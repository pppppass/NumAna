%! TeX encoding = UTF-8
%! TeX program = LuaLaTeX

\documentclass[english, nochinese]{pnote}
\usepackage[paper]{pdef}
\usepackage{pgf}

\DeclareMathOperator\ope{\mathrm{E}}
\DeclareMathOperator\opvar{\mathrm{Var}}

\title{Report of Project of Chapter 7}
\author{Zhihan Li, 1600010653}
\date{June 1, 2019}

\begin{document}

\maketitle

\textbf{Problem (Page 243 Exercise 2).} We consider two algorithm here. The first algorithm is the standard Box--Muller method, which generates independent $ U, V \sim \mathcal{U} \sbr{ 0, 1 } $ and then apply
\begin{gather}
X = \sqrt{ -2 \ln U } \cos \rbr{ 2 \spi V }, \\
Y = \sqrt{ -2 \ln U } \sin \rbr{ 2 \spi V }.
\end{gather}
The second algorithm is the method described in the textbook, which generates independent $ U, V \sim \mathcal{I} \sbr{ -1, 1 } $, rejects samples satisfying $ R^2 = U^2 + V^2 \ge 1 $ and then compute
\begin{gather}
X = U \sqrt{ -2 \ln R^2 / R^2 }, \\
Y = V \sqrt{ -2 \ln R^2 / R^2 }.
\end{gather}
The main efficiency difference comes from the comparison of special function evaluation ($\sin$ and $\cos$) and the $ 1 - \spi / 4 $ rejection rate. On some low-end machines (for example 8051), the evaluation of special functions is very slow since iterative computation must be performed, while the rejection is more acceptable. On modern day machines with FPU, the evaluation of special functions is very fast and rejection matters more. We generate $10^7$ pairs of samples and test the CPU time. The test is performed $100$ times, and both mean and variance are calculated. The CPU times are $\input{Text1.txt}$ and $\input{Text2.txt}$ respectively. We can see that the mean time of the first algorithm is faster. The second algorithm suffers from slightly large variance. We plot the scatter plot of $1000$ samples in Figure \ref{Fig:Algo1} and \ref{Fig:Algo2} respectively. We can see that the samples forms a multi-variate Gaussian distribution.

\begin{figure}[htbp]
\centering
%% Creator: Matplotlib, PGF backend
%%
%% To include the figure in your LaTeX document, write
%%   \input{<filename>.pgf}
%%
%% Make sure the required packages are loaded in your preamble
%%   \usepackage{pgf}
%%
%% Figures using additional raster images can only be included by \input if
%% they are in the same directory as the main LaTeX file. For loading figures
%% from other directories you can use the `import` package
%%   \usepackage{import}
%% and then include the figures with
%%   \import{<path to file>}{<filename>.pgf}
%%
%% Matplotlib used the following preamble
%%   \usepackage{fontspec}
%%   \setmainfont{DejaVuSerif.ttf}[Path=/home/lzh/anaconda3/envs/numana/lib/python3.7/site-packages/matplotlib/mpl-data/fonts/ttf/]
%%   \setsansfont{DejaVuSans.ttf}[Path=/home/lzh/anaconda3/envs/numana/lib/python3.7/site-packages/matplotlib/mpl-data/fonts/ttf/]
%%   \setmonofont{DejaVuSansMono.ttf}[Path=/home/lzh/anaconda3/envs/numana/lib/python3.7/site-packages/matplotlib/mpl-data/fonts/ttf/]
%%
\begingroup%
\makeatletter%
\begin{pgfpicture}%
\pgfpathrectangle{\pgfpointorigin}{\pgfqpoint{6.000000in}{6.000000in}}%
\pgfusepath{use as bounding box, clip}%
\begin{pgfscope}%
\pgfsetbuttcap%
\pgfsetmiterjoin%
\definecolor{currentfill}{rgb}{1.000000,1.000000,1.000000}%
\pgfsetfillcolor{currentfill}%
\pgfsetlinewidth{0.000000pt}%
\definecolor{currentstroke}{rgb}{1.000000,1.000000,1.000000}%
\pgfsetstrokecolor{currentstroke}%
\pgfsetdash{}{0pt}%
\pgfpathmoveto{\pgfqpoint{0.000000in}{0.000000in}}%
\pgfpathlineto{\pgfqpoint{6.000000in}{0.000000in}}%
\pgfpathlineto{\pgfqpoint{6.000000in}{6.000000in}}%
\pgfpathlineto{\pgfqpoint{0.000000in}{6.000000in}}%
\pgfpathclose%
\pgfusepath{fill}%
\end{pgfscope}%
\begin{pgfscope}%
\pgfsetbuttcap%
\pgfsetmiterjoin%
\definecolor{currentfill}{rgb}{1.000000,1.000000,1.000000}%
\pgfsetfillcolor{currentfill}%
\pgfsetlinewidth{0.000000pt}%
\definecolor{currentstroke}{rgb}{0.000000,0.000000,0.000000}%
\pgfsetstrokecolor{currentstroke}%
\pgfsetstrokeopacity{0.000000}%
\pgfsetdash{}{0pt}%
\pgfpathmoveto{\pgfqpoint{0.600000in}{0.600000in}}%
\pgfpathlineto{\pgfqpoint{4.500000in}{0.600000in}}%
\pgfpathlineto{\pgfqpoint{4.500000in}{4.500000in}}%
\pgfpathlineto{\pgfqpoint{0.600000in}{4.500000in}}%
\pgfpathclose%
\pgfusepath{fill}%
\end{pgfscope}%
\begin{pgfscope}%
\pgfpathrectangle{\pgfqpoint{0.600000in}{0.600000in}}{\pgfqpoint{3.900000in}{3.900000in}}%
\pgfusepath{clip}%
\pgfsetbuttcap%
\pgfsetroundjoin%
\definecolor{currentfill}{rgb}{0.121569,0.466667,0.705882}%
\pgfsetfillcolor{currentfill}%
\pgfsetlinewidth{1.003750pt}%
\definecolor{currentstroke}{rgb}{0.121569,0.466667,0.705882}%
\pgfsetstrokecolor{currentstroke}%
\pgfsetdash{}{0pt}%
\pgfpathmoveto{\pgfqpoint{2.549406in}{1.543625in}}%
\pgfpathcurveto{\pgfqpoint{2.560456in}{1.543625in}}{\pgfqpoint{2.571055in}{1.548015in}}{\pgfqpoint{2.578869in}{1.555828in}}%
\pgfpathcurveto{\pgfqpoint{2.586682in}{1.563642in}}{\pgfqpoint{2.591073in}{1.574241in}}{\pgfqpoint{2.591073in}{1.585291in}}%
\pgfpathcurveto{\pgfqpoint{2.591073in}{1.596341in}}{\pgfqpoint{2.586682in}{1.606940in}}{\pgfqpoint{2.578869in}{1.614754in}}%
\pgfpathcurveto{\pgfqpoint{2.571055in}{1.622568in}}{\pgfqpoint{2.560456in}{1.626958in}}{\pgfqpoint{2.549406in}{1.626958in}}%
\pgfpathcurveto{\pgfqpoint{2.538356in}{1.626958in}}{\pgfqpoint{2.527757in}{1.622568in}}{\pgfqpoint{2.519943in}{1.614754in}}%
\pgfpathcurveto{\pgfqpoint{2.512129in}{1.606940in}}{\pgfqpoint{2.507739in}{1.596341in}}{\pgfqpoint{2.507739in}{1.585291in}}%
\pgfpathcurveto{\pgfqpoint{2.507739in}{1.574241in}}{\pgfqpoint{2.512129in}{1.563642in}}{\pgfqpoint{2.519943in}{1.555828in}}%
\pgfpathcurveto{\pgfqpoint{2.527757in}{1.548015in}}{\pgfqpoint{2.538356in}{1.543625in}}{\pgfqpoint{2.549406in}{1.543625in}}%
\pgfpathclose%
\pgfusepath{stroke,fill}%
\end{pgfscope}%
\begin{pgfscope}%
\pgfpathrectangle{\pgfqpoint{0.600000in}{0.600000in}}{\pgfqpoint{3.900000in}{3.900000in}}%
\pgfusepath{clip}%
\pgfsetbuttcap%
\pgfsetroundjoin%
\definecolor{currentfill}{rgb}{0.121569,0.466667,0.705882}%
\pgfsetfillcolor{currentfill}%
\pgfsetlinewidth{1.003750pt}%
\definecolor{currentstroke}{rgb}{0.121569,0.466667,0.705882}%
\pgfsetstrokecolor{currentstroke}%
\pgfsetdash{}{0pt}%
\pgfpathmoveto{\pgfqpoint{3.312224in}{1.701387in}}%
\pgfpathcurveto{\pgfqpoint{3.323274in}{1.701387in}}{\pgfqpoint{3.333873in}{1.705777in}}{\pgfqpoint{3.341687in}{1.713591in}}%
\pgfpathcurveto{\pgfqpoint{3.349501in}{1.721404in}}{\pgfqpoint{3.353891in}{1.732003in}}{\pgfqpoint{3.353891in}{1.743053in}}%
\pgfpathcurveto{\pgfqpoint{3.353891in}{1.754104in}}{\pgfqpoint{3.349501in}{1.764703in}}{\pgfqpoint{3.341687in}{1.772516in}}%
\pgfpathcurveto{\pgfqpoint{3.333873in}{1.780330in}}{\pgfqpoint{3.323274in}{1.784720in}}{\pgfqpoint{3.312224in}{1.784720in}}%
\pgfpathcurveto{\pgfqpoint{3.301174in}{1.784720in}}{\pgfqpoint{3.290575in}{1.780330in}}{\pgfqpoint{3.282762in}{1.772516in}}%
\pgfpathcurveto{\pgfqpoint{3.274948in}{1.764703in}}{\pgfqpoint{3.270558in}{1.754104in}}{\pgfqpoint{3.270558in}{1.743053in}}%
\pgfpathcurveto{\pgfqpoint{3.270558in}{1.732003in}}{\pgfqpoint{3.274948in}{1.721404in}}{\pgfqpoint{3.282762in}{1.713591in}}%
\pgfpathcurveto{\pgfqpoint{3.290575in}{1.705777in}}{\pgfqpoint{3.301174in}{1.701387in}}{\pgfqpoint{3.312224in}{1.701387in}}%
\pgfpathclose%
\pgfusepath{stroke,fill}%
\end{pgfscope}%
\begin{pgfscope}%
\pgfpathrectangle{\pgfqpoint{0.600000in}{0.600000in}}{\pgfqpoint{3.900000in}{3.900000in}}%
\pgfusepath{clip}%
\pgfsetbuttcap%
\pgfsetroundjoin%
\definecolor{currentfill}{rgb}{0.121569,0.466667,0.705882}%
\pgfsetfillcolor{currentfill}%
\pgfsetlinewidth{1.003750pt}%
\definecolor{currentstroke}{rgb}{0.121569,0.466667,0.705882}%
\pgfsetstrokecolor{currentstroke}%
\pgfsetdash{}{0pt}%
\pgfpathmoveto{\pgfqpoint{2.669975in}{1.983979in}}%
\pgfpathcurveto{\pgfqpoint{2.681025in}{1.983979in}}{\pgfqpoint{2.691624in}{1.988369in}}{\pgfqpoint{2.699438in}{1.996183in}}%
\pgfpathcurveto{\pgfqpoint{2.707251in}{2.003997in}}{\pgfqpoint{2.711642in}{2.014596in}}{\pgfqpoint{2.711642in}{2.025646in}}%
\pgfpathcurveto{\pgfqpoint{2.711642in}{2.036696in}}{\pgfqpoint{2.707251in}{2.047295in}}{\pgfqpoint{2.699438in}{2.055109in}}%
\pgfpathcurveto{\pgfqpoint{2.691624in}{2.062922in}}{\pgfqpoint{2.681025in}{2.067312in}}{\pgfqpoint{2.669975in}{2.067312in}}%
\pgfpathcurveto{\pgfqpoint{2.658925in}{2.067312in}}{\pgfqpoint{2.648326in}{2.062922in}}{\pgfqpoint{2.640512in}{2.055109in}}%
\pgfpathcurveto{\pgfqpoint{2.632699in}{2.047295in}}{\pgfqpoint{2.628308in}{2.036696in}}{\pgfqpoint{2.628308in}{2.025646in}}%
\pgfpathcurveto{\pgfqpoint{2.628308in}{2.014596in}}{\pgfqpoint{2.632699in}{2.003997in}}{\pgfqpoint{2.640512in}{1.996183in}}%
\pgfpathcurveto{\pgfqpoint{2.648326in}{1.988369in}}{\pgfqpoint{2.658925in}{1.983979in}}{\pgfqpoint{2.669975in}{1.983979in}}%
\pgfpathclose%
\pgfusepath{stroke,fill}%
\end{pgfscope}%
\begin{pgfscope}%
\pgfpathrectangle{\pgfqpoint{0.600000in}{0.600000in}}{\pgfqpoint{3.900000in}{3.900000in}}%
\pgfusepath{clip}%
\pgfsetbuttcap%
\pgfsetroundjoin%
\definecolor{currentfill}{rgb}{0.121569,0.466667,0.705882}%
\pgfsetfillcolor{currentfill}%
\pgfsetlinewidth{1.003750pt}%
\definecolor{currentstroke}{rgb}{0.121569,0.466667,0.705882}%
\pgfsetstrokecolor{currentstroke}%
\pgfsetdash{}{0pt}%
\pgfpathmoveto{\pgfqpoint{2.251177in}{2.831532in}}%
\pgfpathcurveto{\pgfqpoint{2.262227in}{2.831532in}}{\pgfqpoint{2.272826in}{2.835923in}}{\pgfqpoint{2.280639in}{2.843736in}}%
\pgfpathcurveto{\pgfqpoint{2.288453in}{2.851550in}}{\pgfqpoint{2.292843in}{2.862149in}}{\pgfqpoint{2.292843in}{2.873199in}}%
\pgfpathcurveto{\pgfqpoint{2.292843in}{2.884249in}}{\pgfqpoint{2.288453in}{2.894848in}}{\pgfqpoint{2.280639in}{2.902662in}}%
\pgfpathcurveto{\pgfqpoint{2.272826in}{2.910475in}}{\pgfqpoint{2.262227in}{2.914866in}}{\pgfqpoint{2.251177in}{2.914866in}}%
\pgfpathcurveto{\pgfqpoint{2.240126in}{2.914866in}}{\pgfqpoint{2.229527in}{2.910475in}}{\pgfqpoint{2.221714in}{2.902662in}}%
\pgfpathcurveto{\pgfqpoint{2.213900in}{2.894848in}}{\pgfqpoint{2.209510in}{2.884249in}}{\pgfqpoint{2.209510in}{2.873199in}}%
\pgfpathcurveto{\pgfqpoint{2.209510in}{2.862149in}}{\pgfqpoint{2.213900in}{2.851550in}}{\pgfqpoint{2.221714in}{2.843736in}}%
\pgfpathcurveto{\pgfqpoint{2.229527in}{2.835923in}}{\pgfqpoint{2.240126in}{2.831532in}}{\pgfqpoint{2.251177in}{2.831532in}}%
\pgfpathclose%
\pgfusepath{stroke,fill}%
\end{pgfscope}%
\begin{pgfscope}%
\pgfpathrectangle{\pgfqpoint{0.600000in}{0.600000in}}{\pgfqpoint{3.900000in}{3.900000in}}%
\pgfusepath{clip}%
\pgfsetbuttcap%
\pgfsetroundjoin%
\definecolor{currentfill}{rgb}{0.121569,0.466667,0.705882}%
\pgfsetfillcolor{currentfill}%
\pgfsetlinewidth{1.003750pt}%
\definecolor{currentstroke}{rgb}{0.121569,0.466667,0.705882}%
\pgfsetstrokecolor{currentstroke}%
\pgfsetdash{}{0pt}%
\pgfpathmoveto{\pgfqpoint{2.541790in}{2.242028in}}%
\pgfpathcurveto{\pgfqpoint{2.552840in}{2.242028in}}{\pgfqpoint{2.563439in}{2.246418in}}{\pgfqpoint{2.571253in}{2.254232in}}%
\pgfpathcurveto{\pgfqpoint{2.579067in}{2.262045in}}{\pgfqpoint{2.583457in}{2.272644in}}{\pgfqpoint{2.583457in}{2.283695in}}%
\pgfpathcurveto{\pgfqpoint{2.583457in}{2.294745in}}{\pgfqpoint{2.579067in}{2.305344in}}{\pgfqpoint{2.571253in}{2.313157in}}%
\pgfpathcurveto{\pgfqpoint{2.563439in}{2.320971in}}{\pgfqpoint{2.552840in}{2.325361in}}{\pgfqpoint{2.541790in}{2.325361in}}%
\pgfpathcurveto{\pgfqpoint{2.530740in}{2.325361in}}{\pgfqpoint{2.520141in}{2.320971in}}{\pgfqpoint{2.512328in}{2.313157in}}%
\pgfpathcurveto{\pgfqpoint{2.504514in}{2.305344in}}{\pgfqpoint{2.500124in}{2.294745in}}{\pgfqpoint{2.500124in}{2.283695in}}%
\pgfpathcurveto{\pgfqpoint{2.500124in}{2.272644in}}{\pgfqpoint{2.504514in}{2.262045in}}{\pgfqpoint{2.512328in}{2.254232in}}%
\pgfpathcurveto{\pgfqpoint{2.520141in}{2.246418in}}{\pgfqpoint{2.530740in}{2.242028in}}{\pgfqpoint{2.541790in}{2.242028in}}%
\pgfpathclose%
\pgfusepath{stroke,fill}%
\end{pgfscope}%
\begin{pgfscope}%
\pgfpathrectangle{\pgfqpoint{0.600000in}{0.600000in}}{\pgfqpoint{3.900000in}{3.900000in}}%
\pgfusepath{clip}%
\pgfsetbuttcap%
\pgfsetroundjoin%
\definecolor{currentfill}{rgb}{0.121569,0.466667,0.705882}%
\pgfsetfillcolor{currentfill}%
\pgfsetlinewidth{1.003750pt}%
\definecolor{currentstroke}{rgb}{0.121569,0.466667,0.705882}%
\pgfsetstrokecolor{currentstroke}%
\pgfsetdash{}{0pt}%
\pgfpathmoveto{\pgfqpoint{2.154473in}{3.026745in}}%
\pgfpathcurveto{\pgfqpoint{2.165523in}{3.026745in}}{\pgfqpoint{2.176122in}{3.031135in}}{\pgfqpoint{2.183936in}{3.038949in}}%
\pgfpathcurveto{\pgfqpoint{2.191750in}{3.046762in}}{\pgfqpoint{2.196140in}{3.057361in}}{\pgfqpoint{2.196140in}{3.068411in}}%
\pgfpathcurveto{\pgfqpoint{2.196140in}{3.079462in}}{\pgfqpoint{2.191750in}{3.090061in}}{\pgfqpoint{2.183936in}{3.097874in}}%
\pgfpathcurveto{\pgfqpoint{2.176122in}{3.105688in}}{\pgfqpoint{2.165523in}{3.110078in}}{\pgfqpoint{2.154473in}{3.110078in}}%
\pgfpathcurveto{\pgfqpoint{2.143423in}{3.110078in}}{\pgfqpoint{2.132824in}{3.105688in}}{\pgfqpoint{2.125010in}{3.097874in}}%
\pgfpathcurveto{\pgfqpoint{2.117197in}{3.090061in}}{\pgfqpoint{2.112807in}{3.079462in}}{\pgfqpoint{2.112807in}{3.068411in}}%
\pgfpathcurveto{\pgfqpoint{2.112807in}{3.057361in}}{\pgfqpoint{2.117197in}{3.046762in}}{\pgfqpoint{2.125010in}{3.038949in}}%
\pgfpathcurveto{\pgfqpoint{2.132824in}{3.031135in}}{\pgfqpoint{2.143423in}{3.026745in}}{\pgfqpoint{2.154473in}{3.026745in}}%
\pgfpathclose%
\pgfusepath{stroke,fill}%
\end{pgfscope}%
\begin{pgfscope}%
\pgfpathrectangle{\pgfqpoint{0.600000in}{0.600000in}}{\pgfqpoint{3.900000in}{3.900000in}}%
\pgfusepath{clip}%
\pgfsetbuttcap%
\pgfsetroundjoin%
\definecolor{currentfill}{rgb}{0.121569,0.466667,0.705882}%
\pgfsetfillcolor{currentfill}%
\pgfsetlinewidth{1.003750pt}%
\definecolor{currentstroke}{rgb}{0.121569,0.466667,0.705882}%
\pgfsetstrokecolor{currentstroke}%
\pgfsetdash{}{0pt}%
\pgfpathmoveto{\pgfqpoint{2.524013in}{2.912383in}}%
\pgfpathcurveto{\pgfqpoint{2.535064in}{2.912383in}}{\pgfqpoint{2.545663in}{2.916773in}}{\pgfqpoint{2.553476in}{2.924587in}}%
\pgfpathcurveto{\pgfqpoint{2.561290in}{2.932401in}}{\pgfqpoint{2.565680in}{2.943000in}}{\pgfqpoint{2.565680in}{2.954050in}}%
\pgfpathcurveto{\pgfqpoint{2.565680in}{2.965100in}}{\pgfqpoint{2.561290in}{2.975699in}}{\pgfqpoint{2.553476in}{2.983513in}}%
\pgfpathcurveto{\pgfqpoint{2.545663in}{2.991326in}}{\pgfqpoint{2.535064in}{2.995716in}}{\pgfqpoint{2.524013in}{2.995716in}}%
\pgfpathcurveto{\pgfqpoint{2.512963in}{2.995716in}}{\pgfqpoint{2.502364in}{2.991326in}}{\pgfqpoint{2.494551in}{2.983513in}}%
\pgfpathcurveto{\pgfqpoint{2.486737in}{2.975699in}}{\pgfqpoint{2.482347in}{2.965100in}}{\pgfqpoint{2.482347in}{2.954050in}}%
\pgfpathcurveto{\pgfqpoint{2.482347in}{2.943000in}}{\pgfqpoint{2.486737in}{2.932401in}}{\pgfqpoint{2.494551in}{2.924587in}}%
\pgfpathcurveto{\pgfqpoint{2.502364in}{2.916773in}}{\pgfqpoint{2.512963in}{2.912383in}}{\pgfqpoint{2.524013in}{2.912383in}}%
\pgfpathclose%
\pgfusepath{stroke,fill}%
\end{pgfscope}%
\begin{pgfscope}%
\pgfpathrectangle{\pgfqpoint{0.600000in}{0.600000in}}{\pgfqpoint{3.900000in}{3.900000in}}%
\pgfusepath{clip}%
\pgfsetbuttcap%
\pgfsetroundjoin%
\definecolor{currentfill}{rgb}{0.121569,0.466667,0.705882}%
\pgfsetfillcolor{currentfill}%
\pgfsetlinewidth{1.003750pt}%
\definecolor{currentstroke}{rgb}{0.121569,0.466667,0.705882}%
\pgfsetstrokecolor{currentstroke}%
\pgfsetdash{}{0pt}%
\pgfpathmoveto{\pgfqpoint{2.667290in}{1.818126in}}%
\pgfpathcurveto{\pgfqpoint{2.678340in}{1.818126in}}{\pgfqpoint{2.688939in}{1.822516in}}{\pgfqpoint{2.696753in}{1.830330in}}%
\pgfpathcurveto{\pgfqpoint{2.704566in}{1.838144in}}{\pgfqpoint{2.708956in}{1.848743in}}{\pgfqpoint{2.708956in}{1.859793in}}%
\pgfpathcurveto{\pgfqpoint{2.708956in}{1.870843in}}{\pgfqpoint{2.704566in}{1.881442in}}{\pgfqpoint{2.696753in}{1.889256in}}%
\pgfpathcurveto{\pgfqpoint{2.688939in}{1.897069in}}{\pgfqpoint{2.678340in}{1.901460in}}{\pgfqpoint{2.667290in}{1.901460in}}%
\pgfpathcurveto{\pgfqpoint{2.656240in}{1.901460in}}{\pgfqpoint{2.645641in}{1.897069in}}{\pgfqpoint{2.637827in}{1.889256in}}%
\pgfpathcurveto{\pgfqpoint{2.630013in}{1.881442in}}{\pgfqpoint{2.625623in}{1.870843in}}{\pgfqpoint{2.625623in}{1.859793in}}%
\pgfpathcurveto{\pgfqpoint{2.625623in}{1.848743in}}{\pgfqpoint{2.630013in}{1.838144in}}{\pgfqpoint{2.637827in}{1.830330in}}%
\pgfpathcurveto{\pgfqpoint{2.645641in}{1.822516in}}{\pgfqpoint{2.656240in}{1.818126in}}{\pgfqpoint{2.667290in}{1.818126in}}%
\pgfpathclose%
\pgfusepath{stroke,fill}%
\end{pgfscope}%
\begin{pgfscope}%
\pgfpathrectangle{\pgfqpoint{0.600000in}{0.600000in}}{\pgfqpoint{3.900000in}{3.900000in}}%
\pgfusepath{clip}%
\pgfsetbuttcap%
\pgfsetroundjoin%
\definecolor{currentfill}{rgb}{0.121569,0.466667,0.705882}%
\pgfsetfillcolor{currentfill}%
\pgfsetlinewidth{1.003750pt}%
\definecolor{currentstroke}{rgb}{0.121569,0.466667,0.705882}%
\pgfsetstrokecolor{currentstroke}%
\pgfsetdash{}{0pt}%
\pgfpathmoveto{\pgfqpoint{2.285342in}{2.371828in}}%
\pgfpathcurveto{\pgfqpoint{2.296392in}{2.371828in}}{\pgfqpoint{2.306991in}{2.376218in}}{\pgfqpoint{2.314804in}{2.384032in}}%
\pgfpathcurveto{\pgfqpoint{2.322618in}{2.391845in}}{\pgfqpoint{2.327008in}{2.402444in}}{\pgfqpoint{2.327008in}{2.413495in}}%
\pgfpathcurveto{\pgfqpoint{2.327008in}{2.424545in}}{\pgfqpoint{2.322618in}{2.435144in}}{\pgfqpoint{2.314804in}{2.442957in}}%
\pgfpathcurveto{\pgfqpoint{2.306991in}{2.450771in}}{\pgfqpoint{2.296392in}{2.455161in}}{\pgfqpoint{2.285342in}{2.455161in}}%
\pgfpathcurveto{\pgfqpoint{2.274291in}{2.455161in}}{\pgfqpoint{2.263692in}{2.450771in}}{\pgfqpoint{2.255879in}{2.442957in}}%
\pgfpathcurveto{\pgfqpoint{2.248065in}{2.435144in}}{\pgfqpoint{2.243675in}{2.424545in}}{\pgfqpoint{2.243675in}{2.413495in}}%
\pgfpathcurveto{\pgfqpoint{2.243675in}{2.402444in}}{\pgfqpoint{2.248065in}{2.391845in}}{\pgfqpoint{2.255879in}{2.384032in}}%
\pgfpathcurveto{\pgfqpoint{2.263692in}{2.376218in}}{\pgfqpoint{2.274291in}{2.371828in}}{\pgfqpoint{2.285342in}{2.371828in}}%
\pgfpathclose%
\pgfusepath{stroke,fill}%
\end{pgfscope}%
\begin{pgfscope}%
\pgfpathrectangle{\pgfqpoint{0.600000in}{0.600000in}}{\pgfqpoint{3.900000in}{3.900000in}}%
\pgfusepath{clip}%
\pgfsetbuttcap%
\pgfsetroundjoin%
\definecolor{currentfill}{rgb}{0.121569,0.466667,0.705882}%
\pgfsetfillcolor{currentfill}%
\pgfsetlinewidth{1.003750pt}%
\definecolor{currentstroke}{rgb}{0.121569,0.466667,0.705882}%
\pgfsetstrokecolor{currentstroke}%
\pgfsetdash{}{0pt}%
\pgfpathmoveto{\pgfqpoint{2.914045in}{2.717512in}}%
\pgfpathcurveto{\pgfqpoint{2.925095in}{2.717512in}}{\pgfqpoint{2.935694in}{2.721902in}}{\pgfqpoint{2.943507in}{2.729716in}}%
\pgfpathcurveto{\pgfqpoint{2.951321in}{2.737530in}}{\pgfqpoint{2.955711in}{2.748129in}}{\pgfqpoint{2.955711in}{2.759179in}}%
\pgfpathcurveto{\pgfqpoint{2.955711in}{2.770229in}}{\pgfqpoint{2.951321in}{2.780828in}}{\pgfqpoint{2.943507in}{2.788642in}}%
\pgfpathcurveto{\pgfqpoint{2.935694in}{2.796455in}}{\pgfqpoint{2.925095in}{2.800845in}}{\pgfqpoint{2.914045in}{2.800845in}}%
\pgfpathcurveto{\pgfqpoint{2.902994in}{2.800845in}}{\pgfqpoint{2.892395in}{2.796455in}}{\pgfqpoint{2.884582in}{2.788642in}}%
\pgfpathcurveto{\pgfqpoint{2.876768in}{2.780828in}}{\pgfqpoint{2.872378in}{2.770229in}}{\pgfqpoint{2.872378in}{2.759179in}}%
\pgfpathcurveto{\pgfqpoint{2.872378in}{2.748129in}}{\pgfqpoint{2.876768in}{2.737530in}}{\pgfqpoint{2.884582in}{2.729716in}}%
\pgfpathcurveto{\pgfqpoint{2.892395in}{2.721902in}}{\pgfqpoint{2.902994in}{2.717512in}}{\pgfqpoint{2.914045in}{2.717512in}}%
\pgfpathclose%
\pgfusepath{stroke,fill}%
\end{pgfscope}%
\begin{pgfscope}%
\pgfpathrectangle{\pgfqpoint{0.600000in}{0.600000in}}{\pgfqpoint{3.900000in}{3.900000in}}%
\pgfusepath{clip}%
\pgfsetbuttcap%
\pgfsetroundjoin%
\definecolor{currentfill}{rgb}{0.121569,0.466667,0.705882}%
\pgfsetfillcolor{currentfill}%
\pgfsetlinewidth{1.003750pt}%
\definecolor{currentstroke}{rgb}{0.121569,0.466667,0.705882}%
\pgfsetstrokecolor{currentstroke}%
\pgfsetdash{}{0pt}%
\pgfpathmoveto{\pgfqpoint{3.045769in}{2.918190in}}%
\pgfpathcurveto{\pgfqpoint{3.056819in}{2.918190in}}{\pgfqpoint{3.067418in}{2.922581in}}{\pgfqpoint{3.075232in}{2.930394in}}%
\pgfpathcurveto{\pgfqpoint{3.083046in}{2.938208in}}{\pgfqpoint{3.087436in}{2.948807in}}{\pgfqpoint{3.087436in}{2.959857in}}%
\pgfpathcurveto{\pgfqpoint{3.087436in}{2.970907in}}{\pgfqpoint{3.083046in}{2.981506in}}{\pgfqpoint{3.075232in}{2.989320in}}%
\pgfpathcurveto{\pgfqpoint{3.067418in}{2.997133in}}{\pgfqpoint{3.056819in}{3.001524in}}{\pgfqpoint{3.045769in}{3.001524in}}%
\pgfpathcurveto{\pgfqpoint{3.034719in}{3.001524in}}{\pgfqpoint{3.024120in}{2.997133in}}{\pgfqpoint{3.016306in}{2.989320in}}%
\pgfpathcurveto{\pgfqpoint{3.008493in}{2.981506in}}{\pgfqpoint{3.004103in}{2.970907in}}{\pgfqpoint{3.004103in}{2.959857in}}%
\pgfpathcurveto{\pgfqpoint{3.004103in}{2.948807in}}{\pgfqpoint{3.008493in}{2.938208in}}{\pgfqpoint{3.016306in}{2.930394in}}%
\pgfpathcurveto{\pgfqpoint{3.024120in}{2.922581in}}{\pgfqpoint{3.034719in}{2.918190in}}{\pgfqpoint{3.045769in}{2.918190in}}%
\pgfpathclose%
\pgfusepath{stroke,fill}%
\end{pgfscope}%
\begin{pgfscope}%
\pgfpathrectangle{\pgfqpoint{0.600000in}{0.600000in}}{\pgfqpoint{3.900000in}{3.900000in}}%
\pgfusepath{clip}%
\pgfsetbuttcap%
\pgfsetroundjoin%
\definecolor{currentfill}{rgb}{0.121569,0.466667,0.705882}%
\pgfsetfillcolor{currentfill}%
\pgfsetlinewidth{1.003750pt}%
\definecolor{currentstroke}{rgb}{0.121569,0.466667,0.705882}%
\pgfsetstrokecolor{currentstroke}%
\pgfsetdash{}{0pt}%
\pgfpathmoveto{\pgfqpoint{2.112958in}{2.866721in}}%
\pgfpathcurveto{\pgfqpoint{2.124008in}{2.866721in}}{\pgfqpoint{2.134607in}{2.871112in}}{\pgfqpoint{2.142421in}{2.878925in}}%
\pgfpathcurveto{\pgfqpoint{2.150235in}{2.886739in}}{\pgfqpoint{2.154625in}{2.897338in}}{\pgfqpoint{2.154625in}{2.908388in}}%
\pgfpathcurveto{\pgfqpoint{2.154625in}{2.919438in}}{\pgfqpoint{2.150235in}{2.930037in}}{\pgfqpoint{2.142421in}{2.937851in}}%
\pgfpathcurveto{\pgfqpoint{2.134607in}{2.945664in}}{\pgfqpoint{2.124008in}{2.950055in}}{\pgfqpoint{2.112958in}{2.950055in}}%
\pgfpathcurveto{\pgfqpoint{2.101908in}{2.950055in}}{\pgfqpoint{2.091309in}{2.945664in}}{\pgfqpoint{2.083495in}{2.937851in}}%
\pgfpathcurveto{\pgfqpoint{2.075682in}{2.930037in}}{\pgfqpoint{2.071292in}{2.919438in}}{\pgfqpoint{2.071292in}{2.908388in}}%
\pgfpathcurveto{\pgfqpoint{2.071292in}{2.897338in}}{\pgfqpoint{2.075682in}{2.886739in}}{\pgfqpoint{2.083495in}{2.878925in}}%
\pgfpathcurveto{\pgfqpoint{2.091309in}{2.871112in}}{\pgfqpoint{2.101908in}{2.866721in}}{\pgfqpoint{2.112958in}{2.866721in}}%
\pgfpathclose%
\pgfusepath{stroke,fill}%
\end{pgfscope}%
\begin{pgfscope}%
\pgfpathrectangle{\pgfqpoint{0.600000in}{0.600000in}}{\pgfqpoint{3.900000in}{3.900000in}}%
\pgfusepath{clip}%
\pgfsetbuttcap%
\pgfsetroundjoin%
\definecolor{currentfill}{rgb}{0.121569,0.466667,0.705882}%
\pgfsetfillcolor{currentfill}%
\pgfsetlinewidth{1.003750pt}%
\definecolor{currentstroke}{rgb}{0.121569,0.466667,0.705882}%
\pgfsetstrokecolor{currentstroke}%
\pgfsetdash{}{0pt}%
\pgfpathmoveto{\pgfqpoint{3.077367in}{2.369626in}}%
\pgfpathcurveto{\pgfqpoint{3.088417in}{2.369626in}}{\pgfqpoint{3.099016in}{2.374017in}}{\pgfqpoint{3.106830in}{2.381830in}}%
\pgfpathcurveto{\pgfqpoint{3.114643in}{2.389644in}}{\pgfqpoint{3.119033in}{2.400243in}}{\pgfqpoint{3.119033in}{2.411293in}}%
\pgfpathcurveto{\pgfqpoint{3.119033in}{2.422343in}}{\pgfqpoint{3.114643in}{2.432942in}}{\pgfqpoint{3.106830in}{2.440756in}}%
\pgfpathcurveto{\pgfqpoint{3.099016in}{2.448569in}}{\pgfqpoint{3.088417in}{2.452960in}}{\pgfqpoint{3.077367in}{2.452960in}}%
\pgfpathcurveto{\pgfqpoint{3.066317in}{2.452960in}}{\pgfqpoint{3.055718in}{2.448569in}}{\pgfqpoint{3.047904in}{2.440756in}}%
\pgfpathcurveto{\pgfqpoint{3.040090in}{2.432942in}}{\pgfqpoint{3.035700in}{2.422343in}}{\pgfqpoint{3.035700in}{2.411293in}}%
\pgfpathcurveto{\pgfqpoint{3.035700in}{2.400243in}}{\pgfqpoint{3.040090in}{2.389644in}}{\pgfqpoint{3.047904in}{2.381830in}}%
\pgfpathcurveto{\pgfqpoint{3.055718in}{2.374017in}}{\pgfqpoint{3.066317in}{2.369626in}}{\pgfqpoint{3.077367in}{2.369626in}}%
\pgfpathclose%
\pgfusepath{stroke,fill}%
\end{pgfscope}%
\begin{pgfscope}%
\pgfpathrectangle{\pgfqpoint{0.600000in}{0.600000in}}{\pgfqpoint{3.900000in}{3.900000in}}%
\pgfusepath{clip}%
\pgfsetbuttcap%
\pgfsetroundjoin%
\definecolor{currentfill}{rgb}{0.121569,0.466667,0.705882}%
\pgfsetfillcolor{currentfill}%
\pgfsetlinewidth{1.003750pt}%
\definecolor{currentstroke}{rgb}{0.121569,0.466667,0.705882}%
\pgfsetstrokecolor{currentstroke}%
\pgfsetdash{}{0pt}%
\pgfpathmoveto{\pgfqpoint{2.691820in}{2.742029in}}%
\pgfpathcurveto{\pgfqpoint{2.702870in}{2.742029in}}{\pgfqpoint{2.713469in}{2.746419in}}{\pgfqpoint{2.721283in}{2.754233in}}%
\pgfpathcurveto{\pgfqpoint{2.729096in}{2.762047in}}{\pgfqpoint{2.733487in}{2.772646in}}{\pgfqpoint{2.733487in}{2.783696in}}%
\pgfpathcurveto{\pgfqpoint{2.733487in}{2.794746in}}{\pgfqpoint{2.729096in}{2.805345in}}{\pgfqpoint{2.721283in}{2.813159in}}%
\pgfpathcurveto{\pgfqpoint{2.713469in}{2.820972in}}{\pgfqpoint{2.702870in}{2.825362in}}{\pgfqpoint{2.691820in}{2.825362in}}%
\pgfpathcurveto{\pgfqpoint{2.680770in}{2.825362in}}{\pgfqpoint{2.670171in}{2.820972in}}{\pgfqpoint{2.662357in}{2.813159in}}%
\pgfpathcurveto{\pgfqpoint{2.654544in}{2.805345in}}{\pgfqpoint{2.650153in}{2.794746in}}{\pgfqpoint{2.650153in}{2.783696in}}%
\pgfpathcurveto{\pgfqpoint{2.650153in}{2.772646in}}{\pgfqpoint{2.654544in}{2.762047in}}{\pgfqpoint{2.662357in}{2.754233in}}%
\pgfpathcurveto{\pgfqpoint{2.670171in}{2.746419in}}{\pgfqpoint{2.680770in}{2.742029in}}{\pgfqpoint{2.691820in}{2.742029in}}%
\pgfpathclose%
\pgfusepath{stroke,fill}%
\end{pgfscope}%
\begin{pgfscope}%
\pgfpathrectangle{\pgfqpoint{0.600000in}{0.600000in}}{\pgfqpoint{3.900000in}{3.900000in}}%
\pgfusepath{clip}%
\pgfsetbuttcap%
\pgfsetroundjoin%
\definecolor{currentfill}{rgb}{0.121569,0.466667,0.705882}%
\pgfsetfillcolor{currentfill}%
\pgfsetlinewidth{1.003750pt}%
\definecolor{currentstroke}{rgb}{0.121569,0.466667,0.705882}%
\pgfsetstrokecolor{currentstroke}%
\pgfsetdash{}{0pt}%
\pgfpathmoveto{\pgfqpoint{2.496396in}{3.330565in}}%
\pgfpathcurveto{\pgfqpoint{2.507447in}{3.330565in}}{\pgfqpoint{2.518046in}{3.334955in}}{\pgfqpoint{2.525859in}{3.342769in}}%
\pgfpathcurveto{\pgfqpoint{2.533673in}{3.350583in}}{\pgfqpoint{2.538063in}{3.361182in}}{\pgfqpoint{2.538063in}{3.372232in}}%
\pgfpathcurveto{\pgfqpoint{2.538063in}{3.383282in}}{\pgfqpoint{2.533673in}{3.393881in}}{\pgfqpoint{2.525859in}{3.401695in}}%
\pgfpathcurveto{\pgfqpoint{2.518046in}{3.409508in}}{\pgfqpoint{2.507447in}{3.413899in}}{\pgfqpoint{2.496396in}{3.413899in}}%
\pgfpathcurveto{\pgfqpoint{2.485346in}{3.413899in}}{\pgfqpoint{2.474747in}{3.409508in}}{\pgfqpoint{2.466934in}{3.401695in}}%
\pgfpathcurveto{\pgfqpoint{2.459120in}{3.393881in}}{\pgfqpoint{2.454730in}{3.383282in}}{\pgfqpoint{2.454730in}{3.372232in}}%
\pgfpathcurveto{\pgfqpoint{2.454730in}{3.361182in}}{\pgfqpoint{2.459120in}{3.350583in}}{\pgfqpoint{2.466934in}{3.342769in}}%
\pgfpathcurveto{\pgfqpoint{2.474747in}{3.334955in}}{\pgfqpoint{2.485346in}{3.330565in}}{\pgfqpoint{2.496396in}{3.330565in}}%
\pgfpathclose%
\pgfusepath{stroke,fill}%
\end{pgfscope}%
\begin{pgfscope}%
\pgfpathrectangle{\pgfqpoint{0.600000in}{0.600000in}}{\pgfqpoint{3.900000in}{3.900000in}}%
\pgfusepath{clip}%
\pgfsetbuttcap%
\pgfsetroundjoin%
\definecolor{currentfill}{rgb}{0.121569,0.466667,0.705882}%
\pgfsetfillcolor{currentfill}%
\pgfsetlinewidth{1.003750pt}%
\definecolor{currentstroke}{rgb}{0.121569,0.466667,0.705882}%
\pgfsetstrokecolor{currentstroke}%
\pgfsetdash{}{0pt}%
\pgfpathmoveto{\pgfqpoint{2.362404in}{2.588225in}}%
\pgfpathcurveto{\pgfqpoint{2.373454in}{2.588225in}}{\pgfqpoint{2.384053in}{2.592616in}}{\pgfqpoint{2.391867in}{2.600429in}}%
\pgfpathcurveto{\pgfqpoint{2.399680in}{2.608243in}}{\pgfqpoint{2.404071in}{2.618842in}}{\pgfqpoint{2.404071in}{2.629892in}}%
\pgfpathcurveto{\pgfqpoint{2.404071in}{2.640942in}}{\pgfqpoint{2.399680in}{2.651541in}}{\pgfqpoint{2.391867in}{2.659355in}}%
\pgfpathcurveto{\pgfqpoint{2.384053in}{2.667169in}}{\pgfqpoint{2.373454in}{2.671559in}}{\pgfqpoint{2.362404in}{2.671559in}}%
\pgfpathcurveto{\pgfqpoint{2.351354in}{2.671559in}}{\pgfqpoint{2.340755in}{2.667169in}}{\pgfqpoint{2.332941in}{2.659355in}}%
\pgfpathcurveto{\pgfqpoint{2.325127in}{2.651541in}}{\pgfqpoint{2.320737in}{2.640942in}}{\pgfqpoint{2.320737in}{2.629892in}}%
\pgfpathcurveto{\pgfqpoint{2.320737in}{2.618842in}}{\pgfqpoint{2.325127in}{2.608243in}}{\pgfqpoint{2.332941in}{2.600429in}}%
\pgfpathcurveto{\pgfqpoint{2.340755in}{2.592616in}}{\pgfqpoint{2.351354in}{2.588225in}}{\pgfqpoint{2.362404in}{2.588225in}}%
\pgfpathclose%
\pgfusepath{stroke,fill}%
\end{pgfscope}%
\begin{pgfscope}%
\pgfpathrectangle{\pgfqpoint{0.600000in}{0.600000in}}{\pgfqpoint{3.900000in}{3.900000in}}%
\pgfusepath{clip}%
\pgfsetbuttcap%
\pgfsetroundjoin%
\definecolor{currentfill}{rgb}{0.121569,0.466667,0.705882}%
\pgfsetfillcolor{currentfill}%
\pgfsetlinewidth{1.003750pt}%
\definecolor{currentstroke}{rgb}{0.121569,0.466667,0.705882}%
\pgfsetstrokecolor{currentstroke}%
\pgfsetdash{}{0pt}%
\pgfpathmoveto{\pgfqpoint{2.795369in}{2.761907in}}%
\pgfpathcurveto{\pgfqpoint{2.806419in}{2.761907in}}{\pgfqpoint{2.817018in}{2.766297in}}{\pgfqpoint{2.824832in}{2.774111in}}%
\pgfpathcurveto{\pgfqpoint{2.832645in}{2.781925in}}{\pgfqpoint{2.837036in}{2.792524in}}{\pgfqpoint{2.837036in}{2.803574in}}%
\pgfpathcurveto{\pgfqpoint{2.837036in}{2.814624in}}{\pgfqpoint{2.832645in}{2.825223in}}{\pgfqpoint{2.824832in}{2.833037in}}%
\pgfpathcurveto{\pgfqpoint{2.817018in}{2.840850in}}{\pgfqpoint{2.806419in}{2.845240in}}{\pgfqpoint{2.795369in}{2.845240in}}%
\pgfpathcurveto{\pgfqpoint{2.784319in}{2.845240in}}{\pgfqpoint{2.773720in}{2.840850in}}{\pgfqpoint{2.765906in}{2.833037in}}%
\pgfpathcurveto{\pgfqpoint{2.758092in}{2.825223in}}{\pgfqpoint{2.753702in}{2.814624in}}{\pgfqpoint{2.753702in}{2.803574in}}%
\pgfpathcurveto{\pgfqpoint{2.753702in}{2.792524in}}{\pgfqpoint{2.758092in}{2.781925in}}{\pgfqpoint{2.765906in}{2.774111in}}%
\pgfpathcurveto{\pgfqpoint{2.773720in}{2.766297in}}{\pgfqpoint{2.784319in}{2.761907in}}{\pgfqpoint{2.795369in}{2.761907in}}%
\pgfpathclose%
\pgfusepath{stroke,fill}%
\end{pgfscope}%
\begin{pgfscope}%
\pgfpathrectangle{\pgfqpoint{0.600000in}{0.600000in}}{\pgfqpoint{3.900000in}{3.900000in}}%
\pgfusepath{clip}%
\pgfsetbuttcap%
\pgfsetroundjoin%
\definecolor{currentfill}{rgb}{0.121569,0.466667,0.705882}%
\pgfsetfillcolor{currentfill}%
\pgfsetlinewidth{1.003750pt}%
\definecolor{currentstroke}{rgb}{0.121569,0.466667,0.705882}%
\pgfsetstrokecolor{currentstroke}%
\pgfsetdash{}{0pt}%
\pgfpathmoveto{\pgfqpoint{3.618259in}{2.090357in}}%
\pgfpathcurveto{\pgfqpoint{3.629309in}{2.090357in}}{\pgfqpoint{3.639908in}{2.094747in}}{\pgfqpoint{3.647722in}{2.102560in}}%
\pgfpathcurveto{\pgfqpoint{3.655535in}{2.110374in}}{\pgfqpoint{3.659925in}{2.120973in}}{\pgfqpoint{3.659925in}{2.132023in}}%
\pgfpathcurveto{\pgfqpoint{3.659925in}{2.143073in}}{\pgfqpoint{3.655535in}{2.153672in}}{\pgfqpoint{3.647722in}{2.161486in}}%
\pgfpathcurveto{\pgfqpoint{3.639908in}{2.169300in}}{\pgfqpoint{3.629309in}{2.173690in}}{\pgfqpoint{3.618259in}{2.173690in}}%
\pgfpathcurveto{\pgfqpoint{3.607209in}{2.173690in}}{\pgfqpoint{3.596610in}{2.169300in}}{\pgfqpoint{3.588796in}{2.161486in}}%
\pgfpathcurveto{\pgfqpoint{3.580982in}{2.153672in}}{\pgfqpoint{3.576592in}{2.143073in}}{\pgfqpoint{3.576592in}{2.132023in}}%
\pgfpathcurveto{\pgfqpoint{3.576592in}{2.120973in}}{\pgfqpoint{3.580982in}{2.110374in}}{\pgfqpoint{3.588796in}{2.102560in}}%
\pgfpathcurveto{\pgfqpoint{3.596610in}{2.094747in}}{\pgfqpoint{3.607209in}{2.090357in}}{\pgfqpoint{3.618259in}{2.090357in}}%
\pgfpathclose%
\pgfusepath{stroke,fill}%
\end{pgfscope}%
\begin{pgfscope}%
\pgfpathrectangle{\pgfqpoint{0.600000in}{0.600000in}}{\pgfqpoint{3.900000in}{3.900000in}}%
\pgfusepath{clip}%
\pgfsetbuttcap%
\pgfsetroundjoin%
\definecolor{currentfill}{rgb}{0.121569,0.466667,0.705882}%
\pgfsetfillcolor{currentfill}%
\pgfsetlinewidth{1.003750pt}%
\definecolor{currentstroke}{rgb}{0.121569,0.466667,0.705882}%
\pgfsetstrokecolor{currentstroke}%
\pgfsetdash{}{0pt}%
\pgfpathmoveto{\pgfqpoint{3.336725in}{3.653708in}}%
\pgfpathcurveto{\pgfqpoint{3.347775in}{3.653708in}}{\pgfqpoint{3.358374in}{3.658098in}}{\pgfqpoint{3.366188in}{3.665911in}}%
\pgfpathcurveto{\pgfqpoint{3.374001in}{3.673725in}}{\pgfqpoint{3.378392in}{3.684324in}}{\pgfqpoint{3.378392in}{3.695374in}}%
\pgfpathcurveto{\pgfqpoint{3.378392in}{3.706424in}}{\pgfqpoint{3.374001in}{3.717023in}}{\pgfqpoint{3.366188in}{3.724837in}}%
\pgfpathcurveto{\pgfqpoint{3.358374in}{3.732651in}}{\pgfqpoint{3.347775in}{3.737041in}}{\pgfqpoint{3.336725in}{3.737041in}}%
\pgfpathcurveto{\pgfqpoint{3.325675in}{3.737041in}}{\pgfqpoint{3.315076in}{3.732651in}}{\pgfqpoint{3.307262in}{3.724837in}}%
\pgfpathcurveto{\pgfqpoint{3.299449in}{3.717023in}}{\pgfqpoint{3.295058in}{3.706424in}}{\pgfqpoint{3.295058in}{3.695374in}}%
\pgfpathcurveto{\pgfqpoint{3.295058in}{3.684324in}}{\pgfqpoint{3.299449in}{3.673725in}}{\pgfqpoint{3.307262in}{3.665911in}}%
\pgfpathcurveto{\pgfqpoint{3.315076in}{3.658098in}}{\pgfqpoint{3.325675in}{3.653708in}}{\pgfqpoint{3.336725in}{3.653708in}}%
\pgfpathclose%
\pgfusepath{stroke,fill}%
\end{pgfscope}%
\begin{pgfscope}%
\pgfpathrectangle{\pgfqpoint{0.600000in}{0.600000in}}{\pgfqpoint{3.900000in}{3.900000in}}%
\pgfusepath{clip}%
\pgfsetbuttcap%
\pgfsetroundjoin%
\definecolor{currentfill}{rgb}{0.121569,0.466667,0.705882}%
\pgfsetfillcolor{currentfill}%
\pgfsetlinewidth{1.003750pt}%
\definecolor{currentstroke}{rgb}{0.121569,0.466667,0.705882}%
\pgfsetstrokecolor{currentstroke}%
\pgfsetdash{}{0pt}%
\pgfpathmoveto{\pgfqpoint{2.951030in}{3.096337in}}%
\pgfpathcurveto{\pgfqpoint{2.962081in}{3.096337in}}{\pgfqpoint{2.972680in}{3.100727in}}{\pgfqpoint{2.980493in}{3.108541in}}%
\pgfpathcurveto{\pgfqpoint{2.988307in}{3.116354in}}{\pgfqpoint{2.992697in}{3.126953in}}{\pgfqpoint{2.992697in}{3.138004in}}%
\pgfpathcurveto{\pgfqpoint{2.992697in}{3.149054in}}{\pgfqpoint{2.988307in}{3.159653in}}{\pgfqpoint{2.980493in}{3.167466in}}%
\pgfpathcurveto{\pgfqpoint{2.972680in}{3.175280in}}{\pgfqpoint{2.962081in}{3.179670in}}{\pgfqpoint{2.951030in}{3.179670in}}%
\pgfpathcurveto{\pgfqpoint{2.939980in}{3.179670in}}{\pgfqpoint{2.929381in}{3.175280in}}{\pgfqpoint{2.921568in}{3.167466in}}%
\pgfpathcurveto{\pgfqpoint{2.913754in}{3.159653in}}{\pgfqpoint{2.909364in}{3.149054in}}{\pgfqpoint{2.909364in}{3.138004in}}%
\pgfpathcurveto{\pgfqpoint{2.909364in}{3.126953in}}{\pgfqpoint{2.913754in}{3.116354in}}{\pgfqpoint{2.921568in}{3.108541in}}%
\pgfpathcurveto{\pgfqpoint{2.929381in}{3.100727in}}{\pgfqpoint{2.939980in}{3.096337in}}{\pgfqpoint{2.951030in}{3.096337in}}%
\pgfpathclose%
\pgfusepath{stroke,fill}%
\end{pgfscope}%
\begin{pgfscope}%
\pgfpathrectangle{\pgfqpoint{0.600000in}{0.600000in}}{\pgfqpoint{3.900000in}{3.900000in}}%
\pgfusepath{clip}%
\pgfsetbuttcap%
\pgfsetroundjoin%
\definecolor{currentfill}{rgb}{0.121569,0.466667,0.705882}%
\pgfsetfillcolor{currentfill}%
\pgfsetlinewidth{1.003750pt}%
\definecolor{currentstroke}{rgb}{0.121569,0.466667,0.705882}%
\pgfsetstrokecolor{currentstroke}%
\pgfsetdash{}{0pt}%
\pgfpathmoveto{\pgfqpoint{2.964011in}{2.103829in}}%
\pgfpathcurveto{\pgfqpoint{2.975061in}{2.103829in}}{\pgfqpoint{2.985660in}{2.108219in}}{\pgfqpoint{2.993474in}{2.116033in}}%
\pgfpathcurveto{\pgfqpoint{3.001288in}{2.123847in}}{\pgfqpoint{3.005678in}{2.134446in}}{\pgfqpoint{3.005678in}{2.145496in}}%
\pgfpathcurveto{\pgfqpoint{3.005678in}{2.156546in}}{\pgfqpoint{3.001288in}{2.167145in}}{\pgfqpoint{2.993474in}{2.174958in}}%
\pgfpathcurveto{\pgfqpoint{2.985660in}{2.182772in}}{\pgfqpoint{2.975061in}{2.187162in}}{\pgfqpoint{2.964011in}{2.187162in}}%
\pgfpathcurveto{\pgfqpoint{2.952961in}{2.187162in}}{\pgfqpoint{2.942362in}{2.182772in}}{\pgfqpoint{2.934549in}{2.174958in}}%
\pgfpathcurveto{\pgfqpoint{2.926735in}{2.167145in}}{\pgfqpoint{2.922345in}{2.156546in}}{\pgfqpoint{2.922345in}{2.145496in}}%
\pgfpathcurveto{\pgfqpoint{2.922345in}{2.134446in}}{\pgfqpoint{2.926735in}{2.123847in}}{\pgfqpoint{2.934549in}{2.116033in}}%
\pgfpathcurveto{\pgfqpoint{2.942362in}{2.108219in}}{\pgfqpoint{2.952961in}{2.103829in}}{\pgfqpoint{2.964011in}{2.103829in}}%
\pgfpathclose%
\pgfusepath{stroke,fill}%
\end{pgfscope}%
\begin{pgfscope}%
\pgfpathrectangle{\pgfqpoint{0.600000in}{0.600000in}}{\pgfqpoint{3.900000in}{3.900000in}}%
\pgfusepath{clip}%
\pgfsetbuttcap%
\pgfsetroundjoin%
\definecolor{currentfill}{rgb}{0.121569,0.466667,0.705882}%
\pgfsetfillcolor{currentfill}%
\pgfsetlinewidth{1.003750pt}%
\definecolor{currentstroke}{rgb}{0.121569,0.466667,0.705882}%
\pgfsetstrokecolor{currentstroke}%
\pgfsetdash{}{0pt}%
\pgfpathmoveto{\pgfqpoint{2.491870in}{3.171929in}}%
\pgfpathcurveto{\pgfqpoint{2.502920in}{3.171929in}}{\pgfqpoint{2.513519in}{3.176320in}}{\pgfqpoint{2.521333in}{3.184133in}}%
\pgfpathcurveto{\pgfqpoint{2.529146in}{3.191947in}}{\pgfqpoint{2.533537in}{3.202546in}}{\pgfqpoint{2.533537in}{3.213596in}}%
\pgfpathcurveto{\pgfqpoint{2.533537in}{3.224646in}}{\pgfqpoint{2.529146in}{3.235245in}}{\pgfqpoint{2.521333in}{3.243059in}}%
\pgfpathcurveto{\pgfqpoint{2.513519in}{3.250872in}}{\pgfqpoint{2.502920in}{3.255263in}}{\pgfqpoint{2.491870in}{3.255263in}}%
\pgfpathcurveto{\pgfqpoint{2.480820in}{3.255263in}}{\pgfqpoint{2.470221in}{3.250872in}}{\pgfqpoint{2.462407in}{3.243059in}}%
\pgfpathcurveto{\pgfqpoint{2.454594in}{3.235245in}}{\pgfqpoint{2.450203in}{3.224646in}}{\pgfqpoint{2.450203in}{3.213596in}}%
\pgfpathcurveto{\pgfqpoint{2.450203in}{3.202546in}}{\pgfqpoint{2.454594in}{3.191947in}}{\pgfqpoint{2.462407in}{3.184133in}}%
\pgfpathcurveto{\pgfqpoint{2.470221in}{3.176320in}}{\pgfqpoint{2.480820in}{3.171929in}}{\pgfqpoint{2.491870in}{3.171929in}}%
\pgfpathclose%
\pgfusepath{stroke,fill}%
\end{pgfscope}%
\begin{pgfscope}%
\pgfpathrectangle{\pgfqpoint{0.600000in}{0.600000in}}{\pgfqpoint{3.900000in}{3.900000in}}%
\pgfusepath{clip}%
\pgfsetbuttcap%
\pgfsetroundjoin%
\definecolor{currentfill}{rgb}{0.121569,0.466667,0.705882}%
\pgfsetfillcolor{currentfill}%
\pgfsetlinewidth{1.003750pt}%
\definecolor{currentstroke}{rgb}{0.121569,0.466667,0.705882}%
\pgfsetstrokecolor{currentstroke}%
\pgfsetdash{}{0pt}%
\pgfpathmoveto{\pgfqpoint{2.648395in}{1.733050in}}%
\pgfpathcurveto{\pgfqpoint{2.659445in}{1.733050in}}{\pgfqpoint{2.670044in}{1.737440in}}{\pgfqpoint{2.677858in}{1.745254in}}%
\pgfpathcurveto{\pgfqpoint{2.685671in}{1.753067in}}{\pgfqpoint{2.690062in}{1.763666in}}{\pgfqpoint{2.690062in}{1.774717in}}%
\pgfpathcurveto{\pgfqpoint{2.690062in}{1.785767in}}{\pgfqpoint{2.685671in}{1.796366in}}{\pgfqpoint{2.677858in}{1.804179in}}%
\pgfpathcurveto{\pgfqpoint{2.670044in}{1.811993in}}{\pgfqpoint{2.659445in}{1.816383in}}{\pgfqpoint{2.648395in}{1.816383in}}%
\pgfpathcurveto{\pgfqpoint{2.637345in}{1.816383in}}{\pgfqpoint{2.626746in}{1.811993in}}{\pgfqpoint{2.618932in}{1.804179in}}%
\pgfpathcurveto{\pgfqpoint{2.611118in}{1.796366in}}{\pgfqpoint{2.606728in}{1.785767in}}{\pgfqpoint{2.606728in}{1.774717in}}%
\pgfpathcurveto{\pgfqpoint{2.606728in}{1.763666in}}{\pgfqpoint{2.611118in}{1.753067in}}{\pgfqpoint{2.618932in}{1.745254in}}%
\pgfpathcurveto{\pgfqpoint{2.626746in}{1.737440in}}{\pgfqpoint{2.637345in}{1.733050in}}{\pgfqpoint{2.648395in}{1.733050in}}%
\pgfpathclose%
\pgfusepath{stroke,fill}%
\end{pgfscope}%
\begin{pgfscope}%
\pgfpathrectangle{\pgfqpoint{0.600000in}{0.600000in}}{\pgfqpoint{3.900000in}{3.900000in}}%
\pgfusepath{clip}%
\pgfsetbuttcap%
\pgfsetroundjoin%
\definecolor{currentfill}{rgb}{0.121569,0.466667,0.705882}%
\pgfsetfillcolor{currentfill}%
\pgfsetlinewidth{1.003750pt}%
\definecolor{currentstroke}{rgb}{0.121569,0.466667,0.705882}%
\pgfsetstrokecolor{currentstroke}%
\pgfsetdash{}{0pt}%
\pgfpathmoveto{\pgfqpoint{2.692906in}{1.433760in}}%
\pgfpathcurveto{\pgfqpoint{2.703956in}{1.433760in}}{\pgfqpoint{2.714555in}{1.438150in}}{\pgfqpoint{2.722368in}{1.445964in}}%
\pgfpathcurveto{\pgfqpoint{2.730182in}{1.453777in}}{\pgfqpoint{2.734572in}{1.464376in}}{\pgfqpoint{2.734572in}{1.475426in}}%
\pgfpathcurveto{\pgfqpoint{2.734572in}{1.486476in}}{\pgfqpoint{2.730182in}{1.497076in}}{\pgfqpoint{2.722368in}{1.504889in}}%
\pgfpathcurveto{\pgfqpoint{2.714555in}{1.512703in}}{\pgfqpoint{2.703956in}{1.517093in}}{\pgfqpoint{2.692906in}{1.517093in}}%
\pgfpathcurveto{\pgfqpoint{2.681855in}{1.517093in}}{\pgfqpoint{2.671256in}{1.512703in}}{\pgfqpoint{2.663443in}{1.504889in}}%
\pgfpathcurveto{\pgfqpoint{2.655629in}{1.497076in}}{\pgfqpoint{2.651239in}{1.486476in}}{\pgfqpoint{2.651239in}{1.475426in}}%
\pgfpathcurveto{\pgfqpoint{2.651239in}{1.464376in}}{\pgfqpoint{2.655629in}{1.453777in}}{\pgfqpoint{2.663443in}{1.445964in}}%
\pgfpathcurveto{\pgfqpoint{2.671256in}{1.438150in}}{\pgfqpoint{2.681855in}{1.433760in}}{\pgfqpoint{2.692906in}{1.433760in}}%
\pgfpathclose%
\pgfusepath{stroke,fill}%
\end{pgfscope}%
\begin{pgfscope}%
\pgfpathrectangle{\pgfqpoint{0.600000in}{0.600000in}}{\pgfqpoint{3.900000in}{3.900000in}}%
\pgfusepath{clip}%
\pgfsetbuttcap%
\pgfsetroundjoin%
\definecolor{currentfill}{rgb}{0.121569,0.466667,0.705882}%
\pgfsetfillcolor{currentfill}%
\pgfsetlinewidth{1.003750pt}%
\definecolor{currentstroke}{rgb}{0.121569,0.466667,0.705882}%
\pgfsetstrokecolor{currentstroke}%
\pgfsetdash{}{0pt}%
\pgfpathmoveto{\pgfqpoint{2.633965in}{2.928931in}}%
\pgfpathcurveto{\pgfqpoint{2.645015in}{2.928931in}}{\pgfqpoint{2.655614in}{2.933321in}}{\pgfqpoint{2.663428in}{2.941135in}}%
\pgfpathcurveto{\pgfqpoint{2.671242in}{2.948949in}}{\pgfqpoint{2.675632in}{2.959548in}}{\pgfqpoint{2.675632in}{2.970598in}}%
\pgfpathcurveto{\pgfqpoint{2.675632in}{2.981648in}}{\pgfqpoint{2.671242in}{2.992247in}}{\pgfqpoint{2.663428in}{3.000060in}}%
\pgfpathcurveto{\pgfqpoint{2.655614in}{3.007874in}}{\pgfqpoint{2.645015in}{3.012264in}}{\pgfqpoint{2.633965in}{3.012264in}}%
\pgfpathcurveto{\pgfqpoint{2.622915in}{3.012264in}}{\pgfqpoint{2.612316in}{3.007874in}}{\pgfqpoint{2.604502in}{3.000060in}}%
\pgfpathcurveto{\pgfqpoint{2.596689in}{2.992247in}}{\pgfqpoint{2.592299in}{2.981648in}}{\pgfqpoint{2.592299in}{2.970598in}}%
\pgfpathcurveto{\pgfqpoint{2.592299in}{2.959548in}}{\pgfqpoint{2.596689in}{2.948949in}}{\pgfqpoint{2.604502in}{2.941135in}}%
\pgfpathcurveto{\pgfqpoint{2.612316in}{2.933321in}}{\pgfqpoint{2.622915in}{2.928931in}}{\pgfqpoint{2.633965in}{2.928931in}}%
\pgfpathclose%
\pgfusepath{stroke,fill}%
\end{pgfscope}%
\begin{pgfscope}%
\pgfpathrectangle{\pgfqpoint{0.600000in}{0.600000in}}{\pgfqpoint{3.900000in}{3.900000in}}%
\pgfusepath{clip}%
\pgfsetbuttcap%
\pgfsetroundjoin%
\definecolor{currentfill}{rgb}{0.121569,0.466667,0.705882}%
\pgfsetfillcolor{currentfill}%
\pgfsetlinewidth{1.003750pt}%
\definecolor{currentstroke}{rgb}{0.121569,0.466667,0.705882}%
\pgfsetstrokecolor{currentstroke}%
\pgfsetdash{}{0pt}%
\pgfpathmoveto{\pgfqpoint{2.228222in}{2.707714in}}%
\pgfpathcurveto{\pgfqpoint{2.239272in}{2.707714in}}{\pgfqpoint{2.249871in}{2.712105in}}{\pgfqpoint{2.257685in}{2.719918in}}%
\pgfpathcurveto{\pgfqpoint{2.265499in}{2.727732in}}{\pgfqpoint{2.269889in}{2.738331in}}{\pgfqpoint{2.269889in}{2.749381in}}%
\pgfpathcurveto{\pgfqpoint{2.269889in}{2.760431in}}{\pgfqpoint{2.265499in}{2.771030in}}{\pgfqpoint{2.257685in}{2.778844in}}%
\pgfpathcurveto{\pgfqpoint{2.249871in}{2.786657in}}{\pgfqpoint{2.239272in}{2.791048in}}{\pgfqpoint{2.228222in}{2.791048in}}%
\pgfpathcurveto{\pgfqpoint{2.217172in}{2.791048in}}{\pgfqpoint{2.206573in}{2.786657in}}{\pgfqpoint{2.198759in}{2.778844in}}%
\pgfpathcurveto{\pgfqpoint{2.190946in}{2.771030in}}{\pgfqpoint{2.186555in}{2.760431in}}{\pgfqpoint{2.186555in}{2.749381in}}%
\pgfpathcurveto{\pgfqpoint{2.186555in}{2.738331in}}{\pgfqpoint{2.190946in}{2.727732in}}{\pgfqpoint{2.198759in}{2.719918in}}%
\pgfpathcurveto{\pgfqpoint{2.206573in}{2.712105in}}{\pgfqpoint{2.217172in}{2.707714in}}{\pgfqpoint{2.228222in}{2.707714in}}%
\pgfpathclose%
\pgfusepath{stroke,fill}%
\end{pgfscope}%
\begin{pgfscope}%
\pgfpathrectangle{\pgfqpoint{0.600000in}{0.600000in}}{\pgfqpoint{3.900000in}{3.900000in}}%
\pgfusepath{clip}%
\pgfsetbuttcap%
\pgfsetroundjoin%
\definecolor{currentfill}{rgb}{0.121569,0.466667,0.705882}%
\pgfsetfillcolor{currentfill}%
\pgfsetlinewidth{1.003750pt}%
\definecolor{currentstroke}{rgb}{0.121569,0.466667,0.705882}%
\pgfsetstrokecolor{currentstroke}%
\pgfsetdash{}{0pt}%
\pgfpathmoveto{\pgfqpoint{2.981831in}{2.304984in}}%
\pgfpathcurveto{\pgfqpoint{2.992881in}{2.304984in}}{\pgfqpoint{3.003480in}{2.309375in}}{\pgfqpoint{3.011293in}{2.317188in}}%
\pgfpathcurveto{\pgfqpoint{3.019107in}{2.325002in}}{\pgfqpoint{3.023497in}{2.335601in}}{\pgfqpoint{3.023497in}{2.346651in}}%
\pgfpathcurveto{\pgfqpoint{3.023497in}{2.357701in}}{\pgfqpoint{3.019107in}{2.368300in}}{\pgfqpoint{3.011293in}{2.376114in}}%
\pgfpathcurveto{\pgfqpoint{3.003480in}{2.383927in}}{\pgfqpoint{2.992881in}{2.388318in}}{\pgfqpoint{2.981831in}{2.388318in}}%
\pgfpathcurveto{\pgfqpoint{2.970780in}{2.388318in}}{\pgfqpoint{2.960181in}{2.383927in}}{\pgfqpoint{2.952368in}{2.376114in}}%
\pgfpathcurveto{\pgfqpoint{2.944554in}{2.368300in}}{\pgfqpoint{2.940164in}{2.357701in}}{\pgfqpoint{2.940164in}{2.346651in}}%
\pgfpathcurveto{\pgfqpoint{2.940164in}{2.335601in}}{\pgfqpoint{2.944554in}{2.325002in}}{\pgfqpoint{2.952368in}{2.317188in}}%
\pgfpathcurveto{\pgfqpoint{2.960181in}{2.309375in}}{\pgfqpoint{2.970780in}{2.304984in}}{\pgfqpoint{2.981831in}{2.304984in}}%
\pgfpathclose%
\pgfusepath{stroke,fill}%
\end{pgfscope}%
\begin{pgfscope}%
\pgfpathrectangle{\pgfqpoint{0.600000in}{0.600000in}}{\pgfqpoint{3.900000in}{3.900000in}}%
\pgfusepath{clip}%
\pgfsetbuttcap%
\pgfsetroundjoin%
\definecolor{currentfill}{rgb}{0.121569,0.466667,0.705882}%
\pgfsetfillcolor{currentfill}%
\pgfsetlinewidth{1.003750pt}%
\definecolor{currentstroke}{rgb}{0.121569,0.466667,0.705882}%
\pgfsetstrokecolor{currentstroke}%
\pgfsetdash{}{0pt}%
\pgfpathmoveto{\pgfqpoint{2.319383in}{1.952160in}}%
\pgfpathcurveto{\pgfqpoint{2.330433in}{1.952160in}}{\pgfqpoint{2.341032in}{1.956550in}}{\pgfqpoint{2.348845in}{1.964364in}}%
\pgfpathcurveto{\pgfqpoint{2.356659in}{1.972178in}}{\pgfqpoint{2.361049in}{1.982777in}}{\pgfqpoint{2.361049in}{1.993827in}}%
\pgfpathcurveto{\pgfqpoint{2.361049in}{2.004877in}}{\pgfqpoint{2.356659in}{2.015476in}}{\pgfqpoint{2.348845in}{2.023290in}}%
\pgfpathcurveto{\pgfqpoint{2.341032in}{2.031103in}}{\pgfqpoint{2.330433in}{2.035494in}}{\pgfqpoint{2.319383in}{2.035494in}}%
\pgfpathcurveto{\pgfqpoint{2.308333in}{2.035494in}}{\pgfqpoint{2.297733in}{2.031103in}}{\pgfqpoint{2.289920in}{2.023290in}}%
\pgfpathcurveto{\pgfqpoint{2.282106in}{2.015476in}}{\pgfqpoint{2.277716in}{2.004877in}}{\pgfqpoint{2.277716in}{1.993827in}}%
\pgfpathcurveto{\pgfqpoint{2.277716in}{1.982777in}}{\pgfqpoint{2.282106in}{1.972178in}}{\pgfqpoint{2.289920in}{1.964364in}}%
\pgfpathcurveto{\pgfqpoint{2.297733in}{1.956550in}}{\pgfqpoint{2.308333in}{1.952160in}}{\pgfqpoint{2.319383in}{1.952160in}}%
\pgfpathclose%
\pgfusepath{stroke,fill}%
\end{pgfscope}%
\begin{pgfscope}%
\pgfpathrectangle{\pgfqpoint{0.600000in}{0.600000in}}{\pgfqpoint{3.900000in}{3.900000in}}%
\pgfusepath{clip}%
\pgfsetbuttcap%
\pgfsetroundjoin%
\definecolor{currentfill}{rgb}{0.121569,0.466667,0.705882}%
\pgfsetfillcolor{currentfill}%
\pgfsetlinewidth{1.003750pt}%
\definecolor{currentstroke}{rgb}{0.121569,0.466667,0.705882}%
\pgfsetstrokecolor{currentstroke}%
\pgfsetdash{}{0pt}%
\pgfpathmoveto{\pgfqpoint{2.036129in}{2.092382in}}%
\pgfpathcurveto{\pgfqpoint{2.047179in}{2.092382in}}{\pgfqpoint{2.057778in}{2.096772in}}{\pgfqpoint{2.065592in}{2.104586in}}%
\pgfpathcurveto{\pgfqpoint{2.073405in}{2.112399in}}{\pgfqpoint{2.077795in}{2.122998in}}{\pgfqpoint{2.077795in}{2.134049in}}%
\pgfpathcurveto{\pgfqpoint{2.077795in}{2.145099in}}{\pgfqpoint{2.073405in}{2.155698in}}{\pgfqpoint{2.065592in}{2.163511in}}%
\pgfpathcurveto{\pgfqpoint{2.057778in}{2.171325in}}{\pgfqpoint{2.047179in}{2.175715in}}{\pgfqpoint{2.036129in}{2.175715in}}%
\pgfpathcurveto{\pgfqpoint{2.025079in}{2.175715in}}{\pgfqpoint{2.014480in}{2.171325in}}{\pgfqpoint{2.006666in}{2.163511in}}%
\pgfpathcurveto{\pgfqpoint{1.998852in}{2.155698in}}{\pgfqpoint{1.994462in}{2.145099in}}{\pgfqpoint{1.994462in}{2.134049in}}%
\pgfpathcurveto{\pgfqpoint{1.994462in}{2.122998in}}{\pgfqpoint{1.998852in}{2.112399in}}{\pgfqpoint{2.006666in}{2.104586in}}%
\pgfpathcurveto{\pgfqpoint{2.014480in}{2.096772in}}{\pgfqpoint{2.025079in}{2.092382in}}{\pgfqpoint{2.036129in}{2.092382in}}%
\pgfpathclose%
\pgfusepath{stroke,fill}%
\end{pgfscope}%
\begin{pgfscope}%
\pgfpathrectangle{\pgfqpoint{0.600000in}{0.600000in}}{\pgfqpoint{3.900000in}{3.900000in}}%
\pgfusepath{clip}%
\pgfsetbuttcap%
\pgfsetroundjoin%
\definecolor{currentfill}{rgb}{0.121569,0.466667,0.705882}%
\pgfsetfillcolor{currentfill}%
\pgfsetlinewidth{1.003750pt}%
\definecolor{currentstroke}{rgb}{0.121569,0.466667,0.705882}%
\pgfsetstrokecolor{currentstroke}%
\pgfsetdash{}{0pt}%
\pgfpathmoveto{\pgfqpoint{2.187863in}{2.109738in}}%
\pgfpathcurveto{\pgfqpoint{2.198913in}{2.109738in}}{\pgfqpoint{2.209512in}{2.114128in}}{\pgfqpoint{2.217326in}{2.121942in}}%
\pgfpathcurveto{\pgfqpoint{2.225139in}{2.129756in}}{\pgfqpoint{2.229530in}{2.140355in}}{\pgfqpoint{2.229530in}{2.151405in}}%
\pgfpathcurveto{\pgfqpoint{2.229530in}{2.162455in}}{\pgfqpoint{2.225139in}{2.173054in}}{\pgfqpoint{2.217326in}{2.180868in}}%
\pgfpathcurveto{\pgfqpoint{2.209512in}{2.188681in}}{\pgfqpoint{2.198913in}{2.193071in}}{\pgfqpoint{2.187863in}{2.193071in}}%
\pgfpathcurveto{\pgfqpoint{2.176813in}{2.193071in}}{\pgfqpoint{2.166214in}{2.188681in}}{\pgfqpoint{2.158400in}{2.180868in}}%
\pgfpathcurveto{\pgfqpoint{2.150587in}{2.173054in}}{\pgfqpoint{2.146196in}{2.162455in}}{\pgfqpoint{2.146196in}{2.151405in}}%
\pgfpathcurveto{\pgfqpoint{2.146196in}{2.140355in}}{\pgfqpoint{2.150587in}{2.129756in}}{\pgfqpoint{2.158400in}{2.121942in}}%
\pgfpathcurveto{\pgfqpoint{2.166214in}{2.114128in}}{\pgfqpoint{2.176813in}{2.109738in}}{\pgfqpoint{2.187863in}{2.109738in}}%
\pgfpathclose%
\pgfusepath{stroke,fill}%
\end{pgfscope}%
\begin{pgfscope}%
\pgfpathrectangle{\pgfqpoint{0.600000in}{0.600000in}}{\pgfqpoint{3.900000in}{3.900000in}}%
\pgfusepath{clip}%
\pgfsetbuttcap%
\pgfsetroundjoin%
\definecolor{currentfill}{rgb}{0.121569,0.466667,0.705882}%
\pgfsetfillcolor{currentfill}%
\pgfsetlinewidth{1.003750pt}%
\definecolor{currentstroke}{rgb}{0.121569,0.466667,0.705882}%
\pgfsetstrokecolor{currentstroke}%
\pgfsetdash{}{0pt}%
\pgfpathmoveto{\pgfqpoint{2.120908in}{2.038290in}}%
\pgfpathcurveto{\pgfqpoint{2.131958in}{2.038290in}}{\pgfqpoint{2.142557in}{2.042680in}}{\pgfqpoint{2.150371in}{2.050494in}}%
\pgfpathcurveto{\pgfqpoint{2.158185in}{2.058307in}}{\pgfqpoint{2.162575in}{2.068906in}}{\pgfqpoint{2.162575in}{2.079957in}}%
\pgfpathcurveto{\pgfqpoint{2.162575in}{2.091007in}}{\pgfqpoint{2.158185in}{2.101606in}}{\pgfqpoint{2.150371in}{2.109419in}}%
\pgfpathcurveto{\pgfqpoint{2.142557in}{2.117233in}}{\pgfqpoint{2.131958in}{2.121623in}}{\pgfqpoint{2.120908in}{2.121623in}}%
\pgfpathcurveto{\pgfqpoint{2.109858in}{2.121623in}}{\pgfqpoint{2.099259in}{2.117233in}}{\pgfqpoint{2.091445in}{2.109419in}}%
\pgfpathcurveto{\pgfqpoint{2.083632in}{2.101606in}}{\pgfqpoint{2.079241in}{2.091007in}}{\pgfqpoint{2.079241in}{2.079957in}}%
\pgfpathcurveto{\pgfqpoint{2.079241in}{2.068906in}}{\pgfqpoint{2.083632in}{2.058307in}}{\pgfqpoint{2.091445in}{2.050494in}}%
\pgfpathcurveto{\pgfqpoint{2.099259in}{2.042680in}}{\pgfqpoint{2.109858in}{2.038290in}}{\pgfqpoint{2.120908in}{2.038290in}}%
\pgfpathclose%
\pgfusepath{stroke,fill}%
\end{pgfscope}%
\begin{pgfscope}%
\pgfpathrectangle{\pgfqpoint{0.600000in}{0.600000in}}{\pgfqpoint{3.900000in}{3.900000in}}%
\pgfusepath{clip}%
\pgfsetbuttcap%
\pgfsetroundjoin%
\definecolor{currentfill}{rgb}{0.121569,0.466667,0.705882}%
\pgfsetfillcolor{currentfill}%
\pgfsetlinewidth{1.003750pt}%
\definecolor{currentstroke}{rgb}{0.121569,0.466667,0.705882}%
\pgfsetstrokecolor{currentstroke}%
\pgfsetdash{}{0pt}%
\pgfpathmoveto{\pgfqpoint{3.539884in}{2.259216in}}%
\pgfpathcurveto{\pgfqpoint{3.550935in}{2.259216in}}{\pgfqpoint{3.561534in}{2.263606in}}{\pgfqpoint{3.569347in}{2.271420in}}%
\pgfpathcurveto{\pgfqpoint{3.577161in}{2.279233in}}{\pgfqpoint{3.581551in}{2.289833in}}{\pgfqpoint{3.581551in}{2.300883in}}%
\pgfpathcurveto{\pgfqpoint{3.581551in}{2.311933in}}{\pgfqpoint{3.577161in}{2.322532in}}{\pgfqpoint{3.569347in}{2.330345in}}%
\pgfpathcurveto{\pgfqpoint{3.561534in}{2.338159in}}{\pgfqpoint{3.550935in}{2.342549in}}{\pgfqpoint{3.539884in}{2.342549in}}%
\pgfpathcurveto{\pgfqpoint{3.528834in}{2.342549in}}{\pgfqpoint{3.518235in}{2.338159in}}{\pgfqpoint{3.510422in}{2.330345in}}%
\pgfpathcurveto{\pgfqpoint{3.502608in}{2.322532in}}{\pgfqpoint{3.498218in}{2.311933in}}{\pgfqpoint{3.498218in}{2.300883in}}%
\pgfpathcurveto{\pgfqpoint{3.498218in}{2.289833in}}{\pgfqpoint{3.502608in}{2.279233in}}{\pgfqpoint{3.510422in}{2.271420in}}%
\pgfpathcurveto{\pgfqpoint{3.518235in}{2.263606in}}{\pgfqpoint{3.528834in}{2.259216in}}{\pgfqpoint{3.539884in}{2.259216in}}%
\pgfpathclose%
\pgfusepath{stroke,fill}%
\end{pgfscope}%
\begin{pgfscope}%
\pgfpathrectangle{\pgfqpoint{0.600000in}{0.600000in}}{\pgfqpoint{3.900000in}{3.900000in}}%
\pgfusepath{clip}%
\pgfsetbuttcap%
\pgfsetroundjoin%
\definecolor{currentfill}{rgb}{0.121569,0.466667,0.705882}%
\pgfsetfillcolor{currentfill}%
\pgfsetlinewidth{1.003750pt}%
\definecolor{currentstroke}{rgb}{0.121569,0.466667,0.705882}%
\pgfsetstrokecolor{currentstroke}%
\pgfsetdash{}{0pt}%
\pgfpathmoveto{\pgfqpoint{1.596533in}{2.840255in}}%
\pgfpathcurveto{\pgfqpoint{1.607584in}{2.840255in}}{\pgfqpoint{1.618183in}{2.844646in}}{\pgfqpoint{1.625996in}{2.852459in}}%
\pgfpathcurveto{\pgfqpoint{1.633810in}{2.860273in}}{\pgfqpoint{1.638200in}{2.870872in}}{\pgfqpoint{1.638200in}{2.881922in}}%
\pgfpathcurveto{\pgfqpoint{1.638200in}{2.892972in}}{\pgfqpoint{1.633810in}{2.903571in}}{\pgfqpoint{1.625996in}{2.911385in}}%
\pgfpathcurveto{\pgfqpoint{1.618183in}{2.919198in}}{\pgfqpoint{1.607584in}{2.923589in}}{\pgfqpoint{1.596533in}{2.923589in}}%
\pgfpathcurveto{\pgfqpoint{1.585483in}{2.923589in}}{\pgfqpoint{1.574884in}{2.919198in}}{\pgfqpoint{1.567071in}{2.911385in}}%
\pgfpathcurveto{\pgfqpoint{1.559257in}{2.903571in}}{\pgfqpoint{1.554867in}{2.892972in}}{\pgfqpoint{1.554867in}{2.881922in}}%
\pgfpathcurveto{\pgfqpoint{1.554867in}{2.870872in}}{\pgfqpoint{1.559257in}{2.860273in}}{\pgfqpoint{1.567071in}{2.852459in}}%
\pgfpathcurveto{\pgfqpoint{1.574884in}{2.844646in}}{\pgfqpoint{1.585483in}{2.840255in}}{\pgfqpoint{1.596533in}{2.840255in}}%
\pgfpathclose%
\pgfusepath{stroke,fill}%
\end{pgfscope}%
\begin{pgfscope}%
\pgfpathrectangle{\pgfqpoint{0.600000in}{0.600000in}}{\pgfqpoint{3.900000in}{3.900000in}}%
\pgfusepath{clip}%
\pgfsetbuttcap%
\pgfsetroundjoin%
\definecolor{currentfill}{rgb}{0.121569,0.466667,0.705882}%
\pgfsetfillcolor{currentfill}%
\pgfsetlinewidth{1.003750pt}%
\definecolor{currentstroke}{rgb}{0.121569,0.466667,0.705882}%
\pgfsetstrokecolor{currentstroke}%
\pgfsetdash{}{0pt}%
\pgfpathmoveto{\pgfqpoint{3.285897in}{2.282367in}}%
\pgfpathcurveto{\pgfqpoint{3.296947in}{2.282367in}}{\pgfqpoint{3.307546in}{2.286758in}}{\pgfqpoint{3.315360in}{2.294571in}}%
\pgfpathcurveto{\pgfqpoint{3.323173in}{2.302385in}}{\pgfqpoint{3.327564in}{2.312984in}}{\pgfqpoint{3.327564in}{2.324034in}}%
\pgfpathcurveto{\pgfqpoint{3.327564in}{2.335084in}}{\pgfqpoint{3.323173in}{2.345683in}}{\pgfqpoint{3.315360in}{2.353497in}}%
\pgfpathcurveto{\pgfqpoint{3.307546in}{2.361310in}}{\pgfqpoint{3.296947in}{2.365701in}}{\pgfqpoint{3.285897in}{2.365701in}}%
\pgfpathcurveto{\pgfqpoint{3.274847in}{2.365701in}}{\pgfqpoint{3.264248in}{2.361310in}}{\pgfqpoint{3.256434in}{2.353497in}}%
\pgfpathcurveto{\pgfqpoint{3.248621in}{2.345683in}}{\pgfqpoint{3.244230in}{2.335084in}}{\pgfqpoint{3.244230in}{2.324034in}}%
\pgfpathcurveto{\pgfqpoint{3.244230in}{2.312984in}}{\pgfqpoint{3.248621in}{2.302385in}}{\pgfqpoint{3.256434in}{2.294571in}}%
\pgfpathcurveto{\pgfqpoint{3.264248in}{2.286758in}}{\pgfqpoint{3.274847in}{2.282367in}}{\pgfqpoint{3.285897in}{2.282367in}}%
\pgfpathclose%
\pgfusepath{stroke,fill}%
\end{pgfscope}%
\begin{pgfscope}%
\pgfpathrectangle{\pgfqpoint{0.600000in}{0.600000in}}{\pgfqpoint{3.900000in}{3.900000in}}%
\pgfusepath{clip}%
\pgfsetbuttcap%
\pgfsetroundjoin%
\definecolor{currentfill}{rgb}{0.121569,0.466667,0.705882}%
\pgfsetfillcolor{currentfill}%
\pgfsetlinewidth{1.003750pt}%
\definecolor{currentstroke}{rgb}{0.121569,0.466667,0.705882}%
\pgfsetstrokecolor{currentstroke}%
\pgfsetdash{}{0pt}%
\pgfpathmoveto{\pgfqpoint{1.959999in}{2.947227in}}%
\pgfpathcurveto{\pgfqpoint{1.971049in}{2.947227in}}{\pgfqpoint{1.981648in}{2.951617in}}{\pgfqpoint{1.989462in}{2.959431in}}%
\pgfpathcurveto{\pgfqpoint{1.997275in}{2.967244in}}{\pgfqpoint{2.001666in}{2.977843in}}{\pgfqpoint{2.001666in}{2.988893in}}%
\pgfpathcurveto{\pgfqpoint{2.001666in}{2.999944in}}{\pgfqpoint{1.997275in}{3.010543in}}{\pgfqpoint{1.989462in}{3.018356in}}%
\pgfpathcurveto{\pgfqpoint{1.981648in}{3.026170in}}{\pgfqpoint{1.971049in}{3.030560in}}{\pgfqpoint{1.959999in}{3.030560in}}%
\pgfpathcurveto{\pgfqpoint{1.948949in}{3.030560in}}{\pgfqpoint{1.938350in}{3.026170in}}{\pgfqpoint{1.930536in}{3.018356in}}%
\pgfpathcurveto{\pgfqpoint{1.922723in}{3.010543in}}{\pgfqpoint{1.918332in}{2.999944in}}{\pgfqpoint{1.918332in}{2.988893in}}%
\pgfpathcurveto{\pgfqpoint{1.918332in}{2.977843in}}{\pgfqpoint{1.922723in}{2.967244in}}{\pgfqpoint{1.930536in}{2.959431in}}%
\pgfpathcurveto{\pgfqpoint{1.938350in}{2.951617in}}{\pgfqpoint{1.948949in}{2.947227in}}{\pgfqpoint{1.959999in}{2.947227in}}%
\pgfpathclose%
\pgfusepath{stroke,fill}%
\end{pgfscope}%
\begin{pgfscope}%
\pgfpathrectangle{\pgfqpoint{0.600000in}{0.600000in}}{\pgfqpoint{3.900000in}{3.900000in}}%
\pgfusepath{clip}%
\pgfsetbuttcap%
\pgfsetroundjoin%
\definecolor{currentfill}{rgb}{0.121569,0.466667,0.705882}%
\pgfsetfillcolor{currentfill}%
\pgfsetlinewidth{1.003750pt}%
\definecolor{currentstroke}{rgb}{0.121569,0.466667,0.705882}%
\pgfsetstrokecolor{currentstroke}%
\pgfsetdash{}{0pt}%
\pgfpathmoveto{\pgfqpoint{3.604731in}{2.202522in}}%
\pgfpathcurveto{\pgfqpoint{3.615781in}{2.202522in}}{\pgfqpoint{3.626380in}{2.206912in}}{\pgfqpoint{3.634194in}{2.214726in}}%
\pgfpathcurveto{\pgfqpoint{3.642007in}{2.222539in}}{\pgfqpoint{3.646397in}{2.233138in}}{\pgfqpoint{3.646397in}{2.244188in}}%
\pgfpathcurveto{\pgfqpoint{3.646397in}{2.255239in}}{\pgfqpoint{3.642007in}{2.265838in}}{\pgfqpoint{3.634194in}{2.273651in}}%
\pgfpathcurveto{\pgfqpoint{3.626380in}{2.281465in}}{\pgfqpoint{3.615781in}{2.285855in}}{\pgfqpoint{3.604731in}{2.285855in}}%
\pgfpathcurveto{\pgfqpoint{3.593681in}{2.285855in}}{\pgfqpoint{3.583082in}{2.281465in}}{\pgfqpoint{3.575268in}{2.273651in}}%
\pgfpathcurveto{\pgfqpoint{3.567454in}{2.265838in}}{\pgfqpoint{3.563064in}{2.255239in}}{\pgfqpoint{3.563064in}{2.244188in}}%
\pgfpathcurveto{\pgfqpoint{3.563064in}{2.233138in}}{\pgfqpoint{3.567454in}{2.222539in}}{\pgfqpoint{3.575268in}{2.214726in}}%
\pgfpathcurveto{\pgfqpoint{3.583082in}{2.206912in}}{\pgfqpoint{3.593681in}{2.202522in}}{\pgfqpoint{3.604731in}{2.202522in}}%
\pgfpathclose%
\pgfusepath{stroke,fill}%
\end{pgfscope}%
\begin{pgfscope}%
\pgfpathrectangle{\pgfqpoint{0.600000in}{0.600000in}}{\pgfqpoint{3.900000in}{3.900000in}}%
\pgfusepath{clip}%
\pgfsetbuttcap%
\pgfsetroundjoin%
\definecolor{currentfill}{rgb}{0.121569,0.466667,0.705882}%
\pgfsetfillcolor{currentfill}%
\pgfsetlinewidth{1.003750pt}%
\definecolor{currentstroke}{rgb}{0.121569,0.466667,0.705882}%
\pgfsetstrokecolor{currentstroke}%
\pgfsetdash{}{0pt}%
\pgfpathmoveto{\pgfqpoint{2.469742in}{2.466822in}}%
\pgfpathcurveto{\pgfqpoint{2.480792in}{2.466822in}}{\pgfqpoint{2.491391in}{2.471212in}}{\pgfqpoint{2.499204in}{2.479025in}}%
\pgfpathcurveto{\pgfqpoint{2.507018in}{2.486839in}}{\pgfqpoint{2.511408in}{2.497438in}}{\pgfqpoint{2.511408in}{2.508488in}}%
\pgfpathcurveto{\pgfqpoint{2.511408in}{2.519538in}}{\pgfqpoint{2.507018in}{2.530137in}}{\pgfqpoint{2.499204in}{2.537951in}}%
\pgfpathcurveto{\pgfqpoint{2.491391in}{2.545765in}}{\pgfqpoint{2.480792in}{2.550155in}}{\pgfqpoint{2.469742in}{2.550155in}}%
\pgfpathcurveto{\pgfqpoint{2.458692in}{2.550155in}}{\pgfqpoint{2.448092in}{2.545765in}}{\pgfqpoint{2.440279in}{2.537951in}}%
\pgfpathcurveto{\pgfqpoint{2.432465in}{2.530137in}}{\pgfqpoint{2.428075in}{2.519538in}}{\pgfqpoint{2.428075in}{2.508488in}}%
\pgfpathcurveto{\pgfqpoint{2.428075in}{2.497438in}}{\pgfqpoint{2.432465in}{2.486839in}}{\pgfqpoint{2.440279in}{2.479025in}}%
\pgfpathcurveto{\pgfqpoint{2.448092in}{2.471212in}}{\pgfqpoint{2.458692in}{2.466822in}}{\pgfqpoint{2.469742in}{2.466822in}}%
\pgfpathclose%
\pgfusepath{stroke,fill}%
\end{pgfscope}%
\begin{pgfscope}%
\pgfpathrectangle{\pgfqpoint{0.600000in}{0.600000in}}{\pgfqpoint{3.900000in}{3.900000in}}%
\pgfusepath{clip}%
\pgfsetbuttcap%
\pgfsetroundjoin%
\definecolor{currentfill}{rgb}{0.121569,0.466667,0.705882}%
\pgfsetfillcolor{currentfill}%
\pgfsetlinewidth{1.003750pt}%
\definecolor{currentstroke}{rgb}{0.121569,0.466667,0.705882}%
\pgfsetstrokecolor{currentstroke}%
\pgfsetdash{}{0pt}%
\pgfpathmoveto{\pgfqpoint{2.712027in}{2.730979in}}%
\pgfpathcurveto{\pgfqpoint{2.723077in}{2.730979in}}{\pgfqpoint{2.733676in}{2.735369in}}{\pgfqpoint{2.741489in}{2.743182in}}%
\pgfpathcurveto{\pgfqpoint{2.749303in}{2.750996in}}{\pgfqpoint{2.753693in}{2.761595in}}{\pgfqpoint{2.753693in}{2.772645in}}%
\pgfpathcurveto{\pgfqpoint{2.753693in}{2.783695in}}{\pgfqpoint{2.749303in}{2.794294in}}{\pgfqpoint{2.741489in}{2.802108in}}%
\pgfpathcurveto{\pgfqpoint{2.733676in}{2.809922in}}{\pgfqpoint{2.723077in}{2.814312in}}{\pgfqpoint{2.712027in}{2.814312in}}%
\pgfpathcurveto{\pgfqpoint{2.700977in}{2.814312in}}{\pgfqpoint{2.690377in}{2.809922in}}{\pgfqpoint{2.682564in}{2.802108in}}%
\pgfpathcurveto{\pgfqpoint{2.674750in}{2.794294in}}{\pgfqpoint{2.670360in}{2.783695in}}{\pgfqpoint{2.670360in}{2.772645in}}%
\pgfpathcurveto{\pgfqpoint{2.670360in}{2.761595in}}{\pgfqpoint{2.674750in}{2.750996in}}{\pgfqpoint{2.682564in}{2.743182in}}%
\pgfpathcurveto{\pgfqpoint{2.690377in}{2.735369in}}{\pgfqpoint{2.700977in}{2.730979in}}{\pgfqpoint{2.712027in}{2.730979in}}%
\pgfpathclose%
\pgfusepath{stroke,fill}%
\end{pgfscope}%
\begin{pgfscope}%
\pgfpathrectangle{\pgfqpoint{0.600000in}{0.600000in}}{\pgfqpoint{3.900000in}{3.900000in}}%
\pgfusepath{clip}%
\pgfsetbuttcap%
\pgfsetroundjoin%
\definecolor{currentfill}{rgb}{0.121569,0.466667,0.705882}%
\pgfsetfillcolor{currentfill}%
\pgfsetlinewidth{1.003750pt}%
\definecolor{currentstroke}{rgb}{0.121569,0.466667,0.705882}%
\pgfsetstrokecolor{currentstroke}%
\pgfsetdash{}{0pt}%
\pgfpathmoveto{\pgfqpoint{2.419785in}{2.326109in}}%
\pgfpathcurveto{\pgfqpoint{2.430835in}{2.326109in}}{\pgfqpoint{2.441434in}{2.330499in}}{\pgfqpoint{2.449248in}{2.338313in}}%
\pgfpathcurveto{\pgfqpoint{2.457061in}{2.346126in}}{\pgfqpoint{2.461451in}{2.356725in}}{\pgfqpoint{2.461451in}{2.367775in}}%
\pgfpathcurveto{\pgfqpoint{2.461451in}{2.378826in}}{\pgfqpoint{2.457061in}{2.389425in}}{\pgfqpoint{2.449248in}{2.397238in}}%
\pgfpathcurveto{\pgfqpoint{2.441434in}{2.405052in}}{\pgfqpoint{2.430835in}{2.409442in}}{\pgfqpoint{2.419785in}{2.409442in}}%
\pgfpathcurveto{\pgfqpoint{2.408735in}{2.409442in}}{\pgfqpoint{2.398136in}{2.405052in}}{\pgfqpoint{2.390322in}{2.397238in}}%
\pgfpathcurveto{\pgfqpoint{2.382508in}{2.389425in}}{\pgfqpoint{2.378118in}{2.378826in}}{\pgfqpoint{2.378118in}{2.367775in}}%
\pgfpathcurveto{\pgfqpoint{2.378118in}{2.356725in}}{\pgfqpoint{2.382508in}{2.346126in}}{\pgfqpoint{2.390322in}{2.338313in}}%
\pgfpathcurveto{\pgfqpoint{2.398136in}{2.330499in}}{\pgfqpoint{2.408735in}{2.326109in}}{\pgfqpoint{2.419785in}{2.326109in}}%
\pgfpathclose%
\pgfusepath{stroke,fill}%
\end{pgfscope}%
\begin{pgfscope}%
\pgfpathrectangle{\pgfqpoint{0.600000in}{0.600000in}}{\pgfqpoint{3.900000in}{3.900000in}}%
\pgfusepath{clip}%
\pgfsetbuttcap%
\pgfsetroundjoin%
\definecolor{currentfill}{rgb}{0.121569,0.466667,0.705882}%
\pgfsetfillcolor{currentfill}%
\pgfsetlinewidth{1.003750pt}%
\definecolor{currentstroke}{rgb}{0.121569,0.466667,0.705882}%
\pgfsetstrokecolor{currentstroke}%
\pgfsetdash{}{0pt}%
\pgfpathmoveto{\pgfqpoint{3.705896in}{2.172510in}}%
\pgfpathcurveto{\pgfqpoint{3.716946in}{2.172510in}}{\pgfqpoint{3.727545in}{2.176900in}}{\pgfqpoint{3.735359in}{2.184714in}}%
\pgfpathcurveto{\pgfqpoint{3.743172in}{2.192527in}}{\pgfqpoint{3.747563in}{2.203126in}}{\pgfqpoint{3.747563in}{2.214176in}}%
\pgfpathcurveto{\pgfqpoint{3.747563in}{2.225226in}}{\pgfqpoint{3.743172in}{2.235826in}}{\pgfqpoint{3.735359in}{2.243639in}}%
\pgfpathcurveto{\pgfqpoint{3.727545in}{2.251453in}}{\pgfqpoint{3.716946in}{2.255843in}}{\pgfqpoint{3.705896in}{2.255843in}}%
\pgfpathcurveto{\pgfqpoint{3.694846in}{2.255843in}}{\pgfqpoint{3.684247in}{2.251453in}}{\pgfqpoint{3.676433in}{2.243639in}}%
\pgfpathcurveto{\pgfqpoint{3.668620in}{2.235826in}}{\pgfqpoint{3.664229in}{2.225226in}}{\pgfqpoint{3.664229in}{2.214176in}}%
\pgfpathcurveto{\pgfqpoint{3.664229in}{2.203126in}}{\pgfqpoint{3.668620in}{2.192527in}}{\pgfqpoint{3.676433in}{2.184714in}}%
\pgfpathcurveto{\pgfqpoint{3.684247in}{2.176900in}}{\pgfqpoint{3.694846in}{2.172510in}}{\pgfqpoint{3.705896in}{2.172510in}}%
\pgfpathclose%
\pgfusepath{stroke,fill}%
\end{pgfscope}%
\begin{pgfscope}%
\pgfpathrectangle{\pgfqpoint{0.600000in}{0.600000in}}{\pgfqpoint{3.900000in}{3.900000in}}%
\pgfusepath{clip}%
\pgfsetbuttcap%
\pgfsetroundjoin%
\definecolor{currentfill}{rgb}{0.121569,0.466667,0.705882}%
\pgfsetfillcolor{currentfill}%
\pgfsetlinewidth{1.003750pt}%
\definecolor{currentstroke}{rgb}{0.121569,0.466667,0.705882}%
\pgfsetstrokecolor{currentstroke}%
\pgfsetdash{}{0pt}%
\pgfpathmoveto{\pgfqpoint{2.579775in}{2.377230in}}%
\pgfpathcurveto{\pgfqpoint{2.590826in}{2.377230in}}{\pgfqpoint{2.601425in}{2.381620in}}{\pgfqpoint{2.609238in}{2.389434in}}%
\pgfpathcurveto{\pgfqpoint{2.617052in}{2.397247in}}{\pgfqpoint{2.621442in}{2.407846in}}{\pgfqpoint{2.621442in}{2.418896in}}%
\pgfpathcurveto{\pgfqpoint{2.621442in}{2.429946in}}{\pgfqpoint{2.617052in}{2.440545in}}{\pgfqpoint{2.609238in}{2.448359in}}%
\pgfpathcurveto{\pgfqpoint{2.601425in}{2.456173in}}{\pgfqpoint{2.590826in}{2.460563in}}{\pgfqpoint{2.579775in}{2.460563in}}%
\pgfpathcurveto{\pgfqpoint{2.568725in}{2.460563in}}{\pgfqpoint{2.558126in}{2.456173in}}{\pgfqpoint{2.550313in}{2.448359in}}%
\pgfpathcurveto{\pgfqpoint{2.542499in}{2.440545in}}{\pgfqpoint{2.538109in}{2.429946in}}{\pgfqpoint{2.538109in}{2.418896in}}%
\pgfpathcurveto{\pgfqpoint{2.538109in}{2.407846in}}{\pgfqpoint{2.542499in}{2.397247in}}{\pgfqpoint{2.550313in}{2.389434in}}%
\pgfpathcurveto{\pgfqpoint{2.558126in}{2.381620in}}{\pgfqpoint{2.568725in}{2.377230in}}{\pgfqpoint{2.579775in}{2.377230in}}%
\pgfpathclose%
\pgfusepath{stroke,fill}%
\end{pgfscope}%
\begin{pgfscope}%
\pgfpathrectangle{\pgfqpoint{0.600000in}{0.600000in}}{\pgfqpoint{3.900000in}{3.900000in}}%
\pgfusepath{clip}%
\pgfsetbuttcap%
\pgfsetroundjoin%
\definecolor{currentfill}{rgb}{0.121569,0.466667,0.705882}%
\pgfsetfillcolor{currentfill}%
\pgfsetlinewidth{1.003750pt}%
\definecolor{currentstroke}{rgb}{0.121569,0.466667,0.705882}%
\pgfsetstrokecolor{currentstroke}%
\pgfsetdash{}{0pt}%
\pgfpathmoveto{\pgfqpoint{2.245461in}{2.362087in}}%
\pgfpathcurveto{\pgfqpoint{2.256511in}{2.362087in}}{\pgfqpoint{2.267110in}{2.366477in}}{\pgfqpoint{2.274924in}{2.374291in}}%
\pgfpathcurveto{\pgfqpoint{2.282737in}{2.382105in}}{\pgfqpoint{2.287128in}{2.392704in}}{\pgfqpoint{2.287128in}{2.403754in}}%
\pgfpathcurveto{\pgfqpoint{2.287128in}{2.414804in}}{\pgfqpoint{2.282737in}{2.425403in}}{\pgfqpoint{2.274924in}{2.433217in}}%
\pgfpathcurveto{\pgfqpoint{2.267110in}{2.441030in}}{\pgfqpoint{2.256511in}{2.445420in}}{\pgfqpoint{2.245461in}{2.445420in}}%
\pgfpathcurveto{\pgfqpoint{2.234411in}{2.445420in}}{\pgfqpoint{2.223812in}{2.441030in}}{\pgfqpoint{2.215998in}{2.433217in}}%
\pgfpathcurveto{\pgfqpoint{2.208185in}{2.425403in}}{\pgfqpoint{2.203794in}{2.414804in}}{\pgfqpoint{2.203794in}{2.403754in}}%
\pgfpathcurveto{\pgfqpoint{2.203794in}{2.392704in}}{\pgfqpoint{2.208185in}{2.382105in}}{\pgfqpoint{2.215998in}{2.374291in}}%
\pgfpathcurveto{\pgfqpoint{2.223812in}{2.366477in}}{\pgfqpoint{2.234411in}{2.362087in}}{\pgfqpoint{2.245461in}{2.362087in}}%
\pgfpathclose%
\pgfusepath{stroke,fill}%
\end{pgfscope}%
\begin{pgfscope}%
\pgfpathrectangle{\pgfqpoint{0.600000in}{0.600000in}}{\pgfqpoint{3.900000in}{3.900000in}}%
\pgfusepath{clip}%
\pgfsetbuttcap%
\pgfsetroundjoin%
\definecolor{currentfill}{rgb}{0.121569,0.466667,0.705882}%
\pgfsetfillcolor{currentfill}%
\pgfsetlinewidth{1.003750pt}%
\definecolor{currentstroke}{rgb}{0.121569,0.466667,0.705882}%
\pgfsetstrokecolor{currentstroke}%
\pgfsetdash{}{0pt}%
\pgfpathmoveto{\pgfqpoint{3.343351in}{2.318136in}}%
\pgfpathcurveto{\pgfqpoint{3.354401in}{2.318136in}}{\pgfqpoint{3.365000in}{2.322526in}}{\pgfqpoint{3.372814in}{2.330340in}}%
\pgfpathcurveto{\pgfqpoint{3.380628in}{2.338153in}}{\pgfqpoint{3.385018in}{2.348752in}}{\pgfqpoint{3.385018in}{2.359802in}}%
\pgfpathcurveto{\pgfqpoint{3.385018in}{2.370852in}}{\pgfqpoint{3.380628in}{2.381451in}}{\pgfqpoint{3.372814in}{2.389265in}}%
\pgfpathcurveto{\pgfqpoint{3.365000in}{2.397079in}}{\pgfqpoint{3.354401in}{2.401469in}}{\pgfqpoint{3.343351in}{2.401469in}}%
\pgfpathcurveto{\pgfqpoint{3.332301in}{2.401469in}}{\pgfqpoint{3.321702in}{2.397079in}}{\pgfqpoint{3.313889in}{2.389265in}}%
\pgfpathcurveto{\pgfqpoint{3.306075in}{2.381451in}}{\pgfqpoint{3.301685in}{2.370852in}}{\pgfqpoint{3.301685in}{2.359802in}}%
\pgfpathcurveto{\pgfqpoint{3.301685in}{2.348752in}}{\pgfqpoint{3.306075in}{2.338153in}}{\pgfqpoint{3.313889in}{2.330340in}}%
\pgfpathcurveto{\pgfqpoint{3.321702in}{2.322526in}}{\pgfqpoint{3.332301in}{2.318136in}}{\pgfqpoint{3.343351in}{2.318136in}}%
\pgfpathclose%
\pgfusepath{stroke,fill}%
\end{pgfscope}%
\begin{pgfscope}%
\pgfpathrectangle{\pgfqpoint{0.600000in}{0.600000in}}{\pgfqpoint{3.900000in}{3.900000in}}%
\pgfusepath{clip}%
\pgfsetbuttcap%
\pgfsetroundjoin%
\definecolor{currentfill}{rgb}{0.121569,0.466667,0.705882}%
\pgfsetfillcolor{currentfill}%
\pgfsetlinewidth{1.003750pt}%
\definecolor{currentstroke}{rgb}{0.121569,0.466667,0.705882}%
\pgfsetstrokecolor{currentstroke}%
\pgfsetdash{}{0pt}%
\pgfpathmoveto{\pgfqpoint{2.055574in}{2.715469in}}%
\pgfpathcurveto{\pgfqpoint{2.066624in}{2.715469in}}{\pgfqpoint{2.077223in}{2.719860in}}{\pgfqpoint{2.085037in}{2.727673in}}%
\pgfpathcurveto{\pgfqpoint{2.092851in}{2.735487in}}{\pgfqpoint{2.097241in}{2.746086in}}{\pgfqpoint{2.097241in}{2.757136in}}%
\pgfpathcurveto{\pgfqpoint{2.097241in}{2.768186in}}{\pgfqpoint{2.092851in}{2.778785in}}{\pgfqpoint{2.085037in}{2.786599in}}%
\pgfpathcurveto{\pgfqpoint{2.077223in}{2.794412in}}{\pgfqpoint{2.066624in}{2.798803in}}{\pgfqpoint{2.055574in}{2.798803in}}%
\pgfpathcurveto{\pgfqpoint{2.044524in}{2.798803in}}{\pgfqpoint{2.033925in}{2.794412in}}{\pgfqpoint{2.026111in}{2.786599in}}%
\pgfpathcurveto{\pgfqpoint{2.018298in}{2.778785in}}{\pgfqpoint{2.013908in}{2.768186in}}{\pgfqpoint{2.013908in}{2.757136in}}%
\pgfpathcurveto{\pgfqpoint{2.013908in}{2.746086in}}{\pgfqpoint{2.018298in}{2.735487in}}{\pgfqpoint{2.026111in}{2.727673in}}%
\pgfpathcurveto{\pgfqpoint{2.033925in}{2.719860in}}{\pgfqpoint{2.044524in}{2.715469in}}{\pgfqpoint{2.055574in}{2.715469in}}%
\pgfpathclose%
\pgfusepath{stroke,fill}%
\end{pgfscope}%
\begin{pgfscope}%
\pgfpathrectangle{\pgfqpoint{0.600000in}{0.600000in}}{\pgfqpoint{3.900000in}{3.900000in}}%
\pgfusepath{clip}%
\pgfsetbuttcap%
\pgfsetroundjoin%
\definecolor{currentfill}{rgb}{0.121569,0.466667,0.705882}%
\pgfsetfillcolor{currentfill}%
\pgfsetlinewidth{1.003750pt}%
\definecolor{currentstroke}{rgb}{0.121569,0.466667,0.705882}%
\pgfsetstrokecolor{currentstroke}%
\pgfsetdash{}{0pt}%
\pgfpathmoveto{\pgfqpoint{3.186714in}{2.175845in}}%
\pgfpathcurveto{\pgfqpoint{3.197764in}{2.175845in}}{\pgfqpoint{3.208363in}{2.180236in}}{\pgfqpoint{3.216177in}{2.188049in}}%
\pgfpathcurveto{\pgfqpoint{3.223991in}{2.195863in}}{\pgfqpoint{3.228381in}{2.206462in}}{\pgfqpoint{3.228381in}{2.217512in}}%
\pgfpathcurveto{\pgfqpoint{3.228381in}{2.228562in}}{\pgfqpoint{3.223991in}{2.239161in}}{\pgfqpoint{3.216177in}{2.246975in}}%
\pgfpathcurveto{\pgfqpoint{3.208363in}{2.254788in}}{\pgfqpoint{3.197764in}{2.259179in}}{\pgfqpoint{3.186714in}{2.259179in}}%
\pgfpathcurveto{\pgfqpoint{3.175664in}{2.259179in}}{\pgfqpoint{3.165065in}{2.254788in}}{\pgfqpoint{3.157252in}{2.246975in}}%
\pgfpathcurveto{\pgfqpoint{3.149438in}{2.239161in}}{\pgfqpoint{3.145048in}{2.228562in}}{\pgfqpoint{3.145048in}{2.217512in}}%
\pgfpathcurveto{\pgfqpoint{3.145048in}{2.206462in}}{\pgfqpoint{3.149438in}{2.195863in}}{\pgfqpoint{3.157252in}{2.188049in}}%
\pgfpathcurveto{\pgfqpoint{3.165065in}{2.180236in}}{\pgfqpoint{3.175664in}{2.175845in}}{\pgfqpoint{3.186714in}{2.175845in}}%
\pgfpathclose%
\pgfusepath{stroke,fill}%
\end{pgfscope}%
\begin{pgfscope}%
\pgfpathrectangle{\pgfqpoint{0.600000in}{0.600000in}}{\pgfqpoint{3.900000in}{3.900000in}}%
\pgfusepath{clip}%
\pgfsetbuttcap%
\pgfsetroundjoin%
\definecolor{currentfill}{rgb}{0.121569,0.466667,0.705882}%
\pgfsetfillcolor{currentfill}%
\pgfsetlinewidth{1.003750pt}%
\definecolor{currentstroke}{rgb}{0.121569,0.466667,0.705882}%
\pgfsetstrokecolor{currentstroke}%
\pgfsetdash{}{0pt}%
\pgfpathmoveto{\pgfqpoint{2.711673in}{1.150403in}}%
\pgfpathcurveto{\pgfqpoint{2.722723in}{1.150403in}}{\pgfqpoint{2.733322in}{1.154793in}}{\pgfqpoint{2.741136in}{1.162607in}}%
\pgfpathcurveto{\pgfqpoint{2.748950in}{1.170420in}}{\pgfqpoint{2.753340in}{1.181019in}}{\pgfqpoint{2.753340in}{1.192069in}}%
\pgfpathcurveto{\pgfqpoint{2.753340in}{1.203120in}}{\pgfqpoint{2.748950in}{1.213719in}}{\pgfqpoint{2.741136in}{1.221532in}}%
\pgfpathcurveto{\pgfqpoint{2.733322in}{1.229346in}}{\pgfqpoint{2.722723in}{1.233736in}}{\pgfqpoint{2.711673in}{1.233736in}}%
\pgfpathcurveto{\pgfqpoint{2.700623in}{1.233736in}}{\pgfqpoint{2.690024in}{1.229346in}}{\pgfqpoint{2.682210in}{1.221532in}}%
\pgfpathcurveto{\pgfqpoint{2.674397in}{1.213719in}}{\pgfqpoint{2.670007in}{1.203120in}}{\pgfqpoint{2.670007in}{1.192069in}}%
\pgfpathcurveto{\pgfqpoint{2.670007in}{1.181019in}}{\pgfqpoint{2.674397in}{1.170420in}}{\pgfqpoint{2.682210in}{1.162607in}}%
\pgfpathcurveto{\pgfqpoint{2.690024in}{1.154793in}}{\pgfqpoint{2.700623in}{1.150403in}}{\pgfqpoint{2.711673in}{1.150403in}}%
\pgfpathclose%
\pgfusepath{stroke,fill}%
\end{pgfscope}%
\begin{pgfscope}%
\pgfpathrectangle{\pgfqpoint{0.600000in}{0.600000in}}{\pgfqpoint{3.900000in}{3.900000in}}%
\pgfusepath{clip}%
\pgfsetbuttcap%
\pgfsetroundjoin%
\definecolor{currentfill}{rgb}{0.121569,0.466667,0.705882}%
\pgfsetfillcolor{currentfill}%
\pgfsetlinewidth{1.003750pt}%
\definecolor{currentstroke}{rgb}{0.121569,0.466667,0.705882}%
\pgfsetstrokecolor{currentstroke}%
\pgfsetdash{}{0pt}%
\pgfpathmoveto{\pgfqpoint{2.202470in}{2.284803in}}%
\pgfpathcurveto{\pgfqpoint{2.213520in}{2.284803in}}{\pgfqpoint{2.224119in}{2.289193in}}{\pgfqpoint{2.231933in}{2.297007in}}%
\pgfpathcurveto{\pgfqpoint{2.239746in}{2.304821in}}{\pgfqpoint{2.244136in}{2.315420in}}{\pgfqpoint{2.244136in}{2.326470in}}%
\pgfpathcurveto{\pgfqpoint{2.244136in}{2.337520in}}{\pgfqpoint{2.239746in}{2.348119in}}{\pgfqpoint{2.231933in}{2.355933in}}%
\pgfpathcurveto{\pgfqpoint{2.224119in}{2.363746in}}{\pgfqpoint{2.213520in}{2.368137in}}{\pgfqpoint{2.202470in}{2.368137in}}%
\pgfpathcurveto{\pgfqpoint{2.191420in}{2.368137in}}{\pgfqpoint{2.180821in}{2.363746in}}{\pgfqpoint{2.173007in}{2.355933in}}%
\pgfpathcurveto{\pgfqpoint{2.165193in}{2.348119in}}{\pgfqpoint{2.160803in}{2.337520in}}{\pgfqpoint{2.160803in}{2.326470in}}%
\pgfpathcurveto{\pgfqpoint{2.160803in}{2.315420in}}{\pgfqpoint{2.165193in}{2.304821in}}{\pgfqpoint{2.173007in}{2.297007in}}%
\pgfpathcurveto{\pgfqpoint{2.180821in}{2.289193in}}{\pgfqpoint{2.191420in}{2.284803in}}{\pgfqpoint{2.202470in}{2.284803in}}%
\pgfpathclose%
\pgfusepath{stroke,fill}%
\end{pgfscope}%
\begin{pgfscope}%
\pgfpathrectangle{\pgfqpoint{0.600000in}{0.600000in}}{\pgfqpoint{3.900000in}{3.900000in}}%
\pgfusepath{clip}%
\pgfsetbuttcap%
\pgfsetroundjoin%
\definecolor{currentfill}{rgb}{0.121569,0.466667,0.705882}%
\pgfsetfillcolor{currentfill}%
\pgfsetlinewidth{1.003750pt}%
\definecolor{currentstroke}{rgb}{0.121569,0.466667,0.705882}%
\pgfsetstrokecolor{currentstroke}%
\pgfsetdash{}{0pt}%
\pgfpathmoveto{\pgfqpoint{2.129613in}{1.986010in}}%
\pgfpathcurveto{\pgfqpoint{2.140663in}{1.986010in}}{\pgfqpoint{2.151262in}{1.990400in}}{\pgfqpoint{2.159076in}{1.998214in}}%
\pgfpathcurveto{\pgfqpoint{2.166890in}{2.006027in}}{\pgfqpoint{2.171280in}{2.016626in}}{\pgfqpoint{2.171280in}{2.027676in}}%
\pgfpathcurveto{\pgfqpoint{2.171280in}{2.038727in}}{\pgfqpoint{2.166890in}{2.049326in}}{\pgfqpoint{2.159076in}{2.057139in}}%
\pgfpathcurveto{\pgfqpoint{2.151262in}{2.064953in}}{\pgfqpoint{2.140663in}{2.069343in}}{\pgfqpoint{2.129613in}{2.069343in}}%
\pgfpathcurveto{\pgfqpoint{2.118563in}{2.069343in}}{\pgfqpoint{2.107964in}{2.064953in}}{\pgfqpoint{2.100150in}{2.057139in}}%
\pgfpathcurveto{\pgfqpoint{2.092337in}{2.049326in}}{\pgfqpoint{2.087947in}{2.038727in}}{\pgfqpoint{2.087947in}{2.027676in}}%
\pgfpathcurveto{\pgfqpoint{2.087947in}{2.016626in}}{\pgfqpoint{2.092337in}{2.006027in}}{\pgfqpoint{2.100150in}{1.998214in}}%
\pgfpathcurveto{\pgfqpoint{2.107964in}{1.990400in}}{\pgfqpoint{2.118563in}{1.986010in}}{\pgfqpoint{2.129613in}{1.986010in}}%
\pgfpathclose%
\pgfusepath{stroke,fill}%
\end{pgfscope}%
\begin{pgfscope}%
\pgfpathrectangle{\pgfqpoint{0.600000in}{0.600000in}}{\pgfqpoint{3.900000in}{3.900000in}}%
\pgfusepath{clip}%
\pgfsetbuttcap%
\pgfsetroundjoin%
\definecolor{currentfill}{rgb}{0.121569,0.466667,0.705882}%
\pgfsetfillcolor{currentfill}%
\pgfsetlinewidth{1.003750pt}%
\definecolor{currentstroke}{rgb}{0.121569,0.466667,0.705882}%
\pgfsetstrokecolor{currentstroke}%
\pgfsetdash{}{0pt}%
\pgfpathmoveto{\pgfqpoint{2.905860in}{2.673116in}}%
\pgfpathcurveto{\pgfqpoint{2.916910in}{2.673116in}}{\pgfqpoint{2.927509in}{2.677506in}}{\pgfqpoint{2.935323in}{2.685319in}}%
\pgfpathcurveto{\pgfqpoint{2.943136in}{2.693133in}}{\pgfqpoint{2.947526in}{2.703732in}}{\pgfqpoint{2.947526in}{2.714782in}}%
\pgfpathcurveto{\pgfqpoint{2.947526in}{2.725832in}}{\pgfqpoint{2.943136in}{2.736431in}}{\pgfqpoint{2.935323in}{2.744245in}}%
\pgfpathcurveto{\pgfqpoint{2.927509in}{2.752059in}}{\pgfqpoint{2.916910in}{2.756449in}}{\pgfqpoint{2.905860in}{2.756449in}}%
\pgfpathcurveto{\pgfqpoint{2.894810in}{2.756449in}}{\pgfqpoint{2.884211in}{2.752059in}}{\pgfqpoint{2.876397in}{2.744245in}}%
\pgfpathcurveto{\pgfqpoint{2.868583in}{2.736431in}}{\pgfqpoint{2.864193in}{2.725832in}}{\pgfqpoint{2.864193in}{2.714782in}}%
\pgfpathcurveto{\pgfqpoint{2.864193in}{2.703732in}}{\pgfqpoint{2.868583in}{2.693133in}}{\pgfqpoint{2.876397in}{2.685319in}}%
\pgfpathcurveto{\pgfqpoint{2.884211in}{2.677506in}}{\pgfqpoint{2.894810in}{2.673116in}}{\pgfqpoint{2.905860in}{2.673116in}}%
\pgfpathclose%
\pgfusepath{stroke,fill}%
\end{pgfscope}%
\begin{pgfscope}%
\pgfpathrectangle{\pgfqpoint{0.600000in}{0.600000in}}{\pgfqpoint{3.900000in}{3.900000in}}%
\pgfusepath{clip}%
\pgfsetbuttcap%
\pgfsetroundjoin%
\definecolor{currentfill}{rgb}{0.121569,0.466667,0.705882}%
\pgfsetfillcolor{currentfill}%
\pgfsetlinewidth{1.003750pt}%
\definecolor{currentstroke}{rgb}{0.121569,0.466667,0.705882}%
\pgfsetstrokecolor{currentstroke}%
\pgfsetdash{}{0pt}%
\pgfpathmoveto{\pgfqpoint{2.739095in}{1.280176in}}%
\pgfpathcurveto{\pgfqpoint{2.750146in}{1.280176in}}{\pgfqpoint{2.760745in}{1.284566in}}{\pgfqpoint{2.768558in}{1.292380in}}%
\pgfpathcurveto{\pgfqpoint{2.776372in}{1.300193in}}{\pgfqpoint{2.780762in}{1.310792in}}{\pgfqpoint{2.780762in}{1.321843in}}%
\pgfpathcurveto{\pgfqpoint{2.780762in}{1.332893in}}{\pgfqpoint{2.776372in}{1.343492in}}{\pgfqpoint{2.768558in}{1.351305in}}%
\pgfpathcurveto{\pgfqpoint{2.760745in}{1.359119in}}{\pgfqpoint{2.750146in}{1.363509in}}{\pgfqpoint{2.739095in}{1.363509in}}%
\pgfpathcurveto{\pgfqpoint{2.728045in}{1.363509in}}{\pgfqpoint{2.717446in}{1.359119in}}{\pgfqpoint{2.709633in}{1.351305in}}%
\pgfpathcurveto{\pgfqpoint{2.701819in}{1.343492in}}{\pgfqpoint{2.697429in}{1.332893in}}{\pgfqpoint{2.697429in}{1.321843in}}%
\pgfpathcurveto{\pgfqpoint{2.697429in}{1.310792in}}{\pgfqpoint{2.701819in}{1.300193in}}{\pgfqpoint{2.709633in}{1.292380in}}%
\pgfpathcurveto{\pgfqpoint{2.717446in}{1.284566in}}{\pgfqpoint{2.728045in}{1.280176in}}{\pgfqpoint{2.739095in}{1.280176in}}%
\pgfpathclose%
\pgfusepath{stroke,fill}%
\end{pgfscope}%
\begin{pgfscope}%
\pgfpathrectangle{\pgfqpoint{0.600000in}{0.600000in}}{\pgfqpoint{3.900000in}{3.900000in}}%
\pgfusepath{clip}%
\pgfsetbuttcap%
\pgfsetroundjoin%
\definecolor{currentfill}{rgb}{0.121569,0.466667,0.705882}%
\pgfsetfillcolor{currentfill}%
\pgfsetlinewidth{1.003750pt}%
\definecolor{currentstroke}{rgb}{0.121569,0.466667,0.705882}%
\pgfsetstrokecolor{currentstroke}%
\pgfsetdash{}{0pt}%
\pgfpathmoveto{\pgfqpoint{2.861372in}{2.805429in}}%
\pgfpathcurveto{\pgfqpoint{2.872422in}{2.805429in}}{\pgfqpoint{2.883021in}{2.809819in}}{\pgfqpoint{2.890834in}{2.817632in}}%
\pgfpathcurveto{\pgfqpoint{2.898648in}{2.825446in}}{\pgfqpoint{2.903038in}{2.836045in}}{\pgfqpoint{2.903038in}{2.847095in}}%
\pgfpathcurveto{\pgfqpoint{2.903038in}{2.858145in}}{\pgfqpoint{2.898648in}{2.868744in}}{\pgfqpoint{2.890834in}{2.876558in}}%
\pgfpathcurveto{\pgfqpoint{2.883021in}{2.884372in}}{\pgfqpoint{2.872422in}{2.888762in}}{\pgfqpoint{2.861372in}{2.888762in}}%
\pgfpathcurveto{\pgfqpoint{2.850321in}{2.888762in}}{\pgfqpoint{2.839722in}{2.884372in}}{\pgfqpoint{2.831909in}{2.876558in}}%
\pgfpathcurveto{\pgfqpoint{2.824095in}{2.868744in}}{\pgfqpoint{2.819705in}{2.858145in}}{\pgfqpoint{2.819705in}{2.847095in}}%
\pgfpathcurveto{\pgfqpoint{2.819705in}{2.836045in}}{\pgfqpoint{2.824095in}{2.825446in}}{\pgfqpoint{2.831909in}{2.817632in}}%
\pgfpathcurveto{\pgfqpoint{2.839722in}{2.809819in}}{\pgfqpoint{2.850321in}{2.805429in}}{\pgfqpoint{2.861372in}{2.805429in}}%
\pgfpathclose%
\pgfusepath{stroke,fill}%
\end{pgfscope}%
\begin{pgfscope}%
\pgfpathrectangle{\pgfqpoint{0.600000in}{0.600000in}}{\pgfqpoint{3.900000in}{3.900000in}}%
\pgfusepath{clip}%
\pgfsetbuttcap%
\pgfsetroundjoin%
\definecolor{currentfill}{rgb}{0.121569,0.466667,0.705882}%
\pgfsetfillcolor{currentfill}%
\pgfsetlinewidth{1.003750pt}%
\definecolor{currentstroke}{rgb}{0.121569,0.466667,0.705882}%
\pgfsetstrokecolor{currentstroke}%
\pgfsetdash{}{0pt}%
\pgfpathmoveto{\pgfqpoint{3.257249in}{2.865944in}}%
\pgfpathcurveto{\pgfqpoint{3.268300in}{2.865944in}}{\pgfqpoint{3.278899in}{2.870335in}}{\pgfqpoint{3.286712in}{2.878148in}}%
\pgfpathcurveto{\pgfqpoint{3.294526in}{2.885962in}}{\pgfqpoint{3.298916in}{2.896561in}}{\pgfqpoint{3.298916in}{2.907611in}}%
\pgfpathcurveto{\pgfqpoint{3.298916in}{2.918661in}}{\pgfqpoint{3.294526in}{2.929260in}}{\pgfqpoint{3.286712in}{2.937074in}}%
\pgfpathcurveto{\pgfqpoint{3.278899in}{2.944888in}}{\pgfqpoint{3.268300in}{2.949278in}}{\pgfqpoint{3.257249in}{2.949278in}}%
\pgfpathcurveto{\pgfqpoint{3.246199in}{2.949278in}}{\pgfqpoint{3.235600in}{2.944888in}}{\pgfqpoint{3.227787in}{2.937074in}}%
\pgfpathcurveto{\pgfqpoint{3.219973in}{2.929260in}}{\pgfqpoint{3.215583in}{2.918661in}}{\pgfqpoint{3.215583in}{2.907611in}}%
\pgfpathcurveto{\pgfqpoint{3.215583in}{2.896561in}}{\pgfqpoint{3.219973in}{2.885962in}}{\pgfqpoint{3.227787in}{2.878148in}}%
\pgfpathcurveto{\pgfqpoint{3.235600in}{2.870335in}}{\pgfqpoint{3.246199in}{2.865944in}}{\pgfqpoint{3.257249in}{2.865944in}}%
\pgfpathclose%
\pgfusepath{stroke,fill}%
\end{pgfscope}%
\begin{pgfscope}%
\pgfpathrectangle{\pgfqpoint{0.600000in}{0.600000in}}{\pgfqpoint{3.900000in}{3.900000in}}%
\pgfusepath{clip}%
\pgfsetbuttcap%
\pgfsetroundjoin%
\definecolor{currentfill}{rgb}{0.121569,0.466667,0.705882}%
\pgfsetfillcolor{currentfill}%
\pgfsetlinewidth{1.003750pt}%
\definecolor{currentstroke}{rgb}{0.121569,0.466667,0.705882}%
\pgfsetstrokecolor{currentstroke}%
\pgfsetdash{}{0pt}%
\pgfpathmoveto{\pgfqpoint{2.256525in}{2.841400in}}%
\pgfpathcurveto{\pgfqpoint{2.267575in}{2.841400in}}{\pgfqpoint{2.278174in}{2.845791in}}{\pgfqpoint{2.285988in}{2.853604in}}%
\pgfpathcurveto{\pgfqpoint{2.293801in}{2.861418in}}{\pgfqpoint{2.298192in}{2.872017in}}{\pgfqpoint{2.298192in}{2.883067in}}%
\pgfpathcurveto{\pgfqpoint{2.298192in}{2.894117in}}{\pgfqpoint{2.293801in}{2.904716in}}{\pgfqpoint{2.285988in}{2.912530in}}%
\pgfpathcurveto{\pgfqpoint{2.278174in}{2.920344in}}{\pgfqpoint{2.267575in}{2.924734in}}{\pgfqpoint{2.256525in}{2.924734in}}%
\pgfpathcurveto{\pgfqpoint{2.245475in}{2.924734in}}{\pgfqpoint{2.234876in}{2.920344in}}{\pgfqpoint{2.227062in}{2.912530in}}%
\pgfpathcurveto{\pgfqpoint{2.219249in}{2.904716in}}{\pgfqpoint{2.214858in}{2.894117in}}{\pgfqpoint{2.214858in}{2.883067in}}%
\pgfpathcurveto{\pgfqpoint{2.214858in}{2.872017in}}{\pgfqpoint{2.219249in}{2.861418in}}{\pgfqpoint{2.227062in}{2.853604in}}%
\pgfpathcurveto{\pgfqpoint{2.234876in}{2.845791in}}{\pgfqpoint{2.245475in}{2.841400in}}{\pgfqpoint{2.256525in}{2.841400in}}%
\pgfpathclose%
\pgfusepath{stroke,fill}%
\end{pgfscope}%
\begin{pgfscope}%
\pgfpathrectangle{\pgfqpoint{0.600000in}{0.600000in}}{\pgfqpoint{3.900000in}{3.900000in}}%
\pgfusepath{clip}%
\pgfsetbuttcap%
\pgfsetroundjoin%
\definecolor{currentfill}{rgb}{0.121569,0.466667,0.705882}%
\pgfsetfillcolor{currentfill}%
\pgfsetlinewidth{1.003750pt}%
\definecolor{currentstroke}{rgb}{0.121569,0.466667,0.705882}%
\pgfsetstrokecolor{currentstroke}%
\pgfsetdash{}{0pt}%
\pgfpathmoveto{\pgfqpoint{2.689070in}{2.891487in}}%
\pgfpathcurveto{\pgfqpoint{2.700121in}{2.891487in}}{\pgfqpoint{2.710720in}{2.895877in}}{\pgfqpoint{2.718533in}{2.903691in}}%
\pgfpathcurveto{\pgfqpoint{2.726347in}{2.911505in}}{\pgfqpoint{2.730737in}{2.922104in}}{\pgfqpoint{2.730737in}{2.933154in}}%
\pgfpathcurveto{\pgfqpoint{2.730737in}{2.944204in}}{\pgfqpoint{2.726347in}{2.954803in}}{\pgfqpoint{2.718533in}{2.962617in}}%
\pgfpathcurveto{\pgfqpoint{2.710720in}{2.970430in}}{\pgfqpoint{2.700121in}{2.974820in}}{\pgfqpoint{2.689070in}{2.974820in}}%
\pgfpathcurveto{\pgfqpoint{2.678020in}{2.974820in}}{\pgfqpoint{2.667421in}{2.970430in}}{\pgfqpoint{2.659608in}{2.962617in}}%
\pgfpathcurveto{\pgfqpoint{2.651794in}{2.954803in}}{\pgfqpoint{2.647404in}{2.944204in}}{\pgfqpoint{2.647404in}{2.933154in}}%
\pgfpathcurveto{\pgfqpoint{2.647404in}{2.922104in}}{\pgfqpoint{2.651794in}{2.911505in}}{\pgfqpoint{2.659608in}{2.903691in}}%
\pgfpathcurveto{\pgfqpoint{2.667421in}{2.895877in}}{\pgfqpoint{2.678020in}{2.891487in}}{\pgfqpoint{2.689070in}{2.891487in}}%
\pgfpathclose%
\pgfusepath{stroke,fill}%
\end{pgfscope}%
\begin{pgfscope}%
\pgfpathrectangle{\pgfqpoint{0.600000in}{0.600000in}}{\pgfqpoint{3.900000in}{3.900000in}}%
\pgfusepath{clip}%
\pgfsetbuttcap%
\pgfsetroundjoin%
\definecolor{currentfill}{rgb}{0.121569,0.466667,0.705882}%
\pgfsetfillcolor{currentfill}%
\pgfsetlinewidth{1.003750pt}%
\definecolor{currentstroke}{rgb}{0.121569,0.466667,0.705882}%
\pgfsetstrokecolor{currentstroke}%
\pgfsetdash{}{0pt}%
\pgfpathmoveto{\pgfqpoint{3.071861in}{3.351495in}}%
\pgfpathcurveto{\pgfqpoint{3.082911in}{3.351495in}}{\pgfqpoint{3.093510in}{3.355885in}}{\pgfqpoint{3.101323in}{3.363699in}}%
\pgfpathcurveto{\pgfqpoint{3.109137in}{3.371512in}}{\pgfqpoint{3.113527in}{3.382111in}}{\pgfqpoint{3.113527in}{3.393161in}}%
\pgfpathcurveto{\pgfqpoint{3.113527in}{3.404211in}}{\pgfqpoint{3.109137in}{3.414810in}}{\pgfqpoint{3.101323in}{3.422624in}}%
\pgfpathcurveto{\pgfqpoint{3.093510in}{3.430438in}}{\pgfqpoint{3.082911in}{3.434828in}}{\pgfqpoint{3.071861in}{3.434828in}}%
\pgfpathcurveto{\pgfqpoint{3.060810in}{3.434828in}}{\pgfqpoint{3.050211in}{3.430438in}}{\pgfqpoint{3.042398in}{3.422624in}}%
\pgfpathcurveto{\pgfqpoint{3.034584in}{3.414810in}}{\pgfqpoint{3.030194in}{3.404211in}}{\pgfqpoint{3.030194in}{3.393161in}}%
\pgfpathcurveto{\pgfqpoint{3.030194in}{3.382111in}}{\pgfqpoint{3.034584in}{3.371512in}}{\pgfqpoint{3.042398in}{3.363699in}}%
\pgfpathcurveto{\pgfqpoint{3.050211in}{3.355885in}}{\pgfqpoint{3.060810in}{3.351495in}}{\pgfqpoint{3.071861in}{3.351495in}}%
\pgfpathclose%
\pgfusepath{stroke,fill}%
\end{pgfscope}%
\begin{pgfscope}%
\pgfpathrectangle{\pgfqpoint{0.600000in}{0.600000in}}{\pgfqpoint{3.900000in}{3.900000in}}%
\pgfusepath{clip}%
\pgfsetbuttcap%
\pgfsetroundjoin%
\definecolor{currentfill}{rgb}{0.121569,0.466667,0.705882}%
\pgfsetfillcolor{currentfill}%
\pgfsetlinewidth{1.003750pt}%
\definecolor{currentstroke}{rgb}{0.121569,0.466667,0.705882}%
\pgfsetstrokecolor{currentstroke}%
\pgfsetdash{}{0pt}%
\pgfpathmoveto{\pgfqpoint{2.254797in}{2.410491in}}%
\pgfpathcurveto{\pgfqpoint{2.265847in}{2.410491in}}{\pgfqpoint{2.276446in}{2.414881in}}{\pgfqpoint{2.284260in}{2.422695in}}%
\pgfpathcurveto{\pgfqpoint{2.292074in}{2.430509in}}{\pgfqpoint{2.296464in}{2.441108in}}{\pgfqpoint{2.296464in}{2.452158in}}%
\pgfpathcurveto{\pgfqpoint{2.296464in}{2.463208in}}{\pgfqpoint{2.292074in}{2.473807in}}{\pgfqpoint{2.284260in}{2.481621in}}%
\pgfpathcurveto{\pgfqpoint{2.276446in}{2.489434in}}{\pgfqpoint{2.265847in}{2.493824in}}{\pgfqpoint{2.254797in}{2.493824in}}%
\pgfpathcurveto{\pgfqpoint{2.243747in}{2.493824in}}{\pgfqpoint{2.233148in}{2.489434in}}{\pgfqpoint{2.225334in}{2.481621in}}%
\pgfpathcurveto{\pgfqpoint{2.217521in}{2.473807in}}{\pgfqpoint{2.213131in}{2.463208in}}{\pgfqpoint{2.213131in}{2.452158in}}%
\pgfpathcurveto{\pgfqpoint{2.213131in}{2.441108in}}{\pgfqpoint{2.217521in}{2.430509in}}{\pgfqpoint{2.225334in}{2.422695in}}%
\pgfpathcurveto{\pgfqpoint{2.233148in}{2.414881in}}{\pgfqpoint{2.243747in}{2.410491in}}{\pgfqpoint{2.254797in}{2.410491in}}%
\pgfpathclose%
\pgfusepath{stroke,fill}%
\end{pgfscope}%
\begin{pgfscope}%
\pgfpathrectangle{\pgfqpoint{0.600000in}{0.600000in}}{\pgfqpoint{3.900000in}{3.900000in}}%
\pgfusepath{clip}%
\pgfsetbuttcap%
\pgfsetroundjoin%
\definecolor{currentfill}{rgb}{0.121569,0.466667,0.705882}%
\pgfsetfillcolor{currentfill}%
\pgfsetlinewidth{1.003750pt}%
\definecolor{currentstroke}{rgb}{0.121569,0.466667,0.705882}%
\pgfsetstrokecolor{currentstroke}%
\pgfsetdash{}{0pt}%
\pgfpathmoveto{\pgfqpoint{1.784078in}{2.373486in}}%
\pgfpathcurveto{\pgfqpoint{1.795128in}{2.373486in}}{\pgfqpoint{1.805727in}{2.377876in}}{\pgfqpoint{1.813541in}{2.385690in}}%
\pgfpathcurveto{\pgfqpoint{1.821355in}{2.393504in}}{\pgfqpoint{1.825745in}{2.404103in}}{\pgfqpoint{1.825745in}{2.415153in}}%
\pgfpathcurveto{\pgfqpoint{1.825745in}{2.426203in}}{\pgfqpoint{1.821355in}{2.436802in}}{\pgfqpoint{1.813541in}{2.444616in}}%
\pgfpathcurveto{\pgfqpoint{1.805727in}{2.452429in}}{\pgfqpoint{1.795128in}{2.456819in}}{\pgfqpoint{1.784078in}{2.456819in}}%
\pgfpathcurveto{\pgfqpoint{1.773028in}{2.456819in}}{\pgfqpoint{1.762429in}{2.452429in}}{\pgfqpoint{1.754615in}{2.444616in}}%
\pgfpathcurveto{\pgfqpoint{1.746802in}{2.436802in}}{\pgfqpoint{1.742412in}{2.426203in}}{\pgfqpoint{1.742412in}{2.415153in}}%
\pgfpathcurveto{\pgfqpoint{1.742412in}{2.404103in}}{\pgfqpoint{1.746802in}{2.393504in}}{\pgfqpoint{1.754615in}{2.385690in}}%
\pgfpathcurveto{\pgfqpoint{1.762429in}{2.377876in}}{\pgfqpoint{1.773028in}{2.373486in}}{\pgfqpoint{1.784078in}{2.373486in}}%
\pgfpathclose%
\pgfusepath{stroke,fill}%
\end{pgfscope}%
\begin{pgfscope}%
\pgfpathrectangle{\pgfqpoint{0.600000in}{0.600000in}}{\pgfqpoint{3.900000in}{3.900000in}}%
\pgfusepath{clip}%
\pgfsetbuttcap%
\pgfsetroundjoin%
\definecolor{currentfill}{rgb}{0.121569,0.466667,0.705882}%
\pgfsetfillcolor{currentfill}%
\pgfsetlinewidth{1.003750pt}%
\definecolor{currentstroke}{rgb}{0.121569,0.466667,0.705882}%
\pgfsetstrokecolor{currentstroke}%
\pgfsetdash{}{0pt}%
\pgfpathmoveto{\pgfqpoint{2.700165in}{3.154867in}}%
\pgfpathcurveto{\pgfqpoint{2.711215in}{3.154867in}}{\pgfqpoint{2.721814in}{3.159257in}}{\pgfqpoint{2.729628in}{3.167071in}}%
\pgfpathcurveto{\pgfqpoint{2.737441in}{3.174885in}}{\pgfqpoint{2.741831in}{3.185484in}}{\pgfqpoint{2.741831in}{3.196534in}}%
\pgfpathcurveto{\pgfqpoint{2.741831in}{3.207584in}}{\pgfqpoint{2.737441in}{3.218183in}}{\pgfqpoint{2.729628in}{3.225997in}}%
\pgfpathcurveto{\pgfqpoint{2.721814in}{3.233810in}}{\pgfqpoint{2.711215in}{3.238201in}}{\pgfqpoint{2.700165in}{3.238201in}}%
\pgfpathcurveto{\pgfqpoint{2.689115in}{3.238201in}}{\pgfqpoint{2.678516in}{3.233810in}}{\pgfqpoint{2.670702in}{3.225997in}}%
\pgfpathcurveto{\pgfqpoint{2.662888in}{3.218183in}}{\pgfqpoint{2.658498in}{3.207584in}}{\pgfqpoint{2.658498in}{3.196534in}}%
\pgfpathcurveto{\pgfqpoint{2.658498in}{3.185484in}}{\pgfqpoint{2.662888in}{3.174885in}}{\pgfqpoint{2.670702in}{3.167071in}}%
\pgfpathcurveto{\pgfqpoint{2.678516in}{3.159257in}}{\pgfqpoint{2.689115in}{3.154867in}}{\pgfqpoint{2.700165in}{3.154867in}}%
\pgfpathclose%
\pgfusepath{stroke,fill}%
\end{pgfscope}%
\begin{pgfscope}%
\pgfpathrectangle{\pgfqpoint{0.600000in}{0.600000in}}{\pgfqpoint{3.900000in}{3.900000in}}%
\pgfusepath{clip}%
\pgfsetbuttcap%
\pgfsetroundjoin%
\definecolor{currentfill}{rgb}{0.121569,0.466667,0.705882}%
\pgfsetfillcolor{currentfill}%
\pgfsetlinewidth{1.003750pt}%
\definecolor{currentstroke}{rgb}{0.121569,0.466667,0.705882}%
\pgfsetstrokecolor{currentstroke}%
\pgfsetdash{}{0pt}%
\pgfpathmoveto{\pgfqpoint{2.957455in}{2.511839in}}%
\pgfpathcurveto{\pgfqpoint{2.968505in}{2.511839in}}{\pgfqpoint{2.979104in}{2.516229in}}{\pgfqpoint{2.986917in}{2.524043in}}%
\pgfpathcurveto{\pgfqpoint{2.994731in}{2.531856in}}{\pgfqpoint{2.999121in}{2.542456in}}{\pgfqpoint{2.999121in}{2.553506in}}%
\pgfpathcurveto{\pgfqpoint{2.999121in}{2.564556in}}{\pgfqpoint{2.994731in}{2.575155in}}{\pgfqpoint{2.986917in}{2.582968in}}%
\pgfpathcurveto{\pgfqpoint{2.979104in}{2.590782in}}{\pgfqpoint{2.968505in}{2.595172in}}{\pgfqpoint{2.957455in}{2.595172in}}%
\pgfpathcurveto{\pgfqpoint{2.946404in}{2.595172in}}{\pgfqpoint{2.935805in}{2.590782in}}{\pgfqpoint{2.927992in}{2.582968in}}%
\pgfpathcurveto{\pgfqpoint{2.920178in}{2.575155in}}{\pgfqpoint{2.915788in}{2.564556in}}{\pgfqpoint{2.915788in}{2.553506in}}%
\pgfpathcurveto{\pgfqpoint{2.915788in}{2.542456in}}{\pgfqpoint{2.920178in}{2.531856in}}{\pgfqpoint{2.927992in}{2.524043in}}%
\pgfpathcurveto{\pgfqpoint{2.935805in}{2.516229in}}{\pgfqpoint{2.946404in}{2.511839in}}{\pgfqpoint{2.957455in}{2.511839in}}%
\pgfpathclose%
\pgfusepath{stroke,fill}%
\end{pgfscope}%
\begin{pgfscope}%
\pgfpathrectangle{\pgfqpoint{0.600000in}{0.600000in}}{\pgfqpoint{3.900000in}{3.900000in}}%
\pgfusepath{clip}%
\pgfsetbuttcap%
\pgfsetroundjoin%
\definecolor{currentfill}{rgb}{0.121569,0.466667,0.705882}%
\pgfsetfillcolor{currentfill}%
\pgfsetlinewidth{1.003750pt}%
\definecolor{currentstroke}{rgb}{0.121569,0.466667,0.705882}%
\pgfsetstrokecolor{currentstroke}%
\pgfsetdash{}{0pt}%
\pgfpathmoveto{\pgfqpoint{2.024959in}{2.583266in}}%
\pgfpathcurveto{\pgfqpoint{2.036009in}{2.583266in}}{\pgfqpoint{2.046608in}{2.587657in}}{\pgfqpoint{2.054422in}{2.595470in}}%
\pgfpathcurveto{\pgfqpoint{2.062235in}{2.603284in}}{\pgfqpoint{2.066626in}{2.613883in}}{\pgfqpoint{2.066626in}{2.624933in}}%
\pgfpathcurveto{\pgfqpoint{2.066626in}{2.635983in}}{\pgfqpoint{2.062235in}{2.646582in}}{\pgfqpoint{2.054422in}{2.654396in}}%
\pgfpathcurveto{\pgfqpoint{2.046608in}{2.662210in}}{\pgfqpoint{2.036009in}{2.666600in}}{\pgfqpoint{2.024959in}{2.666600in}}%
\pgfpathcurveto{\pgfqpoint{2.013909in}{2.666600in}}{\pgfqpoint{2.003310in}{2.662210in}}{\pgfqpoint{1.995496in}{2.654396in}}%
\pgfpathcurveto{\pgfqpoint{1.987682in}{2.646582in}}{\pgfqpoint{1.983292in}{2.635983in}}{\pgfqpoint{1.983292in}{2.624933in}}%
\pgfpathcurveto{\pgfqpoint{1.983292in}{2.613883in}}{\pgfqpoint{1.987682in}{2.603284in}}{\pgfqpoint{1.995496in}{2.595470in}}%
\pgfpathcurveto{\pgfqpoint{2.003310in}{2.587657in}}{\pgfqpoint{2.013909in}{2.583266in}}{\pgfqpoint{2.024959in}{2.583266in}}%
\pgfpathclose%
\pgfusepath{stroke,fill}%
\end{pgfscope}%
\begin{pgfscope}%
\pgfpathrectangle{\pgfqpoint{0.600000in}{0.600000in}}{\pgfqpoint{3.900000in}{3.900000in}}%
\pgfusepath{clip}%
\pgfsetbuttcap%
\pgfsetroundjoin%
\definecolor{currentfill}{rgb}{0.121569,0.466667,0.705882}%
\pgfsetfillcolor{currentfill}%
\pgfsetlinewidth{1.003750pt}%
\definecolor{currentstroke}{rgb}{0.121569,0.466667,0.705882}%
\pgfsetstrokecolor{currentstroke}%
\pgfsetdash{}{0pt}%
\pgfpathmoveto{\pgfqpoint{2.563794in}{2.328120in}}%
\pgfpathcurveto{\pgfqpoint{2.574844in}{2.328120in}}{\pgfqpoint{2.585443in}{2.332511in}}{\pgfqpoint{2.593257in}{2.340324in}}%
\pgfpathcurveto{\pgfqpoint{2.601071in}{2.348138in}}{\pgfqpoint{2.605461in}{2.358737in}}{\pgfqpoint{2.605461in}{2.369787in}}%
\pgfpathcurveto{\pgfqpoint{2.605461in}{2.380837in}}{\pgfqpoint{2.601071in}{2.391436in}}{\pgfqpoint{2.593257in}{2.399250in}}%
\pgfpathcurveto{\pgfqpoint{2.585443in}{2.407063in}}{\pgfqpoint{2.574844in}{2.411454in}}{\pgfqpoint{2.563794in}{2.411454in}}%
\pgfpathcurveto{\pgfqpoint{2.552744in}{2.411454in}}{\pgfqpoint{2.542145in}{2.407063in}}{\pgfqpoint{2.534331in}{2.399250in}}%
\pgfpathcurveto{\pgfqpoint{2.526518in}{2.391436in}}{\pgfqpoint{2.522127in}{2.380837in}}{\pgfqpoint{2.522127in}{2.369787in}}%
\pgfpathcurveto{\pgfqpoint{2.522127in}{2.358737in}}{\pgfqpoint{2.526518in}{2.348138in}}{\pgfqpoint{2.534331in}{2.340324in}}%
\pgfpathcurveto{\pgfqpoint{2.542145in}{2.332511in}}{\pgfqpoint{2.552744in}{2.328120in}}{\pgfqpoint{2.563794in}{2.328120in}}%
\pgfpathclose%
\pgfusepath{stroke,fill}%
\end{pgfscope}%
\begin{pgfscope}%
\pgfpathrectangle{\pgfqpoint{0.600000in}{0.600000in}}{\pgfqpoint{3.900000in}{3.900000in}}%
\pgfusepath{clip}%
\pgfsetbuttcap%
\pgfsetroundjoin%
\definecolor{currentfill}{rgb}{0.121569,0.466667,0.705882}%
\pgfsetfillcolor{currentfill}%
\pgfsetlinewidth{1.003750pt}%
\definecolor{currentstroke}{rgb}{0.121569,0.466667,0.705882}%
\pgfsetstrokecolor{currentstroke}%
\pgfsetdash{}{0pt}%
\pgfpathmoveto{\pgfqpoint{3.586557in}{2.604860in}}%
\pgfpathcurveto{\pgfqpoint{3.597607in}{2.604860in}}{\pgfqpoint{3.608206in}{2.609250in}}{\pgfqpoint{3.616020in}{2.617064in}}%
\pgfpathcurveto{\pgfqpoint{3.623834in}{2.624877in}}{\pgfqpoint{3.628224in}{2.635476in}}{\pgfqpoint{3.628224in}{2.646526in}}%
\pgfpathcurveto{\pgfqpoint{3.628224in}{2.657577in}}{\pgfqpoint{3.623834in}{2.668176in}}{\pgfqpoint{3.616020in}{2.675989in}}%
\pgfpathcurveto{\pgfqpoint{3.608206in}{2.683803in}}{\pgfqpoint{3.597607in}{2.688193in}}{\pgfqpoint{3.586557in}{2.688193in}}%
\pgfpathcurveto{\pgfqpoint{3.575507in}{2.688193in}}{\pgfqpoint{3.564908in}{2.683803in}}{\pgfqpoint{3.557094in}{2.675989in}}%
\pgfpathcurveto{\pgfqpoint{3.549281in}{2.668176in}}{\pgfqpoint{3.544890in}{2.657577in}}{\pgfqpoint{3.544890in}{2.646526in}}%
\pgfpathcurveto{\pgfqpoint{3.544890in}{2.635476in}}{\pgfqpoint{3.549281in}{2.624877in}}{\pgfqpoint{3.557094in}{2.617064in}}%
\pgfpathcurveto{\pgfqpoint{3.564908in}{2.609250in}}{\pgfqpoint{3.575507in}{2.604860in}}{\pgfqpoint{3.586557in}{2.604860in}}%
\pgfpathclose%
\pgfusepath{stroke,fill}%
\end{pgfscope}%
\begin{pgfscope}%
\pgfpathrectangle{\pgfqpoint{0.600000in}{0.600000in}}{\pgfqpoint{3.900000in}{3.900000in}}%
\pgfusepath{clip}%
\pgfsetbuttcap%
\pgfsetroundjoin%
\definecolor{currentfill}{rgb}{0.121569,0.466667,0.705882}%
\pgfsetfillcolor{currentfill}%
\pgfsetlinewidth{1.003750pt}%
\definecolor{currentstroke}{rgb}{0.121569,0.466667,0.705882}%
\pgfsetstrokecolor{currentstroke}%
\pgfsetdash{}{0pt}%
\pgfpathmoveto{\pgfqpoint{2.888895in}{2.008122in}}%
\pgfpathcurveto{\pgfqpoint{2.899945in}{2.008122in}}{\pgfqpoint{2.910544in}{2.012512in}}{\pgfqpoint{2.918358in}{2.020326in}}%
\pgfpathcurveto{\pgfqpoint{2.926171in}{2.028139in}}{\pgfqpoint{2.930561in}{2.038738in}}{\pgfqpoint{2.930561in}{2.049788in}}%
\pgfpathcurveto{\pgfqpoint{2.930561in}{2.060839in}}{\pgfqpoint{2.926171in}{2.071438in}}{\pgfqpoint{2.918358in}{2.079251in}}%
\pgfpathcurveto{\pgfqpoint{2.910544in}{2.087065in}}{\pgfqpoint{2.899945in}{2.091455in}}{\pgfqpoint{2.888895in}{2.091455in}}%
\pgfpathcurveto{\pgfqpoint{2.877845in}{2.091455in}}{\pgfqpoint{2.867246in}{2.087065in}}{\pgfqpoint{2.859432in}{2.079251in}}%
\pgfpathcurveto{\pgfqpoint{2.851618in}{2.071438in}}{\pgfqpoint{2.847228in}{2.060839in}}{\pgfqpoint{2.847228in}{2.049788in}}%
\pgfpathcurveto{\pgfqpoint{2.847228in}{2.038738in}}{\pgfqpoint{2.851618in}{2.028139in}}{\pgfqpoint{2.859432in}{2.020326in}}%
\pgfpathcurveto{\pgfqpoint{2.867246in}{2.012512in}}{\pgfqpoint{2.877845in}{2.008122in}}{\pgfqpoint{2.888895in}{2.008122in}}%
\pgfpathclose%
\pgfusepath{stroke,fill}%
\end{pgfscope}%
\begin{pgfscope}%
\pgfpathrectangle{\pgfqpoint{0.600000in}{0.600000in}}{\pgfqpoint{3.900000in}{3.900000in}}%
\pgfusepath{clip}%
\pgfsetbuttcap%
\pgfsetroundjoin%
\definecolor{currentfill}{rgb}{0.121569,0.466667,0.705882}%
\pgfsetfillcolor{currentfill}%
\pgfsetlinewidth{1.003750pt}%
\definecolor{currentstroke}{rgb}{0.121569,0.466667,0.705882}%
\pgfsetstrokecolor{currentstroke}%
\pgfsetdash{}{0pt}%
\pgfpathmoveto{\pgfqpoint{2.794102in}{2.665424in}}%
\pgfpathcurveto{\pgfqpoint{2.805152in}{2.665424in}}{\pgfqpoint{2.815751in}{2.669814in}}{\pgfqpoint{2.823565in}{2.677627in}}%
\pgfpathcurveto{\pgfqpoint{2.831378in}{2.685441in}}{\pgfqpoint{2.835768in}{2.696040in}}{\pgfqpoint{2.835768in}{2.707090in}}%
\pgfpathcurveto{\pgfqpoint{2.835768in}{2.718140in}}{\pgfqpoint{2.831378in}{2.728739in}}{\pgfqpoint{2.823565in}{2.736553in}}%
\pgfpathcurveto{\pgfqpoint{2.815751in}{2.744367in}}{\pgfqpoint{2.805152in}{2.748757in}}{\pgfqpoint{2.794102in}{2.748757in}}%
\pgfpathcurveto{\pgfqpoint{2.783052in}{2.748757in}}{\pgfqpoint{2.772453in}{2.744367in}}{\pgfqpoint{2.764639in}{2.736553in}}%
\pgfpathcurveto{\pgfqpoint{2.756825in}{2.728739in}}{\pgfqpoint{2.752435in}{2.718140in}}{\pgfqpoint{2.752435in}{2.707090in}}%
\pgfpathcurveto{\pgfqpoint{2.752435in}{2.696040in}}{\pgfqpoint{2.756825in}{2.685441in}}{\pgfqpoint{2.764639in}{2.677627in}}%
\pgfpathcurveto{\pgfqpoint{2.772453in}{2.669814in}}{\pgfqpoint{2.783052in}{2.665424in}}{\pgfqpoint{2.794102in}{2.665424in}}%
\pgfpathclose%
\pgfusepath{stroke,fill}%
\end{pgfscope}%
\begin{pgfscope}%
\pgfpathrectangle{\pgfqpoint{0.600000in}{0.600000in}}{\pgfqpoint{3.900000in}{3.900000in}}%
\pgfusepath{clip}%
\pgfsetbuttcap%
\pgfsetroundjoin%
\definecolor{currentfill}{rgb}{0.121569,0.466667,0.705882}%
\pgfsetfillcolor{currentfill}%
\pgfsetlinewidth{1.003750pt}%
\definecolor{currentstroke}{rgb}{0.121569,0.466667,0.705882}%
\pgfsetstrokecolor{currentstroke}%
\pgfsetdash{}{0pt}%
\pgfpathmoveto{\pgfqpoint{2.618850in}{2.232525in}}%
\pgfpathcurveto{\pgfqpoint{2.629900in}{2.232525in}}{\pgfqpoint{2.640499in}{2.236915in}}{\pgfqpoint{2.648313in}{2.244729in}}%
\pgfpathcurveto{\pgfqpoint{2.656127in}{2.252542in}}{\pgfqpoint{2.660517in}{2.263141in}}{\pgfqpoint{2.660517in}{2.274191in}}%
\pgfpathcurveto{\pgfqpoint{2.660517in}{2.285242in}}{\pgfqpoint{2.656127in}{2.295841in}}{\pgfqpoint{2.648313in}{2.303654in}}%
\pgfpathcurveto{\pgfqpoint{2.640499in}{2.311468in}}{\pgfqpoint{2.629900in}{2.315858in}}{\pgfqpoint{2.618850in}{2.315858in}}%
\pgfpathcurveto{\pgfqpoint{2.607800in}{2.315858in}}{\pgfqpoint{2.597201in}{2.311468in}}{\pgfqpoint{2.589387in}{2.303654in}}%
\pgfpathcurveto{\pgfqpoint{2.581574in}{2.295841in}}{\pgfqpoint{2.577184in}{2.285242in}}{\pgfqpoint{2.577184in}{2.274191in}}%
\pgfpathcurveto{\pgfqpoint{2.577184in}{2.263141in}}{\pgfqpoint{2.581574in}{2.252542in}}{\pgfqpoint{2.589387in}{2.244729in}}%
\pgfpathcurveto{\pgfqpoint{2.597201in}{2.236915in}}{\pgfqpoint{2.607800in}{2.232525in}}{\pgfqpoint{2.618850in}{2.232525in}}%
\pgfpathclose%
\pgfusepath{stroke,fill}%
\end{pgfscope}%
\begin{pgfscope}%
\pgfpathrectangle{\pgfqpoint{0.600000in}{0.600000in}}{\pgfqpoint{3.900000in}{3.900000in}}%
\pgfusepath{clip}%
\pgfsetbuttcap%
\pgfsetroundjoin%
\definecolor{currentfill}{rgb}{0.121569,0.466667,0.705882}%
\pgfsetfillcolor{currentfill}%
\pgfsetlinewidth{1.003750pt}%
\definecolor{currentstroke}{rgb}{0.121569,0.466667,0.705882}%
\pgfsetstrokecolor{currentstroke}%
\pgfsetdash{}{0pt}%
\pgfpathmoveto{\pgfqpoint{2.985380in}{2.822525in}}%
\pgfpathcurveto{\pgfqpoint{2.996430in}{2.822525in}}{\pgfqpoint{3.007030in}{2.826916in}}{\pgfqpoint{3.014843in}{2.834729in}}%
\pgfpathcurveto{\pgfqpoint{3.022657in}{2.842543in}}{\pgfqpoint{3.027047in}{2.853142in}}{\pgfqpoint{3.027047in}{2.864192in}}%
\pgfpathcurveto{\pgfqpoint{3.027047in}{2.875242in}}{\pgfqpoint{3.022657in}{2.885841in}}{\pgfqpoint{3.014843in}{2.893655in}}%
\pgfpathcurveto{\pgfqpoint{3.007030in}{2.901468in}}{\pgfqpoint{2.996430in}{2.905859in}}{\pgfqpoint{2.985380in}{2.905859in}}%
\pgfpathcurveto{\pgfqpoint{2.974330in}{2.905859in}}{\pgfqpoint{2.963731in}{2.901468in}}{\pgfqpoint{2.955918in}{2.893655in}}%
\pgfpathcurveto{\pgfqpoint{2.948104in}{2.885841in}}{\pgfqpoint{2.943714in}{2.875242in}}{\pgfqpoint{2.943714in}{2.864192in}}%
\pgfpathcurveto{\pgfqpoint{2.943714in}{2.853142in}}{\pgfqpoint{2.948104in}{2.842543in}}{\pgfqpoint{2.955918in}{2.834729in}}%
\pgfpathcurveto{\pgfqpoint{2.963731in}{2.826916in}}{\pgfqpoint{2.974330in}{2.822525in}}{\pgfqpoint{2.985380in}{2.822525in}}%
\pgfpathclose%
\pgfusepath{stroke,fill}%
\end{pgfscope}%
\begin{pgfscope}%
\pgfpathrectangle{\pgfqpoint{0.600000in}{0.600000in}}{\pgfqpoint{3.900000in}{3.900000in}}%
\pgfusepath{clip}%
\pgfsetbuttcap%
\pgfsetroundjoin%
\definecolor{currentfill}{rgb}{0.121569,0.466667,0.705882}%
\pgfsetfillcolor{currentfill}%
\pgfsetlinewidth{1.003750pt}%
\definecolor{currentstroke}{rgb}{0.121569,0.466667,0.705882}%
\pgfsetstrokecolor{currentstroke}%
\pgfsetdash{}{0pt}%
\pgfpathmoveto{\pgfqpoint{2.544450in}{2.193269in}}%
\pgfpathcurveto{\pgfqpoint{2.555500in}{2.193269in}}{\pgfqpoint{2.566099in}{2.197659in}}{\pgfqpoint{2.573912in}{2.205472in}}%
\pgfpathcurveto{\pgfqpoint{2.581726in}{2.213286in}}{\pgfqpoint{2.586116in}{2.223885in}}{\pgfqpoint{2.586116in}{2.234935in}}%
\pgfpathcurveto{\pgfqpoint{2.586116in}{2.245985in}}{\pgfqpoint{2.581726in}{2.256584in}}{\pgfqpoint{2.573912in}{2.264398in}}%
\pgfpathcurveto{\pgfqpoint{2.566099in}{2.272212in}}{\pgfqpoint{2.555500in}{2.276602in}}{\pgfqpoint{2.544450in}{2.276602in}}%
\pgfpathcurveto{\pgfqpoint{2.533400in}{2.276602in}}{\pgfqpoint{2.522801in}{2.272212in}}{\pgfqpoint{2.514987in}{2.264398in}}%
\pgfpathcurveto{\pgfqpoint{2.507173in}{2.256584in}}{\pgfqpoint{2.502783in}{2.245985in}}{\pgfqpoint{2.502783in}{2.234935in}}%
\pgfpathcurveto{\pgfqpoint{2.502783in}{2.223885in}}{\pgfqpoint{2.507173in}{2.213286in}}{\pgfqpoint{2.514987in}{2.205472in}}%
\pgfpathcurveto{\pgfqpoint{2.522801in}{2.197659in}}{\pgfqpoint{2.533400in}{2.193269in}}{\pgfqpoint{2.544450in}{2.193269in}}%
\pgfpathclose%
\pgfusepath{stroke,fill}%
\end{pgfscope}%
\begin{pgfscope}%
\pgfpathrectangle{\pgfqpoint{0.600000in}{0.600000in}}{\pgfqpoint{3.900000in}{3.900000in}}%
\pgfusepath{clip}%
\pgfsetbuttcap%
\pgfsetroundjoin%
\definecolor{currentfill}{rgb}{0.121569,0.466667,0.705882}%
\pgfsetfillcolor{currentfill}%
\pgfsetlinewidth{1.003750pt}%
\definecolor{currentstroke}{rgb}{0.121569,0.466667,0.705882}%
\pgfsetstrokecolor{currentstroke}%
\pgfsetdash{}{0pt}%
\pgfpathmoveto{\pgfqpoint{2.325681in}{2.491186in}}%
\pgfpathcurveto{\pgfqpoint{2.336731in}{2.491186in}}{\pgfqpoint{2.347330in}{2.495576in}}{\pgfqpoint{2.355144in}{2.503390in}}%
\pgfpathcurveto{\pgfqpoint{2.362957in}{2.511204in}}{\pgfqpoint{2.367347in}{2.521803in}}{\pgfqpoint{2.367347in}{2.532853in}}%
\pgfpathcurveto{\pgfqpoint{2.367347in}{2.543903in}}{\pgfqpoint{2.362957in}{2.554502in}}{\pgfqpoint{2.355144in}{2.562315in}}%
\pgfpathcurveto{\pgfqpoint{2.347330in}{2.570129in}}{\pgfqpoint{2.336731in}{2.574519in}}{\pgfqpoint{2.325681in}{2.574519in}}%
\pgfpathcurveto{\pgfqpoint{2.314631in}{2.574519in}}{\pgfqpoint{2.304032in}{2.570129in}}{\pgfqpoint{2.296218in}{2.562315in}}%
\pgfpathcurveto{\pgfqpoint{2.288404in}{2.554502in}}{\pgfqpoint{2.284014in}{2.543903in}}{\pgfqpoint{2.284014in}{2.532853in}}%
\pgfpathcurveto{\pgfqpoint{2.284014in}{2.521803in}}{\pgfqpoint{2.288404in}{2.511204in}}{\pgfqpoint{2.296218in}{2.503390in}}%
\pgfpathcurveto{\pgfqpoint{2.304032in}{2.495576in}}{\pgfqpoint{2.314631in}{2.491186in}}{\pgfqpoint{2.325681in}{2.491186in}}%
\pgfpathclose%
\pgfusepath{stroke,fill}%
\end{pgfscope}%
\begin{pgfscope}%
\pgfpathrectangle{\pgfqpoint{0.600000in}{0.600000in}}{\pgfqpoint{3.900000in}{3.900000in}}%
\pgfusepath{clip}%
\pgfsetbuttcap%
\pgfsetroundjoin%
\definecolor{currentfill}{rgb}{0.121569,0.466667,0.705882}%
\pgfsetfillcolor{currentfill}%
\pgfsetlinewidth{1.003750pt}%
\definecolor{currentstroke}{rgb}{0.121569,0.466667,0.705882}%
\pgfsetstrokecolor{currentstroke}%
\pgfsetdash{}{0pt}%
\pgfpathmoveto{\pgfqpoint{2.796934in}{2.448142in}}%
\pgfpathcurveto{\pgfqpoint{2.807984in}{2.448142in}}{\pgfqpoint{2.818583in}{2.452532in}}{\pgfqpoint{2.826397in}{2.460346in}}%
\pgfpathcurveto{\pgfqpoint{2.834211in}{2.468160in}}{\pgfqpoint{2.838601in}{2.478759in}}{\pgfqpoint{2.838601in}{2.489809in}}%
\pgfpathcurveto{\pgfqpoint{2.838601in}{2.500859in}}{\pgfqpoint{2.834211in}{2.511458in}}{\pgfqpoint{2.826397in}{2.519272in}}%
\pgfpathcurveto{\pgfqpoint{2.818583in}{2.527085in}}{\pgfqpoint{2.807984in}{2.531476in}}{\pgfqpoint{2.796934in}{2.531476in}}%
\pgfpathcurveto{\pgfqpoint{2.785884in}{2.531476in}}{\pgfqpoint{2.775285in}{2.527085in}}{\pgfqpoint{2.767471in}{2.519272in}}%
\pgfpathcurveto{\pgfqpoint{2.759658in}{2.511458in}}{\pgfqpoint{2.755268in}{2.500859in}}{\pgfqpoint{2.755268in}{2.489809in}}%
\pgfpathcurveto{\pgfqpoint{2.755268in}{2.478759in}}{\pgfqpoint{2.759658in}{2.468160in}}{\pgfqpoint{2.767471in}{2.460346in}}%
\pgfpathcurveto{\pgfqpoint{2.775285in}{2.452532in}}{\pgfqpoint{2.785884in}{2.448142in}}{\pgfqpoint{2.796934in}{2.448142in}}%
\pgfpathclose%
\pgfusepath{stroke,fill}%
\end{pgfscope}%
\begin{pgfscope}%
\pgfpathrectangle{\pgfqpoint{0.600000in}{0.600000in}}{\pgfqpoint{3.900000in}{3.900000in}}%
\pgfusepath{clip}%
\pgfsetbuttcap%
\pgfsetroundjoin%
\definecolor{currentfill}{rgb}{0.121569,0.466667,0.705882}%
\pgfsetfillcolor{currentfill}%
\pgfsetlinewidth{1.003750pt}%
\definecolor{currentstroke}{rgb}{0.121569,0.466667,0.705882}%
\pgfsetstrokecolor{currentstroke}%
\pgfsetdash{}{0pt}%
\pgfpathmoveto{\pgfqpoint{2.010280in}{2.597678in}}%
\pgfpathcurveto{\pgfqpoint{2.021330in}{2.597678in}}{\pgfqpoint{2.031929in}{2.602069in}}{\pgfqpoint{2.039743in}{2.609882in}}%
\pgfpathcurveto{\pgfqpoint{2.047556in}{2.617696in}}{\pgfqpoint{2.051946in}{2.628295in}}{\pgfqpoint{2.051946in}{2.639345in}}%
\pgfpathcurveto{\pgfqpoint{2.051946in}{2.650395in}}{\pgfqpoint{2.047556in}{2.660994in}}{\pgfqpoint{2.039743in}{2.668808in}}%
\pgfpathcurveto{\pgfqpoint{2.031929in}{2.676621in}}{\pgfqpoint{2.021330in}{2.681012in}}{\pgfqpoint{2.010280in}{2.681012in}}%
\pgfpathcurveto{\pgfqpoint{1.999230in}{2.681012in}}{\pgfqpoint{1.988631in}{2.676621in}}{\pgfqpoint{1.980817in}{2.668808in}}%
\pgfpathcurveto{\pgfqpoint{1.973003in}{2.660994in}}{\pgfqpoint{1.968613in}{2.650395in}}{\pgfqpoint{1.968613in}{2.639345in}}%
\pgfpathcurveto{\pgfqpoint{1.968613in}{2.628295in}}{\pgfqpoint{1.973003in}{2.617696in}}{\pgfqpoint{1.980817in}{2.609882in}}%
\pgfpathcurveto{\pgfqpoint{1.988631in}{2.602069in}}{\pgfqpoint{1.999230in}{2.597678in}}{\pgfqpoint{2.010280in}{2.597678in}}%
\pgfpathclose%
\pgfusepath{stroke,fill}%
\end{pgfscope}%
\begin{pgfscope}%
\pgfpathrectangle{\pgfqpoint{0.600000in}{0.600000in}}{\pgfqpoint{3.900000in}{3.900000in}}%
\pgfusepath{clip}%
\pgfsetbuttcap%
\pgfsetroundjoin%
\definecolor{currentfill}{rgb}{0.121569,0.466667,0.705882}%
\pgfsetfillcolor{currentfill}%
\pgfsetlinewidth{1.003750pt}%
\definecolor{currentstroke}{rgb}{0.121569,0.466667,0.705882}%
\pgfsetstrokecolor{currentstroke}%
\pgfsetdash{}{0pt}%
\pgfpathmoveto{\pgfqpoint{2.152556in}{1.958797in}}%
\pgfpathcurveto{\pgfqpoint{2.163606in}{1.958797in}}{\pgfqpoint{2.174205in}{1.963187in}}{\pgfqpoint{2.182019in}{1.971001in}}%
\pgfpathcurveto{\pgfqpoint{2.189833in}{1.978814in}}{\pgfqpoint{2.194223in}{1.989413in}}{\pgfqpoint{2.194223in}{2.000463in}}%
\pgfpathcurveto{\pgfqpoint{2.194223in}{2.011513in}}{\pgfqpoint{2.189833in}{2.022112in}}{\pgfqpoint{2.182019in}{2.029926in}}%
\pgfpathcurveto{\pgfqpoint{2.174205in}{2.037740in}}{\pgfqpoint{2.163606in}{2.042130in}}{\pgfqpoint{2.152556in}{2.042130in}}%
\pgfpathcurveto{\pgfqpoint{2.141506in}{2.042130in}}{\pgfqpoint{2.130907in}{2.037740in}}{\pgfqpoint{2.123093in}{2.029926in}}%
\pgfpathcurveto{\pgfqpoint{2.115280in}{2.022112in}}{\pgfqpoint{2.110889in}{2.011513in}}{\pgfqpoint{2.110889in}{2.000463in}}%
\pgfpathcurveto{\pgfqpoint{2.110889in}{1.989413in}}{\pgfqpoint{2.115280in}{1.978814in}}{\pgfqpoint{2.123093in}{1.971001in}}%
\pgfpathcurveto{\pgfqpoint{2.130907in}{1.963187in}}{\pgfqpoint{2.141506in}{1.958797in}}{\pgfqpoint{2.152556in}{1.958797in}}%
\pgfpathclose%
\pgfusepath{stroke,fill}%
\end{pgfscope}%
\begin{pgfscope}%
\pgfpathrectangle{\pgfqpoint{0.600000in}{0.600000in}}{\pgfqpoint{3.900000in}{3.900000in}}%
\pgfusepath{clip}%
\pgfsetbuttcap%
\pgfsetroundjoin%
\definecolor{currentfill}{rgb}{0.121569,0.466667,0.705882}%
\pgfsetfillcolor{currentfill}%
\pgfsetlinewidth{1.003750pt}%
\definecolor{currentstroke}{rgb}{0.121569,0.466667,0.705882}%
\pgfsetstrokecolor{currentstroke}%
\pgfsetdash{}{0pt}%
\pgfpathmoveto{\pgfqpoint{2.327642in}{2.375062in}}%
\pgfpathcurveto{\pgfqpoint{2.338692in}{2.375062in}}{\pgfqpoint{2.349291in}{2.379453in}}{\pgfqpoint{2.357105in}{2.387266in}}%
\pgfpathcurveto{\pgfqpoint{2.364918in}{2.395080in}}{\pgfqpoint{2.369308in}{2.405679in}}{\pgfqpoint{2.369308in}{2.416729in}}%
\pgfpathcurveto{\pgfqpoint{2.369308in}{2.427779in}}{\pgfqpoint{2.364918in}{2.438378in}}{\pgfqpoint{2.357105in}{2.446192in}}%
\pgfpathcurveto{\pgfqpoint{2.349291in}{2.454005in}}{\pgfqpoint{2.338692in}{2.458396in}}{\pgfqpoint{2.327642in}{2.458396in}}%
\pgfpathcurveto{\pgfqpoint{2.316592in}{2.458396in}}{\pgfqpoint{2.305993in}{2.454005in}}{\pgfqpoint{2.298179in}{2.446192in}}%
\pgfpathcurveto{\pgfqpoint{2.290365in}{2.438378in}}{\pgfqpoint{2.285975in}{2.427779in}}{\pgfqpoint{2.285975in}{2.416729in}}%
\pgfpathcurveto{\pgfqpoint{2.285975in}{2.405679in}}{\pgfqpoint{2.290365in}{2.395080in}}{\pgfqpoint{2.298179in}{2.387266in}}%
\pgfpathcurveto{\pgfqpoint{2.305993in}{2.379453in}}{\pgfqpoint{2.316592in}{2.375062in}}{\pgfqpoint{2.327642in}{2.375062in}}%
\pgfpathclose%
\pgfusepath{stroke,fill}%
\end{pgfscope}%
\begin{pgfscope}%
\pgfpathrectangle{\pgfqpoint{0.600000in}{0.600000in}}{\pgfqpoint{3.900000in}{3.900000in}}%
\pgfusepath{clip}%
\pgfsetbuttcap%
\pgfsetroundjoin%
\definecolor{currentfill}{rgb}{0.121569,0.466667,0.705882}%
\pgfsetfillcolor{currentfill}%
\pgfsetlinewidth{1.003750pt}%
\definecolor{currentstroke}{rgb}{0.121569,0.466667,0.705882}%
\pgfsetstrokecolor{currentstroke}%
\pgfsetdash{}{0pt}%
\pgfpathmoveto{\pgfqpoint{3.439629in}{2.504681in}}%
\pgfpathcurveto{\pgfqpoint{3.450680in}{2.504681in}}{\pgfqpoint{3.461279in}{2.509071in}}{\pgfqpoint{3.469092in}{2.516885in}}%
\pgfpathcurveto{\pgfqpoint{3.476906in}{2.524699in}}{\pgfqpoint{3.481296in}{2.535298in}}{\pgfqpoint{3.481296in}{2.546348in}}%
\pgfpathcurveto{\pgfqpoint{3.481296in}{2.557398in}}{\pgfqpoint{3.476906in}{2.567997in}}{\pgfqpoint{3.469092in}{2.575811in}}%
\pgfpathcurveto{\pgfqpoint{3.461279in}{2.583624in}}{\pgfqpoint{3.450680in}{2.588014in}}{\pgfqpoint{3.439629in}{2.588014in}}%
\pgfpathcurveto{\pgfqpoint{3.428579in}{2.588014in}}{\pgfqpoint{3.417980in}{2.583624in}}{\pgfqpoint{3.410167in}{2.575811in}}%
\pgfpathcurveto{\pgfqpoint{3.402353in}{2.567997in}}{\pgfqpoint{3.397963in}{2.557398in}}{\pgfqpoint{3.397963in}{2.546348in}}%
\pgfpathcurveto{\pgfqpoint{3.397963in}{2.535298in}}{\pgfqpoint{3.402353in}{2.524699in}}{\pgfqpoint{3.410167in}{2.516885in}}%
\pgfpathcurveto{\pgfqpoint{3.417980in}{2.509071in}}{\pgfqpoint{3.428579in}{2.504681in}}{\pgfqpoint{3.439629in}{2.504681in}}%
\pgfpathclose%
\pgfusepath{stroke,fill}%
\end{pgfscope}%
\begin{pgfscope}%
\pgfpathrectangle{\pgfqpoint{0.600000in}{0.600000in}}{\pgfqpoint{3.900000in}{3.900000in}}%
\pgfusepath{clip}%
\pgfsetbuttcap%
\pgfsetroundjoin%
\definecolor{currentfill}{rgb}{0.121569,0.466667,0.705882}%
\pgfsetfillcolor{currentfill}%
\pgfsetlinewidth{1.003750pt}%
\definecolor{currentstroke}{rgb}{0.121569,0.466667,0.705882}%
\pgfsetstrokecolor{currentstroke}%
\pgfsetdash{}{0pt}%
\pgfpathmoveto{\pgfqpoint{2.220227in}{1.581452in}}%
\pgfpathcurveto{\pgfqpoint{2.231277in}{1.581452in}}{\pgfqpoint{2.241876in}{1.585843in}}{\pgfqpoint{2.249690in}{1.593656in}}%
\pgfpathcurveto{\pgfqpoint{2.257503in}{1.601470in}}{\pgfqpoint{2.261894in}{1.612069in}}{\pgfqpoint{2.261894in}{1.623119in}}%
\pgfpathcurveto{\pgfqpoint{2.261894in}{1.634169in}}{\pgfqpoint{2.257503in}{1.644768in}}{\pgfqpoint{2.249690in}{1.652582in}}%
\pgfpathcurveto{\pgfqpoint{2.241876in}{1.660395in}}{\pgfqpoint{2.231277in}{1.664786in}}{\pgfqpoint{2.220227in}{1.664786in}}%
\pgfpathcurveto{\pgfqpoint{2.209177in}{1.664786in}}{\pgfqpoint{2.198578in}{1.660395in}}{\pgfqpoint{2.190764in}{1.652582in}}%
\pgfpathcurveto{\pgfqpoint{2.182950in}{1.644768in}}{\pgfqpoint{2.178560in}{1.634169in}}{\pgfqpoint{2.178560in}{1.623119in}}%
\pgfpathcurveto{\pgfqpoint{2.178560in}{1.612069in}}{\pgfqpoint{2.182950in}{1.601470in}}{\pgfqpoint{2.190764in}{1.593656in}}%
\pgfpathcurveto{\pgfqpoint{2.198578in}{1.585843in}}{\pgfqpoint{2.209177in}{1.581452in}}{\pgfqpoint{2.220227in}{1.581452in}}%
\pgfpathclose%
\pgfusepath{stroke,fill}%
\end{pgfscope}%
\begin{pgfscope}%
\pgfpathrectangle{\pgfqpoint{0.600000in}{0.600000in}}{\pgfqpoint{3.900000in}{3.900000in}}%
\pgfusepath{clip}%
\pgfsetbuttcap%
\pgfsetroundjoin%
\definecolor{currentfill}{rgb}{0.121569,0.466667,0.705882}%
\pgfsetfillcolor{currentfill}%
\pgfsetlinewidth{1.003750pt}%
\definecolor{currentstroke}{rgb}{0.121569,0.466667,0.705882}%
\pgfsetstrokecolor{currentstroke}%
\pgfsetdash{}{0pt}%
\pgfpathmoveto{\pgfqpoint{2.980463in}{2.470930in}}%
\pgfpathcurveto{\pgfqpoint{2.991514in}{2.470930in}}{\pgfqpoint{3.002113in}{2.475321in}}{\pgfqpoint{3.009926in}{2.483134in}}%
\pgfpathcurveto{\pgfqpoint{3.017740in}{2.490948in}}{\pgfqpoint{3.022130in}{2.501547in}}{\pgfqpoint{3.022130in}{2.512597in}}%
\pgfpathcurveto{\pgfqpoint{3.022130in}{2.523647in}}{\pgfqpoint{3.017740in}{2.534246in}}{\pgfqpoint{3.009926in}{2.542060in}}%
\pgfpathcurveto{\pgfqpoint{3.002113in}{2.549873in}}{\pgfqpoint{2.991514in}{2.554264in}}{\pgfqpoint{2.980463in}{2.554264in}}%
\pgfpathcurveto{\pgfqpoint{2.969413in}{2.554264in}}{\pgfqpoint{2.958814in}{2.549873in}}{\pgfqpoint{2.951001in}{2.542060in}}%
\pgfpathcurveto{\pgfqpoint{2.943187in}{2.534246in}}{\pgfqpoint{2.938797in}{2.523647in}}{\pgfqpoint{2.938797in}{2.512597in}}%
\pgfpathcurveto{\pgfqpoint{2.938797in}{2.501547in}}{\pgfqpoint{2.943187in}{2.490948in}}{\pgfqpoint{2.951001in}{2.483134in}}%
\pgfpathcurveto{\pgfqpoint{2.958814in}{2.475321in}}{\pgfqpoint{2.969413in}{2.470930in}}{\pgfqpoint{2.980463in}{2.470930in}}%
\pgfpathclose%
\pgfusepath{stroke,fill}%
\end{pgfscope}%
\begin{pgfscope}%
\pgfpathrectangle{\pgfqpoint{0.600000in}{0.600000in}}{\pgfqpoint{3.900000in}{3.900000in}}%
\pgfusepath{clip}%
\pgfsetbuttcap%
\pgfsetroundjoin%
\definecolor{currentfill}{rgb}{0.121569,0.466667,0.705882}%
\pgfsetfillcolor{currentfill}%
\pgfsetlinewidth{1.003750pt}%
\definecolor{currentstroke}{rgb}{0.121569,0.466667,0.705882}%
\pgfsetstrokecolor{currentstroke}%
\pgfsetdash{}{0pt}%
\pgfpathmoveto{\pgfqpoint{2.281304in}{2.113597in}}%
\pgfpathcurveto{\pgfqpoint{2.292354in}{2.113597in}}{\pgfqpoint{2.302953in}{2.117987in}}{\pgfqpoint{2.310767in}{2.125800in}}%
\pgfpathcurveto{\pgfqpoint{2.318581in}{2.133614in}}{\pgfqpoint{2.322971in}{2.144213in}}{\pgfqpoint{2.322971in}{2.155263in}}%
\pgfpathcurveto{\pgfqpoint{2.322971in}{2.166313in}}{\pgfqpoint{2.318581in}{2.176912in}}{\pgfqpoint{2.310767in}{2.184726in}}%
\pgfpathcurveto{\pgfqpoint{2.302953in}{2.192540in}}{\pgfqpoint{2.292354in}{2.196930in}}{\pgfqpoint{2.281304in}{2.196930in}}%
\pgfpathcurveto{\pgfqpoint{2.270254in}{2.196930in}}{\pgfqpoint{2.259655in}{2.192540in}}{\pgfqpoint{2.251841in}{2.184726in}}%
\pgfpathcurveto{\pgfqpoint{2.244028in}{2.176912in}}{\pgfqpoint{2.239638in}{2.166313in}}{\pgfqpoint{2.239638in}{2.155263in}}%
\pgfpathcurveto{\pgfqpoint{2.239638in}{2.144213in}}{\pgfqpoint{2.244028in}{2.133614in}}{\pgfqpoint{2.251841in}{2.125800in}}%
\pgfpathcurveto{\pgfqpoint{2.259655in}{2.117987in}}{\pgfqpoint{2.270254in}{2.113597in}}{\pgfqpoint{2.281304in}{2.113597in}}%
\pgfpathclose%
\pgfusepath{stroke,fill}%
\end{pgfscope}%
\begin{pgfscope}%
\pgfpathrectangle{\pgfqpoint{0.600000in}{0.600000in}}{\pgfqpoint{3.900000in}{3.900000in}}%
\pgfusepath{clip}%
\pgfsetbuttcap%
\pgfsetroundjoin%
\definecolor{currentfill}{rgb}{0.121569,0.466667,0.705882}%
\pgfsetfillcolor{currentfill}%
\pgfsetlinewidth{1.003750pt}%
\definecolor{currentstroke}{rgb}{0.121569,0.466667,0.705882}%
\pgfsetstrokecolor{currentstroke}%
\pgfsetdash{}{0pt}%
\pgfpathmoveto{\pgfqpoint{2.425851in}{2.236703in}}%
\pgfpathcurveto{\pgfqpoint{2.436901in}{2.236703in}}{\pgfqpoint{2.447500in}{2.241093in}}{\pgfqpoint{2.455314in}{2.248907in}}%
\pgfpathcurveto{\pgfqpoint{2.463127in}{2.256721in}}{\pgfqpoint{2.467518in}{2.267320in}}{\pgfqpoint{2.467518in}{2.278370in}}%
\pgfpathcurveto{\pgfqpoint{2.467518in}{2.289420in}}{\pgfqpoint{2.463127in}{2.300019in}}{\pgfqpoint{2.455314in}{2.307832in}}%
\pgfpathcurveto{\pgfqpoint{2.447500in}{2.315646in}}{\pgfqpoint{2.436901in}{2.320036in}}{\pgfqpoint{2.425851in}{2.320036in}}%
\pgfpathcurveto{\pgfqpoint{2.414801in}{2.320036in}}{\pgfqpoint{2.404202in}{2.315646in}}{\pgfqpoint{2.396388in}{2.307832in}}%
\pgfpathcurveto{\pgfqpoint{2.388574in}{2.300019in}}{\pgfqpoint{2.384184in}{2.289420in}}{\pgfqpoint{2.384184in}{2.278370in}}%
\pgfpathcurveto{\pgfqpoint{2.384184in}{2.267320in}}{\pgfqpoint{2.388574in}{2.256721in}}{\pgfqpoint{2.396388in}{2.248907in}}%
\pgfpathcurveto{\pgfqpoint{2.404202in}{2.241093in}}{\pgfqpoint{2.414801in}{2.236703in}}{\pgfqpoint{2.425851in}{2.236703in}}%
\pgfpathclose%
\pgfusepath{stroke,fill}%
\end{pgfscope}%
\begin{pgfscope}%
\pgfpathrectangle{\pgfqpoint{0.600000in}{0.600000in}}{\pgfqpoint{3.900000in}{3.900000in}}%
\pgfusepath{clip}%
\pgfsetbuttcap%
\pgfsetroundjoin%
\definecolor{currentfill}{rgb}{0.121569,0.466667,0.705882}%
\pgfsetfillcolor{currentfill}%
\pgfsetlinewidth{1.003750pt}%
\definecolor{currentstroke}{rgb}{0.121569,0.466667,0.705882}%
\pgfsetstrokecolor{currentstroke}%
\pgfsetdash{}{0pt}%
\pgfpathmoveto{\pgfqpoint{2.487986in}{2.618404in}}%
\pgfpathcurveto{\pgfqpoint{2.499036in}{2.618404in}}{\pgfqpoint{2.509635in}{2.622794in}}{\pgfqpoint{2.517449in}{2.630608in}}%
\pgfpathcurveto{\pgfqpoint{2.525262in}{2.638421in}}{\pgfqpoint{2.529653in}{2.649020in}}{\pgfqpoint{2.529653in}{2.660070in}}%
\pgfpathcurveto{\pgfqpoint{2.529653in}{2.671120in}}{\pgfqpoint{2.525262in}{2.681719in}}{\pgfqpoint{2.517449in}{2.689533in}}%
\pgfpathcurveto{\pgfqpoint{2.509635in}{2.697347in}}{\pgfqpoint{2.499036in}{2.701737in}}{\pgfqpoint{2.487986in}{2.701737in}}%
\pgfpathcurveto{\pgfqpoint{2.476936in}{2.701737in}}{\pgfqpoint{2.466337in}{2.697347in}}{\pgfqpoint{2.458523in}{2.689533in}}%
\pgfpathcurveto{\pgfqpoint{2.450710in}{2.681719in}}{\pgfqpoint{2.446319in}{2.671120in}}{\pgfqpoint{2.446319in}{2.660070in}}%
\pgfpathcurveto{\pgfqpoint{2.446319in}{2.649020in}}{\pgfqpoint{2.450710in}{2.638421in}}{\pgfqpoint{2.458523in}{2.630608in}}%
\pgfpathcurveto{\pgfqpoint{2.466337in}{2.622794in}}{\pgfqpoint{2.476936in}{2.618404in}}{\pgfqpoint{2.487986in}{2.618404in}}%
\pgfpathclose%
\pgfusepath{stroke,fill}%
\end{pgfscope}%
\begin{pgfscope}%
\pgfpathrectangle{\pgfqpoint{0.600000in}{0.600000in}}{\pgfqpoint{3.900000in}{3.900000in}}%
\pgfusepath{clip}%
\pgfsetbuttcap%
\pgfsetroundjoin%
\definecolor{currentfill}{rgb}{0.121569,0.466667,0.705882}%
\pgfsetfillcolor{currentfill}%
\pgfsetlinewidth{1.003750pt}%
\definecolor{currentstroke}{rgb}{0.121569,0.466667,0.705882}%
\pgfsetstrokecolor{currentstroke}%
\pgfsetdash{}{0pt}%
\pgfpathmoveto{\pgfqpoint{2.734801in}{2.796978in}}%
\pgfpathcurveto{\pgfqpoint{2.745851in}{2.796978in}}{\pgfqpoint{2.756450in}{2.801368in}}{\pgfqpoint{2.764264in}{2.809182in}}%
\pgfpathcurveto{\pgfqpoint{2.772077in}{2.816995in}}{\pgfqpoint{2.776468in}{2.827594in}}{\pgfqpoint{2.776468in}{2.838645in}}%
\pgfpathcurveto{\pgfqpoint{2.776468in}{2.849695in}}{\pgfqpoint{2.772077in}{2.860294in}}{\pgfqpoint{2.764264in}{2.868107in}}%
\pgfpathcurveto{\pgfqpoint{2.756450in}{2.875921in}}{\pgfqpoint{2.745851in}{2.880311in}}{\pgfqpoint{2.734801in}{2.880311in}}%
\pgfpathcurveto{\pgfqpoint{2.723751in}{2.880311in}}{\pgfqpoint{2.713152in}{2.875921in}}{\pgfqpoint{2.705338in}{2.868107in}}%
\pgfpathcurveto{\pgfqpoint{2.697525in}{2.860294in}}{\pgfqpoint{2.693134in}{2.849695in}}{\pgfqpoint{2.693134in}{2.838645in}}%
\pgfpathcurveto{\pgfqpoint{2.693134in}{2.827594in}}{\pgfqpoint{2.697525in}{2.816995in}}{\pgfqpoint{2.705338in}{2.809182in}}%
\pgfpathcurveto{\pgfqpoint{2.713152in}{2.801368in}}{\pgfqpoint{2.723751in}{2.796978in}}{\pgfqpoint{2.734801in}{2.796978in}}%
\pgfpathclose%
\pgfusepath{stroke,fill}%
\end{pgfscope}%
\begin{pgfscope}%
\pgfpathrectangle{\pgfqpoint{0.600000in}{0.600000in}}{\pgfqpoint{3.900000in}{3.900000in}}%
\pgfusepath{clip}%
\pgfsetbuttcap%
\pgfsetroundjoin%
\definecolor{currentfill}{rgb}{0.121569,0.466667,0.705882}%
\pgfsetfillcolor{currentfill}%
\pgfsetlinewidth{1.003750pt}%
\definecolor{currentstroke}{rgb}{0.121569,0.466667,0.705882}%
\pgfsetstrokecolor{currentstroke}%
\pgfsetdash{}{0pt}%
\pgfpathmoveto{\pgfqpoint{2.327750in}{2.340902in}}%
\pgfpathcurveto{\pgfqpoint{2.338800in}{2.340902in}}{\pgfqpoint{2.349399in}{2.345292in}}{\pgfqpoint{2.357213in}{2.353106in}}%
\pgfpathcurveto{\pgfqpoint{2.365026in}{2.360920in}}{\pgfqpoint{2.369417in}{2.371519in}}{\pgfqpoint{2.369417in}{2.382569in}}%
\pgfpathcurveto{\pgfqpoint{2.369417in}{2.393619in}}{\pgfqpoint{2.365026in}{2.404218in}}{\pgfqpoint{2.357213in}{2.412031in}}%
\pgfpathcurveto{\pgfqpoint{2.349399in}{2.419845in}}{\pgfqpoint{2.338800in}{2.424235in}}{\pgfqpoint{2.327750in}{2.424235in}}%
\pgfpathcurveto{\pgfqpoint{2.316700in}{2.424235in}}{\pgfqpoint{2.306101in}{2.419845in}}{\pgfqpoint{2.298287in}{2.412031in}}%
\pgfpathcurveto{\pgfqpoint{2.290473in}{2.404218in}}{\pgfqpoint{2.286083in}{2.393619in}}{\pgfqpoint{2.286083in}{2.382569in}}%
\pgfpathcurveto{\pgfqpoint{2.286083in}{2.371519in}}{\pgfqpoint{2.290473in}{2.360920in}}{\pgfqpoint{2.298287in}{2.353106in}}%
\pgfpathcurveto{\pgfqpoint{2.306101in}{2.345292in}}{\pgfqpoint{2.316700in}{2.340902in}}{\pgfqpoint{2.327750in}{2.340902in}}%
\pgfpathclose%
\pgfusepath{stroke,fill}%
\end{pgfscope}%
\begin{pgfscope}%
\pgfpathrectangle{\pgfqpoint{0.600000in}{0.600000in}}{\pgfqpoint{3.900000in}{3.900000in}}%
\pgfusepath{clip}%
\pgfsetbuttcap%
\pgfsetroundjoin%
\definecolor{currentfill}{rgb}{0.121569,0.466667,0.705882}%
\pgfsetfillcolor{currentfill}%
\pgfsetlinewidth{1.003750pt}%
\definecolor{currentstroke}{rgb}{0.121569,0.466667,0.705882}%
\pgfsetstrokecolor{currentstroke}%
\pgfsetdash{}{0pt}%
\pgfpathmoveto{\pgfqpoint{3.221335in}{2.429334in}}%
\pgfpathcurveto{\pgfqpoint{3.232385in}{2.429334in}}{\pgfqpoint{3.242984in}{2.433724in}}{\pgfqpoint{3.250797in}{2.441537in}}%
\pgfpathcurveto{\pgfqpoint{3.258611in}{2.449351in}}{\pgfqpoint{3.263001in}{2.459950in}}{\pgfqpoint{3.263001in}{2.471000in}}%
\pgfpathcurveto{\pgfqpoint{3.263001in}{2.482050in}}{\pgfqpoint{3.258611in}{2.492649in}}{\pgfqpoint{3.250797in}{2.500463in}}%
\pgfpathcurveto{\pgfqpoint{3.242984in}{2.508277in}}{\pgfqpoint{3.232385in}{2.512667in}}{\pgfqpoint{3.221335in}{2.512667in}}%
\pgfpathcurveto{\pgfqpoint{3.210285in}{2.512667in}}{\pgfqpoint{3.199685in}{2.508277in}}{\pgfqpoint{3.191872in}{2.500463in}}%
\pgfpathcurveto{\pgfqpoint{3.184058in}{2.492649in}}{\pgfqpoint{3.179668in}{2.482050in}}{\pgfqpoint{3.179668in}{2.471000in}}%
\pgfpathcurveto{\pgfqpoint{3.179668in}{2.459950in}}{\pgfqpoint{3.184058in}{2.449351in}}{\pgfqpoint{3.191872in}{2.441537in}}%
\pgfpathcurveto{\pgfqpoint{3.199685in}{2.433724in}}{\pgfqpoint{3.210285in}{2.429334in}}{\pgfqpoint{3.221335in}{2.429334in}}%
\pgfpathclose%
\pgfusepath{stroke,fill}%
\end{pgfscope}%
\begin{pgfscope}%
\pgfpathrectangle{\pgfqpoint{0.600000in}{0.600000in}}{\pgfqpoint{3.900000in}{3.900000in}}%
\pgfusepath{clip}%
\pgfsetbuttcap%
\pgfsetroundjoin%
\definecolor{currentfill}{rgb}{0.121569,0.466667,0.705882}%
\pgfsetfillcolor{currentfill}%
\pgfsetlinewidth{1.003750pt}%
\definecolor{currentstroke}{rgb}{0.121569,0.466667,0.705882}%
\pgfsetstrokecolor{currentstroke}%
\pgfsetdash{}{0pt}%
\pgfpathmoveto{\pgfqpoint{2.284864in}{2.569285in}}%
\pgfpathcurveto{\pgfqpoint{2.295914in}{2.569285in}}{\pgfqpoint{2.306513in}{2.573675in}}{\pgfqpoint{2.314327in}{2.581489in}}%
\pgfpathcurveto{\pgfqpoint{2.322140in}{2.589302in}}{\pgfqpoint{2.326531in}{2.599901in}}{\pgfqpoint{2.326531in}{2.610952in}}%
\pgfpathcurveto{\pgfqpoint{2.326531in}{2.622002in}}{\pgfqpoint{2.322140in}{2.632601in}}{\pgfqpoint{2.314327in}{2.640414in}}%
\pgfpathcurveto{\pgfqpoint{2.306513in}{2.648228in}}{\pgfqpoint{2.295914in}{2.652618in}}{\pgfqpoint{2.284864in}{2.652618in}}%
\pgfpathcurveto{\pgfqpoint{2.273814in}{2.652618in}}{\pgfqpoint{2.263215in}{2.648228in}}{\pgfqpoint{2.255401in}{2.640414in}}%
\pgfpathcurveto{\pgfqpoint{2.247587in}{2.632601in}}{\pgfqpoint{2.243197in}{2.622002in}}{\pgfqpoint{2.243197in}{2.610952in}}%
\pgfpathcurveto{\pgfqpoint{2.243197in}{2.599901in}}{\pgfqpoint{2.247587in}{2.589302in}}{\pgfqpoint{2.255401in}{2.581489in}}%
\pgfpathcurveto{\pgfqpoint{2.263215in}{2.573675in}}{\pgfqpoint{2.273814in}{2.569285in}}{\pgfqpoint{2.284864in}{2.569285in}}%
\pgfpathclose%
\pgfusepath{stroke,fill}%
\end{pgfscope}%
\begin{pgfscope}%
\pgfpathrectangle{\pgfqpoint{0.600000in}{0.600000in}}{\pgfqpoint{3.900000in}{3.900000in}}%
\pgfusepath{clip}%
\pgfsetbuttcap%
\pgfsetroundjoin%
\definecolor{currentfill}{rgb}{0.121569,0.466667,0.705882}%
\pgfsetfillcolor{currentfill}%
\pgfsetlinewidth{1.003750pt}%
\definecolor{currentstroke}{rgb}{0.121569,0.466667,0.705882}%
\pgfsetstrokecolor{currentstroke}%
\pgfsetdash{}{0pt}%
\pgfpathmoveto{\pgfqpoint{3.030862in}{1.797298in}}%
\pgfpathcurveto{\pgfqpoint{3.041912in}{1.797298in}}{\pgfqpoint{3.052512in}{1.801688in}}{\pgfqpoint{3.060325in}{1.809501in}}%
\pgfpathcurveto{\pgfqpoint{3.068139in}{1.817315in}}{\pgfqpoint{3.072529in}{1.827914in}}{\pgfqpoint{3.072529in}{1.838964in}}%
\pgfpathcurveto{\pgfqpoint{3.072529in}{1.850014in}}{\pgfqpoint{3.068139in}{1.860613in}}{\pgfqpoint{3.060325in}{1.868427in}}%
\pgfpathcurveto{\pgfqpoint{3.052512in}{1.876241in}}{\pgfqpoint{3.041912in}{1.880631in}}{\pgfqpoint{3.030862in}{1.880631in}}%
\pgfpathcurveto{\pgfqpoint{3.019812in}{1.880631in}}{\pgfqpoint{3.009213in}{1.876241in}}{\pgfqpoint{3.001400in}{1.868427in}}%
\pgfpathcurveto{\pgfqpoint{2.993586in}{1.860613in}}{\pgfqpoint{2.989196in}{1.850014in}}{\pgfqpoint{2.989196in}{1.838964in}}%
\pgfpathcurveto{\pgfqpoint{2.989196in}{1.827914in}}{\pgfqpoint{2.993586in}{1.817315in}}{\pgfqpoint{3.001400in}{1.809501in}}%
\pgfpathcurveto{\pgfqpoint{3.009213in}{1.801688in}}{\pgfqpoint{3.019812in}{1.797298in}}{\pgfqpoint{3.030862in}{1.797298in}}%
\pgfpathclose%
\pgfusepath{stroke,fill}%
\end{pgfscope}%
\begin{pgfscope}%
\pgfpathrectangle{\pgfqpoint{0.600000in}{0.600000in}}{\pgfqpoint{3.900000in}{3.900000in}}%
\pgfusepath{clip}%
\pgfsetbuttcap%
\pgfsetroundjoin%
\definecolor{currentfill}{rgb}{0.121569,0.466667,0.705882}%
\pgfsetfillcolor{currentfill}%
\pgfsetlinewidth{1.003750pt}%
\definecolor{currentstroke}{rgb}{0.121569,0.466667,0.705882}%
\pgfsetstrokecolor{currentstroke}%
\pgfsetdash{}{0pt}%
\pgfpathmoveto{\pgfqpoint{2.524625in}{2.209750in}}%
\pgfpathcurveto{\pgfqpoint{2.535675in}{2.209750in}}{\pgfqpoint{2.546274in}{2.214140in}}{\pgfqpoint{2.554088in}{2.221954in}}%
\pgfpathcurveto{\pgfqpoint{2.561901in}{2.229767in}}{\pgfqpoint{2.566291in}{2.240366in}}{\pgfqpoint{2.566291in}{2.251416in}}%
\pgfpathcurveto{\pgfqpoint{2.566291in}{2.262466in}}{\pgfqpoint{2.561901in}{2.273065in}}{\pgfqpoint{2.554088in}{2.280879in}}%
\pgfpathcurveto{\pgfqpoint{2.546274in}{2.288693in}}{\pgfqpoint{2.535675in}{2.293083in}}{\pgfqpoint{2.524625in}{2.293083in}}%
\pgfpathcurveto{\pgfqpoint{2.513575in}{2.293083in}}{\pgfqpoint{2.502976in}{2.288693in}}{\pgfqpoint{2.495162in}{2.280879in}}%
\pgfpathcurveto{\pgfqpoint{2.487348in}{2.273065in}}{\pgfqpoint{2.482958in}{2.262466in}}{\pgfqpoint{2.482958in}{2.251416in}}%
\pgfpathcurveto{\pgfqpoint{2.482958in}{2.240366in}}{\pgfqpoint{2.487348in}{2.229767in}}{\pgfqpoint{2.495162in}{2.221954in}}%
\pgfpathcurveto{\pgfqpoint{2.502976in}{2.214140in}}{\pgfqpoint{2.513575in}{2.209750in}}{\pgfqpoint{2.524625in}{2.209750in}}%
\pgfpathclose%
\pgfusepath{stroke,fill}%
\end{pgfscope}%
\begin{pgfscope}%
\pgfpathrectangle{\pgfqpoint{0.600000in}{0.600000in}}{\pgfqpoint{3.900000in}{3.900000in}}%
\pgfusepath{clip}%
\pgfsetbuttcap%
\pgfsetroundjoin%
\definecolor{currentfill}{rgb}{0.121569,0.466667,0.705882}%
\pgfsetfillcolor{currentfill}%
\pgfsetlinewidth{1.003750pt}%
\definecolor{currentstroke}{rgb}{0.121569,0.466667,0.705882}%
\pgfsetstrokecolor{currentstroke}%
\pgfsetdash{}{0pt}%
\pgfpathmoveto{\pgfqpoint{1.528621in}{2.658251in}}%
\pgfpathcurveto{\pgfqpoint{1.539672in}{2.658251in}}{\pgfqpoint{1.550271in}{2.662641in}}{\pgfqpoint{1.558084in}{2.670455in}}%
\pgfpathcurveto{\pgfqpoint{1.565898in}{2.678268in}}{\pgfqpoint{1.570288in}{2.688867in}}{\pgfqpoint{1.570288in}{2.699917in}}%
\pgfpathcurveto{\pgfqpoint{1.570288in}{2.710967in}}{\pgfqpoint{1.565898in}{2.721566in}}{\pgfqpoint{1.558084in}{2.729380in}}%
\pgfpathcurveto{\pgfqpoint{1.550271in}{2.737194in}}{\pgfqpoint{1.539672in}{2.741584in}}{\pgfqpoint{1.528621in}{2.741584in}}%
\pgfpathcurveto{\pgfqpoint{1.517571in}{2.741584in}}{\pgfqpoint{1.506972in}{2.737194in}}{\pgfqpoint{1.499159in}{2.729380in}}%
\pgfpathcurveto{\pgfqpoint{1.491345in}{2.721566in}}{\pgfqpoint{1.486955in}{2.710967in}}{\pgfqpoint{1.486955in}{2.699917in}}%
\pgfpathcurveto{\pgfqpoint{1.486955in}{2.688867in}}{\pgfqpoint{1.491345in}{2.678268in}}{\pgfqpoint{1.499159in}{2.670455in}}%
\pgfpathcurveto{\pgfqpoint{1.506972in}{2.662641in}}{\pgfqpoint{1.517571in}{2.658251in}}{\pgfqpoint{1.528621in}{2.658251in}}%
\pgfpathclose%
\pgfusepath{stroke,fill}%
\end{pgfscope}%
\begin{pgfscope}%
\pgfpathrectangle{\pgfqpoint{0.600000in}{0.600000in}}{\pgfqpoint{3.900000in}{3.900000in}}%
\pgfusepath{clip}%
\pgfsetbuttcap%
\pgfsetroundjoin%
\definecolor{currentfill}{rgb}{0.121569,0.466667,0.705882}%
\pgfsetfillcolor{currentfill}%
\pgfsetlinewidth{1.003750pt}%
\definecolor{currentstroke}{rgb}{0.121569,0.466667,0.705882}%
\pgfsetstrokecolor{currentstroke}%
\pgfsetdash{}{0pt}%
\pgfpathmoveto{\pgfqpoint{2.192361in}{2.513743in}}%
\pgfpathcurveto{\pgfqpoint{2.203411in}{2.513743in}}{\pgfqpoint{2.214010in}{2.518134in}}{\pgfqpoint{2.221824in}{2.525947in}}%
\pgfpathcurveto{\pgfqpoint{2.229637in}{2.533761in}}{\pgfqpoint{2.234028in}{2.544360in}}{\pgfqpoint{2.234028in}{2.555410in}}%
\pgfpathcurveto{\pgfqpoint{2.234028in}{2.566460in}}{\pgfqpoint{2.229637in}{2.577059in}}{\pgfqpoint{2.221824in}{2.584873in}}%
\pgfpathcurveto{\pgfqpoint{2.214010in}{2.592686in}}{\pgfqpoint{2.203411in}{2.597077in}}{\pgfqpoint{2.192361in}{2.597077in}}%
\pgfpathcurveto{\pgfqpoint{2.181311in}{2.597077in}}{\pgfqpoint{2.170712in}{2.592686in}}{\pgfqpoint{2.162898in}{2.584873in}}%
\pgfpathcurveto{\pgfqpoint{2.155085in}{2.577059in}}{\pgfqpoint{2.150694in}{2.566460in}}{\pgfqpoint{2.150694in}{2.555410in}}%
\pgfpathcurveto{\pgfqpoint{2.150694in}{2.544360in}}{\pgfqpoint{2.155085in}{2.533761in}}{\pgfqpoint{2.162898in}{2.525947in}}%
\pgfpathcurveto{\pgfqpoint{2.170712in}{2.518134in}}{\pgfqpoint{2.181311in}{2.513743in}}{\pgfqpoint{2.192361in}{2.513743in}}%
\pgfpathclose%
\pgfusepath{stroke,fill}%
\end{pgfscope}%
\begin{pgfscope}%
\pgfpathrectangle{\pgfqpoint{0.600000in}{0.600000in}}{\pgfqpoint{3.900000in}{3.900000in}}%
\pgfusepath{clip}%
\pgfsetbuttcap%
\pgfsetroundjoin%
\definecolor{currentfill}{rgb}{0.121569,0.466667,0.705882}%
\pgfsetfillcolor{currentfill}%
\pgfsetlinewidth{1.003750pt}%
\definecolor{currentstroke}{rgb}{0.121569,0.466667,0.705882}%
\pgfsetstrokecolor{currentstroke}%
\pgfsetdash{}{0pt}%
\pgfpathmoveto{\pgfqpoint{2.038220in}{2.696006in}}%
\pgfpathcurveto{\pgfqpoint{2.049270in}{2.696006in}}{\pgfqpoint{2.059869in}{2.700396in}}{\pgfqpoint{2.067683in}{2.708210in}}%
\pgfpathcurveto{\pgfqpoint{2.075496in}{2.716023in}}{\pgfqpoint{2.079887in}{2.726622in}}{\pgfqpoint{2.079887in}{2.737672in}}%
\pgfpathcurveto{\pgfqpoint{2.079887in}{2.748722in}}{\pgfqpoint{2.075496in}{2.759321in}}{\pgfqpoint{2.067683in}{2.767135in}}%
\pgfpathcurveto{\pgfqpoint{2.059869in}{2.774949in}}{\pgfqpoint{2.049270in}{2.779339in}}{\pgfqpoint{2.038220in}{2.779339in}}%
\pgfpathcurveto{\pgfqpoint{2.027170in}{2.779339in}}{\pgfqpoint{2.016571in}{2.774949in}}{\pgfqpoint{2.008757in}{2.767135in}}%
\pgfpathcurveto{\pgfqpoint{2.000944in}{2.759321in}}{\pgfqpoint{1.996553in}{2.748722in}}{\pgfqpoint{1.996553in}{2.737672in}}%
\pgfpathcurveto{\pgfqpoint{1.996553in}{2.726622in}}{\pgfqpoint{2.000944in}{2.716023in}}{\pgfqpoint{2.008757in}{2.708210in}}%
\pgfpathcurveto{\pgfqpoint{2.016571in}{2.700396in}}{\pgfqpoint{2.027170in}{2.696006in}}{\pgfqpoint{2.038220in}{2.696006in}}%
\pgfpathclose%
\pgfusepath{stroke,fill}%
\end{pgfscope}%
\begin{pgfscope}%
\pgfpathrectangle{\pgfqpoint{0.600000in}{0.600000in}}{\pgfqpoint{3.900000in}{3.900000in}}%
\pgfusepath{clip}%
\pgfsetbuttcap%
\pgfsetroundjoin%
\definecolor{currentfill}{rgb}{0.121569,0.466667,0.705882}%
\pgfsetfillcolor{currentfill}%
\pgfsetlinewidth{1.003750pt}%
\definecolor{currentstroke}{rgb}{0.121569,0.466667,0.705882}%
\pgfsetstrokecolor{currentstroke}%
\pgfsetdash{}{0pt}%
\pgfpathmoveto{\pgfqpoint{1.897025in}{3.664092in}}%
\pgfpathcurveto{\pgfqpoint{1.908075in}{3.664092in}}{\pgfqpoint{1.918674in}{3.668483in}}{\pgfqpoint{1.926487in}{3.676296in}}%
\pgfpathcurveto{\pgfqpoint{1.934301in}{3.684110in}}{\pgfqpoint{1.938691in}{3.694709in}}{\pgfqpoint{1.938691in}{3.705759in}}%
\pgfpathcurveto{\pgfqpoint{1.938691in}{3.716809in}}{\pgfqpoint{1.934301in}{3.727408in}}{\pgfqpoint{1.926487in}{3.735222in}}%
\pgfpathcurveto{\pgfqpoint{1.918674in}{3.743035in}}{\pgfqpoint{1.908075in}{3.747426in}}{\pgfqpoint{1.897025in}{3.747426in}}%
\pgfpathcurveto{\pgfqpoint{1.885975in}{3.747426in}}{\pgfqpoint{1.875376in}{3.743035in}}{\pgfqpoint{1.867562in}{3.735222in}}%
\pgfpathcurveto{\pgfqpoint{1.859748in}{3.727408in}}{\pgfqpoint{1.855358in}{3.716809in}}{\pgfqpoint{1.855358in}{3.705759in}}%
\pgfpathcurveto{\pgfqpoint{1.855358in}{3.694709in}}{\pgfqpoint{1.859748in}{3.684110in}}{\pgfqpoint{1.867562in}{3.676296in}}%
\pgfpathcurveto{\pgfqpoint{1.875376in}{3.668483in}}{\pgfqpoint{1.885975in}{3.664092in}}{\pgfqpoint{1.897025in}{3.664092in}}%
\pgfpathclose%
\pgfusepath{stroke,fill}%
\end{pgfscope}%
\begin{pgfscope}%
\pgfpathrectangle{\pgfqpoint{0.600000in}{0.600000in}}{\pgfqpoint{3.900000in}{3.900000in}}%
\pgfusepath{clip}%
\pgfsetbuttcap%
\pgfsetroundjoin%
\definecolor{currentfill}{rgb}{0.121569,0.466667,0.705882}%
\pgfsetfillcolor{currentfill}%
\pgfsetlinewidth{1.003750pt}%
\definecolor{currentstroke}{rgb}{0.121569,0.466667,0.705882}%
\pgfsetstrokecolor{currentstroke}%
\pgfsetdash{}{0pt}%
\pgfpathmoveto{\pgfqpoint{1.968338in}{2.176219in}}%
\pgfpathcurveto{\pgfqpoint{1.979388in}{2.176219in}}{\pgfqpoint{1.989987in}{2.180609in}}{\pgfqpoint{1.997800in}{2.188423in}}%
\pgfpathcurveto{\pgfqpoint{2.005614in}{2.196237in}}{\pgfqpoint{2.010004in}{2.206836in}}{\pgfqpoint{2.010004in}{2.217886in}}%
\pgfpathcurveto{\pgfqpoint{2.010004in}{2.228936in}}{\pgfqpoint{2.005614in}{2.239535in}}{\pgfqpoint{1.997800in}{2.247348in}}%
\pgfpathcurveto{\pgfqpoint{1.989987in}{2.255162in}}{\pgfqpoint{1.979388in}{2.259552in}}{\pgfqpoint{1.968338in}{2.259552in}}%
\pgfpathcurveto{\pgfqpoint{1.957288in}{2.259552in}}{\pgfqpoint{1.946689in}{2.255162in}}{\pgfqpoint{1.938875in}{2.247348in}}%
\pgfpathcurveto{\pgfqpoint{1.931061in}{2.239535in}}{\pgfqpoint{1.926671in}{2.228936in}}{\pgfqpoint{1.926671in}{2.217886in}}%
\pgfpathcurveto{\pgfqpoint{1.926671in}{2.206836in}}{\pgfqpoint{1.931061in}{2.196237in}}{\pgfqpoint{1.938875in}{2.188423in}}%
\pgfpathcurveto{\pgfqpoint{1.946689in}{2.180609in}}{\pgfqpoint{1.957288in}{2.176219in}}{\pgfqpoint{1.968338in}{2.176219in}}%
\pgfpathclose%
\pgfusepath{stroke,fill}%
\end{pgfscope}%
\begin{pgfscope}%
\pgfpathrectangle{\pgfqpoint{0.600000in}{0.600000in}}{\pgfqpoint{3.900000in}{3.900000in}}%
\pgfusepath{clip}%
\pgfsetbuttcap%
\pgfsetroundjoin%
\definecolor{currentfill}{rgb}{0.121569,0.466667,0.705882}%
\pgfsetfillcolor{currentfill}%
\pgfsetlinewidth{1.003750pt}%
\definecolor{currentstroke}{rgb}{0.121569,0.466667,0.705882}%
\pgfsetstrokecolor{currentstroke}%
\pgfsetdash{}{0pt}%
\pgfpathmoveto{\pgfqpoint{2.035734in}{1.500780in}}%
\pgfpathcurveto{\pgfqpoint{2.046785in}{1.500780in}}{\pgfqpoint{2.057384in}{1.505170in}}{\pgfqpoint{2.065197in}{1.512984in}}%
\pgfpathcurveto{\pgfqpoint{2.073011in}{1.520798in}}{\pgfqpoint{2.077401in}{1.531397in}}{\pgfqpoint{2.077401in}{1.542447in}}%
\pgfpathcurveto{\pgfqpoint{2.077401in}{1.553497in}}{\pgfqpoint{2.073011in}{1.564096in}}{\pgfqpoint{2.065197in}{1.571910in}}%
\pgfpathcurveto{\pgfqpoint{2.057384in}{1.579723in}}{\pgfqpoint{2.046785in}{1.584113in}}{\pgfqpoint{2.035734in}{1.584113in}}%
\pgfpathcurveto{\pgfqpoint{2.024684in}{1.584113in}}{\pgfqpoint{2.014085in}{1.579723in}}{\pgfqpoint{2.006272in}{1.571910in}}%
\pgfpathcurveto{\pgfqpoint{1.998458in}{1.564096in}}{\pgfqpoint{1.994068in}{1.553497in}}{\pgfqpoint{1.994068in}{1.542447in}}%
\pgfpathcurveto{\pgfqpoint{1.994068in}{1.531397in}}{\pgfqpoint{1.998458in}{1.520798in}}{\pgfqpoint{2.006272in}{1.512984in}}%
\pgfpathcurveto{\pgfqpoint{2.014085in}{1.505170in}}{\pgfqpoint{2.024684in}{1.500780in}}{\pgfqpoint{2.035734in}{1.500780in}}%
\pgfpathclose%
\pgfusepath{stroke,fill}%
\end{pgfscope}%
\begin{pgfscope}%
\pgfpathrectangle{\pgfqpoint{0.600000in}{0.600000in}}{\pgfqpoint{3.900000in}{3.900000in}}%
\pgfusepath{clip}%
\pgfsetbuttcap%
\pgfsetroundjoin%
\definecolor{currentfill}{rgb}{0.121569,0.466667,0.705882}%
\pgfsetfillcolor{currentfill}%
\pgfsetlinewidth{1.003750pt}%
\definecolor{currentstroke}{rgb}{0.121569,0.466667,0.705882}%
\pgfsetstrokecolor{currentstroke}%
\pgfsetdash{}{0pt}%
\pgfpathmoveto{\pgfqpoint{2.597469in}{3.035519in}}%
\pgfpathcurveto{\pgfqpoint{2.608519in}{3.035519in}}{\pgfqpoint{2.619118in}{3.039909in}}{\pgfqpoint{2.626932in}{3.047723in}}%
\pgfpathcurveto{\pgfqpoint{2.634745in}{3.055536in}}{\pgfqpoint{2.639136in}{3.066135in}}{\pgfqpoint{2.639136in}{3.077185in}}%
\pgfpathcurveto{\pgfqpoint{2.639136in}{3.088236in}}{\pgfqpoint{2.634745in}{3.098835in}}{\pgfqpoint{2.626932in}{3.106648in}}%
\pgfpathcurveto{\pgfqpoint{2.619118in}{3.114462in}}{\pgfqpoint{2.608519in}{3.118852in}}{\pgfqpoint{2.597469in}{3.118852in}}%
\pgfpathcurveto{\pgfqpoint{2.586419in}{3.118852in}}{\pgfqpoint{2.575820in}{3.114462in}}{\pgfqpoint{2.568006in}{3.106648in}}%
\pgfpathcurveto{\pgfqpoint{2.560193in}{3.098835in}}{\pgfqpoint{2.555802in}{3.088236in}}{\pgfqpoint{2.555802in}{3.077185in}}%
\pgfpathcurveto{\pgfqpoint{2.555802in}{3.066135in}}{\pgfqpoint{2.560193in}{3.055536in}}{\pgfqpoint{2.568006in}{3.047723in}}%
\pgfpathcurveto{\pgfqpoint{2.575820in}{3.039909in}}{\pgfqpoint{2.586419in}{3.035519in}}{\pgfqpoint{2.597469in}{3.035519in}}%
\pgfpathclose%
\pgfusepath{stroke,fill}%
\end{pgfscope}%
\begin{pgfscope}%
\pgfpathrectangle{\pgfqpoint{0.600000in}{0.600000in}}{\pgfqpoint{3.900000in}{3.900000in}}%
\pgfusepath{clip}%
\pgfsetbuttcap%
\pgfsetroundjoin%
\definecolor{currentfill}{rgb}{0.121569,0.466667,0.705882}%
\pgfsetfillcolor{currentfill}%
\pgfsetlinewidth{1.003750pt}%
\definecolor{currentstroke}{rgb}{0.121569,0.466667,0.705882}%
\pgfsetstrokecolor{currentstroke}%
\pgfsetdash{}{0pt}%
\pgfpathmoveto{\pgfqpoint{2.930647in}{2.406392in}}%
\pgfpathcurveto{\pgfqpoint{2.941697in}{2.406392in}}{\pgfqpoint{2.952296in}{2.410782in}}{\pgfqpoint{2.960110in}{2.418595in}}%
\pgfpathcurveto{\pgfqpoint{2.967924in}{2.426409in}}{\pgfqpoint{2.972314in}{2.437008in}}{\pgfqpoint{2.972314in}{2.448058in}}%
\pgfpathcurveto{\pgfqpoint{2.972314in}{2.459108in}}{\pgfqpoint{2.967924in}{2.469707in}}{\pgfqpoint{2.960110in}{2.477521in}}%
\pgfpathcurveto{\pgfqpoint{2.952296in}{2.485335in}}{\pgfqpoint{2.941697in}{2.489725in}}{\pgfqpoint{2.930647in}{2.489725in}}%
\pgfpathcurveto{\pgfqpoint{2.919597in}{2.489725in}}{\pgfqpoint{2.908998in}{2.485335in}}{\pgfqpoint{2.901184in}{2.477521in}}%
\pgfpathcurveto{\pgfqpoint{2.893371in}{2.469707in}}{\pgfqpoint{2.888981in}{2.459108in}}{\pgfqpoint{2.888981in}{2.448058in}}%
\pgfpathcurveto{\pgfqpoint{2.888981in}{2.437008in}}{\pgfqpoint{2.893371in}{2.426409in}}{\pgfqpoint{2.901184in}{2.418595in}}%
\pgfpathcurveto{\pgfqpoint{2.908998in}{2.410782in}}{\pgfqpoint{2.919597in}{2.406392in}}{\pgfqpoint{2.930647in}{2.406392in}}%
\pgfpathclose%
\pgfusepath{stroke,fill}%
\end{pgfscope}%
\begin{pgfscope}%
\pgfpathrectangle{\pgfqpoint{0.600000in}{0.600000in}}{\pgfqpoint{3.900000in}{3.900000in}}%
\pgfusepath{clip}%
\pgfsetbuttcap%
\pgfsetroundjoin%
\definecolor{currentfill}{rgb}{0.121569,0.466667,0.705882}%
\pgfsetfillcolor{currentfill}%
\pgfsetlinewidth{1.003750pt}%
\definecolor{currentstroke}{rgb}{0.121569,0.466667,0.705882}%
\pgfsetstrokecolor{currentstroke}%
\pgfsetdash{}{0pt}%
\pgfpathmoveto{\pgfqpoint{2.797425in}{2.633289in}}%
\pgfpathcurveto{\pgfqpoint{2.808475in}{2.633289in}}{\pgfqpoint{2.819074in}{2.637679in}}{\pgfqpoint{2.826888in}{2.645493in}}%
\pgfpathcurveto{\pgfqpoint{2.834702in}{2.653306in}}{\pgfqpoint{2.839092in}{2.663905in}}{\pgfqpoint{2.839092in}{2.674955in}}%
\pgfpathcurveto{\pgfqpoint{2.839092in}{2.686006in}}{\pgfqpoint{2.834702in}{2.696605in}}{\pgfqpoint{2.826888in}{2.704418in}}%
\pgfpathcurveto{\pgfqpoint{2.819074in}{2.712232in}}{\pgfqpoint{2.808475in}{2.716622in}}{\pgfqpoint{2.797425in}{2.716622in}}%
\pgfpathcurveto{\pgfqpoint{2.786375in}{2.716622in}}{\pgfqpoint{2.775776in}{2.712232in}}{\pgfqpoint{2.767962in}{2.704418in}}%
\pgfpathcurveto{\pgfqpoint{2.760149in}{2.696605in}}{\pgfqpoint{2.755759in}{2.686006in}}{\pgfqpoint{2.755759in}{2.674955in}}%
\pgfpathcurveto{\pgfqpoint{2.755759in}{2.663905in}}{\pgfqpoint{2.760149in}{2.653306in}}{\pgfqpoint{2.767962in}{2.645493in}}%
\pgfpathcurveto{\pgfqpoint{2.775776in}{2.637679in}}{\pgfqpoint{2.786375in}{2.633289in}}{\pgfqpoint{2.797425in}{2.633289in}}%
\pgfpathclose%
\pgfusepath{stroke,fill}%
\end{pgfscope}%
\begin{pgfscope}%
\pgfpathrectangle{\pgfqpoint{0.600000in}{0.600000in}}{\pgfqpoint{3.900000in}{3.900000in}}%
\pgfusepath{clip}%
\pgfsetbuttcap%
\pgfsetroundjoin%
\definecolor{currentfill}{rgb}{0.121569,0.466667,0.705882}%
\pgfsetfillcolor{currentfill}%
\pgfsetlinewidth{1.003750pt}%
\definecolor{currentstroke}{rgb}{0.121569,0.466667,0.705882}%
\pgfsetstrokecolor{currentstroke}%
\pgfsetdash{}{0pt}%
\pgfpathmoveto{\pgfqpoint{2.127777in}{2.891002in}}%
\pgfpathcurveto{\pgfqpoint{2.138827in}{2.891002in}}{\pgfqpoint{2.149426in}{2.895392in}}{\pgfqpoint{2.157240in}{2.903206in}}%
\pgfpathcurveto{\pgfqpoint{2.165054in}{2.911019in}}{\pgfqpoint{2.169444in}{2.921618in}}{\pgfqpoint{2.169444in}{2.932668in}}%
\pgfpathcurveto{\pgfqpoint{2.169444in}{2.943718in}}{\pgfqpoint{2.165054in}{2.954318in}}{\pgfqpoint{2.157240in}{2.962131in}}%
\pgfpathcurveto{\pgfqpoint{2.149426in}{2.969945in}}{\pgfqpoint{2.138827in}{2.974335in}}{\pgfqpoint{2.127777in}{2.974335in}}%
\pgfpathcurveto{\pgfqpoint{2.116727in}{2.974335in}}{\pgfqpoint{2.106128in}{2.969945in}}{\pgfqpoint{2.098314in}{2.962131in}}%
\pgfpathcurveto{\pgfqpoint{2.090501in}{2.954318in}}{\pgfqpoint{2.086110in}{2.943718in}}{\pgfqpoint{2.086110in}{2.932668in}}%
\pgfpathcurveto{\pgfqpoint{2.086110in}{2.921618in}}{\pgfqpoint{2.090501in}{2.911019in}}{\pgfqpoint{2.098314in}{2.903206in}}%
\pgfpathcurveto{\pgfqpoint{2.106128in}{2.895392in}}{\pgfqpoint{2.116727in}{2.891002in}}{\pgfqpoint{2.127777in}{2.891002in}}%
\pgfpathclose%
\pgfusepath{stroke,fill}%
\end{pgfscope}%
\begin{pgfscope}%
\pgfpathrectangle{\pgfqpoint{0.600000in}{0.600000in}}{\pgfqpoint{3.900000in}{3.900000in}}%
\pgfusepath{clip}%
\pgfsetbuttcap%
\pgfsetroundjoin%
\definecolor{currentfill}{rgb}{0.121569,0.466667,0.705882}%
\pgfsetfillcolor{currentfill}%
\pgfsetlinewidth{1.003750pt}%
\definecolor{currentstroke}{rgb}{0.121569,0.466667,0.705882}%
\pgfsetstrokecolor{currentstroke}%
\pgfsetdash{}{0pt}%
\pgfpathmoveto{\pgfqpoint{2.170853in}{0.727464in}}%
\pgfpathcurveto{\pgfqpoint{2.181904in}{0.727464in}}{\pgfqpoint{2.192503in}{0.731855in}}{\pgfqpoint{2.200316in}{0.739668in}}%
\pgfpathcurveto{\pgfqpoint{2.208130in}{0.747482in}}{\pgfqpoint{2.212520in}{0.758081in}}{\pgfqpoint{2.212520in}{0.769131in}}%
\pgfpathcurveto{\pgfqpoint{2.212520in}{0.780181in}}{\pgfqpoint{2.208130in}{0.790780in}}{\pgfqpoint{2.200316in}{0.798594in}}%
\pgfpathcurveto{\pgfqpoint{2.192503in}{0.806407in}}{\pgfqpoint{2.181904in}{0.810798in}}{\pgfqpoint{2.170853in}{0.810798in}}%
\pgfpathcurveto{\pgfqpoint{2.159803in}{0.810798in}}{\pgfqpoint{2.149204in}{0.806407in}}{\pgfqpoint{2.141391in}{0.798594in}}%
\pgfpathcurveto{\pgfqpoint{2.133577in}{0.790780in}}{\pgfqpoint{2.129187in}{0.780181in}}{\pgfqpoint{2.129187in}{0.769131in}}%
\pgfpathcurveto{\pgfqpoint{2.129187in}{0.758081in}}{\pgfqpoint{2.133577in}{0.747482in}}{\pgfqpoint{2.141391in}{0.739668in}}%
\pgfpathcurveto{\pgfqpoint{2.149204in}{0.731855in}}{\pgfqpoint{2.159803in}{0.727464in}}{\pgfqpoint{2.170853in}{0.727464in}}%
\pgfpathclose%
\pgfusepath{stroke,fill}%
\end{pgfscope}%
\begin{pgfscope}%
\pgfpathrectangle{\pgfqpoint{0.600000in}{0.600000in}}{\pgfqpoint{3.900000in}{3.900000in}}%
\pgfusepath{clip}%
\pgfsetbuttcap%
\pgfsetroundjoin%
\definecolor{currentfill}{rgb}{0.121569,0.466667,0.705882}%
\pgfsetfillcolor{currentfill}%
\pgfsetlinewidth{1.003750pt}%
\definecolor{currentstroke}{rgb}{0.121569,0.466667,0.705882}%
\pgfsetstrokecolor{currentstroke}%
\pgfsetdash{}{0pt}%
\pgfpathmoveto{\pgfqpoint{3.780355in}{2.671975in}}%
\pgfpathcurveto{\pgfqpoint{3.791405in}{2.671975in}}{\pgfqpoint{3.802005in}{2.676365in}}{\pgfqpoint{3.809818in}{2.684179in}}%
\pgfpathcurveto{\pgfqpoint{3.817632in}{2.691992in}}{\pgfqpoint{3.822022in}{2.702591in}}{\pgfqpoint{3.822022in}{2.713641in}}%
\pgfpathcurveto{\pgfqpoint{3.822022in}{2.724692in}}{\pgfqpoint{3.817632in}{2.735291in}}{\pgfqpoint{3.809818in}{2.743104in}}%
\pgfpathcurveto{\pgfqpoint{3.802005in}{2.750918in}}{\pgfqpoint{3.791405in}{2.755308in}}{\pgfqpoint{3.780355in}{2.755308in}}%
\pgfpathcurveto{\pgfqpoint{3.769305in}{2.755308in}}{\pgfqpoint{3.758706in}{2.750918in}}{\pgfqpoint{3.750893in}{2.743104in}}%
\pgfpathcurveto{\pgfqpoint{3.743079in}{2.735291in}}{\pgfqpoint{3.738689in}{2.724692in}}{\pgfqpoint{3.738689in}{2.713641in}}%
\pgfpathcurveto{\pgfqpoint{3.738689in}{2.702591in}}{\pgfqpoint{3.743079in}{2.691992in}}{\pgfqpoint{3.750893in}{2.684179in}}%
\pgfpathcurveto{\pgfqpoint{3.758706in}{2.676365in}}{\pgfqpoint{3.769305in}{2.671975in}}{\pgfqpoint{3.780355in}{2.671975in}}%
\pgfpathclose%
\pgfusepath{stroke,fill}%
\end{pgfscope}%
\begin{pgfscope}%
\pgfpathrectangle{\pgfqpoint{0.600000in}{0.600000in}}{\pgfqpoint{3.900000in}{3.900000in}}%
\pgfusepath{clip}%
\pgfsetbuttcap%
\pgfsetroundjoin%
\definecolor{currentfill}{rgb}{0.121569,0.466667,0.705882}%
\pgfsetfillcolor{currentfill}%
\pgfsetlinewidth{1.003750pt}%
\definecolor{currentstroke}{rgb}{0.121569,0.466667,0.705882}%
\pgfsetstrokecolor{currentstroke}%
\pgfsetdash{}{0pt}%
\pgfpathmoveto{\pgfqpoint{2.287042in}{2.231952in}}%
\pgfpathcurveto{\pgfqpoint{2.298092in}{2.231952in}}{\pgfqpoint{2.308692in}{2.236342in}}{\pgfqpoint{2.316505in}{2.244156in}}%
\pgfpathcurveto{\pgfqpoint{2.324319in}{2.251969in}}{\pgfqpoint{2.328709in}{2.262568in}}{\pgfqpoint{2.328709in}{2.273618in}}%
\pgfpathcurveto{\pgfqpoint{2.328709in}{2.284669in}}{\pgfqpoint{2.324319in}{2.295268in}}{\pgfqpoint{2.316505in}{2.303081in}}%
\pgfpathcurveto{\pgfqpoint{2.308692in}{2.310895in}}{\pgfqpoint{2.298092in}{2.315285in}}{\pgfqpoint{2.287042in}{2.315285in}}%
\pgfpathcurveto{\pgfqpoint{2.275992in}{2.315285in}}{\pgfqpoint{2.265393in}{2.310895in}}{\pgfqpoint{2.257580in}{2.303081in}}%
\pgfpathcurveto{\pgfqpoint{2.249766in}{2.295268in}}{\pgfqpoint{2.245376in}{2.284669in}}{\pgfqpoint{2.245376in}{2.273618in}}%
\pgfpathcurveto{\pgfqpoint{2.245376in}{2.262568in}}{\pgfqpoint{2.249766in}{2.251969in}}{\pgfqpoint{2.257580in}{2.244156in}}%
\pgfpathcurveto{\pgfqpoint{2.265393in}{2.236342in}}{\pgfqpoint{2.275992in}{2.231952in}}{\pgfqpoint{2.287042in}{2.231952in}}%
\pgfpathclose%
\pgfusepath{stroke,fill}%
\end{pgfscope}%
\begin{pgfscope}%
\pgfpathrectangle{\pgfqpoint{0.600000in}{0.600000in}}{\pgfqpoint{3.900000in}{3.900000in}}%
\pgfusepath{clip}%
\pgfsetbuttcap%
\pgfsetroundjoin%
\definecolor{currentfill}{rgb}{0.121569,0.466667,0.705882}%
\pgfsetfillcolor{currentfill}%
\pgfsetlinewidth{1.003750pt}%
\definecolor{currentstroke}{rgb}{0.121569,0.466667,0.705882}%
\pgfsetstrokecolor{currentstroke}%
\pgfsetdash{}{0pt}%
\pgfpathmoveto{\pgfqpoint{2.364710in}{2.450769in}}%
\pgfpathcurveto{\pgfqpoint{2.375760in}{2.450769in}}{\pgfqpoint{2.386359in}{2.455159in}}{\pgfqpoint{2.394173in}{2.462972in}}%
\pgfpathcurveto{\pgfqpoint{2.401986in}{2.470786in}}{\pgfqpoint{2.406376in}{2.481385in}}{\pgfqpoint{2.406376in}{2.492435in}}%
\pgfpathcurveto{\pgfqpoint{2.406376in}{2.503485in}}{\pgfqpoint{2.401986in}{2.514084in}}{\pgfqpoint{2.394173in}{2.521898in}}%
\pgfpathcurveto{\pgfqpoint{2.386359in}{2.529712in}}{\pgfqpoint{2.375760in}{2.534102in}}{\pgfqpoint{2.364710in}{2.534102in}}%
\pgfpathcurveto{\pgfqpoint{2.353660in}{2.534102in}}{\pgfqpoint{2.343061in}{2.529712in}}{\pgfqpoint{2.335247in}{2.521898in}}%
\pgfpathcurveto{\pgfqpoint{2.327433in}{2.514084in}}{\pgfqpoint{2.323043in}{2.503485in}}{\pgfqpoint{2.323043in}{2.492435in}}%
\pgfpathcurveto{\pgfqpoint{2.323043in}{2.481385in}}{\pgfqpoint{2.327433in}{2.470786in}}{\pgfqpoint{2.335247in}{2.462972in}}%
\pgfpathcurveto{\pgfqpoint{2.343061in}{2.455159in}}{\pgfqpoint{2.353660in}{2.450769in}}{\pgfqpoint{2.364710in}{2.450769in}}%
\pgfpathclose%
\pgfusepath{stroke,fill}%
\end{pgfscope}%
\begin{pgfscope}%
\pgfpathrectangle{\pgfqpoint{0.600000in}{0.600000in}}{\pgfqpoint{3.900000in}{3.900000in}}%
\pgfusepath{clip}%
\pgfsetbuttcap%
\pgfsetroundjoin%
\definecolor{currentfill}{rgb}{0.121569,0.466667,0.705882}%
\pgfsetfillcolor{currentfill}%
\pgfsetlinewidth{1.003750pt}%
\definecolor{currentstroke}{rgb}{0.121569,0.466667,0.705882}%
\pgfsetstrokecolor{currentstroke}%
\pgfsetdash{}{0pt}%
\pgfpathmoveto{\pgfqpoint{2.016920in}{2.932881in}}%
\pgfpathcurveto{\pgfqpoint{2.027970in}{2.932881in}}{\pgfqpoint{2.038569in}{2.937271in}}{\pgfqpoint{2.046383in}{2.945085in}}%
\pgfpathcurveto{\pgfqpoint{2.054197in}{2.952899in}}{\pgfqpoint{2.058587in}{2.963498in}}{\pgfqpoint{2.058587in}{2.974548in}}%
\pgfpathcurveto{\pgfqpoint{2.058587in}{2.985598in}}{\pgfqpoint{2.054197in}{2.996197in}}{\pgfqpoint{2.046383in}{3.004011in}}%
\pgfpathcurveto{\pgfqpoint{2.038569in}{3.011824in}}{\pgfqpoint{2.027970in}{3.016214in}}{\pgfqpoint{2.016920in}{3.016214in}}%
\pgfpathcurveto{\pgfqpoint{2.005870in}{3.016214in}}{\pgfqpoint{1.995271in}{3.011824in}}{\pgfqpoint{1.987458in}{3.004011in}}%
\pgfpathcurveto{\pgfqpoint{1.979644in}{2.996197in}}{\pgfqpoint{1.975254in}{2.985598in}}{\pgfqpoint{1.975254in}{2.974548in}}%
\pgfpathcurveto{\pgfqpoint{1.975254in}{2.963498in}}{\pgfqpoint{1.979644in}{2.952899in}}{\pgfqpoint{1.987458in}{2.945085in}}%
\pgfpathcurveto{\pgfqpoint{1.995271in}{2.937271in}}{\pgfqpoint{2.005870in}{2.932881in}}{\pgfqpoint{2.016920in}{2.932881in}}%
\pgfpathclose%
\pgfusepath{stroke,fill}%
\end{pgfscope}%
\begin{pgfscope}%
\pgfpathrectangle{\pgfqpoint{0.600000in}{0.600000in}}{\pgfqpoint{3.900000in}{3.900000in}}%
\pgfusepath{clip}%
\pgfsetbuttcap%
\pgfsetroundjoin%
\definecolor{currentfill}{rgb}{0.121569,0.466667,0.705882}%
\pgfsetfillcolor{currentfill}%
\pgfsetlinewidth{1.003750pt}%
\definecolor{currentstroke}{rgb}{0.121569,0.466667,0.705882}%
\pgfsetstrokecolor{currentstroke}%
\pgfsetdash{}{0pt}%
\pgfpathmoveto{\pgfqpoint{2.220167in}{3.371614in}}%
\pgfpathcurveto{\pgfqpoint{2.231217in}{3.371614in}}{\pgfqpoint{2.241816in}{3.376004in}}{\pgfqpoint{2.249629in}{3.383818in}}%
\pgfpathcurveto{\pgfqpoint{2.257443in}{3.391631in}}{\pgfqpoint{2.261833in}{3.402230in}}{\pgfqpoint{2.261833in}{3.413280in}}%
\pgfpathcurveto{\pgfqpoint{2.261833in}{3.424331in}}{\pgfqpoint{2.257443in}{3.434930in}}{\pgfqpoint{2.249629in}{3.442743in}}%
\pgfpathcurveto{\pgfqpoint{2.241816in}{3.450557in}}{\pgfqpoint{2.231217in}{3.454947in}}{\pgfqpoint{2.220167in}{3.454947in}}%
\pgfpathcurveto{\pgfqpoint{2.209117in}{3.454947in}}{\pgfqpoint{2.198517in}{3.450557in}}{\pgfqpoint{2.190704in}{3.442743in}}%
\pgfpathcurveto{\pgfqpoint{2.182890in}{3.434930in}}{\pgfqpoint{2.178500in}{3.424331in}}{\pgfqpoint{2.178500in}{3.413280in}}%
\pgfpathcurveto{\pgfqpoint{2.178500in}{3.402230in}}{\pgfqpoint{2.182890in}{3.391631in}}{\pgfqpoint{2.190704in}{3.383818in}}%
\pgfpathcurveto{\pgfqpoint{2.198517in}{3.376004in}}{\pgfqpoint{2.209117in}{3.371614in}}{\pgfqpoint{2.220167in}{3.371614in}}%
\pgfpathclose%
\pgfusepath{stroke,fill}%
\end{pgfscope}%
\begin{pgfscope}%
\pgfpathrectangle{\pgfqpoint{0.600000in}{0.600000in}}{\pgfqpoint{3.900000in}{3.900000in}}%
\pgfusepath{clip}%
\pgfsetbuttcap%
\pgfsetroundjoin%
\definecolor{currentfill}{rgb}{0.121569,0.466667,0.705882}%
\pgfsetfillcolor{currentfill}%
\pgfsetlinewidth{1.003750pt}%
\definecolor{currentstroke}{rgb}{0.121569,0.466667,0.705882}%
\pgfsetstrokecolor{currentstroke}%
\pgfsetdash{}{0pt}%
\pgfpathmoveto{\pgfqpoint{1.854632in}{1.924997in}}%
\pgfpathcurveto{\pgfqpoint{1.865682in}{1.924997in}}{\pgfqpoint{1.876281in}{1.929387in}}{\pgfqpoint{1.884095in}{1.937201in}}%
\pgfpathcurveto{\pgfqpoint{1.891908in}{1.945015in}}{\pgfqpoint{1.896298in}{1.955614in}}{\pgfqpoint{1.896298in}{1.966664in}}%
\pgfpathcurveto{\pgfqpoint{1.896298in}{1.977714in}}{\pgfqpoint{1.891908in}{1.988313in}}{\pgfqpoint{1.884095in}{1.996127in}}%
\pgfpathcurveto{\pgfqpoint{1.876281in}{2.003940in}}{\pgfqpoint{1.865682in}{2.008330in}}{\pgfqpoint{1.854632in}{2.008330in}}%
\pgfpathcurveto{\pgfqpoint{1.843582in}{2.008330in}}{\pgfqpoint{1.832983in}{2.003940in}}{\pgfqpoint{1.825169in}{1.996127in}}%
\pgfpathcurveto{\pgfqpoint{1.817355in}{1.988313in}}{\pgfqpoint{1.812965in}{1.977714in}}{\pgfqpoint{1.812965in}{1.966664in}}%
\pgfpathcurveto{\pgfqpoint{1.812965in}{1.955614in}}{\pgfqpoint{1.817355in}{1.945015in}}{\pgfqpoint{1.825169in}{1.937201in}}%
\pgfpathcurveto{\pgfqpoint{1.832983in}{1.929387in}}{\pgfqpoint{1.843582in}{1.924997in}}{\pgfqpoint{1.854632in}{1.924997in}}%
\pgfpathclose%
\pgfusepath{stroke,fill}%
\end{pgfscope}%
\begin{pgfscope}%
\pgfpathrectangle{\pgfqpoint{0.600000in}{0.600000in}}{\pgfqpoint{3.900000in}{3.900000in}}%
\pgfusepath{clip}%
\pgfsetbuttcap%
\pgfsetroundjoin%
\definecolor{currentfill}{rgb}{0.121569,0.466667,0.705882}%
\pgfsetfillcolor{currentfill}%
\pgfsetlinewidth{1.003750pt}%
\definecolor{currentstroke}{rgb}{0.121569,0.466667,0.705882}%
\pgfsetstrokecolor{currentstroke}%
\pgfsetdash{}{0pt}%
\pgfpathmoveto{\pgfqpoint{2.934224in}{3.119277in}}%
\pgfpathcurveto{\pgfqpoint{2.945275in}{3.119277in}}{\pgfqpoint{2.955874in}{3.123668in}}{\pgfqpoint{2.963687in}{3.131481in}}%
\pgfpathcurveto{\pgfqpoint{2.971501in}{3.139295in}}{\pgfqpoint{2.975891in}{3.149894in}}{\pgfqpoint{2.975891in}{3.160944in}}%
\pgfpathcurveto{\pgfqpoint{2.975891in}{3.171994in}}{\pgfqpoint{2.971501in}{3.182593in}}{\pgfqpoint{2.963687in}{3.190407in}}%
\pgfpathcurveto{\pgfqpoint{2.955874in}{3.198220in}}{\pgfqpoint{2.945275in}{3.202611in}}{\pgfqpoint{2.934224in}{3.202611in}}%
\pgfpathcurveto{\pgfqpoint{2.923174in}{3.202611in}}{\pgfqpoint{2.912575in}{3.198220in}}{\pgfqpoint{2.904762in}{3.190407in}}%
\pgfpathcurveto{\pgfqpoint{2.896948in}{3.182593in}}{\pgfqpoint{2.892558in}{3.171994in}}{\pgfqpoint{2.892558in}{3.160944in}}%
\pgfpathcurveto{\pgfqpoint{2.892558in}{3.149894in}}{\pgfqpoint{2.896948in}{3.139295in}}{\pgfqpoint{2.904762in}{3.131481in}}%
\pgfpathcurveto{\pgfqpoint{2.912575in}{3.123668in}}{\pgfqpoint{2.923174in}{3.119277in}}{\pgfqpoint{2.934224in}{3.119277in}}%
\pgfpathclose%
\pgfusepath{stroke,fill}%
\end{pgfscope}%
\begin{pgfscope}%
\pgfpathrectangle{\pgfqpoint{0.600000in}{0.600000in}}{\pgfqpoint{3.900000in}{3.900000in}}%
\pgfusepath{clip}%
\pgfsetbuttcap%
\pgfsetroundjoin%
\definecolor{currentfill}{rgb}{0.121569,0.466667,0.705882}%
\pgfsetfillcolor{currentfill}%
\pgfsetlinewidth{1.003750pt}%
\definecolor{currentstroke}{rgb}{0.121569,0.466667,0.705882}%
\pgfsetstrokecolor{currentstroke}%
\pgfsetdash{}{0pt}%
\pgfpathmoveto{\pgfqpoint{2.758906in}{1.904122in}}%
\pgfpathcurveto{\pgfqpoint{2.769956in}{1.904122in}}{\pgfqpoint{2.780555in}{1.908513in}}{\pgfqpoint{2.788369in}{1.916326in}}%
\pgfpathcurveto{\pgfqpoint{2.796182in}{1.924140in}}{\pgfqpoint{2.800572in}{1.934739in}}{\pgfqpoint{2.800572in}{1.945789in}}%
\pgfpathcurveto{\pgfqpoint{2.800572in}{1.956839in}}{\pgfqpoint{2.796182in}{1.967438in}}{\pgfqpoint{2.788369in}{1.975252in}}%
\pgfpathcurveto{\pgfqpoint{2.780555in}{1.983066in}}{\pgfqpoint{2.769956in}{1.987456in}}{\pgfqpoint{2.758906in}{1.987456in}}%
\pgfpathcurveto{\pgfqpoint{2.747856in}{1.987456in}}{\pgfqpoint{2.737257in}{1.983066in}}{\pgfqpoint{2.729443in}{1.975252in}}%
\pgfpathcurveto{\pgfqpoint{2.721629in}{1.967438in}}{\pgfqpoint{2.717239in}{1.956839in}}{\pgfqpoint{2.717239in}{1.945789in}}%
\pgfpathcurveto{\pgfqpoint{2.717239in}{1.934739in}}{\pgfqpoint{2.721629in}{1.924140in}}{\pgfqpoint{2.729443in}{1.916326in}}%
\pgfpathcurveto{\pgfqpoint{2.737257in}{1.908513in}}{\pgfqpoint{2.747856in}{1.904122in}}{\pgfqpoint{2.758906in}{1.904122in}}%
\pgfpathclose%
\pgfusepath{stroke,fill}%
\end{pgfscope}%
\begin{pgfscope}%
\pgfpathrectangle{\pgfqpoint{0.600000in}{0.600000in}}{\pgfqpoint{3.900000in}{3.900000in}}%
\pgfusepath{clip}%
\pgfsetbuttcap%
\pgfsetroundjoin%
\definecolor{currentfill}{rgb}{0.121569,0.466667,0.705882}%
\pgfsetfillcolor{currentfill}%
\pgfsetlinewidth{1.003750pt}%
\definecolor{currentstroke}{rgb}{0.121569,0.466667,0.705882}%
\pgfsetstrokecolor{currentstroke}%
\pgfsetdash{}{0pt}%
\pgfpathmoveto{\pgfqpoint{2.360894in}{1.582360in}}%
\pgfpathcurveto{\pgfqpoint{2.371944in}{1.582360in}}{\pgfqpoint{2.382543in}{1.586750in}}{\pgfqpoint{2.390357in}{1.594564in}}%
\pgfpathcurveto{\pgfqpoint{2.398170in}{1.602377in}}{\pgfqpoint{2.402560in}{1.612976in}}{\pgfqpoint{2.402560in}{1.624026in}}%
\pgfpathcurveto{\pgfqpoint{2.402560in}{1.635076in}}{\pgfqpoint{2.398170in}{1.645675in}}{\pgfqpoint{2.390357in}{1.653489in}}%
\pgfpathcurveto{\pgfqpoint{2.382543in}{1.661303in}}{\pgfqpoint{2.371944in}{1.665693in}}{\pgfqpoint{2.360894in}{1.665693in}}%
\pgfpathcurveto{\pgfqpoint{2.349844in}{1.665693in}}{\pgfqpoint{2.339245in}{1.661303in}}{\pgfqpoint{2.331431in}{1.653489in}}%
\pgfpathcurveto{\pgfqpoint{2.323617in}{1.645675in}}{\pgfqpoint{2.319227in}{1.635076in}}{\pgfqpoint{2.319227in}{1.624026in}}%
\pgfpathcurveto{\pgfqpoint{2.319227in}{1.612976in}}{\pgfqpoint{2.323617in}{1.602377in}}{\pgfqpoint{2.331431in}{1.594564in}}%
\pgfpathcurveto{\pgfqpoint{2.339245in}{1.586750in}}{\pgfqpoint{2.349844in}{1.582360in}}{\pgfqpoint{2.360894in}{1.582360in}}%
\pgfpathclose%
\pgfusepath{stroke,fill}%
\end{pgfscope}%
\begin{pgfscope}%
\pgfpathrectangle{\pgfqpoint{0.600000in}{0.600000in}}{\pgfqpoint{3.900000in}{3.900000in}}%
\pgfusepath{clip}%
\pgfsetbuttcap%
\pgfsetroundjoin%
\definecolor{currentfill}{rgb}{0.121569,0.466667,0.705882}%
\pgfsetfillcolor{currentfill}%
\pgfsetlinewidth{1.003750pt}%
\definecolor{currentstroke}{rgb}{0.121569,0.466667,0.705882}%
\pgfsetstrokecolor{currentstroke}%
\pgfsetdash{}{0pt}%
\pgfpathmoveto{\pgfqpoint{2.942845in}{2.994003in}}%
\pgfpathcurveto{\pgfqpoint{2.953895in}{2.994003in}}{\pgfqpoint{2.964494in}{2.998393in}}{\pgfqpoint{2.972308in}{3.006207in}}%
\pgfpathcurveto{\pgfqpoint{2.980121in}{3.014020in}}{\pgfqpoint{2.984511in}{3.024619in}}{\pgfqpoint{2.984511in}{3.035670in}}%
\pgfpathcurveto{\pgfqpoint{2.984511in}{3.046720in}}{\pgfqpoint{2.980121in}{3.057319in}}{\pgfqpoint{2.972308in}{3.065132in}}%
\pgfpathcurveto{\pgfqpoint{2.964494in}{3.072946in}}{\pgfqpoint{2.953895in}{3.077336in}}{\pgfqpoint{2.942845in}{3.077336in}}%
\pgfpathcurveto{\pgfqpoint{2.931795in}{3.077336in}}{\pgfqpoint{2.921196in}{3.072946in}}{\pgfqpoint{2.913382in}{3.065132in}}%
\pgfpathcurveto{\pgfqpoint{2.905568in}{3.057319in}}{\pgfqpoint{2.901178in}{3.046720in}}{\pgfqpoint{2.901178in}{3.035670in}}%
\pgfpathcurveto{\pgfqpoint{2.901178in}{3.024619in}}{\pgfqpoint{2.905568in}{3.014020in}}{\pgfqpoint{2.913382in}{3.006207in}}%
\pgfpathcurveto{\pgfqpoint{2.921196in}{2.998393in}}{\pgfqpoint{2.931795in}{2.994003in}}{\pgfqpoint{2.942845in}{2.994003in}}%
\pgfpathclose%
\pgfusepath{stroke,fill}%
\end{pgfscope}%
\begin{pgfscope}%
\pgfpathrectangle{\pgfqpoint{0.600000in}{0.600000in}}{\pgfqpoint{3.900000in}{3.900000in}}%
\pgfusepath{clip}%
\pgfsetbuttcap%
\pgfsetroundjoin%
\definecolor{currentfill}{rgb}{0.121569,0.466667,0.705882}%
\pgfsetfillcolor{currentfill}%
\pgfsetlinewidth{1.003750pt}%
\definecolor{currentstroke}{rgb}{0.121569,0.466667,0.705882}%
\pgfsetstrokecolor{currentstroke}%
\pgfsetdash{}{0pt}%
\pgfpathmoveto{\pgfqpoint{2.572897in}{2.997989in}}%
\pgfpathcurveto{\pgfqpoint{2.583947in}{2.997989in}}{\pgfqpoint{2.594546in}{3.002379in}}{\pgfqpoint{2.602359in}{3.010193in}}%
\pgfpathcurveto{\pgfqpoint{2.610173in}{3.018006in}}{\pgfqpoint{2.614563in}{3.028605in}}{\pgfqpoint{2.614563in}{3.039656in}}%
\pgfpathcurveto{\pgfqpoint{2.614563in}{3.050706in}}{\pgfqpoint{2.610173in}{3.061305in}}{\pgfqpoint{2.602359in}{3.069118in}}%
\pgfpathcurveto{\pgfqpoint{2.594546in}{3.076932in}}{\pgfqpoint{2.583947in}{3.081322in}}{\pgfqpoint{2.572897in}{3.081322in}}%
\pgfpathcurveto{\pgfqpoint{2.561847in}{3.081322in}}{\pgfqpoint{2.551248in}{3.076932in}}{\pgfqpoint{2.543434in}{3.069118in}}%
\pgfpathcurveto{\pgfqpoint{2.535620in}{3.061305in}}{\pgfqpoint{2.531230in}{3.050706in}}{\pgfqpoint{2.531230in}{3.039656in}}%
\pgfpathcurveto{\pgfqpoint{2.531230in}{3.028605in}}{\pgfqpoint{2.535620in}{3.018006in}}{\pgfqpoint{2.543434in}{3.010193in}}%
\pgfpathcurveto{\pgfqpoint{2.551248in}{3.002379in}}{\pgfqpoint{2.561847in}{2.997989in}}{\pgfqpoint{2.572897in}{2.997989in}}%
\pgfpathclose%
\pgfusepath{stroke,fill}%
\end{pgfscope}%
\begin{pgfscope}%
\pgfpathrectangle{\pgfqpoint{0.600000in}{0.600000in}}{\pgfqpoint{3.900000in}{3.900000in}}%
\pgfusepath{clip}%
\pgfsetbuttcap%
\pgfsetroundjoin%
\definecolor{currentfill}{rgb}{0.121569,0.466667,0.705882}%
\pgfsetfillcolor{currentfill}%
\pgfsetlinewidth{1.003750pt}%
\definecolor{currentstroke}{rgb}{0.121569,0.466667,0.705882}%
\pgfsetstrokecolor{currentstroke}%
\pgfsetdash{}{0pt}%
\pgfpathmoveto{\pgfqpoint{3.274625in}{1.750920in}}%
\pgfpathcurveto{\pgfqpoint{3.285675in}{1.750920in}}{\pgfqpoint{3.296274in}{1.755310in}}{\pgfqpoint{3.304087in}{1.763124in}}%
\pgfpathcurveto{\pgfqpoint{3.311901in}{1.770938in}}{\pgfqpoint{3.316291in}{1.781537in}}{\pgfqpoint{3.316291in}{1.792587in}}%
\pgfpathcurveto{\pgfqpoint{3.316291in}{1.803637in}}{\pgfqpoint{3.311901in}{1.814236in}}{\pgfqpoint{3.304087in}{1.822050in}}%
\pgfpathcurveto{\pgfqpoint{3.296274in}{1.829863in}}{\pgfqpoint{3.285675in}{1.834254in}}{\pgfqpoint{3.274625in}{1.834254in}}%
\pgfpathcurveto{\pgfqpoint{3.263574in}{1.834254in}}{\pgfqpoint{3.252975in}{1.829863in}}{\pgfqpoint{3.245162in}{1.822050in}}%
\pgfpathcurveto{\pgfqpoint{3.237348in}{1.814236in}}{\pgfqpoint{3.232958in}{1.803637in}}{\pgfqpoint{3.232958in}{1.792587in}}%
\pgfpathcurveto{\pgfqpoint{3.232958in}{1.781537in}}{\pgfqpoint{3.237348in}{1.770938in}}{\pgfqpoint{3.245162in}{1.763124in}}%
\pgfpathcurveto{\pgfqpoint{3.252975in}{1.755310in}}{\pgfqpoint{3.263574in}{1.750920in}}{\pgfqpoint{3.274625in}{1.750920in}}%
\pgfpathclose%
\pgfusepath{stroke,fill}%
\end{pgfscope}%
\begin{pgfscope}%
\pgfpathrectangle{\pgfqpoint{0.600000in}{0.600000in}}{\pgfqpoint{3.900000in}{3.900000in}}%
\pgfusepath{clip}%
\pgfsetbuttcap%
\pgfsetroundjoin%
\definecolor{currentfill}{rgb}{0.121569,0.466667,0.705882}%
\pgfsetfillcolor{currentfill}%
\pgfsetlinewidth{1.003750pt}%
\definecolor{currentstroke}{rgb}{0.121569,0.466667,0.705882}%
\pgfsetstrokecolor{currentstroke}%
\pgfsetdash{}{0pt}%
\pgfpathmoveto{\pgfqpoint{2.580266in}{1.415581in}}%
\pgfpathcurveto{\pgfqpoint{2.591316in}{1.415581in}}{\pgfqpoint{2.601915in}{1.419971in}}{\pgfqpoint{2.609729in}{1.427785in}}%
\pgfpathcurveto{\pgfqpoint{2.617542in}{1.435599in}}{\pgfqpoint{2.621933in}{1.446198in}}{\pgfqpoint{2.621933in}{1.457248in}}%
\pgfpathcurveto{\pgfqpoint{2.621933in}{1.468298in}}{\pgfqpoint{2.617542in}{1.478897in}}{\pgfqpoint{2.609729in}{1.486711in}}%
\pgfpathcurveto{\pgfqpoint{2.601915in}{1.494524in}}{\pgfqpoint{2.591316in}{1.498915in}}{\pgfqpoint{2.580266in}{1.498915in}}%
\pgfpathcurveto{\pgfqpoint{2.569216in}{1.498915in}}{\pgfqpoint{2.558617in}{1.494524in}}{\pgfqpoint{2.550803in}{1.486711in}}%
\pgfpathcurveto{\pgfqpoint{2.542990in}{1.478897in}}{\pgfqpoint{2.538599in}{1.468298in}}{\pgfqpoint{2.538599in}{1.457248in}}%
\pgfpathcurveto{\pgfqpoint{2.538599in}{1.446198in}}{\pgfqpoint{2.542990in}{1.435599in}}{\pgfqpoint{2.550803in}{1.427785in}}%
\pgfpathcurveto{\pgfqpoint{2.558617in}{1.419971in}}{\pgfqpoint{2.569216in}{1.415581in}}{\pgfqpoint{2.580266in}{1.415581in}}%
\pgfpathclose%
\pgfusepath{stroke,fill}%
\end{pgfscope}%
\begin{pgfscope}%
\pgfpathrectangle{\pgfqpoint{0.600000in}{0.600000in}}{\pgfqpoint{3.900000in}{3.900000in}}%
\pgfusepath{clip}%
\pgfsetbuttcap%
\pgfsetroundjoin%
\definecolor{currentfill}{rgb}{0.121569,0.466667,0.705882}%
\pgfsetfillcolor{currentfill}%
\pgfsetlinewidth{1.003750pt}%
\definecolor{currentstroke}{rgb}{0.121569,0.466667,0.705882}%
\pgfsetstrokecolor{currentstroke}%
\pgfsetdash{}{0pt}%
\pgfpathmoveto{\pgfqpoint{1.881551in}{2.259410in}}%
\pgfpathcurveto{\pgfqpoint{1.892602in}{2.259410in}}{\pgfqpoint{1.903201in}{2.263801in}}{\pgfqpoint{1.911014in}{2.271614in}}%
\pgfpathcurveto{\pgfqpoint{1.918828in}{2.279428in}}{\pgfqpoint{1.923218in}{2.290027in}}{\pgfqpoint{1.923218in}{2.301077in}}%
\pgfpathcurveto{\pgfqpoint{1.923218in}{2.312127in}}{\pgfqpoint{1.918828in}{2.322726in}}{\pgfqpoint{1.911014in}{2.330540in}}%
\pgfpathcurveto{\pgfqpoint{1.903201in}{2.338353in}}{\pgfqpoint{1.892602in}{2.342744in}}{\pgfqpoint{1.881551in}{2.342744in}}%
\pgfpathcurveto{\pgfqpoint{1.870501in}{2.342744in}}{\pgfqpoint{1.859902in}{2.338353in}}{\pgfqpoint{1.852089in}{2.330540in}}%
\pgfpathcurveto{\pgfqpoint{1.844275in}{2.322726in}}{\pgfqpoint{1.839885in}{2.312127in}}{\pgfqpoint{1.839885in}{2.301077in}}%
\pgfpathcurveto{\pgfqpoint{1.839885in}{2.290027in}}{\pgfqpoint{1.844275in}{2.279428in}}{\pgfqpoint{1.852089in}{2.271614in}}%
\pgfpathcurveto{\pgfqpoint{1.859902in}{2.263801in}}{\pgfqpoint{1.870501in}{2.259410in}}{\pgfqpoint{1.881551in}{2.259410in}}%
\pgfpathclose%
\pgfusepath{stroke,fill}%
\end{pgfscope}%
\begin{pgfscope}%
\pgfpathrectangle{\pgfqpoint{0.600000in}{0.600000in}}{\pgfqpoint{3.900000in}{3.900000in}}%
\pgfusepath{clip}%
\pgfsetbuttcap%
\pgfsetroundjoin%
\definecolor{currentfill}{rgb}{0.121569,0.466667,0.705882}%
\pgfsetfillcolor{currentfill}%
\pgfsetlinewidth{1.003750pt}%
\definecolor{currentstroke}{rgb}{0.121569,0.466667,0.705882}%
\pgfsetstrokecolor{currentstroke}%
\pgfsetdash{}{0pt}%
\pgfpathmoveto{\pgfqpoint{1.718743in}{3.315144in}}%
\pgfpathcurveto{\pgfqpoint{1.729793in}{3.315144in}}{\pgfqpoint{1.740392in}{3.319535in}}{\pgfqpoint{1.748206in}{3.327348in}}%
\pgfpathcurveto{\pgfqpoint{1.756019in}{3.335162in}}{\pgfqpoint{1.760410in}{3.345761in}}{\pgfqpoint{1.760410in}{3.356811in}}%
\pgfpathcurveto{\pgfqpoint{1.760410in}{3.367861in}}{\pgfqpoint{1.756019in}{3.378460in}}{\pgfqpoint{1.748206in}{3.386274in}}%
\pgfpathcurveto{\pgfqpoint{1.740392in}{3.394087in}}{\pgfqpoint{1.729793in}{3.398478in}}{\pgfqpoint{1.718743in}{3.398478in}}%
\pgfpathcurveto{\pgfqpoint{1.707693in}{3.398478in}}{\pgfqpoint{1.697094in}{3.394087in}}{\pgfqpoint{1.689280in}{3.386274in}}%
\pgfpathcurveto{\pgfqpoint{1.681467in}{3.378460in}}{\pgfqpoint{1.677076in}{3.367861in}}{\pgfqpoint{1.677076in}{3.356811in}}%
\pgfpathcurveto{\pgfqpoint{1.677076in}{3.345761in}}{\pgfqpoint{1.681467in}{3.335162in}}{\pgfqpoint{1.689280in}{3.327348in}}%
\pgfpathcurveto{\pgfqpoint{1.697094in}{3.319535in}}{\pgfqpoint{1.707693in}{3.315144in}}{\pgfqpoint{1.718743in}{3.315144in}}%
\pgfpathclose%
\pgfusepath{stroke,fill}%
\end{pgfscope}%
\begin{pgfscope}%
\pgfpathrectangle{\pgfqpoint{0.600000in}{0.600000in}}{\pgfqpoint{3.900000in}{3.900000in}}%
\pgfusepath{clip}%
\pgfsetbuttcap%
\pgfsetroundjoin%
\definecolor{currentfill}{rgb}{0.121569,0.466667,0.705882}%
\pgfsetfillcolor{currentfill}%
\pgfsetlinewidth{1.003750pt}%
\definecolor{currentstroke}{rgb}{0.121569,0.466667,0.705882}%
\pgfsetstrokecolor{currentstroke}%
\pgfsetdash{}{0pt}%
\pgfpathmoveto{\pgfqpoint{2.672546in}{2.026296in}}%
\pgfpathcurveto{\pgfqpoint{2.683596in}{2.026296in}}{\pgfqpoint{2.694195in}{2.030686in}}{\pgfqpoint{2.702009in}{2.038499in}}%
\pgfpathcurveto{\pgfqpoint{2.709822in}{2.046313in}}{\pgfqpoint{2.714212in}{2.056912in}}{\pgfqpoint{2.714212in}{2.067962in}}%
\pgfpathcurveto{\pgfqpoint{2.714212in}{2.079012in}}{\pgfqpoint{2.709822in}{2.089611in}}{\pgfqpoint{2.702009in}{2.097425in}}%
\pgfpathcurveto{\pgfqpoint{2.694195in}{2.105239in}}{\pgfqpoint{2.683596in}{2.109629in}}{\pgfqpoint{2.672546in}{2.109629in}}%
\pgfpathcurveto{\pgfqpoint{2.661496in}{2.109629in}}{\pgfqpoint{2.650897in}{2.105239in}}{\pgfqpoint{2.643083in}{2.097425in}}%
\pgfpathcurveto{\pgfqpoint{2.635269in}{2.089611in}}{\pgfqpoint{2.630879in}{2.079012in}}{\pgfqpoint{2.630879in}{2.067962in}}%
\pgfpathcurveto{\pgfqpoint{2.630879in}{2.056912in}}{\pgfqpoint{2.635269in}{2.046313in}}{\pgfqpoint{2.643083in}{2.038499in}}%
\pgfpathcurveto{\pgfqpoint{2.650897in}{2.030686in}}{\pgfqpoint{2.661496in}{2.026296in}}{\pgfqpoint{2.672546in}{2.026296in}}%
\pgfpathclose%
\pgfusepath{stroke,fill}%
\end{pgfscope}%
\begin{pgfscope}%
\pgfpathrectangle{\pgfqpoint{0.600000in}{0.600000in}}{\pgfqpoint{3.900000in}{3.900000in}}%
\pgfusepath{clip}%
\pgfsetbuttcap%
\pgfsetroundjoin%
\definecolor{currentfill}{rgb}{0.121569,0.466667,0.705882}%
\pgfsetfillcolor{currentfill}%
\pgfsetlinewidth{1.003750pt}%
\definecolor{currentstroke}{rgb}{0.121569,0.466667,0.705882}%
\pgfsetstrokecolor{currentstroke}%
\pgfsetdash{}{0pt}%
\pgfpathmoveto{\pgfqpoint{2.853433in}{1.905751in}}%
\pgfpathcurveto{\pgfqpoint{2.864483in}{1.905751in}}{\pgfqpoint{2.875082in}{1.910141in}}{\pgfqpoint{2.882896in}{1.917955in}}%
\pgfpathcurveto{\pgfqpoint{2.890709in}{1.925769in}}{\pgfqpoint{2.895100in}{1.936368in}}{\pgfqpoint{2.895100in}{1.947418in}}%
\pgfpathcurveto{\pgfqpoint{2.895100in}{1.958468in}}{\pgfqpoint{2.890709in}{1.969067in}}{\pgfqpoint{2.882896in}{1.976880in}}%
\pgfpathcurveto{\pgfqpoint{2.875082in}{1.984694in}}{\pgfqpoint{2.864483in}{1.989084in}}{\pgfqpoint{2.853433in}{1.989084in}}%
\pgfpathcurveto{\pgfqpoint{2.842383in}{1.989084in}}{\pgfqpoint{2.831784in}{1.984694in}}{\pgfqpoint{2.823970in}{1.976880in}}%
\pgfpathcurveto{\pgfqpoint{2.816157in}{1.969067in}}{\pgfqpoint{2.811766in}{1.958468in}}{\pgfqpoint{2.811766in}{1.947418in}}%
\pgfpathcurveto{\pgfqpoint{2.811766in}{1.936368in}}{\pgfqpoint{2.816157in}{1.925769in}}{\pgfqpoint{2.823970in}{1.917955in}}%
\pgfpathcurveto{\pgfqpoint{2.831784in}{1.910141in}}{\pgfqpoint{2.842383in}{1.905751in}}{\pgfqpoint{2.853433in}{1.905751in}}%
\pgfpathclose%
\pgfusepath{stroke,fill}%
\end{pgfscope}%
\begin{pgfscope}%
\pgfpathrectangle{\pgfqpoint{0.600000in}{0.600000in}}{\pgfqpoint{3.900000in}{3.900000in}}%
\pgfusepath{clip}%
\pgfsetbuttcap%
\pgfsetroundjoin%
\definecolor{currentfill}{rgb}{0.121569,0.466667,0.705882}%
\pgfsetfillcolor{currentfill}%
\pgfsetlinewidth{1.003750pt}%
\definecolor{currentstroke}{rgb}{0.121569,0.466667,0.705882}%
\pgfsetstrokecolor{currentstroke}%
\pgfsetdash{}{0pt}%
\pgfpathmoveto{\pgfqpoint{2.473606in}{1.954086in}}%
\pgfpathcurveto{\pgfqpoint{2.484657in}{1.954086in}}{\pgfqpoint{2.495256in}{1.958476in}}{\pgfqpoint{2.503069in}{1.966290in}}%
\pgfpathcurveto{\pgfqpoint{2.510883in}{1.974104in}}{\pgfqpoint{2.515273in}{1.984703in}}{\pgfqpoint{2.515273in}{1.995753in}}%
\pgfpathcurveto{\pgfqpoint{2.515273in}{2.006803in}}{\pgfqpoint{2.510883in}{2.017402in}}{\pgfqpoint{2.503069in}{2.025215in}}%
\pgfpathcurveto{\pgfqpoint{2.495256in}{2.033029in}}{\pgfqpoint{2.484657in}{2.037419in}}{\pgfqpoint{2.473606in}{2.037419in}}%
\pgfpathcurveto{\pgfqpoint{2.462556in}{2.037419in}}{\pgfqpoint{2.451957in}{2.033029in}}{\pgfqpoint{2.444144in}{2.025215in}}%
\pgfpathcurveto{\pgfqpoint{2.436330in}{2.017402in}}{\pgfqpoint{2.431940in}{2.006803in}}{\pgfqpoint{2.431940in}{1.995753in}}%
\pgfpathcurveto{\pgfqpoint{2.431940in}{1.984703in}}{\pgfqpoint{2.436330in}{1.974104in}}{\pgfqpoint{2.444144in}{1.966290in}}%
\pgfpathcurveto{\pgfqpoint{2.451957in}{1.958476in}}{\pgfqpoint{2.462556in}{1.954086in}}{\pgfqpoint{2.473606in}{1.954086in}}%
\pgfpathclose%
\pgfusepath{stroke,fill}%
\end{pgfscope}%
\begin{pgfscope}%
\pgfpathrectangle{\pgfqpoint{0.600000in}{0.600000in}}{\pgfqpoint{3.900000in}{3.900000in}}%
\pgfusepath{clip}%
\pgfsetbuttcap%
\pgfsetroundjoin%
\definecolor{currentfill}{rgb}{0.121569,0.466667,0.705882}%
\pgfsetfillcolor{currentfill}%
\pgfsetlinewidth{1.003750pt}%
\definecolor{currentstroke}{rgb}{0.121569,0.466667,0.705882}%
\pgfsetstrokecolor{currentstroke}%
\pgfsetdash{}{0pt}%
\pgfpathmoveto{\pgfqpoint{2.503042in}{3.120514in}}%
\pgfpathcurveto{\pgfqpoint{2.514092in}{3.120514in}}{\pgfqpoint{2.524691in}{3.124905in}}{\pgfqpoint{2.532505in}{3.132718in}}%
\pgfpathcurveto{\pgfqpoint{2.540318in}{3.140532in}}{\pgfqpoint{2.544709in}{3.151131in}}{\pgfqpoint{2.544709in}{3.162181in}}%
\pgfpathcurveto{\pgfqpoint{2.544709in}{3.173231in}}{\pgfqpoint{2.540318in}{3.183830in}}{\pgfqpoint{2.532505in}{3.191644in}}%
\pgfpathcurveto{\pgfqpoint{2.524691in}{3.199457in}}{\pgfqpoint{2.514092in}{3.203848in}}{\pgfqpoint{2.503042in}{3.203848in}}%
\pgfpathcurveto{\pgfqpoint{2.491992in}{3.203848in}}{\pgfqpoint{2.481393in}{3.199457in}}{\pgfqpoint{2.473579in}{3.191644in}}%
\pgfpathcurveto{\pgfqpoint{2.465766in}{3.183830in}}{\pgfqpoint{2.461375in}{3.173231in}}{\pgfqpoint{2.461375in}{3.162181in}}%
\pgfpathcurveto{\pgfqpoint{2.461375in}{3.151131in}}{\pgfqpoint{2.465766in}{3.140532in}}{\pgfqpoint{2.473579in}{3.132718in}}%
\pgfpathcurveto{\pgfqpoint{2.481393in}{3.124905in}}{\pgfqpoint{2.491992in}{3.120514in}}{\pgfqpoint{2.503042in}{3.120514in}}%
\pgfpathclose%
\pgfusepath{stroke,fill}%
\end{pgfscope}%
\begin{pgfscope}%
\pgfpathrectangle{\pgfqpoint{0.600000in}{0.600000in}}{\pgfqpoint{3.900000in}{3.900000in}}%
\pgfusepath{clip}%
\pgfsetbuttcap%
\pgfsetroundjoin%
\definecolor{currentfill}{rgb}{0.121569,0.466667,0.705882}%
\pgfsetfillcolor{currentfill}%
\pgfsetlinewidth{1.003750pt}%
\definecolor{currentstroke}{rgb}{0.121569,0.466667,0.705882}%
\pgfsetstrokecolor{currentstroke}%
\pgfsetdash{}{0pt}%
\pgfpathmoveto{\pgfqpoint{1.945282in}{2.548029in}}%
\pgfpathcurveto{\pgfqpoint{1.956332in}{2.548029in}}{\pgfqpoint{1.966931in}{2.552420in}}{\pgfqpoint{1.974745in}{2.560233in}}%
\pgfpathcurveto{\pgfqpoint{1.982559in}{2.568047in}}{\pgfqpoint{1.986949in}{2.578646in}}{\pgfqpoint{1.986949in}{2.589696in}}%
\pgfpathcurveto{\pgfqpoint{1.986949in}{2.600746in}}{\pgfqpoint{1.982559in}{2.611345in}}{\pgfqpoint{1.974745in}{2.619159in}}%
\pgfpathcurveto{\pgfqpoint{1.966931in}{2.626973in}}{\pgfqpoint{1.956332in}{2.631363in}}{\pgfqpoint{1.945282in}{2.631363in}}%
\pgfpathcurveto{\pgfqpoint{1.934232in}{2.631363in}}{\pgfqpoint{1.923633in}{2.626973in}}{\pgfqpoint{1.915819in}{2.619159in}}%
\pgfpathcurveto{\pgfqpoint{1.908006in}{2.611345in}}{\pgfqpoint{1.903616in}{2.600746in}}{\pgfqpoint{1.903616in}{2.589696in}}%
\pgfpathcurveto{\pgfqpoint{1.903616in}{2.578646in}}{\pgfqpoint{1.908006in}{2.568047in}}{\pgfqpoint{1.915819in}{2.560233in}}%
\pgfpathcurveto{\pgfqpoint{1.923633in}{2.552420in}}{\pgfqpoint{1.934232in}{2.548029in}}{\pgfqpoint{1.945282in}{2.548029in}}%
\pgfpathclose%
\pgfusepath{stroke,fill}%
\end{pgfscope}%
\begin{pgfscope}%
\pgfpathrectangle{\pgfqpoint{0.600000in}{0.600000in}}{\pgfqpoint{3.900000in}{3.900000in}}%
\pgfusepath{clip}%
\pgfsetbuttcap%
\pgfsetroundjoin%
\definecolor{currentfill}{rgb}{0.121569,0.466667,0.705882}%
\pgfsetfillcolor{currentfill}%
\pgfsetlinewidth{1.003750pt}%
\definecolor{currentstroke}{rgb}{0.121569,0.466667,0.705882}%
\pgfsetstrokecolor{currentstroke}%
\pgfsetdash{}{0pt}%
\pgfpathmoveto{\pgfqpoint{2.908171in}{2.012554in}}%
\pgfpathcurveto{\pgfqpoint{2.919221in}{2.012554in}}{\pgfqpoint{2.929820in}{2.016945in}}{\pgfqpoint{2.937633in}{2.024758in}}%
\pgfpathcurveto{\pgfqpoint{2.945447in}{2.032572in}}{\pgfqpoint{2.949837in}{2.043171in}}{\pgfqpoint{2.949837in}{2.054221in}}%
\pgfpathcurveto{\pgfqpoint{2.949837in}{2.065271in}}{\pgfqpoint{2.945447in}{2.075870in}}{\pgfqpoint{2.937633in}{2.083684in}}%
\pgfpathcurveto{\pgfqpoint{2.929820in}{2.091497in}}{\pgfqpoint{2.919221in}{2.095888in}}{\pgfqpoint{2.908171in}{2.095888in}}%
\pgfpathcurveto{\pgfqpoint{2.897120in}{2.095888in}}{\pgfqpoint{2.886521in}{2.091497in}}{\pgfqpoint{2.878708in}{2.083684in}}%
\pgfpathcurveto{\pgfqpoint{2.870894in}{2.075870in}}{\pgfqpoint{2.866504in}{2.065271in}}{\pgfqpoint{2.866504in}{2.054221in}}%
\pgfpathcurveto{\pgfqpoint{2.866504in}{2.043171in}}{\pgfqpoint{2.870894in}{2.032572in}}{\pgfqpoint{2.878708in}{2.024758in}}%
\pgfpathcurveto{\pgfqpoint{2.886521in}{2.016945in}}{\pgfqpoint{2.897120in}{2.012554in}}{\pgfqpoint{2.908171in}{2.012554in}}%
\pgfpathclose%
\pgfusepath{stroke,fill}%
\end{pgfscope}%
\begin{pgfscope}%
\pgfpathrectangle{\pgfqpoint{0.600000in}{0.600000in}}{\pgfqpoint{3.900000in}{3.900000in}}%
\pgfusepath{clip}%
\pgfsetbuttcap%
\pgfsetroundjoin%
\definecolor{currentfill}{rgb}{0.121569,0.466667,0.705882}%
\pgfsetfillcolor{currentfill}%
\pgfsetlinewidth{1.003750pt}%
\definecolor{currentstroke}{rgb}{0.121569,0.466667,0.705882}%
\pgfsetstrokecolor{currentstroke}%
\pgfsetdash{}{0pt}%
\pgfpathmoveto{\pgfqpoint{2.413476in}{2.669287in}}%
\pgfpathcurveto{\pgfqpoint{2.424526in}{2.669287in}}{\pgfqpoint{2.435125in}{2.673678in}}{\pgfqpoint{2.442939in}{2.681491in}}%
\pgfpathcurveto{\pgfqpoint{2.450752in}{2.689305in}}{\pgfqpoint{2.455142in}{2.699904in}}{\pgfqpoint{2.455142in}{2.710954in}}%
\pgfpathcurveto{\pgfqpoint{2.455142in}{2.722004in}}{\pgfqpoint{2.450752in}{2.732603in}}{\pgfqpoint{2.442939in}{2.740417in}}%
\pgfpathcurveto{\pgfqpoint{2.435125in}{2.748230in}}{\pgfqpoint{2.424526in}{2.752621in}}{\pgfqpoint{2.413476in}{2.752621in}}%
\pgfpathcurveto{\pgfqpoint{2.402426in}{2.752621in}}{\pgfqpoint{2.391827in}{2.748230in}}{\pgfqpoint{2.384013in}{2.740417in}}%
\pgfpathcurveto{\pgfqpoint{2.376199in}{2.732603in}}{\pgfqpoint{2.371809in}{2.722004in}}{\pgfqpoint{2.371809in}{2.710954in}}%
\pgfpathcurveto{\pgfqpoint{2.371809in}{2.699904in}}{\pgfqpoint{2.376199in}{2.689305in}}{\pgfqpoint{2.384013in}{2.681491in}}%
\pgfpathcurveto{\pgfqpoint{2.391827in}{2.673678in}}{\pgfqpoint{2.402426in}{2.669287in}}{\pgfqpoint{2.413476in}{2.669287in}}%
\pgfpathclose%
\pgfusepath{stroke,fill}%
\end{pgfscope}%
\begin{pgfscope}%
\pgfpathrectangle{\pgfqpoint{0.600000in}{0.600000in}}{\pgfqpoint{3.900000in}{3.900000in}}%
\pgfusepath{clip}%
\pgfsetbuttcap%
\pgfsetroundjoin%
\definecolor{currentfill}{rgb}{0.121569,0.466667,0.705882}%
\pgfsetfillcolor{currentfill}%
\pgfsetlinewidth{1.003750pt}%
\definecolor{currentstroke}{rgb}{0.121569,0.466667,0.705882}%
\pgfsetstrokecolor{currentstroke}%
\pgfsetdash{}{0pt}%
\pgfpathmoveto{\pgfqpoint{2.102771in}{2.163529in}}%
\pgfpathcurveto{\pgfqpoint{2.113821in}{2.163529in}}{\pgfqpoint{2.124420in}{2.167919in}}{\pgfqpoint{2.132234in}{2.175733in}}%
\pgfpathcurveto{\pgfqpoint{2.140047in}{2.183546in}}{\pgfqpoint{2.144438in}{2.194145in}}{\pgfqpoint{2.144438in}{2.205195in}}%
\pgfpathcurveto{\pgfqpoint{2.144438in}{2.216245in}}{\pgfqpoint{2.140047in}{2.226845in}}{\pgfqpoint{2.132234in}{2.234658in}}%
\pgfpathcurveto{\pgfqpoint{2.124420in}{2.242472in}}{\pgfqpoint{2.113821in}{2.246862in}}{\pgfqpoint{2.102771in}{2.246862in}}%
\pgfpathcurveto{\pgfqpoint{2.091721in}{2.246862in}}{\pgfqpoint{2.081122in}{2.242472in}}{\pgfqpoint{2.073308in}{2.234658in}}%
\pgfpathcurveto{\pgfqpoint{2.065495in}{2.226845in}}{\pgfqpoint{2.061104in}{2.216245in}}{\pgfqpoint{2.061104in}{2.205195in}}%
\pgfpathcurveto{\pgfqpoint{2.061104in}{2.194145in}}{\pgfqpoint{2.065495in}{2.183546in}}{\pgfqpoint{2.073308in}{2.175733in}}%
\pgfpathcurveto{\pgfqpoint{2.081122in}{2.167919in}}{\pgfqpoint{2.091721in}{2.163529in}}{\pgfqpoint{2.102771in}{2.163529in}}%
\pgfpathclose%
\pgfusepath{stroke,fill}%
\end{pgfscope}%
\begin{pgfscope}%
\pgfpathrectangle{\pgfqpoint{0.600000in}{0.600000in}}{\pgfqpoint{3.900000in}{3.900000in}}%
\pgfusepath{clip}%
\pgfsetbuttcap%
\pgfsetroundjoin%
\definecolor{currentfill}{rgb}{0.121569,0.466667,0.705882}%
\pgfsetfillcolor{currentfill}%
\pgfsetlinewidth{1.003750pt}%
\definecolor{currentstroke}{rgb}{0.121569,0.466667,0.705882}%
\pgfsetstrokecolor{currentstroke}%
\pgfsetdash{}{0pt}%
\pgfpathmoveto{\pgfqpoint{2.074376in}{2.693816in}}%
\pgfpathcurveto{\pgfqpoint{2.085426in}{2.693816in}}{\pgfqpoint{2.096025in}{2.698207in}}{\pgfqpoint{2.103839in}{2.706020in}}%
\pgfpathcurveto{\pgfqpoint{2.111652in}{2.713834in}}{\pgfqpoint{2.116043in}{2.724433in}}{\pgfqpoint{2.116043in}{2.735483in}}%
\pgfpathcurveto{\pgfqpoint{2.116043in}{2.746533in}}{\pgfqpoint{2.111652in}{2.757132in}}{\pgfqpoint{2.103839in}{2.764946in}}%
\pgfpathcurveto{\pgfqpoint{2.096025in}{2.772759in}}{\pgfqpoint{2.085426in}{2.777150in}}{\pgfqpoint{2.074376in}{2.777150in}}%
\pgfpathcurveto{\pgfqpoint{2.063326in}{2.777150in}}{\pgfqpoint{2.052727in}{2.772759in}}{\pgfqpoint{2.044913in}{2.764946in}}%
\pgfpathcurveto{\pgfqpoint{2.037100in}{2.757132in}}{\pgfqpoint{2.032709in}{2.746533in}}{\pgfqpoint{2.032709in}{2.735483in}}%
\pgfpathcurveto{\pgfqpoint{2.032709in}{2.724433in}}{\pgfqpoint{2.037100in}{2.713834in}}{\pgfqpoint{2.044913in}{2.706020in}}%
\pgfpathcurveto{\pgfqpoint{2.052727in}{2.698207in}}{\pgfqpoint{2.063326in}{2.693816in}}{\pgfqpoint{2.074376in}{2.693816in}}%
\pgfpathclose%
\pgfusepath{stroke,fill}%
\end{pgfscope}%
\begin{pgfscope}%
\pgfpathrectangle{\pgfqpoint{0.600000in}{0.600000in}}{\pgfqpoint{3.900000in}{3.900000in}}%
\pgfusepath{clip}%
\pgfsetbuttcap%
\pgfsetroundjoin%
\definecolor{currentfill}{rgb}{0.121569,0.466667,0.705882}%
\pgfsetfillcolor{currentfill}%
\pgfsetlinewidth{1.003750pt}%
\definecolor{currentstroke}{rgb}{0.121569,0.466667,0.705882}%
\pgfsetstrokecolor{currentstroke}%
\pgfsetdash{}{0pt}%
\pgfpathmoveto{\pgfqpoint{2.640250in}{2.926268in}}%
\pgfpathcurveto{\pgfqpoint{2.651301in}{2.926268in}}{\pgfqpoint{2.661900in}{2.930658in}}{\pgfqpoint{2.669713in}{2.938472in}}%
\pgfpathcurveto{\pgfqpoint{2.677527in}{2.946286in}}{\pgfqpoint{2.681917in}{2.956885in}}{\pgfqpoint{2.681917in}{2.967935in}}%
\pgfpathcurveto{\pgfqpoint{2.681917in}{2.978985in}}{\pgfqpoint{2.677527in}{2.989584in}}{\pgfqpoint{2.669713in}{2.997397in}}%
\pgfpathcurveto{\pgfqpoint{2.661900in}{3.005211in}}{\pgfqpoint{2.651301in}{3.009601in}}{\pgfqpoint{2.640250in}{3.009601in}}%
\pgfpathcurveto{\pgfqpoint{2.629200in}{3.009601in}}{\pgfqpoint{2.618601in}{3.005211in}}{\pgfqpoint{2.610788in}{2.997397in}}%
\pgfpathcurveto{\pgfqpoint{2.602974in}{2.989584in}}{\pgfqpoint{2.598584in}{2.978985in}}{\pgfqpoint{2.598584in}{2.967935in}}%
\pgfpathcurveto{\pgfqpoint{2.598584in}{2.956885in}}{\pgfqpoint{2.602974in}{2.946286in}}{\pgfqpoint{2.610788in}{2.938472in}}%
\pgfpathcurveto{\pgfqpoint{2.618601in}{2.930658in}}{\pgfqpoint{2.629200in}{2.926268in}}{\pgfqpoint{2.640250in}{2.926268in}}%
\pgfpathclose%
\pgfusepath{stroke,fill}%
\end{pgfscope}%
\begin{pgfscope}%
\pgfpathrectangle{\pgfqpoint{0.600000in}{0.600000in}}{\pgfqpoint{3.900000in}{3.900000in}}%
\pgfusepath{clip}%
\pgfsetbuttcap%
\pgfsetroundjoin%
\definecolor{currentfill}{rgb}{0.121569,0.466667,0.705882}%
\pgfsetfillcolor{currentfill}%
\pgfsetlinewidth{1.003750pt}%
\definecolor{currentstroke}{rgb}{0.121569,0.466667,0.705882}%
\pgfsetstrokecolor{currentstroke}%
\pgfsetdash{}{0pt}%
\pgfpathmoveto{\pgfqpoint{2.419908in}{2.860030in}}%
\pgfpathcurveto{\pgfqpoint{2.430958in}{2.860030in}}{\pgfqpoint{2.441557in}{2.864420in}}{\pgfqpoint{2.449371in}{2.872234in}}%
\pgfpathcurveto{\pgfqpoint{2.457184in}{2.880048in}}{\pgfqpoint{2.461575in}{2.890647in}}{\pgfqpoint{2.461575in}{2.901697in}}%
\pgfpathcurveto{\pgfqpoint{2.461575in}{2.912747in}}{\pgfqpoint{2.457184in}{2.923346in}}{\pgfqpoint{2.449371in}{2.931160in}}%
\pgfpathcurveto{\pgfqpoint{2.441557in}{2.938973in}}{\pgfqpoint{2.430958in}{2.943363in}}{\pgfqpoint{2.419908in}{2.943363in}}%
\pgfpathcurveto{\pgfqpoint{2.408858in}{2.943363in}}{\pgfqpoint{2.398259in}{2.938973in}}{\pgfqpoint{2.390445in}{2.931160in}}%
\pgfpathcurveto{\pgfqpoint{2.382631in}{2.923346in}}{\pgfqpoint{2.378241in}{2.912747in}}{\pgfqpoint{2.378241in}{2.901697in}}%
\pgfpathcurveto{\pgfqpoint{2.378241in}{2.890647in}}{\pgfqpoint{2.382631in}{2.880048in}}{\pgfqpoint{2.390445in}{2.872234in}}%
\pgfpathcurveto{\pgfqpoint{2.398259in}{2.864420in}}{\pgfqpoint{2.408858in}{2.860030in}}{\pgfqpoint{2.419908in}{2.860030in}}%
\pgfpathclose%
\pgfusepath{stroke,fill}%
\end{pgfscope}%
\begin{pgfscope}%
\pgfpathrectangle{\pgfqpoint{0.600000in}{0.600000in}}{\pgfqpoint{3.900000in}{3.900000in}}%
\pgfusepath{clip}%
\pgfsetbuttcap%
\pgfsetroundjoin%
\definecolor{currentfill}{rgb}{0.121569,0.466667,0.705882}%
\pgfsetfillcolor{currentfill}%
\pgfsetlinewidth{1.003750pt}%
\definecolor{currentstroke}{rgb}{0.121569,0.466667,0.705882}%
\pgfsetstrokecolor{currentstroke}%
\pgfsetdash{}{0pt}%
\pgfpathmoveto{\pgfqpoint{2.850735in}{2.698633in}}%
\pgfpathcurveto{\pgfqpoint{2.861785in}{2.698633in}}{\pgfqpoint{2.872384in}{2.703023in}}{\pgfqpoint{2.880198in}{2.710837in}}%
\pgfpathcurveto{\pgfqpoint{2.888011in}{2.718651in}}{\pgfqpoint{2.892401in}{2.729250in}}{\pgfqpoint{2.892401in}{2.740300in}}%
\pgfpathcurveto{\pgfqpoint{2.892401in}{2.751350in}}{\pgfqpoint{2.888011in}{2.761949in}}{\pgfqpoint{2.880198in}{2.769763in}}%
\pgfpathcurveto{\pgfqpoint{2.872384in}{2.777576in}}{\pgfqpoint{2.861785in}{2.781966in}}{\pgfqpoint{2.850735in}{2.781966in}}%
\pgfpathcurveto{\pgfqpoint{2.839685in}{2.781966in}}{\pgfqpoint{2.829086in}{2.777576in}}{\pgfqpoint{2.821272in}{2.769763in}}%
\pgfpathcurveto{\pgfqpoint{2.813458in}{2.761949in}}{\pgfqpoint{2.809068in}{2.751350in}}{\pgfqpoint{2.809068in}{2.740300in}}%
\pgfpathcurveto{\pgfqpoint{2.809068in}{2.729250in}}{\pgfqpoint{2.813458in}{2.718651in}}{\pgfqpoint{2.821272in}{2.710837in}}%
\pgfpathcurveto{\pgfqpoint{2.829086in}{2.703023in}}{\pgfqpoint{2.839685in}{2.698633in}}{\pgfqpoint{2.850735in}{2.698633in}}%
\pgfpathclose%
\pgfusepath{stroke,fill}%
\end{pgfscope}%
\begin{pgfscope}%
\pgfpathrectangle{\pgfqpoint{0.600000in}{0.600000in}}{\pgfqpoint{3.900000in}{3.900000in}}%
\pgfusepath{clip}%
\pgfsetbuttcap%
\pgfsetroundjoin%
\definecolor{currentfill}{rgb}{0.121569,0.466667,0.705882}%
\pgfsetfillcolor{currentfill}%
\pgfsetlinewidth{1.003750pt}%
\definecolor{currentstroke}{rgb}{0.121569,0.466667,0.705882}%
\pgfsetstrokecolor{currentstroke}%
\pgfsetdash{}{0pt}%
\pgfpathmoveto{\pgfqpoint{2.766940in}{2.412163in}}%
\pgfpathcurveto{\pgfqpoint{2.777990in}{2.412163in}}{\pgfqpoint{2.788589in}{2.416553in}}{\pgfqpoint{2.796403in}{2.424366in}}%
\pgfpathcurveto{\pgfqpoint{2.804216in}{2.432180in}}{\pgfqpoint{2.808607in}{2.442779in}}{\pgfqpoint{2.808607in}{2.453829in}}%
\pgfpathcurveto{\pgfqpoint{2.808607in}{2.464879in}}{\pgfqpoint{2.804216in}{2.475478in}}{\pgfqpoint{2.796403in}{2.483292in}}%
\pgfpathcurveto{\pgfqpoint{2.788589in}{2.491106in}}{\pgfqpoint{2.777990in}{2.495496in}}{\pgfqpoint{2.766940in}{2.495496in}}%
\pgfpathcurveto{\pgfqpoint{2.755890in}{2.495496in}}{\pgfqpoint{2.745291in}{2.491106in}}{\pgfqpoint{2.737477in}{2.483292in}}%
\pgfpathcurveto{\pgfqpoint{2.729664in}{2.475478in}}{\pgfqpoint{2.725273in}{2.464879in}}{\pgfqpoint{2.725273in}{2.453829in}}%
\pgfpathcurveto{\pgfqpoint{2.725273in}{2.442779in}}{\pgfqpoint{2.729664in}{2.432180in}}{\pgfqpoint{2.737477in}{2.424366in}}%
\pgfpathcurveto{\pgfqpoint{2.745291in}{2.416553in}}{\pgfqpoint{2.755890in}{2.412163in}}{\pgfqpoint{2.766940in}{2.412163in}}%
\pgfpathclose%
\pgfusepath{stroke,fill}%
\end{pgfscope}%
\begin{pgfscope}%
\pgfpathrectangle{\pgfqpoint{0.600000in}{0.600000in}}{\pgfqpoint{3.900000in}{3.900000in}}%
\pgfusepath{clip}%
\pgfsetbuttcap%
\pgfsetroundjoin%
\definecolor{currentfill}{rgb}{0.121569,0.466667,0.705882}%
\pgfsetfillcolor{currentfill}%
\pgfsetlinewidth{1.003750pt}%
\definecolor{currentstroke}{rgb}{0.121569,0.466667,0.705882}%
\pgfsetstrokecolor{currentstroke}%
\pgfsetdash{}{0pt}%
\pgfpathmoveto{\pgfqpoint{2.452841in}{2.418751in}}%
\pgfpathcurveto{\pgfqpoint{2.463891in}{2.418751in}}{\pgfqpoint{2.474490in}{2.423141in}}{\pgfqpoint{2.482303in}{2.430955in}}%
\pgfpathcurveto{\pgfqpoint{2.490117in}{2.438768in}}{\pgfqpoint{2.494507in}{2.449367in}}{\pgfqpoint{2.494507in}{2.460417in}}%
\pgfpathcurveto{\pgfqpoint{2.494507in}{2.471468in}}{\pgfqpoint{2.490117in}{2.482067in}}{\pgfqpoint{2.482303in}{2.489880in}}%
\pgfpathcurveto{\pgfqpoint{2.474490in}{2.497694in}}{\pgfqpoint{2.463891in}{2.502084in}}{\pgfqpoint{2.452841in}{2.502084in}}%
\pgfpathcurveto{\pgfqpoint{2.441791in}{2.502084in}}{\pgfqpoint{2.431191in}{2.497694in}}{\pgfqpoint{2.423378in}{2.489880in}}%
\pgfpathcurveto{\pgfqpoint{2.415564in}{2.482067in}}{\pgfqpoint{2.411174in}{2.471468in}}{\pgfqpoint{2.411174in}{2.460417in}}%
\pgfpathcurveto{\pgfqpoint{2.411174in}{2.449367in}}{\pgfqpoint{2.415564in}{2.438768in}}{\pgfqpoint{2.423378in}{2.430955in}}%
\pgfpathcurveto{\pgfqpoint{2.431191in}{2.423141in}}{\pgfqpoint{2.441791in}{2.418751in}}{\pgfqpoint{2.452841in}{2.418751in}}%
\pgfpathclose%
\pgfusepath{stroke,fill}%
\end{pgfscope}%
\begin{pgfscope}%
\pgfpathrectangle{\pgfqpoint{0.600000in}{0.600000in}}{\pgfqpoint{3.900000in}{3.900000in}}%
\pgfusepath{clip}%
\pgfsetbuttcap%
\pgfsetroundjoin%
\definecolor{currentfill}{rgb}{0.121569,0.466667,0.705882}%
\pgfsetfillcolor{currentfill}%
\pgfsetlinewidth{1.003750pt}%
\definecolor{currentstroke}{rgb}{0.121569,0.466667,0.705882}%
\pgfsetstrokecolor{currentstroke}%
\pgfsetdash{}{0pt}%
\pgfpathmoveto{\pgfqpoint{3.454890in}{3.286415in}}%
\pgfpathcurveto{\pgfqpoint{3.465940in}{3.286415in}}{\pgfqpoint{3.476539in}{3.290805in}}{\pgfqpoint{3.484353in}{3.298619in}}%
\pgfpathcurveto{\pgfqpoint{3.492166in}{3.306433in}}{\pgfqpoint{3.496557in}{3.317032in}}{\pgfqpoint{3.496557in}{3.328082in}}%
\pgfpathcurveto{\pgfqpoint{3.496557in}{3.339132in}}{\pgfqpoint{3.492166in}{3.349731in}}{\pgfqpoint{3.484353in}{3.357545in}}%
\pgfpathcurveto{\pgfqpoint{3.476539in}{3.365358in}}{\pgfqpoint{3.465940in}{3.369748in}}{\pgfqpoint{3.454890in}{3.369748in}}%
\pgfpathcurveto{\pgfqpoint{3.443840in}{3.369748in}}{\pgfqpoint{3.433241in}{3.365358in}}{\pgfqpoint{3.425427in}{3.357545in}}%
\pgfpathcurveto{\pgfqpoint{3.417614in}{3.349731in}}{\pgfqpoint{3.413223in}{3.339132in}}{\pgfqpoint{3.413223in}{3.328082in}}%
\pgfpathcurveto{\pgfqpoint{3.413223in}{3.317032in}}{\pgfqpoint{3.417614in}{3.306433in}}{\pgfqpoint{3.425427in}{3.298619in}}%
\pgfpathcurveto{\pgfqpoint{3.433241in}{3.290805in}}{\pgfqpoint{3.443840in}{3.286415in}}{\pgfqpoint{3.454890in}{3.286415in}}%
\pgfpathclose%
\pgfusepath{stroke,fill}%
\end{pgfscope}%
\begin{pgfscope}%
\pgfpathrectangle{\pgfqpoint{0.600000in}{0.600000in}}{\pgfqpoint{3.900000in}{3.900000in}}%
\pgfusepath{clip}%
\pgfsetbuttcap%
\pgfsetroundjoin%
\definecolor{currentfill}{rgb}{0.121569,0.466667,0.705882}%
\pgfsetfillcolor{currentfill}%
\pgfsetlinewidth{1.003750pt}%
\definecolor{currentstroke}{rgb}{0.121569,0.466667,0.705882}%
\pgfsetstrokecolor{currentstroke}%
\pgfsetdash{}{0pt}%
\pgfpathmoveto{\pgfqpoint{2.990129in}{2.729863in}}%
\pgfpathcurveto{\pgfqpoint{3.001179in}{2.729863in}}{\pgfqpoint{3.011778in}{2.734254in}}{\pgfqpoint{3.019592in}{2.742067in}}%
\pgfpathcurveto{\pgfqpoint{3.027405in}{2.749881in}}{\pgfqpoint{3.031796in}{2.760480in}}{\pgfqpoint{3.031796in}{2.771530in}}%
\pgfpathcurveto{\pgfqpoint{3.031796in}{2.782580in}}{\pgfqpoint{3.027405in}{2.793179in}}{\pgfqpoint{3.019592in}{2.800993in}}%
\pgfpathcurveto{\pgfqpoint{3.011778in}{2.808806in}}{\pgfqpoint{3.001179in}{2.813197in}}{\pgfqpoint{2.990129in}{2.813197in}}%
\pgfpathcurveto{\pgfqpoint{2.979079in}{2.813197in}}{\pgfqpoint{2.968480in}{2.808806in}}{\pgfqpoint{2.960666in}{2.800993in}}%
\pgfpathcurveto{\pgfqpoint{2.952853in}{2.793179in}}{\pgfqpoint{2.948462in}{2.782580in}}{\pgfqpoint{2.948462in}{2.771530in}}%
\pgfpathcurveto{\pgfqpoint{2.948462in}{2.760480in}}{\pgfqpoint{2.952853in}{2.749881in}}{\pgfqpoint{2.960666in}{2.742067in}}%
\pgfpathcurveto{\pgfqpoint{2.968480in}{2.734254in}}{\pgfqpoint{2.979079in}{2.729863in}}{\pgfqpoint{2.990129in}{2.729863in}}%
\pgfpathclose%
\pgfusepath{stroke,fill}%
\end{pgfscope}%
\begin{pgfscope}%
\pgfpathrectangle{\pgfqpoint{0.600000in}{0.600000in}}{\pgfqpoint{3.900000in}{3.900000in}}%
\pgfusepath{clip}%
\pgfsetbuttcap%
\pgfsetroundjoin%
\definecolor{currentfill}{rgb}{0.121569,0.466667,0.705882}%
\pgfsetfillcolor{currentfill}%
\pgfsetlinewidth{1.003750pt}%
\definecolor{currentstroke}{rgb}{0.121569,0.466667,0.705882}%
\pgfsetstrokecolor{currentstroke}%
\pgfsetdash{}{0pt}%
\pgfpathmoveto{\pgfqpoint{2.604974in}{3.060412in}}%
\pgfpathcurveto{\pgfqpoint{2.616024in}{3.060412in}}{\pgfqpoint{2.626623in}{3.064802in}}{\pgfqpoint{2.634437in}{3.072615in}}%
\pgfpathcurveto{\pgfqpoint{2.642251in}{3.080429in}}{\pgfqpoint{2.646641in}{3.091028in}}{\pgfqpoint{2.646641in}{3.102078in}}%
\pgfpathcurveto{\pgfqpoint{2.646641in}{3.113128in}}{\pgfqpoint{2.642251in}{3.123727in}}{\pgfqpoint{2.634437in}{3.131541in}}%
\pgfpathcurveto{\pgfqpoint{2.626623in}{3.139355in}}{\pgfqpoint{2.616024in}{3.143745in}}{\pgfqpoint{2.604974in}{3.143745in}}%
\pgfpathcurveto{\pgfqpoint{2.593924in}{3.143745in}}{\pgfqpoint{2.583325in}{3.139355in}}{\pgfqpoint{2.575511in}{3.131541in}}%
\pgfpathcurveto{\pgfqpoint{2.567698in}{3.123727in}}{\pgfqpoint{2.563308in}{3.113128in}}{\pgfqpoint{2.563308in}{3.102078in}}%
\pgfpathcurveto{\pgfqpoint{2.563308in}{3.091028in}}{\pgfqpoint{2.567698in}{3.080429in}}{\pgfqpoint{2.575511in}{3.072615in}}%
\pgfpathcurveto{\pgfqpoint{2.583325in}{3.064802in}}{\pgfqpoint{2.593924in}{3.060412in}}{\pgfqpoint{2.604974in}{3.060412in}}%
\pgfpathclose%
\pgfusepath{stroke,fill}%
\end{pgfscope}%
\begin{pgfscope}%
\pgfpathrectangle{\pgfqpoint{0.600000in}{0.600000in}}{\pgfqpoint{3.900000in}{3.900000in}}%
\pgfusepath{clip}%
\pgfsetbuttcap%
\pgfsetroundjoin%
\definecolor{currentfill}{rgb}{0.121569,0.466667,0.705882}%
\pgfsetfillcolor{currentfill}%
\pgfsetlinewidth{1.003750pt}%
\definecolor{currentstroke}{rgb}{0.121569,0.466667,0.705882}%
\pgfsetstrokecolor{currentstroke}%
\pgfsetdash{}{0pt}%
\pgfpathmoveto{\pgfqpoint{3.016165in}{2.305047in}}%
\pgfpathcurveto{\pgfqpoint{3.027215in}{2.305047in}}{\pgfqpoint{3.037815in}{2.309437in}}{\pgfqpoint{3.045628in}{2.317251in}}%
\pgfpathcurveto{\pgfqpoint{3.053442in}{2.325064in}}{\pgfqpoint{3.057832in}{2.335663in}}{\pgfqpoint{3.057832in}{2.346713in}}%
\pgfpathcurveto{\pgfqpoint{3.057832in}{2.357764in}}{\pgfqpoint{3.053442in}{2.368363in}}{\pgfqpoint{3.045628in}{2.376176in}}%
\pgfpathcurveto{\pgfqpoint{3.037815in}{2.383990in}}{\pgfqpoint{3.027215in}{2.388380in}}{\pgfqpoint{3.016165in}{2.388380in}}%
\pgfpathcurveto{\pgfqpoint{3.005115in}{2.388380in}}{\pgfqpoint{2.994516in}{2.383990in}}{\pgfqpoint{2.986703in}{2.376176in}}%
\pgfpathcurveto{\pgfqpoint{2.978889in}{2.368363in}}{\pgfqpoint{2.974499in}{2.357764in}}{\pgfqpoint{2.974499in}{2.346713in}}%
\pgfpathcurveto{\pgfqpoint{2.974499in}{2.335663in}}{\pgfqpoint{2.978889in}{2.325064in}}{\pgfqpoint{2.986703in}{2.317251in}}%
\pgfpathcurveto{\pgfqpoint{2.994516in}{2.309437in}}{\pgfqpoint{3.005115in}{2.305047in}}{\pgfqpoint{3.016165in}{2.305047in}}%
\pgfpathclose%
\pgfusepath{stroke,fill}%
\end{pgfscope}%
\begin{pgfscope}%
\pgfpathrectangle{\pgfqpoint{0.600000in}{0.600000in}}{\pgfqpoint{3.900000in}{3.900000in}}%
\pgfusepath{clip}%
\pgfsetbuttcap%
\pgfsetroundjoin%
\definecolor{currentfill}{rgb}{0.121569,0.466667,0.705882}%
\pgfsetfillcolor{currentfill}%
\pgfsetlinewidth{1.003750pt}%
\definecolor{currentstroke}{rgb}{0.121569,0.466667,0.705882}%
\pgfsetstrokecolor{currentstroke}%
\pgfsetdash{}{0pt}%
\pgfpathmoveto{\pgfqpoint{2.065347in}{2.264227in}}%
\pgfpathcurveto{\pgfqpoint{2.076397in}{2.264227in}}{\pgfqpoint{2.086996in}{2.268617in}}{\pgfqpoint{2.094810in}{2.276431in}}%
\pgfpathcurveto{\pgfqpoint{2.102624in}{2.284244in}}{\pgfqpoint{2.107014in}{2.294843in}}{\pgfqpoint{2.107014in}{2.305894in}}%
\pgfpathcurveto{\pgfqpoint{2.107014in}{2.316944in}}{\pgfqpoint{2.102624in}{2.327543in}}{\pgfqpoint{2.094810in}{2.335356in}}%
\pgfpathcurveto{\pgfqpoint{2.086996in}{2.343170in}}{\pgfqpoint{2.076397in}{2.347560in}}{\pgfqpoint{2.065347in}{2.347560in}}%
\pgfpathcurveto{\pgfqpoint{2.054297in}{2.347560in}}{\pgfqpoint{2.043698in}{2.343170in}}{\pgfqpoint{2.035885in}{2.335356in}}%
\pgfpathcurveto{\pgfqpoint{2.028071in}{2.327543in}}{\pgfqpoint{2.023681in}{2.316944in}}{\pgfqpoint{2.023681in}{2.305894in}}%
\pgfpathcurveto{\pgfqpoint{2.023681in}{2.294843in}}{\pgfqpoint{2.028071in}{2.284244in}}{\pgfqpoint{2.035885in}{2.276431in}}%
\pgfpathcurveto{\pgfqpoint{2.043698in}{2.268617in}}{\pgfqpoint{2.054297in}{2.264227in}}{\pgfqpoint{2.065347in}{2.264227in}}%
\pgfpathclose%
\pgfusepath{stroke,fill}%
\end{pgfscope}%
\begin{pgfscope}%
\pgfpathrectangle{\pgfqpoint{0.600000in}{0.600000in}}{\pgfqpoint{3.900000in}{3.900000in}}%
\pgfusepath{clip}%
\pgfsetbuttcap%
\pgfsetroundjoin%
\definecolor{currentfill}{rgb}{0.121569,0.466667,0.705882}%
\pgfsetfillcolor{currentfill}%
\pgfsetlinewidth{1.003750pt}%
\definecolor{currentstroke}{rgb}{0.121569,0.466667,0.705882}%
\pgfsetstrokecolor{currentstroke}%
\pgfsetdash{}{0pt}%
\pgfpathmoveto{\pgfqpoint{3.010951in}{2.945875in}}%
\pgfpathcurveto{\pgfqpoint{3.022001in}{2.945875in}}{\pgfqpoint{3.032600in}{2.950265in}}{\pgfqpoint{3.040414in}{2.958079in}}%
\pgfpathcurveto{\pgfqpoint{3.048228in}{2.965892in}}{\pgfqpoint{3.052618in}{2.976491in}}{\pgfqpoint{3.052618in}{2.987541in}}%
\pgfpathcurveto{\pgfqpoint{3.052618in}{2.998591in}}{\pgfqpoint{3.048228in}{3.009190in}}{\pgfqpoint{3.040414in}{3.017004in}}%
\pgfpathcurveto{\pgfqpoint{3.032600in}{3.024818in}}{\pgfqpoint{3.022001in}{3.029208in}}{\pgfqpoint{3.010951in}{3.029208in}}%
\pgfpathcurveto{\pgfqpoint{2.999901in}{3.029208in}}{\pgfqpoint{2.989302in}{3.024818in}}{\pgfqpoint{2.981489in}{3.017004in}}%
\pgfpathcurveto{\pgfqpoint{2.973675in}{3.009190in}}{\pgfqpoint{2.969285in}{2.998591in}}{\pgfqpoint{2.969285in}{2.987541in}}%
\pgfpathcurveto{\pgfqpoint{2.969285in}{2.976491in}}{\pgfqpoint{2.973675in}{2.965892in}}{\pgfqpoint{2.981489in}{2.958079in}}%
\pgfpathcurveto{\pgfqpoint{2.989302in}{2.950265in}}{\pgfqpoint{2.999901in}{2.945875in}}{\pgfqpoint{3.010951in}{2.945875in}}%
\pgfpathclose%
\pgfusepath{stroke,fill}%
\end{pgfscope}%
\begin{pgfscope}%
\pgfpathrectangle{\pgfqpoint{0.600000in}{0.600000in}}{\pgfqpoint{3.900000in}{3.900000in}}%
\pgfusepath{clip}%
\pgfsetbuttcap%
\pgfsetroundjoin%
\definecolor{currentfill}{rgb}{0.121569,0.466667,0.705882}%
\pgfsetfillcolor{currentfill}%
\pgfsetlinewidth{1.003750pt}%
\definecolor{currentstroke}{rgb}{0.121569,0.466667,0.705882}%
\pgfsetstrokecolor{currentstroke}%
\pgfsetdash{}{0pt}%
\pgfpathmoveto{\pgfqpoint{3.421776in}{2.500706in}}%
\pgfpathcurveto{\pgfqpoint{3.432826in}{2.500706in}}{\pgfqpoint{3.443425in}{2.505096in}}{\pgfqpoint{3.451238in}{2.512910in}}%
\pgfpathcurveto{\pgfqpoint{3.459052in}{2.520724in}}{\pgfqpoint{3.463442in}{2.531323in}}{\pgfqpoint{3.463442in}{2.542373in}}%
\pgfpathcurveto{\pgfqpoint{3.463442in}{2.553423in}}{\pgfqpoint{3.459052in}{2.564022in}}{\pgfqpoint{3.451238in}{2.571835in}}%
\pgfpathcurveto{\pgfqpoint{3.443425in}{2.579649in}}{\pgfqpoint{3.432826in}{2.584039in}}{\pgfqpoint{3.421776in}{2.584039in}}%
\pgfpathcurveto{\pgfqpoint{3.410725in}{2.584039in}}{\pgfqpoint{3.400126in}{2.579649in}}{\pgfqpoint{3.392313in}{2.571835in}}%
\pgfpathcurveto{\pgfqpoint{3.384499in}{2.564022in}}{\pgfqpoint{3.380109in}{2.553423in}}{\pgfqpoint{3.380109in}{2.542373in}}%
\pgfpathcurveto{\pgfqpoint{3.380109in}{2.531323in}}{\pgfqpoint{3.384499in}{2.520724in}}{\pgfqpoint{3.392313in}{2.512910in}}%
\pgfpathcurveto{\pgfqpoint{3.400126in}{2.505096in}}{\pgfqpoint{3.410725in}{2.500706in}}{\pgfqpoint{3.421776in}{2.500706in}}%
\pgfpathclose%
\pgfusepath{stroke,fill}%
\end{pgfscope}%
\begin{pgfscope}%
\pgfpathrectangle{\pgfqpoint{0.600000in}{0.600000in}}{\pgfqpoint{3.900000in}{3.900000in}}%
\pgfusepath{clip}%
\pgfsetbuttcap%
\pgfsetroundjoin%
\definecolor{currentfill}{rgb}{0.121569,0.466667,0.705882}%
\pgfsetfillcolor{currentfill}%
\pgfsetlinewidth{1.003750pt}%
\definecolor{currentstroke}{rgb}{0.121569,0.466667,0.705882}%
\pgfsetstrokecolor{currentstroke}%
\pgfsetdash{}{0pt}%
\pgfpathmoveto{\pgfqpoint{2.913803in}{2.089243in}}%
\pgfpathcurveto{\pgfqpoint{2.924853in}{2.089243in}}{\pgfqpoint{2.935452in}{2.093633in}}{\pgfqpoint{2.943266in}{2.101447in}}%
\pgfpathcurveto{\pgfqpoint{2.951079in}{2.109260in}}{\pgfqpoint{2.955470in}{2.119859in}}{\pgfqpoint{2.955470in}{2.130910in}}%
\pgfpathcurveto{\pgfqpoint{2.955470in}{2.141960in}}{\pgfqpoint{2.951079in}{2.152559in}}{\pgfqpoint{2.943266in}{2.160372in}}%
\pgfpathcurveto{\pgfqpoint{2.935452in}{2.168186in}}{\pgfqpoint{2.924853in}{2.172576in}}{\pgfqpoint{2.913803in}{2.172576in}}%
\pgfpathcurveto{\pgfqpoint{2.902753in}{2.172576in}}{\pgfqpoint{2.892154in}{2.168186in}}{\pgfqpoint{2.884340in}{2.160372in}}%
\pgfpathcurveto{\pgfqpoint{2.876527in}{2.152559in}}{\pgfqpoint{2.872136in}{2.141960in}}{\pgfqpoint{2.872136in}{2.130910in}}%
\pgfpathcurveto{\pgfqpoint{2.872136in}{2.119859in}}{\pgfqpoint{2.876527in}{2.109260in}}{\pgfqpoint{2.884340in}{2.101447in}}%
\pgfpathcurveto{\pgfqpoint{2.892154in}{2.093633in}}{\pgfqpoint{2.902753in}{2.089243in}}{\pgfqpoint{2.913803in}{2.089243in}}%
\pgfpathclose%
\pgfusepath{stroke,fill}%
\end{pgfscope}%
\begin{pgfscope}%
\pgfpathrectangle{\pgfqpoint{0.600000in}{0.600000in}}{\pgfqpoint{3.900000in}{3.900000in}}%
\pgfusepath{clip}%
\pgfsetbuttcap%
\pgfsetroundjoin%
\definecolor{currentfill}{rgb}{0.121569,0.466667,0.705882}%
\pgfsetfillcolor{currentfill}%
\pgfsetlinewidth{1.003750pt}%
\definecolor{currentstroke}{rgb}{0.121569,0.466667,0.705882}%
\pgfsetstrokecolor{currentstroke}%
\pgfsetdash{}{0pt}%
\pgfpathmoveto{\pgfqpoint{2.448112in}{1.704084in}}%
\pgfpathcurveto{\pgfqpoint{2.459162in}{1.704084in}}{\pgfqpoint{2.469761in}{1.708474in}}{\pgfqpoint{2.477575in}{1.716288in}}%
\pgfpathcurveto{\pgfqpoint{2.485389in}{1.724101in}}{\pgfqpoint{2.489779in}{1.734700in}}{\pgfqpoint{2.489779in}{1.745751in}}%
\pgfpathcurveto{\pgfqpoint{2.489779in}{1.756801in}}{\pgfqpoint{2.485389in}{1.767400in}}{\pgfqpoint{2.477575in}{1.775213in}}%
\pgfpathcurveto{\pgfqpoint{2.469761in}{1.783027in}}{\pgfqpoint{2.459162in}{1.787417in}}{\pgfqpoint{2.448112in}{1.787417in}}%
\pgfpathcurveto{\pgfqpoint{2.437062in}{1.787417in}}{\pgfqpoint{2.426463in}{1.783027in}}{\pgfqpoint{2.418650in}{1.775213in}}%
\pgfpathcurveto{\pgfqpoint{2.410836in}{1.767400in}}{\pgfqpoint{2.406446in}{1.756801in}}{\pgfqpoint{2.406446in}{1.745751in}}%
\pgfpathcurveto{\pgfqpoint{2.406446in}{1.734700in}}{\pgfqpoint{2.410836in}{1.724101in}}{\pgfqpoint{2.418650in}{1.716288in}}%
\pgfpathcurveto{\pgfqpoint{2.426463in}{1.708474in}}{\pgfqpoint{2.437062in}{1.704084in}}{\pgfqpoint{2.448112in}{1.704084in}}%
\pgfpathclose%
\pgfusepath{stroke,fill}%
\end{pgfscope}%
\begin{pgfscope}%
\pgfpathrectangle{\pgfqpoint{0.600000in}{0.600000in}}{\pgfqpoint{3.900000in}{3.900000in}}%
\pgfusepath{clip}%
\pgfsetbuttcap%
\pgfsetroundjoin%
\definecolor{currentfill}{rgb}{0.121569,0.466667,0.705882}%
\pgfsetfillcolor{currentfill}%
\pgfsetlinewidth{1.003750pt}%
\definecolor{currentstroke}{rgb}{0.121569,0.466667,0.705882}%
\pgfsetstrokecolor{currentstroke}%
\pgfsetdash{}{0pt}%
\pgfpathmoveto{\pgfqpoint{2.228767in}{2.531112in}}%
\pgfpathcurveto{\pgfqpoint{2.239817in}{2.531112in}}{\pgfqpoint{2.250416in}{2.535502in}}{\pgfqpoint{2.258230in}{2.543316in}}%
\pgfpathcurveto{\pgfqpoint{2.266044in}{2.551129in}}{\pgfqpoint{2.270434in}{2.561728in}}{\pgfqpoint{2.270434in}{2.572779in}}%
\pgfpathcurveto{\pgfqpoint{2.270434in}{2.583829in}}{\pgfqpoint{2.266044in}{2.594428in}}{\pgfqpoint{2.258230in}{2.602241in}}%
\pgfpathcurveto{\pgfqpoint{2.250416in}{2.610055in}}{\pgfqpoint{2.239817in}{2.614445in}}{\pgfqpoint{2.228767in}{2.614445in}}%
\pgfpathcurveto{\pgfqpoint{2.217717in}{2.614445in}}{\pgfqpoint{2.207118in}{2.610055in}}{\pgfqpoint{2.199305in}{2.602241in}}%
\pgfpathcurveto{\pgfqpoint{2.191491in}{2.594428in}}{\pgfqpoint{2.187101in}{2.583829in}}{\pgfqpoint{2.187101in}{2.572779in}}%
\pgfpathcurveto{\pgfqpoint{2.187101in}{2.561728in}}{\pgfqpoint{2.191491in}{2.551129in}}{\pgfqpoint{2.199305in}{2.543316in}}%
\pgfpathcurveto{\pgfqpoint{2.207118in}{2.535502in}}{\pgfqpoint{2.217717in}{2.531112in}}{\pgfqpoint{2.228767in}{2.531112in}}%
\pgfpathclose%
\pgfusepath{stroke,fill}%
\end{pgfscope}%
\begin{pgfscope}%
\pgfpathrectangle{\pgfqpoint{0.600000in}{0.600000in}}{\pgfqpoint{3.900000in}{3.900000in}}%
\pgfusepath{clip}%
\pgfsetbuttcap%
\pgfsetroundjoin%
\definecolor{currentfill}{rgb}{0.121569,0.466667,0.705882}%
\pgfsetfillcolor{currentfill}%
\pgfsetlinewidth{1.003750pt}%
\definecolor{currentstroke}{rgb}{0.121569,0.466667,0.705882}%
\pgfsetstrokecolor{currentstroke}%
\pgfsetdash{}{0pt}%
\pgfpathmoveto{\pgfqpoint{2.949609in}{2.848161in}}%
\pgfpathcurveto{\pgfqpoint{2.960660in}{2.848161in}}{\pgfqpoint{2.971259in}{2.852552in}}{\pgfqpoint{2.979072in}{2.860365in}}%
\pgfpathcurveto{\pgfqpoint{2.986886in}{2.868179in}}{\pgfqpoint{2.991276in}{2.878778in}}{\pgfqpoint{2.991276in}{2.889828in}}%
\pgfpathcurveto{\pgfqpoint{2.991276in}{2.900878in}}{\pgfqpoint{2.986886in}{2.911477in}}{\pgfqpoint{2.979072in}{2.919291in}}%
\pgfpathcurveto{\pgfqpoint{2.971259in}{2.927104in}}{\pgfqpoint{2.960660in}{2.931495in}}{\pgfqpoint{2.949609in}{2.931495in}}%
\pgfpathcurveto{\pgfqpoint{2.938559in}{2.931495in}}{\pgfqpoint{2.927960in}{2.927104in}}{\pgfqpoint{2.920147in}{2.919291in}}%
\pgfpathcurveto{\pgfqpoint{2.912333in}{2.911477in}}{\pgfqpoint{2.907943in}{2.900878in}}{\pgfqpoint{2.907943in}{2.889828in}}%
\pgfpathcurveto{\pgfqpoint{2.907943in}{2.878778in}}{\pgfqpoint{2.912333in}{2.868179in}}{\pgfqpoint{2.920147in}{2.860365in}}%
\pgfpathcurveto{\pgfqpoint{2.927960in}{2.852552in}}{\pgfqpoint{2.938559in}{2.848161in}}{\pgfqpoint{2.949609in}{2.848161in}}%
\pgfpathclose%
\pgfusepath{stroke,fill}%
\end{pgfscope}%
\begin{pgfscope}%
\pgfpathrectangle{\pgfqpoint{0.600000in}{0.600000in}}{\pgfqpoint{3.900000in}{3.900000in}}%
\pgfusepath{clip}%
\pgfsetbuttcap%
\pgfsetroundjoin%
\definecolor{currentfill}{rgb}{0.121569,0.466667,0.705882}%
\pgfsetfillcolor{currentfill}%
\pgfsetlinewidth{1.003750pt}%
\definecolor{currentstroke}{rgb}{0.121569,0.466667,0.705882}%
\pgfsetstrokecolor{currentstroke}%
\pgfsetdash{}{0pt}%
\pgfpathmoveto{\pgfqpoint{2.240043in}{1.573077in}}%
\pgfpathcurveto{\pgfqpoint{2.251093in}{1.573077in}}{\pgfqpoint{2.261692in}{1.577467in}}{\pgfqpoint{2.269506in}{1.585281in}}%
\pgfpathcurveto{\pgfqpoint{2.277319in}{1.593094in}}{\pgfqpoint{2.281709in}{1.603693in}}{\pgfqpoint{2.281709in}{1.614743in}}%
\pgfpathcurveto{\pgfqpoint{2.281709in}{1.625793in}}{\pgfqpoint{2.277319in}{1.636393in}}{\pgfqpoint{2.269506in}{1.644206in}}%
\pgfpathcurveto{\pgfqpoint{2.261692in}{1.652020in}}{\pgfqpoint{2.251093in}{1.656410in}}{\pgfqpoint{2.240043in}{1.656410in}}%
\pgfpathcurveto{\pgfqpoint{2.228993in}{1.656410in}}{\pgfqpoint{2.218394in}{1.652020in}}{\pgfqpoint{2.210580in}{1.644206in}}%
\pgfpathcurveto{\pgfqpoint{2.202766in}{1.636393in}}{\pgfqpoint{2.198376in}{1.625793in}}{\pgfqpoint{2.198376in}{1.614743in}}%
\pgfpathcurveto{\pgfqpoint{2.198376in}{1.603693in}}{\pgfqpoint{2.202766in}{1.593094in}}{\pgfqpoint{2.210580in}{1.585281in}}%
\pgfpathcurveto{\pgfqpoint{2.218394in}{1.577467in}}{\pgfqpoint{2.228993in}{1.573077in}}{\pgfqpoint{2.240043in}{1.573077in}}%
\pgfpathclose%
\pgfusepath{stroke,fill}%
\end{pgfscope}%
\begin{pgfscope}%
\pgfpathrectangle{\pgfqpoint{0.600000in}{0.600000in}}{\pgfqpoint{3.900000in}{3.900000in}}%
\pgfusepath{clip}%
\pgfsetbuttcap%
\pgfsetroundjoin%
\definecolor{currentfill}{rgb}{0.121569,0.466667,0.705882}%
\pgfsetfillcolor{currentfill}%
\pgfsetlinewidth{1.003750pt}%
\definecolor{currentstroke}{rgb}{0.121569,0.466667,0.705882}%
\pgfsetstrokecolor{currentstroke}%
\pgfsetdash{}{0pt}%
\pgfpathmoveto{\pgfqpoint{2.937965in}{2.157504in}}%
\pgfpathcurveto{\pgfqpoint{2.949015in}{2.157504in}}{\pgfqpoint{2.959614in}{2.161894in}}{\pgfqpoint{2.967428in}{2.169708in}}%
\pgfpathcurveto{\pgfqpoint{2.975241in}{2.177521in}}{\pgfqpoint{2.979632in}{2.188120in}}{\pgfqpoint{2.979632in}{2.199170in}}%
\pgfpathcurveto{\pgfqpoint{2.979632in}{2.210221in}}{\pgfqpoint{2.975241in}{2.220820in}}{\pgfqpoint{2.967428in}{2.228633in}}%
\pgfpathcurveto{\pgfqpoint{2.959614in}{2.236447in}}{\pgfqpoint{2.949015in}{2.240837in}}{\pgfqpoint{2.937965in}{2.240837in}}%
\pgfpathcurveto{\pgfqpoint{2.926915in}{2.240837in}}{\pgfqpoint{2.916316in}{2.236447in}}{\pgfqpoint{2.908502in}{2.228633in}}%
\pgfpathcurveto{\pgfqpoint{2.900688in}{2.220820in}}{\pgfqpoint{2.896298in}{2.210221in}}{\pgfqpoint{2.896298in}{2.199170in}}%
\pgfpathcurveto{\pgfqpoint{2.896298in}{2.188120in}}{\pgfqpoint{2.900688in}{2.177521in}}{\pgfqpoint{2.908502in}{2.169708in}}%
\pgfpathcurveto{\pgfqpoint{2.916316in}{2.161894in}}{\pgfqpoint{2.926915in}{2.157504in}}{\pgfqpoint{2.937965in}{2.157504in}}%
\pgfpathclose%
\pgfusepath{stroke,fill}%
\end{pgfscope}%
\begin{pgfscope}%
\pgfpathrectangle{\pgfqpoint{0.600000in}{0.600000in}}{\pgfqpoint{3.900000in}{3.900000in}}%
\pgfusepath{clip}%
\pgfsetbuttcap%
\pgfsetroundjoin%
\definecolor{currentfill}{rgb}{0.121569,0.466667,0.705882}%
\pgfsetfillcolor{currentfill}%
\pgfsetlinewidth{1.003750pt}%
\definecolor{currentstroke}{rgb}{0.121569,0.466667,0.705882}%
\pgfsetstrokecolor{currentstroke}%
\pgfsetdash{}{0pt}%
\pgfpathmoveto{\pgfqpoint{2.715233in}{3.153187in}}%
\pgfpathcurveto{\pgfqpoint{2.726283in}{3.153187in}}{\pgfqpoint{2.736882in}{3.157577in}}{\pgfqpoint{2.744696in}{3.165391in}}%
\pgfpathcurveto{\pgfqpoint{2.752509in}{3.173205in}}{\pgfqpoint{2.756900in}{3.183804in}}{\pgfqpoint{2.756900in}{3.194854in}}%
\pgfpathcurveto{\pgfqpoint{2.756900in}{3.205904in}}{\pgfqpoint{2.752509in}{3.216503in}}{\pgfqpoint{2.744696in}{3.224317in}}%
\pgfpathcurveto{\pgfqpoint{2.736882in}{3.232130in}}{\pgfqpoint{2.726283in}{3.236520in}}{\pgfqpoint{2.715233in}{3.236520in}}%
\pgfpathcurveto{\pgfqpoint{2.704183in}{3.236520in}}{\pgfqpoint{2.693584in}{3.232130in}}{\pgfqpoint{2.685770in}{3.224317in}}%
\pgfpathcurveto{\pgfqpoint{2.677956in}{3.216503in}}{\pgfqpoint{2.673566in}{3.205904in}}{\pgfqpoint{2.673566in}{3.194854in}}%
\pgfpathcurveto{\pgfqpoint{2.673566in}{3.183804in}}{\pgfqpoint{2.677956in}{3.173205in}}{\pgfqpoint{2.685770in}{3.165391in}}%
\pgfpathcurveto{\pgfqpoint{2.693584in}{3.157577in}}{\pgfqpoint{2.704183in}{3.153187in}}{\pgfqpoint{2.715233in}{3.153187in}}%
\pgfpathclose%
\pgfusepath{stroke,fill}%
\end{pgfscope}%
\begin{pgfscope}%
\pgfpathrectangle{\pgfqpoint{0.600000in}{0.600000in}}{\pgfqpoint{3.900000in}{3.900000in}}%
\pgfusepath{clip}%
\pgfsetbuttcap%
\pgfsetroundjoin%
\definecolor{currentfill}{rgb}{0.121569,0.466667,0.705882}%
\pgfsetfillcolor{currentfill}%
\pgfsetlinewidth{1.003750pt}%
\definecolor{currentstroke}{rgb}{0.121569,0.466667,0.705882}%
\pgfsetstrokecolor{currentstroke}%
\pgfsetdash{}{0pt}%
\pgfpathmoveto{\pgfqpoint{2.187521in}{1.633463in}}%
\pgfpathcurveto{\pgfqpoint{2.198571in}{1.633463in}}{\pgfqpoint{2.209170in}{1.637853in}}{\pgfqpoint{2.216984in}{1.645667in}}%
\pgfpathcurveto{\pgfqpoint{2.224797in}{1.653481in}}{\pgfqpoint{2.229187in}{1.664080in}}{\pgfqpoint{2.229187in}{1.675130in}}%
\pgfpathcurveto{\pgfqpoint{2.229187in}{1.686180in}}{\pgfqpoint{2.224797in}{1.696779in}}{\pgfqpoint{2.216984in}{1.704592in}}%
\pgfpathcurveto{\pgfqpoint{2.209170in}{1.712406in}}{\pgfqpoint{2.198571in}{1.716796in}}{\pgfqpoint{2.187521in}{1.716796in}}%
\pgfpathcurveto{\pgfqpoint{2.176471in}{1.716796in}}{\pgfqpoint{2.165872in}{1.712406in}}{\pgfqpoint{2.158058in}{1.704592in}}%
\pgfpathcurveto{\pgfqpoint{2.150244in}{1.696779in}}{\pgfqpoint{2.145854in}{1.686180in}}{\pgfqpoint{2.145854in}{1.675130in}}%
\pgfpathcurveto{\pgfqpoint{2.145854in}{1.664080in}}{\pgfqpoint{2.150244in}{1.653481in}}{\pgfqpoint{2.158058in}{1.645667in}}%
\pgfpathcurveto{\pgfqpoint{2.165872in}{1.637853in}}{\pgfqpoint{2.176471in}{1.633463in}}{\pgfqpoint{2.187521in}{1.633463in}}%
\pgfpathclose%
\pgfusepath{stroke,fill}%
\end{pgfscope}%
\begin{pgfscope}%
\pgfpathrectangle{\pgfqpoint{0.600000in}{0.600000in}}{\pgfqpoint{3.900000in}{3.900000in}}%
\pgfusepath{clip}%
\pgfsetbuttcap%
\pgfsetroundjoin%
\definecolor{currentfill}{rgb}{0.121569,0.466667,0.705882}%
\pgfsetfillcolor{currentfill}%
\pgfsetlinewidth{1.003750pt}%
\definecolor{currentstroke}{rgb}{0.121569,0.466667,0.705882}%
\pgfsetstrokecolor{currentstroke}%
\pgfsetdash{}{0pt}%
\pgfpathmoveto{\pgfqpoint{2.468737in}{3.400150in}}%
\pgfpathcurveto{\pgfqpoint{2.479788in}{3.400150in}}{\pgfqpoint{2.490387in}{3.404541in}}{\pgfqpoint{2.498200in}{3.412354in}}%
\pgfpathcurveto{\pgfqpoint{2.506014in}{3.420168in}}{\pgfqpoint{2.510404in}{3.430767in}}{\pgfqpoint{2.510404in}{3.441817in}}%
\pgfpathcurveto{\pgfqpoint{2.510404in}{3.452867in}}{\pgfqpoint{2.506014in}{3.463466in}}{\pgfqpoint{2.498200in}{3.471280in}}%
\pgfpathcurveto{\pgfqpoint{2.490387in}{3.479093in}}{\pgfqpoint{2.479788in}{3.483484in}}{\pgfqpoint{2.468737in}{3.483484in}}%
\pgfpathcurveto{\pgfqpoint{2.457687in}{3.483484in}}{\pgfqpoint{2.447088in}{3.479093in}}{\pgfqpoint{2.439275in}{3.471280in}}%
\pgfpathcurveto{\pgfqpoint{2.431461in}{3.463466in}}{\pgfqpoint{2.427071in}{3.452867in}}{\pgfqpoint{2.427071in}{3.441817in}}%
\pgfpathcurveto{\pgfqpoint{2.427071in}{3.430767in}}{\pgfqpoint{2.431461in}{3.420168in}}{\pgfqpoint{2.439275in}{3.412354in}}%
\pgfpathcurveto{\pgfqpoint{2.447088in}{3.404541in}}{\pgfqpoint{2.457687in}{3.400150in}}{\pgfqpoint{2.468737in}{3.400150in}}%
\pgfpathclose%
\pgfusepath{stroke,fill}%
\end{pgfscope}%
\begin{pgfscope}%
\pgfpathrectangle{\pgfqpoint{0.600000in}{0.600000in}}{\pgfqpoint{3.900000in}{3.900000in}}%
\pgfusepath{clip}%
\pgfsetbuttcap%
\pgfsetroundjoin%
\definecolor{currentfill}{rgb}{0.121569,0.466667,0.705882}%
\pgfsetfillcolor{currentfill}%
\pgfsetlinewidth{1.003750pt}%
\definecolor{currentstroke}{rgb}{0.121569,0.466667,0.705882}%
\pgfsetstrokecolor{currentstroke}%
\pgfsetdash{}{0pt}%
\pgfpathmoveto{\pgfqpoint{1.662091in}{2.549334in}}%
\pgfpathcurveto{\pgfqpoint{1.673141in}{2.549334in}}{\pgfqpoint{1.683740in}{2.553724in}}{\pgfqpoint{1.691554in}{2.561538in}}%
\pgfpathcurveto{\pgfqpoint{1.699367in}{2.569352in}}{\pgfqpoint{1.703758in}{2.579951in}}{\pgfqpoint{1.703758in}{2.591001in}}%
\pgfpathcurveto{\pgfqpoint{1.703758in}{2.602051in}}{\pgfqpoint{1.699367in}{2.612650in}}{\pgfqpoint{1.691554in}{2.620464in}}%
\pgfpathcurveto{\pgfqpoint{1.683740in}{2.628277in}}{\pgfqpoint{1.673141in}{2.632668in}}{\pgfqpoint{1.662091in}{2.632668in}}%
\pgfpathcurveto{\pgfqpoint{1.651041in}{2.632668in}}{\pgfqpoint{1.640442in}{2.628277in}}{\pgfqpoint{1.632628in}{2.620464in}}%
\pgfpathcurveto{\pgfqpoint{1.624814in}{2.612650in}}{\pgfqpoint{1.620424in}{2.602051in}}{\pgfqpoint{1.620424in}{2.591001in}}%
\pgfpathcurveto{\pgfqpoint{1.620424in}{2.579951in}}{\pgfqpoint{1.624814in}{2.569352in}}{\pgfqpoint{1.632628in}{2.561538in}}%
\pgfpathcurveto{\pgfqpoint{1.640442in}{2.553724in}}{\pgfqpoint{1.651041in}{2.549334in}}{\pgfqpoint{1.662091in}{2.549334in}}%
\pgfpathclose%
\pgfusepath{stroke,fill}%
\end{pgfscope}%
\begin{pgfscope}%
\pgfpathrectangle{\pgfqpoint{0.600000in}{0.600000in}}{\pgfqpoint{3.900000in}{3.900000in}}%
\pgfusepath{clip}%
\pgfsetbuttcap%
\pgfsetroundjoin%
\definecolor{currentfill}{rgb}{0.121569,0.466667,0.705882}%
\pgfsetfillcolor{currentfill}%
\pgfsetlinewidth{1.003750pt}%
\definecolor{currentstroke}{rgb}{0.121569,0.466667,0.705882}%
\pgfsetstrokecolor{currentstroke}%
\pgfsetdash{}{0pt}%
\pgfpathmoveto{\pgfqpoint{1.764415in}{2.785293in}}%
\pgfpathcurveto{\pgfqpoint{1.775465in}{2.785293in}}{\pgfqpoint{1.786064in}{2.789683in}}{\pgfqpoint{1.793877in}{2.797496in}}%
\pgfpathcurveto{\pgfqpoint{1.801691in}{2.805310in}}{\pgfqpoint{1.806081in}{2.815909in}}{\pgfqpoint{1.806081in}{2.826959in}}%
\pgfpathcurveto{\pgfqpoint{1.806081in}{2.838009in}}{\pgfqpoint{1.801691in}{2.848608in}}{\pgfqpoint{1.793877in}{2.856422in}}%
\pgfpathcurveto{\pgfqpoint{1.786064in}{2.864236in}}{\pgfqpoint{1.775465in}{2.868626in}}{\pgfqpoint{1.764415in}{2.868626in}}%
\pgfpathcurveto{\pgfqpoint{1.753364in}{2.868626in}}{\pgfqpoint{1.742765in}{2.864236in}}{\pgfqpoint{1.734952in}{2.856422in}}%
\pgfpathcurveto{\pgfqpoint{1.727138in}{2.848608in}}{\pgfqpoint{1.722748in}{2.838009in}}{\pgfqpoint{1.722748in}{2.826959in}}%
\pgfpathcurveto{\pgfqpoint{1.722748in}{2.815909in}}{\pgfqpoint{1.727138in}{2.805310in}}{\pgfqpoint{1.734952in}{2.797496in}}%
\pgfpathcurveto{\pgfqpoint{1.742765in}{2.789683in}}{\pgfqpoint{1.753364in}{2.785293in}}{\pgfqpoint{1.764415in}{2.785293in}}%
\pgfpathclose%
\pgfusepath{stroke,fill}%
\end{pgfscope}%
\begin{pgfscope}%
\pgfpathrectangle{\pgfqpoint{0.600000in}{0.600000in}}{\pgfqpoint{3.900000in}{3.900000in}}%
\pgfusepath{clip}%
\pgfsetbuttcap%
\pgfsetroundjoin%
\definecolor{currentfill}{rgb}{0.121569,0.466667,0.705882}%
\pgfsetfillcolor{currentfill}%
\pgfsetlinewidth{1.003750pt}%
\definecolor{currentstroke}{rgb}{0.121569,0.466667,0.705882}%
\pgfsetstrokecolor{currentstroke}%
\pgfsetdash{}{0pt}%
\pgfpathmoveto{\pgfqpoint{2.297404in}{3.342660in}}%
\pgfpathcurveto{\pgfqpoint{2.308454in}{3.342660in}}{\pgfqpoint{2.319053in}{3.347050in}}{\pgfqpoint{2.326867in}{3.354864in}}%
\pgfpathcurveto{\pgfqpoint{2.334681in}{3.362677in}}{\pgfqpoint{2.339071in}{3.373276in}}{\pgfqpoint{2.339071in}{3.384327in}}%
\pgfpathcurveto{\pgfqpoint{2.339071in}{3.395377in}}{\pgfqpoint{2.334681in}{3.405976in}}{\pgfqpoint{2.326867in}{3.413789in}}%
\pgfpathcurveto{\pgfqpoint{2.319053in}{3.421603in}}{\pgfqpoint{2.308454in}{3.425993in}}{\pgfqpoint{2.297404in}{3.425993in}}%
\pgfpathcurveto{\pgfqpoint{2.286354in}{3.425993in}}{\pgfqpoint{2.275755in}{3.421603in}}{\pgfqpoint{2.267941in}{3.413789in}}%
\pgfpathcurveto{\pgfqpoint{2.260128in}{3.405976in}}{\pgfqpoint{2.255738in}{3.395377in}}{\pgfqpoint{2.255738in}{3.384327in}}%
\pgfpathcurveto{\pgfqpoint{2.255738in}{3.373276in}}{\pgfqpoint{2.260128in}{3.362677in}}{\pgfqpoint{2.267941in}{3.354864in}}%
\pgfpathcurveto{\pgfqpoint{2.275755in}{3.347050in}}{\pgfqpoint{2.286354in}{3.342660in}}{\pgfqpoint{2.297404in}{3.342660in}}%
\pgfpathclose%
\pgfusepath{stroke,fill}%
\end{pgfscope}%
\begin{pgfscope}%
\pgfpathrectangle{\pgfqpoint{0.600000in}{0.600000in}}{\pgfqpoint{3.900000in}{3.900000in}}%
\pgfusepath{clip}%
\pgfsetbuttcap%
\pgfsetroundjoin%
\definecolor{currentfill}{rgb}{0.121569,0.466667,0.705882}%
\pgfsetfillcolor{currentfill}%
\pgfsetlinewidth{1.003750pt}%
\definecolor{currentstroke}{rgb}{0.121569,0.466667,0.705882}%
\pgfsetstrokecolor{currentstroke}%
\pgfsetdash{}{0pt}%
\pgfpathmoveto{\pgfqpoint{2.549295in}{2.032554in}}%
\pgfpathcurveto{\pgfqpoint{2.560345in}{2.032554in}}{\pgfqpoint{2.570945in}{2.036944in}}{\pgfqpoint{2.578758in}{2.044758in}}%
\pgfpathcurveto{\pgfqpoint{2.586572in}{2.052572in}}{\pgfqpoint{2.590962in}{2.063171in}}{\pgfqpoint{2.590962in}{2.074221in}}%
\pgfpathcurveto{\pgfqpoint{2.590962in}{2.085271in}}{\pgfqpoint{2.586572in}{2.095870in}}{\pgfqpoint{2.578758in}{2.103684in}}%
\pgfpathcurveto{\pgfqpoint{2.570945in}{2.111497in}}{\pgfqpoint{2.560345in}{2.115887in}}{\pgfqpoint{2.549295in}{2.115887in}}%
\pgfpathcurveto{\pgfqpoint{2.538245in}{2.115887in}}{\pgfqpoint{2.527646in}{2.111497in}}{\pgfqpoint{2.519833in}{2.103684in}}%
\pgfpathcurveto{\pgfqpoint{2.512019in}{2.095870in}}{\pgfqpoint{2.507629in}{2.085271in}}{\pgfqpoint{2.507629in}{2.074221in}}%
\pgfpathcurveto{\pgfqpoint{2.507629in}{2.063171in}}{\pgfqpoint{2.512019in}{2.052572in}}{\pgfqpoint{2.519833in}{2.044758in}}%
\pgfpathcurveto{\pgfqpoint{2.527646in}{2.036944in}}{\pgfqpoint{2.538245in}{2.032554in}}{\pgfqpoint{2.549295in}{2.032554in}}%
\pgfpathclose%
\pgfusepath{stroke,fill}%
\end{pgfscope}%
\begin{pgfscope}%
\pgfpathrectangle{\pgfqpoint{0.600000in}{0.600000in}}{\pgfqpoint{3.900000in}{3.900000in}}%
\pgfusepath{clip}%
\pgfsetbuttcap%
\pgfsetroundjoin%
\definecolor{currentfill}{rgb}{0.121569,0.466667,0.705882}%
\pgfsetfillcolor{currentfill}%
\pgfsetlinewidth{1.003750pt}%
\definecolor{currentstroke}{rgb}{0.121569,0.466667,0.705882}%
\pgfsetstrokecolor{currentstroke}%
\pgfsetdash{}{0pt}%
\pgfpathmoveto{\pgfqpoint{2.158244in}{2.708282in}}%
\pgfpathcurveto{\pgfqpoint{2.169294in}{2.708282in}}{\pgfqpoint{2.179893in}{2.712672in}}{\pgfqpoint{2.187707in}{2.720486in}}%
\pgfpathcurveto{\pgfqpoint{2.195520in}{2.728299in}}{\pgfqpoint{2.199911in}{2.738898in}}{\pgfqpoint{2.199911in}{2.749948in}}%
\pgfpathcurveto{\pgfqpoint{2.199911in}{2.760998in}}{\pgfqpoint{2.195520in}{2.771597in}}{\pgfqpoint{2.187707in}{2.779411in}}%
\pgfpathcurveto{\pgfqpoint{2.179893in}{2.787225in}}{\pgfqpoint{2.169294in}{2.791615in}}{\pgfqpoint{2.158244in}{2.791615in}}%
\pgfpathcurveto{\pgfqpoint{2.147194in}{2.791615in}}{\pgfqpoint{2.136595in}{2.787225in}}{\pgfqpoint{2.128781in}{2.779411in}}%
\pgfpathcurveto{\pgfqpoint{2.120968in}{2.771597in}}{\pgfqpoint{2.116577in}{2.760998in}}{\pgfqpoint{2.116577in}{2.749948in}}%
\pgfpathcurveto{\pgfqpoint{2.116577in}{2.738898in}}{\pgfqpoint{2.120968in}{2.728299in}}{\pgfqpoint{2.128781in}{2.720486in}}%
\pgfpathcurveto{\pgfqpoint{2.136595in}{2.712672in}}{\pgfqpoint{2.147194in}{2.708282in}}{\pgfqpoint{2.158244in}{2.708282in}}%
\pgfpathclose%
\pgfusepath{stroke,fill}%
\end{pgfscope}%
\begin{pgfscope}%
\pgfpathrectangle{\pgfqpoint{0.600000in}{0.600000in}}{\pgfqpoint{3.900000in}{3.900000in}}%
\pgfusepath{clip}%
\pgfsetbuttcap%
\pgfsetroundjoin%
\definecolor{currentfill}{rgb}{0.121569,0.466667,0.705882}%
\pgfsetfillcolor{currentfill}%
\pgfsetlinewidth{1.003750pt}%
\definecolor{currentstroke}{rgb}{0.121569,0.466667,0.705882}%
\pgfsetstrokecolor{currentstroke}%
\pgfsetdash{}{0pt}%
\pgfpathmoveto{\pgfqpoint{2.244373in}{2.735127in}}%
\pgfpathcurveto{\pgfqpoint{2.255423in}{2.735127in}}{\pgfqpoint{2.266022in}{2.739517in}}{\pgfqpoint{2.273835in}{2.747330in}}%
\pgfpathcurveto{\pgfqpoint{2.281649in}{2.755144in}}{\pgfqpoint{2.286039in}{2.765743in}}{\pgfqpoint{2.286039in}{2.776793in}}%
\pgfpathcurveto{\pgfqpoint{2.286039in}{2.787843in}}{\pgfqpoint{2.281649in}{2.798442in}}{\pgfqpoint{2.273835in}{2.806256in}}%
\pgfpathcurveto{\pgfqpoint{2.266022in}{2.814070in}}{\pgfqpoint{2.255423in}{2.818460in}}{\pgfqpoint{2.244373in}{2.818460in}}%
\pgfpathcurveto{\pgfqpoint{2.233322in}{2.818460in}}{\pgfqpoint{2.222723in}{2.814070in}}{\pgfqpoint{2.214910in}{2.806256in}}%
\pgfpathcurveto{\pgfqpoint{2.207096in}{2.798442in}}{\pgfqpoint{2.202706in}{2.787843in}}{\pgfqpoint{2.202706in}{2.776793in}}%
\pgfpathcurveto{\pgfqpoint{2.202706in}{2.765743in}}{\pgfqpoint{2.207096in}{2.755144in}}{\pgfqpoint{2.214910in}{2.747330in}}%
\pgfpathcurveto{\pgfqpoint{2.222723in}{2.739517in}}{\pgfqpoint{2.233322in}{2.735127in}}{\pgfqpoint{2.244373in}{2.735127in}}%
\pgfpathclose%
\pgfusepath{stroke,fill}%
\end{pgfscope}%
\begin{pgfscope}%
\pgfpathrectangle{\pgfqpoint{0.600000in}{0.600000in}}{\pgfqpoint{3.900000in}{3.900000in}}%
\pgfusepath{clip}%
\pgfsetbuttcap%
\pgfsetroundjoin%
\definecolor{currentfill}{rgb}{0.121569,0.466667,0.705882}%
\pgfsetfillcolor{currentfill}%
\pgfsetlinewidth{1.003750pt}%
\definecolor{currentstroke}{rgb}{0.121569,0.466667,0.705882}%
\pgfsetstrokecolor{currentstroke}%
\pgfsetdash{}{0pt}%
\pgfpathmoveto{\pgfqpoint{2.095071in}{1.950638in}}%
\pgfpathcurveto{\pgfqpoint{2.106121in}{1.950638in}}{\pgfqpoint{2.116720in}{1.955028in}}{\pgfqpoint{2.124534in}{1.962842in}}%
\pgfpathcurveto{\pgfqpoint{2.132348in}{1.970656in}}{\pgfqpoint{2.136738in}{1.981255in}}{\pgfqpoint{2.136738in}{1.992305in}}%
\pgfpathcurveto{\pgfqpoint{2.136738in}{2.003355in}}{\pgfqpoint{2.132348in}{2.013954in}}{\pgfqpoint{2.124534in}{2.021768in}}%
\pgfpathcurveto{\pgfqpoint{2.116720in}{2.029581in}}{\pgfqpoint{2.106121in}{2.033971in}}{\pgfqpoint{2.095071in}{2.033971in}}%
\pgfpathcurveto{\pgfqpoint{2.084021in}{2.033971in}}{\pgfqpoint{2.073422in}{2.029581in}}{\pgfqpoint{2.065608in}{2.021768in}}%
\pgfpathcurveto{\pgfqpoint{2.057795in}{2.013954in}}{\pgfqpoint{2.053405in}{2.003355in}}{\pgfqpoint{2.053405in}{1.992305in}}%
\pgfpathcurveto{\pgfqpoint{2.053405in}{1.981255in}}{\pgfqpoint{2.057795in}{1.970656in}}{\pgfqpoint{2.065608in}{1.962842in}}%
\pgfpathcurveto{\pgfqpoint{2.073422in}{1.955028in}}{\pgfqpoint{2.084021in}{1.950638in}}{\pgfqpoint{2.095071in}{1.950638in}}%
\pgfpathclose%
\pgfusepath{stroke,fill}%
\end{pgfscope}%
\begin{pgfscope}%
\pgfpathrectangle{\pgfqpoint{0.600000in}{0.600000in}}{\pgfqpoint{3.900000in}{3.900000in}}%
\pgfusepath{clip}%
\pgfsetbuttcap%
\pgfsetroundjoin%
\definecolor{currentfill}{rgb}{0.121569,0.466667,0.705882}%
\pgfsetfillcolor{currentfill}%
\pgfsetlinewidth{1.003750pt}%
\definecolor{currentstroke}{rgb}{0.121569,0.466667,0.705882}%
\pgfsetstrokecolor{currentstroke}%
\pgfsetdash{}{0pt}%
\pgfpathmoveto{\pgfqpoint{3.013427in}{2.168009in}}%
\pgfpathcurveto{\pgfqpoint{3.024477in}{2.168009in}}{\pgfqpoint{3.035076in}{2.172399in}}{\pgfqpoint{3.042890in}{2.180213in}}%
\pgfpathcurveto{\pgfqpoint{3.050703in}{2.188026in}}{\pgfqpoint{3.055094in}{2.198625in}}{\pgfqpoint{3.055094in}{2.209675in}}%
\pgfpathcurveto{\pgfqpoint{3.055094in}{2.220726in}}{\pgfqpoint{3.050703in}{2.231325in}}{\pgfqpoint{3.042890in}{2.239138in}}%
\pgfpathcurveto{\pgfqpoint{3.035076in}{2.246952in}}{\pgfqpoint{3.024477in}{2.251342in}}{\pgfqpoint{3.013427in}{2.251342in}}%
\pgfpathcurveto{\pgfqpoint{3.002377in}{2.251342in}}{\pgfqpoint{2.991778in}{2.246952in}}{\pgfqpoint{2.983964in}{2.239138in}}%
\pgfpathcurveto{\pgfqpoint{2.976151in}{2.231325in}}{\pgfqpoint{2.971760in}{2.220726in}}{\pgfqpoint{2.971760in}{2.209675in}}%
\pgfpathcurveto{\pgfqpoint{2.971760in}{2.198625in}}{\pgfqpoint{2.976151in}{2.188026in}}{\pgfqpoint{2.983964in}{2.180213in}}%
\pgfpathcurveto{\pgfqpoint{2.991778in}{2.172399in}}{\pgfqpoint{3.002377in}{2.168009in}}{\pgfqpoint{3.013427in}{2.168009in}}%
\pgfpathclose%
\pgfusepath{stroke,fill}%
\end{pgfscope}%
\begin{pgfscope}%
\pgfpathrectangle{\pgfqpoint{0.600000in}{0.600000in}}{\pgfqpoint{3.900000in}{3.900000in}}%
\pgfusepath{clip}%
\pgfsetbuttcap%
\pgfsetroundjoin%
\definecolor{currentfill}{rgb}{0.121569,0.466667,0.705882}%
\pgfsetfillcolor{currentfill}%
\pgfsetlinewidth{1.003750pt}%
\definecolor{currentstroke}{rgb}{0.121569,0.466667,0.705882}%
\pgfsetstrokecolor{currentstroke}%
\pgfsetdash{}{0pt}%
\pgfpathmoveto{\pgfqpoint{2.977670in}{2.309272in}}%
\pgfpathcurveto{\pgfqpoint{2.988720in}{2.309272in}}{\pgfqpoint{2.999319in}{2.313662in}}{\pgfqpoint{3.007133in}{2.321476in}}%
\pgfpathcurveto{\pgfqpoint{3.014946in}{2.329290in}}{\pgfqpoint{3.019337in}{2.339889in}}{\pgfqpoint{3.019337in}{2.350939in}}%
\pgfpathcurveto{\pgfqpoint{3.019337in}{2.361989in}}{\pgfqpoint{3.014946in}{2.372588in}}{\pgfqpoint{3.007133in}{2.380402in}}%
\pgfpathcurveto{\pgfqpoint{2.999319in}{2.388215in}}{\pgfqpoint{2.988720in}{2.392606in}}{\pgfqpoint{2.977670in}{2.392606in}}%
\pgfpathcurveto{\pgfqpoint{2.966620in}{2.392606in}}{\pgfqpoint{2.956021in}{2.388215in}}{\pgfqpoint{2.948207in}{2.380402in}}%
\pgfpathcurveto{\pgfqpoint{2.940394in}{2.372588in}}{\pgfqpoint{2.936003in}{2.361989in}}{\pgfqpoint{2.936003in}{2.350939in}}%
\pgfpathcurveto{\pgfqpoint{2.936003in}{2.339889in}}{\pgfqpoint{2.940394in}{2.329290in}}{\pgfqpoint{2.948207in}{2.321476in}}%
\pgfpathcurveto{\pgfqpoint{2.956021in}{2.313662in}}{\pgfqpoint{2.966620in}{2.309272in}}{\pgfqpoint{2.977670in}{2.309272in}}%
\pgfpathclose%
\pgfusepath{stroke,fill}%
\end{pgfscope}%
\begin{pgfscope}%
\pgfpathrectangle{\pgfqpoint{0.600000in}{0.600000in}}{\pgfqpoint{3.900000in}{3.900000in}}%
\pgfusepath{clip}%
\pgfsetbuttcap%
\pgfsetroundjoin%
\definecolor{currentfill}{rgb}{0.121569,0.466667,0.705882}%
\pgfsetfillcolor{currentfill}%
\pgfsetlinewidth{1.003750pt}%
\definecolor{currentstroke}{rgb}{0.121569,0.466667,0.705882}%
\pgfsetstrokecolor{currentstroke}%
\pgfsetdash{}{0pt}%
\pgfpathmoveto{\pgfqpoint{3.280568in}{3.887031in}}%
\pgfpathcurveto{\pgfqpoint{3.291619in}{3.887031in}}{\pgfqpoint{3.302218in}{3.891421in}}{\pgfqpoint{3.310031in}{3.899235in}}%
\pgfpathcurveto{\pgfqpoint{3.317845in}{3.907048in}}{\pgfqpoint{3.322235in}{3.917647in}}{\pgfqpoint{3.322235in}{3.928697in}}%
\pgfpathcurveto{\pgfqpoint{3.322235in}{3.939748in}}{\pgfqpoint{3.317845in}{3.950347in}}{\pgfqpoint{3.310031in}{3.958160in}}%
\pgfpathcurveto{\pgfqpoint{3.302218in}{3.965974in}}{\pgfqpoint{3.291619in}{3.970364in}}{\pgfqpoint{3.280568in}{3.970364in}}%
\pgfpathcurveto{\pgfqpoint{3.269518in}{3.970364in}}{\pgfqpoint{3.258919in}{3.965974in}}{\pgfqpoint{3.251106in}{3.958160in}}%
\pgfpathcurveto{\pgfqpoint{3.243292in}{3.950347in}}{\pgfqpoint{3.238902in}{3.939748in}}{\pgfqpoint{3.238902in}{3.928697in}}%
\pgfpathcurveto{\pgfqpoint{3.238902in}{3.917647in}}{\pgfqpoint{3.243292in}{3.907048in}}{\pgfqpoint{3.251106in}{3.899235in}}%
\pgfpathcurveto{\pgfqpoint{3.258919in}{3.891421in}}{\pgfqpoint{3.269518in}{3.887031in}}{\pgfqpoint{3.280568in}{3.887031in}}%
\pgfpathclose%
\pgfusepath{stroke,fill}%
\end{pgfscope}%
\begin{pgfscope}%
\pgfpathrectangle{\pgfqpoint{0.600000in}{0.600000in}}{\pgfqpoint{3.900000in}{3.900000in}}%
\pgfusepath{clip}%
\pgfsetbuttcap%
\pgfsetroundjoin%
\definecolor{currentfill}{rgb}{0.121569,0.466667,0.705882}%
\pgfsetfillcolor{currentfill}%
\pgfsetlinewidth{1.003750pt}%
\definecolor{currentstroke}{rgb}{0.121569,0.466667,0.705882}%
\pgfsetstrokecolor{currentstroke}%
\pgfsetdash{}{0pt}%
\pgfpathmoveto{\pgfqpoint{2.598486in}{2.028839in}}%
\pgfpathcurveto{\pgfqpoint{2.609537in}{2.028839in}}{\pgfqpoint{2.620136in}{2.033229in}}{\pgfqpoint{2.627949in}{2.041042in}}%
\pgfpathcurveto{\pgfqpoint{2.635763in}{2.048856in}}{\pgfqpoint{2.640153in}{2.059455in}}{\pgfqpoint{2.640153in}{2.070505in}}%
\pgfpathcurveto{\pgfqpoint{2.640153in}{2.081555in}}{\pgfqpoint{2.635763in}{2.092154in}}{\pgfqpoint{2.627949in}{2.099968in}}%
\pgfpathcurveto{\pgfqpoint{2.620136in}{2.107782in}}{\pgfqpoint{2.609537in}{2.112172in}}{\pgfqpoint{2.598486in}{2.112172in}}%
\pgfpathcurveto{\pgfqpoint{2.587436in}{2.112172in}}{\pgfqpoint{2.576837in}{2.107782in}}{\pgfqpoint{2.569024in}{2.099968in}}%
\pgfpathcurveto{\pgfqpoint{2.561210in}{2.092154in}}{\pgfqpoint{2.556820in}{2.081555in}}{\pgfqpoint{2.556820in}{2.070505in}}%
\pgfpathcurveto{\pgfqpoint{2.556820in}{2.059455in}}{\pgfqpoint{2.561210in}{2.048856in}}{\pgfqpoint{2.569024in}{2.041042in}}%
\pgfpathcurveto{\pgfqpoint{2.576837in}{2.033229in}}{\pgfqpoint{2.587436in}{2.028839in}}{\pgfqpoint{2.598486in}{2.028839in}}%
\pgfpathclose%
\pgfusepath{stroke,fill}%
\end{pgfscope}%
\begin{pgfscope}%
\pgfpathrectangle{\pgfqpoint{0.600000in}{0.600000in}}{\pgfqpoint{3.900000in}{3.900000in}}%
\pgfusepath{clip}%
\pgfsetbuttcap%
\pgfsetroundjoin%
\definecolor{currentfill}{rgb}{0.121569,0.466667,0.705882}%
\pgfsetfillcolor{currentfill}%
\pgfsetlinewidth{1.003750pt}%
\definecolor{currentstroke}{rgb}{0.121569,0.466667,0.705882}%
\pgfsetstrokecolor{currentstroke}%
\pgfsetdash{}{0pt}%
\pgfpathmoveto{\pgfqpoint{2.608262in}{2.058261in}}%
\pgfpathcurveto{\pgfqpoint{2.619312in}{2.058261in}}{\pgfqpoint{2.629911in}{2.062651in}}{\pgfqpoint{2.637724in}{2.070465in}}%
\pgfpathcurveto{\pgfqpoint{2.645538in}{2.078278in}}{\pgfqpoint{2.649928in}{2.088877in}}{\pgfqpoint{2.649928in}{2.099928in}}%
\pgfpathcurveto{\pgfqpoint{2.649928in}{2.110978in}}{\pgfqpoint{2.645538in}{2.121577in}}{\pgfqpoint{2.637724in}{2.129390in}}%
\pgfpathcurveto{\pgfqpoint{2.629911in}{2.137204in}}{\pgfqpoint{2.619312in}{2.141594in}}{\pgfqpoint{2.608262in}{2.141594in}}%
\pgfpathcurveto{\pgfqpoint{2.597212in}{2.141594in}}{\pgfqpoint{2.586612in}{2.137204in}}{\pgfqpoint{2.578799in}{2.129390in}}%
\pgfpathcurveto{\pgfqpoint{2.570985in}{2.121577in}}{\pgfqpoint{2.566595in}{2.110978in}}{\pgfqpoint{2.566595in}{2.099928in}}%
\pgfpathcurveto{\pgfqpoint{2.566595in}{2.088877in}}{\pgfqpoint{2.570985in}{2.078278in}}{\pgfqpoint{2.578799in}{2.070465in}}%
\pgfpathcurveto{\pgfqpoint{2.586612in}{2.062651in}}{\pgfqpoint{2.597212in}{2.058261in}}{\pgfqpoint{2.608262in}{2.058261in}}%
\pgfpathclose%
\pgfusepath{stroke,fill}%
\end{pgfscope}%
\begin{pgfscope}%
\pgfpathrectangle{\pgfqpoint{0.600000in}{0.600000in}}{\pgfqpoint{3.900000in}{3.900000in}}%
\pgfusepath{clip}%
\pgfsetbuttcap%
\pgfsetroundjoin%
\definecolor{currentfill}{rgb}{0.121569,0.466667,0.705882}%
\pgfsetfillcolor{currentfill}%
\pgfsetlinewidth{1.003750pt}%
\definecolor{currentstroke}{rgb}{0.121569,0.466667,0.705882}%
\pgfsetstrokecolor{currentstroke}%
\pgfsetdash{}{0pt}%
\pgfpathmoveto{\pgfqpoint{2.855026in}{2.274590in}}%
\pgfpathcurveto{\pgfqpoint{2.866076in}{2.274590in}}{\pgfqpoint{2.876675in}{2.278980in}}{\pgfqpoint{2.884488in}{2.286794in}}%
\pgfpathcurveto{\pgfqpoint{2.892302in}{2.294608in}}{\pgfqpoint{2.896692in}{2.305207in}}{\pgfqpoint{2.896692in}{2.316257in}}%
\pgfpathcurveto{\pgfqpoint{2.896692in}{2.327307in}}{\pgfqpoint{2.892302in}{2.337906in}}{\pgfqpoint{2.884488in}{2.345720in}}%
\pgfpathcurveto{\pgfqpoint{2.876675in}{2.353533in}}{\pgfqpoint{2.866076in}{2.357923in}}{\pgfqpoint{2.855026in}{2.357923in}}%
\pgfpathcurveto{\pgfqpoint{2.843976in}{2.357923in}}{\pgfqpoint{2.833376in}{2.353533in}}{\pgfqpoint{2.825563in}{2.345720in}}%
\pgfpathcurveto{\pgfqpoint{2.817749in}{2.337906in}}{\pgfqpoint{2.813359in}{2.327307in}}{\pgfqpoint{2.813359in}{2.316257in}}%
\pgfpathcurveto{\pgfqpoint{2.813359in}{2.305207in}}{\pgfqpoint{2.817749in}{2.294608in}}{\pgfqpoint{2.825563in}{2.286794in}}%
\pgfpathcurveto{\pgfqpoint{2.833376in}{2.278980in}}{\pgfqpoint{2.843976in}{2.274590in}}{\pgfqpoint{2.855026in}{2.274590in}}%
\pgfpathclose%
\pgfusepath{stroke,fill}%
\end{pgfscope}%
\begin{pgfscope}%
\pgfpathrectangle{\pgfqpoint{0.600000in}{0.600000in}}{\pgfqpoint{3.900000in}{3.900000in}}%
\pgfusepath{clip}%
\pgfsetbuttcap%
\pgfsetroundjoin%
\definecolor{currentfill}{rgb}{0.121569,0.466667,0.705882}%
\pgfsetfillcolor{currentfill}%
\pgfsetlinewidth{1.003750pt}%
\definecolor{currentstroke}{rgb}{0.121569,0.466667,0.705882}%
\pgfsetstrokecolor{currentstroke}%
\pgfsetdash{}{0pt}%
\pgfpathmoveto{\pgfqpoint{1.964291in}{1.830361in}}%
\pgfpathcurveto{\pgfqpoint{1.975341in}{1.830361in}}{\pgfqpoint{1.985940in}{1.834751in}}{\pgfqpoint{1.993754in}{1.842565in}}%
\pgfpathcurveto{\pgfqpoint{2.001568in}{1.850379in}}{\pgfqpoint{2.005958in}{1.860978in}}{\pgfqpoint{2.005958in}{1.872028in}}%
\pgfpathcurveto{\pgfqpoint{2.005958in}{1.883078in}}{\pgfqpoint{2.001568in}{1.893677in}}{\pgfqpoint{1.993754in}{1.901491in}}%
\pgfpathcurveto{\pgfqpoint{1.985940in}{1.909304in}}{\pgfqpoint{1.975341in}{1.913694in}}{\pgfqpoint{1.964291in}{1.913694in}}%
\pgfpathcurveto{\pgfqpoint{1.953241in}{1.913694in}}{\pgfqpoint{1.942642in}{1.909304in}}{\pgfqpoint{1.934829in}{1.901491in}}%
\pgfpathcurveto{\pgfqpoint{1.927015in}{1.893677in}}{\pgfqpoint{1.922625in}{1.883078in}}{\pgfqpoint{1.922625in}{1.872028in}}%
\pgfpathcurveto{\pgfqpoint{1.922625in}{1.860978in}}{\pgfqpoint{1.927015in}{1.850379in}}{\pgfqpoint{1.934829in}{1.842565in}}%
\pgfpathcurveto{\pgfqpoint{1.942642in}{1.834751in}}{\pgfqpoint{1.953241in}{1.830361in}}{\pgfqpoint{1.964291in}{1.830361in}}%
\pgfpathclose%
\pgfusepath{stroke,fill}%
\end{pgfscope}%
\begin{pgfscope}%
\pgfpathrectangle{\pgfqpoint{0.600000in}{0.600000in}}{\pgfqpoint{3.900000in}{3.900000in}}%
\pgfusepath{clip}%
\pgfsetbuttcap%
\pgfsetroundjoin%
\definecolor{currentfill}{rgb}{0.121569,0.466667,0.705882}%
\pgfsetfillcolor{currentfill}%
\pgfsetlinewidth{1.003750pt}%
\definecolor{currentstroke}{rgb}{0.121569,0.466667,0.705882}%
\pgfsetstrokecolor{currentstroke}%
\pgfsetdash{}{0pt}%
\pgfpathmoveto{\pgfqpoint{2.175398in}{2.146147in}}%
\pgfpathcurveto{\pgfqpoint{2.186448in}{2.146147in}}{\pgfqpoint{2.197047in}{2.150538in}}{\pgfqpoint{2.204860in}{2.158351in}}%
\pgfpathcurveto{\pgfqpoint{2.212674in}{2.166165in}}{\pgfqpoint{2.217064in}{2.176764in}}{\pgfqpoint{2.217064in}{2.187814in}}%
\pgfpathcurveto{\pgfqpoint{2.217064in}{2.198864in}}{\pgfqpoint{2.212674in}{2.209463in}}{\pgfqpoint{2.204860in}{2.217277in}}%
\pgfpathcurveto{\pgfqpoint{2.197047in}{2.225090in}}{\pgfqpoint{2.186448in}{2.229481in}}{\pgfqpoint{2.175398in}{2.229481in}}%
\pgfpathcurveto{\pgfqpoint{2.164348in}{2.229481in}}{\pgfqpoint{2.153748in}{2.225090in}}{\pgfqpoint{2.145935in}{2.217277in}}%
\pgfpathcurveto{\pgfqpoint{2.138121in}{2.209463in}}{\pgfqpoint{2.133731in}{2.198864in}}{\pgfqpoint{2.133731in}{2.187814in}}%
\pgfpathcurveto{\pgfqpoint{2.133731in}{2.176764in}}{\pgfqpoint{2.138121in}{2.166165in}}{\pgfqpoint{2.145935in}{2.158351in}}%
\pgfpathcurveto{\pgfqpoint{2.153748in}{2.150538in}}{\pgfqpoint{2.164348in}{2.146147in}}{\pgfqpoint{2.175398in}{2.146147in}}%
\pgfpathclose%
\pgfusepath{stroke,fill}%
\end{pgfscope}%
\begin{pgfscope}%
\pgfpathrectangle{\pgfqpoint{0.600000in}{0.600000in}}{\pgfqpoint{3.900000in}{3.900000in}}%
\pgfusepath{clip}%
\pgfsetbuttcap%
\pgfsetroundjoin%
\definecolor{currentfill}{rgb}{0.121569,0.466667,0.705882}%
\pgfsetfillcolor{currentfill}%
\pgfsetlinewidth{1.003750pt}%
\definecolor{currentstroke}{rgb}{0.121569,0.466667,0.705882}%
\pgfsetstrokecolor{currentstroke}%
\pgfsetdash{}{0pt}%
\pgfpathmoveto{\pgfqpoint{3.207244in}{2.822257in}}%
\pgfpathcurveto{\pgfqpoint{3.218294in}{2.822257in}}{\pgfqpoint{3.228893in}{2.826647in}}{\pgfqpoint{3.236707in}{2.834460in}}%
\pgfpathcurveto{\pgfqpoint{3.244521in}{2.842274in}}{\pgfqpoint{3.248911in}{2.852873in}}{\pgfqpoint{3.248911in}{2.863923in}}%
\pgfpathcurveto{\pgfqpoint{3.248911in}{2.874973in}}{\pgfqpoint{3.244521in}{2.885572in}}{\pgfqpoint{3.236707in}{2.893386in}}%
\pgfpathcurveto{\pgfqpoint{3.228893in}{2.901200in}}{\pgfqpoint{3.218294in}{2.905590in}}{\pgfqpoint{3.207244in}{2.905590in}}%
\pgfpathcurveto{\pgfqpoint{3.196194in}{2.905590in}}{\pgfqpoint{3.185595in}{2.901200in}}{\pgfqpoint{3.177781in}{2.893386in}}%
\pgfpathcurveto{\pgfqpoint{3.169968in}{2.885572in}}{\pgfqpoint{3.165578in}{2.874973in}}{\pgfqpoint{3.165578in}{2.863923in}}%
\pgfpathcurveto{\pgfqpoint{3.165578in}{2.852873in}}{\pgfqpoint{3.169968in}{2.842274in}}{\pgfqpoint{3.177781in}{2.834460in}}%
\pgfpathcurveto{\pgfqpoint{3.185595in}{2.826647in}}{\pgfqpoint{3.196194in}{2.822257in}}{\pgfqpoint{3.207244in}{2.822257in}}%
\pgfpathclose%
\pgfusepath{stroke,fill}%
\end{pgfscope}%
\begin{pgfscope}%
\pgfpathrectangle{\pgfqpoint{0.600000in}{0.600000in}}{\pgfqpoint{3.900000in}{3.900000in}}%
\pgfusepath{clip}%
\pgfsetbuttcap%
\pgfsetroundjoin%
\definecolor{currentfill}{rgb}{0.121569,0.466667,0.705882}%
\pgfsetfillcolor{currentfill}%
\pgfsetlinewidth{1.003750pt}%
\definecolor{currentstroke}{rgb}{0.121569,0.466667,0.705882}%
\pgfsetstrokecolor{currentstroke}%
\pgfsetdash{}{0pt}%
\pgfpathmoveto{\pgfqpoint{2.002767in}{3.160582in}}%
\pgfpathcurveto{\pgfqpoint{2.013817in}{3.160582in}}{\pgfqpoint{2.024416in}{3.164972in}}{\pgfqpoint{2.032230in}{3.172786in}}%
\pgfpathcurveto{\pgfqpoint{2.040044in}{3.180600in}}{\pgfqpoint{2.044434in}{3.191199in}}{\pgfqpoint{2.044434in}{3.202249in}}%
\pgfpathcurveto{\pgfqpoint{2.044434in}{3.213299in}}{\pgfqpoint{2.040044in}{3.223898in}}{\pgfqpoint{2.032230in}{3.231712in}}%
\pgfpathcurveto{\pgfqpoint{2.024416in}{3.239525in}}{\pgfqpoint{2.013817in}{3.243915in}}{\pgfqpoint{2.002767in}{3.243915in}}%
\pgfpathcurveto{\pgfqpoint{1.991717in}{3.243915in}}{\pgfqpoint{1.981118in}{3.239525in}}{\pgfqpoint{1.973304in}{3.231712in}}%
\pgfpathcurveto{\pgfqpoint{1.965491in}{3.223898in}}{\pgfqpoint{1.961101in}{3.213299in}}{\pgfqpoint{1.961101in}{3.202249in}}%
\pgfpathcurveto{\pgfqpoint{1.961101in}{3.191199in}}{\pgfqpoint{1.965491in}{3.180600in}}{\pgfqpoint{1.973304in}{3.172786in}}%
\pgfpathcurveto{\pgfqpoint{1.981118in}{3.164972in}}{\pgfqpoint{1.991717in}{3.160582in}}{\pgfqpoint{2.002767in}{3.160582in}}%
\pgfpathclose%
\pgfusepath{stroke,fill}%
\end{pgfscope}%
\begin{pgfscope}%
\pgfpathrectangle{\pgfqpoint{0.600000in}{0.600000in}}{\pgfqpoint{3.900000in}{3.900000in}}%
\pgfusepath{clip}%
\pgfsetbuttcap%
\pgfsetroundjoin%
\definecolor{currentfill}{rgb}{0.121569,0.466667,0.705882}%
\pgfsetfillcolor{currentfill}%
\pgfsetlinewidth{1.003750pt}%
\definecolor{currentstroke}{rgb}{0.121569,0.466667,0.705882}%
\pgfsetstrokecolor{currentstroke}%
\pgfsetdash{}{0pt}%
\pgfpathmoveto{\pgfqpoint{1.931632in}{2.493932in}}%
\pgfpathcurveto{\pgfqpoint{1.942682in}{2.493932in}}{\pgfqpoint{1.953281in}{2.498323in}}{\pgfqpoint{1.961095in}{2.506136in}}%
\pgfpathcurveto{\pgfqpoint{1.968909in}{2.513950in}}{\pgfqpoint{1.973299in}{2.524549in}}{\pgfqpoint{1.973299in}{2.535599in}}%
\pgfpathcurveto{\pgfqpoint{1.973299in}{2.546649in}}{\pgfqpoint{1.968909in}{2.557248in}}{\pgfqpoint{1.961095in}{2.565062in}}%
\pgfpathcurveto{\pgfqpoint{1.953281in}{2.572875in}}{\pgfqpoint{1.942682in}{2.577266in}}{\pgfqpoint{1.931632in}{2.577266in}}%
\pgfpathcurveto{\pgfqpoint{1.920582in}{2.577266in}}{\pgfqpoint{1.909983in}{2.572875in}}{\pgfqpoint{1.902170in}{2.565062in}}%
\pgfpathcurveto{\pgfqpoint{1.894356in}{2.557248in}}{\pgfqpoint{1.889966in}{2.546649in}}{\pgfqpoint{1.889966in}{2.535599in}}%
\pgfpathcurveto{\pgfqpoint{1.889966in}{2.524549in}}{\pgfqpoint{1.894356in}{2.513950in}}{\pgfqpoint{1.902170in}{2.506136in}}%
\pgfpathcurveto{\pgfqpoint{1.909983in}{2.498323in}}{\pgfqpoint{1.920582in}{2.493932in}}{\pgfqpoint{1.931632in}{2.493932in}}%
\pgfpathclose%
\pgfusepath{stroke,fill}%
\end{pgfscope}%
\begin{pgfscope}%
\pgfpathrectangle{\pgfqpoint{0.600000in}{0.600000in}}{\pgfqpoint{3.900000in}{3.900000in}}%
\pgfusepath{clip}%
\pgfsetbuttcap%
\pgfsetroundjoin%
\definecolor{currentfill}{rgb}{0.121569,0.466667,0.705882}%
\pgfsetfillcolor{currentfill}%
\pgfsetlinewidth{1.003750pt}%
\definecolor{currentstroke}{rgb}{0.121569,0.466667,0.705882}%
\pgfsetstrokecolor{currentstroke}%
\pgfsetdash{}{0pt}%
\pgfpathmoveto{\pgfqpoint{2.607227in}{3.313832in}}%
\pgfpathcurveto{\pgfqpoint{2.618278in}{3.313832in}}{\pgfqpoint{2.628877in}{3.318223in}}{\pgfqpoint{2.636690in}{3.326036in}}%
\pgfpathcurveto{\pgfqpoint{2.644504in}{3.333850in}}{\pgfqpoint{2.648894in}{3.344449in}}{\pgfqpoint{2.648894in}{3.355499in}}%
\pgfpathcurveto{\pgfqpoint{2.648894in}{3.366549in}}{\pgfqpoint{2.644504in}{3.377148in}}{\pgfqpoint{2.636690in}{3.384962in}}%
\pgfpathcurveto{\pgfqpoint{2.628877in}{3.392775in}}{\pgfqpoint{2.618278in}{3.397166in}}{\pgfqpoint{2.607227in}{3.397166in}}%
\pgfpathcurveto{\pgfqpoint{2.596177in}{3.397166in}}{\pgfqpoint{2.585578in}{3.392775in}}{\pgfqpoint{2.577765in}{3.384962in}}%
\pgfpathcurveto{\pgfqpoint{2.569951in}{3.377148in}}{\pgfqpoint{2.565561in}{3.366549in}}{\pgfqpoint{2.565561in}{3.355499in}}%
\pgfpathcurveto{\pgfqpoint{2.565561in}{3.344449in}}{\pgfqpoint{2.569951in}{3.333850in}}{\pgfqpoint{2.577765in}{3.326036in}}%
\pgfpathcurveto{\pgfqpoint{2.585578in}{3.318223in}}{\pgfqpoint{2.596177in}{3.313832in}}{\pgfqpoint{2.607227in}{3.313832in}}%
\pgfpathclose%
\pgfusepath{stroke,fill}%
\end{pgfscope}%
\begin{pgfscope}%
\pgfpathrectangle{\pgfqpoint{0.600000in}{0.600000in}}{\pgfqpoint{3.900000in}{3.900000in}}%
\pgfusepath{clip}%
\pgfsetbuttcap%
\pgfsetroundjoin%
\definecolor{currentfill}{rgb}{0.121569,0.466667,0.705882}%
\pgfsetfillcolor{currentfill}%
\pgfsetlinewidth{1.003750pt}%
\definecolor{currentstroke}{rgb}{0.121569,0.466667,0.705882}%
\pgfsetstrokecolor{currentstroke}%
\pgfsetdash{}{0pt}%
\pgfpathmoveto{\pgfqpoint{2.922737in}{2.229186in}}%
\pgfpathcurveto{\pgfqpoint{2.933787in}{2.229186in}}{\pgfqpoint{2.944386in}{2.233576in}}{\pgfqpoint{2.952200in}{2.241390in}}%
\pgfpathcurveto{\pgfqpoint{2.960014in}{2.249204in}}{\pgfqpoint{2.964404in}{2.259803in}}{\pgfqpoint{2.964404in}{2.270853in}}%
\pgfpathcurveto{\pgfqpoint{2.964404in}{2.281903in}}{\pgfqpoint{2.960014in}{2.292502in}}{\pgfqpoint{2.952200in}{2.300316in}}%
\pgfpathcurveto{\pgfqpoint{2.944386in}{2.308129in}}{\pgfqpoint{2.933787in}{2.312519in}}{\pgfqpoint{2.922737in}{2.312519in}}%
\pgfpathcurveto{\pgfqpoint{2.911687in}{2.312519in}}{\pgfqpoint{2.901088in}{2.308129in}}{\pgfqpoint{2.893274in}{2.300316in}}%
\pgfpathcurveto{\pgfqpoint{2.885461in}{2.292502in}}{\pgfqpoint{2.881071in}{2.281903in}}{\pgfqpoint{2.881071in}{2.270853in}}%
\pgfpathcurveto{\pgfqpoint{2.881071in}{2.259803in}}{\pgfqpoint{2.885461in}{2.249204in}}{\pgfqpoint{2.893274in}{2.241390in}}%
\pgfpathcurveto{\pgfqpoint{2.901088in}{2.233576in}}{\pgfqpoint{2.911687in}{2.229186in}}{\pgfqpoint{2.922737in}{2.229186in}}%
\pgfpathclose%
\pgfusepath{stroke,fill}%
\end{pgfscope}%
\begin{pgfscope}%
\pgfpathrectangle{\pgfqpoint{0.600000in}{0.600000in}}{\pgfqpoint{3.900000in}{3.900000in}}%
\pgfusepath{clip}%
\pgfsetbuttcap%
\pgfsetroundjoin%
\definecolor{currentfill}{rgb}{0.121569,0.466667,0.705882}%
\pgfsetfillcolor{currentfill}%
\pgfsetlinewidth{1.003750pt}%
\definecolor{currentstroke}{rgb}{0.121569,0.466667,0.705882}%
\pgfsetstrokecolor{currentstroke}%
\pgfsetdash{}{0pt}%
\pgfpathmoveto{\pgfqpoint{2.991260in}{2.631082in}}%
\pgfpathcurveto{\pgfqpoint{3.002310in}{2.631082in}}{\pgfqpoint{3.012909in}{2.635473in}}{\pgfqpoint{3.020723in}{2.643286in}}%
\pgfpathcurveto{\pgfqpoint{3.028536in}{2.651100in}}{\pgfqpoint{3.032927in}{2.661699in}}{\pgfqpoint{3.032927in}{2.672749in}}%
\pgfpathcurveto{\pgfqpoint{3.032927in}{2.683799in}}{\pgfqpoint{3.028536in}{2.694398in}}{\pgfqpoint{3.020723in}{2.702212in}}%
\pgfpathcurveto{\pgfqpoint{3.012909in}{2.710026in}}{\pgfqpoint{3.002310in}{2.714416in}}{\pgfqpoint{2.991260in}{2.714416in}}%
\pgfpathcurveto{\pgfqpoint{2.980210in}{2.714416in}}{\pgfqpoint{2.969611in}{2.710026in}}{\pgfqpoint{2.961797in}{2.702212in}}%
\pgfpathcurveto{\pgfqpoint{2.953984in}{2.694398in}}{\pgfqpoint{2.949593in}{2.683799in}}{\pgfqpoint{2.949593in}{2.672749in}}%
\pgfpathcurveto{\pgfqpoint{2.949593in}{2.661699in}}{\pgfqpoint{2.953984in}{2.651100in}}{\pgfqpoint{2.961797in}{2.643286in}}%
\pgfpathcurveto{\pgfqpoint{2.969611in}{2.635473in}}{\pgfqpoint{2.980210in}{2.631082in}}{\pgfqpoint{2.991260in}{2.631082in}}%
\pgfpathclose%
\pgfusepath{stroke,fill}%
\end{pgfscope}%
\begin{pgfscope}%
\pgfpathrectangle{\pgfqpoint{0.600000in}{0.600000in}}{\pgfqpoint{3.900000in}{3.900000in}}%
\pgfusepath{clip}%
\pgfsetbuttcap%
\pgfsetroundjoin%
\definecolor{currentfill}{rgb}{0.121569,0.466667,0.705882}%
\pgfsetfillcolor{currentfill}%
\pgfsetlinewidth{1.003750pt}%
\definecolor{currentstroke}{rgb}{0.121569,0.466667,0.705882}%
\pgfsetstrokecolor{currentstroke}%
\pgfsetdash{}{0pt}%
\pgfpathmoveto{\pgfqpoint{2.275261in}{2.374752in}}%
\pgfpathcurveto{\pgfqpoint{2.286311in}{2.374752in}}{\pgfqpoint{2.296910in}{2.379143in}}{\pgfqpoint{2.304724in}{2.386956in}}%
\pgfpathcurveto{\pgfqpoint{2.312537in}{2.394770in}}{\pgfqpoint{2.316928in}{2.405369in}}{\pgfqpoint{2.316928in}{2.416419in}}%
\pgfpathcurveto{\pgfqpoint{2.316928in}{2.427469in}}{\pgfqpoint{2.312537in}{2.438068in}}{\pgfqpoint{2.304724in}{2.445882in}}%
\pgfpathcurveto{\pgfqpoint{2.296910in}{2.453695in}}{\pgfqpoint{2.286311in}{2.458086in}}{\pgfqpoint{2.275261in}{2.458086in}}%
\pgfpathcurveto{\pgfqpoint{2.264211in}{2.458086in}}{\pgfqpoint{2.253612in}{2.453695in}}{\pgfqpoint{2.245798in}{2.445882in}}%
\pgfpathcurveto{\pgfqpoint{2.237985in}{2.438068in}}{\pgfqpoint{2.233594in}{2.427469in}}{\pgfqpoint{2.233594in}{2.416419in}}%
\pgfpathcurveto{\pgfqpoint{2.233594in}{2.405369in}}{\pgfqpoint{2.237985in}{2.394770in}}{\pgfqpoint{2.245798in}{2.386956in}}%
\pgfpathcurveto{\pgfqpoint{2.253612in}{2.379143in}}{\pgfqpoint{2.264211in}{2.374752in}}{\pgfqpoint{2.275261in}{2.374752in}}%
\pgfpathclose%
\pgfusepath{stroke,fill}%
\end{pgfscope}%
\begin{pgfscope}%
\pgfpathrectangle{\pgfqpoint{0.600000in}{0.600000in}}{\pgfqpoint{3.900000in}{3.900000in}}%
\pgfusepath{clip}%
\pgfsetbuttcap%
\pgfsetroundjoin%
\definecolor{currentfill}{rgb}{0.121569,0.466667,0.705882}%
\pgfsetfillcolor{currentfill}%
\pgfsetlinewidth{1.003750pt}%
\definecolor{currentstroke}{rgb}{0.121569,0.466667,0.705882}%
\pgfsetstrokecolor{currentstroke}%
\pgfsetdash{}{0pt}%
\pgfpathmoveto{\pgfqpoint{3.481462in}{2.664515in}}%
\pgfpathcurveto{\pgfqpoint{3.492513in}{2.664515in}}{\pgfqpoint{3.503112in}{2.668905in}}{\pgfqpoint{3.510925in}{2.676719in}}%
\pgfpathcurveto{\pgfqpoint{3.518739in}{2.684532in}}{\pgfqpoint{3.523129in}{2.695131in}}{\pgfqpoint{3.523129in}{2.706181in}}%
\pgfpathcurveto{\pgfqpoint{3.523129in}{2.717232in}}{\pgfqpoint{3.518739in}{2.727831in}}{\pgfqpoint{3.510925in}{2.735644in}}%
\pgfpathcurveto{\pgfqpoint{3.503112in}{2.743458in}}{\pgfqpoint{3.492513in}{2.747848in}}{\pgfqpoint{3.481462in}{2.747848in}}%
\pgfpathcurveto{\pgfqpoint{3.470412in}{2.747848in}}{\pgfqpoint{3.459813in}{2.743458in}}{\pgfqpoint{3.452000in}{2.735644in}}%
\pgfpathcurveto{\pgfqpoint{3.444186in}{2.727831in}}{\pgfqpoint{3.439796in}{2.717232in}}{\pgfqpoint{3.439796in}{2.706181in}}%
\pgfpathcurveto{\pgfqpoint{3.439796in}{2.695131in}}{\pgfqpoint{3.444186in}{2.684532in}}{\pgfqpoint{3.452000in}{2.676719in}}%
\pgfpathcurveto{\pgfqpoint{3.459813in}{2.668905in}}{\pgfqpoint{3.470412in}{2.664515in}}{\pgfqpoint{3.481462in}{2.664515in}}%
\pgfpathclose%
\pgfusepath{stroke,fill}%
\end{pgfscope}%
\begin{pgfscope}%
\pgfpathrectangle{\pgfqpoint{0.600000in}{0.600000in}}{\pgfqpoint{3.900000in}{3.900000in}}%
\pgfusepath{clip}%
\pgfsetbuttcap%
\pgfsetroundjoin%
\definecolor{currentfill}{rgb}{0.121569,0.466667,0.705882}%
\pgfsetfillcolor{currentfill}%
\pgfsetlinewidth{1.003750pt}%
\definecolor{currentstroke}{rgb}{0.121569,0.466667,0.705882}%
\pgfsetstrokecolor{currentstroke}%
\pgfsetdash{}{0pt}%
\pgfpathmoveto{\pgfqpoint{1.972776in}{2.266403in}}%
\pgfpathcurveto{\pgfqpoint{1.983826in}{2.266403in}}{\pgfqpoint{1.994425in}{2.270793in}}{\pgfqpoint{2.002239in}{2.278607in}}%
\pgfpathcurveto{\pgfqpoint{2.010052in}{2.286420in}}{\pgfqpoint{2.014443in}{2.297019in}}{\pgfqpoint{2.014443in}{2.308070in}}%
\pgfpathcurveto{\pgfqpoint{2.014443in}{2.319120in}}{\pgfqpoint{2.010052in}{2.329719in}}{\pgfqpoint{2.002239in}{2.337532in}}%
\pgfpathcurveto{\pgfqpoint{1.994425in}{2.345346in}}{\pgfqpoint{1.983826in}{2.349736in}}{\pgfqpoint{1.972776in}{2.349736in}}%
\pgfpathcurveto{\pgfqpoint{1.961726in}{2.349736in}}{\pgfqpoint{1.951127in}{2.345346in}}{\pgfqpoint{1.943313in}{2.337532in}}%
\pgfpathcurveto{\pgfqpoint{1.935500in}{2.329719in}}{\pgfqpoint{1.931109in}{2.319120in}}{\pgfqpoint{1.931109in}{2.308070in}}%
\pgfpathcurveto{\pgfqpoint{1.931109in}{2.297019in}}{\pgfqpoint{1.935500in}{2.286420in}}{\pgfqpoint{1.943313in}{2.278607in}}%
\pgfpathcurveto{\pgfqpoint{1.951127in}{2.270793in}}{\pgfqpoint{1.961726in}{2.266403in}}{\pgfqpoint{1.972776in}{2.266403in}}%
\pgfpathclose%
\pgfusepath{stroke,fill}%
\end{pgfscope}%
\begin{pgfscope}%
\pgfpathrectangle{\pgfqpoint{0.600000in}{0.600000in}}{\pgfqpoint{3.900000in}{3.900000in}}%
\pgfusepath{clip}%
\pgfsetbuttcap%
\pgfsetroundjoin%
\definecolor{currentfill}{rgb}{0.121569,0.466667,0.705882}%
\pgfsetfillcolor{currentfill}%
\pgfsetlinewidth{1.003750pt}%
\definecolor{currentstroke}{rgb}{0.121569,0.466667,0.705882}%
\pgfsetstrokecolor{currentstroke}%
\pgfsetdash{}{0pt}%
\pgfpathmoveto{\pgfqpoint{1.804758in}{1.888340in}}%
\pgfpathcurveto{\pgfqpoint{1.815808in}{1.888340in}}{\pgfqpoint{1.826407in}{1.892730in}}{\pgfqpoint{1.834220in}{1.900544in}}%
\pgfpathcurveto{\pgfqpoint{1.842034in}{1.908358in}}{\pgfqpoint{1.846424in}{1.918957in}}{\pgfqpoint{1.846424in}{1.930007in}}%
\pgfpathcurveto{\pgfqpoint{1.846424in}{1.941057in}}{\pgfqpoint{1.842034in}{1.951656in}}{\pgfqpoint{1.834220in}{1.959470in}}%
\pgfpathcurveto{\pgfqpoint{1.826407in}{1.967283in}}{\pgfqpoint{1.815808in}{1.971674in}}{\pgfqpoint{1.804758in}{1.971674in}}%
\pgfpathcurveto{\pgfqpoint{1.793707in}{1.971674in}}{\pgfqpoint{1.783108in}{1.967283in}}{\pgfqpoint{1.775295in}{1.959470in}}%
\pgfpathcurveto{\pgfqpoint{1.767481in}{1.951656in}}{\pgfqpoint{1.763091in}{1.941057in}}{\pgfqpoint{1.763091in}{1.930007in}}%
\pgfpathcurveto{\pgfqpoint{1.763091in}{1.918957in}}{\pgfqpoint{1.767481in}{1.908358in}}{\pgfqpoint{1.775295in}{1.900544in}}%
\pgfpathcurveto{\pgfqpoint{1.783108in}{1.892730in}}{\pgfqpoint{1.793707in}{1.888340in}}{\pgfqpoint{1.804758in}{1.888340in}}%
\pgfpathclose%
\pgfusepath{stroke,fill}%
\end{pgfscope}%
\begin{pgfscope}%
\pgfpathrectangle{\pgfqpoint{0.600000in}{0.600000in}}{\pgfqpoint{3.900000in}{3.900000in}}%
\pgfusepath{clip}%
\pgfsetbuttcap%
\pgfsetroundjoin%
\definecolor{currentfill}{rgb}{0.121569,0.466667,0.705882}%
\pgfsetfillcolor{currentfill}%
\pgfsetlinewidth{1.003750pt}%
\definecolor{currentstroke}{rgb}{0.121569,0.466667,0.705882}%
\pgfsetstrokecolor{currentstroke}%
\pgfsetdash{}{0pt}%
\pgfpathmoveto{\pgfqpoint{2.382413in}{2.961350in}}%
\pgfpathcurveto{\pgfqpoint{2.393463in}{2.961350in}}{\pgfqpoint{2.404062in}{2.965740in}}{\pgfqpoint{2.411876in}{2.973554in}}%
\pgfpathcurveto{\pgfqpoint{2.419690in}{2.981368in}}{\pgfqpoint{2.424080in}{2.991967in}}{\pgfqpoint{2.424080in}{3.003017in}}%
\pgfpathcurveto{\pgfqpoint{2.424080in}{3.014067in}}{\pgfqpoint{2.419690in}{3.024666in}}{\pgfqpoint{2.411876in}{3.032480in}}%
\pgfpathcurveto{\pgfqpoint{2.404062in}{3.040293in}}{\pgfqpoint{2.393463in}{3.044684in}}{\pgfqpoint{2.382413in}{3.044684in}}%
\pgfpathcurveto{\pgfqpoint{2.371363in}{3.044684in}}{\pgfqpoint{2.360764in}{3.040293in}}{\pgfqpoint{2.352950in}{3.032480in}}%
\pgfpathcurveto{\pgfqpoint{2.345137in}{3.024666in}}{\pgfqpoint{2.340747in}{3.014067in}}{\pgfqpoint{2.340747in}{3.003017in}}%
\pgfpathcurveto{\pgfqpoint{2.340747in}{2.991967in}}{\pgfqpoint{2.345137in}{2.981368in}}{\pgfqpoint{2.352950in}{2.973554in}}%
\pgfpathcurveto{\pgfqpoint{2.360764in}{2.965740in}}{\pgfqpoint{2.371363in}{2.961350in}}{\pgfqpoint{2.382413in}{2.961350in}}%
\pgfpathclose%
\pgfusepath{stroke,fill}%
\end{pgfscope}%
\begin{pgfscope}%
\pgfpathrectangle{\pgfqpoint{0.600000in}{0.600000in}}{\pgfqpoint{3.900000in}{3.900000in}}%
\pgfusepath{clip}%
\pgfsetbuttcap%
\pgfsetroundjoin%
\definecolor{currentfill}{rgb}{0.121569,0.466667,0.705882}%
\pgfsetfillcolor{currentfill}%
\pgfsetlinewidth{1.003750pt}%
\definecolor{currentstroke}{rgb}{0.121569,0.466667,0.705882}%
\pgfsetstrokecolor{currentstroke}%
\pgfsetdash{}{0pt}%
\pgfpathmoveto{\pgfqpoint{2.565264in}{2.356565in}}%
\pgfpathcurveto{\pgfqpoint{2.576314in}{2.356565in}}{\pgfqpoint{2.586913in}{2.360955in}}{\pgfqpoint{2.594727in}{2.368769in}}%
\pgfpathcurveto{\pgfqpoint{2.602540in}{2.376582in}}{\pgfqpoint{2.606930in}{2.387181in}}{\pgfqpoint{2.606930in}{2.398231in}}%
\pgfpathcurveto{\pgfqpoint{2.606930in}{2.409282in}}{\pgfqpoint{2.602540in}{2.419881in}}{\pgfqpoint{2.594727in}{2.427694in}}%
\pgfpathcurveto{\pgfqpoint{2.586913in}{2.435508in}}{\pgfqpoint{2.576314in}{2.439898in}}{\pgfqpoint{2.565264in}{2.439898in}}%
\pgfpathcurveto{\pgfqpoint{2.554214in}{2.439898in}}{\pgfqpoint{2.543615in}{2.435508in}}{\pgfqpoint{2.535801in}{2.427694in}}%
\pgfpathcurveto{\pgfqpoint{2.527987in}{2.419881in}}{\pgfqpoint{2.523597in}{2.409282in}}{\pgfqpoint{2.523597in}{2.398231in}}%
\pgfpathcurveto{\pgfqpoint{2.523597in}{2.387181in}}{\pgfqpoint{2.527987in}{2.376582in}}{\pgfqpoint{2.535801in}{2.368769in}}%
\pgfpathcurveto{\pgfqpoint{2.543615in}{2.360955in}}{\pgfqpoint{2.554214in}{2.356565in}}{\pgfqpoint{2.565264in}{2.356565in}}%
\pgfpathclose%
\pgfusepath{stroke,fill}%
\end{pgfscope}%
\begin{pgfscope}%
\pgfpathrectangle{\pgfqpoint{0.600000in}{0.600000in}}{\pgfqpoint{3.900000in}{3.900000in}}%
\pgfusepath{clip}%
\pgfsetbuttcap%
\pgfsetroundjoin%
\definecolor{currentfill}{rgb}{0.121569,0.466667,0.705882}%
\pgfsetfillcolor{currentfill}%
\pgfsetlinewidth{1.003750pt}%
\definecolor{currentstroke}{rgb}{0.121569,0.466667,0.705882}%
\pgfsetstrokecolor{currentstroke}%
\pgfsetdash{}{0pt}%
\pgfpathmoveto{\pgfqpoint{2.993427in}{2.712586in}}%
\pgfpathcurveto{\pgfqpoint{3.004477in}{2.712586in}}{\pgfqpoint{3.015076in}{2.716976in}}{\pgfqpoint{3.022890in}{2.724790in}}%
\pgfpathcurveto{\pgfqpoint{3.030703in}{2.732604in}}{\pgfqpoint{3.035093in}{2.743203in}}{\pgfqpoint{3.035093in}{2.754253in}}%
\pgfpathcurveto{\pgfqpoint{3.035093in}{2.765303in}}{\pgfqpoint{3.030703in}{2.775902in}}{\pgfqpoint{3.022890in}{2.783716in}}%
\pgfpathcurveto{\pgfqpoint{3.015076in}{2.791529in}}{\pgfqpoint{3.004477in}{2.795919in}}{\pgfqpoint{2.993427in}{2.795919in}}%
\pgfpathcurveto{\pgfqpoint{2.982377in}{2.795919in}}{\pgfqpoint{2.971778in}{2.791529in}}{\pgfqpoint{2.963964in}{2.783716in}}%
\pgfpathcurveto{\pgfqpoint{2.956150in}{2.775902in}}{\pgfqpoint{2.951760in}{2.765303in}}{\pgfqpoint{2.951760in}{2.754253in}}%
\pgfpathcurveto{\pgfqpoint{2.951760in}{2.743203in}}{\pgfqpoint{2.956150in}{2.732604in}}{\pgfqpoint{2.963964in}{2.724790in}}%
\pgfpathcurveto{\pgfqpoint{2.971778in}{2.716976in}}{\pgfqpoint{2.982377in}{2.712586in}}{\pgfqpoint{2.993427in}{2.712586in}}%
\pgfpathclose%
\pgfusepath{stroke,fill}%
\end{pgfscope}%
\begin{pgfscope}%
\pgfpathrectangle{\pgfqpoint{0.600000in}{0.600000in}}{\pgfqpoint{3.900000in}{3.900000in}}%
\pgfusepath{clip}%
\pgfsetbuttcap%
\pgfsetroundjoin%
\definecolor{currentfill}{rgb}{0.121569,0.466667,0.705882}%
\pgfsetfillcolor{currentfill}%
\pgfsetlinewidth{1.003750pt}%
\definecolor{currentstroke}{rgb}{0.121569,0.466667,0.705882}%
\pgfsetstrokecolor{currentstroke}%
\pgfsetdash{}{0pt}%
\pgfpathmoveto{\pgfqpoint{2.596328in}{1.474060in}}%
\pgfpathcurveto{\pgfqpoint{2.607378in}{1.474060in}}{\pgfqpoint{2.617977in}{1.478450in}}{\pgfqpoint{2.625791in}{1.486264in}}%
\pgfpathcurveto{\pgfqpoint{2.633604in}{1.494078in}}{\pgfqpoint{2.637995in}{1.504677in}}{\pgfqpoint{2.637995in}{1.515727in}}%
\pgfpathcurveto{\pgfqpoint{2.637995in}{1.526777in}}{\pgfqpoint{2.633604in}{1.537376in}}{\pgfqpoint{2.625791in}{1.545190in}}%
\pgfpathcurveto{\pgfqpoint{2.617977in}{1.553003in}}{\pgfqpoint{2.607378in}{1.557393in}}{\pgfqpoint{2.596328in}{1.557393in}}%
\pgfpathcurveto{\pgfqpoint{2.585278in}{1.557393in}}{\pgfqpoint{2.574679in}{1.553003in}}{\pgfqpoint{2.566865in}{1.545190in}}%
\pgfpathcurveto{\pgfqpoint{2.559052in}{1.537376in}}{\pgfqpoint{2.554661in}{1.526777in}}{\pgfqpoint{2.554661in}{1.515727in}}%
\pgfpathcurveto{\pgfqpoint{2.554661in}{1.504677in}}{\pgfqpoint{2.559052in}{1.494078in}}{\pgfqpoint{2.566865in}{1.486264in}}%
\pgfpathcurveto{\pgfqpoint{2.574679in}{1.478450in}}{\pgfqpoint{2.585278in}{1.474060in}}{\pgfqpoint{2.596328in}{1.474060in}}%
\pgfpathclose%
\pgfusepath{stroke,fill}%
\end{pgfscope}%
\begin{pgfscope}%
\pgfpathrectangle{\pgfqpoint{0.600000in}{0.600000in}}{\pgfqpoint{3.900000in}{3.900000in}}%
\pgfusepath{clip}%
\pgfsetbuttcap%
\pgfsetroundjoin%
\definecolor{currentfill}{rgb}{0.121569,0.466667,0.705882}%
\pgfsetfillcolor{currentfill}%
\pgfsetlinewidth{1.003750pt}%
\definecolor{currentstroke}{rgb}{0.121569,0.466667,0.705882}%
\pgfsetstrokecolor{currentstroke}%
\pgfsetdash{}{0pt}%
\pgfpathmoveto{\pgfqpoint{2.413271in}{1.902107in}}%
\pgfpathcurveto{\pgfqpoint{2.424321in}{1.902107in}}{\pgfqpoint{2.434920in}{1.906498in}}{\pgfqpoint{2.442734in}{1.914311in}}%
\pgfpathcurveto{\pgfqpoint{2.450548in}{1.922125in}}{\pgfqpoint{2.454938in}{1.932724in}}{\pgfqpoint{2.454938in}{1.943774in}}%
\pgfpathcurveto{\pgfqpoint{2.454938in}{1.954824in}}{\pgfqpoint{2.450548in}{1.965423in}}{\pgfqpoint{2.442734in}{1.973237in}}%
\pgfpathcurveto{\pgfqpoint{2.434920in}{1.981051in}}{\pgfqpoint{2.424321in}{1.985441in}}{\pgfqpoint{2.413271in}{1.985441in}}%
\pgfpathcurveto{\pgfqpoint{2.402221in}{1.985441in}}{\pgfqpoint{2.391622in}{1.981051in}}{\pgfqpoint{2.383809in}{1.973237in}}%
\pgfpathcurveto{\pgfqpoint{2.375995in}{1.965423in}}{\pgfqpoint{2.371605in}{1.954824in}}{\pgfqpoint{2.371605in}{1.943774in}}%
\pgfpathcurveto{\pgfqpoint{2.371605in}{1.932724in}}{\pgfqpoint{2.375995in}{1.922125in}}{\pgfqpoint{2.383809in}{1.914311in}}%
\pgfpathcurveto{\pgfqpoint{2.391622in}{1.906498in}}{\pgfqpoint{2.402221in}{1.902107in}}{\pgfqpoint{2.413271in}{1.902107in}}%
\pgfpathclose%
\pgfusepath{stroke,fill}%
\end{pgfscope}%
\begin{pgfscope}%
\pgfpathrectangle{\pgfqpoint{0.600000in}{0.600000in}}{\pgfqpoint{3.900000in}{3.900000in}}%
\pgfusepath{clip}%
\pgfsetbuttcap%
\pgfsetroundjoin%
\definecolor{currentfill}{rgb}{0.121569,0.466667,0.705882}%
\pgfsetfillcolor{currentfill}%
\pgfsetlinewidth{1.003750pt}%
\definecolor{currentstroke}{rgb}{0.121569,0.466667,0.705882}%
\pgfsetstrokecolor{currentstroke}%
\pgfsetdash{}{0pt}%
\pgfpathmoveto{\pgfqpoint{1.873512in}{1.683575in}}%
\pgfpathcurveto{\pgfqpoint{1.884562in}{1.683575in}}{\pgfqpoint{1.895161in}{1.687965in}}{\pgfqpoint{1.902975in}{1.695779in}}%
\pgfpathcurveto{\pgfqpoint{1.910789in}{1.703592in}}{\pgfqpoint{1.915179in}{1.714191in}}{\pgfqpoint{1.915179in}{1.725241in}}%
\pgfpathcurveto{\pgfqpoint{1.915179in}{1.736292in}}{\pgfqpoint{1.910789in}{1.746891in}}{\pgfqpoint{1.902975in}{1.754704in}}%
\pgfpathcurveto{\pgfqpoint{1.895161in}{1.762518in}}{\pgfqpoint{1.884562in}{1.766908in}}{\pgfqpoint{1.873512in}{1.766908in}}%
\pgfpathcurveto{\pgfqpoint{1.862462in}{1.766908in}}{\pgfqpoint{1.851863in}{1.762518in}}{\pgfqpoint{1.844050in}{1.754704in}}%
\pgfpathcurveto{\pgfqpoint{1.836236in}{1.746891in}}{\pgfqpoint{1.831846in}{1.736292in}}{\pgfqpoint{1.831846in}{1.725241in}}%
\pgfpathcurveto{\pgfqpoint{1.831846in}{1.714191in}}{\pgfqpoint{1.836236in}{1.703592in}}{\pgfqpoint{1.844050in}{1.695779in}}%
\pgfpathcurveto{\pgfqpoint{1.851863in}{1.687965in}}{\pgfqpoint{1.862462in}{1.683575in}}{\pgfqpoint{1.873512in}{1.683575in}}%
\pgfpathclose%
\pgfusepath{stroke,fill}%
\end{pgfscope}%
\begin{pgfscope}%
\pgfpathrectangle{\pgfqpoint{0.600000in}{0.600000in}}{\pgfqpoint{3.900000in}{3.900000in}}%
\pgfusepath{clip}%
\pgfsetbuttcap%
\pgfsetroundjoin%
\definecolor{currentfill}{rgb}{0.121569,0.466667,0.705882}%
\pgfsetfillcolor{currentfill}%
\pgfsetlinewidth{1.003750pt}%
\definecolor{currentstroke}{rgb}{0.121569,0.466667,0.705882}%
\pgfsetstrokecolor{currentstroke}%
\pgfsetdash{}{0pt}%
\pgfpathmoveto{\pgfqpoint{2.361193in}{1.838180in}}%
\pgfpathcurveto{\pgfqpoint{2.372243in}{1.838180in}}{\pgfqpoint{2.382842in}{1.842571in}}{\pgfqpoint{2.390655in}{1.850384in}}%
\pgfpathcurveto{\pgfqpoint{2.398469in}{1.858198in}}{\pgfqpoint{2.402859in}{1.868797in}}{\pgfqpoint{2.402859in}{1.879847in}}%
\pgfpathcurveto{\pgfqpoint{2.402859in}{1.890897in}}{\pgfqpoint{2.398469in}{1.901496in}}{\pgfqpoint{2.390655in}{1.909310in}}%
\pgfpathcurveto{\pgfqpoint{2.382842in}{1.917123in}}{\pgfqpoint{2.372243in}{1.921514in}}{\pgfqpoint{2.361193in}{1.921514in}}%
\pgfpathcurveto{\pgfqpoint{2.350142in}{1.921514in}}{\pgfqpoint{2.339543in}{1.917123in}}{\pgfqpoint{2.331730in}{1.909310in}}%
\pgfpathcurveto{\pgfqpoint{2.323916in}{1.901496in}}{\pgfqpoint{2.319526in}{1.890897in}}{\pgfqpoint{2.319526in}{1.879847in}}%
\pgfpathcurveto{\pgfqpoint{2.319526in}{1.868797in}}{\pgfqpoint{2.323916in}{1.858198in}}{\pgfqpoint{2.331730in}{1.850384in}}%
\pgfpathcurveto{\pgfqpoint{2.339543in}{1.842571in}}{\pgfqpoint{2.350142in}{1.838180in}}{\pgfqpoint{2.361193in}{1.838180in}}%
\pgfpathclose%
\pgfusepath{stroke,fill}%
\end{pgfscope}%
\begin{pgfscope}%
\pgfpathrectangle{\pgfqpoint{0.600000in}{0.600000in}}{\pgfqpoint{3.900000in}{3.900000in}}%
\pgfusepath{clip}%
\pgfsetbuttcap%
\pgfsetroundjoin%
\definecolor{currentfill}{rgb}{0.121569,0.466667,0.705882}%
\pgfsetfillcolor{currentfill}%
\pgfsetlinewidth{1.003750pt}%
\definecolor{currentstroke}{rgb}{0.121569,0.466667,0.705882}%
\pgfsetstrokecolor{currentstroke}%
\pgfsetdash{}{0pt}%
\pgfpathmoveto{\pgfqpoint{2.450254in}{2.164477in}}%
\pgfpathcurveto{\pgfqpoint{2.461304in}{2.164477in}}{\pgfqpoint{2.471903in}{2.168868in}}{\pgfqpoint{2.479717in}{2.176681in}}%
\pgfpathcurveto{\pgfqpoint{2.487530in}{2.184495in}}{\pgfqpoint{2.491921in}{2.195094in}}{\pgfqpoint{2.491921in}{2.206144in}}%
\pgfpathcurveto{\pgfqpoint{2.491921in}{2.217194in}}{\pgfqpoint{2.487530in}{2.227793in}}{\pgfqpoint{2.479717in}{2.235607in}}%
\pgfpathcurveto{\pgfqpoint{2.471903in}{2.243420in}}{\pgfqpoint{2.461304in}{2.247811in}}{\pgfqpoint{2.450254in}{2.247811in}}%
\pgfpathcurveto{\pgfqpoint{2.439204in}{2.247811in}}{\pgfqpoint{2.428605in}{2.243420in}}{\pgfqpoint{2.420791in}{2.235607in}}%
\pgfpathcurveto{\pgfqpoint{2.412978in}{2.227793in}}{\pgfqpoint{2.408587in}{2.217194in}}{\pgfqpoint{2.408587in}{2.206144in}}%
\pgfpathcurveto{\pgfqpoint{2.408587in}{2.195094in}}{\pgfqpoint{2.412978in}{2.184495in}}{\pgfqpoint{2.420791in}{2.176681in}}%
\pgfpathcurveto{\pgfqpoint{2.428605in}{2.168868in}}{\pgfqpoint{2.439204in}{2.164477in}}{\pgfqpoint{2.450254in}{2.164477in}}%
\pgfpathclose%
\pgfusepath{stroke,fill}%
\end{pgfscope}%
\begin{pgfscope}%
\pgfpathrectangle{\pgfqpoint{0.600000in}{0.600000in}}{\pgfqpoint{3.900000in}{3.900000in}}%
\pgfusepath{clip}%
\pgfsetbuttcap%
\pgfsetroundjoin%
\definecolor{currentfill}{rgb}{0.121569,0.466667,0.705882}%
\pgfsetfillcolor{currentfill}%
\pgfsetlinewidth{1.003750pt}%
\definecolor{currentstroke}{rgb}{0.121569,0.466667,0.705882}%
\pgfsetstrokecolor{currentstroke}%
\pgfsetdash{}{0pt}%
\pgfpathmoveto{\pgfqpoint{2.708370in}{1.691419in}}%
\pgfpathcurveto{\pgfqpoint{2.719420in}{1.691419in}}{\pgfqpoint{2.730019in}{1.695809in}}{\pgfqpoint{2.737833in}{1.703623in}}%
\pgfpathcurveto{\pgfqpoint{2.745646in}{1.711436in}}{\pgfqpoint{2.750036in}{1.722035in}}{\pgfqpoint{2.750036in}{1.733086in}}%
\pgfpathcurveto{\pgfqpoint{2.750036in}{1.744136in}}{\pgfqpoint{2.745646in}{1.754735in}}{\pgfqpoint{2.737833in}{1.762548in}}%
\pgfpathcurveto{\pgfqpoint{2.730019in}{1.770362in}}{\pgfqpoint{2.719420in}{1.774752in}}{\pgfqpoint{2.708370in}{1.774752in}}%
\pgfpathcurveto{\pgfqpoint{2.697320in}{1.774752in}}{\pgfqpoint{2.686721in}{1.770362in}}{\pgfqpoint{2.678907in}{1.762548in}}%
\pgfpathcurveto{\pgfqpoint{2.671093in}{1.754735in}}{\pgfqpoint{2.666703in}{1.744136in}}{\pgfqpoint{2.666703in}{1.733086in}}%
\pgfpathcurveto{\pgfqpoint{2.666703in}{1.722035in}}{\pgfqpoint{2.671093in}{1.711436in}}{\pgfqpoint{2.678907in}{1.703623in}}%
\pgfpathcurveto{\pgfqpoint{2.686721in}{1.695809in}}{\pgfqpoint{2.697320in}{1.691419in}}{\pgfqpoint{2.708370in}{1.691419in}}%
\pgfpathclose%
\pgfusepath{stroke,fill}%
\end{pgfscope}%
\begin{pgfscope}%
\pgfpathrectangle{\pgfqpoint{0.600000in}{0.600000in}}{\pgfqpoint{3.900000in}{3.900000in}}%
\pgfusepath{clip}%
\pgfsetbuttcap%
\pgfsetroundjoin%
\definecolor{currentfill}{rgb}{0.121569,0.466667,0.705882}%
\pgfsetfillcolor{currentfill}%
\pgfsetlinewidth{1.003750pt}%
\definecolor{currentstroke}{rgb}{0.121569,0.466667,0.705882}%
\pgfsetstrokecolor{currentstroke}%
\pgfsetdash{}{0pt}%
\pgfpathmoveto{\pgfqpoint{2.376842in}{3.522239in}}%
\pgfpathcurveto{\pgfqpoint{2.387892in}{3.522239in}}{\pgfqpoint{2.398491in}{3.526629in}}{\pgfqpoint{2.406305in}{3.534442in}}%
\pgfpathcurveto{\pgfqpoint{2.414119in}{3.542256in}}{\pgfqpoint{2.418509in}{3.552855in}}{\pgfqpoint{2.418509in}{3.563905in}}%
\pgfpathcurveto{\pgfqpoint{2.418509in}{3.574955in}}{\pgfqpoint{2.414119in}{3.585554in}}{\pgfqpoint{2.406305in}{3.593368in}}%
\pgfpathcurveto{\pgfqpoint{2.398491in}{3.601182in}}{\pgfqpoint{2.387892in}{3.605572in}}{\pgfqpoint{2.376842in}{3.605572in}}%
\pgfpathcurveto{\pgfqpoint{2.365792in}{3.605572in}}{\pgfqpoint{2.355193in}{3.601182in}}{\pgfqpoint{2.347380in}{3.593368in}}%
\pgfpathcurveto{\pgfqpoint{2.339566in}{3.585554in}}{\pgfqpoint{2.335176in}{3.574955in}}{\pgfqpoint{2.335176in}{3.563905in}}%
\pgfpathcurveto{\pgfqpoint{2.335176in}{3.552855in}}{\pgfqpoint{2.339566in}{3.542256in}}{\pgfqpoint{2.347380in}{3.534442in}}%
\pgfpathcurveto{\pgfqpoint{2.355193in}{3.526629in}}{\pgfqpoint{2.365792in}{3.522239in}}{\pgfqpoint{2.376842in}{3.522239in}}%
\pgfpathclose%
\pgfusepath{stroke,fill}%
\end{pgfscope}%
\begin{pgfscope}%
\pgfpathrectangle{\pgfqpoint{0.600000in}{0.600000in}}{\pgfqpoint{3.900000in}{3.900000in}}%
\pgfusepath{clip}%
\pgfsetbuttcap%
\pgfsetroundjoin%
\definecolor{currentfill}{rgb}{0.121569,0.466667,0.705882}%
\pgfsetfillcolor{currentfill}%
\pgfsetlinewidth{1.003750pt}%
\definecolor{currentstroke}{rgb}{0.121569,0.466667,0.705882}%
\pgfsetstrokecolor{currentstroke}%
\pgfsetdash{}{0pt}%
\pgfpathmoveto{\pgfqpoint{2.813072in}{3.627621in}}%
\pgfpathcurveto{\pgfqpoint{2.824122in}{3.627621in}}{\pgfqpoint{2.834721in}{3.632011in}}{\pgfqpoint{2.842535in}{3.639825in}}%
\pgfpathcurveto{\pgfqpoint{2.850348in}{3.647639in}}{\pgfqpoint{2.854739in}{3.658238in}}{\pgfqpoint{2.854739in}{3.669288in}}%
\pgfpathcurveto{\pgfqpoint{2.854739in}{3.680338in}}{\pgfqpoint{2.850348in}{3.690937in}}{\pgfqpoint{2.842535in}{3.698750in}}%
\pgfpathcurveto{\pgfqpoint{2.834721in}{3.706564in}}{\pgfqpoint{2.824122in}{3.710954in}}{\pgfqpoint{2.813072in}{3.710954in}}%
\pgfpathcurveto{\pgfqpoint{2.802022in}{3.710954in}}{\pgfqpoint{2.791423in}{3.706564in}}{\pgfqpoint{2.783609in}{3.698750in}}%
\pgfpathcurveto{\pgfqpoint{2.775796in}{3.690937in}}{\pgfqpoint{2.771405in}{3.680338in}}{\pgfqpoint{2.771405in}{3.669288in}}%
\pgfpathcurveto{\pgfqpoint{2.771405in}{3.658238in}}{\pgfqpoint{2.775796in}{3.647639in}}{\pgfqpoint{2.783609in}{3.639825in}}%
\pgfpathcurveto{\pgfqpoint{2.791423in}{3.632011in}}{\pgfqpoint{2.802022in}{3.627621in}}{\pgfqpoint{2.813072in}{3.627621in}}%
\pgfpathclose%
\pgfusepath{stroke,fill}%
\end{pgfscope}%
\begin{pgfscope}%
\pgfpathrectangle{\pgfqpoint{0.600000in}{0.600000in}}{\pgfqpoint{3.900000in}{3.900000in}}%
\pgfusepath{clip}%
\pgfsetbuttcap%
\pgfsetroundjoin%
\definecolor{currentfill}{rgb}{0.121569,0.466667,0.705882}%
\pgfsetfillcolor{currentfill}%
\pgfsetlinewidth{1.003750pt}%
\definecolor{currentstroke}{rgb}{0.121569,0.466667,0.705882}%
\pgfsetstrokecolor{currentstroke}%
\pgfsetdash{}{0pt}%
\pgfpathmoveto{\pgfqpoint{1.871609in}{1.877609in}}%
\pgfpathcurveto{\pgfqpoint{1.882659in}{1.877609in}}{\pgfqpoint{1.893258in}{1.881999in}}{\pgfqpoint{1.901072in}{1.889813in}}%
\pgfpathcurveto{\pgfqpoint{1.908885in}{1.897626in}}{\pgfqpoint{1.913275in}{1.908225in}}{\pgfqpoint{1.913275in}{1.919276in}}%
\pgfpathcurveto{\pgfqpoint{1.913275in}{1.930326in}}{\pgfqpoint{1.908885in}{1.940925in}}{\pgfqpoint{1.901072in}{1.948738in}}%
\pgfpathcurveto{\pgfqpoint{1.893258in}{1.956552in}}{\pgfqpoint{1.882659in}{1.960942in}}{\pgfqpoint{1.871609in}{1.960942in}}%
\pgfpathcurveto{\pgfqpoint{1.860559in}{1.960942in}}{\pgfqpoint{1.849960in}{1.956552in}}{\pgfqpoint{1.842146in}{1.948738in}}%
\pgfpathcurveto{\pgfqpoint{1.834332in}{1.940925in}}{\pgfqpoint{1.829942in}{1.930326in}}{\pgfqpoint{1.829942in}{1.919276in}}%
\pgfpathcurveto{\pgfqpoint{1.829942in}{1.908225in}}{\pgfqpoint{1.834332in}{1.897626in}}{\pgfqpoint{1.842146in}{1.889813in}}%
\pgfpathcurveto{\pgfqpoint{1.849960in}{1.881999in}}{\pgfqpoint{1.860559in}{1.877609in}}{\pgfqpoint{1.871609in}{1.877609in}}%
\pgfpathclose%
\pgfusepath{stroke,fill}%
\end{pgfscope}%
\begin{pgfscope}%
\pgfpathrectangle{\pgfqpoint{0.600000in}{0.600000in}}{\pgfqpoint{3.900000in}{3.900000in}}%
\pgfusepath{clip}%
\pgfsetbuttcap%
\pgfsetroundjoin%
\definecolor{currentfill}{rgb}{0.121569,0.466667,0.705882}%
\pgfsetfillcolor{currentfill}%
\pgfsetlinewidth{1.003750pt}%
\definecolor{currentstroke}{rgb}{0.121569,0.466667,0.705882}%
\pgfsetstrokecolor{currentstroke}%
\pgfsetdash{}{0pt}%
\pgfpathmoveto{\pgfqpoint{2.061426in}{2.351938in}}%
\pgfpathcurveto{\pgfqpoint{2.072476in}{2.351938in}}{\pgfqpoint{2.083075in}{2.356328in}}{\pgfqpoint{2.090889in}{2.364141in}}%
\pgfpathcurveto{\pgfqpoint{2.098702in}{2.371955in}}{\pgfqpoint{2.103093in}{2.382554in}}{\pgfqpoint{2.103093in}{2.393604in}}%
\pgfpathcurveto{\pgfqpoint{2.103093in}{2.404654in}}{\pgfqpoint{2.098702in}{2.415253in}}{\pgfqpoint{2.090889in}{2.423067in}}%
\pgfpathcurveto{\pgfqpoint{2.083075in}{2.430881in}}{\pgfqpoint{2.072476in}{2.435271in}}{\pgfqpoint{2.061426in}{2.435271in}}%
\pgfpathcurveto{\pgfqpoint{2.050376in}{2.435271in}}{\pgfqpoint{2.039777in}{2.430881in}}{\pgfqpoint{2.031963in}{2.423067in}}%
\pgfpathcurveto{\pgfqpoint{2.024149in}{2.415253in}}{\pgfqpoint{2.019759in}{2.404654in}}{\pgfqpoint{2.019759in}{2.393604in}}%
\pgfpathcurveto{\pgfqpoint{2.019759in}{2.382554in}}{\pgfqpoint{2.024149in}{2.371955in}}{\pgfqpoint{2.031963in}{2.364141in}}%
\pgfpathcurveto{\pgfqpoint{2.039777in}{2.356328in}}{\pgfqpoint{2.050376in}{2.351938in}}{\pgfqpoint{2.061426in}{2.351938in}}%
\pgfpathclose%
\pgfusepath{stroke,fill}%
\end{pgfscope}%
\begin{pgfscope}%
\pgfpathrectangle{\pgfqpoint{0.600000in}{0.600000in}}{\pgfqpoint{3.900000in}{3.900000in}}%
\pgfusepath{clip}%
\pgfsetbuttcap%
\pgfsetroundjoin%
\definecolor{currentfill}{rgb}{0.121569,0.466667,0.705882}%
\pgfsetfillcolor{currentfill}%
\pgfsetlinewidth{1.003750pt}%
\definecolor{currentstroke}{rgb}{0.121569,0.466667,0.705882}%
\pgfsetstrokecolor{currentstroke}%
\pgfsetdash{}{0pt}%
\pgfpathmoveto{\pgfqpoint{2.935335in}{2.895431in}}%
\pgfpathcurveto{\pgfqpoint{2.946385in}{2.895431in}}{\pgfqpoint{2.956984in}{2.899822in}}{\pgfqpoint{2.964798in}{2.907635in}}%
\pgfpathcurveto{\pgfqpoint{2.972612in}{2.915449in}}{\pgfqpoint{2.977002in}{2.926048in}}{\pgfqpoint{2.977002in}{2.937098in}}%
\pgfpathcurveto{\pgfqpoint{2.977002in}{2.948148in}}{\pgfqpoint{2.972612in}{2.958747in}}{\pgfqpoint{2.964798in}{2.966561in}}%
\pgfpathcurveto{\pgfqpoint{2.956984in}{2.974375in}}{\pgfqpoint{2.946385in}{2.978765in}}{\pgfqpoint{2.935335in}{2.978765in}}%
\pgfpathcurveto{\pgfqpoint{2.924285in}{2.978765in}}{\pgfqpoint{2.913686in}{2.974375in}}{\pgfqpoint{2.905872in}{2.966561in}}%
\pgfpathcurveto{\pgfqpoint{2.898059in}{2.958747in}}{\pgfqpoint{2.893669in}{2.948148in}}{\pgfqpoint{2.893669in}{2.937098in}}%
\pgfpathcurveto{\pgfqpoint{2.893669in}{2.926048in}}{\pgfqpoint{2.898059in}{2.915449in}}{\pgfqpoint{2.905872in}{2.907635in}}%
\pgfpathcurveto{\pgfqpoint{2.913686in}{2.899822in}}{\pgfqpoint{2.924285in}{2.895431in}}{\pgfqpoint{2.935335in}{2.895431in}}%
\pgfpathclose%
\pgfusepath{stroke,fill}%
\end{pgfscope}%
\begin{pgfscope}%
\pgfpathrectangle{\pgfqpoint{0.600000in}{0.600000in}}{\pgfqpoint{3.900000in}{3.900000in}}%
\pgfusepath{clip}%
\pgfsetbuttcap%
\pgfsetroundjoin%
\definecolor{currentfill}{rgb}{0.121569,0.466667,0.705882}%
\pgfsetfillcolor{currentfill}%
\pgfsetlinewidth{1.003750pt}%
\definecolor{currentstroke}{rgb}{0.121569,0.466667,0.705882}%
\pgfsetstrokecolor{currentstroke}%
\pgfsetdash{}{0pt}%
\pgfpathmoveto{\pgfqpoint{3.422313in}{1.717965in}}%
\pgfpathcurveto{\pgfqpoint{3.433363in}{1.717965in}}{\pgfqpoint{3.443962in}{1.722355in}}{\pgfqpoint{3.451775in}{1.730169in}}%
\pgfpathcurveto{\pgfqpoint{3.459589in}{1.737983in}}{\pgfqpoint{3.463979in}{1.748582in}}{\pgfqpoint{3.463979in}{1.759632in}}%
\pgfpathcurveto{\pgfqpoint{3.463979in}{1.770682in}}{\pgfqpoint{3.459589in}{1.781281in}}{\pgfqpoint{3.451775in}{1.789095in}}%
\pgfpathcurveto{\pgfqpoint{3.443962in}{1.796908in}}{\pgfqpoint{3.433363in}{1.801298in}}{\pgfqpoint{3.422313in}{1.801298in}}%
\pgfpathcurveto{\pgfqpoint{3.411263in}{1.801298in}}{\pgfqpoint{3.400663in}{1.796908in}}{\pgfqpoint{3.392850in}{1.789095in}}%
\pgfpathcurveto{\pgfqpoint{3.385036in}{1.781281in}}{\pgfqpoint{3.380646in}{1.770682in}}{\pgfqpoint{3.380646in}{1.759632in}}%
\pgfpathcurveto{\pgfqpoint{3.380646in}{1.748582in}}{\pgfqpoint{3.385036in}{1.737983in}}{\pgfqpoint{3.392850in}{1.730169in}}%
\pgfpathcurveto{\pgfqpoint{3.400663in}{1.722355in}}{\pgfqpoint{3.411263in}{1.717965in}}{\pgfqpoint{3.422313in}{1.717965in}}%
\pgfpathclose%
\pgfusepath{stroke,fill}%
\end{pgfscope}%
\begin{pgfscope}%
\pgfpathrectangle{\pgfqpoint{0.600000in}{0.600000in}}{\pgfqpoint{3.900000in}{3.900000in}}%
\pgfusepath{clip}%
\pgfsetbuttcap%
\pgfsetroundjoin%
\definecolor{currentfill}{rgb}{0.121569,0.466667,0.705882}%
\pgfsetfillcolor{currentfill}%
\pgfsetlinewidth{1.003750pt}%
\definecolor{currentstroke}{rgb}{0.121569,0.466667,0.705882}%
\pgfsetstrokecolor{currentstroke}%
\pgfsetdash{}{0pt}%
\pgfpathmoveto{\pgfqpoint{1.639817in}{2.109174in}}%
\pgfpathcurveto{\pgfqpoint{1.650867in}{2.109174in}}{\pgfqpoint{1.661466in}{2.113565in}}{\pgfqpoint{1.669279in}{2.121378in}}%
\pgfpathcurveto{\pgfqpoint{1.677093in}{2.129192in}}{\pgfqpoint{1.681483in}{2.139791in}}{\pgfqpoint{1.681483in}{2.150841in}}%
\pgfpathcurveto{\pgfqpoint{1.681483in}{2.161891in}}{\pgfqpoint{1.677093in}{2.172490in}}{\pgfqpoint{1.669279in}{2.180304in}}%
\pgfpathcurveto{\pgfqpoint{1.661466in}{2.188117in}}{\pgfqpoint{1.650867in}{2.192508in}}{\pgfqpoint{1.639817in}{2.192508in}}%
\pgfpathcurveto{\pgfqpoint{1.628767in}{2.192508in}}{\pgfqpoint{1.618167in}{2.188117in}}{\pgfqpoint{1.610354in}{2.180304in}}%
\pgfpathcurveto{\pgfqpoint{1.602540in}{2.172490in}}{\pgfqpoint{1.598150in}{2.161891in}}{\pgfqpoint{1.598150in}{2.150841in}}%
\pgfpathcurveto{\pgfqpoint{1.598150in}{2.139791in}}{\pgfqpoint{1.602540in}{2.129192in}}{\pgfqpoint{1.610354in}{2.121378in}}%
\pgfpathcurveto{\pgfqpoint{1.618167in}{2.113565in}}{\pgfqpoint{1.628767in}{2.109174in}}{\pgfqpoint{1.639817in}{2.109174in}}%
\pgfpathclose%
\pgfusepath{stroke,fill}%
\end{pgfscope}%
\begin{pgfscope}%
\pgfpathrectangle{\pgfqpoint{0.600000in}{0.600000in}}{\pgfqpoint{3.900000in}{3.900000in}}%
\pgfusepath{clip}%
\pgfsetbuttcap%
\pgfsetroundjoin%
\definecolor{currentfill}{rgb}{0.121569,0.466667,0.705882}%
\pgfsetfillcolor{currentfill}%
\pgfsetlinewidth{1.003750pt}%
\definecolor{currentstroke}{rgb}{0.121569,0.466667,0.705882}%
\pgfsetstrokecolor{currentstroke}%
\pgfsetdash{}{0pt}%
\pgfpathmoveto{\pgfqpoint{2.154696in}{3.178722in}}%
\pgfpathcurveto{\pgfqpoint{2.165746in}{3.178722in}}{\pgfqpoint{2.176345in}{3.183112in}}{\pgfqpoint{2.184159in}{3.190926in}}%
\pgfpathcurveto{\pgfqpoint{2.191972in}{3.198739in}}{\pgfqpoint{2.196362in}{3.209339in}}{\pgfqpoint{2.196362in}{3.220389in}}%
\pgfpathcurveto{\pgfqpoint{2.196362in}{3.231439in}}{\pgfqpoint{2.191972in}{3.242038in}}{\pgfqpoint{2.184159in}{3.249851in}}%
\pgfpathcurveto{\pgfqpoint{2.176345in}{3.257665in}}{\pgfqpoint{2.165746in}{3.262055in}}{\pgfqpoint{2.154696in}{3.262055in}}%
\pgfpathcurveto{\pgfqpoint{2.143646in}{3.262055in}}{\pgfqpoint{2.133047in}{3.257665in}}{\pgfqpoint{2.125233in}{3.249851in}}%
\pgfpathcurveto{\pgfqpoint{2.117419in}{3.242038in}}{\pgfqpoint{2.113029in}{3.231439in}}{\pgfqpoint{2.113029in}{3.220389in}}%
\pgfpathcurveto{\pgfqpoint{2.113029in}{3.209339in}}{\pgfqpoint{2.117419in}{3.198739in}}{\pgfqpoint{2.125233in}{3.190926in}}%
\pgfpathcurveto{\pgfqpoint{2.133047in}{3.183112in}}{\pgfqpoint{2.143646in}{3.178722in}}{\pgfqpoint{2.154696in}{3.178722in}}%
\pgfpathclose%
\pgfusepath{stroke,fill}%
\end{pgfscope}%
\begin{pgfscope}%
\pgfpathrectangle{\pgfqpoint{0.600000in}{0.600000in}}{\pgfqpoint{3.900000in}{3.900000in}}%
\pgfusepath{clip}%
\pgfsetbuttcap%
\pgfsetroundjoin%
\definecolor{currentfill}{rgb}{0.121569,0.466667,0.705882}%
\pgfsetfillcolor{currentfill}%
\pgfsetlinewidth{1.003750pt}%
\definecolor{currentstroke}{rgb}{0.121569,0.466667,0.705882}%
\pgfsetstrokecolor{currentstroke}%
\pgfsetdash{}{0pt}%
\pgfpathmoveto{\pgfqpoint{2.142106in}{1.691799in}}%
\pgfpathcurveto{\pgfqpoint{2.153157in}{1.691799in}}{\pgfqpoint{2.163756in}{1.696189in}}{\pgfqpoint{2.171569in}{1.704003in}}%
\pgfpathcurveto{\pgfqpoint{2.179383in}{1.711816in}}{\pgfqpoint{2.183773in}{1.722415in}}{\pgfqpoint{2.183773in}{1.733465in}}%
\pgfpathcurveto{\pgfqpoint{2.183773in}{1.744515in}}{\pgfqpoint{2.179383in}{1.755114in}}{\pgfqpoint{2.171569in}{1.762928in}}%
\pgfpathcurveto{\pgfqpoint{2.163756in}{1.770742in}}{\pgfqpoint{2.153157in}{1.775132in}}{\pgfqpoint{2.142106in}{1.775132in}}%
\pgfpathcurveto{\pgfqpoint{2.131056in}{1.775132in}}{\pgfqpoint{2.120457in}{1.770742in}}{\pgfqpoint{2.112644in}{1.762928in}}%
\pgfpathcurveto{\pgfqpoint{2.104830in}{1.755114in}}{\pgfqpoint{2.100440in}{1.744515in}}{\pgfqpoint{2.100440in}{1.733465in}}%
\pgfpathcurveto{\pgfqpoint{2.100440in}{1.722415in}}{\pgfqpoint{2.104830in}{1.711816in}}{\pgfqpoint{2.112644in}{1.704003in}}%
\pgfpathcurveto{\pgfqpoint{2.120457in}{1.696189in}}{\pgfqpoint{2.131056in}{1.691799in}}{\pgfqpoint{2.142106in}{1.691799in}}%
\pgfpathclose%
\pgfusepath{stroke,fill}%
\end{pgfscope}%
\begin{pgfscope}%
\pgfpathrectangle{\pgfqpoint{0.600000in}{0.600000in}}{\pgfqpoint{3.900000in}{3.900000in}}%
\pgfusepath{clip}%
\pgfsetbuttcap%
\pgfsetroundjoin%
\definecolor{currentfill}{rgb}{0.121569,0.466667,0.705882}%
\pgfsetfillcolor{currentfill}%
\pgfsetlinewidth{1.003750pt}%
\definecolor{currentstroke}{rgb}{0.121569,0.466667,0.705882}%
\pgfsetstrokecolor{currentstroke}%
\pgfsetdash{}{0pt}%
\pgfpathmoveto{\pgfqpoint{3.807428in}{2.851244in}}%
\pgfpathcurveto{\pgfqpoint{3.818478in}{2.851244in}}{\pgfqpoint{3.829077in}{2.855634in}}{\pgfqpoint{3.836891in}{2.863448in}}%
\pgfpathcurveto{\pgfqpoint{3.844704in}{2.871261in}}{\pgfqpoint{3.849095in}{2.881860in}}{\pgfqpoint{3.849095in}{2.892910in}}%
\pgfpathcurveto{\pgfqpoint{3.849095in}{2.903961in}}{\pgfqpoint{3.844704in}{2.914560in}}{\pgfqpoint{3.836891in}{2.922373in}}%
\pgfpathcurveto{\pgfqpoint{3.829077in}{2.930187in}}{\pgfqpoint{3.818478in}{2.934577in}}{\pgfqpoint{3.807428in}{2.934577in}}%
\pgfpathcurveto{\pgfqpoint{3.796378in}{2.934577in}}{\pgfqpoint{3.785779in}{2.930187in}}{\pgfqpoint{3.777965in}{2.922373in}}%
\pgfpathcurveto{\pgfqpoint{3.770151in}{2.914560in}}{\pgfqpoint{3.765761in}{2.903961in}}{\pgfqpoint{3.765761in}{2.892910in}}%
\pgfpathcurveto{\pgfqpoint{3.765761in}{2.881860in}}{\pgfqpoint{3.770151in}{2.871261in}}{\pgfqpoint{3.777965in}{2.863448in}}%
\pgfpathcurveto{\pgfqpoint{3.785779in}{2.855634in}}{\pgfqpoint{3.796378in}{2.851244in}}{\pgfqpoint{3.807428in}{2.851244in}}%
\pgfpathclose%
\pgfusepath{stroke,fill}%
\end{pgfscope}%
\begin{pgfscope}%
\pgfpathrectangle{\pgfqpoint{0.600000in}{0.600000in}}{\pgfqpoint{3.900000in}{3.900000in}}%
\pgfusepath{clip}%
\pgfsetbuttcap%
\pgfsetroundjoin%
\definecolor{currentfill}{rgb}{0.121569,0.466667,0.705882}%
\pgfsetfillcolor{currentfill}%
\pgfsetlinewidth{1.003750pt}%
\definecolor{currentstroke}{rgb}{0.121569,0.466667,0.705882}%
\pgfsetstrokecolor{currentstroke}%
\pgfsetdash{}{0pt}%
\pgfpathmoveto{\pgfqpoint{2.850670in}{2.637617in}}%
\pgfpathcurveto{\pgfqpoint{2.861720in}{2.637617in}}{\pgfqpoint{2.872319in}{2.642007in}}{\pgfqpoint{2.880133in}{2.649820in}}%
\pgfpathcurveto{\pgfqpoint{2.887946in}{2.657634in}}{\pgfqpoint{2.892336in}{2.668233in}}{\pgfqpoint{2.892336in}{2.679283in}}%
\pgfpathcurveto{\pgfqpoint{2.892336in}{2.690333in}}{\pgfqpoint{2.887946in}{2.700932in}}{\pgfqpoint{2.880133in}{2.708746in}}%
\pgfpathcurveto{\pgfqpoint{2.872319in}{2.716560in}}{\pgfqpoint{2.861720in}{2.720950in}}{\pgfqpoint{2.850670in}{2.720950in}}%
\pgfpathcurveto{\pgfqpoint{2.839620in}{2.720950in}}{\pgfqpoint{2.829021in}{2.716560in}}{\pgfqpoint{2.821207in}{2.708746in}}%
\pgfpathcurveto{\pgfqpoint{2.813393in}{2.700932in}}{\pgfqpoint{2.809003in}{2.690333in}}{\pgfqpoint{2.809003in}{2.679283in}}%
\pgfpathcurveto{\pgfqpoint{2.809003in}{2.668233in}}{\pgfqpoint{2.813393in}{2.657634in}}{\pgfqpoint{2.821207in}{2.649820in}}%
\pgfpathcurveto{\pgfqpoint{2.829021in}{2.642007in}}{\pgfqpoint{2.839620in}{2.637617in}}{\pgfqpoint{2.850670in}{2.637617in}}%
\pgfpathclose%
\pgfusepath{stroke,fill}%
\end{pgfscope}%
\begin{pgfscope}%
\pgfpathrectangle{\pgfqpoint{0.600000in}{0.600000in}}{\pgfqpoint{3.900000in}{3.900000in}}%
\pgfusepath{clip}%
\pgfsetbuttcap%
\pgfsetroundjoin%
\definecolor{currentfill}{rgb}{0.121569,0.466667,0.705882}%
\pgfsetfillcolor{currentfill}%
\pgfsetlinewidth{1.003750pt}%
\definecolor{currentstroke}{rgb}{0.121569,0.466667,0.705882}%
\pgfsetstrokecolor{currentstroke}%
\pgfsetdash{}{0pt}%
\pgfpathmoveto{\pgfqpoint{2.190489in}{2.738032in}}%
\pgfpathcurveto{\pgfqpoint{2.201539in}{2.738032in}}{\pgfqpoint{2.212138in}{2.742422in}}{\pgfqpoint{2.219952in}{2.750236in}}%
\pgfpathcurveto{\pgfqpoint{2.227766in}{2.758049in}}{\pgfqpoint{2.232156in}{2.768648in}}{\pgfqpoint{2.232156in}{2.779698in}}%
\pgfpathcurveto{\pgfqpoint{2.232156in}{2.790749in}}{\pgfqpoint{2.227766in}{2.801348in}}{\pgfqpoint{2.219952in}{2.809161in}}%
\pgfpathcurveto{\pgfqpoint{2.212138in}{2.816975in}}{\pgfqpoint{2.201539in}{2.821365in}}{\pgfqpoint{2.190489in}{2.821365in}}%
\pgfpathcurveto{\pgfqpoint{2.179439in}{2.821365in}}{\pgfqpoint{2.168840in}{2.816975in}}{\pgfqpoint{2.161026in}{2.809161in}}%
\pgfpathcurveto{\pgfqpoint{2.153213in}{2.801348in}}{\pgfqpoint{2.148823in}{2.790749in}}{\pgfqpoint{2.148823in}{2.779698in}}%
\pgfpathcurveto{\pgfqpoint{2.148823in}{2.768648in}}{\pgfqpoint{2.153213in}{2.758049in}}{\pgfqpoint{2.161026in}{2.750236in}}%
\pgfpathcurveto{\pgfqpoint{2.168840in}{2.742422in}}{\pgfqpoint{2.179439in}{2.738032in}}{\pgfqpoint{2.190489in}{2.738032in}}%
\pgfpathclose%
\pgfusepath{stroke,fill}%
\end{pgfscope}%
\begin{pgfscope}%
\pgfpathrectangle{\pgfqpoint{0.600000in}{0.600000in}}{\pgfqpoint{3.900000in}{3.900000in}}%
\pgfusepath{clip}%
\pgfsetbuttcap%
\pgfsetroundjoin%
\definecolor{currentfill}{rgb}{0.121569,0.466667,0.705882}%
\pgfsetfillcolor{currentfill}%
\pgfsetlinewidth{1.003750pt}%
\definecolor{currentstroke}{rgb}{0.121569,0.466667,0.705882}%
\pgfsetstrokecolor{currentstroke}%
\pgfsetdash{}{0pt}%
\pgfpathmoveto{\pgfqpoint{2.782533in}{2.415723in}}%
\pgfpathcurveto{\pgfqpoint{2.793583in}{2.415723in}}{\pgfqpoint{2.804182in}{2.420113in}}{\pgfqpoint{2.811996in}{2.427927in}}%
\pgfpathcurveto{\pgfqpoint{2.819810in}{2.435741in}}{\pgfqpoint{2.824200in}{2.446340in}}{\pgfqpoint{2.824200in}{2.457390in}}%
\pgfpathcurveto{\pgfqpoint{2.824200in}{2.468440in}}{\pgfqpoint{2.819810in}{2.479039in}}{\pgfqpoint{2.811996in}{2.486852in}}%
\pgfpathcurveto{\pgfqpoint{2.804182in}{2.494666in}}{\pgfqpoint{2.793583in}{2.499056in}}{\pgfqpoint{2.782533in}{2.499056in}}%
\pgfpathcurveto{\pgfqpoint{2.771483in}{2.499056in}}{\pgfqpoint{2.760884in}{2.494666in}}{\pgfqpoint{2.753070in}{2.486852in}}%
\pgfpathcurveto{\pgfqpoint{2.745257in}{2.479039in}}{\pgfqpoint{2.740867in}{2.468440in}}{\pgfqpoint{2.740867in}{2.457390in}}%
\pgfpathcurveto{\pgfqpoint{2.740867in}{2.446340in}}{\pgfqpoint{2.745257in}{2.435741in}}{\pgfqpoint{2.753070in}{2.427927in}}%
\pgfpathcurveto{\pgfqpoint{2.760884in}{2.420113in}}{\pgfqpoint{2.771483in}{2.415723in}}{\pgfqpoint{2.782533in}{2.415723in}}%
\pgfpathclose%
\pgfusepath{stroke,fill}%
\end{pgfscope}%
\begin{pgfscope}%
\pgfpathrectangle{\pgfqpoint{0.600000in}{0.600000in}}{\pgfqpoint{3.900000in}{3.900000in}}%
\pgfusepath{clip}%
\pgfsetbuttcap%
\pgfsetroundjoin%
\definecolor{currentfill}{rgb}{0.121569,0.466667,0.705882}%
\pgfsetfillcolor{currentfill}%
\pgfsetlinewidth{1.003750pt}%
\definecolor{currentstroke}{rgb}{0.121569,0.466667,0.705882}%
\pgfsetstrokecolor{currentstroke}%
\pgfsetdash{}{0pt}%
\pgfpathmoveto{\pgfqpoint{1.515645in}{3.917537in}}%
\pgfpathcurveto{\pgfqpoint{1.526696in}{3.917537in}}{\pgfqpoint{1.537295in}{3.921927in}}{\pgfqpoint{1.545108in}{3.929741in}}%
\pgfpathcurveto{\pgfqpoint{1.552922in}{3.937554in}}{\pgfqpoint{1.557312in}{3.948153in}}{\pgfqpoint{1.557312in}{3.959203in}}%
\pgfpathcurveto{\pgfqpoint{1.557312in}{3.970253in}}{\pgfqpoint{1.552922in}{3.980853in}}{\pgfqpoint{1.545108in}{3.988666in}}%
\pgfpathcurveto{\pgfqpoint{1.537295in}{3.996480in}}{\pgfqpoint{1.526696in}{4.000870in}}{\pgfqpoint{1.515645in}{4.000870in}}%
\pgfpathcurveto{\pgfqpoint{1.504595in}{4.000870in}}{\pgfqpoint{1.493996in}{3.996480in}}{\pgfqpoint{1.486183in}{3.988666in}}%
\pgfpathcurveto{\pgfqpoint{1.478369in}{3.980853in}}{\pgfqpoint{1.473979in}{3.970253in}}{\pgfqpoint{1.473979in}{3.959203in}}%
\pgfpathcurveto{\pgfqpoint{1.473979in}{3.948153in}}{\pgfqpoint{1.478369in}{3.937554in}}{\pgfqpoint{1.486183in}{3.929741in}}%
\pgfpathcurveto{\pgfqpoint{1.493996in}{3.921927in}}{\pgfqpoint{1.504595in}{3.917537in}}{\pgfqpoint{1.515645in}{3.917537in}}%
\pgfpathclose%
\pgfusepath{stroke,fill}%
\end{pgfscope}%
\begin{pgfscope}%
\pgfpathrectangle{\pgfqpoint{0.600000in}{0.600000in}}{\pgfqpoint{3.900000in}{3.900000in}}%
\pgfusepath{clip}%
\pgfsetbuttcap%
\pgfsetroundjoin%
\definecolor{currentfill}{rgb}{0.121569,0.466667,0.705882}%
\pgfsetfillcolor{currentfill}%
\pgfsetlinewidth{1.003750pt}%
\definecolor{currentstroke}{rgb}{0.121569,0.466667,0.705882}%
\pgfsetstrokecolor{currentstroke}%
\pgfsetdash{}{0pt}%
\pgfpathmoveto{\pgfqpoint{2.732680in}{2.498479in}}%
\pgfpathcurveto{\pgfqpoint{2.743730in}{2.498479in}}{\pgfqpoint{2.754329in}{2.502869in}}{\pgfqpoint{2.762143in}{2.510683in}}%
\pgfpathcurveto{\pgfqpoint{2.769957in}{2.518497in}}{\pgfqpoint{2.774347in}{2.529096in}}{\pgfqpoint{2.774347in}{2.540146in}}%
\pgfpathcurveto{\pgfqpoint{2.774347in}{2.551196in}}{\pgfqpoint{2.769957in}{2.561795in}}{\pgfqpoint{2.762143in}{2.569608in}}%
\pgfpathcurveto{\pgfqpoint{2.754329in}{2.577422in}}{\pgfqpoint{2.743730in}{2.581812in}}{\pgfqpoint{2.732680in}{2.581812in}}%
\pgfpathcurveto{\pgfqpoint{2.721630in}{2.581812in}}{\pgfqpoint{2.711031in}{2.577422in}}{\pgfqpoint{2.703217in}{2.569608in}}%
\pgfpathcurveto{\pgfqpoint{2.695404in}{2.561795in}}{\pgfqpoint{2.691014in}{2.551196in}}{\pgfqpoint{2.691014in}{2.540146in}}%
\pgfpathcurveto{\pgfqpoint{2.691014in}{2.529096in}}{\pgfqpoint{2.695404in}{2.518497in}}{\pgfqpoint{2.703217in}{2.510683in}}%
\pgfpathcurveto{\pgfqpoint{2.711031in}{2.502869in}}{\pgfqpoint{2.721630in}{2.498479in}}{\pgfqpoint{2.732680in}{2.498479in}}%
\pgfpathclose%
\pgfusepath{stroke,fill}%
\end{pgfscope}%
\begin{pgfscope}%
\pgfpathrectangle{\pgfqpoint{0.600000in}{0.600000in}}{\pgfqpoint{3.900000in}{3.900000in}}%
\pgfusepath{clip}%
\pgfsetbuttcap%
\pgfsetroundjoin%
\definecolor{currentfill}{rgb}{0.121569,0.466667,0.705882}%
\pgfsetfillcolor{currentfill}%
\pgfsetlinewidth{1.003750pt}%
\definecolor{currentstroke}{rgb}{0.121569,0.466667,0.705882}%
\pgfsetstrokecolor{currentstroke}%
\pgfsetdash{}{0pt}%
\pgfpathmoveto{\pgfqpoint{3.276971in}{2.401299in}}%
\pgfpathcurveto{\pgfqpoint{3.288022in}{2.401299in}}{\pgfqpoint{3.298621in}{2.405689in}}{\pgfqpoint{3.306434in}{2.413503in}}%
\pgfpathcurveto{\pgfqpoint{3.314248in}{2.421316in}}{\pgfqpoint{3.318638in}{2.431915in}}{\pgfqpoint{3.318638in}{2.442966in}}%
\pgfpathcurveto{\pgfqpoint{3.318638in}{2.454016in}}{\pgfqpoint{3.314248in}{2.464615in}}{\pgfqpoint{3.306434in}{2.472428in}}%
\pgfpathcurveto{\pgfqpoint{3.298621in}{2.480242in}}{\pgfqpoint{3.288022in}{2.484632in}}{\pgfqpoint{3.276971in}{2.484632in}}%
\pgfpathcurveto{\pgfqpoint{3.265921in}{2.484632in}}{\pgfqpoint{3.255322in}{2.480242in}}{\pgfqpoint{3.247509in}{2.472428in}}%
\pgfpathcurveto{\pgfqpoint{3.239695in}{2.464615in}}{\pgfqpoint{3.235305in}{2.454016in}}{\pgfqpoint{3.235305in}{2.442966in}}%
\pgfpathcurveto{\pgfqpoint{3.235305in}{2.431915in}}{\pgfqpoint{3.239695in}{2.421316in}}{\pgfqpoint{3.247509in}{2.413503in}}%
\pgfpathcurveto{\pgfqpoint{3.255322in}{2.405689in}}{\pgfqpoint{3.265921in}{2.401299in}}{\pgfqpoint{3.276971in}{2.401299in}}%
\pgfpathclose%
\pgfusepath{stroke,fill}%
\end{pgfscope}%
\begin{pgfscope}%
\pgfpathrectangle{\pgfqpoint{0.600000in}{0.600000in}}{\pgfqpoint{3.900000in}{3.900000in}}%
\pgfusepath{clip}%
\pgfsetbuttcap%
\pgfsetroundjoin%
\definecolor{currentfill}{rgb}{0.121569,0.466667,0.705882}%
\pgfsetfillcolor{currentfill}%
\pgfsetlinewidth{1.003750pt}%
\definecolor{currentstroke}{rgb}{0.121569,0.466667,0.705882}%
\pgfsetstrokecolor{currentstroke}%
\pgfsetdash{}{0pt}%
\pgfpathmoveto{\pgfqpoint{1.800489in}{3.671709in}}%
\pgfpathcurveto{\pgfqpoint{1.811539in}{3.671709in}}{\pgfqpoint{1.822138in}{3.676099in}}{\pgfqpoint{1.829952in}{3.683913in}}%
\pgfpathcurveto{\pgfqpoint{1.837766in}{3.691727in}}{\pgfqpoint{1.842156in}{3.702326in}}{\pgfqpoint{1.842156in}{3.713376in}}%
\pgfpathcurveto{\pgfqpoint{1.842156in}{3.724426in}}{\pgfqpoint{1.837766in}{3.735025in}}{\pgfqpoint{1.829952in}{3.742838in}}%
\pgfpathcurveto{\pgfqpoint{1.822138in}{3.750652in}}{\pgfqpoint{1.811539in}{3.755042in}}{\pgfqpoint{1.800489in}{3.755042in}}%
\pgfpathcurveto{\pgfqpoint{1.789439in}{3.755042in}}{\pgfqpoint{1.778840in}{3.750652in}}{\pgfqpoint{1.771027in}{3.742838in}}%
\pgfpathcurveto{\pgfqpoint{1.763213in}{3.735025in}}{\pgfqpoint{1.758823in}{3.724426in}}{\pgfqpoint{1.758823in}{3.713376in}}%
\pgfpathcurveto{\pgfqpoint{1.758823in}{3.702326in}}{\pgfqpoint{1.763213in}{3.691727in}}{\pgfqpoint{1.771027in}{3.683913in}}%
\pgfpathcurveto{\pgfqpoint{1.778840in}{3.676099in}}{\pgfqpoint{1.789439in}{3.671709in}}{\pgfqpoint{1.800489in}{3.671709in}}%
\pgfpathclose%
\pgfusepath{stroke,fill}%
\end{pgfscope}%
\begin{pgfscope}%
\pgfpathrectangle{\pgfqpoint{0.600000in}{0.600000in}}{\pgfqpoint{3.900000in}{3.900000in}}%
\pgfusepath{clip}%
\pgfsetbuttcap%
\pgfsetroundjoin%
\definecolor{currentfill}{rgb}{0.121569,0.466667,0.705882}%
\pgfsetfillcolor{currentfill}%
\pgfsetlinewidth{1.003750pt}%
\definecolor{currentstroke}{rgb}{0.121569,0.466667,0.705882}%
\pgfsetstrokecolor{currentstroke}%
\pgfsetdash{}{0pt}%
\pgfpathmoveto{\pgfqpoint{2.275495in}{2.620592in}}%
\pgfpathcurveto{\pgfqpoint{2.286545in}{2.620592in}}{\pgfqpoint{2.297144in}{2.624982in}}{\pgfqpoint{2.304958in}{2.632796in}}%
\pgfpathcurveto{\pgfqpoint{2.312771in}{2.640609in}}{\pgfqpoint{2.317162in}{2.651208in}}{\pgfqpoint{2.317162in}{2.662258in}}%
\pgfpathcurveto{\pgfqpoint{2.317162in}{2.673308in}}{\pgfqpoint{2.312771in}{2.683907in}}{\pgfqpoint{2.304958in}{2.691721in}}%
\pgfpathcurveto{\pgfqpoint{2.297144in}{2.699535in}}{\pgfqpoint{2.286545in}{2.703925in}}{\pgfqpoint{2.275495in}{2.703925in}}%
\pgfpathcurveto{\pgfqpoint{2.264445in}{2.703925in}}{\pgfqpoint{2.253846in}{2.699535in}}{\pgfqpoint{2.246032in}{2.691721in}}%
\pgfpathcurveto{\pgfqpoint{2.238219in}{2.683907in}}{\pgfqpoint{2.233828in}{2.673308in}}{\pgfqpoint{2.233828in}{2.662258in}}%
\pgfpathcurveto{\pgfqpoint{2.233828in}{2.651208in}}{\pgfqpoint{2.238219in}{2.640609in}}{\pgfqpoint{2.246032in}{2.632796in}}%
\pgfpathcurveto{\pgfqpoint{2.253846in}{2.624982in}}{\pgfqpoint{2.264445in}{2.620592in}}{\pgfqpoint{2.275495in}{2.620592in}}%
\pgfpathclose%
\pgfusepath{stroke,fill}%
\end{pgfscope}%
\begin{pgfscope}%
\pgfpathrectangle{\pgfqpoint{0.600000in}{0.600000in}}{\pgfqpoint{3.900000in}{3.900000in}}%
\pgfusepath{clip}%
\pgfsetbuttcap%
\pgfsetroundjoin%
\definecolor{currentfill}{rgb}{0.121569,0.466667,0.705882}%
\pgfsetfillcolor{currentfill}%
\pgfsetlinewidth{1.003750pt}%
\definecolor{currentstroke}{rgb}{0.121569,0.466667,0.705882}%
\pgfsetstrokecolor{currentstroke}%
\pgfsetdash{}{0pt}%
\pgfpathmoveto{\pgfqpoint{2.161300in}{2.323625in}}%
\pgfpathcurveto{\pgfqpoint{2.172350in}{2.323625in}}{\pgfqpoint{2.182949in}{2.328016in}}{\pgfqpoint{2.190763in}{2.335829in}}%
\pgfpathcurveto{\pgfqpoint{2.198576in}{2.343643in}}{\pgfqpoint{2.202967in}{2.354242in}}{\pgfqpoint{2.202967in}{2.365292in}}%
\pgfpathcurveto{\pgfqpoint{2.202967in}{2.376342in}}{\pgfqpoint{2.198576in}{2.386941in}}{\pgfqpoint{2.190763in}{2.394755in}}%
\pgfpathcurveto{\pgfqpoint{2.182949in}{2.402568in}}{\pgfqpoint{2.172350in}{2.406959in}}{\pgfqpoint{2.161300in}{2.406959in}}%
\pgfpathcurveto{\pgfqpoint{2.150250in}{2.406959in}}{\pgfqpoint{2.139651in}{2.402568in}}{\pgfqpoint{2.131837in}{2.394755in}}%
\pgfpathcurveto{\pgfqpoint{2.124023in}{2.386941in}}{\pgfqpoint{2.119633in}{2.376342in}}{\pgfqpoint{2.119633in}{2.365292in}}%
\pgfpathcurveto{\pgfqpoint{2.119633in}{2.354242in}}{\pgfqpoint{2.124023in}{2.343643in}}{\pgfqpoint{2.131837in}{2.335829in}}%
\pgfpathcurveto{\pgfqpoint{2.139651in}{2.328016in}}{\pgfqpoint{2.150250in}{2.323625in}}{\pgfqpoint{2.161300in}{2.323625in}}%
\pgfpathclose%
\pgfusepath{stroke,fill}%
\end{pgfscope}%
\begin{pgfscope}%
\pgfpathrectangle{\pgfqpoint{0.600000in}{0.600000in}}{\pgfqpoint{3.900000in}{3.900000in}}%
\pgfusepath{clip}%
\pgfsetbuttcap%
\pgfsetroundjoin%
\definecolor{currentfill}{rgb}{0.121569,0.466667,0.705882}%
\pgfsetfillcolor{currentfill}%
\pgfsetlinewidth{1.003750pt}%
\definecolor{currentstroke}{rgb}{0.121569,0.466667,0.705882}%
\pgfsetstrokecolor{currentstroke}%
\pgfsetdash{}{0pt}%
\pgfpathmoveto{\pgfqpoint{3.010030in}{3.499855in}}%
\pgfpathcurveto{\pgfqpoint{3.021080in}{3.499855in}}{\pgfqpoint{3.031679in}{3.504246in}}{\pgfqpoint{3.039493in}{3.512059in}}%
\pgfpathcurveto{\pgfqpoint{3.047307in}{3.519873in}}{\pgfqpoint{3.051697in}{3.530472in}}{\pgfqpoint{3.051697in}{3.541522in}}%
\pgfpathcurveto{\pgfqpoint{3.051697in}{3.552572in}}{\pgfqpoint{3.047307in}{3.563171in}}{\pgfqpoint{3.039493in}{3.570985in}}%
\pgfpathcurveto{\pgfqpoint{3.031679in}{3.578799in}}{\pgfqpoint{3.021080in}{3.583189in}}{\pgfqpoint{3.010030in}{3.583189in}}%
\pgfpathcurveto{\pgfqpoint{2.998980in}{3.583189in}}{\pgfqpoint{2.988381in}{3.578799in}}{\pgfqpoint{2.980567in}{3.570985in}}%
\pgfpathcurveto{\pgfqpoint{2.972754in}{3.563171in}}{\pgfqpoint{2.968364in}{3.552572in}}{\pgfqpoint{2.968364in}{3.541522in}}%
\pgfpathcurveto{\pgfqpoint{2.968364in}{3.530472in}}{\pgfqpoint{2.972754in}{3.519873in}}{\pgfqpoint{2.980567in}{3.512059in}}%
\pgfpathcurveto{\pgfqpoint{2.988381in}{3.504246in}}{\pgfqpoint{2.998980in}{3.499855in}}{\pgfqpoint{3.010030in}{3.499855in}}%
\pgfpathclose%
\pgfusepath{stroke,fill}%
\end{pgfscope}%
\begin{pgfscope}%
\pgfpathrectangle{\pgfqpoint{0.600000in}{0.600000in}}{\pgfqpoint{3.900000in}{3.900000in}}%
\pgfusepath{clip}%
\pgfsetbuttcap%
\pgfsetroundjoin%
\definecolor{currentfill}{rgb}{0.121569,0.466667,0.705882}%
\pgfsetfillcolor{currentfill}%
\pgfsetlinewidth{1.003750pt}%
\definecolor{currentstroke}{rgb}{0.121569,0.466667,0.705882}%
\pgfsetstrokecolor{currentstroke}%
\pgfsetdash{}{0pt}%
\pgfpathmoveto{\pgfqpoint{2.202058in}{1.714573in}}%
\pgfpathcurveto{\pgfqpoint{2.213108in}{1.714573in}}{\pgfqpoint{2.223707in}{1.718963in}}{\pgfqpoint{2.231521in}{1.726777in}}%
\pgfpathcurveto{\pgfqpoint{2.239335in}{1.734590in}}{\pgfqpoint{2.243725in}{1.745190in}}{\pgfqpoint{2.243725in}{1.756240in}}%
\pgfpathcurveto{\pgfqpoint{2.243725in}{1.767290in}}{\pgfqpoint{2.239335in}{1.777889in}}{\pgfqpoint{2.231521in}{1.785702in}}%
\pgfpathcurveto{\pgfqpoint{2.223707in}{1.793516in}}{\pgfqpoint{2.213108in}{1.797906in}}{\pgfqpoint{2.202058in}{1.797906in}}%
\pgfpathcurveto{\pgfqpoint{2.191008in}{1.797906in}}{\pgfqpoint{2.180409in}{1.793516in}}{\pgfqpoint{2.172595in}{1.785702in}}%
\pgfpathcurveto{\pgfqpoint{2.164782in}{1.777889in}}{\pgfqpoint{2.160392in}{1.767290in}}{\pgfqpoint{2.160392in}{1.756240in}}%
\pgfpathcurveto{\pgfqpoint{2.160392in}{1.745190in}}{\pgfqpoint{2.164782in}{1.734590in}}{\pgfqpoint{2.172595in}{1.726777in}}%
\pgfpathcurveto{\pgfqpoint{2.180409in}{1.718963in}}{\pgfqpoint{2.191008in}{1.714573in}}{\pgfqpoint{2.202058in}{1.714573in}}%
\pgfpathclose%
\pgfusepath{stroke,fill}%
\end{pgfscope}%
\begin{pgfscope}%
\pgfpathrectangle{\pgfqpoint{0.600000in}{0.600000in}}{\pgfqpoint{3.900000in}{3.900000in}}%
\pgfusepath{clip}%
\pgfsetbuttcap%
\pgfsetroundjoin%
\definecolor{currentfill}{rgb}{0.121569,0.466667,0.705882}%
\pgfsetfillcolor{currentfill}%
\pgfsetlinewidth{1.003750pt}%
\definecolor{currentstroke}{rgb}{0.121569,0.466667,0.705882}%
\pgfsetstrokecolor{currentstroke}%
\pgfsetdash{}{0pt}%
\pgfpathmoveto{\pgfqpoint{2.461428in}{2.500742in}}%
\pgfpathcurveto{\pgfqpoint{2.472478in}{2.500742in}}{\pgfqpoint{2.483077in}{2.505132in}}{\pgfqpoint{2.490891in}{2.512946in}}%
\pgfpathcurveto{\pgfqpoint{2.498704in}{2.520759in}}{\pgfqpoint{2.503095in}{2.531358in}}{\pgfqpoint{2.503095in}{2.542408in}}%
\pgfpathcurveto{\pgfqpoint{2.503095in}{2.553458in}}{\pgfqpoint{2.498704in}{2.564058in}}{\pgfqpoint{2.490891in}{2.571871in}}%
\pgfpathcurveto{\pgfqpoint{2.483077in}{2.579685in}}{\pgfqpoint{2.472478in}{2.584075in}}{\pgfqpoint{2.461428in}{2.584075in}}%
\pgfpathcurveto{\pgfqpoint{2.450378in}{2.584075in}}{\pgfqpoint{2.439779in}{2.579685in}}{\pgfqpoint{2.431965in}{2.571871in}}%
\pgfpathcurveto{\pgfqpoint{2.424151in}{2.564058in}}{\pgfqpoint{2.419761in}{2.553458in}}{\pgfqpoint{2.419761in}{2.542408in}}%
\pgfpathcurveto{\pgfqpoint{2.419761in}{2.531358in}}{\pgfqpoint{2.424151in}{2.520759in}}{\pgfqpoint{2.431965in}{2.512946in}}%
\pgfpathcurveto{\pgfqpoint{2.439779in}{2.505132in}}{\pgfqpoint{2.450378in}{2.500742in}}{\pgfqpoint{2.461428in}{2.500742in}}%
\pgfpathclose%
\pgfusepath{stroke,fill}%
\end{pgfscope}%
\begin{pgfscope}%
\pgfpathrectangle{\pgfqpoint{0.600000in}{0.600000in}}{\pgfqpoint{3.900000in}{3.900000in}}%
\pgfusepath{clip}%
\pgfsetbuttcap%
\pgfsetroundjoin%
\definecolor{currentfill}{rgb}{0.121569,0.466667,0.705882}%
\pgfsetfillcolor{currentfill}%
\pgfsetlinewidth{1.003750pt}%
\definecolor{currentstroke}{rgb}{0.121569,0.466667,0.705882}%
\pgfsetstrokecolor{currentstroke}%
\pgfsetdash{}{0pt}%
\pgfpathmoveto{\pgfqpoint{2.320630in}{2.556859in}}%
\pgfpathcurveto{\pgfqpoint{2.331680in}{2.556859in}}{\pgfqpoint{2.342279in}{2.561250in}}{\pgfqpoint{2.350093in}{2.569063in}}%
\pgfpathcurveto{\pgfqpoint{2.357906in}{2.576877in}}{\pgfqpoint{2.362296in}{2.587476in}}{\pgfqpoint{2.362296in}{2.598526in}}%
\pgfpathcurveto{\pgfqpoint{2.362296in}{2.609576in}}{\pgfqpoint{2.357906in}{2.620175in}}{\pgfqpoint{2.350093in}{2.627989in}}%
\pgfpathcurveto{\pgfqpoint{2.342279in}{2.635802in}}{\pgfqpoint{2.331680in}{2.640193in}}{\pgfqpoint{2.320630in}{2.640193in}}%
\pgfpathcurveto{\pgfqpoint{2.309580in}{2.640193in}}{\pgfqpoint{2.298981in}{2.635802in}}{\pgfqpoint{2.291167in}{2.627989in}}%
\pgfpathcurveto{\pgfqpoint{2.283353in}{2.620175in}}{\pgfqpoint{2.278963in}{2.609576in}}{\pgfqpoint{2.278963in}{2.598526in}}%
\pgfpathcurveto{\pgfqpoint{2.278963in}{2.587476in}}{\pgfqpoint{2.283353in}{2.576877in}}{\pgfqpoint{2.291167in}{2.569063in}}%
\pgfpathcurveto{\pgfqpoint{2.298981in}{2.561250in}}{\pgfqpoint{2.309580in}{2.556859in}}{\pgfqpoint{2.320630in}{2.556859in}}%
\pgfpathclose%
\pgfusepath{stroke,fill}%
\end{pgfscope}%
\begin{pgfscope}%
\pgfpathrectangle{\pgfqpoint{0.600000in}{0.600000in}}{\pgfqpoint{3.900000in}{3.900000in}}%
\pgfusepath{clip}%
\pgfsetbuttcap%
\pgfsetroundjoin%
\definecolor{currentfill}{rgb}{0.121569,0.466667,0.705882}%
\pgfsetfillcolor{currentfill}%
\pgfsetlinewidth{1.003750pt}%
\definecolor{currentstroke}{rgb}{0.121569,0.466667,0.705882}%
\pgfsetstrokecolor{currentstroke}%
\pgfsetdash{}{0pt}%
\pgfpathmoveto{\pgfqpoint{2.026111in}{2.852312in}}%
\pgfpathcurveto{\pgfqpoint{2.037161in}{2.852312in}}{\pgfqpoint{2.047760in}{2.856702in}}{\pgfqpoint{2.055574in}{2.864515in}}%
\pgfpathcurveto{\pgfqpoint{2.063388in}{2.872329in}}{\pgfqpoint{2.067778in}{2.882928in}}{\pgfqpoint{2.067778in}{2.893978in}}%
\pgfpathcurveto{\pgfqpoint{2.067778in}{2.905028in}}{\pgfqpoint{2.063388in}{2.915627in}}{\pgfqpoint{2.055574in}{2.923441in}}%
\pgfpathcurveto{\pgfqpoint{2.047760in}{2.931255in}}{\pgfqpoint{2.037161in}{2.935645in}}{\pgfqpoint{2.026111in}{2.935645in}}%
\pgfpathcurveto{\pgfqpoint{2.015061in}{2.935645in}}{\pgfqpoint{2.004462in}{2.931255in}}{\pgfqpoint{1.996648in}{2.923441in}}%
\pgfpathcurveto{\pgfqpoint{1.988835in}{2.915627in}}{\pgfqpoint{1.984444in}{2.905028in}}{\pgfqpoint{1.984444in}{2.893978in}}%
\pgfpathcurveto{\pgfqpoint{1.984444in}{2.882928in}}{\pgfqpoint{1.988835in}{2.872329in}}{\pgfqpoint{1.996648in}{2.864515in}}%
\pgfpathcurveto{\pgfqpoint{2.004462in}{2.856702in}}{\pgfqpoint{2.015061in}{2.852312in}}{\pgfqpoint{2.026111in}{2.852312in}}%
\pgfpathclose%
\pgfusepath{stroke,fill}%
\end{pgfscope}%
\begin{pgfscope}%
\pgfpathrectangle{\pgfqpoint{0.600000in}{0.600000in}}{\pgfqpoint{3.900000in}{3.900000in}}%
\pgfusepath{clip}%
\pgfsetbuttcap%
\pgfsetroundjoin%
\definecolor{currentfill}{rgb}{0.121569,0.466667,0.705882}%
\pgfsetfillcolor{currentfill}%
\pgfsetlinewidth{1.003750pt}%
\definecolor{currentstroke}{rgb}{0.121569,0.466667,0.705882}%
\pgfsetstrokecolor{currentstroke}%
\pgfsetdash{}{0pt}%
\pgfpathmoveto{\pgfqpoint{2.836168in}{2.760702in}}%
\pgfpathcurveto{\pgfqpoint{2.847218in}{2.760702in}}{\pgfqpoint{2.857817in}{2.765092in}}{\pgfqpoint{2.865630in}{2.772906in}}%
\pgfpathcurveto{\pgfqpoint{2.873444in}{2.780720in}}{\pgfqpoint{2.877834in}{2.791319in}}{\pgfqpoint{2.877834in}{2.802369in}}%
\pgfpathcurveto{\pgfqpoint{2.877834in}{2.813419in}}{\pgfqpoint{2.873444in}{2.824018in}}{\pgfqpoint{2.865630in}{2.831832in}}%
\pgfpathcurveto{\pgfqpoint{2.857817in}{2.839645in}}{\pgfqpoint{2.847218in}{2.844035in}}{\pgfqpoint{2.836168in}{2.844035in}}%
\pgfpathcurveto{\pgfqpoint{2.825118in}{2.844035in}}{\pgfqpoint{2.814518in}{2.839645in}}{\pgfqpoint{2.806705in}{2.831832in}}%
\pgfpathcurveto{\pgfqpoint{2.798891in}{2.824018in}}{\pgfqpoint{2.794501in}{2.813419in}}{\pgfqpoint{2.794501in}{2.802369in}}%
\pgfpathcurveto{\pgfqpoint{2.794501in}{2.791319in}}{\pgfqpoint{2.798891in}{2.780720in}}{\pgfqpoint{2.806705in}{2.772906in}}%
\pgfpathcurveto{\pgfqpoint{2.814518in}{2.765092in}}{\pgfqpoint{2.825118in}{2.760702in}}{\pgfqpoint{2.836168in}{2.760702in}}%
\pgfpathclose%
\pgfusepath{stroke,fill}%
\end{pgfscope}%
\begin{pgfscope}%
\pgfpathrectangle{\pgfqpoint{0.600000in}{0.600000in}}{\pgfqpoint{3.900000in}{3.900000in}}%
\pgfusepath{clip}%
\pgfsetbuttcap%
\pgfsetroundjoin%
\definecolor{currentfill}{rgb}{0.121569,0.466667,0.705882}%
\pgfsetfillcolor{currentfill}%
\pgfsetlinewidth{1.003750pt}%
\definecolor{currentstroke}{rgb}{0.121569,0.466667,0.705882}%
\pgfsetstrokecolor{currentstroke}%
\pgfsetdash{}{0pt}%
\pgfpathmoveto{\pgfqpoint{2.233640in}{1.552418in}}%
\pgfpathcurveto{\pgfqpoint{2.244690in}{1.552418in}}{\pgfqpoint{2.255289in}{1.556808in}}{\pgfqpoint{2.263103in}{1.564622in}}%
\pgfpathcurveto{\pgfqpoint{2.270916in}{1.572435in}}{\pgfqpoint{2.275306in}{1.583035in}}{\pgfqpoint{2.275306in}{1.594085in}}%
\pgfpathcurveto{\pgfqpoint{2.275306in}{1.605135in}}{\pgfqpoint{2.270916in}{1.615734in}}{\pgfqpoint{2.263103in}{1.623547in}}%
\pgfpathcurveto{\pgfqpoint{2.255289in}{1.631361in}}{\pgfqpoint{2.244690in}{1.635751in}}{\pgfqpoint{2.233640in}{1.635751in}}%
\pgfpathcurveto{\pgfqpoint{2.222590in}{1.635751in}}{\pgfqpoint{2.211991in}{1.631361in}}{\pgfqpoint{2.204177in}{1.623547in}}%
\pgfpathcurveto{\pgfqpoint{2.196363in}{1.615734in}}{\pgfqpoint{2.191973in}{1.605135in}}{\pgfqpoint{2.191973in}{1.594085in}}%
\pgfpathcurveto{\pgfqpoint{2.191973in}{1.583035in}}{\pgfqpoint{2.196363in}{1.572435in}}{\pgfqpoint{2.204177in}{1.564622in}}%
\pgfpathcurveto{\pgfqpoint{2.211991in}{1.556808in}}{\pgfqpoint{2.222590in}{1.552418in}}{\pgfqpoint{2.233640in}{1.552418in}}%
\pgfpathclose%
\pgfusepath{stroke,fill}%
\end{pgfscope}%
\begin{pgfscope}%
\pgfpathrectangle{\pgfqpoint{0.600000in}{0.600000in}}{\pgfqpoint{3.900000in}{3.900000in}}%
\pgfusepath{clip}%
\pgfsetbuttcap%
\pgfsetroundjoin%
\definecolor{currentfill}{rgb}{0.121569,0.466667,0.705882}%
\pgfsetfillcolor{currentfill}%
\pgfsetlinewidth{1.003750pt}%
\definecolor{currentstroke}{rgb}{0.121569,0.466667,0.705882}%
\pgfsetstrokecolor{currentstroke}%
\pgfsetdash{}{0pt}%
\pgfpathmoveto{\pgfqpoint{1.723063in}{2.349471in}}%
\pgfpathcurveto{\pgfqpoint{1.734113in}{2.349471in}}{\pgfqpoint{1.744712in}{2.353861in}}{\pgfqpoint{1.752526in}{2.361675in}}%
\pgfpathcurveto{\pgfqpoint{1.760339in}{2.369489in}}{\pgfqpoint{1.764729in}{2.380088in}}{\pgfqpoint{1.764729in}{2.391138in}}%
\pgfpathcurveto{\pgfqpoint{1.764729in}{2.402188in}}{\pgfqpoint{1.760339in}{2.412787in}}{\pgfqpoint{1.752526in}{2.420601in}}%
\pgfpathcurveto{\pgfqpoint{1.744712in}{2.428414in}}{\pgfqpoint{1.734113in}{2.432804in}}{\pgfqpoint{1.723063in}{2.432804in}}%
\pgfpathcurveto{\pgfqpoint{1.712013in}{2.432804in}}{\pgfqpoint{1.701414in}{2.428414in}}{\pgfqpoint{1.693600in}{2.420601in}}%
\pgfpathcurveto{\pgfqpoint{1.685786in}{2.412787in}}{\pgfqpoint{1.681396in}{2.402188in}}{\pgfqpoint{1.681396in}{2.391138in}}%
\pgfpathcurveto{\pgfqpoint{1.681396in}{2.380088in}}{\pgfqpoint{1.685786in}{2.369489in}}{\pgfqpoint{1.693600in}{2.361675in}}%
\pgfpathcurveto{\pgfqpoint{1.701414in}{2.353861in}}{\pgfqpoint{1.712013in}{2.349471in}}{\pgfqpoint{1.723063in}{2.349471in}}%
\pgfpathclose%
\pgfusepath{stroke,fill}%
\end{pgfscope}%
\begin{pgfscope}%
\pgfpathrectangle{\pgfqpoint{0.600000in}{0.600000in}}{\pgfqpoint{3.900000in}{3.900000in}}%
\pgfusepath{clip}%
\pgfsetbuttcap%
\pgfsetroundjoin%
\definecolor{currentfill}{rgb}{0.121569,0.466667,0.705882}%
\pgfsetfillcolor{currentfill}%
\pgfsetlinewidth{1.003750pt}%
\definecolor{currentstroke}{rgb}{0.121569,0.466667,0.705882}%
\pgfsetstrokecolor{currentstroke}%
\pgfsetdash{}{0pt}%
\pgfpathmoveto{\pgfqpoint{3.259834in}{3.470231in}}%
\pgfpathcurveto{\pgfqpoint{3.270884in}{3.470231in}}{\pgfqpoint{3.281483in}{3.474621in}}{\pgfqpoint{3.289296in}{3.482435in}}%
\pgfpathcurveto{\pgfqpoint{3.297110in}{3.490248in}}{\pgfqpoint{3.301500in}{3.500847in}}{\pgfqpoint{3.301500in}{3.511897in}}%
\pgfpathcurveto{\pgfqpoint{3.301500in}{3.522947in}}{\pgfqpoint{3.297110in}{3.533547in}}{\pgfqpoint{3.289296in}{3.541360in}}%
\pgfpathcurveto{\pgfqpoint{3.281483in}{3.549174in}}{\pgfqpoint{3.270884in}{3.553564in}}{\pgfqpoint{3.259834in}{3.553564in}}%
\pgfpathcurveto{\pgfqpoint{3.248784in}{3.553564in}}{\pgfqpoint{3.238184in}{3.549174in}}{\pgfqpoint{3.230371in}{3.541360in}}%
\pgfpathcurveto{\pgfqpoint{3.222557in}{3.533547in}}{\pgfqpoint{3.218167in}{3.522947in}}{\pgfqpoint{3.218167in}{3.511897in}}%
\pgfpathcurveto{\pgfqpoint{3.218167in}{3.500847in}}{\pgfqpoint{3.222557in}{3.490248in}}{\pgfqpoint{3.230371in}{3.482435in}}%
\pgfpathcurveto{\pgfqpoint{3.238184in}{3.474621in}}{\pgfqpoint{3.248784in}{3.470231in}}{\pgfqpoint{3.259834in}{3.470231in}}%
\pgfpathclose%
\pgfusepath{stroke,fill}%
\end{pgfscope}%
\begin{pgfscope}%
\pgfpathrectangle{\pgfqpoint{0.600000in}{0.600000in}}{\pgfqpoint{3.900000in}{3.900000in}}%
\pgfusepath{clip}%
\pgfsetbuttcap%
\pgfsetroundjoin%
\definecolor{currentfill}{rgb}{0.121569,0.466667,0.705882}%
\pgfsetfillcolor{currentfill}%
\pgfsetlinewidth{1.003750pt}%
\definecolor{currentstroke}{rgb}{0.121569,0.466667,0.705882}%
\pgfsetstrokecolor{currentstroke}%
\pgfsetdash{}{0pt}%
\pgfpathmoveto{\pgfqpoint{3.143935in}{2.852551in}}%
\pgfpathcurveto{\pgfqpoint{3.154985in}{2.852551in}}{\pgfqpoint{3.165584in}{2.856942in}}{\pgfqpoint{3.173398in}{2.864755in}}%
\pgfpathcurveto{\pgfqpoint{3.181211in}{2.872569in}}{\pgfqpoint{3.185601in}{2.883168in}}{\pgfqpoint{3.185601in}{2.894218in}}%
\pgfpathcurveto{\pgfqpoint{3.185601in}{2.905268in}}{\pgfqpoint{3.181211in}{2.915867in}}{\pgfqpoint{3.173398in}{2.923681in}}%
\pgfpathcurveto{\pgfqpoint{3.165584in}{2.931494in}}{\pgfqpoint{3.154985in}{2.935885in}}{\pgfqpoint{3.143935in}{2.935885in}}%
\pgfpathcurveto{\pgfqpoint{3.132885in}{2.935885in}}{\pgfqpoint{3.122286in}{2.931494in}}{\pgfqpoint{3.114472in}{2.923681in}}%
\pgfpathcurveto{\pgfqpoint{3.106658in}{2.915867in}}{\pgfqpoint{3.102268in}{2.905268in}}{\pgfqpoint{3.102268in}{2.894218in}}%
\pgfpathcurveto{\pgfqpoint{3.102268in}{2.883168in}}{\pgfqpoint{3.106658in}{2.872569in}}{\pgfqpoint{3.114472in}{2.864755in}}%
\pgfpathcurveto{\pgfqpoint{3.122286in}{2.856942in}}{\pgfqpoint{3.132885in}{2.852551in}}{\pgfqpoint{3.143935in}{2.852551in}}%
\pgfpathclose%
\pgfusepath{stroke,fill}%
\end{pgfscope}%
\begin{pgfscope}%
\pgfpathrectangle{\pgfqpoint{0.600000in}{0.600000in}}{\pgfqpoint{3.900000in}{3.900000in}}%
\pgfusepath{clip}%
\pgfsetbuttcap%
\pgfsetroundjoin%
\definecolor{currentfill}{rgb}{0.121569,0.466667,0.705882}%
\pgfsetfillcolor{currentfill}%
\pgfsetlinewidth{1.003750pt}%
\definecolor{currentstroke}{rgb}{0.121569,0.466667,0.705882}%
\pgfsetstrokecolor{currentstroke}%
\pgfsetdash{}{0pt}%
\pgfpathmoveto{\pgfqpoint{3.033060in}{2.557731in}}%
\pgfpathcurveto{\pgfqpoint{3.044110in}{2.557731in}}{\pgfqpoint{3.054709in}{2.562121in}}{\pgfqpoint{3.062523in}{2.569935in}}%
\pgfpathcurveto{\pgfqpoint{3.070337in}{2.577749in}}{\pgfqpoint{3.074727in}{2.588348in}}{\pgfqpoint{3.074727in}{2.599398in}}%
\pgfpathcurveto{\pgfqpoint{3.074727in}{2.610448in}}{\pgfqpoint{3.070337in}{2.621047in}}{\pgfqpoint{3.062523in}{2.628861in}}%
\pgfpathcurveto{\pgfqpoint{3.054709in}{2.636674in}}{\pgfqpoint{3.044110in}{2.641065in}}{\pgfqpoint{3.033060in}{2.641065in}}%
\pgfpathcurveto{\pgfqpoint{3.022010in}{2.641065in}}{\pgfqpoint{3.011411in}{2.636674in}}{\pgfqpoint{3.003597in}{2.628861in}}%
\pgfpathcurveto{\pgfqpoint{2.995784in}{2.621047in}}{\pgfqpoint{2.991394in}{2.610448in}}{\pgfqpoint{2.991394in}{2.599398in}}%
\pgfpathcurveto{\pgfqpoint{2.991394in}{2.588348in}}{\pgfqpoint{2.995784in}{2.577749in}}{\pgfqpoint{3.003597in}{2.569935in}}%
\pgfpathcurveto{\pgfqpoint{3.011411in}{2.562121in}}{\pgfqpoint{3.022010in}{2.557731in}}{\pgfqpoint{3.033060in}{2.557731in}}%
\pgfpathclose%
\pgfusepath{stroke,fill}%
\end{pgfscope}%
\begin{pgfscope}%
\pgfpathrectangle{\pgfqpoint{0.600000in}{0.600000in}}{\pgfqpoint{3.900000in}{3.900000in}}%
\pgfusepath{clip}%
\pgfsetbuttcap%
\pgfsetroundjoin%
\definecolor{currentfill}{rgb}{0.121569,0.466667,0.705882}%
\pgfsetfillcolor{currentfill}%
\pgfsetlinewidth{1.003750pt}%
\definecolor{currentstroke}{rgb}{0.121569,0.466667,0.705882}%
\pgfsetstrokecolor{currentstroke}%
\pgfsetdash{}{0pt}%
\pgfpathmoveto{\pgfqpoint{1.790370in}{2.090129in}}%
\pgfpathcurveto{\pgfqpoint{1.801420in}{2.090129in}}{\pgfqpoint{1.812019in}{2.094519in}}{\pgfqpoint{1.819833in}{2.102333in}}%
\pgfpathcurveto{\pgfqpoint{1.827646in}{2.110146in}}{\pgfqpoint{1.832037in}{2.120745in}}{\pgfqpoint{1.832037in}{2.131795in}}%
\pgfpathcurveto{\pgfqpoint{1.832037in}{2.142846in}}{\pgfqpoint{1.827646in}{2.153445in}}{\pgfqpoint{1.819833in}{2.161258in}}%
\pgfpathcurveto{\pgfqpoint{1.812019in}{2.169072in}}{\pgfqpoint{1.801420in}{2.173462in}}{\pgfqpoint{1.790370in}{2.173462in}}%
\pgfpathcurveto{\pgfqpoint{1.779320in}{2.173462in}}{\pgfqpoint{1.768721in}{2.169072in}}{\pgfqpoint{1.760907in}{2.161258in}}%
\pgfpathcurveto{\pgfqpoint{1.753094in}{2.153445in}}{\pgfqpoint{1.748703in}{2.142846in}}{\pgfqpoint{1.748703in}{2.131795in}}%
\pgfpathcurveto{\pgfqpoint{1.748703in}{2.120745in}}{\pgfqpoint{1.753094in}{2.110146in}}{\pgfqpoint{1.760907in}{2.102333in}}%
\pgfpathcurveto{\pgfqpoint{1.768721in}{2.094519in}}{\pgfqpoint{1.779320in}{2.090129in}}{\pgfqpoint{1.790370in}{2.090129in}}%
\pgfpathclose%
\pgfusepath{stroke,fill}%
\end{pgfscope}%
\begin{pgfscope}%
\pgfpathrectangle{\pgfqpoint{0.600000in}{0.600000in}}{\pgfqpoint{3.900000in}{3.900000in}}%
\pgfusepath{clip}%
\pgfsetbuttcap%
\pgfsetroundjoin%
\definecolor{currentfill}{rgb}{0.121569,0.466667,0.705882}%
\pgfsetfillcolor{currentfill}%
\pgfsetlinewidth{1.003750pt}%
\definecolor{currentstroke}{rgb}{0.121569,0.466667,0.705882}%
\pgfsetstrokecolor{currentstroke}%
\pgfsetdash{}{0pt}%
\pgfpathmoveto{\pgfqpoint{2.460006in}{3.484937in}}%
\pgfpathcurveto{\pgfqpoint{2.471056in}{3.484937in}}{\pgfqpoint{2.481655in}{3.489327in}}{\pgfqpoint{2.489469in}{3.497141in}}%
\pgfpathcurveto{\pgfqpoint{2.497282in}{3.504954in}}{\pgfqpoint{2.501672in}{3.515553in}}{\pgfqpoint{2.501672in}{3.526603in}}%
\pgfpathcurveto{\pgfqpoint{2.501672in}{3.537653in}}{\pgfqpoint{2.497282in}{3.548252in}}{\pgfqpoint{2.489469in}{3.556066in}}%
\pgfpathcurveto{\pgfqpoint{2.481655in}{3.563880in}}{\pgfqpoint{2.471056in}{3.568270in}}{\pgfqpoint{2.460006in}{3.568270in}}%
\pgfpathcurveto{\pgfqpoint{2.448956in}{3.568270in}}{\pgfqpoint{2.438357in}{3.563880in}}{\pgfqpoint{2.430543in}{3.556066in}}%
\pgfpathcurveto{\pgfqpoint{2.422729in}{3.548252in}}{\pgfqpoint{2.418339in}{3.537653in}}{\pgfqpoint{2.418339in}{3.526603in}}%
\pgfpathcurveto{\pgfqpoint{2.418339in}{3.515553in}}{\pgfqpoint{2.422729in}{3.504954in}}{\pgfqpoint{2.430543in}{3.497141in}}%
\pgfpathcurveto{\pgfqpoint{2.438357in}{3.489327in}}{\pgfqpoint{2.448956in}{3.484937in}}{\pgfqpoint{2.460006in}{3.484937in}}%
\pgfpathclose%
\pgfusepath{stroke,fill}%
\end{pgfscope}%
\begin{pgfscope}%
\pgfpathrectangle{\pgfqpoint{0.600000in}{0.600000in}}{\pgfqpoint{3.900000in}{3.900000in}}%
\pgfusepath{clip}%
\pgfsetbuttcap%
\pgfsetroundjoin%
\definecolor{currentfill}{rgb}{0.121569,0.466667,0.705882}%
\pgfsetfillcolor{currentfill}%
\pgfsetlinewidth{1.003750pt}%
\definecolor{currentstroke}{rgb}{0.121569,0.466667,0.705882}%
\pgfsetstrokecolor{currentstroke}%
\pgfsetdash{}{0pt}%
\pgfpathmoveto{\pgfqpoint{3.250416in}{3.436632in}}%
\pgfpathcurveto{\pgfqpoint{3.261467in}{3.436632in}}{\pgfqpoint{3.272066in}{3.441023in}}{\pgfqpoint{3.279879in}{3.448836in}}%
\pgfpathcurveto{\pgfqpoint{3.287693in}{3.456650in}}{\pgfqpoint{3.292083in}{3.467249in}}{\pgfqpoint{3.292083in}{3.478299in}}%
\pgfpathcurveto{\pgfqpoint{3.292083in}{3.489349in}}{\pgfqpoint{3.287693in}{3.499948in}}{\pgfqpoint{3.279879in}{3.507762in}}%
\pgfpathcurveto{\pgfqpoint{3.272066in}{3.515575in}}{\pgfqpoint{3.261467in}{3.519966in}}{\pgfqpoint{3.250416in}{3.519966in}}%
\pgfpathcurveto{\pgfqpoint{3.239366in}{3.519966in}}{\pgfqpoint{3.228767in}{3.515575in}}{\pgfqpoint{3.220954in}{3.507762in}}%
\pgfpathcurveto{\pgfqpoint{3.213140in}{3.499948in}}{\pgfqpoint{3.208750in}{3.489349in}}{\pgfqpoint{3.208750in}{3.478299in}}%
\pgfpathcurveto{\pgfqpoint{3.208750in}{3.467249in}}{\pgfqpoint{3.213140in}{3.456650in}}{\pgfqpoint{3.220954in}{3.448836in}}%
\pgfpathcurveto{\pgfqpoint{3.228767in}{3.441023in}}{\pgfqpoint{3.239366in}{3.436632in}}{\pgfqpoint{3.250416in}{3.436632in}}%
\pgfpathclose%
\pgfusepath{stroke,fill}%
\end{pgfscope}%
\begin{pgfscope}%
\pgfpathrectangle{\pgfqpoint{0.600000in}{0.600000in}}{\pgfqpoint{3.900000in}{3.900000in}}%
\pgfusepath{clip}%
\pgfsetbuttcap%
\pgfsetroundjoin%
\definecolor{currentfill}{rgb}{0.121569,0.466667,0.705882}%
\pgfsetfillcolor{currentfill}%
\pgfsetlinewidth{1.003750pt}%
\definecolor{currentstroke}{rgb}{0.121569,0.466667,0.705882}%
\pgfsetstrokecolor{currentstroke}%
\pgfsetdash{}{0pt}%
\pgfpathmoveto{\pgfqpoint{3.252642in}{3.397385in}}%
\pgfpathcurveto{\pgfqpoint{3.263692in}{3.397385in}}{\pgfqpoint{3.274291in}{3.401775in}}{\pgfqpoint{3.282104in}{3.409588in}}%
\pgfpathcurveto{\pgfqpoint{3.289918in}{3.417402in}}{\pgfqpoint{3.294308in}{3.428001in}}{\pgfqpoint{3.294308in}{3.439051in}}%
\pgfpathcurveto{\pgfqpoint{3.294308in}{3.450101in}}{\pgfqpoint{3.289918in}{3.460700in}}{\pgfqpoint{3.282104in}{3.468514in}}%
\pgfpathcurveto{\pgfqpoint{3.274291in}{3.476328in}}{\pgfqpoint{3.263692in}{3.480718in}}{\pgfqpoint{3.252642in}{3.480718in}}%
\pgfpathcurveto{\pgfqpoint{3.241591in}{3.480718in}}{\pgfqpoint{3.230992in}{3.476328in}}{\pgfqpoint{3.223179in}{3.468514in}}%
\pgfpathcurveto{\pgfqpoint{3.215365in}{3.460700in}}{\pgfqpoint{3.210975in}{3.450101in}}{\pgfqpoint{3.210975in}{3.439051in}}%
\pgfpathcurveto{\pgfqpoint{3.210975in}{3.428001in}}{\pgfqpoint{3.215365in}{3.417402in}}{\pgfqpoint{3.223179in}{3.409588in}}%
\pgfpathcurveto{\pgfqpoint{3.230992in}{3.401775in}}{\pgfqpoint{3.241591in}{3.397385in}}{\pgfqpoint{3.252642in}{3.397385in}}%
\pgfpathclose%
\pgfusepath{stroke,fill}%
\end{pgfscope}%
\begin{pgfscope}%
\pgfpathrectangle{\pgfqpoint{0.600000in}{0.600000in}}{\pgfqpoint{3.900000in}{3.900000in}}%
\pgfusepath{clip}%
\pgfsetbuttcap%
\pgfsetroundjoin%
\definecolor{currentfill}{rgb}{0.121569,0.466667,0.705882}%
\pgfsetfillcolor{currentfill}%
\pgfsetlinewidth{1.003750pt}%
\definecolor{currentstroke}{rgb}{0.121569,0.466667,0.705882}%
\pgfsetstrokecolor{currentstroke}%
\pgfsetdash{}{0pt}%
\pgfpathmoveto{\pgfqpoint{2.439410in}{1.885492in}}%
\pgfpathcurveto{\pgfqpoint{2.450460in}{1.885492in}}{\pgfqpoint{2.461059in}{1.889882in}}{\pgfqpoint{2.468872in}{1.897696in}}%
\pgfpathcurveto{\pgfqpoint{2.476686in}{1.905509in}}{\pgfqpoint{2.481076in}{1.916108in}}{\pgfqpoint{2.481076in}{1.927158in}}%
\pgfpathcurveto{\pgfqpoint{2.481076in}{1.938208in}}{\pgfqpoint{2.476686in}{1.948808in}}{\pgfqpoint{2.468872in}{1.956621in}}%
\pgfpathcurveto{\pgfqpoint{2.461059in}{1.964435in}}{\pgfqpoint{2.450460in}{1.968825in}}{\pgfqpoint{2.439410in}{1.968825in}}%
\pgfpathcurveto{\pgfqpoint{2.428360in}{1.968825in}}{\pgfqpoint{2.417761in}{1.964435in}}{\pgfqpoint{2.409947in}{1.956621in}}%
\pgfpathcurveto{\pgfqpoint{2.402133in}{1.948808in}}{\pgfqpoint{2.397743in}{1.938208in}}{\pgfqpoint{2.397743in}{1.927158in}}%
\pgfpathcurveto{\pgfqpoint{2.397743in}{1.916108in}}{\pgfqpoint{2.402133in}{1.905509in}}{\pgfqpoint{2.409947in}{1.897696in}}%
\pgfpathcurveto{\pgfqpoint{2.417761in}{1.889882in}}{\pgfqpoint{2.428360in}{1.885492in}}{\pgfqpoint{2.439410in}{1.885492in}}%
\pgfpathclose%
\pgfusepath{stroke,fill}%
\end{pgfscope}%
\begin{pgfscope}%
\pgfpathrectangle{\pgfqpoint{0.600000in}{0.600000in}}{\pgfqpoint{3.900000in}{3.900000in}}%
\pgfusepath{clip}%
\pgfsetbuttcap%
\pgfsetroundjoin%
\definecolor{currentfill}{rgb}{0.121569,0.466667,0.705882}%
\pgfsetfillcolor{currentfill}%
\pgfsetlinewidth{1.003750pt}%
\definecolor{currentstroke}{rgb}{0.121569,0.466667,0.705882}%
\pgfsetstrokecolor{currentstroke}%
\pgfsetdash{}{0pt}%
\pgfpathmoveto{\pgfqpoint{1.967321in}{2.284921in}}%
\pgfpathcurveto{\pgfqpoint{1.978371in}{2.284921in}}{\pgfqpoint{1.988970in}{2.289311in}}{\pgfqpoint{1.996783in}{2.297125in}}%
\pgfpathcurveto{\pgfqpoint{2.004597in}{2.304938in}}{\pgfqpoint{2.008987in}{2.315537in}}{\pgfqpoint{2.008987in}{2.326587in}}%
\pgfpathcurveto{\pgfqpoint{2.008987in}{2.337638in}}{\pgfqpoint{2.004597in}{2.348237in}}{\pgfqpoint{1.996783in}{2.356050in}}%
\pgfpathcurveto{\pgfqpoint{1.988970in}{2.363864in}}{\pgfqpoint{1.978371in}{2.368254in}}{\pgfqpoint{1.967321in}{2.368254in}}%
\pgfpathcurveto{\pgfqpoint{1.956270in}{2.368254in}}{\pgfqpoint{1.945671in}{2.363864in}}{\pgfqpoint{1.937858in}{2.356050in}}%
\pgfpathcurveto{\pgfqpoint{1.930044in}{2.348237in}}{\pgfqpoint{1.925654in}{2.337638in}}{\pgfqpoint{1.925654in}{2.326587in}}%
\pgfpathcurveto{\pgfqpoint{1.925654in}{2.315537in}}{\pgfqpoint{1.930044in}{2.304938in}}{\pgfqpoint{1.937858in}{2.297125in}}%
\pgfpathcurveto{\pgfqpoint{1.945671in}{2.289311in}}{\pgfqpoint{1.956270in}{2.284921in}}{\pgfqpoint{1.967321in}{2.284921in}}%
\pgfpathclose%
\pgfusepath{stroke,fill}%
\end{pgfscope}%
\begin{pgfscope}%
\pgfpathrectangle{\pgfqpoint{0.600000in}{0.600000in}}{\pgfqpoint{3.900000in}{3.900000in}}%
\pgfusepath{clip}%
\pgfsetbuttcap%
\pgfsetroundjoin%
\definecolor{currentfill}{rgb}{0.121569,0.466667,0.705882}%
\pgfsetfillcolor{currentfill}%
\pgfsetlinewidth{1.003750pt}%
\definecolor{currentstroke}{rgb}{0.121569,0.466667,0.705882}%
\pgfsetstrokecolor{currentstroke}%
\pgfsetdash{}{0pt}%
\pgfpathmoveto{\pgfqpoint{3.080612in}{2.803894in}}%
\pgfpathcurveto{\pgfqpoint{3.091662in}{2.803894in}}{\pgfqpoint{3.102261in}{2.808284in}}{\pgfqpoint{3.110075in}{2.816097in}}%
\pgfpathcurveto{\pgfqpoint{3.117888in}{2.823911in}}{\pgfqpoint{3.122279in}{2.834510in}}{\pgfqpoint{3.122279in}{2.845560in}}%
\pgfpathcurveto{\pgfqpoint{3.122279in}{2.856610in}}{\pgfqpoint{3.117888in}{2.867209in}}{\pgfqpoint{3.110075in}{2.875023in}}%
\pgfpathcurveto{\pgfqpoint{3.102261in}{2.882837in}}{\pgfqpoint{3.091662in}{2.887227in}}{\pgfqpoint{3.080612in}{2.887227in}}%
\pgfpathcurveto{\pgfqpoint{3.069562in}{2.887227in}}{\pgfqpoint{3.058963in}{2.882837in}}{\pgfqpoint{3.051149in}{2.875023in}}%
\pgfpathcurveto{\pgfqpoint{3.043336in}{2.867209in}}{\pgfqpoint{3.038945in}{2.856610in}}{\pgfqpoint{3.038945in}{2.845560in}}%
\pgfpathcurveto{\pgfqpoint{3.038945in}{2.834510in}}{\pgfqpoint{3.043336in}{2.823911in}}{\pgfqpoint{3.051149in}{2.816097in}}%
\pgfpathcurveto{\pgfqpoint{3.058963in}{2.808284in}}{\pgfqpoint{3.069562in}{2.803894in}}{\pgfqpoint{3.080612in}{2.803894in}}%
\pgfpathclose%
\pgfusepath{stroke,fill}%
\end{pgfscope}%
\begin{pgfscope}%
\pgfpathrectangle{\pgfqpoint{0.600000in}{0.600000in}}{\pgfqpoint{3.900000in}{3.900000in}}%
\pgfusepath{clip}%
\pgfsetbuttcap%
\pgfsetroundjoin%
\definecolor{currentfill}{rgb}{0.121569,0.466667,0.705882}%
\pgfsetfillcolor{currentfill}%
\pgfsetlinewidth{1.003750pt}%
\definecolor{currentstroke}{rgb}{0.121569,0.466667,0.705882}%
\pgfsetstrokecolor{currentstroke}%
\pgfsetdash{}{0pt}%
\pgfpathmoveto{\pgfqpoint{2.324480in}{3.278245in}}%
\pgfpathcurveto{\pgfqpoint{2.335530in}{3.278245in}}{\pgfqpoint{2.346129in}{3.282635in}}{\pgfqpoint{2.353943in}{3.290449in}}%
\pgfpathcurveto{\pgfqpoint{2.361757in}{3.298263in}}{\pgfqpoint{2.366147in}{3.308862in}}{\pgfqpoint{2.366147in}{3.319912in}}%
\pgfpathcurveto{\pgfqpoint{2.366147in}{3.330962in}}{\pgfqpoint{2.361757in}{3.341561in}}{\pgfqpoint{2.353943in}{3.349375in}}%
\pgfpathcurveto{\pgfqpoint{2.346129in}{3.357188in}}{\pgfqpoint{2.335530in}{3.361579in}}{\pgfqpoint{2.324480in}{3.361579in}}%
\pgfpathcurveto{\pgfqpoint{2.313430in}{3.361579in}}{\pgfqpoint{2.302831in}{3.357188in}}{\pgfqpoint{2.295017in}{3.349375in}}%
\pgfpathcurveto{\pgfqpoint{2.287204in}{3.341561in}}{\pgfqpoint{2.282814in}{3.330962in}}{\pgfqpoint{2.282814in}{3.319912in}}%
\pgfpathcurveto{\pgfqpoint{2.282814in}{3.308862in}}{\pgfqpoint{2.287204in}{3.298263in}}{\pgfqpoint{2.295017in}{3.290449in}}%
\pgfpathcurveto{\pgfqpoint{2.302831in}{3.282635in}}{\pgfqpoint{2.313430in}{3.278245in}}{\pgfqpoint{2.324480in}{3.278245in}}%
\pgfpathclose%
\pgfusepath{stroke,fill}%
\end{pgfscope}%
\begin{pgfscope}%
\pgfpathrectangle{\pgfqpoint{0.600000in}{0.600000in}}{\pgfqpoint{3.900000in}{3.900000in}}%
\pgfusepath{clip}%
\pgfsetbuttcap%
\pgfsetroundjoin%
\definecolor{currentfill}{rgb}{0.121569,0.466667,0.705882}%
\pgfsetfillcolor{currentfill}%
\pgfsetlinewidth{1.003750pt}%
\definecolor{currentstroke}{rgb}{0.121569,0.466667,0.705882}%
\pgfsetstrokecolor{currentstroke}%
\pgfsetdash{}{0pt}%
\pgfpathmoveto{\pgfqpoint{3.207079in}{2.894982in}}%
\pgfpathcurveto{\pgfqpoint{3.218129in}{2.894982in}}{\pgfqpoint{3.228728in}{2.899372in}}{\pgfqpoint{3.236542in}{2.907186in}}%
\pgfpathcurveto{\pgfqpoint{3.244356in}{2.914999in}}{\pgfqpoint{3.248746in}{2.925598in}}{\pgfqpoint{3.248746in}{2.936648in}}%
\pgfpathcurveto{\pgfqpoint{3.248746in}{2.947699in}}{\pgfqpoint{3.244356in}{2.958298in}}{\pgfqpoint{3.236542in}{2.966111in}}%
\pgfpathcurveto{\pgfqpoint{3.228728in}{2.973925in}}{\pgfqpoint{3.218129in}{2.978315in}}{\pgfqpoint{3.207079in}{2.978315in}}%
\pgfpathcurveto{\pgfqpoint{3.196029in}{2.978315in}}{\pgfqpoint{3.185430in}{2.973925in}}{\pgfqpoint{3.177617in}{2.966111in}}%
\pgfpathcurveto{\pgfqpoint{3.169803in}{2.958298in}}{\pgfqpoint{3.165413in}{2.947699in}}{\pgfqpoint{3.165413in}{2.936648in}}%
\pgfpathcurveto{\pgfqpoint{3.165413in}{2.925598in}}{\pgfqpoint{3.169803in}{2.914999in}}{\pgfqpoint{3.177617in}{2.907186in}}%
\pgfpathcurveto{\pgfqpoint{3.185430in}{2.899372in}}{\pgfqpoint{3.196029in}{2.894982in}}{\pgfqpoint{3.207079in}{2.894982in}}%
\pgfpathclose%
\pgfusepath{stroke,fill}%
\end{pgfscope}%
\begin{pgfscope}%
\pgfpathrectangle{\pgfqpoint{0.600000in}{0.600000in}}{\pgfqpoint{3.900000in}{3.900000in}}%
\pgfusepath{clip}%
\pgfsetbuttcap%
\pgfsetroundjoin%
\definecolor{currentfill}{rgb}{0.121569,0.466667,0.705882}%
\pgfsetfillcolor{currentfill}%
\pgfsetlinewidth{1.003750pt}%
\definecolor{currentstroke}{rgb}{0.121569,0.466667,0.705882}%
\pgfsetstrokecolor{currentstroke}%
\pgfsetdash{}{0pt}%
\pgfpathmoveto{\pgfqpoint{2.874918in}{3.562698in}}%
\pgfpathcurveto{\pgfqpoint{2.885968in}{3.562698in}}{\pgfqpoint{2.896567in}{3.567089in}}{\pgfqpoint{2.904381in}{3.574902in}}%
\pgfpathcurveto{\pgfqpoint{2.912195in}{3.582716in}}{\pgfqpoint{2.916585in}{3.593315in}}{\pgfqpoint{2.916585in}{3.604365in}}%
\pgfpathcurveto{\pgfqpoint{2.916585in}{3.615415in}}{\pgfqpoint{2.912195in}{3.626014in}}{\pgfqpoint{2.904381in}{3.633828in}}%
\pgfpathcurveto{\pgfqpoint{2.896567in}{3.641641in}}{\pgfqpoint{2.885968in}{3.646032in}}{\pgfqpoint{2.874918in}{3.646032in}}%
\pgfpathcurveto{\pgfqpoint{2.863868in}{3.646032in}}{\pgfqpoint{2.853269in}{3.641641in}}{\pgfqpoint{2.845456in}{3.633828in}}%
\pgfpathcurveto{\pgfqpoint{2.837642in}{3.626014in}}{\pgfqpoint{2.833252in}{3.615415in}}{\pgfqpoint{2.833252in}{3.604365in}}%
\pgfpathcurveto{\pgfqpoint{2.833252in}{3.593315in}}{\pgfqpoint{2.837642in}{3.582716in}}{\pgfqpoint{2.845456in}{3.574902in}}%
\pgfpathcurveto{\pgfqpoint{2.853269in}{3.567089in}}{\pgfqpoint{2.863868in}{3.562698in}}{\pgfqpoint{2.874918in}{3.562698in}}%
\pgfpathclose%
\pgfusepath{stroke,fill}%
\end{pgfscope}%
\begin{pgfscope}%
\pgfpathrectangle{\pgfqpoint{0.600000in}{0.600000in}}{\pgfqpoint{3.900000in}{3.900000in}}%
\pgfusepath{clip}%
\pgfsetbuttcap%
\pgfsetroundjoin%
\definecolor{currentfill}{rgb}{0.121569,0.466667,0.705882}%
\pgfsetfillcolor{currentfill}%
\pgfsetlinewidth{1.003750pt}%
\definecolor{currentstroke}{rgb}{0.121569,0.466667,0.705882}%
\pgfsetstrokecolor{currentstroke}%
\pgfsetdash{}{0pt}%
\pgfpathmoveto{\pgfqpoint{3.052913in}{3.467953in}}%
\pgfpathcurveto{\pgfqpoint{3.063964in}{3.467953in}}{\pgfqpoint{3.074563in}{3.472343in}}{\pgfqpoint{3.082376in}{3.480157in}}%
\pgfpathcurveto{\pgfqpoint{3.090190in}{3.487971in}}{\pgfqpoint{3.094580in}{3.498570in}}{\pgfqpoint{3.094580in}{3.509620in}}%
\pgfpathcurveto{\pgfqpoint{3.094580in}{3.520670in}}{\pgfqpoint{3.090190in}{3.531269in}}{\pgfqpoint{3.082376in}{3.539083in}}%
\pgfpathcurveto{\pgfqpoint{3.074563in}{3.546896in}}{\pgfqpoint{3.063964in}{3.551287in}}{\pgfqpoint{3.052913in}{3.551287in}}%
\pgfpathcurveto{\pgfqpoint{3.041863in}{3.551287in}}{\pgfqpoint{3.031264in}{3.546896in}}{\pgfqpoint{3.023451in}{3.539083in}}%
\pgfpathcurveto{\pgfqpoint{3.015637in}{3.531269in}}{\pgfqpoint{3.011247in}{3.520670in}}{\pgfqpoint{3.011247in}{3.509620in}}%
\pgfpathcurveto{\pgfqpoint{3.011247in}{3.498570in}}{\pgfqpoint{3.015637in}{3.487971in}}{\pgfqpoint{3.023451in}{3.480157in}}%
\pgfpathcurveto{\pgfqpoint{3.031264in}{3.472343in}}{\pgfqpoint{3.041863in}{3.467953in}}{\pgfqpoint{3.052913in}{3.467953in}}%
\pgfpathclose%
\pgfusepath{stroke,fill}%
\end{pgfscope}%
\begin{pgfscope}%
\pgfpathrectangle{\pgfqpoint{0.600000in}{0.600000in}}{\pgfqpoint{3.900000in}{3.900000in}}%
\pgfusepath{clip}%
\pgfsetbuttcap%
\pgfsetroundjoin%
\definecolor{currentfill}{rgb}{0.121569,0.466667,0.705882}%
\pgfsetfillcolor{currentfill}%
\pgfsetlinewidth{1.003750pt}%
\definecolor{currentstroke}{rgb}{0.121569,0.466667,0.705882}%
\pgfsetstrokecolor{currentstroke}%
\pgfsetdash{}{0pt}%
\pgfpathmoveto{\pgfqpoint{2.692791in}{2.235696in}}%
\pgfpathcurveto{\pgfqpoint{2.703841in}{2.235696in}}{\pgfqpoint{2.714440in}{2.240087in}}{\pgfqpoint{2.722254in}{2.247900in}}%
\pgfpathcurveto{\pgfqpoint{2.730067in}{2.255714in}}{\pgfqpoint{2.734458in}{2.266313in}}{\pgfqpoint{2.734458in}{2.277363in}}%
\pgfpathcurveto{\pgfqpoint{2.734458in}{2.288413in}}{\pgfqpoint{2.730067in}{2.299012in}}{\pgfqpoint{2.722254in}{2.306826in}}%
\pgfpathcurveto{\pgfqpoint{2.714440in}{2.314639in}}{\pgfqpoint{2.703841in}{2.319030in}}{\pgfqpoint{2.692791in}{2.319030in}}%
\pgfpathcurveto{\pgfqpoint{2.681741in}{2.319030in}}{\pgfqpoint{2.671142in}{2.314639in}}{\pgfqpoint{2.663328in}{2.306826in}}%
\pgfpathcurveto{\pgfqpoint{2.655515in}{2.299012in}}{\pgfqpoint{2.651124in}{2.288413in}}{\pgfqpoint{2.651124in}{2.277363in}}%
\pgfpathcurveto{\pgfqpoint{2.651124in}{2.266313in}}{\pgfqpoint{2.655515in}{2.255714in}}{\pgfqpoint{2.663328in}{2.247900in}}%
\pgfpathcurveto{\pgfqpoint{2.671142in}{2.240087in}}{\pgfqpoint{2.681741in}{2.235696in}}{\pgfqpoint{2.692791in}{2.235696in}}%
\pgfpathclose%
\pgfusepath{stroke,fill}%
\end{pgfscope}%
\begin{pgfscope}%
\pgfpathrectangle{\pgfqpoint{0.600000in}{0.600000in}}{\pgfqpoint{3.900000in}{3.900000in}}%
\pgfusepath{clip}%
\pgfsetbuttcap%
\pgfsetroundjoin%
\definecolor{currentfill}{rgb}{0.121569,0.466667,0.705882}%
\pgfsetfillcolor{currentfill}%
\pgfsetlinewidth{1.003750pt}%
\definecolor{currentstroke}{rgb}{0.121569,0.466667,0.705882}%
\pgfsetstrokecolor{currentstroke}%
\pgfsetdash{}{0pt}%
\pgfpathmoveto{\pgfqpoint{2.352941in}{2.463353in}}%
\pgfpathcurveto{\pgfqpoint{2.363991in}{2.463353in}}{\pgfqpoint{2.374590in}{2.467743in}}{\pgfqpoint{2.382403in}{2.475556in}}%
\pgfpathcurveto{\pgfqpoint{2.390217in}{2.483370in}}{\pgfqpoint{2.394607in}{2.493969in}}{\pgfqpoint{2.394607in}{2.505019in}}%
\pgfpathcurveto{\pgfqpoint{2.394607in}{2.516069in}}{\pgfqpoint{2.390217in}{2.526668in}}{\pgfqpoint{2.382403in}{2.534482in}}%
\pgfpathcurveto{\pgfqpoint{2.374590in}{2.542296in}}{\pgfqpoint{2.363991in}{2.546686in}}{\pgfqpoint{2.352941in}{2.546686in}}%
\pgfpathcurveto{\pgfqpoint{2.341890in}{2.546686in}}{\pgfqpoint{2.331291in}{2.542296in}}{\pgfqpoint{2.323478in}{2.534482in}}%
\pgfpathcurveto{\pgfqpoint{2.315664in}{2.526668in}}{\pgfqpoint{2.311274in}{2.516069in}}{\pgfqpoint{2.311274in}{2.505019in}}%
\pgfpathcurveto{\pgfqpoint{2.311274in}{2.493969in}}{\pgfqpoint{2.315664in}{2.483370in}}{\pgfqpoint{2.323478in}{2.475556in}}%
\pgfpathcurveto{\pgfqpoint{2.331291in}{2.467743in}}{\pgfqpoint{2.341890in}{2.463353in}}{\pgfqpoint{2.352941in}{2.463353in}}%
\pgfpathclose%
\pgfusepath{stroke,fill}%
\end{pgfscope}%
\begin{pgfscope}%
\pgfpathrectangle{\pgfqpoint{0.600000in}{0.600000in}}{\pgfqpoint{3.900000in}{3.900000in}}%
\pgfusepath{clip}%
\pgfsetbuttcap%
\pgfsetroundjoin%
\definecolor{currentfill}{rgb}{0.121569,0.466667,0.705882}%
\pgfsetfillcolor{currentfill}%
\pgfsetlinewidth{1.003750pt}%
\definecolor{currentstroke}{rgb}{0.121569,0.466667,0.705882}%
\pgfsetstrokecolor{currentstroke}%
\pgfsetdash{}{0pt}%
\pgfpathmoveto{\pgfqpoint{2.359020in}{1.908425in}}%
\pgfpathcurveto{\pgfqpoint{2.370070in}{1.908425in}}{\pgfqpoint{2.380669in}{1.912815in}}{\pgfqpoint{2.388482in}{1.920629in}}%
\pgfpathcurveto{\pgfqpoint{2.396296in}{1.928443in}}{\pgfqpoint{2.400686in}{1.939042in}}{\pgfqpoint{2.400686in}{1.950092in}}%
\pgfpathcurveto{\pgfqpoint{2.400686in}{1.961142in}}{\pgfqpoint{2.396296in}{1.971741in}}{\pgfqpoint{2.388482in}{1.979555in}}%
\pgfpathcurveto{\pgfqpoint{2.380669in}{1.987368in}}{\pgfqpoint{2.370070in}{1.991758in}}{\pgfqpoint{2.359020in}{1.991758in}}%
\pgfpathcurveto{\pgfqpoint{2.347970in}{1.991758in}}{\pgfqpoint{2.337371in}{1.987368in}}{\pgfqpoint{2.329557in}{1.979555in}}%
\pgfpathcurveto{\pgfqpoint{2.321743in}{1.971741in}}{\pgfqpoint{2.317353in}{1.961142in}}{\pgfqpoint{2.317353in}{1.950092in}}%
\pgfpathcurveto{\pgfqpoint{2.317353in}{1.939042in}}{\pgfqpoint{2.321743in}{1.928443in}}{\pgfqpoint{2.329557in}{1.920629in}}%
\pgfpathcurveto{\pgfqpoint{2.337371in}{1.912815in}}{\pgfqpoint{2.347970in}{1.908425in}}{\pgfqpoint{2.359020in}{1.908425in}}%
\pgfpathclose%
\pgfusepath{stroke,fill}%
\end{pgfscope}%
\begin{pgfscope}%
\pgfpathrectangle{\pgfqpoint{0.600000in}{0.600000in}}{\pgfqpoint{3.900000in}{3.900000in}}%
\pgfusepath{clip}%
\pgfsetbuttcap%
\pgfsetroundjoin%
\definecolor{currentfill}{rgb}{0.121569,0.466667,0.705882}%
\pgfsetfillcolor{currentfill}%
\pgfsetlinewidth{1.003750pt}%
\definecolor{currentstroke}{rgb}{0.121569,0.466667,0.705882}%
\pgfsetstrokecolor{currentstroke}%
\pgfsetdash{}{0pt}%
\pgfpathmoveto{\pgfqpoint{2.616314in}{1.915163in}}%
\pgfpathcurveto{\pgfqpoint{2.627364in}{1.915163in}}{\pgfqpoint{2.637963in}{1.919553in}}{\pgfqpoint{2.645776in}{1.927367in}}%
\pgfpathcurveto{\pgfqpoint{2.653590in}{1.935181in}}{\pgfqpoint{2.657980in}{1.945780in}}{\pgfqpoint{2.657980in}{1.956830in}}%
\pgfpathcurveto{\pgfqpoint{2.657980in}{1.967880in}}{\pgfqpoint{2.653590in}{1.978479in}}{\pgfqpoint{2.645776in}{1.986292in}}%
\pgfpathcurveto{\pgfqpoint{2.637963in}{1.994106in}}{\pgfqpoint{2.627364in}{1.998496in}}{\pgfqpoint{2.616314in}{1.998496in}}%
\pgfpathcurveto{\pgfqpoint{2.605264in}{1.998496in}}{\pgfqpoint{2.594665in}{1.994106in}}{\pgfqpoint{2.586851in}{1.986292in}}%
\pgfpathcurveto{\pgfqpoint{2.579037in}{1.978479in}}{\pgfqpoint{2.574647in}{1.967880in}}{\pgfqpoint{2.574647in}{1.956830in}}%
\pgfpathcurveto{\pgfqpoint{2.574647in}{1.945780in}}{\pgfqpoint{2.579037in}{1.935181in}}{\pgfqpoint{2.586851in}{1.927367in}}%
\pgfpathcurveto{\pgfqpoint{2.594665in}{1.919553in}}{\pgfqpoint{2.605264in}{1.915163in}}{\pgfqpoint{2.616314in}{1.915163in}}%
\pgfpathclose%
\pgfusepath{stroke,fill}%
\end{pgfscope}%
\begin{pgfscope}%
\pgfpathrectangle{\pgfqpoint{0.600000in}{0.600000in}}{\pgfqpoint{3.900000in}{3.900000in}}%
\pgfusepath{clip}%
\pgfsetbuttcap%
\pgfsetroundjoin%
\definecolor{currentfill}{rgb}{0.121569,0.466667,0.705882}%
\pgfsetfillcolor{currentfill}%
\pgfsetlinewidth{1.003750pt}%
\definecolor{currentstroke}{rgb}{0.121569,0.466667,0.705882}%
\pgfsetstrokecolor{currentstroke}%
\pgfsetdash{}{0pt}%
\pgfpathmoveto{\pgfqpoint{2.457303in}{2.529873in}}%
\pgfpathcurveto{\pgfqpoint{2.468353in}{2.529873in}}{\pgfqpoint{2.478952in}{2.534263in}}{\pgfqpoint{2.486766in}{2.542077in}}%
\pgfpathcurveto{\pgfqpoint{2.494580in}{2.549891in}}{\pgfqpoint{2.498970in}{2.560490in}}{\pgfqpoint{2.498970in}{2.571540in}}%
\pgfpathcurveto{\pgfqpoint{2.498970in}{2.582590in}}{\pgfqpoint{2.494580in}{2.593189in}}{\pgfqpoint{2.486766in}{2.601003in}}%
\pgfpathcurveto{\pgfqpoint{2.478952in}{2.608816in}}{\pgfqpoint{2.468353in}{2.613206in}}{\pgfqpoint{2.457303in}{2.613206in}}%
\pgfpathcurveto{\pgfqpoint{2.446253in}{2.613206in}}{\pgfqpoint{2.435654in}{2.608816in}}{\pgfqpoint{2.427840in}{2.601003in}}%
\pgfpathcurveto{\pgfqpoint{2.420027in}{2.593189in}}{\pgfqpoint{2.415636in}{2.582590in}}{\pgfqpoint{2.415636in}{2.571540in}}%
\pgfpathcurveto{\pgfqpoint{2.415636in}{2.560490in}}{\pgfqpoint{2.420027in}{2.549891in}}{\pgfqpoint{2.427840in}{2.542077in}}%
\pgfpathcurveto{\pgfqpoint{2.435654in}{2.534263in}}{\pgfqpoint{2.446253in}{2.529873in}}{\pgfqpoint{2.457303in}{2.529873in}}%
\pgfpathclose%
\pgfusepath{stroke,fill}%
\end{pgfscope}%
\begin{pgfscope}%
\pgfpathrectangle{\pgfqpoint{0.600000in}{0.600000in}}{\pgfqpoint{3.900000in}{3.900000in}}%
\pgfusepath{clip}%
\pgfsetbuttcap%
\pgfsetroundjoin%
\definecolor{currentfill}{rgb}{0.121569,0.466667,0.705882}%
\pgfsetfillcolor{currentfill}%
\pgfsetlinewidth{1.003750pt}%
\definecolor{currentstroke}{rgb}{0.121569,0.466667,0.705882}%
\pgfsetstrokecolor{currentstroke}%
\pgfsetdash{}{0pt}%
\pgfpathmoveto{\pgfqpoint{2.904828in}{2.217902in}}%
\pgfpathcurveto{\pgfqpoint{2.915878in}{2.217902in}}{\pgfqpoint{2.926477in}{2.222293in}}{\pgfqpoint{2.934290in}{2.230106in}}%
\pgfpathcurveto{\pgfqpoint{2.942104in}{2.237920in}}{\pgfqpoint{2.946494in}{2.248519in}}{\pgfqpoint{2.946494in}{2.259569in}}%
\pgfpathcurveto{\pgfqpoint{2.946494in}{2.270619in}}{\pgfqpoint{2.942104in}{2.281218in}}{\pgfqpoint{2.934290in}{2.289032in}}%
\pgfpathcurveto{\pgfqpoint{2.926477in}{2.296845in}}{\pgfqpoint{2.915878in}{2.301236in}}{\pgfqpoint{2.904828in}{2.301236in}}%
\pgfpathcurveto{\pgfqpoint{2.893778in}{2.301236in}}{\pgfqpoint{2.883179in}{2.296845in}}{\pgfqpoint{2.875365in}{2.289032in}}%
\pgfpathcurveto{\pgfqpoint{2.867551in}{2.281218in}}{\pgfqpoint{2.863161in}{2.270619in}}{\pgfqpoint{2.863161in}{2.259569in}}%
\pgfpathcurveto{\pgfqpoint{2.863161in}{2.248519in}}{\pgfqpoint{2.867551in}{2.237920in}}{\pgfqpoint{2.875365in}{2.230106in}}%
\pgfpathcurveto{\pgfqpoint{2.883179in}{2.222293in}}{\pgfqpoint{2.893778in}{2.217902in}}{\pgfqpoint{2.904828in}{2.217902in}}%
\pgfpathclose%
\pgfusepath{stroke,fill}%
\end{pgfscope}%
\begin{pgfscope}%
\pgfpathrectangle{\pgfqpoint{0.600000in}{0.600000in}}{\pgfqpoint{3.900000in}{3.900000in}}%
\pgfusepath{clip}%
\pgfsetbuttcap%
\pgfsetroundjoin%
\definecolor{currentfill}{rgb}{0.121569,0.466667,0.705882}%
\pgfsetfillcolor{currentfill}%
\pgfsetlinewidth{1.003750pt}%
\definecolor{currentstroke}{rgb}{0.121569,0.466667,0.705882}%
\pgfsetstrokecolor{currentstroke}%
\pgfsetdash{}{0pt}%
\pgfpathmoveto{\pgfqpoint{2.555341in}{2.219715in}}%
\pgfpathcurveto{\pgfqpoint{2.566391in}{2.219715in}}{\pgfqpoint{2.576990in}{2.224105in}}{\pgfqpoint{2.584803in}{2.231919in}}%
\pgfpathcurveto{\pgfqpoint{2.592617in}{2.239733in}}{\pgfqpoint{2.597007in}{2.250332in}}{\pgfqpoint{2.597007in}{2.261382in}}%
\pgfpathcurveto{\pgfqpoint{2.597007in}{2.272432in}}{\pgfqpoint{2.592617in}{2.283031in}}{\pgfqpoint{2.584803in}{2.290845in}}%
\pgfpathcurveto{\pgfqpoint{2.576990in}{2.298658in}}{\pgfqpoint{2.566391in}{2.303049in}}{\pgfqpoint{2.555341in}{2.303049in}}%
\pgfpathcurveto{\pgfqpoint{2.544291in}{2.303049in}}{\pgfqpoint{2.533692in}{2.298658in}}{\pgfqpoint{2.525878in}{2.290845in}}%
\pgfpathcurveto{\pgfqpoint{2.518064in}{2.283031in}}{\pgfqpoint{2.513674in}{2.272432in}}{\pgfqpoint{2.513674in}{2.261382in}}%
\pgfpathcurveto{\pgfqpoint{2.513674in}{2.250332in}}{\pgfqpoint{2.518064in}{2.239733in}}{\pgfqpoint{2.525878in}{2.231919in}}%
\pgfpathcurveto{\pgfqpoint{2.533692in}{2.224105in}}{\pgfqpoint{2.544291in}{2.219715in}}{\pgfqpoint{2.555341in}{2.219715in}}%
\pgfpathclose%
\pgfusepath{stroke,fill}%
\end{pgfscope}%
\begin{pgfscope}%
\pgfpathrectangle{\pgfqpoint{0.600000in}{0.600000in}}{\pgfqpoint{3.900000in}{3.900000in}}%
\pgfusepath{clip}%
\pgfsetbuttcap%
\pgfsetroundjoin%
\definecolor{currentfill}{rgb}{0.121569,0.466667,0.705882}%
\pgfsetfillcolor{currentfill}%
\pgfsetlinewidth{1.003750pt}%
\definecolor{currentstroke}{rgb}{0.121569,0.466667,0.705882}%
\pgfsetstrokecolor{currentstroke}%
\pgfsetdash{}{0pt}%
\pgfpathmoveto{\pgfqpoint{2.907895in}{2.726428in}}%
\pgfpathcurveto{\pgfqpoint{2.918946in}{2.726428in}}{\pgfqpoint{2.929545in}{2.730818in}}{\pgfqpoint{2.937358in}{2.738631in}}%
\pgfpathcurveto{\pgfqpoint{2.945172in}{2.746445in}}{\pgfqpoint{2.949562in}{2.757044in}}{\pgfqpoint{2.949562in}{2.768094in}}%
\pgfpathcurveto{\pgfqpoint{2.949562in}{2.779144in}}{\pgfqpoint{2.945172in}{2.789743in}}{\pgfqpoint{2.937358in}{2.797557in}}%
\pgfpathcurveto{\pgfqpoint{2.929545in}{2.805371in}}{\pgfqpoint{2.918946in}{2.809761in}}{\pgfqpoint{2.907895in}{2.809761in}}%
\pgfpathcurveto{\pgfqpoint{2.896845in}{2.809761in}}{\pgfqpoint{2.886246in}{2.805371in}}{\pgfqpoint{2.878433in}{2.797557in}}%
\pgfpathcurveto{\pgfqpoint{2.870619in}{2.789743in}}{\pgfqpoint{2.866229in}{2.779144in}}{\pgfqpoint{2.866229in}{2.768094in}}%
\pgfpathcurveto{\pgfqpoint{2.866229in}{2.757044in}}{\pgfqpoint{2.870619in}{2.746445in}}{\pgfqpoint{2.878433in}{2.738631in}}%
\pgfpathcurveto{\pgfqpoint{2.886246in}{2.730818in}}{\pgfqpoint{2.896845in}{2.726428in}}{\pgfqpoint{2.907895in}{2.726428in}}%
\pgfpathclose%
\pgfusepath{stroke,fill}%
\end{pgfscope}%
\begin{pgfscope}%
\pgfpathrectangle{\pgfqpoint{0.600000in}{0.600000in}}{\pgfqpoint{3.900000in}{3.900000in}}%
\pgfusepath{clip}%
\pgfsetbuttcap%
\pgfsetroundjoin%
\definecolor{currentfill}{rgb}{0.121569,0.466667,0.705882}%
\pgfsetfillcolor{currentfill}%
\pgfsetlinewidth{1.003750pt}%
\definecolor{currentstroke}{rgb}{0.121569,0.466667,0.705882}%
\pgfsetstrokecolor{currentstroke}%
\pgfsetdash{}{0pt}%
\pgfpathmoveto{\pgfqpoint{2.307513in}{2.752203in}}%
\pgfpathcurveto{\pgfqpoint{2.318563in}{2.752203in}}{\pgfqpoint{2.329162in}{2.756594in}}{\pgfqpoint{2.336975in}{2.764407in}}%
\pgfpathcurveto{\pgfqpoint{2.344789in}{2.772221in}}{\pgfqpoint{2.349179in}{2.782820in}}{\pgfqpoint{2.349179in}{2.793870in}}%
\pgfpathcurveto{\pgfqpoint{2.349179in}{2.804920in}}{\pgfqpoint{2.344789in}{2.815519in}}{\pgfqpoint{2.336975in}{2.823333in}}%
\pgfpathcurveto{\pgfqpoint{2.329162in}{2.831146in}}{\pgfqpoint{2.318563in}{2.835537in}}{\pgfqpoint{2.307513in}{2.835537in}}%
\pgfpathcurveto{\pgfqpoint{2.296463in}{2.835537in}}{\pgfqpoint{2.285863in}{2.831146in}}{\pgfqpoint{2.278050in}{2.823333in}}%
\pgfpathcurveto{\pgfqpoint{2.270236in}{2.815519in}}{\pgfqpoint{2.265846in}{2.804920in}}{\pgfqpoint{2.265846in}{2.793870in}}%
\pgfpathcurveto{\pgfqpoint{2.265846in}{2.782820in}}{\pgfqpoint{2.270236in}{2.772221in}}{\pgfqpoint{2.278050in}{2.764407in}}%
\pgfpathcurveto{\pgfqpoint{2.285863in}{2.756594in}}{\pgfqpoint{2.296463in}{2.752203in}}{\pgfqpoint{2.307513in}{2.752203in}}%
\pgfpathclose%
\pgfusepath{stroke,fill}%
\end{pgfscope}%
\begin{pgfscope}%
\pgfpathrectangle{\pgfqpoint{0.600000in}{0.600000in}}{\pgfqpoint{3.900000in}{3.900000in}}%
\pgfusepath{clip}%
\pgfsetbuttcap%
\pgfsetroundjoin%
\definecolor{currentfill}{rgb}{0.121569,0.466667,0.705882}%
\pgfsetfillcolor{currentfill}%
\pgfsetlinewidth{1.003750pt}%
\definecolor{currentstroke}{rgb}{0.121569,0.466667,0.705882}%
\pgfsetstrokecolor{currentstroke}%
\pgfsetdash{}{0pt}%
\pgfpathmoveto{\pgfqpoint{2.129718in}{2.993732in}}%
\pgfpathcurveto{\pgfqpoint{2.140768in}{2.993732in}}{\pgfqpoint{2.151367in}{2.998122in}}{\pgfqpoint{2.159181in}{3.005936in}}%
\pgfpathcurveto{\pgfqpoint{2.166995in}{3.013750in}}{\pgfqpoint{2.171385in}{3.024349in}}{\pgfqpoint{2.171385in}{3.035399in}}%
\pgfpathcurveto{\pgfqpoint{2.171385in}{3.046449in}}{\pgfqpoint{2.166995in}{3.057048in}}{\pgfqpoint{2.159181in}{3.064862in}}%
\pgfpathcurveto{\pgfqpoint{2.151367in}{3.072675in}}{\pgfqpoint{2.140768in}{3.077065in}}{\pgfqpoint{2.129718in}{3.077065in}}%
\pgfpathcurveto{\pgfqpoint{2.118668in}{3.077065in}}{\pgfqpoint{2.108069in}{3.072675in}}{\pgfqpoint{2.100256in}{3.064862in}}%
\pgfpathcurveto{\pgfqpoint{2.092442in}{3.057048in}}{\pgfqpoint{2.088052in}{3.046449in}}{\pgfqpoint{2.088052in}{3.035399in}}%
\pgfpathcurveto{\pgfqpoint{2.088052in}{3.024349in}}{\pgfqpoint{2.092442in}{3.013750in}}{\pgfqpoint{2.100256in}{3.005936in}}%
\pgfpathcurveto{\pgfqpoint{2.108069in}{2.998122in}}{\pgfqpoint{2.118668in}{2.993732in}}{\pgfqpoint{2.129718in}{2.993732in}}%
\pgfpathclose%
\pgfusepath{stroke,fill}%
\end{pgfscope}%
\begin{pgfscope}%
\pgfpathrectangle{\pgfqpoint{0.600000in}{0.600000in}}{\pgfqpoint{3.900000in}{3.900000in}}%
\pgfusepath{clip}%
\pgfsetbuttcap%
\pgfsetroundjoin%
\definecolor{currentfill}{rgb}{0.121569,0.466667,0.705882}%
\pgfsetfillcolor{currentfill}%
\pgfsetlinewidth{1.003750pt}%
\definecolor{currentstroke}{rgb}{0.121569,0.466667,0.705882}%
\pgfsetstrokecolor{currentstroke}%
\pgfsetdash{}{0pt}%
\pgfpathmoveto{\pgfqpoint{2.250716in}{2.299112in}}%
\pgfpathcurveto{\pgfqpoint{2.261767in}{2.299112in}}{\pgfqpoint{2.272366in}{2.303502in}}{\pgfqpoint{2.280179in}{2.311315in}}%
\pgfpathcurveto{\pgfqpoint{2.287993in}{2.319129in}}{\pgfqpoint{2.292383in}{2.329728in}}{\pgfqpoint{2.292383in}{2.340778in}}%
\pgfpathcurveto{\pgfqpoint{2.292383in}{2.351828in}}{\pgfqpoint{2.287993in}{2.362427in}}{\pgfqpoint{2.280179in}{2.370241in}}%
\pgfpathcurveto{\pgfqpoint{2.272366in}{2.378055in}}{\pgfqpoint{2.261767in}{2.382445in}}{\pgfqpoint{2.250716in}{2.382445in}}%
\pgfpathcurveto{\pgfqpoint{2.239666in}{2.382445in}}{\pgfqpoint{2.229067in}{2.378055in}}{\pgfqpoint{2.221254in}{2.370241in}}%
\pgfpathcurveto{\pgfqpoint{2.213440in}{2.362427in}}{\pgfqpoint{2.209050in}{2.351828in}}{\pgfqpoint{2.209050in}{2.340778in}}%
\pgfpathcurveto{\pgfqpoint{2.209050in}{2.329728in}}{\pgfqpoint{2.213440in}{2.319129in}}{\pgfqpoint{2.221254in}{2.311315in}}%
\pgfpathcurveto{\pgfqpoint{2.229067in}{2.303502in}}{\pgfqpoint{2.239666in}{2.299112in}}{\pgfqpoint{2.250716in}{2.299112in}}%
\pgfpathclose%
\pgfusepath{stroke,fill}%
\end{pgfscope}%
\begin{pgfscope}%
\pgfpathrectangle{\pgfqpoint{0.600000in}{0.600000in}}{\pgfqpoint{3.900000in}{3.900000in}}%
\pgfusepath{clip}%
\pgfsetbuttcap%
\pgfsetroundjoin%
\definecolor{currentfill}{rgb}{0.121569,0.466667,0.705882}%
\pgfsetfillcolor{currentfill}%
\pgfsetlinewidth{1.003750pt}%
\definecolor{currentstroke}{rgb}{0.121569,0.466667,0.705882}%
\pgfsetstrokecolor{currentstroke}%
\pgfsetdash{}{0pt}%
\pgfpathmoveto{\pgfqpoint{3.434750in}{3.356113in}}%
\pgfpathcurveto{\pgfqpoint{3.445800in}{3.356113in}}{\pgfqpoint{3.456399in}{3.360503in}}{\pgfqpoint{3.464212in}{3.368317in}}%
\pgfpathcurveto{\pgfqpoint{3.472026in}{3.376131in}}{\pgfqpoint{3.476416in}{3.386730in}}{\pgfqpoint{3.476416in}{3.397780in}}%
\pgfpathcurveto{\pgfqpoint{3.476416in}{3.408830in}}{\pgfqpoint{3.472026in}{3.419429in}}{\pgfqpoint{3.464212in}{3.427243in}}%
\pgfpathcurveto{\pgfqpoint{3.456399in}{3.435056in}}{\pgfqpoint{3.445800in}{3.439446in}}{\pgfqpoint{3.434750in}{3.439446in}}%
\pgfpathcurveto{\pgfqpoint{3.423699in}{3.439446in}}{\pgfqpoint{3.413100in}{3.435056in}}{\pgfqpoint{3.405287in}{3.427243in}}%
\pgfpathcurveto{\pgfqpoint{3.397473in}{3.419429in}}{\pgfqpoint{3.393083in}{3.408830in}}{\pgfqpoint{3.393083in}{3.397780in}}%
\pgfpathcurveto{\pgfqpoint{3.393083in}{3.386730in}}{\pgfqpoint{3.397473in}{3.376131in}}{\pgfqpoint{3.405287in}{3.368317in}}%
\pgfpathcurveto{\pgfqpoint{3.413100in}{3.360503in}}{\pgfqpoint{3.423699in}{3.356113in}}{\pgfqpoint{3.434750in}{3.356113in}}%
\pgfpathclose%
\pgfusepath{stroke,fill}%
\end{pgfscope}%
\begin{pgfscope}%
\pgfpathrectangle{\pgfqpoint{0.600000in}{0.600000in}}{\pgfqpoint{3.900000in}{3.900000in}}%
\pgfusepath{clip}%
\pgfsetbuttcap%
\pgfsetroundjoin%
\definecolor{currentfill}{rgb}{0.121569,0.466667,0.705882}%
\pgfsetfillcolor{currentfill}%
\pgfsetlinewidth{1.003750pt}%
\definecolor{currentstroke}{rgb}{0.121569,0.466667,0.705882}%
\pgfsetstrokecolor{currentstroke}%
\pgfsetdash{}{0pt}%
\pgfpathmoveto{\pgfqpoint{2.243228in}{2.049270in}}%
\pgfpathcurveto{\pgfqpoint{2.254278in}{2.049270in}}{\pgfqpoint{2.264877in}{2.053660in}}{\pgfqpoint{2.272691in}{2.061474in}}%
\pgfpathcurveto{\pgfqpoint{2.280504in}{2.069288in}}{\pgfqpoint{2.284895in}{2.079887in}}{\pgfqpoint{2.284895in}{2.090937in}}%
\pgfpathcurveto{\pgfqpoint{2.284895in}{2.101987in}}{\pgfqpoint{2.280504in}{2.112586in}}{\pgfqpoint{2.272691in}{2.120400in}}%
\pgfpathcurveto{\pgfqpoint{2.264877in}{2.128213in}}{\pgfqpoint{2.254278in}{2.132604in}}{\pgfqpoint{2.243228in}{2.132604in}}%
\pgfpathcurveto{\pgfqpoint{2.232178in}{2.132604in}}{\pgfqpoint{2.221579in}{2.128213in}}{\pgfqpoint{2.213765in}{2.120400in}}%
\pgfpathcurveto{\pgfqpoint{2.205951in}{2.112586in}}{\pgfqpoint{2.201561in}{2.101987in}}{\pgfqpoint{2.201561in}{2.090937in}}%
\pgfpathcurveto{\pgfqpoint{2.201561in}{2.079887in}}{\pgfqpoint{2.205951in}{2.069288in}}{\pgfqpoint{2.213765in}{2.061474in}}%
\pgfpathcurveto{\pgfqpoint{2.221579in}{2.053660in}}{\pgfqpoint{2.232178in}{2.049270in}}{\pgfqpoint{2.243228in}{2.049270in}}%
\pgfpathclose%
\pgfusepath{stroke,fill}%
\end{pgfscope}%
\begin{pgfscope}%
\pgfpathrectangle{\pgfqpoint{0.600000in}{0.600000in}}{\pgfqpoint{3.900000in}{3.900000in}}%
\pgfusepath{clip}%
\pgfsetbuttcap%
\pgfsetroundjoin%
\definecolor{currentfill}{rgb}{0.121569,0.466667,0.705882}%
\pgfsetfillcolor{currentfill}%
\pgfsetlinewidth{1.003750pt}%
\definecolor{currentstroke}{rgb}{0.121569,0.466667,0.705882}%
\pgfsetstrokecolor{currentstroke}%
\pgfsetdash{}{0pt}%
\pgfpathmoveto{\pgfqpoint{1.356593in}{2.338178in}}%
\pgfpathcurveto{\pgfqpoint{1.367643in}{2.338178in}}{\pgfqpoint{1.378242in}{2.342569in}}{\pgfqpoint{1.386056in}{2.350382in}}%
\pgfpathcurveto{\pgfqpoint{1.393869in}{2.358196in}}{\pgfqpoint{1.398259in}{2.368795in}}{\pgfqpoint{1.398259in}{2.379845in}}%
\pgfpathcurveto{\pgfqpoint{1.398259in}{2.390895in}}{\pgfqpoint{1.393869in}{2.401494in}}{\pgfqpoint{1.386056in}{2.409308in}}%
\pgfpathcurveto{\pgfqpoint{1.378242in}{2.417121in}}{\pgfqpoint{1.367643in}{2.421512in}}{\pgfqpoint{1.356593in}{2.421512in}}%
\pgfpathcurveto{\pgfqpoint{1.345543in}{2.421512in}}{\pgfqpoint{1.334944in}{2.417121in}}{\pgfqpoint{1.327130in}{2.409308in}}%
\pgfpathcurveto{\pgfqpoint{1.319316in}{2.401494in}}{\pgfqpoint{1.314926in}{2.390895in}}{\pgfqpoint{1.314926in}{2.379845in}}%
\pgfpathcurveto{\pgfqpoint{1.314926in}{2.368795in}}{\pgfqpoint{1.319316in}{2.358196in}}{\pgfqpoint{1.327130in}{2.350382in}}%
\pgfpathcurveto{\pgfqpoint{1.334944in}{2.342569in}}{\pgfqpoint{1.345543in}{2.338178in}}{\pgfqpoint{1.356593in}{2.338178in}}%
\pgfpathclose%
\pgfusepath{stroke,fill}%
\end{pgfscope}%
\begin{pgfscope}%
\pgfpathrectangle{\pgfqpoint{0.600000in}{0.600000in}}{\pgfqpoint{3.900000in}{3.900000in}}%
\pgfusepath{clip}%
\pgfsetbuttcap%
\pgfsetroundjoin%
\definecolor{currentfill}{rgb}{0.121569,0.466667,0.705882}%
\pgfsetfillcolor{currentfill}%
\pgfsetlinewidth{1.003750pt}%
\definecolor{currentstroke}{rgb}{0.121569,0.466667,0.705882}%
\pgfsetstrokecolor{currentstroke}%
\pgfsetdash{}{0pt}%
\pgfpathmoveto{\pgfqpoint{2.713952in}{2.758567in}}%
\pgfpathcurveto{\pgfqpoint{2.725002in}{2.758567in}}{\pgfqpoint{2.735601in}{2.762957in}}{\pgfqpoint{2.743414in}{2.770771in}}%
\pgfpathcurveto{\pgfqpoint{2.751228in}{2.778584in}}{\pgfqpoint{2.755618in}{2.789183in}}{\pgfqpoint{2.755618in}{2.800233in}}%
\pgfpathcurveto{\pgfqpoint{2.755618in}{2.811284in}}{\pgfqpoint{2.751228in}{2.821883in}}{\pgfqpoint{2.743414in}{2.829696in}}%
\pgfpathcurveto{\pgfqpoint{2.735601in}{2.837510in}}{\pgfqpoint{2.725002in}{2.841900in}}{\pgfqpoint{2.713952in}{2.841900in}}%
\pgfpathcurveto{\pgfqpoint{2.702902in}{2.841900in}}{\pgfqpoint{2.692302in}{2.837510in}}{\pgfqpoint{2.684489in}{2.829696in}}%
\pgfpathcurveto{\pgfqpoint{2.676675in}{2.821883in}}{\pgfqpoint{2.672285in}{2.811284in}}{\pgfqpoint{2.672285in}{2.800233in}}%
\pgfpathcurveto{\pgfqpoint{2.672285in}{2.789183in}}{\pgfqpoint{2.676675in}{2.778584in}}{\pgfqpoint{2.684489in}{2.770771in}}%
\pgfpathcurveto{\pgfqpoint{2.692302in}{2.762957in}}{\pgfqpoint{2.702902in}{2.758567in}}{\pgfqpoint{2.713952in}{2.758567in}}%
\pgfpathclose%
\pgfusepath{stroke,fill}%
\end{pgfscope}%
\begin{pgfscope}%
\pgfpathrectangle{\pgfqpoint{0.600000in}{0.600000in}}{\pgfqpoint{3.900000in}{3.900000in}}%
\pgfusepath{clip}%
\pgfsetbuttcap%
\pgfsetroundjoin%
\definecolor{currentfill}{rgb}{0.121569,0.466667,0.705882}%
\pgfsetfillcolor{currentfill}%
\pgfsetlinewidth{1.003750pt}%
\definecolor{currentstroke}{rgb}{0.121569,0.466667,0.705882}%
\pgfsetstrokecolor{currentstroke}%
\pgfsetdash{}{0pt}%
\pgfpathmoveto{\pgfqpoint{3.967206in}{1.481157in}}%
\pgfpathcurveto{\pgfqpoint{3.978256in}{1.481157in}}{\pgfqpoint{3.988855in}{1.485547in}}{\pgfqpoint{3.996669in}{1.493361in}}%
\pgfpathcurveto{\pgfqpoint{4.004482in}{1.501175in}}{\pgfqpoint{4.008873in}{1.511774in}}{\pgfqpoint{4.008873in}{1.522824in}}%
\pgfpathcurveto{\pgfqpoint{4.008873in}{1.533874in}}{\pgfqpoint{4.004482in}{1.544473in}}{\pgfqpoint{3.996669in}{1.552287in}}%
\pgfpathcurveto{\pgfqpoint{3.988855in}{1.560100in}}{\pgfqpoint{3.978256in}{1.564490in}}{\pgfqpoint{3.967206in}{1.564490in}}%
\pgfpathcurveto{\pgfqpoint{3.956156in}{1.564490in}}{\pgfqpoint{3.945557in}{1.560100in}}{\pgfqpoint{3.937743in}{1.552287in}}%
\pgfpathcurveto{\pgfqpoint{3.929930in}{1.544473in}}{\pgfqpoint{3.925539in}{1.533874in}}{\pgfqpoint{3.925539in}{1.522824in}}%
\pgfpathcurveto{\pgfqpoint{3.925539in}{1.511774in}}{\pgfqpoint{3.929930in}{1.501175in}}{\pgfqpoint{3.937743in}{1.493361in}}%
\pgfpathcurveto{\pgfqpoint{3.945557in}{1.485547in}}{\pgfqpoint{3.956156in}{1.481157in}}{\pgfqpoint{3.967206in}{1.481157in}}%
\pgfpathclose%
\pgfusepath{stroke,fill}%
\end{pgfscope}%
\begin{pgfscope}%
\pgfpathrectangle{\pgfqpoint{0.600000in}{0.600000in}}{\pgfqpoint{3.900000in}{3.900000in}}%
\pgfusepath{clip}%
\pgfsetbuttcap%
\pgfsetroundjoin%
\definecolor{currentfill}{rgb}{0.121569,0.466667,0.705882}%
\pgfsetfillcolor{currentfill}%
\pgfsetlinewidth{1.003750pt}%
\definecolor{currentstroke}{rgb}{0.121569,0.466667,0.705882}%
\pgfsetstrokecolor{currentstroke}%
\pgfsetdash{}{0pt}%
\pgfpathmoveto{\pgfqpoint{2.894815in}{3.336181in}}%
\pgfpathcurveto{\pgfqpoint{2.905865in}{3.336181in}}{\pgfqpoint{2.916464in}{3.340571in}}{\pgfqpoint{2.924278in}{3.348385in}}%
\pgfpathcurveto{\pgfqpoint{2.932091in}{3.356198in}}{\pgfqpoint{2.936481in}{3.366797in}}{\pgfqpoint{2.936481in}{3.377847in}}%
\pgfpathcurveto{\pgfqpoint{2.936481in}{3.388897in}}{\pgfqpoint{2.932091in}{3.399496in}}{\pgfqpoint{2.924278in}{3.407310in}}%
\pgfpathcurveto{\pgfqpoint{2.916464in}{3.415124in}}{\pgfqpoint{2.905865in}{3.419514in}}{\pgfqpoint{2.894815in}{3.419514in}}%
\pgfpathcurveto{\pgfqpoint{2.883765in}{3.419514in}}{\pgfqpoint{2.873166in}{3.415124in}}{\pgfqpoint{2.865352in}{3.407310in}}%
\pgfpathcurveto{\pgfqpoint{2.857538in}{3.399496in}}{\pgfqpoint{2.853148in}{3.388897in}}{\pgfqpoint{2.853148in}{3.377847in}}%
\pgfpathcurveto{\pgfqpoint{2.853148in}{3.366797in}}{\pgfqpoint{2.857538in}{3.356198in}}{\pgfqpoint{2.865352in}{3.348385in}}%
\pgfpathcurveto{\pgfqpoint{2.873166in}{3.340571in}}{\pgfqpoint{2.883765in}{3.336181in}}{\pgfqpoint{2.894815in}{3.336181in}}%
\pgfpathclose%
\pgfusepath{stroke,fill}%
\end{pgfscope}%
\begin{pgfscope}%
\pgfpathrectangle{\pgfqpoint{0.600000in}{0.600000in}}{\pgfqpoint{3.900000in}{3.900000in}}%
\pgfusepath{clip}%
\pgfsetbuttcap%
\pgfsetroundjoin%
\definecolor{currentfill}{rgb}{0.121569,0.466667,0.705882}%
\pgfsetfillcolor{currentfill}%
\pgfsetlinewidth{1.003750pt}%
\definecolor{currentstroke}{rgb}{0.121569,0.466667,0.705882}%
\pgfsetstrokecolor{currentstroke}%
\pgfsetdash{}{0pt}%
\pgfpathmoveto{\pgfqpoint{2.606041in}{2.386194in}}%
\pgfpathcurveto{\pgfqpoint{2.617091in}{2.386194in}}{\pgfqpoint{2.627690in}{2.390584in}}{\pgfqpoint{2.635504in}{2.398398in}}%
\pgfpathcurveto{\pgfqpoint{2.643317in}{2.406211in}}{\pgfqpoint{2.647708in}{2.416810in}}{\pgfqpoint{2.647708in}{2.427861in}}%
\pgfpathcurveto{\pgfqpoint{2.647708in}{2.438911in}}{\pgfqpoint{2.643317in}{2.449510in}}{\pgfqpoint{2.635504in}{2.457323in}}%
\pgfpathcurveto{\pgfqpoint{2.627690in}{2.465137in}}{\pgfqpoint{2.617091in}{2.469527in}}{\pgfqpoint{2.606041in}{2.469527in}}%
\pgfpathcurveto{\pgfqpoint{2.594991in}{2.469527in}}{\pgfqpoint{2.584392in}{2.465137in}}{\pgfqpoint{2.576578in}{2.457323in}}%
\pgfpathcurveto{\pgfqpoint{2.568765in}{2.449510in}}{\pgfqpoint{2.564374in}{2.438911in}}{\pgfqpoint{2.564374in}{2.427861in}}%
\pgfpathcurveto{\pgfqpoint{2.564374in}{2.416810in}}{\pgfqpoint{2.568765in}{2.406211in}}{\pgfqpoint{2.576578in}{2.398398in}}%
\pgfpathcurveto{\pgfqpoint{2.584392in}{2.390584in}}{\pgfqpoint{2.594991in}{2.386194in}}{\pgfqpoint{2.606041in}{2.386194in}}%
\pgfpathclose%
\pgfusepath{stroke,fill}%
\end{pgfscope}%
\begin{pgfscope}%
\pgfpathrectangle{\pgfqpoint{0.600000in}{0.600000in}}{\pgfqpoint{3.900000in}{3.900000in}}%
\pgfusepath{clip}%
\pgfsetbuttcap%
\pgfsetroundjoin%
\definecolor{currentfill}{rgb}{0.121569,0.466667,0.705882}%
\pgfsetfillcolor{currentfill}%
\pgfsetlinewidth{1.003750pt}%
\definecolor{currentstroke}{rgb}{0.121569,0.466667,0.705882}%
\pgfsetstrokecolor{currentstroke}%
\pgfsetdash{}{0pt}%
\pgfpathmoveto{\pgfqpoint{2.188573in}{2.285501in}}%
\pgfpathcurveto{\pgfqpoint{2.199623in}{2.285501in}}{\pgfqpoint{2.210222in}{2.289891in}}{\pgfqpoint{2.218036in}{2.297705in}}%
\pgfpathcurveto{\pgfqpoint{2.225849in}{2.305519in}}{\pgfqpoint{2.230239in}{2.316118in}}{\pgfqpoint{2.230239in}{2.327168in}}%
\pgfpathcurveto{\pgfqpoint{2.230239in}{2.338218in}}{\pgfqpoint{2.225849in}{2.348817in}}{\pgfqpoint{2.218036in}{2.356631in}}%
\pgfpathcurveto{\pgfqpoint{2.210222in}{2.364444in}}{\pgfqpoint{2.199623in}{2.368834in}}{\pgfqpoint{2.188573in}{2.368834in}}%
\pgfpathcurveto{\pgfqpoint{2.177523in}{2.368834in}}{\pgfqpoint{2.166924in}{2.364444in}}{\pgfqpoint{2.159110in}{2.356631in}}%
\pgfpathcurveto{\pgfqpoint{2.151296in}{2.348817in}}{\pgfqpoint{2.146906in}{2.338218in}}{\pgfqpoint{2.146906in}{2.327168in}}%
\pgfpathcurveto{\pgfqpoint{2.146906in}{2.316118in}}{\pgfqpoint{2.151296in}{2.305519in}}{\pgfqpoint{2.159110in}{2.297705in}}%
\pgfpathcurveto{\pgfqpoint{2.166924in}{2.289891in}}{\pgfqpoint{2.177523in}{2.285501in}}{\pgfqpoint{2.188573in}{2.285501in}}%
\pgfpathclose%
\pgfusepath{stroke,fill}%
\end{pgfscope}%
\begin{pgfscope}%
\pgfpathrectangle{\pgfqpoint{0.600000in}{0.600000in}}{\pgfqpoint{3.900000in}{3.900000in}}%
\pgfusepath{clip}%
\pgfsetbuttcap%
\pgfsetroundjoin%
\definecolor{currentfill}{rgb}{0.121569,0.466667,0.705882}%
\pgfsetfillcolor{currentfill}%
\pgfsetlinewidth{1.003750pt}%
\definecolor{currentstroke}{rgb}{0.121569,0.466667,0.705882}%
\pgfsetstrokecolor{currentstroke}%
\pgfsetdash{}{0pt}%
\pgfpathmoveto{\pgfqpoint{2.616860in}{2.063577in}}%
\pgfpathcurveto{\pgfqpoint{2.627911in}{2.063577in}}{\pgfqpoint{2.638510in}{2.067967in}}{\pgfqpoint{2.646323in}{2.075781in}}%
\pgfpathcurveto{\pgfqpoint{2.654137in}{2.083594in}}{\pgfqpoint{2.658527in}{2.094193in}}{\pgfqpoint{2.658527in}{2.105244in}}%
\pgfpathcurveto{\pgfqpoint{2.658527in}{2.116294in}}{\pgfqpoint{2.654137in}{2.126893in}}{\pgfqpoint{2.646323in}{2.134706in}}%
\pgfpathcurveto{\pgfqpoint{2.638510in}{2.142520in}}{\pgfqpoint{2.627911in}{2.146910in}}{\pgfqpoint{2.616860in}{2.146910in}}%
\pgfpathcurveto{\pgfqpoint{2.605810in}{2.146910in}}{\pgfqpoint{2.595211in}{2.142520in}}{\pgfqpoint{2.587398in}{2.134706in}}%
\pgfpathcurveto{\pgfqpoint{2.579584in}{2.126893in}}{\pgfqpoint{2.575194in}{2.116294in}}{\pgfqpoint{2.575194in}{2.105244in}}%
\pgfpathcurveto{\pgfqpoint{2.575194in}{2.094193in}}{\pgfqpoint{2.579584in}{2.083594in}}{\pgfqpoint{2.587398in}{2.075781in}}%
\pgfpathcurveto{\pgfqpoint{2.595211in}{2.067967in}}{\pgfqpoint{2.605810in}{2.063577in}}{\pgfqpoint{2.616860in}{2.063577in}}%
\pgfpathclose%
\pgfusepath{stroke,fill}%
\end{pgfscope}%
\begin{pgfscope}%
\pgfpathrectangle{\pgfqpoint{0.600000in}{0.600000in}}{\pgfqpoint{3.900000in}{3.900000in}}%
\pgfusepath{clip}%
\pgfsetbuttcap%
\pgfsetroundjoin%
\definecolor{currentfill}{rgb}{0.121569,0.466667,0.705882}%
\pgfsetfillcolor{currentfill}%
\pgfsetlinewidth{1.003750pt}%
\definecolor{currentstroke}{rgb}{0.121569,0.466667,0.705882}%
\pgfsetstrokecolor{currentstroke}%
\pgfsetdash{}{0pt}%
\pgfpathmoveto{\pgfqpoint{2.722364in}{3.367383in}}%
\pgfpathcurveto{\pgfqpoint{2.733414in}{3.367383in}}{\pgfqpoint{2.744013in}{3.371773in}}{\pgfqpoint{2.751827in}{3.379587in}}%
\pgfpathcurveto{\pgfqpoint{2.759641in}{3.387400in}}{\pgfqpoint{2.764031in}{3.397999in}}{\pgfqpoint{2.764031in}{3.409050in}}%
\pgfpathcurveto{\pgfqpoint{2.764031in}{3.420100in}}{\pgfqpoint{2.759641in}{3.430699in}}{\pgfqpoint{2.751827in}{3.438512in}}%
\pgfpathcurveto{\pgfqpoint{2.744013in}{3.446326in}}{\pgfqpoint{2.733414in}{3.450716in}}{\pgfqpoint{2.722364in}{3.450716in}}%
\pgfpathcurveto{\pgfqpoint{2.711314in}{3.450716in}}{\pgfqpoint{2.700715in}{3.446326in}}{\pgfqpoint{2.692902in}{3.438512in}}%
\pgfpathcurveto{\pgfqpoint{2.685088in}{3.430699in}}{\pgfqpoint{2.680698in}{3.420100in}}{\pgfqpoint{2.680698in}{3.409050in}}%
\pgfpathcurveto{\pgfqpoint{2.680698in}{3.397999in}}{\pgfqpoint{2.685088in}{3.387400in}}{\pgfqpoint{2.692902in}{3.379587in}}%
\pgfpathcurveto{\pgfqpoint{2.700715in}{3.371773in}}{\pgfqpoint{2.711314in}{3.367383in}}{\pgfqpoint{2.722364in}{3.367383in}}%
\pgfpathclose%
\pgfusepath{stroke,fill}%
\end{pgfscope}%
\begin{pgfscope}%
\pgfpathrectangle{\pgfqpoint{0.600000in}{0.600000in}}{\pgfqpoint{3.900000in}{3.900000in}}%
\pgfusepath{clip}%
\pgfsetbuttcap%
\pgfsetroundjoin%
\definecolor{currentfill}{rgb}{0.121569,0.466667,0.705882}%
\pgfsetfillcolor{currentfill}%
\pgfsetlinewidth{1.003750pt}%
\definecolor{currentstroke}{rgb}{0.121569,0.466667,0.705882}%
\pgfsetstrokecolor{currentstroke}%
\pgfsetdash{}{0pt}%
\pgfpathmoveto{\pgfqpoint{2.872798in}{1.863719in}}%
\pgfpathcurveto{\pgfqpoint{2.883849in}{1.863719in}}{\pgfqpoint{2.894448in}{1.868109in}}{\pgfqpoint{2.902261in}{1.875923in}}%
\pgfpathcurveto{\pgfqpoint{2.910075in}{1.883736in}}{\pgfqpoint{2.914465in}{1.894335in}}{\pgfqpoint{2.914465in}{1.905385in}}%
\pgfpathcurveto{\pgfqpoint{2.914465in}{1.916436in}}{\pgfqpoint{2.910075in}{1.927035in}}{\pgfqpoint{2.902261in}{1.934848in}}%
\pgfpathcurveto{\pgfqpoint{2.894448in}{1.942662in}}{\pgfqpoint{2.883849in}{1.947052in}}{\pgfqpoint{2.872798in}{1.947052in}}%
\pgfpathcurveto{\pgfqpoint{2.861748in}{1.947052in}}{\pgfqpoint{2.851149in}{1.942662in}}{\pgfqpoint{2.843336in}{1.934848in}}%
\pgfpathcurveto{\pgfqpoint{2.835522in}{1.927035in}}{\pgfqpoint{2.831132in}{1.916436in}}{\pgfqpoint{2.831132in}{1.905385in}}%
\pgfpathcurveto{\pgfqpoint{2.831132in}{1.894335in}}{\pgfqpoint{2.835522in}{1.883736in}}{\pgfqpoint{2.843336in}{1.875923in}}%
\pgfpathcurveto{\pgfqpoint{2.851149in}{1.868109in}}{\pgfqpoint{2.861748in}{1.863719in}}{\pgfqpoint{2.872798in}{1.863719in}}%
\pgfpathclose%
\pgfusepath{stroke,fill}%
\end{pgfscope}%
\begin{pgfscope}%
\pgfpathrectangle{\pgfqpoint{0.600000in}{0.600000in}}{\pgfqpoint{3.900000in}{3.900000in}}%
\pgfusepath{clip}%
\pgfsetbuttcap%
\pgfsetroundjoin%
\definecolor{currentfill}{rgb}{0.121569,0.466667,0.705882}%
\pgfsetfillcolor{currentfill}%
\pgfsetlinewidth{1.003750pt}%
\definecolor{currentstroke}{rgb}{0.121569,0.466667,0.705882}%
\pgfsetstrokecolor{currentstroke}%
\pgfsetdash{}{0pt}%
\pgfpathmoveto{\pgfqpoint{1.805300in}{1.874947in}}%
\pgfpathcurveto{\pgfqpoint{1.816351in}{1.874947in}}{\pgfqpoint{1.826950in}{1.879337in}}{\pgfqpoint{1.834763in}{1.887151in}}%
\pgfpathcurveto{\pgfqpoint{1.842577in}{1.894964in}}{\pgfqpoint{1.846967in}{1.905563in}}{\pgfqpoint{1.846967in}{1.916614in}}%
\pgfpathcurveto{\pgfqpoint{1.846967in}{1.927664in}}{\pgfqpoint{1.842577in}{1.938263in}}{\pgfqpoint{1.834763in}{1.946076in}}%
\pgfpathcurveto{\pgfqpoint{1.826950in}{1.953890in}}{\pgfqpoint{1.816351in}{1.958280in}}{\pgfqpoint{1.805300in}{1.958280in}}%
\pgfpathcurveto{\pgfqpoint{1.794250in}{1.958280in}}{\pgfqpoint{1.783651in}{1.953890in}}{\pgfqpoint{1.775838in}{1.946076in}}%
\pgfpathcurveto{\pgfqpoint{1.768024in}{1.938263in}}{\pgfqpoint{1.763634in}{1.927664in}}{\pgfqpoint{1.763634in}{1.916614in}}%
\pgfpathcurveto{\pgfqpoint{1.763634in}{1.905563in}}{\pgfqpoint{1.768024in}{1.894964in}}{\pgfqpoint{1.775838in}{1.887151in}}%
\pgfpathcurveto{\pgfqpoint{1.783651in}{1.879337in}}{\pgfqpoint{1.794250in}{1.874947in}}{\pgfqpoint{1.805300in}{1.874947in}}%
\pgfpathclose%
\pgfusepath{stroke,fill}%
\end{pgfscope}%
\begin{pgfscope}%
\pgfpathrectangle{\pgfqpoint{0.600000in}{0.600000in}}{\pgfqpoint{3.900000in}{3.900000in}}%
\pgfusepath{clip}%
\pgfsetbuttcap%
\pgfsetroundjoin%
\definecolor{currentfill}{rgb}{0.121569,0.466667,0.705882}%
\pgfsetfillcolor{currentfill}%
\pgfsetlinewidth{1.003750pt}%
\definecolor{currentstroke}{rgb}{0.121569,0.466667,0.705882}%
\pgfsetstrokecolor{currentstroke}%
\pgfsetdash{}{0pt}%
\pgfpathmoveto{\pgfqpoint{1.583779in}{2.922196in}}%
\pgfpathcurveto{\pgfqpoint{1.594830in}{2.922196in}}{\pgfqpoint{1.605429in}{2.926586in}}{\pgfqpoint{1.613242in}{2.934400in}}%
\pgfpathcurveto{\pgfqpoint{1.621056in}{2.942214in}}{\pgfqpoint{1.625446in}{2.952813in}}{\pgfqpoint{1.625446in}{2.963863in}}%
\pgfpathcurveto{\pgfqpoint{1.625446in}{2.974913in}}{\pgfqpoint{1.621056in}{2.985512in}}{\pgfqpoint{1.613242in}{2.993326in}}%
\pgfpathcurveto{\pgfqpoint{1.605429in}{3.001139in}}{\pgfqpoint{1.594830in}{3.005529in}}{\pgfqpoint{1.583779in}{3.005529in}}%
\pgfpathcurveto{\pgfqpoint{1.572729in}{3.005529in}}{\pgfqpoint{1.562130in}{3.001139in}}{\pgfqpoint{1.554317in}{2.993326in}}%
\pgfpathcurveto{\pgfqpoint{1.546503in}{2.985512in}}{\pgfqpoint{1.542113in}{2.974913in}}{\pgfqpoint{1.542113in}{2.963863in}}%
\pgfpathcurveto{\pgfqpoint{1.542113in}{2.952813in}}{\pgfqpoint{1.546503in}{2.942214in}}{\pgfqpoint{1.554317in}{2.934400in}}%
\pgfpathcurveto{\pgfqpoint{1.562130in}{2.926586in}}{\pgfqpoint{1.572729in}{2.922196in}}{\pgfqpoint{1.583779in}{2.922196in}}%
\pgfpathclose%
\pgfusepath{stroke,fill}%
\end{pgfscope}%
\begin{pgfscope}%
\pgfpathrectangle{\pgfqpoint{0.600000in}{0.600000in}}{\pgfqpoint{3.900000in}{3.900000in}}%
\pgfusepath{clip}%
\pgfsetbuttcap%
\pgfsetroundjoin%
\definecolor{currentfill}{rgb}{0.121569,0.466667,0.705882}%
\pgfsetfillcolor{currentfill}%
\pgfsetlinewidth{1.003750pt}%
\definecolor{currentstroke}{rgb}{0.121569,0.466667,0.705882}%
\pgfsetstrokecolor{currentstroke}%
\pgfsetdash{}{0pt}%
\pgfpathmoveto{\pgfqpoint{0.666135in}{2.288866in}}%
\pgfpathcurveto{\pgfqpoint{0.677185in}{2.288866in}}{\pgfqpoint{0.687784in}{2.293256in}}{\pgfqpoint{0.695598in}{2.301070in}}%
\pgfpathcurveto{\pgfqpoint{0.703411in}{2.308883in}}{\pgfqpoint{0.707802in}{2.319482in}}{\pgfqpoint{0.707802in}{2.330533in}}%
\pgfpathcurveto{\pgfqpoint{0.707802in}{2.341583in}}{\pgfqpoint{0.703411in}{2.352182in}}{\pgfqpoint{0.695598in}{2.359995in}}%
\pgfpathcurveto{\pgfqpoint{0.687784in}{2.367809in}}{\pgfqpoint{0.677185in}{2.372199in}}{\pgfqpoint{0.666135in}{2.372199in}}%
\pgfpathcurveto{\pgfqpoint{0.655085in}{2.372199in}}{\pgfqpoint{0.644486in}{2.367809in}}{\pgfqpoint{0.636672in}{2.359995in}}%
\pgfpathcurveto{\pgfqpoint{0.628859in}{2.352182in}}{\pgfqpoint{0.624468in}{2.341583in}}{\pgfqpoint{0.624468in}{2.330533in}}%
\pgfpathcurveto{\pgfqpoint{0.624468in}{2.319482in}}{\pgfqpoint{0.628859in}{2.308883in}}{\pgfqpoint{0.636672in}{2.301070in}}%
\pgfpathcurveto{\pgfqpoint{0.644486in}{2.293256in}}{\pgfqpoint{0.655085in}{2.288866in}}{\pgfqpoint{0.666135in}{2.288866in}}%
\pgfpathclose%
\pgfusepath{stroke,fill}%
\end{pgfscope}%
\begin{pgfscope}%
\pgfpathrectangle{\pgfqpoint{0.600000in}{0.600000in}}{\pgfqpoint{3.900000in}{3.900000in}}%
\pgfusepath{clip}%
\pgfsetbuttcap%
\pgfsetroundjoin%
\definecolor{currentfill}{rgb}{0.121569,0.466667,0.705882}%
\pgfsetfillcolor{currentfill}%
\pgfsetlinewidth{1.003750pt}%
\definecolor{currentstroke}{rgb}{0.121569,0.466667,0.705882}%
\pgfsetstrokecolor{currentstroke}%
\pgfsetdash{}{0pt}%
\pgfpathmoveto{\pgfqpoint{2.307593in}{1.917735in}}%
\pgfpathcurveto{\pgfqpoint{2.318643in}{1.917735in}}{\pgfqpoint{2.329242in}{1.922126in}}{\pgfqpoint{2.337056in}{1.929939in}}%
\pgfpathcurveto{\pgfqpoint{2.344869in}{1.937753in}}{\pgfqpoint{2.349260in}{1.948352in}}{\pgfqpoint{2.349260in}{1.959402in}}%
\pgfpathcurveto{\pgfqpoint{2.349260in}{1.970452in}}{\pgfqpoint{2.344869in}{1.981051in}}{\pgfqpoint{2.337056in}{1.988865in}}%
\pgfpathcurveto{\pgfqpoint{2.329242in}{1.996678in}}{\pgfqpoint{2.318643in}{2.001069in}}{\pgfqpoint{2.307593in}{2.001069in}}%
\pgfpathcurveto{\pgfqpoint{2.296543in}{2.001069in}}{\pgfqpoint{2.285944in}{1.996678in}}{\pgfqpoint{2.278130in}{1.988865in}}%
\pgfpathcurveto{\pgfqpoint{2.270317in}{1.981051in}}{\pgfqpoint{2.265926in}{1.970452in}}{\pgfqpoint{2.265926in}{1.959402in}}%
\pgfpathcurveto{\pgfqpoint{2.265926in}{1.948352in}}{\pgfqpoint{2.270317in}{1.937753in}}{\pgfqpoint{2.278130in}{1.929939in}}%
\pgfpathcurveto{\pgfqpoint{2.285944in}{1.922126in}}{\pgfqpoint{2.296543in}{1.917735in}}{\pgfqpoint{2.307593in}{1.917735in}}%
\pgfpathclose%
\pgfusepath{stroke,fill}%
\end{pgfscope}%
\begin{pgfscope}%
\pgfpathrectangle{\pgfqpoint{0.600000in}{0.600000in}}{\pgfqpoint{3.900000in}{3.900000in}}%
\pgfusepath{clip}%
\pgfsetbuttcap%
\pgfsetroundjoin%
\definecolor{currentfill}{rgb}{0.121569,0.466667,0.705882}%
\pgfsetfillcolor{currentfill}%
\pgfsetlinewidth{1.003750pt}%
\definecolor{currentstroke}{rgb}{0.121569,0.466667,0.705882}%
\pgfsetstrokecolor{currentstroke}%
\pgfsetdash{}{0pt}%
\pgfpathmoveto{\pgfqpoint{1.438357in}{3.262805in}}%
\pgfpathcurveto{\pgfqpoint{1.449407in}{3.262805in}}{\pgfqpoint{1.460006in}{3.267196in}}{\pgfqpoint{1.467820in}{3.275009in}}%
\pgfpathcurveto{\pgfqpoint{1.475634in}{3.282823in}}{\pgfqpoint{1.480024in}{3.293422in}}{\pgfqpoint{1.480024in}{3.304472in}}%
\pgfpathcurveto{\pgfqpoint{1.480024in}{3.315522in}}{\pgfqpoint{1.475634in}{3.326121in}}{\pgfqpoint{1.467820in}{3.333935in}}%
\pgfpathcurveto{\pgfqpoint{1.460006in}{3.341748in}}{\pgfqpoint{1.449407in}{3.346139in}}{\pgfqpoint{1.438357in}{3.346139in}}%
\pgfpathcurveto{\pgfqpoint{1.427307in}{3.346139in}}{\pgfqpoint{1.416708in}{3.341748in}}{\pgfqpoint{1.408895in}{3.333935in}}%
\pgfpathcurveto{\pgfqpoint{1.401081in}{3.326121in}}{\pgfqpoint{1.396691in}{3.315522in}}{\pgfqpoint{1.396691in}{3.304472in}}%
\pgfpathcurveto{\pgfqpoint{1.396691in}{3.293422in}}{\pgfqpoint{1.401081in}{3.282823in}}{\pgfqpoint{1.408895in}{3.275009in}}%
\pgfpathcurveto{\pgfqpoint{1.416708in}{3.267196in}}{\pgfqpoint{1.427307in}{3.262805in}}{\pgfqpoint{1.438357in}{3.262805in}}%
\pgfpathclose%
\pgfusepath{stroke,fill}%
\end{pgfscope}%
\begin{pgfscope}%
\pgfpathrectangle{\pgfqpoint{0.600000in}{0.600000in}}{\pgfqpoint{3.900000in}{3.900000in}}%
\pgfusepath{clip}%
\pgfsetbuttcap%
\pgfsetroundjoin%
\definecolor{currentfill}{rgb}{0.121569,0.466667,0.705882}%
\pgfsetfillcolor{currentfill}%
\pgfsetlinewidth{1.003750pt}%
\definecolor{currentstroke}{rgb}{0.121569,0.466667,0.705882}%
\pgfsetstrokecolor{currentstroke}%
\pgfsetdash{}{0pt}%
\pgfpathmoveto{\pgfqpoint{2.437962in}{3.098662in}}%
\pgfpathcurveto{\pgfqpoint{2.449012in}{3.098662in}}{\pgfqpoint{2.459611in}{3.103053in}}{\pgfqpoint{2.467424in}{3.110866in}}%
\pgfpathcurveto{\pgfqpoint{2.475238in}{3.118680in}}{\pgfqpoint{2.479628in}{3.129279in}}{\pgfqpoint{2.479628in}{3.140329in}}%
\pgfpathcurveto{\pgfqpoint{2.479628in}{3.151379in}}{\pgfqpoint{2.475238in}{3.161978in}}{\pgfqpoint{2.467424in}{3.169792in}}%
\pgfpathcurveto{\pgfqpoint{2.459611in}{3.177605in}}{\pgfqpoint{2.449012in}{3.181996in}}{\pgfqpoint{2.437962in}{3.181996in}}%
\pgfpathcurveto{\pgfqpoint{2.426911in}{3.181996in}}{\pgfqpoint{2.416312in}{3.177605in}}{\pgfqpoint{2.408499in}{3.169792in}}%
\pgfpathcurveto{\pgfqpoint{2.400685in}{3.161978in}}{\pgfqpoint{2.396295in}{3.151379in}}{\pgfqpoint{2.396295in}{3.140329in}}%
\pgfpathcurveto{\pgfqpoint{2.396295in}{3.129279in}}{\pgfqpoint{2.400685in}{3.118680in}}{\pgfqpoint{2.408499in}{3.110866in}}%
\pgfpathcurveto{\pgfqpoint{2.416312in}{3.103053in}}{\pgfqpoint{2.426911in}{3.098662in}}{\pgfqpoint{2.437962in}{3.098662in}}%
\pgfpathclose%
\pgfusepath{stroke,fill}%
\end{pgfscope}%
\begin{pgfscope}%
\pgfpathrectangle{\pgfqpoint{0.600000in}{0.600000in}}{\pgfqpoint{3.900000in}{3.900000in}}%
\pgfusepath{clip}%
\pgfsetbuttcap%
\pgfsetroundjoin%
\definecolor{currentfill}{rgb}{0.121569,0.466667,0.705882}%
\pgfsetfillcolor{currentfill}%
\pgfsetlinewidth{1.003750pt}%
\definecolor{currentstroke}{rgb}{0.121569,0.466667,0.705882}%
\pgfsetstrokecolor{currentstroke}%
\pgfsetdash{}{0pt}%
\pgfpathmoveto{\pgfqpoint{2.525634in}{2.536233in}}%
\pgfpathcurveto{\pgfqpoint{2.536684in}{2.536233in}}{\pgfqpoint{2.547283in}{2.540623in}}{\pgfqpoint{2.555097in}{2.548436in}}%
\pgfpathcurveto{\pgfqpoint{2.562910in}{2.556250in}}{\pgfqpoint{2.567300in}{2.566849in}}{\pgfqpoint{2.567300in}{2.577899in}}%
\pgfpathcurveto{\pgfqpoint{2.567300in}{2.588949in}}{\pgfqpoint{2.562910in}{2.599548in}}{\pgfqpoint{2.555097in}{2.607362in}}%
\pgfpathcurveto{\pgfqpoint{2.547283in}{2.615176in}}{\pgfqpoint{2.536684in}{2.619566in}}{\pgfqpoint{2.525634in}{2.619566in}}%
\pgfpathcurveto{\pgfqpoint{2.514584in}{2.619566in}}{\pgfqpoint{2.503985in}{2.615176in}}{\pgfqpoint{2.496171in}{2.607362in}}%
\pgfpathcurveto{\pgfqpoint{2.488357in}{2.599548in}}{\pgfqpoint{2.483967in}{2.588949in}}{\pgfqpoint{2.483967in}{2.577899in}}%
\pgfpathcurveto{\pgfqpoint{2.483967in}{2.566849in}}{\pgfqpoint{2.488357in}{2.556250in}}{\pgfqpoint{2.496171in}{2.548436in}}%
\pgfpathcurveto{\pgfqpoint{2.503985in}{2.540623in}}{\pgfqpoint{2.514584in}{2.536233in}}{\pgfqpoint{2.525634in}{2.536233in}}%
\pgfpathclose%
\pgfusepath{stroke,fill}%
\end{pgfscope}%
\begin{pgfscope}%
\pgfpathrectangle{\pgfqpoint{0.600000in}{0.600000in}}{\pgfqpoint{3.900000in}{3.900000in}}%
\pgfusepath{clip}%
\pgfsetbuttcap%
\pgfsetroundjoin%
\definecolor{currentfill}{rgb}{0.121569,0.466667,0.705882}%
\pgfsetfillcolor{currentfill}%
\pgfsetlinewidth{1.003750pt}%
\definecolor{currentstroke}{rgb}{0.121569,0.466667,0.705882}%
\pgfsetstrokecolor{currentstroke}%
\pgfsetdash{}{0pt}%
\pgfpathmoveto{\pgfqpoint{3.239267in}{2.289509in}}%
\pgfpathcurveto{\pgfqpoint{3.250318in}{2.289509in}}{\pgfqpoint{3.260917in}{2.293900in}}{\pgfqpoint{3.268730in}{2.301713in}}%
\pgfpathcurveto{\pgfqpoint{3.276544in}{2.309527in}}{\pgfqpoint{3.280934in}{2.320126in}}{\pgfqpoint{3.280934in}{2.331176in}}%
\pgfpathcurveto{\pgfqpoint{3.280934in}{2.342226in}}{\pgfqpoint{3.276544in}{2.352825in}}{\pgfqpoint{3.268730in}{2.360639in}}%
\pgfpathcurveto{\pgfqpoint{3.260917in}{2.368452in}}{\pgfqpoint{3.250318in}{2.372843in}}{\pgfqpoint{3.239267in}{2.372843in}}%
\pgfpathcurveto{\pgfqpoint{3.228217in}{2.372843in}}{\pgfqpoint{3.217618in}{2.368452in}}{\pgfqpoint{3.209805in}{2.360639in}}%
\pgfpathcurveto{\pgfqpoint{3.201991in}{2.352825in}}{\pgfqpoint{3.197601in}{2.342226in}}{\pgfqpoint{3.197601in}{2.331176in}}%
\pgfpathcurveto{\pgfqpoint{3.197601in}{2.320126in}}{\pgfqpoint{3.201991in}{2.309527in}}{\pgfqpoint{3.209805in}{2.301713in}}%
\pgfpathcurveto{\pgfqpoint{3.217618in}{2.293900in}}{\pgfqpoint{3.228217in}{2.289509in}}{\pgfqpoint{3.239267in}{2.289509in}}%
\pgfpathclose%
\pgfusepath{stroke,fill}%
\end{pgfscope}%
\begin{pgfscope}%
\pgfpathrectangle{\pgfqpoint{0.600000in}{0.600000in}}{\pgfqpoint{3.900000in}{3.900000in}}%
\pgfusepath{clip}%
\pgfsetbuttcap%
\pgfsetroundjoin%
\definecolor{currentfill}{rgb}{0.121569,0.466667,0.705882}%
\pgfsetfillcolor{currentfill}%
\pgfsetlinewidth{1.003750pt}%
\definecolor{currentstroke}{rgb}{0.121569,0.466667,0.705882}%
\pgfsetstrokecolor{currentstroke}%
\pgfsetdash{}{0pt}%
\pgfpathmoveto{\pgfqpoint{1.686414in}{1.751344in}}%
\pgfpathcurveto{\pgfqpoint{1.697464in}{1.751344in}}{\pgfqpoint{1.708063in}{1.755734in}}{\pgfqpoint{1.715877in}{1.763548in}}%
\pgfpathcurveto{\pgfqpoint{1.723691in}{1.771361in}}{\pgfqpoint{1.728081in}{1.781960in}}{\pgfqpoint{1.728081in}{1.793011in}}%
\pgfpathcurveto{\pgfqpoint{1.728081in}{1.804061in}}{\pgfqpoint{1.723691in}{1.814660in}}{\pgfqpoint{1.715877in}{1.822473in}}%
\pgfpathcurveto{\pgfqpoint{1.708063in}{1.830287in}}{\pgfqpoint{1.697464in}{1.834677in}}{\pgfqpoint{1.686414in}{1.834677in}}%
\pgfpathcurveto{\pgfqpoint{1.675364in}{1.834677in}}{\pgfqpoint{1.664765in}{1.830287in}}{\pgfqpoint{1.656951in}{1.822473in}}%
\pgfpathcurveto{\pgfqpoint{1.649138in}{1.814660in}}{\pgfqpoint{1.644747in}{1.804061in}}{\pgfqpoint{1.644747in}{1.793011in}}%
\pgfpathcurveto{\pgfqpoint{1.644747in}{1.781960in}}{\pgfqpoint{1.649138in}{1.771361in}}{\pgfqpoint{1.656951in}{1.763548in}}%
\pgfpathcurveto{\pgfqpoint{1.664765in}{1.755734in}}{\pgfqpoint{1.675364in}{1.751344in}}{\pgfqpoint{1.686414in}{1.751344in}}%
\pgfpathclose%
\pgfusepath{stroke,fill}%
\end{pgfscope}%
\begin{pgfscope}%
\pgfpathrectangle{\pgfqpoint{0.600000in}{0.600000in}}{\pgfqpoint{3.900000in}{3.900000in}}%
\pgfusepath{clip}%
\pgfsetbuttcap%
\pgfsetroundjoin%
\definecolor{currentfill}{rgb}{0.121569,0.466667,0.705882}%
\pgfsetfillcolor{currentfill}%
\pgfsetlinewidth{1.003750pt}%
\definecolor{currentstroke}{rgb}{0.121569,0.466667,0.705882}%
\pgfsetstrokecolor{currentstroke}%
\pgfsetdash{}{0pt}%
\pgfpathmoveto{\pgfqpoint{2.334023in}{2.782875in}}%
\pgfpathcurveto{\pgfqpoint{2.345074in}{2.782875in}}{\pgfqpoint{2.355673in}{2.787266in}}{\pgfqpoint{2.363486in}{2.795079in}}%
\pgfpathcurveto{\pgfqpoint{2.371300in}{2.802893in}}{\pgfqpoint{2.375690in}{2.813492in}}{\pgfqpoint{2.375690in}{2.824542in}}%
\pgfpathcurveto{\pgfqpoint{2.375690in}{2.835592in}}{\pgfqpoint{2.371300in}{2.846191in}}{\pgfqpoint{2.363486in}{2.854005in}}%
\pgfpathcurveto{\pgfqpoint{2.355673in}{2.861819in}}{\pgfqpoint{2.345074in}{2.866209in}}{\pgfqpoint{2.334023in}{2.866209in}}%
\pgfpathcurveto{\pgfqpoint{2.322973in}{2.866209in}}{\pgfqpoint{2.312374in}{2.861819in}}{\pgfqpoint{2.304561in}{2.854005in}}%
\pgfpathcurveto{\pgfqpoint{2.296747in}{2.846191in}}{\pgfqpoint{2.292357in}{2.835592in}}{\pgfqpoint{2.292357in}{2.824542in}}%
\pgfpathcurveto{\pgfqpoint{2.292357in}{2.813492in}}{\pgfqpoint{2.296747in}{2.802893in}}{\pgfqpoint{2.304561in}{2.795079in}}%
\pgfpathcurveto{\pgfqpoint{2.312374in}{2.787266in}}{\pgfqpoint{2.322973in}{2.782875in}}{\pgfqpoint{2.334023in}{2.782875in}}%
\pgfpathclose%
\pgfusepath{stroke,fill}%
\end{pgfscope}%
\begin{pgfscope}%
\pgfpathrectangle{\pgfqpoint{0.600000in}{0.600000in}}{\pgfqpoint{3.900000in}{3.900000in}}%
\pgfusepath{clip}%
\pgfsetbuttcap%
\pgfsetroundjoin%
\definecolor{currentfill}{rgb}{0.121569,0.466667,0.705882}%
\pgfsetfillcolor{currentfill}%
\pgfsetlinewidth{1.003750pt}%
\definecolor{currentstroke}{rgb}{0.121569,0.466667,0.705882}%
\pgfsetstrokecolor{currentstroke}%
\pgfsetdash{}{0pt}%
\pgfpathmoveto{\pgfqpoint{2.600736in}{2.621237in}}%
\pgfpathcurveto{\pgfqpoint{2.611786in}{2.621237in}}{\pgfqpoint{2.622385in}{2.625627in}}{\pgfqpoint{2.630199in}{2.633441in}}%
\pgfpathcurveto{\pgfqpoint{2.638012in}{2.641254in}}{\pgfqpoint{2.642402in}{2.651853in}}{\pgfqpoint{2.642402in}{2.662904in}}%
\pgfpathcurveto{\pgfqpoint{2.642402in}{2.673954in}}{\pgfqpoint{2.638012in}{2.684553in}}{\pgfqpoint{2.630199in}{2.692366in}}%
\pgfpathcurveto{\pgfqpoint{2.622385in}{2.700180in}}{\pgfqpoint{2.611786in}{2.704570in}}{\pgfqpoint{2.600736in}{2.704570in}}%
\pgfpathcurveto{\pgfqpoint{2.589686in}{2.704570in}}{\pgfqpoint{2.579087in}{2.700180in}}{\pgfqpoint{2.571273in}{2.692366in}}%
\pgfpathcurveto{\pgfqpoint{2.563459in}{2.684553in}}{\pgfqpoint{2.559069in}{2.673954in}}{\pgfqpoint{2.559069in}{2.662904in}}%
\pgfpathcurveto{\pgfqpoint{2.559069in}{2.651853in}}{\pgfqpoint{2.563459in}{2.641254in}}{\pgfqpoint{2.571273in}{2.633441in}}%
\pgfpathcurveto{\pgfqpoint{2.579087in}{2.625627in}}{\pgfqpoint{2.589686in}{2.621237in}}{\pgfqpoint{2.600736in}{2.621237in}}%
\pgfpathclose%
\pgfusepath{stroke,fill}%
\end{pgfscope}%
\begin{pgfscope}%
\pgfpathrectangle{\pgfqpoint{0.600000in}{0.600000in}}{\pgfqpoint{3.900000in}{3.900000in}}%
\pgfusepath{clip}%
\pgfsetbuttcap%
\pgfsetroundjoin%
\definecolor{currentfill}{rgb}{0.121569,0.466667,0.705882}%
\pgfsetfillcolor{currentfill}%
\pgfsetlinewidth{1.003750pt}%
\definecolor{currentstroke}{rgb}{0.121569,0.466667,0.705882}%
\pgfsetstrokecolor{currentstroke}%
\pgfsetdash{}{0pt}%
\pgfpathmoveto{\pgfqpoint{2.697256in}{1.857895in}}%
\pgfpathcurveto{\pgfqpoint{2.708306in}{1.857895in}}{\pgfqpoint{2.718905in}{1.862285in}}{\pgfqpoint{2.726719in}{1.870098in}}%
\pgfpathcurveto{\pgfqpoint{2.734533in}{1.877912in}}{\pgfqpoint{2.738923in}{1.888511in}}{\pgfqpoint{2.738923in}{1.899561in}}%
\pgfpathcurveto{\pgfqpoint{2.738923in}{1.910611in}}{\pgfqpoint{2.734533in}{1.921210in}}{\pgfqpoint{2.726719in}{1.929024in}}%
\pgfpathcurveto{\pgfqpoint{2.718905in}{1.936838in}}{\pgfqpoint{2.708306in}{1.941228in}}{\pgfqpoint{2.697256in}{1.941228in}}%
\pgfpathcurveto{\pgfqpoint{2.686206in}{1.941228in}}{\pgfqpoint{2.675607in}{1.936838in}}{\pgfqpoint{2.667793in}{1.929024in}}%
\pgfpathcurveto{\pgfqpoint{2.659980in}{1.921210in}}{\pgfqpoint{2.655589in}{1.910611in}}{\pgfqpoint{2.655589in}{1.899561in}}%
\pgfpathcurveto{\pgfqpoint{2.655589in}{1.888511in}}{\pgfqpoint{2.659980in}{1.877912in}}{\pgfqpoint{2.667793in}{1.870098in}}%
\pgfpathcurveto{\pgfqpoint{2.675607in}{1.862285in}}{\pgfqpoint{2.686206in}{1.857895in}}{\pgfqpoint{2.697256in}{1.857895in}}%
\pgfpathclose%
\pgfusepath{stroke,fill}%
\end{pgfscope}%
\begin{pgfscope}%
\pgfpathrectangle{\pgfqpoint{0.600000in}{0.600000in}}{\pgfqpoint{3.900000in}{3.900000in}}%
\pgfusepath{clip}%
\pgfsetbuttcap%
\pgfsetroundjoin%
\definecolor{currentfill}{rgb}{0.121569,0.466667,0.705882}%
\pgfsetfillcolor{currentfill}%
\pgfsetlinewidth{1.003750pt}%
\definecolor{currentstroke}{rgb}{0.121569,0.466667,0.705882}%
\pgfsetstrokecolor{currentstroke}%
\pgfsetdash{}{0pt}%
\pgfpathmoveto{\pgfqpoint{1.511884in}{2.284755in}}%
\pgfpathcurveto{\pgfqpoint{1.522934in}{2.284755in}}{\pgfqpoint{1.533533in}{2.289145in}}{\pgfqpoint{1.541347in}{2.296959in}}%
\pgfpathcurveto{\pgfqpoint{1.549160in}{2.304772in}}{\pgfqpoint{1.553551in}{2.315371in}}{\pgfqpoint{1.553551in}{2.326422in}}%
\pgfpathcurveto{\pgfqpoint{1.553551in}{2.337472in}}{\pgfqpoint{1.549160in}{2.348071in}}{\pgfqpoint{1.541347in}{2.355884in}}%
\pgfpathcurveto{\pgfqpoint{1.533533in}{2.363698in}}{\pgfqpoint{1.522934in}{2.368088in}}{\pgfqpoint{1.511884in}{2.368088in}}%
\pgfpathcurveto{\pgfqpoint{1.500834in}{2.368088in}}{\pgfqpoint{1.490235in}{2.363698in}}{\pgfqpoint{1.482421in}{2.355884in}}%
\pgfpathcurveto{\pgfqpoint{1.474608in}{2.348071in}}{\pgfqpoint{1.470217in}{2.337472in}}{\pgfqpoint{1.470217in}{2.326422in}}%
\pgfpathcurveto{\pgfqpoint{1.470217in}{2.315371in}}{\pgfqpoint{1.474608in}{2.304772in}}{\pgfqpoint{1.482421in}{2.296959in}}%
\pgfpathcurveto{\pgfqpoint{1.490235in}{2.289145in}}{\pgfqpoint{1.500834in}{2.284755in}}{\pgfqpoint{1.511884in}{2.284755in}}%
\pgfpathclose%
\pgfusepath{stroke,fill}%
\end{pgfscope}%
\begin{pgfscope}%
\pgfpathrectangle{\pgfqpoint{0.600000in}{0.600000in}}{\pgfqpoint{3.900000in}{3.900000in}}%
\pgfusepath{clip}%
\pgfsetbuttcap%
\pgfsetroundjoin%
\definecolor{currentfill}{rgb}{0.121569,0.466667,0.705882}%
\pgfsetfillcolor{currentfill}%
\pgfsetlinewidth{1.003750pt}%
\definecolor{currentstroke}{rgb}{0.121569,0.466667,0.705882}%
\pgfsetstrokecolor{currentstroke}%
\pgfsetdash{}{0pt}%
\pgfpathmoveto{\pgfqpoint{1.907824in}{1.931000in}}%
\pgfpathcurveto{\pgfqpoint{1.918874in}{1.931000in}}{\pgfqpoint{1.929473in}{1.935390in}}{\pgfqpoint{1.937286in}{1.943204in}}%
\pgfpathcurveto{\pgfqpoint{1.945100in}{1.951017in}}{\pgfqpoint{1.949490in}{1.961616in}}{\pgfqpoint{1.949490in}{1.972667in}}%
\pgfpathcurveto{\pgfqpoint{1.949490in}{1.983717in}}{\pgfqpoint{1.945100in}{1.994316in}}{\pgfqpoint{1.937286in}{2.002129in}}%
\pgfpathcurveto{\pgfqpoint{1.929473in}{2.009943in}}{\pgfqpoint{1.918874in}{2.014333in}}{\pgfqpoint{1.907824in}{2.014333in}}%
\pgfpathcurveto{\pgfqpoint{1.896774in}{2.014333in}}{\pgfqpoint{1.886175in}{2.009943in}}{\pgfqpoint{1.878361in}{2.002129in}}%
\pgfpathcurveto{\pgfqpoint{1.870547in}{1.994316in}}{\pgfqpoint{1.866157in}{1.983717in}}{\pgfqpoint{1.866157in}{1.972667in}}%
\pgfpathcurveto{\pgfqpoint{1.866157in}{1.961616in}}{\pgfqpoint{1.870547in}{1.951017in}}{\pgfqpoint{1.878361in}{1.943204in}}%
\pgfpathcurveto{\pgfqpoint{1.886175in}{1.935390in}}{\pgfqpoint{1.896774in}{1.931000in}}{\pgfqpoint{1.907824in}{1.931000in}}%
\pgfpathclose%
\pgfusepath{stroke,fill}%
\end{pgfscope}%
\begin{pgfscope}%
\pgfpathrectangle{\pgfqpoint{0.600000in}{0.600000in}}{\pgfqpoint{3.900000in}{3.900000in}}%
\pgfusepath{clip}%
\pgfsetbuttcap%
\pgfsetroundjoin%
\definecolor{currentfill}{rgb}{0.121569,0.466667,0.705882}%
\pgfsetfillcolor{currentfill}%
\pgfsetlinewidth{1.003750pt}%
\definecolor{currentstroke}{rgb}{0.121569,0.466667,0.705882}%
\pgfsetstrokecolor{currentstroke}%
\pgfsetdash{}{0pt}%
\pgfpathmoveto{\pgfqpoint{2.698382in}{2.841053in}}%
\pgfpathcurveto{\pgfqpoint{2.709432in}{2.841053in}}{\pgfqpoint{2.720031in}{2.845443in}}{\pgfqpoint{2.727844in}{2.853257in}}%
\pgfpathcurveto{\pgfqpoint{2.735658in}{2.861070in}}{\pgfqpoint{2.740048in}{2.871669in}}{\pgfqpoint{2.740048in}{2.882720in}}%
\pgfpathcurveto{\pgfqpoint{2.740048in}{2.893770in}}{\pgfqpoint{2.735658in}{2.904369in}}{\pgfqpoint{2.727844in}{2.912182in}}%
\pgfpathcurveto{\pgfqpoint{2.720031in}{2.919996in}}{\pgfqpoint{2.709432in}{2.924386in}}{\pgfqpoint{2.698382in}{2.924386in}}%
\pgfpathcurveto{\pgfqpoint{2.687332in}{2.924386in}}{\pgfqpoint{2.676733in}{2.919996in}}{\pgfqpoint{2.668919in}{2.912182in}}%
\pgfpathcurveto{\pgfqpoint{2.661105in}{2.904369in}}{\pgfqpoint{2.656715in}{2.893770in}}{\pgfqpoint{2.656715in}{2.882720in}}%
\pgfpathcurveto{\pgfqpoint{2.656715in}{2.871669in}}{\pgfqpoint{2.661105in}{2.861070in}}{\pgfqpoint{2.668919in}{2.853257in}}%
\pgfpathcurveto{\pgfqpoint{2.676733in}{2.845443in}}{\pgfqpoint{2.687332in}{2.841053in}}{\pgfqpoint{2.698382in}{2.841053in}}%
\pgfpathclose%
\pgfusepath{stroke,fill}%
\end{pgfscope}%
\begin{pgfscope}%
\pgfpathrectangle{\pgfqpoint{0.600000in}{0.600000in}}{\pgfqpoint{3.900000in}{3.900000in}}%
\pgfusepath{clip}%
\pgfsetbuttcap%
\pgfsetroundjoin%
\definecolor{currentfill}{rgb}{0.121569,0.466667,0.705882}%
\pgfsetfillcolor{currentfill}%
\pgfsetlinewidth{1.003750pt}%
\definecolor{currentstroke}{rgb}{0.121569,0.466667,0.705882}%
\pgfsetstrokecolor{currentstroke}%
\pgfsetdash{}{0pt}%
\pgfpathmoveto{\pgfqpoint{3.111707in}{2.263706in}}%
\pgfpathcurveto{\pgfqpoint{3.122757in}{2.263706in}}{\pgfqpoint{3.133357in}{2.268096in}}{\pgfqpoint{3.141170in}{2.275910in}}%
\pgfpathcurveto{\pgfqpoint{3.148984in}{2.283723in}}{\pgfqpoint{3.153374in}{2.294322in}}{\pgfqpoint{3.153374in}{2.305372in}}%
\pgfpathcurveto{\pgfqpoint{3.153374in}{2.316423in}}{\pgfqpoint{3.148984in}{2.327022in}}{\pgfqpoint{3.141170in}{2.334835in}}%
\pgfpathcurveto{\pgfqpoint{3.133357in}{2.342649in}}{\pgfqpoint{3.122757in}{2.347039in}}{\pgfqpoint{3.111707in}{2.347039in}}%
\pgfpathcurveto{\pgfqpoint{3.100657in}{2.347039in}}{\pgfqpoint{3.090058in}{2.342649in}}{\pgfqpoint{3.082245in}{2.334835in}}%
\pgfpathcurveto{\pgfqpoint{3.074431in}{2.327022in}}{\pgfqpoint{3.070041in}{2.316423in}}{\pgfqpoint{3.070041in}{2.305372in}}%
\pgfpathcurveto{\pgfqpoint{3.070041in}{2.294322in}}{\pgfqpoint{3.074431in}{2.283723in}}{\pgfqpoint{3.082245in}{2.275910in}}%
\pgfpathcurveto{\pgfqpoint{3.090058in}{2.268096in}}{\pgfqpoint{3.100657in}{2.263706in}}{\pgfqpoint{3.111707in}{2.263706in}}%
\pgfpathclose%
\pgfusepath{stroke,fill}%
\end{pgfscope}%
\begin{pgfscope}%
\pgfpathrectangle{\pgfqpoint{0.600000in}{0.600000in}}{\pgfqpoint{3.900000in}{3.900000in}}%
\pgfusepath{clip}%
\pgfsetbuttcap%
\pgfsetroundjoin%
\definecolor{currentfill}{rgb}{0.121569,0.466667,0.705882}%
\pgfsetfillcolor{currentfill}%
\pgfsetlinewidth{1.003750pt}%
\definecolor{currentstroke}{rgb}{0.121569,0.466667,0.705882}%
\pgfsetstrokecolor{currentstroke}%
\pgfsetdash{}{0pt}%
\pgfpathmoveto{\pgfqpoint{2.054179in}{3.110330in}}%
\pgfpathcurveto{\pgfqpoint{2.065229in}{3.110330in}}{\pgfqpoint{2.075828in}{3.114721in}}{\pgfqpoint{2.083642in}{3.122534in}}%
\pgfpathcurveto{\pgfqpoint{2.091455in}{3.130348in}}{\pgfqpoint{2.095845in}{3.140947in}}{\pgfqpoint{2.095845in}{3.151997in}}%
\pgfpathcurveto{\pgfqpoint{2.095845in}{3.163047in}}{\pgfqpoint{2.091455in}{3.173646in}}{\pgfqpoint{2.083642in}{3.181460in}}%
\pgfpathcurveto{\pgfqpoint{2.075828in}{3.189273in}}{\pgfqpoint{2.065229in}{3.193664in}}{\pgfqpoint{2.054179in}{3.193664in}}%
\pgfpathcurveto{\pgfqpoint{2.043129in}{3.193664in}}{\pgfqpoint{2.032530in}{3.189273in}}{\pgfqpoint{2.024716in}{3.181460in}}%
\pgfpathcurveto{\pgfqpoint{2.016902in}{3.173646in}}{\pgfqpoint{2.012512in}{3.163047in}}{\pgfqpoint{2.012512in}{3.151997in}}%
\pgfpathcurveto{\pgfqpoint{2.012512in}{3.140947in}}{\pgfqpoint{2.016902in}{3.130348in}}{\pgfqpoint{2.024716in}{3.122534in}}%
\pgfpathcurveto{\pgfqpoint{2.032530in}{3.114721in}}{\pgfqpoint{2.043129in}{3.110330in}}{\pgfqpoint{2.054179in}{3.110330in}}%
\pgfpathclose%
\pgfusepath{stroke,fill}%
\end{pgfscope}%
\begin{pgfscope}%
\pgfpathrectangle{\pgfqpoint{0.600000in}{0.600000in}}{\pgfqpoint{3.900000in}{3.900000in}}%
\pgfusepath{clip}%
\pgfsetbuttcap%
\pgfsetroundjoin%
\definecolor{currentfill}{rgb}{0.121569,0.466667,0.705882}%
\pgfsetfillcolor{currentfill}%
\pgfsetlinewidth{1.003750pt}%
\definecolor{currentstroke}{rgb}{0.121569,0.466667,0.705882}%
\pgfsetstrokecolor{currentstroke}%
\pgfsetdash{}{0pt}%
\pgfpathmoveto{\pgfqpoint{1.786536in}{3.297324in}}%
\pgfpathcurveto{\pgfqpoint{1.797586in}{3.297324in}}{\pgfqpoint{1.808185in}{3.301715in}}{\pgfqpoint{1.815998in}{3.309528in}}%
\pgfpathcurveto{\pgfqpoint{1.823812in}{3.317342in}}{\pgfqpoint{1.828202in}{3.327941in}}{\pgfqpoint{1.828202in}{3.338991in}}%
\pgfpathcurveto{\pgfqpoint{1.828202in}{3.350041in}}{\pgfqpoint{1.823812in}{3.360640in}}{\pgfqpoint{1.815998in}{3.368454in}}%
\pgfpathcurveto{\pgfqpoint{1.808185in}{3.376267in}}{\pgfqpoint{1.797586in}{3.380658in}}{\pgfqpoint{1.786536in}{3.380658in}}%
\pgfpathcurveto{\pgfqpoint{1.775485in}{3.380658in}}{\pgfqpoint{1.764886in}{3.376267in}}{\pgfqpoint{1.757073in}{3.368454in}}%
\pgfpathcurveto{\pgfqpoint{1.749259in}{3.360640in}}{\pgfqpoint{1.744869in}{3.350041in}}{\pgfqpoint{1.744869in}{3.338991in}}%
\pgfpathcurveto{\pgfqpoint{1.744869in}{3.327941in}}{\pgfqpoint{1.749259in}{3.317342in}}{\pgfqpoint{1.757073in}{3.309528in}}%
\pgfpathcurveto{\pgfqpoint{1.764886in}{3.301715in}}{\pgfqpoint{1.775485in}{3.297324in}}{\pgfqpoint{1.786536in}{3.297324in}}%
\pgfpathclose%
\pgfusepath{stroke,fill}%
\end{pgfscope}%
\begin{pgfscope}%
\pgfpathrectangle{\pgfqpoint{0.600000in}{0.600000in}}{\pgfqpoint{3.900000in}{3.900000in}}%
\pgfusepath{clip}%
\pgfsetbuttcap%
\pgfsetroundjoin%
\definecolor{currentfill}{rgb}{0.121569,0.466667,0.705882}%
\pgfsetfillcolor{currentfill}%
\pgfsetlinewidth{1.003750pt}%
\definecolor{currentstroke}{rgb}{0.121569,0.466667,0.705882}%
\pgfsetstrokecolor{currentstroke}%
\pgfsetdash{}{0pt}%
\pgfpathmoveto{\pgfqpoint{2.745372in}{2.356594in}}%
\pgfpathcurveto{\pgfqpoint{2.756422in}{2.356594in}}{\pgfqpoint{2.767021in}{2.360984in}}{\pgfqpoint{2.774834in}{2.368798in}}%
\pgfpathcurveto{\pgfqpoint{2.782648in}{2.376612in}}{\pgfqpoint{2.787038in}{2.387211in}}{\pgfqpoint{2.787038in}{2.398261in}}%
\pgfpathcurveto{\pgfqpoint{2.787038in}{2.409311in}}{\pgfqpoint{2.782648in}{2.419910in}}{\pgfqpoint{2.774834in}{2.427724in}}%
\pgfpathcurveto{\pgfqpoint{2.767021in}{2.435537in}}{\pgfqpoint{2.756422in}{2.439928in}}{\pgfqpoint{2.745372in}{2.439928in}}%
\pgfpathcurveto{\pgfqpoint{2.734322in}{2.439928in}}{\pgfqpoint{2.723723in}{2.435537in}}{\pgfqpoint{2.715909in}{2.427724in}}%
\pgfpathcurveto{\pgfqpoint{2.708095in}{2.419910in}}{\pgfqpoint{2.703705in}{2.409311in}}{\pgfqpoint{2.703705in}{2.398261in}}%
\pgfpathcurveto{\pgfqpoint{2.703705in}{2.387211in}}{\pgfqpoint{2.708095in}{2.376612in}}{\pgfqpoint{2.715909in}{2.368798in}}%
\pgfpathcurveto{\pgfqpoint{2.723723in}{2.360984in}}{\pgfqpoint{2.734322in}{2.356594in}}{\pgfqpoint{2.745372in}{2.356594in}}%
\pgfpathclose%
\pgfusepath{stroke,fill}%
\end{pgfscope}%
\begin{pgfscope}%
\pgfpathrectangle{\pgfqpoint{0.600000in}{0.600000in}}{\pgfqpoint{3.900000in}{3.900000in}}%
\pgfusepath{clip}%
\pgfsetbuttcap%
\pgfsetroundjoin%
\definecolor{currentfill}{rgb}{0.121569,0.466667,0.705882}%
\pgfsetfillcolor{currentfill}%
\pgfsetlinewidth{1.003750pt}%
\definecolor{currentstroke}{rgb}{0.121569,0.466667,0.705882}%
\pgfsetstrokecolor{currentstroke}%
\pgfsetdash{}{0pt}%
\pgfpathmoveto{\pgfqpoint{2.614617in}{2.559775in}}%
\pgfpathcurveto{\pgfqpoint{2.625667in}{2.559775in}}{\pgfqpoint{2.636266in}{2.564166in}}{\pgfqpoint{2.644080in}{2.571979in}}%
\pgfpathcurveto{\pgfqpoint{2.651893in}{2.579793in}}{\pgfqpoint{2.656284in}{2.590392in}}{\pgfqpoint{2.656284in}{2.601442in}}%
\pgfpathcurveto{\pgfqpoint{2.656284in}{2.612492in}}{\pgfqpoint{2.651893in}{2.623091in}}{\pgfqpoint{2.644080in}{2.630905in}}%
\pgfpathcurveto{\pgfqpoint{2.636266in}{2.638718in}}{\pgfqpoint{2.625667in}{2.643109in}}{\pgfqpoint{2.614617in}{2.643109in}}%
\pgfpathcurveto{\pgfqpoint{2.603567in}{2.643109in}}{\pgfqpoint{2.592968in}{2.638718in}}{\pgfqpoint{2.585154in}{2.630905in}}%
\pgfpathcurveto{\pgfqpoint{2.577341in}{2.623091in}}{\pgfqpoint{2.572950in}{2.612492in}}{\pgfqpoint{2.572950in}{2.601442in}}%
\pgfpathcurveto{\pgfqpoint{2.572950in}{2.590392in}}{\pgfqpoint{2.577341in}{2.579793in}}{\pgfqpoint{2.585154in}{2.571979in}}%
\pgfpathcurveto{\pgfqpoint{2.592968in}{2.564166in}}{\pgfqpoint{2.603567in}{2.559775in}}{\pgfqpoint{2.614617in}{2.559775in}}%
\pgfpathclose%
\pgfusepath{stroke,fill}%
\end{pgfscope}%
\begin{pgfscope}%
\pgfpathrectangle{\pgfqpoint{0.600000in}{0.600000in}}{\pgfqpoint{3.900000in}{3.900000in}}%
\pgfusepath{clip}%
\pgfsetbuttcap%
\pgfsetroundjoin%
\definecolor{currentfill}{rgb}{0.121569,0.466667,0.705882}%
\pgfsetfillcolor{currentfill}%
\pgfsetlinewidth{1.003750pt}%
\definecolor{currentstroke}{rgb}{0.121569,0.466667,0.705882}%
\pgfsetstrokecolor{currentstroke}%
\pgfsetdash{}{0pt}%
\pgfpathmoveto{\pgfqpoint{3.469833in}{2.114014in}}%
\pgfpathcurveto{\pgfqpoint{3.480883in}{2.114014in}}{\pgfqpoint{3.491482in}{2.118404in}}{\pgfqpoint{3.499296in}{2.126218in}}%
\pgfpathcurveto{\pgfqpoint{3.507109in}{2.134032in}}{\pgfqpoint{3.511499in}{2.144631in}}{\pgfqpoint{3.511499in}{2.155681in}}%
\pgfpathcurveto{\pgfqpoint{3.511499in}{2.166731in}}{\pgfqpoint{3.507109in}{2.177330in}}{\pgfqpoint{3.499296in}{2.185144in}}%
\pgfpathcurveto{\pgfqpoint{3.491482in}{2.192957in}}{\pgfqpoint{3.480883in}{2.197347in}}{\pgfqpoint{3.469833in}{2.197347in}}%
\pgfpathcurveto{\pgfqpoint{3.458783in}{2.197347in}}{\pgfqpoint{3.448184in}{2.192957in}}{\pgfqpoint{3.440370in}{2.185144in}}%
\pgfpathcurveto{\pgfqpoint{3.432556in}{2.177330in}}{\pgfqpoint{3.428166in}{2.166731in}}{\pgfqpoint{3.428166in}{2.155681in}}%
\pgfpathcurveto{\pgfqpoint{3.428166in}{2.144631in}}{\pgfqpoint{3.432556in}{2.134032in}}{\pgfqpoint{3.440370in}{2.126218in}}%
\pgfpathcurveto{\pgfqpoint{3.448184in}{2.118404in}}{\pgfqpoint{3.458783in}{2.114014in}}{\pgfqpoint{3.469833in}{2.114014in}}%
\pgfpathclose%
\pgfusepath{stroke,fill}%
\end{pgfscope}%
\begin{pgfscope}%
\pgfpathrectangle{\pgfqpoint{0.600000in}{0.600000in}}{\pgfqpoint{3.900000in}{3.900000in}}%
\pgfusepath{clip}%
\pgfsetbuttcap%
\pgfsetroundjoin%
\definecolor{currentfill}{rgb}{0.121569,0.466667,0.705882}%
\pgfsetfillcolor{currentfill}%
\pgfsetlinewidth{1.003750pt}%
\definecolor{currentstroke}{rgb}{0.121569,0.466667,0.705882}%
\pgfsetstrokecolor{currentstroke}%
\pgfsetdash{}{0pt}%
\pgfpathmoveto{\pgfqpoint{2.805363in}{2.256592in}}%
\pgfpathcurveto{\pgfqpoint{2.816413in}{2.256592in}}{\pgfqpoint{2.827012in}{2.260982in}}{\pgfqpoint{2.834825in}{2.268796in}}%
\pgfpathcurveto{\pgfqpoint{2.842639in}{2.276609in}}{\pgfqpoint{2.847029in}{2.287208in}}{\pgfqpoint{2.847029in}{2.298259in}}%
\pgfpathcurveto{\pgfqpoint{2.847029in}{2.309309in}}{\pgfqpoint{2.842639in}{2.319908in}}{\pgfqpoint{2.834825in}{2.327721in}}%
\pgfpathcurveto{\pgfqpoint{2.827012in}{2.335535in}}{\pgfqpoint{2.816413in}{2.339925in}}{\pgfqpoint{2.805363in}{2.339925in}}%
\pgfpathcurveto{\pgfqpoint{2.794312in}{2.339925in}}{\pgfqpoint{2.783713in}{2.335535in}}{\pgfqpoint{2.775900in}{2.327721in}}%
\pgfpathcurveto{\pgfqpoint{2.768086in}{2.319908in}}{\pgfqpoint{2.763696in}{2.309309in}}{\pgfqpoint{2.763696in}{2.298259in}}%
\pgfpathcurveto{\pgfqpoint{2.763696in}{2.287208in}}{\pgfqpoint{2.768086in}{2.276609in}}{\pgfqpoint{2.775900in}{2.268796in}}%
\pgfpathcurveto{\pgfqpoint{2.783713in}{2.260982in}}{\pgfqpoint{2.794312in}{2.256592in}}{\pgfqpoint{2.805363in}{2.256592in}}%
\pgfpathclose%
\pgfusepath{stroke,fill}%
\end{pgfscope}%
\begin{pgfscope}%
\pgfpathrectangle{\pgfqpoint{0.600000in}{0.600000in}}{\pgfqpoint{3.900000in}{3.900000in}}%
\pgfusepath{clip}%
\pgfsetbuttcap%
\pgfsetroundjoin%
\definecolor{currentfill}{rgb}{0.121569,0.466667,0.705882}%
\pgfsetfillcolor{currentfill}%
\pgfsetlinewidth{1.003750pt}%
\definecolor{currentstroke}{rgb}{0.121569,0.466667,0.705882}%
\pgfsetstrokecolor{currentstroke}%
\pgfsetdash{}{0pt}%
\pgfpathmoveto{\pgfqpoint{3.139039in}{3.107293in}}%
\pgfpathcurveto{\pgfqpoint{3.150089in}{3.107293in}}{\pgfqpoint{3.160688in}{3.111683in}}{\pgfqpoint{3.168502in}{3.119497in}}%
\pgfpathcurveto{\pgfqpoint{3.176315in}{3.127311in}}{\pgfqpoint{3.180706in}{3.137910in}}{\pgfqpoint{3.180706in}{3.148960in}}%
\pgfpathcurveto{\pgfqpoint{3.180706in}{3.160010in}}{\pgfqpoint{3.176315in}{3.170609in}}{\pgfqpoint{3.168502in}{3.178423in}}%
\pgfpathcurveto{\pgfqpoint{3.160688in}{3.186236in}}{\pgfqpoint{3.150089in}{3.190626in}}{\pgfqpoint{3.139039in}{3.190626in}}%
\pgfpathcurveto{\pgfqpoint{3.127989in}{3.190626in}}{\pgfqpoint{3.117390in}{3.186236in}}{\pgfqpoint{3.109576in}{3.178423in}}%
\pgfpathcurveto{\pgfqpoint{3.101763in}{3.170609in}}{\pgfqpoint{3.097372in}{3.160010in}}{\pgfqpoint{3.097372in}{3.148960in}}%
\pgfpathcurveto{\pgfqpoint{3.097372in}{3.137910in}}{\pgfqpoint{3.101763in}{3.127311in}}{\pgfqpoint{3.109576in}{3.119497in}}%
\pgfpathcurveto{\pgfqpoint{3.117390in}{3.111683in}}{\pgfqpoint{3.127989in}{3.107293in}}{\pgfqpoint{3.139039in}{3.107293in}}%
\pgfpathclose%
\pgfusepath{stroke,fill}%
\end{pgfscope}%
\begin{pgfscope}%
\pgfpathrectangle{\pgfqpoint{0.600000in}{0.600000in}}{\pgfqpoint{3.900000in}{3.900000in}}%
\pgfusepath{clip}%
\pgfsetbuttcap%
\pgfsetroundjoin%
\definecolor{currentfill}{rgb}{0.121569,0.466667,0.705882}%
\pgfsetfillcolor{currentfill}%
\pgfsetlinewidth{1.003750pt}%
\definecolor{currentstroke}{rgb}{0.121569,0.466667,0.705882}%
\pgfsetstrokecolor{currentstroke}%
\pgfsetdash{}{0pt}%
\pgfpathmoveto{\pgfqpoint{2.222674in}{2.975940in}}%
\pgfpathcurveto{\pgfqpoint{2.233724in}{2.975940in}}{\pgfqpoint{2.244323in}{2.980331in}}{\pgfqpoint{2.252137in}{2.988144in}}%
\pgfpathcurveto{\pgfqpoint{2.259951in}{2.995958in}}{\pgfqpoint{2.264341in}{3.006557in}}{\pgfqpoint{2.264341in}{3.017607in}}%
\pgfpathcurveto{\pgfqpoint{2.264341in}{3.028657in}}{\pgfqpoint{2.259951in}{3.039256in}}{\pgfqpoint{2.252137in}{3.047070in}}%
\pgfpathcurveto{\pgfqpoint{2.244323in}{3.054883in}}{\pgfqpoint{2.233724in}{3.059274in}}{\pgfqpoint{2.222674in}{3.059274in}}%
\pgfpathcurveto{\pgfqpoint{2.211624in}{3.059274in}}{\pgfqpoint{2.201025in}{3.054883in}}{\pgfqpoint{2.193212in}{3.047070in}}%
\pgfpathcurveto{\pgfqpoint{2.185398in}{3.039256in}}{\pgfqpoint{2.181008in}{3.028657in}}{\pgfqpoint{2.181008in}{3.017607in}}%
\pgfpathcurveto{\pgfqpoint{2.181008in}{3.006557in}}{\pgfqpoint{2.185398in}{2.995958in}}{\pgfqpoint{2.193212in}{2.988144in}}%
\pgfpathcurveto{\pgfqpoint{2.201025in}{2.980331in}}{\pgfqpoint{2.211624in}{2.975940in}}{\pgfqpoint{2.222674in}{2.975940in}}%
\pgfpathclose%
\pgfusepath{stroke,fill}%
\end{pgfscope}%
\begin{pgfscope}%
\pgfpathrectangle{\pgfqpoint{0.600000in}{0.600000in}}{\pgfqpoint{3.900000in}{3.900000in}}%
\pgfusepath{clip}%
\pgfsetbuttcap%
\pgfsetroundjoin%
\definecolor{currentfill}{rgb}{0.121569,0.466667,0.705882}%
\pgfsetfillcolor{currentfill}%
\pgfsetlinewidth{1.003750pt}%
\definecolor{currentstroke}{rgb}{0.121569,0.466667,0.705882}%
\pgfsetstrokecolor{currentstroke}%
\pgfsetdash{}{0pt}%
\pgfpathmoveto{\pgfqpoint{2.064925in}{3.024915in}}%
\pgfpathcurveto{\pgfqpoint{2.075975in}{3.024915in}}{\pgfqpoint{2.086574in}{3.029305in}}{\pgfqpoint{2.094388in}{3.037119in}}%
\pgfpathcurveto{\pgfqpoint{2.102202in}{3.044932in}}{\pgfqpoint{2.106592in}{3.055532in}}{\pgfqpoint{2.106592in}{3.066582in}}%
\pgfpathcurveto{\pgfqpoint{2.106592in}{3.077632in}}{\pgfqpoint{2.102202in}{3.088231in}}{\pgfqpoint{2.094388in}{3.096044in}}%
\pgfpathcurveto{\pgfqpoint{2.086574in}{3.103858in}}{\pgfqpoint{2.075975in}{3.108248in}}{\pgfqpoint{2.064925in}{3.108248in}}%
\pgfpathcurveto{\pgfqpoint{2.053875in}{3.108248in}}{\pgfqpoint{2.043276in}{3.103858in}}{\pgfqpoint{2.035462in}{3.096044in}}%
\pgfpathcurveto{\pgfqpoint{2.027649in}{3.088231in}}{\pgfqpoint{2.023259in}{3.077632in}}{\pgfqpoint{2.023259in}{3.066582in}}%
\pgfpathcurveto{\pgfqpoint{2.023259in}{3.055532in}}{\pgfqpoint{2.027649in}{3.044932in}}{\pgfqpoint{2.035462in}{3.037119in}}%
\pgfpathcurveto{\pgfqpoint{2.043276in}{3.029305in}}{\pgfqpoint{2.053875in}{3.024915in}}{\pgfqpoint{2.064925in}{3.024915in}}%
\pgfpathclose%
\pgfusepath{stroke,fill}%
\end{pgfscope}%
\begin{pgfscope}%
\pgfpathrectangle{\pgfqpoint{0.600000in}{0.600000in}}{\pgfqpoint{3.900000in}{3.900000in}}%
\pgfusepath{clip}%
\pgfsetbuttcap%
\pgfsetroundjoin%
\definecolor{currentfill}{rgb}{0.121569,0.466667,0.705882}%
\pgfsetfillcolor{currentfill}%
\pgfsetlinewidth{1.003750pt}%
\definecolor{currentstroke}{rgb}{0.121569,0.466667,0.705882}%
\pgfsetstrokecolor{currentstroke}%
\pgfsetdash{}{0pt}%
\pgfpathmoveto{\pgfqpoint{2.471108in}{1.886492in}}%
\pgfpathcurveto{\pgfqpoint{2.482158in}{1.886492in}}{\pgfqpoint{2.492757in}{1.890882in}}{\pgfqpoint{2.500570in}{1.898695in}}%
\pgfpathcurveto{\pgfqpoint{2.508384in}{1.906509in}}{\pgfqpoint{2.512774in}{1.917108in}}{\pgfqpoint{2.512774in}{1.928158in}}%
\pgfpathcurveto{\pgfqpoint{2.512774in}{1.939208in}}{\pgfqpoint{2.508384in}{1.949807in}}{\pgfqpoint{2.500570in}{1.957621in}}%
\pgfpathcurveto{\pgfqpoint{2.492757in}{1.965435in}}{\pgfqpoint{2.482158in}{1.969825in}}{\pgfqpoint{2.471108in}{1.969825in}}%
\pgfpathcurveto{\pgfqpoint{2.460058in}{1.969825in}}{\pgfqpoint{2.449459in}{1.965435in}}{\pgfqpoint{2.441645in}{1.957621in}}%
\pgfpathcurveto{\pgfqpoint{2.433831in}{1.949807in}}{\pgfqpoint{2.429441in}{1.939208in}}{\pgfqpoint{2.429441in}{1.928158in}}%
\pgfpathcurveto{\pgfqpoint{2.429441in}{1.917108in}}{\pgfqpoint{2.433831in}{1.906509in}}{\pgfqpoint{2.441645in}{1.898695in}}%
\pgfpathcurveto{\pgfqpoint{2.449459in}{1.890882in}}{\pgfqpoint{2.460058in}{1.886492in}}{\pgfqpoint{2.471108in}{1.886492in}}%
\pgfpathclose%
\pgfusepath{stroke,fill}%
\end{pgfscope}%
\begin{pgfscope}%
\pgfpathrectangle{\pgfqpoint{0.600000in}{0.600000in}}{\pgfqpoint{3.900000in}{3.900000in}}%
\pgfusepath{clip}%
\pgfsetbuttcap%
\pgfsetroundjoin%
\definecolor{currentfill}{rgb}{0.121569,0.466667,0.705882}%
\pgfsetfillcolor{currentfill}%
\pgfsetlinewidth{1.003750pt}%
\definecolor{currentstroke}{rgb}{0.121569,0.466667,0.705882}%
\pgfsetstrokecolor{currentstroke}%
\pgfsetdash{}{0pt}%
\pgfpathmoveto{\pgfqpoint{2.355870in}{2.513455in}}%
\pgfpathcurveto{\pgfqpoint{2.366920in}{2.513455in}}{\pgfqpoint{2.377519in}{2.517845in}}{\pgfqpoint{2.385333in}{2.525658in}}%
\pgfpathcurveto{\pgfqpoint{2.393146in}{2.533472in}}{\pgfqpoint{2.397537in}{2.544071in}}{\pgfqpoint{2.397537in}{2.555121in}}%
\pgfpathcurveto{\pgfqpoint{2.397537in}{2.566171in}}{\pgfqpoint{2.393146in}{2.576770in}}{\pgfqpoint{2.385333in}{2.584584in}}%
\pgfpathcurveto{\pgfqpoint{2.377519in}{2.592398in}}{\pgfqpoint{2.366920in}{2.596788in}}{\pgfqpoint{2.355870in}{2.596788in}}%
\pgfpathcurveto{\pgfqpoint{2.344820in}{2.596788in}}{\pgfqpoint{2.334221in}{2.592398in}}{\pgfqpoint{2.326407in}{2.584584in}}%
\pgfpathcurveto{\pgfqpoint{2.318594in}{2.576770in}}{\pgfqpoint{2.314203in}{2.566171in}}{\pgfqpoint{2.314203in}{2.555121in}}%
\pgfpathcurveto{\pgfqpoint{2.314203in}{2.544071in}}{\pgfqpoint{2.318594in}{2.533472in}}{\pgfqpoint{2.326407in}{2.525658in}}%
\pgfpathcurveto{\pgfqpoint{2.334221in}{2.517845in}}{\pgfqpoint{2.344820in}{2.513455in}}{\pgfqpoint{2.355870in}{2.513455in}}%
\pgfpathclose%
\pgfusepath{stroke,fill}%
\end{pgfscope}%
\begin{pgfscope}%
\pgfpathrectangle{\pgfqpoint{0.600000in}{0.600000in}}{\pgfqpoint{3.900000in}{3.900000in}}%
\pgfusepath{clip}%
\pgfsetbuttcap%
\pgfsetroundjoin%
\definecolor{currentfill}{rgb}{0.121569,0.466667,0.705882}%
\pgfsetfillcolor{currentfill}%
\pgfsetlinewidth{1.003750pt}%
\definecolor{currentstroke}{rgb}{0.121569,0.466667,0.705882}%
\pgfsetstrokecolor{currentstroke}%
\pgfsetdash{}{0pt}%
\pgfpathmoveto{\pgfqpoint{3.154371in}{2.338674in}}%
\pgfpathcurveto{\pgfqpoint{3.165421in}{2.338674in}}{\pgfqpoint{3.176020in}{2.343064in}}{\pgfqpoint{3.183833in}{2.350878in}}%
\pgfpathcurveto{\pgfqpoint{3.191647in}{2.358691in}}{\pgfqpoint{3.196037in}{2.369290in}}{\pgfqpoint{3.196037in}{2.380340in}}%
\pgfpathcurveto{\pgfqpoint{3.196037in}{2.391390in}}{\pgfqpoint{3.191647in}{2.401989in}}{\pgfqpoint{3.183833in}{2.409803in}}%
\pgfpathcurveto{\pgfqpoint{3.176020in}{2.417617in}}{\pgfqpoint{3.165421in}{2.422007in}}{\pgfqpoint{3.154371in}{2.422007in}}%
\pgfpathcurveto{\pgfqpoint{3.143321in}{2.422007in}}{\pgfqpoint{3.132722in}{2.417617in}}{\pgfqpoint{3.124908in}{2.409803in}}%
\pgfpathcurveto{\pgfqpoint{3.117094in}{2.401989in}}{\pgfqpoint{3.112704in}{2.391390in}}{\pgfqpoint{3.112704in}{2.380340in}}%
\pgfpathcurveto{\pgfqpoint{3.112704in}{2.369290in}}{\pgfqpoint{3.117094in}{2.358691in}}{\pgfqpoint{3.124908in}{2.350878in}}%
\pgfpathcurveto{\pgfqpoint{3.132722in}{2.343064in}}{\pgfqpoint{3.143321in}{2.338674in}}{\pgfqpoint{3.154371in}{2.338674in}}%
\pgfpathclose%
\pgfusepath{stroke,fill}%
\end{pgfscope}%
\begin{pgfscope}%
\pgfpathrectangle{\pgfqpoint{0.600000in}{0.600000in}}{\pgfqpoint{3.900000in}{3.900000in}}%
\pgfusepath{clip}%
\pgfsetbuttcap%
\pgfsetroundjoin%
\definecolor{currentfill}{rgb}{0.121569,0.466667,0.705882}%
\pgfsetfillcolor{currentfill}%
\pgfsetlinewidth{1.003750pt}%
\definecolor{currentstroke}{rgb}{0.121569,0.466667,0.705882}%
\pgfsetstrokecolor{currentstroke}%
\pgfsetdash{}{0pt}%
\pgfpathmoveto{\pgfqpoint{1.794870in}{2.765158in}}%
\pgfpathcurveto{\pgfqpoint{1.805920in}{2.765158in}}{\pgfqpoint{1.816520in}{2.769548in}}{\pgfqpoint{1.824333in}{2.777362in}}%
\pgfpathcurveto{\pgfqpoint{1.832147in}{2.785176in}}{\pgfqpoint{1.836537in}{2.795775in}}{\pgfqpoint{1.836537in}{2.806825in}}%
\pgfpathcurveto{\pgfqpoint{1.836537in}{2.817875in}}{\pgfqpoint{1.832147in}{2.828474in}}{\pgfqpoint{1.824333in}{2.836288in}}%
\pgfpathcurveto{\pgfqpoint{1.816520in}{2.844101in}}{\pgfqpoint{1.805920in}{2.848492in}}{\pgfqpoint{1.794870in}{2.848492in}}%
\pgfpathcurveto{\pgfqpoint{1.783820in}{2.848492in}}{\pgfqpoint{1.773221in}{2.844101in}}{\pgfqpoint{1.765408in}{2.836288in}}%
\pgfpathcurveto{\pgfqpoint{1.757594in}{2.828474in}}{\pgfqpoint{1.753204in}{2.817875in}}{\pgfqpoint{1.753204in}{2.806825in}}%
\pgfpathcurveto{\pgfqpoint{1.753204in}{2.795775in}}{\pgfqpoint{1.757594in}{2.785176in}}{\pgfqpoint{1.765408in}{2.777362in}}%
\pgfpathcurveto{\pgfqpoint{1.773221in}{2.769548in}}{\pgfqpoint{1.783820in}{2.765158in}}{\pgfqpoint{1.794870in}{2.765158in}}%
\pgfpathclose%
\pgfusepath{stroke,fill}%
\end{pgfscope}%
\begin{pgfscope}%
\pgfpathrectangle{\pgfqpoint{0.600000in}{0.600000in}}{\pgfqpoint{3.900000in}{3.900000in}}%
\pgfusepath{clip}%
\pgfsetbuttcap%
\pgfsetroundjoin%
\definecolor{currentfill}{rgb}{0.121569,0.466667,0.705882}%
\pgfsetfillcolor{currentfill}%
\pgfsetlinewidth{1.003750pt}%
\definecolor{currentstroke}{rgb}{0.121569,0.466667,0.705882}%
\pgfsetstrokecolor{currentstroke}%
\pgfsetdash{}{0pt}%
\pgfpathmoveto{\pgfqpoint{2.312642in}{2.724852in}}%
\pgfpathcurveto{\pgfqpoint{2.323692in}{2.724852in}}{\pgfqpoint{2.334292in}{2.729243in}}{\pgfqpoint{2.342105in}{2.737056in}}%
\pgfpathcurveto{\pgfqpoint{2.349919in}{2.744870in}}{\pgfqpoint{2.354309in}{2.755469in}}{\pgfqpoint{2.354309in}{2.766519in}}%
\pgfpathcurveto{\pgfqpoint{2.354309in}{2.777569in}}{\pgfqpoint{2.349919in}{2.788168in}}{\pgfqpoint{2.342105in}{2.795982in}}%
\pgfpathcurveto{\pgfqpoint{2.334292in}{2.803796in}}{\pgfqpoint{2.323692in}{2.808186in}}{\pgfqpoint{2.312642in}{2.808186in}}%
\pgfpathcurveto{\pgfqpoint{2.301592in}{2.808186in}}{\pgfqpoint{2.290993in}{2.803796in}}{\pgfqpoint{2.283180in}{2.795982in}}%
\pgfpathcurveto{\pgfqpoint{2.275366in}{2.788168in}}{\pgfqpoint{2.270976in}{2.777569in}}{\pgfqpoint{2.270976in}{2.766519in}}%
\pgfpathcurveto{\pgfqpoint{2.270976in}{2.755469in}}{\pgfqpoint{2.275366in}{2.744870in}}{\pgfqpoint{2.283180in}{2.737056in}}%
\pgfpathcurveto{\pgfqpoint{2.290993in}{2.729243in}}{\pgfqpoint{2.301592in}{2.724852in}}{\pgfqpoint{2.312642in}{2.724852in}}%
\pgfpathclose%
\pgfusepath{stroke,fill}%
\end{pgfscope}%
\begin{pgfscope}%
\pgfpathrectangle{\pgfqpoint{0.600000in}{0.600000in}}{\pgfqpoint{3.900000in}{3.900000in}}%
\pgfusepath{clip}%
\pgfsetbuttcap%
\pgfsetroundjoin%
\definecolor{currentfill}{rgb}{0.121569,0.466667,0.705882}%
\pgfsetfillcolor{currentfill}%
\pgfsetlinewidth{1.003750pt}%
\definecolor{currentstroke}{rgb}{0.121569,0.466667,0.705882}%
\pgfsetstrokecolor{currentstroke}%
\pgfsetdash{}{0pt}%
\pgfpathmoveto{\pgfqpoint{2.313231in}{2.842483in}}%
\pgfpathcurveto{\pgfqpoint{2.324281in}{2.842483in}}{\pgfqpoint{2.334880in}{2.846874in}}{\pgfqpoint{2.342693in}{2.854687in}}%
\pgfpathcurveto{\pgfqpoint{2.350507in}{2.862501in}}{\pgfqpoint{2.354897in}{2.873100in}}{\pgfqpoint{2.354897in}{2.884150in}}%
\pgfpathcurveto{\pgfqpoint{2.354897in}{2.895200in}}{\pgfqpoint{2.350507in}{2.905799in}}{\pgfqpoint{2.342693in}{2.913613in}}%
\pgfpathcurveto{\pgfqpoint{2.334880in}{2.921426in}}{\pgfqpoint{2.324281in}{2.925817in}}{\pgfqpoint{2.313231in}{2.925817in}}%
\pgfpathcurveto{\pgfqpoint{2.302180in}{2.925817in}}{\pgfqpoint{2.291581in}{2.921426in}}{\pgfqpoint{2.283768in}{2.913613in}}%
\pgfpathcurveto{\pgfqpoint{2.275954in}{2.905799in}}{\pgfqpoint{2.271564in}{2.895200in}}{\pgfqpoint{2.271564in}{2.884150in}}%
\pgfpathcurveto{\pgfqpoint{2.271564in}{2.873100in}}{\pgfqpoint{2.275954in}{2.862501in}}{\pgfqpoint{2.283768in}{2.854687in}}%
\pgfpathcurveto{\pgfqpoint{2.291581in}{2.846874in}}{\pgfqpoint{2.302180in}{2.842483in}}{\pgfqpoint{2.313231in}{2.842483in}}%
\pgfpathclose%
\pgfusepath{stroke,fill}%
\end{pgfscope}%
\begin{pgfscope}%
\pgfpathrectangle{\pgfqpoint{0.600000in}{0.600000in}}{\pgfqpoint{3.900000in}{3.900000in}}%
\pgfusepath{clip}%
\pgfsetbuttcap%
\pgfsetroundjoin%
\definecolor{currentfill}{rgb}{0.121569,0.466667,0.705882}%
\pgfsetfillcolor{currentfill}%
\pgfsetlinewidth{1.003750pt}%
\definecolor{currentstroke}{rgb}{0.121569,0.466667,0.705882}%
\pgfsetstrokecolor{currentstroke}%
\pgfsetdash{}{0pt}%
\pgfpathmoveto{\pgfqpoint{2.852286in}{2.152200in}}%
\pgfpathcurveto{\pgfqpoint{2.863336in}{2.152200in}}{\pgfqpoint{2.873935in}{2.156590in}}{\pgfqpoint{2.881749in}{2.164404in}}%
\pgfpathcurveto{\pgfqpoint{2.889563in}{2.172218in}}{\pgfqpoint{2.893953in}{2.182817in}}{\pgfqpoint{2.893953in}{2.193867in}}%
\pgfpathcurveto{\pgfqpoint{2.893953in}{2.204917in}}{\pgfqpoint{2.889563in}{2.215516in}}{\pgfqpoint{2.881749in}{2.223329in}}%
\pgfpathcurveto{\pgfqpoint{2.873935in}{2.231143in}}{\pgfqpoint{2.863336in}{2.235533in}}{\pgfqpoint{2.852286in}{2.235533in}}%
\pgfpathcurveto{\pgfqpoint{2.841236in}{2.235533in}}{\pgfqpoint{2.830637in}{2.231143in}}{\pgfqpoint{2.822824in}{2.223329in}}%
\pgfpathcurveto{\pgfqpoint{2.815010in}{2.215516in}}{\pgfqpoint{2.810620in}{2.204917in}}{\pgfqpoint{2.810620in}{2.193867in}}%
\pgfpathcurveto{\pgfqpoint{2.810620in}{2.182817in}}{\pgfqpoint{2.815010in}{2.172218in}}{\pgfqpoint{2.822824in}{2.164404in}}%
\pgfpathcurveto{\pgfqpoint{2.830637in}{2.156590in}}{\pgfqpoint{2.841236in}{2.152200in}}{\pgfqpoint{2.852286in}{2.152200in}}%
\pgfpathclose%
\pgfusepath{stroke,fill}%
\end{pgfscope}%
\begin{pgfscope}%
\pgfpathrectangle{\pgfqpoint{0.600000in}{0.600000in}}{\pgfqpoint{3.900000in}{3.900000in}}%
\pgfusepath{clip}%
\pgfsetbuttcap%
\pgfsetroundjoin%
\definecolor{currentfill}{rgb}{0.121569,0.466667,0.705882}%
\pgfsetfillcolor{currentfill}%
\pgfsetlinewidth{1.003750pt}%
\definecolor{currentstroke}{rgb}{0.121569,0.466667,0.705882}%
\pgfsetstrokecolor{currentstroke}%
\pgfsetdash{}{0pt}%
\pgfpathmoveto{\pgfqpoint{2.706872in}{2.072191in}}%
\pgfpathcurveto{\pgfqpoint{2.717922in}{2.072191in}}{\pgfqpoint{2.728521in}{2.076582in}}{\pgfqpoint{2.736335in}{2.084395in}}%
\pgfpathcurveto{\pgfqpoint{2.744148in}{2.092209in}}{\pgfqpoint{2.748538in}{2.102808in}}{\pgfqpoint{2.748538in}{2.113858in}}%
\pgfpathcurveto{\pgfqpoint{2.748538in}{2.124908in}}{\pgfqpoint{2.744148in}{2.135507in}}{\pgfqpoint{2.736335in}{2.143321in}}%
\pgfpathcurveto{\pgfqpoint{2.728521in}{2.151134in}}{\pgfqpoint{2.717922in}{2.155525in}}{\pgfqpoint{2.706872in}{2.155525in}}%
\pgfpathcurveto{\pgfqpoint{2.695822in}{2.155525in}}{\pgfqpoint{2.685223in}{2.151134in}}{\pgfqpoint{2.677409in}{2.143321in}}%
\pgfpathcurveto{\pgfqpoint{2.669595in}{2.135507in}}{\pgfqpoint{2.665205in}{2.124908in}}{\pgfqpoint{2.665205in}{2.113858in}}%
\pgfpathcurveto{\pgfqpoint{2.665205in}{2.102808in}}{\pgfqpoint{2.669595in}{2.092209in}}{\pgfqpoint{2.677409in}{2.084395in}}%
\pgfpathcurveto{\pgfqpoint{2.685223in}{2.076582in}}{\pgfqpoint{2.695822in}{2.072191in}}{\pgfqpoint{2.706872in}{2.072191in}}%
\pgfpathclose%
\pgfusepath{stroke,fill}%
\end{pgfscope}%
\begin{pgfscope}%
\pgfpathrectangle{\pgfqpoint{0.600000in}{0.600000in}}{\pgfqpoint{3.900000in}{3.900000in}}%
\pgfusepath{clip}%
\pgfsetbuttcap%
\pgfsetroundjoin%
\definecolor{currentfill}{rgb}{0.121569,0.466667,0.705882}%
\pgfsetfillcolor{currentfill}%
\pgfsetlinewidth{1.003750pt}%
\definecolor{currentstroke}{rgb}{0.121569,0.466667,0.705882}%
\pgfsetstrokecolor{currentstroke}%
\pgfsetdash{}{0pt}%
\pgfpathmoveto{\pgfqpoint{2.462328in}{2.689762in}}%
\pgfpathcurveto{\pgfqpoint{2.473378in}{2.689762in}}{\pgfqpoint{2.483977in}{2.694153in}}{\pgfqpoint{2.491791in}{2.701966in}}%
\pgfpathcurveto{\pgfqpoint{2.499604in}{2.709780in}}{\pgfqpoint{2.503995in}{2.720379in}}{\pgfqpoint{2.503995in}{2.731429in}}%
\pgfpathcurveto{\pgfqpoint{2.503995in}{2.742479in}}{\pgfqpoint{2.499604in}{2.753078in}}{\pgfqpoint{2.491791in}{2.760892in}}%
\pgfpathcurveto{\pgfqpoint{2.483977in}{2.768705in}}{\pgfqpoint{2.473378in}{2.773096in}}{\pgfqpoint{2.462328in}{2.773096in}}%
\pgfpathcurveto{\pgfqpoint{2.451278in}{2.773096in}}{\pgfqpoint{2.440679in}{2.768705in}}{\pgfqpoint{2.432865in}{2.760892in}}%
\pgfpathcurveto{\pgfqpoint{2.425052in}{2.753078in}}{\pgfqpoint{2.420661in}{2.742479in}}{\pgfqpoint{2.420661in}{2.731429in}}%
\pgfpathcurveto{\pgfqpoint{2.420661in}{2.720379in}}{\pgfqpoint{2.425052in}{2.709780in}}{\pgfqpoint{2.432865in}{2.701966in}}%
\pgfpathcurveto{\pgfqpoint{2.440679in}{2.694153in}}{\pgfqpoint{2.451278in}{2.689762in}}{\pgfqpoint{2.462328in}{2.689762in}}%
\pgfpathclose%
\pgfusepath{stroke,fill}%
\end{pgfscope}%
\begin{pgfscope}%
\pgfpathrectangle{\pgfqpoint{0.600000in}{0.600000in}}{\pgfqpoint{3.900000in}{3.900000in}}%
\pgfusepath{clip}%
\pgfsetbuttcap%
\pgfsetroundjoin%
\definecolor{currentfill}{rgb}{0.121569,0.466667,0.705882}%
\pgfsetfillcolor{currentfill}%
\pgfsetlinewidth{1.003750pt}%
\definecolor{currentstroke}{rgb}{0.121569,0.466667,0.705882}%
\pgfsetstrokecolor{currentstroke}%
\pgfsetdash{}{0pt}%
\pgfpathmoveto{\pgfqpoint{1.983359in}{2.246612in}}%
\pgfpathcurveto{\pgfqpoint{1.994409in}{2.246612in}}{\pgfqpoint{2.005008in}{2.251002in}}{\pgfqpoint{2.012822in}{2.258816in}}%
\pgfpathcurveto{\pgfqpoint{2.020635in}{2.266630in}}{\pgfqpoint{2.025026in}{2.277229in}}{\pgfqpoint{2.025026in}{2.288279in}}%
\pgfpathcurveto{\pgfqpoint{2.025026in}{2.299329in}}{\pgfqpoint{2.020635in}{2.309928in}}{\pgfqpoint{2.012822in}{2.317742in}}%
\pgfpathcurveto{\pgfqpoint{2.005008in}{2.325555in}}{\pgfqpoint{1.994409in}{2.329946in}}{\pgfqpoint{1.983359in}{2.329946in}}%
\pgfpathcurveto{\pgfqpoint{1.972309in}{2.329946in}}{\pgfqpoint{1.961710in}{2.325555in}}{\pgfqpoint{1.953896in}{2.317742in}}%
\pgfpathcurveto{\pgfqpoint{1.946083in}{2.309928in}}{\pgfqpoint{1.941692in}{2.299329in}}{\pgfqpoint{1.941692in}{2.288279in}}%
\pgfpathcurveto{\pgfqpoint{1.941692in}{2.277229in}}{\pgfqpoint{1.946083in}{2.266630in}}{\pgfqpoint{1.953896in}{2.258816in}}%
\pgfpathcurveto{\pgfqpoint{1.961710in}{2.251002in}}{\pgfqpoint{1.972309in}{2.246612in}}{\pgfqpoint{1.983359in}{2.246612in}}%
\pgfpathclose%
\pgfusepath{stroke,fill}%
\end{pgfscope}%
\begin{pgfscope}%
\pgfpathrectangle{\pgfqpoint{0.600000in}{0.600000in}}{\pgfqpoint{3.900000in}{3.900000in}}%
\pgfusepath{clip}%
\pgfsetbuttcap%
\pgfsetroundjoin%
\definecolor{currentfill}{rgb}{0.121569,0.466667,0.705882}%
\pgfsetfillcolor{currentfill}%
\pgfsetlinewidth{1.003750pt}%
\definecolor{currentstroke}{rgb}{0.121569,0.466667,0.705882}%
\pgfsetstrokecolor{currentstroke}%
\pgfsetdash{}{0pt}%
\pgfpathmoveto{\pgfqpoint{3.182913in}{2.285967in}}%
\pgfpathcurveto{\pgfqpoint{3.193963in}{2.285967in}}{\pgfqpoint{3.204562in}{2.290358in}}{\pgfqpoint{3.212375in}{2.298171in}}%
\pgfpathcurveto{\pgfqpoint{3.220189in}{2.305985in}}{\pgfqpoint{3.224579in}{2.316584in}}{\pgfqpoint{3.224579in}{2.327634in}}%
\pgfpathcurveto{\pgfqpoint{3.224579in}{2.338684in}}{\pgfqpoint{3.220189in}{2.349283in}}{\pgfqpoint{3.212375in}{2.357097in}}%
\pgfpathcurveto{\pgfqpoint{3.204562in}{2.364910in}}{\pgfqpoint{3.193963in}{2.369301in}}{\pgfqpoint{3.182913in}{2.369301in}}%
\pgfpathcurveto{\pgfqpoint{3.171862in}{2.369301in}}{\pgfqpoint{3.161263in}{2.364910in}}{\pgfqpoint{3.153450in}{2.357097in}}%
\pgfpathcurveto{\pgfqpoint{3.145636in}{2.349283in}}{\pgfqpoint{3.141246in}{2.338684in}}{\pgfqpoint{3.141246in}{2.327634in}}%
\pgfpathcurveto{\pgfqpoint{3.141246in}{2.316584in}}{\pgfqpoint{3.145636in}{2.305985in}}{\pgfqpoint{3.153450in}{2.298171in}}%
\pgfpathcurveto{\pgfqpoint{3.161263in}{2.290358in}}{\pgfqpoint{3.171862in}{2.285967in}}{\pgfqpoint{3.182913in}{2.285967in}}%
\pgfpathclose%
\pgfusepath{stroke,fill}%
\end{pgfscope}%
\begin{pgfscope}%
\pgfpathrectangle{\pgfqpoint{0.600000in}{0.600000in}}{\pgfqpoint{3.900000in}{3.900000in}}%
\pgfusepath{clip}%
\pgfsetbuttcap%
\pgfsetroundjoin%
\definecolor{currentfill}{rgb}{0.121569,0.466667,0.705882}%
\pgfsetfillcolor{currentfill}%
\pgfsetlinewidth{1.003750pt}%
\definecolor{currentstroke}{rgb}{0.121569,0.466667,0.705882}%
\pgfsetstrokecolor{currentstroke}%
\pgfsetdash{}{0pt}%
\pgfpathmoveto{\pgfqpoint{1.542219in}{3.015758in}}%
\pgfpathcurveto{\pgfqpoint{1.553269in}{3.015758in}}{\pgfqpoint{1.563868in}{3.020148in}}{\pgfqpoint{1.571681in}{3.027962in}}%
\pgfpathcurveto{\pgfqpoint{1.579495in}{3.035776in}}{\pgfqpoint{1.583885in}{3.046375in}}{\pgfqpoint{1.583885in}{3.057425in}}%
\pgfpathcurveto{\pgfqpoint{1.583885in}{3.068475in}}{\pgfqpoint{1.579495in}{3.079074in}}{\pgfqpoint{1.571681in}{3.086888in}}%
\pgfpathcurveto{\pgfqpoint{1.563868in}{3.094701in}}{\pgfqpoint{1.553269in}{3.099091in}}{\pgfqpoint{1.542219in}{3.099091in}}%
\pgfpathcurveto{\pgfqpoint{1.531169in}{3.099091in}}{\pgfqpoint{1.520570in}{3.094701in}}{\pgfqpoint{1.512756in}{3.086888in}}%
\pgfpathcurveto{\pgfqpoint{1.504942in}{3.079074in}}{\pgfqpoint{1.500552in}{3.068475in}}{\pgfqpoint{1.500552in}{3.057425in}}%
\pgfpathcurveto{\pgfqpoint{1.500552in}{3.046375in}}{\pgfqpoint{1.504942in}{3.035776in}}{\pgfqpoint{1.512756in}{3.027962in}}%
\pgfpathcurveto{\pgfqpoint{1.520570in}{3.020148in}}{\pgfqpoint{1.531169in}{3.015758in}}{\pgfqpoint{1.542219in}{3.015758in}}%
\pgfpathclose%
\pgfusepath{stroke,fill}%
\end{pgfscope}%
\begin{pgfscope}%
\pgfpathrectangle{\pgfqpoint{0.600000in}{0.600000in}}{\pgfqpoint{3.900000in}{3.900000in}}%
\pgfusepath{clip}%
\pgfsetbuttcap%
\pgfsetroundjoin%
\definecolor{currentfill}{rgb}{0.121569,0.466667,0.705882}%
\pgfsetfillcolor{currentfill}%
\pgfsetlinewidth{1.003750pt}%
\definecolor{currentstroke}{rgb}{0.121569,0.466667,0.705882}%
\pgfsetstrokecolor{currentstroke}%
\pgfsetdash{}{0pt}%
\pgfpathmoveto{\pgfqpoint{2.304219in}{1.761330in}}%
\pgfpathcurveto{\pgfqpoint{2.315270in}{1.761330in}}{\pgfqpoint{2.325869in}{1.765720in}}{\pgfqpoint{2.333682in}{1.773534in}}%
\pgfpathcurveto{\pgfqpoint{2.341496in}{1.781347in}}{\pgfqpoint{2.345886in}{1.791946in}}{\pgfqpoint{2.345886in}{1.802996in}}%
\pgfpathcurveto{\pgfqpoint{2.345886in}{1.814047in}}{\pgfqpoint{2.341496in}{1.824646in}}{\pgfqpoint{2.333682in}{1.832459in}}%
\pgfpathcurveto{\pgfqpoint{2.325869in}{1.840273in}}{\pgfqpoint{2.315270in}{1.844663in}}{\pgfqpoint{2.304219in}{1.844663in}}%
\pgfpathcurveto{\pgfqpoint{2.293169in}{1.844663in}}{\pgfqpoint{2.282570in}{1.840273in}}{\pgfqpoint{2.274757in}{1.832459in}}%
\pgfpathcurveto{\pgfqpoint{2.266943in}{1.824646in}}{\pgfqpoint{2.262553in}{1.814047in}}{\pgfqpoint{2.262553in}{1.802996in}}%
\pgfpathcurveto{\pgfqpoint{2.262553in}{1.791946in}}{\pgfqpoint{2.266943in}{1.781347in}}{\pgfqpoint{2.274757in}{1.773534in}}%
\pgfpathcurveto{\pgfqpoint{2.282570in}{1.765720in}}{\pgfqpoint{2.293169in}{1.761330in}}{\pgfqpoint{2.304219in}{1.761330in}}%
\pgfpathclose%
\pgfusepath{stroke,fill}%
\end{pgfscope}%
\begin{pgfscope}%
\pgfpathrectangle{\pgfqpoint{0.600000in}{0.600000in}}{\pgfqpoint{3.900000in}{3.900000in}}%
\pgfusepath{clip}%
\pgfsetbuttcap%
\pgfsetroundjoin%
\definecolor{currentfill}{rgb}{0.121569,0.466667,0.705882}%
\pgfsetfillcolor{currentfill}%
\pgfsetlinewidth{1.003750pt}%
\definecolor{currentstroke}{rgb}{0.121569,0.466667,0.705882}%
\pgfsetstrokecolor{currentstroke}%
\pgfsetdash{}{0pt}%
\pgfpathmoveto{\pgfqpoint{1.828276in}{2.346328in}}%
\pgfpathcurveto{\pgfqpoint{1.839326in}{2.346328in}}{\pgfqpoint{1.849925in}{2.350718in}}{\pgfqpoint{1.857738in}{2.358532in}}%
\pgfpathcurveto{\pgfqpoint{1.865552in}{2.366346in}}{\pgfqpoint{1.869942in}{2.376945in}}{\pgfqpoint{1.869942in}{2.387995in}}%
\pgfpathcurveto{\pgfqpoint{1.869942in}{2.399045in}}{\pgfqpoint{1.865552in}{2.409644in}}{\pgfqpoint{1.857738in}{2.417458in}}%
\pgfpathcurveto{\pgfqpoint{1.849925in}{2.425271in}}{\pgfqpoint{1.839326in}{2.429661in}}{\pgfqpoint{1.828276in}{2.429661in}}%
\pgfpathcurveto{\pgfqpoint{1.817225in}{2.429661in}}{\pgfqpoint{1.806626in}{2.425271in}}{\pgfqpoint{1.798813in}{2.417458in}}%
\pgfpathcurveto{\pgfqpoint{1.790999in}{2.409644in}}{\pgfqpoint{1.786609in}{2.399045in}}{\pgfqpoint{1.786609in}{2.387995in}}%
\pgfpathcurveto{\pgfqpoint{1.786609in}{2.376945in}}{\pgfqpoint{1.790999in}{2.366346in}}{\pgfqpoint{1.798813in}{2.358532in}}%
\pgfpathcurveto{\pgfqpoint{1.806626in}{2.350718in}}{\pgfqpoint{1.817225in}{2.346328in}}{\pgfqpoint{1.828276in}{2.346328in}}%
\pgfpathclose%
\pgfusepath{stroke,fill}%
\end{pgfscope}%
\begin{pgfscope}%
\pgfpathrectangle{\pgfqpoint{0.600000in}{0.600000in}}{\pgfqpoint{3.900000in}{3.900000in}}%
\pgfusepath{clip}%
\pgfsetbuttcap%
\pgfsetroundjoin%
\definecolor{currentfill}{rgb}{0.121569,0.466667,0.705882}%
\pgfsetfillcolor{currentfill}%
\pgfsetlinewidth{1.003750pt}%
\definecolor{currentstroke}{rgb}{0.121569,0.466667,0.705882}%
\pgfsetstrokecolor{currentstroke}%
\pgfsetdash{}{0pt}%
\pgfpathmoveto{\pgfqpoint{2.463903in}{1.441867in}}%
\pgfpathcurveto{\pgfqpoint{2.474953in}{1.441867in}}{\pgfqpoint{2.485552in}{1.446257in}}{\pgfqpoint{2.493365in}{1.454071in}}%
\pgfpathcurveto{\pgfqpoint{2.501179in}{1.461885in}}{\pgfqpoint{2.505569in}{1.472484in}}{\pgfqpoint{2.505569in}{1.483534in}}%
\pgfpathcurveto{\pgfqpoint{2.505569in}{1.494584in}}{\pgfqpoint{2.501179in}{1.505183in}}{\pgfqpoint{2.493365in}{1.512997in}}%
\pgfpathcurveto{\pgfqpoint{2.485552in}{1.520810in}}{\pgfqpoint{2.474953in}{1.525200in}}{\pgfqpoint{2.463903in}{1.525200in}}%
\pgfpathcurveto{\pgfqpoint{2.452853in}{1.525200in}}{\pgfqpoint{2.442253in}{1.520810in}}{\pgfqpoint{2.434440in}{1.512997in}}%
\pgfpathcurveto{\pgfqpoint{2.426626in}{1.505183in}}{\pgfqpoint{2.422236in}{1.494584in}}{\pgfqpoint{2.422236in}{1.483534in}}%
\pgfpathcurveto{\pgfqpoint{2.422236in}{1.472484in}}{\pgfqpoint{2.426626in}{1.461885in}}{\pgfqpoint{2.434440in}{1.454071in}}%
\pgfpathcurveto{\pgfqpoint{2.442253in}{1.446257in}}{\pgfqpoint{2.452853in}{1.441867in}}{\pgfqpoint{2.463903in}{1.441867in}}%
\pgfpathclose%
\pgfusepath{stroke,fill}%
\end{pgfscope}%
\begin{pgfscope}%
\pgfpathrectangle{\pgfqpoint{0.600000in}{0.600000in}}{\pgfqpoint{3.900000in}{3.900000in}}%
\pgfusepath{clip}%
\pgfsetbuttcap%
\pgfsetroundjoin%
\definecolor{currentfill}{rgb}{0.121569,0.466667,0.705882}%
\pgfsetfillcolor{currentfill}%
\pgfsetlinewidth{1.003750pt}%
\definecolor{currentstroke}{rgb}{0.121569,0.466667,0.705882}%
\pgfsetstrokecolor{currentstroke}%
\pgfsetdash{}{0pt}%
\pgfpathmoveto{\pgfqpoint{2.675898in}{2.659976in}}%
\pgfpathcurveto{\pgfqpoint{2.686948in}{2.659976in}}{\pgfqpoint{2.697547in}{2.664366in}}{\pgfqpoint{2.705361in}{2.672179in}}%
\pgfpathcurveto{\pgfqpoint{2.713175in}{2.679993in}}{\pgfqpoint{2.717565in}{2.690592in}}{\pgfqpoint{2.717565in}{2.701642in}}%
\pgfpathcurveto{\pgfqpoint{2.717565in}{2.712692in}}{\pgfqpoint{2.713175in}{2.723291in}}{\pgfqpoint{2.705361in}{2.731105in}}%
\pgfpathcurveto{\pgfqpoint{2.697547in}{2.738919in}}{\pgfqpoint{2.686948in}{2.743309in}}{\pgfqpoint{2.675898in}{2.743309in}}%
\pgfpathcurveto{\pgfqpoint{2.664848in}{2.743309in}}{\pgfqpoint{2.654249in}{2.738919in}}{\pgfqpoint{2.646435in}{2.731105in}}%
\pgfpathcurveto{\pgfqpoint{2.638622in}{2.723291in}}{\pgfqpoint{2.634232in}{2.712692in}}{\pgfqpoint{2.634232in}{2.701642in}}%
\pgfpathcurveto{\pgfqpoint{2.634232in}{2.690592in}}{\pgfqpoint{2.638622in}{2.679993in}}{\pgfqpoint{2.646435in}{2.672179in}}%
\pgfpathcurveto{\pgfqpoint{2.654249in}{2.664366in}}{\pgfqpoint{2.664848in}{2.659976in}}{\pgfqpoint{2.675898in}{2.659976in}}%
\pgfpathclose%
\pgfusepath{stroke,fill}%
\end{pgfscope}%
\begin{pgfscope}%
\pgfpathrectangle{\pgfqpoint{0.600000in}{0.600000in}}{\pgfqpoint{3.900000in}{3.900000in}}%
\pgfusepath{clip}%
\pgfsetbuttcap%
\pgfsetroundjoin%
\definecolor{currentfill}{rgb}{0.121569,0.466667,0.705882}%
\pgfsetfillcolor{currentfill}%
\pgfsetlinewidth{1.003750pt}%
\definecolor{currentstroke}{rgb}{0.121569,0.466667,0.705882}%
\pgfsetstrokecolor{currentstroke}%
\pgfsetdash{}{0pt}%
\pgfpathmoveto{\pgfqpoint{2.721428in}{2.534978in}}%
\pgfpathcurveto{\pgfqpoint{2.732478in}{2.534978in}}{\pgfqpoint{2.743078in}{2.539368in}}{\pgfqpoint{2.750891in}{2.547182in}}%
\pgfpathcurveto{\pgfqpoint{2.758705in}{2.554995in}}{\pgfqpoint{2.763095in}{2.565594in}}{\pgfqpoint{2.763095in}{2.576644in}}%
\pgfpathcurveto{\pgfqpoint{2.763095in}{2.587694in}}{\pgfqpoint{2.758705in}{2.598294in}}{\pgfqpoint{2.750891in}{2.606107in}}%
\pgfpathcurveto{\pgfqpoint{2.743078in}{2.613921in}}{\pgfqpoint{2.732478in}{2.618311in}}{\pgfqpoint{2.721428in}{2.618311in}}%
\pgfpathcurveto{\pgfqpoint{2.710378in}{2.618311in}}{\pgfqpoint{2.699779in}{2.613921in}}{\pgfqpoint{2.691966in}{2.606107in}}%
\pgfpathcurveto{\pgfqpoint{2.684152in}{2.598294in}}{\pgfqpoint{2.679762in}{2.587694in}}{\pgfqpoint{2.679762in}{2.576644in}}%
\pgfpathcurveto{\pgfqpoint{2.679762in}{2.565594in}}{\pgfqpoint{2.684152in}{2.554995in}}{\pgfqpoint{2.691966in}{2.547182in}}%
\pgfpathcurveto{\pgfqpoint{2.699779in}{2.539368in}}{\pgfqpoint{2.710378in}{2.534978in}}{\pgfqpoint{2.721428in}{2.534978in}}%
\pgfpathclose%
\pgfusepath{stroke,fill}%
\end{pgfscope}%
\begin{pgfscope}%
\pgfpathrectangle{\pgfqpoint{0.600000in}{0.600000in}}{\pgfqpoint{3.900000in}{3.900000in}}%
\pgfusepath{clip}%
\pgfsetbuttcap%
\pgfsetroundjoin%
\definecolor{currentfill}{rgb}{0.121569,0.466667,0.705882}%
\pgfsetfillcolor{currentfill}%
\pgfsetlinewidth{1.003750pt}%
\definecolor{currentstroke}{rgb}{0.121569,0.466667,0.705882}%
\pgfsetstrokecolor{currentstroke}%
\pgfsetdash{}{0pt}%
\pgfpathmoveto{\pgfqpoint{2.763722in}{3.167454in}}%
\pgfpathcurveto{\pgfqpoint{2.774773in}{3.167454in}}{\pgfqpoint{2.785372in}{3.171845in}}{\pgfqpoint{2.793185in}{3.179658in}}%
\pgfpathcurveto{\pgfqpoint{2.800999in}{3.187472in}}{\pgfqpoint{2.805389in}{3.198071in}}{\pgfqpoint{2.805389in}{3.209121in}}%
\pgfpathcurveto{\pgfqpoint{2.805389in}{3.220171in}}{\pgfqpoint{2.800999in}{3.230770in}}{\pgfqpoint{2.793185in}{3.238584in}}%
\pgfpathcurveto{\pgfqpoint{2.785372in}{3.246398in}}{\pgfqpoint{2.774773in}{3.250788in}}{\pgfqpoint{2.763722in}{3.250788in}}%
\pgfpathcurveto{\pgfqpoint{2.752672in}{3.250788in}}{\pgfqpoint{2.742073in}{3.246398in}}{\pgfqpoint{2.734260in}{3.238584in}}%
\pgfpathcurveto{\pgfqpoint{2.726446in}{3.230770in}}{\pgfqpoint{2.722056in}{3.220171in}}{\pgfqpoint{2.722056in}{3.209121in}}%
\pgfpathcurveto{\pgfqpoint{2.722056in}{3.198071in}}{\pgfqpoint{2.726446in}{3.187472in}}{\pgfqpoint{2.734260in}{3.179658in}}%
\pgfpathcurveto{\pgfqpoint{2.742073in}{3.171845in}}{\pgfqpoint{2.752672in}{3.167454in}}{\pgfqpoint{2.763722in}{3.167454in}}%
\pgfpathclose%
\pgfusepath{stroke,fill}%
\end{pgfscope}%
\begin{pgfscope}%
\pgfpathrectangle{\pgfqpoint{0.600000in}{0.600000in}}{\pgfqpoint{3.900000in}{3.900000in}}%
\pgfusepath{clip}%
\pgfsetbuttcap%
\pgfsetroundjoin%
\definecolor{currentfill}{rgb}{0.121569,0.466667,0.705882}%
\pgfsetfillcolor{currentfill}%
\pgfsetlinewidth{1.003750pt}%
\definecolor{currentstroke}{rgb}{0.121569,0.466667,0.705882}%
\pgfsetstrokecolor{currentstroke}%
\pgfsetdash{}{0pt}%
\pgfpathmoveto{\pgfqpoint{3.616447in}{2.755044in}}%
\pgfpathcurveto{\pgfqpoint{3.627497in}{2.755044in}}{\pgfqpoint{3.638096in}{2.759434in}}{\pgfqpoint{3.645909in}{2.767248in}}%
\pgfpathcurveto{\pgfqpoint{3.653723in}{2.775061in}}{\pgfqpoint{3.658113in}{2.785660in}}{\pgfqpoint{3.658113in}{2.796710in}}%
\pgfpathcurveto{\pgfqpoint{3.658113in}{2.807760in}}{\pgfqpoint{3.653723in}{2.818359in}}{\pgfqpoint{3.645909in}{2.826173in}}%
\pgfpathcurveto{\pgfqpoint{3.638096in}{2.833987in}}{\pgfqpoint{3.627497in}{2.838377in}}{\pgfqpoint{3.616447in}{2.838377in}}%
\pgfpathcurveto{\pgfqpoint{3.605397in}{2.838377in}}{\pgfqpoint{3.594797in}{2.833987in}}{\pgfqpoint{3.586984in}{2.826173in}}%
\pgfpathcurveto{\pgfqpoint{3.579170in}{2.818359in}}{\pgfqpoint{3.574780in}{2.807760in}}{\pgfqpoint{3.574780in}{2.796710in}}%
\pgfpathcurveto{\pgfqpoint{3.574780in}{2.785660in}}{\pgfqpoint{3.579170in}{2.775061in}}{\pgfqpoint{3.586984in}{2.767248in}}%
\pgfpathcurveto{\pgfqpoint{3.594797in}{2.759434in}}{\pgfqpoint{3.605397in}{2.755044in}}{\pgfqpoint{3.616447in}{2.755044in}}%
\pgfpathclose%
\pgfusepath{stroke,fill}%
\end{pgfscope}%
\begin{pgfscope}%
\pgfpathrectangle{\pgfqpoint{0.600000in}{0.600000in}}{\pgfqpoint{3.900000in}{3.900000in}}%
\pgfusepath{clip}%
\pgfsetbuttcap%
\pgfsetroundjoin%
\definecolor{currentfill}{rgb}{0.121569,0.466667,0.705882}%
\pgfsetfillcolor{currentfill}%
\pgfsetlinewidth{1.003750pt}%
\definecolor{currentstroke}{rgb}{0.121569,0.466667,0.705882}%
\pgfsetstrokecolor{currentstroke}%
\pgfsetdash{}{0pt}%
\pgfpathmoveto{\pgfqpoint{1.904212in}{2.747819in}}%
\pgfpathcurveto{\pgfqpoint{1.915262in}{2.747819in}}{\pgfqpoint{1.925861in}{2.752209in}}{\pgfqpoint{1.933675in}{2.760023in}}%
\pgfpathcurveto{\pgfqpoint{1.941488in}{2.767836in}}{\pgfqpoint{1.945879in}{2.778435in}}{\pgfqpoint{1.945879in}{2.789486in}}%
\pgfpathcurveto{\pgfqpoint{1.945879in}{2.800536in}}{\pgfqpoint{1.941488in}{2.811135in}}{\pgfqpoint{1.933675in}{2.818948in}}%
\pgfpathcurveto{\pgfqpoint{1.925861in}{2.826762in}}{\pgfqpoint{1.915262in}{2.831152in}}{\pgfqpoint{1.904212in}{2.831152in}}%
\pgfpathcurveto{\pgfqpoint{1.893162in}{2.831152in}}{\pgfqpoint{1.882563in}{2.826762in}}{\pgfqpoint{1.874749in}{2.818948in}}%
\pgfpathcurveto{\pgfqpoint{1.866936in}{2.811135in}}{\pgfqpoint{1.862545in}{2.800536in}}{\pgfqpoint{1.862545in}{2.789486in}}%
\pgfpathcurveto{\pgfqpoint{1.862545in}{2.778435in}}{\pgfqpoint{1.866936in}{2.767836in}}{\pgfqpoint{1.874749in}{2.760023in}}%
\pgfpathcurveto{\pgfqpoint{1.882563in}{2.752209in}}{\pgfqpoint{1.893162in}{2.747819in}}{\pgfqpoint{1.904212in}{2.747819in}}%
\pgfpathclose%
\pgfusepath{stroke,fill}%
\end{pgfscope}%
\begin{pgfscope}%
\pgfpathrectangle{\pgfqpoint{0.600000in}{0.600000in}}{\pgfqpoint{3.900000in}{3.900000in}}%
\pgfusepath{clip}%
\pgfsetbuttcap%
\pgfsetroundjoin%
\definecolor{currentfill}{rgb}{0.121569,0.466667,0.705882}%
\pgfsetfillcolor{currentfill}%
\pgfsetlinewidth{1.003750pt}%
\definecolor{currentstroke}{rgb}{0.121569,0.466667,0.705882}%
\pgfsetstrokecolor{currentstroke}%
\pgfsetdash{}{0pt}%
\pgfpathmoveto{\pgfqpoint{2.527187in}{2.955803in}}%
\pgfpathcurveto{\pgfqpoint{2.538237in}{2.955803in}}{\pgfqpoint{2.548836in}{2.960193in}}{\pgfqpoint{2.556650in}{2.968007in}}%
\pgfpathcurveto{\pgfqpoint{2.564464in}{2.975820in}}{\pgfqpoint{2.568854in}{2.986419in}}{\pgfqpoint{2.568854in}{2.997469in}}%
\pgfpathcurveto{\pgfqpoint{2.568854in}{3.008520in}}{\pgfqpoint{2.564464in}{3.019119in}}{\pgfqpoint{2.556650in}{3.026932in}}%
\pgfpathcurveto{\pgfqpoint{2.548836in}{3.034746in}}{\pgfqpoint{2.538237in}{3.039136in}}{\pgfqpoint{2.527187in}{3.039136in}}%
\pgfpathcurveto{\pgfqpoint{2.516137in}{3.039136in}}{\pgfqpoint{2.505538in}{3.034746in}}{\pgfqpoint{2.497724in}{3.026932in}}%
\pgfpathcurveto{\pgfqpoint{2.489911in}{3.019119in}}{\pgfqpoint{2.485521in}{3.008520in}}{\pgfqpoint{2.485521in}{2.997469in}}%
\pgfpathcurveto{\pgfqpoint{2.485521in}{2.986419in}}{\pgfqpoint{2.489911in}{2.975820in}}{\pgfqpoint{2.497724in}{2.968007in}}%
\pgfpathcurveto{\pgfqpoint{2.505538in}{2.960193in}}{\pgfqpoint{2.516137in}{2.955803in}}{\pgfqpoint{2.527187in}{2.955803in}}%
\pgfpathclose%
\pgfusepath{stroke,fill}%
\end{pgfscope}%
\begin{pgfscope}%
\pgfpathrectangle{\pgfqpoint{0.600000in}{0.600000in}}{\pgfqpoint{3.900000in}{3.900000in}}%
\pgfusepath{clip}%
\pgfsetbuttcap%
\pgfsetroundjoin%
\definecolor{currentfill}{rgb}{0.121569,0.466667,0.705882}%
\pgfsetfillcolor{currentfill}%
\pgfsetlinewidth{1.003750pt}%
\definecolor{currentstroke}{rgb}{0.121569,0.466667,0.705882}%
\pgfsetstrokecolor{currentstroke}%
\pgfsetdash{}{0pt}%
\pgfpathmoveto{\pgfqpoint{2.819559in}{2.701852in}}%
\pgfpathcurveto{\pgfqpoint{2.830609in}{2.701852in}}{\pgfqpoint{2.841208in}{2.706243in}}{\pgfqpoint{2.849021in}{2.714056in}}%
\pgfpathcurveto{\pgfqpoint{2.856835in}{2.721870in}}{\pgfqpoint{2.861225in}{2.732469in}}{\pgfqpoint{2.861225in}{2.743519in}}%
\pgfpathcurveto{\pgfqpoint{2.861225in}{2.754569in}}{\pgfqpoint{2.856835in}{2.765168in}}{\pgfqpoint{2.849021in}{2.772982in}}%
\pgfpathcurveto{\pgfqpoint{2.841208in}{2.780795in}}{\pgfqpoint{2.830609in}{2.785186in}}{\pgfqpoint{2.819559in}{2.785186in}}%
\pgfpathcurveto{\pgfqpoint{2.808508in}{2.785186in}}{\pgfqpoint{2.797909in}{2.780795in}}{\pgfqpoint{2.790096in}{2.772982in}}%
\pgfpathcurveto{\pgfqpoint{2.782282in}{2.765168in}}{\pgfqpoint{2.777892in}{2.754569in}}{\pgfqpoint{2.777892in}{2.743519in}}%
\pgfpathcurveto{\pgfqpoint{2.777892in}{2.732469in}}{\pgfqpoint{2.782282in}{2.721870in}}{\pgfqpoint{2.790096in}{2.714056in}}%
\pgfpathcurveto{\pgfqpoint{2.797909in}{2.706243in}}{\pgfqpoint{2.808508in}{2.701852in}}{\pgfqpoint{2.819559in}{2.701852in}}%
\pgfpathclose%
\pgfusepath{stroke,fill}%
\end{pgfscope}%
\begin{pgfscope}%
\pgfpathrectangle{\pgfqpoint{0.600000in}{0.600000in}}{\pgfqpoint{3.900000in}{3.900000in}}%
\pgfusepath{clip}%
\pgfsetbuttcap%
\pgfsetroundjoin%
\definecolor{currentfill}{rgb}{0.121569,0.466667,0.705882}%
\pgfsetfillcolor{currentfill}%
\pgfsetlinewidth{1.003750pt}%
\definecolor{currentstroke}{rgb}{0.121569,0.466667,0.705882}%
\pgfsetstrokecolor{currentstroke}%
\pgfsetdash{}{0pt}%
\pgfpathmoveto{\pgfqpoint{2.072864in}{2.475295in}}%
\pgfpathcurveto{\pgfqpoint{2.083914in}{2.475295in}}{\pgfqpoint{2.094513in}{2.479685in}}{\pgfqpoint{2.102326in}{2.487499in}}%
\pgfpathcurveto{\pgfqpoint{2.110140in}{2.495313in}}{\pgfqpoint{2.114530in}{2.505912in}}{\pgfqpoint{2.114530in}{2.516962in}}%
\pgfpathcurveto{\pgfqpoint{2.114530in}{2.528012in}}{\pgfqpoint{2.110140in}{2.538611in}}{\pgfqpoint{2.102326in}{2.546425in}}%
\pgfpathcurveto{\pgfqpoint{2.094513in}{2.554238in}}{\pgfqpoint{2.083914in}{2.558628in}}{\pgfqpoint{2.072864in}{2.558628in}}%
\pgfpathcurveto{\pgfqpoint{2.061813in}{2.558628in}}{\pgfqpoint{2.051214in}{2.554238in}}{\pgfqpoint{2.043401in}{2.546425in}}%
\pgfpathcurveto{\pgfqpoint{2.035587in}{2.538611in}}{\pgfqpoint{2.031197in}{2.528012in}}{\pgfqpoint{2.031197in}{2.516962in}}%
\pgfpathcurveto{\pgfqpoint{2.031197in}{2.505912in}}{\pgfqpoint{2.035587in}{2.495313in}}{\pgfqpoint{2.043401in}{2.487499in}}%
\pgfpathcurveto{\pgfqpoint{2.051214in}{2.479685in}}{\pgfqpoint{2.061813in}{2.475295in}}{\pgfqpoint{2.072864in}{2.475295in}}%
\pgfpathclose%
\pgfusepath{stroke,fill}%
\end{pgfscope}%
\begin{pgfscope}%
\pgfpathrectangle{\pgfqpoint{0.600000in}{0.600000in}}{\pgfqpoint{3.900000in}{3.900000in}}%
\pgfusepath{clip}%
\pgfsetbuttcap%
\pgfsetroundjoin%
\definecolor{currentfill}{rgb}{0.121569,0.466667,0.705882}%
\pgfsetfillcolor{currentfill}%
\pgfsetlinewidth{1.003750pt}%
\definecolor{currentstroke}{rgb}{0.121569,0.466667,0.705882}%
\pgfsetstrokecolor{currentstroke}%
\pgfsetdash{}{0pt}%
\pgfpathmoveto{\pgfqpoint{2.888729in}{3.009007in}}%
\pgfpathcurveto{\pgfqpoint{2.899779in}{3.009007in}}{\pgfqpoint{2.910378in}{3.013398in}}{\pgfqpoint{2.918191in}{3.021211in}}%
\pgfpathcurveto{\pgfqpoint{2.926005in}{3.029025in}}{\pgfqpoint{2.930395in}{3.039624in}}{\pgfqpoint{2.930395in}{3.050674in}}%
\pgfpathcurveto{\pgfqpoint{2.930395in}{3.061724in}}{\pgfqpoint{2.926005in}{3.072323in}}{\pgfqpoint{2.918191in}{3.080137in}}%
\pgfpathcurveto{\pgfqpoint{2.910378in}{3.087950in}}{\pgfqpoint{2.899779in}{3.092341in}}{\pgfqpoint{2.888729in}{3.092341in}}%
\pgfpathcurveto{\pgfqpoint{2.877679in}{3.092341in}}{\pgfqpoint{2.867080in}{3.087950in}}{\pgfqpoint{2.859266in}{3.080137in}}%
\pgfpathcurveto{\pgfqpoint{2.851452in}{3.072323in}}{\pgfqpoint{2.847062in}{3.061724in}}{\pgfqpoint{2.847062in}{3.050674in}}%
\pgfpathcurveto{\pgfqpoint{2.847062in}{3.039624in}}{\pgfqpoint{2.851452in}{3.029025in}}{\pgfqpoint{2.859266in}{3.021211in}}%
\pgfpathcurveto{\pgfqpoint{2.867080in}{3.013398in}}{\pgfqpoint{2.877679in}{3.009007in}}{\pgfqpoint{2.888729in}{3.009007in}}%
\pgfpathclose%
\pgfusepath{stroke,fill}%
\end{pgfscope}%
\begin{pgfscope}%
\pgfpathrectangle{\pgfqpoint{0.600000in}{0.600000in}}{\pgfqpoint{3.900000in}{3.900000in}}%
\pgfusepath{clip}%
\pgfsetbuttcap%
\pgfsetroundjoin%
\definecolor{currentfill}{rgb}{0.121569,0.466667,0.705882}%
\pgfsetfillcolor{currentfill}%
\pgfsetlinewidth{1.003750pt}%
\definecolor{currentstroke}{rgb}{0.121569,0.466667,0.705882}%
\pgfsetstrokecolor{currentstroke}%
\pgfsetdash{}{0pt}%
\pgfpathmoveto{\pgfqpoint{2.932913in}{2.814909in}}%
\pgfpathcurveto{\pgfqpoint{2.943963in}{2.814909in}}{\pgfqpoint{2.954562in}{2.819299in}}{\pgfqpoint{2.962376in}{2.827113in}}%
\pgfpathcurveto{\pgfqpoint{2.970189in}{2.834926in}}{\pgfqpoint{2.974580in}{2.845525in}}{\pgfqpoint{2.974580in}{2.856576in}}%
\pgfpathcurveto{\pgfqpoint{2.974580in}{2.867626in}}{\pgfqpoint{2.970189in}{2.878225in}}{\pgfqpoint{2.962376in}{2.886038in}}%
\pgfpathcurveto{\pgfqpoint{2.954562in}{2.893852in}}{\pgfqpoint{2.943963in}{2.898242in}}{\pgfqpoint{2.932913in}{2.898242in}}%
\pgfpathcurveto{\pgfqpoint{2.921863in}{2.898242in}}{\pgfqpoint{2.911264in}{2.893852in}}{\pgfqpoint{2.903450in}{2.886038in}}%
\pgfpathcurveto{\pgfqpoint{2.895637in}{2.878225in}}{\pgfqpoint{2.891246in}{2.867626in}}{\pgfqpoint{2.891246in}{2.856576in}}%
\pgfpathcurveto{\pgfqpoint{2.891246in}{2.845525in}}{\pgfqpoint{2.895637in}{2.834926in}}{\pgfqpoint{2.903450in}{2.827113in}}%
\pgfpathcurveto{\pgfqpoint{2.911264in}{2.819299in}}{\pgfqpoint{2.921863in}{2.814909in}}{\pgfqpoint{2.932913in}{2.814909in}}%
\pgfpathclose%
\pgfusepath{stroke,fill}%
\end{pgfscope}%
\begin{pgfscope}%
\pgfpathrectangle{\pgfqpoint{0.600000in}{0.600000in}}{\pgfqpoint{3.900000in}{3.900000in}}%
\pgfusepath{clip}%
\pgfsetbuttcap%
\pgfsetroundjoin%
\definecolor{currentfill}{rgb}{0.121569,0.466667,0.705882}%
\pgfsetfillcolor{currentfill}%
\pgfsetlinewidth{1.003750pt}%
\definecolor{currentstroke}{rgb}{0.121569,0.466667,0.705882}%
\pgfsetstrokecolor{currentstroke}%
\pgfsetdash{}{0pt}%
\pgfpathmoveto{\pgfqpoint{3.231417in}{1.918006in}}%
\pgfpathcurveto{\pgfqpoint{3.242467in}{1.918006in}}{\pgfqpoint{3.253066in}{1.922397in}}{\pgfqpoint{3.260880in}{1.930210in}}%
\pgfpathcurveto{\pgfqpoint{3.268693in}{1.938024in}}{\pgfqpoint{3.273084in}{1.948623in}}{\pgfqpoint{3.273084in}{1.959673in}}%
\pgfpathcurveto{\pgfqpoint{3.273084in}{1.970723in}}{\pgfqpoint{3.268693in}{1.981322in}}{\pgfqpoint{3.260880in}{1.989136in}}%
\pgfpathcurveto{\pgfqpoint{3.253066in}{1.996949in}}{\pgfqpoint{3.242467in}{2.001340in}}{\pgfqpoint{3.231417in}{2.001340in}}%
\pgfpathcurveto{\pgfqpoint{3.220367in}{2.001340in}}{\pgfqpoint{3.209768in}{1.996949in}}{\pgfqpoint{3.201954in}{1.989136in}}%
\pgfpathcurveto{\pgfqpoint{3.194141in}{1.981322in}}{\pgfqpoint{3.189750in}{1.970723in}}{\pgfqpoint{3.189750in}{1.959673in}}%
\pgfpathcurveto{\pgfqpoint{3.189750in}{1.948623in}}{\pgfqpoint{3.194141in}{1.938024in}}{\pgfqpoint{3.201954in}{1.930210in}}%
\pgfpathcurveto{\pgfqpoint{3.209768in}{1.922397in}}{\pgfqpoint{3.220367in}{1.918006in}}{\pgfqpoint{3.231417in}{1.918006in}}%
\pgfpathclose%
\pgfusepath{stroke,fill}%
\end{pgfscope}%
\begin{pgfscope}%
\pgfpathrectangle{\pgfqpoint{0.600000in}{0.600000in}}{\pgfqpoint{3.900000in}{3.900000in}}%
\pgfusepath{clip}%
\pgfsetbuttcap%
\pgfsetroundjoin%
\definecolor{currentfill}{rgb}{0.121569,0.466667,0.705882}%
\pgfsetfillcolor{currentfill}%
\pgfsetlinewidth{1.003750pt}%
\definecolor{currentstroke}{rgb}{0.121569,0.466667,0.705882}%
\pgfsetstrokecolor{currentstroke}%
\pgfsetdash{}{0pt}%
\pgfpathmoveto{\pgfqpoint{2.635697in}{2.518787in}}%
\pgfpathcurveto{\pgfqpoint{2.646747in}{2.518787in}}{\pgfqpoint{2.657346in}{2.523177in}}{\pgfqpoint{2.665159in}{2.530991in}}%
\pgfpathcurveto{\pgfqpoint{2.672973in}{2.538804in}}{\pgfqpoint{2.677363in}{2.549403in}}{\pgfqpoint{2.677363in}{2.560454in}}%
\pgfpathcurveto{\pgfqpoint{2.677363in}{2.571504in}}{\pgfqpoint{2.672973in}{2.582103in}}{\pgfqpoint{2.665159in}{2.589916in}}%
\pgfpathcurveto{\pgfqpoint{2.657346in}{2.597730in}}{\pgfqpoint{2.646747in}{2.602120in}}{\pgfqpoint{2.635697in}{2.602120in}}%
\pgfpathcurveto{\pgfqpoint{2.624647in}{2.602120in}}{\pgfqpoint{2.614047in}{2.597730in}}{\pgfqpoint{2.606234in}{2.589916in}}%
\pgfpathcurveto{\pgfqpoint{2.598420in}{2.582103in}}{\pgfqpoint{2.594030in}{2.571504in}}{\pgfqpoint{2.594030in}{2.560454in}}%
\pgfpathcurveto{\pgfqpoint{2.594030in}{2.549403in}}{\pgfqpoint{2.598420in}{2.538804in}}{\pgfqpoint{2.606234in}{2.530991in}}%
\pgfpathcurveto{\pgfqpoint{2.614047in}{2.523177in}}{\pgfqpoint{2.624647in}{2.518787in}}{\pgfqpoint{2.635697in}{2.518787in}}%
\pgfpathclose%
\pgfusepath{stroke,fill}%
\end{pgfscope}%
\begin{pgfscope}%
\pgfpathrectangle{\pgfqpoint{0.600000in}{0.600000in}}{\pgfqpoint{3.900000in}{3.900000in}}%
\pgfusepath{clip}%
\pgfsetbuttcap%
\pgfsetroundjoin%
\definecolor{currentfill}{rgb}{0.121569,0.466667,0.705882}%
\pgfsetfillcolor{currentfill}%
\pgfsetlinewidth{1.003750pt}%
\definecolor{currentstroke}{rgb}{0.121569,0.466667,0.705882}%
\pgfsetstrokecolor{currentstroke}%
\pgfsetdash{}{0pt}%
\pgfpathmoveto{\pgfqpoint{2.753141in}{2.700976in}}%
\pgfpathcurveto{\pgfqpoint{2.764191in}{2.700976in}}{\pgfqpoint{2.774790in}{2.705366in}}{\pgfqpoint{2.782604in}{2.713180in}}%
\pgfpathcurveto{\pgfqpoint{2.790418in}{2.720993in}}{\pgfqpoint{2.794808in}{2.731592in}}{\pgfqpoint{2.794808in}{2.742643in}}%
\pgfpathcurveto{\pgfqpoint{2.794808in}{2.753693in}}{\pgfqpoint{2.790418in}{2.764292in}}{\pgfqpoint{2.782604in}{2.772105in}}%
\pgfpathcurveto{\pgfqpoint{2.774790in}{2.779919in}}{\pgfqpoint{2.764191in}{2.784309in}}{\pgfqpoint{2.753141in}{2.784309in}}%
\pgfpathcurveto{\pgfqpoint{2.742091in}{2.784309in}}{\pgfqpoint{2.731492in}{2.779919in}}{\pgfqpoint{2.723678in}{2.772105in}}%
\pgfpathcurveto{\pgfqpoint{2.715865in}{2.764292in}}{\pgfqpoint{2.711475in}{2.753693in}}{\pgfqpoint{2.711475in}{2.742643in}}%
\pgfpathcurveto{\pgfqpoint{2.711475in}{2.731592in}}{\pgfqpoint{2.715865in}{2.720993in}}{\pgfqpoint{2.723678in}{2.713180in}}%
\pgfpathcurveto{\pgfqpoint{2.731492in}{2.705366in}}{\pgfqpoint{2.742091in}{2.700976in}}{\pgfqpoint{2.753141in}{2.700976in}}%
\pgfpathclose%
\pgfusepath{stroke,fill}%
\end{pgfscope}%
\begin{pgfscope}%
\pgfpathrectangle{\pgfqpoint{0.600000in}{0.600000in}}{\pgfqpoint{3.900000in}{3.900000in}}%
\pgfusepath{clip}%
\pgfsetbuttcap%
\pgfsetroundjoin%
\definecolor{currentfill}{rgb}{0.121569,0.466667,0.705882}%
\pgfsetfillcolor{currentfill}%
\pgfsetlinewidth{1.003750pt}%
\definecolor{currentstroke}{rgb}{0.121569,0.466667,0.705882}%
\pgfsetstrokecolor{currentstroke}%
\pgfsetdash{}{0pt}%
\pgfpathmoveto{\pgfqpoint{1.617602in}{2.070772in}}%
\pgfpathcurveto{\pgfqpoint{1.628652in}{2.070772in}}{\pgfqpoint{1.639251in}{2.075162in}}{\pgfqpoint{1.647064in}{2.082975in}}%
\pgfpathcurveto{\pgfqpoint{1.654878in}{2.090789in}}{\pgfqpoint{1.659268in}{2.101388in}}{\pgfqpoint{1.659268in}{2.112438in}}%
\pgfpathcurveto{\pgfqpoint{1.659268in}{2.123488in}}{\pgfqpoint{1.654878in}{2.134087in}}{\pgfqpoint{1.647064in}{2.141901in}}%
\pgfpathcurveto{\pgfqpoint{1.639251in}{2.149715in}}{\pgfqpoint{1.628652in}{2.154105in}}{\pgfqpoint{1.617602in}{2.154105in}}%
\pgfpathcurveto{\pgfqpoint{1.606551in}{2.154105in}}{\pgfqpoint{1.595952in}{2.149715in}}{\pgfqpoint{1.588139in}{2.141901in}}%
\pgfpathcurveto{\pgfqpoint{1.580325in}{2.134087in}}{\pgfqpoint{1.575935in}{2.123488in}}{\pgfqpoint{1.575935in}{2.112438in}}%
\pgfpathcurveto{\pgfqpoint{1.575935in}{2.101388in}}{\pgfqpoint{1.580325in}{2.090789in}}{\pgfqpoint{1.588139in}{2.082975in}}%
\pgfpathcurveto{\pgfqpoint{1.595952in}{2.075162in}}{\pgfqpoint{1.606551in}{2.070772in}}{\pgfqpoint{1.617602in}{2.070772in}}%
\pgfpathclose%
\pgfusepath{stroke,fill}%
\end{pgfscope}%
\begin{pgfscope}%
\pgfpathrectangle{\pgfqpoint{0.600000in}{0.600000in}}{\pgfqpoint{3.900000in}{3.900000in}}%
\pgfusepath{clip}%
\pgfsetbuttcap%
\pgfsetroundjoin%
\definecolor{currentfill}{rgb}{0.121569,0.466667,0.705882}%
\pgfsetfillcolor{currentfill}%
\pgfsetlinewidth{1.003750pt}%
\definecolor{currentstroke}{rgb}{0.121569,0.466667,0.705882}%
\pgfsetstrokecolor{currentstroke}%
\pgfsetdash{}{0pt}%
\pgfpathmoveto{\pgfqpoint{3.227515in}{1.726195in}}%
\pgfpathcurveto{\pgfqpoint{3.238565in}{1.726195in}}{\pgfqpoint{3.249164in}{1.730586in}}{\pgfqpoint{3.256978in}{1.738399in}}%
\pgfpathcurveto{\pgfqpoint{3.264791in}{1.746213in}}{\pgfqpoint{3.269182in}{1.756812in}}{\pgfqpoint{3.269182in}{1.767862in}}%
\pgfpathcurveto{\pgfqpoint{3.269182in}{1.778912in}}{\pgfqpoint{3.264791in}{1.789511in}}{\pgfqpoint{3.256978in}{1.797325in}}%
\pgfpathcurveto{\pgfqpoint{3.249164in}{1.805138in}}{\pgfqpoint{3.238565in}{1.809529in}}{\pgfqpoint{3.227515in}{1.809529in}}%
\pgfpathcurveto{\pgfqpoint{3.216465in}{1.809529in}}{\pgfqpoint{3.205866in}{1.805138in}}{\pgfqpoint{3.198052in}{1.797325in}}%
\pgfpathcurveto{\pgfqpoint{3.190238in}{1.789511in}}{\pgfqpoint{3.185848in}{1.778912in}}{\pgfqpoint{3.185848in}{1.767862in}}%
\pgfpathcurveto{\pgfqpoint{3.185848in}{1.756812in}}{\pgfqpoint{3.190238in}{1.746213in}}{\pgfqpoint{3.198052in}{1.738399in}}%
\pgfpathcurveto{\pgfqpoint{3.205866in}{1.730586in}}{\pgfqpoint{3.216465in}{1.726195in}}{\pgfqpoint{3.227515in}{1.726195in}}%
\pgfpathclose%
\pgfusepath{stroke,fill}%
\end{pgfscope}%
\begin{pgfscope}%
\pgfpathrectangle{\pgfqpoint{0.600000in}{0.600000in}}{\pgfqpoint{3.900000in}{3.900000in}}%
\pgfusepath{clip}%
\pgfsetbuttcap%
\pgfsetroundjoin%
\definecolor{currentfill}{rgb}{0.121569,0.466667,0.705882}%
\pgfsetfillcolor{currentfill}%
\pgfsetlinewidth{1.003750pt}%
\definecolor{currentstroke}{rgb}{0.121569,0.466667,0.705882}%
\pgfsetstrokecolor{currentstroke}%
\pgfsetdash{}{0pt}%
\pgfpathmoveto{\pgfqpoint{2.019904in}{2.705066in}}%
\pgfpathcurveto{\pgfqpoint{2.030954in}{2.705066in}}{\pgfqpoint{2.041553in}{2.709457in}}{\pgfqpoint{2.049367in}{2.717270in}}%
\pgfpathcurveto{\pgfqpoint{2.057180in}{2.725084in}}{\pgfqpoint{2.061571in}{2.735683in}}{\pgfqpoint{2.061571in}{2.746733in}}%
\pgfpathcurveto{\pgfqpoint{2.061571in}{2.757783in}}{\pgfqpoint{2.057180in}{2.768382in}}{\pgfqpoint{2.049367in}{2.776196in}}%
\pgfpathcurveto{\pgfqpoint{2.041553in}{2.784010in}}{\pgfqpoint{2.030954in}{2.788400in}}{\pgfqpoint{2.019904in}{2.788400in}}%
\pgfpathcurveto{\pgfqpoint{2.008854in}{2.788400in}}{\pgfqpoint{1.998255in}{2.784010in}}{\pgfqpoint{1.990441in}{2.776196in}}%
\pgfpathcurveto{\pgfqpoint{1.982627in}{2.768382in}}{\pgfqpoint{1.978237in}{2.757783in}}{\pgfqpoint{1.978237in}{2.746733in}}%
\pgfpathcurveto{\pgfqpoint{1.978237in}{2.735683in}}{\pgfqpoint{1.982627in}{2.725084in}}{\pgfqpoint{1.990441in}{2.717270in}}%
\pgfpathcurveto{\pgfqpoint{1.998255in}{2.709457in}}{\pgfqpoint{2.008854in}{2.705066in}}{\pgfqpoint{2.019904in}{2.705066in}}%
\pgfpathclose%
\pgfusepath{stroke,fill}%
\end{pgfscope}%
\begin{pgfscope}%
\pgfpathrectangle{\pgfqpoint{0.600000in}{0.600000in}}{\pgfqpoint{3.900000in}{3.900000in}}%
\pgfusepath{clip}%
\pgfsetbuttcap%
\pgfsetroundjoin%
\definecolor{currentfill}{rgb}{0.121569,0.466667,0.705882}%
\pgfsetfillcolor{currentfill}%
\pgfsetlinewidth{1.003750pt}%
\definecolor{currentstroke}{rgb}{0.121569,0.466667,0.705882}%
\pgfsetstrokecolor{currentstroke}%
\pgfsetdash{}{0pt}%
\pgfpathmoveto{\pgfqpoint{1.844445in}{2.833988in}}%
\pgfpathcurveto{\pgfqpoint{1.855495in}{2.833988in}}{\pgfqpoint{1.866094in}{2.838378in}}{\pgfqpoint{1.873908in}{2.846192in}}%
\pgfpathcurveto{\pgfqpoint{1.881722in}{2.854005in}}{\pgfqpoint{1.886112in}{2.864604in}}{\pgfqpoint{1.886112in}{2.875654in}}%
\pgfpathcurveto{\pgfqpoint{1.886112in}{2.886704in}}{\pgfqpoint{1.881722in}{2.897303in}}{\pgfqpoint{1.873908in}{2.905117in}}%
\pgfpathcurveto{\pgfqpoint{1.866094in}{2.912931in}}{\pgfqpoint{1.855495in}{2.917321in}}{\pgfqpoint{1.844445in}{2.917321in}}%
\pgfpathcurveto{\pgfqpoint{1.833395in}{2.917321in}}{\pgfqpoint{1.822796in}{2.912931in}}{\pgfqpoint{1.814983in}{2.905117in}}%
\pgfpathcurveto{\pgfqpoint{1.807169in}{2.897303in}}{\pgfqpoint{1.802779in}{2.886704in}}{\pgfqpoint{1.802779in}{2.875654in}}%
\pgfpathcurveto{\pgfqpoint{1.802779in}{2.864604in}}{\pgfqpoint{1.807169in}{2.854005in}}{\pgfqpoint{1.814983in}{2.846192in}}%
\pgfpathcurveto{\pgfqpoint{1.822796in}{2.838378in}}{\pgfqpoint{1.833395in}{2.833988in}}{\pgfqpoint{1.844445in}{2.833988in}}%
\pgfpathclose%
\pgfusepath{stroke,fill}%
\end{pgfscope}%
\begin{pgfscope}%
\pgfpathrectangle{\pgfqpoint{0.600000in}{0.600000in}}{\pgfqpoint{3.900000in}{3.900000in}}%
\pgfusepath{clip}%
\pgfsetbuttcap%
\pgfsetroundjoin%
\definecolor{currentfill}{rgb}{0.121569,0.466667,0.705882}%
\pgfsetfillcolor{currentfill}%
\pgfsetlinewidth{1.003750pt}%
\definecolor{currentstroke}{rgb}{0.121569,0.466667,0.705882}%
\pgfsetstrokecolor{currentstroke}%
\pgfsetdash{}{0pt}%
\pgfpathmoveto{\pgfqpoint{2.597382in}{2.383956in}}%
\pgfpathcurveto{\pgfqpoint{2.608432in}{2.383956in}}{\pgfqpoint{2.619031in}{2.388346in}}{\pgfqpoint{2.626845in}{2.396159in}}%
\pgfpathcurveto{\pgfqpoint{2.634658in}{2.403973in}}{\pgfqpoint{2.639049in}{2.414572in}}{\pgfqpoint{2.639049in}{2.425622in}}%
\pgfpathcurveto{\pgfqpoint{2.639049in}{2.436672in}}{\pgfqpoint{2.634658in}{2.447271in}}{\pgfqpoint{2.626845in}{2.455085in}}%
\pgfpathcurveto{\pgfqpoint{2.619031in}{2.462899in}}{\pgfqpoint{2.608432in}{2.467289in}}{\pgfqpoint{2.597382in}{2.467289in}}%
\pgfpathcurveto{\pgfqpoint{2.586332in}{2.467289in}}{\pgfqpoint{2.575733in}{2.462899in}}{\pgfqpoint{2.567919in}{2.455085in}}%
\pgfpathcurveto{\pgfqpoint{2.560105in}{2.447271in}}{\pgfqpoint{2.555715in}{2.436672in}}{\pgfqpoint{2.555715in}{2.425622in}}%
\pgfpathcurveto{\pgfqpoint{2.555715in}{2.414572in}}{\pgfqpoint{2.560105in}{2.403973in}}{\pgfqpoint{2.567919in}{2.396159in}}%
\pgfpathcurveto{\pgfqpoint{2.575733in}{2.388346in}}{\pgfqpoint{2.586332in}{2.383956in}}{\pgfqpoint{2.597382in}{2.383956in}}%
\pgfpathclose%
\pgfusepath{stroke,fill}%
\end{pgfscope}%
\begin{pgfscope}%
\pgfpathrectangle{\pgfqpoint{0.600000in}{0.600000in}}{\pgfqpoint{3.900000in}{3.900000in}}%
\pgfusepath{clip}%
\pgfsetbuttcap%
\pgfsetroundjoin%
\definecolor{currentfill}{rgb}{0.121569,0.466667,0.705882}%
\pgfsetfillcolor{currentfill}%
\pgfsetlinewidth{1.003750pt}%
\definecolor{currentstroke}{rgb}{0.121569,0.466667,0.705882}%
\pgfsetstrokecolor{currentstroke}%
\pgfsetdash{}{0pt}%
\pgfpathmoveto{\pgfqpoint{2.784486in}{2.166295in}}%
\pgfpathcurveto{\pgfqpoint{2.795536in}{2.166295in}}{\pgfqpoint{2.806135in}{2.170686in}}{\pgfqpoint{2.813949in}{2.178499in}}%
\pgfpathcurveto{\pgfqpoint{2.821763in}{2.186313in}}{\pgfqpoint{2.826153in}{2.196912in}}{\pgfqpoint{2.826153in}{2.207962in}}%
\pgfpathcurveto{\pgfqpoint{2.826153in}{2.219012in}}{\pgfqpoint{2.821763in}{2.229611in}}{\pgfqpoint{2.813949in}{2.237425in}}%
\pgfpathcurveto{\pgfqpoint{2.806135in}{2.245239in}}{\pgfqpoint{2.795536in}{2.249629in}}{\pgfqpoint{2.784486in}{2.249629in}}%
\pgfpathcurveto{\pgfqpoint{2.773436in}{2.249629in}}{\pgfqpoint{2.762837in}{2.245239in}}{\pgfqpoint{2.755024in}{2.237425in}}%
\pgfpathcurveto{\pgfqpoint{2.747210in}{2.229611in}}{\pgfqpoint{2.742820in}{2.219012in}}{\pgfqpoint{2.742820in}{2.207962in}}%
\pgfpathcurveto{\pgfqpoint{2.742820in}{2.196912in}}{\pgfqpoint{2.747210in}{2.186313in}}{\pgfqpoint{2.755024in}{2.178499in}}%
\pgfpathcurveto{\pgfqpoint{2.762837in}{2.170686in}}{\pgfqpoint{2.773436in}{2.166295in}}{\pgfqpoint{2.784486in}{2.166295in}}%
\pgfpathclose%
\pgfusepath{stroke,fill}%
\end{pgfscope}%
\begin{pgfscope}%
\pgfpathrectangle{\pgfqpoint{0.600000in}{0.600000in}}{\pgfqpoint{3.900000in}{3.900000in}}%
\pgfusepath{clip}%
\pgfsetbuttcap%
\pgfsetroundjoin%
\definecolor{currentfill}{rgb}{0.121569,0.466667,0.705882}%
\pgfsetfillcolor{currentfill}%
\pgfsetlinewidth{1.003750pt}%
\definecolor{currentstroke}{rgb}{0.121569,0.466667,0.705882}%
\pgfsetstrokecolor{currentstroke}%
\pgfsetdash{}{0pt}%
\pgfpathmoveto{\pgfqpoint{2.695170in}{2.636616in}}%
\pgfpathcurveto{\pgfqpoint{2.706220in}{2.636616in}}{\pgfqpoint{2.716819in}{2.641006in}}{\pgfqpoint{2.724633in}{2.648820in}}%
\pgfpathcurveto{\pgfqpoint{2.732446in}{2.656633in}}{\pgfqpoint{2.736837in}{2.667233in}}{\pgfqpoint{2.736837in}{2.678283in}}%
\pgfpathcurveto{\pgfqpoint{2.736837in}{2.689333in}}{\pgfqpoint{2.732446in}{2.699932in}}{\pgfqpoint{2.724633in}{2.707745in}}%
\pgfpathcurveto{\pgfqpoint{2.716819in}{2.715559in}}{\pgfqpoint{2.706220in}{2.719949in}}{\pgfqpoint{2.695170in}{2.719949in}}%
\pgfpathcurveto{\pgfqpoint{2.684120in}{2.719949in}}{\pgfqpoint{2.673521in}{2.715559in}}{\pgfqpoint{2.665707in}{2.707745in}}%
\pgfpathcurveto{\pgfqpoint{2.657894in}{2.699932in}}{\pgfqpoint{2.653503in}{2.689333in}}{\pgfqpoint{2.653503in}{2.678283in}}%
\pgfpathcurveto{\pgfqpoint{2.653503in}{2.667233in}}{\pgfqpoint{2.657894in}{2.656633in}}{\pgfqpoint{2.665707in}{2.648820in}}%
\pgfpathcurveto{\pgfqpoint{2.673521in}{2.641006in}}{\pgfqpoint{2.684120in}{2.636616in}}{\pgfqpoint{2.695170in}{2.636616in}}%
\pgfpathclose%
\pgfusepath{stroke,fill}%
\end{pgfscope}%
\begin{pgfscope}%
\pgfpathrectangle{\pgfqpoint{0.600000in}{0.600000in}}{\pgfqpoint{3.900000in}{3.900000in}}%
\pgfusepath{clip}%
\pgfsetbuttcap%
\pgfsetroundjoin%
\definecolor{currentfill}{rgb}{0.121569,0.466667,0.705882}%
\pgfsetfillcolor{currentfill}%
\pgfsetlinewidth{1.003750pt}%
\definecolor{currentstroke}{rgb}{0.121569,0.466667,0.705882}%
\pgfsetstrokecolor{currentstroke}%
\pgfsetdash{}{0pt}%
\pgfpathmoveto{\pgfqpoint{2.056261in}{2.907391in}}%
\pgfpathcurveto{\pgfqpoint{2.067311in}{2.907391in}}{\pgfqpoint{2.077910in}{2.911781in}}{\pgfqpoint{2.085724in}{2.919594in}}%
\pgfpathcurveto{\pgfqpoint{2.093538in}{2.927408in}}{\pgfqpoint{2.097928in}{2.938007in}}{\pgfqpoint{2.097928in}{2.949057in}}%
\pgfpathcurveto{\pgfqpoint{2.097928in}{2.960107in}}{\pgfqpoint{2.093538in}{2.970706in}}{\pgfqpoint{2.085724in}{2.978520in}}%
\pgfpathcurveto{\pgfqpoint{2.077910in}{2.986334in}}{\pgfqpoint{2.067311in}{2.990724in}}{\pgfqpoint{2.056261in}{2.990724in}}%
\pgfpathcurveto{\pgfqpoint{2.045211in}{2.990724in}}{\pgfqpoint{2.034612in}{2.986334in}}{\pgfqpoint{2.026799in}{2.978520in}}%
\pgfpathcurveto{\pgfqpoint{2.018985in}{2.970706in}}{\pgfqpoint{2.014595in}{2.960107in}}{\pgfqpoint{2.014595in}{2.949057in}}%
\pgfpathcurveto{\pgfqpoint{2.014595in}{2.938007in}}{\pgfqpoint{2.018985in}{2.927408in}}{\pgfqpoint{2.026799in}{2.919594in}}%
\pgfpathcurveto{\pgfqpoint{2.034612in}{2.911781in}}{\pgfqpoint{2.045211in}{2.907391in}}{\pgfqpoint{2.056261in}{2.907391in}}%
\pgfpathclose%
\pgfusepath{stroke,fill}%
\end{pgfscope}%
\begin{pgfscope}%
\pgfpathrectangle{\pgfqpoint{0.600000in}{0.600000in}}{\pgfqpoint{3.900000in}{3.900000in}}%
\pgfusepath{clip}%
\pgfsetbuttcap%
\pgfsetroundjoin%
\definecolor{currentfill}{rgb}{0.121569,0.466667,0.705882}%
\pgfsetfillcolor{currentfill}%
\pgfsetlinewidth{1.003750pt}%
\definecolor{currentstroke}{rgb}{0.121569,0.466667,0.705882}%
\pgfsetstrokecolor{currentstroke}%
\pgfsetdash{}{0pt}%
\pgfpathmoveto{\pgfqpoint{2.536873in}{3.222207in}}%
\pgfpathcurveto{\pgfqpoint{2.547923in}{3.222207in}}{\pgfqpoint{2.558522in}{3.226598in}}{\pgfqpoint{2.566336in}{3.234411in}}%
\pgfpathcurveto{\pgfqpoint{2.574149in}{3.242225in}}{\pgfqpoint{2.578539in}{3.252824in}}{\pgfqpoint{2.578539in}{3.263874in}}%
\pgfpathcurveto{\pgfqpoint{2.578539in}{3.274924in}}{\pgfqpoint{2.574149in}{3.285523in}}{\pgfqpoint{2.566336in}{3.293337in}}%
\pgfpathcurveto{\pgfqpoint{2.558522in}{3.301151in}}{\pgfqpoint{2.547923in}{3.305541in}}{\pgfqpoint{2.536873in}{3.305541in}}%
\pgfpathcurveto{\pgfqpoint{2.525823in}{3.305541in}}{\pgfqpoint{2.515224in}{3.301151in}}{\pgfqpoint{2.507410in}{3.293337in}}%
\pgfpathcurveto{\pgfqpoint{2.499596in}{3.285523in}}{\pgfqpoint{2.495206in}{3.274924in}}{\pgfqpoint{2.495206in}{3.263874in}}%
\pgfpathcurveto{\pgfqpoint{2.495206in}{3.252824in}}{\pgfqpoint{2.499596in}{3.242225in}}{\pgfqpoint{2.507410in}{3.234411in}}%
\pgfpathcurveto{\pgfqpoint{2.515224in}{3.226598in}}{\pgfqpoint{2.525823in}{3.222207in}}{\pgfqpoint{2.536873in}{3.222207in}}%
\pgfpathclose%
\pgfusepath{stroke,fill}%
\end{pgfscope}%
\begin{pgfscope}%
\pgfpathrectangle{\pgfqpoint{0.600000in}{0.600000in}}{\pgfqpoint{3.900000in}{3.900000in}}%
\pgfusepath{clip}%
\pgfsetbuttcap%
\pgfsetroundjoin%
\definecolor{currentfill}{rgb}{0.121569,0.466667,0.705882}%
\pgfsetfillcolor{currentfill}%
\pgfsetlinewidth{1.003750pt}%
\definecolor{currentstroke}{rgb}{0.121569,0.466667,0.705882}%
\pgfsetstrokecolor{currentstroke}%
\pgfsetdash{}{0pt}%
\pgfpathmoveto{\pgfqpoint{1.816563in}{1.786327in}}%
\pgfpathcurveto{\pgfqpoint{1.827613in}{1.786327in}}{\pgfqpoint{1.838213in}{1.790718in}}{\pgfqpoint{1.846026in}{1.798531in}}%
\pgfpathcurveto{\pgfqpoint{1.853840in}{1.806345in}}{\pgfqpoint{1.858230in}{1.816944in}}{\pgfqpoint{1.858230in}{1.827994in}}%
\pgfpathcurveto{\pgfqpoint{1.858230in}{1.839044in}}{\pgfqpoint{1.853840in}{1.849643in}}{\pgfqpoint{1.846026in}{1.857457in}}%
\pgfpathcurveto{\pgfqpoint{1.838213in}{1.865271in}}{\pgfqpoint{1.827613in}{1.869661in}}{\pgfqpoint{1.816563in}{1.869661in}}%
\pgfpathcurveto{\pgfqpoint{1.805513in}{1.869661in}}{\pgfqpoint{1.794914in}{1.865271in}}{\pgfqpoint{1.787101in}{1.857457in}}%
\pgfpathcurveto{\pgfqpoint{1.779287in}{1.849643in}}{\pgfqpoint{1.774897in}{1.839044in}}{\pgfqpoint{1.774897in}{1.827994in}}%
\pgfpathcurveto{\pgfqpoint{1.774897in}{1.816944in}}{\pgfqpoint{1.779287in}{1.806345in}}{\pgfqpoint{1.787101in}{1.798531in}}%
\pgfpathcurveto{\pgfqpoint{1.794914in}{1.790718in}}{\pgfqpoint{1.805513in}{1.786327in}}{\pgfqpoint{1.816563in}{1.786327in}}%
\pgfpathclose%
\pgfusepath{stroke,fill}%
\end{pgfscope}%
\begin{pgfscope}%
\pgfpathrectangle{\pgfqpoint{0.600000in}{0.600000in}}{\pgfqpoint{3.900000in}{3.900000in}}%
\pgfusepath{clip}%
\pgfsetbuttcap%
\pgfsetroundjoin%
\definecolor{currentfill}{rgb}{0.121569,0.466667,0.705882}%
\pgfsetfillcolor{currentfill}%
\pgfsetlinewidth{1.003750pt}%
\definecolor{currentstroke}{rgb}{0.121569,0.466667,0.705882}%
\pgfsetstrokecolor{currentstroke}%
\pgfsetdash{}{0pt}%
\pgfpathmoveto{\pgfqpoint{2.048626in}{2.543437in}}%
\pgfpathcurveto{\pgfqpoint{2.059676in}{2.543437in}}{\pgfqpoint{2.070275in}{2.547828in}}{\pgfqpoint{2.078089in}{2.555641in}}%
\pgfpathcurveto{\pgfqpoint{2.085903in}{2.563455in}}{\pgfqpoint{2.090293in}{2.574054in}}{\pgfqpoint{2.090293in}{2.585104in}}%
\pgfpathcurveto{\pgfqpoint{2.090293in}{2.596154in}}{\pgfqpoint{2.085903in}{2.606753in}}{\pgfqpoint{2.078089in}{2.614567in}}%
\pgfpathcurveto{\pgfqpoint{2.070275in}{2.622380in}}{\pgfqpoint{2.059676in}{2.626771in}}{\pgfqpoint{2.048626in}{2.626771in}}%
\pgfpathcurveto{\pgfqpoint{2.037576in}{2.626771in}}{\pgfqpoint{2.026977in}{2.622380in}}{\pgfqpoint{2.019164in}{2.614567in}}%
\pgfpathcurveto{\pgfqpoint{2.011350in}{2.606753in}}{\pgfqpoint{2.006960in}{2.596154in}}{\pgfqpoint{2.006960in}{2.585104in}}%
\pgfpathcurveto{\pgfqpoint{2.006960in}{2.574054in}}{\pgfqpoint{2.011350in}{2.563455in}}{\pgfqpoint{2.019164in}{2.555641in}}%
\pgfpathcurveto{\pgfqpoint{2.026977in}{2.547828in}}{\pgfqpoint{2.037576in}{2.543437in}}{\pgfqpoint{2.048626in}{2.543437in}}%
\pgfpathclose%
\pgfusepath{stroke,fill}%
\end{pgfscope}%
\begin{pgfscope}%
\pgfpathrectangle{\pgfqpoint{0.600000in}{0.600000in}}{\pgfqpoint{3.900000in}{3.900000in}}%
\pgfusepath{clip}%
\pgfsetbuttcap%
\pgfsetroundjoin%
\definecolor{currentfill}{rgb}{0.121569,0.466667,0.705882}%
\pgfsetfillcolor{currentfill}%
\pgfsetlinewidth{1.003750pt}%
\definecolor{currentstroke}{rgb}{0.121569,0.466667,0.705882}%
\pgfsetstrokecolor{currentstroke}%
\pgfsetdash{}{0pt}%
\pgfpathmoveto{\pgfqpoint{2.665548in}{1.930946in}}%
\pgfpathcurveto{\pgfqpoint{2.676598in}{1.930946in}}{\pgfqpoint{2.687197in}{1.935336in}}{\pgfqpoint{2.695010in}{1.943149in}}%
\pgfpathcurveto{\pgfqpoint{2.702824in}{1.950963in}}{\pgfqpoint{2.707214in}{1.961562in}}{\pgfqpoint{2.707214in}{1.972612in}}%
\pgfpathcurveto{\pgfqpoint{2.707214in}{1.983662in}}{\pgfqpoint{2.702824in}{1.994261in}}{\pgfqpoint{2.695010in}{2.002075in}}%
\pgfpathcurveto{\pgfqpoint{2.687197in}{2.009889in}}{\pgfqpoint{2.676598in}{2.014279in}}{\pgfqpoint{2.665548in}{2.014279in}}%
\pgfpathcurveto{\pgfqpoint{2.654497in}{2.014279in}}{\pgfqpoint{2.643898in}{2.009889in}}{\pgfqpoint{2.636085in}{2.002075in}}%
\pgfpathcurveto{\pgfqpoint{2.628271in}{1.994261in}}{\pgfqpoint{2.623881in}{1.983662in}}{\pgfqpoint{2.623881in}{1.972612in}}%
\pgfpathcurveto{\pgfqpoint{2.623881in}{1.961562in}}{\pgfqpoint{2.628271in}{1.950963in}}{\pgfqpoint{2.636085in}{1.943149in}}%
\pgfpathcurveto{\pgfqpoint{2.643898in}{1.935336in}}{\pgfqpoint{2.654497in}{1.930946in}}{\pgfqpoint{2.665548in}{1.930946in}}%
\pgfpathclose%
\pgfusepath{stroke,fill}%
\end{pgfscope}%
\begin{pgfscope}%
\pgfpathrectangle{\pgfqpoint{0.600000in}{0.600000in}}{\pgfqpoint{3.900000in}{3.900000in}}%
\pgfusepath{clip}%
\pgfsetbuttcap%
\pgfsetroundjoin%
\definecolor{currentfill}{rgb}{0.121569,0.466667,0.705882}%
\pgfsetfillcolor{currentfill}%
\pgfsetlinewidth{1.003750pt}%
\definecolor{currentstroke}{rgb}{0.121569,0.466667,0.705882}%
\pgfsetstrokecolor{currentstroke}%
\pgfsetdash{}{0pt}%
\pgfpathmoveto{\pgfqpoint{1.817043in}{2.877384in}}%
\pgfpathcurveto{\pgfqpoint{1.828093in}{2.877384in}}{\pgfqpoint{1.838692in}{2.881774in}}{\pgfqpoint{1.846506in}{2.889588in}}%
\pgfpathcurveto{\pgfqpoint{1.854319in}{2.897402in}}{\pgfqpoint{1.858710in}{2.908001in}}{\pgfqpoint{1.858710in}{2.919051in}}%
\pgfpathcurveto{\pgfqpoint{1.858710in}{2.930101in}}{\pgfqpoint{1.854319in}{2.940700in}}{\pgfqpoint{1.846506in}{2.948514in}}%
\pgfpathcurveto{\pgfqpoint{1.838692in}{2.956327in}}{\pgfqpoint{1.828093in}{2.960718in}}{\pgfqpoint{1.817043in}{2.960718in}}%
\pgfpathcurveto{\pgfqpoint{1.805993in}{2.960718in}}{\pgfqpoint{1.795394in}{2.956327in}}{\pgfqpoint{1.787580in}{2.948514in}}%
\pgfpathcurveto{\pgfqpoint{1.779767in}{2.940700in}}{\pgfqpoint{1.775376in}{2.930101in}}{\pgfqpoint{1.775376in}{2.919051in}}%
\pgfpathcurveto{\pgfqpoint{1.775376in}{2.908001in}}{\pgfqpoint{1.779767in}{2.897402in}}{\pgfqpoint{1.787580in}{2.889588in}}%
\pgfpathcurveto{\pgfqpoint{1.795394in}{2.881774in}}{\pgfqpoint{1.805993in}{2.877384in}}{\pgfqpoint{1.817043in}{2.877384in}}%
\pgfpathclose%
\pgfusepath{stroke,fill}%
\end{pgfscope}%
\begin{pgfscope}%
\pgfpathrectangle{\pgfqpoint{0.600000in}{0.600000in}}{\pgfqpoint{3.900000in}{3.900000in}}%
\pgfusepath{clip}%
\pgfsetbuttcap%
\pgfsetroundjoin%
\definecolor{currentfill}{rgb}{0.121569,0.466667,0.705882}%
\pgfsetfillcolor{currentfill}%
\pgfsetlinewidth{1.003750pt}%
\definecolor{currentstroke}{rgb}{0.121569,0.466667,0.705882}%
\pgfsetstrokecolor{currentstroke}%
\pgfsetdash{}{0pt}%
\pgfpathmoveto{\pgfqpoint{2.666575in}{3.586470in}}%
\pgfpathcurveto{\pgfqpoint{2.677625in}{3.586470in}}{\pgfqpoint{2.688224in}{3.590860in}}{\pgfqpoint{2.696037in}{3.598674in}}%
\pgfpathcurveto{\pgfqpoint{2.703851in}{3.606487in}}{\pgfqpoint{2.708241in}{3.617086in}}{\pgfqpoint{2.708241in}{3.628137in}}%
\pgfpathcurveto{\pgfqpoint{2.708241in}{3.639187in}}{\pgfqpoint{2.703851in}{3.649786in}}{\pgfqpoint{2.696037in}{3.657599in}}%
\pgfpathcurveto{\pgfqpoint{2.688224in}{3.665413in}}{\pgfqpoint{2.677625in}{3.669803in}}{\pgfqpoint{2.666575in}{3.669803in}}%
\pgfpathcurveto{\pgfqpoint{2.655525in}{3.669803in}}{\pgfqpoint{2.644925in}{3.665413in}}{\pgfqpoint{2.637112in}{3.657599in}}%
\pgfpathcurveto{\pgfqpoint{2.629298in}{3.649786in}}{\pgfqpoint{2.624908in}{3.639187in}}{\pgfqpoint{2.624908in}{3.628137in}}%
\pgfpathcurveto{\pgfqpoint{2.624908in}{3.617086in}}{\pgfqpoint{2.629298in}{3.606487in}}{\pgfqpoint{2.637112in}{3.598674in}}%
\pgfpathcurveto{\pgfqpoint{2.644925in}{3.590860in}}{\pgfqpoint{2.655525in}{3.586470in}}{\pgfqpoint{2.666575in}{3.586470in}}%
\pgfpathclose%
\pgfusepath{stroke,fill}%
\end{pgfscope}%
\begin{pgfscope}%
\pgfpathrectangle{\pgfqpoint{0.600000in}{0.600000in}}{\pgfqpoint{3.900000in}{3.900000in}}%
\pgfusepath{clip}%
\pgfsetbuttcap%
\pgfsetroundjoin%
\definecolor{currentfill}{rgb}{0.121569,0.466667,0.705882}%
\pgfsetfillcolor{currentfill}%
\pgfsetlinewidth{1.003750pt}%
\definecolor{currentstroke}{rgb}{0.121569,0.466667,0.705882}%
\pgfsetstrokecolor{currentstroke}%
\pgfsetdash{}{0pt}%
\pgfpathmoveto{\pgfqpoint{2.346395in}{2.394986in}}%
\pgfpathcurveto{\pgfqpoint{2.357445in}{2.394986in}}{\pgfqpoint{2.368044in}{2.399376in}}{\pgfqpoint{2.375858in}{2.407190in}}%
\pgfpathcurveto{\pgfqpoint{2.383672in}{2.415004in}}{\pgfqpoint{2.388062in}{2.425603in}}{\pgfqpoint{2.388062in}{2.436653in}}%
\pgfpathcurveto{\pgfqpoint{2.388062in}{2.447703in}}{\pgfqpoint{2.383672in}{2.458302in}}{\pgfqpoint{2.375858in}{2.466116in}}%
\pgfpathcurveto{\pgfqpoint{2.368044in}{2.473929in}}{\pgfqpoint{2.357445in}{2.478320in}}{\pgfqpoint{2.346395in}{2.478320in}}%
\pgfpathcurveto{\pgfqpoint{2.335345in}{2.478320in}}{\pgfqpoint{2.324746in}{2.473929in}}{\pgfqpoint{2.316932in}{2.466116in}}%
\pgfpathcurveto{\pgfqpoint{2.309119in}{2.458302in}}{\pgfqpoint{2.304728in}{2.447703in}}{\pgfqpoint{2.304728in}{2.436653in}}%
\pgfpathcurveto{\pgfqpoint{2.304728in}{2.425603in}}{\pgfqpoint{2.309119in}{2.415004in}}{\pgfqpoint{2.316932in}{2.407190in}}%
\pgfpathcurveto{\pgfqpoint{2.324746in}{2.399376in}}{\pgfqpoint{2.335345in}{2.394986in}}{\pgfqpoint{2.346395in}{2.394986in}}%
\pgfpathclose%
\pgfusepath{stroke,fill}%
\end{pgfscope}%
\begin{pgfscope}%
\pgfpathrectangle{\pgfqpoint{0.600000in}{0.600000in}}{\pgfqpoint{3.900000in}{3.900000in}}%
\pgfusepath{clip}%
\pgfsetbuttcap%
\pgfsetroundjoin%
\definecolor{currentfill}{rgb}{0.121569,0.466667,0.705882}%
\pgfsetfillcolor{currentfill}%
\pgfsetlinewidth{1.003750pt}%
\definecolor{currentstroke}{rgb}{0.121569,0.466667,0.705882}%
\pgfsetstrokecolor{currentstroke}%
\pgfsetdash{}{0pt}%
\pgfpathmoveto{\pgfqpoint{2.616194in}{2.185064in}}%
\pgfpathcurveto{\pgfqpoint{2.627244in}{2.185064in}}{\pgfqpoint{2.637843in}{2.189454in}}{\pgfqpoint{2.645657in}{2.197268in}}%
\pgfpathcurveto{\pgfqpoint{2.653471in}{2.205082in}}{\pgfqpoint{2.657861in}{2.215681in}}{\pgfqpoint{2.657861in}{2.226731in}}%
\pgfpathcurveto{\pgfqpoint{2.657861in}{2.237781in}}{\pgfqpoint{2.653471in}{2.248380in}}{\pgfqpoint{2.645657in}{2.256194in}}%
\pgfpathcurveto{\pgfqpoint{2.637843in}{2.264007in}}{\pgfqpoint{2.627244in}{2.268398in}}{\pgfqpoint{2.616194in}{2.268398in}}%
\pgfpathcurveto{\pgfqpoint{2.605144in}{2.268398in}}{\pgfqpoint{2.594545in}{2.264007in}}{\pgfqpoint{2.586732in}{2.256194in}}%
\pgfpathcurveto{\pgfqpoint{2.578918in}{2.248380in}}{\pgfqpoint{2.574528in}{2.237781in}}{\pgfqpoint{2.574528in}{2.226731in}}%
\pgfpathcurveto{\pgfqpoint{2.574528in}{2.215681in}}{\pgfqpoint{2.578918in}{2.205082in}}{\pgfqpoint{2.586732in}{2.197268in}}%
\pgfpathcurveto{\pgfqpoint{2.594545in}{2.189454in}}{\pgfqpoint{2.605144in}{2.185064in}}{\pgfqpoint{2.616194in}{2.185064in}}%
\pgfpathclose%
\pgfusepath{stroke,fill}%
\end{pgfscope}%
\begin{pgfscope}%
\pgfpathrectangle{\pgfqpoint{0.600000in}{0.600000in}}{\pgfqpoint{3.900000in}{3.900000in}}%
\pgfusepath{clip}%
\pgfsetbuttcap%
\pgfsetroundjoin%
\definecolor{currentfill}{rgb}{0.121569,0.466667,0.705882}%
\pgfsetfillcolor{currentfill}%
\pgfsetlinewidth{1.003750pt}%
\definecolor{currentstroke}{rgb}{0.121569,0.466667,0.705882}%
\pgfsetstrokecolor{currentstroke}%
\pgfsetdash{}{0pt}%
\pgfpathmoveto{\pgfqpoint{2.314415in}{2.864726in}}%
\pgfpathcurveto{\pgfqpoint{2.325465in}{2.864726in}}{\pgfqpoint{2.336064in}{2.869117in}}{\pgfqpoint{2.343877in}{2.876930in}}%
\pgfpathcurveto{\pgfqpoint{2.351691in}{2.884744in}}{\pgfqpoint{2.356081in}{2.895343in}}{\pgfqpoint{2.356081in}{2.906393in}}%
\pgfpathcurveto{\pgfqpoint{2.356081in}{2.917443in}}{\pgfqpoint{2.351691in}{2.928042in}}{\pgfqpoint{2.343877in}{2.935856in}}%
\pgfpathcurveto{\pgfqpoint{2.336064in}{2.943670in}}{\pgfqpoint{2.325465in}{2.948060in}}{\pgfqpoint{2.314415in}{2.948060in}}%
\pgfpathcurveto{\pgfqpoint{2.303365in}{2.948060in}}{\pgfqpoint{2.292766in}{2.943670in}}{\pgfqpoint{2.284952in}{2.935856in}}%
\pgfpathcurveto{\pgfqpoint{2.277138in}{2.928042in}}{\pgfqpoint{2.272748in}{2.917443in}}{\pgfqpoint{2.272748in}{2.906393in}}%
\pgfpathcurveto{\pgfqpoint{2.272748in}{2.895343in}}{\pgfqpoint{2.277138in}{2.884744in}}{\pgfqpoint{2.284952in}{2.876930in}}%
\pgfpathcurveto{\pgfqpoint{2.292766in}{2.869117in}}{\pgfqpoint{2.303365in}{2.864726in}}{\pgfqpoint{2.314415in}{2.864726in}}%
\pgfpathclose%
\pgfusepath{stroke,fill}%
\end{pgfscope}%
\begin{pgfscope}%
\pgfpathrectangle{\pgfqpoint{0.600000in}{0.600000in}}{\pgfqpoint{3.900000in}{3.900000in}}%
\pgfusepath{clip}%
\pgfsetbuttcap%
\pgfsetroundjoin%
\definecolor{currentfill}{rgb}{0.121569,0.466667,0.705882}%
\pgfsetfillcolor{currentfill}%
\pgfsetlinewidth{1.003750pt}%
\definecolor{currentstroke}{rgb}{0.121569,0.466667,0.705882}%
\pgfsetstrokecolor{currentstroke}%
\pgfsetdash{}{0pt}%
\pgfpathmoveto{\pgfqpoint{2.599932in}{2.688589in}}%
\pgfpathcurveto{\pgfqpoint{2.610982in}{2.688589in}}{\pgfqpoint{2.621581in}{2.692979in}}{\pgfqpoint{2.629394in}{2.700793in}}%
\pgfpathcurveto{\pgfqpoint{2.637208in}{2.708606in}}{\pgfqpoint{2.641598in}{2.719205in}}{\pgfqpoint{2.641598in}{2.730255in}}%
\pgfpathcurveto{\pgfqpoint{2.641598in}{2.741305in}}{\pgfqpoint{2.637208in}{2.751905in}}{\pgfqpoint{2.629394in}{2.759718in}}%
\pgfpathcurveto{\pgfqpoint{2.621581in}{2.767532in}}{\pgfqpoint{2.610982in}{2.771922in}}{\pgfqpoint{2.599932in}{2.771922in}}%
\pgfpathcurveto{\pgfqpoint{2.588881in}{2.771922in}}{\pgfqpoint{2.578282in}{2.767532in}}{\pgfqpoint{2.570469in}{2.759718in}}%
\pgfpathcurveto{\pgfqpoint{2.562655in}{2.751905in}}{\pgfqpoint{2.558265in}{2.741305in}}{\pgfqpoint{2.558265in}{2.730255in}}%
\pgfpathcurveto{\pgfqpoint{2.558265in}{2.719205in}}{\pgfqpoint{2.562655in}{2.708606in}}{\pgfqpoint{2.570469in}{2.700793in}}%
\pgfpathcurveto{\pgfqpoint{2.578282in}{2.692979in}}{\pgfqpoint{2.588881in}{2.688589in}}{\pgfqpoint{2.599932in}{2.688589in}}%
\pgfpathclose%
\pgfusepath{stroke,fill}%
\end{pgfscope}%
\begin{pgfscope}%
\pgfpathrectangle{\pgfqpoint{0.600000in}{0.600000in}}{\pgfqpoint{3.900000in}{3.900000in}}%
\pgfusepath{clip}%
\pgfsetbuttcap%
\pgfsetroundjoin%
\definecolor{currentfill}{rgb}{0.121569,0.466667,0.705882}%
\pgfsetfillcolor{currentfill}%
\pgfsetlinewidth{1.003750pt}%
\definecolor{currentstroke}{rgb}{0.121569,0.466667,0.705882}%
\pgfsetstrokecolor{currentstroke}%
\pgfsetdash{}{0pt}%
\pgfpathmoveto{\pgfqpoint{2.761127in}{2.153116in}}%
\pgfpathcurveto{\pgfqpoint{2.772177in}{2.153116in}}{\pgfqpoint{2.782776in}{2.157506in}}{\pgfqpoint{2.790590in}{2.165320in}}%
\pgfpathcurveto{\pgfqpoint{2.798404in}{2.173134in}}{\pgfqpoint{2.802794in}{2.183733in}}{\pgfqpoint{2.802794in}{2.194783in}}%
\pgfpathcurveto{\pgfqpoint{2.802794in}{2.205833in}}{\pgfqpoint{2.798404in}{2.216432in}}{\pgfqpoint{2.790590in}{2.224246in}}%
\pgfpathcurveto{\pgfqpoint{2.782776in}{2.232059in}}{\pgfqpoint{2.772177in}{2.236449in}}{\pgfqpoint{2.761127in}{2.236449in}}%
\pgfpathcurveto{\pgfqpoint{2.750077in}{2.236449in}}{\pgfqpoint{2.739478in}{2.232059in}}{\pgfqpoint{2.731665in}{2.224246in}}%
\pgfpathcurveto{\pgfqpoint{2.723851in}{2.216432in}}{\pgfqpoint{2.719461in}{2.205833in}}{\pgfqpoint{2.719461in}{2.194783in}}%
\pgfpathcurveto{\pgfqpoint{2.719461in}{2.183733in}}{\pgfqpoint{2.723851in}{2.173134in}}{\pgfqpoint{2.731665in}{2.165320in}}%
\pgfpathcurveto{\pgfqpoint{2.739478in}{2.157506in}}{\pgfqpoint{2.750077in}{2.153116in}}{\pgfqpoint{2.761127in}{2.153116in}}%
\pgfpathclose%
\pgfusepath{stroke,fill}%
\end{pgfscope}%
\begin{pgfscope}%
\pgfpathrectangle{\pgfqpoint{0.600000in}{0.600000in}}{\pgfqpoint{3.900000in}{3.900000in}}%
\pgfusepath{clip}%
\pgfsetbuttcap%
\pgfsetroundjoin%
\definecolor{currentfill}{rgb}{0.121569,0.466667,0.705882}%
\pgfsetfillcolor{currentfill}%
\pgfsetlinewidth{1.003750pt}%
\definecolor{currentstroke}{rgb}{0.121569,0.466667,0.705882}%
\pgfsetstrokecolor{currentstroke}%
\pgfsetdash{}{0pt}%
\pgfpathmoveto{\pgfqpoint{2.201876in}{3.172628in}}%
\pgfpathcurveto{\pgfqpoint{2.212926in}{3.172628in}}{\pgfqpoint{2.223525in}{3.177018in}}{\pgfqpoint{2.231339in}{3.184832in}}%
\pgfpathcurveto{\pgfqpoint{2.239153in}{3.192645in}}{\pgfqpoint{2.243543in}{3.203244in}}{\pgfqpoint{2.243543in}{3.214295in}}%
\pgfpathcurveto{\pgfqpoint{2.243543in}{3.225345in}}{\pgfqpoint{2.239153in}{3.235944in}}{\pgfqpoint{2.231339in}{3.243757in}}%
\pgfpathcurveto{\pgfqpoint{2.223525in}{3.251571in}}{\pgfqpoint{2.212926in}{3.255961in}}{\pgfqpoint{2.201876in}{3.255961in}}%
\pgfpathcurveto{\pgfqpoint{2.190826in}{3.255961in}}{\pgfqpoint{2.180227in}{3.251571in}}{\pgfqpoint{2.172413in}{3.243757in}}%
\pgfpathcurveto{\pgfqpoint{2.164600in}{3.235944in}}{\pgfqpoint{2.160210in}{3.225345in}}{\pgfqpoint{2.160210in}{3.214295in}}%
\pgfpathcurveto{\pgfqpoint{2.160210in}{3.203244in}}{\pgfqpoint{2.164600in}{3.192645in}}{\pgfqpoint{2.172413in}{3.184832in}}%
\pgfpathcurveto{\pgfqpoint{2.180227in}{3.177018in}}{\pgfqpoint{2.190826in}{3.172628in}}{\pgfqpoint{2.201876in}{3.172628in}}%
\pgfpathclose%
\pgfusepath{stroke,fill}%
\end{pgfscope}%
\begin{pgfscope}%
\pgfpathrectangle{\pgfqpoint{0.600000in}{0.600000in}}{\pgfqpoint{3.900000in}{3.900000in}}%
\pgfusepath{clip}%
\pgfsetbuttcap%
\pgfsetroundjoin%
\definecolor{currentfill}{rgb}{0.121569,0.466667,0.705882}%
\pgfsetfillcolor{currentfill}%
\pgfsetlinewidth{1.003750pt}%
\definecolor{currentstroke}{rgb}{0.121569,0.466667,0.705882}%
\pgfsetstrokecolor{currentstroke}%
\pgfsetdash{}{0pt}%
\pgfpathmoveto{\pgfqpoint{1.809104in}{2.363138in}}%
\pgfpathcurveto{\pgfqpoint{1.820154in}{2.363138in}}{\pgfqpoint{1.830753in}{2.367529in}}{\pgfqpoint{1.838567in}{2.375342in}}%
\pgfpathcurveto{\pgfqpoint{1.846381in}{2.383156in}}{\pgfqpoint{1.850771in}{2.393755in}}{\pgfqpoint{1.850771in}{2.404805in}}%
\pgfpathcurveto{\pgfqpoint{1.850771in}{2.415855in}}{\pgfqpoint{1.846381in}{2.426454in}}{\pgfqpoint{1.838567in}{2.434268in}}%
\pgfpathcurveto{\pgfqpoint{1.830753in}{2.442081in}}{\pgfqpoint{1.820154in}{2.446472in}}{\pgfqpoint{1.809104in}{2.446472in}}%
\pgfpathcurveto{\pgfqpoint{1.798054in}{2.446472in}}{\pgfqpoint{1.787455in}{2.442081in}}{\pgfqpoint{1.779641in}{2.434268in}}%
\pgfpathcurveto{\pgfqpoint{1.771828in}{2.426454in}}{\pgfqpoint{1.767438in}{2.415855in}}{\pgfqpoint{1.767438in}{2.404805in}}%
\pgfpathcurveto{\pgfqpoint{1.767438in}{2.393755in}}{\pgfqpoint{1.771828in}{2.383156in}}{\pgfqpoint{1.779641in}{2.375342in}}%
\pgfpathcurveto{\pgfqpoint{1.787455in}{2.367529in}}{\pgfqpoint{1.798054in}{2.363138in}}{\pgfqpoint{1.809104in}{2.363138in}}%
\pgfpathclose%
\pgfusepath{stroke,fill}%
\end{pgfscope}%
\begin{pgfscope}%
\pgfpathrectangle{\pgfqpoint{0.600000in}{0.600000in}}{\pgfqpoint{3.900000in}{3.900000in}}%
\pgfusepath{clip}%
\pgfsetbuttcap%
\pgfsetroundjoin%
\definecolor{currentfill}{rgb}{0.121569,0.466667,0.705882}%
\pgfsetfillcolor{currentfill}%
\pgfsetlinewidth{1.003750pt}%
\definecolor{currentstroke}{rgb}{0.121569,0.466667,0.705882}%
\pgfsetstrokecolor{currentstroke}%
\pgfsetdash{}{0pt}%
\pgfpathmoveto{\pgfqpoint{3.104623in}{2.563318in}}%
\pgfpathcurveto{\pgfqpoint{3.115673in}{2.563318in}}{\pgfqpoint{3.126272in}{2.567708in}}{\pgfqpoint{3.134086in}{2.575522in}}%
\pgfpathcurveto{\pgfqpoint{3.141900in}{2.583335in}}{\pgfqpoint{3.146290in}{2.593934in}}{\pgfqpoint{3.146290in}{2.604984in}}%
\pgfpathcurveto{\pgfqpoint{3.146290in}{2.616035in}}{\pgfqpoint{3.141900in}{2.626634in}}{\pgfqpoint{3.134086in}{2.634447in}}%
\pgfpathcurveto{\pgfqpoint{3.126272in}{2.642261in}}{\pgfqpoint{3.115673in}{2.646651in}}{\pgfqpoint{3.104623in}{2.646651in}}%
\pgfpathcurveto{\pgfqpoint{3.093573in}{2.646651in}}{\pgfqpoint{3.082974in}{2.642261in}}{\pgfqpoint{3.075160in}{2.634447in}}%
\pgfpathcurveto{\pgfqpoint{3.067347in}{2.626634in}}{\pgfqpoint{3.062957in}{2.616035in}}{\pgfqpoint{3.062957in}{2.604984in}}%
\pgfpathcurveto{\pgfqpoint{3.062957in}{2.593934in}}{\pgfqpoint{3.067347in}{2.583335in}}{\pgfqpoint{3.075160in}{2.575522in}}%
\pgfpathcurveto{\pgfqpoint{3.082974in}{2.567708in}}{\pgfqpoint{3.093573in}{2.563318in}}{\pgfqpoint{3.104623in}{2.563318in}}%
\pgfpathclose%
\pgfusepath{stroke,fill}%
\end{pgfscope}%
\begin{pgfscope}%
\pgfpathrectangle{\pgfqpoint{0.600000in}{0.600000in}}{\pgfqpoint{3.900000in}{3.900000in}}%
\pgfusepath{clip}%
\pgfsetbuttcap%
\pgfsetroundjoin%
\definecolor{currentfill}{rgb}{0.121569,0.466667,0.705882}%
\pgfsetfillcolor{currentfill}%
\pgfsetlinewidth{1.003750pt}%
\definecolor{currentstroke}{rgb}{0.121569,0.466667,0.705882}%
\pgfsetstrokecolor{currentstroke}%
\pgfsetdash{}{0pt}%
\pgfpathmoveto{\pgfqpoint{1.905987in}{2.509625in}}%
\pgfpathcurveto{\pgfqpoint{1.917037in}{2.509625in}}{\pgfqpoint{1.927636in}{2.514015in}}{\pgfqpoint{1.935450in}{2.521829in}}%
\pgfpathcurveto{\pgfqpoint{1.943264in}{2.529643in}}{\pgfqpoint{1.947654in}{2.540242in}}{\pgfqpoint{1.947654in}{2.551292in}}%
\pgfpathcurveto{\pgfqpoint{1.947654in}{2.562342in}}{\pgfqpoint{1.943264in}{2.572941in}}{\pgfqpoint{1.935450in}{2.580755in}}%
\pgfpathcurveto{\pgfqpoint{1.927636in}{2.588568in}}{\pgfqpoint{1.917037in}{2.592959in}}{\pgfqpoint{1.905987in}{2.592959in}}%
\pgfpathcurveto{\pgfqpoint{1.894937in}{2.592959in}}{\pgfqpoint{1.884338in}{2.588568in}}{\pgfqpoint{1.876524in}{2.580755in}}%
\pgfpathcurveto{\pgfqpoint{1.868711in}{2.572941in}}{\pgfqpoint{1.864320in}{2.562342in}}{\pgfqpoint{1.864320in}{2.551292in}}%
\pgfpathcurveto{\pgfqpoint{1.864320in}{2.540242in}}{\pgfqpoint{1.868711in}{2.529643in}}{\pgfqpoint{1.876524in}{2.521829in}}%
\pgfpathcurveto{\pgfqpoint{1.884338in}{2.514015in}}{\pgfqpoint{1.894937in}{2.509625in}}{\pgfqpoint{1.905987in}{2.509625in}}%
\pgfpathclose%
\pgfusepath{stroke,fill}%
\end{pgfscope}%
\begin{pgfscope}%
\pgfpathrectangle{\pgfqpoint{0.600000in}{0.600000in}}{\pgfqpoint{3.900000in}{3.900000in}}%
\pgfusepath{clip}%
\pgfsetbuttcap%
\pgfsetroundjoin%
\definecolor{currentfill}{rgb}{0.121569,0.466667,0.705882}%
\pgfsetfillcolor{currentfill}%
\pgfsetlinewidth{1.003750pt}%
\definecolor{currentstroke}{rgb}{0.121569,0.466667,0.705882}%
\pgfsetstrokecolor{currentstroke}%
\pgfsetdash{}{0pt}%
\pgfpathmoveto{\pgfqpoint{1.698923in}{2.451956in}}%
\pgfpathcurveto{\pgfqpoint{1.709973in}{2.451956in}}{\pgfqpoint{1.720572in}{2.456346in}}{\pgfqpoint{1.728386in}{2.464160in}}%
\pgfpathcurveto{\pgfqpoint{1.736200in}{2.471973in}}{\pgfqpoint{1.740590in}{2.482572in}}{\pgfqpoint{1.740590in}{2.493622in}}%
\pgfpathcurveto{\pgfqpoint{1.740590in}{2.504672in}}{\pgfqpoint{1.736200in}{2.515272in}}{\pgfqpoint{1.728386in}{2.523085in}}%
\pgfpathcurveto{\pgfqpoint{1.720572in}{2.530899in}}{\pgfqpoint{1.709973in}{2.535289in}}{\pgfqpoint{1.698923in}{2.535289in}}%
\pgfpathcurveto{\pgfqpoint{1.687873in}{2.535289in}}{\pgfqpoint{1.677274in}{2.530899in}}{\pgfqpoint{1.669460in}{2.523085in}}%
\pgfpathcurveto{\pgfqpoint{1.661647in}{2.515272in}}{\pgfqpoint{1.657257in}{2.504672in}}{\pgfqpoint{1.657257in}{2.493622in}}%
\pgfpathcurveto{\pgfqpoint{1.657257in}{2.482572in}}{\pgfqpoint{1.661647in}{2.471973in}}{\pgfqpoint{1.669460in}{2.464160in}}%
\pgfpathcurveto{\pgfqpoint{1.677274in}{2.456346in}}{\pgfqpoint{1.687873in}{2.451956in}}{\pgfqpoint{1.698923in}{2.451956in}}%
\pgfpathclose%
\pgfusepath{stroke,fill}%
\end{pgfscope}%
\begin{pgfscope}%
\pgfpathrectangle{\pgfqpoint{0.600000in}{0.600000in}}{\pgfqpoint{3.900000in}{3.900000in}}%
\pgfusepath{clip}%
\pgfsetbuttcap%
\pgfsetroundjoin%
\definecolor{currentfill}{rgb}{0.121569,0.466667,0.705882}%
\pgfsetfillcolor{currentfill}%
\pgfsetlinewidth{1.003750pt}%
\definecolor{currentstroke}{rgb}{0.121569,0.466667,0.705882}%
\pgfsetstrokecolor{currentstroke}%
\pgfsetdash{}{0pt}%
\pgfpathmoveto{\pgfqpoint{1.885050in}{1.958331in}}%
\pgfpathcurveto{\pgfqpoint{1.896100in}{1.958331in}}{\pgfqpoint{1.906699in}{1.962721in}}{\pgfqpoint{1.914513in}{1.970534in}}%
\pgfpathcurveto{\pgfqpoint{1.922327in}{1.978348in}}{\pgfqpoint{1.926717in}{1.988947in}}{\pgfqpoint{1.926717in}{1.999997in}}%
\pgfpathcurveto{\pgfqpoint{1.926717in}{2.011047in}}{\pgfqpoint{1.922327in}{2.021646in}}{\pgfqpoint{1.914513in}{2.029460in}}%
\pgfpathcurveto{\pgfqpoint{1.906699in}{2.037274in}}{\pgfqpoint{1.896100in}{2.041664in}}{\pgfqpoint{1.885050in}{2.041664in}}%
\pgfpathcurveto{\pgfqpoint{1.874000in}{2.041664in}}{\pgfqpoint{1.863401in}{2.037274in}}{\pgfqpoint{1.855587in}{2.029460in}}%
\pgfpathcurveto{\pgfqpoint{1.847774in}{2.021646in}}{\pgfqpoint{1.843384in}{2.011047in}}{\pgfqpoint{1.843384in}{1.999997in}}%
\pgfpathcurveto{\pgfqpoint{1.843384in}{1.988947in}}{\pgfqpoint{1.847774in}{1.978348in}}{\pgfqpoint{1.855587in}{1.970534in}}%
\pgfpathcurveto{\pgfqpoint{1.863401in}{1.962721in}}{\pgfqpoint{1.874000in}{1.958331in}}{\pgfqpoint{1.885050in}{1.958331in}}%
\pgfpathclose%
\pgfusepath{stroke,fill}%
\end{pgfscope}%
\begin{pgfscope}%
\pgfpathrectangle{\pgfqpoint{0.600000in}{0.600000in}}{\pgfqpoint{3.900000in}{3.900000in}}%
\pgfusepath{clip}%
\pgfsetbuttcap%
\pgfsetroundjoin%
\definecolor{currentfill}{rgb}{0.121569,0.466667,0.705882}%
\pgfsetfillcolor{currentfill}%
\pgfsetlinewidth{1.003750pt}%
\definecolor{currentstroke}{rgb}{0.121569,0.466667,0.705882}%
\pgfsetstrokecolor{currentstroke}%
\pgfsetdash{}{0pt}%
\pgfpathmoveto{\pgfqpoint{3.237529in}{2.099718in}}%
\pgfpathcurveto{\pgfqpoint{3.248579in}{2.099718in}}{\pgfqpoint{3.259178in}{2.104108in}}{\pgfqpoint{3.266991in}{2.111922in}}%
\pgfpathcurveto{\pgfqpoint{3.274805in}{2.119736in}}{\pgfqpoint{3.279195in}{2.130335in}}{\pgfqpoint{3.279195in}{2.141385in}}%
\pgfpathcurveto{\pgfqpoint{3.279195in}{2.152435in}}{\pgfqpoint{3.274805in}{2.163034in}}{\pgfqpoint{3.266991in}{2.170848in}}%
\pgfpathcurveto{\pgfqpoint{3.259178in}{2.178661in}}{\pgfqpoint{3.248579in}{2.183051in}}{\pgfqpoint{3.237529in}{2.183051in}}%
\pgfpathcurveto{\pgfqpoint{3.226478in}{2.183051in}}{\pgfqpoint{3.215879in}{2.178661in}}{\pgfqpoint{3.208066in}{2.170848in}}%
\pgfpathcurveto{\pgfqpoint{3.200252in}{2.163034in}}{\pgfqpoint{3.195862in}{2.152435in}}{\pgfqpoint{3.195862in}{2.141385in}}%
\pgfpathcurveto{\pgfqpoint{3.195862in}{2.130335in}}{\pgfqpoint{3.200252in}{2.119736in}}{\pgfqpoint{3.208066in}{2.111922in}}%
\pgfpathcurveto{\pgfqpoint{3.215879in}{2.104108in}}{\pgfqpoint{3.226478in}{2.099718in}}{\pgfqpoint{3.237529in}{2.099718in}}%
\pgfpathclose%
\pgfusepath{stroke,fill}%
\end{pgfscope}%
\begin{pgfscope}%
\pgfpathrectangle{\pgfqpoint{0.600000in}{0.600000in}}{\pgfqpoint{3.900000in}{3.900000in}}%
\pgfusepath{clip}%
\pgfsetbuttcap%
\pgfsetroundjoin%
\definecolor{currentfill}{rgb}{0.121569,0.466667,0.705882}%
\pgfsetfillcolor{currentfill}%
\pgfsetlinewidth{1.003750pt}%
\definecolor{currentstroke}{rgb}{0.121569,0.466667,0.705882}%
\pgfsetstrokecolor{currentstroke}%
\pgfsetdash{}{0pt}%
\pgfpathmoveto{\pgfqpoint{2.396764in}{3.030461in}}%
\pgfpathcurveto{\pgfqpoint{2.407814in}{3.030461in}}{\pgfqpoint{2.418413in}{3.034851in}}{\pgfqpoint{2.426227in}{3.042665in}}%
\pgfpathcurveto{\pgfqpoint{2.434040in}{3.050478in}}{\pgfqpoint{2.438431in}{3.061077in}}{\pgfqpoint{2.438431in}{3.072128in}}%
\pgfpathcurveto{\pgfqpoint{2.438431in}{3.083178in}}{\pgfqpoint{2.434040in}{3.093777in}}{\pgfqpoint{2.426227in}{3.101590in}}%
\pgfpathcurveto{\pgfqpoint{2.418413in}{3.109404in}}{\pgfqpoint{2.407814in}{3.113794in}}{\pgfqpoint{2.396764in}{3.113794in}}%
\pgfpathcurveto{\pgfqpoint{2.385714in}{3.113794in}}{\pgfqpoint{2.375115in}{3.109404in}}{\pgfqpoint{2.367301in}{3.101590in}}%
\pgfpathcurveto{\pgfqpoint{2.359488in}{3.093777in}}{\pgfqpoint{2.355097in}{3.083178in}}{\pgfqpoint{2.355097in}{3.072128in}}%
\pgfpathcurveto{\pgfqpoint{2.355097in}{3.061077in}}{\pgfqpoint{2.359488in}{3.050478in}}{\pgfqpoint{2.367301in}{3.042665in}}%
\pgfpathcurveto{\pgfqpoint{2.375115in}{3.034851in}}{\pgfqpoint{2.385714in}{3.030461in}}{\pgfqpoint{2.396764in}{3.030461in}}%
\pgfpathclose%
\pgfusepath{stroke,fill}%
\end{pgfscope}%
\begin{pgfscope}%
\pgfpathrectangle{\pgfqpoint{0.600000in}{0.600000in}}{\pgfqpoint{3.900000in}{3.900000in}}%
\pgfusepath{clip}%
\pgfsetbuttcap%
\pgfsetroundjoin%
\definecolor{currentfill}{rgb}{0.121569,0.466667,0.705882}%
\pgfsetfillcolor{currentfill}%
\pgfsetlinewidth{1.003750pt}%
\definecolor{currentstroke}{rgb}{0.121569,0.466667,0.705882}%
\pgfsetstrokecolor{currentstroke}%
\pgfsetdash{}{0pt}%
\pgfpathmoveto{\pgfqpoint{2.934145in}{2.991378in}}%
\pgfpathcurveto{\pgfqpoint{2.945195in}{2.991378in}}{\pgfqpoint{2.955794in}{2.995768in}}{\pgfqpoint{2.963607in}{3.003582in}}%
\pgfpathcurveto{\pgfqpoint{2.971421in}{3.011396in}}{\pgfqpoint{2.975811in}{3.021995in}}{\pgfqpoint{2.975811in}{3.033045in}}%
\pgfpathcurveto{\pgfqpoint{2.975811in}{3.044095in}}{\pgfqpoint{2.971421in}{3.054694in}}{\pgfqpoint{2.963607in}{3.062508in}}%
\pgfpathcurveto{\pgfqpoint{2.955794in}{3.070321in}}{\pgfqpoint{2.945195in}{3.074711in}}{\pgfqpoint{2.934145in}{3.074711in}}%
\pgfpathcurveto{\pgfqpoint{2.923094in}{3.074711in}}{\pgfqpoint{2.912495in}{3.070321in}}{\pgfqpoint{2.904682in}{3.062508in}}%
\pgfpathcurveto{\pgfqpoint{2.896868in}{3.054694in}}{\pgfqpoint{2.892478in}{3.044095in}}{\pgfqpoint{2.892478in}{3.033045in}}%
\pgfpathcurveto{\pgfqpoint{2.892478in}{3.021995in}}{\pgfqpoint{2.896868in}{3.011396in}}{\pgfqpoint{2.904682in}{3.003582in}}%
\pgfpathcurveto{\pgfqpoint{2.912495in}{2.995768in}}{\pgfqpoint{2.923094in}{2.991378in}}{\pgfqpoint{2.934145in}{2.991378in}}%
\pgfpathclose%
\pgfusepath{stroke,fill}%
\end{pgfscope}%
\begin{pgfscope}%
\pgfpathrectangle{\pgfqpoint{0.600000in}{0.600000in}}{\pgfqpoint{3.900000in}{3.900000in}}%
\pgfusepath{clip}%
\pgfsetbuttcap%
\pgfsetroundjoin%
\definecolor{currentfill}{rgb}{0.121569,0.466667,0.705882}%
\pgfsetfillcolor{currentfill}%
\pgfsetlinewidth{1.003750pt}%
\definecolor{currentstroke}{rgb}{0.121569,0.466667,0.705882}%
\pgfsetstrokecolor{currentstroke}%
\pgfsetdash{}{0pt}%
\pgfpathmoveto{\pgfqpoint{2.873709in}{2.865271in}}%
\pgfpathcurveto{\pgfqpoint{2.884760in}{2.865271in}}{\pgfqpoint{2.895359in}{2.869661in}}{\pgfqpoint{2.903172in}{2.877474in}}%
\pgfpathcurveto{\pgfqpoint{2.910986in}{2.885288in}}{\pgfqpoint{2.915376in}{2.895887in}}{\pgfqpoint{2.915376in}{2.906937in}}%
\pgfpathcurveto{\pgfqpoint{2.915376in}{2.917987in}}{\pgfqpoint{2.910986in}{2.928586in}}{\pgfqpoint{2.903172in}{2.936400in}}%
\pgfpathcurveto{\pgfqpoint{2.895359in}{2.944214in}}{\pgfqpoint{2.884760in}{2.948604in}}{\pgfqpoint{2.873709in}{2.948604in}}%
\pgfpathcurveto{\pgfqpoint{2.862659in}{2.948604in}}{\pgfqpoint{2.852060in}{2.944214in}}{\pgfqpoint{2.844247in}{2.936400in}}%
\pgfpathcurveto{\pgfqpoint{2.836433in}{2.928586in}}{\pgfqpoint{2.832043in}{2.917987in}}{\pgfqpoint{2.832043in}{2.906937in}}%
\pgfpathcurveto{\pgfqpoint{2.832043in}{2.895887in}}{\pgfqpoint{2.836433in}{2.885288in}}{\pgfqpoint{2.844247in}{2.877474in}}%
\pgfpathcurveto{\pgfqpoint{2.852060in}{2.869661in}}{\pgfqpoint{2.862659in}{2.865271in}}{\pgfqpoint{2.873709in}{2.865271in}}%
\pgfpathclose%
\pgfusepath{stroke,fill}%
\end{pgfscope}%
\begin{pgfscope}%
\pgfpathrectangle{\pgfqpoint{0.600000in}{0.600000in}}{\pgfqpoint{3.900000in}{3.900000in}}%
\pgfusepath{clip}%
\pgfsetbuttcap%
\pgfsetroundjoin%
\definecolor{currentfill}{rgb}{0.121569,0.466667,0.705882}%
\pgfsetfillcolor{currentfill}%
\pgfsetlinewidth{1.003750pt}%
\definecolor{currentstroke}{rgb}{0.121569,0.466667,0.705882}%
\pgfsetstrokecolor{currentstroke}%
\pgfsetdash{}{0pt}%
\pgfpathmoveto{\pgfqpoint{2.232614in}{2.480196in}}%
\pgfpathcurveto{\pgfqpoint{2.243664in}{2.480196in}}{\pgfqpoint{2.254264in}{2.484587in}}{\pgfqpoint{2.262077in}{2.492400in}}%
\pgfpathcurveto{\pgfqpoint{2.269891in}{2.500214in}}{\pgfqpoint{2.274281in}{2.510813in}}{\pgfqpoint{2.274281in}{2.521863in}}%
\pgfpathcurveto{\pgfqpoint{2.274281in}{2.532913in}}{\pgfqpoint{2.269891in}{2.543512in}}{\pgfqpoint{2.262077in}{2.551326in}}%
\pgfpathcurveto{\pgfqpoint{2.254264in}{2.559139in}}{\pgfqpoint{2.243664in}{2.563530in}}{\pgfqpoint{2.232614in}{2.563530in}}%
\pgfpathcurveto{\pgfqpoint{2.221564in}{2.563530in}}{\pgfqpoint{2.210965in}{2.559139in}}{\pgfqpoint{2.203152in}{2.551326in}}%
\pgfpathcurveto{\pgfqpoint{2.195338in}{2.543512in}}{\pgfqpoint{2.190948in}{2.532913in}}{\pgfqpoint{2.190948in}{2.521863in}}%
\pgfpathcurveto{\pgfqpoint{2.190948in}{2.510813in}}{\pgfqpoint{2.195338in}{2.500214in}}{\pgfqpoint{2.203152in}{2.492400in}}%
\pgfpathcurveto{\pgfqpoint{2.210965in}{2.484587in}}{\pgfqpoint{2.221564in}{2.480196in}}{\pgfqpoint{2.232614in}{2.480196in}}%
\pgfpathclose%
\pgfusepath{stroke,fill}%
\end{pgfscope}%
\begin{pgfscope}%
\pgfpathrectangle{\pgfqpoint{0.600000in}{0.600000in}}{\pgfqpoint{3.900000in}{3.900000in}}%
\pgfusepath{clip}%
\pgfsetbuttcap%
\pgfsetroundjoin%
\definecolor{currentfill}{rgb}{0.121569,0.466667,0.705882}%
\pgfsetfillcolor{currentfill}%
\pgfsetlinewidth{1.003750pt}%
\definecolor{currentstroke}{rgb}{0.121569,0.466667,0.705882}%
\pgfsetstrokecolor{currentstroke}%
\pgfsetdash{}{0pt}%
\pgfpathmoveto{\pgfqpoint{2.926233in}{3.085796in}}%
\pgfpathcurveto{\pgfqpoint{2.937283in}{3.085796in}}{\pgfqpoint{2.947882in}{3.090186in}}{\pgfqpoint{2.955696in}{3.098000in}}%
\pgfpathcurveto{\pgfqpoint{2.963510in}{3.105814in}}{\pgfqpoint{2.967900in}{3.116413in}}{\pgfqpoint{2.967900in}{3.127463in}}%
\pgfpathcurveto{\pgfqpoint{2.967900in}{3.138513in}}{\pgfqpoint{2.963510in}{3.149112in}}{\pgfqpoint{2.955696in}{3.156926in}}%
\pgfpathcurveto{\pgfqpoint{2.947882in}{3.164739in}}{\pgfqpoint{2.937283in}{3.169129in}}{\pgfqpoint{2.926233in}{3.169129in}}%
\pgfpathcurveto{\pgfqpoint{2.915183in}{3.169129in}}{\pgfqpoint{2.904584in}{3.164739in}}{\pgfqpoint{2.896771in}{3.156926in}}%
\pgfpathcurveto{\pgfqpoint{2.888957in}{3.149112in}}{\pgfqpoint{2.884567in}{3.138513in}}{\pgfqpoint{2.884567in}{3.127463in}}%
\pgfpathcurveto{\pgfqpoint{2.884567in}{3.116413in}}{\pgfqpoint{2.888957in}{3.105814in}}{\pgfqpoint{2.896771in}{3.098000in}}%
\pgfpathcurveto{\pgfqpoint{2.904584in}{3.090186in}}{\pgfqpoint{2.915183in}{3.085796in}}{\pgfqpoint{2.926233in}{3.085796in}}%
\pgfpathclose%
\pgfusepath{stroke,fill}%
\end{pgfscope}%
\begin{pgfscope}%
\pgfpathrectangle{\pgfqpoint{0.600000in}{0.600000in}}{\pgfqpoint{3.900000in}{3.900000in}}%
\pgfusepath{clip}%
\pgfsetbuttcap%
\pgfsetroundjoin%
\definecolor{currentfill}{rgb}{0.121569,0.466667,0.705882}%
\pgfsetfillcolor{currentfill}%
\pgfsetlinewidth{1.003750pt}%
\definecolor{currentstroke}{rgb}{0.121569,0.466667,0.705882}%
\pgfsetstrokecolor{currentstroke}%
\pgfsetdash{}{0pt}%
\pgfpathmoveto{\pgfqpoint{2.133635in}{2.555865in}}%
\pgfpathcurveto{\pgfqpoint{2.144685in}{2.555865in}}{\pgfqpoint{2.155284in}{2.560255in}}{\pgfqpoint{2.163098in}{2.568069in}}%
\pgfpathcurveto{\pgfqpoint{2.170912in}{2.575882in}}{\pgfqpoint{2.175302in}{2.586481in}}{\pgfqpoint{2.175302in}{2.597532in}}%
\pgfpathcurveto{\pgfqpoint{2.175302in}{2.608582in}}{\pgfqpoint{2.170912in}{2.619181in}}{\pgfqpoint{2.163098in}{2.626994in}}%
\pgfpathcurveto{\pgfqpoint{2.155284in}{2.634808in}}{\pgfqpoint{2.144685in}{2.639198in}}{\pgfqpoint{2.133635in}{2.639198in}}%
\pgfpathcurveto{\pgfqpoint{2.122585in}{2.639198in}}{\pgfqpoint{2.111986in}{2.634808in}}{\pgfqpoint{2.104172in}{2.626994in}}%
\pgfpathcurveto{\pgfqpoint{2.096359in}{2.619181in}}{\pgfqpoint{2.091968in}{2.608582in}}{\pgfqpoint{2.091968in}{2.597532in}}%
\pgfpathcurveto{\pgfqpoint{2.091968in}{2.586481in}}{\pgfqpoint{2.096359in}{2.575882in}}{\pgfqpoint{2.104172in}{2.568069in}}%
\pgfpathcurveto{\pgfqpoint{2.111986in}{2.560255in}}{\pgfqpoint{2.122585in}{2.555865in}}{\pgfqpoint{2.133635in}{2.555865in}}%
\pgfpathclose%
\pgfusepath{stroke,fill}%
\end{pgfscope}%
\begin{pgfscope}%
\pgfpathrectangle{\pgfqpoint{0.600000in}{0.600000in}}{\pgfqpoint{3.900000in}{3.900000in}}%
\pgfusepath{clip}%
\pgfsetbuttcap%
\pgfsetroundjoin%
\definecolor{currentfill}{rgb}{0.121569,0.466667,0.705882}%
\pgfsetfillcolor{currentfill}%
\pgfsetlinewidth{1.003750pt}%
\definecolor{currentstroke}{rgb}{0.121569,0.466667,0.705882}%
\pgfsetstrokecolor{currentstroke}%
\pgfsetdash{}{0pt}%
\pgfpathmoveto{\pgfqpoint{2.806054in}{2.179622in}}%
\pgfpathcurveto{\pgfqpoint{2.817105in}{2.179622in}}{\pgfqpoint{2.827704in}{2.184012in}}{\pgfqpoint{2.835517in}{2.191826in}}%
\pgfpathcurveto{\pgfqpoint{2.843331in}{2.199639in}}{\pgfqpoint{2.847721in}{2.210238in}}{\pgfqpoint{2.847721in}{2.221288in}}%
\pgfpathcurveto{\pgfqpoint{2.847721in}{2.232339in}}{\pgfqpoint{2.843331in}{2.242938in}}{\pgfqpoint{2.835517in}{2.250751in}}%
\pgfpathcurveto{\pgfqpoint{2.827704in}{2.258565in}}{\pgfqpoint{2.817105in}{2.262955in}}{\pgfqpoint{2.806054in}{2.262955in}}%
\pgfpathcurveto{\pgfqpoint{2.795004in}{2.262955in}}{\pgfqpoint{2.784405in}{2.258565in}}{\pgfqpoint{2.776592in}{2.250751in}}%
\pgfpathcurveto{\pgfqpoint{2.768778in}{2.242938in}}{\pgfqpoint{2.764388in}{2.232339in}}{\pgfqpoint{2.764388in}{2.221288in}}%
\pgfpathcurveto{\pgfqpoint{2.764388in}{2.210238in}}{\pgfqpoint{2.768778in}{2.199639in}}{\pgfqpoint{2.776592in}{2.191826in}}%
\pgfpathcurveto{\pgfqpoint{2.784405in}{2.184012in}}{\pgfqpoint{2.795004in}{2.179622in}}{\pgfqpoint{2.806054in}{2.179622in}}%
\pgfpathclose%
\pgfusepath{stroke,fill}%
\end{pgfscope}%
\begin{pgfscope}%
\pgfpathrectangle{\pgfqpoint{0.600000in}{0.600000in}}{\pgfqpoint{3.900000in}{3.900000in}}%
\pgfusepath{clip}%
\pgfsetbuttcap%
\pgfsetroundjoin%
\definecolor{currentfill}{rgb}{0.121569,0.466667,0.705882}%
\pgfsetfillcolor{currentfill}%
\pgfsetlinewidth{1.003750pt}%
\definecolor{currentstroke}{rgb}{0.121569,0.466667,0.705882}%
\pgfsetstrokecolor{currentstroke}%
\pgfsetdash{}{0pt}%
\pgfpathmoveto{\pgfqpoint{2.248349in}{2.560668in}}%
\pgfpathcurveto{\pgfqpoint{2.259400in}{2.560668in}}{\pgfqpoint{2.269999in}{2.565059in}}{\pgfqpoint{2.277812in}{2.572872in}}%
\pgfpathcurveto{\pgfqpoint{2.285626in}{2.580686in}}{\pgfqpoint{2.290016in}{2.591285in}}{\pgfqpoint{2.290016in}{2.602335in}}%
\pgfpathcurveto{\pgfqpoint{2.290016in}{2.613385in}}{\pgfqpoint{2.285626in}{2.623984in}}{\pgfqpoint{2.277812in}{2.631798in}}%
\pgfpathcurveto{\pgfqpoint{2.269999in}{2.639611in}}{\pgfqpoint{2.259400in}{2.644002in}}{\pgfqpoint{2.248349in}{2.644002in}}%
\pgfpathcurveto{\pgfqpoint{2.237299in}{2.644002in}}{\pgfqpoint{2.226700in}{2.639611in}}{\pgfqpoint{2.218887in}{2.631798in}}%
\pgfpathcurveto{\pgfqpoint{2.211073in}{2.623984in}}{\pgfqpoint{2.206683in}{2.613385in}}{\pgfqpoint{2.206683in}{2.602335in}}%
\pgfpathcurveto{\pgfqpoint{2.206683in}{2.591285in}}{\pgfqpoint{2.211073in}{2.580686in}}{\pgfqpoint{2.218887in}{2.572872in}}%
\pgfpathcurveto{\pgfqpoint{2.226700in}{2.565059in}}{\pgfqpoint{2.237299in}{2.560668in}}{\pgfqpoint{2.248349in}{2.560668in}}%
\pgfpathclose%
\pgfusepath{stroke,fill}%
\end{pgfscope}%
\begin{pgfscope}%
\pgfpathrectangle{\pgfqpoint{0.600000in}{0.600000in}}{\pgfqpoint{3.900000in}{3.900000in}}%
\pgfusepath{clip}%
\pgfsetbuttcap%
\pgfsetroundjoin%
\definecolor{currentfill}{rgb}{0.121569,0.466667,0.705882}%
\pgfsetfillcolor{currentfill}%
\pgfsetlinewidth{1.003750pt}%
\definecolor{currentstroke}{rgb}{0.121569,0.466667,0.705882}%
\pgfsetstrokecolor{currentstroke}%
\pgfsetdash{}{0pt}%
\pgfpathmoveto{\pgfqpoint{2.739209in}{2.618422in}}%
\pgfpathcurveto{\pgfqpoint{2.750259in}{2.618422in}}{\pgfqpoint{2.760858in}{2.622812in}}{\pgfqpoint{2.768672in}{2.630625in}}%
\pgfpathcurveto{\pgfqpoint{2.776486in}{2.638439in}}{\pgfqpoint{2.780876in}{2.649038in}}{\pgfqpoint{2.780876in}{2.660088in}}%
\pgfpathcurveto{\pgfqpoint{2.780876in}{2.671138in}}{\pgfqpoint{2.776486in}{2.681737in}}{\pgfqpoint{2.768672in}{2.689551in}}%
\pgfpathcurveto{\pgfqpoint{2.760858in}{2.697365in}}{\pgfqpoint{2.750259in}{2.701755in}}{\pgfqpoint{2.739209in}{2.701755in}}%
\pgfpathcurveto{\pgfqpoint{2.728159in}{2.701755in}}{\pgfqpoint{2.717560in}{2.697365in}}{\pgfqpoint{2.709746in}{2.689551in}}%
\pgfpathcurveto{\pgfqpoint{2.701933in}{2.681737in}}{\pgfqpoint{2.697543in}{2.671138in}}{\pgfqpoint{2.697543in}{2.660088in}}%
\pgfpathcurveto{\pgfqpoint{2.697543in}{2.649038in}}{\pgfqpoint{2.701933in}{2.638439in}}{\pgfqpoint{2.709746in}{2.630625in}}%
\pgfpathcurveto{\pgfqpoint{2.717560in}{2.622812in}}{\pgfqpoint{2.728159in}{2.618422in}}{\pgfqpoint{2.739209in}{2.618422in}}%
\pgfpathclose%
\pgfusepath{stroke,fill}%
\end{pgfscope}%
\begin{pgfscope}%
\pgfpathrectangle{\pgfqpoint{0.600000in}{0.600000in}}{\pgfqpoint{3.900000in}{3.900000in}}%
\pgfusepath{clip}%
\pgfsetbuttcap%
\pgfsetroundjoin%
\definecolor{currentfill}{rgb}{0.121569,0.466667,0.705882}%
\pgfsetfillcolor{currentfill}%
\pgfsetlinewidth{1.003750pt}%
\definecolor{currentstroke}{rgb}{0.121569,0.466667,0.705882}%
\pgfsetstrokecolor{currentstroke}%
\pgfsetdash{}{0pt}%
\pgfpathmoveto{\pgfqpoint{2.122049in}{1.634604in}}%
\pgfpathcurveto{\pgfqpoint{2.133099in}{1.634604in}}{\pgfqpoint{2.143698in}{1.638994in}}{\pgfqpoint{2.151511in}{1.646808in}}%
\pgfpathcurveto{\pgfqpoint{2.159325in}{1.654621in}}{\pgfqpoint{2.163715in}{1.665221in}}{\pgfqpoint{2.163715in}{1.676271in}}%
\pgfpathcurveto{\pgfqpoint{2.163715in}{1.687321in}}{\pgfqpoint{2.159325in}{1.697920in}}{\pgfqpoint{2.151511in}{1.705733in}}%
\pgfpathcurveto{\pgfqpoint{2.143698in}{1.713547in}}{\pgfqpoint{2.133099in}{1.717937in}}{\pgfqpoint{2.122049in}{1.717937in}}%
\pgfpathcurveto{\pgfqpoint{2.110998in}{1.717937in}}{\pgfqpoint{2.100399in}{1.713547in}}{\pgfqpoint{2.092586in}{1.705733in}}%
\pgfpathcurveto{\pgfqpoint{2.084772in}{1.697920in}}{\pgfqpoint{2.080382in}{1.687321in}}{\pgfqpoint{2.080382in}{1.676271in}}%
\pgfpathcurveto{\pgfqpoint{2.080382in}{1.665221in}}{\pgfqpoint{2.084772in}{1.654621in}}{\pgfqpoint{2.092586in}{1.646808in}}%
\pgfpathcurveto{\pgfqpoint{2.100399in}{1.638994in}}{\pgfqpoint{2.110998in}{1.634604in}}{\pgfqpoint{2.122049in}{1.634604in}}%
\pgfpathclose%
\pgfusepath{stroke,fill}%
\end{pgfscope}%
\begin{pgfscope}%
\pgfpathrectangle{\pgfqpoint{0.600000in}{0.600000in}}{\pgfqpoint{3.900000in}{3.900000in}}%
\pgfusepath{clip}%
\pgfsetbuttcap%
\pgfsetroundjoin%
\definecolor{currentfill}{rgb}{0.121569,0.466667,0.705882}%
\pgfsetfillcolor{currentfill}%
\pgfsetlinewidth{1.003750pt}%
\definecolor{currentstroke}{rgb}{0.121569,0.466667,0.705882}%
\pgfsetstrokecolor{currentstroke}%
\pgfsetdash{}{0pt}%
\pgfpathmoveto{\pgfqpoint{2.865866in}{2.309785in}}%
\pgfpathcurveto{\pgfqpoint{2.876916in}{2.309785in}}{\pgfqpoint{2.887515in}{2.314176in}}{\pgfqpoint{2.895329in}{2.321989in}}%
\pgfpathcurveto{\pgfqpoint{2.903142in}{2.329803in}}{\pgfqpoint{2.907533in}{2.340402in}}{\pgfqpoint{2.907533in}{2.351452in}}%
\pgfpathcurveto{\pgfqpoint{2.907533in}{2.362502in}}{\pgfqpoint{2.903142in}{2.373101in}}{\pgfqpoint{2.895329in}{2.380915in}}%
\pgfpathcurveto{\pgfqpoint{2.887515in}{2.388728in}}{\pgfqpoint{2.876916in}{2.393119in}}{\pgfqpoint{2.865866in}{2.393119in}}%
\pgfpathcurveto{\pgfqpoint{2.854816in}{2.393119in}}{\pgfqpoint{2.844217in}{2.388728in}}{\pgfqpoint{2.836403in}{2.380915in}}%
\pgfpathcurveto{\pgfqpoint{2.828590in}{2.373101in}}{\pgfqpoint{2.824199in}{2.362502in}}{\pgfqpoint{2.824199in}{2.351452in}}%
\pgfpathcurveto{\pgfqpoint{2.824199in}{2.340402in}}{\pgfqpoint{2.828590in}{2.329803in}}{\pgfqpoint{2.836403in}{2.321989in}}%
\pgfpathcurveto{\pgfqpoint{2.844217in}{2.314176in}}{\pgfqpoint{2.854816in}{2.309785in}}{\pgfqpoint{2.865866in}{2.309785in}}%
\pgfpathclose%
\pgfusepath{stroke,fill}%
\end{pgfscope}%
\begin{pgfscope}%
\pgfpathrectangle{\pgfqpoint{0.600000in}{0.600000in}}{\pgfqpoint{3.900000in}{3.900000in}}%
\pgfusepath{clip}%
\pgfsetbuttcap%
\pgfsetroundjoin%
\definecolor{currentfill}{rgb}{0.121569,0.466667,0.705882}%
\pgfsetfillcolor{currentfill}%
\pgfsetlinewidth{1.003750pt}%
\definecolor{currentstroke}{rgb}{0.121569,0.466667,0.705882}%
\pgfsetstrokecolor{currentstroke}%
\pgfsetdash{}{0pt}%
\pgfpathmoveto{\pgfqpoint{2.965214in}{1.984778in}}%
\pgfpathcurveto{\pgfqpoint{2.976264in}{1.984778in}}{\pgfqpoint{2.986863in}{1.989169in}}{\pgfqpoint{2.994676in}{1.996982in}}%
\pgfpathcurveto{\pgfqpoint{3.002490in}{2.004796in}}{\pgfqpoint{3.006880in}{2.015395in}}{\pgfqpoint{3.006880in}{2.026445in}}%
\pgfpathcurveto{\pgfqpoint{3.006880in}{2.037495in}}{\pgfqpoint{3.002490in}{2.048094in}}{\pgfqpoint{2.994676in}{2.055908in}}%
\pgfpathcurveto{\pgfqpoint{2.986863in}{2.063721in}}{\pgfqpoint{2.976264in}{2.068112in}}{\pgfqpoint{2.965214in}{2.068112in}}%
\pgfpathcurveto{\pgfqpoint{2.954164in}{2.068112in}}{\pgfqpoint{2.943565in}{2.063721in}}{\pgfqpoint{2.935751in}{2.055908in}}%
\pgfpathcurveto{\pgfqpoint{2.927937in}{2.048094in}}{\pgfqpoint{2.923547in}{2.037495in}}{\pgfqpoint{2.923547in}{2.026445in}}%
\pgfpathcurveto{\pgfqpoint{2.923547in}{2.015395in}}{\pgfqpoint{2.927937in}{2.004796in}}{\pgfqpoint{2.935751in}{1.996982in}}%
\pgfpathcurveto{\pgfqpoint{2.943565in}{1.989169in}}{\pgfqpoint{2.954164in}{1.984778in}}{\pgfqpoint{2.965214in}{1.984778in}}%
\pgfpathclose%
\pgfusepath{stroke,fill}%
\end{pgfscope}%
\begin{pgfscope}%
\pgfpathrectangle{\pgfqpoint{0.600000in}{0.600000in}}{\pgfqpoint{3.900000in}{3.900000in}}%
\pgfusepath{clip}%
\pgfsetbuttcap%
\pgfsetroundjoin%
\definecolor{currentfill}{rgb}{0.121569,0.466667,0.705882}%
\pgfsetfillcolor{currentfill}%
\pgfsetlinewidth{1.003750pt}%
\definecolor{currentstroke}{rgb}{0.121569,0.466667,0.705882}%
\pgfsetstrokecolor{currentstroke}%
\pgfsetdash{}{0pt}%
\pgfpathmoveto{\pgfqpoint{2.193869in}{1.856854in}}%
\pgfpathcurveto{\pgfqpoint{2.204920in}{1.856854in}}{\pgfqpoint{2.215519in}{1.861245in}}{\pgfqpoint{2.223332in}{1.869058in}}%
\pgfpathcurveto{\pgfqpoint{2.231146in}{1.876872in}}{\pgfqpoint{2.235536in}{1.887471in}}{\pgfqpoint{2.235536in}{1.898521in}}%
\pgfpathcurveto{\pgfqpoint{2.235536in}{1.909571in}}{\pgfqpoint{2.231146in}{1.920170in}}{\pgfqpoint{2.223332in}{1.927984in}}%
\pgfpathcurveto{\pgfqpoint{2.215519in}{1.935797in}}{\pgfqpoint{2.204920in}{1.940188in}}{\pgfqpoint{2.193869in}{1.940188in}}%
\pgfpathcurveto{\pgfqpoint{2.182819in}{1.940188in}}{\pgfqpoint{2.172220in}{1.935797in}}{\pgfqpoint{2.164407in}{1.927984in}}%
\pgfpathcurveto{\pgfqpoint{2.156593in}{1.920170in}}{\pgfqpoint{2.152203in}{1.909571in}}{\pgfqpoint{2.152203in}{1.898521in}}%
\pgfpathcurveto{\pgfqpoint{2.152203in}{1.887471in}}{\pgfqpoint{2.156593in}{1.876872in}}{\pgfqpoint{2.164407in}{1.869058in}}%
\pgfpathcurveto{\pgfqpoint{2.172220in}{1.861245in}}{\pgfqpoint{2.182819in}{1.856854in}}{\pgfqpoint{2.193869in}{1.856854in}}%
\pgfpathclose%
\pgfusepath{stroke,fill}%
\end{pgfscope}%
\begin{pgfscope}%
\pgfpathrectangle{\pgfqpoint{0.600000in}{0.600000in}}{\pgfqpoint{3.900000in}{3.900000in}}%
\pgfusepath{clip}%
\pgfsetbuttcap%
\pgfsetroundjoin%
\definecolor{currentfill}{rgb}{0.121569,0.466667,0.705882}%
\pgfsetfillcolor{currentfill}%
\pgfsetlinewidth{1.003750pt}%
\definecolor{currentstroke}{rgb}{0.121569,0.466667,0.705882}%
\pgfsetstrokecolor{currentstroke}%
\pgfsetdash{}{0pt}%
\pgfpathmoveto{\pgfqpoint{1.656790in}{2.575882in}}%
\pgfpathcurveto{\pgfqpoint{1.667841in}{2.575882in}}{\pgfqpoint{1.678440in}{2.580273in}}{\pgfqpoint{1.686253in}{2.588086in}}%
\pgfpathcurveto{\pgfqpoint{1.694067in}{2.595900in}}{\pgfqpoint{1.698457in}{2.606499in}}{\pgfqpoint{1.698457in}{2.617549in}}%
\pgfpathcurveto{\pgfqpoint{1.698457in}{2.628599in}}{\pgfqpoint{1.694067in}{2.639198in}}{\pgfqpoint{1.686253in}{2.647012in}}%
\pgfpathcurveto{\pgfqpoint{1.678440in}{2.654826in}}{\pgfqpoint{1.667841in}{2.659216in}}{\pgfqpoint{1.656790in}{2.659216in}}%
\pgfpathcurveto{\pgfqpoint{1.645740in}{2.659216in}}{\pgfqpoint{1.635141in}{2.654826in}}{\pgfqpoint{1.627328in}{2.647012in}}%
\pgfpathcurveto{\pgfqpoint{1.619514in}{2.639198in}}{\pgfqpoint{1.615124in}{2.628599in}}{\pgfqpoint{1.615124in}{2.617549in}}%
\pgfpathcurveto{\pgfqpoint{1.615124in}{2.606499in}}{\pgfqpoint{1.619514in}{2.595900in}}{\pgfqpoint{1.627328in}{2.588086in}}%
\pgfpathcurveto{\pgfqpoint{1.635141in}{2.580273in}}{\pgfqpoint{1.645740in}{2.575882in}}{\pgfqpoint{1.656790in}{2.575882in}}%
\pgfpathclose%
\pgfusepath{stroke,fill}%
\end{pgfscope}%
\begin{pgfscope}%
\pgfpathrectangle{\pgfqpoint{0.600000in}{0.600000in}}{\pgfqpoint{3.900000in}{3.900000in}}%
\pgfusepath{clip}%
\pgfsetbuttcap%
\pgfsetroundjoin%
\definecolor{currentfill}{rgb}{0.121569,0.466667,0.705882}%
\pgfsetfillcolor{currentfill}%
\pgfsetlinewidth{1.003750pt}%
\definecolor{currentstroke}{rgb}{0.121569,0.466667,0.705882}%
\pgfsetstrokecolor{currentstroke}%
\pgfsetdash{}{0pt}%
\pgfpathmoveto{\pgfqpoint{1.770071in}{2.999403in}}%
\pgfpathcurveto{\pgfqpoint{1.781121in}{2.999403in}}{\pgfqpoint{1.791720in}{3.003793in}}{\pgfqpoint{1.799534in}{3.011607in}}%
\pgfpathcurveto{\pgfqpoint{1.807348in}{3.019420in}}{\pgfqpoint{1.811738in}{3.030019in}}{\pgfqpoint{1.811738in}{3.041069in}}%
\pgfpathcurveto{\pgfqpoint{1.811738in}{3.052120in}}{\pgfqpoint{1.807348in}{3.062719in}}{\pgfqpoint{1.799534in}{3.070532in}}%
\pgfpathcurveto{\pgfqpoint{1.791720in}{3.078346in}}{\pgfqpoint{1.781121in}{3.082736in}}{\pgfqpoint{1.770071in}{3.082736in}}%
\pgfpathcurveto{\pgfqpoint{1.759021in}{3.082736in}}{\pgfqpoint{1.748422in}{3.078346in}}{\pgfqpoint{1.740608in}{3.070532in}}%
\pgfpathcurveto{\pgfqpoint{1.732795in}{3.062719in}}{\pgfqpoint{1.728405in}{3.052120in}}{\pgfqpoint{1.728405in}{3.041069in}}%
\pgfpathcurveto{\pgfqpoint{1.728405in}{3.030019in}}{\pgfqpoint{1.732795in}{3.019420in}}{\pgfqpoint{1.740608in}{3.011607in}}%
\pgfpathcurveto{\pgfqpoint{1.748422in}{3.003793in}}{\pgfqpoint{1.759021in}{2.999403in}}{\pgfqpoint{1.770071in}{2.999403in}}%
\pgfpathclose%
\pgfusepath{stroke,fill}%
\end{pgfscope}%
\begin{pgfscope}%
\pgfpathrectangle{\pgfqpoint{0.600000in}{0.600000in}}{\pgfqpoint{3.900000in}{3.900000in}}%
\pgfusepath{clip}%
\pgfsetbuttcap%
\pgfsetroundjoin%
\definecolor{currentfill}{rgb}{0.121569,0.466667,0.705882}%
\pgfsetfillcolor{currentfill}%
\pgfsetlinewidth{1.003750pt}%
\definecolor{currentstroke}{rgb}{0.121569,0.466667,0.705882}%
\pgfsetstrokecolor{currentstroke}%
\pgfsetdash{}{0pt}%
\pgfpathmoveto{\pgfqpoint{3.036961in}{1.993207in}}%
\pgfpathcurveto{\pgfqpoint{3.048011in}{1.993207in}}{\pgfqpoint{3.058610in}{1.997597in}}{\pgfqpoint{3.066424in}{2.005411in}}%
\pgfpathcurveto{\pgfqpoint{3.074238in}{2.013224in}}{\pgfqpoint{3.078628in}{2.023823in}}{\pgfqpoint{3.078628in}{2.034873in}}%
\pgfpathcurveto{\pgfqpoint{3.078628in}{2.045924in}}{\pgfqpoint{3.074238in}{2.056523in}}{\pgfqpoint{3.066424in}{2.064336in}}%
\pgfpathcurveto{\pgfqpoint{3.058610in}{2.072150in}}{\pgfqpoint{3.048011in}{2.076540in}}{\pgfqpoint{3.036961in}{2.076540in}}%
\pgfpathcurveto{\pgfqpoint{3.025911in}{2.076540in}}{\pgfqpoint{3.015312in}{2.072150in}}{\pgfqpoint{3.007498in}{2.064336in}}%
\pgfpathcurveto{\pgfqpoint{2.999685in}{2.056523in}}{\pgfqpoint{2.995295in}{2.045924in}}{\pgfqpoint{2.995295in}{2.034873in}}%
\pgfpathcurveto{\pgfqpoint{2.995295in}{2.023823in}}{\pgfqpoint{2.999685in}{2.013224in}}{\pgfqpoint{3.007498in}{2.005411in}}%
\pgfpathcurveto{\pgfqpoint{3.015312in}{1.997597in}}{\pgfqpoint{3.025911in}{1.993207in}}{\pgfqpoint{3.036961in}{1.993207in}}%
\pgfpathclose%
\pgfusepath{stroke,fill}%
\end{pgfscope}%
\begin{pgfscope}%
\pgfpathrectangle{\pgfqpoint{0.600000in}{0.600000in}}{\pgfqpoint{3.900000in}{3.900000in}}%
\pgfusepath{clip}%
\pgfsetbuttcap%
\pgfsetroundjoin%
\definecolor{currentfill}{rgb}{0.121569,0.466667,0.705882}%
\pgfsetfillcolor{currentfill}%
\pgfsetlinewidth{1.003750pt}%
\definecolor{currentstroke}{rgb}{0.121569,0.466667,0.705882}%
\pgfsetstrokecolor{currentstroke}%
\pgfsetdash{}{0pt}%
\pgfpathmoveto{\pgfqpoint{2.625646in}{1.808428in}}%
\pgfpathcurveto{\pgfqpoint{2.636696in}{1.808428in}}{\pgfqpoint{2.647295in}{1.812818in}}{\pgfqpoint{2.655109in}{1.820632in}}%
\pgfpathcurveto{\pgfqpoint{2.662922in}{1.828445in}}{\pgfqpoint{2.667313in}{1.839044in}}{\pgfqpoint{2.667313in}{1.850094in}}%
\pgfpathcurveto{\pgfqpoint{2.667313in}{1.861145in}}{\pgfqpoint{2.662922in}{1.871744in}}{\pgfqpoint{2.655109in}{1.879557in}}%
\pgfpathcurveto{\pgfqpoint{2.647295in}{1.887371in}}{\pgfqpoint{2.636696in}{1.891761in}}{\pgfqpoint{2.625646in}{1.891761in}}%
\pgfpathcurveto{\pgfqpoint{2.614596in}{1.891761in}}{\pgfqpoint{2.603997in}{1.887371in}}{\pgfqpoint{2.596183in}{1.879557in}}%
\pgfpathcurveto{\pgfqpoint{2.588369in}{1.871744in}}{\pgfqpoint{2.583979in}{1.861145in}}{\pgfqpoint{2.583979in}{1.850094in}}%
\pgfpathcurveto{\pgfqpoint{2.583979in}{1.839044in}}{\pgfqpoint{2.588369in}{1.828445in}}{\pgfqpoint{2.596183in}{1.820632in}}%
\pgfpathcurveto{\pgfqpoint{2.603997in}{1.812818in}}{\pgfqpoint{2.614596in}{1.808428in}}{\pgfqpoint{2.625646in}{1.808428in}}%
\pgfpathclose%
\pgfusepath{stroke,fill}%
\end{pgfscope}%
\begin{pgfscope}%
\pgfpathrectangle{\pgfqpoint{0.600000in}{0.600000in}}{\pgfqpoint{3.900000in}{3.900000in}}%
\pgfusepath{clip}%
\pgfsetbuttcap%
\pgfsetroundjoin%
\definecolor{currentfill}{rgb}{0.121569,0.466667,0.705882}%
\pgfsetfillcolor{currentfill}%
\pgfsetlinewidth{1.003750pt}%
\definecolor{currentstroke}{rgb}{0.121569,0.466667,0.705882}%
\pgfsetstrokecolor{currentstroke}%
\pgfsetdash{}{0pt}%
\pgfpathmoveto{\pgfqpoint{2.162688in}{2.894632in}}%
\pgfpathcurveto{\pgfqpoint{2.173739in}{2.894632in}}{\pgfqpoint{2.184338in}{2.899022in}}{\pgfqpoint{2.192151in}{2.906836in}}%
\pgfpathcurveto{\pgfqpoint{2.199965in}{2.914650in}}{\pgfqpoint{2.204355in}{2.925249in}}{\pgfqpoint{2.204355in}{2.936299in}}%
\pgfpathcurveto{\pgfqpoint{2.204355in}{2.947349in}}{\pgfqpoint{2.199965in}{2.957948in}}{\pgfqpoint{2.192151in}{2.965762in}}%
\pgfpathcurveto{\pgfqpoint{2.184338in}{2.973575in}}{\pgfqpoint{2.173739in}{2.977966in}}{\pgfqpoint{2.162688in}{2.977966in}}%
\pgfpathcurveto{\pgfqpoint{2.151638in}{2.977966in}}{\pgfqpoint{2.141039in}{2.973575in}}{\pgfqpoint{2.133226in}{2.965762in}}%
\pgfpathcurveto{\pgfqpoint{2.125412in}{2.957948in}}{\pgfqpoint{2.121022in}{2.947349in}}{\pgfqpoint{2.121022in}{2.936299in}}%
\pgfpathcurveto{\pgfqpoint{2.121022in}{2.925249in}}{\pgfqpoint{2.125412in}{2.914650in}}{\pgfqpoint{2.133226in}{2.906836in}}%
\pgfpathcurveto{\pgfqpoint{2.141039in}{2.899022in}}{\pgfqpoint{2.151638in}{2.894632in}}{\pgfqpoint{2.162688in}{2.894632in}}%
\pgfpathclose%
\pgfusepath{stroke,fill}%
\end{pgfscope}%
\begin{pgfscope}%
\pgfpathrectangle{\pgfqpoint{0.600000in}{0.600000in}}{\pgfqpoint{3.900000in}{3.900000in}}%
\pgfusepath{clip}%
\pgfsetbuttcap%
\pgfsetroundjoin%
\definecolor{currentfill}{rgb}{0.121569,0.466667,0.705882}%
\pgfsetfillcolor{currentfill}%
\pgfsetlinewidth{1.003750pt}%
\definecolor{currentstroke}{rgb}{0.121569,0.466667,0.705882}%
\pgfsetstrokecolor{currentstroke}%
\pgfsetdash{}{0pt}%
\pgfpathmoveto{\pgfqpoint{2.683798in}{2.323301in}}%
\pgfpathcurveto{\pgfqpoint{2.694848in}{2.323301in}}{\pgfqpoint{2.705447in}{2.327691in}}{\pgfqpoint{2.713261in}{2.335505in}}%
\pgfpathcurveto{\pgfqpoint{2.721075in}{2.343318in}}{\pgfqpoint{2.725465in}{2.353917in}}{\pgfqpoint{2.725465in}{2.364967in}}%
\pgfpathcurveto{\pgfqpoint{2.725465in}{2.376018in}}{\pgfqpoint{2.721075in}{2.386617in}}{\pgfqpoint{2.713261in}{2.394430in}}%
\pgfpathcurveto{\pgfqpoint{2.705447in}{2.402244in}}{\pgfqpoint{2.694848in}{2.406634in}}{\pgfqpoint{2.683798in}{2.406634in}}%
\pgfpathcurveto{\pgfqpoint{2.672748in}{2.406634in}}{\pgfqpoint{2.662149in}{2.402244in}}{\pgfqpoint{2.654335in}{2.394430in}}%
\pgfpathcurveto{\pgfqpoint{2.646522in}{2.386617in}}{\pgfqpoint{2.642132in}{2.376018in}}{\pgfqpoint{2.642132in}{2.364967in}}%
\pgfpathcurveto{\pgfqpoint{2.642132in}{2.353917in}}{\pgfqpoint{2.646522in}{2.343318in}}{\pgfqpoint{2.654335in}{2.335505in}}%
\pgfpathcurveto{\pgfqpoint{2.662149in}{2.327691in}}{\pgfqpoint{2.672748in}{2.323301in}}{\pgfqpoint{2.683798in}{2.323301in}}%
\pgfpathclose%
\pgfusepath{stroke,fill}%
\end{pgfscope}%
\begin{pgfscope}%
\pgfpathrectangle{\pgfqpoint{0.600000in}{0.600000in}}{\pgfqpoint{3.900000in}{3.900000in}}%
\pgfusepath{clip}%
\pgfsetbuttcap%
\pgfsetroundjoin%
\definecolor{currentfill}{rgb}{0.121569,0.466667,0.705882}%
\pgfsetfillcolor{currentfill}%
\pgfsetlinewidth{1.003750pt}%
\definecolor{currentstroke}{rgb}{0.121569,0.466667,0.705882}%
\pgfsetstrokecolor{currentstroke}%
\pgfsetdash{}{0pt}%
\pgfpathmoveto{\pgfqpoint{3.068726in}{1.879765in}}%
\pgfpathcurveto{\pgfqpoint{3.079776in}{1.879765in}}{\pgfqpoint{3.090375in}{1.884156in}}{\pgfqpoint{3.098189in}{1.891969in}}%
\pgfpathcurveto{\pgfqpoint{3.106002in}{1.899783in}}{\pgfqpoint{3.110393in}{1.910382in}}{\pgfqpoint{3.110393in}{1.921432in}}%
\pgfpathcurveto{\pgfqpoint{3.110393in}{1.932482in}}{\pgfqpoint{3.106002in}{1.943081in}}{\pgfqpoint{3.098189in}{1.950895in}}%
\pgfpathcurveto{\pgfqpoint{3.090375in}{1.958708in}}{\pgfqpoint{3.079776in}{1.963099in}}{\pgfqpoint{3.068726in}{1.963099in}}%
\pgfpathcurveto{\pgfqpoint{3.057676in}{1.963099in}}{\pgfqpoint{3.047077in}{1.958708in}}{\pgfqpoint{3.039263in}{1.950895in}}%
\pgfpathcurveto{\pgfqpoint{3.031450in}{1.943081in}}{\pgfqpoint{3.027059in}{1.932482in}}{\pgfqpoint{3.027059in}{1.921432in}}%
\pgfpathcurveto{\pgfqpoint{3.027059in}{1.910382in}}{\pgfqpoint{3.031450in}{1.899783in}}{\pgfqpoint{3.039263in}{1.891969in}}%
\pgfpathcurveto{\pgfqpoint{3.047077in}{1.884156in}}{\pgfqpoint{3.057676in}{1.879765in}}{\pgfqpoint{3.068726in}{1.879765in}}%
\pgfpathclose%
\pgfusepath{stroke,fill}%
\end{pgfscope}%
\begin{pgfscope}%
\pgfpathrectangle{\pgfqpoint{0.600000in}{0.600000in}}{\pgfqpoint{3.900000in}{3.900000in}}%
\pgfusepath{clip}%
\pgfsetbuttcap%
\pgfsetroundjoin%
\definecolor{currentfill}{rgb}{0.121569,0.466667,0.705882}%
\pgfsetfillcolor{currentfill}%
\pgfsetlinewidth{1.003750pt}%
\definecolor{currentstroke}{rgb}{0.121569,0.466667,0.705882}%
\pgfsetstrokecolor{currentstroke}%
\pgfsetdash{}{0pt}%
\pgfpathmoveto{\pgfqpoint{2.559099in}{2.385847in}}%
\pgfpathcurveto{\pgfqpoint{2.570149in}{2.385847in}}{\pgfqpoint{2.580748in}{2.390237in}}{\pgfqpoint{2.588562in}{2.398051in}}%
\pgfpathcurveto{\pgfqpoint{2.596376in}{2.405865in}}{\pgfqpoint{2.600766in}{2.416464in}}{\pgfqpoint{2.600766in}{2.427514in}}%
\pgfpathcurveto{\pgfqpoint{2.600766in}{2.438564in}}{\pgfqpoint{2.596376in}{2.449163in}}{\pgfqpoint{2.588562in}{2.456977in}}%
\pgfpathcurveto{\pgfqpoint{2.580748in}{2.464790in}}{\pgfqpoint{2.570149in}{2.469181in}}{\pgfqpoint{2.559099in}{2.469181in}}%
\pgfpathcurveto{\pgfqpoint{2.548049in}{2.469181in}}{\pgfqpoint{2.537450in}{2.464790in}}{\pgfqpoint{2.529636in}{2.456977in}}%
\pgfpathcurveto{\pgfqpoint{2.521823in}{2.449163in}}{\pgfqpoint{2.517433in}{2.438564in}}{\pgfqpoint{2.517433in}{2.427514in}}%
\pgfpathcurveto{\pgfqpoint{2.517433in}{2.416464in}}{\pgfqpoint{2.521823in}{2.405865in}}{\pgfqpoint{2.529636in}{2.398051in}}%
\pgfpathcurveto{\pgfqpoint{2.537450in}{2.390237in}}{\pgfqpoint{2.548049in}{2.385847in}}{\pgfqpoint{2.559099in}{2.385847in}}%
\pgfpathclose%
\pgfusepath{stroke,fill}%
\end{pgfscope}%
\begin{pgfscope}%
\pgfpathrectangle{\pgfqpoint{0.600000in}{0.600000in}}{\pgfqpoint{3.900000in}{3.900000in}}%
\pgfusepath{clip}%
\pgfsetbuttcap%
\pgfsetroundjoin%
\definecolor{currentfill}{rgb}{0.121569,0.466667,0.705882}%
\pgfsetfillcolor{currentfill}%
\pgfsetlinewidth{1.003750pt}%
\definecolor{currentstroke}{rgb}{0.121569,0.466667,0.705882}%
\pgfsetstrokecolor{currentstroke}%
\pgfsetdash{}{0pt}%
\pgfpathmoveto{\pgfqpoint{2.957457in}{3.399581in}}%
\pgfpathcurveto{\pgfqpoint{2.968507in}{3.399581in}}{\pgfqpoint{2.979106in}{3.403971in}}{\pgfqpoint{2.986919in}{3.411785in}}%
\pgfpathcurveto{\pgfqpoint{2.994733in}{3.419599in}}{\pgfqpoint{2.999123in}{3.430198in}}{\pgfqpoint{2.999123in}{3.441248in}}%
\pgfpathcurveto{\pgfqpoint{2.999123in}{3.452298in}}{\pgfqpoint{2.994733in}{3.462897in}}{\pgfqpoint{2.986919in}{3.470711in}}%
\pgfpathcurveto{\pgfqpoint{2.979106in}{3.478524in}}{\pgfqpoint{2.968507in}{3.482914in}}{\pgfqpoint{2.957457in}{3.482914in}}%
\pgfpathcurveto{\pgfqpoint{2.946406in}{3.482914in}}{\pgfqpoint{2.935807in}{3.478524in}}{\pgfqpoint{2.927994in}{3.470711in}}%
\pgfpathcurveto{\pgfqpoint{2.920180in}{3.462897in}}{\pgfqpoint{2.915790in}{3.452298in}}{\pgfqpoint{2.915790in}{3.441248in}}%
\pgfpathcurveto{\pgfqpoint{2.915790in}{3.430198in}}{\pgfqpoint{2.920180in}{3.419599in}}{\pgfqpoint{2.927994in}{3.411785in}}%
\pgfpathcurveto{\pgfqpoint{2.935807in}{3.403971in}}{\pgfqpoint{2.946406in}{3.399581in}}{\pgfqpoint{2.957457in}{3.399581in}}%
\pgfpathclose%
\pgfusepath{stroke,fill}%
\end{pgfscope}%
\begin{pgfscope}%
\pgfpathrectangle{\pgfqpoint{0.600000in}{0.600000in}}{\pgfqpoint{3.900000in}{3.900000in}}%
\pgfusepath{clip}%
\pgfsetbuttcap%
\pgfsetroundjoin%
\definecolor{currentfill}{rgb}{0.121569,0.466667,0.705882}%
\pgfsetfillcolor{currentfill}%
\pgfsetlinewidth{1.003750pt}%
\definecolor{currentstroke}{rgb}{0.121569,0.466667,0.705882}%
\pgfsetstrokecolor{currentstroke}%
\pgfsetdash{}{0pt}%
\pgfpathmoveto{\pgfqpoint{2.328793in}{3.052622in}}%
\pgfpathcurveto{\pgfqpoint{2.339843in}{3.052622in}}{\pgfqpoint{2.350442in}{3.057012in}}{\pgfqpoint{2.358256in}{3.064826in}}%
\pgfpathcurveto{\pgfqpoint{2.366070in}{3.072639in}}{\pgfqpoint{2.370460in}{3.083238in}}{\pgfqpoint{2.370460in}{3.094289in}}%
\pgfpathcurveto{\pgfqpoint{2.370460in}{3.105339in}}{\pgfqpoint{2.366070in}{3.115938in}}{\pgfqpoint{2.358256in}{3.123751in}}%
\pgfpathcurveto{\pgfqpoint{2.350442in}{3.131565in}}{\pgfqpoint{2.339843in}{3.135955in}}{\pgfqpoint{2.328793in}{3.135955in}}%
\pgfpathcurveto{\pgfqpoint{2.317743in}{3.135955in}}{\pgfqpoint{2.307144in}{3.131565in}}{\pgfqpoint{2.299330in}{3.123751in}}%
\pgfpathcurveto{\pgfqpoint{2.291517in}{3.115938in}}{\pgfqpoint{2.287127in}{3.105339in}}{\pgfqpoint{2.287127in}{3.094289in}}%
\pgfpathcurveto{\pgfqpoint{2.287127in}{3.083238in}}{\pgfqpoint{2.291517in}{3.072639in}}{\pgfqpoint{2.299330in}{3.064826in}}%
\pgfpathcurveto{\pgfqpoint{2.307144in}{3.057012in}}{\pgfqpoint{2.317743in}{3.052622in}}{\pgfqpoint{2.328793in}{3.052622in}}%
\pgfpathclose%
\pgfusepath{stroke,fill}%
\end{pgfscope}%
\begin{pgfscope}%
\pgfpathrectangle{\pgfqpoint{0.600000in}{0.600000in}}{\pgfqpoint{3.900000in}{3.900000in}}%
\pgfusepath{clip}%
\pgfsetbuttcap%
\pgfsetroundjoin%
\definecolor{currentfill}{rgb}{0.121569,0.466667,0.705882}%
\pgfsetfillcolor{currentfill}%
\pgfsetlinewidth{1.003750pt}%
\definecolor{currentstroke}{rgb}{0.121569,0.466667,0.705882}%
\pgfsetstrokecolor{currentstroke}%
\pgfsetdash{}{0pt}%
\pgfpathmoveto{\pgfqpoint{2.531053in}{2.416709in}}%
\pgfpathcurveto{\pgfqpoint{2.542103in}{2.416709in}}{\pgfqpoint{2.552702in}{2.421099in}}{\pgfqpoint{2.560516in}{2.428913in}}%
\pgfpathcurveto{\pgfqpoint{2.568329in}{2.436726in}}{\pgfqpoint{2.572720in}{2.447326in}}{\pgfqpoint{2.572720in}{2.458376in}}%
\pgfpathcurveto{\pgfqpoint{2.572720in}{2.469426in}}{\pgfqpoint{2.568329in}{2.480025in}}{\pgfqpoint{2.560516in}{2.487838in}}%
\pgfpathcurveto{\pgfqpoint{2.552702in}{2.495652in}}{\pgfqpoint{2.542103in}{2.500042in}}{\pgfqpoint{2.531053in}{2.500042in}}%
\pgfpathcurveto{\pgfqpoint{2.520003in}{2.500042in}}{\pgfqpoint{2.509404in}{2.495652in}}{\pgfqpoint{2.501590in}{2.487838in}}%
\pgfpathcurveto{\pgfqpoint{2.493777in}{2.480025in}}{\pgfqpoint{2.489386in}{2.469426in}}{\pgfqpoint{2.489386in}{2.458376in}}%
\pgfpathcurveto{\pgfqpoint{2.489386in}{2.447326in}}{\pgfqpoint{2.493777in}{2.436726in}}{\pgfqpoint{2.501590in}{2.428913in}}%
\pgfpathcurveto{\pgfqpoint{2.509404in}{2.421099in}}{\pgfqpoint{2.520003in}{2.416709in}}{\pgfqpoint{2.531053in}{2.416709in}}%
\pgfpathclose%
\pgfusepath{stroke,fill}%
\end{pgfscope}%
\begin{pgfscope}%
\pgfpathrectangle{\pgfqpoint{0.600000in}{0.600000in}}{\pgfqpoint{3.900000in}{3.900000in}}%
\pgfusepath{clip}%
\pgfsetbuttcap%
\pgfsetroundjoin%
\definecolor{currentfill}{rgb}{0.121569,0.466667,0.705882}%
\pgfsetfillcolor{currentfill}%
\pgfsetlinewidth{1.003750pt}%
\definecolor{currentstroke}{rgb}{0.121569,0.466667,0.705882}%
\pgfsetstrokecolor{currentstroke}%
\pgfsetdash{}{0pt}%
\pgfpathmoveto{\pgfqpoint{2.080976in}{2.765652in}}%
\pgfpathcurveto{\pgfqpoint{2.092026in}{2.765652in}}{\pgfqpoint{2.102625in}{2.770043in}}{\pgfqpoint{2.110438in}{2.777856in}}%
\pgfpathcurveto{\pgfqpoint{2.118252in}{2.785670in}}{\pgfqpoint{2.122642in}{2.796269in}}{\pgfqpoint{2.122642in}{2.807319in}}%
\pgfpathcurveto{\pgfqpoint{2.122642in}{2.818369in}}{\pgfqpoint{2.118252in}{2.828968in}}{\pgfqpoint{2.110438in}{2.836782in}}%
\pgfpathcurveto{\pgfqpoint{2.102625in}{2.844595in}}{\pgfqpoint{2.092026in}{2.848986in}}{\pgfqpoint{2.080976in}{2.848986in}}%
\pgfpathcurveto{\pgfqpoint{2.069925in}{2.848986in}}{\pgfqpoint{2.059326in}{2.844595in}}{\pgfqpoint{2.051513in}{2.836782in}}%
\pgfpathcurveto{\pgfqpoint{2.043699in}{2.828968in}}{\pgfqpoint{2.039309in}{2.818369in}}{\pgfqpoint{2.039309in}{2.807319in}}%
\pgfpathcurveto{\pgfqpoint{2.039309in}{2.796269in}}{\pgfqpoint{2.043699in}{2.785670in}}{\pgfqpoint{2.051513in}{2.777856in}}%
\pgfpathcurveto{\pgfqpoint{2.059326in}{2.770043in}}{\pgfqpoint{2.069925in}{2.765652in}}{\pgfqpoint{2.080976in}{2.765652in}}%
\pgfpathclose%
\pgfusepath{stroke,fill}%
\end{pgfscope}%
\begin{pgfscope}%
\pgfpathrectangle{\pgfqpoint{0.600000in}{0.600000in}}{\pgfqpoint{3.900000in}{3.900000in}}%
\pgfusepath{clip}%
\pgfsetbuttcap%
\pgfsetroundjoin%
\definecolor{currentfill}{rgb}{0.121569,0.466667,0.705882}%
\pgfsetfillcolor{currentfill}%
\pgfsetlinewidth{1.003750pt}%
\definecolor{currentstroke}{rgb}{0.121569,0.466667,0.705882}%
\pgfsetstrokecolor{currentstroke}%
\pgfsetdash{}{0pt}%
\pgfpathmoveto{\pgfqpoint{2.735235in}{3.161692in}}%
\pgfpathcurveto{\pgfqpoint{2.746285in}{3.161692in}}{\pgfqpoint{2.756884in}{3.166082in}}{\pgfqpoint{2.764698in}{3.173895in}}%
\pgfpathcurveto{\pgfqpoint{2.772511in}{3.181709in}}{\pgfqpoint{2.776902in}{3.192308in}}{\pgfqpoint{2.776902in}{3.203358in}}%
\pgfpathcurveto{\pgfqpoint{2.776902in}{3.214408in}}{\pgfqpoint{2.772511in}{3.225007in}}{\pgfqpoint{2.764698in}{3.232821in}}%
\pgfpathcurveto{\pgfqpoint{2.756884in}{3.240635in}}{\pgfqpoint{2.746285in}{3.245025in}}{\pgfqpoint{2.735235in}{3.245025in}}%
\pgfpathcurveto{\pgfqpoint{2.724185in}{3.245025in}}{\pgfqpoint{2.713586in}{3.240635in}}{\pgfqpoint{2.705772in}{3.232821in}}%
\pgfpathcurveto{\pgfqpoint{2.697959in}{3.225007in}}{\pgfqpoint{2.693568in}{3.214408in}}{\pgfqpoint{2.693568in}{3.203358in}}%
\pgfpathcurveto{\pgfqpoint{2.693568in}{3.192308in}}{\pgfqpoint{2.697959in}{3.181709in}}{\pgfqpoint{2.705772in}{3.173895in}}%
\pgfpathcurveto{\pgfqpoint{2.713586in}{3.166082in}}{\pgfqpoint{2.724185in}{3.161692in}}{\pgfqpoint{2.735235in}{3.161692in}}%
\pgfpathclose%
\pgfusepath{stroke,fill}%
\end{pgfscope}%
\begin{pgfscope}%
\pgfpathrectangle{\pgfqpoint{0.600000in}{0.600000in}}{\pgfqpoint{3.900000in}{3.900000in}}%
\pgfusepath{clip}%
\pgfsetbuttcap%
\pgfsetroundjoin%
\definecolor{currentfill}{rgb}{0.121569,0.466667,0.705882}%
\pgfsetfillcolor{currentfill}%
\pgfsetlinewidth{1.003750pt}%
\definecolor{currentstroke}{rgb}{0.121569,0.466667,0.705882}%
\pgfsetstrokecolor{currentstroke}%
\pgfsetdash{}{0pt}%
\pgfpathmoveto{\pgfqpoint{2.664843in}{2.403715in}}%
\pgfpathcurveto{\pgfqpoint{2.675893in}{2.403715in}}{\pgfqpoint{2.686492in}{2.408105in}}{\pgfqpoint{2.694306in}{2.415919in}}%
\pgfpathcurveto{\pgfqpoint{2.702120in}{2.423732in}}{\pgfqpoint{2.706510in}{2.434331in}}{\pgfqpoint{2.706510in}{2.445382in}}%
\pgfpathcurveto{\pgfqpoint{2.706510in}{2.456432in}}{\pgfqpoint{2.702120in}{2.467031in}}{\pgfqpoint{2.694306in}{2.474844in}}%
\pgfpathcurveto{\pgfqpoint{2.686492in}{2.482658in}}{\pgfqpoint{2.675893in}{2.487048in}}{\pgfqpoint{2.664843in}{2.487048in}}%
\pgfpathcurveto{\pgfqpoint{2.653793in}{2.487048in}}{\pgfqpoint{2.643194in}{2.482658in}}{\pgfqpoint{2.635380in}{2.474844in}}%
\pgfpathcurveto{\pgfqpoint{2.627567in}{2.467031in}}{\pgfqpoint{2.623177in}{2.456432in}}{\pgfqpoint{2.623177in}{2.445382in}}%
\pgfpathcurveto{\pgfqpoint{2.623177in}{2.434331in}}{\pgfqpoint{2.627567in}{2.423732in}}{\pgfqpoint{2.635380in}{2.415919in}}%
\pgfpathcurveto{\pgfqpoint{2.643194in}{2.408105in}}{\pgfqpoint{2.653793in}{2.403715in}}{\pgfqpoint{2.664843in}{2.403715in}}%
\pgfpathclose%
\pgfusepath{stroke,fill}%
\end{pgfscope}%
\begin{pgfscope}%
\pgfpathrectangle{\pgfqpoint{0.600000in}{0.600000in}}{\pgfqpoint{3.900000in}{3.900000in}}%
\pgfusepath{clip}%
\pgfsetbuttcap%
\pgfsetroundjoin%
\definecolor{currentfill}{rgb}{0.121569,0.466667,0.705882}%
\pgfsetfillcolor{currentfill}%
\pgfsetlinewidth{1.003750pt}%
\definecolor{currentstroke}{rgb}{0.121569,0.466667,0.705882}%
\pgfsetstrokecolor{currentstroke}%
\pgfsetdash{}{0pt}%
\pgfpathmoveto{\pgfqpoint{3.979863in}{3.300959in}}%
\pgfpathcurveto{\pgfqpoint{3.990913in}{3.300959in}}{\pgfqpoint{4.001512in}{3.305349in}}{\pgfqpoint{4.009325in}{3.313163in}}%
\pgfpathcurveto{\pgfqpoint{4.017139in}{3.320977in}}{\pgfqpoint{4.021529in}{3.331576in}}{\pgfqpoint{4.021529in}{3.342626in}}%
\pgfpathcurveto{\pgfqpoint{4.021529in}{3.353676in}}{\pgfqpoint{4.017139in}{3.364275in}}{\pgfqpoint{4.009325in}{3.372088in}}%
\pgfpathcurveto{\pgfqpoint{4.001512in}{3.379902in}}{\pgfqpoint{3.990913in}{3.384292in}}{\pgfqpoint{3.979863in}{3.384292in}}%
\pgfpathcurveto{\pgfqpoint{3.968812in}{3.384292in}}{\pgfqpoint{3.958213in}{3.379902in}}{\pgfqpoint{3.950400in}{3.372088in}}%
\pgfpathcurveto{\pgfqpoint{3.942586in}{3.364275in}}{\pgfqpoint{3.938196in}{3.353676in}}{\pgfqpoint{3.938196in}{3.342626in}}%
\pgfpathcurveto{\pgfqpoint{3.938196in}{3.331576in}}{\pgfqpoint{3.942586in}{3.320977in}}{\pgfqpoint{3.950400in}{3.313163in}}%
\pgfpathcurveto{\pgfqpoint{3.958213in}{3.305349in}}{\pgfqpoint{3.968812in}{3.300959in}}{\pgfqpoint{3.979863in}{3.300959in}}%
\pgfpathclose%
\pgfusepath{stroke,fill}%
\end{pgfscope}%
\begin{pgfscope}%
\pgfpathrectangle{\pgfqpoint{0.600000in}{0.600000in}}{\pgfqpoint{3.900000in}{3.900000in}}%
\pgfusepath{clip}%
\pgfsetbuttcap%
\pgfsetroundjoin%
\definecolor{currentfill}{rgb}{0.121569,0.466667,0.705882}%
\pgfsetfillcolor{currentfill}%
\pgfsetlinewidth{1.003750pt}%
\definecolor{currentstroke}{rgb}{0.121569,0.466667,0.705882}%
\pgfsetstrokecolor{currentstroke}%
\pgfsetdash{}{0pt}%
\pgfpathmoveto{\pgfqpoint{3.100552in}{1.923073in}}%
\pgfpathcurveto{\pgfqpoint{3.111602in}{1.923073in}}{\pgfqpoint{3.122201in}{1.927463in}}{\pgfqpoint{3.130015in}{1.935276in}}%
\pgfpathcurveto{\pgfqpoint{3.137829in}{1.943090in}}{\pgfqpoint{3.142219in}{1.953689in}}{\pgfqpoint{3.142219in}{1.964739in}}%
\pgfpathcurveto{\pgfqpoint{3.142219in}{1.975789in}}{\pgfqpoint{3.137829in}{1.986388in}}{\pgfqpoint{3.130015in}{1.994202in}}%
\pgfpathcurveto{\pgfqpoint{3.122201in}{2.002016in}}{\pgfqpoint{3.111602in}{2.006406in}}{\pgfqpoint{3.100552in}{2.006406in}}%
\pgfpathcurveto{\pgfqpoint{3.089502in}{2.006406in}}{\pgfqpoint{3.078903in}{2.002016in}}{\pgfqpoint{3.071089in}{1.994202in}}%
\pgfpathcurveto{\pgfqpoint{3.063276in}{1.986388in}}{\pgfqpoint{3.058886in}{1.975789in}}{\pgfqpoint{3.058886in}{1.964739in}}%
\pgfpathcurveto{\pgfqpoint{3.058886in}{1.953689in}}{\pgfqpoint{3.063276in}{1.943090in}}{\pgfqpoint{3.071089in}{1.935276in}}%
\pgfpathcurveto{\pgfqpoint{3.078903in}{1.927463in}}{\pgfqpoint{3.089502in}{1.923073in}}{\pgfqpoint{3.100552in}{1.923073in}}%
\pgfpathclose%
\pgfusepath{stroke,fill}%
\end{pgfscope}%
\begin{pgfscope}%
\pgfpathrectangle{\pgfqpoint{0.600000in}{0.600000in}}{\pgfqpoint{3.900000in}{3.900000in}}%
\pgfusepath{clip}%
\pgfsetbuttcap%
\pgfsetroundjoin%
\definecolor{currentfill}{rgb}{0.121569,0.466667,0.705882}%
\pgfsetfillcolor{currentfill}%
\pgfsetlinewidth{1.003750pt}%
\definecolor{currentstroke}{rgb}{0.121569,0.466667,0.705882}%
\pgfsetstrokecolor{currentstroke}%
\pgfsetdash{}{0pt}%
\pgfpathmoveto{\pgfqpoint{2.478830in}{2.419561in}}%
\pgfpathcurveto{\pgfqpoint{2.489880in}{2.419561in}}{\pgfqpoint{2.500479in}{2.423951in}}{\pgfqpoint{2.508292in}{2.431765in}}%
\pgfpathcurveto{\pgfqpoint{2.516106in}{2.439579in}}{\pgfqpoint{2.520496in}{2.450178in}}{\pgfqpoint{2.520496in}{2.461228in}}%
\pgfpathcurveto{\pgfqpoint{2.520496in}{2.472278in}}{\pgfqpoint{2.516106in}{2.482877in}}{\pgfqpoint{2.508292in}{2.490690in}}%
\pgfpathcurveto{\pgfqpoint{2.500479in}{2.498504in}}{\pgfqpoint{2.489880in}{2.502894in}}{\pgfqpoint{2.478830in}{2.502894in}}%
\pgfpathcurveto{\pgfqpoint{2.467779in}{2.502894in}}{\pgfqpoint{2.457180in}{2.498504in}}{\pgfqpoint{2.449367in}{2.490690in}}%
\pgfpathcurveto{\pgfqpoint{2.441553in}{2.482877in}}{\pgfqpoint{2.437163in}{2.472278in}}{\pgfqpoint{2.437163in}{2.461228in}}%
\pgfpathcurveto{\pgfqpoint{2.437163in}{2.450178in}}{\pgfqpoint{2.441553in}{2.439579in}}{\pgfqpoint{2.449367in}{2.431765in}}%
\pgfpathcurveto{\pgfqpoint{2.457180in}{2.423951in}}{\pgfqpoint{2.467779in}{2.419561in}}{\pgfqpoint{2.478830in}{2.419561in}}%
\pgfpathclose%
\pgfusepath{stroke,fill}%
\end{pgfscope}%
\begin{pgfscope}%
\pgfpathrectangle{\pgfqpoint{0.600000in}{0.600000in}}{\pgfqpoint{3.900000in}{3.900000in}}%
\pgfusepath{clip}%
\pgfsetbuttcap%
\pgfsetroundjoin%
\definecolor{currentfill}{rgb}{0.121569,0.466667,0.705882}%
\pgfsetfillcolor{currentfill}%
\pgfsetlinewidth{1.003750pt}%
\definecolor{currentstroke}{rgb}{0.121569,0.466667,0.705882}%
\pgfsetstrokecolor{currentstroke}%
\pgfsetdash{}{0pt}%
\pgfpathmoveto{\pgfqpoint{3.009702in}{2.013791in}}%
\pgfpathcurveto{\pgfqpoint{3.020753in}{2.013791in}}{\pgfqpoint{3.031352in}{2.018181in}}{\pgfqpoint{3.039165in}{2.025994in}}%
\pgfpathcurveto{\pgfqpoint{3.046979in}{2.033808in}}{\pgfqpoint{3.051369in}{2.044407in}}{\pgfqpoint{3.051369in}{2.055457in}}%
\pgfpathcurveto{\pgfqpoint{3.051369in}{2.066507in}}{\pgfqpoint{3.046979in}{2.077106in}}{\pgfqpoint{3.039165in}{2.084920in}}%
\pgfpathcurveto{\pgfqpoint{3.031352in}{2.092734in}}{\pgfqpoint{3.020753in}{2.097124in}}{\pgfqpoint{3.009702in}{2.097124in}}%
\pgfpathcurveto{\pgfqpoint{2.998652in}{2.097124in}}{\pgfqpoint{2.988053in}{2.092734in}}{\pgfqpoint{2.980240in}{2.084920in}}%
\pgfpathcurveto{\pgfqpoint{2.972426in}{2.077106in}}{\pgfqpoint{2.968036in}{2.066507in}}{\pgfqpoint{2.968036in}{2.055457in}}%
\pgfpathcurveto{\pgfqpoint{2.968036in}{2.044407in}}{\pgfqpoint{2.972426in}{2.033808in}}{\pgfqpoint{2.980240in}{2.025994in}}%
\pgfpathcurveto{\pgfqpoint{2.988053in}{2.018181in}}{\pgfqpoint{2.998652in}{2.013791in}}{\pgfqpoint{3.009702in}{2.013791in}}%
\pgfpathclose%
\pgfusepath{stroke,fill}%
\end{pgfscope}%
\begin{pgfscope}%
\pgfpathrectangle{\pgfqpoint{0.600000in}{0.600000in}}{\pgfqpoint{3.900000in}{3.900000in}}%
\pgfusepath{clip}%
\pgfsetbuttcap%
\pgfsetroundjoin%
\definecolor{currentfill}{rgb}{0.121569,0.466667,0.705882}%
\pgfsetfillcolor{currentfill}%
\pgfsetlinewidth{1.003750pt}%
\definecolor{currentstroke}{rgb}{0.121569,0.466667,0.705882}%
\pgfsetstrokecolor{currentstroke}%
\pgfsetdash{}{0pt}%
\pgfpathmoveto{\pgfqpoint{3.330068in}{2.483688in}}%
\pgfpathcurveto{\pgfqpoint{3.341118in}{2.483688in}}{\pgfqpoint{3.351717in}{2.488078in}}{\pgfqpoint{3.359531in}{2.495891in}}%
\pgfpathcurveto{\pgfqpoint{3.367344in}{2.503705in}}{\pgfqpoint{3.371735in}{2.514304in}}{\pgfqpoint{3.371735in}{2.525354in}}%
\pgfpathcurveto{\pgfqpoint{3.371735in}{2.536404in}}{\pgfqpoint{3.367344in}{2.547003in}}{\pgfqpoint{3.359531in}{2.554817in}}%
\pgfpathcurveto{\pgfqpoint{3.351717in}{2.562631in}}{\pgfqpoint{3.341118in}{2.567021in}}{\pgfqpoint{3.330068in}{2.567021in}}%
\pgfpathcurveto{\pgfqpoint{3.319018in}{2.567021in}}{\pgfqpoint{3.308419in}{2.562631in}}{\pgfqpoint{3.300605in}{2.554817in}}%
\pgfpathcurveto{\pgfqpoint{3.292791in}{2.547003in}}{\pgfqpoint{3.288401in}{2.536404in}}{\pgfqpoint{3.288401in}{2.525354in}}%
\pgfpathcurveto{\pgfqpoint{3.288401in}{2.514304in}}{\pgfqpoint{3.292791in}{2.503705in}}{\pgfqpoint{3.300605in}{2.495891in}}%
\pgfpathcurveto{\pgfqpoint{3.308419in}{2.488078in}}{\pgfqpoint{3.319018in}{2.483688in}}{\pgfqpoint{3.330068in}{2.483688in}}%
\pgfpathclose%
\pgfusepath{stroke,fill}%
\end{pgfscope}%
\begin{pgfscope}%
\pgfpathrectangle{\pgfqpoint{0.600000in}{0.600000in}}{\pgfqpoint{3.900000in}{3.900000in}}%
\pgfusepath{clip}%
\pgfsetbuttcap%
\pgfsetroundjoin%
\definecolor{currentfill}{rgb}{0.121569,0.466667,0.705882}%
\pgfsetfillcolor{currentfill}%
\pgfsetlinewidth{1.003750pt}%
\definecolor{currentstroke}{rgb}{0.121569,0.466667,0.705882}%
\pgfsetstrokecolor{currentstroke}%
\pgfsetdash{}{0pt}%
\pgfpathmoveto{\pgfqpoint{2.366910in}{3.658974in}}%
\pgfpathcurveto{\pgfqpoint{2.377960in}{3.658974in}}{\pgfqpoint{2.388559in}{3.663364in}}{\pgfqpoint{2.396373in}{3.671178in}}%
\pgfpathcurveto{\pgfqpoint{2.404186in}{3.678992in}}{\pgfqpoint{2.408577in}{3.689591in}}{\pgfqpoint{2.408577in}{3.700641in}}%
\pgfpathcurveto{\pgfqpoint{2.408577in}{3.711691in}}{\pgfqpoint{2.404186in}{3.722290in}}{\pgfqpoint{2.396373in}{3.730104in}}%
\pgfpathcurveto{\pgfqpoint{2.388559in}{3.737917in}}{\pgfqpoint{2.377960in}{3.742307in}}{\pgfqpoint{2.366910in}{3.742307in}}%
\pgfpathcurveto{\pgfqpoint{2.355860in}{3.742307in}}{\pgfqpoint{2.345261in}{3.737917in}}{\pgfqpoint{2.337447in}{3.730104in}}%
\pgfpathcurveto{\pgfqpoint{2.329634in}{3.722290in}}{\pgfqpoint{2.325243in}{3.711691in}}{\pgfqpoint{2.325243in}{3.700641in}}%
\pgfpathcurveto{\pgfqpoint{2.325243in}{3.689591in}}{\pgfqpoint{2.329634in}{3.678992in}}{\pgfqpoint{2.337447in}{3.671178in}}%
\pgfpathcurveto{\pgfqpoint{2.345261in}{3.663364in}}{\pgfqpoint{2.355860in}{3.658974in}}{\pgfqpoint{2.366910in}{3.658974in}}%
\pgfpathclose%
\pgfusepath{stroke,fill}%
\end{pgfscope}%
\begin{pgfscope}%
\pgfpathrectangle{\pgfqpoint{0.600000in}{0.600000in}}{\pgfqpoint{3.900000in}{3.900000in}}%
\pgfusepath{clip}%
\pgfsetbuttcap%
\pgfsetroundjoin%
\definecolor{currentfill}{rgb}{0.121569,0.466667,0.705882}%
\pgfsetfillcolor{currentfill}%
\pgfsetlinewidth{1.003750pt}%
\definecolor{currentstroke}{rgb}{0.121569,0.466667,0.705882}%
\pgfsetstrokecolor{currentstroke}%
\pgfsetdash{}{0pt}%
\pgfpathmoveto{\pgfqpoint{2.854375in}{2.574510in}}%
\pgfpathcurveto{\pgfqpoint{2.865425in}{2.574510in}}{\pgfqpoint{2.876024in}{2.578900in}}{\pgfqpoint{2.883838in}{2.586714in}}%
\pgfpathcurveto{\pgfqpoint{2.891652in}{2.594528in}}{\pgfqpoint{2.896042in}{2.605127in}}{\pgfqpoint{2.896042in}{2.616177in}}%
\pgfpathcurveto{\pgfqpoint{2.896042in}{2.627227in}}{\pgfqpoint{2.891652in}{2.637826in}}{\pgfqpoint{2.883838in}{2.645639in}}%
\pgfpathcurveto{\pgfqpoint{2.876024in}{2.653453in}}{\pgfqpoint{2.865425in}{2.657843in}}{\pgfqpoint{2.854375in}{2.657843in}}%
\pgfpathcurveto{\pgfqpoint{2.843325in}{2.657843in}}{\pgfqpoint{2.832726in}{2.653453in}}{\pgfqpoint{2.824913in}{2.645639in}}%
\pgfpathcurveto{\pgfqpoint{2.817099in}{2.637826in}}{\pgfqpoint{2.812709in}{2.627227in}}{\pgfqpoint{2.812709in}{2.616177in}}%
\pgfpathcurveto{\pgfqpoint{2.812709in}{2.605127in}}{\pgfqpoint{2.817099in}{2.594528in}}{\pgfqpoint{2.824913in}{2.586714in}}%
\pgfpathcurveto{\pgfqpoint{2.832726in}{2.578900in}}{\pgfqpoint{2.843325in}{2.574510in}}{\pgfqpoint{2.854375in}{2.574510in}}%
\pgfpathclose%
\pgfusepath{stroke,fill}%
\end{pgfscope}%
\begin{pgfscope}%
\pgfpathrectangle{\pgfqpoint{0.600000in}{0.600000in}}{\pgfqpoint{3.900000in}{3.900000in}}%
\pgfusepath{clip}%
\pgfsetbuttcap%
\pgfsetroundjoin%
\definecolor{currentfill}{rgb}{0.121569,0.466667,0.705882}%
\pgfsetfillcolor{currentfill}%
\pgfsetlinewidth{1.003750pt}%
\definecolor{currentstroke}{rgb}{0.121569,0.466667,0.705882}%
\pgfsetstrokecolor{currentstroke}%
\pgfsetdash{}{0pt}%
\pgfpathmoveto{\pgfqpoint{2.268124in}{2.760787in}}%
\pgfpathcurveto{\pgfqpoint{2.279174in}{2.760787in}}{\pgfqpoint{2.289773in}{2.765177in}}{\pgfqpoint{2.297586in}{2.772991in}}%
\pgfpathcurveto{\pgfqpoint{2.305400in}{2.780804in}}{\pgfqpoint{2.309790in}{2.791403in}}{\pgfqpoint{2.309790in}{2.802453in}}%
\pgfpathcurveto{\pgfqpoint{2.309790in}{2.813504in}}{\pgfqpoint{2.305400in}{2.824103in}}{\pgfqpoint{2.297586in}{2.831916in}}%
\pgfpathcurveto{\pgfqpoint{2.289773in}{2.839730in}}{\pgfqpoint{2.279174in}{2.844120in}}{\pgfqpoint{2.268124in}{2.844120in}}%
\pgfpathcurveto{\pgfqpoint{2.257074in}{2.844120in}}{\pgfqpoint{2.246475in}{2.839730in}}{\pgfqpoint{2.238661in}{2.831916in}}%
\pgfpathcurveto{\pgfqpoint{2.230847in}{2.824103in}}{\pgfqpoint{2.226457in}{2.813504in}}{\pgfqpoint{2.226457in}{2.802453in}}%
\pgfpathcurveto{\pgfqpoint{2.226457in}{2.791403in}}{\pgfqpoint{2.230847in}{2.780804in}}{\pgfqpoint{2.238661in}{2.772991in}}%
\pgfpathcurveto{\pgfqpoint{2.246475in}{2.765177in}}{\pgfqpoint{2.257074in}{2.760787in}}{\pgfqpoint{2.268124in}{2.760787in}}%
\pgfpathclose%
\pgfusepath{stroke,fill}%
\end{pgfscope}%
\begin{pgfscope}%
\pgfpathrectangle{\pgfqpoint{0.600000in}{0.600000in}}{\pgfqpoint{3.900000in}{3.900000in}}%
\pgfusepath{clip}%
\pgfsetbuttcap%
\pgfsetroundjoin%
\definecolor{currentfill}{rgb}{0.121569,0.466667,0.705882}%
\pgfsetfillcolor{currentfill}%
\pgfsetlinewidth{1.003750pt}%
\definecolor{currentstroke}{rgb}{0.121569,0.466667,0.705882}%
\pgfsetstrokecolor{currentstroke}%
\pgfsetdash{}{0pt}%
\pgfpathmoveto{\pgfqpoint{3.385392in}{2.780691in}}%
\pgfpathcurveto{\pgfqpoint{3.396442in}{2.780691in}}{\pgfqpoint{3.407041in}{2.785081in}}{\pgfqpoint{3.414855in}{2.792895in}}%
\pgfpathcurveto{\pgfqpoint{3.422669in}{2.800708in}}{\pgfqpoint{3.427059in}{2.811307in}}{\pgfqpoint{3.427059in}{2.822357in}}%
\pgfpathcurveto{\pgfqpoint{3.427059in}{2.833407in}}{\pgfqpoint{3.422669in}{2.844006in}}{\pgfqpoint{3.414855in}{2.851820in}}%
\pgfpathcurveto{\pgfqpoint{3.407041in}{2.859634in}}{\pgfqpoint{3.396442in}{2.864024in}}{\pgfqpoint{3.385392in}{2.864024in}}%
\pgfpathcurveto{\pgfqpoint{3.374342in}{2.864024in}}{\pgfqpoint{3.363743in}{2.859634in}}{\pgfqpoint{3.355930in}{2.851820in}}%
\pgfpathcurveto{\pgfqpoint{3.348116in}{2.844006in}}{\pgfqpoint{3.343726in}{2.833407in}}{\pgfqpoint{3.343726in}{2.822357in}}%
\pgfpathcurveto{\pgfqpoint{3.343726in}{2.811307in}}{\pgfqpoint{3.348116in}{2.800708in}}{\pgfqpoint{3.355930in}{2.792895in}}%
\pgfpathcurveto{\pgfqpoint{3.363743in}{2.785081in}}{\pgfqpoint{3.374342in}{2.780691in}}{\pgfqpoint{3.385392in}{2.780691in}}%
\pgfpathclose%
\pgfusepath{stroke,fill}%
\end{pgfscope}%
\begin{pgfscope}%
\pgfpathrectangle{\pgfqpoint{0.600000in}{0.600000in}}{\pgfqpoint{3.900000in}{3.900000in}}%
\pgfusepath{clip}%
\pgfsetbuttcap%
\pgfsetroundjoin%
\definecolor{currentfill}{rgb}{0.121569,0.466667,0.705882}%
\pgfsetfillcolor{currentfill}%
\pgfsetlinewidth{1.003750pt}%
\definecolor{currentstroke}{rgb}{0.121569,0.466667,0.705882}%
\pgfsetstrokecolor{currentstroke}%
\pgfsetdash{}{0pt}%
\pgfpathmoveto{\pgfqpoint{3.304652in}{2.392574in}}%
\pgfpathcurveto{\pgfqpoint{3.315702in}{2.392574in}}{\pgfqpoint{3.326301in}{2.396965in}}{\pgfqpoint{3.334114in}{2.404778in}}%
\pgfpathcurveto{\pgfqpoint{3.341928in}{2.412592in}}{\pgfqpoint{3.346318in}{2.423191in}}{\pgfqpoint{3.346318in}{2.434241in}}%
\pgfpathcurveto{\pgfqpoint{3.346318in}{2.445291in}}{\pgfqpoint{3.341928in}{2.455890in}}{\pgfqpoint{3.334114in}{2.463704in}}%
\pgfpathcurveto{\pgfqpoint{3.326301in}{2.471518in}}{\pgfqpoint{3.315702in}{2.475908in}}{\pgfqpoint{3.304652in}{2.475908in}}%
\pgfpathcurveto{\pgfqpoint{3.293601in}{2.475908in}}{\pgfqpoint{3.283002in}{2.471518in}}{\pgfqpoint{3.275189in}{2.463704in}}%
\pgfpathcurveto{\pgfqpoint{3.267375in}{2.455890in}}{\pgfqpoint{3.262985in}{2.445291in}}{\pgfqpoint{3.262985in}{2.434241in}}%
\pgfpathcurveto{\pgfqpoint{3.262985in}{2.423191in}}{\pgfqpoint{3.267375in}{2.412592in}}{\pgfqpoint{3.275189in}{2.404778in}}%
\pgfpathcurveto{\pgfqpoint{3.283002in}{2.396965in}}{\pgfqpoint{3.293601in}{2.392574in}}{\pgfqpoint{3.304652in}{2.392574in}}%
\pgfpathclose%
\pgfusepath{stroke,fill}%
\end{pgfscope}%
\begin{pgfscope}%
\pgfpathrectangle{\pgfqpoint{0.600000in}{0.600000in}}{\pgfqpoint{3.900000in}{3.900000in}}%
\pgfusepath{clip}%
\pgfsetbuttcap%
\pgfsetroundjoin%
\definecolor{currentfill}{rgb}{0.121569,0.466667,0.705882}%
\pgfsetfillcolor{currentfill}%
\pgfsetlinewidth{1.003750pt}%
\definecolor{currentstroke}{rgb}{0.121569,0.466667,0.705882}%
\pgfsetstrokecolor{currentstroke}%
\pgfsetdash{}{0pt}%
\pgfpathmoveto{\pgfqpoint{2.229218in}{2.283551in}}%
\pgfpathcurveto{\pgfqpoint{2.240268in}{2.283551in}}{\pgfqpoint{2.250867in}{2.287941in}}{\pgfqpoint{2.258681in}{2.295755in}}%
\pgfpathcurveto{\pgfqpoint{2.266495in}{2.303568in}}{\pgfqpoint{2.270885in}{2.314167in}}{\pgfqpoint{2.270885in}{2.325217in}}%
\pgfpathcurveto{\pgfqpoint{2.270885in}{2.336268in}}{\pgfqpoint{2.266495in}{2.346867in}}{\pgfqpoint{2.258681in}{2.354680in}}%
\pgfpathcurveto{\pgfqpoint{2.250867in}{2.362494in}}{\pgfqpoint{2.240268in}{2.366884in}}{\pgfqpoint{2.229218in}{2.366884in}}%
\pgfpathcurveto{\pgfqpoint{2.218168in}{2.366884in}}{\pgfqpoint{2.207569in}{2.362494in}}{\pgfqpoint{2.199755in}{2.354680in}}%
\pgfpathcurveto{\pgfqpoint{2.191942in}{2.346867in}}{\pgfqpoint{2.187552in}{2.336268in}}{\pgfqpoint{2.187552in}{2.325217in}}%
\pgfpathcurveto{\pgfqpoint{2.187552in}{2.314167in}}{\pgfqpoint{2.191942in}{2.303568in}}{\pgfqpoint{2.199755in}{2.295755in}}%
\pgfpathcurveto{\pgfqpoint{2.207569in}{2.287941in}}{\pgfqpoint{2.218168in}{2.283551in}}{\pgfqpoint{2.229218in}{2.283551in}}%
\pgfpathclose%
\pgfusepath{stroke,fill}%
\end{pgfscope}%
\begin{pgfscope}%
\pgfpathrectangle{\pgfqpoint{0.600000in}{0.600000in}}{\pgfqpoint{3.900000in}{3.900000in}}%
\pgfusepath{clip}%
\pgfsetbuttcap%
\pgfsetroundjoin%
\definecolor{currentfill}{rgb}{0.121569,0.466667,0.705882}%
\pgfsetfillcolor{currentfill}%
\pgfsetlinewidth{1.003750pt}%
\definecolor{currentstroke}{rgb}{0.121569,0.466667,0.705882}%
\pgfsetstrokecolor{currentstroke}%
\pgfsetdash{}{0pt}%
\pgfpathmoveto{\pgfqpoint{2.168997in}{2.184260in}}%
\pgfpathcurveto{\pgfqpoint{2.180047in}{2.184260in}}{\pgfqpoint{2.190646in}{2.188651in}}{\pgfqpoint{2.198459in}{2.196464in}}%
\pgfpathcurveto{\pgfqpoint{2.206273in}{2.204278in}}{\pgfqpoint{2.210663in}{2.214877in}}{\pgfqpoint{2.210663in}{2.225927in}}%
\pgfpathcurveto{\pgfqpoint{2.210663in}{2.236977in}}{\pgfqpoint{2.206273in}{2.247576in}}{\pgfqpoint{2.198459in}{2.255390in}}%
\pgfpathcurveto{\pgfqpoint{2.190646in}{2.263203in}}{\pgfqpoint{2.180047in}{2.267594in}}{\pgfqpoint{2.168997in}{2.267594in}}%
\pgfpathcurveto{\pgfqpoint{2.157946in}{2.267594in}}{\pgfqpoint{2.147347in}{2.263203in}}{\pgfqpoint{2.139534in}{2.255390in}}%
\pgfpathcurveto{\pgfqpoint{2.131720in}{2.247576in}}{\pgfqpoint{2.127330in}{2.236977in}}{\pgfqpoint{2.127330in}{2.225927in}}%
\pgfpathcurveto{\pgfqpoint{2.127330in}{2.214877in}}{\pgfqpoint{2.131720in}{2.204278in}}{\pgfqpoint{2.139534in}{2.196464in}}%
\pgfpathcurveto{\pgfqpoint{2.147347in}{2.188651in}}{\pgfqpoint{2.157946in}{2.184260in}}{\pgfqpoint{2.168997in}{2.184260in}}%
\pgfpathclose%
\pgfusepath{stroke,fill}%
\end{pgfscope}%
\begin{pgfscope}%
\pgfpathrectangle{\pgfqpoint{0.600000in}{0.600000in}}{\pgfqpoint{3.900000in}{3.900000in}}%
\pgfusepath{clip}%
\pgfsetbuttcap%
\pgfsetroundjoin%
\definecolor{currentfill}{rgb}{0.121569,0.466667,0.705882}%
\pgfsetfillcolor{currentfill}%
\pgfsetlinewidth{1.003750pt}%
\definecolor{currentstroke}{rgb}{0.121569,0.466667,0.705882}%
\pgfsetstrokecolor{currentstroke}%
\pgfsetdash{}{0pt}%
\pgfpathmoveto{\pgfqpoint{2.279080in}{2.978333in}}%
\pgfpathcurveto{\pgfqpoint{2.290130in}{2.978333in}}{\pgfqpoint{2.300729in}{2.982723in}}{\pgfqpoint{2.308542in}{2.990537in}}%
\pgfpathcurveto{\pgfqpoint{2.316356in}{2.998350in}}{\pgfqpoint{2.320746in}{3.008949in}}{\pgfqpoint{2.320746in}{3.020000in}}%
\pgfpathcurveto{\pgfqpoint{2.320746in}{3.031050in}}{\pgfqpoint{2.316356in}{3.041649in}}{\pgfqpoint{2.308542in}{3.049462in}}%
\pgfpathcurveto{\pgfqpoint{2.300729in}{3.057276in}}{\pgfqpoint{2.290130in}{3.061666in}}{\pgfqpoint{2.279080in}{3.061666in}}%
\pgfpathcurveto{\pgfqpoint{2.268030in}{3.061666in}}{\pgfqpoint{2.257430in}{3.057276in}}{\pgfqpoint{2.249617in}{3.049462in}}%
\pgfpathcurveto{\pgfqpoint{2.241803in}{3.041649in}}{\pgfqpoint{2.237413in}{3.031050in}}{\pgfqpoint{2.237413in}{3.020000in}}%
\pgfpathcurveto{\pgfqpoint{2.237413in}{3.008949in}}{\pgfqpoint{2.241803in}{2.998350in}}{\pgfqpoint{2.249617in}{2.990537in}}%
\pgfpathcurveto{\pgfqpoint{2.257430in}{2.982723in}}{\pgfqpoint{2.268030in}{2.978333in}}{\pgfqpoint{2.279080in}{2.978333in}}%
\pgfpathclose%
\pgfusepath{stroke,fill}%
\end{pgfscope}%
\begin{pgfscope}%
\pgfpathrectangle{\pgfqpoint{0.600000in}{0.600000in}}{\pgfqpoint{3.900000in}{3.900000in}}%
\pgfusepath{clip}%
\pgfsetbuttcap%
\pgfsetroundjoin%
\definecolor{currentfill}{rgb}{0.121569,0.466667,0.705882}%
\pgfsetfillcolor{currentfill}%
\pgfsetlinewidth{1.003750pt}%
\definecolor{currentstroke}{rgb}{0.121569,0.466667,0.705882}%
\pgfsetstrokecolor{currentstroke}%
\pgfsetdash{}{0pt}%
\pgfpathmoveto{\pgfqpoint{2.144471in}{2.810980in}}%
\pgfpathcurveto{\pgfqpoint{2.155522in}{2.810980in}}{\pgfqpoint{2.166121in}{2.815370in}}{\pgfqpoint{2.173934in}{2.823183in}}%
\pgfpathcurveto{\pgfqpoint{2.181748in}{2.830997in}}{\pgfqpoint{2.186138in}{2.841596in}}{\pgfqpoint{2.186138in}{2.852646in}}%
\pgfpathcurveto{\pgfqpoint{2.186138in}{2.863696in}}{\pgfqpoint{2.181748in}{2.874295in}}{\pgfqpoint{2.173934in}{2.882109in}}%
\pgfpathcurveto{\pgfqpoint{2.166121in}{2.889923in}}{\pgfqpoint{2.155522in}{2.894313in}}{\pgfqpoint{2.144471in}{2.894313in}}%
\pgfpathcurveto{\pgfqpoint{2.133421in}{2.894313in}}{\pgfqpoint{2.122822in}{2.889923in}}{\pgfqpoint{2.115009in}{2.882109in}}%
\pgfpathcurveto{\pgfqpoint{2.107195in}{2.874295in}}{\pgfqpoint{2.102805in}{2.863696in}}{\pgfqpoint{2.102805in}{2.852646in}}%
\pgfpathcurveto{\pgfqpoint{2.102805in}{2.841596in}}{\pgfqpoint{2.107195in}{2.830997in}}{\pgfqpoint{2.115009in}{2.823183in}}%
\pgfpathcurveto{\pgfqpoint{2.122822in}{2.815370in}}{\pgfqpoint{2.133421in}{2.810980in}}{\pgfqpoint{2.144471in}{2.810980in}}%
\pgfpathclose%
\pgfusepath{stroke,fill}%
\end{pgfscope}%
\begin{pgfscope}%
\pgfpathrectangle{\pgfqpoint{0.600000in}{0.600000in}}{\pgfqpoint{3.900000in}{3.900000in}}%
\pgfusepath{clip}%
\pgfsetbuttcap%
\pgfsetroundjoin%
\definecolor{currentfill}{rgb}{0.121569,0.466667,0.705882}%
\pgfsetfillcolor{currentfill}%
\pgfsetlinewidth{1.003750pt}%
\definecolor{currentstroke}{rgb}{0.121569,0.466667,0.705882}%
\pgfsetstrokecolor{currentstroke}%
\pgfsetdash{}{0pt}%
\pgfpathmoveto{\pgfqpoint{2.949284in}{2.560849in}}%
\pgfpathcurveto{\pgfqpoint{2.960335in}{2.560849in}}{\pgfqpoint{2.970934in}{2.565239in}}{\pgfqpoint{2.978747in}{2.573053in}}%
\pgfpathcurveto{\pgfqpoint{2.986561in}{2.580866in}}{\pgfqpoint{2.990951in}{2.591465in}}{\pgfqpoint{2.990951in}{2.602515in}}%
\pgfpathcurveto{\pgfqpoint{2.990951in}{2.613565in}}{\pgfqpoint{2.986561in}{2.624164in}}{\pgfqpoint{2.978747in}{2.631978in}}%
\pgfpathcurveto{\pgfqpoint{2.970934in}{2.639792in}}{\pgfqpoint{2.960335in}{2.644182in}}{\pgfqpoint{2.949284in}{2.644182in}}%
\pgfpathcurveto{\pgfqpoint{2.938234in}{2.644182in}}{\pgfqpoint{2.927635in}{2.639792in}}{\pgfqpoint{2.919822in}{2.631978in}}%
\pgfpathcurveto{\pgfqpoint{2.912008in}{2.624164in}}{\pgfqpoint{2.907618in}{2.613565in}}{\pgfqpoint{2.907618in}{2.602515in}}%
\pgfpathcurveto{\pgfqpoint{2.907618in}{2.591465in}}{\pgfqpoint{2.912008in}{2.580866in}}{\pgfqpoint{2.919822in}{2.573053in}}%
\pgfpathcurveto{\pgfqpoint{2.927635in}{2.565239in}}{\pgfqpoint{2.938234in}{2.560849in}}{\pgfqpoint{2.949284in}{2.560849in}}%
\pgfpathclose%
\pgfusepath{stroke,fill}%
\end{pgfscope}%
\begin{pgfscope}%
\pgfpathrectangle{\pgfqpoint{0.600000in}{0.600000in}}{\pgfqpoint{3.900000in}{3.900000in}}%
\pgfusepath{clip}%
\pgfsetbuttcap%
\pgfsetroundjoin%
\definecolor{currentfill}{rgb}{0.121569,0.466667,0.705882}%
\pgfsetfillcolor{currentfill}%
\pgfsetlinewidth{1.003750pt}%
\definecolor{currentstroke}{rgb}{0.121569,0.466667,0.705882}%
\pgfsetstrokecolor{currentstroke}%
\pgfsetdash{}{0pt}%
\pgfpathmoveto{\pgfqpoint{2.019436in}{2.040876in}}%
\pgfpathcurveto{\pgfqpoint{2.030487in}{2.040876in}}{\pgfqpoint{2.041086in}{2.045267in}}{\pgfqpoint{2.048899in}{2.053080in}}%
\pgfpathcurveto{\pgfqpoint{2.056713in}{2.060894in}}{\pgfqpoint{2.061103in}{2.071493in}}{\pgfqpoint{2.061103in}{2.082543in}}%
\pgfpathcurveto{\pgfqpoint{2.061103in}{2.093593in}}{\pgfqpoint{2.056713in}{2.104192in}}{\pgfqpoint{2.048899in}{2.112006in}}%
\pgfpathcurveto{\pgfqpoint{2.041086in}{2.119819in}}{\pgfqpoint{2.030487in}{2.124210in}}{\pgfqpoint{2.019436in}{2.124210in}}%
\pgfpathcurveto{\pgfqpoint{2.008386in}{2.124210in}}{\pgfqpoint{1.997787in}{2.119819in}}{\pgfqpoint{1.989974in}{2.112006in}}%
\pgfpathcurveto{\pgfqpoint{1.982160in}{2.104192in}}{\pgfqpoint{1.977770in}{2.093593in}}{\pgfqpoint{1.977770in}{2.082543in}}%
\pgfpathcurveto{\pgfqpoint{1.977770in}{2.071493in}}{\pgfqpoint{1.982160in}{2.060894in}}{\pgfqpoint{1.989974in}{2.053080in}}%
\pgfpathcurveto{\pgfqpoint{1.997787in}{2.045267in}}{\pgfqpoint{2.008386in}{2.040876in}}{\pgfqpoint{2.019436in}{2.040876in}}%
\pgfpathclose%
\pgfusepath{stroke,fill}%
\end{pgfscope}%
\begin{pgfscope}%
\pgfpathrectangle{\pgfqpoint{0.600000in}{0.600000in}}{\pgfqpoint{3.900000in}{3.900000in}}%
\pgfusepath{clip}%
\pgfsetbuttcap%
\pgfsetroundjoin%
\definecolor{currentfill}{rgb}{0.121569,0.466667,0.705882}%
\pgfsetfillcolor{currentfill}%
\pgfsetlinewidth{1.003750pt}%
\definecolor{currentstroke}{rgb}{0.121569,0.466667,0.705882}%
\pgfsetstrokecolor{currentstroke}%
\pgfsetdash{}{0pt}%
\pgfpathmoveto{\pgfqpoint{3.033543in}{2.316614in}}%
\pgfpathcurveto{\pgfqpoint{3.044593in}{2.316614in}}{\pgfqpoint{3.055192in}{2.321004in}}{\pgfqpoint{3.063006in}{2.328818in}}%
\pgfpathcurveto{\pgfqpoint{3.070819in}{2.336631in}}{\pgfqpoint{3.075210in}{2.347230in}}{\pgfqpoint{3.075210in}{2.358280in}}%
\pgfpathcurveto{\pgfqpoint{3.075210in}{2.369330in}}{\pgfqpoint{3.070819in}{2.379930in}}{\pgfqpoint{3.063006in}{2.387743in}}%
\pgfpathcurveto{\pgfqpoint{3.055192in}{2.395557in}}{\pgfqpoint{3.044593in}{2.399947in}}{\pgfqpoint{3.033543in}{2.399947in}}%
\pgfpathcurveto{\pgfqpoint{3.022493in}{2.399947in}}{\pgfqpoint{3.011894in}{2.395557in}}{\pgfqpoint{3.004080in}{2.387743in}}%
\pgfpathcurveto{\pgfqpoint{2.996267in}{2.379930in}}{\pgfqpoint{2.991876in}{2.369330in}}{\pgfqpoint{2.991876in}{2.358280in}}%
\pgfpathcurveto{\pgfqpoint{2.991876in}{2.347230in}}{\pgfqpoint{2.996267in}{2.336631in}}{\pgfqpoint{3.004080in}{2.328818in}}%
\pgfpathcurveto{\pgfqpoint{3.011894in}{2.321004in}}{\pgfqpoint{3.022493in}{2.316614in}}{\pgfqpoint{3.033543in}{2.316614in}}%
\pgfpathclose%
\pgfusepath{stroke,fill}%
\end{pgfscope}%
\begin{pgfscope}%
\pgfpathrectangle{\pgfqpoint{0.600000in}{0.600000in}}{\pgfqpoint{3.900000in}{3.900000in}}%
\pgfusepath{clip}%
\pgfsetbuttcap%
\pgfsetroundjoin%
\definecolor{currentfill}{rgb}{0.121569,0.466667,0.705882}%
\pgfsetfillcolor{currentfill}%
\pgfsetlinewidth{1.003750pt}%
\definecolor{currentstroke}{rgb}{0.121569,0.466667,0.705882}%
\pgfsetstrokecolor{currentstroke}%
\pgfsetdash{}{0pt}%
\pgfpathmoveto{\pgfqpoint{2.701605in}{2.188997in}}%
\pgfpathcurveto{\pgfqpoint{2.712655in}{2.188997in}}{\pgfqpoint{2.723254in}{2.193387in}}{\pgfqpoint{2.731068in}{2.201201in}}%
\pgfpathcurveto{\pgfqpoint{2.738882in}{2.209014in}}{\pgfqpoint{2.743272in}{2.219613in}}{\pgfqpoint{2.743272in}{2.230664in}}%
\pgfpathcurveto{\pgfqpoint{2.743272in}{2.241714in}}{\pgfqpoint{2.738882in}{2.252313in}}{\pgfqpoint{2.731068in}{2.260126in}}%
\pgfpathcurveto{\pgfqpoint{2.723254in}{2.267940in}}{\pgfqpoint{2.712655in}{2.272330in}}{\pgfqpoint{2.701605in}{2.272330in}}%
\pgfpathcurveto{\pgfqpoint{2.690555in}{2.272330in}}{\pgfqpoint{2.679956in}{2.267940in}}{\pgfqpoint{2.672142in}{2.260126in}}%
\pgfpathcurveto{\pgfqpoint{2.664329in}{2.252313in}}{\pgfqpoint{2.659939in}{2.241714in}}{\pgfqpoint{2.659939in}{2.230664in}}%
\pgfpathcurveto{\pgfqpoint{2.659939in}{2.219613in}}{\pgfqpoint{2.664329in}{2.209014in}}{\pgfqpoint{2.672142in}{2.201201in}}%
\pgfpathcurveto{\pgfqpoint{2.679956in}{2.193387in}}{\pgfqpoint{2.690555in}{2.188997in}}{\pgfqpoint{2.701605in}{2.188997in}}%
\pgfpathclose%
\pgfusepath{stroke,fill}%
\end{pgfscope}%
\begin{pgfscope}%
\pgfpathrectangle{\pgfqpoint{0.600000in}{0.600000in}}{\pgfqpoint{3.900000in}{3.900000in}}%
\pgfusepath{clip}%
\pgfsetbuttcap%
\pgfsetroundjoin%
\definecolor{currentfill}{rgb}{0.121569,0.466667,0.705882}%
\pgfsetfillcolor{currentfill}%
\pgfsetlinewidth{1.003750pt}%
\definecolor{currentstroke}{rgb}{0.121569,0.466667,0.705882}%
\pgfsetstrokecolor{currentstroke}%
\pgfsetdash{}{0pt}%
\pgfpathmoveto{\pgfqpoint{2.310245in}{2.293180in}}%
\pgfpathcurveto{\pgfqpoint{2.321295in}{2.293180in}}{\pgfqpoint{2.331894in}{2.297570in}}{\pgfqpoint{2.339708in}{2.305384in}}%
\pgfpathcurveto{\pgfqpoint{2.347521in}{2.313198in}}{\pgfqpoint{2.351912in}{2.323797in}}{\pgfqpoint{2.351912in}{2.334847in}}%
\pgfpathcurveto{\pgfqpoint{2.351912in}{2.345897in}}{\pgfqpoint{2.347521in}{2.356496in}}{\pgfqpoint{2.339708in}{2.364310in}}%
\pgfpathcurveto{\pgfqpoint{2.331894in}{2.372123in}}{\pgfqpoint{2.321295in}{2.376513in}}{\pgfqpoint{2.310245in}{2.376513in}}%
\pgfpathcurveto{\pgfqpoint{2.299195in}{2.376513in}}{\pgfqpoint{2.288596in}{2.372123in}}{\pgfqpoint{2.280782in}{2.364310in}}%
\pgfpathcurveto{\pgfqpoint{2.272969in}{2.356496in}}{\pgfqpoint{2.268578in}{2.345897in}}{\pgfqpoint{2.268578in}{2.334847in}}%
\pgfpathcurveto{\pgfqpoint{2.268578in}{2.323797in}}{\pgfqpoint{2.272969in}{2.313198in}}{\pgfqpoint{2.280782in}{2.305384in}}%
\pgfpathcurveto{\pgfqpoint{2.288596in}{2.297570in}}{\pgfqpoint{2.299195in}{2.293180in}}{\pgfqpoint{2.310245in}{2.293180in}}%
\pgfpathclose%
\pgfusepath{stroke,fill}%
\end{pgfscope}%
\begin{pgfscope}%
\pgfpathrectangle{\pgfqpoint{0.600000in}{0.600000in}}{\pgfqpoint{3.900000in}{3.900000in}}%
\pgfusepath{clip}%
\pgfsetbuttcap%
\pgfsetroundjoin%
\definecolor{currentfill}{rgb}{0.121569,0.466667,0.705882}%
\pgfsetfillcolor{currentfill}%
\pgfsetlinewidth{1.003750pt}%
\definecolor{currentstroke}{rgb}{0.121569,0.466667,0.705882}%
\pgfsetstrokecolor{currentstroke}%
\pgfsetdash{}{0pt}%
\pgfpathmoveto{\pgfqpoint{2.885812in}{1.797449in}}%
\pgfpathcurveto{\pgfqpoint{2.896862in}{1.797449in}}{\pgfqpoint{2.907461in}{1.801839in}}{\pgfqpoint{2.915275in}{1.809653in}}%
\pgfpathcurveto{\pgfqpoint{2.923089in}{1.817467in}}{\pgfqpoint{2.927479in}{1.828066in}}{\pgfqpoint{2.927479in}{1.839116in}}%
\pgfpathcurveto{\pgfqpoint{2.927479in}{1.850166in}}{\pgfqpoint{2.923089in}{1.860765in}}{\pgfqpoint{2.915275in}{1.868579in}}%
\pgfpathcurveto{\pgfqpoint{2.907461in}{1.876392in}}{\pgfqpoint{2.896862in}{1.880782in}}{\pgfqpoint{2.885812in}{1.880782in}}%
\pgfpathcurveto{\pgfqpoint{2.874762in}{1.880782in}}{\pgfqpoint{2.864163in}{1.876392in}}{\pgfqpoint{2.856349in}{1.868579in}}%
\pgfpathcurveto{\pgfqpoint{2.848536in}{1.860765in}}{\pgfqpoint{2.844145in}{1.850166in}}{\pgfqpoint{2.844145in}{1.839116in}}%
\pgfpathcurveto{\pgfqpoint{2.844145in}{1.828066in}}{\pgfqpoint{2.848536in}{1.817467in}}{\pgfqpoint{2.856349in}{1.809653in}}%
\pgfpathcurveto{\pgfqpoint{2.864163in}{1.801839in}}{\pgfqpoint{2.874762in}{1.797449in}}{\pgfqpoint{2.885812in}{1.797449in}}%
\pgfpathclose%
\pgfusepath{stroke,fill}%
\end{pgfscope}%
\begin{pgfscope}%
\pgfpathrectangle{\pgfqpoint{0.600000in}{0.600000in}}{\pgfqpoint{3.900000in}{3.900000in}}%
\pgfusepath{clip}%
\pgfsetbuttcap%
\pgfsetroundjoin%
\definecolor{currentfill}{rgb}{0.121569,0.466667,0.705882}%
\pgfsetfillcolor{currentfill}%
\pgfsetlinewidth{1.003750pt}%
\definecolor{currentstroke}{rgb}{0.121569,0.466667,0.705882}%
\pgfsetstrokecolor{currentstroke}%
\pgfsetdash{}{0pt}%
\pgfpathmoveto{\pgfqpoint{3.001255in}{3.711815in}}%
\pgfpathcurveto{\pgfqpoint{3.012305in}{3.711815in}}{\pgfqpoint{3.022904in}{3.716205in}}{\pgfqpoint{3.030718in}{3.724019in}}%
\pgfpathcurveto{\pgfqpoint{3.038532in}{3.731832in}}{\pgfqpoint{3.042922in}{3.742431in}}{\pgfqpoint{3.042922in}{3.753481in}}%
\pgfpathcurveto{\pgfqpoint{3.042922in}{3.764532in}}{\pgfqpoint{3.038532in}{3.775131in}}{\pgfqpoint{3.030718in}{3.782944in}}%
\pgfpathcurveto{\pgfqpoint{3.022904in}{3.790758in}}{\pgfqpoint{3.012305in}{3.795148in}}{\pgfqpoint{3.001255in}{3.795148in}}%
\pgfpathcurveto{\pgfqpoint{2.990205in}{3.795148in}}{\pgfqpoint{2.979606in}{3.790758in}}{\pgfqpoint{2.971792in}{3.782944in}}%
\pgfpathcurveto{\pgfqpoint{2.963979in}{3.775131in}}{\pgfqpoint{2.959588in}{3.764532in}}{\pgfqpoint{2.959588in}{3.753481in}}%
\pgfpathcurveto{\pgfqpoint{2.959588in}{3.742431in}}{\pgfqpoint{2.963979in}{3.731832in}}{\pgfqpoint{2.971792in}{3.724019in}}%
\pgfpathcurveto{\pgfqpoint{2.979606in}{3.716205in}}{\pgfqpoint{2.990205in}{3.711815in}}{\pgfqpoint{3.001255in}{3.711815in}}%
\pgfpathclose%
\pgfusepath{stroke,fill}%
\end{pgfscope}%
\begin{pgfscope}%
\pgfpathrectangle{\pgfqpoint{0.600000in}{0.600000in}}{\pgfqpoint{3.900000in}{3.900000in}}%
\pgfusepath{clip}%
\pgfsetbuttcap%
\pgfsetroundjoin%
\definecolor{currentfill}{rgb}{0.121569,0.466667,0.705882}%
\pgfsetfillcolor{currentfill}%
\pgfsetlinewidth{1.003750pt}%
\definecolor{currentstroke}{rgb}{0.121569,0.466667,0.705882}%
\pgfsetstrokecolor{currentstroke}%
\pgfsetdash{}{0pt}%
\pgfpathmoveto{\pgfqpoint{2.784578in}{2.984296in}}%
\pgfpathcurveto{\pgfqpoint{2.795628in}{2.984296in}}{\pgfqpoint{2.806228in}{2.988686in}}{\pgfqpoint{2.814041in}{2.996500in}}%
\pgfpathcurveto{\pgfqpoint{2.821855in}{3.004314in}}{\pgfqpoint{2.826245in}{3.014913in}}{\pgfqpoint{2.826245in}{3.025963in}}%
\pgfpathcurveto{\pgfqpoint{2.826245in}{3.037013in}}{\pgfqpoint{2.821855in}{3.047612in}}{\pgfqpoint{2.814041in}{3.055426in}}%
\pgfpathcurveto{\pgfqpoint{2.806228in}{3.063239in}}{\pgfqpoint{2.795628in}{3.067629in}}{\pgfqpoint{2.784578in}{3.067629in}}%
\pgfpathcurveto{\pgfqpoint{2.773528in}{3.067629in}}{\pgfqpoint{2.762929in}{3.063239in}}{\pgfqpoint{2.755116in}{3.055426in}}%
\pgfpathcurveto{\pgfqpoint{2.747302in}{3.047612in}}{\pgfqpoint{2.742912in}{3.037013in}}{\pgfqpoint{2.742912in}{3.025963in}}%
\pgfpathcurveto{\pgfqpoint{2.742912in}{3.014913in}}{\pgfqpoint{2.747302in}{3.004314in}}{\pgfqpoint{2.755116in}{2.996500in}}%
\pgfpathcurveto{\pgfqpoint{2.762929in}{2.988686in}}{\pgfqpoint{2.773528in}{2.984296in}}{\pgfqpoint{2.784578in}{2.984296in}}%
\pgfpathclose%
\pgfusepath{stroke,fill}%
\end{pgfscope}%
\begin{pgfscope}%
\pgfpathrectangle{\pgfqpoint{0.600000in}{0.600000in}}{\pgfqpoint{3.900000in}{3.900000in}}%
\pgfusepath{clip}%
\pgfsetbuttcap%
\pgfsetroundjoin%
\definecolor{currentfill}{rgb}{0.121569,0.466667,0.705882}%
\pgfsetfillcolor{currentfill}%
\pgfsetlinewidth{1.003750pt}%
\definecolor{currentstroke}{rgb}{0.121569,0.466667,0.705882}%
\pgfsetstrokecolor{currentstroke}%
\pgfsetdash{}{0pt}%
\pgfpathmoveto{\pgfqpoint{1.891590in}{2.494060in}}%
\pgfpathcurveto{\pgfqpoint{1.902640in}{2.494060in}}{\pgfqpoint{1.913239in}{2.498450in}}{\pgfqpoint{1.921052in}{2.506264in}}%
\pgfpathcurveto{\pgfqpoint{1.928866in}{2.514077in}}{\pgfqpoint{1.933256in}{2.524676in}}{\pgfqpoint{1.933256in}{2.535727in}}%
\pgfpathcurveto{\pgfqpoint{1.933256in}{2.546777in}}{\pgfqpoint{1.928866in}{2.557376in}}{\pgfqpoint{1.921052in}{2.565189in}}%
\pgfpathcurveto{\pgfqpoint{1.913239in}{2.573003in}}{\pgfqpoint{1.902640in}{2.577393in}}{\pgfqpoint{1.891590in}{2.577393in}}%
\pgfpathcurveto{\pgfqpoint{1.880539in}{2.577393in}}{\pgfqpoint{1.869940in}{2.573003in}}{\pgfqpoint{1.862127in}{2.565189in}}%
\pgfpathcurveto{\pgfqpoint{1.854313in}{2.557376in}}{\pgfqpoint{1.849923in}{2.546777in}}{\pgfqpoint{1.849923in}{2.535727in}}%
\pgfpathcurveto{\pgfqpoint{1.849923in}{2.524676in}}{\pgfqpoint{1.854313in}{2.514077in}}{\pgfqpoint{1.862127in}{2.506264in}}%
\pgfpathcurveto{\pgfqpoint{1.869940in}{2.498450in}}{\pgfqpoint{1.880539in}{2.494060in}}{\pgfqpoint{1.891590in}{2.494060in}}%
\pgfpathclose%
\pgfusepath{stroke,fill}%
\end{pgfscope}%
\begin{pgfscope}%
\pgfpathrectangle{\pgfqpoint{0.600000in}{0.600000in}}{\pgfqpoint{3.900000in}{3.900000in}}%
\pgfusepath{clip}%
\pgfsetbuttcap%
\pgfsetroundjoin%
\definecolor{currentfill}{rgb}{0.121569,0.466667,0.705882}%
\pgfsetfillcolor{currentfill}%
\pgfsetlinewidth{1.003750pt}%
\definecolor{currentstroke}{rgb}{0.121569,0.466667,0.705882}%
\pgfsetstrokecolor{currentstroke}%
\pgfsetdash{}{0pt}%
\pgfpathmoveto{\pgfqpoint{1.967622in}{2.633848in}}%
\pgfpathcurveto{\pgfqpoint{1.978673in}{2.633848in}}{\pgfqpoint{1.989272in}{2.638239in}}{\pgfqpoint{1.997085in}{2.646052in}}%
\pgfpathcurveto{\pgfqpoint{2.004899in}{2.653866in}}{\pgfqpoint{2.009289in}{2.664465in}}{\pgfqpoint{2.009289in}{2.675515in}}%
\pgfpathcurveto{\pgfqpoint{2.009289in}{2.686565in}}{\pgfqpoint{2.004899in}{2.697164in}}{\pgfqpoint{1.997085in}{2.704978in}}%
\pgfpathcurveto{\pgfqpoint{1.989272in}{2.712791in}}{\pgfqpoint{1.978673in}{2.717182in}}{\pgfqpoint{1.967622in}{2.717182in}}%
\pgfpathcurveto{\pgfqpoint{1.956572in}{2.717182in}}{\pgfqpoint{1.945973in}{2.712791in}}{\pgfqpoint{1.938160in}{2.704978in}}%
\pgfpathcurveto{\pgfqpoint{1.930346in}{2.697164in}}{\pgfqpoint{1.925956in}{2.686565in}}{\pgfqpoint{1.925956in}{2.675515in}}%
\pgfpathcurveto{\pgfqpoint{1.925956in}{2.664465in}}{\pgfqpoint{1.930346in}{2.653866in}}{\pgfqpoint{1.938160in}{2.646052in}}%
\pgfpathcurveto{\pgfqpoint{1.945973in}{2.638239in}}{\pgfqpoint{1.956572in}{2.633848in}}{\pgfqpoint{1.967622in}{2.633848in}}%
\pgfpathclose%
\pgfusepath{stroke,fill}%
\end{pgfscope}%
\begin{pgfscope}%
\pgfpathrectangle{\pgfqpoint{0.600000in}{0.600000in}}{\pgfqpoint{3.900000in}{3.900000in}}%
\pgfusepath{clip}%
\pgfsetbuttcap%
\pgfsetroundjoin%
\definecolor{currentfill}{rgb}{0.121569,0.466667,0.705882}%
\pgfsetfillcolor{currentfill}%
\pgfsetlinewidth{1.003750pt}%
\definecolor{currentstroke}{rgb}{0.121569,0.466667,0.705882}%
\pgfsetstrokecolor{currentstroke}%
\pgfsetdash{}{0pt}%
\pgfpathmoveto{\pgfqpoint{2.950241in}{3.180091in}}%
\pgfpathcurveto{\pgfqpoint{2.961291in}{3.180091in}}{\pgfqpoint{2.971890in}{3.184481in}}{\pgfqpoint{2.979704in}{3.192295in}}%
\pgfpathcurveto{\pgfqpoint{2.987518in}{3.200109in}}{\pgfqpoint{2.991908in}{3.210708in}}{\pgfqpoint{2.991908in}{3.221758in}}%
\pgfpathcurveto{\pgfqpoint{2.991908in}{3.232808in}}{\pgfqpoint{2.987518in}{3.243407in}}{\pgfqpoint{2.979704in}{3.251220in}}%
\pgfpathcurveto{\pgfqpoint{2.971890in}{3.259034in}}{\pgfqpoint{2.961291in}{3.263424in}}{\pgfqpoint{2.950241in}{3.263424in}}%
\pgfpathcurveto{\pgfqpoint{2.939191in}{3.263424in}}{\pgfqpoint{2.928592in}{3.259034in}}{\pgfqpoint{2.920778in}{3.251220in}}%
\pgfpathcurveto{\pgfqpoint{2.912965in}{3.243407in}}{\pgfqpoint{2.908575in}{3.232808in}}{\pgfqpoint{2.908575in}{3.221758in}}%
\pgfpathcurveto{\pgfqpoint{2.908575in}{3.210708in}}{\pgfqpoint{2.912965in}{3.200109in}}{\pgfqpoint{2.920778in}{3.192295in}}%
\pgfpathcurveto{\pgfqpoint{2.928592in}{3.184481in}}{\pgfqpoint{2.939191in}{3.180091in}}{\pgfqpoint{2.950241in}{3.180091in}}%
\pgfpathclose%
\pgfusepath{stroke,fill}%
\end{pgfscope}%
\begin{pgfscope}%
\pgfpathrectangle{\pgfqpoint{0.600000in}{0.600000in}}{\pgfqpoint{3.900000in}{3.900000in}}%
\pgfusepath{clip}%
\pgfsetbuttcap%
\pgfsetroundjoin%
\definecolor{currentfill}{rgb}{0.121569,0.466667,0.705882}%
\pgfsetfillcolor{currentfill}%
\pgfsetlinewidth{1.003750pt}%
\definecolor{currentstroke}{rgb}{0.121569,0.466667,0.705882}%
\pgfsetstrokecolor{currentstroke}%
\pgfsetdash{}{0pt}%
\pgfpathmoveto{\pgfqpoint{2.204485in}{2.683168in}}%
\pgfpathcurveto{\pgfqpoint{2.215535in}{2.683168in}}{\pgfqpoint{2.226134in}{2.687558in}}{\pgfqpoint{2.233948in}{2.695372in}}%
\pgfpathcurveto{\pgfqpoint{2.241761in}{2.703185in}}{\pgfqpoint{2.246151in}{2.713784in}}{\pgfqpoint{2.246151in}{2.724834in}}%
\pgfpathcurveto{\pgfqpoint{2.246151in}{2.735885in}}{\pgfqpoint{2.241761in}{2.746484in}}{\pgfqpoint{2.233948in}{2.754297in}}%
\pgfpathcurveto{\pgfqpoint{2.226134in}{2.762111in}}{\pgfqpoint{2.215535in}{2.766501in}}{\pgfqpoint{2.204485in}{2.766501in}}%
\pgfpathcurveto{\pgfqpoint{2.193435in}{2.766501in}}{\pgfqpoint{2.182836in}{2.762111in}}{\pgfqpoint{2.175022in}{2.754297in}}%
\pgfpathcurveto{\pgfqpoint{2.167208in}{2.746484in}}{\pgfqpoint{2.162818in}{2.735885in}}{\pgfqpoint{2.162818in}{2.724834in}}%
\pgfpathcurveto{\pgfqpoint{2.162818in}{2.713784in}}{\pgfqpoint{2.167208in}{2.703185in}}{\pgfqpoint{2.175022in}{2.695372in}}%
\pgfpathcurveto{\pgfqpoint{2.182836in}{2.687558in}}{\pgfqpoint{2.193435in}{2.683168in}}{\pgfqpoint{2.204485in}{2.683168in}}%
\pgfpathclose%
\pgfusepath{stroke,fill}%
\end{pgfscope}%
\begin{pgfscope}%
\pgfpathrectangle{\pgfqpoint{0.600000in}{0.600000in}}{\pgfqpoint{3.900000in}{3.900000in}}%
\pgfusepath{clip}%
\pgfsetbuttcap%
\pgfsetroundjoin%
\definecolor{currentfill}{rgb}{0.121569,0.466667,0.705882}%
\pgfsetfillcolor{currentfill}%
\pgfsetlinewidth{1.003750pt}%
\definecolor{currentstroke}{rgb}{0.121569,0.466667,0.705882}%
\pgfsetstrokecolor{currentstroke}%
\pgfsetdash{}{0pt}%
\pgfpathmoveto{\pgfqpoint{2.860738in}{2.065898in}}%
\pgfpathcurveto{\pgfqpoint{2.871789in}{2.065898in}}{\pgfqpoint{2.882388in}{2.070288in}}{\pgfqpoint{2.890201in}{2.078102in}}%
\pgfpathcurveto{\pgfqpoint{2.898015in}{2.085916in}}{\pgfqpoint{2.902405in}{2.096515in}}{\pgfqpoint{2.902405in}{2.107565in}}%
\pgfpathcurveto{\pgfqpoint{2.902405in}{2.118615in}}{\pgfqpoint{2.898015in}{2.129214in}}{\pgfqpoint{2.890201in}{2.137028in}}%
\pgfpathcurveto{\pgfqpoint{2.882388in}{2.144841in}}{\pgfqpoint{2.871789in}{2.149231in}}{\pgfqpoint{2.860738in}{2.149231in}}%
\pgfpathcurveto{\pgfqpoint{2.849688in}{2.149231in}}{\pgfqpoint{2.839089in}{2.144841in}}{\pgfqpoint{2.831276in}{2.137028in}}%
\pgfpathcurveto{\pgfqpoint{2.823462in}{2.129214in}}{\pgfqpoint{2.819072in}{2.118615in}}{\pgfqpoint{2.819072in}{2.107565in}}%
\pgfpathcurveto{\pgfqpoint{2.819072in}{2.096515in}}{\pgfqpoint{2.823462in}{2.085916in}}{\pgfqpoint{2.831276in}{2.078102in}}%
\pgfpathcurveto{\pgfqpoint{2.839089in}{2.070288in}}{\pgfqpoint{2.849688in}{2.065898in}}{\pgfqpoint{2.860738in}{2.065898in}}%
\pgfpathclose%
\pgfusepath{stroke,fill}%
\end{pgfscope}%
\begin{pgfscope}%
\pgfpathrectangle{\pgfqpoint{0.600000in}{0.600000in}}{\pgfqpoint{3.900000in}{3.900000in}}%
\pgfusepath{clip}%
\pgfsetbuttcap%
\pgfsetroundjoin%
\definecolor{currentfill}{rgb}{0.121569,0.466667,0.705882}%
\pgfsetfillcolor{currentfill}%
\pgfsetlinewidth{1.003750pt}%
\definecolor{currentstroke}{rgb}{0.121569,0.466667,0.705882}%
\pgfsetstrokecolor{currentstroke}%
\pgfsetdash{}{0pt}%
\pgfpathmoveto{\pgfqpoint{2.797801in}{2.754906in}}%
\pgfpathcurveto{\pgfqpoint{2.808852in}{2.754906in}}{\pgfqpoint{2.819451in}{2.759297in}}{\pgfqpoint{2.827264in}{2.767110in}}%
\pgfpathcurveto{\pgfqpoint{2.835078in}{2.774924in}}{\pgfqpoint{2.839468in}{2.785523in}}{\pgfqpoint{2.839468in}{2.796573in}}%
\pgfpathcurveto{\pgfqpoint{2.839468in}{2.807623in}}{\pgfqpoint{2.835078in}{2.818222in}}{\pgfqpoint{2.827264in}{2.826036in}}%
\pgfpathcurveto{\pgfqpoint{2.819451in}{2.833849in}}{\pgfqpoint{2.808852in}{2.838240in}}{\pgfqpoint{2.797801in}{2.838240in}}%
\pgfpathcurveto{\pgfqpoint{2.786751in}{2.838240in}}{\pgfqpoint{2.776152in}{2.833849in}}{\pgfqpoint{2.768339in}{2.826036in}}%
\pgfpathcurveto{\pgfqpoint{2.760525in}{2.818222in}}{\pgfqpoint{2.756135in}{2.807623in}}{\pgfqpoint{2.756135in}{2.796573in}}%
\pgfpathcurveto{\pgfqpoint{2.756135in}{2.785523in}}{\pgfqpoint{2.760525in}{2.774924in}}{\pgfqpoint{2.768339in}{2.767110in}}%
\pgfpathcurveto{\pgfqpoint{2.776152in}{2.759297in}}{\pgfqpoint{2.786751in}{2.754906in}}{\pgfqpoint{2.797801in}{2.754906in}}%
\pgfpathclose%
\pgfusepath{stroke,fill}%
\end{pgfscope}%
\begin{pgfscope}%
\pgfpathrectangle{\pgfqpoint{0.600000in}{0.600000in}}{\pgfqpoint{3.900000in}{3.900000in}}%
\pgfusepath{clip}%
\pgfsetbuttcap%
\pgfsetroundjoin%
\definecolor{currentfill}{rgb}{0.121569,0.466667,0.705882}%
\pgfsetfillcolor{currentfill}%
\pgfsetlinewidth{1.003750pt}%
\definecolor{currentstroke}{rgb}{0.121569,0.466667,0.705882}%
\pgfsetstrokecolor{currentstroke}%
\pgfsetdash{}{0pt}%
\pgfpathmoveto{\pgfqpoint{2.374055in}{2.413537in}}%
\pgfpathcurveto{\pgfqpoint{2.385106in}{2.413537in}}{\pgfqpoint{2.395705in}{2.417928in}}{\pgfqpoint{2.403518in}{2.425741in}}%
\pgfpathcurveto{\pgfqpoint{2.411332in}{2.433555in}}{\pgfqpoint{2.415722in}{2.444154in}}{\pgfqpoint{2.415722in}{2.455204in}}%
\pgfpathcurveto{\pgfqpoint{2.415722in}{2.466254in}}{\pgfqpoint{2.411332in}{2.476853in}}{\pgfqpoint{2.403518in}{2.484667in}}%
\pgfpathcurveto{\pgfqpoint{2.395705in}{2.492481in}}{\pgfqpoint{2.385106in}{2.496871in}}{\pgfqpoint{2.374055in}{2.496871in}}%
\pgfpathcurveto{\pgfqpoint{2.363005in}{2.496871in}}{\pgfqpoint{2.352406in}{2.492481in}}{\pgfqpoint{2.344593in}{2.484667in}}%
\pgfpathcurveto{\pgfqpoint{2.336779in}{2.476853in}}{\pgfqpoint{2.332389in}{2.466254in}}{\pgfqpoint{2.332389in}{2.455204in}}%
\pgfpathcurveto{\pgfqpoint{2.332389in}{2.444154in}}{\pgfqpoint{2.336779in}{2.433555in}}{\pgfqpoint{2.344593in}{2.425741in}}%
\pgfpathcurveto{\pgfqpoint{2.352406in}{2.417928in}}{\pgfqpoint{2.363005in}{2.413537in}}{\pgfqpoint{2.374055in}{2.413537in}}%
\pgfpathclose%
\pgfusepath{stroke,fill}%
\end{pgfscope}%
\begin{pgfscope}%
\pgfpathrectangle{\pgfqpoint{0.600000in}{0.600000in}}{\pgfqpoint{3.900000in}{3.900000in}}%
\pgfusepath{clip}%
\pgfsetbuttcap%
\pgfsetroundjoin%
\definecolor{currentfill}{rgb}{0.121569,0.466667,0.705882}%
\pgfsetfillcolor{currentfill}%
\pgfsetlinewidth{1.003750pt}%
\definecolor{currentstroke}{rgb}{0.121569,0.466667,0.705882}%
\pgfsetstrokecolor{currentstroke}%
\pgfsetdash{}{0pt}%
\pgfpathmoveto{\pgfqpoint{2.199856in}{1.955185in}}%
\pgfpathcurveto{\pgfqpoint{2.210907in}{1.955185in}}{\pgfqpoint{2.221506in}{1.959575in}}{\pgfqpoint{2.229319in}{1.967389in}}%
\pgfpathcurveto{\pgfqpoint{2.237133in}{1.975202in}}{\pgfqpoint{2.241523in}{1.985801in}}{\pgfqpoint{2.241523in}{1.996852in}}%
\pgfpathcurveto{\pgfqpoint{2.241523in}{2.007902in}}{\pgfqpoint{2.237133in}{2.018501in}}{\pgfqpoint{2.229319in}{2.026314in}}%
\pgfpathcurveto{\pgfqpoint{2.221506in}{2.034128in}}{\pgfqpoint{2.210907in}{2.038518in}}{\pgfqpoint{2.199856in}{2.038518in}}%
\pgfpathcurveto{\pgfqpoint{2.188806in}{2.038518in}}{\pgfqpoint{2.178207in}{2.034128in}}{\pgfqpoint{2.170394in}{2.026314in}}%
\pgfpathcurveto{\pgfqpoint{2.162580in}{2.018501in}}{\pgfqpoint{2.158190in}{2.007902in}}{\pgfqpoint{2.158190in}{1.996852in}}%
\pgfpathcurveto{\pgfqpoint{2.158190in}{1.985801in}}{\pgfqpoint{2.162580in}{1.975202in}}{\pgfqpoint{2.170394in}{1.967389in}}%
\pgfpathcurveto{\pgfqpoint{2.178207in}{1.959575in}}{\pgfqpoint{2.188806in}{1.955185in}}{\pgfqpoint{2.199856in}{1.955185in}}%
\pgfpathclose%
\pgfusepath{stroke,fill}%
\end{pgfscope}%
\begin{pgfscope}%
\pgfpathrectangle{\pgfqpoint{0.600000in}{0.600000in}}{\pgfqpoint{3.900000in}{3.900000in}}%
\pgfusepath{clip}%
\pgfsetbuttcap%
\pgfsetroundjoin%
\definecolor{currentfill}{rgb}{0.121569,0.466667,0.705882}%
\pgfsetfillcolor{currentfill}%
\pgfsetlinewidth{1.003750pt}%
\definecolor{currentstroke}{rgb}{0.121569,0.466667,0.705882}%
\pgfsetstrokecolor{currentstroke}%
\pgfsetdash{}{0pt}%
\pgfpathmoveto{\pgfqpoint{3.410153in}{2.727967in}}%
\pgfpathcurveto{\pgfqpoint{3.421203in}{2.727967in}}{\pgfqpoint{3.431802in}{2.732358in}}{\pgfqpoint{3.439616in}{2.740171in}}%
\pgfpathcurveto{\pgfqpoint{3.447429in}{2.747985in}}{\pgfqpoint{3.451819in}{2.758584in}}{\pgfqpoint{3.451819in}{2.769634in}}%
\pgfpathcurveto{\pgfqpoint{3.451819in}{2.780684in}}{\pgfqpoint{3.447429in}{2.791283in}}{\pgfqpoint{3.439616in}{2.799097in}}%
\pgfpathcurveto{\pgfqpoint{3.431802in}{2.806910in}}{\pgfqpoint{3.421203in}{2.811301in}}{\pgfqpoint{3.410153in}{2.811301in}}%
\pgfpathcurveto{\pgfqpoint{3.399103in}{2.811301in}}{\pgfqpoint{3.388504in}{2.806910in}}{\pgfqpoint{3.380690in}{2.799097in}}%
\pgfpathcurveto{\pgfqpoint{3.372876in}{2.791283in}}{\pgfqpoint{3.368486in}{2.780684in}}{\pgfqpoint{3.368486in}{2.769634in}}%
\pgfpathcurveto{\pgfqpoint{3.368486in}{2.758584in}}{\pgfqpoint{3.372876in}{2.747985in}}{\pgfqpoint{3.380690in}{2.740171in}}%
\pgfpathcurveto{\pgfqpoint{3.388504in}{2.732358in}}{\pgfqpoint{3.399103in}{2.727967in}}{\pgfqpoint{3.410153in}{2.727967in}}%
\pgfpathclose%
\pgfusepath{stroke,fill}%
\end{pgfscope}%
\begin{pgfscope}%
\pgfpathrectangle{\pgfqpoint{0.600000in}{0.600000in}}{\pgfqpoint{3.900000in}{3.900000in}}%
\pgfusepath{clip}%
\pgfsetbuttcap%
\pgfsetroundjoin%
\definecolor{currentfill}{rgb}{0.121569,0.466667,0.705882}%
\pgfsetfillcolor{currentfill}%
\pgfsetlinewidth{1.003750pt}%
\definecolor{currentstroke}{rgb}{0.121569,0.466667,0.705882}%
\pgfsetstrokecolor{currentstroke}%
\pgfsetdash{}{0pt}%
\pgfpathmoveto{\pgfqpoint{3.182218in}{2.936434in}}%
\pgfpathcurveto{\pgfqpoint{3.193269in}{2.936434in}}{\pgfqpoint{3.203868in}{2.940824in}}{\pgfqpoint{3.211681in}{2.948638in}}%
\pgfpathcurveto{\pgfqpoint{3.219495in}{2.956451in}}{\pgfqpoint{3.223885in}{2.967050in}}{\pgfqpoint{3.223885in}{2.978101in}}%
\pgfpathcurveto{\pgfqpoint{3.223885in}{2.989151in}}{\pgfqpoint{3.219495in}{2.999750in}}{\pgfqpoint{3.211681in}{3.007563in}}%
\pgfpathcurveto{\pgfqpoint{3.203868in}{3.015377in}}{\pgfqpoint{3.193269in}{3.019767in}}{\pgfqpoint{3.182218in}{3.019767in}}%
\pgfpathcurveto{\pgfqpoint{3.171168in}{3.019767in}}{\pgfqpoint{3.160569in}{3.015377in}}{\pgfqpoint{3.152756in}{3.007563in}}%
\pgfpathcurveto{\pgfqpoint{3.144942in}{2.999750in}}{\pgfqpoint{3.140552in}{2.989151in}}{\pgfqpoint{3.140552in}{2.978101in}}%
\pgfpathcurveto{\pgfqpoint{3.140552in}{2.967050in}}{\pgfqpoint{3.144942in}{2.956451in}}{\pgfqpoint{3.152756in}{2.948638in}}%
\pgfpathcurveto{\pgfqpoint{3.160569in}{2.940824in}}{\pgfqpoint{3.171168in}{2.936434in}}{\pgfqpoint{3.182218in}{2.936434in}}%
\pgfpathclose%
\pgfusepath{stroke,fill}%
\end{pgfscope}%
\begin{pgfscope}%
\pgfpathrectangle{\pgfqpoint{0.600000in}{0.600000in}}{\pgfqpoint{3.900000in}{3.900000in}}%
\pgfusepath{clip}%
\pgfsetbuttcap%
\pgfsetroundjoin%
\definecolor{currentfill}{rgb}{0.121569,0.466667,0.705882}%
\pgfsetfillcolor{currentfill}%
\pgfsetlinewidth{1.003750pt}%
\definecolor{currentstroke}{rgb}{0.121569,0.466667,0.705882}%
\pgfsetstrokecolor{currentstroke}%
\pgfsetdash{}{0pt}%
\pgfpathmoveto{\pgfqpoint{2.243046in}{1.896883in}}%
\pgfpathcurveto{\pgfqpoint{2.254096in}{1.896883in}}{\pgfqpoint{2.264695in}{1.901273in}}{\pgfqpoint{2.272509in}{1.909086in}}%
\pgfpathcurveto{\pgfqpoint{2.280322in}{1.916900in}}{\pgfqpoint{2.284712in}{1.927499in}}{\pgfqpoint{2.284712in}{1.938549in}}%
\pgfpathcurveto{\pgfqpoint{2.284712in}{1.949599in}}{\pgfqpoint{2.280322in}{1.960198in}}{\pgfqpoint{2.272509in}{1.968012in}}%
\pgfpathcurveto{\pgfqpoint{2.264695in}{1.975826in}}{\pgfqpoint{2.254096in}{1.980216in}}{\pgfqpoint{2.243046in}{1.980216in}}%
\pgfpathcurveto{\pgfqpoint{2.231996in}{1.980216in}}{\pgfqpoint{2.221397in}{1.975826in}}{\pgfqpoint{2.213583in}{1.968012in}}%
\pgfpathcurveto{\pgfqpoint{2.205769in}{1.960198in}}{\pgfqpoint{2.201379in}{1.949599in}}{\pgfqpoint{2.201379in}{1.938549in}}%
\pgfpathcurveto{\pgfqpoint{2.201379in}{1.927499in}}{\pgfqpoint{2.205769in}{1.916900in}}{\pgfqpoint{2.213583in}{1.909086in}}%
\pgfpathcurveto{\pgfqpoint{2.221397in}{1.901273in}}{\pgfqpoint{2.231996in}{1.896883in}}{\pgfqpoint{2.243046in}{1.896883in}}%
\pgfpathclose%
\pgfusepath{stroke,fill}%
\end{pgfscope}%
\begin{pgfscope}%
\pgfpathrectangle{\pgfqpoint{0.600000in}{0.600000in}}{\pgfqpoint{3.900000in}{3.900000in}}%
\pgfusepath{clip}%
\pgfsetbuttcap%
\pgfsetroundjoin%
\definecolor{currentfill}{rgb}{0.121569,0.466667,0.705882}%
\pgfsetfillcolor{currentfill}%
\pgfsetlinewidth{1.003750pt}%
\definecolor{currentstroke}{rgb}{0.121569,0.466667,0.705882}%
\pgfsetstrokecolor{currentstroke}%
\pgfsetdash{}{0pt}%
\pgfpathmoveto{\pgfqpoint{3.075833in}{2.745408in}}%
\pgfpathcurveto{\pgfqpoint{3.086883in}{2.745408in}}{\pgfqpoint{3.097482in}{2.749799in}}{\pgfqpoint{3.105296in}{2.757612in}}%
\pgfpathcurveto{\pgfqpoint{3.113109in}{2.765426in}}{\pgfqpoint{3.117500in}{2.776025in}}{\pgfqpoint{3.117500in}{2.787075in}}%
\pgfpathcurveto{\pgfqpoint{3.117500in}{2.798125in}}{\pgfqpoint{3.113109in}{2.808724in}}{\pgfqpoint{3.105296in}{2.816538in}}%
\pgfpathcurveto{\pgfqpoint{3.097482in}{2.824351in}}{\pgfqpoint{3.086883in}{2.828742in}}{\pgfqpoint{3.075833in}{2.828742in}}%
\pgfpathcurveto{\pgfqpoint{3.064783in}{2.828742in}}{\pgfqpoint{3.054184in}{2.824351in}}{\pgfqpoint{3.046370in}{2.816538in}}%
\pgfpathcurveto{\pgfqpoint{3.038556in}{2.808724in}}{\pgfqpoint{3.034166in}{2.798125in}}{\pgfqpoint{3.034166in}{2.787075in}}%
\pgfpathcurveto{\pgfqpoint{3.034166in}{2.776025in}}{\pgfqpoint{3.038556in}{2.765426in}}{\pgfqpoint{3.046370in}{2.757612in}}%
\pgfpathcurveto{\pgfqpoint{3.054184in}{2.749799in}}{\pgfqpoint{3.064783in}{2.745408in}}{\pgfqpoint{3.075833in}{2.745408in}}%
\pgfpathclose%
\pgfusepath{stroke,fill}%
\end{pgfscope}%
\begin{pgfscope}%
\pgfpathrectangle{\pgfqpoint{0.600000in}{0.600000in}}{\pgfqpoint{3.900000in}{3.900000in}}%
\pgfusepath{clip}%
\pgfsetbuttcap%
\pgfsetroundjoin%
\definecolor{currentfill}{rgb}{0.121569,0.466667,0.705882}%
\pgfsetfillcolor{currentfill}%
\pgfsetlinewidth{1.003750pt}%
\definecolor{currentstroke}{rgb}{0.121569,0.466667,0.705882}%
\pgfsetstrokecolor{currentstroke}%
\pgfsetdash{}{0pt}%
\pgfpathmoveto{\pgfqpoint{2.806161in}{2.709948in}}%
\pgfpathcurveto{\pgfqpoint{2.817211in}{2.709948in}}{\pgfqpoint{2.827810in}{2.714339in}}{\pgfqpoint{2.835624in}{2.722152in}}%
\pgfpathcurveto{\pgfqpoint{2.843438in}{2.729966in}}{\pgfqpoint{2.847828in}{2.740565in}}{\pgfqpoint{2.847828in}{2.751615in}}%
\pgfpathcurveto{\pgfqpoint{2.847828in}{2.762665in}}{\pgfqpoint{2.843438in}{2.773264in}}{\pgfqpoint{2.835624in}{2.781078in}}%
\pgfpathcurveto{\pgfqpoint{2.827810in}{2.788891in}}{\pgfqpoint{2.817211in}{2.793282in}}{\pgfqpoint{2.806161in}{2.793282in}}%
\pgfpathcurveto{\pgfqpoint{2.795111in}{2.793282in}}{\pgfqpoint{2.784512in}{2.788891in}}{\pgfqpoint{2.776699in}{2.781078in}}%
\pgfpathcurveto{\pgfqpoint{2.768885in}{2.773264in}}{\pgfqpoint{2.764495in}{2.762665in}}{\pgfqpoint{2.764495in}{2.751615in}}%
\pgfpathcurveto{\pgfqpoint{2.764495in}{2.740565in}}{\pgfqpoint{2.768885in}{2.729966in}}{\pgfqpoint{2.776699in}{2.722152in}}%
\pgfpathcurveto{\pgfqpoint{2.784512in}{2.714339in}}{\pgfqpoint{2.795111in}{2.709948in}}{\pgfqpoint{2.806161in}{2.709948in}}%
\pgfpathclose%
\pgfusepath{stroke,fill}%
\end{pgfscope}%
\begin{pgfscope}%
\pgfpathrectangle{\pgfqpoint{0.600000in}{0.600000in}}{\pgfqpoint{3.900000in}{3.900000in}}%
\pgfusepath{clip}%
\pgfsetbuttcap%
\pgfsetroundjoin%
\definecolor{currentfill}{rgb}{0.121569,0.466667,0.705882}%
\pgfsetfillcolor{currentfill}%
\pgfsetlinewidth{1.003750pt}%
\definecolor{currentstroke}{rgb}{0.121569,0.466667,0.705882}%
\pgfsetstrokecolor{currentstroke}%
\pgfsetdash{}{0pt}%
\pgfpathmoveto{\pgfqpoint{3.173190in}{2.030800in}}%
\pgfpathcurveto{\pgfqpoint{3.184240in}{2.030800in}}{\pgfqpoint{3.194839in}{2.035190in}}{\pgfqpoint{3.202653in}{2.043004in}}%
\pgfpathcurveto{\pgfqpoint{3.210466in}{2.050818in}}{\pgfqpoint{3.214856in}{2.061417in}}{\pgfqpoint{3.214856in}{2.072467in}}%
\pgfpathcurveto{\pgfqpoint{3.214856in}{2.083517in}}{\pgfqpoint{3.210466in}{2.094116in}}{\pgfqpoint{3.202653in}{2.101930in}}%
\pgfpathcurveto{\pgfqpoint{3.194839in}{2.109743in}}{\pgfqpoint{3.184240in}{2.114134in}}{\pgfqpoint{3.173190in}{2.114134in}}%
\pgfpathcurveto{\pgfqpoint{3.162140in}{2.114134in}}{\pgfqpoint{3.151541in}{2.109743in}}{\pgfqpoint{3.143727in}{2.101930in}}%
\pgfpathcurveto{\pgfqpoint{3.135913in}{2.094116in}}{\pgfqpoint{3.131523in}{2.083517in}}{\pgfqpoint{3.131523in}{2.072467in}}%
\pgfpathcurveto{\pgfqpoint{3.131523in}{2.061417in}}{\pgfqpoint{3.135913in}{2.050818in}}{\pgfqpoint{3.143727in}{2.043004in}}%
\pgfpathcurveto{\pgfqpoint{3.151541in}{2.035190in}}{\pgfqpoint{3.162140in}{2.030800in}}{\pgfqpoint{3.173190in}{2.030800in}}%
\pgfpathclose%
\pgfusepath{stroke,fill}%
\end{pgfscope}%
\begin{pgfscope}%
\pgfpathrectangle{\pgfqpoint{0.600000in}{0.600000in}}{\pgfqpoint{3.900000in}{3.900000in}}%
\pgfusepath{clip}%
\pgfsetbuttcap%
\pgfsetroundjoin%
\definecolor{currentfill}{rgb}{0.121569,0.466667,0.705882}%
\pgfsetfillcolor{currentfill}%
\pgfsetlinewidth{1.003750pt}%
\definecolor{currentstroke}{rgb}{0.121569,0.466667,0.705882}%
\pgfsetstrokecolor{currentstroke}%
\pgfsetdash{}{0pt}%
\pgfpathmoveto{\pgfqpoint{3.063108in}{3.371256in}}%
\pgfpathcurveto{\pgfqpoint{3.074158in}{3.371256in}}{\pgfqpoint{3.084757in}{3.375646in}}{\pgfqpoint{3.092571in}{3.383460in}}%
\pgfpathcurveto{\pgfqpoint{3.100385in}{3.391274in}}{\pgfqpoint{3.104775in}{3.401873in}}{\pgfqpoint{3.104775in}{3.412923in}}%
\pgfpathcurveto{\pgfqpoint{3.104775in}{3.423973in}}{\pgfqpoint{3.100385in}{3.434572in}}{\pgfqpoint{3.092571in}{3.442385in}}%
\pgfpathcurveto{\pgfqpoint{3.084757in}{3.450199in}}{\pgfqpoint{3.074158in}{3.454589in}}{\pgfqpoint{3.063108in}{3.454589in}}%
\pgfpathcurveto{\pgfqpoint{3.052058in}{3.454589in}}{\pgfqpoint{3.041459in}{3.450199in}}{\pgfqpoint{3.033645in}{3.442385in}}%
\pgfpathcurveto{\pgfqpoint{3.025832in}{3.434572in}}{\pgfqpoint{3.021442in}{3.423973in}}{\pgfqpoint{3.021442in}{3.412923in}}%
\pgfpathcurveto{\pgfqpoint{3.021442in}{3.401873in}}{\pgfqpoint{3.025832in}{3.391274in}}{\pgfqpoint{3.033645in}{3.383460in}}%
\pgfpathcurveto{\pgfqpoint{3.041459in}{3.375646in}}{\pgfqpoint{3.052058in}{3.371256in}}{\pgfqpoint{3.063108in}{3.371256in}}%
\pgfpathclose%
\pgfusepath{stroke,fill}%
\end{pgfscope}%
\begin{pgfscope}%
\pgfpathrectangle{\pgfqpoint{0.600000in}{0.600000in}}{\pgfqpoint{3.900000in}{3.900000in}}%
\pgfusepath{clip}%
\pgfsetbuttcap%
\pgfsetroundjoin%
\definecolor{currentfill}{rgb}{0.121569,0.466667,0.705882}%
\pgfsetfillcolor{currentfill}%
\pgfsetlinewidth{1.003750pt}%
\definecolor{currentstroke}{rgb}{0.121569,0.466667,0.705882}%
\pgfsetstrokecolor{currentstroke}%
\pgfsetdash{}{0pt}%
\pgfpathmoveto{\pgfqpoint{2.847330in}{2.552292in}}%
\pgfpathcurveto{\pgfqpoint{2.858381in}{2.552292in}}{\pgfqpoint{2.868980in}{2.556683in}}{\pgfqpoint{2.876793in}{2.564496in}}%
\pgfpathcurveto{\pgfqpoint{2.884607in}{2.572310in}}{\pgfqpoint{2.888997in}{2.582909in}}{\pgfqpoint{2.888997in}{2.593959in}}%
\pgfpathcurveto{\pgfqpoint{2.888997in}{2.605009in}}{\pgfqpoint{2.884607in}{2.615608in}}{\pgfqpoint{2.876793in}{2.623422in}}%
\pgfpathcurveto{\pgfqpoint{2.868980in}{2.631235in}}{\pgfqpoint{2.858381in}{2.635626in}}{\pgfqpoint{2.847330in}{2.635626in}}%
\pgfpathcurveto{\pgfqpoint{2.836280in}{2.635626in}}{\pgfqpoint{2.825681in}{2.631235in}}{\pgfqpoint{2.817868in}{2.623422in}}%
\pgfpathcurveto{\pgfqpoint{2.810054in}{2.615608in}}{\pgfqpoint{2.805664in}{2.605009in}}{\pgfqpoint{2.805664in}{2.593959in}}%
\pgfpathcurveto{\pgfqpoint{2.805664in}{2.582909in}}{\pgfqpoint{2.810054in}{2.572310in}}{\pgfqpoint{2.817868in}{2.564496in}}%
\pgfpathcurveto{\pgfqpoint{2.825681in}{2.556683in}}{\pgfqpoint{2.836280in}{2.552292in}}{\pgfqpoint{2.847330in}{2.552292in}}%
\pgfpathclose%
\pgfusepath{stroke,fill}%
\end{pgfscope}%
\begin{pgfscope}%
\pgfpathrectangle{\pgfqpoint{0.600000in}{0.600000in}}{\pgfqpoint{3.900000in}{3.900000in}}%
\pgfusepath{clip}%
\pgfsetbuttcap%
\pgfsetroundjoin%
\definecolor{currentfill}{rgb}{0.121569,0.466667,0.705882}%
\pgfsetfillcolor{currentfill}%
\pgfsetlinewidth{1.003750pt}%
\definecolor{currentstroke}{rgb}{0.121569,0.466667,0.705882}%
\pgfsetstrokecolor{currentstroke}%
\pgfsetdash{}{0pt}%
\pgfpathmoveto{\pgfqpoint{2.493143in}{2.485362in}}%
\pgfpathcurveto{\pgfqpoint{2.504193in}{2.485362in}}{\pgfqpoint{2.514792in}{2.489753in}}{\pgfqpoint{2.522606in}{2.497566in}}%
\pgfpathcurveto{\pgfqpoint{2.530420in}{2.505380in}}{\pgfqpoint{2.534810in}{2.515979in}}{\pgfqpoint{2.534810in}{2.527029in}}%
\pgfpathcurveto{\pgfqpoint{2.534810in}{2.538079in}}{\pgfqpoint{2.530420in}{2.548678in}}{\pgfqpoint{2.522606in}{2.556492in}}%
\pgfpathcurveto{\pgfqpoint{2.514792in}{2.564305in}}{\pgfqpoint{2.504193in}{2.568696in}}{\pgfqpoint{2.493143in}{2.568696in}}%
\pgfpathcurveto{\pgfqpoint{2.482093in}{2.568696in}}{\pgfqpoint{2.471494in}{2.564305in}}{\pgfqpoint{2.463680in}{2.556492in}}%
\pgfpathcurveto{\pgfqpoint{2.455867in}{2.548678in}}{\pgfqpoint{2.451477in}{2.538079in}}{\pgfqpoint{2.451477in}{2.527029in}}%
\pgfpathcurveto{\pgfqpoint{2.451477in}{2.515979in}}{\pgfqpoint{2.455867in}{2.505380in}}{\pgfqpoint{2.463680in}{2.497566in}}%
\pgfpathcurveto{\pgfqpoint{2.471494in}{2.489753in}}{\pgfqpoint{2.482093in}{2.485362in}}{\pgfqpoint{2.493143in}{2.485362in}}%
\pgfpathclose%
\pgfusepath{stroke,fill}%
\end{pgfscope}%
\begin{pgfscope}%
\pgfpathrectangle{\pgfqpoint{0.600000in}{0.600000in}}{\pgfqpoint{3.900000in}{3.900000in}}%
\pgfusepath{clip}%
\pgfsetbuttcap%
\pgfsetroundjoin%
\definecolor{currentfill}{rgb}{0.121569,0.466667,0.705882}%
\pgfsetfillcolor{currentfill}%
\pgfsetlinewidth{1.003750pt}%
\definecolor{currentstroke}{rgb}{0.121569,0.466667,0.705882}%
\pgfsetstrokecolor{currentstroke}%
\pgfsetdash{}{0pt}%
\pgfpathmoveto{\pgfqpoint{2.367575in}{3.006464in}}%
\pgfpathcurveto{\pgfqpoint{2.378625in}{3.006464in}}{\pgfqpoint{2.389224in}{3.010854in}}{\pgfqpoint{2.397037in}{3.018668in}}%
\pgfpathcurveto{\pgfqpoint{2.404851in}{3.026481in}}{\pgfqpoint{2.409241in}{3.037080in}}{\pgfqpoint{2.409241in}{3.048130in}}%
\pgfpathcurveto{\pgfqpoint{2.409241in}{3.059180in}}{\pgfqpoint{2.404851in}{3.069779in}}{\pgfqpoint{2.397037in}{3.077593in}}%
\pgfpathcurveto{\pgfqpoint{2.389224in}{3.085407in}}{\pgfqpoint{2.378625in}{3.089797in}}{\pgfqpoint{2.367575in}{3.089797in}}%
\pgfpathcurveto{\pgfqpoint{2.356524in}{3.089797in}}{\pgfqpoint{2.345925in}{3.085407in}}{\pgfqpoint{2.338112in}{3.077593in}}%
\pgfpathcurveto{\pgfqpoint{2.330298in}{3.069779in}}{\pgfqpoint{2.325908in}{3.059180in}}{\pgfqpoint{2.325908in}{3.048130in}}%
\pgfpathcurveto{\pgfqpoint{2.325908in}{3.037080in}}{\pgfqpoint{2.330298in}{3.026481in}}{\pgfqpoint{2.338112in}{3.018668in}}%
\pgfpathcurveto{\pgfqpoint{2.345925in}{3.010854in}}{\pgfqpoint{2.356524in}{3.006464in}}{\pgfqpoint{2.367575in}{3.006464in}}%
\pgfpathclose%
\pgfusepath{stroke,fill}%
\end{pgfscope}%
\begin{pgfscope}%
\pgfpathrectangle{\pgfqpoint{0.600000in}{0.600000in}}{\pgfqpoint{3.900000in}{3.900000in}}%
\pgfusepath{clip}%
\pgfsetbuttcap%
\pgfsetroundjoin%
\definecolor{currentfill}{rgb}{0.121569,0.466667,0.705882}%
\pgfsetfillcolor{currentfill}%
\pgfsetlinewidth{1.003750pt}%
\definecolor{currentstroke}{rgb}{0.121569,0.466667,0.705882}%
\pgfsetstrokecolor{currentstroke}%
\pgfsetdash{}{0pt}%
\pgfpathmoveto{\pgfqpoint{3.540106in}{2.207204in}}%
\pgfpathcurveto{\pgfqpoint{3.551156in}{2.207204in}}{\pgfqpoint{3.561755in}{2.211595in}}{\pgfqpoint{3.569569in}{2.219408in}}%
\pgfpathcurveto{\pgfqpoint{3.577383in}{2.227222in}}{\pgfqpoint{3.581773in}{2.237821in}}{\pgfqpoint{3.581773in}{2.248871in}}%
\pgfpathcurveto{\pgfqpoint{3.581773in}{2.259921in}}{\pgfqpoint{3.577383in}{2.270520in}}{\pgfqpoint{3.569569in}{2.278334in}}%
\pgfpathcurveto{\pgfqpoint{3.561755in}{2.286147in}}{\pgfqpoint{3.551156in}{2.290538in}}{\pgfqpoint{3.540106in}{2.290538in}}%
\pgfpathcurveto{\pgfqpoint{3.529056in}{2.290538in}}{\pgfqpoint{3.518457in}{2.286147in}}{\pgfqpoint{3.510643in}{2.278334in}}%
\pgfpathcurveto{\pgfqpoint{3.502830in}{2.270520in}}{\pgfqpoint{3.498440in}{2.259921in}}{\pgfqpoint{3.498440in}{2.248871in}}%
\pgfpathcurveto{\pgfqpoint{3.498440in}{2.237821in}}{\pgfqpoint{3.502830in}{2.227222in}}{\pgfqpoint{3.510643in}{2.219408in}}%
\pgfpathcurveto{\pgfqpoint{3.518457in}{2.211595in}}{\pgfqpoint{3.529056in}{2.207204in}}{\pgfqpoint{3.540106in}{2.207204in}}%
\pgfpathclose%
\pgfusepath{stroke,fill}%
\end{pgfscope}%
\begin{pgfscope}%
\pgfpathrectangle{\pgfqpoint{0.600000in}{0.600000in}}{\pgfqpoint{3.900000in}{3.900000in}}%
\pgfusepath{clip}%
\pgfsetbuttcap%
\pgfsetroundjoin%
\definecolor{currentfill}{rgb}{0.121569,0.466667,0.705882}%
\pgfsetfillcolor{currentfill}%
\pgfsetlinewidth{1.003750pt}%
\definecolor{currentstroke}{rgb}{0.121569,0.466667,0.705882}%
\pgfsetstrokecolor{currentstroke}%
\pgfsetdash{}{0pt}%
\pgfpathmoveto{\pgfqpoint{3.330112in}{3.291334in}}%
\pgfpathcurveto{\pgfqpoint{3.341162in}{3.291334in}}{\pgfqpoint{3.351761in}{3.295724in}}{\pgfqpoint{3.359575in}{3.303538in}}%
\pgfpathcurveto{\pgfqpoint{3.367389in}{3.311351in}}{\pgfqpoint{3.371779in}{3.321950in}}{\pgfqpoint{3.371779in}{3.333000in}}%
\pgfpathcurveto{\pgfqpoint{3.371779in}{3.344051in}}{\pgfqpoint{3.367389in}{3.354650in}}{\pgfqpoint{3.359575in}{3.362463in}}%
\pgfpathcurveto{\pgfqpoint{3.351761in}{3.370277in}}{\pgfqpoint{3.341162in}{3.374667in}}{\pgfqpoint{3.330112in}{3.374667in}}%
\pgfpathcurveto{\pgfqpoint{3.319062in}{3.374667in}}{\pgfqpoint{3.308463in}{3.370277in}}{\pgfqpoint{3.300649in}{3.362463in}}%
\pgfpathcurveto{\pgfqpoint{3.292836in}{3.354650in}}{\pgfqpoint{3.288445in}{3.344051in}}{\pgfqpoint{3.288445in}{3.333000in}}%
\pgfpathcurveto{\pgfqpoint{3.288445in}{3.321950in}}{\pgfqpoint{3.292836in}{3.311351in}}{\pgfqpoint{3.300649in}{3.303538in}}%
\pgfpathcurveto{\pgfqpoint{3.308463in}{3.295724in}}{\pgfqpoint{3.319062in}{3.291334in}}{\pgfqpoint{3.330112in}{3.291334in}}%
\pgfpathclose%
\pgfusepath{stroke,fill}%
\end{pgfscope}%
\begin{pgfscope}%
\pgfpathrectangle{\pgfqpoint{0.600000in}{0.600000in}}{\pgfqpoint{3.900000in}{3.900000in}}%
\pgfusepath{clip}%
\pgfsetbuttcap%
\pgfsetroundjoin%
\definecolor{currentfill}{rgb}{0.121569,0.466667,0.705882}%
\pgfsetfillcolor{currentfill}%
\pgfsetlinewidth{1.003750pt}%
\definecolor{currentstroke}{rgb}{0.121569,0.466667,0.705882}%
\pgfsetstrokecolor{currentstroke}%
\pgfsetdash{}{0pt}%
\pgfpathmoveto{\pgfqpoint{2.110510in}{2.875661in}}%
\pgfpathcurveto{\pgfqpoint{2.121560in}{2.875661in}}{\pgfqpoint{2.132159in}{2.880051in}}{\pgfqpoint{2.139973in}{2.887865in}}%
\pgfpathcurveto{\pgfqpoint{2.147786in}{2.895679in}}{\pgfqpoint{2.152177in}{2.906278in}}{\pgfqpoint{2.152177in}{2.917328in}}%
\pgfpathcurveto{\pgfqpoint{2.152177in}{2.928378in}}{\pgfqpoint{2.147786in}{2.938977in}}{\pgfqpoint{2.139973in}{2.946791in}}%
\pgfpathcurveto{\pgfqpoint{2.132159in}{2.954604in}}{\pgfqpoint{2.121560in}{2.958994in}}{\pgfqpoint{2.110510in}{2.958994in}}%
\pgfpathcurveto{\pgfqpoint{2.099460in}{2.958994in}}{\pgfqpoint{2.088861in}{2.954604in}}{\pgfqpoint{2.081047in}{2.946791in}}%
\pgfpathcurveto{\pgfqpoint{2.073234in}{2.938977in}}{\pgfqpoint{2.068843in}{2.928378in}}{\pgfqpoint{2.068843in}{2.917328in}}%
\pgfpathcurveto{\pgfqpoint{2.068843in}{2.906278in}}{\pgfqpoint{2.073234in}{2.895679in}}{\pgfqpoint{2.081047in}{2.887865in}}%
\pgfpathcurveto{\pgfqpoint{2.088861in}{2.880051in}}{\pgfqpoint{2.099460in}{2.875661in}}{\pgfqpoint{2.110510in}{2.875661in}}%
\pgfpathclose%
\pgfusepath{stroke,fill}%
\end{pgfscope}%
\begin{pgfscope}%
\pgfpathrectangle{\pgfqpoint{0.600000in}{0.600000in}}{\pgfqpoint{3.900000in}{3.900000in}}%
\pgfusepath{clip}%
\pgfsetbuttcap%
\pgfsetroundjoin%
\definecolor{currentfill}{rgb}{0.121569,0.466667,0.705882}%
\pgfsetfillcolor{currentfill}%
\pgfsetlinewidth{1.003750pt}%
\definecolor{currentstroke}{rgb}{0.121569,0.466667,0.705882}%
\pgfsetstrokecolor{currentstroke}%
\pgfsetdash{}{0pt}%
\pgfpathmoveto{\pgfqpoint{1.848508in}{3.065489in}}%
\pgfpathcurveto{\pgfqpoint{1.859558in}{3.065489in}}{\pgfqpoint{1.870157in}{3.069879in}}{\pgfqpoint{1.877971in}{3.077693in}}%
\pgfpathcurveto{\pgfqpoint{1.885784in}{3.085507in}}{\pgfqpoint{1.890175in}{3.096106in}}{\pgfqpoint{1.890175in}{3.107156in}}%
\pgfpathcurveto{\pgfqpoint{1.890175in}{3.118206in}}{\pgfqpoint{1.885784in}{3.128805in}}{\pgfqpoint{1.877971in}{3.136619in}}%
\pgfpathcurveto{\pgfqpoint{1.870157in}{3.144432in}}{\pgfqpoint{1.859558in}{3.148823in}}{\pgfqpoint{1.848508in}{3.148823in}}%
\pgfpathcurveto{\pgfqpoint{1.837458in}{3.148823in}}{\pgfqpoint{1.826859in}{3.144432in}}{\pgfqpoint{1.819045in}{3.136619in}}%
\pgfpathcurveto{\pgfqpoint{1.811232in}{3.128805in}}{\pgfqpoint{1.806841in}{3.118206in}}{\pgfqpoint{1.806841in}{3.107156in}}%
\pgfpathcurveto{\pgfqpoint{1.806841in}{3.096106in}}{\pgfqpoint{1.811232in}{3.085507in}}{\pgfqpoint{1.819045in}{3.077693in}}%
\pgfpathcurveto{\pgfqpoint{1.826859in}{3.069879in}}{\pgfqpoint{1.837458in}{3.065489in}}{\pgfqpoint{1.848508in}{3.065489in}}%
\pgfpathclose%
\pgfusepath{stroke,fill}%
\end{pgfscope}%
\begin{pgfscope}%
\pgfpathrectangle{\pgfqpoint{0.600000in}{0.600000in}}{\pgfqpoint{3.900000in}{3.900000in}}%
\pgfusepath{clip}%
\pgfsetbuttcap%
\pgfsetroundjoin%
\definecolor{currentfill}{rgb}{0.121569,0.466667,0.705882}%
\pgfsetfillcolor{currentfill}%
\pgfsetlinewidth{1.003750pt}%
\definecolor{currentstroke}{rgb}{0.121569,0.466667,0.705882}%
\pgfsetstrokecolor{currentstroke}%
\pgfsetdash{}{0pt}%
\pgfpathmoveto{\pgfqpoint{3.448279in}{2.281213in}}%
\pgfpathcurveto{\pgfqpoint{3.459329in}{2.281213in}}{\pgfqpoint{3.469928in}{2.285604in}}{\pgfqpoint{3.477742in}{2.293417in}}%
\pgfpathcurveto{\pgfqpoint{3.485555in}{2.301231in}}{\pgfqpoint{3.489946in}{2.311830in}}{\pgfqpoint{3.489946in}{2.322880in}}%
\pgfpathcurveto{\pgfqpoint{3.489946in}{2.333930in}}{\pgfqpoint{3.485555in}{2.344529in}}{\pgfqpoint{3.477742in}{2.352343in}}%
\pgfpathcurveto{\pgfqpoint{3.469928in}{2.360156in}}{\pgfqpoint{3.459329in}{2.364547in}}{\pgfqpoint{3.448279in}{2.364547in}}%
\pgfpathcurveto{\pgfqpoint{3.437229in}{2.364547in}}{\pgfqpoint{3.426630in}{2.360156in}}{\pgfqpoint{3.418816in}{2.352343in}}%
\pgfpathcurveto{\pgfqpoint{3.411003in}{2.344529in}}{\pgfqpoint{3.406612in}{2.333930in}}{\pgfqpoint{3.406612in}{2.322880in}}%
\pgfpathcurveto{\pgfqpoint{3.406612in}{2.311830in}}{\pgfqpoint{3.411003in}{2.301231in}}{\pgfqpoint{3.418816in}{2.293417in}}%
\pgfpathcurveto{\pgfqpoint{3.426630in}{2.285604in}}{\pgfqpoint{3.437229in}{2.281213in}}{\pgfqpoint{3.448279in}{2.281213in}}%
\pgfpathclose%
\pgfusepath{stroke,fill}%
\end{pgfscope}%
\begin{pgfscope}%
\pgfpathrectangle{\pgfqpoint{0.600000in}{0.600000in}}{\pgfqpoint{3.900000in}{3.900000in}}%
\pgfusepath{clip}%
\pgfsetbuttcap%
\pgfsetroundjoin%
\definecolor{currentfill}{rgb}{0.121569,0.466667,0.705882}%
\pgfsetfillcolor{currentfill}%
\pgfsetlinewidth{1.003750pt}%
\definecolor{currentstroke}{rgb}{0.121569,0.466667,0.705882}%
\pgfsetstrokecolor{currentstroke}%
\pgfsetdash{}{0pt}%
\pgfpathmoveto{\pgfqpoint{2.405366in}{1.959449in}}%
\pgfpathcurveto{\pgfqpoint{2.416416in}{1.959449in}}{\pgfqpoint{2.427015in}{1.963839in}}{\pgfqpoint{2.434829in}{1.971653in}}%
\pgfpathcurveto{\pgfqpoint{2.442642in}{1.979466in}}{\pgfqpoint{2.447032in}{1.990065in}}{\pgfqpoint{2.447032in}{2.001115in}}%
\pgfpathcurveto{\pgfqpoint{2.447032in}{2.012166in}}{\pgfqpoint{2.442642in}{2.022765in}}{\pgfqpoint{2.434829in}{2.030578in}}%
\pgfpathcurveto{\pgfqpoint{2.427015in}{2.038392in}}{\pgfqpoint{2.416416in}{2.042782in}}{\pgfqpoint{2.405366in}{2.042782in}}%
\pgfpathcurveto{\pgfqpoint{2.394316in}{2.042782in}}{\pgfqpoint{2.383717in}{2.038392in}}{\pgfqpoint{2.375903in}{2.030578in}}%
\pgfpathcurveto{\pgfqpoint{2.368089in}{2.022765in}}{\pgfqpoint{2.363699in}{2.012166in}}{\pgfqpoint{2.363699in}{2.001115in}}%
\pgfpathcurveto{\pgfqpoint{2.363699in}{1.990065in}}{\pgfqpoint{2.368089in}{1.979466in}}{\pgfqpoint{2.375903in}{1.971653in}}%
\pgfpathcurveto{\pgfqpoint{2.383717in}{1.963839in}}{\pgfqpoint{2.394316in}{1.959449in}}{\pgfqpoint{2.405366in}{1.959449in}}%
\pgfpathclose%
\pgfusepath{stroke,fill}%
\end{pgfscope}%
\begin{pgfscope}%
\pgfpathrectangle{\pgfqpoint{0.600000in}{0.600000in}}{\pgfqpoint{3.900000in}{3.900000in}}%
\pgfusepath{clip}%
\pgfsetbuttcap%
\pgfsetroundjoin%
\definecolor{currentfill}{rgb}{0.121569,0.466667,0.705882}%
\pgfsetfillcolor{currentfill}%
\pgfsetlinewidth{1.003750pt}%
\definecolor{currentstroke}{rgb}{0.121569,0.466667,0.705882}%
\pgfsetstrokecolor{currentstroke}%
\pgfsetdash{}{0pt}%
\pgfpathmoveto{\pgfqpoint{2.892113in}{1.830647in}}%
\pgfpathcurveto{\pgfqpoint{2.903163in}{1.830647in}}{\pgfqpoint{2.913762in}{1.835037in}}{\pgfqpoint{2.921576in}{1.842851in}}%
\pgfpathcurveto{\pgfqpoint{2.929389in}{1.850665in}}{\pgfqpoint{2.933780in}{1.861264in}}{\pgfqpoint{2.933780in}{1.872314in}}%
\pgfpathcurveto{\pgfqpoint{2.933780in}{1.883364in}}{\pgfqpoint{2.929389in}{1.893963in}}{\pgfqpoint{2.921576in}{1.901776in}}%
\pgfpathcurveto{\pgfqpoint{2.913762in}{1.909590in}}{\pgfqpoint{2.903163in}{1.913980in}}{\pgfqpoint{2.892113in}{1.913980in}}%
\pgfpathcurveto{\pgfqpoint{2.881063in}{1.913980in}}{\pgfqpoint{2.870464in}{1.909590in}}{\pgfqpoint{2.862650in}{1.901776in}}%
\pgfpathcurveto{\pgfqpoint{2.854837in}{1.893963in}}{\pgfqpoint{2.850446in}{1.883364in}}{\pgfqpoint{2.850446in}{1.872314in}}%
\pgfpathcurveto{\pgfqpoint{2.850446in}{1.861264in}}{\pgfqpoint{2.854837in}{1.850665in}}{\pgfqpoint{2.862650in}{1.842851in}}%
\pgfpathcurveto{\pgfqpoint{2.870464in}{1.835037in}}{\pgfqpoint{2.881063in}{1.830647in}}{\pgfqpoint{2.892113in}{1.830647in}}%
\pgfpathclose%
\pgfusepath{stroke,fill}%
\end{pgfscope}%
\begin{pgfscope}%
\pgfpathrectangle{\pgfqpoint{0.600000in}{0.600000in}}{\pgfqpoint{3.900000in}{3.900000in}}%
\pgfusepath{clip}%
\pgfsetbuttcap%
\pgfsetroundjoin%
\definecolor{currentfill}{rgb}{0.121569,0.466667,0.705882}%
\pgfsetfillcolor{currentfill}%
\pgfsetlinewidth{1.003750pt}%
\definecolor{currentstroke}{rgb}{0.121569,0.466667,0.705882}%
\pgfsetstrokecolor{currentstroke}%
\pgfsetdash{}{0pt}%
\pgfpathmoveto{\pgfqpoint{2.640251in}{1.945706in}}%
\pgfpathcurveto{\pgfqpoint{2.651301in}{1.945706in}}{\pgfqpoint{2.661900in}{1.950097in}}{\pgfqpoint{2.669714in}{1.957910in}}%
\pgfpathcurveto{\pgfqpoint{2.677528in}{1.965724in}}{\pgfqpoint{2.681918in}{1.976323in}}{\pgfqpoint{2.681918in}{1.987373in}}%
\pgfpathcurveto{\pgfqpoint{2.681918in}{1.998423in}}{\pgfqpoint{2.677528in}{2.009022in}}{\pgfqpoint{2.669714in}{2.016836in}}%
\pgfpathcurveto{\pgfqpoint{2.661900in}{2.024649in}}{\pgfqpoint{2.651301in}{2.029040in}}{\pgfqpoint{2.640251in}{2.029040in}}%
\pgfpathcurveto{\pgfqpoint{2.629201in}{2.029040in}}{\pgfqpoint{2.618602in}{2.024649in}}{\pgfqpoint{2.610788in}{2.016836in}}%
\pgfpathcurveto{\pgfqpoint{2.602975in}{2.009022in}}{\pgfqpoint{2.598584in}{1.998423in}}{\pgfqpoint{2.598584in}{1.987373in}}%
\pgfpathcurveto{\pgfqpoint{2.598584in}{1.976323in}}{\pgfqpoint{2.602975in}{1.965724in}}{\pgfqpoint{2.610788in}{1.957910in}}%
\pgfpathcurveto{\pgfqpoint{2.618602in}{1.950097in}}{\pgfqpoint{2.629201in}{1.945706in}}{\pgfqpoint{2.640251in}{1.945706in}}%
\pgfpathclose%
\pgfusepath{stroke,fill}%
\end{pgfscope}%
\begin{pgfscope}%
\pgfpathrectangle{\pgfqpoint{0.600000in}{0.600000in}}{\pgfqpoint{3.900000in}{3.900000in}}%
\pgfusepath{clip}%
\pgfsetbuttcap%
\pgfsetroundjoin%
\definecolor{currentfill}{rgb}{0.121569,0.466667,0.705882}%
\pgfsetfillcolor{currentfill}%
\pgfsetlinewidth{1.003750pt}%
\definecolor{currentstroke}{rgb}{0.121569,0.466667,0.705882}%
\pgfsetstrokecolor{currentstroke}%
\pgfsetdash{}{0pt}%
\pgfpathmoveto{\pgfqpoint{3.774262in}{2.012242in}}%
\pgfpathcurveto{\pgfqpoint{3.785312in}{2.012242in}}{\pgfqpoint{3.795911in}{2.016633in}}{\pgfqpoint{3.803725in}{2.024446in}}%
\pgfpathcurveto{\pgfqpoint{3.811539in}{2.032260in}}{\pgfqpoint{3.815929in}{2.042859in}}{\pgfqpoint{3.815929in}{2.053909in}}%
\pgfpathcurveto{\pgfqpoint{3.815929in}{2.064959in}}{\pgfqpoint{3.811539in}{2.075558in}}{\pgfqpoint{3.803725in}{2.083372in}}%
\pgfpathcurveto{\pgfqpoint{3.795911in}{2.091185in}}{\pgfqpoint{3.785312in}{2.095576in}}{\pgfqpoint{3.774262in}{2.095576in}}%
\pgfpathcurveto{\pgfqpoint{3.763212in}{2.095576in}}{\pgfqpoint{3.752613in}{2.091185in}}{\pgfqpoint{3.744799in}{2.083372in}}%
\pgfpathcurveto{\pgfqpoint{3.736986in}{2.075558in}}{\pgfqpoint{3.732595in}{2.064959in}}{\pgfqpoint{3.732595in}{2.053909in}}%
\pgfpathcurveto{\pgfqpoint{3.732595in}{2.042859in}}{\pgfqpoint{3.736986in}{2.032260in}}{\pgfqpoint{3.744799in}{2.024446in}}%
\pgfpathcurveto{\pgfqpoint{3.752613in}{2.016633in}}{\pgfqpoint{3.763212in}{2.012242in}}{\pgfqpoint{3.774262in}{2.012242in}}%
\pgfpathclose%
\pgfusepath{stroke,fill}%
\end{pgfscope}%
\begin{pgfscope}%
\pgfpathrectangle{\pgfqpoint{0.600000in}{0.600000in}}{\pgfqpoint{3.900000in}{3.900000in}}%
\pgfusepath{clip}%
\pgfsetbuttcap%
\pgfsetroundjoin%
\definecolor{currentfill}{rgb}{0.121569,0.466667,0.705882}%
\pgfsetfillcolor{currentfill}%
\pgfsetlinewidth{1.003750pt}%
\definecolor{currentstroke}{rgb}{0.121569,0.466667,0.705882}%
\pgfsetstrokecolor{currentstroke}%
\pgfsetdash{}{0pt}%
\pgfpathmoveto{\pgfqpoint{2.202416in}{2.608906in}}%
\pgfpathcurveto{\pgfqpoint{2.213466in}{2.608906in}}{\pgfqpoint{2.224065in}{2.613296in}}{\pgfqpoint{2.231879in}{2.621110in}}%
\pgfpathcurveto{\pgfqpoint{2.239692in}{2.628924in}}{\pgfqpoint{2.244083in}{2.639523in}}{\pgfqpoint{2.244083in}{2.650573in}}%
\pgfpathcurveto{\pgfqpoint{2.244083in}{2.661623in}}{\pgfqpoint{2.239692in}{2.672222in}}{\pgfqpoint{2.231879in}{2.680036in}}%
\pgfpathcurveto{\pgfqpoint{2.224065in}{2.687849in}}{\pgfqpoint{2.213466in}{2.692239in}}{\pgfqpoint{2.202416in}{2.692239in}}%
\pgfpathcurveto{\pgfqpoint{2.191366in}{2.692239in}}{\pgfqpoint{2.180767in}{2.687849in}}{\pgfqpoint{2.172953in}{2.680036in}}%
\pgfpathcurveto{\pgfqpoint{2.165140in}{2.672222in}}{\pgfqpoint{2.160749in}{2.661623in}}{\pgfqpoint{2.160749in}{2.650573in}}%
\pgfpathcurveto{\pgfqpoint{2.160749in}{2.639523in}}{\pgfqpoint{2.165140in}{2.628924in}}{\pgfqpoint{2.172953in}{2.621110in}}%
\pgfpathcurveto{\pgfqpoint{2.180767in}{2.613296in}}{\pgfqpoint{2.191366in}{2.608906in}}{\pgfqpoint{2.202416in}{2.608906in}}%
\pgfpathclose%
\pgfusepath{stroke,fill}%
\end{pgfscope}%
\begin{pgfscope}%
\pgfpathrectangle{\pgfqpoint{0.600000in}{0.600000in}}{\pgfqpoint{3.900000in}{3.900000in}}%
\pgfusepath{clip}%
\pgfsetbuttcap%
\pgfsetroundjoin%
\definecolor{currentfill}{rgb}{0.121569,0.466667,0.705882}%
\pgfsetfillcolor{currentfill}%
\pgfsetlinewidth{1.003750pt}%
\definecolor{currentstroke}{rgb}{0.121569,0.466667,0.705882}%
\pgfsetstrokecolor{currentstroke}%
\pgfsetdash{}{0pt}%
\pgfpathmoveto{\pgfqpoint{2.692521in}{2.653265in}}%
\pgfpathcurveto{\pgfqpoint{2.703571in}{2.653265in}}{\pgfqpoint{2.714170in}{2.657656in}}{\pgfqpoint{2.721984in}{2.665469in}}%
\pgfpathcurveto{\pgfqpoint{2.729798in}{2.673283in}}{\pgfqpoint{2.734188in}{2.683882in}}{\pgfqpoint{2.734188in}{2.694932in}}%
\pgfpathcurveto{\pgfqpoint{2.734188in}{2.705982in}}{\pgfqpoint{2.729798in}{2.716581in}}{\pgfqpoint{2.721984in}{2.724395in}}%
\pgfpathcurveto{\pgfqpoint{2.714170in}{2.732208in}}{\pgfqpoint{2.703571in}{2.736599in}}{\pgfqpoint{2.692521in}{2.736599in}}%
\pgfpathcurveto{\pgfqpoint{2.681471in}{2.736599in}}{\pgfqpoint{2.670872in}{2.732208in}}{\pgfqpoint{2.663058in}{2.724395in}}%
\pgfpathcurveto{\pgfqpoint{2.655245in}{2.716581in}}{\pgfqpoint{2.650855in}{2.705982in}}{\pgfqpoint{2.650855in}{2.694932in}}%
\pgfpathcurveto{\pgfqpoint{2.650855in}{2.683882in}}{\pgfqpoint{2.655245in}{2.673283in}}{\pgfqpoint{2.663058in}{2.665469in}}%
\pgfpathcurveto{\pgfqpoint{2.670872in}{2.657656in}}{\pgfqpoint{2.681471in}{2.653265in}}{\pgfqpoint{2.692521in}{2.653265in}}%
\pgfpathclose%
\pgfusepath{stroke,fill}%
\end{pgfscope}%
\begin{pgfscope}%
\pgfpathrectangle{\pgfqpoint{0.600000in}{0.600000in}}{\pgfqpoint{3.900000in}{3.900000in}}%
\pgfusepath{clip}%
\pgfsetbuttcap%
\pgfsetroundjoin%
\definecolor{currentfill}{rgb}{0.121569,0.466667,0.705882}%
\pgfsetfillcolor{currentfill}%
\pgfsetlinewidth{1.003750pt}%
\definecolor{currentstroke}{rgb}{0.121569,0.466667,0.705882}%
\pgfsetstrokecolor{currentstroke}%
\pgfsetdash{}{0pt}%
\pgfpathmoveto{\pgfqpoint{3.514920in}{1.767671in}}%
\pgfpathcurveto{\pgfqpoint{3.525970in}{1.767671in}}{\pgfqpoint{3.536569in}{1.772061in}}{\pgfqpoint{3.544383in}{1.779875in}}%
\pgfpathcurveto{\pgfqpoint{3.552196in}{1.787689in}}{\pgfqpoint{3.556587in}{1.798288in}}{\pgfqpoint{3.556587in}{1.809338in}}%
\pgfpathcurveto{\pgfqpoint{3.556587in}{1.820388in}}{\pgfqpoint{3.552196in}{1.830987in}}{\pgfqpoint{3.544383in}{1.838801in}}%
\pgfpathcurveto{\pgfqpoint{3.536569in}{1.846614in}}{\pgfqpoint{3.525970in}{1.851005in}}{\pgfqpoint{3.514920in}{1.851005in}}%
\pgfpathcurveto{\pgfqpoint{3.503870in}{1.851005in}}{\pgfqpoint{3.493271in}{1.846614in}}{\pgfqpoint{3.485457in}{1.838801in}}%
\pgfpathcurveto{\pgfqpoint{3.477644in}{1.830987in}}{\pgfqpoint{3.473253in}{1.820388in}}{\pgfqpoint{3.473253in}{1.809338in}}%
\pgfpathcurveto{\pgfqpoint{3.473253in}{1.798288in}}{\pgfqpoint{3.477644in}{1.787689in}}{\pgfqpoint{3.485457in}{1.779875in}}%
\pgfpathcurveto{\pgfqpoint{3.493271in}{1.772061in}}{\pgfqpoint{3.503870in}{1.767671in}}{\pgfqpoint{3.514920in}{1.767671in}}%
\pgfpathclose%
\pgfusepath{stroke,fill}%
\end{pgfscope}%
\begin{pgfscope}%
\pgfpathrectangle{\pgfqpoint{0.600000in}{0.600000in}}{\pgfqpoint{3.900000in}{3.900000in}}%
\pgfusepath{clip}%
\pgfsetbuttcap%
\pgfsetroundjoin%
\definecolor{currentfill}{rgb}{0.121569,0.466667,0.705882}%
\pgfsetfillcolor{currentfill}%
\pgfsetlinewidth{1.003750pt}%
\definecolor{currentstroke}{rgb}{0.121569,0.466667,0.705882}%
\pgfsetstrokecolor{currentstroke}%
\pgfsetdash{}{0pt}%
\pgfpathmoveto{\pgfqpoint{3.225358in}{2.001437in}}%
\pgfpathcurveto{\pgfqpoint{3.236408in}{2.001437in}}{\pgfqpoint{3.247007in}{2.005827in}}{\pgfqpoint{3.254821in}{2.013641in}}%
\pgfpathcurveto{\pgfqpoint{3.262635in}{2.021454in}}{\pgfqpoint{3.267025in}{2.032053in}}{\pgfqpoint{3.267025in}{2.043103in}}%
\pgfpathcurveto{\pgfqpoint{3.267025in}{2.054154in}}{\pgfqpoint{3.262635in}{2.064753in}}{\pgfqpoint{3.254821in}{2.072566in}}%
\pgfpathcurveto{\pgfqpoint{3.247007in}{2.080380in}}{\pgfqpoint{3.236408in}{2.084770in}}{\pgfqpoint{3.225358in}{2.084770in}}%
\pgfpathcurveto{\pgfqpoint{3.214308in}{2.084770in}}{\pgfqpoint{3.203709in}{2.080380in}}{\pgfqpoint{3.195896in}{2.072566in}}%
\pgfpathcurveto{\pgfqpoint{3.188082in}{2.064753in}}{\pgfqpoint{3.183692in}{2.054154in}}{\pgfqpoint{3.183692in}{2.043103in}}%
\pgfpathcurveto{\pgfqpoint{3.183692in}{2.032053in}}{\pgfqpoint{3.188082in}{2.021454in}}{\pgfqpoint{3.195896in}{2.013641in}}%
\pgfpathcurveto{\pgfqpoint{3.203709in}{2.005827in}}{\pgfqpoint{3.214308in}{2.001437in}}{\pgfqpoint{3.225358in}{2.001437in}}%
\pgfpathclose%
\pgfusepath{stroke,fill}%
\end{pgfscope}%
\begin{pgfscope}%
\pgfpathrectangle{\pgfqpoint{0.600000in}{0.600000in}}{\pgfqpoint{3.900000in}{3.900000in}}%
\pgfusepath{clip}%
\pgfsetbuttcap%
\pgfsetroundjoin%
\definecolor{currentfill}{rgb}{0.121569,0.466667,0.705882}%
\pgfsetfillcolor{currentfill}%
\pgfsetlinewidth{1.003750pt}%
\definecolor{currentstroke}{rgb}{0.121569,0.466667,0.705882}%
\pgfsetstrokecolor{currentstroke}%
\pgfsetdash{}{0pt}%
\pgfpathmoveto{\pgfqpoint{3.263260in}{2.581378in}}%
\pgfpathcurveto{\pgfqpoint{3.274310in}{2.581378in}}{\pgfqpoint{3.284909in}{2.585769in}}{\pgfqpoint{3.292722in}{2.593582in}}%
\pgfpathcurveto{\pgfqpoint{3.300536in}{2.601396in}}{\pgfqpoint{3.304926in}{2.611995in}}{\pgfqpoint{3.304926in}{2.623045in}}%
\pgfpathcurveto{\pgfqpoint{3.304926in}{2.634095in}}{\pgfqpoint{3.300536in}{2.644694in}}{\pgfqpoint{3.292722in}{2.652508in}}%
\pgfpathcurveto{\pgfqpoint{3.284909in}{2.660321in}}{\pgfqpoint{3.274310in}{2.664712in}}{\pgfqpoint{3.263260in}{2.664712in}}%
\pgfpathcurveto{\pgfqpoint{3.252209in}{2.664712in}}{\pgfqpoint{3.241610in}{2.660321in}}{\pgfqpoint{3.233797in}{2.652508in}}%
\pgfpathcurveto{\pgfqpoint{3.225983in}{2.644694in}}{\pgfqpoint{3.221593in}{2.634095in}}{\pgfqpoint{3.221593in}{2.623045in}}%
\pgfpathcurveto{\pgfqpoint{3.221593in}{2.611995in}}{\pgfqpoint{3.225983in}{2.601396in}}{\pgfqpoint{3.233797in}{2.593582in}}%
\pgfpathcurveto{\pgfqpoint{3.241610in}{2.585769in}}{\pgfqpoint{3.252209in}{2.581378in}}{\pgfqpoint{3.263260in}{2.581378in}}%
\pgfpathclose%
\pgfusepath{stroke,fill}%
\end{pgfscope}%
\begin{pgfscope}%
\pgfpathrectangle{\pgfqpoint{0.600000in}{0.600000in}}{\pgfqpoint{3.900000in}{3.900000in}}%
\pgfusepath{clip}%
\pgfsetbuttcap%
\pgfsetroundjoin%
\definecolor{currentfill}{rgb}{0.121569,0.466667,0.705882}%
\pgfsetfillcolor{currentfill}%
\pgfsetlinewidth{1.003750pt}%
\definecolor{currentstroke}{rgb}{0.121569,0.466667,0.705882}%
\pgfsetstrokecolor{currentstroke}%
\pgfsetdash{}{0pt}%
\pgfpathmoveto{\pgfqpoint{2.127439in}{2.178835in}}%
\pgfpathcurveto{\pgfqpoint{2.138489in}{2.178835in}}{\pgfqpoint{2.149088in}{2.183226in}}{\pgfqpoint{2.156902in}{2.191039in}}%
\pgfpathcurveto{\pgfqpoint{2.164715in}{2.198853in}}{\pgfqpoint{2.169105in}{2.209452in}}{\pgfqpoint{2.169105in}{2.220502in}}%
\pgfpathcurveto{\pgfqpoint{2.169105in}{2.231552in}}{\pgfqpoint{2.164715in}{2.242151in}}{\pgfqpoint{2.156902in}{2.249965in}}%
\pgfpathcurveto{\pgfqpoint{2.149088in}{2.257778in}}{\pgfqpoint{2.138489in}{2.262169in}}{\pgfqpoint{2.127439in}{2.262169in}}%
\pgfpathcurveto{\pgfqpoint{2.116389in}{2.262169in}}{\pgfqpoint{2.105790in}{2.257778in}}{\pgfqpoint{2.097976in}{2.249965in}}%
\pgfpathcurveto{\pgfqpoint{2.090162in}{2.242151in}}{\pgfqpoint{2.085772in}{2.231552in}}{\pgfqpoint{2.085772in}{2.220502in}}%
\pgfpathcurveto{\pgfqpoint{2.085772in}{2.209452in}}{\pgfqpoint{2.090162in}{2.198853in}}{\pgfqpoint{2.097976in}{2.191039in}}%
\pgfpathcurveto{\pgfqpoint{2.105790in}{2.183226in}}{\pgfqpoint{2.116389in}{2.178835in}}{\pgfqpoint{2.127439in}{2.178835in}}%
\pgfpathclose%
\pgfusepath{stroke,fill}%
\end{pgfscope}%
\begin{pgfscope}%
\pgfpathrectangle{\pgfqpoint{0.600000in}{0.600000in}}{\pgfqpoint{3.900000in}{3.900000in}}%
\pgfusepath{clip}%
\pgfsetbuttcap%
\pgfsetroundjoin%
\definecolor{currentfill}{rgb}{0.121569,0.466667,0.705882}%
\pgfsetfillcolor{currentfill}%
\pgfsetlinewidth{1.003750pt}%
\definecolor{currentstroke}{rgb}{0.121569,0.466667,0.705882}%
\pgfsetstrokecolor{currentstroke}%
\pgfsetdash{}{0pt}%
\pgfpathmoveto{\pgfqpoint{1.381436in}{2.810705in}}%
\pgfpathcurveto{\pgfqpoint{1.392486in}{2.810705in}}{\pgfqpoint{1.403085in}{2.815095in}}{\pgfqpoint{1.410899in}{2.822909in}}%
\pgfpathcurveto{\pgfqpoint{1.418712in}{2.830722in}}{\pgfqpoint{1.423103in}{2.841321in}}{\pgfqpoint{1.423103in}{2.852371in}}%
\pgfpathcurveto{\pgfqpoint{1.423103in}{2.863422in}}{\pgfqpoint{1.418712in}{2.874021in}}{\pgfqpoint{1.410899in}{2.881834in}}%
\pgfpathcurveto{\pgfqpoint{1.403085in}{2.889648in}}{\pgfqpoint{1.392486in}{2.894038in}}{\pgfqpoint{1.381436in}{2.894038in}}%
\pgfpathcurveto{\pgfqpoint{1.370386in}{2.894038in}}{\pgfqpoint{1.359787in}{2.889648in}}{\pgfqpoint{1.351973in}{2.881834in}}%
\pgfpathcurveto{\pgfqpoint{1.344159in}{2.874021in}}{\pgfqpoint{1.339769in}{2.863422in}}{\pgfqpoint{1.339769in}{2.852371in}}%
\pgfpathcurveto{\pgfqpoint{1.339769in}{2.841321in}}{\pgfqpoint{1.344159in}{2.830722in}}{\pgfqpoint{1.351973in}{2.822909in}}%
\pgfpathcurveto{\pgfqpoint{1.359787in}{2.815095in}}{\pgfqpoint{1.370386in}{2.810705in}}{\pgfqpoint{1.381436in}{2.810705in}}%
\pgfpathclose%
\pgfusepath{stroke,fill}%
\end{pgfscope}%
\begin{pgfscope}%
\pgfpathrectangle{\pgfqpoint{0.600000in}{0.600000in}}{\pgfqpoint{3.900000in}{3.900000in}}%
\pgfusepath{clip}%
\pgfsetbuttcap%
\pgfsetroundjoin%
\definecolor{currentfill}{rgb}{0.121569,0.466667,0.705882}%
\pgfsetfillcolor{currentfill}%
\pgfsetlinewidth{1.003750pt}%
\definecolor{currentstroke}{rgb}{0.121569,0.466667,0.705882}%
\pgfsetstrokecolor{currentstroke}%
\pgfsetdash{}{0pt}%
\pgfpathmoveto{\pgfqpoint{2.224789in}{2.658560in}}%
\pgfpathcurveto{\pgfqpoint{2.235839in}{2.658560in}}{\pgfqpoint{2.246438in}{2.662950in}}{\pgfqpoint{2.254252in}{2.670763in}}%
\pgfpathcurveto{\pgfqpoint{2.262065in}{2.678577in}}{\pgfqpoint{2.266456in}{2.689176in}}{\pgfqpoint{2.266456in}{2.700226in}}%
\pgfpathcurveto{\pgfqpoint{2.266456in}{2.711276in}}{\pgfqpoint{2.262065in}{2.721875in}}{\pgfqpoint{2.254252in}{2.729689in}}%
\pgfpathcurveto{\pgfqpoint{2.246438in}{2.737503in}}{\pgfqpoint{2.235839in}{2.741893in}}{\pgfqpoint{2.224789in}{2.741893in}}%
\pgfpathcurveto{\pgfqpoint{2.213739in}{2.741893in}}{\pgfqpoint{2.203140in}{2.737503in}}{\pgfqpoint{2.195326in}{2.729689in}}%
\pgfpathcurveto{\pgfqpoint{2.187513in}{2.721875in}}{\pgfqpoint{2.183122in}{2.711276in}}{\pgfqpoint{2.183122in}{2.700226in}}%
\pgfpathcurveto{\pgfqpoint{2.183122in}{2.689176in}}{\pgfqpoint{2.187513in}{2.678577in}}{\pgfqpoint{2.195326in}{2.670763in}}%
\pgfpathcurveto{\pgfqpoint{2.203140in}{2.662950in}}{\pgfqpoint{2.213739in}{2.658560in}}{\pgfqpoint{2.224789in}{2.658560in}}%
\pgfpathclose%
\pgfusepath{stroke,fill}%
\end{pgfscope}%
\begin{pgfscope}%
\pgfpathrectangle{\pgfqpoint{0.600000in}{0.600000in}}{\pgfqpoint{3.900000in}{3.900000in}}%
\pgfusepath{clip}%
\pgfsetbuttcap%
\pgfsetroundjoin%
\definecolor{currentfill}{rgb}{0.121569,0.466667,0.705882}%
\pgfsetfillcolor{currentfill}%
\pgfsetlinewidth{1.003750pt}%
\definecolor{currentstroke}{rgb}{0.121569,0.466667,0.705882}%
\pgfsetstrokecolor{currentstroke}%
\pgfsetdash{}{0pt}%
\pgfpathmoveto{\pgfqpoint{2.830241in}{3.477801in}}%
\pgfpathcurveto{\pgfqpoint{2.841291in}{3.477801in}}{\pgfqpoint{2.851890in}{3.482192in}}{\pgfqpoint{2.859703in}{3.490005in}}%
\pgfpathcurveto{\pgfqpoint{2.867517in}{3.497819in}}{\pgfqpoint{2.871907in}{3.508418in}}{\pgfqpoint{2.871907in}{3.519468in}}%
\pgfpathcurveto{\pgfqpoint{2.871907in}{3.530518in}}{\pgfqpoint{2.867517in}{3.541117in}}{\pgfqpoint{2.859703in}{3.548931in}}%
\pgfpathcurveto{\pgfqpoint{2.851890in}{3.556744in}}{\pgfqpoint{2.841291in}{3.561135in}}{\pgfqpoint{2.830241in}{3.561135in}}%
\pgfpathcurveto{\pgfqpoint{2.819190in}{3.561135in}}{\pgfqpoint{2.808591in}{3.556744in}}{\pgfqpoint{2.800778in}{3.548931in}}%
\pgfpathcurveto{\pgfqpoint{2.792964in}{3.541117in}}{\pgfqpoint{2.788574in}{3.530518in}}{\pgfqpoint{2.788574in}{3.519468in}}%
\pgfpathcurveto{\pgfqpoint{2.788574in}{3.508418in}}{\pgfqpoint{2.792964in}{3.497819in}}{\pgfqpoint{2.800778in}{3.490005in}}%
\pgfpathcurveto{\pgfqpoint{2.808591in}{3.482192in}}{\pgfqpoint{2.819190in}{3.477801in}}{\pgfqpoint{2.830241in}{3.477801in}}%
\pgfpathclose%
\pgfusepath{stroke,fill}%
\end{pgfscope}%
\begin{pgfscope}%
\pgfpathrectangle{\pgfqpoint{0.600000in}{0.600000in}}{\pgfqpoint{3.900000in}{3.900000in}}%
\pgfusepath{clip}%
\pgfsetbuttcap%
\pgfsetroundjoin%
\definecolor{currentfill}{rgb}{0.121569,0.466667,0.705882}%
\pgfsetfillcolor{currentfill}%
\pgfsetlinewidth{1.003750pt}%
\definecolor{currentstroke}{rgb}{0.121569,0.466667,0.705882}%
\pgfsetstrokecolor{currentstroke}%
\pgfsetdash{}{0pt}%
\pgfpathmoveto{\pgfqpoint{2.532800in}{2.295351in}}%
\pgfpathcurveto{\pgfqpoint{2.543850in}{2.295351in}}{\pgfqpoint{2.554449in}{2.299741in}}{\pgfqpoint{2.562263in}{2.307554in}}%
\pgfpathcurveto{\pgfqpoint{2.570076in}{2.315368in}}{\pgfqpoint{2.574466in}{2.325967in}}{\pgfqpoint{2.574466in}{2.337017in}}%
\pgfpathcurveto{\pgfqpoint{2.574466in}{2.348067in}}{\pgfqpoint{2.570076in}{2.358666in}}{\pgfqpoint{2.562263in}{2.366480in}}%
\pgfpathcurveto{\pgfqpoint{2.554449in}{2.374294in}}{\pgfqpoint{2.543850in}{2.378684in}}{\pgfqpoint{2.532800in}{2.378684in}}%
\pgfpathcurveto{\pgfqpoint{2.521750in}{2.378684in}}{\pgfqpoint{2.511151in}{2.374294in}}{\pgfqpoint{2.503337in}{2.366480in}}%
\pgfpathcurveto{\pgfqpoint{2.495523in}{2.358666in}}{\pgfqpoint{2.491133in}{2.348067in}}{\pgfqpoint{2.491133in}{2.337017in}}%
\pgfpathcurveto{\pgfqpoint{2.491133in}{2.325967in}}{\pgfqpoint{2.495523in}{2.315368in}}{\pgfqpoint{2.503337in}{2.307554in}}%
\pgfpathcurveto{\pgfqpoint{2.511151in}{2.299741in}}{\pgfqpoint{2.521750in}{2.295351in}}{\pgfqpoint{2.532800in}{2.295351in}}%
\pgfpathclose%
\pgfusepath{stroke,fill}%
\end{pgfscope}%
\begin{pgfscope}%
\pgfpathrectangle{\pgfqpoint{0.600000in}{0.600000in}}{\pgfqpoint{3.900000in}{3.900000in}}%
\pgfusepath{clip}%
\pgfsetbuttcap%
\pgfsetroundjoin%
\definecolor{currentfill}{rgb}{0.121569,0.466667,0.705882}%
\pgfsetfillcolor{currentfill}%
\pgfsetlinewidth{1.003750pt}%
\definecolor{currentstroke}{rgb}{0.121569,0.466667,0.705882}%
\pgfsetstrokecolor{currentstroke}%
\pgfsetdash{}{0pt}%
\pgfpathmoveto{\pgfqpoint{2.326093in}{3.330038in}}%
\pgfpathcurveto{\pgfqpoint{2.337144in}{3.330038in}}{\pgfqpoint{2.347743in}{3.334428in}}{\pgfqpoint{2.355556in}{3.342242in}}%
\pgfpathcurveto{\pgfqpoint{2.363370in}{3.350055in}}{\pgfqpoint{2.367760in}{3.360654in}}{\pgfqpoint{2.367760in}{3.371704in}}%
\pgfpathcurveto{\pgfqpoint{2.367760in}{3.382754in}}{\pgfqpoint{2.363370in}{3.393353in}}{\pgfqpoint{2.355556in}{3.401167in}}%
\pgfpathcurveto{\pgfqpoint{2.347743in}{3.408981in}}{\pgfqpoint{2.337144in}{3.413371in}}{\pgfqpoint{2.326093in}{3.413371in}}%
\pgfpathcurveto{\pgfqpoint{2.315043in}{3.413371in}}{\pgfqpoint{2.304444in}{3.408981in}}{\pgfqpoint{2.296631in}{3.401167in}}%
\pgfpathcurveto{\pgfqpoint{2.288817in}{3.393353in}}{\pgfqpoint{2.284427in}{3.382754in}}{\pgfqpoint{2.284427in}{3.371704in}}%
\pgfpathcurveto{\pgfqpoint{2.284427in}{3.360654in}}{\pgfqpoint{2.288817in}{3.350055in}}{\pgfqpoint{2.296631in}{3.342242in}}%
\pgfpathcurveto{\pgfqpoint{2.304444in}{3.334428in}}{\pgfqpoint{2.315043in}{3.330038in}}{\pgfqpoint{2.326093in}{3.330038in}}%
\pgfpathclose%
\pgfusepath{stroke,fill}%
\end{pgfscope}%
\begin{pgfscope}%
\pgfpathrectangle{\pgfqpoint{0.600000in}{0.600000in}}{\pgfqpoint{3.900000in}{3.900000in}}%
\pgfusepath{clip}%
\pgfsetbuttcap%
\pgfsetroundjoin%
\definecolor{currentfill}{rgb}{0.121569,0.466667,0.705882}%
\pgfsetfillcolor{currentfill}%
\pgfsetlinewidth{1.003750pt}%
\definecolor{currentstroke}{rgb}{0.121569,0.466667,0.705882}%
\pgfsetstrokecolor{currentstroke}%
\pgfsetdash{}{0pt}%
\pgfpathmoveto{\pgfqpoint{3.511239in}{2.271927in}}%
\pgfpathcurveto{\pgfqpoint{3.522289in}{2.271927in}}{\pgfqpoint{3.532888in}{2.276317in}}{\pgfqpoint{3.540702in}{2.284131in}}%
\pgfpathcurveto{\pgfqpoint{3.548515in}{2.291944in}}{\pgfqpoint{3.552906in}{2.302543in}}{\pgfqpoint{3.552906in}{2.313593in}}%
\pgfpathcurveto{\pgfqpoint{3.552906in}{2.324643in}}{\pgfqpoint{3.548515in}{2.335243in}}{\pgfqpoint{3.540702in}{2.343056in}}%
\pgfpathcurveto{\pgfqpoint{3.532888in}{2.350870in}}{\pgfqpoint{3.522289in}{2.355260in}}{\pgfqpoint{3.511239in}{2.355260in}}%
\pgfpathcurveto{\pgfqpoint{3.500189in}{2.355260in}}{\pgfqpoint{3.489590in}{2.350870in}}{\pgfqpoint{3.481776in}{2.343056in}}%
\pgfpathcurveto{\pgfqpoint{3.473963in}{2.335243in}}{\pgfqpoint{3.469572in}{2.324643in}}{\pgfqpoint{3.469572in}{2.313593in}}%
\pgfpathcurveto{\pgfqpoint{3.469572in}{2.302543in}}{\pgfqpoint{3.473963in}{2.291944in}}{\pgfqpoint{3.481776in}{2.284131in}}%
\pgfpathcurveto{\pgfqpoint{3.489590in}{2.276317in}}{\pgfqpoint{3.500189in}{2.271927in}}{\pgfqpoint{3.511239in}{2.271927in}}%
\pgfpathclose%
\pgfusepath{stroke,fill}%
\end{pgfscope}%
\begin{pgfscope}%
\pgfpathrectangle{\pgfqpoint{0.600000in}{0.600000in}}{\pgfqpoint{3.900000in}{3.900000in}}%
\pgfusepath{clip}%
\pgfsetbuttcap%
\pgfsetroundjoin%
\definecolor{currentfill}{rgb}{0.121569,0.466667,0.705882}%
\pgfsetfillcolor{currentfill}%
\pgfsetlinewidth{1.003750pt}%
\definecolor{currentstroke}{rgb}{0.121569,0.466667,0.705882}%
\pgfsetstrokecolor{currentstroke}%
\pgfsetdash{}{0pt}%
\pgfpathmoveto{\pgfqpoint{2.676947in}{2.778642in}}%
\pgfpathcurveto{\pgfqpoint{2.687997in}{2.778642in}}{\pgfqpoint{2.698596in}{2.783032in}}{\pgfqpoint{2.706410in}{2.790846in}}%
\pgfpathcurveto{\pgfqpoint{2.714223in}{2.798659in}}{\pgfqpoint{2.718613in}{2.809259in}}{\pgfqpoint{2.718613in}{2.820309in}}%
\pgfpathcurveto{\pgfqpoint{2.718613in}{2.831359in}}{\pgfqpoint{2.714223in}{2.841958in}}{\pgfqpoint{2.706410in}{2.849771in}}%
\pgfpathcurveto{\pgfqpoint{2.698596in}{2.857585in}}{\pgfqpoint{2.687997in}{2.861975in}}{\pgfqpoint{2.676947in}{2.861975in}}%
\pgfpathcurveto{\pgfqpoint{2.665897in}{2.861975in}}{\pgfqpoint{2.655298in}{2.857585in}}{\pgfqpoint{2.647484in}{2.849771in}}%
\pgfpathcurveto{\pgfqpoint{2.639670in}{2.841958in}}{\pgfqpoint{2.635280in}{2.831359in}}{\pgfqpoint{2.635280in}{2.820309in}}%
\pgfpathcurveto{\pgfqpoint{2.635280in}{2.809259in}}{\pgfqpoint{2.639670in}{2.798659in}}{\pgfqpoint{2.647484in}{2.790846in}}%
\pgfpathcurveto{\pgfqpoint{2.655298in}{2.783032in}}{\pgfqpoint{2.665897in}{2.778642in}}{\pgfqpoint{2.676947in}{2.778642in}}%
\pgfpathclose%
\pgfusepath{stroke,fill}%
\end{pgfscope}%
\begin{pgfscope}%
\pgfpathrectangle{\pgfqpoint{0.600000in}{0.600000in}}{\pgfqpoint{3.900000in}{3.900000in}}%
\pgfusepath{clip}%
\pgfsetbuttcap%
\pgfsetroundjoin%
\definecolor{currentfill}{rgb}{0.121569,0.466667,0.705882}%
\pgfsetfillcolor{currentfill}%
\pgfsetlinewidth{1.003750pt}%
\definecolor{currentstroke}{rgb}{0.121569,0.466667,0.705882}%
\pgfsetstrokecolor{currentstroke}%
\pgfsetdash{}{0pt}%
\pgfpathmoveto{\pgfqpoint{2.347189in}{2.913945in}}%
\pgfpathcurveto{\pgfqpoint{2.358239in}{2.913945in}}{\pgfqpoint{2.368838in}{2.918335in}}{\pgfqpoint{2.376651in}{2.926149in}}%
\pgfpathcurveto{\pgfqpoint{2.384465in}{2.933962in}}{\pgfqpoint{2.388855in}{2.944561in}}{\pgfqpoint{2.388855in}{2.955611in}}%
\pgfpathcurveto{\pgfqpoint{2.388855in}{2.966662in}}{\pgfqpoint{2.384465in}{2.977261in}}{\pgfqpoint{2.376651in}{2.985074in}}%
\pgfpathcurveto{\pgfqpoint{2.368838in}{2.992888in}}{\pgfqpoint{2.358239in}{2.997278in}}{\pgfqpoint{2.347189in}{2.997278in}}%
\pgfpathcurveto{\pgfqpoint{2.336138in}{2.997278in}}{\pgfqpoint{2.325539in}{2.992888in}}{\pgfqpoint{2.317726in}{2.985074in}}%
\pgfpathcurveto{\pgfqpoint{2.309912in}{2.977261in}}{\pgfqpoint{2.305522in}{2.966662in}}{\pgfqpoint{2.305522in}{2.955611in}}%
\pgfpathcurveto{\pgfqpoint{2.305522in}{2.944561in}}{\pgfqpoint{2.309912in}{2.933962in}}{\pgfqpoint{2.317726in}{2.926149in}}%
\pgfpathcurveto{\pgfqpoint{2.325539in}{2.918335in}}{\pgfqpoint{2.336138in}{2.913945in}}{\pgfqpoint{2.347189in}{2.913945in}}%
\pgfpathclose%
\pgfusepath{stroke,fill}%
\end{pgfscope}%
\begin{pgfscope}%
\pgfpathrectangle{\pgfqpoint{0.600000in}{0.600000in}}{\pgfqpoint{3.900000in}{3.900000in}}%
\pgfusepath{clip}%
\pgfsetbuttcap%
\pgfsetroundjoin%
\definecolor{currentfill}{rgb}{0.121569,0.466667,0.705882}%
\pgfsetfillcolor{currentfill}%
\pgfsetlinewidth{1.003750pt}%
\definecolor{currentstroke}{rgb}{0.121569,0.466667,0.705882}%
\pgfsetstrokecolor{currentstroke}%
\pgfsetdash{}{0pt}%
\pgfpathmoveto{\pgfqpoint{2.812273in}{1.572372in}}%
\pgfpathcurveto{\pgfqpoint{2.823323in}{1.572372in}}{\pgfqpoint{2.833922in}{1.576763in}}{\pgfqpoint{2.841736in}{1.584576in}}%
\pgfpathcurveto{\pgfqpoint{2.849550in}{1.592390in}}{\pgfqpoint{2.853940in}{1.602989in}}{\pgfqpoint{2.853940in}{1.614039in}}%
\pgfpathcurveto{\pgfqpoint{2.853940in}{1.625089in}}{\pgfqpoint{2.849550in}{1.635688in}}{\pgfqpoint{2.841736in}{1.643502in}}%
\pgfpathcurveto{\pgfqpoint{2.833922in}{1.651315in}}{\pgfqpoint{2.823323in}{1.655706in}}{\pgfqpoint{2.812273in}{1.655706in}}%
\pgfpathcurveto{\pgfqpoint{2.801223in}{1.655706in}}{\pgfqpoint{2.790624in}{1.651315in}}{\pgfqpoint{2.782811in}{1.643502in}}%
\pgfpathcurveto{\pgfqpoint{2.774997in}{1.635688in}}{\pgfqpoint{2.770607in}{1.625089in}}{\pgfqpoint{2.770607in}{1.614039in}}%
\pgfpathcurveto{\pgfqpoint{2.770607in}{1.602989in}}{\pgfqpoint{2.774997in}{1.592390in}}{\pgfqpoint{2.782811in}{1.584576in}}%
\pgfpathcurveto{\pgfqpoint{2.790624in}{1.576763in}}{\pgfqpoint{2.801223in}{1.572372in}}{\pgfqpoint{2.812273in}{1.572372in}}%
\pgfpathclose%
\pgfusepath{stroke,fill}%
\end{pgfscope}%
\begin{pgfscope}%
\pgfpathrectangle{\pgfqpoint{0.600000in}{0.600000in}}{\pgfqpoint{3.900000in}{3.900000in}}%
\pgfusepath{clip}%
\pgfsetbuttcap%
\pgfsetroundjoin%
\definecolor{currentfill}{rgb}{0.121569,0.466667,0.705882}%
\pgfsetfillcolor{currentfill}%
\pgfsetlinewidth{1.003750pt}%
\definecolor{currentstroke}{rgb}{0.121569,0.466667,0.705882}%
\pgfsetstrokecolor{currentstroke}%
\pgfsetdash{}{0pt}%
\pgfpathmoveto{\pgfqpoint{2.357658in}{2.097252in}}%
\pgfpathcurveto{\pgfqpoint{2.368708in}{2.097252in}}{\pgfqpoint{2.379307in}{2.101642in}}{\pgfqpoint{2.387121in}{2.109456in}}%
\pgfpathcurveto{\pgfqpoint{2.394935in}{2.117270in}}{\pgfqpoint{2.399325in}{2.127869in}}{\pgfqpoint{2.399325in}{2.138919in}}%
\pgfpathcurveto{\pgfqpoint{2.399325in}{2.149969in}}{\pgfqpoint{2.394935in}{2.160568in}}{\pgfqpoint{2.387121in}{2.168382in}}%
\pgfpathcurveto{\pgfqpoint{2.379307in}{2.176195in}}{\pgfqpoint{2.368708in}{2.180586in}}{\pgfqpoint{2.357658in}{2.180586in}}%
\pgfpathcurveto{\pgfqpoint{2.346608in}{2.180586in}}{\pgfqpoint{2.336009in}{2.176195in}}{\pgfqpoint{2.328195in}{2.168382in}}%
\pgfpathcurveto{\pgfqpoint{2.320382in}{2.160568in}}{\pgfqpoint{2.315992in}{2.149969in}}{\pgfqpoint{2.315992in}{2.138919in}}%
\pgfpathcurveto{\pgfqpoint{2.315992in}{2.127869in}}{\pgfqpoint{2.320382in}{2.117270in}}{\pgfqpoint{2.328195in}{2.109456in}}%
\pgfpathcurveto{\pgfqpoint{2.336009in}{2.101642in}}{\pgfqpoint{2.346608in}{2.097252in}}{\pgfqpoint{2.357658in}{2.097252in}}%
\pgfpathclose%
\pgfusepath{stroke,fill}%
\end{pgfscope}%
\begin{pgfscope}%
\pgfpathrectangle{\pgfqpoint{0.600000in}{0.600000in}}{\pgfqpoint{3.900000in}{3.900000in}}%
\pgfusepath{clip}%
\pgfsetbuttcap%
\pgfsetroundjoin%
\definecolor{currentfill}{rgb}{0.121569,0.466667,0.705882}%
\pgfsetfillcolor{currentfill}%
\pgfsetlinewidth{1.003750pt}%
\definecolor{currentstroke}{rgb}{0.121569,0.466667,0.705882}%
\pgfsetstrokecolor{currentstroke}%
\pgfsetdash{}{0pt}%
\pgfpathmoveto{\pgfqpoint{2.656363in}{3.357919in}}%
\pgfpathcurveto{\pgfqpoint{2.667413in}{3.357919in}}{\pgfqpoint{2.678012in}{3.362309in}}{\pgfqpoint{2.685826in}{3.370123in}}%
\pgfpathcurveto{\pgfqpoint{2.693639in}{3.377936in}}{\pgfqpoint{2.698030in}{3.388535in}}{\pgfqpoint{2.698030in}{3.399585in}}%
\pgfpathcurveto{\pgfqpoint{2.698030in}{3.410635in}}{\pgfqpoint{2.693639in}{3.421234in}}{\pgfqpoint{2.685826in}{3.429048in}}%
\pgfpathcurveto{\pgfqpoint{2.678012in}{3.436862in}}{\pgfqpoint{2.667413in}{3.441252in}}{\pgfqpoint{2.656363in}{3.441252in}}%
\pgfpathcurveto{\pgfqpoint{2.645313in}{3.441252in}}{\pgfqpoint{2.634714in}{3.436862in}}{\pgfqpoint{2.626900in}{3.429048in}}%
\pgfpathcurveto{\pgfqpoint{2.619087in}{3.421234in}}{\pgfqpoint{2.614696in}{3.410635in}}{\pgfqpoint{2.614696in}{3.399585in}}%
\pgfpathcurveto{\pgfqpoint{2.614696in}{3.388535in}}{\pgfqpoint{2.619087in}{3.377936in}}{\pgfqpoint{2.626900in}{3.370123in}}%
\pgfpathcurveto{\pgfqpoint{2.634714in}{3.362309in}}{\pgfqpoint{2.645313in}{3.357919in}}{\pgfqpoint{2.656363in}{3.357919in}}%
\pgfpathclose%
\pgfusepath{stroke,fill}%
\end{pgfscope}%
\begin{pgfscope}%
\pgfpathrectangle{\pgfqpoint{0.600000in}{0.600000in}}{\pgfqpoint{3.900000in}{3.900000in}}%
\pgfusepath{clip}%
\pgfsetbuttcap%
\pgfsetroundjoin%
\definecolor{currentfill}{rgb}{0.121569,0.466667,0.705882}%
\pgfsetfillcolor{currentfill}%
\pgfsetlinewidth{1.003750pt}%
\definecolor{currentstroke}{rgb}{0.121569,0.466667,0.705882}%
\pgfsetstrokecolor{currentstroke}%
\pgfsetdash{}{0pt}%
\pgfpathmoveto{\pgfqpoint{2.384848in}{2.925201in}}%
\pgfpathcurveto{\pgfqpoint{2.395898in}{2.925201in}}{\pgfqpoint{2.406497in}{2.929591in}}{\pgfqpoint{2.414310in}{2.937405in}}%
\pgfpathcurveto{\pgfqpoint{2.422124in}{2.945218in}}{\pgfqpoint{2.426514in}{2.955817in}}{\pgfqpoint{2.426514in}{2.966867in}}%
\pgfpathcurveto{\pgfqpoint{2.426514in}{2.977918in}}{\pgfqpoint{2.422124in}{2.988517in}}{\pgfqpoint{2.414310in}{2.996330in}}%
\pgfpathcurveto{\pgfqpoint{2.406497in}{3.004144in}}{\pgfqpoint{2.395898in}{3.008534in}}{\pgfqpoint{2.384848in}{3.008534in}}%
\pgfpathcurveto{\pgfqpoint{2.373797in}{3.008534in}}{\pgfqpoint{2.363198in}{3.004144in}}{\pgfqpoint{2.355385in}{2.996330in}}%
\pgfpathcurveto{\pgfqpoint{2.347571in}{2.988517in}}{\pgfqpoint{2.343181in}{2.977918in}}{\pgfqpoint{2.343181in}{2.966867in}}%
\pgfpathcurveto{\pgfqpoint{2.343181in}{2.955817in}}{\pgfqpoint{2.347571in}{2.945218in}}{\pgfqpoint{2.355385in}{2.937405in}}%
\pgfpathcurveto{\pgfqpoint{2.363198in}{2.929591in}}{\pgfqpoint{2.373797in}{2.925201in}}{\pgfqpoint{2.384848in}{2.925201in}}%
\pgfpathclose%
\pgfusepath{stroke,fill}%
\end{pgfscope}%
\begin{pgfscope}%
\pgfpathrectangle{\pgfqpoint{0.600000in}{0.600000in}}{\pgfqpoint{3.900000in}{3.900000in}}%
\pgfusepath{clip}%
\pgfsetbuttcap%
\pgfsetroundjoin%
\definecolor{currentfill}{rgb}{0.121569,0.466667,0.705882}%
\pgfsetfillcolor{currentfill}%
\pgfsetlinewidth{1.003750pt}%
\definecolor{currentstroke}{rgb}{0.121569,0.466667,0.705882}%
\pgfsetstrokecolor{currentstroke}%
\pgfsetdash{}{0pt}%
\pgfpathmoveto{\pgfqpoint{2.968526in}{1.912416in}}%
\pgfpathcurveto{\pgfqpoint{2.979576in}{1.912416in}}{\pgfqpoint{2.990175in}{1.916806in}}{\pgfqpoint{2.997989in}{1.924620in}}%
\pgfpathcurveto{\pgfqpoint{3.005802in}{1.932433in}}{\pgfqpoint{3.010193in}{1.943032in}}{\pgfqpoint{3.010193in}{1.954082in}}%
\pgfpathcurveto{\pgfqpoint{3.010193in}{1.965132in}}{\pgfqpoint{3.005802in}{1.975731in}}{\pgfqpoint{2.997989in}{1.983545in}}%
\pgfpathcurveto{\pgfqpoint{2.990175in}{1.991359in}}{\pgfqpoint{2.979576in}{1.995749in}}{\pgfqpoint{2.968526in}{1.995749in}}%
\pgfpathcurveto{\pgfqpoint{2.957476in}{1.995749in}}{\pgfqpoint{2.946877in}{1.991359in}}{\pgfqpoint{2.939063in}{1.983545in}}%
\pgfpathcurveto{\pgfqpoint{2.931250in}{1.975731in}}{\pgfqpoint{2.926859in}{1.965132in}}{\pgfqpoint{2.926859in}{1.954082in}}%
\pgfpathcurveto{\pgfqpoint{2.926859in}{1.943032in}}{\pgfqpoint{2.931250in}{1.932433in}}{\pgfqpoint{2.939063in}{1.924620in}}%
\pgfpathcurveto{\pgfqpoint{2.946877in}{1.916806in}}{\pgfqpoint{2.957476in}{1.912416in}}{\pgfqpoint{2.968526in}{1.912416in}}%
\pgfpathclose%
\pgfusepath{stroke,fill}%
\end{pgfscope}%
\begin{pgfscope}%
\pgfpathrectangle{\pgfqpoint{0.600000in}{0.600000in}}{\pgfqpoint{3.900000in}{3.900000in}}%
\pgfusepath{clip}%
\pgfsetbuttcap%
\pgfsetroundjoin%
\definecolor{currentfill}{rgb}{0.121569,0.466667,0.705882}%
\pgfsetfillcolor{currentfill}%
\pgfsetlinewidth{1.003750pt}%
\definecolor{currentstroke}{rgb}{0.121569,0.466667,0.705882}%
\pgfsetstrokecolor{currentstroke}%
\pgfsetdash{}{0pt}%
\pgfpathmoveto{\pgfqpoint{2.559638in}{2.230936in}}%
\pgfpathcurveto{\pgfqpoint{2.570688in}{2.230936in}}{\pgfqpoint{2.581287in}{2.235326in}}{\pgfqpoint{2.589101in}{2.243139in}}%
\pgfpathcurveto{\pgfqpoint{2.596914in}{2.250953in}}{\pgfqpoint{2.601305in}{2.261552in}}{\pgfqpoint{2.601305in}{2.272602in}}%
\pgfpathcurveto{\pgfqpoint{2.601305in}{2.283652in}}{\pgfqpoint{2.596914in}{2.294251in}}{\pgfqpoint{2.589101in}{2.302065in}}%
\pgfpathcurveto{\pgfqpoint{2.581287in}{2.309879in}}{\pgfqpoint{2.570688in}{2.314269in}}{\pgfqpoint{2.559638in}{2.314269in}}%
\pgfpathcurveto{\pgfqpoint{2.548588in}{2.314269in}}{\pgfqpoint{2.537989in}{2.309879in}}{\pgfqpoint{2.530175in}{2.302065in}}%
\pgfpathcurveto{\pgfqpoint{2.522361in}{2.294251in}}{\pgfqpoint{2.517971in}{2.283652in}}{\pgfqpoint{2.517971in}{2.272602in}}%
\pgfpathcurveto{\pgfqpoint{2.517971in}{2.261552in}}{\pgfqpoint{2.522361in}{2.250953in}}{\pgfqpoint{2.530175in}{2.243139in}}%
\pgfpathcurveto{\pgfqpoint{2.537989in}{2.235326in}}{\pgfqpoint{2.548588in}{2.230936in}}{\pgfqpoint{2.559638in}{2.230936in}}%
\pgfpathclose%
\pgfusepath{stroke,fill}%
\end{pgfscope}%
\begin{pgfscope}%
\pgfpathrectangle{\pgfqpoint{0.600000in}{0.600000in}}{\pgfqpoint{3.900000in}{3.900000in}}%
\pgfusepath{clip}%
\pgfsetbuttcap%
\pgfsetroundjoin%
\definecolor{currentfill}{rgb}{0.121569,0.466667,0.705882}%
\pgfsetfillcolor{currentfill}%
\pgfsetlinewidth{1.003750pt}%
\definecolor{currentstroke}{rgb}{0.121569,0.466667,0.705882}%
\pgfsetstrokecolor{currentstroke}%
\pgfsetdash{}{0pt}%
\pgfpathmoveto{\pgfqpoint{2.721873in}{3.552263in}}%
\pgfpathcurveto{\pgfqpoint{2.732923in}{3.552263in}}{\pgfqpoint{2.743522in}{3.556654in}}{\pgfqpoint{2.751336in}{3.564467in}}%
\pgfpathcurveto{\pgfqpoint{2.759149in}{3.572281in}}{\pgfqpoint{2.763540in}{3.582880in}}{\pgfqpoint{2.763540in}{3.593930in}}%
\pgfpathcurveto{\pgfqpoint{2.763540in}{3.604980in}}{\pgfqpoint{2.759149in}{3.615579in}}{\pgfqpoint{2.751336in}{3.623393in}}%
\pgfpathcurveto{\pgfqpoint{2.743522in}{3.631206in}}{\pgfqpoint{2.732923in}{3.635597in}}{\pgfqpoint{2.721873in}{3.635597in}}%
\pgfpathcurveto{\pgfqpoint{2.710823in}{3.635597in}}{\pgfqpoint{2.700224in}{3.631206in}}{\pgfqpoint{2.692410in}{3.623393in}}%
\pgfpathcurveto{\pgfqpoint{2.684597in}{3.615579in}}{\pgfqpoint{2.680206in}{3.604980in}}{\pgfqpoint{2.680206in}{3.593930in}}%
\pgfpathcurveto{\pgfqpoint{2.680206in}{3.582880in}}{\pgfqpoint{2.684597in}{3.572281in}}{\pgfqpoint{2.692410in}{3.564467in}}%
\pgfpathcurveto{\pgfqpoint{2.700224in}{3.556654in}}{\pgfqpoint{2.710823in}{3.552263in}}{\pgfqpoint{2.721873in}{3.552263in}}%
\pgfpathclose%
\pgfusepath{stroke,fill}%
\end{pgfscope}%
\begin{pgfscope}%
\pgfpathrectangle{\pgfqpoint{0.600000in}{0.600000in}}{\pgfqpoint{3.900000in}{3.900000in}}%
\pgfusepath{clip}%
\pgfsetbuttcap%
\pgfsetroundjoin%
\definecolor{currentfill}{rgb}{0.121569,0.466667,0.705882}%
\pgfsetfillcolor{currentfill}%
\pgfsetlinewidth{1.003750pt}%
\definecolor{currentstroke}{rgb}{0.121569,0.466667,0.705882}%
\pgfsetstrokecolor{currentstroke}%
\pgfsetdash{}{0pt}%
\pgfpathmoveto{\pgfqpoint{2.657658in}{2.537435in}}%
\pgfpathcurveto{\pgfqpoint{2.668708in}{2.537435in}}{\pgfqpoint{2.679307in}{2.541825in}}{\pgfqpoint{2.687121in}{2.549639in}}%
\pgfpathcurveto{\pgfqpoint{2.694934in}{2.557452in}}{\pgfqpoint{2.699324in}{2.568051in}}{\pgfqpoint{2.699324in}{2.579101in}}%
\pgfpathcurveto{\pgfqpoint{2.699324in}{2.590151in}}{\pgfqpoint{2.694934in}{2.600751in}}{\pgfqpoint{2.687121in}{2.608564in}}%
\pgfpathcurveto{\pgfqpoint{2.679307in}{2.616378in}}{\pgfqpoint{2.668708in}{2.620768in}}{\pgfqpoint{2.657658in}{2.620768in}}%
\pgfpathcurveto{\pgfqpoint{2.646608in}{2.620768in}}{\pgfqpoint{2.636009in}{2.616378in}}{\pgfqpoint{2.628195in}{2.608564in}}%
\pgfpathcurveto{\pgfqpoint{2.620381in}{2.600751in}}{\pgfqpoint{2.615991in}{2.590151in}}{\pgfqpoint{2.615991in}{2.579101in}}%
\pgfpathcurveto{\pgfqpoint{2.615991in}{2.568051in}}{\pgfqpoint{2.620381in}{2.557452in}}{\pgfqpoint{2.628195in}{2.549639in}}%
\pgfpathcurveto{\pgfqpoint{2.636009in}{2.541825in}}{\pgfqpoint{2.646608in}{2.537435in}}{\pgfqpoint{2.657658in}{2.537435in}}%
\pgfpathclose%
\pgfusepath{stroke,fill}%
\end{pgfscope}%
\begin{pgfscope}%
\pgfpathrectangle{\pgfqpoint{0.600000in}{0.600000in}}{\pgfqpoint{3.900000in}{3.900000in}}%
\pgfusepath{clip}%
\pgfsetbuttcap%
\pgfsetroundjoin%
\definecolor{currentfill}{rgb}{0.121569,0.466667,0.705882}%
\pgfsetfillcolor{currentfill}%
\pgfsetlinewidth{1.003750pt}%
\definecolor{currentstroke}{rgb}{0.121569,0.466667,0.705882}%
\pgfsetstrokecolor{currentstroke}%
\pgfsetdash{}{0pt}%
\pgfpathmoveto{\pgfqpoint{2.034868in}{2.444532in}}%
\pgfpathcurveto{\pgfqpoint{2.045918in}{2.444532in}}{\pgfqpoint{2.056517in}{2.448923in}}{\pgfqpoint{2.064331in}{2.456736in}}%
\pgfpathcurveto{\pgfqpoint{2.072145in}{2.464550in}}{\pgfqpoint{2.076535in}{2.475149in}}{\pgfqpoint{2.076535in}{2.486199in}}%
\pgfpathcurveto{\pgfqpoint{2.076535in}{2.497249in}}{\pgfqpoint{2.072145in}{2.507848in}}{\pgfqpoint{2.064331in}{2.515662in}}%
\pgfpathcurveto{\pgfqpoint{2.056517in}{2.523475in}}{\pgfqpoint{2.045918in}{2.527866in}}{\pgfqpoint{2.034868in}{2.527866in}}%
\pgfpathcurveto{\pgfqpoint{2.023818in}{2.527866in}}{\pgfqpoint{2.013219in}{2.523475in}}{\pgfqpoint{2.005405in}{2.515662in}}%
\pgfpathcurveto{\pgfqpoint{1.997592in}{2.507848in}}{\pgfqpoint{1.993201in}{2.497249in}}{\pgfqpoint{1.993201in}{2.486199in}}%
\pgfpathcurveto{\pgfqpoint{1.993201in}{2.475149in}}{\pgfqpoint{1.997592in}{2.464550in}}{\pgfqpoint{2.005405in}{2.456736in}}%
\pgfpathcurveto{\pgfqpoint{2.013219in}{2.448923in}}{\pgfqpoint{2.023818in}{2.444532in}}{\pgfqpoint{2.034868in}{2.444532in}}%
\pgfpathclose%
\pgfusepath{stroke,fill}%
\end{pgfscope}%
\begin{pgfscope}%
\pgfpathrectangle{\pgfqpoint{0.600000in}{0.600000in}}{\pgfqpoint{3.900000in}{3.900000in}}%
\pgfusepath{clip}%
\pgfsetbuttcap%
\pgfsetroundjoin%
\definecolor{currentfill}{rgb}{0.121569,0.466667,0.705882}%
\pgfsetfillcolor{currentfill}%
\pgfsetlinewidth{1.003750pt}%
\definecolor{currentstroke}{rgb}{0.121569,0.466667,0.705882}%
\pgfsetstrokecolor{currentstroke}%
\pgfsetdash{}{0pt}%
\pgfpathmoveto{\pgfqpoint{2.778853in}{3.160532in}}%
\pgfpathcurveto{\pgfqpoint{2.789903in}{3.160532in}}{\pgfqpoint{2.800503in}{3.164922in}}{\pgfqpoint{2.808316in}{3.172736in}}%
\pgfpathcurveto{\pgfqpoint{2.816130in}{3.180550in}}{\pgfqpoint{2.820520in}{3.191149in}}{\pgfqpoint{2.820520in}{3.202199in}}%
\pgfpathcurveto{\pgfqpoint{2.820520in}{3.213249in}}{\pgfqpoint{2.816130in}{3.223848in}}{\pgfqpoint{2.808316in}{3.231662in}}%
\pgfpathcurveto{\pgfqpoint{2.800503in}{3.239475in}}{\pgfqpoint{2.789903in}{3.243866in}}{\pgfqpoint{2.778853in}{3.243866in}}%
\pgfpathcurveto{\pgfqpoint{2.767803in}{3.243866in}}{\pgfqpoint{2.757204in}{3.239475in}}{\pgfqpoint{2.749391in}{3.231662in}}%
\pgfpathcurveto{\pgfqpoint{2.741577in}{3.223848in}}{\pgfqpoint{2.737187in}{3.213249in}}{\pgfqpoint{2.737187in}{3.202199in}}%
\pgfpathcurveto{\pgfqpoint{2.737187in}{3.191149in}}{\pgfqpoint{2.741577in}{3.180550in}}{\pgfqpoint{2.749391in}{3.172736in}}%
\pgfpathcurveto{\pgfqpoint{2.757204in}{3.164922in}}{\pgfqpoint{2.767803in}{3.160532in}}{\pgfqpoint{2.778853in}{3.160532in}}%
\pgfpathclose%
\pgfusepath{stroke,fill}%
\end{pgfscope}%
\begin{pgfscope}%
\pgfpathrectangle{\pgfqpoint{0.600000in}{0.600000in}}{\pgfqpoint{3.900000in}{3.900000in}}%
\pgfusepath{clip}%
\pgfsetbuttcap%
\pgfsetroundjoin%
\definecolor{currentfill}{rgb}{0.121569,0.466667,0.705882}%
\pgfsetfillcolor{currentfill}%
\pgfsetlinewidth{1.003750pt}%
\definecolor{currentstroke}{rgb}{0.121569,0.466667,0.705882}%
\pgfsetstrokecolor{currentstroke}%
\pgfsetdash{}{0pt}%
\pgfpathmoveto{\pgfqpoint{2.095232in}{2.463159in}}%
\pgfpathcurveto{\pgfqpoint{2.106282in}{2.463159in}}{\pgfqpoint{2.116881in}{2.467549in}}{\pgfqpoint{2.124695in}{2.475363in}}%
\pgfpathcurveto{\pgfqpoint{2.132509in}{2.483176in}}{\pgfqpoint{2.136899in}{2.493775in}}{\pgfqpoint{2.136899in}{2.504825in}}%
\pgfpathcurveto{\pgfqpoint{2.136899in}{2.515876in}}{\pgfqpoint{2.132509in}{2.526475in}}{\pgfqpoint{2.124695in}{2.534288in}}%
\pgfpathcurveto{\pgfqpoint{2.116881in}{2.542102in}}{\pgfqpoint{2.106282in}{2.546492in}}{\pgfqpoint{2.095232in}{2.546492in}}%
\pgfpathcurveto{\pgfqpoint{2.084182in}{2.546492in}}{\pgfqpoint{2.073583in}{2.542102in}}{\pgfqpoint{2.065770in}{2.534288in}}%
\pgfpathcurveto{\pgfqpoint{2.057956in}{2.526475in}}{\pgfqpoint{2.053566in}{2.515876in}}{\pgfqpoint{2.053566in}{2.504825in}}%
\pgfpathcurveto{\pgfqpoint{2.053566in}{2.493775in}}{\pgfqpoint{2.057956in}{2.483176in}}{\pgfqpoint{2.065770in}{2.475363in}}%
\pgfpathcurveto{\pgfqpoint{2.073583in}{2.467549in}}{\pgfqpoint{2.084182in}{2.463159in}}{\pgfqpoint{2.095232in}{2.463159in}}%
\pgfpathclose%
\pgfusepath{stroke,fill}%
\end{pgfscope}%
\begin{pgfscope}%
\pgfpathrectangle{\pgfqpoint{0.600000in}{0.600000in}}{\pgfqpoint{3.900000in}{3.900000in}}%
\pgfusepath{clip}%
\pgfsetbuttcap%
\pgfsetroundjoin%
\definecolor{currentfill}{rgb}{0.121569,0.466667,0.705882}%
\pgfsetfillcolor{currentfill}%
\pgfsetlinewidth{1.003750pt}%
\definecolor{currentstroke}{rgb}{0.121569,0.466667,0.705882}%
\pgfsetstrokecolor{currentstroke}%
\pgfsetdash{}{0pt}%
\pgfpathmoveto{\pgfqpoint{2.196675in}{2.578797in}}%
\pgfpathcurveto{\pgfqpoint{2.207725in}{2.578797in}}{\pgfqpoint{2.218324in}{2.583187in}}{\pgfqpoint{2.226138in}{2.591001in}}%
\pgfpathcurveto{\pgfqpoint{2.233951in}{2.598814in}}{\pgfqpoint{2.238342in}{2.609413in}}{\pgfqpoint{2.238342in}{2.620464in}}%
\pgfpathcurveto{\pgfqpoint{2.238342in}{2.631514in}}{\pgfqpoint{2.233951in}{2.642113in}}{\pgfqpoint{2.226138in}{2.649926in}}%
\pgfpathcurveto{\pgfqpoint{2.218324in}{2.657740in}}{\pgfqpoint{2.207725in}{2.662130in}}{\pgfqpoint{2.196675in}{2.662130in}}%
\pgfpathcurveto{\pgfqpoint{2.185625in}{2.662130in}}{\pgfqpoint{2.175026in}{2.657740in}}{\pgfqpoint{2.167212in}{2.649926in}}%
\pgfpathcurveto{\pgfqpoint{2.159399in}{2.642113in}}{\pgfqpoint{2.155008in}{2.631514in}}{\pgfqpoint{2.155008in}{2.620464in}}%
\pgfpathcurveto{\pgfqpoint{2.155008in}{2.609413in}}{\pgfqpoint{2.159399in}{2.598814in}}{\pgfqpoint{2.167212in}{2.591001in}}%
\pgfpathcurveto{\pgfqpoint{2.175026in}{2.583187in}}{\pgfqpoint{2.185625in}{2.578797in}}{\pgfqpoint{2.196675in}{2.578797in}}%
\pgfpathclose%
\pgfusepath{stroke,fill}%
\end{pgfscope}%
\begin{pgfscope}%
\pgfpathrectangle{\pgfqpoint{0.600000in}{0.600000in}}{\pgfqpoint{3.900000in}{3.900000in}}%
\pgfusepath{clip}%
\pgfsetbuttcap%
\pgfsetroundjoin%
\definecolor{currentfill}{rgb}{0.121569,0.466667,0.705882}%
\pgfsetfillcolor{currentfill}%
\pgfsetlinewidth{1.003750pt}%
\definecolor{currentstroke}{rgb}{0.121569,0.466667,0.705882}%
\pgfsetstrokecolor{currentstroke}%
\pgfsetdash{}{0pt}%
\pgfpathmoveto{\pgfqpoint{3.167034in}{3.092201in}}%
\pgfpathcurveto{\pgfqpoint{3.178084in}{3.092201in}}{\pgfqpoint{3.188683in}{3.096591in}}{\pgfqpoint{3.196497in}{3.104405in}}%
\pgfpathcurveto{\pgfqpoint{3.204310in}{3.112218in}}{\pgfqpoint{3.208700in}{3.122817in}}{\pgfqpoint{3.208700in}{3.133867in}}%
\pgfpathcurveto{\pgfqpoint{3.208700in}{3.144918in}}{\pgfqpoint{3.204310in}{3.155517in}}{\pgfqpoint{3.196497in}{3.163330in}}%
\pgfpathcurveto{\pgfqpoint{3.188683in}{3.171144in}}{\pgfqpoint{3.178084in}{3.175534in}}{\pgfqpoint{3.167034in}{3.175534in}}%
\pgfpathcurveto{\pgfqpoint{3.155984in}{3.175534in}}{\pgfqpoint{3.145385in}{3.171144in}}{\pgfqpoint{3.137571in}{3.163330in}}%
\pgfpathcurveto{\pgfqpoint{3.129757in}{3.155517in}}{\pgfqpoint{3.125367in}{3.144918in}}{\pgfqpoint{3.125367in}{3.133867in}}%
\pgfpathcurveto{\pgfqpoint{3.125367in}{3.122817in}}{\pgfqpoint{3.129757in}{3.112218in}}{\pgfqpoint{3.137571in}{3.104405in}}%
\pgfpathcurveto{\pgfqpoint{3.145385in}{3.096591in}}{\pgfqpoint{3.155984in}{3.092201in}}{\pgfqpoint{3.167034in}{3.092201in}}%
\pgfpathclose%
\pgfusepath{stroke,fill}%
\end{pgfscope}%
\begin{pgfscope}%
\pgfpathrectangle{\pgfqpoint{0.600000in}{0.600000in}}{\pgfqpoint{3.900000in}{3.900000in}}%
\pgfusepath{clip}%
\pgfsetbuttcap%
\pgfsetroundjoin%
\definecolor{currentfill}{rgb}{0.121569,0.466667,0.705882}%
\pgfsetfillcolor{currentfill}%
\pgfsetlinewidth{1.003750pt}%
\definecolor{currentstroke}{rgb}{0.121569,0.466667,0.705882}%
\pgfsetstrokecolor{currentstroke}%
\pgfsetdash{}{0pt}%
\pgfpathmoveto{\pgfqpoint{2.183752in}{2.181686in}}%
\pgfpathcurveto{\pgfqpoint{2.194802in}{2.181686in}}{\pgfqpoint{2.205401in}{2.186076in}}{\pgfqpoint{2.213215in}{2.193889in}}%
\pgfpathcurveto{\pgfqpoint{2.221028in}{2.201703in}}{\pgfqpoint{2.225418in}{2.212302in}}{\pgfqpoint{2.225418in}{2.223352in}}%
\pgfpathcurveto{\pgfqpoint{2.225418in}{2.234402in}}{\pgfqpoint{2.221028in}{2.245001in}}{\pgfqpoint{2.213215in}{2.252815in}}%
\pgfpathcurveto{\pgfqpoint{2.205401in}{2.260629in}}{\pgfqpoint{2.194802in}{2.265019in}}{\pgfqpoint{2.183752in}{2.265019in}}%
\pgfpathcurveto{\pgfqpoint{2.172702in}{2.265019in}}{\pgfqpoint{2.162103in}{2.260629in}}{\pgfqpoint{2.154289in}{2.252815in}}%
\pgfpathcurveto{\pgfqpoint{2.146475in}{2.245001in}}{\pgfqpoint{2.142085in}{2.234402in}}{\pgfqpoint{2.142085in}{2.223352in}}%
\pgfpathcurveto{\pgfqpoint{2.142085in}{2.212302in}}{\pgfqpoint{2.146475in}{2.201703in}}{\pgfqpoint{2.154289in}{2.193889in}}%
\pgfpathcurveto{\pgfqpoint{2.162103in}{2.186076in}}{\pgfqpoint{2.172702in}{2.181686in}}{\pgfqpoint{2.183752in}{2.181686in}}%
\pgfpathclose%
\pgfusepath{stroke,fill}%
\end{pgfscope}%
\begin{pgfscope}%
\pgfpathrectangle{\pgfqpoint{0.600000in}{0.600000in}}{\pgfqpoint{3.900000in}{3.900000in}}%
\pgfusepath{clip}%
\pgfsetbuttcap%
\pgfsetroundjoin%
\definecolor{currentfill}{rgb}{0.121569,0.466667,0.705882}%
\pgfsetfillcolor{currentfill}%
\pgfsetlinewidth{1.003750pt}%
\definecolor{currentstroke}{rgb}{0.121569,0.466667,0.705882}%
\pgfsetstrokecolor{currentstroke}%
\pgfsetdash{}{0pt}%
\pgfpathmoveto{\pgfqpoint{2.966244in}{2.903875in}}%
\pgfpathcurveto{\pgfqpoint{2.977294in}{2.903875in}}{\pgfqpoint{2.987893in}{2.908266in}}{\pgfqpoint{2.995706in}{2.916079in}}%
\pgfpathcurveto{\pgfqpoint{3.003520in}{2.923893in}}{\pgfqpoint{3.007910in}{2.934492in}}{\pgfqpoint{3.007910in}{2.945542in}}%
\pgfpathcurveto{\pgfqpoint{3.007910in}{2.956592in}}{\pgfqpoint{3.003520in}{2.967191in}}{\pgfqpoint{2.995706in}{2.975005in}}%
\pgfpathcurveto{\pgfqpoint{2.987893in}{2.982818in}}{\pgfqpoint{2.977294in}{2.987209in}}{\pgfqpoint{2.966244in}{2.987209in}}%
\pgfpathcurveto{\pgfqpoint{2.955194in}{2.987209in}}{\pgfqpoint{2.944595in}{2.982818in}}{\pgfqpoint{2.936781in}{2.975005in}}%
\pgfpathcurveto{\pgfqpoint{2.928967in}{2.967191in}}{\pgfqpoint{2.924577in}{2.956592in}}{\pgfqpoint{2.924577in}{2.945542in}}%
\pgfpathcurveto{\pgfqpoint{2.924577in}{2.934492in}}{\pgfqpoint{2.928967in}{2.923893in}}{\pgfqpoint{2.936781in}{2.916079in}}%
\pgfpathcurveto{\pgfqpoint{2.944595in}{2.908266in}}{\pgfqpoint{2.955194in}{2.903875in}}{\pgfqpoint{2.966244in}{2.903875in}}%
\pgfpathclose%
\pgfusepath{stroke,fill}%
\end{pgfscope}%
\begin{pgfscope}%
\pgfpathrectangle{\pgfqpoint{0.600000in}{0.600000in}}{\pgfqpoint{3.900000in}{3.900000in}}%
\pgfusepath{clip}%
\pgfsetbuttcap%
\pgfsetroundjoin%
\definecolor{currentfill}{rgb}{0.121569,0.466667,0.705882}%
\pgfsetfillcolor{currentfill}%
\pgfsetlinewidth{1.003750pt}%
\definecolor{currentstroke}{rgb}{0.121569,0.466667,0.705882}%
\pgfsetstrokecolor{currentstroke}%
\pgfsetdash{}{0pt}%
\pgfpathmoveto{\pgfqpoint{3.387328in}{2.663163in}}%
\pgfpathcurveto{\pgfqpoint{3.398378in}{2.663163in}}{\pgfqpoint{3.408977in}{2.667554in}}{\pgfqpoint{3.416791in}{2.675367in}}%
\pgfpathcurveto{\pgfqpoint{3.424604in}{2.683181in}}{\pgfqpoint{3.428995in}{2.693780in}}{\pgfqpoint{3.428995in}{2.704830in}}%
\pgfpathcurveto{\pgfqpoint{3.428995in}{2.715880in}}{\pgfqpoint{3.424604in}{2.726479in}}{\pgfqpoint{3.416791in}{2.734293in}}%
\pgfpathcurveto{\pgfqpoint{3.408977in}{2.742106in}}{\pgfqpoint{3.398378in}{2.746497in}}{\pgfqpoint{3.387328in}{2.746497in}}%
\pgfpathcurveto{\pgfqpoint{3.376278in}{2.746497in}}{\pgfqpoint{3.365679in}{2.742106in}}{\pgfqpoint{3.357865in}{2.734293in}}%
\pgfpathcurveto{\pgfqpoint{3.350052in}{2.726479in}}{\pgfqpoint{3.345661in}{2.715880in}}{\pgfqpoint{3.345661in}{2.704830in}}%
\pgfpathcurveto{\pgfqpoint{3.345661in}{2.693780in}}{\pgfqpoint{3.350052in}{2.683181in}}{\pgfqpoint{3.357865in}{2.675367in}}%
\pgfpathcurveto{\pgfqpoint{3.365679in}{2.667554in}}{\pgfqpoint{3.376278in}{2.663163in}}{\pgfqpoint{3.387328in}{2.663163in}}%
\pgfpathclose%
\pgfusepath{stroke,fill}%
\end{pgfscope}%
\begin{pgfscope}%
\pgfpathrectangle{\pgfqpoint{0.600000in}{0.600000in}}{\pgfqpoint{3.900000in}{3.900000in}}%
\pgfusepath{clip}%
\pgfsetbuttcap%
\pgfsetroundjoin%
\definecolor{currentfill}{rgb}{0.121569,0.466667,0.705882}%
\pgfsetfillcolor{currentfill}%
\pgfsetlinewidth{1.003750pt}%
\definecolor{currentstroke}{rgb}{0.121569,0.466667,0.705882}%
\pgfsetstrokecolor{currentstroke}%
\pgfsetdash{}{0pt}%
\pgfpathmoveto{\pgfqpoint{2.542119in}{3.224138in}}%
\pgfpathcurveto{\pgfqpoint{2.553169in}{3.224138in}}{\pgfqpoint{2.563768in}{3.228528in}}{\pgfqpoint{2.571582in}{3.236342in}}%
\pgfpathcurveto{\pgfqpoint{2.579396in}{3.244155in}}{\pgfqpoint{2.583786in}{3.254754in}}{\pgfqpoint{2.583786in}{3.265804in}}%
\pgfpathcurveto{\pgfqpoint{2.583786in}{3.276855in}}{\pgfqpoint{2.579396in}{3.287454in}}{\pgfqpoint{2.571582in}{3.295267in}}%
\pgfpathcurveto{\pgfqpoint{2.563768in}{3.303081in}}{\pgfqpoint{2.553169in}{3.307471in}}{\pgfqpoint{2.542119in}{3.307471in}}%
\pgfpathcurveto{\pgfqpoint{2.531069in}{3.307471in}}{\pgfqpoint{2.520470in}{3.303081in}}{\pgfqpoint{2.512656in}{3.295267in}}%
\pgfpathcurveto{\pgfqpoint{2.504843in}{3.287454in}}{\pgfqpoint{2.500453in}{3.276855in}}{\pgfqpoint{2.500453in}{3.265804in}}%
\pgfpathcurveto{\pgfqpoint{2.500453in}{3.254754in}}{\pgfqpoint{2.504843in}{3.244155in}}{\pgfqpoint{2.512656in}{3.236342in}}%
\pgfpathcurveto{\pgfqpoint{2.520470in}{3.228528in}}{\pgfqpoint{2.531069in}{3.224138in}}{\pgfqpoint{2.542119in}{3.224138in}}%
\pgfpathclose%
\pgfusepath{stroke,fill}%
\end{pgfscope}%
\begin{pgfscope}%
\pgfpathrectangle{\pgfqpoint{0.600000in}{0.600000in}}{\pgfqpoint{3.900000in}{3.900000in}}%
\pgfusepath{clip}%
\pgfsetbuttcap%
\pgfsetroundjoin%
\definecolor{currentfill}{rgb}{0.121569,0.466667,0.705882}%
\pgfsetfillcolor{currentfill}%
\pgfsetlinewidth{1.003750pt}%
\definecolor{currentstroke}{rgb}{0.121569,0.466667,0.705882}%
\pgfsetstrokecolor{currentstroke}%
\pgfsetdash{}{0pt}%
\pgfpathmoveto{\pgfqpoint{3.057126in}{2.574681in}}%
\pgfpathcurveto{\pgfqpoint{3.068177in}{2.574681in}}{\pgfqpoint{3.078776in}{2.579071in}}{\pgfqpoint{3.086589in}{2.586885in}}%
\pgfpathcurveto{\pgfqpoint{3.094403in}{2.594699in}}{\pgfqpoint{3.098793in}{2.605298in}}{\pgfqpoint{3.098793in}{2.616348in}}%
\pgfpathcurveto{\pgfqpoint{3.098793in}{2.627398in}}{\pgfqpoint{3.094403in}{2.637997in}}{\pgfqpoint{3.086589in}{2.645811in}}%
\pgfpathcurveto{\pgfqpoint{3.078776in}{2.653624in}}{\pgfqpoint{3.068177in}{2.658014in}}{\pgfqpoint{3.057126in}{2.658014in}}%
\pgfpathcurveto{\pgfqpoint{3.046076in}{2.658014in}}{\pgfqpoint{3.035477in}{2.653624in}}{\pgfqpoint{3.027664in}{2.645811in}}%
\pgfpathcurveto{\pgfqpoint{3.019850in}{2.637997in}}{\pgfqpoint{3.015460in}{2.627398in}}{\pgfqpoint{3.015460in}{2.616348in}}%
\pgfpathcurveto{\pgfqpoint{3.015460in}{2.605298in}}{\pgfqpoint{3.019850in}{2.594699in}}{\pgfqpoint{3.027664in}{2.586885in}}%
\pgfpathcurveto{\pgfqpoint{3.035477in}{2.579071in}}{\pgfqpoint{3.046076in}{2.574681in}}{\pgfqpoint{3.057126in}{2.574681in}}%
\pgfpathclose%
\pgfusepath{stroke,fill}%
\end{pgfscope}%
\begin{pgfscope}%
\pgfpathrectangle{\pgfqpoint{0.600000in}{0.600000in}}{\pgfqpoint{3.900000in}{3.900000in}}%
\pgfusepath{clip}%
\pgfsetbuttcap%
\pgfsetroundjoin%
\definecolor{currentfill}{rgb}{0.121569,0.466667,0.705882}%
\pgfsetfillcolor{currentfill}%
\pgfsetlinewidth{1.003750pt}%
\definecolor{currentstroke}{rgb}{0.121569,0.466667,0.705882}%
\pgfsetstrokecolor{currentstroke}%
\pgfsetdash{}{0pt}%
\pgfpathmoveto{\pgfqpoint{2.349317in}{3.892970in}}%
\pgfpathcurveto{\pgfqpoint{2.360367in}{3.892970in}}{\pgfqpoint{2.370966in}{3.897360in}}{\pgfqpoint{2.378780in}{3.905174in}}%
\pgfpathcurveto{\pgfqpoint{2.386594in}{3.912987in}}{\pgfqpoint{2.390984in}{3.923586in}}{\pgfqpoint{2.390984in}{3.934636in}}%
\pgfpathcurveto{\pgfqpoint{2.390984in}{3.945686in}}{\pgfqpoint{2.386594in}{3.956285in}}{\pgfqpoint{2.378780in}{3.964099in}}%
\pgfpathcurveto{\pgfqpoint{2.370966in}{3.971913in}}{\pgfqpoint{2.360367in}{3.976303in}}{\pgfqpoint{2.349317in}{3.976303in}}%
\pgfpathcurveto{\pgfqpoint{2.338267in}{3.976303in}}{\pgfqpoint{2.327668in}{3.971913in}}{\pgfqpoint{2.319854in}{3.964099in}}%
\pgfpathcurveto{\pgfqpoint{2.312041in}{3.956285in}}{\pgfqpoint{2.307651in}{3.945686in}}{\pgfqpoint{2.307651in}{3.934636in}}%
\pgfpathcurveto{\pgfqpoint{2.307651in}{3.923586in}}{\pgfqpoint{2.312041in}{3.912987in}}{\pgfqpoint{2.319854in}{3.905174in}}%
\pgfpathcurveto{\pgfqpoint{2.327668in}{3.897360in}}{\pgfqpoint{2.338267in}{3.892970in}}{\pgfqpoint{2.349317in}{3.892970in}}%
\pgfpathclose%
\pgfusepath{stroke,fill}%
\end{pgfscope}%
\begin{pgfscope}%
\pgfpathrectangle{\pgfqpoint{0.600000in}{0.600000in}}{\pgfqpoint{3.900000in}{3.900000in}}%
\pgfusepath{clip}%
\pgfsetbuttcap%
\pgfsetroundjoin%
\definecolor{currentfill}{rgb}{0.121569,0.466667,0.705882}%
\pgfsetfillcolor{currentfill}%
\pgfsetlinewidth{1.003750pt}%
\definecolor{currentstroke}{rgb}{0.121569,0.466667,0.705882}%
\pgfsetstrokecolor{currentstroke}%
\pgfsetdash{}{0pt}%
\pgfpathmoveto{\pgfqpoint{2.542833in}{2.332908in}}%
\pgfpathcurveto{\pgfqpoint{2.553883in}{2.332908in}}{\pgfqpoint{2.564482in}{2.337298in}}{\pgfqpoint{2.572296in}{2.345112in}}%
\pgfpathcurveto{\pgfqpoint{2.580110in}{2.352925in}}{\pgfqpoint{2.584500in}{2.363525in}}{\pgfqpoint{2.584500in}{2.374575in}}%
\pgfpathcurveto{\pgfqpoint{2.584500in}{2.385625in}}{\pgfqpoint{2.580110in}{2.396224in}}{\pgfqpoint{2.572296in}{2.404037in}}%
\pgfpathcurveto{\pgfqpoint{2.564482in}{2.411851in}}{\pgfqpoint{2.553883in}{2.416241in}}{\pgfqpoint{2.542833in}{2.416241in}}%
\pgfpathcurveto{\pgfqpoint{2.531783in}{2.416241in}}{\pgfqpoint{2.521184in}{2.411851in}}{\pgfqpoint{2.513370in}{2.404037in}}%
\pgfpathcurveto{\pgfqpoint{2.505557in}{2.396224in}}{\pgfqpoint{2.501167in}{2.385625in}}{\pgfqpoint{2.501167in}{2.374575in}}%
\pgfpathcurveto{\pgfqpoint{2.501167in}{2.363525in}}{\pgfqpoint{2.505557in}{2.352925in}}{\pgfqpoint{2.513370in}{2.345112in}}%
\pgfpathcurveto{\pgfqpoint{2.521184in}{2.337298in}}{\pgfqpoint{2.531783in}{2.332908in}}{\pgfqpoint{2.542833in}{2.332908in}}%
\pgfpathclose%
\pgfusepath{stroke,fill}%
\end{pgfscope}%
\begin{pgfscope}%
\pgfpathrectangle{\pgfqpoint{0.600000in}{0.600000in}}{\pgfqpoint{3.900000in}{3.900000in}}%
\pgfusepath{clip}%
\pgfsetbuttcap%
\pgfsetroundjoin%
\definecolor{currentfill}{rgb}{0.121569,0.466667,0.705882}%
\pgfsetfillcolor{currentfill}%
\pgfsetlinewidth{1.003750pt}%
\definecolor{currentstroke}{rgb}{0.121569,0.466667,0.705882}%
\pgfsetstrokecolor{currentstroke}%
\pgfsetdash{}{0pt}%
\pgfpathmoveto{\pgfqpoint{2.517239in}{3.100807in}}%
\pgfpathcurveto{\pgfqpoint{2.528289in}{3.100807in}}{\pgfqpoint{2.538888in}{3.105198in}}{\pgfqpoint{2.546702in}{3.113011in}}%
\pgfpathcurveto{\pgfqpoint{2.554515in}{3.120825in}}{\pgfqpoint{2.558906in}{3.131424in}}{\pgfqpoint{2.558906in}{3.142474in}}%
\pgfpathcurveto{\pgfqpoint{2.558906in}{3.153524in}}{\pgfqpoint{2.554515in}{3.164123in}}{\pgfqpoint{2.546702in}{3.171937in}}%
\pgfpathcurveto{\pgfqpoint{2.538888in}{3.179750in}}{\pgfqpoint{2.528289in}{3.184141in}}{\pgfqpoint{2.517239in}{3.184141in}}%
\pgfpathcurveto{\pgfqpoint{2.506189in}{3.184141in}}{\pgfqpoint{2.495590in}{3.179750in}}{\pgfqpoint{2.487776in}{3.171937in}}%
\pgfpathcurveto{\pgfqpoint{2.479963in}{3.164123in}}{\pgfqpoint{2.475572in}{3.153524in}}{\pgfqpoint{2.475572in}{3.142474in}}%
\pgfpathcurveto{\pgfqpoint{2.475572in}{3.131424in}}{\pgfqpoint{2.479963in}{3.120825in}}{\pgfqpoint{2.487776in}{3.113011in}}%
\pgfpathcurveto{\pgfqpoint{2.495590in}{3.105198in}}{\pgfqpoint{2.506189in}{3.100807in}}{\pgfqpoint{2.517239in}{3.100807in}}%
\pgfpathclose%
\pgfusepath{stroke,fill}%
\end{pgfscope}%
\begin{pgfscope}%
\pgfpathrectangle{\pgfqpoint{0.600000in}{0.600000in}}{\pgfqpoint{3.900000in}{3.900000in}}%
\pgfusepath{clip}%
\pgfsetbuttcap%
\pgfsetroundjoin%
\definecolor{currentfill}{rgb}{0.121569,0.466667,0.705882}%
\pgfsetfillcolor{currentfill}%
\pgfsetlinewidth{1.003750pt}%
\definecolor{currentstroke}{rgb}{0.121569,0.466667,0.705882}%
\pgfsetstrokecolor{currentstroke}%
\pgfsetdash{}{0pt}%
\pgfpathmoveto{\pgfqpoint{2.491465in}{2.407822in}}%
\pgfpathcurveto{\pgfqpoint{2.502515in}{2.407822in}}{\pgfqpoint{2.513114in}{2.412212in}}{\pgfqpoint{2.520927in}{2.420026in}}%
\pgfpathcurveto{\pgfqpoint{2.528741in}{2.427840in}}{\pgfqpoint{2.533131in}{2.438439in}}{\pgfqpoint{2.533131in}{2.449489in}}%
\pgfpathcurveto{\pgfqpoint{2.533131in}{2.460539in}}{\pgfqpoint{2.528741in}{2.471138in}}{\pgfqpoint{2.520927in}{2.478952in}}%
\pgfpathcurveto{\pgfqpoint{2.513114in}{2.486765in}}{\pgfqpoint{2.502515in}{2.491156in}}{\pgfqpoint{2.491465in}{2.491156in}}%
\pgfpathcurveto{\pgfqpoint{2.480415in}{2.491156in}}{\pgfqpoint{2.469816in}{2.486765in}}{\pgfqpoint{2.462002in}{2.478952in}}%
\pgfpathcurveto{\pgfqpoint{2.454188in}{2.471138in}}{\pgfqpoint{2.449798in}{2.460539in}}{\pgfqpoint{2.449798in}{2.449489in}}%
\pgfpathcurveto{\pgfqpoint{2.449798in}{2.438439in}}{\pgfqpoint{2.454188in}{2.427840in}}{\pgfqpoint{2.462002in}{2.420026in}}%
\pgfpathcurveto{\pgfqpoint{2.469816in}{2.412212in}}{\pgfqpoint{2.480415in}{2.407822in}}{\pgfqpoint{2.491465in}{2.407822in}}%
\pgfpathclose%
\pgfusepath{stroke,fill}%
\end{pgfscope}%
\begin{pgfscope}%
\pgfpathrectangle{\pgfqpoint{0.600000in}{0.600000in}}{\pgfqpoint{3.900000in}{3.900000in}}%
\pgfusepath{clip}%
\pgfsetbuttcap%
\pgfsetroundjoin%
\definecolor{currentfill}{rgb}{0.121569,0.466667,0.705882}%
\pgfsetfillcolor{currentfill}%
\pgfsetlinewidth{1.003750pt}%
\definecolor{currentstroke}{rgb}{0.121569,0.466667,0.705882}%
\pgfsetstrokecolor{currentstroke}%
\pgfsetdash{}{0pt}%
\pgfpathmoveto{\pgfqpoint{2.845054in}{2.600941in}}%
\pgfpathcurveto{\pgfqpoint{2.856104in}{2.600941in}}{\pgfqpoint{2.866703in}{2.605331in}}{\pgfqpoint{2.874517in}{2.613145in}}%
\pgfpathcurveto{\pgfqpoint{2.882331in}{2.620958in}}{\pgfqpoint{2.886721in}{2.631557in}}{\pgfqpoint{2.886721in}{2.642608in}}%
\pgfpathcurveto{\pgfqpoint{2.886721in}{2.653658in}}{\pgfqpoint{2.882331in}{2.664257in}}{\pgfqpoint{2.874517in}{2.672070in}}%
\pgfpathcurveto{\pgfqpoint{2.866703in}{2.679884in}}{\pgfqpoint{2.856104in}{2.684274in}}{\pgfqpoint{2.845054in}{2.684274in}}%
\pgfpathcurveto{\pgfqpoint{2.834004in}{2.684274in}}{\pgfqpoint{2.823405in}{2.679884in}}{\pgfqpoint{2.815591in}{2.672070in}}%
\pgfpathcurveto{\pgfqpoint{2.807778in}{2.664257in}}{\pgfqpoint{2.803387in}{2.653658in}}{\pgfqpoint{2.803387in}{2.642608in}}%
\pgfpathcurveto{\pgfqpoint{2.803387in}{2.631557in}}{\pgfqpoint{2.807778in}{2.620958in}}{\pgfqpoint{2.815591in}{2.613145in}}%
\pgfpathcurveto{\pgfqpoint{2.823405in}{2.605331in}}{\pgfqpoint{2.834004in}{2.600941in}}{\pgfqpoint{2.845054in}{2.600941in}}%
\pgfpathclose%
\pgfusepath{stroke,fill}%
\end{pgfscope}%
\begin{pgfscope}%
\pgfpathrectangle{\pgfqpoint{0.600000in}{0.600000in}}{\pgfqpoint{3.900000in}{3.900000in}}%
\pgfusepath{clip}%
\pgfsetbuttcap%
\pgfsetroundjoin%
\definecolor{currentfill}{rgb}{0.121569,0.466667,0.705882}%
\pgfsetfillcolor{currentfill}%
\pgfsetlinewidth{1.003750pt}%
\definecolor{currentstroke}{rgb}{0.121569,0.466667,0.705882}%
\pgfsetstrokecolor{currentstroke}%
\pgfsetdash{}{0pt}%
\pgfpathmoveto{\pgfqpoint{2.272014in}{3.007541in}}%
\pgfpathcurveto{\pgfqpoint{2.283064in}{3.007541in}}{\pgfqpoint{2.293663in}{3.011932in}}{\pgfqpoint{2.301477in}{3.019745in}}%
\pgfpathcurveto{\pgfqpoint{2.309291in}{3.027559in}}{\pgfqpoint{2.313681in}{3.038158in}}{\pgfqpoint{2.313681in}{3.049208in}}%
\pgfpathcurveto{\pgfqpoint{2.313681in}{3.060258in}}{\pgfqpoint{2.309291in}{3.070857in}}{\pgfqpoint{2.301477in}{3.078671in}}%
\pgfpathcurveto{\pgfqpoint{2.293663in}{3.086484in}}{\pgfqpoint{2.283064in}{3.090875in}}{\pgfqpoint{2.272014in}{3.090875in}}%
\pgfpathcurveto{\pgfqpoint{2.260964in}{3.090875in}}{\pgfqpoint{2.250365in}{3.086484in}}{\pgfqpoint{2.242551in}{3.078671in}}%
\pgfpathcurveto{\pgfqpoint{2.234738in}{3.070857in}}{\pgfqpoint{2.230348in}{3.060258in}}{\pgfqpoint{2.230348in}{3.049208in}}%
\pgfpathcurveto{\pgfqpoint{2.230348in}{3.038158in}}{\pgfqpoint{2.234738in}{3.027559in}}{\pgfqpoint{2.242551in}{3.019745in}}%
\pgfpathcurveto{\pgfqpoint{2.250365in}{3.011932in}}{\pgfqpoint{2.260964in}{3.007541in}}{\pgfqpoint{2.272014in}{3.007541in}}%
\pgfpathclose%
\pgfusepath{stroke,fill}%
\end{pgfscope}%
\begin{pgfscope}%
\pgfpathrectangle{\pgfqpoint{0.600000in}{0.600000in}}{\pgfqpoint{3.900000in}{3.900000in}}%
\pgfusepath{clip}%
\pgfsetbuttcap%
\pgfsetroundjoin%
\definecolor{currentfill}{rgb}{0.121569,0.466667,0.705882}%
\pgfsetfillcolor{currentfill}%
\pgfsetlinewidth{1.003750pt}%
\definecolor{currentstroke}{rgb}{0.121569,0.466667,0.705882}%
\pgfsetstrokecolor{currentstroke}%
\pgfsetdash{}{0pt}%
\pgfpathmoveto{\pgfqpoint{3.103963in}{3.000694in}}%
\pgfpathcurveto{\pgfqpoint{3.115014in}{3.000694in}}{\pgfqpoint{3.125613in}{3.005084in}}{\pgfqpoint{3.133426in}{3.012898in}}%
\pgfpathcurveto{\pgfqpoint{3.141240in}{3.020711in}}{\pgfqpoint{3.145630in}{3.031310in}}{\pgfqpoint{3.145630in}{3.042360in}}%
\pgfpathcurveto{\pgfqpoint{3.145630in}{3.053410in}}{\pgfqpoint{3.141240in}{3.064010in}}{\pgfqpoint{3.133426in}{3.071823in}}%
\pgfpathcurveto{\pgfqpoint{3.125613in}{3.079637in}}{\pgfqpoint{3.115014in}{3.084027in}}{\pgfqpoint{3.103963in}{3.084027in}}%
\pgfpathcurveto{\pgfqpoint{3.092913in}{3.084027in}}{\pgfqpoint{3.082314in}{3.079637in}}{\pgfqpoint{3.074501in}{3.071823in}}%
\pgfpathcurveto{\pgfqpoint{3.066687in}{3.064010in}}{\pgfqpoint{3.062297in}{3.053410in}}{\pgfqpoint{3.062297in}{3.042360in}}%
\pgfpathcurveto{\pgfqpoint{3.062297in}{3.031310in}}{\pgfqpoint{3.066687in}{3.020711in}}{\pgfqpoint{3.074501in}{3.012898in}}%
\pgfpathcurveto{\pgfqpoint{3.082314in}{3.005084in}}{\pgfqpoint{3.092913in}{3.000694in}}{\pgfqpoint{3.103963in}{3.000694in}}%
\pgfpathclose%
\pgfusepath{stroke,fill}%
\end{pgfscope}%
\begin{pgfscope}%
\pgfpathrectangle{\pgfqpoint{0.600000in}{0.600000in}}{\pgfqpoint{3.900000in}{3.900000in}}%
\pgfusepath{clip}%
\pgfsetbuttcap%
\pgfsetroundjoin%
\definecolor{currentfill}{rgb}{0.121569,0.466667,0.705882}%
\pgfsetfillcolor{currentfill}%
\pgfsetlinewidth{1.003750pt}%
\definecolor{currentstroke}{rgb}{0.121569,0.466667,0.705882}%
\pgfsetstrokecolor{currentstroke}%
\pgfsetdash{}{0pt}%
\pgfpathmoveto{\pgfqpoint{3.487059in}{2.314895in}}%
\pgfpathcurveto{\pgfqpoint{3.498109in}{2.314895in}}{\pgfqpoint{3.508708in}{2.319285in}}{\pgfqpoint{3.516522in}{2.327099in}}%
\pgfpathcurveto{\pgfqpoint{3.524336in}{2.334913in}}{\pgfqpoint{3.528726in}{2.345512in}}{\pgfqpoint{3.528726in}{2.356562in}}%
\pgfpathcurveto{\pgfqpoint{3.528726in}{2.367612in}}{\pgfqpoint{3.524336in}{2.378211in}}{\pgfqpoint{3.516522in}{2.386025in}}%
\pgfpathcurveto{\pgfqpoint{3.508708in}{2.393838in}}{\pgfqpoint{3.498109in}{2.398228in}}{\pgfqpoint{3.487059in}{2.398228in}}%
\pgfpathcurveto{\pgfqpoint{3.476009in}{2.398228in}}{\pgfqpoint{3.465410in}{2.393838in}}{\pgfqpoint{3.457596in}{2.386025in}}%
\pgfpathcurveto{\pgfqpoint{3.449783in}{2.378211in}}{\pgfqpoint{3.445393in}{2.367612in}}{\pgfqpoint{3.445393in}{2.356562in}}%
\pgfpathcurveto{\pgfqpoint{3.445393in}{2.345512in}}{\pgfqpoint{3.449783in}{2.334913in}}{\pgfqpoint{3.457596in}{2.327099in}}%
\pgfpathcurveto{\pgfqpoint{3.465410in}{2.319285in}}{\pgfqpoint{3.476009in}{2.314895in}}{\pgfqpoint{3.487059in}{2.314895in}}%
\pgfpathclose%
\pgfusepath{stroke,fill}%
\end{pgfscope}%
\begin{pgfscope}%
\pgfpathrectangle{\pgfqpoint{0.600000in}{0.600000in}}{\pgfqpoint{3.900000in}{3.900000in}}%
\pgfusepath{clip}%
\pgfsetbuttcap%
\pgfsetroundjoin%
\definecolor{currentfill}{rgb}{0.121569,0.466667,0.705882}%
\pgfsetfillcolor{currentfill}%
\pgfsetlinewidth{1.003750pt}%
\definecolor{currentstroke}{rgb}{0.121569,0.466667,0.705882}%
\pgfsetstrokecolor{currentstroke}%
\pgfsetdash{}{0pt}%
\pgfpathmoveto{\pgfqpoint{2.312166in}{2.305721in}}%
\pgfpathcurveto{\pgfqpoint{2.323216in}{2.305721in}}{\pgfqpoint{2.333815in}{2.310111in}}{\pgfqpoint{2.341629in}{2.317924in}}%
\pgfpathcurveto{\pgfqpoint{2.349443in}{2.325738in}}{\pgfqpoint{2.353833in}{2.336337in}}{\pgfqpoint{2.353833in}{2.347387in}}%
\pgfpathcurveto{\pgfqpoint{2.353833in}{2.358437in}}{\pgfqpoint{2.349443in}{2.369036in}}{\pgfqpoint{2.341629in}{2.376850in}}%
\pgfpathcurveto{\pgfqpoint{2.333815in}{2.384664in}}{\pgfqpoint{2.323216in}{2.389054in}}{\pgfqpoint{2.312166in}{2.389054in}}%
\pgfpathcurveto{\pgfqpoint{2.301116in}{2.389054in}}{\pgfqpoint{2.290517in}{2.384664in}}{\pgfqpoint{2.282703in}{2.376850in}}%
\pgfpathcurveto{\pgfqpoint{2.274890in}{2.369036in}}{\pgfqpoint{2.270500in}{2.358437in}}{\pgfqpoint{2.270500in}{2.347387in}}%
\pgfpathcurveto{\pgfqpoint{2.270500in}{2.336337in}}{\pgfqpoint{2.274890in}{2.325738in}}{\pgfqpoint{2.282703in}{2.317924in}}%
\pgfpathcurveto{\pgfqpoint{2.290517in}{2.310111in}}{\pgfqpoint{2.301116in}{2.305721in}}{\pgfqpoint{2.312166in}{2.305721in}}%
\pgfpathclose%
\pgfusepath{stroke,fill}%
\end{pgfscope}%
\begin{pgfscope}%
\pgfpathrectangle{\pgfqpoint{0.600000in}{0.600000in}}{\pgfqpoint{3.900000in}{3.900000in}}%
\pgfusepath{clip}%
\pgfsetbuttcap%
\pgfsetroundjoin%
\definecolor{currentfill}{rgb}{0.121569,0.466667,0.705882}%
\pgfsetfillcolor{currentfill}%
\pgfsetlinewidth{1.003750pt}%
\definecolor{currentstroke}{rgb}{0.121569,0.466667,0.705882}%
\pgfsetstrokecolor{currentstroke}%
\pgfsetdash{}{0pt}%
\pgfpathmoveto{\pgfqpoint{3.003924in}{2.547444in}}%
\pgfpathcurveto{\pgfqpoint{3.014974in}{2.547444in}}{\pgfqpoint{3.025573in}{2.551834in}}{\pgfqpoint{3.033386in}{2.559648in}}%
\pgfpathcurveto{\pgfqpoint{3.041200in}{2.567461in}}{\pgfqpoint{3.045590in}{2.578060in}}{\pgfqpoint{3.045590in}{2.589111in}}%
\pgfpathcurveto{\pgfqpoint{3.045590in}{2.600161in}}{\pgfqpoint{3.041200in}{2.610760in}}{\pgfqpoint{3.033386in}{2.618573in}}%
\pgfpathcurveto{\pgfqpoint{3.025573in}{2.626387in}}{\pgfqpoint{3.014974in}{2.630777in}}{\pgfqpoint{3.003924in}{2.630777in}}%
\pgfpathcurveto{\pgfqpoint{2.992873in}{2.630777in}}{\pgfqpoint{2.982274in}{2.626387in}}{\pgfqpoint{2.974461in}{2.618573in}}%
\pgfpathcurveto{\pgfqpoint{2.966647in}{2.610760in}}{\pgfqpoint{2.962257in}{2.600161in}}{\pgfqpoint{2.962257in}{2.589111in}}%
\pgfpathcurveto{\pgfqpoint{2.962257in}{2.578060in}}{\pgfqpoint{2.966647in}{2.567461in}}{\pgfqpoint{2.974461in}{2.559648in}}%
\pgfpathcurveto{\pgfqpoint{2.982274in}{2.551834in}}{\pgfqpoint{2.992873in}{2.547444in}}{\pgfqpoint{3.003924in}{2.547444in}}%
\pgfpathclose%
\pgfusepath{stroke,fill}%
\end{pgfscope}%
\begin{pgfscope}%
\pgfpathrectangle{\pgfqpoint{0.600000in}{0.600000in}}{\pgfqpoint{3.900000in}{3.900000in}}%
\pgfusepath{clip}%
\pgfsetbuttcap%
\pgfsetroundjoin%
\definecolor{currentfill}{rgb}{0.121569,0.466667,0.705882}%
\pgfsetfillcolor{currentfill}%
\pgfsetlinewidth{1.003750pt}%
\definecolor{currentstroke}{rgb}{0.121569,0.466667,0.705882}%
\pgfsetstrokecolor{currentstroke}%
\pgfsetdash{}{0pt}%
\pgfpathmoveto{\pgfqpoint{2.570735in}{2.532131in}}%
\pgfpathcurveto{\pgfqpoint{2.581785in}{2.532131in}}{\pgfqpoint{2.592384in}{2.536521in}}{\pgfqpoint{2.600198in}{2.544335in}}%
\pgfpathcurveto{\pgfqpoint{2.608011in}{2.552149in}}{\pgfqpoint{2.612402in}{2.562748in}}{\pgfqpoint{2.612402in}{2.573798in}}%
\pgfpathcurveto{\pgfqpoint{2.612402in}{2.584848in}}{\pgfqpoint{2.608011in}{2.595447in}}{\pgfqpoint{2.600198in}{2.603261in}}%
\pgfpathcurveto{\pgfqpoint{2.592384in}{2.611074in}}{\pgfqpoint{2.581785in}{2.615465in}}{\pgfqpoint{2.570735in}{2.615465in}}%
\pgfpathcurveto{\pgfqpoint{2.559685in}{2.615465in}}{\pgfqpoint{2.549086in}{2.611074in}}{\pgfqpoint{2.541272in}{2.603261in}}%
\pgfpathcurveto{\pgfqpoint{2.533459in}{2.595447in}}{\pgfqpoint{2.529068in}{2.584848in}}{\pgfqpoint{2.529068in}{2.573798in}}%
\pgfpathcurveto{\pgfqpoint{2.529068in}{2.562748in}}{\pgfqpoint{2.533459in}{2.552149in}}{\pgfqpoint{2.541272in}{2.544335in}}%
\pgfpathcurveto{\pgfqpoint{2.549086in}{2.536521in}}{\pgfqpoint{2.559685in}{2.532131in}}{\pgfqpoint{2.570735in}{2.532131in}}%
\pgfpathclose%
\pgfusepath{stroke,fill}%
\end{pgfscope}%
\begin{pgfscope}%
\pgfpathrectangle{\pgfqpoint{0.600000in}{0.600000in}}{\pgfqpoint{3.900000in}{3.900000in}}%
\pgfusepath{clip}%
\pgfsetbuttcap%
\pgfsetroundjoin%
\definecolor{currentfill}{rgb}{0.121569,0.466667,0.705882}%
\pgfsetfillcolor{currentfill}%
\pgfsetlinewidth{1.003750pt}%
\definecolor{currentstroke}{rgb}{0.121569,0.466667,0.705882}%
\pgfsetstrokecolor{currentstroke}%
\pgfsetdash{}{0pt}%
\pgfpathmoveto{\pgfqpoint{1.359364in}{3.045378in}}%
\pgfpathcurveto{\pgfqpoint{1.370414in}{3.045378in}}{\pgfqpoint{1.381013in}{3.049768in}}{\pgfqpoint{1.388827in}{3.057581in}}%
\pgfpathcurveto{\pgfqpoint{1.396641in}{3.065395in}}{\pgfqpoint{1.401031in}{3.075994in}}{\pgfqpoint{1.401031in}{3.087044in}}%
\pgfpathcurveto{\pgfqpoint{1.401031in}{3.098094in}}{\pgfqpoint{1.396641in}{3.108693in}}{\pgfqpoint{1.388827in}{3.116507in}}%
\pgfpathcurveto{\pgfqpoint{1.381013in}{3.124321in}}{\pgfqpoint{1.370414in}{3.128711in}}{\pgfqpoint{1.359364in}{3.128711in}}%
\pgfpathcurveto{\pgfqpoint{1.348314in}{3.128711in}}{\pgfqpoint{1.337715in}{3.124321in}}{\pgfqpoint{1.329901in}{3.116507in}}%
\pgfpathcurveto{\pgfqpoint{1.322088in}{3.108693in}}{\pgfqpoint{1.317698in}{3.098094in}}{\pgfqpoint{1.317698in}{3.087044in}}%
\pgfpathcurveto{\pgfqpoint{1.317698in}{3.075994in}}{\pgfqpoint{1.322088in}{3.065395in}}{\pgfqpoint{1.329901in}{3.057581in}}%
\pgfpathcurveto{\pgfqpoint{1.337715in}{3.049768in}}{\pgfqpoint{1.348314in}{3.045378in}}{\pgfqpoint{1.359364in}{3.045378in}}%
\pgfpathclose%
\pgfusepath{stroke,fill}%
\end{pgfscope}%
\begin{pgfscope}%
\pgfpathrectangle{\pgfqpoint{0.600000in}{0.600000in}}{\pgfqpoint{3.900000in}{3.900000in}}%
\pgfusepath{clip}%
\pgfsetbuttcap%
\pgfsetroundjoin%
\definecolor{currentfill}{rgb}{0.121569,0.466667,0.705882}%
\pgfsetfillcolor{currentfill}%
\pgfsetlinewidth{1.003750pt}%
\definecolor{currentstroke}{rgb}{0.121569,0.466667,0.705882}%
\pgfsetstrokecolor{currentstroke}%
\pgfsetdash{}{0pt}%
\pgfpathmoveto{\pgfqpoint{2.605032in}{2.121591in}}%
\pgfpathcurveto{\pgfqpoint{2.616082in}{2.121591in}}{\pgfqpoint{2.626681in}{2.125981in}}{\pgfqpoint{2.634495in}{2.133795in}}%
\pgfpathcurveto{\pgfqpoint{2.642308in}{2.141608in}}{\pgfqpoint{2.646698in}{2.152207in}}{\pgfqpoint{2.646698in}{2.163258in}}%
\pgfpathcurveto{\pgfqpoint{2.646698in}{2.174308in}}{\pgfqpoint{2.642308in}{2.184907in}}{\pgfqpoint{2.634495in}{2.192720in}}%
\pgfpathcurveto{\pgfqpoint{2.626681in}{2.200534in}}{\pgfqpoint{2.616082in}{2.204924in}}{\pgfqpoint{2.605032in}{2.204924in}}%
\pgfpathcurveto{\pgfqpoint{2.593982in}{2.204924in}}{\pgfqpoint{2.583383in}{2.200534in}}{\pgfqpoint{2.575569in}{2.192720in}}%
\pgfpathcurveto{\pgfqpoint{2.567755in}{2.184907in}}{\pgfqpoint{2.563365in}{2.174308in}}{\pgfqpoint{2.563365in}{2.163258in}}%
\pgfpathcurveto{\pgfqpoint{2.563365in}{2.152207in}}{\pgfqpoint{2.567755in}{2.141608in}}{\pgfqpoint{2.575569in}{2.133795in}}%
\pgfpathcurveto{\pgfqpoint{2.583383in}{2.125981in}}{\pgfqpoint{2.593982in}{2.121591in}}{\pgfqpoint{2.605032in}{2.121591in}}%
\pgfpathclose%
\pgfusepath{stroke,fill}%
\end{pgfscope}%
\begin{pgfscope}%
\pgfpathrectangle{\pgfqpoint{0.600000in}{0.600000in}}{\pgfqpoint{3.900000in}{3.900000in}}%
\pgfusepath{clip}%
\pgfsetbuttcap%
\pgfsetroundjoin%
\definecolor{currentfill}{rgb}{0.121569,0.466667,0.705882}%
\pgfsetfillcolor{currentfill}%
\pgfsetlinewidth{1.003750pt}%
\definecolor{currentstroke}{rgb}{0.121569,0.466667,0.705882}%
\pgfsetstrokecolor{currentstroke}%
\pgfsetdash{}{0pt}%
\pgfpathmoveto{\pgfqpoint{3.038247in}{2.627073in}}%
\pgfpathcurveto{\pgfqpoint{3.049297in}{2.627073in}}{\pgfqpoint{3.059896in}{2.631463in}}{\pgfqpoint{3.067710in}{2.639276in}}%
\pgfpathcurveto{\pgfqpoint{3.075524in}{2.647090in}}{\pgfqpoint{3.079914in}{2.657689in}}{\pgfqpoint{3.079914in}{2.668739in}}%
\pgfpathcurveto{\pgfqpoint{3.079914in}{2.679789in}}{\pgfqpoint{3.075524in}{2.690388in}}{\pgfqpoint{3.067710in}{2.698202in}}%
\pgfpathcurveto{\pgfqpoint{3.059896in}{2.706016in}}{\pgfqpoint{3.049297in}{2.710406in}}{\pgfqpoint{3.038247in}{2.710406in}}%
\pgfpathcurveto{\pgfqpoint{3.027197in}{2.710406in}}{\pgfqpoint{3.016598in}{2.706016in}}{\pgfqpoint{3.008784in}{2.698202in}}%
\pgfpathcurveto{\pgfqpoint{3.000971in}{2.690388in}}{\pgfqpoint{2.996580in}{2.679789in}}{\pgfqpoint{2.996580in}{2.668739in}}%
\pgfpathcurveto{\pgfqpoint{2.996580in}{2.657689in}}{\pgfqpoint{3.000971in}{2.647090in}}{\pgfqpoint{3.008784in}{2.639276in}}%
\pgfpathcurveto{\pgfqpoint{3.016598in}{2.631463in}}{\pgfqpoint{3.027197in}{2.627073in}}{\pgfqpoint{3.038247in}{2.627073in}}%
\pgfpathclose%
\pgfusepath{stroke,fill}%
\end{pgfscope}%
\begin{pgfscope}%
\pgfpathrectangle{\pgfqpoint{0.600000in}{0.600000in}}{\pgfqpoint{3.900000in}{3.900000in}}%
\pgfusepath{clip}%
\pgfsetbuttcap%
\pgfsetroundjoin%
\definecolor{currentfill}{rgb}{0.121569,0.466667,0.705882}%
\pgfsetfillcolor{currentfill}%
\pgfsetlinewidth{1.003750pt}%
\definecolor{currentstroke}{rgb}{0.121569,0.466667,0.705882}%
\pgfsetstrokecolor{currentstroke}%
\pgfsetdash{}{0pt}%
\pgfpathmoveto{\pgfqpoint{3.410207in}{2.895259in}}%
\pgfpathcurveto{\pgfqpoint{3.421257in}{2.895259in}}{\pgfqpoint{3.431856in}{2.899649in}}{\pgfqpoint{3.439669in}{2.907463in}}%
\pgfpathcurveto{\pgfqpoint{3.447483in}{2.915276in}}{\pgfqpoint{3.451873in}{2.925875in}}{\pgfqpoint{3.451873in}{2.936926in}}%
\pgfpathcurveto{\pgfqpoint{3.451873in}{2.947976in}}{\pgfqpoint{3.447483in}{2.958575in}}{\pgfqpoint{3.439669in}{2.966388in}}%
\pgfpathcurveto{\pgfqpoint{3.431856in}{2.974202in}}{\pgfqpoint{3.421257in}{2.978592in}}{\pgfqpoint{3.410207in}{2.978592in}}%
\pgfpathcurveto{\pgfqpoint{3.399156in}{2.978592in}}{\pgfqpoint{3.388557in}{2.974202in}}{\pgfqpoint{3.380744in}{2.966388in}}%
\pgfpathcurveto{\pgfqpoint{3.372930in}{2.958575in}}{\pgfqpoint{3.368540in}{2.947976in}}{\pgfqpoint{3.368540in}{2.936926in}}%
\pgfpathcurveto{\pgfqpoint{3.368540in}{2.925875in}}{\pgfqpoint{3.372930in}{2.915276in}}{\pgfqpoint{3.380744in}{2.907463in}}%
\pgfpathcurveto{\pgfqpoint{3.388557in}{2.899649in}}{\pgfqpoint{3.399156in}{2.895259in}}{\pgfqpoint{3.410207in}{2.895259in}}%
\pgfpathclose%
\pgfusepath{stroke,fill}%
\end{pgfscope}%
\begin{pgfscope}%
\pgfpathrectangle{\pgfqpoint{0.600000in}{0.600000in}}{\pgfqpoint{3.900000in}{3.900000in}}%
\pgfusepath{clip}%
\pgfsetbuttcap%
\pgfsetroundjoin%
\definecolor{currentfill}{rgb}{0.121569,0.466667,0.705882}%
\pgfsetfillcolor{currentfill}%
\pgfsetlinewidth{1.003750pt}%
\definecolor{currentstroke}{rgb}{0.121569,0.466667,0.705882}%
\pgfsetstrokecolor{currentstroke}%
\pgfsetdash{}{0pt}%
\pgfpathmoveto{\pgfqpoint{1.944991in}{2.146056in}}%
\pgfpathcurveto{\pgfqpoint{1.956041in}{2.146056in}}{\pgfqpoint{1.966640in}{2.150446in}}{\pgfqpoint{1.974454in}{2.158259in}}%
\pgfpathcurveto{\pgfqpoint{1.982267in}{2.166073in}}{\pgfqpoint{1.986658in}{2.176672in}}{\pgfqpoint{1.986658in}{2.187722in}}%
\pgfpathcurveto{\pgfqpoint{1.986658in}{2.198772in}}{\pgfqpoint{1.982267in}{2.209371in}}{\pgfqpoint{1.974454in}{2.217185in}}%
\pgfpathcurveto{\pgfqpoint{1.966640in}{2.224999in}}{\pgfqpoint{1.956041in}{2.229389in}}{\pgfqpoint{1.944991in}{2.229389in}}%
\pgfpathcurveto{\pgfqpoint{1.933941in}{2.229389in}}{\pgfqpoint{1.923342in}{2.224999in}}{\pgfqpoint{1.915528in}{2.217185in}}%
\pgfpathcurveto{\pgfqpoint{1.907715in}{2.209371in}}{\pgfqpoint{1.903324in}{2.198772in}}{\pgfqpoint{1.903324in}{2.187722in}}%
\pgfpathcurveto{\pgfqpoint{1.903324in}{2.176672in}}{\pgfqpoint{1.907715in}{2.166073in}}{\pgfqpoint{1.915528in}{2.158259in}}%
\pgfpathcurveto{\pgfqpoint{1.923342in}{2.150446in}}{\pgfqpoint{1.933941in}{2.146056in}}{\pgfqpoint{1.944991in}{2.146056in}}%
\pgfpathclose%
\pgfusepath{stroke,fill}%
\end{pgfscope}%
\begin{pgfscope}%
\pgfpathrectangle{\pgfqpoint{0.600000in}{0.600000in}}{\pgfqpoint{3.900000in}{3.900000in}}%
\pgfusepath{clip}%
\pgfsetbuttcap%
\pgfsetroundjoin%
\definecolor{currentfill}{rgb}{0.121569,0.466667,0.705882}%
\pgfsetfillcolor{currentfill}%
\pgfsetlinewidth{1.003750pt}%
\definecolor{currentstroke}{rgb}{0.121569,0.466667,0.705882}%
\pgfsetstrokecolor{currentstroke}%
\pgfsetdash{}{0pt}%
\pgfpathmoveto{\pgfqpoint{2.576595in}{2.501342in}}%
\pgfpathcurveto{\pgfqpoint{2.587645in}{2.501342in}}{\pgfqpoint{2.598244in}{2.505732in}}{\pgfqpoint{2.606058in}{2.513546in}}%
\pgfpathcurveto{\pgfqpoint{2.613871in}{2.521359in}}{\pgfqpoint{2.618261in}{2.531959in}}{\pgfqpoint{2.618261in}{2.543009in}}%
\pgfpathcurveto{\pgfqpoint{2.618261in}{2.554059in}}{\pgfqpoint{2.613871in}{2.564658in}}{\pgfqpoint{2.606058in}{2.572471in}}%
\pgfpathcurveto{\pgfqpoint{2.598244in}{2.580285in}}{\pgfqpoint{2.587645in}{2.584675in}}{\pgfqpoint{2.576595in}{2.584675in}}%
\pgfpathcurveto{\pgfqpoint{2.565545in}{2.584675in}}{\pgfqpoint{2.554946in}{2.580285in}}{\pgfqpoint{2.547132in}{2.572471in}}%
\pgfpathcurveto{\pgfqpoint{2.539318in}{2.564658in}}{\pgfqpoint{2.534928in}{2.554059in}}{\pgfqpoint{2.534928in}{2.543009in}}%
\pgfpathcurveto{\pgfqpoint{2.534928in}{2.531959in}}{\pgfqpoint{2.539318in}{2.521359in}}{\pgfqpoint{2.547132in}{2.513546in}}%
\pgfpathcurveto{\pgfqpoint{2.554946in}{2.505732in}}{\pgfqpoint{2.565545in}{2.501342in}}{\pgfqpoint{2.576595in}{2.501342in}}%
\pgfpathclose%
\pgfusepath{stroke,fill}%
\end{pgfscope}%
\begin{pgfscope}%
\pgfpathrectangle{\pgfqpoint{0.600000in}{0.600000in}}{\pgfqpoint{3.900000in}{3.900000in}}%
\pgfusepath{clip}%
\pgfsetbuttcap%
\pgfsetroundjoin%
\definecolor{currentfill}{rgb}{0.121569,0.466667,0.705882}%
\pgfsetfillcolor{currentfill}%
\pgfsetlinewidth{1.003750pt}%
\definecolor{currentstroke}{rgb}{0.121569,0.466667,0.705882}%
\pgfsetstrokecolor{currentstroke}%
\pgfsetdash{}{0pt}%
\pgfpathmoveto{\pgfqpoint{3.064164in}{1.656614in}}%
\pgfpathcurveto{\pgfqpoint{3.075214in}{1.656614in}}{\pgfqpoint{3.085813in}{1.661004in}}{\pgfqpoint{3.093627in}{1.668817in}}%
\pgfpathcurveto{\pgfqpoint{3.101440in}{1.676631in}}{\pgfqpoint{3.105831in}{1.687230in}}{\pgfqpoint{3.105831in}{1.698280in}}%
\pgfpathcurveto{\pgfqpoint{3.105831in}{1.709330in}}{\pgfqpoint{3.101440in}{1.719929in}}{\pgfqpoint{3.093627in}{1.727743in}}%
\pgfpathcurveto{\pgfqpoint{3.085813in}{1.735557in}}{\pgfqpoint{3.075214in}{1.739947in}}{\pgfqpoint{3.064164in}{1.739947in}}%
\pgfpathcurveto{\pgfqpoint{3.053114in}{1.739947in}}{\pgfqpoint{3.042515in}{1.735557in}}{\pgfqpoint{3.034701in}{1.727743in}}%
\pgfpathcurveto{\pgfqpoint{3.026888in}{1.719929in}}{\pgfqpoint{3.022497in}{1.709330in}}{\pgfqpoint{3.022497in}{1.698280in}}%
\pgfpathcurveto{\pgfqpoint{3.022497in}{1.687230in}}{\pgfqpoint{3.026888in}{1.676631in}}{\pgfqpoint{3.034701in}{1.668817in}}%
\pgfpathcurveto{\pgfqpoint{3.042515in}{1.661004in}}{\pgfqpoint{3.053114in}{1.656614in}}{\pgfqpoint{3.064164in}{1.656614in}}%
\pgfpathclose%
\pgfusepath{stroke,fill}%
\end{pgfscope}%
\begin{pgfscope}%
\pgfpathrectangle{\pgfqpoint{0.600000in}{0.600000in}}{\pgfqpoint{3.900000in}{3.900000in}}%
\pgfusepath{clip}%
\pgfsetbuttcap%
\pgfsetroundjoin%
\definecolor{currentfill}{rgb}{0.121569,0.466667,0.705882}%
\pgfsetfillcolor{currentfill}%
\pgfsetlinewidth{1.003750pt}%
\definecolor{currentstroke}{rgb}{0.121569,0.466667,0.705882}%
\pgfsetstrokecolor{currentstroke}%
\pgfsetdash{}{0pt}%
\pgfpathmoveto{\pgfqpoint{3.182616in}{2.766778in}}%
\pgfpathcurveto{\pgfqpoint{3.193666in}{2.766778in}}{\pgfqpoint{3.204265in}{2.771168in}}{\pgfqpoint{3.212079in}{2.778982in}}%
\pgfpathcurveto{\pgfqpoint{3.219892in}{2.786795in}}{\pgfqpoint{3.224283in}{2.797394in}}{\pgfqpoint{3.224283in}{2.808445in}}%
\pgfpathcurveto{\pgfqpoint{3.224283in}{2.819495in}}{\pgfqpoint{3.219892in}{2.830094in}}{\pgfqpoint{3.212079in}{2.837907in}}%
\pgfpathcurveto{\pgfqpoint{3.204265in}{2.845721in}}{\pgfqpoint{3.193666in}{2.850111in}}{\pgfqpoint{3.182616in}{2.850111in}}%
\pgfpathcurveto{\pgfqpoint{3.171566in}{2.850111in}}{\pgfqpoint{3.160967in}{2.845721in}}{\pgfqpoint{3.153153in}{2.837907in}}%
\pgfpathcurveto{\pgfqpoint{3.145340in}{2.830094in}}{\pgfqpoint{3.140949in}{2.819495in}}{\pgfqpoint{3.140949in}{2.808445in}}%
\pgfpathcurveto{\pgfqpoint{3.140949in}{2.797394in}}{\pgfqpoint{3.145340in}{2.786795in}}{\pgfqpoint{3.153153in}{2.778982in}}%
\pgfpathcurveto{\pgfqpoint{3.160967in}{2.771168in}}{\pgfqpoint{3.171566in}{2.766778in}}{\pgfqpoint{3.182616in}{2.766778in}}%
\pgfpathclose%
\pgfusepath{stroke,fill}%
\end{pgfscope}%
\begin{pgfscope}%
\pgfpathrectangle{\pgfqpoint{0.600000in}{0.600000in}}{\pgfqpoint{3.900000in}{3.900000in}}%
\pgfusepath{clip}%
\pgfsetbuttcap%
\pgfsetroundjoin%
\definecolor{currentfill}{rgb}{0.121569,0.466667,0.705882}%
\pgfsetfillcolor{currentfill}%
\pgfsetlinewidth{1.003750pt}%
\definecolor{currentstroke}{rgb}{0.121569,0.466667,0.705882}%
\pgfsetstrokecolor{currentstroke}%
\pgfsetdash{}{0pt}%
\pgfpathmoveto{\pgfqpoint{3.375892in}{1.945777in}}%
\pgfpathcurveto{\pgfqpoint{3.386942in}{1.945777in}}{\pgfqpoint{3.397541in}{1.950167in}}{\pgfqpoint{3.405355in}{1.957981in}}%
\pgfpathcurveto{\pgfqpoint{3.413168in}{1.965794in}}{\pgfqpoint{3.417559in}{1.976394in}}{\pgfqpoint{3.417559in}{1.987444in}}%
\pgfpathcurveto{\pgfqpoint{3.417559in}{1.998494in}}{\pgfqpoint{3.413168in}{2.009093in}}{\pgfqpoint{3.405355in}{2.016906in}}%
\pgfpathcurveto{\pgfqpoint{3.397541in}{2.024720in}}{\pgfqpoint{3.386942in}{2.029110in}}{\pgfqpoint{3.375892in}{2.029110in}}%
\pgfpathcurveto{\pgfqpoint{3.364842in}{2.029110in}}{\pgfqpoint{3.354243in}{2.024720in}}{\pgfqpoint{3.346429in}{2.016906in}}%
\pgfpathcurveto{\pgfqpoint{3.338615in}{2.009093in}}{\pgfqpoint{3.334225in}{1.998494in}}{\pgfqpoint{3.334225in}{1.987444in}}%
\pgfpathcurveto{\pgfqpoint{3.334225in}{1.976394in}}{\pgfqpoint{3.338615in}{1.965794in}}{\pgfqpoint{3.346429in}{1.957981in}}%
\pgfpathcurveto{\pgfqpoint{3.354243in}{1.950167in}}{\pgfqpoint{3.364842in}{1.945777in}}{\pgfqpoint{3.375892in}{1.945777in}}%
\pgfpathclose%
\pgfusepath{stroke,fill}%
\end{pgfscope}%
\begin{pgfscope}%
\pgfpathrectangle{\pgfqpoint{0.600000in}{0.600000in}}{\pgfqpoint{3.900000in}{3.900000in}}%
\pgfusepath{clip}%
\pgfsetbuttcap%
\pgfsetroundjoin%
\definecolor{currentfill}{rgb}{0.121569,0.466667,0.705882}%
\pgfsetfillcolor{currentfill}%
\pgfsetlinewidth{1.003750pt}%
\definecolor{currentstroke}{rgb}{0.121569,0.466667,0.705882}%
\pgfsetstrokecolor{currentstroke}%
\pgfsetdash{}{0pt}%
\pgfpathmoveto{\pgfqpoint{2.138583in}{2.653750in}}%
\pgfpathcurveto{\pgfqpoint{2.149634in}{2.653750in}}{\pgfqpoint{2.160233in}{2.658140in}}{\pgfqpoint{2.168046in}{2.665954in}}%
\pgfpathcurveto{\pgfqpoint{2.175860in}{2.673767in}}{\pgfqpoint{2.180250in}{2.684366in}}{\pgfqpoint{2.180250in}{2.695416in}}%
\pgfpathcurveto{\pgfqpoint{2.180250in}{2.706466in}}{\pgfqpoint{2.175860in}{2.717066in}}{\pgfqpoint{2.168046in}{2.724879in}}%
\pgfpathcurveto{\pgfqpoint{2.160233in}{2.732693in}}{\pgfqpoint{2.149634in}{2.737083in}}{\pgfqpoint{2.138583in}{2.737083in}}%
\pgfpathcurveto{\pgfqpoint{2.127533in}{2.737083in}}{\pgfqpoint{2.116934in}{2.732693in}}{\pgfqpoint{2.109121in}{2.724879in}}%
\pgfpathcurveto{\pgfqpoint{2.101307in}{2.717066in}}{\pgfqpoint{2.096917in}{2.706466in}}{\pgfqpoint{2.096917in}{2.695416in}}%
\pgfpathcurveto{\pgfqpoint{2.096917in}{2.684366in}}{\pgfqpoint{2.101307in}{2.673767in}}{\pgfqpoint{2.109121in}{2.665954in}}%
\pgfpathcurveto{\pgfqpoint{2.116934in}{2.658140in}}{\pgfqpoint{2.127533in}{2.653750in}}{\pgfqpoint{2.138583in}{2.653750in}}%
\pgfpathclose%
\pgfusepath{stroke,fill}%
\end{pgfscope}%
\begin{pgfscope}%
\pgfpathrectangle{\pgfqpoint{0.600000in}{0.600000in}}{\pgfqpoint{3.900000in}{3.900000in}}%
\pgfusepath{clip}%
\pgfsetbuttcap%
\pgfsetroundjoin%
\definecolor{currentfill}{rgb}{0.121569,0.466667,0.705882}%
\pgfsetfillcolor{currentfill}%
\pgfsetlinewidth{1.003750pt}%
\definecolor{currentstroke}{rgb}{0.121569,0.466667,0.705882}%
\pgfsetstrokecolor{currentstroke}%
\pgfsetdash{}{0pt}%
\pgfpathmoveto{\pgfqpoint{2.845779in}{3.301057in}}%
\pgfpathcurveto{\pgfqpoint{2.856829in}{3.301057in}}{\pgfqpoint{2.867429in}{3.305447in}}{\pgfqpoint{2.875242in}{3.313261in}}%
\pgfpathcurveto{\pgfqpoint{2.883056in}{3.321074in}}{\pgfqpoint{2.887446in}{3.331673in}}{\pgfqpoint{2.887446in}{3.342723in}}%
\pgfpathcurveto{\pgfqpoint{2.887446in}{3.353774in}}{\pgfqpoint{2.883056in}{3.364373in}}{\pgfqpoint{2.875242in}{3.372186in}}%
\pgfpathcurveto{\pgfqpoint{2.867429in}{3.380000in}}{\pgfqpoint{2.856829in}{3.384390in}}{\pgfqpoint{2.845779in}{3.384390in}}%
\pgfpathcurveto{\pgfqpoint{2.834729in}{3.384390in}}{\pgfqpoint{2.824130in}{3.380000in}}{\pgfqpoint{2.816317in}{3.372186in}}%
\pgfpathcurveto{\pgfqpoint{2.808503in}{3.364373in}}{\pgfqpoint{2.804113in}{3.353774in}}{\pgfqpoint{2.804113in}{3.342723in}}%
\pgfpathcurveto{\pgfqpoint{2.804113in}{3.331673in}}{\pgfqpoint{2.808503in}{3.321074in}}{\pgfqpoint{2.816317in}{3.313261in}}%
\pgfpathcurveto{\pgfqpoint{2.824130in}{3.305447in}}{\pgfqpoint{2.834729in}{3.301057in}}{\pgfqpoint{2.845779in}{3.301057in}}%
\pgfpathclose%
\pgfusepath{stroke,fill}%
\end{pgfscope}%
\begin{pgfscope}%
\pgfpathrectangle{\pgfqpoint{0.600000in}{0.600000in}}{\pgfqpoint{3.900000in}{3.900000in}}%
\pgfusepath{clip}%
\pgfsetbuttcap%
\pgfsetroundjoin%
\definecolor{currentfill}{rgb}{0.121569,0.466667,0.705882}%
\pgfsetfillcolor{currentfill}%
\pgfsetlinewidth{1.003750pt}%
\definecolor{currentstroke}{rgb}{0.121569,0.466667,0.705882}%
\pgfsetstrokecolor{currentstroke}%
\pgfsetdash{}{0pt}%
\pgfpathmoveto{\pgfqpoint{2.481875in}{1.759153in}}%
\pgfpathcurveto{\pgfqpoint{2.492925in}{1.759153in}}{\pgfqpoint{2.503524in}{1.763543in}}{\pgfqpoint{2.511338in}{1.771357in}}%
\pgfpathcurveto{\pgfqpoint{2.519152in}{1.779171in}}{\pgfqpoint{2.523542in}{1.789770in}}{\pgfqpoint{2.523542in}{1.800820in}}%
\pgfpathcurveto{\pgfqpoint{2.523542in}{1.811870in}}{\pgfqpoint{2.519152in}{1.822469in}}{\pgfqpoint{2.511338in}{1.830282in}}%
\pgfpathcurveto{\pgfqpoint{2.503524in}{1.838096in}}{\pgfqpoint{2.492925in}{1.842486in}}{\pgfqpoint{2.481875in}{1.842486in}}%
\pgfpathcurveto{\pgfqpoint{2.470825in}{1.842486in}}{\pgfqpoint{2.460226in}{1.838096in}}{\pgfqpoint{2.452412in}{1.830282in}}%
\pgfpathcurveto{\pgfqpoint{2.444599in}{1.822469in}}{\pgfqpoint{2.440209in}{1.811870in}}{\pgfqpoint{2.440209in}{1.800820in}}%
\pgfpathcurveto{\pgfqpoint{2.440209in}{1.789770in}}{\pgfqpoint{2.444599in}{1.779171in}}{\pgfqpoint{2.452412in}{1.771357in}}%
\pgfpathcurveto{\pgfqpoint{2.460226in}{1.763543in}}{\pgfqpoint{2.470825in}{1.759153in}}{\pgfqpoint{2.481875in}{1.759153in}}%
\pgfpathclose%
\pgfusepath{stroke,fill}%
\end{pgfscope}%
\begin{pgfscope}%
\pgfpathrectangle{\pgfqpoint{0.600000in}{0.600000in}}{\pgfqpoint{3.900000in}{3.900000in}}%
\pgfusepath{clip}%
\pgfsetbuttcap%
\pgfsetroundjoin%
\definecolor{currentfill}{rgb}{0.121569,0.466667,0.705882}%
\pgfsetfillcolor{currentfill}%
\pgfsetlinewidth{1.003750pt}%
\definecolor{currentstroke}{rgb}{0.121569,0.466667,0.705882}%
\pgfsetstrokecolor{currentstroke}%
\pgfsetdash{}{0pt}%
\pgfpathmoveto{\pgfqpoint{2.980911in}{1.899753in}}%
\pgfpathcurveto{\pgfqpoint{2.991961in}{1.899753in}}{\pgfqpoint{3.002561in}{1.904143in}}{\pgfqpoint{3.010374in}{1.911957in}}%
\pgfpathcurveto{\pgfqpoint{3.018188in}{1.919770in}}{\pgfqpoint{3.022578in}{1.930369in}}{\pgfqpoint{3.022578in}{1.941420in}}%
\pgfpathcurveto{\pgfqpoint{3.022578in}{1.952470in}}{\pgfqpoint{3.018188in}{1.963069in}}{\pgfqpoint{3.010374in}{1.970882in}}%
\pgfpathcurveto{\pgfqpoint{3.002561in}{1.978696in}}{\pgfqpoint{2.991961in}{1.983086in}}{\pgfqpoint{2.980911in}{1.983086in}}%
\pgfpathcurveto{\pgfqpoint{2.969861in}{1.983086in}}{\pgfqpoint{2.959262in}{1.978696in}}{\pgfqpoint{2.951449in}{1.970882in}}%
\pgfpathcurveto{\pgfqpoint{2.943635in}{1.963069in}}{\pgfqpoint{2.939245in}{1.952470in}}{\pgfqpoint{2.939245in}{1.941420in}}%
\pgfpathcurveto{\pgfqpoint{2.939245in}{1.930369in}}{\pgfqpoint{2.943635in}{1.919770in}}{\pgfqpoint{2.951449in}{1.911957in}}%
\pgfpathcurveto{\pgfqpoint{2.959262in}{1.904143in}}{\pgfqpoint{2.969861in}{1.899753in}}{\pgfqpoint{2.980911in}{1.899753in}}%
\pgfpathclose%
\pgfusepath{stroke,fill}%
\end{pgfscope}%
\begin{pgfscope}%
\pgfpathrectangle{\pgfqpoint{0.600000in}{0.600000in}}{\pgfqpoint{3.900000in}{3.900000in}}%
\pgfusepath{clip}%
\pgfsetbuttcap%
\pgfsetroundjoin%
\definecolor{currentfill}{rgb}{0.121569,0.466667,0.705882}%
\pgfsetfillcolor{currentfill}%
\pgfsetlinewidth{1.003750pt}%
\definecolor{currentstroke}{rgb}{0.121569,0.466667,0.705882}%
\pgfsetstrokecolor{currentstroke}%
\pgfsetdash{}{0pt}%
\pgfpathmoveto{\pgfqpoint{1.933903in}{1.943428in}}%
\pgfpathcurveto{\pgfqpoint{1.944953in}{1.943428in}}{\pgfqpoint{1.955552in}{1.947818in}}{\pgfqpoint{1.963366in}{1.955631in}}%
\pgfpathcurveto{\pgfqpoint{1.971179in}{1.963445in}}{\pgfqpoint{1.975570in}{1.974044in}}{\pgfqpoint{1.975570in}{1.985094in}}%
\pgfpathcurveto{\pgfqpoint{1.975570in}{1.996144in}}{\pgfqpoint{1.971179in}{2.006743in}}{\pgfqpoint{1.963366in}{2.014557in}}%
\pgfpathcurveto{\pgfqpoint{1.955552in}{2.022371in}}{\pgfqpoint{1.944953in}{2.026761in}}{\pgfqpoint{1.933903in}{2.026761in}}%
\pgfpathcurveto{\pgfqpoint{1.922853in}{2.026761in}}{\pgfqpoint{1.912254in}{2.022371in}}{\pgfqpoint{1.904440in}{2.014557in}}%
\pgfpathcurveto{\pgfqpoint{1.896626in}{2.006743in}}{\pgfqpoint{1.892236in}{1.996144in}}{\pgfqpoint{1.892236in}{1.985094in}}%
\pgfpathcurveto{\pgfqpoint{1.892236in}{1.974044in}}{\pgfqpoint{1.896626in}{1.963445in}}{\pgfqpoint{1.904440in}{1.955631in}}%
\pgfpathcurveto{\pgfqpoint{1.912254in}{1.947818in}}{\pgfqpoint{1.922853in}{1.943428in}}{\pgfqpoint{1.933903in}{1.943428in}}%
\pgfpathclose%
\pgfusepath{stroke,fill}%
\end{pgfscope}%
\begin{pgfscope}%
\pgfpathrectangle{\pgfqpoint{0.600000in}{0.600000in}}{\pgfqpoint{3.900000in}{3.900000in}}%
\pgfusepath{clip}%
\pgfsetbuttcap%
\pgfsetroundjoin%
\definecolor{currentfill}{rgb}{0.121569,0.466667,0.705882}%
\pgfsetfillcolor{currentfill}%
\pgfsetlinewidth{1.003750pt}%
\definecolor{currentstroke}{rgb}{0.121569,0.466667,0.705882}%
\pgfsetstrokecolor{currentstroke}%
\pgfsetdash{}{0pt}%
\pgfpathmoveto{\pgfqpoint{2.416747in}{1.446827in}}%
\pgfpathcurveto{\pgfqpoint{2.427797in}{1.446827in}}{\pgfqpoint{2.438396in}{1.451217in}}{\pgfqpoint{2.446210in}{1.459031in}}%
\pgfpathcurveto{\pgfqpoint{2.454023in}{1.466844in}}{\pgfqpoint{2.458414in}{1.477443in}}{\pgfqpoint{2.458414in}{1.488493in}}%
\pgfpathcurveto{\pgfqpoint{2.458414in}{1.499544in}}{\pgfqpoint{2.454023in}{1.510143in}}{\pgfqpoint{2.446210in}{1.517956in}}%
\pgfpathcurveto{\pgfqpoint{2.438396in}{1.525770in}}{\pgfqpoint{2.427797in}{1.530160in}}{\pgfqpoint{2.416747in}{1.530160in}}%
\pgfpathcurveto{\pgfqpoint{2.405697in}{1.530160in}}{\pgfqpoint{2.395098in}{1.525770in}}{\pgfqpoint{2.387284in}{1.517956in}}%
\pgfpathcurveto{\pgfqpoint{2.379470in}{1.510143in}}{\pgfqpoint{2.375080in}{1.499544in}}{\pgfqpoint{2.375080in}{1.488493in}}%
\pgfpathcurveto{\pgfqpoint{2.375080in}{1.477443in}}{\pgfqpoint{2.379470in}{1.466844in}}{\pgfqpoint{2.387284in}{1.459031in}}%
\pgfpathcurveto{\pgfqpoint{2.395098in}{1.451217in}}{\pgfqpoint{2.405697in}{1.446827in}}{\pgfqpoint{2.416747in}{1.446827in}}%
\pgfpathclose%
\pgfusepath{stroke,fill}%
\end{pgfscope}%
\begin{pgfscope}%
\pgfpathrectangle{\pgfqpoint{0.600000in}{0.600000in}}{\pgfqpoint{3.900000in}{3.900000in}}%
\pgfusepath{clip}%
\pgfsetbuttcap%
\pgfsetroundjoin%
\definecolor{currentfill}{rgb}{0.121569,0.466667,0.705882}%
\pgfsetfillcolor{currentfill}%
\pgfsetlinewidth{1.003750pt}%
\definecolor{currentstroke}{rgb}{0.121569,0.466667,0.705882}%
\pgfsetstrokecolor{currentstroke}%
\pgfsetdash{}{0pt}%
\pgfpathmoveto{\pgfqpoint{3.491607in}{2.703785in}}%
\pgfpathcurveto{\pgfqpoint{3.502657in}{2.703785in}}{\pgfqpoint{3.513256in}{2.708175in}}{\pgfqpoint{3.521069in}{2.715989in}}%
\pgfpathcurveto{\pgfqpoint{3.528883in}{2.723802in}}{\pgfqpoint{3.533273in}{2.734401in}}{\pgfqpoint{3.533273in}{2.745452in}}%
\pgfpathcurveto{\pgfqpoint{3.533273in}{2.756502in}}{\pgfqpoint{3.528883in}{2.767101in}}{\pgfqpoint{3.521069in}{2.774914in}}%
\pgfpathcurveto{\pgfqpoint{3.513256in}{2.782728in}}{\pgfqpoint{3.502657in}{2.787118in}}{\pgfqpoint{3.491607in}{2.787118in}}%
\pgfpathcurveto{\pgfqpoint{3.480556in}{2.787118in}}{\pgfqpoint{3.469957in}{2.782728in}}{\pgfqpoint{3.462144in}{2.774914in}}%
\pgfpathcurveto{\pgfqpoint{3.454330in}{2.767101in}}{\pgfqpoint{3.449940in}{2.756502in}}{\pgfqpoint{3.449940in}{2.745452in}}%
\pgfpathcurveto{\pgfqpoint{3.449940in}{2.734401in}}{\pgfqpoint{3.454330in}{2.723802in}}{\pgfqpoint{3.462144in}{2.715989in}}%
\pgfpathcurveto{\pgfqpoint{3.469957in}{2.708175in}}{\pgfqpoint{3.480556in}{2.703785in}}{\pgfqpoint{3.491607in}{2.703785in}}%
\pgfpathclose%
\pgfusepath{stroke,fill}%
\end{pgfscope}%
\begin{pgfscope}%
\pgfpathrectangle{\pgfqpoint{0.600000in}{0.600000in}}{\pgfqpoint{3.900000in}{3.900000in}}%
\pgfusepath{clip}%
\pgfsetbuttcap%
\pgfsetroundjoin%
\definecolor{currentfill}{rgb}{0.121569,0.466667,0.705882}%
\pgfsetfillcolor{currentfill}%
\pgfsetlinewidth{1.003750pt}%
\definecolor{currentstroke}{rgb}{0.121569,0.466667,0.705882}%
\pgfsetstrokecolor{currentstroke}%
\pgfsetdash{}{0pt}%
\pgfpathmoveto{\pgfqpoint{2.845171in}{3.019169in}}%
\pgfpathcurveto{\pgfqpoint{2.856221in}{3.019169in}}{\pgfqpoint{2.866820in}{3.023560in}}{\pgfqpoint{2.874634in}{3.031373in}}%
\pgfpathcurveto{\pgfqpoint{2.882447in}{3.039187in}}{\pgfqpoint{2.886838in}{3.049786in}}{\pgfqpoint{2.886838in}{3.060836in}}%
\pgfpathcurveto{\pgfqpoint{2.886838in}{3.071886in}}{\pgfqpoint{2.882447in}{3.082485in}}{\pgfqpoint{2.874634in}{3.090299in}}%
\pgfpathcurveto{\pgfqpoint{2.866820in}{3.098112in}}{\pgfqpoint{2.856221in}{3.102503in}}{\pgfqpoint{2.845171in}{3.102503in}}%
\pgfpathcurveto{\pgfqpoint{2.834121in}{3.102503in}}{\pgfqpoint{2.823522in}{3.098112in}}{\pgfqpoint{2.815708in}{3.090299in}}%
\pgfpathcurveto{\pgfqpoint{2.807894in}{3.082485in}}{\pgfqpoint{2.803504in}{3.071886in}}{\pgfqpoint{2.803504in}{3.060836in}}%
\pgfpathcurveto{\pgfqpoint{2.803504in}{3.049786in}}{\pgfqpoint{2.807894in}{3.039187in}}{\pgfqpoint{2.815708in}{3.031373in}}%
\pgfpathcurveto{\pgfqpoint{2.823522in}{3.023560in}}{\pgfqpoint{2.834121in}{3.019169in}}{\pgfqpoint{2.845171in}{3.019169in}}%
\pgfpathclose%
\pgfusepath{stroke,fill}%
\end{pgfscope}%
\begin{pgfscope}%
\pgfpathrectangle{\pgfqpoint{0.600000in}{0.600000in}}{\pgfqpoint{3.900000in}{3.900000in}}%
\pgfusepath{clip}%
\pgfsetbuttcap%
\pgfsetroundjoin%
\definecolor{currentfill}{rgb}{0.121569,0.466667,0.705882}%
\pgfsetfillcolor{currentfill}%
\pgfsetlinewidth{1.003750pt}%
\definecolor{currentstroke}{rgb}{0.121569,0.466667,0.705882}%
\pgfsetstrokecolor{currentstroke}%
\pgfsetdash{}{0pt}%
\pgfpathmoveto{\pgfqpoint{2.261036in}{2.878931in}}%
\pgfpathcurveto{\pgfqpoint{2.272087in}{2.878931in}}{\pgfqpoint{2.282686in}{2.883322in}}{\pgfqpoint{2.290499in}{2.891135in}}%
\pgfpathcurveto{\pgfqpoint{2.298313in}{2.898949in}}{\pgfqpoint{2.302703in}{2.909548in}}{\pgfqpoint{2.302703in}{2.920598in}}%
\pgfpathcurveto{\pgfqpoint{2.302703in}{2.931648in}}{\pgfqpoint{2.298313in}{2.942247in}}{\pgfqpoint{2.290499in}{2.950061in}}%
\pgfpathcurveto{\pgfqpoint{2.282686in}{2.957874in}}{\pgfqpoint{2.272087in}{2.962265in}}{\pgfqpoint{2.261036in}{2.962265in}}%
\pgfpathcurveto{\pgfqpoint{2.249986in}{2.962265in}}{\pgfqpoint{2.239387in}{2.957874in}}{\pgfqpoint{2.231574in}{2.950061in}}%
\pgfpathcurveto{\pgfqpoint{2.223760in}{2.942247in}}{\pgfqpoint{2.219370in}{2.931648in}}{\pgfqpoint{2.219370in}{2.920598in}}%
\pgfpathcurveto{\pgfqpoint{2.219370in}{2.909548in}}{\pgfqpoint{2.223760in}{2.898949in}}{\pgfqpoint{2.231574in}{2.891135in}}%
\pgfpathcurveto{\pgfqpoint{2.239387in}{2.883322in}}{\pgfqpoint{2.249986in}{2.878931in}}{\pgfqpoint{2.261036in}{2.878931in}}%
\pgfpathclose%
\pgfusepath{stroke,fill}%
\end{pgfscope}%
\begin{pgfscope}%
\pgfpathrectangle{\pgfqpoint{0.600000in}{0.600000in}}{\pgfqpoint{3.900000in}{3.900000in}}%
\pgfusepath{clip}%
\pgfsetbuttcap%
\pgfsetroundjoin%
\definecolor{currentfill}{rgb}{0.121569,0.466667,0.705882}%
\pgfsetfillcolor{currentfill}%
\pgfsetlinewidth{1.003750pt}%
\definecolor{currentstroke}{rgb}{0.121569,0.466667,0.705882}%
\pgfsetstrokecolor{currentstroke}%
\pgfsetdash{}{0pt}%
\pgfpathmoveto{\pgfqpoint{1.935353in}{2.394373in}}%
\pgfpathcurveto{\pgfqpoint{1.946403in}{2.394373in}}{\pgfqpoint{1.957002in}{2.398763in}}{\pgfqpoint{1.964816in}{2.406576in}}%
\pgfpathcurveto{\pgfqpoint{1.972630in}{2.414390in}}{\pgfqpoint{1.977020in}{2.424989in}}{\pgfqpoint{1.977020in}{2.436039in}}%
\pgfpathcurveto{\pgfqpoint{1.977020in}{2.447089in}}{\pgfqpoint{1.972630in}{2.457688in}}{\pgfqpoint{1.964816in}{2.465502in}}%
\pgfpathcurveto{\pgfqpoint{1.957002in}{2.473316in}}{\pgfqpoint{1.946403in}{2.477706in}}{\pgfqpoint{1.935353in}{2.477706in}}%
\pgfpathcurveto{\pgfqpoint{1.924303in}{2.477706in}}{\pgfqpoint{1.913704in}{2.473316in}}{\pgfqpoint{1.905890in}{2.465502in}}%
\pgfpathcurveto{\pgfqpoint{1.898077in}{2.457688in}}{\pgfqpoint{1.893686in}{2.447089in}}{\pgfqpoint{1.893686in}{2.436039in}}%
\pgfpathcurveto{\pgfqpoint{1.893686in}{2.424989in}}{\pgfqpoint{1.898077in}{2.414390in}}{\pgfqpoint{1.905890in}{2.406576in}}%
\pgfpathcurveto{\pgfqpoint{1.913704in}{2.398763in}}{\pgfqpoint{1.924303in}{2.394373in}}{\pgfqpoint{1.935353in}{2.394373in}}%
\pgfpathclose%
\pgfusepath{stroke,fill}%
\end{pgfscope}%
\begin{pgfscope}%
\pgfpathrectangle{\pgfqpoint{0.600000in}{0.600000in}}{\pgfqpoint{3.900000in}{3.900000in}}%
\pgfusepath{clip}%
\pgfsetbuttcap%
\pgfsetroundjoin%
\definecolor{currentfill}{rgb}{0.121569,0.466667,0.705882}%
\pgfsetfillcolor{currentfill}%
\pgfsetlinewidth{1.003750pt}%
\definecolor{currentstroke}{rgb}{0.121569,0.466667,0.705882}%
\pgfsetstrokecolor{currentstroke}%
\pgfsetdash{}{0pt}%
\pgfpathmoveto{\pgfqpoint{2.644702in}{2.427088in}}%
\pgfpathcurveto{\pgfqpoint{2.655752in}{2.427088in}}{\pgfqpoint{2.666351in}{2.431478in}}{\pgfqpoint{2.674165in}{2.439292in}}%
\pgfpathcurveto{\pgfqpoint{2.681978in}{2.447105in}}{\pgfqpoint{2.686369in}{2.457704in}}{\pgfqpoint{2.686369in}{2.468754in}}%
\pgfpathcurveto{\pgfqpoint{2.686369in}{2.479804in}}{\pgfqpoint{2.681978in}{2.490404in}}{\pgfqpoint{2.674165in}{2.498217in}}%
\pgfpathcurveto{\pgfqpoint{2.666351in}{2.506031in}}{\pgfqpoint{2.655752in}{2.510421in}}{\pgfqpoint{2.644702in}{2.510421in}}%
\pgfpathcurveto{\pgfqpoint{2.633652in}{2.510421in}}{\pgfqpoint{2.623053in}{2.506031in}}{\pgfqpoint{2.615239in}{2.498217in}}%
\pgfpathcurveto{\pgfqpoint{2.607426in}{2.490404in}}{\pgfqpoint{2.603035in}{2.479804in}}{\pgfqpoint{2.603035in}{2.468754in}}%
\pgfpathcurveto{\pgfqpoint{2.603035in}{2.457704in}}{\pgfqpoint{2.607426in}{2.447105in}}{\pgfqpoint{2.615239in}{2.439292in}}%
\pgfpathcurveto{\pgfqpoint{2.623053in}{2.431478in}}{\pgfqpoint{2.633652in}{2.427088in}}{\pgfqpoint{2.644702in}{2.427088in}}%
\pgfpathclose%
\pgfusepath{stroke,fill}%
\end{pgfscope}%
\begin{pgfscope}%
\pgfpathrectangle{\pgfqpoint{0.600000in}{0.600000in}}{\pgfqpoint{3.900000in}{3.900000in}}%
\pgfusepath{clip}%
\pgfsetbuttcap%
\pgfsetroundjoin%
\definecolor{currentfill}{rgb}{0.121569,0.466667,0.705882}%
\pgfsetfillcolor{currentfill}%
\pgfsetlinewidth{1.003750pt}%
\definecolor{currentstroke}{rgb}{0.121569,0.466667,0.705882}%
\pgfsetstrokecolor{currentstroke}%
\pgfsetdash{}{0pt}%
\pgfpathmoveto{\pgfqpoint{2.888715in}{1.778708in}}%
\pgfpathcurveto{\pgfqpoint{2.899766in}{1.778708in}}{\pgfqpoint{2.910365in}{1.783099in}}{\pgfqpoint{2.918178in}{1.790912in}}%
\pgfpathcurveto{\pgfqpoint{2.925992in}{1.798726in}}{\pgfqpoint{2.930382in}{1.809325in}}{\pgfqpoint{2.930382in}{1.820375in}}%
\pgfpathcurveto{\pgfqpoint{2.930382in}{1.831425in}}{\pgfqpoint{2.925992in}{1.842024in}}{\pgfqpoint{2.918178in}{1.849838in}}%
\pgfpathcurveto{\pgfqpoint{2.910365in}{1.857652in}}{\pgfqpoint{2.899766in}{1.862042in}}{\pgfqpoint{2.888715in}{1.862042in}}%
\pgfpathcurveto{\pgfqpoint{2.877665in}{1.862042in}}{\pgfqpoint{2.867066in}{1.857652in}}{\pgfqpoint{2.859253in}{1.849838in}}%
\pgfpathcurveto{\pgfqpoint{2.851439in}{1.842024in}}{\pgfqpoint{2.847049in}{1.831425in}}{\pgfqpoint{2.847049in}{1.820375in}}%
\pgfpathcurveto{\pgfqpoint{2.847049in}{1.809325in}}{\pgfqpoint{2.851439in}{1.798726in}}{\pgfqpoint{2.859253in}{1.790912in}}%
\pgfpathcurveto{\pgfqpoint{2.867066in}{1.783099in}}{\pgfqpoint{2.877665in}{1.778708in}}{\pgfqpoint{2.888715in}{1.778708in}}%
\pgfpathclose%
\pgfusepath{stroke,fill}%
\end{pgfscope}%
\begin{pgfscope}%
\pgfpathrectangle{\pgfqpoint{0.600000in}{0.600000in}}{\pgfqpoint{3.900000in}{3.900000in}}%
\pgfusepath{clip}%
\pgfsetbuttcap%
\pgfsetroundjoin%
\definecolor{currentfill}{rgb}{0.121569,0.466667,0.705882}%
\pgfsetfillcolor{currentfill}%
\pgfsetlinewidth{1.003750pt}%
\definecolor{currentstroke}{rgb}{0.121569,0.466667,0.705882}%
\pgfsetstrokecolor{currentstroke}%
\pgfsetdash{}{0pt}%
\pgfpathmoveto{\pgfqpoint{3.006027in}{3.133533in}}%
\pgfpathcurveto{\pgfqpoint{3.017077in}{3.133533in}}{\pgfqpoint{3.027676in}{3.137924in}}{\pgfqpoint{3.035489in}{3.145737in}}%
\pgfpathcurveto{\pgfqpoint{3.043303in}{3.153551in}}{\pgfqpoint{3.047693in}{3.164150in}}{\pgfqpoint{3.047693in}{3.175200in}}%
\pgfpathcurveto{\pgfqpoint{3.047693in}{3.186250in}}{\pgfqpoint{3.043303in}{3.196849in}}{\pgfqpoint{3.035489in}{3.204663in}}%
\pgfpathcurveto{\pgfqpoint{3.027676in}{3.212476in}}{\pgfqpoint{3.017077in}{3.216867in}}{\pgfqpoint{3.006027in}{3.216867in}}%
\pgfpathcurveto{\pgfqpoint{2.994977in}{3.216867in}}{\pgfqpoint{2.984378in}{3.212476in}}{\pgfqpoint{2.976564in}{3.204663in}}%
\pgfpathcurveto{\pgfqpoint{2.968750in}{3.196849in}}{\pgfqpoint{2.964360in}{3.186250in}}{\pgfqpoint{2.964360in}{3.175200in}}%
\pgfpathcurveto{\pgfqpoint{2.964360in}{3.164150in}}{\pgfqpoint{2.968750in}{3.153551in}}{\pgfqpoint{2.976564in}{3.145737in}}%
\pgfpathcurveto{\pgfqpoint{2.984378in}{3.137924in}}{\pgfqpoint{2.994977in}{3.133533in}}{\pgfqpoint{3.006027in}{3.133533in}}%
\pgfpathclose%
\pgfusepath{stroke,fill}%
\end{pgfscope}%
\begin{pgfscope}%
\pgfpathrectangle{\pgfqpoint{0.600000in}{0.600000in}}{\pgfqpoint{3.900000in}{3.900000in}}%
\pgfusepath{clip}%
\pgfsetbuttcap%
\pgfsetroundjoin%
\definecolor{currentfill}{rgb}{0.121569,0.466667,0.705882}%
\pgfsetfillcolor{currentfill}%
\pgfsetlinewidth{1.003750pt}%
\definecolor{currentstroke}{rgb}{0.121569,0.466667,0.705882}%
\pgfsetstrokecolor{currentstroke}%
\pgfsetdash{}{0pt}%
\pgfpathmoveto{\pgfqpoint{2.486771in}{2.068425in}}%
\pgfpathcurveto{\pgfqpoint{2.497821in}{2.068425in}}{\pgfqpoint{2.508421in}{2.072815in}}{\pgfqpoint{2.516234in}{2.080629in}}%
\pgfpathcurveto{\pgfqpoint{2.524048in}{2.088443in}}{\pgfqpoint{2.528438in}{2.099042in}}{\pgfqpoint{2.528438in}{2.110092in}}%
\pgfpathcurveto{\pgfqpoint{2.528438in}{2.121142in}}{\pgfqpoint{2.524048in}{2.131741in}}{\pgfqpoint{2.516234in}{2.139554in}}%
\pgfpathcurveto{\pgfqpoint{2.508421in}{2.147368in}}{\pgfqpoint{2.497821in}{2.151758in}}{\pgfqpoint{2.486771in}{2.151758in}}%
\pgfpathcurveto{\pgfqpoint{2.475721in}{2.151758in}}{\pgfqpoint{2.465122in}{2.147368in}}{\pgfqpoint{2.457309in}{2.139554in}}%
\pgfpathcurveto{\pgfqpoint{2.449495in}{2.131741in}}{\pgfqpoint{2.445105in}{2.121142in}}{\pgfqpoint{2.445105in}{2.110092in}}%
\pgfpathcurveto{\pgfqpoint{2.445105in}{2.099042in}}{\pgfqpoint{2.449495in}{2.088443in}}{\pgfqpoint{2.457309in}{2.080629in}}%
\pgfpathcurveto{\pgfqpoint{2.465122in}{2.072815in}}{\pgfqpoint{2.475721in}{2.068425in}}{\pgfqpoint{2.486771in}{2.068425in}}%
\pgfpathclose%
\pgfusepath{stroke,fill}%
\end{pgfscope}%
\begin{pgfscope}%
\pgfpathrectangle{\pgfqpoint{0.600000in}{0.600000in}}{\pgfqpoint{3.900000in}{3.900000in}}%
\pgfusepath{clip}%
\pgfsetbuttcap%
\pgfsetroundjoin%
\definecolor{currentfill}{rgb}{0.121569,0.466667,0.705882}%
\pgfsetfillcolor{currentfill}%
\pgfsetlinewidth{1.003750pt}%
\definecolor{currentstroke}{rgb}{0.121569,0.466667,0.705882}%
\pgfsetstrokecolor{currentstroke}%
\pgfsetdash{}{0pt}%
\pgfpathmoveto{\pgfqpoint{2.528916in}{1.335039in}}%
\pgfpathcurveto{\pgfqpoint{2.539966in}{1.335039in}}{\pgfqpoint{2.550565in}{1.339429in}}{\pgfqpoint{2.558379in}{1.347243in}}%
\pgfpathcurveto{\pgfqpoint{2.566192in}{1.355056in}}{\pgfqpoint{2.570583in}{1.365655in}}{\pgfqpoint{2.570583in}{1.376705in}}%
\pgfpathcurveto{\pgfqpoint{2.570583in}{1.387756in}}{\pgfqpoint{2.566192in}{1.398355in}}{\pgfqpoint{2.558379in}{1.406168in}}%
\pgfpathcurveto{\pgfqpoint{2.550565in}{1.413982in}}{\pgfqpoint{2.539966in}{1.418372in}}{\pgfqpoint{2.528916in}{1.418372in}}%
\pgfpathcurveto{\pgfqpoint{2.517866in}{1.418372in}}{\pgfqpoint{2.507267in}{1.413982in}}{\pgfqpoint{2.499453in}{1.406168in}}%
\pgfpathcurveto{\pgfqpoint{2.491639in}{1.398355in}}{\pgfqpoint{2.487249in}{1.387756in}}{\pgfqpoint{2.487249in}{1.376705in}}%
\pgfpathcurveto{\pgfqpoint{2.487249in}{1.365655in}}{\pgfqpoint{2.491639in}{1.355056in}}{\pgfqpoint{2.499453in}{1.347243in}}%
\pgfpathcurveto{\pgfqpoint{2.507267in}{1.339429in}}{\pgfqpoint{2.517866in}{1.335039in}}{\pgfqpoint{2.528916in}{1.335039in}}%
\pgfpathclose%
\pgfusepath{stroke,fill}%
\end{pgfscope}%
\begin{pgfscope}%
\pgfpathrectangle{\pgfqpoint{0.600000in}{0.600000in}}{\pgfqpoint{3.900000in}{3.900000in}}%
\pgfusepath{clip}%
\pgfsetbuttcap%
\pgfsetroundjoin%
\definecolor{currentfill}{rgb}{0.121569,0.466667,0.705882}%
\pgfsetfillcolor{currentfill}%
\pgfsetlinewidth{1.003750pt}%
\definecolor{currentstroke}{rgb}{0.121569,0.466667,0.705882}%
\pgfsetstrokecolor{currentstroke}%
\pgfsetdash{}{0pt}%
\pgfpathmoveto{\pgfqpoint{3.246418in}{2.411865in}}%
\pgfpathcurveto{\pgfqpoint{3.257469in}{2.411865in}}{\pgfqpoint{3.268068in}{2.416255in}}{\pgfqpoint{3.275881in}{2.424068in}}%
\pgfpathcurveto{\pgfqpoint{3.283695in}{2.431882in}}{\pgfqpoint{3.288085in}{2.442481in}}{\pgfqpoint{3.288085in}{2.453531in}}%
\pgfpathcurveto{\pgfqpoint{3.288085in}{2.464581in}}{\pgfqpoint{3.283695in}{2.475180in}}{\pgfqpoint{3.275881in}{2.482994in}}%
\pgfpathcurveto{\pgfqpoint{3.268068in}{2.490808in}}{\pgfqpoint{3.257469in}{2.495198in}}{\pgfqpoint{3.246418in}{2.495198in}}%
\pgfpathcurveto{\pgfqpoint{3.235368in}{2.495198in}}{\pgfqpoint{3.224769in}{2.490808in}}{\pgfqpoint{3.216956in}{2.482994in}}%
\pgfpathcurveto{\pgfqpoint{3.209142in}{2.475180in}}{\pgfqpoint{3.204752in}{2.464581in}}{\pgfqpoint{3.204752in}{2.453531in}}%
\pgfpathcurveto{\pgfqpoint{3.204752in}{2.442481in}}{\pgfqpoint{3.209142in}{2.431882in}}{\pgfqpoint{3.216956in}{2.424068in}}%
\pgfpathcurveto{\pgfqpoint{3.224769in}{2.416255in}}{\pgfqpoint{3.235368in}{2.411865in}}{\pgfqpoint{3.246418in}{2.411865in}}%
\pgfpathclose%
\pgfusepath{stroke,fill}%
\end{pgfscope}%
\begin{pgfscope}%
\pgfpathrectangle{\pgfqpoint{0.600000in}{0.600000in}}{\pgfqpoint{3.900000in}{3.900000in}}%
\pgfusepath{clip}%
\pgfsetbuttcap%
\pgfsetroundjoin%
\definecolor{currentfill}{rgb}{0.121569,0.466667,0.705882}%
\pgfsetfillcolor{currentfill}%
\pgfsetlinewidth{1.003750pt}%
\definecolor{currentstroke}{rgb}{0.121569,0.466667,0.705882}%
\pgfsetstrokecolor{currentstroke}%
\pgfsetdash{}{0pt}%
\pgfpathmoveto{\pgfqpoint{2.067564in}{2.843250in}}%
\pgfpathcurveto{\pgfqpoint{2.078614in}{2.843250in}}{\pgfqpoint{2.089213in}{2.847640in}}{\pgfqpoint{2.097027in}{2.855454in}}%
\pgfpathcurveto{\pgfqpoint{2.104840in}{2.863268in}}{\pgfqpoint{2.109231in}{2.873867in}}{\pgfqpoint{2.109231in}{2.884917in}}%
\pgfpathcurveto{\pgfqpoint{2.109231in}{2.895967in}}{\pgfqpoint{2.104840in}{2.906566in}}{\pgfqpoint{2.097027in}{2.914380in}}%
\pgfpathcurveto{\pgfqpoint{2.089213in}{2.922193in}}{\pgfqpoint{2.078614in}{2.926584in}}{\pgfqpoint{2.067564in}{2.926584in}}%
\pgfpathcurveto{\pgfqpoint{2.056514in}{2.926584in}}{\pgfqpoint{2.045915in}{2.922193in}}{\pgfqpoint{2.038101in}{2.914380in}}%
\pgfpathcurveto{\pgfqpoint{2.030287in}{2.906566in}}{\pgfqpoint{2.025897in}{2.895967in}}{\pgfqpoint{2.025897in}{2.884917in}}%
\pgfpathcurveto{\pgfqpoint{2.025897in}{2.873867in}}{\pgfqpoint{2.030287in}{2.863268in}}{\pgfqpoint{2.038101in}{2.855454in}}%
\pgfpathcurveto{\pgfqpoint{2.045915in}{2.847640in}}{\pgfqpoint{2.056514in}{2.843250in}}{\pgfqpoint{2.067564in}{2.843250in}}%
\pgfpathclose%
\pgfusepath{stroke,fill}%
\end{pgfscope}%
\begin{pgfscope}%
\pgfpathrectangle{\pgfqpoint{0.600000in}{0.600000in}}{\pgfqpoint{3.900000in}{3.900000in}}%
\pgfusepath{clip}%
\pgfsetbuttcap%
\pgfsetroundjoin%
\definecolor{currentfill}{rgb}{0.121569,0.466667,0.705882}%
\pgfsetfillcolor{currentfill}%
\pgfsetlinewidth{1.003750pt}%
\definecolor{currentstroke}{rgb}{0.121569,0.466667,0.705882}%
\pgfsetstrokecolor{currentstroke}%
\pgfsetdash{}{0pt}%
\pgfpathmoveto{\pgfqpoint{1.953446in}{2.588876in}}%
\pgfpathcurveto{\pgfqpoint{1.964496in}{2.588876in}}{\pgfqpoint{1.975095in}{2.593266in}}{\pgfqpoint{1.982908in}{2.601080in}}%
\pgfpathcurveto{\pgfqpoint{1.990722in}{2.608893in}}{\pgfqpoint{1.995112in}{2.619492in}}{\pgfqpoint{1.995112in}{2.630543in}}%
\pgfpathcurveto{\pgfqpoint{1.995112in}{2.641593in}}{\pgfqpoint{1.990722in}{2.652192in}}{\pgfqpoint{1.982908in}{2.660005in}}%
\pgfpathcurveto{\pgfqpoint{1.975095in}{2.667819in}}{\pgfqpoint{1.964496in}{2.672209in}}{\pgfqpoint{1.953446in}{2.672209in}}%
\pgfpathcurveto{\pgfqpoint{1.942395in}{2.672209in}}{\pgfqpoint{1.931796in}{2.667819in}}{\pgfqpoint{1.923983in}{2.660005in}}%
\pgfpathcurveto{\pgfqpoint{1.916169in}{2.652192in}}{\pgfqpoint{1.911779in}{2.641593in}}{\pgfqpoint{1.911779in}{2.630543in}}%
\pgfpathcurveto{\pgfqpoint{1.911779in}{2.619492in}}{\pgfqpoint{1.916169in}{2.608893in}}{\pgfqpoint{1.923983in}{2.601080in}}%
\pgfpathcurveto{\pgfqpoint{1.931796in}{2.593266in}}{\pgfqpoint{1.942395in}{2.588876in}}{\pgfqpoint{1.953446in}{2.588876in}}%
\pgfpathclose%
\pgfusepath{stroke,fill}%
\end{pgfscope}%
\begin{pgfscope}%
\pgfpathrectangle{\pgfqpoint{0.600000in}{0.600000in}}{\pgfqpoint{3.900000in}{3.900000in}}%
\pgfusepath{clip}%
\pgfsetbuttcap%
\pgfsetroundjoin%
\definecolor{currentfill}{rgb}{0.121569,0.466667,0.705882}%
\pgfsetfillcolor{currentfill}%
\pgfsetlinewidth{1.003750pt}%
\definecolor{currentstroke}{rgb}{0.121569,0.466667,0.705882}%
\pgfsetstrokecolor{currentstroke}%
\pgfsetdash{}{0pt}%
\pgfpathmoveto{\pgfqpoint{2.785812in}{2.226663in}}%
\pgfpathcurveto{\pgfqpoint{2.796862in}{2.226663in}}{\pgfqpoint{2.807461in}{2.231054in}}{\pgfqpoint{2.815274in}{2.238867in}}%
\pgfpathcurveto{\pgfqpoint{2.823088in}{2.246681in}}{\pgfqpoint{2.827478in}{2.257280in}}{\pgfqpoint{2.827478in}{2.268330in}}%
\pgfpathcurveto{\pgfqpoint{2.827478in}{2.279380in}}{\pgfqpoint{2.823088in}{2.289979in}}{\pgfqpoint{2.815274in}{2.297793in}}%
\pgfpathcurveto{\pgfqpoint{2.807461in}{2.305606in}}{\pgfqpoint{2.796862in}{2.309997in}}{\pgfqpoint{2.785812in}{2.309997in}}%
\pgfpathcurveto{\pgfqpoint{2.774762in}{2.309997in}}{\pgfqpoint{2.764162in}{2.305606in}}{\pgfqpoint{2.756349in}{2.297793in}}%
\pgfpathcurveto{\pgfqpoint{2.748535in}{2.289979in}}{\pgfqpoint{2.744145in}{2.279380in}}{\pgfqpoint{2.744145in}{2.268330in}}%
\pgfpathcurveto{\pgfqpoint{2.744145in}{2.257280in}}{\pgfqpoint{2.748535in}{2.246681in}}{\pgfqpoint{2.756349in}{2.238867in}}%
\pgfpathcurveto{\pgfqpoint{2.764162in}{2.231054in}}{\pgfqpoint{2.774762in}{2.226663in}}{\pgfqpoint{2.785812in}{2.226663in}}%
\pgfpathclose%
\pgfusepath{stroke,fill}%
\end{pgfscope}%
\begin{pgfscope}%
\pgfpathrectangle{\pgfqpoint{0.600000in}{0.600000in}}{\pgfqpoint{3.900000in}{3.900000in}}%
\pgfusepath{clip}%
\pgfsetbuttcap%
\pgfsetroundjoin%
\definecolor{currentfill}{rgb}{0.121569,0.466667,0.705882}%
\pgfsetfillcolor{currentfill}%
\pgfsetlinewidth{1.003750pt}%
\definecolor{currentstroke}{rgb}{0.121569,0.466667,0.705882}%
\pgfsetstrokecolor{currentstroke}%
\pgfsetdash{}{0pt}%
\pgfpathmoveto{\pgfqpoint{2.699149in}{2.127734in}}%
\pgfpathcurveto{\pgfqpoint{2.710199in}{2.127734in}}{\pgfqpoint{2.720798in}{2.132124in}}{\pgfqpoint{2.728612in}{2.139938in}}%
\pgfpathcurveto{\pgfqpoint{2.736425in}{2.147751in}}{\pgfqpoint{2.740816in}{2.158350in}}{\pgfqpoint{2.740816in}{2.169401in}}%
\pgfpathcurveto{\pgfqpoint{2.740816in}{2.180451in}}{\pgfqpoint{2.736425in}{2.191050in}}{\pgfqpoint{2.728612in}{2.198863in}}%
\pgfpathcurveto{\pgfqpoint{2.720798in}{2.206677in}}{\pgfqpoint{2.710199in}{2.211067in}}{\pgfqpoint{2.699149in}{2.211067in}}%
\pgfpathcurveto{\pgfqpoint{2.688099in}{2.211067in}}{\pgfqpoint{2.677500in}{2.206677in}}{\pgfqpoint{2.669686in}{2.198863in}}%
\pgfpathcurveto{\pgfqpoint{2.661872in}{2.191050in}}{\pgfqpoint{2.657482in}{2.180451in}}{\pgfqpoint{2.657482in}{2.169401in}}%
\pgfpathcurveto{\pgfqpoint{2.657482in}{2.158350in}}{\pgfqpoint{2.661872in}{2.147751in}}{\pgfqpoint{2.669686in}{2.139938in}}%
\pgfpathcurveto{\pgfqpoint{2.677500in}{2.132124in}}{\pgfqpoint{2.688099in}{2.127734in}}{\pgfqpoint{2.699149in}{2.127734in}}%
\pgfpathclose%
\pgfusepath{stroke,fill}%
\end{pgfscope}%
\begin{pgfscope}%
\pgfpathrectangle{\pgfqpoint{0.600000in}{0.600000in}}{\pgfqpoint{3.900000in}{3.900000in}}%
\pgfusepath{clip}%
\pgfsetbuttcap%
\pgfsetroundjoin%
\definecolor{currentfill}{rgb}{0.121569,0.466667,0.705882}%
\pgfsetfillcolor{currentfill}%
\pgfsetlinewidth{1.003750pt}%
\definecolor{currentstroke}{rgb}{0.121569,0.466667,0.705882}%
\pgfsetstrokecolor{currentstroke}%
\pgfsetdash{}{0pt}%
\pgfpathmoveto{\pgfqpoint{1.976688in}{2.631557in}}%
\pgfpathcurveto{\pgfqpoint{1.987738in}{2.631557in}}{\pgfqpoint{1.998337in}{2.635948in}}{\pgfqpoint{2.006151in}{2.643761in}}%
\pgfpathcurveto{\pgfqpoint{2.013965in}{2.651575in}}{\pgfqpoint{2.018355in}{2.662174in}}{\pgfqpoint{2.018355in}{2.673224in}}%
\pgfpathcurveto{\pgfqpoint{2.018355in}{2.684274in}}{\pgfqpoint{2.013965in}{2.694873in}}{\pgfqpoint{2.006151in}{2.702687in}}%
\pgfpathcurveto{\pgfqpoint{1.998337in}{2.710501in}}{\pgfqpoint{1.987738in}{2.714891in}}{\pgfqpoint{1.976688in}{2.714891in}}%
\pgfpathcurveto{\pgfqpoint{1.965638in}{2.714891in}}{\pgfqpoint{1.955039in}{2.710501in}}{\pgfqpoint{1.947225in}{2.702687in}}%
\pgfpathcurveto{\pgfqpoint{1.939412in}{2.694873in}}{\pgfqpoint{1.935022in}{2.684274in}}{\pgfqpoint{1.935022in}{2.673224in}}%
\pgfpathcurveto{\pgfqpoint{1.935022in}{2.662174in}}{\pgfqpoint{1.939412in}{2.651575in}}{\pgfqpoint{1.947225in}{2.643761in}}%
\pgfpathcurveto{\pgfqpoint{1.955039in}{2.635948in}}{\pgfqpoint{1.965638in}{2.631557in}}{\pgfqpoint{1.976688in}{2.631557in}}%
\pgfpathclose%
\pgfusepath{stroke,fill}%
\end{pgfscope}%
\begin{pgfscope}%
\pgfpathrectangle{\pgfqpoint{0.600000in}{0.600000in}}{\pgfqpoint{3.900000in}{3.900000in}}%
\pgfusepath{clip}%
\pgfsetbuttcap%
\pgfsetroundjoin%
\definecolor{currentfill}{rgb}{0.121569,0.466667,0.705882}%
\pgfsetfillcolor{currentfill}%
\pgfsetlinewidth{1.003750pt}%
\definecolor{currentstroke}{rgb}{0.121569,0.466667,0.705882}%
\pgfsetstrokecolor{currentstroke}%
\pgfsetdash{}{0pt}%
\pgfpathmoveto{\pgfqpoint{2.612898in}{1.807526in}}%
\pgfpathcurveto{\pgfqpoint{2.623948in}{1.807526in}}{\pgfqpoint{2.634547in}{1.811916in}}{\pgfqpoint{2.642361in}{1.819730in}}%
\pgfpathcurveto{\pgfqpoint{2.650174in}{1.827543in}}{\pgfqpoint{2.654565in}{1.838142in}}{\pgfqpoint{2.654565in}{1.849192in}}%
\pgfpathcurveto{\pgfqpoint{2.654565in}{1.860243in}}{\pgfqpoint{2.650174in}{1.870842in}}{\pgfqpoint{2.642361in}{1.878655in}}%
\pgfpathcurveto{\pgfqpoint{2.634547in}{1.886469in}}{\pgfqpoint{2.623948in}{1.890859in}}{\pgfqpoint{2.612898in}{1.890859in}}%
\pgfpathcurveto{\pgfqpoint{2.601848in}{1.890859in}}{\pgfqpoint{2.591249in}{1.886469in}}{\pgfqpoint{2.583435in}{1.878655in}}%
\pgfpathcurveto{\pgfqpoint{2.575622in}{1.870842in}}{\pgfqpoint{2.571231in}{1.860243in}}{\pgfqpoint{2.571231in}{1.849192in}}%
\pgfpathcurveto{\pgfqpoint{2.571231in}{1.838142in}}{\pgfqpoint{2.575622in}{1.827543in}}{\pgfqpoint{2.583435in}{1.819730in}}%
\pgfpathcurveto{\pgfqpoint{2.591249in}{1.811916in}}{\pgfqpoint{2.601848in}{1.807526in}}{\pgfqpoint{2.612898in}{1.807526in}}%
\pgfpathclose%
\pgfusepath{stroke,fill}%
\end{pgfscope}%
\begin{pgfscope}%
\pgfpathrectangle{\pgfqpoint{0.600000in}{0.600000in}}{\pgfqpoint{3.900000in}{3.900000in}}%
\pgfusepath{clip}%
\pgfsetbuttcap%
\pgfsetroundjoin%
\definecolor{currentfill}{rgb}{0.121569,0.466667,0.705882}%
\pgfsetfillcolor{currentfill}%
\pgfsetlinewidth{1.003750pt}%
\definecolor{currentstroke}{rgb}{0.121569,0.466667,0.705882}%
\pgfsetstrokecolor{currentstroke}%
\pgfsetdash{}{0pt}%
\pgfpathmoveto{\pgfqpoint{2.956671in}{2.012638in}}%
\pgfpathcurveto{\pgfqpoint{2.967721in}{2.012638in}}{\pgfqpoint{2.978320in}{2.017029in}}{\pgfqpoint{2.986133in}{2.024842in}}%
\pgfpathcurveto{\pgfqpoint{2.993947in}{2.032656in}}{\pgfqpoint{2.998337in}{2.043255in}}{\pgfqpoint{2.998337in}{2.054305in}}%
\pgfpathcurveto{\pgfqpoint{2.998337in}{2.065355in}}{\pgfqpoint{2.993947in}{2.075954in}}{\pgfqpoint{2.986133in}{2.083768in}}%
\pgfpathcurveto{\pgfqpoint{2.978320in}{2.091581in}}{\pgfqpoint{2.967721in}{2.095972in}}{\pgfqpoint{2.956671in}{2.095972in}}%
\pgfpathcurveto{\pgfqpoint{2.945621in}{2.095972in}}{\pgfqpoint{2.935022in}{2.091581in}}{\pgfqpoint{2.927208in}{2.083768in}}%
\pgfpathcurveto{\pgfqpoint{2.919394in}{2.075954in}}{\pgfqpoint{2.915004in}{2.065355in}}{\pgfqpoint{2.915004in}{2.054305in}}%
\pgfpathcurveto{\pgfqpoint{2.915004in}{2.043255in}}{\pgfqpoint{2.919394in}{2.032656in}}{\pgfqpoint{2.927208in}{2.024842in}}%
\pgfpathcurveto{\pgfqpoint{2.935022in}{2.017029in}}{\pgfqpoint{2.945621in}{2.012638in}}{\pgfqpoint{2.956671in}{2.012638in}}%
\pgfpathclose%
\pgfusepath{stroke,fill}%
\end{pgfscope}%
\begin{pgfscope}%
\pgfpathrectangle{\pgfqpoint{0.600000in}{0.600000in}}{\pgfqpoint{3.900000in}{3.900000in}}%
\pgfusepath{clip}%
\pgfsetbuttcap%
\pgfsetroundjoin%
\definecolor{currentfill}{rgb}{0.121569,0.466667,0.705882}%
\pgfsetfillcolor{currentfill}%
\pgfsetlinewidth{1.003750pt}%
\definecolor{currentstroke}{rgb}{0.121569,0.466667,0.705882}%
\pgfsetstrokecolor{currentstroke}%
\pgfsetdash{}{0pt}%
\pgfpathmoveto{\pgfqpoint{3.403356in}{2.156462in}}%
\pgfpathcurveto{\pgfqpoint{3.414406in}{2.156462in}}{\pgfqpoint{3.425005in}{2.160852in}}{\pgfqpoint{3.432819in}{2.168666in}}%
\pgfpathcurveto{\pgfqpoint{3.440633in}{2.176479in}}{\pgfqpoint{3.445023in}{2.187078in}}{\pgfqpoint{3.445023in}{2.198129in}}%
\pgfpathcurveto{\pgfqpoint{3.445023in}{2.209179in}}{\pgfqpoint{3.440633in}{2.219778in}}{\pgfqpoint{3.432819in}{2.227591in}}%
\pgfpathcurveto{\pgfqpoint{3.425005in}{2.235405in}}{\pgfqpoint{3.414406in}{2.239795in}}{\pgfqpoint{3.403356in}{2.239795in}}%
\pgfpathcurveto{\pgfqpoint{3.392306in}{2.239795in}}{\pgfqpoint{3.381707in}{2.235405in}}{\pgfqpoint{3.373893in}{2.227591in}}%
\pgfpathcurveto{\pgfqpoint{3.366080in}{2.219778in}}{\pgfqpoint{3.361690in}{2.209179in}}{\pgfqpoint{3.361690in}{2.198129in}}%
\pgfpathcurveto{\pgfqpoint{3.361690in}{2.187078in}}{\pgfqpoint{3.366080in}{2.176479in}}{\pgfqpoint{3.373893in}{2.168666in}}%
\pgfpathcurveto{\pgfqpoint{3.381707in}{2.160852in}}{\pgfqpoint{3.392306in}{2.156462in}}{\pgfqpoint{3.403356in}{2.156462in}}%
\pgfpathclose%
\pgfusepath{stroke,fill}%
\end{pgfscope}%
\begin{pgfscope}%
\pgfpathrectangle{\pgfqpoint{0.600000in}{0.600000in}}{\pgfqpoint{3.900000in}{3.900000in}}%
\pgfusepath{clip}%
\pgfsetbuttcap%
\pgfsetroundjoin%
\definecolor{currentfill}{rgb}{0.121569,0.466667,0.705882}%
\pgfsetfillcolor{currentfill}%
\pgfsetlinewidth{1.003750pt}%
\definecolor{currentstroke}{rgb}{0.121569,0.466667,0.705882}%
\pgfsetstrokecolor{currentstroke}%
\pgfsetdash{}{0pt}%
\pgfpathmoveto{\pgfqpoint{2.140229in}{3.005688in}}%
\pgfpathcurveto{\pgfqpoint{2.151279in}{3.005688in}}{\pgfqpoint{2.161878in}{3.010078in}}{\pgfqpoint{2.169691in}{3.017892in}}%
\pgfpathcurveto{\pgfqpoint{2.177505in}{3.025705in}}{\pgfqpoint{2.181895in}{3.036304in}}{\pgfqpoint{2.181895in}{3.047354in}}%
\pgfpathcurveto{\pgfqpoint{2.181895in}{3.058405in}}{\pgfqpoint{2.177505in}{3.069004in}}{\pgfqpoint{2.169691in}{3.076817in}}%
\pgfpathcurveto{\pgfqpoint{2.161878in}{3.084631in}}{\pgfqpoint{2.151279in}{3.089021in}}{\pgfqpoint{2.140229in}{3.089021in}}%
\pgfpathcurveto{\pgfqpoint{2.129179in}{3.089021in}}{\pgfqpoint{2.118580in}{3.084631in}}{\pgfqpoint{2.110766in}{3.076817in}}%
\pgfpathcurveto{\pgfqpoint{2.102952in}{3.069004in}}{\pgfqpoint{2.098562in}{3.058405in}}{\pgfqpoint{2.098562in}{3.047354in}}%
\pgfpathcurveto{\pgfqpoint{2.098562in}{3.036304in}}{\pgfqpoint{2.102952in}{3.025705in}}{\pgfqpoint{2.110766in}{3.017892in}}%
\pgfpathcurveto{\pgfqpoint{2.118580in}{3.010078in}}{\pgfqpoint{2.129179in}{3.005688in}}{\pgfqpoint{2.140229in}{3.005688in}}%
\pgfpathclose%
\pgfusepath{stroke,fill}%
\end{pgfscope}%
\begin{pgfscope}%
\pgfpathrectangle{\pgfqpoint{0.600000in}{0.600000in}}{\pgfqpoint{3.900000in}{3.900000in}}%
\pgfusepath{clip}%
\pgfsetbuttcap%
\pgfsetroundjoin%
\definecolor{currentfill}{rgb}{0.121569,0.466667,0.705882}%
\pgfsetfillcolor{currentfill}%
\pgfsetlinewidth{1.003750pt}%
\definecolor{currentstroke}{rgb}{0.121569,0.466667,0.705882}%
\pgfsetstrokecolor{currentstroke}%
\pgfsetdash{}{0pt}%
\pgfpathmoveto{\pgfqpoint{3.123454in}{3.271500in}}%
\pgfpathcurveto{\pgfqpoint{3.134504in}{3.271500in}}{\pgfqpoint{3.145103in}{3.275890in}}{\pgfqpoint{3.152917in}{3.283704in}}%
\pgfpathcurveto{\pgfqpoint{3.160731in}{3.291518in}}{\pgfqpoint{3.165121in}{3.302117in}}{\pgfqpoint{3.165121in}{3.313167in}}%
\pgfpathcurveto{\pgfqpoint{3.165121in}{3.324217in}}{\pgfqpoint{3.160731in}{3.334816in}}{\pgfqpoint{3.152917in}{3.342629in}}%
\pgfpathcurveto{\pgfqpoint{3.145103in}{3.350443in}}{\pgfqpoint{3.134504in}{3.354833in}}{\pgfqpoint{3.123454in}{3.354833in}}%
\pgfpathcurveto{\pgfqpoint{3.112404in}{3.354833in}}{\pgfqpoint{3.101805in}{3.350443in}}{\pgfqpoint{3.093991in}{3.342629in}}%
\pgfpathcurveto{\pgfqpoint{3.086178in}{3.334816in}}{\pgfqpoint{3.081788in}{3.324217in}}{\pgfqpoint{3.081788in}{3.313167in}}%
\pgfpathcurveto{\pgfqpoint{3.081788in}{3.302117in}}{\pgfqpoint{3.086178in}{3.291518in}}{\pgfqpoint{3.093991in}{3.283704in}}%
\pgfpathcurveto{\pgfqpoint{3.101805in}{3.275890in}}{\pgfqpoint{3.112404in}{3.271500in}}{\pgfqpoint{3.123454in}{3.271500in}}%
\pgfpathclose%
\pgfusepath{stroke,fill}%
\end{pgfscope}%
\begin{pgfscope}%
\pgfpathrectangle{\pgfqpoint{0.600000in}{0.600000in}}{\pgfqpoint{3.900000in}{3.900000in}}%
\pgfusepath{clip}%
\pgfsetbuttcap%
\pgfsetroundjoin%
\definecolor{currentfill}{rgb}{0.121569,0.466667,0.705882}%
\pgfsetfillcolor{currentfill}%
\pgfsetlinewidth{1.003750pt}%
\definecolor{currentstroke}{rgb}{0.121569,0.466667,0.705882}%
\pgfsetstrokecolor{currentstroke}%
\pgfsetdash{}{0pt}%
\pgfpathmoveto{\pgfqpoint{1.760443in}{2.135897in}}%
\pgfpathcurveto{\pgfqpoint{1.771493in}{2.135897in}}{\pgfqpoint{1.782092in}{2.140287in}}{\pgfqpoint{1.789906in}{2.148100in}}%
\pgfpathcurveto{\pgfqpoint{1.797720in}{2.155914in}}{\pgfqpoint{1.802110in}{2.166513in}}{\pgfqpoint{1.802110in}{2.177563in}}%
\pgfpathcurveto{\pgfqpoint{1.802110in}{2.188613in}}{\pgfqpoint{1.797720in}{2.199212in}}{\pgfqpoint{1.789906in}{2.207026in}}%
\pgfpathcurveto{\pgfqpoint{1.782092in}{2.214840in}}{\pgfqpoint{1.771493in}{2.219230in}}{\pgfqpoint{1.760443in}{2.219230in}}%
\pgfpathcurveto{\pgfqpoint{1.749393in}{2.219230in}}{\pgfqpoint{1.738794in}{2.214840in}}{\pgfqpoint{1.730980in}{2.207026in}}%
\pgfpathcurveto{\pgfqpoint{1.723167in}{2.199212in}}{\pgfqpoint{1.718777in}{2.188613in}}{\pgfqpoint{1.718777in}{2.177563in}}%
\pgfpathcurveto{\pgfqpoint{1.718777in}{2.166513in}}{\pgfqpoint{1.723167in}{2.155914in}}{\pgfqpoint{1.730980in}{2.148100in}}%
\pgfpathcurveto{\pgfqpoint{1.738794in}{2.140287in}}{\pgfqpoint{1.749393in}{2.135897in}}{\pgfqpoint{1.760443in}{2.135897in}}%
\pgfpathclose%
\pgfusepath{stroke,fill}%
\end{pgfscope}%
\begin{pgfscope}%
\pgfpathrectangle{\pgfqpoint{0.600000in}{0.600000in}}{\pgfqpoint{3.900000in}{3.900000in}}%
\pgfusepath{clip}%
\pgfsetbuttcap%
\pgfsetroundjoin%
\definecolor{currentfill}{rgb}{0.121569,0.466667,0.705882}%
\pgfsetfillcolor{currentfill}%
\pgfsetlinewidth{1.003750pt}%
\definecolor{currentstroke}{rgb}{0.121569,0.466667,0.705882}%
\pgfsetstrokecolor{currentstroke}%
\pgfsetdash{}{0pt}%
\pgfpathmoveto{\pgfqpoint{2.607110in}{1.900988in}}%
\pgfpathcurveto{\pgfqpoint{2.618160in}{1.900988in}}{\pgfqpoint{2.628759in}{1.905379in}}{\pgfqpoint{2.636573in}{1.913192in}}%
\pgfpathcurveto{\pgfqpoint{2.644386in}{1.921006in}}{\pgfqpoint{2.648777in}{1.931605in}}{\pgfqpoint{2.648777in}{1.942655in}}%
\pgfpathcurveto{\pgfqpoint{2.648777in}{1.953705in}}{\pgfqpoint{2.644386in}{1.964304in}}{\pgfqpoint{2.636573in}{1.972118in}}%
\pgfpathcurveto{\pgfqpoint{2.628759in}{1.979931in}}{\pgfqpoint{2.618160in}{1.984322in}}{\pgfqpoint{2.607110in}{1.984322in}}%
\pgfpathcurveto{\pgfqpoint{2.596060in}{1.984322in}}{\pgfqpoint{2.585461in}{1.979931in}}{\pgfqpoint{2.577647in}{1.972118in}}%
\pgfpathcurveto{\pgfqpoint{2.569834in}{1.964304in}}{\pgfqpoint{2.565443in}{1.953705in}}{\pgfqpoint{2.565443in}{1.942655in}}%
\pgfpathcurveto{\pgfqpoint{2.565443in}{1.931605in}}{\pgfqpoint{2.569834in}{1.921006in}}{\pgfqpoint{2.577647in}{1.913192in}}%
\pgfpathcurveto{\pgfqpoint{2.585461in}{1.905379in}}{\pgfqpoint{2.596060in}{1.900988in}}{\pgfqpoint{2.607110in}{1.900988in}}%
\pgfpathclose%
\pgfusepath{stroke,fill}%
\end{pgfscope}%
\begin{pgfscope}%
\pgfpathrectangle{\pgfqpoint{0.600000in}{0.600000in}}{\pgfqpoint{3.900000in}{3.900000in}}%
\pgfusepath{clip}%
\pgfsetbuttcap%
\pgfsetroundjoin%
\definecolor{currentfill}{rgb}{0.121569,0.466667,0.705882}%
\pgfsetfillcolor{currentfill}%
\pgfsetlinewidth{1.003750pt}%
\definecolor{currentstroke}{rgb}{0.121569,0.466667,0.705882}%
\pgfsetstrokecolor{currentstroke}%
\pgfsetdash{}{0pt}%
\pgfpathmoveto{\pgfqpoint{2.815792in}{1.836195in}}%
\pgfpathcurveto{\pgfqpoint{2.826842in}{1.836195in}}{\pgfqpoint{2.837441in}{1.840585in}}{\pgfqpoint{2.845255in}{1.848399in}}%
\pgfpathcurveto{\pgfqpoint{2.853068in}{1.856212in}}{\pgfqpoint{2.857459in}{1.866811in}}{\pgfqpoint{2.857459in}{1.877861in}}%
\pgfpathcurveto{\pgfqpoint{2.857459in}{1.888911in}}{\pgfqpoint{2.853068in}{1.899510in}}{\pgfqpoint{2.845255in}{1.907324in}}%
\pgfpathcurveto{\pgfqpoint{2.837441in}{1.915138in}}{\pgfqpoint{2.826842in}{1.919528in}}{\pgfqpoint{2.815792in}{1.919528in}}%
\pgfpathcurveto{\pgfqpoint{2.804742in}{1.919528in}}{\pgfqpoint{2.794143in}{1.915138in}}{\pgfqpoint{2.786329in}{1.907324in}}%
\pgfpathcurveto{\pgfqpoint{2.778516in}{1.899510in}}{\pgfqpoint{2.774125in}{1.888911in}}{\pgfqpoint{2.774125in}{1.877861in}}%
\pgfpathcurveto{\pgfqpoint{2.774125in}{1.866811in}}{\pgfqpoint{2.778516in}{1.856212in}}{\pgfqpoint{2.786329in}{1.848399in}}%
\pgfpathcurveto{\pgfqpoint{2.794143in}{1.840585in}}{\pgfqpoint{2.804742in}{1.836195in}}{\pgfqpoint{2.815792in}{1.836195in}}%
\pgfpathclose%
\pgfusepath{stroke,fill}%
\end{pgfscope}%
\begin{pgfscope}%
\pgfpathrectangle{\pgfqpoint{0.600000in}{0.600000in}}{\pgfqpoint{3.900000in}{3.900000in}}%
\pgfusepath{clip}%
\pgfsetbuttcap%
\pgfsetroundjoin%
\definecolor{currentfill}{rgb}{0.121569,0.466667,0.705882}%
\pgfsetfillcolor{currentfill}%
\pgfsetlinewidth{1.003750pt}%
\definecolor{currentstroke}{rgb}{0.121569,0.466667,0.705882}%
\pgfsetstrokecolor{currentstroke}%
\pgfsetdash{}{0pt}%
\pgfpathmoveto{\pgfqpoint{2.483917in}{2.113006in}}%
\pgfpathcurveto{\pgfqpoint{2.494967in}{2.113006in}}{\pgfqpoint{2.505566in}{2.117396in}}{\pgfqpoint{2.513380in}{2.125210in}}%
\pgfpathcurveto{\pgfqpoint{2.521193in}{2.133024in}}{\pgfqpoint{2.525584in}{2.143623in}}{\pgfqpoint{2.525584in}{2.154673in}}%
\pgfpathcurveto{\pgfqpoint{2.525584in}{2.165723in}}{\pgfqpoint{2.521193in}{2.176322in}}{\pgfqpoint{2.513380in}{2.184135in}}%
\pgfpathcurveto{\pgfqpoint{2.505566in}{2.191949in}}{\pgfqpoint{2.494967in}{2.196339in}}{\pgfqpoint{2.483917in}{2.196339in}}%
\pgfpathcurveto{\pgfqpoint{2.472867in}{2.196339in}}{\pgfqpoint{2.462268in}{2.191949in}}{\pgfqpoint{2.454454in}{2.184135in}}%
\pgfpathcurveto{\pgfqpoint{2.446641in}{2.176322in}}{\pgfqpoint{2.442250in}{2.165723in}}{\pgfqpoint{2.442250in}{2.154673in}}%
\pgfpathcurveto{\pgfqpoint{2.442250in}{2.143623in}}{\pgfqpoint{2.446641in}{2.133024in}}{\pgfqpoint{2.454454in}{2.125210in}}%
\pgfpathcurveto{\pgfqpoint{2.462268in}{2.117396in}}{\pgfqpoint{2.472867in}{2.113006in}}{\pgfqpoint{2.483917in}{2.113006in}}%
\pgfpathclose%
\pgfusepath{stroke,fill}%
\end{pgfscope}%
\begin{pgfscope}%
\pgfpathrectangle{\pgfqpoint{0.600000in}{0.600000in}}{\pgfqpoint{3.900000in}{3.900000in}}%
\pgfusepath{clip}%
\pgfsetbuttcap%
\pgfsetroundjoin%
\definecolor{currentfill}{rgb}{0.121569,0.466667,0.705882}%
\pgfsetfillcolor{currentfill}%
\pgfsetlinewidth{1.003750pt}%
\definecolor{currentstroke}{rgb}{0.121569,0.466667,0.705882}%
\pgfsetstrokecolor{currentstroke}%
\pgfsetdash{}{0pt}%
\pgfpathmoveto{\pgfqpoint{2.629863in}{2.715207in}}%
\pgfpathcurveto{\pgfqpoint{2.640914in}{2.715207in}}{\pgfqpoint{2.651513in}{2.719597in}}{\pgfqpoint{2.659326in}{2.727411in}}%
\pgfpathcurveto{\pgfqpoint{2.667140in}{2.735224in}}{\pgfqpoint{2.671530in}{2.745823in}}{\pgfqpoint{2.671530in}{2.756874in}}%
\pgfpathcurveto{\pgfqpoint{2.671530in}{2.767924in}}{\pgfqpoint{2.667140in}{2.778523in}}{\pgfqpoint{2.659326in}{2.786336in}}%
\pgfpathcurveto{\pgfqpoint{2.651513in}{2.794150in}}{\pgfqpoint{2.640914in}{2.798540in}}{\pgfqpoint{2.629863in}{2.798540in}}%
\pgfpathcurveto{\pgfqpoint{2.618813in}{2.798540in}}{\pgfqpoint{2.608214in}{2.794150in}}{\pgfqpoint{2.600401in}{2.786336in}}%
\pgfpathcurveto{\pgfqpoint{2.592587in}{2.778523in}}{\pgfqpoint{2.588197in}{2.767924in}}{\pgfqpoint{2.588197in}{2.756874in}}%
\pgfpathcurveto{\pgfqpoint{2.588197in}{2.745823in}}{\pgfqpoint{2.592587in}{2.735224in}}{\pgfqpoint{2.600401in}{2.727411in}}%
\pgfpathcurveto{\pgfqpoint{2.608214in}{2.719597in}}{\pgfqpoint{2.618813in}{2.715207in}}{\pgfqpoint{2.629863in}{2.715207in}}%
\pgfpathclose%
\pgfusepath{stroke,fill}%
\end{pgfscope}%
\begin{pgfscope}%
\pgfpathrectangle{\pgfqpoint{0.600000in}{0.600000in}}{\pgfqpoint{3.900000in}{3.900000in}}%
\pgfusepath{clip}%
\pgfsetbuttcap%
\pgfsetroundjoin%
\definecolor{currentfill}{rgb}{0.121569,0.466667,0.705882}%
\pgfsetfillcolor{currentfill}%
\pgfsetlinewidth{1.003750pt}%
\definecolor{currentstroke}{rgb}{0.121569,0.466667,0.705882}%
\pgfsetstrokecolor{currentstroke}%
\pgfsetdash{}{0pt}%
\pgfpathmoveto{\pgfqpoint{3.109528in}{2.929722in}}%
\pgfpathcurveto{\pgfqpoint{3.120578in}{2.929722in}}{\pgfqpoint{3.131177in}{2.934112in}}{\pgfqpoint{3.138990in}{2.941926in}}%
\pgfpathcurveto{\pgfqpoint{3.146804in}{2.949739in}}{\pgfqpoint{3.151194in}{2.960338in}}{\pgfqpoint{3.151194in}{2.971388in}}%
\pgfpathcurveto{\pgfqpoint{3.151194in}{2.982438in}}{\pgfqpoint{3.146804in}{2.993038in}}{\pgfqpoint{3.138990in}{3.000851in}}%
\pgfpathcurveto{\pgfqpoint{3.131177in}{3.008665in}}{\pgfqpoint{3.120578in}{3.013055in}}{\pgfqpoint{3.109528in}{3.013055in}}%
\pgfpathcurveto{\pgfqpoint{3.098478in}{3.013055in}}{\pgfqpoint{3.087878in}{3.008665in}}{\pgfqpoint{3.080065in}{3.000851in}}%
\pgfpathcurveto{\pgfqpoint{3.072251in}{2.993038in}}{\pgfqpoint{3.067861in}{2.982438in}}{\pgfqpoint{3.067861in}{2.971388in}}%
\pgfpathcurveto{\pgfqpoint{3.067861in}{2.960338in}}{\pgfqpoint{3.072251in}{2.949739in}}{\pgfqpoint{3.080065in}{2.941926in}}%
\pgfpathcurveto{\pgfqpoint{3.087878in}{2.934112in}}{\pgfqpoint{3.098478in}{2.929722in}}{\pgfqpoint{3.109528in}{2.929722in}}%
\pgfpathclose%
\pgfusepath{stroke,fill}%
\end{pgfscope}%
\begin{pgfscope}%
\pgfpathrectangle{\pgfqpoint{0.600000in}{0.600000in}}{\pgfqpoint{3.900000in}{3.900000in}}%
\pgfusepath{clip}%
\pgfsetbuttcap%
\pgfsetroundjoin%
\definecolor{currentfill}{rgb}{0.121569,0.466667,0.705882}%
\pgfsetfillcolor{currentfill}%
\pgfsetlinewidth{1.003750pt}%
\definecolor{currentstroke}{rgb}{0.121569,0.466667,0.705882}%
\pgfsetstrokecolor{currentstroke}%
\pgfsetdash{}{0pt}%
\pgfpathmoveto{\pgfqpoint{3.076369in}{3.427516in}}%
\pgfpathcurveto{\pgfqpoint{3.087419in}{3.427516in}}{\pgfqpoint{3.098018in}{3.431906in}}{\pgfqpoint{3.105832in}{3.439720in}}%
\pgfpathcurveto{\pgfqpoint{3.113645in}{3.447533in}}{\pgfqpoint{3.118035in}{3.458132in}}{\pgfqpoint{3.118035in}{3.469183in}}%
\pgfpathcurveto{\pgfqpoint{3.118035in}{3.480233in}}{\pgfqpoint{3.113645in}{3.490832in}}{\pgfqpoint{3.105832in}{3.498645in}}%
\pgfpathcurveto{\pgfqpoint{3.098018in}{3.506459in}}{\pgfqpoint{3.087419in}{3.510849in}}{\pgfqpoint{3.076369in}{3.510849in}}%
\pgfpathcurveto{\pgfqpoint{3.065319in}{3.510849in}}{\pgfqpoint{3.054720in}{3.506459in}}{\pgfqpoint{3.046906in}{3.498645in}}%
\pgfpathcurveto{\pgfqpoint{3.039092in}{3.490832in}}{\pgfqpoint{3.034702in}{3.480233in}}{\pgfqpoint{3.034702in}{3.469183in}}%
\pgfpathcurveto{\pgfqpoint{3.034702in}{3.458132in}}{\pgfqpoint{3.039092in}{3.447533in}}{\pgfqpoint{3.046906in}{3.439720in}}%
\pgfpathcurveto{\pgfqpoint{3.054720in}{3.431906in}}{\pgfqpoint{3.065319in}{3.427516in}}{\pgfqpoint{3.076369in}{3.427516in}}%
\pgfpathclose%
\pgfusepath{stroke,fill}%
\end{pgfscope}%
\begin{pgfscope}%
\pgfpathrectangle{\pgfqpoint{0.600000in}{0.600000in}}{\pgfqpoint{3.900000in}{3.900000in}}%
\pgfusepath{clip}%
\pgfsetbuttcap%
\pgfsetroundjoin%
\definecolor{currentfill}{rgb}{0.121569,0.466667,0.705882}%
\pgfsetfillcolor{currentfill}%
\pgfsetlinewidth{1.003750pt}%
\definecolor{currentstroke}{rgb}{0.121569,0.466667,0.705882}%
\pgfsetstrokecolor{currentstroke}%
\pgfsetdash{}{0pt}%
\pgfpathmoveto{\pgfqpoint{2.616563in}{2.685906in}}%
\pgfpathcurveto{\pgfqpoint{2.627613in}{2.685906in}}{\pgfqpoint{2.638212in}{2.690296in}}{\pgfqpoint{2.646025in}{2.698110in}}%
\pgfpathcurveto{\pgfqpoint{2.653839in}{2.705923in}}{\pgfqpoint{2.658229in}{2.716522in}}{\pgfqpoint{2.658229in}{2.727572in}}%
\pgfpathcurveto{\pgfqpoint{2.658229in}{2.738622in}}{\pgfqpoint{2.653839in}{2.749221in}}{\pgfqpoint{2.646025in}{2.757035in}}%
\pgfpathcurveto{\pgfqpoint{2.638212in}{2.764849in}}{\pgfqpoint{2.627613in}{2.769239in}}{\pgfqpoint{2.616563in}{2.769239in}}%
\pgfpathcurveto{\pgfqpoint{2.605513in}{2.769239in}}{\pgfqpoint{2.594914in}{2.764849in}}{\pgfqpoint{2.587100in}{2.757035in}}%
\pgfpathcurveto{\pgfqpoint{2.579286in}{2.749221in}}{\pgfqpoint{2.574896in}{2.738622in}}{\pgfqpoint{2.574896in}{2.727572in}}%
\pgfpathcurveto{\pgfqpoint{2.574896in}{2.716522in}}{\pgfqpoint{2.579286in}{2.705923in}}{\pgfqpoint{2.587100in}{2.698110in}}%
\pgfpathcurveto{\pgfqpoint{2.594914in}{2.690296in}}{\pgfqpoint{2.605513in}{2.685906in}}{\pgfqpoint{2.616563in}{2.685906in}}%
\pgfpathclose%
\pgfusepath{stroke,fill}%
\end{pgfscope}%
\begin{pgfscope}%
\pgfpathrectangle{\pgfqpoint{0.600000in}{0.600000in}}{\pgfqpoint{3.900000in}{3.900000in}}%
\pgfusepath{clip}%
\pgfsetbuttcap%
\pgfsetroundjoin%
\definecolor{currentfill}{rgb}{0.121569,0.466667,0.705882}%
\pgfsetfillcolor{currentfill}%
\pgfsetlinewidth{1.003750pt}%
\definecolor{currentstroke}{rgb}{0.121569,0.466667,0.705882}%
\pgfsetstrokecolor{currentstroke}%
\pgfsetdash{}{0pt}%
\pgfpathmoveto{\pgfqpoint{2.828302in}{2.354101in}}%
\pgfpathcurveto{\pgfqpoint{2.839352in}{2.354101in}}{\pgfqpoint{2.849951in}{2.358491in}}{\pgfqpoint{2.857765in}{2.366305in}}%
\pgfpathcurveto{\pgfqpoint{2.865578in}{2.374119in}}{\pgfqpoint{2.869968in}{2.384718in}}{\pgfqpoint{2.869968in}{2.395768in}}%
\pgfpathcurveto{\pgfqpoint{2.869968in}{2.406818in}}{\pgfqpoint{2.865578in}{2.417417in}}{\pgfqpoint{2.857765in}{2.425230in}}%
\pgfpathcurveto{\pgfqpoint{2.849951in}{2.433044in}}{\pgfqpoint{2.839352in}{2.437434in}}{\pgfqpoint{2.828302in}{2.437434in}}%
\pgfpathcurveto{\pgfqpoint{2.817252in}{2.437434in}}{\pgfqpoint{2.806653in}{2.433044in}}{\pgfqpoint{2.798839in}{2.425230in}}%
\pgfpathcurveto{\pgfqpoint{2.791025in}{2.417417in}}{\pgfqpoint{2.786635in}{2.406818in}}{\pgfqpoint{2.786635in}{2.395768in}}%
\pgfpathcurveto{\pgfqpoint{2.786635in}{2.384718in}}{\pgfqpoint{2.791025in}{2.374119in}}{\pgfqpoint{2.798839in}{2.366305in}}%
\pgfpathcurveto{\pgfqpoint{2.806653in}{2.358491in}}{\pgfqpoint{2.817252in}{2.354101in}}{\pgfqpoint{2.828302in}{2.354101in}}%
\pgfpathclose%
\pgfusepath{stroke,fill}%
\end{pgfscope}%
\begin{pgfscope}%
\pgfpathrectangle{\pgfqpoint{0.600000in}{0.600000in}}{\pgfqpoint{3.900000in}{3.900000in}}%
\pgfusepath{clip}%
\pgfsetbuttcap%
\pgfsetroundjoin%
\definecolor{currentfill}{rgb}{0.121569,0.466667,0.705882}%
\pgfsetfillcolor{currentfill}%
\pgfsetlinewidth{1.003750pt}%
\definecolor{currentstroke}{rgb}{0.121569,0.466667,0.705882}%
\pgfsetstrokecolor{currentstroke}%
\pgfsetdash{}{0pt}%
\pgfpathmoveto{\pgfqpoint{3.329801in}{1.940794in}}%
\pgfpathcurveto{\pgfqpoint{3.340851in}{1.940794in}}{\pgfqpoint{3.351450in}{1.945184in}}{\pgfqpoint{3.359264in}{1.952998in}}%
\pgfpathcurveto{\pgfqpoint{3.367077in}{1.960812in}}{\pgfqpoint{3.371467in}{1.971411in}}{\pgfqpoint{3.371467in}{1.982461in}}%
\pgfpathcurveto{\pgfqpoint{3.371467in}{1.993511in}}{\pgfqpoint{3.367077in}{2.004110in}}{\pgfqpoint{3.359264in}{2.011923in}}%
\pgfpathcurveto{\pgfqpoint{3.351450in}{2.019737in}}{\pgfqpoint{3.340851in}{2.024127in}}{\pgfqpoint{3.329801in}{2.024127in}}%
\pgfpathcurveto{\pgfqpoint{3.318751in}{2.024127in}}{\pgfqpoint{3.308152in}{2.019737in}}{\pgfqpoint{3.300338in}{2.011923in}}%
\pgfpathcurveto{\pgfqpoint{3.292524in}{2.004110in}}{\pgfqpoint{3.288134in}{1.993511in}}{\pgfqpoint{3.288134in}{1.982461in}}%
\pgfpathcurveto{\pgfqpoint{3.288134in}{1.971411in}}{\pgfqpoint{3.292524in}{1.960812in}}{\pgfqpoint{3.300338in}{1.952998in}}%
\pgfpathcurveto{\pgfqpoint{3.308152in}{1.945184in}}{\pgfqpoint{3.318751in}{1.940794in}}{\pgfqpoint{3.329801in}{1.940794in}}%
\pgfpathclose%
\pgfusepath{stroke,fill}%
\end{pgfscope}%
\begin{pgfscope}%
\pgfpathrectangle{\pgfqpoint{0.600000in}{0.600000in}}{\pgfqpoint{3.900000in}{3.900000in}}%
\pgfusepath{clip}%
\pgfsetbuttcap%
\pgfsetroundjoin%
\definecolor{currentfill}{rgb}{0.121569,0.466667,0.705882}%
\pgfsetfillcolor{currentfill}%
\pgfsetlinewidth{1.003750pt}%
\definecolor{currentstroke}{rgb}{0.121569,0.466667,0.705882}%
\pgfsetstrokecolor{currentstroke}%
\pgfsetdash{}{0pt}%
\pgfpathmoveto{\pgfqpoint{3.451117in}{2.626613in}}%
\pgfpathcurveto{\pgfqpoint{3.462168in}{2.626613in}}{\pgfqpoint{3.472767in}{2.631003in}}{\pgfqpoint{3.480580in}{2.638817in}}%
\pgfpathcurveto{\pgfqpoint{3.488394in}{2.646630in}}{\pgfqpoint{3.492784in}{2.657229in}}{\pgfqpoint{3.492784in}{2.668279in}}%
\pgfpathcurveto{\pgfqpoint{3.492784in}{2.679330in}}{\pgfqpoint{3.488394in}{2.689929in}}{\pgfqpoint{3.480580in}{2.697742in}}%
\pgfpathcurveto{\pgfqpoint{3.472767in}{2.705556in}}{\pgfqpoint{3.462168in}{2.709946in}}{\pgfqpoint{3.451117in}{2.709946in}}%
\pgfpathcurveto{\pgfqpoint{3.440067in}{2.709946in}}{\pgfqpoint{3.429468in}{2.705556in}}{\pgfqpoint{3.421655in}{2.697742in}}%
\pgfpathcurveto{\pgfqpoint{3.413841in}{2.689929in}}{\pgfqpoint{3.409451in}{2.679330in}}{\pgfqpoint{3.409451in}{2.668279in}}%
\pgfpathcurveto{\pgfqpoint{3.409451in}{2.657229in}}{\pgfqpoint{3.413841in}{2.646630in}}{\pgfqpoint{3.421655in}{2.638817in}}%
\pgfpathcurveto{\pgfqpoint{3.429468in}{2.631003in}}{\pgfqpoint{3.440067in}{2.626613in}}{\pgfqpoint{3.451117in}{2.626613in}}%
\pgfpathclose%
\pgfusepath{stroke,fill}%
\end{pgfscope}%
\begin{pgfscope}%
\pgfpathrectangle{\pgfqpoint{0.600000in}{0.600000in}}{\pgfqpoint{3.900000in}{3.900000in}}%
\pgfusepath{clip}%
\pgfsetbuttcap%
\pgfsetroundjoin%
\definecolor{currentfill}{rgb}{0.121569,0.466667,0.705882}%
\pgfsetfillcolor{currentfill}%
\pgfsetlinewidth{1.003750pt}%
\definecolor{currentstroke}{rgb}{0.121569,0.466667,0.705882}%
\pgfsetstrokecolor{currentstroke}%
\pgfsetdash{}{0pt}%
\pgfpathmoveto{\pgfqpoint{2.873698in}{3.427915in}}%
\pgfpathcurveto{\pgfqpoint{2.884748in}{3.427915in}}{\pgfqpoint{2.895347in}{3.432305in}}{\pgfqpoint{2.903161in}{3.440119in}}%
\pgfpathcurveto{\pgfqpoint{2.910974in}{3.447933in}}{\pgfqpoint{2.915364in}{3.458532in}}{\pgfqpoint{2.915364in}{3.469582in}}%
\pgfpathcurveto{\pgfqpoint{2.915364in}{3.480632in}}{\pgfqpoint{2.910974in}{3.491231in}}{\pgfqpoint{2.903161in}{3.499045in}}%
\pgfpathcurveto{\pgfqpoint{2.895347in}{3.506858in}}{\pgfqpoint{2.884748in}{3.511248in}}{\pgfqpoint{2.873698in}{3.511248in}}%
\pgfpathcurveto{\pgfqpoint{2.862648in}{3.511248in}}{\pgfqpoint{2.852049in}{3.506858in}}{\pgfqpoint{2.844235in}{3.499045in}}%
\pgfpathcurveto{\pgfqpoint{2.836421in}{3.491231in}}{\pgfqpoint{2.832031in}{3.480632in}}{\pgfqpoint{2.832031in}{3.469582in}}%
\pgfpathcurveto{\pgfqpoint{2.832031in}{3.458532in}}{\pgfqpoint{2.836421in}{3.447933in}}{\pgfqpoint{2.844235in}{3.440119in}}%
\pgfpathcurveto{\pgfqpoint{2.852049in}{3.432305in}}{\pgfqpoint{2.862648in}{3.427915in}}{\pgfqpoint{2.873698in}{3.427915in}}%
\pgfpathclose%
\pgfusepath{stroke,fill}%
\end{pgfscope}%
\begin{pgfscope}%
\pgfpathrectangle{\pgfqpoint{0.600000in}{0.600000in}}{\pgfqpoint{3.900000in}{3.900000in}}%
\pgfusepath{clip}%
\pgfsetbuttcap%
\pgfsetroundjoin%
\definecolor{currentfill}{rgb}{0.121569,0.466667,0.705882}%
\pgfsetfillcolor{currentfill}%
\pgfsetlinewidth{1.003750pt}%
\definecolor{currentstroke}{rgb}{0.121569,0.466667,0.705882}%
\pgfsetstrokecolor{currentstroke}%
\pgfsetdash{}{0pt}%
\pgfpathmoveto{\pgfqpoint{2.872686in}{2.202474in}}%
\pgfpathcurveto{\pgfqpoint{2.883736in}{2.202474in}}{\pgfqpoint{2.894335in}{2.206865in}}{\pgfqpoint{2.902149in}{2.214678in}}%
\pgfpathcurveto{\pgfqpoint{2.909962in}{2.222492in}}{\pgfqpoint{2.914353in}{2.233091in}}{\pgfqpoint{2.914353in}{2.244141in}}%
\pgfpathcurveto{\pgfqpoint{2.914353in}{2.255191in}}{\pgfqpoint{2.909962in}{2.265790in}}{\pgfqpoint{2.902149in}{2.273604in}}%
\pgfpathcurveto{\pgfqpoint{2.894335in}{2.281417in}}{\pgfqpoint{2.883736in}{2.285808in}}{\pgfqpoint{2.872686in}{2.285808in}}%
\pgfpathcurveto{\pgfqpoint{2.861636in}{2.285808in}}{\pgfqpoint{2.851037in}{2.281417in}}{\pgfqpoint{2.843223in}{2.273604in}}%
\pgfpathcurveto{\pgfqpoint{2.835410in}{2.265790in}}{\pgfqpoint{2.831019in}{2.255191in}}{\pgfqpoint{2.831019in}{2.244141in}}%
\pgfpathcurveto{\pgfqpoint{2.831019in}{2.233091in}}{\pgfqpoint{2.835410in}{2.222492in}}{\pgfqpoint{2.843223in}{2.214678in}}%
\pgfpathcurveto{\pgfqpoint{2.851037in}{2.206865in}}{\pgfqpoint{2.861636in}{2.202474in}}{\pgfqpoint{2.872686in}{2.202474in}}%
\pgfpathclose%
\pgfusepath{stroke,fill}%
\end{pgfscope}%
\begin{pgfscope}%
\pgfpathrectangle{\pgfqpoint{0.600000in}{0.600000in}}{\pgfqpoint{3.900000in}{3.900000in}}%
\pgfusepath{clip}%
\pgfsetbuttcap%
\pgfsetroundjoin%
\definecolor{currentfill}{rgb}{0.121569,0.466667,0.705882}%
\pgfsetfillcolor{currentfill}%
\pgfsetlinewidth{1.003750pt}%
\definecolor{currentstroke}{rgb}{0.121569,0.466667,0.705882}%
\pgfsetstrokecolor{currentstroke}%
\pgfsetdash{}{0pt}%
\pgfpathmoveto{\pgfqpoint{2.925540in}{3.586717in}}%
\pgfpathcurveto{\pgfqpoint{2.936590in}{3.586717in}}{\pgfqpoint{2.947189in}{3.591107in}}{\pgfqpoint{2.955003in}{3.598921in}}%
\pgfpathcurveto{\pgfqpoint{2.962816in}{3.606734in}}{\pgfqpoint{2.967207in}{3.617334in}}{\pgfqpoint{2.967207in}{3.628384in}}%
\pgfpathcurveto{\pgfqpoint{2.967207in}{3.639434in}}{\pgfqpoint{2.962816in}{3.650033in}}{\pgfqpoint{2.955003in}{3.657846in}}%
\pgfpathcurveto{\pgfqpoint{2.947189in}{3.665660in}}{\pgfqpoint{2.936590in}{3.670050in}}{\pgfqpoint{2.925540in}{3.670050in}}%
\pgfpathcurveto{\pgfqpoint{2.914490in}{3.670050in}}{\pgfqpoint{2.903891in}{3.665660in}}{\pgfqpoint{2.896077in}{3.657846in}}%
\pgfpathcurveto{\pgfqpoint{2.888263in}{3.650033in}}{\pgfqpoint{2.883873in}{3.639434in}}{\pgfqpoint{2.883873in}{3.628384in}}%
\pgfpathcurveto{\pgfqpoint{2.883873in}{3.617334in}}{\pgfqpoint{2.888263in}{3.606734in}}{\pgfqpoint{2.896077in}{3.598921in}}%
\pgfpathcurveto{\pgfqpoint{2.903891in}{3.591107in}}{\pgfqpoint{2.914490in}{3.586717in}}{\pgfqpoint{2.925540in}{3.586717in}}%
\pgfpathclose%
\pgfusepath{stroke,fill}%
\end{pgfscope}%
\begin{pgfscope}%
\pgfpathrectangle{\pgfqpoint{0.600000in}{0.600000in}}{\pgfqpoint{3.900000in}{3.900000in}}%
\pgfusepath{clip}%
\pgfsetbuttcap%
\pgfsetroundjoin%
\definecolor{currentfill}{rgb}{0.121569,0.466667,0.705882}%
\pgfsetfillcolor{currentfill}%
\pgfsetlinewidth{1.003750pt}%
\definecolor{currentstroke}{rgb}{0.121569,0.466667,0.705882}%
\pgfsetstrokecolor{currentstroke}%
\pgfsetdash{}{0pt}%
\pgfpathmoveto{\pgfqpoint{2.764515in}{2.163153in}}%
\pgfpathcurveto{\pgfqpoint{2.775565in}{2.163153in}}{\pgfqpoint{2.786164in}{2.167543in}}{\pgfqpoint{2.793977in}{2.175357in}}%
\pgfpathcurveto{\pgfqpoint{2.801791in}{2.183171in}}{\pgfqpoint{2.806181in}{2.193770in}}{\pgfqpoint{2.806181in}{2.204820in}}%
\pgfpathcurveto{\pgfqpoint{2.806181in}{2.215870in}}{\pgfqpoint{2.801791in}{2.226469in}}{\pgfqpoint{2.793977in}{2.234283in}}%
\pgfpathcurveto{\pgfqpoint{2.786164in}{2.242096in}}{\pgfqpoint{2.775565in}{2.246486in}}{\pgfqpoint{2.764515in}{2.246486in}}%
\pgfpathcurveto{\pgfqpoint{2.753465in}{2.246486in}}{\pgfqpoint{2.742866in}{2.242096in}}{\pgfqpoint{2.735052in}{2.234283in}}%
\pgfpathcurveto{\pgfqpoint{2.727238in}{2.226469in}}{\pgfqpoint{2.722848in}{2.215870in}}{\pgfqpoint{2.722848in}{2.204820in}}%
\pgfpathcurveto{\pgfqpoint{2.722848in}{2.193770in}}{\pgfqpoint{2.727238in}{2.183171in}}{\pgfqpoint{2.735052in}{2.175357in}}%
\pgfpathcurveto{\pgfqpoint{2.742866in}{2.167543in}}{\pgfqpoint{2.753465in}{2.163153in}}{\pgfqpoint{2.764515in}{2.163153in}}%
\pgfpathclose%
\pgfusepath{stroke,fill}%
\end{pgfscope}%
\begin{pgfscope}%
\pgfpathrectangle{\pgfqpoint{0.600000in}{0.600000in}}{\pgfqpoint{3.900000in}{3.900000in}}%
\pgfusepath{clip}%
\pgfsetbuttcap%
\pgfsetroundjoin%
\definecolor{currentfill}{rgb}{0.121569,0.466667,0.705882}%
\pgfsetfillcolor{currentfill}%
\pgfsetlinewidth{1.003750pt}%
\definecolor{currentstroke}{rgb}{0.121569,0.466667,0.705882}%
\pgfsetstrokecolor{currentstroke}%
\pgfsetdash{}{0pt}%
\pgfpathmoveto{\pgfqpoint{2.687539in}{2.069643in}}%
\pgfpathcurveto{\pgfqpoint{2.698589in}{2.069643in}}{\pgfqpoint{2.709188in}{2.074033in}}{\pgfqpoint{2.717002in}{2.081847in}}%
\pgfpathcurveto{\pgfqpoint{2.724815in}{2.089661in}}{\pgfqpoint{2.729206in}{2.100260in}}{\pgfqpoint{2.729206in}{2.111310in}}%
\pgfpathcurveto{\pgfqpoint{2.729206in}{2.122360in}}{\pgfqpoint{2.724815in}{2.132959in}}{\pgfqpoint{2.717002in}{2.140772in}}%
\pgfpathcurveto{\pgfqpoint{2.709188in}{2.148586in}}{\pgfqpoint{2.698589in}{2.152976in}}{\pgfqpoint{2.687539in}{2.152976in}}%
\pgfpathcurveto{\pgfqpoint{2.676489in}{2.152976in}}{\pgfqpoint{2.665890in}{2.148586in}}{\pgfqpoint{2.658076in}{2.140772in}}%
\pgfpathcurveto{\pgfqpoint{2.650263in}{2.132959in}}{\pgfqpoint{2.645872in}{2.122360in}}{\pgfqpoint{2.645872in}{2.111310in}}%
\pgfpathcurveto{\pgfqpoint{2.645872in}{2.100260in}}{\pgfqpoint{2.650263in}{2.089661in}}{\pgfqpoint{2.658076in}{2.081847in}}%
\pgfpathcurveto{\pgfqpoint{2.665890in}{2.074033in}}{\pgfqpoint{2.676489in}{2.069643in}}{\pgfqpoint{2.687539in}{2.069643in}}%
\pgfpathclose%
\pgfusepath{stroke,fill}%
\end{pgfscope}%
\begin{pgfscope}%
\pgfpathrectangle{\pgfqpoint{0.600000in}{0.600000in}}{\pgfqpoint{3.900000in}{3.900000in}}%
\pgfusepath{clip}%
\pgfsetbuttcap%
\pgfsetroundjoin%
\definecolor{currentfill}{rgb}{0.121569,0.466667,0.705882}%
\pgfsetfillcolor{currentfill}%
\pgfsetlinewidth{1.003750pt}%
\definecolor{currentstroke}{rgb}{0.121569,0.466667,0.705882}%
\pgfsetstrokecolor{currentstroke}%
\pgfsetdash{}{0pt}%
\pgfpathmoveto{\pgfqpoint{2.424897in}{1.198804in}}%
\pgfpathcurveto{\pgfqpoint{2.435947in}{1.198804in}}{\pgfqpoint{2.446546in}{1.203194in}}{\pgfqpoint{2.454360in}{1.211007in}}%
\pgfpathcurveto{\pgfqpoint{2.462174in}{1.218821in}}{\pgfqpoint{2.466564in}{1.229420in}}{\pgfqpoint{2.466564in}{1.240470in}}%
\pgfpathcurveto{\pgfqpoint{2.466564in}{1.251520in}}{\pgfqpoint{2.462174in}{1.262119in}}{\pgfqpoint{2.454360in}{1.269933in}}%
\pgfpathcurveto{\pgfqpoint{2.446546in}{1.277747in}}{\pgfqpoint{2.435947in}{1.282137in}}{\pgfqpoint{2.424897in}{1.282137in}}%
\pgfpathcurveto{\pgfqpoint{2.413847in}{1.282137in}}{\pgfqpoint{2.403248in}{1.277747in}}{\pgfqpoint{2.395434in}{1.269933in}}%
\pgfpathcurveto{\pgfqpoint{2.387621in}{1.262119in}}{\pgfqpoint{2.383231in}{1.251520in}}{\pgfqpoint{2.383231in}{1.240470in}}%
\pgfpathcurveto{\pgfqpoint{2.383231in}{1.229420in}}{\pgfqpoint{2.387621in}{1.218821in}}{\pgfqpoint{2.395434in}{1.211007in}}%
\pgfpathcurveto{\pgfqpoint{2.403248in}{1.203194in}}{\pgfqpoint{2.413847in}{1.198804in}}{\pgfqpoint{2.424897in}{1.198804in}}%
\pgfpathclose%
\pgfusepath{stroke,fill}%
\end{pgfscope}%
\begin{pgfscope}%
\pgfpathrectangle{\pgfqpoint{0.600000in}{0.600000in}}{\pgfqpoint{3.900000in}{3.900000in}}%
\pgfusepath{clip}%
\pgfsetbuttcap%
\pgfsetroundjoin%
\definecolor{currentfill}{rgb}{0.121569,0.466667,0.705882}%
\pgfsetfillcolor{currentfill}%
\pgfsetlinewidth{1.003750pt}%
\definecolor{currentstroke}{rgb}{0.121569,0.466667,0.705882}%
\pgfsetstrokecolor{currentstroke}%
\pgfsetdash{}{0pt}%
\pgfpathmoveto{\pgfqpoint{2.831177in}{1.954592in}}%
\pgfpathcurveto{\pgfqpoint{2.842227in}{1.954592in}}{\pgfqpoint{2.852826in}{1.958982in}}{\pgfqpoint{2.860639in}{1.966796in}}%
\pgfpathcurveto{\pgfqpoint{2.868453in}{1.974609in}}{\pgfqpoint{2.872843in}{1.985208in}}{\pgfqpoint{2.872843in}{1.996258in}}%
\pgfpathcurveto{\pgfqpoint{2.872843in}{2.007309in}}{\pgfqpoint{2.868453in}{2.017908in}}{\pgfqpoint{2.860639in}{2.025721in}}%
\pgfpathcurveto{\pgfqpoint{2.852826in}{2.033535in}}{\pgfqpoint{2.842227in}{2.037925in}}{\pgfqpoint{2.831177in}{2.037925in}}%
\pgfpathcurveto{\pgfqpoint{2.820127in}{2.037925in}}{\pgfqpoint{2.809527in}{2.033535in}}{\pgfqpoint{2.801714in}{2.025721in}}%
\pgfpathcurveto{\pgfqpoint{2.793900in}{2.017908in}}{\pgfqpoint{2.789510in}{2.007309in}}{\pgfqpoint{2.789510in}{1.996258in}}%
\pgfpathcurveto{\pgfqpoint{2.789510in}{1.985208in}}{\pgfqpoint{2.793900in}{1.974609in}}{\pgfqpoint{2.801714in}{1.966796in}}%
\pgfpathcurveto{\pgfqpoint{2.809527in}{1.958982in}}{\pgfqpoint{2.820127in}{1.954592in}}{\pgfqpoint{2.831177in}{1.954592in}}%
\pgfpathclose%
\pgfusepath{stroke,fill}%
\end{pgfscope}%
\begin{pgfscope}%
\pgfpathrectangle{\pgfqpoint{0.600000in}{0.600000in}}{\pgfqpoint{3.900000in}{3.900000in}}%
\pgfusepath{clip}%
\pgfsetbuttcap%
\pgfsetroundjoin%
\definecolor{currentfill}{rgb}{0.121569,0.466667,0.705882}%
\pgfsetfillcolor{currentfill}%
\pgfsetlinewidth{1.003750pt}%
\definecolor{currentstroke}{rgb}{0.121569,0.466667,0.705882}%
\pgfsetstrokecolor{currentstroke}%
\pgfsetdash{}{0pt}%
\pgfpathmoveto{\pgfqpoint{2.567020in}{2.120710in}}%
\pgfpathcurveto{\pgfqpoint{2.578070in}{2.120710in}}{\pgfqpoint{2.588669in}{2.125101in}}{\pgfqpoint{2.596483in}{2.132914in}}%
\pgfpathcurveto{\pgfqpoint{2.604296in}{2.140728in}}{\pgfqpoint{2.608687in}{2.151327in}}{\pgfqpoint{2.608687in}{2.162377in}}%
\pgfpathcurveto{\pgfqpoint{2.608687in}{2.173427in}}{\pgfqpoint{2.604296in}{2.184026in}}{\pgfqpoint{2.596483in}{2.191840in}}%
\pgfpathcurveto{\pgfqpoint{2.588669in}{2.199653in}}{\pgfqpoint{2.578070in}{2.204044in}}{\pgfqpoint{2.567020in}{2.204044in}}%
\pgfpathcurveto{\pgfqpoint{2.555970in}{2.204044in}}{\pgfqpoint{2.545371in}{2.199653in}}{\pgfqpoint{2.537557in}{2.191840in}}%
\pgfpathcurveto{\pgfqpoint{2.529743in}{2.184026in}}{\pgfqpoint{2.525353in}{2.173427in}}{\pgfqpoint{2.525353in}{2.162377in}}%
\pgfpathcurveto{\pgfqpoint{2.525353in}{2.151327in}}{\pgfqpoint{2.529743in}{2.140728in}}{\pgfqpoint{2.537557in}{2.132914in}}%
\pgfpathcurveto{\pgfqpoint{2.545371in}{2.125101in}}{\pgfqpoint{2.555970in}{2.120710in}}{\pgfqpoint{2.567020in}{2.120710in}}%
\pgfpathclose%
\pgfusepath{stroke,fill}%
\end{pgfscope}%
\begin{pgfscope}%
\pgfpathrectangle{\pgfqpoint{0.600000in}{0.600000in}}{\pgfqpoint{3.900000in}{3.900000in}}%
\pgfusepath{clip}%
\pgfsetbuttcap%
\pgfsetroundjoin%
\definecolor{currentfill}{rgb}{0.121569,0.466667,0.705882}%
\pgfsetfillcolor{currentfill}%
\pgfsetlinewidth{1.003750pt}%
\definecolor{currentstroke}{rgb}{0.121569,0.466667,0.705882}%
\pgfsetstrokecolor{currentstroke}%
\pgfsetdash{}{0pt}%
\pgfpathmoveto{\pgfqpoint{2.399094in}{3.359078in}}%
\pgfpathcurveto{\pgfqpoint{2.410145in}{3.359078in}}{\pgfqpoint{2.420744in}{3.363469in}}{\pgfqpoint{2.428557in}{3.371282in}}%
\pgfpathcurveto{\pgfqpoint{2.436371in}{3.379096in}}{\pgfqpoint{2.440761in}{3.389695in}}{\pgfqpoint{2.440761in}{3.400745in}}%
\pgfpathcurveto{\pgfqpoint{2.440761in}{3.411795in}}{\pgfqpoint{2.436371in}{3.422394in}}{\pgfqpoint{2.428557in}{3.430208in}}%
\pgfpathcurveto{\pgfqpoint{2.420744in}{3.438021in}}{\pgfqpoint{2.410145in}{3.442412in}}{\pgfqpoint{2.399094in}{3.442412in}}%
\pgfpathcurveto{\pgfqpoint{2.388044in}{3.442412in}}{\pgfqpoint{2.377445in}{3.438021in}}{\pgfqpoint{2.369632in}{3.430208in}}%
\pgfpathcurveto{\pgfqpoint{2.361818in}{3.422394in}}{\pgfqpoint{2.357428in}{3.411795in}}{\pgfqpoint{2.357428in}{3.400745in}}%
\pgfpathcurveto{\pgfqpoint{2.357428in}{3.389695in}}{\pgfqpoint{2.361818in}{3.379096in}}{\pgfqpoint{2.369632in}{3.371282in}}%
\pgfpathcurveto{\pgfqpoint{2.377445in}{3.363469in}}{\pgfqpoint{2.388044in}{3.359078in}}{\pgfqpoint{2.399094in}{3.359078in}}%
\pgfpathclose%
\pgfusepath{stroke,fill}%
\end{pgfscope}%
\begin{pgfscope}%
\pgfpathrectangle{\pgfqpoint{0.600000in}{0.600000in}}{\pgfqpoint{3.900000in}{3.900000in}}%
\pgfusepath{clip}%
\pgfsetbuttcap%
\pgfsetroundjoin%
\definecolor{currentfill}{rgb}{0.121569,0.466667,0.705882}%
\pgfsetfillcolor{currentfill}%
\pgfsetlinewidth{1.003750pt}%
\definecolor{currentstroke}{rgb}{0.121569,0.466667,0.705882}%
\pgfsetstrokecolor{currentstroke}%
\pgfsetdash{}{0pt}%
\pgfpathmoveto{\pgfqpoint{2.224095in}{1.878266in}}%
\pgfpathcurveto{\pgfqpoint{2.235145in}{1.878266in}}{\pgfqpoint{2.245744in}{1.882656in}}{\pgfqpoint{2.253558in}{1.890470in}}%
\pgfpathcurveto{\pgfqpoint{2.261372in}{1.898284in}}{\pgfqpoint{2.265762in}{1.908883in}}{\pgfqpoint{2.265762in}{1.919933in}}%
\pgfpathcurveto{\pgfqpoint{2.265762in}{1.930983in}}{\pgfqpoint{2.261372in}{1.941582in}}{\pgfqpoint{2.253558in}{1.949396in}}%
\pgfpathcurveto{\pgfqpoint{2.245744in}{1.957209in}}{\pgfqpoint{2.235145in}{1.961600in}}{\pgfqpoint{2.224095in}{1.961600in}}%
\pgfpathcurveto{\pgfqpoint{2.213045in}{1.961600in}}{\pgfqpoint{2.202446in}{1.957209in}}{\pgfqpoint{2.194632in}{1.949396in}}%
\pgfpathcurveto{\pgfqpoint{2.186819in}{1.941582in}}{\pgfqpoint{2.182429in}{1.930983in}}{\pgfqpoint{2.182429in}{1.919933in}}%
\pgfpathcurveto{\pgfqpoint{2.182429in}{1.908883in}}{\pgfqpoint{2.186819in}{1.898284in}}{\pgfqpoint{2.194632in}{1.890470in}}%
\pgfpathcurveto{\pgfqpoint{2.202446in}{1.882656in}}{\pgfqpoint{2.213045in}{1.878266in}}{\pgfqpoint{2.224095in}{1.878266in}}%
\pgfpathclose%
\pgfusepath{stroke,fill}%
\end{pgfscope}%
\begin{pgfscope}%
\pgfpathrectangle{\pgfqpoint{0.600000in}{0.600000in}}{\pgfqpoint{3.900000in}{3.900000in}}%
\pgfusepath{clip}%
\pgfsetbuttcap%
\pgfsetroundjoin%
\definecolor{currentfill}{rgb}{0.121569,0.466667,0.705882}%
\pgfsetfillcolor{currentfill}%
\pgfsetlinewidth{1.003750pt}%
\definecolor{currentstroke}{rgb}{0.121569,0.466667,0.705882}%
\pgfsetstrokecolor{currentstroke}%
\pgfsetdash{}{0pt}%
\pgfpathmoveto{\pgfqpoint{2.647472in}{2.057694in}}%
\pgfpathcurveto{\pgfqpoint{2.658522in}{2.057694in}}{\pgfqpoint{2.669121in}{2.062084in}}{\pgfqpoint{2.676935in}{2.069897in}}%
\pgfpathcurveto{\pgfqpoint{2.684749in}{2.077711in}}{\pgfqpoint{2.689139in}{2.088310in}}{\pgfqpoint{2.689139in}{2.099360in}}%
\pgfpathcurveto{\pgfqpoint{2.689139in}{2.110410in}}{\pgfqpoint{2.684749in}{2.121009in}}{\pgfqpoint{2.676935in}{2.128823in}}%
\pgfpathcurveto{\pgfqpoint{2.669121in}{2.136637in}}{\pgfqpoint{2.658522in}{2.141027in}}{\pgfqpoint{2.647472in}{2.141027in}}%
\pgfpathcurveto{\pgfqpoint{2.636422in}{2.141027in}}{\pgfqpoint{2.625823in}{2.136637in}}{\pgfqpoint{2.618009in}{2.128823in}}%
\pgfpathcurveto{\pgfqpoint{2.610196in}{2.121009in}}{\pgfqpoint{2.605805in}{2.110410in}}{\pgfqpoint{2.605805in}{2.099360in}}%
\pgfpathcurveto{\pgfqpoint{2.605805in}{2.088310in}}{\pgfqpoint{2.610196in}{2.077711in}}{\pgfqpoint{2.618009in}{2.069897in}}%
\pgfpathcurveto{\pgfqpoint{2.625823in}{2.062084in}}{\pgfqpoint{2.636422in}{2.057694in}}{\pgfqpoint{2.647472in}{2.057694in}}%
\pgfpathclose%
\pgfusepath{stroke,fill}%
\end{pgfscope}%
\begin{pgfscope}%
\pgfpathrectangle{\pgfqpoint{0.600000in}{0.600000in}}{\pgfqpoint{3.900000in}{3.900000in}}%
\pgfusepath{clip}%
\pgfsetbuttcap%
\pgfsetroundjoin%
\definecolor{currentfill}{rgb}{0.121569,0.466667,0.705882}%
\pgfsetfillcolor{currentfill}%
\pgfsetlinewidth{1.003750pt}%
\definecolor{currentstroke}{rgb}{0.121569,0.466667,0.705882}%
\pgfsetstrokecolor{currentstroke}%
\pgfsetdash{}{0pt}%
\pgfpathmoveto{\pgfqpoint{2.436826in}{1.749842in}}%
\pgfpathcurveto{\pgfqpoint{2.447876in}{1.749842in}}{\pgfqpoint{2.458475in}{1.754233in}}{\pgfqpoint{2.466288in}{1.762046in}}%
\pgfpathcurveto{\pgfqpoint{2.474102in}{1.769860in}}{\pgfqpoint{2.478492in}{1.780459in}}{\pgfqpoint{2.478492in}{1.791509in}}%
\pgfpathcurveto{\pgfqpoint{2.478492in}{1.802559in}}{\pgfqpoint{2.474102in}{1.813158in}}{\pgfqpoint{2.466288in}{1.820972in}}%
\pgfpathcurveto{\pgfqpoint{2.458475in}{1.828785in}}{\pgfqpoint{2.447876in}{1.833176in}}{\pgfqpoint{2.436826in}{1.833176in}}%
\pgfpathcurveto{\pgfqpoint{2.425776in}{1.833176in}}{\pgfqpoint{2.415176in}{1.828785in}}{\pgfqpoint{2.407363in}{1.820972in}}%
\pgfpathcurveto{\pgfqpoint{2.399549in}{1.813158in}}{\pgfqpoint{2.395159in}{1.802559in}}{\pgfqpoint{2.395159in}{1.791509in}}%
\pgfpathcurveto{\pgfqpoint{2.395159in}{1.780459in}}{\pgfqpoint{2.399549in}{1.769860in}}{\pgfqpoint{2.407363in}{1.762046in}}%
\pgfpathcurveto{\pgfqpoint{2.415176in}{1.754233in}}{\pgfqpoint{2.425776in}{1.749842in}}{\pgfqpoint{2.436826in}{1.749842in}}%
\pgfpathclose%
\pgfusepath{stroke,fill}%
\end{pgfscope}%
\begin{pgfscope}%
\pgfpathrectangle{\pgfqpoint{0.600000in}{0.600000in}}{\pgfqpoint{3.900000in}{3.900000in}}%
\pgfusepath{clip}%
\pgfsetbuttcap%
\pgfsetroundjoin%
\definecolor{currentfill}{rgb}{0.121569,0.466667,0.705882}%
\pgfsetfillcolor{currentfill}%
\pgfsetlinewidth{1.003750pt}%
\definecolor{currentstroke}{rgb}{0.121569,0.466667,0.705882}%
\pgfsetstrokecolor{currentstroke}%
\pgfsetdash{}{0pt}%
\pgfpathmoveto{\pgfqpoint{2.998537in}{3.545610in}}%
\pgfpathcurveto{\pgfqpoint{3.009587in}{3.545610in}}{\pgfqpoint{3.020186in}{3.550000in}}{\pgfqpoint{3.028000in}{3.557814in}}%
\pgfpathcurveto{\pgfqpoint{3.035814in}{3.565627in}}{\pgfqpoint{3.040204in}{3.576226in}}{\pgfqpoint{3.040204in}{3.587277in}}%
\pgfpathcurveto{\pgfqpoint{3.040204in}{3.598327in}}{\pgfqpoint{3.035814in}{3.608926in}}{\pgfqpoint{3.028000in}{3.616739in}}%
\pgfpathcurveto{\pgfqpoint{3.020186in}{3.624553in}}{\pgfqpoint{3.009587in}{3.628943in}}{\pgfqpoint{2.998537in}{3.628943in}}%
\pgfpathcurveto{\pgfqpoint{2.987487in}{3.628943in}}{\pgfqpoint{2.976888in}{3.624553in}}{\pgfqpoint{2.969074in}{3.616739in}}%
\pgfpathcurveto{\pgfqpoint{2.961261in}{3.608926in}}{\pgfqpoint{2.956871in}{3.598327in}}{\pgfqpoint{2.956871in}{3.587277in}}%
\pgfpathcurveto{\pgfqpoint{2.956871in}{3.576226in}}{\pgfqpoint{2.961261in}{3.565627in}}{\pgfqpoint{2.969074in}{3.557814in}}%
\pgfpathcurveto{\pgfqpoint{2.976888in}{3.550000in}}{\pgfqpoint{2.987487in}{3.545610in}}{\pgfqpoint{2.998537in}{3.545610in}}%
\pgfpathclose%
\pgfusepath{stroke,fill}%
\end{pgfscope}%
\begin{pgfscope}%
\pgfpathrectangle{\pgfqpoint{0.600000in}{0.600000in}}{\pgfqpoint{3.900000in}{3.900000in}}%
\pgfusepath{clip}%
\pgfsetbuttcap%
\pgfsetroundjoin%
\definecolor{currentfill}{rgb}{0.121569,0.466667,0.705882}%
\pgfsetfillcolor{currentfill}%
\pgfsetlinewidth{1.003750pt}%
\definecolor{currentstroke}{rgb}{0.121569,0.466667,0.705882}%
\pgfsetstrokecolor{currentstroke}%
\pgfsetdash{}{0pt}%
\pgfpathmoveto{\pgfqpoint{1.872756in}{2.269880in}}%
\pgfpathcurveto{\pgfqpoint{1.883806in}{2.269880in}}{\pgfqpoint{1.894405in}{2.274271in}}{\pgfqpoint{1.902218in}{2.282084in}}%
\pgfpathcurveto{\pgfqpoint{1.910032in}{2.289898in}}{\pgfqpoint{1.914422in}{2.300497in}}{\pgfqpoint{1.914422in}{2.311547in}}%
\pgfpathcurveto{\pgfqpoint{1.914422in}{2.322597in}}{\pgfqpoint{1.910032in}{2.333196in}}{\pgfqpoint{1.902218in}{2.341010in}}%
\pgfpathcurveto{\pgfqpoint{1.894405in}{2.348823in}}{\pgfqpoint{1.883806in}{2.353214in}}{\pgfqpoint{1.872756in}{2.353214in}}%
\pgfpathcurveto{\pgfqpoint{1.861706in}{2.353214in}}{\pgfqpoint{1.851107in}{2.348823in}}{\pgfqpoint{1.843293in}{2.341010in}}%
\pgfpathcurveto{\pgfqpoint{1.835479in}{2.333196in}}{\pgfqpoint{1.831089in}{2.322597in}}{\pgfqpoint{1.831089in}{2.311547in}}%
\pgfpathcurveto{\pgfqpoint{1.831089in}{2.300497in}}{\pgfqpoint{1.835479in}{2.289898in}}{\pgfqpoint{1.843293in}{2.282084in}}%
\pgfpathcurveto{\pgfqpoint{1.851107in}{2.274271in}}{\pgfqpoint{1.861706in}{2.269880in}}{\pgfqpoint{1.872756in}{2.269880in}}%
\pgfpathclose%
\pgfusepath{stroke,fill}%
\end{pgfscope}%
\begin{pgfscope}%
\pgfpathrectangle{\pgfqpoint{0.600000in}{0.600000in}}{\pgfqpoint{3.900000in}{3.900000in}}%
\pgfusepath{clip}%
\pgfsetbuttcap%
\pgfsetroundjoin%
\definecolor{currentfill}{rgb}{0.121569,0.466667,0.705882}%
\pgfsetfillcolor{currentfill}%
\pgfsetlinewidth{1.003750pt}%
\definecolor{currentstroke}{rgb}{0.121569,0.466667,0.705882}%
\pgfsetstrokecolor{currentstroke}%
\pgfsetdash{}{0pt}%
\pgfpathmoveto{\pgfqpoint{2.292747in}{2.011237in}}%
\pgfpathcurveto{\pgfqpoint{2.303797in}{2.011237in}}{\pgfqpoint{2.314397in}{2.015628in}}{\pgfqpoint{2.322210in}{2.023441in}}%
\pgfpathcurveto{\pgfqpoint{2.330024in}{2.031255in}}{\pgfqpoint{2.334414in}{2.041854in}}{\pgfqpoint{2.334414in}{2.052904in}}%
\pgfpathcurveto{\pgfqpoint{2.334414in}{2.063954in}}{\pgfqpoint{2.330024in}{2.074553in}}{\pgfqpoint{2.322210in}{2.082367in}}%
\pgfpathcurveto{\pgfqpoint{2.314397in}{2.090180in}}{\pgfqpoint{2.303797in}{2.094571in}}{\pgfqpoint{2.292747in}{2.094571in}}%
\pgfpathcurveto{\pgfqpoint{2.281697in}{2.094571in}}{\pgfqpoint{2.271098in}{2.090180in}}{\pgfqpoint{2.263285in}{2.082367in}}%
\pgfpathcurveto{\pgfqpoint{2.255471in}{2.074553in}}{\pgfqpoint{2.251081in}{2.063954in}}{\pgfqpoint{2.251081in}{2.052904in}}%
\pgfpathcurveto{\pgfqpoint{2.251081in}{2.041854in}}{\pgfqpoint{2.255471in}{2.031255in}}{\pgfqpoint{2.263285in}{2.023441in}}%
\pgfpathcurveto{\pgfqpoint{2.271098in}{2.015628in}}{\pgfqpoint{2.281697in}{2.011237in}}{\pgfqpoint{2.292747in}{2.011237in}}%
\pgfpathclose%
\pgfusepath{stroke,fill}%
\end{pgfscope}%
\begin{pgfscope}%
\pgfpathrectangle{\pgfqpoint{0.600000in}{0.600000in}}{\pgfqpoint{3.900000in}{3.900000in}}%
\pgfusepath{clip}%
\pgfsetbuttcap%
\pgfsetroundjoin%
\definecolor{currentfill}{rgb}{0.121569,0.466667,0.705882}%
\pgfsetfillcolor{currentfill}%
\pgfsetlinewidth{1.003750pt}%
\definecolor{currentstroke}{rgb}{0.121569,0.466667,0.705882}%
\pgfsetstrokecolor{currentstroke}%
\pgfsetdash{}{0pt}%
\pgfpathmoveto{\pgfqpoint{1.972510in}{2.209911in}}%
\pgfpathcurveto{\pgfqpoint{1.983560in}{2.209911in}}{\pgfqpoint{1.994159in}{2.214301in}}{\pgfqpoint{2.001972in}{2.222114in}}%
\pgfpathcurveto{\pgfqpoint{2.009786in}{2.229928in}}{\pgfqpoint{2.014176in}{2.240527in}}{\pgfqpoint{2.014176in}{2.251577in}}%
\pgfpathcurveto{\pgfqpoint{2.014176in}{2.262627in}}{\pgfqpoint{2.009786in}{2.273226in}}{\pgfqpoint{2.001972in}{2.281040in}}%
\pgfpathcurveto{\pgfqpoint{1.994159in}{2.288854in}}{\pgfqpoint{1.983560in}{2.293244in}}{\pgfqpoint{1.972510in}{2.293244in}}%
\pgfpathcurveto{\pgfqpoint{1.961460in}{2.293244in}}{\pgfqpoint{1.950860in}{2.288854in}}{\pgfqpoint{1.943047in}{2.281040in}}%
\pgfpathcurveto{\pgfqpoint{1.935233in}{2.273226in}}{\pgfqpoint{1.930843in}{2.262627in}}{\pgfqpoint{1.930843in}{2.251577in}}%
\pgfpathcurveto{\pgfqpoint{1.930843in}{2.240527in}}{\pgfqpoint{1.935233in}{2.229928in}}{\pgfqpoint{1.943047in}{2.222114in}}%
\pgfpathcurveto{\pgfqpoint{1.950860in}{2.214301in}}{\pgfqpoint{1.961460in}{2.209911in}}{\pgfqpoint{1.972510in}{2.209911in}}%
\pgfpathclose%
\pgfusepath{stroke,fill}%
\end{pgfscope}%
\begin{pgfscope}%
\pgfpathrectangle{\pgfqpoint{0.600000in}{0.600000in}}{\pgfqpoint{3.900000in}{3.900000in}}%
\pgfusepath{clip}%
\pgfsetbuttcap%
\pgfsetroundjoin%
\definecolor{currentfill}{rgb}{0.121569,0.466667,0.705882}%
\pgfsetfillcolor{currentfill}%
\pgfsetlinewidth{1.003750pt}%
\definecolor{currentstroke}{rgb}{0.121569,0.466667,0.705882}%
\pgfsetstrokecolor{currentstroke}%
\pgfsetdash{}{0pt}%
\pgfpathmoveto{\pgfqpoint{3.267421in}{2.373436in}}%
\pgfpathcurveto{\pgfqpoint{3.278471in}{2.373436in}}{\pgfqpoint{3.289070in}{2.377826in}}{\pgfqpoint{3.296884in}{2.385640in}}%
\pgfpathcurveto{\pgfqpoint{3.304697in}{2.393454in}}{\pgfqpoint{3.309087in}{2.404053in}}{\pgfqpoint{3.309087in}{2.415103in}}%
\pgfpathcurveto{\pgfqpoint{3.309087in}{2.426153in}}{\pgfqpoint{3.304697in}{2.436752in}}{\pgfqpoint{3.296884in}{2.444566in}}%
\pgfpathcurveto{\pgfqpoint{3.289070in}{2.452379in}}{\pgfqpoint{3.278471in}{2.456769in}}{\pgfqpoint{3.267421in}{2.456769in}}%
\pgfpathcurveto{\pgfqpoint{3.256371in}{2.456769in}}{\pgfqpoint{3.245772in}{2.452379in}}{\pgfqpoint{3.237958in}{2.444566in}}%
\pgfpathcurveto{\pgfqpoint{3.230144in}{2.436752in}}{\pgfqpoint{3.225754in}{2.426153in}}{\pgfqpoint{3.225754in}{2.415103in}}%
\pgfpathcurveto{\pgfqpoint{3.225754in}{2.404053in}}{\pgfqpoint{3.230144in}{2.393454in}}{\pgfqpoint{3.237958in}{2.385640in}}%
\pgfpathcurveto{\pgfqpoint{3.245772in}{2.377826in}}{\pgfqpoint{3.256371in}{2.373436in}}{\pgfqpoint{3.267421in}{2.373436in}}%
\pgfpathclose%
\pgfusepath{stroke,fill}%
\end{pgfscope}%
\begin{pgfscope}%
\pgfpathrectangle{\pgfqpoint{0.600000in}{0.600000in}}{\pgfqpoint{3.900000in}{3.900000in}}%
\pgfusepath{clip}%
\pgfsetbuttcap%
\pgfsetroundjoin%
\definecolor{currentfill}{rgb}{0.121569,0.466667,0.705882}%
\pgfsetfillcolor{currentfill}%
\pgfsetlinewidth{1.003750pt}%
\definecolor{currentstroke}{rgb}{0.121569,0.466667,0.705882}%
\pgfsetstrokecolor{currentstroke}%
\pgfsetdash{}{0pt}%
\pgfpathmoveto{\pgfqpoint{2.913820in}{2.123636in}}%
\pgfpathcurveto{\pgfqpoint{2.924870in}{2.123636in}}{\pgfqpoint{2.935469in}{2.128027in}}{\pgfqpoint{2.943283in}{2.135840in}}%
\pgfpathcurveto{\pgfqpoint{2.951097in}{2.143654in}}{\pgfqpoint{2.955487in}{2.154253in}}{\pgfqpoint{2.955487in}{2.165303in}}%
\pgfpathcurveto{\pgfqpoint{2.955487in}{2.176353in}}{\pgfqpoint{2.951097in}{2.186952in}}{\pgfqpoint{2.943283in}{2.194766in}}%
\pgfpathcurveto{\pgfqpoint{2.935469in}{2.202579in}}{\pgfqpoint{2.924870in}{2.206970in}}{\pgfqpoint{2.913820in}{2.206970in}}%
\pgfpathcurveto{\pgfqpoint{2.902770in}{2.206970in}}{\pgfqpoint{2.892171in}{2.202579in}}{\pgfqpoint{2.884357in}{2.194766in}}%
\pgfpathcurveto{\pgfqpoint{2.876544in}{2.186952in}}{\pgfqpoint{2.872154in}{2.176353in}}{\pgfqpoint{2.872154in}{2.165303in}}%
\pgfpathcurveto{\pgfqpoint{2.872154in}{2.154253in}}{\pgfqpoint{2.876544in}{2.143654in}}{\pgfqpoint{2.884357in}{2.135840in}}%
\pgfpathcurveto{\pgfqpoint{2.892171in}{2.128027in}}{\pgfqpoint{2.902770in}{2.123636in}}{\pgfqpoint{2.913820in}{2.123636in}}%
\pgfpathclose%
\pgfusepath{stroke,fill}%
\end{pgfscope}%
\begin{pgfscope}%
\pgfpathrectangle{\pgfqpoint{0.600000in}{0.600000in}}{\pgfqpoint{3.900000in}{3.900000in}}%
\pgfusepath{clip}%
\pgfsetbuttcap%
\pgfsetroundjoin%
\definecolor{currentfill}{rgb}{0.121569,0.466667,0.705882}%
\pgfsetfillcolor{currentfill}%
\pgfsetlinewidth{1.003750pt}%
\definecolor{currentstroke}{rgb}{0.121569,0.466667,0.705882}%
\pgfsetstrokecolor{currentstroke}%
\pgfsetdash{}{0pt}%
\pgfpathmoveto{\pgfqpoint{3.903018in}{2.846604in}}%
\pgfpathcurveto{\pgfqpoint{3.914068in}{2.846604in}}{\pgfqpoint{3.924667in}{2.850994in}}{\pgfqpoint{3.932481in}{2.858807in}}%
\pgfpathcurveto{\pgfqpoint{3.940294in}{2.866621in}}{\pgfqpoint{3.944685in}{2.877220in}}{\pgfqpoint{3.944685in}{2.888270in}}%
\pgfpathcurveto{\pgfqpoint{3.944685in}{2.899320in}}{\pgfqpoint{3.940294in}{2.909919in}}{\pgfqpoint{3.932481in}{2.917733in}}%
\pgfpathcurveto{\pgfqpoint{3.924667in}{2.925547in}}{\pgfqpoint{3.914068in}{2.929937in}}{\pgfqpoint{3.903018in}{2.929937in}}%
\pgfpathcurveto{\pgfqpoint{3.891968in}{2.929937in}}{\pgfqpoint{3.881369in}{2.925547in}}{\pgfqpoint{3.873555in}{2.917733in}}%
\pgfpathcurveto{\pgfqpoint{3.865742in}{2.909919in}}{\pgfqpoint{3.861351in}{2.899320in}}{\pgfqpoint{3.861351in}{2.888270in}}%
\pgfpathcurveto{\pgfqpoint{3.861351in}{2.877220in}}{\pgfqpoint{3.865742in}{2.866621in}}{\pgfqpoint{3.873555in}{2.858807in}}%
\pgfpathcurveto{\pgfqpoint{3.881369in}{2.850994in}}{\pgfqpoint{3.891968in}{2.846604in}}{\pgfqpoint{3.903018in}{2.846604in}}%
\pgfpathclose%
\pgfusepath{stroke,fill}%
\end{pgfscope}%
\begin{pgfscope}%
\pgfpathrectangle{\pgfqpoint{0.600000in}{0.600000in}}{\pgfqpoint{3.900000in}{3.900000in}}%
\pgfusepath{clip}%
\pgfsetbuttcap%
\pgfsetroundjoin%
\definecolor{currentfill}{rgb}{0.121569,0.466667,0.705882}%
\pgfsetfillcolor{currentfill}%
\pgfsetlinewidth{1.003750pt}%
\definecolor{currentstroke}{rgb}{0.121569,0.466667,0.705882}%
\pgfsetstrokecolor{currentstroke}%
\pgfsetdash{}{0pt}%
\pgfpathmoveto{\pgfqpoint{2.657121in}{2.883975in}}%
\pgfpathcurveto{\pgfqpoint{2.668171in}{2.883975in}}{\pgfqpoint{2.678770in}{2.888365in}}{\pgfqpoint{2.686584in}{2.896179in}}%
\pgfpathcurveto{\pgfqpoint{2.694397in}{2.903992in}}{\pgfqpoint{2.698788in}{2.914591in}}{\pgfqpoint{2.698788in}{2.925641in}}%
\pgfpathcurveto{\pgfqpoint{2.698788in}{2.936691in}}{\pgfqpoint{2.694397in}{2.947290in}}{\pgfqpoint{2.686584in}{2.955104in}}%
\pgfpathcurveto{\pgfqpoint{2.678770in}{2.962918in}}{\pgfqpoint{2.668171in}{2.967308in}}{\pgfqpoint{2.657121in}{2.967308in}}%
\pgfpathcurveto{\pgfqpoint{2.646071in}{2.967308in}}{\pgfqpoint{2.635472in}{2.962918in}}{\pgfqpoint{2.627658in}{2.955104in}}%
\pgfpathcurveto{\pgfqpoint{2.619845in}{2.947290in}}{\pgfqpoint{2.615454in}{2.936691in}}{\pgfqpoint{2.615454in}{2.925641in}}%
\pgfpathcurveto{\pgfqpoint{2.615454in}{2.914591in}}{\pgfqpoint{2.619845in}{2.903992in}}{\pgfqpoint{2.627658in}{2.896179in}}%
\pgfpathcurveto{\pgfqpoint{2.635472in}{2.888365in}}{\pgfqpoint{2.646071in}{2.883975in}}{\pgfqpoint{2.657121in}{2.883975in}}%
\pgfpathclose%
\pgfusepath{stroke,fill}%
\end{pgfscope}%
\begin{pgfscope}%
\pgfpathrectangle{\pgfqpoint{0.600000in}{0.600000in}}{\pgfqpoint{3.900000in}{3.900000in}}%
\pgfusepath{clip}%
\pgfsetbuttcap%
\pgfsetroundjoin%
\definecolor{currentfill}{rgb}{0.121569,0.466667,0.705882}%
\pgfsetfillcolor{currentfill}%
\pgfsetlinewidth{1.003750pt}%
\definecolor{currentstroke}{rgb}{0.121569,0.466667,0.705882}%
\pgfsetstrokecolor{currentstroke}%
\pgfsetdash{}{0pt}%
\pgfpathmoveto{\pgfqpoint{2.104337in}{3.094106in}}%
\pgfpathcurveto{\pgfqpoint{2.115387in}{3.094106in}}{\pgfqpoint{2.125986in}{3.098496in}}{\pgfqpoint{2.133800in}{3.106310in}}%
\pgfpathcurveto{\pgfqpoint{2.141614in}{3.114123in}}{\pgfqpoint{2.146004in}{3.124722in}}{\pgfqpoint{2.146004in}{3.135772in}}%
\pgfpathcurveto{\pgfqpoint{2.146004in}{3.146823in}}{\pgfqpoint{2.141614in}{3.157422in}}{\pgfqpoint{2.133800in}{3.165235in}}%
\pgfpathcurveto{\pgfqpoint{2.125986in}{3.173049in}}{\pgfqpoint{2.115387in}{3.177439in}}{\pgfqpoint{2.104337in}{3.177439in}}%
\pgfpathcurveto{\pgfqpoint{2.093287in}{3.177439in}}{\pgfqpoint{2.082688in}{3.173049in}}{\pgfqpoint{2.074874in}{3.165235in}}%
\pgfpathcurveto{\pgfqpoint{2.067061in}{3.157422in}}{\pgfqpoint{2.062671in}{3.146823in}}{\pgfqpoint{2.062671in}{3.135772in}}%
\pgfpathcurveto{\pgfqpoint{2.062671in}{3.124722in}}{\pgfqpoint{2.067061in}{3.114123in}}{\pgfqpoint{2.074874in}{3.106310in}}%
\pgfpathcurveto{\pgfqpoint{2.082688in}{3.098496in}}{\pgfqpoint{2.093287in}{3.094106in}}{\pgfqpoint{2.104337in}{3.094106in}}%
\pgfpathclose%
\pgfusepath{stroke,fill}%
\end{pgfscope}%
\begin{pgfscope}%
\pgfpathrectangle{\pgfqpoint{0.600000in}{0.600000in}}{\pgfqpoint{3.900000in}{3.900000in}}%
\pgfusepath{clip}%
\pgfsetbuttcap%
\pgfsetroundjoin%
\definecolor{currentfill}{rgb}{0.121569,0.466667,0.705882}%
\pgfsetfillcolor{currentfill}%
\pgfsetlinewidth{1.003750pt}%
\definecolor{currentstroke}{rgb}{0.121569,0.466667,0.705882}%
\pgfsetstrokecolor{currentstroke}%
\pgfsetdash{}{0pt}%
\pgfpathmoveto{\pgfqpoint{2.884042in}{2.170596in}}%
\pgfpathcurveto{\pgfqpoint{2.895093in}{2.170596in}}{\pgfqpoint{2.905692in}{2.174986in}}{\pgfqpoint{2.913505in}{2.182800in}}%
\pgfpathcurveto{\pgfqpoint{2.921319in}{2.190613in}}{\pgfqpoint{2.925709in}{2.201212in}}{\pgfqpoint{2.925709in}{2.212263in}}%
\pgfpathcurveto{\pgfqpoint{2.925709in}{2.223313in}}{\pgfqpoint{2.921319in}{2.233912in}}{\pgfqpoint{2.913505in}{2.241725in}}%
\pgfpathcurveto{\pgfqpoint{2.905692in}{2.249539in}}{\pgfqpoint{2.895093in}{2.253929in}}{\pgfqpoint{2.884042in}{2.253929in}}%
\pgfpathcurveto{\pgfqpoint{2.872992in}{2.253929in}}{\pgfqpoint{2.862393in}{2.249539in}}{\pgfqpoint{2.854580in}{2.241725in}}%
\pgfpathcurveto{\pgfqpoint{2.846766in}{2.233912in}}{\pgfqpoint{2.842376in}{2.223313in}}{\pgfqpoint{2.842376in}{2.212263in}}%
\pgfpathcurveto{\pgfqpoint{2.842376in}{2.201212in}}{\pgfqpoint{2.846766in}{2.190613in}}{\pgfqpoint{2.854580in}{2.182800in}}%
\pgfpathcurveto{\pgfqpoint{2.862393in}{2.174986in}}{\pgfqpoint{2.872992in}{2.170596in}}{\pgfqpoint{2.884042in}{2.170596in}}%
\pgfpathclose%
\pgfusepath{stroke,fill}%
\end{pgfscope}%
\begin{pgfscope}%
\pgfpathrectangle{\pgfqpoint{0.600000in}{0.600000in}}{\pgfqpoint{3.900000in}{3.900000in}}%
\pgfusepath{clip}%
\pgfsetbuttcap%
\pgfsetroundjoin%
\definecolor{currentfill}{rgb}{0.121569,0.466667,0.705882}%
\pgfsetfillcolor{currentfill}%
\pgfsetlinewidth{1.003750pt}%
\definecolor{currentstroke}{rgb}{0.121569,0.466667,0.705882}%
\pgfsetstrokecolor{currentstroke}%
\pgfsetdash{}{0pt}%
\pgfpathmoveto{\pgfqpoint{3.443829in}{2.434008in}}%
\pgfpathcurveto{\pgfqpoint{3.454879in}{2.434008in}}{\pgfqpoint{3.465478in}{2.438399in}}{\pgfqpoint{3.473292in}{2.446212in}}%
\pgfpathcurveto{\pgfqpoint{3.481105in}{2.454026in}}{\pgfqpoint{3.485496in}{2.464625in}}{\pgfqpoint{3.485496in}{2.475675in}}%
\pgfpathcurveto{\pgfqpoint{3.485496in}{2.486725in}}{\pgfqpoint{3.481105in}{2.497324in}}{\pgfqpoint{3.473292in}{2.505138in}}%
\pgfpathcurveto{\pgfqpoint{3.465478in}{2.512951in}}{\pgfqpoint{3.454879in}{2.517342in}}{\pgfqpoint{3.443829in}{2.517342in}}%
\pgfpathcurveto{\pgfqpoint{3.432779in}{2.517342in}}{\pgfqpoint{3.422180in}{2.512951in}}{\pgfqpoint{3.414366in}{2.505138in}}%
\pgfpathcurveto{\pgfqpoint{3.406553in}{2.497324in}}{\pgfqpoint{3.402162in}{2.486725in}}{\pgfqpoint{3.402162in}{2.475675in}}%
\pgfpathcurveto{\pgfqpoint{3.402162in}{2.464625in}}{\pgfqpoint{3.406553in}{2.454026in}}{\pgfqpoint{3.414366in}{2.446212in}}%
\pgfpathcurveto{\pgfqpoint{3.422180in}{2.438399in}}{\pgfqpoint{3.432779in}{2.434008in}}{\pgfqpoint{3.443829in}{2.434008in}}%
\pgfpathclose%
\pgfusepath{stroke,fill}%
\end{pgfscope}%
\begin{pgfscope}%
\pgfpathrectangle{\pgfqpoint{0.600000in}{0.600000in}}{\pgfqpoint{3.900000in}{3.900000in}}%
\pgfusepath{clip}%
\pgfsetbuttcap%
\pgfsetroundjoin%
\definecolor{currentfill}{rgb}{0.121569,0.466667,0.705882}%
\pgfsetfillcolor{currentfill}%
\pgfsetlinewidth{1.003750pt}%
\definecolor{currentstroke}{rgb}{0.121569,0.466667,0.705882}%
\pgfsetstrokecolor{currentstroke}%
\pgfsetdash{}{0pt}%
\pgfpathmoveto{\pgfqpoint{2.884498in}{2.587131in}}%
\pgfpathcurveto{\pgfqpoint{2.895548in}{2.587131in}}{\pgfqpoint{2.906147in}{2.591521in}}{\pgfqpoint{2.913960in}{2.599335in}}%
\pgfpathcurveto{\pgfqpoint{2.921774in}{2.607148in}}{\pgfqpoint{2.926164in}{2.617747in}}{\pgfqpoint{2.926164in}{2.628797in}}%
\pgfpathcurveto{\pgfqpoint{2.926164in}{2.639848in}}{\pgfqpoint{2.921774in}{2.650447in}}{\pgfqpoint{2.913960in}{2.658260in}}%
\pgfpathcurveto{\pgfqpoint{2.906147in}{2.666074in}}{\pgfqpoint{2.895548in}{2.670464in}}{\pgfqpoint{2.884498in}{2.670464in}}%
\pgfpathcurveto{\pgfqpoint{2.873448in}{2.670464in}}{\pgfqpoint{2.862849in}{2.666074in}}{\pgfqpoint{2.855035in}{2.658260in}}%
\pgfpathcurveto{\pgfqpoint{2.847221in}{2.650447in}}{\pgfqpoint{2.842831in}{2.639848in}}{\pgfqpoint{2.842831in}{2.628797in}}%
\pgfpathcurveto{\pgfqpoint{2.842831in}{2.617747in}}{\pgfqpoint{2.847221in}{2.607148in}}{\pgfqpoint{2.855035in}{2.599335in}}%
\pgfpathcurveto{\pgfqpoint{2.862849in}{2.591521in}}{\pgfqpoint{2.873448in}{2.587131in}}{\pgfqpoint{2.884498in}{2.587131in}}%
\pgfpathclose%
\pgfusepath{stroke,fill}%
\end{pgfscope}%
\begin{pgfscope}%
\pgfpathrectangle{\pgfqpoint{0.600000in}{0.600000in}}{\pgfqpoint{3.900000in}{3.900000in}}%
\pgfusepath{clip}%
\pgfsetbuttcap%
\pgfsetroundjoin%
\definecolor{currentfill}{rgb}{0.121569,0.466667,0.705882}%
\pgfsetfillcolor{currentfill}%
\pgfsetlinewidth{1.003750pt}%
\definecolor{currentstroke}{rgb}{0.121569,0.466667,0.705882}%
\pgfsetstrokecolor{currentstroke}%
\pgfsetdash{}{0pt}%
\pgfpathmoveto{\pgfqpoint{3.264318in}{2.500175in}}%
\pgfpathcurveto{\pgfqpoint{3.275368in}{2.500175in}}{\pgfqpoint{3.285967in}{2.504565in}}{\pgfqpoint{3.293781in}{2.512379in}}%
\pgfpathcurveto{\pgfqpoint{3.301594in}{2.520192in}}{\pgfqpoint{3.305985in}{2.530791in}}{\pgfqpoint{3.305985in}{2.541841in}}%
\pgfpathcurveto{\pgfqpoint{3.305985in}{2.552891in}}{\pgfqpoint{3.301594in}{2.563491in}}{\pgfqpoint{3.293781in}{2.571304in}}%
\pgfpathcurveto{\pgfqpoint{3.285967in}{2.579118in}}{\pgfqpoint{3.275368in}{2.583508in}}{\pgfqpoint{3.264318in}{2.583508in}}%
\pgfpathcurveto{\pgfqpoint{3.253268in}{2.583508in}}{\pgfqpoint{3.242669in}{2.579118in}}{\pgfqpoint{3.234855in}{2.571304in}}%
\pgfpathcurveto{\pgfqpoint{3.227042in}{2.563491in}}{\pgfqpoint{3.222651in}{2.552891in}}{\pgfqpoint{3.222651in}{2.541841in}}%
\pgfpathcurveto{\pgfqpoint{3.222651in}{2.530791in}}{\pgfqpoint{3.227042in}{2.520192in}}{\pgfqpoint{3.234855in}{2.512379in}}%
\pgfpathcurveto{\pgfqpoint{3.242669in}{2.504565in}}{\pgfqpoint{3.253268in}{2.500175in}}{\pgfqpoint{3.264318in}{2.500175in}}%
\pgfpathclose%
\pgfusepath{stroke,fill}%
\end{pgfscope}%
\begin{pgfscope}%
\pgfpathrectangle{\pgfqpoint{0.600000in}{0.600000in}}{\pgfqpoint{3.900000in}{3.900000in}}%
\pgfusepath{clip}%
\pgfsetbuttcap%
\pgfsetroundjoin%
\definecolor{currentfill}{rgb}{0.121569,0.466667,0.705882}%
\pgfsetfillcolor{currentfill}%
\pgfsetlinewidth{1.003750pt}%
\definecolor{currentstroke}{rgb}{0.121569,0.466667,0.705882}%
\pgfsetstrokecolor{currentstroke}%
\pgfsetdash{}{0pt}%
\pgfpathmoveto{\pgfqpoint{2.011985in}{2.179498in}}%
\pgfpathcurveto{\pgfqpoint{2.023035in}{2.179498in}}{\pgfqpoint{2.033634in}{2.183889in}}{\pgfqpoint{2.041447in}{2.191702in}}%
\pgfpathcurveto{\pgfqpoint{2.049261in}{2.199516in}}{\pgfqpoint{2.053651in}{2.210115in}}{\pgfqpoint{2.053651in}{2.221165in}}%
\pgfpathcurveto{\pgfqpoint{2.053651in}{2.232215in}}{\pgfqpoint{2.049261in}{2.242814in}}{\pgfqpoint{2.041447in}{2.250628in}}%
\pgfpathcurveto{\pgfqpoint{2.033634in}{2.258442in}}{\pgfqpoint{2.023035in}{2.262832in}}{\pgfqpoint{2.011985in}{2.262832in}}%
\pgfpathcurveto{\pgfqpoint{2.000935in}{2.262832in}}{\pgfqpoint{1.990335in}{2.258442in}}{\pgfqpoint{1.982522in}{2.250628in}}%
\pgfpathcurveto{\pgfqpoint{1.974708in}{2.242814in}}{\pgfqpoint{1.970318in}{2.232215in}}{\pgfqpoint{1.970318in}{2.221165in}}%
\pgfpathcurveto{\pgfqpoint{1.970318in}{2.210115in}}{\pgfqpoint{1.974708in}{2.199516in}}{\pgfqpoint{1.982522in}{2.191702in}}%
\pgfpathcurveto{\pgfqpoint{1.990335in}{2.183889in}}{\pgfqpoint{2.000935in}{2.179498in}}{\pgfqpoint{2.011985in}{2.179498in}}%
\pgfpathclose%
\pgfusepath{stroke,fill}%
\end{pgfscope}%
\begin{pgfscope}%
\pgfpathrectangle{\pgfqpoint{0.600000in}{0.600000in}}{\pgfqpoint{3.900000in}{3.900000in}}%
\pgfusepath{clip}%
\pgfsetbuttcap%
\pgfsetroundjoin%
\definecolor{currentfill}{rgb}{0.121569,0.466667,0.705882}%
\pgfsetfillcolor{currentfill}%
\pgfsetlinewidth{1.003750pt}%
\definecolor{currentstroke}{rgb}{0.121569,0.466667,0.705882}%
\pgfsetstrokecolor{currentstroke}%
\pgfsetdash{}{0pt}%
\pgfpathmoveto{\pgfqpoint{4.011534in}{2.419578in}}%
\pgfpathcurveto{\pgfqpoint{4.022584in}{2.419578in}}{\pgfqpoint{4.033183in}{2.423968in}}{\pgfqpoint{4.040997in}{2.431782in}}%
\pgfpathcurveto{\pgfqpoint{4.048811in}{2.439596in}}{\pgfqpoint{4.053201in}{2.450195in}}{\pgfqpoint{4.053201in}{2.461245in}}%
\pgfpathcurveto{\pgfqpoint{4.053201in}{2.472295in}}{\pgfqpoint{4.048811in}{2.482894in}}{\pgfqpoint{4.040997in}{2.490708in}}%
\pgfpathcurveto{\pgfqpoint{4.033183in}{2.498521in}}{\pgfqpoint{4.022584in}{2.502911in}}{\pgfqpoint{4.011534in}{2.502911in}}%
\pgfpathcurveto{\pgfqpoint{4.000484in}{2.502911in}}{\pgfqpoint{3.989885in}{2.498521in}}{\pgfqpoint{3.982072in}{2.490708in}}%
\pgfpathcurveto{\pgfqpoint{3.974258in}{2.482894in}}{\pgfqpoint{3.969868in}{2.472295in}}{\pgfqpoint{3.969868in}{2.461245in}}%
\pgfpathcurveto{\pgfqpoint{3.969868in}{2.450195in}}{\pgfqpoint{3.974258in}{2.439596in}}{\pgfqpoint{3.982072in}{2.431782in}}%
\pgfpathcurveto{\pgfqpoint{3.989885in}{2.423968in}}{\pgfqpoint{4.000484in}{2.419578in}}{\pgfqpoint{4.011534in}{2.419578in}}%
\pgfpathclose%
\pgfusepath{stroke,fill}%
\end{pgfscope}%
\begin{pgfscope}%
\pgfpathrectangle{\pgfqpoint{0.600000in}{0.600000in}}{\pgfqpoint{3.900000in}{3.900000in}}%
\pgfusepath{clip}%
\pgfsetbuttcap%
\pgfsetroundjoin%
\definecolor{currentfill}{rgb}{0.121569,0.466667,0.705882}%
\pgfsetfillcolor{currentfill}%
\pgfsetlinewidth{1.003750pt}%
\definecolor{currentstroke}{rgb}{0.121569,0.466667,0.705882}%
\pgfsetstrokecolor{currentstroke}%
\pgfsetdash{}{0pt}%
\pgfpathmoveto{\pgfqpoint{3.497859in}{2.815582in}}%
\pgfpathcurveto{\pgfqpoint{3.508909in}{2.815582in}}{\pgfqpoint{3.519508in}{2.819972in}}{\pgfqpoint{3.527322in}{2.827786in}}%
\pgfpathcurveto{\pgfqpoint{3.535135in}{2.835599in}}{\pgfqpoint{3.539526in}{2.846198in}}{\pgfqpoint{3.539526in}{2.857248in}}%
\pgfpathcurveto{\pgfqpoint{3.539526in}{2.868299in}}{\pgfqpoint{3.535135in}{2.878898in}}{\pgfqpoint{3.527322in}{2.886711in}}%
\pgfpathcurveto{\pgfqpoint{3.519508in}{2.894525in}}{\pgfqpoint{3.508909in}{2.898915in}}{\pgfqpoint{3.497859in}{2.898915in}}%
\pgfpathcurveto{\pgfqpoint{3.486809in}{2.898915in}}{\pgfqpoint{3.476210in}{2.894525in}}{\pgfqpoint{3.468396in}{2.886711in}}%
\pgfpathcurveto{\pgfqpoint{3.460583in}{2.878898in}}{\pgfqpoint{3.456192in}{2.868299in}}{\pgfqpoint{3.456192in}{2.857248in}}%
\pgfpathcurveto{\pgfqpoint{3.456192in}{2.846198in}}{\pgfqpoint{3.460583in}{2.835599in}}{\pgfqpoint{3.468396in}{2.827786in}}%
\pgfpathcurveto{\pgfqpoint{3.476210in}{2.819972in}}{\pgfqpoint{3.486809in}{2.815582in}}{\pgfqpoint{3.497859in}{2.815582in}}%
\pgfpathclose%
\pgfusepath{stroke,fill}%
\end{pgfscope}%
\begin{pgfscope}%
\pgfpathrectangle{\pgfqpoint{0.600000in}{0.600000in}}{\pgfqpoint{3.900000in}{3.900000in}}%
\pgfusepath{clip}%
\pgfsetbuttcap%
\pgfsetroundjoin%
\definecolor{currentfill}{rgb}{0.121569,0.466667,0.705882}%
\pgfsetfillcolor{currentfill}%
\pgfsetlinewidth{1.003750pt}%
\definecolor{currentstroke}{rgb}{0.121569,0.466667,0.705882}%
\pgfsetstrokecolor{currentstroke}%
\pgfsetdash{}{0pt}%
\pgfpathmoveto{\pgfqpoint{3.555012in}{2.477728in}}%
\pgfpathcurveto{\pgfqpoint{3.566062in}{2.477728in}}{\pgfqpoint{3.576661in}{2.482118in}}{\pgfqpoint{3.584475in}{2.489931in}}%
\pgfpathcurveto{\pgfqpoint{3.592288in}{2.497745in}}{\pgfqpoint{3.596679in}{2.508344in}}{\pgfqpoint{3.596679in}{2.519394in}}%
\pgfpathcurveto{\pgfqpoint{3.596679in}{2.530444in}}{\pgfqpoint{3.592288in}{2.541043in}}{\pgfqpoint{3.584475in}{2.548857in}}%
\pgfpathcurveto{\pgfqpoint{3.576661in}{2.556671in}}{\pgfqpoint{3.566062in}{2.561061in}}{\pgfqpoint{3.555012in}{2.561061in}}%
\pgfpathcurveto{\pgfqpoint{3.543962in}{2.561061in}}{\pgfqpoint{3.533363in}{2.556671in}}{\pgfqpoint{3.525549in}{2.548857in}}%
\pgfpathcurveto{\pgfqpoint{3.517735in}{2.541043in}}{\pgfqpoint{3.513345in}{2.530444in}}{\pgfqpoint{3.513345in}{2.519394in}}%
\pgfpathcurveto{\pgfqpoint{3.513345in}{2.508344in}}{\pgfqpoint{3.517735in}{2.497745in}}{\pgfqpoint{3.525549in}{2.489931in}}%
\pgfpathcurveto{\pgfqpoint{3.533363in}{2.482118in}}{\pgfqpoint{3.543962in}{2.477728in}}{\pgfqpoint{3.555012in}{2.477728in}}%
\pgfpathclose%
\pgfusepath{stroke,fill}%
\end{pgfscope}%
\begin{pgfscope}%
\pgfpathrectangle{\pgfqpoint{0.600000in}{0.600000in}}{\pgfqpoint{3.900000in}{3.900000in}}%
\pgfusepath{clip}%
\pgfsetbuttcap%
\pgfsetroundjoin%
\definecolor{currentfill}{rgb}{0.121569,0.466667,0.705882}%
\pgfsetfillcolor{currentfill}%
\pgfsetlinewidth{1.003750pt}%
\definecolor{currentstroke}{rgb}{0.121569,0.466667,0.705882}%
\pgfsetstrokecolor{currentstroke}%
\pgfsetdash{}{0pt}%
\pgfpathmoveto{\pgfqpoint{2.307911in}{1.630388in}}%
\pgfpathcurveto{\pgfqpoint{2.318961in}{1.630388in}}{\pgfqpoint{2.329560in}{1.634778in}}{\pgfqpoint{2.337373in}{1.642592in}}%
\pgfpathcurveto{\pgfqpoint{2.345187in}{1.650406in}}{\pgfqpoint{2.349577in}{1.661005in}}{\pgfqpoint{2.349577in}{1.672055in}}%
\pgfpathcurveto{\pgfqpoint{2.349577in}{1.683105in}}{\pgfqpoint{2.345187in}{1.693704in}}{\pgfqpoint{2.337373in}{1.701518in}}%
\pgfpathcurveto{\pgfqpoint{2.329560in}{1.709331in}}{\pgfqpoint{2.318961in}{1.713722in}}{\pgfqpoint{2.307911in}{1.713722in}}%
\pgfpathcurveto{\pgfqpoint{2.296860in}{1.713722in}}{\pgfqpoint{2.286261in}{1.709331in}}{\pgfqpoint{2.278448in}{1.701518in}}%
\pgfpathcurveto{\pgfqpoint{2.270634in}{1.693704in}}{\pgfqpoint{2.266244in}{1.683105in}}{\pgfqpoint{2.266244in}{1.672055in}}%
\pgfpathcurveto{\pgfqpoint{2.266244in}{1.661005in}}{\pgfqpoint{2.270634in}{1.650406in}}{\pgfqpoint{2.278448in}{1.642592in}}%
\pgfpathcurveto{\pgfqpoint{2.286261in}{1.634778in}}{\pgfqpoint{2.296860in}{1.630388in}}{\pgfqpoint{2.307911in}{1.630388in}}%
\pgfpathclose%
\pgfusepath{stroke,fill}%
\end{pgfscope}%
\begin{pgfscope}%
\pgfpathrectangle{\pgfqpoint{0.600000in}{0.600000in}}{\pgfqpoint{3.900000in}{3.900000in}}%
\pgfusepath{clip}%
\pgfsetbuttcap%
\pgfsetroundjoin%
\definecolor{currentfill}{rgb}{0.121569,0.466667,0.705882}%
\pgfsetfillcolor{currentfill}%
\pgfsetlinewidth{1.003750pt}%
\definecolor{currentstroke}{rgb}{0.121569,0.466667,0.705882}%
\pgfsetstrokecolor{currentstroke}%
\pgfsetdash{}{0pt}%
\pgfpathmoveto{\pgfqpoint{2.318550in}{2.844898in}}%
\pgfpathcurveto{\pgfqpoint{2.329600in}{2.844898in}}{\pgfqpoint{2.340199in}{2.849288in}}{\pgfqpoint{2.348013in}{2.857102in}}%
\pgfpathcurveto{\pgfqpoint{2.355827in}{2.864916in}}{\pgfqpoint{2.360217in}{2.875515in}}{\pgfqpoint{2.360217in}{2.886565in}}%
\pgfpathcurveto{\pgfqpoint{2.360217in}{2.897615in}}{\pgfqpoint{2.355827in}{2.908214in}}{\pgfqpoint{2.348013in}{2.916028in}}%
\pgfpathcurveto{\pgfqpoint{2.340199in}{2.923841in}}{\pgfqpoint{2.329600in}{2.928232in}}{\pgfqpoint{2.318550in}{2.928232in}}%
\pgfpathcurveto{\pgfqpoint{2.307500in}{2.928232in}}{\pgfqpoint{2.296901in}{2.923841in}}{\pgfqpoint{2.289088in}{2.916028in}}%
\pgfpathcurveto{\pgfqpoint{2.281274in}{2.908214in}}{\pgfqpoint{2.276884in}{2.897615in}}{\pgfqpoint{2.276884in}{2.886565in}}%
\pgfpathcurveto{\pgfqpoint{2.276884in}{2.875515in}}{\pgfqpoint{2.281274in}{2.864916in}}{\pgfqpoint{2.289088in}{2.857102in}}%
\pgfpathcurveto{\pgfqpoint{2.296901in}{2.849288in}}{\pgfqpoint{2.307500in}{2.844898in}}{\pgfqpoint{2.318550in}{2.844898in}}%
\pgfpathclose%
\pgfusepath{stroke,fill}%
\end{pgfscope}%
\begin{pgfscope}%
\pgfpathrectangle{\pgfqpoint{0.600000in}{0.600000in}}{\pgfqpoint{3.900000in}{3.900000in}}%
\pgfusepath{clip}%
\pgfsetbuttcap%
\pgfsetroundjoin%
\definecolor{currentfill}{rgb}{0.121569,0.466667,0.705882}%
\pgfsetfillcolor{currentfill}%
\pgfsetlinewidth{1.003750pt}%
\definecolor{currentstroke}{rgb}{0.121569,0.466667,0.705882}%
\pgfsetstrokecolor{currentstroke}%
\pgfsetdash{}{0pt}%
\pgfpathmoveto{\pgfqpoint{2.657158in}{3.669319in}}%
\pgfpathcurveto{\pgfqpoint{2.668208in}{3.669319in}}{\pgfqpoint{2.678807in}{3.673709in}}{\pgfqpoint{2.686621in}{3.681523in}}%
\pgfpathcurveto{\pgfqpoint{2.694435in}{3.689336in}}{\pgfqpoint{2.698825in}{3.699935in}}{\pgfqpoint{2.698825in}{3.710985in}}%
\pgfpathcurveto{\pgfqpoint{2.698825in}{3.722035in}}{\pgfqpoint{2.694435in}{3.732634in}}{\pgfqpoint{2.686621in}{3.740448in}}%
\pgfpathcurveto{\pgfqpoint{2.678807in}{3.748262in}}{\pgfqpoint{2.668208in}{3.752652in}}{\pgfqpoint{2.657158in}{3.752652in}}%
\pgfpathcurveto{\pgfqpoint{2.646108in}{3.752652in}}{\pgfqpoint{2.635509in}{3.748262in}}{\pgfqpoint{2.627695in}{3.740448in}}%
\pgfpathcurveto{\pgfqpoint{2.619882in}{3.732634in}}{\pgfqpoint{2.615491in}{3.722035in}}{\pgfqpoint{2.615491in}{3.710985in}}%
\pgfpathcurveto{\pgfqpoint{2.615491in}{3.699935in}}{\pgfqpoint{2.619882in}{3.689336in}}{\pgfqpoint{2.627695in}{3.681523in}}%
\pgfpathcurveto{\pgfqpoint{2.635509in}{3.673709in}}{\pgfqpoint{2.646108in}{3.669319in}}{\pgfqpoint{2.657158in}{3.669319in}}%
\pgfpathclose%
\pgfusepath{stroke,fill}%
\end{pgfscope}%
\begin{pgfscope}%
\pgfpathrectangle{\pgfqpoint{0.600000in}{0.600000in}}{\pgfqpoint{3.900000in}{3.900000in}}%
\pgfusepath{clip}%
\pgfsetbuttcap%
\pgfsetroundjoin%
\definecolor{currentfill}{rgb}{0.121569,0.466667,0.705882}%
\pgfsetfillcolor{currentfill}%
\pgfsetlinewidth{1.003750pt}%
\definecolor{currentstroke}{rgb}{0.121569,0.466667,0.705882}%
\pgfsetstrokecolor{currentstroke}%
\pgfsetdash{}{0pt}%
\pgfpathmoveto{\pgfqpoint{2.745004in}{2.674120in}}%
\pgfpathcurveto{\pgfqpoint{2.756054in}{2.674120in}}{\pgfqpoint{2.766653in}{2.678510in}}{\pgfqpoint{2.774467in}{2.686324in}}%
\pgfpathcurveto{\pgfqpoint{2.782280in}{2.694138in}}{\pgfqpoint{2.786671in}{2.704737in}}{\pgfqpoint{2.786671in}{2.715787in}}%
\pgfpathcurveto{\pgfqpoint{2.786671in}{2.726837in}}{\pgfqpoint{2.782280in}{2.737436in}}{\pgfqpoint{2.774467in}{2.745250in}}%
\pgfpathcurveto{\pgfqpoint{2.766653in}{2.753063in}}{\pgfqpoint{2.756054in}{2.757454in}}{\pgfqpoint{2.745004in}{2.757454in}}%
\pgfpathcurveto{\pgfqpoint{2.733954in}{2.757454in}}{\pgfqpoint{2.723355in}{2.753063in}}{\pgfqpoint{2.715541in}{2.745250in}}%
\pgfpathcurveto{\pgfqpoint{2.707728in}{2.737436in}}{\pgfqpoint{2.703337in}{2.726837in}}{\pgfqpoint{2.703337in}{2.715787in}}%
\pgfpathcurveto{\pgfqpoint{2.703337in}{2.704737in}}{\pgfqpoint{2.707728in}{2.694138in}}{\pgfqpoint{2.715541in}{2.686324in}}%
\pgfpathcurveto{\pgfqpoint{2.723355in}{2.678510in}}{\pgfqpoint{2.733954in}{2.674120in}}{\pgfqpoint{2.745004in}{2.674120in}}%
\pgfpathclose%
\pgfusepath{stroke,fill}%
\end{pgfscope}%
\begin{pgfscope}%
\pgfpathrectangle{\pgfqpoint{0.600000in}{0.600000in}}{\pgfqpoint{3.900000in}{3.900000in}}%
\pgfusepath{clip}%
\pgfsetbuttcap%
\pgfsetroundjoin%
\definecolor{currentfill}{rgb}{0.121569,0.466667,0.705882}%
\pgfsetfillcolor{currentfill}%
\pgfsetlinewidth{1.003750pt}%
\definecolor{currentstroke}{rgb}{0.121569,0.466667,0.705882}%
\pgfsetstrokecolor{currentstroke}%
\pgfsetdash{}{0pt}%
\pgfpathmoveto{\pgfqpoint{2.200316in}{2.884357in}}%
\pgfpathcurveto{\pgfqpoint{2.211366in}{2.884357in}}{\pgfqpoint{2.221965in}{2.888747in}}{\pgfqpoint{2.229779in}{2.896561in}}%
\pgfpathcurveto{\pgfqpoint{2.237593in}{2.904375in}}{\pgfqpoint{2.241983in}{2.914974in}}{\pgfqpoint{2.241983in}{2.926024in}}%
\pgfpathcurveto{\pgfqpoint{2.241983in}{2.937074in}}{\pgfqpoint{2.237593in}{2.947673in}}{\pgfqpoint{2.229779in}{2.955487in}}%
\pgfpathcurveto{\pgfqpoint{2.221965in}{2.963300in}}{\pgfqpoint{2.211366in}{2.967691in}}{\pgfqpoint{2.200316in}{2.967691in}}%
\pgfpathcurveto{\pgfqpoint{2.189266in}{2.967691in}}{\pgfqpoint{2.178667in}{2.963300in}}{\pgfqpoint{2.170853in}{2.955487in}}%
\pgfpathcurveto{\pgfqpoint{2.163040in}{2.947673in}}{\pgfqpoint{2.158649in}{2.937074in}}{\pgfqpoint{2.158649in}{2.926024in}}%
\pgfpathcurveto{\pgfqpoint{2.158649in}{2.914974in}}{\pgfqpoint{2.163040in}{2.904375in}}{\pgfqpoint{2.170853in}{2.896561in}}%
\pgfpathcurveto{\pgfqpoint{2.178667in}{2.888747in}}{\pgfqpoint{2.189266in}{2.884357in}}{\pgfqpoint{2.200316in}{2.884357in}}%
\pgfpathclose%
\pgfusepath{stroke,fill}%
\end{pgfscope}%
\begin{pgfscope}%
\pgfpathrectangle{\pgfqpoint{0.600000in}{0.600000in}}{\pgfqpoint{3.900000in}{3.900000in}}%
\pgfusepath{clip}%
\pgfsetbuttcap%
\pgfsetroundjoin%
\definecolor{currentfill}{rgb}{0.121569,0.466667,0.705882}%
\pgfsetfillcolor{currentfill}%
\pgfsetlinewidth{1.003750pt}%
\definecolor{currentstroke}{rgb}{0.121569,0.466667,0.705882}%
\pgfsetstrokecolor{currentstroke}%
\pgfsetdash{}{0pt}%
\pgfpathmoveto{\pgfqpoint{2.033586in}{3.079407in}}%
\pgfpathcurveto{\pgfqpoint{2.044636in}{3.079407in}}{\pgfqpoint{2.055235in}{3.083797in}}{\pgfqpoint{2.063049in}{3.091611in}}%
\pgfpathcurveto{\pgfqpoint{2.070863in}{3.099424in}}{\pgfqpoint{2.075253in}{3.110023in}}{\pgfqpoint{2.075253in}{3.121074in}}%
\pgfpathcurveto{\pgfqpoint{2.075253in}{3.132124in}}{\pgfqpoint{2.070863in}{3.142723in}}{\pgfqpoint{2.063049in}{3.150536in}}%
\pgfpathcurveto{\pgfqpoint{2.055235in}{3.158350in}}{\pgfqpoint{2.044636in}{3.162740in}}{\pgfqpoint{2.033586in}{3.162740in}}%
\pgfpathcurveto{\pgfqpoint{2.022536in}{3.162740in}}{\pgfqpoint{2.011937in}{3.158350in}}{\pgfqpoint{2.004123in}{3.150536in}}%
\pgfpathcurveto{\pgfqpoint{1.996310in}{3.142723in}}{\pgfqpoint{1.991920in}{3.132124in}}{\pgfqpoint{1.991920in}{3.121074in}}%
\pgfpathcurveto{\pgfqpoint{1.991920in}{3.110023in}}{\pgfqpoint{1.996310in}{3.099424in}}{\pgfqpoint{2.004123in}{3.091611in}}%
\pgfpathcurveto{\pgfqpoint{2.011937in}{3.083797in}}{\pgfqpoint{2.022536in}{3.079407in}}{\pgfqpoint{2.033586in}{3.079407in}}%
\pgfpathclose%
\pgfusepath{stroke,fill}%
\end{pgfscope}%
\begin{pgfscope}%
\pgfpathrectangle{\pgfqpoint{0.600000in}{0.600000in}}{\pgfqpoint{3.900000in}{3.900000in}}%
\pgfusepath{clip}%
\pgfsetbuttcap%
\pgfsetroundjoin%
\definecolor{currentfill}{rgb}{0.121569,0.466667,0.705882}%
\pgfsetfillcolor{currentfill}%
\pgfsetlinewidth{1.003750pt}%
\definecolor{currentstroke}{rgb}{0.121569,0.466667,0.705882}%
\pgfsetstrokecolor{currentstroke}%
\pgfsetdash{}{0pt}%
\pgfpathmoveto{\pgfqpoint{3.603359in}{2.763371in}}%
\pgfpathcurveto{\pgfqpoint{3.614409in}{2.763371in}}{\pgfqpoint{3.625008in}{2.767761in}}{\pgfqpoint{3.632822in}{2.775574in}}%
\pgfpathcurveto{\pgfqpoint{3.640635in}{2.783388in}}{\pgfqpoint{3.645026in}{2.793987in}}{\pgfqpoint{3.645026in}{2.805037in}}%
\pgfpathcurveto{\pgfqpoint{3.645026in}{2.816087in}}{\pgfqpoint{3.640635in}{2.826686in}}{\pgfqpoint{3.632822in}{2.834500in}}%
\pgfpathcurveto{\pgfqpoint{3.625008in}{2.842314in}}{\pgfqpoint{3.614409in}{2.846704in}}{\pgfqpoint{3.603359in}{2.846704in}}%
\pgfpathcurveto{\pgfqpoint{3.592309in}{2.846704in}}{\pgfqpoint{3.581710in}{2.842314in}}{\pgfqpoint{3.573896in}{2.834500in}}%
\pgfpathcurveto{\pgfqpoint{3.566082in}{2.826686in}}{\pgfqpoint{3.561692in}{2.816087in}}{\pgfqpoint{3.561692in}{2.805037in}}%
\pgfpathcurveto{\pgfqpoint{3.561692in}{2.793987in}}{\pgfqpoint{3.566082in}{2.783388in}}{\pgfqpoint{3.573896in}{2.775574in}}%
\pgfpathcurveto{\pgfqpoint{3.581710in}{2.767761in}}{\pgfqpoint{3.592309in}{2.763371in}}{\pgfqpoint{3.603359in}{2.763371in}}%
\pgfpathclose%
\pgfusepath{stroke,fill}%
\end{pgfscope}%
\begin{pgfscope}%
\pgfpathrectangle{\pgfqpoint{0.600000in}{0.600000in}}{\pgfqpoint{3.900000in}{3.900000in}}%
\pgfusepath{clip}%
\pgfsetbuttcap%
\pgfsetroundjoin%
\definecolor{currentfill}{rgb}{0.121569,0.466667,0.705882}%
\pgfsetfillcolor{currentfill}%
\pgfsetlinewidth{1.003750pt}%
\definecolor{currentstroke}{rgb}{0.121569,0.466667,0.705882}%
\pgfsetstrokecolor{currentstroke}%
\pgfsetdash{}{0pt}%
\pgfpathmoveto{\pgfqpoint{2.619261in}{1.811432in}}%
\pgfpathcurveto{\pgfqpoint{2.630311in}{1.811432in}}{\pgfqpoint{2.640910in}{1.815822in}}{\pgfqpoint{2.648724in}{1.823636in}}%
\pgfpathcurveto{\pgfqpoint{2.656538in}{1.831449in}}{\pgfqpoint{2.660928in}{1.842048in}}{\pgfqpoint{2.660928in}{1.853098in}}%
\pgfpathcurveto{\pgfqpoint{2.660928in}{1.864149in}}{\pgfqpoint{2.656538in}{1.874748in}}{\pgfqpoint{2.648724in}{1.882561in}}%
\pgfpathcurveto{\pgfqpoint{2.640910in}{1.890375in}}{\pgfqpoint{2.630311in}{1.894765in}}{\pgfqpoint{2.619261in}{1.894765in}}%
\pgfpathcurveto{\pgfqpoint{2.608211in}{1.894765in}}{\pgfqpoint{2.597612in}{1.890375in}}{\pgfqpoint{2.589798in}{1.882561in}}%
\pgfpathcurveto{\pgfqpoint{2.581985in}{1.874748in}}{\pgfqpoint{2.577595in}{1.864149in}}{\pgfqpoint{2.577595in}{1.853098in}}%
\pgfpathcurveto{\pgfqpoint{2.577595in}{1.842048in}}{\pgfqpoint{2.581985in}{1.831449in}}{\pgfqpoint{2.589798in}{1.823636in}}%
\pgfpathcurveto{\pgfqpoint{2.597612in}{1.815822in}}{\pgfqpoint{2.608211in}{1.811432in}}{\pgfqpoint{2.619261in}{1.811432in}}%
\pgfpathclose%
\pgfusepath{stroke,fill}%
\end{pgfscope}%
\begin{pgfscope}%
\pgfpathrectangle{\pgfqpoint{0.600000in}{0.600000in}}{\pgfqpoint{3.900000in}{3.900000in}}%
\pgfusepath{clip}%
\pgfsetbuttcap%
\pgfsetroundjoin%
\definecolor{currentfill}{rgb}{0.121569,0.466667,0.705882}%
\pgfsetfillcolor{currentfill}%
\pgfsetlinewidth{1.003750pt}%
\definecolor{currentstroke}{rgb}{0.121569,0.466667,0.705882}%
\pgfsetstrokecolor{currentstroke}%
\pgfsetdash{}{0pt}%
\pgfpathmoveto{\pgfqpoint{2.355541in}{2.463373in}}%
\pgfpathcurveto{\pgfqpoint{2.366591in}{2.463373in}}{\pgfqpoint{2.377190in}{2.467764in}}{\pgfqpoint{2.385003in}{2.475577in}}%
\pgfpathcurveto{\pgfqpoint{2.392817in}{2.483391in}}{\pgfqpoint{2.397207in}{2.493990in}}{\pgfqpoint{2.397207in}{2.505040in}}%
\pgfpathcurveto{\pgfqpoint{2.397207in}{2.516090in}}{\pgfqpoint{2.392817in}{2.526689in}}{\pgfqpoint{2.385003in}{2.534503in}}%
\pgfpathcurveto{\pgfqpoint{2.377190in}{2.542316in}}{\pgfqpoint{2.366591in}{2.546707in}}{\pgfqpoint{2.355541in}{2.546707in}}%
\pgfpathcurveto{\pgfqpoint{2.344491in}{2.546707in}}{\pgfqpoint{2.333892in}{2.542316in}}{\pgfqpoint{2.326078in}{2.534503in}}%
\pgfpathcurveto{\pgfqpoint{2.318264in}{2.526689in}}{\pgfqpoint{2.313874in}{2.516090in}}{\pgfqpoint{2.313874in}{2.505040in}}%
\pgfpathcurveto{\pgfqpoint{2.313874in}{2.493990in}}{\pgfqpoint{2.318264in}{2.483391in}}{\pgfqpoint{2.326078in}{2.475577in}}%
\pgfpathcurveto{\pgfqpoint{2.333892in}{2.467764in}}{\pgfqpoint{2.344491in}{2.463373in}}{\pgfqpoint{2.355541in}{2.463373in}}%
\pgfpathclose%
\pgfusepath{stroke,fill}%
\end{pgfscope}%
\begin{pgfscope}%
\pgfpathrectangle{\pgfqpoint{0.600000in}{0.600000in}}{\pgfqpoint{3.900000in}{3.900000in}}%
\pgfusepath{clip}%
\pgfsetbuttcap%
\pgfsetroundjoin%
\definecolor{currentfill}{rgb}{0.121569,0.466667,0.705882}%
\pgfsetfillcolor{currentfill}%
\pgfsetlinewidth{1.003750pt}%
\definecolor{currentstroke}{rgb}{0.121569,0.466667,0.705882}%
\pgfsetstrokecolor{currentstroke}%
\pgfsetdash{}{0pt}%
\pgfpathmoveto{\pgfqpoint{3.119394in}{2.606258in}}%
\pgfpathcurveto{\pgfqpoint{3.130444in}{2.606258in}}{\pgfqpoint{3.141044in}{2.610648in}}{\pgfqpoint{3.148857in}{2.618462in}}%
\pgfpathcurveto{\pgfqpoint{3.156671in}{2.626276in}}{\pgfqpoint{3.161061in}{2.636875in}}{\pgfqpoint{3.161061in}{2.647925in}}%
\pgfpathcurveto{\pgfqpoint{3.161061in}{2.658975in}}{\pgfqpoint{3.156671in}{2.669574in}}{\pgfqpoint{3.148857in}{2.677388in}}%
\pgfpathcurveto{\pgfqpoint{3.141044in}{2.685201in}}{\pgfqpoint{3.130444in}{2.689592in}}{\pgfqpoint{3.119394in}{2.689592in}}%
\pgfpathcurveto{\pgfqpoint{3.108344in}{2.689592in}}{\pgfqpoint{3.097745in}{2.685201in}}{\pgfqpoint{3.089932in}{2.677388in}}%
\pgfpathcurveto{\pgfqpoint{3.082118in}{2.669574in}}{\pgfqpoint{3.077728in}{2.658975in}}{\pgfqpoint{3.077728in}{2.647925in}}%
\pgfpathcurveto{\pgfqpoint{3.077728in}{2.636875in}}{\pgfqpoint{3.082118in}{2.626276in}}{\pgfqpoint{3.089932in}{2.618462in}}%
\pgfpathcurveto{\pgfqpoint{3.097745in}{2.610648in}}{\pgfqpoint{3.108344in}{2.606258in}}{\pgfqpoint{3.119394in}{2.606258in}}%
\pgfpathclose%
\pgfusepath{stroke,fill}%
\end{pgfscope}%
\begin{pgfscope}%
\pgfpathrectangle{\pgfqpoint{0.600000in}{0.600000in}}{\pgfqpoint{3.900000in}{3.900000in}}%
\pgfusepath{clip}%
\pgfsetbuttcap%
\pgfsetroundjoin%
\definecolor{currentfill}{rgb}{0.121569,0.466667,0.705882}%
\pgfsetfillcolor{currentfill}%
\pgfsetlinewidth{1.003750pt}%
\definecolor{currentstroke}{rgb}{0.121569,0.466667,0.705882}%
\pgfsetstrokecolor{currentstroke}%
\pgfsetdash{}{0pt}%
\pgfpathmoveto{\pgfqpoint{1.587246in}{2.962695in}}%
\pgfpathcurveto{\pgfqpoint{1.598296in}{2.962695in}}{\pgfqpoint{1.608895in}{2.967085in}}{\pgfqpoint{1.616709in}{2.974899in}}%
\pgfpathcurveto{\pgfqpoint{1.624522in}{2.982712in}}{\pgfqpoint{1.628912in}{2.993311in}}{\pgfqpoint{1.628912in}{3.004361in}}%
\pgfpathcurveto{\pgfqpoint{1.628912in}{3.015412in}}{\pgfqpoint{1.624522in}{3.026011in}}{\pgfqpoint{1.616709in}{3.033824in}}%
\pgfpathcurveto{\pgfqpoint{1.608895in}{3.041638in}}{\pgfqpoint{1.598296in}{3.046028in}}{\pgfqpoint{1.587246in}{3.046028in}}%
\pgfpathcurveto{\pgfqpoint{1.576196in}{3.046028in}}{\pgfqpoint{1.565597in}{3.041638in}}{\pgfqpoint{1.557783in}{3.033824in}}%
\pgfpathcurveto{\pgfqpoint{1.549969in}{3.026011in}}{\pgfqpoint{1.545579in}{3.015412in}}{\pgfqpoint{1.545579in}{3.004361in}}%
\pgfpathcurveto{\pgfqpoint{1.545579in}{2.993311in}}{\pgfqpoint{1.549969in}{2.982712in}}{\pgfqpoint{1.557783in}{2.974899in}}%
\pgfpathcurveto{\pgfqpoint{1.565597in}{2.967085in}}{\pgfqpoint{1.576196in}{2.962695in}}{\pgfqpoint{1.587246in}{2.962695in}}%
\pgfpathclose%
\pgfusepath{stroke,fill}%
\end{pgfscope}%
\begin{pgfscope}%
\pgfpathrectangle{\pgfqpoint{0.600000in}{0.600000in}}{\pgfqpoint{3.900000in}{3.900000in}}%
\pgfusepath{clip}%
\pgfsetbuttcap%
\pgfsetroundjoin%
\definecolor{currentfill}{rgb}{0.121569,0.466667,0.705882}%
\pgfsetfillcolor{currentfill}%
\pgfsetlinewidth{1.003750pt}%
\definecolor{currentstroke}{rgb}{0.121569,0.466667,0.705882}%
\pgfsetstrokecolor{currentstroke}%
\pgfsetdash{}{0pt}%
\pgfpathmoveto{\pgfqpoint{2.931393in}{3.544554in}}%
\pgfpathcurveto{\pgfqpoint{2.942443in}{3.544554in}}{\pgfqpoint{2.953042in}{3.548944in}}{\pgfqpoint{2.960856in}{3.556758in}}%
\pgfpathcurveto{\pgfqpoint{2.968669in}{3.564572in}}{\pgfqpoint{2.973060in}{3.575171in}}{\pgfqpoint{2.973060in}{3.586221in}}%
\pgfpathcurveto{\pgfqpoint{2.973060in}{3.597271in}}{\pgfqpoint{2.968669in}{3.607870in}}{\pgfqpoint{2.960856in}{3.615683in}}%
\pgfpathcurveto{\pgfqpoint{2.953042in}{3.623497in}}{\pgfqpoint{2.942443in}{3.627887in}}{\pgfqpoint{2.931393in}{3.627887in}}%
\pgfpathcurveto{\pgfqpoint{2.920343in}{3.627887in}}{\pgfqpoint{2.909744in}{3.623497in}}{\pgfqpoint{2.901930in}{3.615683in}}%
\pgfpathcurveto{\pgfqpoint{2.894116in}{3.607870in}}{\pgfqpoint{2.889726in}{3.597271in}}{\pgfqpoint{2.889726in}{3.586221in}}%
\pgfpathcurveto{\pgfqpoint{2.889726in}{3.575171in}}{\pgfqpoint{2.894116in}{3.564572in}}{\pgfqpoint{2.901930in}{3.556758in}}%
\pgfpathcurveto{\pgfqpoint{2.909744in}{3.548944in}}{\pgfqpoint{2.920343in}{3.544554in}}{\pgfqpoint{2.931393in}{3.544554in}}%
\pgfpathclose%
\pgfusepath{stroke,fill}%
\end{pgfscope}%
\begin{pgfscope}%
\pgfpathrectangle{\pgfqpoint{0.600000in}{0.600000in}}{\pgfqpoint{3.900000in}{3.900000in}}%
\pgfusepath{clip}%
\pgfsetbuttcap%
\pgfsetroundjoin%
\definecolor{currentfill}{rgb}{0.121569,0.466667,0.705882}%
\pgfsetfillcolor{currentfill}%
\pgfsetlinewidth{1.003750pt}%
\definecolor{currentstroke}{rgb}{0.121569,0.466667,0.705882}%
\pgfsetstrokecolor{currentstroke}%
\pgfsetdash{}{0pt}%
\pgfpathmoveto{\pgfqpoint{2.219552in}{3.054205in}}%
\pgfpathcurveto{\pgfqpoint{2.230602in}{3.054205in}}{\pgfqpoint{2.241201in}{3.058595in}}{\pgfqpoint{2.249015in}{3.066409in}}%
\pgfpathcurveto{\pgfqpoint{2.256828in}{3.074222in}}{\pgfqpoint{2.261219in}{3.084821in}}{\pgfqpoint{2.261219in}{3.095872in}}%
\pgfpathcurveto{\pgfqpoint{2.261219in}{3.106922in}}{\pgfqpoint{2.256828in}{3.117521in}}{\pgfqpoint{2.249015in}{3.125334in}}%
\pgfpathcurveto{\pgfqpoint{2.241201in}{3.133148in}}{\pgfqpoint{2.230602in}{3.137538in}}{\pgfqpoint{2.219552in}{3.137538in}}%
\pgfpathcurveto{\pgfqpoint{2.208502in}{3.137538in}}{\pgfqpoint{2.197903in}{3.133148in}}{\pgfqpoint{2.190089in}{3.125334in}}%
\pgfpathcurveto{\pgfqpoint{2.182275in}{3.117521in}}{\pgfqpoint{2.177885in}{3.106922in}}{\pgfqpoint{2.177885in}{3.095872in}}%
\pgfpathcurveto{\pgfqpoint{2.177885in}{3.084821in}}{\pgfqpoint{2.182275in}{3.074222in}}{\pgfqpoint{2.190089in}{3.066409in}}%
\pgfpathcurveto{\pgfqpoint{2.197903in}{3.058595in}}{\pgfqpoint{2.208502in}{3.054205in}}{\pgfqpoint{2.219552in}{3.054205in}}%
\pgfpathclose%
\pgfusepath{stroke,fill}%
\end{pgfscope}%
\begin{pgfscope}%
\pgfpathrectangle{\pgfqpoint{0.600000in}{0.600000in}}{\pgfqpoint{3.900000in}{3.900000in}}%
\pgfusepath{clip}%
\pgfsetbuttcap%
\pgfsetroundjoin%
\definecolor{currentfill}{rgb}{0.121569,0.466667,0.705882}%
\pgfsetfillcolor{currentfill}%
\pgfsetlinewidth{1.003750pt}%
\definecolor{currentstroke}{rgb}{0.121569,0.466667,0.705882}%
\pgfsetstrokecolor{currentstroke}%
\pgfsetdash{}{0pt}%
\pgfpathmoveto{\pgfqpoint{2.321485in}{2.620485in}}%
\pgfpathcurveto{\pgfqpoint{2.332535in}{2.620485in}}{\pgfqpoint{2.343134in}{2.624875in}}{\pgfqpoint{2.350947in}{2.632689in}}%
\pgfpathcurveto{\pgfqpoint{2.358761in}{2.640502in}}{\pgfqpoint{2.363151in}{2.651101in}}{\pgfqpoint{2.363151in}{2.662151in}}%
\pgfpathcurveto{\pgfqpoint{2.363151in}{2.673202in}}{\pgfqpoint{2.358761in}{2.683801in}}{\pgfqpoint{2.350947in}{2.691614in}}%
\pgfpathcurveto{\pgfqpoint{2.343134in}{2.699428in}}{\pgfqpoint{2.332535in}{2.703818in}}{\pgfqpoint{2.321485in}{2.703818in}}%
\pgfpathcurveto{\pgfqpoint{2.310435in}{2.703818in}}{\pgfqpoint{2.299836in}{2.699428in}}{\pgfqpoint{2.292022in}{2.691614in}}%
\pgfpathcurveto{\pgfqpoint{2.284208in}{2.683801in}}{\pgfqpoint{2.279818in}{2.673202in}}{\pgfqpoint{2.279818in}{2.662151in}}%
\pgfpathcurveto{\pgfqpoint{2.279818in}{2.651101in}}{\pgfqpoint{2.284208in}{2.640502in}}{\pgfqpoint{2.292022in}{2.632689in}}%
\pgfpathcurveto{\pgfqpoint{2.299836in}{2.624875in}}{\pgfqpoint{2.310435in}{2.620485in}}{\pgfqpoint{2.321485in}{2.620485in}}%
\pgfpathclose%
\pgfusepath{stroke,fill}%
\end{pgfscope}%
\begin{pgfscope}%
\pgfpathrectangle{\pgfqpoint{0.600000in}{0.600000in}}{\pgfqpoint{3.900000in}{3.900000in}}%
\pgfusepath{clip}%
\pgfsetbuttcap%
\pgfsetroundjoin%
\definecolor{currentfill}{rgb}{0.121569,0.466667,0.705882}%
\pgfsetfillcolor{currentfill}%
\pgfsetlinewidth{1.003750pt}%
\definecolor{currentstroke}{rgb}{0.121569,0.466667,0.705882}%
\pgfsetstrokecolor{currentstroke}%
\pgfsetdash{}{0pt}%
\pgfpathmoveto{\pgfqpoint{2.071982in}{2.255999in}}%
\pgfpathcurveto{\pgfqpoint{2.083033in}{2.255999in}}{\pgfqpoint{2.093632in}{2.260389in}}{\pgfqpoint{2.101445in}{2.268203in}}%
\pgfpathcurveto{\pgfqpoint{2.109259in}{2.276016in}}{\pgfqpoint{2.113649in}{2.286615in}}{\pgfqpoint{2.113649in}{2.297665in}}%
\pgfpathcurveto{\pgfqpoint{2.113649in}{2.308715in}}{\pgfqpoint{2.109259in}{2.319315in}}{\pgfqpoint{2.101445in}{2.327128in}}%
\pgfpathcurveto{\pgfqpoint{2.093632in}{2.334942in}}{\pgfqpoint{2.083033in}{2.339332in}}{\pgfqpoint{2.071982in}{2.339332in}}%
\pgfpathcurveto{\pgfqpoint{2.060932in}{2.339332in}}{\pgfqpoint{2.050333in}{2.334942in}}{\pgfqpoint{2.042520in}{2.327128in}}%
\pgfpathcurveto{\pgfqpoint{2.034706in}{2.319315in}}{\pgfqpoint{2.030316in}{2.308715in}}{\pgfqpoint{2.030316in}{2.297665in}}%
\pgfpathcurveto{\pgfqpoint{2.030316in}{2.286615in}}{\pgfqpoint{2.034706in}{2.276016in}}{\pgfqpoint{2.042520in}{2.268203in}}%
\pgfpathcurveto{\pgfqpoint{2.050333in}{2.260389in}}{\pgfqpoint{2.060932in}{2.255999in}}{\pgfqpoint{2.071982in}{2.255999in}}%
\pgfpathclose%
\pgfusepath{stroke,fill}%
\end{pgfscope}%
\begin{pgfscope}%
\pgfpathrectangle{\pgfqpoint{0.600000in}{0.600000in}}{\pgfqpoint{3.900000in}{3.900000in}}%
\pgfusepath{clip}%
\pgfsetbuttcap%
\pgfsetroundjoin%
\definecolor{currentfill}{rgb}{0.121569,0.466667,0.705882}%
\pgfsetfillcolor{currentfill}%
\pgfsetlinewidth{1.003750pt}%
\definecolor{currentstroke}{rgb}{0.121569,0.466667,0.705882}%
\pgfsetstrokecolor{currentstroke}%
\pgfsetdash{}{0pt}%
\pgfpathmoveto{\pgfqpoint{2.082499in}{3.056724in}}%
\pgfpathcurveto{\pgfqpoint{2.093549in}{3.056724in}}{\pgfqpoint{2.104148in}{3.061114in}}{\pgfqpoint{2.111962in}{3.068927in}}%
\pgfpathcurveto{\pgfqpoint{2.119776in}{3.076741in}}{\pgfqpoint{2.124166in}{3.087340in}}{\pgfqpoint{2.124166in}{3.098390in}}%
\pgfpathcurveto{\pgfqpoint{2.124166in}{3.109440in}}{\pgfqpoint{2.119776in}{3.120039in}}{\pgfqpoint{2.111962in}{3.127853in}}%
\pgfpathcurveto{\pgfqpoint{2.104148in}{3.135667in}}{\pgfqpoint{2.093549in}{3.140057in}}{\pgfqpoint{2.082499in}{3.140057in}}%
\pgfpathcurveto{\pgfqpoint{2.071449in}{3.140057in}}{\pgfqpoint{2.060850in}{3.135667in}}{\pgfqpoint{2.053037in}{3.127853in}}%
\pgfpathcurveto{\pgfqpoint{2.045223in}{3.120039in}}{\pgfqpoint{2.040833in}{3.109440in}}{\pgfqpoint{2.040833in}{3.098390in}}%
\pgfpathcurveto{\pgfqpoint{2.040833in}{3.087340in}}{\pgfqpoint{2.045223in}{3.076741in}}{\pgfqpoint{2.053037in}{3.068927in}}%
\pgfpathcurveto{\pgfqpoint{2.060850in}{3.061114in}}{\pgfqpoint{2.071449in}{3.056724in}}{\pgfqpoint{2.082499in}{3.056724in}}%
\pgfpathclose%
\pgfusepath{stroke,fill}%
\end{pgfscope}%
\begin{pgfscope}%
\pgfpathrectangle{\pgfqpoint{0.600000in}{0.600000in}}{\pgfqpoint{3.900000in}{3.900000in}}%
\pgfusepath{clip}%
\pgfsetbuttcap%
\pgfsetroundjoin%
\definecolor{currentfill}{rgb}{0.121569,0.466667,0.705882}%
\pgfsetfillcolor{currentfill}%
\pgfsetlinewidth{1.003750pt}%
\definecolor{currentstroke}{rgb}{0.121569,0.466667,0.705882}%
\pgfsetstrokecolor{currentstroke}%
\pgfsetdash{}{0pt}%
\pgfpathmoveto{\pgfqpoint{3.533957in}{2.659259in}}%
\pgfpathcurveto{\pgfqpoint{3.545008in}{2.659259in}}{\pgfqpoint{3.555607in}{2.663649in}}{\pgfqpoint{3.563420in}{2.671463in}}%
\pgfpathcurveto{\pgfqpoint{3.571234in}{2.679276in}}{\pgfqpoint{3.575624in}{2.689875in}}{\pgfqpoint{3.575624in}{2.700925in}}%
\pgfpathcurveto{\pgfqpoint{3.575624in}{2.711976in}}{\pgfqpoint{3.571234in}{2.722575in}}{\pgfqpoint{3.563420in}{2.730388in}}%
\pgfpathcurveto{\pgfqpoint{3.555607in}{2.738202in}}{\pgfqpoint{3.545008in}{2.742592in}}{\pgfqpoint{3.533957in}{2.742592in}}%
\pgfpathcurveto{\pgfqpoint{3.522907in}{2.742592in}}{\pgfqpoint{3.512308in}{2.738202in}}{\pgfqpoint{3.504495in}{2.730388in}}%
\pgfpathcurveto{\pgfqpoint{3.496681in}{2.722575in}}{\pgfqpoint{3.492291in}{2.711976in}}{\pgfqpoint{3.492291in}{2.700925in}}%
\pgfpathcurveto{\pgfqpoint{3.492291in}{2.689875in}}{\pgfqpoint{3.496681in}{2.679276in}}{\pgfqpoint{3.504495in}{2.671463in}}%
\pgfpathcurveto{\pgfqpoint{3.512308in}{2.663649in}}{\pgfqpoint{3.522907in}{2.659259in}}{\pgfqpoint{3.533957in}{2.659259in}}%
\pgfpathclose%
\pgfusepath{stroke,fill}%
\end{pgfscope}%
\begin{pgfscope}%
\pgfpathrectangle{\pgfqpoint{0.600000in}{0.600000in}}{\pgfqpoint{3.900000in}{3.900000in}}%
\pgfusepath{clip}%
\pgfsetbuttcap%
\pgfsetroundjoin%
\definecolor{currentfill}{rgb}{0.121569,0.466667,0.705882}%
\pgfsetfillcolor{currentfill}%
\pgfsetlinewidth{1.003750pt}%
\definecolor{currentstroke}{rgb}{0.121569,0.466667,0.705882}%
\pgfsetstrokecolor{currentstroke}%
\pgfsetdash{}{0pt}%
\pgfpathmoveto{\pgfqpoint{2.579127in}{2.557231in}}%
\pgfpathcurveto{\pgfqpoint{2.590177in}{2.557231in}}{\pgfqpoint{2.600776in}{2.561621in}}{\pgfqpoint{2.608590in}{2.569435in}}%
\pgfpathcurveto{\pgfqpoint{2.616403in}{2.577248in}}{\pgfqpoint{2.620794in}{2.587847in}}{\pgfqpoint{2.620794in}{2.598898in}}%
\pgfpathcurveto{\pgfqpoint{2.620794in}{2.609948in}}{\pgfqpoint{2.616403in}{2.620547in}}{\pgfqpoint{2.608590in}{2.628360in}}%
\pgfpathcurveto{\pgfqpoint{2.600776in}{2.636174in}}{\pgfqpoint{2.590177in}{2.640564in}}{\pgfqpoint{2.579127in}{2.640564in}}%
\pgfpathcurveto{\pgfqpoint{2.568077in}{2.640564in}}{\pgfqpoint{2.557478in}{2.636174in}}{\pgfqpoint{2.549664in}{2.628360in}}%
\pgfpathcurveto{\pgfqpoint{2.541851in}{2.620547in}}{\pgfqpoint{2.537460in}{2.609948in}}{\pgfqpoint{2.537460in}{2.598898in}}%
\pgfpathcurveto{\pgfqpoint{2.537460in}{2.587847in}}{\pgfqpoint{2.541851in}{2.577248in}}{\pgfqpoint{2.549664in}{2.569435in}}%
\pgfpathcurveto{\pgfqpoint{2.557478in}{2.561621in}}{\pgfqpoint{2.568077in}{2.557231in}}{\pgfqpoint{2.579127in}{2.557231in}}%
\pgfpathclose%
\pgfusepath{stroke,fill}%
\end{pgfscope}%
\begin{pgfscope}%
\pgfpathrectangle{\pgfqpoint{0.600000in}{0.600000in}}{\pgfqpoint{3.900000in}{3.900000in}}%
\pgfusepath{clip}%
\pgfsetbuttcap%
\pgfsetroundjoin%
\definecolor{currentfill}{rgb}{0.121569,0.466667,0.705882}%
\pgfsetfillcolor{currentfill}%
\pgfsetlinewidth{1.003750pt}%
\definecolor{currentstroke}{rgb}{0.121569,0.466667,0.705882}%
\pgfsetstrokecolor{currentstroke}%
\pgfsetdash{}{0pt}%
\pgfpathmoveto{\pgfqpoint{2.038865in}{1.584863in}}%
\pgfpathcurveto{\pgfqpoint{2.049915in}{1.584863in}}{\pgfqpoint{2.060514in}{1.589254in}}{\pgfqpoint{2.068327in}{1.597067in}}%
\pgfpathcurveto{\pgfqpoint{2.076141in}{1.604881in}}{\pgfqpoint{2.080531in}{1.615480in}}{\pgfqpoint{2.080531in}{1.626530in}}%
\pgfpathcurveto{\pgfqpoint{2.080531in}{1.637580in}}{\pgfqpoint{2.076141in}{1.648179in}}{\pgfqpoint{2.068327in}{1.655993in}}%
\pgfpathcurveto{\pgfqpoint{2.060514in}{1.663806in}}{\pgfqpoint{2.049915in}{1.668197in}}{\pgfqpoint{2.038865in}{1.668197in}}%
\pgfpathcurveto{\pgfqpoint{2.027814in}{1.668197in}}{\pgfqpoint{2.017215in}{1.663806in}}{\pgfqpoint{2.009402in}{1.655993in}}%
\pgfpathcurveto{\pgfqpoint{2.001588in}{1.648179in}}{\pgfqpoint{1.997198in}{1.637580in}}{\pgfqpoint{1.997198in}{1.626530in}}%
\pgfpathcurveto{\pgfqpoint{1.997198in}{1.615480in}}{\pgfqpoint{2.001588in}{1.604881in}}{\pgfqpoint{2.009402in}{1.597067in}}%
\pgfpathcurveto{\pgfqpoint{2.017215in}{1.589254in}}{\pgfqpoint{2.027814in}{1.584863in}}{\pgfqpoint{2.038865in}{1.584863in}}%
\pgfpathclose%
\pgfusepath{stroke,fill}%
\end{pgfscope}%
\begin{pgfscope}%
\pgfpathrectangle{\pgfqpoint{0.600000in}{0.600000in}}{\pgfqpoint{3.900000in}{3.900000in}}%
\pgfusepath{clip}%
\pgfsetbuttcap%
\pgfsetroundjoin%
\definecolor{currentfill}{rgb}{0.121569,0.466667,0.705882}%
\pgfsetfillcolor{currentfill}%
\pgfsetlinewidth{1.003750pt}%
\definecolor{currentstroke}{rgb}{0.121569,0.466667,0.705882}%
\pgfsetstrokecolor{currentstroke}%
\pgfsetdash{}{0pt}%
\pgfpathmoveto{\pgfqpoint{2.476000in}{1.982948in}}%
\pgfpathcurveto{\pgfqpoint{2.487050in}{1.982948in}}{\pgfqpoint{2.497649in}{1.987338in}}{\pgfqpoint{2.505462in}{1.995151in}}%
\pgfpathcurveto{\pgfqpoint{2.513276in}{2.002965in}}{\pgfqpoint{2.517666in}{2.013564in}}{\pgfqpoint{2.517666in}{2.024614in}}%
\pgfpathcurveto{\pgfqpoint{2.517666in}{2.035664in}}{\pgfqpoint{2.513276in}{2.046263in}}{\pgfqpoint{2.505462in}{2.054077in}}%
\pgfpathcurveto{\pgfqpoint{2.497649in}{2.061891in}}{\pgfqpoint{2.487050in}{2.066281in}}{\pgfqpoint{2.476000in}{2.066281in}}%
\pgfpathcurveto{\pgfqpoint{2.464950in}{2.066281in}}{\pgfqpoint{2.454351in}{2.061891in}}{\pgfqpoint{2.446537in}{2.054077in}}%
\pgfpathcurveto{\pgfqpoint{2.438723in}{2.046263in}}{\pgfqpoint{2.434333in}{2.035664in}}{\pgfqpoint{2.434333in}{2.024614in}}%
\pgfpathcurveto{\pgfqpoint{2.434333in}{2.013564in}}{\pgfqpoint{2.438723in}{2.002965in}}{\pgfqpoint{2.446537in}{1.995151in}}%
\pgfpathcurveto{\pgfqpoint{2.454351in}{1.987338in}}{\pgfqpoint{2.464950in}{1.982948in}}{\pgfqpoint{2.476000in}{1.982948in}}%
\pgfpathclose%
\pgfusepath{stroke,fill}%
\end{pgfscope}%
\begin{pgfscope}%
\pgfpathrectangle{\pgfqpoint{0.600000in}{0.600000in}}{\pgfqpoint{3.900000in}{3.900000in}}%
\pgfusepath{clip}%
\pgfsetbuttcap%
\pgfsetroundjoin%
\definecolor{currentfill}{rgb}{0.121569,0.466667,0.705882}%
\pgfsetfillcolor{currentfill}%
\pgfsetlinewidth{1.003750pt}%
\definecolor{currentstroke}{rgb}{0.121569,0.466667,0.705882}%
\pgfsetstrokecolor{currentstroke}%
\pgfsetdash{}{0pt}%
\pgfpathmoveto{\pgfqpoint{2.082127in}{3.129020in}}%
\pgfpathcurveto{\pgfqpoint{2.093177in}{3.129020in}}{\pgfqpoint{2.103776in}{3.133410in}}{\pgfqpoint{2.111590in}{3.141224in}}%
\pgfpathcurveto{\pgfqpoint{2.119403in}{3.149037in}}{\pgfqpoint{2.123794in}{3.159637in}}{\pgfqpoint{2.123794in}{3.170687in}}%
\pgfpathcurveto{\pgfqpoint{2.123794in}{3.181737in}}{\pgfqpoint{2.119403in}{3.192336in}}{\pgfqpoint{2.111590in}{3.200149in}}%
\pgfpathcurveto{\pgfqpoint{2.103776in}{3.207963in}}{\pgfqpoint{2.093177in}{3.212353in}}{\pgfqpoint{2.082127in}{3.212353in}}%
\pgfpathcurveto{\pgfqpoint{2.071077in}{3.212353in}}{\pgfqpoint{2.060478in}{3.207963in}}{\pgfqpoint{2.052664in}{3.200149in}}%
\pgfpathcurveto{\pgfqpoint{2.044851in}{3.192336in}}{\pgfqpoint{2.040460in}{3.181737in}}{\pgfqpoint{2.040460in}{3.170687in}}%
\pgfpathcurveto{\pgfqpoint{2.040460in}{3.159637in}}{\pgfqpoint{2.044851in}{3.149037in}}{\pgfqpoint{2.052664in}{3.141224in}}%
\pgfpathcurveto{\pgfqpoint{2.060478in}{3.133410in}}{\pgfqpoint{2.071077in}{3.129020in}}{\pgfqpoint{2.082127in}{3.129020in}}%
\pgfpathclose%
\pgfusepath{stroke,fill}%
\end{pgfscope}%
\begin{pgfscope}%
\pgfpathrectangle{\pgfqpoint{0.600000in}{0.600000in}}{\pgfqpoint{3.900000in}{3.900000in}}%
\pgfusepath{clip}%
\pgfsetbuttcap%
\pgfsetroundjoin%
\definecolor{currentfill}{rgb}{0.121569,0.466667,0.705882}%
\pgfsetfillcolor{currentfill}%
\pgfsetlinewidth{1.003750pt}%
\definecolor{currentstroke}{rgb}{0.121569,0.466667,0.705882}%
\pgfsetstrokecolor{currentstroke}%
\pgfsetdash{}{0pt}%
\pgfpathmoveto{\pgfqpoint{2.029773in}{2.891698in}}%
\pgfpathcurveto{\pgfqpoint{2.040823in}{2.891698in}}{\pgfqpoint{2.051422in}{2.896088in}}{\pgfqpoint{2.059236in}{2.903902in}}%
\pgfpathcurveto{\pgfqpoint{2.067050in}{2.911716in}}{\pgfqpoint{2.071440in}{2.922315in}}{\pgfqpoint{2.071440in}{2.933365in}}%
\pgfpathcurveto{\pgfqpoint{2.071440in}{2.944415in}}{\pgfqpoint{2.067050in}{2.955014in}}{\pgfqpoint{2.059236in}{2.962828in}}%
\pgfpathcurveto{\pgfqpoint{2.051422in}{2.970641in}}{\pgfqpoint{2.040823in}{2.975031in}}{\pgfqpoint{2.029773in}{2.975031in}}%
\pgfpathcurveto{\pgfqpoint{2.018723in}{2.975031in}}{\pgfqpoint{2.008124in}{2.970641in}}{\pgfqpoint{2.000311in}{2.962828in}}%
\pgfpathcurveto{\pgfqpoint{1.992497in}{2.955014in}}{\pgfqpoint{1.988107in}{2.944415in}}{\pgfqpoint{1.988107in}{2.933365in}}%
\pgfpathcurveto{\pgfqpoint{1.988107in}{2.922315in}}{\pgfqpoint{1.992497in}{2.911716in}}{\pgfqpoint{2.000311in}{2.903902in}}%
\pgfpathcurveto{\pgfqpoint{2.008124in}{2.896088in}}{\pgfqpoint{2.018723in}{2.891698in}}{\pgfqpoint{2.029773in}{2.891698in}}%
\pgfpathclose%
\pgfusepath{stroke,fill}%
\end{pgfscope}%
\begin{pgfscope}%
\pgfpathrectangle{\pgfqpoint{0.600000in}{0.600000in}}{\pgfqpoint{3.900000in}{3.900000in}}%
\pgfusepath{clip}%
\pgfsetbuttcap%
\pgfsetroundjoin%
\definecolor{currentfill}{rgb}{0.121569,0.466667,0.705882}%
\pgfsetfillcolor{currentfill}%
\pgfsetlinewidth{1.003750pt}%
\definecolor{currentstroke}{rgb}{0.121569,0.466667,0.705882}%
\pgfsetstrokecolor{currentstroke}%
\pgfsetdash{}{0pt}%
\pgfpathmoveto{\pgfqpoint{1.979990in}{3.053253in}}%
\pgfpathcurveto{\pgfqpoint{1.991040in}{3.053253in}}{\pgfqpoint{2.001639in}{3.057643in}}{\pgfqpoint{2.009453in}{3.065457in}}%
\pgfpathcurveto{\pgfqpoint{2.017266in}{3.073270in}}{\pgfqpoint{2.021657in}{3.083869in}}{\pgfqpoint{2.021657in}{3.094919in}}%
\pgfpathcurveto{\pgfqpoint{2.021657in}{3.105970in}}{\pgfqpoint{2.017266in}{3.116569in}}{\pgfqpoint{2.009453in}{3.124382in}}%
\pgfpathcurveto{\pgfqpoint{2.001639in}{3.132196in}}{\pgfqpoint{1.991040in}{3.136586in}}{\pgfqpoint{1.979990in}{3.136586in}}%
\pgfpathcurveto{\pgfqpoint{1.968940in}{3.136586in}}{\pgfqpoint{1.958341in}{3.132196in}}{\pgfqpoint{1.950527in}{3.124382in}}%
\pgfpathcurveto{\pgfqpoint{1.942714in}{3.116569in}}{\pgfqpoint{1.938323in}{3.105970in}}{\pgfqpoint{1.938323in}{3.094919in}}%
\pgfpathcurveto{\pgfqpoint{1.938323in}{3.083869in}}{\pgfqpoint{1.942714in}{3.073270in}}{\pgfqpoint{1.950527in}{3.065457in}}%
\pgfpathcurveto{\pgfqpoint{1.958341in}{3.057643in}}{\pgfqpoint{1.968940in}{3.053253in}}{\pgfqpoint{1.979990in}{3.053253in}}%
\pgfpathclose%
\pgfusepath{stroke,fill}%
\end{pgfscope}%
\begin{pgfscope}%
\pgfpathrectangle{\pgfqpoint{0.600000in}{0.600000in}}{\pgfqpoint{3.900000in}{3.900000in}}%
\pgfusepath{clip}%
\pgfsetbuttcap%
\pgfsetroundjoin%
\definecolor{currentfill}{rgb}{0.121569,0.466667,0.705882}%
\pgfsetfillcolor{currentfill}%
\pgfsetlinewidth{1.003750pt}%
\definecolor{currentstroke}{rgb}{0.121569,0.466667,0.705882}%
\pgfsetstrokecolor{currentstroke}%
\pgfsetdash{}{0pt}%
\pgfpathmoveto{\pgfqpoint{2.749650in}{2.643369in}}%
\pgfpathcurveto{\pgfqpoint{2.760700in}{2.643369in}}{\pgfqpoint{2.771299in}{2.647759in}}{\pgfqpoint{2.779113in}{2.655572in}}%
\pgfpathcurveto{\pgfqpoint{2.786927in}{2.663386in}}{\pgfqpoint{2.791317in}{2.673985in}}{\pgfqpoint{2.791317in}{2.685035in}}%
\pgfpathcurveto{\pgfqpoint{2.791317in}{2.696085in}}{\pgfqpoint{2.786927in}{2.706684in}}{\pgfqpoint{2.779113in}{2.714498in}}%
\pgfpathcurveto{\pgfqpoint{2.771299in}{2.722312in}}{\pgfqpoint{2.760700in}{2.726702in}}{\pgfqpoint{2.749650in}{2.726702in}}%
\pgfpathcurveto{\pgfqpoint{2.738600in}{2.726702in}}{\pgfqpoint{2.728001in}{2.722312in}}{\pgfqpoint{2.720188in}{2.714498in}}%
\pgfpathcurveto{\pgfqpoint{2.712374in}{2.706684in}}{\pgfqpoint{2.707984in}{2.696085in}}{\pgfqpoint{2.707984in}{2.685035in}}%
\pgfpathcurveto{\pgfqpoint{2.707984in}{2.673985in}}{\pgfqpoint{2.712374in}{2.663386in}}{\pgfqpoint{2.720188in}{2.655572in}}%
\pgfpathcurveto{\pgfqpoint{2.728001in}{2.647759in}}{\pgfqpoint{2.738600in}{2.643369in}}{\pgfqpoint{2.749650in}{2.643369in}}%
\pgfpathclose%
\pgfusepath{stroke,fill}%
\end{pgfscope}%
\begin{pgfscope}%
\pgfpathrectangle{\pgfqpoint{0.600000in}{0.600000in}}{\pgfqpoint{3.900000in}{3.900000in}}%
\pgfusepath{clip}%
\pgfsetbuttcap%
\pgfsetroundjoin%
\definecolor{currentfill}{rgb}{0.121569,0.466667,0.705882}%
\pgfsetfillcolor{currentfill}%
\pgfsetlinewidth{1.003750pt}%
\definecolor{currentstroke}{rgb}{0.121569,0.466667,0.705882}%
\pgfsetstrokecolor{currentstroke}%
\pgfsetdash{}{0pt}%
\pgfpathmoveto{\pgfqpoint{2.806574in}{2.632494in}}%
\pgfpathcurveto{\pgfqpoint{2.817624in}{2.632494in}}{\pgfqpoint{2.828223in}{2.636884in}}{\pgfqpoint{2.836037in}{2.644698in}}%
\pgfpathcurveto{\pgfqpoint{2.843850in}{2.652511in}}{\pgfqpoint{2.848241in}{2.663110in}}{\pgfqpoint{2.848241in}{2.674160in}}%
\pgfpathcurveto{\pgfqpoint{2.848241in}{2.685210in}}{\pgfqpoint{2.843850in}{2.695810in}}{\pgfqpoint{2.836037in}{2.703623in}}%
\pgfpathcurveto{\pgfqpoint{2.828223in}{2.711437in}}{\pgfqpoint{2.817624in}{2.715827in}}{\pgfqpoint{2.806574in}{2.715827in}}%
\pgfpathcurveto{\pgfqpoint{2.795524in}{2.715827in}}{\pgfqpoint{2.784925in}{2.711437in}}{\pgfqpoint{2.777111in}{2.703623in}}%
\pgfpathcurveto{\pgfqpoint{2.769298in}{2.695810in}}{\pgfqpoint{2.764907in}{2.685210in}}{\pgfqpoint{2.764907in}{2.674160in}}%
\pgfpathcurveto{\pgfqpoint{2.764907in}{2.663110in}}{\pgfqpoint{2.769298in}{2.652511in}}{\pgfqpoint{2.777111in}{2.644698in}}%
\pgfpathcurveto{\pgfqpoint{2.784925in}{2.636884in}}{\pgfqpoint{2.795524in}{2.632494in}}{\pgfqpoint{2.806574in}{2.632494in}}%
\pgfpathclose%
\pgfusepath{stroke,fill}%
\end{pgfscope}%
\begin{pgfscope}%
\pgfpathrectangle{\pgfqpoint{0.600000in}{0.600000in}}{\pgfqpoint{3.900000in}{3.900000in}}%
\pgfusepath{clip}%
\pgfsetbuttcap%
\pgfsetroundjoin%
\definecolor{currentfill}{rgb}{0.121569,0.466667,0.705882}%
\pgfsetfillcolor{currentfill}%
\pgfsetlinewidth{1.003750pt}%
\definecolor{currentstroke}{rgb}{0.121569,0.466667,0.705882}%
\pgfsetstrokecolor{currentstroke}%
\pgfsetdash{}{0pt}%
\pgfpathmoveto{\pgfqpoint{2.893739in}{3.292775in}}%
\pgfpathcurveto{\pgfqpoint{2.904790in}{3.292775in}}{\pgfqpoint{2.915389in}{3.297165in}}{\pgfqpoint{2.923202in}{3.304979in}}%
\pgfpathcurveto{\pgfqpoint{2.931016in}{3.312793in}}{\pgfqpoint{2.935406in}{3.323392in}}{\pgfqpoint{2.935406in}{3.334442in}}%
\pgfpathcurveto{\pgfqpoint{2.935406in}{3.345492in}}{\pgfqpoint{2.931016in}{3.356091in}}{\pgfqpoint{2.923202in}{3.363905in}}%
\pgfpathcurveto{\pgfqpoint{2.915389in}{3.371718in}}{\pgfqpoint{2.904790in}{3.376108in}}{\pgfqpoint{2.893739in}{3.376108in}}%
\pgfpathcurveto{\pgfqpoint{2.882689in}{3.376108in}}{\pgfqpoint{2.872090in}{3.371718in}}{\pgfqpoint{2.864277in}{3.363905in}}%
\pgfpathcurveto{\pgfqpoint{2.856463in}{3.356091in}}{\pgfqpoint{2.852073in}{3.345492in}}{\pgfqpoint{2.852073in}{3.334442in}}%
\pgfpathcurveto{\pgfqpoint{2.852073in}{3.323392in}}{\pgfqpoint{2.856463in}{3.312793in}}{\pgfqpoint{2.864277in}{3.304979in}}%
\pgfpathcurveto{\pgfqpoint{2.872090in}{3.297165in}}{\pgfqpoint{2.882689in}{3.292775in}}{\pgfqpoint{2.893739in}{3.292775in}}%
\pgfpathclose%
\pgfusepath{stroke,fill}%
\end{pgfscope}%
\begin{pgfscope}%
\pgfpathrectangle{\pgfqpoint{0.600000in}{0.600000in}}{\pgfqpoint{3.900000in}{3.900000in}}%
\pgfusepath{clip}%
\pgfsetbuttcap%
\pgfsetroundjoin%
\definecolor{currentfill}{rgb}{0.121569,0.466667,0.705882}%
\pgfsetfillcolor{currentfill}%
\pgfsetlinewidth{1.003750pt}%
\definecolor{currentstroke}{rgb}{0.121569,0.466667,0.705882}%
\pgfsetstrokecolor{currentstroke}%
\pgfsetdash{}{0pt}%
\pgfpathmoveto{\pgfqpoint{2.119393in}{2.252494in}}%
\pgfpathcurveto{\pgfqpoint{2.130444in}{2.252494in}}{\pgfqpoint{2.141043in}{2.256884in}}{\pgfqpoint{2.148856in}{2.264698in}}%
\pgfpathcurveto{\pgfqpoint{2.156670in}{2.272511in}}{\pgfqpoint{2.161060in}{2.283110in}}{\pgfqpoint{2.161060in}{2.294160in}}%
\pgfpathcurveto{\pgfqpoint{2.161060in}{2.305211in}}{\pgfqpoint{2.156670in}{2.315810in}}{\pgfqpoint{2.148856in}{2.323623in}}%
\pgfpathcurveto{\pgfqpoint{2.141043in}{2.331437in}}{\pgfqpoint{2.130444in}{2.335827in}}{\pgfqpoint{2.119393in}{2.335827in}}%
\pgfpathcurveto{\pgfqpoint{2.108343in}{2.335827in}}{\pgfqpoint{2.097744in}{2.331437in}}{\pgfqpoint{2.089931in}{2.323623in}}%
\pgfpathcurveto{\pgfqpoint{2.082117in}{2.315810in}}{\pgfqpoint{2.077727in}{2.305211in}}{\pgfqpoint{2.077727in}{2.294160in}}%
\pgfpathcurveto{\pgfqpoint{2.077727in}{2.283110in}}{\pgfqpoint{2.082117in}{2.272511in}}{\pgfqpoint{2.089931in}{2.264698in}}%
\pgfpathcurveto{\pgfqpoint{2.097744in}{2.256884in}}{\pgfqpoint{2.108343in}{2.252494in}}{\pgfqpoint{2.119393in}{2.252494in}}%
\pgfpathclose%
\pgfusepath{stroke,fill}%
\end{pgfscope}%
\begin{pgfscope}%
\pgfpathrectangle{\pgfqpoint{0.600000in}{0.600000in}}{\pgfqpoint{3.900000in}{3.900000in}}%
\pgfusepath{clip}%
\pgfsetbuttcap%
\pgfsetroundjoin%
\definecolor{currentfill}{rgb}{0.121569,0.466667,0.705882}%
\pgfsetfillcolor{currentfill}%
\pgfsetlinewidth{1.003750pt}%
\definecolor{currentstroke}{rgb}{0.121569,0.466667,0.705882}%
\pgfsetstrokecolor{currentstroke}%
\pgfsetdash{}{0pt}%
\pgfpathmoveto{\pgfqpoint{1.678935in}{2.748981in}}%
\pgfpathcurveto{\pgfqpoint{1.689986in}{2.748981in}}{\pgfqpoint{1.700585in}{2.753372in}}{\pgfqpoint{1.708398in}{2.761185in}}%
\pgfpathcurveto{\pgfqpoint{1.716212in}{2.768999in}}{\pgfqpoint{1.720602in}{2.779598in}}{\pgfqpoint{1.720602in}{2.790648in}}%
\pgfpathcurveto{\pgfqpoint{1.720602in}{2.801698in}}{\pgfqpoint{1.716212in}{2.812297in}}{\pgfqpoint{1.708398in}{2.820111in}}%
\pgfpathcurveto{\pgfqpoint{1.700585in}{2.827924in}}{\pgfqpoint{1.689986in}{2.832315in}}{\pgfqpoint{1.678935in}{2.832315in}}%
\pgfpathcurveto{\pgfqpoint{1.667885in}{2.832315in}}{\pgfqpoint{1.657286in}{2.827924in}}{\pgfqpoint{1.649473in}{2.820111in}}%
\pgfpathcurveto{\pgfqpoint{1.641659in}{2.812297in}}{\pgfqpoint{1.637269in}{2.801698in}}{\pgfqpoint{1.637269in}{2.790648in}}%
\pgfpathcurveto{\pgfqpoint{1.637269in}{2.779598in}}{\pgfqpoint{1.641659in}{2.768999in}}{\pgfqpoint{1.649473in}{2.761185in}}%
\pgfpathcurveto{\pgfqpoint{1.657286in}{2.753372in}}{\pgfqpoint{1.667885in}{2.748981in}}{\pgfqpoint{1.678935in}{2.748981in}}%
\pgfpathclose%
\pgfusepath{stroke,fill}%
\end{pgfscope}%
\begin{pgfscope}%
\pgfpathrectangle{\pgfqpoint{0.600000in}{0.600000in}}{\pgfqpoint{3.900000in}{3.900000in}}%
\pgfusepath{clip}%
\pgfsetbuttcap%
\pgfsetroundjoin%
\definecolor{currentfill}{rgb}{0.121569,0.466667,0.705882}%
\pgfsetfillcolor{currentfill}%
\pgfsetlinewidth{1.003750pt}%
\definecolor{currentstroke}{rgb}{0.121569,0.466667,0.705882}%
\pgfsetstrokecolor{currentstroke}%
\pgfsetdash{}{0pt}%
\pgfpathmoveto{\pgfqpoint{3.016899in}{2.298199in}}%
\pgfpathcurveto{\pgfqpoint{3.027949in}{2.298199in}}{\pgfqpoint{3.038548in}{2.302590in}}{\pgfqpoint{3.046361in}{2.310403in}}%
\pgfpathcurveto{\pgfqpoint{3.054175in}{2.318217in}}{\pgfqpoint{3.058565in}{2.328816in}}{\pgfqpoint{3.058565in}{2.339866in}}%
\pgfpathcurveto{\pgfqpoint{3.058565in}{2.350916in}}{\pgfqpoint{3.054175in}{2.361515in}}{\pgfqpoint{3.046361in}{2.369329in}}%
\pgfpathcurveto{\pgfqpoint{3.038548in}{2.377143in}}{\pgfqpoint{3.027949in}{2.381533in}}{\pgfqpoint{3.016899in}{2.381533in}}%
\pgfpathcurveto{\pgfqpoint{3.005848in}{2.381533in}}{\pgfqpoint{2.995249in}{2.377143in}}{\pgfqpoint{2.987436in}{2.369329in}}%
\pgfpathcurveto{\pgfqpoint{2.979622in}{2.361515in}}{\pgfqpoint{2.975232in}{2.350916in}}{\pgfqpoint{2.975232in}{2.339866in}}%
\pgfpathcurveto{\pgfqpoint{2.975232in}{2.328816in}}{\pgfqpoint{2.979622in}{2.318217in}}{\pgfqpoint{2.987436in}{2.310403in}}%
\pgfpathcurveto{\pgfqpoint{2.995249in}{2.302590in}}{\pgfqpoint{3.005848in}{2.298199in}}{\pgfqpoint{3.016899in}{2.298199in}}%
\pgfpathclose%
\pgfusepath{stroke,fill}%
\end{pgfscope}%
\begin{pgfscope}%
\pgfpathrectangle{\pgfqpoint{0.600000in}{0.600000in}}{\pgfqpoint{3.900000in}{3.900000in}}%
\pgfusepath{clip}%
\pgfsetbuttcap%
\pgfsetroundjoin%
\definecolor{currentfill}{rgb}{0.121569,0.466667,0.705882}%
\pgfsetfillcolor{currentfill}%
\pgfsetlinewidth{1.003750pt}%
\definecolor{currentstroke}{rgb}{0.121569,0.466667,0.705882}%
\pgfsetstrokecolor{currentstroke}%
\pgfsetdash{}{0pt}%
\pgfpathmoveto{\pgfqpoint{1.528434in}{3.178911in}}%
\pgfpathcurveto{\pgfqpoint{1.539484in}{3.178911in}}{\pgfqpoint{1.550083in}{3.183301in}}{\pgfqpoint{1.557896in}{3.191114in}}%
\pgfpathcurveto{\pgfqpoint{1.565710in}{3.198928in}}{\pgfqpoint{1.570100in}{3.209527in}}{\pgfqpoint{1.570100in}{3.220577in}}%
\pgfpathcurveto{\pgfqpoint{1.570100in}{3.231627in}}{\pgfqpoint{1.565710in}{3.242226in}}{\pgfqpoint{1.557896in}{3.250040in}}%
\pgfpathcurveto{\pgfqpoint{1.550083in}{3.257854in}}{\pgfqpoint{1.539484in}{3.262244in}}{\pgfqpoint{1.528434in}{3.262244in}}%
\pgfpathcurveto{\pgfqpoint{1.517384in}{3.262244in}}{\pgfqpoint{1.506785in}{3.257854in}}{\pgfqpoint{1.498971in}{3.250040in}}%
\pgfpathcurveto{\pgfqpoint{1.491157in}{3.242226in}}{\pgfqpoint{1.486767in}{3.231627in}}{\pgfqpoint{1.486767in}{3.220577in}}%
\pgfpathcurveto{\pgfqpoint{1.486767in}{3.209527in}}{\pgfqpoint{1.491157in}{3.198928in}}{\pgfqpoint{1.498971in}{3.191114in}}%
\pgfpathcurveto{\pgfqpoint{1.506785in}{3.183301in}}{\pgfqpoint{1.517384in}{3.178911in}}{\pgfqpoint{1.528434in}{3.178911in}}%
\pgfpathclose%
\pgfusepath{stroke,fill}%
\end{pgfscope}%
\begin{pgfscope}%
\pgfpathrectangle{\pgfqpoint{0.600000in}{0.600000in}}{\pgfqpoint{3.900000in}{3.900000in}}%
\pgfusepath{clip}%
\pgfsetbuttcap%
\pgfsetroundjoin%
\definecolor{currentfill}{rgb}{0.121569,0.466667,0.705882}%
\pgfsetfillcolor{currentfill}%
\pgfsetlinewidth{1.003750pt}%
\definecolor{currentstroke}{rgb}{0.121569,0.466667,0.705882}%
\pgfsetstrokecolor{currentstroke}%
\pgfsetdash{}{0pt}%
\pgfpathmoveto{\pgfqpoint{1.959695in}{2.643956in}}%
\pgfpathcurveto{\pgfqpoint{1.970746in}{2.643956in}}{\pgfqpoint{1.981345in}{2.648347in}}{\pgfqpoint{1.989158in}{2.656160in}}%
\pgfpathcurveto{\pgfqpoint{1.996972in}{2.663974in}}{\pgfqpoint{2.001362in}{2.674573in}}{\pgfqpoint{2.001362in}{2.685623in}}%
\pgfpathcurveto{\pgfqpoint{2.001362in}{2.696673in}}{\pgfqpoint{1.996972in}{2.707272in}}{\pgfqpoint{1.989158in}{2.715086in}}%
\pgfpathcurveto{\pgfqpoint{1.981345in}{2.722899in}}{\pgfqpoint{1.970746in}{2.727290in}}{\pgfqpoint{1.959695in}{2.727290in}}%
\pgfpathcurveto{\pgfqpoint{1.948645in}{2.727290in}}{\pgfqpoint{1.938046in}{2.722899in}}{\pgfqpoint{1.930233in}{2.715086in}}%
\pgfpathcurveto{\pgfqpoint{1.922419in}{2.707272in}}{\pgfqpoint{1.918029in}{2.696673in}}{\pgfqpoint{1.918029in}{2.685623in}}%
\pgfpathcurveto{\pgfqpoint{1.918029in}{2.674573in}}{\pgfqpoint{1.922419in}{2.663974in}}{\pgfqpoint{1.930233in}{2.656160in}}%
\pgfpathcurveto{\pgfqpoint{1.938046in}{2.648347in}}{\pgfqpoint{1.948645in}{2.643956in}}{\pgfqpoint{1.959695in}{2.643956in}}%
\pgfpathclose%
\pgfusepath{stroke,fill}%
\end{pgfscope}%
\begin{pgfscope}%
\pgfpathrectangle{\pgfqpoint{0.600000in}{0.600000in}}{\pgfqpoint{3.900000in}{3.900000in}}%
\pgfusepath{clip}%
\pgfsetbuttcap%
\pgfsetroundjoin%
\definecolor{currentfill}{rgb}{0.121569,0.466667,0.705882}%
\pgfsetfillcolor{currentfill}%
\pgfsetlinewidth{1.003750pt}%
\definecolor{currentstroke}{rgb}{0.121569,0.466667,0.705882}%
\pgfsetstrokecolor{currentstroke}%
\pgfsetdash{}{0pt}%
\pgfpathmoveto{\pgfqpoint{2.866323in}{2.255524in}}%
\pgfpathcurveto{\pgfqpoint{2.877373in}{2.255524in}}{\pgfqpoint{2.887972in}{2.259915in}}{\pgfqpoint{2.895786in}{2.267728in}}%
\pgfpathcurveto{\pgfqpoint{2.903600in}{2.275542in}}{\pgfqpoint{2.907990in}{2.286141in}}{\pgfqpoint{2.907990in}{2.297191in}}%
\pgfpathcurveto{\pgfqpoint{2.907990in}{2.308241in}}{\pgfqpoint{2.903600in}{2.318840in}}{\pgfqpoint{2.895786in}{2.326654in}}%
\pgfpathcurveto{\pgfqpoint{2.887972in}{2.334467in}}{\pgfqpoint{2.877373in}{2.338858in}}{\pgfqpoint{2.866323in}{2.338858in}}%
\pgfpathcurveto{\pgfqpoint{2.855273in}{2.338858in}}{\pgfqpoint{2.844674in}{2.334467in}}{\pgfqpoint{2.836860in}{2.326654in}}%
\pgfpathcurveto{\pgfqpoint{2.829047in}{2.318840in}}{\pgfqpoint{2.824656in}{2.308241in}}{\pgfqpoint{2.824656in}{2.297191in}}%
\pgfpathcurveto{\pgfqpoint{2.824656in}{2.286141in}}{\pgfqpoint{2.829047in}{2.275542in}}{\pgfqpoint{2.836860in}{2.267728in}}%
\pgfpathcurveto{\pgfqpoint{2.844674in}{2.259915in}}{\pgfqpoint{2.855273in}{2.255524in}}{\pgfqpoint{2.866323in}{2.255524in}}%
\pgfpathclose%
\pgfusepath{stroke,fill}%
\end{pgfscope}%
\begin{pgfscope}%
\pgfpathrectangle{\pgfqpoint{0.600000in}{0.600000in}}{\pgfqpoint{3.900000in}{3.900000in}}%
\pgfusepath{clip}%
\pgfsetbuttcap%
\pgfsetroundjoin%
\definecolor{currentfill}{rgb}{0.121569,0.466667,0.705882}%
\pgfsetfillcolor{currentfill}%
\pgfsetlinewidth{1.003750pt}%
\definecolor{currentstroke}{rgb}{0.121569,0.466667,0.705882}%
\pgfsetstrokecolor{currentstroke}%
\pgfsetdash{}{0pt}%
\pgfpathmoveto{\pgfqpoint{2.937708in}{2.690253in}}%
\pgfpathcurveto{\pgfqpoint{2.948758in}{2.690253in}}{\pgfqpoint{2.959357in}{2.694644in}}{\pgfqpoint{2.967170in}{2.702457in}}%
\pgfpathcurveto{\pgfqpoint{2.974984in}{2.710271in}}{\pgfqpoint{2.979374in}{2.720870in}}{\pgfqpoint{2.979374in}{2.731920in}}%
\pgfpathcurveto{\pgfqpoint{2.979374in}{2.742970in}}{\pgfqpoint{2.974984in}{2.753569in}}{\pgfqpoint{2.967170in}{2.761383in}}%
\pgfpathcurveto{\pgfqpoint{2.959357in}{2.769197in}}{\pgfqpoint{2.948758in}{2.773587in}}{\pgfqpoint{2.937708in}{2.773587in}}%
\pgfpathcurveto{\pgfqpoint{2.926657in}{2.773587in}}{\pgfqpoint{2.916058in}{2.769197in}}{\pgfqpoint{2.908245in}{2.761383in}}%
\pgfpathcurveto{\pgfqpoint{2.900431in}{2.753569in}}{\pgfqpoint{2.896041in}{2.742970in}}{\pgfqpoint{2.896041in}{2.731920in}}%
\pgfpathcurveto{\pgfqpoint{2.896041in}{2.720870in}}{\pgfqpoint{2.900431in}{2.710271in}}{\pgfqpoint{2.908245in}{2.702457in}}%
\pgfpathcurveto{\pgfqpoint{2.916058in}{2.694644in}}{\pgfqpoint{2.926657in}{2.690253in}}{\pgfqpoint{2.937708in}{2.690253in}}%
\pgfpathclose%
\pgfusepath{stroke,fill}%
\end{pgfscope}%
\begin{pgfscope}%
\pgfpathrectangle{\pgfqpoint{0.600000in}{0.600000in}}{\pgfqpoint{3.900000in}{3.900000in}}%
\pgfusepath{clip}%
\pgfsetbuttcap%
\pgfsetroundjoin%
\definecolor{currentfill}{rgb}{0.121569,0.466667,0.705882}%
\pgfsetfillcolor{currentfill}%
\pgfsetlinewidth{1.003750pt}%
\definecolor{currentstroke}{rgb}{0.121569,0.466667,0.705882}%
\pgfsetstrokecolor{currentstroke}%
\pgfsetdash{}{0pt}%
\pgfpathmoveto{\pgfqpoint{1.959067in}{2.463943in}}%
\pgfpathcurveto{\pgfqpoint{1.970118in}{2.463943in}}{\pgfqpoint{1.980717in}{2.468333in}}{\pgfqpoint{1.988530in}{2.476147in}}%
\pgfpathcurveto{\pgfqpoint{1.996344in}{2.483961in}}{\pgfqpoint{2.000734in}{2.494560in}}{\pgfqpoint{2.000734in}{2.505610in}}%
\pgfpathcurveto{\pgfqpoint{2.000734in}{2.516660in}}{\pgfqpoint{1.996344in}{2.527259in}}{\pgfqpoint{1.988530in}{2.535073in}}%
\pgfpathcurveto{\pgfqpoint{1.980717in}{2.542886in}}{\pgfqpoint{1.970118in}{2.547277in}}{\pgfqpoint{1.959067in}{2.547277in}}%
\pgfpathcurveto{\pgfqpoint{1.948017in}{2.547277in}}{\pgfqpoint{1.937418in}{2.542886in}}{\pgfqpoint{1.929605in}{2.535073in}}%
\pgfpathcurveto{\pgfqpoint{1.921791in}{2.527259in}}{\pgfqpoint{1.917401in}{2.516660in}}{\pgfqpoint{1.917401in}{2.505610in}}%
\pgfpathcurveto{\pgfqpoint{1.917401in}{2.494560in}}{\pgfqpoint{1.921791in}{2.483961in}}{\pgfqpoint{1.929605in}{2.476147in}}%
\pgfpathcurveto{\pgfqpoint{1.937418in}{2.468333in}}{\pgfqpoint{1.948017in}{2.463943in}}{\pgfqpoint{1.959067in}{2.463943in}}%
\pgfpathclose%
\pgfusepath{stroke,fill}%
\end{pgfscope}%
\begin{pgfscope}%
\pgfpathrectangle{\pgfqpoint{0.600000in}{0.600000in}}{\pgfqpoint{3.900000in}{3.900000in}}%
\pgfusepath{clip}%
\pgfsetbuttcap%
\pgfsetroundjoin%
\definecolor{currentfill}{rgb}{0.121569,0.466667,0.705882}%
\pgfsetfillcolor{currentfill}%
\pgfsetlinewidth{1.003750pt}%
\definecolor{currentstroke}{rgb}{0.121569,0.466667,0.705882}%
\pgfsetstrokecolor{currentstroke}%
\pgfsetdash{}{0pt}%
\pgfpathmoveto{\pgfqpoint{2.567564in}{1.274484in}}%
\pgfpathcurveto{\pgfqpoint{2.578614in}{1.274484in}}{\pgfqpoint{2.589213in}{1.278874in}}{\pgfqpoint{2.597026in}{1.286688in}}%
\pgfpathcurveto{\pgfqpoint{2.604840in}{1.294501in}}{\pgfqpoint{2.609230in}{1.305100in}}{\pgfqpoint{2.609230in}{1.316150in}}%
\pgfpathcurveto{\pgfqpoint{2.609230in}{1.327201in}}{\pgfqpoint{2.604840in}{1.337800in}}{\pgfqpoint{2.597026in}{1.345613in}}%
\pgfpathcurveto{\pgfqpoint{2.589213in}{1.353427in}}{\pgfqpoint{2.578614in}{1.357817in}}{\pgfqpoint{2.567564in}{1.357817in}}%
\pgfpathcurveto{\pgfqpoint{2.556513in}{1.357817in}}{\pgfqpoint{2.545914in}{1.353427in}}{\pgfqpoint{2.538101in}{1.345613in}}%
\pgfpathcurveto{\pgfqpoint{2.530287in}{1.337800in}}{\pgfqpoint{2.525897in}{1.327201in}}{\pgfqpoint{2.525897in}{1.316150in}}%
\pgfpathcurveto{\pgfqpoint{2.525897in}{1.305100in}}{\pgfqpoint{2.530287in}{1.294501in}}{\pgfqpoint{2.538101in}{1.286688in}}%
\pgfpathcurveto{\pgfqpoint{2.545914in}{1.278874in}}{\pgfqpoint{2.556513in}{1.274484in}}{\pgfqpoint{2.567564in}{1.274484in}}%
\pgfpathclose%
\pgfusepath{stroke,fill}%
\end{pgfscope}%
\begin{pgfscope}%
\pgfpathrectangle{\pgfqpoint{0.600000in}{0.600000in}}{\pgfqpoint{3.900000in}{3.900000in}}%
\pgfusepath{clip}%
\pgfsetbuttcap%
\pgfsetroundjoin%
\definecolor{currentfill}{rgb}{0.121569,0.466667,0.705882}%
\pgfsetfillcolor{currentfill}%
\pgfsetlinewidth{1.003750pt}%
\definecolor{currentstroke}{rgb}{0.121569,0.466667,0.705882}%
\pgfsetstrokecolor{currentstroke}%
\pgfsetdash{}{0pt}%
\pgfpathmoveto{\pgfqpoint{2.270451in}{3.564178in}}%
\pgfpathcurveto{\pgfqpoint{2.281501in}{3.564178in}}{\pgfqpoint{2.292100in}{3.568568in}}{\pgfqpoint{2.299913in}{3.576382in}}%
\pgfpathcurveto{\pgfqpoint{2.307727in}{3.584196in}}{\pgfqpoint{2.312117in}{3.594795in}}{\pgfqpoint{2.312117in}{3.605845in}}%
\pgfpathcurveto{\pgfqpoint{2.312117in}{3.616895in}}{\pgfqpoint{2.307727in}{3.627494in}}{\pgfqpoint{2.299913in}{3.635308in}}%
\pgfpathcurveto{\pgfqpoint{2.292100in}{3.643121in}}{\pgfqpoint{2.281501in}{3.647511in}}{\pgfqpoint{2.270451in}{3.647511in}}%
\pgfpathcurveto{\pgfqpoint{2.259400in}{3.647511in}}{\pgfqpoint{2.248801in}{3.643121in}}{\pgfqpoint{2.240988in}{3.635308in}}%
\pgfpathcurveto{\pgfqpoint{2.233174in}{3.627494in}}{\pgfqpoint{2.228784in}{3.616895in}}{\pgfqpoint{2.228784in}{3.605845in}}%
\pgfpathcurveto{\pgfqpoint{2.228784in}{3.594795in}}{\pgfqpoint{2.233174in}{3.584196in}}{\pgfqpoint{2.240988in}{3.576382in}}%
\pgfpathcurveto{\pgfqpoint{2.248801in}{3.568568in}}{\pgfqpoint{2.259400in}{3.564178in}}{\pgfqpoint{2.270451in}{3.564178in}}%
\pgfpathclose%
\pgfusepath{stroke,fill}%
\end{pgfscope}%
\begin{pgfscope}%
\pgfpathrectangle{\pgfqpoint{0.600000in}{0.600000in}}{\pgfqpoint{3.900000in}{3.900000in}}%
\pgfusepath{clip}%
\pgfsetbuttcap%
\pgfsetroundjoin%
\definecolor{currentfill}{rgb}{0.121569,0.466667,0.705882}%
\pgfsetfillcolor{currentfill}%
\pgfsetlinewidth{1.003750pt}%
\definecolor{currentstroke}{rgb}{0.121569,0.466667,0.705882}%
\pgfsetstrokecolor{currentstroke}%
\pgfsetdash{}{0pt}%
\pgfpathmoveto{\pgfqpoint{2.697911in}{2.342042in}}%
\pgfpathcurveto{\pgfqpoint{2.708962in}{2.342042in}}{\pgfqpoint{2.719561in}{2.346433in}}{\pgfqpoint{2.727374in}{2.354246in}}%
\pgfpathcurveto{\pgfqpoint{2.735188in}{2.362060in}}{\pgfqpoint{2.739578in}{2.372659in}}{\pgfqpoint{2.739578in}{2.383709in}}%
\pgfpathcurveto{\pgfqpoint{2.739578in}{2.394759in}}{\pgfqpoint{2.735188in}{2.405358in}}{\pgfqpoint{2.727374in}{2.413172in}}%
\pgfpathcurveto{\pgfqpoint{2.719561in}{2.420986in}}{\pgfqpoint{2.708962in}{2.425376in}}{\pgfqpoint{2.697911in}{2.425376in}}%
\pgfpathcurveto{\pgfqpoint{2.686861in}{2.425376in}}{\pgfqpoint{2.676262in}{2.420986in}}{\pgfqpoint{2.668449in}{2.413172in}}%
\pgfpathcurveto{\pgfqpoint{2.660635in}{2.405358in}}{\pgfqpoint{2.656245in}{2.394759in}}{\pgfqpoint{2.656245in}{2.383709in}}%
\pgfpathcurveto{\pgfqpoint{2.656245in}{2.372659in}}{\pgfqpoint{2.660635in}{2.362060in}}{\pgfqpoint{2.668449in}{2.354246in}}%
\pgfpathcurveto{\pgfqpoint{2.676262in}{2.346433in}}{\pgfqpoint{2.686861in}{2.342042in}}{\pgfqpoint{2.697911in}{2.342042in}}%
\pgfpathclose%
\pgfusepath{stroke,fill}%
\end{pgfscope}%
\begin{pgfscope}%
\pgfpathrectangle{\pgfqpoint{0.600000in}{0.600000in}}{\pgfqpoint{3.900000in}{3.900000in}}%
\pgfusepath{clip}%
\pgfsetbuttcap%
\pgfsetroundjoin%
\definecolor{currentfill}{rgb}{0.121569,0.466667,0.705882}%
\pgfsetfillcolor{currentfill}%
\pgfsetlinewidth{1.003750pt}%
\definecolor{currentstroke}{rgb}{0.121569,0.466667,0.705882}%
\pgfsetstrokecolor{currentstroke}%
\pgfsetdash{}{0pt}%
\pgfpathmoveto{\pgfqpoint{3.262539in}{2.625109in}}%
\pgfpathcurveto{\pgfqpoint{3.273589in}{2.625109in}}{\pgfqpoint{3.284188in}{2.629499in}}{\pgfqpoint{3.292002in}{2.637313in}}%
\pgfpathcurveto{\pgfqpoint{3.299815in}{2.645126in}}{\pgfqpoint{3.304206in}{2.655726in}}{\pgfqpoint{3.304206in}{2.666776in}}%
\pgfpathcurveto{\pgfqpoint{3.304206in}{2.677826in}}{\pgfqpoint{3.299815in}{2.688425in}}{\pgfqpoint{3.292002in}{2.696238in}}%
\pgfpathcurveto{\pgfqpoint{3.284188in}{2.704052in}}{\pgfqpoint{3.273589in}{2.708442in}}{\pgfqpoint{3.262539in}{2.708442in}}%
\pgfpathcurveto{\pgfqpoint{3.251489in}{2.708442in}}{\pgfqpoint{3.240890in}{2.704052in}}{\pgfqpoint{3.233076in}{2.696238in}}%
\pgfpathcurveto{\pgfqpoint{3.225263in}{2.688425in}}{\pgfqpoint{3.220872in}{2.677826in}}{\pgfqpoint{3.220872in}{2.666776in}}%
\pgfpathcurveto{\pgfqpoint{3.220872in}{2.655726in}}{\pgfqpoint{3.225263in}{2.645126in}}{\pgfqpoint{3.233076in}{2.637313in}}%
\pgfpathcurveto{\pgfqpoint{3.240890in}{2.629499in}}{\pgfqpoint{3.251489in}{2.625109in}}{\pgfqpoint{3.262539in}{2.625109in}}%
\pgfpathclose%
\pgfusepath{stroke,fill}%
\end{pgfscope}%
\begin{pgfscope}%
\pgfpathrectangle{\pgfqpoint{0.600000in}{0.600000in}}{\pgfqpoint{3.900000in}{3.900000in}}%
\pgfusepath{clip}%
\pgfsetbuttcap%
\pgfsetroundjoin%
\definecolor{currentfill}{rgb}{0.121569,0.466667,0.705882}%
\pgfsetfillcolor{currentfill}%
\pgfsetlinewidth{1.003750pt}%
\definecolor{currentstroke}{rgb}{0.121569,0.466667,0.705882}%
\pgfsetstrokecolor{currentstroke}%
\pgfsetdash{}{0pt}%
\pgfpathmoveto{\pgfqpoint{2.961549in}{2.639950in}}%
\pgfpathcurveto{\pgfqpoint{2.972599in}{2.639950in}}{\pgfqpoint{2.983198in}{2.644340in}}{\pgfqpoint{2.991012in}{2.652154in}}%
\pgfpathcurveto{\pgfqpoint{2.998825in}{2.659967in}}{\pgfqpoint{3.003216in}{2.670567in}}{\pgfqpoint{3.003216in}{2.681617in}}%
\pgfpathcurveto{\pgfqpoint{3.003216in}{2.692667in}}{\pgfqpoint{2.998825in}{2.703266in}}{\pgfqpoint{2.991012in}{2.711079in}}%
\pgfpathcurveto{\pgfqpoint{2.983198in}{2.718893in}}{\pgfqpoint{2.972599in}{2.723283in}}{\pgfqpoint{2.961549in}{2.723283in}}%
\pgfpathcurveto{\pgfqpoint{2.950499in}{2.723283in}}{\pgfqpoint{2.939900in}{2.718893in}}{\pgfqpoint{2.932086in}{2.711079in}}%
\pgfpathcurveto{\pgfqpoint{2.924273in}{2.703266in}}{\pgfqpoint{2.919882in}{2.692667in}}{\pgfqpoint{2.919882in}{2.681617in}}%
\pgfpathcurveto{\pgfqpoint{2.919882in}{2.670567in}}{\pgfqpoint{2.924273in}{2.659967in}}{\pgfqpoint{2.932086in}{2.652154in}}%
\pgfpathcurveto{\pgfqpoint{2.939900in}{2.644340in}}{\pgfqpoint{2.950499in}{2.639950in}}{\pgfqpoint{2.961549in}{2.639950in}}%
\pgfpathclose%
\pgfusepath{stroke,fill}%
\end{pgfscope}%
\begin{pgfscope}%
\pgfpathrectangle{\pgfqpoint{0.600000in}{0.600000in}}{\pgfqpoint{3.900000in}{3.900000in}}%
\pgfusepath{clip}%
\pgfsetbuttcap%
\pgfsetroundjoin%
\definecolor{currentfill}{rgb}{0.121569,0.466667,0.705882}%
\pgfsetfillcolor{currentfill}%
\pgfsetlinewidth{1.003750pt}%
\definecolor{currentstroke}{rgb}{0.121569,0.466667,0.705882}%
\pgfsetstrokecolor{currentstroke}%
\pgfsetdash{}{0pt}%
\pgfpathmoveto{\pgfqpoint{3.047146in}{2.318489in}}%
\pgfpathcurveto{\pgfqpoint{3.058196in}{2.318489in}}{\pgfqpoint{3.068795in}{2.322879in}}{\pgfqpoint{3.076609in}{2.330693in}}%
\pgfpathcurveto{\pgfqpoint{3.084422in}{2.338507in}}{\pgfqpoint{3.088812in}{2.349106in}}{\pgfqpoint{3.088812in}{2.360156in}}%
\pgfpathcurveto{\pgfqpoint{3.088812in}{2.371206in}}{\pgfqpoint{3.084422in}{2.381805in}}{\pgfqpoint{3.076609in}{2.389619in}}%
\pgfpathcurveto{\pgfqpoint{3.068795in}{2.397432in}}{\pgfqpoint{3.058196in}{2.401822in}}{\pgfqpoint{3.047146in}{2.401822in}}%
\pgfpathcurveto{\pgfqpoint{3.036096in}{2.401822in}}{\pgfqpoint{3.025497in}{2.397432in}}{\pgfqpoint{3.017683in}{2.389619in}}%
\pgfpathcurveto{\pgfqpoint{3.009869in}{2.381805in}}{\pgfqpoint{3.005479in}{2.371206in}}{\pgfqpoint{3.005479in}{2.360156in}}%
\pgfpathcurveto{\pgfqpoint{3.005479in}{2.349106in}}{\pgfqpoint{3.009869in}{2.338507in}}{\pgfqpoint{3.017683in}{2.330693in}}%
\pgfpathcurveto{\pgfqpoint{3.025497in}{2.322879in}}{\pgfqpoint{3.036096in}{2.318489in}}{\pgfqpoint{3.047146in}{2.318489in}}%
\pgfpathclose%
\pgfusepath{stroke,fill}%
\end{pgfscope}%
\begin{pgfscope}%
\pgfpathrectangle{\pgfqpoint{0.600000in}{0.600000in}}{\pgfqpoint{3.900000in}{3.900000in}}%
\pgfusepath{clip}%
\pgfsetbuttcap%
\pgfsetroundjoin%
\definecolor{currentfill}{rgb}{0.121569,0.466667,0.705882}%
\pgfsetfillcolor{currentfill}%
\pgfsetlinewidth{1.003750pt}%
\definecolor{currentstroke}{rgb}{0.121569,0.466667,0.705882}%
\pgfsetstrokecolor{currentstroke}%
\pgfsetdash{}{0pt}%
\pgfpathmoveto{\pgfqpoint{2.274822in}{2.601638in}}%
\pgfpathcurveto{\pgfqpoint{2.285872in}{2.601638in}}{\pgfqpoint{2.296471in}{2.606028in}}{\pgfqpoint{2.304285in}{2.613842in}}%
\pgfpathcurveto{\pgfqpoint{2.312098in}{2.621655in}}{\pgfqpoint{2.316488in}{2.632254in}}{\pgfqpoint{2.316488in}{2.643304in}}%
\pgfpathcurveto{\pgfqpoint{2.316488in}{2.654354in}}{\pgfqpoint{2.312098in}{2.664954in}}{\pgfqpoint{2.304285in}{2.672767in}}%
\pgfpathcurveto{\pgfqpoint{2.296471in}{2.680581in}}{\pgfqpoint{2.285872in}{2.684971in}}{\pgfqpoint{2.274822in}{2.684971in}}%
\pgfpathcurveto{\pgfqpoint{2.263772in}{2.684971in}}{\pgfqpoint{2.253173in}{2.680581in}}{\pgfqpoint{2.245359in}{2.672767in}}%
\pgfpathcurveto{\pgfqpoint{2.237545in}{2.664954in}}{\pgfqpoint{2.233155in}{2.654354in}}{\pgfqpoint{2.233155in}{2.643304in}}%
\pgfpathcurveto{\pgfqpoint{2.233155in}{2.632254in}}{\pgfqpoint{2.237545in}{2.621655in}}{\pgfqpoint{2.245359in}{2.613842in}}%
\pgfpathcurveto{\pgfqpoint{2.253173in}{2.606028in}}{\pgfqpoint{2.263772in}{2.601638in}}{\pgfqpoint{2.274822in}{2.601638in}}%
\pgfpathclose%
\pgfusepath{stroke,fill}%
\end{pgfscope}%
\begin{pgfscope}%
\pgfpathrectangle{\pgfqpoint{0.600000in}{0.600000in}}{\pgfqpoint{3.900000in}{3.900000in}}%
\pgfusepath{clip}%
\pgfsetbuttcap%
\pgfsetroundjoin%
\definecolor{currentfill}{rgb}{0.121569,0.466667,0.705882}%
\pgfsetfillcolor{currentfill}%
\pgfsetlinewidth{1.003750pt}%
\definecolor{currentstroke}{rgb}{0.121569,0.466667,0.705882}%
\pgfsetstrokecolor{currentstroke}%
\pgfsetdash{}{0pt}%
\pgfpathmoveto{\pgfqpoint{2.739213in}{1.995716in}}%
\pgfpathcurveto{\pgfqpoint{2.750264in}{1.995716in}}{\pgfqpoint{2.760863in}{2.000106in}}{\pgfqpoint{2.768676in}{2.007920in}}%
\pgfpathcurveto{\pgfqpoint{2.776490in}{2.015734in}}{\pgfqpoint{2.780880in}{2.026333in}}{\pgfqpoint{2.780880in}{2.037383in}}%
\pgfpathcurveto{\pgfqpoint{2.780880in}{2.048433in}}{\pgfqpoint{2.776490in}{2.059032in}}{\pgfqpoint{2.768676in}{2.066846in}}%
\pgfpathcurveto{\pgfqpoint{2.760863in}{2.074659in}}{\pgfqpoint{2.750264in}{2.079049in}}{\pgfqpoint{2.739213in}{2.079049in}}%
\pgfpathcurveto{\pgfqpoint{2.728163in}{2.079049in}}{\pgfqpoint{2.717564in}{2.074659in}}{\pgfqpoint{2.709751in}{2.066846in}}%
\pgfpathcurveto{\pgfqpoint{2.701937in}{2.059032in}}{\pgfqpoint{2.697547in}{2.048433in}}{\pgfqpoint{2.697547in}{2.037383in}}%
\pgfpathcurveto{\pgfqpoint{2.697547in}{2.026333in}}{\pgfqpoint{2.701937in}{2.015734in}}{\pgfqpoint{2.709751in}{2.007920in}}%
\pgfpathcurveto{\pgfqpoint{2.717564in}{2.000106in}}{\pgfqpoint{2.728163in}{1.995716in}}{\pgfqpoint{2.739213in}{1.995716in}}%
\pgfpathclose%
\pgfusepath{stroke,fill}%
\end{pgfscope}%
\begin{pgfscope}%
\pgfpathrectangle{\pgfqpoint{0.600000in}{0.600000in}}{\pgfqpoint{3.900000in}{3.900000in}}%
\pgfusepath{clip}%
\pgfsetbuttcap%
\pgfsetroundjoin%
\definecolor{currentfill}{rgb}{0.121569,0.466667,0.705882}%
\pgfsetfillcolor{currentfill}%
\pgfsetlinewidth{1.003750pt}%
\definecolor{currentstroke}{rgb}{0.121569,0.466667,0.705882}%
\pgfsetstrokecolor{currentstroke}%
\pgfsetdash{}{0pt}%
\pgfpathmoveto{\pgfqpoint{2.439925in}{2.999566in}}%
\pgfpathcurveto{\pgfqpoint{2.450975in}{2.999566in}}{\pgfqpoint{2.461574in}{3.003956in}}{\pgfqpoint{2.469387in}{3.011770in}}%
\pgfpathcurveto{\pgfqpoint{2.477201in}{3.019583in}}{\pgfqpoint{2.481591in}{3.030182in}}{\pgfqpoint{2.481591in}{3.041232in}}%
\pgfpathcurveto{\pgfqpoint{2.481591in}{3.052282in}}{\pgfqpoint{2.477201in}{3.062881in}}{\pgfqpoint{2.469387in}{3.070695in}}%
\pgfpathcurveto{\pgfqpoint{2.461574in}{3.078509in}}{\pgfqpoint{2.450975in}{3.082899in}}{\pgfqpoint{2.439925in}{3.082899in}}%
\pgfpathcurveto{\pgfqpoint{2.428874in}{3.082899in}}{\pgfqpoint{2.418275in}{3.078509in}}{\pgfqpoint{2.410462in}{3.070695in}}%
\pgfpathcurveto{\pgfqpoint{2.402648in}{3.062881in}}{\pgfqpoint{2.398258in}{3.052282in}}{\pgfqpoint{2.398258in}{3.041232in}}%
\pgfpathcurveto{\pgfqpoint{2.398258in}{3.030182in}}{\pgfqpoint{2.402648in}{3.019583in}}{\pgfqpoint{2.410462in}{3.011770in}}%
\pgfpathcurveto{\pgfqpoint{2.418275in}{3.003956in}}{\pgfqpoint{2.428874in}{2.999566in}}{\pgfqpoint{2.439925in}{2.999566in}}%
\pgfpathclose%
\pgfusepath{stroke,fill}%
\end{pgfscope}%
\begin{pgfscope}%
\pgfpathrectangle{\pgfqpoint{0.600000in}{0.600000in}}{\pgfqpoint{3.900000in}{3.900000in}}%
\pgfusepath{clip}%
\pgfsetbuttcap%
\pgfsetroundjoin%
\definecolor{currentfill}{rgb}{0.121569,0.466667,0.705882}%
\pgfsetfillcolor{currentfill}%
\pgfsetlinewidth{1.003750pt}%
\definecolor{currentstroke}{rgb}{0.121569,0.466667,0.705882}%
\pgfsetstrokecolor{currentstroke}%
\pgfsetdash{}{0pt}%
\pgfpathmoveto{\pgfqpoint{2.653771in}{1.941184in}}%
\pgfpathcurveto{\pgfqpoint{2.664821in}{1.941184in}}{\pgfqpoint{2.675420in}{1.945574in}}{\pgfqpoint{2.683234in}{1.953387in}}%
\pgfpathcurveto{\pgfqpoint{2.691047in}{1.961201in}}{\pgfqpoint{2.695438in}{1.971800in}}{\pgfqpoint{2.695438in}{1.982850in}}%
\pgfpathcurveto{\pgfqpoint{2.695438in}{1.993900in}}{\pgfqpoint{2.691047in}{2.004499in}}{\pgfqpoint{2.683234in}{2.012313in}}%
\pgfpathcurveto{\pgfqpoint{2.675420in}{2.020127in}}{\pgfqpoint{2.664821in}{2.024517in}}{\pgfqpoint{2.653771in}{2.024517in}}%
\pgfpathcurveto{\pgfqpoint{2.642721in}{2.024517in}}{\pgfqpoint{2.632122in}{2.020127in}}{\pgfqpoint{2.624308in}{2.012313in}}%
\pgfpathcurveto{\pgfqpoint{2.616494in}{2.004499in}}{\pgfqpoint{2.612104in}{1.993900in}}{\pgfqpoint{2.612104in}{1.982850in}}%
\pgfpathcurveto{\pgfqpoint{2.612104in}{1.971800in}}{\pgfqpoint{2.616494in}{1.961201in}}{\pgfqpoint{2.624308in}{1.953387in}}%
\pgfpathcurveto{\pgfqpoint{2.632122in}{1.945574in}}{\pgfqpoint{2.642721in}{1.941184in}}{\pgfqpoint{2.653771in}{1.941184in}}%
\pgfpathclose%
\pgfusepath{stroke,fill}%
\end{pgfscope}%
\begin{pgfscope}%
\pgfpathrectangle{\pgfqpoint{0.600000in}{0.600000in}}{\pgfqpoint{3.900000in}{3.900000in}}%
\pgfusepath{clip}%
\pgfsetbuttcap%
\pgfsetroundjoin%
\definecolor{currentfill}{rgb}{0.121569,0.466667,0.705882}%
\pgfsetfillcolor{currentfill}%
\pgfsetlinewidth{1.003750pt}%
\definecolor{currentstroke}{rgb}{0.121569,0.466667,0.705882}%
\pgfsetstrokecolor{currentstroke}%
\pgfsetdash{}{0pt}%
\pgfpathmoveto{\pgfqpoint{2.730545in}{1.366387in}}%
\pgfpathcurveto{\pgfqpoint{2.741595in}{1.366387in}}{\pgfqpoint{2.752194in}{1.370777in}}{\pgfqpoint{2.760008in}{1.378590in}}%
\pgfpathcurveto{\pgfqpoint{2.767822in}{1.386404in}}{\pgfqpoint{2.772212in}{1.397003in}}{\pgfqpoint{2.772212in}{1.408053in}}%
\pgfpathcurveto{\pgfqpoint{2.772212in}{1.419103in}}{\pgfqpoint{2.767822in}{1.429702in}}{\pgfqpoint{2.760008in}{1.437516in}}%
\pgfpathcurveto{\pgfqpoint{2.752194in}{1.445330in}}{\pgfqpoint{2.741595in}{1.449720in}}{\pgfqpoint{2.730545in}{1.449720in}}%
\pgfpathcurveto{\pgfqpoint{2.719495in}{1.449720in}}{\pgfqpoint{2.708896in}{1.445330in}}{\pgfqpoint{2.701083in}{1.437516in}}%
\pgfpathcurveto{\pgfqpoint{2.693269in}{1.429702in}}{\pgfqpoint{2.688879in}{1.419103in}}{\pgfqpoint{2.688879in}{1.408053in}}%
\pgfpathcurveto{\pgfqpoint{2.688879in}{1.397003in}}{\pgfqpoint{2.693269in}{1.386404in}}{\pgfqpoint{2.701083in}{1.378590in}}%
\pgfpathcurveto{\pgfqpoint{2.708896in}{1.370777in}}{\pgfqpoint{2.719495in}{1.366387in}}{\pgfqpoint{2.730545in}{1.366387in}}%
\pgfpathclose%
\pgfusepath{stroke,fill}%
\end{pgfscope}%
\begin{pgfscope}%
\pgfpathrectangle{\pgfqpoint{0.600000in}{0.600000in}}{\pgfqpoint{3.900000in}{3.900000in}}%
\pgfusepath{clip}%
\pgfsetbuttcap%
\pgfsetroundjoin%
\definecolor{currentfill}{rgb}{0.121569,0.466667,0.705882}%
\pgfsetfillcolor{currentfill}%
\pgfsetlinewidth{1.003750pt}%
\definecolor{currentstroke}{rgb}{0.121569,0.466667,0.705882}%
\pgfsetstrokecolor{currentstroke}%
\pgfsetdash{}{0pt}%
\pgfpathmoveto{\pgfqpoint{2.909785in}{1.729495in}}%
\pgfpathcurveto{\pgfqpoint{2.920836in}{1.729495in}}{\pgfqpoint{2.931435in}{1.733886in}}{\pgfqpoint{2.939248in}{1.741699in}}%
\pgfpathcurveto{\pgfqpoint{2.947062in}{1.749513in}}{\pgfqpoint{2.951452in}{1.760112in}}{\pgfqpoint{2.951452in}{1.771162in}}%
\pgfpathcurveto{\pgfqpoint{2.951452in}{1.782212in}}{\pgfqpoint{2.947062in}{1.792811in}}{\pgfqpoint{2.939248in}{1.800625in}}%
\pgfpathcurveto{\pgfqpoint{2.931435in}{1.808438in}}{\pgfqpoint{2.920836in}{1.812829in}}{\pgfqpoint{2.909785in}{1.812829in}}%
\pgfpathcurveto{\pgfqpoint{2.898735in}{1.812829in}}{\pgfqpoint{2.888136in}{1.808438in}}{\pgfqpoint{2.880323in}{1.800625in}}%
\pgfpathcurveto{\pgfqpoint{2.872509in}{1.792811in}}{\pgfqpoint{2.868119in}{1.782212in}}{\pgfqpoint{2.868119in}{1.771162in}}%
\pgfpathcurveto{\pgfqpoint{2.868119in}{1.760112in}}{\pgfqpoint{2.872509in}{1.749513in}}{\pgfqpoint{2.880323in}{1.741699in}}%
\pgfpathcurveto{\pgfqpoint{2.888136in}{1.733886in}}{\pgfqpoint{2.898735in}{1.729495in}}{\pgfqpoint{2.909785in}{1.729495in}}%
\pgfpathclose%
\pgfusepath{stroke,fill}%
\end{pgfscope}%
\begin{pgfscope}%
\pgfpathrectangle{\pgfqpoint{0.600000in}{0.600000in}}{\pgfqpoint{3.900000in}{3.900000in}}%
\pgfusepath{clip}%
\pgfsetbuttcap%
\pgfsetroundjoin%
\definecolor{currentfill}{rgb}{0.121569,0.466667,0.705882}%
\pgfsetfillcolor{currentfill}%
\pgfsetlinewidth{1.003750pt}%
\definecolor{currentstroke}{rgb}{0.121569,0.466667,0.705882}%
\pgfsetstrokecolor{currentstroke}%
\pgfsetdash{}{0pt}%
\pgfpathmoveto{\pgfqpoint{3.210209in}{2.871674in}}%
\pgfpathcurveto{\pgfqpoint{3.221259in}{2.871674in}}{\pgfqpoint{3.231858in}{2.876064in}}{\pgfqpoint{3.239671in}{2.883878in}}%
\pgfpathcurveto{\pgfqpoint{3.247485in}{2.891691in}}{\pgfqpoint{3.251875in}{2.902290in}}{\pgfqpoint{3.251875in}{2.913340in}}%
\pgfpathcurveto{\pgfqpoint{3.251875in}{2.924391in}}{\pgfqpoint{3.247485in}{2.934990in}}{\pgfqpoint{3.239671in}{2.942803in}}%
\pgfpathcurveto{\pgfqpoint{3.231858in}{2.950617in}}{\pgfqpoint{3.221259in}{2.955007in}}{\pgfqpoint{3.210209in}{2.955007in}}%
\pgfpathcurveto{\pgfqpoint{3.199158in}{2.955007in}}{\pgfqpoint{3.188559in}{2.950617in}}{\pgfqpoint{3.180746in}{2.942803in}}%
\pgfpathcurveto{\pgfqpoint{3.172932in}{2.934990in}}{\pgfqpoint{3.168542in}{2.924391in}}{\pgfqpoint{3.168542in}{2.913340in}}%
\pgfpathcurveto{\pgfqpoint{3.168542in}{2.902290in}}{\pgfqpoint{3.172932in}{2.891691in}}{\pgfqpoint{3.180746in}{2.883878in}}%
\pgfpathcurveto{\pgfqpoint{3.188559in}{2.876064in}}{\pgfqpoint{3.199158in}{2.871674in}}{\pgfqpoint{3.210209in}{2.871674in}}%
\pgfpathclose%
\pgfusepath{stroke,fill}%
\end{pgfscope}%
\begin{pgfscope}%
\pgfpathrectangle{\pgfqpoint{0.600000in}{0.600000in}}{\pgfqpoint{3.900000in}{3.900000in}}%
\pgfusepath{clip}%
\pgfsetbuttcap%
\pgfsetroundjoin%
\definecolor{currentfill}{rgb}{0.121569,0.466667,0.705882}%
\pgfsetfillcolor{currentfill}%
\pgfsetlinewidth{1.003750pt}%
\definecolor{currentstroke}{rgb}{0.121569,0.466667,0.705882}%
\pgfsetstrokecolor{currentstroke}%
\pgfsetdash{}{0pt}%
\pgfpathmoveto{\pgfqpoint{1.830461in}{3.124373in}}%
\pgfpathcurveto{\pgfqpoint{1.841511in}{3.124373in}}{\pgfqpoint{1.852110in}{3.128763in}}{\pgfqpoint{1.859923in}{3.136577in}}%
\pgfpathcurveto{\pgfqpoint{1.867737in}{3.144390in}}{\pgfqpoint{1.872127in}{3.154989in}}{\pgfqpoint{1.872127in}{3.166039in}}%
\pgfpathcurveto{\pgfqpoint{1.872127in}{3.177090in}}{\pgfqpoint{1.867737in}{3.187689in}}{\pgfqpoint{1.859923in}{3.195502in}}%
\pgfpathcurveto{\pgfqpoint{1.852110in}{3.203316in}}{\pgfqpoint{1.841511in}{3.207706in}}{\pgfqpoint{1.830461in}{3.207706in}}%
\pgfpathcurveto{\pgfqpoint{1.819410in}{3.207706in}}{\pgfqpoint{1.808811in}{3.203316in}}{\pgfqpoint{1.800998in}{3.195502in}}%
\pgfpathcurveto{\pgfqpoint{1.793184in}{3.187689in}}{\pgfqpoint{1.788794in}{3.177090in}}{\pgfqpoint{1.788794in}{3.166039in}}%
\pgfpathcurveto{\pgfqpoint{1.788794in}{3.154989in}}{\pgfqpoint{1.793184in}{3.144390in}}{\pgfqpoint{1.800998in}{3.136577in}}%
\pgfpathcurveto{\pgfqpoint{1.808811in}{3.128763in}}{\pgfqpoint{1.819410in}{3.124373in}}{\pgfqpoint{1.830461in}{3.124373in}}%
\pgfpathclose%
\pgfusepath{stroke,fill}%
\end{pgfscope}%
\begin{pgfscope}%
\pgfpathrectangle{\pgfqpoint{0.600000in}{0.600000in}}{\pgfqpoint{3.900000in}{3.900000in}}%
\pgfusepath{clip}%
\pgfsetbuttcap%
\pgfsetroundjoin%
\definecolor{currentfill}{rgb}{0.121569,0.466667,0.705882}%
\pgfsetfillcolor{currentfill}%
\pgfsetlinewidth{1.003750pt}%
\definecolor{currentstroke}{rgb}{0.121569,0.466667,0.705882}%
\pgfsetstrokecolor{currentstroke}%
\pgfsetdash{}{0pt}%
\pgfpathmoveto{\pgfqpoint{2.894752in}{2.108166in}}%
\pgfpathcurveto{\pgfqpoint{2.905802in}{2.108166in}}{\pgfqpoint{2.916401in}{2.112557in}}{\pgfqpoint{2.924214in}{2.120370in}}%
\pgfpathcurveto{\pgfqpoint{2.932028in}{2.128184in}}{\pgfqpoint{2.936418in}{2.138783in}}{\pgfqpoint{2.936418in}{2.149833in}}%
\pgfpathcurveto{\pgfqpoint{2.936418in}{2.160883in}}{\pgfqpoint{2.932028in}{2.171482in}}{\pgfqpoint{2.924214in}{2.179296in}}%
\pgfpathcurveto{\pgfqpoint{2.916401in}{2.187110in}}{\pgfqpoint{2.905802in}{2.191500in}}{\pgfqpoint{2.894752in}{2.191500in}}%
\pgfpathcurveto{\pgfqpoint{2.883701in}{2.191500in}}{\pgfqpoint{2.873102in}{2.187110in}}{\pgfqpoint{2.865289in}{2.179296in}}%
\pgfpathcurveto{\pgfqpoint{2.857475in}{2.171482in}}{\pgfqpoint{2.853085in}{2.160883in}}{\pgfqpoint{2.853085in}{2.149833in}}%
\pgfpathcurveto{\pgfqpoint{2.853085in}{2.138783in}}{\pgfqpoint{2.857475in}{2.128184in}}{\pgfqpoint{2.865289in}{2.120370in}}%
\pgfpathcurveto{\pgfqpoint{2.873102in}{2.112557in}}{\pgfqpoint{2.883701in}{2.108166in}}{\pgfqpoint{2.894752in}{2.108166in}}%
\pgfpathclose%
\pgfusepath{stroke,fill}%
\end{pgfscope}%
\begin{pgfscope}%
\pgfpathrectangle{\pgfqpoint{0.600000in}{0.600000in}}{\pgfqpoint{3.900000in}{3.900000in}}%
\pgfusepath{clip}%
\pgfsetbuttcap%
\pgfsetroundjoin%
\definecolor{currentfill}{rgb}{0.121569,0.466667,0.705882}%
\pgfsetfillcolor{currentfill}%
\pgfsetlinewidth{1.003750pt}%
\definecolor{currentstroke}{rgb}{0.121569,0.466667,0.705882}%
\pgfsetstrokecolor{currentstroke}%
\pgfsetdash{}{0pt}%
\pgfpathmoveto{\pgfqpoint{2.687357in}{3.347919in}}%
\pgfpathcurveto{\pgfqpoint{2.698407in}{3.347919in}}{\pgfqpoint{2.709006in}{3.352309in}}{\pgfqpoint{2.716820in}{3.360122in}}%
\pgfpathcurveto{\pgfqpoint{2.724633in}{3.367936in}}{\pgfqpoint{2.729024in}{3.378535in}}{\pgfqpoint{2.729024in}{3.389585in}}%
\pgfpathcurveto{\pgfqpoint{2.729024in}{3.400635in}}{\pgfqpoint{2.724633in}{3.411234in}}{\pgfqpoint{2.716820in}{3.419048in}}%
\pgfpathcurveto{\pgfqpoint{2.709006in}{3.426862in}}{\pgfqpoint{2.698407in}{3.431252in}}{\pgfqpoint{2.687357in}{3.431252in}}%
\pgfpathcurveto{\pgfqpoint{2.676307in}{3.431252in}}{\pgfqpoint{2.665708in}{3.426862in}}{\pgfqpoint{2.657894in}{3.419048in}}%
\pgfpathcurveto{\pgfqpoint{2.650081in}{3.411234in}}{\pgfqpoint{2.645690in}{3.400635in}}{\pgfqpoint{2.645690in}{3.389585in}}%
\pgfpathcurveto{\pgfqpoint{2.645690in}{3.378535in}}{\pgfqpoint{2.650081in}{3.367936in}}{\pgfqpoint{2.657894in}{3.360122in}}%
\pgfpathcurveto{\pgfqpoint{2.665708in}{3.352309in}}{\pgfqpoint{2.676307in}{3.347919in}}{\pgfqpoint{2.687357in}{3.347919in}}%
\pgfpathclose%
\pgfusepath{stroke,fill}%
\end{pgfscope}%
\begin{pgfscope}%
\pgfpathrectangle{\pgfqpoint{0.600000in}{0.600000in}}{\pgfqpoint{3.900000in}{3.900000in}}%
\pgfusepath{clip}%
\pgfsetbuttcap%
\pgfsetroundjoin%
\definecolor{currentfill}{rgb}{0.121569,0.466667,0.705882}%
\pgfsetfillcolor{currentfill}%
\pgfsetlinewidth{1.003750pt}%
\definecolor{currentstroke}{rgb}{0.121569,0.466667,0.705882}%
\pgfsetstrokecolor{currentstroke}%
\pgfsetdash{}{0pt}%
\pgfpathmoveto{\pgfqpoint{1.855357in}{2.148693in}}%
\pgfpathcurveto{\pgfqpoint{1.866407in}{2.148693in}}{\pgfqpoint{1.877006in}{2.153084in}}{\pgfqpoint{1.884820in}{2.160897in}}%
\pgfpathcurveto{\pgfqpoint{1.892633in}{2.168711in}}{\pgfqpoint{1.897023in}{2.179310in}}{\pgfqpoint{1.897023in}{2.190360in}}%
\pgfpathcurveto{\pgfqpoint{1.897023in}{2.201410in}}{\pgfqpoint{1.892633in}{2.212009in}}{\pgfqpoint{1.884820in}{2.219823in}}%
\pgfpathcurveto{\pgfqpoint{1.877006in}{2.227637in}}{\pgfqpoint{1.866407in}{2.232027in}}{\pgfqpoint{1.855357in}{2.232027in}}%
\pgfpathcurveto{\pgfqpoint{1.844307in}{2.232027in}}{\pgfqpoint{1.833708in}{2.227637in}}{\pgfqpoint{1.825894in}{2.219823in}}%
\pgfpathcurveto{\pgfqpoint{1.818080in}{2.212009in}}{\pgfqpoint{1.813690in}{2.201410in}}{\pgfqpoint{1.813690in}{2.190360in}}%
\pgfpathcurveto{\pgfqpoint{1.813690in}{2.179310in}}{\pgfqpoint{1.818080in}{2.168711in}}{\pgfqpoint{1.825894in}{2.160897in}}%
\pgfpathcurveto{\pgfqpoint{1.833708in}{2.153084in}}{\pgfqpoint{1.844307in}{2.148693in}}{\pgfqpoint{1.855357in}{2.148693in}}%
\pgfpathclose%
\pgfusepath{stroke,fill}%
\end{pgfscope}%
\begin{pgfscope}%
\pgfpathrectangle{\pgfqpoint{0.600000in}{0.600000in}}{\pgfqpoint{3.900000in}{3.900000in}}%
\pgfusepath{clip}%
\pgfsetbuttcap%
\pgfsetroundjoin%
\definecolor{currentfill}{rgb}{0.121569,0.466667,0.705882}%
\pgfsetfillcolor{currentfill}%
\pgfsetlinewidth{1.003750pt}%
\definecolor{currentstroke}{rgb}{0.121569,0.466667,0.705882}%
\pgfsetstrokecolor{currentstroke}%
\pgfsetdash{}{0pt}%
\pgfpathmoveto{\pgfqpoint{2.510540in}{2.329578in}}%
\pgfpathcurveto{\pgfqpoint{2.521590in}{2.329578in}}{\pgfqpoint{2.532189in}{2.333968in}}{\pgfqpoint{2.540003in}{2.341782in}}%
\pgfpathcurveto{\pgfqpoint{2.547816in}{2.349596in}}{\pgfqpoint{2.552207in}{2.360195in}}{\pgfqpoint{2.552207in}{2.371245in}}%
\pgfpathcurveto{\pgfqpoint{2.552207in}{2.382295in}}{\pgfqpoint{2.547816in}{2.392894in}}{\pgfqpoint{2.540003in}{2.400708in}}%
\pgfpathcurveto{\pgfqpoint{2.532189in}{2.408521in}}{\pgfqpoint{2.521590in}{2.412911in}}{\pgfqpoint{2.510540in}{2.412911in}}%
\pgfpathcurveto{\pgfqpoint{2.499490in}{2.412911in}}{\pgfqpoint{2.488891in}{2.408521in}}{\pgfqpoint{2.481077in}{2.400708in}}%
\pgfpathcurveto{\pgfqpoint{2.473264in}{2.392894in}}{\pgfqpoint{2.468873in}{2.382295in}}{\pgfqpoint{2.468873in}{2.371245in}}%
\pgfpathcurveto{\pgfqpoint{2.468873in}{2.360195in}}{\pgfqpoint{2.473264in}{2.349596in}}{\pgfqpoint{2.481077in}{2.341782in}}%
\pgfpathcurveto{\pgfqpoint{2.488891in}{2.333968in}}{\pgfqpoint{2.499490in}{2.329578in}}{\pgfqpoint{2.510540in}{2.329578in}}%
\pgfpathclose%
\pgfusepath{stroke,fill}%
\end{pgfscope}%
\begin{pgfscope}%
\pgfpathrectangle{\pgfqpoint{0.600000in}{0.600000in}}{\pgfqpoint{3.900000in}{3.900000in}}%
\pgfusepath{clip}%
\pgfsetbuttcap%
\pgfsetroundjoin%
\definecolor{currentfill}{rgb}{0.121569,0.466667,0.705882}%
\pgfsetfillcolor{currentfill}%
\pgfsetlinewidth{1.003750pt}%
\definecolor{currentstroke}{rgb}{0.121569,0.466667,0.705882}%
\pgfsetstrokecolor{currentstroke}%
\pgfsetdash{}{0pt}%
\pgfpathmoveto{\pgfqpoint{3.273955in}{2.941813in}}%
\pgfpathcurveto{\pgfqpoint{3.285006in}{2.941813in}}{\pgfqpoint{3.295605in}{2.946203in}}{\pgfqpoint{3.303418in}{2.954017in}}%
\pgfpathcurveto{\pgfqpoint{3.311232in}{2.961830in}}{\pgfqpoint{3.315622in}{2.972429in}}{\pgfqpoint{3.315622in}{2.983480in}}%
\pgfpathcurveto{\pgfqpoint{3.315622in}{2.994530in}}{\pgfqpoint{3.311232in}{3.005129in}}{\pgfqpoint{3.303418in}{3.012942in}}%
\pgfpathcurveto{\pgfqpoint{3.295605in}{3.020756in}}{\pgfqpoint{3.285006in}{3.025146in}}{\pgfqpoint{3.273955in}{3.025146in}}%
\pgfpathcurveto{\pgfqpoint{3.262905in}{3.025146in}}{\pgfqpoint{3.252306in}{3.020756in}}{\pgfqpoint{3.244493in}{3.012942in}}%
\pgfpathcurveto{\pgfqpoint{3.236679in}{3.005129in}}{\pgfqpoint{3.232289in}{2.994530in}}{\pgfqpoint{3.232289in}{2.983480in}}%
\pgfpathcurveto{\pgfqpoint{3.232289in}{2.972429in}}{\pgfqpoint{3.236679in}{2.961830in}}{\pgfqpoint{3.244493in}{2.954017in}}%
\pgfpathcurveto{\pgfqpoint{3.252306in}{2.946203in}}{\pgfqpoint{3.262905in}{2.941813in}}{\pgfqpoint{3.273955in}{2.941813in}}%
\pgfpathclose%
\pgfusepath{stroke,fill}%
\end{pgfscope}%
\begin{pgfscope}%
\pgfpathrectangle{\pgfqpoint{0.600000in}{0.600000in}}{\pgfqpoint{3.900000in}{3.900000in}}%
\pgfusepath{clip}%
\pgfsetbuttcap%
\pgfsetroundjoin%
\definecolor{currentfill}{rgb}{0.121569,0.466667,0.705882}%
\pgfsetfillcolor{currentfill}%
\pgfsetlinewidth{1.003750pt}%
\definecolor{currentstroke}{rgb}{0.121569,0.466667,0.705882}%
\pgfsetstrokecolor{currentstroke}%
\pgfsetdash{}{0pt}%
\pgfpathmoveto{\pgfqpoint{2.440335in}{2.496976in}}%
\pgfpathcurveto{\pgfqpoint{2.451385in}{2.496976in}}{\pgfqpoint{2.461984in}{2.501366in}}{\pgfqpoint{2.469797in}{2.509179in}}%
\pgfpathcurveto{\pgfqpoint{2.477611in}{2.516993in}}{\pgfqpoint{2.482001in}{2.527592in}}{\pgfqpoint{2.482001in}{2.538642in}}%
\pgfpathcurveto{\pgfqpoint{2.482001in}{2.549692in}}{\pgfqpoint{2.477611in}{2.560291in}}{\pgfqpoint{2.469797in}{2.568105in}}%
\pgfpathcurveto{\pgfqpoint{2.461984in}{2.575919in}}{\pgfqpoint{2.451385in}{2.580309in}}{\pgfqpoint{2.440335in}{2.580309in}}%
\pgfpathcurveto{\pgfqpoint{2.429285in}{2.580309in}}{\pgfqpoint{2.418686in}{2.575919in}}{\pgfqpoint{2.410872in}{2.568105in}}%
\pgfpathcurveto{\pgfqpoint{2.403058in}{2.560291in}}{\pgfqpoint{2.398668in}{2.549692in}}{\pgfqpoint{2.398668in}{2.538642in}}%
\pgfpathcurveto{\pgfqpoint{2.398668in}{2.527592in}}{\pgfqpoint{2.403058in}{2.516993in}}{\pgfqpoint{2.410872in}{2.509179in}}%
\pgfpathcurveto{\pgfqpoint{2.418686in}{2.501366in}}{\pgfqpoint{2.429285in}{2.496976in}}{\pgfqpoint{2.440335in}{2.496976in}}%
\pgfpathclose%
\pgfusepath{stroke,fill}%
\end{pgfscope}%
\begin{pgfscope}%
\pgfpathrectangle{\pgfqpoint{0.600000in}{0.600000in}}{\pgfqpoint{3.900000in}{3.900000in}}%
\pgfusepath{clip}%
\pgfsetbuttcap%
\pgfsetroundjoin%
\definecolor{currentfill}{rgb}{0.121569,0.466667,0.705882}%
\pgfsetfillcolor{currentfill}%
\pgfsetlinewidth{1.003750pt}%
\definecolor{currentstroke}{rgb}{0.121569,0.466667,0.705882}%
\pgfsetstrokecolor{currentstroke}%
\pgfsetdash{}{0pt}%
\pgfpathmoveto{\pgfqpoint{1.829193in}{2.686688in}}%
\pgfpathcurveto{\pgfqpoint{1.840243in}{2.686688in}}{\pgfqpoint{1.850843in}{2.691078in}}{\pgfqpoint{1.858656in}{2.698891in}}%
\pgfpathcurveto{\pgfqpoint{1.866470in}{2.706705in}}{\pgfqpoint{1.870860in}{2.717304in}}{\pgfqpoint{1.870860in}{2.728354in}}%
\pgfpathcurveto{\pgfqpoint{1.870860in}{2.739404in}}{\pgfqpoint{1.866470in}{2.750003in}}{\pgfqpoint{1.858656in}{2.757817in}}%
\pgfpathcurveto{\pgfqpoint{1.850843in}{2.765631in}}{\pgfqpoint{1.840243in}{2.770021in}}{\pgfqpoint{1.829193in}{2.770021in}}%
\pgfpathcurveto{\pgfqpoint{1.818143in}{2.770021in}}{\pgfqpoint{1.807544in}{2.765631in}}{\pgfqpoint{1.799731in}{2.757817in}}%
\pgfpathcurveto{\pgfqpoint{1.791917in}{2.750003in}}{\pgfqpoint{1.787527in}{2.739404in}}{\pgfqpoint{1.787527in}{2.728354in}}%
\pgfpathcurveto{\pgfqpoint{1.787527in}{2.717304in}}{\pgfqpoint{1.791917in}{2.706705in}}{\pgfqpoint{1.799731in}{2.698891in}}%
\pgfpathcurveto{\pgfqpoint{1.807544in}{2.691078in}}{\pgfqpoint{1.818143in}{2.686688in}}{\pgfqpoint{1.829193in}{2.686688in}}%
\pgfpathclose%
\pgfusepath{stroke,fill}%
\end{pgfscope}%
\begin{pgfscope}%
\pgfpathrectangle{\pgfqpoint{0.600000in}{0.600000in}}{\pgfqpoint{3.900000in}{3.900000in}}%
\pgfusepath{clip}%
\pgfsetbuttcap%
\pgfsetroundjoin%
\definecolor{currentfill}{rgb}{0.121569,0.466667,0.705882}%
\pgfsetfillcolor{currentfill}%
\pgfsetlinewidth{1.003750pt}%
\definecolor{currentstroke}{rgb}{0.121569,0.466667,0.705882}%
\pgfsetstrokecolor{currentstroke}%
\pgfsetdash{}{0pt}%
\pgfpathmoveto{\pgfqpoint{2.766936in}{1.802179in}}%
\pgfpathcurveto{\pgfqpoint{2.777987in}{1.802179in}}{\pgfqpoint{2.788586in}{1.806569in}}{\pgfqpoint{2.796399in}{1.814383in}}%
\pgfpathcurveto{\pgfqpoint{2.804213in}{1.822196in}}{\pgfqpoint{2.808603in}{1.832795in}}{\pgfqpoint{2.808603in}{1.843845in}}%
\pgfpathcurveto{\pgfqpoint{2.808603in}{1.854896in}}{\pgfqpoint{2.804213in}{1.865495in}}{\pgfqpoint{2.796399in}{1.873308in}}%
\pgfpathcurveto{\pgfqpoint{2.788586in}{1.881122in}}{\pgfqpoint{2.777987in}{1.885512in}}{\pgfqpoint{2.766936in}{1.885512in}}%
\pgfpathcurveto{\pgfqpoint{2.755886in}{1.885512in}}{\pgfqpoint{2.745287in}{1.881122in}}{\pgfqpoint{2.737474in}{1.873308in}}%
\pgfpathcurveto{\pgfqpoint{2.729660in}{1.865495in}}{\pgfqpoint{2.725270in}{1.854896in}}{\pgfqpoint{2.725270in}{1.843845in}}%
\pgfpathcurveto{\pgfqpoint{2.725270in}{1.832795in}}{\pgfqpoint{2.729660in}{1.822196in}}{\pgfqpoint{2.737474in}{1.814383in}}%
\pgfpathcurveto{\pgfqpoint{2.745287in}{1.806569in}}{\pgfqpoint{2.755886in}{1.802179in}}{\pgfqpoint{2.766936in}{1.802179in}}%
\pgfpathclose%
\pgfusepath{stroke,fill}%
\end{pgfscope}%
\begin{pgfscope}%
\pgfpathrectangle{\pgfqpoint{0.600000in}{0.600000in}}{\pgfqpoint{3.900000in}{3.900000in}}%
\pgfusepath{clip}%
\pgfsetbuttcap%
\pgfsetroundjoin%
\definecolor{currentfill}{rgb}{0.121569,0.466667,0.705882}%
\pgfsetfillcolor{currentfill}%
\pgfsetlinewidth{1.003750pt}%
\definecolor{currentstroke}{rgb}{0.121569,0.466667,0.705882}%
\pgfsetstrokecolor{currentstroke}%
\pgfsetdash{}{0pt}%
\pgfpathmoveto{\pgfqpoint{1.934773in}{1.865058in}}%
\pgfpathcurveto{\pgfqpoint{1.945823in}{1.865058in}}{\pgfqpoint{1.956422in}{1.869448in}}{\pgfqpoint{1.964236in}{1.877262in}}%
\pgfpathcurveto{\pgfqpoint{1.972050in}{1.885076in}}{\pgfqpoint{1.976440in}{1.895675in}}{\pgfqpoint{1.976440in}{1.906725in}}%
\pgfpathcurveto{\pgfqpoint{1.976440in}{1.917775in}}{\pgfqpoint{1.972050in}{1.928374in}}{\pgfqpoint{1.964236in}{1.936188in}}%
\pgfpathcurveto{\pgfqpoint{1.956422in}{1.944001in}}{\pgfqpoint{1.945823in}{1.948392in}}{\pgfqpoint{1.934773in}{1.948392in}}%
\pgfpathcurveto{\pgfqpoint{1.923723in}{1.948392in}}{\pgfqpoint{1.913124in}{1.944001in}}{\pgfqpoint{1.905311in}{1.936188in}}%
\pgfpathcurveto{\pgfqpoint{1.897497in}{1.928374in}}{\pgfqpoint{1.893107in}{1.917775in}}{\pgfqpoint{1.893107in}{1.906725in}}%
\pgfpathcurveto{\pgfqpoint{1.893107in}{1.895675in}}{\pgfqpoint{1.897497in}{1.885076in}}{\pgfqpoint{1.905311in}{1.877262in}}%
\pgfpathcurveto{\pgfqpoint{1.913124in}{1.869448in}}{\pgfqpoint{1.923723in}{1.865058in}}{\pgfqpoint{1.934773in}{1.865058in}}%
\pgfpathclose%
\pgfusepath{stroke,fill}%
\end{pgfscope}%
\begin{pgfscope}%
\pgfpathrectangle{\pgfqpoint{0.600000in}{0.600000in}}{\pgfqpoint{3.900000in}{3.900000in}}%
\pgfusepath{clip}%
\pgfsetbuttcap%
\pgfsetroundjoin%
\definecolor{currentfill}{rgb}{0.121569,0.466667,0.705882}%
\pgfsetfillcolor{currentfill}%
\pgfsetlinewidth{1.003750pt}%
\definecolor{currentstroke}{rgb}{0.121569,0.466667,0.705882}%
\pgfsetstrokecolor{currentstroke}%
\pgfsetdash{}{0pt}%
\pgfpathmoveto{\pgfqpoint{1.588850in}{2.690377in}}%
\pgfpathcurveto{\pgfqpoint{1.599900in}{2.690377in}}{\pgfqpoint{1.610499in}{2.694768in}}{\pgfqpoint{1.618313in}{2.702581in}}%
\pgfpathcurveto{\pgfqpoint{1.626127in}{2.710395in}}{\pgfqpoint{1.630517in}{2.720994in}}{\pgfqpoint{1.630517in}{2.732044in}}%
\pgfpathcurveto{\pgfqpoint{1.630517in}{2.743094in}}{\pgfqpoint{1.626127in}{2.753693in}}{\pgfqpoint{1.618313in}{2.761507in}}%
\pgfpathcurveto{\pgfqpoint{1.610499in}{2.769321in}}{\pgfqpoint{1.599900in}{2.773711in}}{\pgfqpoint{1.588850in}{2.773711in}}%
\pgfpathcurveto{\pgfqpoint{1.577800in}{2.773711in}}{\pgfqpoint{1.567201in}{2.769321in}}{\pgfqpoint{1.559387in}{2.761507in}}%
\pgfpathcurveto{\pgfqpoint{1.551574in}{2.753693in}}{\pgfqpoint{1.547184in}{2.743094in}}{\pgfqpoint{1.547184in}{2.732044in}}%
\pgfpathcurveto{\pgfqpoint{1.547184in}{2.720994in}}{\pgfqpoint{1.551574in}{2.710395in}}{\pgfqpoint{1.559387in}{2.702581in}}%
\pgfpathcurveto{\pgfqpoint{1.567201in}{2.694768in}}{\pgfqpoint{1.577800in}{2.690377in}}{\pgfqpoint{1.588850in}{2.690377in}}%
\pgfpathclose%
\pgfusepath{stroke,fill}%
\end{pgfscope}%
\begin{pgfscope}%
\pgfpathrectangle{\pgfqpoint{0.600000in}{0.600000in}}{\pgfqpoint{3.900000in}{3.900000in}}%
\pgfusepath{clip}%
\pgfsetbuttcap%
\pgfsetroundjoin%
\definecolor{currentfill}{rgb}{0.121569,0.466667,0.705882}%
\pgfsetfillcolor{currentfill}%
\pgfsetlinewidth{1.003750pt}%
\definecolor{currentstroke}{rgb}{0.121569,0.466667,0.705882}%
\pgfsetstrokecolor{currentstroke}%
\pgfsetdash{}{0pt}%
\pgfpathmoveto{\pgfqpoint{2.188023in}{3.114760in}}%
\pgfpathcurveto{\pgfqpoint{2.199073in}{3.114760in}}{\pgfqpoint{2.209672in}{3.119150in}}{\pgfqpoint{2.217486in}{3.126964in}}%
\pgfpathcurveto{\pgfqpoint{2.225299in}{3.134777in}}{\pgfqpoint{2.229690in}{3.145376in}}{\pgfqpoint{2.229690in}{3.156426in}}%
\pgfpathcurveto{\pgfqpoint{2.229690in}{3.167477in}}{\pgfqpoint{2.225299in}{3.178076in}}{\pgfqpoint{2.217486in}{3.185889in}}%
\pgfpathcurveto{\pgfqpoint{2.209672in}{3.193703in}}{\pgfqpoint{2.199073in}{3.198093in}}{\pgfqpoint{2.188023in}{3.198093in}}%
\pgfpathcurveto{\pgfqpoint{2.176973in}{3.198093in}}{\pgfqpoint{2.166374in}{3.193703in}}{\pgfqpoint{2.158560in}{3.185889in}}%
\pgfpathcurveto{\pgfqpoint{2.150747in}{3.178076in}}{\pgfqpoint{2.146356in}{3.167477in}}{\pgfqpoint{2.146356in}{3.156426in}}%
\pgfpathcurveto{\pgfqpoint{2.146356in}{3.145376in}}{\pgfqpoint{2.150747in}{3.134777in}}{\pgfqpoint{2.158560in}{3.126964in}}%
\pgfpathcurveto{\pgfqpoint{2.166374in}{3.119150in}}{\pgfqpoint{2.176973in}{3.114760in}}{\pgfqpoint{2.188023in}{3.114760in}}%
\pgfpathclose%
\pgfusepath{stroke,fill}%
\end{pgfscope}%
\begin{pgfscope}%
\pgfpathrectangle{\pgfqpoint{0.600000in}{0.600000in}}{\pgfqpoint{3.900000in}{3.900000in}}%
\pgfusepath{clip}%
\pgfsetbuttcap%
\pgfsetroundjoin%
\definecolor{currentfill}{rgb}{0.121569,0.466667,0.705882}%
\pgfsetfillcolor{currentfill}%
\pgfsetlinewidth{1.003750pt}%
\definecolor{currentstroke}{rgb}{0.121569,0.466667,0.705882}%
\pgfsetstrokecolor{currentstroke}%
\pgfsetdash{}{0pt}%
\pgfpathmoveto{\pgfqpoint{2.379909in}{1.983352in}}%
\pgfpathcurveto{\pgfqpoint{2.390959in}{1.983352in}}{\pgfqpoint{2.401558in}{1.987742in}}{\pgfqpoint{2.409372in}{1.995555in}}%
\pgfpathcurveto{\pgfqpoint{2.417186in}{2.003369in}}{\pgfqpoint{2.421576in}{2.013968in}}{\pgfqpoint{2.421576in}{2.025018in}}%
\pgfpathcurveto{\pgfqpoint{2.421576in}{2.036068in}}{\pgfqpoint{2.417186in}{2.046667in}}{\pgfqpoint{2.409372in}{2.054481in}}%
\pgfpathcurveto{\pgfqpoint{2.401558in}{2.062295in}}{\pgfqpoint{2.390959in}{2.066685in}}{\pgfqpoint{2.379909in}{2.066685in}}%
\pgfpathcurveto{\pgfqpoint{2.368859in}{2.066685in}}{\pgfqpoint{2.358260in}{2.062295in}}{\pgfqpoint{2.350446in}{2.054481in}}%
\pgfpathcurveto{\pgfqpoint{2.342633in}{2.046667in}}{\pgfqpoint{2.338243in}{2.036068in}}{\pgfqpoint{2.338243in}{2.025018in}}%
\pgfpathcurveto{\pgfqpoint{2.338243in}{2.013968in}}{\pgfqpoint{2.342633in}{2.003369in}}{\pgfqpoint{2.350446in}{1.995555in}}%
\pgfpathcurveto{\pgfqpoint{2.358260in}{1.987742in}}{\pgfqpoint{2.368859in}{1.983352in}}{\pgfqpoint{2.379909in}{1.983352in}}%
\pgfpathclose%
\pgfusepath{stroke,fill}%
\end{pgfscope}%
\begin{pgfscope}%
\pgfpathrectangle{\pgfqpoint{0.600000in}{0.600000in}}{\pgfqpoint{3.900000in}{3.900000in}}%
\pgfusepath{clip}%
\pgfsetbuttcap%
\pgfsetroundjoin%
\definecolor{currentfill}{rgb}{0.121569,0.466667,0.705882}%
\pgfsetfillcolor{currentfill}%
\pgfsetlinewidth{1.003750pt}%
\definecolor{currentstroke}{rgb}{0.121569,0.466667,0.705882}%
\pgfsetstrokecolor{currentstroke}%
\pgfsetdash{}{0pt}%
\pgfpathmoveto{\pgfqpoint{2.812607in}{2.776127in}}%
\pgfpathcurveto{\pgfqpoint{2.823658in}{2.776127in}}{\pgfqpoint{2.834257in}{2.780517in}}{\pgfqpoint{2.842070in}{2.788331in}}%
\pgfpathcurveto{\pgfqpoint{2.849884in}{2.796144in}}{\pgfqpoint{2.854274in}{2.806743in}}{\pgfqpoint{2.854274in}{2.817793in}}%
\pgfpathcurveto{\pgfqpoint{2.854274in}{2.828844in}}{\pgfqpoint{2.849884in}{2.839443in}}{\pgfqpoint{2.842070in}{2.847256in}}%
\pgfpathcurveto{\pgfqpoint{2.834257in}{2.855070in}}{\pgfqpoint{2.823658in}{2.859460in}}{\pgfqpoint{2.812607in}{2.859460in}}%
\pgfpathcurveto{\pgfqpoint{2.801557in}{2.859460in}}{\pgfqpoint{2.790958in}{2.855070in}}{\pgfqpoint{2.783145in}{2.847256in}}%
\pgfpathcurveto{\pgfqpoint{2.775331in}{2.839443in}}{\pgfqpoint{2.770941in}{2.828844in}}{\pgfqpoint{2.770941in}{2.817793in}}%
\pgfpathcurveto{\pgfqpoint{2.770941in}{2.806743in}}{\pgfqpoint{2.775331in}{2.796144in}}{\pgfqpoint{2.783145in}{2.788331in}}%
\pgfpathcurveto{\pgfqpoint{2.790958in}{2.780517in}}{\pgfqpoint{2.801557in}{2.776127in}}{\pgfqpoint{2.812607in}{2.776127in}}%
\pgfpathclose%
\pgfusepath{stroke,fill}%
\end{pgfscope}%
\begin{pgfscope}%
\pgfpathrectangle{\pgfqpoint{0.600000in}{0.600000in}}{\pgfqpoint{3.900000in}{3.900000in}}%
\pgfusepath{clip}%
\pgfsetbuttcap%
\pgfsetroundjoin%
\definecolor{currentfill}{rgb}{0.121569,0.466667,0.705882}%
\pgfsetfillcolor{currentfill}%
\pgfsetlinewidth{1.003750pt}%
\definecolor{currentstroke}{rgb}{0.121569,0.466667,0.705882}%
\pgfsetstrokecolor{currentstroke}%
\pgfsetdash{}{0pt}%
\pgfpathmoveto{\pgfqpoint{1.972174in}{1.026102in}}%
\pgfpathcurveto{\pgfqpoint{1.983224in}{1.026102in}}{\pgfqpoint{1.993823in}{1.030492in}}{\pgfqpoint{2.001637in}{1.038305in}}%
\pgfpathcurveto{\pgfqpoint{2.009451in}{1.046119in}}{\pgfqpoint{2.013841in}{1.056718in}}{\pgfqpoint{2.013841in}{1.067768in}}%
\pgfpathcurveto{\pgfqpoint{2.013841in}{1.078818in}}{\pgfqpoint{2.009451in}{1.089417in}}{\pgfqpoint{2.001637in}{1.097231in}}%
\pgfpathcurveto{\pgfqpoint{1.993823in}{1.105045in}}{\pgfqpoint{1.983224in}{1.109435in}}{\pgfqpoint{1.972174in}{1.109435in}}%
\pgfpathcurveto{\pgfqpoint{1.961124in}{1.109435in}}{\pgfqpoint{1.950525in}{1.105045in}}{\pgfqpoint{1.942711in}{1.097231in}}%
\pgfpathcurveto{\pgfqpoint{1.934898in}{1.089417in}}{\pgfqpoint{1.930507in}{1.078818in}}{\pgfqpoint{1.930507in}{1.067768in}}%
\pgfpathcurveto{\pgfqpoint{1.930507in}{1.056718in}}{\pgfqpoint{1.934898in}{1.046119in}}{\pgfqpoint{1.942711in}{1.038305in}}%
\pgfpathcurveto{\pgfqpoint{1.950525in}{1.030492in}}{\pgfqpoint{1.961124in}{1.026102in}}{\pgfqpoint{1.972174in}{1.026102in}}%
\pgfpathclose%
\pgfusepath{stroke,fill}%
\end{pgfscope}%
\begin{pgfscope}%
\pgfpathrectangle{\pgfqpoint{0.600000in}{0.600000in}}{\pgfqpoint{3.900000in}{3.900000in}}%
\pgfusepath{clip}%
\pgfsetbuttcap%
\pgfsetroundjoin%
\definecolor{currentfill}{rgb}{0.121569,0.466667,0.705882}%
\pgfsetfillcolor{currentfill}%
\pgfsetlinewidth{1.003750pt}%
\definecolor{currentstroke}{rgb}{0.121569,0.466667,0.705882}%
\pgfsetstrokecolor{currentstroke}%
\pgfsetdash{}{0pt}%
\pgfpathmoveto{\pgfqpoint{2.520446in}{2.884058in}}%
\pgfpathcurveto{\pgfqpoint{2.531496in}{2.884058in}}{\pgfqpoint{2.542095in}{2.888448in}}{\pgfqpoint{2.549908in}{2.896262in}}%
\pgfpathcurveto{\pgfqpoint{2.557722in}{2.904076in}}{\pgfqpoint{2.562112in}{2.914675in}}{\pgfqpoint{2.562112in}{2.925725in}}%
\pgfpathcurveto{\pgfqpoint{2.562112in}{2.936775in}}{\pgfqpoint{2.557722in}{2.947374in}}{\pgfqpoint{2.549908in}{2.955188in}}%
\pgfpathcurveto{\pgfqpoint{2.542095in}{2.963001in}}{\pgfqpoint{2.531496in}{2.967391in}}{\pgfqpoint{2.520446in}{2.967391in}}%
\pgfpathcurveto{\pgfqpoint{2.509396in}{2.967391in}}{\pgfqpoint{2.498797in}{2.963001in}}{\pgfqpoint{2.490983in}{2.955188in}}%
\pgfpathcurveto{\pgfqpoint{2.483169in}{2.947374in}}{\pgfqpoint{2.478779in}{2.936775in}}{\pgfqpoint{2.478779in}{2.925725in}}%
\pgfpathcurveto{\pgfqpoint{2.478779in}{2.914675in}}{\pgfqpoint{2.483169in}{2.904076in}}{\pgfqpoint{2.490983in}{2.896262in}}%
\pgfpathcurveto{\pgfqpoint{2.498797in}{2.888448in}}{\pgfqpoint{2.509396in}{2.884058in}}{\pgfqpoint{2.520446in}{2.884058in}}%
\pgfpathclose%
\pgfusepath{stroke,fill}%
\end{pgfscope}%
\begin{pgfscope}%
\pgfpathrectangle{\pgfqpoint{0.600000in}{0.600000in}}{\pgfqpoint{3.900000in}{3.900000in}}%
\pgfusepath{clip}%
\pgfsetbuttcap%
\pgfsetroundjoin%
\definecolor{currentfill}{rgb}{0.121569,0.466667,0.705882}%
\pgfsetfillcolor{currentfill}%
\pgfsetlinewidth{1.003750pt}%
\definecolor{currentstroke}{rgb}{0.121569,0.466667,0.705882}%
\pgfsetstrokecolor{currentstroke}%
\pgfsetdash{}{0pt}%
\pgfpathmoveto{\pgfqpoint{2.054685in}{3.398874in}}%
\pgfpathcurveto{\pgfqpoint{2.065735in}{3.398874in}}{\pgfqpoint{2.076334in}{3.403264in}}{\pgfqpoint{2.084147in}{3.411078in}}%
\pgfpathcurveto{\pgfqpoint{2.091961in}{3.418892in}}{\pgfqpoint{2.096351in}{3.429491in}}{\pgfqpoint{2.096351in}{3.440541in}}%
\pgfpathcurveto{\pgfqpoint{2.096351in}{3.451591in}}{\pgfqpoint{2.091961in}{3.462190in}}{\pgfqpoint{2.084147in}{3.470004in}}%
\pgfpathcurveto{\pgfqpoint{2.076334in}{3.477817in}}{\pgfqpoint{2.065735in}{3.482208in}}{\pgfqpoint{2.054685in}{3.482208in}}%
\pgfpathcurveto{\pgfqpoint{2.043634in}{3.482208in}}{\pgfqpoint{2.033035in}{3.477817in}}{\pgfqpoint{2.025222in}{3.470004in}}%
\pgfpathcurveto{\pgfqpoint{2.017408in}{3.462190in}}{\pgfqpoint{2.013018in}{3.451591in}}{\pgfqpoint{2.013018in}{3.440541in}}%
\pgfpathcurveto{\pgfqpoint{2.013018in}{3.429491in}}{\pgfqpoint{2.017408in}{3.418892in}}{\pgfqpoint{2.025222in}{3.411078in}}%
\pgfpathcurveto{\pgfqpoint{2.033035in}{3.403264in}}{\pgfqpoint{2.043634in}{3.398874in}}{\pgfqpoint{2.054685in}{3.398874in}}%
\pgfpathclose%
\pgfusepath{stroke,fill}%
\end{pgfscope}%
\begin{pgfscope}%
\pgfpathrectangle{\pgfqpoint{0.600000in}{0.600000in}}{\pgfqpoint{3.900000in}{3.900000in}}%
\pgfusepath{clip}%
\pgfsetbuttcap%
\pgfsetroundjoin%
\definecolor{currentfill}{rgb}{0.121569,0.466667,0.705882}%
\pgfsetfillcolor{currentfill}%
\pgfsetlinewidth{1.003750pt}%
\definecolor{currentstroke}{rgb}{0.121569,0.466667,0.705882}%
\pgfsetstrokecolor{currentstroke}%
\pgfsetdash{}{0pt}%
\pgfpathmoveto{\pgfqpoint{2.599209in}{2.874659in}}%
\pgfpathcurveto{\pgfqpoint{2.610259in}{2.874659in}}{\pgfqpoint{2.620859in}{2.879049in}}{\pgfqpoint{2.628672in}{2.886863in}}%
\pgfpathcurveto{\pgfqpoint{2.636486in}{2.894676in}}{\pgfqpoint{2.640876in}{2.905275in}}{\pgfqpoint{2.640876in}{2.916326in}}%
\pgfpathcurveto{\pgfqpoint{2.640876in}{2.927376in}}{\pgfqpoint{2.636486in}{2.937975in}}{\pgfqpoint{2.628672in}{2.945788in}}%
\pgfpathcurveto{\pgfqpoint{2.620859in}{2.953602in}}{\pgfqpoint{2.610259in}{2.957992in}}{\pgfqpoint{2.599209in}{2.957992in}}%
\pgfpathcurveto{\pgfqpoint{2.588159in}{2.957992in}}{\pgfqpoint{2.577560in}{2.953602in}}{\pgfqpoint{2.569747in}{2.945788in}}%
\pgfpathcurveto{\pgfqpoint{2.561933in}{2.937975in}}{\pgfqpoint{2.557543in}{2.927376in}}{\pgfqpoint{2.557543in}{2.916326in}}%
\pgfpathcurveto{\pgfqpoint{2.557543in}{2.905275in}}{\pgfqpoint{2.561933in}{2.894676in}}{\pgfqpoint{2.569747in}{2.886863in}}%
\pgfpathcurveto{\pgfqpoint{2.577560in}{2.879049in}}{\pgfqpoint{2.588159in}{2.874659in}}{\pgfqpoint{2.599209in}{2.874659in}}%
\pgfpathclose%
\pgfusepath{stroke,fill}%
\end{pgfscope}%
\begin{pgfscope}%
\pgfpathrectangle{\pgfqpoint{0.600000in}{0.600000in}}{\pgfqpoint{3.900000in}{3.900000in}}%
\pgfusepath{clip}%
\pgfsetbuttcap%
\pgfsetroundjoin%
\definecolor{currentfill}{rgb}{0.121569,0.466667,0.705882}%
\pgfsetfillcolor{currentfill}%
\pgfsetlinewidth{1.003750pt}%
\definecolor{currentstroke}{rgb}{0.121569,0.466667,0.705882}%
\pgfsetstrokecolor{currentstroke}%
\pgfsetdash{}{0pt}%
\pgfpathmoveto{\pgfqpoint{2.389965in}{2.327516in}}%
\pgfpathcurveto{\pgfqpoint{2.401015in}{2.327516in}}{\pgfqpoint{2.411614in}{2.331907in}}{\pgfqpoint{2.419428in}{2.339720in}}%
\pgfpathcurveto{\pgfqpoint{2.427241in}{2.347534in}}{\pgfqpoint{2.431631in}{2.358133in}}{\pgfqpoint{2.431631in}{2.369183in}}%
\pgfpathcurveto{\pgfqpoint{2.431631in}{2.380233in}}{\pgfqpoint{2.427241in}{2.390832in}}{\pgfqpoint{2.419428in}{2.398646in}}%
\pgfpathcurveto{\pgfqpoint{2.411614in}{2.406460in}}{\pgfqpoint{2.401015in}{2.410850in}}{\pgfqpoint{2.389965in}{2.410850in}}%
\pgfpathcurveto{\pgfqpoint{2.378915in}{2.410850in}}{\pgfqpoint{2.368316in}{2.406460in}}{\pgfqpoint{2.360502in}{2.398646in}}%
\pgfpathcurveto{\pgfqpoint{2.352688in}{2.390832in}}{\pgfqpoint{2.348298in}{2.380233in}}{\pgfqpoint{2.348298in}{2.369183in}}%
\pgfpathcurveto{\pgfqpoint{2.348298in}{2.358133in}}{\pgfqpoint{2.352688in}{2.347534in}}{\pgfqpoint{2.360502in}{2.339720in}}%
\pgfpathcurveto{\pgfqpoint{2.368316in}{2.331907in}}{\pgfqpoint{2.378915in}{2.327516in}}{\pgfqpoint{2.389965in}{2.327516in}}%
\pgfpathclose%
\pgfusepath{stroke,fill}%
\end{pgfscope}%
\begin{pgfscope}%
\pgfpathrectangle{\pgfqpoint{0.600000in}{0.600000in}}{\pgfqpoint{3.900000in}{3.900000in}}%
\pgfusepath{clip}%
\pgfsetbuttcap%
\pgfsetroundjoin%
\definecolor{currentfill}{rgb}{0.121569,0.466667,0.705882}%
\pgfsetfillcolor{currentfill}%
\pgfsetlinewidth{1.003750pt}%
\definecolor{currentstroke}{rgb}{0.121569,0.466667,0.705882}%
\pgfsetstrokecolor{currentstroke}%
\pgfsetdash{}{0pt}%
\pgfpathmoveto{\pgfqpoint{2.241297in}{2.953961in}}%
\pgfpathcurveto{\pgfqpoint{2.252347in}{2.953961in}}{\pgfqpoint{2.262946in}{2.958351in}}{\pgfqpoint{2.270760in}{2.966164in}}%
\pgfpathcurveto{\pgfqpoint{2.278574in}{2.973978in}}{\pgfqpoint{2.282964in}{2.984577in}}{\pgfqpoint{2.282964in}{2.995627in}}%
\pgfpathcurveto{\pgfqpoint{2.282964in}{3.006677in}}{\pgfqpoint{2.278574in}{3.017276in}}{\pgfqpoint{2.270760in}{3.025090in}}%
\pgfpathcurveto{\pgfqpoint{2.262946in}{3.032904in}}{\pgfqpoint{2.252347in}{3.037294in}}{\pgfqpoint{2.241297in}{3.037294in}}%
\pgfpathcurveto{\pgfqpoint{2.230247in}{3.037294in}}{\pgfqpoint{2.219648in}{3.032904in}}{\pgfqpoint{2.211834in}{3.025090in}}%
\pgfpathcurveto{\pgfqpoint{2.204021in}{3.017276in}}{\pgfqpoint{2.199631in}{3.006677in}}{\pgfqpoint{2.199631in}{2.995627in}}%
\pgfpathcurveto{\pgfqpoint{2.199631in}{2.984577in}}{\pgfqpoint{2.204021in}{2.973978in}}{\pgfqpoint{2.211834in}{2.966164in}}%
\pgfpathcurveto{\pgfqpoint{2.219648in}{2.958351in}}{\pgfqpoint{2.230247in}{2.953961in}}{\pgfqpoint{2.241297in}{2.953961in}}%
\pgfpathclose%
\pgfusepath{stroke,fill}%
\end{pgfscope}%
\begin{pgfscope}%
\pgfpathrectangle{\pgfqpoint{0.600000in}{0.600000in}}{\pgfqpoint{3.900000in}{3.900000in}}%
\pgfusepath{clip}%
\pgfsetbuttcap%
\pgfsetroundjoin%
\definecolor{currentfill}{rgb}{0.121569,0.466667,0.705882}%
\pgfsetfillcolor{currentfill}%
\pgfsetlinewidth{1.003750pt}%
\definecolor{currentstroke}{rgb}{0.121569,0.466667,0.705882}%
\pgfsetstrokecolor{currentstroke}%
\pgfsetdash{}{0pt}%
\pgfpathmoveto{\pgfqpoint{2.362666in}{1.803284in}}%
\pgfpathcurveto{\pgfqpoint{2.373716in}{1.803284in}}{\pgfqpoint{2.384315in}{1.807674in}}{\pgfqpoint{2.392128in}{1.815487in}}%
\pgfpathcurveto{\pgfqpoint{2.399942in}{1.823301in}}{\pgfqpoint{2.404332in}{1.833900in}}{\pgfqpoint{2.404332in}{1.844950in}}%
\pgfpathcurveto{\pgfqpoint{2.404332in}{1.856000in}}{\pgfqpoint{2.399942in}{1.866599in}}{\pgfqpoint{2.392128in}{1.874413in}}%
\pgfpathcurveto{\pgfqpoint{2.384315in}{1.882227in}}{\pgfqpoint{2.373716in}{1.886617in}}{\pgfqpoint{2.362666in}{1.886617in}}%
\pgfpathcurveto{\pgfqpoint{2.351615in}{1.886617in}}{\pgfqpoint{2.341016in}{1.882227in}}{\pgfqpoint{2.333203in}{1.874413in}}%
\pgfpathcurveto{\pgfqpoint{2.325389in}{1.866599in}}{\pgfqpoint{2.320999in}{1.856000in}}{\pgfqpoint{2.320999in}{1.844950in}}%
\pgfpathcurveto{\pgfqpoint{2.320999in}{1.833900in}}{\pgfqpoint{2.325389in}{1.823301in}}{\pgfqpoint{2.333203in}{1.815487in}}%
\pgfpathcurveto{\pgfqpoint{2.341016in}{1.807674in}}{\pgfqpoint{2.351615in}{1.803284in}}{\pgfqpoint{2.362666in}{1.803284in}}%
\pgfpathclose%
\pgfusepath{stroke,fill}%
\end{pgfscope}%
\begin{pgfscope}%
\pgfpathrectangle{\pgfqpoint{0.600000in}{0.600000in}}{\pgfqpoint{3.900000in}{3.900000in}}%
\pgfusepath{clip}%
\pgfsetbuttcap%
\pgfsetroundjoin%
\definecolor{currentfill}{rgb}{0.121569,0.466667,0.705882}%
\pgfsetfillcolor{currentfill}%
\pgfsetlinewidth{1.003750pt}%
\definecolor{currentstroke}{rgb}{0.121569,0.466667,0.705882}%
\pgfsetstrokecolor{currentstroke}%
\pgfsetdash{}{0pt}%
\pgfpathmoveto{\pgfqpoint{1.970439in}{3.518963in}}%
\pgfpathcurveto{\pgfqpoint{1.981489in}{3.518963in}}{\pgfqpoint{1.992088in}{3.523353in}}{\pgfqpoint{1.999902in}{3.531167in}}%
\pgfpathcurveto{\pgfqpoint{2.007716in}{3.538981in}}{\pgfqpoint{2.012106in}{3.549580in}}{\pgfqpoint{2.012106in}{3.560630in}}%
\pgfpathcurveto{\pgfqpoint{2.012106in}{3.571680in}}{\pgfqpoint{2.007716in}{3.582279in}}{\pgfqpoint{1.999902in}{3.590093in}}%
\pgfpathcurveto{\pgfqpoint{1.992088in}{3.597906in}}{\pgfqpoint{1.981489in}{3.602296in}}{\pgfqpoint{1.970439in}{3.602296in}}%
\pgfpathcurveto{\pgfqpoint{1.959389in}{3.602296in}}{\pgfqpoint{1.948790in}{3.597906in}}{\pgfqpoint{1.940976in}{3.590093in}}%
\pgfpathcurveto{\pgfqpoint{1.933163in}{3.582279in}}{\pgfqpoint{1.928772in}{3.571680in}}{\pgfqpoint{1.928772in}{3.560630in}}%
\pgfpathcurveto{\pgfqpoint{1.928772in}{3.549580in}}{\pgfqpoint{1.933163in}{3.538981in}}{\pgfqpoint{1.940976in}{3.531167in}}%
\pgfpathcurveto{\pgfqpoint{1.948790in}{3.523353in}}{\pgfqpoint{1.959389in}{3.518963in}}{\pgfqpoint{1.970439in}{3.518963in}}%
\pgfpathclose%
\pgfusepath{stroke,fill}%
\end{pgfscope}%
\begin{pgfscope}%
\pgfpathrectangle{\pgfqpoint{0.600000in}{0.600000in}}{\pgfqpoint{3.900000in}{3.900000in}}%
\pgfusepath{clip}%
\pgfsetbuttcap%
\pgfsetroundjoin%
\definecolor{currentfill}{rgb}{0.121569,0.466667,0.705882}%
\pgfsetfillcolor{currentfill}%
\pgfsetlinewidth{1.003750pt}%
\definecolor{currentstroke}{rgb}{0.121569,0.466667,0.705882}%
\pgfsetstrokecolor{currentstroke}%
\pgfsetdash{}{0pt}%
\pgfpathmoveto{\pgfqpoint{3.455369in}{2.927889in}}%
\pgfpathcurveto{\pgfqpoint{3.466419in}{2.927889in}}{\pgfqpoint{3.477018in}{2.932279in}}{\pgfqpoint{3.484832in}{2.940093in}}%
\pgfpathcurveto{\pgfqpoint{3.492646in}{2.947906in}}{\pgfqpoint{3.497036in}{2.958505in}}{\pgfqpoint{3.497036in}{2.969555in}}%
\pgfpathcurveto{\pgfqpoint{3.497036in}{2.980605in}}{\pgfqpoint{3.492646in}{2.991204in}}{\pgfqpoint{3.484832in}{2.999018in}}%
\pgfpathcurveto{\pgfqpoint{3.477018in}{3.006832in}}{\pgfqpoint{3.466419in}{3.011222in}}{\pgfqpoint{3.455369in}{3.011222in}}%
\pgfpathcurveto{\pgfqpoint{3.444319in}{3.011222in}}{\pgfqpoint{3.433720in}{3.006832in}}{\pgfqpoint{3.425906in}{2.999018in}}%
\pgfpathcurveto{\pgfqpoint{3.418093in}{2.991204in}}{\pgfqpoint{3.413703in}{2.980605in}}{\pgfqpoint{3.413703in}{2.969555in}}%
\pgfpathcurveto{\pgfqpoint{3.413703in}{2.958505in}}{\pgfqpoint{3.418093in}{2.947906in}}{\pgfqpoint{3.425906in}{2.940093in}}%
\pgfpathcurveto{\pgfqpoint{3.433720in}{2.932279in}}{\pgfqpoint{3.444319in}{2.927889in}}{\pgfqpoint{3.455369in}{2.927889in}}%
\pgfpathclose%
\pgfusepath{stroke,fill}%
\end{pgfscope}%
\begin{pgfscope}%
\pgfpathrectangle{\pgfqpoint{0.600000in}{0.600000in}}{\pgfqpoint{3.900000in}{3.900000in}}%
\pgfusepath{clip}%
\pgfsetbuttcap%
\pgfsetroundjoin%
\definecolor{currentfill}{rgb}{0.121569,0.466667,0.705882}%
\pgfsetfillcolor{currentfill}%
\pgfsetlinewidth{1.003750pt}%
\definecolor{currentstroke}{rgb}{0.121569,0.466667,0.705882}%
\pgfsetstrokecolor{currentstroke}%
\pgfsetdash{}{0pt}%
\pgfpathmoveto{\pgfqpoint{1.785832in}{2.844297in}}%
\pgfpathcurveto{\pgfqpoint{1.796882in}{2.844297in}}{\pgfqpoint{1.807481in}{2.848687in}}{\pgfqpoint{1.815295in}{2.856501in}}%
\pgfpathcurveto{\pgfqpoint{1.823108in}{2.864314in}}{\pgfqpoint{1.827499in}{2.874913in}}{\pgfqpoint{1.827499in}{2.885963in}}%
\pgfpathcurveto{\pgfqpoint{1.827499in}{2.897014in}}{\pgfqpoint{1.823108in}{2.907613in}}{\pgfqpoint{1.815295in}{2.915426in}}%
\pgfpathcurveto{\pgfqpoint{1.807481in}{2.923240in}}{\pgfqpoint{1.796882in}{2.927630in}}{\pgfqpoint{1.785832in}{2.927630in}}%
\pgfpathcurveto{\pgfqpoint{1.774782in}{2.927630in}}{\pgfqpoint{1.764183in}{2.923240in}}{\pgfqpoint{1.756369in}{2.915426in}}%
\pgfpathcurveto{\pgfqpoint{1.748556in}{2.907613in}}{\pgfqpoint{1.744165in}{2.897014in}}{\pgfqpoint{1.744165in}{2.885963in}}%
\pgfpathcurveto{\pgfqpoint{1.744165in}{2.874913in}}{\pgfqpoint{1.748556in}{2.864314in}}{\pgfqpoint{1.756369in}{2.856501in}}%
\pgfpathcurveto{\pgfqpoint{1.764183in}{2.848687in}}{\pgfqpoint{1.774782in}{2.844297in}}{\pgfqpoint{1.785832in}{2.844297in}}%
\pgfpathclose%
\pgfusepath{stroke,fill}%
\end{pgfscope}%
\begin{pgfscope}%
\pgfpathrectangle{\pgfqpoint{0.600000in}{0.600000in}}{\pgfqpoint{3.900000in}{3.900000in}}%
\pgfusepath{clip}%
\pgfsetbuttcap%
\pgfsetroundjoin%
\definecolor{currentfill}{rgb}{0.121569,0.466667,0.705882}%
\pgfsetfillcolor{currentfill}%
\pgfsetlinewidth{1.003750pt}%
\definecolor{currentstroke}{rgb}{0.121569,0.466667,0.705882}%
\pgfsetstrokecolor{currentstroke}%
\pgfsetdash{}{0pt}%
\pgfpathmoveto{\pgfqpoint{2.780817in}{1.709896in}}%
\pgfpathcurveto{\pgfqpoint{2.791867in}{1.709896in}}{\pgfqpoint{2.802466in}{1.714286in}}{\pgfqpoint{2.810280in}{1.722100in}}%
\pgfpathcurveto{\pgfqpoint{2.818094in}{1.729914in}}{\pgfqpoint{2.822484in}{1.740513in}}{\pgfqpoint{2.822484in}{1.751563in}}%
\pgfpathcurveto{\pgfqpoint{2.822484in}{1.762613in}}{\pgfqpoint{2.818094in}{1.773212in}}{\pgfqpoint{2.810280in}{1.781025in}}%
\pgfpathcurveto{\pgfqpoint{2.802466in}{1.788839in}}{\pgfqpoint{2.791867in}{1.793229in}}{\pgfqpoint{2.780817in}{1.793229in}}%
\pgfpathcurveto{\pgfqpoint{2.769767in}{1.793229in}}{\pgfqpoint{2.759168in}{1.788839in}}{\pgfqpoint{2.751354in}{1.781025in}}%
\pgfpathcurveto{\pgfqpoint{2.743541in}{1.773212in}}{\pgfqpoint{2.739151in}{1.762613in}}{\pgfqpoint{2.739151in}{1.751563in}}%
\pgfpathcurveto{\pgfqpoint{2.739151in}{1.740513in}}{\pgfqpoint{2.743541in}{1.729914in}}{\pgfqpoint{2.751354in}{1.722100in}}%
\pgfpathcurveto{\pgfqpoint{2.759168in}{1.714286in}}{\pgfqpoint{2.769767in}{1.709896in}}{\pgfqpoint{2.780817in}{1.709896in}}%
\pgfpathclose%
\pgfusepath{stroke,fill}%
\end{pgfscope}%
\begin{pgfscope}%
\pgfpathrectangle{\pgfqpoint{0.600000in}{0.600000in}}{\pgfqpoint{3.900000in}{3.900000in}}%
\pgfusepath{clip}%
\pgfsetbuttcap%
\pgfsetroundjoin%
\definecolor{currentfill}{rgb}{0.121569,0.466667,0.705882}%
\pgfsetfillcolor{currentfill}%
\pgfsetlinewidth{1.003750pt}%
\definecolor{currentstroke}{rgb}{0.121569,0.466667,0.705882}%
\pgfsetstrokecolor{currentstroke}%
\pgfsetdash{}{0pt}%
\pgfpathmoveto{\pgfqpoint{3.270709in}{2.600311in}}%
\pgfpathcurveto{\pgfqpoint{3.281759in}{2.600311in}}{\pgfqpoint{3.292359in}{2.604701in}}{\pgfqpoint{3.300172in}{2.612515in}}%
\pgfpathcurveto{\pgfqpoint{3.307986in}{2.620328in}}{\pgfqpoint{3.312376in}{2.630927in}}{\pgfqpoint{3.312376in}{2.641978in}}%
\pgfpathcurveto{\pgfqpoint{3.312376in}{2.653028in}}{\pgfqpoint{3.307986in}{2.663627in}}{\pgfqpoint{3.300172in}{2.671440in}}%
\pgfpathcurveto{\pgfqpoint{3.292359in}{2.679254in}}{\pgfqpoint{3.281759in}{2.683644in}}{\pgfqpoint{3.270709in}{2.683644in}}%
\pgfpathcurveto{\pgfqpoint{3.259659in}{2.683644in}}{\pgfqpoint{3.249060in}{2.679254in}}{\pgfqpoint{3.241247in}{2.671440in}}%
\pgfpathcurveto{\pgfqpoint{3.233433in}{2.663627in}}{\pgfqpoint{3.229043in}{2.653028in}}{\pgfqpoint{3.229043in}{2.641978in}}%
\pgfpathcurveto{\pgfqpoint{3.229043in}{2.630927in}}{\pgfqpoint{3.233433in}{2.620328in}}{\pgfqpoint{3.241247in}{2.612515in}}%
\pgfpathcurveto{\pgfqpoint{3.249060in}{2.604701in}}{\pgfqpoint{3.259659in}{2.600311in}}{\pgfqpoint{3.270709in}{2.600311in}}%
\pgfpathclose%
\pgfusepath{stroke,fill}%
\end{pgfscope}%
\begin{pgfscope}%
\pgfpathrectangle{\pgfqpoint{0.600000in}{0.600000in}}{\pgfqpoint{3.900000in}{3.900000in}}%
\pgfusepath{clip}%
\pgfsetbuttcap%
\pgfsetroundjoin%
\definecolor{currentfill}{rgb}{0.121569,0.466667,0.705882}%
\pgfsetfillcolor{currentfill}%
\pgfsetlinewidth{1.003750pt}%
\definecolor{currentstroke}{rgb}{0.121569,0.466667,0.705882}%
\pgfsetstrokecolor{currentstroke}%
\pgfsetdash{}{0pt}%
\pgfpathmoveto{\pgfqpoint{3.061848in}{2.767315in}}%
\pgfpathcurveto{\pgfqpoint{3.072898in}{2.767315in}}{\pgfqpoint{3.083497in}{2.771705in}}{\pgfqpoint{3.091310in}{2.779519in}}%
\pgfpathcurveto{\pgfqpoint{3.099124in}{2.787333in}}{\pgfqpoint{3.103514in}{2.797932in}}{\pgfqpoint{3.103514in}{2.808982in}}%
\pgfpathcurveto{\pgfqpoint{3.103514in}{2.820032in}}{\pgfqpoint{3.099124in}{2.830631in}}{\pgfqpoint{3.091310in}{2.838445in}}%
\pgfpathcurveto{\pgfqpoint{3.083497in}{2.846258in}}{\pgfqpoint{3.072898in}{2.850649in}}{\pgfqpoint{3.061848in}{2.850649in}}%
\pgfpathcurveto{\pgfqpoint{3.050798in}{2.850649in}}{\pgfqpoint{3.040198in}{2.846258in}}{\pgfqpoint{3.032385in}{2.838445in}}%
\pgfpathcurveto{\pgfqpoint{3.024571in}{2.830631in}}{\pgfqpoint{3.020181in}{2.820032in}}{\pgfqpoint{3.020181in}{2.808982in}}%
\pgfpathcurveto{\pgfqpoint{3.020181in}{2.797932in}}{\pgfqpoint{3.024571in}{2.787333in}}{\pgfqpoint{3.032385in}{2.779519in}}%
\pgfpathcurveto{\pgfqpoint{3.040198in}{2.771705in}}{\pgfqpoint{3.050798in}{2.767315in}}{\pgfqpoint{3.061848in}{2.767315in}}%
\pgfpathclose%
\pgfusepath{stroke,fill}%
\end{pgfscope}%
\begin{pgfscope}%
\pgfpathrectangle{\pgfqpoint{0.600000in}{0.600000in}}{\pgfqpoint{3.900000in}{3.900000in}}%
\pgfusepath{clip}%
\pgfsetbuttcap%
\pgfsetroundjoin%
\definecolor{currentfill}{rgb}{0.121569,0.466667,0.705882}%
\pgfsetfillcolor{currentfill}%
\pgfsetlinewidth{1.003750pt}%
\definecolor{currentstroke}{rgb}{0.121569,0.466667,0.705882}%
\pgfsetstrokecolor{currentstroke}%
\pgfsetdash{}{0pt}%
\pgfpathmoveto{\pgfqpoint{2.257275in}{2.323568in}}%
\pgfpathcurveto{\pgfqpoint{2.268325in}{2.323568in}}{\pgfqpoint{2.278924in}{2.327958in}}{\pgfqpoint{2.286738in}{2.335772in}}%
\pgfpathcurveto{\pgfqpoint{2.294551in}{2.343585in}}{\pgfqpoint{2.298941in}{2.354184in}}{\pgfqpoint{2.298941in}{2.365234in}}%
\pgfpathcurveto{\pgfqpoint{2.298941in}{2.376284in}}{\pgfqpoint{2.294551in}{2.386884in}}{\pgfqpoint{2.286738in}{2.394697in}}%
\pgfpathcurveto{\pgfqpoint{2.278924in}{2.402511in}}{\pgfqpoint{2.268325in}{2.406901in}}{\pgfqpoint{2.257275in}{2.406901in}}%
\pgfpathcurveto{\pgfqpoint{2.246225in}{2.406901in}}{\pgfqpoint{2.235626in}{2.402511in}}{\pgfqpoint{2.227812in}{2.394697in}}%
\pgfpathcurveto{\pgfqpoint{2.219998in}{2.386884in}}{\pgfqpoint{2.215608in}{2.376284in}}{\pgfqpoint{2.215608in}{2.365234in}}%
\pgfpathcurveto{\pgfqpoint{2.215608in}{2.354184in}}{\pgfqpoint{2.219998in}{2.343585in}}{\pgfqpoint{2.227812in}{2.335772in}}%
\pgfpathcurveto{\pgfqpoint{2.235626in}{2.327958in}}{\pgfqpoint{2.246225in}{2.323568in}}{\pgfqpoint{2.257275in}{2.323568in}}%
\pgfpathclose%
\pgfusepath{stroke,fill}%
\end{pgfscope}%
\begin{pgfscope}%
\pgfpathrectangle{\pgfqpoint{0.600000in}{0.600000in}}{\pgfqpoint{3.900000in}{3.900000in}}%
\pgfusepath{clip}%
\pgfsetbuttcap%
\pgfsetroundjoin%
\definecolor{currentfill}{rgb}{0.121569,0.466667,0.705882}%
\pgfsetfillcolor{currentfill}%
\pgfsetlinewidth{1.003750pt}%
\definecolor{currentstroke}{rgb}{0.121569,0.466667,0.705882}%
\pgfsetstrokecolor{currentstroke}%
\pgfsetdash{}{0pt}%
\pgfpathmoveto{\pgfqpoint{3.427155in}{2.503033in}}%
\pgfpathcurveto{\pgfqpoint{3.438205in}{2.503033in}}{\pgfqpoint{3.448804in}{2.507423in}}{\pgfqpoint{3.456618in}{2.515237in}}%
\pgfpathcurveto{\pgfqpoint{3.464431in}{2.523051in}}{\pgfqpoint{3.468822in}{2.533650in}}{\pgfqpoint{3.468822in}{2.544700in}}%
\pgfpathcurveto{\pgfqpoint{3.468822in}{2.555750in}}{\pgfqpoint{3.464431in}{2.566349in}}{\pgfqpoint{3.456618in}{2.574162in}}%
\pgfpathcurveto{\pgfqpoint{3.448804in}{2.581976in}}{\pgfqpoint{3.438205in}{2.586366in}}{\pgfqpoint{3.427155in}{2.586366in}}%
\pgfpathcurveto{\pgfqpoint{3.416105in}{2.586366in}}{\pgfqpoint{3.405506in}{2.581976in}}{\pgfqpoint{3.397692in}{2.574162in}}%
\pgfpathcurveto{\pgfqpoint{3.389879in}{2.566349in}}{\pgfqpoint{3.385488in}{2.555750in}}{\pgfqpoint{3.385488in}{2.544700in}}%
\pgfpathcurveto{\pgfqpoint{3.385488in}{2.533650in}}{\pgfqpoint{3.389879in}{2.523051in}}{\pgfqpoint{3.397692in}{2.515237in}}%
\pgfpathcurveto{\pgfqpoint{3.405506in}{2.507423in}}{\pgfqpoint{3.416105in}{2.503033in}}{\pgfqpoint{3.427155in}{2.503033in}}%
\pgfpathclose%
\pgfusepath{stroke,fill}%
\end{pgfscope}%
\begin{pgfscope}%
\pgfpathrectangle{\pgfqpoint{0.600000in}{0.600000in}}{\pgfqpoint{3.900000in}{3.900000in}}%
\pgfusepath{clip}%
\pgfsetbuttcap%
\pgfsetroundjoin%
\definecolor{currentfill}{rgb}{0.121569,0.466667,0.705882}%
\pgfsetfillcolor{currentfill}%
\pgfsetlinewidth{1.003750pt}%
\definecolor{currentstroke}{rgb}{0.121569,0.466667,0.705882}%
\pgfsetstrokecolor{currentstroke}%
\pgfsetdash{}{0pt}%
\pgfpathmoveto{\pgfqpoint{2.680791in}{1.835698in}}%
\pgfpathcurveto{\pgfqpoint{2.691841in}{1.835698in}}{\pgfqpoint{2.702440in}{1.840089in}}{\pgfqpoint{2.710253in}{1.847902in}}%
\pgfpathcurveto{\pgfqpoint{2.718067in}{1.855716in}}{\pgfqpoint{2.722457in}{1.866315in}}{\pgfqpoint{2.722457in}{1.877365in}}%
\pgfpathcurveto{\pgfqpoint{2.722457in}{1.888415in}}{\pgfqpoint{2.718067in}{1.899014in}}{\pgfqpoint{2.710253in}{1.906828in}}%
\pgfpathcurveto{\pgfqpoint{2.702440in}{1.914642in}}{\pgfqpoint{2.691841in}{1.919032in}}{\pgfqpoint{2.680791in}{1.919032in}}%
\pgfpathcurveto{\pgfqpoint{2.669740in}{1.919032in}}{\pgfqpoint{2.659141in}{1.914642in}}{\pgfqpoint{2.651328in}{1.906828in}}%
\pgfpathcurveto{\pgfqpoint{2.643514in}{1.899014in}}{\pgfqpoint{2.639124in}{1.888415in}}{\pgfqpoint{2.639124in}{1.877365in}}%
\pgfpathcurveto{\pgfqpoint{2.639124in}{1.866315in}}{\pgfqpoint{2.643514in}{1.855716in}}{\pgfqpoint{2.651328in}{1.847902in}}%
\pgfpathcurveto{\pgfqpoint{2.659141in}{1.840089in}}{\pgfqpoint{2.669740in}{1.835698in}}{\pgfqpoint{2.680791in}{1.835698in}}%
\pgfpathclose%
\pgfusepath{stroke,fill}%
\end{pgfscope}%
\begin{pgfscope}%
\pgfpathrectangle{\pgfqpoint{0.600000in}{0.600000in}}{\pgfqpoint{3.900000in}{3.900000in}}%
\pgfusepath{clip}%
\pgfsetbuttcap%
\pgfsetroundjoin%
\definecolor{currentfill}{rgb}{0.121569,0.466667,0.705882}%
\pgfsetfillcolor{currentfill}%
\pgfsetlinewidth{1.003750pt}%
\definecolor{currentstroke}{rgb}{0.121569,0.466667,0.705882}%
\pgfsetstrokecolor{currentstroke}%
\pgfsetdash{}{0pt}%
\pgfpathmoveto{\pgfqpoint{2.902566in}{2.071019in}}%
\pgfpathcurveto{\pgfqpoint{2.913616in}{2.071019in}}{\pgfqpoint{2.924215in}{2.075409in}}{\pgfqpoint{2.932029in}{2.083222in}}%
\pgfpathcurveto{\pgfqpoint{2.939843in}{2.091036in}}{\pgfqpoint{2.944233in}{2.101635in}}{\pgfqpoint{2.944233in}{2.112685in}}%
\pgfpathcurveto{\pgfqpoint{2.944233in}{2.123735in}}{\pgfqpoint{2.939843in}{2.134334in}}{\pgfqpoint{2.932029in}{2.142148in}}%
\pgfpathcurveto{\pgfqpoint{2.924215in}{2.149962in}}{\pgfqpoint{2.913616in}{2.154352in}}{\pgfqpoint{2.902566in}{2.154352in}}%
\pgfpathcurveto{\pgfqpoint{2.891516in}{2.154352in}}{\pgfqpoint{2.880917in}{2.149962in}}{\pgfqpoint{2.873104in}{2.142148in}}%
\pgfpathcurveto{\pgfqpoint{2.865290in}{2.134334in}}{\pgfqpoint{2.860900in}{2.123735in}}{\pgfqpoint{2.860900in}{2.112685in}}%
\pgfpathcurveto{\pgfqpoint{2.860900in}{2.101635in}}{\pgfqpoint{2.865290in}{2.091036in}}{\pgfqpoint{2.873104in}{2.083222in}}%
\pgfpathcurveto{\pgfqpoint{2.880917in}{2.075409in}}{\pgfqpoint{2.891516in}{2.071019in}}{\pgfqpoint{2.902566in}{2.071019in}}%
\pgfpathclose%
\pgfusepath{stroke,fill}%
\end{pgfscope}%
\begin{pgfscope}%
\pgfpathrectangle{\pgfqpoint{0.600000in}{0.600000in}}{\pgfqpoint{3.900000in}{3.900000in}}%
\pgfusepath{clip}%
\pgfsetbuttcap%
\pgfsetroundjoin%
\definecolor{currentfill}{rgb}{0.121569,0.466667,0.705882}%
\pgfsetfillcolor{currentfill}%
\pgfsetlinewidth{1.003750pt}%
\definecolor{currentstroke}{rgb}{0.121569,0.466667,0.705882}%
\pgfsetstrokecolor{currentstroke}%
\pgfsetdash{}{0pt}%
\pgfpathmoveto{\pgfqpoint{3.084739in}{1.413843in}}%
\pgfpathcurveto{\pgfqpoint{3.095789in}{1.413843in}}{\pgfqpoint{3.106388in}{1.418234in}}{\pgfqpoint{3.114201in}{1.426047in}}%
\pgfpathcurveto{\pgfqpoint{3.122015in}{1.433861in}}{\pgfqpoint{3.126405in}{1.444460in}}{\pgfqpoint{3.126405in}{1.455510in}}%
\pgfpathcurveto{\pgfqpoint{3.126405in}{1.466560in}}{\pgfqpoint{3.122015in}{1.477159in}}{\pgfqpoint{3.114201in}{1.484973in}}%
\pgfpathcurveto{\pgfqpoint{3.106388in}{1.492786in}}{\pgfqpoint{3.095789in}{1.497177in}}{\pgfqpoint{3.084739in}{1.497177in}}%
\pgfpathcurveto{\pgfqpoint{3.073688in}{1.497177in}}{\pgfqpoint{3.063089in}{1.492786in}}{\pgfqpoint{3.055276in}{1.484973in}}%
\pgfpathcurveto{\pgfqpoint{3.047462in}{1.477159in}}{\pgfqpoint{3.043072in}{1.466560in}}{\pgfqpoint{3.043072in}{1.455510in}}%
\pgfpathcurveto{\pgfqpoint{3.043072in}{1.444460in}}{\pgfqpoint{3.047462in}{1.433861in}}{\pgfqpoint{3.055276in}{1.426047in}}%
\pgfpathcurveto{\pgfqpoint{3.063089in}{1.418234in}}{\pgfqpoint{3.073688in}{1.413843in}}{\pgfqpoint{3.084739in}{1.413843in}}%
\pgfpathclose%
\pgfusepath{stroke,fill}%
\end{pgfscope}%
\begin{pgfscope}%
\pgfpathrectangle{\pgfqpoint{0.600000in}{0.600000in}}{\pgfqpoint{3.900000in}{3.900000in}}%
\pgfusepath{clip}%
\pgfsetbuttcap%
\pgfsetroundjoin%
\definecolor{currentfill}{rgb}{0.121569,0.466667,0.705882}%
\pgfsetfillcolor{currentfill}%
\pgfsetlinewidth{1.003750pt}%
\definecolor{currentstroke}{rgb}{0.121569,0.466667,0.705882}%
\pgfsetstrokecolor{currentstroke}%
\pgfsetdash{}{0pt}%
\pgfpathmoveto{\pgfqpoint{3.031746in}{2.630637in}}%
\pgfpathcurveto{\pgfqpoint{3.042796in}{2.630637in}}{\pgfqpoint{3.053395in}{2.635027in}}{\pgfqpoint{3.061208in}{2.642841in}}%
\pgfpathcurveto{\pgfqpoint{3.069022in}{2.650655in}}{\pgfqpoint{3.073412in}{2.661254in}}{\pgfqpoint{3.073412in}{2.672304in}}%
\pgfpathcurveto{\pgfqpoint{3.073412in}{2.683354in}}{\pgfqpoint{3.069022in}{2.693953in}}{\pgfqpoint{3.061208in}{2.701767in}}%
\pgfpathcurveto{\pgfqpoint{3.053395in}{2.709580in}}{\pgfqpoint{3.042796in}{2.713971in}}{\pgfqpoint{3.031746in}{2.713971in}}%
\pgfpathcurveto{\pgfqpoint{3.020695in}{2.713971in}}{\pgfqpoint{3.010096in}{2.709580in}}{\pgfqpoint{3.002283in}{2.701767in}}%
\pgfpathcurveto{\pgfqpoint{2.994469in}{2.693953in}}{\pgfqpoint{2.990079in}{2.683354in}}{\pgfqpoint{2.990079in}{2.672304in}}%
\pgfpathcurveto{\pgfqpoint{2.990079in}{2.661254in}}{\pgfqpoint{2.994469in}{2.650655in}}{\pgfqpoint{3.002283in}{2.642841in}}%
\pgfpathcurveto{\pgfqpoint{3.010096in}{2.635027in}}{\pgfqpoint{3.020695in}{2.630637in}}{\pgfqpoint{3.031746in}{2.630637in}}%
\pgfpathclose%
\pgfusepath{stroke,fill}%
\end{pgfscope}%
\begin{pgfscope}%
\pgfpathrectangle{\pgfqpoint{0.600000in}{0.600000in}}{\pgfqpoint{3.900000in}{3.900000in}}%
\pgfusepath{clip}%
\pgfsetbuttcap%
\pgfsetroundjoin%
\definecolor{currentfill}{rgb}{0.121569,0.466667,0.705882}%
\pgfsetfillcolor{currentfill}%
\pgfsetlinewidth{1.003750pt}%
\definecolor{currentstroke}{rgb}{0.121569,0.466667,0.705882}%
\pgfsetstrokecolor{currentstroke}%
\pgfsetdash{}{0pt}%
\pgfpathmoveto{\pgfqpoint{1.922258in}{2.654905in}}%
\pgfpathcurveto{\pgfqpoint{1.933308in}{2.654905in}}{\pgfqpoint{1.943907in}{2.659295in}}{\pgfqpoint{1.951721in}{2.667109in}}%
\pgfpathcurveto{\pgfqpoint{1.959535in}{2.674923in}}{\pgfqpoint{1.963925in}{2.685522in}}{\pgfqpoint{1.963925in}{2.696572in}}%
\pgfpathcurveto{\pgfqpoint{1.963925in}{2.707622in}}{\pgfqpoint{1.959535in}{2.718221in}}{\pgfqpoint{1.951721in}{2.726035in}}%
\pgfpathcurveto{\pgfqpoint{1.943907in}{2.733848in}}{\pgfqpoint{1.933308in}{2.738239in}}{\pgfqpoint{1.922258in}{2.738239in}}%
\pgfpathcurveto{\pgfqpoint{1.911208in}{2.738239in}}{\pgfqpoint{1.900609in}{2.733848in}}{\pgfqpoint{1.892796in}{2.726035in}}%
\pgfpathcurveto{\pgfqpoint{1.884982in}{2.718221in}}{\pgfqpoint{1.880592in}{2.707622in}}{\pgfqpoint{1.880592in}{2.696572in}}%
\pgfpathcurveto{\pgfqpoint{1.880592in}{2.685522in}}{\pgfqpoint{1.884982in}{2.674923in}}{\pgfqpoint{1.892796in}{2.667109in}}%
\pgfpathcurveto{\pgfqpoint{1.900609in}{2.659295in}}{\pgfqpoint{1.911208in}{2.654905in}}{\pgfqpoint{1.922258in}{2.654905in}}%
\pgfpathclose%
\pgfusepath{stroke,fill}%
\end{pgfscope}%
\begin{pgfscope}%
\pgfpathrectangle{\pgfqpoint{0.600000in}{0.600000in}}{\pgfqpoint{3.900000in}{3.900000in}}%
\pgfusepath{clip}%
\pgfsetbuttcap%
\pgfsetroundjoin%
\definecolor{currentfill}{rgb}{0.121569,0.466667,0.705882}%
\pgfsetfillcolor{currentfill}%
\pgfsetlinewidth{1.003750pt}%
\definecolor{currentstroke}{rgb}{0.121569,0.466667,0.705882}%
\pgfsetstrokecolor{currentstroke}%
\pgfsetdash{}{0pt}%
\pgfpathmoveto{\pgfqpoint{3.014840in}{1.780088in}}%
\pgfpathcurveto{\pgfqpoint{3.025890in}{1.780088in}}{\pgfqpoint{3.036489in}{1.784479in}}{\pgfqpoint{3.044302in}{1.792292in}}%
\pgfpathcurveto{\pgfqpoint{3.052116in}{1.800106in}}{\pgfqpoint{3.056506in}{1.810705in}}{\pgfqpoint{3.056506in}{1.821755in}}%
\pgfpathcurveto{\pgfqpoint{3.056506in}{1.832805in}}{\pgfqpoint{3.052116in}{1.843404in}}{\pgfqpoint{3.044302in}{1.851218in}}%
\pgfpathcurveto{\pgfqpoint{3.036489in}{1.859032in}}{\pgfqpoint{3.025890in}{1.863422in}}{\pgfqpoint{3.014840in}{1.863422in}}%
\pgfpathcurveto{\pgfqpoint{3.003790in}{1.863422in}}{\pgfqpoint{2.993191in}{1.859032in}}{\pgfqpoint{2.985377in}{1.851218in}}%
\pgfpathcurveto{\pgfqpoint{2.977563in}{1.843404in}}{\pgfqpoint{2.973173in}{1.832805in}}{\pgfqpoint{2.973173in}{1.821755in}}%
\pgfpathcurveto{\pgfqpoint{2.973173in}{1.810705in}}{\pgfqpoint{2.977563in}{1.800106in}}{\pgfqpoint{2.985377in}{1.792292in}}%
\pgfpathcurveto{\pgfqpoint{2.993191in}{1.784479in}}{\pgfqpoint{3.003790in}{1.780088in}}{\pgfqpoint{3.014840in}{1.780088in}}%
\pgfpathclose%
\pgfusepath{stroke,fill}%
\end{pgfscope}%
\begin{pgfscope}%
\pgfpathrectangle{\pgfqpoint{0.600000in}{0.600000in}}{\pgfqpoint{3.900000in}{3.900000in}}%
\pgfusepath{clip}%
\pgfsetbuttcap%
\pgfsetroundjoin%
\definecolor{currentfill}{rgb}{0.121569,0.466667,0.705882}%
\pgfsetfillcolor{currentfill}%
\pgfsetlinewidth{1.003750pt}%
\definecolor{currentstroke}{rgb}{0.121569,0.466667,0.705882}%
\pgfsetstrokecolor{currentstroke}%
\pgfsetdash{}{0pt}%
\pgfpathmoveto{\pgfqpoint{2.113827in}{1.864062in}}%
\pgfpathcurveto{\pgfqpoint{2.124877in}{1.864062in}}{\pgfqpoint{2.135476in}{1.868452in}}{\pgfqpoint{2.143290in}{1.876266in}}%
\pgfpathcurveto{\pgfqpoint{2.151103in}{1.884079in}}{\pgfqpoint{2.155493in}{1.894678in}}{\pgfqpoint{2.155493in}{1.905729in}}%
\pgfpathcurveto{\pgfqpoint{2.155493in}{1.916779in}}{\pgfqpoint{2.151103in}{1.927378in}}{\pgfqpoint{2.143290in}{1.935191in}}%
\pgfpathcurveto{\pgfqpoint{2.135476in}{1.943005in}}{\pgfqpoint{2.124877in}{1.947395in}}{\pgfqpoint{2.113827in}{1.947395in}}%
\pgfpathcurveto{\pgfqpoint{2.102777in}{1.947395in}}{\pgfqpoint{2.092178in}{1.943005in}}{\pgfqpoint{2.084364in}{1.935191in}}%
\pgfpathcurveto{\pgfqpoint{2.076550in}{1.927378in}}{\pgfqpoint{2.072160in}{1.916779in}}{\pgfqpoint{2.072160in}{1.905729in}}%
\pgfpathcurveto{\pgfqpoint{2.072160in}{1.894678in}}{\pgfqpoint{2.076550in}{1.884079in}}{\pgfqpoint{2.084364in}{1.876266in}}%
\pgfpathcurveto{\pgfqpoint{2.092178in}{1.868452in}}{\pgfqpoint{2.102777in}{1.864062in}}{\pgfqpoint{2.113827in}{1.864062in}}%
\pgfpathclose%
\pgfusepath{stroke,fill}%
\end{pgfscope}%
\begin{pgfscope}%
\pgfpathrectangle{\pgfqpoint{0.600000in}{0.600000in}}{\pgfqpoint{3.900000in}{3.900000in}}%
\pgfusepath{clip}%
\pgfsetbuttcap%
\pgfsetroundjoin%
\definecolor{currentfill}{rgb}{0.121569,0.466667,0.705882}%
\pgfsetfillcolor{currentfill}%
\pgfsetlinewidth{1.003750pt}%
\definecolor{currentstroke}{rgb}{0.121569,0.466667,0.705882}%
\pgfsetstrokecolor{currentstroke}%
\pgfsetdash{}{0pt}%
\pgfpathmoveto{\pgfqpoint{2.975452in}{2.260956in}}%
\pgfpathcurveto{\pgfqpoint{2.986502in}{2.260956in}}{\pgfqpoint{2.997101in}{2.265346in}}{\pgfqpoint{3.004914in}{2.273160in}}%
\pgfpathcurveto{\pgfqpoint{3.012728in}{2.280973in}}{\pgfqpoint{3.017118in}{2.291573in}}{\pgfqpoint{3.017118in}{2.302623in}}%
\pgfpathcurveto{\pgfqpoint{3.017118in}{2.313673in}}{\pgfqpoint{3.012728in}{2.324272in}}{\pgfqpoint{3.004914in}{2.332085in}}%
\pgfpathcurveto{\pgfqpoint{2.997101in}{2.339899in}}{\pgfqpoint{2.986502in}{2.344289in}}{\pgfqpoint{2.975452in}{2.344289in}}%
\pgfpathcurveto{\pgfqpoint{2.964402in}{2.344289in}}{\pgfqpoint{2.953802in}{2.339899in}}{\pgfqpoint{2.945989in}{2.332085in}}%
\pgfpathcurveto{\pgfqpoint{2.938175in}{2.324272in}}{\pgfqpoint{2.933785in}{2.313673in}}{\pgfqpoint{2.933785in}{2.302623in}}%
\pgfpathcurveto{\pgfqpoint{2.933785in}{2.291573in}}{\pgfqpoint{2.938175in}{2.280973in}}{\pgfqpoint{2.945989in}{2.273160in}}%
\pgfpathcurveto{\pgfqpoint{2.953802in}{2.265346in}}{\pgfqpoint{2.964402in}{2.260956in}}{\pgfqpoint{2.975452in}{2.260956in}}%
\pgfpathclose%
\pgfusepath{stroke,fill}%
\end{pgfscope}%
\begin{pgfscope}%
\pgfpathrectangle{\pgfqpoint{0.600000in}{0.600000in}}{\pgfqpoint{3.900000in}{3.900000in}}%
\pgfusepath{clip}%
\pgfsetbuttcap%
\pgfsetroundjoin%
\definecolor{currentfill}{rgb}{0.121569,0.466667,0.705882}%
\pgfsetfillcolor{currentfill}%
\pgfsetlinewidth{1.003750pt}%
\definecolor{currentstroke}{rgb}{0.121569,0.466667,0.705882}%
\pgfsetstrokecolor{currentstroke}%
\pgfsetdash{}{0pt}%
\pgfpathmoveto{\pgfqpoint{2.075004in}{2.456201in}}%
\pgfpathcurveto{\pgfqpoint{2.086055in}{2.456201in}}{\pgfqpoint{2.096654in}{2.460592in}}{\pgfqpoint{2.104467in}{2.468405in}}%
\pgfpathcurveto{\pgfqpoint{2.112281in}{2.476219in}}{\pgfqpoint{2.116671in}{2.486818in}}{\pgfqpoint{2.116671in}{2.497868in}}%
\pgfpathcurveto{\pgfqpoint{2.116671in}{2.508918in}}{\pgfqpoint{2.112281in}{2.519517in}}{\pgfqpoint{2.104467in}{2.527331in}}%
\pgfpathcurveto{\pgfqpoint{2.096654in}{2.535145in}}{\pgfqpoint{2.086055in}{2.539535in}}{\pgfqpoint{2.075004in}{2.539535in}}%
\pgfpathcurveto{\pgfqpoint{2.063954in}{2.539535in}}{\pgfqpoint{2.053355in}{2.535145in}}{\pgfqpoint{2.045542in}{2.527331in}}%
\pgfpathcurveto{\pgfqpoint{2.037728in}{2.519517in}}{\pgfqpoint{2.033338in}{2.508918in}}{\pgfqpoint{2.033338in}{2.497868in}}%
\pgfpathcurveto{\pgfqpoint{2.033338in}{2.486818in}}{\pgfqpoint{2.037728in}{2.476219in}}{\pgfqpoint{2.045542in}{2.468405in}}%
\pgfpathcurveto{\pgfqpoint{2.053355in}{2.460592in}}{\pgfqpoint{2.063954in}{2.456201in}}{\pgfqpoint{2.075004in}{2.456201in}}%
\pgfpathclose%
\pgfusepath{stroke,fill}%
\end{pgfscope}%
\begin{pgfscope}%
\pgfpathrectangle{\pgfqpoint{0.600000in}{0.600000in}}{\pgfqpoint{3.900000in}{3.900000in}}%
\pgfusepath{clip}%
\pgfsetbuttcap%
\pgfsetroundjoin%
\definecolor{currentfill}{rgb}{0.121569,0.466667,0.705882}%
\pgfsetfillcolor{currentfill}%
\pgfsetlinewidth{1.003750pt}%
\definecolor{currentstroke}{rgb}{0.121569,0.466667,0.705882}%
\pgfsetstrokecolor{currentstroke}%
\pgfsetdash{}{0pt}%
\pgfpathmoveto{\pgfqpoint{2.502887in}{2.315143in}}%
\pgfpathcurveto{\pgfqpoint{2.513937in}{2.315143in}}{\pgfqpoint{2.524536in}{2.319533in}}{\pgfqpoint{2.532350in}{2.327347in}}%
\pgfpathcurveto{\pgfqpoint{2.540163in}{2.335160in}}{\pgfqpoint{2.544554in}{2.345759in}}{\pgfqpoint{2.544554in}{2.356809in}}%
\pgfpathcurveto{\pgfqpoint{2.544554in}{2.367859in}}{\pgfqpoint{2.540163in}{2.378458in}}{\pgfqpoint{2.532350in}{2.386272in}}%
\pgfpathcurveto{\pgfqpoint{2.524536in}{2.394086in}}{\pgfqpoint{2.513937in}{2.398476in}}{\pgfqpoint{2.502887in}{2.398476in}}%
\pgfpathcurveto{\pgfqpoint{2.491837in}{2.398476in}}{\pgfqpoint{2.481238in}{2.394086in}}{\pgfqpoint{2.473424in}{2.386272in}}%
\pgfpathcurveto{\pgfqpoint{2.465610in}{2.378458in}}{\pgfqpoint{2.461220in}{2.367859in}}{\pgfqpoint{2.461220in}{2.356809in}}%
\pgfpathcurveto{\pgfqpoint{2.461220in}{2.345759in}}{\pgfqpoint{2.465610in}{2.335160in}}{\pgfqpoint{2.473424in}{2.327347in}}%
\pgfpathcurveto{\pgfqpoint{2.481238in}{2.319533in}}{\pgfqpoint{2.491837in}{2.315143in}}{\pgfqpoint{2.502887in}{2.315143in}}%
\pgfpathclose%
\pgfusepath{stroke,fill}%
\end{pgfscope}%
\begin{pgfscope}%
\pgfpathrectangle{\pgfqpoint{0.600000in}{0.600000in}}{\pgfqpoint{3.900000in}{3.900000in}}%
\pgfusepath{clip}%
\pgfsetbuttcap%
\pgfsetroundjoin%
\definecolor{currentfill}{rgb}{0.121569,0.466667,0.705882}%
\pgfsetfillcolor{currentfill}%
\pgfsetlinewidth{1.003750pt}%
\definecolor{currentstroke}{rgb}{0.121569,0.466667,0.705882}%
\pgfsetstrokecolor{currentstroke}%
\pgfsetdash{}{0pt}%
\pgfpathmoveto{\pgfqpoint{1.990095in}{2.455254in}}%
\pgfpathcurveto{\pgfqpoint{2.001146in}{2.455254in}}{\pgfqpoint{2.011745in}{2.459644in}}{\pgfqpoint{2.019558in}{2.467457in}}%
\pgfpathcurveto{\pgfqpoint{2.027372in}{2.475271in}}{\pgfqpoint{2.031762in}{2.485870in}}{\pgfqpoint{2.031762in}{2.496920in}}%
\pgfpathcurveto{\pgfqpoint{2.031762in}{2.507970in}}{\pgfqpoint{2.027372in}{2.518569in}}{\pgfqpoint{2.019558in}{2.526383in}}%
\pgfpathcurveto{\pgfqpoint{2.011745in}{2.534197in}}{\pgfqpoint{2.001146in}{2.538587in}}{\pgfqpoint{1.990095in}{2.538587in}}%
\pgfpathcurveto{\pgfqpoint{1.979045in}{2.538587in}}{\pgfqpoint{1.968446in}{2.534197in}}{\pgfqpoint{1.960633in}{2.526383in}}%
\pgfpathcurveto{\pgfqpoint{1.952819in}{2.518569in}}{\pgfqpoint{1.948429in}{2.507970in}}{\pgfqpoint{1.948429in}{2.496920in}}%
\pgfpathcurveto{\pgfqpoint{1.948429in}{2.485870in}}{\pgfqpoint{1.952819in}{2.475271in}}{\pgfqpoint{1.960633in}{2.467457in}}%
\pgfpathcurveto{\pgfqpoint{1.968446in}{2.459644in}}{\pgfqpoint{1.979045in}{2.455254in}}{\pgfqpoint{1.990095in}{2.455254in}}%
\pgfpathclose%
\pgfusepath{stroke,fill}%
\end{pgfscope}%
\begin{pgfscope}%
\pgfpathrectangle{\pgfqpoint{0.600000in}{0.600000in}}{\pgfqpoint{3.900000in}{3.900000in}}%
\pgfusepath{clip}%
\pgfsetbuttcap%
\pgfsetroundjoin%
\definecolor{currentfill}{rgb}{0.121569,0.466667,0.705882}%
\pgfsetfillcolor{currentfill}%
\pgfsetlinewidth{1.003750pt}%
\definecolor{currentstroke}{rgb}{0.121569,0.466667,0.705882}%
\pgfsetstrokecolor{currentstroke}%
\pgfsetdash{}{0pt}%
\pgfpathmoveto{\pgfqpoint{2.100106in}{2.943839in}}%
\pgfpathcurveto{\pgfqpoint{2.111156in}{2.943839in}}{\pgfqpoint{2.121755in}{2.948229in}}{\pgfqpoint{2.129569in}{2.956042in}}%
\pgfpathcurveto{\pgfqpoint{2.137383in}{2.963856in}}{\pgfqpoint{2.141773in}{2.974455in}}{\pgfqpoint{2.141773in}{2.985505in}}%
\pgfpathcurveto{\pgfqpoint{2.141773in}{2.996555in}}{\pgfqpoint{2.137383in}{3.007154in}}{\pgfqpoint{2.129569in}{3.014968in}}%
\pgfpathcurveto{\pgfqpoint{2.121755in}{3.022782in}}{\pgfqpoint{2.111156in}{3.027172in}}{\pgfqpoint{2.100106in}{3.027172in}}%
\pgfpathcurveto{\pgfqpoint{2.089056in}{3.027172in}}{\pgfqpoint{2.078457in}{3.022782in}}{\pgfqpoint{2.070643in}{3.014968in}}%
\pgfpathcurveto{\pgfqpoint{2.062830in}{3.007154in}}{\pgfqpoint{2.058439in}{2.996555in}}{\pgfqpoint{2.058439in}{2.985505in}}%
\pgfpathcurveto{\pgfqpoint{2.058439in}{2.974455in}}{\pgfqpoint{2.062830in}{2.963856in}}{\pgfqpoint{2.070643in}{2.956042in}}%
\pgfpathcurveto{\pgfqpoint{2.078457in}{2.948229in}}{\pgfqpoint{2.089056in}{2.943839in}}{\pgfqpoint{2.100106in}{2.943839in}}%
\pgfpathclose%
\pgfusepath{stroke,fill}%
\end{pgfscope}%
\begin{pgfscope}%
\pgfpathrectangle{\pgfqpoint{0.600000in}{0.600000in}}{\pgfqpoint{3.900000in}{3.900000in}}%
\pgfusepath{clip}%
\pgfsetbuttcap%
\pgfsetroundjoin%
\definecolor{currentfill}{rgb}{0.121569,0.466667,0.705882}%
\pgfsetfillcolor{currentfill}%
\pgfsetlinewidth{1.003750pt}%
\definecolor{currentstroke}{rgb}{0.121569,0.466667,0.705882}%
\pgfsetstrokecolor{currentstroke}%
\pgfsetdash{}{0pt}%
\pgfpathmoveto{\pgfqpoint{1.588257in}{2.594730in}}%
\pgfpathcurveto{\pgfqpoint{1.599307in}{2.594730in}}{\pgfqpoint{1.609906in}{2.599120in}}{\pgfqpoint{1.617720in}{2.606934in}}%
\pgfpathcurveto{\pgfqpoint{1.625533in}{2.614748in}}{\pgfqpoint{1.629923in}{2.625347in}}{\pgfqpoint{1.629923in}{2.636397in}}%
\pgfpathcurveto{\pgfqpoint{1.629923in}{2.647447in}}{\pgfqpoint{1.625533in}{2.658046in}}{\pgfqpoint{1.617720in}{2.665860in}}%
\pgfpathcurveto{\pgfqpoint{1.609906in}{2.673673in}}{\pgfqpoint{1.599307in}{2.678063in}}{\pgfqpoint{1.588257in}{2.678063in}}%
\pgfpathcurveto{\pgfqpoint{1.577207in}{2.678063in}}{\pgfqpoint{1.566608in}{2.673673in}}{\pgfqpoint{1.558794in}{2.665860in}}%
\pgfpathcurveto{\pgfqpoint{1.550980in}{2.658046in}}{\pgfqpoint{1.546590in}{2.647447in}}{\pgfqpoint{1.546590in}{2.636397in}}%
\pgfpathcurveto{\pgfqpoint{1.546590in}{2.625347in}}{\pgfqpoint{1.550980in}{2.614748in}}{\pgfqpoint{1.558794in}{2.606934in}}%
\pgfpathcurveto{\pgfqpoint{1.566608in}{2.599120in}}{\pgfqpoint{1.577207in}{2.594730in}}{\pgfqpoint{1.588257in}{2.594730in}}%
\pgfpathclose%
\pgfusepath{stroke,fill}%
\end{pgfscope}%
\begin{pgfscope}%
\pgfpathrectangle{\pgfqpoint{0.600000in}{0.600000in}}{\pgfqpoint{3.900000in}{3.900000in}}%
\pgfusepath{clip}%
\pgfsetbuttcap%
\pgfsetroundjoin%
\definecolor{currentfill}{rgb}{0.121569,0.466667,0.705882}%
\pgfsetfillcolor{currentfill}%
\pgfsetlinewidth{1.003750pt}%
\definecolor{currentstroke}{rgb}{0.121569,0.466667,0.705882}%
\pgfsetstrokecolor{currentstroke}%
\pgfsetdash{}{0pt}%
\pgfpathmoveto{\pgfqpoint{1.639681in}{3.133973in}}%
\pgfpathcurveto{\pgfqpoint{1.650731in}{3.133973in}}{\pgfqpoint{1.661330in}{3.138363in}}{\pgfqpoint{1.669144in}{3.146177in}}%
\pgfpathcurveto{\pgfqpoint{1.676957in}{3.153991in}}{\pgfqpoint{1.681348in}{3.164590in}}{\pgfqpoint{1.681348in}{3.175640in}}%
\pgfpathcurveto{\pgfqpoint{1.681348in}{3.186690in}}{\pgfqpoint{1.676957in}{3.197289in}}{\pgfqpoint{1.669144in}{3.205103in}}%
\pgfpathcurveto{\pgfqpoint{1.661330in}{3.212916in}}{\pgfqpoint{1.650731in}{3.217307in}}{\pgfqpoint{1.639681in}{3.217307in}}%
\pgfpathcurveto{\pgfqpoint{1.628631in}{3.217307in}}{\pgfqpoint{1.618032in}{3.212916in}}{\pgfqpoint{1.610218in}{3.205103in}}%
\pgfpathcurveto{\pgfqpoint{1.602405in}{3.197289in}}{\pgfqpoint{1.598014in}{3.186690in}}{\pgfqpoint{1.598014in}{3.175640in}}%
\pgfpathcurveto{\pgfqpoint{1.598014in}{3.164590in}}{\pgfqpoint{1.602405in}{3.153991in}}{\pgfqpoint{1.610218in}{3.146177in}}%
\pgfpathcurveto{\pgfqpoint{1.618032in}{3.138363in}}{\pgfqpoint{1.628631in}{3.133973in}}{\pgfqpoint{1.639681in}{3.133973in}}%
\pgfpathclose%
\pgfusepath{stroke,fill}%
\end{pgfscope}%
\begin{pgfscope}%
\pgfpathrectangle{\pgfqpoint{0.600000in}{0.600000in}}{\pgfqpoint{3.900000in}{3.900000in}}%
\pgfusepath{clip}%
\pgfsetbuttcap%
\pgfsetroundjoin%
\definecolor{currentfill}{rgb}{0.121569,0.466667,0.705882}%
\pgfsetfillcolor{currentfill}%
\pgfsetlinewidth{1.003750pt}%
\definecolor{currentstroke}{rgb}{0.121569,0.466667,0.705882}%
\pgfsetstrokecolor{currentstroke}%
\pgfsetdash{}{0pt}%
\pgfpathmoveto{\pgfqpoint{1.995957in}{2.127707in}}%
\pgfpathcurveto{\pgfqpoint{2.007008in}{2.127707in}}{\pgfqpoint{2.017607in}{2.132097in}}{\pgfqpoint{2.025420in}{2.139911in}}%
\pgfpathcurveto{\pgfqpoint{2.033234in}{2.147724in}}{\pgfqpoint{2.037624in}{2.158323in}}{\pgfqpoint{2.037624in}{2.169373in}}%
\pgfpathcurveto{\pgfqpoint{2.037624in}{2.180423in}}{\pgfqpoint{2.033234in}{2.191022in}}{\pgfqpoint{2.025420in}{2.198836in}}%
\pgfpathcurveto{\pgfqpoint{2.017607in}{2.206650in}}{\pgfqpoint{2.007008in}{2.211040in}}{\pgfqpoint{1.995957in}{2.211040in}}%
\pgfpathcurveto{\pgfqpoint{1.984907in}{2.211040in}}{\pgfqpoint{1.974308in}{2.206650in}}{\pgfqpoint{1.966495in}{2.198836in}}%
\pgfpathcurveto{\pgfqpoint{1.958681in}{2.191022in}}{\pgfqpoint{1.954291in}{2.180423in}}{\pgfqpoint{1.954291in}{2.169373in}}%
\pgfpathcurveto{\pgfqpoint{1.954291in}{2.158323in}}{\pgfqpoint{1.958681in}{2.147724in}}{\pgfqpoint{1.966495in}{2.139911in}}%
\pgfpathcurveto{\pgfqpoint{1.974308in}{2.132097in}}{\pgfqpoint{1.984907in}{2.127707in}}{\pgfqpoint{1.995957in}{2.127707in}}%
\pgfpathclose%
\pgfusepath{stroke,fill}%
\end{pgfscope}%
\begin{pgfscope}%
\pgfpathrectangle{\pgfqpoint{0.600000in}{0.600000in}}{\pgfqpoint{3.900000in}{3.900000in}}%
\pgfusepath{clip}%
\pgfsetbuttcap%
\pgfsetroundjoin%
\definecolor{currentfill}{rgb}{0.121569,0.466667,0.705882}%
\pgfsetfillcolor{currentfill}%
\pgfsetlinewidth{1.003750pt}%
\definecolor{currentstroke}{rgb}{0.121569,0.466667,0.705882}%
\pgfsetstrokecolor{currentstroke}%
\pgfsetdash{}{0pt}%
\pgfpathmoveto{\pgfqpoint{1.766840in}{2.991955in}}%
\pgfpathcurveto{\pgfqpoint{1.777890in}{2.991955in}}{\pgfqpoint{1.788489in}{2.996345in}}{\pgfqpoint{1.796303in}{3.004159in}}%
\pgfpathcurveto{\pgfqpoint{1.804116in}{3.011973in}}{\pgfqpoint{1.808507in}{3.022572in}}{\pgfqpoint{1.808507in}{3.033622in}}%
\pgfpathcurveto{\pgfqpoint{1.808507in}{3.044672in}}{\pgfqpoint{1.804116in}{3.055271in}}{\pgfqpoint{1.796303in}{3.063085in}}%
\pgfpathcurveto{\pgfqpoint{1.788489in}{3.070898in}}{\pgfqpoint{1.777890in}{3.075289in}}{\pgfqpoint{1.766840in}{3.075289in}}%
\pgfpathcurveto{\pgfqpoint{1.755790in}{3.075289in}}{\pgfqpoint{1.745191in}{3.070898in}}{\pgfqpoint{1.737377in}{3.063085in}}%
\pgfpathcurveto{\pgfqpoint{1.729563in}{3.055271in}}{\pgfqpoint{1.725173in}{3.044672in}}{\pgfqpoint{1.725173in}{3.033622in}}%
\pgfpathcurveto{\pgfqpoint{1.725173in}{3.022572in}}{\pgfqpoint{1.729563in}{3.011973in}}{\pgfqpoint{1.737377in}{3.004159in}}%
\pgfpathcurveto{\pgfqpoint{1.745191in}{2.996345in}}{\pgfqpoint{1.755790in}{2.991955in}}{\pgfqpoint{1.766840in}{2.991955in}}%
\pgfpathclose%
\pgfusepath{stroke,fill}%
\end{pgfscope}%
\begin{pgfscope}%
\pgfpathrectangle{\pgfqpoint{0.600000in}{0.600000in}}{\pgfqpoint{3.900000in}{3.900000in}}%
\pgfusepath{clip}%
\pgfsetbuttcap%
\pgfsetroundjoin%
\definecolor{currentfill}{rgb}{0.121569,0.466667,0.705882}%
\pgfsetfillcolor{currentfill}%
\pgfsetlinewidth{1.003750pt}%
\definecolor{currentstroke}{rgb}{0.121569,0.466667,0.705882}%
\pgfsetstrokecolor{currentstroke}%
\pgfsetdash{}{0pt}%
\pgfpathmoveto{\pgfqpoint{2.076959in}{2.777363in}}%
\pgfpathcurveto{\pgfqpoint{2.088009in}{2.777363in}}{\pgfqpoint{2.098608in}{2.781753in}}{\pgfqpoint{2.106421in}{2.789566in}}%
\pgfpathcurveto{\pgfqpoint{2.114235in}{2.797380in}}{\pgfqpoint{2.118625in}{2.807979in}}{\pgfqpoint{2.118625in}{2.819029in}}%
\pgfpathcurveto{\pgfqpoint{2.118625in}{2.830079in}}{\pgfqpoint{2.114235in}{2.840678in}}{\pgfqpoint{2.106421in}{2.848492in}}%
\pgfpathcurveto{\pgfqpoint{2.098608in}{2.856306in}}{\pgfqpoint{2.088009in}{2.860696in}}{\pgfqpoint{2.076959in}{2.860696in}}%
\pgfpathcurveto{\pgfqpoint{2.065908in}{2.860696in}}{\pgfqpoint{2.055309in}{2.856306in}}{\pgfqpoint{2.047496in}{2.848492in}}%
\pgfpathcurveto{\pgfqpoint{2.039682in}{2.840678in}}{\pgfqpoint{2.035292in}{2.830079in}}{\pgfqpoint{2.035292in}{2.819029in}}%
\pgfpathcurveto{\pgfqpoint{2.035292in}{2.807979in}}{\pgfqpoint{2.039682in}{2.797380in}}{\pgfqpoint{2.047496in}{2.789566in}}%
\pgfpathcurveto{\pgfqpoint{2.055309in}{2.781753in}}{\pgfqpoint{2.065908in}{2.777363in}}{\pgfqpoint{2.076959in}{2.777363in}}%
\pgfpathclose%
\pgfusepath{stroke,fill}%
\end{pgfscope}%
\begin{pgfscope}%
\pgfpathrectangle{\pgfqpoint{0.600000in}{0.600000in}}{\pgfqpoint{3.900000in}{3.900000in}}%
\pgfusepath{clip}%
\pgfsetbuttcap%
\pgfsetroundjoin%
\definecolor{currentfill}{rgb}{0.121569,0.466667,0.705882}%
\pgfsetfillcolor{currentfill}%
\pgfsetlinewidth{1.003750pt}%
\definecolor{currentstroke}{rgb}{0.121569,0.466667,0.705882}%
\pgfsetstrokecolor{currentstroke}%
\pgfsetdash{}{0pt}%
\pgfpathmoveto{\pgfqpoint{2.595020in}{2.754207in}}%
\pgfpathcurveto{\pgfqpoint{2.606070in}{2.754207in}}{\pgfqpoint{2.616669in}{2.758597in}}{\pgfqpoint{2.624483in}{2.766411in}}%
\pgfpathcurveto{\pgfqpoint{2.632297in}{2.774224in}}{\pgfqpoint{2.636687in}{2.784823in}}{\pgfqpoint{2.636687in}{2.795874in}}%
\pgfpathcurveto{\pgfqpoint{2.636687in}{2.806924in}}{\pgfqpoint{2.632297in}{2.817523in}}{\pgfqpoint{2.624483in}{2.825336in}}%
\pgfpathcurveto{\pgfqpoint{2.616669in}{2.833150in}}{\pgfqpoint{2.606070in}{2.837540in}}{\pgfqpoint{2.595020in}{2.837540in}}%
\pgfpathcurveto{\pgfqpoint{2.583970in}{2.837540in}}{\pgfqpoint{2.573371in}{2.833150in}}{\pgfqpoint{2.565557in}{2.825336in}}%
\pgfpathcurveto{\pgfqpoint{2.557744in}{2.817523in}}{\pgfqpoint{2.553354in}{2.806924in}}{\pgfqpoint{2.553354in}{2.795874in}}%
\pgfpathcurveto{\pgfqpoint{2.553354in}{2.784823in}}{\pgfqpoint{2.557744in}{2.774224in}}{\pgfqpoint{2.565557in}{2.766411in}}%
\pgfpathcurveto{\pgfqpoint{2.573371in}{2.758597in}}{\pgfqpoint{2.583970in}{2.754207in}}{\pgfqpoint{2.595020in}{2.754207in}}%
\pgfpathclose%
\pgfusepath{stroke,fill}%
\end{pgfscope}%
\begin{pgfscope}%
\pgfpathrectangle{\pgfqpoint{0.600000in}{0.600000in}}{\pgfqpoint{3.900000in}{3.900000in}}%
\pgfusepath{clip}%
\pgfsetbuttcap%
\pgfsetroundjoin%
\definecolor{currentfill}{rgb}{0.121569,0.466667,0.705882}%
\pgfsetfillcolor{currentfill}%
\pgfsetlinewidth{1.003750pt}%
\definecolor{currentstroke}{rgb}{0.121569,0.466667,0.705882}%
\pgfsetstrokecolor{currentstroke}%
\pgfsetdash{}{0pt}%
\pgfpathmoveto{\pgfqpoint{2.705052in}{2.077467in}}%
\pgfpathcurveto{\pgfqpoint{2.716102in}{2.077467in}}{\pgfqpoint{2.726701in}{2.081857in}}{\pgfqpoint{2.734515in}{2.089670in}}%
\pgfpathcurveto{\pgfqpoint{2.742328in}{2.097484in}}{\pgfqpoint{2.746719in}{2.108083in}}{\pgfqpoint{2.746719in}{2.119133in}}%
\pgfpathcurveto{\pgfqpoint{2.746719in}{2.130183in}}{\pgfqpoint{2.742328in}{2.140782in}}{\pgfqpoint{2.734515in}{2.148596in}}%
\pgfpathcurveto{\pgfqpoint{2.726701in}{2.156410in}}{\pgfqpoint{2.716102in}{2.160800in}}{\pgfqpoint{2.705052in}{2.160800in}}%
\pgfpathcurveto{\pgfqpoint{2.694002in}{2.160800in}}{\pgfqpoint{2.683403in}{2.156410in}}{\pgfqpoint{2.675589in}{2.148596in}}%
\pgfpathcurveto{\pgfqpoint{2.667775in}{2.140782in}}{\pgfqpoint{2.663385in}{2.130183in}}{\pgfqpoint{2.663385in}{2.119133in}}%
\pgfpathcurveto{\pgfqpoint{2.663385in}{2.108083in}}{\pgfqpoint{2.667775in}{2.097484in}}{\pgfqpoint{2.675589in}{2.089670in}}%
\pgfpathcurveto{\pgfqpoint{2.683403in}{2.081857in}}{\pgfqpoint{2.694002in}{2.077467in}}{\pgfqpoint{2.705052in}{2.077467in}}%
\pgfpathclose%
\pgfusepath{stroke,fill}%
\end{pgfscope}%
\begin{pgfscope}%
\pgfpathrectangle{\pgfqpoint{0.600000in}{0.600000in}}{\pgfqpoint{3.900000in}{3.900000in}}%
\pgfusepath{clip}%
\pgfsetbuttcap%
\pgfsetroundjoin%
\definecolor{currentfill}{rgb}{0.121569,0.466667,0.705882}%
\pgfsetfillcolor{currentfill}%
\pgfsetlinewidth{1.003750pt}%
\definecolor{currentstroke}{rgb}{0.121569,0.466667,0.705882}%
\pgfsetstrokecolor{currentstroke}%
\pgfsetdash{}{0pt}%
\pgfpathmoveto{\pgfqpoint{3.071408in}{2.410989in}}%
\pgfpathcurveto{\pgfqpoint{3.082458in}{2.410989in}}{\pgfqpoint{3.093057in}{2.415380in}}{\pgfqpoint{3.100871in}{2.423193in}}%
\pgfpathcurveto{\pgfqpoint{3.108685in}{2.431007in}}{\pgfqpoint{3.113075in}{2.441606in}}{\pgfqpoint{3.113075in}{2.452656in}}%
\pgfpathcurveto{\pgfqpoint{3.113075in}{2.463706in}}{\pgfqpoint{3.108685in}{2.474305in}}{\pgfqpoint{3.100871in}{2.482119in}}%
\pgfpathcurveto{\pgfqpoint{3.093057in}{2.489932in}}{\pgfqpoint{3.082458in}{2.494323in}}{\pgfqpoint{3.071408in}{2.494323in}}%
\pgfpathcurveto{\pgfqpoint{3.060358in}{2.494323in}}{\pgfqpoint{3.049759in}{2.489932in}}{\pgfqpoint{3.041946in}{2.482119in}}%
\pgfpathcurveto{\pgfqpoint{3.034132in}{2.474305in}}{\pgfqpoint{3.029742in}{2.463706in}}{\pgfqpoint{3.029742in}{2.452656in}}%
\pgfpathcurveto{\pgfqpoint{3.029742in}{2.441606in}}{\pgfqpoint{3.034132in}{2.431007in}}{\pgfqpoint{3.041946in}{2.423193in}}%
\pgfpathcurveto{\pgfqpoint{3.049759in}{2.415380in}}{\pgfqpoint{3.060358in}{2.410989in}}{\pgfqpoint{3.071408in}{2.410989in}}%
\pgfpathclose%
\pgfusepath{stroke,fill}%
\end{pgfscope}%
\begin{pgfscope}%
\pgfpathrectangle{\pgfqpoint{0.600000in}{0.600000in}}{\pgfqpoint{3.900000in}{3.900000in}}%
\pgfusepath{clip}%
\pgfsetbuttcap%
\pgfsetroundjoin%
\definecolor{currentfill}{rgb}{0.121569,0.466667,0.705882}%
\pgfsetfillcolor{currentfill}%
\pgfsetlinewidth{1.003750pt}%
\definecolor{currentstroke}{rgb}{0.121569,0.466667,0.705882}%
\pgfsetstrokecolor{currentstroke}%
\pgfsetdash{}{0pt}%
\pgfpathmoveto{\pgfqpoint{2.807887in}{1.965453in}}%
\pgfpathcurveto{\pgfqpoint{2.818937in}{1.965453in}}{\pgfqpoint{2.829536in}{1.969844in}}{\pgfqpoint{2.837350in}{1.977657in}}%
\pgfpathcurveto{\pgfqpoint{2.845164in}{1.985471in}}{\pgfqpoint{2.849554in}{1.996070in}}{\pgfqpoint{2.849554in}{2.007120in}}%
\pgfpathcurveto{\pgfqpoint{2.849554in}{2.018170in}}{\pgfqpoint{2.845164in}{2.028769in}}{\pgfqpoint{2.837350in}{2.036583in}}%
\pgfpathcurveto{\pgfqpoint{2.829536in}{2.044397in}}{\pgfqpoint{2.818937in}{2.048787in}}{\pgfqpoint{2.807887in}{2.048787in}}%
\pgfpathcurveto{\pgfqpoint{2.796837in}{2.048787in}}{\pgfqpoint{2.786238in}{2.044397in}}{\pgfqpoint{2.778424in}{2.036583in}}%
\pgfpathcurveto{\pgfqpoint{2.770611in}{2.028769in}}{\pgfqpoint{2.766220in}{2.018170in}}{\pgfqpoint{2.766220in}{2.007120in}}%
\pgfpathcurveto{\pgfqpoint{2.766220in}{1.996070in}}{\pgfqpoint{2.770611in}{1.985471in}}{\pgfqpoint{2.778424in}{1.977657in}}%
\pgfpathcurveto{\pgfqpoint{2.786238in}{1.969844in}}{\pgfqpoint{2.796837in}{1.965453in}}{\pgfqpoint{2.807887in}{1.965453in}}%
\pgfpathclose%
\pgfusepath{stroke,fill}%
\end{pgfscope}%
\begin{pgfscope}%
\pgfpathrectangle{\pgfqpoint{0.600000in}{0.600000in}}{\pgfqpoint{3.900000in}{3.900000in}}%
\pgfusepath{clip}%
\pgfsetbuttcap%
\pgfsetroundjoin%
\definecolor{currentfill}{rgb}{0.121569,0.466667,0.705882}%
\pgfsetfillcolor{currentfill}%
\pgfsetlinewidth{1.003750pt}%
\definecolor{currentstroke}{rgb}{0.121569,0.466667,0.705882}%
\pgfsetstrokecolor{currentstroke}%
\pgfsetdash{}{0pt}%
\pgfpathmoveto{\pgfqpoint{2.364132in}{2.032713in}}%
\pgfpathcurveto{\pgfqpoint{2.375182in}{2.032713in}}{\pgfqpoint{2.385781in}{2.037103in}}{\pgfqpoint{2.393594in}{2.044916in}}%
\pgfpathcurveto{\pgfqpoint{2.401408in}{2.052730in}}{\pgfqpoint{2.405798in}{2.063329in}}{\pgfqpoint{2.405798in}{2.074379in}}%
\pgfpathcurveto{\pgfqpoint{2.405798in}{2.085429in}}{\pgfqpoint{2.401408in}{2.096028in}}{\pgfqpoint{2.393594in}{2.103842in}}%
\pgfpathcurveto{\pgfqpoint{2.385781in}{2.111656in}}{\pgfqpoint{2.375182in}{2.116046in}}{\pgfqpoint{2.364132in}{2.116046in}}%
\pgfpathcurveto{\pgfqpoint{2.353082in}{2.116046in}}{\pgfqpoint{2.342483in}{2.111656in}}{\pgfqpoint{2.334669in}{2.103842in}}%
\pgfpathcurveto{\pgfqpoint{2.326855in}{2.096028in}}{\pgfqpoint{2.322465in}{2.085429in}}{\pgfqpoint{2.322465in}{2.074379in}}%
\pgfpathcurveto{\pgfqpoint{2.322465in}{2.063329in}}{\pgfqpoint{2.326855in}{2.052730in}}{\pgfqpoint{2.334669in}{2.044916in}}%
\pgfpathcurveto{\pgfqpoint{2.342483in}{2.037103in}}{\pgfqpoint{2.353082in}{2.032713in}}{\pgfqpoint{2.364132in}{2.032713in}}%
\pgfpathclose%
\pgfusepath{stroke,fill}%
\end{pgfscope}%
\begin{pgfscope}%
\pgfpathrectangle{\pgfqpoint{0.600000in}{0.600000in}}{\pgfqpoint{3.900000in}{3.900000in}}%
\pgfusepath{clip}%
\pgfsetbuttcap%
\pgfsetroundjoin%
\definecolor{currentfill}{rgb}{0.121569,0.466667,0.705882}%
\pgfsetfillcolor{currentfill}%
\pgfsetlinewidth{1.003750pt}%
\definecolor{currentstroke}{rgb}{0.121569,0.466667,0.705882}%
\pgfsetstrokecolor{currentstroke}%
\pgfsetdash{}{0pt}%
\pgfpathmoveto{\pgfqpoint{1.889523in}{1.778970in}}%
\pgfpathcurveto{\pgfqpoint{1.900573in}{1.778970in}}{\pgfqpoint{1.911172in}{1.783361in}}{\pgfqpoint{1.918986in}{1.791174in}}%
\pgfpathcurveto{\pgfqpoint{1.926800in}{1.798988in}}{\pgfqpoint{1.931190in}{1.809587in}}{\pgfqpoint{1.931190in}{1.820637in}}%
\pgfpathcurveto{\pgfqpoint{1.931190in}{1.831687in}}{\pgfqpoint{1.926800in}{1.842286in}}{\pgfqpoint{1.918986in}{1.850100in}}%
\pgfpathcurveto{\pgfqpoint{1.911172in}{1.857914in}}{\pgfqpoint{1.900573in}{1.862304in}}{\pgfqpoint{1.889523in}{1.862304in}}%
\pgfpathcurveto{\pgfqpoint{1.878473in}{1.862304in}}{\pgfqpoint{1.867874in}{1.857914in}}{\pgfqpoint{1.860060in}{1.850100in}}%
\pgfpathcurveto{\pgfqpoint{1.852247in}{1.842286in}}{\pgfqpoint{1.847857in}{1.831687in}}{\pgfqpoint{1.847857in}{1.820637in}}%
\pgfpathcurveto{\pgfqpoint{1.847857in}{1.809587in}}{\pgfqpoint{1.852247in}{1.798988in}}{\pgfqpoint{1.860060in}{1.791174in}}%
\pgfpathcurveto{\pgfqpoint{1.867874in}{1.783361in}}{\pgfqpoint{1.878473in}{1.778970in}}{\pgfqpoint{1.889523in}{1.778970in}}%
\pgfpathclose%
\pgfusepath{stroke,fill}%
\end{pgfscope}%
\begin{pgfscope}%
\pgfpathrectangle{\pgfqpoint{0.600000in}{0.600000in}}{\pgfqpoint{3.900000in}{3.900000in}}%
\pgfusepath{clip}%
\pgfsetbuttcap%
\pgfsetroundjoin%
\definecolor{currentfill}{rgb}{0.121569,0.466667,0.705882}%
\pgfsetfillcolor{currentfill}%
\pgfsetlinewidth{1.003750pt}%
\definecolor{currentstroke}{rgb}{0.121569,0.466667,0.705882}%
\pgfsetstrokecolor{currentstroke}%
\pgfsetdash{}{0pt}%
\pgfpathmoveto{\pgfqpoint{2.478568in}{2.610670in}}%
\pgfpathcurveto{\pgfqpoint{2.489618in}{2.610670in}}{\pgfqpoint{2.500217in}{2.615060in}}{\pgfqpoint{2.508031in}{2.622874in}}%
\pgfpathcurveto{\pgfqpoint{2.515844in}{2.630687in}}{\pgfqpoint{2.520235in}{2.641286in}}{\pgfqpoint{2.520235in}{2.652336in}}%
\pgfpathcurveto{\pgfqpoint{2.520235in}{2.663386in}}{\pgfqpoint{2.515844in}{2.673986in}}{\pgfqpoint{2.508031in}{2.681799in}}%
\pgfpathcurveto{\pgfqpoint{2.500217in}{2.689613in}}{\pgfqpoint{2.489618in}{2.694003in}}{\pgfqpoint{2.478568in}{2.694003in}}%
\pgfpathcurveto{\pgfqpoint{2.467518in}{2.694003in}}{\pgfqpoint{2.456919in}{2.689613in}}{\pgfqpoint{2.449105in}{2.681799in}}%
\pgfpathcurveto{\pgfqpoint{2.441292in}{2.673986in}}{\pgfqpoint{2.436901in}{2.663386in}}{\pgfqpoint{2.436901in}{2.652336in}}%
\pgfpathcurveto{\pgfqpoint{2.436901in}{2.641286in}}{\pgfqpoint{2.441292in}{2.630687in}}{\pgfqpoint{2.449105in}{2.622874in}}%
\pgfpathcurveto{\pgfqpoint{2.456919in}{2.615060in}}{\pgfqpoint{2.467518in}{2.610670in}}{\pgfqpoint{2.478568in}{2.610670in}}%
\pgfpathclose%
\pgfusepath{stroke,fill}%
\end{pgfscope}%
\begin{pgfscope}%
\pgfpathrectangle{\pgfqpoint{0.600000in}{0.600000in}}{\pgfqpoint{3.900000in}{3.900000in}}%
\pgfusepath{clip}%
\pgfsetbuttcap%
\pgfsetroundjoin%
\definecolor{currentfill}{rgb}{0.121569,0.466667,0.705882}%
\pgfsetfillcolor{currentfill}%
\pgfsetlinewidth{1.003750pt}%
\definecolor{currentstroke}{rgb}{0.121569,0.466667,0.705882}%
\pgfsetstrokecolor{currentstroke}%
\pgfsetdash{}{0pt}%
\pgfpathmoveto{\pgfqpoint{2.084686in}{2.413008in}}%
\pgfpathcurveto{\pgfqpoint{2.095736in}{2.413008in}}{\pgfqpoint{2.106335in}{2.417398in}}{\pgfqpoint{2.114149in}{2.425212in}}%
\pgfpathcurveto{\pgfqpoint{2.121963in}{2.433026in}}{\pgfqpoint{2.126353in}{2.443625in}}{\pgfqpoint{2.126353in}{2.454675in}}%
\pgfpathcurveto{\pgfqpoint{2.126353in}{2.465725in}}{\pgfqpoint{2.121963in}{2.476324in}}{\pgfqpoint{2.114149in}{2.484138in}}%
\pgfpathcurveto{\pgfqpoint{2.106335in}{2.491951in}}{\pgfqpoint{2.095736in}{2.496341in}}{\pgfqpoint{2.084686in}{2.496341in}}%
\pgfpathcurveto{\pgfqpoint{2.073636in}{2.496341in}}{\pgfqpoint{2.063037in}{2.491951in}}{\pgfqpoint{2.055223in}{2.484138in}}%
\pgfpathcurveto{\pgfqpoint{2.047410in}{2.476324in}}{\pgfqpoint{2.043020in}{2.465725in}}{\pgfqpoint{2.043020in}{2.454675in}}%
\pgfpathcurveto{\pgfqpoint{2.043020in}{2.443625in}}{\pgfqpoint{2.047410in}{2.433026in}}{\pgfqpoint{2.055223in}{2.425212in}}%
\pgfpathcurveto{\pgfqpoint{2.063037in}{2.417398in}}{\pgfqpoint{2.073636in}{2.413008in}}{\pgfqpoint{2.084686in}{2.413008in}}%
\pgfpathclose%
\pgfusepath{stroke,fill}%
\end{pgfscope}%
\begin{pgfscope}%
\pgfpathrectangle{\pgfqpoint{0.600000in}{0.600000in}}{\pgfqpoint{3.900000in}{3.900000in}}%
\pgfusepath{clip}%
\pgfsetbuttcap%
\pgfsetroundjoin%
\definecolor{currentfill}{rgb}{0.121569,0.466667,0.705882}%
\pgfsetfillcolor{currentfill}%
\pgfsetlinewidth{1.003750pt}%
\definecolor{currentstroke}{rgb}{0.121569,0.466667,0.705882}%
\pgfsetstrokecolor{currentstroke}%
\pgfsetdash{}{0pt}%
\pgfpathmoveto{\pgfqpoint{3.333619in}{2.765644in}}%
\pgfpathcurveto{\pgfqpoint{3.344669in}{2.765644in}}{\pgfqpoint{3.355268in}{2.770034in}}{\pgfqpoint{3.363082in}{2.777848in}}%
\pgfpathcurveto{\pgfqpoint{3.370896in}{2.785661in}}{\pgfqpoint{3.375286in}{2.796260in}}{\pgfqpoint{3.375286in}{2.807310in}}%
\pgfpathcurveto{\pgfqpoint{3.375286in}{2.818361in}}{\pgfqpoint{3.370896in}{2.828960in}}{\pgfqpoint{3.363082in}{2.836773in}}%
\pgfpathcurveto{\pgfqpoint{3.355268in}{2.844587in}}{\pgfqpoint{3.344669in}{2.848977in}}{\pgfqpoint{3.333619in}{2.848977in}}%
\pgfpathcurveto{\pgfqpoint{3.322569in}{2.848977in}}{\pgfqpoint{3.311970in}{2.844587in}}{\pgfqpoint{3.304156in}{2.836773in}}%
\pgfpathcurveto{\pgfqpoint{3.296343in}{2.828960in}}{\pgfqpoint{3.291952in}{2.818361in}}{\pgfqpoint{3.291952in}{2.807310in}}%
\pgfpathcurveto{\pgfqpoint{3.291952in}{2.796260in}}{\pgfqpoint{3.296343in}{2.785661in}}{\pgfqpoint{3.304156in}{2.777848in}}%
\pgfpathcurveto{\pgfqpoint{3.311970in}{2.770034in}}{\pgfqpoint{3.322569in}{2.765644in}}{\pgfqpoint{3.333619in}{2.765644in}}%
\pgfpathclose%
\pgfusepath{stroke,fill}%
\end{pgfscope}%
\begin{pgfscope}%
\pgfpathrectangle{\pgfqpoint{0.600000in}{0.600000in}}{\pgfqpoint{3.900000in}{3.900000in}}%
\pgfusepath{clip}%
\pgfsetbuttcap%
\pgfsetroundjoin%
\definecolor{currentfill}{rgb}{0.121569,0.466667,0.705882}%
\pgfsetfillcolor{currentfill}%
\pgfsetlinewidth{1.003750pt}%
\definecolor{currentstroke}{rgb}{0.121569,0.466667,0.705882}%
\pgfsetstrokecolor{currentstroke}%
\pgfsetdash{}{0pt}%
\pgfpathmoveto{\pgfqpoint{2.479625in}{2.701349in}}%
\pgfpathcurveto{\pgfqpoint{2.490675in}{2.701349in}}{\pgfqpoint{2.501274in}{2.705739in}}{\pgfqpoint{2.509087in}{2.713553in}}%
\pgfpathcurveto{\pgfqpoint{2.516901in}{2.721366in}}{\pgfqpoint{2.521291in}{2.731965in}}{\pgfqpoint{2.521291in}{2.743015in}}%
\pgfpathcurveto{\pgfqpoint{2.521291in}{2.754066in}}{\pgfqpoint{2.516901in}{2.764665in}}{\pgfqpoint{2.509087in}{2.772478in}}%
\pgfpathcurveto{\pgfqpoint{2.501274in}{2.780292in}}{\pgfqpoint{2.490675in}{2.784682in}}{\pgfqpoint{2.479625in}{2.784682in}}%
\pgfpathcurveto{\pgfqpoint{2.468574in}{2.784682in}}{\pgfqpoint{2.457975in}{2.780292in}}{\pgfqpoint{2.450162in}{2.772478in}}%
\pgfpathcurveto{\pgfqpoint{2.442348in}{2.764665in}}{\pgfqpoint{2.437958in}{2.754066in}}{\pgfqpoint{2.437958in}{2.743015in}}%
\pgfpathcurveto{\pgfqpoint{2.437958in}{2.731965in}}{\pgfqpoint{2.442348in}{2.721366in}}{\pgfqpoint{2.450162in}{2.713553in}}%
\pgfpathcurveto{\pgfqpoint{2.457975in}{2.705739in}}{\pgfqpoint{2.468574in}{2.701349in}}{\pgfqpoint{2.479625in}{2.701349in}}%
\pgfpathclose%
\pgfusepath{stroke,fill}%
\end{pgfscope}%
\begin{pgfscope}%
\pgfpathrectangle{\pgfqpoint{0.600000in}{0.600000in}}{\pgfqpoint{3.900000in}{3.900000in}}%
\pgfusepath{clip}%
\pgfsetbuttcap%
\pgfsetroundjoin%
\definecolor{currentfill}{rgb}{0.121569,0.466667,0.705882}%
\pgfsetfillcolor{currentfill}%
\pgfsetlinewidth{1.003750pt}%
\definecolor{currentstroke}{rgb}{0.121569,0.466667,0.705882}%
\pgfsetstrokecolor{currentstroke}%
\pgfsetdash{}{0pt}%
\pgfpathmoveto{\pgfqpoint{2.151420in}{2.904175in}}%
\pgfpathcurveto{\pgfqpoint{2.162470in}{2.904175in}}{\pgfqpoint{2.173069in}{2.908565in}}{\pgfqpoint{2.180883in}{2.916379in}}%
\pgfpathcurveto{\pgfqpoint{2.188697in}{2.924193in}}{\pgfqpoint{2.193087in}{2.934792in}}{\pgfqpoint{2.193087in}{2.945842in}}%
\pgfpathcurveto{\pgfqpoint{2.193087in}{2.956892in}}{\pgfqpoint{2.188697in}{2.967491in}}{\pgfqpoint{2.180883in}{2.975305in}}%
\pgfpathcurveto{\pgfqpoint{2.173069in}{2.983118in}}{\pgfqpoint{2.162470in}{2.987509in}}{\pgfqpoint{2.151420in}{2.987509in}}%
\pgfpathcurveto{\pgfqpoint{2.140370in}{2.987509in}}{\pgfqpoint{2.129771in}{2.983118in}}{\pgfqpoint{2.121957in}{2.975305in}}%
\pgfpathcurveto{\pgfqpoint{2.114144in}{2.967491in}}{\pgfqpoint{2.109753in}{2.956892in}}{\pgfqpoint{2.109753in}{2.945842in}}%
\pgfpathcurveto{\pgfqpoint{2.109753in}{2.934792in}}{\pgfqpoint{2.114144in}{2.924193in}}{\pgfqpoint{2.121957in}{2.916379in}}%
\pgfpathcurveto{\pgfqpoint{2.129771in}{2.908565in}}{\pgfqpoint{2.140370in}{2.904175in}}{\pgfqpoint{2.151420in}{2.904175in}}%
\pgfpathclose%
\pgfusepath{stroke,fill}%
\end{pgfscope}%
\begin{pgfscope}%
\pgfpathrectangle{\pgfqpoint{0.600000in}{0.600000in}}{\pgfqpoint{3.900000in}{3.900000in}}%
\pgfusepath{clip}%
\pgfsetbuttcap%
\pgfsetroundjoin%
\definecolor{currentfill}{rgb}{0.121569,0.466667,0.705882}%
\pgfsetfillcolor{currentfill}%
\pgfsetlinewidth{1.003750pt}%
\definecolor{currentstroke}{rgb}{0.121569,0.466667,0.705882}%
\pgfsetstrokecolor{currentstroke}%
\pgfsetdash{}{0pt}%
\pgfpathmoveto{\pgfqpoint{1.580430in}{3.014733in}}%
\pgfpathcurveto{\pgfqpoint{1.591481in}{3.014733in}}{\pgfqpoint{1.602080in}{3.019123in}}{\pgfqpoint{1.609893in}{3.026937in}}%
\pgfpathcurveto{\pgfqpoint{1.617707in}{3.034751in}}{\pgfqpoint{1.622097in}{3.045350in}}{\pgfqpoint{1.622097in}{3.056400in}}%
\pgfpathcurveto{\pgfqpoint{1.622097in}{3.067450in}}{\pgfqpoint{1.617707in}{3.078049in}}{\pgfqpoint{1.609893in}{3.085863in}}%
\pgfpathcurveto{\pgfqpoint{1.602080in}{3.093676in}}{\pgfqpoint{1.591481in}{3.098066in}}{\pgfqpoint{1.580430in}{3.098066in}}%
\pgfpathcurveto{\pgfqpoint{1.569380in}{3.098066in}}{\pgfqpoint{1.558781in}{3.093676in}}{\pgfqpoint{1.550968in}{3.085863in}}%
\pgfpathcurveto{\pgfqpoint{1.543154in}{3.078049in}}{\pgfqpoint{1.538764in}{3.067450in}}{\pgfqpoint{1.538764in}{3.056400in}}%
\pgfpathcurveto{\pgfqpoint{1.538764in}{3.045350in}}{\pgfqpoint{1.543154in}{3.034751in}}{\pgfqpoint{1.550968in}{3.026937in}}%
\pgfpathcurveto{\pgfqpoint{1.558781in}{3.019123in}}{\pgfqpoint{1.569380in}{3.014733in}}{\pgfqpoint{1.580430in}{3.014733in}}%
\pgfpathclose%
\pgfusepath{stroke,fill}%
\end{pgfscope}%
\begin{pgfscope}%
\pgfpathrectangle{\pgfqpoint{0.600000in}{0.600000in}}{\pgfqpoint{3.900000in}{3.900000in}}%
\pgfusepath{clip}%
\pgfsetbuttcap%
\pgfsetroundjoin%
\definecolor{currentfill}{rgb}{0.121569,0.466667,0.705882}%
\pgfsetfillcolor{currentfill}%
\pgfsetlinewidth{1.003750pt}%
\definecolor{currentstroke}{rgb}{0.121569,0.466667,0.705882}%
\pgfsetstrokecolor{currentstroke}%
\pgfsetdash{}{0pt}%
\pgfpathmoveto{\pgfqpoint{2.252223in}{1.982998in}}%
\pgfpathcurveto{\pgfqpoint{2.263273in}{1.982998in}}{\pgfqpoint{2.273872in}{1.987388in}}{\pgfqpoint{2.281686in}{1.995202in}}%
\pgfpathcurveto{\pgfqpoint{2.289499in}{2.003016in}}{\pgfqpoint{2.293890in}{2.013615in}}{\pgfqpoint{2.293890in}{2.024665in}}%
\pgfpathcurveto{\pgfqpoint{2.293890in}{2.035715in}}{\pgfqpoint{2.289499in}{2.046314in}}{\pgfqpoint{2.281686in}{2.054128in}}%
\pgfpathcurveto{\pgfqpoint{2.273872in}{2.061941in}}{\pgfqpoint{2.263273in}{2.066331in}}{\pgfqpoint{2.252223in}{2.066331in}}%
\pgfpathcurveto{\pgfqpoint{2.241173in}{2.066331in}}{\pgfqpoint{2.230574in}{2.061941in}}{\pgfqpoint{2.222760in}{2.054128in}}%
\pgfpathcurveto{\pgfqpoint{2.214947in}{2.046314in}}{\pgfqpoint{2.210556in}{2.035715in}}{\pgfqpoint{2.210556in}{2.024665in}}%
\pgfpathcurveto{\pgfqpoint{2.210556in}{2.013615in}}{\pgfqpoint{2.214947in}{2.003016in}}{\pgfqpoint{2.222760in}{1.995202in}}%
\pgfpathcurveto{\pgfqpoint{2.230574in}{1.987388in}}{\pgfqpoint{2.241173in}{1.982998in}}{\pgfqpoint{2.252223in}{1.982998in}}%
\pgfpathclose%
\pgfusepath{stroke,fill}%
\end{pgfscope}%
\begin{pgfscope}%
\pgfpathrectangle{\pgfqpoint{0.600000in}{0.600000in}}{\pgfqpoint{3.900000in}{3.900000in}}%
\pgfusepath{clip}%
\pgfsetbuttcap%
\pgfsetroundjoin%
\definecolor{currentfill}{rgb}{0.121569,0.466667,0.705882}%
\pgfsetfillcolor{currentfill}%
\pgfsetlinewidth{1.003750pt}%
\definecolor{currentstroke}{rgb}{0.121569,0.466667,0.705882}%
\pgfsetstrokecolor{currentstroke}%
\pgfsetdash{}{0pt}%
\pgfpathmoveto{\pgfqpoint{2.245346in}{2.800353in}}%
\pgfpathcurveto{\pgfqpoint{2.256397in}{2.800353in}}{\pgfqpoint{2.266996in}{2.804743in}}{\pgfqpoint{2.274809in}{2.812557in}}%
\pgfpathcurveto{\pgfqpoint{2.282623in}{2.820370in}}{\pgfqpoint{2.287013in}{2.830969in}}{\pgfqpoint{2.287013in}{2.842019in}}%
\pgfpathcurveto{\pgfqpoint{2.287013in}{2.853070in}}{\pgfqpoint{2.282623in}{2.863669in}}{\pgfqpoint{2.274809in}{2.871482in}}%
\pgfpathcurveto{\pgfqpoint{2.266996in}{2.879296in}}{\pgfqpoint{2.256397in}{2.883686in}}{\pgfqpoint{2.245346in}{2.883686in}}%
\pgfpathcurveto{\pgfqpoint{2.234296in}{2.883686in}}{\pgfqpoint{2.223697in}{2.879296in}}{\pgfqpoint{2.215884in}{2.871482in}}%
\pgfpathcurveto{\pgfqpoint{2.208070in}{2.863669in}}{\pgfqpoint{2.203680in}{2.853070in}}{\pgfqpoint{2.203680in}{2.842019in}}%
\pgfpathcurveto{\pgfqpoint{2.203680in}{2.830969in}}{\pgfqpoint{2.208070in}{2.820370in}}{\pgfqpoint{2.215884in}{2.812557in}}%
\pgfpathcurveto{\pgfqpoint{2.223697in}{2.804743in}}{\pgfqpoint{2.234296in}{2.800353in}}{\pgfqpoint{2.245346in}{2.800353in}}%
\pgfpathclose%
\pgfusepath{stroke,fill}%
\end{pgfscope}%
\begin{pgfscope}%
\pgfpathrectangle{\pgfqpoint{0.600000in}{0.600000in}}{\pgfqpoint{3.900000in}{3.900000in}}%
\pgfusepath{clip}%
\pgfsetbuttcap%
\pgfsetroundjoin%
\definecolor{currentfill}{rgb}{0.121569,0.466667,0.705882}%
\pgfsetfillcolor{currentfill}%
\pgfsetlinewidth{1.003750pt}%
\definecolor{currentstroke}{rgb}{0.121569,0.466667,0.705882}%
\pgfsetstrokecolor{currentstroke}%
\pgfsetdash{}{0pt}%
\pgfpathmoveto{\pgfqpoint{3.280393in}{1.909339in}}%
\pgfpathcurveto{\pgfqpoint{3.291443in}{1.909339in}}{\pgfqpoint{3.302042in}{1.913730in}}{\pgfqpoint{3.309856in}{1.921543in}}%
\pgfpathcurveto{\pgfqpoint{3.317670in}{1.929357in}}{\pgfqpoint{3.322060in}{1.939956in}}{\pgfqpoint{3.322060in}{1.951006in}}%
\pgfpathcurveto{\pgfqpoint{3.322060in}{1.962056in}}{\pgfqpoint{3.317670in}{1.972655in}}{\pgfqpoint{3.309856in}{1.980469in}}%
\pgfpathcurveto{\pgfqpoint{3.302042in}{1.988282in}}{\pgfqpoint{3.291443in}{1.992673in}}{\pgfqpoint{3.280393in}{1.992673in}}%
\pgfpathcurveto{\pgfqpoint{3.269343in}{1.992673in}}{\pgfqpoint{3.258744in}{1.988282in}}{\pgfqpoint{3.250930in}{1.980469in}}%
\pgfpathcurveto{\pgfqpoint{3.243117in}{1.972655in}}{\pgfqpoint{3.238726in}{1.962056in}}{\pgfqpoint{3.238726in}{1.951006in}}%
\pgfpathcurveto{\pgfqpoint{3.238726in}{1.939956in}}{\pgfqpoint{3.243117in}{1.929357in}}{\pgfqpoint{3.250930in}{1.921543in}}%
\pgfpathcurveto{\pgfqpoint{3.258744in}{1.913730in}}{\pgfqpoint{3.269343in}{1.909339in}}{\pgfqpoint{3.280393in}{1.909339in}}%
\pgfpathclose%
\pgfusepath{stroke,fill}%
\end{pgfscope}%
\begin{pgfscope}%
\pgfpathrectangle{\pgfqpoint{0.600000in}{0.600000in}}{\pgfqpoint{3.900000in}{3.900000in}}%
\pgfusepath{clip}%
\pgfsetbuttcap%
\pgfsetroundjoin%
\definecolor{currentfill}{rgb}{0.121569,0.466667,0.705882}%
\pgfsetfillcolor{currentfill}%
\pgfsetlinewidth{1.003750pt}%
\definecolor{currentstroke}{rgb}{0.121569,0.466667,0.705882}%
\pgfsetstrokecolor{currentstroke}%
\pgfsetdash{}{0pt}%
\pgfpathmoveto{\pgfqpoint{2.672420in}{2.689499in}}%
\pgfpathcurveto{\pgfqpoint{2.683470in}{2.689499in}}{\pgfqpoint{2.694069in}{2.693889in}}{\pgfqpoint{2.701883in}{2.701703in}}%
\pgfpathcurveto{\pgfqpoint{2.709696in}{2.709517in}}{\pgfqpoint{2.714087in}{2.720116in}}{\pgfqpoint{2.714087in}{2.731166in}}%
\pgfpathcurveto{\pgfqpoint{2.714087in}{2.742216in}}{\pgfqpoint{2.709696in}{2.752815in}}{\pgfqpoint{2.701883in}{2.760629in}}%
\pgfpathcurveto{\pgfqpoint{2.694069in}{2.768442in}}{\pgfqpoint{2.683470in}{2.772832in}}{\pgfqpoint{2.672420in}{2.772832in}}%
\pgfpathcurveto{\pgfqpoint{2.661370in}{2.772832in}}{\pgfqpoint{2.650771in}{2.768442in}}{\pgfqpoint{2.642957in}{2.760629in}}%
\pgfpathcurveto{\pgfqpoint{2.635143in}{2.752815in}}{\pgfqpoint{2.630753in}{2.742216in}}{\pgfqpoint{2.630753in}{2.731166in}}%
\pgfpathcurveto{\pgfqpoint{2.630753in}{2.720116in}}{\pgfqpoint{2.635143in}{2.709517in}}{\pgfqpoint{2.642957in}{2.701703in}}%
\pgfpathcurveto{\pgfqpoint{2.650771in}{2.693889in}}{\pgfqpoint{2.661370in}{2.689499in}}{\pgfqpoint{2.672420in}{2.689499in}}%
\pgfpathclose%
\pgfusepath{stroke,fill}%
\end{pgfscope}%
\begin{pgfscope}%
\pgfpathrectangle{\pgfqpoint{0.600000in}{0.600000in}}{\pgfqpoint{3.900000in}{3.900000in}}%
\pgfusepath{clip}%
\pgfsetbuttcap%
\pgfsetroundjoin%
\definecolor{currentfill}{rgb}{0.121569,0.466667,0.705882}%
\pgfsetfillcolor{currentfill}%
\pgfsetlinewidth{1.003750pt}%
\definecolor{currentstroke}{rgb}{0.121569,0.466667,0.705882}%
\pgfsetstrokecolor{currentstroke}%
\pgfsetdash{}{0pt}%
\pgfpathmoveto{\pgfqpoint{2.756705in}{2.671806in}}%
\pgfpathcurveto{\pgfqpoint{2.767755in}{2.671806in}}{\pgfqpoint{2.778354in}{2.676197in}}{\pgfqpoint{2.786167in}{2.684010in}}%
\pgfpathcurveto{\pgfqpoint{2.793981in}{2.691824in}}{\pgfqpoint{2.798371in}{2.702423in}}{\pgfqpoint{2.798371in}{2.713473in}}%
\pgfpathcurveto{\pgfqpoint{2.798371in}{2.724523in}}{\pgfqpoint{2.793981in}{2.735122in}}{\pgfqpoint{2.786167in}{2.742936in}}%
\pgfpathcurveto{\pgfqpoint{2.778354in}{2.750750in}}{\pgfqpoint{2.767755in}{2.755140in}}{\pgfqpoint{2.756705in}{2.755140in}}%
\pgfpathcurveto{\pgfqpoint{2.745654in}{2.755140in}}{\pgfqpoint{2.735055in}{2.750750in}}{\pgfqpoint{2.727242in}{2.742936in}}%
\pgfpathcurveto{\pgfqpoint{2.719428in}{2.735122in}}{\pgfqpoint{2.715038in}{2.724523in}}{\pgfqpoint{2.715038in}{2.713473in}}%
\pgfpathcurveto{\pgfqpoint{2.715038in}{2.702423in}}{\pgfqpoint{2.719428in}{2.691824in}}{\pgfqpoint{2.727242in}{2.684010in}}%
\pgfpathcurveto{\pgfqpoint{2.735055in}{2.676197in}}{\pgfqpoint{2.745654in}{2.671806in}}{\pgfqpoint{2.756705in}{2.671806in}}%
\pgfpathclose%
\pgfusepath{stroke,fill}%
\end{pgfscope}%
\begin{pgfscope}%
\pgfpathrectangle{\pgfqpoint{0.600000in}{0.600000in}}{\pgfqpoint{3.900000in}{3.900000in}}%
\pgfusepath{clip}%
\pgfsetbuttcap%
\pgfsetroundjoin%
\definecolor{currentfill}{rgb}{0.121569,0.466667,0.705882}%
\pgfsetfillcolor{currentfill}%
\pgfsetlinewidth{1.003750pt}%
\definecolor{currentstroke}{rgb}{0.121569,0.466667,0.705882}%
\pgfsetstrokecolor{currentstroke}%
\pgfsetdash{}{0pt}%
\pgfpathmoveto{\pgfqpoint{3.280657in}{2.529688in}}%
\pgfpathcurveto{\pgfqpoint{3.291707in}{2.529688in}}{\pgfqpoint{3.302306in}{2.534078in}}{\pgfqpoint{3.310120in}{2.541892in}}%
\pgfpathcurveto{\pgfqpoint{3.317933in}{2.549705in}}{\pgfqpoint{3.322324in}{2.560304in}}{\pgfqpoint{3.322324in}{2.571354in}}%
\pgfpathcurveto{\pgfqpoint{3.322324in}{2.582405in}}{\pgfqpoint{3.317933in}{2.593004in}}{\pgfqpoint{3.310120in}{2.600817in}}%
\pgfpathcurveto{\pgfqpoint{3.302306in}{2.608631in}}{\pgfqpoint{3.291707in}{2.613021in}}{\pgfqpoint{3.280657in}{2.613021in}}%
\pgfpathcurveto{\pgfqpoint{3.269607in}{2.613021in}}{\pgfqpoint{3.259008in}{2.608631in}}{\pgfqpoint{3.251194in}{2.600817in}}%
\pgfpathcurveto{\pgfqpoint{3.243380in}{2.593004in}}{\pgfqpoint{3.238990in}{2.582405in}}{\pgfqpoint{3.238990in}{2.571354in}}%
\pgfpathcurveto{\pgfqpoint{3.238990in}{2.560304in}}{\pgfqpoint{3.243380in}{2.549705in}}{\pgfqpoint{3.251194in}{2.541892in}}%
\pgfpathcurveto{\pgfqpoint{3.259008in}{2.534078in}}{\pgfqpoint{3.269607in}{2.529688in}}{\pgfqpoint{3.280657in}{2.529688in}}%
\pgfpathclose%
\pgfusepath{stroke,fill}%
\end{pgfscope}%
\begin{pgfscope}%
\pgfpathrectangle{\pgfqpoint{0.600000in}{0.600000in}}{\pgfqpoint{3.900000in}{3.900000in}}%
\pgfusepath{clip}%
\pgfsetbuttcap%
\pgfsetroundjoin%
\definecolor{currentfill}{rgb}{0.121569,0.466667,0.705882}%
\pgfsetfillcolor{currentfill}%
\pgfsetlinewidth{1.003750pt}%
\definecolor{currentstroke}{rgb}{0.121569,0.466667,0.705882}%
\pgfsetstrokecolor{currentstroke}%
\pgfsetdash{}{0pt}%
\pgfpathmoveto{\pgfqpoint{2.456787in}{2.272028in}}%
\pgfpathcurveto{\pgfqpoint{2.467837in}{2.272028in}}{\pgfqpoint{2.478436in}{2.276418in}}{\pgfqpoint{2.486249in}{2.284232in}}%
\pgfpathcurveto{\pgfqpoint{2.494063in}{2.292045in}}{\pgfqpoint{2.498453in}{2.302644in}}{\pgfqpoint{2.498453in}{2.313694in}}%
\pgfpathcurveto{\pgfqpoint{2.498453in}{2.324745in}}{\pgfqpoint{2.494063in}{2.335344in}}{\pgfqpoint{2.486249in}{2.343157in}}%
\pgfpathcurveto{\pgfqpoint{2.478436in}{2.350971in}}{\pgfqpoint{2.467837in}{2.355361in}}{\pgfqpoint{2.456787in}{2.355361in}}%
\pgfpathcurveto{\pgfqpoint{2.445737in}{2.355361in}}{\pgfqpoint{2.435138in}{2.350971in}}{\pgfqpoint{2.427324in}{2.343157in}}%
\pgfpathcurveto{\pgfqpoint{2.419510in}{2.335344in}}{\pgfqpoint{2.415120in}{2.324745in}}{\pgfqpoint{2.415120in}{2.313694in}}%
\pgfpathcurveto{\pgfqpoint{2.415120in}{2.302644in}}{\pgfqpoint{2.419510in}{2.292045in}}{\pgfqpoint{2.427324in}{2.284232in}}%
\pgfpathcurveto{\pgfqpoint{2.435138in}{2.276418in}}{\pgfqpoint{2.445737in}{2.272028in}}{\pgfqpoint{2.456787in}{2.272028in}}%
\pgfpathclose%
\pgfusepath{stroke,fill}%
\end{pgfscope}%
\begin{pgfscope}%
\pgfpathrectangle{\pgfqpoint{0.600000in}{0.600000in}}{\pgfqpoint{3.900000in}{3.900000in}}%
\pgfusepath{clip}%
\pgfsetbuttcap%
\pgfsetroundjoin%
\definecolor{currentfill}{rgb}{0.121569,0.466667,0.705882}%
\pgfsetfillcolor{currentfill}%
\pgfsetlinewidth{1.003750pt}%
\definecolor{currentstroke}{rgb}{0.121569,0.466667,0.705882}%
\pgfsetstrokecolor{currentstroke}%
\pgfsetdash{}{0pt}%
\pgfpathmoveto{\pgfqpoint{2.441897in}{2.784155in}}%
\pgfpathcurveto{\pgfqpoint{2.452948in}{2.784155in}}{\pgfqpoint{2.463547in}{2.788545in}}{\pgfqpoint{2.471360in}{2.796358in}}%
\pgfpathcurveto{\pgfqpoint{2.479174in}{2.804172in}}{\pgfqpoint{2.483564in}{2.814771in}}{\pgfqpoint{2.483564in}{2.825821in}}%
\pgfpathcurveto{\pgfqpoint{2.483564in}{2.836871in}}{\pgfqpoint{2.479174in}{2.847470in}}{\pgfqpoint{2.471360in}{2.855284in}}%
\pgfpathcurveto{\pgfqpoint{2.463547in}{2.863098in}}{\pgfqpoint{2.452948in}{2.867488in}}{\pgfqpoint{2.441897in}{2.867488in}}%
\pgfpathcurveto{\pgfqpoint{2.430847in}{2.867488in}}{\pgfqpoint{2.420248in}{2.863098in}}{\pgfqpoint{2.412435in}{2.855284in}}%
\pgfpathcurveto{\pgfqpoint{2.404621in}{2.847470in}}{\pgfqpoint{2.400231in}{2.836871in}}{\pgfqpoint{2.400231in}{2.825821in}}%
\pgfpathcurveto{\pgfqpoint{2.400231in}{2.814771in}}{\pgfqpoint{2.404621in}{2.804172in}}{\pgfqpoint{2.412435in}{2.796358in}}%
\pgfpathcurveto{\pgfqpoint{2.420248in}{2.788545in}}{\pgfqpoint{2.430847in}{2.784155in}}{\pgfqpoint{2.441897in}{2.784155in}}%
\pgfpathclose%
\pgfusepath{stroke,fill}%
\end{pgfscope}%
\begin{pgfscope}%
\pgfpathrectangle{\pgfqpoint{0.600000in}{0.600000in}}{\pgfqpoint{3.900000in}{3.900000in}}%
\pgfusepath{clip}%
\pgfsetbuttcap%
\pgfsetroundjoin%
\definecolor{currentfill}{rgb}{0.121569,0.466667,0.705882}%
\pgfsetfillcolor{currentfill}%
\pgfsetlinewidth{1.003750pt}%
\definecolor{currentstroke}{rgb}{0.121569,0.466667,0.705882}%
\pgfsetstrokecolor{currentstroke}%
\pgfsetdash{}{0pt}%
\pgfpathmoveto{\pgfqpoint{2.979534in}{2.504882in}}%
\pgfpathcurveto{\pgfqpoint{2.990584in}{2.504882in}}{\pgfqpoint{3.001184in}{2.509273in}}{\pgfqpoint{3.008997in}{2.517086in}}%
\pgfpathcurveto{\pgfqpoint{3.016811in}{2.524900in}}{\pgfqpoint{3.021201in}{2.535499in}}{\pgfqpoint{3.021201in}{2.546549in}}%
\pgfpathcurveto{\pgfqpoint{3.021201in}{2.557599in}}{\pgfqpoint{3.016811in}{2.568198in}}{\pgfqpoint{3.008997in}{2.576012in}}%
\pgfpathcurveto{\pgfqpoint{3.001184in}{2.583825in}}{\pgfqpoint{2.990584in}{2.588216in}}{\pgfqpoint{2.979534in}{2.588216in}}%
\pgfpathcurveto{\pgfqpoint{2.968484in}{2.588216in}}{\pgfqpoint{2.957885in}{2.583825in}}{\pgfqpoint{2.950072in}{2.576012in}}%
\pgfpathcurveto{\pgfqpoint{2.942258in}{2.568198in}}{\pgfqpoint{2.937868in}{2.557599in}}{\pgfqpoint{2.937868in}{2.546549in}}%
\pgfpathcurveto{\pgfqpoint{2.937868in}{2.535499in}}{\pgfqpoint{2.942258in}{2.524900in}}{\pgfqpoint{2.950072in}{2.517086in}}%
\pgfpathcurveto{\pgfqpoint{2.957885in}{2.509273in}}{\pgfqpoint{2.968484in}{2.504882in}}{\pgfqpoint{2.979534in}{2.504882in}}%
\pgfpathclose%
\pgfusepath{stroke,fill}%
\end{pgfscope}%
\begin{pgfscope}%
\pgfpathrectangle{\pgfqpoint{0.600000in}{0.600000in}}{\pgfqpoint{3.900000in}{3.900000in}}%
\pgfusepath{clip}%
\pgfsetbuttcap%
\pgfsetroundjoin%
\definecolor{currentfill}{rgb}{0.121569,0.466667,0.705882}%
\pgfsetfillcolor{currentfill}%
\pgfsetlinewidth{1.003750pt}%
\definecolor{currentstroke}{rgb}{0.121569,0.466667,0.705882}%
\pgfsetstrokecolor{currentstroke}%
\pgfsetdash{}{0pt}%
\pgfpathmoveto{\pgfqpoint{1.895503in}{2.595494in}}%
\pgfpathcurveto{\pgfqpoint{1.906553in}{2.595494in}}{\pgfqpoint{1.917152in}{2.599885in}}{\pgfqpoint{1.924966in}{2.607698in}}%
\pgfpathcurveto{\pgfqpoint{1.932780in}{2.615512in}}{\pgfqpoint{1.937170in}{2.626111in}}{\pgfqpoint{1.937170in}{2.637161in}}%
\pgfpathcurveto{\pgfqpoint{1.937170in}{2.648211in}}{\pgfqpoint{1.932780in}{2.658810in}}{\pgfqpoint{1.924966in}{2.666624in}}%
\pgfpathcurveto{\pgfqpoint{1.917152in}{2.674437in}}{\pgfqpoint{1.906553in}{2.678828in}}{\pgfqpoint{1.895503in}{2.678828in}}%
\pgfpathcurveto{\pgfqpoint{1.884453in}{2.678828in}}{\pgfqpoint{1.873854in}{2.674437in}}{\pgfqpoint{1.866041in}{2.666624in}}%
\pgfpathcurveto{\pgfqpoint{1.858227in}{2.658810in}}{\pgfqpoint{1.853837in}{2.648211in}}{\pgfqpoint{1.853837in}{2.637161in}}%
\pgfpathcurveto{\pgfqpoint{1.853837in}{2.626111in}}{\pgfqpoint{1.858227in}{2.615512in}}{\pgfqpoint{1.866041in}{2.607698in}}%
\pgfpathcurveto{\pgfqpoint{1.873854in}{2.599885in}}{\pgfqpoint{1.884453in}{2.595494in}}{\pgfqpoint{1.895503in}{2.595494in}}%
\pgfpathclose%
\pgfusepath{stroke,fill}%
\end{pgfscope}%
\begin{pgfscope}%
\pgfpathrectangle{\pgfqpoint{0.600000in}{0.600000in}}{\pgfqpoint{3.900000in}{3.900000in}}%
\pgfusepath{clip}%
\pgfsetbuttcap%
\pgfsetroundjoin%
\definecolor{currentfill}{rgb}{0.121569,0.466667,0.705882}%
\pgfsetfillcolor{currentfill}%
\pgfsetlinewidth{1.003750pt}%
\definecolor{currentstroke}{rgb}{0.121569,0.466667,0.705882}%
\pgfsetstrokecolor{currentstroke}%
\pgfsetdash{}{0pt}%
\pgfpathmoveto{\pgfqpoint{2.345783in}{2.386078in}}%
\pgfpathcurveto{\pgfqpoint{2.356833in}{2.386078in}}{\pgfqpoint{2.367433in}{2.390469in}}{\pgfqpoint{2.375246in}{2.398282in}}%
\pgfpathcurveto{\pgfqpoint{2.383060in}{2.406096in}}{\pgfqpoint{2.387450in}{2.416695in}}{\pgfqpoint{2.387450in}{2.427745in}}%
\pgfpathcurveto{\pgfqpoint{2.387450in}{2.438795in}}{\pgfqpoint{2.383060in}{2.449394in}}{\pgfqpoint{2.375246in}{2.457208in}}%
\pgfpathcurveto{\pgfqpoint{2.367433in}{2.465021in}}{\pgfqpoint{2.356833in}{2.469412in}}{\pgfqpoint{2.345783in}{2.469412in}}%
\pgfpathcurveto{\pgfqpoint{2.334733in}{2.469412in}}{\pgfqpoint{2.324134in}{2.465021in}}{\pgfqpoint{2.316321in}{2.457208in}}%
\pgfpathcurveto{\pgfqpoint{2.308507in}{2.449394in}}{\pgfqpoint{2.304117in}{2.438795in}}{\pgfqpoint{2.304117in}{2.427745in}}%
\pgfpathcurveto{\pgfqpoint{2.304117in}{2.416695in}}{\pgfqpoint{2.308507in}{2.406096in}}{\pgfqpoint{2.316321in}{2.398282in}}%
\pgfpathcurveto{\pgfqpoint{2.324134in}{2.390469in}}{\pgfqpoint{2.334733in}{2.386078in}}{\pgfqpoint{2.345783in}{2.386078in}}%
\pgfpathclose%
\pgfusepath{stroke,fill}%
\end{pgfscope}%
\begin{pgfscope}%
\pgfpathrectangle{\pgfqpoint{0.600000in}{0.600000in}}{\pgfqpoint{3.900000in}{3.900000in}}%
\pgfusepath{clip}%
\pgfsetbuttcap%
\pgfsetroundjoin%
\definecolor{currentfill}{rgb}{0.121569,0.466667,0.705882}%
\pgfsetfillcolor{currentfill}%
\pgfsetlinewidth{1.003750pt}%
\definecolor{currentstroke}{rgb}{0.121569,0.466667,0.705882}%
\pgfsetstrokecolor{currentstroke}%
\pgfsetdash{}{0pt}%
\pgfpathmoveto{\pgfqpoint{3.033553in}{2.790971in}}%
\pgfpathcurveto{\pgfqpoint{3.044603in}{2.790971in}}{\pgfqpoint{3.055202in}{2.795362in}}{\pgfqpoint{3.063015in}{2.803175in}}%
\pgfpathcurveto{\pgfqpoint{3.070829in}{2.810989in}}{\pgfqpoint{3.075219in}{2.821588in}}{\pgfqpoint{3.075219in}{2.832638in}}%
\pgfpathcurveto{\pgfqpoint{3.075219in}{2.843688in}}{\pgfqpoint{3.070829in}{2.854287in}}{\pgfqpoint{3.063015in}{2.862101in}}%
\pgfpathcurveto{\pgfqpoint{3.055202in}{2.869914in}}{\pgfqpoint{3.044603in}{2.874305in}}{\pgfqpoint{3.033553in}{2.874305in}}%
\pgfpathcurveto{\pgfqpoint{3.022502in}{2.874305in}}{\pgfqpoint{3.011903in}{2.869914in}}{\pgfqpoint{3.004090in}{2.862101in}}%
\pgfpathcurveto{\pgfqpoint{2.996276in}{2.854287in}}{\pgfqpoint{2.991886in}{2.843688in}}{\pgfqpoint{2.991886in}{2.832638in}}%
\pgfpathcurveto{\pgfqpoint{2.991886in}{2.821588in}}{\pgfqpoint{2.996276in}{2.810989in}}{\pgfqpoint{3.004090in}{2.803175in}}%
\pgfpathcurveto{\pgfqpoint{3.011903in}{2.795362in}}{\pgfqpoint{3.022502in}{2.790971in}}{\pgfqpoint{3.033553in}{2.790971in}}%
\pgfpathclose%
\pgfusepath{stroke,fill}%
\end{pgfscope}%
\begin{pgfscope}%
\pgfpathrectangle{\pgfqpoint{0.600000in}{0.600000in}}{\pgfqpoint{3.900000in}{3.900000in}}%
\pgfusepath{clip}%
\pgfsetbuttcap%
\pgfsetroundjoin%
\definecolor{currentfill}{rgb}{0.121569,0.466667,0.705882}%
\pgfsetfillcolor{currentfill}%
\pgfsetlinewidth{1.003750pt}%
\definecolor{currentstroke}{rgb}{0.121569,0.466667,0.705882}%
\pgfsetstrokecolor{currentstroke}%
\pgfsetdash{}{0pt}%
\pgfpathmoveto{\pgfqpoint{2.494061in}{2.573582in}}%
\pgfpathcurveto{\pgfqpoint{2.505111in}{2.573582in}}{\pgfqpoint{2.515710in}{2.577972in}}{\pgfqpoint{2.523524in}{2.585786in}}%
\pgfpathcurveto{\pgfqpoint{2.531338in}{2.593599in}}{\pgfqpoint{2.535728in}{2.604198in}}{\pgfqpoint{2.535728in}{2.615248in}}%
\pgfpathcurveto{\pgfqpoint{2.535728in}{2.626299in}}{\pgfqpoint{2.531338in}{2.636898in}}{\pgfqpoint{2.523524in}{2.644711in}}%
\pgfpathcurveto{\pgfqpoint{2.515710in}{2.652525in}}{\pgfqpoint{2.505111in}{2.656915in}}{\pgfqpoint{2.494061in}{2.656915in}}%
\pgfpathcurveto{\pgfqpoint{2.483011in}{2.656915in}}{\pgfqpoint{2.472412in}{2.652525in}}{\pgfqpoint{2.464598in}{2.644711in}}%
\pgfpathcurveto{\pgfqpoint{2.456785in}{2.636898in}}{\pgfqpoint{2.452395in}{2.626299in}}{\pgfqpoint{2.452395in}{2.615248in}}%
\pgfpathcurveto{\pgfqpoint{2.452395in}{2.604198in}}{\pgfqpoint{2.456785in}{2.593599in}}{\pgfqpoint{2.464598in}{2.585786in}}%
\pgfpathcurveto{\pgfqpoint{2.472412in}{2.577972in}}{\pgfqpoint{2.483011in}{2.573582in}}{\pgfqpoint{2.494061in}{2.573582in}}%
\pgfpathclose%
\pgfusepath{stroke,fill}%
\end{pgfscope}%
\begin{pgfscope}%
\pgfpathrectangle{\pgfqpoint{0.600000in}{0.600000in}}{\pgfqpoint{3.900000in}{3.900000in}}%
\pgfusepath{clip}%
\pgfsetbuttcap%
\pgfsetroundjoin%
\definecolor{currentfill}{rgb}{0.121569,0.466667,0.705882}%
\pgfsetfillcolor{currentfill}%
\pgfsetlinewidth{1.003750pt}%
\definecolor{currentstroke}{rgb}{0.121569,0.466667,0.705882}%
\pgfsetstrokecolor{currentstroke}%
\pgfsetdash{}{0pt}%
\pgfpathmoveto{\pgfqpoint{2.310597in}{3.447166in}}%
\pgfpathcurveto{\pgfqpoint{2.321647in}{3.447166in}}{\pgfqpoint{2.332246in}{3.451556in}}{\pgfqpoint{2.340060in}{3.459370in}}%
\pgfpathcurveto{\pgfqpoint{2.347874in}{3.467183in}}{\pgfqpoint{2.352264in}{3.477782in}}{\pgfqpoint{2.352264in}{3.488832in}}%
\pgfpathcurveto{\pgfqpoint{2.352264in}{3.499883in}}{\pgfqpoint{2.347874in}{3.510482in}}{\pgfqpoint{2.340060in}{3.518295in}}%
\pgfpathcurveto{\pgfqpoint{2.332246in}{3.526109in}}{\pgfqpoint{2.321647in}{3.530499in}}{\pgfqpoint{2.310597in}{3.530499in}}%
\pgfpathcurveto{\pgfqpoint{2.299547in}{3.530499in}}{\pgfqpoint{2.288948in}{3.526109in}}{\pgfqpoint{2.281135in}{3.518295in}}%
\pgfpathcurveto{\pgfqpoint{2.273321in}{3.510482in}}{\pgfqpoint{2.268931in}{3.499883in}}{\pgfqpoint{2.268931in}{3.488832in}}%
\pgfpathcurveto{\pgfqpoint{2.268931in}{3.477782in}}{\pgfqpoint{2.273321in}{3.467183in}}{\pgfqpoint{2.281135in}{3.459370in}}%
\pgfpathcurveto{\pgfqpoint{2.288948in}{3.451556in}}{\pgfqpoint{2.299547in}{3.447166in}}{\pgfqpoint{2.310597in}{3.447166in}}%
\pgfpathclose%
\pgfusepath{stroke,fill}%
\end{pgfscope}%
\begin{pgfscope}%
\pgfpathrectangle{\pgfqpoint{0.600000in}{0.600000in}}{\pgfqpoint{3.900000in}{3.900000in}}%
\pgfusepath{clip}%
\pgfsetbuttcap%
\pgfsetroundjoin%
\definecolor{currentfill}{rgb}{0.121569,0.466667,0.705882}%
\pgfsetfillcolor{currentfill}%
\pgfsetlinewidth{1.003750pt}%
\definecolor{currentstroke}{rgb}{0.121569,0.466667,0.705882}%
\pgfsetstrokecolor{currentstroke}%
\pgfsetdash{}{0pt}%
\pgfpathmoveto{\pgfqpoint{2.919573in}{2.487137in}}%
\pgfpathcurveto{\pgfqpoint{2.930623in}{2.487137in}}{\pgfqpoint{2.941222in}{2.491527in}}{\pgfqpoint{2.949036in}{2.499341in}}%
\pgfpathcurveto{\pgfqpoint{2.956849in}{2.507155in}}{\pgfqpoint{2.961239in}{2.517754in}}{\pgfqpoint{2.961239in}{2.528804in}}%
\pgfpathcurveto{\pgfqpoint{2.961239in}{2.539854in}}{\pgfqpoint{2.956849in}{2.550453in}}{\pgfqpoint{2.949036in}{2.558267in}}%
\pgfpathcurveto{\pgfqpoint{2.941222in}{2.566080in}}{\pgfqpoint{2.930623in}{2.570470in}}{\pgfqpoint{2.919573in}{2.570470in}}%
\pgfpathcurveto{\pgfqpoint{2.908523in}{2.570470in}}{\pgfqpoint{2.897924in}{2.566080in}}{\pgfqpoint{2.890110in}{2.558267in}}%
\pgfpathcurveto{\pgfqpoint{2.882296in}{2.550453in}}{\pgfqpoint{2.877906in}{2.539854in}}{\pgfqpoint{2.877906in}{2.528804in}}%
\pgfpathcurveto{\pgfqpoint{2.877906in}{2.517754in}}{\pgfqpoint{2.882296in}{2.507155in}}{\pgfqpoint{2.890110in}{2.499341in}}%
\pgfpathcurveto{\pgfqpoint{2.897924in}{2.491527in}}{\pgfqpoint{2.908523in}{2.487137in}}{\pgfqpoint{2.919573in}{2.487137in}}%
\pgfpathclose%
\pgfusepath{stroke,fill}%
\end{pgfscope}%
\begin{pgfscope}%
\pgfpathrectangle{\pgfqpoint{0.600000in}{0.600000in}}{\pgfqpoint{3.900000in}{3.900000in}}%
\pgfusepath{clip}%
\pgfsetbuttcap%
\pgfsetroundjoin%
\definecolor{currentfill}{rgb}{0.121569,0.466667,0.705882}%
\pgfsetfillcolor{currentfill}%
\pgfsetlinewidth{1.003750pt}%
\definecolor{currentstroke}{rgb}{0.121569,0.466667,0.705882}%
\pgfsetstrokecolor{currentstroke}%
\pgfsetdash{}{0pt}%
\pgfpathmoveto{\pgfqpoint{2.059946in}{2.808166in}}%
\pgfpathcurveto{\pgfqpoint{2.070996in}{2.808166in}}{\pgfqpoint{2.081595in}{2.812556in}}{\pgfqpoint{2.089408in}{2.820370in}}%
\pgfpathcurveto{\pgfqpoint{2.097222in}{2.828184in}}{\pgfqpoint{2.101612in}{2.838783in}}{\pgfqpoint{2.101612in}{2.849833in}}%
\pgfpathcurveto{\pgfqpoint{2.101612in}{2.860883in}}{\pgfqpoint{2.097222in}{2.871482in}}{\pgfqpoint{2.089408in}{2.879296in}}%
\pgfpathcurveto{\pgfqpoint{2.081595in}{2.887109in}}{\pgfqpoint{2.070996in}{2.891500in}}{\pgfqpoint{2.059946in}{2.891500in}}%
\pgfpathcurveto{\pgfqpoint{2.048895in}{2.891500in}}{\pgfqpoint{2.038296in}{2.887109in}}{\pgfqpoint{2.030483in}{2.879296in}}%
\pgfpathcurveto{\pgfqpoint{2.022669in}{2.871482in}}{\pgfqpoint{2.018279in}{2.860883in}}{\pgfqpoint{2.018279in}{2.849833in}}%
\pgfpathcurveto{\pgfqpoint{2.018279in}{2.838783in}}{\pgfqpoint{2.022669in}{2.828184in}}{\pgfqpoint{2.030483in}{2.820370in}}%
\pgfpathcurveto{\pgfqpoint{2.038296in}{2.812556in}}{\pgfqpoint{2.048895in}{2.808166in}}{\pgfqpoint{2.059946in}{2.808166in}}%
\pgfpathclose%
\pgfusepath{stroke,fill}%
\end{pgfscope}%
\begin{pgfscope}%
\pgfpathrectangle{\pgfqpoint{0.600000in}{0.600000in}}{\pgfqpoint{3.900000in}{3.900000in}}%
\pgfusepath{clip}%
\pgfsetbuttcap%
\pgfsetroundjoin%
\definecolor{currentfill}{rgb}{0.121569,0.466667,0.705882}%
\pgfsetfillcolor{currentfill}%
\pgfsetlinewidth{1.003750pt}%
\definecolor{currentstroke}{rgb}{0.121569,0.466667,0.705882}%
\pgfsetstrokecolor{currentstroke}%
\pgfsetdash{}{0pt}%
\pgfpathmoveto{\pgfqpoint{2.105595in}{2.483198in}}%
\pgfpathcurveto{\pgfqpoint{2.116645in}{2.483198in}}{\pgfqpoint{2.127244in}{2.487588in}}{\pgfqpoint{2.135058in}{2.495402in}}%
\pgfpathcurveto{\pgfqpoint{2.142872in}{2.503215in}}{\pgfqpoint{2.147262in}{2.513814in}}{\pgfqpoint{2.147262in}{2.524864in}}%
\pgfpathcurveto{\pgfqpoint{2.147262in}{2.535915in}}{\pgfqpoint{2.142872in}{2.546514in}}{\pgfqpoint{2.135058in}{2.554327in}}%
\pgfpathcurveto{\pgfqpoint{2.127244in}{2.562141in}}{\pgfqpoint{2.116645in}{2.566531in}}{\pgfqpoint{2.105595in}{2.566531in}}%
\pgfpathcurveto{\pgfqpoint{2.094545in}{2.566531in}}{\pgfqpoint{2.083946in}{2.562141in}}{\pgfqpoint{2.076132in}{2.554327in}}%
\pgfpathcurveto{\pgfqpoint{2.068319in}{2.546514in}}{\pgfqpoint{2.063928in}{2.535915in}}{\pgfqpoint{2.063928in}{2.524864in}}%
\pgfpathcurveto{\pgfqpoint{2.063928in}{2.513814in}}{\pgfqpoint{2.068319in}{2.503215in}}{\pgfqpoint{2.076132in}{2.495402in}}%
\pgfpathcurveto{\pgfqpoint{2.083946in}{2.487588in}}{\pgfqpoint{2.094545in}{2.483198in}}{\pgfqpoint{2.105595in}{2.483198in}}%
\pgfpathclose%
\pgfusepath{stroke,fill}%
\end{pgfscope}%
\begin{pgfscope}%
\pgfpathrectangle{\pgfqpoint{0.600000in}{0.600000in}}{\pgfqpoint{3.900000in}{3.900000in}}%
\pgfusepath{clip}%
\pgfsetbuttcap%
\pgfsetroundjoin%
\definecolor{currentfill}{rgb}{0.121569,0.466667,0.705882}%
\pgfsetfillcolor{currentfill}%
\pgfsetlinewidth{1.003750pt}%
\definecolor{currentstroke}{rgb}{0.121569,0.466667,0.705882}%
\pgfsetstrokecolor{currentstroke}%
\pgfsetdash{}{0pt}%
\pgfpathmoveto{\pgfqpoint{2.823492in}{2.104595in}}%
\pgfpathcurveto{\pgfqpoint{2.834542in}{2.104595in}}{\pgfqpoint{2.845141in}{2.108985in}}{\pgfqpoint{2.852955in}{2.116798in}}%
\pgfpathcurveto{\pgfqpoint{2.860769in}{2.124612in}}{\pgfqpoint{2.865159in}{2.135211in}}{\pgfqpoint{2.865159in}{2.146261in}}%
\pgfpathcurveto{\pgfqpoint{2.865159in}{2.157311in}}{\pgfqpoint{2.860769in}{2.167910in}}{\pgfqpoint{2.852955in}{2.175724in}}%
\pgfpathcurveto{\pgfqpoint{2.845141in}{2.183538in}}{\pgfqpoint{2.834542in}{2.187928in}}{\pgfqpoint{2.823492in}{2.187928in}}%
\pgfpathcurveto{\pgfqpoint{2.812442in}{2.187928in}}{\pgfqpoint{2.801843in}{2.183538in}}{\pgfqpoint{2.794029in}{2.175724in}}%
\pgfpathcurveto{\pgfqpoint{2.786216in}{2.167910in}}{\pgfqpoint{2.781826in}{2.157311in}}{\pgfqpoint{2.781826in}{2.146261in}}%
\pgfpathcurveto{\pgfqpoint{2.781826in}{2.135211in}}{\pgfqpoint{2.786216in}{2.124612in}}{\pgfqpoint{2.794029in}{2.116798in}}%
\pgfpathcurveto{\pgfqpoint{2.801843in}{2.108985in}}{\pgfqpoint{2.812442in}{2.104595in}}{\pgfqpoint{2.823492in}{2.104595in}}%
\pgfpathclose%
\pgfusepath{stroke,fill}%
\end{pgfscope}%
\begin{pgfscope}%
\pgfpathrectangle{\pgfqpoint{0.600000in}{0.600000in}}{\pgfqpoint{3.900000in}{3.900000in}}%
\pgfusepath{clip}%
\pgfsetbuttcap%
\pgfsetroundjoin%
\definecolor{currentfill}{rgb}{0.121569,0.466667,0.705882}%
\pgfsetfillcolor{currentfill}%
\pgfsetlinewidth{1.003750pt}%
\definecolor{currentstroke}{rgb}{0.121569,0.466667,0.705882}%
\pgfsetstrokecolor{currentstroke}%
\pgfsetdash{}{0pt}%
\pgfpathmoveto{\pgfqpoint{2.774066in}{2.333168in}}%
\pgfpathcurveto{\pgfqpoint{2.785117in}{2.333168in}}{\pgfqpoint{2.795716in}{2.337558in}}{\pgfqpoint{2.803529in}{2.345372in}}%
\pgfpathcurveto{\pgfqpoint{2.811343in}{2.353186in}}{\pgfqpoint{2.815733in}{2.363785in}}{\pgfqpoint{2.815733in}{2.374835in}}%
\pgfpathcurveto{\pgfqpoint{2.815733in}{2.385885in}}{\pgfqpoint{2.811343in}{2.396484in}}{\pgfqpoint{2.803529in}{2.404298in}}%
\pgfpathcurveto{\pgfqpoint{2.795716in}{2.412111in}}{\pgfqpoint{2.785117in}{2.416502in}}{\pgfqpoint{2.774066in}{2.416502in}}%
\pgfpathcurveto{\pgfqpoint{2.763016in}{2.416502in}}{\pgfqpoint{2.752417in}{2.412111in}}{\pgfqpoint{2.744604in}{2.404298in}}%
\pgfpathcurveto{\pgfqpoint{2.736790in}{2.396484in}}{\pgfqpoint{2.732400in}{2.385885in}}{\pgfqpoint{2.732400in}{2.374835in}}%
\pgfpathcurveto{\pgfqpoint{2.732400in}{2.363785in}}{\pgfqpoint{2.736790in}{2.353186in}}{\pgfqpoint{2.744604in}{2.345372in}}%
\pgfpathcurveto{\pgfqpoint{2.752417in}{2.337558in}}{\pgfqpoint{2.763016in}{2.333168in}}{\pgfqpoint{2.774066in}{2.333168in}}%
\pgfpathclose%
\pgfusepath{stroke,fill}%
\end{pgfscope}%
\begin{pgfscope}%
\pgfpathrectangle{\pgfqpoint{0.600000in}{0.600000in}}{\pgfqpoint{3.900000in}{3.900000in}}%
\pgfusepath{clip}%
\pgfsetbuttcap%
\pgfsetroundjoin%
\definecolor{currentfill}{rgb}{0.121569,0.466667,0.705882}%
\pgfsetfillcolor{currentfill}%
\pgfsetlinewidth{1.003750pt}%
\definecolor{currentstroke}{rgb}{0.121569,0.466667,0.705882}%
\pgfsetstrokecolor{currentstroke}%
\pgfsetdash{}{0pt}%
\pgfpathmoveto{\pgfqpoint{2.728884in}{1.782227in}}%
\pgfpathcurveto{\pgfqpoint{2.739934in}{1.782227in}}{\pgfqpoint{2.750533in}{1.786618in}}{\pgfqpoint{2.758347in}{1.794431in}}%
\pgfpathcurveto{\pgfqpoint{2.766160in}{1.802245in}}{\pgfqpoint{2.770551in}{1.812844in}}{\pgfqpoint{2.770551in}{1.823894in}}%
\pgfpathcurveto{\pgfqpoint{2.770551in}{1.834944in}}{\pgfqpoint{2.766160in}{1.845543in}}{\pgfqpoint{2.758347in}{1.853357in}}%
\pgfpathcurveto{\pgfqpoint{2.750533in}{1.861170in}}{\pgfqpoint{2.739934in}{1.865561in}}{\pgfqpoint{2.728884in}{1.865561in}}%
\pgfpathcurveto{\pgfqpoint{2.717834in}{1.865561in}}{\pgfqpoint{2.707235in}{1.861170in}}{\pgfqpoint{2.699421in}{1.853357in}}%
\pgfpathcurveto{\pgfqpoint{2.691608in}{1.845543in}}{\pgfqpoint{2.687217in}{1.834944in}}{\pgfqpoint{2.687217in}{1.823894in}}%
\pgfpathcurveto{\pgfqpoint{2.687217in}{1.812844in}}{\pgfqpoint{2.691608in}{1.802245in}}{\pgfqpoint{2.699421in}{1.794431in}}%
\pgfpathcurveto{\pgfqpoint{2.707235in}{1.786618in}}{\pgfqpoint{2.717834in}{1.782227in}}{\pgfqpoint{2.728884in}{1.782227in}}%
\pgfpathclose%
\pgfusepath{stroke,fill}%
\end{pgfscope}%
\begin{pgfscope}%
\pgfpathrectangle{\pgfqpoint{0.600000in}{0.600000in}}{\pgfqpoint{3.900000in}{3.900000in}}%
\pgfusepath{clip}%
\pgfsetbuttcap%
\pgfsetroundjoin%
\definecolor{currentfill}{rgb}{0.121569,0.466667,0.705882}%
\pgfsetfillcolor{currentfill}%
\pgfsetlinewidth{1.003750pt}%
\definecolor{currentstroke}{rgb}{0.121569,0.466667,0.705882}%
\pgfsetstrokecolor{currentstroke}%
\pgfsetdash{}{0pt}%
\pgfpathmoveto{\pgfqpoint{2.643153in}{3.029412in}}%
\pgfpathcurveto{\pgfqpoint{2.654203in}{3.029412in}}{\pgfqpoint{2.664802in}{3.033802in}}{\pgfqpoint{2.672616in}{3.041616in}}%
\pgfpathcurveto{\pgfqpoint{2.680429in}{3.049430in}}{\pgfqpoint{2.684819in}{3.060029in}}{\pgfqpoint{2.684819in}{3.071079in}}%
\pgfpathcurveto{\pgfqpoint{2.684819in}{3.082129in}}{\pgfqpoint{2.680429in}{3.092728in}}{\pgfqpoint{2.672616in}{3.100542in}}%
\pgfpathcurveto{\pgfqpoint{2.664802in}{3.108355in}}{\pgfqpoint{2.654203in}{3.112745in}}{\pgfqpoint{2.643153in}{3.112745in}}%
\pgfpathcurveto{\pgfqpoint{2.632103in}{3.112745in}}{\pgfqpoint{2.621504in}{3.108355in}}{\pgfqpoint{2.613690in}{3.100542in}}%
\pgfpathcurveto{\pgfqpoint{2.605876in}{3.092728in}}{\pgfqpoint{2.601486in}{3.082129in}}{\pgfqpoint{2.601486in}{3.071079in}}%
\pgfpathcurveto{\pgfqpoint{2.601486in}{3.060029in}}{\pgfqpoint{2.605876in}{3.049430in}}{\pgfqpoint{2.613690in}{3.041616in}}%
\pgfpathcurveto{\pgfqpoint{2.621504in}{3.033802in}}{\pgfqpoint{2.632103in}{3.029412in}}{\pgfqpoint{2.643153in}{3.029412in}}%
\pgfpathclose%
\pgfusepath{stroke,fill}%
\end{pgfscope}%
\begin{pgfscope}%
\pgfpathrectangle{\pgfqpoint{0.600000in}{0.600000in}}{\pgfqpoint{3.900000in}{3.900000in}}%
\pgfusepath{clip}%
\pgfsetbuttcap%
\pgfsetroundjoin%
\definecolor{currentfill}{rgb}{0.121569,0.466667,0.705882}%
\pgfsetfillcolor{currentfill}%
\pgfsetlinewidth{1.003750pt}%
\definecolor{currentstroke}{rgb}{0.121569,0.466667,0.705882}%
\pgfsetstrokecolor{currentstroke}%
\pgfsetdash{}{0pt}%
\pgfpathmoveto{\pgfqpoint{2.612602in}{3.082911in}}%
\pgfpathcurveto{\pgfqpoint{2.623652in}{3.082911in}}{\pgfqpoint{2.634251in}{3.087301in}}{\pgfqpoint{2.642064in}{3.095115in}}%
\pgfpathcurveto{\pgfqpoint{2.649878in}{3.102928in}}{\pgfqpoint{2.654268in}{3.113527in}}{\pgfqpoint{2.654268in}{3.124577in}}%
\pgfpathcurveto{\pgfqpoint{2.654268in}{3.135628in}}{\pgfqpoint{2.649878in}{3.146227in}}{\pgfqpoint{2.642064in}{3.154040in}}%
\pgfpathcurveto{\pgfqpoint{2.634251in}{3.161854in}}{\pgfqpoint{2.623652in}{3.166244in}}{\pgfqpoint{2.612602in}{3.166244in}}%
\pgfpathcurveto{\pgfqpoint{2.601552in}{3.166244in}}{\pgfqpoint{2.590952in}{3.161854in}}{\pgfqpoint{2.583139in}{3.154040in}}%
\pgfpathcurveto{\pgfqpoint{2.575325in}{3.146227in}}{\pgfqpoint{2.570935in}{3.135628in}}{\pgfqpoint{2.570935in}{3.124577in}}%
\pgfpathcurveto{\pgfqpoint{2.570935in}{3.113527in}}{\pgfqpoint{2.575325in}{3.102928in}}{\pgfqpoint{2.583139in}{3.095115in}}%
\pgfpathcurveto{\pgfqpoint{2.590952in}{3.087301in}}{\pgfqpoint{2.601552in}{3.082911in}}{\pgfqpoint{2.612602in}{3.082911in}}%
\pgfpathclose%
\pgfusepath{stroke,fill}%
\end{pgfscope}%
\begin{pgfscope}%
\pgfpathrectangle{\pgfqpoint{0.600000in}{0.600000in}}{\pgfqpoint{3.900000in}{3.900000in}}%
\pgfusepath{clip}%
\pgfsetbuttcap%
\pgfsetroundjoin%
\definecolor{currentfill}{rgb}{0.121569,0.466667,0.705882}%
\pgfsetfillcolor{currentfill}%
\pgfsetlinewidth{1.003750pt}%
\definecolor{currentstroke}{rgb}{0.121569,0.466667,0.705882}%
\pgfsetstrokecolor{currentstroke}%
\pgfsetdash{}{0pt}%
\pgfpathmoveto{\pgfqpoint{3.176171in}{2.519267in}}%
\pgfpathcurveto{\pgfqpoint{3.187221in}{2.519267in}}{\pgfqpoint{3.197820in}{2.523658in}}{\pgfqpoint{3.205634in}{2.531471in}}%
\pgfpathcurveto{\pgfqpoint{3.213448in}{2.539285in}}{\pgfqpoint{3.217838in}{2.549884in}}{\pgfqpoint{3.217838in}{2.560934in}}%
\pgfpathcurveto{\pgfqpoint{3.217838in}{2.571984in}}{\pgfqpoint{3.213448in}{2.582583in}}{\pgfqpoint{3.205634in}{2.590397in}}%
\pgfpathcurveto{\pgfqpoint{3.197820in}{2.598210in}}{\pgfqpoint{3.187221in}{2.602601in}}{\pgfqpoint{3.176171in}{2.602601in}}%
\pgfpathcurveto{\pgfqpoint{3.165121in}{2.602601in}}{\pgfqpoint{3.154522in}{2.598210in}}{\pgfqpoint{3.146709in}{2.590397in}}%
\pgfpathcurveto{\pgfqpoint{3.138895in}{2.582583in}}{\pgfqpoint{3.134505in}{2.571984in}}{\pgfqpoint{3.134505in}{2.560934in}}%
\pgfpathcurveto{\pgfqpoint{3.134505in}{2.549884in}}{\pgfqpoint{3.138895in}{2.539285in}}{\pgfqpoint{3.146709in}{2.531471in}}%
\pgfpathcurveto{\pgfqpoint{3.154522in}{2.523658in}}{\pgfqpoint{3.165121in}{2.519267in}}{\pgfqpoint{3.176171in}{2.519267in}}%
\pgfpathclose%
\pgfusepath{stroke,fill}%
\end{pgfscope}%
\begin{pgfscope}%
\pgfpathrectangle{\pgfqpoint{0.600000in}{0.600000in}}{\pgfqpoint{3.900000in}{3.900000in}}%
\pgfusepath{clip}%
\pgfsetbuttcap%
\pgfsetroundjoin%
\definecolor{currentfill}{rgb}{0.121569,0.466667,0.705882}%
\pgfsetfillcolor{currentfill}%
\pgfsetlinewidth{1.003750pt}%
\definecolor{currentstroke}{rgb}{0.121569,0.466667,0.705882}%
\pgfsetstrokecolor{currentstroke}%
\pgfsetdash{}{0pt}%
\pgfpathmoveto{\pgfqpoint{2.254962in}{3.223309in}}%
\pgfpathcurveto{\pgfqpoint{2.266012in}{3.223309in}}{\pgfqpoint{2.276611in}{3.227700in}}{\pgfqpoint{2.284425in}{3.235513in}}%
\pgfpathcurveto{\pgfqpoint{2.292239in}{3.243327in}}{\pgfqpoint{2.296629in}{3.253926in}}{\pgfqpoint{2.296629in}{3.264976in}}%
\pgfpathcurveto{\pgfqpoint{2.296629in}{3.276026in}}{\pgfqpoint{2.292239in}{3.286625in}}{\pgfqpoint{2.284425in}{3.294439in}}%
\pgfpathcurveto{\pgfqpoint{2.276611in}{3.302252in}}{\pgfqpoint{2.266012in}{3.306643in}}{\pgfqpoint{2.254962in}{3.306643in}}%
\pgfpathcurveto{\pgfqpoint{2.243912in}{3.306643in}}{\pgfqpoint{2.233313in}{3.302252in}}{\pgfqpoint{2.225499in}{3.294439in}}%
\pgfpathcurveto{\pgfqpoint{2.217686in}{3.286625in}}{\pgfqpoint{2.213295in}{3.276026in}}{\pgfqpoint{2.213295in}{3.264976in}}%
\pgfpathcurveto{\pgfqpoint{2.213295in}{3.253926in}}{\pgfqpoint{2.217686in}{3.243327in}}{\pgfqpoint{2.225499in}{3.235513in}}%
\pgfpathcurveto{\pgfqpoint{2.233313in}{3.227700in}}{\pgfqpoint{2.243912in}{3.223309in}}{\pgfqpoint{2.254962in}{3.223309in}}%
\pgfpathclose%
\pgfusepath{stroke,fill}%
\end{pgfscope}%
\begin{pgfscope}%
\pgfpathrectangle{\pgfqpoint{0.600000in}{0.600000in}}{\pgfqpoint{3.900000in}{3.900000in}}%
\pgfusepath{clip}%
\pgfsetbuttcap%
\pgfsetroundjoin%
\definecolor{currentfill}{rgb}{0.121569,0.466667,0.705882}%
\pgfsetfillcolor{currentfill}%
\pgfsetlinewidth{1.003750pt}%
\definecolor{currentstroke}{rgb}{0.121569,0.466667,0.705882}%
\pgfsetstrokecolor{currentstroke}%
\pgfsetdash{}{0pt}%
\pgfpathmoveto{\pgfqpoint{1.950234in}{2.601233in}}%
\pgfpathcurveto{\pgfqpoint{1.961284in}{2.601233in}}{\pgfqpoint{1.971883in}{2.605624in}}{\pgfqpoint{1.979697in}{2.613437in}}%
\pgfpathcurveto{\pgfqpoint{1.987510in}{2.621251in}}{\pgfqpoint{1.991901in}{2.631850in}}{\pgfqpoint{1.991901in}{2.642900in}}%
\pgfpathcurveto{\pgfqpoint{1.991901in}{2.653950in}}{\pgfqpoint{1.987510in}{2.664549in}}{\pgfqpoint{1.979697in}{2.672363in}}%
\pgfpathcurveto{\pgfqpoint{1.971883in}{2.680176in}}{\pgfqpoint{1.961284in}{2.684567in}}{\pgfqpoint{1.950234in}{2.684567in}}%
\pgfpathcurveto{\pgfqpoint{1.939184in}{2.684567in}}{\pgfqpoint{1.928585in}{2.680176in}}{\pgfqpoint{1.920771in}{2.672363in}}%
\pgfpathcurveto{\pgfqpoint{1.912958in}{2.664549in}}{\pgfqpoint{1.908567in}{2.653950in}}{\pgfqpoint{1.908567in}{2.642900in}}%
\pgfpathcurveto{\pgfqpoint{1.908567in}{2.631850in}}{\pgfqpoint{1.912958in}{2.621251in}}{\pgfqpoint{1.920771in}{2.613437in}}%
\pgfpathcurveto{\pgfqpoint{1.928585in}{2.605624in}}{\pgfqpoint{1.939184in}{2.601233in}}{\pgfqpoint{1.950234in}{2.601233in}}%
\pgfpathclose%
\pgfusepath{stroke,fill}%
\end{pgfscope}%
\begin{pgfscope}%
\pgfpathrectangle{\pgfqpoint{0.600000in}{0.600000in}}{\pgfqpoint{3.900000in}{3.900000in}}%
\pgfusepath{clip}%
\pgfsetbuttcap%
\pgfsetroundjoin%
\definecolor{currentfill}{rgb}{0.121569,0.466667,0.705882}%
\pgfsetfillcolor{currentfill}%
\pgfsetlinewidth{1.003750pt}%
\definecolor{currentstroke}{rgb}{0.121569,0.466667,0.705882}%
\pgfsetstrokecolor{currentstroke}%
\pgfsetdash{}{0pt}%
\pgfpathmoveto{\pgfqpoint{2.923401in}{2.123131in}}%
\pgfpathcurveto{\pgfqpoint{2.934451in}{2.123131in}}{\pgfqpoint{2.945050in}{2.127521in}}{\pgfqpoint{2.952864in}{2.135335in}}%
\pgfpathcurveto{\pgfqpoint{2.960678in}{2.143148in}}{\pgfqpoint{2.965068in}{2.153747in}}{\pgfqpoint{2.965068in}{2.164797in}}%
\pgfpathcurveto{\pgfqpoint{2.965068in}{2.175848in}}{\pgfqpoint{2.960678in}{2.186447in}}{\pgfqpoint{2.952864in}{2.194260in}}%
\pgfpathcurveto{\pgfqpoint{2.945050in}{2.202074in}}{\pgfqpoint{2.934451in}{2.206464in}}{\pgfqpoint{2.923401in}{2.206464in}}%
\pgfpathcurveto{\pgfqpoint{2.912351in}{2.206464in}}{\pgfqpoint{2.901752in}{2.202074in}}{\pgfqpoint{2.893939in}{2.194260in}}%
\pgfpathcurveto{\pgfqpoint{2.886125in}{2.186447in}}{\pgfqpoint{2.881735in}{2.175848in}}{\pgfqpoint{2.881735in}{2.164797in}}%
\pgfpathcurveto{\pgfqpoint{2.881735in}{2.153747in}}{\pgfqpoint{2.886125in}{2.143148in}}{\pgfqpoint{2.893939in}{2.135335in}}%
\pgfpathcurveto{\pgfqpoint{2.901752in}{2.127521in}}{\pgfqpoint{2.912351in}{2.123131in}}{\pgfqpoint{2.923401in}{2.123131in}}%
\pgfpathclose%
\pgfusepath{stroke,fill}%
\end{pgfscope}%
\begin{pgfscope}%
\pgfpathrectangle{\pgfqpoint{0.600000in}{0.600000in}}{\pgfqpoint{3.900000in}{3.900000in}}%
\pgfusepath{clip}%
\pgfsetbuttcap%
\pgfsetroundjoin%
\definecolor{currentfill}{rgb}{0.121569,0.466667,0.705882}%
\pgfsetfillcolor{currentfill}%
\pgfsetlinewidth{1.003750pt}%
\definecolor{currentstroke}{rgb}{0.121569,0.466667,0.705882}%
\pgfsetstrokecolor{currentstroke}%
\pgfsetdash{}{0pt}%
\pgfpathmoveto{\pgfqpoint{2.355333in}{3.259420in}}%
\pgfpathcurveto{\pgfqpoint{2.366383in}{3.259420in}}{\pgfqpoint{2.376982in}{3.263811in}}{\pgfqpoint{2.384795in}{3.271624in}}%
\pgfpathcurveto{\pgfqpoint{2.392609in}{3.279438in}}{\pgfqpoint{2.396999in}{3.290037in}}{\pgfqpoint{2.396999in}{3.301087in}}%
\pgfpathcurveto{\pgfqpoint{2.396999in}{3.312137in}}{\pgfqpoint{2.392609in}{3.322736in}}{\pgfqpoint{2.384795in}{3.330550in}}%
\pgfpathcurveto{\pgfqpoint{2.376982in}{3.338363in}}{\pgfqpoint{2.366383in}{3.342754in}}{\pgfqpoint{2.355333in}{3.342754in}}%
\pgfpathcurveto{\pgfqpoint{2.344283in}{3.342754in}}{\pgfqpoint{2.333684in}{3.338363in}}{\pgfqpoint{2.325870in}{3.330550in}}%
\pgfpathcurveto{\pgfqpoint{2.318056in}{3.322736in}}{\pgfqpoint{2.313666in}{3.312137in}}{\pgfqpoint{2.313666in}{3.301087in}}%
\pgfpathcurveto{\pgfqpoint{2.313666in}{3.290037in}}{\pgfqpoint{2.318056in}{3.279438in}}{\pgfqpoint{2.325870in}{3.271624in}}%
\pgfpathcurveto{\pgfqpoint{2.333684in}{3.263811in}}{\pgfqpoint{2.344283in}{3.259420in}}{\pgfqpoint{2.355333in}{3.259420in}}%
\pgfpathclose%
\pgfusepath{stroke,fill}%
\end{pgfscope}%
\begin{pgfscope}%
\pgfpathrectangle{\pgfqpoint{0.600000in}{0.600000in}}{\pgfqpoint{3.900000in}{3.900000in}}%
\pgfusepath{clip}%
\pgfsetbuttcap%
\pgfsetroundjoin%
\definecolor{currentfill}{rgb}{0.121569,0.466667,0.705882}%
\pgfsetfillcolor{currentfill}%
\pgfsetlinewidth{1.003750pt}%
\definecolor{currentstroke}{rgb}{0.121569,0.466667,0.705882}%
\pgfsetstrokecolor{currentstroke}%
\pgfsetdash{}{0pt}%
\pgfpathmoveto{\pgfqpoint{1.424397in}{2.733791in}}%
\pgfpathcurveto{\pgfqpoint{1.435447in}{2.733791in}}{\pgfqpoint{1.446046in}{2.738181in}}{\pgfqpoint{1.453860in}{2.745995in}}%
\pgfpathcurveto{\pgfqpoint{1.461674in}{2.753808in}}{\pgfqpoint{1.466064in}{2.764407in}}{\pgfqpoint{1.466064in}{2.775457in}}%
\pgfpathcurveto{\pgfqpoint{1.466064in}{2.786507in}}{\pgfqpoint{1.461674in}{2.797107in}}{\pgfqpoint{1.453860in}{2.804920in}}%
\pgfpathcurveto{\pgfqpoint{1.446046in}{2.812734in}}{\pgfqpoint{1.435447in}{2.817124in}}{\pgfqpoint{1.424397in}{2.817124in}}%
\pgfpathcurveto{\pgfqpoint{1.413347in}{2.817124in}}{\pgfqpoint{1.402748in}{2.812734in}}{\pgfqpoint{1.394934in}{2.804920in}}%
\pgfpathcurveto{\pgfqpoint{1.387121in}{2.797107in}}{\pgfqpoint{1.382731in}{2.786507in}}{\pgfqpoint{1.382731in}{2.775457in}}%
\pgfpathcurveto{\pgfqpoint{1.382731in}{2.764407in}}{\pgfqpoint{1.387121in}{2.753808in}}{\pgfqpoint{1.394934in}{2.745995in}}%
\pgfpathcurveto{\pgfqpoint{1.402748in}{2.738181in}}{\pgfqpoint{1.413347in}{2.733791in}}{\pgfqpoint{1.424397in}{2.733791in}}%
\pgfpathclose%
\pgfusepath{stroke,fill}%
\end{pgfscope}%
\begin{pgfscope}%
\pgfpathrectangle{\pgfqpoint{0.600000in}{0.600000in}}{\pgfqpoint{3.900000in}{3.900000in}}%
\pgfusepath{clip}%
\pgfsetbuttcap%
\pgfsetroundjoin%
\definecolor{currentfill}{rgb}{0.121569,0.466667,0.705882}%
\pgfsetfillcolor{currentfill}%
\pgfsetlinewidth{1.003750pt}%
\definecolor{currentstroke}{rgb}{0.121569,0.466667,0.705882}%
\pgfsetstrokecolor{currentstroke}%
\pgfsetdash{}{0pt}%
\pgfpathmoveto{\pgfqpoint{1.644373in}{2.261548in}}%
\pgfpathcurveto{\pgfqpoint{1.655423in}{2.261548in}}{\pgfqpoint{1.666022in}{2.265938in}}{\pgfqpoint{1.673836in}{2.273752in}}%
\pgfpathcurveto{\pgfqpoint{1.681650in}{2.281565in}}{\pgfqpoint{1.686040in}{2.292164in}}{\pgfqpoint{1.686040in}{2.303214in}}%
\pgfpathcurveto{\pgfqpoint{1.686040in}{2.314265in}}{\pgfqpoint{1.681650in}{2.324864in}}{\pgfqpoint{1.673836in}{2.332677in}}%
\pgfpathcurveto{\pgfqpoint{1.666022in}{2.340491in}}{\pgfqpoint{1.655423in}{2.344881in}}{\pgfqpoint{1.644373in}{2.344881in}}%
\pgfpathcurveto{\pgfqpoint{1.633323in}{2.344881in}}{\pgfqpoint{1.622724in}{2.340491in}}{\pgfqpoint{1.614910in}{2.332677in}}%
\pgfpathcurveto{\pgfqpoint{1.607097in}{2.324864in}}{\pgfqpoint{1.602707in}{2.314265in}}{\pgfqpoint{1.602707in}{2.303214in}}%
\pgfpathcurveto{\pgfqpoint{1.602707in}{2.292164in}}{\pgfqpoint{1.607097in}{2.281565in}}{\pgfqpoint{1.614910in}{2.273752in}}%
\pgfpathcurveto{\pgfqpoint{1.622724in}{2.265938in}}{\pgfqpoint{1.633323in}{2.261548in}}{\pgfqpoint{1.644373in}{2.261548in}}%
\pgfpathclose%
\pgfusepath{stroke,fill}%
\end{pgfscope}%
\begin{pgfscope}%
\pgfpathrectangle{\pgfqpoint{0.600000in}{0.600000in}}{\pgfqpoint{3.900000in}{3.900000in}}%
\pgfusepath{clip}%
\pgfsetbuttcap%
\pgfsetroundjoin%
\definecolor{currentfill}{rgb}{0.121569,0.466667,0.705882}%
\pgfsetfillcolor{currentfill}%
\pgfsetlinewidth{1.003750pt}%
\definecolor{currentstroke}{rgb}{0.121569,0.466667,0.705882}%
\pgfsetstrokecolor{currentstroke}%
\pgfsetdash{}{0pt}%
\pgfpathmoveto{\pgfqpoint{2.458614in}{2.350494in}}%
\pgfpathcurveto{\pgfqpoint{2.469664in}{2.350494in}}{\pgfqpoint{2.480263in}{2.354884in}}{\pgfqpoint{2.488076in}{2.362697in}}%
\pgfpathcurveto{\pgfqpoint{2.495890in}{2.370511in}}{\pgfqpoint{2.500280in}{2.381110in}}{\pgfqpoint{2.500280in}{2.392160in}}%
\pgfpathcurveto{\pgfqpoint{2.500280in}{2.403210in}}{\pgfqpoint{2.495890in}{2.413809in}}{\pgfqpoint{2.488076in}{2.421623in}}%
\pgfpathcurveto{\pgfqpoint{2.480263in}{2.429437in}}{\pgfqpoint{2.469664in}{2.433827in}}{\pgfqpoint{2.458614in}{2.433827in}}%
\pgfpathcurveto{\pgfqpoint{2.447564in}{2.433827in}}{\pgfqpoint{2.436964in}{2.429437in}}{\pgfqpoint{2.429151in}{2.421623in}}%
\pgfpathcurveto{\pgfqpoint{2.421337in}{2.413809in}}{\pgfqpoint{2.416947in}{2.403210in}}{\pgfqpoint{2.416947in}{2.392160in}}%
\pgfpathcurveto{\pgfqpoint{2.416947in}{2.381110in}}{\pgfqpoint{2.421337in}{2.370511in}}{\pgfqpoint{2.429151in}{2.362697in}}%
\pgfpathcurveto{\pgfqpoint{2.436964in}{2.354884in}}{\pgfqpoint{2.447564in}{2.350494in}}{\pgfqpoint{2.458614in}{2.350494in}}%
\pgfpathclose%
\pgfusepath{stroke,fill}%
\end{pgfscope}%
\begin{pgfscope}%
\pgfpathrectangle{\pgfqpoint{0.600000in}{0.600000in}}{\pgfqpoint{3.900000in}{3.900000in}}%
\pgfusepath{clip}%
\pgfsetbuttcap%
\pgfsetroundjoin%
\definecolor{currentfill}{rgb}{0.121569,0.466667,0.705882}%
\pgfsetfillcolor{currentfill}%
\pgfsetlinewidth{1.003750pt}%
\definecolor{currentstroke}{rgb}{0.121569,0.466667,0.705882}%
\pgfsetstrokecolor{currentstroke}%
\pgfsetdash{}{0pt}%
\pgfpathmoveto{\pgfqpoint{2.661203in}{2.815972in}}%
\pgfpathcurveto{\pgfqpoint{2.672253in}{2.815972in}}{\pgfqpoint{2.682852in}{2.820363in}}{\pgfqpoint{2.690665in}{2.828176in}}%
\pgfpathcurveto{\pgfqpoint{2.698479in}{2.835990in}}{\pgfqpoint{2.702869in}{2.846589in}}{\pgfqpoint{2.702869in}{2.857639in}}%
\pgfpathcurveto{\pgfqpoint{2.702869in}{2.868689in}}{\pgfqpoint{2.698479in}{2.879288in}}{\pgfqpoint{2.690665in}{2.887102in}}%
\pgfpathcurveto{\pgfqpoint{2.682852in}{2.894916in}}{\pgfqpoint{2.672253in}{2.899306in}}{\pgfqpoint{2.661203in}{2.899306in}}%
\pgfpathcurveto{\pgfqpoint{2.650152in}{2.899306in}}{\pgfqpoint{2.639553in}{2.894916in}}{\pgfqpoint{2.631740in}{2.887102in}}%
\pgfpathcurveto{\pgfqpoint{2.623926in}{2.879288in}}{\pgfqpoint{2.619536in}{2.868689in}}{\pgfqpoint{2.619536in}{2.857639in}}%
\pgfpathcurveto{\pgfqpoint{2.619536in}{2.846589in}}{\pgfqpoint{2.623926in}{2.835990in}}{\pgfqpoint{2.631740in}{2.828176in}}%
\pgfpathcurveto{\pgfqpoint{2.639553in}{2.820363in}}{\pgfqpoint{2.650152in}{2.815972in}}{\pgfqpoint{2.661203in}{2.815972in}}%
\pgfpathclose%
\pgfusepath{stroke,fill}%
\end{pgfscope}%
\begin{pgfscope}%
\pgfpathrectangle{\pgfqpoint{0.600000in}{0.600000in}}{\pgfqpoint{3.900000in}{3.900000in}}%
\pgfusepath{clip}%
\pgfsetbuttcap%
\pgfsetroundjoin%
\definecolor{currentfill}{rgb}{0.121569,0.466667,0.705882}%
\pgfsetfillcolor{currentfill}%
\pgfsetlinewidth{1.003750pt}%
\definecolor{currentstroke}{rgb}{0.121569,0.466667,0.705882}%
\pgfsetstrokecolor{currentstroke}%
\pgfsetdash{}{0pt}%
\pgfpathmoveto{\pgfqpoint{2.900429in}{2.333205in}}%
\pgfpathcurveto{\pgfqpoint{2.911479in}{2.333205in}}{\pgfqpoint{2.922078in}{2.337596in}}{\pgfqpoint{2.929892in}{2.345409in}}%
\pgfpathcurveto{\pgfqpoint{2.937705in}{2.353223in}}{\pgfqpoint{2.942096in}{2.363822in}}{\pgfqpoint{2.942096in}{2.374872in}}%
\pgfpathcurveto{\pgfqpoint{2.942096in}{2.385922in}}{\pgfqpoint{2.937705in}{2.396521in}}{\pgfqpoint{2.929892in}{2.404335in}}%
\pgfpathcurveto{\pgfqpoint{2.922078in}{2.412148in}}{\pgfqpoint{2.911479in}{2.416539in}}{\pgfqpoint{2.900429in}{2.416539in}}%
\pgfpathcurveto{\pgfqpoint{2.889379in}{2.416539in}}{\pgfqpoint{2.878780in}{2.412148in}}{\pgfqpoint{2.870966in}{2.404335in}}%
\pgfpathcurveto{\pgfqpoint{2.863152in}{2.396521in}}{\pgfqpoint{2.858762in}{2.385922in}}{\pgfqpoint{2.858762in}{2.374872in}}%
\pgfpathcurveto{\pgfqpoint{2.858762in}{2.363822in}}{\pgfqpoint{2.863152in}{2.353223in}}{\pgfqpoint{2.870966in}{2.345409in}}%
\pgfpathcurveto{\pgfqpoint{2.878780in}{2.337596in}}{\pgfqpoint{2.889379in}{2.333205in}}{\pgfqpoint{2.900429in}{2.333205in}}%
\pgfpathclose%
\pgfusepath{stroke,fill}%
\end{pgfscope}%
\begin{pgfscope}%
\pgfpathrectangle{\pgfqpoint{0.600000in}{0.600000in}}{\pgfqpoint{3.900000in}{3.900000in}}%
\pgfusepath{clip}%
\pgfsetbuttcap%
\pgfsetroundjoin%
\definecolor{currentfill}{rgb}{0.121569,0.466667,0.705882}%
\pgfsetfillcolor{currentfill}%
\pgfsetlinewidth{1.003750pt}%
\definecolor{currentstroke}{rgb}{0.121569,0.466667,0.705882}%
\pgfsetstrokecolor{currentstroke}%
\pgfsetdash{}{0pt}%
\pgfpathmoveto{\pgfqpoint{3.467068in}{2.437613in}}%
\pgfpathcurveto{\pgfqpoint{3.478118in}{2.437613in}}{\pgfqpoint{3.488717in}{2.442004in}}{\pgfqpoint{3.496531in}{2.449817in}}%
\pgfpathcurveto{\pgfqpoint{3.504344in}{2.457631in}}{\pgfqpoint{3.508735in}{2.468230in}}{\pgfqpoint{3.508735in}{2.479280in}}%
\pgfpathcurveto{\pgfqpoint{3.508735in}{2.490330in}}{\pgfqpoint{3.504344in}{2.500929in}}{\pgfqpoint{3.496531in}{2.508743in}}%
\pgfpathcurveto{\pgfqpoint{3.488717in}{2.516556in}}{\pgfqpoint{3.478118in}{2.520947in}}{\pgfqpoint{3.467068in}{2.520947in}}%
\pgfpathcurveto{\pgfqpoint{3.456018in}{2.520947in}}{\pgfqpoint{3.445419in}{2.516556in}}{\pgfqpoint{3.437605in}{2.508743in}}%
\pgfpathcurveto{\pgfqpoint{3.429792in}{2.500929in}}{\pgfqpoint{3.425401in}{2.490330in}}{\pgfqpoint{3.425401in}{2.479280in}}%
\pgfpathcurveto{\pgfqpoint{3.425401in}{2.468230in}}{\pgfqpoint{3.429792in}{2.457631in}}{\pgfqpoint{3.437605in}{2.449817in}}%
\pgfpathcurveto{\pgfqpoint{3.445419in}{2.442004in}}{\pgfqpoint{3.456018in}{2.437613in}}{\pgfqpoint{3.467068in}{2.437613in}}%
\pgfpathclose%
\pgfusepath{stroke,fill}%
\end{pgfscope}%
\begin{pgfscope}%
\pgfpathrectangle{\pgfqpoint{0.600000in}{0.600000in}}{\pgfqpoint{3.900000in}{3.900000in}}%
\pgfusepath{clip}%
\pgfsetbuttcap%
\pgfsetroundjoin%
\definecolor{currentfill}{rgb}{0.121569,0.466667,0.705882}%
\pgfsetfillcolor{currentfill}%
\pgfsetlinewidth{1.003750pt}%
\definecolor{currentstroke}{rgb}{0.121569,0.466667,0.705882}%
\pgfsetstrokecolor{currentstroke}%
\pgfsetdash{}{0pt}%
\pgfpathmoveto{\pgfqpoint{2.237790in}{3.144366in}}%
\pgfpathcurveto{\pgfqpoint{2.248840in}{3.144366in}}{\pgfqpoint{2.259439in}{3.148756in}}{\pgfqpoint{2.267253in}{3.156569in}}%
\pgfpathcurveto{\pgfqpoint{2.275066in}{3.164383in}}{\pgfqpoint{2.279456in}{3.174982in}}{\pgfqpoint{2.279456in}{3.186032in}}%
\pgfpathcurveto{\pgfqpoint{2.279456in}{3.197082in}}{\pgfqpoint{2.275066in}{3.207681in}}{\pgfqpoint{2.267253in}{3.215495in}}%
\pgfpathcurveto{\pgfqpoint{2.259439in}{3.223309in}}{\pgfqpoint{2.248840in}{3.227699in}}{\pgfqpoint{2.237790in}{3.227699in}}%
\pgfpathcurveto{\pgfqpoint{2.226740in}{3.227699in}}{\pgfqpoint{2.216141in}{3.223309in}}{\pgfqpoint{2.208327in}{3.215495in}}%
\pgfpathcurveto{\pgfqpoint{2.200513in}{3.207681in}}{\pgfqpoint{2.196123in}{3.197082in}}{\pgfqpoint{2.196123in}{3.186032in}}%
\pgfpathcurveto{\pgfqpoint{2.196123in}{3.174982in}}{\pgfqpoint{2.200513in}{3.164383in}}{\pgfqpoint{2.208327in}{3.156569in}}%
\pgfpathcurveto{\pgfqpoint{2.216141in}{3.148756in}}{\pgfqpoint{2.226740in}{3.144366in}}{\pgfqpoint{2.237790in}{3.144366in}}%
\pgfpathclose%
\pgfusepath{stroke,fill}%
\end{pgfscope}%
\begin{pgfscope}%
\pgfpathrectangle{\pgfqpoint{0.600000in}{0.600000in}}{\pgfqpoint{3.900000in}{3.900000in}}%
\pgfusepath{clip}%
\pgfsetbuttcap%
\pgfsetroundjoin%
\definecolor{currentfill}{rgb}{0.121569,0.466667,0.705882}%
\pgfsetfillcolor{currentfill}%
\pgfsetlinewidth{1.003750pt}%
\definecolor{currentstroke}{rgb}{0.121569,0.466667,0.705882}%
\pgfsetstrokecolor{currentstroke}%
\pgfsetdash{}{0pt}%
\pgfpathmoveto{\pgfqpoint{2.621577in}{2.847018in}}%
\pgfpathcurveto{\pgfqpoint{2.632627in}{2.847018in}}{\pgfqpoint{2.643226in}{2.851408in}}{\pgfqpoint{2.651039in}{2.859222in}}%
\pgfpathcurveto{\pgfqpoint{2.658853in}{2.867035in}}{\pgfqpoint{2.663243in}{2.877634in}}{\pgfqpoint{2.663243in}{2.888685in}}%
\pgfpathcurveto{\pgfqpoint{2.663243in}{2.899735in}}{\pgfqpoint{2.658853in}{2.910334in}}{\pgfqpoint{2.651039in}{2.918147in}}%
\pgfpathcurveto{\pgfqpoint{2.643226in}{2.925961in}}{\pgfqpoint{2.632627in}{2.930351in}}{\pgfqpoint{2.621577in}{2.930351in}}%
\pgfpathcurveto{\pgfqpoint{2.610526in}{2.930351in}}{\pgfqpoint{2.599927in}{2.925961in}}{\pgfqpoint{2.592114in}{2.918147in}}%
\pgfpathcurveto{\pgfqpoint{2.584300in}{2.910334in}}{\pgfqpoint{2.579910in}{2.899735in}}{\pgfqpoint{2.579910in}{2.888685in}}%
\pgfpathcurveto{\pgfqpoint{2.579910in}{2.877634in}}{\pgfqpoint{2.584300in}{2.867035in}}{\pgfqpoint{2.592114in}{2.859222in}}%
\pgfpathcurveto{\pgfqpoint{2.599927in}{2.851408in}}{\pgfqpoint{2.610526in}{2.847018in}}{\pgfqpoint{2.621577in}{2.847018in}}%
\pgfpathclose%
\pgfusepath{stroke,fill}%
\end{pgfscope}%
\begin{pgfscope}%
\pgfpathrectangle{\pgfqpoint{0.600000in}{0.600000in}}{\pgfqpoint{3.900000in}{3.900000in}}%
\pgfusepath{clip}%
\pgfsetbuttcap%
\pgfsetroundjoin%
\definecolor{currentfill}{rgb}{0.121569,0.466667,0.705882}%
\pgfsetfillcolor{currentfill}%
\pgfsetlinewidth{1.003750pt}%
\definecolor{currentstroke}{rgb}{0.121569,0.466667,0.705882}%
\pgfsetstrokecolor{currentstroke}%
\pgfsetdash{}{0pt}%
\pgfpathmoveto{\pgfqpoint{2.665515in}{3.127647in}}%
\pgfpathcurveto{\pgfqpoint{2.676565in}{3.127647in}}{\pgfqpoint{2.687164in}{3.132037in}}{\pgfqpoint{2.694978in}{3.139851in}}%
\pgfpathcurveto{\pgfqpoint{2.702792in}{3.147664in}}{\pgfqpoint{2.707182in}{3.158263in}}{\pgfqpoint{2.707182in}{3.169313in}}%
\pgfpathcurveto{\pgfqpoint{2.707182in}{3.180364in}}{\pgfqpoint{2.702792in}{3.190963in}}{\pgfqpoint{2.694978in}{3.198776in}}%
\pgfpathcurveto{\pgfqpoint{2.687164in}{3.206590in}}{\pgfqpoint{2.676565in}{3.210980in}}{\pgfqpoint{2.665515in}{3.210980in}}%
\pgfpathcurveto{\pgfqpoint{2.654465in}{3.210980in}}{\pgfqpoint{2.643866in}{3.206590in}}{\pgfqpoint{2.636052in}{3.198776in}}%
\pgfpathcurveto{\pgfqpoint{2.628239in}{3.190963in}}{\pgfqpoint{2.623849in}{3.180364in}}{\pgfqpoint{2.623849in}{3.169313in}}%
\pgfpathcurveto{\pgfqpoint{2.623849in}{3.158263in}}{\pgfqpoint{2.628239in}{3.147664in}}{\pgfqpoint{2.636052in}{3.139851in}}%
\pgfpathcurveto{\pgfqpoint{2.643866in}{3.132037in}}{\pgfqpoint{2.654465in}{3.127647in}}{\pgfqpoint{2.665515in}{3.127647in}}%
\pgfpathclose%
\pgfusepath{stroke,fill}%
\end{pgfscope}%
\begin{pgfscope}%
\pgfpathrectangle{\pgfqpoint{0.600000in}{0.600000in}}{\pgfqpoint{3.900000in}{3.900000in}}%
\pgfusepath{clip}%
\pgfsetbuttcap%
\pgfsetroundjoin%
\definecolor{currentfill}{rgb}{0.121569,0.466667,0.705882}%
\pgfsetfillcolor{currentfill}%
\pgfsetlinewidth{1.003750pt}%
\definecolor{currentstroke}{rgb}{0.121569,0.466667,0.705882}%
\pgfsetstrokecolor{currentstroke}%
\pgfsetdash{}{0pt}%
\pgfpathmoveto{\pgfqpoint{2.417756in}{1.977249in}}%
\pgfpathcurveto{\pgfqpoint{2.428806in}{1.977249in}}{\pgfqpoint{2.439405in}{1.981639in}}{\pgfqpoint{2.447219in}{1.989453in}}%
\pgfpathcurveto{\pgfqpoint{2.455033in}{1.997266in}}{\pgfqpoint{2.459423in}{2.007865in}}{\pgfqpoint{2.459423in}{2.018915in}}%
\pgfpathcurveto{\pgfqpoint{2.459423in}{2.029965in}}{\pgfqpoint{2.455033in}{2.040565in}}{\pgfqpoint{2.447219in}{2.048378in}}%
\pgfpathcurveto{\pgfqpoint{2.439405in}{2.056192in}}{\pgfqpoint{2.428806in}{2.060582in}}{\pgfqpoint{2.417756in}{2.060582in}}%
\pgfpathcurveto{\pgfqpoint{2.406706in}{2.060582in}}{\pgfqpoint{2.396107in}{2.056192in}}{\pgfqpoint{2.388293in}{2.048378in}}%
\pgfpathcurveto{\pgfqpoint{2.380480in}{2.040565in}}{\pgfqpoint{2.376090in}{2.029965in}}{\pgfqpoint{2.376090in}{2.018915in}}%
\pgfpathcurveto{\pgfqpoint{2.376090in}{2.007865in}}{\pgfqpoint{2.380480in}{1.997266in}}{\pgfqpoint{2.388293in}{1.989453in}}%
\pgfpathcurveto{\pgfqpoint{2.396107in}{1.981639in}}{\pgfqpoint{2.406706in}{1.977249in}}{\pgfqpoint{2.417756in}{1.977249in}}%
\pgfpathclose%
\pgfusepath{stroke,fill}%
\end{pgfscope}%
\begin{pgfscope}%
\pgfpathrectangle{\pgfqpoint{0.600000in}{0.600000in}}{\pgfqpoint{3.900000in}{3.900000in}}%
\pgfusepath{clip}%
\pgfsetbuttcap%
\pgfsetroundjoin%
\definecolor{currentfill}{rgb}{0.121569,0.466667,0.705882}%
\pgfsetfillcolor{currentfill}%
\pgfsetlinewidth{1.003750pt}%
\definecolor{currentstroke}{rgb}{0.121569,0.466667,0.705882}%
\pgfsetstrokecolor{currentstroke}%
\pgfsetdash{}{0pt}%
\pgfpathmoveto{\pgfqpoint{3.072646in}{1.684758in}}%
\pgfpathcurveto{\pgfqpoint{3.083696in}{1.684758in}}{\pgfqpoint{3.094295in}{1.689149in}}{\pgfqpoint{3.102109in}{1.696962in}}%
\pgfpathcurveto{\pgfqpoint{3.109922in}{1.704776in}}{\pgfqpoint{3.114313in}{1.715375in}}{\pgfqpoint{3.114313in}{1.726425in}}%
\pgfpathcurveto{\pgfqpoint{3.114313in}{1.737475in}}{\pgfqpoint{3.109922in}{1.748074in}}{\pgfqpoint{3.102109in}{1.755888in}}%
\pgfpathcurveto{\pgfqpoint{3.094295in}{1.763701in}}{\pgfqpoint{3.083696in}{1.768092in}}{\pgfqpoint{3.072646in}{1.768092in}}%
\pgfpathcurveto{\pgfqpoint{3.061596in}{1.768092in}}{\pgfqpoint{3.050997in}{1.763701in}}{\pgfqpoint{3.043183in}{1.755888in}}%
\pgfpathcurveto{\pgfqpoint{3.035370in}{1.748074in}}{\pgfqpoint{3.030979in}{1.737475in}}{\pgfqpoint{3.030979in}{1.726425in}}%
\pgfpathcurveto{\pgfqpoint{3.030979in}{1.715375in}}{\pgfqpoint{3.035370in}{1.704776in}}{\pgfqpoint{3.043183in}{1.696962in}}%
\pgfpathcurveto{\pgfqpoint{3.050997in}{1.689149in}}{\pgfqpoint{3.061596in}{1.684758in}}{\pgfqpoint{3.072646in}{1.684758in}}%
\pgfpathclose%
\pgfusepath{stroke,fill}%
\end{pgfscope}%
\begin{pgfscope}%
\pgfpathrectangle{\pgfqpoint{0.600000in}{0.600000in}}{\pgfqpoint{3.900000in}{3.900000in}}%
\pgfusepath{clip}%
\pgfsetbuttcap%
\pgfsetroundjoin%
\definecolor{currentfill}{rgb}{0.121569,0.466667,0.705882}%
\pgfsetfillcolor{currentfill}%
\pgfsetlinewidth{1.003750pt}%
\definecolor{currentstroke}{rgb}{0.121569,0.466667,0.705882}%
\pgfsetstrokecolor{currentstroke}%
\pgfsetdash{}{0pt}%
\pgfpathmoveto{\pgfqpoint{2.496935in}{1.651820in}}%
\pgfpathcurveto{\pgfqpoint{2.507985in}{1.651820in}}{\pgfqpoint{2.518584in}{1.656210in}}{\pgfqpoint{2.526398in}{1.664024in}}%
\pgfpathcurveto{\pgfqpoint{2.534212in}{1.671838in}}{\pgfqpoint{2.538602in}{1.682437in}}{\pgfqpoint{2.538602in}{1.693487in}}%
\pgfpathcurveto{\pgfqpoint{2.538602in}{1.704537in}}{\pgfqpoint{2.534212in}{1.715136in}}{\pgfqpoint{2.526398in}{1.722950in}}%
\pgfpathcurveto{\pgfqpoint{2.518584in}{1.730763in}}{\pgfqpoint{2.507985in}{1.735154in}}{\pgfqpoint{2.496935in}{1.735154in}}%
\pgfpathcurveto{\pgfqpoint{2.485885in}{1.735154in}}{\pgfqpoint{2.475286in}{1.730763in}}{\pgfqpoint{2.467472in}{1.722950in}}%
\pgfpathcurveto{\pgfqpoint{2.459659in}{1.715136in}}{\pgfqpoint{2.455269in}{1.704537in}}{\pgfqpoint{2.455269in}{1.693487in}}%
\pgfpathcurveto{\pgfqpoint{2.455269in}{1.682437in}}{\pgfqpoint{2.459659in}{1.671838in}}{\pgfqpoint{2.467472in}{1.664024in}}%
\pgfpathcurveto{\pgfqpoint{2.475286in}{1.656210in}}{\pgfqpoint{2.485885in}{1.651820in}}{\pgfqpoint{2.496935in}{1.651820in}}%
\pgfpathclose%
\pgfusepath{stroke,fill}%
\end{pgfscope}%
\begin{pgfscope}%
\pgfpathrectangle{\pgfqpoint{0.600000in}{0.600000in}}{\pgfqpoint{3.900000in}{3.900000in}}%
\pgfusepath{clip}%
\pgfsetbuttcap%
\pgfsetroundjoin%
\definecolor{currentfill}{rgb}{0.121569,0.466667,0.705882}%
\pgfsetfillcolor{currentfill}%
\pgfsetlinewidth{1.003750pt}%
\definecolor{currentstroke}{rgb}{0.121569,0.466667,0.705882}%
\pgfsetstrokecolor{currentstroke}%
\pgfsetdash{}{0pt}%
\pgfpathmoveto{\pgfqpoint{1.936319in}{2.442768in}}%
\pgfpathcurveto{\pgfqpoint{1.947369in}{2.442768in}}{\pgfqpoint{1.957968in}{2.447158in}}{\pgfqpoint{1.965782in}{2.454971in}}%
\pgfpathcurveto{\pgfqpoint{1.973596in}{2.462785in}}{\pgfqpoint{1.977986in}{2.473384in}}{\pgfqpoint{1.977986in}{2.484434in}}%
\pgfpathcurveto{\pgfqpoint{1.977986in}{2.495484in}}{\pgfqpoint{1.973596in}{2.506083in}}{\pgfqpoint{1.965782in}{2.513897in}}%
\pgfpathcurveto{\pgfqpoint{1.957968in}{2.521711in}}{\pgfqpoint{1.947369in}{2.526101in}}{\pgfqpoint{1.936319in}{2.526101in}}%
\pgfpathcurveto{\pgfqpoint{1.925269in}{2.526101in}}{\pgfqpoint{1.914670in}{2.521711in}}{\pgfqpoint{1.906856in}{2.513897in}}%
\pgfpathcurveto{\pgfqpoint{1.899043in}{2.506083in}}{\pgfqpoint{1.894653in}{2.495484in}}{\pgfqpoint{1.894653in}{2.484434in}}%
\pgfpathcurveto{\pgfqpoint{1.894653in}{2.473384in}}{\pgfqpoint{1.899043in}{2.462785in}}{\pgfqpoint{1.906856in}{2.454971in}}%
\pgfpathcurveto{\pgfqpoint{1.914670in}{2.447158in}}{\pgfqpoint{1.925269in}{2.442768in}}{\pgfqpoint{1.936319in}{2.442768in}}%
\pgfpathclose%
\pgfusepath{stroke,fill}%
\end{pgfscope}%
\begin{pgfscope}%
\pgfpathrectangle{\pgfqpoint{0.600000in}{0.600000in}}{\pgfqpoint{3.900000in}{3.900000in}}%
\pgfusepath{clip}%
\pgfsetbuttcap%
\pgfsetroundjoin%
\definecolor{currentfill}{rgb}{0.121569,0.466667,0.705882}%
\pgfsetfillcolor{currentfill}%
\pgfsetlinewidth{1.003750pt}%
\definecolor{currentstroke}{rgb}{0.121569,0.466667,0.705882}%
\pgfsetstrokecolor{currentstroke}%
\pgfsetdash{}{0pt}%
\pgfpathmoveto{\pgfqpoint{2.959115in}{2.156479in}}%
\pgfpathcurveto{\pgfqpoint{2.970165in}{2.156479in}}{\pgfqpoint{2.980764in}{2.160869in}}{\pgfqpoint{2.988578in}{2.168683in}}%
\pgfpathcurveto{\pgfqpoint{2.996392in}{2.176497in}}{\pgfqpoint{3.000782in}{2.187096in}}{\pgfqpoint{3.000782in}{2.198146in}}%
\pgfpathcurveto{\pgfqpoint{3.000782in}{2.209196in}}{\pgfqpoint{2.996392in}{2.219795in}}{\pgfqpoint{2.988578in}{2.227608in}}%
\pgfpathcurveto{\pgfqpoint{2.980764in}{2.235422in}}{\pgfqpoint{2.970165in}{2.239812in}}{\pgfqpoint{2.959115in}{2.239812in}}%
\pgfpathcurveto{\pgfqpoint{2.948065in}{2.239812in}}{\pgfqpoint{2.937466in}{2.235422in}}{\pgfqpoint{2.929652in}{2.227608in}}%
\pgfpathcurveto{\pgfqpoint{2.921839in}{2.219795in}}{\pgfqpoint{2.917449in}{2.209196in}}{\pgfqpoint{2.917449in}{2.198146in}}%
\pgfpathcurveto{\pgfqpoint{2.917449in}{2.187096in}}{\pgfqpoint{2.921839in}{2.176497in}}{\pgfqpoint{2.929652in}{2.168683in}}%
\pgfpathcurveto{\pgfqpoint{2.937466in}{2.160869in}}{\pgfqpoint{2.948065in}{2.156479in}}{\pgfqpoint{2.959115in}{2.156479in}}%
\pgfpathclose%
\pgfusepath{stroke,fill}%
\end{pgfscope}%
\begin{pgfscope}%
\pgfpathrectangle{\pgfqpoint{0.600000in}{0.600000in}}{\pgfqpoint{3.900000in}{3.900000in}}%
\pgfusepath{clip}%
\pgfsetbuttcap%
\pgfsetroundjoin%
\definecolor{currentfill}{rgb}{0.121569,0.466667,0.705882}%
\pgfsetfillcolor{currentfill}%
\pgfsetlinewidth{1.003750pt}%
\definecolor{currentstroke}{rgb}{0.121569,0.466667,0.705882}%
\pgfsetstrokecolor{currentstroke}%
\pgfsetdash{}{0pt}%
\pgfpathmoveto{\pgfqpoint{2.577618in}{2.109620in}}%
\pgfpathcurveto{\pgfqpoint{2.588668in}{2.109620in}}{\pgfqpoint{2.599267in}{2.114010in}}{\pgfqpoint{2.607081in}{2.121823in}}%
\pgfpathcurveto{\pgfqpoint{2.614894in}{2.129637in}}{\pgfqpoint{2.619285in}{2.140236in}}{\pgfqpoint{2.619285in}{2.151286in}}%
\pgfpathcurveto{\pgfqpoint{2.619285in}{2.162336in}}{\pgfqpoint{2.614894in}{2.172935in}}{\pgfqpoint{2.607081in}{2.180749in}}%
\pgfpathcurveto{\pgfqpoint{2.599267in}{2.188563in}}{\pgfqpoint{2.588668in}{2.192953in}}{\pgfqpoint{2.577618in}{2.192953in}}%
\pgfpathcurveto{\pgfqpoint{2.566568in}{2.192953in}}{\pgfqpoint{2.555969in}{2.188563in}}{\pgfqpoint{2.548155in}{2.180749in}}%
\pgfpathcurveto{\pgfqpoint{2.540341in}{2.172935in}}{\pgfqpoint{2.535951in}{2.162336in}}{\pgfqpoint{2.535951in}{2.151286in}}%
\pgfpathcurveto{\pgfqpoint{2.535951in}{2.140236in}}{\pgfqpoint{2.540341in}{2.129637in}}{\pgfqpoint{2.548155in}{2.121823in}}%
\pgfpathcurveto{\pgfqpoint{2.555969in}{2.114010in}}{\pgfqpoint{2.566568in}{2.109620in}}{\pgfqpoint{2.577618in}{2.109620in}}%
\pgfpathclose%
\pgfusepath{stroke,fill}%
\end{pgfscope}%
\begin{pgfscope}%
\pgfpathrectangle{\pgfqpoint{0.600000in}{0.600000in}}{\pgfqpoint{3.900000in}{3.900000in}}%
\pgfusepath{clip}%
\pgfsetbuttcap%
\pgfsetroundjoin%
\definecolor{currentfill}{rgb}{0.121569,0.466667,0.705882}%
\pgfsetfillcolor{currentfill}%
\pgfsetlinewidth{1.003750pt}%
\definecolor{currentstroke}{rgb}{0.121569,0.466667,0.705882}%
\pgfsetstrokecolor{currentstroke}%
\pgfsetdash{}{0pt}%
\pgfpathmoveto{\pgfqpoint{2.366347in}{2.539569in}}%
\pgfpathcurveto{\pgfqpoint{2.377397in}{2.539569in}}{\pgfqpoint{2.387996in}{2.543959in}}{\pgfqpoint{2.395810in}{2.551772in}}%
\pgfpathcurveto{\pgfqpoint{2.403623in}{2.559586in}}{\pgfqpoint{2.408014in}{2.570185in}}{\pgfqpoint{2.408014in}{2.581235in}}%
\pgfpathcurveto{\pgfqpoint{2.408014in}{2.592285in}}{\pgfqpoint{2.403623in}{2.602884in}}{\pgfqpoint{2.395810in}{2.610698in}}%
\pgfpathcurveto{\pgfqpoint{2.387996in}{2.618512in}}{\pgfqpoint{2.377397in}{2.622902in}}{\pgfqpoint{2.366347in}{2.622902in}}%
\pgfpathcurveto{\pgfqpoint{2.355297in}{2.622902in}}{\pgfqpoint{2.344698in}{2.618512in}}{\pgfqpoint{2.336884in}{2.610698in}}%
\pgfpathcurveto{\pgfqpoint{2.329070in}{2.602884in}}{\pgfqpoint{2.324680in}{2.592285in}}{\pgfqpoint{2.324680in}{2.581235in}}%
\pgfpathcurveto{\pgfqpoint{2.324680in}{2.570185in}}{\pgfqpoint{2.329070in}{2.559586in}}{\pgfqpoint{2.336884in}{2.551772in}}%
\pgfpathcurveto{\pgfqpoint{2.344698in}{2.543959in}}{\pgfqpoint{2.355297in}{2.539569in}}{\pgfqpoint{2.366347in}{2.539569in}}%
\pgfpathclose%
\pgfusepath{stroke,fill}%
\end{pgfscope}%
\begin{pgfscope}%
\pgfpathrectangle{\pgfqpoint{0.600000in}{0.600000in}}{\pgfqpoint{3.900000in}{3.900000in}}%
\pgfusepath{clip}%
\pgfsetbuttcap%
\pgfsetroundjoin%
\definecolor{currentfill}{rgb}{0.121569,0.466667,0.705882}%
\pgfsetfillcolor{currentfill}%
\pgfsetlinewidth{1.003750pt}%
\definecolor{currentstroke}{rgb}{0.121569,0.466667,0.705882}%
\pgfsetstrokecolor{currentstroke}%
\pgfsetdash{}{0pt}%
\pgfpathmoveto{\pgfqpoint{2.713549in}{1.741414in}}%
\pgfpathcurveto{\pgfqpoint{2.724599in}{1.741414in}}{\pgfqpoint{2.735198in}{1.745804in}}{\pgfqpoint{2.743012in}{1.753618in}}%
\pgfpathcurveto{\pgfqpoint{2.750826in}{1.761432in}}{\pgfqpoint{2.755216in}{1.772031in}}{\pgfqpoint{2.755216in}{1.783081in}}%
\pgfpathcurveto{\pgfqpoint{2.755216in}{1.794131in}}{\pgfqpoint{2.750826in}{1.804730in}}{\pgfqpoint{2.743012in}{1.812543in}}%
\pgfpathcurveto{\pgfqpoint{2.735198in}{1.820357in}}{\pgfqpoint{2.724599in}{1.824747in}}{\pgfqpoint{2.713549in}{1.824747in}}%
\pgfpathcurveto{\pgfqpoint{2.702499in}{1.824747in}}{\pgfqpoint{2.691900in}{1.820357in}}{\pgfqpoint{2.684087in}{1.812543in}}%
\pgfpathcurveto{\pgfqpoint{2.676273in}{1.804730in}}{\pgfqpoint{2.671883in}{1.794131in}}{\pgfqpoint{2.671883in}{1.783081in}}%
\pgfpathcurveto{\pgfqpoint{2.671883in}{1.772031in}}{\pgfqpoint{2.676273in}{1.761432in}}{\pgfqpoint{2.684087in}{1.753618in}}%
\pgfpathcurveto{\pgfqpoint{2.691900in}{1.745804in}}{\pgfqpoint{2.702499in}{1.741414in}}{\pgfqpoint{2.713549in}{1.741414in}}%
\pgfpathclose%
\pgfusepath{stroke,fill}%
\end{pgfscope}%
\begin{pgfscope}%
\pgfpathrectangle{\pgfqpoint{0.600000in}{0.600000in}}{\pgfqpoint{3.900000in}{3.900000in}}%
\pgfusepath{clip}%
\pgfsetbuttcap%
\pgfsetroundjoin%
\definecolor{currentfill}{rgb}{0.121569,0.466667,0.705882}%
\pgfsetfillcolor{currentfill}%
\pgfsetlinewidth{1.003750pt}%
\definecolor{currentstroke}{rgb}{0.121569,0.466667,0.705882}%
\pgfsetstrokecolor{currentstroke}%
\pgfsetdash{}{0pt}%
\pgfpathmoveto{\pgfqpoint{2.435445in}{2.604744in}}%
\pgfpathcurveto{\pgfqpoint{2.446495in}{2.604744in}}{\pgfqpoint{2.457094in}{2.609134in}}{\pgfqpoint{2.464908in}{2.616948in}}%
\pgfpathcurveto{\pgfqpoint{2.472721in}{2.624761in}}{\pgfqpoint{2.477112in}{2.635360in}}{\pgfqpoint{2.477112in}{2.646410in}}%
\pgfpathcurveto{\pgfqpoint{2.477112in}{2.657460in}}{\pgfqpoint{2.472721in}{2.668059in}}{\pgfqpoint{2.464908in}{2.675873in}}%
\pgfpathcurveto{\pgfqpoint{2.457094in}{2.683687in}}{\pgfqpoint{2.446495in}{2.688077in}}{\pgfqpoint{2.435445in}{2.688077in}}%
\pgfpathcurveto{\pgfqpoint{2.424395in}{2.688077in}}{\pgfqpoint{2.413796in}{2.683687in}}{\pgfqpoint{2.405982in}{2.675873in}}%
\pgfpathcurveto{\pgfqpoint{2.398169in}{2.668059in}}{\pgfqpoint{2.393778in}{2.657460in}}{\pgfqpoint{2.393778in}{2.646410in}}%
\pgfpathcurveto{\pgfqpoint{2.393778in}{2.635360in}}{\pgfqpoint{2.398169in}{2.624761in}}{\pgfqpoint{2.405982in}{2.616948in}}%
\pgfpathcurveto{\pgfqpoint{2.413796in}{2.609134in}}{\pgfqpoint{2.424395in}{2.604744in}}{\pgfqpoint{2.435445in}{2.604744in}}%
\pgfpathclose%
\pgfusepath{stroke,fill}%
\end{pgfscope}%
\begin{pgfscope}%
\pgfpathrectangle{\pgfqpoint{0.600000in}{0.600000in}}{\pgfqpoint{3.900000in}{3.900000in}}%
\pgfusepath{clip}%
\pgfsetbuttcap%
\pgfsetroundjoin%
\definecolor{currentfill}{rgb}{0.121569,0.466667,0.705882}%
\pgfsetfillcolor{currentfill}%
\pgfsetlinewidth{1.003750pt}%
\definecolor{currentstroke}{rgb}{0.121569,0.466667,0.705882}%
\pgfsetstrokecolor{currentstroke}%
\pgfsetdash{}{0pt}%
\pgfpathmoveto{\pgfqpoint{2.311296in}{1.937143in}}%
\pgfpathcurveto{\pgfqpoint{2.322346in}{1.937143in}}{\pgfqpoint{2.332945in}{1.941533in}}{\pgfqpoint{2.340759in}{1.949347in}}%
\pgfpathcurveto{\pgfqpoint{2.348573in}{1.957160in}}{\pgfqpoint{2.352963in}{1.967759in}}{\pgfqpoint{2.352963in}{1.978809in}}%
\pgfpathcurveto{\pgfqpoint{2.352963in}{1.989860in}}{\pgfqpoint{2.348573in}{2.000459in}}{\pgfqpoint{2.340759in}{2.008272in}}%
\pgfpathcurveto{\pgfqpoint{2.332945in}{2.016086in}}{\pgfqpoint{2.322346in}{2.020476in}}{\pgfqpoint{2.311296in}{2.020476in}}%
\pgfpathcurveto{\pgfqpoint{2.300246in}{2.020476in}}{\pgfqpoint{2.289647in}{2.016086in}}{\pgfqpoint{2.281833in}{2.008272in}}%
\pgfpathcurveto{\pgfqpoint{2.274020in}{2.000459in}}{\pgfqpoint{2.269629in}{1.989860in}}{\pgfqpoint{2.269629in}{1.978809in}}%
\pgfpathcurveto{\pgfqpoint{2.269629in}{1.967759in}}{\pgfqpoint{2.274020in}{1.957160in}}{\pgfqpoint{2.281833in}{1.949347in}}%
\pgfpathcurveto{\pgfqpoint{2.289647in}{1.941533in}}{\pgfqpoint{2.300246in}{1.937143in}}{\pgfqpoint{2.311296in}{1.937143in}}%
\pgfpathclose%
\pgfusepath{stroke,fill}%
\end{pgfscope}%
\begin{pgfscope}%
\pgfpathrectangle{\pgfqpoint{0.600000in}{0.600000in}}{\pgfqpoint{3.900000in}{3.900000in}}%
\pgfusepath{clip}%
\pgfsetbuttcap%
\pgfsetroundjoin%
\definecolor{currentfill}{rgb}{0.121569,0.466667,0.705882}%
\pgfsetfillcolor{currentfill}%
\pgfsetlinewidth{1.003750pt}%
\definecolor{currentstroke}{rgb}{0.121569,0.466667,0.705882}%
\pgfsetstrokecolor{currentstroke}%
\pgfsetdash{}{0pt}%
\pgfpathmoveto{\pgfqpoint{3.246574in}{2.660248in}}%
\pgfpathcurveto{\pgfqpoint{3.257624in}{2.660248in}}{\pgfqpoint{3.268223in}{2.664638in}}{\pgfqpoint{3.276037in}{2.672452in}}%
\pgfpathcurveto{\pgfqpoint{3.283850in}{2.680265in}}{\pgfqpoint{3.288240in}{2.690864in}}{\pgfqpoint{3.288240in}{2.701914in}}%
\pgfpathcurveto{\pgfqpoint{3.288240in}{2.712964in}}{\pgfqpoint{3.283850in}{2.723563in}}{\pgfqpoint{3.276037in}{2.731377in}}%
\pgfpathcurveto{\pgfqpoint{3.268223in}{2.739191in}}{\pgfqpoint{3.257624in}{2.743581in}}{\pgfqpoint{3.246574in}{2.743581in}}%
\pgfpathcurveto{\pgfqpoint{3.235524in}{2.743581in}}{\pgfqpoint{3.224925in}{2.739191in}}{\pgfqpoint{3.217111in}{2.731377in}}%
\pgfpathcurveto{\pgfqpoint{3.209297in}{2.723563in}}{\pgfqpoint{3.204907in}{2.712964in}}{\pgfqpoint{3.204907in}{2.701914in}}%
\pgfpathcurveto{\pgfqpoint{3.204907in}{2.690864in}}{\pgfqpoint{3.209297in}{2.680265in}}{\pgfqpoint{3.217111in}{2.672452in}}%
\pgfpathcurveto{\pgfqpoint{3.224925in}{2.664638in}}{\pgfqpoint{3.235524in}{2.660248in}}{\pgfqpoint{3.246574in}{2.660248in}}%
\pgfpathclose%
\pgfusepath{stroke,fill}%
\end{pgfscope}%
\begin{pgfscope}%
\pgfpathrectangle{\pgfqpoint{0.600000in}{0.600000in}}{\pgfqpoint{3.900000in}{3.900000in}}%
\pgfusepath{clip}%
\pgfsetbuttcap%
\pgfsetroundjoin%
\definecolor{currentfill}{rgb}{0.121569,0.466667,0.705882}%
\pgfsetfillcolor{currentfill}%
\pgfsetlinewidth{1.003750pt}%
\definecolor{currentstroke}{rgb}{0.121569,0.466667,0.705882}%
\pgfsetstrokecolor{currentstroke}%
\pgfsetdash{}{0pt}%
\pgfpathmoveto{\pgfqpoint{1.794028in}{1.889908in}}%
\pgfpathcurveto{\pgfqpoint{1.805078in}{1.889908in}}{\pgfqpoint{1.815677in}{1.894298in}}{\pgfqpoint{1.823491in}{1.902112in}}%
\pgfpathcurveto{\pgfqpoint{1.831305in}{1.909926in}}{\pgfqpoint{1.835695in}{1.920525in}}{\pgfqpoint{1.835695in}{1.931575in}}%
\pgfpathcurveto{\pgfqpoint{1.835695in}{1.942625in}}{\pgfqpoint{1.831305in}{1.953224in}}{\pgfqpoint{1.823491in}{1.961038in}}%
\pgfpathcurveto{\pgfqpoint{1.815677in}{1.968851in}}{\pgfqpoint{1.805078in}{1.973242in}}{\pgfqpoint{1.794028in}{1.973242in}}%
\pgfpathcurveto{\pgfqpoint{1.782978in}{1.973242in}}{\pgfqpoint{1.772379in}{1.968851in}}{\pgfqpoint{1.764566in}{1.961038in}}%
\pgfpathcurveto{\pgfqpoint{1.756752in}{1.953224in}}{\pgfqpoint{1.752362in}{1.942625in}}{\pgfqpoint{1.752362in}{1.931575in}}%
\pgfpathcurveto{\pgfqpoint{1.752362in}{1.920525in}}{\pgfqpoint{1.756752in}{1.909926in}}{\pgfqpoint{1.764566in}{1.902112in}}%
\pgfpathcurveto{\pgfqpoint{1.772379in}{1.894298in}}{\pgfqpoint{1.782978in}{1.889908in}}{\pgfqpoint{1.794028in}{1.889908in}}%
\pgfpathclose%
\pgfusepath{stroke,fill}%
\end{pgfscope}%
\begin{pgfscope}%
\pgfpathrectangle{\pgfqpoint{0.600000in}{0.600000in}}{\pgfqpoint{3.900000in}{3.900000in}}%
\pgfusepath{clip}%
\pgfsetbuttcap%
\pgfsetroundjoin%
\definecolor{currentfill}{rgb}{0.121569,0.466667,0.705882}%
\pgfsetfillcolor{currentfill}%
\pgfsetlinewidth{1.003750pt}%
\definecolor{currentstroke}{rgb}{0.121569,0.466667,0.705882}%
\pgfsetstrokecolor{currentstroke}%
\pgfsetdash{}{0pt}%
\pgfpathmoveto{\pgfqpoint{3.414099in}{2.715340in}}%
\pgfpathcurveto{\pgfqpoint{3.425149in}{2.715340in}}{\pgfqpoint{3.435748in}{2.719730in}}{\pgfqpoint{3.443562in}{2.727544in}}%
\pgfpathcurveto{\pgfqpoint{3.451375in}{2.735357in}}{\pgfqpoint{3.455766in}{2.745956in}}{\pgfqpoint{3.455766in}{2.757006in}}%
\pgfpathcurveto{\pgfqpoint{3.455766in}{2.768056in}}{\pgfqpoint{3.451375in}{2.778655in}}{\pgfqpoint{3.443562in}{2.786469in}}%
\pgfpathcurveto{\pgfqpoint{3.435748in}{2.794283in}}{\pgfqpoint{3.425149in}{2.798673in}}{\pgfqpoint{3.414099in}{2.798673in}}%
\pgfpathcurveto{\pgfqpoint{3.403049in}{2.798673in}}{\pgfqpoint{3.392450in}{2.794283in}}{\pgfqpoint{3.384636in}{2.786469in}}%
\pgfpathcurveto{\pgfqpoint{3.376823in}{2.778655in}}{\pgfqpoint{3.372432in}{2.768056in}}{\pgfqpoint{3.372432in}{2.757006in}}%
\pgfpathcurveto{\pgfqpoint{3.372432in}{2.745956in}}{\pgfqpoint{3.376823in}{2.735357in}}{\pgfqpoint{3.384636in}{2.727544in}}%
\pgfpathcurveto{\pgfqpoint{3.392450in}{2.719730in}}{\pgfqpoint{3.403049in}{2.715340in}}{\pgfqpoint{3.414099in}{2.715340in}}%
\pgfpathclose%
\pgfusepath{stroke,fill}%
\end{pgfscope}%
\begin{pgfscope}%
\pgfpathrectangle{\pgfqpoint{0.600000in}{0.600000in}}{\pgfqpoint{3.900000in}{3.900000in}}%
\pgfusepath{clip}%
\pgfsetbuttcap%
\pgfsetroundjoin%
\definecolor{currentfill}{rgb}{0.121569,0.466667,0.705882}%
\pgfsetfillcolor{currentfill}%
\pgfsetlinewidth{1.003750pt}%
\definecolor{currentstroke}{rgb}{0.121569,0.466667,0.705882}%
\pgfsetstrokecolor{currentstroke}%
\pgfsetdash{}{0pt}%
\pgfpathmoveto{\pgfqpoint{2.467638in}{1.974925in}}%
\pgfpathcurveto{\pgfqpoint{2.478688in}{1.974925in}}{\pgfqpoint{2.489287in}{1.979316in}}{\pgfqpoint{2.497101in}{1.987129in}}%
\pgfpathcurveto{\pgfqpoint{2.504914in}{1.994943in}}{\pgfqpoint{2.509305in}{2.005542in}}{\pgfqpoint{2.509305in}{2.016592in}}%
\pgfpathcurveto{\pgfqpoint{2.509305in}{2.027642in}}{\pgfqpoint{2.504914in}{2.038241in}}{\pgfqpoint{2.497101in}{2.046055in}}%
\pgfpathcurveto{\pgfqpoint{2.489287in}{2.053868in}}{\pgfqpoint{2.478688in}{2.058259in}}{\pgfqpoint{2.467638in}{2.058259in}}%
\pgfpathcurveto{\pgfqpoint{2.456588in}{2.058259in}}{\pgfqpoint{2.445989in}{2.053868in}}{\pgfqpoint{2.438175in}{2.046055in}}%
\pgfpathcurveto{\pgfqpoint{2.430362in}{2.038241in}}{\pgfqpoint{2.425971in}{2.027642in}}{\pgfqpoint{2.425971in}{2.016592in}}%
\pgfpathcurveto{\pgfqpoint{2.425971in}{2.005542in}}{\pgfqpoint{2.430362in}{1.994943in}}{\pgfqpoint{2.438175in}{1.987129in}}%
\pgfpathcurveto{\pgfqpoint{2.445989in}{1.979316in}}{\pgfqpoint{2.456588in}{1.974925in}}{\pgfqpoint{2.467638in}{1.974925in}}%
\pgfpathclose%
\pgfusepath{stroke,fill}%
\end{pgfscope}%
\begin{pgfscope}%
\pgfpathrectangle{\pgfqpoint{0.600000in}{0.600000in}}{\pgfqpoint{3.900000in}{3.900000in}}%
\pgfusepath{clip}%
\pgfsetbuttcap%
\pgfsetroundjoin%
\definecolor{currentfill}{rgb}{0.121569,0.466667,0.705882}%
\pgfsetfillcolor{currentfill}%
\pgfsetlinewidth{1.003750pt}%
\definecolor{currentstroke}{rgb}{0.121569,0.466667,0.705882}%
\pgfsetstrokecolor{currentstroke}%
\pgfsetdash{}{0pt}%
\pgfpathmoveto{\pgfqpoint{2.624513in}{2.403973in}}%
\pgfpathcurveto{\pgfqpoint{2.635563in}{2.403973in}}{\pgfqpoint{2.646162in}{2.408364in}}{\pgfqpoint{2.653976in}{2.416177in}}%
\pgfpathcurveto{\pgfqpoint{2.661790in}{2.423991in}}{\pgfqpoint{2.666180in}{2.434590in}}{\pgfqpoint{2.666180in}{2.445640in}}%
\pgfpathcurveto{\pgfqpoint{2.666180in}{2.456690in}}{\pgfqpoint{2.661790in}{2.467289in}}{\pgfqpoint{2.653976in}{2.475103in}}%
\pgfpathcurveto{\pgfqpoint{2.646162in}{2.482916in}}{\pgfqpoint{2.635563in}{2.487307in}}{\pgfqpoint{2.624513in}{2.487307in}}%
\pgfpathcurveto{\pgfqpoint{2.613463in}{2.487307in}}{\pgfqpoint{2.602864in}{2.482916in}}{\pgfqpoint{2.595050in}{2.475103in}}%
\pgfpathcurveto{\pgfqpoint{2.587237in}{2.467289in}}{\pgfqpoint{2.582847in}{2.456690in}}{\pgfqpoint{2.582847in}{2.445640in}}%
\pgfpathcurveto{\pgfqpoint{2.582847in}{2.434590in}}{\pgfqpoint{2.587237in}{2.423991in}}{\pgfqpoint{2.595050in}{2.416177in}}%
\pgfpathcurveto{\pgfqpoint{2.602864in}{2.408364in}}{\pgfqpoint{2.613463in}{2.403973in}}{\pgfqpoint{2.624513in}{2.403973in}}%
\pgfpathclose%
\pgfusepath{stroke,fill}%
\end{pgfscope}%
\begin{pgfscope}%
\pgfpathrectangle{\pgfqpoint{0.600000in}{0.600000in}}{\pgfqpoint{3.900000in}{3.900000in}}%
\pgfusepath{clip}%
\pgfsetbuttcap%
\pgfsetroundjoin%
\definecolor{currentfill}{rgb}{0.121569,0.466667,0.705882}%
\pgfsetfillcolor{currentfill}%
\pgfsetlinewidth{1.003750pt}%
\definecolor{currentstroke}{rgb}{0.121569,0.466667,0.705882}%
\pgfsetstrokecolor{currentstroke}%
\pgfsetdash{}{0pt}%
\pgfpathmoveto{\pgfqpoint{2.328908in}{2.371086in}}%
\pgfpathcurveto{\pgfqpoint{2.339958in}{2.371086in}}{\pgfqpoint{2.350557in}{2.375476in}}{\pgfqpoint{2.358370in}{2.383290in}}%
\pgfpathcurveto{\pgfqpoint{2.366184in}{2.391103in}}{\pgfqpoint{2.370574in}{2.401702in}}{\pgfqpoint{2.370574in}{2.412752in}}%
\pgfpathcurveto{\pgfqpoint{2.370574in}{2.423802in}}{\pgfqpoint{2.366184in}{2.434401in}}{\pgfqpoint{2.358370in}{2.442215in}}%
\pgfpathcurveto{\pgfqpoint{2.350557in}{2.450029in}}{\pgfqpoint{2.339958in}{2.454419in}}{\pgfqpoint{2.328908in}{2.454419in}}%
\pgfpathcurveto{\pgfqpoint{2.317858in}{2.454419in}}{\pgfqpoint{2.307259in}{2.450029in}}{\pgfqpoint{2.299445in}{2.442215in}}%
\pgfpathcurveto{\pgfqpoint{2.291631in}{2.434401in}}{\pgfqpoint{2.287241in}{2.423802in}}{\pgfqpoint{2.287241in}{2.412752in}}%
\pgfpathcurveto{\pgfqpoint{2.287241in}{2.401702in}}{\pgfqpoint{2.291631in}{2.391103in}}{\pgfqpoint{2.299445in}{2.383290in}}%
\pgfpathcurveto{\pgfqpoint{2.307259in}{2.375476in}}{\pgfqpoint{2.317858in}{2.371086in}}{\pgfqpoint{2.328908in}{2.371086in}}%
\pgfpathclose%
\pgfusepath{stroke,fill}%
\end{pgfscope}%
\begin{pgfscope}%
\pgfpathrectangle{\pgfqpoint{0.600000in}{0.600000in}}{\pgfqpoint{3.900000in}{3.900000in}}%
\pgfusepath{clip}%
\pgfsetbuttcap%
\pgfsetroundjoin%
\definecolor{currentfill}{rgb}{0.121569,0.466667,0.705882}%
\pgfsetfillcolor{currentfill}%
\pgfsetlinewidth{1.003750pt}%
\definecolor{currentstroke}{rgb}{0.121569,0.466667,0.705882}%
\pgfsetstrokecolor{currentstroke}%
\pgfsetdash{}{0pt}%
\pgfpathmoveto{\pgfqpoint{2.776465in}{1.895033in}}%
\pgfpathcurveto{\pgfqpoint{2.787516in}{1.895033in}}{\pgfqpoint{2.798115in}{1.899423in}}{\pgfqpoint{2.805928in}{1.907237in}}%
\pgfpathcurveto{\pgfqpoint{2.813742in}{1.915051in}}{\pgfqpoint{2.818132in}{1.925650in}}{\pgfqpoint{2.818132in}{1.936700in}}%
\pgfpathcurveto{\pgfqpoint{2.818132in}{1.947750in}}{\pgfqpoint{2.813742in}{1.958349in}}{\pgfqpoint{2.805928in}{1.966163in}}%
\pgfpathcurveto{\pgfqpoint{2.798115in}{1.973976in}}{\pgfqpoint{2.787516in}{1.978367in}}{\pgfqpoint{2.776465in}{1.978367in}}%
\pgfpathcurveto{\pgfqpoint{2.765415in}{1.978367in}}{\pgfqpoint{2.754816in}{1.973976in}}{\pgfqpoint{2.747003in}{1.966163in}}%
\pgfpathcurveto{\pgfqpoint{2.739189in}{1.958349in}}{\pgfqpoint{2.734799in}{1.947750in}}{\pgfqpoint{2.734799in}{1.936700in}}%
\pgfpathcurveto{\pgfqpoint{2.734799in}{1.925650in}}{\pgfqpoint{2.739189in}{1.915051in}}{\pgfqpoint{2.747003in}{1.907237in}}%
\pgfpathcurveto{\pgfqpoint{2.754816in}{1.899423in}}{\pgfqpoint{2.765415in}{1.895033in}}{\pgfqpoint{2.776465in}{1.895033in}}%
\pgfpathclose%
\pgfusepath{stroke,fill}%
\end{pgfscope}%
\begin{pgfscope}%
\pgfpathrectangle{\pgfqpoint{0.600000in}{0.600000in}}{\pgfqpoint{3.900000in}{3.900000in}}%
\pgfusepath{clip}%
\pgfsetbuttcap%
\pgfsetroundjoin%
\definecolor{currentfill}{rgb}{0.121569,0.466667,0.705882}%
\pgfsetfillcolor{currentfill}%
\pgfsetlinewidth{1.003750pt}%
\definecolor{currentstroke}{rgb}{0.121569,0.466667,0.705882}%
\pgfsetstrokecolor{currentstroke}%
\pgfsetdash{}{0pt}%
\pgfpathmoveto{\pgfqpoint{2.717358in}{2.500772in}}%
\pgfpathcurveto{\pgfqpoint{2.728408in}{2.500772in}}{\pgfqpoint{2.739007in}{2.505162in}}{\pgfqpoint{2.746820in}{2.512976in}}%
\pgfpathcurveto{\pgfqpoint{2.754634in}{2.520790in}}{\pgfqpoint{2.759024in}{2.531389in}}{\pgfqpoint{2.759024in}{2.542439in}}%
\pgfpathcurveto{\pgfqpoint{2.759024in}{2.553489in}}{\pgfqpoint{2.754634in}{2.564088in}}{\pgfqpoint{2.746820in}{2.571902in}}%
\pgfpathcurveto{\pgfqpoint{2.739007in}{2.579715in}}{\pgfqpoint{2.728408in}{2.584106in}}{\pgfqpoint{2.717358in}{2.584106in}}%
\pgfpathcurveto{\pgfqpoint{2.706308in}{2.584106in}}{\pgfqpoint{2.695709in}{2.579715in}}{\pgfqpoint{2.687895in}{2.571902in}}%
\pgfpathcurveto{\pgfqpoint{2.680081in}{2.564088in}}{\pgfqpoint{2.675691in}{2.553489in}}{\pgfqpoint{2.675691in}{2.542439in}}%
\pgfpathcurveto{\pgfqpoint{2.675691in}{2.531389in}}{\pgfqpoint{2.680081in}{2.520790in}}{\pgfqpoint{2.687895in}{2.512976in}}%
\pgfpathcurveto{\pgfqpoint{2.695709in}{2.505162in}}{\pgfqpoint{2.706308in}{2.500772in}}{\pgfqpoint{2.717358in}{2.500772in}}%
\pgfpathclose%
\pgfusepath{stroke,fill}%
\end{pgfscope}%
\begin{pgfscope}%
\pgfpathrectangle{\pgfqpoint{0.600000in}{0.600000in}}{\pgfqpoint{3.900000in}{3.900000in}}%
\pgfusepath{clip}%
\pgfsetbuttcap%
\pgfsetroundjoin%
\definecolor{currentfill}{rgb}{0.121569,0.466667,0.705882}%
\pgfsetfillcolor{currentfill}%
\pgfsetlinewidth{1.003750pt}%
\definecolor{currentstroke}{rgb}{0.121569,0.466667,0.705882}%
\pgfsetstrokecolor{currentstroke}%
\pgfsetdash{}{0pt}%
\pgfpathmoveto{\pgfqpoint{2.243611in}{1.293259in}}%
\pgfpathcurveto{\pgfqpoint{2.254661in}{1.293259in}}{\pgfqpoint{2.265260in}{1.297649in}}{\pgfqpoint{2.273073in}{1.305463in}}%
\pgfpathcurveto{\pgfqpoint{2.280887in}{1.313276in}}{\pgfqpoint{2.285277in}{1.323875in}}{\pgfqpoint{2.285277in}{1.334926in}}%
\pgfpathcurveto{\pgfqpoint{2.285277in}{1.345976in}}{\pgfqpoint{2.280887in}{1.356575in}}{\pgfqpoint{2.273073in}{1.364388in}}%
\pgfpathcurveto{\pgfqpoint{2.265260in}{1.372202in}}{\pgfqpoint{2.254661in}{1.376592in}}{\pgfqpoint{2.243611in}{1.376592in}}%
\pgfpathcurveto{\pgfqpoint{2.232561in}{1.376592in}}{\pgfqpoint{2.221962in}{1.372202in}}{\pgfqpoint{2.214148in}{1.364388in}}%
\pgfpathcurveto{\pgfqpoint{2.206334in}{1.356575in}}{\pgfqpoint{2.201944in}{1.345976in}}{\pgfqpoint{2.201944in}{1.334926in}}%
\pgfpathcurveto{\pgfqpoint{2.201944in}{1.323875in}}{\pgfqpoint{2.206334in}{1.313276in}}{\pgfqpoint{2.214148in}{1.305463in}}%
\pgfpathcurveto{\pgfqpoint{2.221962in}{1.297649in}}{\pgfqpoint{2.232561in}{1.293259in}}{\pgfqpoint{2.243611in}{1.293259in}}%
\pgfpathclose%
\pgfusepath{stroke,fill}%
\end{pgfscope}%
\begin{pgfscope}%
\pgfpathrectangle{\pgfqpoint{0.600000in}{0.600000in}}{\pgfqpoint{3.900000in}{3.900000in}}%
\pgfusepath{clip}%
\pgfsetbuttcap%
\pgfsetroundjoin%
\definecolor{currentfill}{rgb}{0.121569,0.466667,0.705882}%
\pgfsetfillcolor{currentfill}%
\pgfsetlinewidth{1.003750pt}%
\definecolor{currentstroke}{rgb}{0.121569,0.466667,0.705882}%
\pgfsetstrokecolor{currentstroke}%
\pgfsetdash{}{0pt}%
\pgfpathmoveto{\pgfqpoint{2.420005in}{2.284713in}}%
\pgfpathcurveto{\pgfqpoint{2.431055in}{2.284713in}}{\pgfqpoint{2.441654in}{2.289103in}}{\pgfqpoint{2.449467in}{2.296917in}}%
\pgfpathcurveto{\pgfqpoint{2.457281in}{2.304730in}}{\pgfqpoint{2.461671in}{2.315330in}}{\pgfqpoint{2.461671in}{2.326380in}}%
\pgfpathcurveto{\pgfqpoint{2.461671in}{2.337430in}}{\pgfqpoint{2.457281in}{2.348029in}}{\pgfqpoint{2.449467in}{2.355842in}}%
\pgfpathcurveto{\pgfqpoint{2.441654in}{2.363656in}}{\pgfqpoint{2.431055in}{2.368046in}}{\pgfqpoint{2.420005in}{2.368046in}}%
\pgfpathcurveto{\pgfqpoint{2.408954in}{2.368046in}}{\pgfqpoint{2.398355in}{2.363656in}}{\pgfqpoint{2.390542in}{2.355842in}}%
\pgfpathcurveto{\pgfqpoint{2.382728in}{2.348029in}}{\pgfqpoint{2.378338in}{2.337430in}}{\pgfqpoint{2.378338in}{2.326380in}}%
\pgfpathcurveto{\pgfqpoint{2.378338in}{2.315330in}}{\pgfqpoint{2.382728in}{2.304730in}}{\pgfqpoint{2.390542in}{2.296917in}}%
\pgfpathcurveto{\pgfqpoint{2.398355in}{2.289103in}}{\pgfqpoint{2.408954in}{2.284713in}}{\pgfqpoint{2.420005in}{2.284713in}}%
\pgfpathclose%
\pgfusepath{stroke,fill}%
\end{pgfscope}%
\begin{pgfscope}%
\pgfpathrectangle{\pgfqpoint{0.600000in}{0.600000in}}{\pgfqpoint{3.900000in}{3.900000in}}%
\pgfusepath{clip}%
\pgfsetbuttcap%
\pgfsetroundjoin%
\definecolor{currentfill}{rgb}{0.121569,0.466667,0.705882}%
\pgfsetfillcolor{currentfill}%
\pgfsetlinewidth{1.003750pt}%
\definecolor{currentstroke}{rgb}{0.121569,0.466667,0.705882}%
\pgfsetstrokecolor{currentstroke}%
\pgfsetdash{}{0pt}%
\pgfpathmoveto{\pgfqpoint{3.107285in}{2.258225in}}%
\pgfpathcurveto{\pgfqpoint{3.118335in}{2.258225in}}{\pgfqpoint{3.128934in}{2.262615in}}{\pgfqpoint{3.136748in}{2.270429in}}%
\pgfpathcurveto{\pgfqpoint{3.144562in}{2.278242in}}{\pgfqpoint{3.148952in}{2.288841in}}{\pgfqpoint{3.148952in}{2.299891in}}%
\pgfpathcurveto{\pgfqpoint{3.148952in}{2.310942in}}{\pgfqpoint{3.144562in}{2.321541in}}{\pgfqpoint{3.136748in}{2.329354in}}%
\pgfpathcurveto{\pgfqpoint{3.128934in}{2.337168in}}{\pgfqpoint{3.118335in}{2.341558in}}{\pgfqpoint{3.107285in}{2.341558in}}%
\pgfpathcurveto{\pgfqpoint{3.096235in}{2.341558in}}{\pgfqpoint{3.085636in}{2.337168in}}{\pgfqpoint{3.077822in}{2.329354in}}%
\pgfpathcurveto{\pgfqpoint{3.070009in}{2.321541in}}{\pgfqpoint{3.065619in}{2.310942in}}{\pgfqpoint{3.065619in}{2.299891in}}%
\pgfpathcurveto{\pgfqpoint{3.065619in}{2.288841in}}{\pgfqpoint{3.070009in}{2.278242in}}{\pgfqpoint{3.077822in}{2.270429in}}%
\pgfpathcurveto{\pgfqpoint{3.085636in}{2.262615in}}{\pgfqpoint{3.096235in}{2.258225in}}{\pgfqpoint{3.107285in}{2.258225in}}%
\pgfpathclose%
\pgfusepath{stroke,fill}%
\end{pgfscope}%
\begin{pgfscope}%
\pgfpathrectangle{\pgfqpoint{0.600000in}{0.600000in}}{\pgfqpoint{3.900000in}{3.900000in}}%
\pgfusepath{clip}%
\pgfsetbuttcap%
\pgfsetroundjoin%
\definecolor{currentfill}{rgb}{0.121569,0.466667,0.705882}%
\pgfsetfillcolor{currentfill}%
\pgfsetlinewidth{1.003750pt}%
\definecolor{currentstroke}{rgb}{0.121569,0.466667,0.705882}%
\pgfsetstrokecolor{currentstroke}%
\pgfsetdash{}{0pt}%
\pgfpathmoveto{\pgfqpoint{3.459770in}{2.668339in}}%
\pgfpathcurveto{\pgfqpoint{3.470820in}{2.668339in}}{\pgfqpoint{3.481419in}{2.672729in}}{\pgfqpoint{3.489233in}{2.680543in}}%
\pgfpathcurveto{\pgfqpoint{3.497047in}{2.688356in}}{\pgfqpoint{3.501437in}{2.698955in}}{\pgfqpoint{3.501437in}{2.710006in}}%
\pgfpathcurveto{\pgfqpoint{3.501437in}{2.721056in}}{\pgfqpoint{3.497047in}{2.731655in}}{\pgfqpoint{3.489233in}{2.739468in}}%
\pgfpathcurveto{\pgfqpoint{3.481419in}{2.747282in}}{\pgfqpoint{3.470820in}{2.751672in}}{\pgfqpoint{3.459770in}{2.751672in}}%
\pgfpathcurveto{\pgfqpoint{3.448720in}{2.751672in}}{\pgfqpoint{3.438121in}{2.747282in}}{\pgfqpoint{3.430307in}{2.739468in}}%
\pgfpathcurveto{\pgfqpoint{3.422494in}{2.731655in}}{\pgfqpoint{3.418103in}{2.721056in}}{\pgfqpoint{3.418103in}{2.710006in}}%
\pgfpathcurveto{\pgfqpoint{3.418103in}{2.698955in}}{\pgfqpoint{3.422494in}{2.688356in}}{\pgfqpoint{3.430307in}{2.680543in}}%
\pgfpathcurveto{\pgfqpoint{3.438121in}{2.672729in}}{\pgfqpoint{3.448720in}{2.668339in}}{\pgfqpoint{3.459770in}{2.668339in}}%
\pgfpathclose%
\pgfusepath{stroke,fill}%
\end{pgfscope}%
\begin{pgfscope}%
\pgfpathrectangle{\pgfqpoint{0.600000in}{0.600000in}}{\pgfqpoint{3.900000in}{3.900000in}}%
\pgfusepath{clip}%
\pgfsetbuttcap%
\pgfsetroundjoin%
\definecolor{currentfill}{rgb}{0.121569,0.466667,0.705882}%
\pgfsetfillcolor{currentfill}%
\pgfsetlinewidth{1.003750pt}%
\definecolor{currentstroke}{rgb}{0.121569,0.466667,0.705882}%
\pgfsetstrokecolor{currentstroke}%
\pgfsetdash{}{0pt}%
\pgfpathmoveto{\pgfqpoint{3.131705in}{2.779848in}}%
\pgfpathcurveto{\pgfqpoint{3.142755in}{2.779848in}}{\pgfqpoint{3.153354in}{2.784238in}}{\pgfqpoint{3.161167in}{2.792052in}}%
\pgfpathcurveto{\pgfqpoint{3.168981in}{2.799865in}}{\pgfqpoint{3.173371in}{2.810464in}}{\pgfqpoint{3.173371in}{2.821514in}}%
\pgfpathcurveto{\pgfqpoint{3.173371in}{2.832565in}}{\pgfqpoint{3.168981in}{2.843164in}}{\pgfqpoint{3.161167in}{2.850977in}}%
\pgfpathcurveto{\pgfqpoint{3.153354in}{2.858791in}}{\pgfqpoint{3.142755in}{2.863181in}}{\pgfqpoint{3.131705in}{2.863181in}}%
\pgfpathcurveto{\pgfqpoint{3.120655in}{2.863181in}}{\pgfqpoint{3.110056in}{2.858791in}}{\pgfqpoint{3.102242in}{2.850977in}}%
\pgfpathcurveto{\pgfqpoint{3.094428in}{2.843164in}}{\pgfqpoint{3.090038in}{2.832565in}}{\pgfqpoint{3.090038in}{2.821514in}}%
\pgfpathcurveto{\pgfqpoint{3.090038in}{2.810464in}}{\pgfqpoint{3.094428in}{2.799865in}}{\pgfqpoint{3.102242in}{2.792052in}}%
\pgfpathcurveto{\pgfqpoint{3.110056in}{2.784238in}}{\pgfqpoint{3.120655in}{2.779848in}}{\pgfqpoint{3.131705in}{2.779848in}}%
\pgfpathclose%
\pgfusepath{stroke,fill}%
\end{pgfscope}%
\begin{pgfscope}%
\pgfpathrectangle{\pgfqpoint{0.600000in}{0.600000in}}{\pgfqpoint{3.900000in}{3.900000in}}%
\pgfusepath{clip}%
\pgfsetbuttcap%
\pgfsetroundjoin%
\definecolor{currentfill}{rgb}{0.121569,0.466667,0.705882}%
\pgfsetfillcolor{currentfill}%
\pgfsetlinewidth{1.003750pt}%
\definecolor{currentstroke}{rgb}{0.121569,0.466667,0.705882}%
\pgfsetstrokecolor{currentstroke}%
\pgfsetdash{}{0pt}%
\pgfpathmoveto{\pgfqpoint{2.856189in}{2.694091in}}%
\pgfpathcurveto{\pgfqpoint{2.867239in}{2.694091in}}{\pgfqpoint{2.877838in}{2.698481in}}{\pgfqpoint{2.885652in}{2.706295in}}%
\pgfpathcurveto{\pgfqpoint{2.893466in}{2.714108in}}{\pgfqpoint{2.897856in}{2.724707in}}{\pgfqpoint{2.897856in}{2.735757in}}%
\pgfpathcurveto{\pgfqpoint{2.897856in}{2.746807in}}{\pgfqpoint{2.893466in}{2.757406in}}{\pgfqpoint{2.885652in}{2.765220in}}%
\pgfpathcurveto{\pgfqpoint{2.877838in}{2.773034in}}{\pgfqpoint{2.867239in}{2.777424in}}{\pgfqpoint{2.856189in}{2.777424in}}%
\pgfpathcurveto{\pgfqpoint{2.845139in}{2.777424in}}{\pgfqpoint{2.834540in}{2.773034in}}{\pgfqpoint{2.826726in}{2.765220in}}%
\pgfpathcurveto{\pgfqpoint{2.818913in}{2.757406in}}{\pgfqpoint{2.814523in}{2.746807in}}{\pgfqpoint{2.814523in}{2.735757in}}%
\pgfpathcurveto{\pgfqpoint{2.814523in}{2.724707in}}{\pgfqpoint{2.818913in}{2.714108in}}{\pgfqpoint{2.826726in}{2.706295in}}%
\pgfpathcurveto{\pgfqpoint{2.834540in}{2.698481in}}{\pgfqpoint{2.845139in}{2.694091in}}{\pgfqpoint{2.856189in}{2.694091in}}%
\pgfpathclose%
\pgfusepath{stroke,fill}%
\end{pgfscope}%
\begin{pgfscope}%
\pgfpathrectangle{\pgfqpoint{0.600000in}{0.600000in}}{\pgfqpoint{3.900000in}{3.900000in}}%
\pgfusepath{clip}%
\pgfsetbuttcap%
\pgfsetroundjoin%
\definecolor{currentfill}{rgb}{0.121569,0.466667,0.705882}%
\pgfsetfillcolor{currentfill}%
\pgfsetlinewidth{1.003750pt}%
\definecolor{currentstroke}{rgb}{0.121569,0.466667,0.705882}%
\pgfsetstrokecolor{currentstroke}%
\pgfsetdash{}{0pt}%
\pgfpathmoveto{\pgfqpoint{2.534430in}{1.957687in}}%
\pgfpathcurveto{\pgfqpoint{2.545480in}{1.957687in}}{\pgfqpoint{2.556079in}{1.962077in}}{\pgfqpoint{2.563893in}{1.969890in}}%
\pgfpathcurveto{\pgfqpoint{2.571706in}{1.977704in}}{\pgfqpoint{2.576097in}{1.988303in}}{\pgfqpoint{2.576097in}{1.999353in}}%
\pgfpathcurveto{\pgfqpoint{2.576097in}{2.010403in}}{\pgfqpoint{2.571706in}{2.021002in}}{\pgfqpoint{2.563893in}{2.028816in}}%
\pgfpathcurveto{\pgfqpoint{2.556079in}{2.036630in}}{\pgfqpoint{2.545480in}{2.041020in}}{\pgfqpoint{2.534430in}{2.041020in}}%
\pgfpathcurveto{\pgfqpoint{2.523380in}{2.041020in}}{\pgfqpoint{2.512781in}{2.036630in}}{\pgfqpoint{2.504967in}{2.028816in}}%
\pgfpathcurveto{\pgfqpoint{2.497154in}{2.021002in}}{\pgfqpoint{2.492763in}{2.010403in}}{\pgfqpoint{2.492763in}{1.999353in}}%
\pgfpathcurveto{\pgfqpoint{2.492763in}{1.988303in}}{\pgfqpoint{2.497154in}{1.977704in}}{\pgfqpoint{2.504967in}{1.969890in}}%
\pgfpathcurveto{\pgfqpoint{2.512781in}{1.962077in}}{\pgfqpoint{2.523380in}{1.957687in}}{\pgfqpoint{2.534430in}{1.957687in}}%
\pgfpathclose%
\pgfusepath{stroke,fill}%
\end{pgfscope}%
\begin{pgfscope}%
\pgfpathrectangle{\pgfqpoint{0.600000in}{0.600000in}}{\pgfqpoint{3.900000in}{3.900000in}}%
\pgfusepath{clip}%
\pgfsetbuttcap%
\pgfsetroundjoin%
\definecolor{currentfill}{rgb}{0.121569,0.466667,0.705882}%
\pgfsetfillcolor{currentfill}%
\pgfsetlinewidth{1.003750pt}%
\definecolor{currentstroke}{rgb}{0.121569,0.466667,0.705882}%
\pgfsetstrokecolor{currentstroke}%
\pgfsetdash{}{0pt}%
\pgfpathmoveto{\pgfqpoint{2.934215in}{2.496439in}}%
\pgfpathcurveto{\pgfqpoint{2.945265in}{2.496439in}}{\pgfqpoint{2.955864in}{2.500829in}}{\pgfqpoint{2.963678in}{2.508643in}}%
\pgfpathcurveto{\pgfqpoint{2.971492in}{2.516457in}}{\pgfqpoint{2.975882in}{2.527056in}}{\pgfqpoint{2.975882in}{2.538106in}}%
\pgfpathcurveto{\pgfqpoint{2.975882in}{2.549156in}}{\pgfqpoint{2.971492in}{2.559755in}}{\pgfqpoint{2.963678in}{2.567568in}}%
\pgfpathcurveto{\pgfqpoint{2.955864in}{2.575382in}}{\pgfqpoint{2.945265in}{2.579772in}}{\pgfqpoint{2.934215in}{2.579772in}}%
\pgfpathcurveto{\pgfqpoint{2.923165in}{2.579772in}}{\pgfqpoint{2.912566in}{2.575382in}}{\pgfqpoint{2.904752in}{2.567568in}}%
\pgfpathcurveto{\pgfqpoint{2.896939in}{2.559755in}}{\pgfqpoint{2.892548in}{2.549156in}}{\pgfqpoint{2.892548in}{2.538106in}}%
\pgfpathcurveto{\pgfqpoint{2.892548in}{2.527056in}}{\pgfqpoint{2.896939in}{2.516457in}}{\pgfqpoint{2.904752in}{2.508643in}}%
\pgfpathcurveto{\pgfqpoint{2.912566in}{2.500829in}}{\pgfqpoint{2.923165in}{2.496439in}}{\pgfqpoint{2.934215in}{2.496439in}}%
\pgfpathclose%
\pgfusepath{stroke,fill}%
\end{pgfscope}%
\begin{pgfscope}%
\pgfpathrectangle{\pgfqpoint{0.600000in}{0.600000in}}{\pgfqpoint{3.900000in}{3.900000in}}%
\pgfusepath{clip}%
\pgfsetbuttcap%
\pgfsetroundjoin%
\definecolor{currentfill}{rgb}{0.121569,0.466667,0.705882}%
\pgfsetfillcolor{currentfill}%
\pgfsetlinewidth{1.003750pt}%
\definecolor{currentstroke}{rgb}{0.121569,0.466667,0.705882}%
\pgfsetstrokecolor{currentstroke}%
\pgfsetdash{}{0pt}%
\pgfpathmoveto{\pgfqpoint{2.039334in}{3.226240in}}%
\pgfpathcurveto{\pgfqpoint{2.050384in}{3.226240in}}{\pgfqpoint{2.060983in}{3.230630in}}{\pgfqpoint{2.068796in}{3.238444in}}%
\pgfpathcurveto{\pgfqpoint{2.076610in}{3.246257in}}{\pgfqpoint{2.081000in}{3.256856in}}{\pgfqpoint{2.081000in}{3.267907in}}%
\pgfpathcurveto{\pgfqpoint{2.081000in}{3.278957in}}{\pgfqpoint{2.076610in}{3.289556in}}{\pgfqpoint{2.068796in}{3.297369in}}%
\pgfpathcurveto{\pgfqpoint{2.060983in}{3.305183in}}{\pgfqpoint{2.050384in}{3.309573in}}{\pgfqpoint{2.039334in}{3.309573in}}%
\pgfpathcurveto{\pgfqpoint{2.028283in}{3.309573in}}{\pgfqpoint{2.017684in}{3.305183in}}{\pgfqpoint{2.009871in}{3.297369in}}%
\pgfpathcurveto{\pgfqpoint{2.002057in}{3.289556in}}{\pgfqpoint{1.997667in}{3.278957in}}{\pgfqpoint{1.997667in}{3.267907in}}%
\pgfpathcurveto{\pgfqpoint{1.997667in}{3.256856in}}{\pgfqpoint{2.002057in}{3.246257in}}{\pgfqpoint{2.009871in}{3.238444in}}%
\pgfpathcurveto{\pgfqpoint{2.017684in}{3.230630in}}{\pgfqpoint{2.028283in}{3.226240in}}{\pgfqpoint{2.039334in}{3.226240in}}%
\pgfpathclose%
\pgfusepath{stroke,fill}%
\end{pgfscope}%
\begin{pgfscope}%
\pgfpathrectangle{\pgfqpoint{0.600000in}{0.600000in}}{\pgfqpoint{3.900000in}{3.900000in}}%
\pgfusepath{clip}%
\pgfsetbuttcap%
\pgfsetroundjoin%
\definecolor{currentfill}{rgb}{0.121569,0.466667,0.705882}%
\pgfsetfillcolor{currentfill}%
\pgfsetlinewidth{1.003750pt}%
\definecolor{currentstroke}{rgb}{0.121569,0.466667,0.705882}%
\pgfsetstrokecolor{currentstroke}%
\pgfsetdash{}{0pt}%
\pgfpathmoveto{\pgfqpoint{1.959538in}{1.931712in}}%
\pgfpathcurveto{\pgfqpoint{1.970588in}{1.931712in}}{\pgfqpoint{1.981187in}{1.936102in}}{\pgfqpoint{1.989001in}{1.943916in}}%
\pgfpathcurveto{\pgfqpoint{1.996815in}{1.951729in}}{\pgfqpoint{2.001205in}{1.962328in}}{\pgfqpoint{2.001205in}{1.973378in}}%
\pgfpathcurveto{\pgfqpoint{2.001205in}{1.984429in}}{\pgfqpoint{1.996815in}{1.995028in}}{\pgfqpoint{1.989001in}{2.002841in}}%
\pgfpathcurveto{\pgfqpoint{1.981187in}{2.010655in}}{\pgfqpoint{1.970588in}{2.015045in}}{\pgfqpoint{1.959538in}{2.015045in}}%
\pgfpathcurveto{\pgfqpoint{1.948488in}{2.015045in}}{\pgfqpoint{1.937889in}{2.010655in}}{\pgfqpoint{1.930075in}{2.002841in}}%
\pgfpathcurveto{\pgfqpoint{1.922262in}{1.995028in}}{\pgfqpoint{1.917872in}{1.984429in}}{\pgfqpoint{1.917872in}{1.973378in}}%
\pgfpathcurveto{\pgfqpoint{1.917872in}{1.962328in}}{\pgfqpoint{1.922262in}{1.951729in}}{\pgfqpoint{1.930075in}{1.943916in}}%
\pgfpathcurveto{\pgfqpoint{1.937889in}{1.936102in}}{\pgfqpoint{1.948488in}{1.931712in}}{\pgfqpoint{1.959538in}{1.931712in}}%
\pgfpathclose%
\pgfusepath{stroke,fill}%
\end{pgfscope}%
\begin{pgfscope}%
\pgfpathrectangle{\pgfqpoint{0.600000in}{0.600000in}}{\pgfqpoint{3.900000in}{3.900000in}}%
\pgfusepath{clip}%
\pgfsetbuttcap%
\pgfsetroundjoin%
\definecolor{currentfill}{rgb}{0.121569,0.466667,0.705882}%
\pgfsetfillcolor{currentfill}%
\pgfsetlinewidth{1.003750pt}%
\definecolor{currentstroke}{rgb}{0.121569,0.466667,0.705882}%
\pgfsetstrokecolor{currentstroke}%
\pgfsetdash{}{0pt}%
\pgfpathmoveto{\pgfqpoint{2.779023in}{2.324392in}}%
\pgfpathcurveto{\pgfqpoint{2.790073in}{2.324392in}}{\pgfqpoint{2.800672in}{2.328783in}}{\pgfqpoint{2.808486in}{2.336596in}}%
\pgfpathcurveto{\pgfqpoint{2.816299in}{2.344410in}}{\pgfqpoint{2.820690in}{2.355009in}}{\pgfqpoint{2.820690in}{2.366059in}}%
\pgfpathcurveto{\pgfqpoint{2.820690in}{2.377109in}}{\pgfqpoint{2.816299in}{2.387708in}}{\pgfqpoint{2.808486in}{2.395522in}}%
\pgfpathcurveto{\pgfqpoint{2.800672in}{2.403335in}}{\pgfqpoint{2.790073in}{2.407726in}}{\pgfqpoint{2.779023in}{2.407726in}}%
\pgfpathcurveto{\pgfqpoint{2.767973in}{2.407726in}}{\pgfqpoint{2.757374in}{2.403335in}}{\pgfqpoint{2.749560in}{2.395522in}}%
\pgfpathcurveto{\pgfqpoint{2.741746in}{2.387708in}}{\pgfqpoint{2.737356in}{2.377109in}}{\pgfqpoint{2.737356in}{2.366059in}}%
\pgfpathcurveto{\pgfqpoint{2.737356in}{2.355009in}}{\pgfqpoint{2.741746in}{2.344410in}}{\pgfqpoint{2.749560in}{2.336596in}}%
\pgfpathcurveto{\pgfqpoint{2.757374in}{2.328783in}}{\pgfqpoint{2.767973in}{2.324392in}}{\pgfqpoint{2.779023in}{2.324392in}}%
\pgfpathclose%
\pgfusepath{stroke,fill}%
\end{pgfscope}%
\begin{pgfscope}%
\pgfpathrectangle{\pgfqpoint{0.600000in}{0.600000in}}{\pgfqpoint{3.900000in}{3.900000in}}%
\pgfusepath{clip}%
\pgfsetbuttcap%
\pgfsetroundjoin%
\definecolor{currentfill}{rgb}{0.121569,0.466667,0.705882}%
\pgfsetfillcolor{currentfill}%
\pgfsetlinewidth{1.003750pt}%
\definecolor{currentstroke}{rgb}{0.121569,0.466667,0.705882}%
\pgfsetstrokecolor{currentstroke}%
\pgfsetdash{}{0pt}%
\pgfpathmoveto{\pgfqpoint{2.913364in}{1.995451in}}%
\pgfpathcurveto{\pgfqpoint{2.924414in}{1.995451in}}{\pgfqpoint{2.935013in}{1.999841in}}{\pgfqpoint{2.942827in}{2.007655in}}%
\pgfpathcurveto{\pgfqpoint{2.950641in}{2.015468in}}{\pgfqpoint{2.955031in}{2.026067in}}{\pgfqpoint{2.955031in}{2.037118in}}%
\pgfpathcurveto{\pgfqpoint{2.955031in}{2.048168in}}{\pgfqpoint{2.950641in}{2.058767in}}{\pgfqpoint{2.942827in}{2.066580in}}%
\pgfpathcurveto{\pgfqpoint{2.935013in}{2.074394in}}{\pgfqpoint{2.924414in}{2.078784in}}{\pgfqpoint{2.913364in}{2.078784in}}%
\pgfpathcurveto{\pgfqpoint{2.902314in}{2.078784in}}{\pgfqpoint{2.891715in}{2.074394in}}{\pgfqpoint{2.883901in}{2.066580in}}%
\pgfpathcurveto{\pgfqpoint{2.876088in}{2.058767in}}{\pgfqpoint{2.871697in}{2.048168in}}{\pgfqpoint{2.871697in}{2.037118in}}%
\pgfpathcurveto{\pgfqpoint{2.871697in}{2.026067in}}{\pgfqpoint{2.876088in}{2.015468in}}{\pgfqpoint{2.883901in}{2.007655in}}%
\pgfpathcurveto{\pgfqpoint{2.891715in}{1.999841in}}{\pgfqpoint{2.902314in}{1.995451in}}{\pgfqpoint{2.913364in}{1.995451in}}%
\pgfpathclose%
\pgfusepath{stroke,fill}%
\end{pgfscope}%
\begin{pgfscope}%
\pgfpathrectangle{\pgfqpoint{0.600000in}{0.600000in}}{\pgfqpoint{3.900000in}{3.900000in}}%
\pgfusepath{clip}%
\pgfsetbuttcap%
\pgfsetroundjoin%
\definecolor{currentfill}{rgb}{0.121569,0.466667,0.705882}%
\pgfsetfillcolor{currentfill}%
\pgfsetlinewidth{1.003750pt}%
\definecolor{currentstroke}{rgb}{0.121569,0.466667,0.705882}%
\pgfsetstrokecolor{currentstroke}%
\pgfsetdash{}{0pt}%
\pgfpathmoveto{\pgfqpoint{1.974556in}{3.084282in}}%
\pgfpathcurveto{\pgfqpoint{1.985606in}{3.084282in}}{\pgfqpoint{1.996205in}{3.088673in}}{\pgfqpoint{2.004019in}{3.096486in}}%
\pgfpathcurveto{\pgfqpoint{2.011833in}{3.104300in}}{\pgfqpoint{2.016223in}{3.114899in}}{\pgfqpoint{2.016223in}{3.125949in}}%
\pgfpathcurveto{\pgfqpoint{2.016223in}{3.136999in}}{\pgfqpoint{2.011833in}{3.147598in}}{\pgfqpoint{2.004019in}{3.155412in}}%
\pgfpathcurveto{\pgfqpoint{1.996205in}{3.163225in}}{\pgfqpoint{1.985606in}{3.167616in}}{\pgfqpoint{1.974556in}{3.167616in}}%
\pgfpathcurveto{\pgfqpoint{1.963506in}{3.167616in}}{\pgfqpoint{1.952907in}{3.163225in}}{\pgfqpoint{1.945094in}{3.155412in}}%
\pgfpathcurveto{\pgfqpoint{1.937280in}{3.147598in}}{\pgfqpoint{1.932890in}{3.136999in}}{\pgfqpoint{1.932890in}{3.125949in}}%
\pgfpathcurveto{\pgfqpoint{1.932890in}{3.114899in}}{\pgfqpoint{1.937280in}{3.104300in}}{\pgfqpoint{1.945094in}{3.096486in}}%
\pgfpathcurveto{\pgfqpoint{1.952907in}{3.088673in}}{\pgfqpoint{1.963506in}{3.084282in}}{\pgfqpoint{1.974556in}{3.084282in}}%
\pgfpathclose%
\pgfusepath{stroke,fill}%
\end{pgfscope}%
\begin{pgfscope}%
\pgfpathrectangle{\pgfqpoint{0.600000in}{0.600000in}}{\pgfqpoint{3.900000in}{3.900000in}}%
\pgfusepath{clip}%
\pgfsetbuttcap%
\pgfsetroundjoin%
\definecolor{currentfill}{rgb}{0.121569,0.466667,0.705882}%
\pgfsetfillcolor{currentfill}%
\pgfsetlinewidth{1.003750pt}%
\definecolor{currentstroke}{rgb}{0.121569,0.466667,0.705882}%
\pgfsetstrokecolor{currentstroke}%
\pgfsetdash{}{0pt}%
\pgfpathmoveto{\pgfqpoint{2.224267in}{2.341487in}}%
\pgfpathcurveto{\pgfqpoint{2.235317in}{2.341487in}}{\pgfqpoint{2.245916in}{2.345877in}}{\pgfqpoint{2.253729in}{2.353691in}}%
\pgfpathcurveto{\pgfqpoint{2.261543in}{2.361505in}}{\pgfqpoint{2.265933in}{2.372104in}}{\pgfqpoint{2.265933in}{2.383154in}}%
\pgfpathcurveto{\pgfqpoint{2.265933in}{2.394204in}}{\pgfqpoint{2.261543in}{2.404803in}}{\pgfqpoint{2.253729in}{2.412616in}}%
\pgfpathcurveto{\pgfqpoint{2.245916in}{2.420430in}}{\pgfqpoint{2.235317in}{2.424820in}}{\pgfqpoint{2.224267in}{2.424820in}}%
\pgfpathcurveto{\pgfqpoint{2.213217in}{2.424820in}}{\pgfqpoint{2.202618in}{2.420430in}}{\pgfqpoint{2.194804in}{2.412616in}}%
\pgfpathcurveto{\pgfqpoint{2.186990in}{2.404803in}}{\pgfqpoint{2.182600in}{2.394204in}}{\pgfqpoint{2.182600in}{2.383154in}}%
\pgfpathcurveto{\pgfqpoint{2.182600in}{2.372104in}}{\pgfqpoint{2.186990in}{2.361505in}}{\pgfqpoint{2.194804in}{2.353691in}}%
\pgfpathcurveto{\pgfqpoint{2.202618in}{2.345877in}}{\pgfqpoint{2.213217in}{2.341487in}}{\pgfqpoint{2.224267in}{2.341487in}}%
\pgfpathclose%
\pgfusepath{stroke,fill}%
\end{pgfscope}%
\begin{pgfscope}%
\pgfpathrectangle{\pgfqpoint{0.600000in}{0.600000in}}{\pgfqpoint{3.900000in}{3.900000in}}%
\pgfusepath{clip}%
\pgfsetbuttcap%
\pgfsetroundjoin%
\definecolor{currentfill}{rgb}{0.121569,0.466667,0.705882}%
\pgfsetfillcolor{currentfill}%
\pgfsetlinewidth{1.003750pt}%
\definecolor{currentstroke}{rgb}{0.121569,0.466667,0.705882}%
\pgfsetstrokecolor{currentstroke}%
\pgfsetdash{}{0pt}%
\pgfpathmoveto{\pgfqpoint{1.947231in}{2.193610in}}%
\pgfpathcurveto{\pgfqpoint{1.958281in}{2.193610in}}{\pgfqpoint{1.968880in}{2.198001in}}{\pgfqpoint{1.976694in}{2.205814in}}%
\pgfpathcurveto{\pgfqpoint{1.984508in}{2.213628in}}{\pgfqpoint{1.988898in}{2.224227in}}{\pgfqpoint{1.988898in}{2.235277in}}%
\pgfpathcurveto{\pgfqpoint{1.988898in}{2.246327in}}{\pgfqpoint{1.984508in}{2.256926in}}{\pgfqpoint{1.976694in}{2.264740in}}%
\pgfpathcurveto{\pgfqpoint{1.968880in}{2.272553in}}{\pgfqpoint{1.958281in}{2.276944in}}{\pgfqpoint{1.947231in}{2.276944in}}%
\pgfpathcurveto{\pgfqpoint{1.936181in}{2.276944in}}{\pgfqpoint{1.925582in}{2.272553in}}{\pgfqpoint{1.917768in}{2.264740in}}%
\pgfpathcurveto{\pgfqpoint{1.909955in}{2.256926in}}{\pgfqpoint{1.905565in}{2.246327in}}{\pgfqpoint{1.905565in}{2.235277in}}%
\pgfpathcurveto{\pgfqpoint{1.905565in}{2.224227in}}{\pgfqpoint{1.909955in}{2.213628in}}{\pgfqpoint{1.917768in}{2.205814in}}%
\pgfpathcurveto{\pgfqpoint{1.925582in}{2.198001in}}{\pgfqpoint{1.936181in}{2.193610in}}{\pgfqpoint{1.947231in}{2.193610in}}%
\pgfpathclose%
\pgfusepath{stroke,fill}%
\end{pgfscope}%
\begin{pgfscope}%
\pgfpathrectangle{\pgfqpoint{0.600000in}{0.600000in}}{\pgfqpoint{3.900000in}{3.900000in}}%
\pgfusepath{clip}%
\pgfsetbuttcap%
\pgfsetroundjoin%
\definecolor{currentfill}{rgb}{0.121569,0.466667,0.705882}%
\pgfsetfillcolor{currentfill}%
\pgfsetlinewidth{1.003750pt}%
\definecolor{currentstroke}{rgb}{0.121569,0.466667,0.705882}%
\pgfsetstrokecolor{currentstroke}%
\pgfsetdash{}{0pt}%
\pgfpathmoveto{\pgfqpoint{3.018384in}{4.053278in}}%
\pgfpathcurveto{\pgfqpoint{3.029434in}{4.053278in}}{\pgfqpoint{3.040033in}{4.057668in}}{\pgfqpoint{3.047847in}{4.065482in}}%
\pgfpathcurveto{\pgfqpoint{3.055661in}{4.073296in}}{\pgfqpoint{3.060051in}{4.083895in}}{\pgfqpoint{3.060051in}{4.094945in}}%
\pgfpathcurveto{\pgfqpoint{3.060051in}{4.105995in}}{\pgfqpoint{3.055661in}{4.116594in}}{\pgfqpoint{3.047847in}{4.124408in}}%
\pgfpathcurveto{\pgfqpoint{3.040033in}{4.132221in}}{\pgfqpoint{3.029434in}{4.136611in}}{\pgfqpoint{3.018384in}{4.136611in}}%
\pgfpathcurveto{\pgfqpoint{3.007334in}{4.136611in}}{\pgfqpoint{2.996735in}{4.132221in}}{\pgfqpoint{2.988922in}{4.124408in}}%
\pgfpathcurveto{\pgfqpoint{2.981108in}{4.116594in}}{\pgfqpoint{2.976718in}{4.105995in}}{\pgfqpoint{2.976718in}{4.094945in}}%
\pgfpathcurveto{\pgfqpoint{2.976718in}{4.083895in}}{\pgfqpoint{2.981108in}{4.073296in}}{\pgfqpoint{2.988922in}{4.065482in}}%
\pgfpathcurveto{\pgfqpoint{2.996735in}{4.057668in}}{\pgfqpoint{3.007334in}{4.053278in}}{\pgfqpoint{3.018384in}{4.053278in}}%
\pgfpathclose%
\pgfusepath{stroke,fill}%
\end{pgfscope}%
\begin{pgfscope}%
\pgfpathrectangle{\pgfqpoint{0.600000in}{0.600000in}}{\pgfqpoint{3.900000in}{3.900000in}}%
\pgfusepath{clip}%
\pgfsetbuttcap%
\pgfsetroundjoin%
\definecolor{currentfill}{rgb}{0.121569,0.466667,0.705882}%
\pgfsetfillcolor{currentfill}%
\pgfsetlinewidth{1.003750pt}%
\definecolor{currentstroke}{rgb}{0.121569,0.466667,0.705882}%
\pgfsetstrokecolor{currentstroke}%
\pgfsetdash{}{0pt}%
\pgfpathmoveto{\pgfqpoint{2.297137in}{3.162262in}}%
\pgfpathcurveto{\pgfqpoint{2.308187in}{3.162262in}}{\pgfqpoint{2.318786in}{3.166653in}}{\pgfqpoint{2.326600in}{3.174466in}}%
\pgfpathcurveto{\pgfqpoint{2.334413in}{3.182280in}}{\pgfqpoint{2.338803in}{3.192879in}}{\pgfqpoint{2.338803in}{3.203929in}}%
\pgfpathcurveto{\pgfqpoint{2.338803in}{3.214979in}}{\pgfqpoint{2.334413in}{3.225578in}}{\pgfqpoint{2.326600in}{3.233392in}}%
\pgfpathcurveto{\pgfqpoint{2.318786in}{3.241206in}}{\pgfqpoint{2.308187in}{3.245596in}}{\pgfqpoint{2.297137in}{3.245596in}}%
\pgfpathcurveto{\pgfqpoint{2.286087in}{3.245596in}}{\pgfqpoint{2.275488in}{3.241206in}}{\pgfqpoint{2.267674in}{3.233392in}}%
\pgfpathcurveto{\pgfqpoint{2.259860in}{3.225578in}}{\pgfqpoint{2.255470in}{3.214979in}}{\pgfqpoint{2.255470in}{3.203929in}}%
\pgfpathcurveto{\pgfqpoint{2.255470in}{3.192879in}}{\pgfqpoint{2.259860in}{3.182280in}}{\pgfqpoint{2.267674in}{3.174466in}}%
\pgfpathcurveto{\pgfqpoint{2.275488in}{3.166653in}}{\pgfqpoint{2.286087in}{3.162262in}}{\pgfqpoint{2.297137in}{3.162262in}}%
\pgfpathclose%
\pgfusepath{stroke,fill}%
\end{pgfscope}%
\begin{pgfscope}%
\pgfpathrectangle{\pgfqpoint{0.600000in}{0.600000in}}{\pgfqpoint{3.900000in}{3.900000in}}%
\pgfusepath{clip}%
\pgfsetbuttcap%
\pgfsetroundjoin%
\definecolor{currentfill}{rgb}{0.121569,0.466667,0.705882}%
\pgfsetfillcolor{currentfill}%
\pgfsetlinewidth{1.003750pt}%
\definecolor{currentstroke}{rgb}{0.121569,0.466667,0.705882}%
\pgfsetstrokecolor{currentstroke}%
\pgfsetdash{}{0pt}%
\pgfpathmoveto{\pgfqpoint{1.945553in}{1.915258in}}%
\pgfpathcurveto{\pgfqpoint{1.956603in}{1.915258in}}{\pgfqpoint{1.967202in}{1.919648in}}{\pgfqpoint{1.975016in}{1.927462in}}%
\pgfpathcurveto{\pgfqpoint{1.982829in}{1.935275in}}{\pgfqpoint{1.987220in}{1.945874in}}{\pgfqpoint{1.987220in}{1.956924in}}%
\pgfpathcurveto{\pgfqpoint{1.987220in}{1.967975in}}{\pgfqpoint{1.982829in}{1.978574in}}{\pgfqpoint{1.975016in}{1.986387in}}%
\pgfpathcurveto{\pgfqpoint{1.967202in}{1.994201in}}{\pgfqpoint{1.956603in}{1.998591in}}{\pgfqpoint{1.945553in}{1.998591in}}%
\pgfpathcurveto{\pgfqpoint{1.934503in}{1.998591in}}{\pgfqpoint{1.923904in}{1.994201in}}{\pgfqpoint{1.916090in}{1.986387in}}%
\pgfpathcurveto{\pgfqpoint{1.908277in}{1.978574in}}{\pgfqpoint{1.903886in}{1.967975in}}{\pgfqpoint{1.903886in}{1.956924in}}%
\pgfpathcurveto{\pgfqpoint{1.903886in}{1.945874in}}{\pgfqpoint{1.908277in}{1.935275in}}{\pgfqpoint{1.916090in}{1.927462in}}%
\pgfpathcurveto{\pgfqpoint{1.923904in}{1.919648in}}{\pgfqpoint{1.934503in}{1.915258in}}{\pgfqpoint{1.945553in}{1.915258in}}%
\pgfpathclose%
\pgfusepath{stroke,fill}%
\end{pgfscope}%
\begin{pgfscope}%
\pgfpathrectangle{\pgfqpoint{0.600000in}{0.600000in}}{\pgfqpoint{3.900000in}{3.900000in}}%
\pgfusepath{clip}%
\pgfsetbuttcap%
\pgfsetroundjoin%
\definecolor{currentfill}{rgb}{0.121569,0.466667,0.705882}%
\pgfsetfillcolor{currentfill}%
\pgfsetlinewidth{1.003750pt}%
\definecolor{currentstroke}{rgb}{0.121569,0.466667,0.705882}%
\pgfsetstrokecolor{currentstroke}%
\pgfsetdash{}{0pt}%
\pgfpathmoveto{\pgfqpoint{2.246180in}{2.930006in}}%
\pgfpathcurveto{\pgfqpoint{2.257230in}{2.930006in}}{\pgfqpoint{2.267829in}{2.934396in}}{\pgfqpoint{2.275643in}{2.942210in}}%
\pgfpathcurveto{\pgfqpoint{2.283456in}{2.950024in}}{\pgfqpoint{2.287847in}{2.960623in}}{\pgfqpoint{2.287847in}{2.971673in}}%
\pgfpathcurveto{\pgfqpoint{2.287847in}{2.982723in}}{\pgfqpoint{2.283456in}{2.993322in}}{\pgfqpoint{2.275643in}{3.001136in}}%
\pgfpathcurveto{\pgfqpoint{2.267829in}{3.008949in}}{\pgfqpoint{2.257230in}{3.013340in}}{\pgfqpoint{2.246180in}{3.013340in}}%
\pgfpathcurveto{\pgfqpoint{2.235130in}{3.013340in}}{\pgfqpoint{2.224531in}{3.008949in}}{\pgfqpoint{2.216717in}{3.001136in}}%
\pgfpathcurveto{\pgfqpoint{2.208904in}{2.993322in}}{\pgfqpoint{2.204513in}{2.982723in}}{\pgfqpoint{2.204513in}{2.971673in}}%
\pgfpathcurveto{\pgfqpoint{2.204513in}{2.960623in}}{\pgfqpoint{2.208904in}{2.950024in}}{\pgfqpoint{2.216717in}{2.942210in}}%
\pgfpathcurveto{\pgfqpoint{2.224531in}{2.934396in}}{\pgfqpoint{2.235130in}{2.930006in}}{\pgfqpoint{2.246180in}{2.930006in}}%
\pgfpathclose%
\pgfusepath{stroke,fill}%
\end{pgfscope}%
\begin{pgfscope}%
\pgfpathrectangle{\pgfqpoint{0.600000in}{0.600000in}}{\pgfqpoint{3.900000in}{3.900000in}}%
\pgfusepath{clip}%
\pgfsetbuttcap%
\pgfsetroundjoin%
\definecolor{currentfill}{rgb}{0.121569,0.466667,0.705882}%
\pgfsetfillcolor{currentfill}%
\pgfsetlinewidth{1.003750pt}%
\definecolor{currentstroke}{rgb}{0.121569,0.466667,0.705882}%
\pgfsetstrokecolor{currentstroke}%
\pgfsetdash{}{0pt}%
\pgfpathmoveto{\pgfqpoint{1.300916in}{2.199184in}}%
\pgfpathcurveto{\pgfqpoint{1.311966in}{2.199184in}}{\pgfqpoint{1.322565in}{2.203575in}}{\pgfqpoint{1.330379in}{2.211388in}}%
\pgfpathcurveto{\pgfqpoint{1.338192in}{2.219202in}}{\pgfqpoint{1.342583in}{2.229801in}}{\pgfqpoint{1.342583in}{2.240851in}}%
\pgfpathcurveto{\pgfqpoint{1.342583in}{2.251901in}}{\pgfqpoint{1.338192in}{2.262500in}}{\pgfqpoint{1.330379in}{2.270314in}}%
\pgfpathcurveto{\pgfqpoint{1.322565in}{2.278127in}}{\pgfqpoint{1.311966in}{2.282518in}}{\pgfqpoint{1.300916in}{2.282518in}}%
\pgfpathcurveto{\pgfqpoint{1.289866in}{2.282518in}}{\pgfqpoint{1.279267in}{2.278127in}}{\pgfqpoint{1.271453in}{2.270314in}}%
\pgfpathcurveto{\pgfqpoint{1.263640in}{2.262500in}}{\pgfqpoint{1.259249in}{2.251901in}}{\pgfqpoint{1.259249in}{2.240851in}}%
\pgfpathcurveto{\pgfqpoint{1.259249in}{2.229801in}}{\pgfqpoint{1.263640in}{2.219202in}}{\pgfqpoint{1.271453in}{2.211388in}}%
\pgfpathcurveto{\pgfqpoint{1.279267in}{2.203575in}}{\pgfqpoint{1.289866in}{2.199184in}}{\pgfqpoint{1.300916in}{2.199184in}}%
\pgfpathclose%
\pgfusepath{stroke,fill}%
\end{pgfscope}%
\begin{pgfscope}%
\pgfpathrectangle{\pgfqpoint{0.600000in}{0.600000in}}{\pgfqpoint{3.900000in}{3.900000in}}%
\pgfusepath{clip}%
\pgfsetbuttcap%
\pgfsetroundjoin%
\definecolor{currentfill}{rgb}{0.121569,0.466667,0.705882}%
\pgfsetfillcolor{currentfill}%
\pgfsetlinewidth{1.003750pt}%
\definecolor{currentstroke}{rgb}{0.121569,0.466667,0.705882}%
\pgfsetstrokecolor{currentstroke}%
\pgfsetdash{}{0pt}%
\pgfpathmoveto{\pgfqpoint{2.279809in}{3.573200in}}%
\pgfpathcurveto{\pgfqpoint{2.290859in}{3.573200in}}{\pgfqpoint{2.301458in}{3.577590in}}{\pgfqpoint{2.309272in}{3.585404in}}%
\pgfpathcurveto{\pgfqpoint{2.317085in}{3.593218in}}{\pgfqpoint{2.321476in}{3.603817in}}{\pgfqpoint{2.321476in}{3.614867in}}%
\pgfpathcurveto{\pgfqpoint{2.321476in}{3.625917in}}{\pgfqpoint{2.317085in}{3.636516in}}{\pgfqpoint{2.309272in}{3.644330in}}%
\pgfpathcurveto{\pgfqpoint{2.301458in}{3.652143in}}{\pgfqpoint{2.290859in}{3.656533in}}{\pgfqpoint{2.279809in}{3.656533in}}%
\pgfpathcurveto{\pgfqpoint{2.268759in}{3.656533in}}{\pgfqpoint{2.258160in}{3.652143in}}{\pgfqpoint{2.250346in}{3.644330in}}%
\pgfpathcurveto{\pgfqpoint{2.242533in}{3.636516in}}{\pgfqpoint{2.238142in}{3.625917in}}{\pgfqpoint{2.238142in}{3.614867in}}%
\pgfpathcurveto{\pgfqpoint{2.238142in}{3.603817in}}{\pgfqpoint{2.242533in}{3.593218in}}{\pgfqpoint{2.250346in}{3.585404in}}%
\pgfpathcurveto{\pgfqpoint{2.258160in}{3.577590in}}{\pgfqpoint{2.268759in}{3.573200in}}{\pgfqpoint{2.279809in}{3.573200in}}%
\pgfpathclose%
\pgfusepath{stroke,fill}%
\end{pgfscope}%
\begin{pgfscope}%
\pgfpathrectangle{\pgfqpoint{0.600000in}{0.600000in}}{\pgfqpoint{3.900000in}{3.900000in}}%
\pgfusepath{clip}%
\pgfsetbuttcap%
\pgfsetroundjoin%
\definecolor{currentfill}{rgb}{0.121569,0.466667,0.705882}%
\pgfsetfillcolor{currentfill}%
\pgfsetlinewidth{1.003750pt}%
\definecolor{currentstroke}{rgb}{0.121569,0.466667,0.705882}%
\pgfsetstrokecolor{currentstroke}%
\pgfsetdash{}{0pt}%
\pgfpathmoveto{\pgfqpoint{3.404343in}{2.476623in}}%
\pgfpathcurveto{\pgfqpoint{3.415394in}{2.476623in}}{\pgfqpoint{3.425993in}{2.481013in}}{\pgfqpoint{3.433806in}{2.488826in}}%
\pgfpathcurveto{\pgfqpoint{3.441620in}{2.496640in}}{\pgfqpoint{3.446010in}{2.507239in}}{\pgfqpoint{3.446010in}{2.518289in}}%
\pgfpathcurveto{\pgfqpoint{3.446010in}{2.529339in}}{\pgfqpoint{3.441620in}{2.539938in}}{\pgfqpoint{3.433806in}{2.547752in}}%
\pgfpathcurveto{\pgfqpoint{3.425993in}{2.555566in}}{\pgfqpoint{3.415394in}{2.559956in}}{\pgfqpoint{3.404343in}{2.559956in}}%
\pgfpathcurveto{\pgfqpoint{3.393293in}{2.559956in}}{\pgfqpoint{3.382694in}{2.555566in}}{\pgfqpoint{3.374881in}{2.547752in}}%
\pgfpathcurveto{\pgfqpoint{3.367067in}{2.539938in}}{\pgfqpoint{3.362677in}{2.529339in}}{\pgfqpoint{3.362677in}{2.518289in}}%
\pgfpathcurveto{\pgfqpoint{3.362677in}{2.507239in}}{\pgfqpoint{3.367067in}{2.496640in}}{\pgfqpoint{3.374881in}{2.488826in}}%
\pgfpathcurveto{\pgfqpoint{3.382694in}{2.481013in}}{\pgfqpoint{3.393293in}{2.476623in}}{\pgfqpoint{3.404343in}{2.476623in}}%
\pgfpathclose%
\pgfusepath{stroke,fill}%
\end{pgfscope}%
\begin{pgfscope}%
\pgfpathrectangle{\pgfqpoint{0.600000in}{0.600000in}}{\pgfqpoint{3.900000in}{3.900000in}}%
\pgfusepath{clip}%
\pgfsetbuttcap%
\pgfsetroundjoin%
\definecolor{currentfill}{rgb}{0.121569,0.466667,0.705882}%
\pgfsetfillcolor{currentfill}%
\pgfsetlinewidth{1.003750pt}%
\definecolor{currentstroke}{rgb}{0.121569,0.466667,0.705882}%
\pgfsetstrokecolor{currentstroke}%
\pgfsetdash{}{0pt}%
\pgfpathmoveto{\pgfqpoint{2.896678in}{2.479409in}}%
\pgfpathcurveto{\pgfqpoint{2.907728in}{2.479409in}}{\pgfqpoint{2.918328in}{2.483799in}}{\pgfqpoint{2.926141in}{2.491613in}}%
\pgfpathcurveto{\pgfqpoint{2.933955in}{2.499427in}}{\pgfqpoint{2.938345in}{2.510026in}}{\pgfqpoint{2.938345in}{2.521076in}}%
\pgfpathcurveto{\pgfqpoint{2.938345in}{2.532126in}}{\pgfqpoint{2.933955in}{2.542725in}}{\pgfqpoint{2.926141in}{2.550539in}}%
\pgfpathcurveto{\pgfqpoint{2.918328in}{2.558352in}}{\pgfqpoint{2.907728in}{2.562742in}}{\pgfqpoint{2.896678in}{2.562742in}}%
\pgfpathcurveto{\pgfqpoint{2.885628in}{2.562742in}}{\pgfqpoint{2.875029in}{2.558352in}}{\pgfqpoint{2.867216in}{2.550539in}}%
\pgfpathcurveto{\pgfqpoint{2.859402in}{2.542725in}}{\pgfqpoint{2.855012in}{2.532126in}}{\pgfqpoint{2.855012in}{2.521076in}}%
\pgfpathcurveto{\pgfqpoint{2.855012in}{2.510026in}}{\pgfqpoint{2.859402in}{2.499427in}}{\pgfqpoint{2.867216in}{2.491613in}}%
\pgfpathcurveto{\pgfqpoint{2.875029in}{2.483799in}}{\pgfqpoint{2.885628in}{2.479409in}}{\pgfqpoint{2.896678in}{2.479409in}}%
\pgfpathclose%
\pgfusepath{stroke,fill}%
\end{pgfscope}%
\begin{pgfscope}%
\pgfpathrectangle{\pgfqpoint{0.600000in}{0.600000in}}{\pgfqpoint{3.900000in}{3.900000in}}%
\pgfusepath{clip}%
\pgfsetbuttcap%
\pgfsetroundjoin%
\definecolor{currentfill}{rgb}{0.121569,0.466667,0.705882}%
\pgfsetfillcolor{currentfill}%
\pgfsetlinewidth{1.003750pt}%
\definecolor{currentstroke}{rgb}{0.121569,0.466667,0.705882}%
\pgfsetstrokecolor{currentstroke}%
\pgfsetdash{}{0pt}%
\pgfpathmoveto{\pgfqpoint{1.887511in}{2.627940in}}%
\pgfpathcurveto{\pgfqpoint{1.898561in}{2.627940in}}{\pgfqpoint{1.909160in}{2.632330in}}{\pgfqpoint{1.916974in}{2.640144in}}%
\pgfpathcurveto{\pgfqpoint{1.924788in}{2.647958in}}{\pgfqpoint{1.929178in}{2.658557in}}{\pgfqpoint{1.929178in}{2.669607in}}%
\pgfpathcurveto{\pgfqpoint{1.929178in}{2.680657in}}{\pgfqpoint{1.924788in}{2.691256in}}{\pgfqpoint{1.916974in}{2.699070in}}%
\pgfpathcurveto{\pgfqpoint{1.909160in}{2.706883in}}{\pgfqpoint{1.898561in}{2.711273in}}{\pgfqpoint{1.887511in}{2.711273in}}%
\pgfpathcurveto{\pgfqpoint{1.876461in}{2.711273in}}{\pgfqpoint{1.865862in}{2.706883in}}{\pgfqpoint{1.858048in}{2.699070in}}%
\pgfpathcurveto{\pgfqpoint{1.850235in}{2.691256in}}{\pgfqpoint{1.845845in}{2.680657in}}{\pgfqpoint{1.845845in}{2.669607in}}%
\pgfpathcurveto{\pgfqpoint{1.845845in}{2.658557in}}{\pgfqpoint{1.850235in}{2.647958in}}{\pgfqpoint{1.858048in}{2.640144in}}%
\pgfpathcurveto{\pgfqpoint{1.865862in}{2.632330in}}{\pgfqpoint{1.876461in}{2.627940in}}{\pgfqpoint{1.887511in}{2.627940in}}%
\pgfpathclose%
\pgfusepath{stroke,fill}%
\end{pgfscope}%
\begin{pgfscope}%
\pgfpathrectangle{\pgfqpoint{0.600000in}{0.600000in}}{\pgfqpoint{3.900000in}{3.900000in}}%
\pgfusepath{clip}%
\pgfsetbuttcap%
\pgfsetroundjoin%
\definecolor{currentfill}{rgb}{0.121569,0.466667,0.705882}%
\pgfsetfillcolor{currentfill}%
\pgfsetlinewidth{1.003750pt}%
\definecolor{currentstroke}{rgb}{0.121569,0.466667,0.705882}%
\pgfsetstrokecolor{currentstroke}%
\pgfsetdash{}{0pt}%
\pgfpathmoveto{\pgfqpoint{1.776264in}{3.098034in}}%
\pgfpathcurveto{\pgfqpoint{1.787314in}{3.098034in}}{\pgfqpoint{1.797913in}{3.102424in}}{\pgfqpoint{1.805727in}{3.110238in}}%
\pgfpathcurveto{\pgfqpoint{1.813540in}{3.118051in}}{\pgfqpoint{1.817931in}{3.128650in}}{\pgfqpoint{1.817931in}{3.139700in}}%
\pgfpathcurveto{\pgfqpoint{1.817931in}{3.150751in}}{\pgfqpoint{1.813540in}{3.161350in}}{\pgfqpoint{1.805727in}{3.169163in}}%
\pgfpathcurveto{\pgfqpoint{1.797913in}{3.176977in}}{\pgfqpoint{1.787314in}{3.181367in}}{\pgfqpoint{1.776264in}{3.181367in}}%
\pgfpathcurveto{\pgfqpoint{1.765214in}{3.181367in}}{\pgfqpoint{1.754615in}{3.176977in}}{\pgfqpoint{1.746801in}{3.169163in}}%
\pgfpathcurveto{\pgfqpoint{1.738988in}{3.161350in}}{\pgfqpoint{1.734597in}{3.150751in}}{\pgfqpoint{1.734597in}{3.139700in}}%
\pgfpathcurveto{\pgfqpoint{1.734597in}{3.128650in}}{\pgfqpoint{1.738988in}{3.118051in}}{\pgfqpoint{1.746801in}{3.110238in}}%
\pgfpathcurveto{\pgfqpoint{1.754615in}{3.102424in}}{\pgfqpoint{1.765214in}{3.098034in}}{\pgfqpoint{1.776264in}{3.098034in}}%
\pgfpathclose%
\pgfusepath{stroke,fill}%
\end{pgfscope}%
\begin{pgfscope}%
\pgfpathrectangle{\pgfqpoint{0.600000in}{0.600000in}}{\pgfqpoint{3.900000in}{3.900000in}}%
\pgfusepath{clip}%
\pgfsetbuttcap%
\pgfsetroundjoin%
\definecolor{currentfill}{rgb}{0.121569,0.466667,0.705882}%
\pgfsetfillcolor{currentfill}%
\pgfsetlinewidth{1.003750pt}%
\definecolor{currentstroke}{rgb}{0.121569,0.466667,0.705882}%
\pgfsetstrokecolor{currentstroke}%
\pgfsetdash{}{0pt}%
\pgfpathmoveto{\pgfqpoint{2.211483in}{2.378096in}}%
\pgfpathcurveto{\pgfqpoint{2.222534in}{2.378096in}}{\pgfqpoint{2.233133in}{2.382486in}}{\pgfqpoint{2.240946in}{2.390300in}}%
\pgfpathcurveto{\pgfqpoint{2.248760in}{2.398113in}}{\pgfqpoint{2.253150in}{2.408712in}}{\pgfqpoint{2.253150in}{2.419762in}}%
\pgfpathcurveto{\pgfqpoint{2.253150in}{2.430812in}}{\pgfqpoint{2.248760in}{2.441412in}}{\pgfqpoint{2.240946in}{2.449225in}}%
\pgfpathcurveto{\pgfqpoint{2.233133in}{2.457039in}}{\pgfqpoint{2.222534in}{2.461429in}}{\pgfqpoint{2.211483in}{2.461429in}}%
\pgfpathcurveto{\pgfqpoint{2.200433in}{2.461429in}}{\pgfqpoint{2.189834in}{2.457039in}}{\pgfqpoint{2.182021in}{2.449225in}}%
\pgfpathcurveto{\pgfqpoint{2.174207in}{2.441412in}}{\pgfqpoint{2.169817in}{2.430812in}}{\pgfqpoint{2.169817in}{2.419762in}}%
\pgfpathcurveto{\pgfqpoint{2.169817in}{2.408712in}}{\pgfqpoint{2.174207in}{2.398113in}}{\pgfqpoint{2.182021in}{2.390300in}}%
\pgfpathcurveto{\pgfqpoint{2.189834in}{2.382486in}}{\pgfqpoint{2.200433in}{2.378096in}}{\pgfqpoint{2.211483in}{2.378096in}}%
\pgfpathclose%
\pgfusepath{stroke,fill}%
\end{pgfscope}%
\begin{pgfscope}%
\pgfpathrectangle{\pgfqpoint{0.600000in}{0.600000in}}{\pgfqpoint{3.900000in}{3.900000in}}%
\pgfusepath{clip}%
\pgfsetbuttcap%
\pgfsetroundjoin%
\definecolor{currentfill}{rgb}{0.121569,0.466667,0.705882}%
\pgfsetfillcolor{currentfill}%
\pgfsetlinewidth{1.003750pt}%
\definecolor{currentstroke}{rgb}{0.121569,0.466667,0.705882}%
\pgfsetstrokecolor{currentstroke}%
\pgfsetdash{}{0pt}%
\pgfpathmoveto{\pgfqpoint{3.647083in}{2.430247in}}%
\pgfpathcurveto{\pgfqpoint{3.658133in}{2.430247in}}{\pgfqpoint{3.668732in}{2.434638in}}{\pgfqpoint{3.676546in}{2.442451in}}%
\pgfpathcurveto{\pgfqpoint{3.684360in}{2.450265in}}{\pgfqpoint{3.688750in}{2.460864in}}{\pgfqpoint{3.688750in}{2.471914in}}%
\pgfpathcurveto{\pgfqpoint{3.688750in}{2.482964in}}{\pgfqpoint{3.684360in}{2.493563in}}{\pgfqpoint{3.676546in}{2.501377in}}%
\pgfpathcurveto{\pgfqpoint{3.668732in}{2.509190in}}{\pgfqpoint{3.658133in}{2.513581in}}{\pgfqpoint{3.647083in}{2.513581in}}%
\pgfpathcurveto{\pgfqpoint{3.636033in}{2.513581in}}{\pgfqpoint{3.625434in}{2.509190in}}{\pgfqpoint{3.617620in}{2.501377in}}%
\pgfpathcurveto{\pgfqpoint{3.609807in}{2.493563in}}{\pgfqpoint{3.605417in}{2.482964in}}{\pgfqpoint{3.605417in}{2.471914in}}%
\pgfpathcurveto{\pgfqpoint{3.605417in}{2.460864in}}{\pgfqpoint{3.609807in}{2.450265in}}{\pgfqpoint{3.617620in}{2.442451in}}%
\pgfpathcurveto{\pgfqpoint{3.625434in}{2.434638in}}{\pgfqpoint{3.636033in}{2.430247in}}{\pgfqpoint{3.647083in}{2.430247in}}%
\pgfpathclose%
\pgfusepath{stroke,fill}%
\end{pgfscope}%
\begin{pgfscope}%
\pgfpathrectangle{\pgfqpoint{0.600000in}{0.600000in}}{\pgfqpoint{3.900000in}{3.900000in}}%
\pgfusepath{clip}%
\pgfsetbuttcap%
\pgfsetroundjoin%
\definecolor{currentfill}{rgb}{0.121569,0.466667,0.705882}%
\pgfsetfillcolor{currentfill}%
\pgfsetlinewidth{1.003750pt}%
\definecolor{currentstroke}{rgb}{0.121569,0.466667,0.705882}%
\pgfsetstrokecolor{currentstroke}%
\pgfsetdash{}{0pt}%
\pgfpathmoveto{\pgfqpoint{2.454137in}{2.894090in}}%
\pgfpathcurveto{\pgfqpoint{2.465187in}{2.894090in}}{\pgfqpoint{2.475786in}{2.898480in}}{\pgfqpoint{2.483600in}{2.906294in}}%
\pgfpathcurveto{\pgfqpoint{2.491413in}{2.914107in}}{\pgfqpoint{2.495804in}{2.924706in}}{\pgfqpoint{2.495804in}{2.935756in}}%
\pgfpathcurveto{\pgfqpoint{2.495804in}{2.946807in}}{\pgfqpoint{2.491413in}{2.957406in}}{\pgfqpoint{2.483600in}{2.965219in}}%
\pgfpathcurveto{\pgfqpoint{2.475786in}{2.973033in}}{\pgfqpoint{2.465187in}{2.977423in}}{\pgfqpoint{2.454137in}{2.977423in}}%
\pgfpathcurveto{\pgfqpoint{2.443087in}{2.977423in}}{\pgfqpoint{2.432488in}{2.973033in}}{\pgfqpoint{2.424674in}{2.965219in}}%
\pgfpathcurveto{\pgfqpoint{2.416860in}{2.957406in}}{\pgfqpoint{2.412470in}{2.946807in}}{\pgfqpoint{2.412470in}{2.935756in}}%
\pgfpathcurveto{\pgfqpoint{2.412470in}{2.924706in}}{\pgfqpoint{2.416860in}{2.914107in}}{\pgfqpoint{2.424674in}{2.906294in}}%
\pgfpathcurveto{\pgfqpoint{2.432488in}{2.898480in}}{\pgfqpoint{2.443087in}{2.894090in}}{\pgfqpoint{2.454137in}{2.894090in}}%
\pgfpathclose%
\pgfusepath{stroke,fill}%
\end{pgfscope}%
\begin{pgfscope}%
\pgfpathrectangle{\pgfqpoint{0.600000in}{0.600000in}}{\pgfqpoint{3.900000in}{3.900000in}}%
\pgfusepath{clip}%
\pgfsetbuttcap%
\pgfsetroundjoin%
\definecolor{currentfill}{rgb}{0.121569,0.466667,0.705882}%
\pgfsetfillcolor{currentfill}%
\pgfsetlinewidth{1.003750pt}%
\definecolor{currentstroke}{rgb}{0.121569,0.466667,0.705882}%
\pgfsetstrokecolor{currentstroke}%
\pgfsetdash{}{0pt}%
\pgfpathmoveto{\pgfqpoint{2.176052in}{2.753455in}}%
\pgfpathcurveto{\pgfqpoint{2.187102in}{2.753455in}}{\pgfqpoint{2.197701in}{2.757845in}}{\pgfqpoint{2.205514in}{2.765659in}}%
\pgfpathcurveto{\pgfqpoint{2.213328in}{2.773472in}}{\pgfqpoint{2.217718in}{2.784071in}}{\pgfqpoint{2.217718in}{2.795121in}}%
\pgfpathcurveto{\pgfqpoint{2.217718in}{2.806172in}}{\pgfqpoint{2.213328in}{2.816771in}}{\pgfqpoint{2.205514in}{2.824584in}}%
\pgfpathcurveto{\pgfqpoint{2.197701in}{2.832398in}}{\pgfqpoint{2.187102in}{2.836788in}}{\pgfqpoint{2.176052in}{2.836788in}}%
\pgfpathcurveto{\pgfqpoint{2.165002in}{2.836788in}}{\pgfqpoint{2.154403in}{2.832398in}}{\pgfqpoint{2.146589in}{2.824584in}}%
\pgfpathcurveto{\pgfqpoint{2.138775in}{2.816771in}}{\pgfqpoint{2.134385in}{2.806172in}}{\pgfqpoint{2.134385in}{2.795121in}}%
\pgfpathcurveto{\pgfqpoint{2.134385in}{2.784071in}}{\pgfqpoint{2.138775in}{2.773472in}}{\pgfqpoint{2.146589in}{2.765659in}}%
\pgfpathcurveto{\pgfqpoint{2.154403in}{2.757845in}}{\pgfqpoint{2.165002in}{2.753455in}}{\pgfqpoint{2.176052in}{2.753455in}}%
\pgfpathclose%
\pgfusepath{stroke,fill}%
\end{pgfscope}%
\begin{pgfscope}%
\pgfpathrectangle{\pgfqpoint{0.600000in}{0.600000in}}{\pgfqpoint{3.900000in}{3.900000in}}%
\pgfusepath{clip}%
\pgfsetbuttcap%
\pgfsetroundjoin%
\definecolor{currentfill}{rgb}{0.121569,0.466667,0.705882}%
\pgfsetfillcolor{currentfill}%
\pgfsetlinewidth{1.003750pt}%
\definecolor{currentstroke}{rgb}{0.121569,0.466667,0.705882}%
\pgfsetstrokecolor{currentstroke}%
\pgfsetdash{}{0pt}%
\pgfpathmoveto{\pgfqpoint{1.969177in}{2.835482in}}%
\pgfpathcurveto{\pgfqpoint{1.980227in}{2.835482in}}{\pgfqpoint{1.990826in}{2.839872in}}{\pgfqpoint{1.998640in}{2.847686in}}%
\pgfpathcurveto{\pgfqpoint{2.006454in}{2.855500in}}{\pgfqpoint{2.010844in}{2.866099in}}{\pgfqpoint{2.010844in}{2.877149in}}%
\pgfpathcurveto{\pgfqpoint{2.010844in}{2.888199in}}{\pgfqpoint{2.006454in}{2.898798in}}{\pgfqpoint{1.998640in}{2.906612in}}%
\pgfpathcurveto{\pgfqpoint{1.990826in}{2.914425in}}{\pgfqpoint{1.980227in}{2.918815in}}{\pgfqpoint{1.969177in}{2.918815in}}%
\pgfpathcurveto{\pgfqpoint{1.958127in}{2.918815in}}{\pgfqpoint{1.947528in}{2.914425in}}{\pgfqpoint{1.939714in}{2.906612in}}%
\pgfpathcurveto{\pgfqpoint{1.931901in}{2.898798in}}{\pgfqpoint{1.927511in}{2.888199in}}{\pgfqpoint{1.927511in}{2.877149in}}%
\pgfpathcurveto{\pgfqpoint{1.927511in}{2.866099in}}{\pgfqpoint{1.931901in}{2.855500in}}{\pgfqpoint{1.939714in}{2.847686in}}%
\pgfpathcurveto{\pgfqpoint{1.947528in}{2.839872in}}{\pgfqpoint{1.958127in}{2.835482in}}{\pgfqpoint{1.969177in}{2.835482in}}%
\pgfpathclose%
\pgfusepath{stroke,fill}%
\end{pgfscope}%
\begin{pgfscope}%
\pgfpathrectangle{\pgfqpoint{0.600000in}{0.600000in}}{\pgfqpoint{3.900000in}{3.900000in}}%
\pgfusepath{clip}%
\pgfsetbuttcap%
\pgfsetroundjoin%
\definecolor{currentfill}{rgb}{0.121569,0.466667,0.705882}%
\pgfsetfillcolor{currentfill}%
\pgfsetlinewidth{1.003750pt}%
\definecolor{currentstroke}{rgb}{0.121569,0.466667,0.705882}%
\pgfsetstrokecolor{currentstroke}%
\pgfsetdash{}{0pt}%
\pgfpathmoveto{\pgfqpoint{2.361795in}{3.649730in}}%
\pgfpathcurveto{\pgfqpoint{2.372845in}{3.649730in}}{\pgfqpoint{2.383444in}{3.654120in}}{\pgfqpoint{2.391258in}{3.661934in}}%
\pgfpathcurveto{\pgfqpoint{2.399071in}{3.669748in}}{\pgfqpoint{2.403461in}{3.680347in}}{\pgfqpoint{2.403461in}{3.691397in}}%
\pgfpathcurveto{\pgfqpoint{2.403461in}{3.702447in}}{\pgfqpoint{2.399071in}{3.713046in}}{\pgfqpoint{2.391258in}{3.720860in}}%
\pgfpathcurveto{\pgfqpoint{2.383444in}{3.728673in}}{\pgfqpoint{2.372845in}{3.733063in}}{\pgfqpoint{2.361795in}{3.733063in}}%
\pgfpathcurveto{\pgfqpoint{2.350745in}{3.733063in}}{\pgfqpoint{2.340146in}{3.728673in}}{\pgfqpoint{2.332332in}{3.720860in}}%
\pgfpathcurveto{\pgfqpoint{2.324518in}{3.713046in}}{\pgfqpoint{2.320128in}{3.702447in}}{\pgfqpoint{2.320128in}{3.691397in}}%
\pgfpathcurveto{\pgfqpoint{2.320128in}{3.680347in}}{\pgfqpoint{2.324518in}{3.669748in}}{\pgfqpoint{2.332332in}{3.661934in}}%
\pgfpathcurveto{\pgfqpoint{2.340146in}{3.654120in}}{\pgfqpoint{2.350745in}{3.649730in}}{\pgfqpoint{2.361795in}{3.649730in}}%
\pgfpathclose%
\pgfusepath{stroke,fill}%
\end{pgfscope}%
\begin{pgfscope}%
\pgfpathrectangle{\pgfqpoint{0.600000in}{0.600000in}}{\pgfqpoint{3.900000in}{3.900000in}}%
\pgfusepath{clip}%
\pgfsetbuttcap%
\pgfsetroundjoin%
\definecolor{currentfill}{rgb}{0.121569,0.466667,0.705882}%
\pgfsetfillcolor{currentfill}%
\pgfsetlinewidth{1.003750pt}%
\definecolor{currentstroke}{rgb}{0.121569,0.466667,0.705882}%
\pgfsetstrokecolor{currentstroke}%
\pgfsetdash{}{0pt}%
\pgfpathmoveto{\pgfqpoint{2.288813in}{2.285056in}}%
\pgfpathcurveto{\pgfqpoint{2.299863in}{2.285056in}}{\pgfqpoint{2.310462in}{2.289446in}}{\pgfqpoint{2.318276in}{2.297260in}}%
\pgfpathcurveto{\pgfqpoint{2.326089in}{2.305074in}}{\pgfqpoint{2.330479in}{2.315673in}}{\pgfqpoint{2.330479in}{2.326723in}}%
\pgfpathcurveto{\pgfqpoint{2.330479in}{2.337773in}}{\pgfqpoint{2.326089in}{2.348372in}}{\pgfqpoint{2.318276in}{2.356186in}}%
\pgfpathcurveto{\pgfqpoint{2.310462in}{2.363999in}}{\pgfqpoint{2.299863in}{2.368389in}}{\pgfqpoint{2.288813in}{2.368389in}}%
\pgfpathcurveto{\pgfqpoint{2.277763in}{2.368389in}}{\pgfqpoint{2.267164in}{2.363999in}}{\pgfqpoint{2.259350in}{2.356186in}}%
\pgfpathcurveto{\pgfqpoint{2.251536in}{2.348372in}}{\pgfqpoint{2.247146in}{2.337773in}}{\pgfqpoint{2.247146in}{2.326723in}}%
\pgfpathcurveto{\pgfqpoint{2.247146in}{2.315673in}}{\pgfqpoint{2.251536in}{2.305074in}}{\pgfqpoint{2.259350in}{2.297260in}}%
\pgfpathcurveto{\pgfqpoint{2.267164in}{2.289446in}}{\pgfqpoint{2.277763in}{2.285056in}}{\pgfqpoint{2.288813in}{2.285056in}}%
\pgfpathclose%
\pgfusepath{stroke,fill}%
\end{pgfscope}%
\begin{pgfscope}%
\pgfpathrectangle{\pgfqpoint{0.600000in}{0.600000in}}{\pgfqpoint{3.900000in}{3.900000in}}%
\pgfusepath{clip}%
\pgfsetbuttcap%
\pgfsetroundjoin%
\definecolor{currentfill}{rgb}{0.121569,0.466667,0.705882}%
\pgfsetfillcolor{currentfill}%
\pgfsetlinewidth{1.003750pt}%
\definecolor{currentstroke}{rgb}{0.121569,0.466667,0.705882}%
\pgfsetstrokecolor{currentstroke}%
\pgfsetdash{}{0pt}%
\pgfpathmoveto{\pgfqpoint{2.383684in}{3.271821in}}%
\pgfpathcurveto{\pgfqpoint{2.394734in}{3.271821in}}{\pgfqpoint{2.405333in}{3.276212in}}{\pgfqpoint{2.413146in}{3.284025in}}%
\pgfpathcurveto{\pgfqpoint{2.420960in}{3.291839in}}{\pgfqpoint{2.425350in}{3.302438in}}{\pgfqpoint{2.425350in}{3.313488in}}%
\pgfpathcurveto{\pgfqpoint{2.425350in}{3.324538in}}{\pgfqpoint{2.420960in}{3.335137in}}{\pgfqpoint{2.413146in}{3.342951in}}%
\pgfpathcurveto{\pgfqpoint{2.405333in}{3.350764in}}{\pgfqpoint{2.394734in}{3.355155in}}{\pgfqpoint{2.383684in}{3.355155in}}%
\pgfpathcurveto{\pgfqpoint{2.372633in}{3.355155in}}{\pgfqpoint{2.362034in}{3.350764in}}{\pgfqpoint{2.354221in}{3.342951in}}%
\pgfpathcurveto{\pgfqpoint{2.346407in}{3.335137in}}{\pgfqpoint{2.342017in}{3.324538in}}{\pgfqpoint{2.342017in}{3.313488in}}%
\pgfpathcurveto{\pgfqpoint{2.342017in}{3.302438in}}{\pgfqpoint{2.346407in}{3.291839in}}{\pgfqpoint{2.354221in}{3.284025in}}%
\pgfpathcurveto{\pgfqpoint{2.362034in}{3.276212in}}{\pgfqpoint{2.372633in}{3.271821in}}{\pgfqpoint{2.383684in}{3.271821in}}%
\pgfpathclose%
\pgfusepath{stroke,fill}%
\end{pgfscope}%
\begin{pgfscope}%
\pgfpathrectangle{\pgfqpoint{0.600000in}{0.600000in}}{\pgfqpoint{3.900000in}{3.900000in}}%
\pgfusepath{clip}%
\pgfsetbuttcap%
\pgfsetroundjoin%
\definecolor{currentfill}{rgb}{0.121569,0.466667,0.705882}%
\pgfsetfillcolor{currentfill}%
\pgfsetlinewidth{1.003750pt}%
\definecolor{currentstroke}{rgb}{0.121569,0.466667,0.705882}%
\pgfsetstrokecolor{currentstroke}%
\pgfsetdash{}{0pt}%
\pgfpathmoveto{\pgfqpoint{2.522852in}{3.367346in}}%
\pgfpathcurveto{\pgfqpoint{2.533902in}{3.367346in}}{\pgfqpoint{2.544501in}{3.371736in}}{\pgfqpoint{2.552315in}{3.379550in}}%
\pgfpathcurveto{\pgfqpoint{2.560128in}{3.387363in}}{\pgfqpoint{2.564519in}{3.397962in}}{\pgfqpoint{2.564519in}{3.409013in}}%
\pgfpathcurveto{\pgfqpoint{2.564519in}{3.420063in}}{\pgfqpoint{2.560128in}{3.430662in}}{\pgfqpoint{2.552315in}{3.438475in}}%
\pgfpathcurveto{\pgfqpoint{2.544501in}{3.446289in}}{\pgfqpoint{2.533902in}{3.450679in}}{\pgfqpoint{2.522852in}{3.450679in}}%
\pgfpathcurveto{\pgfqpoint{2.511802in}{3.450679in}}{\pgfqpoint{2.501203in}{3.446289in}}{\pgfqpoint{2.493389in}{3.438475in}}%
\pgfpathcurveto{\pgfqpoint{2.485575in}{3.430662in}}{\pgfqpoint{2.481185in}{3.420063in}}{\pgfqpoint{2.481185in}{3.409013in}}%
\pgfpathcurveto{\pgfqpoint{2.481185in}{3.397962in}}{\pgfqpoint{2.485575in}{3.387363in}}{\pgfqpoint{2.493389in}{3.379550in}}%
\pgfpathcurveto{\pgfqpoint{2.501203in}{3.371736in}}{\pgfqpoint{2.511802in}{3.367346in}}{\pgfqpoint{2.522852in}{3.367346in}}%
\pgfpathclose%
\pgfusepath{stroke,fill}%
\end{pgfscope}%
\begin{pgfscope}%
\pgfpathrectangle{\pgfqpoint{0.600000in}{0.600000in}}{\pgfqpoint{3.900000in}{3.900000in}}%
\pgfusepath{clip}%
\pgfsetbuttcap%
\pgfsetroundjoin%
\definecolor{currentfill}{rgb}{0.121569,0.466667,0.705882}%
\pgfsetfillcolor{currentfill}%
\pgfsetlinewidth{1.003750pt}%
\definecolor{currentstroke}{rgb}{0.121569,0.466667,0.705882}%
\pgfsetstrokecolor{currentstroke}%
\pgfsetdash{}{0pt}%
\pgfpathmoveto{\pgfqpoint{2.521431in}{2.233186in}}%
\pgfpathcurveto{\pgfqpoint{2.532481in}{2.233186in}}{\pgfqpoint{2.543080in}{2.237577in}}{\pgfqpoint{2.550893in}{2.245390in}}%
\pgfpathcurveto{\pgfqpoint{2.558707in}{2.253204in}}{\pgfqpoint{2.563097in}{2.263803in}}{\pgfqpoint{2.563097in}{2.274853in}}%
\pgfpathcurveto{\pgfqpoint{2.563097in}{2.285903in}}{\pgfqpoint{2.558707in}{2.296502in}}{\pgfqpoint{2.550893in}{2.304316in}}%
\pgfpathcurveto{\pgfqpoint{2.543080in}{2.312129in}}{\pgfqpoint{2.532481in}{2.316520in}}{\pgfqpoint{2.521431in}{2.316520in}}%
\pgfpathcurveto{\pgfqpoint{2.510381in}{2.316520in}}{\pgfqpoint{2.499781in}{2.312129in}}{\pgfqpoint{2.491968in}{2.304316in}}%
\pgfpathcurveto{\pgfqpoint{2.484154in}{2.296502in}}{\pgfqpoint{2.479764in}{2.285903in}}{\pgfqpoint{2.479764in}{2.274853in}}%
\pgfpathcurveto{\pgfqpoint{2.479764in}{2.263803in}}{\pgfqpoint{2.484154in}{2.253204in}}{\pgfqpoint{2.491968in}{2.245390in}}%
\pgfpathcurveto{\pgfqpoint{2.499781in}{2.237577in}}{\pgfqpoint{2.510381in}{2.233186in}}{\pgfqpoint{2.521431in}{2.233186in}}%
\pgfpathclose%
\pgfusepath{stroke,fill}%
\end{pgfscope}%
\begin{pgfscope}%
\pgfpathrectangle{\pgfqpoint{0.600000in}{0.600000in}}{\pgfqpoint{3.900000in}{3.900000in}}%
\pgfusepath{clip}%
\pgfsetbuttcap%
\pgfsetroundjoin%
\definecolor{currentfill}{rgb}{0.121569,0.466667,0.705882}%
\pgfsetfillcolor{currentfill}%
\pgfsetlinewidth{1.003750pt}%
\definecolor{currentstroke}{rgb}{0.121569,0.466667,0.705882}%
\pgfsetstrokecolor{currentstroke}%
\pgfsetdash{}{0pt}%
\pgfpathmoveto{\pgfqpoint{3.017465in}{2.720949in}}%
\pgfpathcurveto{\pgfqpoint{3.028515in}{2.720949in}}{\pgfqpoint{3.039114in}{2.725339in}}{\pgfqpoint{3.046928in}{2.733153in}}%
\pgfpathcurveto{\pgfqpoint{3.054742in}{2.740966in}}{\pgfqpoint{3.059132in}{2.751565in}}{\pgfqpoint{3.059132in}{2.762615in}}%
\pgfpathcurveto{\pgfqpoint{3.059132in}{2.773665in}}{\pgfqpoint{3.054742in}{2.784265in}}{\pgfqpoint{3.046928in}{2.792078in}}%
\pgfpathcurveto{\pgfqpoint{3.039114in}{2.799892in}}{\pgfqpoint{3.028515in}{2.804282in}}{\pgfqpoint{3.017465in}{2.804282in}}%
\pgfpathcurveto{\pgfqpoint{3.006415in}{2.804282in}}{\pgfqpoint{2.995816in}{2.799892in}}{\pgfqpoint{2.988003in}{2.792078in}}%
\pgfpathcurveto{\pgfqpoint{2.980189in}{2.784265in}}{\pgfqpoint{2.975799in}{2.773665in}}{\pgfqpoint{2.975799in}{2.762615in}}%
\pgfpathcurveto{\pgfqpoint{2.975799in}{2.751565in}}{\pgfqpoint{2.980189in}{2.740966in}}{\pgfqpoint{2.988003in}{2.733153in}}%
\pgfpathcurveto{\pgfqpoint{2.995816in}{2.725339in}}{\pgfqpoint{3.006415in}{2.720949in}}{\pgfqpoint{3.017465in}{2.720949in}}%
\pgfpathclose%
\pgfusepath{stroke,fill}%
\end{pgfscope}%
\begin{pgfscope}%
\pgfpathrectangle{\pgfqpoint{0.600000in}{0.600000in}}{\pgfqpoint{3.900000in}{3.900000in}}%
\pgfusepath{clip}%
\pgfsetbuttcap%
\pgfsetroundjoin%
\definecolor{currentfill}{rgb}{0.121569,0.466667,0.705882}%
\pgfsetfillcolor{currentfill}%
\pgfsetlinewidth{1.003750pt}%
\definecolor{currentstroke}{rgb}{0.121569,0.466667,0.705882}%
\pgfsetstrokecolor{currentstroke}%
\pgfsetdash{}{0pt}%
\pgfpathmoveto{\pgfqpoint{2.540979in}{2.734031in}}%
\pgfpathcurveto{\pgfqpoint{2.552029in}{2.734031in}}{\pgfqpoint{2.562628in}{2.738421in}}{\pgfqpoint{2.570442in}{2.746235in}}%
\pgfpathcurveto{\pgfqpoint{2.578255in}{2.754048in}}{\pgfqpoint{2.582646in}{2.764648in}}{\pgfqpoint{2.582646in}{2.775698in}}%
\pgfpathcurveto{\pgfqpoint{2.582646in}{2.786748in}}{\pgfqpoint{2.578255in}{2.797347in}}{\pgfqpoint{2.570442in}{2.805160in}}%
\pgfpathcurveto{\pgfqpoint{2.562628in}{2.812974in}}{\pgfqpoint{2.552029in}{2.817364in}}{\pgfqpoint{2.540979in}{2.817364in}}%
\pgfpathcurveto{\pgfqpoint{2.529929in}{2.817364in}}{\pgfqpoint{2.519330in}{2.812974in}}{\pgfqpoint{2.511516in}{2.805160in}}%
\pgfpathcurveto{\pgfqpoint{2.503703in}{2.797347in}}{\pgfqpoint{2.499312in}{2.786748in}}{\pgfqpoint{2.499312in}{2.775698in}}%
\pgfpathcurveto{\pgfqpoint{2.499312in}{2.764648in}}{\pgfqpoint{2.503703in}{2.754048in}}{\pgfqpoint{2.511516in}{2.746235in}}%
\pgfpathcurveto{\pgfqpoint{2.519330in}{2.738421in}}{\pgfqpoint{2.529929in}{2.734031in}}{\pgfqpoint{2.540979in}{2.734031in}}%
\pgfpathclose%
\pgfusepath{stroke,fill}%
\end{pgfscope}%
\begin{pgfscope}%
\pgfpathrectangle{\pgfqpoint{0.600000in}{0.600000in}}{\pgfqpoint{3.900000in}{3.900000in}}%
\pgfusepath{clip}%
\pgfsetbuttcap%
\pgfsetroundjoin%
\definecolor{currentfill}{rgb}{0.121569,0.466667,0.705882}%
\pgfsetfillcolor{currentfill}%
\pgfsetlinewidth{1.003750pt}%
\definecolor{currentstroke}{rgb}{0.121569,0.466667,0.705882}%
\pgfsetstrokecolor{currentstroke}%
\pgfsetdash{}{0pt}%
\pgfpathmoveto{\pgfqpoint{2.676402in}{3.124822in}}%
\pgfpathcurveto{\pgfqpoint{2.687452in}{3.124822in}}{\pgfqpoint{2.698052in}{3.129213in}}{\pgfqpoint{2.705865in}{3.137026in}}%
\pgfpathcurveto{\pgfqpoint{2.713679in}{3.144840in}}{\pgfqpoint{2.718069in}{3.155439in}}{\pgfqpoint{2.718069in}{3.166489in}}%
\pgfpathcurveto{\pgfqpoint{2.718069in}{3.177539in}}{\pgfqpoint{2.713679in}{3.188138in}}{\pgfqpoint{2.705865in}{3.195952in}}%
\pgfpathcurveto{\pgfqpoint{2.698052in}{3.203765in}}{\pgfqpoint{2.687452in}{3.208156in}}{\pgfqpoint{2.676402in}{3.208156in}}%
\pgfpathcurveto{\pgfqpoint{2.665352in}{3.208156in}}{\pgfqpoint{2.654753in}{3.203765in}}{\pgfqpoint{2.646940in}{3.195952in}}%
\pgfpathcurveto{\pgfqpoint{2.639126in}{3.188138in}}{\pgfqpoint{2.634736in}{3.177539in}}{\pgfqpoint{2.634736in}{3.166489in}}%
\pgfpathcurveto{\pgfqpoint{2.634736in}{3.155439in}}{\pgfqpoint{2.639126in}{3.144840in}}{\pgfqpoint{2.646940in}{3.137026in}}%
\pgfpathcurveto{\pgfqpoint{2.654753in}{3.129213in}}{\pgfqpoint{2.665352in}{3.124822in}}{\pgfqpoint{2.676402in}{3.124822in}}%
\pgfpathclose%
\pgfusepath{stroke,fill}%
\end{pgfscope}%
\begin{pgfscope}%
\pgfpathrectangle{\pgfqpoint{0.600000in}{0.600000in}}{\pgfqpoint{3.900000in}{3.900000in}}%
\pgfusepath{clip}%
\pgfsetbuttcap%
\pgfsetroundjoin%
\definecolor{currentfill}{rgb}{0.121569,0.466667,0.705882}%
\pgfsetfillcolor{currentfill}%
\pgfsetlinewidth{1.003750pt}%
\definecolor{currentstroke}{rgb}{0.121569,0.466667,0.705882}%
\pgfsetstrokecolor{currentstroke}%
\pgfsetdash{}{0pt}%
\pgfpathmoveto{\pgfqpoint{2.447287in}{2.649866in}}%
\pgfpathcurveto{\pgfqpoint{2.458337in}{2.649866in}}{\pgfqpoint{2.468936in}{2.654256in}}{\pgfqpoint{2.476749in}{2.662070in}}%
\pgfpathcurveto{\pgfqpoint{2.484563in}{2.669883in}}{\pgfqpoint{2.488953in}{2.680483in}}{\pgfqpoint{2.488953in}{2.691533in}}%
\pgfpathcurveto{\pgfqpoint{2.488953in}{2.702583in}}{\pgfqpoint{2.484563in}{2.713182in}}{\pgfqpoint{2.476749in}{2.720995in}}%
\pgfpathcurveto{\pgfqpoint{2.468936in}{2.728809in}}{\pgfqpoint{2.458337in}{2.733199in}}{\pgfqpoint{2.447287in}{2.733199in}}%
\pgfpathcurveto{\pgfqpoint{2.436236in}{2.733199in}}{\pgfqpoint{2.425637in}{2.728809in}}{\pgfqpoint{2.417824in}{2.720995in}}%
\pgfpathcurveto{\pgfqpoint{2.410010in}{2.713182in}}{\pgfqpoint{2.405620in}{2.702583in}}{\pgfqpoint{2.405620in}{2.691533in}}%
\pgfpathcurveto{\pgfqpoint{2.405620in}{2.680483in}}{\pgfqpoint{2.410010in}{2.669883in}}{\pgfqpoint{2.417824in}{2.662070in}}%
\pgfpathcurveto{\pgfqpoint{2.425637in}{2.654256in}}{\pgfqpoint{2.436236in}{2.649866in}}{\pgfqpoint{2.447287in}{2.649866in}}%
\pgfpathclose%
\pgfusepath{stroke,fill}%
\end{pgfscope}%
\begin{pgfscope}%
\pgfpathrectangle{\pgfqpoint{0.600000in}{0.600000in}}{\pgfqpoint{3.900000in}{3.900000in}}%
\pgfusepath{clip}%
\pgfsetbuttcap%
\pgfsetroundjoin%
\definecolor{currentfill}{rgb}{0.121569,0.466667,0.705882}%
\pgfsetfillcolor{currentfill}%
\pgfsetlinewidth{1.003750pt}%
\definecolor{currentstroke}{rgb}{0.121569,0.466667,0.705882}%
\pgfsetstrokecolor{currentstroke}%
\pgfsetdash{}{0pt}%
\pgfpathmoveto{\pgfqpoint{3.273313in}{2.923333in}}%
\pgfpathcurveto{\pgfqpoint{3.284363in}{2.923333in}}{\pgfqpoint{3.294962in}{2.927724in}}{\pgfqpoint{3.302776in}{2.935537in}}%
\pgfpathcurveto{\pgfqpoint{3.310590in}{2.943351in}}{\pgfqpoint{3.314980in}{2.953950in}}{\pgfqpoint{3.314980in}{2.965000in}}%
\pgfpathcurveto{\pgfqpoint{3.314980in}{2.976050in}}{\pgfqpoint{3.310590in}{2.986649in}}{\pgfqpoint{3.302776in}{2.994463in}}%
\pgfpathcurveto{\pgfqpoint{3.294962in}{3.002276in}}{\pgfqpoint{3.284363in}{3.006667in}}{\pgfqpoint{3.273313in}{3.006667in}}%
\pgfpathcurveto{\pgfqpoint{3.262263in}{3.006667in}}{\pgfqpoint{3.251664in}{3.002276in}}{\pgfqpoint{3.243850in}{2.994463in}}%
\pgfpathcurveto{\pgfqpoint{3.236037in}{2.986649in}}{\pgfqpoint{3.231647in}{2.976050in}}{\pgfqpoint{3.231647in}{2.965000in}}%
\pgfpathcurveto{\pgfqpoint{3.231647in}{2.953950in}}{\pgfqpoint{3.236037in}{2.943351in}}{\pgfqpoint{3.243850in}{2.935537in}}%
\pgfpathcurveto{\pgfqpoint{3.251664in}{2.927724in}}{\pgfqpoint{3.262263in}{2.923333in}}{\pgfqpoint{3.273313in}{2.923333in}}%
\pgfpathclose%
\pgfusepath{stroke,fill}%
\end{pgfscope}%
\begin{pgfscope}%
\pgfpathrectangle{\pgfqpoint{0.600000in}{0.600000in}}{\pgfqpoint{3.900000in}{3.900000in}}%
\pgfusepath{clip}%
\pgfsetbuttcap%
\pgfsetroundjoin%
\definecolor{currentfill}{rgb}{0.121569,0.466667,0.705882}%
\pgfsetfillcolor{currentfill}%
\pgfsetlinewidth{1.003750pt}%
\definecolor{currentstroke}{rgb}{0.121569,0.466667,0.705882}%
\pgfsetstrokecolor{currentstroke}%
\pgfsetdash{}{0pt}%
\pgfpathmoveto{\pgfqpoint{3.067576in}{2.099100in}}%
\pgfpathcurveto{\pgfqpoint{3.078626in}{2.099100in}}{\pgfqpoint{3.089225in}{2.103490in}}{\pgfqpoint{3.097039in}{2.111304in}}%
\pgfpathcurveto{\pgfqpoint{3.104852in}{2.119117in}}{\pgfqpoint{3.109243in}{2.129716in}}{\pgfqpoint{3.109243in}{2.140766in}}%
\pgfpathcurveto{\pgfqpoint{3.109243in}{2.151816in}}{\pgfqpoint{3.104852in}{2.162415in}}{\pgfqpoint{3.097039in}{2.170229in}}%
\pgfpathcurveto{\pgfqpoint{3.089225in}{2.178043in}}{\pgfqpoint{3.078626in}{2.182433in}}{\pgfqpoint{3.067576in}{2.182433in}}%
\pgfpathcurveto{\pgfqpoint{3.056526in}{2.182433in}}{\pgfqpoint{3.045927in}{2.178043in}}{\pgfqpoint{3.038113in}{2.170229in}}%
\pgfpathcurveto{\pgfqpoint{3.030300in}{2.162415in}}{\pgfqpoint{3.025909in}{2.151816in}}{\pgfqpoint{3.025909in}{2.140766in}}%
\pgfpathcurveto{\pgfqpoint{3.025909in}{2.129716in}}{\pgfqpoint{3.030300in}{2.119117in}}{\pgfqpoint{3.038113in}{2.111304in}}%
\pgfpathcurveto{\pgfqpoint{3.045927in}{2.103490in}}{\pgfqpoint{3.056526in}{2.099100in}}{\pgfqpoint{3.067576in}{2.099100in}}%
\pgfpathclose%
\pgfusepath{stroke,fill}%
\end{pgfscope}%
\begin{pgfscope}%
\pgfpathrectangle{\pgfqpoint{0.600000in}{0.600000in}}{\pgfqpoint{3.900000in}{3.900000in}}%
\pgfusepath{clip}%
\pgfsetbuttcap%
\pgfsetroundjoin%
\definecolor{currentfill}{rgb}{0.121569,0.466667,0.705882}%
\pgfsetfillcolor{currentfill}%
\pgfsetlinewidth{1.003750pt}%
\definecolor{currentstroke}{rgb}{0.121569,0.466667,0.705882}%
\pgfsetstrokecolor{currentstroke}%
\pgfsetdash{}{0pt}%
\pgfpathmoveto{\pgfqpoint{1.982727in}{2.860298in}}%
\pgfpathcurveto{\pgfqpoint{1.993777in}{2.860298in}}{\pgfqpoint{2.004376in}{2.864688in}}{\pgfqpoint{2.012190in}{2.872502in}}%
\pgfpathcurveto{\pgfqpoint{2.020004in}{2.880316in}}{\pgfqpoint{2.024394in}{2.890915in}}{\pgfqpoint{2.024394in}{2.901965in}}%
\pgfpathcurveto{\pgfqpoint{2.024394in}{2.913015in}}{\pgfqpoint{2.020004in}{2.923614in}}{\pgfqpoint{2.012190in}{2.931428in}}%
\pgfpathcurveto{\pgfqpoint{2.004376in}{2.939241in}}{\pgfqpoint{1.993777in}{2.943631in}}{\pgfqpoint{1.982727in}{2.943631in}}%
\pgfpathcurveto{\pgfqpoint{1.971677in}{2.943631in}}{\pgfqpoint{1.961078in}{2.939241in}}{\pgfqpoint{1.953264in}{2.931428in}}%
\pgfpathcurveto{\pgfqpoint{1.945451in}{2.923614in}}{\pgfqpoint{1.941061in}{2.913015in}}{\pgfqpoint{1.941061in}{2.901965in}}%
\pgfpathcurveto{\pgfqpoint{1.941061in}{2.890915in}}{\pgfqpoint{1.945451in}{2.880316in}}{\pgfqpoint{1.953264in}{2.872502in}}%
\pgfpathcurveto{\pgfqpoint{1.961078in}{2.864688in}}{\pgfqpoint{1.971677in}{2.860298in}}{\pgfqpoint{1.982727in}{2.860298in}}%
\pgfpathclose%
\pgfusepath{stroke,fill}%
\end{pgfscope}%
\begin{pgfscope}%
\pgfpathrectangle{\pgfqpoint{0.600000in}{0.600000in}}{\pgfqpoint{3.900000in}{3.900000in}}%
\pgfusepath{clip}%
\pgfsetbuttcap%
\pgfsetroundjoin%
\definecolor{currentfill}{rgb}{0.121569,0.466667,0.705882}%
\pgfsetfillcolor{currentfill}%
\pgfsetlinewidth{1.003750pt}%
\definecolor{currentstroke}{rgb}{0.121569,0.466667,0.705882}%
\pgfsetstrokecolor{currentstroke}%
\pgfsetdash{}{0pt}%
\pgfpathmoveto{\pgfqpoint{3.047765in}{1.706536in}}%
\pgfpathcurveto{\pgfqpoint{3.058815in}{1.706536in}}{\pgfqpoint{3.069414in}{1.710927in}}{\pgfqpoint{3.077228in}{1.718740in}}%
\pgfpathcurveto{\pgfqpoint{3.085042in}{1.726554in}}{\pgfqpoint{3.089432in}{1.737153in}}{\pgfqpoint{3.089432in}{1.748203in}}%
\pgfpathcurveto{\pgfqpoint{3.089432in}{1.759253in}}{\pgfqpoint{3.085042in}{1.769852in}}{\pgfqpoint{3.077228in}{1.777666in}}%
\pgfpathcurveto{\pgfqpoint{3.069414in}{1.785479in}}{\pgfqpoint{3.058815in}{1.789870in}}{\pgfqpoint{3.047765in}{1.789870in}}%
\pgfpathcurveto{\pgfqpoint{3.036715in}{1.789870in}}{\pgfqpoint{3.026116in}{1.785479in}}{\pgfqpoint{3.018302in}{1.777666in}}%
\pgfpathcurveto{\pgfqpoint{3.010489in}{1.769852in}}{\pgfqpoint{3.006099in}{1.759253in}}{\pgfqpoint{3.006099in}{1.748203in}}%
\pgfpathcurveto{\pgfqpoint{3.006099in}{1.737153in}}{\pgfqpoint{3.010489in}{1.726554in}}{\pgfqpoint{3.018302in}{1.718740in}}%
\pgfpathcurveto{\pgfqpoint{3.026116in}{1.710927in}}{\pgfqpoint{3.036715in}{1.706536in}}{\pgfqpoint{3.047765in}{1.706536in}}%
\pgfpathclose%
\pgfusepath{stroke,fill}%
\end{pgfscope}%
\begin{pgfscope}%
\pgfpathrectangle{\pgfqpoint{0.600000in}{0.600000in}}{\pgfqpoint{3.900000in}{3.900000in}}%
\pgfusepath{clip}%
\pgfsetbuttcap%
\pgfsetroundjoin%
\definecolor{currentfill}{rgb}{0.121569,0.466667,0.705882}%
\pgfsetfillcolor{currentfill}%
\pgfsetlinewidth{1.003750pt}%
\definecolor{currentstroke}{rgb}{0.121569,0.466667,0.705882}%
\pgfsetstrokecolor{currentstroke}%
\pgfsetdash{}{0pt}%
\pgfpathmoveto{\pgfqpoint{1.639921in}{2.490344in}}%
\pgfpathcurveto{\pgfqpoint{1.650971in}{2.490344in}}{\pgfqpoint{1.661570in}{2.494734in}}{\pgfqpoint{1.669384in}{2.502548in}}%
\pgfpathcurveto{\pgfqpoint{1.677198in}{2.510361in}}{\pgfqpoint{1.681588in}{2.520960in}}{\pgfqpoint{1.681588in}{2.532010in}}%
\pgfpathcurveto{\pgfqpoint{1.681588in}{2.543060in}}{\pgfqpoint{1.677198in}{2.553659in}}{\pgfqpoint{1.669384in}{2.561473in}}%
\pgfpathcurveto{\pgfqpoint{1.661570in}{2.569287in}}{\pgfqpoint{1.650971in}{2.573677in}}{\pgfqpoint{1.639921in}{2.573677in}}%
\pgfpathcurveto{\pgfqpoint{1.628871in}{2.573677in}}{\pgfqpoint{1.618272in}{2.569287in}}{\pgfqpoint{1.610459in}{2.561473in}}%
\pgfpathcurveto{\pgfqpoint{1.602645in}{2.553659in}}{\pgfqpoint{1.598255in}{2.543060in}}{\pgfqpoint{1.598255in}{2.532010in}}%
\pgfpathcurveto{\pgfqpoint{1.598255in}{2.520960in}}{\pgfqpoint{1.602645in}{2.510361in}}{\pgfqpoint{1.610459in}{2.502548in}}%
\pgfpathcurveto{\pgfqpoint{1.618272in}{2.494734in}}{\pgfqpoint{1.628871in}{2.490344in}}{\pgfqpoint{1.639921in}{2.490344in}}%
\pgfpathclose%
\pgfusepath{stroke,fill}%
\end{pgfscope}%
\begin{pgfscope}%
\pgfpathrectangle{\pgfqpoint{0.600000in}{0.600000in}}{\pgfqpoint{3.900000in}{3.900000in}}%
\pgfusepath{clip}%
\pgfsetbuttcap%
\pgfsetroundjoin%
\definecolor{currentfill}{rgb}{0.121569,0.466667,0.705882}%
\pgfsetfillcolor{currentfill}%
\pgfsetlinewidth{1.003750pt}%
\definecolor{currentstroke}{rgb}{0.121569,0.466667,0.705882}%
\pgfsetstrokecolor{currentstroke}%
\pgfsetdash{}{0pt}%
\pgfpathmoveto{\pgfqpoint{1.792964in}{2.242396in}}%
\pgfpathcurveto{\pgfqpoint{1.804014in}{2.242396in}}{\pgfqpoint{1.814613in}{2.246786in}}{\pgfqpoint{1.822427in}{2.254600in}}%
\pgfpathcurveto{\pgfqpoint{1.830240in}{2.262413in}}{\pgfqpoint{1.834631in}{2.273012in}}{\pgfqpoint{1.834631in}{2.284062in}}%
\pgfpathcurveto{\pgfqpoint{1.834631in}{2.295112in}}{\pgfqpoint{1.830240in}{2.305711in}}{\pgfqpoint{1.822427in}{2.313525in}}%
\pgfpathcurveto{\pgfqpoint{1.814613in}{2.321339in}}{\pgfqpoint{1.804014in}{2.325729in}}{\pgfqpoint{1.792964in}{2.325729in}}%
\pgfpathcurveto{\pgfqpoint{1.781914in}{2.325729in}}{\pgfqpoint{1.771315in}{2.321339in}}{\pgfqpoint{1.763501in}{2.313525in}}%
\pgfpathcurveto{\pgfqpoint{1.755688in}{2.305711in}}{\pgfqpoint{1.751297in}{2.295112in}}{\pgfqpoint{1.751297in}{2.284062in}}%
\pgfpathcurveto{\pgfqpoint{1.751297in}{2.273012in}}{\pgfqpoint{1.755688in}{2.262413in}}{\pgfqpoint{1.763501in}{2.254600in}}%
\pgfpathcurveto{\pgfqpoint{1.771315in}{2.246786in}}{\pgfqpoint{1.781914in}{2.242396in}}{\pgfqpoint{1.792964in}{2.242396in}}%
\pgfpathclose%
\pgfusepath{stroke,fill}%
\end{pgfscope}%
\begin{pgfscope}%
\pgfpathrectangle{\pgfqpoint{0.600000in}{0.600000in}}{\pgfqpoint{3.900000in}{3.900000in}}%
\pgfusepath{clip}%
\pgfsetbuttcap%
\pgfsetroundjoin%
\definecolor{currentfill}{rgb}{0.121569,0.466667,0.705882}%
\pgfsetfillcolor{currentfill}%
\pgfsetlinewidth{1.003750pt}%
\definecolor{currentstroke}{rgb}{0.121569,0.466667,0.705882}%
\pgfsetstrokecolor{currentstroke}%
\pgfsetdash{}{0pt}%
\pgfpathmoveto{\pgfqpoint{2.650951in}{2.586963in}}%
\pgfpathcurveto{\pgfqpoint{2.662001in}{2.586963in}}{\pgfqpoint{2.672600in}{2.591353in}}{\pgfqpoint{2.680414in}{2.599167in}}%
\pgfpathcurveto{\pgfqpoint{2.688227in}{2.606980in}}{\pgfqpoint{2.692618in}{2.617579in}}{\pgfqpoint{2.692618in}{2.628630in}}%
\pgfpathcurveto{\pgfqpoint{2.692618in}{2.639680in}}{\pgfqpoint{2.688227in}{2.650279in}}{\pgfqpoint{2.680414in}{2.658092in}}%
\pgfpathcurveto{\pgfqpoint{2.672600in}{2.665906in}}{\pgfqpoint{2.662001in}{2.670296in}}{\pgfqpoint{2.650951in}{2.670296in}}%
\pgfpathcurveto{\pgfqpoint{2.639901in}{2.670296in}}{\pgfqpoint{2.629302in}{2.665906in}}{\pgfqpoint{2.621488in}{2.658092in}}%
\pgfpathcurveto{\pgfqpoint{2.613675in}{2.650279in}}{\pgfqpoint{2.609284in}{2.639680in}}{\pgfqpoint{2.609284in}{2.628630in}}%
\pgfpathcurveto{\pgfqpoint{2.609284in}{2.617579in}}{\pgfqpoint{2.613675in}{2.606980in}}{\pgfqpoint{2.621488in}{2.599167in}}%
\pgfpathcurveto{\pgfqpoint{2.629302in}{2.591353in}}{\pgfqpoint{2.639901in}{2.586963in}}{\pgfqpoint{2.650951in}{2.586963in}}%
\pgfpathclose%
\pgfusepath{stroke,fill}%
\end{pgfscope}%
\begin{pgfscope}%
\pgfpathrectangle{\pgfqpoint{0.600000in}{0.600000in}}{\pgfqpoint{3.900000in}{3.900000in}}%
\pgfusepath{clip}%
\pgfsetbuttcap%
\pgfsetroundjoin%
\definecolor{currentfill}{rgb}{0.121569,0.466667,0.705882}%
\pgfsetfillcolor{currentfill}%
\pgfsetlinewidth{1.003750pt}%
\definecolor{currentstroke}{rgb}{0.121569,0.466667,0.705882}%
\pgfsetstrokecolor{currentstroke}%
\pgfsetdash{}{0pt}%
\pgfpathmoveto{\pgfqpoint{2.791489in}{1.518335in}}%
\pgfpathcurveto{\pgfqpoint{2.802539in}{1.518335in}}{\pgfqpoint{2.813138in}{1.522726in}}{\pgfqpoint{2.820952in}{1.530539in}}%
\pgfpathcurveto{\pgfqpoint{2.828766in}{1.538353in}}{\pgfqpoint{2.833156in}{1.548952in}}{\pgfqpoint{2.833156in}{1.560002in}}%
\pgfpathcurveto{\pgfqpoint{2.833156in}{1.571052in}}{\pgfqpoint{2.828766in}{1.581651in}}{\pgfqpoint{2.820952in}{1.589465in}}%
\pgfpathcurveto{\pgfqpoint{2.813138in}{1.597279in}}{\pgfqpoint{2.802539in}{1.601669in}}{\pgfqpoint{2.791489in}{1.601669in}}%
\pgfpathcurveto{\pgfqpoint{2.780439in}{1.601669in}}{\pgfqpoint{2.769840in}{1.597279in}}{\pgfqpoint{2.762026in}{1.589465in}}%
\pgfpathcurveto{\pgfqpoint{2.754213in}{1.581651in}}{\pgfqpoint{2.749823in}{1.571052in}}{\pgfqpoint{2.749823in}{1.560002in}}%
\pgfpathcurveto{\pgfqpoint{2.749823in}{1.548952in}}{\pgfqpoint{2.754213in}{1.538353in}}{\pgfqpoint{2.762026in}{1.530539in}}%
\pgfpathcurveto{\pgfqpoint{2.769840in}{1.522726in}}{\pgfqpoint{2.780439in}{1.518335in}}{\pgfqpoint{2.791489in}{1.518335in}}%
\pgfpathclose%
\pgfusepath{stroke,fill}%
\end{pgfscope}%
\begin{pgfscope}%
\pgfpathrectangle{\pgfqpoint{0.600000in}{0.600000in}}{\pgfqpoint{3.900000in}{3.900000in}}%
\pgfusepath{clip}%
\pgfsetbuttcap%
\pgfsetroundjoin%
\definecolor{currentfill}{rgb}{0.121569,0.466667,0.705882}%
\pgfsetfillcolor{currentfill}%
\pgfsetlinewidth{1.003750pt}%
\definecolor{currentstroke}{rgb}{0.121569,0.466667,0.705882}%
\pgfsetstrokecolor{currentstroke}%
\pgfsetdash{}{0pt}%
\pgfpathmoveto{\pgfqpoint{2.083167in}{2.418967in}}%
\pgfpathcurveto{\pgfqpoint{2.094217in}{2.418967in}}{\pgfqpoint{2.104817in}{2.423357in}}{\pgfqpoint{2.112630in}{2.431171in}}%
\pgfpathcurveto{\pgfqpoint{2.120444in}{2.438985in}}{\pgfqpoint{2.124834in}{2.449584in}}{\pgfqpoint{2.124834in}{2.460634in}}%
\pgfpathcurveto{\pgfqpoint{2.124834in}{2.471684in}}{\pgfqpoint{2.120444in}{2.482283in}}{\pgfqpoint{2.112630in}{2.490097in}}%
\pgfpathcurveto{\pgfqpoint{2.104817in}{2.497910in}}{\pgfqpoint{2.094217in}{2.502301in}}{\pgfqpoint{2.083167in}{2.502301in}}%
\pgfpathcurveto{\pgfqpoint{2.072117in}{2.502301in}}{\pgfqpoint{2.061518in}{2.497910in}}{\pgfqpoint{2.053705in}{2.490097in}}%
\pgfpathcurveto{\pgfqpoint{2.045891in}{2.482283in}}{\pgfqpoint{2.041501in}{2.471684in}}{\pgfqpoint{2.041501in}{2.460634in}}%
\pgfpathcurveto{\pgfqpoint{2.041501in}{2.449584in}}{\pgfqpoint{2.045891in}{2.438985in}}{\pgfqpoint{2.053705in}{2.431171in}}%
\pgfpathcurveto{\pgfqpoint{2.061518in}{2.423357in}}{\pgfqpoint{2.072117in}{2.418967in}}{\pgfqpoint{2.083167in}{2.418967in}}%
\pgfpathclose%
\pgfusepath{stroke,fill}%
\end{pgfscope}%
\begin{pgfscope}%
\pgfpathrectangle{\pgfqpoint{0.600000in}{0.600000in}}{\pgfqpoint{3.900000in}{3.900000in}}%
\pgfusepath{clip}%
\pgfsetbuttcap%
\pgfsetroundjoin%
\definecolor{currentfill}{rgb}{0.121569,0.466667,0.705882}%
\pgfsetfillcolor{currentfill}%
\pgfsetlinewidth{1.003750pt}%
\definecolor{currentstroke}{rgb}{0.121569,0.466667,0.705882}%
\pgfsetstrokecolor{currentstroke}%
\pgfsetdash{}{0pt}%
\pgfpathmoveto{\pgfqpoint{2.639873in}{2.676931in}}%
\pgfpathcurveto{\pgfqpoint{2.650923in}{2.676931in}}{\pgfqpoint{2.661522in}{2.681321in}}{\pgfqpoint{2.669336in}{2.689134in}}%
\pgfpathcurveto{\pgfqpoint{2.677149in}{2.696948in}}{\pgfqpoint{2.681540in}{2.707547in}}{\pgfqpoint{2.681540in}{2.718597in}}%
\pgfpathcurveto{\pgfqpoint{2.681540in}{2.729647in}}{\pgfqpoint{2.677149in}{2.740246in}}{\pgfqpoint{2.669336in}{2.748060in}}%
\pgfpathcurveto{\pgfqpoint{2.661522in}{2.755874in}}{\pgfqpoint{2.650923in}{2.760264in}}{\pgfqpoint{2.639873in}{2.760264in}}%
\pgfpathcurveto{\pgfqpoint{2.628823in}{2.760264in}}{\pgfqpoint{2.618224in}{2.755874in}}{\pgfqpoint{2.610410in}{2.748060in}}%
\pgfpathcurveto{\pgfqpoint{2.602597in}{2.740246in}}{\pgfqpoint{2.598206in}{2.729647in}}{\pgfqpoint{2.598206in}{2.718597in}}%
\pgfpathcurveto{\pgfqpoint{2.598206in}{2.707547in}}{\pgfqpoint{2.602597in}{2.696948in}}{\pgfqpoint{2.610410in}{2.689134in}}%
\pgfpathcurveto{\pgfqpoint{2.618224in}{2.681321in}}{\pgfqpoint{2.628823in}{2.676931in}}{\pgfqpoint{2.639873in}{2.676931in}}%
\pgfpathclose%
\pgfusepath{stroke,fill}%
\end{pgfscope}%
\begin{pgfscope}%
\pgfpathrectangle{\pgfqpoint{0.600000in}{0.600000in}}{\pgfqpoint{3.900000in}{3.900000in}}%
\pgfusepath{clip}%
\pgfsetbuttcap%
\pgfsetroundjoin%
\definecolor{currentfill}{rgb}{0.121569,0.466667,0.705882}%
\pgfsetfillcolor{currentfill}%
\pgfsetlinewidth{1.003750pt}%
\definecolor{currentstroke}{rgb}{0.121569,0.466667,0.705882}%
\pgfsetstrokecolor{currentstroke}%
\pgfsetdash{}{0pt}%
\pgfpathmoveto{\pgfqpoint{1.409309in}{1.319434in}}%
\pgfpathcurveto{\pgfqpoint{1.420359in}{1.319434in}}{\pgfqpoint{1.430958in}{1.323824in}}{\pgfqpoint{1.438771in}{1.331638in}}%
\pgfpathcurveto{\pgfqpoint{1.446585in}{1.339451in}}{\pgfqpoint{1.450975in}{1.350050in}}{\pgfqpoint{1.450975in}{1.361100in}}%
\pgfpathcurveto{\pgfqpoint{1.450975in}{1.372150in}}{\pgfqpoint{1.446585in}{1.382749in}}{\pgfqpoint{1.438771in}{1.390563in}}%
\pgfpathcurveto{\pgfqpoint{1.430958in}{1.398377in}}{\pgfqpoint{1.420359in}{1.402767in}}{\pgfqpoint{1.409309in}{1.402767in}}%
\pgfpathcurveto{\pgfqpoint{1.398259in}{1.402767in}}{\pgfqpoint{1.387660in}{1.398377in}}{\pgfqpoint{1.379846in}{1.390563in}}%
\pgfpathcurveto{\pgfqpoint{1.372032in}{1.382749in}}{\pgfqpoint{1.367642in}{1.372150in}}{\pgfqpoint{1.367642in}{1.361100in}}%
\pgfpathcurveto{\pgfqpoint{1.367642in}{1.350050in}}{\pgfqpoint{1.372032in}{1.339451in}}{\pgfqpoint{1.379846in}{1.331638in}}%
\pgfpathcurveto{\pgfqpoint{1.387660in}{1.323824in}}{\pgfqpoint{1.398259in}{1.319434in}}{\pgfqpoint{1.409309in}{1.319434in}}%
\pgfpathclose%
\pgfusepath{stroke,fill}%
\end{pgfscope}%
\begin{pgfscope}%
\pgfpathrectangle{\pgfqpoint{0.600000in}{0.600000in}}{\pgfqpoint{3.900000in}{3.900000in}}%
\pgfusepath{clip}%
\pgfsetbuttcap%
\pgfsetroundjoin%
\definecolor{currentfill}{rgb}{0.121569,0.466667,0.705882}%
\pgfsetfillcolor{currentfill}%
\pgfsetlinewidth{1.003750pt}%
\definecolor{currentstroke}{rgb}{0.121569,0.466667,0.705882}%
\pgfsetstrokecolor{currentstroke}%
\pgfsetdash{}{0pt}%
\pgfpathmoveto{\pgfqpoint{1.763635in}{2.004181in}}%
\pgfpathcurveto{\pgfqpoint{1.774686in}{2.004181in}}{\pgfqpoint{1.785285in}{2.008571in}}{\pgfqpoint{1.793098in}{2.016385in}}%
\pgfpathcurveto{\pgfqpoint{1.800912in}{2.024199in}}{\pgfqpoint{1.805302in}{2.034798in}}{\pgfqpoint{1.805302in}{2.045848in}}%
\pgfpathcurveto{\pgfqpoint{1.805302in}{2.056898in}}{\pgfqpoint{1.800912in}{2.067497in}}{\pgfqpoint{1.793098in}{2.075311in}}%
\pgfpathcurveto{\pgfqpoint{1.785285in}{2.083124in}}{\pgfqpoint{1.774686in}{2.087515in}}{\pgfqpoint{1.763635in}{2.087515in}}%
\pgfpathcurveto{\pgfqpoint{1.752585in}{2.087515in}}{\pgfqpoint{1.741986in}{2.083124in}}{\pgfqpoint{1.734173in}{2.075311in}}%
\pgfpathcurveto{\pgfqpoint{1.726359in}{2.067497in}}{\pgfqpoint{1.721969in}{2.056898in}}{\pgfqpoint{1.721969in}{2.045848in}}%
\pgfpathcurveto{\pgfqpoint{1.721969in}{2.034798in}}{\pgfqpoint{1.726359in}{2.024199in}}{\pgfqpoint{1.734173in}{2.016385in}}%
\pgfpathcurveto{\pgfqpoint{1.741986in}{2.008571in}}{\pgfqpoint{1.752585in}{2.004181in}}{\pgfqpoint{1.763635in}{2.004181in}}%
\pgfpathclose%
\pgfusepath{stroke,fill}%
\end{pgfscope}%
\begin{pgfscope}%
\pgfpathrectangle{\pgfqpoint{0.600000in}{0.600000in}}{\pgfqpoint{3.900000in}{3.900000in}}%
\pgfusepath{clip}%
\pgfsetbuttcap%
\pgfsetroundjoin%
\definecolor{currentfill}{rgb}{0.121569,0.466667,0.705882}%
\pgfsetfillcolor{currentfill}%
\pgfsetlinewidth{1.003750pt}%
\definecolor{currentstroke}{rgb}{0.121569,0.466667,0.705882}%
\pgfsetstrokecolor{currentstroke}%
\pgfsetdash{}{0pt}%
\pgfpathmoveto{\pgfqpoint{1.787556in}{1.935280in}}%
\pgfpathcurveto{\pgfqpoint{1.798606in}{1.935280in}}{\pgfqpoint{1.809205in}{1.939671in}}{\pgfqpoint{1.817019in}{1.947484in}}%
\pgfpathcurveto{\pgfqpoint{1.824833in}{1.955298in}}{\pgfqpoint{1.829223in}{1.965897in}}{\pgfqpoint{1.829223in}{1.976947in}}%
\pgfpathcurveto{\pgfqpoint{1.829223in}{1.987997in}}{\pgfqpoint{1.824833in}{1.998596in}}{\pgfqpoint{1.817019in}{2.006410in}}%
\pgfpathcurveto{\pgfqpoint{1.809205in}{2.014223in}}{\pgfqpoint{1.798606in}{2.018614in}}{\pgfqpoint{1.787556in}{2.018614in}}%
\pgfpathcurveto{\pgfqpoint{1.776506in}{2.018614in}}{\pgfqpoint{1.765907in}{2.014223in}}{\pgfqpoint{1.758093in}{2.006410in}}%
\pgfpathcurveto{\pgfqpoint{1.750280in}{1.998596in}}{\pgfqpoint{1.745889in}{1.987997in}}{\pgfqpoint{1.745889in}{1.976947in}}%
\pgfpathcurveto{\pgfqpoint{1.745889in}{1.965897in}}{\pgfqpoint{1.750280in}{1.955298in}}{\pgfqpoint{1.758093in}{1.947484in}}%
\pgfpathcurveto{\pgfqpoint{1.765907in}{1.939671in}}{\pgfqpoint{1.776506in}{1.935280in}}{\pgfqpoint{1.787556in}{1.935280in}}%
\pgfpathclose%
\pgfusepath{stroke,fill}%
\end{pgfscope}%
\begin{pgfscope}%
\pgfpathrectangle{\pgfqpoint{0.600000in}{0.600000in}}{\pgfqpoint{3.900000in}{3.900000in}}%
\pgfusepath{clip}%
\pgfsetbuttcap%
\pgfsetroundjoin%
\definecolor{currentfill}{rgb}{0.121569,0.466667,0.705882}%
\pgfsetfillcolor{currentfill}%
\pgfsetlinewidth{1.003750pt}%
\definecolor{currentstroke}{rgb}{0.121569,0.466667,0.705882}%
\pgfsetstrokecolor{currentstroke}%
\pgfsetdash{}{0pt}%
\pgfpathmoveto{\pgfqpoint{1.877466in}{2.831831in}}%
\pgfpathcurveto{\pgfqpoint{1.888517in}{2.831831in}}{\pgfqpoint{1.899116in}{2.836222in}}{\pgfqpoint{1.906929in}{2.844035in}}%
\pgfpathcurveto{\pgfqpoint{1.914743in}{2.851849in}}{\pgfqpoint{1.919133in}{2.862448in}}{\pgfqpoint{1.919133in}{2.873498in}}%
\pgfpathcurveto{\pgfqpoint{1.919133in}{2.884548in}}{\pgfqpoint{1.914743in}{2.895147in}}{\pgfqpoint{1.906929in}{2.902961in}}%
\pgfpathcurveto{\pgfqpoint{1.899116in}{2.910775in}}{\pgfqpoint{1.888517in}{2.915165in}}{\pgfqpoint{1.877466in}{2.915165in}}%
\pgfpathcurveto{\pgfqpoint{1.866416in}{2.915165in}}{\pgfqpoint{1.855817in}{2.910775in}}{\pgfqpoint{1.848004in}{2.902961in}}%
\pgfpathcurveto{\pgfqpoint{1.840190in}{2.895147in}}{\pgfqpoint{1.835800in}{2.884548in}}{\pgfqpoint{1.835800in}{2.873498in}}%
\pgfpathcurveto{\pgfqpoint{1.835800in}{2.862448in}}{\pgfqpoint{1.840190in}{2.851849in}}{\pgfqpoint{1.848004in}{2.844035in}}%
\pgfpathcurveto{\pgfqpoint{1.855817in}{2.836222in}}{\pgfqpoint{1.866416in}{2.831831in}}{\pgfqpoint{1.877466in}{2.831831in}}%
\pgfpathclose%
\pgfusepath{stroke,fill}%
\end{pgfscope}%
\begin{pgfscope}%
\pgfpathrectangle{\pgfqpoint{0.600000in}{0.600000in}}{\pgfqpoint{3.900000in}{3.900000in}}%
\pgfusepath{clip}%
\pgfsetbuttcap%
\pgfsetroundjoin%
\definecolor{currentfill}{rgb}{0.121569,0.466667,0.705882}%
\pgfsetfillcolor{currentfill}%
\pgfsetlinewidth{1.003750pt}%
\definecolor{currentstroke}{rgb}{0.121569,0.466667,0.705882}%
\pgfsetstrokecolor{currentstroke}%
\pgfsetdash{}{0pt}%
\pgfpathmoveto{\pgfqpoint{2.068423in}{2.358224in}}%
\pgfpathcurveto{\pgfqpoint{2.079473in}{2.358224in}}{\pgfqpoint{2.090072in}{2.362614in}}{\pgfqpoint{2.097886in}{2.370428in}}%
\pgfpathcurveto{\pgfqpoint{2.105699in}{2.378241in}}{\pgfqpoint{2.110089in}{2.388840in}}{\pgfqpoint{2.110089in}{2.399890in}}%
\pgfpathcurveto{\pgfqpoint{2.110089in}{2.410941in}}{\pgfqpoint{2.105699in}{2.421540in}}{\pgfqpoint{2.097886in}{2.429353in}}%
\pgfpathcurveto{\pgfqpoint{2.090072in}{2.437167in}}{\pgfqpoint{2.079473in}{2.441557in}}{\pgfqpoint{2.068423in}{2.441557in}}%
\pgfpathcurveto{\pgfqpoint{2.057373in}{2.441557in}}{\pgfqpoint{2.046774in}{2.437167in}}{\pgfqpoint{2.038960in}{2.429353in}}%
\pgfpathcurveto{\pgfqpoint{2.031146in}{2.421540in}}{\pgfqpoint{2.026756in}{2.410941in}}{\pgfqpoint{2.026756in}{2.399890in}}%
\pgfpathcurveto{\pgfqpoint{2.026756in}{2.388840in}}{\pgfqpoint{2.031146in}{2.378241in}}{\pgfqpoint{2.038960in}{2.370428in}}%
\pgfpathcurveto{\pgfqpoint{2.046774in}{2.362614in}}{\pgfqpoint{2.057373in}{2.358224in}}{\pgfqpoint{2.068423in}{2.358224in}}%
\pgfpathclose%
\pgfusepath{stroke,fill}%
\end{pgfscope}%
\begin{pgfscope}%
\pgfpathrectangle{\pgfqpoint{0.600000in}{0.600000in}}{\pgfqpoint{3.900000in}{3.900000in}}%
\pgfusepath{clip}%
\pgfsetbuttcap%
\pgfsetroundjoin%
\definecolor{currentfill}{rgb}{0.121569,0.466667,0.705882}%
\pgfsetfillcolor{currentfill}%
\pgfsetlinewidth{1.003750pt}%
\definecolor{currentstroke}{rgb}{0.121569,0.466667,0.705882}%
\pgfsetstrokecolor{currentstroke}%
\pgfsetdash{}{0pt}%
\pgfpathmoveto{\pgfqpoint{2.341561in}{2.319118in}}%
\pgfpathcurveto{\pgfqpoint{2.352612in}{2.319118in}}{\pgfqpoint{2.363211in}{2.323508in}}{\pgfqpoint{2.371024in}{2.331322in}}%
\pgfpathcurveto{\pgfqpoint{2.378838in}{2.339135in}}{\pgfqpoint{2.383228in}{2.349734in}}{\pgfqpoint{2.383228in}{2.360784in}}%
\pgfpathcurveto{\pgfqpoint{2.383228in}{2.371834in}}{\pgfqpoint{2.378838in}{2.382433in}}{\pgfqpoint{2.371024in}{2.390247in}}%
\pgfpathcurveto{\pgfqpoint{2.363211in}{2.398061in}}{\pgfqpoint{2.352612in}{2.402451in}}{\pgfqpoint{2.341561in}{2.402451in}}%
\pgfpathcurveto{\pgfqpoint{2.330511in}{2.402451in}}{\pgfqpoint{2.319912in}{2.398061in}}{\pgfqpoint{2.312099in}{2.390247in}}%
\pgfpathcurveto{\pgfqpoint{2.304285in}{2.382433in}}{\pgfqpoint{2.299895in}{2.371834in}}{\pgfqpoint{2.299895in}{2.360784in}}%
\pgfpathcurveto{\pgfqpoint{2.299895in}{2.349734in}}{\pgfqpoint{2.304285in}{2.339135in}}{\pgfqpoint{2.312099in}{2.331322in}}%
\pgfpathcurveto{\pgfqpoint{2.319912in}{2.323508in}}{\pgfqpoint{2.330511in}{2.319118in}}{\pgfqpoint{2.341561in}{2.319118in}}%
\pgfpathclose%
\pgfusepath{stroke,fill}%
\end{pgfscope}%
\begin{pgfscope}%
\pgfpathrectangle{\pgfqpoint{0.600000in}{0.600000in}}{\pgfqpoint{3.900000in}{3.900000in}}%
\pgfusepath{clip}%
\pgfsetbuttcap%
\pgfsetroundjoin%
\definecolor{currentfill}{rgb}{0.121569,0.466667,0.705882}%
\pgfsetfillcolor{currentfill}%
\pgfsetlinewidth{1.003750pt}%
\definecolor{currentstroke}{rgb}{0.121569,0.466667,0.705882}%
\pgfsetstrokecolor{currentstroke}%
\pgfsetdash{}{0pt}%
\pgfpathmoveto{\pgfqpoint{2.615085in}{3.127823in}}%
\pgfpathcurveto{\pgfqpoint{2.626135in}{3.127823in}}{\pgfqpoint{2.636734in}{3.132213in}}{\pgfqpoint{2.644548in}{3.140027in}}%
\pgfpathcurveto{\pgfqpoint{2.652362in}{3.147840in}}{\pgfqpoint{2.656752in}{3.158439in}}{\pgfqpoint{2.656752in}{3.169489in}}%
\pgfpathcurveto{\pgfqpoint{2.656752in}{3.180540in}}{\pgfqpoint{2.652362in}{3.191139in}}{\pgfqpoint{2.644548in}{3.198952in}}%
\pgfpathcurveto{\pgfqpoint{2.636734in}{3.206766in}}{\pgfqpoint{2.626135in}{3.211156in}}{\pgfqpoint{2.615085in}{3.211156in}}%
\pgfpathcurveto{\pgfqpoint{2.604035in}{3.211156in}}{\pgfqpoint{2.593436in}{3.206766in}}{\pgfqpoint{2.585622in}{3.198952in}}%
\pgfpathcurveto{\pgfqpoint{2.577809in}{3.191139in}}{\pgfqpoint{2.573418in}{3.180540in}}{\pgfqpoint{2.573418in}{3.169489in}}%
\pgfpathcurveto{\pgfqpoint{2.573418in}{3.158439in}}{\pgfqpoint{2.577809in}{3.147840in}}{\pgfqpoint{2.585622in}{3.140027in}}%
\pgfpathcurveto{\pgfqpoint{2.593436in}{3.132213in}}{\pgfqpoint{2.604035in}{3.127823in}}{\pgfqpoint{2.615085in}{3.127823in}}%
\pgfpathclose%
\pgfusepath{stroke,fill}%
\end{pgfscope}%
\begin{pgfscope}%
\pgfpathrectangle{\pgfqpoint{0.600000in}{0.600000in}}{\pgfqpoint{3.900000in}{3.900000in}}%
\pgfusepath{clip}%
\pgfsetbuttcap%
\pgfsetroundjoin%
\definecolor{currentfill}{rgb}{0.121569,0.466667,0.705882}%
\pgfsetfillcolor{currentfill}%
\pgfsetlinewidth{1.003750pt}%
\definecolor{currentstroke}{rgb}{0.121569,0.466667,0.705882}%
\pgfsetstrokecolor{currentstroke}%
\pgfsetdash{}{0pt}%
\pgfpathmoveto{\pgfqpoint{3.738099in}{2.197888in}}%
\pgfpathcurveto{\pgfqpoint{3.749149in}{2.197888in}}{\pgfqpoint{3.759748in}{2.202278in}}{\pgfqpoint{3.767562in}{2.210092in}}%
\pgfpathcurveto{\pgfqpoint{3.775376in}{2.217906in}}{\pgfqpoint{3.779766in}{2.228505in}}{\pgfqpoint{3.779766in}{2.239555in}}%
\pgfpathcurveto{\pgfqpoint{3.779766in}{2.250605in}}{\pgfqpoint{3.775376in}{2.261204in}}{\pgfqpoint{3.767562in}{2.269017in}}%
\pgfpathcurveto{\pgfqpoint{3.759748in}{2.276831in}}{\pgfqpoint{3.749149in}{2.281221in}}{\pgfqpoint{3.738099in}{2.281221in}}%
\pgfpathcurveto{\pgfqpoint{3.727049in}{2.281221in}}{\pgfqpoint{3.716450in}{2.276831in}}{\pgfqpoint{3.708637in}{2.269017in}}%
\pgfpathcurveto{\pgfqpoint{3.700823in}{2.261204in}}{\pgfqpoint{3.696433in}{2.250605in}}{\pgfqpoint{3.696433in}{2.239555in}}%
\pgfpathcurveto{\pgfqpoint{3.696433in}{2.228505in}}{\pgfqpoint{3.700823in}{2.217906in}}{\pgfqpoint{3.708637in}{2.210092in}}%
\pgfpathcurveto{\pgfqpoint{3.716450in}{2.202278in}}{\pgfqpoint{3.727049in}{2.197888in}}{\pgfqpoint{3.738099in}{2.197888in}}%
\pgfpathclose%
\pgfusepath{stroke,fill}%
\end{pgfscope}%
\begin{pgfscope}%
\pgfpathrectangle{\pgfqpoint{0.600000in}{0.600000in}}{\pgfqpoint{3.900000in}{3.900000in}}%
\pgfusepath{clip}%
\pgfsetbuttcap%
\pgfsetroundjoin%
\definecolor{currentfill}{rgb}{0.121569,0.466667,0.705882}%
\pgfsetfillcolor{currentfill}%
\pgfsetlinewidth{1.003750pt}%
\definecolor{currentstroke}{rgb}{0.121569,0.466667,0.705882}%
\pgfsetstrokecolor{currentstroke}%
\pgfsetdash{}{0pt}%
\pgfpathmoveto{\pgfqpoint{3.339581in}{3.344044in}}%
\pgfpathcurveto{\pgfqpoint{3.350631in}{3.344044in}}{\pgfqpoint{3.361230in}{3.348435in}}{\pgfqpoint{3.369044in}{3.356248in}}%
\pgfpathcurveto{\pgfqpoint{3.376857in}{3.364062in}}{\pgfqpoint{3.381248in}{3.374661in}}{\pgfqpoint{3.381248in}{3.385711in}}%
\pgfpathcurveto{\pgfqpoint{3.381248in}{3.396761in}}{\pgfqpoint{3.376857in}{3.407360in}}{\pgfqpoint{3.369044in}{3.415174in}}%
\pgfpathcurveto{\pgfqpoint{3.361230in}{3.422987in}}{\pgfqpoint{3.350631in}{3.427378in}}{\pgfqpoint{3.339581in}{3.427378in}}%
\pgfpathcurveto{\pgfqpoint{3.328531in}{3.427378in}}{\pgfqpoint{3.317932in}{3.422987in}}{\pgfqpoint{3.310118in}{3.415174in}}%
\pgfpathcurveto{\pgfqpoint{3.302305in}{3.407360in}}{\pgfqpoint{3.297914in}{3.396761in}}{\pgfqpoint{3.297914in}{3.385711in}}%
\pgfpathcurveto{\pgfqpoint{3.297914in}{3.374661in}}{\pgfqpoint{3.302305in}{3.364062in}}{\pgfqpoint{3.310118in}{3.356248in}}%
\pgfpathcurveto{\pgfqpoint{3.317932in}{3.348435in}}{\pgfqpoint{3.328531in}{3.344044in}}{\pgfqpoint{3.339581in}{3.344044in}}%
\pgfpathclose%
\pgfusepath{stroke,fill}%
\end{pgfscope}%
\begin{pgfscope}%
\pgfpathrectangle{\pgfqpoint{0.600000in}{0.600000in}}{\pgfqpoint{3.900000in}{3.900000in}}%
\pgfusepath{clip}%
\pgfsetbuttcap%
\pgfsetroundjoin%
\definecolor{currentfill}{rgb}{0.121569,0.466667,0.705882}%
\pgfsetfillcolor{currentfill}%
\pgfsetlinewidth{1.003750pt}%
\definecolor{currentstroke}{rgb}{0.121569,0.466667,0.705882}%
\pgfsetstrokecolor{currentstroke}%
\pgfsetdash{}{0pt}%
\pgfpathmoveto{\pgfqpoint{2.110323in}{2.947308in}}%
\pgfpathcurveto{\pgfqpoint{2.121373in}{2.947308in}}{\pgfqpoint{2.131972in}{2.951698in}}{\pgfqpoint{2.139785in}{2.959512in}}%
\pgfpathcurveto{\pgfqpoint{2.147599in}{2.967325in}}{\pgfqpoint{2.151989in}{2.977924in}}{\pgfqpoint{2.151989in}{2.988975in}}%
\pgfpathcurveto{\pgfqpoint{2.151989in}{3.000025in}}{\pgfqpoint{2.147599in}{3.010624in}}{\pgfqpoint{2.139785in}{3.018437in}}%
\pgfpathcurveto{\pgfqpoint{2.131972in}{3.026251in}}{\pgfqpoint{2.121373in}{3.030641in}}{\pgfqpoint{2.110323in}{3.030641in}}%
\pgfpathcurveto{\pgfqpoint{2.099272in}{3.030641in}}{\pgfqpoint{2.088673in}{3.026251in}}{\pgfqpoint{2.080860in}{3.018437in}}%
\pgfpathcurveto{\pgfqpoint{2.073046in}{3.010624in}}{\pgfqpoint{2.068656in}{3.000025in}}{\pgfqpoint{2.068656in}{2.988975in}}%
\pgfpathcurveto{\pgfqpoint{2.068656in}{2.977924in}}{\pgfqpoint{2.073046in}{2.967325in}}{\pgfqpoint{2.080860in}{2.959512in}}%
\pgfpathcurveto{\pgfqpoint{2.088673in}{2.951698in}}{\pgfqpoint{2.099272in}{2.947308in}}{\pgfqpoint{2.110323in}{2.947308in}}%
\pgfpathclose%
\pgfusepath{stroke,fill}%
\end{pgfscope}%
\begin{pgfscope}%
\pgfpathrectangle{\pgfqpoint{0.600000in}{0.600000in}}{\pgfqpoint{3.900000in}{3.900000in}}%
\pgfusepath{clip}%
\pgfsetbuttcap%
\pgfsetroundjoin%
\definecolor{currentfill}{rgb}{0.121569,0.466667,0.705882}%
\pgfsetfillcolor{currentfill}%
\pgfsetlinewidth{1.003750pt}%
\definecolor{currentstroke}{rgb}{0.121569,0.466667,0.705882}%
\pgfsetstrokecolor{currentstroke}%
\pgfsetdash{}{0pt}%
\pgfpathmoveto{\pgfqpoint{1.548442in}{2.602695in}}%
\pgfpathcurveto{\pgfqpoint{1.559492in}{2.602695in}}{\pgfqpoint{1.570091in}{2.607085in}}{\pgfqpoint{1.577905in}{2.614899in}}%
\pgfpathcurveto{\pgfqpoint{1.585718in}{2.622712in}}{\pgfqpoint{1.590109in}{2.633311in}}{\pgfqpoint{1.590109in}{2.644361in}}%
\pgfpathcurveto{\pgfqpoint{1.590109in}{2.655411in}}{\pgfqpoint{1.585718in}{2.666011in}}{\pgfqpoint{1.577905in}{2.673824in}}%
\pgfpathcurveto{\pgfqpoint{1.570091in}{2.681638in}}{\pgfqpoint{1.559492in}{2.686028in}}{\pgfqpoint{1.548442in}{2.686028in}}%
\pgfpathcurveto{\pgfqpoint{1.537392in}{2.686028in}}{\pgfqpoint{1.526793in}{2.681638in}}{\pgfqpoint{1.518979in}{2.673824in}}%
\pgfpathcurveto{\pgfqpoint{1.511165in}{2.666011in}}{\pgfqpoint{1.506775in}{2.655411in}}{\pgfqpoint{1.506775in}{2.644361in}}%
\pgfpathcurveto{\pgfqpoint{1.506775in}{2.633311in}}{\pgfqpoint{1.511165in}{2.622712in}}{\pgfqpoint{1.518979in}{2.614899in}}%
\pgfpathcurveto{\pgfqpoint{1.526793in}{2.607085in}}{\pgfqpoint{1.537392in}{2.602695in}}{\pgfqpoint{1.548442in}{2.602695in}}%
\pgfpathclose%
\pgfusepath{stroke,fill}%
\end{pgfscope}%
\begin{pgfscope}%
\pgfpathrectangle{\pgfqpoint{0.600000in}{0.600000in}}{\pgfqpoint{3.900000in}{3.900000in}}%
\pgfusepath{clip}%
\pgfsetbuttcap%
\pgfsetroundjoin%
\definecolor{currentfill}{rgb}{0.121569,0.466667,0.705882}%
\pgfsetfillcolor{currentfill}%
\pgfsetlinewidth{1.003750pt}%
\definecolor{currentstroke}{rgb}{0.121569,0.466667,0.705882}%
\pgfsetstrokecolor{currentstroke}%
\pgfsetdash{}{0pt}%
\pgfpathmoveto{\pgfqpoint{1.629314in}{3.029828in}}%
\pgfpathcurveto{\pgfqpoint{1.640365in}{3.029828in}}{\pgfqpoint{1.650964in}{3.034218in}}{\pgfqpoint{1.658777in}{3.042032in}}%
\pgfpathcurveto{\pgfqpoint{1.666591in}{3.049846in}}{\pgfqpoint{1.670981in}{3.060445in}}{\pgfqpoint{1.670981in}{3.071495in}}%
\pgfpathcurveto{\pgfqpoint{1.670981in}{3.082545in}}{\pgfqpoint{1.666591in}{3.093144in}}{\pgfqpoint{1.658777in}{3.100958in}}%
\pgfpathcurveto{\pgfqpoint{1.650964in}{3.108771in}}{\pgfqpoint{1.640365in}{3.113161in}}{\pgfqpoint{1.629314in}{3.113161in}}%
\pgfpathcurveto{\pgfqpoint{1.618264in}{3.113161in}}{\pgfqpoint{1.607665in}{3.108771in}}{\pgfqpoint{1.599852in}{3.100958in}}%
\pgfpathcurveto{\pgfqpoint{1.592038in}{3.093144in}}{\pgfqpoint{1.587648in}{3.082545in}}{\pgfqpoint{1.587648in}{3.071495in}}%
\pgfpathcurveto{\pgfqpoint{1.587648in}{3.060445in}}{\pgfqpoint{1.592038in}{3.049846in}}{\pgfqpoint{1.599852in}{3.042032in}}%
\pgfpathcurveto{\pgfqpoint{1.607665in}{3.034218in}}{\pgfqpoint{1.618264in}{3.029828in}}{\pgfqpoint{1.629314in}{3.029828in}}%
\pgfpathclose%
\pgfusepath{stroke,fill}%
\end{pgfscope}%
\begin{pgfscope}%
\pgfpathrectangle{\pgfqpoint{0.600000in}{0.600000in}}{\pgfqpoint{3.900000in}{3.900000in}}%
\pgfusepath{clip}%
\pgfsetbuttcap%
\pgfsetroundjoin%
\definecolor{currentfill}{rgb}{0.121569,0.466667,0.705882}%
\pgfsetfillcolor{currentfill}%
\pgfsetlinewidth{1.003750pt}%
\definecolor{currentstroke}{rgb}{0.121569,0.466667,0.705882}%
\pgfsetstrokecolor{currentstroke}%
\pgfsetdash{}{0pt}%
\pgfpathmoveto{\pgfqpoint{2.545102in}{2.564304in}}%
\pgfpathcurveto{\pgfqpoint{2.556152in}{2.564304in}}{\pgfqpoint{2.566751in}{2.568694in}}{\pgfqpoint{2.574564in}{2.576508in}}%
\pgfpathcurveto{\pgfqpoint{2.582378in}{2.584322in}}{\pgfqpoint{2.586768in}{2.594921in}}{\pgfqpoint{2.586768in}{2.605971in}}%
\pgfpathcurveto{\pgfqpoint{2.586768in}{2.617021in}}{\pgfqpoint{2.582378in}{2.627620in}}{\pgfqpoint{2.574564in}{2.635434in}}%
\pgfpathcurveto{\pgfqpoint{2.566751in}{2.643247in}}{\pgfqpoint{2.556152in}{2.647638in}}{\pgfqpoint{2.545102in}{2.647638in}}%
\pgfpathcurveto{\pgfqpoint{2.534052in}{2.647638in}}{\pgfqpoint{2.523453in}{2.643247in}}{\pgfqpoint{2.515639in}{2.635434in}}%
\pgfpathcurveto{\pgfqpoint{2.507825in}{2.627620in}}{\pgfqpoint{2.503435in}{2.617021in}}{\pgfqpoint{2.503435in}{2.605971in}}%
\pgfpathcurveto{\pgfqpoint{2.503435in}{2.594921in}}{\pgfqpoint{2.507825in}{2.584322in}}{\pgfqpoint{2.515639in}{2.576508in}}%
\pgfpathcurveto{\pgfqpoint{2.523453in}{2.568694in}}{\pgfqpoint{2.534052in}{2.564304in}}{\pgfqpoint{2.545102in}{2.564304in}}%
\pgfpathclose%
\pgfusepath{stroke,fill}%
\end{pgfscope}%
\begin{pgfscope}%
\pgfpathrectangle{\pgfqpoint{0.600000in}{0.600000in}}{\pgfqpoint{3.900000in}{3.900000in}}%
\pgfusepath{clip}%
\pgfsetbuttcap%
\pgfsetroundjoin%
\definecolor{currentfill}{rgb}{0.121569,0.466667,0.705882}%
\pgfsetfillcolor{currentfill}%
\pgfsetlinewidth{1.003750pt}%
\definecolor{currentstroke}{rgb}{0.121569,0.466667,0.705882}%
\pgfsetstrokecolor{currentstroke}%
\pgfsetdash{}{0pt}%
\pgfpathmoveto{\pgfqpoint{2.243484in}{1.964374in}}%
\pgfpathcurveto{\pgfqpoint{2.254534in}{1.964374in}}{\pgfqpoint{2.265133in}{1.968765in}}{\pgfqpoint{2.272947in}{1.976578in}}%
\pgfpathcurveto{\pgfqpoint{2.280761in}{1.984392in}}{\pgfqpoint{2.285151in}{1.994991in}}{\pgfqpoint{2.285151in}{2.006041in}}%
\pgfpathcurveto{\pgfqpoint{2.285151in}{2.017091in}}{\pgfqpoint{2.280761in}{2.027690in}}{\pgfqpoint{2.272947in}{2.035504in}}%
\pgfpathcurveto{\pgfqpoint{2.265133in}{2.043317in}}{\pgfqpoint{2.254534in}{2.047708in}}{\pgfqpoint{2.243484in}{2.047708in}}%
\pgfpathcurveto{\pgfqpoint{2.232434in}{2.047708in}}{\pgfqpoint{2.221835in}{2.043317in}}{\pgfqpoint{2.214021in}{2.035504in}}%
\pgfpathcurveto{\pgfqpoint{2.206208in}{2.027690in}}{\pgfqpoint{2.201818in}{2.017091in}}{\pgfqpoint{2.201818in}{2.006041in}}%
\pgfpathcurveto{\pgfqpoint{2.201818in}{1.994991in}}{\pgfqpoint{2.206208in}{1.984392in}}{\pgfqpoint{2.214021in}{1.976578in}}%
\pgfpathcurveto{\pgfqpoint{2.221835in}{1.968765in}}{\pgfqpoint{2.232434in}{1.964374in}}{\pgfqpoint{2.243484in}{1.964374in}}%
\pgfpathclose%
\pgfusepath{stroke,fill}%
\end{pgfscope}%
\begin{pgfscope}%
\pgfpathrectangle{\pgfqpoint{0.600000in}{0.600000in}}{\pgfqpoint{3.900000in}{3.900000in}}%
\pgfusepath{clip}%
\pgfsetbuttcap%
\pgfsetroundjoin%
\definecolor{currentfill}{rgb}{0.121569,0.466667,0.705882}%
\pgfsetfillcolor{currentfill}%
\pgfsetlinewidth{1.003750pt}%
\definecolor{currentstroke}{rgb}{0.121569,0.466667,0.705882}%
\pgfsetstrokecolor{currentstroke}%
\pgfsetdash{}{0pt}%
\pgfpathmoveto{\pgfqpoint{2.443028in}{1.859051in}}%
\pgfpathcurveto{\pgfqpoint{2.454078in}{1.859051in}}{\pgfqpoint{2.464677in}{1.863441in}}{\pgfqpoint{2.472490in}{1.871255in}}%
\pgfpathcurveto{\pgfqpoint{2.480304in}{1.879068in}}{\pgfqpoint{2.484694in}{1.889667in}}{\pgfqpoint{2.484694in}{1.900718in}}%
\pgfpathcurveto{\pgfqpoint{2.484694in}{1.911768in}}{\pgfqpoint{2.480304in}{1.922367in}}{\pgfqpoint{2.472490in}{1.930180in}}%
\pgfpathcurveto{\pgfqpoint{2.464677in}{1.937994in}}{\pgfqpoint{2.454078in}{1.942384in}}{\pgfqpoint{2.443028in}{1.942384in}}%
\pgfpathcurveto{\pgfqpoint{2.431977in}{1.942384in}}{\pgfqpoint{2.421378in}{1.937994in}}{\pgfqpoint{2.413565in}{1.930180in}}%
\pgfpathcurveto{\pgfqpoint{2.405751in}{1.922367in}}{\pgfqpoint{2.401361in}{1.911768in}}{\pgfqpoint{2.401361in}{1.900718in}}%
\pgfpathcurveto{\pgfqpoint{2.401361in}{1.889667in}}{\pgfqpoint{2.405751in}{1.879068in}}{\pgfqpoint{2.413565in}{1.871255in}}%
\pgfpathcurveto{\pgfqpoint{2.421378in}{1.863441in}}{\pgfqpoint{2.431977in}{1.859051in}}{\pgfqpoint{2.443028in}{1.859051in}}%
\pgfpathclose%
\pgfusepath{stroke,fill}%
\end{pgfscope}%
\begin{pgfscope}%
\pgfpathrectangle{\pgfqpoint{0.600000in}{0.600000in}}{\pgfqpoint{3.900000in}{3.900000in}}%
\pgfusepath{clip}%
\pgfsetbuttcap%
\pgfsetroundjoin%
\definecolor{currentfill}{rgb}{0.121569,0.466667,0.705882}%
\pgfsetfillcolor{currentfill}%
\pgfsetlinewidth{1.003750pt}%
\definecolor{currentstroke}{rgb}{0.121569,0.466667,0.705882}%
\pgfsetstrokecolor{currentstroke}%
\pgfsetdash{}{0pt}%
\pgfpathmoveto{\pgfqpoint{3.254127in}{1.869516in}}%
\pgfpathcurveto{\pgfqpoint{3.265177in}{1.869516in}}{\pgfqpoint{3.275776in}{1.873906in}}{\pgfqpoint{3.283589in}{1.881720in}}%
\pgfpathcurveto{\pgfqpoint{3.291403in}{1.889534in}}{\pgfqpoint{3.295793in}{1.900133in}}{\pgfqpoint{3.295793in}{1.911183in}}%
\pgfpathcurveto{\pgfqpoint{3.295793in}{1.922233in}}{\pgfqpoint{3.291403in}{1.932832in}}{\pgfqpoint{3.283589in}{1.940646in}}%
\pgfpathcurveto{\pgfqpoint{3.275776in}{1.948459in}}{\pgfqpoint{3.265177in}{1.952849in}}{\pgfqpoint{3.254127in}{1.952849in}}%
\pgfpathcurveto{\pgfqpoint{3.243077in}{1.952849in}}{\pgfqpoint{3.232478in}{1.948459in}}{\pgfqpoint{3.224664in}{1.940646in}}%
\pgfpathcurveto{\pgfqpoint{3.216850in}{1.932832in}}{\pgfqpoint{3.212460in}{1.922233in}}{\pgfqpoint{3.212460in}{1.911183in}}%
\pgfpathcurveto{\pgfqpoint{3.212460in}{1.900133in}}{\pgfqpoint{3.216850in}{1.889534in}}{\pgfqpoint{3.224664in}{1.881720in}}%
\pgfpathcurveto{\pgfqpoint{3.232478in}{1.873906in}}{\pgfqpoint{3.243077in}{1.869516in}}{\pgfqpoint{3.254127in}{1.869516in}}%
\pgfpathclose%
\pgfusepath{stroke,fill}%
\end{pgfscope}%
\begin{pgfscope}%
\pgfpathrectangle{\pgfqpoint{0.600000in}{0.600000in}}{\pgfqpoint{3.900000in}{3.900000in}}%
\pgfusepath{clip}%
\pgfsetbuttcap%
\pgfsetroundjoin%
\definecolor{currentfill}{rgb}{0.121569,0.466667,0.705882}%
\pgfsetfillcolor{currentfill}%
\pgfsetlinewidth{1.003750pt}%
\definecolor{currentstroke}{rgb}{0.121569,0.466667,0.705882}%
\pgfsetstrokecolor{currentstroke}%
\pgfsetdash{}{0pt}%
\pgfpathmoveto{\pgfqpoint{3.327961in}{2.641423in}}%
\pgfpathcurveto{\pgfqpoint{3.339011in}{2.641423in}}{\pgfqpoint{3.349610in}{2.645814in}}{\pgfqpoint{3.357424in}{2.653627in}}%
\pgfpathcurveto{\pgfqpoint{3.365237in}{2.661441in}}{\pgfqpoint{3.369628in}{2.672040in}}{\pgfqpoint{3.369628in}{2.683090in}}%
\pgfpathcurveto{\pgfqpoint{3.369628in}{2.694140in}}{\pgfqpoint{3.365237in}{2.704739in}}{\pgfqpoint{3.357424in}{2.712553in}}%
\pgfpathcurveto{\pgfqpoint{3.349610in}{2.720367in}}{\pgfqpoint{3.339011in}{2.724757in}}{\pgfqpoint{3.327961in}{2.724757in}}%
\pgfpathcurveto{\pgfqpoint{3.316911in}{2.724757in}}{\pgfqpoint{3.306312in}{2.720367in}}{\pgfqpoint{3.298498in}{2.712553in}}%
\pgfpathcurveto{\pgfqpoint{3.290684in}{2.704739in}}{\pgfqpoint{3.286294in}{2.694140in}}{\pgfqpoint{3.286294in}{2.683090in}}%
\pgfpathcurveto{\pgfqpoint{3.286294in}{2.672040in}}{\pgfqpoint{3.290684in}{2.661441in}}{\pgfqpoint{3.298498in}{2.653627in}}%
\pgfpathcurveto{\pgfqpoint{3.306312in}{2.645814in}}{\pgfqpoint{3.316911in}{2.641423in}}{\pgfqpoint{3.327961in}{2.641423in}}%
\pgfpathclose%
\pgfusepath{stroke,fill}%
\end{pgfscope}%
\begin{pgfscope}%
\pgfpathrectangle{\pgfqpoint{0.600000in}{0.600000in}}{\pgfqpoint{3.900000in}{3.900000in}}%
\pgfusepath{clip}%
\pgfsetbuttcap%
\pgfsetroundjoin%
\definecolor{currentfill}{rgb}{0.121569,0.466667,0.705882}%
\pgfsetfillcolor{currentfill}%
\pgfsetlinewidth{1.003750pt}%
\definecolor{currentstroke}{rgb}{0.121569,0.466667,0.705882}%
\pgfsetstrokecolor{currentstroke}%
\pgfsetdash{}{0pt}%
\pgfpathmoveto{\pgfqpoint{2.889025in}{2.770438in}}%
\pgfpathcurveto{\pgfqpoint{2.900075in}{2.770438in}}{\pgfqpoint{2.910674in}{2.774828in}}{\pgfqpoint{2.918488in}{2.782642in}}%
\pgfpathcurveto{\pgfqpoint{2.926302in}{2.790455in}}{\pgfqpoint{2.930692in}{2.801054in}}{\pgfqpoint{2.930692in}{2.812105in}}%
\pgfpathcurveto{\pgfqpoint{2.930692in}{2.823155in}}{\pgfqpoint{2.926302in}{2.833754in}}{\pgfqpoint{2.918488in}{2.841567in}}%
\pgfpathcurveto{\pgfqpoint{2.910674in}{2.849381in}}{\pgfqpoint{2.900075in}{2.853771in}}{\pgfqpoint{2.889025in}{2.853771in}}%
\pgfpathcurveto{\pgfqpoint{2.877975in}{2.853771in}}{\pgfqpoint{2.867376in}{2.849381in}}{\pgfqpoint{2.859562in}{2.841567in}}%
\pgfpathcurveto{\pgfqpoint{2.851749in}{2.833754in}}{\pgfqpoint{2.847359in}{2.823155in}}{\pgfqpoint{2.847359in}{2.812105in}}%
\pgfpathcurveto{\pgfqpoint{2.847359in}{2.801054in}}{\pgfqpoint{2.851749in}{2.790455in}}{\pgfqpoint{2.859562in}{2.782642in}}%
\pgfpathcurveto{\pgfqpoint{2.867376in}{2.774828in}}{\pgfqpoint{2.877975in}{2.770438in}}{\pgfqpoint{2.889025in}{2.770438in}}%
\pgfpathclose%
\pgfusepath{stroke,fill}%
\end{pgfscope}%
\begin{pgfscope}%
\pgfpathrectangle{\pgfqpoint{0.600000in}{0.600000in}}{\pgfqpoint{3.900000in}{3.900000in}}%
\pgfusepath{clip}%
\pgfsetbuttcap%
\pgfsetroundjoin%
\definecolor{currentfill}{rgb}{0.121569,0.466667,0.705882}%
\pgfsetfillcolor{currentfill}%
\pgfsetlinewidth{1.003750pt}%
\definecolor{currentstroke}{rgb}{0.121569,0.466667,0.705882}%
\pgfsetstrokecolor{currentstroke}%
\pgfsetdash{}{0pt}%
\pgfpathmoveto{\pgfqpoint{1.945449in}{2.410540in}}%
\pgfpathcurveto{\pgfqpoint{1.956499in}{2.410540in}}{\pgfqpoint{1.967098in}{2.414930in}}{\pgfqpoint{1.974912in}{2.422744in}}%
\pgfpathcurveto{\pgfqpoint{1.982725in}{2.430558in}}{\pgfqpoint{1.987115in}{2.441157in}}{\pgfqpoint{1.987115in}{2.452207in}}%
\pgfpathcurveto{\pgfqpoint{1.987115in}{2.463257in}}{\pgfqpoint{1.982725in}{2.473856in}}{\pgfqpoint{1.974912in}{2.481670in}}%
\pgfpathcurveto{\pgfqpoint{1.967098in}{2.489483in}}{\pgfqpoint{1.956499in}{2.493874in}}{\pgfqpoint{1.945449in}{2.493874in}}%
\pgfpathcurveto{\pgfqpoint{1.934399in}{2.493874in}}{\pgfqpoint{1.923800in}{2.489483in}}{\pgfqpoint{1.915986in}{2.481670in}}%
\pgfpathcurveto{\pgfqpoint{1.908172in}{2.473856in}}{\pgfqpoint{1.903782in}{2.463257in}}{\pgfqpoint{1.903782in}{2.452207in}}%
\pgfpathcurveto{\pgfqpoint{1.903782in}{2.441157in}}{\pgfqpoint{1.908172in}{2.430558in}}{\pgfqpoint{1.915986in}{2.422744in}}%
\pgfpathcurveto{\pgfqpoint{1.923800in}{2.414930in}}{\pgfqpoint{1.934399in}{2.410540in}}{\pgfqpoint{1.945449in}{2.410540in}}%
\pgfpathclose%
\pgfusepath{stroke,fill}%
\end{pgfscope}%
\begin{pgfscope}%
\pgfpathrectangle{\pgfqpoint{0.600000in}{0.600000in}}{\pgfqpoint{3.900000in}{3.900000in}}%
\pgfusepath{clip}%
\pgfsetbuttcap%
\pgfsetroundjoin%
\definecolor{currentfill}{rgb}{0.121569,0.466667,0.705882}%
\pgfsetfillcolor{currentfill}%
\pgfsetlinewidth{1.003750pt}%
\definecolor{currentstroke}{rgb}{0.121569,0.466667,0.705882}%
\pgfsetstrokecolor{currentstroke}%
\pgfsetdash{}{0pt}%
\pgfpathmoveto{\pgfqpoint{2.403243in}{2.887095in}}%
\pgfpathcurveto{\pgfqpoint{2.414294in}{2.887095in}}{\pgfqpoint{2.424893in}{2.891485in}}{\pgfqpoint{2.432706in}{2.899299in}}%
\pgfpathcurveto{\pgfqpoint{2.440520in}{2.907112in}}{\pgfqpoint{2.444910in}{2.917711in}}{\pgfqpoint{2.444910in}{2.928761in}}%
\pgfpathcurveto{\pgfqpoint{2.444910in}{2.939811in}}{\pgfqpoint{2.440520in}{2.950411in}}{\pgfqpoint{2.432706in}{2.958224in}}%
\pgfpathcurveto{\pgfqpoint{2.424893in}{2.966038in}}{\pgfqpoint{2.414294in}{2.970428in}}{\pgfqpoint{2.403243in}{2.970428in}}%
\pgfpathcurveto{\pgfqpoint{2.392193in}{2.970428in}}{\pgfqpoint{2.381594in}{2.966038in}}{\pgfqpoint{2.373781in}{2.958224in}}%
\pgfpathcurveto{\pgfqpoint{2.365967in}{2.950411in}}{\pgfqpoint{2.361577in}{2.939811in}}{\pgfqpoint{2.361577in}{2.928761in}}%
\pgfpathcurveto{\pgfqpoint{2.361577in}{2.917711in}}{\pgfqpoint{2.365967in}{2.907112in}}{\pgfqpoint{2.373781in}{2.899299in}}%
\pgfpathcurveto{\pgfqpoint{2.381594in}{2.891485in}}{\pgfqpoint{2.392193in}{2.887095in}}{\pgfqpoint{2.403243in}{2.887095in}}%
\pgfpathclose%
\pgfusepath{stroke,fill}%
\end{pgfscope}%
\begin{pgfscope}%
\pgfpathrectangle{\pgfqpoint{0.600000in}{0.600000in}}{\pgfqpoint{3.900000in}{3.900000in}}%
\pgfusepath{clip}%
\pgfsetbuttcap%
\pgfsetroundjoin%
\definecolor{currentfill}{rgb}{0.121569,0.466667,0.705882}%
\pgfsetfillcolor{currentfill}%
\pgfsetlinewidth{1.003750pt}%
\definecolor{currentstroke}{rgb}{0.121569,0.466667,0.705882}%
\pgfsetstrokecolor{currentstroke}%
\pgfsetdash{}{0pt}%
\pgfpathmoveto{\pgfqpoint{2.613328in}{2.107049in}}%
\pgfpathcurveto{\pgfqpoint{2.624378in}{2.107049in}}{\pgfqpoint{2.634977in}{2.111439in}}{\pgfqpoint{2.642791in}{2.119253in}}%
\pgfpathcurveto{\pgfqpoint{2.650604in}{2.127066in}}{\pgfqpoint{2.654994in}{2.137665in}}{\pgfqpoint{2.654994in}{2.148715in}}%
\pgfpathcurveto{\pgfqpoint{2.654994in}{2.159765in}}{\pgfqpoint{2.650604in}{2.170364in}}{\pgfqpoint{2.642791in}{2.178178in}}%
\pgfpathcurveto{\pgfqpoint{2.634977in}{2.185992in}}{\pgfqpoint{2.624378in}{2.190382in}}{\pgfqpoint{2.613328in}{2.190382in}}%
\pgfpathcurveto{\pgfqpoint{2.602278in}{2.190382in}}{\pgfqpoint{2.591679in}{2.185992in}}{\pgfqpoint{2.583865in}{2.178178in}}%
\pgfpathcurveto{\pgfqpoint{2.576051in}{2.170364in}}{\pgfqpoint{2.571661in}{2.159765in}}{\pgfqpoint{2.571661in}{2.148715in}}%
\pgfpathcurveto{\pgfqpoint{2.571661in}{2.137665in}}{\pgfqpoint{2.576051in}{2.127066in}}{\pgfqpoint{2.583865in}{2.119253in}}%
\pgfpathcurveto{\pgfqpoint{2.591679in}{2.111439in}}{\pgfqpoint{2.602278in}{2.107049in}}{\pgfqpoint{2.613328in}{2.107049in}}%
\pgfpathclose%
\pgfusepath{stroke,fill}%
\end{pgfscope}%
\begin{pgfscope}%
\pgfpathrectangle{\pgfqpoint{0.600000in}{0.600000in}}{\pgfqpoint{3.900000in}{3.900000in}}%
\pgfusepath{clip}%
\pgfsetbuttcap%
\pgfsetroundjoin%
\definecolor{currentfill}{rgb}{0.121569,0.466667,0.705882}%
\pgfsetfillcolor{currentfill}%
\pgfsetlinewidth{1.003750pt}%
\definecolor{currentstroke}{rgb}{0.121569,0.466667,0.705882}%
\pgfsetstrokecolor{currentstroke}%
\pgfsetdash{}{0pt}%
\pgfpathmoveto{\pgfqpoint{2.256291in}{2.130053in}}%
\pgfpathcurveto{\pgfqpoint{2.267341in}{2.130053in}}{\pgfqpoint{2.277940in}{2.134443in}}{\pgfqpoint{2.285754in}{2.142257in}}%
\pgfpathcurveto{\pgfqpoint{2.293568in}{2.150070in}}{\pgfqpoint{2.297958in}{2.160669in}}{\pgfqpoint{2.297958in}{2.171720in}}%
\pgfpathcurveto{\pgfqpoint{2.297958in}{2.182770in}}{\pgfqpoint{2.293568in}{2.193369in}}{\pgfqpoint{2.285754in}{2.201182in}}%
\pgfpathcurveto{\pgfqpoint{2.277940in}{2.208996in}}{\pgfqpoint{2.267341in}{2.213386in}}{\pgfqpoint{2.256291in}{2.213386in}}%
\pgfpathcurveto{\pgfqpoint{2.245241in}{2.213386in}}{\pgfqpoint{2.234642in}{2.208996in}}{\pgfqpoint{2.226828in}{2.201182in}}%
\pgfpathcurveto{\pgfqpoint{2.219015in}{2.193369in}}{\pgfqpoint{2.214625in}{2.182770in}}{\pgfqpoint{2.214625in}{2.171720in}}%
\pgfpathcurveto{\pgfqpoint{2.214625in}{2.160669in}}{\pgfqpoint{2.219015in}{2.150070in}}{\pgfqpoint{2.226828in}{2.142257in}}%
\pgfpathcurveto{\pgfqpoint{2.234642in}{2.134443in}}{\pgfqpoint{2.245241in}{2.130053in}}{\pgfqpoint{2.256291in}{2.130053in}}%
\pgfpathclose%
\pgfusepath{stroke,fill}%
\end{pgfscope}%
\begin{pgfscope}%
\pgfpathrectangle{\pgfqpoint{0.600000in}{0.600000in}}{\pgfqpoint{3.900000in}{3.900000in}}%
\pgfusepath{clip}%
\pgfsetbuttcap%
\pgfsetroundjoin%
\definecolor{currentfill}{rgb}{0.121569,0.466667,0.705882}%
\pgfsetfillcolor{currentfill}%
\pgfsetlinewidth{1.003750pt}%
\definecolor{currentstroke}{rgb}{0.121569,0.466667,0.705882}%
\pgfsetstrokecolor{currentstroke}%
\pgfsetdash{}{0pt}%
\pgfpathmoveto{\pgfqpoint{3.088620in}{3.022161in}}%
\pgfpathcurveto{\pgfqpoint{3.099670in}{3.022161in}}{\pgfqpoint{3.110269in}{3.026551in}}{\pgfqpoint{3.118083in}{3.034365in}}%
\pgfpathcurveto{\pgfqpoint{3.125896in}{3.042178in}}{\pgfqpoint{3.130287in}{3.052777in}}{\pgfqpoint{3.130287in}{3.063827in}}%
\pgfpathcurveto{\pgfqpoint{3.130287in}{3.074877in}}{\pgfqpoint{3.125896in}{3.085476in}}{\pgfqpoint{3.118083in}{3.093290in}}%
\pgfpathcurveto{\pgfqpoint{3.110269in}{3.101104in}}{\pgfqpoint{3.099670in}{3.105494in}}{\pgfqpoint{3.088620in}{3.105494in}}%
\pgfpathcurveto{\pgfqpoint{3.077570in}{3.105494in}}{\pgfqpoint{3.066971in}{3.101104in}}{\pgfqpoint{3.059157in}{3.093290in}}%
\pgfpathcurveto{\pgfqpoint{3.051344in}{3.085476in}}{\pgfqpoint{3.046953in}{3.074877in}}{\pgfqpoint{3.046953in}{3.063827in}}%
\pgfpathcurveto{\pgfqpoint{3.046953in}{3.052777in}}{\pgfqpoint{3.051344in}{3.042178in}}{\pgfqpoint{3.059157in}{3.034365in}}%
\pgfpathcurveto{\pgfqpoint{3.066971in}{3.026551in}}{\pgfqpoint{3.077570in}{3.022161in}}{\pgfqpoint{3.088620in}{3.022161in}}%
\pgfpathclose%
\pgfusepath{stroke,fill}%
\end{pgfscope}%
\begin{pgfscope}%
\pgfpathrectangle{\pgfqpoint{0.600000in}{0.600000in}}{\pgfqpoint{3.900000in}{3.900000in}}%
\pgfusepath{clip}%
\pgfsetbuttcap%
\pgfsetroundjoin%
\definecolor{currentfill}{rgb}{0.121569,0.466667,0.705882}%
\pgfsetfillcolor{currentfill}%
\pgfsetlinewidth{1.003750pt}%
\definecolor{currentstroke}{rgb}{0.121569,0.466667,0.705882}%
\pgfsetstrokecolor{currentstroke}%
\pgfsetdash{}{0pt}%
\pgfpathmoveto{\pgfqpoint{2.000178in}{2.597007in}}%
\pgfpathcurveto{\pgfqpoint{2.011228in}{2.597007in}}{\pgfqpoint{2.021827in}{2.601397in}}{\pgfqpoint{2.029640in}{2.609211in}}%
\pgfpathcurveto{\pgfqpoint{2.037454in}{2.617024in}}{\pgfqpoint{2.041844in}{2.627623in}}{\pgfqpoint{2.041844in}{2.638674in}}%
\pgfpathcurveto{\pgfqpoint{2.041844in}{2.649724in}}{\pgfqpoint{2.037454in}{2.660323in}}{\pgfqpoint{2.029640in}{2.668136in}}%
\pgfpathcurveto{\pgfqpoint{2.021827in}{2.675950in}}{\pgfqpoint{2.011228in}{2.680340in}}{\pgfqpoint{2.000178in}{2.680340in}}%
\pgfpathcurveto{\pgfqpoint{1.989128in}{2.680340in}}{\pgfqpoint{1.978529in}{2.675950in}}{\pgfqpoint{1.970715in}{2.668136in}}%
\pgfpathcurveto{\pgfqpoint{1.962901in}{2.660323in}}{\pgfqpoint{1.958511in}{2.649724in}}{\pgfqpoint{1.958511in}{2.638674in}}%
\pgfpathcurveto{\pgfqpoint{1.958511in}{2.627623in}}{\pgfqpoint{1.962901in}{2.617024in}}{\pgfqpoint{1.970715in}{2.609211in}}%
\pgfpathcurveto{\pgfqpoint{1.978529in}{2.601397in}}{\pgfqpoint{1.989128in}{2.597007in}}{\pgfqpoint{2.000178in}{2.597007in}}%
\pgfpathclose%
\pgfusepath{stroke,fill}%
\end{pgfscope}%
\begin{pgfscope}%
\pgfpathrectangle{\pgfqpoint{0.600000in}{0.600000in}}{\pgfqpoint{3.900000in}{3.900000in}}%
\pgfusepath{clip}%
\pgfsetbuttcap%
\pgfsetroundjoin%
\definecolor{currentfill}{rgb}{0.121569,0.466667,0.705882}%
\pgfsetfillcolor{currentfill}%
\pgfsetlinewidth{1.003750pt}%
\definecolor{currentstroke}{rgb}{0.121569,0.466667,0.705882}%
\pgfsetstrokecolor{currentstroke}%
\pgfsetdash{}{0pt}%
\pgfpathmoveto{\pgfqpoint{2.669916in}{3.139629in}}%
\pgfpathcurveto{\pgfqpoint{2.680966in}{3.139629in}}{\pgfqpoint{2.691565in}{3.144020in}}{\pgfqpoint{2.699379in}{3.151833in}}%
\pgfpathcurveto{\pgfqpoint{2.707193in}{3.159647in}}{\pgfqpoint{2.711583in}{3.170246in}}{\pgfqpoint{2.711583in}{3.181296in}}%
\pgfpathcurveto{\pgfqpoint{2.711583in}{3.192346in}}{\pgfqpoint{2.707193in}{3.202945in}}{\pgfqpoint{2.699379in}{3.210759in}}%
\pgfpathcurveto{\pgfqpoint{2.691565in}{3.218572in}}{\pgfqpoint{2.680966in}{3.222963in}}{\pgfqpoint{2.669916in}{3.222963in}}%
\pgfpathcurveto{\pgfqpoint{2.658866in}{3.222963in}}{\pgfqpoint{2.648267in}{3.218572in}}{\pgfqpoint{2.640454in}{3.210759in}}%
\pgfpathcurveto{\pgfqpoint{2.632640in}{3.202945in}}{\pgfqpoint{2.628250in}{3.192346in}}{\pgfqpoint{2.628250in}{3.181296in}}%
\pgfpathcurveto{\pgfqpoint{2.628250in}{3.170246in}}{\pgfqpoint{2.632640in}{3.159647in}}{\pgfqpoint{2.640454in}{3.151833in}}%
\pgfpathcurveto{\pgfqpoint{2.648267in}{3.144020in}}{\pgfqpoint{2.658866in}{3.139629in}}{\pgfqpoint{2.669916in}{3.139629in}}%
\pgfpathclose%
\pgfusepath{stroke,fill}%
\end{pgfscope}%
\begin{pgfscope}%
\pgfpathrectangle{\pgfqpoint{0.600000in}{0.600000in}}{\pgfqpoint{3.900000in}{3.900000in}}%
\pgfusepath{clip}%
\pgfsetbuttcap%
\pgfsetroundjoin%
\definecolor{currentfill}{rgb}{0.121569,0.466667,0.705882}%
\pgfsetfillcolor{currentfill}%
\pgfsetlinewidth{1.003750pt}%
\definecolor{currentstroke}{rgb}{0.121569,0.466667,0.705882}%
\pgfsetstrokecolor{currentstroke}%
\pgfsetdash{}{0pt}%
\pgfpathmoveto{\pgfqpoint{2.496927in}{2.871718in}}%
\pgfpathcurveto{\pgfqpoint{2.507977in}{2.871718in}}{\pgfqpoint{2.518576in}{2.876108in}}{\pgfqpoint{2.526389in}{2.883922in}}%
\pgfpathcurveto{\pgfqpoint{2.534203in}{2.891735in}}{\pgfqpoint{2.538593in}{2.902334in}}{\pgfqpoint{2.538593in}{2.913384in}}%
\pgfpathcurveto{\pgfqpoint{2.538593in}{2.924435in}}{\pgfqpoint{2.534203in}{2.935034in}}{\pgfqpoint{2.526389in}{2.942847in}}%
\pgfpathcurveto{\pgfqpoint{2.518576in}{2.950661in}}{\pgfqpoint{2.507977in}{2.955051in}}{\pgfqpoint{2.496927in}{2.955051in}}%
\pgfpathcurveto{\pgfqpoint{2.485876in}{2.955051in}}{\pgfqpoint{2.475277in}{2.950661in}}{\pgfqpoint{2.467464in}{2.942847in}}%
\pgfpathcurveto{\pgfqpoint{2.459650in}{2.935034in}}{\pgfqpoint{2.455260in}{2.924435in}}{\pgfqpoint{2.455260in}{2.913384in}}%
\pgfpathcurveto{\pgfqpoint{2.455260in}{2.902334in}}{\pgfqpoint{2.459650in}{2.891735in}}{\pgfqpoint{2.467464in}{2.883922in}}%
\pgfpathcurveto{\pgfqpoint{2.475277in}{2.876108in}}{\pgfqpoint{2.485876in}{2.871718in}}{\pgfqpoint{2.496927in}{2.871718in}}%
\pgfpathclose%
\pgfusepath{stroke,fill}%
\end{pgfscope}%
\begin{pgfscope}%
\pgfpathrectangle{\pgfqpoint{0.600000in}{0.600000in}}{\pgfqpoint{3.900000in}{3.900000in}}%
\pgfusepath{clip}%
\pgfsetbuttcap%
\pgfsetroundjoin%
\definecolor{currentfill}{rgb}{0.121569,0.466667,0.705882}%
\pgfsetfillcolor{currentfill}%
\pgfsetlinewidth{1.003750pt}%
\definecolor{currentstroke}{rgb}{0.121569,0.466667,0.705882}%
\pgfsetstrokecolor{currentstroke}%
\pgfsetdash{}{0pt}%
\pgfpathmoveto{\pgfqpoint{2.079176in}{2.261343in}}%
\pgfpathcurveto{\pgfqpoint{2.090226in}{2.261343in}}{\pgfqpoint{2.100825in}{2.265734in}}{\pgfqpoint{2.108639in}{2.273547in}}%
\pgfpathcurveto{\pgfqpoint{2.116452in}{2.281361in}}{\pgfqpoint{2.120843in}{2.291960in}}{\pgfqpoint{2.120843in}{2.303010in}}%
\pgfpathcurveto{\pgfqpoint{2.120843in}{2.314060in}}{\pgfqpoint{2.116452in}{2.324659in}}{\pgfqpoint{2.108639in}{2.332473in}}%
\pgfpathcurveto{\pgfqpoint{2.100825in}{2.340287in}}{\pgfqpoint{2.090226in}{2.344677in}}{\pgfqpoint{2.079176in}{2.344677in}}%
\pgfpathcurveto{\pgfqpoint{2.068126in}{2.344677in}}{\pgfqpoint{2.057527in}{2.340287in}}{\pgfqpoint{2.049713in}{2.332473in}}%
\pgfpathcurveto{\pgfqpoint{2.041899in}{2.324659in}}{\pgfqpoint{2.037509in}{2.314060in}}{\pgfqpoint{2.037509in}{2.303010in}}%
\pgfpathcurveto{\pgfqpoint{2.037509in}{2.291960in}}{\pgfqpoint{2.041899in}{2.281361in}}{\pgfqpoint{2.049713in}{2.273547in}}%
\pgfpathcurveto{\pgfqpoint{2.057527in}{2.265734in}}{\pgfqpoint{2.068126in}{2.261343in}}{\pgfqpoint{2.079176in}{2.261343in}}%
\pgfpathclose%
\pgfusepath{stroke,fill}%
\end{pgfscope}%
\begin{pgfscope}%
\pgfpathrectangle{\pgfqpoint{0.600000in}{0.600000in}}{\pgfqpoint{3.900000in}{3.900000in}}%
\pgfusepath{clip}%
\pgfsetbuttcap%
\pgfsetroundjoin%
\definecolor{currentfill}{rgb}{0.121569,0.466667,0.705882}%
\pgfsetfillcolor{currentfill}%
\pgfsetlinewidth{1.003750pt}%
\definecolor{currentstroke}{rgb}{0.121569,0.466667,0.705882}%
\pgfsetstrokecolor{currentstroke}%
\pgfsetdash{}{0pt}%
\pgfpathmoveto{\pgfqpoint{2.772941in}{2.781004in}}%
\pgfpathcurveto{\pgfqpoint{2.783991in}{2.781004in}}{\pgfqpoint{2.794590in}{2.785395in}}{\pgfqpoint{2.802404in}{2.793208in}}%
\pgfpathcurveto{\pgfqpoint{2.810218in}{2.801022in}}{\pgfqpoint{2.814608in}{2.811621in}}{\pgfqpoint{2.814608in}{2.822671in}}%
\pgfpathcurveto{\pgfqpoint{2.814608in}{2.833721in}}{\pgfqpoint{2.810218in}{2.844320in}}{\pgfqpoint{2.802404in}{2.852134in}}%
\pgfpathcurveto{\pgfqpoint{2.794590in}{2.859947in}}{\pgfqpoint{2.783991in}{2.864338in}}{\pgfqpoint{2.772941in}{2.864338in}}%
\pgfpathcurveto{\pgfqpoint{2.761891in}{2.864338in}}{\pgfqpoint{2.751292in}{2.859947in}}{\pgfqpoint{2.743478in}{2.852134in}}%
\pgfpathcurveto{\pgfqpoint{2.735665in}{2.844320in}}{\pgfqpoint{2.731275in}{2.833721in}}{\pgfqpoint{2.731275in}{2.822671in}}%
\pgfpathcurveto{\pgfqpoint{2.731275in}{2.811621in}}{\pgfqpoint{2.735665in}{2.801022in}}{\pgfqpoint{2.743478in}{2.793208in}}%
\pgfpathcurveto{\pgfqpoint{2.751292in}{2.785395in}}{\pgfqpoint{2.761891in}{2.781004in}}{\pgfqpoint{2.772941in}{2.781004in}}%
\pgfpathclose%
\pgfusepath{stroke,fill}%
\end{pgfscope}%
\begin{pgfscope}%
\pgfpathrectangle{\pgfqpoint{0.600000in}{0.600000in}}{\pgfqpoint{3.900000in}{3.900000in}}%
\pgfusepath{clip}%
\pgfsetbuttcap%
\pgfsetroundjoin%
\definecolor{currentfill}{rgb}{0.121569,0.466667,0.705882}%
\pgfsetfillcolor{currentfill}%
\pgfsetlinewidth{1.003750pt}%
\definecolor{currentstroke}{rgb}{0.121569,0.466667,0.705882}%
\pgfsetstrokecolor{currentstroke}%
\pgfsetdash{}{0pt}%
\pgfpathmoveto{\pgfqpoint{2.579183in}{3.085168in}}%
\pgfpathcurveto{\pgfqpoint{2.590233in}{3.085168in}}{\pgfqpoint{2.600832in}{3.089558in}}{\pgfqpoint{2.608646in}{3.097372in}}%
\pgfpathcurveto{\pgfqpoint{2.616460in}{3.105185in}}{\pgfqpoint{2.620850in}{3.115784in}}{\pgfqpoint{2.620850in}{3.126834in}}%
\pgfpathcurveto{\pgfqpoint{2.620850in}{3.137884in}}{\pgfqpoint{2.616460in}{3.148484in}}{\pgfqpoint{2.608646in}{3.156297in}}%
\pgfpathcurveto{\pgfqpoint{2.600832in}{3.164111in}}{\pgfqpoint{2.590233in}{3.168501in}}{\pgfqpoint{2.579183in}{3.168501in}}%
\pgfpathcurveto{\pgfqpoint{2.568133in}{3.168501in}}{\pgfqpoint{2.557534in}{3.164111in}}{\pgfqpoint{2.549720in}{3.156297in}}%
\pgfpathcurveto{\pgfqpoint{2.541907in}{3.148484in}}{\pgfqpoint{2.537517in}{3.137884in}}{\pgfqpoint{2.537517in}{3.126834in}}%
\pgfpathcurveto{\pgfqpoint{2.537517in}{3.115784in}}{\pgfqpoint{2.541907in}{3.105185in}}{\pgfqpoint{2.549720in}{3.097372in}}%
\pgfpathcurveto{\pgfqpoint{2.557534in}{3.089558in}}{\pgfqpoint{2.568133in}{3.085168in}}{\pgfqpoint{2.579183in}{3.085168in}}%
\pgfpathclose%
\pgfusepath{stroke,fill}%
\end{pgfscope}%
\begin{pgfscope}%
\pgfpathrectangle{\pgfqpoint{0.600000in}{0.600000in}}{\pgfqpoint{3.900000in}{3.900000in}}%
\pgfusepath{clip}%
\pgfsetbuttcap%
\pgfsetroundjoin%
\definecolor{currentfill}{rgb}{0.121569,0.466667,0.705882}%
\pgfsetfillcolor{currentfill}%
\pgfsetlinewidth{1.003750pt}%
\definecolor{currentstroke}{rgb}{0.121569,0.466667,0.705882}%
\pgfsetstrokecolor{currentstroke}%
\pgfsetdash{}{0pt}%
\pgfpathmoveto{\pgfqpoint{2.300130in}{3.642937in}}%
\pgfpathcurveto{\pgfqpoint{2.311180in}{3.642937in}}{\pgfqpoint{2.321779in}{3.647327in}}{\pgfqpoint{2.329593in}{3.655141in}}%
\pgfpathcurveto{\pgfqpoint{2.337407in}{3.662954in}}{\pgfqpoint{2.341797in}{3.673553in}}{\pgfqpoint{2.341797in}{3.684604in}}%
\pgfpathcurveto{\pgfqpoint{2.341797in}{3.695654in}}{\pgfqpoint{2.337407in}{3.706253in}}{\pgfqpoint{2.329593in}{3.714066in}}%
\pgfpathcurveto{\pgfqpoint{2.321779in}{3.721880in}}{\pgfqpoint{2.311180in}{3.726270in}}{\pgfqpoint{2.300130in}{3.726270in}}%
\pgfpathcurveto{\pgfqpoint{2.289080in}{3.726270in}}{\pgfqpoint{2.278481in}{3.721880in}}{\pgfqpoint{2.270667in}{3.714066in}}%
\pgfpathcurveto{\pgfqpoint{2.262854in}{3.706253in}}{\pgfqpoint{2.258464in}{3.695654in}}{\pgfqpoint{2.258464in}{3.684604in}}%
\pgfpathcurveto{\pgfqpoint{2.258464in}{3.673553in}}{\pgfqpoint{2.262854in}{3.662954in}}{\pgfqpoint{2.270667in}{3.655141in}}%
\pgfpathcurveto{\pgfqpoint{2.278481in}{3.647327in}}{\pgfqpoint{2.289080in}{3.642937in}}{\pgfqpoint{2.300130in}{3.642937in}}%
\pgfpathclose%
\pgfusepath{stroke,fill}%
\end{pgfscope}%
\begin{pgfscope}%
\pgfpathrectangle{\pgfqpoint{0.600000in}{0.600000in}}{\pgfqpoint{3.900000in}{3.900000in}}%
\pgfusepath{clip}%
\pgfsetbuttcap%
\pgfsetroundjoin%
\definecolor{currentfill}{rgb}{0.121569,0.466667,0.705882}%
\pgfsetfillcolor{currentfill}%
\pgfsetlinewidth{1.003750pt}%
\definecolor{currentstroke}{rgb}{0.121569,0.466667,0.705882}%
\pgfsetstrokecolor{currentstroke}%
\pgfsetdash{}{0pt}%
\pgfpathmoveto{\pgfqpoint{2.761906in}{2.464435in}}%
\pgfpathcurveto{\pgfqpoint{2.772956in}{2.464435in}}{\pgfqpoint{2.783555in}{2.468825in}}{\pgfqpoint{2.791369in}{2.476639in}}%
\pgfpathcurveto{\pgfqpoint{2.799182in}{2.484453in}}{\pgfqpoint{2.803572in}{2.495052in}}{\pgfqpoint{2.803572in}{2.506102in}}%
\pgfpathcurveto{\pgfqpoint{2.803572in}{2.517152in}}{\pgfqpoint{2.799182in}{2.527751in}}{\pgfqpoint{2.791369in}{2.535565in}}%
\pgfpathcurveto{\pgfqpoint{2.783555in}{2.543378in}}{\pgfqpoint{2.772956in}{2.547768in}}{\pgfqpoint{2.761906in}{2.547768in}}%
\pgfpathcurveto{\pgfqpoint{2.750856in}{2.547768in}}{\pgfqpoint{2.740257in}{2.543378in}}{\pgfqpoint{2.732443in}{2.535565in}}%
\pgfpathcurveto{\pgfqpoint{2.724629in}{2.527751in}}{\pgfqpoint{2.720239in}{2.517152in}}{\pgfqpoint{2.720239in}{2.506102in}}%
\pgfpathcurveto{\pgfqpoint{2.720239in}{2.495052in}}{\pgfqpoint{2.724629in}{2.484453in}}{\pgfqpoint{2.732443in}{2.476639in}}%
\pgfpathcurveto{\pgfqpoint{2.740257in}{2.468825in}}{\pgfqpoint{2.750856in}{2.464435in}}{\pgfqpoint{2.761906in}{2.464435in}}%
\pgfpathclose%
\pgfusepath{stroke,fill}%
\end{pgfscope}%
\begin{pgfscope}%
\pgfpathrectangle{\pgfqpoint{0.600000in}{0.600000in}}{\pgfqpoint{3.900000in}{3.900000in}}%
\pgfusepath{clip}%
\pgfsetbuttcap%
\pgfsetroundjoin%
\definecolor{currentfill}{rgb}{0.121569,0.466667,0.705882}%
\pgfsetfillcolor{currentfill}%
\pgfsetlinewidth{1.003750pt}%
\definecolor{currentstroke}{rgb}{0.121569,0.466667,0.705882}%
\pgfsetstrokecolor{currentstroke}%
\pgfsetdash{}{0pt}%
\pgfpathmoveto{\pgfqpoint{3.440396in}{2.228833in}}%
\pgfpathcurveto{\pgfqpoint{3.451446in}{2.228833in}}{\pgfqpoint{3.462045in}{2.233223in}}{\pgfqpoint{3.469859in}{2.241037in}}%
\pgfpathcurveto{\pgfqpoint{3.477672in}{2.248850in}}{\pgfqpoint{3.482063in}{2.259449in}}{\pgfqpoint{3.482063in}{2.270500in}}%
\pgfpathcurveto{\pgfqpoint{3.482063in}{2.281550in}}{\pgfqpoint{3.477672in}{2.292149in}}{\pgfqpoint{3.469859in}{2.299962in}}%
\pgfpathcurveto{\pgfqpoint{3.462045in}{2.307776in}}{\pgfqpoint{3.451446in}{2.312166in}}{\pgfqpoint{3.440396in}{2.312166in}}%
\pgfpathcurveto{\pgfqpoint{3.429346in}{2.312166in}}{\pgfqpoint{3.418747in}{2.307776in}}{\pgfqpoint{3.410933in}{2.299962in}}%
\pgfpathcurveto{\pgfqpoint{3.403119in}{2.292149in}}{\pgfqpoint{3.398729in}{2.281550in}}{\pgfqpoint{3.398729in}{2.270500in}}%
\pgfpathcurveto{\pgfqpoint{3.398729in}{2.259449in}}{\pgfqpoint{3.403119in}{2.248850in}}{\pgfqpoint{3.410933in}{2.241037in}}%
\pgfpathcurveto{\pgfqpoint{3.418747in}{2.233223in}}{\pgfqpoint{3.429346in}{2.228833in}}{\pgfqpoint{3.440396in}{2.228833in}}%
\pgfpathclose%
\pgfusepath{stroke,fill}%
\end{pgfscope}%
\begin{pgfscope}%
\pgfpathrectangle{\pgfqpoint{0.600000in}{0.600000in}}{\pgfqpoint{3.900000in}{3.900000in}}%
\pgfusepath{clip}%
\pgfsetbuttcap%
\pgfsetroundjoin%
\definecolor{currentfill}{rgb}{0.121569,0.466667,0.705882}%
\pgfsetfillcolor{currentfill}%
\pgfsetlinewidth{1.003750pt}%
\definecolor{currentstroke}{rgb}{0.121569,0.466667,0.705882}%
\pgfsetstrokecolor{currentstroke}%
\pgfsetdash{}{0pt}%
\pgfpathmoveto{\pgfqpoint{2.442817in}{3.029811in}}%
\pgfpathcurveto{\pgfqpoint{2.453867in}{3.029811in}}{\pgfqpoint{2.464466in}{3.034201in}}{\pgfqpoint{2.472279in}{3.042015in}}%
\pgfpathcurveto{\pgfqpoint{2.480093in}{3.049829in}}{\pgfqpoint{2.484483in}{3.060428in}}{\pgfqpoint{2.484483in}{3.071478in}}%
\pgfpathcurveto{\pgfqpoint{2.484483in}{3.082528in}}{\pgfqpoint{2.480093in}{3.093127in}}{\pgfqpoint{2.472279in}{3.100941in}}%
\pgfpathcurveto{\pgfqpoint{2.464466in}{3.108754in}}{\pgfqpoint{2.453867in}{3.113144in}}{\pgfqpoint{2.442817in}{3.113144in}}%
\pgfpathcurveto{\pgfqpoint{2.431766in}{3.113144in}}{\pgfqpoint{2.421167in}{3.108754in}}{\pgfqpoint{2.413354in}{3.100941in}}%
\pgfpathcurveto{\pgfqpoint{2.405540in}{3.093127in}}{\pgfqpoint{2.401150in}{3.082528in}}{\pgfqpoint{2.401150in}{3.071478in}}%
\pgfpathcurveto{\pgfqpoint{2.401150in}{3.060428in}}{\pgfqpoint{2.405540in}{3.049829in}}{\pgfqpoint{2.413354in}{3.042015in}}%
\pgfpathcurveto{\pgfqpoint{2.421167in}{3.034201in}}{\pgfqpoint{2.431766in}{3.029811in}}{\pgfqpoint{2.442817in}{3.029811in}}%
\pgfpathclose%
\pgfusepath{stroke,fill}%
\end{pgfscope}%
\begin{pgfscope}%
\pgfpathrectangle{\pgfqpoint{0.600000in}{0.600000in}}{\pgfqpoint{3.900000in}{3.900000in}}%
\pgfusepath{clip}%
\pgfsetbuttcap%
\pgfsetroundjoin%
\definecolor{currentfill}{rgb}{0.121569,0.466667,0.705882}%
\pgfsetfillcolor{currentfill}%
\pgfsetlinewidth{1.003750pt}%
\definecolor{currentstroke}{rgb}{0.121569,0.466667,0.705882}%
\pgfsetstrokecolor{currentstroke}%
\pgfsetdash{}{0pt}%
\pgfpathmoveto{\pgfqpoint{2.931005in}{2.916229in}}%
\pgfpathcurveto{\pgfqpoint{2.942055in}{2.916229in}}{\pgfqpoint{2.952654in}{2.920619in}}{\pgfqpoint{2.960468in}{2.928433in}}%
\pgfpathcurveto{\pgfqpoint{2.968282in}{2.936247in}}{\pgfqpoint{2.972672in}{2.946846in}}{\pgfqpoint{2.972672in}{2.957896in}}%
\pgfpathcurveto{\pgfqpoint{2.972672in}{2.968946in}}{\pgfqpoint{2.968282in}{2.979545in}}{\pgfqpoint{2.960468in}{2.987359in}}%
\pgfpathcurveto{\pgfqpoint{2.952654in}{2.995172in}}{\pgfqpoint{2.942055in}{2.999563in}}{\pgfqpoint{2.931005in}{2.999563in}}%
\pgfpathcurveto{\pgfqpoint{2.919955in}{2.999563in}}{\pgfqpoint{2.909356in}{2.995172in}}{\pgfqpoint{2.901542in}{2.987359in}}%
\pgfpathcurveto{\pgfqpoint{2.893729in}{2.979545in}}{\pgfqpoint{2.889338in}{2.968946in}}{\pgfqpoint{2.889338in}{2.957896in}}%
\pgfpathcurveto{\pgfqpoint{2.889338in}{2.946846in}}{\pgfqpoint{2.893729in}{2.936247in}}{\pgfqpoint{2.901542in}{2.928433in}}%
\pgfpathcurveto{\pgfqpoint{2.909356in}{2.920619in}}{\pgfqpoint{2.919955in}{2.916229in}}{\pgfqpoint{2.931005in}{2.916229in}}%
\pgfpathclose%
\pgfusepath{stroke,fill}%
\end{pgfscope}%
\begin{pgfscope}%
\pgfpathrectangle{\pgfqpoint{0.600000in}{0.600000in}}{\pgfqpoint{3.900000in}{3.900000in}}%
\pgfusepath{clip}%
\pgfsetbuttcap%
\pgfsetroundjoin%
\definecolor{currentfill}{rgb}{0.121569,0.466667,0.705882}%
\pgfsetfillcolor{currentfill}%
\pgfsetlinewidth{1.003750pt}%
\definecolor{currentstroke}{rgb}{0.121569,0.466667,0.705882}%
\pgfsetstrokecolor{currentstroke}%
\pgfsetdash{}{0pt}%
\pgfpathmoveto{\pgfqpoint{2.825115in}{3.075458in}}%
\pgfpathcurveto{\pgfqpoint{2.836165in}{3.075458in}}{\pgfqpoint{2.846764in}{3.079848in}}{\pgfqpoint{2.854577in}{3.087662in}}%
\pgfpathcurveto{\pgfqpoint{2.862391in}{3.095476in}}{\pgfqpoint{2.866781in}{3.106075in}}{\pgfqpoint{2.866781in}{3.117125in}}%
\pgfpathcurveto{\pgfqpoint{2.866781in}{3.128175in}}{\pgfqpoint{2.862391in}{3.138774in}}{\pgfqpoint{2.854577in}{3.146588in}}%
\pgfpathcurveto{\pgfqpoint{2.846764in}{3.154401in}}{\pgfqpoint{2.836165in}{3.158791in}}{\pgfqpoint{2.825115in}{3.158791in}}%
\pgfpathcurveto{\pgfqpoint{2.814064in}{3.158791in}}{\pgfqpoint{2.803465in}{3.154401in}}{\pgfqpoint{2.795652in}{3.146588in}}%
\pgfpathcurveto{\pgfqpoint{2.787838in}{3.138774in}}{\pgfqpoint{2.783448in}{3.128175in}}{\pgfqpoint{2.783448in}{3.117125in}}%
\pgfpathcurveto{\pgfqpoint{2.783448in}{3.106075in}}{\pgfqpoint{2.787838in}{3.095476in}}{\pgfqpoint{2.795652in}{3.087662in}}%
\pgfpathcurveto{\pgfqpoint{2.803465in}{3.079848in}}{\pgfqpoint{2.814064in}{3.075458in}}{\pgfqpoint{2.825115in}{3.075458in}}%
\pgfpathclose%
\pgfusepath{stroke,fill}%
\end{pgfscope}%
\begin{pgfscope}%
\pgfpathrectangle{\pgfqpoint{0.600000in}{0.600000in}}{\pgfqpoint{3.900000in}{3.900000in}}%
\pgfusepath{clip}%
\pgfsetbuttcap%
\pgfsetroundjoin%
\definecolor{currentfill}{rgb}{0.121569,0.466667,0.705882}%
\pgfsetfillcolor{currentfill}%
\pgfsetlinewidth{1.003750pt}%
\definecolor{currentstroke}{rgb}{0.121569,0.466667,0.705882}%
\pgfsetstrokecolor{currentstroke}%
\pgfsetdash{}{0pt}%
\pgfpathmoveto{\pgfqpoint{2.740978in}{2.427477in}}%
\pgfpathcurveto{\pgfqpoint{2.752028in}{2.427477in}}{\pgfqpoint{2.762627in}{2.431867in}}{\pgfqpoint{2.770440in}{2.439680in}}%
\pgfpathcurveto{\pgfqpoint{2.778254in}{2.447494in}}{\pgfqpoint{2.782644in}{2.458093in}}{\pgfqpoint{2.782644in}{2.469143in}}%
\pgfpathcurveto{\pgfqpoint{2.782644in}{2.480193in}}{\pgfqpoint{2.778254in}{2.490792in}}{\pgfqpoint{2.770440in}{2.498606in}}%
\pgfpathcurveto{\pgfqpoint{2.762627in}{2.506420in}}{\pgfqpoint{2.752028in}{2.510810in}}{\pgfqpoint{2.740978in}{2.510810in}}%
\pgfpathcurveto{\pgfqpoint{2.729927in}{2.510810in}}{\pgfqpoint{2.719328in}{2.506420in}}{\pgfqpoint{2.711515in}{2.498606in}}%
\pgfpathcurveto{\pgfqpoint{2.703701in}{2.490792in}}{\pgfqpoint{2.699311in}{2.480193in}}{\pgfqpoint{2.699311in}{2.469143in}}%
\pgfpathcurveto{\pgfqpoint{2.699311in}{2.458093in}}{\pgfqpoint{2.703701in}{2.447494in}}{\pgfqpoint{2.711515in}{2.439680in}}%
\pgfpathcurveto{\pgfqpoint{2.719328in}{2.431867in}}{\pgfqpoint{2.729927in}{2.427477in}}{\pgfqpoint{2.740978in}{2.427477in}}%
\pgfpathclose%
\pgfusepath{stroke,fill}%
\end{pgfscope}%
\begin{pgfscope}%
\pgfpathrectangle{\pgfqpoint{0.600000in}{0.600000in}}{\pgfqpoint{3.900000in}{3.900000in}}%
\pgfusepath{clip}%
\pgfsetbuttcap%
\pgfsetroundjoin%
\definecolor{currentfill}{rgb}{0.121569,0.466667,0.705882}%
\pgfsetfillcolor{currentfill}%
\pgfsetlinewidth{1.003750pt}%
\definecolor{currentstroke}{rgb}{0.121569,0.466667,0.705882}%
\pgfsetstrokecolor{currentstroke}%
\pgfsetdash{}{0pt}%
\pgfpathmoveto{\pgfqpoint{2.882485in}{2.620208in}}%
\pgfpathcurveto{\pgfqpoint{2.893535in}{2.620208in}}{\pgfqpoint{2.904134in}{2.624598in}}{\pgfqpoint{2.911947in}{2.632412in}}%
\pgfpathcurveto{\pgfqpoint{2.919761in}{2.640226in}}{\pgfqpoint{2.924151in}{2.650825in}}{\pgfqpoint{2.924151in}{2.661875in}}%
\pgfpathcurveto{\pgfqpoint{2.924151in}{2.672925in}}{\pgfqpoint{2.919761in}{2.683524in}}{\pgfqpoint{2.911947in}{2.691338in}}%
\pgfpathcurveto{\pgfqpoint{2.904134in}{2.699151in}}{\pgfqpoint{2.893535in}{2.703542in}}{\pgfqpoint{2.882485in}{2.703542in}}%
\pgfpathcurveto{\pgfqpoint{2.871435in}{2.703542in}}{\pgfqpoint{2.860835in}{2.699151in}}{\pgfqpoint{2.853022in}{2.691338in}}%
\pgfpathcurveto{\pgfqpoint{2.845208in}{2.683524in}}{\pgfqpoint{2.840818in}{2.672925in}}{\pgfqpoint{2.840818in}{2.661875in}}%
\pgfpathcurveto{\pgfqpoint{2.840818in}{2.650825in}}{\pgfqpoint{2.845208in}{2.640226in}}{\pgfqpoint{2.853022in}{2.632412in}}%
\pgfpathcurveto{\pgfqpoint{2.860835in}{2.624598in}}{\pgfqpoint{2.871435in}{2.620208in}}{\pgfqpoint{2.882485in}{2.620208in}}%
\pgfpathclose%
\pgfusepath{stroke,fill}%
\end{pgfscope}%
\begin{pgfscope}%
\pgfpathrectangle{\pgfqpoint{0.600000in}{0.600000in}}{\pgfqpoint{3.900000in}{3.900000in}}%
\pgfusepath{clip}%
\pgfsetbuttcap%
\pgfsetroundjoin%
\definecolor{currentfill}{rgb}{0.121569,0.466667,0.705882}%
\pgfsetfillcolor{currentfill}%
\pgfsetlinewidth{1.003750pt}%
\definecolor{currentstroke}{rgb}{0.121569,0.466667,0.705882}%
\pgfsetstrokecolor{currentstroke}%
\pgfsetdash{}{0pt}%
\pgfpathmoveto{\pgfqpoint{2.594900in}{3.150223in}}%
\pgfpathcurveto{\pgfqpoint{2.605951in}{3.150223in}}{\pgfqpoint{2.616550in}{3.154614in}}{\pgfqpoint{2.624363in}{3.162427in}}%
\pgfpathcurveto{\pgfqpoint{2.632177in}{3.170241in}}{\pgfqpoint{2.636567in}{3.180840in}}{\pgfqpoint{2.636567in}{3.191890in}}%
\pgfpathcurveto{\pgfqpoint{2.636567in}{3.202940in}}{\pgfqpoint{2.632177in}{3.213539in}}{\pgfqpoint{2.624363in}{3.221353in}}%
\pgfpathcurveto{\pgfqpoint{2.616550in}{3.229166in}}{\pgfqpoint{2.605951in}{3.233557in}}{\pgfqpoint{2.594900in}{3.233557in}}%
\pgfpathcurveto{\pgfqpoint{2.583850in}{3.233557in}}{\pgfqpoint{2.573251in}{3.229166in}}{\pgfqpoint{2.565438in}{3.221353in}}%
\pgfpathcurveto{\pgfqpoint{2.557624in}{3.213539in}}{\pgfqpoint{2.553234in}{3.202940in}}{\pgfqpoint{2.553234in}{3.191890in}}%
\pgfpathcurveto{\pgfqpoint{2.553234in}{3.180840in}}{\pgfqpoint{2.557624in}{3.170241in}}{\pgfqpoint{2.565438in}{3.162427in}}%
\pgfpathcurveto{\pgfqpoint{2.573251in}{3.154614in}}{\pgfqpoint{2.583850in}{3.150223in}}{\pgfqpoint{2.594900in}{3.150223in}}%
\pgfpathclose%
\pgfusepath{stroke,fill}%
\end{pgfscope}%
\begin{pgfscope}%
\pgfpathrectangle{\pgfqpoint{0.600000in}{0.600000in}}{\pgfqpoint{3.900000in}{3.900000in}}%
\pgfusepath{clip}%
\pgfsetbuttcap%
\pgfsetroundjoin%
\definecolor{currentfill}{rgb}{0.121569,0.466667,0.705882}%
\pgfsetfillcolor{currentfill}%
\pgfsetlinewidth{1.003750pt}%
\definecolor{currentstroke}{rgb}{0.121569,0.466667,0.705882}%
\pgfsetstrokecolor{currentstroke}%
\pgfsetdash{}{0pt}%
\pgfpathmoveto{\pgfqpoint{2.316711in}{2.340541in}}%
\pgfpathcurveto{\pgfqpoint{2.327761in}{2.340541in}}{\pgfqpoint{2.338360in}{2.344931in}}{\pgfqpoint{2.346174in}{2.352745in}}%
\pgfpathcurveto{\pgfqpoint{2.353987in}{2.360558in}}{\pgfqpoint{2.358378in}{2.371157in}}{\pgfqpoint{2.358378in}{2.382207in}}%
\pgfpathcurveto{\pgfqpoint{2.358378in}{2.393257in}}{\pgfqpoint{2.353987in}{2.403857in}}{\pgfqpoint{2.346174in}{2.411670in}}%
\pgfpathcurveto{\pgfqpoint{2.338360in}{2.419484in}}{\pgfqpoint{2.327761in}{2.423874in}}{\pgfqpoint{2.316711in}{2.423874in}}%
\pgfpathcurveto{\pgfqpoint{2.305661in}{2.423874in}}{\pgfqpoint{2.295062in}{2.419484in}}{\pgfqpoint{2.287248in}{2.411670in}}%
\pgfpathcurveto{\pgfqpoint{2.279435in}{2.403857in}}{\pgfqpoint{2.275044in}{2.393257in}}{\pgfqpoint{2.275044in}{2.382207in}}%
\pgfpathcurveto{\pgfqpoint{2.275044in}{2.371157in}}{\pgfqpoint{2.279435in}{2.360558in}}{\pgfqpoint{2.287248in}{2.352745in}}%
\pgfpathcurveto{\pgfqpoint{2.295062in}{2.344931in}}{\pgfqpoint{2.305661in}{2.340541in}}{\pgfqpoint{2.316711in}{2.340541in}}%
\pgfpathclose%
\pgfusepath{stroke,fill}%
\end{pgfscope}%
\begin{pgfscope}%
\pgfpathrectangle{\pgfqpoint{0.600000in}{0.600000in}}{\pgfqpoint{3.900000in}{3.900000in}}%
\pgfusepath{clip}%
\pgfsetbuttcap%
\pgfsetroundjoin%
\definecolor{currentfill}{rgb}{0.121569,0.466667,0.705882}%
\pgfsetfillcolor{currentfill}%
\pgfsetlinewidth{1.003750pt}%
\definecolor{currentstroke}{rgb}{0.121569,0.466667,0.705882}%
\pgfsetstrokecolor{currentstroke}%
\pgfsetdash{}{0pt}%
\pgfpathmoveto{\pgfqpoint{2.275459in}{2.891349in}}%
\pgfpathcurveto{\pgfqpoint{2.286509in}{2.891349in}}{\pgfqpoint{2.297108in}{2.895739in}}{\pgfqpoint{2.304922in}{2.903553in}}%
\pgfpathcurveto{\pgfqpoint{2.312735in}{2.911367in}}{\pgfqpoint{2.317126in}{2.921966in}}{\pgfqpoint{2.317126in}{2.933016in}}%
\pgfpathcurveto{\pgfqpoint{2.317126in}{2.944066in}}{\pgfqpoint{2.312735in}{2.954665in}}{\pgfqpoint{2.304922in}{2.962479in}}%
\pgfpathcurveto{\pgfqpoint{2.297108in}{2.970292in}}{\pgfqpoint{2.286509in}{2.974683in}}{\pgfqpoint{2.275459in}{2.974683in}}%
\pgfpathcurveto{\pgfqpoint{2.264409in}{2.974683in}}{\pgfqpoint{2.253810in}{2.970292in}}{\pgfqpoint{2.245996in}{2.962479in}}%
\pgfpathcurveto{\pgfqpoint{2.238183in}{2.954665in}}{\pgfqpoint{2.233792in}{2.944066in}}{\pgfqpoint{2.233792in}{2.933016in}}%
\pgfpathcurveto{\pgfqpoint{2.233792in}{2.921966in}}{\pgfqpoint{2.238183in}{2.911367in}}{\pgfqpoint{2.245996in}{2.903553in}}%
\pgfpathcurveto{\pgfqpoint{2.253810in}{2.895739in}}{\pgfqpoint{2.264409in}{2.891349in}}{\pgfqpoint{2.275459in}{2.891349in}}%
\pgfpathclose%
\pgfusepath{stroke,fill}%
\end{pgfscope}%
\begin{pgfscope}%
\pgfpathrectangle{\pgfqpoint{0.600000in}{0.600000in}}{\pgfqpoint{3.900000in}{3.900000in}}%
\pgfusepath{clip}%
\pgfsetbuttcap%
\pgfsetroundjoin%
\definecolor{currentfill}{rgb}{0.121569,0.466667,0.705882}%
\pgfsetfillcolor{currentfill}%
\pgfsetlinewidth{1.003750pt}%
\definecolor{currentstroke}{rgb}{0.121569,0.466667,0.705882}%
\pgfsetstrokecolor{currentstroke}%
\pgfsetdash{}{0pt}%
\pgfpathmoveto{\pgfqpoint{3.532132in}{2.042527in}}%
\pgfpathcurveto{\pgfqpoint{3.543182in}{2.042527in}}{\pgfqpoint{3.553782in}{2.046917in}}{\pgfqpoint{3.561595in}{2.054731in}}%
\pgfpathcurveto{\pgfqpoint{3.569409in}{2.062544in}}{\pgfqpoint{3.573799in}{2.073143in}}{\pgfqpoint{3.573799in}{2.084193in}}%
\pgfpathcurveto{\pgfqpoint{3.573799in}{2.095243in}}{\pgfqpoint{3.569409in}{2.105843in}}{\pgfqpoint{3.561595in}{2.113656in}}%
\pgfpathcurveto{\pgfqpoint{3.553782in}{2.121470in}}{\pgfqpoint{3.543182in}{2.125860in}}{\pgfqpoint{3.532132in}{2.125860in}}%
\pgfpathcurveto{\pgfqpoint{3.521082in}{2.125860in}}{\pgfqpoint{3.510483in}{2.121470in}}{\pgfqpoint{3.502670in}{2.113656in}}%
\pgfpathcurveto{\pgfqpoint{3.494856in}{2.105843in}}{\pgfqpoint{3.490466in}{2.095243in}}{\pgfqpoint{3.490466in}{2.084193in}}%
\pgfpathcurveto{\pgfqpoint{3.490466in}{2.073143in}}{\pgfqpoint{3.494856in}{2.062544in}}{\pgfqpoint{3.502670in}{2.054731in}}%
\pgfpathcurveto{\pgfqpoint{3.510483in}{2.046917in}}{\pgfqpoint{3.521082in}{2.042527in}}{\pgfqpoint{3.532132in}{2.042527in}}%
\pgfpathclose%
\pgfusepath{stroke,fill}%
\end{pgfscope}%
\begin{pgfscope}%
\pgfpathrectangle{\pgfqpoint{0.600000in}{0.600000in}}{\pgfqpoint{3.900000in}{3.900000in}}%
\pgfusepath{clip}%
\pgfsetbuttcap%
\pgfsetroundjoin%
\definecolor{currentfill}{rgb}{0.121569,0.466667,0.705882}%
\pgfsetfillcolor{currentfill}%
\pgfsetlinewidth{1.003750pt}%
\definecolor{currentstroke}{rgb}{0.121569,0.466667,0.705882}%
\pgfsetstrokecolor{currentstroke}%
\pgfsetdash{}{0pt}%
\pgfpathmoveto{\pgfqpoint{2.487264in}{1.621669in}}%
\pgfpathcurveto{\pgfqpoint{2.498314in}{1.621669in}}{\pgfqpoint{2.508913in}{1.626059in}}{\pgfqpoint{2.516727in}{1.633873in}}%
\pgfpathcurveto{\pgfqpoint{2.524540in}{1.641687in}}{\pgfqpoint{2.528930in}{1.652286in}}{\pgfqpoint{2.528930in}{1.663336in}}%
\pgfpathcurveto{\pgfqpoint{2.528930in}{1.674386in}}{\pgfqpoint{2.524540in}{1.684985in}}{\pgfqpoint{2.516727in}{1.692799in}}%
\pgfpathcurveto{\pgfqpoint{2.508913in}{1.700612in}}{\pgfqpoint{2.498314in}{1.705003in}}{\pgfqpoint{2.487264in}{1.705003in}}%
\pgfpathcurveto{\pgfqpoint{2.476214in}{1.705003in}}{\pgfqpoint{2.465615in}{1.700612in}}{\pgfqpoint{2.457801in}{1.692799in}}%
\pgfpathcurveto{\pgfqpoint{2.449987in}{1.684985in}}{\pgfqpoint{2.445597in}{1.674386in}}{\pgfqpoint{2.445597in}{1.663336in}}%
\pgfpathcurveto{\pgfqpoint{2.445597in}{1.652286in}}{\pgfqpoint{2.449987in}{1.641687in}}{\pgfqpoint{2.457801in}{1.633873in}}%
\pgfpathcurveto{\pgfqpoint{2.465615in}{1.626059in}}{\pgfqpoint{2.476214in}{1.621669in}}{\pgfqpoint{2.487264in}{1.621669in}}%
\pgfpathclose%
\pgfusepath{stroke,fill}%
\end{pgfscope}%
\begin{pgfscope}%
\pgfpathrectangle{\pgfqpoint{0.600000in}{0.600000in}}{\pgfqpoint{3.900000in}{3.900000in}}%
\pgfusepath{clip}%
\pgfsetbuttcap%
\pgfsetroundjoin%
\definecolor{currentfill}{rgb}{0.121569,0.466667,0.705882}%
\pgfsetfillcolor{currentfill}%
\pgfsetlinewidth{1.003750pt}%
\definecolor{currentstroke}{rgb}{0.121569,0.466667,0.705882}%
\pgfsetstrokecolor{currentstroke}%
\pgfsetdash{}{0pt}%
\pgfpathmoveto{\pgfqpoint{3.428893in}{1.942274in}}%
\pgfpathcurveto{\pgfqpoint{3.439944in}{1.942274in}}{\pgfqpoint{3.450543in}{1.946664in}}{\pgfqpoint{3.458356in}{1.954478in}}%
\pgfpathcurveto{\pgfqpoint{3.466170in}{1.962292in}}{\pgfqpoint{3.470560in}{1.972891in}}{\pgfqpoint{3.470560in}{1.983941in}}%
\pgfpathcurveto{\pgfqpoint{3.470560in}{1.994991in}}{\pgfqpoint{3.466170in}{2.005590in}}{\pgfqpoint{3.458356in}{2.013404in}}%
\pgfpathcurveto{\pgfqpoint{3.450543in}{2.021217in}}{\pgfqpoint{3.439944in}{2.025607in}}{\pgfqpoint{3.428893in}{2.025607in}}%
\pgfpathcurveto{\pgfqpoint{3.417843in}{2.025607in}}{\pgfqpoint{3.407244in}{2.021217in}}{\pgfqpoint{3.399431in}{2.013404in}}%
\pgfpathcurveto{\pgfqpoint{3.391617in}{2.005590in}}{\pgfqpoint{3.387227in}{1.994991in}}{\pgfqpoint{3.387227in}{1.983941in}}%
\pgfpathcurveto{\pgfqpoint{3.387227in}{1.972891in}}{\pgfqpoint{3.391617in}{1.962292in}}{\pgfqpoint{3.399431in}{1.954478in}}%
\pgfpathcurveto{\pgfqpoint{3.407244in}{1.946664in}}{\pgfqpoint{3.417843in}{1.942274in}}{\pgfqpoint{3.428893in}{1.942274in}}%
\pgfpathclose%
\pgfusepath{stroke,fill}%
\end{pgfscope}%
\begin{pgfscope}%
\pgfpathrectangle{\pgfqpoint{0.600000in}{0.600000in}}{\pgfqpoint{3.900000in}{3.900000in}}%
\pgfusepath{clip}%
\pgfsetbuttcap%
\pgfsetroundjoin%
\definecolor{currentfill}{rgb}{0.121569,0.466667,0.705882}%
\pgfsetfillcolor{currentfill}%
\pgfsetlinewidth{1.003750pt}%
\definecolor{currentstroke}{rgb}{0.121569,0.466667,0.705882}%
\pgfsetstrokecolor{currentstroke}%
\pgfsetdash{}{0pt}%
\pgfpathmoveto{\pgfqpoint{2.822422in}{2.460613in}}%
\pgfpathcurveto{\pgfqpoint{2.833473in}{2.460613in}}{\pgfqpoint{2.844072in}{2.465003in}}{\pgfqpoint{2.851885in}{2.472817in}}%
\pgfpathcurveto{\pgfqpoint{2.859699in}{2.480630in}}{\pgfqpoint{2.864089in}{2.491229in}}{\pgfqpoint{2.864089in}{2.502279in}}%
\pgfpathcurveto{\pgfqpoint{2.864089in}{2.513329in}}{\pgfqpoint{2.859699in}{2.523929in}}{\pgfqpoint{2.851885in}{2.531742in}}%
\pgfpathcurveto{\pgfqpoint{2.844072in}{2.539556in}}{\pgfqpoint{2.833473in}{2.543946in}}{\pgfqpoint{2.822422in}{2.543946in}}%
\pgfpathcurveto{\pgfqpoint{2.811372in}{2.543946in}}{\pgfqpoint{2.800773in}{2.539556in}}{\pgfqpoint{2.792960in}{2.531742in}}%
\pgfpathcurveto{\pgfqpoint{2.785146in}{2.523929in}}{\pgfqpoint{2.780756in}{2.513329in}}{\pgfqpoint{2.780756in}{2.502279in}}%
\pgfpathcurveto{\pgfqpoint{2.780756in}{2.491229in}}{\pgfqpoint{2.785146in}{2.480630in}}{\pgfqpoint{2.792960in}{2.472817in}}%
\pgfpathcurveto{\pgfqpoint{2.800773in}{2.465003in}}{\pgfqpoint{2.811372in}{2.460613in}}{\pgfqpoint{2.822422in}{2.460613in}}%
\pgfpathclose%
\pgfusepath{stroke,fill}%
\end{pgfscope}%
\begin{pgfscope}%
\pgfpathrectangle{\pgfqpoint{0.600000in}{0.600000in}}{\pgfqpoint{3.900000in}{3.900000in}}%
\pgfusepath{clip}%
\pgfsetbuttcap%
\pgfsetroundjoin%
\definecolor{currentfill}{rgb}{0.121569,0.466667,0.705882}%
\pgfsetfillcolor{currentfill}%
\pgfsetlinewidth{1.003750pt}%
\definecolor{currentstroke}{rgb}{0.121569,0.466667,0.705882}%
\pgfsetstrokecolor{currentstroke}%
\pgfsetdash{}{0pt}%
\pgfpathmoveto{\pgfqpoint{2.787480in}{2.283423in}}%
\pgfpathcurveto{\pgfqpoint{2.798530in}{2.283423in}}{\pgfqpoint{2.809129in}{2.287813in}}{\pgfqpoint{2.816943in}{2.295626in}}%
\pgfpathcurveto{\pgfqpoint{2.824756in}{2.303440in}}{\pgfqpoint{2.829146in}{2.314039in}}{\pgfqpoint{2.829146in}{2.325089in}}%
\pgfpathcurveto{\pgfqpoint{2.829146in}{2.336139in}}{\pgfqpoint{2.824756in}{2.346738in}}{\pgfqpoint{2.816943in}{2.354552in}}%
\pgfpathcurveto{\pgfqpoint{2.809129in}{2.362366in}}{\pgfqpoint{2.798530in}{2.366756in}}{\pgfqpoint{2.787480in}{2.366756in}}%
\pgfpathcurveto{\pgfqpoint{2.776430in}{2.366756in}}{\pgfqpoint{2.765831in}{2.362366in}}{\pgfqpoint{2.758017in}{2.354552in}}%
\pgfpathcurveto{\pgfqpoint{2.750203in}{2.346738in}}{\pgfqpoint{2.745813in}{2.336139in}}{\pgfqpoint{2.745813in}{2.325089in}}%
\pgfpathcurveto{\pgfqpoint{2.745813in}{2.314039in}}{\pgfqpoint{2.750203in}{2.303440in}}{\pgfqpoint{2.758017in}{2.295626in}}%
\pgfpathcurveto{\pgfqpoint{2.765831in}{2.287813in}}{\pgfqpoint{2.776430in}{2.283423in}}{\pgfqpoint{2.787480in}{2.283423in}}%
\pgfpathclose%
\pgfusepath{stroke,fill}%
\end{pgfscope}%
\begin{pgfscope}%
\pgfpathrectangle{\pgfqpoint{0.600000in}{0.600000in}}{\pgfqpoint{3.900000in}{3.900000in}}%
\pgfusepath{clip}%
\pgfsetbuttcap%
\pgfsetroundjoin%
\definecolor{currentfill}{rgb}{0.121569,0.466667,0.705882}%
\pgfsetfillcolor{currentfill}%
\pgfsetlinewidth{1.003750pt}%
\definecolor{currentstroke}{rgb}{0.121569,0.466667,0.705882}%
\pgfsetstrokecolor{currentstroke}%
\pgfsetdash{}{0pt}%
\pgfpathmoveto{\pgfqpoint{1.822853in}{2.249951in}}%
\pgfpathcurveto{\pgfqpoint{1.833903in}{2.249951in}}{\pgfqpoint{1.844503in}{2.254342in}}{\pgfqpoint{1.852316in}{2.262155in}}%
\pgfpathcurveto{\pgfqpoint{1.860130in}{2.269969in}}{\pgfqpoint{1.864520in}{2.280568in}}{\pgfqpoint{1.864520in}{2.291618in}}%
\pgfpathcurveto{\pgfqpoint{1.864520in}{2.302668in}}{\pgfqpoint{1.860130in}{2.313267in}}{\pgfqpoint{1.852316in}{2.321081in}}%
\pgfpathcurveto{\pgfqpoint{1.844503in}{2.328894in}}{\pgfqpoint{1.833903in}{2.333285in}}{\pgfqpoint{1.822853in}{2.333285in}}%
\pgfpathcurveto{\pgfqpoint{1.811803in}{2.333285in}}{\pgfqpoint{1.801204in}{2.328894in}}{\pgfqpoint{1.793391in}{2.321081in}}%
\pgfpathcurveto{\pgfqpoint{1.785577in}{2.313267in}}{\pgfqpoint{1.781187in}{2.302668in}}{\pgfqpoint{1.781187in}{2.291618in}}%
\pgfpathcurveto{\pgfqpoint{1.781187in}{2.280568in}}{\pgfqpoint{1.785577in}{2.269969in}}{\pgfqpoint{1.793391in}{2.262155in}}%
\pgfpathcurveto{\pgfqpoint{1.801204in}{2.254342in}}{\pgfqpoint{1.811803in}{2.249951in}}{\pgfqpoint{1.822853in}{2.249951in}}%
\pgfpathclose%
\pgfusepath{stroke,fill}%
\end{pgfscope}%
\begin{pgfscope}%
\pgfpathrectangle{\pgfqpoint{0.600000in}{0.600000in}}{\pgfqpoint{3.900000in}{3.900000in}}%
\pgfusepath{clip}%
\pgfsetbuttcap%
\pgfsetroundjoin%
\definecolor{currentfill}{rgb}{0.121569,0.466667,0.705882}%
\pgfsetfillcolor{currentfill}%
\pgfsetlinewidth{1.003750pt}%
\definecolor{currentstroke}{rgb}{0.121569,0.466667,0.705882}%
\pgfsetstrokecolor{currentstroke}%
\pgfsetdash{}{0pt}%
\pgfpathmoveto{\pgfqpoint{2.043594in}{3.326118in}}%
\pgfpathcurveto{\pgfqpoint{2.054644in}{3.326118in}}{\pgfqpoint{2.065243in}{3.330509in}}{\pgfqpoint{2.073057in}{3.338322in}}%
\pgfpathcurveto{\pgfqpoint{2.080871in}{3.346136in}}{\pgfqpoint{2.085261in}{3.356735in}}{\pgfqpoint{2.085261in}{3.367785in}}%
\pgfpathcurveto{\pgfqpoint{2.085261in}{3.378835in}}{\pgfqpoint{2.080871in}{3.389434in}}{\pgfqpoint{2.073057in}{3.397248in}}%
\pgfpathcurveto{\pgfqpoint{2.065243in}{3.405061in}}{\pgfqpoint{2.054644in}{3.409452in}}{\pgfqpoint{2.043594in}{3.409452in}}%
\pgfpathcurveto{\pgfqpoint{2.032544in}{3.409452in}}{\pgfqpoint{2.021945in}{3.405061in}}{\pgfqpoint{2.014131in}{3.397248in}}%
\pgfpathcurveto{\pgfqpoint{2.006318in}{3.389434in}}{\pgfqpoint{2.001927in}{3.378835in}}{\pgfqpoint{2.001927in}{3.367785in}}%
\pgfpathcurveto{\pgfqpoint{2.001927in}{3.356735in}}{\pgfqpoint{2.006318in}{3.346136in}}{\pgfqpoint{2.014131in}{3.338322in}}%
\pgfpathcurveto{\pgfqpoint{2.021945in}{3.330509in}}{\pgfqpoint{2.032544in}{3.326118in}}{\pgfqpoint{2.043594in}{3.326118in}}%
\pgfpathclose%
\pgfusepath{stroke,fill}%
\end{pgfscope}%
\begin{pgfscope}%
\pgfpathrectangle{\pgfqpoint{0.600000in}{0.600000in}}{\pgfqpoint{3.900000in}{3.900000in}}%
\pgfusepath{clip}%
\pgfsetbuttcap%
\pgfsetroundjoin%
\definecolor{currentfill}{rgb}{0.121569,0.466667,0.705882}%
\pgfsetfillcolor{currentfill}%
\pgfsetlinewidth{1.003750pt}%
\definecolor{currentstroke}{rgb}{0.121569,0.466667,0.705882}%
\pgfsetstrokecolor{currentstroke}%
\pgfsetdash{}{0pt}%
\pgfpathmoveto{\pgfqpoint{2.783294in}{2.431534in}}%
\pgfpathcurveto{\pgfqpoint{2.794344in}{2.431534in}}{\pgfqpoint{2.804943in}{2.435924in}}{\pgfqpoint{2.812757in}{2.443738in}}%
\pgfpathcurveto{\pgfqpoint{2.820571in}{2.451552in}}{\pgfqpoint{2.824961in}{2.462151in}}{\pgfqpoint{2.824961in}{2.473201in}}%
\pgfpathcurveto{\pgfqpoint{2.824961in}{2.484251in}}{\pgfqpoint{2.820571in}{2.494850in}}{\pgfqpoint{2.812757in}{2.502664in}}%
\pgfpathcurveto{\pgfqpoint{2.804943in}{2.510477in}}{\pgfqpoint{2.794344in}{2.514867in}}{\pgfqpoint{2.783294in}{2.514867in}}%
\pgfpathcurveto{\pgfqpoint{2.772244in}{2.514867in}}{\pgfqpoint{2.761645in}{2.510477in}}{\pgfqpoint{2.753831in}{2.502664in}}%
\pgfpathcurveto{\pgfqpoint{2.746018in}{2.494850in}}{\pgfqpoint{2.741627in}{2.484251in}}{\pgfqpoint{2.741627in}{2.473201in}}%
\pgfpathcurveto{\pgfqpoint{2.741627in}{2.462151in}}{\pgfqpoint{2.746018in}{2.451552in}}{\pgfqpoint{2.753831in}{2.443738in}}%
\pgfpathcurveto{\pgfqpoint{2.761645in}{2.435924in}}{\pgfqpoint{2.772244in}{2.431534in}}{\pgfqpoint{2.783294in}{2.431534in}}%
\pgfpathclose%
\pgfusepath{stroke,fill}%
\end{pgfscope}%
\begin{pgfscope}%
\pgfpathrectangle{\pgfqpoint{0.600000in}{0.600000in}}{\pgfqpoint{3.900000in}{3.900000in}}%
\pgfusepath{clip}%
\pgfsetbuttcap%
\pgfsetroundjoin%
\definecolor{currentfill}{rgb}{0.121569,0.466667,0.705882}%
\pgfsetfillcolor{currentfill}%
\pgfsetlinewidth{1.003750pt}%
\definecolor{currentstroke}{rgb}{0.121569,0.466667,0.705882}%
\pgfsetstrokecolor{currentstroke}%
\pgfsetdash{}{0pt}%
\pgfpathmoveto{\pgfqpoint{3.122573in}{2.196664in}}%
\pgfpathcurveto{\pgfqpoint{3.133623in}{2.196664in}}{\pgfqpoint{3.144222in}{2.201054in}}{\pgfqpoint{3.152036in}{2.208868in}}%
\pgfpathcurveto{\pgfqpoint{3.159850in}{2.216681in}}{\pgfqpoint{3.164240in}{2.227280in}}{\pgfqpoint{3.164240in}{2.238331in}}%
\pgfpathcurveto{\pgfqpoint{3.164240in}{2.249381in}}{\pgfqpoint{3.159850in}{2.259980in}}{\pgfqpoint{3.152036in}{2.267793in}}%
\pgfpathcurveto{\pgfqpoint{3.144222in}{2.275607in}}{\pgfqpoint{3.133623in}{2.279997in}}{\pgfqpoint{3.122573in}{2.279997in}}%
\pgfpathcurveto{\pgfqpoint{3.111523in}{2.279997in}}{\pgfqpoint{3.100924in}{2.275607in}}{\pgfqpoint{3.093110in}{2.267793in}}%
\pgfpathcurveto{\pgfqpoint{3.085297in}{2.259980in}}{\pgfqpoint{3.080906in}{2.249381in}}{\pgfqpoint{3.080906in}{2.238331in}}%
\pgfpathcurveto{\pgfqpoint{3.080906in}{2.227280in}}{\pgfqpoint{3.085297in}{2.216681in}}{\pgfqpoint{3.093110in}{2.208868in}}%
\pgfpathcurveto{\pgfqpoint{3.100924in}{2.201054in}}{\pgfqpoint{3.111523in}{2.196664in}}{\pgfqpoint{3.122573in}{2.196664in}}%
\pgfpathclose%
\pgfusepath{stroke,fill}%
\end{pgfscope}%
\begin{pgfscope}%
\pgfpathrectangle{\pgfqpoint{0.600000in}{0.600000in}}{\pgfqpoint{3.900000in}{3.900000in}}%
\pgfusepath{clip}%
\pgfsetbuttcap%
\pgfsetroundjoin%
\definecolor{currentfill}{rgb}{0.121569,0.466667,0.705882}%
\pgfsetfillcolor{currentfill}%
\pgfsetlinewidth{1.003750pt}%
\definecolor{currentstroke}{rgb}{0.121569,0.466667,0.705882}%
\pgfsetstrokecolor{currentstroke}%
\pgfsetdash{}{0pt}%
\pgfpathmoveto{\pgfqpoint{2.937220in}{2.259462in}}%
\pgfpathcurveto{\pgfqpoint{2.948270in}{2.259462in}}{\pgfqpoint{2.958869in}{2.263852in}}{\pgfqpoint{2.966683in}{2.271666in}}%
\pgfpathcurveto{\pgfqpoint{2.974496in}{2.279479in}}{\pgfqpoint{2.978887in}{2.290078in}}{\pgfqpoint{2.978887in}{2.301128in}}%
\pgfpathcurveto{\pgfqpoint{2.978887in}{2.312179in}}{\pgfqpoint{2.974496in}{2.322778in}}{\pgfqpoint{2.966683in}{2.330591in}}%
\pgfpathcurveto{\pgfqpoint{2.958869in}{2.338405in}}{\pgfqpoint{2.948270in}{2.342795in}}{\pgfqpoint{2.937220in}{2.342795in}}%
\pgfpathcurveto{\pgfqpoint{2.926170in}{2.342795in}}{\pgfqpoint{2.915571in}{2.338405in}}{\pgfqpoint{2.907757in}{2.330591in}}%
\pgfpathcurveto{\pgfqpoint{2.899943in}{2.322778in}}{\pgfqpoint{2.895553in}{2.312179in}}{\pgfqpoint{2.895553in}{2.301128in}}%
\pgfpathcurveto{\pgfqpoint{2.895553in}{2.290078in}}{\pgfqpoint{2.899943in}{2.279479in}}{\pgfqpoint{2.907757in}{2.271666in}}%
\pgfpathcurveto{\pgfqpoint{2.915571in}{2.263852in}}{\pgfqpoint{2.926170in}{2.259462in}}{\pgfqpoint{2.937220in}{2.259462in}}%
\pgfpathclose%
\pgfusepath{stroke,fill}%
\end{pgfscope}%
\begin{pgfscope}%
\pgfpathrectangle{\pgfqpoint{0.600000in}{0.600000in}}{\pgfqpoint{3.900000in}{3.900000in}}%
\pgfusepath{clip}%
\pgfsetbuttcap%
\pgfsetroundjoin%
\definecolor{currentfill}{rgb}{0.121569,0.466667,0.705882}%
\pgfsetfillcolor{currentfill}%
\pgfsetlinewidth{1.003750pt}%
\definecolor{currentstroke}{rgb}{0.121569,0.466667,0.705882}%
\pgfsetstrokecolor{currentstroke}%
\pgfsetdash{}{0pt}%
\pgfpathmoveto{\pgfqpoint{2.428852in}{2.871416in}}%
\pgfpathcurveto{\pgfqpoint{2.439902in}{2.871416in}}{\pgfqpoint{2.450502in}{2.875806in}}{\pgfqpoint{2.458315in}{2.883620in}}%
\pgfpathcurveto{\pgfqpoint{2.466129in}{2.891433in}}{\pgfqpoint{2.470519in}{2.902032in}}{\pgfqpoint{2.470519in}{2.913083in}}%
\pgfpathcurveto{\pgfqpoint{2.470519in}{2.924133in}}{\pgfqpoint{2.466129in}{2.934732in}}{\pgfqpoint{2.458315in}{2.942545in}}%
\pgfpathcurveto{\pgfqpoint{2.450502in}{2.950359in}}{\pgfqpoint{2.439902in}{2.954749in}}{\pgfqpoint{2.428852in}{2.954749in}}%
\pgfpathcurveto{\pgfqpoint{2.417802in}{2.954749in}}{\pgfqpoint{2.407203in}{2.950359in}}{\pgfqpoint{2.399390in}{2.942545in}}%
\pgfpathcurveto{\pgfqpoint{2.391576in}{2.934732in}}{\pgfqpoint{2.387186in}{2.924133in}}{\pgfqpoint{2.387186in}{2.913083in}}%
\pgfpathcurveto{\pgfqpoint{2.387186in}{2.902032in}}{\pgfqpoint{2.391576in}{2.891433in}}{\pgfqpoint{2.399390in}{2.883620in}}%
\pgfpathcurveto{\pgfqpoint{2.407203in}{2.875806in}}{\pgfqpoint{2.417802in}{2.871416in}}{\pgfqpoint{2.428852in}{2.871416in}}%
\pgfpathclose%
\pgfusepath{stroke,fill}%
\end{pgfscope}%
\begin{pgfscope}%
\pgfpathrectangle{\pgfqpoint{0.600000in}{0.600000in}}{\pgfqpoint{3.900000in}{3.900000in}}%
\pgfusepath{clip}%
\pgfsetbuttcap%
\pgfsetroundjoin%
\definecolor{currentfill}{rgb}{0.121569,0.466667,0.705882}%
\pgfsetfillcolor{currentfill}%
\pgfsetlinewidth{1.003750pt}%
\definecolor{currentstroke}{rgb}{0.121569,0.466667,0.705882}%
\pgfsetstrokecolor{currentstroke}%
\pgfsetdash{}{0pt}%
\pgfpathmoveto{\pgfqpoint{2.399401in}{2.251782in}}%
\pgfpathcurveto{\pgfqpoint{2.410451in}{2.251782in}}{\pgfqpoint{2.421050in}{2.256172in}}{\pgfqpoint{2.428864in}{2.263986in}}%
\pgfpathcurveto{\pgfqpoint{2.436678in}{2.271800in}}{\pgfqpoint{2.441068in}{2.282399in}}{\pgfqpoint{2.441068in}{2.293449in}}%
\pgfpathcurveto{\pgfqpoint{2.441068in}{2.304499in}}{\pgfqpoint{2.436678in}{2.315098in}}{\pgfqpoint{2.428864in}{2.322912in}}%
\pgfpathcurveto{\pgfqpoint{2.421050in}{2.330725in}}{\pgfqpoint{2.410451in}{2.335115in}}{\pgfqpoint{2.399401in}{2.335115in}}%
\pgfpathcurveto{\pgfqpoint{2.388351in}{2.335115in}}{\pgfqpoint{2.377752in}{2.330725in}}{\pgfqpoint{2.369938in}{2.322912in}}%
\pgfpathcurveto{\pgfqpoint{2.362125in}{2.315098in}}{\pgfqpoint{2.357735in}{2.304499in}}{\pgfqpoint{2.357735in}{2.293449in}}%
\pgfpathcurveto{\pgfqpoint{2.357735in}{2.282399in}}{\pgfqpoint{2.362125in}{2.271800in}}{\pgfqpoint{2.369938in}{2.263986in}}%
\pgfpathcurveto{\pgfqpoint{2.377752in}{2.256172in}}{\pgfqpoint{2.388351in}{2.251782in}}{\pgfqpoint{2.399401in}{2.251782in}}%
\pgfpathclose%
\pgfusepath{stroke,fill}%
\end{pgfscope}%
\begin{pgfscope}%
\pgfpathrectangle{\pgfqpoint{0.600000in}{0.600000in}}{\pgfqpoint{3.900000in}{3.900000in}}%
\pgfusepath{clip}%
\pgfsetbuttcap%
\pgfsetroundjoin%
\definecolor{currentfill}{rgb}{0.121569,0.466667,0.705882}%
\pgfsetfillcolor{currentfill}%
\pgfsetlinewidth{1.003750pt}%
\definecolor{currentstroke}{rgb}{0.121569,0.466667,0.705882}%
\pgfsetstrokecolor{currentstroke}%
\pgfsetdash{}{0pt}%
\pgfpathmoveto{\pgfqpoint{2.641464in}{2.790268in}}%
\pgfpathcurveto{\pgfqpoint{2.652514in}{2.790268in}}{\pgfqpoint{2.663113in}{2.794658in}}{\pgfqpoint{2.670926in}{2.802472in}}%
\pgfpathcurveto{\pgfqpoint{2.678740in}{2.810286in}}{\pgfqpoint{2.683130in}{2.820885in}}{\pgfqpoint{2.683130in}{2.831935in}}%
\pgfpathcurveto{\pgfqpoint{2.683130in}{2.842985in}}{\pgfqpoint{2.678740in}{2.853584in}}{\pgfqpoint{2.670926in}{2.861398in}}%
\pgfpathcurveto{\pgfqpoint{2.663113in}{2.869211in}}{\pgfqpoint{2.652514in}{2.873601in}}{\pgfqpoint{2.641464in}{2.873601in}}%
\pgfpathcurveto{\pgfqpoint{2.630414in}{2.873601in}}{\pgfqpoint{2.619815in}{2.869211in}}{\pgfqpoint{2.612001in}{2.861398in}}%
\pgfpathcurveto{\pgfqpoint{2.604187in}{2.853584in}}{\pgfqpoint{2.599797in}{2.842985in}}{\pgfqpoint{2.599797in}{2.831935in}}%
\pgfpathcurveto{\pgfqpoint{2.599797in}{2.820885in}}{\pgfqpoint{2.604187in}{2.810286in}}{\pgfqpoint{2.612001in}{2.802472in}}%
\pgfpathcurveto{\pgfqpoint{2.619815in}{2.794658in}}{\pgfqpoint{2.630414in}{2.790268in}}{\pgfqpoint{2.641464in}{2.790268in}}%
\pgfpathclose%
\pgfusepath{stroke,fill}%
\end{pgfscope}%
\begin{pgfscope}%
\pgfpathrectangle{\pgfqpoint{0.600000in}{0.600000in}}{\pgfqpoint{3.900000in}{3.900000in}}%
\pgfusepath{clip}%
\pgfsetbuttcap%
\pgfsetroundjoin%
\definecolor{currentfill}{rgb}{0.121569,0.466667,0.705882}%
\pgfsetfillcolor{currentfill}%
\pgfsetlinewidth{1.003750pt}%
\definecolor{currentstroke}{rgb}{0.121569,0.466667,0.705882}%
\pgfsetstrokecolor{currentstroke}%
\pgfsetdash{}{0pt}%
\pgfpathmoveto{\pgfqpoint{2.720380in}{2.347582in}}%
\pgfpathcurveto{\pgfqpoint{2.731431in}{2.347582in}}{\pgfqpoint{2.742030in}{2.351972in}}{\pgfqpoint{2.749843in}{2.359786in}}%
\pgfpathcurveto{\pgfqpoint{2.757657in}{2.367599in}}{\pgfqpoint{2.762047in}{2.378198in}}{\pgfqpoint{2.762047in}{2.389249in}}%
\pgfpathcurveto{\pgfqpoint{2.762047in}{2.400299in}}{\pgfqpoint{2.757657in}{2.410898in}}{\pgfqpoint{2.749843in}{2.418711in}}%
\pgfpathcurveto{\pgfqpoint{2.742030in}{2.426525in}}{\pgfqpoint{2.731431in}{2.430915in}}{\pgfqpoint{2.720380in}{2.430915in}}%
\pgfpathcurveto{\pgfqpoint{2.709330in}{2.430915in}}{\pgfqpoint{2.698731in}{2.426525in}}{\pgfqpoint{2.690918in}{2.418711in}}%
\pgfpathcurveto{\pgfqpoint{2.683104in}{2.410898in}}{\pgfqpoint{2.678714in}{2.400299in}}{\pgfqpoint{2.678714in}{2.389249in}}%
\pgfpathcurveto{\pgfqpoint{2.678714in}{2.378198in}}{\pgfqpoint{2.683104in}{2.367599in}}{\pgfqpoint{2.690918in}{2.359786in}}%
\pgfpathcurveto{\pgfqpoint{2.698731in}{2.351972in}}{\pgfqpoint{2.709330in}{2.347582in}}{\pgfqpoint{2.720380in}{2.347582in}}%
\pgfpathclose%
\pgfusepath{stroke,fill}%
\end{pgfscope}%
\begin{pgfscope}%
\pgfpathrectangle{\pgfqpoint{0.600000in}{0.600000in}}{\pgfqpoint{3.900000in}{3.900000in}}%
\pgfusepath{clip}%
\pgfsetbuttcap%
\pgfsetroundjoin%
\definecolor{currentfill}{rgb}{0.121569,0.466667,0.705882}%
\pgfsetfillcolor{currentfill}%
\pgfsetlinewidth{1.003750pt}%
\definecolor{currentstroke}{rgb}{0.121569,0.466667,0.705882}%
\pgfsetstrokecolor{currentstroke}%
\pgfsetdash{}{0pt}%
\pgfpathmoveto{\pgfqpoint{3.023760in}{2.624199in}}%
\pgfpathcurveto{\pgfqpoint{3.034810in}{2.624199in}}{\pgfqpoint{3.045409in}{2.628589in}}{\pgfqpoint{3.053223in}{2.636403in}}%
\pgfpathcurveto{\pgfqpoint{3.061036in}{2.644217in}}{\pgfqpoint{3.065427in}{2.654816in}}{\pgfqpoint{3.065427in}{2.665866in}}%
\pgfpathcurveto{\pgfqpoint{3.065427in}{2.676916in}}{\pgfqpoint{3.061036in}{2.687515in}}{\pgfqpoint{3.053223in}{2.695329in}}%
\pgfpathcurveto{\pgfqpoint{3.045409in}{2.703142in}}{\pgfqpoint{3.034810in}{2.707532in}}{\pgfqpoint{3.023760in}{2.707532in}}%
\pgfpathcurveto{\pgfqpoint{3.012710in}{2.707532in}}{\pgfqpoint{3.002111in}{2.703142in}}{\pgfqpoint{2.994297in}{2.695329in}}%
\pgfpathcurveto{\pgfqpoint{2.986484in}{2.687515in}}{\pgfqpoint{2.982093in}{2.676916in}}{\pgfqpoint{2.982093in}{2.665866in}}%
\pgfpathcurveto{\pgfqpoint{2.982093in}{2.654816in}}{\pgfqpoint{2.986484in}{2.644217in}}{\pgfqpoint{2.994297in}{2.636403in}}%
\pgfpathcurveto{\pgfqpoint{3.002111in}{2.628589in}}{\pgfqpoint{3.012710in}{2.624199in}}{\pgfqpoint{3.023760in}{2.624199in}}%
\pgfpathclose%
\pgfusepath{stroke,fill}%
\end{pgfscope}%
\begin{pgfscope}%
\pgfpathrectangle{\pgfqpoint{0.600000in}{0.600000in}}{\pgfqpoint{3.900000in}{3.900000in}}%
\pgfusepath{clip}%
\pgfsetbuttcap%
\pgfsetroundjoin%
\definecolor{currentfill}{rgb}{0.121569,0.466667,0.705882}%
\pgfsetfillcolor{currentfill}%
\pgfsetlinewidth{1.003750pt}%
\definecolor{currentstroke}{rgb}{0.121569,0.466667,0.705882}%
\pgfsetstrokecolor{currentstroke}%
\pgfsetdash{}{0pt}%
\pgfpathmoveto{\pgfqpoint{2.773609in}{3.183913in}}%
\pgfpathcurveto{\pgfqpoint{2.784660in}{3.183913in}}{\pgfqpoint{2.795259in}{3.188303in}}{\pgfqpoint{2.803072in}{3.196116in}}%
\pgfpathcurveto{\pgfqpoint{2.810886in}{3.203930in}}{\pgfqpoint{2.815276in}{3.214529in}}{\pgfqpoint{2.815276in}{3.225579in}}%
\pgfpathcurveto{\pgfqpoint{2.815276in}{3.236629in}}{\pgfqpoint{2.810886in}{3.247228in}}{\pgfqpoint{2.803072in}{3.255042in}}%
\pgfpathcurveto{\pgfqpoint{2.795259in}{3.262856in}}{\pgfqpoint{2.784660in}{3.267246in}}{\pgfqpoint{2.773609in}{3.267246in}}%
\pgfpathcurveto{\pgfqpoint{2.762559in}{3.267246in}}{\pgfqpoint{2.751960in}{3.262856in}}{\pgfqpoint{2.744147in}{3.255042in}}%
\pgfpathcurveto{\pgfqpoint{2.736333in}{3.247228in}}{\pgfqpoint{2.731943in}{3.236629in}}{\pgfqpoint{2.731943in}{3.225579in}}%
\pgfpathcurveto{\pgfqpoint{2.731943in}{3.214529in}}{\pgfqpoint{2.736333in}{3.203930in}}{\pgfqpoint{2.744147in}{3.196116in}}%
\pgfpathcurveto{\pgfqpoint{2.751960in}{3.188303in}}{\pgfqpoint{2.762559in}{3.183913in}}{\pgfqpoint{2.773609in}{3.183913in}}%
\pgfpathclose%
\pgfusepath{stroke,fill}%
\end{pgfscope}%
\begin{pgfscope}%
\pgfpathrectangle{\pgfqpoint{0.600000in}{0.600000in}}{\pgfqpoint{3.900000in}{3.900000in}}%
\pgfusepath{clip}%
\pgfsetbuttcap%
\pgfsetroundjoin%
\definecolor{currentfill}{rgb}{0.121569,0.466667,0.705882}%
\pgfsetfillcolor{currentfill}%
\pgfsetlinewidth{1.003750pt}%
\definecolor{currentstroke}{rgb}{0.121569,0.466667,0.705882}%
\pgfsetstrokecolor{currentstroke}%
\pgfsetdash{}{0pt}%
\pgfpathmoveto{\pgfqpoint{1.414451in}{2.613481in}}%
\pgfpathcurveto{\pgfqpoint{1.425501in}{2.613481in}}{\pgfqpoint{1.436100in}{2.617871in}}{\pgfqpoint{1.443914in}{2.625685in}}%
\pgfpathcurveto{\pgfqpoint{1.451727in}{2.633499in}}{\pgfqpoint{1.456117in}{2.644098in}}{\pgfqpoint{1.456117in}{2.655148in}}%
\pgfpathcurveto{\pgfqpoint{1.456117in}{2.666198in}}{\pgfqpoint{1.451727in}{2.676797in}}{\pgfqpoint{1.443914in}{2.684611in}}%
\pgfpathcurveto{\pgfqpoint{1.436100in}{2.692424in}}{\pgfqpoint{1.425501in}{2.696815in}}{\pgfqpoint{1.414451in}{2.696815in}}%
\pgfpathcurveto{\pgfqpoint{1.403401in}{2.696815in}}{\pgfqpoint{1.392802in}{2.692424in}}{\pgfqpoint{1.384988in}{2.684611in}}%
\pgfpathcurveto{\pgfqpoint{1.377174in}{2.676797in}}{\pgfqpoint{1.372784in}{2.666198in}}{\pgfqpoint{1.372784in}{2.655148in}}%
\pgfpathcurveto{\pgfqpoint{1.372784in}{2.644098in}}{\pgfqpoint{1.377174in}{2.633499in}}{\pgfqpoint{1.384988in}{2.625685in}}%
\pgfpathcurveto{\pgfqpoint{1.392802in}{2.617871in}}{\pgfqpoint{1.403401in}{2.613481in}}{\pgfqpoint{1.414451in}{2.613481in}}%
\pgfpathclose%
\pgfusepath{stroke,fill}%
\end{pgfscope}%
\begin{pgfscope}%
\pgfpathrectangle{\pgfqpoint{0.600000in}{0.600000in}}{\pgfqpoint{3.900000in}{3.900000in}}%
\pgfusepath{clip}%
\pgfsetbuttcap%
\pgfsetroundjoin%
\definecolor{currentfill}{rgb}{0.121569,0.466667,0.705882}%
\pgfsetfillcolor{currentfill}%
\pgfsetlinewidth{1.003750pt}%
\definecolor{currentstroke}{rgb}{0.121569,0.466667,0.705882}%
\pgfsetstrokecolor{currentstroke}%
\pgfsetdash{}{0pt}%
\pgfpathmoveto{\pgfqpoint{2.464222in}{3.049776in}}%
\pgfpathcurveto{\pgfqpoint{2.475272in}{3.049776in}}{\pgfqpoint{2.485871in}{3.054166in}}{\pgfqpoint{2.493684in}{3.061980in}}%
\pgfpathcurveto{\pgfqpoint{2.501498in}{3.069794in}}{\pgfqpoint{2.505888in}{3.080393in}}{\pgfqpoint{2.505888in}{3.091443in}}%
\pgfpathcurveto{\pgfqpoint{2.505888in}{3.102493in}}{\pgfqpoint{2.501498in}{3.113092in}}{\pgfqpoint{2.493684in}{3.120906in}}%
\pgfpathcurveto{\pgfqpoint{2.485871in}{3.128719in}}{\pgfqpoint{2.475272in}{3.133110in}}{\pgfqpoint{2.464222in}{3.133110in}}%
\pgfpathcurveto{\pgfqpoint{2.453172in}{3.133110in}}{\pgfqpoint{2.442572in}{3.128719in}}{\pgfqpoint{2.434759in}{3.120906in}}%
\pgfpathcurveto{\pgfqpoint{2.426945in}{3.113092in}}{\pgfqpoint{2.422555in}{3.102493in}}{\pgfqpoint{2.422555in}{3.091443in}}%
\pgfpathcurveto{\pgfqpoint{2.422555in}{3.080393in}}{\pgfqpoint{2.426945in}{3.069794in}}{\pgfqpoint{2.434759in}{3.061980in}}%
\pgfpathcurveto{\pgfqpoint{2.442572in}{3.054166in}}{\pgfqpoint{2.453172in}{3.049776in}}{\pgfqpoint{2.464222in}{3.049776in}}%
\pgfpathclose%
\pgfusepath{stroke,fill}%
\end{pgfscope}%
\begin{pgfscope}%
\pgfpathrectangle{\pgfqpoint{0.600000in}{0.600000in}}{\pgfqpoint{3.900000in}{3.900000in}}%
\pgfusepath{clip}%
\pgfsetbuttcap%
\pgfsetroundjoin%
\definecolor{currentfill}{rgb}{0.121569,0.466667,0.705882}%
\pgfsetfillcolor{currentfill}%
\pgfsetlinewidth{1.003750pt}%
\definecolor{currentstroke}{rgb}{0.121569,0.466667,0.705882}%
\pgfsetstrokecolor{currentstroke}%
\pgfsetdash{}{0pt}%
\pgfpathmoveto{\pgfqpoint{2.892605in}{2.437218in}}%
\pgfpathcurveto{\pgfqpoint{2.903655in}{2.437218in}}{\pgfqpoint{2.914254in}{2.441609in}}{\pgfqpoint{2.922068in}{2.449422in}}%
\pgfpathcurveto{\pgfqpoint{2.929881in}{2.457236in}}{\pgfqpoint{2.934272in}{2.467835in}}{\pgfqpoint{2.934272in}{2.478885in}}%
\pgfpathcurveto{\pgfqpoint{2.934272in}{2.489935in}}{\pgfqpoint{2.929881in}{2.500534in}}{\pgfqpoint{2.922068in}{2.508348in}}%
\pgfpathcurveto{\pgfqpoint{2.914254in}{2.516162in}}{\pgfqpoint{2.903655in}{2.520552in}}{\pgfqpoint{2.892605in}{2.520552in}}%
\pgfpathcurveto{\pgfqpoint{2.881555in}{2.520552in}}{\pgfqpoint{2.870956in}{2.516162in}}{\pgfqpoint{2.863142in}{2.508348in}}%
\pgfpathcurveto{\pgfqpoint{2.855328in}{2.500534in}}{\pgfqpoint{2.850938in}{2.489935in}}{\pgfqpoint{2.850938in}{2.478885in}}%
\pgfpathcurveto{\pgfqpoint{2.850938in}{2.467835in}}{\pgfqpoint{2.855328in}{2.457236in}}{\pgfqpoint{2.863142in}{2.449422in}}%
\pgfpathcurveto{\pgfqpoint{2.870956in}{2.441609in}}{\pgfqpoint{2.881555in}{2.437218in}}{\pgfqpoint{2.892605in}{2.437218in}}%
\pgfpathclose%
\pgfusepath{stroke,fill}%
\end{pgfscope}%
\begin{pgfscope}%
\pgfpathrectangle{\pgfqpoint{0.600000in}{0.600000in}}{\pgfqpoint{3.900000in}{3.900000in}}%
\pgfusepath{clip}%
\pgfsetbuttcap%
\pgfsetroundjoin%
\definecolor{currentfill}{rgb}{0.121569,0.466667,0.705882}%
\pgfsetfillcolor{currentfill}%
\pgfsetlinewidth{1.003750pt}%
\definecolor{currentstroke}{rgb}{0.121569,0.466667,0.705882}%
\pgfsetstrokecolor{currentstroke}%
\pgfsetdash{}{0pt}%
\pgfpathmoveto{\pgfqpoint{1.854784in}{3.162272in}}%
\pgfpathcurveto{\pgfqpoint{1.865834in}{3.162272in}}{\pgfqpoint{1.876433in}{3.166663in}}{\pgfqpoint{1.884247in}{3.174476in}}%
\pgfpathcurveto{\pgfqpoint{1.892060in}{3.182290in}}{\pgfqpoint{1.896451in}{3.192889in}}{\pgfqpoint{1.896451in}{3.203939in}}%
\pgfpathcurveto{\pgfqpoint{1.896451in}{3.214989in}}{\pgfqpoint{1.892060in}{3.225588in}}{\pgfqpoint{1.884247in}{3.233402in}}%
\pgfpathcurveto{\pgfqpoint{1.876433in}{3.241215in}}{\pgfqpoint{1.865834in}{3.245606in}}{\pgfqpoint{1.854784in}{3.245606in}}%
\pgfpathcurveto{\pgfqpoint{1.843734in}{3.245606in}}{\pgfqpoint{1.833135in}{3.241215in}}{\pgfqpoint{1.825321in}{3.233402in}}%
\pgfpathcurveto{\pgfqpoint{1.817507in}{3.225588in}}{\pgfqpoint{1.813117in}{3.214989in}}{\pgfqpoint{1.813117in}{3.203939in}}%
\pgfpathcurveto{\pgfqpoint{1.813117in}{3.192889in}}{\pgfqpoint{1.817507in}{3.182290in}}{\pgfqpoint{1.825321in}{3.174476in}}%
\pgfpathcurveto{\pgfqpoint{1.833135in}{3.166663in}}{\pgfqpoint{1.843734in}{3.162272in}}{\pgfqpoint{1.854784in}{3.162272in}}%
\pgfpathclose%
\pgfusepath{stroke,fill}%
\end{pgfscope}%
\begin{pgfscope}%
\pgfpathrectangle{\pgfqpoint{0.600000in}{0.600000in}}{\pgfqpoint{3.900000in}{3.900000in}}%
\pgfusepath{clip}%
\pgfsetbuttcap%
\pgfsetroundjoin%
\definecolor{currentfill}{rgb}{0.121569,0.466667,0.705882}%
\pgfsetfillcolor{currentfill}%
\pgfsetlinewidth{1.003750pt}%
\definecolor{currentstroke}{rgb}{0.121569,0.466667,0.705882}%
\pgfsetstrokecolor{currentstroke}%
\pgfsetdash{}{0pt}%
\pgfpathmoveto{\pgfqpoint{1.878022in}{2.632401in}}%
\pgfpathcurveto{\pgfqpoint{1.889072in}{2.632401in}}{\pgfqpoint{1.899671in}{2.636792in}}{\pgfqpoint{1.907485in}{2.644605in}}%
\pgfpathcurveto{\pgfqpoint{1.915299in}{2.652419in}}{\pgfqpoint{1.919689in}{2.663018in}}{\pgfqpoint{1.919689in}{2.674068in}}%
\pgfpathcurveto{\pgfqpoint{1.919689in}{2.685118in}}{\pgfqpoint{1.915299in}{2.695717in}}{\pgfqpoint{1.907485in}{2.703531in}}%
\pgfpathcurveto{\pgfqpoint{1.899671in}{2.711344in}}{\pgfqpoint{1.889072in}{2.715735in}}{\pgfqpoint{1.878022in}{2.715735in}}%
\pgfpathcurveto{\pgfqpoint{1.866972in}{2.715735in}}{\pgfqpoint{1.856373in}{2.711344in}}{\pgfqpoint{1.848559in}{2.703531in}}%
\pgfpathcurveto{\pgfqpoint{1.840746in}{2.695717in}}{\pgfqpoint{1.836355in}{2.685118in}}{\pgfqpoint{1.836355in}{2.674068in}}%
\pgfpathcurveto{\pgfqpoint{1.836355in}{2.663018in}}{\pgfqpoint{1.840746in}{2.652419in}}{\pgfqpoint{1.848559in}{2.644605in}}%
\pgfpathcurveto{\pgfqpoint{1.856373in}{2.636792in}}{\pgfqpoint{1.866972in}{2.632401in}}{\pgfqpoint{1.878022in}{2.632401in}}%
\pgfpathclose%
\pgfusepath{stroke,fill}%
\end{pgfscope}%
\begin{pgfscope}%
\pgfpathrectangle{\pgfqpoint{0.600000in}{0.600000in}}{\pgfqpoint{3.900000in}{3.900000in}}%
\pgfusepath{clip}%
\pgfsetbuttcap%
\pgfsetroundjoin%
\definecolor{currentfill}{rgb}{0.121569,0.466667,0.705882}%
\pgfsetfillcolor{currentfill}%
\pgfsetlinewidth{1.003750pt}%
\definecolor{currentstroke}{rgb}{0.121569,0.466667,0.705882}%
\pgfsetstrokecolor{currentstroke}%
\pgfsetdash{}{0pt}%
\pgfpathmoveto{\pgfqpoint{2.743139in}{1.532688in}}%
\pgfpathcurveto{\pgfqpoint{2.754189in}{1.532688in}}{\pgfqpoint{2.764788in}{1.537078in}}{\pgfqpoint{2.772601in}{1.544892in}}%
\pgfpathcurveto{\pgfqpoint{2.780415in}{1.552706in}}{\pgfqpoint{2.784805in}{1.563305in}}{\pgfqpoint{2.784805in}{1.574355in}}%
\pgfpathcurveto{\pgfqpoint{2.784805in}{1.585405in}}{\pgfqpoint{2.780415in}{1.596004in}}{\pgfqpoint{2.772601in}{1.603817in}}%
\pgfpathcurveto{\pgfqpoint{2.764788in}{1.611631in}}{\pgfqpoint{2.754189in}{1.616021in}}{\pgfqpoint{2.743139in}{1.616021in}}%
\pgfpathcurveto{\pgfqpoint{2.732089in}{1.616021in}}{\pgfqpoint{2.721489in}{1.611631in}}{\pgfqpoint{2.713676in}{1.603817in}}%
\pgfpathcurveto{\pgfqpoint{2.705862in}{1.596004in}}{\pgfqpoint{2.701472in}{1.585405in}}{\pgfqpoint{2.701472in}{1.574355in}}%
\pgfpathcurveto{\pgfqpoint{2.701472in}{1.563305in}}{\pgfqpoint{2.705862in}{1.552706in}}{\pgfqpoint{2.713676in}{1.544892in}}%
\pgfpathcurveto{\pgfqpoint{2.721489in}{1.537078in}}{\pgfqpoint{2.732089in}{1.532688in}}{\pgfqpoint{2.743139in}{1.532688in}}%
\pgfpathclose%
\pgfusepath{stroke,fill}%
\end{pgfscope}%
\begin{pgfscope}%
\pgfpathrectangle{\pgfqpoint{0.600000in}{0.600000in}}{\pgfqpoint{3.900000in}{3.900000in}}%
\pgfusepath{clip}%
\pgfsetbuttcap%
\pgfsetroundjoin%
\definecolor{currentfill}{rgb}{0.121569,0.466667,0.705882}%
\pgfsetfillcolor{currentfill}%
\pgfsetlinewidth{1.003750pt}%
\definecolor{currentstroke}{rgb}{0.121569,0.466667,0.705882}%
\pgfsetstrokecolor{currentstroke}%
\pgfsetdash{}{0pt}%
\pgfpathmoveto{\pgfqpoint{2.601617in}{2.137190in}}%
\pgfpathcurveto{\pgfqpoint{2.612667in}{2.137190in}}{\pgfqpoint{2.623266in}{2.141580in}}{\pgfqpoint{2.631080in}{2.149394in}}%
\pgfpathcurveto{\pgfqpoint{2.638893in}{2.157208in}}{\pgfqpoint{2.643283in}{2.167807in}}{\pgfqpoint{2.643283in}{2.178857in}}%
\pgfpathcurveto{\pgfqpoint{2.643283in}{2.189907in}}{\pgfqpoint{2.638893in}{2.200506in}}{\pgfqpoint{2.631080in}{2.208319in}}%
\pgfpathcurveto{\pgfqpoint{2.623266in}{2.216133in}}{\pgfqpoint{2.612667in}{2.220523in}}{\pgfqpoint{2.601617in}{2.220523in}}%
\pgfpathcurveto{\pgfqpoint{2.590567in}{2.220523in}}{\pgfqpoint{2.579968in}{2.216133in}}{\pgfqpoint{2.572154in}{2.208319in}}%
\pgfpathcurveto{\pgfqpoint{2.564340in}{2.200506in}}{\pgfqpoint{2.559950in}{2.189907in}}{\pgfqpoint{2.559950in}{2.178857in}}%
\pgfpathcurveto{\pgfqpoint{2.559950in}{2.167807in}}{\pgfqpoint{2.564340in}{2.157208in}}{\pgfqpoint{2.572154in}{2.149394in}}%
\pgfpathcurveto{\pgfqpoint{2.579968in}{2.141580in}}{\pgfqpoint{2.590567in}{2.137190in}}{\pgfqpoint{2.601617in}{2.137190in}}%
\pgfpathclose%
\pgfusepath{stroke,fill}%
\end{pgfscope}%
\begin{pgfscope}%
\pgfpathrectangle{\pgfqpoint{0.600000in}{0.600000in}}{\pgfqpoint{3.900000in}{3.900000in}}%
\pgfusepath{clip}%
\pgfsetbuttcap%
\pgfsetroundjoin%
\definecolor{currentfill}{rgb}{0.121569,0.466667,0.705882}%
\pgfsetfillcolor{currentfill}%
\pgfsetlinewidth{1.003750pt}%
\definecolor{currentstroke}{rgb}{0.121569,0.466667,0.705882}%
\pgfsetstrokecolor{currentstroke}%
\pgfsetdash{}{0pt}%
\pgfpathmoveto{\pgfqpoint{1.735137in}{2.228960in}}%
\pgfpathcurveto{\pgfqpoint{1.746187in}{2.228960in}}{\pgfqpoint{1.756786in}{2.233350in}}{\pgfqpoint{1.764599in}{2.241164in}}%
\pgfpathcurveto{\pgfqpoint{1.772413in}{2.248978in}}{\pgfqpoint{1.776803in}{2.259577in}}{\pgfqpoint{1.776803in}{2.270627in}}%
\pgfpathcurveto{\pgfqpoint{1.776803in}{2.281677in}}{\pgfqpoint{1.772413in}{2.292276in}}{\pgfqpoint{1.764599in}{2.300090in}}%
\pgfpathcurveto{\pgfqpoint{1.756786in}{2.307903in}}{\pgfqpoint{1.746187in}{2.312293in}}{\pgfqpoint{1.735137in}{2.312293in}}%
\pgfpathcurveto{\pgfqpoint{1.724086in}{2.312293in}}{\pgfqpoint{1.713487in}{2.307903in}}{\pgfqpoint{1.705674in}{2.300090in}}%
\pgfpathcurveto{\pgfqpoint{1.697860in}{2.292276in}}{\pgfqpoint{1.693470in}{2.281677in}}{\pgfqpoint{1.693470in}{2.270627in}}%
\pgfpathcurveto{\pgfqpoint{1.693470in}{2.259577in}}{\pgfqpoint{1.697860in}{2.248978in}}{\pgfqpoint{1.705674in}{2.241164in}}%
\pgfpathcurveto{\pgfqpoint{1.713487in}{2.233350in}}{\pgfqpoint{1.724086in}{2.228960in}}{\pgfqpoint{1.735137in}{2.228960in}}%
\pgfpathclose%
\pgfusepath{stroke,fill}%
\end{pgfscope}%
\begin{pgfscope}%
\pgfpathrectangle{\pgfqpoint{0.600000in}{0.600000in}}{\pgfqpoint{3.900000in}{3.900000in}}%
\pgfusepath{clip}%
\pgfsetbuttcap%
\pgfsetroundjoin%
\definecolor{currentfill}{rgb}{0.121569,0.466667,0.705882}%
\pgfsetfillcolor{currentfill}%
\pgfsetlinewidth{1.003750pt}%
\definecolor{currentstroke}{rgb}{0.121569,0.466667,0.705882}%
\pgfsetstrokecolor{currentstroke}%
\pgfsetdash{}{0pt}%
\pgfpathmoveto{\pgfqpoint{1.842476in}{1.773059in}}%
\pgfpathcurveto{\pgfqpoint{1.853526in}{1.773059in}}{\pgfqpoint{1.864125in}{1.777450in}}{\pgfqpoint{1.871939in}{1.785263in}}%
\pgfpathcurveto{\pgfqpoint{1.879752in}{1.793077in}}{\pgfqpoint{1.884143in}{1.803676in}}{\pgfqpoint{1.884143in}{1.814726in}}%
\pgfpathcurveto{\pgfqpoint{1.884143in}{1.825776in}}{\pgfqpoint{1.879752in}{1.836375in}}{\pgfqpoint{1.871939in}{1.844189in}}%
\pgfpathcurveto{\pgfqpoint{1.864125in}{1.852002in}}{\pgfqpoint{1.853526in}{1.856393in}}{\pgfqpoint{1.842476in}{1.856393in}}%
\pgfpathcurveto{\pgfqpoint{1.831426in}{1.856393in}}{\pgfqpoint{1.820827in}{1.852002in}}{\pgfqpoint{1.813013in}{1.844189in}}%
\pgfpathcurveto{\pgfqpoint{1.805200in}{1.836375in}}{\pgfqpoint{1.800809in}{1.825776in}}{\pgfqpoint{1.800809in}{1.814726in}}%
\pgfpathcurveto{\pgfqpoint{1.800809in}{1.803676in}}{\pgfqpoint{1.805200in}{1.793077in}}{\pgfqpoint{1.813013in}{1.785263in}}%
\pgfpathcurveto{\pgfqpoint{1.820827in}{1.777450in}}{\pgfqpoint{1.831426in}{1.773059in}}{\pgfqpoint{1.842476in}{1.773059in}}%
\pgfpathclose%
\pgfusepath{stroke,fill}%
\end{pgfscope}%
\begin{pgfscope}%
\pgfpathrectangle{\pgfqpoint{0.600000in}{0.600000in}}{\pgfqpoint{3.900000in}{3.900000in}}%
\pgfusepath{clip}%
\pgfsetbuttcap%
\pgfsetroundjoin%
\definecolor{currentfill}{rgb}{0.121569,0.466667,0.705882}%
\pgfsetfillcolor{currentfill}%
\pgfsetlinewidth{1.003750pt}%
\definecolor{currentstroke}{rgb}{0.121569,0.466667,0.705882}%
\pgfsetstrokecolor{currentstroke}%
\pgfsetdash{}{0pt}%
\pgfpathmoveto{\pgfqpoint{3.852645in}{2.561714in}}%
\pgfpathcurveto{\pgfqpoint{3.863695in}{2.561714in}}{\pgfqpoint{3.874294in}{2.566104in}}{\pgfqpoint{3.882108in}{2.573918in}}%
\pgfpathcurveto{\pgfqpoint{3.889922in}{2.581731in}}{\pgfqpoint{3.894312in}{2.592330in}}{\pgfqpoint{3.894312in}{2.603381in}}%
\pgfpathcurveto{\pgfqpoint{3.894312in}{2.614431in}}{\pgfqpoint{3.889922in}{2.625030in}}{\pgfqpoint{3.882108in}{2.632843in}}%
\pgfpathcurveto{\pgfqpoint{3.874294in}{2.640657in}}{\pgfqpoint{3.863695in}{2.645047in}}{\pgfqpoint{3.852645in}{2.645047in}}%
\pgfpathcurveto{\pgfqpoint{3.841595in}{2.645047in}}{\pgfqpoint{3.830996in}{2.640657in}}{\pgfqpoint{3.823183in}{2.632843in}}%
\pgfpathcurveto{\pgfqpoint{3.815369in}{2.625030in}}{\pgfqpoint{3.810979in}{2.614431in}}{\pgfqpoint{3.810979in}{2.603381in}}%
\pgfpathcurveto{\pgfqpoint{3.810979in}{2.592330in}}{\pgfqpoint{3.815369in}{2.581731in}}{\pgfqpoint{3.823183in}{2.573918in}}%
\pgfpathcurveto{\pgfqpoint{3.830996in}{2.566104in}}{\pgfqpoint{3.841595in}{2.561714in}}{\pgfqpoint{3.852645in}{2.561714in}}%
\pgfpathclose%
\pgfusepath{stroke,fill}%
\end{pgfscope}%
\begin{pgfscope}%
\pgfpathrectangle{\pgfqpoint{0.600000in}{0.600000in}}{\pgfqpoint{3.900000in}{3.900000in}}%
\pgfusepath{clip}%
\pgfsetbuttcap%
\pgfsetroundjoin%
\definecolor{currentfill}{rgb}{0.121569,0.466667,0.705882}%
\pgfsetfillcolor{currentfill}%
\pgfsetlinewidth{1.003750pt}%
\definecolor{currentstroke}{rgb}{0.121569,0.466667,0.705882}%
\pgfsetstrokecolor{currentstroke}%
\pgfsetdash{}{0pt}%
\pgfpathmoveto{\pgfqpoint{2.923334in}{2.030125in}}%
\pgfpathcurveto{\pgfqpoint{2.934384in}{2.030125in}}{\pgfqpoint{2.944983in}{2.034516in}}{\pgfqpoint{2.952797in}{2.042329in}}%
\pgfpathcurveto{\pgfqpoint{2.960610in}{2.050143in}}{\pgfqpoint{2.965001in}{2.060742in}}{\pgfqpoint{2.965001in}{2.071792in}}%
\pgfpathcurveto{\pgfqpoint{2.965001in}{2.082842in}}{\pgfqpoint{2.960610in}{2.093441in}}{\pgfqpoint{2.952797in}{2.101255in}}%
\pgfpathcurveto{\pgfqpoint{2.944983in}{2.109068in}}{\pgfqpoint{2.934384in}{2.113459in}}{\pgfqpoint{2.923334in}{2.113459in}}%
\pgfpathcurveto{\pgfqpoint{2.912284in}{2.113459in}}{\pgfqpoint{2.901685in}{2.109068in}}{\pgfqpoint{2.893871in}{2.101255in}}%
\pgfpathcurveto{\pgfqpoint{2.886058in}{2.093441in}}{\pgfqpoint{2.881667in}{2.082842in}}{\pgfqpoint{2.881667in}{2.071792in}}%
\pgfpathcurveto{\pgfqpoint{2.881667in}{2.060742in}}{\pgfqpoint{2.886058in}{2.050143in}}{\pgfqpoint{2.893871in}{2.042329in}}%
\pgfpathcurveto{\pgfqpoint{2.901685in}{2.034516in}}{\pgfqpoint{2.912284in}{2.030125in}}{\pgfqpoint{2.923334in}{2.030125in}}%
\pgfpathclose%
\pgfusepath{stroke,fill}%
\end{pgfscope}%
\begin{pgfscope}%
\pgfpathrectangle{\pgfqpoint{0.600000in}{0.600000in}}{\pgfqpoint{3.900000in}{3.900000in}}%
\pgfusepath{clip}%
\pgfsetbuttcap%
\pgfsetroundjoin%
\definecolor{currentfill}{rgb}{0.121569,0.466667,0.705882}%
\pgfsetfillcolor{currentfill}%
\pgfsetlinewidth{1.003750pt}%
\definecolor{currentstroke}{rgb}{0.121569,0.466667,0.705882}%
\pgfsetstrokecolor{currentstroke}%
\pgfsetdash{}{0pt}%
\pgfpathmoveto{\pgfqpoint{2.450749in}{1.723696in}}%
\pgfpathcurveto{\pgfqpoint{2.461799in}{1.723696in}}{\pgfqpoint{2.472398in}{1.728086in}}{\pgfqpoint{2.480212in}{1.735900in}}%
\pgfpathcurveto{\pgfqpoint{2.488025in}{1.743713in}}{\pgfqpoint{2.492416in}{1.754312in}}{\pgfqpoint{2.492416in}{1.765363in}}%
\pgfpathcurveto{\pgfqpoint{2.492416in}{1.776413in}}{\pgfqpoint{2.488025in}{1.787012in}}{\pgfqpoint{2.480212in}{1.794825in}}%
\pgfpathcurveto{\pgfqpoint{2.472398in}{1.802639in}}{\pgfqpoint{2.461799in}{1.807029in}}{\pgfqpoint{2.450749in}{1.807029in}}%
\pgfpathcurveto{\pgfqpoint{2.439699in}{1.807029in}}{\pgfqpoint{2.429100in}{1.802639in}}{\pgfqpoint{2.421286in}{1.794825in}}%
\pgfpathcurveto{\pgfqpoint{2.413473in}{1.787012in}}{\pgfqpoint{2.409082in}{1.776413in}}{\pgfqpoint{2.409082in}{1.765363in}}%
\pgfpathcurveto{\pgfqpoint{2.409082in}{1.754312in}}{\pgfqpoint{2.413473in}{1.743713in}}{\pgfqpoint{2.421286in}{1.735900in}}%
\pgfpathcurveto{\pgfqpoint{2.429100in}{1.728086in}}{\pgfqpoint{2.439699in}{1.723696in}}{\pgfqpoint{2.450749in}{1.723696in}}%
\pgfpathclose%
\pgfusepath{stroke,fill}%
\end{pgfscope}%
\begin{pgfscope}%
\pgfpathrectangle{\pgfqpoint{0.600000in}{0.600000in}}{\pgfqpoint{3.900000in}{3.900000in}}%
\pgfusepath{clip}%
\pgfsetbuttcap%
\pgfsetroundjoin%
\definecolor{currentfill}{rgb}{0.121569,0.466667,0.705882}%
\pgfsetfillcolor{currentfill}%
\pgfsetlinewidth{1.003750pt}%
\definecolor{currentstroke}{rgb}{0.121569,0.466667,0.705882}%
\pgfsetstrokecolor{currentstroke}%
\pgfsetdash{}{0pt}%
\pgfpathmoveto{\pgfqpoint{2.159338in}{2.617531in}}%
\pgfpathcurveto{\pgfqpoint{2.170389in}{2.617531in}}{\pgfqpoint{2.180988in}{2.621922in}}{\pgfqpoint{2.188801in}{2.629735in}}%
\pgfpathcurveto{\pgfqpoint{2.196615in}{2.637549in}}{\pgfqpoint{2.201005in}{2.648148in}}{\pgfqpoint{2.201005in}{2.659198in}}%
\pgfpathcurveto{\pgfqpoint{2.201005in}{2.670248in}}{\pgfqpoint{2.196615in}{2.680847in}}{\pgfqpoint{2.188801in}{2.688661in}}%
\pgfpathcurveto{\pgfqpoint{2.180988in}{2.696474in}}{\pgfqpoint{2.170389in}{2.700865in}}{\pgfqpoint{2.159338in}{2.700865in}}%
\pgfpathcurveto{\pgfqpoint{2.148288in}{2.700865in}}{\pgfqpoint{2.137689in}{2.696474in}}{\pgfqpoint{2.129876in}{2.688661in}}%
\pgfpathcurveto{\pgfqpoint{2.122062in}{2.680847in}}{\pgfqpoint{2.117672in}{2.670248in}}{\pgfqpoint{2.117672in}{2.659198in}}%
\pgfpathcurveto{\pgfqpoint{2.117672in}{2.648148in}}{\pgfqpoint{2.122062in}{2.637549in}}{\pgfqpoint{2.129876in}{2.629735in}}%
\pgfpathcurveto{\pgfqpoint{2.137689in}{2.621922in}}{\pgfqpoint{2.148288in}{2.617531in}}{\pgfqpoint{2.159338in}{2.617531in}}%
\pgfpathclose%
\pgfusepath{stroke,fill}%
\end{pgfscope}%
\begin{pgfscope}%
\pgfpathrectangle{\pgfqpoint{0.600000in}{0.600000in}}{\pgfqpoint{3.900000in}{3.900000in}}%
\pgfusepath{clip}%
\pgfsetbuttcap%
\pgfsetroundjoin%
\definecolor{currentfill}{rgb}{0.121569,0.466667,0.705882}%
\pgfsetfillcolor{currentfill}%
\pgfsetlinewidth{1.003750pt}%
\definecolor{currentstroke}{rgb}{0.121569,0.466667,0.705882}%
\pgfsetstrokecolor{currentstroke}%
\pgfsetdash{}{0pt}%
\pgfpathmoveto{\pgfqpoint{2.693629in}{1.998404in}}%
\pgfpathcurveto{\pgfqpoint{2.704679in}{1.998404in}}{\pgfqpoint{2.715278in}{2.002794in}}{\pgfqpoint{2.723092in}{2.010608in}}%
\pgfpathcurveto{\pgfqpoint{2.730906in}{2.018422in}}{\pgfqpoint{2.735296in}{2.029021in}}{\pgfqpoint{2.735296in}{2.040071in}}%
\pgfpathcurveto{\pgfqpoint{2.735296in}{2.051121in}}{\pgfqpoint{2.730906in}{2.061720in}}{\pgfqpoint{2.723092in}{2.069534in}}%
\pgfpathcurveto{\pgfqpoint{2.715278in}{2.077347in}}{\pgfqpoint{2.704679in}{2.081737in}}{\pgfqpoint{2.693629in}{2.081737in}}%
\pgfpathcurveto{\pgfqpoint{2.682579in}{2.081737in}}{\pgfqpoint{2.671980in}{2.077347in}}{\pgfqpoint{2.664166in}{2.069534in}}%
\pgfpathcurveto{\pgfqpoint{2.656353in}{2.061720in}}{\pgfqpoint{2.651963in}{2.051121in}}{\pgfqpoint{2.651963in}{2.040071in}}%
\pgfpathcurveto{\pgfqpoint{2.651963in}{2.029021in}}{\pgfqpoint{2.656353in}{2.018422in}}{\pgfqpoint{2.664166in}{2.010608in}}%
\pgfpathcurveto{\pgfqpoint{2.671980in}{2.002794in}}{\pgfqpoint{2.682579in}{1.998404in}}{\pgfqpoint{2.693629in}{1.998404in}}%
\pgfpathclose%
\pgfusepath{stroke,fill}%
\end{pgfscope}%
\begin{pgfscope}%
\pgfpathrectangle{\pgfqpoint{0.600000in}{0.600000in}}{\pgfqpoint{3.900000in}{3.900000in}}%
\pgfusepath{clip}%
\pgfsetbuttcap%
\pgfsetroundjoin%
\definecolor{currentfill}{rgb}{0.121569,0.466667,0.705882}%
\pgfsetfillcolor{currentfill}%
\pgfsetlinewidth{1.003750pt}%
\definecolor{currentstroke}{rgb}{0.121569,0.466667,0.705882}%
\pgfsetstrokecolor{currentstroke}%
\pgfsetdash{}{0pt}%
\pgfpathmoveto{\pgfqpoint{2.459840in}{2.257082in}}%
\pgfpathcurveto{\pgfqpoint{2.470890in}{2.257082in}}{\pgfqpoint{2.481489in}{2.261472in}}{\pgfqpoint{2.489303in}{2.269285in}}%
\pgfpathcurveto{\pgfqpoint{2.497117in}{2.277099in}}{\pgfqpoint{2.501507in}{2.287698in}}{\pgfqpoint{2.501507in}{2.298748in}}%
\pgfpathcurveto{\pgfqpoint{2.501507in}{2.309798in}}{\pgfqpoint{2.497117in}{2.320397in}}{\pgfqpoint{2.489303in}{2.328211in}}%
\pgfpathcurveto{\pgfqpoint{2.481489in}{2.336025in}}{\pgfqpoint{2.470890in}{2.340415in}}{\pgfqpoint{2.459840in}{2.340415in}}%
\pgfpathcurveto{\pgfqpoint{2.448790in}{2.340415in}}{\pgfqpoint{2.438191in}{2.336025in}}{\pgfqpoint{2.430377in}{2.328211in}}%
\pgfpathcurveto{\pgfqpoint{2.422564in}{2.320397in}}{\pgfqpoint{2.418174in}{2.309798in}}{\pgfqpoint{2.418174in}{2.298748in}}%
\pgfpathcurveto{\pgfqpoint{2.418174in}{2.287698in}}{\pgfqpoint{2.422564in}{2.277099in}}{\pgfqpoint{2.430377in}{2.269285in}}%
\pgfpathcurveto{\pgfqpoint{2.438191in}{2.261472in}}{\pgfqpoint{2.448790in}{2.257082in}}{\pgfqpoint{2.459840in}{2.257082in}}%
\pgfpathclose%
\pgfusepath{stroke,fill}%
\end{pgfscope}%
\begin{pgfscope}%
\pgfpathrectangle{\pgfqpoint{0.600000in}{0.600000in}}{\pgfqpoint{3.900000in}{3.900000in}}%
\pgfusepath{clip}%
\pgfsetbuttcap%
\pgfsetroundjoin%
\definecolor{currentfill}{rgb}{0.121569,0.466667,0.705882}%
\pgfsetfillcolor{currentfill}%
\pgfsetlinewidth{1.003750pt}%
\definecolor{currentstroke}{rgb}{0.121569,0.466667,0.705882}%
\pgfsetstrokecolor{currentstroke}%
\pgfsetdash{}{0pt}%
\pgfpathmoveto{\pgfqpoint{3.607351in}{2.060106in}}%
\pgfpathcurveto{\pgfqpoint{3.618401in}{2.060106in}}{\pgfqpoint{3.629000in}{2.064497in}}{\pgfqpoint{3.636814in}{2.072310in}}%
\pgfpathcurveto{\pgfqpoint{3.644628in}{2.080124in}}{\pgfqpoint{3.649018in}{2.090723in}}{\pgfqpoint{3.649018in}{2.101773in}}%
\pgfpathcurveto{\pgfqpoint{3.649018in}{2.112823in}}{\pgfqpoint{3.644628in}{2.123422in}}{\pgfqpoint{3.636814in}{2.131236in}}%
\pgfpathcurveto{\pgfqpoint{3.629000in}{2.139050in}}{\pgfqpoint{3.618401in}{2.143440in}}{\pgfqpoint{3.607351in}{2.143440in}}%
\pgfpathcurveto{\pgfqpoint{3.596301in}{2.143440in}}{\pgfqpoint{3.585702in}{2.139050in}}{\pgfqpoint{3.577888in}{2.131236in}}%
\pgfpathcurveto{\pgfqpoint{3.570075in}{2.123422in}}{\pgfqpoint{3.565685in}{2.112823in}}{\pgfqpoint{3.565685in}{2.101773in}}%
\pgfpathcurveto{\pgfqpoint{3.565685in}{2.090723in}}{\pgfqpoint{3.570075in}{2.080124in}}{\pgfqpoint{3.577888in}{2.072310in}}%
\pgfpathcurveto{\pgfqpoint{3.585702in}{2.064497in}}{\pgfqpoint{3.596301in}{2.060106in}}{\pgfqpoint{3.607351in}{2.060106in}}%
\pgfpathclose%
\pgfusepath{stroke,fill}%
\end{pgfscope}%
\begin{pgfscope}%
\pgfpathrectangle{\pgfqpoint{0.600000in}{0.600000in}}{\pgfqpoint{3.900000in}{3.900000in}}%
\pgfusepath{clip}%
\pgfsetbuttcap%
\pgfsetroundjoin%
\definecolor{currentfill}{rgb}{0.121569,0.466667,0.705882}%
\pgfsetfillcolor{currentfill}%
\pgfsetlinewidth{1.003750pt}%
\definecolor{currentstroke}{rgb}{0.121569,0.466667,0.705882}%
\pgfsetstrokecolor{currentstroke}%
\pgfsetdash{}{0pt}%
\pgfpathmoveto{\pgfqpoint{2.809978in}{2.335171in}}%
\pgfpathcurveto{\pgfqpoint{2.821028in}{2.335171in}}{\pgfqpoint{2.831627in}{2.339561in}}{\pgfqpoint{2.839440in}{2.347375in}}%
\pgfpathcurveto{\pgfqpoint{2.847254in}{2.355189in}}{\pgfqpoint{2.851644in}{2.365788in}}{\pgfqpoint{2.851644in}{2.376838in}}%
\pgfpathcurveto{\pgfqpoint{2.851644in}{2.387888in}}{\pgfqpoint{2.847254in}{2.398487in}}{\pgfqpoint{2.839440in}{2.406300in}}%
\pgfpathcurveto{\pgfqpoint{2.831627in}{2.414114in}}{\pgfqpoint{2.821028in}{2.418504in}}{\pgfqpoint{2.809978in}{2.418504in}}%
\pgfpathcurveto{\pgfqpoint{2.798927in}{2.418504in}}{\pgfqpoint{2.788328in}{2.414114in}}{\pgfqpoint{2.780515in}{2.406300in}}%
\pgfpathcurveto{\pgfqpoint{2.772701in}{2.398487in}}{\pgfqpoint{2.768311in}{2.387888in}}{\pgfqpoint{2.768311in}{2.376838in}}%
\pgfpathcurveto{\pgfqpoint{2.768311in}{2.365788in}}{\pgfqpoint{2.772701in}{2.355189in}}{\pgfqpoint{2.780515in}{2.347375in}}%
\pgfpathcurveto{\pgfqpoint{2.788328in}{2.339561in}}{\pgfqpoint{2.798927in}{2.335171in}}{\pgfqpoint{2.809978in}{2.335171in}}%
\pgfpathclose%
\pgfusepath{stroke,fill}%
\end{pgfscope}%
\begin{pgfscope}%
\pgfpathrectangle{\pgfqpoint{0.600000in}{0.600000in}}{\pgfqpoint{3.900000in}{3.900000in}}%
\pgfusepath{clip}%
\pgfsetbuttcap%
\pgfsetroundjoin%
\definecolor{currentfill}{rgb}{0.121569,0.466667,0.705882}%
\pgfsetfillcolor{currentfill}%
\pgfsetlinewidth{1.003750pt}%
\definecolor{currentstroke}{rgb}{0.121569,0.466667,0.705882}%
\pgfsetstrokecolor{currentstroke}%
\pgfsetdash{}{0pt}%
\pgfpathmoveto{\pgfqpoint{3.600376in}{1.912771in}}%
\pgfpathcurveto{\pgfqpoint{3.611426in}{1.912771in}}{\pgfqpoint{3.622025in}{1.917161in}}{\pgfqpoint{3.629838in}{1.924974in}}%
\pgfpathcurveto{\pgfqpoint{3.637652in}{1.932788in}}{\pgfqpoint{3.642042in}{1.943387in}}{\pgfqpoint{3.642042in}{1.954437in}}%
\pgfpathcurveto{\pgfqpoint{3.642042in}{1.965487in}}{\pgfqpoint{3.637652in}{1.976086in}}{\pgfqpoint{3.629838in}{1.983900in}}%
\pgfpathcurveto{\pgfqpoint{3.622025in}{1.991714in}}{\pgfqpoint{3.611426in}{1.996104in}}{\pgfqpoint{3.600376in}{1.996104in}}%
\pgfpathcurveto{\pgfqpoint{3.589326in}{1.996104in}}{\pgfqpoint{3.578727in}{1.991714in}}{\pgfqpoint{3.570913in}{1.983900in}}%
\pgfpathcurveto{\pgfqpoint{3.563099in}{1.976086in}}{\pgfqpoint{3.558709in}{1.965487in}}{\pgfqpoint{3.558709in}{1.954437in}}%
\pgfpathcurveto{\pgfqpoint{3.558709in}{1.943387in}}{\pgfqpoint{3.563099in}{1.932788in}}{\pgfqpoint{3.570913in}{1.924974in}}%
\pgfpathcurveto{\pgfqpoint{3.578727in}{1.917161in}}{\pgfqpoint{3.589326in}{1.912771in}}{\pgfqpoint{3.600376in}{1.912771in}}%
\pgfpathclose%
\pgfusepath{stroke,fill}%
\end{pgfscope}%
\begin{pgfscope}%
\pgfpathrectangle{\pgfqpoint{0.600000in}{0.600000in}}{\pgfqpoint{3.900000in}{3.900000in}}%
\pgfusepath{clip}%
\pgfsetbuttcap%
\pgfsetroundjoin%
\definecolor{currentfill}{rgb}{0.121569,0.466667,0.705882}%
\pgfsetfillcolor{currentfill}%
\pgfsetlinewidth{1.003750pt}%
\definecolor{currentstroke}{rgb}{0.121569,0.466667,0.705882}%
\pgfsetstrokecolor{currentstroke}%
\pgfsetdash{}{0pt}%
\pgfpathmoveto{\pgfqpoint{2.731614in}{1.830543in}}%
\pgfpathcurveto{\pgfqpoint{2.742664in}{1.830543in}}{\pgfqpoint{2.753263in}{1.834933in}}{\pgfqpoint{2.761076in}{1.842747in}}%
\pgfpathcurveto{\pgfqpoint{2.768890in}{1.850560in}}{\pgfqpoint{2.773280in}{1.861159in}}{\pgfqpoint{2.773280in}{1.872210in}}%
\pgfpathcurveto{\pgfqpoint{2.773280in}{1.883260in}}{\pgfqpoint{2.768890in}{1.893859in}}{\pgfqpoint{2.761076in}{1.901672in}}%
\pgfpathcurveto{\pgfqpoint{2.753263in}{1.909486in}}{\pgfqpoint{2.742664in}{1.913876in}}{\pgfqpoint{2.731614in}{1.913876in}}%
\pgfpathcurveto{\pgfqpoint{2.720564in}{1.913876in}}{\pgfqpoint{2.709965in}{1.909486in}}{\pgfqpoint{2.702151in}{1.901672in}}%
\pgfpathcurveto{\pgfqpoint{2.694337in}{1.893859in}}{\pgfqpoint{2.689947in}{1.883260in}}{\pgfqpoint{2.689947in}{1.872210in}}%
\pgfpathcurveto{\pgfqpoint{2.689947in}{1.861159in}}{\pgfqpoint{2.694337in}{1.850560in}}{\pgfqpoint{2.702151in}{1.842747in}}%
\pgfpathcurveto{\pgfqpoint{2.709965in}{1.834933in}}{\pgfqpoint{2.720564in}{1.830543in}}{\pgfqpoint{2.731614in}{1.830543in}}%
\pgfpathclose%
\pgfusepath{stroke,fill}%
\end{pgfscope}%
\begin{pgfscope}%
\pgfpathrectangle{\pgfqpoint{0.600000in}{0.600000in}}{\pgfqpoint{3.900000in}{3.900000in}}%
\pgfusepath{clip}%
\pgfsetbuttcap%
\pgfsetroundjoin%
\definecolor{currentfill}{rgb}{0.121569,0.466667,0.705882}%
\pgfsetfillcolor{currentfill}%
\pgfsetlinewidth{1.003750pt}%
\definecolor{currentstroke}{rgb}{0.121569,0.466667,0.705882}%
\pgfsetstrokecolor{currentstroke}%
\pgfsetdash{}{0pt}%
\pgfpathmoveto{\pgfqpoint{2.692553in}{2.881750in}}%
\pgfpathcurveto{\pgfqpoint{2.703603in}{2.881750in}}{\pgfqpoint{2.714202in}{2.886140in}}{\pgfqpoint{2.722016in}{2.893954in}}%
\pgfpathcurveto{\pgfqpoint{2.729829in}{2.901767in}}{\pgfqpoint{2.734220in}{2.912366in}}{\pgfqpoint{2.734220in}{2.923417in}}%
\pgfpathcurveto{\pgfqpoint{2.734220in}{2.934467in}}{\pgfqpoint{2.729829in}{2.945066in}}{\pgfqpoint{2.722016in}{2.952879in}}%
\pgfpathcurveto{\pgfqpoint{2.714202in}{2.960693in}}{\pgfqpoint{2.703603in}{2.965083in}}{\pgfqpoint{2.692553in}{2.965083in}}%
\pgfpathcurveto{\pgfqpoint{2.681503in}{2.965083in}}{\pgfqpoint{2.670904in}{2.960693in}}{\pgfqpoint{2.663090in}{2.952879in}}%
\pgfpathcurveto{\pgfqpoint{2.655277in}{2.945066in}}{\pgfqpoint{2.650886in}{2.934467in}}{\pgfqpoint{2.650886in}{2.923417in}}%
\pgfpathcurveto{\pgfqpoint{2.650886in}{2.912366in}}{\pgfqpoint{2.655277in}{2.901767in}}{\pgfqpoint{2.663090in}{2.893954in}}%
\pgfpathcurveto{\pgfqpoint{2.670904in}{2.886140in}}{\pgfqpoint{2.681503in}{2.881750in}}{\pgfqpoint{2.692553in}{2.881750in}}%
\pgfpathclose%
\pgfusepath{stroke,fill}%
\end{pgfscope}%
\begin{pgfscope}%
\pgfpathrectangle{\pgfqpoint{0.600000in}{0.600000in}}{\pgfqpoint{3.900000in}{3.900000in}}%
\pgfusepath{clip}%
\pgfsetbuttcap%
\pgfsetroundjoin%
\definecolor{currentfill}{rgb}{0.121569,0.466667,0.705882}%
\pgfsetfillcolor{currentfill}%
\pgfsetlinewidth{1.003750pt}%
\definecolor{currentstroke}{rgb}{0.121569,0.466667,0.705882}%
\pgfsetstrokecolor{currentstroke}%
\pgfsetdash{}{0pt}%
\pgfpathmoveto{\pgfqpoint{1.674804in}{2.979269in}}%
\pgfpathcurveto{\pgfqpoint{1.685854in}{2.979269in}}{\pgfqpoint{1.696453in}{2.983659in}}{\pgfqpoint{1.704267in}{2.991473in}}%
\pgfpathcurveto{\pgfqpoint{1.712081in}{2.999287in}}{\pgfqpoint{1.716471in}{3.009886in}}{\pgfqpoint{1.716471in}{3.020936in}}%
\pgfpathcurveto{\pgfqpoint{1.716471in}{3.031986in}}{\pgfqpoint{1.712081in}{3.042585in}}{\pgfqpoint{1.704267in}{3.050398in}}%
\pgfpathcurveto{\pgfqpoint{1.696453in}{3.058212in}}{\pgfqpoint{1.685854in}{3.062602in}}{\pgfqpoint{1.674804in}{3.062602in}}%
\pgfpathcurveto{\pgfqpoint{1.663754in}{3.062602in}}{\pgfqpoint{1.653155in}{3.058212in}}{\pgfqpoint{1.645341in}{3.050398in}}%
\pgfpathcurveto{\pgfqpoint{1.637528in}{3.042585in}}{\pgfqpoint{1.633138in}{3.031986in}}{\pgfqpoint{1.633138in}{3.020936in}}%
\pgfpathcurveto{\pgfqpoint{1.633138in}{3.009886in}}{\pgfqpoint{1.637528in}{2.999287in}}{\pgfqpoint{1.645341in}{2.991473in}}%
\pgfpathcurveto{\pgfqpoint{1.653155in}{2.983659in}}{\pgfqpoint{1.663754in}{2.979269in}}{\pgfqpoint{1.674804in}{2.979269in}}%
\pgfpathclose%
\pgfusepath{stroke,fill}%
\end{pgfscope}%
\begin{pgfscope}%
\pgfpathrectangle{\pgfqpoint{0.600000in}{0.600000in}}{\pgfqpoint{3.900000in}{3.900000in}}%
\pgfusepath{clip}%
\pgfsetbuttcap%
\pgfsetroundjoin%
\definecolor{currentfill}{rgb}{0.121569,0.466667,0.705882}%
\pgfsetfillcolor{currentfill}%
\pgfsetlinewidth{1.003750pt}%
\definecolor{currentstroke}{rgb}{0.121569,0.466667,0.705882}%
\pgfsetstrokecolor{currentstroke}%
\pgfsetdash{}{0pt}%
\pgfpathmoveto{\pgfqpoint{3.007057in}{2.785213in}}%
\pgfpathcurveto{\pgfqpoint{3.018107in}{2.785213in}}{\pgfqpoint{3.028706in}{2.789604in}}{\pgfqpoint{3.036520in}{2.797417in}}%
\pgfpathcurveto{\pgfqpoint{3.044333in}{2.805231in}}{\pgfqpoint{3.048724in}{2.815830in}}{\pgfqpoint{3.048724in}{2.826880in}}%
\pgfpathcurveto{\pgfqpoint{3.048724in}{2.837930in}}{\pgfqpoint{3.044333in}{2.848529in}}{\pgfqpoint{3.036520in}{2.856343in}}%
\pgfpathcurveto{\pgfqpoint{3.028706in}{2.864156in}}{\pgfqpoint{3.018107in}{2.868547in}}{\pgfqpoint{3.007057in}{2.868547in}}%
\pgfpathcurveto{\pgfqpoint{2.996007in}{2.868547in}}{\pgfqpoint{2.985408in}{2.864156in}}{\pgfqpoint{2.977594in}{2.856343in}}%
\pgfpathcurveto{\pgfqpoint{2.969781in}{2.848529in}}{\pgfqpoint{2.965390in}{2.837930in}}{\pgfqpoint{2.965390in}{2.826880in}}%
\pgfpathcurveto{\pgfqpoint{2.965390in}{2.815830in}}{\pgfqpoint{2.969781in}{2.805231in}}{\pgfqpoint{2.977594in}{2.797417in}}%
\pgfpathcurveto{\pgfqpoint{2.985408in}{2.789604in}}{\pgfqpoint{2.996007in}{2.785213in}}{\pgfqpoint{3.007057in}{2.785213in}}%
\pgfpathclose%
\pgfusepath{stroke,fill}%
\end{pgfscope}%
\begin{pgfscope}%
\pgfpathrectangle{\pgfqpoint{0.600000in}{0.600000in}}{\pgfqpoint{3.900000in}{3.900000in}}%
\pgfusepath{clip}%
\pgfsetbuttcap%
\pgfsetroundjoin%
\definecolor{currentfill}{rgb}{0.121569,0.466667,0.705882}%
\pgfsetfillcolor{currentfill}%
\pgfsetlinewidth{1.003750pt}%
\definecolor{currentstroke}{rgb}{0.121569,0.466667,0.705882}%
\pgfsetstrokecolor{currentstroke}%
\pgfsetdash{}{0pt}%
\pgfpathmoveto{\pgfqpoint{2.457045in}{2.607472in}}%
\pgfpathcurveto{\pgfqpoint{2.468096in}{2.607472in}}{\pgfqpoint{2.478695in}{2.611862in}}{\pgfqpoint{2.486508in}{2.619675in}}%
\pgfpathcurveto{\pgfqpoint{2.494322in}{2.627489in}}{\pgfqpoint{2.498712in}{2.638088in}}{\pgfqpoint{2.498712in}{2.649138in}}%
\pgfpathcurveto{\pgfqpoint{2.498712in}{2.660188in}}{\pgfqpoint{2.494322in}{2.670787in}}{\pgfqpoint{2.486508in}{2.678601in}}%
\pgfpathcurveto{\pgfqpoint{2.478695in}{2.686415in}}{\pgfqpoint{2.468096in}{2.690805in}}{\pgfqpoint{2.457045in}{2.690805in}}%
\pgfpathcurveto{\pgfqpoint{2.445995in}{2.690805in}}{\pgfqpoint{2.435396in}{2.686415in}}{\pgfqpoint{2.427583in}{2.678601in}}%
\pgfpathcurveto{\pgfqpoint{2.419769in}{2.670787in}}{\pgfqpoint{2.415379in}{2.660188in}}{\pgfqpoint{2.415379in}{2.649138in}}%
\pgfpathcurveto{\pgfqpoint{2.415379in}{2.638088in}}{\pgfqpoint{2.419769in}{2.627489in}}{\pgfqpoint{2.427583in}{2.619675in}}%
\pgfpathcurveto{\pgfqpoint{2.435396in}{2.611862in}}{\pgfqpoint{2.445995in}{2.607472in}}{\pgfqpoint{2.457045in}{2.607472in}}%
\pgfpathclose%
\pgfusepath{stroke,fill}%
\end{pgfscope}%
\begin{pgfscope}%
\pgfpathrectangle{\pgfqpoint{0.600000in}{0.600000in}}{\pgfqpoint{3.900000in}{3.900000in}}%
\pgfusepath{clip}%
\pgfsetbuttcap%
\pgfsetroundjoin%
\definecolor{currentfill}{rgb}{0.121569,0.466667,0.705882}%
\pgfsetfillcolor{currentfill}%
\pgfsetlinewidth{1.003750pt}%
\definecolor{currentstroke}{rgb}{0.121569,0.466667,0.705882}%
\pgfsetstrokecolor{currentstroke}%
\pgfsetdash{}{0pt}%
\pgfpathmoveto{\pgfqpoint{2.616953in}{2.063422in}}%
\pgfpathcurveto{\pgfqpoint{2.628003in}{2.063422in}}{\pgfqpoint{2.638602in}{2.067812in}}{\pgfqpoint{2.646416in}{2.075626in}}%
\pgfpathcurveto{\pgfqpoint{2.654229in}{2.083439in}}{\pgfqpoint{2.658620in}{2.094038in}}{\pgfqpoint{2.658620in}{2.105088in}}%
\pgfpathcurveto{\pgfqpoint{2.658620in}{2.116139in}}{\pgfqpoint{2.654229in}{2.126738in}}{\pgfqpoint{2.646416in}{2.134551in}}%
\pgfpathcurveto{\pgfqpoint{2.638602in}{2.142365in}}{\pgfqpoint{2.628003in}{2.146755in}}{\pgfqpoint{2.616953in}{2.146755in}}%
\pgfpathcurveto{\pgfqpoint{2.605903in}{2.146755in}}{\pgfqpoint{2.595304in}{2.142365in}}{\pgfqpoint{2.587490in}{2.134551in}}%
\pgfpathcurveto{\pgfqpoint{2.579677in}{2.126738in}}{\pgfqpoint{2.575286in}{2.116139in}}{\pgfqpoint{2.575286in}{2.105088in}}%
\pgfpathcurveto{\pgfqpoint{2.575286in}{2.094038in}}{\pgfqpoint{2.579677in}{2.083439in}}{\pgfqpoint{2.587490in}{2.075626in}}%
\pgfpathcurveto{\pgfqpoint{2.595304in}{2.067812in}}{\pgfqpoint{2.605903in}{2.063422in}}{\pgfqpoint{2.616953in}{2.063422in}}%
\pgfpathclose%
\pgfusepath{stroke,fill}%
\end{pgfscope}%
\begin{pgfscope}%
\pgfpathrectangle{\pgfqpoint{0.600000in}{0.600000in}}{\pgfqpoint{3.900000in}{3.900000in}}%
\pgfusepath{clip}%
\pgfsetbuttcap%
\pgfsetroundjoin%
\definecolor{currentfill}{rgb}{0.121569,0.466667,0.705882}%
\pgfsetfillcolor{currentfill}%
\pgfsetlinewidth{1.003750pt}%
\definecolor{currentstroke}{rgb}{0.121569,0.466667,0.705882}%
\pgfsetstrokecolor{currentstroke}%
\pgfsetdash{}{0pt}%
\pgfpathmoveto{\pgfqpoint{1.924978in}{2.387885in}}%
\pgfpathcurveto{\pgfqpoint{1.936028in}{2.387885in}}{\pgfqpoint{1.946627in}{2.392275in}}{\pgfqpoint{1.954440in}{2.400089in}}%
\pgfpathcurveto{\pgfqpoint{1.962254in}{2.407902in}}{\pgfqpoint{1.966644in}{2.418501in}}{\pgfqpoint{1.966644in}{2.429551in}}%
\pgfpathcurveto{\pgfqpoint{1.966644in}{2.440601in}}{\pgfqpoint{1.962254in}{2.451201in}}{\pgfqpoint{1.954440in}{2.459014in}}%
\pgfpathcurveto{\pgfqpoint{1.946627in}{2.466828in}}{\pgfqpoint{1.936028in}{2.471218in}}{\pgfqpoint{1.924978in}{2.471218in}}%
\pgfpathcurveto{\pgfqpoint{1.913928in}{2.471218in}}{\pgfqpoint{1.903328in}{2.466828in}}{\pgfqpoint{1.895515in}{2.459014in}}%
\pgfpathcurveto{\pgfqpoint{1.887701in}{2.451201in}}{\pgfqpoint{1.883311in}{2.440601in}}{\pgfqpoint{1.883311in}{2.429551in}}%
\pgfpathcurveto{\pgfqpoint{1.883311in}{2.418501in}}{\pgfqpoint{1.887701in}{2.407902in}}{\pgfqpoint{1.895515in}{2.400089in}}%
\pgfpathcurveto{\pgfqpoint{1.903328in}{2.392275in}}{\pgfqpoint{1.913928in}{2.387885in}}{\pgfqpoint{1.924978in}{2.387885in}}%
\pgfpathclose%
\pgfusepath{stroke,fill}%
\end{pgfscope}%
\begin{pgfscope}%
\pgfpathrectangle{\pgfqpoint{0.600000in}{0.600000in}}{\pgfqpoint{3.900000in}{3.900000in}}%
\pgfusepath{clip}%
\pgfsetbuttcap%
\pgfsetroundjoin%
\definecolor{currentfill}{rgb}{0.121569,0.466667,0.705882}%
\pgfsetfillcolor{currentfill}%
\pgfsetlinewidth{1.003750pt}%
\definecolor{currentstroke}{rgb}{0.121569,0.466667,0.705882}%
\pgfsetstrokecolor{currentstroke}%
\pgfsetdash{}{0pt}%
\pgfpathmoveto{\pgfqpoint{2.695517in}{2.484026in}}%
\pgfpathcurveto{\pgfqpoint{2.706567in}{2.484026in}}{\pgfqpoint{2.717166in}{2.488417in}}{\pgfqpoint{2.724980in}{2.496230in}}%
\pgfpathcurveto{\pgfqpoint{2.732793in}{2.504044in}}{\pgfqpoint{2.737184in}{2.514643in}}{\pgfqpoint{2.737184in}{2.525693in}}%
\pgfpathcurveto{\pgfqpoint{2.737184in}{2.536743in}}{\pgfqpoint{2.732793in}{2.547342in}}{\pgfqpoint{2.724980in}{2.555156in}}%
\pgfpathcurveto{\pgfqpoint{2.717166in}{2.562969in}}{\pgfqpoint{2.706567in}{2.567360in}}{\pgfqpoint{2.695517in}{2.567360in}}%
\pgfpathcurveto{\pgfqpoint{2.684467in}{2.567360in}}{\pgfqpoint{2.673868in}{2.562969in}}{\pgfqpoint{2.666054in}{2.555156in}}%
\pgfpathcurveto{\pgfqpoint{2.658241in}{2.547342in}}{\pgfqpoint{2.653850in}{2.536743in}}{\pgfqpoint{2.653850in}{2.525693in}}%
\pgfpathcurveto{\pgfqpoint{2.653850in}{2.514643in}}{\pgfqpoint{2.658241in}{2.504044in}}{\pgfqpoint{2.666054in}{2.496230in}}%
\pgfpathcurveto{\pgfqpoint{2.673868in}{2.488417in}}{\pgfqpoint{2.684467in}{2.484026in}}{\pgfqpoint{2.695517in}{2.484026in}}%
\pgfpathclose%
\pgfusepath{stroke,fill}%
\end{pgfscope}%
\begin{pgfscope}%
\pgfpathrectangle{\pgfqpoint{0.600000in}{0.600000in}}{\pgfqpoint{3.900000in}{3.900000in}}%
\pgfusepath{clip}%
\pgfsetbuttcap%
\pgfsetroundjoin%
\definecolor{currentfill}{rgb}{0.121569,0.466667,0.705882}%
\pgfsetfillcolor{currentfill}%
\pgfsetlinewidth{1.003750pt}%
\definecolor{currentstroke}{rgb}{0.121569,0.466667,0.705882}%
\pgfsetstrokecolor{currentstroke}%
\pgfsetdash{}{0pt}%
\pgfpathmoveto{\pgfqpoint{2.506460in}{2.279661in}}%
\pgfpathcurveto{\pgfqpoint{2.517510in}{2.279661in}}{\pgfqpoint{2.528109in}{2.284051in}}{\pgfqpoint{2.535923in}{2.291865in}}%
\pgfpathcurveto{\pgfqpoint{2.543736in}{2.299679in}}{\pgfqpoint{2.548126in}{2.310278in}}{\pgfqpoint{2.548126in}{2.321328in}}%
\pgfpathcurveto{\pgfqpoint{2.548126in}{2.332378in}}{\pgfqpoint{2.543736in}{2.342977in}}{\pgfqpoint{2.535923in}{2.350791in}}%
\pgfpathcurveto{\pgfqpoint{2.528109in}{2.358604in}}{\pgfqpoint{2.517510in}{2.362995in}}{\pgfqpoint{2.506460in}{2.362995in}}%
\pgfpathcurveto{\pgfqpoint{2.495410in}{2.362995in}}{\pgfqpoint{2.484811in}{2.358604in}}{\pgfqpoint{2.476997in}{2.350791in}}%
\pgfpathcurveto{\pgfqpoint{2.469183in}{2.342977in}}{\pgfqpoint{2.464793in}{2.332378in}}{\pgfqpoint{2.464793in}{2.321328in}}%
\pgfpathcurveto{\pgfqpoint{2.464793in}{2.310278in}}{\pgfqpoint{2.469183in}{2.299679in}}{\pgfqpoint{2.476997in}{2.291865in}}%
\pgfpathcurveto{\pgfqpoint{2.484811in}{2.284051in}}{\pgfqpoint{2.495410in}{2.279661in}}{\pgfqpoint{2.506460in}{2.279661in}}%
\pgfpathclose%
\pgfusepath{stroke,fill}%
\end{pgfscope}%
\begin{pgfscope}%
\pgfpathrectangle{\pgfqpoint{0.600000in}{0.600000in}}{\pgfqpoint{3.900000in}{3.900000in}}%
\pgfusepath{clip}%
\pgfsetbuttcap%
\pgfsetroundjoin%
\definecolor{currentfill}{rgb}{0.121569,0.466667,0.705882}%
\pgfsetfillcolor{currentfill}%
\pgfsetlinewidth{1.003750pt}%
\definecolor{currentstroke}{rgb}{0.121569,0.466667,0.705882}%
\pgfsetstrokecolor{currentstroke}%
\pgfsetdash{}{0pt}%
\pgfpathmoveto{\pgfqpoint{2.992361in}{2.790707in}}%
\pgfpathcurveto{\pgfqpoint{3.003411in}{2.790707in}}{\pgfqpoint{3.014010in}{2.795097in}}{\pgfqpoint{3.021823in}{2.802911in}}%
\pgfpathcurveto{\pgfqpoint{3.029637in}{2.810725in}}{\pgfqpoint{3.034027in}{2.821324in}}{\pgfqpoint{3.034027in}{2.832374in}}%
\pgfpathcurveto{\pgfqpoint{3.034027in}{2.843424in}}{\pgfqpoint{3.029637in}{2.854023in}}{\pgfqpoint{3.021823in}{2.861837in}}%
\pgfpathcurveto{\pgfqpoint{3.014010in}{2.869650in}}{\pgfqpoint{3.003411in}{2.874040in}}{\pgfqpoint{2.992361in}{2.874040in}}%
\pgfpathcurveto{\pgfqpoint{2.981311in}{2.874040in}}{\pgfqpoint{2.970712in}{2.869650in}}{\pgfqpoint{2.962898in}{2.861837in}}%
\pgfpathcurveto{\pgfqpoint{2.955084in}{2.854023in}}{\pgfqpoint{2.950694in}{2.843424in}}{\pgfqpoint{2.950694in}{2.832374in}}%
\pgfpathcurveto{\pgfqpoint{2.950694in}{2.821324in}}{\pgfqpoint{2.955084in}{2.810725in}}{\pgfqpoint{2.962898in}{2.802911in}}%
\pgfpathcurveto{\pgfqpoint{2.970712in}{2.795097in}}{\pgfqpoint{2.981311in}{2.790707in}}{\pgfqpoint{2.992361in}{2.790707in}}%
\pgfpathclose%
\pgfusepath{stroke,fill}%
\end{pgfscope}%
\begin{pgfscope}%
\pgfpathrectangle{\pgfqpoint{0.600000in}{0.600000in}}{\pgfqpoint{3.900000in}{3.900000in}}%
\pgfusepath{clip}%
\pgfsetbuttcap%
\pgfsetroundjoin%
\definecolor{currentfill}{rgb}{0.121569,0.466667,0.705882}%
\pgfsetfillcolor{currentfill}%
\pgfsetlinewidth{1.003750pt}%
\definecolor{currentstroke}{rgb}{0.121569,0.466667,0.705882}%
\pgfsetstrokecolor{currentstroke}%
\pgfsetdash{}{0pt}%
\pgfpathmoveto{\pgfqpoint{2.692222in}{2.733277in}}%
\pgfpathcurveto{\pgfqpoint{2.703272in}{2.733277in}}{\pgfqpoint{2.713871in}{2.737668in}}{\pgfqpoint{2.721685in}{2.745481in}}%
\pgfpathcurveto{\pgfqpoint{2.729499in}{2.753295in}}{\pgfqpoint{2.733889in}{2.763894in}}{\pgfqpoint{2.733889in}{2.774944in}}%
\pgfpathcurveto{\pgfqpoint{2.733889in}{2.785994in}}{\pgfqpoint{2.729499in}{2.796593in}}{\pgfqpoint{2.721685in}{2.804407in}}%
\pgfpathcurveto{\pgfqpoint{2.713871in}{2.812220in}}{\pgfqpoint{2.703272in}{2.816611in}}{\pgfqpoint{2.692222in}{2.816611in}}%
\pgfpathcurveto{\pgfqpoint{2.681172in}{2.816611in}}{\pgfqpoint{2.670573in}{2.812220in}}{\pgfqpoint{2.662759in}{2.804407in}}%
\pgfpathcurveto{\pgfqpoint{2.654946in}{2.796593in}}{\pgfqpoint{2.650555in}{2.785994in}}{\pgfqpoint{2.650555in}{2.774944in}}%
\pgfpathcurveto{\pgfqpoint{2.650555in}{2.763894in}}{\pgfqpoint{2.654946in}{2.753295in}}{\pgfqpoint{2.662759in}{2.745481in}}%
\pgfpathcurveto{\pgfqpoint{2.670573in}{2.737668in}}{\pgfqpoint{2.681172in}{2.733277in}}{\pgfqpoint{2.692222in}{2.733277in}}%
\pgfpathclose%
\pgfusepath{stroke,fill}%
\end{pgfscope}%
\begin{pgfscope}%
\pgfpathrectangle{\pgfqpoint{0.600000in}{0.600000in}}{\pgfqpoint{3.900000in}{3.900000in}}%
\pgfusepath{clip}%
\pgfsetbuttcap%
\pgfsetroundjoin%
\definecolor{currentfill}{rgb}{0.121569,0.466667,0.705882}%
\pgfsetfillcolor{currentfill}%
\pgfsetlinewidth{1.003750pt}%
\definecolor{currentstroke}{rgb}{0.121569,0.466667,0.705882}%
\pgfsetstrokecolor{currentstroke}%
\pgfsetdash{}{0pt}%
\pgfpathmoveto{\pgfqpoint{3.151934in}{1.892745in}}%
\pgfpathcurveto{\pgfqpoint{3.162984in}{1.892745in}}{\pgfqpoint{3.173583in}{1.897135in}}{\pgfqpoint{3.181397in}{1.904949in}}%
\pgfpathcurveto{\pgfqpoint{3.189210in}{1.912762in}}{\pgfqpoint{3.193600in}{1.923361in}}{\pgfqpoint{3.193600in}{1.934412in}}%
\pgfpathcurveto{\pgfqpoint{3.193600in}{1.945462in}}{\pgfqpoint{3.189210in}{1.956061in}}{\pgfqpoint{3.181397in}{1.963874in}}%
\pgfpathcurveto{\pgfqpoint{3.173583in}{1.971688in}}{\pgfqpoint{3.162984in}{1.976078in}}{\pgfqpoint{3.151934in}{1.976078in}}%
\pgfpathcurveto{\pgfqpoint{3.140884in}{1.976078in}}{\pgfqpoint{3.130285in}{1.971688in}}{\pgfqpoint{3.122471in}{1.963874in}}%
\pgfpathcurveto{\pgfqpoint{3.114657in}{1.956061in}}{\pgfqpoint{3.110267in}{1.945462in}}{\pgfqpoint{3.110267in}{1.934412in}}%
\pgfpathcurveto{\pgfqpoint{3.110267in}{1.923361in}}{\pgfqpoint{3.114657in}{1.912762in}}{\pgfqpoint{3.122471in}{1.904949in}}%
\pgfpathcurveto{\pgfqpoint{3.130285in}{1.897135in}}{\pgfqpoint{3.140884in}{1.892745in}}{\pgfqpoint{3.151934in}{1.892745in}}%
\pgfpathclose%
\pgfusepath{stroke,fill}%
\end{pgfscope}%
\begin{pgfscope}%
\pgfpathrectangle{\pgfqpoint{0.600000in}{0.600000in}}{\pgfqpoint{3.900000in}{3.900000in}}%
\pgfusepath{clip}%
\pgfsetbuttcap%
\pgfsetroundjoin%
\definecolor{currentfill}{rgb}{0.121569,0.466667,0.705882}%
\pgfsetfillcolor{currentfill}%
\pgfsetlinewidth{1.003750pt}%
\definecolor{currentstroke}{rgb}{0.121569,0.466667,0.705882}%
\pgfsetstrokecolor{currentstroke}%
\pgfsetdash{}{0pt}%
\pgfpathmoveto{\pgfqpoint{3.031091in}{2.794482in}}%
\pgfpathcurveto{\pgfqpoint{3.042141in}{2.794482in}}{\pgfqpoint{3.052740in}{2.798872in}}{\pgfqpoint{3.060553in}{2.806686in}}%
\pgfpathcurveto{\pgfqpoint{3.068367in}{2.814500in}}{\pgfqpoint{3.072757in}{2.825099in}}{\pgfqpoint{3.072757in}{2.836149in}}%
\pgfpathcurveto{\pgfqpoint{3.072757in}{2.847199in}}{\pgfqpoint{3.068367in}{2.857798in}}{\pgfqpoint{3.060553in}{2.865611in}}%
\pgfpathcurveto{\pgfqpoint{3.052740in}{2.873425in}}{\pgfqpoint{3.042141in}{2.877815in}}{\pgfqpoint{3.031091in}{2.877815in}}%
\pgfpathcurveto{\pgfqpoint{3.020040in}{2.877815in}}{\pgfqpoint{3.009441in}{2.873425in}}{\pgfqpoint{3.001628in}{2.865611in}}%
\pgfpathcurveto{\pgfqpoint{2.993814in}{2.857798in}}{\pgfqpoint{2.989424in}{2.847199in}}{\pgfqpoint{2.989424in}{2.836149in}}%
\pgfpathcurveto{\pgfqpoint{2.989424in}{2.825099in}}{\pgfqpoint{2.993814in}{2.814500in}}{\pgfqpoint{3.001628in}{2.806686in}}%
\pgfpathcurveto{\pgfqpoint{3.009441in}{2.798872in}}{\pgfqpoint{3.020040in}{2.794482in}}{\pgfqpoint{3.031091in}{2.794482in}}%
\pgfpathclose%
\pgfusepath{stroke,fill}%
\end{pgfscope}%
\begin{pgfscope}%
\pgfpathrectangle{\pgfqpoint{0.600000in}{0.600000in}}{\pgfqpoint{3.900000in}{3.900000in}}%
\pgfusepath{clip}%
\pgfsetbuttcap%
\pgfsetroundjoin%
\definecolor{currentfill}{rgb}{0.121569,0.466667,0.705882}%
\pgfsetfillcolor{currentfill}%
\pgfsetlinewidth{1.003750pt}%
\definecolor{currentstroke}{rgb}{0.121569,0.466667,0.705882}%
\pgfsetstrokecolor{currentstroke}%
\pgfsetdash{}{0pt}%
\pgfpathmoveto{\pgfqpoint{2.572743in}{2.252120in}}%
\pgfpathcurveto{\pgfqpoint{2.583794in}{2.252120in}}{\pgfqpoint{2.594393in}{2.256510in}}{\pgfqpoint{2.602206in}{2.264324in}}%
\pgfpathcurveto{\pgfqpoint{2.610020in}{2.272137in}}{\pgfqpoint{2.614410in}{2.282736in}}{\pgfqpoint{2.614410in}{2.293786in}}%
\pgfpathcurveto{\pgfqpoint{2.614410in}{2.304836in}}{\pgfqpoint{2.610020in}{2.315436in}}{\pgfqpoint{2.602206in}{2.323249in}}%
\pgfpathcurveto{\pgfqpoint{2.594393in}{2.331063in}}{\pgfqpoint{2.583794in}{2.335453in}}{\pgfqpoint{2.572743in}{2.335453in}}%
\pgfpathcurveto{\pgfqpoint{2.561693in}{2.335453in}}{\pgfqpoint{2.551094in}{2.331063in}}{\pgfqpoint{2.543281in}{2.323249in}}%
\pgfpathcurveto{\pgfqpoint{2.535467in}{2.315436in}}{\pgfqpoint{2.531077in}{2.304836in}}{\pgfqpoint{2.531077in}{2.293786in}}%
\pgfpathcurveto{\pgfqpoint{2.531077in}{2.282736in}}{\pgfqpoint{2.535467in}{2.272137in}}{\pgfqpoint{2.543281in}{2.264324in}}%
\pgfpathcurveto{\pgfqpoint{2.551094in}{2.256510in}}{\pgfqpoint{2.561693in}{2.252120in}}{\pgfqpoint{2.572743in}{2.252120in}}%
\pgfpathclose%
\pgfusepath{stroke,fill}%
\end{pgfscope}%
\begin{pgfscope}%
\pgfpathrectangle{\pgfqpoint{0.600000in}{0.600000in}}{\pgfqpoint{3.900000in}{3.900000in}}%
\pgfusepath{clip}%
\pgfsetbuttcap%
\pgfsetroundjoin%
\definecolor{currentfill}{rgb}{0.121569,0.466667,0.705882}%
\pgfsetfillcolor{currentfill}%
\pgfsetlinewidth{1.003750pt}%
\definecolor{currentstroke}{rgb}{0.121569,0.466667,0.705882}%
\pgfsetstrokecolor{currentstroke}%
\pgfsetdash{}{0pt}%
\pgfpathmoveto{\pgfqpoint{2.830132in}{2.244200in}}%
\pgfpathcurveto{\pgfqpoint{2.841182in}{2.244200in}}{\pgfqpoint{2.851781in}{2.248591in}}{\pgfqpoint{2.859595in}{2.256404in}}%
\pgfpathcurveto{\pgfqpoint{2.867408in}{2.264218in}}{\pgfqpoint{2.871799in}{2.274817in}}{\pgfqpoint{2.871799in}{2.285867in}}%
\pgfpathcurveto{\pgfqpoint{2.871799in}{2.296917in}}{\pgfqpoint{2.867408in}{2.307516in}}{\pgfqpoint{2.859595in}{2.315330in}}%
\pgfpathcurveto{\pgfqpoint{2.851781in}{2.323143in}}{\pgfqpoint{2.841182in}{2.327534in}}{\pgfqpoint{2.830132in}{2.327534in}}%
\pgfpathcurveto{\pgfqpoint{2.819082in}{2.327534in}}{\pgfqpoint{2.808483in}{2.323143in}}{\pgfqpoint{2.800669in}{2.315330in}}%
\pgfpathcurveto{\pgfqpoint{2.792856in}{2.307516in}}{\pgfqpoint{2.788465in}{2.296917in}}{\pgfqpoint{2.788465in}{2.285867in}}%
\pgfpathcurveto{\pgfqpoint{2.788465in}{2.274817in}}{\pgfqpoint{2.792856in}{2.264218in}}{\pgfqpoint{2.800669in}{2.256404in}}%
\pgfpathcurveto{\pgfqpoint{2.808483in}{2.248591in}}{\pgfqpoint{2.819082in}{2.244200in}}{\pgfqpoint{2.830132in}{2.244200in}}%
\pgfpathclose%
\pgfusepath{stroke,fill}%
\end{pgfscope}%
\begin{pgfscope}%
\pgfpathrectangle{\pgfqpoint{0.600000in}{0.600000in}}{\pgfqpoint{3.900000in}{3.900000in}}%
\pgfusepath{clip}%
\pgfsetbuttcap%
\pgfsetroundjoin%
\definecolor{currentfill}{rgb}{0.121569,0.466667,0.705882}%
\pgfsetfillcolor{currentfill}%
\pgfsetlinewidth{1.003750pt}%
\definecolor{currentstroke}{rgb}{0.121569,0.466667,0.705882}%
\pgfsetstrokecolor{currentstroke}%
\pgfsetdash{}{0pt}%
\pgfpathmoveto{\pgfqpoint{2.129756in}{2.655845in}}%
\pgfpathcurveto{\pgfqpoint{2.140806in}{2.655845in}}{\pgfqpoint{2.151405in}{2.660235in}}{\pgfqpoint{2.159219in}{2.668049in}}%
\pgfpathcurveto{\pgfqpoint{2.167032in}{2.675863in}}{\pgfqpoint{2.171423in}{2.686462in}}{\pgfqpoint{2.171423in}{2.697512in}}%
\pgfpathcurveto{\pgfqpoint{2.171423in}{2.708562in}}{\pgfqpoint{2.167032in}{2.719161in}}{\pgfqpoint{2.159219in}{2.726974in}}%
\pgfpathcurveto{\pgfqpoint{2.151405in}{2.734788in}}{\pgfqpoint{2.140806in}{2.739178in}}{\pgfqpoint{2.129756in}{2.739178in}}%
\pgfpathcurveto{\pgfqpoint{2.118706in}{2.739178in}}{\pgfqpoint{2.108107in}{2.734788in}}{\pgfqpoint{2.100293in}{2.726974in}}%
\pgfpathcurveto{\pgfqpoint{2.092479in}{2.719161in}}{\pgfqpoint{2.088089in}{2.708562in}}{\pgfqpoint{2.088089in}{2.697512in}}%
\pgfpathcurveto{\pgfqpoint{2.088089in}{2.686462in}}{\pgfqpoint{2.092479in}{2.675863in}}{\pgfqpoint{2.100293in}{2.668049in}}%
\pgfpathcurveto{\pgfqpoint{2.108107in}{2.660235in}}{\pgfqpoint{2.118706in}{2.655845in}}{\pgfqpoint{2.129756in}{2.655845in}}%
\pgfpathclose%
\pgfusepath{stroke,fill}%
\end{pgfscope}%
\begin{pgfscope}%
\pgfpathrectangle{\pgfqpoint{0.600000in}{0.600000in}}{\pgfqpoint{3.900000in}{3.900000in}}%
\pgfusepath{clip}%
\pgfsetbuttcap%
\pgfsetroundjoin%
\definecolor{currentfill}{rgb}{0.121569,0.466667,0.705882}%
\pgfsetfillcolor{currentfill}%
\pgfsetlinewidth{1.003750pt}%
\definecolor{currentstroke}{rgb}{0.121569,0.466667,0.705882}%
\pgfsetstrokecolor{currentstroke}%
\pgfsetdash{}{0pt}%
\pgfpathmoveto{\pgfqpoint{2.476767in}{2.733954in}}%
\pgfpathcurveto{\pgfqpoint{2.487818in}{2.733954in}}{\pgfqpoint{2.498417in}{2.738344in}}{\pgfqpoint{2.506230in}{2.746157in}}%
\pgfpathcurveto{\pgfqpoint{2.514044in}{2.753971in}}{\pgfqpoint{2.518434in}{2.764570in}}{\pgfqpoint{2.518434in}{2.775620in}}%
\pgfpathcurveto{\pgfqpoint{2.518434in}{2.786670in}}{\pgfqpoint{2.514044in}{2.797269in}}{\pgfqpoint{2.506230in}{2.805083in}}%
\pgfpathcurveto{\pgfqpoint{2.498417in}{2.812897in}}{\pgfqpoint{2.487818in}{2.817287in}}{\pgfqpoint{2.476767in}{2.817287in}}%
\pgfpathcurveto{\pgfqpoint{2.465717in}{2.817287in}}{\pgfqpoint{2.455118in}{2.812897in}}{\pgfqpoint{2.447305in}{2.805083in}}%
\pgfpathcurveto{\pgfqpoint{2.439491in}{2.797269in}}{\pgfqpoint{2.435101in}{2.786670in}}{\pgfqpoint{2.435101in}{2.775620in}}%
\pgfpathcurveto{\pgfqpoint{2.435101in}{2.764570in}}{\pgfqpoint{2.439491in}{2.753971in}}{\pgfqpoint{2.447305in}{2.746157in}}%
\pgfpathcurveto{\pgfqpoint{2.455118in}{2.738344in}}{\pgfqpoint{2.465717in}{2.733954in}}{\pgfqpoint{2.476767in}{2.733954in}}%
\pgfpathclose%
\pgfusepath{stroke,fill}%
\end{pgfscope}%
\begin{pgfscope}%
\pgfpathrectangle{\pgfqpoint{0.600000in}{0.600000in}}{\pgfqpoint{3.900000in}{3.900000in}}%
\pgfusepath{clip}%
\pgfsetbuttcap%
\pgfsetroundjoin%
\definecolor{currentfill}{rgb}{0.121569,0.466667,0.705882}%
\pgfsetfillcolor{currentfill}%
\pgfsetlinewidth{1.003750pt}%
\definecolor{currentstroke}{rgb}{0.121569,0.466667,0.705882}%
\pgfsetstrokecolor{currentstroke}%
\pgfsetdash{}{0pt}%
\pgfpathmoveto{\pgfqpoint{2.474117in}{2.946962in}}%
\pgfpathcurveto{\pgfqpoint{2.485168in}{2.946962in}}{\pgfqpoint{2.495767in}{2.951352in}}{\pgfqpoint{2.503580in}{2.959165in}}%
\pgfpathcurveto{\pgfqpoint{2.511394in}{2.966979in}}{\pgfqpoint{2.515784in}{2.977578in}}{\pgfqpoint{2.515784in}{2.988628in}}%
\pgfpathcurveto{\pgfqpoint{2.515784in}{2.999678in}}{\pgfqpoint{2.511394in}{3.010277in}}{\pgfqpoint{2.503580in}{3.018091in}}%
\pgfpathcurveto{\pgfqpoint{2.495767in}{3.025905in}}{\pgfqpoint{2.485168in}{3.030295in}}{\pgfqpoint{2.474117in}{3.030295in}}%
\pgfpathcurveto{\pgfqpoint{2.463067in}{3.030295in}}{\pgfqpoint{2.452468in}{3.025905in}}{\pgfqpoint{2.444655in}{3.018091in}}%
\pgfpathcurveto{\pgfqpoint{2.436841in}{3.010277in}}{\pgfqpoint{2.432451in}{2.999678in}}{\pgfqpoint{2.432451in}{2.988628in}}%
\pgfpathcurveto{\pgfqpoint{2.432451in}{2.977578in}}{\pgfqpoint{2.436841in}{2.966979in}}{\pgfqpoint{2.444655in}{2.959165in}}%
\pgfpathcurveto{\pgfqpoint{2.452468in}{2.951352in}}{\pgfqpoint{2.463067in}{2.946962in}}{\pgfqpoint{2.474117in}{2.946962in}}%
\pgfpathclose%
\pgfusepath{stroke,fill}%
\end{pgfscope}%
\begin{pgfscope}%
\pgfpathrectangle{\pgfqpoint{0.600000in}{0.600000in}}{\pgfqpoint{3.900000in}{3.900000in}}%
\pgfusepath{clip}%
\pgfsetbuttcap%
\pgfsetroundjoin%
\definecolor{currentfill}{rgb}{0.121569,0.466667,0.705882}%
\pgfsetfillcolor{currentfill}%
\pgfsetlinewidth{1.003750pt}%
\definecolor{currentstroke}{rgb}{0.121569,0.466667,0.705882}%
\pgfsetstrokecolor{currentstroke}%
\pgfsetdash{}{0pt}%
\pgfpathmoveto{\pgfqpoint{3.112435in}{2.870681in}}%
\pgfpathcurveto{\pgfqpoint{3.123485in}{2.870681in}}{\pgfqpoint{3.134084in}{2.875071in}}{\pgfqpoint{3.141898in}{2.882884in}}%
\pgfpathcurveto{\pgfqpoint{3.149712in}{2.890698in}}{\pgfqpoint{3.154102in}{2.901297in}}{\pgfqpoint{3.154102in}{2.912347in}}%
\pgfpathcurveto{\pgfqpoint{3.154102in}{2.923397in}}{\pgfqpoint{3.149712in}{2.933996in}}{\pgfqpoint{3.141898in}{2.941810in}}%
\pgfpathcurveto{\pgfqpoint{3.134084in}{2.949624in}}{\pgfqpoint{3.123485in}{2.954014in}}{\pgfqpoint{3.112435in}{2.954014in}}%
\pgfpathcurveto{\pgfqpoint{3.101385in}{2.954014in}}{\pgfqpoint{3.090786in}{2.949624in}}{\pgfqpoint{3.082972in}{2.941810in}}%
\pgfpathcurveto{\pgfqpoint{3.075159in}{2.933996in}}{\pgfqpoint{3.070769in}{2.923397in}}{\pgfqpoint{3.070769in}{2.912347in}}%
\pgfpathcurveto{\pgfqpoint{3.070769in}{2.901297in}}{\pgfqpoint{3.075159in}{2.890698in}}{\pgfqpoint{3.082972in}{2.882884in}}%
\pgfpathcurveto{\pgfqpoint{3.090786in}{2.875071in}}{\pgfqpoint{3.101385in}{2.870681in}}{\pgfqpoint{3.112435in}{2.870681in}}%
\pgfpathclose%
\pgfusepath{stroke,fill}%
\end{pgfscope}%
\begin{pgfscope}%
\pgfpathrectangle{\pgfqpoint{0.600000in}{0.600000in}}{\pgfqpoint{3.900000in}{3.900000in}}%
\pgfusepath{clip}%
\pgfsetbuttcap%
\pgfsetroundjoin%
\definecolor{currentfill}{rgb}{0.121569,0.466667,0.705882}%
\pgfsetfillcolor{currentfill}%
\pgfsetlinewidth{1.003750pt}%
\definecolor{currentstroke}{rgb}{0.121569,0.466667,0.705882}%
\pgfsetstrokecolor{currentstroke}%
\pgfsetdash{}{0pt}%
\pgfpathmoveto{\pgfqpoint{3.624017in}{2.101758in}}%
\pgfpathcurveto{\pgfqpoint{3.635067in}{2.101758in}}{\pgfqpoint{3.645666in}{2.106148in}}{\pgfqpoint{3.653480in}{2.113961in}}%
\pgfpathcurveto{\pgfqpoint{3.661293in}{2.121775in}}{\pgfqpoint{3.665683in}{2.132374in}}{\pgfqpoint{3.665683in}{2.143424in}}%
\pgfpathcurveto{\pgfqpoint{3.665683in}{2.154474in}}{\pgfqpoint{3.661293in}{2.165073in}}{\pgfqpoint{3.653480in}{2.172887in}}%
\pgfpathcurveto{\pgfqpoint{3.645666in}{2.180701in}}{\pgfqpoint{3.635067in}{2.185091in}}{\pgfqpoint{3.624017in}{2.185091in}}%
\pgfpathcurveto{\pgfqpoint{3.612967in}{2.185091in}}{\pgfqpoint{3.602368in}{2.180701in}}{\pgfqpoint{3.594554in}{2.172887in}}%
\pgfpathcurveto{\pgfqpoint{3.586740in}{2.165073in}}{\pgfqpoint{3.582350in}{2.154474in}}{\pgfqpoint{3.582350in}{2.143424in}}%
\pgfpathcurveto{\pgfqpoint{3.582350in}{2.132374in}}{\pgfqpoint{3.586740in}{2.121775in}}{\pgfqpoint{3.594554in}{2.113961in}}%
\pgfpathcurveto{\pgfqpoint{3.602368in}{2.106148in}}{\pgfqpoint{3.612967in}{2.101758in}}{\pgfqpoint{3.624017in}{2.101758in}}%
\pgfpathclose%
\pgfusepath{stroke,fill}%
\end{pgfscope}%
\begin{pgfscope}%
\pgfpathrectangle{\pgfqpoint{0.600000in}{0.600000in}}{\pgfqpoint{3.900000in}{3.900000in}}%
\pgfusepath{clip}%
\pgfsetbuttcap%
\pgfsetroundjoin%
\definecolor{currentfill}{rgb}{0.121569,0.466667,0.705882}%
\pgfsetfillcolor{currentfill}%
\pgfsetlinewidth{1.003750pt}%
\definecolor{currentstroke}{rgb}{0.121569,0.466667,0.705882}%
\pgfsetstrokecolor{currentstroke}%
\pgfsetdash{}{0pt}%
\pgfpathmoveto{\pgfqpoint{3.001195in}{2.391768in}}%
\pgfpathcurveto{\pgfqpoint{3.012245in}{2.391768in}}{\pgfqpoint{3.022844in}{2.396158in}}{\pgfqpoint{3.030658in}{2.403972in}}%
\pgfpathcurveto{\pgfqpoint{3.038472in}{2.411786in}}{\pgfqpoint{3.042862in}{2.422385in}}{\pgfqpoint{3.042862in}{2.433435in}}%
\pgfpathcurveto{\pgfqpoint{3.042862in}{2.444485in}}{\pgfqpoint{3.038472in}{2.455084in}}{\pgfqpoint{3.030658in}{2.462898in}}%
\pgfpathcurveto{\pgfqpoint{3.022844in}{2.470711in}}{\pgfqpoint{3.012245in}{2.475101in}}{\pgfqpoint{3.001195in}{2.475101in}}%
\pgfpathcurveto{\pgfqpoint{2.990145in}{2.475101in}}{\pgfqpoint{2.979546in}{2.470711in}}{\pgfqpoint{2.971733in}{2.462898in}}%
\pgfpathcurveto{\pgfqpoint{2.963919in}{2.455084in}}{\pgfqpoint{2.959529in}{2.444485in}}{\pgfqpoint{2.959529in}{2.433435in}}%
\pgfpathcurveto{\pgfqpoint{2.959529in}{2.422385in}}{\pgfqpoint{2.963919in}{2.411786in}}{\pgfqpoint{2.971733in}{2.403972in}}%
\pgfpathcurveto{\pgfqpoint{2.979546in}{2.396158in}}{\pgfqpoint{2.990145in}{2.391768in}}{\pgfqpoint{3.001195in}{2.391768in}}%
\pgfpathclose%
\pgfusepath{stroke,fill}%
\end{pgfscope}%
\begin{pgfscope}%
\pgfpathrectangle{\pgfqpoint{0.600000in}{0.600000in}}{\pgfqpoint{3.900000in}{3.900000in}}%
\pgfusepath{clip}%
\pgfsetbuttcap%
\pgfsetroundjoin%
\definecolor{currentfill}{rgb}{0.121569,0.466667,0.705882}%
\pgfsetfillcolor{currentfill}%
\pgfsetlinewidth{1.003750pt}%
\definecolor{currentstroke}{rgb}{0.121569,0.466667,0.705882}%
\pgfsetstrokecolor{currentstroke}%
\pgfsetdash{}{0pt}%
\pgfpathmoveto{\pgfqpoint{3.097695in}{1.778689in}}%
\pgfpathcurveto{\pgfqpoint{3.108745in}{1.778689in}}{\pgfqpoint{3.119344in}{1.783079in}}{\pgfqpoint{3.127158in}{1.790893in}}%
\pgfpathcurveto{\pgfqpoint{3.134971in}{1.798706in}}{\pgfqpoint{3.139362in}{1.809305in}}{\pgfqpoint{3.139362in}{1.820356in}}%
\pgfpathcurveto{\pgfqpoint{3.139362in}{1.831406in}}{\pgfqpoint{3.134971in}{1.842005in}}{\pgfqpoint{3.127158in}{1.849818in}}%
\pgfpathcurveto{\pgfqpoint{3.119344in}{1.857632in}}{\pgfqpoint{3.108745in}{1.862022in}}{\pgfqpoint{3.097695in}{1.862022in}}%
\pgfpathcurveto{\pgfqpoint{3.086645in}{1.862022in}}{\pgfqpoint{3.076046in}{1.857632in}}{\pgfqpoint{3.068232in}{1.849818in}}%
\pgfpathcurveto{\pgfqpoint{3.060419in}{1.842005in}}{\pgfqpoint{3.056028in}{1.831406in}}{\pgfqpoint{3.056028in}{1.820356in}}%
\pgfpathcurveto{\pgfqpoint{3.056028in}{1.809305in}}{\pgfqpoint{3.060419in}{1.798706in}}{\pgfqpoint{3.068232in}{1.790893in}}%
\pgfpathcurveto{\pgfqpoint{3.076046in}{1.783079in}}{\pgfqpoint{3.086645in}{1.778689in}}{\pgfqpoint{3.097695in}{1.778689in}}%
\pgfpathclose%
\pgfusepath{stroke,fill}%
\end{pgfscope}%
\begin{pgfscope}%
\pgfpathrectangle{\pgfqpoint{0.600000in}{0.600000in}}{\pgfqpoint{3.900000in}{3.900000in}}%
\pgfusepath{clip}%
\pgfsetbuttcap%
\pgfsetroundjoin%
\definecolor{currentfill}{rgb}{0.121569,0.466667,0.705882}%
\pgfsetfillcolor{currentfill}%
\pgfsetlinewidth{1.003750pt}%
\definecolor{currentstroke}{rgb}{0.121569,0.466667,0.705882}%
\pgfsetstrokecolor{currentstroke}%
\pgfsetdash{}{0pt}%
\pgfpathmoveto{\pgfqpoint{2.666782in}{2.055143in}}%
\pgfpathcurveto{\pgfqpoint{2.677832in}{2.055143in}}{\pgfqpoint{2.688431in}{2.059533in}}{\pgfqpoint{2.696244in}{2.067347in}}%
\pgfpathcurveto{\pgfqpoint{2.704058in}{2.075160in}}{\pgfqpoint{2.708448in}{2.085759in}}{\pgfqpoint{2.708448in}{2.096810in}}%
\pgfpathcurveto{\pgfqpoint{2.708448in}{2.107860in}}{\pgfqpoint{2.704058in}{2.118459in}}{\pgfqpoint{2.696244in}{2.126272in}}%
\pgfpathcurveto{\pgfqpoint{2.688431in}{2.134086in}}{\pgfqpoint{2.677832in}{2.138476in}}{\pgfqpoint{2.666782in}{2.138476in}}%
\pgfpathcurveto{\pgfqpoint{2.655732in}{2.138476in}}{\pgfqpoint{2.645133in}{2.134086in}}{\pgfqpoint{2.637319in}{2.126272in}}%
\pgfpathcurveto{\pgfqpoint{2.629505in}{2.118459in}}{\pgfqpoint{2.625115in}{2.107860in}}{\pgfqpoint{2.625115in}{2.096810in}}%
\pgfpathcurveto{\pgfqpoint{2.625115in}{2.085759in}}{\pgfqpoint{2.629505in}{2.075160in}}{\pgfqpoint{2.637319in}{2.067347in}}%
\pgfpathcurveto{\pgfqpoint{2.645133in}{2.059533in}}{\pgfqpoint{2.655732in}{2.055143in}}{\pgfqpoint{2.666782in}{2.055143in}}%
\pgfpathclose%
\pgfusepath{stroke,fill}%
\end{pgfscope}%
\begin{pgfscope}%
\pgfpathrectangle{\pgfqpoint{0.600000in}{0.600000in}}{\pgfqpoint{3.900000in}{3.900000in}}%
\pgfusepath{clip}%
\pgfsetbuttcap%
\pgfsetroundjoin%
\definecolor{currentfill}{rgb}{0.121569,0.466667,0.705882}%
\pgfsetfillcolor{currentfill}%
\pgfsetlinewidth{1.003750pt}%
\definecolor{currentstroke}{rgb}{0.121569,0.466667,0.705882}%
\pgfsetstrokecolor{currentstroke}%
\pgfsetdash{}{0pt}%
\pgfpathmoveto{\pgfqpoint{2.597045in}{2.879217in}}%
\pgfpathcurveto{\pgfqpoint{2.608095in}{2.879217in}}{\pgfqpoint{2.618695in}{2.883607in}}{\pgfqpoint{2.626508in}{2.891421in}}%
\pgfpathcurveto{\pgfqpoint{2.634322in}{2.899234in}}{\pgfqpoint{2.638712in}{2.909833in}}{\pgfqpoint{2.638712in}{2.920883in}}%
\pgfpathcurveto{\pgfqpoint{2.638712in}{2.931934in}}{\pgfqpoint{2.634322in}{2.942533in}}{\pgfqpoint{2.626508in}{2.950346in}}%
\pgfpathcurveto{\pgfqpoint{2.618695in}{2.958160in}}{\pgfqpoint{2.608095in}{2.962550in}}{\pgfqpoint{2.597045in}{2.962550in}}%
\pgfpathcurveto{\pgfqpoint{2.585995in}{2.962550in}}{\pgfqpoint{2.575396in}{2.958160in}}{\pgfqpoint{2.567583in}{2.950346in}}%
\pgfpathcurveto{\pgfqpoint{2.559769in}{2.942533in}}{\pgfqpoint{2.555379in}{2.931934in}}{\pgfqpoint{2.555379in}{2.920883in}}%
\pgfpathcurveto{\pgfqpoint{2.555379in}{2.909833in}}{\pgfqpoint{2.559769in}{2.899234in}}{\pgfqpoint{2.567583in}{2.891421in}}%
\pgfpathcurveto{\pgfqpoint{2.575396in}{2.883607in}}{\pgfqpoint{2.585995in}{2.879217in}}{\pgfqpoint{2.597045in}{2.879217in}}%
\pgfpathclose%
\pgfusepath{stroke,fill}%
\end{pgfscope}%
\begin{pgfscope}%
\pgfpathrectangle{\pgfqpoint{0.600000in}{0.600000in}}{\pgfqpoint{3.900000in}{3.900000in}}%
\pgfusepath{clip}%
\pgfsetbuttcap%
\pgfsetroundjoin%
\definecolor{currentfill}{rgb}{0.121569,0.466667,0.705882}%
\pgfsetfillcolor{currentfill}%
\pgfsetlinewidth{1.003750pt}%
\definecolor{currentstroke}{rgb}{0.121569,0.466667,0.705882}%
\pgfsetstrokecolor{currentstroke}%
\pgfsetdash{}{0pt}%
\pgfpathmoveto{\pgfqpoint{2.699537in}{1.365811in}}%
\pgfpathcurveto{\pgfqpoint{2.710587in}{1.365811in}}{\pgfqpoint{2.721186in}{1.370201in}}{\pgfqpoint{2.729000in}{1.378015in}}%
\pgfpathcurveto{\pgfqpoint{2.736813in}{1.385828in}}{\pgfqpoint{2.741203in}{1.396427in}}{\pgfqpoint{2.741203in}{1.407477in}}%
\pgfpathcurveto{\pgfqpoint{2.741203in}{1.418527in}}{\pgfqpoint{2.736813in}{1.429126in}}{\pgfqpoint{2.729000in}{1.436940in}}%
\pgfpathcurveto{\pgfqpoint{2.721186in}{1.444754in}}{\pgfqpoint{2.710587in}{1.449144in}}{\pgfqpoint{2.699537in}{1.449144in}}%
\pgfpathcurveto{\pgfqpoint{2.688487in}{1.449144in}}{\pgfqpoint{2.677888in}{1.444754in}}{\pgfqpoint{2.670074in}{1.436940in}}%
\pgfpathcurveto{\pgfqpoint{2.662260in}{1.429126in}}{\pgfqpoint{2.657870in}{1.418527in}}{\pgfqpoint{2.657870in}{1.407477in}}%
\pgfpathcurveto{\pgfqpoint{2.657870in}{1.396427in}}{\pgfqpoint{2.662260in}{1.385828in}}{\pgfqpoint{2.670074in}{1.378015in}}%
\pgfpathcurveto{\pgfqpoint{2.677888in}{1.370201in}}{\pgfqpoint{2.688487in}{1.365811in}}{\pgfqpoint{2.699537in}{1.365811in}}%
\pgfpathclose%
\pgfusepath{stroke,fill}%
\end{pgfscope}%
\begin{pgfscope}%
\pgfpathrectangle{\pgfqpoint{0.600000in}{0.600000in}}{\pgfqpoint{3.900000in}{3.900000in}}%
\pgfusepath{clip}%
\pgfsetbuttcap%
\pgfsetroundjoin%
\definecolor{currentfill}{rgb}{0.121569,0.466667,0.705882}%
\pgfsetfillcolor{currentfill}%
\pgfsetlinewidth{1.003750pt}%
\definecolor{currentstroke}{rgb}{0.121569,0.466667,0.705882}%
\pgfsetstrokecolor{currentstroke}%
\pgfsetdash{}{0pt}%
\pgfpathmoveto{\pgfqpoint{3.176169in}{2.797721in}}%
\pgfpathcurveto{\pgfqpoint{3.187220in}{2.797721in}}{\pgfqpoint{3.197819in}{2.802111in}}{\pgfqpoint{3.205632in}{2.809925in}}%
\pgfpathcurveto{\pgfqpoint{3.213446in}{2.817738in}}{\pgfqpoint{3.217836in}{2.828337in}}{\pgfqpoint{3.217836in}{2.839388in}}%
\pgfpathcurveto{\pgfqpoint{3.217836in}{2.850438in}}{\pgfqpoint{3.213446in}{2.861037in}}{\pgfqpoint{3.205632in}{2.868850in}}%
\pgfpathcurveto{\pgfqpoint{3.197819in}{2.876664in}}{\pgfqpoint{3.187220in}{2.881054in}}{\pgfqpoint{3.176169in}{2.881054in}}%
\pgfpathcurveto{\pgfqpoint{3.165119in}{2.881054in}}{\pgfqpoint{3.154520in}{2.876664in}}{\pgfqpoint{3.146707in}{2.868850in}}%
\pgfpathcurveto{\pgfqpoint{3.138893in}{2.861037in}}{\pgfqpoint{3.134503in}{2.850438in}}{\pgfqpoint{3.134503in}{2.839388in}}%
\pgfpathcurveto{\pgfqpoint{3.134503in}{2.828337in}}{\pgfqpoint{3.138893in}{2.817738in}}{\pgfqpoint{3.146707in}{2.809925in}}%
\pgfpathcurveto{\pgfqpoint{3.154520in}{2.802111in}}{\pgfqpoint{3.165119in}{2.797721in}}{\pgfqpoint{3.176169in}{2.797721in}}%
\pgfpathclose%
\pgfusepath{stroke,fill}%
\end{pgfscope}%
\begin{pgfscope}%
\pgfpathrectangle{\pgfqpoint{0.600000in}{0.600000in}}{\pgfqpoint{3.900000in}{3.900000in}}%
\pgfusepath{clip}%
\pgfsetbuttcap%
\pgfsetroundjoin%
\definecolor{currentfill}{rgb}{0.121569,0.466667,0.705882}%
\pgfsetfillcolor{currentfill}%
\pgfsetlinewidth{1.003750pt}%
\definecolor{currentstroke}{rgb}{0.121569,0.466667,0.705882}%
\pgfsetstrokecolor{currentstroke}%
\pgfsetdash{}{0pt}%
\pgfpathmoveto{\pgfqpoint{2.885689in}{2.199655in}}%
\pgfpathcurveto{\pgfqpoint{2.896739in}{2.199655in}}{\pgfqpoint{2.907338in}{2.204046in}}{\pgfqpoint{2.915152in}{2.211859in}}%
\pgfpathcurveto{\pgfqpoint{2.922965in}{2.219673in}}{\pgfqpoint{2.927356in}{2.230272in}}{\pgfqpoint{2.927356in}{2.241322in}}%
\pgfpathcurveto{\pgfqpoint{2.927356in}{2.252372in}}{\pgfqpoint{2.922965in}{2.262971in}}{\pgfqpoint{2.915152in}{2.270785in}}%
\pgfpathcurveto{\pgfqpoint{2.907338in}{2.278598in}}{\pgfqpoint{2.896739in}{2.282989in}}{\pgfqpoint{2.885689in}{2.282989in}}%
\pgfpathcurveto{\pgfqpoint{2.874639in}{2.282989in}}{\pgfqpoint{2.864040in}{2.278598in}}{\pgfqpoint{2.856226in}{2.270785in}}%
\pgfpathcurveto{\pgfqpoint{2.848413in}{2.262971in}}{\pgfqpoint{2.844022in}{2.252372in}}{\pgfqpoint{2.844022in}{2.241322in}}%
\pgfpathcurveto{\pgfqpoint{2.844022in}{2.230272in}}{\pgfqpoint{2.848413in}{2.219673in}}{\pgfqpoint{2.856226in}{2.211859in}}%
\pgfpathcurveto{\pgfqpoint{2.864040in}{2.204046in}}{\pgfqpoint{2.874639in}{2.199655in}}{\pgfqpoint{2.885689in}{2.199655in}}%
\pgfpathclose%
\pgfusepath{stroke,fill}%
\end{pgfscope}%
\begin{pgfscope}%
\pgfpathrectangle{\pgfqpoint{0.600000in}{0.600000in}}{\pgfqpoint{3.900000in}{3.900000in}}%
\pgfusepath{clip}%
\pgfsetbuttcap%
\pgfsetroundjoin%
\definecolor{currentfill}{rgb}{0.121569,0.466667,0.705882}%
\pgfsetfillcolor{currentfill}%
\pgfsetlinewidth{1.003750pt}%
\definecolor{currentstroke}{rgb}{0.121569,0.466667,0.705882}%
\pgfsetstrokecolor{currentstroke}%
\pgfsetdash{}{0pt}%
\pgfpathmoveto{\pgfqpoint{2.627992in}{2.479142in}}%
\pgfpathcurveto{\pgfqpoint{2.639042in}{2.479142in}}{\pgfqpoint{2.649641in}{2.483533in}}{\pgfqpoint{2.657455in}{2.491346in}}%
\pgfpathcurveto{\pgfqpoint{2.665268in}{2.499160in}}{\pgfqpoint{2.669659in}{2.509759in}}{\pgfqpoint{2.669659in}{2.520809in}}%
\pgfpathcurveto{\pgfqpoint{2.669659in}{2.531859in}}{\pgfqpoint{2.665268in}{2.542458in}}{\pgfqpoint{2.657455in}{2.550272in}}%
\pgfpathcurveto{\pgfqpoint{2.649641in}{2.558086in}}{\pgfqpoint{2.639042in}{2.562476in}}{\pgfqpoint{2.627992in}{2.562476in}}%
\pgfpathcurveto{\pgfqpoint{2.616942in}{2.562476in}}{\pgfqpoint{2.606343in}{2.558086in}}{\pgfqpoint{2.598529in}{2.550272in}}%
\pgfpathcurveto{\pgfqpoint{2.590716in}{2.542458in}}{\pgfqpoint{2.586325in}{2.531859in}}{\pgfqpoint{2.586325in}{2.520809in}}%
\pgfpathcurveto{\pgfqpoint{2.586325in}{2.509759in}}{\pgfqpoint{2.590716in}{2.499160in}}{\pgfqpoint{2.598529in}{2.491346in}}%
\pgfpathcurveto{\pgfqpoint{2.606343in}{2.483533in}}{\pgfqpoint{2.616942in}{2.479142in}}{\pgfqpoint{2.627992in}{2.479142in}}%
\pgfpathclose%
\pgfusepath{stroke,fill}%
\end{pgfscope}%
\begin{pgfscope}%
\pgfpathrectangle{\pgfqpoint{0.600000in}{0.600000in}}{\pgfqpoint{3.900000in}{3.900000in}}%
\pgfusepath{clip}%
\pgfsetbuttcap%
\pgfsetroundjoin%
\definecolor{currentfill}{rgb}{0.121569,0.466667,0.705882}%
\pgfsetfillcolor{currentfill}%
\pgfsetlinewidth{1.003750pt}%
\definecolor{currentstroke}{rgb}{0.121569,0.466667,0.705882}%
\pgfsetstrokecolor{currentstroke}%
\pgfsetdash{}{0pt}%
\pgfpathmoveto{\pgfqpoint{2.861053in}{2.122108in}}%
\pgfpathcurveto{\pgfqpoint{2.872103in}{2.122108in}}{\pgfqpoint{2.882702in}{2.126499in}}{\pgfqpoint{2.890516in}{2.134312in}}%
\pgfpathcurveto{\pgfqpoint{2.898330in}{2.142126in}}{\pgfqpoint{2.902720in}{2.152725in}}{\pgfqpoint{2.902720in}{2.163775in}}%
\pgfpathcurveto{\pgfqpoint{2.902720in}{2.174825in}}{\pgfqpoint{2.898330in}{2.185424in}}{\pgfqpoint{2.890516in}{2.193238in}}%
\pgfpathcurveto{\pgfqpoint{2.882702in}{2.201051in}}{\pgfqpoint{2.872103in}{2.205442in}}{\pgfqpoint{2.861053in}{2.205442in}}%
\pgfpathcurveto{\pgfqpoint{2.850003in}{2.205442in}}{\pgfqpoint{2.839404in}{2.201051in}}{\pgfqpoint{2.831590in}{2.193238in}}%
\pgfpathcurveto{\pgfqpoint{2.823777in}{2.185424in}}{\pgfqpoint{2.819387in}{2.174825in}}{\pgfqpoint{2.819387in}{2.163775in}}%
\pgfpathcurveto{\pgfqpoint{2.819387in}{2.152725in}}{\pgfqpoint{2.823777in}{2.142126in}}{\pgfqpoint{2.831590in}{2.134312in}}%
\pgfpathcurveto{\pgfqpoint{2.839404in}{2.126499in}}{\pgfqpoint{2.850003in}{2.122108in}}{\pgfqpoint{2.861053in}{2.122108in}}%
\pgfpathclose%
\pgfusepath{stroke,fill}%
\end{pgfscope}%
\begin{pgfscope}%
\pgfpathrectangle{\pgfqpoint{0.600000in}{0.600000in}}{\pgfqpoint{3.900000in}{3.900000in}}%
\pgfusepath{clip}%
\pgfsetbuttcap%
\pgfsetroundjoin%
\definecolor{currentfill}{rgb}{0.121569,0.466667,0.705882}%
\pgfsetfillcolor{currentfill}%
\pgfsetlinewidth{1.003750pt}%
\definecolor{currentstroke}{rgb}{0.121569,0.466667,0.705882}%
\pgfsetstrokecolor{currentstroke}%
\pgfsetdash{}{0pt}%
\pgfpathmoveto{\pgfqpoint{3.556532in}{3.127359in}}%
\pgfpathcurveto{\pgfqpoint{3.567583in}{3.127359in}}{\pgfqpoint{3.578182in}{3.131750in}}{\pgfqpoint{3.585995in}{3.139563in}}%
\pgfpathcurveto{\pgfqpoint{3.593809in}{3.147377in}}{\pgfqpoint{3.598199in}{3.157976in}}{\pgfqpoint{3.598199in}{3.169026in}}%
\pgfpathcurveto{\pgfqpoint{3.598199in}{3.180076in}}{\pgfqpoint{3.593809in}{3.190675in}}{\pgfqpoint{3.585995in}{3.198489in}}%
\pgfpathcurveto{\pgfqpoint{3.578182in}{3.206302in}}{\pgfqpoint{3.567583in}{3.210693in}}{\pgfqpoint{3.556532in}{3.210693in}}%
\pgfpathcurveto{\pgfqpoint{3.545482in}{3.210693in}}{\pgfqpoint{3.534883in}{3.206302in}}{\pgfqpoint{3.527070in}{3.198489in}}%
\pgfpathcurveto{\pgfqpoint{3.519256in}{3.190675in}}{\pgfqpoint{3.514866in}{3.180076in}}{\pgfqpoint{3.514866in}{3.169026in}}%
\pgfpathcurveto{\pgfqpoint{3.514866in}{3.157976in}}{\pgfqpoint{3.519256in}{3.147377in}}{\pgfqpoint{3.527070in}{3.139563in}}%
\pgfpathcurveto{\pgfqpoint{3.534883in}{3.131750in}}{\pgfqpoint{3.545482in}{3.127359in}}{\pgfqpoint{3.556532in}{3.127359in}}%
\pgfpathclose%
\pgfusepath{stroke,fill}%
\end{pgfscope}%
\begin{pgfscope}%
\pgfpathrectangle{\pgfqpoint{0.600000in}{0.600000in}}{\pgfqpoint{3.900000in}{3.900000in}}%
\pgfusepath{clip}%
\pgfsetbuttcap%
\pgfsetroundjoin%
\definecolor{currentfill}{rgb}{0.121569,0.466667,0.705882}%
\pgfsetfillcolor{currentfill}%
\pgfsetlinewidth{1.003750pt}%
\definecolor{currentstroke}{rgb}{0.121569,0.466667,0.705882}%
\pgfsetstrokecolor{currentstroke}%
\pgfsetdash{}{0pt}%
\pgfpathmoveto{\pgfqpoint{2.952625in}{3.797061in}}%
\pgfpathcurveto{\pgfqpoint{2.963675in}{3.797061in}}{\pgfqpoint{2.974274in}{3.801451in}}{\pgfqpoint{2.982088in}{3.809265in}}%
\pgfpathcurveto{\pgfqpoint{2.989902in}{3.817079in}}{\pgfqpoint{2.994292in}{3.827678in}}{\pgfqpoint{2.994292in}{3.838728in}}%
\pgfpathcurveto{\pgfqpoint{2.994292in}{3.849778in}}{\pgfqpoint{2.989902in}{3.860377in}}{\pgfqpoint{2.982088in}{3.868191in}}%
\pgfpathcurveto{\pgfqpoint{2.974274in}{3.876004in}}{\pgfqpoint{2.963675in}{3.880394in}}{\pgfqpoint{2.952625in}{3.880394in}}%
\pgfpathcurveto{\pgfqpoint{2.941575in}{3.880394in}}{\pgfqpoint{2.930976in}{3.876004in}}{\pgfqpoint{2.923162in}{3.868191in}}%
\pgfpathcurveto{\pgfqpoint{2.915349in}{3.860377in}}{\pgfqpoint{2.910958in}{3.849778in}}{\pgfqpoint{2.910958in}{3.838728in}}%
\pgfpathcurveto{\pgfqpoint{2.910958in}{3.827678in}}{\pgfqpoint{2.915349in}{3.817079in}}{\pgfqpoint{2.923162in}{3.809265in}}%
\pgfpathcurveto{\pgfqpoint{2.930976in}{3.801451in}}{\pgfqpoint{2.941575in}{3.797061in}}{\pgfqpoint{2.952625in}{3.797061in}}%
\pgfpathclose%
\pgfusepath{stroke,fill}%
\end{pgfscope}%
\begin{pgfscope}%
\pgfpathrectangle{\pgfqpoint{0.600000in}{0.600000in}}{\pgfqpoint{3.900000in}{3.900000in}}%
\pgfusepath{clip}%
\pgfsetbuttcap%
\pgfsetroundjoin%
\definecolor{currentfill}{rgb}{0.121569,0.466667,0.705882}%
\pgfsetfillcolor{currentfill}%
\pgfsetlinewidth{1.003750pt}%
\definecolor{currentstroke}{rgb}{0.121569,0.466667,0.705882}%
\pgfsetstrokecolor{currentstroke}%
\pgfsetdash{}{0pt}%
\pgfpathmoveto{\pgfqpoint{3.772528in}{3.026491in}}%
\pgfpathcurveto{\pgfqpoint{3.783578in}{3.026491in}}{\pgfqpoint{3.794177in}{3.030881in}}{\pgfqpoint{3.801991in}{3.038695in}}%
\pgfpathcurveto{\pgfqpoint{3.809804in}{3.046509in}}{\pgfqpoint{3.814195in}{3.057108in}}{\pgfqpoint{3.814195in}{3.068158in}}%
\pgfpathcurveto{\pgfqpoint{3.814195in}{3.079208in}}{\pgfqpoint{3.809804in}{3.089807in}}{\pgfqpoint{3.801991in}{3.097621in}}%
\pgfpathcurveto{\pgfqpoint{3.794177in}{3.105434in}}{\pgfqpoint{3.783578in}{3.109824in}}{\pgfqpoint{3.772528in}{3.109824in}}%
\pgfpathcurveto{\pgfqpoint{3.761478in}{3.109824in}}{\pgfqpoint{3.750879in}{3.105434in}}{\pgfqpoint{3.743065in}{3.097621in}}%
\pgfpathcurveto{\pgfqpoint{3.735252in}{3.089807in}}{\pgfqpoint{3.730861in}{3.079208in}}{\pgfqpoint{3.730861in}{3.068158in}}%
\pgfpathcurveto{\pgfqpoint{3.730861in}{3.057108in}}{\pgfqpoint{3.735252in}{3.046509in}}{\pgfqpoint{3.743065in}{3.038695in}}%
\pgfpathcurveto{\pgfqpoint{3.750879in}{3.030881in}}{\pgfqpoint{3.761478in}{3.026491in}}{\pgfqpoint{3.772528in}{3.026491in}}%
\pgfpathclose%
\pgfusepath{stroke,fill}%
\end{pgfscope}%
\begin{pgfscope}%
\pgfpathrectangle{\pgfqpoint{0.600000in}{0.600000in}}{\pgfqpoint{3.900000in}{3.900000in}}%
\pgfusepath{clip}%
\pgfsetbuttcap%
\pgfsetroundjoin%
\definecolor{currentfill}{rgb}{0.121569,0.466667,0.705882}%
\pgfsetfillcolor{currentfill}%
\pgfsetlinewidth{1.003750pt}%
\definecolor{currentstroke}{rgb}{0.121569,0.466667,0.705882}%
\pgfsetstrokecolor{currentstroke}%
\pgfsetdash{}{0pt}%
\pgfpathmoveto{\pgfqpoint{2.575875in}{3.249391in}}%
\pgfpathcurveto{\pgfqpoint{2.586925in}{3.249391in}}{\pgfqpoint{2.597524in}{3.253781in}}{\pgfqpoint{2.605338in}{3.261594in}}%
\pgfpathcurveto{\pgfqpoint{2.613151in}{3.269408in}}{\pgfqpoint{2.617542in}{3.280007in}}{\pgfqpoint{2.617542in}{3.291057in}}%
\pgfpathcurveto{\pgfqpoint{2.617542in}{3.302107in}}{\pgfqpoint{2.613151in}{3.312706in}}{\pgfqpoint{2.605338in}{3.320520in}}%
\pgfpathcurveto{\pgfqpoint{2.597524in}{3.328334in}}{\pgfqpoint{2.586925in}{3.332724in}}{\pgfqpoint{2.575875in}{3.332724in}}%
\pgfpathcurveto{\pgfqpoint{2.564825in}{3.332724in}}{\pgfqpoint{2.554226in}{3.328334in}}{\pgfqpoint{2.546412in}{3.320520in}}%
\pgfpathcurveto{\pgfqpoint{2.538599in}{3.312706in}}{\pgfqpoint{2.534208in}{3.302107in}}{\pgfqpoint{2.534208in}{3.291057in}}%
\pgfpathcurveto{\pgfqpoint{2.534208in}{3.280007in}}{\pgfqpoint{2.538599in}{3.269408in}}{\pgfqpoint{2.546412in}{3.261594in}}%
\pgfpathcurveto{\pgfqpoint{2.554226in}{3.253781in}}{\pgfqpoint{2.564825in}{3.249391in}}{\pgfqpoint{2.575875in}{3.249391in}}%
\pgfpathclose%
\pgfusepath{stroke,fill}%
\end{pgfscope}%
\begin{pgfscope}%
\pgfpathrectangle{\pgfqpoint{0.600000in}{0.600000in}}{\pgfqpoint{3.900000in}{3.900000in}}%
\pgfusepath{clip}%
\pgfsetbuttcap%
\pgfsetroundjoin%
\definecolor{currentfill}{rgb}{0.121569,0.466667,0.705882}%
\pgfsetfillcolor{currentfill}%
\pgfsetlinewidth{1.003750pt}%
\definecolor{currentstroke}{rgb}{0.121569,0.466667,0.705882}%
\pgfsetstrokecolor{currentstroke}%
\pgfsetdash{}{0pt}%
\pgfpathmoveto{\pgfqpoint{2.531844in}{2.859919in}}%
\pgfpathcurveto{\pgfqpoint{2.542894in}{2.859919in}}{\pgfqpoint{2.553493in}{2.864309in}}{\pgfqpoint{2.561306in}{2.872123in}}%
\pgfpathcurveto{\pgfqpoint{2.569120in}{2.879936in}}{\pgfqpoint{2.573510in}{2.890535in}}{\pgfqpoint{2.573510in}{2.901586in}}%
\pgfpathcurveto{\pgfqpoint{2.573510in}{2.912636in}}{\pgfqpoint{2.569120in}{2.923235in}}{\pgfqpoint{2.561306in}{2.931048in}}%
\pgfpathcurveto{\pgfqpoint{2.553493in}{2.938862in}}{\pgfqpoint{2.542894in}{2.943252in}}{\pgfqpoint{2.531844in}{2.943252in}}%
\pgfpathcurveto{\pgfqpoint{2.520793in}{2.943252in}}{\pgfqpoint{2.510194in}{2.938862in}}{\pgfqpoint{2.502381in}{2.931048in}}%
\pgfpathcurveto{\pgfqpoint{2.494567in}{2.923235in}}{\pgfqpoint{2.490177in}{2.912636in}}{\pgfqpoint{2.490177in}{2.901586in}}%
\pgfpathcurveto{\pgfqpoint{2.490177in}{2.890535in}}{\pgfqpoint{2.494567in}{2.879936in}}{\pgfqpoint{2.502381in}{2.872123in}}%
\pgfpathcurveto{\pgfqpoint{2.510194in}{2.864309in}}{\pgfqpoint{2.520793in}{2.859919in}}{\pgfqpoint{2.531844in}{2.859919in}}%
\pgfpathclose%
\pgfusepath{stroke,fill}%
\end{pgfscope}%
\begin{pgfscope}%
\pgfpathrectangle{\pgfqpoint{0.600000in}{0.600000in}}{\pgfqpoint{3.900000in}{3.900000in}}%
\pgfusepath{clip}%
\pgfsetbuttcap%
\pgfsetroundjoin%
\definecolor{currentfill}{rgb}{0.121569,0.466667,0.705882}%
\pgfsetfillcolor{currentfill}%
\pgfsetlinewidth{1.003750pt}%
\definecolor{currentstroke}{rgb}{0.121569,0.466667,0.705882}%
\pgfsetstrokecolor{currentstroke}%
\pgfsetdash{}{0pt}%
\pgfpathmoveto{\pgfqpoint{2.144700in}{3.211184in}}%
\pgfpathcurveto{\pgfqpoint{2.155751in}{3.211184in}}{\pgfqpoint{2.166350in}{3.215575in}}{\pgfqpoint{2.174163in}{3.223388in}}%
\pgfpathcurveto{\pgfqpoint{2.181977in}{3.231202in}}{\pgfqpoint{2.186367in}{3.241801in}}{\pgfqpoint{2.186367in}{3.252851in}}%
\pgfpathcurveto{\pgfqpoint{2.186367in}{3.263901in}}{\pgfqpoint{2.181977in}{3.274500in}}{\pgfqpoint{2.174163in}{3.282314in}}%
\pgfpathcurveto{\pgfqpoint{2.166350in}{3.290127in}}{\pgfqpoint{2.155751in}{3.294518in}}{\pgfqpoint{2.144700in}{3.294518in}}%
\pgfpathcurveto{\pgfqpoint{2.133650in}{3.294518in}}{\pgfqpoint{2.123051in}{3.290127in}}{\pgfqpoint{2.115238in}{3.282314in}}%
\pgfpathcurveto{\pgfqpoint{2.107424in}{3.274500in}}{\pgfqpoint{2.103034in}{3.263901in}}{\pgfqpoint{2.103034in}{3.252851in}}%
\pgfpathcurveto{\pgfqpoint{2.103034in}{3.241801in}}{\pgfqpoint{2.107424in}{3.231202in}}{\pgfqpoint{2.115238in}{3.223388in}}%
\pgfpathcurveto{\pgfqpoint{2.123051in}{3.215575in}}{\pgfqpoint{2.133650in}{3.211184in}}{\pgfqpoint{2.144700in}{3.211184in}}%
\pgfpathclose%
\pgfusepath{stroke,fill}%
\end{pgfscope}%
\begin{pgfscope}%
\pgfpathrectangle{\pgfqpoint{0.600000in}{0.600000in}}{\pgfqpoint{3.900000in}{3.900000in}}%
\pgfusepath{clip}%
\pgfsetbuttcap%
\pgfsetroundjoin%
\definecolor{currentfill}{rgb}{0.121569,0.466667,0.705882}%
\pgfsetfillcolor{currentfill}%
\pgfsetlinewidth{1.003750pt}%
\definecolor{currentstroke}{rgb}{0.121569,0.466667,0.705882}%
\pgfsetstrokecolor{currentstroke}%
\pgfsetdash{}{0pt}%
\pgfpathmoveto{\pgfqpoint{2.326312in}{3.236177in}}%
\pgfpathcurveto{\pgfqpoint{2.337363in}{3.236177in}}{\pgfqpoint{2.347962in}{3.240568in}}{\pgfqpoint{2.355775in}{3.248381in}}%
\pgfpathcurveto{\pgfqpoint{2.363589in}{3.256195in}}{\pgfqpoint{2.367979in}{3.266794in}}{\pgfqpoint{2.367979in}{3.277844in}}%
\pgfpathcurveto{\pgfqpoint{2.367979in}{3.288894in}}{\pgfqpoint{2.363589in}{3.299493in}}{\pgfqpoint{2.355775in}{3.307307in}}%
\pgfpathcurveto{\pgfqpoint{2.347962in}{3.315120in}}{\pgfqpoint{2.337363in}{3.319511in}}{\pgfqpoint{2.326312in}{3.319511in}}%
\pgfpathcurveto{\pgfqpoint{2.315262in}{3.319511in}}{\pgfqpoint{2.304663in}{3.315120in}}{\pgfqpoint{2.296850in}{3.307307in}}%
\pgfpathcurveto{\pgfqpoint{2.289036in}{3.299493in}}{\pgfqpoint{2.284646in}{3.288894in}}{\pgfqpoint{2.284646in}{3.277844in}}%
\pgfpathcurveto{\pgfqpoint{2.284646in}{3.266794in}}{\pgfqpoint{2.289036in}{3.256195in}}{\pgfqpoint{2.296850in}{3.248381in}}%
\pgfpathcurveto{\pgfqpoint{2.304663in}{3.240568in}}{\pgfqpoint{2.315262in}{3.236177in}}{\pgfqpoint{2.326312in}{3.236177in}}%
\pgfpathclose%
\pgfusepath{stroke,fill}%
\end{pgfscope}%
\begin{pgfscope}%
\pgfpathrectangle{\pgfqpoint{0.600000in}{0.600000in}}{\pgfqpoint{3.900000in}{3.900000in}}%
\pgfusepath{clip}%
\pgfsetbuttcap%
\pgfsetroundjoin%
\definecolor{currentfill}{rgb}{0.121569,0.466667,0.705882}%
\pgfsetfillcolor{currentfill}%
\pgfsetlinewidth{1.003750pt}%
\definecolor{currentstroke}{rgb}{0.121569,0.466667,0.705882}%
\pgfsetstrokecolor{currentstroke}%
\pgfsetdash{}{0pt}%
\pgfpathmoveto{\pgfqpoint{1.971665in}{2.867991in}}%
\pgfpathcurveto{\pgfqpoint{1.982715in}{2.867991in}}{\pgfqpoint{1.993314in}{2.872382in}}{\pgfqpoint{2.001128in}{2.880195in}}%
\pgfpathcurveto{\pgfqpoint{2.008941in}{2.888009in}}{\pgfqpoint{2.013332in}{2.898608in}}{\pgfqpoint{2.013332in}{2.909658in}}%
\pgfpathcurveto{\pgfqpoint{2.013332in}{2.920708in}}{\pgfqpoint{2.008941in}{2.931307in}}{\pgfqpoint{2.001128in}{2.939121in}}%
\pgfpathcurveto{\pgfqpoint{1.993314in}{2.946934in}}{\pgfqpoint{1.982715in}{2.951325in}}{\pgfqpoint{1.971665in}{2.951325in}}%
\pgfpathcurveto{\pgfqpoint{1.960615in}{2.951325in}}{\pgfqpoint{1.950016in}{2.946934in}}{\pgfqpoint{1.942202in}{2.939121in}}%
\pgfpathcurveto{\pgfqpoint{1.934389in}{2.931307in}}{\pgfqpoint{1.929998in}{2.920708in}}{\pgfqpoint{1.929998in}{2.909658in}}%
\pgfpathcurveto{\pgfqpoint{1.929998in}{2.898608in}}{\pgfqpoint{1.934389in}{2.888009in}}{\pgfqpoint{1.942202in}{2.880195in}}%
\pgfpathcurveto{\pgfqpoint{1.950016in}{2.872382in}}{\pgfqpoint{1.960615in}{2.867991in}}{\pgfqpoint{1.971665in}{2.867991in}}%
\pgfpathclose%
\pgfusepath{stroke,fill}%
\end{pgfscope}%
\begin{pgfscope}%
\pgfpathrectangle{\pgfqpoint{0.600000in}{0.600000in}}{\pgfqpoint{3.900000in}{3.900000in}}%
\pgfusepath{clip}%
\pgfsetbuttcap%
\pgfsetroundjoin%
\definecolor{currentfill}{rgb}{0.121569,0.466667,0.705882}%
\pgfsetfillcolor{currentfill}%
\pgfsetlinewidth{1.003750pt}%
\definecolor{currentstroke}{rgb}{0.121569,0.466667,0.705882}%
\pgfsetstrokecolor{currentstroke}%
\pgfsetdash{}{0pt}%
\pgfpathmoveto{\pgfqpoint{2.394376in}{2.933646in}}%
\pgfpathcurveto{\pgfqpoint{2.405426in}{2.933646in}}{\pgfqpoint{2.416025in}{2.938036in}}{\pgfqpoint{2.423839in}{2.945850in}}%
\pgfpathcurveto{\pgfqpoint{2.431652in}{2.953663in}}{\pgfqpoint{2.436042in}{2.964262in}}{\pgfqpoint{2.436042in}{2.975312in}}%
\pgfpathcurveto{\pgfqpoint{2.436042in}{2.986363in}}{\pgfqpoint{2.431652in}{2.996962in}}{\pgfqpoint{2.423839in}{3.004775in}}%
\pgfpathcurveto{\pgfqpoint{2.416025in}{3.012589in}}{\pgfqpoint{2.405426in}{3.016979in}}{\pgfqpoint{2.394376in}{3.016979in}}%
\pgfpathcurveto{\pgfqpoint{2.383326in}{3.016979in}}{\pgfqpoint{2.372727in}{3.012589in}}{\pgfqpoint{2.364913in}{3.004775in}}%
\pgfpathcurveto{\pgfqpoint{2.357099in}{2.996962in}}{\pgfqpoint{2.352709in}{2.986363in}}{\pgfqpoint{2.352709in}{2.975312in}}%
\pgfpathcurveto{\pgfqpoint{2.352709in}{2.964262in}}{\pgfqpoint{2.357099in}{2.953663in}}{\pgfqpoint{2.364913in}{2.945850in}}%
\pgfpathcurveto{\pgfqpoint{2.372727in}{2.938036in}}{\pgfqpoint{2.383326in}{2.933646in}}{\pgfqpoint{2.394376in}{2.933646in}}%
\pgfpathclose%
\pgfusepath{stroke,fill}%
\end{pgfscope}%
\begin{pgfscope}%
\pgfpathrectangle{\pgfqpoint{0.600000in}{0.600000in}}{\pgfqpoint{3.900000in}{3.900000in}}%
\pgfusepath{clip}%
\pgfsetbuttcap%
\pgfsetroundjoin%
\definecolor{currentfill}{rgb}{0.121569,0.466667,0.705882}%
\pgfsetfillcolor{currentfill}%
\pgfsetlinewidth{1.003750pt}%
\definecolor{currentstroke}{rgb}{0.121569,0.466667,0.705882}%
\pgfsetstrokecolor{currentstroke}%
\pgfsetdash{}{0pt}%
\pgfpathmoveto{\pgfqpoint{2.946355in}{2.559949in}}%
\pgfpathcurveto{\pgfqpoint{2.957406in}{2.559949in}}{\pgfqpoint{2.968005in}{2.564339in}}{\pgfqpoint{2.975818in}{2.572152in}}%
\pgfpathcurveto{\pgfqpoint{2.983632in}{2.579966in}}{\pgfqpoint{2.988022in}{2.590565in}}{\pgfqpoint{2.988022in}{2.601615in}}%
\pgfpathcurveto{\pgfqpoint{2.988022in}{2.612665in}}{\pgfqpoint{2.983632in}{2.623264in}}{\pgfqpoint{2.975818in}{2.631078in}}%
\pgfpathcurveto{\pgfqpoint{2.968005in}{2.638892in}}{\pgfqpoint{2.957406in}{2.643282in}}{\pgfqpoint{2.946355in}{2.643282in}}%
\pgfpathcurveto{\pgfqpoint{2.935305in}{2.643282in}}{\pgfqpoint{2.924706in}{2.638892in}}{\pgfqpoint{2.916893in}{2.631078in}}%
\pgfpathcurveto{\pgfqpoint{2.909079in}{2.623264in}}{\pgfqpoint{2.904689in}{2.612665in}}{\pgfqpoint{2.904689in}{2.601615in}}%
\pgfpathcurveto{\pgfqpoint{2.904689in}{2.590565in}}{\pgfqpoint{2.909079in}{2.579966in}}{\pgfqpoint{2.916893in}{2.572152in}}%
\pgfpathcurveto{\pgfqpoint{2.924706in}{2.564339in}}{\pgfqpoint{2.935305in}{2.559949in}}{\pgfqpoint{2.946355in}{2.559949in}}%
\pgfpathclose%
\pgfusepath{stroke,fill}%
\end{pgfscope}%
\begin{pgfscope}%
\pgfpathrectangle{\pgfqpoint{0.600000in}{0.600000in}}{\pgfqpoint{3.900000in}{3.900000in}}%
\pgfusepath{clip}%
\pgfsetbuttcap%
\pgfsetroundjoin%
\definecolor{currentfill}{rgb}{0.121569,0.466667,0.705882}%
\pgfsetfillcolor{currentfill}%
\pgfsetlinewidth{1.003750pt}%
\definecolor{currentstroke}{rgb}{0.121569,0.466667,0.705882}%
\pgfsetstrokecolor{currentstroke}%
\pgfsetdash{}{0pt}%
\pgfpathmoveto{\pgfqpoint{2.286990in}{2.035694in}}%
\pgfpathcurveto{\pgfqpoint{2.298040in}{2.035694in}}{\pgfqpoint{2.308639in}{2.040084in}}{\pgfqpoint{2.316452in}{2.047898in}}%
\pgfpathcurveto{\pgfqpoint{2.324266in}{2.055712in}}{\pgfqpoint{2.328656in}{2.066311in}}{\pgfqpoint{2.328656in}{2.077361in}}%
\pgfpathcurveto{\pgfqpoint{2.328656in}{2.088411in}}{\pgfqpoint{2.324266in}{2.099010in}}{\pgfqpoint{2.316452in}{2.106823in}}%
\pgfpathcurveto{\pgfqpoint{2.308639in}{2.114637in}}{\pgfqpoint{2.298040in}{2.119027in}}{\pgfqpoint{2.286990in}{2.119027in}}%
\pgfpathcurveto{\pgfqpoint{2.275940in}{2.119027in}}{\pgfqpoint{2.265341in}{2.114637in}}{\pgfqpoint{2.257527in}{2.106823in}}%
\pgfpathcurveto{\pgfqpoint{2.249713in}{2.099010in}}{\pgfqpoint{2.245323in}{2.088411in}}{\pgfqpoint{2.245323in}{2.077361in}}%
\pgfpathcurveto{\pgfqpoint{2.245323in}{2.066311in}}{\pgfqpoint{2.249713in}{2.055712in}}{\pgfqpoint{2.257527in}{2.047898in}}%
\pgfpathcurveto{\pgfqpoint{2.265341in}{2.040084in}}{\pgfqpoint{2.275940in}{2.035694in}}{\pgfqpoint{2.286990in}{2.035694in}}%
\pgfpathclose%
\pgfusepath{stroke,fill}%
\end{pgfscope}%
\begin{pgfscope}%
\pgfpathrectangle{\pgfqpoint{0.600000in}{0.600000in}}{\pgfqpoint{3.900000in}{3.900000in}}%
\pgfusepath{clip}%
\pgfsetbuttcap%
\pgfsetroundjoin%
\definecolor{currentfill}{rgb}{0.121569,0.466667,0.705882}%
\pgfsetfillcolor{currentfill}%
\pgfsetlinewidth{1.003750pt}%
\definecolor{currentstroke}{rgb}{0.121569,0.466667,0.705882}%
\pgfsetstrokecolor{currentstroke}%
\pgfsetdash{}{0pt}%
\pgfpathmoveto{\pgfqpoint{2.803228in}{2.090078in}}%
\pgfpathcurveto{\pgfqpoint{2.814278in}{2.090078in}}{\pgfqpoint{2.824877in}{2.094469in}}{\pgfqpoint{2.832691in}{2.102282in}}%
\pgfpathcurveto{\pgfqpoint{2.840504in}{2.110096in}}{\pgfqpoint{2.844895in}{2.120695in}}{\pgfqpoint{2.844895in}{2.131745in}}%
\pgfpathcurveto{\pgfqpoint{2.844895in}{2.142795in}}{\pgfqpoint{2.840504in}{2.153394in}}{\pgfqpoint{2.832691in}{2.161208in}}%
\pgfpathcurveto{\pgfqpoint{2.824877in}{2.169021in}}{\pgfqpoint{2.814278in}{2.173412in}}{\pgfqpoint{2.803228in}{2.173412in}}%
\pgfpathcurveto{\pgfqpoint{2.792178in}{2.173412in}}{\pgfqpoint{2.781579in}{2.169021in}}{\pgfqpoint{2.773765in}{2.161208in}}%
\pgfpathcurveto{\pgfqpoint{2.765952in}{2.153394in}}{\pgfqpoint{2.761561in}{2.142795in}}{\pgfqpoint{2.761561in}{2.131745in}}%
\pgfpathcurveto{\pgfqpoint{2.761561in}{2.120695in}}{\pgfqpoint{2.765952in}{2.110096in}}{\pgfqpoint{2.773765in}{2.102282in}}%
\pgfpathcurveto{\pgfqpoint{2.781579in}{2.094469in}}{\pgfqpoint{2.792178in}{2.090078in}}{\pgfqpoint{2.803228in}{2.090078in}}%
\pgfpathclose%
\pgfusepath{stroke,fill}%
\end{pgfscope}%
\begin{pgfscope}%
\pgfpathrectangle{\pgfqpoint{0.600000in}{0.600000in}}{\pgfqpoint{3.900000in}{3.900000in}}%
\pgfusepath{clip}%
\pgfsetbuttcap%
\pgfsetroundjoin%
\definecolor{currentfill}{rgb}{0.121569,0.466667,0.705882}%
\pgfsetfillcolor{currentfill}%
\pgfsetlinewidth{1.003750pt}%
\definecolor{currentstroke}{rgb}{0.121569,0.466667,0.705882}%
\pgfsetstrokecolor{currentstroke}%
\pgfsetdash{}{0pt}%
\pgfpathmoveto{\pgfqpoint{2.488471in}{3.059801in}}%
\pgfpathcurveto{\pgfqpoint{2.499521in}{3.059801in}}{\pgfqpoint{2.510120in}{3.064191in}}{\pgfqpoint{2.517934in}{3.072005in}}%
\pgfpathcurveto{\pgfqpoint{2.525748in}{3.079818in}}{\pgfqpoint{2.530138in}{3.090417in}}{\pgfqpoint{2.530138in}{3.101468in}}%
\pgfpathcurveto{\pgfqpoint{2.530138in}{3.112518in}}{\pgfqpoint{2.525748in}{3.123117in}}{\pgfqpoint{2.517934in}{3.130930in}}%
\pgfpathcurveto{\pgfqpoint{2.510120in}{3.138744in}}{\pgfqpoint{2.499521in}{3.143134in}}{\pgfqpoint{2.488471in}{3.143134in}}%
\pgfpathcurveto{\pgfqpoint{2.477421in}{3.143134in}}{\pgfqpoint{2.466822in}{3.138744in}}{\pgfqpoint{2.459008in}{3.130930in}}%
\pgfpathcurveto{\pgfqpoint{2.451195in}{3.123117in}}{\pgfqpoint{2.446805in}{3.112518in}}{\pgfqpoint{2.446805in}{3.101468in}}%
\pgfpathcurveto{\pgfqpoint{2.446805in}{3.090417in}}{\pgfqpoint{2.451195in}{3.079818in}}{\pgfqpoint{2.459008in}{3.072005in}}%
\pgfpathcurveto{\pgfqpoint{2.466822in}{3.064191in}}{\pgfqpoint{2.477421in}{3.059801in}}{\pgfqpoint{2.488471in}{3.059801in}}%
\pgfpathclose%
\pgfusepath{stroke,fill}%
\end{pgfscope}%
\begin{pgfscope}%
\pgfpathrectangle{\pgfqpoint{0.600000in}{0.600000in}}{\pgfqpoint{3.900000in}{3.900000in}}%
\pgfusepath{clip}%
\pgfsetbuttcap%
\pgfsetroundjoin%
\definecolor{currentfill}{rgb}{0.121569,0.466667,0.705882}%
\pgfsetfillcolor{currentfill}%
\pgfsetlinewidth{1.003750pt}%
\definecolor{currentstroke}{rgb}{0.121569,0.466667,0.705882}%
\pgfsetstrokecolor{currentstroke}%
\pgfsetdash{}{0pt}%
\pgfpathmoveto{\pgfqpoint{2.421142in}{2.513407in}}%
\pgfpathcurveto{\pgfqpoint{2.432192in}{2.513407in}}{\pgfqpoint{2.442791in}{2.517797in}}{\pgfqpoint{2.450604in}{2.525611in}}%
\pgfpathcurveto{\pgfqpoint{2.458418in}{2.533424in}}{\pgfqpoint{2.462808in}{2.544023in}}{\pgfqpoint{2.462808in}{2.555074in}}%
\pgfpathcurveto{\pgfqpoint{2.462808in}{2.566124in}}{\pgfqpoint{2.458418in}{2.576723in}}{\pgfqpoint{2.450604in}{2.584536in}}%
\pgfpathcurveto{\pgfqpoint{2.442791in}{2.592350in}}{\pgfqpoint{2.432192in}{2.596740in}}{\pgfqpoint{2.421142in}{2.596740in}}%
\pgfpathcurveto{\pgfqpoint{2.410091in}{2.596740in}}{\pgfqpoint{2.399492in}{2.592350in}}{\pgfqpoint{2.391679in}{2.584536in}}%
\pgfpathcurveto{\pgfqpoint{2.383865in}{2.576723in}}{\pgfqpoint{2.379475in}{2.566124in}}{\pgfqpoint{2.379475in}{2.555074in}}%
\pgfpathcurveto{\pgfqpoint{2.379475in}{2.544023in}}{\pgfqpoint{2.383865in}{2.533424in}}{\pgfqpoint{2.391679in}{2.525611in}}%
\pgfpathcurveto{\pgfqpoint{2.399492in}{2.517797in}}{\pgfqpoint{2.410091in}{2.513407in}}{\pgfqpoint{2.421142in}{2.513407in}}%
\pgfpathclose%
\pgfusepath{stroke,fill}%
\end{pgfscope}%
\begin{pgfscope}%
\pgfpathrectangle{\pgfqpoint{0.600000in}{0.600000in}}{\pgfqpoint{3.900000in}{3.900000in}}%
\pgfusepath{clip}%
\pgfsetbuttcap%
\pgfsetroundjoin%
\definecolor{currentfill}{rgb}{0.121569,0.466667,0.705882}%
\pgfsetfillcolor{currentfill}%
\pgfsetlinewidth{1.003750pt}%
\definecolor{currentstroke}{rgb}{0.121569,0.466667,0.705882}%
\pgfsetstrokecolor{currentstroke}%
\pgfsetdash{}{0pt}%
\pgfpathmoveto{\pgfqpoint{3.232033in}{1.829216in}}%
\pgfpathcurveto{\pgfqpoint{3.243083in}{1.829216in}}{\pgfqpoint{3.253682in}{1.833606in}}{\pgfqpoint{3.261496in}{1.841420in}}%
\pgfpathcurveto{\pgfqpoint{3.269310in}{1.849233in}}{\pgfqpoint{3.273700in}{1.859832in}}{\pgfqpoint{3.273700in}{1.870883in}}%
\pgfpathcurveto{\pgfqpoint{3.273700in}{1.881933in}}{\pgfqpoint{3.269310in}{1.892532in}}{\pgfqpoint{3.261496in}{1.900345in}}%
\pgfpathcurveto{\pgfqpoint{3.253682in}{1.908159in}}{\pgfqpoint{3.243083in}{1.912549in}}{\pgfqpoint{3.232033in}{1.912549in}}%
\pgfpathcurveto{\pgfqpoint{3.220983in}{1.912549in}}{\pgfqpoint{3.210384in}{1.908159in}}{\pgfqpoint{3.202570in}{1.900345in}}%
\pgfpathcurveto{\pgfqpoint{3.194757in}{1.892532in}}{\pgfqpoint{3.190366in}{1.881933in}}{\pgfqpoint{3.190366in}{1.870883in}}%
\pgfpathcurveto{\pgfqpoint{3.190366in}{1.859832in}}{\pgfqpoint{3.194757in}{1.849233in}}{\pgfqpoint{3.202570in}{1.841420in}}%
\pgfpathcurveto{\pgfqpoint{3.210384in}{1.833606in}}{\pgfqpoint{3.220983in}{1.829216in}}{\pgfqpoint{3.232033in}{1.829216in}}%
\pgfpathclose%
\pgfusepath{stroke,fill}%
\end{pgfscope}%
\begin{pgfscope}%
\pgfpathrectangle{\pgfqpoint{0.600000in}{0.600000in}}{\pgfqpoint{3.900000in}{3.900000in}}%
\pgfusepath{clip}%
\pgfsetbuttcap%
\pgfsetroundjoin%
\definecolor{currentfill}{rgb}{0.121569,0.466667,0.705882}%
\pgfsetfillcolor{currentfill}%
\pgfsetlinewidth{1.003750pt}%
\definecolor{currentstroke}{rgb}{0.121569,0.466667,0.705882}%
\pgfsetstrokecolor{currentstroke}%
\pgfsetdash{}{0pt}%
\pgfpathmoveto{\pgfqpoint{2.308588in}{2.032759in}}%
\pgfpathcurveto{\pgfqpoint{2.319638in}{2.032759in}}{\pgfqpoint{2.330237in}{2.037150in}}{\pgfqpoint{2.338051in}{2.044963in}}%
\pgfpathcurveto{\pgfqpoint{2.345865in}{2.052777in}}{\pgfqpoint{2.350255in}{2.063376in}}{\pgfqpoint{2.350255in}{2.074426in}}%
\pgfpathcurveto{\pgfqpoint{2.350255in}{2.085476in}}{\pgfqpoint{2.345865in}{2.096075in}}{\pgfqpoint{2.338051in}{2.103889in}}%
\pgfpathcurveto{\pgfqpoint{2.330237in}{2.111702in}}{\pgfqpoint{2.319638in}{2.116093in}}{\pgfqpoint{2.308588in}{2.116093in}}%
\pgfpathcurveto{\pgfqpoint{2.297538in}{2.116093in}}{\pgfqpoint{2.286939in}{2.111702in}}{\pgfqpoint{2.279125in}{2.103889in}}%
\pgfpathcurveto{\pgfqpoint{2.271312in}{2.096075in}}{\pgfqpoint{2.266921in}{2.085476in}}{\pgfqpoint{2.266921in}{2.074426in}}%
\pgfpathcurveto{\pgfqpoint{2.266921in}{2.063376in}}{\pgfqpoint{2.271312in}{2.052777in}}{\pgfqpoint{2.279125in}{2.044963in}}%
\pgfpathcurveto{\pgfqpoint{2.286939in}{2.037150in}}{\pgfqpoint{2.297538in}{2.032759in}}{\pgfqpoint{2.308588in}{2.032759in}}%
\pgfpathclose%
\pgfusepath{stroke,fill}%
\end{pgfscope}%
\begin{pgfscope}%
\pgfpathrectangle{\pgfqpoint{0.600000in}{0.600000in}}{\pgfqpoint{3.900000in}{3.900000in}}%
\pgfusepath{clip}%
\pgfsetbuttcap%
\pgfsetroundjoin%
\definecolor{currentfill}{rgb}{0.121569,0.466667,0.705882}%
\pgfsetfillcolor{currentfill}%
\pgfsetlinewidth{1.003750pt}%
\definecolor{currentstroke}{rgb}{0.121569,0.466667,0.705882}%
\pgfsetstrokecolor{currentstroke}%
\pgfsetdash{}{0pt}%
\pgfpathmoveto{\pgfqpoint{2.664766in}{2.304047in}}%
\pgfpathcurveto{\pgfqpoint{2.675816in}{2.304047in}}{\pgfqpoint{2.686415in}{2.308438in}}{\pgfqpoint{2.694229in}{2.316251in}}%
\pgfpathcurveto{\pgfqpoint{2.702043in}{2.324065in}}{\pgfqpoint{2.706433in}{2.334664in}}{\pgfqpoint{2.706433in}{2.345714in}}%
\pgfpathcurveto{\pgfqpoint{2.706433in}{2.356764in}}{\pgfqpoint{2.702043in}{2.367363in}}{\pgfqpoint{2.694229in}{2.375177in}}%
\pgfpathcurveto{\pgfqpoint{2.686415in}{2.382990in}}{\pgfqpoint{2.675816in}{2.387381in}}{\pgfqpoint{2.664766in}{2.387381in}}%
\pgfpathcurveto{\pgfqpoint{2.653716in}{2.387381in}}{\pgfqpoint{2.643117in}{2.382990in}}{\pgfqpoint{2.635303in}{2.375177in}}%
\pgfpathcurveto{\pgfqpoint{2.627490in}{2.367363in}}{\pgfqpoint{2.623100in}{2.356764in}}{\pgfqpoint{2.623100in}{2.345714in}}%
\pgfpathcurveto{\pgfqpoint{2.623100in}{2.334664in}}{\pgfqpoint{2.627490in}{2.324065in}}{\pgfqpoint{2.635303in}{2.316251in}}%
\pgfpathcurveto{\pgfqpoint{2.643117in}{2.308438in}}{\pgfqpoint{2.653716in}{2.304047in}}{\pgfqpoint{2.664766in}{2.304047in}}%
\pgfpathclose%
\pgfusepath{stroke,fill}%
\end{pgfscope}%
\begin{pgfscope}%
\pgfpathrectangle{\pgfqpoint{0.600000in}{0.600000in}}{\pgfqpoint{3.900000in}{3.900000in}}%
\pgfusepath{clip}%
\pgfsetbuttcap%
\pgfsetroundjoin%
\definecolor{currentfill}{rgb}{0.121569,0.466667,0.705882}%
\pgfsetfillcolor{currentfill}%
\pgfsetlinewidth{1.003750pt}%
\definecolor{currentstroke}{rgb}{0.121569,0.466667,0.705882}%
\pgfsetstrokecolor{currentstroke}%
\pgfsetdash{}{0pt}%
\pgfpathmoveto{\pgfqpoint{2.283801in}{3.568836in}}%
\pgfpathcurveto{\pgfqpoint{2.294851in}{3.568836in}}{\pgfqpoint{2.305450in}{3.573226in}}{\pgfqpoint{2.313264in}{3.581040in}}%
\pgfpathcurveto{\pgfqpoint{2.321077in}{3.588854in}}{\pgfqpoint{2.325467in}{3.599453in}}{\pgfqpoint{2.325467in}{3.610503in}}%
\pgfpathcurveto{\pgfqpoint{2.325467in}{3.621553in}}{\pgfqpoint{2.321077in}{3.632152in}}{\pgfqpoint{2.313264in}{3.639966in}}%
\pgfpathcurveto{\pgfqpoint{2.305450in}{3.647779in}}{\pgfqpoint{2.294851in}{3.652169in}}{\pgfqpoint{2.283801in}{3.652169in}}%
\pgfpathcurveto{\pgfqpoint{2.272751in}{3.652169in}}{\pgfqpoint{2.262152in}{3.647779in}}{\pgfqpoint{2.254338in}{3.639966in}}%
\pgfpathcurveto{\pgfqpoint{2.246524in}{3.632152in}}{\pgfqpoint{2.242134in}{3.621553in}}{\pgfqpoint{2.242134in}{3.610503in}}%
\pgfpathcurveto{\pgfqpoint{2.242134in}{3.599453in}}{\pgfqpoint{2.246524in}{3.588854in}}{\pgfqpoint{2.254338in}{3.581040in}}%
\pgfpathcurveto{\pgfqpoint{2.262152in}{3.573226in}}{\pgfqpoint{2.272751in}{3.568836in}}{\pgfqpoint{2.283801in}{3.568836in}}%
\pgfpathclose%
\pgfusepath{stroke,fill}%
\end{pgfscope}%
\begin{pgfscope}%
\pgfpathrectangle{\pgfqpoint{0.600000in}{0.600000in}}{\pgfqpoint{3.900000in}{3.900000in}}%
\pgfusepath{clip}%
\pgfsetbuttcap%
\pgfsetroundjoin%
\definecolor{currentfill}{rgb}{0.121569,0.466667,0.705882}%
\pgfsetfillcolor{currentfill}%
\pgfsetlinewidth{1.003750pt}%
\definecolor{currentstroke}{rgb}{0.121569,0.466667,0.705882}%
\pgfsetstrokecolor{currentstroke}%
\pgfsetdash{}{0pt}%
\pgfpathmoveto{\pgfqpoint{3.120773in}{2.709411in}}%
\pgfpathcurveto{\pgfqpoint{3.131823in}{2.709411in}}{\pgfqpoint{3.142422in}{2.713801in}}{\pgfqpoint{3.150236in}{2.721615in}}%
\pgfpathcurveto{\pgfqpoint{3.158050in}{2.729429in}}{\pgfqpoint{3.162440in}{2.740028in}}{\pgfqpoint{3.162440in}{2.751078in}}%
\pgfpathcurveto{\pgfqpoint{3.162440in}{2.762128in}}{\pgfqpoint{3.158050in}{2.772727in}}{\pgfqpoint{3.150236in}{2.780541in}}%
\pgfpathcurveto{\pgfqpoint{3.142422in}{2.788354in}}{\pgfqpoint{3.131823in}{2.792745in}}{\pgfqpoint{3.120773in}{2.792745in}}%
\pgfpathcurveto{\pgfqpoint{3.109723in}{2.792745in}}{\pgfqpoint{3.099124in}{2.788354in}}{\pgfqpoint{3.091310in}{2.780541in}}%
\pgfpathcurveto{\pgfqpoint{3.083497in}{2.772727in}}{\pgfqpoint{3.079107in}{2.762128in}}{\pgfqpoint{3.079107in}{2.751078in}}%
\pgfpathcurveto{\pgfqpoint{3.079107in}{2.740028in}}{\pgfqpoint{3.083497in}{2.729429in}}{\pgfqpoint{3.091310in}{2.721615in}}%
\pgfpathcurveto{\pgfqpoint{3.099124in}{2.713801in}}{\pgfqpoint{3.109723in}{2.709411in}}{\pgfqpoint{3.120773in}{2.709411in}}%
\pgfpathclose%
\pgfusepath{stroke,fill}%
\end{pgfscope}%
\begin{pgfscope}%
\pgfpathrectangle{\pgfqpoint{0.600000in}{0.600000in}}{\pgfqpoint{3.900000in}{3.900000in}}%
\pgfusepath{clip}%
\pgfsetbuttcap%
\pgfsetroundjoin%
\definecolor{currentfill}{rgb}{0.121569,0.466667,0.705882}%
\pgfsetfillcolor{currentfill}%
\pgfsetlinewidth{1.003750pt}%
\definecolor{currentstroke}{rgb}{0.121569,0.466667,0.705882}%
\pgfsetstrokecolor{currentstroke}%
\pgfsetdash{}{0pt}%
\pgfpathmoveto{\pgfqpoint{2.450281in}{2.851984in}}%
\pgfpathcurveto{\pgfqpoint{2.461331in}{2.851984in}}{\pgfqpoint{2.471930in}{2.856374in}}{\pgfqpoint{2.479744in}{2.864188in}}%
\pgfpathcurveto{\pgfqpoint{2.487557in}{2.872001in}}{\pgfqpoint{2.491948in}{2.882600in}}{\pgfqpoint{2.491948in}{2.893650in}}%
\pgfpathcurveto{\pgfqpoint{2.491948in}{2.904701in}}{\pgfqpoint{2.487557in}{2.915300in}}{\pgfqpoint{2.479744in}{2.923113in}}%
\pgfpathcurveto{\pgfqpoint{2.471930in}{2.930927in}}{\pgfqpoint{2.461331in}{2.935317in}}{\pgfqpoint{2.450281in}{2.935317in}}%
\pgfpathcurveto{\pgfqpoint{2.439231in}{2.935317in}}{\pgfqpoint{2.428632in}{2.930927in}}{\pgfqpoint{2.420818in}{2.923113in}}%
\pgfpathcurveto{\pgfqpoint{2.413005in}{2.915300in}}{\pgfqpoint{2.408614in}{2.904701in}}{\pgfqpoint{2.408614in}{2.893650in}}%
\pgfpathcurveto{\pgfqpoint{2.408614in}{2.882600in}}{\pgfqpoint{2.413005in}{2.872001in}}{\pgfqpoint{2.420818in}{2.864188in}}%
\pgfpathcurveto{\pgfqpoint{2.428632in}{2.856374in}}{\pgfqpoint{2.439231in}{2.851984in}}{\pgfqpoint{2.450281in}{2.851984in}}%
\pgfpathclose%
\pgfusepath{stroke,fill}%
\end{pgfscope}%
\begin{pgfscope}%
\pgfpathrectangle{\pgfqpoint{0.600000in}{0.600000in}}{\pgfqpoint{3.900000in}{3.900000in}}%
\pgfusepath{clip}%
\pgfsetbuttcap%
\pgfsetroundjoin%
\definecolor{currentfill}{rgb}{0.121569,0.466667,0.705882}%
\pgfsetfillcolor{currentfill}%
\pgfsetlinewidth{1.003750pt}%
\definecolor{currentstroke}{rgb}{0.121569,0.466667,0.705882}%
\pgfsetstrokecolor{currentstroke}%
\pgfsetdash{}{0pt}%
\pgfpathmoveto{\pgfqpoint{2.667079in}{0.671671in}}%
\pgfpathcurveto{\pgfqpoint{2.678129in}{0.671671in}}{\pgfqpoint{2.688728in}{0.676062in}}{\pgfqpoint{2.696542in}{0.683875in}}%
\pgfpathcurveto{\pgfqpoint{2.704356in}{0.691689in}}{\pgfqpoint{2.708746in}{0.702288in}}{\pgfqpoint{2.708746in}{0.713338in}}%
\pgfpathcurveto{\pgfqpoint{2.708746in}{0.724388in}}{\pgfqpoint{2.704356in}{0.734987in}}{\pgfqpoint{2.696542in}{0.742801in}}%
\pgfpathcurveto{\pgfqpoint{2.688728in}{0.750614in}}{\pgfqpoint{2.678129in}{0.755005in}}{\pgfqpoint{2.667079in}{0.755005in}}%
\pgfpathcurveto{\pgfqpoint{2.656029in}{0.755005in}}{\pgfqpoint{2.645430in}{0.750614in}}{\pgfqpoint{2.637616in}{0.742801in}}%
\pgfpathcurveto{\pgfqpoint{2.629803in}{0.734987in}}{\pgfqpoint{2.625413in}{0.724388in}}{\pgfqpoint{2.625413in}{0.713338in}}%
\pgfpathcurveto{\pgfqpoint{2.625413in}{0.702288in}}{\pgfqpoint{2.629803in}{0.691689in}}{\pgfqpoint{2.637616in}{0.683875in}}%
\pgfpathcurveto{\pgfqpoint{2.645430in}{0.676062in}}{\pgfqpoint{2.656029in}{0.671671in}}{\pgfqpoint{2.667079in}{0.671671in}}%
\pgfpathclose%
\pgfusepath{stroke,fill}%
\end{pgfscope}%
\begin{pgfscope}%
\pgfpathrectangle{\pgfqpoint{0.600000in}{0.600000in}}{\pgfqpoint{3.900000in}{3.900000in}}%
\pgfusepath{clip}%
\pgfsetbuttcap%
\pgfsetroundjoin%
\definecolor{currentfill}{rgb}{0.121569,0.466667,0.705882}%
\pgfsetfillcolor{currentfill}%
\pgfsetlinewidth{1.003750pt}%
\definecolor{currentstroke}{rgb}{0.121569,0.466667,0.705882}%
\pgfsetstrokecolor{currentstroke}%
\pgfsetdash{}{0pt}%
\pgfpathmoveto{\pgfqpoint{2.591519in}{2.751651in}}%
\pgfpathcurveto{\pgfqpoint{2.602570in}{2.751651in}}{\pgfqpoint{2.613169in}{2.756041in}}{\pgfqpoint{2.620982in}{2.763855in}}%
\pgfpathcurveto{\pgfqpoint{2.628796in}{2.771669in}}{\pgfqpoint{2.633186in}{2.782268in}}{\pgfqpoint{2.633186in}{2.793318in}}%
\pgfpathcurveto{\pgfqpoint{2.633186in}{2.804368in}}{\pgfqpoint{2.628796in}{2.814967in}}{\pgfqpoint{2.620982in}{2.822781in}}%
\pgfpathcurveto{\pgfqpoint{2.613169in}{2.830594in}}{\pgfqpoint{2.602570in}{2.834984in}}{\pgfqpoint{2.591519in}{2.834984in}}%
\pgfpathcurveto{\pgfqpoint{2.580469in}{2.834984in}}{\pgfqpoint{2.569870in}{2.830594in}}{\pgfqpoint{2.562057in}{2.822781in}}%
\pgfpathcurveto{\pgfqpoint{2.554243in}{2.814967in}}{\pgfqpoint{2.549853in}{2.804368in}}{\pgfqpoint{2.549853in}{2.793318in}}%
\pgfpathcurveto{\pgfqpoint{2.549853in}{2.782268in}}{\pgfqpoint{2.554243in}{2.771669in}}{\pgfqpoint{2.562057in}{2.763855in}}%
\pgfpathcurveto{\pgfqpoint{2.569870in}{2.756041in}}{\pgfqpoint{2.580469in}{2.751651in}}{\pgfqpoint{2.591519in}{2.751651in}}%
\pgfpathclose%
\pgfusepath{stroke,fill}%
\end{pgfscope}%
\begin{pgfscope}%
\pgfpathrectangle{\pgfqpoint{0.600000in}{0.600000in}}{\pgfqpoint{3.900000in}{3.900000in}}%
\pgfusepath{clip}%
\pgfsetbuttcap%
\pgfsetroundjoin%
\definecolor{currentfill}{rgb}{0.121569,0.466667,0.705882}%
\pgfsetfillcolor{currentfill}%
\pgfsetlinewidth{1.003750pt}%
\definecolor{currentstroke}{rgb}{0.121569,0.466667,0.705882}%
\pgfsetstrokecolor{currentstroke}%
\pgfsetdash{}{0pt}%
\pgfpathmoveto{\pgfqpoint{2.656657in}{2.212173in}}%
\pgfpathcurveto{\pgfqpoint{2.667707in}{2.212173in}}{\pgfqpoint{2.678306in}{2.216564in}}{\pgfqpoint{2.686120in}{2.224377in}}%
\pgfpathcurveto{\pgfqpoint{2.693933in}{2.232191in}}{\pgfqpoint{2.698324in}{2.242790in}}{\pgfqpoint{2.698324in}{2.253840in}}%
\pgfpathcurveto{\pgfqpoint{2.698324in}{2.264890in}}{\pgfqpoint{2.693933in}{2.275489in}}{\pgfqpoint{2.686120in}{2.283303in}}%
\pgfpathcurveto{\pgfqpoint{2.678306in}{2.291116in}}{\pgfqpoint{2.667707in}{2.295507in}}{\pgfqpoint{2.656657in}{2.295507in}}%
\pgfpathcurveto{\pgfqpoint{2.645607in}{2.295507in}}{\pgfqpoint{2.635008in}{2.291116in}}{\pgfqpoint{2.627194in}{2.283303in}}%
\pgfpathcurveto{\pgfqpoint{2.619380in}{2.275489in}}{\pgfqpoint{2.614990in}{2.264890in}}{\pgfqpoint{2.614990in}{2.253840in}}%
\pgfpathcurveto{\pgfqpoint{2.614990in}{2.242790in}}{\pgfqpoint{2.619380in}{2.232191in}}{\pgfqpoint{2.627194in}{2.224377in}}%
\pgfpathcurveto{\pgfqpoint{2.635008in}{2.216564in}}{\pgfqpoint{2.645607in}{2.212173in}}{\pgfqpoint{2.656657in}{2.212173in}}%
\pgfpathclose%
\pgfusepath{stroke,fill}%
\end{pgfscope}%
\begin{pgfscope}%
\pgfpathrectangle{\pgfqpoint{0.600000in}{0.600000in}}{\pgfqpoint{3.900000in}{3.900000in}}%
\pgfusepath{clip}%
\pgfsetbuttcap%
\pgfsetroundjoin%
\definecolor{currentfill}{rgb}{0.121569,0.466667,0.705882}%
\pgfsetfillcolor{currentfill}%
\pgfsetlinewidth{1.003750pt}%
\definecolor{currentstroke}{rgb}{0.121569,0.466667,0.705882}%
\pgfsetstrokecolor{currentstroke}%
\pgfsetdash{}{0pt}%
\pgfpathmoveto{\pgfqpoint{2.827870in}{3.483559in}}%
\pgfpathcurveto{\pgfqpoint{2.838920in}{3.483559in}}{\pgfqpoint{2.849519in}{3.487949in}}{\pgfqpoint{2.857333in}{3.495763in}}%
\pgfpathcurveto{\pgfqpoint{2.865146in}{3.503576in}}{\pgfqpoint{2.869537in}{3.514176in}}{\pgfqpoint{2.869537in}{3.525226in}}%
\pgfpathcurveto{\pgfqpoint{2.869537in}{3.536276in}}{\pgfqpoint{2.865146in}{3.546875in}}{\pgfqpoint{2.857333in}{3.554688in}}%
\pgfpathcurveto{\pgfqpoint{2.849519in}{3.562502in}}{\pgfqpoint{2.838920in}{3.566892in}}{\pgfqpoint{2.827870in}{3.566892in}}%
\pgfpathcurveto{\pgfqpoint{2.816820in}{3.566892in}}{\pgfqpoint{2.806221in}{3.562502in}}{\pgfqpoint{2.798407in}{3.554688in}}%
\pgfpathcurveto{\pgfqpoint{2.790593in}{3.546875in}}{\pgfqpoint{2.786203in}{3.536276in}}{\pgfqpoint{2.786203in}{3.525226in}}%
\pgfpathcurveto{\pgfqpoint{2.786203in}{3.514176in}}{\pgfqpoint{2.790593in}{3.503576in}}{\pgfqpoint{2.798407in}{3.495763in}}%
\pgfpathcurveto{\pgfqpoint{2.806221in}{3.487949in}}{\pgfqpoint{2.816820in}{3.483559in}}{\pgfqpoint{2.827870in}{3.483559in}}%
\pgfpathclose%
\pgfusepath{stroke,fill}%
\end{pgfscope}%
\begin{pgfscope}%
\pgfpathrectangle{\pgfqpoint{0.600000in}{0.600000in}}{\pgfqpoint{3.900000in}{3.900000in}}%
\pgfusepath{clip}%
\pgfsetbuttcap%
\pgfsetroundjoin%
\definecolor{currentfill}{rgb}{0.121569,0.466667,0.705882}%
\pgfsetfillcolor{currentfill}%
\pgfsetlinewidth{1.003750pt}%
\definecolor{currentstroke}{rgb}{0.121569,0.466667,0.705882}%
\pgfsetstrokecolor{currentstroke}%
\pgfsetdash{}{0pt}%
\pgfpathmoveto{\pgfqpoint{2.308699in}{2.218251in}}%
\pgfpathcurveto{\pgfqpoint{2.319749in}{2.218251in}}{\pgfqpoint{2.330348in}{2.222641in}}{\pgfqpoint{2.338161in}{2.230455in}}%
\pgfpathcurveto{\pgfqpoint{2.345975in}{2.238269in}}{\pgfqpoint{2.350365in}{2.248868in}}{\pgfqpoint{2.350365in}{2.259918in}}%
\pgfpathcurveto{\pgfqpoint{2.350365in}{2.270968in}}{\pgfqpoint{2.345975in}{2.281567in}}{\pgfqpoint{2.338161in}{2.289381in}}%
\pgfpathcurveto{\pgfqpoint{2.330348in}{2.297194in}}{\pgfqpoint{2.319749in}{2.301585in}}{\pgfqpoint{2.308699in}{2.301585in}}%
\pgfpathcurveto{\pgfqpoint{2.297649in}{2.301585in}}{\pgfqpoint{2.287049in}{2.297194in}}{\pgfqpoint{2.279236in}{2.289381in}}%
\pgfpathcurveto{\pgfqpoint{2.271422in}{2.281567in}}{\pgfqpoint{2.267032in}{2.270968in}}{\pgfqpoint{2.267032in}{2.259918in}}%
\pgfpathcurveto{\pgfqpoint{2.267032in}{2.248868in}}{\pgfqpoint{2.271422in}{2.238269in}}{\pgfqpoint{2.279236in}{2.230455in}}%
\pgfpathcurveto{\pgfqpoint{2.287049in}{2.222641in}}{\pgfqpoint{2.297649in}{2.218251in}}{\pgfqpoint{2.308699in}{2.218251in}}%
\pgfpathclose%
\pgfusepath{stroke,fill}%
\end{pgfscope}%
\begin{pgfscope}%
\pgfpathrectangle{\pgfqpoint{0.600000in}{0.600000in}}{\pgfqpoint{3.900000in}{3.900000in}}%
\pgfusepath{clip}%
\pgfsetbuttcap%
\pgfsetroundjoin%
\definecolor{currentfill}{rgb}{0.121569,0.466667,0.705882}%
\pgfsetfillcolor{currentfill}%
\pgfsetlinewidth{1.003750pt}%
\definecolor{currentstroke}{rgb}{0.121569,0.466667,0.705882}%
\pgfsetstrokecolor{currentstroke}%
\pgfsetdash{}{0pt}%
\pgfpathmoveto{\pgfqpoint{2.916114in}{2.542433in}}%
\pgfpathcurveto{\pgfqpoint{2.927164in}{2.542433in}}{\pgfqpoint{2.937763in}{2.546823in}}{\pgfqpoint{2.945576in}{2.554637in}}%
\pgfpathcurveto{\pgfqpoint{2.953390in}{2.562451in}}{\pgfqpoint{2.957780in}{2.573050in}}{\pgfqpoint{2.957780in}{2.584100in}}%
\pgfpathcurveto{\pgfqpoint{2.957780in}{2.595150in}}{\pgfqpoint{2.953390in}{2.605749in}}{\pgfqpoint{2.945576in}{2.613563in}}%
\pgfpathcurveto{\pgfqpoint{2.937763in}{2.621376in}}{\pgfqpoint{2.927164in}{2.625766in}}{\pgfqpoint{2.916114in}{2.625766in}}%
\pgfpathcurveto{\pgfqpoint{2.905063in}{2.625766in}}{\pgfqpoint{2.894464in}{2.621376in}}{\pgfqpoint{2.886651in}{2.613563in}}%
\pgfpathcurveto{\pgfqpoint{2.878837in}{2.605749in}}{\pgfqpoint{2.874447in}{2.595150in}}{\pgfqpoint{2.874447in}{2.584100in}}%
\pgfpathcurveto{\pgfqpoint{2.874447in}{2.573050in}}{\pgfqpoint{2.878837in}{2.562451in}}{\pgfqpoint{2.886651in}{2.554637in}}%
\pgfpathcurveto{\pgfqpoint{2.894464in}{2.546823in}}{\pgfqpoint{2.905063in}{2.542433in}}{\pgfqpoint{2.916114in}{2.542433in}}%
\pgfpathclose%
\pgfusepath{stroke,fill}%
\end{pgfscope}%
\begin{pgfscope}%
\pgfpathrectangle{\pgfqpoint{0.600000in}{0.600000in}}{\pgfqpoint{3.900000in}{3.900000in}}%
\pgfusepath{clip}%
\pgfsetbuttcap%
\pgfsetroundjoin%
\definecolor{currentfill}{rgb}{0.121569,0.466667,0.705882}%
\pgfsetfillcolor{currentfill}%
\pgfsetlinewidth{1.003750pt}%
\definecolor{currentstroke}{rgb}{0.121569,0.466667,0.705882}%
\pgfsetstrokecolor{currentstroke}%
\pgfsetdash{}{0pt}%
\pgfpathmoveto{\pgfqpoint{1.931559in}{2.553345in}}%
\pgfpathcurveto{\pgfqpoint{1.942609in}{2.553345in}}{\pgfqpoint{1.953208in}{2.557735in}}{\pgfqpoint{1.961022in}{2.565549in}}%
\pgfpathcurveto{\pgfqpoint{1.968835in}{2.573362in}}{\pgfqpoint{1.973226in}{2.583961in}}{\pgfqpoint{1.973226in}{2.595011in}}%
\pgfpathcurveto{\pgfqpoint{1.973226in}{2.606062in}}{\pgfqpoint{1.968835in}{2.616661in}}{\pgfqpoint{1.961022in}{2.624474in}}%
\pgfpathcurveto{\pgfqpoint{1.953208in}{2.632288in}}{\pgfqpoint{1.942609in}{2.636678in}}{\pgfqpoint{1.931559in}{2.636678in}}%
\pgfpathcurveto{\pgfqpoint{1.920509in}{2.636678in}}{\pgfqpoint{1.909910in}{2.632288in}}{\pgfqpoint{1.902096in}{2.624474in}}%
\pgfpathcurveto{\pgfqpoint{1.894283in}{2.616661in}}{\pgfqpoint{1.889892in}{2.606062in}}{\pgfqpoint{1.889892in}{2.595011in}}%
\pgfpathcurveto{\pgfqpoint{1.889892in}{2.583961in}}{\pgfqpoint{1.894283in}{2.573362in}}{\pgfqpoint{1.902096in}{2.565549in}}%
\pgfpathcurveto{\pgfqpoint{1.909910in}{2.557735in}}{\pgfqpoint{1.920509in}{2.553345in}}{\pgfqpoint{1.931559in}{2.553345in}}%
\pgfpathclose%
\pgfusepath{stroke,fill}%
\end{pgfscope}%
\begin{pgfscope}%
\pgfpathrectangle{\pgfqpoint{0.600000in}{0.600000in}}{\pgfqpoint{3.900000in}{3.900000in}}%
\pgfusepath{clip}%
\pgfsetbuttcap%
\pgfsetroundjoin%
\definecolor{currentfill}{rgb}{0.121569,0.466667,0.705882}%
\pgfsetfillcolor{currentfill}%
\pgfsetlinewidth{1.003750pt}%
\definecolor{currentstroke}{rgb}{0.121569,0.466667,0.705882}%
\pgfsetstrokecolor{currentstroke}%
\pgfsetdash{}{0pt}%
\pgfpathmoveto{\pgfqpoint{3.109538in}{2.637611in}}%
\pgfpathcurveto{\pgfqpoint{3.120588in}{2.637611in}}{\pgfqpoint{3.131187in}{2.642002in}}{\pgfqpoint{3.139001in}{2.649815in}}%
\pgfpathcurveto{\pgfqpoint{3.146815in}{2.657629in}}{\pgfqpoint{3.151205in}{2.668228in}}{\pgfqpoint{3.151205in}{2.679278in}}%
\pgfpathcurveto{\pgfqpoint{3.151205in}{2.690328in}}{\pgfqpoint{3.146815in}{2.700927in}}{\pgfqpoint{3.139001in}{2.708741in}}%
\pgfpathcurveto{\pgfqpoint{3.131187in}{2.716554in}}{\pgfqpoint{3.120588in}{2.720945in}}{\pgfqpoint{3.109538in}{2.720945in}}%
\pgfpathcurveto{\pgfqpoint{3.098488in}{2.720945in}}{\pgfqpoint{3.087889in}{2.716554in}}{\pgfqpoint{3.080075in}{2.708741in}}%
\pgfpathcurveto{\pgfqpoint{3.072262in}{2.700927in}}{\pgfqpoint{3.067872in}{2.690328in}}{\pgfqpoint{3.067872in}{2.679278in}}%
\pgfpathcurveto{\pgfqpoint{3.067872in}{2.668228in}}{\pgfqpoint{3.072262in}{2.657629in}}{\pgfqpoint{3.080075in}{2.649815in}}%
\pgfpathcurveto{\pgfqpoint{3.087889in}{2.642002in}}{\pgfqpoint{3.098488in}{2.637611in}}{\pgfqpoint{3.109538in}{2.637611in}}%
\pgfpathclose%
\pgfusepath{stroke,fill}%
\end{pgfscope}%
\begin{pgfscope}%
\pgfpathrectangle{\pgfqpoint{0.600000in}{0.600000in}}{\pgfqpoint{3.900000in}{3.900000in}}%
\pgfusepath{clip}%
\pgfsetbuttcap%
\pgfsetroundjoin%
\definecolor{currentfill}{rgb}{0.121569,0.466667,0.705882}%
\pgfsetfillcolor{currentfill}%
\pgfsetlinewidth{1.003750pt}%
\definecolor{currentstroke}{rgb}{0.121569,0.466667,0.705882}%
\pgfsetstrokecolor{currentstroke}%
\pgfsetdash{}{0pt}%
\pgfpathmoveto{\pgfqpoint{2.546956in}{2.528887in}}%
\pgfpathcurveto{\pgfqpoint{2.558006in}{2.528887in}}{\pgfqpoint{2.568605in}{2.533278in}}{\pgfqpoint{2.576419in}{2.541091in}}%
\pgfpathcurveto{\pgfqpoint{2.584233in}{2.548905in}}{\pgfqpoint{2.588623in}{2.559504in}}{\pgfqpoint{2.588623in}{2.570554in}}%
\pgfpathcurveto{\pgfqpoint{2.588623in}{2.581604in}}{\pgfqpoint{2.584233in}{2.592203in}}{\pgfqpoint{2.576419in}{2.600017in}}%
\pgfpathcurveto{\pgfqpoint{2.568605in}{2.607830in}}{\pgfqpoint{2.558006in}{2.612221in}}{\pgfqpoint{2.546956in}{2.612221in}}%
\pgfpathcurveto{\pgfqpoint{2.535906in}{2.612221in}}{\pgfqpoint{2.525307in}{2.607830in}}{\pgfqpoint{2.517493in}{2.600017in}}%
\pgfpathcurveto{\pgfqpoint{2.509680in}{2.592203in}}{\pgfqpoint{2.505290in}{2.581604in}}{\pgfqpoint{2.505290in}{2.570554in}}%
\pgfpathcurveto{\pgfqpoint{2.505290in}{2.559504in}}{\pgfqpoint{2.509680in}{2.548905in}}{\pgfqpoint{2.517493in}{2.541091in}}%
\pgfpathcurveto{\pgfqpoint{2.525307in}{2.533278in}}{\pgfqpoint{2.535906in}{2.528887in}}{\pgfqpoint{2.546956in}{2.528887in}}%
\pgfpathclose%
\pgfusepath{stroke,fill}%
\end{pgfscope}%
\begin{pgfscope}%
\pgfpathrectangle{\pgfqpoint{0.600000in}{0.600000in}}{\pgfqpoint{3.900000in}{3.900000in}}%
\pgfusepath{clip}%
\pgfsetbuttcap%
\pgfsetroundjoin%
\definecolor{currentfill}{rgb}{0.121569,0.466667,0.705882}%
\pgfsetfillcolor{currentfill}%
\pgfsetlinewidth{1.003750pt}%
\definecolor{currentstroke}{rgb}{0.121569,0.466667,0.705882}%
\pgfsetstrokecolor{currentstroke}%
\pgfsetdash{}{0pt}%
\pgfpathmoveto{\pgfqpoint{2.087499in}{1.463814in}}%
\pgfpathcurveto{\pgfqpoint{2.098549in}{1.463814in}}{\pgfqpoint{2.109148in}{1.468205in}}{\pgfqpoint{2.116961in}{1.476018in}}%
\pgfpathcurveto{\pgfqpoint{2.124775in}{1.483832in}}{\pgfqpoint{2.129165in}{1.494431in}}{\pgfqpoint{2.129165in}{1.505481in}}%
\pgfpathcurveto{\pgfqpoint{2.129165in}{1.516531in}}{\pgfqpoint{2.124775in}{1.527130in}}{\pgfqpoint{2.116961in}{1.534944in}}%
\pgfpathcurveto{\pgfqpoint{2.109148in}{1.542757in}}{\pgfqpoint{2.098549in}{1.547148in}}{\pgfqpoint{2.087499in}{1.547148in}}%
\pgfpathcurveto{\pgfqpoint{2.076448in}{1.547148in}}{\pgfqpoint{2.065849in}{1.542757in}}{\pgfqpoint{2.058036in}{1.534944in}}%
\pgfpathcurveto{\pgfqpoint{2.050222in}{1.527130in}}{\pgfqpoint{2.045832in}{1.516531in}}{\pgfqpoint{2.045832in}{1.505481in}}%
\pgfpathcurveto{\pgfqpoint{2.045832in}{1.494431in}}{\pgfqpoint{2.050222in}{1.483832in}}{\pgfqpoint{2.058036in}{1.476018in}}%
\pgfpathcurveto{\pgfqpoint{2.065849in}{1.468205in}}{\pgfqpoint{2.076448in}{1.463814in}}{\pgfqpoint{2.087499in}{1.463814in}}%
\pgfpathclose%
\pgfusepath{stroke,fill}%
\end{pgfscope}%
\begin{pgfscope}%
\pgfpathrectangle{\pgfqpoint{0.600000in}{0.600000in}}{\pgfqpoint{3.900000in}{3.900000in}}%
\pgfusepath{clip}%
\pgfsetbuttcap%
\pgfsetroundjoin%
\definecolor{currentfill}{rgb}{0.121569,0.466667,0.705882}%
\pgfsetfillcolor{currentfill}%
\pgfsetlinewidth{1.003750pt}%
\definecolor{currentstroke}{rgb}{0.121569,0.466667,0.705882}%
\pgfsetstrokecolor{currentstroke}%
\pgfsetdash{}{0pt}%
\pgfpathmoveto{\pgfqpoint{3.220696in}{2.737641in}}%
\pgfpathcurveto{\pgfqpoint{3.231746in}{2.737641in}}{\pgfqpoint{3.242346in}{2.742032in}}{\pgfqpoint{3.250159in}{2.749845in}}%
\pgfpathcurveto{\pgfqpoint{3.257973in}{2.757659in}}{\pgfqpoint{3.262363in}{2.768258in}}{\pgfqpoint{3.262363in}{2.779308in}}%
\pgfpathcurveto{\pgfqpoint{3.262363in}{2.790358in}}{\pgfqpoint{3.257973in}{2.800957in}}{\pgfqpoint{3.250159in}{2.808771in}}%
\pgfpathcurveto{\pgfqpoint{3.242346in}{2.816584in}}{\pgfqpoint{3.231746in}{2.820975in}}{\pgfqpoint{3.220696in}{2.820975in}}%
\pgfpathcurveto{\pgfqpoint{3.209646in}{2.820975in}}{\pgfqpoint{3.199047in}{2.816584in}}{\pgfqpoint{3.191234in}{2.808771in}}%
\pgfpathcurveto{\pgfqpoint{3.183420in}{2.800957in}}{\pgfqpoint{3.179030in}{2.790358in}}{\pgfqpoint{3.179030in}{2.779308in}}%
\pgfpathcurveto{\pgfqpoint{3.179030in}{2.768258in}}{\pgfqpoint{3.183420in}{2.757659in}}{\pgfqpoint{3.191234in}{2.749845in}}%
\pgfpathcurveto{\pgfqpoint{3.199047in}{2.742032in}}{\pgfqpoint{3.209646in}{2.737641in}}{\pgfqpoint{3.220696in}{2.737641in}}%
\pgfpathclose%
\pgfusepath{stroke,fill}%
\end{pgfscope}%
\begin{pgfscope}%
\pgfpathrectangle{\pgfqpoint{0.600000in}{0.600000in}}{\pgfqpoint{3.900000in}{3.900000in}}%
\pgfusepath{clip}%
\pgfsetbuttcap%
\pgfsetroundjoin%
\definecolor{currentfill}{rgb}{0.121569,0.466667,0.705882}%
\pgfsetfillcolor{currentfill}%
\pgfsetlinewidth{1.003750pt}%
\definecolor{currentstroke}{rgb}{0.121569,0.466667,0.705882}%
\pgfsetstrokecolor{currentstroke}%
\pgfsetdash{}{0pt}%
\pgfpathmoveto{\pgfqpoint{2.338856in}{2.360852in}}%
\pgfpathcurveto{\pgfqpoint{2.349906in}{2.360852in}}{\pgfqpoint{2.360505in}{2.365242in}}{\pgfqpoint{2.368319in}{2.373056in}}%
\pgfpathcurveto{\pgfqpoint{2.376132in}{2.380869in}}{\pgfqpoint{2.380523in}{2.391468in}}{\pgfqpoint{2.380523in}{2.402519in}}%
\pgfpathcurveto{\pgfqpoint{2.380523in}{2.413569in}}{\pgfqpoint{2.376132in}{2.424168in}}{\pgfqpoint{2.368319in}{2.431981in}}%
\pgfpathcurveto{\pgfqpoint{2.360505in}{2.439795in}}{\pgfqpoint{2.349906in}{2.444185in}}{\pgfqpoint{2.338856in}{2.444185in}}%
\pgfpathcurveto{\pgfqpoint{2.327806in}{2.444185in}}{\pgfqpoint{2.317207in}{2.439795in}}{\pgfqpoint{2.309393in}{2.431981in}}%
\pgfpathcurveto{\pgfqpoint{2.301579in}{2.424168in}}{\pgfqpoint{2.297189in}{2.413569in}}{\pgfqpoint{2.297189in}{2.402519in}}%
\pgfpathcurveto{\pgfqpoint{2.297189in}{2.391468in}}{\pgfqpoint{2.301579in}{2.380869in}}{\pgfqpoint{2.309393in}{2.373056in}}%
\pgfpathcurveto{\pgfqpoint{2.317207in}{2.365242in}}{\pgfqpoint{2.327806in}{2.360852in}}{\pgfqpoint{2.338856in}{2.360852in}}%
\pgfpathclose%
\pgfusepath{stroke,fill}%
\end{pgfscope}%
\begin{pgfscope}%
\pgfpathrectangle{\pgfqpoint{0.600000in}{0.600000in}}{\pgfqpoint{3.900000in}{3.900000in}}%
\pgfusepath{clip}%
\pgfsetbuttcap%
\pgfsetroundjoin%
\definecolor{currentfill}{rgb}{0.121569,0.466667,0.705882}%
\pgfsetfillcolor{currentfill}%
\pgfsetlinewidth{1.003750pt}%
\definecolor{currentstroke}{rgb}{0.121569,0.466667,0.705882}%
\pgfsetstrokecolor{currentstroke}%
\pgfsetdash{}{0pt}%
\pgfpathmoveto{\pgfqpoint{2.167944in}{1.839339in}}%
\pgfpathcurveto{\pgfqpoint{2.178994in}{1.839339in}}{\pgfqpoint{2.189593in}{1.843729in}}{\pgfqpoint{2.197407in}{1.851543in}}%
\pgfpathcurveto{\pgfqpoint{2.205220in}{1.859357in}}{\pgfqpoint{2.209611in}{1.869956in}}{\pgfqpoint{2.209611in}{1.881006in}}%
\pgfpathcurveto{\pgfqpoint{2.209611in}{1.892056in}}{\pgfqpoint{2.205220in}{1.902655in}}{\pgfqpoint{2.197407in}{1.910469in}}%
\pgfpathcurveto{\pgfqpoint{2.189593in}{1.918282in}}{\pgfqpoint{2.178994in}{1.922672in}}{\pgfqpoint{2.167944in}{1.922672in}}%
\pgfpathcurveto{\pgfqpoint{2.156894in}{1.922672in}}{\pgfqpoint{2.146295in}{1.918282in}}{\pgfqpoint{2.138481in}{1.910469in}}%
\pgfpathcurveto{\pgfqpoint{2.130667in}{1.902655in}}{\pgfqpoint{2.126277in}{1.892056in}}{\pgfqpoint{2.126277in}{1.881006in}}%
\pgfpathcurveto{\pgfqpoint{2.126277in}{1.869956in}}{\pgfqpoint{2.130667in}{1.859357in}}{\pgfqpoint{2.138481in}{1.851543in}}%
\pgfpathcurveto{\pgfqpoint{2.146295in}{1.843729in}}{\pgfqpoint{2.156894in}{1.839339in}}{\pgfqpoint{2.167944in}{1.839339in}}%
\pgfpathclose%
\pgfusepath{stroke,fill}%
\end{pgfscope}%
\begin{pgfscope}%
\pgfpathrectangle{\pgfqpoint{0.600000in}{0.600000in}}{\pgfqpoint{3.900000in}{3.900000in}}%
\pgfusepath{clip}%
\pgfsetbuttcap%
\pgfsetroundjoin%
\definecolor{currentfill}{rgb}{0.121569,0.466667,0.705882}%
\pgfsetfillcolor{currentfill}%
\pgfsetlinewidth{1.003750pt}%
\definecolor{currentstroke}{rgb}{0.121569,0.466667,0.705882}%
\pgfsetstrokecolor{currentstroke}%
\pgfsetdash{}{0pt}%
\pgfpathmoveto{\pgfqpoint{2.537669in}{3.214333in}}%
\pgfpathcurveto{\pgfqpoint{2.548719in}{3.214333in}}{\pgfqpoint{2.559318in}{3.218723in}}{\pgfqpoint{2.567132in}{3.226537in}}%
\pgfpathcurveto{\pgfqpoint{2.574945in}{3.234351in}}{\pgfqpoint{2.579335in}{3.244950in}}{\pgfqpoint{2.579335in}{3.256000in}}%
\pgfpathcurveto{\pgfqpoint{2.579335in}{3.267050in}}{\pgfqpoint{2.574945in}{3.277649in}}{\pgfqpoint{2.567132in}{3.285462in}}%
\pgfpathcurveto{\pgfqpoint{2.559318in}{3.293276in}}{\pgfqpoint{2.548719in}{3.297666in}}{\pgfqpoint{2.537669in}{3.297666in}}%
\pgfpathcurveto{\pgfqpoint{2.526619in}{3.297666in}}{\pgfqpoint{2.516020in}{3.293276in}}{\pgfqpoint{2.508206in}{3.285462in}}%
\pgfpathcurveto{\pgfqpoint{2.500392in}{3.277649in}}{\pgfqpoint{2.496002in}{3.267050in}}{\pgfqpoint{2.496002in}{3.256000in}}%
\pgfpathcurveto{\pgfqpoint{2.496002in}{3.244950in}}{\pgfqpoint{2.500392in}{3.234351in}}{\pgfqpoint{2.508206in}{3.226537in}}%
\pgfpathcurveto{\pgfqpoint{2.516020in}{3.218723in}}{\pgfqpoint{2.526619in}{3.214333in}}{\pgfqpoint{2.537669in}{3.214333in}}%
\pgfpathclose%
\pgfusepath{stroke,fill}%
\end{pgfscope}%
\begin{pgfscope}%
\pgfpathrectangle{\pgfqpoint{0.600000in}{0.600000in}}{\pgfqpoint{3.900000in}{3.900000in}}%
\pgfusepath{clip}%
\pgfsetbuttcap%
\pgfsetroundjoin%
\definecolor{currentfill}{rgb}{0.121569,0.466667,0.705882}%
\pgfsetfillcolor{currentfill}%
\pgfsetlinewidth{1.003750pt}%
\definecolor{currentstroke}{rgb}{0.121569,0.466667,0.705882}%
\pgfsetstrokecolor{currentstroke}%
\pgfsetdash{}{0pt}%
\pgfpathmoveto{\pgfqpoint{3.378704in}{1.961536in}}%
\pgfpathcurveto{\pgfqpoint{3.389754in}{1.961536in}}{\pgfqpoint{3.400353in}{1.965926in}}{\pgfqpoint{3.408166in}{1.973740in}}%
\pgfpathcurveto{\pgfqpoint{3.415980in}{1.981554in}}{\pgfqpoint{3.420370in}{1.992153in}}{\pgfqpoint{3.420370in}{2.003203in}}%
\pgfpathcurveto{\pgfqpoint{3.420370in}{2.014253in}}{\pgfqpoint{3.415980in}{2.024852in}}{\pgfqpoint{3.408166in}{2.032666in}}%
\pgfpathcurveto{\pgfqpoint{3.400353in}{2.040479in}}{\pgfqpoint{3.389754in}{2.044870in}}{\pgfqpoint{3.378704in}{2.044870in}}%
\pgfpathcurveto{\pgfqpoint{3.367654in}{2.044870in}}{\pgfqpoint{3.357055in}{2.040479in}}{\pgfqpoint{3.349241in}{2.032666in}}%
\pgfpathcurveto{\pgfqpoint{3.341427in}{2.024852in}}{\pgfqpoint{3.337037in}{2.014253in}}{\pgfqpoint{3.337037in}{2.003203in}}%
\pgfpathcurveto{\pgfqpoint{3.337037in}{1.992153in}}{\pgfqpoint{3.341427in}{1.981554in}}{\pgfqpoint{3.349241in}{1.973740in}}%
\pgfpathcurveto{\pgfqpoint{3.357055in}{1.965926in}}{\pgfqpoint{3.367654in}{1.961536in}}{\pgfqpoint{3.378704in}{1.961536in}}%
\pgfpathclose%
\pgfusepath{stroke,fill}%
\end{pgfscope}%
\begin{pgfscope}%
\pgfpathrectangle{\pgfqpoint{0.600000in}{0.600000in}}{\pgfqpoint{3.900000in}{3.900000in}}%
\pgfusepath{clip}%
\pgfsetbuttcap%
\pgfsetroundjoin%
\definecolor{currentfill}{rgb}{0.121569,0.466667,0.705882}%
\pgfsetfillcolor{currentfill}%
\pgfsetlinewidth{1.003750pt}%
\definecolor{currentstroke}{rgb}{0.121569,0.466667,0.705882}%
\pgfsetstrokecolor{currentstroke}%
\pgfsetdash{}{0pt}%
\pgfpathmoveto{\pgfqpoint{2.554651in}{2.038825in}}%
\pgfpathcurveto{\pgfqpoint{2.565702in}{2.038825in}}{\pgfqpoint{2.576301in}{2.043215in}}{\pgfqpoint{2.584114in}{2.051029in}}%
\pgfpathcurveto{\pgfqpoint{2.591928in}{2.058842in}}{\pgfqpoint{2.596318in}{2.069441in}}{\pgfqpoint{2.596318in}{2.080491in}}%
\pgfpathcurveto{\pgfqpoint{2.596318in}{2.091542in}}{\pgfqpoint{2.591928in}{2.102141in}}{\pgfqpoint{2.584114in}{2.109954in}}%
\pgfpathcurveto{\pgfqpoint{2.576301in}{2.117768in}}{\pgfqpoint{2.565702in}{2.122158in}}{\pgfqpoint{2.554651in}{2.122158in}}%
\pgfpathcurveto{\pgfqpoint{2.543601in}{2.122158in}}{\pgfqpoint{2.533002in}{2.117768in}}{\pgfqpoint{2.525189in}{2.109954in}}%
\pgfpathcurveto{\pgfqpoint{2.517375in}{2.102141in}}{\pgfqpoint{2.512985in}{2.091542in}}{\pgfqpoint{2.512985in}{2.080491in}}%
\pgfpathcurveto{\pgfqpoint{2.512985in}{2.069441in}}{\pgfqpoint{2.517375in}{2.058842in}}{\pgfqpoint{2.525189in}{2.051029in}}%
\pgfpathcurveto{\pgfqpoint{2.533002in}{2.043215in}}{\pgfqpoint{2.543601in}{2.038825in}}{\pgfqpoint{2.554651in}{2.038825in}}%
\pgfpathclose%
\pgfusepath{stroke,fill}%
\end{pgfscope}%
\begin{pgfscope}%
\pgfpathrectangle{\pgfqpoint{0.600000in}{0.600000in}}{\pgfqpoint{3.900000in}{3.900000in}}%
\pgfusepath{clip}%
\pgfsetbuttcap%
\pgfsetroundjoin%
\definecolor{currentfill}{rgb}{0.121569,0.466667,0.705882}%
\pgfsetfillcolor{currentfill}%
\pgfsetlinewidth{1.003750pt}%
\definecolor{currentstroke}{rgb}{0.121569,0.466667,0.705882}%
\pgfsetstrokecolor{currentstroke}%
\pgfsetdash{}{0pt}%
\pgfpathmoveto{\pgfqpoint{2.363929in}{1.986773in}}%
\pgfpathcurveto{\pgfqpoint{2.374979in}{1.986773in}}{\pgfqpoint{2.385578in}{1.991163in}}{\pgfqpoint{2.393392in}{1.998977in}}%
\pgfpathcurveto{\pgfqpoint{2.401206in}{2.006790in}}{\pgfqpoint{2.405596in}{2.017390in}}{\pgfqpoint{2.405596in}{2.028440in}}%
\pgfpathcurveto{\pgfqpoint{2.405596in}{2.039490in}}{\pgfqpoint{2.401206in}{2.050089in}}{\pgfqpoint{2.393392in}{2.057902in}}%
\pgfpathcurveto{\pgfqpoint{2.385578in}{2.065716in}}{\pgfqpoint{2.374979in}{2.070106in}}{\pgfqpoint{2.363929in}{2.070106in}}%
\pgfpathcurveto{\pgfqpoint{2.352879in}{2.070106in}}{\pgfqpoint{2.342280in}{2.065716in}}{\pgfqpoint{2.334466in}{2.057902in}}%
\pgfpathcurveto{\pgfqpoint{2.326653in}{2.050089in}}{\pgfqpoint{2.322263in}{2.039490in}}{\pgfqpoint{2.322263in}{2.028440in}}%
\pgfpathcurveto{\pgfqpoint{2.322263in}{2.017390in}}{\pgfqpoint{2.326653in}{2.006790in}}{\pgfqpoint{2.334466in}{1.998977in}}%
\pgfpathcurveto{\pgfqpoint{2.342280in}{1.991163in}}{\pgfqpoint{2.352879in}{1.986773in}}{\pgfqpoint{2.363929in}{1.986773in}}%
\pgfpathclose%
\pgfusepath{stroke,fill}%
\end{pgfscope}%
\begin{pgfscope}%
\pgfpathrectangle{\pgfqpoint{0.600000in}{0.600000in}}{\pgfqpoint{3.900000in}{3.900000in}}%
\pgfusepath{clip}%
\pgfsetbuttcap%
\pgfsetroundjoin%
\definecolor{currentfill}{rgb}{0.121569,0.466667,0.705882}%
\pgfsetfillcolor{currentfill}%
\pgfsetlinewidth{1.003750pt}%
\definecolor{currentstroke}{rgb}{0.121569,0.466667,0.705882}%
\pgfsetstrokecolor{currentstroke}%
\pgfsetdash{}{0pt}%
\pgfpathmoveto{\pgfqpoint{2.195701in}{2.557940in}}%
\pgfpathcurveto{\pgfqpoint{2.206751in}{2.557940in}}{\pgfqpoint{2.217350in}{2.562330in}}{\pgfqpoint{2.225164in}{2.570144in}}%
\pgfpathcurveto{\pgfqpoint{2.232978in}{2.577957in}}{\pgfqpoint{2.237368in}{2.588556in}}{\pgfqpoint{2.237368in}{2.599606in}}%
\pgfpathcurveto{\pgfqpoint{2.237368in}{2.610656in}}{\pgfqpoint{2.232978in}{2.621256in}}{\pgfqpoint{2.225164in}{2.629069in}}%
\pgfpathcurveto{\pgfqpoint{2.217350in}{2.636883in}}{\pgfqpoint{2.206751in}{2.641273in}}{\pgfqpoint{2.195701in}{2.641273in}}%
\pgfpathcurveto{\pgfqpoint{2.184651in}{2.641273in}}{\pgfqpoint{2.174052in}{2.636883in}}{\pgfqpoint{2.166239in}{2.629069in}}%
\pgfpathcurveto{\pgfqpoint{2.158425in}{2.621256in}}{\pgfqpoint{2.154035in}{2.610656in}}{\pgfqpoint{2.154035in}{2.599606in}}%
\pgfpathcurveto{\pgfqpoint{2.154035in}{2.588556in}}{\pgfqpoint{2.158425in}{2.577957in}}{\pgfqpoint{2.166239in}{2.570144in}}%
\pgfpathcurveto{\pgfqpoint{2.174052in}{2.562330in}}{\pgfqpoint{2.184651in}{2.557940in}}{\pgfqpoint{2.195701in}{2.557940in}}%
\pgfpathclose%
\pgfusepath{stroke,fill}%
\end{pgfscope}%
\begin{pgfscope}%
\pgfpathrectangle{\pgfqpoint{0.600000in}{0.600000in}}{\pgfqpoint{3.900000in}{3.900000in}}%
\pgfusepath{clip}%
\pgfsetbuttcap%
\pgfsetroundjoin%
\definecolor{currentfill}{rgb}{0.121569,0.466667,0.705882}%
\pgfsetfillcolor{currentfill}%
\pgfsetlinewidth{1.003750pt}%
\definecolor{currentstroke}{rgb}{0.121569,0.466667,0.705882}%
\pgfsetstrokecolor{currentstroke}%
\pgfsetdash{}{0pt}%
\pgfpathmoveto{\pgfqpoint{2.374137in}{2.278625in}}%
\pgfpathcurveto{\pgfqpoint{2.385187in}{2.278625in}}{\pgfqpoint{2.395786in}{2.283015in}}{\pgfqpoint{2.403600in}{2.290829in}}%
\pgfpathcurveto{\pgfqpoint{2.411413in}{2.298643in}}{\pgfqpoint{2.415804in}{2.309242in}}{\pgfqpoint{2.415804in}{2.320292in}}%
\pgfpathcurveto{\pgfqpoint{2.415804in}{2.331342in}}{\pgfqpoint{2.411413in}{2.341941in}}{\pgfqpoint{2.403600in}{2.349754in}}%
\pgfpathcurveto{\pgfqpoint{2.395786in}{2.357568in}}{\pgfqpoint{2.385187in}{2.361958in}}{\pgfqpoint{2.374137in}{2.361958in}}%
\pgfpathcurveto{\pgfqpoint{2.363087in}{2.361958in}}{\pgfqpoint{2.352488in}{2.357568in}}{\pgfqpoint{2.344674in}{2.349754in}}%
\pgfpathcurveto{\pgfqpoint{2.336861in}{2.341941in}}{\pgfqpoint{2.332470in}{2.331342in}}{\pgfqpoint{2.332470in}{2.320292in}}%
\pgfpathcurveto{\pgfqpoint{2.332470in}{2.309242in}}{\pgfqpoint{2.336861in}{2.298643in}}{\pgfqpoint{2.344674in}{2.290829in}}%
\pgfpathcurveto{\pgfqpoint{2.352488in}{2.283015in}}{\pgfqpoint{2.363087in}{2.278625in}}{\pgfqpoint{2.374137in}{2.278625in}}%
\pgfpathclose%
\pgfusepath{stroke,fill}%
\end{pgfscope}%
\begin{pgfscope}%
\pgfpathrectangle{\pgfqpoint{0.600000in}{0.600000in}}{\pgfqpoint{3.900000in}{3.900000in}}%
\pgfusepath{clip}%
\pgfsetbuttcap%
\pgfsetroundjoin%
\definecolor{currentfill}{rgb}{0.121569,0.466667,0.705882}%
\pgfsetfillcolor{currentfill}%
\pgfsetlinewidth{1.003750pt}%
\definecolor{currentstroke}{rgb}{0.121569,0.466667,0.705882}%
\pgfsetstrokecolor{currentstroke}%
\pgfsetdash{}{0pt}%
\pgfpathmoveto{\pgfqpoint{2.557149in}{2.363275in}}%
\pgfpathcurveto{\pgfqpoint{2.568200in}{2.363275in}}{\pgfqpoint{2.578799in}{2.367665in}}{\pgfqpoint{2.586612in}{2.375479in}}%
\pgfpathcurveto{\pgfqpoint{2.594426in}{2.383292in}}{\pgfqpoint{2.598816in}{2.393891in}}{\pgfqpoint{2.598816in}{2.404942in}}%
\pgfpathcurveto{\pgfqpoint{2.598816in}{2.415992in}}{\pgfqpoint{2.594426in}{2.426591in}}{\pgfqpoint{2.586612in}{2.434404in}}%
\pgfpathcurveto{\pgfqpoint{2.578799in}{2.442218in}}{\pgfqpoint{2.568200in}{2.446608in}}{\pgfqpoint{2.557149in}{2.446608in}}%
\pgfpathcurveto{\pgfqpoint{2.546099in}{2.446608in}}{\pgfqpoint{2.535500in}{2.442218in}}{\pgfqpoint{2.527687in}{2.434404in}}%
\pgfpathcurveto{\pgfqpoint{2.519873in}{2.426591in}}{\pgfqpoint{2.515483in}{2.415992in}}{\pgfqpoint{2.515483in}{2.404942in}}%
\pgfpathcurveto{\pgfqpoint{2.515483in}{2.393891in}}{\pgfqpoint{2.519873in}{2.383292in}}{\pgfqpoint{2.527687in}{2.375479in}}%
\pgfpathcurveto{\pgfqpoint{2.535500in}{2.367665in}}{\pgfqpoint{2.546099in}{2.363275in}}{\pgfqpoint{2.557149in}{2.363275in}}%
\pgfpathclose%
\pgfusepath{stroke,fill}%
\end{pgfscope}%
\begin{pgfscope}%
\pgfpathrectangle{\pgfqpoint{0.600000in}{0.600000in}}{\pgfqpoint{3.900000in}{3.900000in}}%
\pgfusepath{clip}%
\pgfsetbuttcap%
\pgfsetroundjoin%
\definecolor{currentfill}{rgb}{0.121569,0.466667,0.705882}%
\pgfsetfillcolor{currentfill}%
\pgfsetlinewidth{1.003750pt}%
\definecolor{currentstroke}{rgb}{0.121569,0.466667,0.705882}%
\pgfsetstrokecolor{currentstroke}%
\pgfsetdash{}{0pt}%
\pgfpathmoveto{\pgfqpoint{2.776193in}{3.662985in}}%
\pgfpathcurveto{\pgfqpoint{2.787243in}{3.662985in}}{\pgfqpoint{2.797842in}{3.667375in}}{\pgfqpoint{2.805656in}{3.675189in}}%
\pgfpathcurveto{\pgfqpoint{2.813469in}{3.683002in}}{\pgfqpoint{2.817860in}{3.693601in}}{\pgfqpoint{2.817860in}{3.704651in}}%
\pgfpathcurveto{\pgfqpoint{2.817860in}{3.715702in}}{\pgfqpoint{2.813469in}{3.726301in}}{\pgfqpoint{2.805656in}{3.734114in}}%
\pgfpathcurveto{\pgfqpoint{2.797842in}{3.741928in}}{\pgfqpoint{2.787243in}{3.746318in}}{\pgfqpoint{2.776193in}{3.746318in}}%
\pgfpathcurveto{\pgfqpoint{2.765143in}{3.746318in}}{\pgfqpoint{2.754544in}{3.741928in}}{\pgfqpoint{2.746730in}{3.734114in}}%
\pgfpathcurveto{\pgfqpoint{2.738917in}{3.726301in}}{\pgfqpoint{2.734526in}{3.715702in}}{\pgfqpoint{2.734526in}{3.704651in}}%
\pgfpathcurveto{\pgfqpoint{2.734526in}{3.693601in}}{\pgfqpoint{2.738917in}{3.683002in}}{\pgfqpoint{2.746730in}{3.675189in}}%
\pgfpathcurveto{\pgfqpoint{2.754544in}{3.667375in}}{\pgfqpoint{2.765143in}{3.662985in}}{\pgfqpoint{2.776193in}{3.662985in}}%
\pgfpathclose%
\pgfusepath{stroke,fill}%
\end{pgfscope}%
\begin{pgfscope}%
\pgfpathrectangle{\pgfqpoint{0.600000in}{0.600000in}}{\pgfqpoint{3.900000in}{3.900000in}}%
\pgfusepath{clip}%
\pgfsetbuttcap%
\pgfsetroundjoin%
\definecolor{currentfill}{rgb}{0.121569,0.466667,0.705882}%
\pgfsetfillcolor{currentfill}%
\pgfsetlinewidth{1.003750pt}%
\definecolor{currentstroke}{rgb}{0.121569,0.466667,0.705882}%
\pgfsetstrokecolor{currentstroke}%
\pgfsetdash{}{0pt}%
\pgfpathmoveto{\pgfqpoint{2.426712in}{2.965387in}}%
\pgfpathcurveto{\pgfqpoint{2.437762in}{2.965387in}}{\pgfqpoint{2.448361in}{2.969777in}}{\pgfqpoint{2.456175in}{2.977591in}}%
\pgfpathcurveto{\pgfqpoint{2.463988in}{2.985404in}}{\pgfqpoint{2.468379in}{2.996003in}}{\pgfqpoint{2.468379in}{3.007053in}}%
\pgfpathcurveto{\pgfqpoint{2.468379in}{3.018104in}}{\pgfqpoint{2.463988in}{3.028703in}}{\pgfqpoint{2.456175in}{3.036516in}}%
\pgfpathcurveto{\pgfqpoint{2.448361in}{3.044330in}}{\pgfqpoint{2.437762in}{3.048720in}}{\pgfqpoint{2.426712in}{3.048720in}}%
\pgfpathcurveto{\pgfqpoint{2.415662in}{3.048720in}}{\pgfqpoint{2.405063in}{3.044330in}}{\pgfqpoint{2.397249in}{3.036516in}}%
\pgfpathcurveto{\pgfqpoint{2.389436in}{3.028703in}}{\pgfqpoint{2.385045in}{3.018104in}}{\pgfqpoint{2.385045in}{3.007053in}}%
\pgfpathcurveto{\pgfqpoint{2.385045in}{2.996003in}}{\pgfqpoint{2.389436in}{2.985404in}}{\pgfqpoint{2.397249in}{2.977591in}}%
\pgfpathcurveto{\pgfqpoint{2.405063in}{2.969777in}}{\pgfqpoint{2.415662in}{2.965387in}}{\pgfqpoint{2.426712in}{2.965387in}}%
\pgfpathclose%
\pgfusepath{stroke,fill}%
\end{pgfscope}%
\begin{pgfscope}%
\pgfpathrectangle{\pgfqpoint{0.600000in}{0.600000in}}{\pgfqpoint{3.900000in}{3.900000in}}%
\pgfusepath{clip}%
\pgfsetbuttcap%
\pgfsetroundjoin%
\definecolor{currentfill}{rgb}{0.121569,0.466667,0.705882}%
\pgfsetfillcolor{currentfill}%
\pgfsetlinewidth{1.003750pt}%
\definecolor{currentstroke}{rgb}{0.121569,0.466667,0.705882}%
\pgfsetstrokecolor{currentstroke}%
\pgfsetdash{}{0pt}%
\pgfpathmoveto{\pgfqpoint{2.525697in}{2.154470in}}%
\pgfpathcurveto{\pgfqpoint{2.536747in}{2.154470in}}{\pgfqpoint{2.547346in}{2.158860in}}{\pgfqpoint{2.555160in}{2.166674in}}%
\pgfpathcurveto{\pgfqpoint{2.562974in}{2.174487in}}{\pgfqpoint{2.567364in}{2.185086in}}{\pgfqpoint{2.567364in}{2.196137in}}%
\pgfpathcurveto{\pgfqpoint{2.567364in}{2.207187in}}{\pgfqpoint{2.562974in}{2.217786in}}{\pgfqpoint{2.555160in}{2.225599in}}%
\pgfpathcurveto{\pgfqpoint{2.547346in}{2.233413in}}{\pgfqpoint{2.536747in}{2.237803in}}{\pgfqpoint{2.525697in}{2.237803in}}%
\pgfpathcurveto{\pgfqpoint{2.514647in}{2.237803in}}{\pgfqpoint{2.504048in}{2.233413in}}{\pgfqpoint{2.496234in}{2.225599in}}%
\pgfpathcurveto{\pgfqpoint{2.488421in}{2.217786in}}{\pgfqpoint{2.484031in}{2.207187in}}{\pgfqpoint{2.484031in}{2.196137in}}%
\pgfpathcurveto{\pgfqpoint{2.484031in}{2.185086in}}{\pgfqpoint{2.488421in}{2.174487in}}{\pgfqpoint{2.496234in}{2.166674in}}%
\pgfpathcurveto{\pgfqpoint{2.504048in}{2.158860in}}{\pgfqpoint{2.514647in}{2.154470in}}{\pgfqpoint{2.525697in}{2.154470in}}%
\pgfpathclose%
\pgfusepath{stroke,fill}%
\end{pgfscope}%
\begin{pgfscope}%
\pgfpathrectangle{\pgfqpoint{0.600000in}{0.600000in}}{\pgfqpoint{3.900000in}{3.900000in}}%
\pgfusepath{clip}%
\pgfsetbuttcap%
\pgfsetroundjoin%
\definecolor{currentfill}{rgb}{0.121569,0.466667,0.705882}%
\pgfsetfillcolor{currentfill}%
\pgfsetlinewidth{1.003750pt}%
\definecolor{currentstroke}{rgb}{0.121569,0.466667,0.705882}%
\pgfsetstrokecolor{currentstroke}%
\pgfsetdash{}{0pt}%
\pgfpathmoveto{\pgfqpoint{2.938537in}{2.486872in}}%
\pgfpathcurveto{\pgfqpoint{2.949587in}{2.486872in}}{\pgfqpoint{2.960186in}{2.491262in}}{\pgfqpoint{2.968000in}{2.499076in}}%
\pgfpathcurveto{\pgfqpoint{2.975813in}{2.506889in}}{\pgfqpoint{2.980203in}{2.517488in}}{\pgfqpoint{2.980203in}{2.528538in}}%
\pgfpathcurveto{\pgfqpoint{2.980203in}{2.539589in}}{\pgfqpoint{2.975813in}{2.550188in}}{\pgfqpoint{2.968000in}{2.558001in}}%
\pgfpathcurveto{\pgfqpoint{2.960186in}{2.565815in}}{\pgfqpoint{2.949587in}{2.570205in}}{\pgfqpoint{2.938537in}{2.570205in}}%
\pgfpathcurveto{\pgfqpoint{2.927487in}{2.570205in}}{\pgfqpoint{2.916888in}{2.565815in}}{\pgfqpoint{2.909074in}{2.558001in}}%
\pgfpathcurveto{\pgfqpoint{2.901260in}{2.550188in}}{\pgfqpoint{2.896870in}{2.539589in}}{\pgfqpoint{2.896870in}{2.528538in}}%
\pgfpathcurveto{\pgfqpoint{2.896870in}{2.517488in}}{\pgfqpoint{2.901260in}{2.506889in}}{\pgfqpoint{2.909074in}{2.499076in}}%
\pgfpathcurveto{\pgfqpoint{2.916888in}{2.491262in}}{\pgfqpoint{2.927487in}{2.486872in}}{\pgfqpoint{2.938537in}{2.486872in}}%
\pgfpathclose%
\pgfusepath{stroke,fill}%
\end{pgfscope}%
\begin{pgfscope}%
\pgfpathrectangle{\pgfqpoint{0.600000in}{0.600000in}}{\pgfqpoint{3.900000in}{3.900000in}}%
\pgfusepath{clip}%
\pgfsetbuttcap%
\pgfsetroundjoin%
\definecolor{currentfill}{rgb}{0.121569,0.466667,0.705882}%
\pgfsetfillcolor{currentfill}%
\pgfsetlinewidth{1.003750pt}%
\definecolor{currentstroke}{rgb}{0.121569,0.466667,0.705882}%
\pgfsetstrokecolor{currentstroke}%
\pgfsetdash{}{0pt}%
\pgfpathmoveto{\pgfqpoint{2.251904in}{2.117383in}}%
\pgfpathcurveto{\pgfqpoint{2.262954in}{2.117383in}}{\pgfqpoint{2.273553in}{2.121774in}}{\pgfqpoint{2.281367in}{2.129587in}}%
\pgfpathcurveto{\pgfqpoint{2.289180in}{2.137401in}}{\pgfqpoint{2.293571in}{2.148000in}}{\pgfqpoint{2.293571in}{2.159050in}}%
\pgfpathcurveto{\pgfqpoint{2.293571in}{2.170100in}}{\pgfqpoint{2.289180in}{2.180699in}}{\pgfqpoint{2.281367in}{2.188513in}}%
\pgfpathcurveto{\pgfqpoint{2.273553in}{2.196326in}}{\pgfqpoint{2.262954in}{2.200717in}}{\pgfqpoint{2.251904in}{2.200717in}}%
\pgfpathcurveto{\pgfqpoint{2.240854in}{2.200717in}}{\pgfqpoint{2.230255in}{2.196326in}}{\pgfqpoint{2.222441in}{2.188513in}}%
\pgfpathcurveto{\pgfqpoint{2.214627in}{2.180699in}}{\pgfqpoint{2.210237in}{2.170100in}}{\pgfqpoint{2.210237in}{2.159050in}}%
\pgfpathcurveto{\pgfqpoint{2.210237in}{2.148000in}}{\pgfqpoint{2.214627in}{2.137401in}}{\pgfqpoint{2.222441in}{2.129587in}}%
\pgfpathcurveto{\pgfqpoint{2.230255in}{2.121774in}}{\pgfqpoint{2.240854in}{2.117383in}}{\pgfqpoint{2.251904in}{2.117383in}}%
\pgfpathclose%
\pgfusepath{stroke,fill}%
\end{pgfscope}%
\begin{pgfscope}%
\pgfpathrectangle{\pgfqpoint{0.600000in}{0.600000in}}{\pgfqpoint{3.900000in}{3.900000in}}%
\pgfusepath{clip}%
\pgfsetbuttcap%
\pgfsetroundjoin%
\definecolor{currentfill}{rgb}{0.121569,0.466667,0.705882}%
\pgfsetfillcolor{currentfill}%
\pgfsetlinewidth{1.003750pt}%
\definecolor{currentstroke}{rgb}{0.121569,0.466667,0.705882}%
\pgfsetstrokecolor{currentstroke}%
\pgfsetdash{}{0pt}%
\pgfpathmoveto{\pgfqpoint{2.425351in}{2.591594in}}%
\pgfpathcurveto{\pgfqpoint{2.436401in}{2.591594in}}{\pgfqpoint{2.447000in}{2.595984in}}{\pgfqpoint{2.454814in}{2.603798in}}%
\pgfpathcurveto{\pgfqpoint{2.462627in}{2.611611in}}{\pgfqpoint{2.467018in}{2.622210in}}{\pgfqpoint{2.467018in}{2.633260in}}%
\pgfpathcurveto{\pgfqpoint{2.467018in}{2.644311in}}{\pgfqpoint{2.462627in}{2.654910in}}{\pgfqpoint{2.454814in}{2.662723in}}%
\pgfpathcurveto{\pgfqpoint{2.447000in}{2.670537in}}{\pgfqpoint{2.436401in}{2.674927in}}{\pgfqpoint{2.425351in}{2.674927in}}%
\pgfpathcurveto{\pgfqpoint{2.414301in}{2.674927in}}{\pgfqpoint{2.403702in}{2.670537in}}{\pgfqpoint{2.395888in}{2.662723in}}%
\pgfpathcurveto{\pgfqpoint{2.388075in}{2.654910in}}{\pgfqpoint{2.383684in}{2.644311in}}{\pgfqpoint{2.383684in}{2.633260in}}%
\pgfpathcurveto{\pgfqpoint{2.383684in}{2.622210in}}{\pgfqpoint{2.388075in}{2.611611in}}{\pgfqpoint{2.395888in}{2.603798in}}%
\pgfpathcurveto{\pgfqpoint{2.403702in}{2.595984in}}{\pgfqpoint{2.414301in}{2.591594in}}{\pgfqpoint{2.425351in}{2.591594in}}%
\pgfpathclose%
\pgfusepath{stroke,fill}%
\end{pgfscope}%
\begin{pgfscope}%
\pgfpathrectangle{\pgfqpoint{0.600000in}{0.600000in}}{\pgfqpoint{3.900000in}{3.900000in}}%
\pgfusepath{clip}%
\pgfsetbuttcap%
\pgfsetroundjoin%
\definecolor{currentfill}{rgb}{0.121569,0.466667,0.705882}%
\pgfsetfillcolor{currentfill}%
\pgfsetlinewidth{1.003750pt}%
\definecolor{currentstroke}{rgb}{0.121569,0.466667,0.705882}%
\pgfsetstrokecolor{currentstroke}%
\pgfsetdash{}{0pt}%
\pgfpathmoveto{\pgfqpoint{1.939889in}{1.283260in}}%
\pgfpathcurveto{\pgfqpoint{1.950939in}{1.283260in}}{\pgfqpoint{1.961538in}{1.287651in}}{\pgfqpoint{1.969351in}{1.295464in}}%
\pgfpathcurveto{\pgfqpoint{1.977165in}{1.303278in}}{\pgfqpoint{1.981555in}{1.313877in}}{\pgfqpoint{1.981555in}{1.324927in}}%
\pgfpathcurveto{\pgfqpoint{1.981555in}{1.335977in}}{\pgfqpoint{1.977165in}{1.346576in}}{\pgfqpoint{1.969351in}{1.354390in}}%
\pgfpathcurveto{\pgfqpoint{1.961538in}{1.362203in}}{\pgfqpoint{1.950939in}{1.366594in}}{\pgfqpoint{1.939889in}{1.366594in}}%
\pgfpathcurveto{\pgfqpoint{1.928838in}{1.366594in}}{\pgfqpoint{1.918239in}{1.362203in}}{\pgfqpoint{1.910426in}{1.354390in}}%
\pgfpathcurveto{\pgfqpoint{1.902612in}{1.346576in}}{\pgfqpoint{1.898222in}{1.335977in}}{\pgfqpoint{1.898222in}{1.324927in}}%
\pgfpathcurveto{\pgfqpoint{1.898222in}{1.313877in}}{\pgfqpoint{1.902612in}{1.303278in}}{\pgfqpoint{1.910426in}{1.295464in}}%
\pgfpathcurveto{\pgfqpoint{1.918239in}{1.287651in}}{\pgfqpoint{1.928838in}{1.283260in}}{\pgfqpoint{1.939889in}{1.283260in}}%
\pgfpathclose%
\pgfusepath{stroke,fill}%
\end{pgfscope}%
\begin{pgfscope}%
\pgfpathrectangle{\pgfqpoint{0.600000in}{0.600000in}}{\pgfqpoint{3.900000in}{3.900000in}}%
\pgfusepath{clip}%
\pgfsetbuttcap%
\pgfsetroundjoin%
\definecolor{currentfill}{rgb}{0.121569,0.466667,0.705882}%
\pgfsetfillcolor{currentfill}%
\pgfsetlinewidth{1.003750pt}%
\definecolor{currentstroke}{rgb}{0.121569,0.466667,0.705882}%
\pgfsetstrokecolor{currentstroke}%
\pgfsetdash{}{0pt}%
\pgfpathmoveto{\pgfqpoint{2.226248in}{2.575349in}}%
\pgfpathcurveto{\pgfqpoint{2.237298in}{2.575349in}}{\pgfqpoint{2.247897in}{2.579740in}}{\pgfqpoint{2.255711in}{2.587553in}}%
\pgfpathcurveto{\pgfqpoint{2.263524in}{2.595367in}}{\pgfqpoint{2.267915in}{2.605966in}}{\pgfqpoint{2.267915in}{2.617016in}}%
\pgfpathcurveto{\pgfqpoint{2.267915in}{2.628066in}}{\pgfqpoint{2.263524in}{2.638665in}}{\pgfqpoint{2.255711in}{2.646479in}}%
\pgfpathcurveto{\pgfqpoint{2.247897in}{2.654292in}}{\pgfqpoint{2.237298in}{2.658683in}}{\pgfqpoint{2.226248in}{2.658683in}}%
\pgfpathcurveto{\pgfqpoint{2.215198in}{2.658683in}}{\pgfqpoint{2.204599in}{2.654292in}}{\pgfqpoint{2.196785in}{2.646479in}}%
\pgfpathcurveto{\pgfqpoint{2.188972in}{2.638665in}}{\pgfqpoint{2.184581in}{2.628066in}}{\pgfqpoint{2.184581in}{2.617016in}}%
\pgfpathcurveto{\pgfqpoint{2.184581in}{2.605966in}}{\pgfqpoint{2.188972in}{2.595367in}}{\pgfqpoint{2.196785in}{2.587553in}}%
\pgfpathcurveto{\pgfqpoint{2.204599in}{2.579740in}}{\pgfqpoint{2.215198in}{2.575349in}}{\pgfqpoint{2.226248in}{2.575349in}}%
\pgfpathclose%
\pgfusepath{stroke,fill}%
\end{pgfscope}%
\begin{pgfscope}%
\pgfpathrectangle{\pgfqpoint{0.600000in}{0.600000in}}{\pgfqpoint{3.900000in}{3.900000in}}%
\pgfusepath{clip}%
\pgfsetbuttcap%
\pgfsetroundjoin%
\definecolor{currentfill}{rgb}{0.121569,0.466667,0.705882}%
\pgfsetfillcolor{currentfill}%
\pgfsetlinewidth{1.003750pt}%
\definecolor{currentstroke}{rgb}{0.121569,0.466667,0.705882}%
\pgfsetstrokecolor{currentstroke}%
\pgfsetdash{}{0pt}%
\pgfpathmoveto{\pgfqpoint{1.980312in}{1.676494in}}%
\pgfpathcurveto{\pgfqpoint{1.991362in}{1.676494in}}{\pgfqpoint{2.001961in}{1.680884in}}{\pgfqpoint{2.009775in}{1.688698in}}%
\pgfpathcurveto{\pgfqpoint{2.017588in}{1.696511in}}{\pgfqpoint{2.021979in}{1.707110in}}{\pgfqpoint{2.021979in}{1.718161in}}%
\pgfpathcurveto{\pgfqpoint{2.021979in}{1.729211in}}{\pgfqpoint{2.017588in}{1.739810in}}{\pgfqpoint{2.009775in}{1.747623in}}%
\pgfpathcurveto{\pgfqpoint{2.001961in}{1.755437in}}{\pgfqpoint{1.991362in}{1.759827in}}{\pgfqpoint{1.980312in}{1.759827in}}%
\pgfpathcurveto{\pgfqpoint{1.969262in}{1.759827in}}{\pgfqpoint{1.958663in}{1.755437in}}{\pgfqpoint{1.950849in}{1.747623in}}%
\pgfpathcurveto{\pgfqpoint{1.943035in}{1.739810in}}{\pgfqpoint{1.938645in}{1.729211in}}{\pgfqpoint{1.938645in}{1.718161in}}%
\pgfpathcurveto{\pgfqpoint{1.938645in}{1.707110in}}{\pgfqpoint{1.943035in}{1.696511in}}{\pgfqpoint{1.950849in}{1.688698in}}%
\pgfpathcurveto{\pgfqpoint{1.958663in}{1.680884in}}{\pgfqpoint{1.969262in}{1.676494in}}{\pgfqpoint{1.980312in}{1.676494in}}%
\pgfpathclose%
\pgfusepath{stroke,fill}%
\end{pgfscope}%
\begin{pgfscope}%
\pgfpathrectangle{\pgfqpoint{0.600000in}{0.600000in}}{\pgfqpoint{3.900000in}{3.900000in}}%
\pgfusepath{clip}%
\pgfsetbuttcap%
\pgfsetroundjoin%
\definecolor{currentfill}{rgb}{0.121569,0.466667,0.705882}%
\pgfsetfillcolor{currentfill}%
\pgfsetlinewidth{1.003750pt}%
\definecolor{currentstroke}{rgb}{0.121569,0.466667,0.705882}%
\pgfsetstrokecolor{currentstroke}%
\pgfsetdash{}{0pt}%
\pgfpathmoveto{\pgfqpoint{2.974604in}{2.618829in}}%
\pgfpathcurveto{\pgfqpoint{2.985654in}{2.618829in}}{\pgfqpoint{2.996253in}{2.623219in}}{\pgfqpoint{3.004066in}{2.631033in}}%
\pgfpathcurveto{\pgfqpoint{3.011880in}{2.638846in}}{\pgfqpoint{3.016270in}{2.649445in}}{\pgfqpoint{3.016270in}{2.660495in}}%
\pgfpathcurveto{\pgfqpoint{3.016270in}{2.671546in}}{\pgfqpoint{3.011880in}{2.682145in}}{\pgfqpoint{3.004066in}{2.689958in}}%
\pgfpathcurveto{\pgfqpoint{2.996253in}{2.697772in}}{\pgfqpoint{2.985654in}{2.702162in}}{\pgfqpoint{2.974604in}{2.702162in}}%
\pgfpathcurveto{\pgfqpoint{2.963553in}{2.702162in}}{\pgfqpoint{2.952954in}{2.697772in}}{\pgfqpoint{2.945141in}{2.689958in}}%
\pgfpathcurveto{\pgfqpoint{2.937327in}{2.682145in}}{\pgfqpoint{2.932937in}{2.671546in}}{\pgfqpoint{2.932937in}{2.660495in}}%
\pgfpathcurveto{\pgfqpoint{2.932937in}{2.649445in}}{\pgfqpoint{2.937327in}{2.638846in}}{\pgfqpoint{2.945141in}{2.631033in}}%
\pgfpathcurveto{\pgfqpoint{2.952954in}{2.623219in}}{\pgfqpoint{2.963553in}{2.618829in}}{\pgfqpoint{2.974604in}{2.618829in}}%
\pgfpathclose%
\pgfusepath{stroke,fill}%
\end{pgfscope}%
\begin{pgfscope}%
\pgfpathrectangle{\pgfqpoint{0.600000in}{0.600000in}}{\pgfqpoint{3.900000in}{3.900000in}}%
\pgfusepath{clip}%
\pgfsetbuttcap%
\pgfsetroundjoin%
\definecolor{currentfill}{rgb}{0.121569,0.466667,0.705882}%
\pgfsetfillcolor{currentfill}%
\pgfsetlinewidth{1.003750pt}%
\definecolor{currentstroke}{rgb}{0.121569,0.466667,0.705882}%
\pgfsetstrokecolor{currentstroke}%
\pgfsetdash{}{0pt}%
\pgfpathmoveto{\pgfqpoint{2.342923in}{2.619660in}}%
\pgfpathcurveto{\pgfqpoint{2.353973in}{2.619660in}}{\pgfqpoint{2.364572in}{2.624050in}}{\pgfqpoint{2.372386in}{2.631864in}}%
\pgfpathcurveto{\pgfqpoint{2.380200in}{2.639678in}}{\pgfqpoint{2.384590in}{2.650277in}}{\pgfqpoint{2.384590in}{2.661327in}}%
\pgfpathcurveto{\pgfqpoint{2.384590in}{2.672377in}}{\pgfqpoint{2.380200in}{2.682976in}}{\pgfqpoint{2.372386in}{2.690790in}}%
\pgfpathcurveto{\pgfqpoint{2.364572in}{2.698603in}}{\pgfqpoint{2.353973in}{2.702994in}}{\pgfqpoint{2.342923in}{2.702994in}}%
\pgfpathcurveto{\pgfqpoint{2.331873in}{2.702994in}}{\pgfqpoint{2.321274in}{2.698603in}}{\pgfqpoint{2.313460in}{2.690790in}}%
\pgfpathcurveto{\pgfqpoint{2.305647in}{2.682976in}}{\pgfqpoint{2.301256in}{2.672377in}}{\pgfqpoint{2.301256in}{2.661327in}}%
\pgfpathcurveto{\pgfqpoint{2.301256in}{2.650277in}}{\pgfqpoint{2.305647in}{2.639678in}}{\pgfqpoint{2.313460in}{2.631864in}}%
\pgfpathcurveto{\pgfqpoint{2.321274in}{2.624050in}}{\pgfqpoint{2.331873in}{2.619660in}}{\pgfqpoint{2.342923in}{2.619660in}}%
\pgfpathclose%
\pgfusepath{stroke,fill}%
\end{pgfscope}%
\begin{pgfscope}%
\pgfpathrectangle{\pgfqpoint{0.600000in}{0.600000in}}{\pgfqpoint{3.900000in}{3.900000in}}%
\pgfusepath{clip}%
\pgfsetbuttcap%
\pgfsetroundjoin%
\definecolor{currentfill}{rgb}{0.121569,0.466667,0.705882}%
\pgfsetfillcolor{currentfill}%
\pgfsetlinewidth{1.003750pt}%
\definecolor{currentstroke}{rgb}{0.121569,0.466667,0.705882}%
\pgfsetstrokecolor{currentstroke}%
\pgfsetdash{}{0pt}%
\pgfpathmoveto{\pgfqpoint{1.971452in}{3.009783in}}%
\pgfpathcurveto{\pgfqpoint{1.982502in}{3.009783in}}{\pgfqpoint{1.993101in}{3.014174in}}{\pgfqpoint{2.000914in}{3.021987in}}%
\pgfpathcurveto{\pgfqpoint{2.008728in}{3.029801in}}{\pgfqpoint{2.013118in}{3.040400in}}{\pgfqpoint{2.013118in}{3.051450in}}%
\pgfpathcurveto{\pgfqpoint{2.013118in}{3.062500in}}{\pgfqpoint{2.008728in}{3.073099in}}{\pgfqpoint{2.000914in}{3.080913in}}%
\pgfpathcurveto{\pgfqpoint{1.993101in}{3.088726in}}{\pgfqpoint{1.982502in}{3.093117in}}{\pgfqpoint{1.971452in}{3.093117in}}%
\pgfpathcurveto{\pgfqpoint{1.960401in}{3.093117in}}{\pgfqpoint{1.949802in}{3.088726in}}{\pgfqpoint{1.941989in}{3.080913in}}%
\pgfpathcurveto{\pgfqpoint{1.934175in}{3.073099in}}{\pgfqpoint{1.929785in}{3.062500in}}{\pgfqpoint{1.929785in}{3.051450in}}%
\pgfpathcurveto{\pgfqpoint{1.929785in}{3.040400in}}{\pgfqpoint{1.934175in}{3.029801in}}{\pgfqpoint{1.941989in}{3.021987in}}%
\pgfpathcurveto{\pgfqpoint{1.949802in}{3.014174in}}{\pgfqpoint{1.960401in}{3.009783in}}{\pgfqpoint{1.971452in}{3.009783in}}%
\pgfpathclose%
\pgfusepath{stroke,fill}%
\end{pgfscope}%
\begin{pgfscope}%
\pgfpathrectangle{\pgfqpoint{0.600000in}{0.600000in}}{\pgfqpoint{3.900000in}{3.900000in}}%
\pgfusepath{clip}%
\pgfsetbuttcap%
\pgfsetroundjoin%
\definecolor{currentfill}{rgb}{0.121569,0.466667,0.705882}%
\pgfsetfillcolor{currentfill}%
\pgfsetlinewidth{1.003750pt}%
\definecolor{currentstroke}{rgb}{0.121569,0.466667,0.705882}%
\pgfsetstrokecolor{currentstroke}%
\pgfsetdash{}{0pt}%
\pgfpathmoveto{\pgfqpoint{3.430732in}{3.015165in}}%
\pgfpathcurveto{\pgfqpoint{3.441782in}{3.015165in}}{\pgfqpoint{3.452381in}{3.019555in}}{\pgfqpoint{3.460195in}{3.027368in}}%
\pgfpathcurveto{\pgfqpoint{3.468009in}{3.035182in}}{\pgfqpoint{3.472399in}{3.045781in}}{\pgfqpoint{3.472399in}{3.056831in}}%
\pgfpathcurveto{\pgfqpoint{3.472399in}{3.067881in}}{\pgfqpoint{3.468009in}{3.078480in}}{\pgfqpoint{3.460195in}{3.086294in}}%
\pgfpathcurveto{\pgfqpoint{3.452381in}{3.094108in}}{\pgfqpoint{3.441782in}{3.098498in}}{\pgfqpoint{3.430732in}{3.098498in}}%
\pgfpathcurveto{\pgfqpoint{3.419682in}{3.098498in}}{\pgfqpoint{3.409083in}{3.094108in}}{\pgfqpoint{3.401269in}{3.086294in}}%
\pgfpathcurveto{\pgfqpoint{3.393456in}{3.078480in}}{\pgfqpoint{3.389065in}{3.067881in}}{\pgfqpoint{3.389065in}{3.056831in}}%
\pgfpathcurveto{\pgfqpoint{3.389065in}{3.045781in}}{\pgfqpoint{3.393456in}{3.035182in}}{\pgfqpoint{3.401269in}{3.027368in}}%
\pgfpathcurveto{\pgfqpoint{3.409083in}{3.019555in}}{\pgfqpoint{3.419682in}{3.015165in}}{\pgfqpoint{3.430732in}{3.015165in}}%
\pgfpathclose%
\pgfusepath{stroke,fill}%
\end{pgfscope}%
\begin{pgfscope}%
\pgfpathrectangle{\pgfqpoint{0.600000in}{0.600000in}}{\pgfqpoint{3.900000in}{3.900000in}}%
\pgfusepath{clip}%
\pgfsetbuttcap%
\pgfsetroundjoin%
\definecolor{currentfill}{rgb}{0.121569,0.466667,0.705882}%
\pgfsetfillcolor{currentfill}%
\pgfsetlinewidth{1.003750pt}%
\definecolor{currentstroke}{rgb}{0.121569,0.466667,0.705882}%
\pgfsetstrokecolor{currentstroke}%
\pgfsetdash{}{0pt}%
\pgfpathmoveto{\pgfqpoint{3.096901in}{2.758940in}}%
\pgfpathcurveto{\pgfqpoint{3.107951in}{2.758940in}}{\pgfqpoint{3.118550in}{2.763330in}}{\pgfqpoint{3.126364in}{2.771144in}}%
\pgfpathcurveto{\pgfqpoint{3.134177in}{2.778958in}}{\pgfqpoint{3.138567in}{2.789557in}}{\pgfqpoint{3.138567in}{2.800607in}}%
\pgfpathcurveto{\pgfqpoint{3.138567in}{2.811657in}}{\pgfqpoint{3.134177in}{2.822256in}}{\pgfqpoint{3.126364in}{2.830070in}}%
\pgfpathcurveto{\pgfqpoint{3.118550in}{2.837883in}}{\pgfqpoint{3.107951in}{2.842273in}}{\pgfqpoint{3.096901in}{2.842273in}}%
\pgfpathcurveto{\pgfqpoint{3.085851in}{2.842273in}}{\pgfqpoint{3.075252in}{2.837883in}}{\pgfqpoint{3.067438in}{2.830070in}}%
\pgfpathcurveto{\pgfqpoint{3.059624in}{2.822256in}}{\pgfqpoint{3.055234in}{2.811657in}}{\pgfqpoint{3.055234in}{2.800607in}}%
\pgfpathcurveto{\pgfqpoint{3.055234in}{2.789557in}}{\pgfqpoint{3.059624in}{2.778958in}}{\pgfqpoint{3.067438in}{2.771144in}}%
\pgfpathcurveto{\pgfqpoint{3.075252in}{2.763330in}}{\pgfqpoint{3.085851in}{2.758940in}}{\pgfqpoint{3.096901in}{2.758940in}}%
\pgfpathclose%
\pgfusepath{stroke,fill}%
\end{pgfscope}%
\begin{pgfscope}%
\pgfpathrectangle{\pgfqpoint{0.600000in}{0.600000in}}{\pgfqpoint{3.900000in}{3.900000in}}%
\pgfusepath{clip}%
\pgfsetbuttcap%
\pgfsetroundjoin%
\definecolor{currentfill}{rgb}{0.121569,0.466667,0.705882}%
\pgfsetfillcolor{currentfill}%
\pgfsetlinewidth{1.003750pt}%
\definecolor{currentstroke}{rgb}{0.121569,0.466667,0.705882}%
\pgfsetstrokecolor{currentstroke}%
\pgfsetdash{}{0pt}%
\pgfpathmoveto{\pgfqpoint{1.807855in}{2.746843in}}%
\pgfpathcurveto{\pgfqpoint{1.818905in}{2.746843in}}{\pgfqpoint{1.829504in}{2.751234in}}{\pgfqpoint{1.837318in}{2.759047in}}%
\pgfpathcurveto{\pgfqpoint{1.845132in}{2.766861in}}{\pgfqpoint{1.849522in}{2.777460in}}{\pgfqpoint{1.849522in}{2.788510in}}%
\pgfpathcurveto{\pgfqpoint{1.849522in}{2.799560in}}{\pgfqpoint{1.845132in}{2.810159in}}{\pgfqpoint{1.837318in}{2.817973in}}%
\pgfpathcurveto{\pgfqpoint{1.829504in}{2.825786in}}{\pgfqpoint{1.818905in}{2.830177in}}{\pgfqpoint{1.807855in}{2.830177in}}%
\pgfpathcurveto{\pgfqpoint{1.796805in}{2.830177in}}{\pgfqpoint{1.786206in}{2.825786in}}{\pgfqpoint{1.778393in}{2.817973in}}%
\pgfpathcurveto{\pgfqpoint{1.770579in}{2.810159in}}{\pgfqpoint{1.766189in}{2.799560in}}{\pgfqpoint{1.766189in}{2.788510in}}%
\pgfpathcurveto{\pgfqpoint{1.766189in}{2.777460in}}{\pgfqpoint{1.770579in}{2.766861in}}{\pgfqpoint{1.778393in}{2.759047in}}%
\pgfpathcurveto{\pgfqpoint{1.786206in}{2.751234in}}{\pgfqpoint{1.796805in}{2.746843in}}{\pgfqpoint{1.807855in}{2.746843in}}%
\pgfpathclose%
\pgfusepath{stroke,fill}%
\end{pgfscope}%
\begin{pgfscope}%
\pgfpathrectangle{\pgfqpoint{0.600000in}{0.600000in}}{\pgfqpoint{3.900000in}{3.900000in}}%
\pgfusepath{clip}%
\pgfsetbuttcap%
\pgfsetroundjoin%
\definecolor{currentfill}{rgb}{0.121569,0.466667,0.705882}%
\pgfsetfillcolor{currentfill}%
\pgfsetlinewidth{1.003750pt}%
\definecolor{currentstroke}{rgb}{0.121569,0.466667,0.705882}%
\pgfsetstrokecolor{currentstroke}%
\pgfsetdash{}{0pt}%
\pgfpathmoveto{\pgfqpoint{3.012453in}{2.489415in}}%
\pgfpathcurveto{\pgfqpoint{3.023503in}{2.489415in}}{\pgfqpoint{3.034102in}{2.493806in}}{\pgfqpoint{3.041916in}{2.501619in}}%
\pgfpathcurveto{\pgfqpoint{3.049729in}{2.509433in}}{\pgfqpoint{3.054119in}{2.520032in}}{\pgfqpoint{3.054119in}{2.531082in}}%
\pgfpathcurveto{\pgfqpoint{3.054119in}{2.542132in}}{\pgfqpoint{3.049729in}{2.552731in}}{\pgfqpoint{3.041916in}{2.560545in}}%
\pgfpathcurveto{\pgfqpoint{3.034102in}{2.568358in}}{\pgfqpoint{3.023503in}{2.572749in}}{\pgfqpoint{3.012453in}{2.572749in}}%
\pgfpathcurveto{\pgfqpoint{3.001403in}{2.572749in}}{\pgfqpoint{2.990804in}{2.568358in}}{\pgfqpoint{2.982990in}{2.560545in}}%
\pgfpathcurveto{\pgfqpoint{2.975176in}{2.552731in}}{\pgfqpoint{2.970786in}{2.542132in}}{\pgfqpoint{2.970786in}{2.531082in}}%
\pgfpathcurveto{\pgfqpoint{2.970786in}{2.520032in}}{\pgfqpoint{2.975176in}{2.509433in}}{\pgfqpoint{2.982990in}{2.501619in}}%
\pgfpathcurveto{\pgfqpoint{2.990804in}{2.493806in}}{\pgfqpoint{3.001403in}{2.489415in}}{\pgfqpoint{3.012453in}{2.489415in}}%
\pgfpathclose%
\pgfusepath{stroke,fill}%
\end{pgfscope}%
\begin{pgfscope}%
\pgfpathrectangle{\pgfqpoint{0.600000in}{0.600000in}}{\pgfqpoint{3.900000in}{3.900000in}}%
\pgfusepath{clip}%
\pgfsetbuttcap%
\pgfsetroundjoin%
\definecolor{currentfill}{rgb}{0.121569,0.466667,0.705882}%
\pgfsetfillcolor{currentfill}%
\pgfsetlinewidth{1.003750pt}%
\definecolor{currentstroke}{rgb}{0.121569,0.466667,0.705882}%
\pgfsetstrokecolor{currentstroke}%
\pgfsetdash{}{0pt}%
\pgfpathmoveto{\pgfqpoint{2.594540in}{2.374933in}}%
\pgfpathcurveto{\pgfqpoint{2.605590in}{2.374933in}}{\pgfqpoint{2.616189in}{2.379323in}}{\pgfqpoint{2.624003in}{2.387137in}}%
\pgfpathcurveto{\pgfqpoint{2.631816in}{2.394951in}}{\pgfqpoint{2.636207in}{2.405550in}}{\pgfqpoint{2.636207in}{2.416600in}}%
\pgfpathcurveto{\pgfqpoint{2.636207in}{2.427650in}}{\pgfqpoint{2.631816in}{2.438249in}}{\pgfqpoint{2.624003in}{2.446062in}}%
\pgfpathcurveto{\pgfqpoint{2.616189in}{2.453876in}}{\pgfqpoint{2.605590in}{2.458266in}}{\pgfqpoint{2.594540in}{2.458266in}}%
\pgfpathcurveto{\pgfqpoint{2.583490in}{2.458266in}}{\pgfqpoint{2.572891in}{2.453876in}}{\pgfqpoint{2.565077in}{2.446062in}}%
\pgfpathcurveto{\pgfqpoint{2.557264in}{2.438249in}}{\pgfqpoint{2.552873in}{2.427650in}}{\pgfqpoint{2.552873in}{2.416600in}}%
\pgfpathcurveto{\pgfqpoint{2.552873in}{2.405550in}}{\pgfqpoint{2.557264in}{2.394951in}}{\pgfqpoint{2.565077in}{2.387137in}}%
\pgfpathcurveto{\pgfqpoint{2.572891in}{2.379323in}}{\pgfqpoint{2.583490in}{2.374933in}}{\pgfqpoint{2.594540in}{2.374933in}}%
\pgfpathclose%
\pgfusepath{stroke,fill}%
\end{pgfscope}%
\begin{pgfscope}%
\pgfpathrectangle{\pgfqpoint{0.600000in}{0.600000in}}{\pgfqpoint{3.900000in}{3.900000in}}%
\pgfusepath{clip}%
\pgfsetbuttcap%
\pgfsetroundjoin%
\definecolor{currentfill}{rgb}{0.121569,0.466667,0.705882}%
\pgfsetfillcolor{currentfill}%
\pgfsetlinewidth{1.003750pt}%
\definecolor{currentstroke}{rgb}{0.121569,0.466667,0.705882}%
\pgfsetstrokecolor{currentstroke}%
\pgfsetdash{}{0pt}%
\pgfpathmoveto{\pgfqpoint{2.895309in}{2.419994in}}%
\pgfpathcurveto{\pgfqpoint{2.906359in}{2.419994in}}{\pgfqpoint{2.916958in}{2.424384in}}{\pgfqpoint{2.924771in}{2.432197in}}%
\pgfpathcurveto{\pgfqpoint{2.932585in}{2.440011in}}{\pgfqpoint{2.936975in}{2.450610in}}{\pgfqpoint{2.936975in}{2.461660in}}%
\pgfpathcurveto{\pgfqpoint{2.936975in}{2.472710in}}{\pgfqpoint{2.932585in}{2.483309in}}{\pgfqpoint{2.924771in}{2.491123in}}%
\pgfpathcurveto{\pgfqpoint{2.916958in}{2.498937in}}{\pgfqpoint{2.906359in}{2.503327in}}{\pgfqpoint{2.895309in}{2.503327in}}%
\pgfpathcurveto{\pgfqpoint{2.884259in}{2.503327in}}{\pgfqpoint{2.873660in}{2.498937in}}{\pgfqpoint{2.865846in}{2.491123in}}%
\pgfpathcurveto{\pgfqpoint{2.858032in}{2.483309in}}{\pgfqpoint{2.853642in}{2.472710in}}{\pgfqpoint{2.853642in}{2.461660in}}%
\pgfpathcurveto{\pgfqpoint{2.853642in}{2.450610in}}{\pgfqpoint{2.858032in}{2.440011in}}{\pgfqpoint{2.865846in}{2.432197in}}%
\pgfpathcurveto{\pgfqpoint{2.873660in}{2.424384in}}{\pgfqpoint{2.884259in}{2.419994in}}{\pgfqpoint{2.895309in}{2.419994in}}%
\pgfpathclose%
\pgfusepath{stroke,fill}%
\end{pgfscope}%
\begin{pgfscope}%
\pgfpathrectangle{\pgfqpoint{0.600000in}{0.600000in}}{\pgfqpoint{3.900000in}{3.900000in}}%
\pgfusepath{clip}%
\pgfsetbuttcap%
\pgfsetroundjoin%
\definecolor{currentfill}{rgb}{0.121569,0.466667,0.705882}%
\pgfsetfillcolor{currentfill}%
\pgfsetlinewidth{1.003750pt}%
\definecolor{currentstroke}{rgb}{0.121569,0.466667,0.705882}%
\pgfsetstrokecolor{currentstroke}%
\pgfsetdash{}{0pt}%
\pgfpathmoveto{\pgfqpoint{2.916867in}{2.778058in}}%
\pgfpathcurveto{\pgfqpoint{2.927917in}{2.778058in}}{\pgfqpoint{2.938516in}{2.782448in}}{\pgfqpoint{2.946330in}{2.790262in}}%
\pgfpathcurveto{\pgfqpoint{2.954143in}{2.798076in}}{\pgfqpoint{2.958534in}{2.808675in}}{\pgfqpoint{2.958534in}{2.819725in}}%
\pgfpathcurveto{\pgfqpoint{2.958534in}{2.830775in}}{\pgfqpoint{2.954143in}{2.841374in}}{\pgfqpoint{2.946330in}{2.849188in}}%
\pgfpathcurveto{\pgfqpoint{2.938516in}{2.857001in}}{\pgfqpoint{2.927917in}{2.861392in}}{\pgfqpoint{2.916867in}{2.861392in}}%
\pgfpathcurveto{\pgfqpoint{2.905817in}{2.861392in}}{\pgfqpoint{2.895218in}{2.857001in}}{\pgfqpoint{2.887404in}{2.849188in}}%
\pgfpathcurveto{\pgfqpoint{2.879591in}{2.841374in}}{\pgfqpoint{2.875200in}{2.830775in}}{\pgfqpoint{2.875200in}{2.819725in}}%
\pgfpathcurveto{\pgfqpoint{2.875200in}{2.808675in}}{\pgfqpoint{2.879591in}{2.798076in}}{\pgfqpoint{2.887404in}{2.790262in}}%
\pgfpathcurveto{\pgfqpoint{2.895218in}{2.782448in}}{\pgfqpoint{2.905817in}{2.778058in}}{\pgfqpoint{2.916867in}{2.778058in}}%
\pgfpathclose%
\pgfusepath{stroke,fill}%
\end{pgfscope}%
\begin{pgfscope}%
\pgfpathrectangle{\pgfqpoint{0.600000in}{0.600000in}}{\pgfqpoint{3.900000in}{3.900000in}}%
\pgfusepath{clip}%
\pgfsetbuttcap%
\pgfsetroundjoin%
\definecolor{currentfill}{rgb}{0.121569,0.466667,0.705882}%
\pgfsetfillcolor{currentfill}%
\pgfsetlinewidth{1.003750pt}%
\definecolor{currentstroke}{rgb}{0.121569,0.466667,0.705882}%
\pgfsetstrokecolor{currentstroke}%
\pgfsetdash{}{0pt}%
\pgfpathmoveto{\pgfqpoint{1.691762in}{2.971814in}}%
\pgfpathcurveto{\pgfqpoint{1.702812in}{2.971814in}}{\pgfqpoint{1.713411in}{2.976204in}}{\pgfqpoint{1.721225in}{2.984018in}}%
\pgfpathcurveto{\pgfqpoint{1.729038in}{2.991832in}}{\pgfqpoint{1.733428in}{3.002431in}}{\pgfqpoint{1.733428in}{3.013481in}}%
\pgfpathcurveto{\pgfqpoint{1.733428in}{3.024531in}}{\pgfqpoint{1.729038in}{3.035130in}}{\pgfqpoint{1.721225in}{3.042943in}}%
\pgfpathcurveto{\pgfqpoint{1.713411in}{3.050757in}}{\pgfqpoint{1.702812in}{3.055147in}}{\pgfqpoint{1.691762in}{3.055147in}}%
\pgfpathcurveto{\pgfqpoint{1.680712in}{3.055147in}}{\pgfqpoint{1.670113in}{3.050757in}}{\pgfqpoint{1.662299in}{3.042943in}}%
\pgfpathcurveto{\pgfqpoint{1.654485in}{3.035130in}}{\pgfqpoint{1.650095in}{3.024531in}}{\pgfqpoint{1.650095in}{3.013481in}}%
\pgfpathcurveto{\pgfqpoint{1.650095in}{3.002431in}}{\pgfqpoint{1.654485in}{2.991832in}}{\pgfqpoint{1.662299in}{2.984018in}}%
\pgfpathcurveto{\pgfqpoint{1.670113in}{2.976204in}}{\pgfqpoint{1.680712in}{2.971814in}}{\pgfqpoint{1.691762in}{2.971814in}}%
\pgfpathclose%
\pgfusepath{stroke,fill}%
\end{pgfscope}%
\begin{pgfscope}%
\pgfpathrectangle{\pgfqpoint{0.600000in}{0.600000in}}{\pgfqpoint{3.900000in}{3.900000in}}%
\pgfusepath{clip}%
\pgfsetbuttcap%
\pgfsetroundjoin%
\definecolor{currentfill}{rgb}{0.121569,0.466667,0.705882}%
\pgfsetfillcolor{currentfill}%
\pgfsetlinewidth{1.003750pt}%
\definecolor{currentstroke}{rgb}{0.121569,0.466667,0.705882}%
\pgfsetstrokecolor{currentstroke}%
\pgfsetdash{}{0pt}%
\pgfpathmoveto{\pgfqpoint{2.321986in}{1.877375in}}%
\pgfpathcurveto{\pgfqpoint{2.333036in}{1.877375in}}{\pgfqpoint{2.343635in}{1.881765in}}{\pgfqpoint{2.351449in}{1.889579in}}%
\pgfpathcurveto{\pgfqpoint{2.359263in}{1.897392in}}{\pgfqpoint{2.363653in}{1.907991in}}{\pgfqpoint{2.363653in}{1.919041in}}%
\pgfpathcurveto{\pgfqpoint{2.363653in}{1.930091in}}{\pgfqpoint{2.359263in}{1.940691in}}{\pgfqpoint{2.351449in}{1.948504in}}%
\pgfpathcurveto{\pgfqpoint{2.343635in}{1.956318in}}{\pgfqpoint{2.333036in}{1.960708in}}{\pgfqpoint{2.321986in}{1.960708in}}%
\pgfpathcurveto{\pgfqpoint{2.310936in}{1.960708in}}{\pgfqpoint{2.300337in}{1.956318in}}{\pgfqpoint{2.292523in}{1.948504in}}%
\pgfpathcurveto{\pgfqpoint{2.284710in}{1.940691in}}{\pgfqpoint{2.280320in}{1.930091in}}{\pgfqpoint{2.280320in}{1.919041in}}%
\pgfpathcurveto{\pgfqpoint{2.280320in}{1.907991in}}{\pgfqpoint{2.284710in}{1.897392in}}{\pgfqpoint{2.292523in}{1.889579in}}%
\pgfpathcurveto{\pgfqpoint{2.300337in}{1.881765in}}{\pgfqpoint{2.310936in}{1.877375in}}{\pgfqpoint{2.321986in}{1.877375in}}%
\pgfpathclose%
\pgfusepath{stroke,fill}%
\end{pgfscope}%
\begin{pgfscope}%
\pgfpathrectangle{\pgfqpoint{0.600000in}{0.600000in}}{\pgfqpoint{3.900000in}{3.900000in}}%
\pgfusepath{clip}%
\pgfsetbuttcap%
\pgfsetroundjoin%
\definecolor{currentfill}{rgb}{0.121569,0.466667,0.705882}%
\pgfsetfillcolor{currentfill}%
\pgfsetlinewidth{1.003750pt}%
\definecolor{currentstroke}{rgb}{0.121569,0.466667,0.705882}%
\pgfsetstrokecolor{currentstroke}%
\pgfsetdash{}{0pt}%
\pgfpathmoveto{\pgfqpoint{2.601343in}{2.228680in}}%
\pgfpathcurveto{\pgfqpoint{2.612393in}{2.228680in}}{\pgfqpoint{2.622992in}{2.233070in}}{\pgfqpoint{2.630806in}{2.240884in}}%
\pgfpathcurveto{\pgfqpoint{2.638620in}{2.248698in}}{\pgfqpoint{2.643010in}{2.259297in}}{\pgfqpoint{2.643010in}{2.270347in}}%
\pgfpathcurveto{\pgfqpoint{2.643010in}{2.281397in}}{\pgfqpoint{2.638620in}{2.291996in}}{\pgfqpoint{2.630806in}{2.299810in}}%
\pgfpathcurveto{\pgfqpoint{2.622992in}{2.307623in}}{\pgfqpoint{2.612393in}{2.312013in}}{\pgfqpoint{2.601343in}{2.312013in}}%
\pgfpathcurveto{\pgfqpoint{2.590293in}{2.312013in}}{\pgfqpoint{2.579694in}{2.307623in}}{\pgfqpoint{2.571881in}{2.299810in}}%
\pgfpathcurveto{\pgfqpoint{2.564067in}{2.291996in}}{\pgfqpoint{2.559677in}{2.281397in}}{\pgfqpoint{2.559677in}{2.270347in}}%
\pgfpathcurveto{\pgfqpoint{2.559677in}{2.259297in}}{\pgfqpoint{2.564067in}{2.248698in}}{\pgfqpoint{2.571881in}{2.240884in}}%
\pgfpathcurveto{\pgfqpoint{2.579694in}{2.233070in}}{\pgfqpoint{2.590293in}{2.228680in}}{\pgfqpoint{2.601343in}{2.228680in}}%
\pgfpathclose%
\pgfusepath{stroke,fill}%
\end{pgfscope}%
\begin{pgfscope}%
\pgfpathrectangle{\pgfqpoint{0.600000in}{0.600000in}}{\pgfqpoint{3.900000in}{3.900000in}}%
\pgfusepath{clip}%
\pgfsetbuttcap%
\pgfsetroundjoin%
\definecolor{currentfill}{rgb}{0.121569,0.466667,0.705882}%
\pgfsetfillcolor{currentfill}%
\pgfsetlinewidth{1.003750pt}%
\definecolor{currentstroke}{rgb}{0.121569,0.466667,0.705882}%
\pgfsetstrokecolor{currentstroke}%
\pgfsetdash{}{0pt}%
\pgfpathmoveto{\pgfqpoint{1.322666in}{3.110976in}}%
\pgfpathcurveto{\pgfqpoint{1.333716in}{3.110976in}}{\pgfqpoint{1.344315in}{3.115366in}}{\pgfqpoint{1.352129in}{3.123180in}}%
\pgfpathcurveto{\pgfqpoint{1.359942in}{3.130994in}}{\pgfqpoint{1.364333in}{3.141593in}}{\pgfqpoint{1.364333in}{3.152643in}}%
\pgfpathcurveto{\pgfqpoint{1.364333in}{3.163693in}}{\pgfqpoint{1.359942in}{3.174292in}}{\pgfqpoint{1.352129in}{3.182105in}}%
\pgfpathcurveto{\pgfqpoint{1.344315in}{3.189919in}}{\pgfqpoint{1.333716in}{3.194309in}}{\pgfqpoint{1.322666in}{3.194309in}}%
\pgfpathcurveto{\pgfqpoint{1.311616in}{3.194309in}}{\pgfqpoint{1.301017in}{3.189919in}}{\pgfqpoint{1.293203in}{3.182105in}}%
\pgfpathcurveto{\pgfqpoint{1.285390in}{3.174292in}}{\pgfqpoint{1.280999in}{3.163693in}}{\pgfqpoint{1.280999in}{3.152643in}}%
\pgfpathcurveto{\pgfqpoint{1.280999in}{3.141593in}}{\pgfqpoint{1.285390in}{3.130994in}}{\pgfqpoint{1.293203in}{3.123180in}}%
\pgfpathcurveto{\pgfqpoint{1.301017in}{3.115366in}}{\pgfqpoint{1.311616in}{3.110976in}}{\pgfqpoint{1.322666in}{3.110976in}}%
\pgfpathclose%
\pgfusepath{stroke,fill}%
\end{pgfscope}%
\begin{pgfscope}%
\pgfpathrectangle{\pgfqpoint{0.600000in}{0.600000in}}{\pgfqpoint{3.900000in}{3.900000in}}%
\pgfusepath{clip}%
\pgfsetbuttcap%
\pgfsetroundjoin%
\definecolor{currentfill}{rgb}{0.121569,0.466667,0.705882}%
\pgfsetfillcolor{currentfill}%
\pgfsetlinewidth{1.003750pt}%
\definecolor{currentstroke}{rgb}{0.121569,0.466667,0.705882}%
\pgfsetstrokecolor{currentstroke}%
\pgfsetdash{}{0pt}%
\pgfpathmoveto{\pgfqpoint{2.856970in}{1.685690in}}%
\pgfpathcurveto{\pgfqpoint{2.868020in}{1.685690in}}{\pgfqpoint{2.878619in}{1.690080in}}{\pgfqpoint{2.886433in}{1.697894in}}%
\pgfpathcurveto{\pgfqpoint{2.894246in}{1.705707in}}{\pgfqpoint{2.898637in}{1.716306in}}{\pgfqpoint{2.898637in}{1.727357in}}%
\pgfpathcurveto{\pgfqpoint{2.898637in}{1.738407in}}{\pgfqpoint{2.894246in}{1.749006in}}{\pgfqpoint{2.886433in}{1.756819in}}%
\pgfpathcurveto{\pgfqpoint{2.878619in}{1.764633in}}{\pgfqpoint{2.868020in}{1.769023in}}{\pgfqpoint{2.856970in}{1.769023in}}%
\pgfpathcurveto{\pgfqpoint{2.845920in}{1.769023in}}{\pgfqpoint{2.835321in}{1.764633in}}{\pgfqpoint{2.827507in}{1.756819in}}%
\pgfpathcurveto{\pgfqpoint{2.819694in}{1.749006in}}{\pgfqpoint{2.815303in}{1.738407in}}{\pgfqpoint{2.815303in}{1.727357in}}%
\pgfpathcurveto{\pgfqpoint{2.815303in}{1.716306in}}{\pgfqpoint{2.819694in}{1.705707in}}{\pgfqpoint{2.827507in}{1.697894in}}%
\pgfpathcurveto{\pgfqpoint{2.835321in}{1.690080in}}{\pgfqpoint{2.845920in}{1.685690in}}{\pgfqpoint{2.856970in}{1.685690in}}%
\pgfpathclose%
\pgfusepath{stroke,fill}%
\end{pgfscope}%
\begin{pgfscope}%
\pgfpathrectangle{\pgfqpoint{0.600000in}{0.600000in}}{\pgfqpoint{3.900000in}{3.900000in}}%
\pgfusepath{clip}%
\pgfsetbuttcap%
\pgfsetroundjoin%
\definecolor{currentfill}{rgb}{0.121569,0.466667,0.705882}%
\pgfsetfillcolor{currentfill}%
\pgfsetlinewidth{1.003750pt}%
\definecolor{currentstroke}{rgb}{0.121569,0.466667,0.705882}%
\pgfsetstrokecolor{currentstroke}%
\pgfsetdash{}{0pt}%
\pgfpathmoveto{\pgfqpoint{2.798220in}{1.453959in}}%
\pgfpathcurveto{\pgfqpoint{2.809270in}{1.453959in}}{\pgfqpoint{2.819869in}{1.458350in}}{\pgfqpoint{2.827683in}{1.466163in}}%
\pgfpathcurveto{\pgfqpoint{2.835496in}{1.473977in}}{\pgfqpoint{2.839886in}{1.484576in}}{\pgfqpoint{2.839886in}{1.495626in}}%
\pgfpathcurveto{\pgfqpoint{2.839886in}{1.506676in}}{\pgfqpoint{2.835496in}{1.517275in}}{\pgfqpoint{2.827683in}{1.525089in}}%
\pgfpathcurveto{\pgfqpoint{2.819869in}{1.532902in}}{\pgfqpoint{2.809270in}{1.537293in}}{\pgfqpoint{2.798220in}{1.537293in}}%
\pgfpathcurveto{\pgfqpoint{2.787170in}{1.537293in}}{\pgfqpoint{2.776571in}{1.532902in}}{\pgfqpoint{2.768757in}{1.525089in}}%
\pgfpathcurveto{\pgfqpoint{2.760943in}{1.517275in}}{\pgfqpoint{2.756553in}{1.506676in}}{\pgfqpoint{2.756553in}{1.495626in}}%
\pgfpathcurveto{\pgfqpoint{2.756553in}{1.484576in}}{\pgfqpoint{2.760943in}{1.473977in}}{\pgfqpoint{2.768757in}{1.466163in}}%
\pgfpathcurveto{\pgfqpoint{2.776571in}{1.458350in}}{\pgfqpoint{2.787170in}{1.453959in}}{\pgfqpoint{2.798220in}{1.453959in}}%
\pgfpathclose%
\pgfusepath{stroke,fill}%
\end{pgfscope}%
\begin{pgfscope}%
\pgfpathrectangle{\pgfqpoint{0.600000in}{0.600000in}}{\pgfqpoint{3.900000in}{3.900000in}}%
\pgfusepath{clip}%
\pgfsetbuttcap%
\pgfsetroundjoin%
\definecolor{currentfill}{rgb}{0.121569,0.466667,0.705882}%
\pgfsetfillcolor{currentfill}%
\pgfsetlinewidth{1.003750pt}%
\definecolor{currentstroke}{rgb}{0.121569,0.466667,0.705882}%
\pgfsetstrokecolor{currentstroke}%
\pgfsetdash{}{0pt}%
\pgfpathmoveto{\pgfqpoint{2.734048in}{2.746033in}}%
\pgfpathcurveto{\pgfqpoint{2.745098in}{2.746033in}}{\pgfqpoint{2.755697in}{2.750423in}}{\pgfqpoint{2.763510in}{2.758237in}}%
\pgfpathcurveto{\pgfqpoint{2.771324in}{2.766050in}}{\pgfqpoint{2.775714in}{2.776649in}}{\pgfqpoint{2.775714in}{2.787699in}}%
\pgfpathcurveto{\pgfqpoint{2.775714in}{2.798750in}}{\pgfqpoint{2.771324in}{2.809349in}}{\pgfqpoint{2.763510in}{2.817162in}}%
\pgfpathcurveto{\pgfqpoint{2.755697in}{2.824976in}}{\pgfqpoint{2.745098in}{2.829366in}}{\pgfqpoint{2.734048in}{2.829366in}}%
\pgfpathcurveto{\pgfqpoint{2.722997in}{2.829366in}}{\pgfqpoint{2.712398in}{2.824976in}}{\pgfqpoint{2.704585in}{2.817162in}}%
\pgfpathcurveto{\pgfqpoint{2.696771in}{2.809349in}}{\pgfqpoint{2.692381in}{2.798750in}}{\pgfqpoint{2.692381in}{2.787699in}}%
\pgfpathcurveto{\pgfqpoint{2.692381in}{2.776649in}}{\pgfqpoint{2.696771in}{2.766050in}}{\pgfqpoint{2.704585in}{2.758237in}}%
\pgfpathcurveto{\pgfqpoint{2.712398in}{2.750423in}}{\pgfqpoint{2.722997in}{2.746033in}}{\pgfqpoint{2.734048in}{2.746033in}}%
\pgfpathclose%
\pgfusepath{stroke,fill}%
\end{pgfscope}%
\begin{pgfscope}%
\pgfpathrectangle{\pgfqpoint{0.600000in}{0.600000in}}{\pgfqpoint{3.900000in}{3.900000in}}%
\pgfusepath{clip}%
\pgfsetbuttcap%
\pgfsetroundjoin%
\definecolor{currentfill}{rgb}{0.121569,0.466667,0.705882}%
\pgfsetfillcolor{currentfill}%
\pgfsetlinewidth{1.003750pt}%
\definecolor{currentstroke}{rgb}{0.121569,0.466667,0.705882}%
\pgfsetstrokecolor{currentstroke}%
\pgfsetdash{}{0pt}%
\pgfpathmoveto{\pgfqpoint{2.432338in}{3.255832in}}%
\pgfpathcurveto{\pgfqpoint{2.443388in}{3.255832in}}{\pgfqpoint{2.453987in}{3.260222in}}{\pgfqpoint{2.461801in}{3.268036in}}%
\pgfpathcurveto{\pgfqpoint{2.469615in}{3.275849in}}{\pgfqpoint{2.474005in}{3.286448in}}{\pgfqpoint{2.474005in}{3.297499in}}%
\pgfpathcurveto{\pgfqpoint{2.474005in}{3.308549in}}{\pgfqpoint{2.469615in}{3.319148in}}{\pgfqpoint{2.461801in}{3.326961in}}%
\pgfpathcurveto{\pgfqpoint{2.453987in}{3.334775in}}{\pgfqpoint{2.443388in}{3.339165in}}{\pgfqpoint{2.432338in}{3.339165in}}%
\pgfpathcurveto{\pgfqpoint{2.421288in}{3.339165in}}{\pgfqpoint{2.410689in}{3.334775in}}{\pgfqpoint{2.402875in}{3.326961in}}%
\pgfpathcurveto{\pgfqpoint{2.395062in}{3.319148in}}{\pgfqpoint{2.390671in}{3.308549in}}{\pgfqpoint{2.390671in}{3.297499in}}%
\pgfpathcurveto{\pgfqpoint{2.390671in}{3.286448in}}{\pgfqpoint{2.395062in}{3.275849in}}{\pgfqpoint{2.402875in}{3.268036in}}%
\pgfpathcurveto{\pgfqpoint{2.410689in}{3.260222in}}{\pgfqpoint{2.421288in}{3.255832in}}{\pgfqpoint{2.432338in}{3.255832in}}%
\pgfpathclose%
\pgfusepath{stroke,fill}%
\end{pgfscope}%
\begin{pgfscope}%
\pgfpathrectangle{\pgfqpoint{0.600000in}{0.600000in}}{\pgfqpoint{3.900000in}{3.900000in}}%
\pgfusepath{clip}%
\pgfsetbuttcap%
\pgfsetroundjoin%
\definecolor{currentfill}{rgb}{0.121569,0.466667,0.705882}%
\pgfsetfillcolor{currentfill}%
\pgfsetlinewidth{1.003750pt}%
\definecolor{currentstroke}{rgb}{0.121569,0.466667,0.705882}%
\pgfsetstrokecolor{currentstroke}%
\pgfsetdash{}{0pt}%
\pgfpathmoveto{\pgfqpoint{3.751370in}{1.783971in}}%
\pgfpathcurveto{\pgfqpoint{3.762420in}{1.783971in}}{\pgfqpoint{3.773019in}{1.788361in}}{\pgfqpoint{3.780832in}{1.796174in}}%
\pgfpathcurveto{\pgfqpoint{3.788646in}{1.803988in}}{\pgfqpoint{3.793036in}{1.814587in}}{\pgfqpoint{3.793036in}{1.825637in}}%
\pgfpathcurveto{\pgfqpoint{3.793036in}{1.836687in}}{\pgfqpoint{3.788646in}{1.847286in}}{\pgfqpoint{3.780832in}{1.855100in}}%
\pgfpathcurveto{\pgfqpoint{3.773019in}{1.862914in}}{\pgfqpoint{3.762420in}{1.867304in}}{\pgfqpoint{3.751370in}{1.867304in}}%
\pgfpathcurveto{\pgfqpoint{3.740319in}{1.867304in}}{\pgfqpoint{3.729720in}{1.862914in}}{\pgfqpoint{3.721907in}{1.855100in}}%
\pgfpathcurveto{\pgfqpoint{3.714093in}{1.847286in}}{\pgfqpoint{3.709703in}{1.836687in}}{\pgfqpoint{3.709703in}{1.825637in}}%
\pgfpathcurveto{\pgfqpoint{3.709703in}{1.814587in}}{\pgfqpoint{3.714093in}{1.803988in}}{\pgfqpoint{3.721907in}{1.796174in}}%
\pgfpathcurveto{\pgfqpoint{3.729720in}{1.788361in}}{\pgfqpoint{3.740319in}{1.783971in}}{\pgfqpoint{3.751370in}{1.783971in}}%
\pgfpathclose%
\pgfusepath{stroke,fill}%
\end{pgfscope}%
\begin{pgfscope}%
\pgfpathrectangle{\pgfqpoint{0.600000in}{0.600000in}}{\pgfqpoint{3.900000in}{3.900000in}}%
\pgfusepath{clip}%
\pgfsetbuttcap%
\pgfsetroundjoin%
\definecolor{currentfill}{rgb}{0.121569,0.466667,0.705882}%
\pgfsetfillcolor{currentfill}%
\pgfsetlinewidth{1.003750pt}%
\definecolor{currentstroke}{rgb}{0.121569,0.466667,0.705882}%
\pgfsetstrokecolor{currentstroke}%
\pgfsetdash{}{0pt}%
\pgfpathmoveto{\pgfqpoint{2.993524in}{2.198624in}}%
\pgfpathcurveto{\pgfqpoint{3.004575in}{2.198624in}}{\pgfqpoint{3.015174in}{2.203015in}}{\pgfqpoint{3.022987in}{2.210828in}}%
\pgfpathcurveto{\pgfqpoint{3.030801in}{2.218642in}}{\pgfqpoint{3.035191in}{2.229241in}}{\pgfqpoint{3.035191in}{2.240291in}}%
\pgfpathcurveto{\pgfqpoint{3.035191in}{2.251341in}}{\pgfqpoint{3.030801in}{2.261940in}}{\pgfqpoint{3.022987in}{2.269754in}}%
\pgfpathcurveto{\pgfqpoint{3.015174in}{2.277567in}}{\pgfqpoint{3.004575in}{2.281958in}}{\pgfqpoint{2.993524in}{2.281958in}}%
\pgfpathcurveto{\pgfqpoint{2.982474in}{2.281958in}}{\pgfqpoint{2.971875in}{2.277567in}}{\pgfqpoint{2.964062in}{2.269754in}}%
\pgfpathcurveto{\pgfqpoint{2.956248in}{2.261940in}}{\pgfqpoint{2.951858in}{2.251341in}}{\pgfqpoint{2.951858in}{2.240291in}}%
\pgfpathcurveto{\pgfqpoint{2.951858in}{2.229241in}}{\pgfqpoint{2.956248in}{2.218642in}}{\pgfqpoint{2.964062in}{2.210828in}}%
\pgfpathcurveto{\pgfqpoint{2.971875in}{2.203015in}}{\pgfqpoint{2.982474in}{2.198624in}}{\pgfqpoint{2.993524in}{2.198624in}}%
\pgfpathclose%
\pgfusepath{stroke,fill}%
\end{pgfscope}%
\begin{pgfscope}%
\pgfpathrectangle{\pgfqpoint{0.600000in}{0.600000in}}{\pgfqpoint{3.900000in}{3.900000in}}%
\pgfusepath{clip}%
\pgfsetbuttcap%
\pgfsetroundjoin%
\definecolor{currentfill}{rgb}{0.121569,0.466667,0.705882}%
\pgfsetfillcolor{currentfill}%
\pgfsetlinewidth{1.003750pt}%
\definecolor{currentstroke}{rgb}{0.121569,0.466667,0.705882}%
\pgfsetstrokecolor{currentstroke}%
\pgfsetdash{}{0pt}%
\pgfpathmoveto{\pgfqpoint{1.072209in}{2.301013in}}%
\pgfpathcurveto{\pgfqpoint{1.083259in}{2.301013in}}{\pgfqpoint{1.093858in}{2.305404in}}{\pgfqpoint{1.101672in}{2.313217in}}%
\pgfpathcurveto{\pgfqpoint{1.109486in}{2.321031in}}{\pgfqpoint{1.113876in}{2.331630in}}{\pgfqpoint{1.113876in}{2.342680in}}%
\pgfpathcurveto{\pgfqpoint{1.113876in}{2.353730in}}{\pgfqpoint{1.109486in}{2.364329in}}{\pgfqpoint{1.101672in}{2.372143in}}%
\pgfpathcurveto{\pgfqpoint{1.093858in}{2.379956in}}{\pgfqpoint{1.083259in}{2.384347in}}{\pgfqpoint{1.072209in}{2.384347in}}%
\pgfpathcurveto{\pgfqpoint{1.061159in}{2.384347in}}{\pgfqpoint{1.050560in}{2.379956in}}{\pgfqpoint{1.042746in}{2.372143in}}%
\pgfpathcurveto{\pgfqpoint{1.034933in}{2.364329in}}{\pgfqpoint{1.030543in}{2.353730in}}{\pgfqpoint{1.030543in}{2.342680in}}%
\pgfpathcurveto{\pgfqpoint{1.030543in}{2.331630in}}{\pgfqpoint{1.034933in}{2.321031in}}{\pgfqpoint{1.042746in}{2.313217in}}%
\pgfpathcurveto{\pgfqpoint{1.050560in}{2.305404in}}{\pgfqpoint{1.061159in}{2.301013in}}{\pgfqpoint{1.072209in}{2.301013in}}%
\pgfpathclose%
\pgfusepath{stroke,fill}%
\end{pgfscope}%
\begin{pgfscope}%
\pgfpathrectangle{\pgfqpoint{0.600000in}{0.600000in}}{\pgfqpoint{3.900000in}{3.900000in}}%
\pgfusepath{clip}%
\pgfsetbuttcap%
\pgfsetroundjoin%
\definecolor{currentfill}{rgb}{0.121569,0.466667,0.705882}%
\pgfsetfillcolor{currentfill}%
\pgfsetlinewidth{1.003750pt}%
\definecolor{currentstroke}{rgb}{0.121569,0.466667,0.705882}%
\pgfsetstrokecolor{currentstroke}%
\pgfsetdash{}{0pt}%
\pgfpathmoveto{\pgfqpoint{2.616894in}{2.358522in}}%
\pgfpathcurveto{\pgfqpoint{2.627944in}{2.358522in}}{\pgfqpoint{2.638543in}{2.362913in}}{\pgfqpoint{2.646356in}{2.370726in}}%
\pgfpathcurveto{\pgfqpoint{2.654170in}{2.378540in}}{\pgfqpoint{2.658560in}{2.389139in}}{\pgfqpoint{2.658560in}{2.400189in}}%
\pgfpathcurveto{\pgfqpoint{2.658560in}{2.411239in}}{\pgfqpoint{2.654170in}{2.421838in}}{\pgfqpoint{2.646356in}{2.429652in}}%
\pgfpathcurveto{\pgfqpoint{2.638543in}{2.437466in}}{\pgfqpoint{2.627944in}{2.441856in}}{\pgfqpoint{2.616894in}{2.441856in}}%
\pgfpathcurveto{\pgfqpoint{2.605843in}{2.441856in}}{\pgfqpoint{2.595244in}{2.437466in}}{\pgfqpoint{2.587431in}{2.429652in}}%
\pgfpathcurveto{\pgfqpoint{2.579617in}{2.421838in}}{\pgfqpoint{2.575227in}{2.411239in}}{\pgfqpoint{2.575227in}{2.400189in}}%
\pgfpathcurveto{\pgfqpoint{2.575227in}{2.389139in}}{\pgfqpoint{2.579617in}{2.378540in}}{\pgfqpoint{2.587431in}{2.370726in}}%
\pgfpathcurveto{\pgfqpoint{2.595244in}{2.362913in}}{\pgfqpoint{2.605843in}{2.358522in}}{\pgfqpoint{2.616894in}{2.358522in}}%
\pgfpathclose%
\pgfusepath{stroke,fill}%
\end{pgfscope}%
\begin{pgfscope}%
\pgfpathrectangle{\pgfqpoint{0.600000in}{0.600000in}}{\pgfqpoint{3.900000in}{3.900000in}}%
\pgfusepath{clip}%
\pgfsetbuttcap%
\pgfsetroundjoin%
\definecolor{currentfill}{rgb}{0.121569,0.466667,0.705882}%
\pgfsetfillcolor{currentfill}%
\pgfsetlinewidth{1.003750pt}%
\definecolor{currentstroke}{rgb}{0.121569,0.466667,0.705882}%
\pgfsetstrokecolor{currentstroke}%
\pgfsetdash{}{0pt}%
\pgfpathmoveto{\pgfqpoint{2.238886in}{1.840085in}}%
\pgfpathcurveto{\pgfqpoint{2.249936in}{1.840085in}}{\pgfqpoint{2.260535in}{1.844476in}}{\pgfqpoint{2.268348in}{1.852289in}}%
\pgfpathcurveto{\pgfqpoint{2.276162in}{1.860103in}}{\pgfqpoint{2.280552in}{1.870702in}}{\pgfqpoint{2.280552in}{1.881752in}}%
\pgfpathcurveto{\pgfqpoint{2.280552in}{1.892802in}}{\pgfqpoint{2.276162in}{1.903401in}}{\pgfqpoint{2.268348in}{1.911215in}}%
\pgfpathcurveto{\pgfqpoint{2.260535in}{1.919028in}}{\pgfqpoint{2.249936in}{1.923419in}}{\pgfqpoint{2.238886in}{1.923419in}}%
\pgfpathcurveto{\pgfqpoint{2.227836in}{1.923419in}}{\pgfqpoint{2.217237in}{1.919028in}}{\pgfqpoint{2.209423in}{1.911215in}}%
\pgfpathcurveto{\pgfqpoint{2.201609in}{1.903401in}}{\pgfqpoint{2.197219in}{1.892802in}}{\pgfqpoint{2.197219in}{1.881752in}}%
\pgfpathcurveto{\pgfqpoint{2.197219in}{1.870702in}}{\pgfqpoint{2.201609in}{1.860103in}}{\pgfqpoint{2.209423in}{1.852289in}}%
\pgfpathcurveto{\pgfqpoint{2.217237in}{1.844476in}}{\pgfqpoint{2.227836in}{1.840085in}}{\pgfqpoint{2.238886in}{1.840085in}}%
\pgfpathclose%
\pgfusepath{stroke,fill}%
\end{pgfscope}%
\begin{pgfscope}%
\pgfpathrectangle{\pgfqpoint{0.600000in}{0.600000in}}{\pgfqpoint{3.900000in}{3.900000in}}%
\pgfusepath{clip}%
\pgfsetbuttcap%
\pgfsetroundjoin%
\definecolor{currentfill}{rgb}{0.121569,0.466667,0.705882}%
\pgfsetfillcolor{currentfill}%
\pgfsetlinewidth{1.003750pt}%
\definecolor{currentstroke}{rgb}{0.121569,0.466667,0.705882}%
\pgfsetstrokecolor{currentstroke}%
\pgfsetdash{}{0pt}%
\pgfpathmoveto{\pgfqpoint{2.641550in}{3.081496in}}%
\pgfpathcurveto{\pgfqpoint{2.652600in}{3.081496in}}{\pgfqpoint{2.663199in}{3.085886in}}{\pgfqpoint{2.671013in}{3.093700in}}%
\pgfpathcurveto{\pgfqpoint{2.678827in}{3.101513in}}{\pgfqpoint{2.683217in}{3.112112in}}{\pgfqpoint{2.683217in}{3.123162in}}%
\pgfpathcurveto{\pgfqpoint{2.683217in}{3.134212in}}{\pgfqpoint{2.678827in}{3.144812in}}{\pgfqpoint{2.671013in}{3.152625in}}%
\pgfpathcurveto{\pgfqpoint{2.663199in}{3.160439in}}{\pgfqpoint{2.652600in}{3.164829in}}{\pgfqpoint{2.641550in}{3.164829in}}%
\pgfpathcurveto{\pgfqpoint{2.630500in}{3.164829in}}{\pgfqpoint{2.619901in}{3.160439in}}{\pgfqpoint{2.612087in}{3.152625in}}%
\pgfpathcurveto{\pgfqpoint{2.604274in}{3.144812in}}{\pgfqpoint{2.599884in}{3.134212in}}{\pgfqpoint{2.599884in}{3.123162in}}%
\pgfpathcurveto{\pgfqpoint{2.599884in}{3.112112in}}{\pgfqpoint{2.604274in}{3.101513in}}{\pgfqpoint{2.612087in}{3.093700in}}%
\pgfpathcurveto{\pgfqpoint{2.619901in}{3.085886in}}{\pgfqpoint{2.630500in}{3.081496in}}{\pgfqpoint{2.641550in}{3.081496in}}%
\pgfpathclose%
\pgfusepath{stroke,fill}%
\end{pgfscope}%
\begin{pgfscope}%
\pgfpathrectangle{\pgfqpoint{0.600000in}{0.600000in}}{\pgfqpoint{3.900000in}{3.900000in}}%
\pgfusepath{clip}%
\pgfsetbuttcap%
\pgfsetroundjoin%
\definecolor{currentfill}{rgb}{0.121569,0.466667,0.705882}%
\pgfsetfillcolor{currentfill}%
\pgfsetlinewidth{1.003750pt}%
\definecolor{currentstroke}{rgb}{0.121569,0.466667,0.705882}%
\pgfsetstrokecolor{currentstroke}%
\pgfsetdash{}{0pt}%
\pgfpathmoveto{\pgfqpoint{1.884972in}{3.022540in}}%
\pgfpathcurveto{\pgfqpoint{1.896022in}{3.022540in}}{\pgfqpoint{1.906621in}{3.026930in}}{\pgfqpoint{1.914435in}{3.034743in}}%
\pgfpathcurveto{\pgfqpoint{1.922249in}{3.042557in}}{\pgfqpoint{1.926639in}{3.053156in}}{\pgfqpoint{1.926639in}{3.064206in}}%
\pgfpathcurveto{\pgfqpoint{1.926639in}{3.075256in}}{\pgfqpoint{1.922249in}{3.085855in}}{\pgfqpoint{1.914435in}{3.093669in}}%
\pgfpathcurveto{\pgfqpoint{1.906621in}{3.101483in}}{\pgfqpoint{1.896022in}{3.105873in}}{\pgfqpoint{1.884972in}{3.105873in}}%
\pgfpathcurveto{\pgfqpoint{1.873922in}{3.105873in}}{\pgfqpoint{1.863323in}{3.101483in}}{\pgfqpoint{1.855509in}{3.093669in}}%
\pgfpathcurveto{\pgfqpoint{1.847696in}{3.085855in}}{\pgfqpoint{1.843306in}{3.075256in}}{\pgfqpoint{1.843306in}{3.064206in}}%
\pgfpathcurveto{\pgfqpoint{1.843306in}{3.053156in}}{\pgfqpoint{1.847696in}{3.042557in}}{\pgfqpoint{1.855509in}{3.034743in}}%
\pgfpathcurveto{\pgfqpoint{1.863323in}{3.026930in}}{\pgfqpoint{1.873922in}{3.022540in}}{\pgfqpoint{1.884972in}{3.022540in}}%
\pgfpathclose%
\pgfusepath{stroke,fill}%
\end{pgfscope}%
\begin{pgfscope}%
\pgfpathrectangle{\pgfqpoint{0.600000in}{0.600000in}}{\pgfqpoint{3.900000in}{3.900000in}}%
\pgfusepath{clip}%
\pgfsetbuttcap%
\pgfsetroundjoin%
\definecolor{currentfill}{rgb}{0.121569,0.466667,0.705882}%
\pgfsetfillcolor{currentfill}%
\pgfsetlinewidth{1.003750pt}%
\definecolor{currentstroke}{rgb}{0.121569,0.466667,0.705882}%
\pgfsetstrokecolor{currentstroke}%
\pgfsetdash{}{0pt}%
\pgfpathmoveto{\pgfqpoint{2.375964in}{2.674622in}}%
\pgfpathcurveto{\pgfqpoint{2.387014in}{2.674622in}}{\pgfqpoint{2.397613in}{2.679012in}}{\pgfqpoint{2.405427in}{2.686826in}}%
\pgfpathcurveto{\pgfqpoint{2.413241in}{2.694639in}}{\pgfqpoint{2.417631in}{2.705238in}}{\pgfqpoint{2.417631in}{2.716288in}}%
\pgfpathcurveto{\pgfqpoint{2.417631in}{2.727339in}}{\pgfqpoint{2.413241in}{2.737938in}}{\pgfqpoint{2.405427in}{2.745751in}}%
\pgfpathcurveto{\pgfqpoint{2.397613in}{2.753565in}}{\pgfqpoint{2.387014in}{2.757955in}}{\pgfqpoint{2.375964in}{2.757955in}}%
\pgfpathcurveto{\pgfqpoint{2.364914in}{2.757955in}}{\pgfqpoint{2.354315in}{2.753565in}}{\pgfqpoint{2.346501in}{2.745751in}}%
\pgfpathcurveto{\pgfqpoint{2.338688in}{2.737938in}}{\pgfqpoint{2.334298in}{2.727339in}}{\pgfqpoint{2.334298in}{2.716288in}}%
\pgfpathcurveto{\pgfqpoint{2.334298in}{2.705238in}}{\pgfqpoint{2.338688in}{2.694639in}}{\pgfqpoint{2.346501in}{2.686826in}}%
\pgfpathcurveto{\pgfqpoint{2.354315in}{2.679012in}}{\pgfqpoint{2.364914in}{2.674622in}}{\pgfqpoint{2.375964in}{2.674622in}}%
\pgfpathclose%
\pgfusepath{stroke,fill}%
\end{pgfscope}%
\begin{pgfscope}%
\pgfpathrectangle{\pgfqpoint{0.600000in}{0.600000in}}{\pgfqpoint{3.900000in}{3.900000in}}%
\pgfusepath{clip}%
\pgfsetbuttcap%
\pgfsetroundjoin%
\definecolor{currentfill}{rgb}{0.121569,0.466667,0.705882}%
\pgfsetfillcolor{currentfill}%
\pgfsetlinewidth{1.003750pt}%
\definecolor{currentstroke}{rgb}{0.121569,0.466667,0.705882}%
\pgfsetstrokecolor{currentstroke}%
\pgfsetdash{}{0pt}%
\pgfpathmoveto{\pgfqpoint{1.907941in}{1.941634in}}%
\pgfpathcurveto{\pgfqpoint{1.918991in}{1.941634in}}{\pgfqpoint{1.929590in}{1.946024in}}{\pgfqpoint{1.937404in}{1.953838in}}%
\pgfpathcurveto{\pgfqpoint{1.945217in}{1.961651in}}{\pgfqpoint{1.949607in}{1.972250in}}{\pgfqpoint{1.949607in}{1.983301in}}%
\pgfpathcurveto{\pgfqpoint{1.949607in}{1.994351in}}{\pgfqpoint{1.945217in}{2.004950in}}{\pgfqpoint{1.937404in}{2.012763in}}%
\pgfpathcurveto{\pgfqpoint{1.929590in}{2.020577in}}{\pgfqpoint{1.918991in}{2.024967in}}{\pgfqpoint{1.907941in}{2.024967in}}%
\pgfpathcurveto{\pgfqpoint{1.896891in}{2.024967in}}{\pgfqpoint{1.886292in}{2.020577in}}{\pgfqpoint{1.878478in}{2.012763in}}%
\pgfpathcurveto{\pgfqpoint{1.870664in}{2.004950in}}{\pgfqpoint{1.866274in}{1.994351in}}{\pgfqpoint{1.866274in}{1.983301in}}%
\pgfpathcurveto{\pgfqpoint{1.866274in}{1.972250in}}{\pgfqpoint{1.870664in}{1.961651in}}{\pgfqpoint{1.878478in}{1.953838in}}%
\pgfpathcurveto{\pgfqpoint{1.886292in}{1.946024in}}{\pgfqpoint{1.896891in}{1.941634in}}{\pgfqpoint{1.907941in}{1.941634in}}%
\pgfpathclose%
\pgfusepath{stroke,fill}%
\end{pgfscope}%
\begin{pgfscope}%
\pgfpathrectangle{\pgfqpoint{0.600000in}{0.600000in}}{\pgfqpoint{3.900000in}{3.900000in}}%
\pgfusepath{clip}%
\pgfsetbuttcap%
\pgfsetroundjoin%
\definecolor{currentfill}{rgb}{0.121569,0.466667,0.705882}%
\pgfsetfillcolor{currentfill}%
\pgfsetlinewidth{1.003750pt}%
\definecolor{currentstroke}{rgb}{0.121569,0.466667,0.705882}%
\pgfsetstrokecolor{currentstroke}%
\pgfsetdash{}{0pt}%
\pgfpathmoveto{\pgfqpoint{1.810448in}{2.343380in}}%
\pgfpathcurveto{\pgfqpoint{1.821498in}{2.343380in}}{\pgfqpoint{1.832097in}{2.347770in}}{\pgfqpoint{1.839911in}{2.355584in}}%
\pgfpathcurveto{\pgfqpoint{1.847724in}{2.363397in}}{\pgfqpoint{1.852114in}{2.373996in}}{\pgfqpoint{1.852114in}{2.385046in}}%
\pgfpathcurveto{\pgfqpoint{1.852114in}{2.396096in}}{\pgfqpoint{1.847724in}{2.406695in}}{\pgfqpoint{1.839911in}{2.414509in}}%
\pgfpathcurveto{\pgfqpoint{1.832097in}{2.422323in}}{\pgfqpoint{1.821498in}{2.426713in}}{\pgfqpoint{1.810448in}{2.426713in}}%
\pgfpathcurveto{\pgfqpoint{1.799398in}{2.426713in}}{\pgfqpoint{1.788799in}{2.422323in}}{\pgfqpoint{1.780985in}{2.414509in}}%
\pgfpathcurveto{\pgfqpoint{1.773171in}{2.406695in}}{\pgfqpoint{1.768781in}{2.396096in}}{\pgfqpoint{1.768781in}{2.385046in}}%
\pgfpathcurveto{\pgfqpoint{1.768781in}{2.373996in}}{\pgfqpoint{1.773171in}{2.363397in}}{\pgfqpoint{1.780985in}{2.355584in}}%
\pgfpathcurveto{\pgfqpoint{1.788799in}{2.347770in}}{\pgfqpoint{1.799398in}{2.343380in}}{\pgfqpoint{1.810448in}{2.343380in}}%
\pgfpathclose%
\pgfusepath{stroke,fill}%
\end{pgfscope}%
\begin{pgfscope}%
\pgfpathrectangle{\pgfqpoint{0.600000in}{0.600000in}}{\pgfqpoint{3.900000in}{3.900000in}}%
\pgfusepath{clip}%
\pgfsetbuttcap%
\pgfsetroundjoin%
\definecolor{currentfill}{rgb}{0.121569,0.466667,0.705882}%
\pgfsetfillcolor{currentfill}%
\pgfsetlinewidth{1.003750pt}%
\definecolor{currentstroke}{rgb}{0.121569,0.466667,0.705882}%
\pgfsetstrokecolor{currentstroke}%
\pgfsetdash{}{0pt}%
\pgfpathmoveto{\pgfqpoint{3.391919in}{2.028273in}}%
\pgfpathcurveto{\pgfqpoint{3.402969in}{2.028273in}}{\pgfqpoint{3.413568in}{2.032663in}}{\pgfqpoint{3.421382in}{2.040477in}}%
\pgfpathcurveto{\pgfqpoint{3.429195in}{2.048290in}}{\pgfqpoint{3.433586in}{2.058889in}}{\pgfqpoint{3.433586in}{2.069939in}}%
\pgfpathcurveto{\pgfqpoint{3.433586in}{2.080990in}}{\pgfqpoint{3.429195in}{2.091589in}}{\pgfqpoint{3.421382in}{2.099402in}}%
\pgfpathcurveto{\pgfqpoint{3.413568in}{2.107216in}}{\pgfqpoint{3.402969in}{2.111606in}}{\pgfqpoint{3.391919in}{2.111606in}}%
\pgfpathcurveto{\pgfqpoint{3.380869in}{2.111606in}}{\pgfqpoint{3.370270in}{2.107216in}}{\pgfqpoint{3.362456in}{2.099402in}}%
\pgfpathcurveto{\pgfqpoint{3.354643in}{2.091589in}}{\pgfqpoint{3.350252in}{2.080990in}}{\pgfqpoint{3.350252in}{2.069939in}}%
\pgfpathcurveto{\pgfqpoint{3.350252in}{2.058889in}}{\pgfqpoint{3.354643in}{2.048290in}}{\pgfqpoint{3.362456in}{2.040477in}}%
\pgfpathcurveto{\pgfqpoint{3.370270in}{2.032663in}}{\pgfqpoint{3.380869in}{2.028273in}}{\pgfqpoint{3.391919in}{2.028273in}}%
\pgfpathclose%
\pgfusepath{stroke,fill}%
\end{pgfscope}%
\begin{pgfscope}%
\pgfpathrectangle{\pgfqpoint{0.600000in}{0.600000in}}{\pgfqpoint{3.900000in}{3.900000in}}%
\pgfusepath{clip}%
\pgfsetbuttcap%
\pgfsetroundjoin%
\definecolor{currentfill}{rgb}{0.121569,0.466667,0.705882}%
\pgfsetfillcolor{currentfill}%
\pgfsetlinewidth{1.003750pt}%
\definecolor{currentstroke}{rgb}{0.121569,0.466667,0.705882}%
\pgfsetstrokecolor{currentstroke}%
\pgfsetdash{}{0pt}%
\pgfpathmoveto{\pgfqpoint{3.520903in}{2.311638in}}%
\pgfpathcurveto{\pgfqpoint{3.531953in}{2.311638in}}{\pgfqpoint{3.542552in}{2.316028in}}{\pgfqpoint{3.550366in}{2.323842in}}%
\pgfpathcurveto{\pgfqpoint{3.558179in}{2.331656in}}{\pgfqpoint{3.562570in}{2.342255in}}{\pgfqpoint{3.562570in}{2.353305in}}%
\pgfpathcurveto{\pgfqpoint{3.562570in}{2.364355in}}{\pgfqpoint{3.558179in}{2.374954in}}{\pgfqpoint{3.550366in}{2.382768in}}%
\pgfpathcurveto{\pgfqpoint{3.542552in}{2.390581in}}{\pgfqpoint{3.531953in}{2.394971in}}{\pgfqpoint{3.520903in}{2.394971in}}%
\pgfpathcurveto{\pgfqpoint{3.509853in}{2.394971in}}{\pgfqpoint{3.499254in}{2.390581in}}{\pgfqpoint{3.491440in}{2.382768in}}%
\pgfpathcurveto{\pgfqpoint{3.483627in}{2.374954in}}{\pgfqpoint{3.479236in}{2.364355in}}{\pgfqpoint{3.479236in}{2.353305in}}%
\pgfpathcurveto{\pgfqpoint{3.479236in}{2.342255in}}{\pgfqpoint{3.483627in}{2.331656in}}{\pgfqpoint{3.491440in}{2.323842in}}%
\pgfpathcurveto{\pgfqpoint{3.499254in}{2.316028in}}{\pgfqpoint{3.509853in}{2.311638in}}{\pgfqpoint{3.520903in}{2.311638in}}%
\pgfpathclose%
\pgfusepath{stroke,fill}%
\end{pgfscope}%
\begin{pgfscope}%
\pgfpathrectangle{\pgfqpoint{0.600000in}{0.600000in}}{\pgfqpoint{3.900000in}{3.900000in}}%
\pgfusepath{clip}%
\pgfsetbuttcap%
\pgfsetroundjoin%
\definecolor{currentfill}{rgb}{0.121569,0.466667,0.705882}%
\pgfsetfillcolor{currentfill}%
\pgfsetlinewidth{1.003750pt}%
\definecolor{currentstroke}{rgb}{0.121569,0.466667,0.705882}%
\pgfsetstrokecolor{currentstroke}%
\pgfsetdash{}{0pt}%
\pgfpathmoveto{\pgfqpoint{2.432091in}{2.088231in}}%
\pgfpathcurveto{\pgfqpoint{2.443141in}{2.088231in}}{\pgfqpoint{2.453741in}{2.092621in}}{\pgfqpoint{2.461554in}{2.100435in}}%
\pgfpathcurveto{\pgfqpoint{2.469368in}{2.108249in}}{\pgfqpoint{2.473758in}{2.118848in}}{\pgfqpoint{2.473758in}{2.129898in}}%
\pgfpathcurveto{\pgfqpoint{2.473758in}{2.140948in}}{\pgfqpoint{2.469368in}{2.151547in}}{\pgfqpoint{2.461554in}{2.159361in}}%
\pgfpathcurveto{\pgfqpoint{2.453741in}{2.167174in}}{\pgfqpoint{2.443141in}{2.171564in}}{\pgfqpoint{2.432091in}{2.171564in}}%
\pgfpathcurveto{\pgfqpoint{2.421041in}{2.171564in}}{\pgfqpoint{2.410442in}{2.167174in}}{\pgfqpoint{2.402629in}{2.159361in}}%
\pgfpathcurveto{\pgfqpoint{2.394815in}{2.151547in}}{\pgfqpoint{2.390425in}{2.140948in}}{\pgfqpoint{2.390425in}{2.129898in}}%
\pgfpathcurveto{\pgfqpoint{2.390425in}{2.118848in}}{\pgfqpoint{2.394815in}{2.108249in}}{\pgfqpoint{2.402629in}{2.100435in}}%
\pgfpathcurveto{\pgfqpoint{2.410442in}{2.092621in}}{\pgfqpoint{2.421041in}{2.088231in}}{\pgfqpoint{2.432091in}{2.088231in}}%
\pgfpathclose%
\pgfusepath{stroke,fill}%
\end{pgfscope}%
\begin{pgfscope}%
\pgfpathrectangle{\pgfqpoint{0.600000in}{0.600000in}}{\pgfqpoint{3.900000in}{3.900000in}}%
\pgfusepath{clip}%
\pgfsetbuttcap%
\pgfsetroundjoin%
\definecolor{currentfill}{rgb}{0.121569,0.466667,0.705882}%
\pgfsetfillcolor{currentfill}%
\pgfsetlinewidth{1.003750pt}%
\definecolor{currentstroke}{rgb}{0.121569,0.466667,0.705882}%
\pgfsetstrokecolor{currentstroke}%
\pgfsetdash{}{0pt}%
\pgfpathmoveto{\pgfqpoint{3.213041in}{2.308032in}}%
\pgfpathcurveto{\pgfqpoint{3.224091in}{2.308032in}}{\pgfqpoint{3.234690in}{2.312422in}}{\pgfqpoint{3.242504in}{2.320236in}}%
\pgfpathcurveto{\pgfqpoint{3.250318in}{2.328049in}}{\pgfqpoint{3.254708in}{2.338648in}}{\pgfqpoint{3.254708in}{2.349698in}}%
\pgfpathcurveto{\pgfqpoint{3.254708in}{2.360749in}}{\pgfqpoint{3.250318in}{2.371348in}}{\pgfqpoint{3.242504in}{2.379161in}}%
\pgfpathcurveto{\pgfqpoint{3.234690in}{2.386975in}}{\pgfqpoint{3.224091in}{2.391365in}}{\pgfqpoint{3.213041in}{2.391365in}}%
\pgfpathcurveto{\pgfqpoint{3.201991in}{2.391365in}}{\pgfqpoint{3.191392in}{2.386975in}}{\pgfqpoint{3.183578in}{2.379161in}}%
\pgfpathcurveto{\pgfqpoint{3.175765in}{2.371348in}}{\pgfqpoint{3.171375in}{2.360749in}}{\pgfqpoint{3.171375in}{2.349698in}}%
\pgfpathcurveto{\pgfqpoint{3.171375in}{2.338648in}}{\pgfqpoint{3.175765in}{2.328049in}}{\pgfqpoint{3.183578in}{2.320236in}}%
\pgfpathcurveto{\pgfqpoint{3.191392in}{2.312422in}}{\pgfqpoint{3.201991in}{2.308032in}}{\pgfqpoint{3.213041in}{2.308032in}}%
\pgfpathclose%
\pgfusepath{stroke,fill}%
\end{pgfscope}%
\begin{pgfscope}%
\pgfpathrectangle{\pgfqpoint{0.600000in}{0.600000in}}{\pgfqpoint{3.900000in}{3.900000in}}%
\pgfusepath{clip}%
\pgfsetbuttcap%
\pgfsetroundjoin%
\definecolor{currentfill}{rgb}{0.121569,0.466667,0.705882}%
\pgfsetfillcolor{currentfill}%
\pgfsetlinewidth{1.003750pt}%
\definecolor{currentstroke}{rgb}{0.121569,0.466667,0.705882}%
\pgfsetstrokecolor{currentstroke}%
\pgfsetdash{}{0pt}%
\pgfpathmoveto{\pgfqpoint{2.524600in}{2.532459in}}%
\pgfpathcurveto{\pgfqpoint{2.535651in}{2.532459in}}{\pgfqpoint{2.546250in}{2.536850in}}{\pgfqpoint{2.554063in}{2.544663in}}%
\pgfpathcurveto{\pgfqpoint{2.561877in}{2.552477in}}{\pgfqpoint{2.566267in}{2.563076in}}{\pgfqpoint{2.566267in}{2.574126in}}%
\pgfpathcurveto{\pgfqpoint{2.566267in}{2.585176in}}{\pgfqpoint{2.561877in}{2.595775in}}{\pgfqpoint{2.554063in}{2.603589in}}%
\pgfpathcurveto{\pgfqpoint{2.546250in}{2.611402in}}{\pgfqpoint{2.535651in}{2.615793in}}{\pgfqpoint{2.524600in}{2.615793in}}%
\pgfpathcurveto{\pgfqpoint{2.513550in}{2.615793in}}{\pgfqpoint{2.502951in}{2.611402in}}{\pgfqpoint{2.495138in}{2.603589in}}%
\pgfpathcurveto{\pgfqpoint{2.487324in}{2.595775in}}{\pgfqpoint{2.482934in}{2.585176in}}{\pgfqpoint{2.482934in}{2.574126in}}%
\pgfpathcurveto{\pgfqpoint{2.482934in}{2.563076in}}{\pgfqpoint{2.487324in}{2.552477in}}{\pgfqpoint{2.495138in}{2.544663in}}%
\pgfpathcurveto{\pgfqpoint{2.502951in}{2.536850in}}{\pgfqpoint{2.513550in}{2.532459in}}{\pgfqpoint{2.524600in}{2.532459in}}%
\pgfpathclose%
\pgfusepath{stroke,fill}%
\end{pgfscope}%
\begin{pgfscope}%
\pgfpathrectangle{\pgfqpoint{0.600000in}{0.600000in}}{\pgfqpoint{3.900000in}{3.900000in}}%
\pgfusepath{clip}%
\pgfsetbuttcap%
\pgfsetroundjoin%
\definecolor{currentfill}{rgb}{0.121569,0.466667,0.705882}%
\pgfsetfillcolor{currentfill}%
\pgfsetlinewidth{1.003750pt}%
\definecolor{currentstroke}{rgb}{0.121569,0.466667,0.705882}%
\pgfsetstrokecolor{currentstroke}%
\pgfsetdash{}{0pt}%
\pgfpathmoveto{\pgfqpoint{2.309942in}{2.393970in}}%
\pgfpathcurveto{\pgfqpoint{2.320992in}{2.393970in}}{\pgfqpoint{2.331591in}{2.398361in}}{\pgfqpoint{2.339405in}{2.406174in}}%
\pgfpathcurveto{\pgfqpoint{2.347219in}{2.413988in}}{\pgfqpoint{2.351609in}{2.424587in}}{\pgfqpoint{2.351609in}{2.435637in}}%
\pgfpathcurveto{\pgfqpoint{2.351609in}{2.446687in}}{\pgfqpoint{2.347219in}{2.457286in}}{\pgfqpoint{2.339405in}{2.465100in}}%
\pgfpathcurveto{\pgfqpoint{2.331591in}{2.472913in}}{\pgfqpoint{2.320992in}{2.477304in}}{\pgfqpoint{2.309942in}{2.477304in}}%
\pgfpathcurveto{\pgfqpoint{2.298892in}{2.477304in}}{\pgfqpoint{2.288293in}{2.472913in}}{\pgfqpoint{2.280479in}{2.465100in}}%
\pgfpathcurveto{\pgfqpoint{2.272666in}{2.457286in}}{\pgfqpoint{2.268275in}{2.446687in}}{\pgfqpoint{2.268275in}{2.435637in}}%
\pgfpathcurveto{\pgfqpoint{2.268275in}{2.424587in}}{\pgfqpoint{2.272666in}{2.413988in}}{\pgfqpoint{2.280479in}{2.406174in}}%
\pgfpathcurveto{\pgfqpoint{2.288293in}{2.398361in}}{\pgfqpoint{2.298892in}{2.393970in}}{\pgfqpoint{2.309942in}{2.393970in}}%
\pgfpathclose%
\pgfusepath{stroke,fill}%
\end{pgfscope}%
\begin{pgfscope}%
\pgfpathrectangle{\pgfqpoint{0.600000in}{0.600000in}}{\pgfqpoint{3.900000in}{3.900000in}}%
\pgfusepath{clip}%
\pgfsetbuttcap%
\pgfsetroundjoin%
\definecolor{currentfill}{rgb}{0.121569,0.466667,0.705882}%
\pgfsetfillcolor{currentfill}%
\pgfsetlinewidth{1.003750pt}%
\definecolor{currentstroke}{rgb}{0.121569,0.466667,0.705882}%
\pgfsetstrokecolor{currentstroke}%
\pgfsetdash{}{0pt}%
\pgfpathmoveto{\pgfqpoint{2.297448in}{2.348800in}}%
\pgfpathcurveto{\pgfqpoint{2.308499in}{2.348800in}}{\pgfqpoint{2.319098in}{2.353191in}}{\pgfqpoint{2.326911in}{2.361004in}}%
\pgfpathcurveto{\pgfqpoint{2.334725in}{2.368818in}}{\pgfqpoint{2.339115in}{2.379417in}}{\pgfqpoint{2.339115in}{2.390467in}}%
\pgfpathcurveto{\pgfqpoint{2.339115in}{2.401517in}}{\pgfqpoint{2.334725in}{2.412116in}}{\pgfqpoint{2.326911in}{2.419930in}}%
\pgfpathcurveto{\pgfqpoint{2.319098in}{2.427743in}}{\pgfqpoint{2.308499in}{2.432134in}}{\pgfqpoint{2.297448in}{2.432134in}}%
\pgfpathcurveto{\pgfqpoint{2.286398in}{2.432134in}}{\pgfqpoint{2.275799in}{2.427743in}}{\pgfqpoint{2.267986in}{2.419930in}}%
\pgfpathcurveto{\pgfqpoint{2.260172in}{2.412116in}}{\pgfqpoint{2.255782in}{2.401517in}}{\pgfqpoint{2.255782in}{2.390467in}}%
\pgfpathcurveto{\pgfqpoint{2.255782in}{2.379417in}}{\pgfqpoint{2.260172in}{2.368818in}}{\pgfqpoint{2.267986in}{2.361004in}}%
\pgfpathcurveto{\pgfqpoint{2.275799in}{2.353191in}}{\pgfqpoint{2.286398in}{2.348800in}}{\pgfqpoint{2.297448in}{2.348800in}}%
\pgfpathclose%
\pgfusepath{stroke,fill}%
\end{pgfscope}%
\begin{pgfscope}%
\pgfpathrectangle{\pgfqpoint{0.600000in}{0.600000in}}{\pgfqpoint{3.900000in}{3.900000in}}%
\pgfusepath{clip}%
\pgfsetbuttcap%
\pgfsetroundjoin%
\definecolor{currentfill}{rgb}{0.121569,0.466667,0.705882}%
\pgfsetfillcolor{currentfill}%
\pgfsetlinewidth{1.003750pt}%
\definecolor{currentstroke}{rgb}{0.121569,0.466667,0.705882}%
\pgfsetstrokecolor{currentstroke}%
\pgfsetdash{}{0pt}%
\pgfpathmoveto{\pgfqpoint{2.021154in}{2.556089in}}%
\pgfpathcurveto{\pgfqpoint{2.032204in}{2.556089in}}{\pgfqpoint{2.042803in}{2.560479in}}{\pgfqpoint{2.050617in}{2.568293in}}%
\pgfpathcurveto{\pgfqpoint{2.058430in}{2.576106in}}{\pgfqpoint{2.062821in}{2.586705in}}{\pgfqpoint{2.062821in}{2.597755in}}%
\pgfpathcurveto{\pgfqpoint{2.062821in}{2.608806in}}{\pgfqpoint{2.058430in}{2.619405in}}{\pgfqpoint{2.050617in}{2.627218in}}%
\pgfpathcurveto{\pgfqpoint{2.042803in}{2.635032in}}{\pgfqpoint{2.032204in}{2.639422in}}{\pgfqpoint{2.021154in}{2.639422in}}%
\pgfpathcurveto{\pgfqpoint{2.010104in}{2.639422in}}{\pgfqpoint{1.999505in}{2.635032in}}{\pgfqpoint{1.991691in}{2.627218in}}%
\pgfpathcurveto{\pgfqpoint{1.983878in}{2.619405in}}{\pgfqpoint{1.979487in}{2.608806in}}{\pgfqpoint{1.979487in}{2.597755in}}%
\pgfpathcurveto{\pgfqpoint{1.979487in}{2.586705in}}{\pgfqpoint{1.983878in}{2.576106in}}{\pgfqpoint{1.991691in}{2.568293in}}%
\pgfpathcurveto{\pgfqpoint{1.999505in}{2.560479in}}{\pgfqpoint{2.010104in}{2.556089in}}{\pgfqpoint{2.021154in}{2.556089in}}%
\pgfpathclose%
\pgfusepath{stroke,fill}%
\end{pgfscope}%
\begin{pgfscope}%
\pgfpathrectangle{\pgfqpoint{0.600000in}{0.600000in}}{\pgfqpoint{3.900000in}{3.900000in}}%
\pgfusepath{clip}%
\pgfsetbuttcap%
\pgfsetroundjoin%
\definecolor{currentfill}{rgb}{0.121569,0.466667,0.705882}%
\pgfsetfillcolor{currentfill}%
\pgfsetlinewidth{1.003750pt}%
\definecolor{currentstroke}{rgb}{0.121569,0.466667,0.705882}%
\pgfsetstrokecolor{currentstroke}%
\pgfsetdash{}{0pt}%
\pgfpathmoveto{\pgfqpoint{2.017050in}{2.810886in}}%
\pgfpathcurveto{\pgfqpoint{2.028100in}{2.810886in}}{\pgfqpoint{2.038699in}{2.815276in}}{\pgfqpoint{2.046513in}{2.823089in}}%
\pgfpathcurveto{\pgfqpoint{2.054327in}{2.830903in}}{\pgfqpoint{2.058717in}{2.841502in}}{\pgfqpoint{2.058717in}{2.852552in}}%
\pgfpathcurveto{\pgfqpoint{2.058717in}{2.863602in}}{\pgfqpoint{2.054327in}{2.874201in}}{\pgfqpoint{2.046513in}{2.882015in}}%
\pgfpathcurveto{\pgfqpoint{2.038699in}{2.889829in}}{\pgfqpoint{2.028100in}{2.894219in}}{\pgfqpoint{2.017050in}{2.894219in}}%
\pgfpathcurveto{\pgfqpoint{2.006000in}{2.894219in}}{\pgfqpoint{1.995401in}{2.889829in}}{\pgfqpoint{1.987587in}{2.882015in}}%
\pgfpathcurveto{\pgfqpoint{1.979774in}{2.874201in}}{\pgfqpoint{1.975383in}{2.863602in}}{\pgfqpoint{1.975383in}{2.852552in}}%
\pgfpathcurveto{\pgfqpoint{1.975383in}{2.841502in}}{\pgfqpoint{1.979774in}{2.830903in}}{\pgfqpoint{1.987587in}{2.823089in}}%
\pgfpathcurveto{\pgfqpoint{1.995401in}{2.815276in}}{\pgfqpoint{2.006000in}{2.810886in}}{\pgfqpoint{2.017050in}{2.810886in}}%
\pgfpathclose%
\pgfusepath{stroke,fill}%
\end{pgfscope}%
\begin{pgfscope}%
\pgfpathrectangle{\pgfqpoint{0.600000in}{0.600000in}}{\pgfqpoint{3.900000in}{3.900000in}}%
\pgfusepath{clip}%
\pgfsetbuttcap%
\pgfsetroundjoin%
\definecolor{currentfill}{rgb}{0.121569,0.466667,0.705882}%
\pgfsetfillcolor{currentfill}%
\pgfsetlinewidth{1.003750pt}%
\definecolor{currentstroke}{rgb}{0.121569,0.466667,0.705882}%
\pgfsetstrokecolor{currentstroke}%
\pgfsetdash{}{0pt}%
\pgfpathmoveto{\pgfqpoint{3.192172in}{2.197570in}}%
\pgfpathcurveto{\pgfqpoint{3.203222in}{2.197570in}}{\pgfqpoint{3.213821in}{2.201960in}}{\pgfqpoint{3.221634in}{2.209774in}}%
\pgfpathcurveto{\pgfqpoint{3.229448in}{2.217588in}}{\pgfqpoint{3.233838in}{2.228187in}}{\pgfqpoint{3.233838in}{2.239237in}}%
\pgfpathcurveto{\pgfqpoint{3.233838in}{2.250287in}}{\pgfqpoint{3.229448in}{2.260886in}}{\pgfqpoint{3.221634in}{2.268699in}}%
\pgfpathcurveto{\pgfqpoint{3.213821in}{2.276513in}}{\pgfqpoint{3.203222in}{2.280903in}}{\pgfqpoint{3.192172in}{2.280903in}}%
\pgfpathcurveto{\pgfqpoint{3.181122in}{2.280903in}}{\pgfqpoint{3.170523in}{2.276513in}}{\pgfqpoint{3.162709in}{2.268699in}}%
\pgfpathcurveto{\pgfqpoint{3.154895in}{2.260886in}}{\pgfqpoint{3.150505in}{2.250287in}}{\pgfqpoint{3.150505in}{2.239237in}}%
\pgfpathcurveto{\pgfqpoint{3.150505in}{2.228187in}}{\pgfqpoint{3.154895in}{2.217588in}}{\pgfqpoint{3.162709in}{2.209774in}}%
\pgfpathcurveto{\pgfqpoint{3.170523in}{2.201960in}}{\pgfqpoint{3.181122in}{2.197570in}}{\pgfqpoint{3.192172in}{2.197570in}}%
\pgfpathclose%
\pgfusepath{stroke,fill}%
\end{pgfscope}%
\begin{pgfscope}%
\pgfpathrectangle{\pgfqpoint{0.600000in}{0.600000in}}{\pgfqpoint{3.900000in}{3.900000in}}%
\pgfusepath{clip}%
\pgfsetbuttcap%
\pgfsetroundjoin%
\definecolor{currentfill}{rgb}{0.121569,0.466667,0.705882}%
\pgfsetfillcolor{currentfill}%
\pgfsetlinewidth{1.003750pt}%
\definecolor{currentstroke}{rgb}{0.121569,0.466667,0.705882}%
\pgfsetstrokecolor{currentstroke}%
\pgfsetdash{}{0pt}%
\pgfpathmoveto{\pgfqpoint{2.976045in}{2.545361in}}%
\pgfpathcurveto{\pgfqpoint{2.987096in}{2.545361in}}{\pgfqpoint{2.997695in}{2.549751in}}{\pgfqpoint{3.005508in}{2.557565in}}%
\pgfpathcurveto{\pgfqpoint{3.013322in}{2.565379in}}{\pgfqpoint{3.017712in}{2.575978in}}{\pgfqpoint{3.017712in}{2.587028in}}%
\pgfpathcurveto{\pgfqpoint{3.017712in}{2.598078in}}{\pgfqpoint{3.013322in}{2.608677in}}{\pgfqpoint{3.005508in}{2.616491in}}%
\pgfpathcurveto{\pgfqpoint{2.997695in}{2.624304in}}{\pgfqpoint{2.987096in}{2.628695in}}{\pgfqpoint{2.976045in}{2.628695in}}%
\pgfpathcurveto{\pgfqpoint{2.964995in}{2.628695in}}{\pgfqpoint{2.954396in}{2.624304in}}{\pgfqpoint{2.946583in}{2.616491in}}%
\pgfpathcurveto{\pgfqpoint{2.938769in}{2.608677in}}{\pgfqpoint{2.934379in}{2.598078in}}{\pgfqpoint{2.934379in}{2.587028in}}%
\pgfpathcurveto{\pgfqpoint{2.934379in}{2.575978in}}{\pgfqpoint{2.938769in}{2.565379in}}{\pgfqpoint{2.946583in}{2.557565in}}%
\pgfpathcurveto{\pgfqpoint{2.954396in}{2.549751in}}{\pgfqpoint{2.964995in}{2.545361in}}{\pgfqpoint{2.976045in}{2.545361in}}%
\pgfpathclose%
\pgfusepath{stroke,fill}%
\end{pgfscope}%
\begin{pgfscope}%
\pgfpathrectangle{\pgfqpoint{0.600000in}{0.600000in}}{\pgfqpoint{3.900000in}{3.900000in}}%
\pgfusepath{clip}%
\pgfsetbuttcap%
\pgfsetroundjoin%
\definecolor{currentfill}{rgb}{0.121569,0.466667,0.705882}%
\pgfsetfillcolor{currentfill}%
\pgfsetlinewidth{1.003750pt}%
\definecolor{currentstroke}{rgb}{0.121569,0.466667,0.705882}%
\pgfsetstrokecolor{currentstroke}%
\pgfsetdash{}{0pt}%
\pgfpathmoveto{\pgfqpoint{2.618654in}{1.655176in}}%
\pgfpathcurveto{\pgfqpoint{2.629704in}{1.655176in}}{\pgfqpoint{2.640303in}{1.659566in}}{\pgfqpoint{2.648117in}{1.667380in}}%
\pgfpathcurveto{\pgfqpoint{2.655930in}{1.675194in}}{\pgfqpoint{2.660321in}{1.685793in}}{\pgfqpoint{2.660321in}{1.696843in}}%
\pgfpathcurveto{\pgfqpoint{2.660321in}{1.707893in}}{\pgfqpoint{2.655930in}{1.718492in}}{\pgfqpoint{2.648117in}{1.726306in}}%
\pgfpathcurveto{\pgfqpoint{2.640303in}{1.734119in}}{\pgfqpoint{2.629704in}{1.738509in}}{\pgfqpoint{2.618654in}{1.738509in}}%
\pgfpathcurveto{\pgfqpoint{2.607604in}{1.738509in}}{\pgfqpoint{2.597005in}{1.734119in}}{\pgfqpoint{2.589191in}{1.726306in}}%
\pgfpathcurveto{\pgfqpoint{2.581378in}{1.718492in}}{\pgfqpoint{2.576987in}{1.707893in}}{\pgfqpoint{2.576987in}{1.696843in}}%
\pgfpathcurveto{\pgfqpoint{2.576987in}{1.685793in}}{\pgfqpoint{2.581378in}{1.675194in}}{\pgfqpoint{2.589191in}{1.667380in}}%
\pgfpathcurveto{\pgfqpoint{2.597005in}{1.659566in}}{\pgfqpoint{2.607604in}{1.655176in}}{\pgfqpoint{2.618654in}{1.655176in}}%
\pgfpathclose%
\pgfusepath{stroke,fill}%
\end{pgfscope}%
\begin{pgfscope}%
\pgfpathrectangle{\pgfqpoint{0.600000in}{0.600000in}}{\pgfqpoint{3.900000in}{3.900000in}}%
\pgfusepath{clip}%
\pgfsetbuttcap%
\pgfsetroundjoin%
\definecolor{currentfill}{rgb}{0.121569,0.466667,0.705882}%
\pgfsetfillcolor{currentfill}%
\pgfsetlinewidth{1.003750pt}%
\definecolor{currentstroke}{rgb}{0.121569,0.466667,0.705882}%
\pgfsetstrokecolor{currentstroke}%
\pgfsetdash{}{0pt}%
\pgfpathmoveto{\pgfqpoint{2.559844in}{2.622178in}}%
\pgfpathcurveto{\pgfqpoint{2.570894in}{2.622178in}}{\pgfqpoint{2.581493in}{2.626568in}}{\pgfqpoint{2.589307in}{2.634382in}}%
\pgfpathcurveto{\pgfqpoint{2.597121in}{2.642196in}}{\pgfqpoint{2.601511in}{2.652795in}}{\pgfqpoint{2.601511in}{2.663845in}}%
\pgfpathcurveto{\pgfqpoint{2.601511in}{2.674895in}}{\pgfqpoint{2.597121in}{2.685494in}}{\pgfqpoint{2.589307in}{2.693307in}}%
\pgfpathcurveto{\pgfqpoint{2.581493in}{2.701121in}}{\pgfqpoint{2.570894in}{2.705511in}}{\pgfqpoint{2.559844in}{2.705511in}}%
\pgfpathcurveto{\pgfqpoint{2.548794in}{2.705511in}}{\pgfqpoint{2.538195in}{2.701121in}}{\pgfqpoint{2.530382in}{2.693307in}}%
\pgfpathcurveto{\pgfqpoint{2.522568in}{2.685494in}}{\pgfqpoint{2.518178in}{2.674895in}}{\pgfqpoint{2.518178in}{2.663845in}}%
\pgfpathcurveto{\pgfqpoint{2.518178in}{2.652795in}}{\pgfqpoint{2.522568in}{2.642196in}}{\pgfqpoint{2.530382in}{2.634382in}}%
\pgfpathcurveto{\pgfqpoint{2.538195in}{2.626568in}}{\pgfqpoint{2.548794in}{2.622178in}}{\pgfqpoint{2.559844in}{2.622178in}}%
\pgfpathclose%
\pgfusepath{stroke,fill}%
\end{pgfscope}%
\begin{pgfscope}%
\pgfpathrectangle{\pgfqpoint{0.600000in}{0.600000in}}{\pgfqpoint{3.900000in}{3.900000in}}%
\pgfusepath{clip}%
\pgfsetbuttcap%
\pgfsetroundjoin%
\definecolor{currentfill}{rgb}{0.121569,0.466667,0.705882}%
\pgfsetfillcolor{currentfill}%
\pgfsetlinewidth{1.003750pt}%
\definecolor{currentstroke}{rgb}{0.121569,0.466667,0.705882}%
\pgfsetstrokecolor{currentstroke}%
\pgfsetdash{}{0pt}%
\pgfpathmoveto{\pgfqpoint{2.581714in}{2.681993in}}%
\pgfpathcurveto{\pgfqpoint{2.592764in}{2.681993in}}{\pgfqpoint{2.603363in}{2.686384in}}{\pgfqpoint{2.611177in}{2.694197in}}%
\pgfpathcurveto{\pgfqpoint{2.618991in}{2.702011in}}{\pgfqpoint{2.623381in}{2.712610in}}{\pgfqpoint{2.623381in}{2.723660in}}%
\pgfpathcurveto{\pgfqpoint{2.623381in}{2.734710in}}{\pgfqpoint{2.618991in}{2.745309in}}{\pgfqpoint{2.611177in}{2.753123in}}%
\pgfpathcurveto{\pgfqpoint{2.603363in}{2.760936in}}{\pgfqpoint{2.592764in}{2.765327in}}{\pgfqpoint{2.581714in}{2.765327in}}%
\pgfpathcurveto{\pgfqpoint{2.570664in}{2.765327in}}{\pgfqpoint{2.560065in}{2.760936in}}{\pgfqpoint{2.552251in}{2.753123in}}%
\pgfpathcurveto{\pgfqpoint{2.544438in}{2.745309in}}{\pgfqpoint{2.540047in}{2.734710in}}{\pgfqpoint{2.540047in}{2.723660in}}%
\pgfpathcurveto{\pgfqpoint{2.540047in}{2.712610in}}{\pgfqpoint{2.544438in}{2.702011in}}{\pgfqpoint{2.552251in}{2.694197in}}%
\pgfpathcurveto{\pgfqpoint{2.560065in}{2.686384in}}{\pgfqpoint{2.570664in}{2.681993in}}{\pgfqpoint{2.581714in}{2.681993in}}%
\pgfpathclose%
\pgfusepath{stroke,fill}%
\end{pgfscope}%
\begin{pgfscope}%
\pgfpathrectangle{\pgfqpoint{0.600000in}{0.600000in}}{\pgfqpoint{3.900000in}{3.900000in}}%
\pgfusepath{clip}%
\pgfsetbuttcap%
\pgfsetroundjoin%
\definecolor{currentfill}{rgb}{0.121569,0.466667,0.705882}%
\pgfsetfillcolor{currentfill}%
\pgfsetlinewidth{1.003750pt}%
\definecolor{currentstroke}{rgb}{0.121569,0.466667,0.705882}%
\pgfsetstrokecolor{currentstroke}%
\pgfsetdash{}{0pt}%
\pgfpathmoveto{\pgfqpoint{2.230551in}{4.026383in}}%
\pgfpathcurveto{\pgfqpoint{2.241601in}{4.026383in}}{\pgfqpoint{2.252200in}{4.030773in}}{\pgfqpoint{2.260014in}{4.038587in}}%
\pgfpathcurveto{\pgfqpoint{2.267827in}{4.046400in}}{\pgfqpoint{2.272217in}{4.056999in}}{\pgfqpoint{2.272217in}{4.068049in}}%
\pgfpathcurveto{\pgfqpoint{2.272217in}{4.079100in}}{\pgfqpoint{2.267827in}{4.089699in}}{\pgfqpoint{2.260014in}{4.097512in}}%
\pgfpathcurveto{\pgfqpoint{2.252200in}{4.105326in}}{\pgfqpoint{2.241601in}{4.109716in}}{\pgfqpoint{2.230551in}{4.109716in}}%
\pgfpathcurveto{\pgfqpoint{2.219501in}{4.109716in}}{\pgfqpoint{2.208902in}{4.105326in}}{\pgfqpoint{2.201088in}{4.097512in}}%
\pgfpathcurveto{\pgfqpoint{2.193274in}{4.089699in}}{\pgfqpoint{2.188884in}{4.079100in}}{\pgfqpoint{2.188884in}{4.068049in}}%
\pgfpathcurveto{\pgfqpoint{2.188884in}{4.056999in}}{\pgfqpoint{2.193274in}{4.046400in}}{\pgfqpoint{2.201088in}{4.038587in}}%
\pgfpathcurveto{\pgfqpoint{2.208902in}{4.030773in}}{\pgfqpoint{2.219501in}{4.026383in}}{\pgfqpoint{2.230551in}{4.026383in}}%
\pgfpathclose%
\pgfusepath{stroke,fill}%
\end{pgfscope}%
\begin{pgfscope}%
\pgfpathrectangle{\pgfqpoint{0.600000in}{0.600000in}}{\pgfqpoint{3.900000in}{3.900000in}}%
\pgfusepath{clip}%
\pgfsetbuttcap%
\pgfsetroundjoin%
\definecolor{currentfill}{rgb}{0.121569,0.466667,0.705882}%
\pgfsetfillcolor{currentfill}%
\pgfsetlinewidth{1.003750pt}%
\definecolor{currentstroke}{rgb}{0.121569,0.466667,0.705882}%
\pgfsetstrokecolor{currentstroke}%
\pgfsetdash{}{0pt}%
\pgfpathmoveto{\pgfqpoint{3.083890in}{2.278008in}}%
\pgfpathcurveto{\pgfqpoint{3.094940in}{2.278008in}}{\pgfqpoint{3.105539in}{2.282398in}}{\pgfqpoint{3.113353in}{2.290212in}}%
\pgfpathcurveto{\pgfqpoint{3.121166in}{2.298025in}}{\pgfqpoint{3.125557in}{2.308624in}}{\pgfqpoint{3.125557in}{2.319674in}}%
\pgfpathcurveto{\pgfqpoint{3.125557in}{2.330725in}}{\pgfqpoint{3.121166in}{2.341324in}}{\pgfqpoint{3.113353in}{2.349137in}}%
\pgfpathcurveto{\pgfqpoint{3.105539in}{2.356951in}}{\pgfqpoint{3.094940in}{2.361341in}}{\pgfqpoint{3.083890in}{2.361341in}}%
\pgfpathcurveto{\pgfqpoint{3.072840in}{2.361341in}}{\pgfqpoint{3.062241in}{2.356951in}}{\pgfqpoint{3.054427in}{2.349137in}}%
\pgfpathcurveto{\pgfqpoint{3.046613in}{2.341324in}}{\pgfqpoint{3.042223in}{2.330725in}}{\pgfqpoint{3.042223in}{2.319674in}}%
\pgfpathcurveto{\pgfqpoint{3.042223in}{2.308624in}}{\pgfqpoint{3.046613in}{2.298025in}}{\pgfqpoint{3.054427in}{2.290212in}}%
\pgfpathcurveto{\pgfqpoint{3.062241in}{2.282398in}}{\pgfqpoint{3.072840in}{2.278008in}}{\pgfqpoint{3.083890in}{2.278008in}}%
\pgfpathclose%
\pgfusepath{stroke,fill}%
\end{pgfscope}%
\begin{pgfscope}%
\pgfpathrectangle{\pgfqpoint{0.600000in}{0.600000in}}{\pgfqpoint{3.900000in}{3.900000in}}%
\pgfusepath{clip}%
\pgfsetbuttcap%
\pgfsetroundjoin%
\definecolor{currentfill}{rgb}{0.121569,0.466667,0.705882}%
\pgfsetfillcolor{currentfill}%
\pgfsetlinewidth{1.003750pt}%
\definecolor{currentstroke}{rgb}{0.121569,0.466667,0.705882}%
\pgfsetstrokecolor{currentstroke}%
\pgfsetdash{}{0pt}%
\pgfpathmoveto{\pgfqpoint{2.632162in}{2.869917in}}%
\pgfpathcurveto{\pgfqpoint{2.643212in}{2.869917in}}{\pgfqpoint{2.653811in}{2.874307in}}{\pgfqpoint{2.661625in}{2.882121in}}%
\pgfpathcurveto{\pgfqpoint{2.669439in}{2.889935in}}{\pgfqpoint{2.673829in}{2.900534in}}{\pgfqpoint{2.673829in}{2.911584in}}%
\pgfpathcurveto{\pgfqpoint{2.673829in}{2.922634in}}{\pgfqpoint{2.669439in}{2.933233in}}{\pgfqpoint{2.661625in}{2.941047in}}%
\pgfpathcurveto{\pgfqpoint{2.653811in}{2.948860in}}{\pgfqpoint{2.643212in}{2.953251in}}{\pgfqpoint{2.632162in}{2.953251in}}%
\pgfpathcurveto{\pgfqpoint{2.621112in}{2.953251in}}{\pgfqpoint{2.610513in}{2.948860in}}{\pgfqpoint{2.602699in}{2.941047in}}%
\pgfpathcurveto{\pgfqpoint{2.594886in}{2.933233in}}{\pgfqpoint{2.590495in}{2.922634in}}{\pgfqpoint{2.590495in}{2.911584in}}%
\pgfpathcurveto{\pgfqpoint{2.590495in}{2.900534in}}{\pgfqpoint{2.594886in}{2.889935in}}{\pgfqpoint{2.602699in}{2.882121in}}%
\pgfpathcurveto{\pgfqpoint{2.610513in}{2.874307in}}{\pgfqpoint{2.621112in}{2.869917in}}{\pgfqpoint{2.632162in}{2.869917in}}%
\pgfpathclose%
\pgfusepath{stroke,fill}%
\end{pgfscope}%
\begin{pgfscope}%
\pgfpathrectangle{\pgfqpoint{0.600000in}{0.600000in}}{\pgfqpoint{3.900000in}{3.900000in}}%
\pgfusepath{clip}%
\pgfsetbuttcap%
\pgfsetroundjoin%
\definecolor{currentfill}{rgb}{0.121569,0.466667,0.705882}%
\pgfsetfillcolor{currentfill}%
\pgfsetlinewidth{1.003750pt}%
\definecolor{currentstroke}{rgb}{0.121569,0.466667,0.705882}%
\pgfsetstrokecolor{currentstroke}%
\pgfsetdash{}{0pt}%
\pgfpathmoveto{\pgfqpoint{2.742873in}{1.891377in}}%
\pgfpathcurveto{\pgfqpoint{2.753923in}{1.891377in}}{\pgfqpoint{2.764522in}{1.895767in}}{\pgfqpoint{2.772335in}{1.903581in}}%
\pgfpathcurveto{\pgfqpoint{2.780149in}{1.911395in}}{\pgfqpoint{2.784539in}{1.921994in}}{\pgfqpoint{2.784539in}{1.933044in}}%
\pgfpathcurveto{\pgfqpoint{2.784539in}{1.944094in}}{\pgfqpoint{2.780149in}{1.954693in}}{\pgfqpoint{2.772335in}{1.962507in}}%
\pgfpathcurveto{\pgfqpoint{2.764522in}{1.970320in}}{\pgfqpoint{2.753923in}{1.974710in}}{\pgfqpoint{2.742873in}{1.974710in}}%
\pgfpathcurveto{\pgfqpoint{2.731823in}{1.974710in}}{\pgfqpoint{2.721223in}{1.970320in}}{\pgfqpoint{2.713410in}{1.962507in}}%
\pgfpathcurveto{\pgfqpoint{2.705596in}{1.954693in}}{\pgfqpoint{2.701206in}{1.944094in}}{\pgfqpoint{2.701206in}{1.933044in}}%
\pgfpathcurveto{\pgfqpoint{2.701206in}{1.921994in}}{\pgfqpoint{2.705596in}{1.911395in}}{\pgfqpoint{2.713410in}{1.903581in}}%
\pgfpathcurveto{\pgfqpoint{2.721223in}{1.895767in}}{\pgfqpoint{2.731823in}{1.891377in}}{\pgfqpoint{2.742873in}{1.891377in}}%
\pgfpathclose%
\pgfusepath{stroke,fill}%
\end{pgfscope}%
\begin{pgfscope}%
\pgfpathrectangle{\pgfqpoint{0.600000in}{0.600000in}}{\pgfqpoint{3.900000in}{3.900000in}}%
\pgfusepath{clip}%
\pgfsetbuttcap%
\pgfsetroundjoin%
\definecolor{currentfill}{rgb}{0.121569,0.466667,0.705882}%
\pgfsetfillcolor{currentfill}%
\pgfsetlinewidth{1.003750pt}%
\definecolor{currentstroke}{rgb}{0.121569,0.466667,0.705882}%
\pgfsetstrokecolor{currentstroke}%
\pgfsetdash{}{0pt}%
\pgfpathmoveto{\pgfqpoint{2.632389in}{3.106774in}}%
\pgfpathcurveto{\pgfqpoint{2.643439in}{3.106774in}}{\pgfqpoint{2.654038in}{3.111164in}}{\pgfqpoint{2.661852in}{3.118978in}}%
\pgfpathcurveto{\pgfqpoint{2.669665in}{3.126792in}}{\pgfqpoint{2.674056in}{3.137391in}}{\pgfqpoint{2.674056in}{3.148441in}}%
\pgfpathcurveto{\pgfqpoint{2.674056in}{3.159491in}}{\pgfqpoint{2.669665in}{3.170090in}}{\pgfqpoint{2.661852in}{3.177904in}}%
\pgfpathcurveto{\pgfqpoint{2.654038in}{3.185717in}}{\pgfqpoint{2.643439in}{3.190107in}}{\pgfqpoint{2.632389in}{3.190107in}}%
\pgfpathcurveto{\pgfqpoint{2.621339in}{3.190107in}}{\pgfqpoint{2.610740in}{3.185717in}}{\pgfqpoint{2.602926in}{3.177904in}}%
\pgfpathcurveto{\pgfqpoint{2.595113in}{3.170090in}}{\pgfqpoint{2.590722in}{3.159491in}}{\pgfqpoint{2.590722in}{3.148441in}}%
\pgfpathcurveto{\pgfqpoint{2.590722in}{3.137391in}}{\pgfqpoint{2.595113in}{3.126792in}}{\pgfqpoint{2.602926in}{3.118978in}}%
\pgfpathcurveto{\pgfqpoint{2.610740in}{3.111164in}}{\pgfqpoint{2.621339in}{3.106774in}}{\pgfqpoint{2.632389in}{3.106774in}}%
\pgfpathclose%
\pgfusepath{stroke,fill}%
\end{pgfscope}%
\begin{pgfscope}%
\pgfpathrectangle{\pgfqpoint{0.600000in}{0.600000in}}{\pgfqpoint{3.900000in}{3.900000in}}%
\pgfusepath{clip}%
\pgfsetbuttcap%
\pgfsetroundjoin%
\definecolor{currentfill}{rgb}{0.121569,0.466667,0.705882}%
\pgfsetfillcolor{currentfill}%
\pgfsetlinewidth{1.003750pt}%
\definecolor{currentstroke}{rgb}{0.121569,0.466667,0.705882}%
\pgfsetstrokecolor{currentstroke}%
\pgfsetdash{}{0pt}%
\pgfpathmoveto{\pgfqpoint{2.932037in}{1.475886in}}%
\pgfpathcurveto{\pgfqpoint{2.943087in}{1.475886in}}{\pgfqpoint{2.953686in}{1.480277in}}{\pgfqpoint{2.961500in}{1.488090in}}%
\pgfpathcurveto{\pgfqpoint{2.969313in}{1.495904in}}{\pgfqpoint{2.973703in}{1.506503in}}{\pgfqpoint{2.973703in}{1.517553in}}%
\pgfpathcurveto{\pgfqpoint{2.973703in}{1.528603in}}{\pgfqpoint{2.969313in}{1.539202in}}{\pgfqpoint{2.961500in}{1.547016in}}%
\pgfpathcurveto{\pgfqpoint{2.953686in}{1.554829in}}{\pgfqpoint{2.943087in}{1.559220in}}{\pgfqpoint{2.932037in}{1.559220in}}%
\pgfpathcurveto{\pgfqpoint{2.920987in}{1.559220in}}{\pgfqpoint{2.910388in}{1.554829in}}{\pgfqpoint{2.902574in}{1.547016in}}%
\pgfpathcurveto{\pgfqpoint{2.894760in}{1.539202in}}{\pgfqpoint{2.890370in}{1.528603in}}{\pgfqpoint{2.890370in}{1.517553in}}%
\pgfpathcurveto{\pgfqpoint{2.890370in}{1.506503in}}{\pgfqpoint{2.894760in}{1.495904in}}{\pgfqpoint{2.902574in}{1.488090in}}%
\pgfpathcurveto{\pgfqpoint{2.910388in}{1.480277in}}{\pgfqpoint{2.920987in}{1.475886in}}{\pgfqpoint{2.932037in}{1.475886in}}%
\pgfpathclose%
\pgfusepath{stroke,fill}%
\end{pgfscope}%
\begin{pgfscope}%
\pgfpathrectangle{\pgfqpoint{0.600000in}{0.600000in}}{\pgfqpoint{3.900000in}{3.900000in}}%
\pgfusepath{clip}%
\pgfsetbuttcap%
\pgfsetroundjoin%
\definecolor{currentfill}{rgb}{0.121569,0.466667,0.705882}%
\pgfsetfillcolor{currentfill}%
\pgfsetlinewidth{1.003750pt}%
\definecolor{currentstroke}{rgb}{0.121569,0.466667,0.705882}%
\pgfsetstrokecolor{currentstroke}%
\pgfsetdash{}{0pt}%
\pgfpathmoveto{\pgfqpoint{2.746429in}{3.402874in}}%
\pgfpathcurveto{\pgfqpoint{2.757480in}{3.402874in}}{\pgfqpoint{2.768079in}{3.407264in}}{\pgfqpoint{2.775892in}{3.415077in}}%
\pgfpathcurveto{\pgfqpoint{2.783706in}{3.422891in}}{\pgfqpoint{2.788096in}{3.433490in}}{\pgfqpoint{2.788096in}{3.444540in}}%
\pgfpathcurveto{\pgfqpoint{2.788096in}{3.455590in}}{\pgfqpoint{2.783706in}{3.466189in}}{\pgfqpoint{2.775892in}{3.474003in}}%
\pgfpathcurveto{\pgfqpoint{2.768079in}{3.481817in}}{\pgfqpoint{2.757480in}{3.486207in}}{\pgfqpoint{2.746429in}{3.486207in}}%
\pgfpathcurveto{\pgfqpoint{2.735379in}{3.486207in}}{\pgfqpoint{2.724780in}{3.481817in}}{\pgfqpoint{2.716967in}{3.474003in}}%
\pgfpathcurveto{\pgfqpoint{2.709153in}{3.466189in}}{\pgfqpoint{2.704763in}{3.455590in}}{\pgfqpoint{2.704763in}{3.444540in}}%
\pgfpathcurveto{\pgfqpoint{2.704763in}{3.433490in}}{\pgfqpoint{2.709153in}{3.422891in}}{\pgfqpoint{2.716967in}{3.415077in}}%
\pgfpathcurveto{\pgfqpoint{2.724780in}{3.407264in}}{\pgfqpoint{2.735379in}{3.402874in}}{\pgfqpoint{2.746429in}{3.402874in}}%
\pgfpathclose%
\pgfusepath{stroke,fill}%
\end{pgfscope}%
\begin{pgfscope}%
\pgfpathrectangle{\pgfqpoint{0.600000in}{0.600000in}}{\pgfqpoint{3.900000in}{3.900000in}}%
\pgfusepath{clip}%
\pgfsetbuttcap%
\pgfsetroundjoin%
\definecolor{currentfill}{rgb}{0.121569,0.466667,0.705882}%
\pgfsetfillcolor{currentfill}%
\pgfsetlinewidth{1.003750pt}%
\definecolor{currentstroke}{rgb}{0.121569,0.466667,0.705882}%
\pgfsetstrokecolor{currentstroke}%
\pgfsetdash{}{0pt}%
\pgfpathmoveto{\pgfqpoint{2.879393in}{2.091128in}}%
\pgfpathcurveto{\pgfqpoint{2.890443in}{2.091128in}}{\pgfqpoint{2.901042in}{2.095518in}}{\pgfqpoint{2.908856in}{2.103332in}}%
\pgfpathcurveto{\pgfqpoint{2.916669in}{2.111146in}}{\pgfqpoint{2.921060in}{2.121745in}}{\pgfqpoint{2.921060in}{2.132795in}}%
\pgfpathcurveto{\pgfqpoint{2.921060in}{2.143845in}}{\pgfqpoint{2.916669in}{2.154444in}}{\pgfqpoint{2.908856in}{2.162257in}}%
\pgfpathcurveto{\pgfqpoint{2.901042in}{2.170071in}}{\pgfqpoint{2.890443in}{2.174461in}}{\pgfqpoint{2.879393in}{2.174461in}}%
\pgfpathcurveto{\pgfqpoint{2.868343in}{2.174461in}}{\pgfqpoint{2.857744in}{2.170071in}}{\pgfqpoint{2.849930in}{2.162257in}}%
\pgfpathcurveto{\pgfqpoint{2.842117in}{2.154444in}}{\pgfqpoint{2.837726in}{2.143845in}}{\pgfqpoint{2.837726in}{2.132795in}}%
\pgfpathcurveto{\pgfqpoint{2.837726in}{2.121745in}}{\pgfqpoint{2.842117in}{2.111146in}}{\pgfqpoint{2.849930in}{2.103332in}}%
\pgfpathcurveto{\pgfqpoint{2.857744in}{2.095518in}}{\pgfqpoint{2.868343in}{2.091128in}}{\pgfqpoint{2.879393in}{2.091128in}}%
\pgfpathclose%
\pgfusepath{stroke,fill}%
\end{pgfscope}%
\begin{pgfscope}%
\pgfpathrectangle{\pgfqpoint{0.600000in}{0.600000in}}{\pgfqpoint{3.900000in}{3.900000in}}%
\pgfusepath{clip}%
\pgfsetbuttcap%
\pgfsetroundjoin%
\definecolor{currentfill}{rgb}{0.121569,0.466667,0.705882}%
\pgfsetfillcolor{currentfill}%
\pgfsetlinewidth{1.003750pt}%
\definecolor{currentstroke}{rgb}{0.121569,0.466667,0.705882}%
\pgfsetstrokecolor{currentstroke}%
\pgfsetdash{}{0pt}%
\pgfpathmoveto{\pgfqpoint{1.789493in}{2.805721in}}%
\pgfpathcurveto{\pgfqpoint{1.800543in}{2.805721in}}{\pgfqpoint{1.811142in}{2.810111in}}{\pgfqpoint{1.818956in}{2.817925in}}%
\pgfpathcurveto{\pgfqpoint{1.826769in}{2.825739in}}{\pgfqpoint{1.831159in}{2.836338in}}{\pgfqpoint{1.831159in}{2.847388in}}%
\pgfpathcurveto{\pgfqpoint{1.831159in}{2.858438in}}{\pgfqpoint{1.826769in}{2.869037in}}{\pgfqpoint{1.818956in}{2.876851in}}%
\pgfpathcurveto{\pgfqpoint{1.811142in}{2.884664in}}{\pgfqpoint{1.800543in}{2.889055in}}{\pgfqpoint{1.789493in}{2.889055in}}%
\pgfpathcurveto{\pgfqpoint{1.778443in}{2.889055in}}{\pgfqpoint{1.767844in}{2.884664in}}{\pgfqpoint{1.760030in}{2.876851in}}%
\pgfpathcurveto{\pgfqpoint{1.752216in}{2.869037in}}{\pgfqpoint{1.747826in}{2.858438in}}{\pgfqpoint{1.747826in}{2.847388in}}%
\pgfpathcurveto{\pgfqpoint{1.747826in}{2.836338in}}{\pgfqpoint{1.752216in}{2.825739in}}{\pgfqpoint{1.760030in}{2.817925in}}%
\pgfpathcurveto{\pgfqpoint{1.767844in}{2.810111in}}{\pgfqpoint{1.778443in}{2.805721in}}{\pgfqpoint{1.789493in}{2.805721in}}%
\pgfpathclose%
\pgfusepath{stroke,fill}%
\end{pgfscope}%
\begin{pgfscope}%
\pgfpathrectangle{\pgfqpoint{0.600000in}{0.600000in}}{\pgfqpoint{3.900000in}{3.900000in}}%
\pgfusepath{clip}%
\pgfsetbuttcap%
\pgfsetroundjoin%
\definecolor{currentfill}{rgb}{0.121569,0.466667,0.705882}%
\pgfsetfillcolor{currentfill}%
\pgfsetlinewidth{1.003750pt}%
\definecolor{currentstroke}{rgb}{0.121569,0.466667,0.705882}%
\pgfsetstrokecolor{currentstroke}%
\pgfsetdash{}{0pt}%
\pgfpathmoveto{\pgfqpoint{3.539075in}{2.096863in}}%
\pgfpathcurveto{\pgfqpoint{3.550125in}{2.096863in}}{\pgfqpoint{3.560725in}{2.101254in}}{\pgfqpoint{3.568538in}{2.109067in}}%
\pgfpathcurveto{\pgfqpoint{3.576352in}{2.116881in}}{\pgfqpoint{3.580742in}{2.127480in}}{\pgfqpoint{3.580742in}{2.138530in}}%
\pgfpathcurveto{\pgfqpoint{3.580742in}{2.149580in}}{\pgfqpoint{3.576352in}{2.160179in}}{\pgfqpoint{3.568538in}{2.167993in}}%
\pgfpathcurveto{\pgfqpoint{3.560725in}{2.175806in}}{\pgfqpoint{3.550125in}{2.180197in}}{\pgfqpoint{3.539075in}{2.180197in}}%
\pgfpathcurveto{\pgfqpoint{3.528025in}{2.180197in}}{\pgfqpoint{3.517426in}{2.175806in}}{\pgfqpoint{3.509613in}{2.167993in}}%
\pgfpathcurveto{\pgfqpoint{3.501799in}{2.160179in}}{\pgfqpoint{3.497409in}{2.149580in}}{\pgfqpoint{3.497409in}{2.138530in}}%
\pgfpathcurveto{\pgfqpoint{3.497409in}{2.127480in}}{\pgfqpoint{3.501799in}{2.116881in}}{\pgfqpoint{3.509613in}{2.109067in}}%
\pgfpathcurveto{\pgfqpoint{3.517426in}{2.101254in}}{\pgfqpoint{3.528025in}{2.096863in}}{\pgfqpoint{3.539075in}{2.096863in}}%
\pgfpathclose%
\pgfusepath{stroke,fill}%
\end{pgfscope}%
\begin{pgfscope}%
\pgfpathrectangle{\pgfqpoint{0.600000in}{0.600000in}}{\pgfqpoint{3.900000in}{3.900000in}}%
\pgfusepath{clip}%
\pgfsetbuttcap%
\pgfsetroundjoin%
\definecolor{currentfill}{rgb}{0.121569,0.466667,0.705882}%
\pgfsetfillcolor{currentfill}%
\pgfsetlinewidth{1.003750pt}%
\definecolor{currentstroke}{rgb}{0.121569,0.466667,0.705882}%
\pgfsetstrokecolor{currentstroke}%
\pgfsetdash{}{0pt}%
\pgfpathmoveto{\pgfqpoint{2.346651in}{2.411449in}}%
\pgfpathcurveto{\pgfqpoint{2.357701in}{2.411449in}}{\pgfqpoint{2.368300in}{2.415839in}}{\pgfqpoint{2.376114in}{2.423653in}}%
\pgfpathcurveto{\pgfqpoint{2.383927in}{2.431467in}}{\pgfqpoint{2.388317in}{2.442066in}}{\pgfqpoint{2.388317in}{2.453116in}}%
\pgfpathcurveto{\pgfqpoint{2.388317in}{2.464166in}}{\pgfqpoint{2.383927in}{2.474765in}}{\pgfqpoint{2.376114in}{2.482579in}}%
\pgfpathcurveto{\pgfqpoint{2.368300in}{2.490392in}}{\pgfqpoint{2.357701in}{2.494783in}}{\pgfqpoint{2.346651in}{2.494783in}}%
\pgfpathcurveto{\pgfqpoint{2.335601in}{2.494783in}}{\pgfqpoint{2.325002in}{2.490392in}}{\pgfqpoint{2.317188in}{2.482579in}}%
\pgfpathcurveto{\pgfqpoint{2.309374in}{2.474765in}}{\pgfqpoint{2.304984in}{2.464166in}}{\pgfqpoint{2.304984in}{2.453116in}}%
\pgfpathcurveto{\pgfqpoint{2.304984in}{2.442066in}}{\pgfqpoint{2.309374in}{2.431467in}}{\pgfqpoint{2.317188in}{2.423653in}}%
\pgfpathcurveto{\pgfqpoint{2.325002in}{2.415839in}}{\pgfqpoint{2.335601in}{2.411449in}}{\pgfqpoint{2.346651in}{2.411449in}}%
\pgfpathclose%
\pgfusepath{stroke,fill}%
\end{pgfscope}%
\begin{pgfscope}%
\pgfpathrectangle{\pgfqpoint{0.600000in}{0.600000in}}{\pgfqpoint{3.900000in}{3.900000in}}%
\pgfusepath{clip}%
\pgfsetbuttcap%
\pgfsetroundjoin%
\definecolor{currentfill}{rgb}{0.121569,0.466667,0.705882}%
\pgfsetfillcolor{currentfill}%
\pgfsetlinewidth{1.003750pt}%
\definecolor{currentstroke}{rgb}{0.121569,0.466667,0.705882}%
\pgfsetstrokecolor{currentstroke}%
\pgfsetdash{}{0pt}%
\pgfpathmoveto{\pgfqpoint{2.611831in}{1.945863in}}%
\pgfpathcurveto{\pgfqpoint{2.622881in}{1.945863in}}{\pgfqpoint{2.633480in}{1.950253in}}{\pgfqpoint{2.641294in}{1.958067in}}%
\pgfpathcurveto{\pgfqpoint{2.649107in}{1.965881in}}{\pgfqpoint{2.653498in}{1.976480in}}{\pgfqpoint{2.653498in}{1.987530in}}%
\pgfpathcurveto{\pgfqpoint{2.653498in}{1.998580in}}{\pgfqpoint{2.649107in}{2.009179in}}{\pgfqpoint{2.641294in}{2.016993in}}%
\pgfpathcurveto{\pgfqpoint{2.633480in}{2.024806in}}{\pgfqpoint{2.622881in}{2.029196in}}{\pgfqpoint{2.611831in}{2.029196in}}%
\pgfpathcurveto{\pgfqpoint{2.600781in}{2.029196in}}{\pgfqpoint{2.590182in}{2.024806in}}{\pgfqpoint{2.582368in}{2.016993in}}%
\pgfpathcurveto{\pgfqpoint{2.574555in}{2.009179in}}{\pgfqpoint{2.570164in}{1.998580in}}{\pgfqpoint{2.570164in}{1.987530in}}%
\pgfpathcurveto{\pgfqpoint{2.570164in}{1.976480in}}{\pgfqpoint{2.574555in}{1.965881in}}{\pgfqpoint{2.582368in}{1.958067in}}%
\pgfpathcurveto{\pgfqpoint{2.590182in}{1.950253in}}{\pgfqpoint{2.600781in}{1.945863in}}{\pgfqpoint{2.611831in}{1.945863in}}%
\pgfpathclose%
\pgfusepath{stroke,fill}%
\end{pgfscope}%
\begin{pgfscope}%
\pgfpathrectangle{\pgfqpoint{0.600000in}{0.600000in}}{\pgfqpoint{3.900000in}{3.900000in}}%
\pgfusepath{clip}%
\pgfsetbuttcap%
\pgfsetroundjoin%
\definecolor{currentfill}{rgb}{0.121569,0.466667,0.705882}%
\pgfsetfillcolor{currentfill}%
\pgfsetlinewidth{1.003750pt}%
\definecolor{currentstroke}{rgb}{0.121569,0.466667,0.705882}%
\pgfsetstrokecolor{currentstroke}%
\pgfsetdash{}{0pt}%
\pgfpathmoveto{\pgfqpoint{3.232235in}{2.299296in}}%
\pgfpathcurveto{\pgfqpoint{3.243285in}{2.299296in}}{\pgfqpoint{3.253884in}{2.303686in}}{\pgfqpoint{3.261698in}{2.311500in}}%
\pgfpathcurveto{\pgfqpoint{3.269511in}{2.319313in}}{\pgfqpoint{3.273902in}{2.329912in}}{\pgfqpoint{3.273902in}{2.340962in}}%
\pgfpathcurveto{\pgfqpoint{3.273902in}{2.352013in}}{\pgfqpoint{3.269511in}{2.362612in}}{\pgfqpoint{3.261698in}{2.370425in}}%
\pgfpathcurveto{\pgfqpoint{3.253884in}{2.378239in}}{\pgfqpoint{3.243285in}{2.382629in}}{\pgfqpoint{3.232235in}{2.382629in}}%
\pgfpathcurveto{\pgfqpoint{3.221185in}{2.382629in}}{\pgfqpoint{3.210586in}{2.378239in}}{\pgfqpoint{3.202772in}{2.370425in}}%
\pgfpathcurveto{\pgfqpoint{3.194959in}{2.362612in}}{\pgfqpoint{3.190568in}{2.352013in}}{\pgfqpoint{3.190568in}{2.340962in}}%
\pgfpathcurveto{\pgfqpoint{3.190568in}{2.329912in}}{\pgfqpoint{3.194959in}{2.319313in}}{\pgfqpoint{3.202772in}{2.311500in}}%
\pgfpathcurveto{\pgfqpoint{3.210586in}{2.303686in}}{\pgfqpoint{3.221185in}{2.299296in}}{\pgfqpoint{3.232235in}{2.299296in}}%
\pgfpathclose%
\pgfusepath{stroke,fill}%
\end{pgfscope}%
\begin{pgfscope}%
\pgfpathrectangle{\pgfqpoint{0.600000in}{0.600000in}}{\pgfqpoint{3.900000in}{3.900000in}}%
\pgfusepath{clip}%
\pgfsetbuttcap%
\pgfsetroundjoin%
\definecolor{currentfill}{rgb}{0.121569,0.466667,0.705882}%
\pgfsetfillcolor{currentfill}%
\pgfsetlinewidth{1.003750pt}%
\definecolor{currentstroke}{rgb}{0.121569,0.466667,0.705882}%
\pgfsetstrokecolor{currentstroke}%
\pgfsetdash{}{0pt}%
\pgfpathmoveto{\pgfqpoint{2.347151in}{2.280815in}}%
\pgfpathcurveto{\pgfqpoint{2.358201in}{2.280815in}}{\pgfqpoint{2.368800in}{2.285205in}}{\pgfqpoint{2.376614in}{2.293019in}}%
\pgfpathcurveto{\pgfqpoint{2.384428in}{2.300833in}}{\pgfqpoint{2.388818in}{2.311432in}}{\pgfqpoint{2.388818in}{2.322482in}}%
\pgfpathcurveto{\pgfqpoint{2.388818in}{2.333532in}}{\pgfqpoint{2.384428in}{2.344131in}}{\pgfqpoint{2.376614in}{2.351945in}}%
\pgfpathcurveto{\pgfqpoint{2.368800in}{2.359758in}}{\pgfqpoint{2.358201in}{2.364149in}}{\pgfqpoint{2.347151in}{2.364149in}}%
\pgfpathcurveto{\pgfqpoint{2.336101in}{2.364149in}}{\pgfqpoint{2.325502in}{2.359758in}}{\pgfqpoint{2.317688in}{2.351945in}}%
\pgfpathcurveto{\pgfqpoint{2.309875in}{2.344131in}}{\pgfqpoint{2.305484in}{2.333532in}}{\pgfqpoint{2.305484in}{2.322482in}}%
\pgfpathcurveto{\pgfqpoint{2.305484in}{2.311432in}}{\pgfqpoint{2.309875in}{2.300833in}}{\pgfqpoint{2.317688in}{2.293019in}}%
\pgfpathcurveto{\pgfqpoint{2.325502in}{2.285205in}}{\pgfqpoint{2.336101in}{2.280815in}}{\pgfqpoint{2.347151in}{2.280815in}}%
\pgfpathclose%
\pgfusepath{stroke,fill}%
\end{pgfscope}%
\begin{pgfscope}%
\pgfpathrectangle{\pgfqpoint{0.600000in}{0.600000in}}{\pgfqpoint{3.900000in}{3.900000in}}%
\pgfusepath{clip}%
\pgfsetbuttcap%
\pgfsetroundjoin%
\definecolor{currentfill}{rgb}{0.121569,0.466667,0.705882}%
\pgfsetfillcolor{currentfill}%
\pgfsetlinewidth{1.003750pt}%
\definecolor{currentstroke}{rgb}{0.121569,0.466667,0.705882}%
\pgfsetstrokecolor{currentstroke}%
\pgfsetdash{}{0pt}%
\pgfpathmoveto{\pgfqpoint{2.847821in}{3.372894in}}%
\pgfpathcurveto{\pgfqpoint{2.858871in}{3.372894in}}{\pgfqpoint{2.869470in}{3.377284in}}{\pgfqpoint{2.877284in}{3.385098in}}%
\pgfpathcurveto{\pgfqpoint{2.885097in}{3.392911in}}{\pgfqpoint{2.889488in}{3.403510in}}{\pgfqpoint{2.889488in}{3.414561in}}%
\pgfpathcurveto{\pgfqpoint{2.889488in}{3.425611in}}{\pgfqpoint{2.885097in}{3.436210in}}{\pgfqpoint{2.877284in}{3.444023in}}%
\pgfpathcurveto{\pgfqpoint{2.869470in}{3.451837in}}{\pgfqpoint{2.858871in}{3.456227in}}{\pgfqpoint{2.847821in}{3.456227in}}%
\pgfpathcurveto{\pgfqpoint{2.836771in}{3.456227in}}{\pgfqpoint{2.826172in}{3.451837in}}{\pgfqpoint{2.818358in}{3.444023in}}%
\pgfpathcurveto{\pgfqpoint{2.810545in}{3.436210in}}{\pgfqpoint{2.806154in}{3.425611in}}{\pgfqpoint{2.806154in}{3.414561in}}%
\pgfpathcurveto{\pgfqpoint{2.806154in}{3.403510in}}{\pgfqpoint{2.810545in}{3.392911in}}{\pgfqpoint{2.818358in}{3.385098in}}%
\pgfpathcurveto{\pgfqpoint{2.826172in}{3.377284in}}{\pgfqpoint{2.836771in}{3.372894in}}{\pgfqpoint{2.847821in}{3.372894in}}%
\pgfpathclose%
\pgfusepath{stroke,fill}%
\end{pgfscope}%
\begin{pgfscope}%
\pgfpathrectangle{\pgfqpoint{0.600000in}{0.600000in}}{\pgfqpoint{3.900000in}{3.900000in}}%
\pgfusepath{clip}%
\pgfsetbuttcap%
\pgfsetroundjoin%
\definecolor{currentfill}{rgb}{0.121569,0.466667,0.705882}%
\pgfsetfillcolor{currentfill}%
\pgfsetlinewidth{1.003750pt}%
\definecolor{currentstroke}{rgb}{0.121569,0.466667,0.705882}%
\pgfsetstrokecolor{currentstroke}%
\pgfsetdash{}{0pt}%
\pgfpathmoveto{\pgfqpoint{1.705727in}{2.149543in}}%
\pgfpathcurveto{\pgfqpoint{1.716777in}{2.149543in}}{\pgfqpoint{1.727376in}{2.153933in}}{\pgfqpoint{1.735189in}{2.161746in}}%
\pgfpathcurveto{\pgfqpoint{1.743003in}{2.169560in}}{\pgfqpoint{1.747393in}{2.180159in}}{\pgfqpoint{1.747393in}{2.191209in}}%
\pgfpathcurveto{\pgfqpoint{1.747393in}{2.202259in}}{\pgfqpoint{1.743003in}{2.212858in}}{\pgfqpoint{1.735189in}{2.220672in}}%
\pgfpathcurveto{\pgfqpoint{1.727376in}{2.228486in}}{\pgfqpoint{1.716777in}{2.232876in}}{\pgfqpoint{1.705727in}{2.232876in}}%
\pgfpathcurveto{\pgfqpoint{1.694676in}{2.232876in}}{\pgfqpoint{1.684077in}{2.228486in}}{\pgfqpoint{1.676264in}{2.220672in}}%
\pgfpathcurveto{\pgfqpoint{1.668450in}{2.212858in}}{\pgfqpoint{1.664060in}{2.202259in}}{\pgfqpoint{1.664060in}{2.191209in}}%
\pgfpathcurveto{\pgfqpoint{1.664060in}{2.180159in}}{\pgfqpoint{1.668450in}{2.169560in}}{\pgfqpoint{1.676264in}{2.161746in}}%
\pgfpathcurveto{\pgfqpoint{1.684077in}{2.153933in}}{\pgfqpoint{1.694676in}{2.149543in}}{\pgfqpoint{1.705727in}{2.149543in}}%
\pgfpathclose%
\pgfusepath{stroke,fill}%
\end{pgfscope}%
\begin{pgfscope}%
\pgfpathrectangle{\pgfqpoint{0.600000in}{0.600000in}}{\pgfqpoint{3.900000in}{3.900000in}}%
\pgfusepath{clip}%
\pgfsetbuttcap%
\pgfsetroundjoin%
\definecolor{currentfill}{rgb}{0.121569,0.466667,0.705882}%
\pgfsetfillcolor{currentfill}%
\pgfsetlinewidth{1.003750pt}%
\definecolor{currentstroke}{rgb}{0.121569,0.466667,0.705882}%
\pgfsetstrokecolor{currentstroke}%
\pgfsetdash{}{0pt}%
\pgfpathmoveto{\pgfqpoint{2.453852in}{2.921662in}}%
\pgfpathcurveto{\pgfqpoint{2.464902in}{2.921662in}}{\pgfqpoint{2.475501in}{2.926053in}}{\pgfqpoint{2.483315in}{2.933866in}}%
\pgfpathcurveto{\pgfqpoint{2.491128in}{2.941680in}}{\pgfqpoint{2.495518in}{2.952279in}}{\pgfqpoint{2.495518in}{2.963329in}}%
\pgfpathcurveto{\pgfqpoint{2.495518in}{2.974379in}}{\pgfqpoint{2.491128in}{2.984978in}}{\pgfqpoint{2.483315in}{2.992792in}}%
\pgfpathcurveto{\pgfqpoint{2.475501in}{3.000605in}}{\pgfqpoint{2.464902in}{3.004996in}}{\pgfqpoint{2.453852in}{3.004996in}}%
\pgfpathcurveto{\pgfqpoint{2.442802in}{3.004996in}}{\pgfqpoint{2.432203in}{3.000605in}}{\pgfqpoint{2.424389in}{2.992792in}}%
\pgfpathcurveto{\pgfqpoint{2.416575in}{2.984978in}}{\pgfqpoint{2.412185in}{2.974379in}}{\pgfqpoint{2.412185in}{2.963329in}}%
\pgfpathcurveto{\pgfqpoint{2.412185in}{2.952279in}}{\pgfqpoint{2.416575in}{2.941680in}}{\pgfqpoint{2.424389in}{2.933866in}}%
\pgfpathcurveto{\pgfqpoint{2.432203in}{2.926053in}}{\pgfqpoint{2.442802in}{2.921662in}}{\pgfqpoint{2.453852in}{2.921662in}}%
\pgfpathclose%
\pgfusepath{stroke,fill}%
\end{pgfscope}%
\begin{pgfscope}%
\pgfpathrectangle{\pgfqpoint{0.600000in}{0.600000in}}{\pgfqpoint{3.900000in}{3.900000in}}%
\pgfusepath{clip}%
\pgfsetbuttcap%
\pgfsetroundjoin%
\definecolor{currentfill}{rgb}{0.121569,0.466667,0.705882}%
\pgfsetfillcolor{currentfill}%
\pgfsetlinewidth{1.003750pt}%
\definecolor{currentstroke}{rgb}{0.121569,0.466667,0.705882}%
\pgfsetstrokecolor{currentstroke}%
\pgfsetdash{}{0pt}%
\pgfpathmoveto{\pgfqpoint{1.879147in}{2.743066in}}%
\pgfpathcurveto{\pgfqpoint{1.890197in}{2.743066in}}{\pgfqpoint{1.900796in}{2.747456in}}{\pgfqpoint{1.908610in}{2.755269in}}%
\pgfpathcurveto{\pgfqpoint{1.916423in}{2.763083in}}{\pgfqpoint{1.920814in}{2.773682in}}{\pgfqpoint{1.920814in}{2.784732in}}%
\pgfpathcurveto{\pgfqpoint{1.920814in}{2.795782in}}{\pgfqpoint{1.916423in}{2.806381in}}{\pgfqpoint{1.908610in}{2.814195in}}%
\pgfpathcurveto{\pgfqpoint{1.900796in}{2.822009in}}{\pgfqpoint{1.890197in}{2.826399in}}{\pgfqpoint{1.879147in}{2.826399in}}%
\pgfpathcurveto{\pgfqpoint{1.868097in}{2.826399in}}{\pgfqpoint{1.857498in}{2.822009in}}{\pgfqpoint{1.849684in}{2.814195in}}%
\pgfpathcurveto{\pgfqpoint{1.841871in}{2.806381in}}{\pgfqpoint{1.837480in}{2.795782in}}{\pgfqpoint{1.837480in}{2.784732in}}%
\pgfpathcurveto{\pgfqpoint{1.837480in}{2.773682in}}{\pgfqpoint{1.841871in}{2.763083in}}{\pgfqpoint{1.849684in}{2.755269in}}%
\pgfpathcurveto{\pgfqpoint{1.857498in}{2.747456in}}{\pgfqpoint{1.868097in}{2.743066in}}{\pgfqpoint{1.879147in}{2.743066in}}%
\pgfpathclose%
\pgfusepath{stroke,fill}%
\end{pgfscope}%
\begin{pgfscope}%
\pgfpathrectangle{\pgfqpoint{0.600000in}{0.600000in}}{\pgfqpoint{3.900000in}{3.900000in}}%
\pgfusepath{clip}%
\pgfsetbuttcap%
\pgfsetroundjoin%
\definecolor{currentfill}{rgb}{0.121569,0.466667,0.705882}%
\pgfsetfillcolor{currentfill}%
\pgfsetlinewidth{1.003750pt}%
\definecolor{currentstroke}{rgb}{0.121569,0.466667,0.705882}%
\pgfsetstrokecolor{currentstroke}%
\pgfsetdash{}{0pt}%
\pgfpathmoveto{\pgfqpoint{1.982711in}{2.294942in}}%
\pgfpathcurveto{\pgfqpoint{1.993761in}{2.294942in}}{\pgfqpoint{2.004360in}{2.299332in}}{\pgfqpoint{2.012174in}{2.307146in}}%
\pgfpathcurveto{\pgfqpoint{2.019987in}{2.314959in}}{\pgfqpoint{2.024378in}{2.325558in}}{\pgfqpoint{2.024378in}{2.336609in}}%
\pgfpathcurveto{\pgfqpoint{2.024378in}{2.347659in}}{\pgfqpoint{2.019987in}{2.358258in}}{\pgfqpoint{2.012174in}{2.366071in}}%
\pgfpathcurveto{\pgfqpoint{2.004360in}{2.373885in}}{\pgfqpoint{1.993761in}{2.378275in}}{\pgfqpoint{1.982711in}{2.378275in}}%
\pgfpathcurveto{\pgfqpoint{1.971661in}{2.378275in}}{\pgfqpoint{1.961062in}{2.373885in}}{\pgfqpoint{1.953248in}{2.366071in}}%
\pgfpathcurveto{\pgfqpoint{1.945435in}{2.358258in}}{\pgfqpoint{1.941044in}{2.347659in}}{\pgfqpoint{1.941044in}{2.336609in}}%
\pgfpathcurveto{\pgfqpoint{1.941044in}{2.325558in}}{\pgfqpoint{1.945435in}{2.314959in}}{\pgfqpoint{1.953248in}{2.307146in}}%
\pgfpathcurveto{\pgfqpoint{1.961062in}{2.299332in}}{\pgfqpoint{1.971661in}{2.294942in}}{\pgfqpoint{1.982711in}{2.294942in}}%
\pgfpathclose%
\pgfusepath{stroke,fill}%
\end{pgfscope}%
\begin{pgfscope}%
\pgfpathrectangle{\pgfqpoint{0.600000in}{0.600000in}}{\pgfqpoint{3.900000in}{3.900000in}}%
\pgfusepath{clip}%
\pgfsetbuttcap%
\pgfsetroundjoin%
\definecolor{currentfill}{rgb}{0.121569,0.466667,0.705882}%
\pgfsetfillcolor{currentfill}%
\pgfsetlinewidth{1.003750pt}%
\definecolor{currentstroke}{rgb}{0.121569,0.466667,0.705882}%
\pgfsetstrokecolor{currentstroke}%
\pgfsetdash{}{0pt}%
\pgfpathmoveto{\pgfqpoint{3.328371in}{2.274988in}}%
\pgfpathcurveto{\pgfqpoint{3.339421in}{2.274988in}}{\pgfqpoint{3.350020in}{2.279378in}}{\pgfqpoint{3.357834in}{2.287192in}}%
\pgfpathcurveto{\pgfqpoint{3.365647in}{2.295005in}}{\pgfqpoint{3.370038in}{2.305604in}}{\pgfqpoint{3.370038in}{2.316655in}}%
\pgfpathcurveto{\pgfqpoint{3.370038in}{2.327705in}}{\pgfqpoint{3.365647in}{2.338304in}}{\pgfqpoint{3.357834in}{2.346117in}}%
\pgfpathcurveto{\pgfqpoint{3.350020in}{2.353931in}}{\pgfqpoint{3.339421in}{2.358321in}}{\pgfqpoint{3.328371in}{2.358321in}}%
\pgfpathcurveto{\pgfqpoint{3.317321in}{2.358321in}}{\pgfqpoint{3.306722in}{2.353931in}}{\pgfqpoint{3.298908in}{2.346117in}}%
\pgfpathcurveto{\pgfqpoint{3.291095in}{2.338304in}}{\pgfqpoint{3.286704in}{2.327705in}}{\pgfqpoint{3.286704in}{2.316655in}}%
\pgfpathcurveto{\pgfqpoint{3.286704in}{2.305604in}}{\pgfqpoint{3.291095in}{2.295005in}}{\pgfqpoint{3.298908in}{2.287192in}}%
\pgfpathcurveto{\pgfqpoint{3.306722in}{2.279378in}}{\pgfqpoint{3.317321in}{2.274988in}}{\pgfqpoint{3.328371in}{2.274988in}}%
\pgfpathclose%
\pgfusepath{stroke,fill}%
\end{pgfscope}%
\begin{pgfscope}%
\pgfpathrectangle{\pgfqpoint{0.600000in}{0.600000in}}{\pgfqpoint{3.900000in}{3.900000in}}%
\pgfusepath{clip}%
\pgfsetbuttcap%
\pgfsetroundjoin%
\definecolor{currentfill}{rgb}{0.121569,0.466667,0.705882}%
\pgfsetfillcolor{currentfill}%
\pgfsetlinewidth{1.003750pt}%
\definecolor{currentstroke}{rgb}{0.121569,0.466667,0.705882}%
\pgfsetstrokecolor{currentstroke}%
\pgfsetdash{}{0pt}%
\pgfpathmoveto{\pgfqpoint{2.947717in}{1.458047in}}%
\pgfpathcurveto{\pgfqpoint{2.958767in}{1.458047in}}{\pgfqpoint{2.969366in}{1.462437in}}{\pgfqpoint{2.977180in}{1.470251in}}%
\pgfpathcurveto{\pgfqpoint{2.984994in}{1.478064in}}{\pgfqpoint{2.989384in}{1.488663in}}{\pgfqpoint{2.989384in}{1.499713in}}%
\pgfpathcurveto{\pgfqpoint{2.989384in}{1.510763in}}{\pgfqpoint{2.984994in}{1.521362in}}{\pgfqpoint{2.977180in}{1.529176in}}%
\pgfpathcurveto{\pgfqpoint{2.969366in}{1.536990in}}{\pgfqpoint{2.958767in}{1.541380in}}{\pgfqpoint{2.947717in}{1.541380in}}%
\pgfpathcurveto{\pgfqpoint{2.936667in}{1.541380in}}{\pgfqpoint{2.926068in}{1.536990in}}{\pgfqpoint{2.918254in}{1.529176in}}%
\pgfpathcurveto{\pgfqpoint{2.910441in}{1.521362in}}{\pgfqpoint{2.906051in}{1.510763in}}{\pgfqpoint{2.906051in}{1.499713in}}%
\pgfpathcurveto{\pgfqpoint{2.906051in}{1.488663in}}{\pgfqpoint{2.910441in}{1.478064in}}{\pgfqpoint{2.918254in}{1.470251in}}%
\pgfpathcurveto{\pgfqpoint{2.926068in}{1.462437in}}{\pgfqpoint{2.936667in}{1.458047in}}{\pgfqpoint{2.947717in}{1.458047in}}%
\pgfpathclose%
\pgfusepath{stroke,fill}%
\end{pgfscope}%
\begin{pgfscope}%
\pgfpathrectangle{\pgfqpoint{0.600000in}{0.600000in}}{\pgfqpoint{3.900000in}{3.900000in}}%
\pgfusepath{clip}%
\pgfsetbuttcap%
\pgfsetroundjoin%
\definecolor{currentfill}{rgb}{0.121569,0.466667,0.705882}%
\pgfsetfillcolor{currentfill}%
\pgfsetlinewidth{1.003750pt}%
\definecolor{currentstroke}{rgb}{0.121569,0.466667,0.705882}%
\pgfsetstrokecolor{currentstroke}%
\pgfsetdash{}{0pt}%
\pgfpathmoveto{\pgfqpoint{3.979695in}{2.890108in}}%
\pgfpathcurveto{\pgfqpoint{3.990745in}{2.890108in}}{\pgfqpoint{4.001344in}{2.894498in}}{\pgfqpoint{4.009158in}{2.902311in}}%
\pgfpathcurveto{\pgfqpoint{4.016972in}{2.910125in}}{\pgfqpoint{4.021362in}{2.920724in}}{\pgfqpoint{4.021362in}{2.931774in}}%
\pgfpathcurveto{\pgfqpoint{4.021362in}{2.942824in}}{\pgfqpoint{4.016972in}{2.953423in}}{\pgfqpoint{4.009158in}{2.961237in}}%
\pgfpathcurveto{\pgfqpoint{4.001344in}{2.969051in}}{\pgfqpoint{3.990745in}{2.973441in}}{\pgfqpoint{3.979695in}{2.973441in}}%
\pgfpathcurveto{\pgfqpoint{3.968645in}{2.973441in}}{\pgfqpoint{3.958046in}{2.969051in}}{\pgfqpoint{3.950232in}{2.961237in}}%
\pgfpathcurveto{\pgfqpoint{3.942419in}{2.953423in}}{\pgfqpoint{3.938029in}{2.942824in}}{\pgfqpoint{3.938029in}{2.931774in}}%
\pgfpathcurveto{\pgfqpoint{3.938029in}{2.920724in}}{\pgfqpoint{3.942419in}{2.910125in}}{\pgfqpoint{3.950232in}{2.902311in}}%
\pgfpathcurveto{\pgfqpoint{3.958046in}{2.894498in}}{\pgfqpoint{3.968645in}{2.890108in}}{\pgfqpoint{3.979695in}{2.890108in}}%
\pgfpathclose%
\pgfusepath{stroke,fill}%
\end{pgfscope}%
\begin{pgfscope}%
\pgfpathrectangle{\pgfqpoint{0.600000in}{0.600000in}}{\pgfqpoint{3.900000in}{3.900000in}}%
\pgfusepath{clip}%
\pgfsetbuttcap%
\pgfsetroundjoin%
\definecolor{currentfill}{rgb}{0.121569,0.466667,0.705882}%
\pgfsetfillcolor{currentfill}%
\pgfsetlinewidth{1.003750pt}%
\definecolor{currentstroke}{rgb}{0.121569,0.466667,0.705882}%
\pgfsetstrokecolor{currentstroke}%
\pgfsetdash{}{0pt}%
\pgfpathmoveto{\pgfqpoint{2.714959in}{2.977063in}}%
\pgfpathcurveto{\pgfqpoint{2.726009in}{2.977063in}}{\pgfqpoint{2.736608in}{2.981453in}}{\pgfqpoint{2.744422in}{2.989267in}}%
\pgfpathcurveto{\pgfqpoint{2.752235in}{2.997081in}}{\pgfqpoint{2.756626in}{3.007680in}}{\pgfqpoint{2.756626in}{3.018730in}}%
\pgfpathcurveto{\pgfqpoint{2.756626in}{3.029780in}}{\pgfqpoint{2.752235in}{3.040379in}}{\pgfqpoint{2.744422in}{3.048192in}}%
\pgfpathcurveto{\pgfqpoint{2.736608in}{3.056006in}}{\pgfqpoint{2.726009in}{3.060396in}}{\pgfqpoint{2.714959in}{3.060396in}}%
\pgfpathcurveto{\pgfqpoint{2.703909in}{3.060396in}}{\pgfqpoint{2.693310in}{3.056006in}}{\pgfqpoint{2.685496in}{3.048192in}}%
\pgfpathcurveto{\pgfqpoint{2.677683in}{3.040379in}}{\pgfqpoint{2.673292in}{3.029780in}}{\pgfqpoint{2.673292in}{3.018730in}}%
\pgfpathcurveto{\pgfqpoint{2.673292in}{3.007680in}}{\pgfqpoint{2.677683in}{2.997081in}}{\pgfqpoint{2.685496in}{2.989267in}}%
\pgfpathcurveto{\pgfqpoint{2.693310in}{2.981453in}}{\pgfqpoint{2.703909in}{2.977063in}}{\pgfqpoint{2.714959in}{2.977063in}}%
\pgfpathclose%
\pgfusepath{stroke,fill}%
\end{pgfscope}%
\begin{pgfscope}%
\pgfpathrectangle{\pgfqpoint{0.600000in}{0.600000in}}{\pgfqpoint{3.900000in}{3.900000in}}%
\pgfusepath{clip}%
\pgfsetbuttcap%
\pgfsetroundjoin%
\definecolor{currentfill}{rgb}{0.121569,0.466667,0.705882}%
\pgfsetfillcolor{currentfill}%
\pgfsetlinewidth{1.003750pt}%
\definecolor{currentstroke}{rgb}{0.121569,0.466667,0.705882}%
\pgfsetstrokecolor{currentstroke}%
\pgfsetdash{}{0pt}%
\pgfpathmoveto{\pgfqpoint{3.158860in}{3.032986in}}%
\pgfpathcurveto{\pgfqpoint{3.169910in}{3.032986in}}{\pgfqpoint{3.180509in}{3.037376in}}{\pgfqpoint{3.188323in}{3.045189in}}%
\pgfpathcurveto{\pgfqpoint{3.196136in}{3.053003in}}{\pgfqpoint{3.200527in}{3.063602in}}{\pgfqpoint{3.200527in}{3.074652in}}%
\pgfpathcurveto{\pgfqpoint{3.200527in}{3.085702in}}{\pgfqpoint{3.196136in}{3.096301in}}{\pgfqpoint{3.188323in}{3.104115in}}%
\pgfpathcurveto{\pgfqpoint{3.180509in}{3.111929in}}{\pgfqpoint{3.169910in}{3.116319in}}{\pgfqpoint{3.158860in}{3.116319in}}%
\pgfpathcurveto{\pgfqpoint{3.147810in}{3.116319in}}{\pgfqpoint{3.137211in}{3.111929in}}{\pgfqpoint{3.129397in}{3.104115in}}%
\pgfpathcurveto{\pgfqpoint{3.121584in}{3.096301in}}{\pgfqpoint{3.117193in}{3.085702in}}{\pgfqpoint{3.117193in}{3.074652in}}%
\pgfpathcurveto{\pgfqpoint{3.117193in}{3.063602in}}{\pgfqpoint{3.121584in}{3.053003in}}{\pgfqpoint{3.129397in}{3.045189in}}%
\pgfpathcurveto{\pgfqpoint{3.137211in}{3.037376in}}{\pgfqpoint{3.147810in}{3.032986in}}{\pgfqpoint{3.158860in}{3.032986in}}%
\pgfpathclose%
\pgfusepath{stroke,fill}%
\end{pgfscope}%
\begin{pgfscope}%
\pgfpathrectangle{\pgfqpoint{0.600000in}{0.600000in}}{\pgfqpoint{3.900000in}{3.900000in}}%
\pgfusepath{clip}%
\pgfsetbuttcap%
\pgfsetroundjoin%
\definecolor{currentfill}{rgb}{0.121569,0.466667,0.705882}%
\pgfsetfillcolor{currentfill}%
\pgfsetlinewidth{1.003750pt}%
\definecolor{currentstroke}{rgb}{0.121569,0.466667,0.705882}%
\pgfsetstrokecolor{currentstroke}%
\pgfsetdash{}{0pt}%
\pgfpathmoveto{\pgfqpoint{2.273614in}{2.248049in}}%
\pgfpathcurveto{\pgfqpoint{2.284664in}{2.248049in}}{\pgfqpoint{2.295263in}{2.252440in}}{\pgfqpoint{2.303076in}{2.260253in}}%
\pgfpathcurveto{\pgfqpoint{2.310890in}{2.268067in}}{\pgfqpoint{2.315280in}{2.278666in}}{\pgfqpoint{2.315280in}{2.289716in}}%
\pgfpathcurveto{\pgfqpoint{2.315280in}{2.300766in}}{\pgfqpoint{2.310890in}{2.311365in}}{\pgfqpoint{2.303076in}{2.319179in}}%
\pgfpathcurveto{\pgfqpoint{2.295263in}{2.326993in}}{\pgfqpoint{2.284664in}{2.331383in}}{\pgfqpoint{2.273614in}{2.331383in}}%
\pgfpathcurveto{\pgfqpoint{2.262564in}{2.331383in}}{\pgfqpoint{2.251965in}{2.326993in}}{\pgfqpoint{2.244151in}{2.319179in}}%
\pgfpathcurveto{\pgfqpoint{2.236337in}{2.311365in}}{\pgfqpoint{2.231947in}{2.300766in}}{\pgfqpoint{2.231947in}{2.289716in}}%
\pgfpathcurveto{\pgfqpoint{2.231947in}{2.278666in}}{\pgfqpoint{2.236337in}{2.268067in}}{\pgfqpoint{2.244151in}{2.260253in}}%
\pgfpathcurveto{\pgfqpoint{2.251965in}{2.252440in}}{\pgfqpoint{2.262564in}{2.248049in}}{\pgfqpoint{2.273614in}{2.248049in}}%
\pgfpathclose%
\pgfusepath{stroke,fill}%
\end{pgfscope}%
\begin{pgfscope}%
\pgfpathrectangle{\pgfqpoint{0.600000in}{0.600000in}}{\pgfqpoint{3.900000in}{3.900000in}}%
\pgfusepath{clip}%
\pgfsetbuttcap%
\pgfsetroundjoin%
\definecolor{currentfill}{rgb}{0.121569,0.466667,0.705882}%
\pgfsetfillcolor{currentfill}%
\pgfsetlinewidth{1.003750pt}%
\definecolor{currentstroke}{rgb}{0.121569,0.466667,0.705882}%
\pgfsetstrokecolor{currentstroke}%
\pgfsetdash{}{0pt}%
\pgfpathmoveto{\pgfqpoint{2.929130in}{2.781410in}}%
\pgfpathcurveto{\pgfqpoint{2.940180in}{2.781410in}}{\pgfqpoint{2.950779in}{2.785800in}}{\pgfqpoint{2.958592in}{2.793614in}}%
\pgfpathcurveto{\pgfqpoint{2.966406in}{2.801427in}}{\pgfqpoint{2.970796in}{2.812026in}}{\pgfqpoint{2.970796in}{2.823076in}}%
\pgfpathcurveto{\pgfqpoint{2.970796in}{2.834126in}}{\pgfqpoint{2.966406in}{2.844725in}}{\pgfqpoint{2.958592in}{2.852539in}}%
\pgfpathcurveto{\pgfqpoint{2.950779in}{2.860353in}}{\pgfqpoint{2.940180in}{2.864743in}}{\pgfqpoint{2.929130in}{2.864743in}}%
\pgfpathcurveto{\pgfqpoint{2.918080in}{2.864743in}}{\pgfqpoint{2.907481in}{2.860353in}}{\pgfqpoint{2.899667in}{2.852539in}}%
\pgfpathcurveto{\pgfqpoint{2.891853in}{2.844725in}}{\pgfqpoint{2.887463in}{2.834126in}}{\pgfqpoint{2.887463in}{2.823076in}}%
\pgfpathcurveto{\pgfqpoint{2.887463in}{2.812026in}}{\pgfqpoint{2.891853in}{2.801427in}}{\pgfqpoint{2.899667in}{2.793614in}}%
\pgfpathcurveto{\pgfqpoint{2.907481in}{2.785800in}}{\pgfqpoint{2.918080in}{2.781410in}}{\pgfqpoint{2.929130in}{2.781410in}}%
\pgfpathclose%
\pgfusepath{stroke,fill}%
\end{pgfscope}%
\begin{pgfscope}%
\pgfpathrectangle{\pgfqpoint{0.600000in}{0.600000in}}{\pgfqpoint{3.900000in}{3.900000in}}%
\pgfusepath{clip}%
\pgfsetbuttcap%
\pgfsetroundjoin%
\definecolor{currentfill}{rgb}{0.121569,0.466667,0.705882}%
\pgfsetfillcolor{currentfill}%
\pgfsetlinewidth{1.003750pt}%
\definecolor{currentstroke}{rgb}{0.121569,0.466667,0.705882}%
\pgfsetstrokecolor{currentstroke}%
\pgfsetdash{}{0pt}%
\pgfpathmoveto{\pgfqpoint{2.490464in}{2.545817in}}%
\pgfpathcurveto{\pgfqpoint{2.501514in}{2.545817in}}{\pgfqpoint{2.512113in}{2.550207in}}{\pgfqpoint{2.519927in}{2.558020in}}%
\pgfpathcurveto{\pgfqpoint{2.527740in}{2.565834in}}{\pgfqpoint{2.532131in}{2.576433in}}{\pgfqpoint{2.532131in}{2.587483in}}%
\pgfpathcurveto{\pgfqpoint{2.532131in}{2.598533in}}{\pgfqpoint{2.527740in}{2.609132in}}{\pgfqpoint{2.519927in}{2.616946in}}%
\pgfpathcurveto{\pgfqpoint{2.512113in}{2.624760in}}{\pgfqpoint{2.501514in}{2.629150in}}{\pgfqpoint{2.490464in}{2.629150in}}%
\pgfpathcurveto{\pgfqpoint{2.479414in}{2.629150in}}{\pgfqpoint{2.468815in}{2.624760in}}{\pgfqpoint{2.461001in}{2.616946in}}%
\pgfpathcurveto{\pgfqpoint{2.453188in}{2.609132in}}{\pgfqpoint{2.448797in}{2.598533in}}{\pgfqpoint{2.448797in}{2.587483in}}%
\pgfpathcurveto{\pgfqpoint{2.448797in}{2.576433in}}{\pgfqpoint{2.453188in}{2.565834in}}{\pgfqpoint{2.461001in}{2.558020in}}%
\pgfpathcurveto{\pgfqpoint{2.468815in}{2.550207in}}{\pgfqpoint{2.479414in}{2.545817in}}{\pgfqpoint{2.490464in}{2.545817in}}%
\pgfpathclose%
\pgfusepath{stroke,fill}%
\end{pgfscope}%
\begin{pgfscope}%
\pgfpathrectangle{\pgfqpoint{0.600000in}{0.600000in}}{\pgfqpoint{3.900000in}{3.900000in}}%
\pgfusepath{clip}%
\pgfsetbuttcap%
\pgfsetroundjoin%
\definecolor{currentfill}{rgb}{0.121569,0.466667,0.705882}%
\pgfsetfillcolor{currentfill}%
\pgfsetlinewidth{1.003750pt}%
\definecolor{currentstroke}{rgb}{0.121569,0.466667,0.705882}%
\pgfsetstrokecolor{currentstroke}%
\pgfsetdash{}{0pt}%
\pgfpathmoveto{\pgfqpoint{2.469810in}{2.547102in}}%
\pgfpathcurveto{\pgfqpoint{2.480860in}{2.547102in}}{\pgfqpoint{2.491459in}{2.551492in}}{\pgfqpoint{2.499272in}{2.559306in}}%
\pgfpathcurveto{\pgfqpoint{2.507086in}{2.567120in}}{\pgfqpoint{2.511476in}{2.577719in}}{\pgfqpoint{2.511476in}{2.588769in}}%
\pgfpathcurveto{\pgfqpoint{2.511476in}{2.599819in}}{\pgfqpoint{2.507086in}{2.610418in}}{\pgfqpoint{2.499272in}{2.618232in}}%
\pgfpathcurveto{\pgfqpoint{2.491459in}{2.626045in}}{\pgfqpoint{2.480860in}{2.630435in}}{\pgfqpoint{2.469810in}{2.630435in}}%
\pgfpathcurveto{\pgfqpoint{2.458760in}{2.630435in}}{\pgfqpoint{2.448161in}{2.626045in}}{\pgfqpoint{2.440347in}{2.618232in}}%
\pgfpathcurveto{\pgfqpoint{2.432533in}{2.610418in}}{\pgfqpoint{2.428143in}{2.599819in}}{\pgfqpoint{2.428143in}{2.588769in}}%
\pgfpathcurveto{\pgfqpoint{2.428143in}{2.577719in}}{\pgfqpoint{2.432533in}{2.567120in}}{\pgfqpoint{2.440347in}{2.559306in}}%
\pgfpathcurveto{\pgfqpoint{2.448161in}{2.551492in}}{\pgfqpoint{2.458760in}{2.547102in}}{\pgfqpoint{2.469810in}{2.547102in}}%
\pgfpathclose%
\pgfusepath{stroke,fill}%
\end{pgfscope}%
\begin{pgfscope}%
\pgfpathrectangle{\pgfqpoint{0.600000in}{0.600000in}}{\pgfqpoint{3.900000in}{3.900000in}}%
\pgfusepath{clip}%
\pgfsetbuttcap%
\pgfsetroundjoin%
\definecolor{currentfill}{rgb}{0.121569,0.466667,0.705882}%
\pgfsetfillcolor{currentfill}%
\pgfsetlinewidth{1.003750pt}%
\definecolor{currentstroke}{rgb}{0.121569,0.466667,0.705882}%
\pgfsetstrokecolor{currentstroke}%
\pgfsetdash{}{0pt}%
\pgfpathmoveto{\pgfqpoint{2.995855in}{1.475931in}}%
\pgfpathcurveto{\pgfqpoint{3.006905in}{1.475931in}}{\pgfqpoint{3.017504in}{1.480321in}}{\pgfqpoint{3.025317in}{1.488134in}}%
\pgfpathcurveto{\pgfqpoint{3.033131in}{1.495948in}}{\pgfqpoint{3.037521in}{1.506547in}}{\pgfqpoint{3.037521in}{1.517597in}}%
\pgfpathcurveto{\pgfqpoint{3.037521in}{1.528647in}}{\pgfqpoint{3.033131in}{1.539246in}}{\pgfqpoint{3.025317in}{1.547060in}}%
\pgfpathcurveto{\pgfqpoint{3.017504in}{1.554874in}}{\pgfqpoint{3.006905in}{1.559264in}}{\pgfqpoint{2.995855in}{1.559264in}}%
\pgfpathcurveto{\pgfqpoint{2.984805in}{1.559264in}}{\pgfqpoint{2.974205in}{1.554874in}}{\pgfqpoint{2.966392in}{1.547060in}}%
\pgfpathcurveto{\pgfqpoint{2.958578in}{1.539246in}}{\pgfqpoint{2.954188in}{1.528647in}}{\pgfqpoint{2.954188in}{1.517597in}}%
\pgfpathcurveto{\pgfqpoint{2.954188in}{1.506547in}}{\pgfqpoint{2.958578in}{1.495948in}}{\pgfqpoint{2.966392in}{1.488134in}}%
\pgfpathcurveto{\pgfqpoint{2.974205in}{1.480321in}}{\pgfqpoint{2.984805in}{1.475931in}}{\pgfqpoint{2.995855in}{1.475931in}}%
\pgfpathclose%
\pgfusepath{stroke,fill}%
\end{pgfscope}%
\begin{pgfscope}%
\pgfpathrectangle{\pgfqpoint{0.600000in}{0.600000in}}{\pgfqpoint{3.900000in}{3.900000in}}%
\pgfusepath{clip}%
\pgfsetbuttcap%
\pgfsetroundjoin%
\definecolor{currentfill}{rgb}{0.121569,0.466667,0.705882}%
\pgfsetfillcolor{currentfill}%
\pgfsetlinewidth{1.003750pt}%
\definecolor{currentstroke}{rgb}{0.121569,0.466667,0.705882}%
\pgfsetstrokecolor{currentstroke}%
\pgfsetdash{}{0pt}%
\pgfpathmoveto{\pgfqpoint{2.727925in}{2.859516in}}%
\pgfpathcurveto{\pgfqpoint{2.738975in}{2.859516in}}{\pgfqpoint{2.749574in}{2.863906in}}{\pgfqpoint{2.757387in}{2.871720in}}%
\pgfpathcurveto{\pgfqpoint{2.765201in}{2.879534in}}{\pgfqpoint{2.769591in}{2.890133in}}{\pgfqpoint{2.769591in}{2.901183in}}%
\pgfpathcurveto{\pgfqpoint{2.769591in}{2.912233in}}{\pgfqpoint{2.765201in}{2.922832in}}{\pgfqpoint{2.757387in}{2.930646in}}%
\pgfpathcurveto{\pgfqpoint{2.749574in}{2.938459in}}{\pgfqpoint{2.738975in}{2.942849in}}{\pgfqpoint{2.727925in}{2.942849in}}%
\pgfpathcurveto{\pgfqpoint{2.716874in}{2.942849in}}{\pgfqpoint{2.706275in}{2.938459in}}{\pgfqpoint{2.698462in}{2.930646in}}%
\pgfpathcurveto{\pgfqpoint{2.690648in}{2.922832in}}{\pgfqpoint{2.686258in}{2.912233in}}{\pgfqpoint{2.686258in}{2.901183in}}%
\pgfpathcurveto{\pgfqpoint{2.686258in}{2.890133in}}{\pgfqpoint{2.690648in}{2.879534in}}{\pgfqpoint{2.698462in}{2.871720in}}%
\pgfpathcurveto{\pgfqpoint{2.706275in}{2.863906in}}{\pgfqpoint{2.716874in}{2.859516in}}{\pgfqpoint{2.727925in}{2.859516in}}%
\pgfpathclose%
\pgfusepath{stroke,fill}%
\end{pgfscope}%
\begin{pgfscope}%
\pgfpathrectangle{\pgfqpoint{0.600000in}{0.600000in}}{\pgfqpoint{3.900000in}{3.900000in}}%
\pgfusepath{clip}%
\pgfsetbuttcap%
\pgfsetroundjoin%
\definecolor{currentfill}{rgb}{0.121569,0.466667,0.705882}%
\pgfsetfillcolor{currentfill}%
\pgfsetlinewidth{1.003750pt}%
\definecolor{currentstroke}{rgb}{0.121569,0.466667,0.705882}%
\pgfsetstrokecolor{currentstroke}%
\pgfsetdash{}{0pt}%
\pgfpathmoveto{\pgfqpoint{2.893348in}{3.465567in}}%
\pgfpathcurveto{\pgfqpoint{2.904398in}{3.465567in}}{\pgfqpoint{2.914997in}{3.469957in}}{\pgfqpoint{2.922811in}{3.477771in}}%
\pgfpathcurveto{\pgfqpoint{2.930624in}{3.485584in}}{\pgfqpoint{2.935014in}{3.496183in}}{\pgfqpoint{2.935014in}{3.507234in}}%
\pgfpathcurveto{\pgfqpoint{2.935014in}{3.518284in}}{\pgfqpoint{2.930624in}{3.528883in}}{\pgfqpoint{2.922811in}{3.536696in}}%
\pgfpathcurveto{\pgfqpoint{2.914997in}{3.544510in}}{\pgfqpoint{2.904398in}{3.548900in}}{\pgfqpoint{2.893348in}{3.548900in}}%
\pgfpathcurveto{\pgfqpoint{2.882298in}{3.548900in}}{\pgfqpoint{2.871699in}{3.544510in}}{\pgfqpoint{2.863885in}{3.536696in}}%
\pgfpathcurveto{\pgfqpoint{2.856071in}{3.528883in}}{\pgfqpoint{2.851681in}{3.518284in}}{\pgfqpoint{2.851681in}{3.507234in}}%
\pgfpathcurveto{\pgfqpoint{2.851681in}{3.496183in}}{\pgfqpoint{2.856071in}{3.485584in}}{\pgfqpoint{2.863885in}{3.477771in}}%
\pgfpathcurveto{\pgfqpoint{2.871699in}{3.469957in}}{\pgfqpoint{2.882298in}{3.465567in}}{\pgfqpoint{2.893348in}{3.465567in}}%
\pgfpathclose%
\pgfusepath{stroke,fill}%
\end{pgfscope}%
\begin{pgfscope}%
\pgfpathrectangle{\pgfqpoint{0.600000in}{0.600000in}}{\pgfqpoint{3.900000in}{3.900000in}}%
\pgfusepath{clip}%
\pgfsetbuttcap%
\pgfsetroundjoin%
\definecolor{currentfill}{rgb}{0.121569,0.466667,0.705882}%
\pgfsetfillcolor{currentfill}%
\pgfsetlinewidth{1.003750pt}%
\definecolor{currentstroke}{rgb}{0.121569,0.466667,0.705882}%
\pgfsetstrokecolor{currentstroke}%
\pgfsetdash{}{0pt}%
\pgfpathmoveto{\pgfqpoint{2.413819in}{3.004199in}}%
\pgfpathcurveto{\pgfqpoint{2.424869in}{3.004199in}}{\pgfqpoint{2.435468in}{3.008589in}}{\pgfqpoint{2.443282in}{3.016402in}}%
\pgfpathcurveto{\pgfqpoint{2.451095in}{3.024216in}}{\pgfqpoint{2.455485in}{3.034815in}}{\pgfqpoint{2.455485in}{3.045865in}}%
\pgfpathcurveto{\pgfqpoint{2.455485in}{3.056915in}}{\pgfqpoint{2.451095in}{3.067514in}}{\pgfqpoint{2.443282in}{3.075328in}}%
\pgfpathcurveto{\pgfqpoint{2.435468in}{3.083142in}}{\pgfqpoint{2.424869in}{3.087532in}}{\pgfqpoint{2.413819in}{3.087532in}}%
\pgfpathcurveto{\pgfqpoint{2.402769in}{3.087532in}}{\pgfqpoint{2.392170in}{3.083142in}}{\pgfqpoint{2.384356in}{3.075328in}}%
\pgfpathcurveto{\pgfqpoint{2.376542in}{3.067514in}}{\pgfqpoint{2.372152in}{3.056915in}}{\pgfqpoint{2.372152in}{3.045865in}}%
\pgfpathcurveto{\pgfqpoint{2.372152in}{3.034815in}}{\pgfqpoint{2.376542in}{3.024216in}}{\pgfqpoint{2.384356in}{3.016402in}}%
\pgfpathcurveto{\pgfqpoint{2.392170in}{3.008589in}}{\pgfqpoint{2.402769in}{3.004199in}}{\pgfqpoint{2.413819in}{3.004199in}}%
\pgfpathclose%
\pgfusepath{stroke,fill}%
\end{pgfscope}%
\begin{pgfscope}%
\pgfpathrectangle{\pgfqpoint{0.600000in}{0.600000in}}{\pgfqpoint{3.900000in}{3.900000in}}%
\pgfusepath{clip}%
\pgfsetbuttcap%
\pgfsetroundjoin%
\definecolor{currentfill}{rgb}{0.121569,0.466667,0.705882}%
\pgfsetfillcolor{currentfill}%
\pgfsetlinewidth{1.003750pt}%
\definecolor{currentstroke}{rgb}{0.121569,0.466667,0.705882}%
\pgfsetstrokecolor{currentstroke}%
\pgfsetdash{}{0pt}%
\pgfpathmoveto{\pgfqpoint{2.544627in}{2.213689in}}%
\pgfpathcurveto{\pgfqpoint{2.555677in}{2.213689in}}{\pgfqpoint{2.566276in}{2.218079in}}{\pgfqpoint{2.574090in}{2.225892in}}%
\pgfpathcurveto{\pgfqpoint{2.581904in}{2.233706in}}{\pgfqpoint{2.586294in}{2.244305in}}{\pgfqpoint{2.586294in}{2.255355in}}%
\pgfpathcurveto{\pgfqpoint{2.586294in}{2.266405in}}{\pgfqpoint{2.581904in}{2.277004in}}{\pgfqpoint{2.574090in}{2.284818in}}%
\pgfpathcurveto{\pgfqpoint{2.566276in}{2.292632in}}{\pgfqpoint{2.555677in}{2.297022in}}{\pgfqpoint{2.544627in}{2.297022in}}%
\pgfpathcurveto{\pgfqpoint{2.533577in}{2.297022in}}{\pgfqpoint{2.522978in}{2.292632in}}{\pgfqpoint{2.515164in}{2.284818in}}%
\pgfpathcurveto{\pgfqpoint{2.507351in}{2.277004in}}{\pgfqpoint{2.502960in}{2.266405in}}{\pgfqpoint{2.502960in}{2.255355in}}%
\pgfpathcurveto{\pgfqpoint{2.502960in}{2.244305in}}{\pgfqpoint{2.507351in}{2.233706in}}{\pgfqpoint{2.515164in}{2.225892in}}%
\pgfpathcurveto{\pgfqpoint{2.522978in}{2.218079in}}{\pgfqpoint{2.533577in}{2.213689in}}{\pgfqpoint{2.544627in}{2.213689in}}%
\pgfpathclose%
\pgfusepath{stroke,fill}%
\end{pgfscope}%
\begin{pgfscope}%
\pgfpathrectangle{\pgfqpoint{0.600000in}{0.600000in}}{\pgfqpoint{3.900000in}{3.900000in}}%
\pgfusepath{clip}%
\pgfsetbuttcap%
\pgfsetroundjoin%
\definecolor{currentfill}{rgb}{0.121569,0.466667,0.705882}%
\pgfsetfillcolor{currentfill}%
\pgfsetlinewidth{1.003750pt}%
\definecolor{currentstroke}{rgb}{0.121569,0.466667,0.705882}%
\pgfsetstrokecolor{currentstroke}%
\pgfsetdash{}{0pt}%
\pgfpathmoveto{\pgfqpoint{2.426307in}{2.784667in}}%
\pgfpathcurveto{\pgfqpoint{2.437357in}{2.784667in}}{\pgfqpoint{2.447956in}{2.789058in}}{\pgfqpoint{2.455770in}{2.796871in}}%
\pgfpathcurveto{\pgfqpoint{2.463584in}{2.804685in}}{\pgfqpoint{2.467974in}{2.815284in}}{\pgfqpoint{2.467974in}{2.826334in}}%
\pgfpathcurveto{\pgfqpoint{2.467974in}{2.837384in}}{\pgfqpoint{2.463584in}{2.847983in}}{\pgfqpoint{2.455770in}{2.855797in}}%
\pgfpathcurveto{\pgfqpoint{2.447956in}{2.863610in}}{\pgfqpoint{2.437357in}{2.868001in}}{\pgfqpoint{2.426307in}{2.868001in}}%
\pgfpathcurveto{\pgfqpoint{2.415257in}{2.868001in}}{\pgfqpoint{2.404658in}{2.863610in}}{\pgfqpoint{2.396844in}{2.855797in}}%
\pgfpathcurveto{\pgfqpoint{2.389031in}{2.847983in}}{\pgfqpoint{2.384640in}{2.837384in}}{\pgfqpoint{2.384640in}{2.826334in}}%
\pgfpathcurveto{\pgfqpoint{2.384640in}{2.815284in}}{\pgfqpoint{2.389031in}{2.804685in}}{\pgfqpoint{2.396844in}{2.796871in}}%
\pgfpathcurveto{\pgfqpoint{2.404658in}{2.789058in}}{\pgfqpoint{2.415257in}{2.784667in}}{\pgfqpoint{2.426307in}{2.784667in}}%
\pgfpathclose%
\pgfusepath{stroke,fill}%
\end{pgfscope}%
\begin{pgfscope}%
\pgfpathrectangle{\pgfqpoint{0.600000in}{0.600000in}}{\pgfqpoint{3.900000in}{3.900000in}}%
\pgfusepath{clip}%
\pgfsetbuttcap%
\pgfsetroundjoin%
\definecolor{currentfill}{rgb}{0.121569,0.466667,0.705882}%
\pgfsetfillcolor{currentfill}%
\pgfsetlinewidth{1.003750pt}%
\definecolor{currentstroke}{rgb}{0.121569,0.466667,0.705882}%
\pgfsetstrokecolor{currentstroke}%
\pgfsetdash{}{0pt}%
\pgfpathmoveto{\pgfqpoint{3.503558in}{2.444489in}}%
\pgfpathcurveto{\pgfqpoint{3.514608in}{2.444489in}}{\pgfqpoint{3.525207in}{2.448880in}}{\pgfqpoint{3.533021in}{2.456693in}}%
\pgfpathcurveto{\pgfqpoint{3.540835in}{2.464507in}}{\pgfqpoint{3.545225in}{2.475106in}}{\pgfqpoint{3.545225in}{2.486156in}}%
\pgfpathcurveto{\pgfqpoint{3.545225in}{2.497206in}}{\pgfqpoint{3.540835in}{2.507805in}}{\pgfqpoint{3.533021in}{2.515619in}}%
\pgfpathcurveto{\pgfqpoint{3.525207in}{2.523432in}}{\pgfqpoint{3.514608in}{2.527823in}}{\pgfqpoint{3.503558in}{2.527823in}}%
\pgfpathcurveto{\pgfqpoint{3.492508in}{2.527823in}}{\pgfqpoint{3.481909in}{2.523432in}}{\pgfqpoint{3.474095in}{2.515619in}}%
\pgfpathcurveto{\pgfqpoint{3.466282in}{2.507805in}}{\pgfqpoint{3.461892in}{2.497206in}}{\pgfqpoint{3.461892in}{2.486156in}}%
\pgfpathcurveto{\pgfqpoint{3.461892in}{2.475106in}}{\pgfqpoint{3.466282in}{2.464507in}}{\pgfqpoint{3.474095in}{2.456693in}}%
\pgfpathcurveto{\pgfqpoint{3.481909in}{2.448880in}}{\pgfqpoint{3.492508in}{2.444489in}}{\pgfqpoint{3.503558in}{2.444489in}}%
\pgfpathclose%
\pgfusepath{stroke,fill}%
\end{pgfscope}%
\begin{pgfscope}%
\pgfpathrectangle{\pgfqpoint{0.600000in}{0.600000in}}{\pgfqpoint{3.900000in}{3.900000in}}%
\pgfusepath{clip}%
\pgfsetbuttcap%
\pgfsetroundjoin%
\definecolor{currentfill}{rgb}{0.121569,0.466667,0.705882}%
\pgfsetfillcolor{currentfill}%
\pgfsetlinewidth{1.003750pt}%
\definecolor{currentstroke}{rgb}{0.121569,0.466667,0.705882}%
\pgfsetstrokecolor{currentstroke}%
\pgfsetdash{}{0pt}%
\pgfpathmoveto{\pgfqpoint{3.117927in}{2.716502in}}%
\pgfpathcurveto{\pgfqpoint{3.128978in}{2.716502in}}{\pgfqpoint{3.139577in}{2.720892in}}{\pgfqpoint{3.147390in}{2.728705in}}%
\pgfpathcurveto{\pgfqpoint{3.155204in}{2.736519in}}{\pgfqpoint{3.159594in}{2.747118in}}{\pgfqpoint{3.159594in}{2.758168in}}%
\pgfpathcurveto{\pgfqpoint{3.159594in}{2.769218in}}{\pgfqpoint{3.155204in}{2.779817in}}{\pgfqpoint{3.147390in}{2.787631in}}%
\pgfpathcurveto{\pgfqpoint{3.139577in}{2.795445in}}{\pgfqpoint{3.128978in}{2.799835in}}{\pgfqpoint{3.117927in}{2.799835in}}%
\pgfpathcurveto{\pgfqpoint{3.106877in}{2.799835in}}{\pgfqpoint{3.096278in}{2.795445in}}{\pgfqpoint{3.088465in}{2.787631in}}%
\pgfpathcurveto{\pgfqpoint{3.080651in}{2.779817in}}{\pgfqpoint{3.076261in}{2.769218in}}{\pgfqpoint{3.076261in}{2.758168in}}%
\pgfpathcurveto{\pgfqpoint{3.076261in}{2.747118in}}{\pgfqpoint{3.080651in}{2.736519in}}{\pgfqpoint{3.088465in}{2.728705in}}%
\pgfpathcurveto{\pgfqpoint{3.096278in}{2.720892in}}{\pgfqpoint{3.106877in}{2.716502in}}{\pgfqpoint{3.117927in}{2.716502in}}%
\pgfpathclose%
\pgfusepath{stroke,fill}%
\end{pgfscope}%
\begin{pgfscope}%
\pgfpathrectangle{\pgfqpoint{0.600000in}{0.600000in}}{\pgfqpoint{3.900000in}{3.900000in}}%
\pgfusepath{clip}%
\pgfsetbuttcap%
\pgfsetroundjoin%
\definecolor{currentfill}{rgb}{0.121569,0.466667,0.705882}%
\pgfsetfillcolor{currentfill}%
\pgfsetlinewidth{1.003750pt}%
\definecolor{currentstroke}{rgb}{0.121569,0.466667,0.705882}%
\pgfsetstrokecolor{currentstroke}%
\pgfsetdash{}{0pt}%
\pgfpathmoveto{\pgfqpoint{2.921868in}{3.099484in}}%
\pgfpathcurveto{\pgfqpoint{2.932918in}{3.099484in}}{\pgfqpoint{2.943517in}{3.103874in}}{\pgfqpoint{2.951330in}{3.111688in}}%
\pgfpathcurveto{\pgfqpoint{2.959144in}{3.119501in}}{\pgfqpoint{2.963534in}{3.130100in}}{\pgfqpoint{2.963534in}{3.141150in}}%
\pgfpathcurveto{\pgfqpoint{2.963534in}{3.152200in}}{\pgfqpoint{2.959144in}{3.162800in}}{\pgfqpoint{2.951330in}{3.170613in}}%
\pgfpathcurveto{\pgfqpoint{2.943517in}{3.178427in}}{\pgfqpoint{2.932918in}{3.182817in}}{\pgfqpoint{2.921868in}{3.182817in}}%
\pgfpathcurveto{\pgfqpoint{2.910817in}{3.182817in}}{\pgfqpoint{2.900218in}{3.178427in}}{\pgfqpoint{2.892405in}{3.170613in}}%
\pgfpathcurveto{\pgfqpoint{2.884591in}{3.162800in}}{\pgfqpoint{2.880201in}{3.152200in}}{\pgfqpoint{2.880201in}{3.141150in}}%
\pgfpathcurveto{\pgfqpoint{2.880201in}{3.130100in}}{\pgfqpoint{2.884591in}{3.119501in}}{\pgfqpoint{2.892405in}{3.111688in}}%
\pgfpathcurveto{\pgfqpoint{2.900218in}{3.103874in}}{\pgfqpoint{2.910817in}{3.099484in}}{\pgfqpoint{2.921868in}{3.099484in}}%
\pgfpathclose%
\pgfusepath{stroke,fill}%
\end{pgfscope}%
\begin{pgfscope}%
\pgfpathrectangle{\pgfqpoint{0.600000in}{0.600000in}}{\pgfqpoint{3.900000in}{3.900000in}}%
\pgfusepath{clip}%
\pgfsetbuttcap%
\pgfsetroundjoin%
\definecolor{currentfill}{rgb}{0.121569,0.466667,0.705882}%
\pgfsetfillcolor{currentfill}%
\pgfsetlinewidth{1.003750pt}%
\definecolor{currentstroke}{rgb}{0.121569,0.466667,0.705882}%
\pgfsetstrokecolor{currentstroke}%
\pgfsetdash{}{0pt}%
\pgfpathmoveto{\pgfqpoint{2.026236in}{2.191459in}}%
\pgfpathcurveto{\pgfqpoint{2.037286in}{2.191459in}}{\pgfqpoint{2.047885in}{2.195849in}}{\pgfqpoint{2.055699in}{2.203663in}}%
\pgfpathcurveto{\pgfqpoint{2.063512in}{2.211477in}}{\pgfqpoint{2.067903in}{2.222076in}}{\pgfqpoint{2.067903in}{2.233126in}}%
\pgfpathcurveto{\pgfqpoint{2.067903in}{2.244176in}}{\pgfqpoint{2.063512in}{2.254775in}}{\pgfqpoint{2.055699in}{2.262588in}}%
\pgfpathcurveto{\pgfqpoint{2.047885in}{2.270402in}}{\pgfqpoint{2.037286in}{2.274792in}}{\pgfqpoint{2.026236in}{2.274792in}}%
\pgfpathcurveto{\pgfqpoint{2.015186in}{2.274792in}}{\pgfqpoint{2.004587in}{2.270402in}}{\pgfqpoint{1.996773in}{2.262588in}}%
\pgfpathcurveto{\pgfqpoint{1.988959in}{2.254775in}}{\pgfqpoint{1.984569in}{2.244176in}}{\pgfqpoint{1.984569in}{2.233126in}}%
\pgfpathcurveto{\pgfqpoint{1.984569in}{2.222076in}}{\pgfqpoint{1.988959in}{2.211477in}}{\pgfqpoint{1.996773in}{2.203663in}}%
\pgfpathcurveto{\pgfqpoint{2.004587in}{2.195849in}}{\pgfqpoint{2.015186in}{2.191459in}}{\pgfqpoint{2.026236in}{2.191459in}}%
\pgfpathclose%
\pgfusepath{stroke,fill}%
\end{pgfscope}%
\begin{pgfscope}%
\pgfpathrectangle{\pgfqpoint{0.600000in}{0.600000in}}{\pgfqpoint{3.900000in}{3.900000in}}%
\pgfusepath{clip}%
\pgfsetbuttcap%
\pgfsetroundjoin%
\definecolor{currentfill}{rgb}{0.121569,0.466667,0.705882}%
\pgfsetfillcolor{currentfill}%
\pgfsetlinewidth{1.003750pt}%
\definecolor{currentstroke}{rgb}{0.121569,0.466667,0.705882}%
\pgfsetstrokecolor{currentstroke}%
\pgfsetdash{}{0pt}%
\pgfpathmoveto{\pgfqpoint{3.945730in}{1.551978in}}%
\pgfpathcurveto{\pgfqpoint{3.956780in}{1.551978in}}{\pgfqpoint{3.967379in}{1.556368in}}{\pgfqpoint{3.975192in}{1.564181in}}%
\pgfpathcurveto{\pgfqpoint{3.983006in}{1.571995in}}{\pgfqpoint{3.987396in}{1.582594in}}{\pgfqpoint{3.987396in}{1.593644in}}%
\pgfpathcurveto{\pgfqpoint{3.987396in}{1.604694in}}{\pgfqpoint{3.983006in}{1.615293in}}{\pgfqpoint{3.975192in}{1.623107in}}%
\pgfpathcurveto{\pgfqpoint{3.967379in}{1.630921in}}{\pgfqpoint{3.956780in}{1.635311in}}{\pgfqpoint{3.945730in}{1.635311in}}%
\pgfpathcurveto{\pgfqpoint{3.934679in}{1.635311in}}{\pgfqpoint{3.924080in}{1.630921in}}{\pgfqpoint{3.916267in}{1.623107in}}%
\pgfpathcurveto{\pgfqpoint{3.908453in}{1.615293in}}{\pgfqpoint{3.904063in}{1.604694in}}{\pgfqpoint{3.904063in}{1.593644in}}%
\pgfpathcurveto{\pgfqpoint{3.904063in}{1.582594in}}{\pgfqpoint{3.908453in}{1.571995in}}{\pgfqpoint{3.916267in}{1.564181in}}%
\pgfpathcurveto{\pgfqpoint{3.924080in}{1.556368in}}{\pgfqpoint{3.934679in}{1.551978in}}{\pgfqpoint{3.945730in}{1.551978in}}%
\pgfpathclose%
\pgfusepath{stroke,fill}%
\end{pgfscope}%
\begin{pgfscope}%
\pgfpathrectangle{\pgfqpoint{0.600000in}{0.600000in}}{\pgfqpoint{3.900000in}{3.900000in}}%
\pgfusepath{clip}%
\pgfsetbuttcap%
\pgfsetroundjoin%
\definecolor{currentfill}{rgb}{0.121569,0.466667,0.705882}%
\pgfsetfillcolor{currentfill}%
\pgfsetlinewidth{1.003750pt}%
\definecolor{currentstroke}{rgb}{0.121569,0.466667,0.705882}%
\pgfsetstrokecolor{currentstroke}%
\pgfsetdash{}{0pt}%
\pgfpathmoveto{\pgfqpoint{2.676894in}{3.186750in}}%
\pgfpathcurveto{\pgfqpoint{2.687944in}{3.186750in}}{\pgfqpoint{2.698543in}{3.191140in}}{\pgfqpoint{2.706357in}{3.198954in}}%
\pgfpathcurveto{\pgfqpoint{2.714170in}{3.206767in}}{\pgfqpoint{2.718561in}{3.217366in}}{\pgfqpoint{2.718561in}{3.228417in}}%
\pgfpathcurveto{\pgfqpoint{2.718561in}{3.239467in}}{\pgfqpoint{2.714170in}{3.250066in}}{\pgfqpoint{2.706357in}{3.257879in}}%
\pgfpathcurveto{\pgfqpoint{2.698543in}{3.265693in}}{\pgfqpoint{2.687944in}{3.270083in}}{\pgfqpoint{2.676894in}{3.270083in}}%
\pgfpathcurveto{\pgfqpoint{2.665844in}{3.270083in}}{\pgfqpoint{2.655245in}{3.265693in}}{\pgfqpoint{2.647431in}{3.257879in}}%
\pgfpathcurveto{\pgfqpoint{2.639618in}{3.250066in}}{\pgfqpoint{2.635227in}{3.239467in}}{\pgfqpoint{2.635227in}{3.228417in}}%
\pgfpathcurveto{\pgfqpoint{2.635227in}{3.217366in}}{\pgfqpoint{2.639618in}{3.206767in}}{\pgfqpoint{2.647431in}{3.198954in}}%
\pgfpathcurveto{\pgfqpoint{2.655245in}{3.191140in}}{\pgfqpoint{2.665844in}{3.186750in}}{\pgfqpoint{2.676894in}{3.186750in}}%
\pgfpathclose%
\pgfusepath{stroke,fill}%
\end{pgfscope}%
\begin{pgfscope}%
\pgfpathrectangle{\pgfqpoint{0.600000in}{0.600000in}}{\pgfqpoint{3.900000in}{3.900000in}}%
\pgfusepath{clip}%
\pgfsetbuttcap%
\pgfsetroundjoin%
\definecolor{currentfill}{rgb}{0.121569,0.466667,0.705882}%
\pgfsetfillcolor{currentfill}%
\pgfsetlinewidth{1.003750pt}%
\definecolor{currentstroke}{rgb}{0.121569,0.466667,0.705882}%
\pgfsetstrokecolor{currentstroke}%
\pgfsetdash{}{0pt}%
\pgfpathmoveto{\pgfqpoint{2.202233in}{2.777074in}}%
\pgfpathcurveto{\pgfqpoint{2.213284in}{2.777074in}}{\pgfqpoint{2.223883in}{2.781464in}}{\pgfqpoint{2.231696in}{2.789278in}}%
\pgfpathcurveto{\pgfqpoint{2.239510in}{2.797092in}}{\pgfqpoint{2.243900in}{2.807691in}}{\pgfqpoint{2.243900in}{2.818741in}}%
\pgfpathcurveto{\pgfqpoint{2.243900in}{2.829791in}}{\pgfqpoint{2.239510in}{2.840390in}}{\pgfqpoint{2.231696in}{2.848203in}}%
\pgfpathcurveto{\pgfqpoint{2.223883in}{2.856017in}}{\pgfqpoint{2.213284in}{2.860407in}}{\pgfqpoint{2.202233in}{2.860407in}}%
\pgfpathcurveto{\pgfqpoint{2.191183in}{2.860407in}}{\pgfqpoint{2.180584in}{2.856017in}}{\pgfqpoint{2.172771in}{2.848203in}}%
\pgfpathcurveto{\pgfqpoint{2.164957in}{2.840390in}}{\pgfqpoint{2.160567in}{2.829791in}}{\pgfqpoint{2.160567in}{2.818741in}}%
\pgfpathcurveto{\pgfqpoint{2.160567in}{2.807691in}}{\pgfqpoint{2.164957in}{2.797092in}}{\pgfqpoint{2.172771in}{2.789278in}}%
\pgfpathcurveto{\pgfqpoint{2.180584in}{2.781464in}}{\pgfqpoint{2.191183in}{2.777074in}}{\pgfqpoint{2.202233in}{2.777074in}}%
\pgfpathclose%
\pgfusepath{stroke,fill}%
\end{pgfscope}%
\begin{pgfscope}%
\pgfpathrectangle{\pgfqpoint{0.600000in}{0.600000in}}{\pgfqpoint{3.900000in}{3.900000in}}%
\pgfusepath{clip}%
\pgfsetbuttcap%
\pgfsetroundjoin%
\definecolor{currentfill}{rgb}{0.121569,0.466667,0.705882}%
\pgfsetfillcolor{currentfill}%
\pgfsetlinewidth{1.003750pt}%
\definecolor{currentstroke}{rgb}{0.121569,0.466667,0.705882}%
\pgfsetstrokecolor{currentstroke}%
\pgfsetdash{}{0pt}%
\pgfpathmoveto{\pgfqpoint{1.673885in}{2.258674in}}%
\pgfpathcurveto{\pgfqpoint{1.684935in}{2.258674in}}{\pgfqpoint{1.695534in}{2.263064in}}{\pgfqpoint{1.703348in}{2.270878in}}%
\pgfpathcurveto{\pgfqpoint{1.711161in}{2.278692in}}{\pgfqpoint{1.715551in}{2.289291in}}{\pgfqpoint{1.715551in}{2.300341in}}%
\pgfpathcurveto{\pgfqpoint{1.715551in}{2.311391in}}{\pgfqpoint{1.711161in}{2.321990in}}{\pgfqpoint{1.703348in}{2.329804in}}%
\pgfpathcurveto{\pgfqpoint{1.695534in}{2.337617in}}{\pgfqpoint{1.684935in}{2.342008in}}{\pgfqpoint{1.673885in}{2.342008in}}%
\pgfpathcurveto{\pgfqpoint{1.662835in}{2.342008in}}{\pgfqpoint{1.652236in}{2.337617in}}{\pgfqpoint{1.644422in}{2.329804in}}%
\pgfpathcurveto{\pgfqpoint{1.636608in}{2.321990in}}{\pgfqpoint{1.632218in}{2.311391in}}{\pgfqpoint{1.632218in}{2.300341in}}%
\pgfpathcurveto{\pgfqpoint{1.632218in}{2.289291in}}{\pgfqpoint{1.636608in}{2.278692in}}{\pgfqpoint{1.644422in}{2.270878in}}%
\pgfpathcurveto{\pgfqpoint{1.652236in}{2.263064in}}{\pgfqpoint{1.662835in}{2.258674in}}{\pgfqpoint{1.673885in}{2.258674in}}%
\pgfpathclose%
\pgfusepath{stroke,fill}%
\end{pgfscope}%
\begin{pgfscope}%
\pgfpathrectangle{\pgfqpoint{0.600000in}{0.600000in}}{\pgfqpoint{3.900000in}{3.900000in}}%
\pgfusepath{clip}%
\pgfsetbuttcap%
\pgfsetroundjoin%
\definecolor{currentfill}{rgb}{0.121569,0.466667,0.705882}%
\pgfsetfillcolor{currentfill}%
\pgfsetlinewidth{1.003750pt}%
\definecolor{currentstroke}{rgb}{0.121569,0.466667,0.705882}%
\pgfsetstrokecolor{currentstroke}%
\pgfsetdash{}{0pt}%
\pgfpathmoveto{\pgfqpoint{2.668784in}{2.546569in}}%
\pgfpathcurveto{\pgfqpoint{2.679834in}{2.546569in}}{\pgfqpoint{2.690433in}{2.550959in}}{\pgfqpoint{2.698247in}{2.558773in}}%
\pgfpathcurveto{\pgfqpoint{2.706060in}{2.566586in}}{\pgfqpoint{2.710451in}{2.577185in}}{\pgfqpoint{2.710451in}{2.588236in}}%
\pgfpathcurveto{\pgfqpoint{2.710451in}{2.599286in}}{\pgfqpoint{2.706060in}{2.609885in}}{\pgfqpoint{2.698247in}{2.617698in}}%
\pgfpathcurveto{\pgfqpoint{2.690433in}{2.625512in}}{\pgfqpoint{2.679834in}{2.629902in}}{\pgfqpoint{2.668784in}{2.629902in}}%
\pgfpathcurveto{\pgfqpoint{2.657734in}{2.629902in}}{\pgfqpoint{2.647135in}{2.625512in}}{\pgfqpoint{2.639321in}{2.617698in}}%
\pgfpathcurveto{\pgfqpoint{2.631507in}{2.609885in}}{\pgfqpoint{2.627117in}{2.599286in}}{\pgfqpoint{2.627117in}{2.588236in}}%
\pgfpathcurveto{\pgfqpoint{2.627117in}{2.577185in}}{\pgfqpoint{2.631507in}{2.566586in}}{\pgfqpoint{2.639321in}{2.558773in}}%
\pgfpathcurveto{\pgfqpoint{2.647135in}{2.550959in}}{\pgfqpoint{2.657734in}{2.546569in}}{\pgfqpoint{2.668784in}{2.546569in}}%
\pgfpathclose%
\pgfusepath{stroke,fill}%
\end{pgfscope}%
\begin{pgfscope}%
\pgfpathrectangle{\pgfqpoint{0.600000in}{0.600000in}}{\pgfqpoint{3.900000in}{3.900000in}}%
\pgfusepath{clip}%
\pgfsetbuttcap%
\pgfsetroundjoin%
\definecolor{currentfill}{rgb}{0.121569,0.466667,0.705882}%
\pgfsetfillcolor{currentfill}%
\pgfsetlinewidth{1.003750pt}%
\definecolor{currentstroke}{rgb}{0.121569,0.466667,0.705882}%
\pgfsetstrokecolor{currentstroke}%
\pgfsetdash{}{0pt}%
\pgfpathmoveto{\pgfqpoint{3.143681in}{2.606507in}}%
\pgfpathcurveto{\pgfqpoint{3.154732in}{2.606507in}}{\pgfqpoint{3.165331in}{2.610898in}}{\pgfqpoint{3.173144in}{2.618711in}}%
\pgfpathcurveto{\pgfqpoint{3.180958in}{2.626525in}}{\pgfqpoint{3.185348in}{2.637124in}}{\pgfqpoint{3.185348in}{2.648174in}}%
\pgfpathcurveto{\pgfqpoint{3.185348in}{2.659224in}}{\pgfqpoint{3.180958in}{2.669823in}}{\pgfqpoint{3.173144in}{2.677637in}}%
\pgfpathcurveto{\pgfqpoint{3.165331in}{2.685450in}}{\pgfqpoint{3.154732in}{2.689841in}}{\pgfqpoint{3.143681in}{2.689841in}}%
\pgfpathcurveto{\pgfqpoint{3.132631in}{2.689841in}}{\pgfqpoint{3.122032in}{2.685450in}}{\pgfqpoint{3.114219in}{2.677637in}}%
\pgfpathcurveto{\pgfqpoint{3.106405in}{2.669823in}}{\pgfqpoint{3.102015in}{2.659224in}}{\pgfqpoint{3.102015in}{2.648174in}}%
\pgfpathcurveto{\pgfqpoint{3.102015in}{2.637124in}}{\pgfqpoint{3.106405in}{2.626525in}}{\pgfqpoint{3.114219in}{2.618711in}}%
\pgfpathcurveto{\pgfqpoint{3.122032in}{2.610898in}}{\pgfqpoint{3.132631in}{2.606507in}}{\pgfqpoint{3.143681in}{2.606507in}}%
\pgfpathclose%
\pgfusepath{stroke,fill}%
\end{pgfscope}%
\begin{pgfscope}%
\pgfpathrectangle{\pgfqpoint{0.600000in}{0.600000in}}{\pgfqpoint{3.900000in}{3.900000in}}%
\pgfusepath{clip}%
\pgfsetbuttcap%
\pgfsetroundjoin%
\definecolor{currentfill}{rgb}{0.121569,0.466667,0.705882}%
\pgfsetfillcolor{currentfill}%
\pgfsetlinewidth{1.003750pt}%
\definecolor{currentstroke}{rgb}{0.121569,0.466667,0.705882}%
\pgfsetstrokecolor{currentstroke}%
\pgfsetdash{}{0pt}%
\pgfpathmoveto{\pgfqpoint{2.306143in}{2.272665in}}%
\pgfpathcurveto{\pgfqpoint{2.317193in}{2.272665in}}{\pgfqpoint{2.327792in}{2.277056in}}{\pgfqpoint{2.335606in}{2.284869in}}%
\pgfpathcurveto{\pgfqpoint{2.343419in}{2.292683in}}{\pgfqpoint{2.347810in}{2.303282in}}{\pgfqpoint{2.347810in}{2.314332in}}%
\pgfpathcurveto{\pgfqpoint{2.347810in}{2.325382in}}{\pgfqpoint{2.343419in}{2.335981in}}{\pgfqpoint{2.335606in}{2.343795in}}%
\pgfpathcurveto{\pgfqpoint{2.327792in}{2.351608in}}{\pgfqpoint{2.317193in}{2.355999in}}{\pgfqpoint{2.306143in}{2.355999in}}%
\pgfpathcurveto{\pgfqpoint{2.295093in}{2.355999in}}{\pgfqpoint{2.284494in}{2.351608in}}{\pgfqpoint{2.276680in}{2.343795in}}%
\pgfpathcurveto{\pgfqpoint{2.268866in}{2.335981in}}{\pgfqpoint{2.264476in}{2.325382in}}{\pgfqpoint{2.264476in}{2.314332in}}%
\pgfpathcurveto{\pgfqpoint{2.264476in}{2.303282in}}{\pgfqpoint{2.268866in}{2.292683in}}{\pgfqpoint{2.276680in}{2.284869in}}%
\pgfpathcurveto{\pgfqpoint{2.284494in}{2.277056in}}{\pgfqpoint{2.295093in}{2.272665in}}{\pgfqpoint{2.306143in}{2.272665in}}%
\pgfpathclose%
\pgfusepath{stroke,fill}%
\end{pgfscope}%
\begin{pgfscope}%
\pgfpathrectangle{\pgfqpoint{0.600000in}{0.600000in}}{\pgfqpoint{3.900000in}{3.900000in}}%
\pgfusepath{clip}%
\pgfsetbuttcap%
\pgfsetroundjoin%
\definecolor{currentfill}{rgb}{0.121569,0.466667,0.705882}%
\pgfsetfillcolor{currentfill}%
\pgfsetlinewidth{1.003750pt}%
\definecolor{currentstroke}{rgb}{0.121569,0.466667,0.705882}%
\pgfsetstrokecolor{currentstroke}%
\pgfsetdash{}{0pt}%
\pgfpathmoveto{\pgfqpoint{2.689500in}{2.773338in}}%
\pgfpathcurveto{\pgfqpoint{2.700550in}{2.773338in}}{\pgfqpoint{2.711149in}{2.777728in}}{\pgfqpoint{2.718963in}{2.785542in}}%
\pgfpathcurveto{\pgfqpoint{2.726776in}{2.793355in}}{\pgfqpoint{2.731167in}{2.803954in}}{\pgfqpoint{2.731167in}{2.815005in}}%
\pgfpathcurveto{\pgfqpoint{2.731167in}{2.826055in}}{\pgfqpoint{2.726776in}{2.836654in}}{\pgfqpoint{2.718963in}{2.844467in}}%
\pgfpathcurveto{\pgfqpoint{2.711149in}{2.852281in}}{\pgfqpoint{2.700550in}{2.856671in}}{\pgfqpoint{2.689500in}{2.856671in}}%
\pgfpathcurveto{\pgfqpoint{2.678450in}{2.856671in}}{\pgfqpoint{2.667851in}{2.852281in}}{\pgfqpoint{2.660037in}{2.844467in}}%
\pgfpathcurveto{\pgfqpoint{2.652224in}{2.836654in}}{\pgfqpoint{2.647833in}{2.826055in}}{\pgfqpoint{2.647833in}{2.815005in}}%
\pgfpathcurveto{\pgfqpoint{2.647833in}{2.803954in}}{\pgfqpoint{2.652224in}{2.793355in}}{\pgfqpoint{2.660037in}{2.785542in}}%
\pgfpathcurveto{\pgfqpoint{2.667851in}{2.777728in}}{\pgfqpoint{2.678450in}{2.773338in}}{\pgfqpoint{2.689500in}{2.773338in}}%
\pgfpathclose%
\pgfusepath{stroke,fill}%
\end{pgfscope}%
\begin{pgfscope}%
\pgfpathrectangle{\pgfqpoint{0.600000in}{0.600000in}}{\pgfqpoint{3.900000in}{3.900000in}}%
\pgfusepath{clip}%
\pgfsetbuttcap%
\pgfsetroundjoin%
\definecolor{currentfill}{rgb}{0.121569,0.466667,0.705882}%
\pgfsetfillcolor{currentfill}%
\pgfsetlinewidth{1.003750pt}%
\definecolor{currentstroke}{rgb}{0.121569,0.466667,0.705882}%
\pgfsetstrokecolor{currentstroke}%
\pgfsetdash{}{0pt}%
\pgfpathmoveto{\pgfqpoint{2.478384in}{2.753859in}}%
\pgfpathcurveto{\pgfqpoint{2.489434in}{2.753859in}}{\pgfqpoint{2.500033in}{2.758250in}}{\pgfqpoint{2.507847in}{2.766063in}}%
\pgfpathcurveto{\pgfqpoint{2.515660in}{2.773877in}}{\pgfqpoint{2.520051in}{2.784476in}}{\pgfqpoint{2.520051in}{2.795526in}}%
\pgfpathcurveto{\pgfqpoint{2.520051in}{2.806576in}}{\pgfqpoint{2.515660in}{2.817175in}}{\pgfqpoint{2.507847in}{2.824989in}}%
\pgfpathcurveto{\pgfqpoint{2.500033in}{2.832802in}}{\pgfqpoint{2.489434in}{2.837193in}}{\pgfqpoint{2.478384in}{2.837193in}}%
\pgfpathcurveto{\pgfqpoint{2.467334in}{2.837193in}}{\pgfqpoint{2.456735in}{2.832802in}}{\pgfqpoint{2.448921in}{2.824989in}}%
\pgfpathcurveto{\pgfqpoint{2.441108in}{2.817175in}}{\pgfqpoint{2.436717in}{2.806576in}}{\pgfqpoint{2.436717in}{2.795526in}}%
\pgfpathcurveto{\pgfqpoint{2.436717in}{2.784476in}}{\pgfqpoint{2.441108in}{2.773877in}}{\pgfqpoint{2.448921in}{2.766063in}}%
\pgfpathcurveto{\pgfqpoint{2.456735in}{2.758250in}}{\pgfqpoint{2.467334in}{2.753859in}}{\pgfqpoint{2.478384in}{2.753859in}}%
\pgfpathclose%
\pgfusepath{stroke,fill}%
\end{pgfscope}%
\begin{pgfscope}%
\pgfpathrectangle{\pgfqpoint{0.600000in}{0.600000in}}{\pgfqpoint{3.900000in}{3.900000in}}%
\pgfusepath{clip}%
\pgfsetbuttcap%
\pgfsetroundjoin%
\definecolor{currentfill}{rgb}{0.121569,0.466667,0.705882}%
\pgfsetfillcolor{currentfill}%
\pgfsetlinewidth{1.003750pt}%
\definecolor{currentstroke}{rgb}{0.121569,0.466667,0.705882}%
\pgfsetstrokecolor{currentstroke}%
\pgfsetdash{}{0pt}%
\pgfpathmoveto{\pgfqpoint{1.977010in}{1.888299in}}%
\pgfpathcurveto{\pgfqpoint{1.988060in}{1.888299in}}{\pgfqpoint{1.998659in}{1.892689in}}{\pgfqpoint{2.006472in}{1.900503in}}%
\pgfpathcurveto{\pgfqpoint{2.014286in}{1.908316in}}{\pgfqpoint{2.018676in}{1.918915in}}{\pgfqpoint{2.018676in}{1.929966in}}%
\pgfpathcurveto{\pgfqpoint{2.018676in}{1.941016in}}{\pgfqpoint{2.014286in}{1.951615in}}{\pgfqpoint{2.006472in}{1.959428in}}%
\pgfpathcurveto{\pgfqpoint{1.998659in}{1.967242in}}{\pgfqpoint{1.988060in}{1.971632in}}{\pgfqpoint{1.977010in}{1.971632in}}%
\pgfpathcurveto{\pgfqpoint{1.965959in}{1.971632in}}{\pgfqpoint{1.955360in}{1.967242in}}{\pgfqpoint{1.947547in}{1.959428in}}%
\pgfpathcurveto{\pgfqpoint{1.939733in}{1.951615in}}{\pgfqpoint{1.935343in}{1.941016in}}{\pgfqpoint{1.935343in}{1.929966in}}%
\pgfpathcurveto{\pgfqpoint{1.935343in}{1.918915in}}{\pgfqpoint{1.939733in}{1.908316in}}{\pgfqpoint{1.947547in}{1.900503in}}%
\pgfpathcurveto{\pgfqpoint{1.955360in}{1.892689in}}{\pgfqpoint{1.965959in}{1.888299in}}{\pgfqpoint{1.977010in}{1.888299in}}%
\pgfpathclose%
\pgfusepath{stroke,fill}%
\end{pgfscope}%
\begin{pgfscope}%
\pgfpathrectangle{\pgfqpoint{0.600000in}{0.600000in}}{\pgfqpoint{3.900000in}{3.900000in}}%
\pgfusepath{clip}%
\pgfsetbuttcap%
\pgfsetroundjoin%
\definecolor{currentfill}{rgb}{0.121569,0.466667,0.705882}%
\pgfsetfillcolor{currentfill}%
\pgfsetlinewidth{1.003750pt}%
\definecolor{currentstroke}{rgb}{0.121569,0.466667,0.705882}%
\pgfsetstrokecolor{currentstroke}%
\pgfsetdash{}{0pt}%
\pgfpathmoveto{\pgfqpoint{2.757775in}{2.555659in}}%
\pgfpathcurveto{\pgfqpoint{2.768825in}{2.555659in}}{\pgfqpoint{2.779424in}{2.560049in}}{\pgfqpoint{2.787238in}{2.567863in}}%
\pgfpathcurveto{\pgfqpoint{2.795051in}{2.575676in}}{\pgfqpoint{2.799441in}{2.586275in}}{\pgfqpoint{2.799441in}{2.597325in}}%
\pgfpathcurveto{\pgfqpoint{2.799441in}{2.608376in}}{\pgfqpoint{2.795051in}{2.618975in}}{\pgfqpoint{2.787238in}{2.626788in}}%
\pgfpathcurveto{\pgfqpoint{2.779424in}{2.634602in}}{\pgfqpoint{2.768825in}{2.638992in}}{\pgfqpoint{2.757775in}{2.638992in}}%
\pgfpathcurveto{\pgfqpoint{2.746725in}{2.638992in}}{\pgfqpoint{2.736126in}{2.634602in}}{\pgfqpoint{2.728312in}{2.626788in}}%
\pgfpathcurveto{\pgfqpoint{2.720498in}{2.618975in}}{\pgfqpoint{2.716108in}{2.608376in}}{\pgfqpoint{2.716108in}{2.597325in}}%
\pgfpathcurveto{\pgfqpoint{2.716108in}{2.586275in}}{\pgfqpoint{2.720498in}{2.575676in}}{\pgfqpoint{2.728312in}{2.567863in}}%
\pgfpathcurveto{\pgfqpoint{2.736126in}{2.560049in}}{\pgfqpoint{2.746725in}{2.555659in}}{\pgfqpoint{2.757775in}{2.555659in}}%
\pgfpathclose%
\pgfusepath{stroke,fill}%
\end{pgfscope}%
\begin{pgfscope}%
\pgfpathrectangle{\pgfqpoint{0.600000in}{0.600000in}}{\pgfqpoint{3.900000in}{3.900000in}}%
\pgfusepath{clip}%
\pgfsetbuttcap%
\pgfsetroundjoin%
\definecolor{currentfill}{rgb}{0.121569,0.466667,0.705882}%
\pgfsetfillcolor{currentfill}%
\pgfsetlinewidth{1.003750pt}%
\definecolor{currentstroke}{rgb}{0.121569,0.466667,0.705882}%
\pgfsetstrokecolor{currentstroke}%
\pgfsetdash{}{0pt}%
\pgfpathmoveto{\pgfqpoint{2.336764in}{1.940826in}}%
\pgfpathcurveto{\pgfqpoint{2.347814in}{1.940826in}}{\pgfqpoint{2.358413in}{1.945216in}}{\pgfqpoint{2.366227in}{1.953030in}}%
\pgfpathcurveto{\pgfqpoint{2.374040in}{1.960843in}}{\pgfqpoint{2.378431in}{1.971442in}}{\pgfqpoint{2.378431in}{1.982493in}}%
\pgfpathcurveto{\pgfqpoint{2.378431in}{1.993543in}}{\pgfqpoint{2.374040in}{2.004142in}}{\pgfqpoint{2.366227in}{2.011955in}}%
\pgfpathcurveto{\pgfqpoint{2.358413in}{2.019769in}}{\pgfqpoint{2.347814in}{2.024159in}}{\pgfqpoint{2.336764in}{2.024159in}}%
\pgfpathcurveto{\pgfqpoint{2.325714in}{2.024159in}}{\pgfqpoint{2.315115in}{2.019769in}}{\pgfqpoint{2.307301in}{2.011955in}}%
\pgfpathcurveto{\pgfqpoint{2.299487in}{2.004142in}}{\pgfqpoint{2.295097in}{1.993543in}}{\pgfqpoint{2.295097in}{1.982493in}}%
\pgfpathcurveto{\pgfqpoint{2.295097in}{1.971442in}}{\pgfqpoint{2.299487in}{1.960843in}}{\pgfqpoint{2.307301in}{1.953030in}}%
\pgfpathcurveto{\pgfqpoint{2.315115in}{1.945216in}}{\pgfqpoint{2.325714in}{1.940826in}}{\pgfqpoint{2.336764in}{1.940826in}}%
\pgfpathclose%
\pgfusepath{stroke,fill}%
\end{pgfscope}%
\begin{pgfscope}%
\pgfpathrectangle{\pgfqpoint{0.600000in}{0.600000in}}{\pgfqpoint{3.900000in}{3.900000in}}%
\pgfusepath{clip}%
\pgfsetbuttcap%
\pgfsetroundjoin%
\definecolor{currentfill}{rgb}{0.121569,0.466667,0.705882}%
\pgfsetfillcolor{currentfill}%
\pgfsetlinewidth{1.003750pt}%
\definecolor{currentstroke}{rgb}{0.121569,0.466667,0.705882}%
\pgfsetstrokecolor{currentstroke}%
\pgfsetdash{}{0pt}%
\pgfpathmoveto{\pgfqpoint{2.401397in}{2.872296in}}%
\pgfpathcurveto{\pgfqpoint{2.412447in}{2.872296in}}{\pgfqpoint{2.423046in}{2.876687in}}{\pgfqpoint{2.430860in}{2.884500in}}%
\pgfpathcurveto{\pgfqpoint{2.438673in}{2.892314in}}{\pgfqpoint{2.443064in}{2.902913in}}{\pgfqpoint{2.443064in}{2.913963in}}%
\pgfpathcurveto{\pgfqpoint{2.443064in}{2.925013in}}{\pgfqpoint{2.438673in}{2.935612in}}{\pgfqpoint{2.430860in}{2.943426in}}%
\pgfpathcurveto{\pgfqpoint{2.423046in}{2.951240in}}{\pgfqpoint{2.412447in}{2.955630in}}{\pgfqpoint{2.401397in}{2.955630in}}%
\pgfpathcurveto{\pgfqpoint{2.390347in}{2.955630in}}{\pgfqpoint{2.379748in}{2.951240in}}{\pgfqpoint{2.371934in}{2.943426in}}%
\pgfpathcurveto{\pgfqpoint{2.364120in}{2.935612in}}{\pgfqpoint{2.359730in}{2.925013in}}{\pgfqpoint{2.359730in}{2.913963in}}%
\pgfpathcurveto{\pgfqpoint{2.359730in}{2.902913in}}{\pgfqpoint{2.364120in}{2.892314in}}{\pgfqpoint{2.371934in}{2.884500in}}%
\pgfpathcurveto{\pgfqpoint{2.379748in}{2.876687in}}{\pgfqpoint{2.390347in}{2.872296in}}{\pgfqpoint{2.401397in}{2.872296in}}%
\pgfpathclose%
\pgfusepath{stroke,fill}%
\end{pgfscope}%
\begin{pgfscope}%
\pgfpathrectangle{\pgfqpoint{0.600000in}{0.600000in}}{\pgfqpoint{3.900000in}{3.900000in}}%
\pgfusepath{clip}%
\pgfsetbuttcap%
\pgfsetroundjoin%
\definecolor{currentfill}{rgb}{0.121569,0.466667,0.705882}%
\pgfsetfillcolor{currentfill}%
\pgfsetlinewidth{1.003750pt}%
\definecolor{currentstroke}{rgb}{0.121569,0.466667,0.705882}%
\pgfsetstrokecolor{currentstroke}%
\pgfsetdash{}{0pt}%
\pgfpathmoveto{\pgfqpoint{2.794954in}{3.410738in}}%
\pgfpathcurveto{\pgfqpoint{2.806005in}{3.410738in}}{\pgfqpoint{2.816604in}{3.415128in}}{\pgfqpoint{2.824417in}{3.422941in}}%
\pgfpathcurveto{\pgfqpoint{2.832231in}{3.430755in}}{\pgfqpoint{2.836621in}{3.441354in}}{\pgfqpoint{2.836621in}{3.452404in}}%
\pgfpathcurveto{\pgfqpoint{2.836621in}{3.463454in}}{\pgfqpoint{2.832231in}{3.474053in}}{\pgfqpoint{2.824417in}{3.481867in}}%
\pgfpathcurveto{\pgfqpoint{2.816604in}{3.489681in}}{\pgfqpoint{2.806005in}{3.494071in}}{\pgfqpoint{2.794954in}{3.494071in}}%
\pgfpathcurveto{\pgfqpoint{2.783904in}{3.494071in}}{\pgfqpoint{2.773305in}{3.489681in}}{\pgfqpoint{2.765492in}{3.481867in}}%
\pgfpathcurveto{\pgfqpoint{2.757678in}{3.474053in}}{\pgfqpoint{2.753288in}{3.463454in}}{\pgfqpoint{2.753288in}{3.452404in}}%
\pgfpathcurveto{\pgfqpoint{2.753288in}{3.441354in}}{\pgfqpoint{2.757678in}{3.430755in}}{\pgfqpoint{2.765492in}{3.422941in}}%
\pgfpathcurveto{\pgfqpoint{2.773305in}{3.415128in}}{\pgfqpoint{2.783904in}{3.410738in}}{\pgfqpoint{2.794954in}{3.410738in}}%
\pgfpathclose%
\pgfusepath{stroke,fill}%
\end{pgfscope}%
\begin{pgfscope}%
\pgfpathrectangle{\pgfqpoint{0.600000in}{0.600000in}}{\pgfqpoint{3.900000in}{3.900000in}}%
\pgfusepath{clip}%
\pgfsetbuttcap%
\pgfsetroundjoin%
\definecolor{currentfill}{rgb}{0.121569,0.466667,0.705882}%
\pgfsetfillcolor{currentfill}%
\pgfsetlinewidth{1.003750pt}%
\definecolor{currentstroke}{rgb}{0.121569,0.466667,0.705882}%
\pgfsetstrokecolor{currentstroke}%
\pgfsetdash{}{0pt}%
\pgfpathmoveto{\pgfqpoint{2.746008in}{3.050652in}}%
\pgfpathcurveto{\pgfqpoint{2.757058in}{3.050652in}}{\pgfqpoint{2.767657in}{3.055043in}}{\pgfqpoint{2.775471in}{3.062856in}}%
\pgfpathcurveto{\pgfqpoint{2.783284in}{3.070670in}}{\pgfqpoint{2.787674in}{3.081269in}}{\pgfqpoint{2.787674in}{3.092319in}}%
\pgfpathcurveto{\pgfqpoint{2.787674in}{3.103369in}}{\pgfqpoint{2.783284in}{3.113968in}}{\pgfqpoint{2.775471in}{3.121782in}}%
\pgfpathcurveto{\pgfqpoint{2.767657in}{3.129595in}}{\pgfqpoint{2.757058in}{3.133986in}}{\pgfqpoint{2.746008in}{3.133986in}}%
\pgfpathcurveto{\pgfqpoint{2.734958in}{3.133986in}}{\pgfqpoint{2.724359in}{3.129595in}}{\pgfqpoint{2.716545in}{3.121782in}}%
\pgfpathcurveto{\pgfqpoint{2.708731in}{3.113968in}}{\pgfqpoint{2.704341in}{3.103369in}}{\pgfqpoint{2.704341in}{3.092319in}}%
\pgfpathcurveto{\pgfqpoint{2.704341in}{3.081269in}}{\pgfqpoint{2.708731in}{3.070670in}}{\pgfqpoint{2.716545in}{3.062856in}}%
\pgfpathcurveto{\pgfqpoint{2.724359in}{3.055043in}}{\pgfqpoint{2.734958in}{3.050652in}}{\pgfqpoint{2.746008in}{3.050652in}}%
\pgfpathclose%
\pgfusepath{stroke,fill}%
\end{pgfscope}%
\begin{pgfscope}%
\pgfpathrectangle{\pgfqpoint{0.600000in}{0.600000in}}{\pgfqpoint{3.900000in}{3.900000in}}%
\pgfusepath{clip}%
\pgfsetbuttcap%
\pgfsetroundjoin%
\definecolor{currentfill}{rgb}{0.121569,0.466667,0.705882}%
\pgfsetfillcolor{currentfill}%
\pgfsetlinewidth{1.003750pt}%
\definecolor{currentstroke}{rgb}{0.121569,0.466667,0.705882}%
\pgfsetstrokecolor{currentstroke}%
\pgfsetdash{}{0pt}%
\pgfpathmoveto{\pgfqpoint{2.703079in}{2.996311in}}%
\pgfpathcurveto{\pgfqpoint{2.714129in}{2.996311in}}{\pgfqpoint{2.724728in}{3.000701in}}{\pgfqpoint{2.732542in}{3.008514in}}%
\pgfpathcurveto{\pgfqpoint{2.740356in}{3.016328in}}{\pgfqpoint{2.744746in}{3.026927in}}{\pgfqpoint{2.744746in}{3.037977in}}%
\pgfpathcurveto{\pgfqpoint{2.744746in}{3.049027in}}{\pgfqpoint{2.740356in}{3.059626in}}{\pgfqpoint{2.732542in}{3.067440in}}%
\pgfpathcurveto{\pgfqpoint{2.724728in}{3.075254in}}{\pgfqpoint{2.714129in}{3.079644in}}{\pgfqpoint{2.703079in}{3.079644in}}%
\pgfpathcurveto{\pgfqpoint{2.692029in}{3.079644in}}{\pgfqpoint{2.681430in}{3.075254in}}{\pgfqpoint{2.673616in}{3.067440in}}%
\pgfpathcurveto{\pgfqpoint{2.665803in}{3.059626in}}{\pgfqpoint{2.661412in}{3.049027in}}{\pgfqpoint{2.661412in}{3.037977in}}%
\pgfpathcurveto{\pgfqpoint{2.661412in}{3.026927in}}{\pgfqpoint{2.665803in}{3.016328in}}{\pgfqpoint{2.673616in}{3.008514in}}%
\pgfpathcurveto{\pgfqpoint{2.681430in}{3.000701in}}{\pgfqpoint{2.692029in}{2.996311in}}{\pgfqpoint{2.703079in}{2.996311in}}%
\pgfpathclose%
\pgfusepath{stroke,fill}%
\end{pgfscope}%
\begin{pgfscope}%
\pgfpathrectangle{\pgfqpoint{0.600000in}{0.600000in}}{\pgfqpoint{3.900000in}{3.900000in}}%
\pgfusepath{clip}%
\pgfsetbuttcap%
\pgfsetroundjoin%
\definecolor{currentfill}{rgb}{0.121569,0.466667,0.705882}%
\pgfsetfillcolor{currentfill}%
\pgfsetlinewidth{1.003750pt}%
\definecolor{currentstroke}{rgb}{0.121569,0.466667,0.705882}%
\pgfsetstrokecolor{currentstroke}%
\pgfsetdash{}{0pt}%
\pgfpathmoveto{\pgfqpoint{2.882910in}{2.521926in}}%
\pgfpathcurveto{\pgfqpoint{2.893960in}{2.521926in}}{\pgfqpoint{2.904559in}{2.526316in}}{\pgfqpoint{2.912373in}{2.534130in}}%
\pgfpathcurveto{\pgfqpoint{2.920186in}{2.541944in}}{\pgfqpoint{2.924576in}{2.552543in}}{\pgfqpoint{2.924576in}{2.563593in}}%
\pgfpathcurveto{\pgfqpoint{2.924576in}{2.574643in}}{\pgfqpoint{2.920186in}{2.585242in}}{\pgfqpoint{2.912373in}{2.593055in}}%
\pgfpathcurveto{\pgfqpoint{2.904559in}{2.600869in}}{\pgfqpoint{2.893960in}{2.605259in}}{\pgfqpoint{2.882910in}{2.605259in}}%
\pgfpathcurveto{\pgfqpoint{2.871860in}{2.605259in}}{\pgfqpoint{2.861261in}{2.600869in}}{\pgfqpoint{2.853447in}{2.593055in}}%
\pgfpathcurveto{\pgfqpoint{2.845633in}{2.585242in}}{\pgfqpoint{2.841243in}{2.574643in}}{\pgfqpoint{2.841243in}{2.563593in}}%
\pgfpathcurveto{\pgfqpoint{2.841243in}{2.552543in}}{\pgfqpoint{2.845633in}{2.541944in}}{\pgfqpoint{2.853447in}{2.534130in}}%
\pgfpathcurveto{\pgfqpoint{2.861261in}{2.526316in}}{\pgfqpoint{2.871860in}{2.521926in}}{\pgfqpoint{2.882910in}{2.521926in}}%
\pgfpathclose%
\pgfusepath{stroke,fill}%
\end{pgfscope}%
\begin{pgfscope}%
\pgfpathrectangle{\pgfqpoint{0.600000in}{0.600000in}}{\pgfqpoint{3.900000in}{3.900000in}}%
\pgfusepath{clip}%
\pgfsetbuttcap%
\pgfsetroundjoin%
\definecolor{currentfill}{rgb}{0.121569,0.466667,0.705882}%
\pgfsetfillcolor{currentfill}%
\pgfsetlinewidth{1.003750pt}%
\definecolor{currentstroke}{rgb}{0.121569,0.466667,0.705882}%
\pgfsetstrokecolor{currentstroke}%
\pgfsetdash{}{0pt}%
\pgfpathmoveto{\pgfqpoint{2.451485in}{2.588117in}}%
\pgfpathcurveto{\pgfqpoint{2.462535in}{2.588117in}}{\pgfqpoint{2.473134in}{2.592507in}}{\pgfqpoint{2.480948in}{2.600321in}}%
\pgfpathcurveto{\pgfqpoint{2.488762in}{2.608134in}}{\pgfqpoint{2.493152in}{2.618733in}}{\pgfqpoint{2.493152in}{2.629784in}}%
\pgfpathcurveto{\pgfqpoint{2.493152in}{2.640834in}}{\pgfqpoint{2.488762in}{2.651433in}}{\pgfqpoint{2.480948in}{2.659246in}}%
\pgfpathcurveto{\pgfqpoint{2.473134in}{2.667060in}}{\pgfqpoint{2.462535in}{2.671450in}}{\pgfqpoint{2.451485in}{2.671450in}}%
\pgfpathcurveto{\pgfqpoint{2.440435in}{2.671450in}}{\pgfqpoint{2.429836in}{2.667060in}}{\pgfqpoint{2.422023in}{2.659246in}}%
\pgfpathcurveto{\pgfqpoint{2.414209in}{2.651433in}}{\pgfqpoint{2.409819in}{2.640834in}}{\pgfqpoint{2.409819in}{2.629784in}}%
\pgfpathcurveto{\pgfqpoint{2.409819in}{2.618733in}}{\pgfqpoint{2.414209in}{2.608134in}}{\pgfqpoint{2.422023in}{2.600321in}}%
\pgfpathcurveto{\pgfqpoint{2.429836in}{2.592507in}}{\pgfqpoint{2.440435in}{2.588117in}}{\pgfqpoint{2.451485in}{2.588117in}}%
\pgfpathclose%
\pgfusepath{stroke,fill}%
\end{pgfscope}%
\begin{pgfscope}%
\pgfpathrectangle{\pgfqpoint{0.600000in}{0.600000in}}{\pgfqpoint{3.900000in}{3.900000in}}%
\pgfusepath{clip}%
\pgfsetbuttcap%
\pgfsetroundjoin%
\definecolor{currentfill}{rgb}{0.121569,0.466667,0.705882}%
\pgfsetfillcolor{currentfill}%
\pgfsetlinewidth{1.003750pt}%
\definecolor{currentstroke}{rgb}{0.121569,0.466667,0.705882}%
\pgfsetstrokecolor{currentstroke}%
\pgfsetdash{}{0pt}%
\pgfpathmoveto{\pgfqpoint{2.510902in}{3.059969in}}%
\pgfpathcurveto{\pgfqpoint{2.521952in}{3.059969in}}{\pgfqpoint{2.532551in}{3.064359in}}{\pgfqpoint{2.540364in}{3.072172in}}%
\pgfpathcurveto{\pgfqpoint{2.548178in}{3.079986in}}{\pgfqpoint{2.552568in}{3.090585in}}{\pgfqpoint{2.552568in}{3.101635in}}%
\pgfpathcurveto{\pgfqpoint{2.552568in}{3.112685in}}{\pgfqpoint{2.548178in}{3.123284in}}{\pgfqpoint{2.540364in}{3.131098in}}%
\pgfpathcurveto{\pgfqpoint{2.532551in}{3.138912in}}{\pgfqpoint{2.521952in}{3.143302in}}{\pgfqpoint{2.510902in}{3.143302in}}%
\pgfpathcurveto{\pgfqpoint{2.499851in}{3.143302in}}{\pgfqpoint{2.489252in}{3.138912in}}{\pgfqpoint{2.481439in}{3.131098in}}%
\pgfpathcurveto{\pgfqpoint{2.473625in}{3.123284in}}{\pgfqpoint{2.469235in}{3.112685in}}{\pgfqpoint{2.469235in}{3.101635in}}%
\pgfpathcurveto{\pgfqpoint{2.469235in}{3.090585in}}{\pgfqpoint{2.473625in}{3.079986in}}{\pgfqpoint{2.481439in}{3.072172in}}%
\pgfpathcurveto{\pgfqpoint{2.489252in}{3.064359in}}{\pgfqpoint{2.499851in}{3.059969in}}{\pgfqpoint{2.510902in}{3.059969in}}%
\pgfpathclose%
\pgfusepath{stroke,fill}%
\end{pgfscope}%
\begin{pgfscope}%
\pgfpathrectangle{\pgfqpoint{0.600000in}{0.600000in}}{\pgfqpoint{3.900000in}{3.900000in}}%
\pgfusepath{clip}%
\pgfsetbuttcap%
\pgfsetroundjoin%
\definecolor{currentfill}{rgb}{0.121569,0.466667,0.705882}%
\pgfsetfillcolor{currentfill}%
\pgfsetlinewidth{1.003750pt}%
\definecolor{currentstroke}{rgb}{0.121569,0.466667,0.705882}%
\pgfsetstrokecolor{currentstroke}%
\pgfsetdash{}{0pt}%
\pgfpathmoveto{\pgfqpoint{3.320112in}{3.105446in}}%
\pgfpathcurveto{\pgfqpoint{3.331162in}{3.105446in}}{\pgfqpoint{3.341762in}{3.109837in}}{\pgfqpoint{3.349575in}{3.117650in}}%
\pgfpathcurveto{\pgfqpoint{3.357389in}{3.125464in}}{\pgfqpoint{3.361779in}{3.136063in}}{\pgfqpoint{3.361779in}{3.147113in}}%
\pgfpathcurveto{\pgfqpoint{3.361779in}{3.158163in}}{\pgfqpoint{3.357389in}{3.168762in}}{\pgfqpoint{3.349575in}{3.176576in}}%
\pgfpathcurveto{\pgfqpoint{3.341762in}{3.184389in}}{\pgfqpoint{3.331162in}{3.188780in}}{\pgfqpoint{3.320112in}{3.188780in}}%
\pgfpathcurveto{\pgfqpoint{3.309062in}{3.188780in}}{\pgfqpoint{3.298463in}{3.184389in}}{\pgfqpoint{3.290650in}{3.176576in}}%
\pgfpathcurveto{\pgfqpoint{3.282836in}{3.168762in}}{\pgfqpoint{3.278446in}{3.158163in}}{\pgfqpoint{3.278446in}{3.147113in}}%
\pgfpathcurveto{\pgfqpoint{3.278446in}{3.136063in}}{\pgfqpoint{3.282836in}{3.125464in}}{\pgfqpoint{3.290650in}{3.117650in}}%
\pgfpathcurveto{\pgfqpoint{3.298463in}{3.109837in}}{\pgfqpoint{3.309062in}{3.105446in}}{\pgfqpoint{3.320112in}{3.105446in}}%
\pgfpathclose%
\pgfusepath{stroke,fill}%
\end{pgfscope}%
\begin{pgfscope}%
\pgfpathrectangle{\pgfqpoint{0.600000in}{0.600000in}}{\pgfqpoint{3.900000in}{3.900000in}}%
\pgfusepath{clip}%
\pgfsetbuttcap%
\pgfsetroundjoin%
\definecolor{currentfill}{rgb}{0.121569,0.466667,0.705882}%
\pgfsetfillcolor{currentfill}%
\pgfsetlinewidth{1.003750pt}%
\definecolor{currentstroke}{rgb}{0.121569,0.466667,0.705882}%
\pgfsetstrokecolor{currentstroke}%
\pgfsetdash{}{0pt}%
\pgfpathmoveto{\pgfqpoint{2.126737in}{2.391562in}}%
\pgfpathcurveto{\pgfqpoint{2.137787in}{2.391562in}}{\pgfqpoint{2.148386in}{2.395952in}}{\pgfqpoint{2.156200in}{2.403766in}}%
\pgfpathcurveto{\pgfqpoint{2.164014in}{2.411579in}}{\pgfqpoint{2.168404in}{2.422178in}}{\pgfqpoint{2.168404in}{2.433228in}}%
\pgfpathcurveto{\pgfqpoint{2.168404in}{2.444279in}}{\pgfqpoint{2.164014in}{2.454878in}}{\pgfqpoint{2.156200in}{2.462691in}}%
\pgfpathcurveto{\pgfqpoint{2.148386in}{2.470505in}}{\pgfqpoint{2.137787in}{2.474895in}}{\pgfqpoint{2.126737in}{2.474895in}}%
\pgfpathcurveto{\pgfqpoint{2.115687in}{2.474895in}}{\pgfqpoint{2.105088in}{2.470505in}}{\pgfqpoint{2.097274in}{2.462691in}}%
\pgfpathcurveto{\pgfqpoint{2.089461in}{2.454878in}}{\pgfqpoint{2.085071in}{2.444279in}}{\pgfqpoint{2.085071in}{2.433228in}}%
\pgfpathcurveto{\pgfqpoint{2.085071in}{2.422178in}}{\pgfqpoint{2.089461in}{2.411579in}}{\pgfqpoint{2.097274in}{2.403766in}}%
\pgfpathcurveto{\pgfqpoint{2.105088in}{2.395952in}}{\pgfqpoint{2.115687in}{2.391562in}}{\pgfqpoint{2.126737in}{2.391562in}}%
\pgfpathclose%
\pgfusepath{stroke,fill}%
\end{pgfscope}%
\begin{pgfscope}%
\pgfpathrectangle{\pgfqpoint{0.600000in}{0.600000in}}{\pgfqpoint{3.900000in}{3.900000in}}%
\pgfusepath{clip}%
\pgfsetbuttcap%
\pgfsetroundjoin%
\definecolor{currentfill}{rgb}{0.121569,0.466667,0.705882}%
\pgfsetfillcolor{currentfill}%
\pgfsetlinewidth{1.003750pt}%
\definecolor{currentstroke}{rgb}{0.121569,0.466667,0.705882}%
\pgfsetstrokecolor{currentstroke}%
\pgfsetdash{}{0pt}%
\pgfpathmoveto{\pgfqpoint{2.420325in}{2.069810in}}%
\pgfpathcurveto{\pgfqpoint{2.431375in}{2.069810in}}{\pgfqpoint{2.441974in}{2.074200in}}{\pgfqpoint{2.449788in}{2.082014in}}%
\pgfpathcurveto{\pgfqpoint{2.457601in}{2.089827in}}{\pgfqpoint{2.461992in}{2.100426in}}{\pgfqpoint{2.461992in}{2.111477in}}%
\pgfpathcurveto{\pgfqpoint{2.461992in}{2.122527in}}{\pgfqpoint{2.457601in}{2.133126in}}{\pgfqpoint{2.449788in}{2.140939in}}%
\pgfpathcurveto{\pgfqpoint{2.441974in}{2.148753in}}{\pgfqpoint{2.431375in}{2.153143in}}{\pgfqpoint{2.420325in}{2.153143in}}%
\pgfpathcurveto{\pgfqpoint{2.409275in}{2.153143in}}{\pgfqpoint{2.398676in}{2.148753in}}{\pgfqpoint{2.390862in}{2.140939in}}%
\pgfpathcurveto{\pgfqpoint{2.383049in}{2.133126in}}{\pgfqpoint{2.378658in}{2.122527in}}{\pgfqpoint{2.378658in}{2.111477in}}%
\pgfpathcurveto{\pgfqpoint{2.378658in}{2.100426in}}{\pgfqpoint{2.383049in}{2.089827in}}{\pgfqpoint{2.390862in}{2.082014in}}%
\pgfpathcurveto{\pgfqpoint{2.398676in}{2.074200in}}{\pgfqpoint{2.409275in}{2.069810in}}{\pgfqpoint{2.420325in}{2.069810in}}%
\pgfpathclose%
\pgfusepath{stroke,fill}%
\end{pgfscope}%
\begin{pgfscope}%
\pgfpathrectangle{\pgfqpoint{0.600000in}{0.600000in}}{\pgfqpoint{3.900000in}{3.900000in}}%
\pgfusepath{clip}%
\pgfsetbuttcap%
\pgfsetroundjoin%
\definecolor{currentfill}{rgb}{0.121569,0.466667,0.705882}%
\pgfsetfillcolor{currentfill}%
\pgfsetlinewidth{1.003750pt}%
\definecolor{currentstroke}{rgb}{0.121569,0.466667,0.705882}%
\pgfsetstrokecolor{currentstroke}%
\pgfsetdash{}{0pt}%
\pgfpathmoveto{\pgfqpoint{2.391135in}{2.004054in}}%
\pgfpathcurveto{\pgfqpoint{2.402185in}{2.004054in}}{\pgfqpoint{2.412784in}{2.008445in}}{\pgfqpoint{2.420597in}{2.016258in}}%
\pgfpathcurveto{\pgfqpoint{2.428411in}{2.024072in}}{\pgfqpoint{2.432801in}{2.034671in}}{\pgfqpoint{2.432801in}{2.045721in}}%
\pgfpathcurveto{\pgfqpoint{2.432801in}{2.056771in}}{\pgfqpoint{2.428411in}{2.067370in}}{\pgfqpoint{2.420597in}{2.075184in}}%
\pgfpathcurveto{\pgfqpoint{2.412784in}{2.082997in}}{\pgfqpoint{2.402185in}{2.087388in}}{\pgfqpoint{2.391135in}{2.087388in}}%
\pgfpathcurveto{\pgfqpoint{2.380085in}{2.087388in}}{\pgfqpoint{2.369486in}{2.082997in}}{\pgfqpoint{2.361672in}{2.075184in}}%
\pgfpathcurveto{\pgfqpoint{2.353858in}{2.067370in}}{\pgfqpoint{2.349468in}{2.056771in}}{\pgfqpoint{2.349468in}{2.045721in}}%
\pgfpathcurveto{\pgfqpoint{2.349468in}{2.034671in}}{\pgfqpoint{2.353858in}{2.024072in}}{\pgfqpoint{2.361672in}{2.016258in}}%
\pgfpathcurveto{\pgfqpoint{2.369486in}{2.008445in}}{\pgfqpoint{2.380085in}{2.004054in}}{\pgfqpoint{2.391135in}{2.004054in}}%
\pgfpathclose%
\pgfusepath{stroke,fill}%
\end{pgfscope}%
\begin{pgfscope}%
\pgfpathrectangle{\pgfqpoint{0.600000in}{0.600000in}}{\pgfqpoint{3.900000in}{3.900000in}}%
\pgfusepath{clip}%
\pgfsetbuttcap%
\pgfsetroundjoin%
\definecolor{currentfill}{rgb}{0.121569,0.466667,0.705882}%
\pgfsetfillcolor{currentfill}%
\pgfsetlinewidth{1.003750pt}%
\definecolor{currentstroke}{rgb}{0.121569,0.466667,0.705882}%
\pgfsetstrokecolor{currentstroke}%
\pgfsetdash{}{0pt}%
\pgfpathmoveto{\pgfqpoint{2.763118in}{2.233597in}}%
\pgfpathcurveto{\pgfqpoint{2.774168in}{2.233597in}}{\pgfqpoint{2.784767in}{2.237988in}}{\pgfqpoint{2.792580in}{2.245801in}}%
\pgfpathcurveto{\pgfqpoint{2.800394in}{2.253615in}}{\pgfqpoint{2.804784in}{2.264214in}}{\pgfqpoint{2.804784in}{2.275264in}}%
\pgfpathcurveto{\pgfqpoint{2.804784in}{2.286314in}}{\pgfqpoint{2.800394in}{2.296913in}}{\pgfqpoint{2.792580in}{2.304727in}}%
\pgfpathcurveto{\pgfqpoint{2.784767in}{2.312540in}}{\pgfqpoint{2.774168in}{2.316931in}}{\pgfqpoint{2.763118in}{2.316931in}}%
\pgfpathcurveto{\pgfqpoint{2.752068in}{2.316931in}}{\pgfqpoint{2.741469in}{2.312540in}}{\pgfqpoint{2.733655in}{2.304727in}}%
\pgfpathcurveto{\pgfqpoint{2.725841in}{2.296913in}}{\pgfqpoint{2.721451in}{2.286314in}}{\pgfqpoint{2.721451in}{2.275264in}}%
\pgfpathcurveto{\pgfqpoint{2.721451in}{2.264214in}}{\pgfqpoint{2.725841in}{2.253615in}}{\pgfqpoint{2.733655in}{2.245801in}}%
\pgfpathcurveto{\pgfqpoint{2.741469in}{2.237988in}}{\pgfqpoint{2.752068in}{2.233597in}}{\pgfqpoint{2.763118in}{2.233597in}}%
\pgfpathclose%
\pgfusepath{stroke,fill}%
\end{pgfscope}%
\begin{pgfscope}%
\pgfpathrectangle{\pgfqpoint{0.600000in}{0.600000in}}{\pgfqpoint{3.900000in}{3.900000in}}%
\pgfusepath{clip}%
\pgfsetbuttcap%
\pgfsetroundjoin%
\definecolor{currentfill}{rgb}{0.121569,0.466667,0.705882}%
\pgfsetfillcolor{currentfill}%
\pgfsetlinewidth{1.003750pt}%
\definecolor{currentstroke}{rgb}{0.121569,0.466667,0.705882}%
\pgfsetstrokecolor{currentstroke}%
\pgfsetdash{}{0pt}%
\pgfpathmoveto{\pgfqpoint{3.243348in}{2.380043in}}%
\pgfpathcurveto{\pgfqpoint{3.254398in}{2.380043in}}{\pgfqpoint{3.264997in}{2.384433in}}{\pgfqpoint{3.272811in}{2.392247in}}%
\pgfpathcurveto{\pgfqpoint{3.280624in}{2.400061in}}{\pgfqpoint{3.285015in}{2.410660in}}{\pgfqpoint{3.285015in}{2.421710in}}%
\pgfpathcurveto{\pgfqpoint{3.285015in}{2.432760in}}{\pgfqpoint{3.280624in}{2.443359in}}{\pgfqpoint{3.272811in}{2.451173in}}%
\pgfpathcurveto{\pgfqpoint{3.264997in}{2.458986in}}{\pgfqpoint{3.254398in}{2.463377in}}{\pgfqpoint{3.243348in}{2.463377in}}%
\pgfpathcurveto{\pgfqpoint{3.232298in}{2.463377in}}{\pgfqpoint{3.221699in}{2.458986in}}{\pgfqpoint{3.213885in}{2.451173in}}%
\pgfpathcurveto{\pgfqpoint{3.206072in}{2.443359in}}{\pgfqpoint{3.201681in}{2.432760in}}{\pgfqpoint{3.201681in}{2.421710in}}%
\pgfpathcurveto{\pgfqpoint{3.201681in}{2.410660in}}{\pgfqpoint{3.206072in}{2.400061in}}{\pgfqpoint{3.213885in}{2.392247in}}%
\pgfpathcurveto{\pgfqpoint{3.221699in}{2.384433in}}{\pgfqpoint{3.232298in}{2.380043in}}{\pgfqpoint{3.243348in}{2.380043in}}%
\pgfpathclose%
\pgfusepath{stroke,fill}%
\end{pgfscope}%
\begin{pgfscope}%
\pgfpathrectangle{\pgfqpoint{0.600000in}{0.600000in}}{\pgfqpoint{3.900000in}{3.900000in}}%
\pgfusepath{clip}%
\pgfsetbuttcap%
\pgfsetroundjoin%
\definecolor{currentfill}{rgb}{0.121569,0.466667,0.705882}%
\pgfsetfillcolor{currentfill}%
\pgfsetlinewidth{1.003750pt}%
\definecolor{currentstroke}{rgb}{0.121569,0.466667,0.705882}%
\pgfsetstrokecolor{currentstroke}%
\pgfsetdash{}{0pt}%
\pgfpathmoveto{\pgfqpoint{2.555490in}{1.962801in}}%
\pgfpathcurveto{\pgfqpoint{2.566540in}{1.962801in}}{\pgfqpoint{2.577139in}{1.967191in}}{\pgfqpoint{2.584953in}{1.975005in}}%
\pgfpathcurveto{\pgfqpoint{2.592766in}{1.982819in}}{\pgfqpoint{2.597156in}{1.993418in}}{\pgfqpoint{2.597156in}{2.004468in}}%
\pgfpathcurveto{\pgfqpoint{2.597156in}{2.015518in}}{\pgfqpoint{2.592766in}{2.026117in}}{\pgfqpoint{2.584953in}{2.033931in}}%
\pgfpathcurveto{\pgfqpoint{2.577139in}{2.041744in}}{\pgfqpoint{2.566540in}{2.046134in}}{\pgfqpoint{2.555490in}{2.046134in}}%
\pgfpathcurveto{\pgfqpoint{2.544440in}{2.046134in}}{\pgfqpoint{2.533841in}{2.041744in}}{\pgfqpoint{2.526027in}{2.033931in}}%
\pgfpathcurveto{\pgfqpoint{2.518213in}{2.026117in}}{\pgfqpoint{2.513823in}{2.015518in}}{\pgfqpoint{2.513823in}{2.004468in}}%
\pgfpathcurveto{\pgfqpoint{2.513823in}{1.993418in}}{\pgfqpoint{2.518213in}{1.982819in}}{\pgfqpoint{2.526027in}{1.975005in}}%
\pgfpathcurveto{\pgfqpoint{2.533841in}{1.967191in}}{\pgfqpoint{2.544440in}{1.962801in}}{\pgfqpoint{2.555490in}{1.962801in}}%
\pgfpathclose%
\pgfusepath{stroke,fill}%
\end{pgfscope}%
\begin{pgfscope}%
\pgfpathrectangle{\pgfqpoint{0.600000in}{0.600000in}}{\pgfqpoint{3.900000in}{3.900000in}}%
\pgfusepath{clip}%
\pgfsetbuttcap%
\pgfsetroundjoin%
\definecolor{currentfill}{rgb}{0.121569,0.466667,0.705882}%
\pgfsetfillcolor{currentfill}%
\pgfsetlinewidth{1.003750pt}%
\definecolor{currentstroke}{rgb}{0.121569,0.466667,0.705882}%
\pgfsetstrokecolor{currentstroke}%
\pgfsetdash{}{0pt}%
\pgfpathmoveto{\pgfqpoint{2.979607in}{3.014577in}}%
\pgfpathcurveto{\pgfqpoint{2.990657in}{3.014577in}}{\pgfqpoint{3.001256in}{3.018968in}}{\pgfqpoint{3.009069in}{3.026781in}}%
\pgfpathcurveto{\pgfqpoint{3.016883in}{3.034595in}}{\pgfqpoint{3.021273in}{3.045194in}}{\pgfqpoint{3.021273in}{3.056244in}}%
\pgfpathcurveto{\pgfqpoint{3.021273in}{3.067294in}}{\pgfqpoint{3.016883in}{3.077893in}}{\pgfqpoint{3.009069in}{3.085707in}}%
\pgfpathcurveto{\pgfqpoint{3.001256in}{3.093520in}}{\pgfqpoint{2.990657in}{3.097911in}}{\pgfqpoint{2.979607in}{3.097911in}}%
\pgfpathcurveto{\pgfqpoint{2.968556in}{3.097911in}}{\pgfqpoint{2.957957in}{3.093520in}}{\pgfqpoint{2.950144in}{3.085707in}}%
\pgfpathcurveto{\pgfqpoint{2.942330in}{3.077893in}}{\pgfqpoint{2.937940in}{3.067294in}}{\pgfqpoint{2.937940in}{3.056244in}}%
\pgfpathcurveto{\pgfqpoint{2.937940in}{3.045194in}}{\pgfqpoint{2.942330in}{3.034595in}}{\pgfqpoint{2.950144in}{3.026781in}}%
\pgfpathcurveto{\pgfqpoint{2.957957in}{3.018968in}}{\pgfqpoint{2.968556in}{3.014577in}}{\pgfqpoint{2.979607in}{3.014577in}}%
\pgfpathclose%
\pgfusepath{stroke,fill}%
\end{pgfscope}%
\begin{pgfscope}%
\pgfpathrectangle{\pgfqpoint{0.600000in}{0.600000in}}{\pgfqpoint{3.900000in}{3.900000in}}%
\pgfusepath{clip}%
\pgfsetbuttcap%
\pgfsetroundjoin%
\definecolor{currentfill}{rgb}{0.121569,0.466667,0.705882}%
\pgfsetfillcolor{currentfill}%
\pgfsetlinewidth{1.003750pt}%
\definecolor{currentstroke}{rgb}{0.121569,0.466667,0.705882}%
\pgfsetstrokecolor{currentstroke}%
\pgfsetdash{}{0pt}%
\pgfpathmoveto{\pgfqpoint{1.528665in}{1.737829in}}%
\pgfpathcurveto{\pgfqpoint{1.539716in}{1.737829in}}{\pgfqpoint{1.550315in}{1.742219in}}{\pgfqpoint{1.558128in}{1.750032in}}%
\pgfpathcurveto{\pgfqpoint{1.565942in}{1.757846in}}{\pgfqpoint{1.570332in}{1.768445in}}{\pgfqpoint{1.570332in}{1.779495in}}%
\pgfpathcurveto{\pgfqpoint{1.570332in}{1.790545in}}{\pgfqpoint{1.565942in}{1.801144in}}{\pgfqpoint{1.558128in}{1.808958in}}%
\pgfpathcurveto{\pgfqpoint{1.550315in}{1.816772in}}{\pgfqpoint{1.539716in}{1.821162in}}{\pgfqpoint{1.528665in}{1.821162in}}%
\pgfpathcurveto{\pgfqpoint{1.517615in}{1.821162in}}{\pgfqpoint{1.507016in}{1.816772in}}{\pgfqpoint{1.499203in}{1.808958in}}%
\pgfpathcurveto{\pgfqpoint{1.491389in}{1.801144in}}{\pgfqpoint{1.486999in}{1.790545in}}{\pgfqpoint{1.486999in}{1.779495in}}%
\pgfpathcurveto{\pgfqpoint{1.486999in}{1.768445in}}{\pgfqpoint{1.491389in}{1.757846in}}{\pgfqpoint{1.499203in}{1.750032in}}%
\pgfpathcurveto{\pgfqpoint{1.507016in}{1.742219in}}{\pgfqpoint{1.517615in}{1.737829in}}{\pgfqpoint{1.528665in}{1.737829in}}%
\pgfpathclose%
\pgfusepath{stroke,fill}%
\end{pgfscope}%
\begin{pgfscope}%
\pgfpathrectangle{\pgfqpoint{0.600000in}{0.600000in}}{\pgfqpoint{3.900000in}{3.900000in}}%
\pgfusepath{clip}%
\pgfsetbuttcap%
\pgfsetroundjoin%
\definecolor{currentfill}{rgb}{0.121569,0.466667,0.705882}%
\pgfsetfillcolor{currentfill}%
\pgfsetlinewidth{1.003750pt}%
\definecolor{currentstroke}{rgb}{0.121569,0.466667,0.705882}%
\pgfsetstrokecolor{currentstroke}%
\pgfsetdash{}{0pt}%
\pgfpathmoveto{\pgfqpoint{2.650745in}{2.610428in}}%
\pgfpathcurveto{\pgfqpoint{2.661795in}{2.610428in}}{\pgfqpoint{2.672394in}{2.614819in}}{\pgfqpoint{2.680208in}{2.622632in}}%
\pgfpathcurveto{\pgfqpoint{2.688022in}{2.630446in}}{\pgfqpoint{2.692412in}{2.641045in}}{\pgfqpoint{2.692412in}{2.652095in}}%
\pgfpathcurveto{\pgfqpoint{2.692412in}{2.663145in}}{\pgfqpoint{2.688022in}{2.673744in}}{\pgfqpoint{2.680208in}{2.681558in}}%
\pgfpathcurveto{\pgfqpoint{2.672394in}{2.689371in}}{\pgfqpoint{2.661795in}{2.693762in}}{\pgfqpoint{2.650745in}{2.693762in}}%
\pgfpathcurveto{\pgfqpoint{2.639695in}{2.693762in}}{\pgfqpoint{2.629096in}{2.689371in}}{\pgfqpoint{2.621282in}{2.681558in}}%
\pgfpathcurveto{\pgfqpoint{2.613469in}{2.673744in}}{\pgfqpoint{2.609078in}{2.663145in}}{\pgfqpoint{2.609078in}{2.652095in}}%
\pgfpathcurveto{\pgfqpoint{2.609078in}{2.641045in}}{\pgfqpoint{2.613469in}{2.630446in}}{\pgfqpoint{2.621282in}{2.622632in}}%
\pgfpathcurveto{\pgfqpoint{2.629096in}{2.614819in}}{\pgfqpoint{2.639695in}{2.610428in}}{\pgfqpoint{2.650745in}{2.610428in}}%
\pgfpathclose%
\pgfusepath{stroke,fill}%
\end{pgfscope}%
\begin{pgfscope}%
\pgfpathrectangle{\pgfqpoint{0.600000in}{0.600000in}}{\pgfqpoint{3.900000in}{3.900000in}}%
\pgfusepath{clip}%
\pgfsetbuttcap%
\pgfsetroundjoin%
\definecolor{currentfill}{rgb}{0.121569,0.466667,0.705882}%
\pgfsetfillcolor{currentfill}%
\pgfsetlinewidth{1.003750pt}%
\definecolor{currentstroke}{rgb}{0.121569,0.466667,0.705882}%
\pgfsetstrokecolor{currentstroke}%
\pgfsetdash{}{0pt}%
\pgfpathmoveto{\pgfqpoint{2.592550in}{3.032417in}}%
\pgfpathcurveto{\pgfqpoint{2.603600in}{3.032417in}}{\pgfqpoint{2.614199in}{3.036808in}}{\pgfqpoint{2.622013in}{3.044621in}}%
\pgfpathcurveto{\pgfqpoint{2.629827in}{3.052435in}}{\pgfqpoint{2.634217in}{3.063034in}}{\pgfqpoint{2.634217in}{3.074084in}}%
\pgfpathcurveto{\pgfqpoint{2.634217in}{3.085134in}}{\pgfqpoint{2.629827in}{3.095733in}}{\pgfqpoint{2.622013in}{3.103547in}}%
\pgfpathcurveto{\pgfqpoint{2.614199in}{3.111360in}}{\pgfqpoint{2.603600in}{3.115751in}}{\pgfqpoint{2.592550in}{3.115751in}}%
\pgfpathcurveto{\pgfqpoint{2.581500in}{3.115751in}}{\pgfqpoint{2.570901in}{3.111360in}}{\pgfqpoint{2.563087in}{3.103547in}}%
\pgfpathcurveto{\pgfqpoint{2.555274in}{3.095733in}}{\pgfqpoint{2.550883in}{3.085134in}}{\pgfqpoint{2.550883in}{3.074084in}}%
\pgfpathcurveto{\pgfqpoint{2.550883in}{3.063034in}}{\pgfqpoint{2.555274in}{3.052435in}}{\pgfqpoint{2.563087in}{3.044621in}}%
\pgfpathcurveto{\pgfqpoint{2.570901in}{3.036808in}}{\pgfqpoint{2.581500in}{3.032417in}}{\pgfqpoint{2.592550in}{3.032417in}}%
\pgfpathclose%
\pgfusepath{stroke,fill}%
\end{pgfscope}%
\begin{pgfscope}%
\pgfpathrectangle{\pgfqpoint{0.600000in}{0.600000in}}{\pgfqpoint{3.900000in}{3.900000in}}%
\pgfusepath{clip}%
\pgfsetbuttcap%
\pgfsetroundjoin%
\definecolor{currentfill}{rgb}{0.121569,0.466667,0.705882}%
\pgfsetfillcolor{currentfill}%
\pgfsetlinewidth{1.003750pt}%
\definecolor{currentstroke}{rgb}{0.121569,0.466667,0.705882}%
\pgfsetstrokecolor{currentstroke}%
\pgfsetdash{}{0pt}%
\pgfpathmoveto{\pgfqpoint{1.945206in}{2.821500in}}%
\pgfpathcurveto{\pgfqpoint{1.956256in}{2.821500in}}{\pgfqpoint{1.966855in}{2.825891in}}{\pgfqpoint{1.974668in}{2.833704in}}%
\pgfpathcurveto{\pgfqpoint{1.982482in}{2.841518in}}{\pgfqpoint{1.986872in}{2.852117in}}{\pgfqpoint{1.986872in}{2.863167in}}%
\pgfpathcurveto{\pgfqpoint{1.986872in}{2.874217in}}{\pgfqpoint{1.982482in}{2.884816in}}{\pgfqpoint{1.974668in}{2.892630in}}%
\pgfpathcurveto{\pgfqpoint{1.966855in}{2.900443in}}{\pgfqpoint{1.956256in}{2.904834in}}{\pgfqpoint{1.945206in}{2.904834in}}%
\pgfpathcurveto{\pgfqpoint{1.934156in}{2.904834in}}{\pgfqpoint{1.923556in}{2.900443in}}{\pgfqpoint{1.915743in}{2.892630in}}%
\pgfpathcurveto{\pgfqpoint{1.907929in}{2.884816in}}{\pgfqpoint{1.903539in}{2.874217in}}{\pgfqpoint{1.903539in}{2.863167in}}%
\pgfpathcurveto{\pgfqpoint{1.903539in}{2.852117in}}{\pgfqpoint{1.907929in}{2.841518in}}{\pgfqpoint{1.915743in}{2.833704in}}%
\pgfpathcurveto{\pgfqpoint{1.923556in}{2.825891in}}{\pgfqpoint{1.934156in}{2.821500in}}{\pgfqpoint{1.945206in}{2.821500in}}%
\pgfpathclose%
\pgfusepath{stroke,fill}%
\end{pgfscope}%
\begin{pgfscope}%
\pgfpathrectangle{\pgfqpoint{0.600000in}{0.600000in}}{\pgfqpoint{3.900000in}{3.900000in}}%
\pgfusepath{clip}%
\pgfsetbuttcap%
\pgfsetroundjoin%
\definecolor{currentfill}{rgb}{0.121569,0.466667,0.705882}%
\pgfsetfillcolor{currentfill}%
\pgfsetlinewidth{1.003750pt}%
\definecolor{currentstroke}{rgb}{0.121569,0.466667,0.705882}%
\pgfsetstrokecolor{currentstroke}%
\pgfsetdash{}{0pt}%
\pgfpathmoveto{\pgfqpoint{3.312499in}{2.453359in}}%
\pgfpathcurveto{\pgfqpoint{3.323550in}{2.453359in}}{\pgfqpoint{3.334149in}{2.457749in}}{\pgfqpoint{3.341962in}{2.465563in}}%
\pgfpathcurveto{\pgfqpoint{3.349776in}{2.473376in}}{\pgfqpoint{3.354166in}{2.483975in}}{\pgfqpoint{3.354166in}{2.495025in}}%
\pgfpathcurveto{\pgfqpoint{3.354166in}{2.506076in}}{\pgfqpoint{3.349776in}{2.516675in}}{\pgfqpoint{3.341962in}{2.524488in}}%
\pgfpathcurveto{\pgfqpoint{3.334149in}{2.532302in}}{\pgfqpoint{3.323550in}{2.536692in}}{\pgfqpoint{3.312499in}{2.536692in}}%
\pgfpathcurveto{\pgfqpoint{3.301449in}{2.536692in}}{\pgfqpoint{3.290850in}{2.532302in}}{\pgfqpoint{3.283037in}{2.524488in}}%
\pgfpathcurveto{\pgfqpoint{3.275223in}{2.516675in}}{\pgfqpoint{3.270833in}{2.506076in}}{\pgfqpoint{3.270833in}{2.495025in}}%
\pgfpathcurveto{\pgfqpoint{3.270833in}{2.483975in}}{\pgfqpoint{3.275223in}{2.473376in}}{\pgfqpoint{3.283037in}{2.465563in}}%
\pgfpathcurveto{\pgfqpoint{3.290850in}{2.457749in}}{\pgfqpoint{3.301449in}{2.453359in}}{\pgfqpoint{3.312499in}{2.453359in}}%
\pgfpathclose%
\pgfusepath{stroke,fill}%
\end{pgfscope}%
\begin{pgfscope}%
\pgfpathrectangle{\pgfqpoint{0.600000in}{0.600000in}}{\pgfqpoint{3.900000in}{3.900000in}}%
\pgfusepath{clip}%
\pgfsetbuttcap%
\pgfsetroundjoin%
\definecolor{currentfill}{rgb}{0.121569,0.466667,0.705882}%
\pgfsetfillcolor{currentfill}%
\pgfsetlinewidth{1.003750pt}%
\definecolor{currentstroke}{rgb}{0.121569,0.466667,0.705882}%
\pgfsetstrokecolor{currentstroke}%
\pgfsetdash{}{0pt}%
\pgfpathmoveto{\pgfqpoint{2.870419in}{2.109996in}}%
\pgfpathcurveto{\pgfqpoint{2.881469in}{2.109996in}}{\pgfqpoint{2.892068in}{2.114387in}}{\pgfqpoint{2.899882in}{2.122200in}}%
\pgfpathcurveto{\pgfqpoint{2.907695in}{2.130014in}}{\pgfqpoint{2.912086in}{2.140613in}}{\pgfqpoint{2.912086in}{2.151663in}}%
\pgfpathcurveto{\pgfqpoint{2.912086in}{2.162713in}}{\pgfqpoint{2.907695in}{2.173312in}}{\pgfqpoint{2.899882in}{2.181126in}}%
\pgfpathcurveto{\pgfqpoint{2.892068in}{2.188939in}}{\pgfqpoint{2.881469in}{2.193330in}}{\pgfqpoint{2.870419in}{2.193330in}}%
\pgfpathcurveto{\pgfqpoint{2.859369in}{2.193330in}}{\pgfqpoint{2.848770in}{2.188939in}}{\pgfqpoint{2.840956in}{2.181126in}}%
\pgfpathcurveto{\pgfqpoint{2.833143in}{2.173312in}}{\pgfqpoint{2.828752in}{2.162713in}}{\pgfqpoint{2.828752in}{2.151663in}}%
\pgfpathcurveto{\pgfqpoint{2.828752in}{2.140613in}}{\pgfqpoint{2.833143in}{2.130014in}}{\pgfqpoint{2.840956in}{2.122200in}}%
\pgfpathcurveto{\pgfqpoint{2.848770in}{2.114387in}}{\pgfqpoint{2.859369in}{2.109996in}}{\pgfqpoint{2.870419in}{2.109996in}}%
\pgfpathclose%
\pgfusepath{stroke,fill}%
\end{pgfscope}%
\begin{pgfscope}%
\pgfpathrectangle{\pgfqpoint{0.600000in}{0.600000in}}{\pgfqpoint{3.900000in}{3.900000in}}%
\pgfusepath{clip}%
\pgfsetbuttcap%
\pgfsetroundjoin%
\definecolor{currentfill}{rgb}{0.121569,0.466667,0.705882}%
\pgfsetfillcolor{currentfill}%
\pgfsetlinewidth{1.003750pt}%
\definecolor{currentstroke}{rgb}{0.121569,0.466667,0.705882}%
\pgfsetstrokecolor{currentstroke}%
\pgfsetdash{}{0pt}%
\pgfpathmoveto{\pgfqpoint{3.082932in}{2.811047in}}%
\pgfpathcurveto{\pgfqpoint{3.093983in}{2.811047in}}{\pgfqpoint{3.104582in}{2.815437in}}{\pgfqpoint{3.112395in}{2.823251in}}%
\pgfpathcurveto{\pgfqpoint{3.120209in}{2.831064in}}{\pgfqpoint{3.124599in}{2.841663in}}{\pgfqpoint{3.124599in}{2.852714in}}%
\pgfpathcurveto{\pgfqpoint{3.124599in}{2.863764in}}{\pgfqpoint{3.120209in}{2.874363in}}{\pgfqpoint{3.112395in}{2.882176in}}%
\pgfpathcurveto{\pgfqpoint{3.104582in}{2.889990in}}{\pgfqpoint{3.093983in}{2.894380in}}{\pgfqpoint{3.082932in}{2.894380in}}%
\pgfpathcurveto{\pgfqpoint{3.071882in}{2.894380in}}{\pgfqpoint{3.061283in}{2.889990in}}{\pgfqpoint{3.053470in}{2.882176in}}%
\pgfpathcurveto{\pgfqpoint{3.045656in}{2.874363in}}{\pgfqpoint{3.041266in}{2.863764in}}{\pgfqpoint{3.041266in}{2.852714in}}%
\pgfpathcurveto{\pgfqpoint{3.041266in}{2.841663in}}{\pgfqpoint{3.045656in}{2.831064in}}{\pgfqpoint{3.053470in}{2.823251in}}%
\pgfpathcurveto{\pgfqpoint{3.061283in}{2.815437in}}{\pgfqpoint{3.071882in}{2.811047in}}{\pgfqpoint{3.082932in}{2.811047in}}%
\pgfpathclose%
\pgfusepath{stroke,fill}%
\end{pgfscope}%
\begin{pgfscope}%
\pgfpathrectangle{\pgfqpoint{0.600000in}{0.600000in}}{\pgfqpoint{3.900000in}{3.900000in}}%
\pgfusepath{clip}%
\pgfsetbuttcap%
\pgfsetroundjoin%
\definecolor{currentfill}{rgb}{0.121569,0.466667,0.705882}%
\pgfsetfillcolor{currentfill}%
\pgfsetlinewidth{1.003750pt}%
\definecolor{currentstroke}{rgb}{0.121569,0.466667,0.705882}%
\pgfsetstrokecolor{currentstroke}%
\pgfsetdash{}{0pt}%
\pgfpathmoveto{\pgfqpoint{2.315085in}{1.957038in}}%
\pgfpathcurveto{\pgfqpoint{2.326135in}{1.957038in}}{\pgfqpoint{2.336734in}{1.961428in}}{\pgfqpoint{2.344548in}{1.969242in}}%
\pgfpathcurveto{\pgfqpoint{2.352361in}{1.977055in}}{\pgfqpoint{2.356752in}{1.987654in}}{\pgfqpoint{2.356752in}{1.998704in}}%
\pgfpathcurveto{\pgfqpoint{2.356752in}{2.009755in}}{\pgfqpoint{2.352361in}{2.020354in}}{\pgfqpoint{2.344548in}{2.028167in}}%
\pgfpathcurveto{\pgfqpoint{2.336734in}{2.035981in}}{\pgfqpoint{2.326135in}{2.040371in}}{\pgfqpoint{2.315085in}{2.040371in}}%
\pgfpathcurveto{\pgfqpoint{2.304035in}{2.040371in}}{\pgfqpoint{2.293436in}{2.035981in}}{\pgfqpoint{2.285622in}{2.028167in}}%
\pgfpathcurveto{\pgfqpoint{2.277809in}{2.020354in}}{\pgfqpoint{2.273418in}{2.009755in}}{\pgfqpoint{2.273418in}{1.998704in}}%
\pgfpathcurveto{\pgfqpoint{2.273418in}{1.987654in}}{\pgfqpoint{2.277809in}{1.977055in}}{\pgfqpoint{2.285622in}{1.969242in}}%
\pgfpathcurveto{\pgfqpoint{2.293436in}{1.961428in}}{\pgfqpoint{2.304035in}{1.957038in}}{\pgfqpoint{2.315085in}{1.957038in}}%
\pgfpathclose%
\pgfusepath{stroke,fill}%
\end{pgfscope}%
\begin{pgfscope}%
\pgfpathrectangle{\pgfqpoint{0.600000in}{0.600000in}}{\pgfqpoint{3.900000in}{3.900000in}}%
\pgfusepath{clip}%
\pgfsetbuttcap%
\pgfsetroundjoin%
\definecolor{currentfill}{rgb}{0.121569,0.466667,0.705882}%
\pgfsetfillcolor{currentfill}%
\pgfsetlinewidth{1.003750pt}%
\definecolor{currentstroke}{rgb}{0.121569,0.466667,0.705882}%
\pgfsetstrokecolor{currentstroke}%
\pgfsetdash{}{0pt}%
\pgfpathmoveto{\pgfqpoint{2.320786in}{2.596896in}}%
\pgfpathcurveto{\pgfqpoint{2.331836in}{2.596896in}}{\pgfqpoint{2.342435in}{2.601286in}}{\pgfqpoint{2.350249in}{2.609100in}}%
\pgfpathcurveto{\pgfqpoint{2.358063in}{2.616914in}}{\pgfqpoint{2.362453in}{2.627513in}}{\pgfqpoint{2.362453in}{2.638563in}}%
\pgfpathcurveto{\pgfqpoint{2.362453in}{2.649613in}}{\pgfqpoint{2.358063in}{2.660212in}}{\pgfqpoint{2.350249in}{2.668025in}}%
\pgfpathcurveto{\pgfqpoint{2.342435in}{2.675839in}}{\pgfqpoint{2.331836in}{2.680229in}}{\pgfqpoint{2.320786in}{2.680229in}}%
\pgfpathcurveto{\pgfqpoint{2.309736in}{2.680229in}}{\pgfqpoint{2.299137in}{2.675839in}}{\pgfqpoint{2.291324in}{2.668025in}}%
\pgfpathcurveto{\pgfqpoint{2.283510in}{2.660212in}}{\pgfqpoint{2.279120in}{2.649613in}}{\pgfqpoint{2.279120in}{2.638563in}}%
\pgfpathcurveto{\pgfqpoint{2.279120in}{2.627513in}}{\pgfqpoint{2.283510in}{2.616914in}}{\pgfqpoint{2.291324in}{2.609100in}}%
\pgfpathcurveto{\pgfqpoint{2.299137in}{2.601286in}}{\pgfqpoint{2.309736in}{2.596896in}}{\pgfqpoint{2.320786in}{2.596896in}}%
\pgfpathclose%
\pgfusepath{stroke,fill}%
\end{pgfscope}%
\begin{pgfscope}%
\pgfpathrectangle{\pgfqpoint{0.600000in}{0.600000in}}{\pgfqpoint{3.900000in}{3.900000in}}%
\pgfusepath{clip}%
\pgfsetbuttcap%
\pgfsetroundjoin%
\definecolor{currentfill}{rgb}{0.121569,0.466667,0.705882}%
\pgfsetfillcolor{currentfill}%
\pgfsetlinewidth{1.003750pt}%
\definecolor{currentstroke}{rgb}{0.121569,0.466667,0.705882}%
\pgfsetstrokecolor{currentstroke}%
\pgfsetdash{}{0pt}%
\pgfpathmoveto{\pgfqpoint{2.851511in}{2.031548in}}%
\pgfpathcurveto{\pgfqpoint{2.862562in}{2.031548in}}{\pgfqpoint{2.873161in}{2.035939in}}{\pgfqpoint{2.880974in}{2.043752in}}%
\pgfpathcurveto{\pgfqpoint{2.888788in}{2.051566in}}{\pgfqpoint{2.893178in}{2.062165in}}{\pgfqpoint{2.893178in}{2.073215in}}%
\pgfpathcurveto{\pgfqpoint{2.893178in}{2.084265in}}{\pgfqpoint{2.888788in}{2.094864in}}{\pgfqpoint{2.880974in}{2.102678in}}%
\pgfpathcurveto{\pgfqpoint{2.873161in}{2.110491in}}{\pgfqpoint{2.862562in}{2.114882in}}{\pgfqpoint{2.851511in}{2.114882in}}%
\pgfpathcurveto{\pgfqpoint{2.840461in}{2.114882in}}{\pgfqpoint{2.829862in}{2.110491in}}{\pgfqpoint{2.822049in}{2.102678in}}%
\pgfpathcurveto{\pgfqpoint{2.814235in}{2.094864in}}{\pgfqpoint{2.809845in}{2.084265in}}{\pgfqpoint{2.809845in}{2.073215in}}%
\pgfpathcurveto{\pgfqpoint{2.809845in}{2.062165in}}{\pgfqpoint{2.814235in}{2.051566in}}{\pgfqpoint{2.822049in}{2.043752in}}%
\pgfpathcurveto{\pgfqpoint{2.829862in}{2.035939in}}{\pgfqpoint{2.840461in}{2.031548in}}{\pgfqpoint{2.851511in}{2.031548in}}%
\pgfpathclose%
\pgfusepath{stroke,fill}%
\end{pgfscope}%
\begin{pgfscope}%
\pgfpathrectangle{\pgfqpoint{0.600000in}{0.600000in}}{\pgfqpoint{3.900000in}{3.900000in}}%
\pgfusepath{clip}%
\pgfsetbuttcap%
\pgfsetroundjoin%
\definecolor{currentfill}{rgb}{0.121569,0.466667,0.705882}%
\pgfsetfillcolor{currentfill}%
\pgfsetlinewidth{1.003750pt}%
\definecolor{currentstroke}{rgb}{0.121569,0.466667,0.705882}%
\pgfsetstrokecolor{currentstroke}%
\pgfsetdash{}{0pt}%
\pgfpathmoveto{\pgfqpoint{2.576530in}{2.922097in}}%
\pgfpathcurveto{\pgfqpoint{2.587580in}{2.922097in}}{\pgfqpoint{2.598179in}{2.926488in}}{\pgfqpoint{2.605993in}{2.934301in}}%
\pgfpathcurveto{\pgfqpoint{2.613807in}{2.942115in}}{\pgfqpoint{2.618197in}{2.952714in}}{\pgfqpoint{2.618197in}{2.963764in}}%
\pgfpathcurveto{\pgfqpoint{2.618197in}{2.974814in}}{\pgfqpoint{2.613807in}{2.985413in}}{\pgfqpoint{2.605993in}{2.993227in}}%
\pgfpathcurveto{\pgfqpoint{2.598179in}{3.001040in}}{\pgfqpoint{2.587580in}{3.005431in}}{\pgfqpoint{2.576530in}{3.005431in}}%
\pgfpathcurveto{\pgfqpoint{2.565480in}{3.005431in}}{\pgfqpoint{2.554881in}{3.001040in}}{\pgfqpoint{2.547067in}{2.993227in}}%
\pgfpathcurveto{\pgfqpoint{2.539254in}{2.985413in}}{\pgfqpoint{2.534863in}{2.974814in}}{\pgfqpoint{2.534863in}{2.963764in}}%
\pgfpathcurveto{\pgfqpoint{2.534863in}{2.952714in}}{\pgfqpoint{2.539254in}{2.942115in}}{\pgfqpoint{2.547067in}{2.934301in}}%
\pgfpathcurveto{\pgfqpoint{2.554881in}{2.926488in}}{\pgfqpoint{2.565480in}{2.922097in}}{\pgfqpoint{2.576530in}{2.922097in}}%
\pgfpathclose%
\pgfusepath{stroke,fill}%
\end{pgfscope}%
\begin{pgfscope}%
\pgfpathrectangle{\pgfqpoint{0.600000in}{0.600000in}}{\pgfqpoint{3.900000in}{3.900000in}}%
\pgfusepath{clip}%
\pgfsetbuttcap%
\pgfsetroundjoin%
\definecolor{currentfill}{rgb}{0.121569,0.466667,0.705882}%
\pgfsetfillcolor{currentfill}%
\pgfsetlinewidth{1.003750pt}%
\definecolor{currentstroke}{rgb}{0.121569,0.466667,0.705882}%
\pgfsetstrokecolor{currentstroke}%
\pgfsetdash{}{0pt}%
\pgfpathmoveto{\pgfqpoint{3.048688in}{2.629089in}}%
\pgfpathcurveto{\pgfqpoint{3.059739in}{2.629089in}}{\pgfqpoint{3.070338in}{2.633480in}}{\pgfqpoint{3.078151in}{2.641293in}}%
\pgfpathcurveto{\pgfqpoint{3.085965in}{2.649107in}}{\pgfqpoint{3.090355in}{2.659706in}}{\pgfqpoint{3.090355in}{2.670756in}}%
\pgfpathcurveto{\pgfqpoint{3.090355in}{2.681806in}}{\pgfqpoint{3.085965in}{2.692405in}}{\pgfqpoint{3.078151in}{2.700219in}}%
\pgfpathcurveto{\pgfqpoint{3.070338in}{2.708033in}}{\pgfqpoint{3.059739in}{2.712423in}}{\pgfqpoint{3.048688in}{2.712423in}}%
\pgfpathcurveto{\pgfqpoint{3.037638in}{2.712423in}}{\pgfqpoint{3.027039in}{2.708033in}}{\pgfqpoint{3.019226in}{2.700219in}}%
\pgfpathcurveto{\pgfqpoint{3.011412in}{2.692405in}}{\pgfqpoint{3.007022in}{2.681806in}}{\pgfqpoint{3.007022in}{2.670756in}}%
\pgfpathcurveto{\pgfqpoint{3.007022in}{2.659706in}}{\pgfqpoint{3.011412in}{2.649107in}}{\pgfqpoint{3.019226in}{2.641293in}}%
\pgfpathcurveto{\pgfqpoint{3.027039in}{2.633480in}}{\pgfqpoint{3.037638in}{2.629089in}}{\pgfqpoint{3.048688in}{2.629089in}}%
\pgfpathclose%
\pgfusepath{stroke,fill}%
\end{pgfscope}%
\begin{pgfscope}%
\pgfpathrectangle{\pgfqpoint{0.600000in}{0.600000in}}{\pgfqpoint{3.900000in}{3.900000in}}%
\pgfusepath{clip}%
\pgfsetbuttcap%
\pgfsetroundjoin%
\definecolor{currentfill}{rgb}{0.121569,0.466667,0.705882}%
\pgfsetfillcolor{currentfill}%
\pgfsetlinewidth{1.003750pt}%
\definecolor{currentstroke}{rgb}{0.121569,0.466667,0.705882}%
\pgfsetstrokecolor{currentstroke}%
\pgfsetdash{}{0pt}%
\pgfpathmoveto{\pgfqpoint{3.059453in}{2.654390in}}%
\pgfpathcurveto{\pgfqpoint{3.070503in}{2.654390in}}{\pgfqpoint{3.081102in}{2.658780in}}{\pgfqpoint{3.088915in}{2.666594in}}%
\pgfpathcurveto{\pgfqpoint{3.096729in}{2.674407in}}{\pgfqpoint{3.101119in}{2.685006in}}{\pgfqpoint{3.101119in}{2.696056in}}%
\pgfpathcurveto{\pgfqpoint{3.101119in}{2.707106in}}{\pgfqpoint{3.096729in}{2.717705in}}{\pgfqpoint{3.088915in}{2.725519in}}%
\pgfpathcurveto{\pgfqpoint{3.081102in}{2.733333in}}{\pgfqpoint{3.070503in}{2.737723in}}{\pgfqpoint{3.059453in}{2.737723in}}%
\pgfpathcurveto{\pgfqpoint{3.048403in}{2.737723in}}{\pgfqpoint{3.037804in}{2.733333in}}{\pgfqpoint{3.029990in}{2.725519in}}%
\pgfpathcurveto{\pgfqpoint{3.022176in}{2.717705in}}{\pgfqpoint{3.017786in}{2.707106in}}{\pgfqpoint{3.017786in}{2.696056in}}%
\pgfpathcurveto{\pgfqpoint{3.017786in}{2.685006in}}{\pgfqpoint{3.022176in}{2.674407in}}{\pgfqpoint{3.029990in}{2.666594in}}%
\pgfpathcurveto{\pgfqpoint{3.037804in}{2.658780in}}{\pgfqpoint{3.048403in}{2.654390in}}{\pgfqpoint{3.059453in}{2.654390in}}%
\pgfpathclose%
\pgfusepath{stroke,fill}%
\end{pgfscope}%
\begin{pgfscope}%
\pgfpathrectangle{\pgfqpoint{0.600000in}{0.600000in}}{\pgfqpoint{3.900000in}{3.900000in}}%
\pgfusepath{clip}%
\pgfsetbuttcap%
\pgfsetroundjoin%
\definecolor{currentfill}{rgb}{0.121569,0.466667,0.705882}%
\pgfsetfillcolor{currentfill}%
\pgfsetlinewidth{1.003750pt}%
\definecolor{currentstroke}{rgb}{0.121569,0.466667,0.705882}%
\pgfsetstrokecolor{currentstroke}%
\pgfsetdash{}{0pt}%
\pgfpathmoveto{\pgfqpoint{2.659067in}{2.748995in}}%
\pgfpathcurveto{\pgfqpoint{2.670117in}{2.748995in}}{\pgfqpoint{2.680716in}{2.753385in}}{\pgfqpoint{2.688530in}{2.761199in}}%
\pgfpathcurveto{\pgfqpoint{2.696343in}{2.769012in}}{\pgfqpoint{2.700733in}{2.779611in}}{\pgfqpoint{2.700733in}{2.790661in}}%
\pgfpathcurveto{\pgfqpoint{2.700733in}{2.801711in}}{\pgfqpoint{2.696343in}{2.812310in}}{\pgfqpoint{2.688530in}{2.820124in}}%
\pgfpathcurveto{\pgfqpoint{2.680716in}{2.827938in}}{\pgfqpoint{2.670117in}{2.832328in}}{\pgfqpoint{2.659067in}{2.832328in}}%
\pgfpathcurveto{\pgfqpoint{2.648017in}{2.832328in}}{\pgfqpoint{2.637418in}{2.827938in}}{\pgfqpoint{2.629604in}{2.820124in}}%
\pgfpathcurveto{\pgfqpoint{2.621790in}{2.812310in}}{\pgfqpoint{2.617400in}{2.801711in}}{\pgfqpoint{2.617400in}{2.790661in}}%
\pgfpathcurveto{\pgfqpoint{2.617400in}{2.779611in}}{\pgfqpoint{2.621790in}{2.769012in}}{\pgfqpoint{2.629604in}{2.761199in}}%
\pgfpathcurveto{\pgfqpoint{2.637418in}{2.753385in}}{\pgfqpoint{2.648017in}{2.748995in}}{\pgfqpoint{2.659067in}{2.748995in}}%
\pgfpathclose%
\pgfusepath{stroke,fill}%
\end{pgfscope}%
\begin{pgfscope}%
\pgfpathrectangle{\pgfqpoint{0.600000in}{0.600000in}}{\pgfqpoint{3.900000in}{3.900000in}}%
\pgfusepath{clip}%
\pgfsetbuttcap%
\pgfsetroundjoin%
\definecolor{currentfill}{rgb}{0.121569,0.466667,0.705882}%
\pgfsetfillcolor{currentfill}%
\pgfsetlinewidth{1.003750pt}%
\definecolor{currentstroke}{rgb}{0.121569,0.466667,0.705882}%
\pgfsetstrokecolor{currentstroke}%
\pgfsetdash{}{0pt}%
\pgfpathmoveto{\pgfqpoint{3.133202in}{2.965904in}}%
\pgfpathcurveto{\pgfqpoint{3.144252in}{2.965904in}}{\pgfqpoint{3.154851in}{2.970295in}}{\pgfqpoint{3.162665in}{2.978108in}}%
\pgfpathcurveto{\pgfqpoint{3.170478in}{2.985922in}}{\pgfqpoint{3.174868in}{2.996521in}}{\pgfqpoint{3.174868in}{3.007571in}}%
\pgfpathcurveto{\pgfqpoint{3.174868in}{3.018621in}}{\pgfqpoint{3.170478in}{3.029220in}}{\pgfqpoint{3.162665in}{3.037034in}}%
\pgfpathcurveto{\pgfqpoint{3.154851in}{3.044847in}}{\pgfqpoint{3.144252in}{3.049238in}}{\pgfqpoint{3.133202in}{3.049238in}}%
\pgfpathcurveto{\pgfqpoint{3.122152in}{3.049238in}}{\pgfqpoint{3.111553in}{3.044847in}}{\pgfqpoint{3.103739in}{3.037034in}}%
\pgfpathcurveto{\pgfqpoint{3.095925in}{3.029220in}}{\pgfqpoint{3.091535in}{3.018621in}}{\pgfqpoint{3.091535in}{3.007571in}}%
\pgfpathcurveto{\pgfqpoint{3.091535in}{2.996521in}}{\pgfqpoint{3.095925in}{2.985922in}}{\pgfqpoint{3.103739in}{2.978108in}}%
\pgfpathcurveto{\pgfqpoint{3.111553in}{2.970295in}}{\pgfqpoint{3.122152in}{2.965904in}}{\pgfqpoint{3.133202in}{2.965904in}}%
\pgfpathclose%
\pgfusepath{stroke,fill}%
\end{pgfscope}%
\begin{pgfscope}%
\pgfpathrectangle{\pgfqpoint{0.600000in}{0.600000in}}{\pgfqpoint{3.900000in}{3.900000in}}%
\pgfusepath{clip}%
\pgfsetbuttcap%
\pgfsetroundjoin%
\definecolor{currentfill}{rgb}{0.121569,0.466667,0.705882}%
\pgfsetfillcolor{currentfill}%
\pgfsetlinewidth{1.003750pt}%
\definecolor{currentstroke}{rgb}{0.121569,0.466667,0.705882}%
\pgfsetstrokecolor{currentstroke}%
\pgfsetdash{}{0pt}%
\pgfpathmoveto{\pgfqpoint{2.886004in}{2.269285in}}%
\pgfpathcurveto{\pgfqpoint{2.897054in}{2.269285in}}{\pgfqpoint{2.907653in}{2.273675in}}{\pgfqpoint{2.915467in}{2.281489in}}%
\pgfpathcurveto{\pgfqpoint{2.923280in}{2.289303in}}{\pgfqpoint{2.927671in}{2.299902in}}{\pgfqpoint{2.927671in}{2.310952in}}%
\pgfpathcurveto{\pgfqpoint{2.927671in}{2.322002in}}{\pgfqpoint{2.923280in}{2.332601in}}{\pgfqpoint{2.915467in}{2.340414in}}%
\pgfpathcurveto{\pgfqpoint{2.907653in}{2.348228in}}{\pgfqpoint{2.897054in}{2.352618in}}{\pgfqpoint{2.886004in}{2.352618in}}%
\pgfpathcurveto{\pgfqpoint{2.874954in}{2.352618in}}{\pgfqpoint{2.864355in}{2.348228in}}{\pgfqpoint{2.856541in}{2.340414in}}%
\pgfpathcurveto{\pgfqpoint{2.848728in}{2.332601in}}{\pgfqpoint{2.844337in}{2.322002in}}{\pgfqpoint{2.844337in}{2.310952in}}%
\pgfpathcurveto{\pgfqpoint{2.844337in}{2.299902in}}{\pgfqpoint{2.848728in}{2.289303in}}{\pgfqpoint{2.856541in}{2.281489in}}%
\pgfpathcurveto{\pgfqpoint{2.864355in}{2.273675in}}{\pgfqpoint{2.874954in}{2.269285in}}{\pgfqpoint{2.886004in}{2.269285in}}%
\pgfpathclose%
\pgfusepath{stroke,fill}%
\end{pgfscope}%
\begin{pgfscope}%
\pgfpathrectangle{\pgfqpoint{0.600000in}{0.600000in}}{\pgfqpoint{3.900000in}{3.900000in}}%
\pgfusepath{clip}%
\pgfsetbuttcap%
\pgfsetroundjoin%
\definecolor{currentfill}{rgb}{0.121569,0.466667,0.705882}%
\pgfsetfillcolor{currentfill}%
\pgfsetlinewidth{1.003750pt}%
\definecolor{currentstroke}{rgb}{0.121569,0.466667,0.705882}%
\pgfsetstrokecolor{currentstroke}%
\pgfsetdash{}{0pt}%
\pgfpathmoveto{\pgfqpoint{2.628552in}{3.122806in}}%
\pgfpathcurveto{\pgfqpoint{2.639602in}{3.122806in}}{\pgfqpoint{2.650201in}{3.127196in}}{\pgfqpoint{2.658015in}{3.135010in}}%
\pgfpathcurveto{\pgfqpoint{2.665828in}{3.142823in}}{\pgfqpoint{2.670219in}{3.153422in}}{\pgfqpoint{2.670219in}{3.164473in}}%
\pgfpathcurveto{\pgfqpoint{2.670219in}{3.175523in}}{\pgfqpoint{2.665828in}{3.186122in}}{\pgfqpoint{2.658015in}{3.193935in}}%
\pgfpathcurveto{\pgfqpoint{2.650201in}{3.201749in}}{\pgfqpoint{2.639602in}{3.206139in}}{\pgfqpoint{2.628552in}{3.206139in}}%
\pgfpathcurveto{\pgfqpoint{2.617502in}{3.206139in}}{\pgfqpoint{2.606903in}{3.201749in}}{\pgfqpoint{2.599089in}{3.193935in}}%
\pgfpathcurveto{\pgfqpoint{2.591276in}{3.186122in}}{\pgfqpoint{2.586885in}{3.175523in}}{\pgfqpoint{2.586885in}{3.164473in}}%
\pgfpathcurveto{\pgfqpoint{2.586885in}{3.153422in}}{\pgfqpoint{2.591276in}{3.142823in}}{\pgfqpoint{2.599089in}{3.135010in}}%
\pgfpathcurveto{\pgfqpoint{2.606903in}{3.127196in}}{\pgfqpoint{2.617502in}{3.122806in}}{\pgfqpoint{2.628552in}{3.122806in}}%
\pgfpathclose%
\pgfusepath{stroke,fill}%
\end{pgfscope}%
\begin{pgfscope}%
\pgfpathrectangle{\pgfqpoint{0.600000in}{0.600000in}}{\pgfqpoint{3.900000in}{3.900000in}}%
\pgfusepath{clip}%
\pgfsetbuttcap%
\pgfsetroundjoin%
\definecolor{currentfill}{rgb}{0.121569,0.466667,0.705882}%
\pgfsetfillcolor{currentfill}%
\pgfsetlinewidth{1.003750pt}%
\definecolor{currentstroke}{rgb}{0.121569,0.466667,0.705882}%
\pgfsetstrokecolor{currentstroke}%
\pgfsetdash{}{0pt}%
\pgfpathmoveto{\pgfqpoint{2.950820in}{1.652864in}}%
\pgfpathcurveto{\pgfqpoint{2.961870in}{1.652864in}}{\pgfqpoint{2.972469in}{1.657254in}}{\pgfqpoint{2.980282in}{1.665068in}}%
\pgfpathcurveto{\pgfqpoint{2.988096in}{1.672882in}}{\pgfqpoint{2.992486in}{1.683481in}}{\pgfqpoint{2.992486in}{1.694531in}}%
\pgfpathcurveto{\pgfqpoint{2.992486in}{1.705581in}}{\pgfqpoint{2.988096in}{1.716180in}}{\pgfqpoint{2.980282in}{1.723993in}}%
\pgfpathcurveto{\pgfqpoint{2.972469in}{1.731807in}}{\pgfqpoint{2.961870in}{1.736197in}}{\pgfqpoint{2.950820in}{1.736197in}}%
\pgfpathcurveto{\pgfqpoint{2.939769in}{1.736197in}}{\pgfqpoint{2.929170in}{1.731807in}}{\pgfqpoint{2.921357in}{1.723993in}}%
\pgfpathcurveto{\pgfqpoint{2.913543in}{1.716180in}}{\pgfqpoint{2.909153in}{1.705581in}}{\pgfqpoint{2.909153in}{1.694531in}}%
\pgfpathcurveto{\pgfqpoint{2.909153in}{1.683481in}}{\pgfqpoint{2.913543in}{1.672882in}}{\pgfqpoint{2.921357in}{1.665068in}}%
\pgfpathcurveto{\pgfqpoint{2.929170in}{1.657254in}}{\pgfqpoint{2.939769in}{1.652864in}}{\pgfqpoint{2.950820in}{1.652864in}}%
\pgfpathclose%
\pgfusepath{stroke,fill}%
\end{pgfscope}%
\begin{pgfscope}%
\pgfpathrectangle{\pgfqpoint{0.600000in}{0.600000in}}{\pgfqpoint{3.900000in}{3.900000in}}%
\pgfusepath{clip}%
\pgfsetbuttcap%
\pgfsetroundjoin%
\definecolor{currentfill}{rgb}{0.121569,0.466667,0.705882}%
\pgfsetfillcolor{currentfill}%
\pgfsetlinewidth{1.003750pt}%
\definecolor{currentstroke}{rgb}{0.121569,0.466667,0.705882}%
\pgfsetstrokecolor{currentstroke}%
\pgfsetdash{}{0pt}%
\pgfpathmoveto{\pgfqpoint{2.677748in}{1.981952in}}%
\pgfpathcurveto{\pgfqpoint{2.688799in}{1.981952in}}{\pgfqpoint{2.699398in}{1.986342in}}{\pgfqpoint{2.707211in}{1.994156in}}%
\pgfpathcurveto{\pgfqpoint{2.715025in}{2.001969in}}{\pgfqpoint{2.719415in}{2.012569in}}{\pgfqpoint{2.719415in}{2.023619in}}%
\pgfpathcurveto{\pgfqpoint{2.719415in}{2.034669in}}{\pgfqpoint{2.715025in}{2.045268in}}{\pgfqpoint{2.707211in}{2.053081in}}%
\pgfpathcurveto{\pgfqpoint{2.699398in}{2.060895in}}{\pgfqpoint{2.688799in}{2.065285in}}{\pgfqpoint{2.677748in}{2.065285in}}%
\pgfpathcurveto{\pgfqpoint{2.666698in}{2.065285in}}{\pgfqpoint{2.656099in}{2.060895in}}{\pgfqpoint{2.648286in}{2.053081in}}%
\pgfpathcurveto{\pgfqpoint{2.640472in}{2.045268in}}{\pgfqpoint{2.636082in}{2.034669in}}{\pgfqpoint{2.636082in}{2.023619in}}%
\pgfpathcurveto{\pgfqpoint{2.636082in}{2.012569in}}{\pgfqpoint{2.640472in}{2.001969in}}{\pgfqpoint{2.648286in}{1.994156in}}%
\pgfpathcurveto{\pgfqpoint{2.656099in}{1.986342in}}{\pgfqpoint{2.666698in}{1.981952in}}{\pgfqpoint{2.677748in}{1.981952in}}%
\pgfpathclose%
\pgfusepath{stroke,fill}%
\end{pgfscope}%
\begin{pgfscope}%
\pgfpathrectangle{\pgfqpoint{0.600000in}{0.600000in}}{\pgfqpoint{3.900000in}{3.900000in}}%
\pgfusepath{clip}%
\pgfsetbuttcap%
\pgfsetroundjoin%
\definecolor{currentfill}{rgb}{0.121569,0.466667,0.705882}%
\pgfsetfillcolor{currentfill}%
\pgfsetlinewidth{1.003750pt}%
\definecolor{currentstroke}{rgb}{0.121569,0.466667,0.705882}%
\pgfsetstrokecolor{currentstroke}%
\pgfsetdash{}{0pt}%
\pgfpathmoveto{\pgfqpoint{2.262987in}{2.755081in}}%
\pgfpathcurveto{\pgfqpoint{2.274037in}{2.755081in}}{\pgfqpoint{2.284636in}{2.759471in}}{\pgfqpoint{2.292450in}{2.767285in}}%
\pgfpathcurveto{\pgfqpoint{2.300264in}{2.775099in}}{\pgfqpoint{2.304654in}{2.785698in}}{\pgfqpoint{2.304654in}{2.796748in}}%
\pgfpathcurveto{\pgfqpoint{2.304654in}{2.807798in}}{\pgfqpoint{2.300264in}{2.818397in}}{\pgfqpoint{2.292450in}{2.826210in}}%
\pgfpathcurveto{\pgfqpoint{2.284636in}{2.834024in}}{\pgfqpoint{2.274037in}{2.838414in}}{\pgfqpoint{2.262987in}{2.838414in}}%
\pgfpathcurveto{\pgfqpoint{2.251937in}{2.838414in}}{\pgfqpoint{2.241338in}{2.834024in}}{\pgfqpoint{2.233524in}{2.826210in}}%
\pgfpathcurveto{\pgfqpoint{2.225711in}{2.818397in}}{\pgfqpoint{2.221321in}{2.807798in}}{\pgfqpoint{2.221321in}{2.796748in}}%
\pgfpathcurveto{\pgfqpoint{2.221321in}{2.785698in}}{\pgfqpoint{2.225711in}{2.775099in}}{\pgfqpoint{2.233524in}{2.767285in}}%
\pgfpathcurveto{\pgfqpoint{2.241338in}{2.759471in}}{\pgfqpoint{2.251937in}{2.755081in}}{\pgfqpoint{2.262987in}{2.755081in}}%
\pgfpathclose%
\pgfusepath{stroke,fill}%
\end{pgfscope}%
\begin{pgfscope}%
\pgfpathrectangle{\pgfqpoint{0.600000in}{0.600000in}}{\pgfqpoint{3.900000in}{3.900000in}}%
\pgfusepath{clip}%
\pgfsetbuttcap%
\pgfsetroundjoin%
\definecolor{currentfill}{rgb}{0.121569,0.466667,0.705882}%
\pgfsetfillcolor{currentfill}%
\pgfsetlinewidth{1.003750pt}%
\definecolor{currentstroke}{rgb}{0.121569,0.466667,0.705882}%
\pgfsetstrokecolor{currentstroke}%
\pgfsetdash{}{0pt}%
\pgfpathmoveto{\pgfqpoint{2.089483in}{3.647043in}}%
\pgfpathcurveto{\pgfqpoint{2.100533in}{3.647043in}}{\pgfqpoint{2.111132in}{3.651434in}}{\pgfqpoint{2.118946in}{3.659247in}}%
\pgfpathcurveto{\pgfqpoint{2.126759in}{3.667061in}}{\pgfqpoint{2.131150in}{3.677660in}}{\pgfqpoint{2.131150in}{3.688710in}}%
\pgfpathcurveto{\pgfqpoint{2.131150in}{3.699760in}}{\pgfqpoint{2.126759in}{3.710359in}}{\pgfqpoint{2.118946in}{3.718173in}}%
\pgfpathcurveto{\pgfqpoint{2.111132in}{3.725987in}}{\pgfqpoint{2.100533in}{3.730377in}}{\pgfqpoint{2.089483in}{3.730377in}}%
\pgfpathcurveto{\pgfqpoint{2.078433in}{3.730377in}}{\pgfqpoint{2.067834in}{3.725987in}}{\pgfqpoint{2.060020in}{3.718173in}}%
\pgfpathcurveto{\pgfqpoint{2.052207in}{3.710359in}}{\pgfqpoint{2.047816in}{3.699760in}}{\pgfqpoint{2.047816in}{3.688710in}}%
\pgfpathcurveto{\pgfqpoint{2.047816in}{3.677660in}}{\pgfqpoint{2.052207in}{3.667061in}}{\pgfqpoint{2.060020in}{3.659247in}}%
\pgfpathcurveto{\pgfqpoint{2.067834in}{3.651434in}}{\pgfqpoint{2.078433in}{3.647043in}}{\pgfqpoint{2.089483in}{3.647043in}}%
\pgfpathclose%
\pgfusepath{stroke,fill}%
\end{pgfscope}%
\begin{pgfscope}%
\pgfpathrectangle{\pgfqpoint{0.600000in}{0.600000in}}{\pgfqpoint{3.900000in}{3.900000in}}%
\pgfusepath{clip}%
\pgfsetbuttcap%
\pgfsetroundjoin%
\definecolor{currentfill}{rgb}{0.121569,0.466667,0.705882}%
\pgfsetfillcolor{currentfill}%
\pgfsetlinewidth{1.003750pt}%
\definecolor{currentstroke}{rgb}{0.121569,0.466667,0.705882}%
\pgfsetstrokecolor{currentstroke}%
\pgfsetdash{}{0pt}%
\pgfpathmoveto{\pgfqpoint{3.010471in}{3.426923in}}%
\pgfpathcurveto{\pgfqpoint{3.021521in}{3.426923in}}{\pgfqpoint{3.032120in}{3.431313in}}{\pgfqpoint{3.039934in}{3.439127in}}%
\pgfpathcurveto{\pgfqpoint{3.047747in}{3.446940in}}{\pgfqpoint{3.052138in}{3.457539in}}{\pgfqpoint{3.052138in}{3.468589in}}%
\pgfpathcurveto{\pgfqpoint{3.052138in}{3.479640in}}{\pgfqpoint{3.047747in}{3.490239in}}{\pgfqpoint{3.039934in}{3.498052in}}%
\pgfpathcurveto{\pgfqpoint{3.032120in}{3.505866in}}{\pgfqpoint{3.021521in}{3.510256in}}{\pgfqpoint{3.010471in}{3.510256in}}%
\pgfpathcurveto{\pgfqpoint{2.999421in}{3.510256in}}{\pgfqpoint{2.988822in}{3.505866in}}{\pgfqpoint{2.981008in}{3.498052in}}%
\pgfpathcurveto{\pgfqpoint{2.973194in}{3.490239in}}{\pgfqpoint{2.968804in}{3.479640in}}{\pgfqpoint{2.968804in}{3.468589in}}%
\pgfpathcurveto{\pgfqpoint{2.968804in}{3.457539in}}{\pgfqpoint{2.973194in}{3.446940in}}{\pgfqpoint{2.981008in}{3.439127in}}%
\pgfpathcurveto{\pgfqpoint{2.988822in}{3.431313in}}{\pgfqpoint{2.999421in}{3.426923in}}{\pgfqpoint{3.010471in}{3.426923in}}%
\pgfpathclose%
\pgfusepath{stroke,fill}%
\end{pgfscope}%
\begin{pgfscope}%
\pgfpathrectangle{\pgfqpoint{0.600000in}{0.600000in}}{\pgfqpoint{3.900000in}{3.900000in}}%
\pgfusepath{clip}%
\pgfsetbuttcap%
\pgfsetroundjoin%
\definecolor{currentfill}{rgb}{0.121569,0.466667,0.705882}%
\pgfsetfillcolor{currentfill}%
\pgfsetlinewidth{1.003750pt}%
\definecolor{currentstroke}{rgb}{0.121569,0.466667,0.705882}%
\pgfsetstrokecolor{currentstroke}%
\pgfsetdash{}{0pt}%
\pgfpathmoveto{\pgfqpoint{2.652749in}{2.401565in}}%
\pgfpathcurveto{\pgfqpoint{2.663800in}{2.401565in}}{\pgfqpoint{2.674399in}{2.405955in}}{\pgfqpoint{2.682212in}{2.413769in}}%
\pgfpathcurveto{\pgfqpoint{2.690026in}{2.421583in}}{\pgfqpoint{2.694416in}{2.432182in}}{\pgfqpoint{2.694416in}{2.443232in}}%
\pgfpathcurveto{\pgfqpoint{2.694416in}{2.454282in}}{\pgfqpoint{2.690026in}{2.464881in}}{\pgfqpoint{2.682212in}{2.472695in}}%
\pgfpathcurveto{\pgfqpoint{2.674399in}{2.480508in}}{\pgfqpoint{2.663800in}{2.484898in}}{\pgfqpoint{2.652749in}{2.484898in}}%
\pgfpathcurveto{\pgfqpoint{2.641699in}{2.484898in}}{\pgfqpoint{2.631100in}{2.480508in}}{\pgfqpoint{2.623287in}{2.472695in}}%
\pgfpathcurveto{\pgfqpoint{2.615473in}{2.464881in}}{\pgfqpoint{2.611083in}{2.454282in}}{\pgfqpoint{2.611083in}{2.443232in}}%
\pgfpathcurveto{\pgfqpoint{2.611083in}{2.432182in}}{\pgfqpoint{2.615473in}{2.421583in}}{\pgfqpoint{2.623287in}{2.413769in}}%
\pgfpathcurveto{\pgfqpoint{2.631100in}{2.405955in}}{\pgfqpoint{2.641699in}{2.401565in}}{\pgfqpoint{2.652749in}{2.401565in}}%
\pgfpathclose%
\pgfusepath{stroke,fill}%
\end{pgfscope}%
\begin{pgfscope}%
\pgfsetbuttcap%
\pgfsetroundjoin%
\definecolor{currentfill}{rgb}{0.000000,0.000000,0.000000}%
\pgfsetfillcolor{currentfill}%
\pgfsetlinewidth{0.803000pt}%
\definecolor{currentstroke}{rgb}{0.000000,0.000000,0.000000}%
\pgfsetstrokecolor{currentstroke}%
\pgfsetdash{}{0pt}%
\pgfsys@defobject{currentmarker}{\pgfqpoint{0.000000in}{0.000000in}}{\pgfqpoint{0.000000in}{0.048611in}}{%
\pgfpathmoveto{\pgfqpoint{0.000000in}{0.000000in}}%
\pgfpathlineto{\pgfqpoint{0.000000in}{0.048611in}}%
\pgfusepath{stroke,fill}%
}%
\begin{pgfscope}%
\pgfsys@transformshift{1.010526in}{0.600000in}%
\pgfsys@useobject{currentmarker}{}%
\end{pgfscope}%
\end{pgfscope}%
\begin{pgfscope}%
\pgfsetbuttcap%
\pgfsetroundjoin%
\definecolor{currentfill}{rgb}{0.000000,0.000000,0.000000}%
\pgfsetfillcolor{currentfill}%
\pgfsetlinewidth{0.803000pt}%
\definecolor{currentstroke}{rgb}{0.000000,0.000000,0.000000}%
\pgfsetstrokecolor{currentstroke}%
\pgfsetdash{}{0pt}%
\pgfsys@defobject{currentmarker}{\pgfqpoint{0.000000in}{-0.048611in}}{\pgfqpoint{0.000000in}{0.000000in}}{%
\pgfpathmoveto{\pgfqpoint{0.000000in}{0.000000in}}%
\pgfpathlineto{\pgfqpoint{0.000000in}{-0.048611in}}%
\pgfusepath{stroke,fill}%
}%
\begin{pgfscope}%
\pgfsys@transformshift{1.010526in}{4.500000in}%
\pgfsys@useobject{currentmarker}{}%
\end{pgfscope}%
\end{pgfscope}%
\begin{pgfscope}%
\definecolor{textcolor}{rgb}{0.000000,0.000000,0.000000}%
\pgfsetstrokecolor{textcolor}%
\pgfsetfillcolor{textcolor}%
\pgftext[x=1.010526in,y=0.551389in,,top]{\color{textcolor}\sffamily\fontsize{10.000000}{12.000000}\selectfont −3}%
\end{pgfscope}%
\begin{pgfscope}%
\pgfsetbuttcap%
\pgfsetroundjoin%
\definecolor{currentfill}{rgb}{0.000000,0.000000,0.000000}%
\pgfsetfillcolor{currentfill}%
\pgfsetlinewidth{0.803000pt}%
\definecolor{currentstroke}{rgb}{0.000000,0.000000,0.000000}%
\pgfsetstrokecolor{currentstroke}%
\pgfsetdash{}{0pt}%
\pgfsys@defobject{currentmarker}{\pgfqpoint{0.000000in}{0.000000in}}{\pgfqpoint{0.000000in}{0.048611in}}{%
\pgfpathmoveto{\pgfqpoint{0.000000in}{0.000000in}}%
\pgfpathlineto{\pgfqpoint{0.000000in}{0.048611in}}%
\pgfusepath{stroke,fill}%
}%
\begin{pgfscope}%
\pgfsys@transformshift{1.523684in}{0.600000in}%
\pgfsys@useobject{currentmarker}{}%
\end{pgfscope}%
\end{pgfscope}%
\begin{pgfscope}%
\pgfsetbuttcap%
\pgfsetroundjoin%
\definecolor{currentfill}{rgb}{0.000000,0.000000,0.000000}%
\pgfsetfillcolor{currentfill}%
\pgfsetlinewidth{0.803000pt}%
\definecolor{currentstroke}{rgb}{0.000000,0.000000,0.000000}%
\pgfsetstrokecolor{currentstroke}%
\pgfsetdash{}{0pt}%
\pgfsys@defobject{currentmarker}{\pgfqpoint{0.000000in}{-0.048611in}}{\pgfqpoint{0.000000in}{0.000000in}}{%
\pgfpathmoveto{\pgfqpoint{0.000000in}{0.000000in}}%
\pgfpathlineto{\pgfqpoint{0.000000in}{-0.048611in}}%
\pgfusepath{stroke,fill}%
}%
\begin{pgfscope}%
\pgfsys@transformshift{1.523684in}{4.500000in}%
\pgfsys@useobject{currentmarker}{}%
\end{pgfscope}%
\end{pgfscope}%
\begin{pgfscope}%
\definecolor{textcolor}{rgb}{0.000000,0.000000,0.000000}%
\pgfsetstrokecolor{textcolor}%
\pgfsetfillcolor{textcolor}%
\pgftext[x=1.523684in,y=0.551389in,,top]{\color{textcolor}\sffamily\fontsize{10.000000}{12.000000}\selectfont −2}%
\end{pgfscope}%
\begin{pgfscope}%
\pgfsetbuttcap%
\pgfsetroundjoin%
\definecolor{currentfill}{rgb}{0.000000,0.000000,0.000000}%
\pgfsetfillcolor{currentfill}%
\pgfsetlinewidth{0.803000pt}%
\definecolor{currentstroke}{rgb}{0.000000,0.000000,0.000000}%
\pgfsetstrokecolor{currentstroke}%
\pgfsetdash{}{0pt}%
\pgfsys@defobject{currentmarker}{\pgfqpoint{0.000000in}{0.000000in}}{\pgfqpoint{0.000000in}{0.048611in}}{%
\pgfpathmoveto{\pgfqpoint{0.000000in}{0.000000in}}%
\pgfpathlineto{\pgfqpoint{0.000000in}{0.048611in}}%
\pgfusepath{stroke,fill}%
}%
\begin{pgfscope}%
\pgfsys@transformshift{2.036842in}{0.600000in}%
\pgfsys@useobject{currentmarker}{}%
\end{pgfscope}%
\end{pgfscope}%
\begin{pgfscope}%
\pgfsetbuttcap%
\pgfsetroundjoin%
\definecolor{currentfill}{rgb}{0.000000,0.000000,0.000000}%
\pgfsetfillcolor{currentfill}%
\pgfsetlinewidth{0.803000pt}%
\definecolor{currentstroke}{rgb}{0.000000,0.000000,0.000000}%
\pgfsetstrokecolor{currentstroke}%
\pgfsetdash{}{0pt}%
\pgfsys@defobject{currentmarker}{\pgfqpoint{0.000000in}{-0.048611in}}{\pgfqpoint{0.000000in}{0.000000in}}{%
\pgfpathmoveto{\pgfqpoint{0.000000in}{0.000000in}}%
\pgfpathlineto{\pgfqpoint{0.000000in}{-0.048611in}}%
\pgfusepath{stroke,fill}%
}%
\begin{pgfscope}%
\pgfsys@transformshift{2.036842in}{4.500000in}%
\pgfsys@useobject{currentmarker}{}%
\end{pgfscope}%
\end{pgfscope}%
\begin{pgfscope}%
\definecolor{textcolor}{rgb}{0.000000,0.000000,0.000000}%
\pgfsetstrokecolor{textcolor}%
\pgfsetfillcolor{textcolor}%
\pgftext[x=2.036842in,y=0.551389in,,top]{\color{textcolor}\sffamily\fontsize{10.000000}{12.000000}\selectfont −1}%
\end{pgfscope}%
\begin{pgfscope}%
\pgfsetbuttcap%
\pgfsetroundjoin%
\definecolor{currentfill}{rgb}{0.000000,0.000000,0.000000}%
\pgfsetfillcolor{currentfill}%
\pgfsetlinewidth{0.803000pt}%
\definecolor{currentstroke}{rgb}{0.000000,0.000000,0.000000}%
\pgfsetstrokecolor{currentstroke}%
\pgfsetdash{}{0pt}%
\pgfsys@defobject{currentmarker}{\pgfqpoint{0.000000in}{0.000000in}}{\pgfqpoint{0.000000in}{0.048611in}}{%
\pgfpathmoveto{\pgfqpoint{0.000000in}{0.000000in}}%
\pgfpathlineto{\pgfqpoint{0.000000in}{0.048611in}}%
\pgfusepath{stroke,fill}%
}%
\begin{pgfscope}%
\pgfsys@transformshift{2.550000in}{0.600000in}%
\pgfsys@useobject{currentmarker}{}%
\end{pgfscope}%
\end{pgfscope}%
\begin{pgfscope}%
\pgfsetbuttcap%
\pgfsetroundjoin%
\definecolor{currentfill}{rgb}{0.000000,0.000000,0.000000}%
\pgfsetfillcolor{currentfill}%
\pgfsetlinewidth{0.803000pt}%
\definecolor{currentstroke}{rgb}{0.000000,0.000000,0.000000}%
\pgfsetstrokecolor{currentstroke}%
\pgfsetdash{}{0pt}%
\pgfsys@defobject{currentmarker}{\pgfqpoint{0.000000in}{-0.048611in}}{\pgfqpoint{0.000000in}{0.000000in}}{%
\pgfpathmoveto{\pgfqpoint{0.000000in}{0.000000in}}%
\pgfpathlineto{\pgfqpoint{0.000000in}{-0.048611in}}%
\pgfusepath{stroke,fill}%
}%
\begin{pgfscope}%
\pgfsys@transformshift{2.550000in}{4.500000in}%
\pgfsys@useobject{currentmarker}{}%
\end{pgfscope}%
\end{pgfscope}%
\begin{pgfscope}%
\definecolor{textcolor}{rgb}{0.000000,0.000000,0.000000}%
\pgfsetstrokecolor{textcolor}%
\pgfsetfillcolor{textcolor}%
\pgftext[x=2.550000in,y=0.551389in,,top]{\color{textcolor}\sffamily\fontsize{10.000000}{12.000000}\selectfont 0}%
\end{pgfscope}%
\begin{pgfscope}%
\pgfsetbuttcap%
\pgfsetroundjoin%
\definecolor{currentfill}{rgb}{0.000000,0.000000,0.000000}%
\pgfsetfillcolor{currentfill}%
\pgfsetlinewidth{0.803000pt}%
\definecolor{currentstroke}{rgb}{0.000000,0.000000,0.000000}%
\pgfsetstrokecolor{currentstroke}%
\pgfsetdash{}{0pt}%
\pgfsys@defobject{currentmarker}{\pgfqpoint{0.000000in}{0.000000in}}{\pgfqpoint{0.000000in}{0.048611in}}{%
\pgfpathmoveto{\pgfqpoint{0.000000in}{0.000000in}}%
\pgfpathlineto{\pgfqpoint{0.000000in}{0.048611in}}%
\pgfusepath{stroke,fill}%
}%
\begin{pgfscope}%
\pgfsys@transformshift{3.063158in}{0.600000in}%
\pgfsys@useobject{currentmarker}{}%
\end{pgfscope}%
\end{pgfscope}%
\begin{pgfscope}%
\pgfsetbuttcap%
\pgfsetroundjoin%
\definecolor{currentfill}{rgb}{0.000000,0.000000,0.000000}%
\pgfsetfillcolor{currentfill}%
\pgfsetlinewidth{0.803000pt}%
\definecolor{currentstroke}{rgb}{0.000000,0.000000,0.000000}%
\pgfsetstrokecolor{currentstroke}%
\pgfsetdash{}{0pt}%
\pgfsys@defobject{currentmarker}{\pgfqpoint{0.000000in}{-0.048611in}}{\pgfqpoint{0.000000in}{0.000000in}}{%
\pgfpathmoveto{\pgfqpoint{0.000000in}{0.000000in}}%
\pgfpathlineto{\pgfqpoint{0.000000in}{-0.048611in}}%
\pgfusepath{stroke,fill}%
}%
\begin{pgfscope}%
\pgfsys@transformshift{3.063158in}{4.500000in}%
\pgfsys@useobject{currentmarker}{}%
\end{pgfscope}%
\end{pgfscope}%
\begin{pgfscope}%
\definecolor{textcolor}{rgb}{0.000000,0.000000,0.000000}%
\pgfsetstrokecolor{textcolor}%
\pgfsetfillcolor{textcolor}%
\pgftext[x=3.063158in,y=0.551389in,,top]{\color{textcolor}\sffamily\fontsize{10.000000}{12.000000}\selectfont 1}%
\end{pgfscope}%
\begin{pgfscope}%
\pgfsetbuttcap%
\pgfsetroundjoin%
\definecolor{currentfill}{rgb}{0.000000,0.000000,0.000000}%
\pgfsetfillcolor{currentfill}%
\pgfsetlinewidth{0.803000pt}%
\definecolor{currentstroke}{rgb}{0.000000,0.000000,0.000000}%
\pgfsetstrokecolor{currentstroke}%
\pgfsetdash{}{0pt}%
\pgfsys@defobject{currentmarker}{\pgfqpoint{0.000000in}{0.000000in}}{\pgfqpoint{0.000000in}{0.048611in}}{%
\pgfpathmoveto{\pgfqpoint{0.000000in}{0.000000in}}%
\pgfpathlineto{\pgfqpoint{0.000000in}{0.048611in}}%
\pgfusepath{stroke,fill}%
}%
\begin{pgfscope}%
\pgfsys@transformshift{3.576316in}{0.600000in}%
\pgfsys@useobject{currentmarker}{}%
\end{pgfscope}%
\end{pgfscope}%
\begin{pgfscope}%
\pgfsetbuttcap%
\pgfsetroundjoin%
\definecolor{currentfill}{rgb}{0.000000,0.000000,0.000000}%
\pgfsetfillcolor{currentfill}%
\pgfsetlinewidth{0.803000pt}%
\definecolor{currentstroke}{rgb}{0.000000,0.000000,0.000000}%
\pgfsetstrokecolor{currentstroke}%
\pgfsetdash{}{0pt}%
\pgfsys@defobject{currentmarker}{\pgfqpoint{0.000000in}{-0.048611in}}{\pgfqpoint{0.000000in}{0.000000in}}{%
\pgfpathmoveto{\pgfqpoint{0.000000in}{0.000000in}}%
\pgfpathlineto{\pgfqpoint{0.000000in}{-0.048611in}}%
\pgfusepath{stroke,fill}%
}%
\begin{pgfscope}%
\pgfsys@transformshift{3.576316in}{4.500000in}%
\pgfsys@useobject{currentmarker}{}%
\end{pgfscope}%
\end{pgfscope}%
\begin{pgfscope}%
\definecolor{textcolor}{rgb}{0.000000,0.000000,0.000000}%
\pgfsetstrokecolor{textcolor}%
\pgfsetfillcolor{textcolor}%
\pgftext[x=3.576316in,y=0.551389in,,top]{\color{textcolor}\sffamily\fontsize{10.000000}{12.000000}\selectfont 2}%
\end{pgfscope}%
\begin{pgfscope}%
\pgfsetbuttcap%
\pgfsetroundjoin%
\definecolor{currentfill}{rgb}{0.000000,0.000000,0.000000}%
\pgfsetfillcolor{currentfill}%
\pgfsetlinewidth{0.803000pt}%
\definecolor{currentstroke}{rgb}{0.000000,0.000000,0.000000}%
\pgfsetstrokecolor{currentstroke}%
\pgfsetdash{}{0pt}%
\pgfsys@defobject{currentmarker}{\pgfqpoint{0.000000in}{0.000000in}}{\pgfqpoint{0.000000in}{0.048611in}}{%
\pgfpathmoveto{\pgfqpoint{0.000000in}{0.000000in}}%
\pgfpathlineto{\pgfqpoint{0.000000in}{0.048611in}}%
\pgfusepath{stroke,fill}%
}%
\begin{pgfscope}%
\pgfsys@transformshift{4.089474in}{0.600000in}%
\pgfsys@useobject{currentmarker}{}%
\end{pgfscope}%
\end{pgfscope}%
\begin{pgfscope}%
\pgfsetbuttcap%
\pgfsetroundjoin%
\definecolor{currentfill}{rgb}{0.000000,0.000000,0.000000}%
\pgfsetfillcolor{currentfill}%
\pgfsetlinewidth{0.803000pt}%
\definecolor{currentstroke}{rgb}{0.000000,0.000000,0.000000}%
\pgfsetstrokecolor{currentstroke}%
\pgfsetdash{}{0pt}%
\pgfsys@defobject{currentmarker}{\pgfqpoint{0.000000in}{-0.048611in}}{\pgfqpoint{0.000000in}{0.000000in}}{%
\pgfpathmoveto{\pgfqpoint{0.000000in}{0.000000in}}%
\pgfpathlineto{\pgfqpoint{0.000000in}{-0.048611in}}%
\pgfusepath{stroke,fill}%
}%
\begin{pgfscope}%
\pgfsys@transformshift{4.089474in}{4.500000in}%
\pgfsys@useobject{currentmarker}{}%
\end{pgfscope}%
\end{pgfscope}%
\begin{pgfscope}%
\definecolor{textcolor}{rgb}{0.000000,0.000000,0.000000}%
\pgfsetstrokecolor{textcolor}%
\pgfsetfillcolor{textcolor}%
\pgftext[x=4.089474in,y=0.551389in,,top]{\color{textcolor}\sffamily\fontsize{10.000000}{12.000000}\selectfont 3}%
\end{pgfscope}%
\begin{pgfscope}%
\pgfsetbuttcap%
\pgfsetroundjoin%
\definecolor{currentfill}{rgb}{0.000000,0.000000,0.000000}%
\pgfsetfillcolor{currentfill}%
\pgfsetlinewidth{0.803000pt}%
\definecolor{currentstroke}{rgb}{0.000000,0.000000,0.000000}%
\pgfsetstrokecolor{currentstroke}%
\pgfsetdash{}{0pt}%
\pgfsys@defobject{currentmarker}{\pgfqpoint{0.000000in}{0.000000in}}{\pgfqpoint{0.048611in}{0.000000in}}{%
\pgfpathmoveto{\pgfqpoint{0.000000in}{0.000000in}}%
\pgfpathlineto{\pgfqpoint{0.048611in}{0.000000in}}%
\pgfusepath{stroke,fill}%
}%
\begin{pgfscope}%
\pgfsys@transformshift{0.600000in}{1.010526in}%
\pgfsys@useobject{currentmarker}{}%
\end{pgfscope}%
\end{pgfscope}%
\begin{pgfscope}%
\pgfsetbuttcap%
\pgfsetroundjoin%
\definecolor{currentfill}{rgb}{0.000000,0.000000,0.000000}%
\pgfsetfillcolor{currentfill}%
\pgfsetlinewidth{0.803000pt}%
\definecolor{currentstroke}{rgb}{0.000000,0.000000,0.000000}%
\pgfsetstrokecolor{currentstroke}%
\pgfsetdash{}{0pt}%
\pgfsys@defobject{currentmarker}{\pgfqpoint{-0.048611in}{0.000000in}}{\pgfqpoint{0.000000in}{0.000000in}}{%
\pgfpathmoveto{\pgfqpoint{0.000000in}{0.000000in}}%
\pgfpathlineto{\pgfqpoint{-0.048611in}{0.000000in}}%
\pgfusepath{stroke,fill}%
}%
\begin{pgfscope}%
\pgfsys@transformshift{4.500000in}{1.010526in}%
\pgfsys@useobject{currentmarker}{}%
\end{pgfscope}%
\end{pgfscope}%
\begin{pgfscope}%
\definecolor{textcolor}{rgb}{0.000000,0.000000,0.000000}%
\pgfsetstrokecolor{textcolor}%
\pgfsetfillcolor{textcolor}%
\pgftext[x=0.346650in,y=0.957765in,left,base]{\color{textcolor}\sffamily\fontsize{10.000000}{12.000000}\selectfont −3}%
\end{pgfscope}%
\begin{pgfscope}%
\pgfsetbuttcap%
\pgfsetroundjoin%
\definecolor{currentfill}{rgb}{0.000000,0.000000,0.000000}%
\pgfsetfillcolor{currentfill}%
\pgfsetlinewidth{0.803000pt}%
\definecolor{currentstroke}{rgb}{0.000000,0.000000,0.000000}%
\pgfsetstrokecolor{currentstroke}%
\pgfsetdash{}{0pt}%
\pgfsys@defobject{currentmarker}{\pgfqpoint{0.000000in}{0.000000in}}{\pgfqpoint{0.048611in}{0.000000in}}{%
\pgfpathmoveto{\pgfqpoint{0.000000in}{0.000000in}}%
\pgfpathlineto{\pgfqpoint{0.048611in}{0.000000in}}%
\pgfusepath{stroke,fill}%
}%
\begin{pgfscope}%
\pgfsys@transformshift{0.600000in}{1.523684in}%
\pgfsys@useobject{currentmarker}{}%
\end{pgfscope}%
\end{pgfscope}%
\begin{pgfscope}%
\pgfsetbuttcap%
\pgfsetroundjoin%
\definecolor{currentfill}{rgb}{0.000000,0.000000,0.000000}%
\pgfsetfillcolor{currentfill}%
\pgfsetlinewidth{0.803000pt}%
\definecolor{currentstroke}{rgb}{0.000000,0.000000,0.000000}%
\pgfsetstrokecolor{currentstroke}%
\pgfsetdash{}{0pt}%
\pgfsys@defobject{currentmarker}{\pgfqpoint{-0.048611in}{0.000000in}}{\pgfqpoint{0.000000in}{0.000000in}}{%
\pgfpathmoveto{\pgfqpoint{0.000000in}{0.000000in}}%
\pgfpathlineto{\pgfqpoint{-0.048611in}{0.000000in}}%
\pgfusepath{stroke,fill}%
}%
\begin{pgfscope}%
\pgfsys@transformshift{4.500000in}{1.523684in}%
\pgfsys@useobject{currentmarker}{}%
\end{pgfscope}%
\end{pgfscope}%
\begin{pgfscope}%
\definecolor{textcolor}{rgb}{0.000000,0.000000,0.000000}%
\pgfsetstrokecolor{textcolor}%
\pgfsetfillcolor{textcolor}%
\pgftext[x=0.346650in,y=1.470923in,left,base]{\color{textcolor}\sffamily\fontsize{10.000000}{12.000000}\selectfont −2}%
\end{pgfscope}%
\begin{pgfscope}%
\pgfsetbuttcap%
\pgfsetroundjoin%
\definecolor{currentfill}{rgb}{0.000000,0.000000,0.000000}%
\pgfsetfillcolor{currentfill}%
\pgfsetlinewidth{0.803000pt}%
\definecolor{currentstroke}{rgb}{0.000000,0.000000,0.000000}%
\pgfsetstrokecolor{currentstroke}%
\pgfsetdash{}{0pt}%
\pgfsys@defobject{currentmarker}{\pgfqpoint{0.000000in}{0.000000in}}{\pgfqpoint{0.048611in}{0.000000in}}{%
\pgfpathmoveto{\pgfqpoint{0.000000in}{0.000000in}}%
\pgfpathlineto{\pgfqpoint{0.048611in}{0.000000in}}%
\pgfusepath{stroke,fill}%
}%
\begin{pgfscope}%
\pgfsys@transformshift{0.600000in}{2.036842in}%
\pgfsys@useobject{currentmarker}{}%
\end{pgfscope}%
\end{pgfscope}%
\begin{pgfscope}%
\pgfsetbuttcap%
\pgfsetroundjoin%
\definecolor{currentfill}{rgb}{0.000000,0.000000,0.000000}%
\pgfsetfillcolor{currentfill}%
\pgfsetlinewidth{0.803000pt}%
\definecolor{currentstroke}{rgb}{0.000000,0.000000,0.000000}%
\pgfsetstrokecolor{currentstroke}%
\pgfsetdash{}{0pt}%
\pgfsys@defobject{currentmarker}{\pgfqpoint{-0.048611in}{0.000000in}}{\pgfqpoint{0.000000in}{0.000000in}}{%
\pgfpathmoveto{\pgfqpoint{0.000000in}{0.000000in}}%
\pgfpathlineto{\pgfqpoint{-0.048611in}{0.000000in}}%
\pgfusepath{stroke,fill}%
}%
\begin{pgfscope}%
\pgfsys@transformshift{4.500000in}{2.036842in}%
\pgfsys@useobject{currentmarker}{}%
\end{pgfscope}%
\end{pgfscope}%
\begin{pgfscope}%
\definecolor{textcolor}{rgb}{0.000000,0.000000,0.000000}%
\pgfsetstrokecolor{textcolor}%
\pgfsetfillcolor{textcolor}%
\pgftext[x=0.346650in,y=1.984081in,left,base]{\color{textcolor}\sffamily\fontsize{10.000000}{12.000000}\selectfont −1}%
\end{pgfscope}%
\begin{pgfscope}%
\pgfsetbuttcap%
\pgfsetroundjoin%
\definecolor{currentfill}{rgb}{0.000000,0.000000,0.000000}%
\pgfsetfillcolor{currentfill}%
\pgfsetlinewidth{0.803000pt}%
\definecolor{currentstroke}{rgb}{0.000000,0.000000,0.000000}%
\pgfsetstrokecolor{currentstroke}%
\pgfsetdash{}{0pt}%
\pgfsys@defobject{currentmarker}{\pgfqpoint{0.000000in}{0.000000in}}{\pgfqpoint{0.048611in}{0.000000in}}{%
\pgfpathmoveto{\pgfqpoint{0.000000in}{0.000000in}}%
\pgfpathlineto{\pgfqpoint{0.048611in}{0.000000in}}%
\pgfusepath{stroke,fill}%
}%
\begin{pgfscope}%
\pgfsys@transformshift{0.600000in}{2.550000in}%
\pgfsys@useobject{currentmarker}{}%
\end{pgfscope}%
\end{pgfscope}%
\begin{pgfscope}%
\pgfsetbuttcap%
\pgfsetroundjoin%
\definecolor{currentfill}{rgb}{0.000000,0.000000,0.000000}%
\pgfsetfillcolor{currentfill}%
\pgfsetlinewidth{0.803000pt}%
\definecolor{currentstroke}{rgb}{0.000000,0.000000,0.000000}%
\pgfsetstrokecolor{currentstroke}%
\pgfsetdash{}{0pt}%
\pgfsys@defobject{currentmarker}{\pgfqpoint{-0.048611in}{0.000000in}}{\pgfqpoint{0.000000in}{0.000000in}}{%
\pgfpathmoveto{\pgfqpoint{0.000000in}{0.000000in}}%
\pgfpathlineto{\pgfqpoint{-0.048611in}{0.000000in}}%
\pgfusepath{stroke,fill}%
}%
\begin{pgfscope}%
\pgfsys@transformshift{4.500000in}{2.550000in}%
\pgfsys@useobject{currentmarker}{}%
\end{pgfscope}%
\end{pgfscope}%
\begin{pgfscope}%
\definecolor{textcolor}{rgb}{0.000000,0.000000,0.000000}%
\pgfsetstrokecolor{textcolor}%
\pgfsetfillcolor{textcolor}%
\pgftext[x=0.463024in,y=2.497238in,left,base]{\color{textcolor}\sffamily\fontsize{10.000000}{12.000000}\selectfont 0}%
\end{pgfscope}%
\begin{pgfscope}%
\pgfsetbuttcap%
\pgfsetroundjoin%
\definecolor{currentfill}{rgb}{0.000000,0.000000,0.000000}%
\pgfsetfillcolor{currentfill}%
\pgfsetlinewidth{0.803000pt}%
\definecolor{currentstroke}{rgb}{0.000000,0.000000,0.000000}%
\pgfsetstrokecolor{currentstroke}%
\pgfsetdash{}{0pt}%
\pgfsys@defobject{currentmarker}{\pgfqpoint{0.000000in}{0.000000in}}{\pgfqpoint{0.048611in}{0.000000in}}{%
\pgfpathmoveto{\pgfqpoint{0.000000in}{0.000000in}}%
\pgfpathlineto{\pgfqpoint{0.048611in}{0.000000in}}%
\pgfusepath{stroke,fill}%
}%
\begin{pgfscope}%
\pgfsys@transformshift{0.600000in}{3.063158in}%
\pgfsys@useobject{currentmarker}{}%
\end{pgfscope}%
\end{pgfscope}%
\begin{pgfscope}%
\pgfsetbuttcap%
\pgfsetroundjoin%
\definecolor{currentfill}{rgb}{0.000000,0.000000,0.000000}%
\pgfsetfillcolor{currentfill}%
\pgfsetlinewidth{0.803000pt}%
\definecolor{currentstroke}{rgb}{0.000000,0.000000,0.000000}%
\pgfsetstrokecolor{currentstroke}%
\pgfsetdash{}{0pt}%
\pgfsys@defobject{currentmarker}{\pgfqpoint{-0.048611in}{0.000000in}}{\pgfqpoint{0.000000in}{0.000000in}}{%
\pgfpathmoveto{\pgfqpoint{0.000000in}{0.000000in}}%
\pgfpathlineto{\pgfqpoint{-0.048611in}{0.000000in}}%
\pgfusepath{stroke,fill}%
}%
\begin{pgfscope}%
\pgfsys@transformshift{4.500000in}{3.063158in}%
\pgfsys@useobject{currentmarker}{}%
\end{pgfscope}%
\end{pgfscope}%
\begin{pgfscope}%
\definecolor{textcolor}{rgb}{0.000000,0.000000,0.000000}%
\pgfsetstrokecolor{textcolor}%
\pgfsetfillcolor{textcolor}%
\pgftext[x=0.463024in,y=3.010396in,left,base]{\color{textcolor}\sffamily\fontsize{10.000000}{12.000000}\selectfont 1}%
\end{pgfscope}%
\begin{pgfscope}%
\pgfsetbuttcap%
\pgfsetroundjoin%
\definecolor{currentfill}{rgb}{0.000000,0.000000,0.000000}%
\pgfsetfillcolor{currentfill}%
\pgfsetlinewidth{0.803000pt}%
\definecolor{currentstroke}{rgb}{0.000000,0.000000,0.000000}%
\pgfsetstrokecolor{currentstroke}%
\pgfsetdash{}{0pt}%
\pgfsys@defobject{currentmarker}{\pgfqpoint{0.000000in}{0.000000in}}{\pgfqpoint{0.048611in}{0.000000in}}{%
\pgfpathmoveto{\pgfqpoint{0.000000in}{0.000000in}}%
\pgfpathlineto{\pgfqpoint{0.048611in}{0.000000in}}%
\pgfusepath{stroke,fill}%
}%
\begin{pgfscope}%
\pgfsys@transformshift{0.600000in}{3.576316in}%
\pgfsys@useobject{currentmarker}{}%
\end{pgfscope}%
\end{pgfscope}%
\begin{pgfscope}%
\pgfsetbuttcap%
\pgfsetroundjoin%
\definecolor{currentfill}{rgb}{0.000000,0.000000,0.000000}%
\pgfsetfillcolor{currentfill}%
\pgfsetlinewidth{0.803000pt}%
\definecolor{currentstroke}{rgb}{0.000000,0.000000,0.000000}%
\pgfsetstrokecolor{currentstroke}%
\pgfsetdash{}{0pt}%
\pgfsys@defobject{currentmarker}{\pgfqpoint{-0.048611in}{0.000000in}}{\pgfqpoint{0.000000in}{0.000000in}}{%
\pgfpathmoveto{\pgfqpoint{0.000000in}{0.000000in}}%
\pgfpathlineto{\pgfqpoint{-0.048611in}{0.000000in}}%
\pgfusepath{stroke,fill}%
}%
\begin{pgfscope}%
\pgfsys@transformshift{4.500000in}{3.576316in}%
\pgfsys@useobject{currentmarker}{}%
\end{pgfscope}%
\end{pgfscope}%
\begin{pgfscope}%
\definecolor{textcolor}{rgb}{0.000000,0.000000,0.000000}%
\pgfsetstrokecolor{textcolor}%
\pgfsetfillcolor{textcolor}%
\pgftext[x=0.463024in,y=3.523554in,left,base]{\color{textcolor}\sffamily\fontsize{10.000000}{12.000000}\selectfont 2}%
\end{pgfscope}%
\begin{pgfscope}%
\pgfsetbuttcap%
\pgfsetroundjoin%
\definecolor{currentfill}{rgb}{0.000000,0.000000,0.000000}%
\pgfsetfillcolor{currentfill}%
\pgfsetlinewidth{0.803000pt}%
\definecolor{currentstroke}{rgb}{0.000000,0.000000,0.000000}%
\pgfsetstrokecolor{currentstroke}%
\pgfsetdash{}{0pt}%
\pgfsys@defobject{currentmarker}{\pgfqpoint{0.000000in}{0.000000in}}{\pgfqpoint{0.048611in}{0.000000in}}{%
\pgfpathmoveto{\pgfqpoint{0.000000in}{0.000000in}}%
\pgfpathlineto{\pgfqpoint{0.048611in}{0.000000in}}%
\pgfusepath{stroke,fill}%
}%
\begin{pgfscope}%
\pgfsys@transformshift{0.600000in}{4.089474in}%
\pgfsys@useobject{currentmarker}{}%
\end{pgfscope}%
\end{pgfscope}%
\begin{pgfscope}%
\pgfsetbuttcap%
\pgfsetroundjoin%
\definecolor{currentfill}{rgb}{0.000000,0.000000,0.000000}%
\pgfsetfillcolor{currentfill}%
\pgfsetlinewidth{0.803000pt}%
\definecolor{currentstroke}{rgb}{0.000000,0.000000,0.000000}%
\pgfsetstrokecolor{currentstroke}%
\pgfsetdash{}{0pt}%
\pgfsys@defobject{currentmarker}{\pgfqpoint{-0.048611in}{0.000000in}}{\pgfqpoint{0.000000in}{0.000000in}}{%
\pgfpathmoveto{\pgfqpoint{0.000000in}{0.000000in}}%
\pgfpathlineto{\pgfqpoint{-0.048611in}{0.000000in}}%
\pgfusepath{stroke,fill}%
}%
\begin{pgfscope}%
\pgfsys@transformshift{4.500000in}{4.089474in}%
\pgfsys@useobject{currentmarker}{}%
\end{pgfscope}%
\end{pgfscope}%
\begin{pgfscope}%
\definecolor{textcolor}{rgb}{0.000000,0.000000,0.000000}%
\pgfsetstrokecolor{textcolor}%
\pgfsetfillcolor{textcolor}%
\pgftext[x=0.463024in,y=4.036712in,left,base]{\color{textcolor}\sffamily\fontsize{10.000000}{12.000000}\selectfont 3}%
\end{pgfscope}%
\begin{pgfscope}%
\pgfsetrectcap%
\pgfsetmiterjoin%
\pgfsetlinewidth{0.803000pt}%
\definecolor{currentstroke}{rgb}{0.000000,0.000000,0.000000}%
\pgfsetstrokecolor{currentstroke}%
\pgfsetdash{}{0pt}%
\pgfpathmoveto{\pgfqpoint{0.600000in}{0.600000in}}%
\pgfpathlineto{\pgfqpoint{0.600000in}{4.500000in}}%
\pgfusepath{stroke}%
\end{pgfscope}%
\begin{pgfscope}%
\pgfsetrectcap%
\pgfsetmiterjoin%
\pgfsetlinewidth{0.803000pt}%
\definecolor{currentstroke}{rgb}{0.000000,0.000000,0.000000}%
\pgfsetstrokecolor{currentstroke}%
\pgfsetdash{}{0pt}%
\pgfpathmoveto{\pgfqpoint{4.500000in}{0.600000in}}%
\pgfpathlineto{\pgfqpoint{4.500000in}{4.500000in}}%
\pgfusepath{stroke}%
\end{pgfscope}%
\begin{pgfscope}%
\pgfsetrectcap%
\pgfsetmiterjoin%
\pgfsetlinewidth{0.803000pt}%
\definecolor{currentstroke}{rgb}{0.000000,0.000000,0.000000}%
\pgfsetstrokecolor{currentstroke}%
\pgfsetdash{}{0pt}%
\pgfpathmoveto{\pgfqpoint{0.600000in}{0.600000in}}%
\pgfpathlineto{\pgfqpoint{4.500000in}{0.600000in}}%
\pgfusepath{stroke}%
\end{pgfscope}%
\begin{pgfscope}%
\pgfsetrectcap%
\pgfsetmiterjoin%
\pgfsetlinewidth{0.803000pt}%
\definecolor{currentstroke}{rgb}{0.000000,0.000000,0.000000}%
\pgfsetstrokecolor{currentstroke}%
\pgfsetdash{}{0pt}%
\pgfpathmoveto{\pgfqpoint{0.600000in}{4.500000in}}%
\pgfpathlineto{\pgfqpoint{4.500000in}{4.500000in}}%
\pgfusepath{stroke}%
\end{pgfscope}%
\begin{pgfscope}%
\pgfsetbuttcap%
\pgfsetmiterjoin%
\definecolor{currentfill}{rgb}{1.000000,1.000000,1.000000}%
\pgfsetfillcolor{currentfill}%
\pgfsetlinewidth{0.000000pt}%
\definecolor{currentstroke}{rgb}{0.000000,0.000000,0.000000}%
\pgfsetstrokecolor{currentstroke}%
\pgfsetstrokeopacity{0.000000}%
\pgfsetdash{}{0pt}%
\pgfpathmoveto{\pgfqpoint{0.600000in}{4.530000in}}%
\pgfpathlineto{\pgfqpoint{4.500000in}{4.530000in}}%
\pgfpathlineto{\pgfqpoint{4.500000in}{5.730000in}}%
\pgfpathlineto{\pgfqpoint{0.600000in}{5.730000in}}%
\pgfpathclose%
\pgfusepath{fill}%
\end{pgfscope}%
\begin{pgfscope}%
\pgfpathrectangle{\pgfqpoint{0.600000in}{4.530000in}}{\pgfqpoint{3.900000in}{1.200000in}}%
\pgfusepath{clip}%
\pgfsetbuttcap%
\pgfsetmiterjoin%
\definecolor{currentfill}{rgb}{0.121569,0.466667,0.705882}%
\pgfsetfillcolor{currentfill}%
\pgfsetlinewidth{0.000000pt}%
\definecolor{currentstroke}{rgb}{0.000000,0.000000,0.000000}%
\pgfsetstrokecolor{currentstroke}%
\pgfsetstrokeopacity{0.000000}%
\pgfsetdash{}{0pt}%
\pgfpathmoveto{\pgfqpoint{0.600000in}{4.530000in}}%
\pgfpathlineto{\pgfqpoint{0.728289in}{4.530000in}}%
\pgfpathlineto{\pgfqpoint{0.728289in}{4.541204in}}%
\pgfpathlineto{\pgfqpoint{0.600000in}{4.541204in}}%
\pgfpathclose%
\pgfusepath{fill}%
\end{pgfscope}%
\begin{pgfscope}%
\pgfpathrectangle{\pgfqpoint{0.600000in}{4.530000in}}{\pgfqpoint{3.900000in}{1.200000in}}%
\pgfusepath{clip}%
\pgfsetbuttcap%
\pgfsetmiterjoin%
\definecolor{currentfill}{rgb}{0.121569,0.466667,0.705882}%
\pgfsetfillcolor{currentfill}%
\pgfsetlinewidth{0.000000pt}%
\definecolor{currentstroke}{rgb}{0.000000,0.000000,0.000000}%
\pgfsetstrokecolor{currentstroke}%
\pgfsetstrokeopacity{0.000000}%
\pgfsetdash{}{0pt}%
\pgfpathmoveto{\pgfqpoint{0.728289in}{4.530000in}}%
\pgfpathlineto{\pgfqpoint{0.856579in}{4.530000in}}%
\pgfpathlineto{\pgfqpoint{0.856579in}{4.530000in}}%
\pgfpathlineto{\pgfqpoint{0.728289in}{4.530000in}}%
\pgfpathclose%
\pgfusepath{fill}%
\end{pgfscope}%
\begin{pgfscope}%
\pgfpathrectangle{\pgfqpoint{0.600000in}{4.530000in}}{\pgfqpoint{3.900000in}{1.200000in}}%
\pgfusepath{clip}%
\pgfsetbuttcap%
\pgfsetmiterjoin%
\definecolor{currentfill}{rgb}{0.121569,0.466667,0.705882}%
\pgfsetfillcolor{currentfill}%
\pgfsetlinewidth{0.000000pt}%
\definecolor{currentstroke}{rgb}{0.000000,0.000000,0.000000}%
\pgfsetstrokecolor{currentstroke}%
\pgfsetstrokeopacity{0.000000}%
\pgfsetdash{}{0pt}%
\pgfpathmoveto{\pgfqpoint{0.856579in}{4.530000in}}%
\pgfpathlineto{\pgfqpoint{0.984868in}{4.530000in}}%
\pgfpathlineto{\pgfqpoint{0.984868in}{4.530000in}}%
\pgfpathlineto{\pgfqpoint{0.856579in}{4.530000in}}%
\pgfpathclose%
\pgfusepath{fill}%
\end{pgfscope}%
\begin{pgfscope}%
\pgfpathrectangle{\pgfqpoint{0.600000in}{4.530000in}}{\pgfqpoint{3.900000in}{1.200000in}}%
\pgfusepath{clip}%
\pgfsetbuttcap%
\pgfsetmiterjoin%
\definecolor{currentfill}{rgb}{0.121569,0.466667,0.705882}%
\pgfsetfillcolor{currentfill}%
\pgfsetlinewidth{0.000000pt}%
\definecolor{currentstroke}{rgb}{0.000000,0.000000,0.000000}%
\pgfsetstrokecolor{currentstroke}%
\pgfsetstrokeopacity{0.000000}%
\pgfsetdash{}{0pt}%
\pgfpathmoveto{\pgfqpoint{0.984868in}{4.530000in}}%
\pgfpathlineto{\pgfqpoint{1.113158in}{4.530000in}}%
\pgfpathlineto{\pgfqpoint{1.113158in}{4.541204in}}%
\pgfpathlineto{\pgfqpoint{0.984868in}{4.541204in}}%
\pgfpathclose%
\pgfusepath{fill}%
\end{pgfscope}%
\begin{pgfscope}%
\pgfpathrectangle{\pgfqpoint{0.600000in}{4.530000in}}{\pgfqpoint{3.900000in}{1.200000in}}%
\pgfusepath{clip}%
\pgfsetbuttcap%
\pgfsetmiterjoin%
\definecolor{currentfill}{rgb}{0.121569,0.466667,0.705882}%
\pgfsetfillcolor{currentfill}%
\pgfsetlinewidth{0.000000pt}%
\definecolor{currentstroke}{rgb}{0.000000,0.000000,0.000000}%
\pgfsetstrokecolor{currentstroke}%
\pgfsetstrokeopacity{0.000000}%
\pgfsetdash{}{0pt}%
\pgfpathmoveto{\pgfqpoint{1.113158in}{4.530000in}}%
\pgfpathlineto{\pgfqpoint{1.241447in}{4.530000in}}%
\pgfpathlineto{\pgfqpoint{1.241447in}{4.530000in}}%
\pgfpathlineto{\pgfqpoint{1.113158in}{4.530000in}}%
\pgfpathclose%
\pgfusepath{fill}%
\end{pgfscope}%
\begin{pgfscope}%
\pgfpathrectangle{\pgfqpoint{0.600000in}{4.530000in}}{\pgfqpoint{3.900000in}{1.200000in}}%
\pgfusepath{clip}%
\pgfsetbuttcap%
\pgfsetmiterjoin%
\definecolor{currentfill}{rgb}{0.121569,0.466667,0.705882}%
\pgfsetfillcolor{currentfill}%
\pgfsetlinewidth{0.000000pt}%
\definecolor{currentstroke}{rgb}{0.000000,0.000000,0.000000}%
\pgfsetstrokecolor{currentstroke}%
\pgfsetstrokeopacity{0.000000}%
\pgfsetdash{}{0pt}%
\pgfpathmoveto{\pgfqpoint{1.241447in}{4.530000in}}%
\pgfpathlineto{\pgfqpoint{1.369737in}{4.530000in}}%
\pgfpathlineto{\pgfqpoint{1.369737in}{4.574818in}}%
\pgfpathlineto{\pgfqpoint{1.241447in}{4.574818in}}%
\pgfpathclose%
\pgfusepath{fill}%
\end{pgfscope}%
\begin{pgfscope}%
\pgfpathrectangle{\pgfqpoint{0.600000in}{4.530000in}}{\pgfqpoint{3.900000in}{1.200000in}}%
\pgfusepath{clip}%
\pgfsetbuttcap%
\pgfsetmiterjoin%
\definecolor{currentfill}{rgb}{0.121569,0.466667,0.705882}%
\pgfsetfillcolor{currentfill}%
\pgfsetlinewidth{0.000000pt}%
\definecolor{currentstroke}{rgb}{0.000000,0.000000,0.000000}%
\pgfsetstrokecolor{currentstroke}%
\pgfsetstrokeopacity{0.000000}%
\pgfsetdash{}{0pt}%
\pgfpathmoveto{\pgfqpoint{1.369737in}{4.530000in}}%
\pgfpathlineto{\pgfqpoint{1.498026in}{4.530000in}}%
\pgfpathlineto{\pgfqpoint{1.498026in}{4.586022in}}%
\pgfpathlineto{\pgfqpoint{1.369737in}{4.586022in}}%
\pgfpathclose%
\pgfusepath{fill}%
\end{pgfscope}%
\begin{pgfscope}%
\pgfpathrectangle{\pgfqpoint{0.600000in}{4.530000in}}{\pgfqpoint{3.900000in}{1.200000in}}%
\pgfusepath{clip}%
\pgfsetbuttcap%
\pgfsetmiterjoin%
\definecolor{currentfill}{rgb}{0.121569,0.466667,0.705882}%
\pgfsetfillcolor{currentfill}%
\pgfsetlinewidth{0.000000pt}%
\definecolor{currentstroke}{rgb}{0.000000,0.000000,0.000000}%
\pgfsetstrokecolor{currentstroke}%
\pgfsetstrokeopacity{0.000000}%
\pgfsetdash{}{0pt}%
\pgfpathmoveto{\pgfqpoint{1.498026in}{4.530000in}}%
\pgfpathlineto{\pgfqpoint{1.626316in}{4.530000in}}%
\pgfpathlineto{\pgfqpoint{1.626316in}{4.686863in}}%
\pgfpathlineto{\pgfqpoint{1.498026in}{4.686863in}}%
\pgfpathclose%
\pgfusepath{fill}%
\end{pgfscope}%
\begin{pgfscope}%
\pgfpathrectangle{\pgfqpoint{0.600000in}{4.530000in}}{\pgfqpoint{3.900000in}{1.200000in}}%
\pgfusepath{clip}%
\pgfsetbuttcap%
\pgfsetmiterjoin%
\definecolor{currentfill}{rgb}{0.121569,0.466667,0.705882}%
\pgfsetfillcolor{currentfill}%
\pgfsetlinewidth{0.000000pt}%
\definecolor{currentstroke}{rgb}{0.000000,0.000000,0.000000}%
\pgfsetstrokecolor{currentstroke}%
\pgfsetstrokeopacity{0.000000}%
\pgfsetdash{}{0pt}%
\pgfpathmoveto{\pgfqpoint{1.626316in}{4.530000in}}%
\pgfpathlineto{\pgfqpoint{1.754605in}{4.530000in}}%
\pgfpathlineto{\pgfqpoint{1.754605in}{4.720476in}}%
\pgfpathlineto{\pgfqpoint{1.626316in}{4.720476in}}%
\pgfpathclose%
\pgfusepath{fill}%
\end{pgfscope}%
\begin{pgfscope}%
\pgfpathrectangle{\pgfqpoint{0.600000in}{4.530000in}}{\pgfqpoint{3.900000in}{1.200000in}}%
\pgfusepath{clip}%
\pgfsetbuttcap%
\pgfsetmiterjoin%
\definecolor{currentfill}{rgb}{0.121569,0.466667,0.705882}%
\pgfsetfillcolor{currentfill}%
\pgfsetlinewidth{0.000000pt}%
\definecolor{currentstroke}{rgb}{0.000000,0.000000,0.000000}%
\pgfsetstrokecolor{currentstroke}%
\pgfsetstrokeopacity{0.000000}%
\pgfsetdash{}{0pt}%
\pgfpathmoveto{\pgfqpoint{1.754605in}{4.530000in}}%
\pgfpathlineto{\pgfqpoint{1.882895in}{4.530000in}}%
\pgfpathlineto{\pgfqpoint{1.882895in}{4.978179in}}%
\pgfpathlineto{\pgfqpoint{1.754605in}{4.978179in}}%
\pgfpathclose%
\pgfusepath{fill}%
\end{pgfscope}%
\begin{pgfscope}%
\pgfpathrectangle{\pgfqpoint{0.600000in}{4.530000in}}{\pgfqpoint{3.900000in}{1.200000in}}%
\pgfusepath{clip}%
\pgfsetbuttcap%
\pgfsetmiterjoin%
\definecolor{currentfill}{rgb}{0.121569,0.466667,0.705882}%
\pgfsetfillcolor{currentfill}%
\pgfsetlinewidth{0.000000pt}%
\definecolor{currentstroke}{rgb}{0.000000,0.000000,0.000000}%
\pgfsetstrokecolor{currentstroke}%
\pgfsetstrokeopacity{0.000000}%
\pgfsetdash{}{0pt}%
\pgfpathmoveto{\pgfqpoint{1.882895in}{4.530000in}}%
\pgfpathlineto{\pgfqpoint{2.011184in}{4.530000in}}%
\pgfpathlineto{\pgfqpoint{2.011184in}{5.157451in}}%
\pgfpathlineto{\pgfqpoint{1.882895in}{5.157451in}}%
\pgfpathclose%
\pgfusepath{fill}%
\end{pgfscope}%
\begin{pgfscope}%
\pgfpathrectangle{\pgfqpoint{0.600000in}{4.530000in}}{\pgfqpoint{3.900000in}{1.200000in}}%
\pgfusepath{clip}%
\pgfsetbuttcap%
\pgfsetmiterjoin%
\definecolor{currentfill}{rgb}{0.121569,0.466667,0.705882}%
\pgfsetfillcolor{currentfill}%
\pgfsetlinewidth{0.000000pt}%
\definecolor{currentstroke}{rgb}{0.000000,0.000000,0.000000}%
\pgfsetstrokecolor{currentstroke}%
\pgfsetstrokeopacity{0.000000}%
\pgfsetdash{}{0pt}%
\pgfpathmoveto{\pgfqpoint{2.011184in}{4.530000in}}%
\pgfpathlineto{\pgfqpoint{2.139474in}{4.530000in}}%
\pgfpathlineto{\pgfqpoint{2.139474in}{5.235882in}}%
\pgfpathlineto{\pgfqpoint{2.011184in}{5.235882in}}%
\pgfpathclose%
\pgfusepath{fill}%
\end{pgfscope}%
\begin{pgfscope}%
\pgfpathrectangle{\pgfqpoint{0.600000in}{4.530000in}}{\pgfqpoint{3.900000in}{1.200000in}}%
\pgfusepath{clip}%
\pgfsetbuttcap%
\pgfsetmiterjoin%
\definecolor{currentfill}{rgb}{0.121569,0.466667,0.705882}%
\pgfsetfillcolor{currentfill}%
\pgfsetlinewidth{0.000000pt}%
\definecolor{currentstroke}{rgb}{0.000000,0.000000,0.000000}%
\pgfsetstrokecolor{currentstroke}%
\pgfsetstrokeopacity{0.000000}%
\pgfsetdash{}{0pt}%
\pgfpathmoveto{\pgfqpoint{2.139474in}{4.530000in}}%
\pgfpathlineto{\pgfqpoint{2.267763in}{4.530000in}}%
\pgfpathlineto{\pgfqpoint{2.267763in}{5.359132in}}%
\pgfpathlineto{\pgfqpoint{2.139474in}{5.359132in}}%
\pgfpathclose%
\pgfusepath{fill}%
\end{pgfscope}%
\begin{pgfscope}%
\pgfpathrectangle{\pgfqpoint{0.600000in}{4.530000in}}{\pgfqpoint{3.900000in}{1.200000in}}%
\pgfusepath{clip}%
\pgfsetbuttcap%
\pgfsetmiterjoin%
\definecolor{currentfill}{rgb}{0.121569,0.466667,0.705882}%
\pgfsetfillcolor{currentfill}%
\pgfsetlinewidth{0.000000pt}%
\definecolor{currentstroke}{rgb}{0.000000,0.000000,0.000000}%
\pgfsetstrokecolor{currentstroke}%
\pgfsetstrokeopacity{0.000000}%
\pgfsetdash{}{0pt}%
\pgfpathmoveto{\pgfqpoint{2.267763in}{4.530000in}}%
\pgfpathlineto{\pgfqpoint{2.396053in}{4.530000in}}%
\pgfpathlineto{\pgfqpoint{2.396053in}{5.560812in}}%
\pgfpathlineto{\pgfqpoint{2.267763in}{5.560812in}}%
\pgfpathclose%
\pgfusepath{fill}%
\end{pgfscope}%
\begin{pgfscope}%
\pgfpathrectangle{\pgfqpoint{0.600000in}{4.530000in}}{\pgfqpoint{3.900000in}{1.200000in}}%
\pgfusepath{clip}%
\pgfsetbuttcap%
\pgfsetmiterjoin%
\definecolor{currentfill}{rgb}{0.121569,0.466667,0.705882}%
\pgfsetfillcolor{currentfill}%
\pgfsetlinewidth{0.000000pt}%
\definecolor{currentstroke}{rgb}{0.000000,0.000000,0.000000}%
\pgfsetstrokecolor{currentstroke}%
\pgfsetstrokeopacity{0.000000}%
\pgfsetdash{}{0pt}%
\pgfpathmoveto{\pgfqpoint{2.396053in}{4.530000in}}%
\pgfpathlineto{\pgfqpoint{2.524342in}{4.530000in}}%
\pgfpathlineto{\pgfqpoint{2.524342in}{5.527199in}}%
\pgfpathlineto{\pgfqpoint{2.396053in}{5.527199in}}%
\pgfpathclose%
\pgfusepath{fill}%
\end{pgfscope}%
\begin{pgfscope}%
\pgfpathrectangle{\pgfqpoint{0.600000in}{4.530000in}}{\pgfqpoint{3.900000in}{1.200000in}}%
\pgfusepath{clip}%
\pgfsetbuttcap%
\pgfsetmiterjoin%
\definecolor{currentfill}{rgb}{0.121569,0.466667,0.705882}%
\pgfsetfillcolor{currentfill}%
\pgfsetlinewidth{0.000000pt}%
\definecolor{currentstroke}{rgb}{0.000000,0.000000,0.000000}%
\pgfsetstrokecolor{currentstroke}%
\pgfsetstrokeopacity{0.000000}%
\pgfsetdash{}{0pt}%
\pgfpathmoveto{\pgfqpoint{2.524342in}{4.530000in}}%
\pgfpathlineto{\pgfqpoint{2.652632in}{4.530000in}}%
\pgfpathlineto{\pgfqpoint{2.652632in}{5.672857in}}%
\pgfpathlineto{\pgfqpoint{2.524342in}{5.672857in}}%
\pgfpathclose%
\pgfusepath{fill}%
\end{pgfscope}%
\begin{pgfscope}%
\pgfpathrectangle{\pgfqpoint{0.600000in}{4.530000in}}{\pgfqpoint{3.900000in}{1.200000in}}%
\pgfusepath{clip}%
\pgfsetbuttcap%
\pgfsetmiterjoin%
\definecolor{currentfill}{rgb}{0.121569,0.466667,0.705882}%
\pgfsetfillcolor{currentfill}%
\pgfsetlinewidth{0.000000pt}%
\definecolor{currentstroke}{rgb}{0.000000,0.000000,0.000000}%
\pgfsetstrokecolor{currentstroke}%
\pgfsetstrokeopacity{0.000000}%
\pgfsetdash{}{0pt}%
\pgfpathmoveto{\pgfqpoint{2.652632in}{4.530000in}}%
\pgfpathlineto{\pgfqpoint{2.780921in}{4.530000in}}%
\pgfpathlineto{\pgfqpoint{2.780921in}{5.672857in}}%
\pgfpathlineto{\pgfqpoint{2.652632in}{5.672857in}}%
\pgfpathclose%
\pgfusepath{fill}%
\end{pgfscope}%
\begin{pgfscope}%
\pgfpathrectangle{\pgfqpoint{0.600000in}{4.530000in}}{\pgfqpoint{3.900000in}{1.200000in}}%
\pgfusepath{clip}%
\pgfsetbuttcap%
\pgfsetmiterjoin%
\definecolor{currentfill}{rgb}{0.121569,0.466667,0.705882}%
\pgfsetfillcolor{currentfill}%
\pgfsetlinewidth{0.000000pt}%
\definecolor{currentstroke}{rgb}{0.000000,0.000000,0.000000}%
\pgfsetstrokecolor{currentstroke}%
\pgfsetstrokeopacity{0.000000}%
\pgfsetdash{}{0pt}%
\pgfpathmoveto{\pgfqpoint{2.780921in}{4.530000in}}%
\pgfpathlineto{\pgfqpoint{2.909211in}{4.530000in}}%
\pgfpathlineto{\pgfqpoint{2.909211in}{5.493585in}}%
\pgfpathlineto{\pgfqpoint{2.780921in}{5.493585in}}%
\pgfpathclose%
\pgfusepath{fill}%
\end{pgfscope}%
\begin{pgfscope}%
\pgfpathrectangle{\pgfqpoint{0.600000in}{4.530000in}}{\pgfqpoint{3.900000in}{1.200000in}}%
\pgfusepath{clip}%
\pgfsetbuttcap%
\pgfsetmiterjoin%
\definecolor{currentfill}{rgb}{0.121569,0.466667,0.705882}%
\pgfsetfillcolor{currentfill}%
\pgfsetlinewidth{0.000000pt}%
\definecolor{currentstroke}{rgb}{0.000000,0.000000,0.000000}%
\pgfsetstrokecolor{currentstroke}%
\pgfsetstrokeopacity{0.000000}%
\pgfsetdash{}{0pt}%
\pgfpathmoveto{\pgfqpoint{2.909211in}{4.530000in}}%
\pgfpathlineto{\pgfqpoint{3.037500in}{4.530000in}}%
\pgfpathlineto{\pgfqpoint{3.037500in}{5.504790in}}%
\pgfpathlineto{\pgfqpoint{2.909211in}{5.504790in}}%
\pgfpathclose%
\pgfusepath{fill}%
\end{pgfscope}%
\begin{pgfscope}%
\pgfpathrectangle{\pgfqpoint{0.600000in}{4.530000in}}{\pgfqpoint{3.900000in}{1.200000in}}%
\pgfusepath{clip}%
\pgfsetbuttcap%
\pgfsetmiterjoin%
\definecolor{currentfill}{rgb}{0.121569,0.466667,0.705882}%
\pgfsetfillcolor{currentfill}%
\pgfsetlinewidth{0.000000pt}%
\definecolor{currentstroke}{rgb}{0.000000,0.000000,0.000000}%
\pgfsetstrokecolor{currentstroke}%
\pgfsetstrokeopacity{0.000000}%
\pgfsetdash{}{0pt}%
\pgfpathmoveto{\pgfqpoint{3.037500in}{4.530000in}}%
\pgfpathlineto{\pgfqpoint{3.165789in}{4.530000in}}%
\pgfpathlineto{\pgfqpoint{3.165789in}{5.056611in}}%
\pgfpathlineto{\pgfqpoint{3.037500in}{5.056611in}}%
\pgfpathclose%
\pgfusepath{fill}%
\end{pgfscope}%
\begin{pgfscope}%
\pgfpathrectangle{\pgfqpoint{0.600000in}{4.530000in}}{\pgfqpoint{3.900000in}{1.200000in}}%
\pgfusepath{clip}%
\pgfsetbuttcap%
\pgfsetmiterjoin%
\definecolor{currentfill}{rgb}{0.121569,0.466667,0.705882}%
\pgfsetfillcolor{currentfill}%
\pgfsetlinewidth{0.000000pt}%
\definecolor{currentstroke}{rgb}{0.000000,0.000000,0.000000}%
\pgfsetstrokecolor{currentstroke}%
\pgfsetstrokeopacity{0.000000}%
\pgfsetdash{}{0pt}%
\pgfpathmoveto{\pgfqpoint{3.165789in}{4.530000in}}%
\pgfpathlineto{\pgfqpoint{3.294079in}{4.530000in}}%
\pgfpathlineto{\pgfqpoint{3.294079in}{5.011793in}}%
\pgfpathlineto{\pgfqpoint{3.165789in}{5.011793in}}%
\pgfpathclose%
\pgfusepath{fill}%
\end{pgfscope}%
\begin{pgfscope}%
\pgfpathrectangle{\pgfqpoint{0.600000in}{4.530000in}}{\pgfqpoint{3.900000in}{1.200000in}}%
\pgfusepath{clip}%
\pgfsetbuttcap%
\pgfsetmiterjoin%
\definecolor{currentfill}{rgb}{0.121569,0.466667,0.705882}%
\pgfsetfillcolor{currentfill}%
\pgfsetlinewidth{0.000000pt}%
\definecolor{currentstroke}{rgb}{0.000000,0.000000,0.000000}%
\pgfsetstrokecolor{currentstroke}%
\pgfsetstrokeopacity{0.000000}%
\pgfsetdash{}{0pt}%
\pgfpathmoveto{\pgfqpoint{3.294079in}{4.530000in}}%
\pgfpathlineto{\pgfqpoint{3.422368in}{4.530000in}}%
\pgfpathlineto{\pgfqpoint{3.422368in}{4.810112in}}%
\pgfpathlineto{\pgfqpoint{3.294079in}{4.810112in}}%
\pgfpathclose%
\pgfusepath{fill}%
\end{pgfscope}%
\begin{pgfscope}%
\pgfpathrectangle{\pgfqpoint{0.600000in}{4.530000in}}{\pgfqpoint{3.900000in}{1.200000in}}%
\pgfusepath{clip}%
\pgfsetbuttcap%
\pgfsetmiterjoin%
\definecolor{currentfill}{rgb}{0.121569,0.466667,0.705882}%
\pgfsetfillcolor{currentfill}%
\pgfsetlinewidth{0.000000pt}%
\definecolor{currentstroke}{rgb}{0.000000,0.000000,0.000000}%
\pgfsetstrokecolor{currentstroke}%
\pgfsetstrokeopacity{0.000000}%
\pgfsetdash{}{0pt}%
\pgfpathmoveto{\pgfqpoint{3.422368in}{4.530000in}}%
\pgfpathlineto{\pgfqpoint{3.550658in}{4.530000in}}%
\pgfpathlineto{\pgfqpoint{3.550658in}{4.832521in}}%
\pgfpathlineto{\pgfqpoint{3.422368in}{4.832521in}}%
\pgfpathclose%
\pgfusepath{fill}%
\end{pgfscope}%
\begin{pgfscope}%
\pgfpathrectangle{\pgfqpoint{0.600000in}{4.530000in}}{\pgfqpoint{3.900000in}{1.200000in}}%
\pgfusepath{clip}%
\pgfsetbuttcap%
\pgfsetmiterjoin%
\definecolor{currentfill}{rgb}{0.121569,0.466667,0.705882}%
\pgfsetfillcolor{currentfill}%
\pgfsetlinewidth{0.000000pt}%
\definecolor{currentstroke}{rgb}{0.000000,0.000000,0.000000}%
\pgfsetstrokecolor{currentstroke}%
\pgfsetstrokeopacity{0.000000}%
\pgfsetdash{}{0pt}%
\pgfpathmoveto{\pgfqpoint{3.550658in}{4.530000in}}%
\pgfpathlineto{\pgfqpoint{3.678947in}{4.530000in}}%
\pgfpathlineto{\pgfqpoint{3.678947in}{4.653249in}}%
\pgfpathlineto{\pgfqpoint{3.550658in}{4.653249in}}%
\pgfpathclose%
\pgfusepath{fill}%
\end{pgfscope}%
\begin{pgfscope}%
\pgfpathrectangle{\pgfqpoint{0.600000in}{4.530000in}}{\pgfqpoint{3.900000in}{1.200000in}}%
\pgfusepath{clip}%
\pgfsetbuttcap%
\pgfsetmiterjoin%
\definecolor{currentfill}{rgb}{0.121569,0.466667,0.705882}%
\pgfsetfillcolor{currentfill}%
\pgfsetlinewidth{0.000000pt}%
\definecolor{currentstroke}{rgb}{0.000000,0.000000,0.000000}%
\pgfsetstrokecolor{currentstroke}%
\pgfsetstrokeopacity{0.000000}%
\pgfsetdash{}{0pt}%
\pgfpathmoveto{\pgfqpoint{3.678947in}{4.530000in}}%
\pgfpathlineto{\pgfqpoint{3.807237in}{4.530000in}}%
\pgfpathlineto{\pgfqpoint{3.807237in}{4.597227in}}%
\pgfpathlineto{\pgfqpoint{3.678947in}{4.597227in}}%
\pgfpathclose%
\pgfusepath{fill}%
\end{pgfscope}%
\begin{pgfscope}%
\pgfpathrectangle{\pgfqpoint{0.600000in}{4.530000in}}{\pgfqpoint{3.900000in}{1.200000in}}%
\pgfusepath{clip}%
\pgfsetbuttcap%
\pgfsetmiterjoin%
\definecolor{currentfill}{rgb}{0.121569,0.466667,0.705882}%
\pgfsetfillcolor{currentfill}%
\pgfsetlinewidth{0.000000pt}%
\definecolor{currentstroke}{rgb}{0.000000,0.000000,0.000000}%
\pgfsetstrokecolor{currentstroke}%
\pgfsetstrokeopacity{0.000000}%
\pgfsetdash{}{0pt}%
\pgfpathmoveto{\pgfqpoint{3.807237in}{4.530000in}}%
\pgfpathlineto{\pgfqpoint{3.935526in}{4.530000in}}%
\pgfpathlineto{\pgfqpoint{3.935526in}{4.563613in}}%
\pgfpathlineto{\pgfqpoint{3.807237in}{4.563613in}}%
\pgfpathclose%
\pgfusepath{fill}%
\end{pgfscope}%
\begin{pgfscope}%
\pgfpathrectangle{\pgfqpoint{0.600000in}{4.530000in}}{\pgfqpoint{3.900000in}{1.200000in}}%
\pgfusepath{clip}%
\pgfsetbuttcap%
\pgfsetmiterjoin%
\definecolor{currentfill}{rgb}{0.121569,0.466667,0.705882}%
\pgfsetfillcolor{currentfill}%
\pgfsetlinewidth{0.000000pt}%
\definecolor{currentstroke}{rgb}{0.000000,0.000000,0.000000}%
\pgfsetstrokecolor{currentstroke}%
\pgfsetstrokeopacity{0.000000}%
\pgfsetdash{}{0pt}%
\pgfpathmoveto{\pgfqpoint{3.935526in}{4.530000in}}%
\pgfpathlineto{\pgfqpoint{4.063816in}{4.530000in}}%
\pgfpathlineto{\pgfqpoint{4.063816in}{4.586022in}}%
\pgfpathlineto{\pgfqpoint{3.935526in}{4.586022in}}%
\pgfpathclose%
\pgfusepath{fill}%
\end{pgfscope}%
\begin{pgfscope}%
\pgfpathrectangle{\pgfqpoint{0.600000in}{4.530000in}}{\pgfqpoint{3.900000in}{1.200000in}}%
\pgfusepath{clip}%
\pgfsetbuttcap%
\pgfsetmiterjoin%
\definecolor{currentfill}{rgb}{0.121569,0.466667,0.705882}%
\pgfsetfillcolor{currentfill}%
\pgfsetlinewidth{0.000000pt}%
\definecolor{currentstroke}{rgb}{0.000000,0.000000,0.000000}%
\pgfsetstrokecolor{currentstroke}%
\pgfsetstrokeopacity{0.000000}%
\pgfsetdash{}{0pt}%
\pgfpathmoveto{\pgfqpoint{4.063816in}{4.530000in}}%
\pgfpathlineto{\pgfqpoint{4.192105in}{4.530000in}}%
\pgfpathlineto{\pgfqpoint{4.192105in}{4.530000in}}%
\pgfpathlineto{\pgfqpoint{4.063816in}{4.530000in}}%
\pgfpathclose%
\pgfusepath{fill}%
\end{pgfscope}%
\begin{pgfscope}%
\pgfpathrectangle{\pgfqpoint{0.600000in}{4.530000in}}{\pgfqpoint{3.900000in}{1.200000in}}%
\pgfusepath{clip}%
\pgfsetbuttcap%
\pgfsetmiterjoin%
\definecolor{currentfill}{rgb}{0.121569,0.466667,0.705882}%
\pgfsetfillcolor{currentfill}%
\pgfsetlinewidth{0.000000pt}%
\definecolor{currentstroke}{rgb}{0.000000,0.000000,0.000000}%
\pgfsetstrokecolor{currentstroke}%
\pgfsetstrokeopacity{0.000000}%
\pgfsetdash{}{0pt}%
\pgfpathmoveto{\pgfqpoint{4.192105in}{4.530000in}}%
\pgfpathlineto{\pgfqpoint{4.320395in}{4.530000in}}%
\pgfpathlineto{\pgfqpoint{4.320395in}{4.530000in}}%
\pgfpathlineto{\pgfqpoint{4.192105in}{4.530000in}}%
\pgfpathclose%
\pgfusepath{fill}%
\end{pgfscope}%
\begin{pgfscope}%
\pgfpathrectangle{\pgfqpoint{0.600000in}{4.530000in}}{\pgfqpoint{3.900000in}{1.200000in}}%
\pgfusepath{clip}%
\pgfsetbuttcap%
\pgfsetmiterjoin%
\definecolor{currentfill}{rgb}{0.121569,0.466667,0.705882}%
\pgfsetfillcolor{currentfill}%
\pgfsetlinewidth{0.000000pt}%
\definecolor{currentstroke}{rgb}{0.000000,0.000000,0.000000}%
\pgfsetstrokecolor{currentstroke}%
\pgfsetstrokeopacity{0.000000}%
\pgfsetdash{}{0pt}%
\pgfpathmoveto{\pgfqpoint{4.320395in}{4.530000in}}%
\pgfpathlineto{\pgfqpoint{4.448684in}{4.530000in}}%
\pgfpathlineto{\pgfqpoint{4.448684in}{4.530000in}}%
\pgfpathlineto{\pgfqpoint{4.320395in}{4.530000in}}%
\pgfpathclose%
\pgfusepath{fill}%
\end{pgfscope}%
\begin{pgfscope}%
\pgfpathrectangle{\pgfqpoint{0.600000in}{4.530000in}}{\pgfqpoint{3.900000in}{1.200000in}}%
\pgfusepath{clip}%
\pgfsetbuttcap%
\pgfsetmiterjoin%
\definecolor{currentfill}{rgb}{0.121569,0.466667,0.705882}%
\pgfsetfillcolor{currentfill}%
\pgfsetlinewidth{0.000000pt}%
\definecolor{currentstroke}{rgb}{0.000000,0.000000,0.000000}%
\pgfsetstrokecolor{currentstroke}%
\pgfsetstrokeopacity{0.000000}%
\pgfsetdash{}{0pt}%
\pgfpathmoveto{\pgfqpoint{4.448684in}{4.530000in}}%
\pgfpathlineto{\pgfqpoint{4.576974in}{4.530000in}}%
\pgfpathlineto{\pgfqpoint{4.576974in}{4.530000in}}%
\pgfpathlineto{\pgfqpoint{4.448684in}{4.530000in}}%
\pgfpathclose%
\pgfusepath{fill}%
\end{pgfscope}%
\begin{pgfscope}%
\pgfsetbuttcap%
\pgfsetroundjoin%
\definecolor{currentfill}{rgb}{0.000000,0.000000,0.000000}%
\pgfsetfillcolor{currentfill}%
\pgfsetlinewidth{0.803000pt}%
\definecolor{currentstroke}{rgb}{0.000000,0.000000,0.000000}%
\pgfsetstrokecolor{currentstroke}%
\pgfsetdash{}{0pt}%
\pgfsys@defobject{currentmarker}{\pgfqpoint{0.000000in}{0.000000in}}{\pgfqpoint{0.000000in}{0.048611in}}{%
\pgfpathmoveto{\pgfqpoint{0.000000in}{0.000000in}}%
\pgfpathlineto{\pgfqpoint{0.000000in}{0.048611in}}%
\pgfusepath{stroke,fill}%
}%
\begin{pgfscope}%
\pgfsys@transformshift{1.010526in}{4.530000in}%
\pgfsys@useobject{currentmarker}{}%
\end{pgfscope}%
\end{pgfscope}%
\begin{pgfscope}%
\pgfsetbuttcap%
\pgfsetroundjoin%
\definecolor{currentfill}{rgb}{0.000000,0.000000,0.000000}%
\pgfsetfillcolor{currentfill}%
\pgfsetlinewidth{0.803000pt}%
\definecolor{currentstroke}{rgb}{0.000000,0.000000,0.000000}%
\pgfsetstrokecolor{currentstroke}%
\pgfsetdash{}{0pt}%
\pgfsys@defobject{currentmarker}{\pgfqpoint{0.000000in}{0.000000in}}{\pgfqpoint{0.000000in}{0.048611in}}{%
\pgfpathmoveto{\pgfqpoint{0.000000in}{0.000000in}}%
\pgfpathlineto{\pgfqpoint{0.000000in}{0.048611in}}%
\pgfusepath{stroke,fill}%
}%
\begin{pgfscope}%
\pgfsys@transformshift{1.523684in}{4.530000in}%
\pgfsys@useobject{currentmarker}{}%
\end{pgfscope}%
\end{pgfscope}%
\begin{pgfscope}%
\pgfsetbuttcap%
\pgfsetroundjoin%
\definecolor{currentfill}{rgb}{0.000000,0.000000,0.000000}%
\pgfsetfillcolor{currentfill}%
\pgfsetlinewidth{0.803000pt}%
\definecolor{currentstroke}{rgb}{0.000000,0.000000,0.000000}%
\pgfsetstrokecolor{currentstroke}%
\pgfsetdash{}{0pt}%
\pgfsys@defobject{currentmarker}{\pgfqpoint{0.000000in}{0.000000in}}{\pgfqpoint{0.000000in}{0.048611in}}{%
\pgfpathmoveto{\pgfqpoint{0.000000in}{0.000000in}}%
\pgfpathlineto{\pgfqpoint{0.000000in}{0.048611in}}%
\pgfusepath{stroke,fill}%
}%
\begin{pgfscope}%
\pgfsys@transformshift{2.036842in}{4.530000in}%
\pgfsys@useobject{currentmarker}{}%
\end{pgfscope}%
\end{pgfscope}%
\begin{pgfscope}%
\pgfsetbuttcap%
\pgfsetroundjoin%
\definecolor{currentfill}{rgb}{0.000000,0.000000,0.000000}%
\pgfsetfillcolor{currentfill}%
\pgfsetlinewidth{0.803000pt}%
\definecolor{currentstroke}{rgb}{0.000000,0.000000,0.000000}%
\pgfsetstrokecolor{currentstroke}%
\pgfsetdash{}{0pt}%
\pgfsys@defobject{currentmarker}{\pgfqpoint{0.000000in}{0.000000in}}{\pgfqpoint{0.000000in}{0.048611in}}{%
\pgfpathmoveto{\pgfqpoint{0.000000in}{0.000000in}}%
\pgfpathlineto{\pgfqpoint{0.000000in}{0.048611in}}%
\pgfusepath{stroke,fill}%
}%
\begin{pgfscope}%
\pgfsys@transformshift{2.550000in}{4.530000in}%
\pgfsys@useobject{currentmarker}{}%
\end{pgfscope}%
\end{pgfscope}%
\begin{pgfscope}%
\pgfsetbuttcap%
\pgfsetroundjoin%
\definecolor{currentfill}{rgb}{0.000000,0.000000,0.000000}%
\pgfsetfillcolor{currentfill}%
\pgfsetlinewidth{0.803000pt}%
\definecolor{currentstroke}{rgb}{0.000000,0.000000,0.000000}%
\pgfsetstrokecolor{currentstroke}%
\pgfsetdash{}{0pt}%
\pgfsys@defobject{currentmarker}{\pgfqpoint{0.000000in}{0.000000in}}{\pgfqpoint{0.000000in}{0.048611in}}{%
\pgfpathmoveto{\pgfqpoint{0.000000in}{0.000000in}}%
\pgfpathlineto{\pgfqpoint{0.000000in}{0.048611in}}%
\pgfusepath{stroke,fill}%
}%
\begin{pgfscope}%
\pgfsys@transformshift{3.063158in}{4.530000in}%
\pgfsys@useobject{currentmarker}{}%
\end{pgfscope}%
\end{pgfscope}%
\begin{pgfscope}%
\pgfsetbuttcap%
\pgfsetroundjoin%
\definecolor{currentfill}{rgb}{0.000000,0.000000,0.000000}%
\pgfsetfillcolor{currentfill}%
\pgfsetlinewidth{0.803000pt}%
\definecolor{currentstroke}{rgb}{0.000000,0.000000,0.000000}%
\pgfsetstrokecolor{currentstroke}%
\pgfsetdash{}{0pt}%
\pgfsys@defobject{currentmarker}{\pgfqpoint{0.000000in}{0.000000in}}{\pgfqpoint{0.000000in}{0.048611in}}{%
\pgfpathmoveto{\pgfqpoint{0.000000in}{0.000000in}}%
\pgfpathlineto{\pgfqpoint{0.000000in}{0.048611in}}%
\pgfusepath{stroke,fill}%
}%
\begin{pgfscope}%
\pgfsys@transformshift{3.576316in}{4.530000in}%
\pgfsys@useobject{currentmarker}{}%
\end{pgfscope}%
\end{pgfscope}%
\begin{pgfscope}%
\pgfsetbuttcap%
\pgfsetroundjoin%
\definecolor{currentfill}{rgb}{0.000000,0.000000,0.000000}%
\pgfsetfillcolor{currentfill}%
\pgfsetlinewidth{0.803000pt}%
\definecolor{currentstroke}{rgb}{0.000000,0.000000,0.000000}%
\pgfsetstrokecolor{currentstroke}%
\pgfsetdash{}{0pt}%
\pgfsys@defobject{currentmarker}{\pgfqpoint{0.000000in}{0.000000in}}{\pgfqpoint{0.000000in}{0.048611in}}{%
\pgfpathmoveto{\pgfqpoint{0.000000in}{0.000000in}}%
\pgfpathlineto{\pgfqpoint{0.000000in}{0.048611in}}%
\pgfusepath{stroke,fill}%
}%
\begin{pgfscope}%
\pgfsys@transformshift{4.089474in}{4.530000in}%
\pgfsys@useobject{currentmarker}{}%
\end{pgfscope}%
\end{pgfscope}%
\begin{pgfscope}%
\pgfsetbuttcap%
\pgfsetroundjoin%
\definecolor{currentfill}{rgb}{0.000000,0.000000,0.000000}%
\pgfsetfillcolor{currentfill}%
\pgfsetlinewidth{0.803000pt}%
\definecolor{currentstroke}{rgb}{0.000000,0.000000,0.000000}%
\pgfsetstrokecolor{currentstroke}%
\pgfsetdash{}{0pt}%
\pgfsys@defobject{currentmarker}{\pgfqpoint{0.000000in}{0.000000in}}{\pgfqpoint{0.048611in}{0.000000in}}{%
\pgfpathmoveto{\pgfqpoint{0.000000in}{0.000000in}}%
\pgfpathlineto{\pgfqpoint{0.048611in}{0.000000in}}%
\pgfusepath{stroke,fill}%
}%
\begin{pgfscope}%
\pgfsys@transformshift{0.600000in}{4.530000in}%
\pgfsys@useobject{currentmarker}{}%
\end{pgfscope}%
\end{pgfscope}%
\begin{pgfscope}%
\definecolor{textcolor}{rgb}{0.000000,0.000000,0.000000}%
\pgfsetstrokecolor{textcolor}%
\pgfsetfillcolor{textcolor}%
\pgftext[x=0.463024in,y=4.477238in,left,base]{\color{textcolor}\sffamily\fontsize{10.000000}{12.000000}\selectfont 0}%
\end{pgfscope}%
\begin{pgfscope}%
\pgfsetbuttcap%
\pgfsetroundjoin%
\definecolor{currentfill}{rgb}{0.000000,0.000000,0.000000}%
\pgfsetfillcolor{currentfill}%
\pgfsetlinewidth{0.803000pt}%
\definecolor{currentstroke}{rgb}{0.000000,0.000000,0.000000}%
\pgfsetstrokecolor{currentstroke}%
\pgfsetdash{}{0pt}%
\pgfsys@defobject{currentmarker}{\pgfqpoint{0.000000in}{0.000000in}}{\pgfqpoint{0.048611in}{0.000000in}}{%
\pgfpathmoveto{\pgfqpoint{0.000000in}{0.000000in}}%
\pgfpathlineto{\pgfqpoint{0.048611in}{0.000000in}}%
\pgfusepath{stroke,fill}%
}%
\begin{pgfscope}%
\pgfsys@transformshift{0.600000in}{5.090224in}%
\pgfsys@useobject{currentmarker}{}%
\end{pgfscope}%
\end{pgfscope}%
\begin{pgfscope}%
\definecolor{textcolor}{rgb}{0.000000,0.000000,0.000000}%
\pgfsetstrokecolor{textcolor}%
\pgfsetfillcolor{textcolor}%
\pgftext[x=0.374658in,y=5.037463in,left,base]{\color{textcolor}\sffamily\fontsize{10.000000}{12.000000}\selectfont 50}%
\end{pgfscope}%
\begin{pgfscope}%
\pgfsetbuttcap%
\pgfsetroundjoin%
\definecolor{currentfill}{rgb}{0.000000,0.000000,0.000000}%
\pgfsetfillcolor{currentfill}%
\pgfsetlinewidth{0.803000pt}%
\definecolor{currentstroke}{rgb}{0.000000,0.000000,0.000000}%
\pgfsetstrokecolor{currentstroke}%
\pgfsetdash{}{0pt}%
\pgfsys@defobject{currentmarker}{\pgfqpoint{0.000000in}{0.000000in}}{\pgfqpoint{0.048611in}{0.000000in}}{%
\pgfpathmoveto{\pgfqpoint{0.000000in}{0.000000in}}%
\pgfpathlineto{\pgfqpoint{0.048611in}{0.000000in}}%
\pgfusepath{stroke,fill}%
}%
\begin{pgfscope}%
\pgfsys@transformshift{0.600000in}{5.650448in}%
\pgfsys@useobject{currentmarker}{}%
\end{pgfscope}%
\end{pgfscope}%
\begin{pgfscope}%
\definecolor{textcolor}{rgb}{0.000000,0.000000,0.000000}%
\pgfsetstrokecolor{textcolor}%
\pgfsetfillcolor{textcolor}%
\pgftext[x=0.286293in,y=5.597687in,left,base]{\color{textcolor}\sffamily\fontsize{10.000000}{12.000000}\selectfont 100}%
\end{pgfscope}%
\begin{pgfscope}%
\pgfsetrectcap%
\pgfsetmiterjoin%
\pgfsetlinewidth{0.803000pt}%
\definecolor{currentstroke}{rgb}{0.000000,0.000000,0.000000}%
\pgfsetstrokecolor{currentstroke}%
\pgfsetdash{}{0pt}%
\pgfpathmoveto{\pgfqpoint{0.600000in}{4.530000in}}%
\pgfpathlineto{\pgfqpoint{0.600000in}{5.730000in}}%
\pgfusepath{stroke}%
\end{pgfscope}%
\begin{pgfscope}%
\pgfsetrectcap%
\pgfsetmiterjoin%
\pgfsetlinewidth{0.803000pt}%
\definecolor{currentstroke}{rgb}{0.000000,0.000000,0.000000}%
\pgfsetstrokecolor{currentstroke}%
\pgfsetdash{}{0pt}%
\pgfpathmoveto{\pgfqpoint{4.500000in}{4.530000in}}%
\pgfpathlineto{\pgfqpoint{4.500000in}{5.730000in}}%
\pgfusepath{stroke}%
\end{pgfscope}%
\begin{pgfscope}%
\pgfsetrectcap%
\pgfsetmiterjoin%
\pgfsetlinewidth{0.803000pt}%
\definecolor{currentstroke}{rgb}{0.000000,0.000000,0.000000}%
\pgfsetstrokecolor{currentstroke}%
\pgfsetdash{}{0pt}%
\pgfpathmoveto{\pgfqpoint{0.600000in}{4.530000in}}%
\pgfpathlineto{\pgfqpoint{4.500000in}{4.530000in}}%
\pgfusepath{stroke}%
\end{pgfscope}%
\begin{pgfscope}%
\pgfsetrectcap%
\pgfsetmiterjoin%
\pgfsetlinewidth{0.803000pt}%
\definecolor{currentstroke}{rgb}{0.000000,0.000000,0.000000}%
\pgfsetstrokecolor{currentstroke}%
\pgfsetdash{}{0pt}%
\pgfpathmoveto{\pgfqpoint{0.600000in}{5.730000in}}%
\pgfpathlineto{\pgfqpoint{4.500000in}{5.730000in}}%
\pgfusepath{stroke}%
\end{pgfscope}%
\begin{pgfscope}%
\pgfsetbuttcap%
\pgfsetmiterjoin%
\definecolor{currentfill}{rgb}{1.000000,1.000000,1.000000}%
\pgfsetfillcolor{currentfill}%
\pgfsetlinewidth{0.000000pt}%
\definecolor{currentstroke}{rgb}{0.000000,0.000000,0.000000}%
\pgfsetstrokecolor{currentstroke}%
\pgfsetstrokeopacity{0.000000}%
\pgfsetdash{}{0pt}%
\pgfpathmoveto{\pgfqpoint{4.530000in}{0.600000in}}%
\pgfpathlineto{\pgfqpoint{5.730000in}{0.600000in}}%
\pgfpathlineto{\pgfqpoint{5.730000in}{4.500000in}}%
\pgfpathlineto{\pgfqpoint{4.530000in}{4.500000in}}%
\pgfpathclose%
\pgfusepath{fill}%
\end{pgfscope}%
\begin{pgfscope}%
\pgfpathrectangle{\pgfqpoint{4.530000in}{0.600000in}}{\pgfqpoint{1.200000in}{3.900000in}}%
\pgfusepath{clip}%
\pgfsetbuttcap%
\pgfsetmiterjoin%
\definecolor{currentfill}{rgb}{0.121569,0.466667,0.705882}%
\pgfsetfillcolor{currentfill}%
\pgfsetlinewidth{0.000000pt}%
\definecolor{currentstroke}{rgb}{0.000000,0.000000,0.000000}%
\pgfsetstrokecolor{currentstroke}%
\pgfsetstrokeopacity{0.000000}%
\pgfsetdash{}{0pt}%
\pgfpathmoveto{\pgfqpoint{4.530000in}{0.600000in}}%
\pgfpathlineto{\pgfqpoint{4.540681in}{0.600000in}}%
\pgfpathlineto{\pgfqpoint{4.540681in}{0.728289in}}%
\pgfpathlineto{\pgfqpoint{4.530000in}{0.728289in}}%
\pgfpathclose%
\pgfusepath{fill}%
\end{pgfscope}%
\begin{pgfscope}%
\pgfpathrectangle{\pgfqpoint{4.530000in}{0.600000in}}{\pgfqpoint{1.200000in}{3.900000in}}%
\pgfusepath{clip}%
\pgfsetbuttcap%
\pgfsetmiterjoin%
\definecolor{currentfill}{rgb}{0.121569,0.466667,0.705882}%
\pgfsetfillcolor{currentfill}%
\pgfsetlinewidth{0.000000pt}%
\definecolor{currentstroke}{rgb}{0.000000,0.000000,0.000000}%
\pgfsetstrokecolor{currentstroke}%
\pgfsetstrokeopacity{0.000000}%
\pgfsetdash{}{0pt}%
\pgfpathmoveto{\pgfqpoint{4.530000in}{0.728289in}}%
\pgfpathlineto{\pgfqpoint{4.540681in}{0.728289in}}%
\pgfpathlineto{\pgfqpoint{4.540681in}{0.856579in}}%
\pgfpathlineto{\pgfqpoint{4.530000in}{0.856579in}}%
\pgfpathclose%
\pgfusepath{fill}%
\end{pgfscope}%
\begin{pgfscope}%
\pgfpathrectangle{\pgfqpoint{4.530000in}{0.600000in}}{\pgfqpoint{1.200000in}{3.900000in}}%
\pgfusepath{clip}%
\pgfsetbuttcap%
\pgfsetmiterjoin%
\definecolor{currentfill}{rgb}{0.121569,0.466667,0.705882}%
\pgfsetfillcolor{currentfill}%
\pgfsetlinewidth{0.000000pt}%
\definecolor{currentstroke}{rgb}{0.000000,0.000000,0.000000}%
\pgfsetstrokecolor{currentstroke}%
\pgfsetstrokeopacity{0.000000}%
\pgfsetdash{}{0pt}%
\pgfpathmoveto{\pgfqpoint{4.530000in}{0.856579in}}%
\pgfpathlineto{\pgfqpoint{4.530000in}{0.856579in}}%
\pgfpathlineto{\pgfqpoint{4.530000in}{0.984868in}}%
\pgfpathlineto{\pgfqpoint{4.530000in}{0.984868in}}%
\pgfpathclose%
\pgfusepath{fill}%
\end{pgfscope}%
\begin{pgfscope}%
\pgfpathrectangle{\pgfqpoint{4.530000in}{0.600000in}}{\pgfqpoint{1.200000in}{3.900000in}}%
\pgfusepath{clip}%
\pgfsetbuttcap%
\pgfsetmiterjoin%
\definecolor{currentfill}{rgb}{0.121569,0.466667,0.705882}%
\pgfsetfillcolor{currentfill}%
\pgfsetlinewidth{0.000000pt}%
\definecolor{currentstroke}{rgb}{0.000000,0.000000,0.000000}%
\pgfsetstrokecolor{currentstroke}%
\pgfsetstrokeopacity{0.000000}%
\pgfsetdash{}{0pt}%
\pgfpathmoveto{\pgfqpoint{4.530000in}{0.984868in}}%
\pgfpathlineto{\pgfqpoint{4.540681in}{0.984868in}}%
\pgfpathlineto{\pgfqpoint{4.540681in}{1.113158in}}%
\pgfpathlineto{\pgfqpoint{4.530000in}{1.113158in}}%
\pgfpathclose%
\pgfusepath{fill}%
\end{pgfscope}%
\begin{pgfscope}%
\pgfpathrectangle{\pgfqpoint{4.530000in}{0.600000in}}{\pgfqpoint{1.200000in}{3.900000in}}%
\pgfusepath{clip}%
\pgfsetbuttcap%
\pgfsetmiterjoin%
\definecolor{currentfill}{rgb}{0.121569,0.466667,0.705882}%
\pgfsetfillcolor{currentfill}%
\pgfsetlinewidth{0.000000pt}%
\definecolor{currentstroke}{rgb}{0.000000,0.000000,0.000000}%
\pgfsetstrokecolor{currentstroke}%
\pgfsetstrokeopacity{0.000000}%
\pgfsetdash{}{0pt}%
\pgfpathmoveto{\pgfqpoint{4.530000in}{1.113158in}}%
\pgfpathlineto{\pgfqpoint{4.551362in}{1.113158in}}%
\pgfpathlineto{\pgfqpoint{4.551362in}{1.241447in}}%
\pgfpathlineto{\pgfqpoint{4.530000in}{1.241447in}}%
\pgfpathclose%
\pgfusepath{fill}%
\end{pgfscope}%
\begin{pgfscope}%
\pgfpathrectangle{\pgfqpoint{4.530000in}{0.600000in}}{\pgfqpoint{1.200000in}{3.900000in}}%
\pgfusepath{clip}%
\pgfsetbuttcap%
\pgfsetmiterjoin%
\definecolor{currentfill}{rgb}{0.121569,0.466667,0.705882}%
\pgfsetfillcolor{currentfill}%
\pgfsetlinewidth{0.000000pt}%
\definecolor{currentstroke}{rgb}{0.000000,0.000000,0.000000}%
\pgfsetstrokecolor{currentstroke}%
\pgfsetstrokeopacity{0.000000}%
\pgfsetdash{}{0pt}%
\pgfpathmoveto{\pgfqpoint{4.530000in}{1.241447in}}%
\pgfpathlineto{\pgfqpoint{4.583405in}{1.241447in}}%
\pgfpathlineto{\pgfqpoint{4.583405in}{1.369737in}}%
\pgfpathlineto{\pgfqpoint{4.530000in}{1.369737in}}%
\pgfpathclose%
\pgfusepath{fill}%
\end{pgfscope}%
\begin{pgfscope}%
\pgfpathrectangle{\pgfqpoint{4.530000in}{0.600000in}}{\pgfqpoint{1.200000in}{3.900000in}}%
\pgfusepath{clip}%
\pgfsetbuttcap%
\pgfsetmiterjoin%
\definecolor{currentfill}{rgb}{0.121569,0.466667,0.705882}%
\pgfsetfillcolor{currentfill}%
\pgfsetlinewidth{0.000000pt}%
\definecolor{currentstroke}{rgb}{0.000000,0.000000,0.000000}%
\pgfsetstrokecolor{currentstroke}%
\pgfsetstrokeopacity{0.000000}%
\pgfsetdash{}{0pt}%
\pgfpathmoveto{\pgfqpoint{4.530000in}{1.369737in}}%
\pgfpathlineto{\pgfqpoint{4.626128in}{1.369737in}}%
\pgfpathlineto{\pgfqpoint{4.626128in}{1.498026in}}%
\pgfpathlineto{\pgfqpoint{4.530000in}{1.498026in}}%
\pgfpathclose%
\pgfusepath{fill}%
\end{pgfscope}%
\begin{pgfscope}%
\pgfpathrectangle{\pgfqpoint{4.530000in}{0.600000in}}{\pgfqpoint{1.200000in}{3.900000in}}%
\pgfusepath{clip}%
\pgfsetbuttcap%
\pgfsetmiterjoin%
\definecolor{currentfill}{rgb}{0.121569,0.466667,0.705882}%
\pgfsetfillcolor{currentfill}%
\pgfsetlinewidth{0.000000pt}%
\definecolor{currentstroke}{rgb}{0.000000,0.000000,0.000000}%
\pgfsetstrokecolor{currentstroke}%
\pgfsetstrokeopacity{0.000000}%
\pgfsetdash{}{0pt}%
\pgfpathmoveto{\pgfqpoint{4.530000in}{1.498026in}}%
\pgfpathlineto{\pgfqpoint{4.700895in}{1.498026in}}%
\pgfpathlineto{\pgfqpoint{4.700895in}{1.626316in}}%
\pgfpathlineto{\pgfqpoint{4.530000in}{1.626316in}}%
\pgfpathclose%
\pgfusepath{fill}%
\end{pgfscope}%
\begin{pgfscope}%
\pgfpathrectangle{\pgfqpoint{4.530000in}{0.600000in}}{\pgfqpoint{1.200000in}{3.900000in}}%
\pgfusepath{clip}%
\pgfsetbuttcap%
\pgfsetmiterjoin%
\definecolor{currentfill}{rgb}{0.121569,0.466667,0.705882}%
\pgfsetfillcolor{currentfill}%
\pgfsetlinewidth{0.000000pt}%
\definecolor{currentstroke}{rgb}{0.000000,0.000000,0.000000}%
\pgfsetstrokecolor{currentstroke}%
\pgfsetstrokeopacity{0.000000}%
\pgfsetdash{}{0pt}%
\pgfpathmoveto{\pgfqpoint{4.530000in}{1.626316in}}%
\pgfpathlineto{\pgfqpoint{4.732937in}{1.626316in}}%
\pgfpathlineto{\pgfqpoint{4.732937in}{1.754605in}}%
\pgfpathlineto{\pgfqpoint{4.530000in}{1.754605in}}%
\pgfpathclose%
\pgfusepath{fill}%
\end{pgfscope}%
\begin{pgfscope}%
\pgfpathrectangle{\pgfqpoint{4.530000in}{0.600000in}}{\pgfqpoint{1.200000in}{3.900000in}}%
\pgfusepath{clip}%
\pgfsetbuttcap%
\pgfsetmiterjoin%
\definecolor{currentfill}{rgb}{0.121569,0.466667,0.705882}%
\pgfsetfillcolor{currentfill}%
\pgfsetlinewidth{0.000000pt}%
\definecolor{currentstroke}{rgb}{0.000000,0.000000,0.000000}%
\pgfsetstrokecolor{currentstroke}%
\pgfsetstrokeopacity{0.000000}%
\pgfsetdash{}{0pt}%
\pgfpathmoveto{\pgfqpoint{4.530000in}{1.754605in}}%
\pgfpathlineto{\pgfqpoint{4.946555in}{1.754605in}}%
\pgfpathlineto{\pgfqpoint{4.946555in}{1.882895in}}%
\pgfpathlineto{\pgfqpoint{4.530000in}{1.882895in}}%
\pgfpathclose%
\pgfusepath{fill}%
\end{pgfscope}%
\begin{pgfscope}%
\pgfpathrectangle{\pgfqpoint{4.530000in}{0.600000in}}{\pgfqpoint{1.200000in}{3.900000in}}%
\pgfusepath{clip}%
\pgfsetbuttcap%
\pgfsetmiterjoin%
\definecolor{currentfill}{rgb}{0.121569,0.466667,0.705882}%
\pgfsetfillcolor{currentfill}%
\pgfsetlinewidth{0.000000pt}%
\definecolor{currentstroke}{rgb}{0.000000,0.000000,0.000000}%
\pgfsetstrokecolor{currentstroke}%
\pgfsetstrokeopacity{0.000000}%
\pgfsetdash{}{0pt}%
\pgfpathmoveto{\pgfqpoint{4.530000in}{1.882895in}}%
\pgfpathlineto{\pgfqpoint{5.213578in}{1.882895in}}%
\pgfpathlineto{\pgfqpoint{5.213578in}{2.011184in}}%
\pgfpathlineto{\pgfqpoint{4.530000in}{2.011184in}}%
\pgfpathclose%
\pgfusepath{fill}%
\end{pgfscope}%
\begin{pgfscope}%
\pgfpathrectangle{\pgfqpoint{4.530000in}{0.600000in}}{\pgfqpoint{1.200000in}{3.900000in}}%
\pgfusepath{clip}%
\pgfsetbuttcap%
\pgfsetmiterjoin%
\definecolor{currentfill}{rgb}{0.121569,0.466667,0.705882}%
\pgfsetfillcolor{currentfill}%
\pgfsetlinewidth{0.000000pt}%
\definecolor{currentstroke}{rgb}{0.000000,0.000000,0.000000}%
\pgfsetstrokecolor{currentstroke}%
\pgfsetstrokeopacity{0.000000}%
\pgfsetdash{}{0pt}%
\pgfpathmoveto{\pgfqpoint{4.530000in}{2.011184in}}%
\pgfpathlineto{\pgfqpoint{5.181535in}{2.011184in}}%
\pgfpathlineto{\pgfqpoint{5.181535in}{2.139474in}}%
\pgfpathlineto{\pgfqpoint{4.530000in}{2.139474in}}%
\pgfpathclose%
\pgfusepath{fill}%
\end{pgfscope}%
\begin{pgfscope}%
\pgfpathrectangle{\pgfqpoint{4.530000in}{0.600000in}}{\pgfqpoint{1.200000in}{3.900000in}}%
\pgfusepath{clip}%
\pgfsetbuttcap%
\pgfsetmiterjoin%
\definecolor{currentfill}{rgb}{0.121569,0.466667,0.705882}%
\pgfsetfillcolor{currentfill}%
\pgfsetlinewidth{0.000000pt}%
\definecolor{currentstroke}{rgb}{0.000000,0.000000,0.000000}%
\pgfsetstrokecolor{currentstroke}%
\pgfsetstrokeopacity{0.000000}%
\pgfsetdash{}{0pt}%
\pgfpathmoveto{\pgfqpoint{4.530000in}{2.139474in}}%
\pgfpathlineto{\pgfqpoint{5.277664in}{2.139474in}}%
\pgfpathlineto{\pgfqpoint{5.277664in}{2.267763in}}%
\pgfpathlineto{\pgfqpoint{4.530000in}{2.267763in}}%
\pgfpathclose%
\pgfusepath{fill}%
\end{pgfscope}%
\begin{pgfscope}%
\pgfpathrectangle{\pgfqpoint{4.530000in}{0.600000in}}{\pgfqpoint{1.200000in}{3.900000in}}%
\pgfusepath{clip}%
\pgfsetbuttcap%
\pgfsetmiterjoin%
\definecolor{currentfill}{rgb}{0.121569,0.466667,0.705882}%
\pgfsetfillcolor{currentfill}%
\pgfsetlinewidth{0.000000pt}%
\definecolor{currentstroke}{rgb}{0.000000,0.000000,0.000000}%
\pgfsetstrokecolor{currentstroke}%
\pgfsetstrokeopacity{0.000000}%
\pgfsetdash{}{0pt}%
\pgfpathmoveto{\pgfqpoint{4.530000in}{2.267763in}}%
\pgfpathlineto{\pgfqpoint{5.672857in}{2.267763in}}%
\pgfpathlineto{\pgfqpoint{5.672857in}{2.396053in}}%
\pgfpathlineto{\pgfqpoint{4.530000in}{2.396053in}}%
\pgfpathclose%
\pgfusepath{fill}%
\end{pgfscope}%
\begin{pgfscope}%
\pgfpathrectangle{\pgfqpoint{4.530000in}{0.600000in}}{\pgfqpoint{1.200000in}{3.900000in}}%
\pgfusepath{clip}%
\pgfsetbuttcap%
\pgfsetmiterjoin%
\definecolor{currentfill}{rgb}{0.121569,0.466667,0.705882}%
\pgfsetfillcolor{currentfill}%
\pgfsetlinewidth{0.000000pt}%
\definecolor{currentstroke}{rgb}{0.000000,0.000000,0.000000}%
\pgfsetstrokecolor{currentstroke}%
\pgfsetstrokeopacity{0.000000}%
\pgfsetdash{}{0pt}%
\pgfpathmoveto{\pgfqpoint{4.530000in}{2.396053in}}%
\pgfpathlineto{\pgfqpoint{5.405834in}{2.396053in}}%
\pgfpathlineto{\pgfqpoint{5.405834in}{2.524342in}}%
\pgfpathlineto{\pgfqpoint{4.530000in}{2.524342in}}%
\pgfpathclose%
\pgfusepath{fill}%
\end{pgfscope}%
\begin{pgfscope}%
\pgfpathrectangle{\pgfqpoint{4.530000in}{0.600000in}}{\pgfqpoint{1.200000in}{3.900000in}}%
\pgfusepath{clip}%
\pgfsetbuttcap%
\pgfsetmiterjoin%
\definecolor{currentfill}{rgb}{0.121569,0.466667,0.705882}%
\pgfsetfillcolor{currentfill}%
\pgfsetlinewidth{0.000000pt}%
\definecolor{currentstroke}{rgb}{0.000000,0.000000,0.000000}%
\pgfsetstrokecolor{currentstroke}%
\pgfsetstrokeopacity{0.000000}%
\pgfsetdash{}{0pt}%
\pgfpathmoveto{\pgfqpoint{4.530000in}{2.524342in}}%
\pgfpathlineto{\pgfqpoint{5.576729in}{2.524342in}}%
\pgfpathlineto{\pgfqpoint{5.576729in}{2.652632in}}%
\pgfpathlineto{\pgfqpoint{4.530000in}{2.652632in}}%
\pgfpathclose%
\pgfusepath{fill}%
\end{pgfscope}%
\begin{pgfscope}%
\pgfpathrectangle{\pgfqpoint{4.530000in}{0.600000in}}{\pgfqpoint{1.200000in}{3.900000in}}%
\pgfusepath{clip}%
\pgfsetbuttcap%
\pgfsetmiterjoin%
\definecolor{currentfill}{rgb}{0.121569,0.466667,0.705882}%
\pgfsetfillcolor{currentfill}%
\pgfsetlinewidth{0.000000pt}%
\definecolor{currentstroke}{rgb}{0.000000,0.000000,0.000000}%
\pgfsetstrokecolor{currentstroke}%
\pgfsetstrokeopacity{0.000000}%
\pgfsetdash{}{0pt}%
\pgfpathmoveto{\pgfqpoint{4.530000in}{2.652632in}}%
\pgfpathlineto{\pgfqpoint{5.534005in}{2.652632in}}%
\pgfpathlineto{\pgfqpoint{5.534005in}{2.780921in}}%
\pgfpathlineto{\pgfqpoint{4.530000in}{2.780921in}}%
\pgfpathclose%
\pgfusepath{fill}%
\end{pgfscope}%
\begin{pgfscope}%
\pgfpathrectangle{\pgfqpoint{4.530000in}{0.600000in}}{\pgfqpoint{1.200000in}{3.900000in}}%
\pgfusepath{clip}%
\pgfsetbuttcap%
\pgfsetmiterjoin%
\definecolor{currentfill}{rgb}{0.121569,0.466667,0.705882}%
\pgfsetfillcolor{currentfill}%
\pgfsetlinewidth{0.000000pt}%
\definecolor{currentstroke}{rgb}{0.000000,0.000000,0.000000}%
\pgfsetstrokecolor{currentstroke}%
\pgfsetstrokeopacity{0.000000}%
\pgfsetdash{}{0pt}%
\pgfpathmoveto{\pgfqpoint{4.530000in}{2.780921in}}%
\pgfpathlineto{\pgfqpoint{5.491282in}{2.780921in}}%
\pgfpathlineto{\pgfqpoint{5.491282in}{2.909211in}}%
\pgfpathlineto{\pgfqpoint{4.530000in}{2.909211in}}%
\pgfpathclose%
\pgfusepath{fill}%
\end{pgfscope}%
\begin{pgfscope}%
\pgfpathrectangle{\pgfqpoint{4.530000in}{0.600000in}}{\pgfqpoint{1.200000in}{3.900000in}}%
\pgfusepath{clip}%
\pgfsetbuttcap%
\pgfsetmiterjoin%
\definecolor{currentfill}{rgb}{0.121569,0.466667,0.705882}%
\pgfsetfillcolor{currentfill}%
\pgfsetlinewidth{0.000000pt}%
\definecolor{currentstroke}{rgb}{0.000000,0.000000,0.000000}%
\pgfsetstrokecolor{currentstroke}%
\pgfsetstrokeopacity{0.000000}%
\pgfsetdash{}{0pt}%
\pgfpathmoveto{\pgfqpoint{4.530000in}{2.909211in}}%
\pgfpathlineto{\pgfqpoint{5.266983in}{2.909211in}}%
\pgfpathlineto{\pgfqpoint{5.266983in}{3.037500in}}%
\pgfpathlineto{\pgfqpoint{4.530000in}{3.037500in}}%
\pgfpathclose%
\pgfusepath{fill}%
\end{pgfscope}%
\begin{pgfscope}%
\pgfpathrectangle{\pgfqpoint{4.530000in}{0.600000in}}{\pgfqpoint{1.200000in}{3.900000in}}%
\pgfusepath{clip}%
\pgfsetbuttcap%
\pgfsetmiterjoin%
\definecolor{currentfill}{rgb}{0.121569,0.466667,0.705882}%
\pgfsetfillcolor{currentfill}%
\pgfsetlinewidth{0.000000pt}%
\definecolor{currentstroke}{rgb}{0.000000,0.000000,0.000000}%
\pgfsetstrokecolor{currentstroke}%
\pgfsetstrokeopacity{0.000000}%
\pgfsetdash{}{0pt}%
\pgfpathmoveto{\pgfqpoint{4.530000in}{3.037500in}}%
\pgfpathlineto{\pgfqpoint{5.192216in}{3.037500in}}%
\pgfpathlineto{\pgfqpoint{5.192216in}{3.165789in}}%
\pgfpathlineto{\pgfqpoint{4.530000in}{3.165789in}}%
\pgfpathclose%
\pgfusepath{fill}%
\end{pgfscope}%
\begin{pgfscope}%
\pgfpathrectangle{\pgfqpoint{4.530000in}{0.600000in}}{\pgfqpoint{1.200000in}{3.900000in}}%
\pgfusepath{clip}%
\pgfsetbuttcap%
\pgfsetmiterjoin%
\definecolor{currentfill}{rgb}{0.121569,0.466667,0.705882}%
\pgfsetfillcolor{currentfill}%
\pgfsetlinewidth{0.000000pt}%
\definecolor{currentstroke}{rgb}{0.000000,0.000000,0.000000}%
\pgfsetstrokecolor{currentstroke}%
\pgfsetstrokeopacity{0.000000}%
\pgfsetdash{}{0pt}%
\pgfpathmoveto{\pgfqpoint{4.530000in}{3.165789in}}%
\pgfpathlineto{\pgfqpoint{4.893151in}{3.165789in}}%
\pgfpathlineto{\pgfqpoint{4.893151in}{3.294079in}}%
\pgfpathlineto{\pgfqpoint{4.530000in}{3.294079in}}%
\pgfpathclose%
\pgfusepath{fill}%
\end{pgfscope}%
\begin{pgfscope}%
\pgfpathrectangle{\pgfqpoint{4.530000in}{0.600000in}}{\pgfqpoint{1.200000in}{3.900000in}}%
\pgfusepath{clip}%
\pgfsetbuttcap%
\pgfsetmiterjoin%
\definecolor{currentfill}{rgb}{0.121569,0.466667,0.705882}%
\pgfsetfillcolor{currentfill}%
\pgfsetlinewidth{0.000000pt}%
\definecolor{currentstroke}{rgb}{0.000000,0.000000,0.000000}%
\pgfsetstrokecolor{currentstroke}%
\pgfsetstrokeopacity{0.000000}%
\pgfsetdash{}{0pt}%
\pgfpathmoveto{\pgfqpoint{4.530000in}{3.294079in}}%
\pgfpathlineto{\pgfqpoint{4.850427in}{3.294079in}}%
\pgfpathlineto{\pgfqpoint{4.850427in}{3.422368in}}%
\pgfpathlineto{\pgfqpoint{4.530000in}{3.422368in}}%
\pgfpathclose%
\pgfusepath{fill}%
\end{pgfscope}%
\begin{pgfscope}%
\pgfpathrectangle{\pgfqpoint{4.530000in}{0.600000in}}{\pgfqpoint{1.200000in}{3.900000in}}%
\pgfusepath{clip}%
\pgfsetbuttcap%
\pgfsetmiterjoin%
\definecolor{currentfill}{rgb}{0.121569,0.466667,0.705882}%
\pgfsetfillcolor{currentfill}%
\pgfsetlinewidth{0.000000pt}%
\definecolor{currentstroke}{rgb}{0.000000,0.000000,0.000000}%
\pgfsetstrokecolor{currentstroke}%
\pgfsetstrokeopacity{0.000000}%
\pgfsetdash{}{0pt}%
\pgfpathmoveto{\pgfqpoint{4.530000in}{3.422368in}}%
\pgfpathlineto{\pgfqpoint{4.722256in}{3.422368in}}%
\pgfpathlineto{\pgfqpoint{4.722256in}{3.550658in}}%
\pgfpathlineto{\pgfqpoint{4.530000in}{3.550658in}}%
\pgfpathclose%
\pgfusepath{fill}%
\end{pgfscope}%
\begin{pgfscope}%
\pgfpathrectangle{\pgfqpoint{4.530000in}{0.600000in}}{\pgfqpoint{1.200000in}{3.900000in}}%
\pgfusepath{clip}%
\pgfsetbuttcap%
\pgfsetmiterjoin%
\definecolor{currentfill}{rgb}{0.121569,0.466667,0.705882}%
\pgfsetfillcolor{currentfill}%
\pgfsetlinewidth{0.000000pt}%
\definecolor{currentstroke}{rgb}{0.000000,0.000000,0.000000}%
\pgfsetstrokecolor{currentstroke}%
\pgfsetstrokeopacity{0.000000}%
\pgfsetdash{}{0pt}%
\pgfpathmoveto{\pgfqpoint{4.530000in}{3.550658in}}%
\pgfpathlineto{\pgfqpoint{4.658171in}{3.550658in}}%
\pgfpathlineto{\pgfqpoint{4.658171in}{3.678947in}}%
\pgfpathlineto{\pgfqpoint{4.530000in}{3.678947in}}%
\pgfpathclose%
\pgfusepath{fill}%
\end{pgfscope}%
\begin{pgfscope}%
\pgfpathrectangle{\pgfqpoint{4.530000in}{0.600000in}}{\pgfqpoint{1.200000in}{3.900000in}}%
\pgfusepath{clip}%
\pgfsetbuttcap%
\pgfsetmiterjoin%
\definecolor{currentfill}{rgb}{0.121569,0.466667,0.705882}%
\pgfsetfillcolor{currentfill}%
\pgfsetlinewidth{0.000000pt}%
\definecolor{currentstroke}{rgb}{0.000000,0.000000,0.000000}%
\pgfsetstrokecolor{currentstroke}%
\pgfsetstrokeopacity{0.000000}%
\pgfsetdash{}{0pt}%
\pgfpathmoveto{\pgfqpoint{4.530000in}{3.678947in}}%
\pgfpathlineto{\pgfqpoint{4.636809in}{3.678947in}}%
\pgfpathlineto{\pgfqpoint{4.636809in}{3.807237in}}%
\pgfpathlineto{\pgfqpoint{4.530000in}{3.807237in}}%
\pgfpathclose%
\pgfusepath{fill}%
\end{pgfscope}%
\begin{pgfscope}%
\pgfpathrectangle{\pgfqpoint{4.530000in}{0.600000in}}{\pgfqpoint{1.200000in}{3.900000in}}%
\pgfusepath{clip}%
\pgfsetbuttcap%
\pgfsetmiterjoin%
\definecolor{currentfill}{rgb}{0.121569,0.466667,0.705882}%
\pgfsetfillcolor{currentfill}%
\pgfsetlinewidth{0.000000pt}%
\definecolor{currentstroke}{rgb}{0.000000,0.000000,0.000000}%
\pgfsetstrokecolor{currentstroke}%
\pgfsetstrokeopacity{0.000000}%
\pgfsetdash{}{0pt}%
\pgfpathmoveto{\pgfqpoint{4.530000in}{3.807237in}}%
\pgfpathlineto{\pgfqpoint{4.562043in}{3.807237in}}%
\pgfpathlineto{\pgfqpoint{4.562043in}{3.935526in}}%
\pgfpathlineto{\pgfqpoint{4.530000in}{3.935526in}}%
\pgfpathclose%
\pgfusepath{fill}%
\end{pgfscope}%
\begin{pgfscope}%
\pgfpathrectangle{\pgfqpoint{4.530000in}{0.600000in}}{\pgfqpoint{1.200000in}{3.900000in}}%
\pgfusepath{clip}%
\pgfsetbuttcap%
\pgfsetmiterjoin%
\definecolor{currentfill}{rgb}{0.121569,0.466667,0.705882}%
\pgfsetfillcolor{currentfill}%
\pgfsetlinewidth{0.000000pt}%
\definecolor{currentstroke}{rgb}{0.000000,0.000000,0.000000}%
\pgfsetstrokecolor{currentstroke}%
\pgfsetstrokeopacity{0.000000}%
\pgfsetdash{}{0pt}%
\pgfpathmoveto{\pgfqpoint{4.530000in}{3.935526in}}%
\pgfpathlineto{\pgfqpoint{4.540681in}{3.935526in}}%
\pgfpathlineto{\pgfqpoint{4.540681in}{4.063816in}}%
\pgfpathlineto{\pgfqpoint{4.530000in}{4.063816in}}%
\pgfpathclose%
\pgfusepath{fill}%
\end{pgfscope}%
\begin{pgfscope}%
\pgfpathrectangle{\pgfqpoint{4.530000in}{0.600000in}}{\pgfqpoint{1.200000in}{3.900000in}}%
\pgfusepath{clip}%
\pgfsetbuttcap%
\pgfsetmiterjoin%
\definecolor{currentfill}{rgb}{0.121569,0.466667,0.705882}%
\pgfsetfillcolor{currentfill}%
\pgfsetlinewidth{0.000000pt}%
\definecolor{currentstroke}{rgb}{0.000000,0.000000,0.000000}%
\pgfsetstrokecolor{currentstroke}%
\pgfsetstrokeopacity{0.000000}%
\pgfsetdash{}{0pt}%
\pgfpathmoveto{\pgfqpoint{4.530000in}{4.063816in}}%
\pgfpathlineto{\pgfqpoint{4.551362in}{4.063816in}}%
\pgfpathlineto{\pgfqpoint{4.551362in}{4.192105in}}%
\pgfpathlineto{\pgfqpoint{4.530000in}{4.192105in}}%
\pgfpathclose%
\pgfusepath{fill}%
\end{pgfscope}%
\begin{pgfscope}%
\pgfpathrectangle{\pgfqpoint{4.530000in}{0.600000in}}{\pgfqpoint{1.200000in}{3.900000in}}%
\pgfusepath{clip}%
\pgfsetbuttcap%
\pgfsetmiterjoin%
\definecolor{currentfill}{rgb}{0.121569,0.466667,0.705882}%
\pgfsetfillcolor{currentfill}%
\pgfsetlinewidth{0.000000pt}%
\definecolor{currentstroke}{rgb}{0.000000,0.000000,0.000000}%
\pgfsetstrokecolor{currentstroke}%
\pgfsetstrokeopacity{0.000000}%
\pgfsetdash{}{0pt}%
\pgfpathmoveto{\pgfqpoint{4.530000in}{4.192105in}}%
\pgfpathlineto{\pgfqpoint{4.530000in}{4.192105in}}%
\pgfpathlineto{\pgfqpoint{4.530000in}{4.320395in}}%
\pgfpathlineto{\pgfqpoint{4.530000in}{4.320395in}}%
\pgfpathclose%
\pgfusepath{fill}%
\end{pgfscope}%
\begin{pgfscope}%
\pgfpathrectangle{\pgfqpoint{4.530000in}{0.600000in}}{\pgfqpoint{1.200000in}{3.900000in}}%
\pgfusepath{clip}%
\pgfsetbuttcap%
\pgfsetmiterjoin%
\definecolor{currentfill}{rgb}{0.121569,0.466667,0.705882}%
\pgfsetfillcolor{currentfill}%
\pgfsetlinewidth{0.000000pt}%
\definecolor{currentstroke}{rgb}{0.000000,0.000000,0.000000}%
\pgfsetstrokecolor{currentstroke}%
\pgfsetstrokeopacity{0.000000}%
\pgfsetdash{}{0pt}%
\pgfpathmoveto{\pgfqpoint{4.530000in}{4.320395in}}%
\pgfpathlineto{\pgfqpoint{4.530000in}{4.320395in}}%
\pgfpathlineto{\pgfqpoint{4.530000in}{4.448684in}}%
\pgfpathlineto{\pgfqpoint{4.530000in}{4.448684in}}%
\pgfpathclose%
\pgfusepath{fill}%
\end{pgfscope}%
\begin{pgfscope}%
\pgfpathrectangle{\pgfqpoint{4.530000in}{0.600000in}}{\pgfqpoint{1.200000in}{3.900000in}}%
\pgfusepath{clip}%
\pgfsetbuttcap%
\pgfsetmiterjoin%
\definecolor{currentfill}{rgb}{0.121569,0.466667,0.705882}%
\pgfsetfillcolor{currentfill}%
\pgfsetlinewidth{0.000000pt}%
\definecolor{currentstroke}{rgb}{0.000000,0.000000,0.000000}%
\pgfsetstrokecolor{currentstroke}%
\pgfsetstrokeopacity{0.000000}%
\pgfsetdash{}{0pt}%
\pgfpathmoveto{\pgfqpoint{4.530000in}{4.448684in}}%
\pgfpathlineto{\pgfqpoint{4.530000in}{4.448684in}}%
\pgfpathlineto{\pgfqpoint{4.530000in}{4.576974in}}%
\pgfpathlineto{\pgfqpoint{4.530000in}{4.576974in}}%
\pgfpathclose%
\pgfusepath{fill}%
\end{pgfscope}%
\begin{pgfscope}%
\pgfsetbuttcap%
\pgfsetroundjoin%
\definecolor{currentfill}{rgb}{0.000000,0.000000,0.000000}%
\pgfsetfillcolor{currentfill}%
\pgfsetlinewidth{0.803000pt}%
\definecolor{currentstroke}{rgb}{0.000000,0.000000,0.000000}%
\pgfsetstrokecolor{currentstroke}%
\pgfsetdash{}{0pt}%
\pgfsys@defobject{currentmarker}{\pgfqpoint{0.000000in}{0.000000in}}{\pgfqpoint{0.000000in}{0.048611in}}{%
\pgfpathmoveto{\pgfqpoint{0.000000in}{0.000000in}}%
\pgfpathlineto{\pgfqpoint{0.000000in}{0.048611in}}%
\pgfusepath{stroke,fill}%
}%
\begin{pgfscope}%
\pgfsys@transformshift{4.530000in}{0.600000in}%
\pgfsys@useobject{currentmarker}{}%
\end{pgfscope}%
\end{pgfscope}%
\begin{pgfscope}%
\definecolor{textcolor}{rgb}{0.000000,0.000000,0.000000}%
\pgfsetstrokecolor{textcolor}%
\pgfsetfillcolor{textcolor}%
\pgftext[x=4.530000in,y=0.551389in,,top]{\color{textcolor}\sffamily\fontsize{10.000000}{12.000000}\selectfont 0}%
\end{pgfscope}%
\begin{pgfscope}%
\pgfsetbuttcap%
\pgfsetroundjoin%
\definecolor{currentfill}{rgb}{0.000000,0.000000,0.000000}%
\pgfsetfillcolor{currentfill}%
\pgfsetlinewidth{0.803000pt}%
\definecolor{currentstroke}{rgb}{0.000000,0.000000,0.000000}%
\pgfsetstrokecolor{currentstroke}%
\pgfsetdash{}{0pt}%
\pgfsys@defobject{currentmarker}{\pgfqpoint{0.000000in}{0.000000in}}{\pgfqpoint{0.000000in}{0.048611in}}{%
\pgfpathmoveto{\pgfqpoint{0.000000in}{0.000000in}}%
\pgfpathlineto{\pgfqpoint{0.000000in}{0.048611in}}%
\pgfusepath{stroke,fill}%
}%
\begin{pgfscope}%
\pgfsys@transformshift{5.598091in}{0.600000in}%
\pgfsys@useobject{currentmarker}{}%
\end{pgfscope}%
\end{pgfscope}%
\begin{pgfscope}%
\definecolor{textcolor}{rgb}{0.000000,0.000000,0.000000}%
\pgfsetstrokecolor{textcolor}%
\pgfsetfillcolor{textcolor}%
\pgftext[x=5.598091in,y=0.551389in,,top]{\color{textcolor}\sffamily\fontsize{10.000000}{12.000000}\selectfont 100}%
\end{pgfscope}%
\begin{pgfscope}%
\pgfsetbuttcap%
\pgfsetroundjoin%
\definecolor{currentfill}{rgb}{0.000000,0.000000,0.000000}%
\pgfsetfillcolor{currentfill}%
\pgfsetlinewidth{0.803000pt}%
\definecolor{currentstroke}{rgb}{0.000000,0.000000,0.000000}%
\pgfsetstrokecolor{currentstroke}%
\pgfsetdash{}{0pt}%
\pgfsys@defobject{currentmarker}{\pgfqpoint{0.000000in}{0.000000in}}{\pgfqpoint{0.048611in}{0.000000in}}{%
\pgfpathmoveto{\pgfqpoint{0.000000in}{0.000000in}}%
\pgfpathlineto{\pgfqpoint{0.048611in}{0.000000in}}%
\pgfusepath{stroke,fill}%
}%
\begin{pgfscope}%
\pgfsys@transformshift{4.530000in}{1.010526in}%
\pgfsys@useobject{currentmarker}{}%
\end{pgfscope}%
\end{pgfscope}%
\begin{pgfscope}%
\pgfsetbuttcap%
\pgfsetroundjoin%
\definecolor{currentfill}{rgb}{0.000000,0.000000,0.000000}%
\pgfsetfillcolor{currentfill}%
\pgfsetlinewidth{0.803000pt}%
\definecolor{currentstroke}{rgb}{0.000000,0.000000,0.000000}%
\pgfsetstrokecolor{currentstroke}%
\pgfsetdash{}{0pt}%
\pgfsys@defobject{currentmarker}{\pgfqpoint{0.000000in}{0.000000in}}{\pgfqpoint{0.048611in}{0.000000in}}{%
\pgfpathmoveto{\pgfqpoint{0.000000in}{0.000000in}}%
\pgfpathlineto{\pgfqpoint{0.048611in}{0.000000in}}%
\pgfusepath{stroke,fill}%
}%
\begin{pgfscope}%
\pgfsys@transformshift{4.530000in}{1.523684in}%
\pgfsys@useobject{currentmarker}{}%
\end{pgfscope}%
\end{pgfscope}%
\begin{pgfscope}%
\pgfsetbuttcap%
\pgfsetroundjoin%
\definecolor{currentfill}{rgb}{0.000000,0.000000,0.000000}%
\pgfsetfillcolor{currentfill}%
\pgfsetlinewidth{0.803000pt}%
\definecolor{currentstroke}{rgb}{0.000000,0.000000,0.000000}%
\pgfsetstrokecolor{currentstroke}%
\pgfsetdash{}{0pt}%
\pgfsys@defobject{currentmarker}{\pgfqpoint{0.000000in}{0.000000in}}{\pgfqpoint{0.048611in}{0.000000in}}{%
\pgfpathmoveto{\pgfqpoint{0.000000in}{0.000000in}}%
\pgfpathlineto{\pgfqpoint{0.048611in}{0.000000in}}%
\pgfusepath{stroke,fill}%
}%
\begin{pgfscope}%
\pgfsys@transformshift{4.530000in}{2.036842in}%
\pgfsys@useobject{currentmarker}{}%
\end{pgfscope}%
\end{pgfscope}%
\begin{pgfscope}%
\pgfsetbuttcap%
\pgfsetroundjoin%
\definecolor{currentfill}{rgb}{0.000000,0.000000,0.000000}%
\pgfsetfillcolor{currentfill}%
\pgfsetlinewidth{0.803000pt}%
\definecolor{currentstroke}{rgb}{0.000000,0.000000,0.000000}%
\pgfsetstrokecolor{currentstroke}%
\pgfsetdash{}{0pt}%
\pgfsys@defobject{currentmarker}{\pgfqpoint{0.000000in}{0.000000in}}{\pgfqpoint{0.048611in}{0.000000in}}{%
\pgfpathmoveto{\pgfqpoint{0.000000in}{0.000000in}}%
\pgfpathlineto{\pgfqpoint{0.048611in}{0.000000in}}%
\pgfusepath{stroke,fill}%
}%
\begin{pgfscope}%
\pgfsys@transformshift{4.530000in}{2.550000in}%
\pgfsys@useobject{currentmarker}{}%
\end{pgfscope}%
\end{pgfscope}%
\begin{pgfscope}%
\pgfsetbuttcap%
\pgfsetroundjoin%
\definecolor{currentfill}{rgb}{0.000000,0.000000,0.000000}%
\pgfsetfillcolor{currentfill}%
\pgfsetlinewidth{0.803000pt}%
\definecolor{currentstroke}{rgb}{0.000000,0.000000,0.000000}%
\pgfsetstrokecolor{currentstroke}%
\pgfsetdash{}{0pt}%
\pgfsys@defobject{currentmarker}{\pgfqpoint{0.000000in}{0.000000in}}{\pgfqpoint{0.048611in}{0.000000in}}{%
\pgfpathmoveto{\pgfqpoint{0.000000in}{0.000000in}}%
\pgfpathlineto{\pgfqpoint{0.048611in}{0.000000in}}%
\pgfusepath{stroke,fill}%
}%
\begin{pgfscope}%
\pgfsys@transformshift{4.530000in}{3.063158in}%
\pgfsys@useobject{currentmarker}{}%
\end{pgfscope}%
\end{pgfscope}%
\begin{pgfscope}%
\pgfsetbuttcap%
\pgfsetroundjoin%
\definecolor{currentfill}{rgb}{0.000000,0.000000,0.000000}%
\pgfsetfillcolor{currentfill}%
\pgfsetlinewidth{0.803000pt}%
\definecolor{currentstroke}{rgb}{0.000000,0.000000,0.000000}%
\pgfsetstrokecolor{currentstroke}%
\pgfsetdash{}{0pt}%
\pgfsys@defobject{currentmarker}{\pgfqpoint{0.000000in}{0.000000in}}{\pgfqpoint{0.048611in}{0.000000in}}{%
\pgfpathmoveto{\pgfqpoint{0.000000in}{0.000000in}}%
\pgfpathlineto{\pgfqpoint{0.048611in}{0.000000in}}%
\pgfusepath{stroke,fill}%
}%
\begin{pgfscope}%
\pgfsys@transformshift{4.530000in}{3.576316in}%
\pgfsys@useobject{currentmarker}{}%
\end{pgfscope}%
\end{pgfscope}%
\begin{pgfscope}%
\pgfsetbuttcap%
\pgfsetroundjoin%
\definecolor{currentfill}{rgb}{0.000000,0.000000,0.000000}%
\pgfsetfillcolor{currentfill}%
\pgfsetlinewidth{0.803000pt}%
\definecolor{currentstroke}{rgb}{0.000000,0.000000,0.000000}%
\pgfsetstrokecolor{currentstroke}%
\pgfsetdash{}{0pt}%
\pgfsys@defobject{currentmarker}{\pgfqpoint{0.000000in}{0.000000in}}{\pgfqpoint{0.048611in}{0.000000in}}{%
\pgfpathmoveto{\pgfqpoint{0.000000in}{0.000000in}}%
\pgfpathlineto{\pgfqpoint{0.048611in}{0.000000in}}%
\pgfusepath{stroke,fill}%
}%
\begin{pgfscope}%
\pgfsys@transformshift{4.530000in}{4.089474in}%
\pgfsys@useobject{currentmarker}{}%
\end{pgfscope}%
\end{pgfscope}%
\begin{pgfscope}%
\pgfsetrectcap%
\pgfsetmiterjoin%
\pgfsetlinewidth{0.803000pt}%
\definecolor{currentstroke}{rgb}{0.000000,0.000000,0.000000}%
\pgfsetstrokecolor{currentstroke}%
\pgfsetdash{}{0pt}%
\pgfpathmoveto{\pgfqpoint{4.530000in}{0.600000in}}%
\pgfpathlineto{\pgfqpoint{4.530000in}{4.500000in}}%
\pgfusepath{stroke}%
\end{pgfscope}%
\begin{pgfscope}%
\pgfsetrectcap%
\pgfsetmiterjoin%
\pgfsetlinewidth{0.803000pt}%
\definecolor{currentstroke}{rgb}{0.000000,0.000000,0.000000}%
\pgfsetstrokecolor{currentstroke}%
\pgfsetdash{}{0pt}%
\pgfpathmoveto{\pgfqpoint{5.730000in}{0.600000in}}%
\pgfpathlineto{\pgfqpoint{5.730000in}{4.500000in}}%
\pgfusepath{stroke}%
\end{pgfscope}%
\begin{pgfscope}%
\pgfsetrectcap%
\pgfsetmiterjoin%
\pgfsetlinewidth{0.803000pt}%
\definecolor{currentstroke}{rgb}{0.000000,0.000000,0.000000}%
\pgfsetstrokecolor{currentstroke}%
\pgfsetdash{}{0pt}%
\pgfpathmoveto{\pgfqpoint{4.530000in}{0.600000in}}%
\pgfpathlineto{\pgfqpoint{5.730000in}{0.600000in}}%
\pgfusepath{stroke}%
\end{pgfscope}%
\begin{pgfscope}%
\pgfsetrectcap%
\pgfsetmiterjoin%
\pgfsetlinewidth{0.803000pt}%
\definecolor{currentstroke}{rgb}{0.000000,0.000000,0.000000}%
\pgfsetstrokecolor{currentstroke}%
\pgfsetdash{}{0pt}%
\pgfpathmoveto{\pgfqpoint{4.530000in}{4.500000in}}%
\pgfpathlineto{\pgfqpoint{5.730000in}{4.500000in}}%
\pgfusepath{stroke}%
\end{pgfscope}%
\end{pgfpicture}%
\makeatother%
\endgroup%

\caption{Scatter plot of $1000$ pairs of samples from the first algorithm}
\label{Fig:Algo1}
\end{figure}

\begin{figure}[htbp]
\centering
%% Creator: Matplotlib, PGF backend
%%
%% To include the figure in your LaTeX document, write
%%   \input{<filename>.pgf}
%%
%% Make sure the required packages are loaded in your preamble
%%   \usepackage{pgf}
%%
%% Figures using additional raster images can only be included by \input if
%% they are in the same directory as the main LaTeX file. For loading figures
%% from other directories you can use the `import` package
%%   \usepackage{import}
%% and then include the figures with
%%   \import{<path to file>}{<filename>.pgf}
%%
%% Matplotlib used the following preamble
%%   \usepackage{fontspec}
%%   \setmainfont{DejaVuSerif.ttf}[Path=/home/lzh/anaconda3/envs/numana/lib/python3.7/site-packages/matplotlib/mpl-data/fonts/ttf/]
%%   \setsansfont{DejaVuSans.ttf}[Path=/home/lzh/anaconda3/envs/numana/lib/python3.7/site-packages/matplotlib/mpl-data/fonts/ttf/]
%%   \setmonofont{DejaVuSansMono.ttf}[Path=/home/lzh/anaconda3/envs/numana/lib/python3.7/site-packages/matplotlib/mpl-data/fonts/ttf/]
%%
\begingroup%
\makeatletter%
\begin{pgfpicture}%
\pgfpathrectangle{\pgfpointorigin}{\pgfqpoint{6.000000in}{6.000000in}}%
\pgfusepath{use as bounding box, clip}%
\begin{pgfscope}%
\pgfsetbuttcap%
\pgfsetmiterjoin%
\definecolor{currentfill}{rgb}{1.000000,1.000000,1.000000}%
\pgfsetfillcolor{currentfill}%
\pgfsetlinewidth{0.000000pt}%
\definecolor{currentstroke}{rgb}{1.000000,1.000000,1.000000}%
\pgfsetstrokecolor{currentstroke}%
\pgfsetdash{}{0pt}%
\pgfpathmoveto{\pgfqpoint{0.000000in}{0.000000in}}%
\pgfpathlineto{\pgfqpoint{6.000000in}{0.000000in}}%
\pgfpathlineto{\pgfqpoint{6.000000in}{6.000000in}}%
\pgfpathlineto{\pgfqpoint{0.000000in}{6.000000in}}%
\pgfpathclose%
\pgfusepath{fill}%
\end{pgfscope}%
\begin{pgfscope}%
\pgfsetbuttcap%
\pgfsetmiterjoin%
\definecolor{currentfill}{rgb}{1.000000,1.000000,1.000000}%
\pgfsetfillcolor{currentfill}%
\pgfsetlinewidth{0.000000pt}%
\definecolor{currentstroke}{rgb}{0.000000,0.000000,0.000000}%
\pgfsetstrokecolor{currentstroke}%
\pgfsetstrokeopacity{0.000000}%
\pgfsetdash{}{0pt}%
\pgfpathmoveto{\pgfqpoint{0.600000in}{0.600000in}}%
\pgfpathlineto{\pgfqpoint{4.500000in}{0.600000in}}%
\pgfpathlineto{\pgfqpoint{4.500000in}{4.500000in}}%
\pgfpathlineto{\pgfqpoint{0.600000in}{4.500000in}}%
\pgfpathclose%
\pgfusepath{fill}%
\end{pgfscope}%
\begin{pgfscope}%
\pgfpathrectangle{\pgfqpoint{0.600000in}{0.600000in}}{\pgfqpoint{3.900000in}{3.900000in}}%
\pgfusepath{clip}%
\pgfsetbuttcap%
\pgfsetroundjoin%
\definecolor{currentfill}{rgb}{0.121569,0.466667,0.705882}%
\pgfsetfillcolor{currentfill}%
\pgfsetlinewidth{1.003750pt}%
\definecolor{currentstroke}{rgb}{0.121569,0.466667,0.705882}%
\pgfsetstrokecolor{currentstroke}%
\pgfsetdash{}{0pt}%
\pgfpathmoveto{\pgfqpoint{2.193249in}{2.779173in}}%
\pgfpathcurveto{\pgfqpoint{2.204299in}{2.779173in}}{\pgfqpoint{2.214898in}{2.783563in}}{\pgfqpoint{2.222712in}{2.791377in}}%
\pgfpathcurveto{\pgfqpoint{2.230526in}{2.799190in}}{\pgfqpoint{2.234916in}{2.809789in}}{\pgfqpoint{2.234916in}{2.820839in}}%
\pgfpathcurveto{\pgfqpoint{2.234916in}{2.831890in}}{\pgfqpoint{2.230526in}{2.842489in}}{\pgfqpoint{2.222712in}{2.850302in}}%
\pgfpathcurveto{\pgfqpoint{2.214898in}{2.858116in}}{\pgfqpoint{2.204299in}{2.862506in}}{\pgfqpoint{2.193249in}{2.862506in}}%
\pgfpathcurveto{\pgfqpoint{2.182199in}{2.862506in}}{\pgfqpoint{2.171600in}{2.858116in}}{\pgfqpoint{2.163786in}{2.850302in}}%
\pgfpathcurveto{\pgfqpoint{2.155973in}{2.842489in}}{\pgfqpoint{2.151582in}{2.831890in}}{\pgfqpoint{2.151582in}{2.820839in}}%
\pgfpathcurveto{\pgfqpoint{2.151582in}{2.809789in}}{\pgfqpoint{2.155973in}{2.799190in}}{\pgfqpoint{2.163786in}{2.791377in}}%
\pgfpathcurveto{\pgfqpoint{2.171600in}{2.783563in}}{\pgfqpoint{2.182199in}{2.779173in}}{\pgfqpoint{2.193249in}{2.779173in}}%
\pgfpathclose%
\pgfusepath{stroke,fill}%
\end{pgfscope}%
\begin{pgfscope}%
\pgfpathrectangle{\pgfqpoint{0.600000in}{0.600000in}}{\pgfqpoint{3.900000in}{3.900000in}}%
\pgfusepath{clip}%
\pgfsetbuttcap%
\pgfsetroundjoin%
\definecolor{currentfill}{rgb}{0.121569,0.466667,0.705882}%
\pgfsetfillcolor{currentfill}%
\pgfsetlinewidth{1.003750pt}%
\definecolor{currentstroke}{rgb}{0.121569,0.466667,0.705882}%
\pgfsetstrokecolor{currentstroke}%
\pgfsetdash{}{0pt}%
\pgfpathmoveto{\pgfqpoint{2.743981in}{3.225500in}}%
\pgfpathcurveto{\pgfqpoint{2.755031in}{3.225500in}}{\pgfqpoint{2.765630in}{3.229891in}}{\pgfqpoint{2.773443in}{3.237704in}}%
\pgfpathcurveto{\pgfqpoint{2.781257in}{3.245518in}}{\pgfqpoint{2.785647in}{3.256117in}}{\pgfqpoint{2.785647in}{3.267167in}}%
\pgfpathcurveto{\pgfqpoint{2.785647in}{3.278217in}}{\pgfqpoint{2.781257in}{3.288816in}}{\pgfqpoint{2.773443in}{3.296630in}}%
\pgfpathcurveto{\pgfqpoint{2.765630in}{3.304443in}}{\pgfqpoint{2.755031in}{3.308834in}}{\pgfqpoint{2.743981in}{3.308834in}}%
\pgfpathcurveto{\pgfqpoint{2.732931in}{3.308834in}}{\pgfqpoint{2.722331in}{3.304443in}}{\pgfqpoint{2.714518in}{3.296630in}}%
\pgfpathcurveto{\pgfqpoint{2.706704in}{3.288816in}}{\pgfqpoint{2.702314in}{3.278217in}}{\pgfqpoint{2.702314in}{3.267167in}}%
\pgfpathcurveto{\pgfqpoint{2.702314in}{3.256117in}}{\pgfqpoint{2.706704in}{3.245518in}}{\pgfqpoint{2.714518in}{3.237704in}}%
\pgfpathcurveto{\pgfqpoint{2.722331in}{3.229891in}}{\pgfqpoint{2.732931in}{3.225500in}}{\pgfqpoint{2.743981in}{3.225500in}}%
\pgfpathclose%
\pgfusepath{stroke,fill}%
\end{pgfscope}%
\begin{pgfscope}%
\pgfpathrectangle{\pgfqpoint{0.600000in}{0.600000in}}{\pgfqpoint{3.900000in}{3.900000in}}%
\pgfusepath{clip}%
\pgfsetbuttcap%
\pgfsetroundjoin%
\definecolor{currentfill}{rgb}{0.121569,0.466667,0.705882}%
\pgfsetfillcolor{currentfill}%
\pgfsetlinewidth{1.003750pt}%
\definecolor{currentstroke}{rgb}{0.121569,0.466667,0.705882}%
\pgfsetstrokecolor{currentstroke}%
\pgfsetdash{}{0pt}%
\pgfpathmoveto{\pgfqpoint{3.291205in}{2.002225in}}%
\pgfpathcurveto{\pgfqpoint{3.302255in}{2.002225in}}{\pgfqpoint{3.312854in}{2.006615in}}{\pgfqpoint{3.320668in}{2.014429in}}%
\pgfpathcurveto{\pgfqpoint{3.328481in}{2.022242in}}{\pgfqpoint{3.332872in}{2.032841in}}{\pgfqpoint{3.332872in}{2.043892in}}%
\pgfpathcurveto{\pgfqpoint{3.332872in}{2.054942in}}{\pgfqpoint{3.328481in}{2.065541in}}{\pgfqpoint{3.320668in}{2.073354in}}%
\pgfpathcurveto{\pgfqpoint{3.312854in}{2.081168in}}{\pgfqpoint{3.302255in}{2.085558in}}{\pgfqpoint{3.291205in}{2.085558in}}%
\pgfpathcurveto{\pgfqpoint{3.280155in}{2.085558in}}{\pgfqpoint{3.269556in}{2.081168in}}{\pgfqpoint{3.261742in}{2.073354in}}%
\pgfpathcurveto{\pgfqpoint{3.253929in}{2.065541in}}{\pgfqpoint{3.249538in}{2.054942in}}{\pgfqpoint{3.249538in}{2.043892in}}%
\pgfpathcurveto{\pgfqpoint{3.249538in}{2.032841in}}{\pgfqpoint{3.253929in}{2.022242in}}{\pgfqpoint{3.261742in}{2.014429in}}%
\pgfpathcurveto{\pgfqpoint{3.269556in}{2.006615in}}{\pgfqpoint{3.280155in}{2.002225in}}{\pgfqpoint{3.291205in}{2.002225in}}%
\pgfpathclose%
\pgfusepath{stroke,fill}%
\end{pgfscope}%
\begin{pgfscope}%
\pgfpathrectangle{\pgfqpoint{0.600000in}{0.600000in}}{\pgfqpoint{3.900000in}{3.900000in}}%
\pgfusepath{clip}%
\pgfsetbuttcap%
\pgfsetroundjoin%
\definecolor{currentfill}{rgb}{0.121569,0.466667,0.705882}%
\pgfsetfillcolor{currentfill}%
\pgfsetlinewidth{1.003750pt}%
\definecolor{currentstroke}{rgb}{0.121569,0.466667,0.705882}%
\pgfsetstrokecolor{currentstroke}%
\pgfsetdash{}{0pt}%
\pgfpathmoveto{\pgfqpoint{2.837106in}{2.696532in}}%
\pgfpathcurveto{\pgfqpoint{2.848156in}{2.696532in}}{\pgfqpoint{2.858755in}{2.700922in}}{\pgfqpoint{2.866568in}{2.708736in}}%
\pgfpathcurveto{\pgfqpoint{2.874382in}{2.716549in}}{\pgfqpoint{2.878772in}{2.727148in}}{\pgfqpoint{2.878772in}{2.738198in}}%
\pgfpathcurveto{\pgfqpoint{2.878772in}{2.749249in}}{\pgfqpoint{2.874382in}{2.759848in}}{\pgfqpoint{2.866568in}{2.767661in}}%
\pgfpathcurveto{\pgfqpoint{2.858755in}{2.775475in}}{\pgfqpoint{2.848156in}{2.779865in}}{\pgfqpoint{2.837106in}{2.779865in}}%
\pgfpathcurveto{\pgfqpoint{2.826055in}{2.779865in}}{\pgfqpoint{2.815456in}{2.775475in}}{\pgfqpoint{2.807643in}{2.767661in}}%
\pgfpathcurveto{\pgfqpoint{2.799829in}{2.759848in}}{\pgfqpoint{2.795439in}{2.749249in}}{\pgfqpoint{2.795439in}{2.738198in}}%
\pgfpathcurveto{\pgfqpoint{2.795439in}{2.727148in}}{\pgfqpoint{2.799829in}{2.716549in}}{\pgfqpoint{2.807643in}{2.708736in}}%
\pgfpathcurveto{\pgfqpoint{2.815456in}{2.700922in}}{\pgfqpoint{2.826055in}{2.696532in}}{\pgfqpoint{2.837106in}{2.696532in}}%
\pgfpathclose%
\pgfusepath{stroke,fill}%
\end{pgfscope}%
\begin{pgfscope}%
\pgfpathrectangle{\pgfqpoint{0.600000in}{0.600000in}}{\pgfqpoint{3.900000in}{3.900000in}}%
\pgfusepath{clip}%
\pgfsetbuttcap%
\pgfsetroundjoin%
\definecolor{currentfill}{rgb}{0.121569,0.466667,0.705882}%
\pgfsetfillcolor{currentfill}%
\pgfsetlinewidth{1.003750pt}%
\definecolor{currentstroke}{rgb}{0.121569,0.466667,0.705882}%
\pgfsetstrokecolor{currentstroke}%
\pgfsetdash{}{0pt}%
\pgfpathmoveto{\pgfqpoint{2.166581in}{1.468867in}}%
\pgfpathcurveto{\pgfqpoint{2.177631in}{1.468867in}}{\pgfqpoint{2.188230in}{1.473257in}}{\pgfqpoint{2.196043in}{1.481071in}}%
\pgfpathcurveto{\pgfqpoint{2.203857in}{1.488885in}}{\pgfqpoint{2.208247in}{1.499484in}}{\pgfqpoint{2.208247in}{1.510534in}}%
\pgfpathcurveto{\pgfqpoint{2.208247in}{1.521584in}}{\pgfqpoint{2.203857in}{1.532183in}}{\pgfqpoint{2.196043in}{1.539997in}}%
\pgfpathcurveto{\pgfqpoint{2.188230in}{1.547810in}}{\pgfqpoint{2.177631in}{1.552200in}}{\pgfqpoint{2.166581in}{1.552200in}}%
\pgfpathcurveto{\pgfqpoint{2.155530in}{1.552200in}}{\pgfqpoint{2.144931in}{1.547810in}}{\pgfqpoint{2.137118in}{1.539997in}}%
\pgfpathcurveto{\pgfqpoint{2.129304in}{1.532183in}}{\pgfqpoint{2.124914in}{1.521584in}}{\pgfqpoint{2.124914in}{1.510534in}}%
\pgfpathcurveto{\pgfqpoint{2.124914in}{1.499484in}}{\pgfqpoint{2.129304in}{1.488885in}}{\pgfqpoint{2.137118in}{1.481071in}}%
\pgfpathcurveto{\pgfqpoint{2.144931in}{1.473257in}}{\pgfqpoint{2.155530in}{1.468867in}}{\pgfqpoint{2.166581in}{1.468867in}}%
\pgfpathclose%
\pgfusepath{stroke,fill}%
\end{pgfscope}%
\begin{pgfscope}%
\pgfpathrectangle{\pgfqpoint{0.600000in}{0.600000in}}{\pgfqpoint{3.900000in}{3.900000in}}%
\pgfusepath{clip}%
\pgfsetbuttcap%
\pgfsetroundjoin%
\definecolor{currentfill}{rgb}{0.121569,0.466667,0.705882}%
\pgfsetfillcolor{currentfill}%
\pgfsetlinewidth{1.003750pt}%
\definecolor{currentstroke}{rgb}{0.121569,0.466667,0.705882}%
\pgfsetstrokecolor{currentstroke}%
\pgfsetdash{}{0pt}%
\pgfpathmoveto{\pgfqpoint{3.003822in}{2.040345in}}%
\pgfpathcurveto{\pgfqpoint{3.014872in}{2.040345in}}{\pgfqpoint{3.025471in}{2.044735in}}{\pgfqpoint{3.033285in}{2.052548in}}%
\pgfpathcurveto{\pgfqpoint{3.041099in}{2.060362in}}{\pgfqpoint{3.045489in}{2.070961in}}{\pgfqpoint{3.045489in}{2.082011in}}%
\pgfpathcurveto{\pgfqpoint{3.045489in}{2.093061in}}{\pgfqpoint{3.041099in}{2.103660in}}{\pgfqpoint{3.033285in}{2.111474in}}%
\pgfpathcurveto{\pgfqpoint{3.025471in}{2.119288in}}{\pgfqpoint{3.014872in}{2.123678in}}{\pgfqpoint{3.003822in}{2.123678in}}%
\pgfpathcurveto{\pgfqpoint{2.992772in}{2.123678in}}{\pgfqpoint{2.982173in}{2.119288in}}{\pgfqpoint{2.974359in}{2.111474in}}%
\pgfpathcurveto{\pgfqpoint{2.966546in}{2.103660in}}{\pgfqpoint{2.962156in}{2.093061in}}{\pgfqpoint{2.962156in}{2.082011in}}%
\pgfpathcurveto{\pgfqpoint{2.962156in}{2.070961in}}{\pgfqpoint{2.966546in}{2.060362in}}{\pgfqpoint{2.974359in}{2.052548in}}%
\pgfpathcurveto{\pgfqpoint{2.982173in}{2.044735in}}{\pgfqpoint{2.992772in}{2.040345in}}{\pgfqpoint{3.003822in}{2.040345in}}%
\pgfpathclose%
\pgfusepath{stroke,fill}%
\end{pgfscope}%
\begin{pgfscope}%
\pgfpathrectangle{\pgfqpoint{0.600000in}{0.600000in}}{\pgfqpoint{3.900000in}{3.900000in}}%
\pgfusepath{clip}%
\pgfsetbuttcap%
\pgfsetroundjoin%
\definecolor{currentfill}{rgb}{0.121569,0.466667,0.705882}%
\pgfsetfillcolor{currentfill}%
\pgfsetlinewidth{1.003750pt}%
\definecolor{currentstroke}{rgb}{0.121569,0.466667,0.705882}%
\pgfsetstrokecolor{currentstroke}%
\pgfsetdash{}{0pt}%
\pgfpathmoveto{\pgfqpoint{2.285130in}{3.201892in}}%
\pgfpathcurveto{\pgfqpoint{2.296180in}{3.201892in}}{\pgfqpoint{2.306779in}{3.206282in}}{\pgfqpoint{2.314592in}{3.214096in}}%
\pgfpathcurveto{\pgfqpoint{2.322406in}{3.221909in}}{\pgfqpoint{2.326796in}{3.232509in}}{\pgfqpoint{2.326796in}{3.243559in}}%
\pgfpathcurveto{\pgfqpoint{2.326796in}{3.254609in}}{\pgfqpoint{2.322406in}{3.265208in}}{\pgfqpoint{2.314592in}{3.273021in}}%
\pgfpathcurveto{\pgfqpoint{2.306779in}{3.280835in}}{\pgfqpoint{2.296180in}{3.285225in}}{\pgfqpoint{2.285130in}{3.285225in}}%
\pgfpathcurveto{\pgfqpoint{2.274079in}{3.285225in}}{\pgfqpoint{2.263480in}{3.280835in}}{\pgfqpoint{2.255667in}{3.273021in}}%
\pgfpathcurveto{\pgfqpoint{2.247853in}{3.265208in}}{\pgfqpoint{2.243463in}{3.254609in}}{\pgfqpoint{2.243463in}{3.243559in}}%
\pgfpathcurveto{\pgfqpoint{2.243463in}{3.232509in}}{\pgfqpoint{2.247853in}{3.221909in}}{\pgfqpoint{2.255667in}{3.214096in}}%
\pgfpathcurveto{\pgfqpoint{2.263480in}{3.206282in}}{\pgfqpoint{2.274079in}{3.201892in}}{\pgfqpoint{2.285130in}{3.201892in}}%
\pgfpathclose%
\pgfusepath{stroke,fill}%
\end{pgfscope}%
\begin{pgfscope}%
\pgfpathrectangle{\pgfqpoint{0.600000in}{0.600000in}}{\pgfqpoint{3.900000in}{3.900000in}}%
\pgfusepath{clip}%
\pgfsetbuttcap%
\pgfsetroundjoin%
\definecolor{currentfill}{rgb}{0.121569,0.466667,0.705882}%
\pgfsetfillcolor{currentfill}%
\pgfsetlinewidth{1.003750pt}%
\definecolor{currentstroke}{rgb}{0.121569,0.466667,0.705882}%
\pgfsetstrokecolor{currentstroke}%
\pgfsetdash{}{0pt}%
\pgfpathmoveto{\pgfqpoint{3.140834in}{2.638112in}}%
\pgfpathcurveto{\pgfqpoint{3.151884in}{2.638112in}}{\pgfqpoint{3.162483in}{2.642502in}}{\pgfqpoint{3.170296in}{2.650316in}}%
\pgfpathcurveto{\pgfqpoint{3.178110in}{2.658130in}}{\pgfqpoint{3.182500in}{2.668729in}}{\pgfqpoint{3.182500in}{2.679779in}}%
\pgfpathcurveto{\pgfqpoint{3.182500in}{2.690829in}}{\pgfqpoint{3.178110in}{2.701428in}}{\pgfqpoint{3.170296in}{2.709241in}}%
\pgfpathcurveto{\pgfqpoint{3.162483in}{2.717055in}}{\pgfqpoint{3.151884in}{2.721445in}}{\pgfqpoint{3.140834in}{2.721445in}}%
\pgfpathcurveto{\pgfqpoint{3.129783in}{2.721445in}}{\pgfqpoint{3.119184in}{2.717055in}}{\pgfqpoint{3.111371in}{2.709241in}}%
\pgfpathcurveto{\pgfqpoint{3.103557in}{2.701428in}}{\pgfqpoint{3.099167in}{2.690829in}}{\pgfqpoint{3.099167in}{2.679779in}}%
\pgfpathcurveto{\pgfqpoint{3.099167in}{2.668729in}}{\pgfqpoint{3.103557in}{2.658130in}}{\pgfqpoint{3.111371in}{2.650316in}}%
\pgfpathcurveto{\pgfqpoint{3.119184in}{2.642502in}}{\pgfqpoint{3.129783in}{2.638112in}}{\pgfqpoint{3.140834in}{2.638112in}}%
\pgfpathclose%
\pgfusepath{stroke,fill}%
\end{pgfscope}%
\begin{pgfscope}%
\pgfpathrectangle{\pgfqpoint{0.600000in}{0.600000in}}{\pgfqpoint{3.900000in}{3.900000in}}%
\pgfusepath{clip}%
\pgfsetbuttcap%
\pgfsetroundjoin%
\definecolor{currentfill}{rgb}{0.121569,0.466667,0.705882}%
\pgfsetfillcolor{currentfill}%
\pgfsetlinewidth{1.003750pt}%
\definecolor{currentstroke}{rgb}{0.121569,0.466667,0.705882}%
\pgfsetstrokecolor{currentstroke}%
\pgfsetdash{}{0pt}%
\pgfpathmoveto{\pgfqpoint{2.668415in}{2.279240in}}%
\pgfpathcurveto{\pgfqpoint{2.679465in}{2.279240in}}{\pgfqpoint{2.690064in}{2.283630in}}{\pgfqpoint{2.697878in}{2.291443in}}%
\pgfpathcurveto{\pgfqpoint{2.705691in}{2.299257in}}{\pgfqpoint{2.710082in}{2.309856in}}{\pgfqpoint{2.710082in}{2.320906in}}%
\pgfpathcurveto{\pgfqpoint{2.710082in}{2.331956in}}{\pgfqpoint{2.705691in}{2.342555in}}{\pgfqpoint{2.697878in}{2.350369in}}%
\pgfpathcurveto{\pgfqpoint{2.690064in}{2.358183in}}{\pgfqpoint{2.679465in}{2.362573in}}{\pgfqpoint{2.668415in}{2.362573in}}%
\pgfpathcurveto{\pgfqpoint{2.657365in}{2.362573in}}{\pgfqpoint{2.646766in}{2.358183in}}{\pgfqpoint{2.638952in}{2.350369in}}%
\pgfpathcurveto{\pgfqpoint{2.631139in}{2.342555in}}{\pgfqpoint{2.626748in}{2.331956in}}{\pgfqpoint{2.626748in}{2.320906in}}%
\pgfpathcurveto{\pgfqpoint{2.626748in}{2.309856in}}{\pgfqpoint{2.631139in}{2.299257in}}{\pgfqpoint{2.638952in}{2.291443in}}%
\pgfpathcurveto{\pgfqpoint{2.646766in}{2.283630in}}{\pgfqpoint{2.657365in}{2.279240in}}{\pgfqpoint{2.668415in}{2.279240in}}%
\pgfpathclose%
\pgfusepath{stroke,fill}%
\end{pgfscope}%
\begin{pgfscope}%
\pgfpathrectangle{\pgfqpoint{0.600000in}{0.600000in}}{\pgfqpoint{3.900000in}{3.900000in}}%
\pgfusepath{clip}%
\pgfsetbuttcap%
\pgfsetroundjoin%
\definecolor{currentfill}{rgb}{0.121569,0.466667,0.705882}%
\pgfsetfillcolor{currentfill}%
\pgfsetlinewidth{1.003750pt}%
\definecolor{currentstroke}{rgb}{0.121569,0.466667,0.705882}%
\pgfsetstrokecolor{currentstroke}%
\pgfsetdash{}{0pt}%
\pgfpathmoveto{\pgfqpoint{2.493189in}{2.006709in}}%
\pgfpathcurveto{\pgfqpoint{2.504239in}{2.006709in}}{\pgfqpoint{2.514838in}{2.011099in}}{\pgfqpoint{2.522652in}{2.018913in}}%
\pgfpathcurveto{\pgfqpoint{2.530466in}{2.026726in}}{\pgfqpoint{2.534856in}{2.037325in}}{\pgfqpoint{2.534856in}{2.048376in}}%
\pgfpathcurveto{\pgfqpoint{2.534856in}{2.059426in}}{\pgfqpoint{2.530466in}{2.070025in}}{\pgfqpoint{2.522652in}{2.077838in}}%
\pgfpathcurveto{\pgfqpoint{2.514838in}{2.085652in}}{\pgfqpoint{2.504239in}{2.090042in}}{\pgfqpoint{2.493189in}{2.090042in}}%
\pgfpathcurveto{\pgfqpoint{2.482139in}{2.090042in}}{\pgfqpoint{2.471540in}{2.085652in}}{\pgfqpoint{2.463727in}{2.077838in}}%
\pgfpathcurveto{\pgfqpoint{2.455913in}{2.070025in}}{\pgfqpoint{2.451523in}{2.059426in}}{\pgfqpoint{2.451523in}{2.048376in}}%
\pgfpathcurveto{\pgfqpoint{2.451523in}{2.037325in}}{\pgfqpoint{2.455913in}{2.026726in}}{\pgfqpoint{2.463727in}{2.018913in}}%
\pgfpathcurveto{\pgfqpoint{2.471540in}{2.011099in}}{\pgfqpoint{2.482139in}{2.006709in}}{\pgfqpoint{2.493189in}{2.006709in}}%
\pgfpathclose%
\pgfusepath{stroke,fill}%
\end{pgfscope}%
\begin{pgfscope}%
\pgfpathrectangle{\pgfqpoint{0.600000in}{0.600000in}}{\pgfqpoint{3.900000in}{3.900000in}}%
\pgfusepath{clip}%
\pgfsetbuttcap%
\pgfsetroundjoin%
\definecolor{currentfill}{rgb}{0.121569,0.466667,0.705882}%
\pgfsetfillcolor{currentfill}%
\pgfsetlinewidth{1.003750pt}%
\definecolor{currentstroke}{rgb}{0.121569,0.466667,0.705882}%
\pgfsetstrokecolor{currentstroke}%
\pgfsetdash{}{0pt}%
\pgfpathmoveto{\pgfqpoint{3.021077in}{1.369769in}}%
\pgfpathcurveto{\pgfqpoint{3.032127in}{1.369769in}}{\pgfqpoint{3.042726in}{1.374160in}}{\pgfqpoint{3.050539in}{1.381973in}}%
\pgfpathcurveto{\pgfqpoint{3.058353in}{1.389787in}}{\pgfqpoint{3.062743in}{1.400386in}}{\pgfqpoint{3.062743in}{1.411436in}}%
\pgfpathcurveto{\pgfqpoint{3.062743in}{1.422486in}}{\pgfqpoint{3.058353in}{1.433085in}}{\pgfqpoint{3.050539in}{1.440899in}}%
\pgfpathcurveto{\pgfqpoint{3.042726in}{1.448713in}}{\pgfqpoint{3.032127in}{1.453103in}}{\pgfqpoint{3.021077in}{1.453103in}}%
\pgfpathcurveto{\pgfqpoint{3.010026in}{1.453103in}}{\pgfqpoint{2.999427in}{1.448713in}}{\pgfqpoint{2.991614in}{1.440899in}}%
\pgfpathcurveto{\pgfqpoint{2.983800in}{1.433085in}}{\pgfqpoint{2.979410in}{1.422486in}}{\pgfqpoint{2.979410in}{1.411436in}}%
\pgfpathcurveto{\pgfqpoint{2.979410in}{1.400386in}}{\pgfqpoint{2.983800in}{1.389787in}}{\pgfqpoint{2.991614in}{1.381973in}}%
\pgfpathcurveto{\pgfqpoint{2.999427in}{1.374160in}}{\pgfqpoint{3.010026in}{1.369769in}}{\pgfqpoint{3.021077in}{1.369769in}}%
\pgfpathclose%
\pgfusepath{stroke,fill}%
\end{pgfscope}%
\begin{pgfscope}%
\pgfpathrectangle{\pgfqpoint{0.600000in}{0.600000in}}{\pgfqpoint{3.900000in}{3.900000in}}%
\pgfusepath{clip}%
\pgfsetbuttcap%
\pgfsetroundjoin%
\definecolor{currentfill}{rgb}{0.121569,0.466667,0.705882}%
\pgfsetfillcolor{currentfill}%
\pgfsetlinewidth{1.003750pt}%
\definecolor{currentstroke}{rgb}{0.121569,0.466667,0.705882}%
\pgfsetstrokecolor{currentstroke}%
\pgfsetdash{}{0pt}%
\pgfpathmoveto{\pgfqpoint{2.591358in}{2.785156in}}%
\pgfpathcurveto{\pgfqpoint{2.602408in}{2.785156in}}{\pgfqpoint{2.613007in}{2.789546in}}{\pgfqpoint{2.620821in}{2.797360in}}%
\pgfpathcurveto{\pgfqpoint{2.628634in}{2.805174in}}{\pgfqpoint{2.633025in}{2.815773in}}{\pgfqpoint{2.633025in}{2.826823in}}%
\pgfpathcurveto{\pgfqpoint{2.633025in}{2.837873in}}{\pgfqpoint{2.628634in}{2.848472in}}{\pgfqpoint{2.620821in}{2.856286in}}%
\pgfpathcurveto{\pgfqpoint{2.613007in}{2.864099in}}{\pgfqpoint{2.602408in}{2.868490in}}{\pgfqpoint{2.591358in}{2.868490in}}%
\pgfpathcurveto{\pgfqpoint{2.580308in}{2.868490in}}{\pgfqpoint{2.569709in}{2.864099in}}{\pgfqpoint{2.561895in}{2.856286in}}%
\pgfpathcurveto{\pgfqpoint{2.554081in}{2.848472in}}{\pgfqpoint{2.549691in}{2.837873in}}{\pgfqpoint{2.549691in}{2.826823in}}%
\pgfpathcurveto{\pgfqpoint{2.549691in}{2.815773in}}{\pgfqpoint{2.554081in}{2.805174in}}{\pgfqpoint{2.561895in}{2.797360in}}%
\pgfpathcurveto{\pgfqpoint{2.569709in}{2.789546in}}{\pgfqpoint{2.580308in}{2.785156in}}{\pgfqpoint{2.591358in}{2.785156in}}%
\pgfpathclose%
\pgfusepath{stroke,fill}%
\end{pgfscope}%
\begin{pgfscope}%
\pgfpathrectangle{\pgfqpoint{0.600000in}{0.600000in}}{\pgfqpoint{3.900000in}{3.900000in}}%
\pgfusepath{clip}%
\pgfsetbuttcap%
\pgfsetroundjoin%
\definecolor{currentfill}{rgb}{0.121569,0.466667,0.705882}%
\pgfsetfillcolor{currentfill}%
\pgfsetlinewidth{1.003750pt}%
\definecolor{currentstroke}{rgb}{0.121569,0.466667,0.705882}%
\pgfsetstrokecolor{currentstroke}%
\pgfsetdash{}{0pt}%
\pgfpathmoveto{\pgfqpoint{2.589252in}{2.472380in}}%
\pgfpathcurveto{\pgfqpoint{2.600303in}{2.472380in}}{\pgfqpoint{2.610902in}{2.476770in}}{\pgfqpoint{2.618715in}{2.484584in}}%
\pgfpathcurveto{\pgfqpoint{2.626529in}{2.492398in}}{\pgfqpoint{2.630919in}{2.502997in}}{\pgfqpoint{2.630919in}{2.514047in}}%
\pgfpathcurveto{\pgfqpoint{2.630919in}{2.525097in}}{\pgfqpoint{2.626529in}{2.535696in}}{\pgfqpoint{2.618715in}{2.543510in}}%
\pgfpathcurveto{\pgfqpoint{2.610902in}{2.551323in}}{\pgfqpoint{2.600303in}{2.555714in}}{\pgfqpoint{2.589252in}{2.555714in}}%
\pgfpathcurveto{\pgfqpoint{2.578202in}{2.555714in}}{\pgfqpoint{2.567603in}{2.551323in}}{\pgfqpoint{2.559790in}{2.543510in}}%
\pgfpathcurveto{\pgfqpoint{2.551976in}{2.535696in}}{\pgfqpoint{2.547586in}{2.525097in}}{\pgfqpoint{2.547586in}{2.514047in}}%
\pgfpathcurveto{\pgfqpoint{2.547586in}{2.502997in}}{\pgfqpoint{2.551976in}{2.492398in}}{\pgfqpoint{2.559790in}{2.484584in}}%
\pgfpathcurveto{\pgfqpoint{2.567603in}{2.476770in}}{\pgfqpoint{2.578202in}{2.472380in}}{\pgfqpoint{2.589252in}{2.472380in}}%
\pgfpathclose%
\pgfusepath{stroke,fill}%
\end{pgfscope}%
\begin{pgfscope}%
\pgfpathrectangle{\pgfqpoint{0.600000in}{0.600000in}}{\pgfqpoint{3.900000in}{3.900000in}}%
\pgfusepath{clip}%
\pgfsetbuttcap%
\pgfsetroundjoin%
\definecolor{currentfill}{rgb}{0.121569,0.466667,0.705882}%
\pgfsetfillcolor{currentfill}%
\pgfsetlinewidth{1.003750pt}%
\definecolor{currentstroke}{rgb}{0.121569,0.466667,0.705882}%
\pgfsetstrokecolor{currentstroke}%
\pgfsetdash{}{0pt}%
\pgfpathmoveto{\pgfqpoint{2.095013in}{2.022645in}}%
\pgfpathcurveto{\pgfqpoint{2.106063in}{2.022645in}}{\pgfqpoint{2.116662in}{2.027035in}}{\pgfqpoint{2.124475in}{2.034849in}}%
\pgfpathcurveto{\pgfqpoint{2.132289in}{2.042663in}}{\pgfqpoint{2.136679in}{2.053262in}}{\pgfqpoint{2.136679in}{2.064312in}}%
\pgfpathcurveto{\pgfqpoint{2.136679in}{2.075362in}}{\pgfqpoint{2.132289in}{2.085961in}}{\pgfqpoint{2.124475in}{2.093774in}}%
\pgfpathcurveto{\pgfqpoint{2.116662in}{2.101588in}}{\pgfqpoint{2.106063in}{2.105978in}}{\pgfqpoint{2.095013in}{2.105978in}}%
\pgfpathcurveto{\pgfqpoint{2.083962in}{2.105978in}}{\pgfqpoint{2.073363in}{2.101588in}}{\pgfqpoint{2.065550in}{2.093774in}}%
\pgfpathcurveto{\pgfqpoint{2.057736in}{2.085961in}}{\pgfqpoint{2.053346in}{2.075362in}}{\pgfqpoint{2.053346in}{2.064312in}}%
\pgfpathcurveto{\pgfqpoint{2.053346in}{2.053262in}}{\pgfqpoint{2.057736in}{2.042663in}}{\pgfqpoint{2.065550in}{2.034849in}}%
\pgfpathcurveto{\pgfqpoint{2.073363in}{2.027035in}}{\pgfqpoint{2.083962in}{2.022645in}}{\pgfqpoint{2.095013in}{2.022645in}}%
\pgfpathclose%
\pgfusepath{stroke,fill}%
\end{pgfscope}%
\begin{pgfscope}%
\pgfpathrectangle{\pgfqpoint{0.600000in}{0.600000in}}{\pgfqpoint{3.900000in}{3.900000in}}%
\pgfusepath{clip}%
\pgfsetbuttcap%
\pgfsetroundjoin%
\definecolor{currentfill}{rgb}{0.121569,0.466667,0.705882}%
\pgfsetfillcolor{currentfill}%
\pgfsetlinewidth{1.003750pt}%
\definecolor{currentstroke}{rgb}{0.121569,0.466667,0.705882}%
\pgfsetstrokecolor{currentstroke}%
\pgfsetdash{}{0pt}%
\pgfpathmoveto{\pgfqpoint{2.947414in}{2.448287in}}%
\pgfpathcurveto{\pgfqpoint{2.958464in}{2.448287in}}{\pgfqpoint{2.969063in}{2.452678in}}{\pgfqpoint{2.976877in}{2.460491in}}%
\pgfpathcurveto{\pgfqpoint{2.984690in}{2.468305in}}{\pgfqpoint{2.989080in}{2.478904in}}{\pgfqpoint{2.989080in}{2.489954in}}%
\pgfpathcurveto{\pgfqpoint{2.989080in}{2.501004in}}{\pgfqpoint{2.984690in}{2.511603in}}{\pgfqpoint{2.976877in}{2.519417in}}%
\pgfpathcurveto{\pgfqpoint{2.969063in}{2.527230in}}{\pgfqpoint{2.958464in}{2.531621in}}{\pgfqpoint{2.947414in}{2.531621in}}%
\pgfpathcurveto{\pgfqpoint{2.936364in}{2.531621in}}{\pgfqpoint{2.925765in}{2.527230in}}{\pgfqpoint{2.917951in}{2.519417in}}%
\pgfpathcurveto{\pgfqpoint{2.910137in}{2.511603in}}{\pgfqpoint{2.905747in}{2.501004in}}{\pgfqpoint{2.905747in}{2.489954in}}%
\pgfpathcurveto{\pgfqpoint{2.905747in}{2.478904in}}{\pgfqpoint{2.910137in}{2.468305in}}{\pgfqpoint{2.917951in}{2.460491in}}%
\pgfpathcurveto{\pgfqpoint{2.925765in}{2.452678in}}{\pgfqpoint{2.936364in}{2.448287in}}{\pgfqpoint{2.947414in}{2.448287in}}%
\pgfpathclose%
\pgfusepath{stroke,fill}%
\end{pgfscope}%
\begin{pgfscope}%
\pgfpathrectangle{\pgfqpoint{0.600000in}{0.600000in}}{\pgfqpoint{3.900000in}{3.900000in}}%
\pgfusepath{clip}%
\pgfsetbuttcap%
\pgfsetroundjoin%
\definecolor{currentfill}{rgb}{0.121569,0.466667,0.705882}%
\pgfsetfillcolor{currentfill}%
\pgfsetlinewidth{1.003750pt}%
\definecolor{currentstroke}{rgb}{0.121569,0.466667,0.705882}%
\pgfsetstrokecolor{currentstroke}%
\pgfsetdash{}{0pt}%
\pgfpathmoveto{\pgfqpoint{2.702167in}{2.312574in}}%
\pgfpathcurveto{\pgfqpoint{2.713218in}{2.312574in}}{\pgfqpoint{2.723817in}{2.316965in}}{\pgfqpoint{2.731630in}{2.324778in}}%
\pgfpathcurveto{\pgfqpoint{2.739444in}{2.332592in}}{\pgfqpoint{2.743834in}{2.343191in}}{\pgfqpoint{2.743834in}{2.354241in}}%
\pgfpathcurveto{\pgfqpoint{2.743834in}{2.365291in}}{\pgfqpoint{2.739444in}{2.375890in}}{\pgfqpoint{2.731630in}{2.383704in}}%
\pgfpathcurveto{\pgfqpoint{2.723817in}{2.391517in}}{\pgfqpoint{2.713218in}{2.395908in}}{\pgfqpoint{2.702167in}{2.395908in}}%
\pgfpathcurveto{\pgfqpoint{2.691117in}{2.395908in}}{\pgfqpoint{2.680518in}{2.391517in}}{\pgfqpoint{2.672705in}{2.383704in}}%
\pgfpathcurveto{\pgfqpoint{2.664891in}{2.375890in}}{\pgfqpoint{2.660501in}{2.365291in}}{\pgfqpoint{2.660501in}{2.354241in}}%
\pgfpathcurveto{\pgfqpoint{2.660501in}{2.343191in}}{\pgfqpoint{2.664891in}{2.332592in}}{\pgfqpoint{2.672705in}{2.324778in}}%
\pgfpathcurveto{\pgfqpoint{2.680518in}{2.316965in}}{\pgfqpoint{2.691117in}{2.312574in}}{\pgfqpoint{2.702167in}{2.312574in}}%
\pgfpathclose%
\pgfusepath{stroke,fill}%
\end{pgfscope}%
\begin{pgfscope}%
\pgfpathrectangle{\pgfqpoint{0.600000in}{0.600000in}}{\pgfqpoint{3.900000in}{3.900000in}}%
\pgfusepath{clip}%
\pgfsetbuttcap%
\pgfsetroundjoin%
\definecolor{currentfill}{rgb}{0.121569,0.466667,0.705882}%
\pgfsetfillcolor{currentfill}%
\pgfsetlinewidth{1.003750pt}%
\definecolor{currentstroke}{rgb}{0.121569,0.466667,0.705882}%
\pgfsetstrokecolor{currentstroke}%
\pgfsetdash{}{0pt}%
\pgfpathmoveto{\pgfqpoint{2.363906in}{1.962986in}}%
\pgfpathcurveto{\pgfqpoint{2.374956in}{1.962986in}}{\pgfqpoint{2.385555in}{1.967376in}}{\pgfqpoint{2.393368in}{1.975189in}}%
\pgfpathcurveto{\pgfqpoint{2.401182in}{1.983003in}}{\pgfqpoint{2.405572in}{1.993602in}}{\pgfqpoint{2.405572in}{2.004652in}}%
\pgfpathcurveto{\pgfqpoint{2.405572in}{2.015702in}}{\pgfqpoint{2.401182in}{2.026301in}}{\pgfqpoint{2.393368in}{2.034115in}}%
\pgfpathcurveto{\pgfqpoint{2.385555in}{2.041929in}}{\pgfqpoint{2.374956in}{2.046319in}}{\pgfqpoint{2.363906in}{2.046319in}}%
\pgfpathcurveto{\pgfqpoint{2.352855in}{2.046319in}}{\pgfqpoint{2.342256in}{2.041929in}}{\pgfqpoint{2.334443in}{2.034115in}}%
\pgfpathcurveto{\pgfqpoint{2.326629in}{2.026301in}}{\pgfqpoint{2.322239in}{2.015702in}}{\pgfqpoint{2.322239in}{2.004652in}}%
\pgfpathcurveto{\pgfqpoint{2.322239in}{1.993602in}}{\pgfqpoint{2.326629in}{1.983003in}}{\pgfqpoint{2.334443in}{1.975189in}}%
\pgfpathcurveto{\pgfqpoint{2.342256in}{1.967376in}}{\pgfqpoint{2.352855in}{1.962986in}}{\pgfqpoint{2.363906in}{1.962986in}}%
\pgfpathclose%
\pgfusepath{stroke,fill}%
\end{pgfscope}%
\begin{pgfscope}%
\pgfpathrectangle{\pgfqpoint{0.600000in}{0.600000in}}{\pgfqpoint{3.900000in}{3.900000in}}%
\pgfusepath{clip}%
\pgfsetbuttcap%
\pgfsetroundjoin%
\definecolor{currentfill}{rgb}{0.121569,0.466667,0.705882}%
\pgfsetfillcolor{currentfill}%
\pgfsetlinewidth{1.003750pt}%
\definecolor{currentstroke}{rgb}{0.121569,0.466667,0.705882}%
\pgfsetstrokecolor{currentstroke}%
\pgfsetdash{}{0pt}%
\pgfpathmoveto{\pgfqpoint{2.592126in}{3.049524in}}%
\pgfpathcurveto{\pgfqpoint{2.603176in}{3.049524in}}{\pgfqpoint{2.613775in}{3.053914in}}{\pgfqpoint{2.621588in}{3.061728in}}%
\pgfpathcurveto{\pgfqpoint{2.629402in}{3.069541in}}{\pgfqpoint{2.633792in}{3.080140in}}{\pgfqpoint{2.633792in}{3.091190in}}%
\pgfpathcurveto{\pgfqpoint{2.633792in}{3.102241in}}{\pgfqpoint{2.629402in}{3.112840in}}{\pgfqpoint{2.621588in}{3.120653in}}%
\pgfpathcurveto{\pgfqpoint{2.613775in}{3.128467in}}{\pgfqpoint{2.603176in}{3.132857in}}{\pgfqpoint{2.592126in}{3.132857in}}%
\pgfpathcurveto{\pgfqpoint{2.581075in}{3.132857in}}{\pgfqpoint{2.570476in}{3.128467in}}{\pgfqpoint{2.562663in}{3.120653in}}%
\pgfpathcurveto{\pgfqpoint{2.554849in}{3.112840in}}{\pgfqpoint{2.550459in}{3.102241in}}{\pgfqpoint{2.550459in}{3.091190in}}%
\pgfpathcurveto{\pgfqpoint{2.550459in}{3.080140in}}{\pgfqpoint{2.554849in}{3.069541in}}{\pgfqpoint{2.562663in}{3.061728in}}%
\pgfpathcurveto{\pgfqpoint{2.570476in}{3.053914in}}{\pgfqpoint{2.581075in}{3.049524in}}{\pgfqpoint{2.592126in}{3.049524in}}%
\pgfpathclose%
\pgfusepath{stroke,fill}%
\end{pgfscope}%
\begin{pgfscope}%
\pgfpathrectangle{\pgfqpoint{0.600000in}{0.600000in}}{\pgfqpoint{3.900000in}{3.900000in}}%
\pgfusepath{clip}%
\pgfsetbuttcap%
\pgfsetroundjoin%
\definecolor{currentfill}{rgb}{0.121569,0.466667,0.705882}%
\pgfsetfillcolor{currentfill}%
\pgfsetlinewidth{1.003750pt}%
\definecolor{currentstroke}{rgb}{0.121569,0.466667,0.705882}%
\pgfsetstrokecolor{currentstroke}%
\pgfsetdash{}{0pt}%
\pgfpathmoveto{\pgfqpoint{2.306330in}{1.679248in}}%
\pgfpathcurveto{\pgfqpoint{2.317380in}{1.679248in}}{\pgfqpoint{2.327979in}{1.683638in}}{\pgfqpoint{2.335792in}{1.691451in}}%
\pgfpathcurveto{\pgfqpoint{2.343606in}{1.699265in}}{\pgfqpoint{2.347996in}{1.709864in}}{\pgfqpoint{2.347996in}{1.720914in}}%
\pgfpathcurveto{\pgfqpoint{2.347996in}{1.731964in}}{\pgfqpoint{2.343606in}{1.742563in}}{\pgfqpoint{2.335792in}{1.750377in}}%
\pgfpathcurveto{\pgfqpoint{2.327979in}{1.758191in}}{\pgfqpoint{2.317380in}{1.762581in}}{\pgfqpoint{2.306330in}{1.762581in}}%
\pgfpathcurveto{\pgfqpoint{2.295279in}{1.762581in}}{\pgfqpoint{2.284680in}{1.758191in}}{\pgfqpoint{2.276867in}{1.750377in}}%
\pgfpathcurveto{\pgfqpoint{2.269053in}{1.742563in}}{\pgfqpoint{2.264663in}{1.731964in}}{\pgfqpoint{2.264663in}{1.720914in}}%
\pgfpathcurveto{\pgfqpoint{2.264663in}{1.709864in}}{\pgfqpoint{2.269053in}{1.699265in}}{\pgfqpoint{2.276867in}{1.691451in}}%
\pgfpathcurveto{\pgfqpoint{2.284680in}{1.683638in}}{\pgfqpoint{2.295279in}{1.679248in}}{\pgfqpoint{2.306330in}{1.679248in}}%
\pgfpathclose%
\pgfusepath{stroke,fill}%
\end{pgfscope}%
\begin{pgfscope}%
\pgfpathrectangle{\pgfqpoint{0.600000in}{0.600000in}}{\pgfqpoint{3.900000in}{3.900000in}}%
\pgfusepath{clip}%
\pgfsetbuttcap%
\pgfsetroundjoin%
\definecolor{currentfill}{rgb}{0.121569,0.466667,0.705882}%
\pgfsetfillcolor{currentfill}%
\pgfsetlinewidth{1.003750pt}%
\definecolor{currentstroke}{rgb}{0.121569,0.466667,0.705882}%
\pgfsetstrokecolor{currentstroke}%
\pgfsetdash{}{0pt}%
\pgfpathmoveto{\pgfqpoint{2.171717in}{3.056445in}}%
\pgfpathcurveto{\pgfqpoint{2.182767in}{3.056445in}}{\pgfqpoint{2.193366in}{3.060836in}}{\pgfqpoint{2.201180in}{3.068649in}}%
\pgfpathcurveto{\pgfqpoint{2.208993in}{3.076463in}}{\pgfqpoint{2.213384in}{3.087062in}}{\pgfqpoint{2.213384in}{3.098112in}}%
\pgfpathcurveto{\pgfqpoint{2.213384in}{3.109162in}}{\pgfqpoint{2.208993in}{3.119761in}}{\pgfqpoint{2.201180in}{3.127575in}}%
\pgfpathcurveto{\pgfqpoint{2.193366in}{3.135389in}}{\pgfqpoint{2.182767in}{3.139779in}}{\pgfqpoint{2.171717in}{3.139779in}}%
\pgfpathcurveto{\pgfqpoint{2.160667in}{3.139779in}}{\pgfqpoint{2.150068in}{3.135389in}}{\pgfqpoint{2.142254in}{3.127575in}}%
\pgfpathcurveto{\pgfqpoint{2.134440in}{3.119761in}}{\pgfqpoint{2.130050in}{3.109162in}}{\pgfqpoint{2.130050in}{3.098112in}}%
\pgfpathcurveto{\pgfqpoint{2.130050in}{3.087062in}}{\pgfqpoint{2.134440in}{3.076463in}}{\pgfqpoint{2.142254in}{3.068649in}}%
\pgfpathcurveto{\pgfqpoint{2.150068in}{3.060836in}}{\pgfqpoint{2.160667in}{3.056445in}}{\pgfqpoint{2.171717in}{3.056445in}}%
\pgfpathclose%
\pgfusepath{stroke,fill}%
\end{pgfscope}%
\begin{pgfscope}%
\pgfpathrectangle{\pgfqpoint{0.600000in}{0.600000in}}{\pgfqpoint{3.900000in}{3.900000in}}%
\pgfusepath{clip}%
\pgfsetbuttcap%
\pgfsetroundjoin%
\definecolor{currentfill}{rgb}{0.121569,0.466667,0.705882}%
\pgfsetfillcolor{currentfill}%
\pgfsetlinewidth{1.003750pt}%
\definecolor{currentstroke}{rgb}{0.121569,0.466667,0.705882}%
\pgfsetstrokecolor{currentstroke}%
\pgfsetdash{}{0pt}%
\pgfpathmoveto{\pgfqpoint{2.367046in}{2.634638in}}%
\pgfpathcurveto{\pgfqpoint{2.378096in}{2.634638in}}{\pgfqpoint{2.388695in}{2.639028in}}{\pgfqpoint{2.396509in}{2.646842in}}%
\pgfpathcurveto{\pgfqpoint{2.404322in}{2.654655in}}{\pgfqpoint{2.408713in}{2.665254in}}{\pgfqpoint{2.408713in}{2.676304in}}%
\pgfpathcurveto{\pgfqpoint{2.408713in}{2.687354in}}{\pgfqpoint{2.404322in}{2.697954in}}{\pgfqpoint{2.396509in}{2.705767in}}%
\pgfpathcurveto{\pgfqpoint{2.388695in}{2.713581in}}{\pgfqpoint{2.378096in}{2.717971in}}{\pgfqpoint{2.367046in}{2.717971in}}%
\pgfpathcurveto{\pgfqpoint{2.355996in}{2.717971in}}{\pgfqpoint{2.345397in}{2.713581in}}{\pgfqpoint{2.337583in}{2.705767in}}%
\pgfpathcurveto{\pgfqpoint{2.329770in}{2.697954in}}{\pgfqpoint{2.325379in}{2.687354in}}{\pgfqpoint{2.325379in}{2.676304in}}%
\pgfpathcurveto{\pgfqpoint{2.325379in}{2.665254in}}{\pgfqpoint{2.329770in}{2.654655in}}{\pgfqpoint{2.337583in}{2.646842in}}%
\pgfpathcurveto{\pgfqpoint{2.345397in}{2.639028in}}{\pgfqpoint{2.355996in}{2.634638in}}{\pgfqpoint{2.367046in}{2.634638in}}%
\pgfpathclose%
\pgfusepath{stroke,fill}%
\end{pgfscope}%
\begin{pgfscope}%
\pgfpathrectangle{\pgfqpoint{0.600000in}{0.600000in}}{\pgfqpoint{3.900000in}{3.900000in}}%
\pgfusepath{clip}%
\pgfsetbuttcap%
\pgfsetroundjoin%
\definecolor{currentfill}{rgb}{0.121569,0.466667,0.705882}%
\pgfsetfillcolor{currentfill}%
\pgfsetlinewidth{1.003750pt}%
\definecolor{currentstroke}{rgb}{0.121569,0.466667,0.705882}%
\pgfsetstrokecolor{currentstroke}%
\pgfsetdash{}{0pt}%
\pgfpathmoveto{\pgfqpoint{2.913717in}{2.009611in}}%
\pgfpathcurveto{\pgfqpoint{2.924767in}{2.009611in}}{\pgfqpoint{2.935366in}{2.014001in}}{\pgfqpoint{2.943180in}{2.021815in}}%
\pgfpathcurveto{\pgfqpoint{2.950994in}{2.029629in}}{\pgfqpoint{2.955384in}{2.040228in}}{\pgfqpoint{2.955384in}{2.051278in}}%
\pgfpathcurveto{\pgfqpoint{2.955384in}{2.062328in}}{\pgfqpoint{2.950994in}{2.072927in}}{\pgfqpoint{2.943180in}{2.080741in}}%
\pgfpathcurveto{\pgfqpoint{2.935366in}{2.088554in}}{\pgfqpoint{2.924767in}{2.092944in}}{\pgfqpoint{2.913717in}{2.092944in}}%
\pgfpathcurveto{\pgfqpoint{2.902667in}{2.092944in}}{\pgfqpoint{2.892068in}{2.088554in}}{\pgfqpoint{2.884254in}{2.080741in}}%
\pgfpathcurveto{\pgfqpoint{2.876441in}{2.072927in}}{\pgfqpoint{2.872051in}{2.062328in}}{\pgfqpoint{2.872051in}{2.051278in}}%
\pgfpathcurveto{\pgfqpoint{2.872051in}{2.040228in}}{\pgfqpoint{2.876441in}{2.029629in}}{\pgfqpoint{2.884254in}{2.021815in}}%
\pgfpathcurveto{\pgfqpoint{2.892068in}{2.014001in}}{\pgfqpoint{2.902667in}{2.009611in}}{\pgfqpoint{2.913717in}{2.009611in}}%
\pgfpathclose%
\pgfusepath{stroke,fill}%
\end{pgfscope}%
\begin{pgfscope}%
\pgfpathrectangle{\pgfqpoint{0.600000in}{0.600000in}}{\pgfqpoint{3.900000in}{3.900000in}}%
\pgfusepath{clip}%
\pgfsetbuttcap%
\pgfsetroundjoin%
\definecolor{currentfill}{rgb}{0.121569,0.466667,0.705882}%
\pgfsetfillcolor{currentfill}%
\pgfsetlinewidth{1.003750pt}%
\definecolor{currentstroke}{rgb}{0.121569,0.466667,0.705882}%
\pgfsetstrokecolor{currentstroke}%
\pgfsetdash{}{0pt}%
\pgfpathmoveto{\pgfqpoint{3.299013in}{2.255738in}}%
\pgfpathcurveto{\pgfqpoint{3.310063in}{2.255738in}}{\pgfqpoint{3.320662in}{2.260128in}}{\pgfqpoint{3.328475in}{2.267942in}}%
\pgfpathcurveto{\pgfqpoint{3.336289in}{2.275755in}}{\pgfqpoint{3.340679in}{2.286354in}}{\pgfqpoint{3.340679in}{2.297404in}}%
\pgfpathcurveto{\pgfqpoint{3.340679in}{2.308454in}}{\pgfqpoint{3.336289in}{2.319053in}}{\pgfqpoint{3.328475in}{2.326867in}}%
\pgfpathcurveto{\pgfqpoint{3.320662in}{2.334681in}}{\pgfqpoint{3.310063in}{2.339071in}}{\pgfqpoint{3.299013in}{2.339071in}}%
\pgfpathcurveto{\pgfqpoint{3.287962in}{2.339071in}}{\pgfqpoint{3.277363in}{2.334681in}}{\pgfqpoint{3.269550in}{2.326867in}}%
\pgfpathcurveto{\pgfqpoint{3.261736in}{2.319053in}}{\pgfqpoint{3.257346in}{2.308454in}}{\pgfqpoint{3.257346in}{2.297404in}}%
\pgfpathcurveto{\pgfqpoint{3.257346in}{2.286354in}}{\pgfqpoint{3.261736in}{2.275755in}}{\pgfqpoint{3.269550in}{2.267942in}}%
\pgfpathcurveto{\pgfqpoint{3.277363in}{2.260128in}}{\pgfqpoint{3.287962in}{2.255738in}}{\pgfqpoint{3.299013in}{2.255738in}}%
\pgfpathclose%
\pgfusepath{stroke,fill}%
\end{pgfscope}%
\begin{pgfscope}%
\pgfpathrectangle{\pgfqpoint{0.600000in}{0.600000in}}{\pgfqpoint{3.900000in}{3.900000in}}%
\pgfusepath{clip}%
\pgfsetbuttcap%
\pgfsetroundjoin%
\definecolor{currentfill}{rgb}{0.121569,0.466667,0.705882}%
\pgfsetfillcolor{currentfill}%
\pgfsetlinewidth{1.003750pt}%
\definecolor{currentstroke}{rgb}{0.121569,0.466667,0.705882}%
\pgfsetstrokecolor{currentstroke}%
\pgfsetdash{}{0pt}%
\pgfpathmoveto{\pgfqpoint{2.653114in}{2.806227in}}%
\pgfpathcurveto{\pgfqpoint{2.664164in}{2.806227in}}{\pgfqpoint{2.674763in}{2.810617in}}{\pgfqpoint{2.682577in}{2.818431in}}%
\pgfpathcurveto{\pgfqpoint{2.690391in}{2.826245in}}{\pgfqpoint{2.694781in}{2.836844in}}{\pgfqpoint{2.694781in}{2.847894in}}%
\pgfpathcurveto{\pgfqpoint{2.694781in}{2.858944in}}{\pgfqpoint{2.690391in}{2.869543in}}{\pgfqpoint{2.682577in}{2.877357in}}%
\pgfpathcurveto{\pgfqpoint{2.674763in}{2.885170in}}{\pgfqpoint{2.664164in}{2.889561in}}{\pgfqpoint{2.653114in}{2.889561in}}%
\pgfpathcurveto{\pgfqpoint{2.642064in}{2.889561in}}{\pgfqpoint{2.631465in}{2.885170in}}{\pgfqpoint{2.623651in}{2.877357in}}%
\pgfpathcurveto{\pgfqpoint{2.615838in}{2.869543in}}{\pgfqpoint{2.611448in}{2.858944in}}{\pgfqpoint{2.611448in}{2.847894in}}%
\pgfpathcurveto{\pgfqpoint{2.611448in}{2.836844in}}{\pgfqpoint{2.615838in}{2.826245in}}{\pgfqpoint{2.623651in}{2.818431in}}%
\pgfpathcurveto{\pgfqpoint{2.631465in}{2.810617in}}{\pgfqpoint{2.642064in}{2.806227in}}{\pgfqpoint{2.653114in}{2.806227in}}%
\pgfpathclose%
\pgfusepath{stroke,fill}%
\end{pgfscope}%
\begin{pgfscope}%
\pgfpathrectangle{\pgfqpoint{0.600000in}{0.600000in}}{\pgfqpoint{3.900000in}{3.900000in}}%
\pgfusepath{clip}%
\pgfsetbuttcap%
\pgfsetroundjoin%
\definecolor{currentfill}{rgb}{0.121569,0.466667,0.705882}%
\pgfsetfillcolor{currentfill}%
\pgfsetlinewidth{1.003750pt}%
\definecolor{currentstroke}{rgb}{0.121569,0.466667,0.705882}%
\pgfsetstrokecolor{currentstroke}%
\pgfsetdash{}{0pt}%
\pgfpathmoveto{\pgfqpoint{2.563116in}{3.524800in}}%
\pgfpathcurveto{\pgfqpoint{2.574166in}{3.524800in}}{\pgfqpoint{2.584765in}{3.529190in}}{\pgfqpoint{2.592578in}{3.537004in}}%
\pgfpathcurveto{\pgfqpoint{2.600392in}{3.544818in}}{\pgfqpoint{2.604782in}{3.555417in}}{\pgfqpoint{2.604782in}{3.566467in}}%
\pgfpathcurveto{\pgfqpoint{2.604782in}{3.577517in}}{\pgfqpoint{2.600392in}{3.588116in}}{\pgfqpoint{2.592578in}{3.595930in}}%
\pgfpathcurveto{\pgfqpoint{2.584765in}{3.603743in}}{\pgfqpoint{2.574166in}{3.608133in}}{\pgfqpoint{2.563116in}{3.608133in}}%
\pgfpathcurveto{\pgfqpoint{2.552065in}{3.608133in}}{\pgfqpoint{2.541466in}{3.603743in}}{\pgfqpoint{2.533653in}{3.595930in}}%
\pgfpathcurveto{\pgfqpoint{2.525839in}{3.588116in}}{\pgfqpoint{2.521449in}{3.577517in}}{\pgfqpoint{2.521449in}{3.566467in}}%
\pgfpathcurveto{\pgfqpoint{2.521449in}{3.555417in}}{\pgfqpoint{2.525839in}{3.544818in}}{\pgfqpoint{2.533653in}{3.537004in}}%
\pgfpathcurveto{\pgfqpoint{2.541466in}{3.529190in}}{\pgfqpoint{2.552065in}{3.524800in}}{\pgfqpoint{2.563116in}{3.524800in}}%
\pgfpathclose%
\pgfusepath{stroke,fill}%
\end{pgfscope}%
\begin{pgfscope}%
\pgfpathrectangle{\pgfqpoint{0.600000in}{0.600000in}}{\pgfqpoint{3.900000in}{3.900000in}}%
\pgfusepath{clip}%
\pgfsetbuttcap%
\pgfsetroundjoin%
\definecolor{currentfill}{rgb}{0.121569,0.466667,0.705882}%
\pgfsetfillcolor{currentfill}%
\pgfsetlinewidth{1.003750pt}%
\definecolor{currentstroke}{rgb}{0.121569,0.466667,0.705882}%
\pgfsetstrokecolor{currentstroke}%
\pgfsetdash{}{0pt}%
\pgfpathmoveto{\pgfqpoint{1.937203in}{3.546785in}}%
\pgfpathcurveto{\pgfqpoint{1.948253in}{3.546785in}}{\pgfqpoint{1.958852in}{3.551176in}}{\pgfqpoint{1.966666in}{3.558989in}}%
\pgfpathcurveto{\pgfqpoint{1.974479in}{3.566803in}}{\pgfqpoint{1.978869in}{3.577402in}}{\pgfqpoint{1.978869in}{3.588452in}}%
\pgfpathcurveto{\pgfqpoint{1.978869in}{3.599502in}}{\pgfqpoint{1.974479in}{3.610101in}}{\pgfqpoint{1.966666in}{3.617915in}}%
\pgfpathcurveto{\pgfqpoint{1.958852in}{3.625729in}}{\pgfqpoint{1.948253in}{3.630119in}}{\pgfqpoint{1.937203in}{3.630119in}}%
\pgfpathcurveto{\pgfqpoint{1.926153in}{3.630119in}}{\pgfqpoint{1.915554in}{3.625729in}}{\pgfqpoint{1.907740in}{3.617915in}}%
\pgfpathcurveto{\pgfqpoint{1.899926in}{3.610101in}}{\pgfqpoint{1.895536in}{3.599502in}}{\pgfqpoint{1.895536in}{3.588452in}}%
\pgfpathcurveto{\pgfqpoint{1.895536in}{3.577402in}}{\pgfqpoint{1.899926in}{3.566803in}}{\pgfqpoint{1.907740in}{3.558989in}}%
\pgfpathcurveto{\pgfqpoint{1.915554in}{3.551176in}}{\pgfqpoint{1.926153in}{3.546785in}}{\pgfqpoint{1.937203in}{3.546785in}}%
\pgfpathclose%
\pgfusepath{stroke,fill}%
\end{pgfscope}%
\begin{pgfscope}%
\pgfpathrectangle{\pgfqpoint{0.600000in}{0.600000in}}{\pgfqpoint{3.900000in}{3.900000in}}%
\pgfusepath{clip}%
\pgfsetbuttcap%
\pgfsetroundjoin%
\definecolor{currentfill}{rgb}{0.121569,0.466667,0.705882}%
\pgfsetfillcolor{currentfill}%
\pgfsetlinewidth{1.003750pt}%
\definecolor{currentstroke}{rgb}{0.121569,0.466667,0.705882}%
\pgfsetstrokecolor{currentstroke}%
\pgfsetdash{}{0pt}%
\pgfpathmoveto{\pgfqpoint{3.108154in}{3.475217in}}%
\pgfpathcurveto{\pgfqpoint{3.119204in}{3.475217in}}{\pgfqpoint{3.129803in}{3.479607in}}{\pgfqpoint{3.137616in}{3.487421in}}%
\pgfpathcurveto{\pgfqpoint{3.145430in}{3.495234in}}{\pgfqpoint{3.149820in}{3.505833in}}{\pgfqpoint{3.149820in}{3.516884in}}%
\pgfpathcurveto{\pgfqpoint{3.149820in}{3.527934in}}{\pgfqpoint{3.145430in}{3.538533in}}{\pgfqpoint{3.137616in}{3.546346in}}%
\pgfpathcurveto{\pgfqpoint{3.129803in}{3.554160in}}{\pgfqpoint{3.119204in}{3.558550in}}{\pgfqpoint{3.108154in}{3.558550in}}%
\pgfpathcurveto{\pgfqpoint{3.097103in}{3.558550in}}{\pgfqpoint{3.086504in}{3.554160in}}{\pgfqpoint{3.078691in}{3.546346in}}%
\pgfpathcurveto{\pgfqpoint{3.070877in}{3.538533in}}{\pgfqpoint{3.066487in}{3.527934in}}{\pgfqpoint{3.066487in}{3.516884in}}%
\pgfpathcurveto{\pgfqpoint{3.066487in}{3.505833in}}{\pgfqpoint{3.070877in}{3.495234in}}{\pgfqpoint{3.078691in}{3.487421in}}%
\pgfpathcurveto{\pgfqpoint{3.086504in}{3.479607in}}{\pgfqpoint{3.097103in}{3.475217in}}{\pgfqpoint{3.108154in}{3.475217in}}%
\pgfpathclose%
\pgfusepath{stroke,fill}%
\end{pgfscope}%
\begin{pgfscope}%
\pgfpathrectangle{\pgfqpoint{0.600000in}{0.600000in}}{\pgfqpoint{3.900000in}{3.900000in}}%
\pgfusepath{clip}%
\pgfsetbuttcap%
\pgfsetroundjoin%
\definecolor{currentfill}{rgb}{0.121569,0.466667,0.705882}%
\pgfsetfillcolor{currentfill}%
\pgfsetlinewidth{1.003750pt}%
\definecolor{currentstroke}{rgb}{0.121569,0.466667,0.705882}%
\pgfsetstrokecolor{currentstroke}%
\pgfsetdash{}{0pt}%
\pgfpathmoveto{\pgfqpoint{2.238904in}{3.633181in}}%
\pgfpathcurveto{\pgfqpoint{2.249954in}{3.633181in}}{\pgfqpoint{2.260553in}{3.637571in}}{\pgfqpoint{2.268366in}{3.645384in}}%
\pgfpathcurveto{\pgfqpoint{2.276180in}{3.653198in}}{\pgfqpoint{2.280570in}{3.663797in}}{\pgfqpoint{2.280570in}{3.674847in}}%
\pgfpathcurveto{\pgfqpoint{2.280570in}{3.685897in}}{\pgfqpoint{2.276180in}{3.696496in}}{\pgfqpoint{2.268366in}{3.704310in}}%
\pgfpathcurveto{\pgfqpoint{2.260553in}{3.712124in}}{\pgfqpoint{2.249954in}{3.716514in}}{\pgfqpoint{2.238904in}{3.716514in}}%
\pgfpathcurveto{\pgfqpoint{2.227854in}{3.716514in}}{\pgfqpoint{2.217255in}{3.712124in}}{\pgfqpoint{2.209441in}{3.704310in}}%
\pgfpathcurveto{\pgfqpoint{2.201627in}{3.696496in}}{\pgfqpoint{2.197237in}{3.685897in}}{\pgfqpoint{2.197237in}{3.674847in}}%
\pgfpathcurveto{\pgfqpoint{2.197237in}{3.663797in}}{\pgfqpoint{2.201627in}{3.653198in}}{\pgfqpoint{2.209441in}{3.645384in}}%
\pgfpathcurveto{\pgfqpoint{2.217255in}{3.637571in}}{\pgfqpoint{2.227854in}{3.633181in}}{\pgfqpoint{2.238904in}{3.633181in}}%
\pgfpathclose%
\pgfusepath{stroke,fill}%
\end{pgfscope}%
\begin{pgfscope}%
\pgfpathrectangle{\pgfqpoint{0.600000in}{0.600000in}}{\pgfqpoint{3.900000in}{3.900000in}}%
\pgfusepath{clip}%
\pgfsetbuttcap%
\pgfsetroundjoin%
\definecolor{currentfill}{rgb}{0.121569,0.466667,0.705882}%
\pgfsetfillcolor{currentfill}%
\pgfsetlinewidth{1.003750pt}%
\definecolor{currentstroke}{rgb}{0.121569,0.466667,0.705882}%
\pgfsetstrokecolor{currentstroke}%
\pgfsetdash{}{0pt}%
\pgfpathmoveto{\pgfqpoint{1.967281in}{2.420969in}}%
\pgfpathcurveto{\pgfqpoint{1.978331in}{2.420969in}}{\pgfqpoint{1.988930in}{2.425360in}}{\pgfqpoint{1.996744in}{2.433173in}}%
\pgfpathcurveto{\pgfqpoint{2.004558in}{2.440987in}}{\pgfqpoint{2.008948in}{2.451586in}}{\pgfqpoint{2.008948in}{2.462636in}}%
\pgfpathcurveto{\pgfqpoint{2.008948in}{2.473686in}}{\pgfqpoint{2.004558in}{2.484285in}}{\pgfqpoint{1.996744in}{2.492099in}}%
\pgfpathcurveto{\pgfqpoint{1.988930in}{2.499913in}}{\pgfqpoint{1.978331in}{2.504303in}}{\pgfqpoint{1.967281in}{2.504303in}}%
\pgfpathcurveto{\pgfqpoint{1.956231in}{2.504303in}}{\pgfqpoint{1.945632in}{2.499913in}}{\pgfqpoint{1.937818in}{2.492099in}}%
\pgfpathcurveto{\pgfqpoint{1.930005in}{2.484285in}}{\pgfqpoint{1.925615in}{2.473686in}}{\pgfqpoint{1.925615in}{2.462636in}}%
\pgfpathcurveto{\pgfqpoint{1.925615in}{2.451586in}}{\pgfqpoint{1.930005in}{2.440987in}}{\pgfqpoint{1.937818in}{2.433173in}}%
\pgfpathcurveto{\pgfqpoint{1.945632in}{2.425360in}}{\pgfqpoint{1.956231in}{2.420969in}}{\pgfqpoint{1.967281in}{2.420969in}}%
\pgfpathclose%
\pgfusepath{stroke,fill}%
\end{pgfscope}%
\begin{pgfscope}%
\pgfpathrectangle{\pgfqpoint{0.600000in}{0.600000in}}{\pgfqpoint{3.900000in}{3.900000in}}%
\pgfusepath{clip}%
\pgfsetbuttcap%
\pgfsetroundjoin%
\definecolor{currentfill}{rgb}{0.121569,0.466667,0.705882}%
\pgfsetfillcolor{currentfill}%
\pgfsetlinewidth{1.003750pt}%
\definecolor{currentstroke}{rgb}{0.121569,0.466667,0.705882}%
\pgfsetstrokecolor{currentstroke}%
\pgfsetdash{}{0pt}%
\pgfpathmoveto{\pgfqpoint{2.486127in}{2.673130in}}%
\pgfpathcurveto{\pgfqpoint{2.497177in}{2.673130in}}{\pgfqpoint{2.507776in}{2.677521in}}{\pgfqpoint{2.515590in}{2.685334in}}%
\pgfpathcurveto{\pgfqpoint{2.523403in}{2.693148in}}{\pgfqpoint{2.527794in}{2.703747in}}{\pgfqpoint{2.527794in}{2.714797in}}%
\pgfpathcurveto{\pgfqpoint{2.527794in}{2.725847in}}{\pgfqpoint{2.523403in}{2.736446in}}{\pgfqpoint{2.515590in}{2.744260in}}%
\pgfpathcurveto{\pgfqpoint{2.507776in}{2.752073in}}{\pgfqpoint{2.497177in}{2.756464in}}{\pgfqpoint{2.486127in}{2.756464in}}%
\pgfpathcurveto{\pgfqpoint{2.475077in}{2.756464in}}{\pgfqpoint{2.464478in}{2.752073in}}{\pgfqpoint{2.456664in}{2.744260in}}%
\pgfpathcurveto{\pgfqpoint{2.448850in}{2.736446in}}{\pgfqpoint{2.444460in}{2.725847in}}{\pgfqpoint{2.444460in}{2.714797in}}%
\pgfpathcurveto{\pgfqpoint{2.444460in}{2.703747in}}{\pgfqpoint{2.448850in}{2.693148in}}{\pgfqpoint{2.456664in}{2.685334in}}%
\pgfpathcurveto{\pgfqpoint{2.464478in}{2.677521in}}{\pgfqpoint{2.475077in}{2.673130in}}{\pgfqpoint{2.486127in}{2.673130in}}%
\pgfpathclose%
\pgfusepath{stroke,fill}%
\end{pgfscope}%
\begin{pgfscope}%
\pgfpathrectangle{\pgfqpoint{0.600000in}{0.600000in}}{\pgfqpoint{3.900000in}{3.900000in}}%
\pgfusepath{clip}%
\pgfsetbuttcap%
\pgfsetroundjoin%
\definecolor{currentfill}{rgb}{0.121569,0.466667,0.705882}%
\pgfsetfillcolor{currentfill}%
\pgfsetlinewidth{1.003750pt}%
\definecolor{currentstroke}{rgb}{0.121569,0.466667,0.705882}%
\pgfsetstrokecolor{currentstroke}%
\pgfsetdash{}{0pt}%
\pgfpathmoveto{\pgfqpoint{1.694703in}{1.894424in}}%
\pgfpathcurveto{\pgfqpoint{1.705753in}{1.894424in}}{\pgfqpoint{1.716352in}{1.898815in}}{\pgfqpoint{1.724166in}{1.906628in}}%
\pgfpathcurveto{\pgfqpoint{1.731980in}{1.914442in}}{\pgfqpoint{1.736370in}{1.925041in}}{\pgfqpoint{1.736370in}{1.936091in}}%
\pgfpathcurveto{\pgfqpoint{1.736370in}{1.947141in}}{\pgfqpoint{1.731980in}{1.957740in}}{\pgfqpoint{1.724166in}{1.965554in}}%
\pgfpathcurveto{\pgfqpoint{1.716352in}{1.973367in}}{\pgfqpoint{1.705753in}{1.977758in}}{\pgfqpoint{1.694703in}{1.977758in}}%
\pgfpathcurveto{\pgfqpoint{1.683653in}{1.977758in}}{\pgfqpoint{1.673054in}{1.973367in}}{\pgfqpoint{1.665240in}{1.965554in}}%
\pgfpathcurveto{\pgfqpoint{1.657427in}{1.957740in}}{\pgfqpoint{1.653036in}{1.947141in}}{\pgfqpoint{1.653036in}{1.936091in}}%
\pgfpathcurveto{\pgfqpoint{1.653036in}{1.925041in}}{\pgfqpoint{1.657427in}{1.914442in}}{\pgfqpoint{1.665240in}{1.906628in}}%
\pgfpathcurveto{\pgfqpoint{1.673054in}{1.898815in}}{\pgfqpoint{1.683653in}{1.894424in}}{\pgfqpoint{1.694703in}{1.894424in}}%
\pgfpathclose%
\pgfusepath{stroke,fill}%
\end{pgfscope}%
\begin{pgfscope}%
\pgfpathrectangle{\pgfqpoint{0.600000in}{0.600000in}}{\pgfqpoint{3.900000in}{3.900000in}}%
\pgfusepath{clip}%
\pgfsetbuttcap%
\pgfsetroundjoin%
\definecolor{currentfill}{rgb}{0.121569,0.466667,0.705882}%
\pgfsetfillcolor{currentfill}%
\pgfsetlinewidth{1.003750pt}%
\definecolor{currentstroke}{rgb}{0.121569,0.466667,0.705882}%
\pgfsetstrokecolor{currentstroke}%
\pgfsetdash{}{0pt}%
\pgfpathmoveto{\pgfqpoint{2.690167in}{2.530306in}}%
\pgfpathcurveto{\pgfqpoint{2.701217in}{2.530306in}}{\pgfqpoint{2.711816in}{2.534697in}}{\pgfqpoint{2.719630in}{2.542510in}}%
\pgfpathcurveto{\pgfqpoint{2.727443in}{2.550324in}}{\pgfqpoint{2.731834in}{2.560923in}}{\pgfqpoint{2.731834in}{2.571973in}}%
\pgfpathcurveto{\pgfqpoint{2.731834in}{2.583023in}}{\pgfqpoint{2.727443in}{2.593622in}}{\pgfqpoint{2.719630in}{2.601436in}}%
\pgfpathcurveto{\pgfqpoint{2.711816in}{2.609249in}}{\pgfqpoint{2.701217in}{2.613640in}}{\pgfqpoint{2.690167in}{2.613640in}}%
\pgfpathcurveto{\pgfqpoint{2.679117in}{2.613640in}}{\pgfqpoint{2.668518in}{2.609249in}}{\pgfqpoint{2.660704in}{2.601436in}}%
\pgfpathcurveto{\pgfqpoint{2.652891in}{2.593622in}}{\pgfqpoint{2.648500in}{2.583023in}}{\pgfqpoint{2.648500in}{2.571973in}}%
\pgfpathcurveto{\pgfqpoint{2.648500in}{2.560923in}}{\pgfqpoint{2.652891in}{2.550324in}}{\pgfqpoint{2.660704in}{2.542510in}}%
\pgfpathcurveto{\pgfqpoint{2.668518in}{2.534697in}}{\pgfqpoint{2.679117in}{2.530306in}}{\pgfqpoint{2.690167in}{2.530306in}}%
\pgfpathclose%
\pgfusepath{stroke,fill}%
\end{pgfscope}%
\begin{pgfscope}%
\pgfpathrectangle{\pgfqpoint{0.600000in}{0.600000in}}{\pgfqpoint{3.900000in}{3.900000in}}%
\pgfusepath{clip}%
\pgfsetbuttcap%
\pgfsetroundjoin%
\definecolor{currentfill}{rgb}{0.121569,0.466667,0.705882}%
\pgfsetfillcolor{currentfill}%
\pgfsetlinewidth{1.003750pt}%
\definecolor{currentstroke}{rgb}{0.121569,0.466667,0.705882}%
\pgfsetstrokecolor{currentstroke}%
\pgfsetdash{}{0pt}%
\pgfpathmoveto{\pgfqpoint{2.905630in}{2.639801in}}%
\pgfpathcurveto{\pgfqpoint{2.916680in}{2.639801in}}{\pgfqpoint{2.927279in}{2.644191in}}{\pgfqpoint{2.935092in}{2.652004in}}%
\pgfpathcurveto{\pgfqpoint{2.942906in}{2.659818in}}{\pgfqpoint{2.947296in}{2.670417in}}{\pgfqpoint{2.947296in}{2.681467in}}%
\pgfpathcurveto{\pgfqpoint{2.947296in}{2.692517in}}{\pgfqpoint{2.942906in}{2.703116in}}{\pgfqpoint{2.935092in}{2.710930in}}%
\pgfpathcurveto{\pgfqpoint{2.927279in}{2.718744in}}{\pgfqpoint{2.916680in}{2.723134in}}{\pgfqpoint{2.905630in}{2.723134in}}%
\pgfpathcurveto{\pgfqpoint{2.894579in}{2.723134in}}{\pgfqpoint{2.883980in}{2.718744in}}{\pgfqpoint{2.876167in}{2.710930in}}%
\pgfpathcurveto{\pgfqpoint{2.868353in}{2.703116in}}{\pgfqpoint{2.863963in}{2.692517in}}{\pgfqpoint{2.863963in}{2.681467in}}%
\pgfpathcurveto{\pgfqpoint{2.863963in}{2.670417in}}{\pgfqpoint{2.868353in}{2.659818in}}{\pgfqpoint{2.876167in}{2.652004in}}%
\pgfpathcurveto{\pgfqpoint{2.883980in}{2.644191in}}{\pgfqpoint{2.894579in}{2.639801in}}{\pgfqpoint{2.905630in}{2.639801in}}%
\pgfpathclose%
\pgfusepath{stroke,fill}%
\end{pgfscope}%
\begin{pgfscope}%
\pgfpathrectangle{\pgfqpoint{0.600000in}{0.600000in}}{\pgfqpoint{3.900000in}{3.900000in}}%
\pgfusepath{clip}%
\pgfsetbuttcap%
\pgfsetroundjoin%
\definecolor{currentfill}{rgb}{0.121569,0.466667,0.705882}%
\pgfsetfillcolor{currentfill}%
\pgfsetlinewidth{1.003750pt}%
\definecolor{currentstroke}{rgb}{0.121569,0.466667,0.705882}%
\pgfsetstrokecolor{currentstroke}%
\pgfsetdash{}{0pt}%
\pgfpathmoveto{\pgfqpoint{3.233143in}{2.667842in}}%
\pgfpathcurveto{\pgfqpoint{3.244193in}{2.667842in}}{\pgfqpoint{3.254792in}{2.672232in}}{\pgfqpoint{3.262606in}{2.680046in}}%
\pgfpathcurveto{\pgfqpoint{3.270420in}{2.687860in}}{\pgfqpoint{3.274810in}{2.698459in}}{\pgfqpoint{3.274810in}{2.709509in}}%
\pgfpathcurveto{\pgfqpoint{3.274810in}{2.720559in}}{\pgfqpoint{3.270420in}{2.731158in}}{\pgfqpoint{3.262606in}{2.738972in}}%
\pgfpathcurveto{\pgfqpoint{3.254792in}{2.746785in}}{\pgfqpoint{3.244193in}{2.751175in}}{\pgfqpoint{3.233143in}{2.751175in}}%
\pgfpathcurveto{\pgfqpoint{3.222093in}{2.751175in}}{\pgfqpoint{3.211494in}{2.746785in}}{\pgfqpoint{3.203680in}{2.738972in}}%
\pgfpathcurveto{\pgfqpoint{3.195867in}{2.731158in}}{\pgfqpoint{3.191476in}{2.720559in}}{\pgfqpoint{3.191476in}{2.709509in}}%
\pgfpathcurveto{\pgfqpoint{3.191476in}{2.698459in}}{\pgfqpoint{3.195867in}{2.687860in}}{\pgfqpoint{3.203680in}{2.680046in}}%
\pgfpathcurveto{\pgfqpoint{3.211494in}{2.672232in}}{\pgfqpoint{3.222093in}{2.667842in}}{\pgfqpoint{3.233143in}{2.667842in}}%
\pgfpathclose%
\pgfusepath{stroke,fill}%
\end{pgfscope}%
\begin{pgfscope}%
\pgfpathrectangle{\pgfqpoint{0.600000in}{0.600000in}}{\pgfqpoint{3.900000in}{3.900000in}}%
\pgfusepath{clip}%
\pgfsetbuttcap%
\pgfsetroundjoin%
\definecolor{currentfill}{rgb}{0.121569,0.466667,0.705882}%
\pgfsetfillcolor{currentfill}%
\pgfsetlinewidth{1.003750pt}%
\definecolor{currentstroke}{rgb}{0.121569,0.466667,0.705882}%
\pgfsetstrokecolor{currentstroke}%
\pgfsetdash{}{0pt}%
\pgfpathmoveto{\pgfqpoint{3.564712in}{1.702185in}}%
\pgfpathcurveto{\pgfqpoint{3.575762in}{1.702185in}}{\pgfqpoint{3.586361in}{1.706575in}}{\pgfqpoint{3.594175in}{1.714389in}}%
\pgfpathcurveto{\pgfqpoint{3.601989in}{1.722202in}}{\pgfqpoint{3.606379in}{1.732801in}}{\pgfqpoint{3.606379in}{1.743851in}}%
\pgfpathcurveto{\pgfqpoint{3.606379in}{1.754901in}}{\pgfqpoint{3.601989in}{1.765501in}}{\pgfqpoint{3.594175in}{1.773314in}}%
\pgfpathcurveto{\pgfqpoint{3.586361in}{1.781128in}}{\pgfqpoint{3.575762in}{1.785518in}}{\pgfqpoint{3.564712in}{1.785518in}}%
\pgfpathcurveto{\pgfqpoint{3.553662in}{1.785518in}}{\pgfqpoint{3.543063in}{1.781128in}}{\pgfqpoint{3.535249in}{1.773314in}}%
\pgfpathcurveto{\pgfqpoint{3.527436in}{1.765501in}}{\pgfqpoint{3.523046in}{1.754901in}}{\pgfqpoint{3.523046in}{1.743851in}}%
\pgfpathcurveto{\pgfqpoint{3.523046in}{1.732801in}}{\pgfqpoint{3.527436in}{1.722202in}}{\pgfqpoint{3.535249in}{1.714389in}}%
\pgfpathcurveto{\pgfqpoint{3.543063in}{1.706575in}}{\pgfqpoint{3.553662in}{1.702185in}}{\pgfqpoint{3.564712in}{1.702185in}}%
\pgfpathclose%
\pgfusepath{stroke,fill}%
\end{pgfscope}%
\begin{pgfscope}%
\pgfpathrectangle{\pgfqpoint{0.600000in}{0.600000in}}{\pgfqpoint{3.900000in}{3.900000in}}%
\pgfusepath{clip}%
\pgfsetbuttcap%
\pgfsetroundjoin%
\definecolor{currentfill}{rgb}{0.121569,0.466667,0.705882}%
\pgfsetfillcolor{currentfill}%
\pgfsetlinewidth{1.003750pt}%
\definecolor{currentstroke}{rgb}{0.121569,0.466667,0.705882}%
\pgfsetstrokecolor{currentstroke}%
\pgfsetdash{}{0pt}%
\pgfpathmoveto{\pgfqpoint{2.447684in}{2.856117in}}%
\pgfpathcurveto{\pgfqpoint{2.458734in}{2.856117in}}{\pgfqpoint{2.469333in}{2.860508in}}{\pgfqpoint{2.477147in}{2.868321in}}%
\pgfpathcurveto{\pgfqpoint{2.484961in}{2.876135in}}{\pgfqpoint{2.489351in}{2.886734in}}{\pgfqpoint{2.489351in}{2.897784in}}%
\pgfpathcurveto{\pgfqpoint{2.489351in}{2.908834in}}{\pgfqpoint{2.484961in}{2.919433in}}{\pgfqpoint{2.477147in}{2.927247in}}%
\pgfpathcurveto{\pgfqpoint{2.469333in}{2.935060in}}{\pgfqpoint{2.458734in}{2.939451in}}{\pgfqpoint{2.447684in}{2.939451in}}%
\pgfpathcurveto{\pgfqpoint{2.436634in}{2.939451in}}{\pgfqpoint{2.426035in}{2.935060in}}{\pgfqpoint{2.418221in}{2.927247in}}%
\pgfpathcurveto{\pgfqpoint{2.410408in}{2.919433in}}{\pgfqpoint{2.406017in}{2.908834in}}{\pgfqpoint{2.406017in}{2.897784in}}%
\pgfpathcurveto{\pgfqpoint{2.406017in}{2.886734in}}{\pgfqpoint{2.410408in}{2.876135in}}{\pgfqpoint{2.418221in}{2.868321in}}%
\pgfpathcurveto{\pgfqpoint{2.426035in}{2.860508in}}{\pgfqpoint{2.436634in}{2.856117in}}{\pgfqpoint{2.447684in}{2.856117in}}%
\pgfpathclose%
\pgfusepath{stroke,fill}%
\end{pgfscope}%
\begin{pgfscope}%
\pgfpathrectangle{\pgfqpoint{0.600000in}{0.600000in}}{\pgfqpoint{3.900000in}{3.900000in}}%
\pgfusepath{clip}%
\pgfsetbuttcap%
\pgfsetroundjoin%
\definecolor{currentfill}{rgb}{0.121569,0.466667,0.705882}%
\pgfsetfillcolor{currentfill}%
\pgfsetlinewidth{1.003750pt}%
\definecolor{currentstroke}{rgb}{0.121569,0.466667,0.705882}%
\pgfsetstrokecolor{currentstroke}%
\pgfsetdash{}{0pt}%
\pgfpathmoveto{\pgfqpoint{3.362016in}{2.839424in}}%
\pgfpathcurveto{\pgfqpoint{3.373066in}{2.839424in}}{\pgfqpoint{3.383665in}{2.843814in}}{\pgfqpoint{3.391478in}{2.851628in}}%
\pgfpathcurveto{\pgfqpoint{3.399292in}{2.859441in}}{\pgfqpoint{3.403682in}{2.870040in}}{\pgfqpoint{3.403682in}{2.881090in}}%
\pgfpathcurveto{\pgfqpoint{3.403682in}{2.892141in}}{\pgfqpoint{3.399292in}{2.902740in}}{\pgfqpoint{3.391478in}{2.910553in}}%
\pgfpathcurveto{\pgfqpoint{3.383665in}{2.918367in}}{\pgfqpoint{3.373066in}{2.922757in}}{\pgfqpoint{3.362016in}{2.922757in}}%
\pgfpathcurveto{\pgfqpoint{3.350965in}{2.922757in}}{\pgfqpoint{3.340366in}{2.918367in}}{\pgfqpoint{3.332553in}{2.910553in}}%
\pgfpathcurveto{\pgfqpoint{3.324739in}{2.902740in}}{\pgfqpoint{3.320349in}{2.892141in}}{\pgfqpoint{3.320349in}{2.881090in}}%
\pgfpathcurveto{\pgfqpoint{3.320349in}{2.870040in}}{\pgfqpoint{3.324739in}{2.859441in}}{\pgfqpoint{3.332553in}{2.851628in}}%
\pgfpathcurveto{\pgfqpoint{3.340366in}{2.843814in}}{\pgfqpoint{3.350965in}{2.839424in}}{\pgfqpoint{3.362016in}{2.839424in}}%
\pgfpathclose%
\pgfusepath{stroke,fill}%
\end{pgfscope}%
\begin{pgfscope}%
\pgfpathrectangle{\pgfqpoint{0.600000in}{0.600000in}}{\pgfqpoint{3.900000in}{3.900000in}}%
\pgfusepath{clip}%
\pgfsetbuttcap%
\pgfsetroundjoin%
\definecolor{currentfill}{rgb}{0.121569,0.466667,0.705882}%
\pgfsetfillcolor{currentfill}%
\pgfsetlinewidth{1.003750pt}%
\definecolor{currentstroke}{rgb}{0.121569,0.466667,0.705882}%
\pgfsetstrokecolor{currentstroke}%
\pgfsetdash{}{0pt}%
\pgfpathmoveto{\pgfqpoint{2.043934in}{3.478949in}}%
\pgfpathcurveto{\pgfqpoint{2.054984in}{3.478949in}}{\pgfqpoint{2.065583in}{3.483340in}}{\pgfqpoint{2.073397in}{3.491153in}}%
\pgfpathcurveto{\pgfqpoint{2.081210in}{3.498967in}}{\pgfqpoint{2.085601in}{3.509566in}}{\pgfqpoint{2.085601in}{3.520616in}}%
\pgfpathcurveto{\pgfqpoint{2.085601in}{3.531666in}}{\pgfqpoint{2.081210in}{3.542265in}}{\pgfqpoint{2.073397in}{3.550079in}}%
\pgfpathcurveto{\pgfqpoint{2.065583in}{3.557893in}}{\pgfqpoint{2.054984in}{3.562283in}}{\pgfqpoint{2.043934in}{3.562283in}}%
\pgfpathcurveto{\pgfqpoint{2.032884in}{3.562283in}}{\pgfqpoint{2.022285in}{3.557893in}}{\pgfqpoint{2.014471in}{3.550079in}}%
\pgfpathcurveto{\pgfqpoint{2.006658in}{3.542265in}}{\pgfqpoint{2.002267in}{3.531666in}}{\pgfqpoint{2.002267in}{3.520616in}}%
\pgfpathcurveto{\pgfqpoint{2.002267in}{3.509566in}}{\pgfqpoint{2.006658in}{3.498967in}}{\pgfqpoint{2.014471in}{3.491153in}}%
\pgfpathcurveto{\pgfqpoint{2.022285in}{3.483340in}}{\pgfqpoint{2.032884in}{3.478949in}}{\pgfqpoint{2.043934in}{3.478949in}}%
\pgfpathclose%
\pgfusepath{stroke,fill}%
\end{pgfscope}%
\begin{pgfscope}%
\pgfpathrectangle{\pgfqpoint{0.600000in}{0.600000in}}{\pgfqpoint{3.900000in}{3.900000in}}%
\pgfusepath{clip}%
\pgfsetbuttcap%
\pgfsetroundjoin%
\definecolor{currentfill}{rgb}{0.121569,0.466667,0.705882}%
\pgfsetfillcolor{currentfill}%
\pgfsetlinewidth{1.003750pt}%
\definecolor{currentstroke}{rgb}{0.121569,0.466667,0.705882}%
\pgfsetstrokecolor{currentstroke}%
\pgfsetdash{}{0pt}%
\pgfpathmoveto{\pgfqpoint{2.591965in}{2.435040in}}%
\pgfpathcurveto{\pgfqpoint{2.603015in}{2.435040in}}{\pgfqpoint{2.613614in}{2.439430in}}{\pgfqpoint{2.621428in}{2.447243in}}%
\pgfpathcurveto{\pgfqpoint{2.629242in}{2.455057in}}{\pgfqpoint{2.633632in}{2.465656in}}{\pgfqpoint{2.633632in}{2.476706in}}%
\pgfpathcurveto{\pgfqpoint{2.633632in}{2.487756in}}{\pgfqpoint{2.629242in}{2.498355in}}{\pgfqpoint{2.621428in}{2.506169in}}%
\pgfpathcurveto{\pgfqpoint{2.613614in}{2.513983in}}{\pgfqpoint{2.603015in}{2.518373in}}{\pgfqpoint{2.591965in}{2.518373in}}%
\pgfpathcurveto{\pgfqpoint{2.580915in}{2.518373in}}{\pgfqpoint{2.570316in}{2.513983in}}{\pgfqpoint{2.562502in}{2.506169in}}%
\pgfpathcurveto{\pgfqpoint{2.554689in}{2.498355in}}{\pgfqpoint{2.550298in}{2.487756in}}{\pgfqpoint{2.550298in}{2.476706in}}%
\pgfpathcurveto{\pgfqpoint{2.550298in}{2.465656in}}{\pgfqpoint{2.554689in}{2.455057in}}{\pgfqpoint{2.562502in}{2.447243in}}%
\pgfpathcurveto{\pgfqpoint{2.570316in}{2.439430in}}{\pgfqpoint{2.580915in}{2.435040in}}{\pgfqpoint{2.591965in}{2.435040in}}%
\pgfpathclose%
\pgfusepath{stroke,fill}%
\end{pgfscope}%
\begin{pgfscope}%
\pgfpathrectangle{\pgfqpoint{0.600000in}{0.600000in}}{\pgfqpoint{3.900000in}{3.900000in}}%
\pgfusepath{clip}%
\pgfsetbuttcap%
\pgfsetroundjoin%
\definecolor{currentfill}{rgb}{0.121569,0.466667,0.705882}%
\pgfsetfillcolor{currentfill}%
\pgfsetlinewidth{1.003750pt}%
\definecolor{currentstroke}{rgb}{0.121569,0.466667,0.705882}%
\pgfsetstrokecolor{currentstroke}%
\pgfsetdash{}{0pt}%
\pgfpathmoveto{\pgfqpoint{2.738737in}{2.157081in}}%
\pgfpathcurveto{\pgfqpoint{2.749787in}{2.157081in}}{\pgfqpoint{2.760386in}{2.161471in}}{\pgfqpoint{2.768200in}{2.169285in}}%
\pgfpathcurveto{\pgfqpoint{2.776014in}{2.177098in}}{\pgfqpoint{2.780404in}{2.187697in}}{\pgfqpoint{2.780404in}{2.198748in}}%
\pgfpathcurveto{\pgfqpoint{2.780404in}{2.209798in}}{\pgfqpoint{2.776014in}{2.220397in}}{\pgfqpoint{2.768200in}{2.228210in}}%
\pgfpathcurveto{\pgfqpoint{2.760386in}{2.236024in}}{\pgfqpoint{2.749787in}{2.240414in}}{\pgfqpoint{2.738737in}{2.240414in}}%
\pgfpathcurveto{\pgfqpoint{2.727687in}{2.240414in}}{\pgfqpoint{2.717088in}{2.236024in}}{\pgfqpoint{2.709274in}{2.228210in}}%
\pgfpathcurveto{\pgfqpoint{2.701461in}{2.220397in}}{\pgfqpoint{2.697070in}{2.209798in}}{\pgfqpoint{2.697070in}{2.198748in}}%
\pgfpathcurveto{\pgfqpoint{2.697070in}{2.187697in}}{\pgfqpoint{2.701461in}{2.177098in}}{\pgfqpoint{2.709274in}{2.169285in}}%
\pgfpathcurveto{\pgfqpoint{2.717088in}{2.161471in}}{\pgfqpoint{2.727687in}{2.157081in}}{\pgfqpoint{2.738737in}{2.157081in}}%
\pgfpathclose%
\pgfusepath{stroke,fill}%
\end{pgfscope}%
\begin{pgfscope}%
\pgfpathrectangle{\pgfqpoint{0.600000in}{0.600000in}}{\pgfqpoint{3.900000in}{3.900000in}}%
\pgfusepath{clip}%
\pgfsetbuttcap%
\pgfsetroundjoin%
\definecolor{currentfill}{rgb}{0.121569,0.466667,0.705882}%
\pgfsetfillcolor{currentfill}%
\pgfsetlinewidth{1.003750pt}%
\definecolor{currentstroke}{rgb}{0.121569,0.466667,0.705882}%
\pgfsetstrokecolor{currentstroke}%
\pgfsetdash{}{0pt}%
\pgfpathmoveto{\pgfqpoint{2.440567in}{2.271448in}}%
\pgfpathcurveto{\pgfqpoint{2.451617in}{2.271448in}}{\pgfqpoint{2.462216in}{2.275838in}}{\pgfqpoint{2.470030in}{2.283652in}}%
\pgfpathcurveto{\pgfqpoint{2.477843in}{2.291466in}}{\pgfqpoint{2.482234in}{2.302065in}}{\pgfqpoint{2.482234in}{2.313115in}}%
\pgfpathcurveto{\pgfqpoint{2.482234in}{2.324165in}}{\pgfqpoint{2.477843in}{2.334764in}}{\pgfqpoint{2.470030in}{2.342578in}}%
\pgfpathcurveto{\pgfqpoint{2.462216in}{2.350391in}}{\pgfqpoint{2.451617in}{2.354782in}}{\pgfqpoint{2.440567in}{2.354782in}}%
\pgfpathcurveto{\pgfqpoint{2.429517in}{2.354782in}}{\pgfqpoint{2.418918in}{2.350391in}}{\pgfqpoint{2.411104in}{2.342578in}}%
\pgfpathcurveto{\pgfqpoint{2.403291in}{2.334764in}}{\pgfqpoint{2.398900in}{2.324165in}}{\pgfqpoint{2.398900in}{2.313115in}}%
\pgfpathcurveto{\pgfqpoint{2.398900in}{2.302065in}}{\pgfqpoint{2.403291in}{2.291466in}}{\pgfqpoint{2.411104in}{2.283652in}}%
\pgfpathcurveto{\pgfqpoint{2.418918in}{2.275838in}}{\pgfqpoint{2.429517in}{2.271448in}}{\pgfqpoint{2.440567in}{2.271448in}}%
\pgfpathclose%
\pgfusepath{stroke,fill}%
\end{pgfscope}%
\begin{pgfscope}%
\pgfpathrectangle{\pgfqpoint{0.600000in}{0.600000in}}{\pgfqpoint{3.900000in}{3.900000in}}%
\pgfusepath{clip}%
\pgfsetbuttcap%
\pgfsetroundjoin%
\definecolor{currentfill}{rgb}{0.121569,0.466667,0.705882}%
\pgfsetfillcolor{currentfill}%
\pgfsetlinewidth{1.003750pt}%
\definecolor{currentstroke}{rgb}{0.121569,0.466667,0.705882}%
\pgfsetstrokecolor{currentstroke}%
\pgfsetdash{}{0pt}%
\pgfpathmoveto{\pgfqpoint{3.281553in}{1.982453in}}%
\pgfpathcurveto{\pgfqpoint{3.292603in}{1.982453in}}{\pgfqpoint{3.303202in}{1.986843in}}{\pgfqpoint{3.311016in}{1.994657in}}%
\pgfpathcurveto{\pgfqpoint{3.318829in}{2.002471in}}{\pgfqpoint{3.323220in}{2.013070in}}{\pgfqpoint{3.323220in}{2.024120in}}%
\pgfpathcurveto{\pgfqpoint{3.323220in}{2.035170in}}{\pgfqpoint{3.318829in}{2.045769in}}{\pgfqpoint{3.311016in}{2.053583in}}%
\pgfpathcurveto{\pgfqpoint{3.303202in}{2.061396in}}{\pgfqpoint{3.292603in}{2.065786in}}{\pgfqpoint{3.281553in}{2.065786in}}%
\pgfpathcurveto{\pgfqpoint{3.270503in}{2.065786in}}{\pgfqpoint{3.259904in}{2.061396in}}{\pgfqpoint{3.252090in}{2.053583in}}%
\pgfpathcurveto{\pgfqpoint{3.244276in}{2.045769in}}{\pgfqpoint{3.239886in}{2.035170in}}{\pgfqpoint{3.239886in}{2.024120in}}%
\pgfpathcurveto{\pgfqpoint{3.239886in}{2.013070in}}{\pgfqpoint{3.244276in}{2.002471in}}{\pgfqpoint{3.252090in}{1.994657in}}%
\pgfpathcurveto{\pgfqpoint{3.259904in}{1.986843in}}{\pgfqpoint{3.270503in}{1.982453in}}{\pgfqpoint{3.281553in}{1.982453in}}%
\pgfpathclose%
\pgfusepath{stroke,fill}%
\end{pgfscope}%
\begin{pgfscope}%
\pgfpathrectangle{\pgfqpoint{0.600000in}{0.600000in}}{\pgfqpoint{3.900000in}{3.900000in}}%
\pgfusepath{clip}%
\pgfsetbuttcap%
\pgfsetroundjoin%
\definecolor{currentfill}{rgb}{0.121569,0.466667,0.705882}%
\pgfsetfillcolor{currentfill}%
\pgfsetlinewidth{1.003750pt}%
\definecolor{currentstroke}{rgb}{0.121569,0.466667,0.705882}%
\pgfsetstrokecolor{currentstroke}%
\pgfsetdash{}{0pt}%
\pgfpathmoveto{\pgfqpoint{2.869838in}{2.082596in}}%
\pgfpathcurveto{\pgfqpoint{2.880888in}{2.082596in}}{\pgfqpoint{2.891487in}{2.086987in}}{\pgfqpoint{2.899300in}{2.094800in}}%
\pgfpathcurveto{\pgfqpoint{2.907114in}{2.102614in}}{\pgfqpoint{2.911504in}{2.113213in}}{\pgfqpoint{2.911504in}{2.124263in}}%
\pgfpathcurveto{\pgfqpoint{2.911504in}{2.135313in}}{\pgfqpoint{2.907114in}{2.145912in}}{\pgfqpoint{2.899300in}{2.153726in}}%
\pgfpathcurveto{\pgfqpoint{2.891487in}{2.161539in}}{\pgfqpoint{2.880888in}{2.165930in}}{\pgfqpoint{2.869838in}{2.165930in}}%
\pgfpathcurveto{\pgfqpoint{2.858787in}{2.165930in}}{\pgfqpoint{2.848188in}{2.161539in}}{\pgfqpoint{2.840375in}{2.153726in}}%
\pgfpathcurveto{\pgfqpoint{2.832561in}{2.145912in}}{\pgfqpoint{2.828171in}{2.135313in}}{\pgfqpoint{2.828171in}{2.124263in}}%
\pgfpathcurveto{\pgfqpoint{2.828171in}{2.113213in}}{\pgfqpoint{2.832561in}{2.102614in}}{\pgfqpoint{2.840375in}{2.094800in}}%
\pgfpathcurveto{\pgfqpoint{2.848188in}{2.086987in}}{\pgfqpoint{2.858787in}{2.082596in}}{\pgfqpoint{2.869838in}{2.082596in}}%
\pgfpathclose%
\pgfusepath{stroke,fill}%
\end{pgfscope}%
\begin{pgfscope}%
\pgfpathrectangle{\pgfqpoint{0.600000in}{0.600000in}}{\pgfqpoint{3.900000in}{3.900000in}}%
\pgfusepath{clip}%
\pgfsetbuttcap%
\pgfsetroundjoin%
\definecolor{currentfill}{rgb}{0.121569,0.466667,0.705882}%
\pgfsetfillcolor{currentfill}%
\pgfsetlinewidth{1.003750pt}%
\definecolor{currentstroke}{rgb}{0.121569,0.466667,0.705882}%
\pgfsetstrokecolor{currentstroke}%
\pgfsetdash{}{0pt}%
\pgfpathmoveto{\pgfqpoint{2.415319in}{2.376460in}}%
\pgfpathcurveto{\pgfqpoint{2.426370in}{2.376460in}}{\pgfqpoint{2.436969in}{2.380851in}}{\pgfqpoint{2.444782in}{2.388664in}}%
\pgfpathcurveto{\pgfqpoint{2.452596in}{2.396478in}}{\pgfqpoint{2.456986in}{2.407077in}}{\pgfqpoint{2.456986in}{2.418127in}}%
\pgfpathcurveto{\pgfqpoint{2.456986in}{2.429177in}}{\pgfqpoint{2.452596in}{2.439776in}}{\pgfqpoint{2.444782in}{2.447590in}}%
\pgfpathcurveto{\pgfqpoint{2.436969in}{2.455403in}}{\pgfqpoint{2.426370in}{2.459794in}}{\pgfqpoint{2.415319in}{2.459794in}}%
\pgfpathcurveto{\pgfqpoint{2.404269in}{2.459794in}}{\pgfqpoint{2.393670in}{2.455403in}}{\pgfqpoint{2.385857in}{2.447590in}}%
\pgfpathcurveto{\pgfqpoint{2.378043in}{2.439776in}}{\pgfqpoint{2.373653in}{2.429177in}}{\pgfqpoint{2.373653in}{2.418127in}}%
\pgfpathcurveto{\pgfqpoint{2.373653in}{2.407077in}}{\pgfqpoint{2.378043in}{2.396478in}}{\pgfqpoint{2.385857in}{2.388664in}}%
\pgfpathcurveto{\pgfqpoint{2.393670in}{2.380851in}}{\pgfqpoint{2.404269in}{2.376460in}}{\pgfqpoint{2.415319in}{2.376460in}}%
\pgfpathclose%
\pgfusepath{stroke,fill}%
\end{pgfscope}%
\begin{pgfscope}%
\pgfpathrectangle{\pgfqpoint{0.600000in}{0.600000in}}{\pgfqpoint{3.900000in}{3.900000in}}%
\pgfusepath{clip}%
\pgfsetbuttcap%
\pgfsetroundjoin%
\definecolor{currentfill}{rgb}{0.121569,0.466667,0.705882}%
\pgfsetfillcolor{currentfill}%
\pgfsetlinewidth{1.003750pt}%
\definecolor{currentstroke}{rgb}{0.121569,0.466667,0.705882}%
\pgfsetstrokecolor{currentstroke}%
\pgfsetdash{}{0pt}%
\pgfpathmoveto{\pgfqpoint{3.189316in}{2.606351in}}%
\pgfpathcurveto{\pgfqpoint{3.200366in}{2.606351in}}{\pgfqpoint{3.210965in}{2.610742in}}{\pgfqpoint{3.218779in}{2.618555in}}%
\pgfpathcurveto{\pgfqpoint{3.226593in}{2.626369in}}{\pgfqpoint{3.230983in}{2.636968in}}{\pgfqpoint{3.230983in}{2.648018in}}%
\pgfpathcurveto{\pgfqpoint{3.230983in}{2.659068in}}{\pgfqpoint{3.226593in}{2.669667in}}{\pgfqpoint{3.218779in}{2.677481in}}%
\pgfpathcurveto{\pgfqpoint{3.210965in}{2.685294in}}{\pgfqpoint{3.200366in}{2.689685in}}{\pgfqpoint{3.189316in}{2.689685in}}%
\pgfpathcurveto{\pgfqpoint{3.178266in}{2.689685in}}{\pgfqpoint{3.167667in}{2.685294in}}{\pgfqpoint{3.159853in}{2.677481in}}%
\pgfpathcurveto{\pgfqpoint{3.152040in}{2.669667in}}{\pgfqpoint{3.147650in}{2.659068in}}{\pgfqpoint{3.147650in}{2.648018in}}%
\pgfpathcurveto{\pgfqpoint{3.147650in}{2.636968in}}{\pgfqpoint{3.152040in}{2.626369in}}{\pgfqpoint{3.159853in}{2.618555in}}%
\pgfpathcurveto{\pgfqpoint{3.167667in}{2.610742in}}{\pgfqpoint{3.178266in}{2.606351in}}{\pgfqpoint{3.189316in}{2.606351in}}%
\pgfpathclose%
\pgfusepath{stroke,fill}%
\end{pgfscope}%
\begin{pgfscope}%
\pgfpathrectangle{\pgfqpoint{0.600000in}{0.600000in}}{\pgfqpoint{3.900000in}{3.900000in}}%
\pgfusepath{clip}%
\pgfsetbuttcap%
\pgfsetroundjoin%
\definecolor{currentfill}{rgb}{0.121569,0.466667,0.705882}%
\pgfsetfillcolor{currentfill}%
\pgfsetlinewidth{1.003750pt}%
\definecolor{currentstroke}{rgb}{0.121569,0.466667,0.705882}%
\pgfsetstrokecolor{currentstroke}%
\pgfsetdash{}{0pt}%
\pgfpathmoveto{\pgfqpoint{1.538070in}{2.661826in}}%
\pgfpathcurveto{\pgfqpoint{1.549120in}{2.661826in}}{\pgfqpoint{1.559719in}{2.666217in}}{\pgfqpoint{1.567533in}{2.674030in}}%
\pgfpathcurveto{\pgfqpoint{1.575346in}{2.681844in}}{\pgfqpoint{1.579737in}{2.692443in}}{\pgfqpoint{1.579737in}{2.703493in}}%
\pgfpathcurveto{\pgfqpoint{1.579737in}{2.714543in}}{\pgfqpoint{1.575346in}{2.725142in}}{\pgfqpoint{1.567533in}{2.732956in}}%
\pgfpathcurveto{\pgfqpoint{1.559719in}{2.740769in}}{\pgfqpoint{1.549120in}{2.745160in}}{\pgfqpoint{1.538070in}{2.745160in}}%
\pgfpathcurveto{\pgfqpoint{1.527020in}{2.745160in}}{\pgfqpoint{1.516421in}{2.740769in}}{\pgfqpoint{1.508607in}{2.732956in}}%
\pgfpathcurveto{\pgfqpoint{1.500793in}{2.725142in}}{\pgfqpoint{1.496403in}{2.714543in}}{\pgfqpoint{1.496403in}{2.703493in}}%
\pgfpathcurveto{\pgfqpoint{1.496403in}{2.692443in}}{\pgfqpoint{1.500793in}{2.681844in}}{\pgfqpoint{1.508607in}{2.674030in}}%
\pgfpathcurveto{\pgfqpoint{1.516421in}{2.666217in}}{\pgfqpoint{1.527020in}{2.661826in}}{\pgfqpoint{1.538070in}{2.661826in}}%
\pgfpathclose%
\pgfusepath{stroke,fill}%
\end{pgfscope}%
\begin{pgfscope}%
\pgfpathrectangle{\pgfqpoint{0.600000in}{0.600000in}}{\pgfqpoint{3.900000in}{3.900000in}}%
\pgfusepath{clip}%
\pgfsetbuttcap%
\pgfsetroundjoin%
\definecolor{currentfill}{rgb}{0.121569,0.466667,0.705882}%
\pgfsetfillcolor{currentfill}%
\pgfsetlinewidth{1.003750pt}%
\definecolor{currentstroke}{rgb}{0.121569,0.466667,0.705882}%
\pgfsetstrokecolor{currentstroke}%
\pgfsetdash{}{0pt}%
\pgfpathmoveto{\pgfqpoint{2.380133in}{1.780000in}}%
\pgfpathcurveto{\pgfqpoint{2.391183in}{1.780000in}}{\pgfqpoint{2.401782in}{1.784390in}}{\pgfqpoint{2.409595in}{1.792204in}}%
\pgfpathcurveto{\pgfqpoint{2.417409in}{1.800017in}}{\pgfqpoint{2.421799in}{1.810616in}}{\pgfqpoint{2.421799in}{1.821666in}}%
\pgfpathcurveto{\pgfqpoint{2.421799in}{1.832717in}}{\pgfqpoint{2.417409in}{1.843316in}}{\pgfqpoint{2.409595in}{1.851129in}}%
\pgfpathcurveto{\pgfqpoint{2.401782in}{1.858943in}}{\pgfqpoint{2.391183in}{1.863333in}}{\pgfqpoint{2.380133in}{1.863333in}}%
\pgfpathcurveto{\pgfqpoint{2.369082in}{1.863333in}}{\pgfqpoint{2.358483in}{1.858943in}}{\pgfqpoint{2.350670in}{1.851129in}}%
\pgfpathcurveto{\pgfqpoint{2.342856in}{1.843316in}}{\pgfqpoint{2.338466in}{1.832717in}}{\pgfqpoint{2.338466in}{1.821666in}}%
\pgfpathcurveto{\pgfqpoint{2.338466in}{1.810616in}}{\pgfqpoint{2.342856in}{1.800017in}}{\pgfqpoint{2.350670in}{1.792204in}}%
\pgfpathcurveto{\pgfqpoint{2.358483in}{1.784390in}}{\pgfqpoint{2.369082in}{1.780000in}}{\pgfqpoint{2.380133in}{1.780000in}}%
\pgfpathclose%
\pgfusepath{stroke,fill}%
\end{pgfscope}%
\begin{pgfscope}%
\pgfpathrectangle{\pgfqpoint{0.600000in}{0.600000in}}{\pgfqpoint{3.900000in}{3.900000in}}%
\pgfusepath{clip}%
\pgfsetbuttcap%
\pgfsetroundjoin%
\definecolor{currentfill}{rgb}{0.121569,0.466667,0.705882}%
\pgfsetfillcolor{currentfill}%
\pgfsetlinewidth{1.003750pt}%
\definecolor{currentstroke}{rgb}{0.121569,0.466667,0.705882}%
\pgfsetstrokecolor{currentstroke}%
\pgfsetdash{}{0pt}%
\pgfpathmoveto{\pgfqpoint{3.853940in}{2.167058in}}%
\pgfpathcurveto{\pgfqpoint{3.864990in}{2.167058in}}{\pgfqpoint{3.875589in}{2.171448in}}{\pgfqpoint{3.883403in}{2.179261in}}%
\pgfpathcurveto{\pgfqpoint{3.891216in}{2.187075in}}{\pgfqpoint{3.895607in}{2.197674in}}{\pgfqpoint{3.895607in}{2.208724in}}%
\pgfpathcurveto{\pgfqpoint{3.895607in}{2.219774in}}{\pgfqpoint{3.891216in}{2.230373in}}{\pgfqpoint{3.883403in}{2.238187in}}%
\pgfpathcurveto{\pgfqpoint{3.875589in}{2.246001in}}{\pgfqpoint{3.864990in}{2.250391in}}{\pgfqpoint{3.853940in}{2.250391in}}%
\pgfpathcurveto{\pgfqpoint{3.842890in}{2.250391in}}{\pgfqpoint{3.832291in}{2.246001in}}{\pgfqpoint{3.824477in}{2.238187in}}%
\pgfpathcurveto{\pgfqpoint{3.816664in}{2.230373in}}{\pgfqpoint{3.812273in}{2.219774in}}{\pgfqpoint{3.812273in}{2.208724in}}%
\pgfpathcurveto{\pgfqpoint{3.812273in}{2.197674in}}{\pgfqpoint{3.816664in}{2.187075in}}{\pgfqpoint{3.824477in}{2.179261in}}%
\pgfpathcurveto{\pgfqpoint{3.832291in}{2.171448in}}{\pgfqpoint{3.842890in}{2.167058in}}{\pgfqpoint{3.853940in}{2.167058in}}%
\pgfpathclose%
\pgfusepath{stroke,fill}%
\end{pgfscope}%
\begin{pgfscope}%
\pgfpathrectangle{\pgfqpoint{0.600000in}{0.600000in}}{\pgfqpoint{3.900000in}{3.900000in}}%
\pgfusepath{clip}%
\pgfsetbuttcap%
\pgfsetroundjoin%
\definecolor{currentfill}{rgb}{0.121569,0.466667,0.705882}%
\pgfsetfillcolor{currentfill}%
\pgfsetlinewidth{1.003750pt}%
\definecolor{currentstroke}{rgb}{0.121569,0.466667,0.705882}%
\pgfsetstrokecolor{currentstroke}%
\pgfsetdash{}{0pt}%
\pgfpathmoveto{\pgfqpoint{2.549988in}{3.134071in}}%
\pgfpathcurveto{\pgfqpoint{2.561038in}{3.134071in}}{\pgfqpoint{2.571637in}{3.138462in}}{\pgfqpoint{2.579451in}{3.146275in}}%
\pgfpathcurveto{\pgfqpoint{2.587264in}{3.154089in}}{\pgfqpoint{2.591655in}{3.164688in}}{\pgfqpoint{2.591655in}{3.175738in}}%
\pgfpathcurveto{\pgfqpoint{2.591655in}{3.186788in}}{\pgfqpoint{2.587264in}{3.197387in}}{\pgfqpoint{2.579451in}{3.205201in}}%
\pgfpathcurveto{\pgfqpoint{2.571637in}{3.213014in}}{\pgfqpoint{2.561038in}{3.217405in}}{\pgfqpoint{2.549988in}{3.217405in}}%
\pgfpathcurveto{\pgfqpoint{2.538938in}{3.217405in}}{\pgfqpoint{2.528339in}{3.213014in}}{\pgfqpoint{2.520525in}{3.205201in}}%
\pgfpathcurveto{\pgfqpoint{2.512712in}{3.197387in}}{\pgfqpoint{2.508321in}{3.186788in}}{\pgfqpoint{2.508321in}{3.175738in}}%
\pgfpathcurveto{\pgfqpoint{2.508321in}{3.164688in}}{\pgfqpoint{2.512712in}{3.154089in}}{\pgfqpoint{2.520525in}{3.146275in}}%
\pgfpathcurveto{\pgfqpoint{2.528339in}{3.138462in}}{\pgfqpoint{2.538938in}{3.134071in}}{\pgfqpoint{2.549988in}{3.134071in}}%
\pgfpathclose%
\pgfusepath{stroke,fill}%
\end{pgfscope}%
\begin{pgfscope}%
\pgfpathrectangle{\pgfqpoint{0.600000in}{0.600000in}}{\pgfqpoint{3.900000in}{3.900000in}}%
\pgfusepath{clip}%
\pgfsetbuttcap%
\pgfsetroundjoin%
\definecolor{currentfill}{rgb}{0.121569,0.466667,0.705882}%
\pgfsetfillcolor{currentfill}%
\pgfsetlinewidth{1.003750pt}%
\definecolor{currentstroke}{rgb}{0.121569,0.466667,0.705882}%
\pgfsetstrokecolor{currentstroke}%
\pgfsetdash{}{0pt}%
\pgfpathmoveto{\pgfqpoint{2.780980in}{2.694881in}}%
\pgfpathcurveto{\pgfqpoint{2.792030in}{2.694881in}}{\pgfqpoint{2.802629in}{2.699271in}}{\pgfqpoint{2.810442in}{2.707085in}}%
\pgfpathcurveto{\pgfqpoint{2.818256in}{2.714898in}}{\pgfqpoint{2.822646in}{2.725498in}}{\pgfqpoint{2.822646in}{2.736548in}}%
\pgfpathcurveto{\pgfqpoint{2.822646in}{2.747598in}}{\pgfqpoint{2.818256in}{2.758197in}}{\pgfqpoint{2.810442in}{2.766010in}}%
\pgfpathcurveto{\pgfqpoint{2.802629in}{2.773824in}}{\pgfqpoint{2.792030in}{2.778214in}}{\pgfqpoint{2.780980in}{2.778214in}}%
\pgfpathcurveto{\pgfqpoint{2.769930in}{2.778214in}}{\pgfqpoint{2.759330in}{2.773824in}}{\pgfqpoint{2.751517in}{2.766010in}}%
\pgfpathcurveto{\pgfqpoint{2.743703in}{2.758197in}}{\pgfqpoint{2.739313in}{2.747598in}}{\pgfqpoint{2.739313in}{2.736548in}}%
\pgfpathcurveto{\pgfqpoint{2.739313in}{2.725498in}}{\pgfqpoint{2.743703in}{2.714898in}}{\pgfqpoint{2.751517in}{2.707085in}}%
\pgfpathcurveto{\pgfqpoint{2.759330in}{2.699271in}}{\pgfqpoint{2.769930in}{2.694881in}}{\pgfqpoint{2.780980in}{2.694881in}}%
\pgfpathclose%
\pgfusepath{stroke,fill}%
\end{pgfscope}%
\begin{pgfscope}%
\pgfpathrectangle{\pgfqpoint{0.600000in}{0.600000in}}{\pgfqpoint{3.900000in}{3.900000in}}%
\pgfusepath{clip}%
\pgfsetbuttcap%
\pgfsetroundjoin%
\definecolor{currentfill}{rgb}{0.121569,0.466667,0.705882}%
\pgfsetfillcolor{currentfill}%
\pgfsetlinewidth{1.003750pt}%
\definecolor{currentstroke}{rgb}{0.121569,0.466667,0.705882}%
\pgfsetstrokecolor{currentstroke}%
\pgfsetdash{}{0pt}%
\pgfpathmoveto{\pgfqpoint{2.638921in}{2.054517in}}%
\pgfpathcurveto{\pgfqpoint{2.649971in}{2.054517in}}{\pgfqpoint{2.660570in}{2.058907in}}{\pgfqpoint{2.668384in}{2.066720in}}%
\pgfpathcurveto{\pgfqpoint{2.676197in}{2.074534in}}{\pgfqpoint{2.680588in}{2.085133in}}{\pgfqpoint{2.680588in}{2.096183in}}%
\pgfpathcurveto{\pgfqpoint{2.680588in}{2.107233in}}{\pgfqpoint{2.676197in}{2.117832in}}{\pgfqpoint{2.668384in}{2.125646in}}%
\pgfpathcurveto{\pgfqpoint{2.660570in}{2.133460in}}{\pgfqpoint{2.649971in}{2.137850in}}{\pgfqpoint{2.638921in}{2.137850in}}%
\pgfpathcurveto{\pgfqpoint{2.627871in}{2.137850in}}{\pgfqpoint{2.617272in}{2.133460in}}{\pgfqpoint{2.609458in}{2.125646in}}%
\pgfpathcurveto{\pgfqpoint{2.601645in}{2.117832in}}{\pgfqpoint{2.597254in}{2.107233in}}{\pgfqpoint{2.597254in}{2.096183in}}%
\pgfpathcurveto{\pgfqpoint{2.597254in}{2.085133in}}{\pgfqpoint{2.601645in}{2.074534in}}{\pgfqpoint{2.609458in}{2.066720in}}%
\pgfpathcurveto{\pgfqpoint{2.617272in}{2.058907in}}{\pgfqpoint{2.627871in}{2.054517in}}{\pgfqpoint{2.638921in}{2.054517in}}%
\pgfpathclose%
\pgfusepath{stroke,fill}%
\end{pgfscope}%
\begin{pgfscope}%
\pgfpathrectangle{\pgfqpoint{0.600000in}{0.600000in}}{\pgfqpoint{3.900000in}{3.900000in}}%
\pgfusepath{clip}%
\pgfsetbuttcap%
\pgfsetroundjoin%
\definecolor{currentfill}{rgb}{0.121569,0.466667,0.705882}%
\pgfsetfillcolor{currentfill}%
\pgfsetlinewidth{1.003750pt}%
\definecolor{currentstroke}{rgb}{0.121569,0.466667,0.705882}%
\pgfsetstrokecolor{currentstroke}%
\pgfsetdash{}{0pt}%
\pgfpathmoveto{\pgfqpoint{2.913271in}{2.781972in}}%
\pgfpathcurveto{\pgfqpoint{2.924321in}{2.781972in}}{\pgfqpoint{2.934920in}{2.786362in}}{\pgfqpoint{2.942734in}{2.794176in}}%
\pgfpathcurveto{\pgfqpoint{2.950547in}{2.801989in}}{\pgfqpoint{2.954938in}{2.812588in}}{\pgfqpoint{2.954938in}{2.823639in}}%
\pgfpathcurveto{\pgfqpoint{2.954938in}{2.834689in}}{\pgfqpoint{2.950547in}{2.845288in}}{\pgfqpoint{2.942734in}{2.853101in}}%
\pgfpathcurveto{\pgfqpoint{2.934920in}{2.860915in}}{\pgfqpoint{2.924321in}{2.865305in}}{\pgfqpoint{2.913271in}{2.865305in}}%
\pgfpathcurveto{\pgfqpoint{2.902221in}{2.865305in}}{\pgfqpoint{2.891622in}{2.860915in}}{\pgfqpoint{2.883808in}{2.853101in}}%
\pgfpathcurveto{\pgfqpoint{2.875995in}{2.845288in}}{\pgfqpoint{2.871604in}{2.834689in}}{\pgfqpoint{2.871604in}{2.823639in}}%
\pgfpathcurveto{\pgfqpoint{2.871604in}{2.812588in}}{\pgfqpoint{2.875995in}{2.801989in}}{\pgfqpoint{2.883808in}{2.794176in}}%
\pgfpathcurveto{\pgfqpoint{2.891622in}{2.786362in}}{\pgfqpoint{2.902221in}{2.781972in}}{\pgfqpoint{2.913271in}{2.781972in}}%
\pgfpathclose%
\pgfusepath{stroke,fill}%
\end{pgfscope}%
\begin{pgfscope}%
\pgfpathrectangle{\pgfqpoint{0.600000in}{0.600000in}}{\pgfqpoint{3.900000in}{3.900000in}}%
\pgfusepath{clip}%
\pgfsetbuttcap%
\pgfsetroundjoin%
\definecolor{currentfill}{rgb}{0.121569,0.466667,0.705882}%
\pgfsetfillcolor{currentfill}%
\pgfsetlinewidth{1.003750pt}%
\definecolor{currentstroke}{rgb}{0.121569,0.466667,0.705882}%
\pgfsetstrokecolor{currentstroke}%
\pgfsetdash{}{0pt}%
\pgfpathmoveto{\pgfqpoint{3.011055in}{2.522042in}}%
\pgfpathcurveto{\pgfqpoint{3.022105in}{2.522042in}}{\pgfqpoint{3.032704in}{2.526432in}}{\pgfqpoint{3.040518in}{2.534246in}}%
\pgfpathcurveto{\pgfqpoint{3.048331in}{2.542060in}}{\pgfqpoint{3.052722in}{2.552659in}}{\pgfqpoint{3.052722in}{2.563709in}}%
\pgfpathcurveto{\pgfqpoint{3.052722in}{2.574759in}}{\pgfqpoint{3.048331in}{2.585358in}}{\pgfqpoint{3.040518in}{2.593172in}}%
\pgfpathcurveto{\pgfqpoint{3.032704in}{2.600985in}}{\pgfqpoint{3.022105in}{2.605376in}}{\pgfqpoint{3.011055in}{2.605376in}}%
\pgfpathcurveto{\pgfqpoint{3.000005in}{2.605376in}}{\pgfqpoint{2.989406in}{2.600985in}}{\pgfqpoint{2.981592in}{2.593172in}}%
\pgfpathcurveto{\pgfqpoint{2.973779in}{2.585358in}}{\pgfqpoint{2.969388in}{2.574759in}}{\pgfqpoint{2.969388in}{2.563709in}}%
\pgfpathcurveto{\pgfqpoint{2.969388in}{2.552659in}}{\pgfqpoint{2.973779in}{2.542060in}}{\pgfqpoint{2.981592in}{2.534246in}}%
\pgfpathcurveto{\pgfqpoint{2.989406in}{2.526432in}}{\pgfqpoint{3.000005in}{2.522042in}}{\pgfqpoint{3.011055in}{2.522042in}}%
\pgfpathclose%
\pgfusepath{stroke,fill}%
\end{pgfscope}%
\begin{pgfscope}%
\pgfpathrectangle{\pgfqpoint{0.600000in}{0.600000in}}{\pgfqpoint{3.900000in}{3.900000in}}%
\pgfusepath{clip}%
\pgfsetbuttcap%
\pgfsetroundjoin%
\definecolor{currentfill}{rgb}{0.121569,0.466667,0.705882}%
\pgfsetfillcolor{currentfill}%
\pgfsetlinewidth{1.003750pt}%
\definecolor{currentstroke}{rgb}{0.121569,0.466667,0.705882}%
\pgfsetstrokecolor{currentstroke}%
\pgfsetdash{}{0pt}%
\pgfpathmoveto{\pgfqpoint{3.882548in}{1.985225in}}%
\pgfpathcurveto{\pgfqpoint{3.893598in}{1.985225in}}{\pgfqpoint{3.904197in}{1.989615in}}{\pgfqpoint{3.912011in}{1.997429in}}%
\pgfpathcurveto{\pgfqpoint{3.919825in}{2.005242in}}{\pgfqpoint{3.924215in}{2.015841in}}{\pgfqpoint{3.924215in}{2.026891in}}%
\pgfpathcurveto{\pgfqpoint{3.924215in}{2.037941in}}{\pgfqpoint{3.919825in}{2.048540in}}{\pgfqpoint{3.912011in}{2.056354in}}%
\pgfpathcurveto{\pgfqpoint{3.904197in}{2.064168in}}{\pgfqpoint{3.893598in}{2.068558in}}{\pgfqpoint{3.882548in}{2.068558in}}%
\pgfpathcurveto{\pgfqpoint{3.871498in}{2.068558in}}{\pgfqpoint{3.860899in}{2.064168in}}{\pgfqpoint{3.853085in}{2.056354in}}%
\pgfpathcurveto{\pgfqpoint{3.845272in}{2.048540in}}{\pgfqpoint{3.840882in}{2.037941in}}{\pgfqpoint{3.840882in}{2.026891in}}%
\pgfpathcurveto{\pgfqpoint{3.840882in}{2.015841in}}{\pgfqpoint{3.845272in}{2.005242in}}{\pgfqpoint{3.853085in}{1.997429in}}%
\pgfpathcurveto{\pgfqpoint{3.860899in}{1.989615in}}{\pgfqpoint{3.871498in}{1.985225in}}{\pgfqpoint{3.882548in}{1.985225in}}%
\pgfpathclose%
\pgfusepath{stroke,fill}%
\end{pgfscope}%
\begin{pgfscope}%
\pgfpathrectangle{\pgfqpoint{0.600000in}{0.600000in}}{\pgfqpoint{3.900000in}{3.900000in}}%
\pgfusepath{clip}%
\pgfsetbuttcap%
\pgfsetroundjoin%
\definecolor{currentfill}{rgb}{0.121569,0.466667,0.705882}%
\pgfsetfillcolor{currentfill}%
\pgfsetlinewidth{1.003750pt}%
\definecolor{currentstroke}{rgb}{0.121569,0.466667,0.705882}%
\pgfsetstrokecolor{currentstroke}%
\pgfsetdash{}{0pt}%
\pgfpathmoveto{\pgfqpoint{2.039501in}{3.439291in}}%
\pgfpathcurveto{\pgfqpoint{2.050551in}{3.439291in}}{\pgfqpoint{2.061150in}{3.443682in}}{\pgfqpoint{2.068963in}{3.451495in}}%
\pgfpathcurveto{\pgfqpoint{2.076777in}{3.459309in}}{\pgfqpoint{2.081167in}{3.469908in}}{\pgfqpoint{2.081167in}{3.480958in}}%
\pgfpathcurveto{\pgfqpoint{2.081167in}{3.492008in}}{\pgfqpoint{2.076777in}{3.502607in}}{\pgfqpoint{2.068963in}{3.510421in}}%
\pgfpathcurveto{\pgfqpoint{2.061150in}{3.518235in}}{\pgfqpoint{2.050551in}{3.522625in}}{\pgfqpoint{2.039501in}{3.522625in}}%
\pgfpathcurveto{\pgfqpoint{2.028450in}{3.522625in}}{\pgfqpoint{2.017851in}{3.518235in}}{\pgfqpoint{2.010038in}{3.510421in}}%
\pgfpathcurveto{\pgfqpoint{2.002224in}{3.502607in}}{\pgfqpoint{1.997834in}{3.492008in}}{\pgfqpoint{1.997834in}{3.480958in}}%
\pgfpathcurveto{\pgfqpoint{1.997834in}{3.469908in}}{\pgfqpoint{2.002224in}{3.459309in}}{\pgfqpoint{2.010038in}{3.451495in}}%
\pgfpathcurveto{\pgfqpoint{2.017851in}{3.443682in}}{\pgfqpoint{2.028450in}{3.439291in}}{\pgfqpoint{2.039501in}{3.439291in}}%
\pgfpathclose%
\pgfusepath{stroke,fill}%
\end{pgfscope}%
\begin{pgfscope}%
\pgfpathrectangle{\pgfqpoint{0.600000in}{0.600000in}}{\pgfqpoint{3.900000in}{3.900000in}}%
\pgfusepath{clip}%
\pgfsetbuttcap%
\pgfsetroundjoin%
\definecolor{currentfill}{rgb}{0.121569,0.466667,0.705882}%
\pgfsetfillcolor{currentfill}%
\pgfsetlinewidth{1.003750pt}%
\definecolor{currentstroke}{rgb}{0.121569,0.466667,0.705882}%
\pgfsetstrokecolor{currentstroke}%
\pgfsetdash{}{0pt}%
\pgfpathmoveto{\pgfqpoint{3.049524in}{2.621238in}}%
\pgfpathcurveto{\pgfqpoint{3.060574in}{2.621238in}}{\pgfqpoint{3.071173in}{2.625628in}}{\pgfqpoint{3.078987in}{2.633442in}}%
\pgfpathcurveto{\pgfqpoint{3.086800in}{2.641255in}}{\pgfqpoint{3.091191in}{2.651854in}}{\pgfqpoint{3.091191in}{2.662904in}}%
\pgfpathcurveto{\pgfqpoint{3.091191in}{2.673955in}}{\pgfqpoint{3.086800in}{2.684554in}}{\pgfqpoint{3.078987in}{2.692367in}}%
\pgfpathcurveto{\pgfqpoint{3.071173in}{2.700181in}}{\pgfqpoint{3.060574in}{2.704571in}}{\pgfqpoint{3.049524in}{2.704571in}}%
\pgfpathcurveto{\pgfqpoint{3.038474in}{2.704571in}}{\pgfqpoint{3.027875in}{2.700181in}}{\pgfqpoint{3.020061in}{2.692367in}}%
\pgfpathcurveto{\pgfqpoint{3.012248in}{2.684554in}}{\pgfqpoint{3.007857in}{2.673955in}}{\pgfqpoint{3.007857in}{2.662904in}}%
\pgfpathcurveto{\pgfqpoint{3.007857in}{2.651854in}}{\pgfqpoint{3.012248in}{2.641255in}}{\pgfqpoint{3.020061in}{2.633442in}}%
\pgfpathcurveto{\pgfqpoint{3.027875in}{2.625628in}}{\pgfqpoint{3.038474in}{2.621238in}}{\pgfqpoint{3.049524in}{2.621238in}}%
\pgfpathclose%
\pgfusepath{stroke,fill}%
\end{pgfscope}%
\begin{pgfscope}%
\pgfpathrectangle{\pgfqpoint{0.600000in}{0.600000in}}{\pgfqpoint{3.900000in}{3.900000in}}%
\pgfusepath{clip}%
\pgfsetbuttcap%
\pgfsetroundjoin%
\definecolor{currentfill}{rgb}{0.121569,0.466667,0.705882}%
\pgfsetfillcolor{currentfill}%
\pgfsetlinewidth{1.003750pt}%
\definecolor{currentstroke}{rgb}{0.121569,0.466667,0.705882}%
\pgfsetstrokecolor{currentstroke}%
\pgfsetdash{}{0pt}%
\pgfpathmoveto{\pgfqpoint{2.113102in}{2.759069in}}%
\pgfpathcurveto{\pgfqpoint{2.124153in}{2.759069in}}{\pgfqpoint{2.134752in}{2.763460in}}{\pgfqpoint{2.142565in}{2.771273in}}%
\pgfpathcurveto{\pgfqpoint{2.150379in}{2.779087in}}{\pgfqpoint{2.154769in}{2.789686in}}{\pgfqpoint{2.154769in}{2.800736in}}%
\pgfpathcurveto{\pgfqpoint{2.154769in}{2.811786in}}{\pgfqpoint{2.150379in}{2.822385in}}{\pgfqpoint{2.142565in}{2.830199in}}%
\pgfpathcurveto{\pgfqpoint{2.134752in}{2.838013in}}{\pgfqpoint{2.124153in}{2.842403in}}{\pgfqpoint{2.113102in}{2.842403in}}%
\pgfpathcurveto{\pgfqpoint{2.102052in}{2.842403in}}{\pgfqpoint{2.091453in}{2.838013in}}{\pgfqpoint{2.083640in}{2.830199in}}%
\pgfpathcurveto{\pgfqpoint{2.075826in}{2.822385in}}{\pgfqpoint{2.071436in}{2.811786in}}{\pgfqpoint{2.071436in}{2.800736in}}%
\pgfpathcurveto{\pgfqpoint{2.071436in}{2.789686in}}{\pgfqpoint{2.075826in}{2.779087in}}{\pgfqpoint{2.083640in}{2.771273in}}%
\pgfpathcurveto{\pgfqpoint{2.091453in}{2.763460in}}{\pgfqpoint{2.102052in}{2.759069in}}{\pgfqpoint{2.113102in}{2.759069in}}%
\pgfpathclose%
\pgfusepath{stroke,fill}%
\end{pgfscope}%
\begin{pgfscope}%
\pgfpathrectangle{\pgfqpoint{0.600000in}{0.600000in}}{\pgfqpoint{3.900000in}{3.900000in}}%
\pgfusepath{clip}%
\pgfsetbuttcap%
\pgfsetroundjoin%
\definecolor{currentfill}{rgb}{0.121569,0.466667,0.705882}%
\pgfsetfillcolor{currentfill}%
\pgfsetlinewidth{1.003750pt}%
\definecolor{currentstroke}{rgb}{0.121569,0.466667,0.705882}%
\pgfsetstrokecolor{currentstroke}%
\pgfsetdash{}{0pt}%
\pgfpathmoveto{\pgfqpoint{3.203408in}{3.189369in}}%
\pgfpathcurveto{\pgfqpoint{3.214458in}{3.189369in}}{\pgfqpoint{3.225057in}{3.193759in}}{\pgfqpoint{3.232871in}{3.201573in}}%
\pgfpathcurveto{\pgfqpoint{3.240684in}{3.209386in}}{\pgfqpoint{3.245075in}{3.219985in}}{\pgfqpoint{3.245075in}{3.231036in}}%
\pgfpathcurveto{\pgfqpoint{3.245075in}{3.242086in}}{\pgfqpoint{3.240684in}{3.252685in}}{\pgfqpoint{3.232871in}{3.260498in}}%
\pgfpathcurveto{\pgfqpoint{3.225057in}{3.268312in}}{\pgfqpoint{3.214458in}{3.272702in}}{\pgfqpoint{3.203408in}{3.272702in}}%
\pgfpathcurveto{\pgfqpoint{3.192358in}{3.272702in}}{\pgfqpoint{3.181759in}{3.268312in}}{\pgfqpoint{3.173945in}{3.260498in}}%
\pgfpathcurveto{\pgfqpoint{3.166132in}{3.252685in}}{\pgfqpoint{3.161741in}{3.242086in}}{\pgfqpoint{3.161741in}{3.231036in}}%
\pgfpathcurveto{\pgfqpoint{3.161741in}{3.219985in}}{\pgfqpoint{3.166132in}{3.209386in}}{\pgfqpoint{3.173945in}{3.201573in}}%
\pgfpathcurveto{\pgfqpoint{3.181759in}{3.193759in}}{\pgfqpoint{3.192358in}{3.189369in}}{\pgfqpoint{3.203408in}{3.189369in}}%
\pgfpathclose%
\pgfusepath{stroke,fill}%
\end{pgfscope}%
\begin{pgfscope}%
\pgfpathrectangle{\pgfqpoint{0.600000in}{0.600000in}}{\pgfqpoint{3.900000in}{3.900000in}}%
\pgfusepath{clip}%
\pgfsetbuttcap%
\pgfsetroundjoin%
\definecolor{currentfill}{rgb}{0.121569,0.466667,0.705882}%
\pgfsetfillcolor{currentfill}%
\pgfsetlinewidth{1.003750pt}%
\definecolor{currentstroke}{rgb}{0.121569,0.466667,0.705882}%
\pgfsetstrokecolor{currentstroke}%
\pgfsetdash{}{0pt}%
\pgfpathmoveto{\pgfqpoint{3.004118in}{2.748147in}}%
\pgfpathcurveto{\pgfqpoint{3.015168in}{2.748147in}}{\pgfqpoint{3.025767in}{2.752538in}}{\pgfqpoint{3.033581in}{2.760351in}}%
\pgfpathcurveto{\pgfqpoint{3.041394in}{2.768165in}}{\pgfqpoint{3.045785in}{2.778764in}}{\pgfqpoint{3.045785in}{2.789814in}}%
\pgfpathcurveto{\pgfqpoint{3.045785in}{2.800864in}}{\pgfqpoint{3.041394in}{2.811463in}}{\pgfqpoint{3.033581in}{2.819277in}}%
\pgfpathcurveto{\pgfqpoint{3.025767in}{2.827090in}}{\pgfqpoint{3.015168in}{2.831481in}}{\pgfqpoint{3.004118in}{2.831481in}}%
\pgfpathcurveto{\pgfqpoint{2.993068in}{2.831481in}}{\pgfqpoint{2.982469in}{2.827090in}}{\pgfqpoint{2.974655in}{2.819277in}}%
\pgfpathcurveto{\pgfqpoint{2.966842in}{2.811463in}}{\pgfqpoint{2.962451in}{2.800864in}}{\pgfqpoint{2.962451in}{2.789814in}}%
\pgfpathcurveto{\pgfqpoint{2.962451in}{2.778764in}}{\pgfqpoint{2.966842in}{2.768165in}}{\pgfqpoint{2.974655in}{2.760351in}}%
\pgfpathcurveto{\pgfqpoint{2.982469in}{2.752538in}}{\pgfqpoint{2.993068in}{2.748147in}}{\pgfqpoint{3.004118in}{2.748147in}}%
\pgfpathclose%
\pgfusepath{stroke,fill}%
\end{pgfscope}%
\begin{pgfscope}%
\pgfpathrectangle{\pgfqpoint{0.600000in}{0.600000in}}{\pgfqpoint{3.900000in}{3.900000in}}%
\pgfusepath{clip}%
\pgfsetbuttcap%
\pgfsetroundjoin%
\definecolor{currentfill}{rgb}{0.121569,0.466667,0.705882}%
\pgfsetfillcolor{currentfill}%
\pgfsetlinewidth{1.003750pt}%
\definecolor{currentstroke}{rgb}{0.121569,0.466667,0.705882}%
\pgfsetstrokecolor{currentstroke}%
\pgfsetdash{}{0pt}%
\pgfpathmoveto{\pgfqpoint{2.584094in}{2.496126in}}%
\pgfpathcurveto{\pgfqpoint{2.595144in}{2.496126in}}{\pgfqpoint{2.605743in}{2.500516in}}{\pgfqpoint{2.613557in}{2.508330in}}%
\pgfpathcurveto{\pgfqpoint{2.621370in}{2.516143in}}{\pgfqpoint{2.625760in}{2.526742in}}{\pgfqpoint{2.625760in}{2.537792in}}%
\pgfpathcurveto{\pgfqpoint{2.625760in}{2.548842in}}{\pgfqpoint{2.621370in}{2.559442in}}{\pgfqpoint{2.613557in}{2.567255in}}%
\pgfpathcurveto{\pgfqpoint{2.605743in}{2.575069in}}{\pgfqpoint{2.595144in}{2.579459in}}{\pgfqpoint{2.584094in}{2.579459in}}%
\pgfpathcurveto{\pgfqpoint{2.573044in}{2.579459in}}{\pgfqpoint{2.562445in}{2.575069in}}{\pgfqpoint{2.554631in}{2.567255in}}%
\pgfpathcurveto{\pgfqpoint{2.546817in}{2.559442in}}{\pgfqpoint{2.542427in}{2.548842in}}{\pgfqpoint{2.542427in}{2.537792in}}%
\pgfpathcurveto{\pgfqpoint{2.542427in}{2.526742in}}{\pgfqpoint{2.546817in}{2.516143in}}{\pgfqpoint{2.554631in}{2.508330in}}%
\pgfpathcurveto{\pgfqpoint{2.562445in}{2.500516in}}{\pgfqpoint{2.573044in}{2.496126in}}{\pgfqpoint{2.584094in}{2.496126in}}%
\pgfpathclose%
\pgfusepath{stroke,fill}%
\end{pgfscope}%
\begin{pgfscope}%
\pgfpathrectangle{\pgfqpoint{0.600000in}{0.600000in}}{\pgfqpoint{3.900000in}{3.900000in}}%
\pgfusepath{clip}%
\pgfsetbuttcap%
\pgfsetroundjoin%
\definecolor{currentfill}{rgb}{0.121569,0.466667,0.705882}%
\pgfsetfillcolor{currentfill}%
\pgfsetlinewidth{1.003750pt}%
\definecolor{currentstroke}{rgb}{0.121569,0.466667,0.705882}%
\pgfsetstrokecolor{currentstroke}%
\pgfsetdash{}{0pt}%
\pgfpathmoveto{\pgfqpoint{2.760850in}{2.268988in}}%
\pgfpathcurveto{\pgfqpoint{2.771901in}{2.268988in}}{\pgfqpoint{2.782500in}{2.273378in}}{\pgfqpoint{2.790313in}{2.281192in}}%
\pgfpathcurveto{\pgfqpoint{2.798127in}{2.289006in}}{\pgfqpoint{2.802517in}{2.299605in}}{\pgfqpoint{2.802517in}{2.310655in}}%
\pgfpathcurveto{\pgfqpoint{2.802517in}{2.321705in}}{\pgfqpoint{2.798127in}{2.332304in}}{\pgfqpoint{2.790313in}{2.340118in}}%
\pgfpathcurveto{\pgfqpoint{2.782500in}{2.347931in}}{\pgfqpoint{2.771901in}{2.352322in}}{\pgfqpoint{2.760850in}{2.352322in}}%
\pgfpathcurveto{\pgfqpoint{2.749800in}{2.352322in}}{\pgfqpoint{2.739201in}{2.347931in}}{\pgfqpoint{2.731388in}{2.340118in}}%
\pgfpathcurveto{\pgfqpoint{2.723574in}{2.332304in}}{\pgfqpoint{2.719184in}{2.321705in}}{\pgfqpoint{2.719184in}{2.310655in}}%
\pgfpathcurveto{\pgfqpoint{2.719184in}{2.299605in}}{\pgfqpoint{2.723574in}{2.289006in}}{\pgfqpoint{2.731388in}{2.281192in}}%
\pgfpathcurveto{\pgfqpoint{2.739201in}{2.273378in}}{\pgfqpoint{2.749800in}{2.268988in}}{\pgfqpoint{2.760850in}{2.268988in}}%
\pgfpathclose%
\pgfusepath{stroke,fill}%
\end{pgfscope}%
\begin{pgfscope}%
\pgfpathrectangle{\pgfqpoint{0.600000in}{0.600000in}}{\pgfqpoint{3.900000in}{3.900000in}}%
\pgfusepath{clip}%
\pgfsetbuttcap%
\pgfsetroundjoin%
\definecolor{currentfill}{rgb}{0.121569,0.466667,0.705882}%
\pgfsetfillcolor{currentfill}%
\pgfsetlinewidth{1.003750pt}%
\definecolor{currentstroke}{rgb}{0.121569,0.466667,0.705882}%
\pgfsetstrokecolor{currentstroke}%
\pgfsetdash{}{0pt}%
\pgfpathmoveto{\pgfqpoint{3.073460in}{2.656399in}}%
\pgfpathcurveto{\pgfqpoint{3.084510in}{2.656399in}}{\pgfqpoint{3.095109in}{2.660789in}}{\pgfqpoint{3.102923in}{2.668602in}}%
\pgfpathcurveto{\pgfqpoint{3.110736in}{2.676416in}}{\pgfqpoint{3.115126in}{2.687015in}}{\pgfqpoint{3.115126in}{2.698065in}}%
\pgfpathcurveto{\pgfqpoint{3.115126in}{2.709115in}}{\pgfqpoint{3.110736in}{2.719714in}}{\pgfqpoint{3.102923in}{2.727528in}}%
\pgfpathcurveto{\pgfqpoint{3.095109in}{2.735342in}}{\pgfqpoint{3.084510in}{2.739732in}}{\pgfqpoint{3.073460in}{2.739732in}}%
\pgfpathcurveto{\pgfqpoint{3.062410in}{2.739732in}}{\pgfqpoint{3.051811in}{2.735342in}}{\pgfqpoint{3.043997in}{2.727528in}}%
\pgfpathcurveto{\pgfqpoint{3.036183in}{2.719714in}}{\pgfqpoint{3.031793in}{2.709115in}}{\pgfqpoint{3.031793in}{2.698065in}}%
\pgfpathcurveto{\pgfqpoint{3.031793in}{2.687015in}}{\pgfqpoint{3.036183in}{2.676416in}}{\pgfqpoint{3.043997in}{2.668602in}}%
\pgfpathcurveto{\pgfqpoint{3.051811in}{2.660789in}}{\pgfqpoint{3.062410in}{2.656399in}}{\pgfqpoint{3.073460in}{2.656399in}}%
\pgfpathclose%
\pgfusepath{stroke,fill}%
\end{pgfscope}%
\begin{pgfscope}%
\pgfpathrectangle{\pgfqpoint{0.600000in}{0.600000in}}{\pgfqpoint{3.900000in}{3.900000in}}%
\pgfusepath{clip}%
\pgfsetbuttcap%
\pgfsetroundjoin%
\definecolor{currentfill}{rgb}{0.121569,0.466667,0.705882}%
\pgfsetfillcolor{currentfill}%
\pgfsetlinewidth{1.003750pt}%
\definecolor{currentstroke}{rgb}{0.121569,0.466667,0.705882}%
\pgfsetstrokecolor{currentstroke}%
\pgfsetdash{}{0pt}%
\pgfpathmoveto{\pgfqpoint{2.523724in}{2.666353in}}%
\pgfpathcurveto{\pgfqpoint{2.534774in}{2.666353in}}{\pgfqpoint{2.545373in}{2.670743in}}{\pgfqpoint{2.553187in}{2.678557in}}%
\pgfpathcurveto{\pgfqpoint{2.561001in}{2.686370in}}{\pgfqpoint{2.565391in}{2.696969in}}{\pgfqpoint{2.565391in}{2.708020in}}%
\pgfpathcurveto{\pgfqpoint{2.565391in}{2.719070in}}{\pgfqpoint{2.561001in}{2.729669in}}{\pgfqpoint{2.553187in}{2.737482in}}%
\pgfpathcurveto{\pgfqpoint{2.545373in}{2.745296in}}{\pgfqpoint{2.534774in}{2.749686in}}{\pgfqpoint{2.523724in}{2.749686in}}%
\pgfpathcurveto{\pgfqpoint{2.512674in}{2.749686in}}{\pgfqpoint{2.502075in}{2.745296in}}{\pgfqpoint{2.494261in}{2.737482in}}%
\pgfpathcurveto{\pgfqpoint{2.486448in}{2.729669in}}{\pgfqpoint{2.482058in}{2.719070in}}{\pgfqpoint{2.482058in}{2.708020in}}%
\pgfpathcurveto{\pgfqpoint{2.482058in}{2.696969in}}{\pgfqpoint{2.486448in}{2.686370in}}{\pgfqpoint{2.494261in}{2.678557in}}%
\pgfpathcurveto{\pgfqpoint{2.502075in}{2.670743in}}{\pgfqpoint{2.512674in}{2.666353in}}{\pgfqpoint{2.523724in}{2.666353in}}%
\pgfpathclose%
\pgfusepath{stroke,fill}%
\end{pgfscope}%
\begin{pgfscope}%
\pgfpathrectangle{\pgfqpoint{0.600000in}{0.600000in}}{\pgfqpoint{3.900000in}{3.900000in}}%
\pgfusepath{clip}%
\pgfsetbuttcap%
\pgfsetroundjoin%
\definecolor{currentfill}{rgb}{0.121569,0.466667,0.705882}%
\pgfsetfillcolor{currentfill}%
\pgfsetlinewidth{1.003750pt}%
\definecolor{currentstroke}{rgb}{0.121569,0.466667,0.705882}%
\pgfsetstrokecolor{currentstroke}%
\pgfsetdash{}{0pt}%
\pgfpathmoveto{\pgfqpoint{3.108888in}{2.453849in}}%
\pgfpathcurveto{\pgfqpoint{3.119938in}{2.453849in}}{\pgfqpoint{3.130537in}{2.458239in}}{\pgfqpoint{3.138351in}{2.466052in}}%
\pgfpathcurveto{\pgfqpoint{3.146165in}{2.473866in}}{\pgfqpoint{3.150555in}{2.484465in}}{\pgfqpoint{3.150555in}{2.495515in}}%
\pgfpathcurveto{\pgfqpoint{3.150555in}{2.506565in}}{\pgfqpoint{3.146165in}{2.517164in}}{\pgfqpoint{3.138351in}{2.524978in}}%
\pgfpathcurveto{\pgfqpoint{3.130537in}{2.532792in}}{\pgfqpoint{3.119938in}{2.537182in}}{\pgfqpoint{3.108888in}{2.537182in}}%
\pgfpathcurveto{\pgfqpoint{3.097838in}{2.537182in}}{\pgfqpoint{3.087239in}{2.532792in}}{\pgfqpoint{3.079425in}{2.524978in}}%
\pgfpathcurveto{\pgfqpoint{3.071612in}{2.517164in}}{\pgfqpoint{3.067222in}{2.506565in}}{\pgfqpoint{3.067222in}{2.495515in}}%
\pgfpathcurveto{\pgfqpoint{3.067222in}{2.484465in}}{\pgfqpoint{3.071612in}{2.473866in}}{\pgfqpoint{3.079425in}{2.466052in}}%
\pgfpathcurveto{\pgfqpoint{3.087239in}{2.458239in}}{\pgfqpoint{3.097838in}{2.453849in}}{\pgfqpoint{3.108888in}{2.453849in}}%
\pgfpathclose%
\pgfusepath{stroke,fill}%
\end{pgfscope}%
\begin{pgfscope}%
\pgfpathrectangle{\pgfqpoint{0.600000in}{0.600000in}}{\pgfqpoint{3.900000in}{3.900000in}}%
\pgfusepath{clip}%
\pgfsetbuttcap%
\pgfsetroundjoin%
\definecolor{currentfill}{rgb}{0.121569,0.466667,0.705882}%
\pgfsetfillcolor{currentfill}%
\pgfsetlinewidth{1.003750pt}%
\definecolor{currentstroke}{rgb}{0.121569,0.466667,0.705882}%
\pgfsetstrokecolor{currentstroke}%
\pgfsetdash{}{0pt}%
\pgfpathmoveto{\pgfqpoint{2.309747in}{2.835405in}}%
\pgfpathcurveto{\pgfqpoint{2.320798in}{2.835405in}}{\pgfqpoint{2.331397in}{2.839795in}}{\pgfqpoint{2.339210in}{2.847609in}}%
\pgfpathcurveto{\pgfqpoint{2.347024in}{2.855422in}}{\pgfqpoint{2.351414in}{2.866022in}}{\pgfqpoint{2.351414in}{2.877072in}}%
\pgfpathcurveto{\pgfqpoint{2.351414in}{2.888122in}}{\pgfqpoint{2.347024in}{2.898721in}}{\pgfqpoint{2.339210in}{2.906534in}}%
\pgfpathcurveto{\pgfqpoint{2.331397in}{2.914348in}}{\pgfqpoint{2.320798in}{2.918738in}}{\pgfqpoint{2.309747in}{2.918738in}}%
\pgfpathcurveto{\pgfqpoint{2.298697in}{2.918738in}}{\pgfqpoint{2.288098in}{2.914348in}}{\pgfqpoint{2.280285in}{2.906534in}}%
\pgfpathcurveto{\pgfqpoint{2.272471in}{2.898721in}}{\pgfqpoint{2.268081in}{2.888122in}}{\pgfqpoint{2.268081in}{2.877072in}}%
\pgfpathcurveto{\pgfqpoint{2.268081in}{2.866022in}}{\pgfqpoint{2.272471in}{2.855422in}}{\pgfqpoint{2.280285in}{2.847609in}}%
\pgfpathcurveto{\pgfqpoint{2.288098in}{2.839795in}}{\pgfqpoint{2.298697in}{2.835405in}}{\pgfqpoint{2.309747in}{2.835405in}}%
\pgfpathclose%
\pgfusepath{stroke,fill}%
\end{pgfscope}%
\begin{pgfscope}%
\pgfpathrectangle{\pgfqpoint{0.600000in}{0.600000in}}{\pgfqpoint{3.900000in}{3.900000in}}%
\pgfusepath{clip}%
\pgfsetbuttcap%
\pgfsetroundjoin%
\definecolor{currentfill}{rgb}{0.121569,0.466667,0.705882}%
\pgfsetfillcolor{currentfill}%
\pgfsetlinewidth{1.003750pt}%
\definecolor{currentstroke}{rgb}{0.121569,0.466667,0.705882}%
\pgfsetstrokecolor{currentstroke}%
\pgfsetdash{}{0pt}%
\pgfpathmoveto{\pgfqpoint{2.910461in}{2.756704in}}%
\pgfpathcurveto{\pgfqpoint{2.921511in}{2.756704in}}{\pgfqpoint{2.932110in}{2.761094in}}{\pgfqpoint{2.939923in}{2.768908in}}%
\pgfpathcurveto{\pgfqpoint{2.947737in}{2.776721in}}{\pgfqpoint{2.952127in}{2.787320in}}{\pgfqpoint{2.952127in}{2.798370in}}%
\pgfpathcurveto{\pgfqpoint{2.952127in}{2.809421in}}{\pgfqpoint{2.947737in}{2.820020in}}{\pgfqpoint{2.939923in}{2.827833in}}%
\pgfpathcurveto{\pgfqpoint{2.932110in}{2.835647in}}{\pgfqpoint{2.921511in}{2.840037in}}{\pgfqpoint{2.910461in}{2.840037in}}%
\pgfpathcurveto{\pgfqpoint{2.899411in}{2.840037in}}{\pgfqpoint{2.888812in}{2.835647in}}{\pgfqpoint{2.880998in}{2.827833in}}%
\pgfpathcurveto{\pgfqpoint{2.873184in}{2.820020in}}{\pgfqpoint{2.868794in}{2.809421in}}{\pgfqpoint{2.868794in}{2.798370in}}%
\pgfpathcurveto{\pgfqpoint{2.868794in}{2.787320in}}{\pgfqpoint{2.873184in}{2.776721in}}{\pgfqpoint{2.880998in}{2.768908in}}%
\pgfpathcurveto{\pgfqpoint{2.888812in}{2.761094in}}{\pgfqpoint{2.899411in}{2.756704in}}{\pgfqpoint{2.910461in}{2.756704in}}%
\pgfpathclose%
\pgfusepath{stroke,fill}%
\end{pgfscope}%
\begin{pgfscope}%
\pgfpathrectangle{\pgfqpoint{0.600000in}{0.600000in}}{\pgfqpoint{3.900000in}{3.900000in}}%
\pgfusepath{clip}%
\pgfsetbuttcap%
\pgfsetroundjoin%
\definecolor{currentfill}{rgb}{0.121569,0.466667,0.705882}%
\pgfsetfillcolor{currentfill}%
\pgfsetlinewidth{1.003750pt}%
\definecolor{currentstroke}{rgb}{0.121569,0.466667,0.705882}%
\pgfsetstrokecolor{currentstroke}%
\pgfsetdash{}{0pt}%
\pgfpathmoveto{\pgfqpoint{1.984257in}{2.472656in}}%
\pgfpathcurveto{\pgfqpoint{1.995307in}{2.472656in}}{\pgfqpoint{2.005906in}{2.477046in}}{\pgfqpoint{2.013720in}{2.484860in}}%
\pgfpathcurveto{\pgfqpoint{2.021534in}{2.492673in}}{\pgfqpoint{2.025924in}{2.503272in}}{\pgfqpoint{2.025924in}{2.514322in}}%
\pgfpathcurveto{\pgfqpoint{2.025924in}{2.525372in}}{\pgfqpoint{2.021534in}{2.535972in}}{\pgfqpoint{2.013720in}{2.543785in}}%
\pgfpathcurveto{\pgfqpoint{2.005906in}{2.551599in}}{\pgfqpoint{1.995307in}{2.555989in}}{\pgfqpoint{1.984257in}{2.555989in}}%
\pgfpathcurveto{\pgfqpoint{1.973207in}{2.555989in}}{\pgfqpoint{1.962608in}{2.551599in}}{\pgfqpoint{1.954795in}{2.543785in}}%
\pgfpathcurveto{\pgfqpoint{1.946981in}{2.535972in}}{\pgfqpoint{1.942591in}{2.525372in}}{\pgfqpoint{1.942591in}{2.514322in}}%
\pgfpathcurveto{\pgfqpoint{1.942591in}{2.503272in}}{\pgfqpoint{1.946981in}{2.492673in}}{\pgfqpoint{1.954795in}{2.484860in}}%
\pgfpathcurveto{\pgfqpoint{1.962608in}{2.477046in}}{\pgfqpoint{1.973207in}{2.472656in}}{\pgfqpoint{1.984257in}{2.472656in}}%
\pgfpathclose%
\pgfusepath{stroke,fill}%
\end{pgfscope}%
\begin{pgfscope}%
\pgfpathrectangle{\pgfqpoint{0.600000in}{0.600000in}}{\pgfqpoint{3.900000in}{3.900000in}}%
\pgfusepath{clip}%
\pgfsetbuttcap%
\pgfsetroundjoin%
\definecolor{currentfill}{rgb}{0.121569,0.466667,0.705882}%
\pgfsetfillcolor{currentfill}%
\pgfsetlinewidth{1.003750pt}%
\definecolor{currentstroke}{rgb}{0.121569,0.466667,0.705882}%
\pgfsetstrokecolor{currentstroke}%
\pgfsetdash{}{0pt}%
\pgfpathmoveto{\pgfqpoint{3.321016in}{2.501806in}}%
\pgfpathcurveto{\pgfqpoint{3.332067in}{2.501806in}}{\pgfqpoint{3.342666in}{2.506196in}}{\pgfqpoint{3.350479in}{2.514009in}}%
\pgfpathcurveto{\pgfqpoint{3.358293in}{2.521823in}}{\pgfqpoint{3.362683in}{2.532422in}}{\pgfqpoint{3.362683in}{2.543472in}}%
\pgfpathcurveto{\pgfqpoint{3.362683in}{2.554522in}}{\pgfqpoint{3.358293in}{2.565121in}}{\pgfqpoint{3.350479in}{2.572935in}}%
\pgfpathcurveto{\pgfqpoint{3.342666in}{2.580749in}}{\pgfqpoint{3.332067in}{2.585139in}}{\pgfqpoint{3.321016in}{2.585139in}}%
\pgfpathcurveto{\pgfqpoint{3.309966in}{2.585139in}}{\pgfqpoint{3.299367in}{2.580749in}}{\pgfqpoint{3.291554in}{2.572935in}}%
\pgfpathcurveto{\pgfqpoint{3.283740in}{2.565121in}}{\pgfqpoint{3.279350in}{2.554522in}}{\pgfqpoint{3.279350in}{2.543472in}}%
\pgfpathcurveto{\pgfqpoint{3.279350in}{2.532422in}}{\pgfqpoint{3.283740in}{2.521823in}}{\pgfqpoint{3.291554in}{2.514009in}}%
\pgfpathcurveto{\pgfqpoint{3.299367in}{2.506196in}}{\pgfqpoint{3.309966in}{2.501806in}}{\pgfqpoint{3.321016in}{2.501806in}}%
\pgfpathclose%
\pgfusepath{stroke,fill}%
\end{pgfscope}%
\begin{pgfscope}%
\pgfpathrectangle{\pgfqpoint{0.600000in}{0.600000in}}{\pgfqpoint{3.900000in}{3.900000in}}%
\pgfusepath{clip}%
\pgfsetbuttcap%
\pgfsetroundjoin%
\definecolor{currentfill}{rgb}{0.121569,0.466667,0.705882}%
\pgfsetfillcolor{currentfill}%
\pgfsetlinewidth{1.003750pt}%
\definecolor{currentstroke}{rgb}{0.121569,0.466667,0.705882}%
\pgfsetstrokecolor{currentstroke}%
\pgfsetdash{}{0pt}%
\pgfpathmoveto{\pgfqpoint{3.597349in}{1.656988in}}%
\pgfpathcurveto{\pgfqpoint{3.608399in}{1.656988in}}{\pgfqpoint{3.618999in}{1.661378in}}{\pgfqpoint{3.626812in}{1.669191in}}%
\pgfpathcurveto{\pgfqpoint{3.634626in}{1.677005in}}{\pgfqpoint{3.639016in}{1.687604in}}{\pgfqpoint{3.639016in}{1.698654in}}%
\pgfpathcurveto{\pgfqpoint{3.639016in}{1.709704in}}{\pgfqpoint{3.634626in}{1.720303in}}{\pgfqpoint{3.626812in}{1.728117in}}%
\pgfpathcurveto{\pgfqpoint{3.618999in}{1.735931in}}{\pgfqpoint{3.608399in}{1.740321in}}{\pgfqpoint{3.597349in}{1.740321in}}%
\pgfpathcurveto{\pgfqpoint{3.586299in}{1.740321in}}{\pgfqpoint{3.575700in}{1.735931in}}{\pgfqpoint{3.567887in}{1.728117in}}%
\pgfpathcurveto{\pgfqpoint{3.560073in}{1.720303in}}{\pgfqpoint{3.555683in}{1.709704in}}{\pgfqpoint{3.555683in}{1.698654in}}%
\pgfpathcurveto{\pgfqpoint{3.555683in}{1.687604in}}{\pgfqpoint{3.560073in}{1.677005in}}{\pgfqpoint{3.567887in}{1.669191in}}%
\pgfpathcurveto{\pgfqpoint{3.575700in}{1.661378in}}{\pgfqpoint{3.586299in}{1.656988in}}{\pgfqpoint{3.597349in}{1.656988in}}%
\pgfpathclose%
\pgfusepath{stroke,fill}%
\end{pgfscope}%
\begin{pgfscope}%
\pgfpathrectangle{\pgfqpoint{0.600000in}{0.600000in}}{\pgfqpoint{3.900000in}{3.900000in}}%
\pgfusepath{clip}%
\pgfsetbuttcap%
\pgfsetroundjoin%
\definecolor{currentfill}{rgb}{0.121569,0.466667,0.705882}%
\pgfsetfillcolor{currentfill}%
\pgfsetlinewidth{1.003750pt}%
\definecolor{currentstroke}{rgb}{0.121569,0.466667,0.705882}%
\pgfsetstrokecolor{currentstroke}%
\pgfsetdash{}{0pt}%
\pgfpathmoveto{\pgfqpoint{2.446245in}{2.470795in}}%
\pgfpathcurveto{\pgfqpoint{2.457295in}{2.470795in}}{\pgfqpoint{2.467894in}{2.475186in}}{\pgfqpoint{2.475708in}{2.482999in}}%
\pgfpathcurveto{\pgfqpoint{2.483522in}{2.490813in}}{\pgfqpoint{2.487912in}{2.501412in}}{\pgfqpoint{2.487912in}{2.512462in}}%
\pgfpathcurveto{\pgfqpoint{2.487912in}{2.523512in}}{\pgfqpoint{2.483522in}{2.534111in}}{\pgfqpoint{2.475708in}{2.541925in}}%
\pgfpathcurveto{\pgfqpoint{2.467894in}{2.549739in}}{\pgfqpoint{2.457295in}{2.554129in}}{\pgfqpoint{2.446245in}{2.554129in}}%
\pgfpathcurveto{\pgfqpoint{2.435195in}{2.554129in}}{\pgfqpoint{2.424596in}{2.549739in}}{\pgfqpoint{2.416783in}{2.541925in}}%
\pgfpathcurveto{\pgfqpoint{2.408969in}{2.534111in}}{\pgfqpoint{2.404579in}{2.523512in}}{\pgfqpoint{2.404579in}{2.512462in}}%
\pgfpathcurveto{\pgfqpoint{2.404579in}{2.501412in}}{\pgfqpoint{2.408969in}{2.490813in}}{\pgfqpoint{2.416783in}{2.482999in}}%
\pgfpathcurveto{\pgfqpoint{2.424596in}{2.475186in}}{\pgfqpoint{2.435195in}{2.470795in}}{\pgfqpoint{2.446245in}{2.470795in}}%
\pgfpathclose%
\pgfusepath{stroke,fill}%
\end{pgfscope}%
\begin{pgfscope}%
\pgfpathrectangle{\pgfqpoint{0.600000in}{0.600000in}}{\pgfqpoint{3.900000in}{3.900000in}}%
\pgfusepath{clip}%
\pgfsetbuttcap%
\pgfsetroundjoin%
\definecolor{currentfill}{rgb}{0.121569,0.466667,0.705882}%
\pgfsetfillcolor{currentfill}%
\pgfsetlinewidth{1.003750pt}%
\definecolor{currentstroke}{rgb}{0.121569,0.466667,0.705882}%
\pgfsetstrokecolor{currentstroke}%
\pgfsetdash{}{0pt}%
\pgfpathmoveto{\pgfqpoint{1.712304in}{3.451071in}}%
\pgfpathcurveto{\pgfqpoint{1.723354in}{3.451071in}}{\pgfqpoint{1.733953in}{3.455461in}}{\pgfqpoint{1.741766in}{3.463275in}}%
\pgfpathcurveto{\pgfqpoint{1.749580in}{3.471088in}}{\pgfqpoint{1.753970in}{3.481687in}}{\pgfqpoint{1.753970in}{3.492737in}}%
\pgfpathcurveto{\pgfqpoint{1.753970in}{3.503788in}}{\pgfqpoint{1.749580in}{3.514387in}}{\pgfqpoint{1.741766in}{3.522200in}}%
\pgfpathcurveto{\pgfqpoint{1.733953in}{3.530014in}}{\pgfqpoint{1.723354in}{3.534404in}}{\pgfqpoint{1.712304in}{3.534404in}}%
\pgfpathcurveto{\pgfqpoint{1.701254in}{3.534404in}}{\pgfqpoint{1.690654in}{3.530014in}}{\pgfqpoint{1.682841in}{3.522200in}}%
\pgfpathcurveto{\pgfqpoint{1.675027in}{3.514387in}}{\pgfqpoint{1.670637in}{3.503788in}}{\pgfqpoint{1.670637in}{3.492737in}}%
\pgfpathcurveto{\pgfqpoint{1.670637in}{3.481687in}}{\pgfqpoint{1.675027in}{3.471088in}}{\pgfqpoint{1.682841in}{3.463275in}}%
\pgfpathcurveto{\pgfqpoint{1.690654in}{3.455461in}}{\pgfqpoint{1.701254in}{3.451071in}}{\pgfqpoint{1.712304in}{3.451071in}}%
\pgfpathclose%
\pgfusepath{stroke,fill}%
\end{pgfscope}%
\begin{pgfscope}%
\pgfpathrectangle{\pgfqpoint{0.600000in}{0.600000in}}{\pgfqpoint{3.900000in}{3.900000in}}%
\pgfusepath{clip}%
\pgfsetbuttcap%
\pgfsetroundjoin%
\definecolor{currentfill}{rgb}{0.121569,0.466667,0.705882}%
\pgfsetfillcolor{currentfill}%
\pgfsetlinewidth{1.003750pt}%
\definecolor{currentstroke}{rgb}{0.121569,0.466667,0.705882}%
\pgfsetstrokecolor{currentstroke}%
\pgfsetdash{}{0pt}%
\pgfpathmoveto{\pgfqpoint{2.235431in}{2.641884in}}%
\pgfpathcurveto{\pgfqpoint{2.246481in}{2.641884in}}{\pgfqpoint{2.257080in}{2.646274in}}{\pgfqpoint{2.264893in}{2.654088in}}%
\pgfpathcurveto{\pgfqpoint{2.272707in}{2.661902in}}{\pgfqpoint{2.277097in}{2.672501in}}{\pgfqpoint{2.277097in}{2.683551in}}%
\pgfpathcurveto{\pgfqpoint{2.277097in}{2.694601in}}{\pgfqpoint{2.272707in}{2.705200in}}{\pgfqpoint{2.264893in}{2.713013in}}%
\pgfpathcurveto{\pgfqpoint{2.257080in}{2.720827in}}{\pgfqpoint{2.246481in}{2.725217in}}{\pgfqpoint{2.235431in}{2.725217in}}%
\pgfpathcurveto{\pgfqpoint{2.224381in}{2.725217in}}{\pgfqpoint{2.213782in}{2.720827in}}{\pgfqpoint{2.205968in}{2.713013in}}%
\pgfpathcurveto{\pgfqpoint{2.198154in}{2.705200in}}{\pgfqpoint{2.193764in}{2.694601in}}{\pgfqpoint{2.193764in}{2.683551in}}%
\pgfpathcurveto{\pgfqpoint{2.193764in}{2.672501in}}{\pgfqpoint{2.198154in}{2.661902in}}{\pgfqpoint{2.205968in}{2.654088in}}%
\pgfpathcurveto{\pgfqpoint{2.213782in}{2.646274in}}{\pgfqpoint{2.224381in}{2.641884in}}{\pgfqpoint{2.235431in}{2.641884in}}%
\pgfpathclose%
\pgfusepath{stroke,fill}%
\end{pgfscope}%
\begin{pgfscope}%
\pgfpathrectangle{\pgfqpoint{0.600000in}{0.600000in}}{\pgfqpoint{3.900000in}{3.900000in}}%
\pgfusepath{clip}%
\pgfsetbuttcap%
\pgfsetroundjoin%
\definecolor{currentfill}{rgb}{0.121569,0.466667,0.705882}%
\pgfsetfillcolor{currentfill}%
\pgfsetlinewidth{1.003750pt}%
\definecolor{currentstroke}{rgb}{0.121569,0.466667,0.705882}%
\pgfsetstrokecolor{currentstroke}%
\pgfsetdash{}{0pt}%
\pgfpathmoveto{\pgfqpoint{2.796685in}{1.762682in}}%
\pgfpathcurveto{\pgfqpoint{2.807735in}{1.762682in}}{\pgfqpoint{2.818334in}{1.767073in}}{\pgfqpoint{2.826148in}{1.774886in}}%
\pgfpathcurveto{\pgfqpoint{2.833961in}{1.782700in}}{\pgfqpoint{2.838351in}{1.793299in}}{\pgfqpoint{2.838351in}{1.804349in}}%
\pgfpathcurveto{\pgfqpoint{2.838351in}{1.815399in}}{\pgfqpoint{2.833961in}{1.825998in}}{\pgfqpoint{2.826148in}{1.833812in}}%
\pgfpathcurveto{\pgfqpoint{2.818334in}{1.841625in}}{\pgfqpoint{2.807735in}{1.846016in}}{\pgfqpoint{2.796685in}{1.846016in}}%
\pgfpathcurveto{\pgfqpoint{2.785635in}{1.846016in}}{\pgfqpoint{2.775036in}{1.841625in}}{\pgfqpoint{2.767222in}{1.833812in}}%
\pgfpathcurveto{\pgfqpoint{2.759408in}{1.825998in}}{\pgfqpoint{2.755018in}{1.815399in}}{\pgfqpoint{2.755018in}{1.804349in}}%
\pgfpathcurveto{\pgfqpoint{2.755018in}{1.793299in}}{\pgfqpoint{2.759408in}{1.782700in}}{\pgfqpoint{2.767222in}{1.774886in}}%
\pgfpathcurveto{\pgfqpoint{2.775036in}{1.767073in}}{\pgfqpoint{2.785635in}{1.762682in}}{\pgfqpoint{2.796685in}{1.762682in}}%
\pgfpathclose%
\pgfusepath{stroke,fill}%
\end{pgfscope}%
\begin{pgfscope}%
\pgfpathrectangle{\pgfqpoint{0.600000in}{0.600000in}}{\pgfqpoint{3.900000in}{3.900000in}}%
\pgfusepath{clip}%
\pgfsetbuttcap%
\pgfsetroundjoin%
\definecolor{currentfill}{rgb}{0.121569,0.466667,0.705882}%
\pgfsetfillcolor{currentfill}%
\pgfsetlinewidth{1.003750pt}%
\definecolor{currentstroke}{rgb}{0.121569,0.466667,0.705882}%
\pgfsetstrokecolor{currentstroke}%
\pgfsetdash{}{0pt}%
\pgfpathmoveto{\pgfqpoint{2.940090in}{1.360062in}}%
\pgfpathcurveto{\pgfqpoint{2.951140in}{1.360062in}}{\pgfqpoint{2.961739in}{1.364452in}}{\pgfqpoint{2.969553in}{1.372266in}}%
\pgfpathcurveto{\pgfqpoint{2.977366in}{1.380080in}}{\pgfqpoint{2.981756in}{1.390679in}}{\pgfqpoint{2.981756in}{1.401729in}}%
\pgfpathcurveto{\pgfqpoint{2.981756in}{1.412779in}}{\pgfqpoint{2.977366in}{1.423378in}}{\pgfqpoint{2.969553in}{1.431191in}}%
\pgfpathcurveto{\pgfqpoint{2.961739in}{1.439005in}}{\pgfqpoint{2.951140in}{1.443395in}}{\pgfqpoint{2.940090in}{1.443395in}}%
\pgfpathcurveto{\pgfqpoint{2.929040in}{1.443395in}}{\pgfqpoint{2.918441in}{1.439005in}}{\pgfqpoint{2.910627in}{1.431191in}}%
\pgfpathcurveto{\pgfqpoint{2.902813in}{1.423378in}}{\pgfqpoint{2.898423in}{1.412779in}}{\pgfqpoint{2.898423in}{1.401729in}}%
\pgfpathcurveto{\pgfqpoint{2.898423in}{1.390679in}}{\pgfqpoint{2.902813in}{1.380080in}}{\pgfqpoint{2.910627in}{1.372266in}}%
\pgfpathcurveto{\pgfqpoint{2.918441in}{1.364452in}}{\pgfqpoint{2.929040in}{1.360062in}}{\pgfqpoint{2.940090in}{1.360062in}}%
\pgfpathclose%
\pgfusepath{stroke,fill}%
\end{pgfscope}%
\begin{pgfscope}%
\pgfpathrectangle{\pgfqpoint{0.600000in}{0.600000in}}{\pgfqpoint{3.900000in}{3.900000in}}%
\pgfusepath{clip}%
\pgfsetbuttcap%
\pgfsetroundjoin%
\definecolor{currentfill}{rgb}{0.121569,0.466667,0.705882}%
\pgfsetfillcolor{currentfill}%
\pgfsetlinewidth{1.003750pt}%
\definecolor{currentstroke}{rgb}{0.121569,0.466667,0.705882}%
\pgfsetstrokecolor{currentstroke}%
\pgfsetdash{}{0pt}%
\pgfpathmoveto{\pgfqpoint{3.230126in}{2.847554in}}%
\pgfpathcurveto{\pgfqpoint{3.241176in}{2.847554in}}{\pgfqpoint{3.251775in}{2.851944in}}{\pgfqpoint{3.259588in}{2.859758in}}%
\pgfpathcurveto{\pgfqpoint{3.267402in}{2.867572in}}{\pgfqpoint{3.271792in}{2.878171in}}{\pgfqpoint{3.271792in}{2.889221in}}%
\pgfpathcurveto{\pgfqpoint{3.271792in}{2.900271in}}{\pgfqpoint{3.267402in}{2.910870in}}{\pgfqpoint{3.259588in}{2.918684in}}%
\pgfpathcurveto{\pgfqpoint{3.251775in}{2.926497in}}{\pgfqpoint{3.241176in}{2.930888in}}{\pgfqpoint{3.230126in}{2.930888in}}%
\pgfpathcurveto{\pgfqpoint{3.219076in}{2.930888in}}{\pgfqpoint{3.208477in}{2.926497in}}{\pgfqpoint{3.200663in}{2.918684in}}%
\pgfpathcurveto{\pgfqpoint{3.192849in}{2.910870in}}{\pgfqpoint{3.188459in}{2.900271in}}{\pgfqpoint{3.188459in}{2.889221in}}%
\pgfpathcurveto{\pgfqpoint{3.188459in}{2.878171in}}{\pgfqpoint{3.192849in}{2.867572in}}{\pgfqpoint{3.200663in}{2.859758in}}%
\pgfpathcurveto{\pgfqpoint{3.208477in}{2.851944in}}{\pgfqpoint{3.219076in}{2.847554in}}{\pgfqpoint{3.230126in}{2.847554in}}%
\pgfpathclose%
\pgfusepath{stroke,fill}%
\end{pgfscope}%
\begin{pgfscope}%
\pgfpathrectangle{\pgfqpoint{0.600000in}{0.600000in}}{\pgfqpoint{3.900000in}{3.900000in}}%
\pgfusepath{clip}%
\pgfsetbuttcap%
\pgfsetroundjoin%
\definecolor{currentfill}{rgb}{0.121569,0.466667,0.705882}%
\pgfsetfillcolor{currentfill}%
\pgfsetlinewidth{1.003750pt}%
\definecolor{currentstroke}{rgb}{0.121569,0.466667,0.705882}%
\pgfsetstrokecolor{currentstroke}%
\pgfsetdash{}{0pt}%
\pgfpathmoveto{\pgfqpoint{2.934807in}{2.551134in}}%
\pgfpathcurveto{\pgfqpoint{2.945857in}{2.551134in}}{\pgfqpoint{2.956456in}{2.555525in}}{\pgfqpoint{2.964270in}{2.563338in}}%
\pgfpathcurveto{\pgfqpoint{2.972083in}{2.571152in}}{\pgfqpoint{2.976473in}{2.581751in}}{\pgfqpoint{2.976473in}{2.592801in}}%
\pgfpathcurveto{\pgfqpoint{2.976473in}{2.603851in}}{\pgfqpoint{2.972083in}{2.614450in}}{\pgfqpoint{2.964270in}{2.622264in}}%
\pgfpathcurveto{\pgfqpoint{2.956456in}{2.630078in}}{\pgfqpoint{2.945857in}{2.634468in}}{\pgfqpoint{2.934807in}{2.634468in}}%
\pgfpathcurveto{\pgfqpoint{2.923757in}{2.634468in}}{\pgfqpoint{2.913158in}{2.630078in}}{\pgfqpoint{2.905344in}{2.622264in}}%
\pgfpathcurveto{\pgfqpoint{2.897530in}{2.614450in}}{\pgfqpoint{2.893140in}{2.603851in}}{\pgfqpoint{2.893140in}{2.592801in}}%
\pgfpathcurveto{\pgfqpoint{2.893140in}{2.581751in}}{\pgfqpoint{2.897530in}{2.571152in}}{\pgfqpoint{2.905344in}{2.563338in}}%
\pgfpathcurveto{\pgfqpoint{2.913158in}{2.555525in}}{\pgfqpoint{2.923757in}{2.551134in}}{\pgfqpoint{2.934807in}{2.551134in}}%
\pgfpathclose%
\pgfusepath{stroke,fill}%
\end{pgfscope}%
\begin{pgfscope}%
\pgfpathrectangle{\pgfqpoint{0.600000in}{0.600000in}}{\pgfqpoint{3.900000in}{3.900000in}}%
\pgfusepath{clip}%
\pgfsetbuttcap%
\pgfsetroundjoin%
\definecolor{currentfill}{rgb}{0.121569,0.466667,0.705882}%
\pgfsetfillcolor{currentfill}%
\pgfsetlinewidth{1.003750pt}%
\definecolor{currentstroke}{rgb}{0.121569,0.466667,0.705882}%
\pgfsetstrokecolor{currentstroke}%
\pgfsetdash{}{0pt}%
\pgfpathmoveto{\pgfqpoint{1.819360in}{1.598593in}}%
\pgfpathcurveto{\pgfqpoint{1.830410in}{1.598593in}}{\pgfqpoint{1.841009in}{1.602983in}}{\pgfqpoint{1.848822in}{1.610797in}}%
\pgfpathcurveto{\pgfqpoint{1.856636in}{1.618610in}}{\pgfqpoint{1.861026in}{1.629210in}}{\pgfqpoint{1.861026in}{1.640260in}}%
\pgfpathcurveto{\pgfqpoint{1.861026in}{1.651310in}}{\pgfqpoint{1.856636in}{1.661909in}}{\pgfqpoint{1.848822in}{1.669722in}}%
\pgfpathcurveto{\pgfqpoint{1.841009in}{1.677536in}}{\pgfqpoint{1.830410in}{1.681926in}}{\pgfqpoint{1.819360in}{1.681926in}}%
\pgfpathcurveto{\pgfqpoint{1.808309in}{1.681926in}}{\pgfqpoint{1.797710in}{1.677536in}}{\pgfqpoint{1.789897in}{1.669722in}}%
\pgfpathcurveto{\pgfqpoint{1.782083in}{1.661909in}}{\pgfqpoint{1.777693in}{1.651310in}}{\pgfqpoint{1.777693in}{1.640260in}}%
\pgfpathcurveto{\pgfqpoint{1.777693in}{1.629210in}}{\pgfqpoint{1.782083in}{1.618610in}}{\pgfqpoint{1.789897in}{1.610797in}}%
\pgfpathcurveto{\pgfqpoint{1.797710in}{1.602983in}}{\pgfqpoint{1.808309in}{1.598593in}}{\pgfqpoint{1.819360in}{1.598593in}}%
\pgfpathclose%
\pgfusepath{stroke,fill}%
\end{pgfscope}%
\begin{pgfscope}%
\pgfpathrectangle{\pgfqpoint{0.600000in}{0.600000in}}{\pgfqpoint{3.900000in}{3.900000in}}%
\pgfusepath{clip}%
\pgfsetbuttcap%
\pgfsetroundjoin%
\definecolor{currentfill}{rgb}{0.121569,0.466667,0.705882}%
\pgfsetfillcolor{currentfill}%
\pgfsetlinewidth{1.003750pt}%
\definecolor{currentstroke}{rgb}{0.121569,0.466667,0.705882}%
\pgfsetstrokecolor{currentstroke}%
\pgfsetdash{}{0pt}%
\pgfpathmoveto{\pgfqpoint{2.049506in}{2.190720in}}%
\pgfpathcurveto{\pgfqpoint{2.060556in}{2.190720in}}{\pgfqpoint{2.071155in}{2.195111in}}{\pgfqpoint{2.078969in}{2.202924in}}%
\pgfpathcurveto{\pgfqpoint{2.086782in}{2.210738in}}{\pgfqpoint{2.091173in}{2.221337in}}{\pgfqpoint{2.091173in}{2.232387in}}%
\pgfpathcurveto{\pgfqpoint{2.091173in}{2.243437in}}{\pgfqpoint{2.086782in}{2.254036in}}{\pgfqpoint{2.078969in}{2.261850in}}%
\pgfpathcurveto{\pgfqpoint{2.071155in}{2.269664in}}{\pgfqpoint{2.060556in}{2.274054in}}{\pgfqpoint{2.049506in}{2.274054in}}%
\pgfpathcurveto{\pgfqpoint{2.038456in}{2.274054in}}{\pgfqpoint{2.027857in}{2.269664in}}{\pgfqpoint{2.020043in}{2.261850in}}%
\pgfpathcurveto{\pgfqpoint{2.012229in}{2.254036in}}{\pgfqpoint{2.007839in}{2.243437in}}{\pgfqpoint{2.007839in}{2.232387in}}%
\pgfpathcurveto{\pgfqpoint{2.007839in}{2.221337in}}{\pgfqpoint{2.012229in}{2.210738in}}{\pgfqpoint{2.020043in}{2.202924in}}%
\pgfpathcurveto{\pgfqpoint{2.027857in}{2.195111in}}{\pgfqpoint{2.038456in}{2.190720in}}{\pgfqpoint{2.049506in}{2.190720in}}%
\pgfpathclose%
\pgfusepath{stroke,fill}%
\end{pgfscope}%
\begin{pgfscope}%
\pgfpathrectangle{\pgfqpoint{0.600000in}{0.600000in}}{\pgfqpoint{3.900000in}{3.900000in}}%
\pgfusepath{clip}%
\pgfsetbuttcap%
\pgfsetroundjoin%
\definecolor{currentfill}{rgb}{0.121569,0.466667,0.705882}%
\pgfsetfillcolor{currentfill}%
\pgfsetlinewidth{1.003750pt}%
\definecolor{currentstroke}{rgb}{0.121569,0.466667,0.705882}%
\pgfsetstrokecolor{currentstroke}%
\pgfsetdash{}{0pt}%
\pgfpathmoveto{\pgfqpoint{1.889938in}{2.760533in}}%
\pgfpathcurveto{\pgfqpoint{1.900988in}{2.760533in}}{\pgfqpoint{1.911587in}{2.764924in}}{\pgfqpoint{1.919401in}{2.772737in}}%
\pgfpathcurveto{\pgfqpoint{1.927215in}{2.780551in}}{\pgfqpoint{1.931605in}{2.791150in}}{\pgfqpoint{1.931605in}{2.802200in}}%
\pgfpathcurveto{\pgfqpoint{1.931605in}{2.813250in}}{\pgfqpoint{1.927215in}{2.823849in}}{\pgfqpoint{1.919401in}{2.831663in}}%
\pgfpathcurveto{\pgfqpoint{1.911587in}{2.839476in}}{\pgfqpoint{1.900988in}{2.843867in}}{\pgfqpoint{1.889938in}{2.843867in}}%
\pgfpathcurveto{\pgfqpoint{1.878888in}{2.843867in}}{\pgfqpoint{1.868289in}{2.839476in}}{\pgfqpoint{1.860475in}{2.831663in}}%
\pgfpathcurveto{\pgfqpoint{1.852662in}{2.823849in}}{\pgfqpoint{1.848272in}{2.813250in}}{\pgfqpoint{1.848272in}{2.802200in}}%
\pgfpathcurveto{\pgfqpoint{1.848272in}{2.791150in}}{\pgfqpoint{1.852662in}{2.780551in}}{\pgfqpoint{1.860475in}{2.772737in}}%
\pgfpathcurveto{\pgfqpoint{1.868289in}{2.764924in}}{\pgfqpoint{1.878888in}{2.760533in}}{\pgfqpoint{1.889938in}{2.760533in}}%
\pgfpathclose%
\pgfusepath{stroke,fill}%
\end{pgfscope}%
\begin{pgfscope}%
\pgfpathrectangle{\pgfqpoint{0.600000in}{0.600000in}}{\pgfqpoint{3.900000in}{3.900000in}}%
\pgfusepath{clip}%
\pgfsetbuttcap%
\pgfsetroundjoin%
\definecolor{currentfill}{rgb}{0.121569,0.466667,0.705882}%
\pgfsetfillcolor{currentfill}%
\pgfsetlinewidth{1.003750pt}%
\definecolor{currentstroke}{rgb}{0.121569,0.466667,0.705882}%
\pgfsetstrokecolor{currentstroke}%
\pgfsetdash{}{0pt}%
\pgfpathmoveto{\pgfqpoint{2.344697in}{1.964297in}}%
\pgfpathcurveto{\pgfqpoint{2.355747in}{1.964297in}}{\pgfqpoint{2.366346in}{1.968687in}}{\pgfqpoint{2.374160in}{1.976500in}}%
\pgfpathcurveto{\pgfqpoint{2.381974in}{1.984314in}}{\pgfqpoint{2.386364in}{1.994913in}}{\pgfqpoint{2.386364in}{2.005963in}}%
\pgfpathcurveto{\pgfqpoint{2.386364in}{2.017013in}}{\pgfqpoint{2.381974in}{2.027612in}}{\pgfqpoint{2.374160in}{2.035426in}}%
\pgfpathcurveto{\pgfqpoint{2.366346in}{2.043240in}}{\pgfqpoint{2.355747in}{2.047630in}}{\pgfqpoint{2.344697in}{2.047630in}}%
\pgfpathcurveto{\pgfqpoint{2.333647in}{2.047630in}}{\pgfqpoint{2.323048in}{2.043240in}}{\pgfqpoint{2.315234in}{2.035426in}}%
\pgfpathcurveto{\pgfqpoint{2.307421in}{2.027612in}}{\pgfqpoint{2.303031in}{2.017013in}}{\pgfqpoint{2.303031in}{2.005963in}}%
\pgfpathcurveto{\pgfqpoint{2.303031in}{1.994913in}}{\pgfqpoint{2.307421in}{1.984314in}}{\pgfqpoint{2.315234in}{1.976500in}}%
\pgfpathcurveto{\pgfqpoint{2.323048in}{1.968687in}}{\pgfqpoint{2.333647in}{1.964297in}}{\pgfqpoint{2.344697in}{1.964297in}}%
\pgfpathclose%
\pgfusepath{stroke,fill}%
\end{pgfscope}%
\begin{pgfscope}%
\pgfpathrectangle{\pgfqpoint{0.600000in}{0.600000in}}{\pgfqpoint{3.900000in}{3.900000in}}%
\pgfusepath{clip}%
\pgfsetbuttcap%
\pgfsetroundjoin%
\definecolor{currentfill}{rgb}{0.121569,0.466667,0.705882}%
\pgfsetfillcolor{currentfill}%
\pgfsetlinewidth{1.003750pt}%
\definecolor{currentstroke}{rgb}{0.121569,0.466667,0.705882}%
\pgfsetstrokecolor{currentstroke}%
\pgfsetdash{}{0pt}%
\pgfpathmoveto{\pgfqpoint{2.456257in}{3.222382in}}%
\pgfpathcurveto{\pgfqpoint{2.467307in}{3.222382in}}{\pgfqpoint{2.477906in}{3.226772in}}{\pgfqpoint{2.485719in}{3.234586in}}%
\pgfpathcurveto{\pgfqpoint{2.493533in}{3.242400in}}{\pgfqpoint{2.497923in}{3.252999in}}{\pgfqpoint{2.497923in}{3.264049in}}%
\pgfpathcurveto{\pgfqpoint{2.497923in}{3.275099in}}{\pgfqpoint{2.493533in}{3.285698in}}{\pgfqpoint{2.485719in}{3.293512in}}%
\pgfpathcurveto{\pgfqpoint{2.477906in}{3.301325in}}{\pgfqpoint{2.467307in}{3.305715in}}{\pgfqpoint{2.456257in}{3.305715in}}%
\pgfpathcurveto{\pgfqpoint{2.445206in}{3.305715in}}{\pgfqpoint{2.434607in}{3.301325in}}{\pgfqpoint{2.426794in}{3.293512in}}%
\pgfpathcurveto{\pgfqpoint{2.418980in}{3.285698in}}{\pgfqpoint{2.414590in}{3.275099in}}{\pgfqpoint{2.414590in}{3.264049in}}%
\pgfpathcurveto{\pgfqpoint{2.414590in}{3.252999in}}{\pgfqpoint{2.418980in}{3.242400in}}{\pgfqpoint{2.426794in}{3.234586in}}%
\pgfpathcurveto{\pgfqpoint{2.434607in}{3.226772in}}{\pgfqpoint{2.445206in}{3.222382in}}{\pgfqpoint{2.456257in}{3.222382in}}%
\pgfpathclose%
\pgfusepath{stroke,fill}%
\end{pgfscope}%
\begin{pgfscope}%
\pgfpathrectangle{\pgfqpoint{0.600000in}{0.600000in}}{\pgfqpoint{3.900000in}{3.900000in}}%
\pgfusepath{clip}%
\pgfsetbuttcap%
\pgfsetroundjoin%
\definecolor{currentfill}{rgb}{0.121569,0.466667,0.705882}%
\pgfsetfillcolor{currentfill}%
\pgfsetlinewidth{1.003750pt}%
\definecolor{currentstroke}{rgb}{0.121569,0.466667,0.705882}%
\pgfsetstrokecolor{currentstroke}%
\pgfsetdash{}{0pt}%
\pgfpathmoveto{\pgfqpoint{2.116398in}{2.806836in}}%
\pgfpathcurveto{\pgfqpoint{2.127448in}{2.806836in}}{\pgfqpoint{2.138047in}{2.811226in}}{\pgfqpoint{2.145861in}{2.819040in}}%
\pgfpathcurveto{\pgfqpoint{2.153675in}{2.826854in}}{\pgfqpoint{2.158065in}{2.837453in}}{\pgfqpoint{2.158065in}{2.848503in}}%
\pgfpathcurveto{\pgfqpoint{2.158065in}{2.859553in}}{\pgfqpoint{2.153675in}{2.870152in}}{\pgfqpoint{2.145861in}{2.877966in}}%
\pgfpathcurveto{\pgfqpoint{2.138047in}{2.885779in}}{\pgfqpoint{2.127448in}{2.890170in}}{\pgfqpoint{2.116398in}{2.890170in}}%
\pgfpathcurveto{\pgfqpoint{2.105348in}{2.890170in}}{\pgfqpoint{2.094749in}{2.885779in}}{\pgfqpoint{2.086935in}{2.877966in}}%
\pgfpathcurveto{\pgfqpoint{2.079122in}{2.870152in}}{\pgfqpoint{2.074731in}{2.859553in}}{\pgfqpoint{2.074731in}{2.848503in}}%
\pgfpathcurveto{\pgfqpoint{2.074731in}{2.837453in}}{\pgfqpoint{2.079122in}{2.826854in}}{\pgfqpoint{2.086935in}{2.819040in}}%
\pgfpathcurveto{\pgfqpoint{2.094749in}{2.811226in}}{\pgfqpoint{2.105348in}{2.806836in}}{\pgfqpoint{2.116398in}{2.806836in}}%
\pgfpathclose%
\pgfusepath{stroke,fill}%
\end{pgfscope}%
\begin{pgfscope}%
\pgfpathrectangle{\pgfqpoint{0.600000in}{0.600000in}}{\pgfqpoint{3.900000in}{3.900000in}}%
\pgfusepath{clip}%
\pgfsetbuttcap%
\pgfsetroundjoin%
\definecolor{currentfill}{rgb}{0.121569,0.466667,0.705882}%
\pgfsetfillcolor{currentfill}%
\pgfsetlinewidth{1.003750pt}%
\definecolor{currentstroke}{rgb}{0.121569,0.466667,0.705882}%
\pgfsetstrokecolor{currentstroke}%
\pgfsetdash{}{0pt}%
\pgfpathmoveto{\pgfqpoint{2.511639in}{1.918868in}}%
\pgfpathcurveto{\pgfqpoint{2.522690in}{1.918868in}}{\pgfqpoint{2.533289in}{1.923258in}}{\pgfqpoint{2.541102in}{1.931072in}}%
\pgfpathcurveto{\pgfqpoint{2.548916in}{1.938885in}}{\pgfqpoint{2.553306in}{1.949484in}}{\pgfqpoint{2.553306in}{1.960535in}}%
\pgfpathcurveto{\pgfqpoint{2.553306in}{1.971585in}}{\pgfqpoint{2.548916in}{1.982184in}}{\pgfqpoint{2.541102in}{1.989997in}}%
\pgfpathcurveto{\pgfqpoint{2.533289in}{1.997811in}}{\pgfqpoint{2.522690in}{2.002201in}}{\pgfqpoint{2.511639in}{2.002201in}}%
\pgfpathcurveto{\pgfqpoint{2.500589in}{2.002201in}}{\pgfqpoint{2.489990in}{1.997811in}}{\pgfqpoint{2.482177in}{1.989997in}}%
\pgfpathcurveto{\pgfqpoint{2.474363in}{1.982184in}}{\pgfqpoint{2.469973in}{1.971585in}}{\pgfqpoint{2.469973in}{1.960535in}}%
\pgfpathcurveto{\pgfqpoint{2.469973in}{1.949484in}}{\pgfqpoint{2.474363in}{1.938885in}}{\pgfqpoint{2.482177in}{1.931072in}}%
\pgfpathcurveto{\pgfqpoint{2.489990in}{1.923258in}}{\pgfqpoint{2.500589in}{1.918868in}}{\pgfqpoint{2.511639in}{1.918868in}}%
\pgfpathclose%
\pgfusepath{stroke,fill}%
\end{pgfscope}%
\begin{pgfscope}%
\pgfpathrectangle{\pgfqpoint{0.600000in}{0.600000in}}{\pgfqpoint{3.900000in}{3.900000in}}%
\pgfusepath{clip}%
\pgfsetbuttcap%
\pgfsetroundjoin%
\definecolor{currentfill}{rgb}{0.121569,0.466667,0.705882}%
\pgfsetfillcolor{currentfill}%
\pgfsetlinewidth{1.003750pt}%
\definecolor{currentstroke}{rgb}{0.121569,0.466667,0.705882}%
\pgfsetstrokecolor{currentstroke}%
\pgfsetdash{}{0pt}%
\pgfpathmoveto{\pgfqpoint{2.898963in}{1.836203in}}%
\pgfpathcurveto{\pgfqpoint{2.910014in}{1.836203in}}{\pgfqpoint{2.920613in}{1.840594in}}{\pgfqpoint{2.928426in}{1.848407in}}%
\pgfpathcurveto{\pgfqpoint{2.936240in}{1.856221in}}{\pgfqpoint{2.940630in}{1.866820in}}{\pgfqpoint{2.940630in}{1.877870in}}%
\pgfpathcurveto{\pgfqpoint{2.940630in}{1.888920in}}{\pgfqpoint{2.936240in}{1.899519in}}{\pgfqpoint{2.928426in}{1.907333in}}%
\pgfpathcurveto{\pgfqpoint{2.920613in}{1.915146in}}{\pgfqpoint{2.910014in}{1.919537in}}{\pgfqpoint{2.898963in}{1.919537in}}%
\pgfpathcurveto{\pgfqpoint{2.887913in}{1.919537in}}{\pgfqpoint{2.877314in}{1.915146in}}{\pgfqpoint{2.869501in}{1.907333in}}%
\pgfpathcurveto{\pgfqpoint{2.861687in}{1.899519in}}{\pgfqpoint{2.857297in}{1.888920in}}{\pgfqpoint{2.857297in}{1.877870in}}%
\pgfpathcurveto{\pgfqpoint{2.857297in}{1.866820in}}{\pgfqpoint{2.861687in}{1.856221in}}{\pgfqpoint{2.869501in}{1.848407in}}%
\pgfpathcurveto{\pgfqpoint{2.877314in}{1.840594in}}{\pgfqpoint{2.887913in}{1.836203in}}{\pgfqpoint{2.898963in}{1.836203in}}%
\pgfpathclose%
\pgfusepath{stroke,fill}%
\end{pgfscope}%
\begin{pgfscope}%
\pgfpathrectangle{\pgfqpoint{0.600000in}{0.600000in}}{\pgfqpoint{3.900000in}{3.900000in}}%
\pgfusepath{clip}%
\pgfsetbuttcap%
\pgfsetroundjoin%
\definecolor{currentfill}{rgb}{0.121569,0.466667,0.705882}%
\pgfsetfillcolor{currentfill}%
\pgfsetlinewidth{1.003750pt}%
\definecolor{currentstroke}{rgb}{0.121569,0.466667,0.705882}%
\pgfsetstrokecolor{currentstroke}%
\pgfsetdash{}{0pt}%
\pgfpathmoveto{\pgfqpoint{2.339501in}{2.643366in}}%
\pgfpathcurveto{\pgfqpoint{2.350551in}{2.643366in}}{\pgfqpoint{2.361150in}{2.647756in}}{\pgfqpoint{2.368964in}{2.655570in}}%
\pgfpathcurveto{\pgfqpoint{2.376777in}{2.663383in}}{\pgfqpoint{2.381167in}{2.673982in}}{\pgfqpoint{2.381167in}{2.685032in}}%
\pgfpathcurveto{\pgfqpoint{2.381167in}{2.696083in}}{\pgfqpoint{2.376777in}{2.706682in}}{\pgfqpoint{2.368964in}{2.714495in}}%
\pgfpathcurveto{\pgfqpoint{2.361150in}{2.722309in}}{\pgfqpoint{2.350551in}{2.726699in}}{\pgfqpoint{2.339501in}{2.726699in}}%
\pgfpathcurveto{\pgfqpoint{2.328451in}{2.726699in}}{\pgfqpoint{2.317852in}{2.722309in}}{\pgfqpoint{2.310038in}{2.714495in}}%
\pgfpathcurveto{\pgfqpoint{2.302224in}{2.706682in}}{\pgfqpoint{2.297834in}{2.696083in}}{\pgfqpoint{2.297834in}{2.685032in}}%
\pgfpathcurveto{\pgfqpoint{2.297834in}{2.673982in}}{\pgfqpoint{2.302224in}{2.663383in}}{\pgfqpoint{2.310038in}{2.655570in}}%
\pgfpathcurveto{\pgfqpoint{2.317852in}{2.647756in}}{\pgfqpoint{2.328451in}{2.643366in}}{\pgfqpoint{2.339501in}{2.643366in}}%
\pgfpathclose%
\pgfusepath{stroke,fill}%
\end{pgfscope}%
\begin{pgfscope}%
\pgfpathrectangle{\pgfqpoint{0.600000in}{0.600000in}}{\pgfqpoint{3.900000in}{3.900000in}}%
\pgfusepath{clip}%
\pgfsetbuttcap%
\pgfsetroundjoin%
\definecolor{currentfill}{rgb}{0.121569,0.466667,0.705882}%
\pgfsetfillcolor{currentfill}%
\pgfsetlinewidth{1.003750pt}%
\definecolor{currentstroke}{rgb}{0.121569,0.466667,0.705882}%
\pgfsetstrokecolor{currentstroke}%
\pgfsetdash{}{0pt}%
\pgfpathmoveto{\pgfqpoint{1.480833in}{3.016287in}}%
\pgfpathcurveto{\pgfqpoint{1.491884in}{3.016287in}}{\pgfqpoint{1.502483in}{3.020678in}}{\pgfqpoint{1.510296in}{3.028491in}}%
\pgfpathcurveto{\pgfqpoint{1.518110in}{3.036305in}}{\pgfqpoint{1.522500in}{3.046904in}}{\pgfqpoint{1.522500in}{3.057954in}}%
\pgfpathcurveto{\pgfqpoint{1.522500in}{3.069004in}}{\pgfqpoint{1.518110in}{3.079603in}}{\pgfqpoint{1.510296in}{3.087417in}}%
\pgfpathcurveto{\pgfqpoint{1.502483in}{3.095231in}}{\pgfqpoint{1.491884in}{3.099621in}}{\pgfqpoint{1.480833in}{3.099621in}}%
\pgfpathcurveto{\pgfqpoint{1.469783in}{3.099621in}}{\pgfqpoint{1.459184in}{3.095231in}}{\pgfqpoint{1.451371in}{3.087417in}}%
\pgfpathcurveto{\pgfqpoint{1.443557in}{3.079603in}}{\pgfqpoint{1.439167in}{3.069004in}}{\pgfqpoint{1.439167in}{3.057954in}}%
\pgfpathcurveto{\pgfqpoint{1.439167in}{3.046904in}}{\pgfqpoint{1.443557in}{3.036305in}}{\pgfqpoint{1.451371in}{3.028491in}}%
\pgfpathcurveto{\pgfqpoint{1.459184in}{3.020678in}}{\pgfqpoint{1.469783in}{3.016287in}}{\pgfqpoint{1.480833in}{3.016287in}}%
\pgfpathclose%
\pgfusepath{stroke,fill}%
\end{pgfscope}%
\begin{pgfscope}%
\pgfpathrectangle{\pgfqpoint{0.600000in}{0.600000in}}{\pgfqpoint{3.900000in}{3.900000in}}%
\pgfusepath{clip}%
\pgfsetbuttcap%
\pgfsetroundjoin%
\definecolor{currentfill}{rgb}{0.121569,0.466667,0.705882}%
\pgfsetfillcolor{currentfill}%
\pgfsetlinewidth{1.003750pt}%
\definecolor{currentstroke}{rgb}{0.121569,0.466667,0.705882}%
\pgfsetstrokecolor{currentstroke}%
\pgfsetdash{}{0pt}%
\pgfpathmoveto{\pgfqpoint{2.195227in}{2.405185in}}%
\pgfpathcurveto{\pgfqpoint{2.206277in}{2.405185in}}{\pgfqpoint{2.216876in}{2.409575in}}{\pgfqpoint{2.224689in}{2.417389in}}%
\pgfpathcurveto{\pgfqpoint{2.232503in}{2.425202in}}{\pgfqpoint{2.236893in}{2.435801in}}{\pgfqpoint{2.236893in}{2.446852in}}%
\pgfpathcurveto{\pgfqpoint{2.236893in}{2.457902in}}{\pgfqpoint{2.232503in}{2.468501in}}{\pgfqpoint{2.224689in}{2.476314in}}%
\pgfpathcurveto{\pgfqpoint{2.216876in}{2.484128in}}{\pgfqpoint{2.206277in}{2.488518in}}{\pgfqpoint{2.195227in}{2.488518in}}%
\pgfpathcurveto{\pgfqpoint{2.184177in}{2.488518in}}{\pgfqpoint{2.173578in}{2.484128in}}{\pgfqpoint{2.165764in}{2.476314in}}%
\pgfpathcurveto{\pgfqpoint{2.157950in}{2.468501in}}{\pgfqpoint{2.153560in}{2.457902in}}{\pgfqpoint{2.153560in}{2.446852in}}%
\pgfpathcurveto{\pgfqpoint{2.153560in}{2.435801in}}{\pgfqpoint{2.157950in}{2.425202in}}{\pgfqpoint{2.165764in}{2.417389in}}%
\pgfpathcurveto{\pgfqpoint{2.173578in}{2.409575in}}{\pgfqpoint{2.184177in}{2.405185in}}{\pgfqpoint{2.195227in}{2.405185in}}%
\pgfpathclose%
\pgfusepath{stroke,fill}%
\end{pgfscope}%
\begin{pgfscope}%
\pgfpathrectangle{\pgfqpoint{0.600000in}{0.600000in}}{\pgfqpoint{3.900000in}{3.900000in}}%
\pgfusepath{clip}%
\pgfsetbuttcap%
\pgfsetroundjoin%
\definecolor{currentfill}{rgb}{0.121569,0.466667,0.705882}%
\pgfsetfillcolor{currentfill}%
\pgfsetlinewidth{1.003750pt}%
\definecolor{currentstroke}{rgb}{0.121569,0.466667,0.705882}%
\pgfsetstrokecolor{currentstroke}%
\pgfsetdash{}{0pt}%
\pgfpathmoveto{\pgfqpoint{2.826293in}{3.147554in}}%
\pgfpathcurveto{\pgfqpoint{2.837343in}{3.147554in}}{\pgfqpoint{2.847942in}{3.151945in}}{\pgfqpoint{2.855756in}{3.159758in}}%
\pgfpathcurveto{\pgfqpoint{2.863569in}{3.167572in}}{\pgfqpoint{2.867959in}{3.178171in}}{\pgfqpoint{2.867959in}{3.189221in}}%
\pgfpathcurveto{\pgfqpoint{2.867959in}{3.200271in}}{\pgfqpoint{2.863569in}{3.210870in}}{\pgfqpoint{2.855756in}{3.218684in}}%
\pgfpathcurveto{\pgfqpoint{2.847942in}{3.226497in}}{\pgfqpoint{2.837343in}{3.230888in}}{\pgfqpoint{2.826293in}{3.230888in}}%
\pgfpathcurveto{\pgfqpoint{2.815243in}{3.230888in}}{\pgfqpoint{2.804644in}{3.226497in}}{\pgfqpoint{2.796830in}{3.218684in}}%
\pgfpathcurveto{\pgfqpoint{2.789016in}{3.210870in}}{\pgfqpoint{2.784626in}{3.200271in}}{\pgfqpoint{2.784626in}{3.189221in}}%
\pgfpathcurveto{\pgfqpoint{2.784626in}{3.178171in}}{\pgfqpoint{2.789016in}{3.167572in}}{\pgfqpoint{2.796830in}{3.159758in}}%
\pgfpathcurveto{\pgfqpoint{2.804644in}{3.151945in}}{\pgfqpoint{2.815243in}{3.147554in}}{\pgfqpoint{2.826293in}{3.147554in}}%
\pgfpathclose%
\pgfusepath{stroke,fill}%
\end{pgfscope}%
\begin{pgfscope}%
\pgfpathrectangle{\pgfqpoint{0.600000in}{0.600000in}}{\pgfqpoint{3.900000in}{3.900000in}}%
\pgfusepath{clip}%
\pgfsetbuttcap%
\pgfsetroundjoin%
\definecolor{currentfill}{rgb}{0.121569,0.466667,0.705882}%
\pgfsetfillcolor{currentfill}%
\pgfsetlinewidth{1.003750pt}%
\definecolor{currentstroke}{rgb}{0.121569,0.466667,0.705882}%
\pgfsetstrokecolor{currentstroke}%
\pgfsetdash{}{0pt}%
\pgfpathmoveto{\pgfqpoint{2.396142in}{3.142709in}}%
\pgfpathcurveto{\pgfqpoint{2.407193in}{3.142709in}}{\pgfqpoint{2.417792in}{3.147100in}}{\pgfqpoint{2.425605in}{3.154913in}}%
\pgfpathcurveto{\pgfqpoint{2.433419in}{3.162727in}}{\pgfqpoint{2.437809in}{3.173326in}}{\pgfqpoint{2.437809in}{3.184376in}}%
\pgfpathcurveto{\pgfqpoint{2.437809in}{3.195426in}}{\pgfqpoint{2.433419in}{3.206025in}}{\pgfqpoint{2.425605in}{3.213839in}}%
\pgfpathcurveto{\pgfqpoint{2.417792in}{3.221653in}}{\pgfqpoint{2.407193in}{3.226043in}}{\pgfqpoint{2.396142in}{3.226043in}}%
\pgfpathcurveto{\pgfqpoint{2.385092in}{3.226043in}}{\pgfqpoint{2.374493in}{3.221653in}}{\pgfqpoint{2.366680in}{3.213839in}}%
\pgfpathcurveto{\pgfqpoint{2.358866in}{3.206025in}}{\pgfqpoint{2.354476in}{3.195426in}}{\pgfqpoint{2.354476in}{3.184376in}}%
\pgfpathcurveto{\pgfqpoint{2.354476in}{3.173326in}}{\pgfqpoint{2.358866in}{3.162727in}}{\pgfqpoint{2.366680in}{3.154913in}}%
\pgfpathcurveto{\pgfqpoint{2.374493in}{3.147100in}}{\pgfqpoint{2.385092in}{3.142709in}}{\pgfqpoint{2.396142in}{3.142709in}}%
\pgfpathclose%
\pgfusepath{stroke,fill}%
\end{pgfscope}%
\begin{pgfscope}%
\pgfpathrectangle{\pgfqpoint{0.600000in}{0.600000in}}{\pgfqpoint{3.900000in}{3.900000in}}%
\pgfusepath{clip}%
\pgfsetbuttcap%
\pgfsetroundjoin%
\definecolor{currentfill}{rgb}{0.121569,0.466667,0.705882}%
\pgfsetfillcolor{currentfill}%
\pgfsetlinewidth{1.003750pt}%
\definecolor{currentstroke}{rgb}{0.121569,0.466667,0.705882}%
\pgfsetstrokecolor{currentstroke}%
\pgfsetdash{}{0pt}%
\pgfpathmoveto{\pgfqpoint{2.748370in}{3.380275in}}%
\pgfpathcurveto{\pgfqpoint{2.759420in}{3.380275in}}{\pgfqpoint{2.770019in}{3.384665in}}{\pgfqpoint{2.777833in}{3.392479in}}%
\pgfpathcurveto{\pgfqpoint{2.785647in}{3.400292in}}{\pgfqpoint{2.790037in}{3.410891in}}{\pgfqpoint{2.790037in}{3.421942in}}%
\pgfpathcurveto{\pgfqpoint{2.790037in}{3.432992in}}{\pgfqpoint{2.785647in}{3.443591in}}{\pgfqpoint{2.777833in}{3.451404in}}%
\pgfpathcurveto{\pgfqpoint{2.770019in}{3.459218in}}{\pgfqpoint{2.759420in}{3.463608in}}{\pgfqpoint{2.748370in}{3.463608in}}%
\pgfpathcurveto{\pgfqpoint{2.737320in}{3.463608in}}{\pgfqpoint{2.726721in}{3.459218in}}{\pgfqpoint{2.718907in}{3.451404in}}%
\pgfpathcurveto{\pgfqpoint{2.711094in}{3.443591in}}{\pgfqpoint{2.706704in}{3.432992in}}{\pgfqpoint{2.706704in}{3.421942in}}%
\pgfpathcurveto{\pgfqpoint{2.706704in}{3.410891in}}{\pgfqpoint{2.711094in}{3.400292in}}{\pgfqpoint{2.718907in}{3.392479in}}%
\pgfpathcurveto{\pgfqpoint{2.726721in}{3.384665in}}{\pgfqpoint{2.737320in}{3.380275in}}{\pgfqpoint{2.748370in}{3.380275in}}%
\pgfpathclose%
\pgfusepath{stroke,fill}%
\end{pgfscope}%
\begin{pgfscope}%
\pgfpathrectangle{\pgfqpoint{0.600000in}{0.600000in}}{\pgfqpoint{3.900000in}{3.900000in}}%
\pgfusepath{clip}%
\pgfsetbuttcap%
\pgfsetroundjoin%
\definecolor{currentfill}{rgb}{0.121569,0.466667,0.705882}%
\pgfsetfillcolor{currentfill}%
\pgfsetlinewidth{1.003750pt}%
\definecolor{currentstroke}{rgb}{0.121569,0.466667,0.705882}%
\pgfsetstrokecolor{currentstroke}%
\pgfsetdash{}{0pt}%
\pgfpathmoveto{\pgfqpoint{2.508462in}{1.625243in}}%
\pgfpathcurveto{\pgfqpoint{2.519513in}{1.625243in}}{\pgfqpoint{2.530112in}{1.629633in}}{\pgfqpoint{2.537925in}{1.637447in}}%
\pgfpathcurveto{\pgfqpoint{2.545739in}{1.645260in}}{\pgfqpoint{2.550129in}{1.655859in}}{\pgfqpoint{2.550129in}{1.666910in}}%
\pgfpathcurveto{\pgfqpoint{2.550129in}{1.677960in}}{\pgfqpoint{2.545739in}{1.688559in}}{\pgfqpoint{2.537925in}{1.696372in}}%
\pgfpathcurveto{\pgfqpoint{2.530112in}{1.704186in}}{\pgfqpoint{2.519513in}{1.708576in}}{\pgfqpoint{2.508462in}{1.708576in}}%
\pgfpathcurveto{\pgfqpoint{2.497412in}{1.708576in}}{\pgfqpoint{2.486813in}{1.704186in}}{\pgfqpoint{2.479000in}{1.696372in}}%
\pgfpathcurveto{\pgfqpoint{2.471186in}{1.688559in}}{\pgfqpoint{2.466796in}{1.677960in}}{\pgfqpoint{2.466796in}{1.666910in}}%
\pgfpathcurveto{\pgfqpoint{2.466796in}{1.655859in}}{\pgfqpoint{2.471186in}{1.645260in}}{\pgfqpoint{2.479000in}{1.637447in}}%
\pgfpathcurveto{\pgfqpoint{2.486813in}{1.629633in}}{\pgfqpoint{2.497412in}{1.625243in}}{\pgfqpoint{2.508462in}{1.625243in}}%
\pgfpathclose%
\pgfusepath{stroke,fill}%
\end{pgfscope}%
\begin{pgfscope}%
\pgfpathrectangle{\pgfqpoint{0.600000in}{0.600000in}}{\pgfqpoint{3.900000in}{3.900000in}}%
\pgfusepath{clip}%
\pgfsetbuttcap%
\pgfsetroundjoin%
\definecolor{currentfill}{rgb}{0.121569,0.466667,0.705882}%
\pgfsetfillcolor{currentfill}%
\pgfsetlinewidth{1.003750pt}%
\definecolor{currentstroke}{rgb}{0.121569,0.466667,0.705882}%
\pgfsetstrokecolor{currentstroke}%
\pgfsetdash{}{0pt}%
\pgfpathmoveto{\pgfqpoint{2.154594in}{0.533753in}}%
\pgfpathcurveto{\pgfqpoint{2.165644in}{0.533753in}}{\pgfqpoint{2.176243in}{0.538144in}}{\pgfqpoint{2.184057in}{0.545957in}}%
\pgfpathcurveto{\pgfqpoint{2.191871in}{0.553771in}}{\pgfqpoint{2.196261in}{0.564370in}}{\pgfqpoint{2.196261in}{0.575420in}}%
\pgfpathcurveto{\pgfqpoint{2.196261in}{0.586470in}}{\pgfqpoint{2.191871in}{0.597069in}}{\pgfqpoint{2.184057in}{0.604883in}}%
\pgfpathcurveto{\pgfqpoint{2.176243in}{0.612697in}}{\pgfqpoint{2.165644in}{0.617087in}}{\pgfqpoint{2.154594in}{0.617087in}}%
\pgfpathcurveto{\pgfqpoint{2.143544in}{0.617087in}}{\pgfqpoint{2.132945in}{0.612697in}}{\pgfqpoint{2.125131in}{0.604883in}}%
\pgfpathcurveto{\pgfqpoint{2.117318in}{0.597069in}}{\pgfqpoint{2.112927in}{0.586470in}}{\pgfqpoint{2.112927in}{0.575420in}}%
\pgfpathcurveto{\pgfqpoint{2.112927in}{0.564370in}}{\pgfqpoint{2.117318in}{0.553771in}}{\pgfqpoint{2.125131in}{0.545957in}}%
\pgfpathcurveto{\pgfqpoint{2.132945in}{0.538144in}}{\pgfqpoint{2.143544in}{0.533753in}}{\pgfqpoint{2.154594in}{0.533753in}}%
\pgfpathclose%
\pgfusepath{stroke,fill}%
\end{pgfscope}%
\begin{pgfscope}%
\pgfpathrectangle{\pgfqpoint{0.600000in}{0.600000in}}{\pgfqpoint{3.900000in}{3.900000in}}%
\pgfusepath{clip}%
\pgfsetbuttcap%
\pgfsetroundjoin%
\definecolor{currentfill}{rgb}{0.121569,0.466667,0.705882}%
\pgfsetfillcolor{currentfill}%
\pgfsetlinewidth{1.003750pt}%
\definecolor{currentstroke}{rgb}{0.121569,0.466667,0.705882}%
\pgfsetstrokecolor{currentstroke}%
\pgfsetdash{}{0pt}%
\pgfpathmoveto{\pgfqpoint{2.535084in}{3.121889in}}%
\pgfpathcurveto{\pgfqpoint{2.546134in}{3.121889in}}{\pgfqpoint{2.556733in}{3.126279in}}{\pgfqpoint{2.564546in}{3.134092in}}%
\pgfpathcurveto{\pgfqpoint{2.572360in}{3.141906in}}{\pgfqpoint{2.576750in}{3.152505in}}{\pgfqpoint{2.576750in}{3.163555in}}%
\pgfpathcurveto{\pgfqpoint{2.576750in}{3.174605in}}{\pgfqpoint{2.572360in}{3.185204in}}{\pgfqpoint{2.564546in}{3.193018in}}%
\pgfpathcurveto{\pgfqpoint{2.556733in}{3.200832in}}{\pgfqpoint{2.546134in}{3.205222in}}{\pgfqpoint{2.535084in}{3.205222in}}%
\pgfpathcurveto{\pgfqpoint{2.524034in}{3.205222in}}{\pgfqpoint{2.513434in}{3.200832in}}{\pgfqpoint{2.505621in}{3.193018in}}%
\pgfpathcurveto{\pgfqpoint{2.497807in}{3.185204in}}{\pgfqpoint{2.493417in}{3.174605in}}{\pgfqpoint{2.493417in}{3.163555in}}%
\pgfpathcurveto{\pgfqpoint{2.493417in}{3.152505in}}{\pgfqpoint{2.497807in}{3.141906in}}{\pgfqpoint{2.505621in}{3.134092in}}%
\pgfpathcurveto{\pgfqpoint{2.513434in}{3.126279in}}{\pgfqpoint{2.524034in}{3.121889in}}{\pgfqpoint{2.535084in}{3.121889in}}%
\pgfpathclose%
\pgfusepath{stroke,fill}%
\end{pgfscope}%
\begin{pgfscope}%
\pgfpathrectangle{\pgfqpoint{0.600000in}{0.600000in}}{\pgfqpoint{3.900000in}{3.900000in}}%
\pgfusepath{clip}%
\pgfsetbuttcap%
\pgfsetroundjoin%
\definecolor{currentfill}{rgb}{0.121569,0.466667,0.705882}%
\pgfsetfillcolor{currentfill}%
\pgfsetlinewidth{1.003750pt}%
\definecolor{currentstroke}{rgb}{0.121569,0.466667,0.705882}%
\pgfsetstrokecolor{currentstroke}%
\pgfsetdash{}{0pt}%
\pgfpathmoveto{\pgfqpoint{2.895226in}{2.394570in}}%
\pgfpathcurveto{\pgfqpoint{2.906276in}{2.394570in}}{\pgfqpoint{2.916875in}{2.398960in}}{\pgfqpoint{2.924689in}{2.406774in}}%
\pgfpathcurveto{\pgfqpoint{2.932503in}{2.414587in}}{\pgfqpoint{2.936893in}{2.425186in}}{\pgfqpoint{2.936893in}{2.436236in}}%
\pgfpathcurveto{\pgfqpoint{2.936893in}{2.447286in}}{\pgfqpoint{2.932503in}{2.457886in}}{\pgfqpoint{2.924689in}{2.465699in}}%
\pgfpathcurveto{\pgfqpoint{2.916875in}{2.473513in}}{\pgfqpoint{2.906276in}{2.477903in}}{\pgfqpoint{2.895226in}{2.477903in}}%
\pgfpathcurveto{\pgfqpoint{2.884176in}{2.477903in}}{\pgfqpoint{2.873577in}{2.473513in}}{\pgfqpoint{2.865764in}{2.465699in}}%
\pgfpathcurveto{\pgfqpoint{2.857950in}{2.457886in}}{\pgfqpoint{2.853560in}{2.447286in}}{\pgfqpoint{2.853560in}{2.436236in}}%
\pgfpathcurveto{\pgfqpoint{2.853560in}{2.425186in}}{\pgfqpoint{2.857950in}{2.414587in}}{\pgfqpoint{2.865764in}{2.406774in}}%
\pgfpathcurveto{\pgfqpoint{2.873577in}{2.398960in}}{\pgfqpoint{2.884176in}{2.394570in}}{\pgfqpoint{2.895226in}{2.394570in}}%
\pgfpathclose%
\pgfusepath{stroke,fill}%
\end{pgfscope}%
\begin{pgfscope}%
\pgfpathrectangle{\pgfqpoint{0.600000in}{0.600000in}}{\pgfqpoint{3.900000in}{3.900000in}}%
\pgfusepath{clip}%
\pgfsetbuttcap%
\pgfsetroundjoin%
\definecolor{currentfill}{rgb}{0.121569,0.466667,0.705882}%
\pgfsetfillcolor{currentfill}%
\pgfsetlinewidth{1.003750pt}%
\definecolor{currentstroke}{rgb}{0.121569,0.466667,0.705882}%
\pgfsetstrokecolor{currentstroke}%
\pgfsetdash{}{0pt}%
\pgfpathmoveto{\pgfqpoint{3.050534in}{3.650957in}}%
\pgfpathcurveto{\pgfqpoint{3.061584in}{3.650957in}}{\pgfqpoint{3.072183in}{3.655347in}}{\pgfqpoint{3.079996in}{3.663161in}}%
\pgfpathcurveto{\pgfqpoint{3.087810in}{3.670974in}}{\pgfqpoint{3.092200in}{3.681573in}}{\pgfqpoint{3.092200in}{3.692623in}}%
\pgfpathcurveto{\pgfqpoint{3.092200in}{3.703673in}}{\pgfqpoint{3.087810in}{3.714272in}}{\pgfqpoint{3.079996in}{3.722086in}}%
\pgfpathcurveto{\pgfqpoint{3.072183in}{3.729900in}}{\pgfqpoint{3.061584in}{3.734290in}}{\pgfqpoint{3.050534in}{3.734290in}}%
\pgfpathcurveto{\pgfqpoint{3.039483in}{3.734290in}}{\pgfqpoint{3.028884in}{3.729900in}}{\pgfqpoint{3.021071in}{3.722086in}}%
\pgfpathcurveto{\pgfqpoint{3.013257in}{3.714272in}}{\pgfqpoint{3.008867in}{3.703673in}}{\pgfqpoint{3.008867in}{3.692623in}}%
\pgfpathcurveto{\pgfqpoint{3.008867in}{3.681573in}}{\pgfqpoint{3.013257in}{3.670974in}}{\pgfqpoint{3.021071in}{3.663161in}}%
\pgfpathcurveto{\pgfqpoint{3.028884in}{3.655347in}}{\pgfqpoint{3.039483in}{3.650957in}}{\pgfqpoint{3.050534in}{3.650957in}}%
\pgfpathclose%
\pgfusepath{stroke,fill}%
\end{pgfscope}%
\begin{pgfscope}%
\pgfpathrectangle{\pgfqpoint{0.600000in}{0.600000in}}{\pgfqpoint{3.900000in}{3.900000in}}%
\pgfusepath{clip}%
\pgfsetbuttcap%
\pgfsetroundjoin%
\definecolor{currentfill}{rgb}{0.121569,0.466667,0.705882}%
\pgfsetfillcolor{currentfill}%
\pgfsetlinewidth{1.003750pt}%
\definecolor{currentstroke}{rgb}{0.121569,0.466667,0.705882}%
\pgfsetstrokecolor{currentstroke}%
\pgfsetdash{}{0pt}%
\pgfpathmoveto{\pgfqpoint{3.614621in}{1.933785in}}%
\pgfpathcurveto{\pgfqpoint{3.625672in}{1.933785in}}{\pgfqpoint{3.636271in}{1.938176in}}{\pgfqpoint{3.644084in}{1.945989in}}%
\pgfpathcurveto{\pgfqpoint{3.651898in}{1.953803in}}{\pgfqpoint{3.656288in}{1.964402in}}{\pgfqpoint{3.656288in}{1.975452in}}%
\pgfpathcurveto{\pgfqpoint{3.656288in}{1.986502in}}{\pgfqpoint{3.651898in}{1.997101in}}{\pgfqpoint{3.644084in}{2.004915in}}%
\pgfpathcurveto{\pgfqpoint{3.636271in}{2.012728in}}{\pgfqpoint{3.625672in}{2.017119in}}{\pgfqpoint{3.614621in}{2.017119in}}%
\pgfpathcurveto{\pgfqpoint{3.603571in}{2.017119in}}{\pgfqpoint{3.592972in}{2.012728in}}{\pgfqpoint{3.585159in}{2.004915in}}%
\pgfpathcurveto{\pgfqpoint{3.577345in}{1.997101in}}{\pgfqpoint{3.572955in}{1.986502in}}{\pgfqpoint{3.572955in}{1.975452in}}%
\pgfpathcurveto{\pgfqpoint{3.572955in}{1.964402in}}{\pgfqpoint{3.577345in}{1.953803in}}{\pgfqpoint{3.585159in}{1.945989in}}%
\pgfpathcurveto{\pgfqpoint{3.592972in}{1.938176in}}{\pgfqpoint{3.603571in}{1.933785in}}{\pgfqpoint{3.614621in}{1.933785in}}%
\pgfpathclose%
\pgfusepath{stroke,fill}%
\end{pgfscope}%
\begin{pgfscope}%
\pgfpathrectangle{\pgfqpoint{0.600000in}{0.600000in}}{\pgfqpoint{3.900000in}{3.900000in}}%
\pgfusepath{clip}%
\pgfsetbuttcap%
\pgfsetroundjoin%
\definecolor{currentfill}{rgb}{0.121569,0.466667,0.705882}%
\pgfsetfillcolor{currentfill}%
\pgfsetlinewidth{1.003750pt}%
\definecolor{currentstroke}{rgb}{0.121569,0.466667,0.705882}%
\pgfsetstrokecolor{currentstroke}%
\pgfsetdash{}{0pt}%
\pgfpathmoveto{\pgfqpoint{2.909182in}{2.015141in}}%
\pgfpathcurveto{\pgfqpoint{2.920232in}{2.015141in}}{\pgfqpoint{2.930831in}{2.019531in}}{\pgfqpoint{2.938644in}{2.027345in}}%
\pgfpathcurveto{\pgfqpoint{2.946458in}{2.035158in}}{\pgfqpoint{2.950848in}{2.045757in}}{\pgfqpoint{2.950848in}{2.056807in}}%
\pgfpathcurveto{\pgfqpoint{2.950848in}{2.067858in}}{\pgfqpoint{2.946458in}{2.078457in}}{\pgfqpoint{2.938644in}{2.086270in}}%
\pgfpathcurveto{\pgfqpoint{2.930831in}{2.094084in}}{\pgfqpoint{2.920232in}{2.098474in}}{\pgfqpoint{2.909182in}{2.098474in}}%
\pgfpathcurveto{\pgfqpoint{2.898131in}{2.098474in}}{\pgfqpoint{2.887532in}{2.094084in}}{\pgfqpoint{2.879719in}{2.086270in}}%
\pgfpathcurveto{\pgfqpoint{2.871905in}{2.078457in}}{\pgfqpoint{2.867515in}{2.067858in}}{\pgfqpoint{2.867515in}{2.056807in}}%
\pgfpathcurveto{\pgfqpoint{2.867515in}{2.045757in}}{\pgfqpoint{2.871905in}{2.035158in}}{\pgfqpoint{2.879719in}{2.027345in}}%
\pgfpathcurveto{\pgfqpoint{2.887532in}{2.019531in}}{\pgfqpoint{2.898131in}{2.015141in}}{\pgfqpoint{2.909182in}{2.015141in}}%
\pgfpathclose%
\pgfusepath{stroke,fill}%
\end{pgfscope}%
\begin{pgfscope}%
\pgfpathrectangle{\pgfqpoint{0.600000in}{0.600000in}}{\pgfqpoint{3.900000in}{3.900000in}}%
\pgfusepath{clip}%
\pgfsetbuttcap%
\pgfsetroundjoin%
\definecolor{currentfill}{rgb}{0.121569,0.466667,0.705882}%
\pgfsetfillcolor{currentfill}%
\pgfsetlinewidth{1.003750pt}%
\definecolor{currentstroke}{rgb}{0.121569,0.466667,0.705882}%
\pgfsetstrokecolor{currentstroke}%
\pgfsetdash{}{0pt}%
\pgfpathmoveto{\pgfqpoint{3.087110in}{2.116916in}}%
\pgfpathcurveto{\pgfqpoint{3.098160in}{2.116916in}}{\pgfqpoint{3.108759in}{2.121306in}}{\pgfqpoint{3.116573in}{2.129120in}}%
\pgfpathcurveto{\pgfqpoint{3.124387in}{2.136933in}}{\pgfqpoint{3.128777in}{2.147532in}}{\pgfqpoint{3.128777in}{2.158582in}}%
\pgfpathcurveto{\pgfqpoint{3.128777in}{2.169633in}}{\pgfqpoint{3.124387in}{2.180232in}}{\pgfqpoint{3.116573in}{2.188045in}}%
\pgfpathcurveto{\pgfqpoint{3.108759in}{2.195859in}}{\pgfqpoint{3.098160in}{2.200249in}}{\pgfqpoint{3.087110in}{2.200249in}}%
\pgfpathcurveto{\pgfqpoint{3.076060in}{2.200249in}}{\pgfqpoint{3.065461in}{2.195859in}}{\pgfqpoint{3.057648in}{2.188045in}}%
\pgfpathcurveto{\pgfqpoint{3.049834in}{2.180232in}}{\pgfqpoint{3.045444in}{2.169633in}}{\pgfqpoint{3.045444in}{2.158582in}}%
\pgfpathcurveto{\pgfqpoint{3.045444in}{2.147532in}}{\pgfqpoint{3.049834in}{2.136933in}}{\pgfqpoint{3.057648in}{2.129120in}}%
\pgfpathcurveto{\pgfqpoint{3.065461in}{2.121306in}}{\pgfqpoint{3.076060in}{2.116916in}}{\pgfqpoint{3.087110in}{2.116916in}}%
\pgfpathclose%
\pgfusepath{stroke,fill}%
\end{pgfscope}%
\begin{pgfscope}%
\pgfpathrectangle{\pgfqpoint{0.600000in}{0.600000in}}{\pgfqpoint{3.900000in}{3.900000in}}%
\pgfusepath{clip}%
\pgfsetbuttcap%
\pgfsetroundjoin%
\definecolor{currentfill}{rgb}{0.121569,0.466667,0.705882}%
\pgfsetfillcolor{currentfill}%
\pgfsetlinewidth{1.003750pt}%
\definecolor{currentstroke}{rgb}{0.121569,0.466667,0.705882}%
\pgfsetstrokecolor{currentstroke}%
\pgfsetdash{}{0pt}%
\pgfpathmoveto{\pgfqpoint{2.739504in}{2.556399in}}%
\pgfpathcurveto{\pgfqpoint{2.750554in}{2.556399in}}{\pgfqpoint{2.761153in}{2.560789in}}{\pgfqpoint{2.768966in}{2.568603in}}%
\pgfpathcurveto{\pgfqpoint{2.776780in}{2.576416in}}{\pgfqpoint{2.781170in}{2.587015in}}{\pgfqpoint{2.781170in}{2.598065in}}%
\pgfpathcurveto{\pgfqpoint{2.781170in}{2.609115in}}{\pgfqpoint{2.776780in}{2.619715in}}{\pgfqpoint{2.768966in}{2.627528in}}%
\pgfpathcurveto{\pgfqpoint{2.761153in}{2.635342in}}{\pgfqpoint{2.750554in}{2.639732in}}{\pgfqpoint{2.739504in}{2.639732in}}%
\pgfpathcurveto{\pgfqpoint{2.728453in}{2.639732in}}{\pgfqpoint{2.717854in}{2.635342in}}{\pgfqpoint{2.710041in}{2.627528in}}%
\pgfpathcurveto{\pgfqpoint{2.702227in}{2.619715in}}{\pgfqpoint{2.697837in}{2.609115in}}{\pgfqpoint{2.697837in}{2.598065in}}%
\pgfpathcurveto{\pgfqpoint{2.697837in}{2.587015in}}{\pgfqpoint{2.702227in}{2.576416in}}{\pgfqpoint{2.710041in}{2.568603in}}%
\pgfpathcurveto{\pgfqpoint{2.717854in}{2.560789in}}{\pgfqpoint{2.728453in}{2.556399in}}{\pgfqpoint{2.739504in}{2.556399in}}%
\pgfpathclose%
\pgfusepath{stroke,fill}%
\end{pgfscope}%
\begin{pgfscope}%
\pgfpathrectangle{\pgfqpoint{0.600000in}{0.600000in}}{\pgfqpoint{3.900000in}{3.900000in}}%
\pgfusepath{clip}%
\pgfsetbuttcap%
\pgfsetroundjoin%
\definecolor{currentfill}{rgb}{0.121569,0.466667,0.705882}%
\pgfsetfillcolor{currentfill}%
\pgfsetlinewidth{1.003750pt}%
\definecolor{currentstroke}{rgb}{0.121569,0.466667,0.705882}%
\pgfsetstrokecolor{currentstroke}%
\pgfsetdash{}{0pt}%
\pgfpathmoveto{\pgfqpoint{2.652401in}{2.174623in}}%
\pgfpathcurveto{\pgfqpoint{2.663452in}{2.174623in}}{\pgfqpoint{2.674051in}{2.179014in}}{\pgfqpoint{2.681864in}{2.186827in}}%
\pgfpathcurveto{\pgfqpoint{2.689678in}{2.194641in}}{\pgfqpoint{2.694068in}{2.205240in}}{\pgfqpoint{2.694068in}{2.216290in}}%
\pgfpathcurveto{\pgfqpoint{2.694068in}{2.227340in}}{\pgfqpoint{2.689678in}{2.237939in}}{\pgfqpoint{2.681864in}{2.245753in}}%
\pgfpathcurveto{\pgfqpoint{2.674051in}{2.253566in}}{\pgfqpoint{2.663452in}{2.257957in}}{\pgfqpoint{2.652401in}{2.257957in}}%
\pgfpathcurveto{\pgfqpoint{2.641351in}{2.257957in}}{\pgfqpoint{2.630752in}{2.253566in}}{\pgfqpoint{2.622939in}{2.245753in}}%
\pgfpathcurveto{\pgfqpoint{2.615125in}{2.237939in}}{\pgfqpoint{2.610735in}{2.227340in}}{\pgfqpoint{2.610735in}{2.216290in}}%
\pgfpathcurveto{\pgfqpoint{2.610735in}{2.205240in}}{\pgfqpoint{2.615125in}{2.194641in}}{\pgfqpoint{2.622939in}{2.186827in}}%
\pgfpathcurveto{\pgfqpoint{2.630752in}{2.179014in}}{\pgfqpoint{2.641351in}{2.174623in}}{\pgfqpoint{2.652401in}{2.174623in}}%
\pgfpathclose%
\pgfusepath{stroke,fill}%
\end{pgfscope}%
\begin{pgfscope}%
\pgfpathrectangle{\pgfqpoint{0.600000in}{0.600000in}}{\pgfqpoint{3.900000in}{3.900000in}}%
\pgfusepath{clip}%
\pgfsetbuttcap%
\pgfsetroundjoin%
\definecolor{currentfill}{rgb}{0.121569,0.466667,0.705882}%
\pgfsetfillcolor{currentfill}%
\pgfsetlinewidth{1.003750pt}%
\definecolor{currentstroke}{rgb}{0.121569,0.466667,0.705882}%
\pgfsetstrokecolor{currentstroke}%
\pgfsetdash{}{0pt}%
\pgfpathmoveto{\pgfqpoint{2.719097in}{1.725411in}}%
\pgfpathcurveto{\pgfqpoint{2.730147in}{1.725411in}}{\pgfqpoint{2.740746in}{1.729801in}}{\pgfqpoint{2.748560in}{1.737615in}}%
\pgfpathcurveto{\pgfqpoint{2.756374in}{1.745429in}}{\pgfqpoint{2.760764in}{1.756028in}}{\pgfqpoint{2.760764in}{1.767078in}}%
\pgfpathcurveto{\pgfqpoint{2.760764in}{1.778128in}}{\pgfqpoint{2.756374in}{1.788727in}}{\pgfqpoint{2.748560in}{1.796541in}}%
\pgfpathcurveto{\pgfqpoint{2.740746in}{1.804354in}}{\pgfqpoint{2.730147in}{1.808745in}}{\pgfqpoint{2.719097in}{1.808745in}}%
\pgfpathcurveto{\pgfqpoint{2.708047in}{1.808745in}}{\pgfqpoint{2.697448in}{1.804354in}}{\pgfqpoint{2.689634in}{1.796541in}}%
\pgfpathcurveto{\pgfqpoint{2.681821in}{1.788727in}}{\pgfqpoint{2.677431in}{1.778128in}}{\pgfqpoint{2.677431in}{1.767078in}}%
\pgfpathcurveto{\pgfqpoint{2.677431in}{1.756028in}}{\pgfqpoint{2.681821in}{1.745429in}}{\pgfqpoint{2.689634in}{1.737615in}}%
\pgfpathcurveto{\pgfqpoint{2.697448in}{1.729801in}}{\pgfqpoint{2.708047in}{1.725411in}}{\pgfqpoint{2.719097in}{1.725411in}}%
\pgfpathclose%
\pgfusepath{stroke,fill}%
\end{pgfscope}%
\begin{pgfscope}%
\pgfpathrectangle{\pgfqpoint{0.600000in}{0.600000in}}{\pgfqpoint{3.900000in}{3.900000in}}%
\pgfusepath{clip}%
\pgfsetbuttcap%
\pgfsetroundjoin%
\definecolor{currentfill}{rgb}{0.121569,0.466667,0.705882}%
\pgfsetfillcolor{currentfill}%
\pgfsetlinewidth{1.003750pt}%
\definecolor{currentstroke}{rgb}{0.121569,0.466667,0.705882}%
\pgfsetstrokecolor{currentstroke}%
\pgfsetdash{}{0pt}%
\pgfpathmoveto{\pgfqpoint{2.634213in}{2.835183in}}%
\pgfpathcurveto{\pgfqpoint{2.645263in}{2.835183in}}{\pgfqpoint{2.655862in}{2.839573in}}{\pgfqpoint{2.663676in}{2.847387in}}%
\pgfpathcurveto{\pgfqpoint{2.671489in}{2.855201in}}{\pgfqpoint{2.675879in}{2.865800in}}{\pgfqpoint{2.675879in}{2.876850in}}%
\pgfpathcurveto{\pgfqpoint{2.675879in}{2.887900in}}{\pgfqpoint{2.671489in}{2.898499in}}{\pgfqpoint{2.663676in}{2.906313in}}%
\pgfpathcurveto{\pgfqpoint{2.655862in}{2.914126in}}{\pgfqpoint{2.645263in}{2.918517in}}{\pgfqpoint{2.634213in}{2.918517in}}%
\pgfpathcurveto{\pgfqpoint{2.623163in}{2.918517in}}{\pgfqpoint{2.612564in}{2.914126in}}{\pgfqpoint{2.604750in}{2.906313in}}%
\pgfpathcurveto{\pgfqpoint{2.596936in}{2.898499in}}{\pgfqpoint{2.592546in}{2.887900in}}{\pgfqpoint{2.592546in}{2.876850in}}%
\pgfpathcurveto{\pgfqpoint{2.592546in}{2.865800in}}{\pgfqpoint{2.596936in}{2.855201in}}{\pgfqpoint{2.604750in}{2.847387in}}%
\pgfpathcurveto{\pgfqpoint{2.612564in}{2.839573in}}{\pgfqpoint{2.623163in}{2.835183in}}{\pgfqpoint{2.634213in}{2.835183in}}%
\pgfpathclose%
\pgfusepath{stroke,fill}%
\end{pgfscope}%
\begin{pgfscope}%
\pgfpathrectangle{\pgfqpoint{0.600000in}{0.600000in}}{\pgfqpoint{3.900000in}{3.900000in}}%
\pgfusepath{clip}%
\pgfsetbuttcap%
\pgfsetroundjoin%
\definecolor{currentfill}{rgb}{0.121569,0.466667,0.705882}%
\pgfsetfillcolor{currentfill}%
\pgfsetlinewidth{1.003750pt}%
\definecolor{currentstroke}{rgb}{0.121569,0.466667,0.705882}%
\pgfsetstrokecolor{currentstroke}%
\pgfsetdash{}{0pt}%
\pgfpathmoveto{\pgfqpoint{3.445439in}{3.435699in}}%
\pgfpathcurveto{\pgfqpoint{3.456489in}{3.435699in}}{\pgfqpoint{3.467088in}{3.440089in}}{\pgfqpoint{3.474902in}{3.447902in}}%
\pgfpathcurveto{\pgfqpoint{3.482716in}{3.455716in}}{\pgfqpoint{3.487106in}{3.466315in}}{\pgfqpoint{3.487106in}{3.477365in}}%
\pgfpathcurveto{\pgfqpoint{3.487106in}{3.488415in}}{\pgfqpoint{3.482716in}{3.499014in}}{\pgfqpoint{3.474902in}{3.506828in}}%
\pgfpathcurveto{\pgfqpoint{3.467088in}{3.514642in}}{\pgfqpoint{3.456489in}{3.519032in}}{\pgfqpoint{3.445439in}{3.519032in}}%
\pgfpathcurveto{\pgfqpoint{3.434389in}{3.519032in}}{\pgfqpoint{3.423790in}{3.514642in}}{\pgfqpoint{3.415976in}{3.506828in}}%
\pgfpathcurveto{\pgfqpoint{3.408163in}{3.499014in}}{\pgfqpoint{3.403772in}{3.488415in}}{\pgfqpoint{3.403772in}{3.477365in}}%
\pgfpathcurveto{\pgfqpoint{3.403772in}{3.466315in}}{\pgfqpoint{3.408163in}{3.455716in}}{\pgfqpoint{3.415976in}{3.447902in}}%
\pgfpathcurveto{\pgfqpoint{3.423790in}{3.440089in}}{\pgfqpoint{3.434389in}{3.435699in}}{\pgfqpoint{3.445439in}{3.435699in}}%
\pgfpathclose%
\pgfusepath{stroke,fill}%
\end{pgfscope}%
\begin{pgfscope}%
\pgfpathrectangle{\pgfqpoint{0.600000in}{0.600000in}}{\pgfqpoint{3.900000in}{3.900000in}}%
\pgfusepath{clip}%
\pgfsetbuttcap%
\pgfsetroundjoin%
\definecolor{currentfill}{rgb}{0.121569,0.466667,0.705882}%
\pgfsetfillcolor{currentfill}%
\pgfsetlinewidth{1.003750pt}%
\definecolor{currentstroke}{rgb}{0.121569,0.466667,0.705882}%
\pgfsetstrokecolor{currentstroke}%
\pgfsetdash{}{0pt}%
\pgfpathmoveto{\pgfqpoint{2.499978in}{1.975112in}}%
\pgfpathcurveto{\pgfqpoint{2.511028in}{1.975112in}}{\pgfqpoint{2.521627in}{1.979503in}}{\pgfqpoint{2.529441in}{1.987316in}}%
\pgfpathcurveto{\pgfqpoint{2.537254in}{1.995130in}}{\pgfqpoint{2.541644in}{2.005729in}}{\pgfqpoint{2.541644in}{2.016779in}}%
\pgfpathcurveto{\pgfqpoint{2.541644in}{2.027829in}}{\pgfqpoint{2.537254in}{2.038428in}}{\pgfqpoint{2.529441in}{2.046242in}}%
\pgfpathcurveto{\pgfqpoint{2.521627in}{2.054056in}}{\pgfqpoint{2.511028in}{2.058446in}}{\pgfqpoint{2.499978in}{2.058446in}}%
\pgfpathcurveto{\pgfqpoint{2.488928in}{2.058446in}}{\pgfqpoint{2.478329in}{2.054056in}}{\pgfqpoint{2.470515in}{2.046242in}}%
\pgfpathcurveto{\pgfqpoint{2.462701in}{2.038428in}}{\pgfqpoint{2.458311in}{2.027829in}}{\pgfqpoint{2.458311in}{2.016779in}}%
\pgfpathcurveto{\pgfqpoint{2.458311in}{2.005729in}}{\pgfqpoint{2.462701in}{1.995130in}}{\pgfqpoint{2.470515in}{1.987316in}}%
\pgfpathcurveto{\pgfqpoint{2.478329in}{1.979503in}}{\pgfqpoint{2.488928in}{1.975112in}}{\pgfqpoint{2.499978in}{1.975112in}}%
\pgfpathclose%
\pgfusepath{stroke,fill}%
\end{pgfscope}%
\begin{pgfscope}%
\pgfpathrectangle{\pgfqpoint{0.600000in}{0.600000in}}{\pgfqpoint{3.900000in}{3.900000in}}%
\pgfusepath{clip}%
\pgfsetbuttcap%
\pgfsetroundjoin%
\definecolor{currentfill}{rgb}{0.121569,0.466667,0.705882}%
\pgfsetfillcolor{currentfill}%
\pgfsetlinewidth{1.003750pt}%
\definecolor{currentstroke}{rgb}{0.121569,0.466667,0.705882}%
\pgfsetstrokecolor{currentstroke}%
\pgfsetdash{}{0pt}%
\pgfpathmoveto{\pgfqpoint{2.638193in}{3.069022in}}%
\pgfpathcurveto{\pgfqpoint{2.649243in}{3.069022in}}{\pgfqpoint{2.659842in}{3.073412in}}{\pgfqpoint{2.667655in}{3.081226in}}%
\pgfpathcurveto{\pgfqpoint{2.675469in}{3.089040in}}{\pgfqpoint{2.679859in}{3.099639in}}{\pgfqpoint{2.679859in}{3.110689in}}%
\pgfpathcurveto{\pgfqpoint{2.679859in}{3.121739in}}{\pgfqpoint{2.675469in}{3.132338in}}{\pgfqpoint{2.667655in}{3.140151in}}%
\pgfpathcurveto{\pgfqpoint{2.659842in}{3.147965in}}{\pgfqpoint{2.649243in}{3.152355in}}{\pgfqpoint{2.638193in}{3.152355in}}%
\pgfpathcurveto{\pgfqpoint{2.627142in}{3.152355in}}{\pgfqpoint{2.616543in}{3.147965in}}{\pgfqpoint{2.608730in}{3.140151in}}%
\pgfpathcurveto{\pgfqpoint{2.600916in}{3.132338in}}{\pgfqpoint{2.596526in}{3.121739in}}{\pgfqpoint{2.596526in}{3.110689in}}%
\pgfpathcurveto{\pgfqpoint{2.596526in}{3.099639in}}{\pgfqpoint{2.600916in}{3.089040in}}{\pgfqpoint{2.608730in}{3.081226in}}%
\pgfpathcurveto{\pgfqpoint{2.616543in}{3.073412in}}{\pgfqpoint{2.627142in}{3.069022in}}{\pgfqpoint{2.638193in}{3.069022in}}%
\pgfpathclose%
\pgfusepath{stroke,fill}%
\end{pgfscope}%
\begin{pgfscope}%
\pgfpathrectangle{\pgfqpoint{0.600000in}{0.600000in}}{\pgfqpoint{3.900000in}{3.900000in}}%
\pgfusepath{clip}%
\pgfsetbuttcap%
\pgfsetroundjoin%
\definecolor{currentfill}{rgb}{0.121569,0.466667,0.705882}%
\pgfsetfillcolor{currentfill}%
\pgfsetlinewidth{1.003750pt}%
\definecolor{currentstroke}{rgb}{0.121569,0.466667,0.705882}%
\pgfsetstrokecolor{currentstroke}%
\pgfsetdash{}{0pt}%
\pgfpathmoveto{\pgfqpoint{2.073393in}{3.023143in}}%
\pgfpathcurveto{\pgfqpoint{2.084443in}{3.023143in}}{\pgfqpoint{2.095042in}{3.027534in}}{\pgfqpoint{2.102856in}{3.035347in}}%
\pgfpathcurveto{\pgfqpoint{2.110670in}{3.043161in}}{\pgfqpoint{2.115060in}{3.053760in}}{\pgfqpoint{2.115060in}{3.064810in}}%
\pgfpathcurveto{\pgfqpoint{2.115060in}{3.075860in}}{\pgfqpoint{2.110670in}{3.086459in}}{\pgfqpoint{2.102856in}{3.094273in}}%
\pgfpathcurveto{\pgfqpoint{2.095042in}{3.102087in}}{\pgfqpoint{2.084443in}{3.106477in}}{\pgfqpoint{2.073393in}{3.106477in}}%
\pgfpathcurveto{\pgfqpoint{2.062343in}{3.106477in}}{\pgfqpoint{2.051744in}{3.102087in}}{\pgfqpoint{2.043930in}{3.094273in}}%
\pgfpathcurveto{\pgfqpoint{2.036117in}{3.086459in}}{\pgfqpoint{2.031727in}{3.075860in}}{\pgfqpoint{2.031727in}{3.064810in}}%
\pgfpathcurveto{\pgfqpoint{2.031727in}{3.053760in}}{\pgfqpoint{2.036117in}{3.043161in}}{\pgfqpoint{2.043930in}{3.035347in}}%
\pgfpathcurveto{\pgfqpoint{2.051744in}{3.027534in}}{\pgfqpoint{2.062343in}{3.023143in}}{\pgfqpoint{2.073393in}{3.023143in}}%
\pgfpathclose%
\pgfusepath{stroke,fill}%
\end{pgfscope}%
\begin{pgfscope}%
\pgfpathrectangle{\pgfqpoint{0.600000in}{0.600000in}}{\pgfqpoint{3.900000in}{3.900000in}}%
\pgfusepath{clip}%
\pgfsetbuttcap%
\pgfsetroundjoin%
\definecolor{currentfill}{rgb}{0.121569,0.466667,0.705882}%
\pgfsetfillcolor{currentfill}%
\pgfsetlinewidth{1.003750pt}%
\definecolor{currentstroke}{rgb}{0.121569,0.466667,0.705882}%
\pgfsetstrokecolor{currentstroke}%
\pgfsetdash{}{0pt}%
\pgfpathmoveto{\pgfqpoint{3.231714in}{2.484425in}}%
\pgfpathcurveto{\pgfqpoint{3.242764in}{2.484425in}}{\pgfqpoint{3.253363in}{2.488815in}}{\pgfqpoint{3.261177in}{2.496629in}}%
\pgfpathcurveto{\pgfqpoint{3.268990in}{2.504443in}}{\pgfqpoint{3.273380in}{2.515042in}}{\pgfqpoint{3.273380in}{2.526092in}}%
\pgfpathcurveto{\pgfqpoint{3.273380in}{2.537142in}}{\pgfqpoint{3.268990in}{2.547741in}}{\pgfqpoint{3.261177in}{2.555554in}}%
\pgfpathcurveto{\pgfqpoint{3.253363in}{2.563368in}}{\pgfqpoint{3.242764in}{2.567758in}}{\pgfqpoint{3.231714in}{2.567758in}}%
\pgfpathcurveto{\pgfqpoint{3.220664in}{2.567758in}}{\pgfqpoint{3.210065in}{2.563368in}}{\pgfqpoint{3.202251in}{2.555554in}}%
\pgfpathcurveto{\pgfqpoint{3.194437in}{2.547741in}}{\pgfqpoint{3.190047in}{2.537142in}}{\pgfqpoint{3.190047in}{2.526092in}}%
\pgfpathcurveto{\pgfqpoint{3.190047in}{2.515042in}}{\pgfqpoint{3.194437in}{2.504443in}}{\pgfqpoint{3.202251in}{2.496629in}}%
\pgfpathcurveto{\pgfqpoint{3.210065in}{2.488815in}}{\pgfqpoint{3.220664in}{2.484425in}}{\pgfqpoint{3.231714in}{2.484425in}}%
\pgfpathclose%
\pgfusepath{stroke,fill}%
\end{pgfscope}%
\begin{pgfscope}%
\pgfpathrectangle{\pgfqpoint{0.600000in}{0.600000in}}{\pgfqpoint{3.900000in}{3.900000in}}%
\pgfusepath{clip}%
\pgfsetbuttcap%
\pgfsetroundjoin%
\definecolor{currentfill}{rgb}{0.121569,0.466667,0.705882}%
\pgfsetfillcolor{currentfill}%
\pgfsetlinewidth{1.003750pt}%
\definecolor{currentstroke}{rgb}{0.121569,0.466667,0.705882}%
\pgfsetstrokecolor{currentstroke}%
\pgfsetdash{}{0pt}%
\pgfpathmoveto{\pgfqpoint{2.663834in}{2.033854in}}%
\pgfpathcurveto{\pgfqpoint{2.674884in}{2.033854in}}{\pgfqpoint{2.685483in}{2.038244in}}{\pgfqpoint{2.693296in}{2.046058in}}%
\pgfpathcurveto{\pgfqpoint{2.701110in}{2.053871in}}{\pgfqpoint{2.705500in}{2.064470in}}{\pgfqpoint{2.705500in}{2.075520in}}%
\pgfpathcurveto{\pgfqpoint{2.705500in}{2.086571in}}{\pgfqpoint{2.701110in}{2.097170in}}{\pgfqpoint{2.693296in}{2.104983in}}%
\pgfpathcurveto{\pgfqpoint{2.685483in}{2.112797in}}{\pgfqpoint{2.674884in}{2.117187in}}{\pgfqpoint{2.663834in}{2.117187in}}%
\pgfpathcurveto{\pgfqpoint{2.652783in}{2.117187in}}{\pgfqpoint{2.642184in}{2.112797in}}{\pgfqpoint{2.634371in}{2.104983in}}%
\pgfpathcurveto{\pgfqpoint{2.626557in}{2.097170in}}{\pgfqpoint{2.622167in}{2.086571in}}{\pgfqpoint{2.622167in}{2.075520in}}%
\pgfpathcurveto{\pgfqpoint{2.622167in}{2.064470in}}{\pgfqpoint{2.626557in}{2.053871in}}{\pgfqpoint{2.634371in}{2.046058in}}%
\pgfpathcurveto{\pgfqpoint{2.642184in}{2.038244in}}{\pgfqpoint{2.652783in}{2.033854in}}{\pgfqpoint{2.663834in}{2.033854in}}%
\pgfpathclose%
\pgfusepath{stroke,fill}%
\end{pgfscope}%
\begin{pgfscope}%
\pgfpathrectangle{\pgfqpoint{0.600000in}{0.600000in}}{\pgfqpoint{3.900000in}{3.900000in}}%
\pgfusepath{clip}%
\pgfsetbuttcap%
\pgfsetroundjoin%
\definecolor{currentfill}{rgb}{0.121569,0.466667,0.705882}%
\pgfsetfillcolor{currentfill}%
\pgfsetlinewidth{1.003750pt}%
\definecolor{currentstroke}{rgb}{0.121569,0.466667,0.705882}%
\pgfsetstrokecolor{currentstroke}%
\pgfsetdash{}{0pt}%
\pgfpathmoveto{\pgfqpoint{2.120506in}{2.758547in}}%
\pgfpathcurveto{\pgfqpoint{2.131556in}{2.758547in}}{\pgfqpoint{2.142155in}{2.762937in}}{\pgfqpoint{2.149969in}{2.770751in}}%
\pgfpathcurveto{\pgfqpoint{2.157782in}{2.778564in}}{\pgfqpoint{2.162173in}{2.789163in}}{\pgfqpoint{2.162173in}{2.800213in}}%
\pgfpathcurveto{\pgfqpoint{2.162173in}{2.811264in}}{\pgfqpoint{2.157782in}{2.821863in}}{\pgfqpoint{2.149969in}{2.829676in}}%
\pgfpathcurveto{\pgfqpoint{2.142155in}{2.837490in}}{\pgfqpoint{2.131556in}{2.841880in}}{\pgfqpoint{2.120506in}{2.841880in}}%
\pgfpathcurveto{\pgfqpoint{2.109456in}{2.841880in}}{\pgfqpoint{2.098857in}{2.837490in}}{\pgfqpoint{2.091043in}{2.829676in}}%
\pgfpathcurveto{\pgfqpoint{2.083229in}{2.821863in}}{\pgfqpoint{2.078839in}{2.811264in}}{\pgfqpoint{2.078839in}{2.800213in}}%
\pgfpathcurveto{\pgfqpoint{2.078839in}{2.789163in}}{\pgfqpoint{2.083229in}{2.778564in}}{\pgfqpoint{2.091043in}{2.770751in}}%
\pgfpathcurveto{\pgfqpoint{2.098857in}{2.762937in}}{\pgfqpoint{2.109456in}{2.758547in}}{\pgfqpoint{2.120506in}{2.758547in}}%
\pgfpathclose%
\pgfusepath{stroke,fill}%
\end{pgfscope}%
\begin{pgfscope}%
\pgfpathrectangle{\pgfqpoint{0.600000in}{0.600000in}}{\pgfqpoint{3.900000in}{3.900000in}}%
\pgfusepath{clip}%
\pgfsetbuttcap%
\pgfsetroundjoin%
\definecolor{currentfill}{rgb}{0.121569,0.466667,0.705882}%
\pgfsetfillcolor{currentfill}%
\pgfsetlinewidth{1.003750pt}%
\definecolor{currentstroke}{rgb}{0.121569,0.466667,0.705882}%
\pgfsetstrokecolor{currentstroke}%
\pgfsetdash{}{0pt}%
\pgfpathmoveto{\pgfqpoint{2.670035in}{2.993170in}}%
\pgfpathcurveto{\pgfqpoint{2.681086in}{2.993170in}}{\pgfqpoint{2.691685in}{2.997561in}}{\pgfqpoint{2.699498in}{3.005374in}}%
\pgfpathcurveto{\pgfqpoint{2.707312in}{3.013188in}}{\pgfqpoint{2.711702in}{3.023787in}}{\pgfqpoint{2.711702in}{3.034837in}}%
\pgfpathcurveto{\pgfqpoint{2.711702in}{3.045887in}}{\pgfqpoint{2.707312in}{3.056486in}}{\pgfqpoint{2.699498in}{3.064300in}}%
\pgfpathcurveto{\pgfqpoint{2.691685in}{3.072113in}}{\pgfqpoint{2.681086in}{3.076504in}}{\pgfqpoint{2.670035in}{3.076504in}}%
\pgfpathcurveto{\pgfqpoint{2.658985in}{3.076504in}}{\pgfqpoint{2.648386in}{3.072113in}}{\pgfqpoint{2.640573in}{3.064300in}}%
\pgfpathcurveto{\pgfqpoint{2.632759in}{3.056486in}}{\pgfqpoint{2.628369in}{3.045887in}}{\pgfqpoint{2.628369in}{3.034837in}}%
\pgfpathcurveto{\pgfqpoint{2.628369in}{3.023787in}}{\pgfqpoint{2.632759in}{3.013188in}}{\pgfqpoint{2.640573in}{3.005374in}}%
\pgfpathcurveto{\pgfqpoint{2.648386in}{2.997561in}}{\pgfqpoint{2.658985in}{2.993170in}}{\pgfqpoint{2.670035in}{2.993170in}}%
\pgfpathclose%
\pgfusepath{stroke,fill}%
\end{pgfscope}%
\begin{pgfscope}%
\pgfpathrectangle{\pgfqpoint{0.600000in}{0.600000in}}{\pgfqpoint{3.900000in}{3.900000in}}%
\pgfusepath{clip}%
\pgfsetbuttcap%
\pgfsetroundjoin%
\definecolor{currentfill}{rgb}{0.121569,0.466667,0.705882}%
\pgfsetfillcolor{currentfill}%
\pgfsetlinewidth{1.003750pt}%
\definecolor{currentstroke}{rgb}{0.121569,0.466667,0.705882}%
\pgfsetstrokecolor{currentstroke}%
\pgfsetdash{}{0pt}%
\pgfpathmoveto{\pgfqpoint{2.379297in}{1.790012in}}%
\pgfpathcurveto{\pgfqpoint{2.390347in}{1.790012in}}{\pgfqpoint{2.400946in}{1.794402in}}{\pgfqpoint{2.408760in}{1.802216in}}%
\pgfpathcurveto{\pgfqpoint{2.416574in}{1.810029in}}{\pgfqpoint{2.420964in}{1.820628in}}{\pgfqpoint{2.420964in}{1.831678in}}%
\pgfpathcurveto{\pgfqpoint{2.420964in}{1.842729in}}{\pgfqpoint{2.416574in}{1.853328in}}{\pgfqpoint{2.408760in}{1.861141in}}%
\pgfpathcurveto{\pgfqpoint{2.400946in}{1.868955in}}{\pgfqpoint{2.390347in}{1.873345in}}{\pgfqpoint{2.379297in}{1.873345in}}%
\pgfpathcurveto{\pgfqpoint{2.368247in}{1.873345in}}{\pgfqpoint{2.357648in}{1.868955in}}{\pgfqpoint{2.349834in}{1.861141in}}%
\pgfpathcurveto{\pgfqpoint{2.342021in}{1.853328in}}{\pgfqpoint{2.337631in}{1.842729in}}{\pgfqpoint{2.337631in}{1.831678in}}%
\pgfpathcurveto{\pgfqpoint{2.337631in}{1.820628in}}{\pgfqpoint{2.342021in}{1.810029in}}{\pgfqpoint{2.349834in}{1.802216in}}%
\pgfpathcurveto{\pgfqpoint{2.357648in}{1.794402in}}{\pgfqpoint{2.368247in}{1.790012in}}{\pgfqpoint{2.379297in}{1.790012in}}%
\pgfpathclose%
\pgfusepath{stroke,fill}%
\end{pgfscope}%
\begin{pgfscope}%
\pgfpathrectangle{\pgfqpoint{0.600000in}{0.600000in}}{\pgfqpoint{3.900000in}{3.900000in}}%
\pgfusepath{clip}%
\pgfsetbuttcap%
\pgfsetroundjoin%
\definecolor{currentfill}{rgb}{0.121569,0.466667,0.705882}%
\pgfsetfillcolor{currentfill}%
\pgfsetlinewidth{1.003750pt}%
\definecolor{currentstroke}{rgb}{0.121569,0.466667,0.705882}%
\pgfsetstrokecolor{currentstroke}%
\pgfsetdash{}{0pt}%
\pgfpathmoveto{\pgfqpoint{2.062767in}{2.795754in}}%
\pgfpathcurveto{\pgfqpoint{2.073817in}{2.795754in}}{\pgfqpoint{2.084416in}{2.800144in}}{\pgfqpoint{2.092230in}{2.807957in}}%
\pgfpathcurveto{\pgfqpoint{2.100043in}{2.815771in}}{\pgfqpoint{2.104434in}{2.826370in}}{\pgfqpoint{2.104434in}{2.837420in}}%
\pgfpathcurveto{\pgfqpoint{2.104434in}{2.848470in}}{\pgfqpoint{2.100043in}{2.859069in}}{\pgfqpoint{2.092230in}{2.866883in}}%
\pgfpathcurveto{\pgfqpoint{2.084416in}{2.874697in}}{\pgfqpoint{2.073817in}{2.879087in}}{\pgfqpoint{2.062767in}{2.879087in}}%
\pgfpathcurveto{\pgfqpoint{2.051717in}{2.879087in}}{\pgfqpoint{2.041118in}{2.874697in}}{\pgfqpoint{2.033304in}{2.866883in}}%
\pgfpathcurveto{\pgfqpoint{2.025491in}{2.859069in}}{\pgfqpoint{2.021100in}{2.848470in}}{\pgfqpoint{2.021100in}{2.837420in}}%
\pgfpathcurveto{\pgfqpoint{2.021100in}{2.826370in}}{\pgfqpoint{2.025491in}{2.815771in}}{\pgfqpoint{2.033304in}{2.807957in}}%
\pgfpathcurveto{\pgfqpoint{2.041118in}{2.800144in}}{\pgfqpoint{2.051717in}{2.795754in}}{\pgfqpoint{2.062767in}{2.795754in}}%
\pgfpathclose%
\pgfusepath{stroke,fill}%
\end{pgfscope}%
\begin{pgfscope}%
\pgfpathrectangle{\pgfqpoint{0.600000in}{0.600000in}}{\pgfqpoint{3.900000in}{3.900000in}}%
\pgfusepath{clip}%
\pgfsetbuttcap%
\pgfsetroundjoin%
\definecolor{currentfill}{rgb}{0.121569,0.466667,0.705882}%
\pgfsetfillcolor{currentfill}%
\pgfsetlinewidth{1.003750pt}%
\definecolor{currentstroke}{rgb}{0.121569,0.466667,0.705882}%
\pgfsetstrokecolor{currentstroke}%
\pgfsetdash{}{0pt}%
\pgfpathmoveto{\pgfqpoint{2.112679in}{2.142903in}}%
\pgfpathcurveto{\pgfqpoint{2.123729in}{2.142903in}}{\pgfqpoint{2.134328in}{2.147293in}}{\pgfqpoint{2.142142in}{2.155107in}}%
\pgfpathcurveto{\pgfqpoint{2.149955in}{2.162921in}}{\pgfqpoint{2.154345in}{2.173520in}}{\pgfqpoint{2.154345in}{2.184570in}}%
\pgfpathcurveto{\pgfqpoint{2.154345in}{2.195620in}}{\pgfqpoint{2.149955in}{2.206219in}}{\pgfqpoint{2.142142in}{2.214032in}}%
\pgfpathcurveto{\pgfqpoint{2.134328in}{2.221846in}}{\pgfqpoint{2.123729in}{2.226236in}}{\pgfqpoint{2.112679in}{2.226236in}}%
\pgfpathcurveto{\pgfqpoint{2.101629in}{2.226236in}}{\pgfqpoint{2.091030in}{2.221846in}}{\pgfqpoint{2.083216in}{2.214032in}}%
\pgfpathcurveto{\pgfqpoint{2.075402in}{2.206219in}}{\pgfqpoint{2.071012in}{2.195620in}}{\pgfqpoint{2.071012in}{2.184570in}}%
\pgfpathcurveto{\pgfqpoint{2.071012in}{2.173520in}}{\pgfqpoint{2.075402in}{2.162921in}}{\pgfqpoint{2.083216in}{2.155107in}}%
\pgfpathcurveto{\pgfqpoint{2.091030in}{2.147293in}}{\pgfqpoint{2.101629in}{2.142903in}}{\pgfqpoint{2.112679in}{2.142903in}}%
\pgfpathclose%
\pgfusepath{stroke,fill}%
\end{pgfscope}%
\begin{pgfscope}%
\pgfpathrectangle{\pgfqpoint{0.600000in}{0.600000in}}{\pgfqpoint{3.900000in}{3.900000in}}%
\pgfusepath{clip}%
\pgfsetbuttcap%
\pgfsetroundjoin%
\definecolor{currentfill}{rgb}{0.121569,0.466667,0.705882}%
\pgfsetfillcolor{currentfill}%
\pgfsetlinewidth{1.003750pt}%
\definecolor{currentstroke}{rgb}{0.121569,0.466667,0.705882}%
\pgfsetstrokecolor{currentstroke}%
\pgfsetdash{}{0pt}%
\pgfpathmoveto{\pgfqpoint{1.761548in}{2.487422in}}%
\pgfpathcurveto{\pgfqpoint{1.772598in}{2.487422in}}{\pgfqpoint{1.783197in}{2.491812in}}{\pgfqpoint{1.791011in}{2.499626in}}%
\pgfpathcurveto{\pgfqpoint{1.798824in}{2.507439in}}{\pgfqpoint{1.803214in}{2.518038in}}{\pgfqpoint{1.803214in}{2.529088in}}%
\pgfpathcurveto{\pgfqpoint{1.803214in}{2.540138in}}{\pgfqpoint{1.798824in}{2.550737in}}{\pgfqpoint{1.791011in}{2.558551in}}%
\pgfpathcurveto{\pgfqpoint{1.783197in}{2.566365in}}{\pgfqpoint{1.772598in}{2.570755in}}{\pgfqpoint{1.761548in}{2.570755in}}%
\pgfpathcurveto{\pgfqpoint{1.750498in}{2.570755in}}{\pgfqpoint{1.739899in}{2.566365in}}{\pgfqpoint{1.732085in}{2.558551in}}%
\pgfpathcurveto{\pgfqpoint{1.724271in}{2.550737in}}{\pgfqpoint{1.719881in}{2.540138in}}{\pgfqpoint{1.719881in}{2.529088in}}%
\pgfpathcurveto{\pgfqpoint{1.719881in}{2.518038in}}{\pgfqpoint{1.724271in}{2.507439in}}{\pgfqpoint{1.732085in}{2.499626in}}%
\pgfpathcurveto{\pgfqpoint{1.739899in}{2.491812in}}{\pgfqpoint{1.750498in}{2.487422in}}{\pgfqpoint{1.761548in}{2.487422in}}%
\pgfpathclose%
\pgfusepath{stroke,fill}%
\end{pgfscope}%
\begin{pgfscope}%
\pgfpathrectangle{\pgfqpoint{0.600000in}{0.600000in}}{\pgfqpoint{3.900000in}{3.900000in}}%
\pgfusepath{clip}%
\pgfsetbuttcap%
\pgfsetroundjoin%
\definecolor{currentfill}{rgb}{0.121569,0.466667,0.705882}%
\pgfsetfillcolor{currentfill}%
\pgfsetlinewidth{1.003750pt}%
\definecolor{currentstroke}{rgb}{0.121569,0.466667,0.705882}%
\pgfsetstrokecolor{currentstroke}%
\pgfsetdash{}{0pt}%
\pgfpathmoveto{\pgfqpoint{1.689573in}{2.308423in}}%
\pgfpathcurveto{\pgfqpoint{1.700623in}{2.308423in}}{\pgfqpoint{1.711222in}{2.312813in}}{\pgfqpoint{1.719036in}{2.320627in}}%
\pgfpathcurveto{\pgfqpoint{1.726849in}{2.328441in}}{\pgfqpoint{1.731240in}{2.339040in}}{\pgfqpoint{1.731240in}{2.350090in}}%
\pgfpathcurveto{\pgfqpoint{1.731240in}{2.361140in}}{\pgfqpoint{1.726849in}{2.371739in}}{\pgfqpoint{1.719036in}{2.379553in}}%
\pgfpathcurveto{\pgfqpoint{1.711222in}{2.387366in}}{\pgfqpoint{1.700623in}{2.391756in}}{\pgfqpoint{1.689573in}{2.391756in}}%
\pgfpathcurveto{\pgfqpoint{1.678523in}{2.391756in}}{\pgfqpoint{1.667924in}{2.387366in}}{\pgfqpoint{1.660110in}{2.379553in}}%
\pgfpathcurveto{\pgfqpoint{1.652296in}{2.371739in}}{\pgfqpoint{1.647906in}{2.361140in}}{\pgfqpoint{1.647906in}{2.350090in}}%
\pgfpathcurveto{\pgfqpoint{1.647906in}{2.339040in}}{\pgfqpoint{1.652296in}{2.328441in}}{\pgfqpoint{1.660110in}{2.320627in}}%
\pgfpathcurveto{\pgfqpoint{1.667924in}{2.312813in}}{\pgfqpoint{1.678523in}{2.308423in}}{\pgfqpoint{1.689573in}{2.308423in}}%
\pgfpathclose%
\pgfusepath{stroke,fill}%
\end{pgfscope}%
\begin{pgfscope}%
\pgfpathrectangle{\pgfqpoint{0.600000in}{0.600000in}}{\pgfqpoint{3.900000in}{3.900000in}}%
\pgfusepath{clip}%
\pgfsetbuttcap%
\pgfsetroundjoin%
\definecolor{currentfill}{rgb}{0.121569,0.466667,0.705882}%
\pgfsetfillcolor{currentfill}%
\pgfsetlinewidth{1.003750pt}%
\definecolor{currentstroke}{rgb}{0.121569,0.466667,0.705882}%
\pgfsetstrokecolor{currentstroke}%
\pgfsetdash{}{0pt}%
\pgfpathmoveto{\pgfqpoint{2.031592in}{2.108918in}}%
\pgfpathcurveto{\pgfqpoint{2.042642in}{2.108918in}}{\pgfqpoint{2.053241in}{2.113308in}}{\pgfqpoint{2.061055in}{2.121122in}}%
\pgfpathcurveto{\pgfqpoint{2.068869in}{2.128936in}}{\pgfqpoint{2.073259in}{2.139535in}}{\pgfqpoint{2.073259in}{2.150585in}}%
\pgfpathcurveto{\pgfqpoint{2.073259in}{2.161635in}}{\pgfqpoint{2.068869in}{2.172234in}}{\pgfqpoint{2.061055in}{2.180048in}}%
\pgfpathcurveto{\pgfqpoint{2.053241in}{2.187861in}}{\pgfqpoint{2.042642in}{2.192251in}}{\pgfqpoint{2.031592in}{2.192251in}}%
\pgfpathcurveto{\pgfqpoint{2.020542in}{2.192251in}}{\pgfqpoint{2.009943in}{2.187861in}}{\pgfqpoint{2.002129in}{2.180048in}}%
\pgfpathcurveto{\pgfqpoint{1.994316in}{2.172234in}}{\pgfqpoint{1.989926in}{2.161635in}}{\pgfqpoint{1.989926in}{2.150585in}}%
\pgfpathcurveto{\pgfqpoint{1.989926in}{2.139535in}}{\pgfqpoint{1.994316in}{2.128936in}}{\pgfqpoint{2.002129in}{2.121122in}}%
\pgfpathcurveto{\pgfqpoint{2.009943in}{2.113308in}}{\pgfqpoint{2.020542in}{2.108918in}}{\pgfqpoint{2.031592in}{2.108918in}}%
\pgfpathclose%
\pgfusepath{stroke,fill}%
\end{pgfscope}%
\begin{pgfscope}%
\pgfpathrectangle{\pgfqpoint{0.600000in}{0.600000in}}{\pgfqpoint{3.900000in}{3.900000in}}%
\pgfusepath{clip}%
\pgfsetbuttcap%
\pgfsetroundjoin%
\definecolor{currentfill}{rgb}{0.121569,0.466667,0.705882}%
\pgfsetfillcolor{currentfill}%
\pgfsetlinewidth{1.003750pt}%
\definecolor{currentstroke}{rgb}{0.121569,0.466667,0.705882}%
\pgfsetstrokecolor{currentstroke}%
\pgfsetdash{}{0pt}%
\pgfpathmoveto{\pgfqpoint{2.939125in}{3.153551in}}%
\pgfpathcurveto{\pgfqpoint{2.950175in}{3.153551in}}{\pgfqpoint{2.960774in}{3.157941in}}{\pgfqpoint{2.968588in}{3.165755in}}%
\pgfpathcurveto{\pgfqpoint{2.976401in}{3.173568in}}{\pgfqpoint{2.980791in}{3.184167in}}{\pgfqpoint{2.980791in}{3.195217in}}%
\pgfpathcurveto{\pgfqpoint{2.980791in}{3.206268in}}{\pgfqpoint{2.976401in}{3.216867in}}{\pgfqpoint{2.968588in}{3.224680in}}%
\pgfpathcurveto{\pgfqpoint{2.960774in}{3.232494in}}{\pgfqpoint{2.950175in}{3.236884in}}{\pgfqpoint{2.939125in}{3.236884in}}%
\pgfpathcurveto{\pgfqpoint{2.928075in}{3.236884in}}{\pgfqpoint{2.917476in}{3.232494in}}{\pgfqpoint{2.909662in}{3.224680in}}%
\pgfpathcurveto{\pgfqpoint{2.901848in}{3.216867in}}{\pgfqpoint{2.897458in}{3.206268in}}{\pgfqpoint{2.897458in}{3.195217in}}%
\pgfpathcurveto{\pgfqpoint{2.897458in}{3.184167in}}{\pgfqpoint{2.901848in}{3.173568in}}{\pgfqpoint{2.909662in}{3.165755in}}%
\pgfpathcurveto{\pgfqpoint{2.917476in}{3.157941in}}{\pgfqpoint{2.928075in}{3.153551in}}{\pgfqpoint{2.939125in}{3.153551in}}%
\pgfpathclose%
\pgfusepath{stroke,fill}%
\end{pgfscope}%
\begin{pgfscope}%
\pgfpathrectangle{\pgfqpoint{0.600000in}{0.600000in}}{\pgfqpoint{3.900000in}{3.900000in}}%
\pgfusepath{clip}%
\pgfsetbuttcap%
\pgfsetroundjoin%
\definecolor{currentfill}{rgb}{0.121569,0.466667,0.705882}%
\pgfsetfillcolor{currentfill}%
\pgfsetlinewidth{1.003750pt}%
\definecolor{currentstroke}{rgb}{0.121569,0.466667,0.705882}%
\pgfsetstrokecolor{currentstroke}%
\pgfsetdash{}{0pt}%
\pgfpathmoveto{\pgfqpoint{3.448627in}{2.157873in}}%
\pgfpathcurveto{\pgfqpoint{3.459677in}{2.157873in}}{\pgfqpoint{3.470276in}{2.162263in}}{\pgfqpoint{3.478089in}{2.170077in}}%
\pgfpathcurveto{\pgfqpoint{3.485903in}{2.177890in}}{\pgfqpoint{3.490293in}{2.188489in}}{\pgfqpoint{3.490293in}{2.199539in}}%
\pgfpathcurveto{\pgfqpoint{3.490293in}{2.210590in}}{\pgfqpoint{3.485903in}{2.221189in}}{\pgfqpoint{3.478089in}{2.229002in}}%
\pgfpathcurveto{\pgfqpoint{3.470276in}{2.236816in}}{\pgfqpoint{3.459677in}{2.241206in}}{\pgfqpoint{3.448627in}{2.241206in}}%
\pgfpathcurveto{\pgfqpoint{3.437577in}{2.241206in}}{\pgfqpoint{3.426978in}{2.236816in}}{\pgfqpoint{3.419164in}{2.229002in}}%
\pgfpathcurveto{\pgfqpoint{3.411350in}{2.221189in}}{\pgfqpoint{3.406960in}{2.210590in}}{\pgfqpoint{3.406960in}{2.199539in}}%
\pgfpathcurveto{\pgfqpoint{3.406960in}{2.188489in}}{\pgfqpoint{3.411350in}{2.177890in}}{\pgfqpoint{3.419164in}{2.170077in}}%
\pgfpathcurveto{\pgfqpoint{3.426978in}{2.162263in}}{\pgfqpoint{3.437577in}{2.157873in}}{\pgfqpoint{3.448627in}{2.157873in}}%
\pgfpathclose%
\pgfusepath{stroke,fill}%
\end{pgfscope}%
\begin{pgfscope}%
\pgfpathrectangle{\pgfqpoint{0.600000in}{0.600000in}}{\pgfqpoint{3.900000in}{3.900000in}}%
\pgfusepath{clip}%
\pgfsetbuttcap%
\pgfsetroundjoin%
\definecolor{currentfill}{rgb}{0.121569,0.466667,0.705882}%
\pgfsetfillcolor{currentfill}%
\pgfsetlinewidth{1.003750pt}%
\definecolor{currentstroke}{rgb}{0.121569,0.466667,0.705882}%
\pgfsetstrokecolor{currentstroke}%
\pgfsetdash{}{0pt}%
\pgfpathmoveto{\pgfqpoint{3.280564in}{2.222351in}}%
\pgfpathcurveto{\pgfqpoint{3.291614in}{2.222351in}}{\pgfqpoint{3.302213in}{2.226741in}}{\pgfqpoint{3.310027in}{2.234555in}}%
\pgfpathcurveto{\pgfqpoint{3.317840in}{2.242368in}}{\pgfqpoint{3.322231in}{2.252967in}}{\pgfqpoint{3.322231in}{2.264017in}}%
\pgfpathcurveto{\pgfqpoint{3.322231in}{2.275068in}}{\pgfqpoint{3.317840in}{2.285667in}}{\pgfqpoint{3.310027in}{2.293480in}}%
\pgfpathcurveto{\pgfqpoint{3.302213in}{2.301294in}}{\pgfqpoint{3.291614in}{2.305684in}}{\pgfqpoint{3.280564in}{2.305684in}}%
\pgfpathcurveto{\pgfqpoint{3.269514in}{2.305684in}}{\pgfqpoint{3.258915in}{2.301294in}}{\pgfqpoint{3.251101in}{2.293480in}}%
\pgfpathcurveto{\pgfqpoint{3.243287in}{2.285667in}}{\pgfqpoint{3.238897in}{2.275068in}}{\pgfqpoint{3.238897in}{2.264017in}}%
\pgfpathcurveto{\pgfqpoint{3.238897in}{2.252967in}}{\pgfqpoint{3.243287in}{2.242368in}}{\pgfqpoint{3.251101in}{2.234555in}}%
\pgfpathcurveto{\pgfqpoint{3.258915in}{2.226741in}}{\pgfqpoint{3.269514in}{2.222351in}}{\pgfqpoint{3.280564in}{2.222351in}}%
\pgfpathclose%
\pgfusepath{stroke,fill}%
\end{pgfscope}%
\begin{pgfscope}%
\pgfpathrectangle{\pgfqpoint{0.600000in}{0.600000in}}{\pgfqpoint{3.900000in}{3.900000in}}%
\pgfusepath{clip}%
\pgfsetbuttcap%
\pgfsetroundjoin%
\definecolor{currentfill}{rgb}{0.121569,0.466667,0.705882}%
\pgfsetfillcolor{currentfill}%
\pgfsetlinewidth{1.003750pt}%
\definecolor{currentstroke}{rgb}{0.121569,0.466667,0.705882}%
\pgfsetstrokecolor{currentstroke}%
\pgfsetdash{}{0pt}%
\pgfpathmoveto{\pgfqpoint{1.876026in}{3.263007in}}%
\pgfpathcurveto{\pgfqpoint{1.887077in}{3.263007in}}{\pgfqpoint{1.897676in}{3.267397in}}{\pgfqpoint{1.905489in}{3.275211in}}%
\pgfpathcurveto{\pgfqpoint{1.913303in}{3.283025in}}{\pgfqpoint{1.917693in}{3.293624in}}{\pgfqpoint{1.917693in}{3.304674in}}%
\pgfpathcurveto{\pgfqpoint{1.917693in}{3.315724in}}{\pgfqpoint{1.913303in}{3.326323in}}{\pgfqpoint{1.905489in}{3.334137in}}%
\pgfpathcurveto{\pgfqpoint{1.897676in}{3.341950in}}{\pgfqpoint{1.887077in}{3.346341in}}{\pgfqpoint{1.876026in}{3.346341in}}%
\pgfpathcurveto{\pgfqpoint{1.864976in}{3.346341in}}{\pgfqpoint{1.854377in}{3.341950in}}{\pgfqpoint{1.846564in}{3.334137in}}%
\pgfpathcurveto{\pgfqpoint{1.838750in}{3.326323in}}{\pgfqpoint{1.834360in}{3.315724in}}{\pgfqpoint{1.834360in}{3.304674in}}%
\pgfpathcurveto{\pgfqpoint{1.834360in}{3.293624in}}{\pgfqpoint{1.838750in}{3.283025in}}{\pgfqpoint{1.846564in}{3.275211in}}%
\pgfpathcurveto{\pgfqpoint{1.854377in}{3.267397in}}{\pgfqpoint{1.864976in}{3.263007in}}{\pgfqpoint{1.876026in}{3.263007in}}%
\pgfpathclose%
\pgfusepath{stroke,fill}%
\end{pgfscope}%
\begin{pgfscope}%
\pgfpathrectangle{\pgfqpoint{0.600000in}{0.600000in}}{\pgfqpoint{3.900000in}{3.900000in}}%
\pgfusepath{clip}%
\pgfsetbuttcap%
\pgfsetroundjoin%
\definecolor{currentfill}{rgb}{0.121569,0.466667,0.705882}%
\pgfsetfillcolor{currentfill}%
\pgfsetlinewidth{1.003750pt}%
\definecolor{currentstroke}{rgb}{0.121569,0.466667,0.705882}%
\pgfsetstrokecolor{currentstroke}%
\pgfsetdash{}{0pt}%
\pgfpathmoveto{\pgfqpoint{2.590784in}{2.989740in}}%
\pgfpathcurveto{\pgfqpoint{2.601834in}{2.989740in}}{\pgfqpoint{2.612433in}{2.994131in}}{\pgfqpoint{2.620247in}{3.001944in}}%
\pgfpathcurveto{\pgfqpoint{2.628060in}{3.009758in}}{\pgfqpoint{2.632451in}{3.020357in}}{\pgfqpoint{2.632451in}{3.031407in}}%
\pgfpathcurveto{\pgfqpoint{2.632451in}{3.042457in}}{\pgfqpoint{2.628060in}{3.053056in}}{\pgfqpoint{2.620247in}{3.060870in}}%
\pgfpathcurveto{\pgfqpoint{2.612433in}{3.068683in}}{\pgfqpoint{2.601834in}{3.073074in}}{\pgfqpoint{2.590784in}{3.073074in}}%
\pgfpathcurveto{\pgfqpoint{2.579734in}{3.073074in}}{\pgfqpoint{2.569135in}{3.068683in}}{\pgfqpoint{2.561321in}{3.060870in}}%
\pgfpathcurveto{\pgfqpoint{2.553508in}{3.053056in}}{\pgfqpoint{2.549117in}{3.042457in}}{\pgfqpoint{2.549117in}{3.031407in}}%
\pgfpathcurveto{\pgfqpoint{2.549117in}{3.020357in}}{\pgfqpoint{2.553508in}{3.009758in}}{\pgfqpoint{2.561321in}{3.001944in}}%
\pgfpathcurveto{\pgfqpoint{2.569135in}{2.994131in}}{\pgfqpoint{2.579734in}{2.989740in}}{\pgfqpoint{2.590784in}{2.989740in}}%
\pgfpathclose%
\pgfusepath{stroke,fill}%
\end{pgfscope}%
\begin{pgfscope}%
\pgfpathrectangle{\pgfqpoint{0.600000in}{0.600000in}}{\pgfqpoint{3.900000in}{3.900000in}}%
\pgfusepath{clip}%
\pgfsetbuttcap%
\pgfsetroundjoin%
\definecolor{currentfill}{rgb}{0.121569,0.466667,0.705882}%
\pgfsetfillcolor{currentfill}%
\pgfsetlinewidth{1.003750pt}%
\definecolor{currentstroke}{rgb}{0.121569,0.466667,0.705882}%
\pgfsetstrokecolor{currentstroke}%
\pgfsetdash{}{0pt}%
\pgfpathmoveto{\pgfqpoint{2.653379in}{2.794848in}}%
\pgfpathcurveto{\pgfqpoint{2.664429in}{2.794848in}}{\pgfqpoint{2.675028in}{2.799238in}}{\pgfqpoint{2.682842in}{2.807052in}}%
\pgfpathcurveto{\pgfqpoint{2.690655in}{2.814866in}}{\pgfqpoint{2.695045in}{2.825465in}}{\pgfqpoint{2.695045in}{2.836515in}}%
\pgfpathcurveto{\pgfqpoint{2.695045in}{2.847565in}}{\pgfqpoint{2.690655in}{2.858164in}}{\pgfqpoint{2.682842in}{2.865978in}}%
\pgfpathcurveto{\pgfqpoint{2.675028in}{2.873791in}}{\pgfqpoint{2.664429in}{2.878182in}}{\pgfqpoint{2.653379in}{2.878182in}}%
\pgfpathcurveto{\pgfqpoint{2.642329in}{2.878182in}}{\pgfqpoint{2.631730in}{2.873791in}}{\pgfqpoint{2.623916in}{2.865978in}}%
\pgfpathcurveto{\pgfqpoint{2.616102in}{2.858164in}}{\pgfqpoint{2.611712in}{2.847565in}}{\pgfqpoint{2.611712in}{2.836515in}}%
\pgfpathcurveto{\pgfqpoint{2.611712in}{2.825465in}}{\pgfqpoint{2.616102in}{2.814866in}}{\pgfqpoint{2.623916in}{2.807052in}}%
\pgfpathcurveto{\pgfqpoint{2.631730in}{2.799238in}}{\pgfqpoint{2.642329in}{2.794848in}}{\pgfqpoint{2.653379in}{2.794848in}}%
\pgfpathclose%
\pgfusepath{stroke,fill}%
\end{pgfscope}%
\begin{pgfscope}%
\pgfpathrectangle{\pgfqpoint{0.600000in}{0.600000in}}{\pgfqpoint{3.900000in}{3.900000in}}%
\pgfusepath{clip}%
\pgfsetbuttcap%
\pgfsetroundjoin%
\definecolor{currentfill}{rgb}{0.121569,0.466667,0.705882}%
\pgfsetfillcolor{currentfill}%
\pgfsetlinewidth{1.003750pt}%
\definecolor{currentstroke}{rgb}{0.121569,0.466667,0.705882}%
\pgfsetstrokecolor{currentstroke}%
\pgfsetdash{}{0pt}%
\pgfpathmoveto{\pgfqpoint{2.895494in}{3.149380in}}%
\pgfpathcurveto{\pgfqpoint{2.906544in}{3.149380in}}{\pgfqpoint{2.917143in}{3.153771in}}{\pgfqpoint{2.924957in}{3.161584in}}%
\pgfpathcurveto{\pgfqpoint{2.932771in}{3.169398in}}{\pgfqpoint{2.937161in}{3.179997in}}{\pgfqpoint{2.937161in}{3.191047in}}%
\pgfpathcurveto{\pgfqpoint{2.937161in}{3.202097in}}{\pgfqpoint{2.932771in}{3.212696in}}{\pgfqpoint{2.924957in}{3.220510in}}%
\pgfpathcurveto{\pgfqpoint{2.917143in}{3.228324in}}{\pgfqpoint{2.906544in}{3.232714in}}{\pgfqpoint{2.895494in}{3.232714in}}%
\pgfpathcurveto{\pgfqpoint{2.884444in}{3.232714in}}{\pgfqpoint{2.873845in}{3.228324in}}{\pgfqpoint{2.866031in}{3.220510in}}%
\pgfpathcurveto{\pgfqpoint{2.858218in}{3.212696in}}{\pgfqpoint{2.853828in}{3.202097in}}{\pgfqpoint{2.853828in}{3.191047in}}%
\pgfpathcurveto{\pgfqpoint{2.853828in}{3.179997in}}{\pgfqpoint{2.858218in}{3.169398in}}{\pgfqpoint{2.866031in}{3.161584in}}%
\pgfpathcurveto{\pgfqpoint{2.873845in}{3.153771in}}{\pgfqpoint{2.884444in}{3.149380in}}{\pgfqpoint{2.895494in}{3.149380in}}%
\pgfpathclose%
\pgfusepath{stroke,fill}%
\end{pgfscope}%
\begin{pgfscope}%
\pgfpathrectangle{\pgfqpoint{0.600000in}{0.600000in}}{\pgfqpoint{3.900000in}{3.900000in}}%
\pgfusepath{clip}%
\pgfsetbuttcap%
\pgfsetroundjoin%
\definecolor{currentfill}{rgb}{0.121569,0.466667,0.705882}%
\pgfsetfillcolor{currentfill}%
\pgfsetlinewidth{1.003750pt}%
\definecolor{currentstroke}{rgb}{0.121569,0.466667,0.705882}%
\pgfsetstrokecolor{currentstroke}%
\pgfsetdash{}{0pt}%
\pgfpathmoveto{\pgfqpoint{2.920640in}{3.076874in}}%
\pgfpathcurveto{\pgfqpoint{2.931690in}{3.076874in}}{\pgfqpoint{2.942289in}{3.081265in}}{\pgfqpoint{2.950103in}{3.089078in}}%
\pgfpathcurveto{\pgfqpoint{2.957917in}{3.096892in}}{\pgfqpoint{2.962307in}{3.107491in}}{\pgfqpoint{2.962307in}{3.118541in}}%
\pgfpathcurveto{\pgfqpoint{2.962307in}{3.129591in}}{\pgfqpoint{2.957917in}{3.140190in}}{\pgfqpoint{2.950103in}{3.148004in}}%
\pgfpathcurveto{\pgfqpoint{2.942289in}{3.155817in}}{\pgfqpoint{2.931690in}{3.160208in}}{\pgfqpoint{2.920640in}{3.160208in}}%
\pgfpathcurveto{\pgfqpoint{2.909590in}{3.160208in}}{\pgfqpoint{2.898991in}{3.155817in}}{\pgfqpoint{2.891177in}{3.148004in}}%
\pgfpathcurveto{\pgfqpoint{2.883364in}{3.140190in}}{\pgfqpoint{2.878974in}{3.129591in}}{\pgfqpoint{2.878974in}{3.118541in}}%
\pgfpathcurveto{\pgfqpoint{2.878974in}{3.107491in}}{\pgfqpoint{2.883364in}{3.096892in}}{\pgfqpoint{2.891177in}{3.089078in}}%
\pgfpathcurveto{\pgfqpoint{2.898991in}{3.081265in}}{\pgfqpoint{2.909590in}{3.076874in}}{\pgfqpoint{2.920640in}{3.076874in}}%
\pgfpathclose%
\pgfusepath{stroke,fill}%
\end{pgfscope}%
\begin{pgfscope}%
\pgfpathrectangle{\pgfqpoint{0.600000in}{0.600000in}}{\pgfqpoint{3.900000in}{3.900000in}}%
\pgfusepath{clip}%
\pgfsetbuttcap%
\pgfsetroundjoin%
\definecolor{currentfill}{rgb}{0.121569,0.466667,0.705882}%
\pgfsetfillcolor{currentfill}%
\pgfsetlinewidth{1.003750pt}%
\definecolor{currentstroke}{rgb}{0.121569,0.466667,0.705882}%
\pgfsetstrokecolor{currentstroke}%
\pgfsetdash{}{0pt}%
\pgfpathmoveto{\pgfqpoint{2.685519in}{2.718356in}}%
\pgfpathcurveto{\pgfqpoint{2.696569in}{2.718356in}}{\pgfqpoint{2.707168in}{2.722746in}}{\pgfqpoint{2.714982in}{2.730560in}}%
\pgfpathcurveto{\pgfqpoint{2.722795in}{2.738373in}}{\pgfqpoint{2.727186in}{2.748972in}}{\pgfqpoint{2.727186in}{2.760023in}}%
\pgfpathcurveto{\pgfqpoint{2.727186in}{2.771073in}}{\pgfqpoint{2.722795in}{2.781672in}}{\pgfqpoint{2.714982in}{2.789485in}}%
\pgfpathcurveto{\pgfqpoint{2.707168in}{2.797299in}}{\pgfqpoint{2.696569in}{2.801689in}}{\pgfqpoint{2.685519in}{2.801689in}}%
\pgfpathcurveto{\pgfqpoint{2.674469in}{2.801689in}}{\pgfqpoint{2.663870in}{2.797299in}}{\pgfqpoint{2.656056in}{2.789485in}}%
\pgfpathcurveto{\pgfqpoint{2.648242in}{2.781672in}}{\pgfqpoint{2.643852in}{2.771073in}}{\pgfqpoint{2.643852in}{2.760023in}}%
\pgfpathcurveto{\pgfqpoint{2.643852in}{2.748972in}}{\pgfqpoint{2.648242in}{2.738373in}}{\pgfqpoint{2.656056in}{2.730560in}}%
\pgfpathcurveto{\pgfqpoint{2.663870in}{2.722746in}}{\pgfqpoint{2.674469in}{2.718356in}}{\pgfqpoint{2.685519in}{2.718356in}}%
\pgfpathclose%
\pgfusepath{stroke,fill}%
\end{pgfscope}%
\begin{pgfscope}%
\pgfpathrectangle{\pgfqpoint{0.600000in}{0.600000in}}{\pgfqpoint{3.900000in}{3.900000in}}%
\pgfusepath{clip}%
\pgfsetbuttcap%
\pgfsetroundjoin%
\definecolor{currentfill}{rgb}{0.121569,0.466667,0.705882}%
\pgfsetfillcolor{currentfill}%
\pgfsetlinewidth{1.003750pt}%
\definecolor{currentstroke}{rgb}{0.121569,0.466667,0.705882}%
\pgfsetstrokecolor{currentstroke}%
\pgfsetdash{}{0pt}%
\pgfpathmoveto{\pgfqpoint{1.918873in}{2.813843in}}%
\pgfpathcurveto{\pgfqpoint{1.929923in}{2.813843in}}{\pgfqpoint{1.940522in}{2.818234in}}{\pgfqpoint{1.948336in}{2.826047in}}%
\pgfpathcurveto{\pgfqpoint{1.956149in}{2.833861in}}{\pgfqpoint{1.960540in}{2.844460in}}{\pgfqpoint{1.960540in}{2.855510in}}%
\pgfpathcurveto{\pgfqpoint{1.960540in}{2.866560in}}{\pgfqpoint{1.956149in}{2.877159in}}{\pgfqpoint{1.948336in}{2.884973in}}%
\pgfpathcurveto{\pgfqpoint{1.940522in}{2.892787in}}{\pgfqpoint{1.929923in}{2.897177in}}{\pgfqpoint{1.918873in}{2.897177in}}%
\pgfpathcurveto{\pgfqpoint{1.907823in}{2.897177in}}{\pgfqpoint{1.897224in}{2.892787in}}{\pgfqpoint{1.889410in}{2.884973in}}%
\pgfpathcurveto{\pgfqpoint{1.881596in}{2.877159in}}{\pgfqpoint{1.877206in}{2.866560in}}{\pgfqpoint{1.877206in}{2.855510in}}%
\pgfpathcurveto{\pgfqpoint{1.877206in}{2.844460in}}{\pgfqpoint{1.881596in}{2.833861in}}{\pgfqpoint{1.889410in}{2.826047in}}%
\pgfpathcurveto{\pgfqpoint{1.897224in}{2.818234in}}{\pgfqpoint{1.907823in}{2.813843in}}{\pgfqpoint{1.918873in}{2.813843in}}%
\pgfpathclose%
\pgfusepath{stroke,fill}%
\end{pgfscope}%
\begin{pgfscope}%
\pgfpathrectangle{\pgfqpoint{0.600000in}{0.600000in}}{\pgfqpoint{3.900000in}{3.900000in}}%
\pgfusepath{clip}%
\pgfsetbuttcap%
\pgfsetroundjoin%
\definecolor{currentfill}{rgb}{0.121569,0.466667,0.705882}%
\pgfsetfillcolor{currentfill}%
\pgfsetlinewidth{1.003750pt}%
\definecolor{currentstroke}{rgb}{0.121569,0.466667,0.705882}%
\pgfsetstrokecolor{currentstroke}%
\pgfsetdash{}{0pt}%
\pgfpathmoveto{\pgfqpoint{3.238621in}{3.375004in}}%
\pgfpathcurveto{\pgfqpoint{3.249671in}{3.375004in}}{\pgfqpoint{3.260270in}{3.379394in}}{\pgfqpoint{3.268084in}{3.387208in}}%
\pgfpathcurveto{\pgfqpoint{3.275898in}{3.395022in}}{\pgfqpoint{3.280288in}{3.405621in}}{\pgfqpoint{3.280288in}{3.416671in}}%
\pgfpathcurveto{\pgfqpoint{3.280288in}{3.427721in}}{\pgfqpoint{3.275898in}{3.438320in}}{\pgfqpoint{3.268084in}{3.446134in}}%
\pgfpathcurveto{\pgfqpoint{3.260270in}{3.453947in}}{\pgfqpoint{3.249671in}{3.458337in}}{\pgfqpoint{3.238621in}{3.458337in}}%
\pgfpathcurveto{\pgfqpoint{3.227571in}{3.458337in}}{\pgfqpoint{3.216972in}{3.453947in}}{\pgfqpoint{3.209158in}{3.446134in}}%
\pgfpathcurveto{\pgfqpoint{3.201345in}{3.438320in}}{\pgfqpoint{3.196955in}{3.427721in}}{\pgfqpoint{3.196955in}{3.416671in}}%
\pgfpathcurveto{\pgfqpoint{3.196955in}{3.405621in}}{\pgfqpoint{3.201345in}{3.395022in}}{\pgfqpoint{3.209158in}{3.387208in}}%
\pgfpathcurveto{\pgfqpoint{3.216972in}{3.379394in}}{\pgfqpoint{3.227571in}{3.375004in}}{\pgfqpoint{3.238621in}{3.375004in}}%
\pgfpathclose%
\pgfusepath{stroke,fill}%
\end{pgfscope}%
\begin{pgfscope}%
\pgfpathrectangle{\pgfqpoint{0.600000in}{0.600000in}}{\pgfqpoint{3.900000in}{3.900000in}}%
\pgfusepath{clip}%
\pgfsetbuttcap%
\pgfsetroundjoin%
\definecolor{currentfill}{rgb}{0.121569,0.466667,0.705882}%
\pgfsetfillcolor{currentfill}%
\pgfsetlinewidth{1.003750pt}%
\definecolor{currentstroke}{rgb}{0.121569,0.466667,0.705882}%
\pgfsetstrokecolor{currentstroke}%
\pgfsetdash{}{0pt}%
\pgfpathmoveto{\pgfqpoint{2.449897in}{2.189706in}}%
\pgfpathcurveto{\pgfqpoint{2.460947in}{2.189706in}}{\pgfqpoint{2.471546in}{2.194096in}}{\pgfqpoint{2.479360in}{2.201910in}}%
\pgfpathcurveto{\pgfqpoint{2.487173in}{2.209723in}}{\pgfqpoint{2.491564in}{2.220322in}}{\pgfqpoint{2.491564in}{2.231372in}}%
\pgfpathcurveto{\pgfqpoint{2.491564in}{2.242422in}}{\pgfqpoint{2.487173in}{2.253021in}}{\pgfqpoint{2.479360in}{2.260835in}}%
\pgfpathcurveto{\pgfqpoint{2.471546in}{2.268649in}}{\pgfqpoint{2.460947in}{2.273039in}}{\pgfqpoint{2.449897in}{2.273039in}}%
\pgfpathcurveto{\pgfqpoint{2.438847in}{2.273039in}}{\pgfqpoint{2.428248in}{2.268649in}}{\pgfqpoint{2.420434in}{2.260835in}}%
\pgfpathcurveto{\pgfqpoint{2.412621in}{2.253021in}}{\pgfqpoint{2.408230in}{2.242422in}}{\pgfqpoint{2.408230in}{2.231372in}}%
\pgfpathcurveto{\pgfqpoint{2.408230in}{2.220322in}}{\pgfqpoint{2.412621in}{2.209723in}}{\pgfqpoint{2.420434in}{2.201910in}}%
\pgfpathcurveto{\pgfqpoint{2.428248in}{2.194096in}}{\pgfqpoint{2.438847in}{2.189706in}}{\pgfqpoint{2.449897in}{2.189706in}}%
\pgfpathclose%
\pgfusepath{stroke,fill}%
\end{pgfscope}%
\begin{pgfscope}%
\pgfpathrectangle{\pgfqpoint{0.600000in}{0.600000in}}{\pgfqpoint{3.900000in}{3.900000in}}%
\pgfusepath{clip}%
\pgfsetbuttcap%
\pgfsetroundjoin%
\definecolor{currentfill}{rgb}{0.121569,0.466667,0.705882}%
\pgfsetfillcolor{currentfill}%
\pgfsetlinewidth{1.003750pt}%
\definecolor{currentstroke}{rgb}{0.121569,0.466667,0.705882}%
\pgfsetstrokecolor{currentstroke}%
\pgfsetdash{}{0pt}%
\pgfpathmoveto{\pgfqpoint{1.876498in}{2.130377in}}%
\pgfpathcurveto{\pgfqpoint{1.887548in}{2.130377in}}{\pgfqpoint{1.898147in}{2.134767in}}{\pgfqpoint{1.905960in}{2.142580in}}%
\pgfpathcurveto{\pgfqpoint{1.913774in}{2.150394in}}{\pgfqpoint{1.918164in}{2.160993in}}{\pgfqpoint{1.918164in}{2.172043in}}%
\pgfpathcurveto{\pgfqpoint{1.918164in}{2.183093in}}{\pgfqpoint{1.913774in}{2.193692in}}{\pgfqpoint{1.905960in}{2.201506in}}%
\pgfpathcurveto{\pgfqpoint{1.898147in}{2.209320in}}{\pgfqpoint{1.887548in}{2.213710in}}{\pgfqpoint{1.876498in}{2.213710in}}%
\pgfpathcurveto{\pgfqpoint{1.865447in}{2.213710in}}{\pgfqpoint{1.854848in}{2.209320in}}{\pgfqpoint{1.847035in}{2.201506in}}%
\pgfpathcurveto{\pgfqpoint{1.839221in}{2.193692in}}{\pgfqpoint{1.834831in}{2.183093in}}{\pgfqpoint{1.834831in}{2.172043in}}%
\pgfpathcurveto{\pgfqpoint{1.834831in}{2.160993in}}{\pgfqpoint{1.839221in}{2.150394in}}{\pgfqpoint{1.847035in}{2.142580in}}%
\pgfpathcurveto{\pgfqpoint{1.854848in}{2.134767in}}{\pgfqpoint{1.865447in}{2.130377in}}{\pgfqpoint{1.876498in}{2.130377in}}%
\pgfpathclose%
\pgfusepath{stroke,fill}%
\end{pgfscope}%
\begin{pgfscope}%
\pgfpathrectangle{\pgfqpoint{0.600000in}{0.600000in}}{\pgfqpoint{3.900000in}{3.900000in}}%
\pgfusepath{clip}%
\pgfsetbuttcap%
\pgfsetroundjoin%
\definecolor{currentfill}{rgb}{0.121569,0.466667,0.705882}%
\pgfsetfillcolor{currentfill}%
\pgfsetlinewidth{1.003750pt}%
\definecolor{currentstroke}{rgb}{0.121569,0.466667,0.705882}%
\pgfsetstrokecolor{currentstroke}%
\pgfsetdash{}{0pt}%
\pgfpathmoveto{\pgfqpoint{0.705857in}{2.925807in}}%
\pgfpathcurveto{\pgfqpoint{0.716907in}{2.925807in}}{\pgfqpoint{0.727506in}{2.930197in}}{\pgfqpoint{0.735319in}{2.938011in}}%
\pgfpathcurveto{\pgfqpoint{0.743133in}{2.945824in}}{\pgfqpoint{0.747523in}{2.956423in}}{\pgfqpoint{0.747523in}{2.967473in}}%
\pgfpathcurveto{\pgfqpoint{0.747523in}{2.978523in}}{\pgfqpoint{0.743133in}{2.989123in}}{\pgfqpoint{0.735319in}{2.996936in}}%
\pgfpathcurveto{\pgfqpoint{0.727506in}{3.004750in}}{\pgfqpoint{0.716907in}{3.009140in}}{\pgfqpoint{0.705857in}{3.009140in}}%
\pgfpathcurveto{\pgfqpoint{0.694807in}{3.009140in}}{\pgfqpoint{0.684208in}{3.004750in}}{\pgfqpoint{0.676394in}{2.996936in}}%
\pgfpathcurveto{\pgfqpoint{0.668580in}{2.989123in}}{\pgfqpoint{0.664190in}{2.978523in}}{\pgfqpoint{0.664190in}{2.967473in}}%
\pgfpathcurveto{\pgfqpoint{0.664190in}{2.956423in}}{\pgfqpoint{0.668580in}{2.945824in}}{\pgfqpoint{0.676394in}{2.938011in}}%
\pgfpathcurveto{\pgfqpoint{0.684208in}{2.930197in}}{\pgfqpoint{0.694807in}{2.925807in}}{\pgfqpoint{0.705857in}{2.925807in}}%
\pgfpathclose%
\pgfusepath{stroke,fill}%
\end{pgfscope}%
\begin{pgfscope}%
\pgfpathrectangle{\pgfqpoint{0.600000in}{0.600000in}}{\pgfqpoint{3.900000in}{3.900000in}}%
\pgfusepath{clip}%
\pgfsetbuttcap%
\pgfsetroundjoin%
\definecolor{currentfill}{rgb}{0.121569,0.466667,0.705882}%
\pgfsetfillcolor{currentfill}%
\pgfsetlinewidth{1.003750pt}%
\definecolor{currentstroke}{rgb}{0.121569,0.466667,0.705882}%
\pgfsetstrokecolor{currentstroke}%
\pgfsetdash{}{0pt}%
\pgfpathmoveto{\pgfqpoint{2.143436in}{2.002690in}}%
\pgfpathcurveto{\pgfqpoint{2.154487in}{2.002690in}}{\pgfqpoint{2.165086in}{2.007080in}}{\pgfqpoint{2.172899in}{2.014893in}}%
\pgfpathcurveto{\pgfqpoint{2.180713in}{2.022707in}}{\pgfqpoint{2.185103in}{2.033306in}}{\pgfqpoint{2.185103in}{2.044356in}}%
\pgfpathcurveto{\pgfqpoint{2.185103in}{2.055406in}}{\pgfqpoint{2.180713in}{2.066005in}}{\pgfqpoint{2.172899in}{2.073819in}}%
\pgfpathcurveto{\pgfqpoint{2.165086in}{2.081633in}}{\pgfqpoint{2.154487in}{2.086023in}}{\pgfqpoint{2.143436in}{2.086023in}}%
\pgfpathcurveto{\pgfqpoint{2.132386in}{2.086023in}}{\pgfqpoint{2.121787in}{2.081633in}}{\pgfqpoint{2.113974in}{2.073819in}}%
\pgfpathcurveto{\pgfqpoint{2.106160in}{2.066005in}}{\pgfqpoint{2.101770in}{2.055406in}}{\pgfqpoint{2.101770in}{2.044356in}}%
\pgfpathcurveto{\pgfqpoint{2.101770in}{2.033306in}}{\pgfqpoint{2.106160in}{2.022707in}}{\pgfqpoint{2.113974in}{2.014893in}}%
\pgfpathcurveto{\pgfqpoint{2.121787in}{2.007080in}}{\pgfqpoint{2.132386in}{2.002690in}}{\pgfqpoint{2.143436in}{2.002690in}}%
\pgfpathclose%
\pgfusepath{stroke,fill}%
\end{pgfscope}%
\begin{pgfscope}%
\pgfpathrectangle{\pgfqpoint{0.600000in}{0.600000in}}{\pgfqpoint{3.900000in}{3.900000in}}%
\pgfusepath{clip}%
\pgfsetbuttcap%
\pgfsetroundjoin%
\definecolor{currentfill}{rgb}{0.121569,0.466667,0.705882}%
\pgfsetfillcolor{currentfill}%
\pgfsetlinewidth{1.003750pt}%
\definecolor{currentstroke}{rgb}{0.121569,0.466667,0.705882}%
\pgfsetstrokecolor{currentstroke}%
\pgfsetdash{}{0pt}%
\pgfpathmoveto{\pgfqpoint{2.701206in}{2.878420in}}%
\pgfpathcurveto{\pgfqpoint{2.712256in}{2.878420in}}{\pgfqpoint{2.722855in}{2.882810in}}{\pgfqpoint{2.730669in}{2.890624in}}%
\pgfpathcurveto{\pgfqpoint{2.738483in}{2.898437in}}{\pgfqpoint{2.742873in}{2.909036in}}{\pgfqpoint{2.742873in}{2.920086in}}%
\pgfpathcurveto{\pgfqpoint{2.742873in}{2.931137in}}{\pgfqpoint{2.738483in}{2.941736in}}{\pgfqpoint{2.730669in}{2.949549in}}%
\pgfpathcurveto{\pgfqpoint{2.722855in}{2.957363in}}{\pgfqpoint{2.712256in}{2.961753in}}{\pgfqpoint{2.701206in}{2.961753in}}%
\pgfpathcurveto{\pgfqpoint{2.690156in}{2.961753in}}{\pgfqpoint{2.679557in}{2.957363in}}{\pgfqpoint{2.671743in}{2.949549in}}%
\pgfpathcurveto{\pgfqpoint{2.663930in}{2.941736in}}{\pgfqpoint{2.659539in}{2.931137in}}{\pgfqpoint{2.659539in}{2.920086in}}%
\pgfpathcurveto{\pgfqpoint{2.659539in}{2.909036in}}{\pgfqpoint{2.663930in}{2.898437in}}{\pgfqpoint{2.671743in}{2.890624in}}%
\pgfpathcurveto{\pgfqpoint{2.679557in}{2.882810in}}{\pgfqpoint{2.690156in}{2.878420in}}{\pgfqpoint{2.701206in}{2.878420in}}%
\pgfpathclose%
\pgfusepath{stroke,fill}%
\end{pgfscope}%
\begin{pgfscope}%
\pgfpathrectangle{\pgfqpoint{0.600000in}{0.600000in}}{\pgfqpoint{3.900000in}{3.900000in}}%
\pgfusepath{clip}%
\pgfsetbuttcap%
\pgfsetroundjoin%
\definecolor{currentfill}{rgb}{0.121569,0.466667,0.705882}%
\pgfsetfillcolor{currentfill}%
\pgfsetlinewidth{1.003750pt}%
\definecolor{currentstroke}{rgb}{0.121569,0.466667,0.705882}%
\pgfsetstrokecolor{currentstroke}%
\pgfsetdash{}{0pt}%
\pgfpathmoveto{\pgfqpoint{2.606368in}{2.358144in}}%
\pgfpathcurveto{\pgfqpoint{2.617418in}{2.358144in}}{\pgfqpoint{2.628017in}{2.362535in}}{\pgfqpoint{2.635831in}{2.370348in}}%
\pgfpathcurveto{\pgfqpoint{2.643644in}{2.378162in}}{\pgfqpoint{2.648035in}{2.388761in}}{\pgfqpoint{2.648035in}{2.399811in}}%
\pgfpathcurveto{\pgfqpoint{2.648035in}{2.410861in}}{\pgfqpoint{2.643644in}{2.421460in}}{\pgfqpoint{2.635831in}{2.429274in}}%
\pgfpathcurveto{\pgfqpoint{2.628017in}{2.437088in}}{\pgfqpoint{2.617418in}{2.441478in}}{\pgfqpoint{2.606368in}{2.441478in}}%
\pgfpathcurveto{\pgfqpoint{2.595318in}{2.441478in}}{\pgfqpoint{2.584719in}{2.437088in}}{\pgfqpoint{2.576905in}{2.429274in}}%
\pgfpathcurveto{\pgfqpoint{2.569092in}{2.421460in}}{\pgfqpoint{2.564701in}{2.410861in}}{\pgfqpoint{2.564701in}{2.399811in}}%
\pgfpathcurveto{\pgfqpoint{2.564701in}{2.388761in}}{\pgfqpoint{2.569092in}{2.378162in}}{\pgfqpoint{2.576905in}{2.370348in}}%
\pgfpathcurveto{\pgfqpoint{2.584719in}{2.362535in}}{\pgfqpoint{2.595318in}{2.358144in}}{\pgfqpoint{2.606368in}{2.358144in}}%
\pgfpathclose%
\pgfusepath{stroke,fill}%
\end{pgfscope}%
\begin{pgfscope}%
\pgfpathrectangle{\pgfqpoint{0.600000in}{0.600000in}}{\pgfqpoint{3.900000in}{3.900000in}}%
\pgfusepath{clip}%
\pgfsetbuttcap%
\pgfsetroundjoin%
\definecolor{currentfill}{rgb}{0.121569,0.466667,0.705882}%
\pgfsetfillcolor{currentfill}%
\pgfsetlinewidth{1.003750pt}%
\definecolor{currentstroke}{rgb}{0.121569,0.466667,0.705882}%
\pgfsetstrokecolor{currentstroke}%
\pgfsetdash{}{0pt}%
\pgfpathmoveto{\pgfqpoint{3.160923in}{2.638668in}}%
\pgfpathcurveto{\pgfqpoint{3.171973in}{2.638668in}}{\pgfqpoint{3.182572in}{2.643059in}}{\pgfqpoint{3.190385in}{2.650872in}}%
\pgfpathcurveto{\pgfqpoint{3.198199in}{2.658686in}}{\pgfqpoint{3.202589in}{2.669285in}}{\pgfqpoint{3.202589in}{2.680335in}}%
\pgfpathcurveto{\pgfqpoint{3.202589in}{2.691385in}}{\pgfqpoint{3.198199in}{2.701984in}}{\pgfqpoint{3.190385in}{2.709798in}}%
\pgfpathcurveto{\pgfqpoint{3.182572in}{2.717612in}}{\pgfqpoint{3.171973in}{2.722002in}}{\pgfqpoint{3.160923in}{2.722002in}}%
\pgfpathcurveto{\pgfqpoint{3.149873in}{2.722002in}}{\pgfqpoint{3.139274in}{2.717612in}}{\pgfqpoint{3.131460in}{2.709798in}}%
\pgfpathcurveto{\pgfqpoint{3.123646in}{2.701984in}}{\pgfqpoint{3.119256in}{2.691385in}}{\pgfqpoint{3.119256in}{2.680335in}}%
\pgfpathcurveto{\pgfqpoint{3.119256in}{2.669285in}}{\pgfqpoint{3.123646in}{2.658686in}}{\pgfqpoint{3.131460in}{2.650872in}}%
\pgfpathcurveto{\pgfqpoint{3.139274in}{2.643059in}}{\pgfqpoint{3.149873in}{2.638668in}}{\pgfqpoint{3.160923in}{2.638668in}}%
\pgfpathclose%
\pgfusepath{stroke,fill}%
\end{pgfscope}%
\begin{pgfscope}%
\pgfpathrectangle{\pgfqpoint{0.600000in}{0.600000in}}{\pgfqpoint{3.900000in}{3.900000in}}%
\pgfusepath{clip}%
\pgfsetbuttcap%
\pgfsetroundjoin%
\definecolor{currentfill}{rgb}{0.121569,0.466667,0.705882}%
\pgfsetfillcolor{currentfill}%
\pgfsetlinewidth{1.003750pt}%
\definecolor{currentstroke}{rgb}{0.121569,0.466667,0.705882}%
\pgfsetstrokecolor{currentstroke}%
\pgfsetdash{}{0pt}%
\pgfpathmoveto{\pgfqpoint{2.022017in}{3.859460in}}%
\pgfpathcurveto{\pgfqpoint{2.033067in}{3.859460in}}{\pgfqpoint{2.043666in}{3.863851in}}{\pgfqpoint{2.051479in}{3.871664in}}%
\pgfpathcurveto{\pgfqpoint{2.059293in}{3.879478in}}{\pgfqpoint{2.063683in}{3.890077in}}{\pgfqpoint{2.063683in}{3.901127in}}%
\pgfpathcurveto{\pgfqpoint{2.063683in}{3.912177in}}{\pgfqpoint{2.059293in}{3.922776in}}{\pgfqpoint{2.051479in}{3.930590in}}%
\pgfpathcurveto{\pgfqpoint{2.043666in}{3.938404in}}{\pgfqpoint{2.033067in}{3.942794in}}{\pgfqpoint{2.022017in}{3.942794in}}%
\pgfpathcurveto{\pgfqpoint{2.010967in}{3.942794in}}{\pgfqpoint{2.000367in}{3.938404in}}{\pgfqpoint{1.992554in}{3.930590in}}%
\pgfpathcurveto{\pgfqpoint{1.984740in}{3.922776in}}{\pgfqpoint{1.980350in}{3.912177in}}{\pgfqpoint{1.980350in}{3.901127in}}%
\pgfpathcurveto{\pgfqpoint{1.980350in}{3.890077in}}{\pgfqpoint{1.984740in}{3.879478in}}{\pgfqpoint{1.992554in}{3.871664in}}%
\pgfpathcurveto{\pgfqpoint{2.000367in}{3.863851in}}{\pgfqpoint{2.010967in}{3.859460in}}{\pgfqpoint{2.022017in}{3.859460in}}%
\pgfpathclose%
\pgfusepath{stroke,fill}%
\end{pgfscope}%
\begin{pgfscope}%
\pgfpathrectangle{\pgfqpoint{0.600000in}{0.600000in}}{\pgfqpoint{3.900000in}{3.900000in}}%
\pgfusepath{clip}%
\pgfsetbuttcap%
\pgfsetroundjoin%
\definecolor{currentfill}{rgb}{0.121569,0.466667,0.705882}%
\pgfsetfillcolor{currentfill}%
\pgfsetlinewidth{1.003750pt}%
\definecolor{currentstroke}{rgb}{0.121569,0.466667,0.705882}%
\pgfsetstrokecolor{currentstroke}%
\pgfsetdash{}{0pt}%
\pgfpathmoveto{\pgfqpoint{1.968342in}{2.701762in}}%
\pgfpathcurveto{\pgfqpoint{1.979392in}{2.701762in}}{\pgfqpoint{1.989991in}{2.706152in}}{\pgfqpoint{1.997805in}{2.713965in}}%
\pgfpathcurveto{\pgfqpoint{2.005618in}{2.721779in}}{\pgfqpoint{2.010009in}{2.732378in}}{\pgfqpoint{2.010009in}{2.743428in}}%
\pgfpathcurveto{\pgfqpoint{2.010009in}{2.754478in}}{\pgfqpoint{2.005618in}{2.765077in}}{\pgfqpoint{1.997805in}{2.772891in}}%
\pgfpathcurveto{\pgfqpoint{1.989991in}{2.780705in}}{\pgfqpoint{1.979392in}{2.785095in}}{\pgfqpoint{1.968342in}{2.785095in}}%
\pgfpathcurveto{\pgfqpoint{1.957292in}{2.785095in}}{\pgfqpoint{1.946693in}{2.780705in}}{\pgfqpoint{1.938879in}{2.772891in}}%
\pgfpathcurveto{\pgfqpoint{1.931065in}{2.765077in}}{\pgfqpoint{1.926675in}{2.754478in}}{\pgfqpoint{1.926675in}{2.743428in}}%
\pgfpathcurveto{\pgfqpoint{1.926675in}{2.732378in}}{\pgfqpoint{1.931065in}{2.721779in}}{\pgfqpoint{1.938879in}{2.713965in}}%
\pgfpathcurveto{\pgfqpoint{1.946693in}{2.706152in}}{\pgfqpoint{1.957292in}{2.701762in}}{\pgfqpoint{1.968342in}{2.701762in}}%
\pgfpathclose%
\pgfusepath{stroke,fill}%
\end{pgfscope}%
\begin{pgfscope}%
\pgfpathrectangle{\pgfqpoint{0.600000in}{0.600000in}}{\pgfqpoint{3.900000in}{3.900000in}}%
\pgfusepath{clip}%
\pgfsetbuttcap%
\pgfsetroundjoin%
\definecolor{currentfill}{rgb}{0.121569,0.466667,0.705882}%
\pgfsetfillcolor{currentfill}%
\pgfsetlinewidth{1.003750pt}%
\definecolor{currentstroke}{rgb}{0.121569,0.466667,0.705882}%
\pgfsetstrokecolor{currentstroke}%
\pgfsetdash{}{0pt}%
\pgfpathmoveto{\pgfqpoint{3.070033in}{1.799848in}}%
\pgfpathcurveto{\pgfqpoint{3.081083in}{1.799848in}}{\pgfqpoint{3.091682in}{1.804238in}}{\pgfqpoint{3.099495in}{1.812052in}}%
\pgfpathcurveto{\pgfqpoint{3.107309in}{1.819865in}}{\pgfqpoint{3.111699in}{1.830464in}}{\pgfqpoint{3.111699in}{1.841515in}}%
\pgfpathcurveto{\pgfqpoint{3.111699in}{1.852565in}}{\pgfqpoint{3.107309in}{1.863164in}}{\pgfqpoint{3.099495in}{1.870977in}}%
\pgfpathcurveto{\pgfqpoint{3.091682in}{1.878791in}}{\pgfqpoint{3.081083in}{1.883181in}}{\pgfqpoint{3.070033in}{1.883181in}}%
\pgfpathcurveto{\pgfqpoint{3.058982in}{1.883181in}}{\pgfqpoint{3.048383in}{1.878791in}}{\pgfqpoint{3.040570in}{1.870977in}}%
\pgfpathcurveto{\pgfqpoint{3.032756in}{1.863164in}}{\pgfqpoint{3.028366in}{1.852565in}}{\pgfqpoint{3.028366in}{1.841515in}}%
\pgfpathcurveto{\pgfqpoint{3.028366in}{1.830464in}}{\pgfqpoint{3.032756in}{1.819865in}}{\pgfqpoint{3.040570in}{1.812052in}}%
\pgfpathcurveto{\pgfqpoint{3.048383in}{1.804238in}}{\pgfqpoint{3.058982in}{1.799848in}}{\pgfqpoint{3.070033in}{1.799848in}}%
\pgfpathclose%
\pgfusepath{stroke,fill}%
\end{pgfscope}%
\begin{pgfscope}%
\pgfpathrectangle{\pgfqpoint{0.600000in}{0.600000in}}{\pgfqpoint{3.900000in}{3.900000in}}%
\pgfusepath{clip}%
\pgfsetbuttcap%
\pgfsetroundjoin%
\definecolor{currentfill}{rgb}{0.121569,0.466667,0.705882}%
\pgfsetfillcolor{currentfill}%
\pgfsetlinewidth{1.003750pt}%
\definecolor{currentstroke}{rgb}{0.121569,0.466667,0.705882}%
\pgfsetstrokecolor{currentstroke}%
\pgfsetdash{}{0pt}%
\pgfpathmoveto{\pgfqpoint{2.647799in}{2.198174in}}%
\pgfpathcurveto{\pgfqpoint{2.658849in}{2.198174in}}{\pgfqpoint{2.669448in}{2.202564in}}{\pgfqpoint{2.677262in}{2.210378in}}%
\pgfpathcurveto{\pgfqpoint{2.685076in}{2.218192in}}{\pgfqpoint{2.689466in}{2.228791in}}{\pgfqpoint{2.689466in}{2.239841in}}%
\pgfpathcurveto{\pgfqpoint{2.689466in}{2.250891in}}{\pgfqpoint{2.685076in}{2.261490in}}{\pgfqpoint{2.677262in}{2.269304in}}%
\pgfpathcurveto{\pgfqpoint{2.669448in}{2.277117in}}{\pgfqpoint{2.658849in}{2.281508in}}{\pgfqpoint{2.647799in}{2.281508in}}%
\pgfpathcurveto{\pgfqpoint{2.636749in}{2.281508in}}{\pgfqpoint{2.626150in}{2.277117in}}{\pgfqpoint{2.618337in}{2.269304in}}%
\pgfpathcurveto{\pgfqpoint{2.610523in}{2.261490in}}{\pgfqpoint{2.606133in}{2.250891in}}{\pgfqpoint{2.606133in}{2.239841in}}%
\pgfpathcurveto{\pgfqpoint{2.606133in}{2.228791in}}{\pgfqpoint{2.610523in}{2.218192in}}{\pgfqpoint{2.618337in}{2.210378in}}%
\pgfpathcurveto{\pgfqpoint{2.626150in}{2.202564in}}{\pgfqpoint{2.636749in}{2.198174in}}{\pgfqpoint{2.647799in}{2.198174in}}%
\pgfpathclose%
\pgfusepath{stroke,fill}%
\end{pgfscope}%
\begin{pgfscope}%
\pgfpathrectangle{\pgfqpoint{0.600000in}{0.600000in}}{\pgfqpoint{3.900000in}{3.900000in}}%
\pgfusepath{clip}%
\pgfsetbuttcap%
\pgfsetroundjoin%
\definecolor{currentfill}{rgb}{0.121569,0.466667,0.705882}%
\pgfsetfillcolor{currentfill}%
\pgfsetlinewidth{1.003750pt}%
\definecolor{currentstroke}{rgb}{0.121569,0.466667,0.705882}%
\pgfsetstrokecolor{currentstroke}%
\pgfsetdash{}{0pt}%
\pgfpathmoveto{\pgfqpoint{2.276689in}{2.698601in}}%
\pgfpathcurveto{\pgfqpoint{2.287739in}{2.698601in}}{\pgfqpoint{2.298338in}{2.702992in}}{\pgfqpoint{2.306152in}{2.710805in}}%
\pgfpathcurveto{\pgfqpoint{2.313965in}{2.718619in}}{\pgfqpoint{2.318356in}{2.729218in}}{\pgfqpoint{2.318356in}{2.740268in}}%
\pgfpathcurveto{\pgfqpoint{2.318356in}{2.751318in}}{\pgfqpoint{2.313965in}{2.761917in}}{\pgfqpoint{2.306152in}{2.769731in}}%
\pgfpathcurveto{\pgfqpoint{2.298338in}{2.777544in}}{\pgfqpoint{2.287739in}{2.781935in}}{\pgfqpoint{2.276689in}{2.781935in}}%
\pgfpathcurveto{\pgfqpoint{2.265639in}{2.781935in}}{\pgfqpoint{2.255040in}{2.777544in}}{\pgfqpoint{2.247226in}{2.769731in}}%
\pgfpathcurveto{\pgfqpoint{2.239413in}{2.761917in}}{\pgfqpoint{2.235022in}{2.751318in}}{\pgfqpoint{2.235022in}{2.740268in}}%
\pgfpathcurveto{\pgfqpoint{2.235022in}{2.729218in}}{\pgfqpoint{2.239413in}{2.718619in}}{\pgfqpoint{2.247226in}{2.710805in}}%
\pgfpathcurveto{\pgfqpoint{2.255040in}{2.702992in}}{\pgfqpoint{2.265639in}{2.698601in}}{\pgfqpoint{2.276689in}{2.698601in}}%
\pgfpathclose%
\pgfusepath{stroke,fill}%
\end{pgfscope}%
\begin{pgfscope}%
\pgfpathrectangle{\pgfqpoint{0.600000in}{0.600000in}}{\pgfqpoint{3.900000in}{3.900000in}}%
\pgfusepath{clip}%
\pgfsetbuttcap%
\pgfsetroundjoin%
\definecolor{currentfill}{rgb}{0.121569,0.466667,0.705882}%
\pgfsetfillcolor{currentfill}%
\pgfsetlinewidth{1.003750pt}%
\definecolor{currentstroke}{rgb}{0.121569,0.466667,0.705882}%
\pgfsetstrokecolor{currentstroke}%
\pgfsetdash{}{0pt}%
\pgfpathmoveto{\pgfqpoint{2.464068in}{3.445712in}}%
\pgfpathcurveto{\pgfqpoint{2.475118in}{3.445712in}}{\pgfqpoint{2.485717in}{3.450102in}}{\pgfqpoint{2.493531in}{3.457916in}}%
\pgfpathcurveto{\pgfqpoint{2.501344in}{3.465729in}}{\pgfqpoint{2.505735in}{3.476328in}}{\pgfqpoint{2.505735in}{3.487379in}}%
\pgfpathcurveto{\pgfqpoint{2.505735in}{3.498429in}}{\pgfqpoint{2.501344in}{3.509028in}}{\pgfqpoint{2.493531in}{3.516841in}}%
\pgfpathcurveto{\pgfqpoint{2.485717in}{3.524655in}}{\pgfqpoint{2.475118in}{3.529045in}}{\pgfqpoint{2.464068in}{3.529045in}}%
\pgfpathcurveto{\pgfqpoint{2.453018in}{3.529045in}}{\pgfqpoint{2.442419in}{3.524655in}}{\pgfqpoint{2.434605in}{3.516841in}}%
\pgfpathcurveto{\pgfqpoint{2.426792in}{3.509028in}}{\pgfqpoint{2.422401in}{3.498429in}}{\pgfqpoint{2.422401in}{3.487379in}}%
\pgfpathcurveto{\pgfqpoint{2.422401in}{3.476328in}}{\pgfqpoint{2.426792in}{3.465729in}}{\pgfqpoint{2.434605in}{3.457916in}}%
\pgfpathcurveto{\pgfqpoint{2.442419in}{3.450102in}}{\pgfqpoint{2.453018in}{3.445712in}}{\pgfqpoint{2.464068in}{3.445712in}}%
\pgfpathclose%
\pgfusepath{stroke,fill}%
\end{pgfscope}%
\begin{pgfscope}%
\pgfpathrectangle{\pgfqpoint{0.600000in}{0.600000in}}{\pgfqpoint{3.900000in}{3.900000in}}%
\pgfusepath{clip}%
\pgfsetbuttcap%
\pgfsetroundjoin%
\definecolor{currentfill}{rgb}{0.121569,0.466667,0.705882}%
\pgfsetfillcolor{currentfill}%
\pgfsetlinewidth{1.003750pt}%
\definecolor{currentstroke}{rgb}{0.121569,0.466667,0.705882}%
\pgfsetstrokecolor{currentstroke}%
\pgfsetdash{}{0pt}%
\pgfpathmoveto{\pgfqpoint{2.119661in}{2.665676in}}%
\pgfpathcurveto{\pgfqpoint{2.130711in}{2.665676in}}{\pgfqpoint{2.141310in}{2.670066in}}{\pgfqpoint{2.149124in}{2.677879in}}%
\pgfpathcurveto{\pgfqpoint{2.156937in}{2.685693in}}{\pgfqpoint{2.161328in}{2.696292in}}{\pgfqpoint{2.161328in}{2.707342in}}%
\pgfpathcurveto{\pgfqpoint{2.161328in}{2.718392in}}{\pgfqpoint{2.156937in}{2.728991in}}{\pgfqpoint{2.149124in}{2.736805in}}%
\pgfpathcurveto{\pgfqpoint{2.141310in}{2.744619in}}{\pgfqpoint{2.130711in}{2.749009in}}{\pgfqpoint{2.119661in}{2.749009in}}%
\pgfpathcurveto{\pgfqpoint{2.108611in}{2.749009in}}{\pgfqpoint{2.098012in}{2.744619in}}{\pgfqpoint{2.090198in}{2.736805in}}%
\pgfpathcurveto{\pgfqpoint{2.082385in}{2.728991in}}{\pgfqpoint{2.077994in}{2.718392in}}{\pgfqpoint{2.077994in}{2.707342in}}%
\pgfpathcurveto{\pgfqpoint{2.077994in}{2.696292in}}{\pgfqpoint{2.082385in}{2.685693in}}{\pgfqpoint{2.090198in}{2.677879in}}%
\pgfpathcurveto{\pgfqpoint{2.098012in}{2.670066in}}{\pgfqpoint{2.108611in}{2.665676in}}{\pgfqpoint{2.119661in}{2.665676in}}%
\pgfpathclose%
\pgfusepath{stroke,fill}%
\end{pgfscope}%
\begin{pgfscope}%
\pgfpathrectangle{\pgfqpoint{0.600000in}{0.600000in}}{\pgfqpoint{3.900000in}{3.900000in}}%
\pgfusepath{clip}%
\pgfsetbuttcap%
\pgfsetroundjoin%
\definecolor{currentfill}{rgb}{0.121569,0.466667,0.705882}%
\pgfsetfillcolor{currentfill}%
\pgfsetlinewidth{1.003750pt}%
\definecolor{currentstroke}{rgb}{0.121569,0.466667,0.705882}%
\pgfsetstrokecolor{currentstroke}%
\pgfsetdash{}{0pt}%
\pgfpathmoveto{\pgfqpoint{2.150228in}{3.319642in}}%
\pgfpathcurveto{\pgfqpoint{2.161278in}{3.319642in}}{\pgfqpoint{2.171877in}{3.324033in}}{\pgfqpoint{2.179691in}{3.331846in}}%
\pgfpathcurveto{\pgfqpoint{2.187505in}{3.339660in}}{\pgfqpoint{2.191895in}{3.350259in}}{\pgfqpoint{2.191895in}{3.361309in}}%
\pgfpathcurveto{\pgfqpoint{2.191895in}{3.372359in}}{\pgfqpoint{2.187505in}{3.382958in}}{\pgfqpoint{2.179691in}{3.390772in}}%
\pgfpathcurveto{\pgfqpoint{2.171877in}{3.398585in}}{\pgfqpoint{2.161278in}{3.402976in}}{\pgfqpoint{2.150228in}{3.402976in}}%
\pgfpathcurveto{\pgfqpoint{2.139178in}{3.402976in}}{\pgfqpoint{2.128579in}{3.398585in}}{\pgfqpoint{2.120766in}{3.390772in}}%
\pgfpathcurveto{\pgfqpoint{2.112952in}{3.382958in}}{\pgfqpoint{2.108562in}{3.372359in}}{\pgfqpoint{2.108562in}{3.361309in}}%
\pgfpathcurveto{\pgfqpoint{2.108562in}{3.350259in}}{\pgfqpoint{2.112952in}{3.339660in}}{\pgfqpoint{2.120766in}{3.331846in}}%
\pgfpathcurveto{\pgfqpoint{2.128579in}{3.324033in}}{\pgfqpoint{2.139178in}{3.319642in}}{\pgfqpoint{2.150228in}{3.319642in}}%
\pgfpathclose%
\pgfusepath{stroke,fill}%
\end{pgfscope}%
\begin{pgfscope}%
\pgfpathrectangle{\pgfqpoint{0.600000in}{0.600000in}}{\pgfqpoint{3.900000in}{3.900000in}}%
\pgfusepath{clip}%
\pgfsetbuttcap%
\pgfsetroundjoin%
\definecolor{currentfill}{rgb}{0.121569,0.466667,0.705882}%
\pgfsetfillcolor{currentfill}%
\pgfsetlinewidth{1.003750pt}%
\definecolor{currentstroke}{rgb}{0.121569,0.466667,0.705882}%
\pgfsetstrokecolor{currentstroke}%
\pgfsetdash{}{0pt}%
\pgfpathmoveto{\pgfqpoint{3.046382in}{2.866798in}}%
\pgfpathcurveto{\pgfqpoint{3.057432in}{2.866798in}}{\pgfqpoint{3.068031in}{2.871188in}}{\pgfqpoint{3.075844in}{2.879002in}}%
\pgfpathcurveto{\pgfqpoint{3.083658in}{2.886816in}}{\pgfqpoint{3.088048in}{2.897415in}}{\pgfqpoint{3.088048in}{2.908465in}}%
\pgfpathcurveto{\pgfqpoint{3.088048in}{2.919515in}}{\pgfqpoint{3.083658in}{2.930114in}}{\pgfqpoint{3.075844in}{2.937928in}}%
\pgfpathcurveto{\pgfqpoint{3.068031in}{2.945741in}}{\pgfqpoint{3.057432in}{2.950132in}}{\pgfqpoint{3.046382in}{2.950132in}}%
\pgfpathcurveto{\pgfqpoint{3.035331in}{2.950132in}}{\pgfqpoint{3.024732in}{2.945741in}}{\pgfqpoint{3.016919in}{2.937928in}}%
\pgfpathcurveto{\pgfqpoint{3.009105in}{2.930114in}}{\pgfqpoint{3.004715in}{2.919515in}}{\pgfqpoint{3.004715in}{2.908465in}}%
\pgfpathcurveto{\pgfqpoint{3.004715in}{2.897415in}}{\pgfqpoint{3.009105in}{2.886816in}}{\pgfqpoint{3.016919in}{2.879002in}}%
\pgfpathcurveto{\pgfqpoint{3.024732in}{2.871188in}}{\pgfqpoint{3.035331in}{2.866798in}}{\pgfqpoint{3.046382in}{2.866798in}}%
\pgfpathclose%
\pgfusepath{stroke,fill}%
\end{pgfscope}%
\begin{pgfscope}%
\pgfpathrectangle{\pgfqpoint{0.600000in}{0.600000in}}{\pgfqpoint{3.900000in}{3.900000in}}%
\pgfusepath{clip}%
\pgfsetbuttcap%
\pgfsetroundjoin%
\definecolor{currentfill}{rgb}{0.121569,0.466667,0.705882}%
\pgfsetfillcolor{currentfill}%
\pgfsetlinewidth{1.003750pt}%
\definecolor{currentstroke}{rgb}{0.121569,0.466667,0.705882}%
\pgfsetstrokecolor{currentstroke}%
\pgfsetdash{}{0pt}%
\pgfpathmoveto{\pgfqpoint{2.181384in}{2.955012in}}%
\pgfpathcurveto{\pgfqpoint{2.192434in}{2.955012in}}{\pgfqpoint{2.203034in}{2.959403in}}{\pgfqpoint{2.210847in}{2.967216in}}%
\pgfpathcurveto{\pgfqpoint{2.218661in}{2.975030in}}{\pgfqpoint{2.223051in}{2.985629in}}{\pgfqpoint{2.223051in}{2.996679in}}%
\pgfpathcurveto{\pgfqpoint{2.223051in}{3.007729in}}{\pgfqpoint{2.218661in}{3.018328in}}{\pgfqpoint{2.210847in}{3.026142in}}%
\pgfpathcurveto{\pgfqpoint{2.203034in}{3.033955in}}{\pgfqpoint{2.192434in}{3.038346in}}{\pgfqpoint{2.181384in}{3.038346in}}%
\pgfpathcurveto{\pgfqpoint{2.170334in}{3.038346in}}{\pgfqpoint{2.159735in}{3.033955in}}{\pgfqpoint{2.151922in}{3.026142in}}%
\pgfpathcurveto{\pgfqpoint{2.144108in}{3.018328in}}{\pgfqpoint{2.139718in}{3.007729in}}{\pgfqpoint{2.139718in}{2.996679in}}%
\pgfpathcurveto{\pgfqpoint{2.139718in}{2.985629in}}{\pgfqpoint{2.144108in}{2.975030in}}{\pgfqpoint{2.151922in}{2.967216in}}%
\pgfpathcurveto{\pgfqpoint{2.159735in}{2.959403in}}{\pgfqpoint{2.170334in}{2.955012in}}{\pgfqpoint{2.181384in}{2.955012in}}%
\pgfpathclose%
\pgfusepath{stroke,fill}%
\end{pgfscope}%
\begin{pgfscope}%
\pgfpathrectangle{\pgfqpoint{0.600000in}{0.600000in}}{\pgfqpoint{3.900000in}{3.900000in}}%
\pgfusepath{clip}%
\pgfsetbuttcap%
\pgfsetroundjoin%
\definecolor{currentfill}{rgb}{0.121569,0.466667,0.705882}%
\pgfsetfillcolor{currentfill}%
\pgfsetlinewidth{1.003750pt}%
\definecolor{currentstroke}{rgb}{0.121569,0.466667,0.705882}%
\pgfsetstrokecolor{currentstroke}%
\pgfsetdash{}{0pt}%
\pgfpathmoveto{\pgfqpoint{2.205784in}{2.298450in}}%
\pgfpathcurveto{\pgfqpoint{2.216834in}{2.298450in}}{\pgfqpoint{2.227433in}{2.302841in}}{\pgfqpoint{2.235247in}{2.310654in}}%
\pgfpathcurveto{\pgfqpoint{2.243061in}{2.318468in}}{\pgfqpoint{2.247451in}{2.329067in}}{\pgfqpoint{2.247451in}{2.340117in}}%
\pgfpathcurveto{\pgfqpoint{2.247451in}{2.351167in}}{\pgfqpoint{2.243061in}{2.361766in}}{\pgfqpoint{2.235247in}{2.369580in}}%
\pgfpathcurveto{\pgfqpoint{2.227433in}{2.377394in}}{\pgfqpoint{2.216834in}{2.381784in}}{\pgfqpoint{2.205784in}{2.381784in}}%
\pgfpathcurveto{\pgfqpoint{2.194734in}{2.381784in}}{\pgfqpoint{2.184135in}{2.377394in}}{\pgfqpoint{2.176321in}{2.369580in}}%
\pgfpathcurveto{\pgfqpoint{2.168508in}{2.361766in}}{\pgfqpoint{2.164117in}{2.351167in}}{\pgfqpoint{2.164117in}{2.340117in}}%
\pgfpathcurveto{\pgfqpoint{2.164117in}{2.329067in}}{\pgfqpoint{2.168508in}{2.318468in}}{\pgfqpoint{2.176321in}{2.310654in}}%
\pgfpathcurveto{\pgfqpoint{2.184135in}{2.302841in}}{\pgfqpoint{2.194734in}{2.298450in}}{\pgfqpoint{2.205784in}{2.298450in}}%
\pgfpathclose%
\pgfusepath{stroke,fill}%
\end{pgfscope}%
\begin{pgfscope}%
\pgfpathrectangle{\pgfqpoint{0.600000in}{0.600000in}}{\pgfqpoint{3.900000in}{3.900000in}}%
\pgfusepath{clip}%
\pgfsetbuttcap%
\pgfsetroundjoin%
\definecolor{currentfill}{rgb}{0.121569,0.466667,0.705882}%
\pgfsetfillcolor{currentfill}%
\pgfsetlinewidth{1.003750pt}%
\definecolor{currentstroke}{rgb}{0.121569,0.466667,0.705882}%
\pgfsetstrokecolor{currentstroke}%
\pgfsetdash{}{0pt}%
\pgfpathmoveto{\pgfqpoint{1.926159in}{2.753032in}}%
\pgfpathcurveto{\pgfqpoint{1.937209in}{2.753032in}}{\pgfqpoint{1.947809in}{2.757423in}}{\pgfqpoint{1.955622in}{2.765236in}}%
\pgfpathcurveto{\pgfqpoint{1.963436in}{2.773050in}}{\pgfqpoint{1.967826in}{2.783649in}}{\pgfqpoint{1.967826in}{2.794699in}}%
\pgfpathcurveto{\pgfqpoint{1.967826in}{2.805749in}}{\pgfqpoint{1.963436in}{2.816348in}}{\pgfqpoint{1.955622in}{2.824162in}}%
\pgfpathcurveto{\pgfqpoint{1.947809in}{2.831976in}}{\pgfqpoint{1.937209in}{2.836366in}}{\pgfqpoint{1.926159in}{2.836366in}}%
\pgfpathcurveto{\pgfqpoint{1.915109in}{2.836366in}}{\pgfqpoint{1.904510in}{2.831976in}}{\pgfqpoint{1.896697in}{2.824162in}}%
\pgfpathcurveto{\pgfqpoint{1.888883in}{2.816348in}}{\pgfqpoint{1.884493in}{2.805749in}}{\pgfqpoint{1.884493in}{2.794699in}}%
\pgfpathcurveto{\pgfqpoint{1.884493in}{2.783649in}}{\pgfqpoint{1.888883in}{2.773050in}}{\pgfqpoint{1.896697in}{2.765236in}}%
\pgfpathcurveto{\pgfqpoint{1.904510in}{2.757423in}}{\pgfqpoint{1.915109in}{2.753032in}}{\pgfqpoint{1.926159in}{2.753032in}}%
\pgfpathclose%
\pgfusepath{stroke,fill}%
\end{pgfscope}%
\begin{pgfscope}%
\pgfpathrectangle{\pgfqpoint{0.600000in}{0.600000in}}{\pgfqpoint{3.900000in}{3.900000in}}%
\pgfusepath{clip}%
\pgfsetbuttcap%
\pgfsetroundjoin%
\definecolor{currentfill}{rgb}{0.121569,0.466667,0.705882}%
\pgfsetfillcolor{currentfill}%
\pgfsetlinewidth{1.003750pt}%
\definecolor{currentstroke}{rgb}{0.121569,0.466667,0.705882}%
\pgfsetstrokecolor{currentstroke}%
\pgfsetdash{}{0pt}%
\pgfpathmoveto{\pgfqpoint{3.670989in}{3.026233in}}%
\pgfpathcurveto{\pgfqpoint{3.682039in}{3.026233in}}{\pgfqpoint{3.692638in}{3.030623in}}{\pgfqpoint{3.700452in}{3.038437in}}%
\pgfpathcurveto{\pgfqpoint{3.708265in}{3.046250in}}{\pgfqpoint{3.712656in}{3.056849in}}{\pgfqpoint{3.712656in}{3.067899in}}%
\pgfpathcurveto{\pgfqpoint{3.712656in}{3.078950in}}{\pgfqpoint{3.708265in}{3.089549in}}{\pgfqpoint{3.700452in}{3.097362in}}%
\pgfpathcurveto{\pgfqpoint{3.692638in}{3.105176in}}{\pgfqpoint{3.682039in}{3.109566in}}{\pgfqpoint{3.670989in}{3.109566in}}%
\pgfpathcurveto{\pgfqpoint{3.659939in}{3.109566in}}{\pgfqpoint{3.649340in}{3.105176in}}{\pgfqpoint{3.641526in}{3.097362in}}%
\pgfpathcurveto{\pgfqpoint{3.633713in}{3.089549in}}{\pgfqpoint{3.629322in}{3.078950in}}{\pgfqpoint{3.629322in}{3.067899in}}%
\pgfpathcurveto{\pgfqpoint{3.629322in}{3.056849in}}{\pgfqpoint{3.633713in}{3.046250in}}{\pgfqpoint{3.641526in}{3.038437in}}%
\pgfpathcurveto{\pgfqpoint{3.649340in}{3.030623in}}{\pgfqpoint{3.659939in}{3.026233in}}{\pgfqpoint{3.670989in}{3.026233in}}%
\pgfpathclose%
\pgfusepath{stroke,fill}%
\end{pgfscope}%
\begin{pgfscope}%
\pgfpathrectangle{\pgfqpoint{0.600000in}{0.600000in}}{\pgfqpoint{3.900000in}{3.900000in}}%
\pgfusepath{clip}%
\pgfsetbuttcap%
\pgfsetroundjoin%
\definecolor{currentfill}{rgb}{0.121569,0.466667,0.705882}%
\pgfsetfillcolor{currentfill}%
\pgfsetlinewidth{1.003750pt}%
\definecolor{currentstroke}{rgb}{0.121569,0.466667,0.705882}%
\pgfsetstrokecolor{currentstroke}%
\pgfsetdash{}{0pt}%
\pgfpathmoveto{\pgfqpoint{2.645077in}{1.980924in}}%
\pgfpathcurveto{\pgfqpoint{2.656127in}{1.980924in}}{\pgfqpoint{2.666726in}{1.985314in}}{\pgfqpoint{2.674539in}{1.993128in}}%
\pgfpathcurveto{\pgfqpoint{2.682353in}{2.000941in}}{\pgfqpoint{2.686743in}{2.011541in}}{\pgfqpoint{2.686743in}{2.022591in}}%
\pgfpathcurveto{\pgfqpoint{2.686743in}{2.033641in}}{\pgfqpoint{2.682353in}{2.044240in}}{\pgfqpoint{2.674539in}{2.052053in}}%
\pgfpathcurveto{\pgfqpoint{2.666726in}{2.059867in}}{\pgfqpoint{2.656127in}{2.064257in}}{\pgfqpoint{2.645077in}{2.064257in}}%
\pgfpathcurveto{\pgfqpoint{2.634026in}{2.064257in}}{\pgfqpoint{2.623427in}{2.059867in}}{\pgfqpoint{2.615614in}{2.052053in}}%
\pgfpathcurveto{\pgfqpoint{2.607800in}{2.044240in}}{\pgfqpoint{2.603410in}{2.033641in}}{\pgfqpoint{2.603410in}{2.022591in}}%
\pgfpathcurveto{\pgfqpoint{2.603410in}{2.011541in}}{\pgfqpoint{2.607800in}{2.000941in}}{\pgfqpoint{2.615614in}{1.993128in}}%
\pgfpathcurveto{\pgfqpoint{2.623427in}{1.985314in}}{\pgfqpoint{2.634026in}{1.980924in}}{\pgfqpoint{2.645077in}{1.980924in}}%
\pgfpathclose%
\pgfusepath{stroke,fill}%
\end{pgfscope}%
\begin{pgfscope}%
\pgfpathrectangle{\pgfqpoint{0.600000in}{0.600000in}}{\pgfqpoint{3.900000in}{3.900000in}}%
\pgfusepath{clip}%
\pgfsetbuttcap%
\pgfsetroundjoin%
\definecolor{currentfill}{rgb}{0.121569,0.466667,0.705882}%
\pgfsetfillcolor{currentfill}%
\pgfsetlinewidth{1.003750pt}%
\definecolor{currentstroke}{rgb}{0.121569,0.466667,0.705882}%
\pgfsetstrokecolor{currentstroke}%
\pgfsetdash{}{0pt}%
\pgfpathmoveto{\pgfqpoint{1.954827in}{2.621244in}}%
\pgfpathcurveto{\pgfqpoint{1.965877in}{2.621244in}}{\pgfqpoint{1.976477in}{2.625634in}}{\pgfqpoint{1.984290in}{2.633448in}}%
\pgfpathcurveto{\pgfqpoint{1.992104in}{2.641261in}}{\pgfqpoint{1.996494in}{2.651860in}}{\pgfqpoint{1.996494in}{2.662910in}}%
\pgfpathcurveto{\pgfqpoint{1.996494in}{2.673961in}}{\pgfqpoint{1.992104in}{2.684560in}}{\pgfqpoint{1.984290in}{2.692373in}}%
\pgfpathcurveto{\pgfqpoint{1.976477in}{2.700187in}}{\pgfqpoint{1.965877in}{2.704577in}}{\pgfqpoint{1.954827in}{2.704577in}}%
\pgfpathcurveto{\pgfqpoint{1.943777in}{2.704577in}}{\pgfqpoint{1.933178in}{2.700187in}}{\pgfqpoint{1.925365in}{2.692373in}}%
\pgfpathcurveto{\pgfqpoint{1.917551in}{2.684560in}}{\pgfqpoint{1.913161in}{2.673961in}}{\pgfqpoint{1.913161in}{2.662910in}}%
\pgfpathcurveto{\pgfqpoint{1.913161in}{2.651860in}}{\pgfqpoint{1.917551in}{2.641261in}}{\pgfqpoint{1.925365in}{2.633448in}}%
\pgfpathcurveto{\pgfqpoint{1.933178in}{2.625634in}}{\pgfqpoint{1.943777in}{2.621244in}}{\pgfqpoint{1.954827in}{2.621244in}}%
\pgfpathclose%
\pgfusepath{stroke,fill}%
\end{pgfscope}%
\begin{pgfscope}%
\pgfpathrectangle{\pgfqpoint{0.600000in}{0.600000in}}{\pgfqpoint{3.900000in}{3.900000in}}%
\pgfusepath{clip}%
\pgfsetbuttcap%
\pgfsetroundjoin%
\definecolor{currentfill}{rgb}{0.121569,0.466667,0.705882}%
\pgfsetfillcolor{currentfill}%
\pgfsetlinewidth{1.003750pt}%
\definecolor{currentstroke}{rgb}{0.121569,0.466667,0.705882}%
\pgfsetstrokecolor{currentstroke}%
\pgfsetdash{}{0pt}%
\pgfpathmoveto{\pgfqpoint{1.909329in}{1.931156in}}%
\pgfpathcurveto{\pgfqpoint{1.920379in}{1.931156in}}{\pgfqpoint{1.930978in}{1.935546in}}{\pgfqpoint{1.938792in}{1.943360in}}%
\pgfpathcurveto{\pgfqpoint{1.946606in}{1.951174in}}{\pgfqpoint{1.950996in}{1.961773in}}{\pgfqpoint{1.950996in}{1.972823in}}%
\pgfpathcurveto{\pgfqpoint{1.950996in}{1.983873in}}{\pgfqpoint{1.946606in}{1.994472in}}{\pgfqpoint{1.938792in}{2.002286in}}%
\pgfpathcurveto{\pgfqpoint{1.930978in}{2.010099in}}{\pgfqpoint{1.920379in}{2.014489in}}{\pgfqpoint{1.909329in}{2.014489in}}%
\pgfpathcurveto{\pgfqpoint{1.898279in}{2.014489in}}{\pgfqpoint{1.887680in}{2.010099in}}{\pgfqpoint{1.879867in}{2.002286in}}%
\pgfpathcurveto{\pgfqpoint{1.872053in}{1.994472in}}{\pgfqpoint{1.867663in}{1.983873in}}{\pgfqpoint{1.867663in}{1.972823in}}%
\pgfpathcurveto{\pgfqpoint{1.867663in}{1.961773in}}{\pgfqpoint{1.872053in}{1.951174in}}{\pgfqpoint{1.879867in}{1.943360in}}%
\pgfpathcurveto{\pgfqpoint{1.887680in}{1.935546in}}{\pgfqpoint{1.898279in}{1.931156in}}{\pgfqpoint{1.909329in}{1.931156in}}%
\pgfpathclose%
\pgfusepath{stroke,fill}%
\end{pgfscope}%
\begin{pgfscope}%
\pgfpathrectangle{\pgfqpoint{0.600000in}{0.600000in}}{\pgfqpoint{3.900000in}{3.900000in}}%
\pgfusepath{clip}%
\pgfsetbuttcap%
\pgfsetroundjoin%
\definecolor{currentfill}{rgb}{0.121569,0.466667,0.705882}%
\pgfsetfillcolor{currentfill}%
\pgfsetlinewidth{1.003750pt}%
\definecolor{currentstroke}{rgb}{0.121569,0.466667,0.705882}%
\pgfsetstrokecolor{currentstroke}%
\pgfsetdash{}{0pt}%
\pgfpathmoveto{\pgfqpoint{2.009718in}{2.831809in}}%
\pgfpathcurveto{\pgfqpoint{2.020768in}{2.831809in}}{\pgfqpoint{2.031367in}{2.836199in}}{\pgfqpoint{2.039181in}{2.844012in}}%
\pgfpathcurveto{\pgfqpoint{2.046994in}{2.851826in}}{\pgfqpoint{2.051385in}{2.862425in}}{\pgfqpoint{2.051385in}{2.873475in}}%
\pgfpathcurveto{\pgfqpoint{2.051385in}{2.884525in}}{\pgfqpoint{2.046994in}{2.895124in}}{\pgfqpoint{2.039181in}{2.902938in}}%
\pgfpathcurveto{\pgfqpoint{2.031367in}{2.910752in}}{\pgfqpoint{2.020768in}{2.915142in}}{\pgfqpoint{2.009718in}{2.915142in}}%
\pgfpathcurveto{\pgfqpoint{1.998668in}{2.915142in}}{\pgfqpoint{1.988069in}{2.910752in}}{\pgfqpoint{1.980255in}{2.902938in}}%
\pgfpathcurveto{\pgfqpoint{1.972441in}{2.895124in}}{\pgfqpoint{1.968051in}{2.884525in}}{\pgfqpoint{1.968051in}{2.873475in}}%
\pgfpathcurveto{\pgfqpoint{1.968051in}{2.862425in}}{\pgfqpoint{1.972441in}{2.851826in}}{\pgfqpoint{1.980255in}{2.844012in}}%
\pgfpathcurveto{\pgfqpoint{1.988069in}{2.836199in}}{\pgfqpoint{1.998668in}{2.831809in}}{\pgfqpoint{2.009718in}{2.831809in}}%
\pgfpathclose%
\pgfusepath{stroke,fill}%
\end{pgfscope}%
\begin{pgfscope}%
\pgfpathrectangle{\pgfqpoint{0.600000in}{0.600000in}}{\pgfqpoint{3.900000in}{3.900000in}}%
\pgfusepath{clip}%
\pgfsetbuttcap%
\pgfsetroundjoin%
\definecolor{currentfill}{rgb}{0.121569,0.466667,0.705882}%
\pgfsetfillcolor{currentfill}%
\pgfsetlinewidth{1.003750pt}%
\definecolor{currentstroke}{rgb}{0.121569,0.466667,0.705882}%
\pgfsetstrokecolor{currentstroke}%
\pgfsetdash{}{0pt}%
\pgfpathmoveto{\pgfqpoint{3.386995in}{2.143866in}}%
\pgfpathcurveto{\pgfqpoint{3.398045in}{2.143866in}}{\pgfqpoint{3.408644in}{2.148256in}}{\pgfqpoint{3.416458in}{2.156070in}}%
\pgfpathcurveto{\pgfqpoint{3.424271in}{2.163884in}}{\pgfqpoint{3.428662in}{2.174483in}}{\pgfqpoint{3.428662in}{2.185533in}}%
\pgfpathcurveto{\pgfqpoint{3.428662in}{2.196583in}}{\pgfqpoint{3.424271in}{2.207182in}}{\pgfqpoint{3.416458in}{2.214996in}}%
\pgfpathcurveto{\pgfqpoint{3.408644in}{2.222809in}}{\pgfqpoint{3.398045in}{2.227199in}}{\pgfqpoint{3.386995in}{2.227199in}}%
\pgfpathcurveto{\pgfqpoint{3.375945in}{2.227199in}}{\pgfqpoint{3.365346in}{2.222809in}}{\pgfqpoint{3.357532in}{2.214996in}}%
\pgfpathcurveto{\pgfqpoint{3.349719in}{2.207182in}}{\pgfqpoint{3.345328in}{2.196583in}}{\pgfqpoint{3.345328in}{2.185533in}}%
\pgfpathcurveto{\pgfqpoint{3.345328in}{2.174483in}}{\pgfqpoint{3.349719in}{2.163884in}}{\pgfqpoint{3.357532in}{2.156070in}}%
\pgfpathcurveto{\pgfqpoint{3.365346in}{2.148256in}}{\pgfqpoint{3.375945in}{2.143866in}}{\pgfqpoint{3.386995in}{2.143866in}}%
\pgfpathclose%
\pgfusepath{stroke,fill}%
\end{pgfscope}%
\begin{pgfscope}%
\pgfpathrectangle{\pgfqpoint{0.600000in}{0.600000in}}{\pgfqpoint{3.900000in}{3.900000in}}%
\pgfusepath{clip}%
\pgfsetbuttcap%
\pgfsetroundjoin%
\definecolor{currentfill}{rgb}{0.121569,0.466667,0.705882}%
\pgfsetfillcolor{currentfill}%
\pgfsetlinewidth{1.003750pt}%
\definecolor{currentstroke}{rgb}{0.121569,0.466667,0.705882}%
\pgfsetstrokecolor{currentstroke}%
\pgfsetdash{}{0pt}%
\pgfpathmoveto{\pgfqpoint{2.528258in}{2.581697in}}%
\pgfpathcurveto{\pgfqpoint{2.539309in}{2.581697in}}{\pgfqpoint{2.549908in}{2.586087in}}{\pgfqpoint{2.557721in}{2.593901in}}%
\pgfpathcurveto{\pgfqpoint{2.565535in}{2.601714in}}{\pgfqpoint{2.569925in}{2.612313in}}{\pgfqpoint{2.569925in}{2.623363in}}%
\pgfpathcurveto{\pgfqpoint{2.569925in}{2.634413in}}{\pgfqpoint{2.565535in}{2.645013in}}{\pgfqpoint{2.557721in}{2.652826in}}%
\pgfpathcurveto{\pgfqpoint{2.549908in}{2.660640in}}{\pgfqpoint{2.539309in}{2.665030in}}{\pgfqpoint{2.528258in}{2.665030in}}%
\pgfpathcurveto{\pgfqpoint{2.517208in}{2.665030in}}{\pgfqpoint{2.506609in}{2.660640in}}{\pgfqpoint{2.498796in}{2.652826in}}%
\pgfpathcurveto{\pgfqpoint{2.490982in}{2.645013in}}{\pgfqpoint{2.486592in}{2.634413in}}{\pgfqpoint{2.486592in}{2.623363in}}%
\pgfpathcurveto{\pgfqpoint{2.486592in}{2.612313in}}{\pgfqpoint{2.490982in}{2.601714in}}{\pgfqpoint{2.498796in}{2.593901in}}%
\pgfpathcurveto{\pgfqpoint{2.506609in}{2.586087in}}{\pgfqpoint{2.517208in}{2.581697in}}{\pgfqpoint{2.528258in}{2.581697in}}%
\pgfpathclose%
\pgfusepath{stroke,fill}%
\end{pgfscope}%
\begin{pgfscope}%
\pgfpathrectangle{\pgfqpoint{0.600000in}{0.600000in}}{\pgfqpoint{3.900000in}{3.900000in}}%
\pgfusepath{clip}%
\pgfsetbuttcap%
\pgfsetroundjoin%
\definecolor{currentfill}{rgb}{0.121569,0.466667,0.705882}%
\pgfsetfillcolor{currentfill}%
\pgfsetlinewidth{1.003750pt}%
\definecolor{currentstroke}{rgb}{0.121569,0.466667,0.705882}%
\pgfsetstrokecolor{currentstroke}%
\pgfsetdash{}{0pt}%
\pgfpathmoveto{\pgfqpoint{2.522559in}{2.499127in}}%
\pgfpathcurveto{\pgfqpoint{2.533609in}{2.499127in}}{\pgfqpoint{2.544208in}{2.503517in}}{\pgfqpoint{2.552022in}{2.511331in}}%
\pgfpathcurveto{\pgfqpoint{2.559835in}{2.519144in}}{\pgfqpoint{2.564225in}{2.529743in}}{\pgfqpoint{2.564225in}{2.540793in}}%
\pgfpathcurveto{\pgfqpoint{2.564225in}{2.551844in}}{\pgfqpoint{2.559835in}{2.562443in}}{\pgfqpoint{2.552022in}{2.570256in}}%
\pgfpathcurveto{\pgfqpoint{2.544208in}{2.578070in}}{\pgfqpoint{2.533609in}{2.582460in}}{\pgfqpoint{2.522559in}{2.582460in}}%
\pgfpathcurveto{\pgfqpoint{2.511509in}{2.582460in}}{\pgfqpoint{2.500910in}{2.578070in}}{\pgfqpoint{2.493096in}{2.570256in}}%
\pgfpathcurveto{\pgfqpoint{2.485282in}{2.562443in}}{\pgfqpoint{2.480892in}{2.551844in}}{\pgfqpoint{2.480892in}{2.540793in}}%
\pgfpathcurveto{\pgfqpoint{2.480892in}{2.529743in}}{\pgfqpoint{2.485282in}{2.519144in}}{\pgfqpoint{2.493096in}{2.511331in}}%
\pgfpathcurveto{\pgfqpoint{2.500910in}{2.503517in}}{\pgfqpoint{2.511509in}{2.499127in}}{\pgfqpoint{2.522559in}{2.499127in}}%
\pgfpathclose%
\pgfusepath{stroke,fill}%
\end{pgfscope}%
\begin{pgfscope}%
\pgfpathrectangle{\pgfqpoint{0.600000in}{0.600000in}}{\pgfqpoint{3.900000in}{3.900000in}}%
\pgfusepath{clip}%
\pgfsetbuttcap%
\pgfsetroundjoin%
\definecolor{currentfill}{rgb}{0.121569,0.466667,0.705882}%
\pgfsetfillcolor{currentfill}%
\pgfsetlinewidth{1.003750pt}%
\definecolor{currentstroke}{rgb}{0.121569,0.466667,0.705882}%
\pgfsetstrokecolor{currentstroke}%
\pgfsetdash{}{0pt}%
\pgfpathmoveto{\pgfqpoint{3.149376in}{2.401367in}}%
\pgfpathcurveto{\pgfqpoint{3.160426in}{2.401367in}}{\pgfqpoint{3.171025in}{2.405757in}}{\pgfqpoint{3.178839in}{2.413570in}}%
\pgfpathcurveto{\pgfqpoint{3.186652in}{2.421384in}}{\pgfqpoint{3.191043in}{2.431983in}}{\pgfqpoint{3.191043in}{2.443033in}}%
\pgfpathcurveto{\pgfqpoint{3.191043in}{2.454083in}}{\pgfqpoint{3.186652in}{2.464682in}}{\pgfqpoint{3.178839in}{2.472496in}}%
\pgfpathcurveto{\pgfqpoint{3.171025in}{2.480310in}}{\pgfqpoint{3.160426in}{2.484700in}}{\pgfqpoint{3.149376in}{2.484700in}}%
\pgfpathcurveto{\pgfqpoint{3.138326in}{2.484700in}}{\pgfqpoint{3.127727in}{2.480310in}}{\pgfqpoint{3.119913in}{2.472496in}}%
\pgfpathcurveto{\pgfqpoint{3.112100in}{2.464682in}}{\pgfqpoint{3.107709in}{2.454083in}}{\pgfqpoint{3.107709in}{2.443033in}}%
\pgfpathcurveto{\pgfqpoint{3.107709in}{2.431983in}}{\pgfqpoint{3.112100in}{2.421384in}}{\pgfqpoint{3.119913in}{2.413570in}}%
\pgfpathcurveto{\pgfqpoint{3.127727in}{2.405757in}}{\pgfqpoint{3.138326in}{2.401367in}}{\pgfqpoint{3.149376in}{2.401367in}}%
\pgfpathclose%
\pgfusepath{stroke,fill}%
\end{pgfscope}%
\begin{pgfscope}%
\pgfpathrectangle{\pgfqpoint{0.600000in}{0.600000in}}{\pgfqpoint{3.900000in}{3.900000in}}%
\pgfusepath{clip}%
\pgfsetbuttcap%
\pgfsetroundjoin%
\definecolor{currentfill}{rgb}{0.121569,0.466667,0.705882}%
\pgfsetfillcolor{currentfill}%
\pgfsetlinewidth{1.003750pt}%
\definecolor{currentstroke}{rgb}{0.121569,0.466667,0.705882}%
\pgfsetstrokecolor{currentstroke}%
\pgfsetdash{}{0pt}%
\pgfpathmoveto{\pgfqpoint{3.439749in}{2.816991in}}%
\pgfpathcurveto{\pgfqpoint{3.450799in}{2.816991in}}{\pgfqpoint{3.461398in}{2.821381in}}{\pgfqpoint{3.469212in}{2.829195in}}%
\pgfpathcurveto{\pgfqpoint{3.477025in}{2.837008in}}{\pgfqpoint{3.481416in}{2.847607in}}{\pgfqpoint{3.481416in}{2.858657in}}%
\pgfpathcurveto{\pgfqpoint{3.481416in}{2.869708in}}{\pgfqpoint{3.477025in}{2.880307in}}{\pgfqpoint{3.469212in}{2.888120in}}%
\pgfpathcurveto{\pgfqpoint{3.461398in}{2.895934in}}{\pgfqpoint{3.450799in}{2.900324in}}{\pgfqpoint{3.439749in}{2.900324in}}%
\pgfpathcurveto{\pgfqpoint{3.428699in}{2.900324in}}{\pgfqpoint{3.418100in}{2.895934in}}{\pgfqpoint{3.410286in}{2.888120in}}%
\pgfpathcurveto{\pgfqpoint{3.402473in}{2.880307in}}{\pgfqpoint{3.398082in}{2.869708in}}{\pgfqpoint{3.398082in}{2.858657in}}%
\pgfpathcurveto{\pgfqpoint{3.398082in}{2.847607in}}{\pgfqpoint{3.402473in}{2.837008in}}{\pgfqpoint{3.410286in}{2.829195in}}%
\pgfpathcurveto{\pgfqpoint{3.418100in}{2.821381in}}{\pgfqpoint{3.428699in}{2.816991in}}{\pgfqpoint{3.439749in}{2.816991in}}%
\pgfpathclose%
\pgfusepath{stroke,fill}%
\end{pgfscope}%
\begin{pgfscope}%
\pgfpathrectangle{\pgfqpoint{0.600000in}{0.600000in}}{\pgfqpoint{3.900000in}{3.900000in}}%
\pgfusepath{clip}%
\pgfsetbuttcap%
\pgfsetroundjoin%
\definecolor{currentfill}{rgb}{0.121569,0.466667,0.705882}%
\pgfsetfillcolor{currentfill}%
\pgfsetlinewidth{1.003750pt}%
\definecolor{currentstroke}{rgb}{0.121569,0.466667,0.705882}%
\pgfsetstrokecolor{currentstroke}%
\pgfsetdash{}{0pt}%
\pgfpathmoveto{\pgfqpoint{1.987789in}{2.907088in}}%
\pgfpathcurveto{\pgfqpoint{1.998839in}{2.907088in}}{\pgfqpoint{2.009438in}{2.911478in}}{\pgfqpoint{2.017252in}{2.919292in}}%
\pgfpathcurveto{\pgfqpoint{2.025065in}{2.927106in}}{\pgfqpoint{2.029456in}{2.937705in}}{\pgfqpoint{2.029456in}{2.948755in}}%
\pgfpathcurveto{\pgfqpoint{2.029456in}{2.959805in}}{\pgfqpoint{2.025065in}{2.970404in}}{\pgfqpoint{2.017252in}{2.978217in}}%
\pgfpathcurveto{\pgfqpoint{2.009438in}{2.986031in}}{\pgfqpoint{1.998839in}{2.990421in}}{\pgfqpoint{1.987789in}{2.990421in}}%
\pgfpathcurveto{\pgfqpoint{1.976739in}{2.990421in}}{\pgfqpoint{1.966140in}{2.986031in}}{\pgfqpoint{1.958326in}{2.978217in}}%
\pgfpathcurveto{\pgfqpoint{1.950512in}{2.970404in}}{\pgfqpoint{1.946122in}{2.959805in}}{\pgfqpoint{1.946122in}{2.948755in}}%
\pgfpathcurveto{\pgfqpoint{1.946122in}{2.937705in}}{\pgfqpoint{1.950512in}{2.927106in}}{\pgfqpoint{1.958326in}{2.919292in}}%
\pgfpathcurveto{\pgfqpoint{1.966140in}{2.911478in}}{\pgfqpoint{1.976739in}{2.907088in}}{\pgfqpoint{1.987789in}{2.907088in}}%
\pgfpathclose%
\pgfusepath{stroke,fill}%
\end{pgfscope}%
\begin{pgfscope}%
\pgfpathrectangle{\pgfqpoint{0.600000in}{0.600000in}}{\pgfqpoint{3.900000in}{3.900000in}}%
\pgfusepath{clip}%
\pgfsetbuttcap%
\pgfsetroundjoin%
\definecolor{currentfill}{rgb}{0.121569,0.466667,0.705882}%
\pgfsetfillcolor{currentfill}%
\pgfsetlinewidth{1.003750pt}%
\definecolor{currentstroke}{rgb}{0.121569,0.466667,0.705882}%
\pgfsetstrokecolor{currentstroke}%
\pgfsetdash{}{0pt}%
\pgfpathmoveto{\pgfqpoint{2.727235in}{2.513305in}}%
\pgfpathcurveto{\pgfqpoint{2.738286in}{2.513305in}}{\pgfqpoint{2.748885in}{2.517695in}}{\pgfqpoint{2.756698in}{2.525509in}}%
\pgfpathcurveto{\pgfqpoint{2.764512in}{2.533323in}}{\pgfqpoint{2.768902in}{2.543922in}}{\pgfqpoint{2.768902in}{2.554972in}}%
\pgfpathcurveto{\pgfqpoint{2.768902in}{2.566022in}}{\pgfqpoint{2.764512in}{2.576621in}}{\pgfqpoint{2.756698in}{2.584435in}}%
\pgfpathcurveto{\pgfqpoint{2.748885in}{2.592248in}}{\pgfqpoint{2.738286in}{2.596638in}}{\pgfqpoint{2.727235in}{2.596638in}}%
\pgfpathcurveto{\pgfqpoint{2.716185in}{2.596638in}}{\pgfqpoint{2.705586in}{2.592248in}}{\pgfqpoint{2.697773in}{2.584435in}}%
\pgfpathcurveto{\pgfqpoint{2.689959in}{2.576621in}}{\pgfqpoint{2.685569in}{2.566022in}}{\pgfqpoint{2.685569in}{2.554972in}}%
\pgfpathcurveto{\pgfqpoint{2.685569in}{2.543922in}}{\pgfqpoint{2.689959in}{2.533323in}}{\pgfqpoint{2.697773in}{2.525509in}}%
\pgfpathcurveto{\pgfqpoint{2.705586in}{2.517695in}}{\pgfqpoint{2.716185in}{2.513305in}}{\pgfqpoint{2.727235in}{2.513305in}}%
\pgfpathclose%
\pgfusepath{stroke,fill}%
\end{pgfscope}%
\begin{pgfscope}%
\pgfpathrectangle{\pgfqpoint{0.600000in}{0.600000in}}{\pgfqpoint{3.900000in}{3.900000in}}%
\pgfusepath{clip}%
\pgfsetbuttcap%
\pgfsetroundjoin%
\definecolor{currentfill}{rgb}{0.121569,0.466667,0.705882}%
\pgfsetfillcolor{currentfill}%
\pgfsetlinewidth{1.003750pt}%
\definecolor{currentstroke}{rgb}{0.121569,0.466667,0.705882}%
\pgfsetstrokecolor{currentstroke}%
\pgfsetdash{}{0pt}%
\pgfpathmoveto{\pgfqpoint{3.027003in}{2.468852in}}%
\pgfpathcurveto{\pgfqpoint{3.038053in}{2.468852in}}{\pgfqpoint{3.048652in}{2.473242in}}{\pgfqpoint{3.056466in}{2.481056in}}%
\pgfpathcurveto{\pgfqpoint{3.064279in}{2.488870in}}{\pgfqpoint{3.068670in}{2.499469in}}{\pgfqpoint{3.068670in}{2.510519in}}%
\pgfpathcurveto{\pgfqpoint{3.068670in}{2.521569in}}{\pgfqpoint{3.064279in}{2.532168in}}{\pgfqpoint{3.056466in}{2.539982in}}%
\pgfpathcurveto{\pgfqpoint{3.048652in}{2.547795in}}{\pgfqpoint{3.038053in}{2.552186in}}{\pgfqpoint{3.027003in}{2.552186in}}%
\pgfpathcurveto{\pgfqpoint{3.015953in}{2.552186in}}{\pgfqpoint{3.005354in}{2.547795in}}{\pgfqpoint{2.997540in}{2.539982in}}%
\pgfpathcurveto{\pgfqpoint{2.989727in}{2.532168in}}{\pgfqpoint{2.985336in}{2.521569in}}{\pgfqpoint{2.985336in}{2.510519in}}%
\pgfpathcurveto{\pgfqpoint{2.985336in}{2.499469in}}{\pgfqpoint{2.989727in}{2.488870in}}{\pgfqpoint{2.997540in}{2.481056in}}%
\pgfpathcurveto{\pgfqpoint{3.005354in}{2.473242in}}{\pgfqpoint{3.015953in}{2.468852in}}{\pgfqpoint{3.027003in}{2.468852in}}%
\pgfpathclose%
\pgfusepath{stroke,fill}%
\end{pgfscope}%
\begin{pgfscope}%
\pgfpathrectangle{\pgfqpoint{0.600000in}{0.600000in}}{\pgfqpoint{3.900000in}{3.900000in}}%
\pgfusepath{clip}%
\pgfsetbuttcap%
\pgfsetroundjoin%
\definecolor{currentfill}{rgb}{0.121569,0.466667,0.705882}%
\pgfsetfillcolor{currentfill}%
\pgfsetlinewidth{1.003750pt}%
\definecolor{currentstroke}{rgb}{0.121569,0.466667,0.705882}%
\pgfsetstrokecolor{currentstroke}%
\pgfsetdash{}{0pt}%
\pgfpathmoveto{\pgfqpoint{2.197640in}{1.237675in}}%
\pgfpathcurveto{\pgfqpoint{2.208690in}{1.237675in}}{\pgfqpoint{2.219289in}{1.242066in}}{\pgfqpoint{2.227103in}{1.249879in}}%
\pgfpathcurveto{\pgfqpoint{2.234917in}{1.257693in}}{\pgfqpoint{2.239307in}{1.268292in}}{\pgfqpoint{2.239307in}{1.279342in}}%
\pgfpathcurveto{\pgfqpoint{2.239307in}{1.290392in}}{\pgfqpoint{2.234917in}{1.300991in}}{\pgfqpoint{2.227103in}{1.308805in}}%
\pgfpathcurveto{\pgfqpoint{2.219289in}{1.316618in}}{\pgfqpoint{2.208690in}{1.321009in}}{\pgfqpoint{2.197640in}{1.321009in}}%
\pgfpathcurveto{\pgfqpoint{2.186590in}{1.321009in}}{\pgfqpoint{2.175991in}{1.316618in}}{\pgfqpoint{2.168177in}{1.308805in}}%
\pgfpathcurveto{\pgfqpoint{2.160364in}{1.300991in}}{\pgfqpoint{2.155973in}{1.290392in}}{\pgfqpoint{2.155973in}{1.279342in}}%
\pgfpathcurveto{\pgfqpoint{2.155973in}{1.268292in}}{\pgfqpoint{2.160364in}{1.257693in}}{\pgfqpoint{2.168177in}{1.249879in}}%
\pgfpathcurveto{\pgfqpoint{2.175991in}{1.242066in}}{\pgfqpoint{2.186590in}{1.237675in}}{\pgfqpoint{2.197640in}{1.237675in}}%
\pgfpathclose%
\pgfusepath{stroke,fill}%
\end{pgfscope}%
\begin{pgfscope}%
\pgfpathrectangle{\pgfqpoint{0.600000in}{0.600000in}}{\pgfqpoint{3.900000in}{3.900000in}}%
\pgfusepath{clip}%
\pgfsetbuttcap%
\pgfsetroundjoin%
\definecolor{currentfill}{rgb}{0.121569,0.466667,0.705882}%
\pgfsetfillcolor{currentfill}%
\pgfsetlinewidth{1.003750pt}%
\definecolor{currentstroke}{rgb}{0.121569,0.466667,0.705882}%
\pgfsetstrokecolor{currentstroke}%
\pgfsetdash{}{0pt}%
\pgfpathmoveto{\pgfqpoint{2.707051in}{2.274434in}}%
\pgfpathcurveto{\pgfqpoint{2.718101in}{2.274434in}}{\pgfqpoint{2.728700in}{2.278825in}}{\pgfqpoint{2.736514in}{2.286638in}}%
\pgfpathcurveto{\pgfqpoint{2.744328in}{2.294452in}}{\pgfqpoint{2.748718in}{2.305051in}}{\pgfqpoint{2.748718in}{2.316101in}}%
\pgfpathcurveto{\pgfqpoint{2.748718in}{2.327151in}}{\pgfqpoint{2.744328in}{2.337750in}}{\pgfqpoint{2.736514in}{2.345564in}}%
\pgfpathcurveto{\pgfqpoint{2.728700in}{2.353377in}}{\pgfqpoint{2.718101in}{2.357768in}}{\pgfqpoint{2.707051in}{2.357768in}}%
\pgfpathcurveto{\pgfqpoint{2.696001in}{2.357768in}}{\pgfqpoint{2.685402in}{2.353377in}}{\pgfqpoint{2.677588in}{2.345564in}}%
\pgfpathcurveto{\pgfqpoint{2.669775in}{2.337750in}}{\pgfqpoint{2.665385in}{2.327151in}}{\pgfqpoint{2.665385in}{2.316101in}}%
\pgfpathcurveto{\pgfqpoint{2.665385in}{2.305051in}}{\pgfqpoint{2.669775in}{2.294452in}}{\pgfqpoint{2.677588in}{2.286638in}}%
\pgfpathcurveto{\pgfqpoint{2.685402in}{2.278825in}}{\pgfqpoint{2.696001in}{2.274434in}}{\pgfqpoint{2.707051in}{2.274434in}}%
\pgfpathclose%
\pgfusepath{stroke,fill}%
\end{pgfscope}%
\begin{pgfscope}%
\pgfpathrectangle{\pgfqpoint{0.600000in}{0.600000in}}{\pgfqpoint{3.900000in}{3.900000in}}%
\pgfusepath{clip}%
\pgfsetbuttcap%
\pgfsetroundjoin%
\definecolor{currentfill}{rgb}{0.121569,0.466667,0.705882}%
\pgfsetfillcolor{currentfill}%
\pgfsetlinewidth{1.003750pt}%
\definecolor{currentstroke}{rgb}{0.121569,0.466667,0.705882}%
\pgfsetstrokecolor{currentstroke}%
\pgfsetdash{}{0pt}%
\pgfpathmoveto{\pgfqpoint{2.142437in}{2.737511in}}%
\pgfpathcurveto{\pgfqpoint{2.153487in}{2.737511in}}{\pgfqpoint{2.164086in}{2.741901in}}{\pgfqpoint{2.171900in}{2.749715in}}%
\pgfpathcurveto{\pgfqpoint{2.179713in}{2.757528in}}{\pgfqpoint{2.184104in}{2.768127in}}{\pgfqpoint{2.184104in}{2.779177in}}%
\pgfpathcurveto{\pgfqpoint{2.184104in}{2.790228in}}{\pgfqpoint{2.179713in}{2.800827in}}{\pgfqpoint{2.171900in}{2.808640in}}%
\pgfpathcurveto{\pgfqpoint{2.164086in}{2.816454in}}{\pgfqpoint{2.153487in}{2.820844in}}{\pgfqpoint{2.142437in}{2.820844in}}%
\pgfpathcurveto{\pgfqpoint{2.131387in}{2.820844in}}{\pgfqpoint{2.120788in}{2.816454in}}{\pgfqpoint{2.112974in}{2.808640in}}%
\pgfpathcurveto{\pgfqpoint{2.105161in}{2.800827in}}{\pgfqpoint{2.100770in}{2.790228in}}{\pgfqpoint{2.100770in}{2.779177in}}%
\pgfpathcurveto{\pgfqpoint{2.100770in}{2.768127in}}{\pgfqpoint{2.105161in}{2.757528in}}{\pgfqpoint{2.112974in}{2.749715in}}%
\pgfpathcurveto{\pgfqpoint{2.120788in}{2.741901in}}{\pgfqpoint{2.131387in}{2.737511in}}{\pgfqpoint{2.142437in}{2.737511in}}%
\pgfpathclose%
\pgfusepath{stroke,fill}%
\end{pgfscope}%
\begin{pgfscope}%
\pgfpathrectangle{\pgfqpoint{0.600000in}{0.600000in}}{\pgfqpoint{3.900000in}{3.900000in}}%
\pgfusepath{clip}%
\pgfsetbuttcap%
\pgfsetroundjoin%
\definecolor{currentfill}{rgb}{0.121569,0.466667,0.705882}%
\pgfsetfillcolor{currentfill}%
\pgfsetlinewidth{1.003750pt}%
\definecolor{currentstroke}{rgb}{0.121569,0.466667,0.705882}%
\pgfsetstrokecolor{currentstroke}%
\pgfsetdash{}{0pt}%
\pgfpathmoveto{\pgfqpoint{1.681839in}{2.617690in}}%
\pgfpathcurveto{\pgfqpoint{1.692889in}{2.617690in}}{\pgfqpoint{1.703488in}{2.622081in}}{\pgfqpoint{1.711302in}{2.629894in}}%
\pgfpathcurveto{\pgfqpoint{1.719115in}{2.637708in}}{\pgfqpoint{1.723506in}{2.648307in}}{\pgfqpoint{1.723506in}{2.659357in}}%
\pgfpathcurveto{\pgfqpoint{1.723506in}{2.670407in}}{\pgfqpoint{1.719115in}{2.681006in}}{\pgfqpoint{1.711302in}{2.688820in}}%
\pgfpathcurveto{\pgfqpoint{1.703488in}{2.696634in}}{\pgfqpoint{1.692889in}{2.701024in}}{\pgfqpoint{1.681839in}{2.701024in}}%
\pgfpathcurveto{\pgfqpoint{1.670789in}{2.701024in}}{\pgfqpoint{1.660190in}{2.696634in}}{\pgfqpoint{1.652376in}{2.688820in}}%
\pgfpathcurveto{\pgfqpoint{1.644562in}{2.681006in}}{\pgfqpoint{1.640172in}{2.670407in}}{\pgfqpoint{1.640172in}{2.659357in}}%
\pgfpathcurveto{\pgfqpoint{1.640172in}{2.648307in}}{\pgfqpoint{1.644562in}{2.637708in}}{\pgfqpoint{1.652376in}{2.629894in}}%
\pgfpathcurveto{\pgfqpoint{1.660190in}{2.622081in}}{\pgfqpoint{1.670789in}{2.617690in}}{\pgfqpoint{1.681839in}{2.617690in}}%
\pgfpathclose%
\pgfusepath{stroke,fill}%
\end{pgfscope}%
\begin{pgfscope}%
\pgfpathrectangle{\pgfqpoint{0.600000in}{0.600000in}}{\pgfqpoint{3.900000in}{3.900000in}}%
\pgfusepath{clip}%
\pgfsetbuttcap%
\pgfsetroundjoin%
\definecolor{currentfill}{rgb}{0.121569,0.466667,0.705882}%
\pgfsetfillcolor{currentfill}%
\pgfsetlinewidth{1.003750pt}%
\definecolor{currentstroke}{rgb}{0.121569,0.466667,0.705882}%
\pgfsetstrokecolor{currentstroke}%
\pgfsetdash{}{0pt}%
\pgfpathmoveto{\pgfqpoint{2.462221in}{2.107991in}}%
\pgfpathcurveto{\pgfqpoint{2.473271in}{2.107991in}}{\pgfqpoint{2.483870in}{2.112381in}}{\pgfqpoint{2.491684in}{2.120195in}}%
\pgfpathcurveto{\pgfqpoint{2.499497in}{2.128009in}}{\pgfqpoint{2.503887in}{2.138608in}}{\pgfqpoint{2.503887in}{2.149658in}}%
\pgfpathcurveto{\pgfqpoint{2.503887in}{2.160708in}}{\pgfqpoint{2.499497in}{2.171307in}}{\pgfqpoint{2.491684in}{2.179121in}}%
\pgfpathcurveto{\pgfqpoint{2.483870in}{2.186934in}}{\pgfqpoint{2.473271in}{2.191325in}}{\pgfqpoint{2.462221in}{2.191325in}}%
\pgfpathcurveto{\pgfqpoint{2.451171in}{2.191325in}}{\pgfqpoint{2.440572in}{2.186934in}}{\pgfqpoint{2.432758in}{2.179121in}}%
\pgfpathcurveto{\pgfqpoint{2.424944in}{2.171307in}}{\pgfqpoint{2.420554in}{2.160708in}}{\pgfqpoint{2.420554in}{2.149658in}}%
\pgfpathcurveto{\pgfqpoint{2.420554in}{2.138608in}}{\pgfqpoint{2.424944in}{2.128009in}}{\pgfqpoint{2.432758in}{2.120195in}}%
\pgfpathcurveto{\pgfqpoint{2.440572in}{2.112381in}}{\pgfqpoint{2.451171in}{2.107991in}}{\pgfqpoint{2.462221in}{2.107991in}}%
\pgfpathclose%
\pgfusepath{stroke,fill}%
\end{pgfscope}%
\begin{pgfscope}%
\pgfpathrectangle{\pgfqpoint{0.600000in}{0.600000in}}{\pgfqpoint{3.900000in}{3.900000in}}%
\pgfusepath{clip}%
\pgfsetbuttcap%
\pgfsetroundjoin%
\definecolor{currentfill}{rgb}{0.121569,0.466667,0.705882}%
\pgfsetfillcolor{currentfill}%
\pgfsetlinewidth{1.003750pt}%
\definecolor{currentstroke}{rgb}{0.121569,0.466667,0.705882}%
\pgfsetstrokecolor{currentstroke}%
\pgfsetdash{}{0pt}%
\pgfpathmoveto{\pgfqpoint{1.785120in}{2.743828in}}%
\pgfpathcurveto{\pgfqpoint{1.796170in}{2.743828in}}{\pgfqpoint{1.806769in}{2.748219in}}{\pgfqpoint{1.814582in}{2.756032in}}%
\pgfpathcurveto{\pgfqpoint{1.822396in}{2.763846in}}{\pgfqpoint{1.826786in}{2.774445in}}{\pgfqpoint{1.826786in}{2.785495in}}%
\pgfpathcurveto{\pgfqpoint{1.826786in}{2.796545in}}{\pgfqpoint{1.822396in}{2.807144in}}{\pgfqpoint{1.814582in}{2.814958in}}%
\pgfpathcurveto{\pgfqpoint{1.806769in}{2.822771in}}{\pgfqpoint{1.796170in}{2.827162in}}{\pgfqpoint{1.785120in}{2.827162in}}%
\pgfpathcurveto{\pgfqpoint{1.774070in}{2.827162in}}{\pgfqpoint{1.763470in}{2.822771in}}{\pgfqpoint{1.755657in}{2.814958in}}%
\pgfpathcurveto{\pgfqpoint{1.747843in}{2.807144in}}{\pgfqpoint{1.743453in}{2.796545in}}{\pgfqpoint{1.743453in}{2.785495in}}%
\pgfpathcurveto{\pgfqpoint{1.743453in}{2.774445in}}{\pgfqpoint{1.747843in}{2.763846in}}{\pgfqpoint{1.755657in}{2.756032in}}%
\pgfpathcurveto{\pgfqpoint{1.763470in}{2.748219in}}{\pgfqpoint{1.774070in}{2.743828in}}{\pgfqpoint{1.785120in}{2.743828in}}%
\pgfpathclose%
\pgfusepath{stroke,fill}%
\end{pgfscope}%
\begin{pgfscope}%
\pgfpathrectangle{\pgfqpoint{0.600000in}{0.600000in}}{\pgfqpoint{3.900000in}{3.900000in}}%
\pgfusepath{clip}%
\pgfsetbuttcap%
\pgfsetroundjoin%
\definecolor{currentfill}{rgb}{0.121569,0.466667,0.705882}%
\pgfsetfillcolor{currentfill}%
\pgfsetlinewidth{1.003750pt}%
\definecolor{currentstroke}{rgb}{0.121569,0.466667,0.705882}%
\pgfsetstrokecolor{currentstroke}%
\pgfsetdash{}{0pt}%
\pgfpathmoveto{\pgfqpoint{2.180705in}{2.251920in}}%
\pgfpathcurveto{\pgfqpoint{2.191755in}{2.251920in}}{\pgfqpoint{2.202354in}{2.256310in}}{\pgfqpoint{2.210168in}{2.264124in}}%
\pgfpathcurveto{\pgfqpoint{2.217982in}{2.271937in}}{\pgfqpoint{2.222372in}{2.282537in}}{\pgfqpoint{2.222372in}{2.293587in}}%
\pgfpathcurveto{\pgfqpoint{2.222372in}{2.304637in}}{\pgfqpoint{2.217982in}{2.315236in}}{\pgfqpoint{2.210168in}{2.323049in}}%
\pgfpathcurveto{\pgfqpoint{2.202354in}{2.330863in}}{\pgfqpoint{2.191755in}{2.335253in}}{\pgfqpoint{2.180705in}{2.335253in}}%
\pgfpathcurveto{\pgfqpoint{2.169655in}{2.335253in}}{\pgfqpoint{2.159056in}{2.330863in}}{\pgfqpoint{2.151242in}{2.323049in}}%
\pgfpathcurveto{\pgfqpoint{2.143429in}{2.315236in}}{\pgfqpoint{2.139039in}{2.304637in}}{\pgfqpoint{2.139039in}{2.293587in}}%
\pgfpathcurveto{\pgfqpoint{2.139039in}{2.282537in}}{\pgfqpoint{2.143429in}{2.271937in}}{\pgfqpoint{2.151242in}{2.264124in}}%
\pgfpathcurveto{\pgfqpoint{2.159056in}{2.256310in}}{\pgfqpoint{2.169655in}{2.251920in}}{\pgfqpoint{2.180705in}{2.251920in}}%
\pgfpathclose%
\pgfusepath{stroke,fill}%
\end{pgfscope}%
\begin{pgfscope}%
\pgfpathrectangle{\pgfqpoint{0.600000in}{0.600000in}}{\pgfqpoint{3.900000in}{3.900000in}}%
\pgfusepath{clip}%
\pgfsetbuttcap%
\pgfsetroundjoin%
\definecolor{currentfill}{rgb}{0.121569,0.466667,0.705882}%
\pgfsetfillcolor{currentfill}%
\pgfsetlinewidth{1.003750pt}%
\definecolor{currentstroke}{rgb}{0.121569,0.466667,0.705882}%
\pgfsetstrokecolor{currentstroke}%
\pgfsetdash{}{0pt}%
\pgfpathmoveto{\pgfqpoint{2.417965in}{3.417535in}}%
\pgfpathcurveto{\pgfqpoint{2.429015in}{3.417535in}}{\pgfqpoint{2.439614in}{3.421926in}}{\pgfqpoint{2.447428in}{3.429739in}}%
\pgfpathcurveto{\pgfqpoint{2.455241in}{3.437553in}}{\pgfqpoint{2.459632in}{3.448152in}}{\pgfqpoint{2.459632in}{3.459202in}}%
\pgfpathcurveto{\pgfqpoint{2.459632in}{3.470252in}}{\pgfqpoint{2.455241in}{3.480851in}}{\pgfqpoint{2.447428in}{3.488665in}}%
\pgfpathcurveto{\pgfqpoint{2.439614in}{3.496478in}}{\pgfqpoint{2.429015in}{3.500869in}}{\pgfqpoint{2.417965in}{3.500869in}}%
\pgfpathcurveto{\pgfqpoint{2.406915in}{3.500869in}}{\pgfqpoint{2.396316in}{3.496478in}}{\pgfqpoint{2.388502in}{3.488665in}}%
\pgfpathcurveto{\pgfqpoint{2.380689in}{3.480851in}}{\pgfqpoint{2.376298in}{3.470252in}}{\pgfqpoint{2.376298in}{3.459202in}}%
\pgfpathcurveto{\pgfqpoint{2.376298in}{3.448152in}}{\pgfqpoint{2.380689in}{3.437553in}}{\pgfqpoint{2.388502in}{3.429739in}}%
\pgfpathcurveto{\pgfqpoint{2.396316in}{3.421926in}}{\pgfqpoint{2.406915in}{3.417535in}}{\pgfqpoint{2.417965in}{3.417535in}}%
\pgfpathclose%
\pgfusepath{stroke,fill}%
\end{pgfscope}%
\begin{pgfscope}%
\pgfpathrectangle{\pgfqpoint{0.600000in}{0.600000in}}{\pgfqpoint{3.900000in}{3.900000in}}%
\pgfusepath{clip}%
\pgfsetbuttcap%
\pgfsetroundjoin%
\definecolor{currentfill}{rgb}{0.121569,0.466667,0.705882}%
\pgfsetfillcolor{currentfill}%
\pgfsetlinewidth{1.003750pt}%
\definecolor{currentstroke}{rgb}{0.121569,0.466667,0.705882}%
\pgfsetstrokecolor{currentstroke}%
\pgfsetdash{}{0pt}%
\pgfpathmoveto{\pgfqpoint{2.014598in}{3.888033in}}%
\pgfpathcurveto{\pgfqpoint{2.025648in}{3.888033in}}{\pgfqpoint{2.036247in}{3.892423in}}{\pgfqpoint{2.044061in}{3.900237in}}%
\pgfpathcurveto{\pgfqpoint{2.051875in}{3.908051in}}{\pgfqpoint{2.056265in}{3.918650in}}{\pgfqpoint{2.056265in}{3.929700in}}%
\pgfpathcurveto{\pgfqpoint{2.056265in}{3.940750in}}{\pgfqpoint{2.051875in}{3.951349in}}{\pgfqpoint{2.044061in}{3.959163in}}%
\pgfpathcurveto{\pgfqpoint{2.036247in}{3.966976in}}{\pgfqpoint{2.025648in}{3.971367in}}{\pgfqpoint{2.014598in}{3.971367in}}%
\pgfpathcurveto{\pgfqpoint{2.003548in}{3.971367in}}{\pgfqpoint{1.992949in}{3.966976in}}{\pgfqpoint{1.985135in}{3.959163in}}%
\pgfpathcurveto{\pgfqpoint{1.977322in}{3.951349in}}{\pgfqpoint{1.972931in}{3.940750in}}{\pgfqpoint{1.972931in}{3.929700in}}%
\pgfpathcurveto{\pgfqpoint{1.972931in}{3.918650in}}{\pgfqpoint{1.977322in}{3.908051in}}{\pgfqpoint{1.985135in}{3.900237in}}%
\pgfpathcurveto{\pgfqpoint{1.992949in}{3.892423in}}{\pgfqpoint{2.003548in}{3.888033in}}{\pgfqpoint{2.014598in}{3.888033in}}%
\pgfpathclose%
\pgfusepath{stroke,fill}%
\end{pgfscope}%
\begin{pgfscope}%
\pgfpathrectangle{\pgfqpoint{0.600000in}{0.600000in}}{\pgfqpoint{3.900000in}{3.900000in}}%
\pgfusepath{clip}%
\pgfsetbuttcap%
\pgfsetroundjoin%
\definecolor{currentfill}{rgb}{0.121569,0.466667,0.705882}%
\pgfsetfillcolor{currentfill}%
\pgfsetlinewidth{1.003750pt}%
\definecolor{currentstroke}{rgb}{0.121569,0.466667,0.705882}%
\pgfsetstrokecolor{currentstroke}%
\pgfsetdash{}{0pt}%
\pgfpathmoveto{\pgfqpoint{2.546254in}{2.077267in}}%
\pgfpathcurveto{\pgfqpoint{2.557304in}{2.077267in}}{\pgfqpoint{2.567903in}{2.081657in}}{\pgfqpoint{2.575717in}{2.089471in}}%
\pgfpathcurveto{\pgfqpoint{2.583530in}{2.097284in}}{\pgfqpoint{2.587921in}{2.107883in}}{\pgfqpoint{2.587921in}{2.118933in}}%
\pgfpathcurveto{\pgfqpoint{2.587921in}{2.129983in}}{\pgfqpoint{2.583530in}{2.140582in}}{\pgfqpoint{2.575717in}{2.148396in}}%
\pgfpathcurveto{\pgfqpoint{2.567903in}{2.156210in}}{\pgfqpoint{2.557304in}{2.160600in}}{\pgfqpoint{2.546254in}{2.160600in}}%
\pgfpathcurveto{\pgfqpoint{2.535204in}{2.160600in}}{\pgfqpoint{2.524605in}{2.156210in}}{\pgfqpoint{2.516791in}{2.148396in}}%
\pgfpathcurveto{\pgfqpoint{2.508977in}{2.140582in}}{\pgfqpoint{2.504587in}{2.129983in}}{\pgfqpoint{2.504587in}{2.118933in}}%
\pgfpathcurveto{\pgfqpoint{2.504587in}{2.107883in}}{\pgfqpoint{2.508977in}{2.097284in}}{\pgfqpoint{2.516791in}{2.089471in}}%
\pgfpathcurveto{\pgfqpoint{2.524605in}{2.081657in}}{\pgfqpoint{2.535204in}{2.077267in}}{\pgfqpoint{2.546254in}{2.077267in}}%
\pgfpathclose%
\pgfusepath{stroke,fill}%
\end{pgfscope}%
\begin{pgfscope}%
\pgfpathrectangle{\pgfqpoint{0.600000in}{0.600000in}}{\pgfqpoint{3.900000in}{3.900000in}}%
\pgfusepath{clip}%
\pgfsetbuttcap%
\pgfsetroundjoin%
\definecolor{currentfill}{rgb}{0.121569,0.466667,0.705882}%
\pgfsetfillcolor{currentfill}%
\pgfsetlinewidth{1.003750pt}%
\definecolor{currentstroke}{rgb}{0.121569,0.466667,0.705882}%
\pgfsetstrokecolor{currentstroke}%
\pgfsetdash{}{0pt}%
\pgfpathmoveto{\pgfqpoint{2.013318in}{1.973569in}}%
\pgfpathcurveto{\pgfqpoint{2.024368in}{1.973569in}}{\pgfqpoint{2.034967in}{1.977959in}}{\pgfqpoint{2.042780in}{1.985773in}}%
\pgfpathcurveto{\pgfqpoint{2.050594in}{1.993586in}}{\pgfqpoint{2.054984in}{2.004185in}}{\pgfqpoint{2.054984in}{2.015235in}}%
\pgfpathcurveto{\pgfqpoint{2.054984in}{2.026285in}}{\pgfqpoint{2.050594in}{2.036885in}}{\pgfqpoint{2.042780in}{2.044698in}}%
\pgfpathcurveto{\pgfqpoint{2.034967in}{2.052512in}}{\pgfqpoint{2.024368in}{2.056902in}}{\pgfqpoint{2.013318in}{2.056902in}}%
\pgfpathcurveto{\pgfqpoint{2.002268in}{2.056902in}}{\pgfqpoint{1.991668in}{2.052512in}}{\pgfqpoint{1.983855in}{2.044698in}}%
\pgfpathcurveto{\pgfqpoint{1.976041in}{2.036885in}}{\pgfqpoint{1.971651in}{2.026285in}}{\pgfqpoint{1.971651in}{2.015235in}}%
\pgfpathcurveto{\pgfqpoint{1.971651in}{2.004185in}}{\pgfqpoint{1.976041in}{1.993586in}}{\pgfqpoint{1.983855in}{1.985773in}}%
\pgfpathcurveto{\pgfqpoint{1.991668in}{1.977959in}}{\pgfqpoint{2.002268in}{1.973569in}}{\pgfqpoint{2.013318in}{1.973569in}}%
\pgfpathclose%
\pgfusepath{stroke,fill}%
\end{pgfscope}%
\begin{pgfscope}%
\pgfpathrectangle{\pgfqpoint{0.600000in}{0.600000in}}{\pgfqpoint{3.900000in}{3.900000in}}%
\pgfusepath{clip}%
\pgfsetbuttcap%
\pgfsetroundjoin%
\definecolor{currentfill}{rgb}{0.121569,0.466667,0.705882}%
\pgfsetfillcolor{currentfill}%
\pgfsetlinewidth{1.003750pt}%
\definecolor{currentstroke}{rgb}{0.121569,0.466667,0.705882}%
\pgfsetstrokecolor{currentstroke}%
\pgfsetdash{}{0pt}%
\pgfpathmoveto{\pgfqpoint{2.422468in}{2.193672in}}%
\pgfpathcurveto{\pgfqpoint{2.433518in}{2.193672in}}{\pgfqpoint{2.444117in}{2.198062in}}{\pgfqpoint{2.451930in}{2.205876in}}%
\pgfpathcurveto{\pgfqpoint{2.459744in}{2.213690in}}{\pgfqpoint{2.464134in}{2.224289in}}{\pgfqpoint{2.464134in}{2.235339in}}%
\pgfpathcurveto{\pgfqpoint{2.464134in}{2.246389in}}{\pgfqpoint{2.459744in}{2.256988in}}{\pgfqpoint{2.451930in}{2.264802in}}%
\pgfpathcurveto{\pgfqpoint{2.444117in}{2.272615in}}{\pgfqpoint{2.433518in}{2.277005in}}{\pgfqpoint{2.422468in}{2.277005in}}%
\pgfpathcurveto{\pgfqpoint{2.411418in}{2.277005in}}{\pgfqpoint{2.400819in}{2.272615in}}{\pgfqpoint{2.393005in}{2.264802in}}%
\pgfpathcurveto{\pgfqpoint{2.385191in}{2.256988in}}{\pgfqpoint{2.380801in}{2.246389in}}{\pgfqpoint{2.380801in}{2.235339in}}%
\pgfpathcurveto{\pgfqpoint{2.380801in}{2.224289in}}{\pgfqpoint{2.385191in}{2.213690in}}{\pgfqpoint{2.393005in}{2.205876in}}%
\pgfpathcurveto{\pgfqpoint{2.400819in}{2.198062in}}{\pgfqpoint{2.411418in}{2.193672in}}{\pgfqpoint{2.422468in}{2.193672in}}%
\pgfpathclose%
\pgfusepath{stroke,fill}%
\end{pgfscope}%
\begin{pgfscope}%
\pgfpathrectangle{\pgfqpoint{0.600000in}{0.600000in}}{\pgfqpoint{3.900000in}{3.900000in}}%
\pgfusepath{clip}%
\pgfsetbuttcap%
\pgfsetroundjoin%
\definecolor{currentfill}{rgb}{0.121569,0.466667,0.705882}%
\pgfsetfillcolor{currentfill}%
\pgfsetlinewidth{1.003750pt}%
\definecolor{currentstroke}{rgb}{0.121569,0.466667,0.705882}%
\pgfsetstrokecolor{currentstroke}%
\pgfsetdash{}{0pt}%
\pgfpathmoveto{\pgfqpoint{2.517933in}{2.484529in}}%
\pgfpathcurveto{\pgfqpoint{2.528984in}{2.484529in}}{\pgfqpoint{2.539583in}{2.488919in}}{\pgfqpoint{2.547396in}{2.496733in}}%
\pgfpathcurveto{\pgfqpoint{2.555210in}{2.504547in}}{\pgfqpoint{2.559600in}{2.515146in}}{\pgfqpoint{2.559600in}{2.526196in}}%
\pgfpathcurveto{\pgfqpoint{2.559600in}{2.537246in}}{\pgfqpoint{2.555210in}{2.547845in}}{\pgfqpoint{2.547396in}{2.555659in}}%
\pgfpathcurveto{\pgfqpoint{2.539583in}{2.563472in}}{\pgfqpoint{2.528984in}{2.567863in}}{\pgfqpoint{2.517933in}{2.567863in}}%
\pgfpathcurveto{\pgfqpoint{2.506883in}{2.567863in}}{\pgfqpoint{2.496284in}{2.563472in}}{\pgfqpoint{2.488471in}{2.555659in}}%
\pgfpathcurveto{\pgfqpoint{2.480657in}{2.547845in}}{\pgfqpoint{2.476267in}{2.537246in}}{\pgfqpoint{2.476267in}{2.526196in}}%
\pgfpathcurveto{\pgfqpoint{2.476267in}{2.515146in}}{\pgfqpoint{2.480657in}{2.504547in}}{\pgfqpoint{2.488471in}{2.496733in}}%
\pgfpathcurveto{\pgfqpoint{2.496284in}{2.488919in}}{\pgfqpoint{2.506883in}{2.484529in}}{\pgfqpoint{2.517933in}{2.484529in}}%
\pgfpathclose%
\pgfusepath{stroke,fill}%
\end{pgfscope}%
\begin{pgfscope}%
\pgfpathrectangle{\pgfqpoint{0.600000in}{0.600000in}}{\pgfqpoint{3.900000in}{3.900000in}}%
\pgfusepath{clip}%
\pgfsetbuttcap%
\pgfsetroundjoin%
\definecolor{currentfill}{rgb}{0.121569,0.466667,0.705882}%
\pgfsetfillcolor{currentfill}%
\pgfsetlinewidth{1.003750pt}%
\definecolor{currentstroke}{rgb}{0.121569,0.466667,0.705882}%
\pgfsetstrokecolor{currentstroke}%
\pgfsetdash{}{0pt}%
\pgfpathmoveto{\pgfqpoint{2.738214in}{2.691716in}}%
\pgfpathcurveto{\pgfqpoint{2.749264in}{2.691716in}}{\pgfqpoint{2.759863in}{2.696106in}}{\pgfqpoint{2.767677in}{2.703920in}}%
\pgfpathcurveto{\pgfqpoint{2.775490in}{2.711733in}}{\pgfqpoint{2.779881in}{2.722332in}}{\pgfqpoint{2.779881in}{2.733382in}}%
\pgfpathcurveto{\pgfqpoint{2.779881in}{2.744433in}}{\pgfqpoint{2.775490in}{2.755032in}}{\pgfqpoint{2.767677in}{2.762845in}}%
\pgfpathcurveto{\pgfqpoint{2.759863in}{2.770659in}}{\pgfqpoint{2.749264in}{2.775049in}}{\pgfqpoint{2.738214in}{2.775049in}}%
\pgfpathcurveto{\pgfqpoint{2.727164in}{2.775049in}}{\pgfqpoint{2.716565in}{2.770659in}}{\pgfqpoint{2.708751in}{2.762845in}}%
\pgfpathcurveto{\pgfqpoint{2.700938in}{2.755032in}}{\pgfqpoint{2.696547in}{2.744433in}}{\pgfqpoint{2.696547in}{2.733382in}}%
\pgfpathcurveto{\pgfqpoint{2.696547in}{2.722332in}}{\pgfqpoint{2.700938in}{2.711733in}}{\pgfqpoint{2.708751in}{2.703920in}}%
\pgfpathcurveto{\pgfqpoint{2.716565in}{2.696106in}}{\pgfqpoint{2.727164in}{2.691716in}}{\pgfqpoint{2.738214in}{2.691716in}}%
\pgfpathclose%
\pgfusepath{stroke,fill}%
\end{pgfscope}%
\begin{pgfscope}%
\pgfpathrectangle{\pgfqpoint{0.600000in}{0.600000in}}{\pgfqpoint{3.900000in}{3.900000in}}%
\pgfusepath{clip}%
\pgfsetbuttcap%
\pgfsetroundjoin%
\definecolor{currentfill}{rgb}{0.121569,0.466667,0.705882}%
\pgfsetfillcolor{currentfill}%
\pgfsetlinewidth{1.003750pt}%
\definecolor{currentstroke}{rgb}{0.121569,0.466667,0.705882}%
\pgfsetstrokecolor{currentstroke}%
\pgfsetdash{}{0pt}%
\pgfpathmoveto{\pgfqpoint{2.956733in}{2.542486in}}%
\pgfpathcurveto{\pgfqpoint{2.967783in}{2.542486in}}{\pgfqpoint{2.978382in}{2.546876in}}{\pgfqpoint{2.986196in}{2.554690in}}%
\pgfpathcurveto{\pgfqpoint{2.994010in}{2.562504in}}{\pgfqpoint{2.998400in}{2.573103in}}{\pgfqpoint{2.998400in}{2.584153in}}%
\pgfpathcurveto{\pgfqpoint{2.998400in}{2.595203in}}{\pgfqpoint{2.994010in}{2.605802in}}{\pgfqpoint{2.986196in}{2.613616in}}%
\pgfpathcurveto{\pgfqpoint{2.978382in}{2.621429in}}{\pgfqpoint{2.967783in}{2.625820in}}{\pgfqpoint{2.956733in}{2.625820in}}%
\pgfpathcurveto{\pgfqpoint{2.945683in}{2.625820in}}{\pgfqpoint{2.935084in}{2.621429in}}{\pgfqpoint{2.927270in}{2.613616in}}%
\pgfpathcurveto{\pgfqpoint{2.919457in}{2.605802in}}{\pgfqpoint{2.915067in}{2.595203in}}{\pgfqpoint{2.915067in}{2.584153in}}%
\pgfpathcurveto{\pgfqpoint{2.915067in}{2.573103in}}{\pgfqpoint{2.919457in}{2.562504in}}{\pgfqpoint{2.927270in}{2.554690in}}%
\pgfpathcurveto{\pgfqpoint{2.935084in}{2.546876in}}{\pgfqpoint{2.945683in}{2.542486in}}{\pgfqpoint{2.956733in}{2.542486in}}%
\pgfpathclose%
\pgfusepath{stroke,fill}%
\end{pgfscope}%
\begin{pgfscope}%
\pgfpathrectangle{\pgfqpoint{0.600000in}{0.600000in}}{\pgfqpoint{3.900000in}{3.900000in}}%
\pgfusepath{clip}%
\pgfsetbuttcap%
\pgfsetroundjoin%
\definecolor{currentfill}{rgb}{0.121569,0.466667,0.705882}%
\pgfsetfillcolor{currentfill}%
\pgfsetlinewidth{1.003750pt}%
\definecolor{currentstroke}{rgb}{0.121569,0.466667,0.705882}%
\pgfsetstrokecolor{currentstroke}%
\pgfsetdash{}{0pt}%
\pgfpathmoveto{\pgfqpoint{2.411492in}{3.472821in}}%
\pgfpathcurveto{\pgfqpoint{2.422542in}{3.472821in}}{\pgfqpoint{2.433141in}{3.477211in}}{\pgfqpoint{2.440955in}{3.485025in}}%
\pgfpathcurveto{\pgfqpoint{2.448768in}{3.492838in}}{\pgfqpoint{2.453158in}{3.503437in}}{\pgfqpoint{2.453158in}{3.514488in}}%
\pgfpathcurveto{\pgfqpoint{2.453158in}{3.525538in}}{\pgfqpoint{2.448768in}{3.536137in}}{\pgfqpoint{2.440955in}{3.543950in}}%
\pgfpathcurveto{\pgfqpoint{2.433141in}{3.551764in}}{\pgfqpoint{2.422542in}{3.556154in}}{\pgfqpoint{2.411492in}{3.556154in}}%
\pgfpathcurveto{\pgfqpoint{2.400442in}{3.556154in}}{\pgfqpoint{2.389843in}{3.551764in}}{\pgfqpoint{2.382029in}{3.543950in}}%
\pgfpathcurveto{\pgfqpoint{2.374215in}{3.536137in}}{\pgfqpoint{2.369825in}{3.525538in}}{\pgfqpoint{2.369825in}{3.514488in}}%
\pgfpathcurveto{\pgfqpoint{2.369825in}{3.503437in}}{\pgfqpoint{2.374215in}{3.492838in}}{\pgfqpoint{2.382029in}{3.485025in}}%
\pgfpathcurveto{\pgfqpoint{2.389843in}{3.477211in}}{\pgfqpoint{2.400442in}{3.472821in}}{\pgfqpoint{2.411492in}{3.472821in}}%
\pgfpathclose%
\pgfusepath{stroke,fill}%
\end{pgfscope}%
\begin{pgfscope}%
\pgfpathrectangle{\pgfqpoint{0.600000in}{0.600000in}}{\pgfqpoint{3.900000in}{3.900000in}}%
\pgfusepath{clip}%
\pgfsetbuttcap%
\pgfsetroundjoin%
\definecolor{currentfill}{rgb}{0.121569,0.466667,0.705882}%
\pgfsetfillcolor{currentfill}%
\pgfsetlinewidth{1.003750pt}%
\definecolor{currentstroke}{rgb}{0.121569,0.466667,0.705882}%
\pgfsetstrokecolor{currentstroke}%
\pgfsetdash{}{0pt}%
\pgfpathmoveto{\pgfqpoint{2.575520in}{3.318769in}}%
\pgfpathcurveto{\pgfqpoint{2.586570in}{3.318769in}}{\pgfqpoint{2.597169in}{3.323159in}}{\pgfqpoint{2.604983in}{3.330973in}}%
\pgfpathcurveto{\pgfqpoint{2.612796in}{3.338787in}}{\pgfqpoint{2.617187in}{3.349386in}}{\pgfqpoint{2.617187in}{3.360436in}}%
\pgfpathcurveto{\pgfqpoint{2.617187in}{3.371486in}}{\pgfqpoint{2.612796in}{3.382085in}}{\pgfqpoint{2.604983in}{3.389899in}}%
\pgfpathcurveto{\pgfqpoint{2.597169in}{3.397712in}}{\pgfqpoint{2.586570in}{3.402102in}}{\pgfqpoint{2.575520in}{3.402102in}}%
\pgfpathcurveto{\pgfqpoint{2.564470in}{3.402102in}}{\pgfqpoint{2.553871in}{3.397712in}}{\pgfqpoint{2.546057in}{3.389899in}}%
\pgfpathcurveto{\pgfqpoint{2.538244in}{3.382085in}}{\pgfqpoint{2.533853in}{3.371486in}}{\pgfqpoint{2.533853in}{3.360436in}}%
\pgfpathcurveto{\pgfqpoint{2.533853in}{3.349386in}}{\pgfqpoint{2.538244in}{3.338787in}}{\pgfqpoint{2.546057in}{3.330973in}}%
\pgfpathcurveto{\pgfqpoint{2.553871in}{3.323159in}}{\pgfqpoint{2.564470in}{3.318769in}}{\pgfqpoint{2.575520in}{3.318769in}}%
\pgfpathclose%
\pgfusepath{stroke,fill}%
\end{pgfscope}%
\begin{pgfscope}%
\pgfpathrectangle{\pgfqpoint{0.600000in}{0.600000in}}{\pgfqpoint{3.900000in}{3.900000in}}%
\pgfusepath{clip}%
\pgfsetbuttcap%
\pgfsetroundjoin%
\definecolor{currentfill}{rgb}{0.121569,0.466667,0.705882}%
\pgfsetfillcolor{currentfill}%
\pgfsetlinewidth{1.003750pt}%
\definecolor{currentstroke}{rgb}{0.121569,0.466667,0.705882}%
\pgfsetstrokecolor{currentstroke}%
\pgfsetdash{}{0pt}%
\pgfpathmoveto{\pgfqpoint{2.732706in}{2.494586in}}%
\pgfpathcurveto{\pgfqpoint{2.743756in}{2.494586in}}{\pgfqpoint{2.754355in}{2.498977in}}{\pgfqpoint{2.762169in}{2.506790in}}%
\pgfpathcurveto{\pgfqpoint{2.769983in}{2.514604in}}{\pgfqpoint{2.774373in}{2.525203in}}{\pgfqpoint{2.774373in}{2.536253in}}%
\pgfpathcurveto{\pgfqpoint{2.774373in}{2.547303in}}{\pgfqpoint{2.769983in}{2.557902in}}{\pgfqpoint{2.762169in}{2.565716in}}%
\pgfpathcurveto{\pgfqpoint{2.754355in}{2.573529in}}{\pgfqpoint{2.743756in}{2.577920in}}{\pgfqpoint{2.732706in}{2.577920in}}%
\pgfpathcurveto{\pgfqpoint{2.721656in}{2.577920in}}{\pgfqpoint{2.711057in}{2.573529in}}{\pgfqpoint{2.703243in}{2.565716in}}%
\pgfpathcurveto{\pgfqpoint{2.695430in}{2.557902in}}{\pgfqpoint{2.691039in}{2.547303in}}{\pgfqpoint{2.691039in}{2.536253in}}%
\pgfpathcurveto{\pgfqpoint{2.691039in}{2.525203in}}{\pgfqpoint{2.695430in}{2.514604in}}{\pgfqpoint{2.703243in}{2.506790in}}%
\pgfpathcurveto{\pgfqpoint{2.711057in}{2.498977in}}{\pgfqpoint{2.721656in}{2.494586in}}{\pgfqpoint{2.732706in}{2.494586in}}%
\pgfpathclose%
\pgfusepath{stroke,fill}%
\end{pgfscope}%
\begin{pgfscope}%
\pgfpathrectangle{\pgfqpoint{0.600000in}{0.600000in}}{\pgfqpoint{3.900000in}{3.900000in}}%
\pgfusepath{clip}%
\pgfsetbuttcap%
\pgfsetroundjoin%
\definecolor{currentfill}{rgb}{0.121569,0.466667,0.705882}%
\pgfsetfillcolor{currentfill}%
\pgfsetlinewidth{1.003750pt}%
\definecolor{currentstroke}{rgb}{0.121569,0.466667,0.705882}%
\pgfsetstrokecolor{currentstroke}%
\pgfsetdash{}{0pt}%
\pgfpathmoveto{\pgfqpoint{2.713864in}{2.883196in}}%
\pgfpathcurveto{\pgfqpoint{2.724914in}{2.883196in}}{\pgfqpoint{2.735513in}{2.887586in}}{\pgfqpoint{2.743327in}{2.895400in}}%
\pgfpathcurveto{\pgfqpoint{2.751141in}{2.903214in}}{\pgfqpoint{2.755531in}{2.913813in}}{\pgfqpoint{2.755531in}{2.924863in}}%
\pgfpathcurveto{\pgfqpoint{2.755531in}{2.935913in}}{\pgfqpoint{2.751141in}{2.946512in}}{\pgfqpoint{2.743327in}{2.954326in}}%
\pgfpathcurveto{\pgfqpoint{2.735513in}{2.962139in}}{\pgfqpoint{2.724914in}{2.966529in}}{\pgfqpoint{2.713864in}{2.966529in}}%
\pgfpathcurveto{\pgfqpoint{2.702814in}{2.966529in}}{\pgfqpoint{2.692215in}{2.962139in}}{\pgfqpoint{2.684402in}{2.954326in}}%
\pgfpathcurveto{\pgfqpoint{2.676588in}{2.946512in}}{\pgfqpoint{2.672198in}{2.935913in}}{\pgfqpoint{2.672198in}{2.924863in}}%
\pgfpathcurveto{\pgfqpoint{2.672198in}{2.913813in}}{\pgfqpoint{2.676588in}{2.903214in}}{\pgfqpoint{2.684402in}{2.895400in}}%
\pgfpathcurveto{\pgfqpoint{2.692215in}{2.887586in}}{\pgfqpoint{2.702814in}{2.883196in}}{\pgfqpoint{2.713864in}{2.883196in}}%
\pgfpathclose%
\pgfusepath{stroke,fill}%
\end{pgfscope}%
\begin{pgfscope}%
\pgfpathrectangle{\pgfqpoint{0.600000in}{0.600000in}}{\pgfqpoint{3.900000in}{3.900000in}}%
\pgfusepath{clip}%
\pgfsetbuttcap%
\pgfsetroundjoin%
\definecolor{currentfill}{rgb}{0.121569,0.466667,0.705882}%
\pgfsetfillcolor{currentfill}%
\pgfsetlinewidth{1.003750pt}%
\definecolor{currentstroke}{rgb}{0.121569,0.466667,0.705882}%
\pgfsetstrokecolor{currentstroke}%
\pgfsetdash{}{0pt}%
\pgfpathmoveto{\pgfqpoint{2.862017in}{2.731493in}}%
\pgfpathcurveto{\pgfqpoint{2.873067in}{2.731493in}}{\pgfqpoint{2.883666in}{2.735884in}}{\pgfqpoint{2.891480in}{2.743697in}}%
\pgfpathcurveto{\pgfqpoint{2.899294in}{2.751511in}}{\pgfqpoint{2.903684in}{2.762110in}}{\pgfqpoint{2.903684in}{2.773160in}}%
\pgfpathcurveto{\pgfqpoint{2.903684in}{2.784210in}}{\pgfqpoint{2.899294in}{2.794809in}}{\pgfqpoint{2.891480in}{2.802623in}}%
\pgfpathcurveto{\pgfqpoint{2.883666in}{2.810437in}}{\pgfqpoint{2.873067in}{2.814827in}}{\pgfqpoint{2.862017in}{2.814827in}}%
\pgfpathcurveto{\pgfqpoint{2.850967in}{2.814827in}}{\pgfqpoint{2.840368in}{2.810437in}}{\pgfqpoint{2.832554in}{2.802623in}}%
\pgfpathcurveto{\pgfqpoint{2.824741in}{2.794809in}}{\pgfqpoint{2.820350in}{2.784210in}}{\pgfqpoint{2.820350in}{2.773160in}}%
\pgfpathcurveto{\pgfqpoint{2.820350in}{2.762110in}}{\pgfqpoint{2.824741in}{2.751511in}}{\pgfqpoint{2.832554in}{2.743697in}}%
\pgfpathcurveto{\pgfqpoint{2.840368in}{2.735884in}}{\pgfqpoint{2.850967in}{2.731493in}}{\pgfqpoint{2.862017in}{2.731493in}}%
\pgfpathclose%
\pgfusepath{stroke,fill}%
\end{pgfscope}%
\begin{pgfscope}%
\pgfpathrectangle{\pgfqpoint{0.600000in}{0.600000in}}{\pgfqpoint{3.900000in}{3.900000in}}%
\pgfusepath{clip}%
\pgfsetbuttcap%
\pgfsetroundjoin%
\definecolor{currentfill}{rgb}{0.121569,0.466667,0.705882}%
\pgfsetfillcolor{currentfill}%
\pgfsetlinewidth{1.003750pt}%
\definecolor{currentstroke}{rgb}{0.121569,0.466667,0.705882}%
\pgfsetstrokecolor{currentstroke}%
\pgfsetdash{}{0pt}%
\pgfpathmoveto{\pgfqpoint{2.676109in}{2.267709in}}%
\pgfpathcurveto{\pgfqpoint{2.687159in}{2.267709in}}{\pgfqpoint{2.697758in}{2.272099in}}{\pgfqpoint{2.705572in}{2.279913in}}%
\pgfpathcurveto{\pgfqpoint{2.713385in}{2.287726in}}{\pgfqpoint{2.717776in}{2.298325in}}{\pgfqpoint{2.717776in}{2.309375in}}%
\pgfpathcurveto{\pgfqpoint{2.717776in}{2.320426in}}{\pgfqpoint{2.713385in}{2.331025in}}{\pgfqpoint{2.705572in}{2.338838in}}%
\pgfpathcurveto{\pgfqpoint{2.697758in}{2.346652in}}{\pgfqpoint{2.687159in}{2.351042in}}{\pgfqpoint{2.676109in}{2.351042in}}%
\pgfpathcurveto{\pgfqpoint{2.665059in}{2.351042in}}{\pgfqpoint{2.654460in}{2.346652in}}{\pgfqpoint{2.646646in}{2.338838in}}%
\pgfpathcurveto{\pgfqpoint{2.638833in}{2.331025in}}{\pgfqpoint{2.634442in}{2.320426in}}{\pgfqpoint{2.634442in}{2.309375in}}%
\pgfpathcurveto{\pgfqpoint{2.634442in}{2.298325in}}{\pgfqpoint{2.638833in}{2.287726in}}{\pgfqpoint{2.646646in}{2.279913in}}%
\pgfpathcurveto{\pgfqpoint{2.654460in}{2.272099in}}{\pgfqpoint{2.665059in}{2.267709in}}{\pgfqpoint{2.676109in}{2.267709in}}%
\pgfpathclose%
\pgfusepath{stroke,fill}%
\end{pgfscope}%
\begin{pgfscope}%
\pgfpathrectangle{\pgfqpoint{0.600000in}{0.600000in}}{\pgfqpoint{3.900000in}{3.900000in}}%
\pgfusepath{clip}%
\pgfsetbuttcap%
\pgfsetroundjoin%
\definecolor{currentfill}{rgb}{0.121569,0.466667,0.705882}%
\pgfsetfillcolor{currentfill}%
\pgfsetlinewidth{1.003750pt}%
\definecolor{currentstroke}{rgb}{0.121569,0.466667,0.705882}%
\pgfsetstrokecolor{currentstroke}%
\pgfsetdash{}{0pt}%
\pgfpathmoveto{\pgfqpoint{3.173056in}{2.246793in}}%
\pgfpathcurveto{\pgfqpoint{3.184106in}{2.246793in}}{\pgfqpoint{3.194705in}{2.251183in}}{\pgfqpoint{3.202519in}{2.258997in}}%
\pgfpathcurveto{\pgfqpoint{3.210333in}{2.266810in}}{\pgfqpoint{3.214723in}{2.277409in}}{\pgfqpoint{3.214723in}{2.288459in}}%
\pgfpathcurveto{\pgfqpoint{3.214723in}{2.299509in}}{\pgfqpoint{3.210333in}{2.310108in}}{\pgfqpoint{3.202519in}{2.317922in}}%
\pgfpathcurveto{\pgfqpoint{3.194705in}{2.325736in}}{\pgfqpoint{3.184106in}{2.330126in}}{\pgfqpoint{3.173056in}{2.330126in}}%
\pgfpathcurveto{\pgfqpoint{3.162006in}{2.330126in}}{\pgfqpoint{3.151407in}{2.325736in}}{\pgfqpoint{3.143593in}{2.317922in}}%
\pgfpathcurveto{\pgfqpoint{3.135780in}{2.310108in}}{\pgfqpoint{3.131389in}{2.299509in}}{\pgfqpoint{3.131389in}{2.288459in}}%
\pgfpathcurveto{\pgfqpoint{3.131389in}{2.277409in}}{\pgfqpoint{3.135780in}{2.266810in}}{\pgfqpoint{3.143593in}{2.258997in}}%
\pgfpathcurveto{\pgfqpoint{3.151407in}{2.251183in}}{\pgfqpoint{3.162006in}{2.246793in}}{\pgfqpoint{3.173056in}{2.246793in}}%
\pgfpathclose%
\pgfusepath{stroke,fill}%
\end{pgfscope}%
\begin{pgfscope}%
\pgfpathrectangle{\pgfqpoint{0.600000in}{0.600000in}}{\pgfqpoint{3.900000in}{3.900000in}}%
\pgfusepath{clip}%
\pgfsetbuttcap%
\pgfsetroundjoin%
\definecolor{currentfill}{rgb}{0.121569,0.466667,0.705882}%
\pgfsetfillcolor{currentfill}%
\pgfsetlinewidth{1.003750pt}%
\definecolor{currentstroke}{rgb}{0.121569,0.466667,0.705882}%
\pgfsetstrokecolor{currentstroke}%
\pgfsetdash{}{0pt}%
\pgfpathmoveto{\pgfqpoint{2.205860in}{1.412840in}}%
\pgfpathcurveto{\pgfqpoint{2.216910in}{1.412840in}}{\pgfqpoint{2.227509in}{1.417230in}}{\pgfqpoint{2.235323in}{1.425044in}}%
\pgfpathcurveto{\pgfqpoint{2.243136in}{1.432857in}}{\pgfqpoint{2.247527in}{1.443456in}}{\pgfqpoint{2.247527in}{1.454506in}}%
\pgfpathcurveto{\pgfqpoint{2.247527in}{1.465556in}}{\pgfqpoint{2.243136in}{1.476156in}}{\pgfqpoint{2.235323in}{1.483969in}}%
\pgfpathcurveto{\pgfqpoint{2.227509in}{1.491783in}}{\pgfqpoint{2.216910in}{1.496173in}}{\pgfqpoint{2.205860in}{1.496173in}}%
\pgfpathcurveto{\pgfqpoint{2.194810in}{1.496173in}}{\pgfqpoint{2.184211in}{1.491783in}}{\pgfqpoint{2.176397in}{1.483969in}}%
\pgfpathcurveto{\pgfqpoint{2.168583in}{1.476156in}}{\pgfqpoint{2.164193in}{1.465556in}}{\pgfqpoint{2.164193in}{1.454506in}}%
\pgfpathcurveto{\pgfqpoint{2.164193in}{1.443456in}}{\pgfqpoint{2.168583in}{1.432857in}}{\pgfqpoint{2.176397in}{1.425044in}}%
\pgfpathcurveto{\pgfqpoint{2.184211in}{1.417230in}}{\pgfqpoint{2.194810in}{1.412840in}}{\pgfqpoint{2.205860in}{1.412840in}}%
\pgfpathclose%
\pgfusepath{stroke,fill}%
\end{pgfscope}%
\begin{pgfscope}%
\pgfpathrectangle{\pgfqpoint{0.600000in}{0.600000in}}{\pgfqpoint{3.900000in}{3.900000in}}%
\pgfusepath{clip}%
\pgfsetbuttcap%
\pgfsetroundjoin%
\definecolor{currentfill}{rgb}{0.121569,0.466667,0.705882}%
\pgfsetfillcolor{currentfill}%
\pgfsetlinewidth{1.003750pt}%
\definecolor{currentstroke}{rgb}{0.121569,0.466667,0.705882}%
\pgfsetstrokecolor{currentstroke}%
\pgfsetdash{}{0pt}%
\pgfpathmoveto{\pgfqpoint{3.258112in}{2.757168in}}%
\pgfpathcurveto{\pgfqpoint{3.269162in}{2.757168in}}{\pgfqpoint{3.279761in}{2.761558in}}{\pgfqpoint{3.287575in}{2.769372in}}%
\pgfpathcurveto{\pgfqpoint{3.295388in}{2.777186in}}{\pgfqpoint{3.299779in}{2.787785in}}{\pgfqpoint{3.299779in}{2.798835in}}%
\pgfpathcurveto{\pgfqpoint{3.299779in}{2.809885in}}{\pgfqpoint{3.295388in}{2.820484in}}{\pgfqpoint{3.287575in}{2.828298in}}%
\pgfpathcurveto{\pgfqpoint{3.279761in}{2.836111in}}{\pgfqpoint{3.269162in}{2.840502in}}{\pgfqpoint{3.258112in}{2.840502in}}%
\pgfpathcurveto{\pgfqpoint{3.247062in}{2.840502in}}{\pgfqpoint{3.236463in}{2.836111in}}{\pgfqpoint{3.228649in}{2.828298in}}%
\pgfpathcurveto{\pgfqpoint{3.220836in}{2.820484in}}{\pgfqpoint{3.216445in}{2.809885in}}{\pgfqpoint{3.216445in}{2.798835in}}%
\pgfpathcurveto{\pgfqpoint{3.216445in}{2.787785in}}{\pgfqpoint{3.220836in}{2.777186in}}{\pgfqpoint{3.228649in}{2.769372in}}%
\pgfpathcurveto{\pgfqpoint{3.236463in}{2.761558in}}{\pgfqpoint{3.247062in}{2.757168in}}{\pgfqpoint{3.258112in}{2.757168in}}%
\pgfpathclose%
\pgfusepath{stroke,fill}%
\end{pgfscope}%
\begin{pgfscope}%
\pgfpathrectangle{\pgfqpoint{0.600000in}{0.600000in}}{\pgfqpoint{3.900000in}{3.900000in}}%
\pgfusepath{clip}%
\pgfsetbuttcap%
\pgfsetroundjoin%
\definecolor{currentfill}{rgb}{0.121569,0.466667,0.705882}%
\pgfsetfillcolor{currentfill}%
\pgfsetlinewidth{1.003750pt}%
\definecolor{currentstroke}{rgb}{0.121569,0.466667,0.705882}%
\pgfsetstrokecolor{currentstroke}%
\pgfsetdash{}{0pt}%
\pgfpathmoveto{\pgfqpoint{2.937787in}{3.508910in}}%
\pgfpathcurveto{\pgfqpoint{2.948837in}{3.508910in}}{\pgfqpoint{2.959436in}{3.513300in}}{\pgfqpoint{2.967250in}{3.521113in}}%
\pgfpathcurveto{\pgfqpoint{2.975063in}{3.528927in}}{\pgfqpoint{2.979454in}{3.539526in}}{\pgfqpoint{2.979454in}{3.550576in}}%
\pgfpathcurveto{\pgfqpoint{2.979454in}{3.561626in}}{\pgfqpoint{2.975063in}{3.572225in}}{\pgfqpoint{2.967250in}{3.580039in}}%
\pgfpathcurveto{\pgfqpoint{2.959436in}{3.587853in}}{\pgfqpoint{2.948837in}{3.592243in}}{\pgfqpoint{2.937787in}{3.592243in}}%
\pgfpathcurveto{\pgfqpoint{2.926737in}{3.592243in}}{\pgfqpoint{2.916138in}{3.587853in}}{\pgfqpoint{2.908324in}{3.580039in}}%
\pgfpathcurveto{\pgfqpoint{2.900511in}{3.572225in}}{\pgfqpoint{2.896120in}{3.561626in}}{\pgfqpoint{2.896120in}{3.550576in}}%
\pgfpathcurveto{\pgfqpoint{2.896120in}{3.539526in}}{\pgfqpoint{2.900511in}{3.528927in}}{\pgfqpoint{2.908324in}{3.521113in}}%
\pgfpathcurveto{\pgfqpoint{2.916138in}{3.513300in}}{\pgfqpoint{2.926737in}{3.508910in}}{\pgfqpoint{2.937787in}{3.508910in}}%
\pgfpathclose%
\pgfusepath{stroke,fill}%
\end{pgfscope}%
\begin{pgfscope}%
\pgfpathrectangle{\pgfqpoint{0.600000in}{0.600000in}}{\pgfqpoint{3.900000in}{3.900000in}}%
\pgfusepath{clip}%
\pgfsetbuttcap%
\pgfsetroundjoin%
\definecolor{currentfill}{rgb}{0.121569,0.466667,0.705882}%
\pgfsetfillcolor{currentfill}%
\pgfsetlinewidth{1.003750pt}%
\definecolor{currentstroke}{rgb}{0.121569,0.466667,0.705882}%
\pgfsetstrokecolor{currentstroke}%
\pgfsetdash{}{0pt}%
\pgfpathmoveto{\pgfqpoint{2.174630in}{2.527710in}}%
\pgfpathcurveto{\pgfqpoint{2.185680in}{2.527710in}}{\pgfqpoint{2.196279in}{2.532100in}}{\pgfqpoint{2.204092in}{2.539914in}}%
\pgfpathcurveto{\pgfqpoint{2.211906in}{2.547727in}}{\pgfqpoint{2.216296in}{2.558326in}}{\pgfqpoint{2.216296in}{2.569377in}}%
\pgfpathcurveto{\pgfqpoint{2.216296in}{2.580427in}}{\pgfqpoint{2.211906in}{2.591026in}}{\pgfqpoint{2.204092in}{2.598839in}}%
\pgfpathcurveto{\pgfqpoint{2.196279in}{2.606653in}}{\pgfqpoint{2.185680in}{2.611043in}}{\pgfqpoint{2.174630in}{2.611043in}}%
\pgfpathcurveto{\pgfqpoint{2.163580in}{2.611043in}}{\pgfqpoint{2.152981in}{2.606653in}}{\pgfqpoint{2.145167in}{2.598839in}}%
\pgfpathcurveto{\pgfqpoint{2.137353in}{2.591026in}}{\pgfqpoint{2.132963in}{2.580427in}}{\pgfqpoint{2.132963in}{2.569377in}}%
\pgfpathcurveto{\pgfqpoint{2.132963in}{2.558326in}}{\pgfqpoint{2.137353in}{2.547727in}}{\pgfqpoint{2.145167in}{2.539914in}}%
\pgfpathcurveto{\pgfqpoint{2.152981in}{2.532100in}}{\pgfqpoint{2.163580in}{2.527710in}}{\pgfqpoint{2.174630in}{2.527710in}}%
\pgfpathclose%
\pgfusepath{stroke,fill}%
\end{pgfscope}%
\begin{pgfscope}%
\pgfpathrectangle{\pgfqpoint{0.600000in}{0.600000in}}{\pgfqpoint{3.900000in}{3.900000in}}%
\pgfusepath{clip}%
\pgfsetbuttcap%
\pgfsetroundjoin%
\definecolor{currentfill}{rgb}{0.121569,0.466667,0.705882}%
\pgfsetfillcolor{currentfill}%
\pgfsetlinewidth{1.003750pt}%
\definecolor{currentstroke}{rgb}{0.121569,0.466667,0.705882}%
\pgfsetstrokecolor{currentstroke}%
\pgfsetdash{}{0pt}%
\pgfpathmoveto{\pgfqpoint{2.676482in}{2.382370in}}%
\pgfpathcurveto{\pgfqpoint{2.687532in}{2.382370in}}{\pgfqpoint{2.698131in}{2.386760in}}{\pgfqpoint{2.705945in}{2.394574in}}%
\pgfpathcurveto{\pgfqpoint{2.713758in}{2.402388in}}{\pgfqpoint{2.718148in}{2.412987in}}{\pgfqpoint{2.718148in}{2.424037in}}%
\pgfpathcurveto{\pgfqpoint{2.718148in}{2.435087in}}{\pgfqpoint{2.713758in}{2.445686in}}{\pgfqpoint{2.705945in}{2.453499in}}%
\pgfpathcurveto{\pgfqpoint{2.698131in}{2.461313in}}{\pgfqpoint{2.687532in}{2.465703in}}{\pgfqpoint{2.676482in}{2.465703in}}%
\pgfpathcurveto{\pgfqpoint{2.665432in}{2.465703in}}{\pgfqpoint{2.654833in}{2.461313in}}{\pgfqpoint{2.647019in}{2.453499in}}%
\pgfpathcurveto{\pgfqpoint{2.639205in}{2.445686in}}{\pgfqpoint{2.634815in}{2.435087in}}{\pgfqpoint{2.634815in}{2.424037in}}%
\pgfpathcurveto{\pgfqpoint{2.634815in}{2.412987in}}{\pgfqpoint{2.639205in}{2.402388in}}{\pgfqpoint{2.647019in}{2.394574in}}%
\pgfpathcurveto{\pgfqpoint{2.654833in}{2.386760in}}{\pgfqpoint{2.665432in}{2.382370in}}{\pgfqpoint{2.676482in}{2.382370in}}%
\pgfpathclose%
\pgfusepath{stroke,fill}%
\end{pgfscope}%
\begin{pgfscope}%
\pgfpathrectangle{\pgfqpoint{0.600000in}{0.600000in}}{\pgfqpoint{3.900000in}{3.900000in}}%
\pgfusepath{clip}%
\pgfsetbuttcap%
\pgfsetroundjoin%
\definecolor{currentfill}{rgb}{0.121569,0.466667,0.705882}%
\pgfsetfillcolor{currentfill}%
\pgfsetlinewidth{1.003750pt}%
\definecolor{currentstroke}{rgb}{0.121569,0.466667,0.705882}%
\pgfsetstrokecolor{currentstroke}%
\pgfsetdash{}{0pt}%
\pgfpathmoveto{\pgfqpoint{2.265964in}{2.194176in}}%
\pgfpathcurveto{\pgfqpoint{2.277015in}{2.194176in}}{\pgfqpoint{2.287614in}{2.198566in}}{\pgfqpoint{2.295427in}{2.206380in}}%
\pgfpathcurveto{\pgfqpoint{2.303241in}{2.214193in}}{\pgfqpoint{2.307631in}{2.224792in}}{\pgfqpoint{2.307631in}{2.235843in}}%
\pgfpathcurveto{\pgfqpoint{2.307631in}{2.246893in}}{\pgfqpoint{2.303241in}{2.257492in}}{\pgfqpoint{2.295427in}{2.265305in}}%
\pgfpathcurveto{\pgfqpoint{2.287614in}{2.273119in}}{\pgfqpoint{2.277015in}{2.277509in}}{\pgfqpoint{2.265964in}{2.277509in}}%
\pgfpathcurveto{\pgfqpoint{2.254914in}{2.277509in}}{\pgfqpoint{2.244315in}{2.273119in}}{\pgfqpoint{2.236502in}{2.265305in}}%
\pgfpathcurveto{\pgfqpoint{2.228688in}{2.257492in}}{\pgfqpoint{2.224298in}{2.246893in}}{\pgfqpoint{2.224298in}{2.235843in}}%
\pgfpathcurveto{\pgfqpoint{2.224298in}{2.224792in}}{\pgfqpoint{2.228688in}{2.214193in}}{\pgfqpoint{2.236502in}{2.206380in}}%
\pgfpathcurveto{\pgfqpoint{2.244315in}{2.198566in}}{\pgfqpoint{2.254914in}{2.194176in}}{\pgfqpoint{2.265964in}{2.194176in}}%
\pgfpathclose%
\pgfusepath{stroke,fill}%
\end{pgfscope}%
\begin{pgfscope}%
\pgfpathrectangle{\pgfqpoint{0.600000in}{0.600000in}}{\pgfqpoint{3.900000in}{3.900000in}}%
\pgfusepath{clip}%
\pgfsetbuttcap%
\pgfsetroundjoin%
\definecolor{currentfill}{rgb}{0.121569,0.466667,0.705882}%
\pgfsetfillcolor{currentfill}%
\pgfsetlinewidth{1.003750pt}%
\definecolor{currentstroke}{rgb}{0.121569,0.466667,0.705882}%
\pgfsetstrokecolor{currentstroke}%
\pgfsetdash{}{0pt}%
\pgfpathmoveto{\pgfqpoint{3.389484in}{2.859633in}}%
\pgfpathcurveto{\pgfqpoint{3.400534in}{2.859633in}}{\pgfqpoint{3.411133in}{2.864023in}}{\pgfqpoint{3.418947in}{2.871837in}}%
\pgfpathcurveto{\pgfqpoint{3.426761in}{2.879651in}}{\pgfqpoint{3.431151in}{2.890250in}}{\pgfqpoint{3.431151in}{2.901300in}}%
\pgfpathcurveto{\pgfqpoint{3.431151in}{2.912350in}}{\pgfqpoint{3.426761in}{2.922949in}}{\pgfqpoint{3.418947in}{2.930763in}}%
\pgfpathcurveto{\pgfqpoint{3.411133in}{2.938576in}}{\pgfqpoint{3.400534in}{2.942967in}}{\pgfqpoint{3.389484in}{2.942967in}}%
\pgfpathcurveto{\pgfqpoint{3.378434in}{2.942967in}}{\pgfqpoint{3.367835in}{2.938576in}}{\pgfqpoint{3.360021in}{2.930763in}}%
\pgfpathcurveto{\pgfqpoint{3.352208in}{2.922949in}}{\pgfqpoint{3.347818in}{2.912350in}}{\pgfqpoint{3.347818in}{2.901300in}}%
\pgfpathcurveto{\pgfqpoint{3.347818in}{2.890250in}}{\pgfqpoint{3.352208in}{2.879651in}}{\pgfqpoint{3.360021in}{2.871837in}}%
\pgfpathcurveto{\pgfqpoint{3.367835in}{2.864023in}}{\pgfqpoint{3.378434in}{2.859633in}}{\pgfqpoint{3.389484in}{2.859633in}}%
\pgfpathclose%
\pgfusepath{stroke,fill}%
\end{pgfscope}%
\begin{pgfscope}%
\pgfpathrectangle{\pgfqpoint{0.600000in}{0.600000in}}{\pgfqpoint{3.900000in}{3.900000in}}%
\pgfusepath{clip}%
\pgfsetbuttcap%
\pgfsetroundjoin%
\definecolor{currentfill}{rgb}{0.121569,0.466667,0.705882}%
\pgfsetfillcolor{currentfill}%
\pgfsetlinewidth{1.003750pt}%
\definecolor{currentstroke}{rgb}{0.121569,0.466667,0.705882}%
\pgfsetstrokecolor{currentstroke}%
\pgfsetdash{}{0pt}%
\pgfpathmoveto{\pgfqpoint{2.917358in}{3.063690in}}%
\pgfpathcurveto{\pgfqpoint{2.928408in}{3.063690in}}{\pgfqpoint{2.939007in}{3.068080in}}{\pgfqpoint{2.946821in}{3.075894in}}%
\pgfpathcurveto{\pgfqpoint{2.954634in}{3.083708in}}{\pgfqpoint{2.959025in}{3.094307in}}{\pgfqpoint{2.959025in}{3.105357in}}%
\pgfpathcurveto{\pgfqpoint{2.959025in}{3.116407in}}{\pgfqpoint{2.954634in}{3.127006in}}{\pgfqpoint{2.946821in}{3.134819in}}%
\pgfpathcurveto{\pgfqpoint{2.939007in}{3.142633in}}{\pgfqpoint{2.928408in}{3.147023in}}{\pgfqpoint{2.917358in}{3.147023in}}%
\pgfpathcurveto{\pgfqpoint{2.906308in}{3.147023in}}{\pgfqpoint{2.895709in}{3.142633in}}{\pgfqpoint{2.887895in}{3.134819in}}%
\pgfpathcurveto{\pgfqpoint{2.880082in}{3.127006in}}{\pgfqpoint{2.875691in}{3.116407in}}{\pgfqpoint{2.875691in}{3.105357in}}%
\pgfpathcurveto{\pgfqpoint{2.875691in}{3.094307in}}{\pgfqpoint{2.880082in}{3.083708in}}{\pgfqpoint{2.887895in}{3.075894in}}%
\pgfpathcurveto{\pgfqpoint{2.895709in}{3.068080in}}{\pgfqpoint{2.906308in}{3.063690in}}{\pgfqpoint{2.917358in}{3.063690in}}%
\pgfpathclose%
\pgfusepath{stroke,fill}%
\end{pgfscope}%
\begin{pgfscope}%
\pgfpathrectangle{\pgfqpoint{0.600000in}{0.600000in}}{\pgfqpoint{3.900000in}{3.900000in}}%
\pgfusepath{clip}%
\pgfsetbuttcap%
\pgfsetroundjoin%
\definecolor{currentfill}{rgb}{0.121569,0.466667,0.705882}%
\pgfsetfillcolor{currentfill}%
\pgfsetlinewidth{1.003750pt}%
\definecolor{currentstroke}{rgb}{0.121569,0.466667,0.705882}%
\pgfsetstrokecolor{currentstroke}%
\pgfsetdash{}{0pt}%
\pgfpathmoveto{\pgfqpoint{2.194438in}{2.133738in}}%
\pgfpathcurveto{\pgfqpoint{2.205488in}{2.133738in}}{\pgfqpoint{2.216087in}{2.138129in}}{\pgfqpoint{2.223901in}{2.145942in}}%
\pgfpathcurveto{\pgfqpoint{2.231715in}{2.153756in}}{\pgfqpoint{2.236105in}{2.164355in}}{\pgfqpoint{2.236105in}{2.175405in}}%
\pgfpathcurveto{\pgfqpoint{2.236105in}{2.186455in}}{\pgfqpoint{2.231715in}{2.197054in}}{\pgfqpoint{2.223901in}{2.204868in}}%
\pgfpathcurveto{\pgfqpoint{2.216087in}{2.212681in}}{\pgfqpoint{2.205488in}{2.217072in}}{\pgfqpoint{2.194438in}{2.217072in}}%
\pgfpathcurveto{\pgfqpoint{2.183388in}{2.217072in}}{\pgfqpoint{2.172789in}{2.212681in}}{\pgfqpoint{2.164975in}{2.204868in}}%
\pgfpathcurveto{\pgfqpoint{2.157162in}{2.197054in}}{\pgfqpoint{2.152772in}{2.186455in}}{\pgfqpoint{2.152772in}{2.175405in}}%
\pgfpathcurveto{\pgfqpoint{2.152772in}{2.164355in}}{\pgfqpoint{2.157162in}{2.153756in}}{\pgfqpoint{2.164975in}{2.145942in}}%
\pgfpathcurveto{\pgfqpoint{2.172789in}{2.138129in}}{\pgfqpoint{2.183388in}{2.133738in}}{\pgfqpoint{2.194438in}{2.133738in}}%
\pgfpathclose%
\pgfusepath{stroke,fill}%
\end{pgfscope}%
\begin{pgfscope}%
\pgfpathrectangle{\pgfqpoint{0.600000in}{0.600000in}}{\pgfqpoint{3.900000in}{3.900000in}}%
\pgfusepath{clip}%
\pgfsetbuttcap%
\pgfsetroundjoin%
\definecolor{currentfill}{rgb}{0.121569,0.466667,0.705882}%
\pgfsetfillcolor{currentfill}%
\pgfsetlinewidth{1.003750pt}%
\definecolor{currentstroke}{rgb}{0.121569,0.466667,0.705882}%
\pgfsetstrokecolor{currentstroke}%
\pgfsetdash{}{0pt}%
\pgfpathmoveto{\pgfqpoint{2.324069in}{3.083420in}}%
\pgfpathcurveto{\pgfqpoint{2.335119in}{3.083420in}}{\pgfqpoint{2.345718in}{3.087810in}}{\pgfqpoint{2.353532in}{3.095624in}}%
\pgfpathcurveto{\pgfqpoint{2.361346in}{3.103438in}}{\pgfqpoint{2.365736in}{3.114037in}}{\pgfqpoint{2.365736in}{3.125087in}}%
\pgfpathcurveto{\pgfqpoint{2.365736in}{3.136137in}}{\pgfqpoint{2.361346in}{3.146736in}}{\pgfqpoint{2.353532in}{3.154550in}}%
\pgfpathcurveto{\pgfqpoint{2.345718in}{3.162363in}}{\pgfqpoint{2.335119in}{3.166754in}}{\pgfqpoint{2.324069in}{3.166754in}}%
\pgfpathcurveto{\pgfqpoint{2.313019in}{3.166754in}}{\pgfqpoint{2.302420in}{3.162363in}}{\pgfqpoint{2.294606in}{3.154550in}}%
\pgfpathcurveto{\pgfqpoint{2.286793in}{3.146736in}}{\pgfqpoint{2.282403in}{3.136137in}}{\pgfqpoint{2.282403in}{3.125087in}}%
\pgfpathcurveto{\pgfqpoint{2.282403in}{3.114037in}}{\pgfqpoint{2.286793in}{3.103438in}}{\pgfqpoint{2.294606in}{3.095624in}}%
\pgfpathcurveto{\pgfqpoint{2.302420in}{3.087810in}}{\pgfqpoint{2.313019in}{3.083420in}}{\pgfqpoint{2.324069in}{3.083420in}}%
\pgfpathclose%
\pgfusepath{stroke,fill}%
\end{pgfscope}%
\begin{pgfscope}%
\pgfpathrectangle{\pgfqpoint{0.600000in}{0.600000in}}{\pgfqpoint{3.900000in}{3.900000in}}%
\pgfusepath{clip}%
\pgfsetbuttcap%
\pgfsetroundjoin%
\definecolor{currentfill}{rgb}{0.121569,0.466667,0.705882}%
\pgfsetfillcolor{currentfill}%
\pgfsetlinewidth{1.003750pt}%
\definecolor{currentstroke}{rgb}{0.121569,0.466667,0.705882}%
\pgfsetstrokecolor{currentstroke}%
\pgfsetdash{}{0pt}%
\pgfpathmoveto{\pgfqpoint{1.980778in}{2.697735in}}%
\pgfpathcurveto{\pgfqpoint{1.991828in}{2.697735in}}{\pgfqpoint{2.002427in}{2.702125in}}{\pgfqpoint{2.010241in}{2.709938in}}%
\pgfpathcurveto{\pgfqpoint{2.018054in}{2.717752in}}{\pgfqpoint{2.022445in}{2.728351in}}{\pgfqpoint{2.022445in}{2.739401in}}%
\pgfpathcurveto{\pgfqpoint{2.022445in}{2.750451in}}{\pgfqpoint{2.018054in}{2.761050in}}{\pgfqpoint{2.010241in}{2.768864in}}%
\pgfpathcurveto{\pgfqpoint{2.002427in}{2.776678in}}{\pgfqpoint{1.991828in}{2.781068in}}{\pgfqpoint{1.980778in}{2.781068in}}%
\pgfpathcurveto{\pgfqpoint{1.969728in}{2.781068in}}{\pgfqpoint{1.959129in}{2.776678in}}{\pgfqpoint{1.951315in}{2.768864in}}%
\pgfpathcurveto{\pgfqpoint{1.943502in}{2.761050in}}{\pgfqpoint{1.939111in}{2.750451in}}{\pgfqpoint{1.939111in}{2.739401in}}%
\pgfpathcurveto{\pgfqpoint{1.939111in}{2.728351in}}{\pgfqpoint{1.943502in}{2.717752in}}{\pgfqpoint{1.951315in}{2.709938in}}%
\pgfpathcurveto{\pgfqpoint{1.959129in}{2.702125in}}{\pgfqpoint{1.969728in}{2.697735in}}{\pgfqpoint{1.980778in}{2.697735in}}%
\pgfpathclose%
\pgfusepath{stroke,fill}%
\end{pgfscope}%
\begin{pgfscope}%
\pgfpathrectangle{\pgfqpoint{0.600000in}{0.600000in}}{\pgfqpoint{3.900000in}{3.900000in}}%
\pgfusepath{clip}%
\pgfsetbuttcap%
\pgfsetroundjoin%
\definecolor{currentfill}{rgb}{0.121569,0.466667,0.705882}%
\pgfsetfillcolor{currentfill}%
\pgfsetlinewidth{1.003750pt}%
\definecolor{currentstroke}{rgb}{0.121569,0.466667,0.705882}%
\pgfsetstrokecolor{currentstroke}%
\pgfsetdash{}{0pt}%
\pgfpathmoveto{\pgfqpoint{2.026997in}{2.419048in}}%
\pgfpathcurveto{\pgfqpoint{2.038048in}{2.419048in}}{\pgfqpoint{2.048647in}{2.423438in}}{\pgfqpoint{2.056460in}{2.431252in}}%
\pgfpathcurveto{\pgfqpoint{2.064274in}{2.439066in}}{\pgfqpoint{2.068664in}{2.449665in}}{\pgfqpoint{2.068664in}{2.460715in}}%
\pgfpathcurveto{\pgfqpoint{2.068664in}{2.471765in}}{\pgfqpoint{2.064274in}{2.482364in}}{\pgfqpoint{2.056460in}{2.490177in}}%
\pgfpathcurveto{\pgfqpoint{2.048647in}{2.497991in}}{\pgfqpoint{2.038048in}{2.502381in}}{\pgfqpoint{2.026997in}{2.502381in}}%
\pgfpathcurveto{\pgfqpoint{2.015947in}{2.502381in}}{\pgfqpoint{2.005348in}{2.497991in}}{\pgfqpoint{1.997535in}{2.490177in}}%
\pgfpathcurveto{\pgfqpoint{1.989721in}{2.482364in}}{\pgfqpoint{1.985331in}{2.471765in}}{\pgfqpoint{1.985331in}{2.460715in}}%
\pgfpathcurveto{\pgfqpoint{1.985331in}{2.449665in}}{\pgfqpoint{1.989721in}{2.439066in}}{\pgfqpoint{1.997535in}{2.431252in}}%
\pgfpathcurveto{\pgfqpoint{2.005348in}{2.423438in}}{\pgfqpoint{2.015947in}{2.419048in}}{\pgfqpoint{2.026997in}{2.419048in}}%
\pgfpathclose%
\pgfusepath{stroke,fill}%
\end{pgfscope}%
\begin{pgfscope}%
\pgfpathrectangle{\pgfqpoint{0.600000in}{0.600000in}}{\pgfqpoint{3.900000in}{3.900000in}}%
\pgfusepath{clip}%
\pgfsetbuttcap%
\pgfsetroundjoin%
\definecolor{currentfill}{rgb}{0.121569,0.466667,0.705882}%
\pgfsetfillcolor{currentfill}%
\pgfsetlinewidth{1.003750pt}%
\definecolor{currentstroke}{rgb}{0.121569,0.466667,0.705882}%
\pgfsetstrokecolor{currentstroke}%
\pgfsetdash{}{0pt}%
\pgfpathmoveto{\pgfqpoint{2.510542in}{2.509793in}}%
\pgfpathcurveto{\pgfqpoint{2.521592in}{2.509793in}}{\pgfqpoint{2.532191in}{2.514183in}}{\pgfqpoint{2.540005in}{2.521997in}}%
\pgfpathcurveto{\pgfqpoint{2.547819in}{2.529811in}}{\pgfqpoint{2.552209in}{2.540410in}}{\pgfqpoint{2.552209in}{2.551460in}}%
\pgfpathcurveto{\pgfqpoint{2.552209in}{2.562510in}}{\pgfqpoint{2.547819in}{2.573109in}}{\pgfqpoint{2.540005in}{2.580923in}}%
\pgfpathcurveto{\pgfqpoint{2.532191in}{2.588736in}}{\pgfqpoint{2.521592in}{2.593126in}}{\pgfqpoint{2.510542in}{2.593126in}}%
\pgfpathcurveto{\pgfqpoint{2.499492in}{2.593126in}}{\pgfqpoint{2.488893in}{2.588736in}}{\pgfqpoint{2.481079in}{2.580923in}}%
\pgfpathcurveto{\pgfqpoint{2.473266in}{2.573109in}}{\pgfqpoint{2.468876in}{2.562510in}}{\pgfqpoint{2.468876in}{2.551460in}}%
\pgfpathcurveto{\pgfqpoint{2.468876in}{2.540410in}}{\pgfqpoint{2.473266in}{2.529811in}}{\pgfqpoint{2.481079in}{2.521997in}}%
\pgfpathcurveto{\pgfqpoint{2.488893in}{2.514183in}}{\pgfqpoint{2.499492in}{2.509793in}}{\pgfqpoint{2.510542in}{2.509793in}}%
\pgfpathclose%
\pgfusepath{stroke,fill}%
\end{pgfscope}%
\begin{pgfscope}%
\pgfpathrectangle{\pgfqpoint{0.600000in}{0.600000in}}{\pgfqpoint{3.900000in}{3.900000in}}%
\pgfusepath{clip}%
\pgfsetbuttcap%
\pgfsetroundjoin%
\definecolor{currentfill}{rgb}{0.121569,0.466667,0.705882}%
\pgfsetfillcolor{currentfill}%
\pgfsetlinewidth{1.003750pt}%
\definecolor{currentstroke}{rgb}{0.121569,0.466667,0.705882}%
\pgfsetstrokecolor{currentstroke}%
\pgfsetdash{}{0pt}%
\pgfpathmoveto{\pgfqpoint{2.347178in}{3.491787in}}%
\pgfpathcurveto{\pgfqpoint{2.358229in}{3.491787in}}{\pgfqpoint{2.368828in}{3.496177in}}{\pgfqpoint{2.376641in}{3.503990in}}%
\pgfpathcurveto{\pgfqpoint{2.384455in}{3.511804in}}{\pgfqpoint{2.388845in}{3.522403in}}{\pgfqpoint{2.388845in}{3.533453in}}%
\pgfpathcurveto{\pgfqpoint{2.388845in}{3.544503in}}{\pgfqpoint{2.384455in}{3.555102in}}{\pgfqpoint{2.376641in}{3.562916in}}%
\pgfpathcurveto{\pgfqpoint{2.368828in}{3.570730in}}{\pgfqpoint{2.358229in}{3.575120in}}{\pgfqpoint{2.347178in}{3.575120in}}%
\pgfpathcurveto{\pgfqpoint{2.336128in}{3.575120in}}{\pgfqpoint{2.325529in}{3.570730in}}{\pgfqpoint{2.317716in}{3.562916in}}%
\pgfpathcurveto{\pgfqpoint{2.309902in}{3.555102in}}{\pgfqpoint{2.305512in}{3.544503in}}{\pgfqpoint{2.305512in}{3.533453in}}%
\pgfpathcurveto{\pgfqpoint{2.305512in}{3.522403in}}{\pgfqpoint{2.309902in}{3.511804in}}{\pgfqpoint{2.317716in}{3.503990in}}%
\pgfpathcurveto{\pgfqpoint{2.325529in}{3.496177in}}{\pgfqpoint{2.336128in}{3.491787in}}{\pgfqpoint{2.347178in}{3.491787in}}%
\pgfpathclose%
\pgfusepath{stroke,fill}%
\end{pgfscope}%
\begin{pgfscope}%
\pgfpathrectangle{\pgfqpoint{0.600000in}{0.600000in}}{\pgfqpoint{3.900000in}{3.900000in}}%
\pgfusepath{clip}%
\pgfsetbuttcap%
\pgfsetroundjoin%
\definecolor{currentfill}{rgb}{0.121569,0.466667,0.705882}%
\pgfsetfillcolor{currentfill}%
\pgfsetlinewidth{1.003750pt}%
\definecolor{currentstroke}{rgb}{0.121569,0.466667,0.705882}%
\pgfsetstrokecolor{currentstroke}%
\pgfsetdash{}{0pt}%
\pgfpathmoveto{\pgfqpoint{2.332013in}{2.464084in}}%
\pgfpathcurveto{\pgfqpoint{2.343063in}{2.464084in}}{\pgfqpoint{2.353662in}{2.468474in}}{\pgfqpoint{2.361475in}{2.476288in}}%
\pgfpathcurveto{\pgfqpoint{2.369289in}{2.484102in}}{\pgfqpoint{2.373679in}{2.494701in}}{\pgfqpoint{2.373679in}{2.505751in}}%
\pgfpathcurveto{\pgfqpoint{2.373679in}{2.516801in}}{\pgfqpoint{2.369289in}{2.527400in}}{\pgfqpoint{2.361475in}{2.535214in}}%
\pgfpathcurveto{\pgfqpoint{2.353662in}{2.543027in}}{\pgfqpoint{2.343063in}{2.547418in}}{\pgfqpoint{2.332013in}{2.547418in}}%
\pgfpathcurveto{\pgfqpoint{2.320962in}{2.547418in}}{\pgfqpoint{2.310363in}{2.543027in}}{\pgfqpoint{2.302550in}{2.535214in}}%
\pgfpathcurveto{\pgfqpoint{2.294736in}{2.527400in}}{\pgfqpoint{2.290346in}{2.516801in}}{\pgfqpoint{2.290346in}{2.505751in}}%
\pgfpathcurveto{\pgfqpoint{2.290346in}{2.494701in}}{\pgfqpoint{2.294736in}{2.484102in}}{\pgfqpoint{2.302550in}{2.476288in}}%
\pgfpathcurveto{\pgfqpoint{2.310363in}{2.468474in}}{\pgfqpoint{2.320962in}{2.464084in}}{\pgfqpoint{2.332013in}{2.464084in}}%
\pgfpathclose%
\pgfusepath{stroke,fill}%
\end{pgfscope}%
\begin{pgfscope}%
\pgfpathrectangle{\pgfqpoint{0.600000in}{0.600000in}}{\pgfqpoint{3.900000in}{3.900000in}}%
\pgfusepath{clip}%
\pgfsetbuttcap%
\pgfsetroundjoin%
\definecolor{currentfill}{rgb}{0.121569,0.466667,0.705882}%
\pgfsetfillcolor{currentfill}%
\pgfsetlinewidth{1.003750pt}%
\definecolor{currentstroke}{rgb}{0.121569,0.466667,0.705882}%
\pgfsetstrokecolor{currentstroke}%
\pgfsetdash{}{0pt}%
\pgfpathmoveto{\pgfqpoint{2.566145in}{1.579019in}}%
\pgfpathcurveto{\pgfqpoint{2.577195in}{1.579019in}}{\pgfqpoint{2.587794in}{1.583409in}}{\pgfqpoint{2.595608in}{1.591223in}}%
\pgfpathcurveto{\pgfqpoint{2.603421in}{1.599036in}}{\pgfqpoint{2.607812in}{1.609635in}}{\pgfqpoint{2.607812in}{1.620685in}}%
\pgfpathcurveto{\pgfqpoint{2.607812in}{1.631736in}}{\pgfqpoint{2.603421in}{1.642335in}}{\pgfqpoint{2.595608in}{1.650148in}}%
\pgfpathcurveto{\pgfqpoint{2.587794in}{1.657962in}}{\pgfqpoint{2.577195in}{1.662352in}}{\pgfqpoint{2.566145in}{1.662352in}}%
\pgfpathcurveto{\pgfqpoint{2.555095in}{1.662352in}}{\pgfqpoint{2.544496in}{1.657962in}}{\pgfqpoint{2.536682in}{1.650148in}}%
\pgfpathcurveto{\pgfqpoint{2.528869in}{1.642335in}}{\pgfqpoint{2.524478in}{1.631736in}}{\pgfqpoint{2.524478in}{1.620685in}}%
\pgfpathcurveto{\pgfqpoint{2.524478in}{1.609635in}}{\pgfqpoint{2.528869in}{1.599036in}}{\pgfqpoint{2.536682in}{1.591223in}}%
\pgfpathcurveto{\pgfqpoint{2.544496in}{1.583409in}}{\pgfqpoint{2.555095in}{1.579019in}}{\pgfqpoint{2.566145in}{1.579019in}}%
\pgfpathclose%
\pgfusepath{stroke,fill}%
\end{pgfscope}%
\begin{pgfscope}%
\pgfpathrectangle{\pgfqpoint{0.600000in}{0.600000in}}{\pgfqpoint{3.900000in}{3.900000in}}%
\pgfusepath{clip}%
\pgfsetbuttcap%
\pgfsetroundjoin%
\definecolor{currentfill}{rgb}{0.121569,0.466667,0.705882}%
\pgfsetfillcolor{currentfill}%
\pgfsetlinewidth{1.003750pt}%
\definecolor{currentstroke}{rgb}{0.121569,0.466667,0.705882}%
\pgfsetstrokecolor{currentstroke}%
\pgfsetdash{}{0pt}%
\pgfpathmoveto{\pgfqpoint{2.478698in}{2.756626in}}%
\pgfpathcurveto{\pgfqpoint{2.489748in}{2.756626in}}{\pgfqpoint{2.500347in}{2.761016in}}{\pgfqpoint{2.508161in}{2.768830in}}%
\pgfpathcurveto{\pgfqpoint{2.515974in}{2.776643in}}{\pgfqpoint{2.520364in}{2.787242in}}{\pgfqpoint{2.520364in}{2.798292in}}%
\pgfpathcurveto{\pgfqpoint{2.520364in}{2.809343in}}{\pgfqpoint{2.515974in}{2.819942in}}{\pgfqpoint{2.508161in}{2.827755in}}%
\pgfpathcurveto{\pgfqpoint{2.500347in}{2.835569in}}{\pgfqpoint{2.489748in}{2.839959in}}{\pgfqpoint{2.478698in}{2.839959in}}%
\pgfpathcurveto{\pgfqpoint{2.467648in}{2.839959in}}{\pgfqpoint{2.457049in}{2.835569in}}{\pgfqpoint{2.449235in}{2.827755in}}%
\pgfpathcurveto{\pgfqpoint{2.441421in}{2.819942in}}{\pgfqpoint{2.437031in}{2.809343in}}{\pgfqpoint{2.437031in}{2.798292in}}%
\pgfpathcurveto{\pgfqpoint{2.437031in}{2.787242in}}{\pgfqpoint{2.441421in}{2.776643in}}{\pgfqpoint{2.449235in}{2.768830in}}%
\pgfpathcurveto{\pgfqpoint{2.457049in}{2.761016in}}{\pgfqpoint{2.467648in}{2.756626in}}{\pgfqpoint{2.478698in}{2.756626in}}%
\pgfpathclose%
\pgfusepath{stroke,fill}%
\end{pgfscope}%
\begin{pgfscope}%
\pgfpathrectangle{\pgfqpoint{0.600000in}{0.600000in}}{\pgfqpoint{3.900000in}{3.900000in}}%
\pgfusepath{clip}%
\pgfsetbuttcap%
\pgfsetroundjoin%
\definecolor{currentfill}{rgb}{0.121569,0.466667,0.705882}%
\pgfsetfillcolor{currentfill}%
\pgfsetlinewidth{1.003750pt}%
\definecolor{currentstroke}{rgb}{0.121569,0.466667,0.705882}%
\pgfsetstrokecolor{currentstroke}%
\pgfsetdash{}{0pt}%
\pgfpathmoveto{\pgfqpoint{2.176512in}{2.610620in}}%
\pgfpathcurveto{\pgfqpoint{2.187562in}{2.610620in}}{\pgfqpoint{2.198161in}{2.615011in}}{\pgfqpoint{2.205975in}{2.622824in}}%
\pgfpathcurveto{\pgfqpoint{2.213788in}{2.630638in}}{\pgfqpoint{2.218179in}{2.641237in}}{\pgfqpoint{2.218179in}{2.652287in}}%
\pgfpathcurveto{\pgfqpoint{2.218179in}{2.663337in}}{\pgfqpoint{2.213788in}{2.673936in}}{\pgfqpoint{2.205975in}{2.681750in}}%
\pgfpathcurveto{\pgfqpoint{2.198161in}{2.689564in}}{\pgfqpoint{2.187562in}{2.693954in}}{\pgfqpoint{2.176512in}{2.693954in}}%
\pgfpathcurveto{\pgfqpoint{2.165462in}{2.693954in}}{\pgfqpoint{2.154863in}{2.689564in}}{\pgfqpoint{2.147049in}{2.681750in}}%
\pgfpathcurveto{\pgfqpoint{2.139236in}{2.673936in}}{\pgfqpoint{2.134845in}{2.663337in}}{\pgfqpoint{2.134845in}{2.652287in}}%
\pgfpathcurveto{\pgfqpoint{2.134845in}{2.641237in}}{\pgfqpoint{2.139236in}{2.630638in}}{\pgfqpoint{2.147049in}{2.622824in}}%
\pgfpathcurveto{\pgfqpoint{2.154863in}{2.615011in}}{\pgfqpoint{2.165462in}{2.610620in}}{\pgfqpoint{2.176512in}{2.610620in}}%
\pgfpathclose%
\pgfusepath{stroke,fill}%
\end{pgfscope}%
\begin{pgfscope}%
\pgfpathrectangle{\pgfqpoint{0.600000in}{0.600000in}}{\pgfqpoint{3.900000in}{3.900000in}}%
\pgfusepath{clip}%
\pgfsetbuttcap%
\pgfsetroundjoin%
\definecolor{currentfill}{rgb}{0.121569,0.466667,0.705882}%
\pgfsetfillcolor{currentfill}%
\pgfsetlinewidth{1.003750pt}%
\definecolor{currentstroke}{rgb}{0.121569,0.466667,0.705882}%
\pgfsetstrokecolor{currentstroke}%
\pgfsetdash{}{0pt}%
\pgfpathmoveto{\pgfqpoint{3.151215in}{2.213233in}}%
\pgfpathcurveto{\pgfqpoint{3.162265in}{2.213233in}}{\pgfqpoint{3.172864in}{2.217623in}}{\pgfqpoint{3.180678in}{2.225437in}}%
\pgfpathcurveto{\pgfqpoint{3.188491in}{2.233250in}}{\pgfqpoint{3.192882in}{2.243849in}}{\pgfqpoint{3.192882in}{2.254899in}}%
\pgfpathcurveto{\pgfqpoint{3.192882in}{2.265949in}}{\pgfqpoint{3.188491in}{2.276548in}}{\pgfqpoint{3.180678in}{2.284362in}}%
\pgfpathcurveto{\pgfqpoint{3.172864in}{2.292176in}}{\pgfqpoint{3.162265in}{2.296566in}}{\pgfqpoint{3.151215in}{2.296566in}}%
\pgfpathcurveto{\pgfqpoint{3.140165in}{2.296566in}}{\pgfqpoint{3.129566in}{2.292176in}}{\pgfqpoint{3.121752in}{2.284362in}}%
\pgfpathcurveto{\pgfqpoint{3.113938in}{2.276548in}}{\pgfqpoint{3.109548in}{2.265949in}}{\pgfqpoint{3.109548in}{2.254899in}}%
\pgfpathcurveto{\pgfqpoint{3.109548in}{2.243849in}}{\pgfqpoint{3.113938in}{2.233250in}}{\pgfqpoint{3.121752in}{2.225437in}}%
\pgfpathcurveto{\pgfqpoint{3.129566in}{2.217623in}}{\pgfqpoint{3.140165in}{2.213233in}}{\pgfqpoint{3.151215in}{2.213233in}}%
\pgfpathclose%
\pgfusepath{stroke,fill}%
\end{pgfscope}%
\begin{pgfscope}%
\pgfpathrectangle{\pgfqpoint{0.600000in}{0.600000in}}{\pgfqpoint{3.900000in}{3.900000in}}%
\pgfusepath{clip}%
\pgfsetbuttcap%
\pgfsetroundjoin%
\definecolor{currentfill}{rgb}{0.121569,0.466667,0.705882}%
\pgfsetfillcolor{currentfill}%
\pgfsetlinewidth{1.003750pt}%
\definecolor{currentstroke}{rgb}{0.121569,0.466667,0.705882}%
\pgfsetstrokecolor{currentstroke}%
\pgfsetdash{}{0pt}%
\pgfpathmoveto{\pgfqpoint{2.371308in}{3.234702in}}%
\pgfpathcurveto{\pgfqpoint{2.382359in}{3.234702in}}{\pgfqpoint{2.392958in}{3.239093in}}{\pgfqpoint{2.400771in}{3.246906in}}%
\pgfpathcurveto{\pgfqpoint{2.408585in}{3.254720in}}{\pgfqpoint{2.412975in}{3.265319in}}{\pgfqpoint{2.412975in}{3.276369in}}%
\pgfpathcurveto{\pgfqpoint{2.412975in}{3.287419in}}{\pgfqpoint{2.408585in}{3.298018in}}{\pgfqpoint{2.400771in}{3.305832in}}%
\pgfpathcurveto{\pgfqpoint{2.392958in}{3.313645in}}{\pgfqpoint{2.382359in}{3.318036in}}{\pgfqpoint{2.371308in}{3.318036in}}%
\pgfpathcurveto{\pgfqpoint{2.360258in}{3.318036in}}{\pgfqpoint{2.349659in}{3.313645in}}{\pgfqpoint{2.341846in}{3.305832in}}%
\pgfpathcurveto{\pgfqpoint{2.334032in}{3.298018in}}{\pgfqpoint{2.329642in}{3.287419in}}{\pgfqpoint{2.329642in}{3.276369in}}%
\pgfpathcurveto{\pgfqpoint{2.329642in}{3.265319in}}{\pgfqpoint{2.334032in}{3.254720in}}{\pgfqpoint{2.341846in}{3.246906in}}%
\pgfpathcurveto{\pgfqpoint{2.349659in}{3.239093in}}{\pgfqpoint{2.360258in}{3.234702in}}{\pgfqpoint{2.371308in}{3.234702in}}%
\pgfpathclose%
\pgfusepath{stroke,fill}%
\end{pgfscope}%
\begin{pgfscope}%
\pgfpathrectangle{\pgfqpoint{0.600000in}{0.600000in}}{\pgfqpoint{3.900000in}{3.900000in}}%
\pgfusepath{clip}%
\pgfsetbuttcap%
\pgfsetroundjoin%
\definecolor{currentfill}{rgb}{0.121569,0.466667,0.705882}%
\pgfsetfillcolor{currentfill}%
\pgfsetlinewidth{1.003750pt}%
\definecolor{currentstroke}{rgb}{0.121569,0.466667,0.705882}%
\pgfsetstrokecolor{currentstroke}%
\pgfsetdash{}{0pt}%
\pgfpathmoveto{\pgfqpoint{2.024668in}{2.554704in}}%
\pgfpathcurveto{\pgfqpoint{2.035718in}{2.554704in}}{\pgfqpoint{2.046317in}{2.559094in}}{\pgfqpoint{2.054130in}{2.566908in}}%
\pgfpathcurveto{\pgfqpoint{2.061944in}{2.574722in}}{\pgfqpoint{2.066334in}{2.585321in}}{\pgfqpoint{2.066334in}{2.596371in}}%
\pgfpathcurveto{\pgfqpoint{2.066334in}{2.607421in}}{\pgfqpoint{2.061944in}{2.618020in}}{\pgfqpoint{2.054130in}{2.625833in}}%
\pgfpathcurveto{\pgfqpoint{2.046317in}{2.633647in}}{\pgfqpoint{2.035718in}{2.638037in}}{\pgfqpoint{2.024668in}{2.638037in}}%
\pgfpathcurveto{\pgfqpoint{2.013617in}{2.638037in}}{\pgfqpoint{2.003018in}{2.633647in}}{\pgfqpoint{1.995205in}{2.625833in}}%
\pgfpathcurveto{\pgfqpoint{1.987391in}{2.618020in}}{\pgfqpoint{1.983001in}{2.607421in}}{\pgfqpoint{1.983001in}{2.596371in}}%
\pgfpathcurveto{\pgfqpoint{1.983001in}{2.585321in}}{\pgfqpoint{1.987391in}{2.574722in}}{\pgfqpoint{1.995205in}{2.566908in}}%
\pgfpathcurveto{\pgfqpoint{2.003018in}{2.559094in}}{\pgfqpoint{2.013617in}{2.554704in}}{\pgfqpoint{2.024668in}{2.554704in}}%
\pgfpathclose%
\pgfusepath{stroke,fill}%
\end{pgfscope}%
\begin{pgfscope}%
\pgfpathrectangle{\pgfqpoint{0.600000in}{0.600000in}}{\pgfqpoint{3.900000in}{3.900000in}}%
\pgfusepath{clip}%
\pgfsetbuttcap%
\pgfsetroundjoin%
\definecolor{currentfill}{rgb}{0.121569,0.466667,0.705882}%
\pgfsetfillcolor{currentfill}%
\pgfsetlinewidth{1.003750pt}%
\definecolor{currentstroke}{rgb}{0.121569,0.466667,0.705882}%
\pgfsetstrokecolor{currentstroke}%
\pgfsetdash{}{0pt}%
\pgfpathmoveto{\pgfqpoint{1.843518in}{2.828345in}}%
\pgfpathcurveto{\pgfqpoint{1.854568in}{2.828345in}}{\pgfqpoint{1.865167in}{2.832736in}}{\pgfqpoint{1.872980in}{2.840549in}}%
\pgfpathcurveto{\pgfqpoint{1.880794in}{2.848363in}}{\pgfqpoint{1.885184in}{2.858962in}}{\pgfqpoint{1.885184in}{2.870012in}}%
\pgfpathcurveto{\pgfqpoint{1.885184in}{2.881062in}}{\pgfqpoint{1.880794in}{2.891661in}}{\pgfqpoint{1.872980in}{2.899475in}}%
\pgfpathcurveto{\pgfqpoint{1.865167in}{2.907288in}}{\pgfqpoint{1.854568in}{2.911679in}}{\pgfqpoint{1.843518in}{2.911679in}}%
\pgfpathcurveto{\pgfqpoint{1.832467in}{2.911679in}}{\pgfqpoint{1.821868in}{2.907288in}}{\pgfqpoint{1.814055in}{2.899475in}}%
\pgfpathcurveto{\pgfqpoint{1.806241in}{2.891661in}}{\pgfqpoint{1.801851in}{2.881062in}}{\pgfqpoint{1.801851in}{2.870012in}}%
\pgfpathcurveto{\pgfqpoint{1.801851in}{2.858962in}}{\pgfqpoint{1.806241in}{2.848363in}}{\pgfqpoint{1.814055in}{2.840549in}}%
\pgfpathcurveto{\pgfqpoint{1.821868in}{2.832736in}}{\pgfqpoint{1.832467in}{2.828345in}}{\pgfqpoint{1.843518in}{2.828345in}}%
\pgfpathclose%
\pgfusepath{stroke,fill}%
\end{pgfscope}%
\begin{pgfscope}%
\pgfpathrectangle{\pgfqpoint{0.600000in}{0.600000in}}{\pgfqpoint{3.900000in}{3.900000in}}%
\pgfusepath{clip}%
\pgfsetbuttcap%
\pgfsetroundjoin%
\definecolor{currentfill}{rgb}{0.121569,0.466667,0.705882}%
\pgfsetfillcolor{currentfill}%
\pgfsetlinewidth{1.003750pt}%
\definecolor{currentstroke}{rgb}{0.121569,0.466667,0.705882}%
\pgfsetstrokecolor{currentstroke}%
\pgfsetdash{}{0pt}%
\pgfpathmoveto{\pgfqpoint{2.830396in}{2.187968in}}%
\pgfpathcurveto{\pgfqpoint{2.841446in}{2.187968in}}{\pgfqpoint{2.852045in}{2.192358in}}{\pgfqpoint{2.859858in}{2.200172in}}%
\pgfpathcurveto{\pgfqpoint{2.867672in}{2.207985in}}{\pgfqpoint{2.872062in}{2.218584in}}{\pgfqpoint{2.872062in}{2.229635in}}%
\pgfpathcurveto{\pgfqpoint{2.872062in}{2.240685in}}{\pgfqpoint{2.867672in}{2.251284in}}{\pgfqpoint{2.859858in}{2.259097in}}%
\pgfpathcurveto{\pgfqpoint{2.852045in}{2.266911in}}{\pgfqpoint{2.841446in}{2.271301in}}{\pgfqpoint{2.830396in}{2.271301in}}%
\pgfpathcurveto{\pgfqpoint{2.819345in}{2.271301in}}{\pgfqpoint{2.808746in}{2.266911in}}{\pgfqpoint{2.800933in}{2.259097in}}%
\pgfpathcurveto{\pgfqpoint{2.793119in}{2.251284in}}{\pgfqpoint{2.788729in}{2.240685in}}{\pgfqpoint{2.788729in}{2.229635in}}%
\pgfpathcurveto{\pgfqpoint{2.788729in}{2.218584in}}{\pgfqpoint{2.793119in}{2.207985in}}{\pgfqpoint{2.800933in}{2.200172in}}%
\pgfpathcurveto{\pgfqpoint{2.808746in}{2.192358in}}{\pgfqpoint{2.819345in}{2.187968in}}{\pgfqpoint{2.830396in}{2.187968in}}%
\pgfpathclose%
\pgfusepath{stroke,fill}%
\end{pgfscope}%
\begin{pgfscope}%
\pgfpathrectangle{\pgfqpoint{0.600000in}{0.600000in}}{\pgfqpoint{3.900000in}{3.900000in}}%
\pgfusepath{clip}%
\pgfsetbuttcap%
\pgfsetroundjoin%
\definecolor{currentfill}{rgb}{0.121569,0.466667,0.705882}%
\pgfsetfillcolor{currentfill}%
\pgfsetlinewidth{1.003750pt}%
\definecolor{currentstroke}{rgb}{0.121569,0.466667,0.705882}%
\pgfsetstrokecolor{currentstroke}%
\pgfsetdash{}{0pt}%
\pgfpathmoveto{\pgfqpoint{2.541448in}{2.892067in}}%
\pgfpathcurveto{\pgfqpoint{2.552498in}{2.892067in}}{\pgfqpoint{2.563097in}{2.896457in}}{\pgfqpoint{2.570910in}{2.904271in}}%
\pgfpathcurveto{\pgfqpoint{2.578724in}{2.912084in}}{\pgfqpoint{2.583114in}{2.922684in}}{\pgfqpoint{2.583114in}{2.933734in}}%
\pgfpathcurveto{\pgfqpoint{2.583114in}{2.944784in}}{\pgfqpoint{2.578724in}{2.955383in}}{\pgfqpoint{2.570910in}{2.963196in}}%
\pgfpathcurveto{\pgfqpoint{2.563097in}{2.971010in}}{\pgfqpoint{2.552498in}{2.975400in}}{\pgfqpoint{2.541448in}{2.975400in}}%
\pgfpathcurveto{\pgfqpoint{2.530397in}{2.975400in}}{\pgfqpoint{2.519798in}{2.971010in}}{\pgfqpoint{2.511985in}{2.963196in}}%
\pgfpathcurveto{\pgfqpoint{2.504171in}{2.955383in}}{\pgfqpoint{2.499781in}{2.944784in}}{\pgfqpoint{2.499781in}{2.933734in}}%
\pgfpathcurveto{\pgfqpoint{2.499781in}{2.922684in}}{\pgfqpoint{2.504171in}{2.912084in}}{\pgfqpoint{2.511985in}{2.904271in}}%
\pgfpathcurveto{\pgfqpoint{2.519798in}{2.896457in}}{\pgfqpoint{2.530397in}{2.892067in}}{\pgfqpoint{2.541448in}{2.892067in}}%
\pgfpathclose%
\pgfusepath{stroke,fill}%
\end{pgfscope}%
\begin{pgfscope}%
\pgfpathrectangle{\pgfqpoint{0.600000in}{0.600000in}}{\pgfqpoint{3.900000in}{3.900000in}}%
\pgfusepath{clip}%
\pgfsetbuttcap%
\pgfsetroundjoin%
\definecolor{currentfill}{rgb}{0.121569,0.466667,0.705882}%
\pgfsetfillcolor{currentfill}%
\pgfsetlinewidth{1.003750pt}%
\definecolor{currentstroke}{rgb}{0.121569,0.466667,0.705882}%
\pgfsetstrokecolor{currentstroke}%
\pgfsetdash{}{0pt}%
\pgfpathmoveto{\pgfqpoint{1.833908in}{1.964824in}}%
\pgfpathcurveto{\pgfqpoint{1.844959in}{1.964824in}}{\pgfqpoint{1.855558in}{1.969214in}}{\pgfqpoint{1.863371in}{1.977028in}}%
\pgfpathcurveto{\pgfqpoint{1.871185in}{1.984841in}}{\pgfqpoint{1.875575in}{1.995441in}}{\pgfqpoint{1.875575in}{2.006491in}}%
\pgfpathcurveto{\pgfqpoint{1.875575in}{2.017541in}}{\pgfqpoint{1.871185in}{2.028140in}}{\pgfqpoint{1.863371in}{2.035953in}}%
\pgfpathcurveto{\pgfqpoint{1.855558in}{2.043767in}}{\pgfqpoint{1.844959in}{2.048157in}}{\pgfqpoint{1.833908in}{2.048157in}}%
\pgfpathcurveto{\pgfqpoint{1.822858in}{2.048157in}}{\pgfqpoint{1.812259in}{2.043767in}}{\pgfqpoint{1.804446in}{2.035953in}}%
\pgfpathcurveto{\pgfqpoint{1.796632in}{2.028140in}}{\pgfqpoint{1.792242in}{2.017541in}}{\pgfqpoint{1.792242in}{2.006491in}}%
\pgfpathcurveto{\pgfqpoint{1.792242in}{1.995441in}}{\pgfqpoint{1.796632in}{1.984841in}}{\pgfqpoint{1.804446in}{1.977028in}}%
\pgfpathcurveto{\pgfqpoint{1.812259in}{1.969214in}}{\pgfqpoint{1.822858in}{1.964824in}}{\pgfqpoint{1.833908in}{1.964824in}}%
\pgfpathclose%
\pgfusepath{stroke,fill}%
\end{pgfscope}%
\begin{pgfscope}%
\pgfpathrectangle{\pgfqpoint{0.600000in}{0.600000in}}{\pgfqpoint{3.900000in}{3.900000in}}%
\pgfusepath{clip}%
\pgfsetbuttcap%
\pgfsetroundjoin%
\definecolor{currentfill}{rgb}{0.121569,0.466667,0.705882}%
\pgfsetfillcolor{currentfill}%
\pgfsetlinewidth{1.003750pt}%
\definecolor{currentstroke}{rgb}{0.121569,0.466667,0.705882}%
\pgfsetstrokecolor{currentstroke}%
\pgfsetdash{}{0pt}%
\pgfpathmoveto{\pgfqpoint{2.137793in}{2.376362in}}%
\pgfpathcurveto{\pgfqpoint{2.148843in}{2.376362in}}{\pgfqpoint{2.159442in}{2.380752in}}{\pgfqpoint{2.167256in}{2.388565in}}%
\pgfpathcurveto{\pgfqpoint{2.175069in}{2.396379in}}{\pgfqpoint{2.179459in}{2.406978in}}{\pgfqpoint{2.179459in}{2.418028in}}%
\pgfpathcurveto{\pgfqpoint{2.179459in}{2.429078in}}{\pgfqpoint{2.175069in}{2.439677in}}{\pgfqpoint{2.167256in}{2.447491in}}%
\pgfpathcurveto{\pgfqpoint{2.159442in}{2.455305in}}{\pgfqpoint{2.148843in}{2.459695in}}{\pgfqpoint{2.137793in}{2.459695in}}%
\pgfpathcurveto{\pgfqpoint{2.126743in}{2.459695in}}{\pgfqpoint{2.116144in}{2.455305in}}{\pgfqpoint{2.108330in}{2.447491in}}%
\pgfpathcurveto{\pgfqpoint{2.100516in}{2.439677in}}{\pgfqpoint{2.096126in}{2.429078in}}{\pgfqpoint{2.096126in}{2.418028in}}%
\pgfpathcurveto{\pgfqpoint{2.096126in}{2.406978in}}{\pgfqpoint{2.100516in}{2.396379in}}{\pgfqpoint{2.108330in}{2.388565in}}%
\pgfpathcurveto{\pgfqpoint{2.116144in}{2.380752in}}{\pgfqpoint{2.126743in}{2.376362in}}{\pgfqpoint{2.137793in}{2.376362in}}%
\pgfpathclose%
\pgfusepath{stroke,fill}%
\end{pgfscope}%
\begin{pgfscope}%
\pgfpathrectangle{\pgfqpoint{0.600000in}{0.600000in}}{\pgfqpoint{3.900000in}{3.900000in}}%
\pgfusepath{clip}%
\pgfsetbuttcap%
\pgfsetroundjoin%
\definecolor{currentfill}{rgb}{0.121569,0.466667,0.705882}%
\pgfsetfillcolor{currentfill}%
\pgfsetlinewidth{1.003750pt}%
\definecolor{currentstroke}{rgb}{0.121569,0.466667,0.705882}%
\pgfsetstrokecolor{currentstroke}%
\pgfsetdash{}{0pt}%
\pgfpathmoveto{\pgfqpoint{2.701134in}{2.708943in}}%
\pgfpathcurveto{\pgfqpoint{2.712184in}{2.708943in}}{\pgfqpoint{2.722783in}{2.713334in}}{\pgfqpoint{2.730597in}{2.721147in}}%
\pgfpathcurveto{\pgfqpoint{2.738411in}{2.728961in}}{\pgfqpoint{2.742801in}{2.739560in}}{\pgfqpoint{2.742801in}{2.750610in}}%
\pgfpathcurveto{\pgfqpoint{2.742801in}{2.761660in}}{\pgfqpoint{2.738411in}{2.772259in}}{\pgfqpoint{2.730597in}{2.780073in}}%
\pgfpathcurveto{\pgfqpoint{2.722783in}{2.787886in}}{\pgfqpoint{2.712184in}{2.792277in}}{\pgfqpoint{2.701134in}{2.792277in}}%
\pgfpathcurveto{\pgfqpoint{2.690084in}{2.792277in}}{\pgfqpoint{2.679485in}{2.787886in}}{\pgfqpoint{2.671671in}{2.780073in}}%
\pgfpathcurveto{\pgfqpoint{2.663858in}{2.772259in}}{\pgfqpoint{2.659468in}{2.761660in}}{\pgfqpoint{2.659468in}{2.750610in}}%
\pgfpathcurveto{\pgfqpoint{2.659468in}{2.739560in}}{\pgfqpoint{2.663858in}{2.728961in}}{\pgfqpoint{2.671671in}{2.721147in}}%
\pgfpathcurveto{\pgfqpoint{2.679485in}{2.713334in}}{\pgfqpoint{2.690084in}{2.708943in}}{\pgfqpoint{2.701134in}{2.708943in}}%
\pgfpathclose%
\pgfusepath{stroke,fill}%
\end{pgfscope}%
\begin{pgfscope}%
\pgfpathrectangle{\pgfqpoint{0.600000in}{0.600000in}}{\pgfqpoint{3.900000in}{3.900000in}}%
\pgfusepath{clip}%
\pgfsetbuttcap%
\pgfsetroundjoin%
\definecolor{currentfill}{rgb}{0.121569,0.466667,0.705882}%
\pgfsetfillcolor{currentfill}%
\pgfsetlinewidth{1.003750pt}%
\definecolor{currentstroke}{rgb}{0.121569,0.466667,0.705882}%
\pgfsetstrokecolor{currentstroke}%
\pgfsetdash{}{0pt}%
\pgfpathmoveto{\pgfqpoint{2.358292in}{2.207533in}}%
\pgfpathcurveto{\pgfqpoint{2.369342in}{2.207533in}}{\pgfqpoint{2.379941in}{2.211923in}}{\pgfqpoint{2.387754in}{2.219737in}}%
\pgfpathcurveto{\pgfqpoint{2.395568in}{2.227550in}}{\pgfqpoint{2.399958in}{2.238149in}}{\pgfqpoint{2.399958in}{2.249199in}}%
\pgfpathcurveto{\pgfqpoint{2.399958in}{2.260250in}}{\pgfqpoint{2.395568in}{2.270849in}}{\pgfqpoint{2.387754in}{2.278662in}}%
\pgfpathcurveto{\pgfqpoint{2.379941in}{2.286476in}}{\pgfqpoint{2.369342in}{2.290866in}}{\pgfqpoint{2.358292in}{2.290866in}}%
\pgfpathcurveto{\pgfqpoint{2.347242in}{2.290866in}}{\pgfqpoint{2.336642in}{2.286476in}}{\pgfqpoint{2.328829in}{2.278662in}}%
\pgfpathcurveto{\pgfqpoint{2.321015in}{2.270849in}}{\pgfqpoint{2.316625in}{2.260250in}}{\pgfqpoint{2.316625in}{2.249199in}}%
\pgfpathcurveto{\pgfqpoint{2.316625in}{2.238149in}}{\pgfqpoint{2.321015in}{2.227550in}}{\pgfqpoint{2.328829in}{2.219737in}}%
\pgfpathcurveto{\pgfqpoint{2.336642in}{2.211923in}}{\pgfqpoint{2.347242in}{2.207533in}}{\pgfqpoint{2.358292in}{2.207533in}}%
\pgfpathclose%
\pgfusepath{stroke,fill}%
\end{pgfscope}%
\begin{pgfscope}%
\pgfpathrectangle{\pgfqpoint{0.600000in}{0.600000in}}{\pgfqpoint{3.900000in}{3.900000in}}%
\pgfusepath{clip}%
\pgfsetbuttcap%
\pgfsetroundjoin%
\definecolor{currentfill}{rgb}{0.121569,0.466667,0.705882}%
\pgfsetfillcolor{currentfill}%
\pgfsetlinewidth{1.003750pt}%
\definecolor{currentstroke}{rgb}{0.121569,0.466667,0.705882}%
\pgfsetstrokecolor{currentstroke}%
\pgfsetdash{}{0pt}%
\pgfpathmoveto{\pgfqpoint{2.820690in}{1.439367in}}%
\pgfpathcurveto{\pgfqpoint{2.831740in}{1.439367in}}{\pgfqpoint{2.842339in}{1.443757in}}{\pgfqpoint{2.850152in}{1.451571in}}%
\pgfpathcurveto{\pgfqpoint{2.857966in}{1.459385in}}{\pgfqpoint{2.862356in}{1.469984in}}{\pgfqpoint{2.862356in}{1.481034in}}%
\pgfpathcurveto{\pgfqpoint{2.862356in}{1.492084in}}{\pgfqpoint{2.857966in}{1.502683in}}{\pgfqpoint{2.850152in}{1.510496in}}%
\pgfpathcurveto{\pgfqpoint{2.842339in}{1.518310in}}{\pgfqpoint{2.831740in}{1.522700in}}{\pgfqpoint{2.820690in}{1.522700in}}%
\pgfpathcurveto{\pgfqpoint{2.809639in}{1.522700in}}{\pgfqpoint{2.799040in}{1.518310in}}{\pgfqpoint{2.791227in}{1.510496in}}%
\pgfpathcurveto{\pgfqpoint{2.783413in}{1.502683in}}{\pgfqpoint{2.779023in}{1.492084in}}{\pgfqpoint{2.779023in}{1.481034in}}%
\pgfpathcurveto{\pgfqpoint{2.779023in}{1.469984in}}{\pgfqpoint{2.783413in}{1.459385in}}{\pgfqpoint{2.791227in}{1.451571in}}%
\pgfpathcurveto{\pgfqpoint{2.799040in}{1.443757in}}{\pgfqpoint{2.809639in}{1.439367in}}{\pgfqpoint{2.820690in}{1.439367in}}%
\pgfpathclose%
\pgfusepath{stroke,fill}%
\end{pgfscope}%
\begin{pgfscope}%
\pgfpathrectangle{\pgfqpoint{0.600000in}{0.600000in}}{\pgfqpoint{3.900000in}{3.900000in}}%
\pgfusepath{clip}%
\pgfsetbuttcap%
\pgfsetroundjoin%
\definecolor{currentfill}{rgb}{0.121569,0.466667,0.705882}%
\pgfsetfillcolor{currentfill}%
\pgfsetlinewidth{1.003750pt}%
\definecolor{currentstroke}{rgb}{0.121569,0.466667,0.705882}%
\pgfsetstrokecolor{currentstroke}%
\pgfsetdash{}{0pt}%
\pgfpathmoveto{\pgfqpoint{1.851422in}{1.715831in}}%
\pgfpathcurveto{\pgfqpoint{1.862472in}{1.715831in}}{\pgfqpoint{1.873071in}{1.720221in}}{\pgfqpoint{1.880884in}{1.728035in}}%
\pgfpathcurveto{\pgfqpoint{1.888698in}{1.735848in}}{\pgfqpoint{1.893088in}{1.746447in}}{\pgfqpoint{1.893088in}{1.757497in}}%
\pgfpathcurveto{\pgfqpoint{1.893088in}{1.768548in}}{\pgfqpoint{1.888698in}{1.779147in}}{\pgfqpoint{1.880884in}{1.786960in}}%
\pgfpathcurveto{\pgfqpoint{1.873071in}{1.794774in}}{\pgfqpoint{1.862472in}{1.799164in}}{\pgfqpoint{1.851422in}{1.799164in}}%
\pgfpathcurveto{\pgfqpoint{1.840371in}{1.799164in}}{\pgfqpoint{1.829772in}{1.794774in}}{\pgfqpoint{1.821959in}{1.786960in}}%
\pgfpathcurveto{\pgfqpoint{1.814145in}{1.779147in}}{\pgfqpoint{1.809755in}{1.768548in}}{\pgfqpoint{1.809755in}{1.757497in}}%
\pgfpathcurveto{\pgfqpoint{1.809755in}{1.746447in}}{\pgfqpoint{1.814145in}{1.735848in}}{\pgfqpoint{1.821959in}{1.728035in}}%
\pgfpathcurveto{\pgfqpoint{1.829772in}{1.720221in}}{\pgfqpoint{1.840371in}{1.715831in}}{\pgfqpoint{1.851422in}{1.715831in}}%
\pgfpathclose%
\pgfusepath{stroke,fill}%
\end{pgfscope}%
\begin{pgfscope}%
\pgfpathrectangle{\pgfqpoint{0.600000in}{0.600000in}}{\pgfqpoint{3.900000in}{3.900000in}}%
\pgfusepath{clip}%
\pgfsetbuttcap%
\pgfsetroundjoin%
\definecolor{currentfill}{rgb}{0.121569,0.466667,0.705882}%
\pgfsetfillcolor{currentfill}%
\pgfsetlinewidth{1.003750pt}%
\definecolor{currentstroke}{rgb}{0.121569,0.466667,0.705882}%
\pgfsetstrokecolor{currentstroke}%
\pgfsetdash{}{0pt}%
\pgfpathmoveto{\pgfqpoint{2.449684in}{3.403705in}}%
\pgfpathcurveto{\pgfqpoint{2.460734in}{3.403705in}}{\pgfqpoint{2.471333in}{3.408095in}}{\pgfqpoint{2.479147in}{3.415909in}}%
\pgfpathcurveto{\pgfqpoint{2.486960in}{3.423722in}}{\pgfqpoint{2.491351in}{3.434322in}}{\pgfqpoint{2.491351in}{3.445372in}}%
\pgfpathcurveto{\pgfqpoint{2.491351in}{3.456422in}}{\pgfqpoint{2.486960in}{3.467021in}}{\pgfqpoint{2.479147in}{3.474834in}}%
\pgfpathcurveto{\pgfqpoint{2.471333in}{3.482648in}}{\pgfqpoint{2.460734in}{3.487038in}}{\pgfqpoint{2.449684in}{3.487038in}}%
\pgfpathcurveto{\pgfqpoint{2.438634in}{3.487038in}}{\pgfqpoint{2.428035in}{3.482648in}}{\pgfqpoint{2.420221in}{3.474834in}}%
\pgfpathcurveto{\pgfqpoint{2.412408in}{3.467021in}}{\pgfqpoint{2.408017in}{3.456422in}}{\pgfqpoint{2.408017in}{3.445372in}}%
\pgfpathcurveto{\pgfqpoint{2.408017in}{3.434322in}}{\pgfqpoint{2.412408in}{3.423722in}}{\pgfqpoint{2.420221in}{3.415909in}}%
\pgfpathcurveto{\pgfqpoint{2.428035in}{3.408095in}}{\pgfqpoint{2.438634in}{3.403705in}}{\pgfqpoint{2.449684in}{3.403705in}}%
\pgfpathclose%
\pgfusepath{stroke,fill}%
\end{pgfscope}%
\begin{pgfscope}%
\pgfpathrectangle{\pgfqpoint{0.600000in}{0.600000in}}{\pgfqpoint{3.900000in}{3.900000in}}%
\pgfusepath{clip}%
\pgfsetbuttcap%
\pgfsetroundjoin%
\definecolor{currentfill}{rgb}{0.121569,0.466667,0.705882}%
\pgfsetfillcolor{currentfill}%
\pgfsetlinewidth{1.003750pt}%
\definecolor{currentstroke}{rgb}{0.121569,0.466667,0.705882}%
\pgfsetstrokecolor{currentstroke}%
\pgfsetdash{}{0pt}%
\pgfpathmoveto{\pgfqpoint{2.945735in}{2.504478in}}%
\pgfpathcurveto{\pgfqpoint{2.956785in}{2.504478in}}{\pgfqpoint{2.967384in}{2.508869in}}{\pgfqpoint{2.975198in}{2.516682in}}%
\pgfpathcurveto{\pgfqpoint{2.983012in}{2.524496in}}{\pgfqpoint{2.987402in}{2.535095in}}{\pgfqpoint{2.987402in}{2.546145in}}%
\pgfpathcurveto{\pgfqpoint{2.987402in}{2.557195in}}{\pgfqpoint{2.983012in}{2.567794in}}{\pgfqpoint{2.975198in}{2.575608in}}%
\pgfpathcurveto{\pgfqpoint{2.967384in}{2.583421in}}{\pgfqpoint{2.956785in}{2.587812in}}{\pgfqpoint{2.945735in}{2.587812in}}%
\pgfpathcurveto{\pgfqpoint{2.934685in}{2.587812in}}{\pgfqpoint{2.924086in}{2.583421in}}{\pgfqpoint{2.916272in}{2.575608in}}%
\pgfpathcurveto{\pgfqpoint{2.908459in}{2.567794in}}{\pgfqpoint{2.904068in}{2.557195in}}{\pgfqpoint{2.904068in}{2.546145in}}%
\pgfpathcurveto{\pgfqpoint{2.904068in}{2.535095in}}{\pgfqpoint{2.908459in}{2.524496in}}{\pgfqpoint{2.916272in}{2.516682in}}%
\pgfpathcurveto{\pgfqpoint{2.924086in}{2.508869in}}{\pgfqpoint{2.934685in}{2.504478in}}{\pgfqpoint{2.945735in}{2.504478in}}%
\pgfpathclose%
\pgfusepath{stroke,fill}%
\end{pgfscope}%
\begin{pgfscope}%
\pgfpathrectangle{\pgfqpoint{0.600000in}{0.600000in}}{\pgfqpoint{3.900000in}{3.900000in}}%
\pgfusepath{clip}%
\pgfsetbuttcap%
\pgfsetroundjoin%
\definecolor{currentfill}{rgb}{0.121569,0.466667,0.705882}%
\pgfsetfillcolor{currentfill}%
\pgfsetlinewidth{1.003750pt}%
\definecolor{currentstroke}{rgb}{0.121569,0.466667,0.705882}%
\pgfsetstrokecolor{currentstroke}%
\pgfsetdash{}{0pt}%
\pgfpathmoveto{\pgfqpoint{2.532023in}{2.814718in}}%
\pgfpathcurveto{\pgfqpoint{2.543073in}{2.814718in}}{\pgfqpoint{2.553672in}{2.819108in}}{\pgfqpoint{2.561485in}{2.826922in}}%
\pgfpathcurveto{\pgfqpoint{2.569299in}{2.834735in}}{\pgfqpoint{2.573689in}{2.845334in}}{\pgfqpoint{2.573689in}{2.856384in}}%
\pgfpathcurveto{\pgfqpoint{2.573689in}{2.867435in}}{\pgfqpoint{2.569299in}{2.878034in}}{\pgfqpoint{2.561485in}{2.885847in}}%
\pgfpathcurveto{\pgfqpoint{2.553672in}{2.893661in}}{\pgfqpoint{2.543073in}{2.898051in}}{\pgfqpoint{2.532023in}{2.898051in}}%
\pgfpathcurveto{\pgfqpoint{2.520972in}{2.898051in}}{\pgfqpoint{2.510373in}{2.893661in}}{\pgfqpoint{2.502560in}{2.885847in}}%
\pgfpathcurveto{\pgfqpoint{2.494746in}{2.878034in}}{\pgfqpoint{2.490356in}{2.867435in}}{\pgfqpoint{2.490356in}{2.856384in}}%
\pgfpathcurveto{\pgfqpoint{2.490356in}{2.845334in}}{\pgfqpoint{2.494746in}{2.834735in}}{\pgfqpoint{2.502560in}{2.826922in}}%
\pgfpathcurveto{\pgfqpoint{2.510373in}{2.819108in}}{\pgfqpoint{2.520972in}{2.814718in}}{\pgfqpoint{2.532023in}{2.814718in}}%
\pgfpathclose%
\pgfusepath{stroke,fill}%
\end{pgfscope}%
\begin{pgfscope}%
\pgfpathrectangle{\pgfqpoint{0.600000in}{0.600000in}}{\pgfqpoint{3.900000in}{3.900000in}}%
\pgfusepath{clip}%
\pgfsetbuttcap%
\pgfsetroundjoin%
\definecolor{currentfill}{rgb}{0.121569,0.466667,0.705882}%
\pgfsetfillcolor{currentfill}%
\pgfsetlinewidth{1.003750pt}%
\definecolor{currentstroke}{rgb}{0.121569,0.466667,0.705882}%
\pgfsetstrokecolor{currentstroke}%
\pgfsetdash{}{0pt}%
\pgfpathmoveto{\pgfqpoint{1.619219in}{2.266945in}}%
\pgfpathcurveto{\pgfqpoint{1.630269in}{2.266945in}}{\pgfqpoint{1.640868in}{2.271335in}}{\pgfqpoint{1.648682in}{2.279149in}}%
\pgfpathcurveto{\pgfqpoint{1.656495in}{2.286963in}}{\pgfqpoint{1.660886in}{2.297562in}}{\pgfqpoint{1.660886in}{2.308612in}}%
\pgfpathcurveto{\pgfqpoint{1.660886in}{2.319662in}}{\pgfqpoint{1.656495in}{2.330261in}}{\pgfqpoint{1.648682in}{2.338075in}}%
\pgfpathcurveto{\pgfqpoint{1.640868in}{2.345888in}}{\pgfqpoint{1.630269in}{2.350278in}}{\pgfqpoint{1.619219in}{2.350278in}}%
\pgfpathcurveto{\pgfqpoint{1.608169in}{2.350278in}}{\pgfqpoint{1.597570in}{2.345888in}}{\pgfqpoint{1.589756in}{2.338075in}}%
\pgfpathcurveto{\pgfqpoint{1.581943in}{2.330261in}}{\pgfqpoint{1.577552in}{2.319662in}}{\pgfqpoint{1.577552in}{2.308612in}}%
\pgfpathcurveto{\pgfqpoint{1.577552in}{2.297562in}}{\pgfqpoint{1.581943in}{2.286963in}}{\pgfqpoint{1.589756in}{2.279149in}}%
\pgfpathcurveto{\pgfqpoint{1.597570in}{2.271335in}}{\pgfqpoint{1.608169in}{2.266945in}}{\pgfqpoint{1.619219in}{2.266945in}}%
\pgfpathclose%
\pgfusepath{stroke,fill}%
\end{pgfscope}%
\begin{pgfscope}%
\pgfpathrectangle{\pgfqpoint{0.600000in}{0.600000in}}{\pgfqpoint{3.900000in}{3.900000in}}%
\pgfusepath{clip}%
\pgfsetbuttcap%
\pgfsetroundjoin%
\definecolor{currentfill}{rgb}{0.121569,0.466667,0.705882}%
\pgfsetfillcolor{currentfill}%
\pgfsetlinewidth{1.003750pt}%
\definecolor{currentstroke}{rgb}{0.121569,0.466667,0.705882}%
\pgfsetstrokecolor{currentstroke}%
\pgfsetdash{}{0pt}%
\pgfpathmoveto{\pgfqpoint{3.142097in}{2.291924in}}%
\pgfpathcurveto{\pgfqpoint{3.153147in}{2.291924in}}{\pgfqpoint{3.163746in}{2.296314in}}{\pgfqpoint{3.171560in}{2.304128in}}%
\pgfpathcurveto{\pgfqpoint{3.179373in}{2.311942in}}{\pgfqpoint{3.183764in}{2.322541in}}{\pgfqpoint{3.183764in}{2.333591in}}%
\pgfpathcurveto{\pgfqpoint{3.183764in}{2.344641in}}{\pgfqpoint{3.179373in}{2.355240in}}{\pgfqpoint{3.171560in}{2.363054in}}%
\pgfpathcurveto{\pgfqpoint{3.163746in}{2.370867in}}{\pgfqpoint{3.153147in}{2.375258in}}{\pgfqpoint{3.142097in}{2.375258in}}%
\pgfpathcurveto{\pgfqpoint{3.131047in}{2.375258in}}{\pgfqpoint{3.120448in}{2.370867in}}{\pgfqpoint{3.112634in}{2.363054in}}%
\pgfpathcurveto{\pgfqpoint{3.104821in}{2.355240in}}{\pgfqpoint{3.100430in}{2.344641in}}{\pgfqpoint{3.100430in}{2.333591in}}%
\pgfpathcurveto{\pgfqpoint{3.100430in}{2.322541in}}{\pgfqpoint{3.104821in}{2.311942in}}{\pgfqpoint{3.112634in}{2.304128in}}%
\pgfpathcurveto{\pgfqpoint{3.120448in}{2.296314in}}{\pgfqpoint{3.131047in}{2.291924in}}{\pgfqpoint{3.142097in}{2.291924in}}%
\pgfpathclose%
\pgfusepath{stroke,fill}%
\end{pgfscope}%
\begin{pgfscope}%
\pgfpathrectangle{\pgfqpoint{0.600000in}{0.600000in}}{\pgfqpoint{3.900000in}{3.900000in}}%
\pgfusepath{clip}%
\pgfsetbuttcap%
\pgfsetroundjoin%
\definecolor{currentfill}{rgb}{0.121569,0.466667,0.705882}%
\pgfsetfillcolor{currentfill}%
\pgfsetlinewidth{1.003750pt}%
\definecolor{currentstroke}{rgb}{0.121569,0.466667,0.705882}%
\pgfsetstrokecolor{currentstroke}%
\pgfsetdash{}{0pt}%
\pgfpathmoveto{\pgfqpoint{3.213072in}{2.065202in}}%
\pgfpathcurveto{\pgfqpoint{3.224122in}{2.065202in}}{\pgfqpoint{3.234721in}{2.069592in}}{\pgfqpoint{3.242535in}{2.077405in}}%
\pgfpathcurveto{\pgfqpoint{3.250348in}{2.085219in}}{\pgfqpoint{3.254739in}{2.095818in}}{\pgfqpoint{3.254739in}{2.106868in}}%
\pgfpathcurveto{\pgfqpoint{3.254739in}{2.117918in}}{\pgfqpoint{3.250348in}{2.128517in}}{\pgfqpoint{3.242535in}{2.136331in}}%
\pgfpathcurveto{\pgfqpoint{3.234721in}{2.144145in}}{\pgfqpoint{3.224122in}{2.148535in}}{\pgfqpoint{3.213072in}{2.148535in}}%
\pgfpathcurveto{\pgfqpoint{3.202022in}{2.148535in}}{\pgfqpoint{3.191423in}{2.144145in}}{\pgfqpoint{3.183609in}{2.136331in}}%
\pgfpathcurveto{\pgfqpoint{3.175796in}{2.128517in}}{\pgfqpoint{3.171405in}{2.117918in}}{\pgfqpoint{3.171405in}{2.106868in}}%
\pgfpathcurveto{\pgfqpoint{3.171405in}{2.095818in}}{\pgfqpoint{3.175796in}{2.085219in}}{\pgfqpoint{3.183609in}{2.077405in}}%
\pgfpathcurveto{\pgfqpoint{3.191423in}{2.069592in}}{\pgfqpoint{3.202022in}{2.065202in}}{\pgfqpoint{3.213072in}{2.065202in}}%
\pgfpathclose%
\pgfusepath{stroke,fill}%
\end{pgfscope}%
\begin{pgfscope}%
\pgfpathrectangle{\pgfqpoint{0.600000in}{0.600000in}}{\pgfqpoint{3.900000in}{3.900000in}}%
\pgfusepath{clip}%
\pgfsetbuttcap%
\pgfsetroundjoin%
\definecolor{currentfill}{rgb}{0.121569,0.466667,0.705882}%
\pgfsetfillcolor{currentfill}%
\pgfsetlinewidth{1.003750pt}%
\definecolor{currentstroke}{rgb}{0.121569,0.466667,0.705882}%
\pgfsetstrokecolor{currentstroke}%
\pgfsetdash{}{0pt}%
\pgfpathmoveto{\pgfqpoint{2.751051in}{2.960998in}}%
\pgfpathcurveto{\pgfqpoint{2.762101in}{2.960998in}}{\pgfqpoint{2.772700in}{2.965388in}}{\pgfqpoint{2.780513in}{2.973202in}}%
\pgfpathcurveto{\pgfqpoint{2.788327in}{2.981015in}}{\pgfqpoint{2.792717in}{2.991614in}}{\pgfqpoint{2.792717in}{3.002664in}}%
\pgfpathcurveto{\pgfqpoint{2.792717in}{3.013715in}}{\pgfqpoint{2.788327in}{3.024314in}}{\pgfqpoint{2.780513in}{3.032127in}}%
\pgfpathcurveto{\pgfqpoint{2.772700in}{3.039941in}}{\pgfqpoint{2.762101in}{3.044331in}}{\pgfqpoint{2.751051in}{3.044331in}}%
\pgfpathcurveto{\pgfqpoint{2.740001in}{3.044331in}}{\pgfqpoint{2.729402in}{3.039941in}}{\pgfqpoint{2.721588in}{3.032127in}}%
\pgfpathcurveto{\pgfqpoint{2.713774in}{3.024314in}}{\pgfqpoint{2.709384in}{3.013715in}}{\pgfqpoint{2.709384in}{3.002664in}}%
\pgfpathcurveto{\pgfqpoint{2.709384in}{2.991614in}}{\pgfqpoint{2.713774in}{2.981015in}}{\pgfqpoint{2.721588in}{2.973202in}}%
\pgfpathcurveto{\pgfqpoint{2.729402in}{2.965388in}}{\pgfqpoint{2.740001in}{2.960998in}}{\pgfqpoint{2.751051in}{2.960998in}}%
\pgfpathclose%
\pgfusepath{stroke,fill}%
\end{pgfscope}%
\begin{pgfscope}%
\pgfpathrectangle{\pgfqpoint{0.600000in}{0.600000in}}{\pgfqpoint{3.900000in}{3.900000in}}%
\pgfusepath{clip}%
\pgfsetbuttcap%
\pgfsetroundjoin%
\definecolor{currentfill}{rgb}{0.121569,0.466667,0.705882}%
\pgfsetfillcolor{currentfill}%
\pgfsetlinewidth{1.003750pt}%
\definecolor{currentstroke}{rgb}{0.121569,0.466667,0.705882}%
\pgfsetstrokecolor{currentstroke}%
\pgfsetdash{}{0pt}%
\pgfpathmoveto{\pgfqpoint{2.845533in}{3.054398in}}%
\pgfpathcurveto{\pgfqpoint{2.856583in}{3.054398in}}{\pgfqpoint{2.867182in}{3.058788in}}{\pgfqpoint{2.874996in}{3.066602in}}%
\pgfpathcurveto{\pgfqpoint{2.882809in}{3.074416in}}{\pgfqpoint{2.887200in}{3.085015in}}{\pgfqpoint{2.887200in}{3.096065in}}%
\pgfpathcurveto{\pgfqpoint{2.887200in}{3.107115in}}{\pgfqpoint{2.882809in}{3.117714in}}{\pgfqpoint{2.874996in}{3.125528in}}%
\pgfpathcurveto{\pgfqpoint{2.867182in}{3.133341in}}{\pgfqpoint{2.856583in}{3.137731in}}{\pgfqpoint{2.845533in}{3.137731in}}%
\pgfpathcurveto{\pgfqpoint{2.834483in}{3.137731in}}{\pgfqpoint{2.823884in}{3.133341in}}{\pgfqpoint{2.816070in}{3.125528in}}%
\pgfpathcurveto{\pgfqpoint{2.808256in}{3.117714in}}{\pgfqpoint{2.803866in}{3.107115in}}{\pgfqpoint{2.803866in}{3.096065in}}%
\pgfpathcurveto{\pgfqpoint{2.803866in}{3.085015in}}{\pgfqpoint{2.808256in}{3.074416in}}{\pgfqpoint{2.816070in}{3.066602in}}%
\pgfpathcurveto{\pgfqpoint{2.823884in}{3.058788in}}{\pgfqpoint{2.834483in}{3.054398in}}{\pgfqpoint{2.845533in}{3.054398in}}%
\pgfpathclose%
\pgfusepath{stroke,fill}%
\end{pgfscope}%
\begin{pgfscope}%
\pgfpathrectangle{\pgfqpoint{0.600000in}{0.600000in}}{\pgfqpoint{3.900000in}{3.900000in}}%
\pgfusepath{clip}%
\pgfsetbuttcap%
\pgfsetroundjoin%
\definecolor{currentfill}{rgb}{0.121569,0.466667,0.705882}%
\pgfsetfillcolor{currentfill}%
\pgfsetlinewidth{1.003750pt}%
\definecolor{currentstroke}{rgb}{0.121569,0.466667,0.705882}%
\pgfsetstrokecolor{currentstroke}%
\pgfsetdash{}{0pt}%
\pgfpathmoveto{\pgfqpoint{2.817420in}{2.396318in}}%
\pgfpathcurveto{\pgfqpoint{2.828470in}{2.396318in}}{\pgfqpoint{2.839069in}{2.400708in}}{\pgfqpoint{2.846883in}{2.408522in}}%
\pgfpathcurveto{\pgfqpoint{2.854696in}{2.416335in}}{\pgfqpoint{2.859086in}{2.426934in}}{\pgfqpoint{2.859086in}{2.437985in}}%
\pgfpathcurveto{\pgfqpoint{2.859086in}{2.449035in}}{\pgfqpoint{2.854696in}{2.459634in}}{\pgfqpoint{2.846883in}{2.467447in}}%
\pgfpathcurveto{\pgfqpoint{2.839069in}{2.475261in}}{\pgfqpoint{2.828470in}{2.479651in}}{\pgfqpoint{2.817420in}{2.479651in}}%
\pgfpathcurveto{\pgfqpoint{2.806370in}{2.479651in}}{\pgfqpoint{2.795771in}{2.475261in}}{\pgfqpoint{2.787957in}{2.467447in}}%
\pgfpathcurveto{\pgfqpoint{2.780143in}{2.459634in}}{\pgfqpoint{2.775753in}{2.449035in}}{\pgfqpoint{2.775753in}{2.437985in}}%
\pgfpathcurveto{\pgfqpoint{2.775753in}{2.426934in}}{\pgfqpoint{2.780143in}{2.416335in}}{\pgfqpoint{2.787957in}{2.408522in}}%
\pgfpathcurveto{\pgfqpoint{2.795771in}{2.400708in}}{\pgfqpoint{2.806370in}{2.396318in}}{\pgfqpoint{2.817420in}{2.396318in}}%
\pgfpathclose%
\pgfusepath{stroke,fill}%
\end{pgfscope}%
\begin{pgfscope}%
\pgfpathrectangle{\pgfqpoint{0.600000in}{0.600000in}}{\pgfqpoint{3.900000in}{3.900000in}}%
\pgfusepath{clip}%
\pgfsetbuttcap%
\pgfsetroundjoin%
\definecolor{currentfill}{rgb}{0.121569,0.466667,0.705882}%
\pgfsetfillcolor{currentfill}%
\pgfsetlinewidth{1.003750pt}%
\definecolor{currentstroke}{rgb}{0.121569,0.466667,0.705882}%
\pgfsetstrokecolor{currentstroke}%
\pgfsetdash{}{0pt}%
\pgfpathmoveto{\pgfqpoint{2.102669in}{3.862408in}}%
\pgfpathcurveto{\pgfqpoint{2.113719in}{3.862408in}}{\pgfqpoint{2.124318in}{3.866798in}}{\pgfqpoint{2.132132in}{3.874611in}}%
\pgfpathcurveto{\pgfqpoint{2.139946in}{3.882425in}}{\pgfqpoint{2.144336in}{3.893024in}}{\pgfqpoint{2.144336in}{3.904074in}}%
\pgfpathcurveto{\pgfqpoint{2.144336in}{3.915124in}}{\pgfqpoint{2.139946in}{3.925723in}}{\pgfqpoint{2.132132in}{3.933537in}}%
\pgfpathcurveto{\pgfqpoint{2.124318in}{3.941351in}}{\pgfqpoint{2.113719in}{3.945741in}}{\pgfqpoint{2.102669in}{3.945741in}}%
\pgfpathcurveto{\pgfqpoint{2.091619in}{3.945741in}}{\pgfqpoint{2.081020in}{3.941351in}}{\pgfqpoint{2.073206in}{3.933537in}}%
\pgfpathcurveto{\pgfqpoint{2.065393in}{3.925723in}}{\pgfqpoint{2.061003in}{3.915124in}}{\pgfqpoint{2.061003in}{3.904074in}}%
\pgfpathcurveto{\pgfqpoint{2.061003in}{3.893024in}}{\pgfqpoint{2.065393in}{3.882425in}}{\pgfqpoint{2.073206in}{3.874611in}}%
\pgfpathcurveto{\pgfqpoint{2.081020in}{3.866798in}}{\pgfqpoint{2.091619in}{3.862408in}}{\pgfqpoint{2.102669in}{3.862408in}}%
\pgfpathclose%
\pgfusepath{stroke,fill}%
\end{pgfscope}%
\begin{pgfscope}%
\pgfpathrectangle{\pgfqpoint{0.600000in}{0.600000in}}{\pgfqpoint{3.900000in}{3.900000in}}%
\pgfusepath{clip}%
\pgfsetbuttcap%
\pgfsetroundjoin%
\definecolor{currentfill}{rgb}{0.121569,0.466667,0.705882}%
\pgfsetfillcolor{currentfill}%
\pgfsetlinewidth{1.003750pt}%
\definecolor{currentstroke}{rgb}{0.121569,0.466667,0.705882}%
\pgfsetstrokecolor{currentstroke}%
\pgfsetdash{}{0pt}%
\pgfpathmoveto{\pgfqpoint{2.496547in}{2.828508in}}%
\pgfpathcurveto{\pgfqpoint{2.507597in}{2.828508in}}{\pgfqpoint{2.518196in}{2.832898in}}{\pgfqpoint{2.526009in}{2.840712in}}%
\pgfpathcurveto{\pgfqpoint{2.533823in}{2.848526in}}{\pgfqpoint{2.538213in}{2.859125in}}{\pgfqpoint{2.538213in}{2.870175in}}%
\pgfpathcurveto{\pgfqpoint{2.538213in}{2.881225in}}{\pgfqpoint{2.533823in}{2.891824in}}{\pgfqpoint{2.526009in}{2.899637in}}%
\pgfpathcurveto{\pgfqpoint{2.518196in}{2.907451in}}{\pgfqpoint{2.507597in}{2.911841in}}{\pgfqpoint{2.496547in}{2.911841in}}%
\pgfpathcurveto{\pgfqpoint{2.485497in}{2.911841in}}{\pgfqpoint{2.474898in}{2.907451in}}{\pgfqpoint{2.467084in}{2.899637in}}%
\pgfpathcurveto{\pgfqpoint{2.459270in}{2.891824in}}{\pgfqpoint{2.454880in}{2.881225in}}{\pgfqpoint{2.454880in}{2.870175in}}%
\pgfpathcurveto{\pgfqpoint{2.454880in}{2.859125in}}{\pgfqpoint{2.459270in}{2.848526in}}{\pgfqpoint{2.467084in}{2.840712in}}%
\pgfpathcurveto{\pgfqpoint{2.474898in}{2.832898in}}{\pgfqpoint{2.485497in}{2.828508in}}{\pgfqpoint{2.496547in}{2.828508in}}%
\pgfpathclose%
\pgfusepath{stroke,fill}%
\end{pgfscope}%
\begin{pgfscope}%
\pgfpathrectangle{\pgfqpoint{0.600000in}{0.600000in}}{\pgfqpoint{3.900000in}{3.900000in}}%
\pgfusepath{clip}%
\pgfsetbuttcap%
\pgfsetroundjoin%
\definecolor{currentfill}{rgb}{0.121569,0.466667,0.705882}%
\pgfsetfillcolor{currentfill}%
\pgfsetlinewidth{1.003750pt}%
\definecolor{currentstroke}{rgb}{0.121569,0.466667,0.705882}%
\pgfsetstrokecolor{currentstroke}%
\pgfsetdash{}{0pt}%
\pgfpathmoveto{\pgfqpoint{2.121102in}{2.430832in}}%
\pgfpathcurveto{\pgfqpoint{2.132152in}{2.430832in}}{\pgfqpoint{2.142751in}{2.435223in}}{\pgfqpoint{2.150565in}{2.443036in}}%
\pgfpathcurveto{\pgfqpoint{2.158379in}{2.450850in}}{\pgfqpoint{2.162769in}{2.461449in}}{\pgfqpoint{2.162769in}{2.472499in}}%
\pgfpathcurveto{\pgfqpoint{2.162769in}{2.483549in}}{\pgfqpoint{2.158379in}{2.494148in}}{\pgfqpoint{2.150565in}{2.501962in}}%
\pgfpathcurveto{\pgfqpoint{2.142751in}{2.509775in}}{\pgfqpoint{2.132152in}{2.514166in}}{\pgfqpoint{2.121102in}{2.514166in}}%
\pgfpathcurveto{\pgfqpoint{2.110052in}{2.514166in}}{\pgfqpoint{2.099453in}{2.509775in}}{\pgfqpoint{2.091640in}{2.501962in}}%
\pgfpathcurveto{\pgfqpoint{2.083826in}{2.494148in}}{\pgfqpoint{2.079436in}{2.483549in}}{\pgfqpoint{2.079436in}{2.472499in}}%
\pgfpathcurveto{\pgfqpoint{2.079436in}{2.461449in}}{\pgfqpoint{2.083826in}{2.450850in}}{\pgfqpoint{2.091640in}{2.443036in}}%
\pgfpathcurveto{\pgfqpoint{2.099453in}{2.435223in}}{\pgfqpoint{2.110052in}{2.430832in}}{\pgfqpoint{2.121102in}{2.430832in}}%
\pgfpathclose%
\pgfusepath{stroke,fill}%
\end{pgfscope}%
\begin{pgfscope}%
\pgfpathrectangle{\pgfqpoint{0.600000in}{0.600000in}}{\pgfqpoint{3.900000in}{3.900000in}}%
\pgfusepath{clip}%
\pgfsetbuttcap%
\pgfsetroundjoin%
\definecolor{currentfill}{rgb}{0.121569,0.466667,0.705882}%
\pgfsetfillcolor{currentfill}%
\pgfsetlinewidth{1.003750pt}%
\definecolor{currentstroke}{rgb}{0.121569,0.466667,0.705882}%
\pgfsetstrokecolor{currentstroke}%
\pgfsetdash{}{0pt}%
\pgfpathmoveto{\pgfqpoint{2.002991in}{3.079599in}}%
\pgfpathcurveto{\pgfqpoint{2.014041in}{3.079599in}}{\pgfqpoint{2.024640in}{3.083989in}}{\pgfqpoint{2.032454in}{3.091803in}}%
\pgfpathcurveto{\pgfqpoint{2.040267in}{3.099616in}}{\pgfqpoint{2.044657in}{3.110215in}}{\pgfqpoint{2.044657in}{3.121265in}}%
\pgfpathcurveto{\pgfqpoint{2.044657in}{3.132315in}}{\pgfqpoint{2.040267in}{3.142914in}}{\pgfqpoint{2.032454in}{3.150728in}}%
\pgfpathcurveto{\pgfqpoint{2.024640in}{3.158542in}}{\pgfqpoint{2.014041in}{3.162932in}}{\pgfqpoint{2.002991in}{3.162932in}}%
\pgfpathcurveto{\pgfqpoint{1.991941in}{3.162932in}}{\pgfqpoint{1.981342in}{3.158542in}}{\pgfqpoint{1.973528in}{3.150728in}}%
\pgfpathcurveto{\pgfqpoint{1.965714in}{3.142914in}}{\pgfqpoint{1.961324in}{3.132315in}}{\pgfqpoint{1.961324in}{3.121265in}}%
\pgfpathcurveto{\pgfqpoint{1.961324in}{3.110215in}}{\pgfqpoint{1.965714in}{3.099616in}}{\pgfqpoint{1.973528in}{3.091803in}}%
\pgfpathcurveto{\pgfqpoint{1.981342in}{3.083989in}}{\pgfqpoint{1.991941in}{3.079599in}}{\pgfqpoint{2.002991in}{3.079599in}}%
\pgfpathclose%
\pgfusepath{stroke,fill}%
\end{pgfscope}%
\begin{pgfscope}%
\pgfpathrectangle{\pgfqpoint{0.600000in}{0.600000in}}{\pgfqpoint{3.900000in}{3.900000in}}%
\pgfusepath{clip}%
\pgfsetbuttcap%
\pgfsetroundjoin%
\definecolor{currentfill}{rgb}{0.121569,0.466667,0.705882}%
\pgfsetfillcolor{currentfill}%
\pgfsetlinewidth{1.003750pt}%
\definecolor{currentstroke}{rgb}{0.121569,0.466667,0.705882}%
\pgfsetstrokecolor{currentstroke}%
\pgfsetdash{}{0pt}%
\pgfpathmoveto{\pgfqpoint{1.456021in}{2.771399in}}%
\pgfpathcurveto{\pgfqpoint{1.467071in}{2.771399in}}{\pgfqpoint{1.477670in}{2.775789in}}{\pgfqpoint{1.485483in}{2.783603in}}%
\pgfpathcurveto{\pgfqpoint{1.493297in}{2.791416in}}{\pgfqpoint{1.497687in}{2.802016in}}{\pgfqpoint{1.497687in}{2.813066in}}%
\pgfpathcurveto{\pgfqpoint{1.497687in}{2.824116in}}{\pgfqpoint{1.493297in}{2.834715in}}{\pgfqpoint{1.485483in}{2.842528in}}%
\pgfpathcurveto{\pgfqpoint{1.477670in}{2.850342in}}{\pgfqpoint{1.467071in}{2.854732in}}{\pgfqpoint{1.456021in}{2.854732in}}%
\pgfpathcurveto{\pgfqpoint{1.444971in}{2.854732in}}{\pgfqpoint{1.434371in}{2.850342in}}{\pgfqpoint{1.426558in}{2.842528in}}%
\pgfpathcurveto{\pgfqpoint{1.418744in}{2.834715in}}{\pgfqpoint{1.414354in}{2.824116in}}{\pgfqpoint{1.414354in}{2.813066in}}%
\pgfpathcurveto{\pgfqpoint{1.414354in}{2.802016in}}{\pgfqpoint{1.418744in}{2.791416in}}{\pgfqpoint{1.426558in}{2.783603in}}%
\pgfpathcurveto{\pgfqpoint{1.434371in}{2.775789in}}{\pgfqpoint{1.444971in}{2.771399in}}{\pgfqpoint{1.456021in}{2.771399in}}%
\pgfpathclose%
\pgfusepath{stroke,fill}%
\end{pgfscope}%
\begin{pgfscope}%
\pgfpathrectangle{\pgfqpoint{0.600000in}{0.600000in}}{\pgfqpoint{3.900000in}{3.900000in}}%
\pgfusepath{clip}%
\pgfsetbuttcap%
\pgfsetroundjoin%
\definecolor{currentfill}{rgb}{0.121569,0.466667,0.705882}%
\pgfsetfillcolor{currentfill}%
\pgfsetlinewidth{1.003750pt}%
\definecolor{currentstroke}{rgb}{0.121569,0.466667,0.705882}%
\pgfsetstrokecolor{currentstroke}%
\pgfsetdash{}{0pt}%
\pgfpathmoveto{\pgfqpoint{2.275969in}{2.676610in}}%
\pgfpathcurveto{\pgfqpoint{2.287020in}{2.676610in}}{\pgfqpoint{2.297619in}{2.681000in}}{\pgfqpoint{2.305432in}{2.688814in}}%
\pgfpathcurveto{\pgfqpoint{2.313246in}{2.696627in}}{\pgfqpoint{2.317636in}{2.707226in}}{\pgfqpoint{2.317636in}{2.718277in}}%
\pgfpathcurveto{\pgfqpoint{2.317636in}{2.729327in}}{\pgfqpoint{2.313246in}{2.739926in}}{\pgfqpoint{2.305432in}{2.747739in}}%
\pgfpathcurveto{\pgfqpoint{2.297619in}{2.755553in}}{\pgfqpoint{2.287020in}{2.759943in}}{\pgfqpoint{2.275969in}{2.759943in}}%
\pgfpathcurveto{\pgfqpoint{2.264919in}{2.759943in}}{\pgfqpoint{2.254320in}{2.755553in}}{\pgfqpoint{2.246507in}{2.747739in}}%
\pgfpathcurveto{\pgfqpoint{2.238693in}{2.739926in}}{\pgfqpoint{2.234303in}{2.729327in}}{\pgfqpoint{2.234303in}{2.718277in}}%
\pgfpathcurveto{\pgfqpoint{2.234303in}{2.707226in}}{\pgfqpoint{2.238693in}{2.696627in}}{\pgfqpoint{2.246507in}{2.688814in}}%
\pgfpathcurveto{\pgfqpoint{2.254320in}{2.681000in}}{\pgfqpoint{2.264919in}{2.676610in}}{\pgfqpoint{2.275969in}{2.676610in}}%
\pgfpathclose%
\pgfusepath{stroke,fill}%
\end{pgfscope}%
\begin{pgfscope}%
\pgfpathrectangle{\pgfqpoint{0.600000in}{0.600000in}}{\pgfqpoint{3.900000in}{3.900000in}}%
\pgfusepath{clip}%
\pgfsetbuttcap%
\pgfsetroundjoin%
\definecolor{currentfill}{rgb}{0.121569,0.466667,0.705882}%
\pgfsetfillcolor{currentfill}%
\pgfsetlinewidth{1.003750pt}%
\definecolor{currentstroke}{rgb}{0.121569,0.466667,0.705882}%
\pgfsetstrokecolor{currentstroke}%
\pgfsetdash{}{0pt}%
\pgfpathmoveto{\pgfqpoint{2.333466in}{1.846557in}}%
\pgfpathcurveto{\pgfqpoint{2.344516in}{1.846557in}}{\pgfqpoint{2.355115in}{1.850947in}}{\pgfqpoint{2.362928in}{1.858760in}}%
\pgfpathcurveto{\pgfqpoint{2.370742in}{1.866574in}}{\pgfqpoint{2.375132in}{1.877173in}}{\pgfqpoint{2.375132in}{1.888223in}}%
\pgfpathcurveto{\pgfqpoint{2.375132in}{1.899273in}}{\pgfqpoint{2.370742in}{1.909872in}}{\pgfqpoint{2.362928in}{1.917686in}}%
\pgfpathcurveto{\pgfqpoint{2.355115in}{1.925500in}}{\pgfqpoint{2.344516in}{1.929890in}}{\pgfqpoint{2.333466in}{1.929890in}}%
\pgfpathcurveto{\pgfqpoint{2.322416in}{1.929890in}}{\pgfqpoint{2.311816in}{1.925500in}}{\pgfqpoint{2.304003in}{1.917686in}}%
\pgfpathcurveto{\pgfqpoint{2.296189in}{1.909872in}}{\pgfqpoint{2.291799in}{1.899273in}}{\pgfqpoint{2.291799in}{1.888223in}}%
\pgfpathcurveto{\pgfqpoint{2.291799in}{1.877173in}}{\pgfqpoint{2.296189in}{1.866574in}}{\pgfqpoint{2.304003in}{1.858760in}}%
\pgfpathcurveto{\pgfqpoint{2.311816in}{1.850947in}}{\pgfqpoint{2.322416in}{1.846557in}}{\pgfqpoint{2.333466in}{1.846557in}}%
\pgfpathclose%
\pgfusepath{stroke,fill}%
\end{pgfscope}%
\begin{pgfscope}%
\pgfpathrectangle{\pgfqpoint{0.600000in}{0.600000in}}{\pgfqpoint{3.900000in}{3.900000in}}%
\pgfusepath{clip}%
\pgfsetbuttcap%
\pgfsetroundjoin%
\definecolor{currentfill}{rgb}{0.121569,0.466667,0.705882}%
\pgfsetfillcolor{currentfill}%
\pgfsetlinewidth{1.003750pt}%
\definecolor{currentstroke}{rgb}{0.121569,0.466667,0.705882}%
\pgfsetstrokecolor{currentstroke}%
\pgfsetdash{}{0pt}%
\pgfpathmoveto{\pgfqpoint{1.466738in}{1.855187in}}%
\pgfpathcurveto{\pgfqpoint{1.477788in}{1.855187in}}{\pgfqpoint{1.488387in}{1.859578in}}{\pgfqpoint{1.496201in}{1.867391in}}%
\pgfpathcurveto{\pgfqpoint{1.504015in}{1.875205in}}{\pgfqpoint{1.508405in}{1.885804in}}{\pgfqpoint{1.508405in}{1.896854in}}%
\pgfpathcurveto{\pgfqpoint{1.508405in}{1.907904in}}{\pgfqpoint{1.504015in}{1.918503in}}{\pgfqpoint{1.496201in}{1.926317in}}%
\pgfpathcurveto{\pgfqpoint{1.488387in}{1.934131in}}{\pgfqpoint{1.477788in}{1.938521in}}{\pgfqpoint{1.466738in}{1.938521in}}%
\pgfpathcurveto{\pgfqpoint{1.455688in}{1.938521in}}{\pgfqpoint{1.445089in}{1.934131in}}{\pgfqpoint{1.437275in}{1.926317in}}%
\pgfpathcurveto{\pgfqpoint{1.429462in}{1.918503in}}{\pgfqpoint{1.425072in}{1.907904in}}{\pgfqpoint{1.425072in}{1.896854in}}%
\pgfpathcurveto{\pgfqpoint{1.425072in}{1.885804in}}{\pgfqpoint{1.429462in}{1.875205in}}{\pgfqpoint{1.437275in}{1.867391in}}%
\pgfpathcurveto{\pgfqpoint{1.445089in}{1.859578in}}{\pgfqpoint{1.455688in}{1.855187in}}{\pgfqpoint{1.466738in}{1.855187in}}%
\pgfpathclose%
\pgfusepath{stroke,fill}%
\end{pgfscope}%
\begin{pgfscope}%
\pgfpathrectangle{\pgfqpoint{0.600000in}{0.600000in}}{\pgfqpoint{3.900000in}{3.900000in}}%
\pgfusepath{clip}%
\pgfsetbuttcap%
\pgfsetroundjoin%
\definecolor{currentfill}{rgb}{0.121569,0.466667,0.705882}%
\pgfsetfillcolor{currentfill}%
\pgfsetlinewidth{1.003750pt}%
\definecolor{currentstroke}{rgb}{0.121569,0.466667,0.705882}%
\pgfsetstrokecolor{currentstroke}%
\pgfsetdash{}{0pt}%
\pgfpathmoveto{\pgfqpoint{2.987809in}{1.929818in}}%
\pgfpathcurveto{\pgfqpoint{2.998859in}{1.929818in}}{\pgfqpoint{3.009458in}{1.934208in}}{\pgfqpoint{3.017272in}{1.942022in}}%
\pgfpathcurveto{\pgfqpoint{3.025085in}{1.949835in}}{\pgfqpoint{3.029476in}{1.960434in}}{\pgfqpoint{3.029476in}{1.971484in}}%
\pgfpathcurveto{\pgfqpoint{3.029476in}{1.982534in}}{\pgfqpoint{3.025085in}{1.993134in}}{\pgfqpoint{3.017272in}{2.000947in}}%
\pgfpathcurveto{\pgfqpoint{3.009458in}{2.008761in}}{\pgfqpoint{2.998859in}{2.013151in}}{\pgfqpoint{2.987809in}{2.013151in}}%
\pgfpathcurveto{\pgfqpoint{2.976759in}{2.013151in}}{\pgfqpoint{2.966160in}{2.008761in}}{\pgfqpoint{2.958346in}{2.000947in}}%
\pgfpathcurveto{\pgfqpoint{2.950533in}{1.993134in}}{\pgfqpoint{2.946142in}{1.982534in}}{\pgfqpoint{2.946142in}{1.971484in}}%
\pgfpathcurveto{\pgfqpoint{2.946142in}{1.960434in}}{\pgfqpoint{2.950533in}{1.949835in}}{\pgfqpoint{2.958346in}{1.942022in}}%
\pgfpathcurveto{\pgfqpoint{2.966160in}{1.934208in}}{\pgfqpoint{2.976759in}{1.929818in}}{\pgfqpoint{2.987809in}{1.929818in}}%
\pgfpathclose%
\pgfusepath{stroke,fill}%
\end{pgfscope}%
\begin{pgfscope}%
\pgfpathrectangle{\pgfqpoint{0.600000in}{0.600000in}}{\pgfqpoint{3.900000in}{3.900000in}}%
\pgfusepath{clip}%
\pgfsetbuttcap%
\pgfsetroundjoin%
\definecolor{currentfill}{rgb}{0.121569,0.466667,0.705882}%
\pgfsetfillcolor{currentfill}%
\pgfsetlinewidth{1.003750pt}%
\definecolor{currentstroke}{rgb}{0.121569,0.466667,0.705882}%
\pgfsetstrokecolor{currentstroke}%
\pgfsetdash{}{0pt}%
\pgfpathmoveto{\pgfqpoint{3.678947in}{2.592438in}}%
\pgfpathcurveto{\pgfqpoint{3.689997in}{2.592438in}}{\pgfqpoint{3.700596in}{2.596828in}}{\pgfqpoint{3.708409in}{2.604642in}}%
\pgfpathcurveto{\pgfqpoint{3.716223in}{2.612455in}}{\pgfqpoint{3.720613in}{2.623054in}}{\pgfqpoint{3.720613in}{2.634104in}}%
\pgfpathcurveto{\pgfqpoint{3.720613in}{2.645155in}}{\pgfqpoint{3.716223in}{2.655754in}}{\pgfqpoint{3.708409in}{2.663567in}}%
\pgfpathcurveto{\pgfqpoint{3.700596in}{2.671381in}}{\pgfqpoint{3.689997in}{2.675771in}}{\pgfqpoint{3.678947in}{2.675771in}}%
\pgfpathcurveto{\pgfqpoint{3.667897in}{2.675771in}}{\pgfqpoint{3.657298in}{2.671381in}}{\pgfqpoint{3.649484in}{2.663567in}}%
\pgfpathcurveto{\pgfqpoint{3.641670in}{2.655754in}}{\pgfqpoint{3.637280in}{2.645155in}}{\pgfqpoint{3.637280in}{2.634104in}}%
\pgfpathcurveto{\pgfqpoint{3.637280in}{2.623054in}}{\pgfqpoint{3.641670in}{2.612455in}}{\pgfqpoint{3.649484in}{2.604642in}}%
\pgfpathcurveto{\pgfqpoint{3.657298in}{2.596828in}}{\pgfqpoint{3.667897in}{2.592438in}}{\pgfqpoint{3.678947in}{2.592438in}}%
\pgfpathclose%
\pgfusepath{stroke,fill}%
\end{pgfscope}%
\begin{pgfscope}%
\pgfpathrectangle{\pgfqpoint{0.600000in}{0.600000in}}{\pgfqpoint{3.900000in}{3.900000in}}%
\pgfusepath{clip}%
\pgfsetbuttcap%
\pgfsetroundjoin%
\definecolor{currentfill}{rgb}{0.121569,0.466667,0.705882}%
\pgfsetfillcolor{currentfill}%
\pgfsetlinewidth{1.003750pt}%
\definecolor{currentstroke}{rgb}{0.121569,0.466667,0.705882}%
\pgfsetstrokecolor{currentstroke}%
\pgfsetdash{}{0pt}%
\pgfpathmoveto{\pgfqpoint{2.549384in}{1.882341in}}%
\pgfpathcurveto{\pgfqpoint{2.560434in}{1.882341in}}{\pgfqpoint{2.571033in}{1.886731in}}{\pgfqpoint{2.578847in}{1.894545in}}%
\pgfpathcurveto{\pgfqpoint{2.586661in}{1.902358in}}{\pgfqpoint{2.591051in}{1.912957in}}{\pgfqpoint{2.591051in}{1.924008in}}%
\pgfpathcurveto{\pgfqpoint{2.591051in}{1.935058in}}{\pgfqpoint{2.586661in}{1.945657in}}{\pgfqpoint{2.578847in}{1.953470in}}%
\pgfpathcurveto{\pgfqpoint{2.571033in}{1.961284in}}{\pgfqpoint{2.560434in}{1.965674in}}{\pgfqpoint{2.549384in}{1.965674in}}%
\pgfpathcurveto{\pgfqpoint{2.538334in}{1.965674in}}{\pgfqpoint{2.527735in}{1.961284in}}{\pgfqpoint{2.519921in}{1.953470in}}%
\pgfpathcurveto{\pgfqpoint{2.512108in}{1.945657in}}{\pgfqpoint{2.507718in}{1.935058in}}{\pgfqpoint{2.507718in}{1.924008in}}%
\pgfpathcurveto{\pgfqpoint{2.507718in}{1.912957in}}{\pgfqpoint{2.512108in}{1.902358in}}{\pgfqpoint{2.519921in}{1.894545in}}%
\pgfpathcurveto{\pgfqpoint{2.527735in}{1.886731in}}{\pgfqpoint{2.538334in}{1.882341in}}{\pgfqpoint{2.549384in}{1.882341in}}%
\pgfpathclose%
\pgfusepath{stroke,fill}%
\end{pgfscope}%
\begin{pgfscope}%
\pgfpathrectangle{\pgfqpoint{0.600000in}{0.600000in}}{\pgfqpoint{3.900000in}{3.900000in}}%
\pgfusepath{clip}%
\pgfsetbuttcap%
\pgfsetroundjoin%
\definecolor{currentfill}{rgb}{0.121569,0.466667,0.705882}%
\pgfsetfillcolor{currentfill}%
\pgfsetlinewidth{1.003750pt}%
\definecolor{currentstroke}{rgb}{0.121569,0.466667,0.705882}%
\pgfsetstrokecolor{currentstroke}%
\pgfsetdash{}{0pt}%
\pgfpathmoveto{\pgfqpoint{2.692829in}{2.087625in}}%
\pgfpathcurveto{\pgfqpoint{2.703879in}{2.087625in}}{\pgfqpoint{2.714478in}{2.092015in}}{\pgfqpoint{2.722292in}{2.099828in}}%
\pgfpathcurveto{\pgfqpoint{2.730105in}{2.107642in}}{\pgfqpoint{2.734496in}{2.118241in}}{\pgfqpoint{2.734496in}{2.129291in}}%
\pgfpathcurveto{\pgfqpoint{2.734496in}{2.140341in}}{\pgfqpoint{2.730105in}{2.150940in}}{\pgfqpoint{2.722292in}{2.158754in}}%
\pgfpathcurveto{\pgfqpoint{2.714478in}{2.166568in}}{\pgfqpoint{2.703879in}{2.170958in}}{\pgfqpoint{2.692829in}{2.170958in}}%
\pgfpathcurveto{\pgfqpoint{2.681779in}{2.170958in}}{\pgfqpoint{2.671180in}{2.166568in}}{\pgfqpoint{2.663366in}{2.158754in}}%
\pgfpathcurveto{\pgfqpoint{2.655552in}{2.150940in}}{\pgfqpoint{2.651162in}{2.140341in}}{\pgfqpoint{2.651162in}{2.129291in}}%
\pgfpathcurveto{\pgfqpoint{2.651162in}{2.118241in}}{\pgfqpoint{2.655552in}{2.107642in}}{\pgfqpoint{2.663366in}{2.099828in}}%
\pgfpathcurveto{\pgfqpoint{2.671180in}{2.092015in}}{\pgfqpoint{2.681779in}{2.087625in}}{\pgfqpoint{2.692829in}{2.087625in}}%
\pgfpathclose%
\pgfusepath{stroke,fill}%
\end{pgfscope}%
\begin{pgfscope}%
\pgfpathrectangle{\pgfqpoint{0.600000in}{0.600000in}}{\pgfqpoint{3.900000in}{3.900000in}}%
\pgfusepath{clip}%
\pgfsetbuttcap%
\pgfsetroundjoin%
\definecolor{currentfill}{rgb}{0.121569,0.466667,0.705882}%
\pgfsetfillcolor{currentfill}%
\pgfsetlinewidth{1.003750pt}%
\definecolor{currentstroke}{rgb}{0.121569,0.466667,0.705882}%
\pgfsetstrokecolor{currentstroke}%
\pgfsetdash{}{0pt}%
\pgfpathmoveto{\pgfqpoint{2.429441in}{2.671017in}}%
\pgfpathcurveto{\pgfqpoint{2.440491in}{2.671017in}}{\pgfqpoint{2.451090in}{2.675407in}}{\pgfqpoint{2.458904in}{2.683221in}}%
\pgfpathcurveto{\pgfqpoint{2.466717in}{2.691035in}}{\pgfqpoint{2.471108in}{2.701634in}}{\pgfqpoint{2.471108in}{2.712684in}}%
\pgfpathcurveto{\pgfqpoint{2.471108in}{2.723734in}}{\pgfqpoint{2.466717in}{2.734333in}}{\pgfqpoint{2.458904in}{2.742146in}}%
\pgfpathcurveto{\pgfqpoint{2.451090in}{2.749960in}}{\pgfqpoint{2.440491in}{2.754350in}}{\pgfqpoint{2.429441in}{2.754350in}}%
\pgfpathcurveto{\pgfqpoint{2.418391in}{2.754350in}}{\pgfqpoint{2.407792in}{2.749960in}}{\pgfqpoint{2.399978in}{2.742146in}}%
\pgfpathcurveto{\pgfqpoint{2.392165in}{2.734333in}}{\pgfqpoint{2.387774in}{2.723734in}}{\pgfqpoint{2.387774in}{2.712684in}}%
\pgfpathcurveto{\pgfqpoint{2.387774in}{2.701634in}}{\pgfqpoint{2.392165in}{2.691035in}}{\pgfqpoint{2.399978in}{2.683221in}}%
\pgfpathcurveto{\pgfqpoint{2.407792in}{2.675407in}}{\pgfqpoint{2.418391in}{2.671017in}}{\pgfqpoint{2.429441in}{2.671017in}}%
\pgfpathclose%
\pgfusepath{stroke,fill}%
\end{pgfscope}%
\begin{pgfscope}%
\pgfpathrectangle{\pgfqpoint{0.600000in}{0.600000in}}{\pgfqpoint{3.900000in}{3.900000in}}%
\pgfusepath{clip}%
\pgfsetbuttcap%
\pgfsetroundjoin%
\definecolor{currentfill}{rgb}{0.121569,0.466667,0.705882}%
\pgfsetfillcolor{currentfill}%
\pgfsetlinewidth{1.003750pt}%
\definecolor{currentstroke}{rgb}{0.121569,0.466667,0.705882}%
\pgfsetstrokecolor{currentstroke}%
\pgfsetdash{}{0pt}%
\pgfpathmoveto{\pgfqpoint{2.004675in}{2.612218in}}%
\pgfpathcurveto{\pgfqpoint{2.015725in}{2.612218in}}{\pgfqpoint{2.026324in}{2.616608in}}{\pgfqpoint{2.034138in}{2.624422in}}%
\pgfpathcurveto{\pgfqpoint{2.041952in}{2.632235in}}{\pgfqpoint{2.046342in}{2.642834in}}{\pgfqpoint{2.046342in}{2.653884in}}%
\pgfpathcurveto{\pgfqpoint{2.046342in}{2.664935in}}{\pgfqpoint{2.041952in}{2.675534in}}{\pgfqpoint{2.034138in}{2.683347in}}%
\pgfpathcurveto{\pgfqpoint{2.026324in}{2.691161in}}{\pgfqpoint{2.015725in}{2.695551in}}{\pgfqpoint{2.004675in}{2.695551in}}%
\pgfpathcurveto{\pgfqpoint{1.993625in}{2.695551in}}{\pgfqpoint{1.983026in}{2.691161in}}{\pgfqpoint{1.975212in}{2.683347in}}%
\pgfpathcurveto{\pgfqpoint{1.967399in}{2.675534in}}{\pgfqpoint{1.963009in}{2.664935in}}{\pgfqpoint{1.963009in}{2.653884in}}%
\pgfpathcurveto{\pgfqpoint{1.963009in}{2.642834in}}{\pgfqpoint{1.967399in}{2.632235in}}{\pgfqpoint{1.975212in}{2.624422in}}%
\pgfpathcurveto{\pgfqpoint{1.983026in}{2.616608in}}{\pgfqpoint{1.993625in}{2.612218in}}{\pgfqpoint{2.004675in}{2.612218in}}%
\pgfpathclose%
\pgfusepath{stroke,fill}%
\end{pgfscope}%
\begin{pgfscope}%
\pgfpathrectangle{\pgfqpoint{0.600000in}{0.600000in}}{\pgfqpoint{3.900000in}{3.900000in}}%
\pgfusepath{clip}%
\pgfsetbuttcap%
\pgfsetroundjoin%
\definecolor{currentfill}{rgb}{0.121569,0.466667,0.705882}%
\pgfsetfillcolor{currentfill}%
\pgfsetlinewidth{1.003750pt}%
\definecolor{currentstroke}{rgb}{0.121569,0.466667,0.705882}%
\pgfsetstrokecolor{currentstroke}%
\pgfsetdash{}{0pt}%
\pgfpathmoveto{\pgfqpoint{3.438261in}{1.390930in}}%
\pgfpathcurveto{\pgfqpoint{3.449311in}{1.390930in}}{\pgfqpoint{3.459910in}{1.395320in}}{\pgfqpoint{3.467723in}{1.403134in}}%
\pgfpathcurveto{\pgfqpoint{3.475537in}{1.410947in}}{\pgfqpoint{3.479927in}{1.421547in}}{\pgfqpoint{3.479927in}{1.432597in}}%
\pgfpathcurveto{\pgfqpoint{3.479927in}{1.443647in}}{\pgfqpoint{3.475537in}{1.454246in}}{\pgfqpoint{3.467723in}{1.462059in}}%
\pgfpathcurveto{\pgfqpoint{3.459910in}{1.469873in}}{\pgfqpoint{3.449311in}{1.474263in}}{\pgfqpoint{3.438261in}{1.474263in}}%
\pgfpathcurveto{\pgfqpoint{3.427210in}{1.474263in}}{\pgfqpoint{3.416611in}{1.469873in}}{\pgfqpoint{3.408798in}{1.462059in}}%
\pgfpathcurveto{\pgfqpoint{3.400984in}{1.454246in}}{\pgfqpoint{3.396594in}{1.443647in}}{\pgfqpoint{3.396594in}{1.432597in}}%
\pgfpathcurveto{\pgfqpoint{3.396594in}{1.421547in}}{\pgfqpoint{3.400984in}{1.410947in}}{\pgfqpoint{3.408798in}{1.403134in}}%
\pgfpathcurveto{\pgfqpoint{3.416611in}{1.395320in}}{\pgfqpoint{3.427210in}{1.390930in}}{\pgfqpoint{3.438261in}{1.390930in}}%
\pgfpathclose%
\pgfusepath{stroke,fill}%
\end{pgfscope}%
\begin{pgfscope}%
\pgfpathrectangle{\pgfqpoint{0.600000in}{0.600000in}}{\pgfqpoint{3.900000in}{3.900000in}}%
\pgfusepath{clip}%
\pgfsetbuttcap%
\pgfsetroundjoin%
\definecolor{currentfill}{rgb}{0.121569,0.466667,0.705882}%
\pgfsetfillcolor{currentfill}%
\pgfsetlinewidth{1.003750pt}%
\definecolor{currentstroke}{rgb}{0.121569,0.466667,0.705882}%
\pgfsetstrokecolor{currentstroke}%
\pgfsetdash{}{0pt}%
\pgfpathmoveto{\pgfqpoint{1.617885in}{2.156395in}}%
\pgfpathcurveto{\pgfqpoint{1.628935in}{2.156395in}}{\pgfqpoint{1.639534in}{2.160785in}}{\pgfqpoint{1.647348in}{2.168599in}}%
\pgfpathcurveto{\pgfqpoint{1.655161in}{2.176412in}}{\pgfqpoint{1.659552in}{2.187011in}}{\pgfqpoint{1.659552in}{2.198062in}}%
\pgfpathcurveto{\pgfqpoint{1.659552in}{2.209112in}}{\pgfqpoint{1.655161in}{2.219711in}}{\pgfqpoint{1.647348in}{2.227524in}}%
\pgfpathcurveto{\pgfqpoint{1.639534in}{2.235338in}}{\pgfqpoint{1.628935in}{2.239728in}}{\pgfqpoint{1.617885in}{2.239728in}}%
\pgfpathcurveto{\pgfqpoint{1.606835in}{2.239728in}}{\pgfqpoint{1.596236in}{2.235338in}}{\pgfqpoint{1.588422in}{2.227524in}}%
\pgfpathcurveto{\pgfqpoint{1.580609in}{2.219711in}}{\pgfqpoint{1.576218in}{2.209112in}}{\pgfqpoint{1.576218in}{2.198062in}}%
\pgfpathcurveto{\pgfqpoint{1.576218in}{2.187011in}}{\pgfqpoint{1.580609in}{2.176412in}}{\pgfqpoint{1.588422in}{2.168599in}}%
\pgfpathcurveto{\pgfqpoint{1.596236in}{2.160785in}}{\pgfqpoint{1.606835in}{2.156395in}}{\pgfqpoint{1.617885in}{2.156395in}}%
\pgfpathclose%
\pgfusepath{stroke,fill}%
\end{pgfscope}%
\begin{pgfscope}%
\pgfpathrectangle{\pgfqpoint{0.600000in}{0.600000in}}{\pgfqpoint{3.900000in}{3.900000in}}%
\pgfusepath{clip}%
\pgfsetbuttcap%
\pgfsetroundjoin%
\definecolor{currentfill}{rgb}{0.121569,0.466667,0.705882}%
\pgfsetfillcolor{currentfill}%
\pgfsetlinewidth{1.003750pt}%
\definecolor{currentstroke}{rgb}{0.121569,0.466667,0.705882}%
\pgfsetstrokecolor{currentstroke}%
\pgfsetdash{}{0pt}%
\pgfpathmoveto{\pgfqpoint{2.795882in}{2.892169in}}%
\pgfpathcurveto{\pgfqpoint{2.806932in}{2.892169in}}{\pgfqpoint{2.817531in}{2.896559in}}{\pgfqpoint{2.825345in}{2.904372in}}%
\pgfpathcurveto{\pgfqpoint{2.833159in}{2.912186in}}{\pgfqpoint{2.837549in}{2.922785in}}{\pgfqpoint{2.837549in}{2.933835in}}%
\pgfpathcurveto{\pgfqpoint{2.837549in}{2.944885in}}{\pgfqpoint{2.833159in}{2.955484in}}{\pgfqpoint{2.825345in}{2.963298in}}%
\pgfpathcurveto{\pgfqpoint{2.817531in}{2.971112in}}{\pgfqpoint{2.806932in}{2.975502in}}{\pgfqpoint{2.795882in}{2.975502in}}%
\pgfpathcurveto{\pgfqpoint{2.784832in}{2.975502in}}{\pgfqpoint{2.774233in}{2.971112in}}{\pgfqpoint{2.766419in}{2.963298in}}%
\pgfpathcurveto{\pgfqpoint{2.758606in}{2.955484in}}{\pgfqpoint{2.754216in}{2.944885in}}{\pgfqpoint{2.754216in}{2.933835in}}%
\pgfpathcurveto{\pgfqpoint{2.754216in}{2.922785in}}{\pgfqpoint{2.758606in}{2.912186in}}{\pgfqpoint{2.766419in}{2.904372in}}%
\pgfpathcurveto{\pgfqpoint{2.774233in}{2.896559in}}{\pgfqpoint{2.784832in}{2.892169in}}{\pgfqpoint{2.795882in}{2.892169in}}%
\pgfpathclose%
\pgfusepath{stroke,fill}%
\end{pgfscope}%
\begin{pgfscope}%
\pgfpathrectangle{\pgfqpoint{0.600000in}{0.600000in}}{\pgfqpoint{3.900000in}{3.900000in}}%
\pgfusepath{clip}%
\pgfsetbuttcap%
\pgfsetroundjoin%
\definecolor{currentfill}{rgb}{0.121569,0.466667,0.705882}%
\pgfsetfillcolor{currentfill}%
\pgfsetlinewidth{1.003750pt}%
\definecolor{currentstroke}{rgb}{0.121569,0.466667,0.705882}%
\pgfsetstrokecolor{currentstroke}%
\pgfsetdash{}{0pt}%
\pgfpathmoveto{\pgfqpoint{2.167665in}{1.319039in}}%
\pgfpathcurveto{\pgfqpoint{2.178715in}{1.319039in}}{\pgfqpoint{2.189314in}{1.323430in}}{\pgfqpoint{2.197127in}{1.331243in}}%
\pgfpathcurveto{\pgfqpoint{2.204941in}{1.339057in}}{\pgfqpoint{2.209331in}{1.349656in}}{\pgfqpoint{2.209331in}{1.360706in}}%
\pgfpathcurveto{\pgfqpoint{2.209331in}{1.371756in}}{\pgfqpoint{2.204941in}{1.382355in}}{\pgfqpoint{2.197127in}{1.390169in}}%
\pgfpathcurveto{\pgfqpoint{2.189314in}{1.397983in}}{\pgfqpoint{2.178715in}{1.402373in}}{\pgfqpoint{2.167665in}{1.402373in}}%
\pgfpathcurveto{\pgfqpoint{2.156614in}{1.402373in}}{\pgfqpoint{2.146015in}{1.397983in}}{\pgfqpoint{2.138202in}{1.390169in}}%
\pgfpathcurveto{\pgfqpoint{2.130388in}{1.382355in}}{\pgfqpoint{2.125998in}{1.371756in}}{\pgfqpoint{2.125998in}{1.360706in}}%
\pgfpathcurveto{\pgfqpoint{2.125998in}{1.349656in}}{\pgfqpoint{2.130388in}{1.339057in}}{\pgfqpoint{2.138202in}{1.331243in}}%
\pgfpathcurveto{\pgfqpoint{2.146015in}{1.323430in}}{\pgfqpoint{2.156614in}{1.319039in}}{\pgfqpoint{2.167665in}{1.319039in}}%
\pgfpathclose%
\pgfusepath{stroke,fill}%
\end{pgfscope}%
\begin{pgfscope}%
\pgfpathrectangle{\pgfqpoint{0.600000in}{0.600000in}}{\pgfqpoint{3.900000in}{3.900000in}}%
\pgfusepath{clip}%
\pgfsetbuttcap%
\pgfsetroundjoin%
\definecolor{currentfill}{rgb}{0.121569,0.466667,0.705882}%
\pgfsetfillcolor{currentfill}%
\pgfsetlinewidth{1.003750pt}%
\definecolor{currentstroke}{rgb}{0.121569,0.466667,0.705882}%
\pgfsetstrokecolor{currentstroke}%
\pgfsetdash{}{0pt}%
\pgfpathmoveto{\pgfqpoint{2.202699in}{1.794113in}}%
\pgfpathcurveto{\pgfqpoint{2.213749in}{1.794113in}}{\pgfqpoint{2.224349in}{1.798504in}}{\pgfqpoint{2.232162in}{1.806317in}}%
\pgfpathcurveto{\pgfqpoint{2.239976in}{1.814131in}}{\pgfqpoint{2.244366in}{1.824730in}}{\pgfqpoint{2.244366in}{1.835780in}}%
\pgfpathcurveto{\pgfqpoint{2.244366in}{1.846830in}}{\pgfqpoint{2.239976in}{1.857429in}}{\pgfqpoint{2.232162in}{1.865243in}}%
\pgfpathcurveto{\pgfqpoint{2.224349in}{1.873056in}}{\pgfqpoint{2.213749in}{1.877447in}}{\pgfqpoint{2.202699in}{1.877447in}}%
\pgfpathcurveto{\pgfqpoint{2.191649in}{1.877447in}}{\pgfqpoint{2.181050in}{1.873056in}}{\pgfqpoint{2.173237in}{1.865243in}}%
\pgfpathcurveto{\pgfqpoint{2.165423in}{1.857429in}}{\pgfqpoint{2.161033in}{1.846830in}}{\pgfqpoint{2.161033in}{1.835780in}}%
\pgfpathcurveto{\pgfqpoint{2.161033in}{1.824730in}}{\pgfqpoint{2.165423in}{1.814131in}}{\pgfqpoint{2.173237in}{1.806317in}}%
\pgfpathcurveto{\pgfqpoint{2.181050in}{1.798504in}}{\pgfqpoint{2.191649in}{1.794113in}}{\pgfqpoint{2.202699in}{1.794113in}}%
\pgfpathclose%
\pgfusepath{stroke,fill}%
\end{pgfscope}%
\begin{pgfscope}%
\pgfpathrectangle{\pgfqpoint{0.600000in}{0.600000in}}{\pgfqpoint{3.900000in}{3.900000in}}%
\pgfusepath{clip}%
\pgfsetbuttcap%
\pgfsetroundjoin%
\definecolor{currentfill}{rgb}{0.121569,0.466667,0.705882}%
\pgfsetfillcolor{currentfill}%
\pgfsetlinewidth{1.003750pt}%
\definecolor{currentstroke}{rgb}{0.121569,0.466667,0.705882}%
\pgfsetstrokecolor{currentstroke}%
\pgfsetdash{}{0pt}%
\pgfpathmoveto{\pgfqpoint{2.056682in}{2.675053in}}%
\pgfpathcurveto{\pgfqpoint{2.067732in}{2.675053in}}{\pgfqpoint{2.078331in}{2.679443in}}{\pgfqpoint{2.086145in}{2.687257in}}%
\pgfpathcurveto{\pgfqpoint{2.093959in}{2.695071in}}{\pgfqpoint{2.098349in}{2.705670in}}{\pgfqpoint{2.098349in}{2.716720in}}%
\pgfpathcurveto{\pgfqpoint{2.098349in}{2.727770in}}{\pgfqpoint{2.093959in}{2.738369in}}{\pgfqpoint{2.086145in}{2.746183in}}%
\pgfpathcurveto{\pgfqpoint{2.078331in}{2.753996in}}{\pgfqpoint{2.067732in}{2.758387in}}{\pgfqpoint{2.056682in}{2.758387in}}%
\pgfpathcurveto{\pgfqpoint{2.045632in}{2.758387in}}{\pgfqpoint{2.035033in}{2.753996in}}{\pgfqpoint{2.027219in}{2.746183in}}%
\pgfpathcurveto{\pgfqpoint{2.019406in}{2.738369in}}{\pgfqpoint{2.015015in}{2.727770in}}{\pgfqpoint{2.015015in}{2.716720in}}%
\pgfpathcurveto{\pgfqpoint{2.015015in}{2.705670in}}{\pgfqpoint{2.019406in}{2.695071in}}{\pgfqpoint{2.027219in}{2.687257in}}%
\pgfpathcurveto{\pgfqpoint{2.035033in}{2.679443in}}{\pgfqpoint{2.045632in}{2.675053in}}{\pgfqpoint{2.056682in}{2.675053in}}%
\pgfpathclose%
\pgfusepath{stroke,fill}%
\end{pgfscope}%
\begin{pgfscope}%
\pgfpathrectangle{\pgfqpoint{0.600000in}{0.600000in}}{\pgfqpoint{3.900000in}{3.900000in}}%
\pgfusepath{clip}%
\pgfsetbuttcap%
\pgfsetroundjoin%
\definecolor{currentfill}{rgb}{0.121569,0.466667,0.705882}%
\pgfsetfillcolor{currentfill}%
\pgfsetlinewidth{1.003750pt}%
\definecolor{currentstroke}{rgb}{0.121569,0.466667,0.705882}%
\pgfsetstrokecolor{currentstroke}%
\pgfsetdash{}{0pt}%
\pgfpathmoveto{\pgfqpoint{3.772428in}{2.394056in}}%
\pgfpathcurveto{\pgfqpoint{3.783478in}{2.394056in}}{\pgfqpoint{3.794077in}{2.398446in}}{\pgfqpoint{3.801891in}{2.406260in}}%
\pgfpathcurveto{\pgfqpoint{3.809704in}{2.414073in}}{\pgfqpoint{3.814095in}{2.424672in}}{\pgfqpoint{3.814095in}{2.435723in}}%
\pgfpathcurveto{\pgfqpoint{3.814095in}{2.446773in}}{\pgfqpoint{3.809704in}{2.457372in}}{\pgfqpoint{3.801891in}{2.465185in}}%
\pgfpathcurveto{\pgfqpoint{3.794077in}{2.472999in}}{\pgfqpoint{3.783478in}{2.477389in}}{\pgfqpoint{3.772428in}{2.477389in}}%
\pgfpathcurveto{\pgfqpoint{3.761378in}{2.477389in}}{\pgfqpoint{3.750779in}{2.472999in}}{\pgfqpoint{3.742965in}{2.465185in}}%
\pgfpathcurveto{\pgfqpoint{3.735152in}{2.457372in}}{\pgfqpoint{3.730761in}{2.446773in}}{\pgfqpoint{3.730761in}{2.435723in}}%
\pgfpathcurveto{\pgfqpoint{3.730761in}{2.424672in}}{\pgfqpoint{3.735152in}{2.414073in}}{\pgfqpoint{3.742965in}{2.406260in}}%
\pgfpathcurveto{\pgfqpoint{3.750779in}{2.398446in}}{\pgfqpoint{3.761378in}{2.394056in}}{\pgfqpoint{3.772428in}{2.394056in}}%
\pgfpathclose%
\pgfusepath{stroke,fill}%
\end{pgfscope}%
\begin{pgfscope}%
\pgfpathrectangle{\pgfqpoint{0.600000in}{0.600000in}}{\pgfqpoint{3.900000in}{3.900000in}}%
\pgfusepath{clip}%
\pgfsetbuttcap%
\pgfsetroundjoin%
\definecolor{currentfill}{rgb}{0.121569,0.466667,0.705882}%
\pgfsetfillcolor{currentfill}%
\pgfsetlinewidth{1.003750pt}%
\definecolor{currentstroke}{rgb}{0.121569,0.466667,0.705882}%
\pgfsetstrokecolor{currentstroke}%
\pgfsetdash{}{0pt}%
\pgfpathmoveto{\pgfqpoint{2.598766in}{3.283507in}}%
\pgfpathcurveto{\pgfqpoint{2.609816in}{3.283507in}}{\pgfqpoint{2.620415in}{3.287897in}}{\pgfqpoint{2.628228in}{3.295711in}}%
\pgfpathcurveto{\pgfqpoint{2.636042in}{3.303524in}}{\pgfqpoint{2.640432in}{3.314123in}}{\pgfqpoint{2.640432in}{3.325173in}}%
\pgfpathcurveto{\pgfqpoint{2.640432in}{3.336224in}}{\pgfqpoint{2.636042in}{3.346823in}}{\pgfqpoint{2.628228in}{3.354636in}}%
\pgfpathcurveto{\pgfqpoint{2.620415in}{3.362450in}}{\pgfqpoint{2.609816in}{3.366840in}}{\pgfqpoint{2.598766in}{3.366840in}}%
\pgfpathcurveto{\pgfqpoint{2.587715in}{3.366840in}}{\pgfqpoint{2.577116in}{3.362450in}}{\pgfqpoint{2.569303in}{3.354636in}}%
\pgfpathcurveto{\pgfqpoint{2.561489in}{3.346823in}}{\pgfqpoint{2.557099in}{3.336224in}}{\pgfqpoint{2.557099in}{3.325173in}}%
\pgfpathcurveto{\pgfqpoint{2.557099in}{3.314123in}}{\pgfqpoint{2.561489in}{3.303524in}}{\pgfqpoint{2.569303in}{3.295711in}}%
\pgfpathcurveto{\pgfqpoint{2.577116in}{3.287897in}}{\pgfqpoint{2.587715in}{3.283507in}}{\pgfqpoint{2.598766in}{3.283507in}}%
\pgfpathclose%
\pgfusepath{stroke,fill}%
\end{pgfscope}%
\begin{pgfscope}%
\pgfpathrectangle{\pgfqpoint{0.600000in}{0.600000in}}{\pgfqpoint{3.900000in}{3.900000in}}%
\pgfusepath{clip}%
\pgfsetbuttcap%
\pgfsetroundjoin%
\definecolor{currentfill}{rgb}{0.121569,0.466667,0.705882}%
\pgfsetfillcolor{currentfill}%
\pgfsetlinewidth{1.003750pt}%
\definecolor{currentstroke}{rgb}{0.121569,0.466667,0.705882}%
\pgfsetstrokecolor{currentstroke}%
\pgfsetdash{}{0pt}%
\pgfpathmoveto{\pgfqpoint{1.701297in}{2.223999in}}%
\pgfpathcurveto{\pgfqpoint{1.712347in}{2.223999in}}{\pgfqpoint{1.722946in}{2.228389in}}{\pgfqpoint{1.730760in}{2.236203in}}%
\pgfpathcurveto{\pgfqpoint{1.738574in}{2.244017in}}{\pgfqpoint{1.742964in}{2.254616in}}{\pgfqpoint{1.742964in}{2.265666in}}%
\pgfpathcurveto{\pgfqpoint{1.742964in}{2.276716in}}{\pgfqpoint{1.738574in}{2.287315in}}{\pgfqpoint{1.730760in}{2.295129in}}%
\pgfpathcurveto{\pgfqpoint{1.722946in}{2.302942in}}{\pgfqpoint{1.712347in}{2.307332in}}{\pgfqpoint{1.701297in}{2.307332in}}%
\pgfpathcurveto{\pgfqpoint{1.690247in}{2.307332in}}{\pgfqpoint{1.679648in}{2.302942in}}{\pgfqpoint{1.671834in}{2.295129in}}%
\pgfpathcurveto{\pgfqpoint{1.664021in}{2.287315in}}{\pgfqpoint{1.659631in}{2.276716in}}{\pgfqpoint{1.659631in}{2.265666in}}%
\pgfpathcurveto{\pgfqpoint{1.659631in}{2.254616in}}{\pgfqpoint{1.664021in}{2.244017in}}{\pgfqpoint{1.671834in}{2.236203in}}%
\pgfpathcurveto{\pgfqpoint{1.679648in}{2.228389in}}{\pgfqpoint{1.690247in}{2.223999in}}{\pgfqpoint{1.701297in}{2.223999in}}%
\pgfpathclose%
\pgfusepath{stroke,fill}%
\end{pgfscope}%
\begin{pgfscope}%
\pgfpathrectangle{\pgfqpoint{0.600000in}{0.600000in}}{\pgfqpoint{3.900000in}{3.900000in}}%
\pgfusepath{clip}%
\pgfsetbuttcap%
\pgfsetroundjoin%
\definecolor{currentfill}{rgb}{0.121569,0.466667,0.705882}%
\pgfsetfillcolor{currentfill}%
\pgfsetlinewidth{1.003750pt}%
\definecolor{currentstroke}{rgb}{0.121569,0.466667,0.705882}%
\pgfsetstrokecolor{currentstroke}%
\pgfsetdash{}{0pt}%
\pgfpathmoveto{\pgfqpoint{2.357867in}{2.377657in}}%
\pgfpathcurveto{\pgfqpoint{2.368917in}{2.377657in}}{\pgfqpoint{2.379516in}{2.382048in}}{\pgfqpoint{2.387330in}{2.389861in}}%
\pgfpathcurveto{\pgfqpoint{2.395143in}{2.397675in}}{\pgfqpoint{2.399534in}{2.408274in}}{\pgfqpoint{2.399534in}{2.419324in}}%
\pgfpathcurveto{\pgfqpoint{2.399534in}{2.430374in}}{\pgfqpoint{2.395143in}{2.440973in}}{\pgfqpoint{2.387330in}{2.448787in}}%
\pgfpathcurveto{\pgfqpoint{2.379516in}{2.456601in}}{\pgfqpoint{2.368917in}{2.460991in}}{\pgfqpoint{2.357867in}{2.460991in}}%
\pgfpathcurveto{\pgfqpoint{2.346817in}{2.460991in}}{\pgfqpoint{2.336218in}{2.456601in}}{\pgfqpoint{2.328404in}{2.448787in}}%
\pgfpathcurveto{\pgfqpoint{2.320591in}{2.440973in}}{\pgfqpoint{2.316200in}{2.430374in}}{\pgfqpoint{2.316200in}{2.419324in}}%
\pgfpathcurveto{\pgfqpoint{2.316200in}{2.408274in}}{\pgfqpoint{2.320591in}{2.397675in}}{\pgfqpoint{2.328404in}{2.389861in}}%
\pgfpathcurveto{\pgfqpoint{2.336218in}{2.382048in}}{\pgfqpoint{2.346817in}{2.377657in}}{\pgfqpoint{2.357867in}{2.377657in}}%
\pgfpathclose%
\pgfusepath{stroke,fill}%
\end{pgfscope}%
\begin{pgfscope}%
\pgfpathrectangle{\pgfqpoint{0.600000in}{0.600000in}}{\pgfqpoint{3.900000in}{3.900000in}}%
\pgfusepath{clip}%
\pgfsetbuttcap%
\pgfsetroundjoin%
\definecolor{currentfill}{rgb}{0.121569,0.466667,0.705882}%
\pgfsetfillcolor{currentfill}%
\pgfsetlinewidth{1.003750pt}%
\definecolor{currentstroke}{rgb}{0.121569,0.466667,0.705882}%
\pgfsetstrokecolor{currentstroke}%
\pgfsetdash{}{0pt}%
\pgfpathmoveto{\pgfqpoint{2.998011in}{2.598426in}}%
\pgfpathcurveto{\pgfqpoint{3.009061in}{2.598426in}}{\pgfqpoint{3.019660in}{2.602817in}}{\pgfqpoint{3.027474in}{2.610630in}}%
\pgfpathcurveto{\pgfqpoint{3.035287in}{2.618444in}}{\pgfqpoint{3.039678in}{2.629043in}}{\pgfqpoint{3.039678in}{2.640093in}}%
\pgfpathcurveto{\pgfqpoint{3.039678in}{2.651143in}}{\pgfqpoint{3.035287in}{2.661742in}}{\pgfqpoint{3.027474in}{2.669556in}}%
\pgfpathcurveto{\pgfqpoint{3.019660in}{2.677369in}}{\pgfqpoint{3.009061in}{2.681760in}}{\pgfqpoint{2.998011in}{2.681760in}}%
\pgfpathcurveto{\pgfqpoint{2.986961in}{2.681760in}}{\pgfqpoint{2.976362in}{2.677369in}}{\pgfqpoint{2.968548in}{2.669556in}}%
\pgfpathcurveto{\pgfqpoint{2.960735in}{2.661742in}}{\pgfqpoint{2.956344in}{2.651143in}}{\pgfqpoint{2.956344in}{2.640093in}}%
\pgfpathcurveto{\pgfqpoint{2.956344in}{2.629043in}}{\pgfqpoint{2.960735in}{2.618444in}}{\pgfqpoint{2.968548in}{2.610630in}}%
\pgfpathcurveto{\pgfqpoint{2.976362in}{2.602817in}}{\pgfqpoint{2.986961in}{2.598426in}}{\pgfqpoint{2.998011in}{2.598426in}}%
\pgfpathclose%
\pgfusepath{stroke,fill}%
\end{pgfscope}%
\begin{pgfscope}%
\pgfpathrectangle{\pgfqpoint{0.600000in}{0.600000in}}{\pgfqpoint{3.900000in}{3.900000in}}%
\pgfusepath{clip}%
\pgfsetbuttcap%
\pgfsetroundjoin%
\definecolor{currentfill}{rgb}{0.121569,0.466667,0.705882}%
\pgfsetfillcolor{currentfill}%
\pgfsetlinewidth{1.003750pt}%
\definecolor{currentstroke}{rgb}{0.121569,0.466667,0.705882}%
\pgfsetstrokecolor{currentstroke}%
\pgfsetdash{}{0pt}%
\pgfpathmoveto{\pgfqpoint{2.864999in}{2.792074in}}%
\pgfpathcurveto{\pgfqpoint{2.876049in}{2.792074in}}{\pgfqpoint{2.886648in}{2.796464in}}{\pgfqpoint{2.894461in}{2.804278in}}%
\pgfpathcurveto{\pgfqpoint{2.902275in}{2.812092in}}{\pgfqpoint{2.906665in}{2.822691in}}{\pgfqpoint{2.906665in}{2.833741in}}%
\pgfpathcurveto{\pgfqpoint{2.906665in}{2.844791in}}{\pgfqpoint{2.902275in}{2.855390in}}{\pgfqpoint{2.894461in}{2.863203in}}%
\pgfpathcurveto{\pgfqpoint{2.886648in}{2.871017in}}{\pgfqpoint{2.876049in}{2.875407in}}{\pgfqpoint{2.864999in}{2.875407in}}%
\pgfpathcurveto{\pgfqpoint{2.853948in}{2.875407in}}{\pgfqpoint{2.843349in}{2.871017in}}{\pgfqpoint{2.835536in}{2.863203in}}%
\pgfpathcurveto{\pgfqpoint{2.827722in}{2.855390in}}{\pgfqpoint{2.823332in}{2.844791in}}{\pgfqpoint{2.823332in}{2.833741in}}%
\pgfpathcurveto{\pgfqpoint{2.823332in}{2.822691in}}{\pgfqpoint{2.827722in}{2.812092in}}{\pgfqpoint{2.835536in}{2.804278in}}%
\pgfpathcurveto{\pgfqpoint{2.843349in}{2.796464in}}{\pgfqpoint{2.853948in}{2.792074in}}{\pgfqpoint{2.864999in}{2.792074in}}%
\pgfpathclose%
\pgfusepath{stroke,fill}%
\end{pgfscope}%
\begin{pgfscope}%
\pgfpathrectangle{\pgfqpoint{0.600000in}{0.600000in}}{\pgfqpoint{3.900000in}{3.900000in}}%
\pgfusepath{clip}%
\pgfsetbuttcap%
\pgfsetroundjoin%
\definecolor{currentfill}{rgb}{0.121569,0.466667,0.705882}%
\pgfsetfillcolor{currentfill}%
\pgfsetlinewidth{1.003750pt}%
\definecolor{currentstroke}{rgb}{0.121569,0.466667,0.705882}%
\pgfsetstrokecolor{currentstroke}%
\pgfsetdash{}{0pt}%
\pgfpathmoveto{\pgfqpoint{3.211485in}{2.006794in}}%
\pgfpathcurveto{\pgfqpoint{3.222535in}{2.006794in}}{\pgfqpoint{3.233134in}{2.011185in}}{\pgfqpoint{3.240948in}{2.018998in}}%
\pgfpathcurveto{\pgfqpoint{3.248761in}{2.026812in}}{\pgfqpoint{3.253152in}{2.037411in}}{\pgfqpoint{3.253152in}{2.048461in}}%
\pgfpathcurveto{\pgfqpoint{3.253152in}{2.059511in}}{\pgfqpoint{3.248761in}{2.070110in}}{\pgfqpoint{3.240948in}{2.077924in}}%
\pgfpathcurveto{\pgfqpoint{3.233134in}{2.085737in}}{\pgfqpoint{3.222535in}{2.090128in}}{\pgfqpoint{3.211485in}{2.090128in}}%
\pgfpathcurveto{\pgfqpoint{3.200435in}{2.090128in}}{\pgfqpoint{3.189836in}{2.085737in}}{\pgfqpoint{3.182022in}{2.077924in}}%
\pgfpathcurveto{\pgfqpoint{3.174209in}{2.070110in}}{\pgfqpoint{3.169818in}{2.059511in}}{\pgfqpoint{3.169818in}{2.048461in}}%
\pgfpathcurveto{\pgfqpoint{3.169818in}{2.037411in}}{\pgfqpoint{3.174209in}{2.026812in}}{\pgfqpoint{3.182022in}{2.018998in}}%
\pgfpathcurveto{\pgfqpoint{3.189836in}{2.011185in}}{\pgfqpoint{3.200435in}{2.006794in}}{\pgfqpoint{3.211485in}{2.006794in}}%
\pgfpathclose%
\pgfusepath{stroke,fill}%
\end{pgfscope}%
\begin{pgfscope}%
\pgfpathrectangle{\pgfqpoint{0.600000in}{0.600000in}}{\pgfqpoint{3.900000in}{3.900000in}}%
\pgfusepath{clip}%
\pgfsetbuttcap%
\pgfsetroundjoin%
\definecolor{currentfill}{rgb}{0.121569,0.466667,0.705882}%
\pgfsetfillcolor{currentfill}%
\pgfsetlinewidth{1.003750pt}%
\definecolor{currentstroke}{rgb}{0.121569,0.466667,0.705882}%
\pgfsetstrokecolor{currentstroke}%
\pgfsetdash{}{0pt}%
\pgfpathmoveto{\pgfqpoint{2.814335in}{2.905679in}}%
\pgfpathcurveto{\pgfqpoint{2.825385in}{2.905679in}}{\pgfqpoint{2.835984in}{2.910069in}}{\pgfqpoint{2.843798in}{2.917883in}}%
\pgfpathcurveto{\pgfqpoint{2.851612in}{2.925697in}}{\pgfqpoint{2.856002in}{2.936296in}}{\pgfqpoint{2.856002in}{2.947346in}}%
\pgfpathcurveto{\pgfqpoint{2.856002in}{2.958396in}}{\pgfqpoint{2.851612in}{2.968995in}}{\pgfqpoint{2.843798in}{2.976809in}}%
\pgfpathcurveto{\pgfqpoint{2.835984in}{2.984622in}}{\pgfqpoint{2.825385in}{2.989012in}}{\pgfqpoint{2.814335in}{2.989012in}}%
\pgfpathcurveto{\pgfqpoint{2.803285in}{2.989012in}}{\pgfqpoint{2.792686in}{2.984622in}}{\pgfqpoint{2.784872in}{2.976809in}}%
\pgfpathcurveto{\pgfqpoint{2.777059in}{2.968995in}}{\pgfqpoint{2.772669in}{2.958396in}}{\pgfqpoint{2.772669in}{2.947346in}}%
\pgfpathcurveto{\pgfqpoint{2.772669in}{2.936296in}}{\pgfqpoint{2.777059in}{2.925697in}}{\pgfqpoint{2.784872in}{2.917883in}}%
\pgfpathcurveto{\pgfqpoint{2.792686in}{2.910069in}}{\pgfqpoint{2.803285in}{2.905679in}}{\pgfqpoint{2.814335in}{2.905679in}}%
\pgfpathclose%
\pgfusepath{stroke,fill}%
\end{pgfscope}%
\begin{pgfscope}%
\pgfpathrectangle{\pgfqpoint{0.600000in}{0.600000in}}{\pgfqpoint{3.900000in}{3.900000in}}%
\pgfusepath{clip}%
\pgfsetbuttcap%
\pgfsetroundjoin%
\definecolor{currentfill}{rgb}{0.121569,0.466667,0.705882}%
\pgfsetfillcolor{currentfill}%
\pgfsetlinewidth{1.003750pt}%
\definecolor{currentstroke}{rgb}{0.121569,0.466667,0.705882}%
\pgfsetstrokecolor{currentstroke}%
\pgfsetdash{}{0pt}%
\pgfpathmoveto{\pgfqpoint{1.949587in}{1.843121in}}%
\pgfpathcurveto{\pgfqpoint{1.960637in}{1.843121in}}{\pgfqpoint{1.971236in}{1.847511in}}{\pgfqpoint{1.979050in}{1.855325in}}%
\pgfpathcurveto{\pgfqpoint{1.986863in}{1.863138in}}{\pgfqpoint{1.991254in}{1.873738in}}{\pgfqpoint{1.991254in}{1.884788in}}%
\pgfpathcurveto{\pgfqpoint{1.991254in}{1.895838in}}{\pgfqpoint{1.986863in}{1.906437in}}{\pgfqpoint{1.979050in}{1.914250in}}%
\pgfpathcurveto{\pgfqpoint{1.971236in}{1.922064in}}{\pgfqpoint{1.960637in}{1.926454in}}{\pgfqpoint{1.949587in}{1.926454in}}%
\pgfpathcurveto{\pgfqpoint{1.938537in}{1.926454in}}{\pgfqpoint{1.927938in}{1.922064in}}{\pgfqpoint{1.920124in}{1.914250in}}%
\pgfpathcurveto{\pgfqpoint{1.912311in}{1.906437in}}{\pgfqpoint{1.907920in}{1.895838in}}{\pgfqpoint{1.907920in}{1.884788in}}%
\pgfpathcurveto{\pgfqpoint{1.907920in}{1.873738in}}{\pgfqpoint{1.912311in}{1.863138in}}{\pgfqpoint{1.920124in}{1.855325in}}%
\pgfpathcurveto{\pgfqpoint{1.927938in}{1.847511in}}{\pgfqpoint{1.938537in}{1.843121in}}{\pgfqpoint{1.949587in}{1.843121in}}%
\pgfpathclose%
\pgfusepath{stroke,fill}%
\end{pgfscope}%
\begin{pgfscope}%
\pgfpathrectangle{\pgfqpoint{0.600000in}{0.600000in}}{\pgfqpoint{3.900000in}{3.900000in}}%
\pgfusepath{clip}%
\pgfsetbuttcap%
\pgfsetroundjoin%
\definecolor{currentfill}{rgb}{0.121569,0.466667,0.705882}%
\pgfsetfillcolor{currentfill}%
\pgfsetlinewidth{1.003750pt}%
\definecolor{currentstroke}{rgb}{0.121569,0.466667,0.705882}%
\pgfsetstrokecolor{currentstroke}%
\pgfsetdash{}{0pt}%
\pgfpathmoveto{\pgfqpoint{1.485956in}{2.711652in}}%
\pgfpathcurveto{\pgfqpoint{1.497006in}{2.711652in}}{\pgfqpoint{1.507605in}{2.716043in}}{\pgfqpoint{1.515418in}{2.723856in}}%
\pgfpathcurveto{\pgfqpoint{1.523232in}{2.731670in}}{\pgfqpoint{1.527622in}{2.742269in}}{\pgfqpoint{1.527622in}{2.753319in}}%
\pgfpathcurveto{\pgfqpoint{1.527622in}{2.764369in}}{\pgfqpoint{1.523232in}{2.774968in}}{\pgfqpoint{1.515418in}{2.782782in}}%
\pgfpathcurveto{\pgfqpoint{1.507605in}{2.790596in}}{\pgfqpoint{1.497006in}{2.794986in}}{\pgfqpoint{1.485956in}{2.794986in}}%
\pgfpathcurveto{\pgfqpoint{1.474905in}{2.794986in}}{\pgfqpoint{1.464306in}{2.790596in}}{\pgfqpoint{1.456493in}{2.782782in}}%
\pgfpathcurveto{\pgfqpoint{1.448679in}{2.774968in}}{\pgfqpoint{1.444289in}{2.764369in}}{\pgfqpoint{1.444289in}{2.753319in}}%
\pgfpathcurveto{\pgfqpoint{1.444289in}{2.742269in}}{\pgfqpoint{1.448679in}{2.731670in}}{\pgfqpoint{1.456493in}{2.723856in}}%
\pgfpathcurveto{\pgfqpoint{1.464306in}{2.716043in}}{\pgfqpoint{1.474905in}{2.711652in}}{\pgfqpoint{1.485956in}{2.711652in}}%
\pgfpathclose%
\pgfusepath{stroke,fill}%
\end{pgfscope}%
\begin{pgfscope}%
\pgfpathrectangle{\pgfqpoint{0.600000in}{0.600000in}}{\pgfqpoint{3.900000in}{3.900000in}}%
\pgfusepath{clip}%
\pgfsetbuttcap%
\pgfsetroundjoin%
\definecolor{currentfill}{rgb}{0.121569,0.466667,0.705882}%
\pgfsetfillcolor{currentfill}%
\pgfsetlinewidth{1.003750pt}%
\definecolor{currentstroke}{rgb}{0.121569,0.466667,0.705882}%
\pgfsetstrokecolor{currentstroke}%
\pgfsetdash{}{0pt}%
\pgfpathmoveto{\pgfqpoint{2.568371in}{2.344890in}}%
\pgfpathcurveto{\pgfqpoint{2.579421in}{2.344890in}}{\pgfqpoint{2.590020in}{2.349280in}}{\pgfqpoint{2.597834in}{2.357094in}}%
\pgfpathcurveto{\pgfqpoint{2.605648in}{2.364908in}}{\pgfqpoint{2.610038in}{2.375507in}}{\pgfqpoint{2.610038in}{2.386557in}}%
\pgfpathcurveto{\pgfqpoint{2.610038in}{2.397607in}}{\pgfqpoint{2.605648in}{2.408206in}}{\pgfqpoint{2.597834in}{2.416020in}}%
\pgfpathcurveto{\pgfqpoint{2.590020in}{2.423833in}}{\pgfqpoint{2.579421in}{2.428223in}}{\pgfqpoint{2.568371in}{2.428223in}}%
\pgfpathcurveto{\pgfqpoint{2.557321in}{2.428223in}}{\pgfqpoint{2.546722in}{2.423833in}}{\pgfqpoint{2.538908in}{2.416020in}}%
\pgfpathcurveto{\pgfqpoint{2.531095in}{2.408206in}}{\pgfqpoint{2.526705in}{2.397607in}}{\pgfqpoint{2.526705in}{2.386557in}}%
\pgfpathcurveto{\pgfqpoint{2.526705in}{2.375507in}}{\pgfqpoint{2.531095in}{2.364908in}}{\pgfqpoint{2.538908in}{2.357094in}}%
\pgfpathcurveto{\pgfqpoint{2.546722in}{2.349280in}}{\pgfqpoint{2.557321in}{2.344890in}}{\pgfqpoint{2.568371in}{2.344890in}}%
\pgfpathclose%
\pgfusepath{stroke,fill}%
\end{pgfscope}%
\begin{pgfscope}%
\pgfpathrectangle{\pgfqpoint{0.600000in}{0.600000in}}{\pgfqpoint{3.900000in}{3.900000in}}%
\pgfusepath{clip}%
\pgfsetbuttcap%
\pgfsetroundjoin%
\definecolor{currentfill}{rgb}{0.121569,0.466667,0.705882}%
\pgfsetfillcolor{currentfill}%
\pgfsetlinewidth{1.003750pt}%
\definecolor{currentstroke}{rgb}{0.121569,0.466667,0.705882}%
\pgfsetstrokecolor{currentstroke}%
\pgfsetdash{}{0pt}%
\pgfpathmoveto{\pgfqpoint{0.957648in}{2.497043in}}%
\pgfpathcurveto{\pgfqpoint{0.968698in}{2.497043in}}{\pgfqpoint{0.979297in}{2.501433in}}{\pgfqpoint{0.987111in}{2.509247in}}%
\pgfpathcurveto{\pgfqpoint{0.994924in}{2.517060in}}{\pgfqpoint{0.999315in}{2.527659in}}{\pgfqpoint{0.999315in}{2.538709in}}%
\pgfpathcurveto{\pgfqpoint{0.999315in}{2.549759in}}{\pgfqpoint{0.994924in}{2.560359in}}{\pgfqpoint{0.987111in}{2.568172in}}%
\pgfpathcurveto{\pgfqpoint{0.979297in}{2.575986in}}{\pgfqpoint{0.968698in}{2.580376in}}{\pgfqpoint{0.957648in}{2.580376in}}%
\pgfpathcurveto{\pgfqpoint{0.946598in}{2.580376in}}{\pgfqpoint{0.935999in}{2.575986in}}{\pgfqpoint{0.928185in}{2.568172in}}%
\pgfpathcurveto{\pgfqpoint{0.920372in}{2.560359in}}{\pgfqpoint{0.915981in}{2.549759in}}{\pgfqpoint{0.915981in}{2.538709in}}%
\pgfpathcurveto{\pgfqpoint{0.915981in}{2.527659in}}{\pgfqpoint{0.920372in}{2.517060in}}{\pgfqpoint{0.928185in}{2.509247in}}%
\pgfpathcurveto{\pgfqpoint{0.935999in}{2.501433in}}{\pgfqpoint{0.946598in}{2.497043in}}{\pgfqpoint{0.957648in}{2.497043in}}%
\pgfpathclose%
\pgfusepath{stroke,fill}%
\end{pgfscope}%
\begin{pgfscope}%
\pgfpathrectangle{\pgfqpoint{0.600000in}{0.600000in}}{\pgfqpoint{3.900000in}{3.900000in}}%
\pgfusepath{clip}%
\pgfsetbuttcap%
\pgfsetroundjoin%
\definecolor{currentfill}{rgb}{0.121569,0.466667,0.705882}%
\pgfsetfillcolor{currentfill}%
\pgfsetlinewidth{1.003750pt}%
\definecolor{currentstroke}{rgb}{0.121569,0.466667,0.705882}%
\pgfsetstrokecolor{currentstroke}%
\pgfsetdash{}{0pt}%
\pgfpathmoveto{\pgfqpoint{1.693801in}{2.544567in}}%
\pgfpathcurveto{\pgfqpoint{1.704851in}{2.544567in}}{\pgfqpoint{1.715450in}{2.548957in}}{\pgfqpoint{1.723264in}{2.556771in}}%
\pgfpathcurveto{\pgfqpoint{1.731078in}{2.564585in}}{\pgfqpoint{1.735468in}{2.575184in}}{\pgfqpoint{1.735468in}{2.586234in}}%
\pgfpathcurveto{\pgfqpoint{1.735468in}{2.597284in}}{\pgfqpoint{1.731078in}{2.607883in}}{\pgfqpoint{1.723264in}{2.615696in}}%
\pgfpathcurveto{\pgfqpoint{1.715450in}{2.623510in}}{\pgfqpoint{1.704851in}{2.627900in}}{\pgfqpoint{1.693801in}{2.627900in}}%
\pgfpathcurveto{\pgfqpoint{1.682751in}{2.627900in}}{\pgfqpoint{1.672152in}{2.623510in}}{\pgfqpoint{1.664338in}{2.615696in}}%
\pgfpathcurveto{\pgfqpoint{1.656525in}{2.607883in}}{\pgfqpoint{1.652135in}{2.597284in}}{\pgfqpoint{1.652135in}{2.586234in}}%
\pgfpathcurveto{\pgfqpoint{1.652135in}{2.575184in}}{\pgfqpoint{1.656525in}{2.564585in}}{\pgfqpoint{1.664338in}{2.556771in}}%
\pgfpathcurveto{\pgfqpoint{1.672152in}{2.548957in}}{\pgfqpoint{1.682751in}{2.544567in}}{\pgfqpoint{1.693801in}{2.544567in}}%
\pgfpathclose%
\pgfusepath{stroke,fill}%
\end{pgfscope}%
\begin{pgfscope}%
\pgfpathrectangle{\pgfqpoint{0.600000in}{0.600000in}}{\pgfqpoint{3.900000in}{3.900000in}}%
\pgfusepath{clip}%
\pgfsetbuttcap%
\pgfsetroundjoin%
\definecolor{currentfill}{rgb}{0.121569,0.466667,0.705882}%
\pgfsetfillcolor{currentfill}%
\pgfsetlinewidth{1.003750pt}%
\definecolor{currentstroke}{rgb}{0.121569,0.466667,0.705882}%
\pgfsetstrokecolor{currentstroke}%
\pgfsetdash{}{0pt}%
\pgfpathmoveto{\pgfqpoint{1.830231in}{2.816593in}}%
\pgfpathcurveto{\pgfqpoint{1.841281in}{2.816593in}}{\pgfqpoint{1.851880in}{2.820983in}}{\pgfqpoint{1.859694in}{2.828797in}}%
\pgfpathcurveto{\pgfqpoint{1.867507in}{2.836611in}}{\pgfqpoint{1.871898in}{2.847210in}}{\pgfqpoint{1.871898in}{2.858260in}}%
\pgfpathcurveto{\pgfqpoint{1.871898in}{2.869310in}}{\pgfqpoint{1.867507in}{2.879909in}}{\pgfqpoint{1.859694in}{2.887723in}}%
\pgfpathcurveto{\pgfqpoint{1.851880in}{2.895536in}}{\pgfqpoint{1.841281in}{2.899926in}}{\pgfqpoint{1.830231in}{2.899926in}}%
\pgfpathcurveto{\pgfqpoint{1.819181in}{2.899926in}}{\pgfqpoint{1.808582in}{2.895536in}}{\pgfqpoint{1.800768in}{2.887723in}}%
\pgfpathcurveto{\pgfqpoint{1.792955in}{2.879909in}}{\pgfqpoint{1.788564in}{2.869310in}}{\pgfqpoint{1.788564in}{2.858260in}}%
\pgfpathcurveto{\pgfqpoint{1.788564in}{2.847210in}}{\pgfqpoint{1.792955in}{2.836611in}}{\pgfqpoint{1.800768in}{2.828797in}}%
\pgfpathcurveto{\pgfqpoint{1.808582in}{2.820983in}}{\pgfqpoint{1.819181in}{2.816593in}}{\pgfqpoint{1.830231in}{2.816593in}}%
\pgfpathclose%
\pgfusepath{stroke,fill}%
\end{pgfscope}%
\begin{pgfscope}%
\pgfpathrectangle{\pgfqpoint{0.600000in}{0.600000in}}{\pgfqpoint{3.900000in}{3.900000in}}%
\pgfusepath{clip}%
\pgfsetbuttcap%
\pgfsetroundjoin%
\definecolor{currentfill}{rgb}{0.121569,0.466667,0.705882}%
\pgfsetfillcolor{currentfill}%
\pgfsetlinewidth{1.003750pt}%
\definecolor{currentstroke}{rgb}{0.121569,0.466667,0.705882}%
\pgfsetstrokecolor{currentstroke}%
\pgfsetdash{}{0pt}%
\pgfpathmoveto{\pgfqpoint{2.422612in}{2.768336in}}%
\pgfpathcurveto{\pgfqpoint{2.433662in}{2.768336in}}{\pgfqpoint{2.444262in}{2.772727in}}{\pgfqpoint{2.452075in}{2.780540in}}%
\pgfpathcurveto{\pgfqpoint{2.459889in}{2.788354in}}{\pgfqpoint{2.464279in}{2.798953in}}{\pgfqpoint{2.464279in}{2.810003in}}%
\pgfpathcurveto{\pgfqpoint{2.464279in}{2.821053in}}{\pgfqpoint{2.459889in}{2.831652in}}{\pgfqpoint{2.452075in}{2.839466in}}%
\pgfpathcurveto{\pgfqpoint{2.444262in}{2.847279in}}{\pgfqpoint{2.433662in}{2.851670in}}{\pgfqpoint{2.422612in}{2.851670in}}%
\pgfpathcurveto{\pgfqpoint{2.411562in}{2.851670in}}{\pgfqpoint{2.400963in}{2.847279in}}{\pgfqpoint{2.393150in}{2.839466in}}%
\pgfpathcurveto{\pgfqpoint{2.385336in}{2.831652in}}{\pgfqpoint{2.380946in}{2.821053in}}{\pgfqpoint{2.380946in}{2.810003in}}%
\pgfpathcurveto{\pgfqpoint{2.380946in}{2.798953in}}{\pgfqpoint{2.385336in}{2.788354in}}{\pgfqpoint{2.393150in}{2.780540in}}%
\pgfpathcurveto{\pgfqpoint{2.400963in}{2.772727in}}{\pgfqpoint{2.411562in}{2.768336in}}{\pgfqpoint{2.422612in}{2.768336in}}%
\pgfpathclose%
\pgfusepath{stroke,fill}%
\end{pgfscope}%
\begin{pgfscope}%
\pgfpathrectangle{\pgfqpoint{0.600000in}{0.600000in}}{\pgfqpoint{3.900000in}{3.900000in}}%
\pgfusepath{clip}%
\pgfsetbuttcap%
\pgfsetroundjoin%
\definecolor{currentfill}{rgb}{0.121569,0.466667,0.705882}%
\pgfsetfillcolor{currentfill}%
\pgfsetlinewidth{1.003750pt}%
\definecolor{currentstroke}{rgb}{0.121569,0.466667,0.705882}%
\pgfsetstrokecolor{currentstroke}%
\pgfsetdash{}{0pt}%
\pgfpathmoveto{\pgfqpoint{2.854035in}{1.619117in}}%
\pgfpathcurveto{\pgfqpoint{2.865085in}{1.619117in}}{\pgfqpoint{2.875684in}{1.623507in}}{\pgfqpoint{2.883498in}{1.631321in}}%
\pgfpathcurveto{\pgfqpoint{2.891312in}{1.639134in}}{\pgfqpoint{2.895702in}{1.649733in}}{\pgfqpoint{2.895702in}{1.660783in}}%
\pgfpathcurveto{\pgfqpoint{2.895702in}{1.671833in}}{\pgfqpoint{2.891312in}{1.682433in}}{\pgfqpoint{2.883498in}{1.690246in}}%
\pgfpathcurveto{\pgfqpoint{2.875684in}{1.698060in}}{\pgfqpoint{2.865085in}{1.702450in}}{\pgfqpoint{2.854035in}{1.702450in}}%
\pgfpathcurveto{\pgfqpoint{2.842985in}{1.702450in}}{\pgfqpoint{2.832386in}{1.698060in}}{\pgfqpoint{2.824573in}{1.690246in}}%
\pgfpathcurveto{\pgfqpoint{2.816759in}{1.682433in}}{\pgfqpoint{2.812369in}{1.671833in}}{\pgfqpoint{2.812369in}{1.660783in}}%
\pgfpathcurveto{\pgfqpoint{2.812369in}{1.649733in}}{\pgfqpoint{2.816759in}{1.639134in}}{\pgfqpoint{2.824573in}{1.631321in}}%
\pgfpathcurveto{\pgfqpoint{2.832386in}{1.623507in}}{\pgfqpoint{2.842985in}{1.619117in}}{\pgfqpoint{2.854035in}{1.619117in}}%
\pgfpathclose%
\pgfusepath{stroke,fill}%
\end{pgfscope}%
\begin{pgfscope}%
\pgfpathrectangle{\pgfqpoint{0.600000in}{0.600000in}}{\pgfqpoint{3.900000in}{3.900000in}}%
\pgfusepath{clip}%
\pgfsetbuttcap%
\pgfsetroundjoin%
\definecolor{currentfill}{rgb}{0.121569,0.466667,0.705882}%
\pgfsetfillcolor{currentfill}%
\pgfsetlinewidth{1.003750pt}%
\definecolor{currentstroke}{rgb}{0.121569,0.466667,0.705882}%
\pgfsetstrokecolor{currentstroke}%
\pgfsetdash{}{0pt}%
\pgfpathmoveto{\pgfqpoint{2.525314in}{1.913745in}}%
\pgfpathcurveto{\pgfqpoint{2.536364in}{1.913745in}}{\pgfqpoint{2.546963in}{1.918136in}}{\pgfqpoint{2.554777in}{1.925949in}}%
\pgfpathcurveto{\pgfqpoint{2.562590in}{1.933763in}}{\pgfqpoint{2.566981in}{1.944362in}}{\pgfqpoint{2.566981in}{1.955412in}}%
\pgfpathcurveto{\pgfqpoint{2.566981in}{1.966462in}}{\pgfqpoint{2.562590in}{1.977061in}}{\pgfqpoint{2.554777in}{1.984875in}}%
\pgfpathcurveto{\pgfqpoint{2.546963in}{1.992688in}}{\pgfqpoint{2.536364in}{1.997079in}}{\pgfqpoint{2.525314in}{1.997079in}}%
\pgfpathcurveto{\pgfqpoint{2.514264in}{1.997079in}}{\pgfqpoint{2.503665in}{1.992688in}}{\pgfqpoint{2.495851in}{1.984875in}}%
\pgfpathcurveto{\pgfqpoint{2.488038in}{1.977061in}}{\pgfqpoint{2.483647in}{1.966462in}}{\pgfqpoint{2.483647in}{1.955412in}}%
\pgfpathcurveto{\pgfqpoint{2.483647in}{1.944362in}}{\pgfqpoint{2.488038in}{1.933763in}}{\pgfqpoint{2.495851in}{1.925949in}}%
\pgfpathcurveto{\pgfqpoint{2.503665in}{1.918136in}}{\pgfqpoint{2.514264in}{1.913745in}}{\pgfqpoint{2.525314in}{1.913745in}}%
\pgfpathclose%
\pgfusepath{stroke,fill}%
\end{pgfscope}%
\begin{pgfscope}%
\pgfpathrectangle{\pgfqpoint{0.600000in}{0.600000in}}{\pgfqpoint{3.900000in}{3.900000in}}%
\pgfusepath{clip}%
\pgfsetbuttcap%
\pgfsetroundjoin%
\definecolor{currentfill}{rgb}{0.121569,0.466667,0.705882}%
\pgfsetfillcolor{currentfill}%
\pgfsetlinewidth{1.003750pt}%
\definecolor{currentstroke}{rgb}{0.121569,0.466667,0.705882}%
\pgfsetstrokecolor{currentstroke}%
\pgfsetdash{}{0pt}%
\pgfpathmoveto{\pgfqpoint{2.732038in}{2.042385in}}%
\pgfpathcurveto{\pgfqpoint{2.743089in}{2.042385in}}{\pgfqpoint{2.753688in}{2.046775in}}{\pgfqpoint{2.761501in}{2.054589in}}%
\pgfpathcurveto{\pgfqpoint{2.769315in}{2.062403in}}{\pgfqpoint{2.773705in}{2.073002in}}{\pgfqpoint{2.773705in}{2.084052in}}%
\pgfpathcurveto{\pgfqpoint{2.773705in}{2.095102in}}{\pgfqpoint{2.769315in}{2.105701in}}{\pgfqpoint{2.761501in}{2.113514in}}%
\pgfpathcurveto{\pgfqpoint{2.753688in}{2.121328in}}{\pgfqpoint{2.743089in}{2.125718in}}{\pgfqpoint{2.732038in}{2.125718in}}%
\pgfpathcurveto{\pgfqpoint{2.720988in}{2.125718in}}{\pgfqpoint{2.710389in}{2.121328in}}{\pgfqpoint{2.702576in}{2.113514in}}%
\pgfpathcurveto{\pgfqpoint{2.694762in}{2.105701in}}{\pgfqpoint{2.690372in}{2.095102in}}{\pgfqpoint{2.690372in}{2.084052in}}%
\pgfpathcurveto{\pgfqpoint{2.690372in}{2.073002in}}{\pgfqpoint{2.694762in}{2.062403in}}{\pgfqpoint{2.702576in}{2.054589in}}%
\pgfpathcurveto{\pgfqpoint{2.710389in}{2.046775in}}{\pgfqpoint{2.720988in}{2.042385in}}{\pgfqpoint{2.732038in}{2.042385in}}%
\pgfpathclose%
\pgfusepath{stroke,fill}%
\end{pgfscope}%
\begin{pgfscope}%
\pgfpathrectangle{\pgfqpoint{0.600000in}{0.600000in}}{\pgfqpoint{3.900000in}{3.900000in}}%
\pgfusepath{clip}%
\pgfsetbuttcap%
\pgfsetroundjoin%
\definecolor{currentfill}{rgb}{0.121569,0.466667,0.705882}%
\pgfsetfillcolor{currentfill}%
\pgfsetlinewidth{1.003750pt}%
\definecolor{currentstroke}{rgb}{0.121569,0.466667,0.705882}%
\pgfsetstrokecolor{currentstroke}%
\pgfsetdash{}{0pt}%
\pgfpathmoveto{\pgfqpoint{3.223040in}{2.537506in}}%
\pgfpathcurveto{\pgfqpoint{3.234090in}{2.537506in}}{\pgfqpoint{3.244689in}{2.541896in}}{\pgfqpoint{3.252503in}{2.549710in}}%
\pgfpathcurveto{\pgfqpoint{3.260316in}{2.557523in}}{\pgfqpoint{3.264707in}{2.568123in}}{\pgfqpoint{3.264707in}{2.579173in}}%
\pgfpathcurveto{\pgfqpoint{3.264707in}{2.590223in}}{\pgfqpoint{3.260316in}{2.600822in}}{\pgfqpoint{3.252503in}{2.608635in}}%
\pgfpathcurveto{\pgfqpoint{3.244689in}{2.616449in}}{\pgfqpoint{3.234090in}{2.620839in}}{\pgfqpoint{3.223040in}{2.620839in}}%
\pgfpathcurveto{\pgfqpoint{3.211990in}{2.620839in}}{\pgfqpoint{3.201391in}{2.616449in}}{\pgfqpoint{3.193577in}{2.608635in}}%
\pgfpathcurveto{\pgfqpoint{3.185764in}{2.600822in}}{\pgfqpoint{3.181373in}{2.590223in}}{\pgfqpoint{3.181373in}{2.579173in}}%
\pgfpathcurveto{\pgfqpoint{3.181373in}{2.568123in}}{\pgfqpoint{3.185764in}{2.557523in}}{\pgfqpoint{3.193577in}{2.549710in}}%
\pgfpathcurveto{\pgfqpoint{3.201391in}{2.541896in}}{\pgfqpoint{3.211990in}{2.537506in}}{\pgfqpoint{3.223040in}{2.537506in}}%
\pgfpathclose%
\pgfusepath{stroke,fill}%
\end{pgfscope}%
\begin{pgfscope}%
\pgfpathrectangle{\pgfqpoint{0.600000in}{0.600000in}}{\pgfqpoint{3.900000in}{3.900000in}}%
\pgfusepath{clip}%
\pgfsetbuttcap%
\pgfsetroundjoin%
\definecolor{currentfill}{rgb}{0.121569,0.466667,0.705882}%
\pgfsetfillcolor{currentfill}%
\pgfsetlinewidth{1.003750pt}%
\definecolor{currentstroke}{rgb}{0.121569,0.466667,0.705882}%
\pgfsetstrokecolor{currentstroke}%
\pgfsetdash{}{0pt}%
\pgfpathmoveto{\pgfqpoint{2.390205in}{1.928450in}}%
\pgfpathcurveto{\pgfqpoint{2.401255in}{1.928450in}}{\pgfqpoint{2.411854in}{1.932840in}}{\pgfqpoint{2.419668in}{1.940654in}}%
\pgfpathcurveto{\pgfqpoint{2.427482in}{1.948467in}}{\pgfqpoint{2.431872in}{1.959066in}}{\pgfqpoint{2.431872in}{1.970116in}}%
\pgfpathcurveto{\pgfqpoint{2.431872in}{1.981166in}}{\pgfqpoint{2.427482in}{1.991765in}}{\pgfqpoint{2.419668in}{1.999579in}}%
\pgfpathcurveto{\pgfqpoint{2.411854in}{2.007393in}}{\pgfqpoint{2.401255in}{2.011783in}}{\pgfqpoint{2.390205in}{2.011783in}}%
\pgfpathcurveto{\pgfqpoint{2.379155in}{2.011783in}}{\pgfqpoint{2.368556in}{2.007393in}}{\pgfqpoint{2.360743in}{1.999579in}}%
\pgfpathcurveto{\pgfqpoint{2.352929in}{1.991765in}}{\pgfqpoint{2.348539in}{1.981166in}}{\pgfqpoint{2.348539in}{1.970116in}}%
\pgfpathcurveto{\pgfqpoint{2.348539in}{1.959066in}}{\pgfqpoint{2.352929in}{1.948467in}}{\pgfqpoint{2.360743in}{1.940654in}}%
\pgfpathcurveto{\pgfqpoint{2.368556in}{1.932840in}}{\pgfqpoint{2.379155in}{1.928450in}}{\pgfqpoint{2.390205in}{1.928450in}}%
\pgfpathclose%
\pgfusepath{stroke,fill}%
\end{pgfscope}%
\begin{pgfscope}%
\pgfpathrectangle{\pgfqpoint{0.600000in}{0.600000in}}{\pgfqpoint{3.900000in}{3.900000in}}%
\pgfusepath{clip}%
\pgfsetbuttcap%
\pgfsetroundjoin%
\definecolor{currentfill}{rgb}{0.121569,0.466667,0.705882}%
\pgfsetfillcolor{currentfill}%
\pgfsetlinewidth{1.003750pt}%
\definecolor{currentstroke}{rgb}{0.121569,0.466667,0.705882}%
\pgfsetstrokecolor{currentstroke}%
\pgfsetdash{}{0pt}%
\pgfpathmoveto{\pgfqpoint{3.483900in}{2.430277in}}%
\pgfpathcurveto{\pgfqpoint{3.494950in}{2.430277in}}{\pgfqpoint{3.505549in}{2.434667in}}{\pgfqpoint{3.513363in}{2.442481in}}%
\pgfpathcurveto{\pgfqpoint{3.521176in}{2.450294in}}{\pgfqpoint{3.525567in}{2.460893in}}{\pgfqpoint{3.525567in}{2.471944in}}%
\pgfpathcurveto{\pgfqpoint{3.525567in}{2.482994in}}{\pgfqpoint{3.521176in}{2.493593in}}{\pgfqpoint{3.513363in}{2.501406in}}%
\pgfpathcurveto{\pgfqpoint{3.505549in}{2.509220in}}{\pgfqpoint{3.494950in}{2.513610in}}{\pgfqpoint{3.483900in}{2.513610in}}%
\pgfpathcurveto{\pgfqpoint{3.472850in}{2.513610in}}{\pgfqpoint{3.462251in}{2.509220in}}{\pgfqpoint{3.454437in}{2.501406in}}%
\pgfpathcurveto{\pgfqpoint{3.446624in}{2.493593in}}{\pgfqpoint{3.442233in}{2.482994in}}{\pgfqpoint{3.442233in}{2.471944in}}%
\pgfpathcurveto{\pgfqpoint{3.442233in}{2.460893in}}{\pgfqpoint{3.446624in}{2.450294in}}{\pgfqpoint{3.454437in}{2.442481in}}%
\pgfpathcurveto{\pgfqpoint{3.462251in}{2.434667in}}{\pgfqpoint{3.472850in}{2.430277in}}{\pgfqpoint{3.483900in}{2.430277in}}%
\pgfpathclose%
\pgfusepath{stroke,fill}%
\end{pgfscope}%
\begin{pgfscope}%
\pgfpathrectangle{\pgfqpoint{0.600000in}{0.600000in}}{\pgfqpoint{3.900000in}{3.900000in}}%
\pgfusepath{clip}%
\pgfsetbuttcap%
\pgfsetroundjoin%
\definecolor{currentfill}{rgb}{0.121569,0.466667,0.705882}%
\pgfsetfillcolor{currentfill}%
\pgfsetlinewidth{1.003750pt}%
\definecolor{currentstroke}{rgb}{0.121569,0.466667,0.705882}%
\pgfsetstrokecolor{currentstroke}%
\pgfsetdash{}{0pt}%
\pgfpathmoveto{\pgfqpoint{2.778823in}{2.879302in}}%
\pgfpathcurveto{\pgfqpoint{2.789873in}{2.879302in}}{\pgfqpoint{2.800472in}{2.883692in}}{\pgfqpoint{2.808285in}{2.891505in}}%
\pgfpathcurveto{\pgfqpoint{2.816099in}{2.899319in}}{\pgfqpoint{2.820489in}{2.909918in}}{\pgfqpoint{2.820489in}{2.920968in}}%
\pgfpathcurveto{\pgfqpoint{2.820489in}{2.932018in}}{\pgfqpoint{2.816099in}{2.942617in}}{\pgfqpoint{2.808285in}{2.950431in}}%
\pgfpathcurveto{\pgfqpoint{2.800472in}{2.958245in}}{\pgfqpoint{2.789873in}{2.962635in}}{\pgfqpoint{2.778823in}{2.962635in}}%
\pgfpathcurveto{\pgfqpoint{2.767772in}{2.962635in}}{\pgfqpoint{2.757173in}{2.958245in}}{\pgfqpoint{2.749360in}{2.950431in}}%
\pgfpathcurveto{\pgfqpoint{2.741546in}{2.942617in}}{\pgfqpoint{2.737156in}{2.932018in}}{\pgfqpoint{2.737156in}{2.920968in}}%
\pgfpathcurveto{\pgfqpoint{2.737156in}{2.909918in}}{\pgfqpoint{2.741546in}{2.899319in}}{\pgfqpoint{2.749360in}{2.891505in}}%
\pgfpathcurveto{\pgfqpoint{2.757173in}{2.883692in}}{\pgfqpoint{2.767772in}{2.879302in}}{\pgfqpoint{2.778823in}{2.879302in}}%
\pgfpathclose%
\pgfusepath{stroke,fill}%
\end{pgfscope}%
\begin{pgfscope}%
\pgfpathrectangle{\pgfqpoint{0.600000in}{0.600000in}}{\pgfqpoint{3.900000in}{3.900000in}}%
\pgfusepath{clip}%
\pgfsetbuttcap%
\pgfsetroundjoin%
\definecolor{currentfill}{rgb}{0.121569,0.466667,0.705882}%
\pgfsetfillcolor{currentfill}%
\pgfsetlinewidth{1.003750pt}%
\definecolor{currentstroke}{rgb}{0.121569,0.466667,0.705882}%
\pgfsetstrokecolor{currentstroke}%
\pgfsetdash{}{0pt}%
\pgfpathmoveto{\pgfqpoint{3.189939in}{2.456410in}}%
\pgfpathcurveto{\pgfqpoint{3.200989in}{2.456410in}}{\pgfqpoint{3.211588in}{2.460800in}}{\pgfqpoint{3.219402in}{2.468614in}}%
\pgfpathcurveto{\pgfqpoint{3.227216in}{2.476428in}}{\pgfqpoint{3.231606in}{2.487027in}}{\pgfqpoint{3.231606in}{2.498077in}}%
\pgfpathcurveto{\pgfqpoint{3.231606in}{2.509127in}}{\pgfqpoint{3.227216in}{2.519726in}}{\pgfqpoint{3.219402in}{2.527540in}}%
\pgfpathcurveto{\pgfqpoint{3.211588in}{2.535353in}}{\pgfqpoint{3.200989in}{2.539744in}}{\pgfqpoint{3.189939in}{2.539744in}}%
\pgfpathcurveto{\pgfqpoint{3.178889in}{2.539744in}}{\pgfqpoint{3.168290in}{2.535353in}}{\pgfqpoint{3.160477in}{2.527540in}}%
\pgfpathcurveto{\pgfqpoint{3.152663in}{2.519726in}}{\pgfqpoint{3.148273in}{2.509127in}}{\pgfqpoint{3.148273in}{2.498077in}}%
\pgfpathcurveto{\pgfqpoint{3.148273in}{2.487027in}}{\pgfqpoint{3.152663in}{2.476428in}}{\pgfqpoint{3.160477in}{2.468614in}}%
\pgfpathcurveto{\pgfqpoint{3.168290in}{2.460800in}}{\pgfqpoint{3.178889in}{2.456410in}}{\pgfqpoint{3.189939in}{2.456410in}}%
\pgfpathclose%
\pgfusepath{stroke,fill}%
\end{pgfscope}%
\begin{pgfscope}%
\pgfpathrectangle{\pgfqpoint{0.600000in}{0.600000in}}{\pgfqpoint{3.900000in}{3.900000in}}%
\pgfusepath{clip}%
\pgfsetbuttcap%
\pgfsetroundjoin%
\definecolor{currentfill}{rgb}{0.121569,0.466667,0.705882}%
\pgfsetfillcolor{currentfill}%
\pgfsetlinewidth{1.003750pt}%
\definecolor{currentstroke}{rgb}{0.121569,0.466667,0.705882}%
\pgfsetstrokecolor{currentstroke}%
\pgfsetdash{}{0pt}%
\pgfpathmoveto{\pgfqpoint{2.071655in}{2.762374in}}%
\pgfpathcurveto{\pgfqpoint{2.082705in}{2.762374in}}{\pgfqpoint{2.093304in}{2.766764in}}{\pgfqpoint{2.101118in}{2.774578in}}%
\pgfpathcurveto{\pgfqpoint{2.108931in}{2.782391in}}{\pgfqpoint{2.113322in}{2.792990in}}{\pgfqpoint{2.113322in}{2.804040in}}%
\pgfpathcurveto{\pgfqpoint{2.113322in}{2.815091in}}{\pgfqpoint{2.108931in}{2.825690in}}{\pgfqpoint{2.101118in}{2.833503in}}%
\pgfpathcurveto{\pgfqpoint{2.093304in}{2.841317in}}{\pgfqpoint{2.082705in}{2.845707in}}{\pgfqpoint{2.071655in}{2.845707in}}%
\pgfpathcurveto{\pgfqpoint{2.060605in}{2.845707in}}{\pgfqpoint{2.050006in}{2.841317in}}{\pgfqpoint{2.042192in}{2.833503in}}%
\pgfpathcurveto{\pgfqpoint{2.034379in}{2.825690in}}{\pgfqpoint{2.029988in}{2.815091in}}{\pgfqpoint{2.029988in}{2.804040in}}%
\pgfpathcurveto{\pgfqpoint{2.029988in}{2.792990in}}{\pgfqpoint{2.034379in}{2.782391in}}{\pgfqpoint{2.042192in}{2.774578in}}%
\pgfpathcurveto{\pgfqpoint{2.050006in}{2.766764in}}{\pgfqpoint{2.060605in}{2.762374in}}{\pgfqpoint{2.071655in}{2.762374in}}%
\pgfpathclose%
\pgfusepath{stroke,fill}%
\end{pgfscope}%
\begin{pgfscope}%
\pgfpathrectangle{\pgfqpoint{0.600000in}{0.600000in}}{\pgfqpoint{3.900000in}{3.900000in}}%
\pgfusepath{clip}%
\pgfsetbuttcap%
\pgfsetroundjoin%
\definecolor{currentfill}{rgb}{0.121569,0.466667,0.705882}%
\pgfsetfillcolor{currentfill}%
\pgfsetlinewidth{1.003750pt}%
\definecolor{currentstroke}{rgb}{0.121569,0.466667,0.705882}%
\pgfsetstrokecolor{currentstroke}%
\pgfsetdash{}{0pt}%
\pgfpathmoveto{\pgfqpoint{2.628623in}{2.628929in}}%
\pgfpathcurveto{\pgfqpoint{2.639673in}{2.628929in}}{\pgfqpoint{2.650272in}{2.633319in}}{\pgfqpoint{2.658086in}{2.641133in}}%
\pgfpathcurveto{\pgfqpoint{2.665899in}{2.648947in}}{\pgfqpoint{2.670290in}{2.659546in}}{\pgfqpoint{2.670290in}{2.670596in}}%
\pgfpathcurveto{\pgfqpoint{2.670290in}{2.681646in}}{\pgfqpoint{2.665899in}{2.692245in}}{\pgfqpoint{2.658086in}{2.700058in}}%
\pgfpathcurveto{\pgfqpoint{2.650272in}{2.707872in}}{\pgfqpoint{2.639673in}{2.712262in}}{\pgfqpoint{2.628623in}{2.712262in}}%
\pgfpathcurveto{\pgfqpoint{2.617573in}{2.712262in}}{\pgfqpoint{2.606974in}{2.707872in}}{\pgfqpoint{2.599160in}{2.700058in}}%
\pgfpathcurveto{\pgfqpoint{2.591347in}{2.692245in}}{\pgfqpoint{2.586956in}{2.681646in}}{\pgfqpoint{2.586956in}{2.670596in}}%
\pgfpathcurveto{\pgfqpoint{2.586956in}{2.659546in}}{\pgfqpoint{2.591347in}{2.648947in}}{\pgfqpoint{2.599160in}{2.641133in}}%
\pgfpathcurveto{\pgfqpoint{2.606974in}{2.633319in}}{\pgfqpoint{2.617573in}{2.628929in}}{\pgfqpoint{2.628623in}{2.628929in}}%
\pgfpathclose%
\pgfusepath{stroke,fill}%
\end{pgfscope}%
\begin{pgfscope}%
\pgfpathrectangle{\pgfqpoint{0.600000in}{0.600000in}}{\pgfqpoint{3.900000in}{3.900000in}}%
\pgfusepath{clip}%
\pgfsetbuttcap%
\pgfsetroundjoin%
\definecolor{currentfill}{rgb}{0.121569,0.466667,0.705882}%
\pgfsetfillcolor{currentfill}%
\pgfsetlinewidth{1.003750pt}%
\definecolor{currentstroke}{rgb}{0.121569,0.466667,0.705882}%
\pgfsetstrokecolor{currentstroke}%
\pgfsetdash{}{0pt}%
\pgfpathmoveto{\pgfqpoint{2.436044in}{3.075665in}}%
\pgfpathcurveto{\pgfqpoint{2.447094in}{3.075665in}}{\pgfqpoint{2.457693in}{3.080055in}}{\pgfqpoint{2.465507in}{3.087869in}}%
\pgfpathcurveto{\pgfqpoint{2.473321in}{3.095683in}}{\pgfqpoint{2.477711in}{3.106282in}}{\pgfqpoint{2.477711in}{3.117332in}}%
\pgfpathcurveto{\pgfqpoint{2.477711in}{3.128382in}}{\pgfqpoint{2.473321in}{3.138981in}}{\pgfqpoint{2.465507in}{3.146795in}}%
\pgfpathcurveto{\pgfqpoint{2.457693in}{3.154608in}}{\pgfqpoint{2.447094in}{3.158998in}}{\pgfqpoint{2.436044in}{3.158998in}}%
\pgfpathcurveto{\pgfqpoint{2.424994in}{3.158998in}}{\pgfqpoint{2.414395in}{3.154608in}}{\pgfqpoint{2.406582in}{3.146795in}}%
\pgfpathcurveto{\pgfqpoint{2.398768in}{3.138981in}}{\pgfqpoint{2.394378in}{3.128382in}}{\pgfqpoint{2.394378in}{3.117332in}}%
\pgfpathcurveto{\pgfqpoint{2.394378in}{3.106282in}}{\pgfqpoint{2.398768in}{3.095683in}}{\pgfqpoint{2.406582in}{3.087869in}}%
\pgfpathcurveto{\pgfqpoint{2.414395in}{3.080055in}}{\pgfqpoint{2.424994in}{3.075665in}}{\pgfqpoint{2.436044in}{3.075665in}}%
\pgfpathclose%
\pgfusepath{stroke,fill}%
\end{pgfscope}%
\begin{pgfscope}%
\pgfpathrectangle{\pgfqpoint{0.600000in}{0.600000in}}{\pgfqpoint{3.900000in}{3.900000in}}%
\pgfusepath{clip}%
\pgfsetbuttcap%
\pgfsetroundjoin%
\definecolor{currentfill}{rgb}{0.121569,0.466667,0.705882}%
\pgfsetfillcolor{currentfill}%
\pgfsetlinewidth{1.003750pt}%
\definecolor{currentstroke}{rgb}{0.121569,0.466667,0.705882}%
\pgfsetstrokecolor{currentstroke}%
\pgfsetdash{}{0pt}%
\pgfpathmoveto{\pgfqpoint{1.948635in}{3.196435in}}%
\pgfpathcurveto{\pgfqpoint{1.959685in}{3.196435in}}{\pgfqpoint{1.970284in}{3.200825in}}{\pgfqpoint{1.978098in}{3.208639in}}%
\pgfpathcurveto{\pgfqpoint{1.985911in}{3.216452in}}{\pgfqpoint{1.990302in}{3.227051in}}{\pgfqpoint{1.990302in}{3.238101in}}%
\pgfpathcurveto{\pgfqpoint{1.990302in}{3.249152in}}{\pgfqpoint{1.985911in}{3.259751in}}{\pgfqpoint{1.978098in}{3.267564in}}%
\pgfpathcurveto{\pgfqpoint{1.970284in}{3.275378in}}{\pgfqpoint{1.959685in}{3.279768in}}{\pgfqpoint{1.948635in}{3.279768in}}%
\pgfpathcurveto{\pgfqpoint{1.937585in}{3.279768in}}{\pgfqpoint{1.926986in}{3.275378in}}{\pgfqpoint{1.919172in}{3.267564in}}%
\pgfpathcurveto{\pgfqpoint{1.911359in}{3.259751in}}{\pgfqpoint{1.906968in}{3.249152in}}{\pgfqpoint{1.906968in}{3.238101in}}%
\pgfpathcurveto{\pgfqpoint{1.906968in}{3.227051in}}{\pgfqpoint{1.911359in}{3.216452in}}{\pgfqpoint{1.919172in}{3.208639in}}%
\pgfpathcurveto{\pgfqpoint{1.926986in}{3.200825in}}{\pgfqpoint{1.937585in}{3.196435in}}{\pgfqpoint{1.948635in}{3.196435in}}%
\pgfpathclose%
\pgfusepath{stroke,fill}%
\end{pgfscope}%
\begin{pgfscope}%
\pgfpathrectangle{\pgfqpoint{0.600000in}{0.600000in}}{\pgfqpoint{3.900000in}{3.900000in}}%
\pgfusepath{clip}%
\pgfsetbuttcap%
\pgfsetroundjoin%
\definecolor{currentfill}{rgb}{0.121569,0.466667,0.705882}%
\pgfsetfillcolor{currentfill}%
\pgfsetlinewidth{1.003750pt}%
\definecolor{currentstroke}{rgb}{0.121569,0.466667,0.705882}%
\pgfsetstrokecolor{currentstroke}%
\pgfsetdash{}{0pt}%
\pgfpathmoveto{\pgfqpoint{1.774775in}{2.475316in}}%
\pgfpathcurveto{\pgfqpoint{1.785825in}{2.475316in}}{\pgfqpoint{1.796424in}{2.479706in}}{\pgfqpoint{1.804237in}{2.487519in}}%
\pgfpathcurveto{\pgfqpoint{1.812051in}{2.495333in}}{\pgfqpoint{1.816441in}{2.505932in}}{\pgfqpoint{1.816441in}{2.516982in}}%
\pgfpathcurveto{\pgfqpoint{1.816441in}{2.528032in}}{\pgfqpoint{1.812051in}{2.538631in}}{\pgfqpoint{1.804237in}{2.546445in}}%
\pgfpathcurveto{\pgfqpoint{1.796424in}{2.554259in}}{\pgfqpoint{1.785825in}{2.558649in}}{\pgfqpoint{1.774775in}{2.558649in}}%
\pgfpathcurveto{\pgfqpoint{1.763725in}{2.558649in}}{\pgfqpoint{1.753126in}{2.554259in}}{\pgfqpoint{1.745312in}{2.546445in}}%
\pgfpathcurveto{\pgfqpoint{1.737498in}{2.538631in}}{\pgfqpoint{1.733108in}{2.528032in}}{\pgfqpoint{1.733108in}{2.516982in}}%
\pgfpathcurveto{\pgfqpoint{1.733108in}{2.505932in}}{\pgfqpoint{1.737498in}{2.495333in}}{\pgfqpoint{1.745312in}{2.487519in}}%
\pgfpathcurveto{\pgfqpoint{1.753126in}{2.479706in}}{\pgfqpoint{1.763725in}{2.475316in}}{\pgfqpoint{1.774775in}{2.475316in}}%
\pgfpathclose%
\pgfusepath{stroke,fill}%
\end{pgfscope}%
\begin{pgfscope}%
\pgfpathrectangle{\pgfqpoint{0.600000in}{0.600000in}}{\pgfqpoint{3.900000in}{3.900000in}}%
\pgfusepath{clip}%
\pgfsetbuttcap%
\pgfsetroundjoin%
\definecolor{currentfill}{rgb}{0.121569,0.466667,0.705882}%
\pgfsetfillcolor{currentfill}%
\pgfsetlinewidth{1.003750pt}%
\definecolor{currentstroke}{rgb}{0.121569,0.466667,0.705882}%
\pgfsetstrokecolor{currentstroke}%
\pgfsetdash{}{0pt}%
\pgfpathmoveto{\pgfqpoint{1.878353in}{2.308565in}}%
\pgfpathcurveto{\pgfqpoint{1.889403in}{2.308565in}}{\pgfqpoint{1.900002in}{2.312955in}}{\pgfqpoint{1.907816in}{2.320769in}}%
\pgfpathcurveto{\pgfqpoint{1.915630in}{2.328583in}}{\pgfqpoint{1.920020in}{2.339182in}}{\pgfqpoint{1.920020in}{2.350232in}}%
\pgfpathcurveto{\pgfqpoint{1.920020in}{2.361282in}}{\pgfqpoint{1.915630in}{2.371881in}}{\pgfqpoint{1.907816in}{2.379694in}}%
\pgfpathcurveto{\pgfqpoint{1.900002in}{2.387508in}}{\pgfqpoint{1.889403in}{2.391898in}}{\pgfqpoint{1.878353in}{2.391898in}}%
\pgfpathcurveto{\pgfqpoint{1.867303in}{2.391898in}}{\pgfqpoint{1.856704in}{2.387508in}}{\pgfqpoint{1.848891in}{2.379694in}}%
\pgfpathcurveto{\pgfqpoint{1.841077in}{2.371881in}}{\pgfqpoint{1.836687in}{2.361282in}}{\pgfqpoint{1.836687in}{2.350232in}}%
\pgfpathcurveto{\pgfqpoint{1.836687in}{2.339182in}}{\pgfqpoint{1.841077in}{2.328583in}}{\pgfqpoint{1.848891in}{2.320769in}}%
\pgfpathcurveto{\pgfqpoint{1.856704in}{2.312955in}}{\pgfqpoint{1.867303in}{2.308565in}}{\pgfqpoint{1.878353in}{2.308565in}}%
\pgfpathclose%
\pgfusepath{stroke,fill}%
\end{pgfscope}%
\begin{pgfscope}%
\pgfpathrectangle{\pgfqpoint{0.600000in}{0.600000in}}{\pgfqpoint{3.900000in}{3.900000in}}%
\pgfusepath{clip}%
\pgfsetbuttcap%
\pgfsetroundjoin%
\definecolor{currentfill}{rgb}{0.121569,0.466667,0.705882}%
\pgfsetfillcolor{currentfill}%
\pgfsetlinewidth{1.003750pt}%
\definecolor{currentstroke}{rgb}{0.121569,0.466667,0.705882}%
\pgfsetstrokecolor{currentstroke}%
\pgfsetdash{}{0pt}%
\pgfpathmoveto{\pgfqpoint{2.396927in}{3.002217in}}%
\pgfpathcurveto{\pgfqpoint{2.407977in}{3.002217in}}{\pgfqpoint{2.418576in}{3.006608in}}{\pgfqpoint{2.426390in}{3.014421in}}%
\pgfpathcurveto{\pgfqpoint{2.434204in}{3.022235in}}{\pgfqpoint{2.438594in}{3.032834in}}{\pgfqpoint{2.438594in}{3.043884in}}%
\pgfpathcurveto{\pgfqpoint{2.438594in}{3.054934in}}{\pgfqpoint{2.434204in}{3.065533in}}{\pgfqpoint{2.426390in}{3.073347in}}%
\pgfpathcurveto{\pgfqpoint{2.418576in}{3.081161in}}{\pgfqpoint{2.407977in}{3.085551in}}{\pgfqpoint{2.396927in}{3.085551in}}%
\pgfpathcurveto{\pgfqpoint{2.385877in}{3.085551in}}{\pgfqpoint{2.375278in}{3.081161in}}{\pgfqpoint{2.367464in}{3.073347in}}%
\pgfpathcurveto{\pgfqpoint{2.359651in}{3.065533in}}{\pgfqpoint{2.355261in}{3.054934in}}{\pgfqpoint{2.355261in}{3.043884in}}%
\pgfpathcurveto{\pgfqpoint{2.355261in}{3.032834in}}{\pgfqpoint{2.359651in}{3.022235in}}{\pgfqpoint{2.367464in}{3.014421in}}%
\pgfpathcurveto{\pgfqpoint{2.375278in}{3.006608in}}{\pgfqpoint{2.385877in}{3.002217in}}{\pgfqpoint{2.396927in}{3.002217in}}%
\pgfpathclose%
\pgfusepath{stroke,fill}%
\end{pgfscope}%
\begin{pgfscope}%
\pgfpathrectangle{\pgfqpoint{0.600000in}{0.600000in}}{\pgfqpoint{3.900000in}{3.900000in}}%
\pgfusepath{clip}%
\pgfsetbuttcap%
\pgfsetroundjoin%
\definecolor{currentfill}{rgb}{0.121569,0.466667,0.705882}%
\pgfsetfillcolor{currentfill}%
\pgfsetlinewidth{1.003750pt}%
\definecolor{currentstroke}{rgb}{0.121569,0.466667,0.705882}%
\pgfsetstrokecolor{currentstroke}%
\pgfsetdash{}{0pt}%
\pgfpathmoveto{\pgfqpoint{2.261025in}{3.211620in}}%
\pgfpathcurveto{\pgfqpoint{2.272076in}{3.211620in}}{\pgfqpoint{2.282675in}{3.216010in}}{\pgfqpoint{2.290488in}{3.223824in}}%
\pgfpathcurveto{\pgfqpoint{2.298302in}{3.231638in}}{\pgfqpoint{2.302692in}{3.242237in}}{\pgfqpoint{2.302692in}{3.253287in}}%
\pgfpathcurveto{\pgfqpoint{2.302692in}{3.264337in}}{\pgfqpoint{2.298302in}{3.274936in}}{\pgfqpoint{2.290488in}{3.282750in}}%
\pgfpathcurveto{\pgfqpoint{2.282675in}{3.290563in}}{\pgfqpoint{2.272076in}{3.294954in}}{\pgfqpoint{2.261025in}{3.294954in}}%
\pgfpathcurveto{\pgfqpoint{2.249975in}{3.294954in}}{\pgfqpoint{2.239376in}{3.290563in}}{\pgfqpoint{2.231563in}{3.282750in}}%
\pgfpathcurveto{\pgfqpoint{2.223749in}{3.274936in}}{\pgfqpoint{2.219359in}{3.264337in}}{\pgfqpoint{2.219359in}{3.253287in}}%
\pgfpathcurveto{\pgfqpoint{2.219359in}{3.242237in}}{\pgfqpoint{2.223749in}{3.231638in}}{\pgfqpoint{2.231563in}{3.223824in}}%
\pgfpathcurveto{\pgfqpoint{2.239376in}{3.216010in}}{\pgfqpoint{2.249975in}{3.211620in}}{\pgfqpoint{2.261025in}{3.211620in}}%
\pgfpathclose%
\pgfusepath{stroke,fill}%
\end{pgfscope}%
\begin{pgfscope}%
\pgfpathrectangle{\pgfqpoint{0.600000in}{0.600000in}}{\pgfqpoint{3.900000in}{3.900000in}}%
\pgfusepath{clip}%
\pgfsetbuttcap%
\pgfsetroundjoin%
\definecolor{currentfill}{rgb}{0.121569,0.466667,0.705882}%
\pgfsetfillcolor{currentfill}%
\pgfsetlinewidth{1.003750pt}%
\definecolor{currentstroke}{rgb}{0.121569,0.466667,0.705882}%
\pgfsetstrokecolor{currentstroke}%
\pgfsetdash{}{0pt}%
\pgfpathmoveto{\pgfqpoint{3.092748in}{1.490711in}}%
\pgfpathcurveto{\pgfqpoint{3.103798in}{1.490711in}}{\pgfqpoint{3.114397in}{1.495101in}}{\pgfqpoint{3.122211in}{1.502915in}}%
\pgfpathcurveto{\pgfqpoint{3.130024in}{1.510729in}}{\pgfqpoint{3.134415in}{1.521328in}}{\pgfqpoint{3.134415in}{1.532378in}}%
\pgfpathcurveto{\pgfqpoint{3.134415in}{1.543428in}}{\pgfqpoint{3.130024in}{1.554027in}}{\pgfqpoint{3.122211in}{1.561841in}}%
\pgfpathcurveto{\pgfqpoint{3.114397in}{1.569654in}}{\pgfqpoint{3.103798in}{1.574045in}}{\pgfqpoint{3.092748in}{1.574045in}}%
\pgfpathcurveto{\pgfqpoint{3.081698in}{1.574045in}}{\pgfqpoint{3.071099in}{1.569654in}}{\pgfqpoint{3.063285in}{1.561841in}}%
\pgfpathcurveto{\pgfqpoint{3.055472in}{1.554027in}}{\pgfqpoint{3.051081in}{1.543428in}}{\pgfqpoint{3.051081in}{1.532378in}}%
\pgfpathcurveto{\pgfqpoint{3.051081in}{1.521328in}}{\pgfqpoint{3.055472in}{1.510729in}}{\pgfqpoint{3.063285in}{1.502915in}}%
\pgfpathcurveto{\pgfqpoint{3.071099in}{1.495101in}}{\pgfqpoint{3.081698in}{1.490711in}}{\pgfqpoint{3.092748in}{1.490711in}}%
\pgfpathclose%
\pgfusepath{stroke,fill}%
\end{pgfscope}%
\begin{pgfscope}%
\pgfpathrectangle{\pgfqpoint{0.600000in}{0.600000in}}{\pgfqpoint{3.900000in}{3.900000in}}%
\pgfusepath{clip}%
\pgfsetbuttcap%
\pgfsetroundjoin%
\definecolor{currentfill}{rgb}{0.121569,0.466667,0.705882}%
\pgfsetfillcolor{currentfill}%
\pgfsetlinewidth{1.003750pt}%
\definecolor{currentstroke}{rgb}{0.121569,0.466667,0.705882}%
\pgfsetstrokecolor{currentstroke}%
\pgfsetdash{}{0pt}%
\pgfpathmoveto{\pgfqpoint{2.328942in}{2.875449in}}%
\pgfpathcurveto{\pgfqpoint{2.339992in}{2.875449in}}{\pgfqpoint{2.350591in}{2.879840in}}{\pgfqpoint{2.358404in}{2.887653in}}%
\pgfpathcurveto{\pgfqpoint{2.366218in}{2.895467in}}{\pgfqpoint{2.370608in}{2.906066in}}{\pgfqpoint{2.370608in}{2.917116in}}%
\pgfpathcurveto{\pgfqpoint{2.370608in}{2.928166in}}{\pgfqpoint{2.366218in}{2.938765in}}{\pgfqpoint{2.358404in}{2.946579in}}%
\pgfpathcurveto{\pgfqpoint{2.350591in}{2.954393in}}{\pgfqpoint{2.339992in}{2.958783in}}{\pgfqpoint{2.328942in}{2.958783in}}%
\pgfpathcurveto{\pgfqpoint{2.317891in}{2.958783in}}{\pgfqpoint{2.307292in}{2.954393in}}{\pgfqpoint{2.299479in}{2.946579in}}%
\pgfpathcurveto{\pgfqpoint{2.291665in}{2.938765in}}{\pgfqpoint{2.287275in}{2.928166in}}{\pgfqpoint{2.287275in}{2.917116in}}%
\pgfpathcurveto{\pgfqpoint{2.287275in}{2.906066in}}{\pgfqpoint{2.291665in}{2.895467in}}{\pgfqpoint{2.299479in}{2.887653in}}%
\pgfpathcurveto{\pgfqpoint{2.307292in}{2.879840in}}{\pgfqpoint{2.317891in}{2.875449in}}{\pgfqpoint{2.328942in}{2.875449in}}%
\pgfpathclose%
\pgfusepath{stroke,fill}%
\end{pgfscope}%
\begin{pgfscope}%
\pgfpathrectangle{\pgfqpoint{0.600000in}{0.600000in}}{\pgfqpoint{3.900000in}{3.900000in}}%
\pgfusepath{clip}%
\pgfsetbuttcap%
\pgfsetroundjoin%
\definecolor{currentfill}{rgb}{0.121569,0.466667,0.705882}%
\pgfsetfillcolor{currentfill}%
\pgfsetlinewidth{1.003750pt}%
\definecolor{currentstroke}{rgb}{0.121569,0.466667,0.705882}%
\pgfsetstrokecolor{currentstroke}%
\pgfsetdash{}{0pt}%
\pgfpathmoveto{\pgfqpoint{2.347469in}{2.317939in}}%
\pgfpathcurveto{\pgfqpoint{2.358519in}{2.317939in}}{\pgfqpoint{2.369118in}{2.322329in}}{\pgfqpoint{2.376932in}{2.330143in}}%
\pgfpathcurveto{\pgfqpoint{2.384746in}{2.337956in}}{\pgfqpoint{2.389136in}{2.348556in}}{\pgfqpoint{2.389136in}{2.359606in}}%
\pgfpathcurveto{\pgfqpoint{2.389136in}{2.370656in}}{\pgfqpoint{2.384746in}{2.381255in}}{\pgfqpoint{2.376932in}{2.389068in}}%
\pgfpathcurveto{\pgfqpoint{2.369118in}{2.396882in}}{\pgfqpoint{2.358519in}{2.401272in}}{\pgfqpoint{2.347469in}{2.401272in}}%
\pgfpathcurveto{\pgfqpoint{2.336419in}{2.401272in}}{\pgfqpoint{2.325820in}{2.396882in}}{\pgfqpoint{2.318006in}{2.389068in}}%
\pgfpathcurveto{\pgfqpoint{2.310193in}{2.381255in}}{\pgfqpoint{2.305802in}{2.370656in}}{\pgfqpoint{2.305802in}{2.359606in}}%
\pgfpathcurveto{\pgfqpoint{2.305802in}{2.348556in}}{\pgfqpoint{2.310193in}{2.337956in}}{\pgfqpoint{2.318006in}{2.330143in}}%
\pgfpathcurveto{\pgfqpoint{2.325820in}{2.322329in}}{\pgfqpoint{2.336419in}{2.317939in}}{\pgfqpoint{2.347469in}{2.317939in}}%
\pgfpathclose%
\pgfusepath{stroke,fill}%
\end{pgfscope}%
\begin{pgfscope}%
\pgfpathrectangle{\pgfqpoint{0.600000in}{0.600000in}}{\pgfqpoint{3.900000in}{3.900000in}}%
\pgfusepath{clip}%
\pgfsetbuttcap%
\pgfsetroundjoin%
\definecolor{currentfill}{rgb}{0.121569,0.466667,0.705882}%
\pgfsetfillcolor{currentfill}%
\pgfsetlinewidth{1.003750pt}%
\definecolor{currentstroke}{rgb}{0.121569,0.466667,0.705882}%
\pgfsetstrokecolor{currentstroke}%
\pgfsetdash{}{0pt}%
\pgfpathmoveto{\pgfqpoint{2.652533in}{1.502739in}}%
\pgfpathcurveto{\pgfqpoint{2.663583in}{1.502739in}}{\pgfqpoint{2.674182in}{1.507129in}}{\pgfqpoint{2.681996in}{1.514943in}}%
\pgfpathcurveto{\pgfqpoint{2.689809in}{1.522757in}}{\pgfqpoint{2.694200in}{1.533356in}}{\pgfqpoint{2.694200in}{1.544406in}}%
\pgfpathcurveto{\pgfqpoint{2.694200in}{1.555456in}}{\pgfqpoint{2.689809in}{1.566055in}}{\pgfqpoint{2.681996in}{1.573869in}}%
\pgfpathcurveto{\pgfqpoint{2.674182in}{1.581682in}}{\pgfqpoint{2.663583in}{1.586073in}}{\pgfqpoint{2.652533in}{1.586073in}}%
\pgfpathcurveto{\pgfqpoint{2.641483in}{1.586073in}}{\pgfqpoint{2.630884in}{1.581682in}}{\pgfqpoint{2.623070in}{1.573869in}}%
\pgfpathcurveto{\pgfqpoint{2.615256in}{1.566055in}}{\pgfqpoint{2.610866in}{1.555456in}}{\pgfqpoint{2.610866in}{1.544406in}}%
\pgfpathcurveto{\pgfqpoint{2.610866in}{1.533356in}}{\pgfqpoint{2.615256in}{1.522757in}}{\pgfqpoint{2.623070in}{1.514943in}}%
\pgfpathcurveto{\pgfqpoint{2.630884in}{1.507129in}}{\pgfqpoint{2.641483in}{1.502739in}}{\pgfqpoint{2.652533in}{1.502739in}}%
\pgfpathclose%
\pgfusepath{stroke,fill}%
\end{pgfscope}%
\begin{pgfscope}%
\pgfpathrectangle{\pgfqpoint{0.600000in}{0.600000in}}{\pgfqpoint{3.900000in}{3.900000in}}%
\pgfusepath{clip}%
\pgfsetbuttcap%
\pgfsetroundjoin%
\definecolor{currentfill}{rgb}{0.121569,0.466667,0.705882}%
\pgfsetfillcolor{currentfill}%
\pgfsetlinewidth{1.003750pt}%
\definecolor{currentstroke}{rgb}{0.121569,0.466667,0.705882}%
\pgfsetstrokecolor{currentstroke}%
\pgfsetdash{}{0pt}%
\pgfpathmoveto{\pgfqpoint{3.438427in}{1.629790in}}%
\pgfpathcurveto{\pgfqpoint{3.449478in}{1.629790in}}{\pgfqpoint{3.460077in}{1.634180in}}{\pgfqpoint{3.467890in}{1.641994in}}%
\pgfpathcurveto{\pgfqpoint{3.475704in}{1.649808in}}{\pgfqpoint{3.480094in}{1.660407in}}{\pgfqpoint{3.480094in}{1.671457in}}%
\pgfpathcurveto{\pgfqpoint{3.480094in}{1.682507in}}{\pgfqpoint{3.475704in}{1.693106in}}{\pgfqpoint{3.467890in}{1.700919in}}%
\pgfpathcurveto{\pgfqpoint{3.460077in}{1.708733in}}{\pgfqpoint{3.449478in}{1.713123in}}{\pgfqpoint{3.438427in}{1.713123in}}%
\pgfpathcurveto{\pgfqpoint{3.427377in}{1.713123in}}{\pgfqpoint{3.416778in}{1.708733in}}{\pgfqpoint{3.408965in}{1.700919in}}%
\pgfpathcurveto{\pgfqpoint{3.401151in}{1.693106in}}{\pgfqpoint{3.396761in}{1.682507in}}{\pgfqpoint{3.396761in}{1.671457in}}%
\pgfpathcurveto{\pgfqpoint{3.396761in}{1.660407in}}{\pgfqpoint{3.401151in}{1.649808in}}{\pgfqpoint{3.408965in}{1.641994in}}%
\pgfpathcurveto{\pgfqpoint{3.416778in}{1.634180in}}{\pgfqpoint{3.427377in}{1.629790in}}{\pgfqpoint{3.438427in}{1.629790in}}%
\pgfpathclose%
\pgfusepath{stroke,fill}%
\end{pgfscope}%
\begin{pgfscope}%
\pgfpathrectangle{\pgfqpoint{0.600000in}{0.600000in}}{\pgfqpoint{3.900000in}{3.900000in}}%
\pgfusepath{clip}%
\pgfsetbuttcap%
\pgfsetroundjoin%
\definecolor{currentfill}{rgb}{0.121569,0.466667,0.705882}%
\pgfsetfillcolor{currentfill}%
\pgfsetlinewidth{1.003750pt}%
\definecolor{currentstroke}{rgb}{0.121569,0.466667,0.705882}%
\pgfsetstrokecolor{currentstroke}%
\pgfsetdash{}{0pt}%
\pgfpathmoveto{\pgfqpoint{2.353389in}{1.815506in}}%
\pgfpathcurveto{\pgfqpoint{2.364439in}{1.815506in}}{\pgfqpoint{2.375038in}{1.819896in}}{\pgfqpoint{2.382852in}{1.827710in}}%
\pgfpathcurveto{\pgfqpoint{2.390665in}{1.835523in}}{\pgfqpoint{2.395056in}{1.846122in}}{\pgfqpoint{2.395056in}{1.857172in}}%
\pgfpathcurveto{\pgfqpoint{2.395056in}{1.868223in}}{\pgfqpoint{2.390665in}{1.878822in}}{\pgfqpoint{2.382852in}{1.886635in}}%
\pgfpathcurveto{\pgfqpoint{2.375038in}{1.894449in}}{\pgfqpoint{2.364439in}{1.898839in}}{\pgfqpoint{2.353389in}{1.898839in}}%
\pgfpathcurveto{\pgfqpoint{2.342339in}{1.898839in}}{\pgfqpoint{2.331740in}{1.894449in}}{\pgfqpoint{2.323926in}{1.886635in}}%
\pgfpathcurveto{\pgfqpoint{2.316112in}{1.878822in}}{\pgfqpoint{2.311722in}{1.868223in}}{\pgfqpoint{2.311722in}{1.857172in}}%
\pgfpathcurveto{\pgfqpoint{2.311722in}{1.846122in}}{\pgfqpoint{2.316112in}{1.835523in}}{\pgfqpoint{2.323926in}{1.827710in}}%
\pgfpathcurveto{\pgfqpoint{2.331740in}{1.819896in}}{\pgfqpoint{2.342339in}{1.815506in}}{\pgfqpoint{2.353389in}{1.815506in}}%
\pgfpathclose%
\pgfusepath{stroke,fill}%
\end{pgfscope}%
\begin{pgfscope}%
\pgfpathrectangle{\pgfqpoint{0.600000in}{0.600000in}}{\pgfqpoint{3.900000in}{3.900000in}}%
\pgfusepath{clip}%
\pgfsetbuttcap%
\pgfsetroundjoin%
\definecolor{currentfill}{rgb}{0.121569,0.466667,0.705882}%
\pgfsetfillcolor{currentfill}%
\pgfsetlinewidth{1.003750pt}%
\definecolor{currentstroke}{rgb}{0.121569,0.466667,0.705882}%
\pgfsetstrokecolor{currentstroke}%
\pgfsetdash{}{0pt}%
\pgfpathmoveto{\pgfqpoint{2.346739in}{2.872651in}}%
\pgfpathcurveto{\pgfqpoint{2.357789in}{2.872651in}}{\pgfqpoint{2.368388in}{2.877041in}}{\pgfqpoint{2.376202in}{2.884855in}}%
\pgfpathcurveto{\pgfqpoint{2.384016in}{2.892668in}}{\pgfqpoint{2.388406in}{2.903267in}}{\pgfqpoint{2.388406in}{2.914318in}}%
\pgfpathcurveto{\pgfqpoint{2.388406in}{2.925368in}}{\pgfqpoint{2.384016in}{2.935967in}}{\pgfqpoint{2.376202in}{2.943780in}}%
\pgfpathcurveto{\pgfqpoint{2.368388in}{2.951594in}}{\pgfqpoint{2.357789in}{2.955984in}}{\pgfqpoint{2.346739in}{2.955984in}}%
\pgfpathcurveto{\pgfqpoint{2.335689in}{2.955984in}}{\pgfqpoint{2.325090in}{2.951594in}}{\pgfqpoint{2.317276in}{2.943780in}}%
\pgfpathcurveto{\pgfqpoint{2.309463in}{2.935967in}}{\pgfqpoint{2.305073in}{2.925368in}}{\pgfqpoint{2.305073in}{2.914318in}}%
\pgfpathcurveto{\pgfqpoint{2.305073in}{2.903267in}}{\pgfqpoint{2.309463in}{2.892668in}}{\pgfqpoint{2.317276in}{2.884855in}}%
\pgfpathcurveto{\pgfqpoint{2.325090in}{2.877041in}}{\pgfqpoint{2.335689in}{2.872651in}}{\pgfqpoint{2.346739in}{2.872651in}}%
\pgfpathclose%
\pgfusepath{stroke,fill}%
\end{pgfscope}%
\begin{pgfscope}%
\pgfpathrectangle{\pgfqpoint{0.600000in}{0.600000in}}{\pgfqpoint{3.900000in}{3.900000in}}%
\pgfusepath{clip}%
\pgfsetbuttcap%
\pgfsetroundjoin%
\definecolor{currentfill}{rgb}{0.121569,0.466667,0.705882}%
\pgfsetfillcolor{currentfill}%
\pgfsetlinewidth{1.003750pt}%
\definecolor{currentstroke}{rgb}{0.121569,0.466667,0.705882}%
\pgfsetstrokecolor{currentstroke}%
\pgfsetdash{}{0pt}%
\pgfpathmoveto{\pgfqpoint{2.631365in}{2.532697in}}%
\pgfpathcurveto{\pgfqpoint{2.642415in}{2.532697in}}{\pgfqpoint{2.653014in}{2.537087in}}{\pgfqpoint{2.660828in}{2.544900in}}%
\pgfpathcurveto{\pgfqpoint{2.668642in}{2.552714in}}{\pgfqpoint{2.673032in}{2.563313in}}{\pgfqpoint{2.673032in}{2.574363in}}%
\pgfpathcurveto{\pgfqpoint{2.673032in}{2.585413in}}{\pgfqpoint{2.668642in}{2.596012in}}{\pgfqpoint{2.660828in}{2.603826in}}%
\pgfpathcurveto{\pgfqpoint{2.653014in}{2.611640in}}{\pgfqpoint{2.642415in}{2.616030in}}{\pgfqpoint{2.631365in}{2.616030in}}%
\pgfpathcurveto{\pgfqpoint{2.620315in}{2.616030in}}{\pgfqpoint{2.609716in}{2.611640in}}{\pgfqpoint{2.601902in}{2.603826in}}%
\pgfpathcurveto{\pgfqpoint{2.594089in}{2.596012in}}{\pgfqpoint{2.589698in}{2.585413in}}{\pgfqpoint{2.589698in}{2.574363in}}%
\pgfpathcurveto{\pgfqpoint{2.589698in}{2.563313in}}{\pgfqpoint{2.594089in}{2.552714in}}{\pgfqpoint{2.601902in}{2.544900in}}%
\pgfpathcurveto{\pgfqpoint{2.609716in}{2.537087in}}{\pgfqpoint{2.620315in}{2.532697in}}{\pgfqpoint{2.631365in}{2.532697in}}%
\pgfpathclose%
\pgfusepath{stroke,fill}%
\end{pgfscope}%
\begin{pgfscope}%
\pgfpathrectangle{\pgfqpoint{0.600000in}{0.600000in}}{\pgfqpoint{3.900000in}{3.900000in}}%
\pgfusepath{clip}%
\pgfsetbuttcap%
\pgfsetroundjoin%
\definecolor{currentfill}{rgb}{0.121569,0.466667,0.705882}%
\pgfsetfillcolor{currentfill}%
\pgfsetlinewidth{1.003750pt}%
\definecolor{currentstroke}{rgb}{0.121569,0.466667,0.705882}%
\pgfsetstrokecolor{currentstroke}%
\pgfsetdash{}{0pt}%
\pgfpathmoveto{\pgfqpoint{2.435906in}{3.040068in}}%
\pgfpathcurveto{\pgfqpoint{2.446956in}{3.040068in}}{\pgfqpoint{2.457555in}{3.044458in}}{\pgfqpoint{2.465369in}{3.052272in}}%
\pgfpathcurveto{\pgfqpoint{2.473183in}{3.060085in}}{\pgfqpoint{2.477573in}{3.070684in}}{\pgfqpoint{2.477573in}{3.081734in}}%
\pgfpathcurveto{\pgfqpoint{2.477573in}{3.092785in}}{\pgfqpoint{2.473183in}{3.103384in}}{\pgfqpoint{2.465369in}{3.111197in}}%
\pgfpathcurveto{\pgfqpoint{2.457555in}{3.119011in}}{\pgfqpoint{2.446956in}{3.123401in}}{\pgfqpoint{2.435906in}{3.123401in}}%
\pgfpathcurveto{\pgfqpoint{2.424856in}{3.123401in}}{\pgfqpoint{2.414257in}{3.119011in}}{\pgfqpoint{2.406443in}{3.111197in}}%
\pgfpathcurveto{\pgfqpoint{2.398630in}{3.103384in}}{\pgfqpoint{2.394239in}{3.092785in}}{\pgfqpoint{2.394239in}{3.081734in}}%
\pgfpathcurveto{\pgfqpoint{2.394239in}{3.070684in}}{\pgfqpoint{2.398630in}{3.060085in}}{\pgfqpoint{2.406443in}{3.052272in}}%
\pgfpathcurveto{\pgfqpoint{2.414257in}{3.044458in}}{\pgfqpoint{2.424856in}{3.040068in}}{\pgfqpoint{2.435906in}{3.040068in}}%
\pgfpathclose%
\pgfusepath{stroke,fill}%
\end{pgfscope}%
\begin{pgfscope}%
\pgfpathrectangle{\pgfqpoint{0.600000in}{0.600000in}}{\pgfqpoint{3.900000in}{3.900000in}}%
\pgfusepath{clip}%
\pgfsetbuttcap%
\pgfsetroundjoin%
\definecolor{currentfill}{rgb}{0.121569,0.466667,0.705882}%
\pgfsetfillcolor{currentfill}%
\pgfsetlinewidth{1.003750pt}%
\definecolor{currentstroke}{rgb}{0.121569,0.466667,0.705882}%
\pgfsetstrokecolor{currentstroke}%
\pgfsetdash{}{0pt}%
\pgfpathmoveto{\pgfqpoint{2.366726in}{2.411136in}}%
\pgfpathcurveto{\pgfqpoint{2.377776in}{2.411136in}}{\pgfqpoint{2.388375in}{2.415526in}}{\pgfqpoint{2.396189in}{2.423340in}}%
\pgfpathcurveto{\pgfqpoint{2.404003in}{2.431153in}}{\pgfqpoint{2.408393in}{2.441752in}}{\pgfqpoint{2.408393in}{2.452802in}}%
\pgfpathcurveto{\pgfqpoint{2.408393in}{2.463853in}}{\pgfqpoint{2.404003in}{2.474452in}}{\pgfqpoint{2.396189in}{2.482265in}}%
\pgfpathcurveto{\pgfqpoint{2.388375in}{2.490079in}}{\pgfqpoint{2.377776in}{2.494469in}}{\pgfqpoint{2.366726in}{2.494469in}}%
\pgfpathcurveto{\pgfqpoint{2.355676in}{2.494469in}}{\pgfqpoint{2.345077in}{2.490079in}}{\pgfqpoint{2.337263in}{2.482265in}}%
\pgfpathcurveto{\pgfqpoint{2.329450in}{2.474452in}}{\pgfqpoint{2.325060in}{2.463853in}}{\pgfqpoint{2.325060in}{2.452802in}}%
\pgfpathcurveto{\pgfqpoint{2.325060in}{2.441752in}}{\pgfqpoint{2.329450in}{2.431153in}}{\pgfqpoint{2.337263in}{2.423340in}}%
\pgfpathcurveto{\pgfqpoint{2.345077in}{2.415526in}}{\pgfqpoint{2.355676in}{2.411136in}}{\pgfqpoint{2.366726in}{2.411136in}}%
\pgfpathclose%
\pgfusepath{stroke,fill}%
\end{pgfscope}%
\begin{pgfscope}%
\pgfpathrectangle{\pgfqpoint{0.600000in}{0.600000in}}{\pgfqpoint{3.900000in}{3.900000in}}%
\pgfusepath{clip}%
\pgfsetbuttcap%
\pgfsetroundjoin%
\definecolor{currentfill}{rgb}{0.121569,0.466667,0.705882}%
\pgfsetfillcolor{currentfill}%
\pgfsetlinewidth{1.003750pt}%
\definecolor{currentstroke}{rgb}{0.121569,0.466667,0.705882}%
\pgfsetstrokecolor{currentstroke}%
\pgfsetdash{}{0pt}%
\pgfpathmoveto{\pgfqpoint{3.249858in}{2.197757in}}%
\pgfpathcurveto{\pgfqpoint{3.260908in}{2.197757in}}{\pgfqpoint{3.271507in}{2.202147in}}{\pgfqpoint{3.279321in}{2.209961in}}%
\pgfpathcurveto{\pgfqpoint{3.287134in}{2.217775in}}{\pgfqpoint{3.291525in}{2.228374in}}{\pgfqpoint{3.291525in}{2.239424in}}%
\pgfpathcurveto{\pgfqpoint{3.291525in}{2.250474in}}{\pgfqpoint{3.287134in}{2.261073in}}{\pgfqpoint{3.279321in}{2.268887in}}%
\pgfpathcurveto{\pgfqpoint{3.271507in}{2.276700in}}{\pgfqpoint{3.260908in}{2.281090in}}{\pgfqpoint{3.249858in}{2.281090in}}%
\pgfpathcurveto{\pgfqpoint{3.238808in}{2.281090in}}{\pgfqpoint{3.228209in}{2.276700in}}{\pgfqpoint{3.220395in}{2.268887in}}%
\pgfpathcurveto{\pgfqpoint{3.212581in}{2.261073in}}{\pgfqpoint{3.208191in}{2.250474in}}{\pgfqpoint{3.208191in}{2.239424in}}%
\pgfpathcurveto{\pgfqpoint{3.208191in}{2.228374in}}{\pgfqpoint{3.212581in}{2.217775in}}{\pgfqpoint{3.220395in}{2.209961in}}%
\pgfpathcurveto{\pgfqpoint{3.228209in}{2.202147in}}{\pgfqpoint{3.238808in}{2.197757in}}{\pgfqpoint{3.249858in}{2.197757in}}%
\pgfpathclose%
\pgfusepath{stroke,fill}%
\end{pgfscope}%
\begin{pgfscope}%
\pgfpathrectangle{\pgfqpoint{0.600000in}{0.600000in}}{\pgfqpoint{3.900000in}{3.900000in}}%
\pgfusepath{clip}%
\pgfsetbuttcap%
\pgfsetroundjoin%
\definecolor{currentfill}{rgb}{0.121569,0.466667,0.705882}%
\pgfsetfillcolor{currentfill}%
\pgfsetlinewidth{1.003750pt}%
\definecolor{currentstroke}{rgb}{0.121569,0.466667,0.705882}%
\pgfsetstrokecolor{currentstroke}%
\pgfsetdash{}{0pt}%
\pgfpathmoveto{\pgfqpoint{3.309359in}{2.807066in}}%
\pgfpathcurveto{\pgfqpoint{3.320409in}{2.807066in}}{\pgfqpoint{3.331008in}{2.811457in}}{\pgfqpoint{3.338822in}{2.819270in}}%
\pgfpathcurveto{\pgfqpoint{3.346635in}{2.827084in}}{\pgfqpoint{3.351026in}{2.837683in}}{\pgfqpoint{3.351026in}{2.848733in}}%
\pgfpathcurveto{\pgfqpoint{3.351026in}{2.859783in}}{\pgfqpoint{3.346635in}{2.870382in}}{\pgfqpoint{3.338822in}{2.878196in}}%
\pgfpathcurveto{\pgfqpoint{3.331008in}{2.886009in}}{\pgfqpoint{3.320409in}{2.890400in}}{\pgfqpoint{3.309359in}{2.890400in}}%
\pgfpathcurveto{\pgfqpoint{3.298309in}{2.890400in}}{\pgfqpoint{3.287710in}{2.886009in}}{\pgfqpoint{3.279896in}{2.878196in}}%
\pgfpathcurveto{\pgfqpoint{3.272083in}{2.870382in}}{\pgfqpoint{3.267692in}{2.859783in}}{\pgfqpoint{3.267692in}{2.848733in}}%
\pgfpathcurveto{\pgfqpoint{3.267692in}{2.837683in}}{\pgfqpoint{3.272083in}{2.827084in}}{\pgfqpoint{3.279896in}{2.819270in}}%
\pgfpathcurveto{\pgfqpoint{3.287710in}{2.811457in}}{\pgfqpoint{3.298309in}{2.807066in}}{\pgfqpoint{3.309359in}{2.807066in}}%
\pgfpathclose%
\pgfusepath{stroke,fill}%
\end{pgfscope}%
\begin{pgfscope}%
\pgfpathrectangle{\pgfqpoint{0.600000in}{0.600000in}}{\pgfqpoint{3.900000in}{3.900000in}}%
\pgfusepath{clip}%
\pgfsetbuttcap%
\pgfsetroundjoin%
\definecolor{currentfill}{rgb}{0.121569,0.466667,0.705882}%
\pgfsetfillcolor{currentfill}%
\pgfsetlinewidth{1.003750pt}%
\definecolor{currentstroke}{rgb}{0.121569,0.466667,0.705882}%
\pgfsetstrokecolor{currentstroke}%
\pgfsetdash{}{0pt}%
\pgfpathmoveto{\pgfqpoint{3.343691in}{3.238966in}}%
\pgfpathcurveto{\pgfqpoint{3.354741in}{3.238966in}}{\pgfqpoint{3.365340in}{3.243356in}}{\pgfqpoint{3.373153in}{3.251170in}}%
\pgfpathcurveto{\pgfqpoint{3.380967in}{3.258984in}}{\pgfqpoint{3.385357in}{3.269583in}}{\pgfqpoint{3.385357in}{3.280633in}}%
\pgfpathcurveto{\pgfqpoint{3.385357in}{3.291683in}}{\pgfqpoint{3.380967in}{3.302282in}}{\pgfqpoint{3.373153in}{3.310095in}}%
\pgfpathcurveto{\pgfqpoint{3.365340in}{3.317909in}}{\pgfqpoint{3.354741in}{3.322299in}}{\pgfqpoint{3.343691in}{3.322299in}}%
\pgfpathcurveto{\pgfqpoint{3.332640in}{3.322299in}}{\pgfqpoint{3.322041in}{3.317909in}}{\pgfqpoint{3.314228in}{3.310095in}}%
\pgfpathcurveto{\pgfqpoint{3.306414in}{3.302282in}}{\pgfqpoint{3.302024in}{3.291683in}}{\pgfqpoint{3.302024in}{3.280633in}}%
\pgfpathcurveto{\pgfqpoint{3.302024in}{3.269583in}}{\pgfqpoint{3.306414in}{3.258984in}}{\pgfqpoint{3.314228in}{3.251170in}}%
\pgfpathcurveto{\pgfqpoint{3.322041in}{3.243356in}}{\pgfqpoint{3.332640in}{3.238966in}}{\pgfqpoint{3.343691in}{3.238966in}}%
\pgfpathclose%
\pgfusepath{stroke,fill}%
\end{pgfscope}%
\begin{pgfscope}%
\pgfpathrectangle{\pgfqpoint{0.600000in}{0.600000in}}{\pgfqpoint{3.900000in}{3.900000in}}%
\pgfusepath{clip}%
\pgfsetbuttcap%
\pgfsetroundjoin%
\definecolor{currentfill}{rgb}{0.121569,0.466667,0.705882}%
\pgfsetfillcolor{currentfill}%
\pgfsetlinewidth{1.003750pt}%
\definecolor{currentstroke}{rgb}{0.121569,0.466667,0.705882}%
\pgfsetstrokecolor{currentstroke}%
\pgfsetdash{}{0pt}%
\pgfpathmoveto{\pgfqpoint{2.958869in}{1.559885in}}%
\pgfpathcurveto{\pgfqpoint{2.969919in}{1.559885in}}{\pgfqpoint{2.980518in}{1.564276in}}{\pgfqpoint{2.988332in}{1.572089in}}%
\pgfpathcurveto{\pgfqpoint{2.996145in}{1.579903in}}{\pgfqpoint{3.000536in}{1.590502in}}{\pgfqpoint{3.000536in}{1.601552in}}%
\pgfpathcurveto{\pgfqpoint{3.000536in}{1.612602in}}{\pgfqpoint{2.996145in}{1.623201in}}{\pgfqpoint{2.988332in}{1.631015in}}%
\pgfpathcurveto{\pgfqpoint{2.980518in}{1.638828in}}{\pgfqpoint{2.969919in}{1.643219in}}{\pgfqpoint{2.958869in}{1.643219in}}%
\pgfpathcurveto{\pgfqpoint{2.947819in}{1.643219in}}{\pgfqpoint{2.937220in}{1.638828in}}{\pgfqpoint{2.929406in}{1.631015in}}%
\pgfpathcurveto{\pgfqpoint{2.921593in}{1.623201in}}{\pgfqpoint{2.917202in}{1.612602in}}{\pgfqpoint{2.917202in}{1.601552in}}%
\pgfpathcurveto{\pgfqpoint{2.917202in}{1.590502in}}{\pgfqpoint{2.921593in}{1.579903in}}{\pgfqpoint{2.929406in}{1.572089in}}%
\pgfpathcurveto{\pgfqpoint{2.937220in}{1.564276in}}{\pgfqpoint{2.947819in}{1.559885in}}{\pgfqpoint{2.958869in}{1.559885in}}%
\pgfpathclose%
\pgfusepath{stroke,fill}%
\end{pgfscope}%
\begin{pgfscope}%
\pgfpathrectangle{\pgfqpoint{0.600000in}{0.600000in}}{\pgfqpoint{3.900000in}{3.900000in}}%
\pgfusepath{clip}%
\pgfsetbuttcap%
\pgfsetroundjoin%
\definecolor{currentfill}{rgb}{0.121569,0.466667,0.705882}%
\pgfsetfillcolor{currentfill}%
\pgfsetlinewidth{1.003750pt}%
\definecolor{currentstroke}{rgb}{0.121569,0.466667,0.705882}%
\pgfsetstrokecolor{currentstroke}%
\pgfsetdash{}{0pt}%
\pgfpathmoveto{\pgfqpoint{3.383646in}{1.768491in}}%
\pgfpathcurveto{\pgfqpoint{3.394696in}{1.768491in}}{\pgfqpoint{3.405295in}{1.772881in}}{\pgfqpoint{3.413109in}{1.780694in}}%
\pgfpathcurveto{\pgfqpoint{3.420923in}{1.788508in}}{\pgfqpoint{3.425313in}{1.799107in}}{\pgfqpoint{3.425313in}{1.810157in}}%
\pgfpathcurveto{\pgfqpoint{3.425313in}{1.821207in}}{\pgfqpoint{3.420923in}{1.831806in}}{\pgfqpoint{3.413109in}{1.839620in}}%
\pgfpathcurveto{\pgfqpoint{3.405295in}{1.847434in}}{\pgfqpoint{3.394696in}{1.851824in}}{\pgfqpoint{3.383646in}{1.851824in}}%
\pgfpathcurveto{\pgfqpoint{3.372596in}{1.851824in}}{\pgfqpoint{3.361997in}{1.847434in}}{\pgfqpoint{3.354183in}{1.839620in}}%
\pgfpathcurveto{\pgfqpoint{3.346370in}{1.831806in}}{\pgfqpoint{3.341980in}{1.821207in}}{\pgfqpoint{3.341980in}{1.810157in}}%
\pgfpathcurveto{\pgfqpoint{3.341980in}{1.799107in}}{\pgfqpoint{3.346370in}{1.788508in}}{\pgfqpoint{3.354183in}{1.780694in}}%
\pgfpathcurveto{\pgfqpoint{3.361997in}{1.772881in}}{\pgfqpoint{3.372596in}{1.768491in}}{\pgfqpoint{3.383646in}{1.768491in}}%
\pgfpathclose%
\pgfusepath{stroke,fill}%
\end{pgfscope}%
\begin{pgfscope}%
\pgfpathrectangle{\pgfqpoint{0.600000in}{0.600000in}}{\pgfqpoint{3.900000in}{3.900000in}}%
\pgfusepath{clip}%
\pgfsetbuttcap%
\pgfsetroundjoin%
\definecolor{currentfill}{rgb}{0.121569,0.466667,0.705882}%
\pgfsetfillcolor{currentfill}%
\pgfsetlinewidth{1.003750pt}%
\definecolor{currentstroke}{rgb}{0.121569,0.466667,0.705882}%
\pgfsetstrokecolor{currentstroke}%
\pgfsetdash{}{0pt}%
\pgfpathmoveto{\pgfqpoint{1.784444in}{3.286910in}}%
\pgfpathcurveto{\pgfqpoint{1.795494in}{3.286910in}}{\pgfqpoint{1.806093in}{3.291300in}}{\pgfqpoint{1.813906in}{3.299114in}}%
\pgfpathcurveto{\pgfqpoint{1.821720in}{3.306928in}}{\pgfqpoint{1.826110in}{3.317527in}}{\pgfqpoint{1.826110in}{3.328577in}}%
\pgfpathcurveto{\pgfqpoint{1.826110in}{3.339627in}}{\pgfqpoint{1.821720in}{3.350226in}}{\pgfqpoint{1.813906in}{3.358040in}}%
\pgfpathcurveto{\pgfqpoint{1.806093in}{3.365853in}}{\pgfqpoint{1.795494in}{3.370243in}}{\pgfqpoint{1.784444in}{3.370243in}}%
\pgfpathcurveto{\pgfqpoint{1.773393in}{3.370243in}}{\pgfqpoint{1.762794in}{3.365853in}}{\pgfqpoint{1.754981in}{3.358040in}}%
\pgfpathcurveto{\pgfqpoint{1.747167in}{3.350226in}}{\pgfqpoint{1.742777in}{3.339627in}}{\pgfqpoint{1.742777in}{3.328577in}}%
\pgfpathcurveto{\pgfqpoint{1.742777in}{3.317527in}}{\pgfqpoint{1.747167in}{3.306928in}}{\pgfqpoint{1.754981in}{3.299114in}}%
\pgfpathcurveto{\pgfqpoint{1.762794in}{3.291300in}}{\pgfqpoint{1.773393in}{3.286910in}}{\pgfqpoint{1.784444in}{3.286910in}}%
\pgfpathclose%
\pgfusepath{stroke,fill}%
\end{pgfscope}%
\begin{pgfscope}%
\pgfpathrectangle{\pgfqpoint{0.600000in}{0.600000in}}{\pgfqpoint{3.900000in}{3.900000in}}%
\pgfusepath{clip}%
\pgfsetbuttcap%
\pgfsetroundjoin%
\definecolor{currentfill}{rgb}{0.121569,0.466667,0.705882}%
\pgfsetfillcolor{currentfill}%
\pgfsetlinewidth{1.003750pt}%
\definecolor{currentstroke}{rgb}{0.121569,0.466667,0.705882}%
\pgfsetstrokecolor{currentstroke}%
\pgfsetdash{}{0pt}%
\pgfpathmoveto{\pgfqpoint{2.622028in}{2.830842in}}%
\pgfpathcurveto{\pgfqpoint{2.633078in}{2.830842in}}{\pgfqpoint{2.643677in}{2.835232in}}{\pgfqpoint{2.651491in}{2.843045in}}%
\pgfpathcurveto{\pgfqpoint{2.659304in}{2.850859in}}{\pgfqpoint{2.663694in}{2.861458in}}{\pgfqpoint{2.663694in}{2.872508in}}%
\pgfpathcurveto{\pgfqpoint{2.663694in}{2.883558in}}{\pgfqpoint{2.659304in}{2.894157in}}{\pgfqpoint{2.651491in}{2.901971in}}%
\pgfpathcurveto{\pgfqpoint{2.643677in}{2.909785in}}{\pgfqpoint{2.633078in}{2.914175in}}{\pgfqpoint{2.622028in}{2.914175in}}%
\pgfpathcurveto{\pgfqpoint{2.610978in}{2.914175in}}{\pgfqpoint{2.600379in}{2.909785in}}{\pgfqpoint{2.592565in}{2.901971in}}%
\pgfpathcurveto{\pgfqpoint{2.584751in}{2.894157in}}{\pgfqpoint{2.580361in}{2.883558in}}{\pgfqpoint{2.580361in}{2.872508in}}%
\pgfpathcurveto{\pgfqpoint{2.580361in}{2.861458in}}{\pgfqpoint{2.584751in}{2.850859in}}{\pgfqpoint{2.592565in}{2.843045in}}%
\pgfpathcurveto{\pgfqpoint{2.600379in}{2.835232in}}{\pgfqpoint{2.610978in}{2.830842in}}{\pgfqpoint{2.622028in}{2.830842in}}%
\pgfpathclose%
\pgfusepath{stroke,fill}%
\end{pgfscope}%
\begin{pgfscope}%
\pgfpathrectangle{\pgfqpoint{0.600000in}{0.600000in}}{\pgfqpoint{3.900000in}{3.900000in}}%
\pgfusepath{clip}%
\pgfsetbuttcap%
\pgfsetroundjoin%
\definecolor{currentfill}{rgb}{0.121569,0.466667,0.705882}%
\pgfsetfillcolor{currentfill}%
\pgfsetlinewidth{1.003750pt}%
\definecolor{currentstroke}{rgb}{0.121569,0.466667,0.705882}%
\pgfsetstrokecolor{currentstroke}%
\pgfsetdash{}{0pt}%
\pgfpathmoveto{\pgfqpoint{2.813804in}{2.801155in}}%
\pgfpathcurveto{\pgfqpoint{2.824854in}{2.801155in}}{\pgfqpoint{2.835453in}{2.805545in}}{\pgfqpoint{2.843267in}{2.813359in}}%
\pgfpathcurveto{\pgfqpoint{2.851081in}{2.821172in}}{\pgfqpoint{2.855471in}{2.831772in}}{\pgfqpoint{2.855471in}{2.842822in}}%
\pgfpathcurveto{\pgfqpoint{2.855471in}{2.853872in}}{\pgfqpoint{2.851081in}{2.864471in}}{\pgfqpoint{2.843267in}{2.872284in}}%
\pgfpathcurveto{\pgfqpoint{2.835453in}{2.880098in}}{\pgfqpoint{2.824854in}{2.884488in}}{\pgfqpoint{2.813804in}{2.884488in}}%
\pgfpathcurveto{\pgfqpoint{2.802754in}{2.884488in}}{\pgfqpoint{2.792155in}{2.880098in}}{\pgfqpoint{2.784342in}{2.872284in}}%
\pgfpathcurveto{\pgfqpoint{2.776528in}{2.864471in}}{\pgfqpoint{2.772138in}{2.853872in}}{\pgfqpoint{2.772138in}{2.842822in}}%
\pgfpathcurveto{\pgfqpoint{2.772138in}{2.831772in}}{\pgfqpoint{2.776528in}{2.821172in}}{\pgfqpoint{2.784342in}{2.813359in}}%
\pgfpathcurveto{\pgfqpoint{2.792155in}{2.805545in}}{\pgfqpoint{2.802754in}{2.801155in}}{\pgfqpoint{2.813804in}{2.801155in}}%
\pgfpathclose%
\pgfusepath{stroke,fill}%
\end{pgfscope}%
\begin{pgfscope}%
\pgfpathrectangle{\pgfqpoint{0.600000in}{0.600000in}}{\pgfqpoint{3.900000in}{3.900000in}}%
\pgfusepath{clip}%
\pgfsetbuttcap%
\pgfsetroundjoin%
\definecolor{currentfill}{rgb}{0.121569,0.466667,0.705882}%
\pgfsetfillcolor{currentfill}%
\pgfsetlinewidth{1.003750pt}%
\definecolor{currentstroke}{rgb}{0.121569,0.466667,0.705882}%
\pgfsetstrokecolor{currentstroke}%
\pgfsetdash{}{0pt}%
\pgfpathmoveto{\pgfqpoint{3.145554in}{2.724187in}}%
\pgfpathcurveto{\pgfqpoint{3.156604in}{2.724187in}}{\pgfqpoint{3.167203in}{2.728577in}}{\pgfqpoint{3.175017in}{2.736391in}}%
\pgfpathcurveto{\pgfqpoint{3.182830in}{2.744205in}}{\pgfqpoint{3.187220in}{2.754804in}}{\pgfqpoint{3.187220in}{2.765854in}}%
\pgfpathcurveto{\pgfqpoint{3.187220in}{2.776904in}}{\pgfqpoint{3.182830in}{2.787503in}}{\pgfqpoint{3.175017in}{2.795317in}}%
\pgfpathcurveto{\pgfqpoint{3.167203in}{2.803130in}}{\pgfqpoint{3.156604in}{2.807520in}}{\pgfqpoint{3.145554in}{2.807520in}}%
\pgfpathcurveto{\pgfqpoint{3.134504in}{2.807520in}}{\pgfqpoint{3.123905in}{2.803130in}}{\pgfqpoint{3.116091in}{2.795317in}}%
\pgfpathcurveto{\pgfqpoint{3.108277in}{2.787503in}}{\pgfqpoint{3.103887in}{2.776904in}}{\pgfqpoint{3.103887in}{2.765854in}}%
\pgfpathcurveto{\pgfqpoint{3.103887in}{2.754804in}}{\pgfqpoint{3.108277in}{2.744205in}}{\pgfqpoint{3.116091in}{2.736391in}}%
\pgfpathcurveto{\pgfqpoint{3.123905in}{2.728577in}}{\pgfqpoint{3.134504in}{2.724187in}}{\pgfqpoint{3.145554in}{2.724187in}}%
\pgfpathclose%
\pgfusepath{stroke,fill}%
\end{pgfscope}%
\begin{pgfscope}%
\pgfpathrectangle{\pgfqpoint{0.600000in}{0.600000in}}{\pgfqpoint{3.900000in}{3.900000in}}%
\pgfusepath{clip}%
\pgfsetbuttcap%
\pgfsetroundjoin%
\definecolor{currentfill}{rgb}{0.121569,0.466667,0.705882}%
\pgfsetfillcolor{currentfill}%
\pgfsetlinewidth{1.003750pt}%
\definecolor{currentstroke}{rgb}{0.121569,0.466667,0.705882}%
\pgfsetstrokecolor{currentstroke}%
\pgfsetdash{}{0pt}%
\pgfpathmoveto{\pgfqpoint{2.265348in}{2.986168in}}%
\pgfpathcurveto{\pgfqpoint{2.276398in}{2.986168in}}{\pgfqpoint{2.286997in}{2.990558in}}{\pgfqpoint{2.294811in}{2.998372in}}%
\pgfpathcurveto{\pgfqpoint{2.302624in}{3.006186in}}{\pgfqpoint{2.307015in}{3.016785in}}{\pgfqpoint{2.307015in}{3.027835in}}%
\pgfpathcurveto{\pgfqpoint{2.307015in}{3.038885in}}{\pgfqpoint{2.302624in}{3.049484in}}{\pgfqpoint{2.294811in}{3.057298in}}%
\pgfpathcurveto{\pgfqpoint{2.286997in}{3.065111in}}{\pgfqpoint{2.276398in}{3.069501in}}{\pgfqpoint{2.265348in}{3.069501in}}%
\pgfpathcurveto{\pgfqpoint{2.254298in}{3.069501in}}{\pgfqpoint{2.243699in}{3.065111in}}{\pgfqpoint{2.235885in}{3.057298in}}%
\pgfpathcurveto{\pgfqpoint{2.228072in}{3.049484in}}{\pgfqpoint{2.223681in}{3.038885in}}{\pgfqpoint{2.223681in}{3.027835in}}%
\pgfpathcurveto{\pgfqpoint{2.223681in}{3.016785in}}{\pgfqpoint{2.228072in}{3.006186in}}{\pgfqpoint{2.235885in}{2.998372in}}%
\pgfpathcurveto{\pgfqpoint{2.243699in}{2.990558in}}{\pgfqpoint{2.254298in}{2.986168in}}{\pgfqpoint{2.265348in}{2.986168in}}%
\pgfpathclose%
\pgfusepath{stroke,fill}%
\end{pgfscope}%
\begin{pgfscope}%
\pgfpathrectangle{\pgfqpoint{0.600000in}{0.600000in}}{\pgfqpoint{3.900000in}{3.900000in}}%
\pgfusepath{clip}%
\pgfsetbuttcap%
\pgfsetroundjoin%
\definecolor{currentfill}{rgb}{0.121569,0.466667,0.705882}%
\pgfsetfillcolor{currentfill}%
\pgfsetlinewidth{1.003750pt}%
\definecolor{currentstroke}{rgb}{0.121569,0.466667,0.705882}%
\pgfsetstrokecolor{currentstroke}%
\pgfsetdash{}{0pt}%
\pgfpathmoveto{\pgfqpoint{2.717517in}{1.878586in}}%
\pgfpathcurveto{\pgfqpoint{2.728567in}{1.878586in}}{\pgfqpoint{2.739166in}{1.882976in}}{\pgfqpoint{2.746980in}{1.890790in}}%
\pgfpathcurveto{\pgfqpoint{2.754794in}{1.898603in}}{\pgfqpoint{2.759184in}{1.909202in}}{\pgfqpoint{2.759184in}{1.920252in}}%
\pgfpathcurveto{\pgfqpoint{2.759184in}{1.931303in}}{\pgfqpoint{2.754794in}{1.941902in}}{\pgfqpoint{2.746980in}{1.949715in}}%
\pgfpathcurveto{\pgfqpoint{2.739166in}{1.957529in}}{\pgfqpoint{2.728567in}{1.961919in}}{\pgfqpoint{2.717517in}{1.961919in}}%
\pgfpathcurveto{\pgfqpoint{2.706467in}{1.961919in}}{\pgfqpoint{2.695868in}{1.957529in}}{\pgfqpoint{2.688054in}{1.949715in}}%
\pgfpathcurveto{\pgfqpoint{2.680241in}{1.941902in}}{\pgfqpoint{2.675851in}{1.931303in}}{\pgfqpoint{2.675851in}{1.920252in}}%
\pgfpathcurveto{\pgfqpoint{2.675851in}{1.909202in}}{\pgfqpoint{2.680241in}{1.898603in}}{\pgfqpoint{2.688054in}{1.890790in}}%
\pgfpathcurveto{\pgfqpoint{2.695868in}{1.882976in}}{\pgfqpoint{2.706467in}{1.878586in}}{\pgfqpoint{2.717517in}{1.878586in}}%
\pgfpathclose%
\pgfusepath{stroke,fill}%
\end{pgfscope}%
\begin{pgfscope}%
\pgfpathrectangle{\pgfqpoint{0.600000in}{0.600000in}}{\pgfqpoint{3.900000in}{3.900000in}}%
\pgfusepath{clip}%
\pgfsetbuttcap%
\pgfsetroundjoin%
\definecolor{currentfill}{rgb}{0.121569,0.466667,0.705882}%
\pgfsetfillcolor{currentfill}%
\pgfsetlinewidth{1.003750pt}%
\definecolor{currentstroke}{rgb}{0.121569,0.466667,0.705882}%
\pgfsetstrokecolor{currentstroke}%
\pgfsetdash{}{0pt}%
\pgfpathmoveto{\pgfqpoint{1.064963in}{2.592167in}}%
\pgfpathcurveto{\pgfqpoint{1.076013in}{2.592167in}}{\pgfqpoint{1.086612in}{2.596557in}}{\pgfqpoint{1.094425in}{2.604371in}}%
\pgfpathcurveto{\pgfqpoint{1.102239in}{2.612184in}}{\pgfqpoint{1.106629in}{2.622783in}}{\pgfqpoint{1.106629in}{2.633833in}}%
\pgfpathcurveto{\pgfqpoint{1.106629in}{2.644884in}}{\pgfqpoint{1.102239in}{2.655483in}}{\pgfqpoint{1.094425in}{2.663296in}}%
\pgfpathcurveto{\pgfqpoint{1.086612in}{2.671110in}}{\pgfqpoint{1.076013in}{2.675500in}}{\pgfqpoint{1.064963in}{2.675500in}}%
\pgfpathcurveto{\pgfqpoint{1.053913in}{2.675500in}}{\pgfqpoint{1.043313in}{2.671110in}}{\pgfqpoint{1.035500in}{2.663296in}}%
\pgfpathcurveto{\pgfqpoint{1.027686in}{2.655483in}}{\pgfqpoint{1.023296in}{2.644884in}}{\pgfqpoint{1.023296in}{2.633833in}}%
\pgfpathcurveto{\pgfqpoint{1.023296in}{2.622783in}}{\pgfqpoint{1.027686in}{2.612184in}}{\pgfqpoint{1.035500in}{2.604371in}}%
\pgfpathcurveto{\pgfqpoint{1.043313in}{2.596557in}}{\pgfqpoint{1.053913in}{2.592167in}}{\pgfqpoint{1.064963in}{2.592167in}}%
\pgfpathclose%
\pgfusepath{stroke,fill}%
\end{pgfscope}%
\begin{pgfscope}%
\pgfpathrectangle{\pgfqpoint{0.600000in}{0.600000in}}{\pgfqpoint{3.900000in}{3.900000in}}%
\pgfusepath{clip}%
\pgfsetbuttcap%
\pgfsetroundjoin%
\definecolor{currentfill}{rgb}{0.121569,0.466667,0.705882}%
\pgfsetfillcolor{currentfill}%
\pgfsetlinewidth{1.003750pt}%
\definecolor{currentstroke}{rgb}{0.121569,0.466667,0.705882}%
\pgfsetstrokecolor{currentstroke}%
\pgfsetdash{}{0pt}%
\pgfpathmoveto{\pgfqpoint{3.012220in}{0.901228in}}%
\pgfpathcurveto{\pgfqpoint{3.023270in}{0.901228in}}{\pgfqpoint{3.033869in}{0.905618in}}{\pgfqpoint{3.041682in}{0.913432in}}%
\pgfpathcurveto{\pgfqpoint{3.049496in}{0.921245in}}{\pgfqpoint{3.053886in}{0.931844in}}{\pgfqpoint{3.053886in}{0.942895in}}%
\pgfpathcurveto{\pgfqpoint{3.053886in}{0.953945in}}{\pgfqpoint{3.049496in}{0.964544in}}{\pgfqpoint{3.041682in}{0.972357in}}%
\pgfpathcurveto{\pgfqpoint{3.033869in}{0.980171in}}{\pgfqpoint{3.023270in}{0.984561in}}{\pgfqpoint{3.012220in}{0.984561in}}%
\pgfpathcurveto{\pgfqpoint{3.001169in}{0.984561in}}{\pgfqpoint{2.990570in}{0.980171in}}{\pgfqpoint{2.982757in}{0.972357in}}%
\pgfpathcurveto{\pgfqpoint{2.974943in}{0.964544in}}{\pgfqpoint{2.970553in}{0.953945in}}{\pgfqpoint{2.970553in}{0.942895in}}%
\pgfpathcurveto{\pgfqpoint{2.970553in}{0.931844in}}{\pgfqpoint{2.974943in}{0.921245in}}{\pgfqpoint{2.982757in}{0.913432in}}%
\pgfpathcurveto{\pgfqpoint{2.990570in}{0.905618in}}{\pgfqpoint{3.001169in}{0.901228in}}{\pgfqpoint{3.012220in}{0.901228in}}%
\pgfpathclose%
\pgfusepath{stroke,fill}%
\end{pgfscope}%
\begin{pgfscope}%
\pgfpathrectangle{\pgfqpoint{0.600000in}{0.600000in}}{\pgfqpoint{3.900000in}{3.900000in}}%
\pgfusepath{clip}%
\pgfsetbuttcap%
\pgfsetroundjoin%
\definecolor{currentfill}{rgb}{0.121569,0.466667,0.705882}%
\pgfsetfillcolor{currentfill}%
\pgfsetlinewidth{1.003750pt}%
\definecolor{currentstroke}{rgb}{0.121569,0.466667,0.705882}%
\pgfsetstrokecolor{currentstroke}%
\pgfsetdash{}{0pt}%
\pgfpathmoveto{\pgfqpoint{2.293510in}{2.047704in}}%
\pgfpathcurveto{\pgfqpoint{2.304561in}{2.047704in}}{\pgfqpoint{2.315160in}{2.052094in}}{\pgfqpoint{2.322973in}{2.059908in}}%
\pgfpathcurveto{\pgfqpoint{2.330787in}{2.067721in}}{\pgfqpoint{2.335177in}{2.078320in}}{\pgfqpoint{2.335177in}{2.089370in}}%
\pgfpathcurveto{\pgfqpoint{2.335177in}{2.100421in}}{\pgfqpoint{2.330787in}{2.111020in}}{\pgfqpoint{2.322973in}{2.118833in}}%
\pgfpathcurveto{\pgfqpoint{2.315160in}{2.126647in}}{\pgfqpoint{2.304561in}{2.131037in}}{\pgfqpoint{2.293510in}{2.131037in}}%
\pgfpathcurveto{\pgfqpoint{2.282460in}{2.131037in}}{\pgfqpoint{2.271861in}{2.126647in}}{\pgfqpoint{2.264048in}{2.118833in}}%
\pgfpathcurveto{\pgfqpoint{2.256234in}{2.111020in}}{\pgfqpoint{2.251844in}{2.100421in}}{\pgfqpoint{2.251844in}{2.089370in}}%
\pgfpathcurveto{\pgfqpoint{2.251844in}{2.078320in}}{\pgfqpoint{2.256234in}{2.067721in}}{\pgfqpoint{2.264048in}{2.059908in}}%
\pgfpathcurveto{\pgfqpoint{2.271861in}{2.052094in}}{\pgfqpoint{2.282460in}{2.047704in}}{\pgfqpoint{2.293510in}{2.047704in}}%
\pgfpathclose%
\pgfusepath{stroke,fill}%
\end{pgfscope}%
\begin{pgfscope}%
\pgfpathrectangle{\pgfqpoint{0.600000in}{0.600000in}}{\pgfqpoint{3.900000in}{3.900000in}}%
\pgfusepath{clip}%
\pgfsetbuttcap%
\pgfsetroundjoin%
\definecolor{currentfill}{rgb}{0.121569,0.466667,0.705882}%
\pgfsetfillcolor{currentfill}%
\pgfsetlinewidth{1.003750pt}%
\definecolor{currentstroke}{rgb}{0.121569,0.466667,0.705882}%
\pgfsetstrokecolor{currentstroke}%
\pgfsetdash{}{0pt}%
\pgfpathmoveto{\pgfqpoint{3.338843in}{2.275163in}}%
\pgfpathcurveto{\pgfqpoint{3.349893in}{2.275163in}}{\pgfqpoint{3.360492in}{2.279554in}}{\pgfqpoint{3.368306in}{2.287367in}}%
\pgfpathcurveto{\pgfqpoint{3.376119in}{2.295181in}}{\pgfqpoint{3.380510in}{2.305780in}}{\pgfqpoint{3.380510in}{2.316830in}}%
\pgfpathcurveto{\pgfqpoint{3.380510in}{2.327880in}}{\pgfqpoint{3.376119in}{2.338479in}}{\pgfqpoint{3.368306in}{2.346293in}}%
\pgfpathcurveto{\pgfqpoint{3.360492in}{2.354106in}}{\pgfqpoint{3.349893in}{2.358497in}}{\pgfqpoint{3.338843in}{2.358497in}}%
\pgfpathcurveto{\pgfqpoint{3.327793in}{2.358497in}}{\pgfqpoint{3.317194in}{2.354106in}}{\pgfqpoint{3.309380in}{2.346293in}}%
\pgfpathcurveto{\pgfqpoint{3.301567in}{2.338479in}}{\pgfqpoint{3.297176in}{2.327880in}}{\pgfqpoint{3.297176in}{2.316830in}}%
\pgfpathcurveto{\pgfqpoint{3.297176in}{2.305780in}}{\pgfqpoint{3.301567in}{2.295181in}}{\pgfqpoint{3.309380in}{2.287367in}}%
\pgfpathcurveto{\pgfqpoint{3.317194in}{2.279554in}}{\pgfqpoint{3.327793in}{2.275163in}}{\pgfqpoint{3.338843in}{2.275163in}}%
\pgfpathclose%
\pgfusepath{stroke,fill}%
\end{pgfscope}%
\begin{pgfscope}%
\pgfpathrectangle{\pgfqpoint{0.600000in}{0.600000in}}{\pgfqpoint{3.900000in}{3.900000in}}%
\pgfusepath{clip}%
\pgfsetbuttcap%
\pgfsetroundjoin%
\definecolor{currentfill}{rgb}{0.121569,0.466667,0.705882}%
\pgfsetfillcolor{currentfill}%
\pgfsetlinewidth{1.003750pt}%
\definecolor{currentstroke}{rgb}{0.121569,0.466667,0.705882}%
\pgfsetstrokecolor{currentstroke}%
\pgfsetdash{}{0pt}%
\pgfpathmoveto{\pgfqpoint{2.675084in}{3.095147in}}%
\pgfpathcurveto{\pgfqpoint{2.686134in}{3.095147in}}{\pgfqpoint{2.696733in}{3.099538in}}{\pgfqpoint{2.704547in}{3.107351in}}%
\pgfpathcurveto{\pgfqpoint{2.712360in}{3.115165in}}{\pgfqpoint{2.716751in}{3.125764in}}{\pgfqpoint{2.716751in}{3.136814in}}%
\pgfpathcurveto{\pgfqpoint{2.716751in}{3.147864in}}{\pgfqpoint{2.712360in}{3.158463in}}{\pgfqpoint{2.704547in}{3.166277in}}%
\pgfpathcurveto{\pgfqpoint{2.696733in}{3.174091in}}{\pgfqpoint{2.686134in}{3.178481in}}{\pgfqpoint{2.675084in}{3.178481in}}%
\pgfpathcurveto{\pgfqpoint{2.664034in}{3.178481in}}{\pgfqpoint{2.653435in}{3.174091in}}{\pgfqpoint{2.645621in}{3.166277in}}%
\pgfpathcurveto{\pgfqpoint{2.637807in}{3.158463in}}{\pgfqpoint{2.633417in}{3.147864in}}{\pgfqpoint{2.633417in}{3.136814in}}%
\pgfpathcurveto{\pgfqpoint{2.633417in}{3.125764in}}{\pgfqpoint{2.637807in}{3.115165in}}{\pgfqpoint{2.645621in}{3.107351in}}%
\pgfpathcurveto{\pgfqpoint{2.653435in}{3.099538in}}{\pgfqpoint{2.664034in}{3.095147in}}{\pgfqpoint{2.675084in}{3.095147in}}%
\pgfpathclose%
\pgfusepath{stroke,fill}%
\end{pgfscope}%
\begin{pgfscope}%
\pgfpathrectangle{\pgfqpoint{0.600000in}{0.600000in}}{\pgfqpoint{3.900000in}{3.900000in}}%
\pgfusepath{clip}%
\pgfsetbuttcap%
\pgfsetroundjoin%
\definecolor{currentfill}{rgb}{0.121569,0.466667,0.705882}%
\pgfsetfillcolor{currentfill}%
\pgfsetlinewidth{1.003750pt}%
\definecolor{currentstroke}{rgb}{0.121569,0.466667,0.705882}%
\pgfsetstrokecolor{currentstroke}%
\pgfsetdash{}{0pt}%
\pgfpathmoveto{\pgfqpoint{2.689567in}{2.329469in}}%
\pgfpathcurveto{\pgfqpoint{2.700617in}{2.329469in}}{\pgfqpoint{2.711216in}{2.333859in}}{\pgfqpoint{2.719029in}{2.341673in}}%
\pgfpathcurveto{\pgfqpoint{2.726843in}{2.349487in}}{\pgfqpoint{2.731233in}{2.360086in}}{\pgfqpoint{2.731233in}{2.371136in}}%
\pgfpathcurveto{\pgfqpoint{2.731233in}{2.382186in}}{\pgfqpoint{2.726843in}{2.392785in}}{\pgfqpoint{2.719029in}{2.400599in}}%
\pgfpathcurveto{\pgfqpoint{2.711216in}{2.408412in}}{\pgfqpoint{2.700617in}{2.412802in}}{\pgfqpoint{2.689567in}{2.412802in}}%
\pgfpathcurveto{\pgfqpoint{2.678517in}{2.412802in}}{\pgfqpoint{2.667918in}{2.408412in}}{\pgfqpoint{2.660104in}{2.400599in}}%
\pgfpathcurveto{\pgfqpoint{2.652290in}{2.392785in}}{\pgfqpoint{2.647900in}{2.382186in}}{\pgfqpoint{2.647900in}{2.371136in}}%
\pgfpathcurveto{\pgfqpoint{2.647900in}{2.360086in}}{\pgfqpoint{2.652290in}{2.349487in}}{\pgfqpoint{2.660104in}{2.341673in}}%
\pgfpathcurveto{\pgfqpoint{2.667918in}{2.333859in}}{\pgfqpoint{2.678517in}{2.329469in}}{\pgfqpoint{2.689567in}{2.329469in}}%
\pgfpathclose%
\pgfusepath{stroke,fill}%
\end{pgfscope}%
\begin{pgfscope}%
\pgfpathrectangle{\pgfqpoint{0.600000in}{0.600000in}}{\pgfqpoint{3.900000in}{3.900000in}}%
\pgfusepath{clip}%
\pgfsetbuttcap%
\pgfsetroundjoin%
\definecolor{currentfill}{rgb}{0.121569,0.466667,0.705882}%
\pgfsetfillcolor{currentfill}%
\pgfsetlinewidth{1.003750pt}%
\definecolor{currentstroke}{rgb}{0.121569,0.466667,0.705882}%
\pgfsetstrokecolor{currentstroke}%
\pgfsetdash{}{0pt}%
\pgfpathmoveto{\pgfqpoint{2.929305in}{2.578207in}}%
\pgfpathcurveto{\pgfqpoint{2.940355in}{2.578207in}}{\pgfqpoint{2.950955in}{2.582597in}}{\pgfqpoint{2.958768in}{2.590411in}}%
\pgfpathcurveto{\pgfqpoint{2.966582in}{2.598224in}}{\pgfqpoint{2.970972in}{2.608823in}}{\pgfqpoint{2.970972in}{2.619874in}}%
\pgfpathcurveto{\pgfqpoint{2.970972in}{2.630924in}}{\pgfqpoint{2.966582in}{2.641523in}}{\pgfqpoint{2.958768in}{2.649336in}}%
\pgfpathcurveto{\pgfqpoint{2.950955in}{2.657150in}}{\pgfqpoint{2.940355in}{2.661540in}}{\pgfqpoint{2.929305in}{2.661540in}}%
\pgfpathcurveto{\pgfqpoint{2.918255in}{2.661540in}}{\pgfqpoint{2.907656in}{2.657150in}}{\pgfqpoint{2.899843in}{2.649336in}}%
\pgfpathcurveto{\pgfqpoint{2.892029in}{2.641523in}}{\pgfqpoint{2.887639in}{2.630924in}}{\pgfqpoint{2.887639in}{2.619874in}}%
\pgfpathcurveto{\pgfqpoint{2.887639in}{2.608823in}}{\pgfqpoint{2.892029in}{2.598224in}}{\pgfqpoint{2.899843in}{2.590411in}}%
\pgfpathcurveto{\pgfqpoint{2.907656in}{2.582597in}}{\pgfqpoint{2.918255in}{2.578207in}}{\pgfqpoint{2.929305in}{2.578207in}}%
\pgfpathclose%
\pgfusepath{stroke,fill}%
\end{pgfscope}%
\begin{pgfscope}%
\pgfpathrectangle{\pgfqpoint{0.600000in}{0.600000in}}{\pgfqpoint{3.900000in}{3.900000in}}%
\pgfusepath{clip}%
\pgfsetbuttcap%
\pgfsetroundjoin%
\definecolor{currentfill}{rgb}{0.121569,0.466667,0.705882}%
\pgfsetfillcolor{currentfill}%
\pgfsetlinewidth{1.003750pt}%
\definecolor{currentstroke}{rgb}{0.121569,0.466667,0.705882}%
\pgfsetstrokecolor{currentstroke}%
\pgfsetdash{}{0pt}%
\pgfpathmoveto{\pgfqpoint{2.193651in}{3.513124in}}%
\pgfpathcurveto{\pgfqpoint{2.204701in}{3.513124in}}{\pgfqpoint{2.215300in}{3.517515in}}{\pgfqpoint{2.223114in}{3.525328in}}%
\pgfpathcurveto{\pgfqpoint{2.230928in}{3.533142in}}{\pgfqpoint{2.235318in}{3.543741in}}{\pgfqpoint{2.235318in}{3.554791in}}%
\pgfpathcurveto{\pgfqpoint{2.235318in}{3.565841in}}{\pgfqpoint{2.230928in}{3.576440in}}{\pgfqpoint{2.223114in}{3.584254in}}%
\pgfpathcurveto{\pgfqpoint{2.215300in}{3.592067in}}{\pgfqpoint{2.204701in}{3.596458in}}{\pgfqpoint{2.193651in}{3.596458in}}%
\pgfpathcurveto{\pgfqpoint{2.182601in}{3.596458in}}{\pgfqpoint{2.172002in}{3.592067in}}{\pgfqpoint{2.164188in}{3.584254in}}%
\pgfpathcurveto{\pgfqpoint{2.156375in}{3.576440in}}{\pgfqpoint{2.151985in}{3.565841in}}{\pgfqpoint{2.151985in}{3.554791in}}%
\pgfpathcurveto{\pgfqpoint{2.151985in}{3.543741in}}{\pgfqpoint{2.156375in}{3.533142in}}{\pgfqpoint{2.164188in}{3.525328in}}%
\pgfpathcurveto{\pgfqpoint{2.172002in}{3.517515in}}{\pgfqpoint{2.182601in}{3.513124in}}{\pgfqpoint{2.193651in}{3.513124in}}%
\pgfpathclose%
\pgfusepath{stroke,fill}%
\end{pgfscope}%
\begin{pgfscope}%
\pgfpathrectangle{\pgfqpoint{0.600000in}{0.600000in}}{\pgfqpoint{3.900000in}{3.900000in}}%
\pgfusepath{clip}%
\pgfsetbuttcap%
\pgfsetroundjoin%
\definecolor{currentfill}{rgb}{0.121569,0.466667,0.705882}%
\pgfsetfillcolor{currentfill}%
\pgfsetlinewidth{1.003750pt}%
\definecolor{currentstroke}{rgb}{0.121569,0.466667,0.705882}%
\pgfsetstrokecolor{currentstroke}%
\pgfsetdash{}{0pt}%
\pgfpathmoveto{\pgfqpoint{2.407534in}{2.167548in}}%
\pgfpathcurveto{\pgfqpoint{2.418584in}{2.167548in}}{\pgfqpoint{2.429183in}{2.171938in}}{\pgfqpoint{2.436997in}{2.179752in}}%
\pgfpathcurveto{\pgfqpoint{2.444811in}{2.187565in}}{\pgfqpoint{2.449201in}{2.198164in}}{\pgfqpoint{2.449201in}{2.209215in}}%
\pgfpathcurveto{\pgfqpoint{2.449201in}{2.220265in}}{\pgfqpoint{2.444811in}{2.230864in}}{\pgfqpoint{2.436997in}{2.238677in}}%
\pgfpathcurveto{\pgfqpoint{2.429183in}{2.246491in}}{\pgfqpoint{2.418584in}{2.250881in}}{\pgfqpoint{2.407534in}{2.250881in}}%
\pgfpathcurveto{\pgfqpoint{2.396484in}{2.250881in}}{\pgfqpoint{2.385885in}{2.246491in}}{\pgfqpoint{2.378071in}{2.238677in}}%
\pgfpathcurveto{\pgfqpoint{2.370258in}{2.230864in}}{\pgfqpoint{2.365867in}{2.220265in}}{\pgfqpoint{2.365867in}{2.209215in}}%
\pgfpathcurveto{\pgfqpoint{2.365867in}{2.198164in}}{\pgfqpoint{2.370258in}{2.187565in}}{\pgfqpoint{2.378071in}{2.179752in}}%
\pgfpathcurveto{\pgfqpoint{2.385885in}{2.171938in}}{\pgfqpoint{2.396484in}{2.167548in}}{\pgfqpoint{2.407534in}{2.167548in}}%
\pgfpathclose%
\pgfusepath{stroke,fill}%
\end{pgfscope}%
\begin{pgfscope}%
\pgfpathrectangle{\pgfqpoint{0.600000in}{0.600000in}}{\pgfqpoint{3.900000in}{3.900000in}}%
\pgfusepath{clip}%
\pgfsetbuttcap%
\pgfsetroundjoin%
\definecolor{currentfill}{rgb}{0.121569,0.466667,0.705882}%
\pgfsetfillcolor{currentfill}%
\pgfsetlinewidth{1.003750pt}%
\definecolor{currentstroke}{rgb}{0.121569,0.466667,0.705882}%
\pgfsetstrokecolor{currentstroke}%
\pgfsetdash{}{0pt}%
\pgfpathmoveto{\pgfqpoint{2.103676in}{3.392226in}}%
\pgfpathcurveto{\pgfqpoint{2.114726in}{3.392226in}}{\pgfqpoint{2.125325in}{3.396616in}}{\pgfqpoint{2.133138in}{3.404429in}}%
\pgfpathcurveto{\pgfqpoint{2.140952in}{3.412243in}}{\pgfqpoint{2.145342in}{3.422842in}}{\pgfqpoint{2.145342in}{3.433892in}}%
\pgfpathcurveto{\pgfqpoint{2.145342in}{3.444942in}}{\pgfqpoint{2.140952in}{3.455541in}}{\pgfqpoint{2.133138in}{3.463355in}}%
\pgfpathcurveto{\pgfqpoint{2.125325in}{3.471169in}}{\pgfqpoint{2.114726in}{3.475559in}}{\pgfqpoint{2.103676in}{3.475559in}}%
\pgfpathcurveto{\pgfqpoint{2.092625in}{3.475559in}}{\pgfqpoint{2.082026in}{3.471169in}}{\pgfqpoint{2.074213in}{3.463355in}}%
\pgfpathcurveto{\pgfqpoint{2.066399in}{3.455541in}}{\pgfqpoint{2.062009in}{3.444942in}}{\pgfqpoint{2.062009in}{3.433892in}}%
\pgfpathcurveto{\pgfqpoint{2.062009in}{3.422842in}}{\pgfqpoint{2.066399in}{3.412243in}}{\pgfqpoint{2.074213in}{3.404429in}}%
\pgfpathcurveto{\pgfqpoint{2.082026in}{3.396616in}}{\pgfqpoint{2.092625in}{3.392226in}}{\pgfqpoint{2.103676in}{3.392226in}}%
\pgfpathclose%
\pgfusepath{stroke,fill}%
\end{pgfscope}%
\begin{pgfscope}%
\pgfpathrectangle{\pgfqpoint{0.600000in}{0.600000in}}{\pgfqpoint{3.900000in}{3.900000in}}%
\pgfusepath{clip}%
\pgfsetbuttcap%
\pgfsetroundjoin%
\definecolor{currentfill}{rgb}{0.121569,0.466667,0.705882}%
\pgfsetfillcolor{currentfill}%
\pgfsetlinewidth{1.003750pt}%
\definecolor{currentstroke}{rgb}{0.121569,0.466667,0.705882}%
\pgfsetstrokecolor{currentstroke}%
\pgfsetdash{}{0pt}%
\pgfpathmoveto{\pgfqpoint{2.578257in}{2.122417in}}%
\pgfpathcurveto{\pgfqpoint{2.589308in}{2.122417in}}{\pgfqpoint{2.599907in}{2.126807in}}{\pgfqpoint{2.607720in}{2.134621in}}%
\pgfpathcurveto{\pgfqpoint{2.615534in}{2.142434in}}{\pgfqpoint{2.619924in}{2.153033in}}{\pgfqpoint{2.619924in}{2.164083in}}%
\pgfpathcurveto{\pgfqpoint{2.619924in}{2.175133in}}{\pgfqpoint{2.615534in}{2.185733in}}{\pgfqpoint{2.607720in}{2.193546in}}%
\pgfpathcurveto{\pgfqpoint{2.599907in}{2.201360in}}{\pgfqpoint{2.589308in}{2.205750in}}{\pgfqpoint{2.578257in}{2.205750in}}%
\pgfpathcurveto{\pgfqpoint{2.567207in}{2.205750in}}{\pgfqpoint{2.556608in}{2.201360in}}{\pgfqpoint{2.548795in}{2.193546in}}%
\pgfpathcurveto{\pgfqpoint{2.540981in}{2.185733in}}{\pgfqpoint{2.536591in}{2.175133in}}{\pgfqpoint{2.536591in}{2.164083in}}%
\pgfpathcurveto{\pgfqpoint{2.536591in}{2.153033in}}{\pgfqpoint{2.540981in}{2.142434in}}{\pgfqpoint{2.548795in}{2.134621in}}%
\pgfpathcurveto{\pgfqpoint{2.556608in}{2.126807in}}{\pgfqpoint{2.567207in}{2.122417in}}{\pgfqpoint{2.578257in}{2.122417in}}%
\pgfpathclose%
\pgfusepath{stroke,fill}%
\end{pgfscope}%
\begin{pgfscope}%
\pgfpathrectangle{\pgfqpoint{0.600000in}{0.600000in}}{\pgfqpoint{3.900000in}{3.900000in}}%
\pgfusepath{clip}%
\pgfsetbuttcap%
\pgfsetroundjoin%
\definecolor{currentfill}{rgb}{0.121569,0.466667,0.705882}%
\pgfsetfillcolor{currentfill}%
\pgfsetlinewidth{1.003750pt}%
\definecolor{currentstroke}{rgb}{0.121569,0.466667,0.705882}%
\pgfsetstrokecolor{currentstroke}%
\pgfsetdash{}{0pt}%
\pgfpathmoveto{\pgfqpoint{2.390092in}{2.841992in}}%
\pgfpathcurveto{\pgfqpoint{2.401142in}{2.841992in}}{\pgfqpoint{2.411741in}{2.846383in}}{\pgfqpoint{2.419555in}{2.854196in}}%
\pgfpathcurveto{\pgfqpoint{2.427368in}{2.862010in}}{\pgfqpoint{2.431759in}{2.872609in}}{\pgfqpoint{2.431759in}{2.883659in}}%
\pgfpathcurveto{\pgfqpoint{2.431759in}{2.894709in}}{\pgfqpoint{2.427368in}{2.905308in}}{\pgfqpoint{2.419555in}{2.913122in}}%
\pgfpathcurveto{\pgfqpoint{2.411741in}{2.920935in}}{\pgfqpoint{2.401142in}{2.925326in}}{\pgfqpoint{2.390092in}{2.925326in}}%
\pgfpathcurveto{\pgfqpoint{2.379042in}{2.925326in}}{\pgfqpoint{2.368443in}{2.920935in}}{\pgfqpoint{2.360629in}{2.913122in}}%
\pgfpathcurveto{\pgfqpoint{2.352816in}{2.905308in}}{\pgfqpoint{2.348425in}{2.894709in}}{\pgfqpoint{2.348425in}{2.883659in}}%
\pgfpathcurveto{\pgfqpoint{2.348425in}{2.872609in}}{\pgfqpoint{2.352816in}{2.862010in}}{\pgfqpoint{2.360629in}{2.854196in}}%
\pgfpathcurveto{\pgfqpoint{2.368443in}{2.846383in}}{\pgfqpoint{2.379042in}{2.841992in}}{\pgfqpoint{2.390092in}{2.841992in}}%
\pgfpathclose%
\pgfusepath{stroke,fill}%
\end{pgfscope}%
\begin{pgfscope}%
\pgfpathrectangle{\pgfqpoint{0.600000in}{0.600000in}}{\pgfqpoint{3.900000in}{3.900000in}}%
\pgfusepath{clip}%
\pgfsetbuttcap%
\pgfsetroundjoin%
\definecolor{currentfill}{rgb}{0.121569,0.466667,0.705882}%
\pgfsetfillcolor{currentfill}%
\pgfsetlinewidth{1.003750pt}%
\definecolor{currentstroke}{rgb}{0.121569,0.466667,0.705882}%
\pgfsetstrokecolor{currentstroke}%
\pgfsetdash{}{0pt}%
\pgfpathmoveto{\pgfqpoint{2.427169in}{2.391461in}}%
\pgfpathcurveto{\pgfqpoint{2.438219in}{2.391461in}}{\pgfqpoint{2.448818in}{2.395851in}}{\pgfqpoint{2.456632in}{2.403665in}}%
\pgfpathcurveto{\pgfqpoint{2.464445in}{2.411478in}}{\pgfqpoint{2.468836in}{2.422077in}}{\pgfqpoint{2.468836in}{2.433128in}}%
\pgfpathcurveto{\pgfqpoint{2.468836in}{2.444178in}}{\pgfqpoint{2.464445in}{2.454777in}}{\pgfqpoint{2.456632in}{2.462590in}}%
\pgfpathcurveto{\pgfqpoint{2.448818in}{2.470404in}}{\pgfqpoint{2.438219in}{2.474794in}}{\pgfqpoint{2.427169in}{2.474794in}}%
\pgfpathcurveto{\pgfqpoint{2.416119in}{2.474794in}}{\pgfqpoint{2.405520in}{2.470404in}}{\pgfqpoint{2.397706in}{2.462590in}}%
\pgfpathcurveto{\pgfqpoint{2.389892in}{2.454777in}}{\pgfqpoint{2.385502in}{2.444178in}}{\pgfqpoint{2.385502in}{2.433128in}}%
\pgfpathcurveto{\pgfqpoint{2.385502in}{2.422077in}}{\pgfqpoint{2.389892in}{2.411478in}}{\pgfqpoint{2.397706in}{2.403665in}}%
\pgfpathcurveto{\pgfqpoint{2.405520in}{2.395851in}}{\pgfqpoint{2.416119in}{2.391461in}}{\pgfqpoint{2.427169in}{2.391461in}}%
\pgfpathclose%
\pgfusepath{stroke,fill}%
\end{pgfscope}%
\begin{pgfscope}%
\pgfpathrectangle{\pgfqpoint{0.600000in}{0.600000in}}{\pgfqpoint{3.900000in}{3.900000in}}%
\pgfusepath{clip}%
\pgfsetbuttcap%
\pgfsetroundjoin%
\definecolor{currentfill}{rgb}{0.121569,0.466667,0.705882}%
\pgfsetfillcolor{currentfill}%
\pgfsetlinewidth{1.003750pt}%
\definecolor{currentstroke}{rgb}{0.121569,0.466667,0.705882}%
\pgfsetstrokecolor{currentstroke}%
\pgfsetdash{}{0pt}%
\pgfpathmoveto{\pgfqpoint{2.633393in}{2.518673in}}%
\pgfpathcurveto{\pgfqpoint{2.644443in}{2.518673in}}{\pgfqpoint{2.655042in}{2.523063in}}{\pgfqpoint{2.662856in}{2.530877in}}%
\pgfpathcurveto{\pgfqpoint{2.670669in}{2.538690in}}{\pgfqpoint{2.675060in}{2.549289in}}{\pgfqpoint{2.675060in}{2.560339in}}%
\pgfpathcurveto{\pgfqpoint{2.675060in}{2.571390in}}{\pgfqpoint{2.670669in}{2.581989in}}{\pgfqpoint{2.662856in}{2.589802in}}%
\pgfpathcurveto{\pgfqpoint{2.655042in}{2.597616in}}{\pgfqpoint{2.644443in}{2.602006in}}{\pgfqpoint{2.633393in}{2.602006in}}%
\pgfpathcurveto{\pgfqpoint{2.622343in}{2.602006in}}{\pgfqpoint{2.611744in}{2.597616in}}{\pgfqpoint{2.603930in}{2.589802in}}%
\pgfpathcurveto{\pgfqpoint{2.596117in}{2.581989in}}{\pgfqpoint{2.591726in}{2.571390in}}{\pgfqpoint{2.591726in}{2.560339in}}%
\pgfpathcurveto{\pgfqpoint{2.591726in}{2.549289in}}{\pgfqpoint{2.596117in}{2.538690in}}{\pgfqpoint{2.603930in}{2.530877in}}%
\pgfpathcurveto{\pgfqpoint{2.611744in}{2.523063in}}{\pgfqpoint{2.622343in}{2.518673in}}{\pgfqpoint{2.633393in}{2.518673in}}%
\pgfpathclose%
\pgfusepath{stroke,fill}%
\end{pgfscope}%
\begin{pgfscope}%
\pgfpathrectangle{\pgfqpoint{0.600000in}{0.600000in}}{\pgfqpoint{3.900000in}{3.900000in}}%
\pgfusepath{clip}%
\pgfsetbuttcap%
\pgfsetroundjoin%
\definecolor{currentfill}{rgb}{0.121569,0.466667,0.705882}%
\pgfsetfillcolor{currentfill}%
\pgfsetlinewidth{1.003750pt}%
\definecolor{currentstroke}{rgb}{0.121569,0.466667,0.705882}%
\pgfsetstrokecolor{currentstroke}%
\pgfsetdash{}{0pt}%
\pgfpathmoveto{\pgfqpoint{2.934915in}{1.640069in}}%
\pgfpathcurveto{\pgfqpoint{2.945965in}{1.640069in}}{\pgfqpoint{2.956564in}{1.644460in}}{\pgfqpoint{2.964377in}{1.652273in}}%
\pgfpathcurveto{\pgfqpoint{2.972191in}{1.660087in}}{\pgfqpoint{2.976581in}{1.670686in}}{\pgfqpoint{2.976581in}{1.681736in}}%
\pgfpathcurveto{\pgfqpoint{2.976581in}{1.692786in}}{\pgfqpoint{2.972191in}{1.703385in}}{\pgfqpoint{2.964377in}{1.711199in}}%
\pgfpathcurveto{\pgfqpoint{2.956564in}{1.719013in}}{\pgfqpoint{2.945965in}{1.723403in}}{\pgfqpoint{2.934915in}{1.723403in}}%
\pgfpathcurveto{\pgfqpoint{2.923864in}{1.723403in}}{\pgfqpoint{2.913265in}{1.719013in}}{\pgfqpoint{2.905452in}{1.711199in}}%
\pgfpathcurveto{\pgfqpoint{2.897638in}{1.703385in}}{\pgfqpoint{2.893248in}{1.692786in}}{\pgfqpoint{2.893248in}{1.681736in}}%
\pgfpathcurveto{\pgfqpoint{2.893248in}{1.670686in}}{\pgfqpoint{2.897638in}{1.660087in}}{\pgfqpoint{2.905452in}{1.652273in}}%
\pgfpathcurveto{\pgfqpoint{2.913265in}{1.644460in}}{\pgfqpoint{2.923864in}{1.640069in}}{\pgfqpoint{2.934915in}{1.640069in}}%
\pgfpathclose%
\pgfusepath{stroke,fill}%
\end{pgfscope}%
\begin{pgfscope}%
\pgfpathrectangle{\pgfqpoint{0.600000in}{0.600000in}}{\pgfqpoint{3.900000in}{3.900000in}}%
\pgfusepath{clip}%
\pgfsetbuttcap%
\pgfsetroundjoin%
\definecolor{currentfill}{rgb}{0.121569,0.466667,0.705882}%
\pgfsetfillcolor{currentfill}%
\pgfsetlinewidth{1.003750pt}%
\definecolor{currentstroke}{rgb}{0.121569,0.466667,0.705882}%
\pgfsetstrokecolor{currentstroke}%
\pgfsetdash{}{0pt}%
\pgfpathmoveto{\pgfqpoint{2.936112in}{1.331790in}}%
\pgfpathcurveto{\pgfqpoint{2.947163in}{1.331790in}}{\pgfqpoint{2.957762in}{1.336180in}}{\pgfqpoint{2.965575in}{1.343994in}}%
\pgfpathcurveto{\pgfqpoint{2.973389in}{1.351807in}}{\pgfqpoint{2.977779in}{1.362406in}}{\pgfqpoint{2.977779in}{1.373456in}}%
\pgfpathcurveto{\pgfqpoint{2.977779in}{1.384507in}}{\pgfqpoint{2.973389in}{1.395106in}}{\pgfqpoint{2.965575in}{1.402919in}}%
\pgfpathcurveto{\pgfqpoint{2.957762in}{1.410733in}}{\pgfqpoint{2.947163in}{1.415123in}}{\pgfqpoint{2.936112in}{1.415123in}}%
\pgfpathcurveto{\pgfqpoint{2.925062in}{1.415123in}}{\pgfqpoint{2.914463in}{1.410733in}}{\pgfqpoint{2.906650in}{1.402919in}}%
\pgfpathcurveto{\pgfqpoint{2.898836in}{1.395106in}}{\pgfqpoint{2.894446in}{1.384507in}}{\pgfqpoint{2.894446in}{1.373456in}}%
\pgfpathcurveto{\pgfqpoint{2.894446in}{1.362406in}}{\pgfqpoint{2.898836in}{1.351807in}}{\pgfqpoint{2.906650in}{1.343994in}}%
\pgfpathcurveto{\pgfqpoint{2.914463in}{1.336180in}}{\pgfqpoint{2.925062in}{1.331790in}}{\pgfqpoint{2.936112in}{1.331790in}}%
\pgfpathclose%
\pgfusepath{stroke,fill}%
\end{pgfscope}%
\begin{pgfscope}%
\pgfpathrectangle{\pgfqpoint{0.600000in}{0.600000in}}{\pgfqpoint{3.900000in}{3.900000in}}%
\pgfusepath{clip}%
\pgfsetbuttcap%
\pgfsetroundjoin%
\definecolor{currentfill}{rgb}{0.121569,0.466667,0.705882}%
\pgfsetfillcolor{currentfill}%
\pgfsetlinewidth{1.003750pt}%
\definecolor{currentstroke}{rgb}{0.121569,0.466667,0.705882}%
\pgfsetstrokecolor{currentstroke}%
\pgfsetdash{}{0pt}%
\pgfpathmoveto{\pgfqpoint{1.867725in}{2.249974in}}%
\pgfpathcurveto{\pgfqpoint{1.878775in}{2.249974in}}{\pgfqpoint{1.889374in}{2.254364in}}{\pgfqpoint{1.897188in}{2.262178in}}%
\pgfpathcurveto{\pgfqpoint{1.905001in}{2.269991in}}{\pgfqpoint{1.909391in}{2.280591in}}{\pgfqpoint{1.909391in}{2.291641in}}%
\pgfpathcurveto{\pgfqpoint{1.909391in}{2.302691in}}{\pgfqpoint{1.905001in}{2.313290in}}{\pgfqpoint{1.897188in}{2.321103in}}%
\pgfpathcurveto{\pgfqpoint{1.889374in}{2.328917in}}{\pgfqpoint{1.878775in}{2.333307in}}{\pgfqpoint{1.867725in}{2.333307in}}%
\pgfpathcurveto{\pgfqpoint{1.856675in}{2.333307in}}{\pgfqpoint{1.846076in}{2.328917in}}{\pgfqpoint{1.838262in}{2.321103in}}%
\pgfpathcurveto{\pgfqpoint{1.830448in}{2.313290in}}{\pgfqpoint{1.826058in}{2.302691in}}{\pgfqpoint{1.826058in}{2.291641in}}%
\pgfpathcurveto{\pgfqpoint{1.826058in}{2.280591in}}{\pgfqpoint{1.830448in}{2.269991in}}{\pgfqpoint{1.838262in}{2.262178in}}%
\pgfpathcurveto{\pgfqpoint{1.846076in}{2.254364in}}{\pgfqpoint{1.856675in}{2.249974in}}{\pgfqpoint{1.867725in}{2.249974in}}%
\pgfpathclose%
\pgfusepath{stroke,fill}%
\end{pgfscope}%
\begin{pgfscope}%
\pgfpathrectangle{\pgfqpoint{0.600000in}{0.600000in}}{\pgfqpoint{3.900000in}{3.900000in}}%
\pgfusepath{clip}%
\pgfsetbuttcap%
\pgfsetroundjoin%
\definecolor{currentfill}{rgb}{0.121569,0.466667,0.705882}%
\pgfsetfillcolor{currentfill}%
\pgfsetlinewidth{1.003750pt}%
\definecolor{currentstroke}{rgb}{0.121569,0.466667,0.705882}%
\pgfsetstrokecolor{currentstroke}%
\pgfsetdash{}{0pt}%
\pgfpathmoveto{\pgfqpoint{2.738339in}{3.436833in}}%
\pgfpathcurveto{\pgfqpoint{2.749389in}{3.436833in}}{\pgfqpoint{2.759988in}{3.441223in}}{\pgfqpoint{2.767802in}{3.449036in}}%
\pgfpathcurveto{\pgfqpoint{2.775615in}{3.456850in}}{\pgfqpoint{2.780005in}{3.467449in}}{\pgfqpoint{2.780005in}{3.478499in}}%
\pgfpathcurveto{\pgfqpoint{2.780005in}{3.489549in}}{\pgfqpoint{2.775615in}{3.500148in}}{\pgfqpoint{2.767802in}{3.507962in}}%
\pgfpathcurveto{\pgfqpoint{2.759988in}{3.515776in}}{\pgfqpoint{2.749389in}{3.520166in}}{\pgfqpoint{2.738339in}{3.520166in}}%
\pgfpathcurveto{\pgfqpoint{2.727289in}{3.520166in}}{\pgfqpoint{2.716690in}{3.515776in}}{\pgfqpoint{2.708876in}{3.507962in}}%
\pgfpathcurveto{\pgfqpoint{2.701062in}{3.500148in}}{\pgfqpoint{2.696672in}{3.489549in}}{\pgfqpoint{2.696672in}{3.478499in}}%
\pgfpathcurveto{\pgfqpoint{2.696672in}{3.467449in}}{\pgfqpoint{2.701062in}{3.456850in}}{\pgfqpoint{2.708876in}{3.449036in}}%
\pgfpathcurveto{\pgfqpoint{2.716690in}{3.441223in}}{\pgfqpoint{2.727289in}{3.436833in}}{\pgfqpoint{2.738339in}{3.436833in}}%
\pgfpathclose%
\pgfusepath{stroke,fill}%
\end{pgfscope}%
\begin{pgfscope}%
\pgfpathrectangle{\pgfqpoint{0.600000in}{0.600000in}}{\pgfqpoint{3.900000in}{3.900000in}}%
\pgfusepath{clip}%
\pgfsetbuttcap%
\pgfsetroundjoin%
\definecolor{currentfill}{rgb}{0.121569,0.466667,0.705882}%
\pgfsetfillcolor{currentfill}%
\pgfsetlinewidth{1.003750pt}%
\definecolor{currentstroke}{rgb}{0.121569,0.466667,0.705882}%
\pgfsetstrokecolor{currentstroke}%
\pgfsetdash{}{0pt}%
\pgfpathmoveto{\pgfqpoint{2.287639in}{3.023506in}}%
\pgfpathcurveto{\pgfqpoint{2.298689in}{3.023506in}}{\pgfqpoint{2.309288in}{3.027896in}}{\pgfqpoint{2.317101in}{3.035710in}}%
\pgfpathcurveto{\pgfqpoint{2.324915in}{3.043523in}}{\pgfqpoint{2.329305in}{3.054122in}}{\pgfqpoint{2.329305in}{3.065172in}}%
\pgfpathcurveto{\pgfqpoint{2.329305in}{3.076223in}}{\pgfqpoint{2.324915in}{3.086822in}}{\pgfqpoint{2.317101in}{3.094635in}}%
\pgfpathcurveto{\pgfqpoint{2.309288in}{3.102449in}}{\pgfqpoint{2.298689in}{3.106839in}}{\pgfqpoint{2.287639in}{3.106839in}}%
\pgfpathcurveto{\pgfqpoint{2.276588in}{3.106839in}}{\pgfqpoint{2.265989in}{3.102449in}}{\pgfqpoint{2.258176in}{3.094635in}}%
\pgfpathcurveto{\pgfqpoint{2.250362in}{3.086822in}}{\pgfqpoint{2.245972in}{3.076223in}}{\pgfqpoint{2.245972in}{3.065172in}}%
\pgfpathcurveto{\pgfqpoint{2.245972in}{3.054122in}}{\pgfqpoint{2.250362in}{3.043523in}}{\pgfqpoint{2.258176in}{3.035710in}}%
\pgfpathcurveto{\pgfqpoint{2.265989in}{3.027896in}}{\pgfqpoint{2.276588in}{3.023506in}}{\pgfqpoint{2.287639in}{3.023506in}}%
\pgfpathclose%
\pgfusepath{stroke,fill}%
\end{pgfscope}%
\begin{pgfscope}%
\pgfpathrectangle{\pgfqpoint{0.600000in}{0.600000in}}{\pgfqpoint{3.900000in}{3.900000in}}%
\pgfusepath{clip}%
\pgfsetbuttcap%
\pgfsetroundjoin%
\definecolor{currentfill}{rgb}{0.121569,0.466667,0.705882}%
\pgfsetfillcolor{currentfill}%
\pgfsetlinewidth{1.003750pt}%
\definecolor{currentstroke}{rgb}{0.121569,0.466667,0.705882}%
\pgfsetstrokecolor{currentstroke}%
\pgfsetdash{}{0pt}%
\pgfpathmoveto{\pgfqpoint{2.662503in}{3.286119in}}%
\pgfpathcurveto{\pgfqpoint{2.673553in}{3.286119in}}{\pgfqpoint{2.684152in}{3.290509in}}{\pgfqpoint{2.691966in}{3.298323in}}%
\pgfpathcurveto{\pgfqpoint{2.699780in}{3.306136in}}{\pgfqpoint{2.704170in}{3.316735in}}{\pgfqpoint{2.704170in}{3.327785in}}%
\pgfpathcurveto{\pgfqpoint{2.704170in}{3.338836in}}{\pgfqpoint{2.699780in}{3.349435in}}{\pgfqpoint{2.691966in}{3.357248in}}%
\pgfpathcurveto{\pgfqpoint{2.684152in}{3.365062in}}{\pgfqpoint{2.673553in}{3.369452in}}{\pgfqpoint{2.662503in}{3.369452in}}%
\pgfpathcurveto{\pgfqpoint{2.651453in}{3.369452in}}{\pgfqpoint{2.640854in}{3.365062in}}{\pgfqpoint{2.633041in}{3.357248in}}%
\pgfpathcurveto{\pgfqpoint{2.625227in}{3.349435in}}{\pgfqpoint{2.620837in}{3.338836in}}{\pgfqpoint{2.620837in}{3.327785in}}%
\pgfpathcurveto{\pgfqpoint{2.620837in}{3.316735in}}{\pgfqpoint{2.625227in}{3.306136in}}{\pgfqpoint{2.633041in}{3.298323in}}%
\pgfpathcurveto{\pgfqpoint{2.640854in}{3.290509in}}{\pgfqpoint{2.651453in}{3.286119in}}{\pgfqpoint{2.662503in}{3.286119in}}%
\pgfpathclose%
\pgfusepath{stroke,fill}%
\end{pgfscope}%
\begin{pgfscope}%
\pgfpathrectangle{\pgfqpoint{0.600000in}{0.600000in}}{\pgfqpoint{3.900000in}{3.900000in}}%
\pgfusepath{clip}%
\pgfsetbuttcap%
\pgfsetroundjoin%
\definecolor{currentfill}{rgb}{0.121569,0.466667,0.705882}%
\pgfsetfillcolor{currentfill}%
\pgfsetlinewidth{1.003750pt}%
\definecolor{currentstroke}{rgb}{0.121569,0.466667,0.705882}%
\pgfsetstrokecolor{currentstroke}%
\pgfsetdash{}{0pt}%
\pgfpathmoveto{\pgfqpoint{3.313454in}{2.386059in}}%
\pgfpathcurveto{\pgfqpoint{3.324504in}{2.386059in}}{\pgfqpoint{3.335103in}{2.390449in}}{\pgfqpoint{3.342916in}{2.398263in}}%
\pgfpathcurveto{\pgfqpoint{3.350730in}{2.406077in}}{\pgfqpoint{3.355120in}{2.416676in}}{\pgfqpoint{3.355120in}{2.427726in}}%
\pgfpathcurveto{\pgfqpoint{3.355120in}{2.438776in}}{\pgfqpoint{3.350730in}{2.449375in}}{\pgfqpoint{3.342916in}{2.457189in}}%
\pgfpathcurveto{\pgfqpoint{3.335103in}{2.465002in}}{\pgfqpoint{3.324504in}{2.469393in}}{\pgfqpoint{3.313454in}{2.469393in}}%
\pgfpathcurveto{\pgfqpoint{3.302403in}{2.469393in}}{\pgfqpoint{3.291804in}{2.465002in}}{\pgfqpoint{3.283991in}{2.457189in}}%
\pgfpathcurveto{\pgfqpoint{3.276177in}{2.449375in}}{\pgfqpoint{3.271787in}{2.438776in}}{\pgfqpoint{3.271787in}{2.427726in}}%
\pgfpathcurveto{\pgfqpoint{3.271787in}{2.416676in}}{\pgfqpoint{3.276177in}{2.406077in}}{\pgfqpoint{3.283991in}{2.398263in}}%
\pgfpathcurveto{\pgfqpoint{3.291804in}{2.390449in}}{\pgfqpoint{3.302403in}{2.386059in}}{\pgfqpoint{3.313454in}{2.386059in}}%
\pgfpathclose%
\pgfusepath{stroke,fill}%
\end{pgfscope}%
\begin{pgfscope}%
\pgfpathrectangle{\pgfqpoint{0.600000in}{0.600000in}}{\pgfqpoint{3.900000in}{3.900000in}}%
\pgfusepath{clip}%
\pgfsetbuttcap%
\pgfsetroundjoin%
\definecolor{currentfill}{rgb}{0.121569,0.466667,0.705882}%
\pgfsetfillcolor{currentfill}%
\pgfsetlinewidth{1.003750pt}%
\definecolor{currentstroke}{rgb}{0.121569,0.466667,0.705882}%
\pgfsetstrokecolor{currentstroke}%
\pgfsetdash{}{0pt}%
\pgfpathmoveto{\pgfqpoint{2.406919in}{2.743590in}}%
\pgfpathcurveto{\pgfqpoint{2.417970in}{2.743590in}}{\pgfqpoint{2.428569in}{2.747980in}}{\pgfqpoint{2.436382in}{2.755794in}}%
\pgfpathcurveto{\pgfqpoint{2.444196in}{2.763607in}}{\pgfqpoint{2.448586in}{2.774207in}}{\pgfqpoint{2.448586in}{2.785257in}}%
\pgfpathcurveto{\pgfqpoint{2.448586in}{2.796307in}}{\pgfqpoint{2.444196in}{2.806906in}}{\pgfqpoint{2.436382in}{2.814719in}}%
\pgfpathcurveto{\pgfqpoint{2.428569in}{2.822533in}}{\pgfqpoint{2.417970in}{2.826923in}}{\pgfqpoint{2.406919in}{2.826923in}}%
\pgfpathcurveto{\pgfqpoint{2.395869in}{2.826923in}}{\pgfqpoint{2.385270in}{2.822533in}}{\pgfqpoint{2.377457in}{2.814719in}}%
\pgfpathcurveto{\pgfqpoint{2.369643in}{2.806906in}}{\pgfqpoint{2.365253in}{2.796307in}}{\pgfqpoint{2.365253in}{2.785257in}}%
\pgfpathcurveto{\pgfqpoint{2.365253in}{2.774207in}}{\pgfqpoint{2.369643in}{2.763607in}}{\pgfqpoint{2.377457in}{2.755794in}}%
\pgfpathcurveto{\pgfqpoint{2.385270in}{2.747980in}}{\pgfqpoint{2.395869in}{2.743590in}}{\pgfqpoint{2.406919in}{2.743590in}}%
\pgfpathclose%
\pgfusepath{stroke,fill}%
\end{pgfscope}%
\begin{pgfscope}%
\pgfpathrectangle{\pgfqpoint{0.600000in}{0.600000in}}{\pgfqpoint{3.900000in}{3.900000in}}%
\pgfusepath{clip}%
\pgfsetbuttcap%
\pgfsetroundjoin%
\definecolor{currentfill}{rgb}{0.121569,0.466667,0.705882}%
\pgfsetfillcolor{currentfill}%
\pgfsetlinewidth{1.003750pt}%
\definecolor{currentstroke}{rgb}{0.121569,0.466667,0.705882}%
\pgfsetstrokecolor{currentstroke}%
\pgfsetdash{}{0pt}%
\pgfpathmoveto{\pgfqpoint{2.539891in}{2.468647in}}%
\pgfpathcurveto{\pgfqpoint{2.550941in}{2.468647in}}{\pgfqpoint{2.561540in}{2.473037in}}{\pgfqpoint{2.569354in}{2.480851in}}%
\pgfpathcurveto{\pgfqpoint{2.577167in}{2.488664in}}{\pgfqpoint{2.581558in}{2.499263in}}{\pgfqpoint{2.581558in}{2.510313in}}%
\pgfpathcurveto{\pgfqpoint{2.581558in}{2.521363in}}{\pgfqpoint{2.577167in}{2.531962in}}{\pgfqpoint{2.569354in}{2.539776in}}%
\pgfpathcurveto{\pgfqpoint{2.561540in}{2.547590in}}{\pgfqpoint{2.550941in}{2.551980in}}{\pgfqpoint{2.539891in}{2.551980in}}%
\pgfpathcurveto{\pgfqpoint{2.528841in}{2.551980in}}{\pgfqpoint{2.518242in}{2.547590in}}{\pgfqpoint{2.510428in}{2.539776in}}%
\pgfpathcurveto{\pgfqpoint{2.502615in}{2.531962in}}{\pgfqpoint{2.498224in}{2.521363in}}{\pgfqpoint{2.498224in}{2.510313in}}%
\pgfpathcurveto{\pgfqpoint{2.498224in}{2.499263in}}{\pgfqpoint{2.502615in}{2.488664in}}{\pgfqpoint{2.510428in}{2.480851in}}%
\pgfpathcurveto{\pgfqpoint{2.518242in}{2.473037in}}{\pgfqpoint{2.528841in}{2.468647in}}{\pgfqpoint{2.539891in}{2.468647in}}%
\pgfpathclose%
\pgfusepath{stroke,fill}%
\end{pgfscope}%
\begin{pgfscope}%
\pgfpathrectangle{\pgfqpoint{0.600000in}{0.600000in}}{\pgfqpoint{3.900000in}{3.900000in}}%
\pgfusepath{clip}%
\pgfsetbuttcap%
\pgfsetroundjoin%
\definecolor{currentfill}{rgb}{0.121569,0.466667,0.705882}%
\pgfsetfillcolor{currentfill}%
\pgfsetlinewidth{1.003750pt}%
\definecolor{currentstroke}{rgb}{0.121569,0.466667,0.705882}%
\pgfsetstrokecolor{currentstroke}%
\pgfsetdash{}{0pt}%
\pgfpathmoveto{\pgfqpoint{3.264213in}{3.452470in}}%
\pgfpathcurveto{\pgfqpoint{3.275263in}{3.452470in}}{\pgfqpoint{3.285862in}{3.456860in}}{\pgfqpoint{3.293675in}{3.464674in}}%
\pgfpathcurveto{\pgfqpoint{3.301489in}{3.472488in}}{\pgfqpoint{3.305879in}{3.483087in}}{\pgfqpoint{3.305879in}{3.494137in}}%
\pgfpathcurveto{\pgfqpoint{3.305879in}{3.505187in}}{\pgfqpoint{3.301489in}{3.515786in}}{\pgfqpoint{3.293675in}{3.523599in}}%
\pgfpathcurveto{\pgfqpoint{3.285862in}{3.531413in}}{\pgfqpoint{3.275263in}{3.535803in}}{\pgfqpoint{3.264213in}{3.535803in}}%
\pgfpathcurveto{\pgfqpoint{3.253162in}{3.535803in}}{\pgfqpoint{3.242563in}{3.531413in}}{\pgfqpoint{3.234750in}{3.523599in}}%
\pgfpathcurveto{\pgfqpoint{3.226936in}{3.515786in}}{\pgfqpoint{3.222546in}{3.505187in}}{\pgfqpoint{3.222546in}{3.494137in}}%
\pgfpathcurveto{\pgfqpoint{3.222546in}{3.483087in}}{\pgfqpoint{3.226936in}{3.472488in}}{\pgfqpoint{3.234750in}{3.464674in}}%
\pgfpathcurveto{\pgfqpoint{3.242563in}{3.456860in}}{\pgfqpoint{3.253162in}{3.452470in}}{\pgfqpoint{3.264213in}{3.452470in}}%
\pgfpathclose%
\pgfusepath{stroke,fill}%
\end{pgfscope}%
\begin{pgfscope}%
\pgfpathrectangle{\pgfqpoint{0.600000in}{0.600000in}}{\pgfqpoint{3.900000in}{3.900000in}}%
\pgfusepath{clip}%
\pgfsetbuttcap%
\pgfsetroundjoin%
\definecolor{currentfill}{rgb}{0.121569,0.466667,0.705882}%
\pgfsetfillcolor{currentfill}%
\pgfsetlinewidth{1.003750pt}%
\definecolor{currentstroke}{rgb}{0.121569,0.466667,0.705882}%
\pgfsetstrokecolor{currentstroke}%
\pgfsetdash{}{0pt}%
\pgfpathmoveto{\pgfqpoint{2.181259in}{2.474339in}}%
\pgfpathcurveto{\pgfqpoint{2.192309in}{2.474339in}}{\pgfqpoint{2.202908in}{2.478729in}}{\pgfqpoint{2.210722in}{2.486542in}}%
\pgfpathcurveto{\pgfqpoint{2.218536in}{2.494356in}}{\pgfqpoint{2.222926in}{2.504955in}}{\pgfqpoint{2.222926in}{2.516005in}}%
\pgfpathcurveto{\pgfqpoint{2.222926in}{2.527055in}}{\pgfqpoint{2.218536in}{2.537654in}}{\pgfqpoint{2.210722in}{2.545468in}}%
\pgfpathcurveto{\pgfqpoint{2.202908in}{2.553282in}}{\pgfqpoint{2.192309in}{2.557672in}}{\pgfqpoint{2.181259in}{2.557672in}}%
\pgfpathcurveto{\pgfqpoint{2.170209in}{2.557672in}}{\pgfqpoint{2.159610in}{2.553282in}}{\pgfqpoint{2.151796in}{2.545468in}}%
\pgfpathcurveto{\pgfqpoint{2.143983in}{2.537654in}}{\pgfqpoint{2.139593in}{2.527055in}}{\pgfqpoint{2.139593in}{2.516005in}}%
\pgfpathcurveto{\pgfqpoint{2.139593in}{2.504955in}}{\pgfqpoint{2.143983in}{2.494356in}}{\pgfqpoint{2.151796in}{2.486542in}}%
\pgfpathcurveto{\pgfqpoint{2.159610in}{2.478729in}}{\pgfqpoint{2.170209in}{2.474339in}}{\pgfqpoint{2.181259in}{2.474339in}}%
\pgfpathclose%
\pgfusepath{stroke,fill}%
\end{pgfscope}%
\begin{pgfscope}%
\pgfpathrectangle{\pgfqpoint{0.600000in}{0.600000in}}{\pgfqpoint{3.900000in}{3.900000in}}%
\pgfusepath{clip}%
\pgfsetbuttcap%
\pgfsetroundjoin%
\definecolor{currentfill}{rgb}{0.121569,0.466667,0.705882}%
\pgfsetfillcolor{currentfill}%
\pgfsetlinewidth{1.003750pt}%
\definecolor{currentstroke}{rgb}{0.121569,0.466667,0.705882}%
\pgfsetstrokecolor{currentstroke}%
\pgfsetdash{}{0pt}%
\pgfpathmoveto{\pgfqpoint{3.281477in}{2.330788in}}%
\pgfpathcurveto{\pgfqpoint{3.292527in}{2.330788in}}{\pgfqpoint{3.303126in}{2.335179in}}{\pgfqpoint{3.310940in}{2.342992in}}%
\pgfpathcurveto{\pgfqpoint{3.318753in}{2.350806in}}{\pgfqpoint{3.323144in}{2.361405in}}{\pgfqpoint{3.323144in}{2.372455in}}%
\pgfpathcurveto{\pgfqpoint{3.323144in}{2.383505in}}{\pgfqpoint{3.318753in}{2.394104in}}{\pgfqpoint{3.310940in}{2.401918in}}%
\pgfpathcurveto{\pgfqpoint{3.303126in}{2.409731in}}{\pgfqpoint{3.292527in}{2.414122in}}{\pgfqpoint{3.281477in}{2.414122in}}%
\pgfpathcurveto{\pgfqpoint{3.270427in}{2.414122in}}{\pgfqpoint{3.259828in}{2.409731in}}{\pgfqpoint{3.252014in}{2.401918in}}%
\pgfpathcurveto{\pgfqpoint{3.244200in}{2.394104in}}{\pgfqpoint{3.239810in}{2.383505in}}{\pgfqpoint{3.239810in}{2.372455in}}%
\pgfpathcurveto{\pgfqpoint{3.239810in}{2.361405in}}{\pgfqpoint{3.244200in}{2.350806in}}{\pgfqpoint{3.252014in}{2.342992in}}%
\pgfpathcurveto{\pgfqpoint{3.259828in}{2.335179in}}{\pgfqpoint{3.270427in}{2.330788in}}{\pgfqpoint{3.281477in}{2.330788in}}%
\pgfpathclose%
\pgfusepath{stroke,fill}%
\end{pgfscope}%
\begin{pgfscope}%
\pgfpathrectangle{\pgfqpoint{0.600000in}{0.600000in}}{\pgfqpoint{3.900000in}{3.900000in}}%
\pgfusepath{clip}%
\pgfsetbuttcap%
\pgfsetroundjoin%
\definecolor{currentfill}{rgb}{0.121569,0.466667,0.705882}%
\pgfsetfillcolor{currentfill}%
\pgfsetlinewidth{1.003750pt}%
\definecolor{currentstroke}{rgb}{0.121569,0.466667,0.705882}%
\pgfsetstrokecolor{currentstroke}%
\pgfsetdash{}{0pt}%
\pgfpathmoveto{\pgfqpoint{2.327106in}{2.323448in}}%
\pgfpathcurveto{\pgfqpoint{2.338156in}{2.323448in}}{\pgfqpoint{2.348755in}{2.327838in}}{\pgfqpoint{2.356569in}{2.335651in}}%
\pgfpathcurveto{\pgfqpoint{2.364382in}{2.343465in}}{\pgfqpoint{2.368773in}{2.354064in}}{\pgfqpoint{2.368773in}{2.365114in}}%
\pgfpathcurveto{\pgfqpoint{2.368773in}{2.376164in}}{\pgfqpoint{2.364382in}{2.386763in}}{\pgfqpoint{2.356569in}{2.394577in}}%
\pgfpathcurveto{\pgfqpoint{2.348755in}{2.402391in}}{\pgfqpoint{2.338156in}{2.406781in}}{\pgfqpoint{2.327106in}{2.406781in}}%
\pgfpathcurveto{\pgfqpoint{2.316056in}{2.406781in}}{\pgfqpoint{2.305457in}{2.402391in}}{\pgfqpoint{2.297643in}{2.394577in}}%
\pgfpathcurveto{\pgfqpoint{2.289829in}{2.386763in}}{\pgfqpoint{2.285439in}{2.376164in}}{\pgfqpoint{2.285439in}{2.365114in}}%
\pgfpathcurveto{\pgfqpoint{2.285439in}{2.354064in}}{\pgfqpoint{2.289829in}{2.343465in}}{\pgfqpoint{2.297643in}{2.335651in}}%
\pgfpathcurveto{\pgfqpoint{2.305457in}{2.327838in}}{\pgfqpoint{2.316056in}{2.323448in}}{\pgfqpoint{2.327106in}{2.323448in}}%
\pgfpathclose%
\pgfusepath{stroke,fill}%
\end{pgfscope}%
\begin{pgfscope}%
\pgfpathrectangle{\pgfqpoint{0.600000in}{0.600000in}}{\pgfqpoint{3.900000in}{3.900000in}}%
\pgfusepath{clip}%
\pgfsetbuttcap%
\pgfsetroundjoin%
\definecolor{currentfill}{rgb}{0.121569,0.466667,0.705882}%
\pgfsetfillcolor{currentfill}%
\pgfsetlinewidth{1.003750pt}%
\definecolor{currentstroke}{rgb}{0.121569,0.466667,0.705882}%
\pgfsetstrokecolor{currentstroke}%
\pgfsetdash{}{0pt}%
\pgfpathmoveto{\pgfqpoint{2.731642in}{2.611655in}}%
\pgfpathcurveto{\pgfqpoint{2.742692in}{2.611655in}}{\pgfqpoint{2.753291in}{2.616045in}}{\pgfqpoint{2.761105in}{2.623859in}}%
\pgfpathcurveto{\pgfqpoint{2.768919in}{2.631673in}}{\pgfqpoint{2.773309in}{2.642272in}}{\pgfqpoint{2.773309in}{2.653322in}}%
\pgfpathcurveto{\pgfqpoint{2.773309in}{2.664372in}}{\pgfqpoint{2.768919in}{2.674971in}}{\pgfqpoint{2.761105in}{2.682785in}}%
\pgfpathcurveto{\pgfqpoint{2.753291in}{2.690598in}}{\pgfqpoint{2.742692in}{2.694988in}}{\pgfqpoint{2.731642in}{2.694988in}}%
\pgfpathcurveto{\pgfqpoint{2.720592in}{2.694988in}}{\pgfqpoint{2.709993in}{2.690598in}}{\pgfqpoint{2.702180in}{2.682785in}}%
\pgfpathcurveto{\pgfqpoint{2.694366in}{2.674971in}}{\pgfqpoint{2.689976in}{2.664372in}}{\pgfqpoint{2.689976in}{2.653322in}}%
\pgfpathcurveto{\pgfqpoint{2.689976in}{2.642272in}}{\pgfqpoint{2.694366in}{2.631673in}}{\pgfqpoint{2.702180in}{2.623859in}}%
\pgfpathcurveto{\pgfqpoint{2.709993in}{2.616045in}}{\pgfqpoint{2.720592in}{2.611655in}}{\pgfqpoint{2.731642in}{2.611655in}}%
\pgfpathclose%
\pgfusepath{stroke,fill}%
\end{pgfscope}%
\begin{pgfscope}%
\pgfpathrectangle{\pgfqpoint{0.600000in}{0.600000in}}{\pgfqpoint{3.900000in}{3.900000in}}%
\pgfusepath{clip}%
\pgfsetbuttcap%
\pgfsetroundjoin%
\definecolor{currentfill}{rgb}{0.121569,0.466667,0.705882}%
\pgfsetfillcolor{currentfill}%
\pgfsetlinewidth{1.003750pt}%
\definecolor{currentstroke}{rgb}{0.121569,0.466667,0.705882}%
\pgfsetstrokecolor{currentstroke}%
\pgfsetdash{}{0pt}%
\pgfpathmoveto{\pgfqpoint{2.030544in}{2.072886in}}%
\pgfpathcurveto{\pgfqpoint{2.041594in}{2.072886in}}{\pgfqpoint{2.052193in}{2.077276in}}{\pgfqpoint{2.060007in}{2.085089in}}%
\pgfpathcurveto{\pgfqpoint{2.067820in}{2.092903in}}{\pgfqpoint{2.072211in}{2.103502in}}{\pgfqpoint{2.072211in}{2.114552in}}%
\pgfpathcurveto{\pgfqpoint{2.072211in}{2.125602in}}{\pgfqpoint{2.067820in}{2.136201in}}{\pgfqpoint{2.060007in}{2.144015in}}%
\pgfpathcurveto{\pgfqpoint{2.052193in}{2.151829in}}{\pgfqpoint{2.041594in}{2.156219in}}{\pgfqpoint{2.030544in}{2.156219in}}%
\pgfpathcurveto{\pgfqpoint{2.019494in}{2.156219in}}{\pgfqpoint{2.008895in}{2.151829in}}{\pgfqpoint{2.001081in}{2.144015in}}%
\pgfpathcurveto{\pgfqpoint{1.993268in}{2.136201in}}{\pgfqpoint{1.988877in}{2.125602in}}{\pgfqpoint{1.988877in}{2.114552in}}%
\pgfpathcurveto{\pgfqpoint{1.988877in}{2.103502in}}{\pgfqpoint{1.993268in}{2.092903in}}{\pgfqpoint{2.001081in}{2.085089in}}%
\pgfpathcurveto{\pgfqpoint{2.008895in}{2.077276in}}{\pgfqpoint{2.019494in}{2.072886in}}{\pgfqpoint{2.030544in}{2.072886in}}%
\pgfpathclose%
\pgfusepath{stroke,fill}%
\end{pgfscope}%
\begin{pgfscope}%
\pgfpathrectangle{\pgfqpoint{0.600000in}{0.600000in}}{\pgfqpoint{3.900000in}{3.900000in}}%
\pgfusepath{clip}%
\pgfsetbuttcap%
\pgfsetroundjoin%
\definecolor{currentfill}{rgb}{0.121569,0.466667,0.705882}%
\pgfsetfillcolor{currentfill}%
\pgfsetlinewidth{1.003750pt}%
\definecolor{currentstroke}{rgb}{0.121569,0.466667,0.705882}%
\pgfsetstrokecolor{currentstroke}%
\pgfsetdash{}{0pt}%
\pgfpathmoveto{\pgfqpoint{2.737306in}{2.334276in}}%
\pgfpathcurveto{\pgfqpoint{2.748356in}{2.334276in}}{\pgfqpoint{2.758955in}{2.338666in}}{\pgfqpoint{2.766768in}{2.346480in}}%
\pgfpathcurveto{\pgfqpoint{2.774582in}{2.354293in}}{\pgfqpoint{2.778972in}{2.364892in}}{\pgfqpoint{2.778972in}{2.375943in}}%
\pgfpathcurveto{\pgfqpoint{2.778972in}{2.386993in}}{\pgfqpoint{2.774582in}{2.397592in}}{\pgfqpoint{2.766768in}{2.405405in}}%
\pgfpathcurveto{\pgfqpoint{2.758955in}{2.413219in}}{\pgfqpoint{2.748356in}{2.417609in}}{\pgfqpoint{2.737306in}{2.417609in}}%
\pgfpathcurveto{\pgfqpoint{2.726255in}{2.417609in}}{\pgfqpoint{2.715656in}{2.413219in}}{\pgfqpoint{2.707843in}{2.405405in}}%
\pgfpathcurveto{\pgfqpoint{2.700029in}{2.397592in}}{\pgfqpoint{2.695639in}{2.386993in}}{\pgfqpoint{2.695639in}{2.375943in}}%
\pgfpathcurveto{\pgfqpoint{2.695639in}{2.364892in}}{\pgfqpoint{2.700029in}{2.354293in}}{\pgfqpoint{2.707843in}{2.346480in}}%
\pgfpathcurveto{\pgfqpoint{2.715656in}{2.338666in}}{\pgfqpoint{2.726255in}{2.334276in}}{\pgfqpoint{2.737306in}{2.334276in}}%
\pgfpathclose%
\pgfusepath{stroke,fill}%
\end{pgfscope}%
\begin{pgfscope}%
\pgfpathrectangle{\pgfqpoint{0.600000in}{0.600000in}}{\pgfqpoint{3.900000in}{3.900000in}}%
\pgfusepath{clip}%
\pgfsetbuttcap%
\pgfsetroundjoin%
\definecolor{currentfill}{rgb}{0.121569,0.466667,0.705882}%
\pgfsetfillcolor{currentfill}%
\pgfsetlinewidth{1.003750pt}%
\definecolor{currentstroke}{rgb}{0.121569,0.466667,0.705882}%
\pgfsetstrokecolor{currentstroke}%
\pgfsetdash{}{0pt}%
\pgfpathmoveto{\pgfqpoint{3.155854in}{1.904272in}}%
\pgfpathcurveto{\pgfqpoint{3.166904in}{1.904272in}}{\pgfqpoint{3.177503in}{1.908662in}}{\pgfqpoint{3.185317in}{1.916476in}}%
\pgfpathcurveto{\pgfqpoint{3.193130in}{1.924289in}}{\pgfqpoint{3.197521in}{1.934888in}}{\pgfqpoint{3.197521in}{1.945939in}}%
\pgfpathcurveto{\pgfqpoint{3.197521in}{1.956989in}}{\pgfqpoint{3.193130in}{1.967588in}}{\pgfqpoint{3.185317in}{1.975401in}}%
\pgfpathcurveto{\pgfqpoint{3.177503in}{1.983215in}}{\pgfqpoint{3.166904in}{1.987605in}}{\pgfqpoint{3.155854in}{1.987605in}}%
\pgfpathcurveto{\pgfqpoint{3.144804in}{1.987605in}}{\pgfqpoint{3.134205in}{1.983215in}}{\pgfqpoint{3.126391in}{1.975401in}}%
\pgfpathcurveto{\pgfqpoint{3.118578in}{1.967588in}}{\pgfqpoint{3.114187in}{1.956989in}}{\pgfqpoint{3.114187in}{1.945939in}}%
\pgfpathcurveto{\pgfqpoint{3.114187in}{1.934888in}}{\pgfqpoint{3.118578in}{1.924289in}}{\pgfqpoint{3.126391in}{1.916476in}}%
\pgfpathcurveto{\pgfqpoint{3.134205in}{1.908662in}}{\pgfqpoint{3.144804in}{1.904272in}}{\pgfqpoint{3.155854in}{1.904272in}}%
\pgfpathclose%
\pgfusepath{stroke,fill}%
\end{pgfscope}%
\begin{pgfscope}%
\pgfpathrectangle{\pgfqpoint{0.600000in}{0.600000in}}{\pgfqpoint{3.900000in}{3.900000in}}%
\pgfusepath{clip}%
\pgfsetbuttcap%
\pgfsetroundjoin%
\definecolor{currentfill}{rgb}{0.121569,0.466667,0.705882}%
\pgfsetfillcolor{currentfill}%
\pgfsetlinewidth{1.003750pt}%
\definecolor{currentstroke}{rgb}{0.121569,0.466667,0.705882}%
\pgfsetstrokecolor{currentstroke}%
\pgfsetdash{}{0pt}%
\pgfpathmoveto{\pgfqpoint{2.285388in}{2.741055in}}%
\pgfpathcurveto{\pgfqpoint{2.296438in}{2.741055in}}{\pgfqpoint{2.307037in}{2.745445in}}{\pgfqpoint{2.314850in}{2.753259in}}%
\pgfpathcurveto{\pgfqpoint{2.322664in}{2.761073in}}{\pgfqpoint{2.327054in}{2.771672in}}{\pgfqpoint{2.327054in}{2.782722in}}%
\pgfpathcurveto{\pgfqpoint{2.327054in}{2.793772in}}{\pgfqpoint{2.322664in}{2.804371in}}{\pgfqpoint{2.314850in}{2.812185in}}%
\pgfpathcurveto{\pgfqpoint{2.307037in}{2.819998in}}{\pgfqpoint{2.296438in}{2.824389in}}{\pgfqpoint{2.285388in}{2.824389in}}%
\pgfpathcurveto{\pgfqpoint{2.274338in}{2.824389in}}{\pgfqpoint{2.263739in}{2.819998in}}{\pgfqpoint{2.255925in}{2.812185in}}%
\pgfpathcurveto{\pgfqpoint{2.248111in}{2.804371in}}{\pgfqpoint{2.243721in}{2.793772in}}{\pgfqpoint{2.243721in}{2.782722in}}%
\pgfpathcurveto{\pgfqpoint{2.243721in}{2.771672in}}{\pgfqpoint{2.248111in}{2.761073in}}{\pgfqpoint{2.255925in}{2.753259in}}%
\pgfpathcurveto{\pgfqpoint{2.263739in}{2.745445in}}{\pgfqpoint{2.274338in}{2.741055in}}{\pgfqpoint{2.285388in}{2.741055in}}%
\pgfpathclose%
\pgfusepath{stroke,fill}%
\end{pgfscope}%
\begin{pgfscope}%
\pgfpathrectangle{\pgfqpoint{0.600000in}{0.600000in}}{\pgfqpoint{3.900000in}{3.900000in}}%
\pgfusepath{clip}%
\pgfsetbuttcap%
\pgfsetroundjoin%
\definecolor{currentfill}{rgb}{0.121569,0.466667,0.705882}%
\pgfsetfillcolor{currentfill}%
\pgfsetlinewidth{1.003750pt}%
\definecolor{currentstroke}{rgb}{0.121569,0.466667,0.705882}%
\pgfsetstrokecolor{currentstroke}%
\pgfsetdash{}{0pt}%
\pgfpathmoveto{\pgfqpoint{3.143508in}{3.115451in}}%
\pgfpathcurveto{\pgfqpoint{3.154558in}{3.115451in}}{\pgfqpoint{3.165157in}{3.119842in}}{\pgfqpoint{3.172971in}{3.127655in}}%
\pgfpathcurveto{\pgfqpoint{3.180784in}{3.135469in}}{\pgfqpoint{3.185174in}{3.146068in}}{\pgfqpoint{3.185174in}{3.157118in}}%
\pgfpathcurveto{\pgfqpoint{3.185174in}{3.168168in}}{\pgfqpoint{3.180784in}{3.178767in}}{\pgfqpoint{3.172971in}{3.186581in}}%
\pgfpathcurveto{\pgfqpoint{3.165157in}{3.194395in}}{\pgfqpoint{3.154558in}{3.198785in}}{\pgfqpoint{3.143508in}{3.198785in}}%
\pgfpathcurveto{\pgfqpoint{3.132458in}{3.198785in}}{\pgfqpoint{3.121859in}{3.194395in}}{\pgfqpoint{3.114045in}{3.186581in}}%
\pgfpathcurveto{\pgfqpoint{3.106231in}{3.178767in}}{\pgfqpoint{3.101841in}{3.168168in}}{\pgfqpoint{3.101841in}{3.157118in}}%
\pgfpathcurveto{\pgfqpoint{3.101841in}{3.146068in}}{\pgfqpoint{3.106231in}{3.135469in}}{\pgfqpoint{3.114045in}{3.127655in}}%
\pgfpathcurveto{\pgfqpoint{3.121859in}{3.119842in}}{\pgfqpoint{3.132458in}{3.115451in}}{\pgfqpoint{3.143508in}{3.115451in}}%
\pgfpathclose%
\pgfusepath{stroke,fill}%
\end{pgfscope}%
\begin{pgfscope}%
\pgfpathrectangle{\pgfqpoint{0.600000in}{0.600000in}}{\pgfqpoint{3.900000in}{3.900000in}}%
\pgfusepath{clip}%
\pgfsetbuttcap%
\pgfsetroundjoin%
\definecolor{currentfill}{rgb}{0.121569,0.466667,0.705882}%
\pgfsetfillcolor{currentfill}%
\pgfsetlinewidth{1.003750pt}%
\definecolor{currentstroke}{rgb}{0.121569,0.466667,0.705882}%
\pgfsetstrokecolor{currentstroke}%
\pgfsetdash{}{0pt}%
\pgfpathmoveto{\pgfqpoint{2.164017in}{2.094099in}}%
\pgfpathcurveto{\pgfqpoint{2.175067in}{2.094099in}}{\pgfqpoint{2.185666in}{2.098489in}}{\pgfqpoint{2.193480in}{2.106303in}}%
\pgfpathcurveto{\pgfqpoint{2.201293in}{2.114116in}}{\pgfqpoint{2.205683in}{2.124715in}}{\pgfqpoint{2.205683in}{2.135765in}}%
\pgfpathcurveto{\pgfqpoint{2.205683in}{2.146816in}}{\pgfqpoint{2.201293in}{2.157415in}}{\pgfqpoint{2.193480in}{2.165228in}}%
\pgfpathcurveto{\pgfqpoint{2.185666in}{2.173042in}}{\pgfqpoint{2.175067in}{2.177432in}}{\pgfqpoint{2.164017in}{2.177432in}}%
\pgfpathcurveto{\pgfqpoint{2.152967in}{2.177432in}}{\pgfqpoint{2.142368in}{2.173042in}}{\pgfqpoint{2.134554in}{2.165228in}}%
\pgfpathcurveto{\pgfqpoint{2.126740in}{2.157415in}}{\pgfqpoint{2.122350in}{2.146816in}}{\pgfqpoint{2.122350in}{2.135765in}}%
\pgfpathcurveto{\pgfqpoint{2.122350in}{2.124715in}}{\pgfqpoint{2.126740in}{2.114116in}}{\pgfqpoint{2.134554in}{2.106303in}}%
\pgfpathcurveto{\pgfqpoint{2.142368in}{2.098489in}}{\pgfqpoint{2.152967in}{2.094099in}}{\pgfqpoint{2.164017in}{2.094099in}}%
\pgfpathclose%
\pgfusepath{stroke,fill}%
\end{pgfscope}%
\begin{pgfscope}%
\pgfpathrectangle{\pgfqpoint{0.600000in}{0.600000in}}{\pgfqpoint{3.900000in}{3.900000in}}%
\pgfusepath{clip}%
\pgfsetbuttcap%
\pgfsetroundjoin%
\definecolor{currentfill}{rgb}{0.121569,0.466667,0.705882}%
\pgfsetfillcolor{currentfill}%
\pgfsetlinewidth{1.003750pt}%
\definecolor{currentstroke}{rgb}{0.121569,0.466667,0.705882}%
\pgfsetstrokecolor{currentstroke}%
\pgfsetdash{}{0pt}%
\pgfpathmoveto{\pgfqpoint{3.119266in}{1.916314in}}%
\pgfpathcurveto{\pgfqpoint{3.130316in}{1.916314in}}{\pgfqpoint{3.140915in}{1.920705in}}{\pgfqpoint{3.148728in}{1.928518in}}%
\pgfpathcurveto{\pgfqpoint{3.156542in}{1.936332in}}{\pgfqpoint{3.160932in}{1.946931in}}{\pgfqpoint{3.160932in}{1.957981in}}%
\pgfpathcurveto{\pgfqpoint{3.160932in}{1.969031in}}{\pgfqpoint{3.156542in}{1.979630in}}{\pgfqpoint{3.148728in}{1.987444in}}%
\pgfpathcurveto{\pgfqpoint{3.140915in}{1.995257in}}{\pgfqpoint{3.130316in}{1.999648in}}{\pgfqpoint{3.119266in}{1.999648in}}%
\pgfpathcurveto{\pgfqpoint{3.108215in}{1.999648in}}{\pgfqpoint{3.097616in}{1.995257in}}{\pgfqpoint{3.089803in}{1.987444in}}%
\pgfpathcurveto{\pgfqpoint{3.081989in}{1.979630in}}{\pgfqpoint{3.077599in}{1.969031in}}{\pgfqpoint{3.077599in}{1.957981in}}%
\pgfpathcurveto{\pgfqpoint{3.077599in}{1.946931in}}{\pgfqpoint{3.081989in}{1.936332in}}{\pgfqpoint{3.089803in}{1.928518in}}%
\pgfpathcurveto{\pgfqpoint{3.097616in}{1.920705in}}{\pgfqpoint{3.108215in}{1.916314in}}{\pgfqpoint{3.119266in}{1.916314in}}%
\pgfpathclose%
\pgfusepath{stroke,fill}%
\end{pgfscope}%
\begin{pgfscope}%
\pgfpathrectangle{\pgfqpoint{0.600000in}{0.600000in}}{\pgfqpoint{3.900000in}{3.900000in}}%
\pgfusepath{clip}%
\pgfsetbuttcap%
\pgfsetroundjoin%
\definecolor{currentfill}{rgb}{0.121569,0.466667,0.705882}%
\pgfsetfillcolor{currentfill}%
\pgfsetlinewidth{1.003750pt}%
\definecolor{currentstroke}{rgb}{0.121569,0.466667,0.705882}%
\pgfsetstrokecolor{currentstroke}%
\pgfsetdash{}{0pt}%
\pgfpathmoveto{\pgfqpoint{2.348910in}{3.027258in}}%
\pgfpathcurveto{\pgfqpoint{2.359960in}{3.027258in}}{\pgfqpoint{2.370559in}{3.031648in}}{\pgfqpoint{2.378373in}{3.039462in}}%
\pgfpathcurveto{\pgfqpoint{2.386187in}{3.047276in}}{\pgfqpoint{2.390577in}{3.057875in}}{\pgfqpoint{2.390577in}{3.068925in}}%
\pgfpathcurveto{\pgfqpoint{2.390577in}{3.079975in}}{\pgfqpoint{2.386187in}{3.090574in}}{\pgfqpoint{2.378373in}{3.098388in}}%
\pgfpathcurveto{\pgfqpoint{2.370559in}{3.106201in}}{\pgfqpoint{2.359960in}{3.110591in}}{\pgfqpoint{2.348910in}{3.110591in}}%
\pgfpathcurveto{\pgfqpoint{2.337860in}{3.110591in}}{\pgfqpoint{2.327261in}{3.106201in}}{\pgfqpoint{2.319447in}{3.098388in}}%
\pgfpathcurveto{\pgfqpoint{2.311634in}{3.090574in}}{\pgfqpoint{2.307243in}{3.079975in}}{\pgfqpoint{2.307243in}{3.068925in}}%
\pgfpathcurveto{\pgfqpoint{2.307243in}{3.057875in}}{\pgfqpoint{2.311634in}{3.047276in}}{\pgfqpoint{2.319447in}{3.039462in}}%
\pgfpathcurveto{\pgfqpoint{2.327261in}{3.031648in}}{\pgfqpoint{2.337860in}{3.027258in}}{\pgfqpoint{2.348910in}{3.027258in}}%
\pgfpathclose%
\pgfusepath{stroke,fill}%
\end{pgfscope}%
\begin{pgfscope}%
\pgfpathrectangle{\pgfqpoint{0.600000in}{0.600000in}}{\pgfqpoint{3.900000in}{3.900000in}}%
\pgfusepath{clip}%
\pgfsetbuttcap%
\pgfsetroundjoin%
\definecolor{currentfill}{rgb}{0.121569,0.466667,0.705882}%
\pgfsetfillcolor{currentfill}%
\pgfsetlinewidth{1.003750pt}%
\definecolor{currentstroke}{rgb}{0.121569,0.466667,0.705882}%
\pgfsetstrokecolor{currentstroke}%
\pgfsetdash{}{0pt}%
\pgfpathmoveto{\pgfqpoint{2.837656in}{2.710316in}}%
\pgfpathcurveto{\pgfqpoint{2.848706in}{2.710316in}}{\pgfqpoint{2.859305in}{2.714707in}}{\pgfqpoint{2.867119in}{2.722520in}}%
\pgfpathcurveto{\pgfqpoint{2.874933in}{2.730334in}}{\pgfqpoint{2.879323in}{2.740933in}}{\pgfqpoint{2.879323in}{2.751983in}}%
\pgfpathcurveto{\pgfqpoint{2.879323in}{2.763033in}}{\pgfqpoint{2.874933in}{2.773632in}}{\pgfqpoint{2.867119in}{2.781446in}}%
\pgfpathcurveto{\pgfqpoint{2.859305in}{2.789259in}}{\pgfqpoint{2.848706in}{2.793650in}}{\pgfqpoint{2.837656in}{2.793650in}}%
\pgfpathcurveto{\pgfqpoint{2.826606in}{2.793650in}}{\pgfqpoint{2.816007in}{2.789259in}}{\pgfqpoint{2.808193in}{2.781446in}}%
\pgfpathcurveto{\pgfqpoint{2.800380in}{2.773632in}}{\pgfqpoint{2.795989in}{2.763033in}}{\pgfqpoint{2.795989in}{2.751983in}}%
\pgfpathcurveto{\pgfqpoint{2.795989in}{2.740933in}}{\pgfqpoint{2.800380in}{2.730334in}}{\pgfqpoint{2.808193in}{2.722520in}}%
\pgfpathcurveto{\pgfqpoint{2.816007in}{2.714707in}}{\pgfqpoint{2.826606in}{2.710316in}}{\pgfqpoint{2.837656in}{2.710316in}}%
\pgfpathclose%
\pgfusepath{stroke,fill}%
\end{pgfscope}%
\begin{pgfscope}%
\pgfpathrectangle{\pgfqpoint{0.600000in}{0.600000in}}{\pgfqpoint{3.900000in}{3.900000in}}%
\pgfusepath{clip}%
\pgfsetbuttcap%
\pgfsetroundjoin%
\definecolor{currentfill}{rgb}{0.121569,0.466667,0.705882}%
\pgfsetfillcolor{currentfill}%
\pgfsetlinewidth{1.003750pt}%
\definecolor{currentstroke}{rgb}{0.121569,0.466667,0.705882}%
\pgfsetstrokecolor{currentstroke}%
\pgfsetdash{}{0pt}%
\pgfpathmoveto{\pgfqpoint{2.343777in}{2.358805in}}%
\pgfpathcurveto{\pgfqpoint{2.354827in}{2.358805in}}{\pgfqpoint{2.365426in}{2.363195in}}{\pgfqpoint{2.373239in}{2.371009in}}%
\pgfpathcurveto{\pgfqpoint{2.381053in}{2.378822in}}{\pgfqpoint{2.385443in}{2.389421in}}{\pgfqpoint{2.385443in}{2.400471in}}%
\pgfpathcurveto{\pgfqpoint{2.385443in}{2.411522in}}{\pgfqpoint{2.381053in}{2.422121in}}{\pgfqpoint{2.373239in}{2.429934in}}%
\pgfpathcurveto{\pgfqpoint{2.365426in}{2.437748in}}{\pgfqpoint{2.354827in}{2.442138in}}{\pgfqpoint{2.343777in}{2.442138in}}%
\pgfpathcurveto{\pgfqpoint{2.332727in}{2.442138in}}{\pgfqpoint{2.322128in}{2.437748in}}{\pgfqpoint{2.314314in}{2.429934in}}%
\pgfpathcurveto{\pgfqpoint{2.306500in}{2.422121in}}{\pgfqpoint{2.302110in}{2.411522in}}{\pgfqpoint{2.302110in}{2.400471in}}%
\pgfpathcurveto{\pgfqpoint{2.302110in}{2.389421in}}{\pgfqpoint{2.306500in}{2.378822in}}{\pgfqpoint{2.314314in}{2.371009in}}%
\pgfpathcurveto{\pgfqpoint{2.322128in}{2.363195in}}{\pgfqpoint{2.332727in}{2.358805in}}{\pgfqpoint{2.343777in}{2.358805in}}%
\pgfpathclose%
\pgfusepath{stroke,fill}%
\end{pgfscope}%
\begin{pgfscope}%
\pgfpathrectangle{\pgfqpoint{0.600000in}{0.600000in}}{\pgfqpoint{3.900000in}{3.900000in}}%
\pgfusepath{clip}%
\pgfsetbuttcap%
\pgfsetroundjoin%
\definecolor{currentfill}{rgb}{0.121569,0.466667,0.705882}%
\pgfsetfillcolor{currentfill}%
\pgfsetlinewidth{1.003750pt}%
\definecolor{currentstroke}{rgb}{0.121569,0.466667,0.705882}%
\pgfsetstrokecolor{currentstroke}%
\pgfsetdash{}{0pt}%
\pgfpathmoveto{\pgfqpoint{3.821703in}{2.758879in}}%
\pgfpathcurveto{\pgfqpoint{3.832753in}{2.758879in}}{\pgfqpoint{3.843352in}{2.763269in}}{\pgfqpoint{3.851165in}{2.771083in}}%
\pgfpathcurveto{\pgfqpoint{3.858979in}{2.778897in}}{\pgfqpoint{3.863369in}{2.789496in}}{\pgfqpoint{3.863369in}{2.800546in}}%
\pgfpathcurveto{\pgfqpoint{3.863369in}{2.811596in}}{\pgfqpoint{3.858979in}{2.822195in}}{\pgfqpoint{3.851165in}{2.830008in}}%
\pgfpathcurveto{\pgfqpoint{3.843352in}{2.837822in}}{\pgfqpoint{3.832753in}{2.842212in}}{\pgfqpoint{3.821703in}{2.842212in}}%
\pgfpathcurveto{\pgfqpoint{3.810652in}{2.842212in}}{\pgfqpoint{3.800053in}{2.837822in}}{\pgfqpoint{3.792240in}{2.830008in}}%
\pgfpathcurveto{\pgfqpoint{3.784426in}{2.822195in}}{\pgfqpoint{3.780036in}{2.811596in}}{\pgfqpoint{3.780036in}{2.800546in}}%
\pgfpathcurveto{\pgfqpoint{3.780036in}{2.789496in}}{\pgfqpoint{3.784426in}{2.778897in}}{\pgfqpoint{3.792240in}{2.771083in}}%
\pgfpathcurveto{\pgfqpoint{3.800053in}{2.763269in}}{\pgfqpoint{3.810652in}{2.758879in}}{\pgfqpoint{3.821703in}{2.758879in}}%
\pgfpathclose%
\pgfusepath{stroke,fill}%
\end{pgfscope}%
\begin{pgfscope}%
\pgfpathrectangle{\pgfqpoint{0.600000in}{0.600000in}}{\pgfqpoint{3.900000in}{3.900000in}}%
\pgfusepath{clip}%
\pgfsetbuttcap%
\pgfsetroundjoin%
\definecolor{currentfill}{rgb}{0.121569,0.466667,0.705882}%
\pgfsetfillcolor{currentfill}%
\pgfsetlinewidth{1.003750pt}%
\definecolor{currentstroke}{rgb}{0.121569,0.466667,0.705882}%
\pgfsetstrokecolor{currentstroke}%
\pgfsetdash{}{0pt}%
\pgfpathmoveto{\pgfqpoint{2.341749in}{1.851609in}}%
\pgfpathcurveto{\pgfqpoint{2.352799in}{1.851609in}}{\pgfqpoint{2.363398in}{1.855999in}}{\pgfqpoint{2.371212in}{1.863813in}}%
\pgfpathcurveto{\pgfqpoint{2.379025in}{1.871627in}}{\pgfqpoint{2.383416in}{1.882226in}}{\pgfqpoint{2.383416in}{1.893276in}}%
\pgfpathcurveto{\pgfqpoint{2.383416in}{1.904326in}}{\pgfqpoint{2.379025in}{1.914925in}}{\pgfqpoint{2.371212in}{1.922739in}}%
\pgfpathcurveto{\pgfqpoint{2.363398in}{1.930552in}}{\pgfqpoint{2.352799in}{1.934943in}}{\pgfqpoint{2.341749in}{1.934943in}}%
\pgfpathcurveto{\pgfqpoint{2.330699in}{1.934943in}}{\pgfqpoint{2.320100in}{1.930552in}}{\pgfqpoint{2.312286in}{1.922739in}}%
\pgfpathcurveto{\pgfqpoint{2.304472in}{1.914925in}}{\pgfqpoint{2.300082in}{1.904326in}}{\pgfqpoint{2.300082in}{1.893276in}}%
\pgfpathcurveto{\pgfqpoint{2.300082in}{1.882226in}}{\pgfqpoint{2.304472in}{1.871627in}}{\pgfqpoint{2.312286in}{1.863813in}}%
\pgfpathcurveto{\pgfqpoint{2.320100in}{1.855999in}}{\pgfqpoint{2.330699in}{1.851609in}}{\pgfqpoint{2.341749in}{1.851609in}}%
\pgfpathclose%
\pgfusepath{stroke,fill}%
\end{pgfscope}%
\begin{pgfscope}%
\pgfpathrectangle{\pgfqpoint{0.600000in}{0.600000in}}{\pgfqpoint{3.900000in}{3.900000in}}%
\pgfusepath{clip}%
\pgfsetbuttcap%
\pgfsetroundjoin%
\definecolor{currentfill}{rgb}{0.121569,0.466667,0.705882}%
\pgfsetfillcolor{currentfill}%
\pgfsetlinewidth{1.003750pt}%
\definecolor{currentstroke}{rgb}{0.121569,0.466667,0.705882}%
\pgfsetstrokecolor{currentstroke}%
\pgfsetdash{}{0pt}%
\pgfpathmoveto{\pgfqpoint{3.601943in}{2.604358in}}%
\pgfpathcurveto{\pgfqpoint{3.612993in}{2.604358in}}{\pgfqpoint{3.623593in}{2.608748in}}{\pgfqpoint{3.631406in}{2.616562in}}%
\pgfpathcurveto{\pgfqpoint{3.639220in}{2.624375in}}{\pgfqpoint{3.643610in}{2.634975in}}{\pgfqpoint{3.643610in}{2.646025in}}%
\pgfpathcurveto{\pgfqpoint{3.643610in}{2.657075in}}{\pgfqpoint{3.639220in}{2.667674in}}{\pgfqpoint{3.631406in}{2.675487in}}%
\pgfpathcurveto{\pgfqpoint{3.623593in}{2.683301in}}{\pgfqpoint{3.612993in}{2.687691in}}{\pgfqpoint{3.601943in}{2.687691in}}%
\pgfpathcurveto{\pgfqpoint{3.590893in}{2.687691in}}{\pgfqpoint{3.580294in}{2.683301in}}{\pgfqpoint{3.572481in}{2.675487in}}%
\pgfpathcurveto{\pgfqpoint{3.564667in}{2.667674in}}{\pgfqpoint{3.560277in}{2.657075in}}{\pgfqpoint{3.560277in}{2.646025in}}%
\pgfpathcurveto{\pgfqpoint{3.560277in}{2.634975in}}{\pgfqpoint{3.564667in}{2.624375in}}{\pgfqpoint{3.572481in}{2.616562in}}%
\pgfpathcurveto{\pgfqpoint{3.580294in}{2.608748in}}{\pgfqpoint{3.590893in}{2.604358in}}{\pgfqpoint{3.601943in}{2.604358in}}%
\pgfpathclose%
\pgfusepath{stroke,fill}%
\end{pgfscope}%
\begin{pgfscope}%
\pgfpathrectangle{\pgfqpoint{0.600000in}{0.600000in}}{\pgfqpoint{3.900000in}{3.900000in}}%
\pgfusepath{clip}%
\pgfsetbuttcap%
\pgfsetroundjoin%
\definecolor{currentfill}{rgb}{0.121569,0.466667,0.705882}%
\pgfsetfillcolor{currentfill}%
\pgfsetlinewidth{1.003750pt}%
\definecolor{currentstroke}{rgb}{0.121569,0.466667,0.705882}%
\pgfsetstrokecolor{currentstroke}%
\pgfsetdash{}{0pt}%
\pgfpathmoveto{\pgfqpoint{3.318823in}{2.422786in}}%
\pgfpathcurveto{\pgfqpoint{3.329874in}{2.422786in}}{\pgfqpoint{3.340473in}{2.427176in}}{\pgfqpoint{3.348286in}{2.434990in}}%
\pgfpathcurveto{\pgfqpoint{3.356100in}{2.442803in}}{\pgfqpoint{3.360490in}{2.453402in}}{\pgfqpoint{3.360490in}{2.464453in}}%
\pgfpathcurveto{\pgfqpoint{3.360490in}{2.475503in}}{\pgfqpoint{3.356100in}{2.486102in}}{\pgfqpoint{3.348286in}{2.493915in}}%
\pgfpathcurveto{\pgfqpoint{3.340473in}{2.501729in}}{\pgfqpoint{3.329874in}{2.506119in}}{\pgfqpoint{3.318823in}{2.506119in}}%
\pgfpathcurveto{\pgfqpoint{3.307773in}{2.506119in}}{\pgfqpoint{3.297174in}{2.501729in}}{\pgfqpoint{3.289361in}{2.493915in}}%
\pgfpathcurveto{\pgfqpoint{3.281547in}{2.486102in}}{\pgfqpoint{3.277157in}{2.475503in}}{\pgfqpoint{3.277157in}{2.464453in}}%
\pgfpathcurveto{\pgfqpoint{3.277157in}{2.453402in}}{\pgfqpoint{3.281547in}{2.442803in}}{\pgfqpoint{3.289361in}{2.434990in}}%
\pgfpathcurveto{\pgfqpoint{3.297174in}{2.427176in}}{\pgfqpoint{3.307773in}{2.422786in}}{\pgfqpoint{3.318823in}{2.422786in}}%
\pgfpathclose%
\pgfusepath{stroke,fill}%
\end{pgfscope}%
\begin{pgfscope}%
\pgfpathrectangle{\pgfqpoint{0.600000in}{0.600000in}}{\pgfqpoint{3.900000in}{3.900000in}}%
\pgfusepath{clip}%
\pgfsetbuttcap%
\pgfsetroundjoin%
\definecolor{currentfill}{rgb}{0.121569,0.466667,0.705882}%
\pgfsetfillcolor{currentfill}%
\pgfsetlinewidth{1.003750pt}%
\definecolor{currentstroke}{rgb}{0.121569,0.466667,0.705882}%
\pgfsetstrokecolor{currentstroke}%
\pgfsetdash{}{0pt}%
\pgfpathmoveto{\pgfqpoint{2.372911in}{2.235155in}}%
\pgfpathcurveto{\pgfqpoint{2.383961in}{2.235155in}}{\pgfqpoint{2.394560in}{2.239545in}}{\pgfqpoint{2.402374in}{2.247359in}}%
\pgfpathcurveto{\pgfqpoint{2.410187in}{2.255172in}}{\pgfqpoint{2.414578in}{2.265771in}}{\pgfqpoint{2.414578in}{2.276821in}}%
\pgfpathcurveto{\pgfqpoint{2.414578in}{2.287871in}}{\pgfqpoint{2.410187in}{2.298470in}}{\pgfqpoint{2.402374in}{2.306284in}}%
\pgfpathcurveto{\pgfqpoint{2.394560in}{2.314098in}}{\pgfqpoint{2.383961in}{2.318488in}}{\pgfqpoint{2.372911in}{2.318488in}}%
\pgfpathcurveto{\pgfqpoint{2.361861in}{2.318488in}}{\pgfqpoint{2.351262in}{2.314098in}}{\pgfqpoint{2.343448in}{2.306284in}}%
\pgfpathcurveto{\pgfqpoint{2.335635in}{2.298470in}}{\pgfqpoint{2.331244in}{2.287871in}}{\pgfqpoint{2.331244in}{2.276821in}}%
\pgfpathcurveto{\pgfqpoint{2.331244in}{2.265771in}}{\pgfqpoint{2.335635in}{2.255172in}}{\pgfqpoint{2.343448in}{2.247359in}}%
\pgfpathcurveto{\pgfqpoint{2.351262in}{2.239545in}}{\pgfqpoint{2.361861in}{2.235155in}}{\pgfqpoint{2.372911in}{2.235155in}}%
\pgfpathclose%
\pgfusepath{stroke,fill}%
\end{pgfscope}%
\begin{pgfscope}%
\pgfpathrectangle{\pgfqpoint{0.600000in}{0.600000in}}{\pgfqpoint{3.900000in}{3.900000in}}%
\pgfusepath{clip}%
\pgfsetbuttcap%
\pgfsetroundjoin%
\definecolor{currentfill}{rgb}{0.121569,0.466667,0.705882}%
\pgfsetfillcolor{currentfill}%
\pgfsetlinewidth{1.003750pt}%
\definecolor{currentstroke}{rgb}{0.121569,0.466667,0.705882}%
\pgfsetstrokecolor{currentstroke}%
\pgfsetdash{}{0pt}%
\pgfpathmoveto{\pgfqpoint{3.339684in}{3.196580in}}%
\pgfpathcurveto{\pgfqpoint{3.350734in}{3.196580in}}{\pgfqpoint{3.361333in}{3.200971in}}{\pgfqpoint{3.369146in}{3.208784in}}%
\pgfpathcurveto{\pgfqpoint{3.376960in}{3.216598in}}{\pgfqpoint{3.381350in}{3.227197in}}{\pgfqpoint{3.381350in}{3.238247in}}%
\pgfpathcurveto{\pgfqpoint{3.381350in}{3.249297in}}{\pgfqpoint{3.376960in}{3.259896in}}{\pgfqpoint{3.369146in}{3.267710in}}%
\pgfpathcurveto{\pgfqpoint{3.361333in}{3.275523in}}{\pgfqpoint{3.350734in}{3.279914in}}{\pgfqpoint{3.339684in}{3.279914in}}%
\pgfpathcurveto{\pgfqpoint{3.328633in}{3.279914in}}{\pgfqpoint{3.318034in}{3.275523in}}{\pgfqpoint{3.310221in}{3.267710in}}%
\pgfpathcurveto{\pgfqpoint{3.302407in}{3.259896in}}{\pgfqpoint{3.298017in}{3.249297in}}{\pgfqpoint{3.298017in}{3.238247in}}%
\pgfpathcurveto{\pgfqpoint{3.298017in}{3.227197in}}{\pgfqpoint{3.302407in}{3.216598in}}{\pgfqpoint{3.310221in}{3.208784in}}%
\pgfpathcurveto{\pgfqpoint{3.318034in}{3.200971in}}{\pgfqpoint{3.328633in}{3.196580in}}{\pgfqpoint{3.339684in}{3.196580in}}%
\pgfpathclose%
\pgfusepath{stroke,fill}%
\end{pgfscope}%
\begin{pgfscope}%
\pgfpathrectangle{\pgfqpoint{0.600000in}{0.600000in}}{\pgfqpoint{3.900000in}{3.900000in}}%
\pgfusepath{clip}%
\pgfsetbuttcap%
\pgfsetroundjoin%
\definecolor{currentfill}{rgb}{0.121569,0.466667,0.705882}%
\pgfsetfillcolor{currentfill}%
\pgfsetlinewidth{1.003750pt}%
\definecolor{currentstroke}{rgb}{0.121569,0.466667,0.705882}%
\pgfsetstrokecolor{currentstroke}%
\pgfsetdash{}{0pt}%
\pgfpathmoveto{\pgfqpoint{2.598868in}{1.974572in}}%
\pgfpathcurveto{\pgfqpoint{2.609918in}{1.974572in}}{\pgfqpoint{2.620517in}{1.978962in}}{\pgfqpoint{2.628331in}{1.986775in}}%
\pgfpathcurveto{\pgfqpoint{2.636144in}{1.994589in}}{\pgfqpoint{2.640535in}{2.005188in}}{\pgfqpoint{2.640535in}{2.016238in}}%
\pgfpathcurveto{\pgfqpoint{2.640535in}{2.027288in}}{\pgfqpoint{2.636144in}{2.037887in}}{\pgfqpoint{2.628331in}{2.045701in}}%
\pgfpathcurveto{\pgfqpoint{2.620517in}{2.053515in}}{\pgfqpoint{2.609918in}{2.057905in}}{\pgfqpoint{2.598868in}{2.057905in}}%
\pgfpathcurveto{\pgfqpoint{2.587818in}{2.057905in}}{\pgfqpoint{2.577219in}{2.053515in}}{\pgfqpoint{2.569405in}{2.045701in}}%
\pgfpathcurveto{\pgfqpoint{2.561592in}{2.037887in}}{\pgfqpoint{2.557201in}{2.027288in}}{\pgfqpoint{2.557201in}{2.016238in}}%
\pgfpathcurveto{\pgfqpoint{2.557201in}{2.005188in}}{\pgfqpoint{2.561592in}{1.994589in}}{\pgfqpoint{2.569405in}{1.986775in}}%
\pgfpathcurveto{\pgfqpoint{2.577219in}{1.978962in}}{\pgfqpoint{2.587818in}{1.974572in}}{\pgfqpoint{2.598868in}{1.974572in}}%
\pgfpathclose%
\pgfusepath{stroke,fill}%
\end{pgfscope}%
\begin{pgfscope}%
\pgfpathrectangle{\pgfqpoint{0.600000in}{0.600000in}}{\pgfqpoint{3.900000in}{3.900000in}}%
\pgfusepath{clip}%
\pgfsetbuttcap%
\pgfsetroundjoin%
\definecolor{currentfill}{rgb}{0.121569,0.466667,0.705882}%
\pgfsetfillcolor{currentfill}%
\pgfsetlinewidth{1.003750pt}%
\definecolor{currentstroke}{rgb}{0.121569,0.466667,0.705882}%
\pgfsetstrokecolor{currentstroke}%
\pgfsetdash{}{0pt}%
\pgfpathmoveto{\pgfqpoint{2.201616in}{1.799724in}}%
\pgfpathcurveto{\pgfqpoint{2.212666in}{1.799724in}}{\pgfqpoint{2.223266in}{1.804114in}}{\pgfqpoint{2.231079in}{1.811928in}}%
\pgfpathcurveto{\pgfqpoint{2.238893in}{1.819741in}}{\pgfqpoint{2.243283in}{1.830340in}}{\pgfqpoint{2.243283in}{1.841390in}}%
\pgfpathcurveto{\pgfqpoint{2.243283in}{1.852441in}}{\pgfqpoint{2.238893in}{1.863040in}}{\pgfqpoint{2.231079in}{1.870853in}}%
\pgfpathcurveto{\pgfqpoint{2.223266in}{1.878667in}}{\pgfqpoint{2.212666in}{1.883057in}}{\pgfqpoint{2.201616in}{1.883057in}}%
\pgfpathcurveto{\pgfqpoint{2.190566in}{1.883057in}}{\pgfqpoint{2.179967in}{1.878667in}}{\pgfqpoint{2.172154in}{1.870853in}}%
\pgfpathcurveto{\pgfqpoint{2.164340in}{1.863040in}}{\pgfqpoint{2.159950in}{1.852441in}}{\pgfqpoint{2.159950in}{1.841390in}}%
\pgfpathcurveto{\pgfqpoint{2.159950in}{1.830340in}}{\pgfqpoint{2.164340in}{1.819741in}}{\pgfqpoint{2.172154in}{1.811928in}}%
\pgfpathcurveto{\pgfqpoint{2.179967in}{1.804114in}}{\pgfqpoint{2.190566in}{1.799724in}}{\pgfqpoint{2.201616in}{1.799724in}}%
\pgfpathclose%
\pgfusepath{stroke,fill}%
\end{pgfscope}%
\begin{pgfscope}%
\pgfpathrectangle{\pgfqpoint{0.600000in}{0.600000in}}{\pgfqpoint{3.900000in}{3.900000in}}%
\pgfusepath{clip}%
\pgfsetbuttcap%
\pgfsetroundjoin%
\definecolor{currentfill}{rgb}{0.121569,0.466667,0.705882}%
\pgfsetfillcolor{currentfill}%
\pgfsetlinewidth{1.003750pt}%
\definecolor{currentstroke}{rgb}{0.121569,0.466667,0.705882}%
\pgfsetstrokecolor{currentstroke}%
\pgfsetdash{}{0pt}%
\pgfpathmoveto{\pgfqpoint{2.579801in}{2.376751in}}%
\pgfpathcurveto{\pgfqpoint{2.590851in}{2.376751in}}{\pgfqpoint{2.601450in}{2.381141in}}{\pgfqpoint{2.609264in}{2.388955in}}%
\pgfpathcurveto{\pgfqpoint{2.617077in}{2.396768in}}{\pgfqpoint{2.621468in}{2.407368in}}{\pgfqpoint{2.621468in}{2.418418in}}%
\pgfpathcurveto{\pgfqpoint{2.621468in}{2.429468in}}{\pgfqpoint{2.617077in}{2.440067in}}{\pgfqpoint{2.609264in}{2.447880in}}%
\pgfpathcurveto{\pgfqpoint{2.601450in}{2.455694in}}{\pgfqpoint{2.590851in}{2.460084in}}{\pgfqpoint{2.579801in}{2.460084in}}%
\pgfpathcurveto{\pgfqpoint{2.568751in}{2.460084in}}{\pgfqpoint{2.558152in}{2.455694in}}{\pgfqpoint{2.550338in}{2.447880in}}%
\pgfpathcurveto{\pgfqpoint{2.542525in}{2.440067in}}{\pgfqpoint{2.538134in}{2.429468in}}{\pgfqpoint{2.538134in}{2.418418in}}%
\pgfpathcurveto{\pgfqpoint{2.538134in}{2.407368in}}{\pgfqpoint{2.542525in}{2.396768in}}{\pgfqpoint{2.550338in}{2.388955in}}%
\pgfpathcurveto{\pgfqpoint{2.558152in}{2.381141in}}{\pgfqpoint{2.568751in}{2.376751in}}{\pgfqpoint{2.579801in}{2.376751in}}%
\pgfpathclose%
\pgfusepath{stroke,fill}%
\end{pgfscope}%
\begin{pgfscope}%
\pgfpathrectangle{\pgfqpoint{0.600000in}{0.600000in}}{\pgfqpoint{3.900000in}{3.900000in}}%
\pgfusepath{clip}%
\pgfsetbuttcap%
\pgfsetroundjoin%
\definecolor{currentfill}{rgb}{0.121569,0.466667,0.705882}%
\pgfsetfillcolor{currentfill}%
\pgfsetlinewidth{1.003750pt}%
\definecolor{currentstroke}{rgb}{0.121569,0.466667,0.705882}%
\pgfsetstrokecolor{currentstroke}%
\pgfsetdash{}{0pt}%
\pgfpathmoveto{\pgfqpoint{2.595135in}{1.634046in}}%
\pgfpathcurveto{\pgfqpoint{2.606185in}{1.634046in}}{\pgfqpoint{2.616784in}{1.638436in}}{\pgfqpoint{2.624597in}{1.646250in}}%
\pgfpathcurveto{\pgfqpoint{2.632411in}{1.654064in}}{\pgfqpoint{2.636801in}{1.664663in}}{\pgfqpoint{2.636801in}{1.675713in}}%
\pgfpathcurveto{\pgfqpoint{2.636801in}{1.686763in}}{\pgfqpoint{2.632411in}{1.697362in}}{\pgfqpoint{2.624597in}{1.705176in}}%
\pgfpathcurveto{\pgfqpoint{2.616784in}{1.712989in}}{\pgfqpoint{2.606185in}{1.717379in}}{\pgfqpoint{2.595135in}{1.717379in}}%
\pgfpathcurveto{\pgfqpoint{2.584084in}{1.717379in}}{\pgfqpoint{2.573485in}{1.712989in}}{\pgfqpoint{2.565672in}{1.705176in}}%
\pgfpathcurveto{\pgfqpoint{2.557858in}{1.697362in}}{\pgfqpoint{2.553468in}{1.686763in}}{\pgfqpoint{2.553468in}{1.675713in}}%
\pgfpathcurveto{\pgfqpoint{2.553468in}{1.664663in}}{\pgfqpoint{2.557858in}{1.654064in}}{\pgfqpoint{2.565672in}{1.646250in}}%
\pgfpathcurveto{\pgfqpoint{2.573485in}{1.638436in}}{\pgfqpoint{2.584084in}{1.634046in}}{\pgfqpoint{2.595135in}{1.634046in}}%
\pgfpathclose%
\pgfusepath{stroke,fill}%
\end{pgfscope}%
\begin{pgfscope}%
\pgfpathrectangle{\pgfqpoint{0.600000in}{0.600000in}}{\pgfqpoint{3.900000in}{3.900000in}}%
\pgfusepath{clip}%
\pgfsetbuttcap%
\pgfsetroundjoin%
\definecolor{currentfill}{rgb}{0.121569,0.466667,0.705882}%
\pgfsetfillcolor{currentfill}%
\pgfsetlinewidth{1.003750pt}%
\definecolor{currentstroke}{rgb}{0.121569,0.466667,0.705882}%
\pgfsetstrokecolor{currentstroke}%
\pgfsetdash{}{0pt}%
\pgfpathmoveto{\pgfqpoint{2.781500in}{1.477706in}}%
\pgfpathcurveto{\pgfqpoint{2.792551in}{1.477706in}}{\pgfqpoint{2.803150in}{1.482096in}}{\pgfqpoint{2.810963in}{1.489910in}}%
\pgfpathcurveto{\pgfqpoint{2.818777in}{1.497723in}}{\pgfqpoint{2.823167in}{1.508322in}}{\pgfqpoint{2.823167in}{1.519372in}}%
\pgfpathcurveto{\pgfqpoint{2.823167in}{1.530423in}}{\pgfqpoint{2.818777in}{1.541022in}}{\pgfqpoint{2.810963in}{1.548835in}}%
\pgfpathcurveto{\pgfqpoint{2.803150in}{1.556649in}}{\pgfqpoint{2.792551in}{1.561039in}}{\pgfqpoint{2.781500in}{1.561039in}}%
\pgfpathcurveto{\pgfqpoint{2.770450in}{1.561039in}}{\pgfqpoint{2.759851in}{1.556649in}}{\pgfqpoint{2.752038in}{1.548835in}}%
\pgfpathcurveto{\pgfqpoint{2.744224in}{1.541022in}}{\pgfqpoint{2.739834in}{1.530423in}}{\pgfqpoint{2.739834in}{1.519372in}}%
\pgfpathcurveto{\pgfqpoint{2.739834in}{1.508322in}}{\pgfqpoint{2.744224in}{1.497723in}}{\pgfqpoint{2.752038in}{1.489910in}}%
\pgfpathcurveto{\pgfqpoint{2.759851in}{1.482096in}}{\pgfqpoint{2.770450in}{1.477706in}}{\pgfqpoint{2.781500in}{1.477706in}}%
\pgfpathclose%
\pgfusepath{stroke,fill}%
\end{pgfscope}%
\begin{pgfscope}%
\pgfpathrectangle{\pgfqpoint{0.600000in}{0.600000in}}{\pgfqpoint{3.900000in}{3.900000in}}%
\pgfusepath{clip}%
\pgfsetbuttcap%
\pgfsetroundjoin%
\definecolor{currentfill}{rgb}{0.121569,0.466667,0.705882}%
\pgfsetfillcolor{currentfill}%
\pgfsetlinewidth{1.003750pt}%
\definecolor{currentstroke}{rgb}{0.121569,0.466667,0.705882}%
\pgfsetstrokecolor{currentstroke}%
\pgfsetdash{}{0pt}%
\pgfpathmoveto{\pgfqpoint{2.387844in}{2.086054in}}%
\pgfpathcurveto{\pgfqpoint{2.398894in}{2.086054in}}{\pgfqpoint{2.409493in}{2.090445in}}{\pgfqpoint{2.417307in}{2.098258in}}%
\pgfpathcurveto{\pgfqpoint{2.425121in}{2.106072in}}{\pgfqpoint{2.429511in}{2.116671in}}{\pgfqpoint{2.429511in}{2.127721in}}%
\pgfpathcurveto{\pgfqpoint{2.429511in}{2.138771in}}{\pgfqpoint{2.425121in}{2.149370in}}{\pgfqpoint{2.417307in}{2.157184in}}%
\pgfpathcurveto{\pgfqpoint{2.409493in}{2.164997in}}{\pgfqpoint{2.398894in}{2.169388in}}{\pgfqpoint{2.387844in}{2.169388in}}%
\pgfpathcurveto{\pgfqpoint{2.376794in}{2.169388in}}{\pgfqpoint{2.366195in}{2.164997in}}{\pgfqpoint{2.358382in}{2.157184in}}%
\pgfpathcurveto{\pgfqpoint{2.350568in}{2.149370in}}{\pgfqpoint{2.346178in}{2.138771in}}{\pgfqpoint{2.346178in}{2.127721in}}%
\pgfpathcurveto{\pgfqpoint{2.346178in}{2.116671in}}{\pgfqpoint{2.350568in}{2.106072in}}{\pgfqpoint{2.358382in}{2.098258in}}%
\pgfpathcurveto{\pgfqpoint{2.366195in}{2.090445in}}{\pgfqpoint{2.376794in}{2.086054in}}{\pgfqpoint{2.387844in}{2.086054in}}%
\pgfpathclose%
\pgfusepath{stroke,fill}%
\end{pgfscope}%
\begin{pgfscope}%
\pgfpathrectangle{\pgfqpoint{0.600000in}{0.600000in}}{\pgfqpoint{3.900000in}{3.900000in}}%
\pgfusepath{clip}%
\pgfsetbuttcap%
\pgfsetroundjoin%
\definecolor{currentfill}{rgb}{0.121569,0.466667,0.705882}%
\pgfsetfillcolor{currentfill}%
\pgfsetlinewidth{1.003750pt}%
\definecolor{currentstroke}{rgb}{0.121569,0.466667,0.705882}%
\pgfsetstrokecolor{currentstroke}%
\pgfsetdash{}{0pt}%
\pgfpathmoveto{\pgfqpoint{3.131908in}{2.705871in}}%
\pgfpathcurveto{\pgfqpoint{3.142958in}{2.705871in}}{\pgfqpoint{3.153557in}{2.710262in}}{\pgfqpoint{3.161371in}{2.718075in}}%
\pgfpathcurveto{\pgfqpoint{3.169184in}{2.725889in}}{\pgfqpoint{3.173574in}{2.736488in}}{\pgfqpoint{3.173574in}{2.747538in}}%
\pgfpathcurveto{\pgfqpoint{3.173574in}{2.758588in}}{\pgfqpoint{3.169184in}{2.769187in}}{\pgfqpoint{3.161371in}{2.777001in}}%
\pgfpathcurveto{\pgfqpoint{3.153557in}{2.784814in}}{\pgfqpoint{3.142958in}{2.789205in}}{\pgfqpoint{3.131908in}{2.789205in}}%
\pgfpathcurveto{\pgfqpoint{3.120858in}{2.789205in}}{\pgfqpoint{3.110259in}{2.784814in}}{\pgfqpoint{3.102445in}{2.777001in}}%
\pgfpathcurveto{\pgfqpoint{3.094631in}{2.769187in}}{\pgfqpoint{3.090241in}{2.758588in}}{\pgfqpoint{3.090241in}{2.747538in}}%
\pgfpathcurveto{\pgfqpoint{3.090241in}{2.736488in}}{\pgfqpoint{3.094631in}{2.725889in}}{\pgfqpoint{3.102445in}{2.718075in}}%
\pgfpathcurveto{\pgfqpoint{3.110259in}{2.710262in}}{\pgfqpoint{3.120858in}{2.705871in}}{\pgfqpoint{3.131908in}{2.705871in}}%
\pgfpathclose%
\pgfusepath{stroke,fill}%
\end{pgfscope}%
\begin{pgfscope}%
\pgfpathrectangle{\pgfqpoint{0.600000in}{0.600000in}}{\pgfqpoint{3.900000in}{3.900000in}}%
\pgfusepath{clip}%
\pgfsetbuttcap%
\pgfsetroundjoin%
\definecolor{currentfill}{rgb}{0.121569,0.466667,0.705882}%
\pgfsetfillcolor{currentfill}%
\pgfsetlinewidth{1.003750pt}%
\definecolor{currentstroke}{rgb}{0.121569,0.466667,0.705882}%
\pgfsetstrokecolor{currentstroke}%
\pgfsetdash{}{0pt}%
\pgfpathmoveto{\pgfqpoint{2.279361in}{2.468726in}}%
\pgfpathcurveto{\pgfqpoint{2.290411in}{2.468726in}}{\pgfqpoint{2.301010in}{2.473116in}}{\pgfqpoint{2.308823in}{2.480929in}}%
\pgfpathcurveto{\pgfqpoint{2.316637in}{2.488743in}}{\pgfqpoint{2.321027in}{2.499342in}}{\pgfqpoint{2.321027in}{2.510392in}}%
\pgfpathcurveto{\pgfqpoint{2.321027in}{2.521442in}}{\pgfqpoint{2.316637in}{2.532041in}}{\pgfqpoint{2.308823in}{2.539855in}}%
\pgfpathcurveto{\pgfqpoint{2.301010in}{2.547669in}}{\pgfqpoint{2.290411in}{2.552059in}}{\pgfqpoint{2.279361in}{2.552059in}}%
\pgfpathcurveto{\pgfqpoint{2.268311in}{2.552059in}}{\pgfqpoint{2.257712in}{2.547669in}}{\pgfqpoint{2.249898in}{2.539855in}}%
\pgfpathcurveto{\pgfqpoint{2.242084in}{2.532041in}}{\pgfqpoint{2.237694in}{2.521442in}}{\pgfqpoint{2.237694in}{2.510392in}}%
\pgfpathcurveto{\pgfqpoint{2.237694in}{2.499342in}}{\pgfqpoint{2.242084in}{2.488743in}}{\pgfqpoint{2.249898in}{2.480929in}}%
\pgfpathcurveto{\pgfqpoint{2.257712in}{2.473116in}}{\pgfqpoint{2.268311in}{2.468726in}}{\pgfqpoint{2.279361in}{2.468726in}}%
\pgfpathclose%
\pgfusepath{stroke,fill}%
\end{pgfscope}%
\begin{pgfscope}%
\pgfpathrectangle{\pgfqpoint{0.600000in}{0.600000in}}{\pgfqpoint{3.900000in}{3.900000in}}%
\pgfusepath{clip}%
\pgfsetbuttcap%
\pgfsetroundjoin%
\definecolor{currentfill}{rgb}{0.121569,0.466667,0.705882}%
\pgfsetfillcolor{currentfill}%
\pgfsetlinewidth{1.003750pt}%
\definecolor{currentstroke}{rgb}{0.121569,0.466667,0.705882}%
\pgfsetstrokecolor{currentstroke}%
\pgfsetdash{}{0pt}%
\pgfpathmoveto{\pgfqpoint{2.931550in}{2.926806in}}%
\pgfpathcurveto{\pgfqpoint{2.942600in}{2.926806in}}{\pgfqpoint{2.953199in}{2.931196in}}{\pgfqpoint{2.961013in}{2.939009in}}%
\pgfpathcurveto{\pgfqpoint{2.968827in}{2.946823in}}{\pgfqpoint{2.973217in}{2.957422in}}{\pgfqpoint{2.973217in}{2.968472in}}%
\pgfpathcurveto{\pgfqpoint{2.973217in}{2.979522in}}{\pgfqpoint{2.968827in}{2.990121in}}{\pgfqpoint{2.961013in}{2.997935in}}%
\pgfpathcurveto{\pgfqpoint{2.953199in}{3.005749in}}{\pgfqpoint{2.942600in}{3.010139in}}{\pgfqpoint{2.931550in}{3.010139in}}%
\pgfpathcurveto{\pgfqpoint{2.920500in}{3.010139in}}{\pgfqpoint{2.909901in}{3.005749in}}{\pgfqpoint{2.902088in}{2.997935in}}%
\pgfpathcurveto{\pgfqpoint{2.894274in}{2.990121in}}{\pgfqpoint{2.889884in}{2.979522in}}{\pgfqpoint{2.889884in}{2.968472in}}%
\pgfpathcurveto{\pgfqpoint{2.889884in}{2.957422in}}{\pgfqpoint{2.894274in}{2.946823in}}{\pgfqpoint{2.902088in}{2.939009in}}%
\pgfpathcurveto{\pgfqpoint{2.909901in}{2.931196in}}{\pgfqpoint{2.920500in}{2.926806in}}{\pgfqpoint{2.931550in}{2.926806in}}%
\pgfpathclose%
\pgfusepath{stroke,fill}%
\end{pgfscope}%
\begin{pgfscope}%
\pgfpathrectangle{\pgfqpoint{0.600000in}{0.600000in}}{\pgfqpoint{3.900000in}{3.900000in}}%
\pgfusepath{clip}%
\pgfsetbuttcap%
\pgfsetroundjoin%
\definecolor{currentfill}{rgb}{0.121569,0.466667,0.705882}%
\pgfsetfillcolor{currentfill}%
\pgfsetlinewidth{1.003750pt}%
\definecolor{currentstroke}{rgb}{0.121569,0.466667,0.705882}%
\pgfsetstrokecolor{currentstroke}%
\pgfsetdash{}{0pt}%
\pgfpathmoveto{\pgfqpoint{2.605433in}{2.293384in}}%
\pgfpathcurveto{\pgfqpoint{2.616483in}{2.293384in}}{\pgfqpoint{2.627082in}{2.297774in}}{\pgfqpoint{2.634895in}{2.305588in}}%
\pgfpathcurveto{\pgfqpoint{2.642709in}{2.313401in}}{\pgfqpoint{2.647099in}{2.324000in}}{\pgfqpoint{2.647099in}{2.335051in}}%
\pgfpathcurveto{\pgfqpoint{2.647099in}{2.346101in}}{\pgfqpoint{2.642709in}{2.356700in}}{\pgfqpoint{2.634895in}{2.364513in}}%
\pgfpathcurveto{\pgfqpoint{2.627082in}{2.372327in}}{\pgfqpoint{2.616483in}{2.376717in}}{\pgfqpoint{2.605433in}{2.376717in}}%
\pgfpathcurveto{\pgfqpoint{2.594383in}{2.376717in}}{\pgfqpoint{2.583783in}{2.372327in}}{\pgfqpoint{2.575970in}{2.364513in}}%
\pgfpathcurveto{\pgfqpoint{2.568156in}{2.356700in}}{\pgfqpoint{2.563766in}{2.346101in}}{\pgfqpoint{2.563766in}{2.335051in}}%
\pgfpathcurveto{\pgfqpoint{2.563766in}{2.324000in}}{\pgfqpoint{2.568156in}{2.313401in}}{\pgfqpoint{2.575970in}{2.305588in}}%
\pgfpathcurveto{\pgfqpoint{2.583783in}{2.297774in}}{\pgfqpoint{2.594383in}{2.293384in}}{\pgfqpoint{2.605433in}{2.293384in}}%
\pgfpathclose%
\pgfusepath{stroke,fill}%
\end{pgfscope}%
\begin{pgfscope}%
\pgfpathrectangle{\pgfqpoint{0.600000in}{0.600000in}}{\pgfqpoint{3.900000in}{3.900000in}}%
\pgfusepath{clip}%
\pgfsetbuttcap%
\pgfsetroundjoin%
\definecolor{currentfill}{rgb}{0.121569,0.466667,0.705882}%
\pgfsetfillcolor{currentfill}%
\pgfsetlinewidth{1.003750pt}%
\definecolor{currentstroke}{rgb}{0.121569,0.466667,0.705882}%
\pgfsetstrokecolor{currentstroke}%
\pgfsetdash{}{0pt}%
\pgfpathmoveto{\pgfqpoint{1.634971in}{3.216109in}}%
\pgfpathcurveto{\pgfqpoint{1.646021in}{3.216109in}}{\pgfqpoint{1.656620in}{3.220499in}}{\pgfqpoint{1.664433in}{3.228312in}}%
\pgfpathcurveto{\pgfqpoint{1.672247in}{3.236126in}}{\pgfqpoint{1.676637in}{3.246725in}}{\pgfqpoint{1.676637in}{3.257775in}}%
\pgfpathcurveto{\pgfqpoint{1.676637in}{3.268825in}}{\pgfqpoint{1.672247in}{3.279424in}}{\pgfqpoint{1.664433in}{3.287238in}}%
\pgfpathcurveto{\pgfqpoint{1.656620in}{3.295052in}}{\pgfqpoint{1.646021in}{3.299442in}}{\pgfqpoint{1.634971in}{3.299442in}}%
\pgfpathcurveto{\pgfqpoint{1.623920in}{3.299442in}}{\pgfqpoint{1.613321in}{3.295052in}}{\pgfqpoint{1.605508in}{3.287238in}}%
\pgfpathcurveto{\pgfqpoint{1.597694in}{3.279424in}}{\pgfqpoint{1.593304in}{3.268825in}}{\pgfqpoint{1.593304in}{3.257775in}}%
\pgfpathcurveto{\pgfqpoint{1.593304in}{3.246725in}}{\pgfqpoint{1.597694in}{3.236126in}}{\pgfqpoint{1.605508in}{3.228312in}}%
\pgfpathcurveto{\pgfqpoint{1.613321in}{3.220499in}}{\pgfqpoint{1.623920in}{3.216109in}}{\pgfqpoint{1.634971in}{3.216109in}}%
\pgfpathclose%
\pgfusepath{stroke,fill}%
\end{pgfscope}%
\begin{pgfscope}%
\pgfpathrectangle{\pgfqpoint{0.600000in}{0.600000in}}{\pgfqpoint{3.900000in}{3.900000in}}%
\pgfusepath{clip}%
\pgfsetbuttcap%
\pgfsetroundjoin%
\definecolor{currentfill}{rgb}{0.121569,0.466667,0.705882}%
\pgfsetfillcolor{currentfill}%
\pgfsetlinewidth{1.003750pt}%
\definecolor{currentstroke}{rgb}{0.121569,0.466667,0.705882}%
\pgfsetstrokecolor{currentstroke}%
\pgfsetdash{}{0pt}%
\pgfpathmoveto{\pgfqpoint{2.415162in}{2.638950in}}%
\pgfpathcurveto{\pgfqpoint{2.426212in}{2.638950in}}{\pgfqpoint{2.436811in}{2.643340in}}{\pgfqpoint{2.444625in}{2.651154in}}%
\pgfpathcurveto{\pgfqpoint{2.452438in}{2.658968in}}{\pgfqpoint{2.456829in}{2.669567in}}{\pgfqpoint{2.456829in}{2.680617in}}%
\pgfpathcurveto{\pgfqpoint{2.456829in}{2.691667in}}{\pgfqpoint{2.452438in}{2.702266in}}{\pgfqpoint{2.444625in}{2.710080in}}%
\pgfpathcurveto{\pgfqpoint{2.436811in}{2.717893in}}{\pgfqpoint{2.426212in}{2.722284in}}{\pgfqpoint{2.415162in}{2.722284in}}%
\pgfpathcurveto{\pgfqpoint{2.404112in}{2.722284in}}{\pgfqpoint{2.393513in}{2.717893in}}{\pgfqpoint{2.385699in}{2.710080in}}%
\pgfpathcurveto{\pgfqpoint{2.377885in}{2.702266in}}{\pgfqpoint{2.373495in}{2.691667in}}{\pgfqpoint{2.373495in}{2.680617in}}%
\pgfpathcurveto{\pgfqpoint{2.373495in}{2.669567in}}{\pgfqpoint{2.377885in}{2.658968in}}{\pgfqpoint{2.385699in}{2.651154in}}%
\pgfpathcurveto{\pgfqpoint{2.393513in}{2.643340in}}{\pgfqpoint{2.404112in}{2.638950in}}{\pgfqpoint{2.415162in}{2.638950in}}%
\pgfpathclose%
\pgfusepath{stroke,fill}%
\end{pgfscope}%
\begin{pgfscope}%
\pgfpathrectangle{\pgfqpoint{0.600000in}{0.600000in}}{\pgfqpoint{3.900000in}{3.900000in}}%
\pgfusepath{clip}%
\pgfsetbuttcap%
\pgfsetroundjoin%
\definecolor{currentfill}{rgb}{0.121569,0.466667,0.705882}%
\pgfsetfillcolor{currentfill}%
\pgfsetlinewidth{1.003750pt}%
\definecolor{currentstroke}{rgb}{0.121569,0.466667,0.705882}%
\pgfsetstrokecolor{currentstroke}%
\pgfsetdash{}{0pt}%
\pgfpathmoveto{\pgfqpoint{2.482389in}{2.171571in}}%
\pgfpathcurveto{\pgfqpoint{2.493439in}{2.171571in}}{\pgfqpoint{2.504038in}{2.175961in}}{\pgfqpoint{2.511852in}{2.183775in}}%
\pgfpathcurveto{\pgfqpoint{2.519665in}{2.191589in}}{\pgfqpoint{2.524056in}{2.202188in}}{\pgfqpoint{2.524056in}{2.213238in}}%
\pgfpathcurveto{\pgfqpoint{2.524056in}{2.224288in}}{\pgfqpoint{2.519665in}{2.234887in}}{\pgfqpoint{2.511852in}{2.242701in}}%
\pgfpathcurveto{\pgfqpoint{2.504038in}{2.250514in}}{\pgfqpoint{2.493439in}{2.254904in}}{\pgfqpoint{2.482389in}{2.254904in}}%
\pgfpathcurveto{\pgfqpoint{2.471339in}{2.254904in}}{\pgfqpoint{2.460740in}{2.250514in}}{\pgfqpoint{2.452926in}{2.242701in}}%
\pgfpathcurveto{\pgfqpoint{2.445113in}{2.234887in}}{\pgfqpoint{2.440722in}{2.224288in}}{\pgfqpoint{2.440722in}{2.213238in}}%
\pgfpathcurveto{\pgfqpoint{2.440722in}{2.202188in}}{\pgfqpoint{2.445113in}{2.191589in}}{\pgfqpoint{2.452926in}{2.183775in}}%
\pgfpathcurveto{\pgfqpoint{2.460740in}{2.175961in}}{\pgfqpoint{2.471339in}{2.171571in}}{\pgfqpoint{2.482389in}{2.171571in}}%
\pgfpathclose%
\pgfusepath{stroke,fill}%
\end{pgfscope}%
\begin{pgfscope}%
\pgfpathrectangle{\pgfqpoint{0.600000in}{0.600000in}}{\pgfqpoint{3.900000in}{3.900000in}}%
\pgfusepath{clip}%
\pgfsetbuttcap%
\pgfsetroundjoin%
\definecolor{currentfill}{rgb}{0.121569,0.466667,0.705882}%
\pgfsetfillcolor{currentfill}%
\pgfsetlinewidth{1.003750pt}%
\definecolor{currentstroke}{rgb}{0.121569,0.466667,0.705882}%
\pgfsetstrokecolor{currentstroke}%
\pgfsetdash{}{0pt}%
\pgfpathmoveto{\pgfqpoint{3.487204in}{2.250570in}}%
\pgfpathcurveto{\pgfqpoint{3.498254in}{2.250570in}}{\pgfqpoint{3.508853in}{2.254960in}}{\pgfqpoint{3.516667in}{2.262774in}}%
\pgfpathcurveto{\pgfqpoint{3.524481in}{2.270587in}}{\pgfqpoint{3.528871in}{2.281186in}}{\pgfqpoint{3.528871in}{2.292236in}}%
\pgfpathcurveto{\pgfqpoint{3.528871in}{2.303287in}}{\pgfqpoint{3.524481in}{2.313886in}}{\pgfqpoint{3.516667in}{2.321699in}}%
\pgfpathcurveto{\pgfqpoint{3.508853in}{2.329513in}}{\pgfqpoint{3.498254in}{2.333903in}}{\pgfqpoint{3.487204in}{2.333903in}}%
\pgfpathcurveto{\pgfqpoint{3.476154in}{2.333903in}}{\pgfqpoint{3.465555in}{2.329513in}}{\pgfqpoint{3.457742in}{2.321699in}}%
\pgfpathcurveto{\pgfqpoint{3.449928in}{2.313886in}}{\pgfqpoint{3.445538in}{2.303287in}}{\pgfqpoint{3.445538in}{2.292236in}}%
\pgfpathcurveto{\pgfqpoint{3.445538in}{2.281186in}}{\pgfqpoint{3.449928in}{2.270587in}}{\pgfqpoint{3.457742in}{2.262774in}}%
\pgfpathcurveto{\pgfqpoint{3.465555in}{2.254960in}}{\pgfqpoint{3.476154in}{2.250570in}}{\pgfqpoint{3.487204in}{2.250570in}}%
\pgfpathclose%
\pgfusepath{stroke,fill}%
\end{pgfscope}%
\begin{pgfscope}%
\pgfpathrectangle{\pgfqpoint{0.600000in}{0.600000in}}{\pgfqpoint{3.900000in}{3.900000in}}%
\pgfusepath{clip}%
\pgfsetbuttcap%
\pgfsetroundjoin%
\definecolor{currentfill}{rgb}{0.121569,0.466667,0.705882}%
\pgfsetfillcolor{currentfill}%
\pgfsetlinewidth{1.003750pt}%
\definecolor{currentstroke}{rgb}{0.121569,0.466667,0.705882}%
\pgfsetstrokecolor{currentstroke}%
\pgfsetdash{}{0pt}%
\pgfpathmoveto{\pgfqpoint{2.234737in}{2.110513in}}%
\pgfpathcurveto{\pgfqpoint{2.245787in}{2.110513in}}{\pgfqpoint{2.256386in}{2.114904in}}{\pgfqpoint{2.264200in}{2.122717in}}%
\pgfpathcurveto{\pgfqpoint{2.272013in}{2.130531in}}{\pgfqpoint{2.276403in}{2.141130in}}{\pgfqpoint{2.276403in}{2.152180in}}%
\pgfpathcurveto{\pgfqpoint{2.276403in}{2.163230in}}{\pgfqpoint{2.272013in}{2.173829in}}{\pgfqpoint{2.264200in}{2.181643in}}%
\pgfpathcurveto{\pgfqpoint{2.256386in}{2.189456in}}{\pgfqpoint{2.245787in}{2.193847in}}{\pgfqpoint{2.234737in}{2.193847in}}%
\pgfpathcurveto{\pgfqpoint{2.223687in}{2.193847in}}{\pgfqpoint{2.213088in}{2.189456in}}{\pgfqpoint{2.205274in}{2.181643in}}%
\pgfpathcurveto{\pgfqpoint{2.197460in}{2.173829in}}{\pgfqpoint{2.193070in}{2.163230in}}{\pgfqpoint{2.193070in}{2.152180in}}%
\pgfpathcurveto{\pgfqpoint{2.193070in}{2.141130in}}{\pgfqpoint{2.197460in}{2.130531in}}{\pgfqpoint{2.205274in}{2.122717in}}%
\pgfpathcurveto{\pgfqpoint{2.213088in}{2.114904in}}{\pgfqpoint{2.223687in}{2.110513in}}{\pgfqpoint{2.234737in}{2.110513in}}%
\pgfpathclose%
\pgfusepath{stroke,fill}%
\end{pgfscope}%
\begin{pgfscope}%
\pgfpathrectangle{\pgfqpoint{0.600000in}{0.600000in}}{\pgfqpoint{3.900000in}{3.900000in}}%
\pgfusepath{clip}%
\pgfsetbuttcap%
\pgfsetroundjoin%
\definecolor{currentfill}{rgb}{0.121569,0.466667,0.705882}%
\pgfsetfillcolor{currentfill}%
\pgfsetlinewidth{1.003750pt}%
\definecolor{currentstroke}{rgb}{0.121569,0.466667,0.705882}%
\pgfsetstrokecolor{currentstroke}%
\pgfsetdash{}{0pt}%
\pgfpathmoveto{\pgfqpoint{2.118960in}{3.148794in}}%
\pgfpathcurveto{\pgfqpoint{2.130010in}{3.148794in}}{\pgfqpoint{2.140609in}{3.153184in}}{\pgfqpoint{2.148423in}{3.160998in}}%
\pgfpathcurveto{\pgfqpoint{2.156237in}{3.168811in}}{\pgfqpoint{2.160627in}{3.179410in}}{\pgfqpoint{2.160627in}{3.190460in}}%
\pgfpathcurveto{\pgfqpoint{2.160627in}{3.201511in}}{\pgfqpoint{2.156237in}{3.212110in}}{\pgfqpoint{2.148423in}{3.219923in}}%
\pgfpathcurveto{\pgfqpoint{2.140609in}{3.227737in}}{\pgfqpoint{2.130010in}{3.232127in}}{\pgfqpoint{2.118960in}{3.232127in}}%
\pgfpathcurveto{\pgfqpoint{2.107910in}{3.232127in}}{\pgfqpoint{2.097311in}{3.227737in}}{\pgfqpoint{2.089497in}{3.219923in}}%
\pgfpathcurveto{\pgfqpoint{2.081684in}{3.212110in}}{\pgfqpoint{2.077294in}{3.201511in}}{\pgfqpoint{2.077294in}{3.190460in}}%
\pgfpathcurveto{\pgfqpoint{2.077294in}{3.179410in}}{\pgfqpoint{2.081684in}{3.168811in}}{\pgfqpoint{2.089497in}{3.160998in}}%
\pgfpathcurveto{\pgfqpoint{2.097311in}{3.153184in}}{\pgfqpoint{2.107910in}{3.148794in}}{\pgfqpoint{2.118960in}{3.148794in}}%
\pgfpathclose%
\pgfusepath{stroke,fill}%
\end{pgfscope}%
\begin{pgfscope}%
\pgfpathrectangle{\pgfqpoint{0.600000in}{0.600000in}}{\pgfqpoint{3.900000in}{3.900000in}}%
\pgfusepath{clip}%
\pgfsetbuttcap%
\pgfsetroundjoin%
\definecolor{currentfill}{rgb}{0.121569,0.466667,0.705882}%
\pgfsetfillcolor{currentfill}%
\pgfsetlinewidth{1.003750pt}%
\definecolor{currentstroke}{rgb}{0.121569,0.466667,0.705882}%
\pgfsetstrokecolor{currentstroke}%
\pgfsetdash{}{0pt}%
\pgfpathmoveto{\pgfqpoint{2.334550in}{3.001392in}}%
\pgfpathcurveto{\pgfqpoint{2.345600in}{3.001392in}}{\pgfqpoint{2.356199in}{3.005783in}}{\pgfqpoint{2.364013in}{3.013596in}}%
\pgfpathcurveto{\pgfqpoint{2.371826in}{3.021410in}}{\pgfqpoint{2.376217in}{3.032009in}}{\pgfqpoint{2.376217in}{3.043059in}}%
\pgfpathcurveto{\pgfqpoint{2.376217in}{3.054109in}}{\pgfqpoint{2.371826in}{3.064708in}}{\pgfqpoint{2.364013in}{3.072522in}}%
\pgfpathcurveto{\pgfqpoint{2.356199in}{3.080335in}}{\pgfqpoint{2.345600in}{3.084726in}}{\pgfqpoint{2.334550in}{3.084726in}}%
\pgfpathcurveto{\pgfqpoint{2.323500in}{3.084726in}}{\pgfqpoint{2.312901in}{3.080335in}}{\pgfqpoint{2.305087in}{3.072522in}}%
\pgfpathcurveto{\pgfqpoint{2.297274in}{3.064708in}}{\pgfqpoint{2.292883in}{3.054109in}}{\pgfqpoint{2.292883in}{3.043059in}}%
\pgfpathcurveto{\pgfqpoint{2.292883in}{3.032009in}}{\pgfqpoint{2.297274in}{3.021410in}}{\pgfqpoint{2.305087in}{3.013596in}}%
\pgfpathcurveto{\pgfqpoint{2.312901in}{3.005783in}}{\pgfqpoint{2.323500in}{3.001392in}}{\pgfqpoint{2.334550in}{3.001392in}}%
\pgfpathclose%
\pgfusepath{stroke,fill}%
\end{pgfscope}%
\begin{pgfscope}%
\pgfpathrectangle{\pgfqpoint{0.600000in}{0.600000in}}{\pgfqpoint{3.900000in}{3.900000in}}%
\pgfusepath{clip}%
\pgfsetbuttcap%
\pgfsetroundjoin%
\definecolor{currentfill}{rgb}{0.121569,0.466667,0.705882}%
\pgfsetfillcolor{currentfill}%
\pgfsetlinewidth{1.003750pt}%
\definecolor{currentstroke}{rgb}{0.121569,0.466667,0.705882}%
\pgfsetstrokecolor{currentstroke}%
\pgfsetdash{}{0pt}%
\pgfpathmoveto{\pgfqpoint{1.814522in}{2.878544in}}%
\pgfpathcurveto{\pgfqpoint{1.825572in}{2.878544in}}{\pgfqpoint{1.836171in}{2.882934in}}{\pgfqpoint{1.843985in}{2.890748in}}%
\pgfpathcurveto{\pgfqpoint{1.851798in}{2.898562in}}{\pgfqpoint{1.856188in}{2.909161in}}{\pgfqpoint{1.856188in}{2.920211in}}%
\pgfpathcurveto{\pgfqpoint{1.856188in}{2.931261in}}{\pgfqpoint{1.851798in}{2.941860in}}{\pgfqpoint{1.843985in}{2.949674in}}%
\pgfpathcurveto{\pgfqpoint{1.836171in}{2.957487in}}{\pgfqpoint{1.825572in}{2.961877in}}{\pgfqpoint{1.814522in}{2.961877in}}%
\pgfpathcurveto{\pgfqpoint{1.803472in}{2.961877in}}{\pgfqpoint{1.792873in}{2.957487in}}{\pgfqpoint{1.785059in}{2.949674in}}%
\pgfpathcurveto{\pgfqpoint{1.777245in}{2.941860in}}{\pgfqpoint{1.772855in}{2.931261in}}{\pgfqpoint{1.772855in}{2.920211in}}%
\pgfpathcurveto{\pgfqpoint{1.772855in}{2.909161in}}{\pgfqpoint{1.777245in}{2.898562in}}{\pgfqpoint{1.785059in}{2.890748in}}%
\pgfpathcurveto{\pgfqpoint{1.792873in}{2.882934in}}{\pgfqpoint{1.803472in}{2.878544in}}{\pgfqpoint{1.814522in}{2.878544in}}%
\pgfpathclose%
\pgfusepath{stroke,fill}%
\end{pgfscope}%
\begin{pgfscope}%
\pgfpathrectangle{\pgfqpoint{0.600000in}{0.600000in}}{\pgfqpoint{3.900000in}{3.900000in}}%
\pgfusepath{clip}%
\pgfsetbuttcap%
\pgfsetroundjoin%
\definecolor{currentfill}{rgb}{0.121569,0.466667,0.705882}%
\pgfsetfillcolor{currentfill}%
\pgfsetlinewidth{1.003750pt}%
\definecolor{currentstroke}{rgb}{0.121569,0.466667,0.705882}%
\pgfsetstrokecolor{currentstroke}%
\pgfsetdash{}{0pt}%
\pgfpathmoveto{\pgfqpoint{2.262481in}{2.679662in}}%
\pgfpathcurveto{\pgfqpoint{2.273531in}{2.679662in}}{\pgfqpoint{2.284130in}{2.684052in}}{\pgfqpoint{2.291944in}{2.691866in}}%
\pgfpathcurveto{\pgfqpoint{2.299757in}{2.699679in}}{\pgfqpoint{2.304148in}{2.710278in}}{\pgfqpoint{2.304148in}{2.721329in}}%
\pgfpathcurveto{\pgfqpoint{2.304148in}{2.732379in}}{\pgfqpoint{2.299757in}{2.742978in}}{\pgfqpoint{2.291944in}{2.750791in}}%
\pgfpathcurveto{\pgfqpoint{2.284130in}{2.758605in}}{\pgfqpoint{2.273531in}{2.762995in}}{\pgfqpoint{2.262481in}{2.762995in}}%
\pgfpathcurveto{\pgfqpoint{2.251431in}{2.762995in}}{\pgfqpoint{2.240832in}{2.758605in}}{\pgfqpoint{2.233018in}{2.750791in}}%
\pgfpathcurveto{\pgfqpoint{2.225205in}{2.742978in}}{\pgfqpoint{2.220814in}{2.732379in}}{\pgfqpoint{2.220814in}{2.721329in}}%
\pgfpathcurveto{\pgfqpoint{2.220814in}{2.710278in}}{\pgfqpoint{2.225205in}{2.699679in}}{\pgfqpoint{2.233018in}{2.691866in}}%
\pgfpathcurveto{\pgfqpoint{2.240832in}{2.684052in}}{\pgfqpoint{2.251431in}{2.679662in}}{\pgfqpoint{2.262481in}{2.679662in}}%
\pgfpathclose%
\pgfusepath{stroke,fill}%
\end{pgfscope}%
\begin{pgfscope}%
\pgfpathrectangle{\pgfqpoint{0.600000in}{0.600000in}}{\pgfqpoint{3.900000in}{3.900000in}}%
\pgfusepath{clip}%
\pgfsetbuttcap%
\pgfsetroundjoin%
\definecolor{currentfill}{rgb}{0.121569,0.466667,0.705882}%
\pgfsetfillcolor{currentfill}%
\pgfsetlinewidth{1.003750pt}%
\definecolor{currentstroke}{rgb}{0.121569,0.466667,0.705882}%
\pgfsetstrokecolor{currentstroke}%
\pgfsetdash{}{0pt}%
\pgfpathmoveto{\pgfqpoint{2.582012in}{1.856708in}}%
\pgfpathcurveto{\pgfqpoint{2.593062in}{1.856708in}}{\pgfqpoint{2.603661in}{1.861099in}}{\pgfqpoint{2.611475in}{1.868912in}}%
\pgfpathcurveto{\pgfqpoint{2.619288in}{1.876726in}}{\pgfqpoint{2.623679in}{1.887325in}}{\pgfqpoint{2.623679in}{1.898375in}}%
\pgfpathcurveto{\pgfqpoint{2.623679in}{1.909425in}}{\pgfqpoint{2.619288in}{1.920024in}}{\pgfqpoint{2.611475in}{1.927838in}}%
\pgfpathcurveto{\pgfqpoint{2.603661in}{1.935651in}}{\pgfqpoint{2.593062in}{1.940042in}}{\pgfqpoint{2.582012in}{1.940042in}}%
\pgfpathcurveto{\pgfqpoint{2.570962in}{1.940042in}}{\pgfqpoint{2.560363in}{1.935651in}}{\pgfqpoint{2.552549in}{1.927838in}}%
\pgfpathcurveto{\pgfqpoint{2.544736in}{1.920024in}}{\pgfqpoint{2.540345in}{1.909425in}}{\pgfqpoint{2.540345in}{1.898375in}}%
\pgfpathcurveto{\pgfqpoint{2.540345in}{1.887325in}}{\pgfqpoint{2.544736in}{1.876726in}}{\pgfqpoint{2.552549in}{1.868912in}}%
\pgfpathcurveto{\pgfqpoint{2.560363in}{1.861099in}}{\pgfqpoint{2.570962in}{1.856708in}}{\pgfqpoint{2.582012in}{1.856708in}}%
\pgfpathclose%
\pgfusepath{stroke,fill}%
\end{pgfscope}%
\begin{pgfscope}%
\pgfpathrectangle{\pgfqpoint{0.600000in}{0.600000in}}{\pgfqpoint{3.900000in}{3.900000in}}%
\pgfusepath{clip}%
\pgfsetbuttcap%
\pgfsetroundjoin%
\definecolor{currentfill}{rgb}{0.121569,0.466667,0.705882}%
\pgfsetfillcolor{currentfill}%
\pgfsetlinewidth{1.003750pt}%
\definecolor{currentstroke}{rgb}{0.121569,0.466667,0.705882}%
\pgfsetstrokecolor{currentstroke}%
\pgfsetdash{}{0pt}%
\pgfpathmoveto{\pgfqpoint{3.246809in}{1.868561in}}%
\pgfpathcurveto{\pgfqpoint{3.257859in}{1.868561in}}{\pgfqpoint{3.268458in}{1.872951in}}{\pgfqpoint{3.276271in}{1.880765in}}%
\pgfpathcurveto{\pgfqpoint{3.284085in}{1.888578in}}{\pgfqpoint{3.288475in}{1.899177in}}{\pgfqpoint{3.288475in}{1.910227in}}%
\pgfpathcurveto{\pgfqpoint{3.288475in}{1.921277in}}{\pgfqpoint{3.284085in}{1.931877in}}{\pgfqpoint{3.276271in}{1.939690in}}%
\pgfpathcurveto{\pgfqpoint{3.268458in}{1.947504in}}{\pgfqpoint{3.257859in}{1.951894in}}{\pgfqpoint{3.246809in}{1.951894in}}%
\pgfpathcurveto{\pgfqpoint{3.235759in}{1.951894in}}{\pgfqpoint{3.225159in}{1.947504in}}{\pgfqpoint{3.217346in}{1.939690in}}%
\pgfpathcurveto{\pgfqpoint{3.209532in}{1.931877in}}{\pgfqpoint{3.205142in}{1.921277in}}{\pgfqpoint{3.205142in}{1.910227in}}%
\pgfpathcurveto{\pgfqpoint{3.205142in}{1.899177in}}{\pgfqpoint{3.209532in}{1.888578in}}{\pgfqpoint{3.217346in}{1.880765in}}%
\pgfpathcurveto{\pgfqpoint{3.225159in}{1.872951in}}{\pgfqpoint{3.235759in}{1.868561in}}{\pgfqpoint{3.246809in}{1.868561in}}%
\pgfpathclose%
\pgfusepath{stroke,fill}%
\end{pgfscope}%
\begin{pgfscope}%
\pgfpathrectangle{\pgfqpoint{0.600000in}{0.600000in}}{\pgfqpoint{3.900000in}{3.900000in}}%
\pgfusepath{clip}%
\pgfsetbuttcap%
\pgfsetroundjoin%
\definecolor{currentfill}{rgb}{0.121569,0.466667,0.705882}%
\pgfsetfillcolor{currentfill}%
\pgfsetlinewidth{1.003750pt}%
\definecolor{currentstroke}{rgb}{0.121569,0.466667,0.705882}%
\pgfsetstrokecolor{currentstroke}%
\pgfsetdash{}{0pt}%
\pgfpathmoveto{\pgfqpoint{1.505964in}{3.786409in}}%
\pgfpathcurveto{\pgfqpoint{1.517014in}{3.786409in}}{\pgfqpoint{1.527613in}{3.790799in}}{\pgfqpoint{1.535427in}{3.798613in}}%
\pgfpathcurveto{\pgfqpoint{1.543240in}{3.806427in}}{\pgfqpoint{1.547631in}{3.817026in}}{\pgfqpoint{1.547631in}{3.828076in}}%
\pgfpathcurveto{\pgfqpoint{1.547631in}{3.839126in}}{\pgfqpoint{1.543240in}{3.849725in}}{\pgfqpoint{1.535427in}{3.857539in}}%
\pgfpathcurveto{\pgfqpoint{1.527613in}{3.865352in}}{\pgfqpoint{1.517014in}{3.869742in}}{\pgfqpoint{1.505964in}{3.869742in}}%
\pgfpathcurveto{\pgfqpoint{1.494914in}{3.869742in}}{\pgfqpoint{1.484315in}{3.865352in}}{\pgfqpoint{1.476501in}{3.857539in}}%
\pgfpathcurveto{\pgfqpoint{1.468688in}{3.849725in}}{\pgfqpoint{1.464297in}{3.839126in}}{\pgfqpoint{1.464297in}{3.828076in}}%
\pgfpathcurveto{\pgfqpoint{1.464297in}{3.817026in}}{\pgfqpoint{1.468688in}{3.806427in}}{\pgfqpoint{1.476501in}{3.798613in}}%
\pgfpathcurveto{\pgfqpoint{1.484315in}{3.790799in}}{\pgfqpoint{1.494914in}{3.786409in}}{\pgfqpoint{1.505964in}{3.786409in}}%
\pgfpathclose%
\pgfusepath{stroke,fill}%
\end{pgfscope}%
\begin{pgfscope}%
\pgfpathrectangle{\pgfqpoint{0.600000in}{0.600000in}}{\pgfqpoint{3.900000in}{3.900000in}}%
\pgfusepath{clip}%
\pgfsetbuttcap%
\pgfsetroundjoin%
\definecolor{currentfill}{rgb}{0.121569,0.466667,0.705882}%
\pgfsetfillcolor{currentfill}%
\pgfsetlinewidth{1.003750pt}%
\definecolor{currentstroke}{rgb}{0.121569,0.466667,0.705882}%
\pgfsetstrokecolor{currentstroke}%
\pgfsetdash{}{0pt}%
\pgfpathmoveto{\pgfqpoint{2.257965in}{2.957945in}}%
\pgfpathcurveto{\pgfqpoint{2.269016in}{2.957945in}}{\pgfqpoint{2.279615in}{2.962336in}}{\pgfqpoint{2.287428in}{2.970149in}}%
\pgfpathcurveto{\pgfqpoint{2.295242in}{2.977963in}}{\pgfqpoint{2.299632in}{2.988562in}}{\pgfqpoint{2.299632in}{2.999612in}}%
\pgfpathcurveto{\pgfqpoint{2.299632in}{3.010662in}}{\pgfqpoint{2.295242in}{3.021261in}}{\pgfqpoint{2.287428in}{3.029075in}}%
\pgfpathcurveto{\pgfqpoint{2.279615in}{3.036888in}}{\pgfqpoint{2.269016in}{3.041279in}}{\pgfqpoint{2.257965in}{3.041279in}}%
\pgfpathcurveto{\pgfqpoint{2.246915in}{3.041279in}}{\pgfqpoint{2.236316in}{3.036888in}}{\pgfqpoint{2.228503in}{3.029075in}}%
\pgfpathcurveto{\pgfqpoint{2.220689in}{3.021261in}}{\pgfqpoint{2.216299in}{3.010662in}}{\pgfqpoint{2.216299in}{2.999612in}}%
\pgfpathcurveto{\pgfqpoint{2.216299in}{2.988562in}}{\pgfqpoint{2.220689in}{2.977963in}}{\pgfqpoint{2.228503in}{2.970149in}}%
\pgfpathcurveto{\pgfqpoint{2.236316in}{2.962336in}}{\pgfqpoint{2.246915in}{2.957945in}}{\pgfqpoint{2.257965in}{2.957945in}}%
\pgfpathclose%
\pgfusepath{stroke,fill}%
\end{pgfscope}%
\begin{pgfscope}%
\pgfpathrectangle{\pgfqpoint{0.600000in}{0.600000in}}{\pgfqpoint{3.900000in}{3.900000in}}%
\pgfusepath{clip}%
\pgfsetbuttcap%
\pgfsetroundjoin%
\definecolor{currentfill}{rgb}{0.121569,0.466667,0.705882}%
\pgfsetfillcolor{currentfill}%
\pgfsetlinewidth{1.003750pt}%
\definecolor{currentstroke}{rgb}{0.121569,0.466667,0.705882}%
\pgfsetstrokecolor{currentstroke}%
\pgfsetdash{}{0pt}%
\pgfpathmoveto{\pgfqpoint{2.321193in}{2.061145in}}%
\pgfpathcurveto{\pgfqpoint{2.332243in}{2.061145in}}{\pgfqpoint{2.342842in}{2.065535in}}{\pgfqpoint{2.350655in}{2.073349in}}%
\pgfpathcurveto{\pgfqpoint{2.358469in}{2.081162in}}{\pgfqpoint{2.362859in}{2.091761in}}{\pgfqpoint{2.362859in}{2.102811in}}%
\pgfpathcurveto{\pgfqpoint{2.362859in}{2.113861in}}{\pgfqpoint{2.358469in}{2.124460in}}{\pgfqpoint{2.350655in}{2.132274in}}%
\pgfpathcurveto{\pgfqpoint{2.342842in}{2.140088in}}{\pgfqpoint{2.332243in}{2.144478in}}{\pgfqpoint{2.321193in}{2.144478in}}%
\pgfpathcurveto{\pgfqpoint{2.310143in}{2.144478in}}{\pgfqpoint{2.299544in}{2.140088in}}{\pgfqpoint{2.291730in}{2.132274in}}%
\pgfpathcurveto{\pgfqpoint{2.283916in}{2.124460in}}{\pgfqpoint{2.279526in}{2.113861in}}{\pgfqpoint{2.279526in}{2.102811in}}%
\pgfpathcurveto{\pgfqpoint{2.279526in}{2.091761in}}{\pgfqpoint{2.283916in}{2.081162in}}{\pgfqpoint{2.291730in}{2.073349in}}%
\pgfpathcurveto{\pgfqpoint{2.299544in}{2.065535in}}{\pgfqpoint{2.310143in}{2.061145in}}{\pgfqpoint{2.321193in}{2.061145in}}%
\pgfpathclose%
\pgfusepath{stroke,fill}%
\end{pgfscope}%
\begin{pgfscope}%
\pgfpathrectangle{\pgfqpoint{0.600000in}{0.600000in}}{\pgfqpoint{3.900000in}{3.900000in}}%
\pgfusepath{clip}%
\pgfsetbuttcap%
\pgfsetroundjoin%
\definecolor{currentfill}{rgb}{0.121569,0.466667,0.705882}%
\pgfsetfillcolor{currentfill}%
\pgfsetlinewidth{1.003750pt}%
\definecolor{currentstroke}{rgb}{0.121569,0.466667,0.705882}%
\pgfsetstrokecolor{currentstroke}%
\pgfsetdash{}{0pt}%
\pgfpathmoveto{\pgfqpoint{3.024836in}{3.084634in}}%
\pgfpathcurveto{\pgfqpoint{3.035886in}{3.084634in}}{\pgfqpoint{3.046485in}{3.089024in}}{\pgfqpoint{3.054299in}{3.096838in}}%
\pgfpathcurveto{\pgfqpoint{3.062112in}{3.104652in}}{\pgfqpoint{3.066503in}{3.115251in}}{\pgfqpoint{3.066503in}{3.126301in}}%
\pgfpathcurveto{\pgfqpoint{3.066503in}{3.137351in}}{\pgfqpoint{3.062112in}{3.147950in}}{\pgfqpoint{3.054299in}{3.155763in}}%
\pgfpathcurveto{\pgfqpoint{3.046485in}{3.163577in}}{\pgfqpoint{3.035886in}{3.167967in}}{\pgfqpoint{3.024836in}{3.167967in}}%
\pgfpathcurveto{\pgfqpoint{3.013786in}{3.167967in}}{\pgfqpoint{3.003187in}{3.163577in}}{\pgfqpoint{2.995373in}{3.155763in}}%
\pgfpathcurveto{\pgfqpoint{2.987559in}{3.147950in}}{\pgfqpoint{2.983169in}{3.137351in}}{\pgfqpoint{2.983169in}{3.126301in}}%
\pgfpathcurveto{\pgfqpoint{2.983169in}{3.115251in}}{\pgfqpoint{2.987559in}{3.104652in}}{\pgfqpoint{2.995373in}{3.096838in}}%
\pgfpathcurveto{\pgfqpoint{3.003187in}{3.089024in}}{\pgfqpoint{3.013786in}{3.084634in}}{\pgfqpoint{3.024836in}{3.084634in}}%
\pgfpathclose%
\pgfusepath{stroke,fill}%
\end{pgfscope}%
\begin{pgfscope}%
\pgfpathrectangle{\pgfqpoint{0.600000in}{0.600000in}}{\pgfqpoint{3.900000in}{3.900000in}}%
\pgfusepath{clip}%
\pgfsetbuttcap%
\pgfsetroundjoin%
\definecolor{currentfill}{rgb}{0.121569,0.466667,0.705882}%
\pgfsetfillcolor{currentfill}%
\pgfsetlinewidth{1.003750pt}%
\definecolor{currentstroke}{rgb}{0.121569,0.466667,0.705882}%
\pgfsetstrokecolor{currentstroke}%
\pgfsetdash{}{0pt}%
\pgfpathmoveto{\pgfqpoint{2.446062in}{2.568517in}}%
\pgfpathcurveto{\pgfqpoint{2.457112in}{2.568517in}}{\pgfqpoint{2.467712in}{2.572908in}}{\pgfqpoint{2.475525in}{2.580721in}}%
\pgfpathcurveto{\pgfqpoint{2.483339in}{2.588535in}}{\pgfqpoint{2.487729in}{2.599134in}}{\pgfqpoint{2.487729in}{2.610184in}}%
\pgfpathcurveto{\pgfqpoint{2.487729in}{2.621234in}}{\pgfqpoint{2.483339in}{2.631833in}}{\pgfqpoint{2.475525in}{2.639647in}}%
\pgfpathcurveto{\pgfqpoint{2.467712in}{2.647460in}}{\pgfqpoint{2.457112in}{2.651851in}}{\pgfqpoint{2.446062in}{2.651851in}}%
\pgfpathcurveto{\pgfqpoint{2.435012in}{2.651851in}}{\pgfqpoint{2.424413in}{2.647460in}}{\pgfqpoint{2.416600in}{2.639647in}}%
\pgfpathcurveto{\pgfqpoint{2.408786in}{2.631833in}}{\pgfqpoint{2.404396in}{2.621234in}}{\pgfqpoint{2.404396in}{2.610184in}}%
\pgfpathcurveto{\pgfqpoint{2.404396in}{2.599134in}}{\pgfqpoint{2.408786in}{2.588535in}}{\pgfqpoint{2.416600in}{2.580721in}}%
\pgfpathcurveto{\pgfqpoint{2.424413in}{2.572908in}}{\pgfqpoint{2.435012in}{2.568517in}}{\pgfqpoint{2.446062in}{2.568517in}}%
\pgfpathclose%
\pgfusepath{stroke,fill}%
\end{pgfscope}%
\begin{pgfscope}%
\pgfpathrectangle{\pgfqpoint{0.600000in}{0.600000in}}{\pgfqpoint{3.900000in}{3.900000in}}%
\pgfusepath{clip}%
\pgfsetbuttcap%
\pgfsetroundjoin%
\definecolor{currentfill}{rgb}{0.121569,0.466667,0.705882}%
\pgfsetfillcolor{currentfill}%
\pgfsetlinewidth{1.003750pt}%
\definecolor{currentstroke}{rgb}{0.121569,0.466667,0.705882}%
\pgfsetstrokecolor{currentstroke}%
\pgfsetdash{}{0pt}%
\pgfpathmoveto{\pgfqpoint{2.518153in}{2.648264in}}%
\pgfpathcurveto{\pgfqpoint{2.529203in}{2.648264in}}{\pgfqpoint{2.539802in}{2.652654in}}{\pgfqpoint{2.547616in}{2.660468in}}%
\pgfpathcurveto{\pgfqpoint{2.555429in}{2.668281in}}{\pgfqpoint{2.559820in}{2.678880in}}{\pgfqpoint{2.559820in}{2.689931in}}%
\pgfpathcurveto{\pgfqpoint{2.559820in}{2.700981in}}{\pgfqpoint{2.555429in}{2.711580in}}{\pgfqpoint{2.547616in}{2.719393in}}%
\pgfpathcurveto{\pgfqpoint{2.539802in}{2.727207in}}{\pgfqpoint{2.529203in}{2.731597in}}{\pgfqpoint{2.518153in}{2.731597in}}%
\pgfpathcurveto{\pgfqpoint{2.507103in}{2.731597in}}{\pgfqpoint{2.496504in}{2.727207in}}{\pgfqpoint{2.488690in}{2.719393in}}%
\pgfpathcurveto{\pgfqpoint{2.480877in}{2.711580in}}{\pgfqpoint{2.476486in}{2.700981in}}{\pgfqpoint{2.476486in}{2.689931in}}%
\pgfpathcurveto{\pgfqpoint{2.476486in}{2.678880in}}{\pgfqpoint{2.480877in}{2.668281in}}{\pgfqpoint{2.488690in}{2.660468in}}%
\pgfpathcurveto{\pgfqpoint{2.496504in}{2.652654in}}{\pgfqpoint{2.507103in}{2.648264in}}{\pgfqpoint{2.518153in}{2.648264in}}%
\pgfpathclose%
\pgfusepath{stroke,fill}%
\end{pgfscope}%
\begin{pgfscope}%
\pgfpathrectangle{\pgfqpoint{0.600000in}{0.600000in}}{\pgfqpoint{3.900000in}{3.900000in}}%
\pgfusepath{clip}%
\pgfsetbuttcap%
\pgfsetroundjoin%
\definecolor{currentfill}{rgb}{0.121569,0.466667,0.705882}%
\pgfsetfillcolor{currentfill}%
\pgfsetlinewidth{1.003750pt}%
\definecolor{currentstroke}{rgb}{0.121569,0.466667,0.705882}%
\pgfsetstrokecolor{currentstroke}%
\pgfsetdash{}{0pt}%
\pgfpathmoveto{\pgfqpoint{2.808753in}{1.226177in}}%
\pgfpathcurveto{\pgfqpoint{2.819804in}{1.226177in}}{\pgfqpoint{2.830403in}{1.230567in}}{\pgfqpoint{2.838216in}{1.238381in}}%
\pgfpathcurveto{\pgfqpoint{2.846030in}{1.246194in}}{\pgfqpoint{2.850420in}{1.256793in}}{\pgfqpoint{2.850420in}{1.267844in}}%
\pgfpathcurveto{\pgfqpoint{2.850420in}{1.278894in}}{\pgfqpoint{2.846030in}{1.289493in}}{\pgfqpoint{2.838216in}{1.297306in}}%
\pgfpathcurveto{\pgfqpoint{2.830403in}{1.305120in}}{\pgfqpoint{2.819804in}{1.309510in}}{\pgfqpoint{2.808753in}{1.309510in}}%
\pgfpathcurveto{\pgfqpoint{2.797703in}{1.309510in}}{\pgfqpoint{2.787104in}{1.305120in}}{\pgfqpoint{2.779291in}{1.297306in}}%
\pgfpathcurveto{\pgfqpoint{2.771477in}{1.289493in}}{\pgfqpoint{2.767087in}{1.278894in}}{\pgfqpoint{2.767087in}{1.267844in}}%
\pgfpathcurveto{\pgfqpoint{2.767087in}{1.256793in}}{\pgfqpoint{2.771477in}{1.246194in}}{\pgfqpoint{2.779291in}{1.238381in}}%
\pgfpathcurveto{\pgfqpoint{2.787104in}{1.230567in}}{\pgfqpoint{2.797703in}{1.226177in}}{\pgfqpoint{2.808753in}{1.226177in}}%
\pgfpathclose%
\pgfusepath{stroke,fill}%
\end{pgfscope}%
\begin{pgfscope}%
\pgfpathrectangle{\pgfqpoint{0.600000in}{0.600000in}}{\pgfqpoint{3.900000in}{3.900000in}}%
\pgfusepath{clip}%
\pgfsetbuttcap%
\pgfsetroundjoin%
\definecolor{currentfill}{rgb}{0.121569,0.466667,0.705882}%
\pgfsetfillcolor{currentfill}%
\pgfsetlinewidth{1.003750pt}%
\definecolor{currentstroke}{rgb}{0.121569,0.466667,0.705882}%
\pgfsetstrokecolor{currentstroke}%
\pgfsetdash{}{0pt}%
\pgfpathmoveto{\pgfqpoint{2.766669in}{0.700898in}}%
\pgfpathcurveto{\pgfqpoint{2.777720in}{0.700898in}}{\pgfqpoint{2.788319in}{0.705289in}}{\pgfqpoint{2.796132in}{0.713102in}}%
\pgfpathcurveto{\pgfqpoint{2.803946in}{0.720916in}}{\pgfqpoint{2.808336in}{0.731515in}}{\pgfqpoint{2.808336in}{0.742565in}}%
\pgfpathcurveto{\pgfqpoint{2.808336in}{0.753615in}}{\pgfqpoint{2.803946in}{0.764214in}}{\pgfqpoint{2.796132in}{0.772028in}}%
\pgfpathcurveto{\pgfqpoint{2.788319in}{0.779841in}}{\pgfqpoint{2.777720in}{0.784232in}}{\pgfqpoint{2.766669in}{0.784232in}}%
\pgfpathcurveto{\pgfqpoint{2.755619in}{0.784232in}}{\pgfqpoint{2.745020in}{0.779841in}}{\pgfqpoint{2.737207in}{0.772028in}}%
\pgfpathcurveto{\pgfqpoint{2.729393in}{0.764214in}}{\pgfqpoint{2.725003in}{0.753615in}}{\pgfqpoint{2.725003in}{0.742565in}}%
\pgfpathcurveto{\pgfqpoint{2.725003in}{0.731515in}}{\pgfqpoint{2.729393in}{0.720916in}}{\pgfqpoint{2.737207in}{0.713102in}}%
\pgfpathcurveto{\pgfqpoint{2.745020in}{0.705289in}}{\pgfqpoint{2.755619in}{0.700898in}}{\pgfqpoint{2.766669in}{0.700898in}}%
\pgfpathclose%
\pgfusepath{stroke,fill}%
\end{pgfscope}%
\begin{pgfscope}%
\pgfpathrectangle{\pgfqpoint{0.600000in}{0.600000in}}{\pgfqpoint{3.900000in}{3.900000in}}%
\pgfusepath{clip}%
\pgfsetbuttcap%
\pgfsetroundjoin%
\definecolor{currentfill}{rgb}{0.121569,0.466667,0.705882}%
\pgfsetfillcolor{currentfill}%
\pgfsetlinewidth{1.003750pt}%
\definecolor{currentstroke}{rgb}{0.121569,0.466667,0.705882}%
\pgfsetstrokecolor{currentstroke}%
\pgfsetdash{}{0pt}%
\pgfpathmoveto{\pgfqpoint{2.744669in}{2.764797in}}%
\pgfpathcurveto{\pgfqpoint{2.755719in}{2.764797in}}{\pgfqpoint{2.766318in}{2.769188in}}{\pgfqpoint{2.774132in}{2.777001in}}%
\pgfpathcurveto{\pgfqpoint{2.781946in}{2.784815in}}{\pgfqpoint{2.786336in}{2.795414in}}{\pgfqpoint{2.786336in}{2.806464in}}%
\pgfpathcurveto{\pgfqpoint{2.786336in}{2.817514in}}{\pgfqpoint{2.781946in}{2.828113in}}{\pgfqpoint{2.774132in}{2.835927in}}%
\pgfpathcurveto{\pgfqpoint{2.766318in}{2.843741in}}{\pgfqpoint{2.755719in}{2.848131in}}{\pgfqpoint{2.744669in}{2.848131in}}%
\pgfpathcurveto{\pgfqpoint{2.733619in}{2.848131in}}{\pgfqpoint{2.723020in}{2.843741in}}{\pgfqpoint{2.715206in}{2.835927in}}%
\pgfpathcurveto{\pgfqpoint{2.707393in}{2.828113in}}{\pgfqpoint{2.703003in}{2.817514in}}{\pgfqpoint{2.703003in}{2.806464in}}%
\pgfpathcurveto{\pgfqpoint{2.703003in}{2.795414in}}{\pgfqpoint{2.707393in}{2.784815in}}{\pgfqpoint{2.715206in}{2.777001in}}%
\pgfpathcurveto{\pgfqpoint{2.723020in}{2.769188in}}{\pgfqpoint{2.733619in}{2.764797in}}{\pgfqpoint{2.744669in}{2.764797in}}%
\pgfpathclose%
\pgfusepath{stroke,fill}%
\end{pgfscope}%
\begin{pgfscope}%
\pgfpathrectangle{\pgfqpoint{0.600000in}{0.600000in}}{\pgfqpoint{3.900000in}{3.900000in}}%
\pgfusepath{clip}%
\pgfsetbuttcap%
\pgfsetroundjoin%
\definecolor{currentfill}{rgb}{0.121569,0.466667,0.705882}%
\pgfsetfillcolor{currentfill}%
\pgfsetlinewidth{1.003750pt}%
\definecolor{currentstroke}{rgb}{0.121569,0.466667,0.705882}%
\pgfsetstrokecolor{currentstroke}%
\pgfsetdash{}{0pt}%
\pgfpathmoveto{\pgfqpoint{2.862212in}{2.931837in}}%
\pgfpathcurveto{\pgfqpoint{2.873262in}{2.931837in}}{\pgfqpoint{2.883861in}{2.936228in}}{\pgfqpoint{2.891675in}{2.944041in}}%
\pgfpathcurveto{\pgfqpoint{2.899488in}{2.951855in}}{\pgfqpoint{2.903878in}{2.962454in}}{\pgfqpoint{2.903878in}{2.973504in}}%
\pgfpathcurveto{\pgfqpoint{2.903878in}{2.984554in}}{\pgfqpoint{2.899488in}{2.995153in}}{\pgfqpoint{2.891675in}{3.002967in}}%
\pgfpathcurveto{\pgfqpoint{2.883861in}{3.010780in}}{\pgfqpoint{2.873262in}{3.015171in}}{\pgfqpoint{2.862212in}{3.015171in}}%
\pgfpathcurveto{\pgfqpoint{2.851162in}{3.015171in}}{\pgfqpoint{2.840563in}{3.010780in}}{\pgfqpoint{2.832749in}{3.002967in}}%
\pgfpathcurveto{\pgfqpoint{2.824935in}{2.995153in}}{\pgfqpoint{2.820545in}{2.984554in}}{\pgfqpoint{2.820545in}{2.973504in}}%
\pgfpathcurveto{\pgfqpoint{2.820545in}{2.962454in}}{\pgfqpoint{2.824935in}{2.951855in}}{\pgfqpoint{2.832749in}{2.944041in}}%
\pgfpathcurveto{\pgfqpoint{2.840563in}{2.936228in}}{\pgfqpoint{2.851162in}{2.931837in}}{\pgfqpoint{2.862212in}{2.931837in}}%
\pgfpathclose%
\pgfusepath{stroke,fill}%
\end{pgfscope}%
\begin{pgfscope}%
\pgfpathrectangle{\pgfqpoint{0.600000in}{0.600000in}}{\pgfqpoint{3.900000in}{3.900000in}}%
\pgfusepath{clip}%
\pgfsetbuttcap%
\pgfsetroundjoin%
\definecolor{currentfill}{rgb}{0.121569,0.466667,0.705882}%
\pgfsetfillcolor{currentfill}%
\pgfsetlinewidth{1.003750pt}%
\definecolor{currentstroke}{rgb}{0.121569,0.466667,0.705882}%
\pgfsetstrokecolor{currentstroke}%
\pgfsetdash{}{0pt}%
\pgfpathmoveto{\pgfqpoint{3.400722in}{1.111478in}}%
\pgfpathcurveto{\pgfqpoint{3.411772in}{1.111478in}}{\pgfqpoint{3.422371in}{1.115868in}}{\pgfqpoint{3.430185in}{1.123682in}}%
\pgfpathcurveto{\pgfqpoint{3.437998in}{1.131496in}}{\pgfqpoint{3.442389in}{1.142095in}}{\pgfqpoint{3.442389in}{1.153145in}}%
\pgfpathcurveto{\pgfqpoint{3.442389in}{1.164195in}}{\pgfqpoint{3.437998in}{1.174794in}}{\pgfqpoint{3.430185in}{1.182608in}}%
\pgfpathcurveto{\pgfqpoint{3.422371in}{1.190421in}}{\pgfqpoint{3.411772in}{1.194811in}}{\pgfqpoint{3.400722in}{1.194811in}}%
\pgfpathcurveto{\pgfqpoint{3.389672in}{1.194811in}}{\pgfqpoint{3.379073in}{1.190421in}}{\pgfqpoint{3.371259in}{1.182608in}}%
\pgfpathcurveto{\pgfqpoint{3.363446in}{1.174794in}}{\pgfqpoint{3.359055in}{1.164195in}}{\pgfqpoint{3.359055in}{1.153145in}}%
\pgfpathcurveto{\pgfqpoint{3.359055in}{1.142095in}}{\pgfqpoint{3.363446in}{1.131496in}}{\pgfqpoint{3.371259in}{1.123682in}}%
\pgfpathcurveto{\pgfqpoint{3.379073in}{1.115868in}}{\pgfqpoint{3.389672in}{1.111478in}}{\pgfqpoint{3.400722in}{1.111478in}}%
\pgfpathclose%
\pgfusepath{stroke,fill}%
\end{pgfscope}%
\begin{pgfscope}%
\pgfpathrectangle{\pgfqpoint{0.600000in}{0.600000in}}{\pgfqpoint{3.900000in}{3.900000in}}%
\pgfusepath{clip}%
\pgfsetbuttcap%
\pgfsetroundjoin%
\definecolor{currentfill}{rgb}{0.121569,0.466667,0.705882}%
\pgfsetfillcolor{currentfill}%
\pgfsetlinewidth{1.003750pt}%
\definecolor{currentstroke}{rgb}{0.121569,0.466667,0.705882}%
\pgfsetstrokecolor{currentstroke}%
\pgfsetdash{}{0pt}%
\pgfpathmoveto{\pgfqpoint{2.256593in}{3.217093in}}%
\pgfpathcurveto{\pgfqpoint{2.267643in}{3.217093in}}{\pgfqpoint{2.278242in}{3.221483in}}{\pgfqpoint{2.286056in}{3.229297in}}%
\pgfpathcurveto{\pgfqpoint{2.293869in}{3.237110in}}{\pgfqpoint{2.298260in}{3.247709in}}{\pgfqpoint{2.298260in}{3.258759in}}%
\pgfpathcurveto{\pgfqpoint{2.298260in}{3.269810in}}{\pgfqpoint{2.293869in}{3.280409in}}{\pgfqpoint{2.286056in}{3.288222in}}%
\pgfpathcurveto{\pgfqpoint{2.278242in}{3.296036in}}{\pgfqpoint{2.267643in}{3.300426in}}{\pgfqpoint{2.256593in}{3.300426in}}%
\pgfpathcurveto{\pgfqpoint{2.245543in}{3.300426in}}{\pgfqpoint{2.234944in}{3.296036in}}{\pgfqpoint{2.227130in}{3.288222in}}%
\pgfpathcurveto{\pgfqpoint{2.219316in}{3.280409in}}{\pgfqpoint{2.214926in}{3.269810in}}{\pgfqpoint{2.214926in}{3.258759in}}%
\pgfpathcurveto{\pgfqpoint{2.214926in}{3.247709in}}{\pgfqpoint{2.219316in}{3.237110in}}{\pgfqpoint{2.227130in}{3.229297in}}%
\pgfpathcurveto{\pgfqpoint{2.234944in}{3.221483in}}{\pgfqpoint{2.245543in}{3.217093in}}{\pgfqpoint{2.256593in}{3.217093in}}%
\pgfpathclose%
\pgfusepath{stroke,fill}%
\end{pgfscope}%
\begin{pgfscope}%
\pgfpathrectangle{\pgfqpoint{0.600000in}{0.600000in}}{\pgfqpoint{3.900000in}{3.900000in}}%
\pgfusepath{clip}%
\pgfsetbuttcap%
\pgfsetroundjoin%
\definecolor{currentfill}{rgb}{0.121569,0.466667,0.705882}%
\pgfsetfillcolor{currentfill}%
\pgfsetlinewidth{1.003750pt}%
\definecolor{currentstroke}{rgb}{0.121569,0.466667,0.705882}%
\pgfsetstrokecolor{currentstroke}%
\pgfsetdash{}{0pt}%
\pgfpathmoveto{\pgfqpoint{2.482460in}{3.088221in}}%
\pgfpathcurveto{\pgfqpoint{2.493511in}{3.088221in}}{\pgfqpoint{2.504110in}{3.092611in}}{\pgfqpoint{2.511923in}{3.100425in}}%
\pgfpathcurveto{\pgfqpoint{2.519737in}{3.108239in}}{\pgfqpoint{2.524127in}{3.118838in}}{\pgfqpoint{2.524127in}{3.129888in}}%
\pgfpathcurveto{\pgfqpoint{2.524127in}{3.140938in}}{\pgfqpoint{2.519737in}{3.151537in}}{\pgfqpoint{2.511923in}{3.159351in}}%
\pgfpathcurveto{\pgfqpoint{2.504110in}{3.167164in}}{\pgfqpoint{2.493511in}{3.171555in}}{\pgfqpoint{2.482460in}{3.171555in}}%
\pgfpathcurveto{\pgfqpoint{2.471410in}{3.171555in}}{\pgfqpoint{2.460811in}{3.167164in}}{\pgfqpoint{2.452998in}{3.159351in}}%
\pgfpathcurveto{\pgfqpoint{2.445184in}{3.151537in}}{\pgfqpoint{2.440794in}{3.140938in}}{\pgfqpoint{2.440794in}{3.129888in}}%
\pgfpathcurveto{\pgfqpoint{2.440794in}{3.118838in}}{\pgfqpoint{2.445184in}{3.108239in}}{\pgfqpoint{2.452998in}{3.100425in}}%
\pgfpathcurveto{\pgfqpoint{2.460811in}{3.092611in}}{\pgfqpoint{2.471410in}{3.088221in}}{\pgfqpoint{2.482460in}{3.088221in}}%
\pgfpathclose%
\pgfusepath{stroke,fill}%
\end{pgfscope}%
\begin{pgfscope}%
\pgfpathrectangle{\pgfqpoint{0.600000in}{0.600000in}}{\pgfqpoint{3.900000in}{3.900000in}}%
\pgfusepath{clip}%
\pgfsetbuttcap%
\pgfsetroundjoin%
\definecolor{currentfill}{rgb}{0.121569,0.466667,0.705882}%
\pgfsetfillcolor{currentfill}%
\pgfsetlinewidth{1.003750pt}%
\definecolor{currentstroke}{rgb}{0.121569,0.466667,0.705882}%
\pgfsetstrokecolor{currentstroke}%
\pgfsetdash{}{0pt}%
\pgfpathmoveto{\pgfqpoint{2.200407in}{1.429614in}}%
\pgfpathcurveto{\pgfqpoint{2.211457in}{1.429614in}}{\pgfqpoint{2.222056in}{1.434004in}}{\pgfqpoint{2.229870in}{1.441818in}}%
\pgfpathcurveto{\pgfqpoint{2.237684in}{1.449632in}}{\pgfqpoint{2.242074in}{1.460231in}}{\pgfqpoint{2.242074in}{1.471281in}}%
\pgfpathcurveto{\pgfqpoint{2.242074in}{1.482331in}}{\pgfqpoint{2.237684in}{1.492930in}}{\pgfqpoint{2.229870in}{1.500744in}}%
\pgfpathcurveto{\pgfqpoint{2.222056in}{1.508557in}}{\pgfqpoint{2.211457in}{1.512948in}}{\pgfqpoint{2.200407in}{1.512948in}}%
\pgfpathcurveto{\pgfqpoint{2.189357in}{1.512948in}}{\pgfqpoint{2.178758in}{1.508557in}}{\pgfqpoint{2.170945in}{1.500744in}}%
\pgfpathcurveto{\pgfqpoint{2.163131in}{1.492930in}}{\pgfqpoint{2.158741in}{1.482331in}}{\pgfqpoint{2.158741in}{1.471281in}}%
\pgfpathcurveto{\pgfqpoint{2.158741in}{1.460231in}}{\pgfqpoint{2.163131in}{1.449632in}}{\pgfqpoint{2.170945in}{1.441818in}}%
\pgfpathcurveto{\pgfqpoint{2.178758in}{1.434004in}}{\pgfqpoint{2.189357in}{1.429614in}}{\pgfqpoint{2.200407in}{1.429614in}}%
\pgfpathclose%
\pgfusepath{stroke,fill}%
\end{pgfscope}%
\begin{pgfscope}%
\pgfpathrectangle{\pgfqpoint{0.600000in}{0.600000in}}{\pgfqpoint{3.900000in}{3.900000in}}%
\pgfusepath{clip}%
\pgfsetbuttcap%
\pgfsetroundjoin%
\definecolor{currentfill}{rgb}{0.121569,0.466667,0.705882}%
\pgfsetfillcolor{currentfill}%
\pgfsetlinewidth{1.003750pt}%
\definecolor{currentstroke}{rgb}{0.121569,0.466667,0.705882}%
\pgfsetstrokecolor{currentstroke}%
\pgfsetdash{}{0pt}%
\pgfpathmoveto{\pgfqpoint{2.402964in}{2.347725in}}%
\pgfpathcurveto{\pgfqpoint{2.414014in}{2.347725in}}{\pgfqpoint{2.424613in}{2.352115in}}{\pgfqpoint{2.432427in}{2.359928in}}%
\pgfpathcurveto{\pgfqpoint{2.440241in}{2.367742in}}{\pgfqpoint{2.444631in}{2.378341in}}{\pgfqpoint{2.444631in}{2.389391in}}%
\pgfpathcurveto{\pgfqpoint{2.444631in}{2.400441in}}{\pgfqpoint{2.440241in}{2.411040in}}{\pgfqpoint{2.432427in}{2.418854in}}%
\pgfpathcurveto{\pgfqpoint{2.424613in}{2.426668in}}{\pgfqpoint{2.414014in}{2.431058in}}{\pgfqpoint{2.402964in}{2.431058in}}%
\pgfpathcurveto{\pgfqpoint{2.391914in}{2.431058in}}{\pgfqpoint{2.381315in}{2.426668in}}{\pgfqpoint{2.373502in}{2.418854in}}%
\pgfpathcurveto{\pgfqpoint{2.365688in}{2.411040in}}{\pgfqpoint{2.361298in}{2.400441in}}{\pgfqpoint{2.361298in}{2.389391in}}%
\pgfpathcurveto{\pgfqpoint{2.361298in}{2.378341in}}{\pgfqpoint{2.365688in}{2.367742in}}{\pgfqpoint{2.373502in}{2.359928in}}%
\pgfpathcurveto{\pgfqpoint{2.381315in}{2.352115in}}{\pgfqpoint{2.391914in}{2.347725in}}{\pgfqpoint{2.402964in}{2.347725in}}%
\pgfpathclose%
\pgfusepath{stroke,fill}%
\end{pgfscope}%
\begin{pgfscope}%
\pgfpathrectangle{\pgfqpoint{0.600000in}{0.600000in}}{\pgfqpoint{3.900000in}{3.900000in}}%
\pgfusepath{clip}%
\pgfsetbuttcap%
\pgfsetroundjoin%
\definecolor{currentfill}{rgb}{0.121569,0.466667,0.705882}%
\pgfsetfillcolor{currentfill}%
\pgfsetlinewidth{1.003750pt}%
\definecolor{currentstroke}{rgb}{0.121569,0.466667,0.705882}%
\pgfsetstrokecolor{currentstroke}%
\pgfsetdash{}{0pt}%
\pgfpathmoveto{\pgfqpoint{1.780297in}{2.712284in}}%
\pgfpathcurveto{\pgfqpoint{1.791348in}{2.712284in}}{\pgfqpoint{1.801947in}{2.716674in}}{\pgfqpoint{1.809760in}{2.724488in}}%
\pgfpathcurveto{\pgfqpoint{1.817574in}{2.732301in}}{\pgfqpoint{1.821964in}{2.742900in}}{\pgfqpoint{1.821964in}{2.753950in}}%
\pgfpathcurveto{\pgfqpoint{1.821964in}{2.765001in}}{\pgfqpoint{1.817574in}{2.775600in}}{\pgfqpoint{1.809760in}{2.783413in}}%
\pgfpathcurveto{\pgfqpoint{1.801947in}{2.791227in}}{\pgfqpoint{1.791348in}{2.795617in}}{\pgfqpoint{1.780297in}{2.795617in}}%
\pgfpathcurveto{\pgfqpoint{1.769247in}{2.795617in}}{\pgfqpoint{1.758648in}{2.791227in}}{\pgfqpoint{1.750835in}{2.783413in}}%
\pgfpathcurveto{\pgfqpoint{1.743021in}{2.775600in}}{\pgfqpoint{1.738631in}{2.765001in}}{\pgfqpoint{1.738631in}{2.753950in}}%
\pgfpathcurveto{\pgfqpoint{1.738631in}{2.742900in}}{\pgfqpoint{1.743021in}{2.732301in}}{\pgfqpoint{1.750835in}{2.724488in}}%
\pgfpathcurveto{\pgfqpoint{1.758648in}{2.716674in}}{\pgfqpoint{1.769247in}{2.712284in}}{\pgfqpoint{1.780297in}{2.712284in}}%
\pgfpathclose%
\pgfusepath{stroke,fill}%
\end{pgfscope}%
\begin{pgfscope}%
\pgfpathrectangle{\pgfqpoint{0.600000in}{0.600000in}}{\pgfqpoint{3.900000in}{3.900000in}}%
\pgfusepath{clip}%
\pgfsetbuttcap%
\pgfsetroundjoin%
\definecolor{currentfill}{rgb}{0.121569,0.466667,0.705882}%
\pgfsetfillcolor{currentfill}%
\pgfsetlinewidth{1.003750pt}%
\definecolor{currentstroke}{rgb}{0.121569,0.466667,0.705882}%
\pgfsetstrokecolor{currentstroke}%
\pgfsetdash{}{0pt}%
\pgfpathmoveto{\pgfqpoint{2.529412in}{3.325821in}}%
\pgfpathcurveto{\pgfqpoint{2.540462in}{3.325821in}}{\pgfqpoint{2.551061in}{3.330211in}}{\pgfqpoint{2.558874in}{3.338025in}}%
\pgfpathcurveto{\pgfqpoint{2.566688in}{3.345838in}}{\pgfqpoint{2.571078in}{3.356437in}}{\pgfqpoint{2.571078in}{3.367487in}}%
\pgfpathcurveto{\pgfqpoint{2.571078in}{3.378537in}}{\pgfqpoint{2.566688in}{3.389137in}}{\pgfqpoint{2.558874in}{3.396950in}}%
\pgfpathcurveto{\pgfqpoint{2.551061in}{3.404764in}}{\pgfqpoint{2.540462in}{3.409154in}}{\pgfqpoint{2.529412in}{3.409154in}}%
\pgfpathcurveto{\pgfqpoint{2.518362in}{3.409154in}}{\pgfqpoint{2.507762in}{3.404764in}}{\pgfqpoint{2.499949in}{3.396950in}}%
\pgfpathcurveto{\pgfqpoint{2.492135in}{3.389137in}}{\pgfqpoint{2.487745in}{3.378537in}}{\pgfqpoint{2.487745in}{3.367487in}}%
\pgfpathcurveto{\pgfqpoint{2.487745in}{3.356437in}}{\pgfqpoint{2.492135in}{3.345838in}}{\pgfqpoint{2.499949in}{3.338025in}}%
\pgfpathcurveto{\pgfqpoint{2.507762in}{3.330211in}}{\pgfqpoint{2.518362in}{3.325821in}}{\pgfqpoint{2.529412in}{3.325821in}}%
\pgfpathclose%
\pgfusepath{stroke,fill}%
\end{pgfscope}%
\begin{pgfscope}%
\pgfpathrectangle{\pgfqpoint{0.600000in}{0.600000in}}{\pgfqpoint{3.900000in}{3.900000in}}%
\pgfusepath{clip}%
\pgfsetbuttcap%
\pgfsetroundjoin%
\definecolor{currentfill}{rgb}{0.121569,0.466667,0.705882}%
\pgfsetfillcolor{currentfill}%
\pgfsetlinewidth{1.003750pt}%
\definecolor{currentstroke}{rgb}{0.121569,0.466667,0.705882}%
\pgfsetstrokecolor{currentstroke}%
\pgfsetdash{}{0pt}%
\pgfpathmoveto{\pgfqpoint{2.300901in}{3.106153in}}%
\pgfpathcurveto{\pgfqpoint{2.311952in}{3.106153in}}{\pgfqpoint{2.322551in}{3.110543in}}{\pgfqpoint{2.330364in}{3.118357in}}%
\pgfpathcurveto{\pgfqpoint{2.338178in}{3.126171in}}{\pgfqpoint{2.342568in}{3.136770in}}{\pgfqpoint{2.342568in}{3.147820in}}%
\pgfpathcurveto{\pgfqpoint{2.342568in}{3.158870in}}{\pgfqpoint{2.338178in}{3.169469in}}{\pgfqpoint{2.330364in}{3.177282in}}%
\pgfpathcurveto{\pgfqpoint{2.322551in}{3.185096in}}{\pgfqpoint{2.311952in}{3.189486in}}{\pgfqpoint{2.300901in}{3.189486in}}%
\pgfpathcurveto{\pgfqpoint{2.289851in}{3.189486in}}{\pgfqpoint{2.279252in}{3.185096in}}{\pgfqpoint{2.271439in}{3.177282in}}%
\pgfpathcurveto{\pgfqpoint{2.263625in}{3.169469in}}{\pgfqpoint{2.259235in}{3.158870in}}{\pgfqpoint{2.259235in}{3.147820in}}%
\pgfpathcurveto{\pgfqpoint{2.259235in}{3.136770in}}{\pgfqpoint{2.263625in}{3.126171in}}{\pgfqpoint{2.271439in}{3.118357in}}%
\pgfpathcurveto{\pgfqpoint{2.279252in}{3.110543in}}{\pgfqpoint{2.289851in}{3.106153in}}{\pgfqpoint{2.300901in}{3.106153in}}%
\pgfpathclose%
\pgfusepath{stroke,fill}%
\end{pgfscope}%
\begin{pgfscope}%
\pgfpathrectangle{\pgfqpoint{0.600000in}{0.600000in}}{\pgfqpoint{3.900000in}{3.900000in}}%
\pgfusepath{clip}%
\pgfsetbuttcap%
\pgfsetroundjoin%
\definecolor{currentfill}{rgb}{0.121569,0.466667,0.705882}%
\pgfsetfillcolor{currentfill}%
\pgfsetlinewidth{1.003750pt}%
\definecolor{currentstroke}{rgb}{0.121569,0.466667,0.705882}%
\pgfsetstrokecolor{currentstroke}%
\pgfsetdash{}{0pt}%
\pgfpathmoveto{\pgfqpoint{3.038394in}{2.969006in}}%
\pgfpathcurveto{\pgfqpoint{3.049444in}{2.969006in}}{\pgfqpoint{3.060043in}{2.973396in}}{\pgfqpoint{3.067857in}{2.981210in}}%
\pgfpathcurveto{\pgfqpoint{3.075670in}{2.989023in}}{\pgfqpoint{3.080061in}{2.999622in}}{\pgfqpoint{3.080061in}{3.010672in}}%
\pgfpathcurveto{\pgfqpoint{3.080061in}{3.021723in}}{\pgfqpoint{3.075670in}{3.032322in}}{\pgfqpoint{3.067857in}{3.040135in}}%
\pgfpathcurveto{\pgfqpoint{3.060043in}{3.047949in}}{\pgfqpoint{3.049444in}{3.052339in}}{\pgfqpoint{3.038394in}{3.052339in}}%
\pgfpathcurveto{\pgfqpoint{3.027344in}{3.052339in}}{\pgfqpoint{3.016745in}{3.047949in}}{\pgfqpoint{3.008931in}{3.040135in}}%
\pgfpathcurveto{\pgfqpoint{3.001118in}{3.032322in}}{\pgfqpoint{2.996727in}{3.021723in}}{\pgfqpoint{2.996727in}{3.010672in}}%
\pgfpathcurveto{\pgfqpoint{2.996727in}{2.999622in}}{\pgfqpoint{3.001118in}{2.989023in}}{\pgfqpoint{3.008931in}{2.981210in}}%
\pgfpathcurveto{\pgfqpoint{3.016745in}{2.973396in}}{\pgfqpoint{3.027344in}{2.969006in}}{\pgfqpoint{3.038394in}{2.969006in}}%
\pgfpathclose%
\pgfusepath{stroke,fill}%
\end{pgfscope}%
\begin{pgfscope}%
\pgfpathrectangle{\pgfqpoint{0.600000in}{0.600000in}}{\pgfqpoint{3.900000in}{3.900000in}}%
\pgfusepath{clip}%
\pgfsetbuttcap%
\pgfsetroundjoin%
\definecolor{currentfill}{rgb}{0.121569,0.466667,0.705882}%
\pgfsetfillcolor{currentfill}%
\pgfsetlinewidth{1.003750pt}%
\definecolor{currentstroke}{rgb}{0.121569,0.466667,0.705882}%
\pgfsetstrokecolor{currentstroke}%
\pgfsetdash{}{0pt}%
\pgfpathmoveto{\pgfqpoint{2.432912in}{2.069833in}}%
\pgfpathcurveto{\pgfqpoint{2.443962in}{2.069833in}}{\pgfqpoint{2.454561in}{2.074223in}}{\pgfqpoint{2.462375in}{2.082037in}}%
\pgfpathcurveto{\pgfqpoint{2.470188in}{2.089851in}}{\pgfqpoint{2.474579in}{2.100450in}}{\pgfqpoint{2.474579in}{2.111500in}}%
\pgfpathcurveto{\pgfqpoint{2.474579in}{2.122550in}}{\pgfqpoint{2.470188in}{2.133149in}}{\pgfqpoint{2.462375in}{2.140962in}}%
\pgfpathcurveto{\pgfqpoint{2.454561in}{2.148776in}}{\pgfqpoint{2.443962in}{2.153166in}}{\pgfqpoint{2.432912in}{2.153166in}}%
\pgfpathcurveto{\pgfqpoint{2.421862in}{2.153166in}}{\pgfqpoint{2.411263in}{2.148776in}}{\pgfqpoint{2.403449in}{2.140962in}}%
\pgfpathcurveto{\pgfqpoint{2.395636in}{2.133149in}}{\pgfqpoint{2.391245in}{2.122550in}}{\pgfqpoint{2.391245in}{2.111500in}}%
\pgfpathcurveto{\pgfqpoint{2.391245in}{2.100450in}}{\pgfqpoint{2.395636in}{2.089851in}}{\pgfqpoint{2.403449in}{2.082037in}}%
\pgfpathcurveto{\pgfqpoint{2.411263in}{2.074223in}}{\pgfqpoint{2.421862in}{2.069833in}}{\pgfqpoint{2.432912in}{2.069833in}}%
\pgfpathclose%
\pgfusepath{stroke,fill}%
\end{pgfscope}%
\begin{pgfscope}%
\pgfpathrectangle{\pgfqpoint{0.600000in}{0.600000in}}{\pgfqpoint{3.900000in}{3.900000in}}%
\pgfusepath{clip}%
\pgfsetbuttcap%
\pgfsetroundjoin%
\definecolor{currentfill}{rgb}{0.121569,0.466667,0.705882}%
\pgfsetfillcolor{currentfill}%
\pgfsetlinewidth{1.003750pt}%
\definecolor{currentstroke}{rgb}{0.121569,0.466667,0.705882}%
\pgfsetstrokecolor{currentstroke}%
\pgfsetdash{}{0pt}%
\pgfpathmoveto{\pgfqpoint{2.310007in}{2.290941in}}%
\pgfpathcurveto{\pgfqpoint{2.321057in}{2.290941in}}{\pgfqpoint{2.331657in}{2.295332in}}{\pgfqpoint{2.339470in}{2.303145in}}%
\pgfpathcurveto{\pgfqpoint{2.347284in}{2.310959in}}{\pgfqpoint{2.351674in}{2.321558in}}{\pgfqpoint{2.351674in}{2.332608in}}%
\pgfpathcurveto{\pgfqpoint{2.351674in}{2.343658in}}{\pgfqpoint{2.347284in}{2.354257in}}{\pgfqpoint{2.339470in}{2.362071in}}%
\pgfpathcurveto{\pgfqpoint{2.331657in}{2.369884in}}{\pgfqpoint{2.321057in}{2.374275in}}{\pgfqpoint{2.310007in}{2.374275in}}%
\pgfpathcurveto{\pgfqpoint{2.298957in}{2.374275in}}{\pgfqpoint{2.288358in}{2.369884in}}{\pgfqpoint{2.280545in}{2.362071in}}%
\pgfpathcurveto{\pgfqpoint{2.272731in}{2.354257in}}{\pgfqpoint{2.268341in}{2.343658in}}{\pgfqpoint{2.268341in}{2.332608in}}%
\pgfpathcurveto{\pgfqpoint{2.268341in}{2.321558in}}{\pgfqpoint{2.272731in}{2.310959in}}{\pgfqpoint{2.280545in}{2.303145in}}%
\pgfpathcurveto{\pgfqpoint{2.288358in}{2.295332in}}{\pgfqpoint{2.298957in}{2.290941in}}{\pgfqpoint{2.310007in}{2.290941in}}%
\pgfpathclose%
\pgfusepath{stroke,fill}%
\end{pgfscope}%
\begin{pgfscope}%
\pgfpathrectangle{\pgfqpoint{0.600000in}{0.600000in}}{\pgfqpoint{3.900000in}{3.900000in}}%
\pgfusepath{clip}%
\pgfsetbuttcap%
\pgfsetroundjoin%
\definecolor{currentfill}{rgb}{0.121569,0.466667,0.705882}%
\pgfsetfillcolor{currentfill}%
\pgfsetlinewidth{1.003750pt}%
\definecolor{currentstroke}{rgb}{0.121569,0.466667,0.705882}%
\pgfsetstrokecolor{currentstroke}%
\pgfsetdash{}{0pt}%
\pgfpathmoveto{\pgfqpoint{2.735809in}{2.885062in}}%
\pgfpathcurveto{\pgfqpoint{2.746859in}{2.885062in}}{\pgfqpoint{2.757458in}{2.889453in}}{\pgfqpoint{2.765272in}{2.897266in}}%
\pgfpathcurveto{\pgfqpoint{2.773085in}{2.905080in}}{\pgfqpoint{2.777476in}{2.915679in}}{\pgfqpoint{2.777476in}{2.926729in}}%
\pgfpathcurveto{\pgfqpoint{2.777476in}{2.937779in}}{\pgfqpoint{2.773085in}{2.948378in}}{\pgfqpoint{2.765272in}{2.956192in}}%
\pgfpathcurveto{\pgfqpoint{2.757458in}{2.964005in}}{\pgfqpoint{2.746859in}{2.968396in}}{\pgfqpoint{2.735809in}{2.968396in}}%
\pgfpathcurveto{\pgfqpoint{2.724759in}{2.968396in}}{\pgfqpoint{2.714160in}{2.964005in}}{\pgfqpoint{2.706346in}{2.956192in}}%
\pgfpathcurveto{\pgfqpoint{2.698533in}{2.948378in}}{\pgfqpoint{2.694142in}{2.937779in}}{\pgfqpoint{2.694142in}{2.926729in}}%
\pgfpathcurveto{\pgfqpoint{2.694142in}{2.915679in}}{\pgfqpoint{2.698533in}{2.905080in}}{\pgfqpoint{2.706346in}{2.897266in}}%
\pgfpathcurveto{\pgfqpoint{2.714160in}{2.889453in}}{\pgfqpoint{2.724759in}{2.885062in}}{\pgfqpoint{2.735809in}{2.885062in}}%
\pgfpathclose%
\pgfusepath{stroke,fill}%
\end{pgfscope}%
\begin{pgfscope}%
\pgfpathrectangle{\pgfqpoint{0.600000in}{0.600000in}}{\pgfqpoint{3.900000in}{3.900000in}}%
\pgfusepath{clip}%
\pgfsetbuttcap%
\pgfsetroundjoin%
\definecolor{currentfill}{rgb}{0.121569,0.466667,0.705882}%
\pgfsetfillcolor{currentfill}%
\pgfsetlinewidth{1.003750pt}%
\definecolor{currentstroke}{rgb}{0.121569,0.466667,0.705882}%
\pgfsetstrokecolor{currentstroke}%
\pgfsetdash{}{0pt}%
\pgfpathmoveto{\pgfqpoint{2.808046in}{2.886783in}}%
\pgfpathcurveto{\pgfqpoint{2.819096in}{2.886783in}}{\pgfqpoint{2.829695in}{2.891174in}}{\pgfqpoint{2.837509in}{2.898987in}}%
\pgfpathcurveto{\pgfqpoint{2.845323in}{2.906801in}}{\pgfqpoint{2.849713in}{2.917400in}}{\pgfqpoint{2.849713in}{2.928450in}}%
\pgfpathcurveto{\pgfqpoint{2.849713in}{2.939500in}}{\pgfqpoint{2.845323in}{2.950099in}}{\pgfqpoint{2.837509in}{2.957913in}}%
\pgfpathcurveto{\pgfqpoint{2.829695in}{2.965726in}}{\pgfqpoint{2.819096in}{2.970117in}}{\pgfqpoint{2.808046in}{2.970117in}}%
\pgfpathcurveto{\pgfqpoint{2.796996in}{2.970117in}}{\pgfqpoint{2.786397in}{2.965726in}}{\pgfqpoint{2.778583in}{2.957913in}}%
\pgfpathcurveto{\pgfqpoint{2.770770in}{2.950099in}}{\pgfqpoint{2.766380in}{2.939500in}}{\pgfqpoint{2.766380in}{2.928450in}}%
\pgfpathcurveto{\pgfqpoint{2.766380in}{2.917400in}}{\pgfqpoint{2.770770in}{2.906801in}}{\pgfqpoint{2.778583in}{2.898987in}}%
\pgfpathcurveto{\pgfqpoint{2.786397in}{2.891174in}}{\pgfqpoint{2.796996in}{2.886783in}}{\pgfqpoint{2.808046in}{2.886783in}}%
\pgfpathclose%
\pgfusepath{stroke,fill}%
\end{pgfscope}%
\begin{pgfscope}%
\pgfpathrectangle{\pgfqpoint{0.600000in}{0.600000in}}{\pgfqpoint{3.900000in}{3.900000in}}%
\pgfusepath{clip}%
\pgfsetbuttcap%
\pgfsetroundjoin%
\definecolor{currentfill}{rgb}{0.121569,0.466667,0.705882}%
\pgfsetfillcolor{currentfill}%
\pgfsetlinewidth{1.003750pt}%
\definecolor{currentstroke}{rgb}{0.121569,0.466667,0.705882}%
\pgfsetstrokecolor{currentstroke}%
\pgfsetdash{}{0pt}%
\pgfpathmoveto{\pgfqpoint{2.861003in}{3.055992in}}%
\pgfpathcurveto{\pgfqpoint{2.872053in}{3.055992in}}{\pgfqpoint{2.882652in}{3.060383in}}{\pgfqpoint{2.890465in}{3.068196in}}%
\pgfpathcurveto{\pgfqpoint{2.898279in}{3.076010in}}{\pgfqpoint{2.902669in}{3.086609in}}{\pgfqpoint{2.902669in}{3.097659in}}%
\pgfpathcurveto{\pgfqpoint{2.902669in}{3.108709in}}{\pgfqpoint{2.898279in}{3.119308in}}{\pgfqpoint{2.890465in}{3.127122in}}%
\pgfpathcurveto{\pgfqpoint{2.882652in}{3.134935in}}{\pgfqpoint{2.872053in}{3.139326in}}{\pgfqpoint{2.861003in}{3.139326in}}%
\pgfpathcurveto{\pgfqpoint{2.849952in}{3.139326in}}{\pgfqpoint{2.839353in}{3.134935in}}{\pgfqpoint{2.831540in}{3.127122in}}%
\pgfpathcurveto{\pgfqpoint{2.823726in}{3.119308in}}{\pgfqpoint{2.819336in}{3.108709in}}{\pgfqpoint{2.819336in}{3.097659in}}%
\pgfpathcurveto{\pgfqpoint{2.819336in}{3.086609in}}{\pgfqpoint{2.823726in}{3.076010in}}{\pgfqpoint{2.831540in}{3.068196in}}%
\pgfpathcurveto{\pgfqpoint{2.839353in}{3.060383in}}{\pgfqpoint{2.849952in}{3.055992in}}{\pgfqpoint{2.861003in}{3.055992in}}%
\pgfpathclose%
\pgfusepath{stroke,fill}%
\end{pgfscope}%
\begin{pgfscope}%
\pgfpathrectangle{\pgfqpoint{0.600000in}{0.600000in}}{\pgfqpoint{3.900000in}{3.900000in}}%
\pgfusepath{clip}%
\pgfsetbuttcap%
\pgfsetroundjoin%
\definecolor{currentfill}{rgb}{0.121569,0.466667,0.705882}%
\pgfsetfillcolor{currentfill}%
\pgfsetlinewidth{1.003750pt}%
\definecolor{currentstroke}{rgb}{0.121569,0.466667,0.705882}%
\pgfsetstrokecolor{currentstroke}%
\pgfsetdash{}{0pt}%
\pgfpathmoveto{\pgfqpoint{2.510271in}{3.179938in}}%
\pgfpathcurveto{\pgfqpoint{2.521322in}{3.179938in}}{\pgfqpoint{2.531921in}{3.184328in}}{\pgfqpoint{2.539734in}{3.192142in}}%
\pgfpathcurveto{\pgfqpoint{2.547548in}{3.199956in}}{\pgfqpoint{2.551938in}{3.210555in}}{\pgfqpoint{2.551938in}{3.221605in}}%
\pgfpathcurveto{\pgfqpoint{2.551938in}{3.232655in}}{\pgfqpoint{2.547548in}{3.243254in}}{\pgfqpoint{2.539734in}{3.251068in}}%
\pgfpathcurveto{\pgfqpoint{2.531921in}{3.258881in}}{\pgfqpoint{2.521322in}{3.263272in}}{\pgfqpoint{2.510271in}{3.263272in}}%
\pgfpathcurveto{\pgfqpoint{2.499221in}{3.263272in}}{\pgfqpoint{2.488622in}{3.258881in}}{\pgfqpoint{2.480809in}{3.251068in}}%
\pgfpathcurveto{\pgfqpoint{2.472995in}{3.243254in}}{\pgfqpoint{2.468605in}{3.232655in}}{\pgfqpoint{2.468605in}{3.221605in}}%
\pgfpathcurveto{\pgfqpoint{2.468605in}{3.210555in}}{\pgfqpoint{2.472995in}{3.199956in}}{\pgfqpoint{2.480809in}{3.192142in}}%
\pgfpathcurveto{\pgfqpoint{2.488622in}{3.184328in}}{\pgfqpoint{2.499221in}{3.179938in}}{\pgfqpoint{2.510271in}{3.179938in}}%
\pgfpathclose%
\pgfusepath{stroke,fill}%
\end{pgfscope}%
\begin{pgfscope}%
\pgfpathrectangle{\pgfqpoint{0.600000in}{0.600000in}}{\pgfqpoint{3.900000in}{3.900000in}}%
\pgfusepath{clip}%
\pgfsetbuttcap%
\pgfsetroundjoin%
\definecolor{currentfill}{rgb}{0.121569,0.466667,0.705882}%
\pgfsetfillcolor{currentfill}%
\pgfsetlinewidth{1.003750pt}%
\definecolor{currentstroke}{rgb}{0.121569,0.466667,0.705882}%
\pgfsetstrokecolor{currentstroke}%
\pgfsetdash{}{0pt}%
\pgfpathmoveto{\pgfqpoint{2.962222in}{2.929740in}}%
\pgfpathcurveto{\pgfqpoint{2.973272in}{2.929740in}}{\pgfqpoint{2.983871in}{2.934131in}}{\pgfqpoint{2.991685in}{2.941944in}}%
\pgfpathcurveto{\pgfqpoint{2.999499in}{2.949758in}}{\pgfqpoint{3.003889in}{2.960357in}}{\pgfqpoint{3.003889in}{2.971407in}}%
\pgfpathcurveto{\pgfqpoint{3.003889in}{2.982457in}}{\pgfqpoint{2.999499in}{2.993056in}}{\pgfqpoint{2.991685in}{3.000870in}}%
\pgfpathcurveto{\pgfqpoint{2.983871in}{3.008684in}}{\pgfqpoint{2.973272in}{3.013074in}}{\pgfqpoint{2.962222in}{3.013074in}}%
\pgfpathcurveto{\pgfqpoint{2.951172in}{3.013074in}}{\pgfqpoint{2.940573in}{3.008684in}}{\pgfqpoint{2.932760in}{3.000870in}}%
\pgfpathcurveto{\pgfqpoint{2.924946in}{2.993056in}}{\pgfqpoint{2.920556in}{2.982457in}}{\pgfqpoint{2.920556in}{2.971407in}}%
\pgfpathcurveto{\pgfqpoint{2.920556in}{2.960357in}}{\pgfqpoint{2.924946in}{2.949758in}}{\pgfqpoint{2.932760in}{2.941944in}}%
\pgfpathcurveto{\pgfqpoint{2.940573in}{2.934131in}}{\pgfqpoint{2.951172in}{2.929740in}}{\pgfqpoint{2.962222in}{2.929740in}}%
\pgfpathclose%
\pgfusepath{stroke,fill}%
\end{pgfscope}%
\begin{pgfscope}%
\pgfpathrectangle{\pgfqpoint{0.600000in}{0.600000in}}{\pgfqpoint{3.900000in}{3.900000in}}%
\pgfusepath{clip}%
\pgfsetbuttcap%
\pgfsetroundjoin%
\definecolor{currentfill}{rgb}{0.121569,0.466667,0.705882}%
\pgfsetfillcolor{currentfill}%
\pgfsetlinewidth{1.003750pt}%
\definecolor{currentstroke}{rgb}{0.121569,0.466667,0.705882}%
\pgfsetstrokecolor{currentstroke}%
\pgfsetdash{}{0pt}%
\pgfpathmoveto{\pgfqpoint{2.067481in}{2.092513in}}%
\pgfpathcurveto{\pgfqpoint{2.078531in}{2.092513in}}{\pgfqpoint{2.089130in}{2.096904in}}{\pgfqpoint{2.096943in}{2.104717in}}%
\pgfpathcurveto{\pgfqpoint{2.104757in}{2.112531in}}{\pgfqpoint{2.109147in}{2.123130in}}{\pgfqpoint{2.109147in}{2.134180in}}%
\pgfpathcurveto{\pgfqpoint{2.109147in}{2.145230in}}{\pgfqpoint{2.104757in}{2.155829in}}{\pgfqpoint{2.096943in}{2.163643in}}%
\pgfpathcurveto{\pgfqpoint{2.089130in}{2.171457in}}{\pgfqpoint{2.078531in}{2.175847in}}{\pgfqpoint{2.067481in}{2.175847in}}%
\pgfpathcurveto{\pgfqpoint{2.056430in}{2.175847in}}{\pgfqpoint{2.045831in}{2.171457in}}{\pgfqpoint{2.038018in}{2.163643in}}%
\pgfpathcurveto{\pgfqpoint{2.030204in}{2.155829in}}{\pgfqpoint{2.025814in}{2.145230in}}{\pgfqpoint{2.025814in}{2.134180in}}%
\pgfpathcurveto{\pgfqpoint{2.025814in}{2.123130in}}{\pgfqpoint{2.030204in}{2.112531in}}{\pgfqpoint{2.038018in}{2.104717in}}%
\pgfpathcurveto{\pgfqpoint{2.045831in}{2.096904in}}{\pgfqpoint{2.056430in}{2.092513in}}{\pgfqpoint{2.067481in}{2.092513in}}%
\pgfpathclose%
\pgfusepath{stroke,fill}%
\end{pgfscope}%
\begin{pgfscope}%
\pgfpathrectangle{\pgfqpoint{0.600000in}{0.600000in}}{\pgfqpoint{3.900000in}{3.900000in}}%
\pgfusepath{clip}%
\pgfsetbuttcap%
\pgfsetroundjoin%
\definecolor{currentfill}{rgb}{0.121569,0.466667,0.705882}%
\pgfsetfillcolor{currentfill}%
\pgfsetlinewidth{1.003750pt}%
\definecolor{currentstroke}{rgb}{0.121569,0.466667,0.705882}%
\pgfsetstrokecolor{currentstroke}%
\pgfsetdash{}{0pt}%
\pgfpathmoveto{\pgfqpoint{2.020087in}{3.307732in}}%
\pgfpathcurveto{\pgfqpoint{2.031137in}{3.307732in}}{\pgfqpoint{2.041736in}{3.312122in}}{\pgfqpoint{2.049550in}{3.319936in}}%
\pgfpathcurveto{\pgfqpoint{2.057363in}{3.327749in}}{\pgfqpoint{2.061753in}{3.338348in}}{\pgfqpoint{2.061753in}{3.349398in}}%
\pgfpathcurveto{\pgfqpoint{2.061753in}{3.360449in}}{\pgfqpoint{2.057363in}{3.371048in}}{\pgfqpoint{2.049550in}{3.378861in}}%
\pgfpathcurveto{\pgfqpoint{2.041736in}{3.386675in}}{\pgfqpoint{2.031137in}{3.391065in}}{\pgfqpoint{2.020087in}{3.391065in}}%
\pgfpathcurveto{\pgfqpoint{2.009037in}{3.391065in}}{\pgfqpoint{1.998438in}{3.386675in}}{\pgfqpoint{1.990624in}{3.378861in}}%
\pgfpathcurveto{\pgfqpoint{1.982810in}{3.371048in}}{\pgfqpoint{1.978420in}{3.360449in}}{\pgfqpoint{1.978420in}{3.349398in}}%
\pgfpathcurveto{\pgfqpoint{1.978420in}{3.338348in}}{\pgfqpoint{1.982810in}{3.327749in}}{\pgfqpoint{1.990624in}{3.319936in}}%
\pgfpathcurveto{\pgfqpoint{1.998438in}{3.312122in}}{\pgfqpoint{2.009037in}{3.307732in}}{\pgfqpoint{2.020087in}{3.307732in}}%
\pgfpathclose%
\pgfusepath{stroke,fill}%
\end{pgfscope}%
\begin{pgfscope}%
\pgfpathrectangle{\pgfqpoint{0.600000in}{0.600000in}}{\pgfqpoint{3.900000in}{3.900000in}}%
\pgfusepath{clip}%
\pgfsetbuttcap%
\pgfsetroundjoin%
\definecolor{currentfill}{rgb}{0.121569,0.466667,0.705882}%
\pgfsetfillcolor{currentfill}%
\pgfsetlinewidth{1.003750pt}%
\definecolor{currentstroke}{rgb}{0.121569,0.466667,0.705882}%
\pgfsetstrokecolor{currentstroke}%
\pgfsetdash{}{0pt}%
\pgfpathmoveto{\pgfqpoint{2.886961in}{3.078133in}}%
\pgfpathcurveto{\pgfqpoint{2.898011in}{3.078133in}}{\pgfqpoint{2.908610in}{3.082523in}}{\pgfqpoint{2.916424in}{3.090337in}}%
\pgfpathcurveto{\pgfqpoint{2.924238in}{3.098151in}}{\pgfqpoint{2.928628in}{3.108750in}}{\pgfqpoint{2.928628in}{3.119800in}}%
\pgfpathcurveto{\pgfqpoint{2.928628in}{3.130850in}}{\pgfqpoint{2.924238in}{3.141449in}}{\pgfqpoint{2.916424in}{3.149263in}}%
\pgfpathcurveto{\pgfqpoint{2.908610in}{3.157076in}}{\pgfqpoint{2.898011in}{3.161467in}}{\pgfqpoint{2.886961in}{3.161467in}}%
\pgfpathcurveto{\pgfqpoint{2.875911in}{3.161467in}}{\pgfqpoint{2.865312in}{3.157076in}}{\pgfqpoint{2.857498in}{3.149263in}}%
\pgfpathcurveto{\pgfqpoint{2.849685in}{3.141449in}}{\pgfqpoint{2.845294in}{3.130850in}}{\pgfqpoint{2.845294in}{3.119800in}}%
\pgfpathcurveto{\pgfqpoint{2.845294in}{3.108750in}}{\pgfqpoint{2.849685in}{3.098151in}}{\pgfqpoint{2.857498in}{3.090337in}}%
\pgfpathcurveto{\pgfqpoint{2.865312in}{3.082523in}}{\pgfqpoint{2.875911in}{3.078133in}}{\pgfqpoint{2.886961in}{3.078133in}}%
\pgfpathclose%
\pgfusepath{stroke,fill}%
\end{pgfscope}%
\begin{pgfscope}%
\pgfpathrectangle{\pgfqpoint{0.600000in}{0.600000in}}{\pgfqpoint{3.900000in}{3.900000in}}%
\pgfusepath{clip}%
\pgfsetbuttcap%
\pgfsetroundjoin%
\definecolor{currentfill}{rgb}{0.121569,0.466667,0.705882}%
\pgfsetfillcolor{currentfill}%
\pgfsetlinewidth{1.003750pt}%
\definecolor{currentstroke}{rgb}{0.121569,0.466667,0.705882}%
\pgfsetstrokecolor{currentstroke}%
\pgfsetdash{}{0pt}%
\pgfpathmoveto{\pgfqpoint{2.083066in}{3.120750in}}%
\pgfpathcurveto{\pgfqpoint{2.094116in}{3.120750in}}{\pgfqpoint{2.104715in}{3.125141in}}{\pgfqpoint{2.112529in}{3.132954in}}%
\pgfpathcurveto{\pgfqpoint{2.120342in}{3.140768in}}{\pgfqpoint{2.124733in}{3.151367in}}{\pgfqpoint{2.124733in}{3.162417in}}%
\pgfpathcurveto{\pgfqpoint{2.124733in}{3.173467in}}{\pgfqpoint{2.120342in}{3.184066in}}{\pgfqpoint{2.112529in}{3.191880in}}%
\pgfpathcurveto{\pgfqpoint{2.104715in}{3.199693in}}{\pgfqpoint{2.094116in}{3.204084in}}{\pgfqpoint{2.083066in}{3.204084in}}%
\pgfpathcurveto{\pgfqpoint{2.072016in}{3.204084in}}{\pgfqpoint{2.061417in}{3.199693in}}{\pgfqpoint{2.053603in}{3.191880in}}%
\pgfpathcurveto{\pgfqpoint{2.045790in}{3.184066in}}{\pgfqpoint{2.041399in}{3.173467in}}{\pgfqpoint{2.041399in}{3.162417in}}%
\pgfpathcurveto{\pgfqpoint{2.041399in}{3.151367in}}{\pgfqpoint{2.045790in}{3.140768in}}{\pgfqpoint{2.053603in}{3.132954in}}%
\pgfpathcurveto{\pgfqpoint{2.061417in}{3.125141in}}{\pgfqpoint{2.072016in}{3.120750in}}{\pgfqpoint{2.083066in}{3.120750in}}%
\pgfpathclose%
\pgfusepath{stroke,fill}%
\end{pgfscope}%
\begin{pgfscope}%
\pgfpathrectangle{\pgfqpoint{0.600000in}{0.600000in}}{\pgfqpoint{3.900000in}{3.900000in}}%
\pgfusepath{clip}%
\pgfsetbuttcap%
\pgfsetroundjoin%
\definecolor{currentfill}{rgb}{0.121569,0.466667,0.705882}%
\pgfsetfillcolor{currentfill}%
\pgfsetlinewidth{1.003750pt}%
\definecolor{currentstroke}{rgb}{0.121569,0.466667,0.705882}%
\pgfsetstrokecolor{currentstroke}%
\pgfsetdash{}{0pt}%
\pgfpathmoveto{\pgfqpoint{1.473883in}{2.974966in}}%
\pgfpathcurveto{\pgfqpoint{1.484933in}{2.974966in}}{\pgfqpoint{1.495532in}{2.979356in}}{\pgfqpoint{1.503345in}{2.987170in}}%
\pgfpathcurveto{\pgfqpoint{1.511159in}{2.994983in}}{\pgfqpoint{1.515549in}{3.005582in}}{\pgfqpoint{1.515549in}{3.016633in}}%
\pgfpathcurveto{\pgfqpoint{1.515549in}{3.027683in}}{\pgfqpoint{1.511159in}{3.038282in}}{\pgfqpoint{1.503345in}{3.046095in}}%
\pgfpathcurveto{\pgfqpoint{1.495532in}{3.053909in}}{\pgfqpoint{1.484933in}{3.058299in}}{\pgfqpoint{1.473883in}{3.058299in}}%
\pgfpathcurveto{\pgfqpoint{1.462832in}{3.058299in}}{\pgfqpoint{1.452233in}{3.053909in}}{\pgfqpoint{1.444420in}{3.046095in}}%
\pgfpathcurveto{\pgfqpoint{1.436606in}{3.038282in}}{\pgfqpoint{1.432216in}{3.027683in}}{\pgfqpoint{1.432216in}{3.016633in}}%
\pgfpathcurveto{\pgfqpoint{1.432216in}{3.005582in}}{\pgfqpoint{1.436606in}{2.994983in}}{\pgfqpoint{1.444420in}{2.987170in}}%
\pgfpathcurveto{\pgfqpoint{1.452233in}{2.979356in}}{\pgfqpoint{1.462832in}{2.974966in}}{\pgfqpoint{1.473883in}{2.974966in}}%
\pgfpathclose%
\pgfusepath{stroke,fill}%
\end{pgfscope}%
\begin{pgfscope}%
\pgfpathrectangle{\pgfqpoint{0.600000in}{0.600000in}}{\pgfqpoint{3.900000in}{3.900000in}}%
\pgfusepath{clip}%
\pgfsetbuttcap%
\pgfsetroundjoin%
\definecolor{currentfill}{rgb}{0.121569,0.466667,0.705882}%
\pgfsetfillcolor{currentfill}%
\pgfsetlinewidth{1.003750pt}%
\definecolor{currentstroke}{rgb}{0.121569,0.466667,0.705882}%
\pgfsetstrokecolor{currentstroke}%
\pgfsetdash{}{0pt}%
\pgfpathmoveto{\pgfqpoint{2.843993in}{3.498773in}}%
\pgfpathcurveto{\pgfqpoint{2.855043in}{3.498773in}}{\pgfqpoint{2.865642in}{3.503163in}}{\pgfqpoint{2.873456in}{3.510977in}}%
\pgfpathcurveto{\pgfqpoint{2.881269in}{3.518790in}}{\pgfqpoint{2.885660in}{3.529389in}}{\pgfqpoint{2.885660in}{3.540439in}}%
\pgfpathcurveto{\pgfqpoint{2.885660in}{3.551490in}}{\pgfqpoint{2.881269in}{3.562089in}}{\pgfqpoint{2.873456in}{3.569902in}}%
\pgfpathcurveto{\pgfqpoint{2.865642in}{3.577716in}}{\pgfqpoint{2.855043in}{3.582106in}}{\pgfqpoint{2.843993in}{3.582106in}}%
\pgfpathcurveto{\pgfqpoint{2.832943in}{3.582106in}}{\pgfqpoint{2.822344in}{3.577716in}}{\pgfqpoint{2.814530in}{3.569902in}}%
\pgfpathcurveto{\pgfqpoint{2.806716in}{3.562089in}}{\pgfqpoint{2.802326in}{3.551490in}}{\pgfqpoint{2.802326in}{3.540439in}}%
\pgfpathcurveto{\pgfqpoint{2.802326in}{3.529389in}}{\pgfqpoint{2.806716in}{3.518790in}}{\pgfqpoint{2.814530in}{3.510977in}}%
\pgfpathcurveto{\pgfqpoint{2.822344in}{3.503163in}}{\pgfqpoint{2.832943in}{3.498773in}}{\pgfqpoint{2.843993in}{3.498773in}}%
\pgfpathclose%
\pgfusepath{stroke,fill}%
\end{pgfscope}%
\begin{pgfscope}%
\pgfpathrectangle{\pgfqpoint{0.600000in}{0.600000in}}{\pgfqpoint{3.900000in}{3.900000in}}%
\pgfusepath{clip}%
\pgfsetbuttcap%
\pgfsetroundjoin%
\definecolor{currentfill}{rgb}{0.121569,0.466667,0.705882}%
\pgfsetfillcolor{currentfill}%
\pgfsetlinewidth{1.003750pt}%
\definecolor{currentstroke}{rgb}{0.121569,0.466667,0.705882}%
\pgfsetstrokecolor{currentstroke}%
\pgfsetdash{}{0pt}%
\pgfpathmoveto{\pgfqpoint{1.797856in}{3.612449in}}%
\pgfpathcurveto{\pgfqpoint{1.808906in}{3.612449in}}{\pgfqpoint{1.819505in}{3.616840in}}{\pgfqpoint{1.827319in}{3.624653in}}%
\pgfpathcurveto{\pgfqpoint{1.835133in}{3.632467in}}{\pgfqpoint{1.839523in}{3.643066in}}{\pgfqpoint{1.839523in}{3.654116in}}%
\pgfpathcurveto{\pgfqpoint{1.839523in}{3.665166in}}{\pgfqpoint{1.835133in}{3.675765in}}{\pgfqpoint{1.827319in}{3.683579in}}%
\pgfpathcurveto{\pgfqpoint{1.819505in}{3.691392in}}{\pgfqpoint{1.808906in}{3.695783in}}{\pgfqpoint{1.797856in}{3.695783in}}%
\pgfpathcurveto{\pgfqpoint{1.786806in}{3.695783in}}{\pgfqpoint{1.776207in}{3.691392in}}{\pgfqpoint{1.768393in}{3.683579in}}%
\pgfpathcurveto{\pgfqpoint{1.760580in}{3.675765in}}{\pgfqpoint{1.756189in}{3.665166in}}{\pgfqpoint{1.756189in}{3.654116in}}%
\pgfpathcurveto{\pgfqpoint{1.756189in}{3.643066in}}{\pgfqpoint{1.760580in}{3.632467in}}{\pgfqpoint{1.768393in}{3.624653in}}%
\pgfpathcurveto{\pgfqpoint{1.776207in}{3.616840in}}{\pgfqpoint{1.786806in}{3.612449in}}{\pgfqpoint{1.797856in}{3.612449in}}%
\pgfpathclose%
\pgfusepath{stroke,fill}%
\end{pgfscope}%
\begin{pgfscope}%
\pgfpathrectangle{\pgfqpoint{0.600000in}{0.600000in}}{\pgfqpoint{3.900000in}{3.900000in}}%
\pgfusepath{clip}%
\pgfsetbuttcap%
\pgfsetroundjoin%
\definecolor{currentfill}{rgb}{0.121569,0.466667,0.705882}%
\pgfsetfillcolor{currentfill}%
\pgfsetlinewidth{1.003750pt}%
\definecolor{currentstroke}{rgb}{0.121569,0.466667,0.705882}%
\pgfsetstrokecolor{currentstroke}%
\pgfsetdash{}{0pt}%
\pgfpathmoveto{\pgfqpoint{2.508971in}{2.649790in}}%
\pgfpathcurveto{\pgfqpoint{2.520021in}{2.649790in}}{\pgfqpoint{2.530620in}{2.654180in}}{\pgfqpoint{2.538433in}{2.661994in}}%
\pgfpathcurveto{\pgfqpoint{2.546247in}{2.669807in}}{\pgfqpoint{2.550637in}{2.680406in}}{\pgfqpoint{2.550637in}{2.691456in}}%
\pgfpathcurveto{\pgfqpoint{2.550637in}{2.702507in}}{\pgfqpoint{2.546247in}{2.713106in}}{\pgfqpoint{2.538433in}{2.720919in}}%
\pgfpathcurveto{\pgfqpoint{2.530620in}{2.728733in}}{\pgfqpoint{2.520021in}{2.733123in}}{\pgfqpoint{2.508971in}{2.733123in}}%
\pgfpathcurveto{\pgfqpoint{2.497920in}{2.733123in}}{\pgfqpoint{2.487321in}{2.728733in}}{\pgfqpoint{2.479508in}{2.720919in}}%
\pgfpathcurveto{\pgfqpoint{2.471694in}{2.713106in}}{\pgfqpoint{2.467304in}{2.702507in}}{\pgfqpoint{2.467304in}{2.691456in}}%
\pgfpathcurveto{\pgfqpoint{2.467304in}{2.680406in}}{\pgfqpoint{2.471694in}{2.669807in}}{\pgfqpoint{2.479508in}{2.661994in}}%
\pgfpathcurveto{\pgfqpoint{2.487321in}{2.654180in}}{\pgfqpoint{2.497920in}{2.649790in}}{\pgfqpoint{2.508971in}{2.649790in}}%
\pgfpathclose%
\pgfusepath{stroke,fill}%
\end{pgfscope}%
\begin{pgfscope}%
\pgfpathrectangle{\pgfqpoint{0.600000in}{0.600000in}}{\pgfqpoint{3.900000in}{3.900000in}}%
\pgfusepath{clip}%
\pgfsetbuttcap%
\pgfsetroundjoin%
\definecolor{currentfill}{rgb}{0.121569,0.466667,0.705882}%
\pgfsetfillcolor{currentfill}%
\pgfsetlinewidth{1.003750pt}%
\definecolor{currentstroke}{rgb}{0.121569,0.466667,0.705882}%
\pgfsetstrokecolor{currentstroke}%
\pgfsetdash{}{0pt}%
\pgfpathmoveto{\pgfqpoint{2.672795in}{3.029942in}}%
\pgfpathcurveto{\pgfqpoint{2.683845in}{3.029942in}}{\pgfqpoint{2.694444in}{3.034332in}}{\pgfqpoint{2.702258in}{3.042146in}}%
\pgfpathcurveto{\pgfqpoint{2.710071in}{3.049959in}}{\pgfqpoint{2.714462in}{3.060558in}}{\pgfqpoint{2.714462in}{3.071609in}}%
\pgfpathcurveto{\pgfqpoint{2.714462in}{3.082659in}}{\pgfqpoint{2.710071in}{3.093258in}}{\pgfqpoint{2.702258in}{3.101071in}}%
\pgfpathcurveto{\pgfqpoint{2.694444in}{3.108885in}}{\pgfqpoint{2.683845in}{3.113275in}}{\pgfqpoint{2.672795in}{3.113275in}}%
\pgfpathcurveto{\pgfqpoint{2.661745in}{3.113275in}}{\pgfqpoint{2.651146in}{3.108885in}}{\pgfqpoint{2.643332in}{3.101071in}}%
\pgfpathcurveto{\pgfqpoint{2.635519in}{3.093258in}}{\pgfqpoint{2.631128in}{3.082659in}}{\pgfqpoint{2.631128in}{3.071609in}}%
\pgfpathcurveto{\pgfqpoint{2.631128in}{3.060558in}}{\pgfqpoint{2.635519in}{3.049959in}}{\pgfqpoint{2.643332in}{3.042146in}}%
\pgfpathcurveto{\pgfqpoint{2.651146in}{3.034332in}}{\pgfqpoint{2.661745in}{3.029942in}}{\pgfqpoint{2.672795in}{3.029942in}}%
\pgfpathclose%
\pgfusepath{stroke,fill}%
\end{pgfscope}%
\begin{pgfscope}%
\pgfpathrectangle{\pgfqpoint{0.600000in}{0.600000in}}{\pgfqpoint{3.900000in}{3.900000in}}%
\pgfusepath{clip}%
\pgfsetbuttcap%
\pgfsetroundjoin%
\definecolor{currentfill}{rgb}{0.121569,0.466667,0.705882}%
\pgfsetfillcolor{currentfill}%
\pgfsetlinewidth{1.003750pt}%
\definecolor{currentstroke}{rgb}{0.121569,0.466667,0.705882}%
\pgfsetstrokecolor{currentstroke}%
\pgfsetdash{}{0pt}%
\pgfpathmoveto{\pgfqpoint{2.888378in}{2.107674in}}%
\pgfpathcurveto{\pgfqpoint{2.899428in}{2.107674in}}{\pgfqpoint{2.910027in}{2.112064in}}{\pgfqpoint{2.917841in}{2.119877in}}%
\pgfpathcurveto{\pgfqpoint{2.925654in}{2.127691in}}{\pgfqpoint{2.930045in}{2.138290in}}{\pgfqpoint{2.930045in}{2.149340in}}%
\pgfpathcurveto{\pgfqpoint{2.930045in}{2.160390in}}{\pgfqpoint{2.925654in}{2.170989in}}{\pgfqpoint{2.917841in}{2.178803in}}%
\pgfpathcurveto{\pgfqpoint{2.910027in}{2.186617in}}{\pgfqpoint{2.899428in}{2.191007in}}{\pgfqpoint{2.888378in}{2.191007in}}%
\pgfpathcurveto{\pgfqpoint{2.877328in}{2.191007in}}{\pgfqpoint{2.866729in}{2.186617in}}{\pgfqpoint{2.858915in}{2.178803in}}%
\pgfpathcurveto{\pgfqpoint{2.851102in}{2.170989in}}{\pgfqpoint{2.846711in}{2.160390in}}{\pgfqpoint{2.846711in}{2.149340in}}%
\pgfpathcurveto{\pgfqpoint{2.846711in}{2.138290in}}{\pgfqpoint{2.851102in}{2.127691in}}{\pgfqpoint{2.858915in}{2.119877in}}%
\pgfpathcurveto{\pgfqpoint{2.866729in}{2.112064in}}{\pgfqpoint{2.877328in}{2.107674in}}{\pgfqpoint{2.888378in}{2.107674in}}%
\pgfpathclose%
\pgfusepath{stroke,fill}%
\end{pgfscope}%
\begin{pgfscope}%
\pgfpathrectangle{\pgfqpoint{0.600000in}{0.600000in}}{\pgfqpoint{3.900000in}{3.900000in}}%
\pgfusepath{clip}%
\pgfsetbuttcap%
\pgfsetroundjoin%
\definecolor{currentfill}{rgb}{0.121569,0.466667,0.705882}%
\pgfsetfillcolor{currentfill}%
\pgfsetlinewidth{1.003750pt}%
\definecolor{currentstroke}{rgb}{0.121569,0.466667,0.705882}%
\pgfsetstrokecolor{currentstroke}%
\pgfsetdash{}{0pt}%
\pgfpathmoveto{\pgfqpoint{1.873086in}{1.812507in}}%
\pgfpathcurveto{\pgfqpoint{1.884136in}{1.812507in}}{\pgfqpoint{1.894735in}{1.816897in}}{\pgfqpoint{1.902549in}{1.824711in}}%
\pgfpathcurveto{\pgfqpoint{1.910362in}{1.832524in}}{\pgfqpoint{1.914753in}{1.843123in}}{\pgfqpoint{1.914753in}{1.854174in}}%
\pgfpathcurveto{\pgfqpoint{1.914753in}{1.865224in}}{\pgfqpoint{1.910362in}{1.875823in}}{\pgfqpoint{1.902549in}{1.883636in}}%
\pgfpathcurveto{\pgfqpoint{1.894735in}{1.891450in}}{\pgfqpoint{1.884136in}{1.895840in}}{\pgfqpoint{1.873086in}{1.895840in}}%
\pgfpathcurveto{\pgfqpoint{1.862036in}{1.895840in}}{\pgfqpoint{1.851437in}{1.891450in}}{\pgfqpoint{1.843623in}{1.883636in}}%
\pgfpathcurveto{\pgfqpoint{1.835810in}{1.875823in}}{\pgfqpoint{1.831419in}{1.865224in}}{\pgfqpoint{1.831419in}{1.854174in}}%
\pgfpathcurveto{\pgfqpoint{1.831419in}{1.843123in}}{\pgfqpoint{1.835810in}{1.832524in}}{\pgfqpoint{1.843623in}{1.824711in}}%
\pgfpathcurveto{\pgfqpoint{1.851437in}{1.816897in}}{\pgfqpoint{1.862036in}{1.812507in}}{\pgfqpoint{1.873086in}{1.812507in}}%
\pgfpathclose%
\pgfusepath{stroke,fill}%
\end{pgfscope}%
\begin{pgfscope}%
\pgfpathrectangle{\pgfqpoint{0.600000in}{0.600000in}}{\pgfqpoint{3.900000in}{3.900000in}}%
\pgfusepath{clip}%
\pgfsetbuttcap%
\pgfsetroundjoin%
\definecolor{currentfill}{rgb}{0.121569,0.466667,0.705882}%
\pgfsetfillcolor{currentfill}%
\pgfsetlinewidth{1.003750pt}%
\definecolor{currentstroke}{rgb}{0.121569,0.466667,0.705882}%
\pgfsetstrokecolor{currentstroke}%
\pgfsetdash{}{0pt}%
\pgfpathmoveto{\pgfqpoint{2.728251in}{2.948475in}}%
\pgfpathcurveto{\pgfqpoint{2.739301in}{2.948475in}}{\pgfqpoint{2.749900in}{2.952865in}}{\pgfqpoint{2.757714in}{2.960679in}}%
\pgfpathcurveto{\pgfqpoint{2.765527in}{2.968493in}}{\pgfqpoint{2.769918in}{2.979092in}}{\pgfqpoint{2.769918in}{2.990142in}}%
\pgfpathcurveto{\pgfqpoint{2.769918in}{3.001192in}}{\pgfqpoint{2.765527in}{3.011791in}}{\pgfqpoint{2.757714in}{3.019604in}}%
\pgfpathcurveto{\pgfqpoint{2.749900in}{3.027418in}}{\pgfqpoint{2.739301in}{3.031808in}}{\pgfqpoint{2.728251in}{3.031808in}}%
\pgfpathcurveto{\pgfqpoint{2.717201in}{3.031808in}}{\pgfqpoint{2.706602in}{3.027418in}}{\pgfqpoint{2.698788in}{3.019604in}}%
\pgfpathcurveto{\pgfqpoint{2.690975in}{3.011791in}}{\pgfqpoint{2.686584in}{3.001192in}}{\pgfqpoint{2.686584in}{2.990142in}}%
\pgfpathcurveto{\pgfqpoint{2.686584in}{2.979092in}}{\pgfqpoint{2.690975in}{2.968493in}}{\pgfqpoint{2.698788in}{2.960679in}}%
\pgfpathcurveto{\pgfqpoint{2.706602in}{2.952865in}}{\pgfqpoint{2.717201in}{2.948475in}}{\pgfqpoint{2.728251in}{2.948475in}}%
\pgfpathclose%
\pgfusepath{stroke,fill}%
\end{pgfscope}%
\begin{pgfscope}%
\pgfpathrectangle{\pgfqpoint{0.600000in}{0.600000in}}{\pgfqpoint{3.900000in}{3.900000in}}%
\pgfusepath{clip}%
\pgfsetbuttcap%
\pgfsetroundjoin%
\definecolor{currentfill}{rgb}{0.121569,0.466667,0.705882}%
\pgfsetfillcolor{currentfill}%
\pgfsetlinewidth{1.003750pt}%
\definecolor{currentstroke}{rgb}{0.121569,0.466667,0.705882}%
\pgfsetstrokecolor{currentstroke}%
\pgfsetdash{}{0pt}%
\pgfpathmoveto{\pgfqpoint{2.117095in}{3.770131in}}%
\pgfpathcurveto{\pgfqpoint{2.128146in}{3.770131in}}{\pgfqpoint{2.138745in}{3.774521in}}{\pgfqpoint{2.146558in}{3.782334in}}%
\pgfpathcurveto{\pgfqpoint{2.154372in}{3.790148in}}{\pgfqpoint{2.158762in}{3.800747in}}{\pgfqpoint{2.158762in}{3.811797in}}%
\pgfpathcurveto{\pgfqpoint{2.158762in}{3.822847in}}{\pgfqpoint{2.154372in}{3.833446in}}{\pgfqpoint{2.146558in}{3.841260in}}%
\pgfpathcurveto{\pgfqpoint{2.138745in}{3.849074in}}{\pgfqpoint{2.128146in}{3.853464in}}{\pgfqpoint{2.117095in}{3.853464in}}%
\pgfpathcurveto{\pgfqpoint{2.106045in}{3.853464in}}{\pgfqpoint{2.095446in}{3.849074in}}{\pgfqpoint{2.087633in}{3.841260in}}%
\pgfpathcurveto{\pgfqpoint{2.079819in}{3.833446in}}{\pgfqpoint{2.075429in}{3.822847in}}{\pgfqpoint{2.075429in}{3.811797in}}%
\pgfpathcurveto{\pgfqpoint{2.075429in}{3.800747in}}{\pgfqpoint{2.079819in}{3.790148in}}{\pgfqpoint{2.087633in}{3.782334in}}%
\pgfpathcurveto{\pgfqpoint{2.095446in}{3.774521in}}{\pgfqpoint{2.106045in}{3.770131in}}{\pgfqpoint{2.117095in}{3.770131in}}%
\pgfpathclose%
\pgfusepath{stroke,fill}%
\end{pgfscope}%
\begin{pgfscope}%
\pgfpathrectangle{\pgfqpoint{0.600000in}{0.600000in}}{\pgfqpoint{3.900000in}{3.900000in}}%
\pgfusepath{clip}%
\pgfsetbuttcap%
\pgfsetroundjoin%
\definecolor{currentfill}{rgb}{0.121569,0.466667,0.705882}%
\pgfsetfillcolor{currentfill}%
\pgfsetlinewidth{1.003750pt}%
\definecolor{currentstroke}{rgb}{0.121569,0.466667,0.705882}%
\pgfsetstrokecolor{currentstroke}%
\pgfsetdash{}{0pt}%
\pgfpathmoveto{\pgfqpoint{2.067315in}{2.849689in}}%
\pgfpathcurveto{\pgfqpoint{2.078366in}{2.849689in}}{\pgfqpoint{2.088965in}{2.854079in}}{\pgfqpoint{2.096778in}{2.861893in}}%
\pgfpathcurveto{\pgfqpoint{2.104592in}{2.869707in}}{\pgfqpoint{2.108982in}{2.880306in}}{\pgfqpoint{2.108982in}{2.891356in}}%
\pgfpathcurveto{\pgfqpoint{2.108982in}{2.902406in}}{\pgfqpoint{2.104592in}{2.913005in}}{\pgfqpoint{2.096778in}{2.920819in}}%
\pgfpathcurveto{\pgfqpoint{2.088965in}{2.928632in}}{\pgfqpoint{2.078366in}{2.933023in}}{\pgfqpoint{2.067315in}{2.933023in}}%
\pgfpathcurveto{\pgfqpoint{2.056265in}{2.933023in}}{\pgfqpoint{2.045666in}{2.928632in}}{\pgfqpoint{2.037853in}{2.920819in}}%
\pgfpathcurveto{\pgfqpoint{2.030039in}{2.913005in}}{\pgfqpoint{2.025649in}{2.902406in}}{\pgfqpoint{2.025649in}{2.891356in}}%
\pgfpathcurveto{\pgfqpoint{2.025649in}{2.880306in}}{\pgfqpoint{2.030039in}{2.869707in}}{\pgfqpoint{2.037853in}{2.861893in}}%
\pgfpathcurveto{\pgfqpoint{2.045666in}{2.854079in}}{\pgfqpoint{2.056265in}{2.849689in}}{\pgfqpoint{2.067315in}{2.849689in}}%
\pgfpathclose%
\pgfusepath{stroke,fill}%
\end{pgfscope}%
\begin{pgfscope}%
\pgfpathrectangle{\pgfqpoint{0.600000in}{0.600000in}}{\pgfqpoint{3.900000in}{3.900000in}}%
\pgfusepath{clip}%
\pgfsetbuttcap%
\pgfsetroundjoin%
\definecolor{currentfill}{rgb}{0.121569,0.466667,0.705882}%
\pgfsetfillcolor{currentfill}%
\pgfsetlinewidth{1.003750pt}%
\definecolor{currentstroke}{rgb}{0.121569,0.466667,0.705882}%
\pgfsetstrokecolor{currentstroke}%
\pgfsetdash{}{0pt}%
\pgfpathmoveto{\pgfqpoint{3.206659in}{2.065376in}}%
\pgfpathcurveto{\pgfqpoint{3.217709in}{2.065376in}}{\pgfqpoint{3.228308in}{2.069766in}}{\pgfqpoint{3.236122in}{2.077580in}}%
\pgfpathcurveto{\pgfqpoint{3.243935in}{2.085393in}}{\pgfqpoint{3.248325in}{2.095992in}}{\pgfqpoint{3.248325in}{2.107042in}}%
\pgfpathcurveto{\pgfqpoint{3.248325in}{2.118092in}}{\pgfqpoint{3.243935in}{2.128691in}}{\pgfqpoint{3.236122in}{2.136505in}}%
\pgfpathcurveto{\pgfqpoint{3.228308in}{2.144319in}}{\pgfqpoint{3.217709in}{2.148709in}}{\pgfqpoint{3.206659in}{2.148709in}}%
\pgfpathcurveto{\pgfqpoint{3.195609in}{2.148709in}}{\pgfqpoint{3.185010in}{2.144319in}}{\pgfqpoint{3.177196in}{2.136505in}}%
\pgfpathcurveto{\pgfqpoint{3.169382in}{2.128691in}}{\pgfqpoint{3.164992in}{2.118092in}}{\pgfqpoint{3.164992in}{2.107042in}}%
\pgfpathcurveto{\pgfqpoint{3.164992in}{2.095992in}}{\pgfqpoint{3.169382in}{2.085393in}}{\pgfqpoint{3.177196in}{2.077580in}}%
\pgfpathcurveto{\pgfqpoint{3.185010in}{2.069766in}}{\pgfqpoint{3.195609in}{2.065376in}}{\pgfqpoint{3.206659in}{2.065376in}}%
\pgfpathclose%
\pgfusepath{stroke,fill}%
\end{pgfscope}%
\begin{pgfscope}%
\pgfpathrectangle{\pgfqpoint{0.600000in}{0.600000in}}{\pgfqpoint{3.900000in}{3.900000in}}%
\pgfusepath{clip}%
\pgfsetbuttcap%
\pgfsetroundjoin%
\definecolor{currentfill}{rgb}{0.121569,0.466667,0.705882}%
\pgfsetfillcolor{currentfill}%
\pgfsetlinewidth{1.003750pt}%
\definecolor{currentstroke}{rgb}{0.121569,0.466667,0.705882}%
\pgfsetstrokecolor{currentstroke}%
\pgfsetdash{}{0pt}%
\pgfpathmoveto{\pgfqpoint{3.224653in}{1.677079in}}%
\pgfpathcurveto{\pgfqpoint{3.235703in}{1.677079in}}{\pgfqpoint{3.246302in}{1.681469in}}{\pgfqpoint{3.254115in}{1.689283in}}%
\pgfpathcurveto{\pgfqpoint{3.261929in}{1.697096in}}{\pgfqpoint{3.266319in}{1.707695in}}{\pgfqpoint{3.266319in}{1.718745in}}%
\pgfpathcurveto{\pgfqpoint{3.266319in}{1.729795in}}{\pgfqpoint{3.261929in}{1.740395in}}{\pgfqpoint{3.254115in}{1.748208in}}%
\pgfpathcurveto{\pgfqpoint{3.246302in}{1.756022in}}{\pgfqpoint{3.235703in}{1.760412in}}{\pgfqpoint{3.224653in}{1.760412in}}%
\pgfpathcurveto{\pgfqpoint{3.213602in}{1.760412in}}{\pgfqpoint{3.203003in}{1.756022in}}{\pgfqpoint{3.195190in}{1.748208in}}%
\pgfpathcurveto{\pgfqpoint{3.187376in}{1.740395in}}{\pgfqpoint{3.182986in}{1.729795in}}{\pgfqpoint{3.182986in}{1.718745in}}%
\pgfpathcurveto{\pgfqpoint{3.182986in}{1.707695in}}{\pgfqpoint{3.187376in}{1.697096in}}{\pgfqpoint{3.195190in}{1.689283in}}%
\pgfpathcurveto{\pgfqpoint{3.203003in}{1.681469in}}{\pgfqpoint{3.213602in}{1.677079in}}{\pgfqpoint{3.224653in}{1.677079in}}%
\pgfpathclose%
\pgfusepath{stroke,fill}%
\end{pgfscope}%
\begin{pgfscope}%
\pgfpathrectangle{\pgfqpoint{0.600000in}{0.600000in}}{\pgfqpoint{3.900000in}{3.900000in}}%
\pgfusepath{clip}%
\pgfsetbuttcap%
\pgfsetroundjoin%
\definecolor{currentfill}{rgb}{0.121569,0.466667,0.705882}%
\pgfsetfillcolor{currentfill}%
\pgfsetlinewidth{1.003750pt}%
\definecolor{currentstroke}{rgb}{0.121569,0.466667,0.705882}%
\pgfsetstrokecolor{currentstroke}%
\pgfsetdash{}{0pt}%
\pgfpathmoveto{\pgfqpoint{1.809254in}{1.947131in}}%
\pgfpathcurveto{\pgfqpoint{1.820304in}{1.947131in}}{\pgfqpoint{1.830903in}{1.951521in}}{\pgfqpoint{1.838716in}{1.959334in}}%
\pgfpathcurveto{\pgfqpoint{1.846530in}{1.967148in}}{\pgfqpoint{1.850920in}{1.977747in}}{\pgfqpoint{1.850920in}{1.988797in}}%
\pgfpathcurveto{\pgfqpoint{1.850920in}{1.999847in}}{\pgfqpoint{1.846530in}{2.010446in}}{\pgfqpoint{1.838716in}{2.018260in}}%
\pgfpathcurveto{\pgfqpoint{1.830903in}{2.026074in}}{\pgfqpoint{1.820304in}{2.030464in}}{\pgfqpoint{1.809254in}{2.030464in}}%
\pgfpathcurveto{\pgfqpoint{1.798203in}{2.030464in}}{\pgfqpoint{1.787604in}{2.026074in}}{\pgfqpoint{1.779791in}{2.018260in}}%
\pgfpathcurveto{\pgfqpoint{1.771977in}{2.010446in}}{\pgfqpoint{1.767587in}{1.999847in}}{\pgfqpoint{1.767587in}{1.988797in}}%
\pgfpathcurveto{\pgfqpoint{1.767587in}{1.977747in}}{\pgfqpoint{1.771977in}{1.967148in}}{\pgfqpoint{1.779791in}{1.959334in}}%
\pgfpathcurveto{\pgfqpoint{1.787604in}{1.951521in}}{\pgfqpoint{1.798203in}{1.947131in}}{\pgfqpoint{1.809254in}{1.947131in}}%
\pgfpathclose%
\pgfusepath{stroke,fill}%
\end{pgfscope}%
\begin{pgfscope}%
\pgfpathrectangle{\pgfqpoint{0.600000in}{0.600000in}}{\pgfqpoint{3.900000in}{3.900000in}}%
\pgfusepath{clip}%
\pgfsetbuttcap%
\pgfsetroundjoin%
\definecolor{currentfill}{rgb}{0.121569,0.466667,0.705882}%
\pgfsetfillcolor{currentfill}%
\pgfsetlinewidth{1.003750pt}%
\definecolor{currentstroke}{rgb}{0.121569,0.466667,0.705882}%
\pgfsetstrokecolor{currentstroke}%
\pgfsetdash{}{0pt}%
\pgfpathmoveto{\pgfqpoint{2.266033in}{3.220636in}}%
\pgfpathcurveto{\pgfqpoint{2.277083in}{3.220636in}}{\pgfqpoint{2.287682in}{3.225026in}}{\pgfqpoint{2.295496in}{3.232840in}}%
\pgfpathcurveto{\pgfqpoint{2.303310in}{3.240653in}}{\pgfqpoint{2.307700in}{3.251252in}}{\pgfqpoint{2.307700in}{3.262302in}}%
\pgfpathcurveto{\pgfqpoint{2.307700in}{3.273353in}}{\pgfqpoint{2.303310in}{3.283952in}}{\pgfqpoint{2.295496in}{3.291765in}}%
\pgfpathcurveto{\pgfqpoint{2.287682in}{3.299579in}}{\pgfqpoint{2.277083in}{3.303969in}}{\pgfqpoint{2.266033in}{3.303969in}}%
\pgfpathcurveto{\pgfqpoint{2.254983in}{3.303969in}}{\pgfqpoint{2.244384in}{3.299579in}}{\pgfqpoint{2.236570in}{3.291765in}}%
\pgfpathcurveto{\pgfqpoint{2.228757in}{3.283952in}}{\pgfqpoint{2.224367in}{3.273353in}}{\pgfqpoint{2.224367in}{3.262302in}}%
\pgfpathcurveto{\pgfqpoint{2.224367in}{3.251252in}}{\pgfqpoint{2.228757in}{3.240653in}}{\pgfqpoint{2.236570in}{3.232840in}}%
\pgfpathcurveto{\pgfqpoint{2.244384in}{3.225026in}}{\pgfqpoint{2.254983in}{3.220636in}}{\pgfqpoint{2.266033in}{3.220636in}}%
\pgfpathclose%
\pgfusepath{stroke,fill}%
\end{pgfscope}%
\begin{pgfscope}%
\pgfpathrectangle{\pgfqpoint{0.600000in}{0.600000in}}{\pgfqpoint{3.900000in}{3.900000in}}%
\pgfusepath{clip}%
\pgfsetbuttcap%
\pgfsetroundjoin%
\definecolor{currentfill}{rgb}{0.121569,0.466667,0.705882}%
\pgfsetfillcolor{currentfill}%
\pgfsetlinewidth{1.003750pt}%
\definecolor{currentstroke}{rgb}{0.121569,0.466667,0.705882}%
\pgfsetstrokecolor{currentstroke}%
\pgfsetdash{}{0pt}%
\pgfpathmoveto{\pgfqpoint{2.951188in}{2.542297in}}%
\pgfpathcurveto{\pgfqpoint{2.962238in}{2.542297in}}{\pgfqpoint{2.972837in}{2.546688in}}{\pgfqpoint{2.980651in}{2.554501in}}%
\pgfpathcurveto{\pgfqpoint{2.988464in}{2.562315in}}{\pgfqpoint{2.992854in}{2.572914in}}{\pgfqpoint{2.992854in}{2.583964in}}%
\pgfpathcurveto{\pgfqpoint{2.992854in}{2.595014in}}{\pgfqpoint{2.988464in}{2.605613in}}{\pgfqpoint{2.980651in}{2.613427in}}%
\pgfpathcurveto{\pgfqpoint{2.972837in}{2.621240in}}{\pgfqpoint{2.962238in}{2.625631in}}{\pgfqpoint{2.951188in}{2.625631in}}%
\pgfpathcurveto{\pgfqpoint{2.940138in}{2.625631in}}{\pgfqpoint{2.929539in}{2.621240in}}{\pgfqpoint{2.921725in}{2.613427in}}%
\pgfpathcurveto{\pgfqpoint{2.913911in}{2.605613in}}{\pgfqpoint{2.909521in}{2.595014in}}{\pgfqpoint{2.909521in}{2.583964in}}%
\pgfpathcurveto{\pgfqpoint{2.909521in}{2.572914in}}{\pgfqpoint{2.913911in}{2.562315in}}{\pgfqpoint{2.921725in}{2.554501in}}%
\pgfpathcurveto{\pgfqpoint{2.929539in}{2.546688in}}{\pgfqpoint{2.940138in}{2.542297in}}{\pgfqpoint{2.951188in}{2.542297in}}%
\pgfpathclose%
\pgfusepath{stroke,fill}%
\end{pgfscope}%
\begin{pgfscope}%
\pgfpathrectangle{\pgfqpoint{0.600000in}{0.600000in}}{\pgfqpoint{3.900000in}{3.900000in}}%
\pgfusepath{clip}%
\pgfsetbuttcap%
\pgfsetroundjoin%
\definecolor{currentfill}{rgb}{0.121569,0.466667,0.705882}%
\pgfsetfillcolor{currentfill}%
\pgfsetlinewidth{1.003750pt}%
\definecolor{currentstroke}{rgb}{0.121569,0.466667,0.705882}%
\pgfsetstrokecolor{currentstroke}%
\pgfsetdash{}{0pt}%
\pgfpathmoveto{\pgfqpoint{2.565336in}{2.271103in}}%
\pgfpathcurveto{\pgfqpoint{2.576386in}{2.271103in}}{\pgfqpoint{2.586985in}{2.275493in}}{\pgfqpoint{2.594799in}{2.283307in}}%
\pgfpathcurveto{\pgfqpoint{2.602613in}{2.291120in}}{\pgfqpoint{2.607003in}{2.301719in}}{\pgfqpoint{2.607003in}{2.312770in}}%
\pgfpathcurveto{\pgfqpoint{2.607003in}{2.323820in}}{\pgfqpoint{2.602613in}{2.334419in}}{\pgfqpoint{2.594799in}{2.342232in}}%
\pgfpathcurveto{\pgfqpoint{2.586985in}{2.350046in}}{\pgfqpoint{2.576386in}{2.354436in}}{\pgfqpoint{2.565336in}{2.354436in}}%
\pgfpathcurveto{\pgfqpoint{2.554286in}{2.354436in}}{\pgfqpoint{2.543687in}{2.350046in}}{\pgfqpoint{2.535873in}{2.342232in}}%
\pgfpathcurveto{\pgfqpoint{2.528060in}{2.334419in}}{\pgfqpoint{2.523669in}{2.323820in}}{\pgfqpoint{2.523669in}{2.312770in}}%
\pgfpathcurveto{\pgfqpoint{2.523669in}{2.301719in}}{\pgfqpoint{2.528060in}{2.291120in}}{\pgfqpoint{2.535873in}{2.283307in}}%
\pgfpathcurveto{\pgfqpoint{2.543687in}{2.275493in}}{\pgfqpoint{2.554286in}{2.271103in}}{\pgfqpoint{2.565336in}{2.271103in}}%
\pgfpathclose%
\pgfusepath{stroke,fill}%
\end{pgfscope}%
\begin{pgfscope}%
\pgfpathrectangle{\pgfqpoint{0.600000in}{0.600000in}}{\pgfqpoint{3.900000in}{3.900000in}}%
\pgfusepath{clip}%
\pgfsetbuttcap%
\pgfsetroundjoin%
\definecolor{currentfill}{rgb}{0.121569,0.466667,0.705882}%
\pgfsetfillcolor{currentfill}%
\pgfsetlinewidth{1.003750pt}%
\definecolor{currentstroke}{rgb}{0.121569,0.466667,0.705882}%
\pgfsetstrokecolor{currentstroke}%
\pgfsetdash{}{0pt}%
\pgfpathmoveto{\pgfqpoint{2.047042in}{2.416352in}}%
\pgfpathcurveto{\pgfqpoint{2.058093in}{2.416352in}}{\pgfqpoint{2.068692in}{2.420742in}}{\pgfqpoint{2.076505in}{2.428556in}}%
\pgfpathcurveto{\pgfqpoint{2.084319in}{2.436369in}}{\pgfqpoint{2.088709in}{2.446968in}}{\pgfqpoint{2.088709in}{2.458018in}}%
\pgfpathcurveto{\pgfqpoint{2.088709in}{2.469068in}}{\pgfqpoint{2.084319in}{2.479667in}}{\pgfqpoint{2.076505in}{2.487481in}}%
\pgfpathcurveto{\pgfqpoint{2.068692in}{2.495295in}}{\pgfqpoint{2.058093in}{2.499685in}}{\pgfqpoint{2.047042in}{2.499685in}}%
\pgfpathcurveto{\pgfqpoint{2.035992in}{2.499685in}}{\pgfqpoint{2.025393in}{2.495295in}}{\pgfqpoint{2.017580in}{2.487481in}}%
\pgfpathcurveto{\pgfqpoint{2.009766in}{2.479667in}}{\pgfqpoint{2.005376in}{2.469068in}}{\pgfqpoint{2.005376in}{2.458018in}}%
\pgfpathcurveto{\pgfqpoint{2.005376in}{2.446968in}}{\pgfqpoint{2.009766in}{2.436369in}}{\pgfqpoint{2.017580in}{2.428556in}}%
\pgfpathcurveto{\pgfqpoint{2.025393in}{2.420742in}}{\pgfqpoint{2.035992in}{2.416352in}}{\pgfqpoint{2.047042in}{2.416352in}}%
\pgfpathclose%
\pgfusepath{stroke,fill}%
\end{pgfscope}%
\begin{pgfscope}%
\pgfpathrectangle{\pgfqpoint{0.600000in}{0.600000in}}{\pgfqpoint{3.900000in}{3.900000in}}%
\pgfusepath{clip}%
\pgfsetbuttcap%
\pgfsetroundjoin%
\definecolor{currentfill}{rgb}{0.121569,0.466667,0.705882}%
\pgfsetfillcolor{currentfill}%
\pgfsetlinewidth{1.003750pt}%
\definecolor{currentstroke}{rgb}{0.121569,0.466667,0.705882}%
\pgfsetstrokecolor{currentstroke}%
\pgfsetdash{}{0pt}%
\pgfpathmoveto{\pgfqpoint{2.304455in}{1.464484in}}%
\pgfpathcurveto{\pgfqpoint{2.315505in}{1.464484in}}{\pgfqpoint{2.326104in}{1.468875in}}{\pgfqpoint{2.333917in}{1.476688in}}%
\pgfpathcurveto{\pgfqpoint{2.341731in}{1.484502in}}{\pgfqpoint{2.346121in}{1.495101in}}{\pgfqpoint{2.346121in}{1.506151in}}%
\pgfpathcurveto{\pgfqpoint{2.346121in}{1.517201in}}{\pgfqpoint{2.341731in}{1.527800in}}{\pgfqpoint{2.333917in}{1.535614in}}%
\pgfpathcurveto{\pgfqpoint{2.326104in}{1.543428in}}{\pgfqpoint{2.315505in}{1.547818in}}{\pgfqpoint{2.304455in}{1.547818in}}%
\pgfpathcurveto{\pgfqpoint{2.293404in}{1.547818in}}{\pgfqpoint{2.282805in}{1.543428in}}{\pgfqpoint{2.274992in}{1.535614in}}%
\pgfpathcurveto{\pgfqpoint{2.267178in}{1.527800in}}{\pgfqpoint{2.262788in}{1.517201in}}{\pgfqpoint{2.262788in}{1.506151in}}%
\pgfpathcurveto{\pgfqpoint{2.262788in}{1.495101in}}{\pgfqpoint{2.267178in}{1.484502in}}{\pgfqpoint{2.274992in}{1.476688in}}%
\pgfpathcurveto{\pgfqpoint{2.282805in}{1.468875in}}{\pgfqpoint{2.293404in}{1.464484in}}{\pgfqpoint{2.304455in}{1.464484in}}%
\pgfpathclose%
\pgfusepath{stroke,fill}%
\end{pgfscope}%
\begin{pgfscope}%
\pgfpathrectangle{\pgfqpoint{0.600000in}{0.600000in}}{\pgfqpoint{3.900000in}{3.900000in}}%
\pgfusepath{clip}%
\pgfsetbuttcap%
\pgfsetroundjoin%
\definecolor{currentfill}{rgb}{0.121569,0.466667,0.705882}%
\pgfsetfillcolor{currentfill}%
\pgfsetlinewidth{1.003750pt}%
\definecolor{currentstroke}{rgb}{0.121569,0.466667,0.705882}%
\pgfsetstrokecolor{currentstroke}%
\pgfsetdash{}{0pt}%
\pgfpathmoveto{\pgfqpoint{3.049990in}{2.413843in}}%
\pgfpathcurveto{\pgfqpoint{3.061040in}{2.413843in}}{\pgfqpoint{3.071639in}{2.418234in}}{\pgfqpoint{3.079453in}{2.426047in}}%
\pgfpathcurveto{\pgfqpoint{3.087267in}{2.433861in}}{\pgfqpoint{3.091657in}{2.444460in}}{\pgfqpoint{3.091657in}{2.455510in}}%
\pgfpathcurveto{\pgfqpoint{3.091657in}{2.466560in}}{\pgfqpoint{3.087267in}{2.477159in}}{\pgfqpoint{3.079453in}{2.484973in}}%
\pgfpathcurveto{\pgfqpoint{3.071639in}{2.492786in}}{\pgfqpoint{3.061040in}{2.497177in}}{\pgfqpoint{3.049990in}{2.497177in}}%
\pgfpathcurveto{\pgfqpoint{3.038940in}{2.497177in}}{\pgfqpoint{3.028341in}{2.492786in}}{\pgfqpoint{3.020527in}{2.484973in}}%
\pgfpathcurveto{\pgfqpoint{3.012714in}{2.477159in}}{\pgfqpoint{3.008323in}{2.466560in}}{\pgfqpoint{3.008323in}{2.455510in}}%
\pgfpathcurveto{\pgfqpoint{3.008323in}{2.444460in}}{\pgfqpoint{3.012714in}{2.433861in}}{\pgfqpoint{3.020527in}{2.426047in}}%
\pgfpathcurveto{\pgfqpoint{3.028341in}{2.418234in}}{\pgfqpoint{3.038940in}{2.413843in}}{\pgfqpoint{3.049990in}{2.413843in}}%
\pgfpathclose%
\pgfusepath{stroke,fill}%
\end{pgfscope}%
\begin{pgfscope}%
\pgfpathrectangle{\pgfqpoint{0.600000in}{0.600000in}}{\pgfqpoint{3.900000in}{3.900000in}}%
\pgfusepath{clip}%
\pgfsetbuttcap%
\pgfsetroundjoin%
\definecolor{currentfill}{rgb}{0.121569,0.466667,0.705882}%
\pgfsetfillcolor{currentfill}%
\pgfsetlinewidth{1.003750pt}%
\definecolor{currentstroke}{rgb}{0.121569,0.466667,0.705882}%
\pgfsetstrokecolor{currentstroke}%
\pgfsetdash{}{0pt}%
\pgfpathmoveto{\pgfqpoint{3.432106in}{3.427221in}}%
\pgfpathcurveto{\pgfqpoint{3.443156in}{3.427221in}}{\pgfqpoint{3.453755in}{3.431611in}}{\pgfqpoint{3.461569in}{3.439425in}}%
\pgfpathcurveto{\pgfqpoint{3.469382in}{3.447239in}}{\pgfqpoint{3.473773in}{3.457838in}}{\pgfqpoint{3.473773in}{3.468888in}}%
\pgfpathcurveto{\pgfqpoint{3.473773in}{3.479938in}}{\pgfqpoint{3.469382in}{3.490537in}}{\pgfqpoint{3.461569in}{3.498351in}}%
\pgfpathcurveto{\pgfqpoint{3.453755in}{3.506164in}}{\pgfqpoint{3.443156in}{3.510555in}}{\pgfqpoint{3.432106in}{3.510555in}}%
\pgfpathcurveto{\pgfqpoint{3.421056in}{3.510555in}}{\pgfqpoint{3.410457in}{3.506164in}}{\pgfqpoint{3.402643in}{3.498351in}}%
\pgfpathcurveto{\pgfqpoint{3.394830in}{3.490537in}}{\pgfqpoint{3.390439in}{3.479938in}}{\pgfqpoint{3.390439in}{3.468888in}}%
\pgfpathcurveto{\pgfqpoint{3.390439in}{3.457838in}}{\pgfqpoint{3.394830in}{3.447239in}}{\pgfqpoint{3.402643in}{3.439425in}}%
\pgfpathcurveto{\pgfqpoint{3.410457in}{3.431611in}}{\pgfqpoint{3.421056in}{3.427221in}}{\pgfqpoint{3.432106in}{3.427221in}}%
\pgfpathclose%
\pgfusepath{stroke,fill}%
\end{pgfscope}%
\begin{pgfscope}%
\pgfpathrectangle{\pgfqpoint{0.600000in}{0.600000in}}{\pgfqpoint{3.900000in}{3.900000in}}%
\pgfusepath{clip}%
\pgfsetbuttcap%
\pgfsetroundjoin%
\definecolor{currentfill}{rgb}{0.121569,0.466667,0.705882}%
\pgfsetfillcolor{currentfill}%
\pgfsetlinewidth{1.003750pt}%
\definecolor{currentstroke}{rgb}{0.121569,0.466667,0.705882}%
\pgfsetstrokecolor{currentstroke}%
\pgfsetdash{}{0pt}%
\pgfpathmoveto{\pgfqpoint{1.860450in}{1.760592in}}%
\pgfpathcurveto{\pgfqpoint{1.871500in}{1.760592in}}{\pgfqpoint{1.882099in}{1.764983in}}{\pgfqpoint{1.889913in}{1.772796in}}%
\pgfpathcurveto{\pgfqpoint{1.897726in}{1.780610in}}{\pgfqpoint{1.902117in}{1.791209in}}{\pgfqpoint{1.902117in}{1.802259in}}%
\pgfpathcurveto{\pgfqpoint{1.902117in}{1.813309in}}{\pgfqpoint{1.897726in}{1.823908in}}{\pgfqpoint{1.889913in}{1.831722in}}%
\pgfpathcurveto{\pgfqpoint{1.882099in}{1.839536in}}{\pgfqpoint{1.871500in}{1.843926in}}{\pgfqpoint{1.860450in}{1.843926in}}%
\pgfpathcurveto{\pgfqpoint{1.849400in}{1.843926in}}{\pgfqpoint{1.838801in}{1.839536in}}{\pgfqpoint{1.830987in}{1.831722in}}%
\pgfpathcurveto{\pgfqpoint{1.823174in}{1.823908in}}{\pgfqpoint{1.818783in}{1.813309in}}{\pgfqpoint{1.818783in}{1.802259in}}%
\pgfpathcurveto{\pgfqpoint{1.818783in}{1.791209in}}{\pgfqpoint{1.823174in}{1.780610in}}{\pgfqpoint{1.830987in}{1.772796in}}%
\pgfpathcurveto{\pgfqpoint{1.838801in}{1.764983in}}{\pgfqpoint{1.849400in}{1.760592in}}{\pgfqpoint{1.860450in}{1.760592in}}%
\pgfpathclose%
\pgfusepath{stroke,fill}%
\end{pgfscope}%
\begin{pgfscope}%
\pgfpathrectangle{\pgfqpoint{0.600000in}{0.600000in}}{\pgfqpoint{3.900000in}{3.900000in}}%
\pgfusepath{clip}%
\pgfsetbuttcap%
\pgfsetroundjoin%
\definecolor{currentfill}{rgb}{0.121569,0.466667,0.705882}%
\pgfsetfillcolor{currentfill}%
\pgfsetlinewidth{1.003750pt}%
\definecolor{currentstroke}{rgb}{0.121569,0.466667,0.705882}%
\pgfsetstrokecolor{currentstroke}%
\pgfsetdash{}{0pt}%
\pgfpathmoveto{\pgfqpoint{2.147005in}{2.688419in}}%
\pgfpathcurveto{\pgfqpoint{2.158055in}{2.688419in}}{\pgfqpoint{2.168654in}{2.692809in}}{\pgfqpoint{2.176467in}{2.700623in}}%
\pgfpathcurveto{\pgfqpoint{2.184281in}{2.708436in}}{\pgfqpoint{2.188671in}{2.719035in}}{\pgfqpoint{2.188671in}{2.730086in}}%
\pgfpathcurveto{\pgfqpoint{2.188671in}{2.741136in}}{\pgfqpoint{2.184281in}{2.751735in}}{\pgfqpoint{2.176467in}{2.759548in}}%
\pgfpathcurveto{\pgfqpoint{2.168654in}{2.767362in}}{\pgfqpoint{2.158055in}{2.771752in}}{\pgfqpoint{2.147005in}{2.771752in}}%
\pgfpathcurveto{\pgfqpoint{2.135954in}{2.771752in}}{\pgfqpoint{2.125355in}{2.767362in}}{\pgfqpoint{2.117542in}{2.759548in}}%
\pgfpathcurveto{\pgfqpoint{2.109728in}{2.751735in}}{\pgfqpoint{2.105338in}{2.741136in}}{\pgfqpoint{2.105338in}{2.730086in}}%
\pgfpathcurveto{\pgfqpoint{2.105338in}{2.719035in}}{\pgfqpoint{2.109728in}{2.708436in}}{\pgfqpoint{2.117542in}{2.700623in}}%
\pgfpathcurveto{\pgfqpoint{2.125355in}{2.692809in}}{\pgfqpoint{2.135954in}{2.688419in}}{\pgfqpoint{2.147005in}{2.688419in}}%
\pgfpathclose%
\pgfusepath{stroke,fill}%
\end{pgfscope}%
\begin{pgfscope}%
\pgfpathrectangle{\pgfqpoint{0.600000in}{0.600000in}}{\pgfqpoint{3.900000in}{3.900000in}}%
\pgfusepath{clip}%
\pgfsetbuttcap%
\pgfsetroundjoin%
\definecolor{currentfill}{rgb}{0.121569,0.466667,0.705882}%
\pgfsetfillcolor{currentfill}%
\pgfsetlinewidth{1.003750pt}%
\definecolor{currentstroke}{rgb}{0.121569,0.466667,0.705882}%
\pgfsetstrokecolor{currentstroke}%
\pgfsetdash{}{0pt}%
\pgfpathmoveto{\pgfqpoint{2.857496in}{3.323003in}}%
\pgfpathcurveto{\pgfqpoint{2.868546in}{3.323003in}}{\pgfqpoint{2.879145in}{3.327393in}}{\pgfqpoint{2.886958in}{3.335207in}}%
\pgfpathcurveto{\pgfqpoint{2.894772in}{3.343020in}}{\pgfqpoint{2.899162in}{3.353619in}}{\pgfqpoint{2.899162in}{3.364669in}}%
\pgfpathcurveto{\pgfqpoint{2.899162in}{3.375719in}}{\pgfqpoint{2.894772in}{3.386318in}}{\pgfqpoint{2.886958in}{3.394132in}}%
\pgfpathcurveto{\pgfqpoint{2.879145in}{3.401946in}}{\pgfqpoint{2.868546in}{3.406336in}}{\pgfqpoint{2.857496in}{3.406336in}}%
\pgfpathcurveto{\pgfqpoint{2.846445in}{3.406336in}}{\pgfqpoint{2.835846in}{3.401946in}}{\pgfqpoint{2.828033in}{3.394132in}}%
\pgfpathcurveto{\pgfqpoint{2.820219in}{3.386318in}}{\pgfqpoint{2.815829in}{3.375719in}}{\pgfqpoint{2.815829in}{3.364669in}}%
\pgfpathcurveto{\pgfqpoint{2.815829in}{3.353619in}}{\pgfqpoint{2.820219in}{3.343020in}}{\pgfqpoint{2.828033in}{3.335207in}}%
\pgfpathcurveto{\pgfqpoint{2.835846in}{3.327393in}}{\pgfqpoint{2.846445in}{3.323003in}}{\pgfqpoint{2.857496in}{3.323003in}}%
\pgfpathclose%
\pgfusepath{stroke,fill}%
\end{pgfscope}%
\begin{pgfscope}%
\pgfpathrectangle{\pgfqpoint{0.600000in}{0.600000in}}{\pgfqpoint{3.900000in}{3.900000in}}%
\pgfusepath{clip}%
\pgfsetbuttcap%
\pgfsetroundjoin%
\definecolor{currentfill}{rgb}{0.121569,0.466667,0.705882}%
\pgfsetfillcolor{currentfill}%
\pgfsetlinewidth{1.003750pt}%
\definecolor{currentstroke}{rgb}{0.121569,0.466667,0.705882}%
\pgfsetstrokecolor{currentstroke}%
\pgfsetdash{}{0pt}%
\pgfpathmoveto{\pgfqpoint{1.854809in}{1.947489in}}%
\pgfpathcurveto{\pgfqpoint{1.865859in}{1.947489in}}{\pgfqpoint{1.876458in}{1.951879in}}{\pgfqpoint{1.884271in}{1.959693in}}%
\pgfpathcurveto{\pgfqpoint{1.892085in}{1.967506in}}{\pgfqpoint{1.896475in}{1.978105in}}{\pgfqpoint{1.896475in}{1.989155in}}%
\pgfpathcurveto{\pgfqpoint{1.896475in}{2.000206in}}{\pgfqpoint{1.892085in}{2.010805in}}{\pgfqpoint{1.884271in}{2.018618in}}%
\pgfpathcurveto{\pgfqpoint{1.876458in}{2.026432in}}{\pgfqpoint{1.865859in}{2.030822in}}{\pgfqpoint{1.854809in}{2.030822in}}%
\pgfpathcurveto{\pgfqpoint{1.843759in}{2.030822in}}{\pgfqpoint{1.833159in}{2.026432in}}{\pgfqpoint{1.825346in}{2.018618in}}%
\pgfpathcurveto{\pgfqpoint{1.817532in}{2.010805in}}{\pgfqpoint{1.813142in}{2.000206in}}{\pgfqpoint{1.813142in}{1.989155in}}%
\pgfpathcurveto{\pgfqpoint{1.813142in}{1.978105in}}{\pgfqpoint{1.817532in}{1.967506in}}{\pgfqpoint{1.825346in}{1.959693in}}%
\pgfpathcurveto{\pgfqpoint{1.833159in}{1.951879in}}{\pgfqpoint{1.843759in}{1.947489in}}{\pgfqpoint{1.854809in}{1.947489in}}%
\pgfpathclose%
\pgfusepath{stroke,fill}%
\end{pgfscope}%
\begin{pgfscope}%
\pgfpathrectangle{\pgfqpoint{0.600000in}{0.600000in}}{\pgfqpoint{3.900000in}{3.900000in}}%
\pgfusepath{clip}%
\pgfsetbuttcap%
\pgfsetroundjoin%
\definecolor{currentfill}{rgb}{0.121569,0.466667,0.705882}%
\pgfsetfillcolor{currentfill}%
\pgfsetlinewidth{1.003750pt}%
\definecolor{currentstroke}{rgb}{0.121569,0.466667,0.705882}%
\pgfsetstrokecolor{currentstroke}%
\pgfsetdash{}{0pt}%
\pgfpathmoveto{\pgfqpoint{2.015725in}{1.371002in}}%
\pgfpathcurveto{\pgfqpoint{2.026775in}{1.371002in}}{\pgfqpoint{2.037374in}{1.375393in}}{\pgfqpoint{2.045188in}{1.383206in}}%
\pgfpathcurveto{\pgfqpoint{2.053001in}{1.391020in}}{\pgfqpoint{2.057391in}{1.401619in}}{\pgfqpoint{2.057391in}{1.412669in}}%
\pgfpathcurveto{\pgfqpoint{2.057391in}{1.423719in}}{\pgfqpoint{2.053001in}{1.434318in}}{\pgfqpoint{2.045188in}{1.442132in}}%
\pgfpathcurveto{\pgfqpoint{2.037374in}{1.449945in}}{\pgfqpoint{2.026775in}{1.454336in}}{\pgfqpoint{2.015725in}{1.454336in}}%
\pgfpathcurveto{\pgfqpoint{2.004675in}{1.454336in}}{\pgfqpoint{1.994076in}{1.449945in}}{\pgfqpoint{1.986262in}{1.442132in}}%
\pgfpathcurveto{\pgfqpoint{1.978448in}{1.434318in}}{\pgfqpoint{1.974058in}{1.423719in}}{\pgfqpoint{1.974058in}{1.412669in}}%
\pgfpathcurveto{\pgfqpoint{1.974058in}{1.401619in}}{\pgfqpoint{1.978448in}{1.391020in}}{\pgfqpoint{1.986262in}{1.383206in}}%
\pgfpathcurveto{\pgfqpoint{1.994076in}{1.375393in}}{\pgfqpoint{2.004675in}{1.371002in}}{\pgfqpoint{2.015725in}{1.371002in}}%
\pgfpathclose%
\pgfusepath{stroke,fill}%
\end{pgfscope}%
\begin{pgfscope}%
\pgfpathrectangle{\pgfqpoint{0.600000in}{0.600000in}}{\pgfqpoint{3.900000in}{3.900000in}}%
\pgfusepath{clip}%
\pgfsetbuttcap%
\pgfsetroundjoin%
\definecolor{currentfill}{rgb}{0.121569,0.466667,0.705882}%
\pgfsetfillcolor{currentfill}%
\pgfsetlinewidth{1.003750pt}%
\definecolor{currentstroke}{rgb}{0.121569,0.466667,0.705882}%
\pgfsetstrokecolor{currentstroke}%
\pgfsetdash{}{0pt}%
\pgfpathmoveto{\pgfqpoint{1.780197in}{2.023907in}}%
\pgfpathcurveto{\pgfqpoint{1.791247in}{2.023907in}}{\pgfqpoint{1.801846in}{2.028297in}}{\pgfqpoint{1.809660in}{2.036111in}}%
\pgfpathcurveto{\pgfqpoint{1.817473in}{2.043925in}}{\pgfqpoint{1.821864in}{2.054524in}}{\pgfqpoint{1.821864in}{2.065574in}}%
\pgfpathcurveto{\pgfqpoint{1.821864in}{2.076624in}}{\pgfqpoint{1.817473in}{2.087223in}}{\pgfqpoint{1.809660in}{2.095037in}}%
\pgfpathcurveto{\pgfqpoint{1.801846in}{2.102850in}}{\pgfqpoint{1.791247in}{2.107241in}}{\pgfqpoint{1.780197in}{2.107241in}}%
\pgfpathcurveto{\pgfqpoint{1.769147in}{2.107241in}}{\pgfqpoint{1.758548in}{2.102850in}}{\pgfqpoint{1.750734in}{2.095037in}}%
\pgfpathcurveto{\pgfqpoint{1.742920in}{2.087223in}}{\pgfqpoint{1.738530in}{2.076624in}}{\pgfqpoint{1.738530in}{2.065574in}}%
\pgfpathcurveto{\pgfqpoint{1.738530in}{2.054524in}}{\pgfqpoint{1.742920in}{2.043925in}}{\pgfqpoint{1.750734in}{2.036111in}}%
\pgfpathcurveto{\pgfqpoint{1.758548in}{2.028297in}}{\pgfqpoint{1.769147in}{2.023907in}}{\pgfqpoint{1.780197in}{2.023907in}}%
\pgfpathclose%
\pgfusepath{stroke,fill}%
\end{pgfscope}%
\begin{pgfscope}%
\pgfpathrectangle{\pgfqpoint{0.600000in}{0.600000in}}{\pgfqpoint{3.900000in}{3.900000in}}%
\pgfusepath{clip}%
\pgfsetbuttcap%
\pgfsetroundjoin%
\definecolor{currentfill}{rgb}{0.121569,0.466667,0.705882}%
\pgfsetfillcolor{currentfill}%
\pgfsetlinewidth{1.003750pt}%
\definecolor{currentstroke}{rgb}{0.121569,0.466667,0.705882}%
\pgfsetstrokecolor{currentstroke}%
\pgfsetdash{}{0pt}%
\pgfpathmoveto{\pgfqpoint{2.254194in}{2.137140in}}%
\pgfpathcurveto{\pgfqpoint{2.265244in}{2.137140in}}{\pgfqpoint{2.275843in}{2.141531in}}{\pgfqpoint{2.283657in}{2.149344in}}%
\pgfpathcurveto{\pgfqpoint{2.291470in}{2.157158in}}{\pgfqpoint{2.295861in}{2.167757in}}{\pgfqpoint{2.295861in}{2.178807in}}%
\pgfpathcurveto{\pgfqpoint{2.295861in}{2.189857in}}{\pgfqpoint{2.291470in}{2.200456in}}{\pgfqpoint{2.283657in}{2.208270in}}%
\pgfpathcurveto{\pgfqpoint{2.275843in}{2.216083in}}{\pgfqpoint{2.265244in}{2.220474in}}{\pgfqpoint{2.254194in}{2.220474in}}%
\pgfpathcurveto{\pgfqpoint{2.243144in}{2.220474in}}{\pgfqpoint{2.232545in}{2.216083in}}{\pgfqpoint{2.224731in}{2.208270in}}%
\pgfpathcurveto{\pgfqpoint{2.216918in}{2.200456in}}{\pgfqpoint{2.212527in}{2.189857in}}{\pgfqpoint{2.212527in}{2.178807in}}%
\pgfpathcurveto{\pgfqpoint{2.212527in}{2.167757in}}{\pgfqpoint{2.216918in}{2.157158in}}{\pgfqpoint{2.224731in}{2.149344in}}%
\pgfpathcurveto{\pgfqpoint{2.232545in}{2.141531in}}{\pgfqpoint{2.243144in}{2.137140in}}{\pgfqpoint{2.254194in}{2.137140in}}%
\pgfpathclose%
\pgfusepath{stroke,fill}%
\end{pgfscope}%
\begin{pgfscope}%
\pgfpathrectangle{\pgfqpoint{0.600000in}{0.600000in}}{\pgfqpoint{3.900000in}{3.900000in}}%
\pgfusepath{clip}%
\pgfsetbuttcap%
\pgfsetroundjoin%
\definecolor{currentfill}{rgb}{0.121569,0.466667,0.705882}%
\pgfsetfillcolor{currentfill}%
\pgfsetlinewidth{1.003750pt}%
\definecolor{currentstroke}{rgb}{0.121569,0.466667,0.705882}%
\pgfsetstrokecolor{currentstroke}%
\pgfsetdash{}{0pt}%
\pgfpathmoveto{\pgfqpoint{3.475322in}{3.160574in}}%
\pgfpathcurveto{\pgfqpoint{3.486372in}{3.160574in}}{\pgfqpoint{3.496971in}{3.164965in}}{\pgfqpoint{3.504785in}{3.172778in}}%
\pgfpathcurveto{\pgfqpoint{3.512599in}{3.180592in}}{\pgfqpoint{3.516989in}{3.191191in}}{\pgfqpoint{3.516989in}{3.202241in}}%
\pgfpathcurveto{\pgfqpoint{3.516989in}{3.213291in}}{\pgfqpoint{3.512599in}{3.223890in}}{\pgfqpoint{3.504785in}{3.231704in}}%
\pgfpathcurveto{\pgfqpoint{3.496971in}{3.239517in}}{\pgfqpoint{3.486372in}{3.243908in}}{\pgfqpoint{3.475322in}{3.243908in}}%
\pgfpathcurveto{\pgfqpoint{3.464272in}{3.243908in}}{\pgfqpoint{3.453673in}{3.239517in}}{\pgfqpoint{3.445860in}{3.231704in}}%
\pgfpathcurveto{\pgfqpoint{3.438046in}{3.223890in}}{\pgfqpoint{3.433656in}{3.213291in}}{\pgfqpoint{3.433656in}{3.202241in}}%
\pgfpathcurveto{\pgfqpoint{3.433656in}{3.191191in}}{\pgfqpoint{3.438046in}{3.180592in}}{\pgfqpoint{3.445860in}{3.172778in}}%
\pgfpathcurveto{\pgfqpoint{3.453673in}{3.164965in}}{\pgfqpoint{3.464272in}{3.160574in}}{\pgfqpoint{3.475322in}{3.160574in}}%
\pgfpathclose%
\pgfusepath{stroke,fill}%
\end{pgfscope}%
\begin{pgfscope}%
\pgfpathrectangle{\pgfqpoint{0.600000in}{0.600000in}}{\pgfqpoint{3.900000in}{3.900000in}}%
\pgfusepath{clip}%
\pgfsetbuttcap%
\pgfsetroundjoin%
\definecolor{currentfill}{rgb}{0.121569,0.466667,0.705882}%
\pgfsetfillcolor{currentfill}%
\pgfsetlinewidth{1.003750pt}%
\definecolor{currentstroke}{rgb}{0.121569,0.466667,0.705882}%
\pgfsetstrokecolor{currentstroke}%
\pgfsetdash{}{0pt}%
\pgfpathmoveto{\pgfqpoint{1.803324in}{2.397068in}}%
\pgfpathcurveto{\pgfqpoint{1.814374in}{2.397068in}}{\pgfqpoint{1.824973in}{2.401458in}}{\pgfqpoint{1.832787in}{2.409272in}}%
\pgfpathcurveto{\pgfqpoint{1.840601in}{2.417085in}}{\pgfqpoint{1.844991in}{2.427684in}}{\pgfqpoint{1.844991in}{2.438735in}}%
\pgfpathcurveto{\pgfqpoint{1.844991in}{2.449785in}}{\pgfqpoint{1.840601in}{2.460384in}}{\pgfqpoint{1.832787in}{2.468197in}}%
\pgfpathcurveto{\pgfqpoint{1.824973in}{2.476011in}}{\pgfqpoint{1.814374in}{2.480401in}}{\pgfqpoint{1.803324in}{2.480401in}}%
\pgfpathcurveto{\pgfqpoint{1.792274in}{2.480401in}}{\pgfqpoint{1.781675in}{2.476011in}}{\pgfqpoint{1.773861in}{2.468197in}}%
\pgfpathcurveto{\pgfqpoint{1.766048in}{2.460384in}}{\pgfqpoint{1.761657in}{2.449785in}}{\pgfqpoint{1.761657in}{2.438735in}}%
\pgfpathcurveto{\pgfqpoint{1.761657in}{2.427684in}}{\pgfqpoint{1.766048in}{2.417085in}}{\pgfqpoint{1.773861in}{2.409272in}}%
\pgfpathcurveto{\pgfqpoint{1.781675in}{2.401458in}}{\pgfqpoint{1.792274in}{2.397068in}}{\pgfqpoint{1.803324in}{2.397068in}}%
\pgfpathclose%
\pgfusepath{stroke,fill}%
\end{pgfscope}%
\begin{pgfscope}%
\pgfpathrectangle{\pgfqpoint{0.600000in}{0.600000in}}{\pgfqpoint{3.900000in}{3.900000in}}%
\pgfusepath{clip}%
\pgfsetbuttcap%
\pgfsetroundjoin%
\definecolor{currentfill}{rgb}{0.121569,0.466667,0.705882}%
\pgfsetfillcolor{currentfill}%
\pgfsetlinewidth{1.003750pt}%
\definecolor{currentstroke}{rgb}{0.121569,0.466667,0.705882}%
\pgfsetstrokecolor{currentstroke}%
\pgfsetdash{}{0pt}%
\pgfpathmoveto{\pgfqpoint{2.633994in}{2.845184in}}%
\pgfpathcurveto{\pgfqpoint{2.645044in}{2.845184in}}{\pgfqpoint{2.655643in}{2.849574in}}{\pgfqpoint{2.663456in}{2.857388in}}%
\pgfpathcurveto{\pgfqpoint{2.671270in}{2.865201in}}{\pgfqpoint{2.675660in}{2.875800in}}{\pgfqpoint{2.675660in}{2.886850in}}%
\pgfpathcurveto{\pgfqpoint{2.675660in}{2.897900in}}{\pgfqpoint{2.671270in}{2.908500in}}{\pgfqpoint{2.663456in}{2.916313in}}%
\pgfpathcurveto{\pgfqpoint{2.655643in}{2.924127in}}{\pgfqpoint{2.645044in}{2.928517in}}{\pgfqpoint{2.633994in}{2.928517in}}%
\pgfpathcurveto{\pgfqpoint{2.622943in}{2.928517in}}{\pgfqpoint{2.612344in}{2.924127in}}{\pgfqpoint{2.604531in}{2.916313in}}%
\pgfpathcurveto{\pgfqpoint{2.596717in}{2.908500in}}{\pgfqpoint{2.592327in}{2.897900in}}{\pgfqpoint{2.592327in}{2.886850in}}%
\pgfpathcurveto{\pgfqpoint{2.592327in}{2.875800in}}{\pgfqpoint{2.596717in}{2.865201in}}{\pgfqpoint{2.604531in}{2.857388in}}%
\pgfpathcurveto{\pgfqpoint{2.612344in}{2.849574in}}{\pgfqpoint{2.622943in}{2.845184in}}{\pgfqpoint{2.633994in}{2.845184in}}%
\pgfpathclose%
\pgfusepath{stroke,fill}%
\end{pgfscope}%
\begin{pgfscope}%
\pgfpathrectangle{\pgfqpoint{0.600000in}{0.600000in}}{\pgfqpoint{3.900000in}{3.900000in}}%
\pgfusepath{clip}%
\pgfsetbuttcap%
\pgfsetroundjoin%
\definecolor{currentfill}{rgb}{0.121569,0.466667,0.705882}%
\pgfsetfillcolor{currentfill}%
\pgfsetlinewidth{1.003750pt}%
\definecolor{currentstroke}{rgb}{0.121569,0.466667,0.705882}%
\pgfsetstrokecolor{currentstroke}%
\pgfsetdash{}{0pt}%
\pgfpathmoveto{\pgfqpoint{2.226986in}{2.440665in}}%
\pgfpathcurveto{\pgfqpoint{2.238037in}{2.440665in}}{\pgfqpoint{2.248636in}{2.445055in}}{\pgfqpoint{2.256449in}{2.452868in}}%
\pgfpathcurveto{\pgfqpoint{2.264263in}{2.460682in}}{\pgfqpoint{2.268653in}{2.471281in}}{\pgfqpoint{2.268653in}{2.482331in}}%
\pgfpathcurveto{\pgfqpoint{2.268653in}{2.493381in}}{\pgfqpoint{2.264263in}{2.503980in}}{\pgfqpoint{2.256449in}{2.511794in}}%
\pgfpathcurveto{\pgfqpoint{2.248636in}{2.519608in}}{\pgfqpoint{2.238037in}{2.523998in}}{\pgfqpoint{2.226986in}{2.523998in}}%
\pgfpathcurveto{\pgfqpoint{2.215936in}{2.523998in}}{\pgfqpoint{2.205337in}{2.519608in}}{\pgfqpoint{2.197524in}{2.511794in}}%
\pgfpathcurveto{\pgfqpoint{2.189710in}{2.503980in}}{\pgfqpoint{2.185320in}{2.493381in}}{\pgfqpoint{2.185320in}{2.482331in}}%
\pgfpathcurveto{\pgfqpoint{2.185320in}{2.471281in}}{\pgfqpoint{2.189710in}{2.460682in}}{\pgfqpoint{2.197524in}{2.452868in}}%
\pgfpathcurveto{\pgfqpoint{2.205337in}{2.445055in}}{\pgfqpoint{2.215936in}{2.440665in}}{\pgfqpoint{2.226986in}{2.440665in}}%
\pgfpathclose%
\pgfusepath{stroke,fill}%
\end{pgfscope}%
\begin{pgfscope}%
\pgfpathrectangle{\pgfqpoint{0.600000in}{0.600000in}}{\pgfqpoint{3.900000in}{3.900000in}}%
\pgfusepath{clip}%
\pgfsetbuttcap%
\pgfsetroundjoin%
\definecolor{currentfill}{rgb}{0.121569,0.466667,0.705882}%
\pgfsetfillcolor{currentfill}%
\pgfsetlinewidth{1.003750pt}%
\definecolor{currentstroke}{rgb}{0.121569,0.466667,0.705882}%
\pgfsetstrokecolor{currentstroke}%
\pgfsetdash{}{0pt}%
\pgfpathmoveto{\pgfqpoint{2.471120in}{0.845301in}}%
\pgfpathcurveto{\pgfqpoint{2.482171in}{0.845301in}}{\pgfqpoint{2.492770in}{0.849692in}}{\pgfqpoint{2.500583in}{0.857505in}}%
\pgfpathcurveto{\pgfqpoint{2.508397in}{0.865319in}}{\pgfqpoint{2.512787in}{0.875918in}}{\pgfqpoint{2.512787in}{0.886968in}}%
\pgfpathcurveto{\pgfqpoint{2.512787in}{0.898018in}}{\pgfqpoint{2.508397in}{0.908617in}}{\pgfqpoint{2.500583in}{0.916431in}}%
\pgfpathcurveto{\pgfqpoint{2.492770in}{0.924244in}}{\pgfqpoint{2.482171in}{0.928635in}}{\pgfqpoint{2.471120in}{0.928635in}}%
\pgfpathcurveto{\pgfqpoint{2.460070in}{0.928635in}}{\pgfqpoint{2.449471in}{0.924244in}}{\pgfqpoint{2.441658in}{0.916431in}}%
\pgfpathcurveto{\pgfqpoint{2.433844in}{0.908617in}}{\pgfqpoint{2.429454in}{0.898018in}}{\pgfqpoint{2.429454in}{0.886968in}}%
\pgfpathcurveto{\pgfqpoint{2.429454in}{0.875918in}}{\pgfqpoint{2.433844in}{0.865319in}}{\pgfqpoint{2.441658in}{0.857505in}}%
\pgfpathcurveto{\pgfqpoint{2.449471in}{0.849692in}}{\pgfqpoint{2.460070in}{0.845301in}}{\pgfqpoint{2.471120in}{0.845301in}}%
\pgfpathclose%
\pgfusepath{stroke,fill}%
\end{pgfscope}%
\begin{pgfscope}%
\pgfpathrectangle{\pgfqpoint{0.600000in}{0.600000in}}{\pgfqpoint{3.900000in}{3.900000in}}%
\pgfusepath{clip}%
\pgfsetbuttcap%
\pgfsetroundjoin%
\definecolor{currentfill}{rgb}{0.121569,0.466667,0.705882}%
\pgfsetfillcolor{currentfill}%
\pgfsetlinewidth{1.003750pt}%
\definecolor{currentstroke}{rgb}{0.121569,0.466667,0.705882}%
\pgfsetstrokecolor{currentstroke}%
\pgfsetdash{}{0pt}%
\pgfpathmoveto{\pgfqpoint{2.708041in}{2.775305in}}%
\pgfpathcurveto{\pgfqpoint{2.719091in}{2.775305in}}{\pgfqpoint{2.729690in}{2.779695in}}{\pgfqpoint{2.737504in}{2.787509in}}%
\pgfpathcurveto{\pgfqpoint{2.745317in}{2.795323in}}{\pgfqpoint{2.749708in}{2.805922in}}{\pgfqpoint{2.749708in}{2.816972in}}%
\pgfpathcurveto{\pgfqpoint{2.749708in}{2.828022in}}{\pgfqpoint{2.745317in}{2.838621in}}{\pgfqpoint{2.737504in}{2.846434in}}%
\pgfpathcurveto{\pgfqpoint{2.729690in}{2.854248in}}{\pgfqpoint{2.719091in}{2.858638in}}{\pgfqpoint{2.708041in}{2.858638in}}%
\pgfpathcurveto{\pgfqpoint{2.696991in}{2.858638in}}{\pgfqpoint{2.686392in}{2.854248in}}{\pgfqpoint{2.678578in}{2.846434in}}%
\pgfpathcurveto{\pgfqpoint{2.670765in}{2.838621in}}{\pgfqpoint{2.666374in}{2.828022in}}{\pgfqpoint{2.666374in}{2.816972in}}%
\pgfpathcurveto{\pgfqpoint{2.666374in}{2.805922in}}{\pgfqpoint{2.670765in}{2.795323in}}{\pgfqpoint{2.678578in}{2.787509in}}%
\pgfpathcurveto{\pgfqpoint{2.686392in}{2.779695in}}{\pgfqpoint{2.696991in}{2.775305in}}{\pgfqpoint{2.708041in}{2.775305in}}%
\pgfpathclose%
\pgfusepath{stroke,fill}%
\end{pgfscope}%
\begin{pgfscope}%
\pgfpathrectangle{\pgfqpoint{0.600000in}{0.600000in}}{\pgfqpoint{3.900000in}{3.900000in}}%
\pgfusepath{clip}%
\pgfsetbuttcap%
\pgfsetroundjoin%
\definecolor{currentfill}{rgb}{0.121569,0.466667,0.705882}%
\pgfsetfillcolor{currentfill}%
\pgfsetlinewidth{1.003750pt}%
\definecolor{currentstroke}{rgb}{0.121569,0.466667,0.705882}%
\pgfsetstrokecolor{currentstroke}%
\pgfsetdash{}{0pt}%
\pgfpathmoveto{\pgfqpoint{2.646331in}{2.307093in}}%
\pgfpathcurveto{\pgfqpoint{2.657381in}{2.307093in}}{\pgfqpoint{2.667980in}{2.311484in}}{\pgfqpoint{2.675794in}{2.319297in}}%
\pgfpathcurveto{\pgfqpoint{2.683607in}{2.327111in}}{\pgfqpoint{2.687998in}{2.337710in}}{\pgfqpoint{2.687998in}{2.348760in}}%
\pgfpathcurveto{\pgfqpoint{2.687998in}{2.359810in}}{\pgfqpoint{2.683607in}{2.370409in}}{\pgfqpoint{2.675794in}{2.378223in}}%
\pgfpathcurveto{\pgfqpoint{2.667980in}{2.386037in}}{\pgfqpoint{2.657381in}{2.390427in}}{\pgfqpoint{2.646331in}{2.390427in}}%
\pgfpathcurveto{\pgfqpoint{2.635281in}{2.390427in}}{\pgfqpoint{2.624682in}{2.386037in}}{\pgfqpoint{2.616868in}{2.378223in}}%
\pgfpathcurveto{\pgfqpoint{2.609055in}{2.370409in}}{\pgfqpoint{2.604664in}{2.359810in}}{\pgfqpoint{2.604664in}{2.348760in}}%
\pgfpathcurveto{\pgfqpoint{2.604664in}{2.337710in}}{\pgfqpoint{2.609055in}{2.327111in}}{\pgfqpoint{2.616868in}{2.319297in}}%
\pgfpathcurveto{\pgfqpoint{2.624682in}{2.311484in}}{\pgfqpoint{2.635281in}{2.307093in}}{\pgfqpoint{2.646331in}{2.307093in}}%
\pgfpathclose%
\pgfusepath{stroke,fill}%
\end{pgfscope}%
\begin{pgfscope}%
\pgfpathrectangle{\pgfqpoint{0.600000in}{0.600000in}}{\pgfqpoint{3.900000in}{3.900000in}}%
\pgfusepath{clip}%
\pgfsetbuttcap%
\pgfsetroundjoin%
\definecolor{currentfill}{rgb}{0.121569,0.466667,0.705882}%
\pgfsetfillcolor{currentfill}%
\pgfsetlinewidth{1.003750pt}%
\definecolor{currentstroke}{rgb}{0.121569,0.466667,0.705882}%
\pgfsetstrokecolor{currentstroke}%
\pgfsetdash{}{0pt}%
\pgfpathmoveto{\pgfqpoint{3.946406in}{3.756072in}}%
\pgfpathcurveto{\pgfqpoint{3.957456in}{3.756072in}}{\pgfqpoint{3.968055in}{3.760462in}}{\pgfqpoint{3.975869in}{3.768276in}}%
\pgfpathcurveto{\pgfqpoint{3.983682in}{3.776090in}}{\pgfqpoint{3.988073in}{3.786689in}}{\pgfqpoint{3.988073in}{3.797739in}}%
\pgfpathcurveto{\pgfqpoint{3.988073in}{3.808789in}}{\pgfqpoint{3.983682in}{3.819388in}}{\pgfqpoint{3.975869in}{3.827201in}}%
\pgfpathcurveto{\pgfqpoint{3.968055in}{3.835015in}}{\pgfqpoint{3.957456in}{3.839405in}}{\pgfqpoint{3.946406in}{3.839405in}}%
\pgfpathcurveto{\pgfqpoint{3.935356in}{3.839405in}}{\pgfqpoint{3.924757in}{3.835015in}}{\pgfqpoint{3.916943in}{3.827201in}}%
\pgfpathcurveto{\pgfqpoint{3.909130in}{3.819388in}}{\pgfqpoint{3.904739in}{3.808789in}}{\pgfqpoint{3.904739in}{3.797739in}}%
\pgfpathcurveto{\pgfqpoint{3.904739in}{3.786689in}}{\pgfqpoint{3.909130in}{3.776090in}}{\pgfqpoint{3.916943in}{3.768276in}}%
\pgfpathcurveto{\pgfqpoint{3.924757in}{3.760462in}}{\pgfqpoint{3.935356in}{3.756072in}}{\pgfqpoint{3.946406in}{3.756072in}}%
\pgfpathclose%
\pgfusepath{stroke,fill}%
\end{pgfscope}%
\begin{pgfscope}%
\pgfpathrectangle{\pgfqpoint{0.600000in}{0.600000in}}{\pgfqpoint{3.900000in}{3.900000in}}%
\pgfusepath{clip}%
\pgfsetbuttcap%
\pgfsetroundjoin%
\definecolor{currentfill}{rgb}{0.121569,0.466667,0.705882}%
\pgfsetfillcolor{currentfill}%
\pgfsetlinewidth{1.003750pt}%
\definecolor{currentstroke}{rgb}{0.121569,0.466667,0.705882}%
\pgfsetstrokecolor{currentstroke}%
\pgfsetdash{}{0pt}%
\pgfpathmoveto{\pgfqpoint{2.248640in}{2.349503in}}%
\pgfpathcurveto{\pgfqpoint{2.259690in}{2.349503in}}{\pgfqpoint{2.270289in}{2.353893in}}{\pgfqpoint{2.278102in}{2.361706in}}%
\pgfpathcurveto{\pgfqpoint{2.285916in}{2.369520in}}{\pgfqpoint{2.290306in}{2.380119in}}{\pgfqpoint{2.290306in}{2.391169in}}%
\pgfpathcurveto{\pgfqpoint{2.290306in}{2.402219in}}{\pgfqpoint{2.285916in}{2.412818in}}{\pgfqpoint{2.278102in}{2.420632in}}%
\pgfpathcurveto{\pgfqpoint{2.270289in}{2.428446in}}{\pgfqpoint{2.259690in}{2.432836in}}{\pgfqpoint{2.248640in}{2.432836in}}%
\pgfpathcurveto{\pgfqpoint{2.237590in}{2.432836in}}{\pgfqpoint{2.226991in}{2.428446in}}{\pgfqpoint{2.219177in}{2.420632in}}%
\pgfpathcurveto{\pgfqpoint{2.211363in}{2.412818in}}{\pgfqpoint{2.206973in}{2.402219in}}{\pgfqpoint{2.206973in}{2.391169in}}%
\pgfpathcurveto{\pgfqpoint{2.206973in}{2.380119in}}{\pgfqpoint{2.211363in}{2.369520in}}{\pgfqpoint{2.219177in}{2.361706in}}%
\pgfpathcurveto{\pgfqpoint{2.226991in}{2.353893in}}{\pgfqpoint{2.237590in}{2.349503in}}{\pgfqpoint{2.248640in}{2.349503in}}%
\pgfpathclose%
\pgfusepath{stroke,fill}%
\end{pgfscope}%
\begin{pgfscope}%
\pgfpathrectangle{\pgfqpoint{0.600000in}{0.600000in}}{\pgfqpoint{3.900000in}{3.900000in}}%
\pgfusepath{clip}%
\pgfsetbuttcap%
\pgfsetroundjoin%
\definecolor{currentfill}{rgb}{0.121569,0.466667,0.705882}%
\pgfsetfillcolor{currentfill}%
\pgfsetlinewidth{1.003750pt}%
\definecolor{currentstroke}{rgb}{0.121569,0.466667,0.705882}%
\pgfsetstrokecolor{currentstroke}%
\pgfsetdash{}{0pt}%
\pgfpathmoveto{\pgfqpoint{2.514300in}{2.376492in}}%
\pgfpathcurveto{\pgfqpoint{2.525350in}{2.376492in}}{\pgfqpoint{2.535949in}{2.380882in}}{\pgfqpoint{2.543762in}{2.388696in}}%
\pgfpathcurveto{\pgfqpoint{2.551576in}{2.396509in}}{\pgfqpoint{2.555966in}{2.407108in}}{\pgfqpoint{2.555966in}{2.418158in}}%
\pgfpathcurveto{\pgfqpoint{2.555966in}{2.429208in}}{\pgfqpoint{2.551576in}{2.439807in}}{\pgfqpoint{2.543762in}{2.447621in}}%
\pgfpathcurveto{\pgfqpoint{2.535949in}{2.455435in}}{\pgfqpoint{2.525350in}{2.459825in}}{\pgfqpoint{2.514300in}{2.459825in}}%
\pgfpathcurveto{\pgfqpoint{2.503250in}{2.459825in}}{\pgfqpoint{2.492651in}{2.455435in}}{\pgfqpoint{2.484837in}{2.447621in}}%
\pgfpathcurveto{\pgfqpoint{2.477023in}{2.439807in}}{\pgfqpoint{2.472633in}{2.429208in}}{\pgfqpoint{2.472633in}{2.418158in}}%
\pgfpathcurveto{\pgfqpoint{2.472633in}{2.407108in}}{\pgfqpoint{2.477023in}{2.396509in}}{\pgfqpoint{2.484837in}{2.388696in}}%
\pgfpathcurveto{\pgfqpoint{2.492651in}{2.380882in}}{\pgfqpoint{2.503250in}{2.376492in}}{\pgfqpoint{2.514300in}{2.376492in}}%
\pgfpathclose%
\pgfusepath{stroke,fill}%
\end{pgfscope}%
\begin{pgfscope}%
\pgfpathrectangle{\pgfqpoint{0.600000in}{0.600000in}}{\pgfqpoint{3.900000in}{3.900000in}}%
\pgfusepath{clip}%
\pgfsetbuttcap%
\pgfsetroundjoin%
\definecolor{currentfill}{rgb}{0.121569,0.466667,0.705882}%
\pgfsetfillcolor{currentfill}%
\pgfsetlinewidth{1.003750pt}%
\definecolor{currentstroke}{rgb}{0.121569,0.466667,0.705882}%
\pgfsetstrokecolor{currentstroke}%
\pgfsetdash{}{0pt}%
\pgfpathmoveto{\pgfqpoint{2.597022in}{2.403167in}}%
\pgfpathcurveto{\pgfqpoint{2.608072in}{2.403167in}}{\pgfqpoint{2.618671in}{2.407557in}}{\pgfqpoint{2.626485in}{2.415370in}}%
\pgfpathcurveto{\pgfqpoint{2.634298in}{2.423184in}}{\pgfqpoint{2.638689in}{2.433783in}}{\pgfqpoint{2.638689in}{2.444833in}}%
\pgfpathcurveto{\pgfqpoint{2.638689in}{2.455883in}}{\pgfqpoint{2.634298in}{2.466482in}}{\pgfqpoint{2.626485in}{2.474296in}}%
\pgfpathcurveto{\pgfqpoint{2.618671in}{2.482110in}}{\pgfqpoint{2.608072in}{2.486500in}}{\pgfqpoint{2.597022in}{2.486500in}}%
\pgfpathcurveto{\pgfqpoint{2.585972in}{2.486500in}}{\pgfqpoint{2.575373in}{2.482110in}}{\pgfqpoint{2.567559in}{2.474296in}}%
\pgfpathcurveto{\pgfqpoint{2.559745in}{2.466482in}}{\pgfqpoint{2.555355in}{2.455883in}}{\pgfqpoint{2.555355in}{2.444833in}}%
\pgfpathcurveto{\pgfqpoint{2.555355in}{2.433783in}}{\pgfqpoint{2.559745in}{2.423184in}}{\pgfqpoint{2.567559in}{2.415370in}}%
\pgfpathcurveto{\pgfqpoint{2.575373in}{2.407557in}}{\pgfqpoint{2.585972in}{2.403167in}}{\pgfqpoint{2.597022in}{2.403167in}}%
\pgfpathclose%
\pgfusepath{stroke,fill}%
\end{pgfscope}%
\begin{pgfscope}%
\pgfpathrectangle{\pgfqpoint{0.600000in}{0.600000in}}{\pgfqpoint{3.900000in}{3.900000in}}%
\pgfusepath{clip}%
\pgfsetbuttcap%
\pgfsetroundjoin%
\definecolor{currentfill}{rgb}{0.121569,0.466667,0.705882}%
\pgfsetfillcolor{currentfill}%
\pgfsetlinewidth{1.003750pt}%
\definecolor{currentstroke}{rgb}{0.121569,0.466667,0.705882}%
\pgfsetstrokecolor{currentstroke}%
\pgfsetdash{}{0pt}%
\pgfpathmoveto{\pgfqpoint{2.612347in}{2.836044in}}%
\pgfpathcurveto{\pgfqpoint{2.623398in}{2.836044in}}{\pgfqpoint{2.633997in}{2.840434in}}{\pgfqpoint{2.641810in}{2.848247in}}%
\pgfpathcurveto{\pgfqpoint{2.649624in}{2.856061in}}{\pgfqpoint{2.654014in}{2.866660in}}{\pgfqpoint{2.654014in}{2.877710in}}%
\pgfpathcurveto{\pgfqpoint{2.654014in}{2.888760in}}{\pgfqpoint{2.649624in}{2.899359in}}{\pgfqpoint{2.641810in}{2.907173in}}%
\pgfpathcurveto{\pgfqpoint{2.633997in}{2.914987in}}{\pgfqpoint{2.623398in}{2.919377in}}{\pgfqpoint{2.612347in}{2.919377in}}%
\pgfpathcurveto{\pgfqpoint{2.601297in}{2.919377in}}{\pgfqpoint{2.590698in}{2.914987in}}{\pgfqpoint{2.582885in}{2.907173in}}%
\pgfpathcurveto{\pgfqpoint{2.575071in}{2.899359in}}{\pgfqpoint{2.570681in}{2.888760in}}{\pgfqpoint{2.570681in}{2.877710in}}%
\pgfpathcurveto{\pgfqpoint{2.570681in}{2.866660in}}{\pgfqpoint{2.575071in}{2.856061in}}{\pgfqpoint{2.582885in}{2.848247in}}%
\pgfpathcurveto{\pgfqpoint{2.590698in}{2.840434in}}{\pgfqpoint{2.601297in}{2.836044in}}{\pgfqpoint{2.612347in}{2.836044in}}%
\pgfpathclose%
\pgfusepath{stroke,fill}%
\end{pgfscope}%
\begin{pgfscope}%
\pgfpathrectangle{\pgfqpoint{0.600000in}{0.600000in}}{\pgfqpoint{3.900000in}{3.900000in}}%
\pgfusepath{clip}%
\pgfsetbuttcap%
\pgfsetroundjoin%
\definecolor{currentfill}{rgb}{0.121569,0.466667,0.705882}%
\pgfsetfillcolor{currentfill}%
\pgfsetlinewidth{1.003750pt}%
\definecolor{currentstroke}{rgb}{0.121569,0.466667,0.705882}%
\pgfsetstrokecolor{currentstroke}%
\pgfsetdash{}{0pt}%
\pgfpathmoveto{\pgfqpoint{3.142442in}{2.420736in}}%
\pgfpathcurveto{\pgfqpoint{3.153492in}{2.420736in}}{\pgfqpoint{3.164091in}{2.425126in}}{\pgfqpoint{3.171905in}{2.432940in}}%
\pgfpathcurveto{\pgfqpoint{3.179718in}{2.440754in}}{\pgfqpoint{3.184108in}{2.451353in}}{\pgfqpoint{3.184108in}{2.462403in}}%
\pgfpathcurveto{\pgfqpoint{3.184108in}{2.473453in}}{\pgfqpoint{3.179718in}{2.484052in}}{\pgfqpoint{3.171905in}{2.491866in}}%
\pgfpathcurveto{\pgfqpoint{3.164091in}{2.499679in}}{\pgfqpoint{3.153492in}{2.504069in}}{\pgfqpoint{3.142442in}{2.504069in}}%
\pgfpathcurveto{\pgfqpoint{3.131392in}{2.504069in}}{\pgfqpoint{3.120793in}{2.499679in}}{\pgfqpoint{3.112979in}{2.491866in}}%
\pgfpathcurveto{\pgfqpoint{3.105165in}{2.484052in}}{\pgfqpoint{3.100775in}{2.473453in}}{\pgfqpoint{3.100775in}{2.462403in}}%
\pgfpathcurveto{\pgfqpoint{3.100775in}{2.451353in}}{\pgfqpoint{3.105165in}{2.440754in}}{\pgfqpoint{3.112979in}{2.432940in}}%
\pgfpathcurveto{\pgfqpoint{3.120793in}{2.425126in}}{\pgfqpoint{3.131392in}{2.420736in}}{\pgfqpoint{3.142442in}{2.420736in}}%
\pgfpathclose%
\pgfusepath{stroke,fill}%
\end{pgfscope}%
\begin{pgfscope}%
\pgfpathrectangle{\pgfqpoint{0.600000in}{0.600000in}}{\pgfqpoint{3.900000in}{3.900000in}}%
\pgfusepath{clip}%
\pgfsetbuttcap%
\pgfsetroundjoin%
\definecolor{currentfill}{rgb}{0.121569,0.466667,0.705882}%
\pgfsetfillcolor{currentfill}%
\pgfsetlinewidth{1.003750pt}%
\definecolor{currentstroke}{rgb}{0.121569,0.466667,0.705882}%
\pgfsetstrokecolor{currentstroke}%
\pgfsetdash{}{0pt}%
\pgfpathmoveto{\pgfqpoint{2.700381in}{3.193044in}}%
\pgfpathcurveto{\pgfqpoint{2.711431in}{3.193044in}}{\pgfqpoint{2.722030in}{3.197434in}}{\pgfqpoint{2.729844in}{3.205248in}}%
\pgfpathcurveto{\pgfqpoint{2.737657in}{3.213061in}}{\pgfqpoint{2.742048in}{3.223660in}}{\pgfqpoint{2.742048in}{3.234710in}}%
\pgfpathcurveto{\pgfqpoint{2.742048in}{3.245760in}}{\pgfqpoint{2.737657in}{3.256360in}}{\pgfqpoint{2.729844in}{3.264173in}}%
\pgfpathcurveto{\pgfqpoint{2.722030in}{3.271987in}}{\pgfqpoint{2.711431in}{3.276377in}}{\pgfqpoint{2.700381in}{3.276377in}}%
\pgfpathcurveto{\pgfqpoint{2.689331in}{3.276377in}}{\pgfqpoint{2.678732in}{3.271987in}}{\pgfqpoint{2.670918in}{3.264173in}}%
\pgfpathcurveto{\pgfqpoint{2.663105in}{3.256360in}}{\pgfqpoint{2.658714in}{3.245760in}}{\pgfqpoint{2.658714in}{3.234710in}}%
\pgfpathcurveto{\pgfqpoint{2.658714in}{3.223660in}}{\pgfqpoint{2.663105in}{3.213061in}}{\pgfqpoint{2.670918in}{3.205248in}}%
\pgfpathcurveto{\pgfqpoint{2.678732in}{3.197434in}}{\pgfqpoint{2.689331in}{3.193044in}}{\pgfqpoint{2.700381in}{3.193044in}}%
\pgfpathclose%
\pgfusepath{stroke,fill}%
\end{pgfscope}%
\begin{pgfscope}%
\pgfpathrectangle{\pgfqpoint{0.600000in}{0.600000in}}{\pgfqpoint{3.900000in}{3.900000in}}%
\pgfusepath{clip}%
\pgfsetbuttcap%
\pgfsetroundjoin%
\definecolor{currentfill}{rgb}{0.121569,0.466667,0.705882}%
\pgfsetfillcolor{currentfill}%
\pgfsetlinewidth{1.003750pt}%
\definecolor{currentstroke}{rgb}{0.121569,0.466667,0.705882}%
\pgfsetstrokecolor{currentstroke}%
\pgfsetdash{}{0pt}%
\pgfpathmoveto{\pgfqpoint{2.967071in}{1.749008in}}%
\pgfpathcurveto{\pgfqpoint{2.978121in}{1.749008in}}{\pgfqpoint{2.988720in}{1.753399in}}{\pgfqpoint{2.996534in}{1.761212in}}%
\pgfpathcurveto{\pgfqpoint{3.004347in}{1.769026in}}{\pgfqpoint{3.008738in}{1.779625in}}{\pgfqpoint{3.008738in}{1.790675in}}%
\pgfpathcurveto{\pgfqpoint{3.008738in}{1.801725in}}{\pgfqpoint{3.004347in}{1.812324in}}{\pgfqpoint{2.996534in}{1.820138in}}%
\pgfpathcurveto{\pgfqpoint{2.988720in}{1.827951in}}{\pgfqpoint{2.978121in}{1.832342in}}{\pgfqpoint{2.967071in}{1.832342in}}%
\pgfpathcurveto{\pgfqpoint{2.956021in}{1.832342in}}{\pgfqpoint{2.945422in}{1.827951in}}{\pgfqpoint{2.937608in}{1.820138in}}%
\pgfpathcurveto{\pgfqpoint{2.929795in}{1.812324in}}{\pgfqpoint{2.925404in}{1.801725in}}{\pgfqpoint{2.925404in}{1.790675in}}%
\pgfpathcurveto{\pgfqpoint{2.925404in}{1.779625in}}{\pgfqpoint{2.929795in}{1.769026in}}{\pgfqpoint{2.937608in}{1.761212in}}%
\pgfpathcurveto{\pgfqpoint{2.945422in}{1.753399in}}{\pgfqpoint{2.956021in}{1.749008in}}{\pgfqpoint{2.967071in}{1.749008in}}%
\pgfpathclose%
\pgfusepath{stroke,fill}%
\end{pgfscope}%
\begin{pgfscope}%
\pgfpathrectangle{\pgfqpoint{0.600000in}{0.600000in}}{\pgfqpoint{3.900000in}{3.900000in}}%
\pgfusepath{clip}%
\pgfsetbuttcap%
\pgfsetroundjoin%
\definecolor{currentfill}{rgb}{0.121569,0.466667,0.705882}%
\pgfsetfillcolor{currentfill}%
\pgfsetlinewidth{1.003750pt}%
\definecolor{currentstroke}{rgb}{0.121569,0.466667,0.705882}%
\pgfsetstrokecolor{currentstroke}%
\pgfsetdash{}{0pt}%
\pgfpathmoveto{\pgfqpoint{2.638817in}{3.283247in}}%
\pgfpathcurveto{\pgfqpoint{2.649867in}{3.283247in}}{\pgfqpoint{2.660466in}{3.287638in}}{\pgfqpoint{2.668280in}{3.295451in}}%
\pgfpathcurveto{\pgfqpoint{2.676093in}{3.303265in}}{\pgfqpoint{2.680483in}{3.313864in}}{\pgfqpoint{2.680483in}{3.324914in}}%
\pgfpathcurveto{\pgfqpoint{2.680483in}{3.335964in}}{\pgfqpoint{2.676093in}{3.346563in}}{\pgfqpoint{2.668280in}{3.354377in}}%
\pgfpathcurveto{\pgfqpoint{2.660466in}{3.362191in}}{\pgfqpoint{2.649867in}{3.366581in}}{\pgfqpoint{2.638817in}{3.366581in}}%
\pgfpathcurveto{\pgfqpoint{2.627767in}{3.366581in}}{\pgfqpoint{2.617168in}{3.362191in}}{\pgfqpoint{2.609354in}{3.354377in}}%
\pgfpathcurveto{\pgfqpoint{2.601540in}{3.346563in}}{\pgfqpoint{2.597150in}{3.335964in}}{\pgfqpoint{2.597150in}{3.324914in}}%
\pgfpathcurveto{\pgfqpoint{2.597150in}{3.313864in}}{\pgfqpoint{2.601540in}{3.303265in}}{\pgfqpoint{2.609354in}{3.295451in}}%
\pgfpathcurveto{\pgfqpoint{2.617168in}{3.287638in}}{\pgfqpoint{2.627767in}{3.283247in}}{\pgfqpoint{2.638817in}{3.283247in}}%
\pgfpathclose%
\pgfusepath{stroke,fill}%
\end{pgfscope}%
\begin{pgfscope}%
\pgfpathrectangle{\pgfqpoint{0.600000in}{0.600000in}}{\pgfqpoint{3.900000in}{3.900000in}}%
\pgfusepath{clip}%
\pgfsetbuttcap%
\pgfsetroundjoin%
\definecolor{currentfill}{rgb}{0.121569,0.466667,0.705882}%
\pgfsetfillcolor{currentfill}%
\pgfsetlinewidth{1.003750pt}%
\definecolor{currentstroke}{rgb}{0.121569,0.466667,0.705882}%
\pgfsetstrokecolor{currentstroke}%
\pgfsetdash{}{0pt}%
\pgfpathmoveto{\pgfqpoint{2.260629in}{2.873341in}}%
\pgfpathcurveto{\pgfqpoint{2.271679in}{2.873341in}}{\pgfqpoint{2.282278in}{2.877731in}}{\pgfqpoint{2.290092in}{2.885545in}}%
\pgfpathcurveto{\pgfqpoint{2.297906in}{2.893358in}}{\pgfqpoint{2.302296in}{2.903957in}}{\pgfqpoint{2.302296in}{2.915008in}}%
\pgfpathcurveto{\pgfqpoint{2.302296in}{2.926058in}}{\pgfqpoint{2.297906in}{2.936657in}}{\pgfqpoint{2.290092in}{2.944470in}}%
\pgfpathcurveto{\pgfqpoint{2.282278in}{2.952284in}}{\pgfqpoint{2.271679in}{2.956674in}}{\pgfqpoint{2.260629in}{2.956674in}}%
\pgfpathcurveto{\pgfqpoint{2.249579in}{2.956674in}}{\pgfqpoint{2.238980in}{2.952284in}}{\pgfqpoint{2.231166in}{2.944470in}}%
\pgfpathcurveto{\pgfqpoint{2.223353in}{2.936657in}}{\pgfqpoint{2.218963in}{2.926058in}}{\pgfqpoint{2.218963in}{2.915008in}}%
\pgfpathcurveto{\pgfqpoint{2.218963in}{2.903957in}}{\pgfqpoint{2.223353in}{2.893358in}}{\pgfqpoint{2.231166in}{2.885545in}}%
\pgfpathcurveto{\pgfqpoint{2.238980in}{2.877731in}}{\pgfqpoint{2.249579in}{2.873341in}}{\pgfqpoint{2.260629in}{2.873341in}}%
\pgfpathclose%
\pgfusepath{stroke,fill}%
\end{pgfscope}%
\begin{pgfscope}%
\pgfpathrectangle{\pgfqpoint{0.600000in}{0.600000in}}{\pgfqpoint{3.900000in}{3.900000in}}%
\pgfusepath{clip}%
\pgfsetbuttcap%
\pgfsetroundjoin%
\definecolor{currentfill}{rgb}{0.121569,0.466667,0.705882}%
\pgfsetfillcolor{currentfill}%
\pgfsetlinewidth{1.003750pt}%
\definecolor{currentstroke}{rgb}{0.121569,0.466667,0.705882}%
\pgfsetstrokecolor{currentstroke}%
\pgfsetdash{}{0pt}%
\pgfpathmoveto{\pgfqpoint{2.430648in}{2.194738in}}%
\pgfpathcurveto{\pgfqpoint{2.441698in}{2.194738in}}{\pgfqpoint{2.452297in}{2.199128in}}{\pgfqpoint{2.460111in}{2.206942in}}%
\pgfpathcurveto{\pgfqpoint{2.467924in}{2.214756in}}{\pgfqpoint{2.472315in}{2.225355in}}{\pgfqpoint{2.472315in}{2.236405in}}%
\pgfpathcurveto{\pgfqpoint{2.472315in}{2.247455in}}{\pgfqpoint{2.467924in}{2.258054in}}{\pgfqpoint{2.460111in}{2.265867in}}%
\pgfpathcurveto{\pgfqpoint{2.452297in}{2.273681in}}{\pgfqpoint{2.441698in}{2.278071in}}{\pgfqpoint{2.430648in}{2.278071in}}%
\pgfpathcurveto{\pgfqpoint{2.419598in}{2.278071in}}{\pgfqpoint{2.408999in}{2.273681in}}{\pgfqpoint{2.401185in}{2.265867in}}%
\pgfpathcurveto{\pgfqpoint{2.393372in}{2.258054in}}{\pgfqpoint{2.388981in}{2.247455in}}{\pgfqpoint{2.388981in}{2.236405in}}%
\pgfpathcurveto{\pgfqpoint{2.388981in}{2.225355in}}{\pgfqpoint{2.393372in}{2.214756in}}{\pgfqpoint{2.401185in}{2.206942in}}%
\pgfpathcurveto{\pgfqpoint{2.408999in}{2.199128in}}{\pgfqpoint{2.419598in}{2.194738in}}{\pgfqpoint{2.430648in}{2.194738in}}%
\pgfpathclose%
\pgfusepath{stroke,fill}%
\end{pgfscope}%
\begin{pgfscope}%
\pgfpathrectangle{\pgfqpoint{0.600000in}{0.600000in}}{\pgfqpoint{3.900000in}{3.900000in}}%
\pgfusepath{clip}%
\pgfsetbuttcap%
\pgfsetroundjoin%
\definecolor{currentfill}{rgb}{0.121569,0.466667,0.705882}%
\pgfsetfillcolor{currentfill}%
\pgfsetlinewidth{1.003750pt}%
\definecolor{currentstroke}{rgb}{0.121569,0.466667,0.705882}%
\pgfsetstrokecolor{currentstroke}%
\pgfsetdash{}{0pt}%
\pgfpathmoveto{\pgfqpoint{1.943206in}{2.293288in}}%
\pgfpathcurveto{\pgfqpoint{1.954256in}{2.293288in}}{\pgfqpoint{1.964855in}{2.297678in}}{\pgfqpoint{1.972668in}{2.305491in}}%
\pgfpathcurveto{\pgfqpoint{1.980482in}{2.313305in}}{\pgfqpoint{1.984872in}{2.323904in}}{\pgfqpoint{1.984872in}{2.334954in}}%
\pgfpathcurveto{\pgfqpoint{1.984872in}{2.346004in}}{\pgfqpoint{1.980482in}{2.356603in}}{\pgfqpoint{1.972668in}{2.364417in}}%
\pgfpathcurveto{\pgfqpoint{1.964855in}{2.372231in}}{\pgfqpoint{1.954256in}{2.376621in}}{\pgfqpoint{1.943206in}{2.376621in}}%
\pgfpathcurveto{\pgfqpoint{1.932155in}{2.376621in}}{\pgfqpoint{1.921556in}{2.372231in}}{\pgfqpoint{1.913743in}{2.364417in}}%
\pgfpathcurveto{\pgfqpoint{1.905929in}{2.356603in}}{\pgfqpoint{1.901539in}{2.346004in}}{\pgfqpoint{1.901539in}{2.334954in}}%
\pgfpathcurveto{\pgfqpoint{1.901539in}{2.323904in}}{\pgfqpoint{1.905929in}{2.313305in}}{\pgfqpoint{1.913743in}{2.305491in}}%
\pgfpathcurveto{\pgfqpoint{1.921556in}{2.297678in}}{\pgfqpoint{1.932155in}{2.293288in}}{\pgfqpoint{1.943206in}{2.293288in}}%
\pgfpathclose%
\pgfusepath{stroke,fill}%
\end{pgfscope}%
\begin{pgfscope}%
\pgfpathrectangle{\pgfqpoint{0.600000in}{0.600000in}}{\pgfqpoint{3.900000in}{3.900000in}}%
\pgfusepath{clip}%
\pgfsetbuttcap%
\pgfsetroundjoin%
\definecolor{currentfill}{rgb}{0.121569,0.466667,0.705882}%
\pgfsetfillcolor{currentfill}%
\pgfsetlinewidth{1.003750pt}%
\definecolor{currentstroke}{rgb}{0.121569,0.466667,0.705882}%
\pgfsetstrokecolor{currentstroke}%
\pgfsetdash{}{0pt}%
\pgfpathmoveto{\pgfqpoint{2.681377in}{3.045860in}}%
\pgfpathcurveto{\pgfqpoint{2.692427in}{3.045860in}}{\pgfqpoint{2.703026in}{3.050250in}}{\pgfqpoint{2.710840in}{3.058064in}}%
\pgfpathcurveto{\pgfqpoint{2.718654in}{3.065877in}}{\pgfqpoint{2.723044in}{3.076476in}}{\pgfqpoint{2.723044in}{3.087526in}}%
\pgfpathcurveto{\pgfqpoint{2.723044in}{3.098577in}}{\pgfqpoint{2.718654in}{3.109176in}}{\pgfqpoint{2.710840in}{3.116989in}}%
\pgfpathcurveto{\pgfqpoint{2.703026in}{3.124803in}}{\pgfqpoint{2.692427in}{3.129193in}}{\pgfqpoint{2.681377in}{3.129193in}}%
\pgfpathcurveto{\pgfqpoint{2.670327in}{3.129193in}}{\pgfqpoint{2.659728in}{3.124803in}}{\pgfqpoint{2.651915in}{3.116989in}}%
\pgfpathcurveto{\pgfqpoint{2.644101in}{3.109176in}}{\pgfqpoint{2.639711in}{3.098577in}}{\pgfqpoint{2.639711in}{3.087526in}}%
\pgfpathcurveto{\pgfqpoint{2.639711in}{3.076476in}}{\pgfqpoint{2.644101in}{3.065877in}}{\pgfqpoint{2.651915in}{3.058064in}}%
\pgfpathcurveto{\pgfqpoint{2.659728in}{3.050250in}}{\pgfqpoint{2.670327in}{3.045860in}}{\pgfqpoint{2.681377in}{3.045860in}}%
\pgfpathclose%
\pgfusepath{stroke,fill}%
\end{pgfscope}%
\begin{pgfscope}%
\pgfpathrectangle{\pgfqpoint{0.600000in}{0.600000in}}{\pgfqpoint{3.900000in}{3.900000in}}%
\pgfusepath{clip}%
\pgfsetbuttcap%
\pgfsetroundjoin%
\definecolor{currentfill}{rgb}{0.121569,0.466667,0.705882}%
\pgfsetfillcolor{currentfill}%
\pgfsetlinewidth{1.003750pt}%
\definecolor{currentstroke}{rgb}{0.121569,0.466667,0.705882}%
\pgfsetstrokecolor{currentstroke}%
\pgfsetdash{}{0pt}%
\pgfpathmoveto{\pgfqpoint{2.174694in}{2.089207in}}%
\pgfpathcurveto{\pgfqpoint{2.185744in}{2.089207in}}{\pgfqpoint{2.196343in}{2.093598in}}{\pgfqpoint{2.204157in}{2.101411in}}%
\pgfpathcurveto{\pgfqpoint{2.211970in}{2.109225in}}{\pgfqpoint{2.216361in}{2.119824in}}{\pgfqpoint{2.216361in}{2.130874in}}%
\pgfpathcurveto{\pgfqpoint{2.216361in}{2.141924in}}{\pgfqpoint{2.211970in}{2.152523in}}{\pgfqpoint{2.204157in}{2.160337in}}%
\pgfpathcurveto{\pgfqpoint{2.196343in}{2.168150in}}{\pgfqpoint{2.185744in}{2.172541in}}{\pgfqpoint{2.174694in}{2.172541in}}%
\pgfpathcurveto{\pgfqpoint{2.163644in}{2.172541in}}{\pgfqpoint{2.153045in}{2.168150in}}{\pgfqpoint{2.145231in}{2.160337in}}%
\pgfpathcurveto{\pgfqpoint{2.137418in}{2.152523in}}{\pgfqpoint{2.133027in}{2.141924in}}{\pgfqpoint{2.133027in}{2.130874in}}%
\pgfpathcurveto{\pgfqpoint{2.133027in}{2.119824in}}{\pgfqpoint{2.137418in}{2.109225in}}{\pgfqpoint{2.145231in}{2.101411in}}%
\pgfpathcurveto{\pgfqpoint{2.153045in}{2.093598in}}{\pgfqpoint{2.163644in}{2.089207in}}{\pgfqpoint{2.174694in}{2.089207in}}%
\pgfpathclose%
\pgfusepath{stroke,fill}%
\end{pgfscope}%
\begin{pgfscope}%
\pgfpathrectangle{\pgfqpoint{0.600000in}{0.600000in}}{\pgfqpoint{3.900000in}{3.900000in}}%
\pgfusepath{clip}%
\pgfsetbuttcap%
\pgfsetroundjoin%
\definecolor{currentfill}{rgb}{0.121569,0.466667,0.705882}%
\pgfsetfillcolor{currentfill}%
\pgfsetlinewidth{1.003750pt}%
\definecolor{currentstroke}{rgb}{0.121569,0.466667,0.705882}%
\pgfsetstrokecolor{currentstroke}%
\pgfsetdash{}{0pt}%
\pgfpathmoveto{\pgfqpoint{1.644674in}{2.876291in}}%
\pgfpathcurveto{\pgfqpoint{1.655724in}{2.876291in}}{\pgfqpoint{1.666323in}{2.880681in}}{\pgfqpoint{1.674136in}{2.888495in}}%
\pgfpathcurveto{\pgfqpoint{1.681950in}{2.896308in}}{\pgfqpoint{1.686340in}{2.906907in}}{\pgfqpoint{1.686340in}{2.917957in}}%
\pgfpathcurveto{\pgfqpoint{1.686340in}{2.929007in}}{\pgfqpoint{1.681950in}{2.939606in}}{\pgfqpoint{1.674136in}{2.947420in}}%
\pgfpathcurveto{\pgfqpoint{1.666323in}{2.955234in}}{\pgfqpoint{1.655724in}{2.959624in}}{\pgfqpoint{1.644674in}{2.959624in}}%
\pgfpathcurveto{\pgfqpoint{1.633623in}{2.959624in}}{\pgfqpoint{1.623024in}{2.955234in}}{\pgfqpoint{1.615211in}{2.947420in}}%
\pgfpathcurveto{\pgfqpoint{1.607397in}{2.939606in}}{\pgfqpoint{1.603007in}{2.929007in}}{\pgfqpoint{1.603007in}{2.917957in}}%
\pgfpathcurveto{\pgfqpoint{1.603007in}{2.906907in}}{\pgfqpoint{1.607397in}{2.896308in}}{\pgfqpoint{1.615211in}{2.888495in}}%
\pgfpathcurveto{\pgfqpoint{1.623024in}{2.880681in}}{\pgfqpoint{1.633623in}{2.876291in}}{\pgfqpoint{1.644674in}{2.876291in}}%
\pgfpathclose%
\pgfusepath{stroke,fill}%
\end{pgfscope}%
\begin{pgfscope}%
\pgfpathrectangle{\pgfqpoint{0.600000in}{0.600000in}}{\pgfqpoint{3.900000in}{3.900000in}}%
\pgfusepath{clip}%
\pgfsetbuttcap%
\pgfsetroundjoin%
\definecolor{currentfill}{rgb}{0.121569,0.466667,0.705882}%
\pgfsetfillcolor{currentfill}%
\pgfsetlinewidth{1.003750pt}%
\definecolor{currentstroke}{rgb}{0.121569,0.466667,0.705882}%
\pgfsetstrokecolor{currentstroke}%
\pgfsetdash{}{0pt}%
\pgfpathmoveto{\pgfqpoint{2.690959in}{2.577605in}}%
\pgfpathcurveto{\pgfqpoint{2.702009in}{2.577605in}}{\pgfqpoint{2.712608in}{2.581995in}}{\pgfqpoint{2.720421in}{2.589809in}}%
\pgfpathcurveto{\pgfqpoint{2.728235in}{2.597622in}}{\pgfqpoint{2.732625in}{2.608222in}}{\pgfqpoint{2.732625in}{2.619272in}}%
\pgfpathcurveto{\pgfqpoint{2.732625in}{2.630322in}}{\pgfqpoint{2.728235in}{2.640921in}}{\pgfqpoint{2.720421in}{2.648734in}}%
\pgfpathcurveto{\pgfqpoint{2.712608in}{2.656548in}}{\pgfqpoint{2.702009in}{2.660938in}}{\pgfqpoint{2.690959in}{2.660938in}}%
\pgfpathcurveto{\pgfqpoint{2.679909in}{2.660938in}}{\pgfqpoint{2.669310in}{2.656548in}}{\pgfqpoint{2.661496in}{2.648734in}}%
\pgfpathcurveto{\pgfqpoint{2.653682in}{2.640921in}}{\pgfqpoint{2.649292in}{2.630322in}}{\pgfqpoint{2.649292in}{2.619272in}}%
\pgfpathcurveto{\pgfqpoint{2.649292in}{2.608222in}}{\pgfqpoint{2.653682in}{2.597622in}}{\pgfqpoint{2.661496in}{2.589809in}}%
\pgfpathcurveto{\pgfqpoint{2.669310in}{2.581995in}}{\pgfqpoint{2.679909in}{2.577605in}}{\pgfqpoint{2.690959in}{2.577605in}}%
\pgfpathclose%
\pgfusepath{stroke,fill}%
\end{pgfscope}%
\begin{pgfscope}%
\pgfpathrectangle{\pgfqpoint{0.600000in}{0.600000in}}{\pgfqpoint{3.900000in}{3.900000in}}%
\pgfusepath{clip}%
\pgfsetbuttcap%
\pgfsetroundjoin%
\definecolor{currentfill}{rgb}{0.121569,0.466667,0.705882}%
\pgfsetfillcolor{currentfill}%
\pgfsetlinewidth{1.003750pt}%
\definecolor{currentstroke}{rgb}{0.121569,0.466667,0.705882}%
\pgfsetstrokecolor{currentstroke}%
\pgfsetdash{}{0pt}%
\pgfpathmoveto{\pgfqpoint{2.443731in}{2.325920in}}%
\pgfpathcurveto{\pgfqpoint{2.454781in}{2.325920in}}{\pgfqpoint{2.465380in}{2.330310in}}{\pgfqpoint{2.473194in}{2.338124in}}%
\pgfpathcurveto{\pgfqpoint{2.481008in}{2.345937in}}{\pgfqpoint{2.485398in}{2.356536in}}{\pgfqpoint{2.485398in}{2.367586in}}%
\pgfpathcurveto{\pgfqpoint{2.485398in}{2.378636in}}{\pgfqpoint{2.481008in}{2.389235in}}{\pgfqpoint{2.473194in}{2.397049in}}%
\pgfpathcurveto{\pgfqpoint{2.465380in}{2.404863in}}{\pgfqpoint{2.454781in}{2.409253in}}{\pgfqpoint{2.443731in}{2.409253in}}%
\pgfpathcurveto{\pgfqpoint{2.432681in}{2.409253in}}{\pgfqpoint{2.422082in}{2.404863in}}{\pgfqpoint{2.414268in}{2.397049in}}%
\pgfpathcurveto{\pgfqpoint{2.406455in}{2.389235in}}{\pgfqpoint{2.402064in}{2.378636in}}{\pgfqpoint{2.402064in}{2.367586in}}%
\pgfpathcurveto{\pgfqpoint{2.402064in}{2.356536in}}{\pgfqpoint{2.406455in}{2.345937in}}{\pgfqpoint{2.414268in}{2.338124in}}%
\pgfpathcurveto{\pgfqpoint{2.422082in}{2.330310in}}{\pgfqpoint{2.432681in}{2.325920in}}{\pgfqpoint{2.443731in}{2.325920in}}%
\pgfpathclose%
\pgfusepath{stroke,fill}%
\end{pgfscope}%
\begin{pgfscope}%
\pgfpathrectangle{\pgfqpoint{0.600000in}{0.600000in}}{\pgfqpoint{3.900000in}{3.900000in}}%
\pgfusepath{clip}%
\pgfsetbuttcap%
\pgfsetroundjoin%
\definecolor{currentfill}{rgb}{0.121569,0.466667,0.705882}%
\pgfsetfillcolor{currentfill}%
\pgfsetlinewidth{1.003750pt}%
\definecolor{currentstroke}{rgb}{0.121569,0.466667,0.705882}%
\pgfsetstrokecolor{currentstroke}%
\pgfsetdash{}{0pt}%
\pgfpathmoveto{\pgfqpoint{2.770262in}{2.515915in}}%
\pgfpathcurveto{\pgfqpoint{2.781312in}{2.515915in}}{\pgfqpoint{2.791911in}{2.520305in}}{\pgfqpoint{2.799725in}{2.528119in}}%
\pgfpathcurveto{\pgfqpoint{2.807538in}{2.535932in}}{\pgfqpoint{2.811928in}{2.546531in}}{\pgfqpoint{2.811928in}{2.557581in}}%
\pgfpathcurveto{\pgfqpoint{2.811928in}{2.568632in}}{\pgfqpoint{2.807538in}{2.579231in}}{\pgfqpoint{2.799725in}{2.587044in}}%
\pgfpathcurveto{\pgfqpoint{2.791911in}{2.594858in}}{\pgfqpoint{2.781312in}{2.599248in}}{\pgfqpoint{2.770262in}{2.599248in}}%
\pgfpathcurveto{\pgfqpoint{2.759212in}{2.599248in}}{\pgfqpoint{2.748613in}{2.594858in}}{\pgfqpoint{2.740799in}{2.587044in}}%
\pgfpathcurveto{\pgfqpoint{2.732985in}{2.579231in}}{\pgfqpoint{2.728595in}{2.568632in}}{\pgfqpoint{2.728595in}{2.557581in}}%
\pgfpathcurveto{\pgfqpoint{2.728595in}{2.546531in}}{\pgfqpoint{2.732985in}{2.535932in}}{\pgfqpoint{2.740799in}{2.528119in}}%
\pgfpathcurveto{\pgfqpoint{2.748613in}{2.520305in}}{\pgfqpoint{2.759212in}{2.515915in}}{\pgfqpoint{2.770262in}{2.515915in}}%
\pgfpathclose%
\pgfusepath{stroke,fill}%
\end{pgfscope}%
\begin{pgfscope}%
\pgfpathrectangle{\pgfqpoint{0.600000in}{0.600000in}}{\pgfqpoint{3.900000in}{3.900000in}}%
\pgfusepath{clip}%
\pgfsetbuttcap%
\pgfsetroundjoin%
\definecolor{currentfill}{rgb}{0.121569,0.466667,0.705882}%
\pgfsetfillcolor{currentfill}%
\pgfsetlinewidth{1.003750pt}%
\definecolor{currentstroke}{rgb}{0.121569,0.466667,0.705882}%
\pgfsetstrokecolor{currentstroke}%
\pgfsetdash{}{0pt}%
\pgfpathmoveto{\pgfqpoint{1.471564in}{2.228881in}}%
\pgfpathcurveto{\pgfqpoint{1.482615in}{2.228881in}}{\pgfqpoint{1.493214in}{2.233271in}}{\pgfqpoint{1.501027in}{2.241085in}}%
\pgfpathcurveto{\pgfqpoint{1.508841in}{2.248899in}}{\pgfqpoint{1.513231in}{2.259498in}}{\pgfqpoint{1.513231in}{2.270548in}}%
\pgfpathcurveto{\pgfqpoint{1.513231in}{2.281598in}}{\pgfqpoint{1.508841in}{2.292197in}}{\pgfqpoint{1.501027in}{2.300011in}}%
\pgfpathcurveto{\pgfqpoint{1.493214in}{2.307824in}}{\pgfqpoint{1.482615in}{2.312214in}}{\pgfqpoint{1.471564in}{2.312214in}}%
\pgfpathcurveto{\pgfqpoint{1.460514in}{2.312214in}}{\pgfqpoint{1.449915in}{2.307824in}}{\pgfqpoint{1.442102in}{2.300011in}}%
\pgfpathcurveto{\pgfqpoint{1.434288in}{2.292197in}}{\pgfqpoint{1.429898in}{2.281598in}}{\pgfqpoint{1.429898in}{2.270548in}}%
\pgfpathcurveto{\pgfqpoint{1.429898in}{2.259498in}}{\pgfqpoint{1.434288in}{2.248899in}}{\pgfqpoint{1.442102in}{2.241085in}}%
\pgfpathcurveto{\pgfqpoint{1.449915in}{2.233271in}}{\pgfqpoint{1.460514in}{2.228881in}}{\pgfqpoint{1.471564in}{2.228881in}}%
\pgfpathclose%
\pgfusepath{stroke,fill}%
\end{pgfscope}%
\begin{pgfscope}%
\pgfpathrectangle{\pgfqpoint{0.600000in}{0.600000in}}{\pgfqpoint{3.900000in}{3.900000in}}%
\pgfusepath{clip}%
\pgfsetbuttcap%
\pgfsetroundjoin%
\definecolor{currentfill}{rgb}{0.121569,0.466667,0.705882}%
\pgfsetfillcolor{currentfill}%
\pgfsetlinewidth{1.003750pt}%
\definecolor{currentstroke}{rgb}{0.121569,0.466667,0.705882}%
\pgfsetstrokecolor{currentstroke}%
\pgfsetdash{}{0pt}%
\pgfpathmoveto{\pgfqpoint{2.260831in}{3.100720in}}%
\pgfpathcurveto{\pgfqpoint{2.271881in}{3.100720in}}{\pgfqpoint{2.282480in}{3.105110in}}{\pgfqpoint{2.290294in}{3.112923in}}%
\pgfpathcurveto{\pgfqpoint{2.298107in}{3.120737in}}{\pgfqpoint{2.302497in}{3.131336in}}{\pgfqpoint{2.302497in}{3.142386in}}%
\pgfpathcurveto{\pgfqpoint{2.302497in}{3.153436in}}{\pgfqpoint{2.298107in}{3.164035in}}{\pgfqpoint{2.290294in}{3.171849in}}%
\pgfpathcurveto{\pgfqpoint{2.282480in}{3.179663in}}{\pgfqpoint{2.271881in}{3.184053in}}{\pgfqpoint{2.260831in}{3.184053in}}%
\pgfpathcurveto{\pgfqpoint{2.249781in}{3.184053in}}{\pgfqpoint{2.239182in}{3.179663in}}{\pgfqpoint{2.231368in}{3.171849in}}%
\pgfpathcurveto{\pgfqpoint{2.223554in}{3.164035in}}{\pgfqpoint{2.219164in}{3.153436in}}{\pgfqpoint{2.219164in}{3.142386in}}%
\pgfpathcurveto{\pgfqpoint{2.219164in}{3.131336in}}{\pgfqpoint{2.223554in}{3.120737in}}{\pgfqpoint{2.231368in}{3.112923in}}%
\pgfpathcurveto{\pgfqpoint{2.239182in}{3.105110in}}{\pgfqpoint{2.249781in}{3.100720in}}{\pgfqpoint{2.260831in}{3.100720in}}%
\pgfpathclose%
\pgfusepath{stroke,fill}%
\end{pgfscope}%
\begin{pgfscope}%
\pgfpathrectangle{\pgfqpoint{0.600000in}{0.600000in}}{\pgfqpoint{3.900000in}{3.900000in}}%
\pgfusepath{clip}%
\pgfsetbuttcap%
\pgfsetroundjoin%
\definecolor{currentfill}{rgb}{0.121569,0.466667,0.705882}%
\pgfsetfillcolor{currentfill}%
\pgfsetlinewidth{1.003750pt}%
\definecolor{currentstroke}{rgb}{0.121569,0.466667,0.705882}%
\pgfsetstrokecolor{currentstroke}%
\pgfsetdash{}{0pt}%
\pgfpathmoveto{\pgfqpoint{1.897610in}{2.810162in}}%
\pgfpathcurveto{\pgfqpoint{1.908660in}{2.810162in}}{\pgfqpoint{1.919259in}{2.814552in}}{\pgfqpoint{1.927073in}{2.822366in}}%
\pgfpathcurveto{\pgfqpoint{1.934886in}{2.830180in}}{\pgfqpoint{1.939277in}{2.840779in}}{\pgfqpoint{1.939277in}{2.851829in}}%
\pgfpathcurveto{\pgfqpoint{1.939277in}{2.862879in}}{\pgfqpoint{1.934886in}{2.873478in}}{\pgfqpoint{1.927073in}{2.881291in}}%
\pgfpathcurveto{\pgfqpoint{1.919259in}{2.889105in}}{\pgfqpoint{1.908660in}{2.893495in}}{\pgfqpoint{1.897610in}{2.893495in}}%
\pgfpathcurveto{\pgfqpoint{1.886560in}{2.893495in}}{\pgfqpoint{1.875961in}{2.889105in}}{\pgfqpoint{1.868147in}{2.881291in}}%
\pgfpathcurveto{\pgfqpoint{1.860334in}{2.873478in}}{\pgfqpoint{1.855943in}{2.862879in}}{\pgfqpoint{1.855943in}{2.851829in}}%
\pgfpathcurveto{\pgfqpoint{1.855943in}{2.840779in}}{\pgfqpoint{1.860334in}{2.830180in}}{\pgfqpoint{1.868147in}{2.822366in}}%
\pgfpathcurveto{\pgfqpoint{1.875961in}{2.814552in}}{\pgfqpoint{1.886560in}{2.810162in}}{\pgfqpoint{1.897610in}{2.810162in}}%
\pgfpathclose%
\pgfusepath{stroke,fill}%
\end{pgfscope}%
\begin{pgfscope}%
\pgfpathrectangle{\pgfqpoint{0.600000in}{0.600000in}}{\pgfqpoint{3.900000in}{3.900000in}}%
\pgfusepath{clip}%
\pgfsetbuttcap%
\pgfsetroundjoin%
\definecolor{currentfill}{rgb}{0.121569,0.466667,0.705882}%
\pgfsetfillcolor{currentfill}%
\pgfsetlinewidth{1.003750pt}%
\definecolor{currentstroke}{rgb}{0.121569,0.466667,0.705882}%
\pgfsetstrokecolor{currentstroke}%
\pgfsetdash{}{0pt}%
\pgfpathmoveto{\pgfqpoint{1.912208in}{2.452033in}}%
\pgfpathcurveto{\pgfqpoint{1.923258in}{2.452033in}}{\pgfqpoint{1.933857in}{2.456423in}}{\pgfqpoint{1.941671in}{2.464237in}}%
\pgfpathcurveto{\pgfqpoint{1.949484in}{2.472050in}}{\pgfqpoint{1.953874in}{2.482649in}}{\pgfqpoint{1.953874in}{2.493700in}}%
\pgfpathcurveto{\pgfqpoint{1.953874in}{2.504750in}}{\pgfqpoint{1.949484in}{2.515349in}}{\pgfqpoint{1.941671in}{2.523162in}}%
\pgfpathcurveto{\pgfqpoint{1.933857in}{2.530976in}}{\pgfqpoint{1.923258in}{2.535366in}}{\pgfqpoint{1.912208in}{2.535366in}}%
\pgfpathcurveto{\pgfqpoint{1.901158in}{2.535366in}}{\pgfqpoint{1.890559in}{2.530976in}}{\pgfqpoint{1.882745in}{2.523162in}}%
\pgfpathcurveto{\pgfqpoint{1.874931in}{2.515349in}}{\pgfqpoint{1.870541in}{2.504750in}}{\pgfqpoint{1.870541in}{2.493700in}}%
\pgfpathcurveto{\pgfqpoint{1.870541in}{2.482649in}}{\pgfqpoint{1.874931in}{2.472050in}}{\pgfqpoint{1.882745in}{2.464237in}}%
\pgfpathcurveto{\pgfqpoint{1.890559in}{2.456423in}}{\pgfqpoint{1.901158in}{2.452033in}}{\pgfqpoint{1.912208in}{2.452033in}}%
\pgfpathclose%
\pgfusepath{stroke,fill}%
\end{pgfscope}%
\begin{pgfscope}%
\pgfpathrectangle{\pgfqpoint{0.600000in}{0.600000in}}{\pgfqpoint{3.900000in}{3.900000in}}%
\pgfusepath{clip}%
\pgfsetbuttcap%
\pgfsetroundjoin%
\definecolor{currentfill}{rgb}{0.121569,0.466667,0.705882}%
\pgfsetfillcolor{currentfill}%
\pgfsetlinewidth{1.003750pt}%
\definecolor{currentstroke}{rgb}{0.121569,0.466667,0.705882}%
\pgfsetstrokecolor{currentstroke}%
\pgfsetdash{}{0pt}%
\pgfpathmoveto{\pgfqpoint{1.994777in}{1.694453in}}%
\pgfpathcurveto{\pgfqpoint{2.005827in}{1.694453in}}{\pgfqpoint{2.016426in}{1.698843in}}{\pgfqpoint{2.024239in}{1.706657in}}%
\pgfpathcurveto{\pgfqpoint{2.032053in}{1.714471in}}{\pgfqpoint{2.036443in}{1.725070in}}{\pgfqpoint{2.036443in}{1.736120in}}%
\pgfpathcurveto{\pgfqpoint{2.036443in}{1.747170in}}{\pgfqpoint{2.032053in}{1.757769in}}{\pgfqpoint{2.024239in}{1.765583in}}%
\pgfpathcurveto{\pgfqpoint{2.016426in}{1.773396in}}{\pgfqpoint{2.005827in}{1.777786in}}{\pgfqpoint{1.994777in}{1.777786in}}%
\pgfpathcurveto{\pgfqpoint{1.983727in}{1.777786in}}{\pgfqpoint{1.973127in}{1.773396in}}{\pgfqpoint{1.965314in}{1.765583in}}%
\pgfpathcurveto{\pgfqpoint{1.957500in}{1.757769in}}{\pgfqpoint{1.953110in}{1.747170in}}{\pgfqpoint{1.953110in}{1.736120in}}%
\pgfpathcurveto{\pgfqpoint{1.953110in}{1.725070in}}{\pgfqpoint{1.957500in}{1.714471in}}{\pgfqpoint{1.965314in}{1.706657in}}%
\pgfpathcurveto{\pgfqpoint{1.973127in}{1.698843in}}{\pgfqpoint{1.983727in}{1.694453in}}{\pgfqpoint{1.994777in}{1.694453in}}%
\pgfpathclose%
\pgfusepath{stroke,fill}%
\end{pgfscope}%
\begin{pgfscope}%
\pgfpathrectangle{\pgfqpoint{0.600000in}{0.600000in}}{\pgfqpoint{3.900000in}{3.900000in}}%
\pgfusepath{clip}%
\pgfsetbuttcap%
\pgfsetroundjoin%
\definecolor{currentfill}{rgb}{0.121569,0.466667,0.705882}%
\pgfsetfillcolor{currentfill}%
\pgfsetlinewidth{1.003750pt}%
\definecolor{currentstroke}{rgb}{0.121569,0.466667,0.705882}%
\pgfsetstrokecolor{currentstroke}%
\pgfsetdash{}{0pt}%
\pgfpathmoveto{\pgfqpoint{2.828837in}{3.424893in}}%
\pgfpathcurveto{\pgfqpoint{2.839888in}{3.424893in}}{\pgfqpoint{2.850487in}{3.429283in}}{\pgfqpoint{2.858300in}{3.437097in}}%
\pgfpathcurveto{\pgfqpoint{2.866114in}{3.444910in}}{\pgfqpoint{2.870504in}{3.455509in}}{\pgfqpoint{2.870504in}{3.466559in}}%
\pgfpathcurveto{\pgfqpoint{2.870504in}{3.477609in}}{\pgfqpoint{2.866114in}{3.488209in}}{\pgfqpoint{2.858300in}{3.496022in}}%
\pgfpathcurveto{\pgfqpoint{2.850487in}{3.503836in}}{\pgfqpoint{2.839888in}{3.508226in}}{\pgfqpoint{2.828837in}{3.508226in}}%
\pgfpathcurveto{\pgfqpoint{2.817787in}{3.508226in}}{\pgfqpoint{2.807188in}{3.503836in}}{\pgfqpoint{2.799375in}{3.496022in}}%
\pgfpathcurveto{\pgfqpoint{2.791561in}{3.488209in}}{\pgfqpoint{2.787171in}{3.477609in}}{\pgfqpoint{2.787171in}{3.466559in}}%
\pgfpathcurveto{\pgfqpoint{2.787171in}{3.455509in}}{\pgfqpoint{2.791561in}{3.444910in}}{\pgfqpoint{2.799375in}{3.437097in}}%
\pgfpathcurveto{\pgfqpoint{2.807188in}{3.429283in}}{\pgfqpoint{2.817787in}{3.424893in}}{\pgfqpoint{2.828837in}{3.424893in}}%
\pgfpathclose%
\pgfusepath{stroke,fill}%
\end{pgfscope}%
\begin{pgfscope}%
\pgfpathrectangle{\pgfqpoint{0.600000in}{0.600000in}}{\pgfqpoint{3.900000in}{3.900000in}}%
\pgfusepath{clip}%
\pgfsetbuttcap%
\pgfsetroundjoin%
\definecolor{currentfill}{rgb}{0.121569,0.466667,0.705882}%
\pgfsetfillcolor{currentfill}%
\pgfsetlinewidth{1.003750pt}%
\definecolor{currentstroke}{rgb}{0.121569,0.466667,0.705882}%
\pgfsetstrokecolor{currentstroke}%
\pgfsetdash{}{0pt}%
\pgfpathmoveto{\pgfqpoint{2.725628in}{2.261384in}}%
\pgfpathcurveto{\pgfqpoint{2.736679in}{2.261384in}}{\pgfqpoint{2.747278in}{2.265774in}}{\pgfqpoint{2.755091in}{2.273587in}}%
\pgfpathcurveto{\pgfqpoint{2.762905in}{2.281401in}}{\pgfqpoint{2.767295in}{2.292000in}}{\pgfqpoint{2.767295in}{2.303050in}}%
\pgfpathcurveto{\pgfqpoint{2.767295in}{2.314100in}}{\pgfqpoint{2.762905in}{2.324699in}}{\pgfqpoint{2.755091in}{2.332513in}}%
\pgfpathcurveto{\pgfqpoint{2.747278in}{2.340327in}}{\pgfqpoint{2.736679in}{2.344717in}}{\pgfqpoint{2.725628in}{2.344717in}}%
\pgfpathcurveto{\pgfqpoint{2.714578in}{2.344717in}}{\pgfqpoint{2.703979in}{2.340327in}}{\pgfqpoint{2.696166in}{2.332513in}}%
\pgfpathcurveto{\pgfqpoint{2.688352in}{2.324699in}}{\pgfqpoint{2.683962in}{2.314100in}}{\pgfqpoint{2.683962in}{2.303050in}}%
\pgfpathcurveto{\pgfqpoint{2.683962in}{2.292000in}}{\pgfqpoint{2.688352in}{2.281401in}}{\pgfqpoint{2.696166in}{2.273587in}}%
\pgfpathcurveto{\pgfqpoint{2.703979in}{2.265774in}}{\pgfqpoint{2.714578in}{2.261384in}}{\pgfqpoint{2.725628in}{2.261384in}}%
\pgfpathclose%
\pgfusepath{stroke,fill}%
\end{pgfscope}%
\begin{pgfscope}%
\pgfpathrectangle{\pgfqpoint{0.600000in}{0.600000in}}{\pgfqpoint{3.900000in}{3.900000in}}%
\pgfusepath{clip}%
\pgfsetbuttcap%
\pgfsetroundjoin%
\definecolor{currentfill}{rgb}{0.121569,0.466667,0.705882}%
\pgfsetfillcolor{currentfill}%
\pgfsetlinewidth{1.003750pt}%
\definecolor{currentstroke}{rgb}{0.121569,0.466667,0.705882}%
\pgfsetstrokecolor{currentstroke}%
\pgfsetdash{}{0pt}%
\pgfpathmoveto{\pgfqpoint{2.996552in}{2.105993in}}%
\pgfpathcurveto{\pgfqpoint{3.007602in}{2.105993in}}{\pgfqpoint{3.018202in}{2.110384in}}{\pgfqpoint{3.026015in}{2.118197in}}%
\pgfpathcurveto{\pgfqpoint{3.033829in}{2.126011in}}{\pgfqpoint{3.038219in}{2.136610in}}{\pgfqpoint{3.038219in}{2.147660in}}%
\pgfpathcurveto{\pgfqpoint{3.038219in}{2.158710in}}{\pgfqpoint{3.033829in}{2.169309in}}{\pgfqpoint{3.026015in}{2.177123in}}%
\pgfpathcurveto{\pgfqpoint{3.018202in}{2.184936in}}{\pgfqpoint{3.007602in}{2.189327in}}{\pgfqpoint{2.996552in}{2.189327in}}%
\pgfpathcurveto{\pgfqpoint{2.985502in}{2.189327in}}{\pgfqpoint{2.974903in}{2.184936in}}{\pgfqpoint{2.967090in}{2.177123in}}%
\pgfpathcurveto{\pgfqpoint{2.959276in}{2.169309in}}{\pgfqpoint{2.954886in}{2.158710in}}{\pgfqpoint{2.954886in}{2.147660in}}%
\pgfpathcurveto{\pgfqpoint{2.954886in}{2.136610in}}{\pgfqpoint{2.959276in}{2.126011in}}{\pgfqpoint{2.967090in}{2.118197in}}%
\pgfpathcurveto{\pgfqpoint{2.974903in}{2.110384in}}{\pgfqpoint{2.985502in}{2.105993in}}{\pgfqpoint{2.996552in}{2.105993in}}%
\pgfpathclose%
\pgfusepath{stroke,fill}%
\end{pgfscope}%
\begin{pgfscope}%
\pgfpathrectangle{\pgfqpoint{0.600000in}{0.600000in}}{\pgfqpoint{3.900000in}{3.900000in}}%
\pgfusepath{clip}%
\pgfsetbuttcap%
\pgfsetroundjoin%
\definecolor{currentfill}{rgb}{0.121569,0.466667,0.705882}%
\pgfsetfillcolor{currentfill}%
\pgfsetlinewidth{1.003750pt}%
\definecolor{currentstroke}{rgb}{0.121569,0.466667,0.705882}%
\pgfsetstrokecolor{currentstroke}%
\pgfsetdash{}{0pt}%
\pgfpathmoveto{\pgfqpoint{2.107360in}{2.300721in}}%
\pgfpathcurveto{\pgfqpoint{2.118410in}{2.300721in}}{\pgfqpoint{2.129009in}{2.305111in}}{\pgfqpoint{2.136823in}{2.312925in}}%
\pgfpathcurveto{\pgfqpoint{2.144636in}{2.320739in}}{\pgfqpoint{2.149027in}{2.331338in}}{\pgfqpoint{2.149027in}{2.342388in}}%
\pgfpathcurveto{\pgfqpoint{2.149027in}{2.353438in}}{\pgfqpoint{2.144636in}{2.364037in}}{\pgfqpoint{2.136823in}{2.371851in}}%
\pgfpathcurveto{\pgfqpoint{2.129009in}{2.379664in}}{\pgfqpoint{2.118410in}{2.384054in}}{\pgfqpoint{2.107360in}{2.384054in}}%
\pgfpathcurveto{\pgfqpoint{2.096310in}{2.384054in}}{\pgfqpoint{2.085711in}{2.379664in}}{\pgfqpoint{2.077897in}{2.371851in}}%
\pgfpathcurveto{\pgfqpoint{2.070084in}{2.364037in}}{\pgfqpoint{2.065693in}{2.353438in}}{\pgfqpoint{2.065693in}{2.342388in}}%
\pgfpathcurveto{\pgfqpoint{2.065693in}{2.331338in}}{\pgfqpoint{2.070084in}{2.320739in}}{\pgfqpoint{2.077897in}{2.312925in}}%
\pgfpathcurveto{\pgfqpoint{2.085711in}{2.305111in}}{\pgfqpoint{2.096310in}{2.300721in}}{\pgfqpoint{2.107360in}{2.300721in}}%
\pgfpathclose%
\pgfusepath{stroke,fill}%
\end{pgfscope}%
\begin{pgfscope}%
\pgfpathrectangle{\pgfqpoint{0.600000in}{0.600000in}}{\pgfqpoint{3.900000in}{3.900000in}}%
\pgfusepath{clip}%
\pgfsetbuttcap%
\pgfsetroundjoin%
\definecolor{currentfill}{rgb}{0.121569,0.466667,0.705882}%
\pgfsetfillcolor{currentfill}%
\pgfsetlinewidth{1.003750pt}%
\definecolor{currentstroke}{rgb}{0.121569,0.466667,0.705882}%
\pgfsetstrokecolor{currentstroke}%
\pgfsetdash{}{0pt}%
\pgfpathmoveto{\pgfqpoint{2.923413in}{2.135281in}}%
\pgfpathcurveto{\pgfqpoint{2.934463in}{2.135281in}}{\pgfqpoint{2.945062in}{2.139671in}}{\pgfqpoint{2.952876in}{2.147485in}}%
\pgfpathcurveto{\pgfqpoint{2.960690in}{2.155298in}}{\pgfqpoint{2.965080in}{2.165897in}}{\pgfqpoint{2.965080in}{2.176947in}}%
\pgfpathcurveto{\pgfqpoint{2.965080in}{2.187997in}}{\pgfqpoint{2.960690in}{2.198596in}}{\pgfqpoint{2.952876in}{2.206410in}}%
\pgfpathcurveto{\pgfqpoint{2.945062in}{2.214224in}}{\pgfqpoint{2.934463in}{2.218614in}}{\pgfqpoint{2.923413in}{2.218614in}}%
\pgfpathcurveto{\pgfqpoint{2.912363in}{2.218614in}}{\pgfqpoint{2.901764in}{2.214224in}}{\pgfqpoint{2.893950in}{2.206410in}}%
\pgfpathcurveto{\pgfqpoint{2.886137in}{2.198596in}}{\pgfqpoint{2.881747in}{2.187997in}}{\pgfqpoint{2.881747in}{2.176947in}}%
\pgfpathcurveto{\pgfqpoint{2.881747in}{2.165897in}}{\pgfqpoint{2.886137in}{2.155298in}}{\pgfqpoint{2.893950in}{2.147485in}}%
\pgfpathcurveto{\pgfqpoint{2.901764in}{2.139671in}}{\pgfqpoint{2.912363in}{2.135281in}}{\pgfqpoint{2.923413in}{2.135281in}}%
\pgfpathclose%
\pgfusepath{stroke,fill}%
\end{pgfscope}%
\begin{pgfscope}%
\pgfpathrectangle{\pgfqpoint{0.600000in}{0.600000in}}{\pgfqpoint{3.900000in}{3.900000in}}%
\pgfusepath{clip}%
\pgfsetbuttcap%
\pgfsetroundjoin%
\definecolor{currentfill}{rgb}{0.121569,0.466667,0.705882}%
\pgfsetfillcolor{currentfill}%
\pgfsetlinewidth{1.003750pt}%
\definecolor{currentstroke}{rgb}{0.121569,0.466667,0.705882}%
\pgfsetstrokecolor{currentstroke}%
\pgfsetdash{}{0pt}%
\pgfpathmoveto{\pgfqpoint{2.962459in}{2.648907in}}%
\pgfpathcurveto{\pgfqpoint{2.973510in}{2.648907in}}{\pgfqpoint{2.984109in}{2.653298in}}{\pgfqpoint{2.991922in}{2.661111in}}%
\pgfpathcurveto{\pgfqpoint{2.999736in}{2.668925in}}{\pgfqpoint{3.004126in}{2.679524in}}{\pgfqpoint{3.004126in}{2.690574in}}%
\pgfpathcurveto{\pgfqpoint{3.004126in}{2.701624in}}{\pgfqpoint{2.999736in}{2.712223in}}{\pgfqpoint{2.991922in}{2.720037in}}%
\pgfpathcurveto{\pgfqpoint{2.984109in}{2.727851in}}{\pgfqpoint{2.973510in}{2.732241in}}{\pgfqpoint{2.962459in}{2.732241in}}%
\pgfpathcurveto{\pgfqpoint{2.951409in}{2.732241in}}{\pgfqpoint{2.940810in}{2.727851in}}{\pgfqpoint{2.932997in}{2.720037in}}%
\pgfpathcurveto{\pgfqpoint{2.925183in}{2.712223in}}{\pgfqpoint{2.920793in}{2.701624in}}{\pgfqpoint{2.920793in}{2.690574in}}%
\pgfpathcurveto{\pgfqpoint{2.920793in}{2.679524in}}{\pgfqpoint{2.925183in}{2.668925in}}{\pgfqpoint{2.932997in}{2.661111in}}%
\pgfpathcurveto{\pgfqpoint{2.940810in}{2.653298in}}{\pgfqpoint{2.951409in}{2.648907in}}{\pgfqpoint{2.962459in}{2.648907in}}%
\pgfpathclose%
\pgfusepath{stroke,fill}%
\end{pgfscope}%
\begin{pgfscope}%
\pgfpathrectangle{\pgfqpoint{0.600000in}{0.600000in}}{\pgfqpoint{3.900000in}{3.900000in}}%
\pgfusepath{clip}%
\pgfsetbuttcap%
\pgfsetroundjoin%
\definecolor{currentfill}{rgb}{0.121569,0.466667,0.705882}%
\pgfsetfillcolor{currentfill}%
\pgfsetlinewidth{1.003750pt}%
\definecolor{currentstroke}{rgb}{0.121569,0.466667,0.705882}%
\pgfsetstrokecolor{currentstroke}%
\pgfsetdash{}{0pt}%
\pgfpathmoveto{\pgfqpoint{3.004579in}{1.543376in}}%
\pgfpathcurveto{\pgfqpoint{3.015629in}{1.543376in}}{\pgfqpoint{3.026228in}{1.547766in}}{\pgfqpoint{3.034041in}{1.555580in}}%
\pgfpathcurveto{\pgfqpoint{3.041855in}{1.563393in}}{\pgfqpoint{3.046245in}{1.573992in}}{\pgfqpoint{3.046245in}{1.585043in}}%
\pgfpathcurveto{\pgfqpoint{3.046245in}{1.596093in}}{\pgfqpoint{3.041855in}{1.606692in}}{\pgfqpoint{3.034041in}{1.614505in}}%
\pgfpathcurveto{\pgfqpoint{3.026228in}{1.622319in}}{\pgfqpoint{3.015629in}{1.626709in}}{\pgfqpoint{3.004579in}{1.626709in}}%
\pgfpathcurveto{\pgfqpoint{2.993528in}{1.626709in}}{\pgfqpoint{2.982929in}{1.622319in}}{\pgfqpoint{2.975116in}{1.614505in}}%
\pgfpathcurveto{\pgfqpoint{2.967302in}{1.606692in}}{\pgfqpoint{2.962912in}{1.596093in}}{\pgfqpoint{2.962912in}{1.585043in}}%
\pgfpathcurveto{\pgfqpoint{2.962912in}{1.573992in}}{\pgfqpoint{2.967302in}{1.563393in}}{\pgfqpoint{2.975116in}{1.555580in}}%
\pgfpathcurveto{\pgfqpoint{2.982929in}{1.547766in}}{\pgfqpoint{2.993528in}{1.543376in}}{\pgfqpoint{3.004579in}{1.543376in}}%
\pgfpathclose%
\pgfusepath{stroke,fill}%
\end{pgfscope}%
\begin{pgfscope}%
\pgfpathrectangle{\pgfqpoint{0.600000in}{0.600000in}}{\pgfqpoint{3.900000in}{3.900000in}}%
\pgfusepath{clip}%
\pgfsetbuttcap%
\pgfsetroundjoin%
\definecolor{currentfill}{rgb}{0.121569,0.466667,0.705882}%
\pgfsetfillcolor{currentfill}%
\pgfsetlinewidth{1.003750pt}%
\definecolor{currentstroke}{rgb}{0.121569,0.466667,0.705882}%
\pgfsetstrokecolor{currentstroke}%
\pgfsetdash{}{0pt}%
\pgfpathmoveto{\pgfqpoint{2.082261in}{3.226196in}}%
\pgfpathcurveto{\pgfqpoint{2.093312in}{3.226196in}}{\pgfqpoint{2.103911in}{3.230586in}}{\pgfqpoint{2.111724in}{3.238400in}}%
\pgfpathcurveto{\pgfqpoint{2.119538in}{3.246214in}}{\pgfqpoint{2.123928in}{3.256813in}}{\pgfqpoint{2.123928in}{3.267863in}}%
\pgfpathcurveto{\pgfqpoint{2.123928in}{3.278913in}}{\pgfqpoint{2.119538in}{3.289512in}}{\pgfqpoint{2.111724in}{3.297326in}}%
\pgfpathcurveto{\pgfqpoint{2.103911in}{3.305139in}}{\pgfqpoint{2.093312in}{3.309530in}}{\pgfqpoint{2.082261in}{3.309530in}}%
\pgfpathcurveto{\pgfqpoint{2.071211in}{3.309530in}}{\pgfqpoint{2.060612in}{3.305139in}}{\pgfqpoint{2.052799in}{3.297326in}}%
\pgfpathcurveto{\pgfqpoint{2.044985in}{3.289512in}}{\pgfqpoint{2.040595in}{3.278913in}}{\pgfqpoint{2.040595in}{3.267863in}}%
\pgfpathcurveto{\pgfqpoint{2.040595in}{3.256813in}}{\pgfqpoint{2.044985in}{3.246214in}}{\pgfqpoint{2.052799in}{3.238400in}}%
\pgfpathcurveto{\pgfqpoint{2.060612in}{3.230586in}}{\pgfqpoint{2.071211in}{3.226196in}}{\pgfqpoint{2.082261in}{3.226196in}}%
\pgfpathclose%
\pgfusepath{stroke,fill}%
\end{pgfscope}%
\begin{pgfscope}%
\pgfpathrectangle{\pgfqpoint{0.600000in}{0.600000in}}{\pgfqpoint{3.900000in}{3.900000in}}%
\pgfusepath{clip}%
\pgfsetbuttcap%
\pgfsetroundjoin%
\definecolor{currentfill}{rgb}{0.121569,0.466667,0.705882}%
\pgfsetfillcolor{currentfill}%
\pgfsetlinewidth{1.003750pt}%
\definecolor{currentstroke}{rgb}{0.121569,0.466667,0.705882}%
\pgfsetstrokecolor{currentstroke}%
\pgfsetdash{}{0pt}%
\pgfpathmoveto{\pgfqpoint{2.259370in}{2.393775in}}%
\pgfpathcurveto{\pgfqpoint{2.270420in}{2.393775in}}{\pgfqpoint{2.281019in}{2.398166in}}{\pgfqpoint{2.288833in}{2.405979in}}%
\pgfpathcurveto{\pgfqpoint{2.296646in}{2.413793in}}{\pgfqpoint{2.301037in}{2.424392in}}{\pgfqpoint{2.301037in}{2.435442in}}%
\pgfpathcurveto{\pgfqpoint{2.301037in}{2.446492in}}{\pgfqpoint{2.296646in}{2.457091in}}{\pgfqpoint{2.288833in}{2.464905in}}%
\pgfpathcurveto{\pgfqpoint{2.281019in}{2.472718in}}{\pgfqpoint{2.270420in}{2.477109in}}{\pgfqpoint{2.259370in}{2.477109in}}%
\pgfpathcurveto{\pgfqpoint{2.248320in}{2.477109in}}{\pgfqpoint{2.237721in}{2.472718in}}{\pgfqpoint{2.229907in}{2.464905in}}%
\pgfpathcurveto{\pgfqpoint{2.222094in}{2.457091in}}{\pgfqpoint{2.217703in}{2.446492in}}{\pgfqpoint{2.217703in}{2.435442in}}%
\pgfpathcurveto{\pgfqpoint{2.217703in}{2.424392in}}{\pgfqpoint{2.222094in}{2.413793in}}{\pgfqpoint{2.229907in}{2.405979in}}%
\pgfpathcurveto{\pgfqpoint{2.237721in}{2.398166in}}{\pgfqpoint{2.248320in}{2.393775in}}{\pgfqpoint{2.259370in}{2.393775in}}%
\pgfpathclose%
\pgfusepath{stroke,fill}%
\end{pgfscope}%
\begin{pgfscope}%
\pgfpathrectangle{\pgfqpoint{0.600000in}{0.600000in}}{\pgfqpoint{3.900000in}{3.900000in}}%
\pgfusepath{clip}%
\pgfsetbuttcap%
\pgfsetroundjoin%
\definecolor{currentfill}{rgb}{0.121569,0.466667,0.705882}%
\pgfsetfillcolor{currentfill}%
\pgfsetlinewidth{1.003750pt}%
\definecolor{currentstroke}{rgb}{0.121569,0.466667,0.705882}%
\pgfsetstrokecolor{currentstroke}%
\pgfsetdash{}{0pt}%
\pgfpathmoveto{\pgfqpoint{1.710501in}{2.271512in}}%
\pgfpathcurveto{\pgfqpoint{1.721551in}{2.271512in}}{\pgfqpoint{1.732150in}{2.275902in}}{\pgfqpoint{1.739964in}{2.283716in}}%
\pgfpathcurveto{\pgfqpoint{1.747777in}{2.291530in}}{\pgfqpoint{1.752168in}{2.302129in}}{\pgfqpoint{1.752168in}{2.313179in}}%
\pgfpathcurveto{\pgfqpoint{1.752168in}{2.324229in}}{\pgfqpoint{1.747777in}{2.334828in}}{\pgfqpoint{1.739964in}{2.342642in}}%
\pgfpathcurveto{\pgfqpoint{1.732150in}{2.350455in}}{\pgfqpoint{1.721551in}{2.354846in}}{\pgfqpoint{1.710501in}{2.354846in}}%
\pgfpathcurveto{\pgfqpoint{1.699451in}{2.354846in}}{\pgfqpoint{1.688852in}{2.350455in}}{\pgfqpoint{1.681038in}{2.342642in}}%
\pgfpathcurveto{\pgfqpoint{1.673225in}{2.334828in}}{\pgfqpoint{1.668834in}{2.324229in}}{\pgfqpoint{1.668834in}{2.313179in}}%
\pgfpathcurveto{\pgfqpoint{1.668834in}{2.302129in}}{\pgfqpoint{1.673225in}{2.291530in}}{\pgfqpoint{1.681038in}{2.283716in}}%
\pgfpathcurveto{\pgfqpoint{1.688852in}{2.275902in}}{\pgfqpoint{1.699451in}{2.271512in}}{\pgfqpoint{1.710501in}{2.271512in}}%
\pgfpathclose%
\pgfusepath{stroke,fill}%
\end{pgfscope}%
\begin{pgfscope}%
\pgfpathrectangle{\pgfqpoint{0.600000in}{0.600000in}}{\pgfqpoint{3.900000in}{3.900000in}}%
\pgfusepath{clip}%
\pgfsetbuttcap%
\pgfsetroundjoin%
\definecolor{currentfill}{rgb}{0.121569,0.466667,0.705882}%
\pgfsetfillcolor{currentfill}%
\pgfsetlinewidth{1.003750pt}%
\definecolor{currentstroke}{rgb}{0.121569,0.466667,0.705882}%
\pgfsetstrokecolor{currentstroke}%
\pgfsetdash{}{0pt}%
\pgfpathmoveto{\pgfqpoint{2.209684in}{2.943348in}}%
\pgfpathcurveto{\pgfqpoint{2.220734in}{2.943348in}}{\pgfqpoint{2.231333in}{2.947738in}}{\pgfqpoint{2.239147in}{2.955552in}}%
\pgfpathcurveto{\pgfqpoint{2.246960in}{2.963366in}}{\pgfqpoint{2.251351in}{2.973965in}}{\pgfqpoint{2.251351in}{2.985015in}}%
\pgfpathcurveto{\pgfqpoint{2.251351in}{2.996065in}}{\pgfqpoint{2.246960in}{3.006664in}}{\pgfqpoint{2.239147in}{3.014478in}}%
\pgfpathcurveto{\pgfqpoint{2.231333in}{3.022291in}}{\pgfqpoint{2.220734in}{3.026681in}}{\pgfqpoint{2.209684in}{3.026681in}}%
\pgfpathcurveto{\pgfqpoint{2.198634in}{3.026681in}}{\pgfqpoint{2.188035in}{3.022291in}}{\pgfqpoint{2.180221in}{3.014478in}}%
\pgfpathcurveto{\pgfqpoint{2.172407in}{3.006664in}}{\pgfqpoint{2.168017in}{2.996065in}}{\pgfqpoint{2.168017in}{2.985015in}}%
\pgfpathcurveto{\pgfqpoint{2.168017in}{2.973965in}}{\pgfqpoint{2.172407in}{2.963366in}}{\pgfqpoint{2.180221in}{2.955552in}}%
\pgfpathcurveto{\pgfqpoint{2.188035in}{2.947738in}}{\pgfqpoint{2.198634in}{2.943348in}}{\pgfqpoint{2.209684in}{2.943348in}}%
\pgfpathclose%
\pgfusepath{stroke,fill}%
\end{pgfscope}%
\begin{pgfscope}%
\pgfpathrectangle{\pgfqpoint{0.600000in}{0.600000in}}{\pgfqpoint{3.900000in}{3.900000in}}%
\pgfusepath{clip}%
\pgfsetbuttcap%
\pgfsetroundjoin%
\definecolor{currentfill}{rgb}{0.121569,0.466667,0.705882}%
\pgfsetfillcolor{currentfill}%
\pgfsetlinewidth{1.003750pt}%
\definecolor{currentstroke}{rgb}{0.121569,0.466667,0.705882}%
\pgfsetstrokecolor{currentstroke}%
\pgfsetdash{}{0pt}%
\pgfpathmoveto{\pgfqpoint{2.532145in}{2.443906in}}%
\pgfpathcurveto{\pgfqpoint{2.543195in}{2.443906in}}{\pgfqpoint{2.553794in}{2.448297in}}{\pgfqpoint{2.561607in}{2.456110in}}%
\pgfpathcurveto{\pgfqpoint{2.569421in}{2.463924in}}{\pgfqpoint{2.573811in}{2.474523in}}{\pgfqpoint{2.573811in}{2.485573in}}%
\pgfpathcurveto{\pgfqpoint{2.573811in}{2.496623in}}{\pgfqpoint{2.569421in}{2.507222in}}{\pgfqpoint{2.561607in}{2.515036in}}%
\pgfpathcurveto{\pgfqpoint{2.553794in}{2.522849in}}{\pgfqpoint{2.543195in}{2.527240in}}{\pgfqpoint{2.532145in}{2.527240in}}%
\pgfpathcurveto{\pgfqpoint{2.521094in}{2.527240in}}{\pgfqpoint{2.510495in}{2.522849in}}{\pgfqpoint{2.502682in}{2.515036in}}%
\pgfpathcurveto{\pgfqpoint{2.494868in}{2.507222in}}{\pgfqpoint{2.490478in}{2.496623in}}{\pgfqpoint{2.490478in}{2.485573in}}%
\pgfpathcurveto{\pgfqpoint{2.490478in}{2.474523in}}{\pgfqpoint{2.494868in}{2.463924in}}{\pgfqpoint{2.502682in}{2.456110in}}%
\pgfpathcurveto{\pgfqpoint{2.510495in}{2.448297in}}{\pgfqpoint{2.521094in}{2.443906in}}{\pgfqpoint{2.532145in}{2.443906in}}%
\pgfpathclose%
\pgfusepath{stroke,fill}%
\end{pgfscope}%
\begin{pgfscope}%
\pgfpathrectangle{\pgfqpoint{0.600000in}{0.600000in}}{\pgfqpoint{3.900000in}{3.900000in}}%
\pgfusepath{clip}%
\pgfsetbuttcap%
\pgfsetroundjoin%
\definecolor{currentfill}{rgb}{0.121569,0.466667,0.705882}%
\pgfsetfillcolor{currentfill}%
\pgfsetlinewidth{1.003750pt}%
\definecolor{currentstroke}{rgb}{0.121569,0.466667,0.705882}%
\pgfsetstrokecolor{currentstroke}%
\pgfsetdash{}{0pt}%
\pgfpathmoveto{\pgfqpoint{2.583798in}{2.101759in}}%
\pgfpathcurveto{\pgfqpoint{2.594848in}{2.101759in}}{\pgfqpoint{2.605447in}{2.106149in}}{\pgfqpoint{2.613261in}{2.113963in}}%
\pgfpathcurveto{\pgfqpoint{2.621075in}{2.121776in}}{\pgfqpoint{2.625465in}{2.132375in}}{\pgfqpoint{2.625465in}{2.143425in}}%
\pgfpathcurveto{\pgfqpoint{2.625465in}{2.154476in}}{\pgfqpoint{2.621075in}{2.165075in}}{\pgfqpoint{2.613261in}{2.172888in}}%
\pgfpathcurveto{\pgfqpoint{2.605447in}{2.180702in}}{\pgfqpoint{2.594848in}{2.185092in}}{\pgfqpoint{2.583798in}{2.185092in}}%
\pgfpathcurveto{\pgfqpoint{2.572748in}{2.185092in}}{\pgfqpoint{2.562149in}{2.180702in}}{\pgfqpoint{2.554335in}{2.172888in}}%
\pgfpathcurveto{\pgfqpoint{2.546522in}{2.165075in}}{\pgfqpoint{2.542132in}{2.154476in}}{\pgfqpoint{2.542132in}{2.143425in}}%
\pgfpathcurveto{\pgfqpoint{2.542132in}{2.132375in}}{\pgfqpoint{2.546522in}{2.121776in}}{\pgfqpoint{2.554335in}{2.113963in}}%
\pgfpathcurveto{\pgfqpoint{2.562149in}{2.106149in}}{\pgfqpoint{2.572748in}{2.101759in}}{\pgfqpoint{2.583798in}{2.101759in}}%
\pgfpathclose%
\pgfusepath{stroke,fill}%
\end{pgfscope}%
\begin{pgfscope}%
\pgfpathrectangle{\pgfqpoint{0.600000in}{0.600000in}}{\pgfqpoint{3.900000in}{3.900000in}}%
\pgfusepath{clip}%
\pgfsetbuttcap%
\pgfsetroundjoin%
\definecolor{currentfill}{rgb}{0.121569,0.466667,0.705882}%
\pgfsetfillcolor{currentfill}%
\pgfsetlinewidth{1.003750pt}%
\definecolor{currentstroke}{rgb}{0.121569,0.466667,0.705882}%
\pgfsetstrokecolor{currentstroke}%
\pgfsetdash{}{0pt}%
\pgfpathmoveto{\pgfqpoint{3.229823in}{2.713790in}}%
\pgfpathcurveto{\pgfqpoint{3.240873in}{2.713790in}}{\pgfqpoint{3.251472in}{2.718181in}}{\pgfqpoint{3.259286in}{2.725994in}}%
\pgfpathcurveto{\pgfqpoint{3.267099in}{2.733808in}}{\pgfqpoint{3.271490in}{2.744407in}}{\pgfqpoint{3.271490in}{2.755457in}}%
\pgfpathcurveto{\pgfqpoint{3.271490in}{2.766507in}}{\pgfqpoint{3.267099in}{2.777106in}}{\pgfqpoint{3.259286in}{2.784920in}}%
\pgfpathcurveto{\pgfqpoint{3.251472in}{2.792733in}}{\pgfqpoint{3.240873in}{2.797124in}}{\pgfqpoint{3.229823in}{2.797124in}}%
\pgfpathcurveto{\pgfqpoint{3.218773in}{2.797124in}}{\pgfqpoint{3.208174in}{2.792733in}}{\pgfqpoint{3.200360in}{2.784920in}}%
\pgfpathcurveto{\pgfqpoint{3.192547in}{2.777106in}}{\pgfqpoint{3.188156in}{2.766507in}}{\pgfqpoint{3.188156in}{2.755457in}}%
\pgfpathcurveto{\pgfqpoint{3.188156in}{2.744407in}}{\pgfqpoint{3.192547in}{2.733808in}}{\pgfqpoint{3.200360in}{2.725994in}}%
\pgfpathcurveto{\pgfqpoint{3.208174in}{2.718181in}}{\pgfqpoint{3.218773in}{2.713790in}}{\pgfqpoint{3.229823in}{2.713790in}}%
\pgfpathclose%
\pgfusepath{stroke,fill}%
\end{pgfscope}%
\begin{pgfscope}%
\pgfpathrectangle{\pgfqpoint{0.600000in}{0.600000in}}{\pgfqpoint{3.900000in}{3.900000in}}%
\pgfusepath{clip}%
\pgfsetbuttcap%
\pgfsetroundjoin%
\definecolor{currentfill}{rgb}{0.121569,0.466667,0.705882}%
\pgfsetfillcolor{currentfill}%
\pgfsetlinewidth{1.003750pt}%
\definecolor{currentstroke}{rgb}{0.121569,0.466667,0.705882}%
\pgfsetstrokecolor{currentstroke}%
\pgfsetdash{}{0pt}%
\pgfpathmoveto{\pgfqpoint{2.322073in}{3.219038in}}%
\pgfpathcurveto{\pgfqpoint{2.333123in}{3.219038in}}{\pgfqpoint{2.343722in}{3.223428in}}{\pgfqpoint{2.351536in}{3.231242in}}%
\pgfpathcurveto{\pgfqpoint{2.359349in}{3.239055in}}{\pgfqpoint{2.363740in}{3.249654in}}{\pgfqpoint{2.363740in}{3.260704in}}%
\pgfpathcurveto{\pgfqpoint{2.363740in}{3.271755in}}{\pgfqpoint{2.359349in}{3.282354in}}{\pgfqpoint{2.351536in}{3.290167in}}%
\pgfpathcurveto{\pgfqpoint{2.343722in}{3.297981in}}{\pgfqpoint{2.333123in}{3.302371in}}{\pgfqpoint{2.322073in}{3.302371in}}%
\pgfpathcurveto{\pgfqpoint{2.311023in}{3.302371in}}{\pgfqpoint{2.300424in}{3.297981in}}{\pgfqpoint{2.292610in}{3.290167in}}%
\pgfpathcurveto{\pgfqpoint{2.284797in}{3.282354in}}{\pgfqpoint{2.280406in}{3.271755in}}{\pgfqpoint{2.280406in}{3.260704in}}%
\pgfpathcurveto{\pgfqpoint{2.280406in}{3.249654in}}{\pgfqpoint{2.284797in}{3.239055in}}{\pgfqpoint{2.292610in}{3.231242in}}%
\pgfpathcurveto{\pgfqpoint{2.300424in}{3.223428in}}{\pgfqpoint{2.311023in}{3.219038in}}{\pgfqpoint{2.322073in}{3.219038in}}%
\pgfpathclose%
\pgfusepath{stroke,fill}%
\end{pgfscope}%
\begin{pgfscope}%
\pgfpathrectangle{\pgfqpoint{0.600000in}{0.600000in}}{\pgfqpoint{3.900000in}{3.900000in}}%
\pgfusepath{clip}%
\pgfsetbuttcap%
\pgfsetroundjoin%
\definecolor{currentfill}{rgb}{0.121569,0.466667,0.705882}%
\pgfsetfillcolor{currentfill}%
\pgfsetlinewidth{1.003750pt}%
\definecolor{currentstroke}{rgb}{0.121569,0.466667,0.705882}%
\pgfsetstrokecolor{currentstroke}%
\pgfsetdash{}{0pt}%
\pgfpathmoveto{\pgfqpoint{2.629757in}{3.087707in}}%
\pgfpathcurveto{\pgfqpoint{2.640807in}{3.087707in}}{\pgfqpoint{2.651406in}{3.092097in}}{\pgfqpoint{2.659219in}{3.099911in}}%
\pgfpathcurveto{\pgfqpoint{2.667033in}{3.107725in}}{\pgfqpoint{2.671423in}{3.118324in}}{\pgfqpoint{2.671423in}{3.129374in}}%
\pgfpathcurveto{\pgfqpoint{2.671423in}{3.140424in}}{\pgfqpoint{2.667033in}{3.151023in}}{\pgfqpoint{2.659219in}{3.158837in}}%
\pgfpathcurveto{\pgfqpoint{2.651406in}{3.166650in}}{\pgfqpoint{2.640807in}{3.171040in}}{\pgfqpoint{2.629757in}{3.171040in}}%
\pgfpathcurveto{\pgfqpoint{2.618706in}{3.171040in}}{\pgfqpoint{2.608107in}{3.166650in}}{\pgfqpoint{2.600294in}{3.158837in}}%
\pgfpathcurveto{\pgfqpoint{2.592480in}{3.151023in}}{\pgfqpoint{2.588090in}{3.140424in}}{\pgfqpoint{2.588090in}{3.129374in}}%
\pgfpathcurveto{\pgfqpoint{2.588090in}{3.118324in}}{\pgfqpoint{2.592480in}{3.107725in}}{\pgfqpoint{2.600294in}{3.099911in}}%
\pgfpathcurveto{\pgfqpoint{2.608107in}{3.092097in}}{\pgfqpoint{2.618706in}{3.087707in}}{\pgfqpoint{2.629757in}{3.087707in}}%
\pgfpathclose%
\pgfusepath{stroke,fill}%
\end{pgfscope}%
\begin{pgfscope}%
\pgfpathrectangle{\pgfqpoint{0.600000in}{0.600000in}}{\pgfqpoint{3.900000in}{3.900000in}}%
\pgfusepath{clip}%
\pgfsetbuttcap%
\pgfsetroundjoin%
\definecolor{currentfill}{rgb}{0.121569,0.466667,0.705882}%
\pgfsetfillcolor{currentfill}%
\pgfsetlinewidth{1.003750pt}%
\definecolor{currentstroke}{rgb}{0.121569,0.466667,0.705882}%
\pgfsetstrokecolor{currentstroke}%
\pgfsetdash{}{0pt}%
\pgfpathmoveto{\pgfqpoint{2.608280in}{2.294671in}}%
\pgfpathcurveto{\pgfqpoint{2.619330in}{2.294671in}}{\pgfqpoint{2.629929in}{2.299061in}}{\pgfqpoint{2.637742in}{2.306874in}}%
\pgfpathcurveto{\pgfqpoint{2.645556in}{2.314688in}}{\pgfqpoint{2.649946in}{2.325287in}}{\pgfqpoint{2.649946in}{2.336337in}}%
\pgfpathcurveto{\pgfqpoint{2.649946in}{2.347387in}}{\pgfqpoint{2.645556in}{2.357986in}}{\pgfqpoint{2.637742in}{2.365800in}}%
\pgfpathcurveto{\pgfqpoint{2.629929in}{2.373614in}}{\pgfqpoint{2.619330in}{2.378004in}}{\pgfqpoint{2.608280in}{2.378004in}}%
\pgfpathcurveto{\pgfqpoint{2.597229in}{2.378004in}}{\pgfqpoint{2.586630in}{2.373614in}}{\pgfqpoint{2.578817in}{2.365800in}}%
\pgfpathcurveto{\pgfqpoint{2.571003in}{2.357986in}}{\pgfqpoint{2.566613in}{2.347387in}}{\pgfqpoint{2.566613in}{2.336337in}}%
\pgfpathcurveto{\pgfqpoint{2.566613in}{2.325287in}}{\pgfqpoint{2.571003in}{2.314688in}}{\pgfqpoint{2.578817in}{2.306874in}}%
\pgfpathcurveto{\pgfqpoint{2.586630in}{2.299061in}}{\pgfqpoint{2.597229in}{2.294671in}}{\pgfqpoint{2.608280in}{2.294671in}}%
\pgfpathclose%
\pgfusepath{stroke,fill}%
\end{pgfscope}%
\begin{pgfscope}%
\pgfpathrectangle{\pgfqpoint{0.600000in}{0.600000in}}{\pgfqpoint{3.900000in}{3.900000in}}%
\pgfusepath{clip}%
\pgfsetbuttcap%
\pgfsetroundjoin%
\definecolor{currentfill}{rgb}{0.121569,0.466667,0.705882}%
\pgfsetfillcolor{currentfill}%
\pgfsetlinewidth{1.003750pt}%
\definecolor{currentstroke}{rgb}{0.121569,0.466667,0.705882}%
\pgfsetstrokecolor{currentstroke}%
\pgfsetdash{}{0pt}%
\pgfpathmoveto{\pgfqpoint{1.375353in}{1.570115in}}%
\pgfpathcurveto{\pgfqpoint{1.386404in}{1.570115in}}{\pgfqpoint{1.397003in}{1.574505in}}{\pgfqpoint{1.404816in}{1.582319in}}%
\pgfpathcurveto{\pgfqpoint{1.412630in}{1.590132in}}{\pgfqpoint{1.417020in}{1.600731in}}{\pgfqpoint{1.417020in}{1.611781in}}%
\pgfpathcurveto{\pgfqpoint{1.417020in}{1.622832in}}{\pgfqpoint{1.412630in}{1.633431in}}{\pgfqpoint{1.404816in}{1.641244in}}%
\pgfpathcurveto{\pgfqpoint{1.397003in}{1.649058in}}{\pgfqpoint{1.386404in}{1.653448in}}{\pgfqpoint{1.375353in}{1.653448in}}%
\pgfpathcurveto{\pgfqpoint{1.364303in}{1.653448in}}{\pgfqpoint{1.353704in}{1.649058in}}{\pgfqpoint{1.345891in}{1.641244in}}%
\pgfpathcurveto{\pgfqpoint{1.338077in}{1.633431in}}{\pgfqpoint{1.333687in}{1.622832in}}{\pgfqpoint{1.333687in}{1.611781in}}%
\pgfpathcurveto{\pgfqpoint{1.333687in}{1.600731in}}{\pgfqpoint{1.338077in}{1.590132in}}{\pgfqpoint{1.345891in}{1.582319in}}%
\pgfpathcurveto{\pgfqpoint{1.353704in}{1.574505in}}{\pgfqpoint{1.364303in}{1.570115in}}{\pgfqpoint{1.375353in}{1.570115in}}%
\pgfpathclose%
\pgfusepath{stroke,fill}%
\end{pgfscope}%
\begin{pgfscope}%
\pgfpathrectangle{\pgfqpoint{0.600000in}{0.600000in}}{\pgfqpoint{3.900000in}{3.900000in}}%
\pgfusepath{clip}%
\pgfsetbuttcap%
\pgfsetroundjoin%
\definecolor{currentfill}{rgb}{0.121569,0.466667,0.705882}%
\pgfsetfillcolor{currentfill}%
\pgfsetlinewidth{1.003750pt}%
\definecolor{currentstroke}{rgb}{0.121569,0.466667,0.705882}%
\pgfsetstrokecolor{currentstroke}%
\pgfsetdash{}{0pt}%
\pgfpathmoveto{\pgfqpoint{2.303793in}{2.833666in}}%
\pgfpathcurveto{\pgfqpoint{2.314843in}{2.833666in}}{\pgfqpoint{2.325442in}{2.838056in}}{\pgfqpoint{2.333256in}{2.845870in}}%
\pgfpathcurveto{\pgfqpoint{2.341070in}{2.853683in}}{\pgfqpoint{2.345460in}{2.864282in}}{\pgfqpoint{2.345460in}{2.875332in}}%
\pgfpathcurveto{\pgfqpoint{2.345460in}{2.886383in}}{\pgfqpoint{2.341070in}{2.896982in}}{\pgfqpoint{2.333256in}{2.904795in}}%
\pgfpathcurveto{\pgfqpoint{2.325442in}{2.912609in}}{\pgfqpoint{2.314843in}{2.916999in}}{\pgfqpoint{2.303793in}{2.916999in}}%
\pgfpathcurveto{\pgfqpoint{2.292743in}{2.916999in}}{\pgfqpoint{2.282144in}{2.912609in}}{\pgfqpoint{2.274330in}{2.904795in}}%
\pgfpathcurveto{\pgfqpoint{2.266517in}{2.896982in}}{\pgfqpoint{2.262127in}{2.886383in}}{\pgfqpoint{2.262127in}{2.875332in}}%
\pgfpathcurveto{\pgfqpoint{2.262127in}{2.864282in}}{\pgfqpoint{2.266517in}{2.853683in}}{\pgfqpoint{2.274330in}{2.845870in}}%
\pgfpathcurveto{\pgfqpoint{2.282144in}{2.838056in}}{\pgfqpoint{2.292743in}{2.833666in}}{\pgfqpoint{2.303793in}{2.833666in}}%
\pgfpathclose%
\pgfusepath{stroke,fill}%
\end{pgfscope}%
\begin{pgfscope}%
\pgfpathrectangle{\pgfqpoint{0.600000in}{0.600000in}}{\pgfqpoint{3.900000in}{3.900000in}}%
\pgfusepath{clip}%
\pgfsetbuttcap%
\pgfsetroundjoin%
\definecolor{currentfill}{rgb}{0.121569,0.466667,0.705882}%
\pgfsetfillcolor{currentfill}%
\pgfsetlinewidth{1.003750pt}%
\definecolor{currentstroke}{rgb}{0.121569,0.466667,0.705882}%
\pgfsetstrokecolor{currentstroke}%
\pgfsetdash{}{0pt}%
\pgfpathmoveto{\pgfqpoint{1.879565in}{3.076711in}}%
\pgfpathcurveto{\pgfqpoint{1.890615in}{3.076711in}}{\pgfqpoint{1.901214in}{3.081101in}}{\pgfqpoint{1.909028in}{3.088915in}}%
\pgfpathcurveto{\pgfqpoint{1.916841in}{3.096728in}}{\pgfqpoint{1.921232in}{3.107328in}}{\pgfqpoint{1.921232in}{3.118378in}}%
\pgfpathcurveto{\pgfqpoint{1.921232in}{3.129428in}}{\pgfqpoint{1.916841in}{3.140027in}}{\pgfqpoint{1.909028in}{3.147840in}}%
\pgfpathcurveto{\pgfqpoint{1.901214in}{3.155654in}}{\pgfqpoint{1.890615in}{3.160044in}}{\pgfqpoint{1.879565in}{3.160044in}}%
\pgfpathcurveto{\pgfqpoint{1.868515in}{3.160044in}}{\pgfqpoint{1.857916in}{3.155654in}}{\pgfqpoint{1.850102in}{3.147840in}}%
\pgfpathcurveto{\pgfqpoint{1.842288in}{3.140027in}}{\pgfqpoint{1.837898in}{3.129428in}}{\pgfqpoint{1.837898in}{3.118378in}}%
\pgfpathcurveto{\pgfqpoint{1.837898in}{3.107328in}}{\pgfqpoint{1.842288in}{3.096728in}}{\pgfqpoint{1.850102in}{3.088915in}}%
\pgfpathcurveto{\pgfqpoint{1.857916in}{3.081101in}}{\pgfqpoint{1.868515in}{3.076711in}}{\pgfqpoint{1.879565in}{3.076711in}}%
\pgfpathclose%
\pgfusepath{stroke,fill}%
\end{pgfscope}%
\begin{pgfscope}%
\pgfpathrectangle{\pgfqpoint{0.600000in}{0.600000in}}{\pgfqpoint{3.900000in}{3.900000in}}%
\pgfusepath{clip}%
\pgfsetbuttcap%
\pgfsetroundjoin%
\definecolor{currentfill}{rgb}{0.121569,0.466667,0.705882}%
\pgfsetfillcolor{currentfill}%
\pgfsetlinewidth{1.003750pt}%
\definecolor{currentstroke}{rgb}{0.121569,0.466667,0.705882}%
\pgfsetstrokecolor{currentstroke}%
\pgfsetdash{}{0pt}%
\pgfpathmoveto{\pgfqpoint{2.663925in}{2.869285in}}%
\pgfpathcurveto{\pgfqpoint{2.674975in}{2.869285in}}{\pgfqpoint{2.685574in}{2.873675in}}{\pgfqpoint{2.693387in}{2.881489in}}%
\pgfpathcurveto{\pgfqpoint{2.701201in}{2.889303in}}{\pgfqpoint{2.705591in}{2.899902in}}{\pgfqpoint{2.705591in}{2.910952in}}%
\pgfpathcurveto{\pgfqpoint{2.705591in}{2.922002in}}{\pgfqpoint{2.701201in}{2.932601in}}{\pgfqpoint{2.693387in}{2.940415in}}%
\pgfpathcurveto{\pgfqpoint{2.685574in}{2.948228in}}{\pgfqpoint{2.674975in}{2.952619in}}{\pgfqpoint{2.663925in}{2.952619in}}%
\pgfpathcurveto{\pgfqpoint{2.652874in}{2.952619in}}{\pgfqpoint{2.642275in}{2.948228in}}{\pgfqpoint{2.634462in}{2.940415in}}%
\pgfpathcurveto{\pgfqpoint{2.626648in}{2.932601in}}{\pgfqpoint{2.622258in}{2.922002in}}{\pgfqpoint{2.622258in}{2.910952in}}%
\pgfpathcurveto{\pgfqpoint{2.622258in}{2.899902in}}{\pgfqpoint{2.626648in}{2.889303in}}{\pgfqpoint{2.634462in}{2.881489in}}%
\pgfpathcurveto{\pgfqpoint{2.642275in}{2.873675in}}{\pgfqpoint{2.652874in}{2.869285in}}{\pgfqpoint{2.663925in}{2.869285in}}%
\pgfpathclose%
\pgfusepath{stroke,fill}%
\end{pgfscope}%
\begin{pgfscope}%
\pgfpathrectangle{\pgfqpoint{0.600000in}{0.600000in}}{\pgfqpoint{3.900000in}{3.900000in}}%
\pgfusepath{clip}%
\pgfsetbuttcap%
\pgfsetroundjoin%
\definecolor{currentfill}{rgb}{0.121569,0.466667,0.705882}%
\pgfsetfillcolor{currentfill}%
\pgfsetlinewidth{1.003750pt}%
\definecolor{currentstroke}{rgb}{0.121569,0.466667,0.705882}%
\pgfsetstrokecolor{currentstroke}%
\pgfsetdash{}{0pt}%
\pgfpathmoveto{\pgfqpoint{3.670881in}{2.639921in}}%
\pgfpathcurveto{\pgfqpoint{3.681931in}{2.639921in}}{\pgfqpoint{3.692530in}{2.644311in}}{\pgfqpoint{3.700344in}{2.652125in}}%
\pgfpathcurveto{\pgfqpoint{3.708158in}{2.659938in}}{\pgfqpoint{3.712548in}{2.670537in}}{\pgfqpoint{3.712548in}{2.681587in}}%
\pgfpathcurveto{\pgfqpoint{3.712548in}{2.692637in}}{\pgfqpoint{3.708158in}{2.703237in}}{\pgfqpoint{3.700344in}{2.711050in}}%
\pgfpathcurveto{\pgfqpoint{3.692530in}{2.718864in}}{\pgfqpoint{3.681931in}{2.723254in}}{\pgfqpoint{3.670881in}{2.723254in}}%
\pgfpathcurveto{\pgfqpoint{3.659831in}{2.723254in}}{\pgfqpoint{3.649232in}{2.718864in}}{\pgfqpoint{3.641418in}{2.711050in}}%
\pgfpathcurveto{\pgfqpoint{3.633605in}{2.703237in}}{\pgfqpoint{3.629215in}{2.692637in}}{\pgfqpoint{3.629215in}{2.681587in}}%
\pgfpathcurveto{\pgfqpoint{3.629215in}{2.670537in}}{\pgfqpoint{3.633605in}{2.659938in}}{\pgfqpoint{3.641418in}{2.652125in}}%
\pgfpathcurveto{\pgfqpoint{3.649232in}{2.644311in}}{\pgfqpoint{3.659831in}{2.639921in}}{\pgfqpoint{3.670881in}{2.639921in}}%
\pgfpathclose%
\pgfusepath{stroke,fill}%
\end{pgfscope}%
\begin{pgfscope}%
\pgfpathrectangle{\pgfqpoint{0.600000in}{0.600000in}}{\pgfqpoint{3.900000in}{3.900000in}}%
\pgfusepath{clip}%
\pgfsetbuttcap%
\pgfsetroundjoin%
\definecolor{currentfill}{rgb}{0.121569,0.466667,0.705882}%
\pgfsetfillcolor{currentfill}%
\pgfsetlinewidth{1.003750pt}%
\definecolor{currentstroke}{rgb}{0.121569,0.466667,0.705882}%
\pgfsetstrokecolor{currentstroke}%
\pgfsetdash{}{0pt}%
\pgfpathmoveto{\pgfqpoint{2.748224in}{2.606563in}}%
\pgfpathcurveto{\pgfqpoint{2.759274in}{2.606563in}}{\pgfqpoint{2.769873in}{2.610953in}}{\pgfqpoint{2.777687in}{2.618767in}}%
\pgfpathcurveto{\pgfqpoint{2.785501in}{2.626580in}}{\pgfqpoint{2.789891in}{2.637180in}}{\pgfqpoint{2.789891in}{2.648230in}}%
\pgfpathcurveto{\pgfqpoint{2.789891in}{2.659280in}}{\pgfqpoint{2.785501in}{2.669879in}}{\pgfqpoint{2.777687in}{2.677692in}}%
\pgfpathcurveto{\pgfqpoint{2.769873in}{2.685506in}}{\pgfqpoint{2.759274in}{2.689896in}}{\pgfqpoint{2.748224in}{2.689896in}}%
\pgfpathcurveto{\pgfqpoint{2.737174in}{2.689896in}}{\pgfqpoint{2.726575in}{2.685506in}}{\pgfqpoint{2.718762in}{2.677692in}}%
\pgfpathcurveto{\pgfqpoint{2.710948in}{2.669879in}}{\pgfqpoint{2.706558in}{2.659280in}}{\pgfqpoint{2.706558in}{2.648230in}}%
\pgfpathcurveto{\pgfqpoint{2.706558in}{2.637180in}}{\pgfqpoint{2.710948in}{2.626580in}}{\pgfqpoint{2.718762in}{2.618767in}}%
\pgfpathcurveto{\pgfqpoint{2.726575in}{2.610953in}}{\pgfqpoint{2.737174in}{2.606563in}}{\pgfqpoint{2.748224in}{2.606563in}}%
\pgfpathclose%
\pgfusepath{stroke,fill}%
\end{pgfscope}%
\begin{pgfscope}%
\pgfpathrectangle{\pgfqpoint{0.600000in}{0.600000in}}{\pgfqpoint{3.900000in}{3.900000in}}%
\pgfusepath{clip}%
\pgfsetbuttcap%
\pgfsetroundjoin%
\definecolor{currentfill}{rgb}{0.121569,0.466667,0.705882}%
\pgfsetfillcolor{currentfill}%
\pgfsetlinewidth{1.003750pt}%
\definecolor{currentstroke}{rgb}{0.121569,0.466667,0.705882}%
\pgfsetstrokecolor{currentstroke}%
\pgfsetdash{}{0pt}%
\pgfpathmoveto{\pgfqpoint{4.032498in}{2.968297in}}%
\pgfpathcurveto{\pgfqpoint{4.043548in}{2.968297in}}{\pgfqpoint{4.054147in}{2.972688in}}{\pgfqpoint{4.061961in}{2.980501in}}%
\pgfpathcurveto{\pgfqpoint{4.069775in}{2.988315in}}{\pgfqpoint{4.074165in}{2.998914in}}{\pgfqpoint{4.074165in}{3.009964in}}%
\pgfpathcurveto{\pgfqpoint{4.074165in}{3.021014in}}{\pgfqpoint{4.069775in}{3.031613in}}{\pgfqpoint{4.061961in}{3.039427in}}%
\pgfpathcurveto{\pgfqpoint{4.054147in}{3.047240in}}{\pgfqpoint{4.043548in}{3.051631in}}{\pgfqpoint{4.032498in}{3.051631in}}%
\pgfpathcurveto{\pgfqpoint{4.021448in}{3.051631in}}{\pgfqpoint{4.010849in}{3.047240in}}{\pgfqpoint{4.003035in}{3.039427in}}%
\pgfpathcurveto{\pgfqpoint{3.995222in}{3.031613in}}{\pgfqpoint{3.990832in}{3.021014in}}{\pgfqpoint{3.990832in}{3.009964in}}%
\pgfpathcurveto{\pgfqpoint{3.990832in}{2.998914in}}{\pgfqpoint{3.995222in}{2.988315in}}{\pgfqpoint{4.003035in}{2.980501in}}%
\pgfpathcurveto{\pgfqpoint{4.010849in}{2.972688in}}{\pgfqpoint{4.021448in}{2.968297in}}{\pgfqpoint{4.032498in}{2.968297in}}%
\pgfpathclose%
\pgfusepath{stroke,fill}%
\end{pgfscope}%
\begin{pgfscope}%
\pgfpathrectangle{\pgfqpoint{0.600000in}{0.600000in}}{\pgfqpoint{3.900000in}{3.900000in}}%
\pgfusepath{clip}%
\pgfsetbuttcap%
\pgfsetroundjoin%
\definecolor{currentfill}{rgb}{0.121569,0.466667,0.705882}%
\pgfsetfillcolor{currentfill}%
\pgfsetlinewidth{1.003750pt}%
\definecolor{currentstroke}{rgb}{0.121569,0.466667,0.705882}%
\pgfsetstrokecolor{currentstroke}%
\pgfsetdash{}{0pt}%
\pgfpathmoveto{\pgfqpoint{2.308159in}{1.324175in}}%
\pgfpathcurveto{\pgfqpoint{2.319209in}{1.324175in}}{\pgfqpoint{2.329808in}{1.328566in}}{\pgfqpoint{2.337622in}{1.336379in}}%
\pgfpathcurveto{\pgfqpoint{2.345435in}{1.344193in}}{\pgfqpoint{2.349825in}{1.354792in}}{\pgfqpoint{2.349825in}{1.365842in}}%
\pgfpathcurveto{\pgfqpoint{2.349825in}{1.376892in}}{\pgfqpoint{2.345435in}{1.387491in}}{\pgfqpoint{2.337622in}{1.395305in}}%
\pgfpathcurveto{\pgfqpoint{2.329808in}{1.403118in}}{\pgfqpoint{2.319209in}{1.407509in}}{\pgfqpoint{2.308159in}{1.407509in}}%
\pgfpathcurveto{\pgfqpoint{2.297109in}{1.407509in}}{\pgfqpoint{2.286510in}{1.403118in}}{\pgfqpoint{2.278696in}{1.395305in}}%
\pgfpathcurveto{\pgfqpoint{2.270882in}{1.387491in}}{\pgfqpoint{2.266492in}{1.376892in}}{\pgfqpoint{2.266492in}{1.365842in}}%
\pgfpathcurveto{\pgfqpoint{2.266492in}{1.354792in}}{\pgfqpoint{2.270882in}{1.344193in}}{\pgfqpoint{2.278696in}{1.336379in}}%
\pgfpathcurveto{\pgfqpoint{2.286510in}{1.328566in}}{\pgfqpoint{2.297109in}{1.324175in}}{\pgfqpoint{2.308159in}{1.324175in}}%
\pgfpathclose%
\pgfusepath{stroke,fill}%
\end{pgfscope}%
\begin{pgfscope}%
\pgfpathrectangle{\pgfqpoint{0.600000in}{0.600000in}}{\pgfqpoint{3.900000in}{3.900000in}}%
\pgfusepath{clip}%
\pgfsetbuttcap%
\pgfsetroundjoin%
\definecolor{currentfill}{rgb}{0.121569,0.466667,0.705882}%
\pgfsetfillcolor{currentfill}%
\pgfsetlinewidth{1.003750pt}%
\definecolor{currentstroke}{rgb}{0.121569,0.466667,0.705882}%
\pgfsetstrokecolor{currentstroke}%
\pgfsetdash{}{0pt}%
\pgfpathmoveto{\pgfqpoint{1.889163in}{2.479756in}}%
\pgfpathcurveto{\pgfqpoint{1.900213in}{2.479756in}}{\pgfqpoint{1.910812in}{2.484146in}}{\pgfqpoint{1.918626in}{2.491960in}}%
\pgfpathcurveto{\pgfqpoint{1.926440in}{2.499774in}}{\pgfqpoint{1.930830in}{2.510373in}}{\pgfqpoint{1.930830in}{2.521423in}}%
\pgfpathcurveto{\pgfqpoint{1.930830in}{2.532473in}}{\pgfqpoint{1.926440in}{2.543072in}}{\pgfqpoint{1.918626in}{2.550886in}}%
\pgfpathcurveto{\pgfqpoint{1.910812in}{2.558699in}}{\pgfqpoint{1.900213in}{2.563090in}}{\pgfqpoint{1.889163in}{2.563090in}}%
\pgfpathcurveto{\pgfqpoint{1.878113in}{2.563090in}}{\pgfqpoint{1.867514in}{2.558699in}}{\pgfqpoint{1.859700in}{2.550886in}}%
\pgfpathcurveto{\pgfqpoint{1.851887in}{2.543072in}}{\pgfqpoint{1.847496in}{2.532473in}}{\pgfqpoint{1.847496in}{2.521423in}}%
\pgfpathcurveto{\pgfqpoint{1.847496in}{2.510373in}}{\pgfqpoint{1.851887in}{2.499774in}}{\pgfqpoint{1.859700in}{2.491960in}}%
\pgfpathcurveto{\pgfqpoint{1.867514in}{2.484146in}}{\pgfqpoint{1.878113in}{2.479756in}}{\pgfqpoint{1.889163in}{2.479756in}}%
\pgfpathclose%
\pgfusepath{stroke,fill}%
\end{pgfscope}%
\begin{pgfscope}%
\pgfpathrectangle{\pgfqpoint{0.600000in}{0.600000in}}{\pgfqpoint{3.900000in}{3.900000in}}%
\pgfusepath{clip}%
\pgfsetbuttcap%
\pgfsetroundjoin%
\definecolor{currentfill}{rgb}{0.121569,0.466667,0.705882}%
\pgfsetfillcolor{currentfill}%
\pgfsetlinewidth{1.003750pt}%
\definecolor{currentstroke}{rgb}{0.121569,0.466667,0.705882}%
\pgfsetstrokecolor{currentstroke}%
\pgfsetdash{}{0pt}%
\pgfpathmoveto{\pgfqpoint{2.112660in}{2.403920in}}%
\pgfpathcurveto{\pgfqpoint{2.123710in}{2.403920in}}{\pgfqpoint{2.134309in}{2.408310in}}{\pgfqpoint{2.142122in}{2.416124in}}%
\pgfpathcurveto{\pgfqpoint{2.149936in}{2.423937in}}{\pgfqpoint{2.154326in}{2.434536in}}{\pgfqpoint{2.154326in}{2.445586in}}%
\pgfpathcurveto{\pgfqpoint{2.154326in}{2.456637in}}{\pgfqpoint{2.149936in}{2.467236in}}{\pgfqpoint{2.142122in}{2.475049in}}%
\pgfpathcurveto{\pgfqpoint{2.134309in}{2.482863in}}{\pgfqpoint{2.123710in}{2.487253in}}{\pgfqpoint{2.112660in}{2.487253in}}%
\pgfpathcurveto{\pgfqpoint{2.101610in}{2.487253in}}{\pgfqpoint{2.091011in}{2.482863in}}{\pgfqpoint{2.083197in}{2.475049in}}%
\pgfpathcurveto{\pgfqpoint{2.075383in}{2.467236in}}{\pgfqpoint{2.070993in}{2.456637in}}{\pgfqpoint{2.070993in}{2.445586in}}%
\pgfpathcurveto{\pgfqpoint{2.070993in}{2.434536in}}{\pgfqpoint{2.075383in}{2.423937in}}{\pgfqpoint{2.083197in}{2.416124in}}%
\pgfpathcurveto{\pgfqpoint{2.091011in}{2.408310in}}{\pgfqpoint{2.101610in}{2.403920in}}{\pgfqpoint{2.112660in}{2.403920in}}%
\pgfpathclose%
\pgfusepath{stroke,fill}%
\end{pgfscope}%
\begin{pgfscope}%
\pgfpathrectangle{\pgfqpoint{0.600000in}{0.600000in}}{\pgfqpoint{3.900000in}{3.900000in}}%
\pgfusepath{clip}%
\pgfsetbuttcap%
\pgfsetroundjoin%
\definecolor{currentfill}{rgb}{0.121569,0.466667,0.705882}%
\pgfsetfillcolor{currentfill}%
\pgfsetlinewidth{1.003750pt}%
\definecolor{currentstroke}{rgb}{0.121569,0.466667,0.705882}%
\pgfsetstrokecolor{currentstroke}%
\pgfsetdash{}{0pt}%
\pgfpathmoveto{\pgfqpoint{1.793261in}{3.462733in}}%
\pgfpathcurveto{\pgfqpoint{1.804311in}{3.462733in}}{\pgfqpoint{1.814910in}{3.467124in}}{\pgfqpoint{1.822724in}{3.474937in}}%
\pgfpathcurveto{\pgfqpoint{1.830537in}{3.482751in}}{\pgfqpoint{1.834928in}{3.493350in}}{\pgfqpoint{1.834928in}{3.504400in}}%
\pgfpathcurveto{\pgfqpoint{1.834928in}{3.515450in}}{\pgfqpoint{1.830537in}{3.526049in}}{\pgfqpoint{1.822724in}{3.533863in}}%
\pgfpathcurveto{\pgfqpoint{1.814910in}{3.541676in}}{\pgfqpoint{1.804311in}{3.546067in}}{\pgfqpoint{1.793261in}{3.546067in}}%
\pgfpathcurveto{\pgfqpoint{1.782211in}{3.546067in}}{\pgfqpoint{1.771612in}{3.541676in}}{\pgfqpoint{1.763798in}{3.533863in}}%
\pgfpathcurveto{\pgfqpoint{1.755985in}{3.526049in}}{\pgfqpoint{1.751594in}{3.515450in}}{\pgfqpoint{1.751594in}{3.504400in}}%
\pgfpathcurveto{\pgfqpoint{1.751594in}{3.493350in}}{\pgfqpoint{1.755985in}{3.482751in}}{\pgfqpoint{1.763798in}{3.474937in}}%
\pgfpathcurveto{\pgfqpoint{1.771612in}{3.467124in}}{\pgfqpoint{1.782211in}{3.462733in}}{\pgfqpoint{1.793261in}{3.462733in}}%
\pgfpathclose%
\pgfusepath{stroke,fill}%
\end{pgfscope}%
\begin{pgfscope}%
\pgfpathrectangle{\pgfqpoint{0.600000in}{0.600000in}}{\pgfqpoint{3.900000in}{3.900000in}}%
\pgfusepath{clip}%
\pgfsetbuttcap%
\pgfsetroundjoin%
\definecolor{currentfill}{rgb}{0.121569,0.466667,0.705882}%
\pgfsetfillcolor{currentfill}%
\pgfsetlinewidth{1.003750pt}%
\definecolor{currentstroke}{rgb}{0.121569,0.466667,0.705882}%
\pgfsetstrokecolor{currentstroke}%
\pgfsetdash{}{0pt}%
\pgfpathmoveto{\pgfqpoint{1.874910in}{2.310104in}}%
\pgfpathcurveto{\pgfqpoint{1.885960in}{2.310104in}}{\pgfqpoint{1.896559in}{2.314494in}}{\pgfqpoint{1.904373in}{2.322308in}}%
\pgfpathcurveto{\pgfqpoint{1.912187in}{2.330121in}}{\pgfqpoint{1.916577in}{2.340720in}}{\pgfqpoint{1.916577in}{2.351771in}}%
\pgfpathcurveto{\pgfqpoint{1.916577in}{2.362821in}}{\pgfqpoint{1.912187in}{2.373420in}}{\pgfqpoint{1.904373in}{2.381233in}}%
\pgfpathcurveto{\pgfqpoint{1.896559in}{2.389047in}}{\pgfqpoint{1.885960in}{2.393437in}}{\pgfqpoint{1.874910in}{2.393437in}}%
\pgfpathcurveto{\pgfqpoint{1.863860in}{2.393437in}}{\pgfqpoint{1.853261in}{2.389047in}}{\pgfqpoint{1.845447in}{2.381233in}}%
\pgfpathcurveto{\pgfqpoint{1.837634in}{2.373420in}}{\pgfqpoint{1.833243in}{2.362821in}}{\pgfqpoint{1.833243in}{2.351771in}}%
\pgfpathcurveto{\pgfqpoint{1.833243in}{2.340720in}}{\pgfqpoint{1.837634in}{2.330121in}}{\pgfqpoint{1.845447in}{2.322308in}}%
\pgfpathcurveto{\pgfqpoint{1.853261in}{2.314494in}}{\pgfqpoint{1.863860in}{2.310104in}}{\pgfqpoint{1.874910in}{2.310104in}}%
\pgfpathclose%
\pgfusepath{stroke,fill}%
\end{pgfscope}%
\begin{pgfscope}%
\pgfpathrectangle{\pgfqpoint{0.600000in}{0.600000in}}{\pgfqpoint{3.900000in}{3.900000in}}%
\pgfusepath{clip}%
\pgfsetbuttcap%
\pgfsetroundjoin%
\definecolor{currentfill}{rgb}{0.121569,0.466667,0.705882}%
\pgfsetfillcolor{currentfill}%
\pgfsetlinewidth{1.003750pt}%
\definecolor{currentstroke}{rgb}{0.121569,0.466667,0.705882}%
\pgfsetstrokecolor{currentstroke}%
\pgfsetdash{}{0pt}%
\pgfpathmoveto{\pgfqpoint{3.372492in}{1.539851in}}%
\pgfpathcurveto{\pgfqpoint{3.383542in}{1.539851in}}{\pgfqpoint{3.394141in}{1.544241in}}{\pgfqpoint{3.401954in}{1.552055in}}%
\pgfpathcurveto{\pgfqpoint{3.409768in}{1.559869in}}{\pgfqpoint{3.414158in}{1.570468in}}{\pgfqpoint{3.414158in}{1.581518in}}%
\pgfpathcurveto{\pgfqpoint{3.414158in}{1.592568in}}{\pgfqpoint{3.409768in}{1.603167in}}{\pgfqpoint{3.401954in}{1.610980in}}%
\pgfpathcurveto{\pgfqpoint{3.394141in}{1.618794in}}{\pgfqpoint{3.383542in}{1.623184in}}{\pgfqpoint{3.372492in}{1.623184in}}%
\pgfpathcurveto{\pgfqpoint{3.361441in}{1.623184in}}{\pgfqpoint{3.350842in}{1.618794in}}{\pgfqpoint{3.343029in}{1.610980in}}%
\pgfpathcurveto{\pgfqpoint{3.335215in}{1.603167in}}{\pgfqpoint{3.330825in}{1.592568in}}{\pgfqpoint{3.330825in}{1.581518in}}%
\pgfpathcurveto{\pgfqpoint{3.330825in}{1.570468in}}{\pgfqpoint{3.335215in}{1.559869in}}{\pgfqpoint{3.343029in}{1.552055in}}%
\pgfpathcurveto{\pgfqpoint{3.350842in}{1.544241in}}{\pgfqpoint{3.361441in}{1.539851in}}{\pgfqpoint{3.372492in}{1.539851in}}%
\pgfpathclose%
\pgfusepath{stroke,fill}%
\end{pgfscope}%
\begin{pgfscope}%
\pgfpathrectangle{\pgfqpoint{0.600000in}{0.600000in}}{\pgfqpoint{3.900000in}{3.900000in}}%
\pgfusepath{clip}%
\pgfsetbuttcap%
\pgfsetroundjoin%
\definecolor{currentfill}{rgb}{0.121569,0.466667,0.705882}%
\pgfsetfillcolor{currentfill}%
\pgfsetlinewidth{1.003750pt}%
\definecolor{currentstroke}{rgb}{0.121569,0.466667,0.705882}%
\pgfsetstrokecolor{currentstroke}%
\pgfsetdash{}{0pt}%
\pgfpathmoveto{\pgfqpoint{2.727225in}{2.381020in}}%
\pgfpathcurveto{\pgfqpoint{2.738275in}{2.381020in}}{\pgfqpoint{2.748874in}{2.385410in}}{\pgfqpoint{2.756688in}{2.393224in}}%
\pgfpathcurveto{\pgfqpoint{2.764502in}{2.401037in}}{\pgfqpoint{2.768892in}{2.411636in}}{\pgfqpoint{2.768892in}{2.422687in}}%
\pgfpathcurveto{\pgfqpoint{2.768892in}{2.433737in}}{\pgfqpoint{2.764502in}{2.444336in}}{\pgfqpoint{2.756688in}{2.452149in}}%
\pgfpathcurveto{\pgfqpoint{2.748874in}{2.459963in}}{\pgfqpoint{2.738275in}{2.464353in}}{\pgfqpoint{2.727225in}{2.464353in}}%
\pgfpathcurveto{\pgfqpoint{2.716175in}{2.464353in}}{\pgfqpoint{2.705576in}{2.459963in}}{\pgfqpoint{2.697762in}{2.452149in}}%
\pgfpathcurveto{\pgfqpoint{2.689949in}{2.444336in}}{\pgfqpoint{2.685559in}{2.433737in}}{\pgfqpoint{2.685559in}{2.422687in}}%
\pgfpathcurveto{\pgfqpoint{2.685559in}{2.411636in}}{\pgfqpoint{2.689949in}{2.401037in}}{\pgfqpoint{2.697762in}{2.393224in}}%
\pgfpathcurveto{\pgfqpoint{2.705576in}{2.385410in}}{\pgfqpoint{2.716175in}{2.381020in}}{\pgfqpoint{2.727225in}{2.381020in}}%
\pgfpathclose%
\pgfusepath{stroke,fill}%
\end{pgfscope}%
\begin{pgfscope}%
\pgfpathrectangle{\pgfqpoint{0.600000in}{0.600000in}}{\pgfqpoint{3.900000in}{3.900000in}}%
\pgfusepath{clip}%
\pgfsetbuttcap%
\pgfsetroundjoin%
\definecolor{currentfill}{rgb}{0.121569,0.466667,0.705882}%
\pgfsetfillcolor{currentfill}%
\pgfsetlinewidth{1.003750pt}%
\definecolor{currentstroke}{rgb}{0.121569,0.466667,0.705882}%
\pgfsetstrokecolor{currentstroke}%
\pgfsetdash{}{0pt}%
\pgfpathmoveto{\pgfqpoint{2.850614in}{3.042710in}}%
\pgfpathcurveto{\pgfqpoint{2.861665in}{3.042710in}}{\pgfqpoint{2.872264in}{3.047100in}}{\pgfqpoint{2.880077in}{3.054914in}}%
\pgfpathcurveto{\pgfqpoint{2.887891in}{3.062727in}}{\pgfqpoint{2.892281in}{3.073326in}}{\pgfqpoint{2.892281in}{3.084377in}}%
\pgfpathcurveto{\pgfqpoint{2.892281in}{3.095427in}}{\pgfqpoint{2.887891in}{3.106026in}}{\pgfqpoint{2.880077in}{3.113839in}}%
\pgfpathcurveto{\pgfqpoint{2.872264in}{3.121653in}}{\pgfqpoint{2.861665in}{3.126043in}}{\pgfqpoint{2.850614in}{3.126043in}}%
\pgfpathcurveto{\pgfqpoint{2.839564in}{3.126043in}}{\pgfqpoint{2.828965in}{3.121653in}}{\pgfqpoint{2.821152in}{3.113839in}}%
\pgfpathcurveto{\pgfqpoint{2.813338in}{3.106026in}}{\pgfqpoint{2.808948in}{3.095427in}}{\pgfqpoint{2.808948in}{3.084377in}}%
\pgfpathcurveto{\pgfqpoint{2.808948in}{3.073326in}}{\pgfqpoint{2.813338in}{3.062727in}}{\pgfqpoint{2.821152in}{3.054914in}}%
\pgfpathcurveto{\pgfqpoint{2.828965in}{3.047100in}}{\pgfqpoint{2.839564in}{3.042710in}}{\pgfqpoint{2.850614in}{3.042710in}}%
\pgfpathclose%
\pgfusepath{stroke,fill}%
\end{pgfscope}%
\begin{pgfscope}%
\pgfpathrectangle{\pgfqpoint{0.600000in}{0.600000in}}{\pgfqpoint{3.900000in}{3.900000in}}%
\pgfusepath{clip}%
\pgfsetbuttcap%
\pgfsetroundjoin%
\definecolor{currentfill}{rgb}{0.121569,0.466667,0.705882}%
\pgfsetfillcolor{currentfill}%
\pgfsetlinewidth{1.003750pt}%
\definecolor{currentstroke}{rgb}{0.121569,0.466667,0.705882}%
\pgfsetstrokecolor{currentstroke}%
\pgfsetdash{}{0pt}%
\pgfpathmoveto{\pgfqpoint{2.588791in}{2.720269in}}%
\pgfpathcurveto{\pgfqpoint{2.599841in}{2.720269in}}{\pgfqpoint{2.610440in}{2.724659in}}{\pgfqpoint{2.618253in}{2.732473in}}%
\pgfpathcurveto{\pgfqpoint{2.626067in}{2.740286in}}{\pgfqpoint{2.630457in}{2.750886in}}{\pgfqpoint{2.630457in}{2.761936in}}%
\pgfpathcurveto{\pgfqpoint{2.630457in}{2.772986in}}{\pgfqpoint{2.626067in}{2.783585in}}{\pgfqpoint{2.618253in}{2.791398in}}%
\pgfpathcurveto{\pgfqpoint{2.610440in}{2.799212in}}{\pgfqpoint{2.599841in}{2.803602in}}{\pgfqpoint{2.588791in}{2.803602in}}%
\pgfpathcurveto{\pgfqpoint{2.577740in}{2.803602in}}{\pgfqpoint{2.567141in}{2.799212in}}{\pgfqpoint{2.559328in}{2.791398in}}%
\pgfpathcurveto{\pgfqpoint{2.551514in}{2.783585in}}{\pgfqpoint{2.547124in}{2.772986in}}{\pgfqpoint{2.547124in}{2.761936in}}%
\pgfpathcurveto{\pgfqpoint{2.547124in}{2.750886in}}{\pgfqpoint{2.551514in}{2.740286in}}{\pgfqpoint{2.559328in}{2.732473in}}%
\pgfpathcurveto{\pgfqpoint{2.567141in}{2.724659in}}{\pgfqpoint{2.577740in}{2.720269in}}{\pgfqpoint{2.588791in}{2.720269in}}%
\pgfpathclose%
\pgfusepath{stroke,fill}%
\end{pgfscope}%
\begin{pgfscope}%
\pgfpathrectangle{\pgfqpoint{0.600000in}{0.600000in}}{\pgfqpoint{3.900000in}{3.900000in}}%
\pgfusepath{clip}%
\pgfsetbuttcap%
\pgfsetroundjoin%
\definecolor{currentfill}{rgb}{0.121569,0.466667,0.705882}%
\pgfsetfillcolor{currentfill}%
\pgfsetlinewidth{1.003750pt}%
\definecolor{currentstroke}{rgb}{0.121569,0.466667,0.705882}%
\pgfsetstrokecolor{currentstroke}%
\pgfsetdash{}{0pt}%
\pgfpathmoveto{\pgfqpoint{2.557788in}{3.192470in}}%
\pgfpathcurveto{\pgfqpoint{2.568838in}{3.192470in}}{\pgfqpoint{2.579437in}{3.196861in}}{\pgfqpoint{2.587251in}{3.204674in}}%
\pgfpathcurveto{\pgfqpoint{2.595064in}{3.212488in}}{\pgfqpoint{2.599454in}{3.223087in}}{\pgfqpoint{2.599454in}{3.234137in}}%
\pgfpathcurveto{\pgfqpoint{2.599454in}{3.245187in}}{\pgfqpoint{2.595064in}{3.255786in}}{\pgfqpoint{2.587251in}{3.263600in}}%
\pgfpathcurveto{\pgfqpoint{2.579437in}{3.271413in}}{\pgfqpoint{2.568838in}{3.275804in}}{\pgfqpoint{2.557788in}{3.275804in}}%
\pgfpathcurveto{\pgfqpoint{2.546738in}{3.275804in}}{\pgfqpoint{2.536139in}{3.271413in}}{\pgfqpoint{2.528325in}{3.263600in}}%
\pgfpathcurveto{\pgfqpoint{2.520511in}{3.255786in}}{\pgfqpoint{2.516121in}{3.245187in}}{\pgfqpoint{2.516121in}{3.234137in}}%
\pgfpathcurveto{\pgfqpoint{2.516121in}{3.223087in}}{\pgfqpoint{2.520511in}{3.212488in}}{\pgfqpoint{2.528325in}{3.204674in}}%
\pgfpathcurveto{\pgfqpoint{2.536139in}{3.196861in}}{\pgfqpoint{2.546738in}{3.192470in}}{\pgfqpoint{2.557788in}{3.192470in}}%
\pgfpathclose%
\pgfusepath{stroke,fill}%
\end{pgfscope}%
\begin{pgfscope}%
\pgfpathrectangle{\pgfqpoint{0.600000in}{0.600000in}}{\pgfqpoint{3.900000in}{3.900000in}}%
\pgfusepath{clip}%
\pgfsetbuttcap%
\pgfsetroundjoin%
\definecolor{currentfill}{rgb}{0.121569,0.466667,0.705882}%
\pgfsetfillcolor{currentfill}%
\pgfsetlinewidth{1.003750pt}%
\definecolor{currentstroke}{rgb}{0.121569,0.466667,0.705882}%
\pgfsetstrokecolor{currentstroke}%
\pgfsetdash{}{0pt}%
\pgfpathmoveto{\pgfqpoint{3.013149in}{3.314992in}}%
\pgfpathcurveto{\pgfqpoint{3.024199in}{3.314992in}}{\pgfqpoint{3.034798in}{3.319382in}}{\pgfqpoint{3.042612in}{3.327196in}}%
\pgfpathcurveto{\pgfqpoint{3.050425in}{3.335009in}}{\pgfqpoint{3.054815in}{3.345608in}}{\pgfqpoint{3.054815in}{3.356659in}}%
\pgfpathcurveto{\pgfqpoint{3.054815in}{3.367709in}}{\pgfqpoint{3.050425in}{3.378308in}}{\pgfqpoint{3.042612in}{3.386121in}}%
\pgfpathcurveto{\pgfqpoint{3.034798in}{3.393935in}}{\pgfqpoint{3.024199in}{3.398325in}}{\pgfqpoint{3.013149in}{3.398325in}}%
\pgfpathcurveto{\pgfqpoint{3.002099in}{3.398325in}}{\pgfqpoint{2.991500in}{3.393935in}}{\pgfqpoint{2.983686in}{3.386121in}}%
\pgfpathcurveto{\pgfqpoint{2.975872in}{3.378308in}}{\pgfqpoint{2.971482in}{3.367709in}}{\pgfqpoint{2.971482in}{3.356659in}}%
\pgfpathcurveto{\pgfqpoint{2.971482in}{3.345608in}}{\pgfqpoint{2.975872in}{3.335009in}}{\pgfqpoint{2.983686in}{3.327196in}}%
\pgfpathcurveto{\pgfqpoint{2.991500in}{3.319382in}}{\pgfqpoint{3.002099in}{3.314992in}}{\pgfqpoint{3.013149in}{3.314992in}}%
\pgfpathclose%
\pgfusepath{stroke,fill}%
\end{pgfscope}%
\begin{pgfscope}%
\pgfpathrectangle{\pgfqpoint{0.600000in}{0.600000in}}{\pgfqpoint{3.900000in}{3.900000in}}%
\pgfusepath{clip}%
\pgfsetbuttcap%
\pgfsetroundjoin%
\definecolor{currentfill}{rgb}{0.121569,0.466667,0.705882}%
\pgfsetfillcolor{currentfill}%
\pgfsetlinewidth{1.003750pt}%
\definecolor{currentstroke}{rgb}{0.121569,0.466667,0.705882}%
\pgfsetstrokecolor{currentstroke}%
\pgfsetdash{}{0pt}%
\pgfpathmoveto{\pgfqpoint{2.015669in}{2.716600in}}%
\pgfpathcurveto{\pgfqpoint{2.026719in}{2.716600in}}{\pgfqpoint{2.037318in}{2.720990in}}{\pgfqpoint{2.045132in}{2.728804in}}%
\pgfpathcurveto{\pgfqpoint{2.052946in}{2.736617in}}{\pgfqpoint{2.057336in}{2.747216in}}{\pgfqpoint{2.057336in}{2.758266in}}%
\pgfpathcurveto{\pgfqpoint{2.057336in}{2.769316in}}{\pgfqpoint{2.052946in}{2.779915in}}{\pgfqpoint{2.045132in}{2.787729in}}%
\pgfpathcurveto{\pgfqpoint{2.037318in}{2.795543in}}{\pgfqpoint{2.026719in}{2.799933in}}{\pgfqpoint{2.015669in}{2.799933in}}%
\pgfpathcurveto{\pgfqpoint{2.004619in}{2.799933in}}{\pgfqpoint{1.994020in}{2.795543in}}{\pgfqpoint{1.986207in}{2.787729in}}%
\pgfpathcurveto{\pgfqpoint{1.978393in}{2.779915in}}{\pgfqpoint{1.974003in}{2.769316in}}{\pgfqpoint{1.974003in}{2.758266in}}%
\pgfpathcurveto{\pgfqpoint{1.974003in}{2.747216in}}{\pgfqpoint{1.978393in}{2.736617in}}{\pgfqpoint{1.986207in}{2.728804in}}%
\pgfpathcurveto{\pgfqpoint{1.994020in}{2.720990in}}{\pgfqpoint{2.004619in}{2.716600in}}{\pgfqpoint{2.015669in}{2.716600in}}%
\pgfpathclose%
\pgfusepath{stroke,fill}%
\end{pgfscope}%
\begin{pgfscope}%
\pgfpathrectangle{\pgfqpoint{0.600000in}{0.600000in}}{\pgfqpoint{3.900000in}{3.900000in}}%
\pgfusepath{clip}%
\pgfsetbuttcap%
\pgfsetroundjoin%
\definecolor{currentfill}{rgb}{0.121569,0.466667,0.705882}%
\pgfsetfillcolor{currentfill}%
\pgfsetlinewidth{1.003750pt}%
\definecolor{currentstroke}{rgb}{0.121569,0.466667,0.705882}%
\pgfsetstrokecolor{currentstroke}%
\pgfsetdash{}{0pt}%
\pgfpathmoveto{\pgfqpoint{2.646890in}{2.476848in}}%
\pgfpathcurveto{\pgfqpoint{2.657940in}{2.476848in}}{\pgfqpoint{2.668539in}{2.481238in}}{\pgfqpoint{2.676352in}{2.489052in}}%
\pgfpathcurveto{\pgfqpoint{2.684166in}{2.496865in}}{\pgfqpoint{2.688556in}{2.507464in}}{\pgfqpoint{2.688556in}{2.518514in}}%
\pgfpathcurveto{\pgfqpoint{2.688556in}{2.529564in}}{\pgfqpoint{2.684166in}{2.540164in}}{\pgfqpoint{2.676352in}{2.547977in}}%
\pgfpathcurveto{\pgfqpoint{2.668539in}{2.555791in}}{\pgfqpoint{2.657940in}{2.560181in}}{\pgfqpoint{2.646890in}{2.560181in}}%
\pgfpathcurveto{\pgfqpoint{2.635839in}{2.560181in}}{\pgfqpoint{2.625240in}{2.555791in}}{\pgfqpoint{2.617427in}{2.547977in}}%
\pgfpathcurveto{\pgfqpoint{2.609613in}{2.540164in}}{\pgfqpoint{2.605223in}{2.529564in}}{\pgfqpoint{2.605223in}{2.518514in}}%
\pgfpathcurveto{\pgfqpoint{2.605223in}{2.507464in}}{\pgfqpoint{2.609613in}{2.496865in}}{\pgfqpoint{2.617427in}{2.489052in}}%
\pgfpathcurveto{\pgfqpoint{2.625240in}{2.481238in}}{\pgfqpoint{2.635839in}{2.476848in}}{\pgfqpoint{2.646890in}{2.476848in}}%
\pgfpathclose%
\pgfusepath{stroke,fill}%
\end{pgfscope}%
\begin{pgfscope}%
\pgfpathrectangle{\pgfqpoint{0.600000in}{0.600000in}}{\pgfqpoint{3.900000in}{3.900000in}}%
\pgfusepath{clip}%
\pgfsetbuttcap%
\pgfsetroundjoin%
\definecolor{currentfill}{rgb}{0.121569,0.466667,0.705882}%
\pgfsetfillcolor{currentfill}%
\pgfsetlinewidth{1.003750pt}%
\definecolor{currentstroke}{rgb}{0.121569,0.466667,0.705882}%
\pgfsetstrokecolor{currentstroke}%
\pgfsetdash{}{0pt}%
\pgfpathmoveto{\pgfqpoint{3.636662in}{2.738895in}}%
\pgfpathcurveto{\pgfqpoint{3.647712in}{2.738895in}}{\pgfqpoint{3.658311in}{2.743285in}}{\pgfqpoint{3.666125in}{2.751098in}}%
\pgfpathcurveto{\pgfqpoint{3.673938in}{2.758912in}}{\pgfqpoint{3.678329in}{2.769511in}}{\pgfqpoint{3.678329in}{2.780561in}}%
\pgfpathcurveto{\pgfqpoint{3.678329in}{2.791611in}}{\pgfqpoint{3.673938in}{2.802210in}}{\pgfqpoint{3.666125in}{2.810024in}}%
\pgfpathcurveto{\pgfqpoint{3.658311in}{2.817838in}}{\pgfqpoint{3.647712in}{2.822228in}}{\pgfqpoint{3.636662in}{2.822228in}}%
\pgfpathcurveto{\pgfqpoint{3.625612in}{2.822228in}}{\pgfqpoint{3.615013in}{2.817838in}}{\pgfqpoint{3.607199in}{2.810024in}}%
\pgfpathcurveto{\pgfqpoint{3.599386in}{2.802210in}}{\pgfqpoint{3.594995in}{2.791611in}}{\pgfqpoint{3.594995in}{2.780561in}}%
\pgfpathcurveto{\pgfqpoint{3.594995in}{2.769511in}}{\pgfqpoint{3.599386in}{2.758912in}}{\pgfqpoint{3.607199in}{2.751098in}}%
\pgfpathcurveto{\pgfqpoint{3.615013in}{2.743285in}}{\pgfqpoint{3.625612in}{2.738895in}}{\pgfqpoint{3.636662in}{2.738895in}}%
\pgfpathclose%
\pgfusepath{stroke,fill}%
\end{pgfscope}%
\begin{pgfscope}%
\pgfpathrectangle{\pgfqpoint{0.600000in}{0.600000in}}{\pgfqpoint{3.900000in}{3.900000in}}%
\pgfusepath{clip}%
\pgfsetbuttcap%
\pgfsetroundjoin%
\definecolor{currentfill}{rgb}{0.121569,0.466667,0.705882}%
\pgfsetfillcolor{currentfill}%
\pgfsetlinewidth{1.003750pt}%
\definecolor{currentstroke}{rgb}{0.121569,0.466667,0.705882}%
\pgfsetstrokecolor{currentstroke}%
\pgfsetdash{}{0pt}%
\pgfpathmoveto{\pgfqpoint{2.799360in}{2.393757in}}%
\pgfpathcurveto{\pgfqpoint{2.810410in}{2.393757in}}{\pgfqpoint{2.821009in}{2.398147in}}{\pgfqpoint{2.828823in}{2.405961in}}%
\pgfpathcurveto{\pgfqpoint{2.836636in}{2.413774in}}{\pgfqpoint{2.841027in}{2.424373in}}{\pgfqpoint{2.841027in}{2.435423in}}%
\pgfpathcurveto{\pgfqpoint{2.841027in}{2.446474in}}{\pgfqpoint{2.836636in}{2.457073in}}{\pgfqpoint{2.828823in}{2.464886in}}%
\pgfpathcurveto{\pgfqpoint{2.821009in}{2.472700in}}{\pgfqpoint{2.810410in}{2.477090in}}{\pgfqpoint{2.799360in}{2.477090in}}%
\pgfpathcurveto{\pgfqpoint{2.788310in}{2.477090in}}{\pgfqpoint{2.777711in}{2.472700in}}{\pgfqpoint{2.769897in}{2.464886in}}%
\pgfpathcurveto{\pgfqpoint{2.762084in}{2.457073in}}{\pgfqpoint{2.757693in}{2.446474in}}{\pgfqpoint{2.757693in}{2.435423in}}%
\pgfpathcurveto{\pgfqpoint{2.757693in}{2.424373in}}{\pgfqpoint{2.762084in}{2.413774in}}{\pgfqpoint{2.769897in}{2.405961in}}%
\pgfpathcurveto{\pgfqpoint{2.777711in}{2.398147in}}{\pgfqpoint{2.788310in}{2.393757in}}{\pgfqpoint{2.799360in}{2.393757in}}%
\pgfpathclose%
\pgfusepath{stroke,fill}%
\end{pgfscope}%
\begin{pgfscope}%
\pgfpathrectangle{\pgfqpoint{0.600000in}{0.600000in}}{\pgfqpoint{3.900000in}{3.900000in}}%
\pgfusepath{clip}%
\pgfsetbuttcap%
\pgfsetroundjoin%
\definecolor{currentfill}{rgb}{0.121569,0.466667,0.705882}%
\pgfsetfillcolor{currentfill}%
\pgfsetlinewidth{1.003750pt}%
\definecolor{currentstroke}{rgb}{0.121569,0.466667,0.705882}%
\pgfsetstrokecolor{currentstroke}%
\pgfsetdash{}{0pt}%
\pgfpathmoveto{\pgfqpoint{2.983704in}{1.412858in}}%
\pgfpathcurveto{\pgfqpoint{2.994754in}{1.412858in}}{\pgfqpoint{3.005353in}{1.417249in}}{\pgfqpoint{3.013166in}{1.425062in}}%
\pgfpathcurveto{\pgfqpoint{3.020980in}{1.432876in}}{\pgfqpoint{3.025370in}{1.443475in}}{\pgfqpoint{3.025370in}{1.454525in}}%
\pgfpathcurveto{\pgfqpoint{3.025370in}{1.465575in}}{\pgfqpoint{3.020980in}{1.476174in}}{\pgfqpoint{3.013166in}{1.483988in}}%
\pgfpathcurveto{\pgfqpoint{3.005353in}{1.491801in}}{\pgfqpoint{2.994754in}{1.496192in}}{\pgfqpoint{2.983704in}{1.496192in}}%
\pgfpathcurveto{\pgfqpoint{2.972653in}{1.496192in}}{\pgfqpoint{2.962054in}{1.491801in}}{\pgfqpoint{2.954241in}{1.483988in}}%
\pgfpathcurveto{\pgfqpoint{2.946427in}{1.476174in}}{\pgfqpoint{2.942037in}{1.465575in}}{\pgfqpoint{2.942037in}{1.454525in}}%
\pgfpathcurveto{\pgfqpoint{2.942037in}{1.443475in}}{\pgfqpoint{2.946427in}{1.432876in}}{\pgfqpoint{2.954241in}{1.425062in}}%
\pgfpathcurveto{\pgfqpoint{2.962054in}{1.417249in}}{\pgfqpoint{2.972653in}{1.412858in}}{\pgfqpoint{2.983704in}{1.412858in}}%
\pgfpathclose%
\pgfusepath{stroke,fill}%
\end{pgfscope}%
\begin{pgfscope}%
\pgfpathrectangle{\pgfqpoint{0.600000in}{0.600000in}}{\pgfqpoint{3.900000in}{3.900000in}}%
\pgfusepath{clip}%
\pgfsetbuttcap%
\pgfsetroundjoin%
\definecolor{currentfill}{rgb}{0.121569,0.466667,0.705882}%
\pgfsetfillcolor{currentfill}%
\pgfsetlinewidth{1.003750pt}%
\definecolor{currentstroke}{rgb}{0.121569,0.466667,0.705882}%
\pgfsetstrokecolor{currentstroke}%
\pgfsetdash{}{0pt}%
\pgfpathmoveto{\pgfqpoint{2.085265in}{2.418631in}}%
\pgfpathcurveto{\pgfqpoint{2.096315in}{2.418631in}}{\pgfqpoint{2.106915in}{2.423022in}}{\pgfqpoint{2.114728in}{2.430835in}}%
\pgfpathcurveto{\pgfqpoint{2.122542in}{2.438649in}}{\pgfqpoint{2.126932in}{2.449248in}}{\pgfqpoint{2.126932in}{2.460298in}}%
\pgfpathcurveto{\pgfqpoint{2.126932in}{2.471348in}}{\pgfqpoint{2.122542in}{2.481947in}}{\pgfqpoint{2.114728in}{2.489761in}}%
\pgfpathcurveto{\pgfqpoint{2.106915in}{2.497574in}}{\pgfqpoint{2.096315in}{2.501965in}}{\pgfqpoint{2.085265in}{2.501965in}}%
\pgfpathcurveto{\pgfqpoint{2.074215in}{2.501965in}}{\pgfqpoint{2.063616in}{2.497574in}}{\pgfqpoint{2.055803in}{2.489761in}}%
\pgfpathcurveto{\pgfqpoint{2.047989in}{2.481947in}}{\pgfqpoint{2.043599in}{2.471348in}}{\pgfqpoint{2.043599in}{2.460298in}}%
\pgfpathcurveto{\pgfqpoint{2.043599in}{2.449248in}}{\pgfqpoint{2.047989in}{2.438649in}}{\pgfqpoint{2.055803in}{2.430835in}}%
\pgfpathcurveto{\pgfqpoint{2.063616in}{2.423022in}}{\pgfqpoint{2.074215in}{2.418631in}}{\pgfqpoint{2.085265in}{2.418631in}}%
\pgfpathclose%
\pgfusepath{stroke,fill}%
\end{pgfscope}%
\begin{pgfscope}%
\pgfpathrectangle{\pgfqpoint{0.600000in}{0.600000in}}{\pgfqpoint{3.900000in}{3.900000in}}%
\pgfusepath{clip}%
\pgfsetbuttcap%
\pgfsetroundjoin%
\definecolor{currentfill}{rgb}{0.121569,0.466667,0.705882}%
\pgfsetfillcolor{currentfill}%
\pgfsetlinewidth{1.003750pt}%
\definecolor{currentstroke}{rgb}{0.121569,0.466667,0.705882}%
\pgfsetstrokecolor{currentstroke}%
\pgfsetdash{}{0pt}%
\pgfpathmoveto{\pgfqpoint{2.552462in}{3.580362in}}%
\pgfpathcurveto{\pgfqpoint{2.563512in}{3.580362in}}{\pgfqpoint{2.574111in}{3.584752in}}{\pgfqpoint{2.581925in}{3.592566in}}%
\pgfpathcurveto{\pgfqpoint{2.589738in}{3.600379in}}{\pgfqpoint{2.594129in}{3.610978in}}{\pgfqpoint{2.594129in}{3.622029in}}%
\pgfpathcurveto{\pgfqpoint{2.594129in}{3.633079in}}{\pgfqpoint{2.589738in}{3.643678in}}{\pgfqpoint{2.581925in}{3.651491in}}%
\pgfpathcurveto{\pgfqpoint{2.574111in}{3.659305in}}{\pgfqpoint{2.563512in}{3.663695in}}{\pgfqpoint{2.552462in}{3.663695in}}%
\pgfpathcurveto{\pgfqpoint{2.541412in}{3.663695in}}{\pgfqpoint{2.530813in}{3.659305in}}{\pgfqpoint{2.522999in}{3.651491in}}%
\pgfpathcurveto{\pgfqpoint{2.515185in}{3.643678in}}{\pgfqpoint{2.510795in}{3.633079in}}{\pgfqpoint{2.510795in}{3.622029in}}%
\pgfpathcurveto{\pgfqpoint{2.510795in}{3.610978in}}{\pgfqpoint{2.515185in}{3.600379in}}{\pgfqpoint{2.522999in}{3.592566in}}%
\pgfpathcurveto{\pgfqpoint{2.530813in}{3.584752in}}{\pgfqpoint{2.541412in}{3.580362in}}{\pgfqpoint{2.552462in}{3.580362in}}%
\pgfpathclose%
\pgfusepath{stroke,fill}%
\end{pgfscope}%
\begin{pgfscope}%
\pgfpathrectangle{\pgfqpoint{0.600000in}{0.600000in}}{\pgfqpoint{3.900000in}{3.900000in}}%
\pgfusepath{clip}%
\pgfsetbuttcap%
\pgfsetroundjoin%
\definecolor{currentfill}{rgb}{0.121569,0.466667,0.705882}%
\pgfsetfillcolor{currentfill}%
\pgfsetlinewidth{1.003750pt}%
\definecolor{currentstroke}{rgb}{0.121569,0.466667,0.705882}%
\pgfsetstrokecolor{currentstroke}%
\pgfsetdash{}{0pt}%
\pgfpathmoveto{\pgfqpoint{3.302047in}{2.079070in}}%
\pgfpathcurveto{\pgfqpoint{3.313097in}{2.079070in}}{\pgfqpoint{3.323696in}{2.083461in}}{\pgfqpoint{3.331510in}{2.091274in}}%
\pgfpathcurveto{\pgfqpoint{3.339324in}{2.099088in}}{\pgfqpoint{3.343714in}{2.109687in}}{\pgfqpoint{3.343714in}{2.120737in}}%
\pgfpathcurveto{\pgfqpoint{3.343714in}{2.131787in}}{\pgfqpoint{3.339324in}{2.142386in}}{\pgfqpoint{3.331510in}{2.150200in}}%
\pgfpathcurveto{\pgfqpoint{3.323696in}{2.158013in}}{\pgfqpoint{3.313097in}{2.162404in}}{\pgfqpoint{3.302047in}{2.162404in}}%
\pgfpathcurveto{\pgfqpoint{3.290997in}{2.162404in}}{\pgfqpoint{3.280398in}{2.158013in}}{\pgfqpoint{3.272584in}{2.150200in}}%
\pgfpathcurveto{\pgfqpoint{3.264771in}{2.142386in}}{\pgfqpoint{3.260381in}{2.131787in}}{\pgfqpoint{3.260381in}{2.120737in}}%
\pgfpathcurveto{\pgfqpoint{3.260381in}{2.109687in}}{\pgfqpoint{3.264771in}{2.099088in}}{\pgfqpoint{3.272584in}{2.091274in}}%
\pgfpathcurveto{\pgfqpoint{3.280398in}{2.083461in}}{\pgfqpoint{3.290997in}{2.079070in}}{\pgfqpoint{3.302047in}{2.079070in}}%
\pgfpathclose%
\pgfusepath{stroke,fill}%
\end{pgfscope}%
\begin{pgfscope}%
\pgfpathrectangle{\pgfqpoint{0.600000in}{0.600000in}}{\pgfqpoint{3.900000in}{3.900000in}}%
\pgfusepath{clip}%
\pgfsetbuttcap%
\pgfsetroundjoin%
\definecolor{currentfill}{rgb}{0.121569,0.466667,0.705882}%
\pgfsetfillcolor{currentfill}%
\pgfsetlinewidth{1.003750pt}%
\definecolor{currentstroke}{rgb}{0.121569,0.466667,0.705882}%
\pgfsetstrokecolor{currentstroke}%
\pgfsetdash{}{0pt}%
\pgfpathmoveto{\pgfqpoint{2.909663in}{2.686164in}}%
\pgfpathcurveto{\pgfqpoint{2.920713in}{2.686164in}}{\pgfqpoint{2.931312in}{2.690554in}}{\pgfqpoint{2.939126in}{2.698368in}}%
\pgfpathcurveto{\pgfqpoint{2.946939in}{2.706181in}}{\pgfqpoint{2.951330in}{2.716780in}}{\pgfqpoint{2.951330in}{2.727831in}}%
\pgfpathcurveto{\pgfqpoint{2.951330in}{2.738881in}}{\pgfqpoint{2.946939in}{2.749480in}}{\pgfqpoint{2.939126in}{2.757293in}}%
\pgfpathcurveto{\pgfqpoint{2.931312in}{2.765107in}}{\pgfqpoint{2.920713in}{2.769497in}}{\pgfqpoint{2.909663in}{2.769497in}}%
\pgfpathcurveto{\pgfqpoint{2.898613in}{2.769497in}}{\pgfqpoint{2.888014in}{2.765107in}}{\pgfqpoint{2.880200in}{2.757293in}}%
\pgfpathcurveto{\pgfqpoint{2.872387in}{2.749480in}}{\pgfqpoint{2.867996in}{2.738881in}}{\pgfqpoint{2.867996in}{2.727831in}}%
\pgfpathcurveto{\pgfqpoint{2.867996in}{2.716780in}}{\pgfqpoint{2.872387in}{2.706181in}}{\pgfqpoint{2.880200in}{2.698368in}}%
\pgfpathcurveto{\pgfqpoint{2.888014in}{2.690554in}}{\pgfqpoint{2.898613in}{2.686164in}}{\pgfqpoint{2.909663in}{2.686164in}}%
\pgfpathclose%
\pgfusepath{stroke,fill}%
\end{pgfscope}%
\begin{pgfscope}%
\pgfpathrectangle{\pgfqpoint{0.600000in}{0.600000in}}{\pgfqpoint{3.900000in}{3.900000in}}%
\pgfusepath{clip}%
\pgfsetbuttcap%
\pgfsetroundjoin%
\definecolor{currentfill}{rgb}{0.121569,0.466667,0.705882}%
\pgfsetfillcolor{currentfill}%
\pgfsetlinewidth{1.003750pt}%
\definecolor{currentstroke}{rgb}{0.121569,0.466667,0.705882}%
\pgfsetstrokecolor{currentstroke}%
\pgfsetdash{}{0pt}%
\pgfpathmoveto{\pgfqpoint{2.985586in}{2.268794in}}%
\pgfpathcurveto{\pgfqpoint{2.996636in}{2.268794in}}{\pgfqpoint{3.007235in}{2.273184in}}{\pgfqpoint{3.015049in}{2.280998in}}%
\pgfpathcurveto{\pgfqpoint{3.022862in}{2.288811in}}{\pgfqpoint{3.027253in}{2.299410in}}{\pgfqpoint{3.027253in}{2.310460in}}%
\pgfpathcurveto{\pgfqpoint{3.027253in}{2.321511in}}{\pgfqpoint{3.022862in}{2.332110in}}{\pgfqpoint{3.015049in}{2.339923in}}%
\pgfpathcurveto{\pgfqpoint{3.007235in}{2.347737in}}{\pgfqpoint{2.996636in}{2.352127in}}{\pgfqpoint{2.985586in}{2.352127in}}%
\pgfpathcurveto{\pgfqpoint{2.974536in}{2.352127in}}{\pgfqpoint{2.963937in}{2.347737in}}{\pgfqpoint{2.956123in}{2.339923in}}%
\pgfpathcurveto{\pgfqpoint{2.948310in}{2.332110in}}{\pgfqpoint{2.943919in}{2.321511in}}{\pgfqpoint{2.943919in}{2.310460in}}%
\pgfpathcurveto{\pgfqpoint{2.943919in}{2.299410in}}{\pgfqpoint{2.948310in}{2.288811in}}{\pgfqpoint{2.956123in}{2.280998in}}%
\pgfpathcurveto{\pgfqpoint{2.963937in}{2.273184in}}{\pgfqpoint{2.974536in}{2.268794in}}{\pgfqpoint{2.985586in}{2.268794in}}%
\pgfpathclose%
\pgfusepath{stroke,fill}%
\end{pgfscope}%
\begin{pgfscope}%
\pgfpathrectangle{\pgfqpoint{0.600000in}{0.600000in}}{\pgfqpoint{3.900000in}{3.900000in}}%
\pgfusepath{clip}%
\pgfsetbuttcap%
\pgfsetroundjoin%
\definecolor{currentfill}{rgb}{0.121569,0.466667,0.705882}%
\pgfsetfillcolor{currentfill}%
\pgfsetlinewidth{1.003750pt}%
\definecolor{currentstroke}{rgb}{0.121569,0.466667,0.705882}%
\pgfsetstrokecolor{currentstroke}%
\pgfsetdash{}{0pt}%
\pgfpathmoveto{\pgfqpoint{1.167140in}{2.045700in}}%
\pgfpathcurveto{\pgfqpoint{1.178190in}{2.045700in}}{\pgfqpoint{1.188789in}{2.050091in}}{\pgfqpoint{1.196603in}{2.057904in}}%
\pgfpathcurveto{\pgfqpoint{1.204417in}{2.065718in}}{\pgfqpoint{1.208807in}{2.076317in}}{\pgfqpoint{1.208807in}{2.087367in}}%
\pgfpathcurveto{\pgfqpoint{1.208807in}{2.098417in}}{\pgfqpoint{1.204417in}{2.109016in}}{\pgfqpoint{1.196603in}{2.116830in}}%
\pgfpathcurveto{\pgfqpoint{1.188789in}{2.124643in}}{\pgfqpoint{1.178190in}{2.129034in}}{\pgfqpoint{1.167140in}{2.129034in}}%
\pgfpathcurveto{\pgfqpoint{1.156090in}{2.129034in}}{\pgfqpoint{1.145491in}{2.124643in}}{\pgfqpoint{1.137678in}{2.116830in}}%
\pgfpathcurveto{\pgfqpoint{1.129864in}{2.109016in}}{\pgfqpoint{1.125474in}{2.098417in}}{\pgfqpoint{1.125474in}{2.087367in}}%
\pgfpathcurveto{\pgfqpoint{1.125474in}{2.076317in}}{\pgfqpoint{1.129864in}{2.065718in}}{\pgfqpoint{1.137678in}{2.057904in}}%
\pgfpathcurveto{\pgfqpoint{1.145491in}{2.050091in}}{\pgfqpoint{1.156090in}{2.045700in}}{\pgfqpoint{1.167140in}{2.045700in}}%
\pgfpathclose%
\pgfusepath{stroke,fill}%
\end{pgfscope}%
\begin{pgfscope}%
\pgfpathrectangle{\pgfqpoint{0.600000in}{0.600000in}}{\pgfqpoint{3.900000in}{3.900000in}}%
\pgfusepath{clip}%
\pgfsetbuttcap%
\pgfsetroundjoin%
\definecolor{currentfill}{rgb}{0.121569,0.466667,0.705882}%
\pgfsetfillcolor{currentfill}%
\pgfsetlinewidth{1.003750pt}%
\definecolor{currentstroke}{rgb}{0.121569,0.466667,0.705882}%
\pgfsetstrokecolor{currentstroke}%
\pgfsetdash{}{0pt}%
\pgfpathmoveto{\pgfqpoint{3.004651in}{2.606402in}}%
\pgfpathcurveto{\pgfqpoint{3.015701in}{2.606402in}}{\pgfqpoint{3.026300in}{2.610792in}}{\pgfqpoint{3.034113in}{2.618606in}}%
\pgfpathcurveto{\pgfqpoint{3.041927in}{2.626419in}}{\pgfqpoint{3.046317in}{2.637018in}}{\pgfqpoint{3.046317in}{2.648069in}}%
\pgfpathcurveto{\pgfqpoint{3.046317in}{2.659119in}}{\pgfqpoint{3.041927in}{2.669718in}}{\pgfqpoint{3.034113in}{2.677531in}}%
\pgfpathcurveto{\pgfqpoint{3.026300in}{2.685345in}}{\pgfqpoint{3.015701in}{2.689735in}}{\pgfqpoint{3.004651in}{2.689735in}}%
\pgfpathcurveto{\pgfqpoint{2.993601in}{2.689735in}}{\pgfqpoint{2.983002in}{2.685345in}}{\pgfqpoint{2.975188in}{2.677531in}}%
\pgfpathcurveto{\pgfqpoint{2.967374in}{2.669718in}}{\pgfqpoint{2.962984in}{2.659119in}}{\pgfqpoint{2.962984in}{2.648069in}}%
\pgfpathcurveto{\pgfqpoint{2.962984in}{2.637018in}}{\pgfqpoint{2.967374in}{2.626419in}}{\pgfqpoint{2.975188in}{2.618606in}}%
\pgfpathcurveto{\pgfqpoint{2.983002in}{2.610792in}}{\pgfqpoint{2.993601in}{2.606402in}}{\pgfqpoint{3.004651in}{2.606402in}}%
\pgfpathclose%
\pgfusepath{stroke,fill}%
\end{pgfscope}%
\begin{pgfscope}%
\pgfpathrectangle{\pgfqpoint{0.600000in}{0.600000in}}{\pgfqpoint{3.900000in}{3.900000in}}%
\pgfusepath{clip}%
\pgfsetbuttcap%
\pgfsetroundjoin%
\definecolor{currentfill}{rgb}{0.121569,0.466667,0.705882}%
\pgfsetfillcolor{currentfill}%
\pgfsetlinewidth{1.003750pt}%
\definecolor{currentstroke}{rgb}{0.121569,0.466667,0.705882}%
\pgfsetstrokecolor{currentstroke}%
\pgfsetdash{}{0pt}%
\pgfpathmoveto{\pgfqpoint{2.602747in}{2.079977in}}%
\pgfpathcurveto{\pgfqpoint{2.613797in}{2.079977in}}{\pgfqpoint{2.624396in}{2.084367in}}{\pgfqpoint{2.632210in}{2.092181in}}%
\pgfpathcurveto{\pgfqpoint{2.640024in}{2.099994in}}{\pgfqpoint{2.644414in}{2.110593in}}{\pgfqpoint{2.644414in}{2.121643in}}%
\pgfpathcurveto{\pgfqpoint{2.644414in}{2.132693in}}{\pgfqpoint{2.640024in}{2.143293in}}{\pgfqpoint{2.632210in}{2.151106in}}%
\pgfpathcurveto{\pgfqpoint{2.624396in}{2.158920in}}{\pgfqpoint{2.613797in}{2.163310in}}{\pgfqpoint{2.602747in}{2.163310in}}%
\pgfpathcurveto{\pgfqpoint{2.591697in}{2.163310in}}{\pgfqpoint{2.581098in}{2.158920in}}{\pgfqpoint{2.573285in}{2.151106in}}%
\pgfpathcurveto{\pgfqpoint{2.565471in}{2.143293in}}{\pgfqpoint{2.561081in}{2.132693in}}{\pgfqpoint{2.561081in}{2.121643in}}%
\pgfpathcurveto{\pgfqpoint{2.561081in}{2.110593in}}{\pgfqpoint{2.565471in}{2.099994in}}{\pgfqpoint{2.573285in}{2.092181in}}%
\pgfpathcurveto{\pgfqpoint{2.581098in}{2.084367in}}{\pgfqpoint{2.591697in}{2.079977in}}{\pgfqpoint{2.602747in}{2.079977in}}%
\pgfpathclose%
\pgfusepath{stroke,fill}%
\end{pgfscope}%
\begin{pgfscope}%
\pgfpathrectangle{\pgfqpoint{0.600000in}{0.600000in}}{\pgfqpoint{3.900000in}{3.900000in}}%
\pgfusepath{clip}%
\pgfsetbuttcap%
\pgfsetroundjoin%
\definecolor{currentfill}{rgb}{0.121569,0.466667,0.705882}%
\pgfsetfillcolor{currentfill}%
\pgfsetlinewidth{1.003750pt}%
\definecolor{currentstroke}{rgb}{0.121569,0.466667,0.705882}%
\pgfsetstrokecolor{currentstroke}%
\pgfsetdash{}{0pt}%
\pgfpathmoveto{\pgfqpoint{2.124123in}{2.238426in}}%
\pgfpathcurveto{\pgfqpoint{2.135173in}{2.238426in}}{\pgfqpoint{2.145773in}{2.242817in}}{\pgfqpoint{2.153586in}{2.250630in}}%
\pgfpathcurveto{\pgfqpoint{2.161400in}{2.258444in}}{\pgfqpoint{2.165790in}{2.269043in}}{\pgfqpoint{2.165790in}{2.280093in}}%
\pgfpathcurveto{\pgfqpoint{2.165790in}{2.291143in}}{\pgfqpoint{2.161400in}{2.301742in}}{\pgfqpoint{2.153586in}{2.309556in}}%
\pgfpathcurveto{\pgfqpoint{2.145773in}{2.317369in}}{\pgfqpoint{2.135173in}{2.321760in}}{\pgfqpoint{2.124123in}{2.321760in}}%
\pgfpathcurveto{\pgfqpoint{2.113073in}{2.321760in}}{\pgfqpoint{2.102474in}{2.317369in}}{\pgfqpoint{2.094661in}{2.309556in}}%
\pgfpathcurveto{\pgfqpoint{2.086847in}{2.301742in}}{\pgfqpoint{2.082457in}{2.291143in}}{\pgfqpoint{2.082457in}{2.280093in}}%
\pgfpathcurveto{\pgfqpoint{2.082457in}{2.269043in}}{\pgfqpoint{2.086847in}{2.258444in}}{\pgfqpoint{2.094661in}{2.250630in}}%
\pgfpathcurveto{\pgfqpoint{2.102474in}{2.242817in}}{\pgfqpoint{2.113073in}{2.238426in}}{\pgfqpoint{2.124123in}{2.238426in}}%
\pgfpathclose%
\pgfusepath{stroke,fill}%
\end{pgfscope}%
\begin{pgfscope}%
\pgfpathrectangle{\pgfqpoint{0.600000in}{0.600000in}}{\pgfqpoint{3.900000in}{3.900000in}}%
\pgfusepath{clip}%
\pgfsetbuttcap%
\pgfsetroundjoin%
\definecolor{currentfill}{rgb}{0.121569,0.466667,0.705882}%
\pgfsetfillcolor{currentfill}%
\pgfsetlinewidth{1.003750pt}%
\definecolor{currentstroke}{rgb}{0.121569,0.466667,0.705882}%
\pgfsetstrokecolor{currentstroke}%
\pgfsetdash{}{0pt}%
\pgfpathmoveto{\pgfqpoint{3.035640in}{1.273137in}}%
\pgfpathcurveto{\pgfqpoint{3.046691in}{1.273137in}}{\pgfqpoint{3.057290in}{1.277528in}}{\pgfqpoint{3.065103in}{1.285341in}}%
\pgfpathcurveto{\pgfqpoint{3.072917in}{1.293155in}}{\pgfqpoint{3.077307in}{1.303754in}}{\pgfqpoint{3.077307in}{1.314804in}}%
\pgfpathcurveto{\pgfqpoint{3.077307in}{1.325854in}}{\pgfqpoint{3.072917in}{1.336453in}}{\pgfqpoint{3.065103in}{1.344267in}}%
\pgfpathcurveto{\pgfqpoint{3.057290in}{1.352080in}}{\pgfqpoint{3.046691in}{1.356471in}}{\pgfqpoint{3.035640in}{1.356471in}}%
\pgfpathcurveto{\pgfqpoint{3.024590in}{1.356471in}}{\pgfqpoint{3.013991in}{1.352080in}}{\pgfqpoint{3.006178in}{1.344267in}}%
\pgfpathcurveto{\pgfqpoint{2.998364in}{1.336453in}}{\pgfqpoint{2.993974in}{1.325854in}}{\pgfqpoint{2.993974in}{1.314804in}}%
\pgfpathcurveto{\pgfqpoint{2.993974in}{1.303754in}}{\pgfqpoint{2.998364in}{1.293155in}}{\pgfqpoint{3.006178in}{1.285341in}}%
\pgfpathcurveto{\pgfqpoint{3.013991in}{1.277528in}}{\pgfqpoint{3.024590in}{1.273137in}}{\pgfqpoint{3.035640in}{1.273137in}}%
\pgfpathclose%
\pgfusepath{stroke,fill}%
\end{pgfscope}%
\begin{pgfscope}%
\pgfpathrectangle{\pgfqpoint{0.600000in}{0.600000in}}{\pgfqpoint{3.900000in}{3.900000in}}%
\pgfusepath{clip}%
\pgfsetbuttcap%
\pgfsetroundjoin%
\definecolor{currentfill}{rgb}{0.121569,0.466667,0.705882}%
\pgfsetfillcolor{currentfill}%
\pgfsetlinewidth{1.003750pt}%
\definecolor{currentstroke}{rgb}{0.121569,0.466667,0.705882}%
\pgfsetstrokecolor{currentstroke}%
\pgfsetdash{}{0pt}%
\pgfpathmoveto{\pgfqpoint{3.567502in}{2.557401in}}%
\pgfpathcurveto{\pgfqpoint{3.578552in}{2.557401in}}{\pgfqpoint{3.589151in}{2.561791in}}{\pgfqpoint{3.596965in}{2.569605in}}%
\pgfpathcurveto{\pgfqpoint{3.604778in}{2.577419in}}{\pgfqpoint{3.609168in}{2.588018in}}{\pgfqpoint{3.609168in}{2.599068in}}%
\pgfpathcurveto{\pgfqpoint{3.609168in}{2.610118in}}{\pgfqpoint{3.604778in}{2.620717in}}{\pgfqpoint{3.596965in}{2.628531in}}%
\pgfpathcurveto{\pgfqpoint{3.589151in}{2.636344in}}{\pgfqpoint{3.578552in}{2.640734in}}{\pgfqpoint{3.567502in}{2.640734in}}%
\pgfpathcurveto{\pgfqpoint{3.556452in}{2.640734in}}{\pgfqpoint{3.545853in}{2.636344in}}{\pgfqpoint{3.538039in}{2.628531in}}%
\pgfpathcurveto{\pgfqpoint{3.530225in}{2.620717in}}{\pgfqpoint{3.525835in}{2.610118in}}{\pgfqpoint{3.525835in}{2.599068in}}%
\pgfpathcurveto{\pgfqpoint{3.525835in}{2.588018in}}{\pgfqpoint{3.530225in}{2.577419in}}{\pgfqpoint{3.538039in}{2.569605in}}%
\pgfpathcurveto{\pgfqpoint{3.545853in}{2.561791in}}{\pgfqpoint{3.556452in}{2.557401in}}{\pgfqpoint{3.567502in}{2.557401in}}%
\pgfpathclose%
\pgfusepath{stroke,fill}%
\end{pgfscope}%
\begin{pgfscope}%
\pgfpathrectangle{\pgfqpoint{0.600000in}{0.600000in}}{\pgfqpoint{3.900000in}{3.900000in}}%
\pgfusepath{clip}%
\pgfsetbuttcap%
\pgfsetroundjoin%
\definecolor{currentfill}{rgb}{0.121569,0.466667,0.705882}%
\pgfsetfillcolor{currentfill}%
\pgfsetlinewidth{1.003750pt}%
\definecolor{currentstroke}{rgb}{0.121569,0.466667,0.705882}%
\pgfsetstrokecolor{currentstroke}%
\pgfsetdash{}{0pt}%
\pgfpathmoveto{\pgfqpoint{2.756722in}{3.029864in}}%
\pgfpathcurveto{\pgfqpoint{2.767772in}{3.029864in}}{\pgfqpoint{2.778371in}{3.034254in}}{\pgfqpoint{2.786184in}{3.042068in}}%
\pgfpathcurveto{\pgfqpoint{2.793998in}{3.049881in}}{\pgfqpoint{2.798388in}{3.060480in}}{\pgfqpoint{2.798388in}{3.071531in}}%
\pgfpathcurveto{\pgfqpoint{2.798388in}{3.082581in}}{\pgfqpoint{2.793998in}{3.093180in}}{\pgfqpoint{2.786184in}{3.100993in}}%
\pgfpathcurveto{\pgfqpoint{2.778371in}{3.108807in}}{\pgfqpoint{2.767772in}{3.113197in}}{\pgfqpoint{2.756722in}{3.113197in}}%
\pgfpathcurveto{\pgfqpoint{2.745671in}{3.113197in}}{\pgfqpoint{2.735072in}{3.108807in}}{\pgfqpoint{2.727259in}{3.100993in}}%
\pgfpathcurveto{\pgfqpoint{2.719445in}{3.093180in}}{\pgfqpoint{2.715055in}{3.082581in}}{\pgfqpoint{2.715055in}{3.071531in}}%
\pgfpathcurveto{\pgfqpoint{2.715055in}{3.060480in}}{\pgfqpoint{2.719445in}{3.049881in}}{\pgfqpoint{2.727259in}{3.042068in}}%
\pgfpathcurveto{\pgfqpoint{2.735072in}{3.034254in}}{\pgfqpoint{2.745671in}{3.029864in}}{\pgfqpoint{2.756722in}{3.029864in}}%
\pgfpathclose%
\pgfusepath{stroke,fill}%
\end{pgfscope}%
\begin{pgfscope}%
\pgfpathrectangle{\pgfqpoint{0.600000in}{0.600000in}}{\pgfqpoint{3.900000in}{3.900000in}}%
\pgfusepath{clip}%
\pgfsetbuttcap%
\pgfsetroundjoin%
\definecolor{currentfill}{rgb}{0.121569,0.466667,0.705882}%
\pgfsetfillcolor{currentfill}%
\pgfsetlinewidth{1.003750pt}%
\definecolor{currentstroke}{rgb}{0.121569,0.466667,0.705882}%
\pgfsetstrokecolor{currentstroke}%
\pgfsetdash{}{0pt}%
\pgfpathmoveto{\pgfqpoint{2.234798in}{3.098021in}}%
\pgfpathcurveto{\pgfqpoint{2.245848in}{3.098021in}}{\pgfqpoint{2.256448in}{3.102411in}}{\pgfqpoint{2.264261in}{3.110224in}}%
\pgfpathcurveto{\pgfqpoint{2.272075in}{3.118038in}}{\pgfqpoint{2.276465in}{3.128637in}}{\pgfqpoint{2.276465in}{3.139687in}}%
\pgfpathcurveto{\pgfqpoint{2.276465in}{3.150737in}}{\pgfqpoint{2.272075in}{3.161336in}}{\pgfqpoint{2.264261in}{3.169150in}}%
\pgfpathcurveto{\pgfqpoint{2.256448in}{3.176964in}}{\pgfqpoint{2.245848in}{3.181354in}}{\pgfqpoint{2.234798in}{3.181354in}}%
\pgfpathcurveto{\pgfqpoint{2.223748in}{3.181354in}}{\pgfqpoint{2.213149in}{3.176964in}}{\pgfqpoint{2.205336in}{3.169150in}}%
\pgfpathcurveto{\pgfqpoint{2.197522in}{3.161336in}}{\pgfqpoint{2.193132in}{3.150737in}}{\pgfqpoint{2.193132in}{3.139687in}}%
\pgfpathcurveto{\pgfqpoint{2.193132in}{3.128637in}}{\pgfqpoint{2.197522in}{3.118038in}}{\pgfqpoint{2.205336in}{3.110224in}}%
\pgfpathcurveto{\pgfqpoint{2.213149in}{3.102411in}}{\pgfqpoint{2.223748in}{3.098021in}}{\pgfqpoint{2.234798in}{3.098021in}}%
\pgfpathclose%
\pgfusepath{stroke,fill}%
\end{pgfscope}%
\begin{pgfscope}%
\pgfpathrectangle{\pgfqpoint{0.600000in}{0.600000in}}{\pgfqpoint{3.900000in}{3.900000in}}%
\pgfusepath{clip}%
\pgfsetbuttcap%
\pgfsetroundjoin%
\definecolor{currentfill}{rgb}{0.121569,0.466667,0.705882}%
\pgfsetfillcolor{currentfill}%
\pgfsetlinewidth{1.003750pt}%
\definecolor{currentstroke}{rgb}{0.121569,0.466667,0.705882}%
\pgfsetstrokecolor{currentstroke}%
\pgfsetdash{}{0pt}%
\pgfpathmoveto{\pgfqpoint{2.775965in}{1.789277in}}%
\pgfpathcurveto{\pgfqpoint{2.787015in}{1.789277in}}{\pgfqpoint{2.797614in}{1.793667in}}{\pgfqpoint{2.805428in}{1.801481in}}%
\pgfpathcurveto{\pgfqpoint{2.813242in}{1.809294in}}{\pgfqpoint{2.817632in}{1.819893in}}{\pgfqpoint{2.817632in}{1.830944in}}%
\pgfpathcurveto{\pgfqpoint{2.817632in}{1.841994in}}{\pgfqpoint{2.813242in}{1.852593in}}{\pgfqpoint{2.805428in}{1.860406in}}%
\pgfpathcurveto{\pgfqpoint{2.797614in}{1.868220in}}{\pgfqpoint{2.787015in}{1.872610in}}{\pgfqpoint{2.775965in}{1.872610in}}%
\pgfpathcurveto{\pgfqpoint{2.764915in}{1.872610in}}{\pgfqpoint{2.754316in}{1.868220in}}{\pgfqpoint{2.746502in}{1.860406in}}%
\pgfpathcurveto{\pgfqpoint{2.738689in}{1.852593in}}{\pgfqpoint{2.734298in}{1.841994in}}{\pgfqpoint{2.734298in}{1.830944in}}%
\pgfpathcurveto{\pgfqpoint{2.734298in}{1.819893in}}{\pgfqpoint{2.738689in}{1.809294in}}{\pgfqpoint{2.746502in}{1.801481in}}%
\pgfpathcurveto{\pgfqpoint{2.754316in}{1.793667in}}{\pgfqpoint{2.764915in}{1.789277in}}{\pgfqpoint{2.775965in}{1.789277in}}%
\pgfpathclose%
\pgfusepath{stroke,fill}%
\end{pgfscope}%
\begin{pgfscope}%
\pgfpathrectangle{\pgfqpoint{0.600000in}{0.600000in}}{\pgfqpoint{3.900000in}{3.900000in}}%
\pgfusepath{clip}%
\pgfsetbuttcap%
\pgfsetroundjoin%
\definecolor{currentfill}{rgb}{0.121569,0.466667,0.705882}%
\pgfsetfillcolor{currentfill}%
\pgfsetlinewidth{1.003750pt}%
\definecolor{currentstroke}{rgb}{0.121569,0.466667,0.705882}%
\pgfsetstrokecolor{currentstroke}%
\pgfsetdash{}{0pt}%
\pgfpathmoveto{\pgfqpoint{2.641058in}{1.705390in}}%
\pgfpathcurveto{\pgfqpoint{2.652108in}{1.705390in}}{\pgfqpoint{2.662707in}{1.709780in}}{\pgfqpoint{2.670521in}{1.717594in}}%
\pgfpathcurveto{\pgfqpoint{2.678335in}{1.725408in}}{\pgfqpoint{2.682725in}{1.736007in}}{\pgfqpoint{2.682725in}{1.747057in}}%
\pgfpathcurveto{\pgfqpoint{2.682725in}{1.758107in}}{\pgfqpoint{2.678335in}{1.768706in}}{\pgfqpoint{2.670521in}{1.776520in}}%
\pgfpathcurveto{\pgfqpoint{2.662707in}{1.784333in}}{\pgfqpoint{2.652108in}{1.788723in}}{\pgfqpoint{2.641058in}{1.788723in}}%
\pgfpathcurveto{\pgfqpoint{2.630008in}{1.788723in}}{\pgfqpoint{2.619409in}{1.784333in}}{\pgfqpoint{2.611596in}{1.776520in}}%
\pgfpathcurveto{\pgfqpoint{2.603782in}{1.768706in}}{\pgfqpoint{2.599392in}{1.758107in}}{\pgfqpoint{2.599392in}{1.747057in}}%
\pgfpathcurveto{\pgfqpoint{2.599392in}{1.736007in}}{\pgfqpoint{2.603782in}{1.725408in}}{\pgfqpoint{2.611596in}{1.717594in}}%
\pgfpathcurveto{\pgfqpoint{2.619409in}{1.709780in}}{\pgfqpoint{2.630008in}{1.705390in}}{\pgfqpoint{2.641058in}{1.705390in}}%
\pgfpathclose%
\pgfusepath{stroke,fill}%
\end{pgfscope}%
\begin{pgfscope}%
\pgfpathrectangle{\pgfqpoint{0.600000in}{0.600000in}}{\pgfqpoint{3.900000in}{3.900000in}}%
\pgfusepath{clip}%
\pgfsetbuttcap%
\pgfsetroundjoin%
\definecolor{currentfill}{rgb}{0.121569,0.466667,0.705882}%
\pgfsetfillcolor{currentfill}%
\pgfsetlinewidth{1.003750pt}%
\definecolor{currentstroke}{rgb}{0.121569,0.466667,0.705882}%
\pgfsetstrokecolor{currentstroke}%
\pgfsetdash{}{0pt}%
\pgfpathmoveto{\pgfqpoint{2.546604in}{2.441136in}}%
\pgfpathcurveto{\pgfqpoint{2.557654in}{2.441136in}}{\pgfqpoint{2.568253in}{2.445526in}}{\pgfqpoint{2.576067in}{2.453340in}}%
\pgfpathcurveto{\pgfqpoint{2.583881in}{2.461154in}}{\pgfqpoint{2.588271in}{2.471753in}}{\pgfqpoint{2.588271in}{2.482803in}}%
\pgfpathcurveto{\pgfqpoint{2.588271in}{2.493853in}}{\pgfqpoint{2.583881in}{2.504452in}}{\pgfqpoint{2.576067in}{2.512265in}}%
\pgfpathcurveto{\pgfqpoint{2.568253in}{2.520079in}}{\pgfqpoint{2.557654in}{2.524469in}}{\pgfqpoint{2.546604in}{2.524469in}}%
\pgfpathcurveto{\pgfqpoint{2.535554in}{2.524469in}}{\pgfqpoint{2.524955in}{2.520079in}}{\pgfqpoint{2.517141in}{2.512265in}}%
\pgfpathcurveto{\pgfqpoint{2.509328in}{2.504452in}}{\pgfqpoint{2.504937in}{2.493853in}}{\pgfqpoint{2.504937in}{2.482803in}}%
\pgfpathcurveto{\pgfqpoint{2.504937in}{2.471753in}}{\pgfqpoint{2.509328in}{2.461154in}}{\pgfqpoint{2.517141in}{2.453340in}}%
\pgfpathcurveto{\pgfqpoint{2.524955in}{2.445526in}}{\pgfqpoint{2.535554in}{2.441136in}}{\pgfqpoint{2.546604in}{2.441136in}}%
\pgfpathclose%
\pgfusepath{stroke,fill}%
\end{pgfscope}%
\begin{pgfscope}%
\pgfpathrectangle{\pgfqpoint{0.600000in}{0.600000in}}{\pgfqpoint{3.900000in}{3.900000in}}%
\pgfusepath{clip}%
\pgfsetbuttcap%
\pgfsetroundjoin%
\definecolor{currentfill}{rgb}{0.121569,0.466667,0.705882}%
\pgfsetfillcolor{currentfill}%
\pgfsetlinewidth{1.003750pt}%
\definecolor{currentstroke}{rgb}{0.121569,0.466667,0.705882}%
\pgfsetstrokecolor{currentstroke}%
\pgfsetdash{}{0pt}%
\pgfpathmoveto{\pgfqpoint{1.976268in}{1.906236in}}%
\pgfpathcurveto{\pgfqpoint{1.987318in}{1.906236in}}{\pgfqpoint{1.997918in}{1.910627in}}{\pgfqpoint{2.005731in}{1.918440in}}%
\pgfpathcurveto{\pgfqpoint{2.013545in}{1.926254in}}{\pgfqpoint{2.017935in}{1.936853in}}{\pgfqpoint{2.017935in}{1.947903in}}%
\pgfpathcurveto{\pgfqpoint{2.017935in}{1.958953in}}{\pgfqpoint{2.013545in}{1.969552in}}{\pgfqpoint{2.005731in}{1.977366in}}%
\pgfpathcurveto{\pgfqpoint{1.997918in}{1.985179in}}{\pgfqpoint{1.987318in}{1.989570in}}{\pgfqpoint{1.976268in}{1.989570in}}%
\pgfpathcurveto{\pgfqpoint{1.965218in}{1.989570in}}{\pgfqpoint{1.954619in}{1.985179in}}{\pgfqpoint{1.946806in}{1.977366in}}%
\pgfpathcurveto{\pgfqpoint{1.938992in}{1.969552in}}{\pgfqpoint{1.934602in}{1.958953in}}{\pgfqpoint{1.934602in}{1.947903in}}%
\pgfpathcurveto{\pgfqpoint{1.934602in}{1.936853in}}{\pgfqpoint{1.938992in}{1.926254in}}{\pgfqpoint{1.946806in}{1.918440in}}%
\pgfpathcurveto{\pgfqpoint{1.954619in}{1.910627in}}{\pgfqpoint{1.965218in}{1.906236in}}{\pgfqpoint{1.976268in}{1.906236in}}%
\pgfpathclose%
\pgfusepath{stroke,fill}%
\end{pgfscope}%
\begin{pgfscope}%
\pgfpathrectangle{\pgfqpoint{0.600000in}{0.600000in}}{\pgfqpoint{3.900000in}{3.900000in}}%
\pgfusepath{clip}%
\pgfsetbuttcap%
\pgfsetroundjoin%
\definecolor{currentfill}{rgb}{0.121569,0.466667,0.705882}%
\pgfsetfillcolor{currentfill}%
\pgfsetlinewidth{1.003750pt}%
\definecolor{currentstroke}{rgb}{0.121569,0.466667,0.705882}%
\pgfsetstrokecolor{currentstroke}%
\pgfsetdash{}{0pt}%
\pgfpathmoveto{\pgfqpoint{2.324783in}{0.742163in}}%
\pgfpathcurveto{\pgfqpoint{2.335833in}{0.742163in}}{\pgfqpoint{2.346432in}{0.746553in}}{\pgfqpoint{2.354246in}{0.754367in}}%
\pgfpathcurveto{\pgfqpoint{2.362059in}{0.762180in}}{\pgfqpoint{2.366450in}{0.772779in}}{\pgfqpoint{2.366450in}{0.783830in}}%
\pgfpathcurveto{\pgfqpoint{2.366450in}{0.794880in}}{\pgfqpoint{2.362059in}{0.805479in}}{\pgfqpoint{2.354246in}{0.813292in}}%
\pgfpathcurveto{\pgfqpoint{2.346432in}{0.821106in}}{\pgfqpoint{2.335833in}{0.825496in}}{\pgfqpoint{2.324783in}{0.825496in}}%
\pgfpathcurveto{\pgfqpoint{2.313733in}{0.825496in}}{\pgfqpoint{2.303134in}{0.821106in}}{\pgfqpoint{2.295320in}{0.813292in}}%
\pgfpathcurveto{\pgfqpoint{2.287506in}{0.805479in}}{\pgfqpoint{2.283116in}{0.794880in}}{\pgfqpoint{2.283116in}{0.783830in}}%
\pgfpathcurveto{\pgfqpoint{2.283116in}{0.772779in}}{\pgfqpoint{2.287506in}{0.762180in}}{\pgfqpoint{2.295320in}{0.754367in}}%
\pgfpathcurveto{\pgfqpoint{2.303134in}{0.746553in}}{\pgfqpoint{2.313733in}{0.742163in}}{\pgfqpoint{2.324783in}{0.742163in}}%
\pgfpathclose%
\pgfusepath{stroke,fill}%
\end{pgfscope}%
\begin{pgfscope}%
\pgfpathrectangle{\pgfqpoint{0.600000in}{0.600000in}}{\pgfqpoint{3.900000in}{3.900000in}}%
\pgfusepath{clip}%
\pgfsetbuttcap%
\pgfsetroundjoin%
\definecolor{currentfill}{rgb}{0.121569,0.466667,0.705882}%
\pgfsetfillcolor{currentfill}%
\pgfsetlinewidth{1.003750pt}%
\definecolor{currentstroke}{rgb}{0.121569,0.466667,0.705882}%
\pgfsetstrokecolor{currentstroke}%
\pgfsetdash{}{0pt}%
\pgfpathmoveto{\pgfqpoint{2.661087in}{3.032261in}}%
\pgfpathcurveto{\pgfqpoint{2.672137in}{3.032261in}}{\pgfqpoint{2.682736in}{3.036651in}}{\pgfqpoint{2.690549in}{3.044464in}}%
\pgfpathcurveto{\pgfqpoint{2.698363in}{3.052278in}}{\pgfqpoint{2.702753in}{3.062877in}}{\pgfqpoint{2.702753in}{3.073927in}}%
\pgfpathcurveto{\pgfqpoint{2.702753in}{3.084977in}}{\pgfqpoint{2.698363in}{3.095576in}}{\pgfqpoint{2.690549in}{3.103390in}}%
\pgfpathcurveto{\pgfqpoint{2.682736in}{3.111204in}}{\pgfqpoint{2.672137in}{3.115594in}}{\pgfqpoint{2.661087in}{3.115594in}}%
\pgfpathcurveto{\pgfqpoint{2.650036in}{3.115594in}}{\pgfqpoint{2.639437in}{3.111204in}}{\pgfqpoint{2.631624in}{3.103390in}}%
\pgfpathcurveto{\pgfqpoint{2.623810in}{3.095576in}}{\pgfqpoint{2.619420in}{3.084977in}}{\pgfqpoint{2.619420in}{3.073927in}}%
\pgfpathcurveto{\pgfqpoint{2.619420in}{3.062877in}}{\pgfqpoint{2.623810in}{3.052278in}}{\pgfqpoint{2.631624in}{3.044464in}}%
\pgfpathcurveto{\pgfqpoint{2.639437in}{3.036651in}}{\pgfqpoint{2.650036in}{3.032261in}}{\pgfqpoint{2.661087in}{3.032261in}}%
\pgfpathclose%
\pgfusepath{stroke,fill}%
\end{pgfscope}%
\begin{pgfscope}%
\pgfpathrectangle{\pgfqpoint{0.600000in}{0.600000in}}{\pgfqpoint{3.900000in}{3.900000in}}%
\pgfusepath{clip}%
\pgfsetbuttcap%
\pgfsetroundjoin%
\definecolor{currentfill}{rgb}{0.121569,0.466667,0.705882}%
\pgfsetfillcolor{currentfill}%
\pgfsetlinewidth{1.003750pt}%
\definecolor{currentstroke}{rgb}{0.121569,0.466667,0.705882}%
\pgfsetstrokecolor{currentstroke}%
\pgfsetdash{}{0pt}%
\pgfpathmoveto{\pgfqpoint{2.034400in}{1.911716in}}%
\pgfpathcurveto{\pgfqpoint{2.045450in}{1.911716in}}{\pgfqpoint{2.056049in}{1.916107in}}{\pgfqpoint{2.063863in}{1.923920in}}%
\pgfpathcurveto{\pgfqpoint{2.071676in}{1.931734in}}{\pgfqpoint{2.076067in}{1.942333in}}{\pgfqpoint{2.076067in}{1.953383in}}%
\pgfpathcurveto{\pgfqpoint{2.076067in}{1.964433in}}{\pgfqpoint{2.071676in}{1.975032in}}{\pgfqpoint{2.063863in}{1.982846in}}%
\pgfpathcurveto{\pgfqpoint{2.056049in}{1.990659in}}{\pgfqpoint{2.045450in}{1.995050in}}{\pgfqpoint{2.034400in}{1.995050in}}%
\pgfpathcurveto{\pgfqpoint{2.023350in}{1.995050in}}{\pgfqpoint{2.012751in}{1.990659in}}{\pgfqpoint{2.004937in}{1.982846in}}%
\pgfpathcurveto{\pgfqpoint{1.997124in}{1.975032in}}{\pgfqpoint{1.992733in}{1.964433in}}{\pgfqpoint{1.992733in}{1.953383in}}%
\pgfpathcurveto{\pgfqpoint{1.992733in}{1.942333in}}{\pgfqpoint{1.997124in}{1.931734in}}{\pgfqpoint{2.004937in}{1.923920in}}%
\pgfpathcurveto{\pgfqpoint{2.012751in}{1.916107in}}{\pgfqpoint{2.023350in}{1.911716in}}{\pgfqpoint{2.034400in}{1.911716in}}%
\pgfpathclose%
\pgfusepath{stroke,fill}%
\end{pgfscope}%
\begin{pgfscope}%
\pgfpathrectangle{\pgfqpoint{0.600000in}{0.600000in}}{\pgfqpoint{3.900000in}{3.900000in}}%
\pgfusepath{clip}%
\pgfsetbuttcap%
\pgfsetroundjoin%
\definecolor{currentfill}{rgb}{0.121569,0.466667,0.705882}%
\pgfsetfillcolor{currentfill}%
\pgfsetlinewidth{1.003750pt}%
\definecolor{currentstroke}{rgb}{0.121569,0.466667,0.705882}%
\pgfsetstrokecolor{currentstroke}%
\pgfsetdash{}{0pt}%
\pgfpathmoveto{\pgfqpoint{2.120558in}{2.476017in}}%
\pgfpathcurveto{\pgfqpoint{2.131608in}{2.476017in}}{\pgfqpoint{2.142207in}{2.480407in}}{\pgfqpoint{2.150021in}{2.488221in}}%
\pgfpathcurveto{\pgfqpoint{2.157835in}{2.496035in}}{\pgfqpoint{2.162225in}{2.506634in}}{\pgfqpoint{2.162225in}{2.517684in}}%
\pgfpathcurveto{\pgfqpoint{2.162225in}{2.528734in}}{\pgfqpoint{2.157835in}{2.539333in}}{\pgfqpoint{2.150021in}{2.547147in}}%
\pgfpathcurveto{\pgfqpoint{2.142207in}{2.554960in}}{\pgfqpoint{2.131608in}{2.559351in}}{\pgfqpoint{2.120558in}{2.559351in}}%
\pgfpathcurveto{\pgfqpoint{2.109508in}{2.559351in}}{\pgfqpoint{2.098909in}{2.554960in}}{\pgfqpoint{2.091095in}{2.547147in}}%
\pgfpathcurveto{\pgfqpoint{2.083282in}{2.539333in}}{\pgfqpoint{2.078891in}{2.528734in}}{\pgfqpoint{2.078891in}{2.517684in}}%
\pgfpathcurveto{\pgfqpoint{2.078891in}{2.506634in}}{\pgfqpoint{2.083282in}{2.496035in}}{\pgfqpoint{2.091095in}{2.488221in}}%
\pgfpathcurveto{\pgfqpoint{2.098909in}{2.480407in}}{\pgfqpoint{2.109508in}{2.476017in}}{\pgfqpoint{2.120558in}{2.476017in}}%
\pgfpathclose%
\pgfusepath{stroke,fill}%
\end{pgfscope}%
\begin{pgfscope}%
\pgfpathrectangle{\pgfqpoint{0.600000in}{0.600000in}}{\pgfqpoint{3.900000in}{3.900000in}}%
\pgfusepath{clip}%
\pgfsetbuttcap%
\pgfsetroundjoin%
\definecolor{currentfill}{rgb}{0.121569,0.466667,0.705882}%
\pgfsetfillcolor{currentfill}%
\pgfsetlinewidth{1.003750pt}%
\definecolor{currentstroke}{rgb}{0.121569,0.466667,0.705882}%
\pgfsetstrokecolor{currentstroke}%
\pgfsetdash{}{0pt}%
\pgfpathmoveto{\pgfqpoint{1.859083in}{2.602008in}}%
\pgfpathcurveto{\pgfqpoint{1.870133in}{2.602008in}}{\pgfqpoint{1.880732in}{2.606399in}}{\pgfqpoint{1.888545in}{2.614212in}}%
\pgfpathcurveto{\pgfqpoint{1.896359in}{2.622026in}}{\pgfqpoint{1.900749in}{2.632625in}}{\pgfqpoint{1.900749in}{2.643675in}}%
\pgfpathcurveto{\pgfqpoint{1.900749in}{2.654725in}}{\pgfqpoint{1.896359in}{2.665324in}}{\pgfqpoint{1.888545in}{2.673138in}}%
\pgfpathcurveto{\pgfqpoint{1.880732in}{2.680951in}}{\pgfqpoint{1.870133in}{2.685342in}}{\pgfqpoint{1.859083in}{2.685342in}}%
\pgfpathcurveto{\pgfqpoint{1.848033in}{2.685342in}}{\pgfqpoint{1.837433in}{2.680951in}}{\pgfqpoint{1.829620in}{2.673138in}}%
\pgfpathcurveto{\pgfqpoint{1.821806in}{2.665324in}}{\pgfqpoint{1.817416in}{2.654725in}}{\pgfqpoint{1.817416in}{2.643675in}}%
\pgfpathcurveto{\pgfqpoint{1.817416in}{2.632625in}}{\pgfqpoint{1.821806in}{2.622026in}}{\pgfqpoint{1.829620in}{2.614212in}}%
\pgfpathcurveto{\pgfqpoint{1.837433in}{2.606399in}}{\pgfqpoint{1.848033in}{2.602008in}}{\pgfqpoint{1.859083in}{2.602008in}}%
\pgfpathclose%
\pgfusepath{stroke,fill}%
\end{pgfscope}%
\begin{pgfscope}%
\pgfpathrectangle{\pgfqpoint{0.600000in}{0.600000in}}{\pgfqpoint{3.900000in}{3.900000in}}%
\pgfusepath{clip}%
\pgfsetbuttcap%
\pgfsetroundjoin%
\definecolor{currentfill}{rgb}{0.121569,0.466667,0.705882}%
\pgfsetfillcolor{currentfill}%
\pgfsetlinewidth{1.003750pt}%
\definecolor{currentstroke}{rgb}{0.121569,0.466667,0.705882}%
\pgfsetstrokecolor{currentstroke}%
\pgfsetdash{}{0pt}%
\pgfpathmoveto{\pgfqpoint{2.791513in}{2.599960in}}%
\pgfpathcurveto{\pgfqpoint{2.802563in}{2.599960in}}{\pgfqpoint{2.813163in}{2.604351in}}{\pgfqpoint{2.820976in}{2.612164in}}%
\pgfpathcurveto{\pgfqpoint{2.828790in}{2.619978in}}{\pgfqpoint{2.833180in}{2.630577in}}{\pgfqpoint{2.833180in}{2.641627in}}%
\pgfpathcurveto{\pgfqpoint{2.833180in}{2.652677in}}{\pgfqpoint{2.828790in}{2.663276in}}{\pgfqpoint{2.820976in}{2.671090in}}%
\pgfpathcurveto{\pgfqpoint{2.813163in}{2.678904in}}{\pgfqpoint{2.802563in}{2.683294in}}{\pgfqpoint{2.791513in}{2.683294in}}%
\pgfpathcurveto{\pgfqpoint{2.780463in}{2.683294in}}{\pgfqpoint{2.769864in}{2.678904in}}{\pgfqpoint{2.762051in}{2.671090in}}%
\pgfpathcurveto{\pgfqpoint{2.754237in}{2.663276in}}{\pgfqpoint{2.749847in}{2.652677in}}{\pgfqpoint{2.749847in}{2.641627in}}%
\pgfpathcurveto{\pgfqpoint{2.749847in}{2.630577in}}{\pgfqpoint{2.754237in}{2.619978in}}{\pgfqpoint{2.762051in}{2.612164in}}%
\pgfpathcurveto{\pgfqpoint{2.769864in}{2.604351in}}{\pgfqpoint{2.780463in}{2.599960in}}{\pgfqpoint{2.791513in}{2.599960in}}%
\pgfpathclose%
\pgfusepath{stroke,fill}%
\end{pgfscope}%
\begin{pgfscope}%
\pgfpathrectangle{\pgfqpoint{0.600000in}{0.600000in}}{\pgfqpoint{3.900000in}{3.900000in}}%
\pgfusepath{clip}%
\pgfsetbuttcap%
\pgfsetroundjoin%
\definecolor{currentfill}{rgb}{0.121569,0.466667,0.705882}%
\pgfsetfillcolor{currentfill}%
\pgfsetlinewidth{1.003750pt}%
\definecolor{currentstroke}{rgb}{0.121569,0.466667,0.705882}%
\pgfsetstrokecolor{currentstroke}%
\pgfsetdash{}{0pt}%
\pgfpathmoveto{\pgfqpoint{2.815357in}{2.252148in}}%
\pgfpathcurveto{\pgfqpoint{2.826407in}{2.252148in}}{\pgfqpoint{2.837006in}{2.256538in}}{\pgfqpoint{2.844819in}{2.264352in}}%
\pgfpathcurveto{\pgfqpoint{2.852633in}{2.272166in}}{\pgfqpoint{2.857023in}{2.282765in}}{\pgfqpoint{2.857023in}{2.293815in}}%
\pgfpathcurveto{\pgfqpoint{2.857023in}{2.304865in}}{\pgfqpoint{2.852633in}{2.315464in}}{\pgfqpoint{2.844819in}{2.323278in}}%
\pgfpathcurveto{\pgfqpoint{2.837006in}{2.331091in}}{\pgfqpoint{2.826407in}{2.335481in}}{\pgfqpoint{2.815357in}{2.335481in}}%
\pgfpathcurveto{\pgfqpoint{2.804307in}{2.335481in}}{\pgfqpoint{2.793708in}{2.331091in}}{\pgfqpoint{2.785894in}{2.323278in}}%
\pgfpathcurveto{\pgfqpoint{2.778080in}{2.315464in}}{\pgfqpoint{2.773690in}{2.304865in}}{\pgfqpoint{2.773690in}{2.293815in}}%
\pgfpathcurveto{\pgfqpoint{2.773690in}{2.282765in}}{\pgfqpoint{2.778080in}{2.272166in}}{\pgfqpoint{2.785894in}{2.264352in}}%
\pgfpathcurveto{\pgfqpoint{2.793708in}{2.256538in}}{\pgfqpoint{2.804307in}{2.252148in}}{\pgfqpoint{2.815357in}{2.252148in}}%
\pgfpathclose%
\pgfusepath{stroke,fill}%
\end{pgfscope}%
\begin{pgfscope}%
\pgfpathrectangle{\pgfqpoint{0.600000in}{0.600000in}}{\pgfqpoint{3.900000in}{3.900000in}}%
\pgfusepath{clip}%
\pgfsetbuttcap%
\pgfsetroundjoin%
\definecolor{currentfill}{rgb}{0.121569,0.466667,0.705882}%
\pgfsetfillcolor{currentfill}%
\pgfsetlinewidth{1.003750pt}%
\definecolor{currentstroke}{rgb}{0.121569,0.466667,0.705882}%
\pgfsetstrokecolor{currentstroke}%
\pgfsetdash{}{0pt}%
\pgfpathmoveto{\pgfqpoint{2.612261in}{2.615681in}}%
\pgfpathcurveto{\pgfqpoint{2.623311in}{2.615681in}}{\pgfqpoint{2.633910in}{2.620071in}}{\pgfqpoint{2.641724in}{2.627885in}}%
\pgfpathcurveto{\pgfqpoint{2.649537in}{2.635699in}}{\pgfqpoint{2.653928in}{2.646298in}}{\pgfqpoint{2.653928in}{2.657348in}}%
\pgfpathcurveto{\pgfqpoint{2.653928in}{2.668398in}}{\pgfqpoint{2.649537in}{2.678997in}}{\pgfqpoint{2.641724in}{2.686811in}}%
\pgfpathcurveto{\pgfqpoint{2.633910in}{2.694624in}}{\pgfqpoint{2.623311in}{2.699015in}}{\pgfqpoint{2.612261in}{2.699015in}}%
\pgfpathcurveto{\pgfqpoint{2.601211in}{2.699015in}}{\pgfqpoint{2.590612in}{2.694624in}}{\pgfqpoint{2.582798in}{2.686811in}}%
\pgfpathcurveto{\pgfqpoint{2.574984in}{2.678997in}}{\pgfqpoint{2.570594in}{2.668398in}}{\pgfqpoint{2.570594in}{2.657348in}}%
\pgfpathcurveto{\pgfqpoint{2.570594in}{2.646298in}}{\pgfqpoint{2.574984in}{2.635699in}}{\pgfqpoint{2.582798in}{2.627885in}}%
\pgfpathcurveto{\pgfqpoint{2.590612in}{2.620071in}}{\pgfqpoint{2.601211in}{2.615681in}}{\pgfqpoint{2.612261in}{2.615681in}}%
\pgfpathclose%
\pgfusepath{stroke,fill}%
\end{pgfscope}%
\begin{pgfscope}%
\pgfpathrectangle{\pgfqpoint{0.600000in}{0.600000in}}{\pgfqpoint{3.900000in}{3.900000in}}%
\pgfusepath{clip}%
\pgfsetbuttcap%
\pgfsetroundjoin%
\definecolor{currentfill}{rgb}{0.121569,0.466667,0.705882}%
\pgfsetfillcolor{currentfill}%
\pgfsetlinewidth{1.003750pt}%
\definecolor{currentstroke}{rgb}{0.121569,0.466667,0.705882}%
\pgfsetstrokecolor{currentstroke}%
\pgfsetdash{}{0pt}%
\pgfpathmoveto{\pgfqpoint{2.033180in}{1.707666in}}%
\pgfpathcurveto{\pgfqpoint{2.044230in}{1.707666in}}{\pgfqpoint{2.054829in}{1.712056in}}{\pgfqpoint{2.062643in}{1.719870in}}%
\pgfpathcurveto{\pgfqpoint{2.070456in}{1.727684in}}{\pgfqpoint{2.074846in}{1.738283in}}{\pgfqpoint{2.074846in}{1.749333in}}%
\pgfpathcurveto{\pgfqpoint{2.074846in}{1.760383in}}{\pgfqpoint{2.070456in}{1.770982in}}{\pgfqpoint{2.062643in}{1.778796in}}%
\pgfpathcurveto{\pgfqpoint{2.054829in}{1.786609in}}{\pgfqpoint{2.044230in}{1.791000in}}{\pgfqpoint{2.033180in}{1.791000in}}%
\pgfpathcurveto{\pgfqpoint{2.022130in}{1.791000in}}{\pgfqpoint{2.011531in}{1.786609in}}{\pgfqpoint{2.003717in}{1.778796in}}%
\pgfpathcurveto{\pgfqpoint{1.995903in}{1.770982in}}{\pgfqpoint{1.991513in}{1.760383in}}{\pgfqpoint{1.991513in}{1.749333in}}%
\pgfpathcurveto{\pgfqpoint{1.991513in}{1.738283in}}{\pgfqpoint{1.995903in}{1.727684in}}{\pgfqpoint{2.003717in}{1.719870in}}%
\pgfpathcurveto{\pgfqpoint{2.011531in}{1.712056in}}{\pgfqpoint{2.022130in}{1.707666in}}{\pgfqpoint{2.033180in}{1.707666in}}%
\pgfpathclose%
\pgfusepath{stroke,fill}%
\end{pgfscope}%
\begin{pgfscope}%
\pgfpathrectangle{\pgfqpoint{0.600000in}{0.600000in}}{\pgfqpoint{3.900000in}{3.900000in}}%
\pgfusepath{clip}%
\pgfsetbuttcap%
\pgfsetroundjoin%
\definecolor{currentfill}{rgb}{0.121569,0.466667,0.705882}%
\pgfsetfillcolor{currentfill}%
\pgfsetlinewidth{1.003750pt}%
\definecolor{currentstroke}{rgb}{0.121569,0.466667,0.705882}%
\pgfsetstrokecolor{currentstroke}%
\pgfsetdash{}{0pt}%
\pgfpathmoveto{\pgfqpoint{2.880687in}{2.192538in}}%
\pgfpathcurveto{\pgfqpoint{2.891737in}{2.192538in}}{\pgfqpoint{2.902336in}{2.196928in}}{\pgfqpoint{2.910150in}{2.204742in}}%
\pgfpathcurveto{\pgfqpoint{2.917964in}{2.212556in}}{\pgfqpoint{2.922354in}{2.223155in}}{\pgfqpoint{2.922354in}{2.234205in}}%
\pgfpathcurveto{\pgfqpoint{2.922354in}{2.245255in}}{\pgfqpoint{2.917964in}{2.255854in}}{\pgfqpoint{2.910150in}{2.263667in}}%
\pgfpathcurveto{\pgfqpoint{2.902336in}{2.271481in}}{\pgfqpoint{2.891737in}{2.275871in}}{\pgfqpoint{2.880687in}{2.275871in}}%
\pgfpathcurveto{\pgfqpoint{2.869637in}{2.275871in}}{\pgfqpoint{2.859038in}{2.271481in}}{\pgfqpoint{2.851224in}{2.263667in}}%
\pgfpathcurveto{\pgfqpoint{2.843411in}{2.255854in}}{\pgfqpoint{2.839021in}{2.245255in}}{\pgfqpoint{2.839021in}{2.234205in}}%
\pgfpathcurveto{\pgfqpoint{2.839021in}{2.223155in}}{\pgfqpoint{2.843411in}{2.212556in}}{\pgfqpoint{2.851224in}{2.204742in}}%
\pgfpathcurveto{\pgfqpoint{2.859038in}{2.196928in}}{\pgfqpoint{2.869637in}{2.192538in}}{\pgfqpoint{2.880687in}{2.192538in}}%
\pgfpathclose%
\pgfusepath{stroke,fill}%
\end{pgfscope}%
\begin{pgfscope}%
\pgfpathrectangle{\pgfqpoint{0.600000in}{0.600000in}}{\pgfqpoint{3.900000in}{3.900000in}}%
\pgfusepath{clip}%
\pgfsetbuttcap%
\pgfsetroundjoin%
\definecolor{currentfill}{rgb}{0.121569,0.466667,0.705882}%
\pgfsetfillcolor{currentfill}%
\pgfsetlinewidth{1.003750pt}%
\definecolor{currentstroke}{rgb}{0.121569,0.466667,0.705882}%
\pgfsetstrokecolor{currentstroke}%
\pgfsetdash{}{0pt}%
\pgfpathmoveto{\pgfqpoint{2.468614in}{1.733225in}}%
\pgfpathcurveto{\pgfqpoint{2.479665in}{1.733225in}}{\pgfqpoint{2.490264in}{1.737616in}}{\pgfqpoint{2.498077in}{1.745429in}}%
\pgfpathcurveto{\pgfqpoint{2.505891in}{1.753243in}}{\pgfqpoint{2.510281in}{1.763842in}}{\pgfqpoint{2.510281in}{1.774892in}}%
\pgfpathcurveto{\pgfqpoint{2.510281in}{1.785942in}}{\pgfqpoint{2.505891in}{1.796541in}}{\pgfqpoint{2.498077in}{1.804355in}}%
\pgfpathcurveto{\pgfqpoint{2.490264in}{1.812169in}}{\pgfqpoint{2.479665in}{1.816559in}}{\pgfqpoint{2.468614in}{1.816559in}}%
\pgfpathcurveto{\pgfqpoint{2.457564in}{1.816559in}}{\pgfqpoint{2.446965in}{1.812169in}}{\pgfqpoint{2.439152in}{1.804355in}}%
\pgfpathcurveto{\pgfqpoint{2.431338in}{1.796541in}}{\pgfqpoint{2.426948in}{1.785942in}}{\pgfqpoint{2.426948in}{1.774892in}}%
\pgfpathcurveto{\pgfqpoint{2.426948in}{1.763842in}}{\pgfqpoint{2.431338in}{1.753243in}}{\pgfqpoint{2.439152in}{1.745429in}}%
\pgfpathcurveto{\pgfqpoint{2.446965in}{1.737616in}}{\pgfqpoint{2.457564in}{1.733225in}}{\pgfqpoint{2.468614in}{1.733225in}}%
\pgfpathclose%
\pgfusepath{stroke,fill}%
\end{pgfscope}%
\begin{pgfscope}%
\pgfpathrectangle{\pgfqpoint{0.600000in}{0.600000in}}{\pgfqpoint{3.900000in}{3.900000in}}%
\pgfusepath{clip}%
\pgfsetbuttcap%
\pgfsetroundjoin%
\definecolor{currentfill}{rgb}{0.121569,0.466667,0.705882}%
\pgfsetfillcolor{currentfill}%
\pgfsetlinewidth{1.003750pt}%
\definecolor{currentstroke}{rgb}{0.121569,0.466667,0.705882}%
\pgfsetstrokecolor{currentstroke}%
\pgfsetdash{}{0pt}%
\pgfpathmoveto{\pgfqpoint{2.827309in}{3.392437in}}%
\pgfpathcurveto{\pgfqpoint{2.838359in}{3.392437in}}{\pgfqpoint{2.848958in}{3.396827in}}{\pgfqpoint{2.856772in}{3.404641in}}%
\pgfpathcurveto{\pgfqpoint{2.864585in}{3.412454in}}{\pgfqpoint{2.868976in}{3.423053in}}{\pgfqpoint{2.868976in}{3.434103in}}%
\pgfpathcurveto{\pgfqpoint{2.868976in}{3.445154in}}{\pgfqpoint{2.864585in}{3.455753in}}{\pgfqpoint{2.856772in}{3.463566in}}%
\pgfpathcurveto{\pgfqpoint{2.848958in}{3.471380in}}{\pgfqpoint{2.838359in}{3.475770in}}{\pgfqpoint{2.827309in}{3.475770in}}%
\pgfpathcurveto{\pgfqpoint{2.816259in}{3.475770in}}{\pgfqpoint{2.805660in}{3.471380in}}{\pgfqpoint{2.797846in}{3.463566in}}%
\pgfpathcurveto{\pgfqpoint{2.790033in}{3.455753in}}{\pgfqpoint{2.785642in}{3.445154in}}{\pgfqpoint{2.785642in}{3.434103in}}%
\pgfpathcurveto{\pgfqpoint{2.785642in}{3.423053in}}{\pgfqpoint{2.790033in}{3.412454in}}{\pgfqpoint{2.797846in}{3.404641in}}%
\pgfpathcurveto{\pgfqpoint{2.805660in}{3.396827in}}{\pgfqpoint{2.816259in}{3.392437in}}{\pgfqpoint{2.827309in}{3.392437in}}%
\pgfpathclose%
\pgfusepath{stroke,fill}%
\end{pgfscope}%
\begin{pgfscope}%
\pgfpathrectangle{\pgfqpoint{0.600000in}{0.600000in}}{\pgfqpoint{3.900000in}{3.900000in}}%
\pgfusepath{clip}%
\pgfsetbuttcap%
\pgfsetroundjoin%
\definecolor{currentfill}{rgb}{0.121569,0.466667,0.705882}%
\pgfsetfillcolor{currentfill}%
\pgfsetlinewidth{1.003750pt}%
\definecolor{currentstroke}{rgb}{0.121569,0.466667,0.705882}%
\pgfsetstrokecolor{currentstroke}%
\pgfsetdash{}{0pt}%
\pgfpathmoveto{\pgfqpoint{2.396509in}{2.663754in}}%
\pgfpathcurveto{\pgfqpoint{2.407559in}{2.663754in}}{\pgfqpoint{2.418158in}{2.668144in}}{\pgfqpoint{2.425972in}{2.675957in}}%
\pgfpathcurveto{\pgfqpoint{2.433785in}{2.683771in}}{\pgfqpoint{2.438176in}{2.694370in}}{\pgfqpoint{2.438176in}{2.705420in}}%
\pgfpathcurveto{\pgfqpoint{2.438176in}{2.716470in}}{\pgfqpoint{2.433785in}{2.727069in}}{\pgfqpoint{2.425972in}{2.734883in}}%
\pgfpathcurveto{\pgfqpoint{2.418158in}{2.742697in}}{\pgfqpoint{2.407559in}{2.747087in}}{\pgfqpoint{2.396509in}{2.747087in}}%
\pgfpathcurveto{\pgfqpoint{2.385459in}{2.747087in}}{\pgfqpoint{2.374860in}{2.742697in}}{\pgfqpoint{2.367046in}{2.734883in}}%
\pgfpathcurveto{\pgfqpoint{2.359233in}{2.727069in}}{\pgfqpoint{2.354842in}{2.716470in}}{\pgfqpoint{2.354842in}{2.705420in}}%
\pgfpathcurveto{\pgfqpoint{2.354842in}{2.694370in}}{\pgfqpoint{2.359233in}{2.683771in}}{\pgfqpoint{2.367046in}{2.675957in}}%
\pgfpathcurveto{\pgfqpoint{2.374860in}{2.668144in}}{\pgfqpoint{2.385459in}{2.663754in}}{\pgfqpoint{2.396509in}{2.663754in}}%
\pgfpathclose%
\pgfusepath{stroke,fill}%
\end{pgfscope}%
\begin{pgfscope}%
\pgfpathrectangle{\pgfqpoint{0.600000in}{0.600000in}}{\pgfqpoint{3.900000in}{3.900000in}}%
\pgfusepath{clip}%
\pgfsetbuttcap%
\pgfsetroundjoin%
\definecolor{currentfill}{rgb}{0.121569,0.466667,0.705882}%
\pgfsetfillcolor{currentfill}%
\pgfsetlinewidth{1.003750pt}%
\definecolor{currentstroke}{rgb}{0.121569,0.466667,0.705882}%
\pgfsetstrokecolor{currentstroke}%
\pgfsetdash{}{0pt}%
\pgfpathmoveto{\pgfqpoint{2.103356in}{2.932325in}}%
\pgfpathcurveto{\pgfqpoint{2.114406in}{2.932325in}}{\pgfqpoint{2.125005in}{2.936715in}}{\pgfqpoint{2.132819in}{2.944529in}}%
\pgfpathcurveto{\pgfqpoint{2.140632in}{2.952342in}}{\pgfqpoint{2.145022in}{2.962941in}}{\pgfqpoint{2.145022in}{2.973991in}}%
\pgfpathcurveto{\pgfqpoint{2.145022in}{2.985041in}}{\pgfqpoint{2.140632in}{2.995640in}}{\pgfqpoint{2.132819in}{3.003454in}}%
\pgfpathcurveto{\pgfqpoint{2.125005in}{3.011268in}}{\pgfqpoint{2.114406in}{3.015658in}}{\pgfqpoint{2.103356in}{3.015658in}}%
\pgfpathcurveto{\pgfqpoint{2.092306in}{3.015658in}}{\pgfqpoint{2.081707in}{3.011268in}}{\pgfqpoint{2.073893in}{3.003454in}}%
\pgfpathcurveto{\pgfqpoint{2.066079in}{2.995640in}}{\pgfqpoint{2.061689in}{2.985041in}}{\pgfqpoint{2.061689in}{2.973991in}}%
\pgfpathcurveto{\pgfqpoint{2.061689in}{2.962941in}}{\pgfqpoint{2.066079in}{2.952342in}}{\pgfqpoint{2.073893in}{2.944529in}}%
\pgfpathcurveto{\pgfqpoint{2.081707in}{2.936715in}}{\pgfqpoint{2.092306in}{2.932325in}}{\pgfqpoint{2.103356in}{2.932325in}}%
\pgfpathclose%
\pgfusepath{stroke,fill}%
\end{pgfscope}%
\begin{pgfscope}%
\pgfpathrectangle{\pgfqpoint{0.600000in}{0.600000in}}{\pgfqpoint{3.900000in}{3.900000in}}%
\pgfusepath{clip}%
\pgfsetbuttcap%
\pgfsetroundjoin%
\definecolor{currentfill}{rgb}{0.121569,0.466667,0.705882}%
\pgfsetfillcolor{currentfill}%
\pgfsetlinewidth{1.003750pt}%
\definecolor{currentstroke}{rgb}{0.121569,0.466667,0.705882}%
\pgfsetstrokecolor{currentstroke}%
\pgfsetdash{}{0pt}%
\pgfpathmoveto{\pgfqpoint{1.359951in}{3.868414in}}%
\pgfpathcurveto{\pgfqpoint{1.371002in}{3.868414in}}{\pgfqpoint{1.381601in}{3.872804in}}{\pgfqpoint{1.389414in}{3.880617in}}%
\pgfpathcurveto{\pgfqpoint{1.397228in}{3.888431in}}{\pgfqpoint{1.401618in}{3.899030in}}{\pgfqpoint{1.401618in}{3.910080in}}%
\pgfpathcurveto{\pgfqpoint{1.401618in}{3.921130in}}{\pgfqpoint{1.397228in}{3.931729in}}{\pgfqpoint{1.389414in}{3.939543in}}%
\pgfpathcurveto{\pgfqpoint{1.381601in}{3.947357in}}{\pgfqpoint{1.371002in}{3.951747in}}{\pgfqpoint{1.359951in}{3.951747in}}%
\pgfpathcurveto{\pgfqpoint{1.348901in}{3.951747in}}{\pgfqpoint{1.338302in}{3.947357in}}{\pgfqpoint{1.330489in}{3.939543in}}%
\pgfpathcurveto{\pgfqpoint{1.322675in}{3.931729in}}{\pgfqpoint{1.318285in}{3.921130in}}{\pgfqpoint{1.318285in}{3.910080in}}%
\pgfpathcurveto{\pgfqpoint{1.318285in}{3.899030in}}{\pgfqpoint{1.322675in}{3.888431in}}{\pgfqpoint{1.330489in}{3.880617in}}%
\pgfpathcurveto{\pgfqpoint{1.338302in}{3.872804in}}{\pgfqpoint{1.348901in}{3.868414in}}{\pgfqpoint{1.359951in}{3.868414in}}%
\pgfpathclose%
\pgfusepath{stroke,fill}%
\end{pgfscope}%
\begin{pgfscope}%
\pgfpathrectangle{\pgfqpoint{0.600000in}{0.600000in}}{\pgfqpoint{3.900000in}{3.900000in}}%
\pgfusepath{clip}%
\pgfsetbuttcap%
\pgfsetroundjoin%
\definecolor{currentfill}{rgb}{0.121569,0.466667,0.705882}%
\pgfsetfillcolor{currentfill}%
\pgfsetlinewidth{1.003750pt}%
\definecolor{currentstroke}{rgb}{0.121569,0.466667,0.705882}%
\pgfsetstrokecolor{currentstroke}%
\pgfsetdash{}{0pt}%
\pgfpathmoveto{\pgfqpoint{2.645577in}{2.999519in}}%
\pgfpathcurveto{\pgfqpoint{2.656627in}{2.999519in}}{\pgfqpoint{2.667226in}{3.003909in}}{\pgfqpoint{2.675040in}{3.011723in}}%
\pgfpathcurveto{\pgfqpoint{2.682853in}{3.019536in}}{\pgfqpoint{2.687244in}{3.030135in}}{\pgfqpoint{2.687244in}{3.041185in}}%
\pgfpathcurveto{\pgfqpoint{2.687244in}{3.052235in}}{\pgfqpoint{2.682853in}{3.062834in}}{\pgfqpoint{2.675040in}{3.070648in}}%
\pgfpathcurveto{\pgfqpoint{2.667226in}{3.078462in}}{\pgfqpoint{2.656627in}{3.082852in}}{\pgfqpoint{2.645577in}{3.082852in}}%
\pgfpathcurveto{\pgfqpoint{2.634527in}{3.082852in}}{\pgfqpoint{2.623928in}{3.078462in}}{\pgfqpoint{2.616114in}{3.070648in}}%
\pgfpathcurveto{\pgfqpoint{2.608301in}{3.062834in}}{\pgfqpoint{2.603910in}{3.052235in}}{\pgfqpoint{2.603910in}{3.041185in}}%
\pgfpathcurveto{\pgfqpoint{2.603910in}{3.030135in}}{\pgfqpoint{2.608301in}{3.019536in}}{\pgfqpoint{2.616114in}{3.011723in}}%
\pgfpathcurveto{\pgfqpoint{2.623928in}{3.003909in}}{\pgfqpoint{2.634527in}{2.999519in}}{\pgfqpoint{2.645577in}{2.999519in}}%
\pgfpathclose%
\pgfusepath{stroke,fill}%
\end{pgfscope}%
\begin{pgfscope}%
\pgfpathrectangle{\pgfqpoint{0.600000in}{0.600000in}}{\pgfqpoint{3.900000in}{3.900000in}}%
\pgfusepath{clip}%
\pgfsetbuttcap%
\pgfsetroundjoin%
\definecolor{currentfill}{rgb}{0.121569,0.466667,0.705882}%
\pgfsetfillcolor{currentfill}%
\pgfsetlinewidth{1.003750pt}%
\definecolor{currentstroke}{rgb}{0.121569,0.466667,0.705882}%
\pgfsetstrokecolor{currentstroke}%
\pgfsetdash{}{0pt}%
\pgfpathmoveto{\pgfqpoint{2.957575in}{2.954597in}}%
\pgfpathcurveto{\pgfqpoint{2.968626in}{2.954597in}}{\pgfqpoint{2.979225in}{2.958987in}}{\pgfqpoint{2.987038in}{2.966801in}}%
\pgfpathcurveto{\pgfqpoint{2.994852in}{2.974614in}}{\pgfqpoint{2.999242in}{2.985213in}}{\pgfqpoint{2.999242in}{2.996263in}}%
\pgfpathcurveto{\pgfqpoint{2.999242in}{3.007313in}}{\pgfqpoint{2.994852in}{3.017912in}}{\pgfqpoint{2.987038in}{3.025726in}}%
\pgfpathcurveto{\pgfqpoint{2.979225in}{3.033540in}}{\pgfqpoint{2.968626in}{3.037930in}}{\pgfqpoint{2.957575in}{3.037930in}}%
\pgfpathcurveto{\pgfqpoint{2.946525in}{3.037930in}}{\pgfqpoint{2.935926in}{3.033540in}}{\pgfqpoint{2.928113in}{3.025726in}}%
\pgfpathcurveto{\pgfqpoint{2.920299in}{3.017912in}}{\pgfqpoint{2.915909in}{3.007313in}}{\pgfqpoint{2.915909in}{2.996263in}}%
\pgfpathcurveto{\pgfqpoint{2.915909in}{2.985213in}}{\pgfqpoint{2.920299in}{2.974614in}}{\pgfqpoint{2.928113in}{2.966801in}}%
\pgfpathcurveto{\pgfqpoint{2.935926in}{2.958987in}}{\pgfqpoint{2.946525in}{2.954597in}}{\pgfqpoint{2.957575in}{2.954597in}}%
\pgfpathclose%
\pgfusepath{stroke,fill}%
\end{pgfscope}%
\begin{pgfscope}%
\pgfpathrectangle{\pgfqpoint{0.600000in}{0.600000in}}{\pgfqpoint{3.900000in}{3.900000in}}%
\pgfusepath{clip}%
\pgfsetbuttcap%
\pgfsetroundjoin%
\definecolor{currentfill}{rgb}{0.121569,0.466667,0.705882}%
\pgfsetfillcolor{currentfill}%
\pgfsetlinewidth{1.003750pt}%
\definecolor{currentstroke}{rgb}{0.121569,0.466667,0.705882}%
\pgfsetstrokecolor{currentstroke}%
\pgfsetdash{}{0pt}%
\pgfpathmoveto{\pgfqpoint{2.925760in}{2.485238in}}%
\pgfpathcurveto{\pgfqpoint{2.936810in}{2.485238in}}{\pgfqpoint{2.947409in}{2.489628in}}{\pgfqpoint{2.955223in}{2.497442in}}%
\pgfpathcurveto{\pgfqpoint{2.963036in}{2.505255in}}{\pgfqpoint{2.967427in}{2.515854in}}{\pgfqpoint{2.967427in}{2.526905in}}%
\pgfpathcurveto{\pgfqpoint{2.967427in}{2.537955in}}{\pgfqpoint{2.963036in}{2.548554in}}{\pgfqpoint{2.955223in}{2.556367in}}%
\pgfpathcurveto{\pgfqpoint{2.947409in}{2.564181in}}{\pgfqpoint{2.936810in}{2.568571in}}{\pgfqpoint{2.925760in}{2.568571in}}%
\pgfpathcurveto{\pgfqpoint{2.914710in}{2.568571in}}{\pgfqpoint{2.904111in}{2.564181in}}{\pgfqpoint{2.896297in}{2.556367in}}%
\pgfpathcurveto{\pgfqpoint{2.888484in}{2.548554in}}{\pgfqpoint{2.884093in}{2.537955in}}{\pgfqpoint{2.884093in}{2.526905in}}%
\pgfpathcurveto{\pgfqpoint{2.884093in}{2.515854in}}{\pgfqpoint{2.888484in}{2.505255in}}{\pgfqpoint{2.896297in}{2.497442in}}%
\pgfpathcurveto{\pgfqpoint{2.904111in}{2.489628in}}{\pgfqpoint{2.914710in}{2.485238in}}{\pgfqpoint{2.925760in}{2.485238in}}%
\pgfpathclose%
\pgfusepath{stroke,fill}%
\end{pgfscope}%
\begin{pgfscope}%
\pgfpathrectangle{\pgfqpoint{0.600000in}{0.600000in}}{\pgfqpoint{3.900000in}{3.900000in}}%
\pgfusepath{clip}%
\pgfsetbuttcap%
\pgfsetroundjoin%
\definecolor{currentfill}{rgb}{0.121569,0.466667,0.705882}%
\pgfsetfillcolor{currentfill}%
\pgfsetlinewidth{1.003750pt}%
\definecolor{currentstroke}{rgb}{0.121569,0.466667,0.705882}%
\pgfsetstrokecolor{currentstroke}%
\pgfsetdash{}{0pt}%
\pgfpathmoveto{\pgfqpoint{2.197381in}{3.037494in}}%
\pgfpathcurveto{\pgfqpoint{2.208431in}{3.037494in}}{\pgfqpoint{2.219030in}{3.041884in}}{\pgfqpoint{2.226843in}{3.049697in}}%
\pgfpathcurveto{\pgfqpoint{2.234657in}{3.057511in}}{\pgfqpoint{2.239047in}{3.068110in}}{\pgfqpoint{2.239047in}{3.079160in}}%
\pgfpathcurveto{\pgfqpoint{2.239047in}{3.090210in}}{\pgfqpoint{2.234657in}{3.100809in}}{\pgfqpoint{2.226843in}{3.108623in}}%
\pgfpathcurveto{\pgfqpoint{2.219030in}{3.116437in}}{\pgfqpoint{2.208431in}{3.120827in}}{\pgfqpoint{2.197381in}{3.120827in}}%
\pgfpathcurveto{\pgfqpoint{2.186330in}{3.120827in}}{\pgfqpoint{2.175731in}{3.116437in}}{\pgfqpoint{2.167918in}{3.108623in}}%
\pgfpathcurveto{\pgfqpoint{2.160104in}{3.100809in}}{\pgfqpoint{2.155714in}{3.090210in}}{\pgfqpoint{2.155714in}{3.079160in}}%
\pgfpathcurveto{\pgfqpoint{2.155714in}{3.068110in}}{\pgfqpoint{2.160104in}{3.057511in}}{\pgfqpoint{2.167918in}{3.049697in}}%
\pgfpathcurveto{\pgfqpoint{2.175731in}{3.041884in}}{\pgfqpoint{2.186330in}{3.037494in}}{\pgfqpoint{2.197381in}{3.037494in}}%
\pgfpathclose%
\pgfusepath{stroke,fill}%
\end{pgfscope}%
\begin{pgfscope}%
\pgfpathrectangle{\pgfqpoint{0.600000in}{0.600000in}}{\pgfqpoint{3.900000in}{3.900000in}}%
\pgfusepath{clip}%
\pgfsetbuttcap%
\pgfsetroundjoin%
\definecolor{currentfill}{rgb}{0.121569,0.466667,0.705882}%
\pgfsetfillcolor{currentfill}%
\pgfsetlinewidth{1.003750pt}%
\definecolor{currentstroke}{rgb}{0.121569,0.466667,0.705882}%
\pgfsetstrokecolor{currentstroke}%
\pgfsetdash{}{0pt}%
\pgfpathmoveto{\pgfqpoint{2.790913in}{2.449806in}}%
\pgfpathcurveto{\pgfqpoint{2.801963in}{2.449806in}}{\pgfqpoint{2.812562in}{2.454197in}}{\pgfqpoint{2.820376in}{2.462010in}}%
\pgfpathcurveto{\pgfqpoint{2.828190in}{2.469824in}}{\pgfqpoint{2.832580in}{2.480423in}}{\pgfqpoint{2.832580in}{2.491473in}}%
\pgfpathcurveto{\pgfqpoint{2.832580in}{2.502523in}}{\pgfqpoint{2.828190in}{2.513122in}}{\pgfqpoint{2.820376in}{2.520936in}}%
\pgfpathcurveto{\pgfqpoint{2.812562in}{2.528750in}}{\pgfqpoint{2.801963in}{2.533140in}}{\pgfqpoint{2.790913in}{2.533140in}}%
\pgfpathcurveto{\pgfqpoint{2.779863in}{2.533140in}}{\pgfqpoint{2.769264in}{2.528750in}}{\pgfqpoint{2.761450in}{2.520936in}}%
\pgfpathcurveto{\pgfqpoint{2.753637in}{2.513122in}}{\pgfqpoint{2.749247in}{2.502523in}}{\pgfqpoint{2.749247in}{2.491473in}}%
\pgfpathcurveto{\pgfqpoint{2.749247in}{2.480423in}}{\pgfqpoint{2.753637in}{2.469824in}}{\pgfqpoint{2.761450in}{2.462010in}}%
\pgfpathcurveto{\pgfqpoint{2.769264in}{2.454197in}}{\pgfqpoint{2.779863in}{2.449806in}}{\pgfqpoint{2.790913in}{2.449806in}}%
\pgfpathclose%
\pgfusepath{stroke,fill}%
\end{pgfscope}%
\begin{pgfscope}%
\pgfpathrectangle{\pgfqpoint{0.600000in}{0.600000in}}{\pgfqpoint{3.900000in}{3.900000in}}%
\pgfusepath{clip}%
\pgfsetbuttcap%
\pgfsetroundjoin%
\definecolor{currentfill}{rgb}{0.121569,0.466667,0.705882}%
\pgfsetfillcolor{currentfill}%
\pgfsetlinewidth{1.003750pt}%
\definecolor{currentstroke}{rgb}{0.121569,0.466667,0.705882}%
\pgfsetstrokecolor{currentstroke}%
\pgfsetdash{}{0pt}%
\pgfpathmoveto{\pgfqpoint{2.458211in}{3.519869in}}%
\pgfpathcurveto{\pgfqpoint{2.469261in}{3.519869in}}{\pgfqpoint{2.479860in}{3.524259in}}{\pgfqpoint{2.487673in}{3.532073in}}%
\pgfpathcurveto{\pgfqpoint{2.495487in}{3.539887in}}{\pgfqpoint{2.499877in}{3.550486in}}{\pgfqpoint{2.499877in}{3.561536in}}%
\pgfpathcurveto{\pgfqpoint{2.499877in}{3.572586in}}{\pgfqpoint{2.495487in}{3.583185in}}{\pgfqpoint{2.487673in}{3.590999in}}%
\pgfpathcurveto{\pgfqpoint{2.479860in}{3.598812in}}{\pgfqpoint{2.469261in}{3.603203in}}{\pgfqpoint{2.458211in}{3.603203in}}%
\pgfpathcurveto{\pgfqpoint{2.447160in}{3.603203in}}{\pgfqpoint{2.436561in}{3.598812in}}{\pgfqpoint{2.428748in}{3.590999in}}%
\pgfpathcurveto{\pgfqpoint{2.420934in}{3.583185in}}{\pgfqpoint{2.416544in}{3.572586in}}{\pgfqpoint{2.416544in}{3.561536in}}%
\pgfpathcurveto{\pgfqpoint{2.416544in}{3.550486in}}{\pgfqpoint{2.420934in}{3.539887in}}{\pgfqpoint{2.428748in}{3.532073in}}%
\pgfpathcurveto{\pgfqpoint{2.436561in}{3.524259in}}{\pgfqpoint{2.447160in}{3.519869in}}{\pgfqpoint{2.458211in}{3.519869in}}%
\pgfpathclose%
\pgfusepath{stroke,fill}%
\end{pgfscope}%
\begin{pgfscope}%
\pgfpathrectangle{\pgfqpoint{0.600000in}{0.600000in}}{\pgfqpoint{3.900000in}{3.900000in}}%
\pgfusepath{clip}%
\pgfsetbuttcap%
\pgfsetroundjoin%
\definecolor{currentfill}{rgb}{0.121569,0.466667,0.705882}%
\pgfsetfillcolor{currentfill}%
\pgfsetlinewidth{1.003750pt}%
\definecolor{currentstroke}{rgb}{0.121569,0.466667,0.705882}%
\pgfsetstrokecolor{currentstroke}%
\pgfsetdash{}{0pt}%
\pgfpathmoveto{\pgfqpoint{2.499709in}{2.311574in}}%
\pgfpathcurveto{\pgfqpoint{2.510759in}{2.311574in}}{\pgfqpoint{2.521358in}{2.315965in}}{\pgfqpoint{2.529172in}{2.323778in}}%
\pgfpathcurveto{\pgfqpoint{2.536986in}{2.331592in}}{\pgfqpoint{2.541376in}{2.342191in}}{\pgfqpoint{2.541376in}{2.353241in}}%
\pgfpathcurveto{\pgfqpoint{2.541376in}{2.364291in}}{\pgfqpoint{2.536986in}{2.374890in}}{\pgfqpoint{2.529172in}{2.382704in}}%
\pgfpathcurveto{\pgfqpoint{2.521358in}{2.390517in}}{\pgfqpoint{2.510759in}{2.394908in}}{\pgfqpoint{2.499709in}{2.394908in}}%
\pgfpathcurveto{\pgfqpoint{2.488659in}{2.394908in}}{\pgfqpoint{2.478060in}{2.390517in}}{\pgfqpoint{2.470246in}{2.382704in}}%
\pgfpathcurveto{\pgfqpoint{2.462433in}{2.374890in}}{\pgfqpoint{2.458042in}{2.364291in}}{\pgfqpoint{2.458042in}{2.353241in}}%
\pgfpathcurveto{\pgfqpoint{2.458042in}{2.342191in}}{\pgfqpoint{2.462433in}{2.331592in}}{\pgfqpoint{2.470246in}{2.323778in}}%
\pgfpathcurveto{\pgfqpoint{2.478060in}{2.315965in}}{\pgfqpoint{2.488659in}{2.311574in}}{\pgfqpoint{2.499709in}{2.311574in}}%
\pgfpathclose%
\pgfusepath{stroke,fill}%
\end{pgfscope}%
\begin{pgfscope}%
\pgfpathrectangle{\pgfqpoint{0.600000in}{0.600000in}}{\pgfqpoint{3.900000in}{3.900000in}}%
\pgfusepath{clip}%
\pgfsetbuttcap%
\pgfsetroundjoin%
\definecolor{currentfill}{rgb}{0.121569,0.466667,0.705882}%
\pgfsetfillcolor{currentfill}%
\pgfsetlinewidth{1.003750pt}%
\definecolor{currentstroke}{rgb}{0.121569,0.466667,0.705882}%
\pgfsetstrokecolor{currentstroke}%
\pgfsetdash{}{0pt}%
\pgfpathmoveto{\pgfqpoint{1.975856in}{2.694493in}}%
\pgfpathcurveto{\pgfqpoint{1.986906in}{2.694493in}}{\pgfqpoint{1.997505in}{2.698883in}}{\pgfqpoint{2.005318in}{2.706697in}}%
\pgfpathcurveto{\pgfqpoint{2.013132in}{2.714510in}}{\pgfqpoint{2.017522in}{2.725109in}}{\pgfqpoint{2.017522in}{2.736159in}}%
\pgfpathcurveto{\pgfqpoint{2.017522in}{2.747209in}}{\pgfqpoint{2.013132in}{2.757809in}}{\pgfqpoint{2.005318in}{2.765622in}}%
\pgfpathcurveto{\pgfqpoint{1.997505in}{2.773436in}}{\pgfqpoint{1.986906in}{2.777826in}}{\pgfqpoint{1.975856in}{2.777826in}}%
\pgfpathcurveto{\pgfqpoint{1.964805in}{2.777826in}}{\pgfqpoint{1.954206in}{2.773436in}}{\pgfqpoint{1.946393in}{2.765622in}}%
\pgfpathcurveto{\pgfqpoint{1.938579in}{2.757809in}}{\pgfqpoint{1.934189in}{2.747209in}}{\pgfqpoint{1.934189in}{2.736159in}}%
\pgfpathcurveto{\pgfqpoint{1.934189in}{2.725109in}}{\pgfqpoint{1.938579in}{2.714510in}}{\pgfqpoint{1.946393in}{2.706697in}}%
\pgfpathcurveto{\pgfqpoint{1.954206in}{2.698883in}}{\pgfqpoint{1.964805in}{2.694493in}}{\pgfqpoint{1.975856in}{2.694493in}}%
\pgfpathclose%
\pgfusepath{stroke,fill}%
\end{pgfscope}%
\begin{pgfscope}%
\pgfpathrectangle{\pgfqpoint{0.600000in}{0.600000in}}{\pgfqpoint{3.900000in}{3.900000in}}%
\pgfusepath{clip}%
\pgfsetbuttcap%
\pgfsetroundjoin%
\definecolor{currentfill}{rgb}{0.121569,0.466667,0.705882}%
\pgfsetfillcolor{currentfill}%
\pgfsetlinewidth{1.003750pt}%
\definecolor{currentstroke}{rgb}{0.121569,0.466667,0.705882}%
\pgfsetstrokecolor{currentstroke}%
\pgfsetdash{}{0pt}%
\pgfpathmoveto{\pgfqpoint{2.829603in}{3.347742in}}%
\pgfpathcurveto{\pgfqpoint{2.840653in}{3.347742in}}{\pgfqpoint{2.851252in}{3.352132in}}{\pgfqpoint{2.859066in}{3.359946in}}%
\pgfpathcurveto{\pgfqpoint{2.866879in}{3.367760in}}{\pgfqpoint{2.871269in}{3.378359in}}{\pgfqpoint{2.871269in}{3.389409in}}%
\pgfpathcurveto{\pgfqpoint{2.871269in}{3.400459in}}{\pgfqpoint{2.866879in}{3.411058in}}{\pgfqpoint{2.859066in}{3.418872in}}%
\pgfpathcurveto{\pgfqpoint{2.851252in}{3.426685in}}{\pgfqpoint{2.840653in}{3.431075in}}{\pgfqpoint{2.829603in}{3.431075in}}%
\pgfpathcurveto{\pgfqpoint{2.818553in}{3.431075in}}{\pgfqpoint{2.807954in}{3.426685in}}{\pgfqpoint{2.800140in}{3.418872in}}%
\pgfpathcurveto{\pgfqpoint{2.792326in}{3.411058in}}{\pgfqpoint{2.787936in}{3.400459in}}{\pgfqpoint{2.787936in}{3.389409in}}%
\pgfpathcurveto{\pgfqpoint{2.787936in}{3.378359in}}{\pgfqpoint{2.792326in}{3.367760in}}{\pgfqpoint{2.800140in}{3.359946in}}%
\pgfpathcurveto{\pgfqpoint{2.807954in}{3.352132in}}{\pgfqpoint{2.818553in}{3.347742in}}{\pgfqpoint{2.829603in}{3.347742in}}%
\pgfpathclose%
\pgfusepath{stroke,fill}%
\end{pgfscope}%
\begin{pgfscope}%
\pgfpathrectangle{\pgfqpoint{0.600000in}{0.600000in}}{\pgfqpoint{3.900000in}{3.900000in}}%
\pgfusepath{clip}%
\pgfsetbuttcap%
\pgfsetroundjoin%
\definecolor{currentfill}{rgb}{0.121569,0.466667,0.705882}%
\pgfsetfillcolor{currentfill}%
\pgfsetlinewidth{1.003750pt}%
\definecolor{currentstroke}{rgb}{0.121569,0.466667,0.705882}%
\pgfsetstrokecolor{currentstroke}%
\pgfsetdash{}{0pt}%
\pgfpathmoveto{\pgfqpoint{3.049479in}{2.624754in}}%
\pgfpathcurveto{\pgfqpoint{3.060529in}{2.624754in}}{\pgfqpoint{3.071128in}{2.629145in}}{\pgfqpoint{3.078942in}{2.636958in}}%
\pgfpathcurveto{\pgfqpoint{3.086755in}{2.644772in}}{\pgfqpoint{3.091146in}{2.655371in}}{\pgfqpoint{3.091146in}{2.666421in}}%
\pgfpathcurveto{\pgfqpoint{3.091146in}{2.677471in}}{\pgfqpoint{3.086755in}{2.688070in}}{\pgfqpoint{3.078942in}{2.695884in}}%
\pgfpathcurveto{\pgfqpoint{3.071128in}{2.703697in}}{\pgfqpoint{3.060529in}{2.708088in}}{\pgfqpoint{3.049479in}{2.708088in}}%
\pgfpathcurveto{\pgfqpoint{3.038429in}{2.708088in}}{\pgfqpoint{3.027830in}{2.703697in}}{\pgfqpoint{3.020016in}{2.695884in}}%
\pgfpathcurveto{\pgfqpoint{3.012202in}{2.688070in}}{\pgfqpoint{3.007812in}{2.677471in}}{\pgfqpoint{3.007812in}{2.666421in}}%
\pgfpathcurveto{\pgfqpoint{3.007812in}{2.655371in}}{\pgfqpoint{3.012202in}{2.644772in}}{\pgfqpoint{3.020016in}{2.636958in}}%
\pgfpathcurveto{\pgfqpoint{3.027830in}{2.629145in}}{\pgfqpoint{3.038429in}{2.624754in}}{\pgfqpoint{3.049479in}{2.624754in}}%
\pgfpathclose%
\pgfusepath{stroke,fill}%
\end{pgfscope}%
\begin{pgfscope}%
\pgfpathrectangle{\pgfqpoint{0.600000in}{0.600000in}}{\pgfqpoint{3.900000in}{3.900000in}}%
\pgfusepath{clip}%
\pgfsetbuttcap%
\pgfsetroundjoin%
\definecolor{currentfill}{rgb}{0.121569,0.466667,0.705882}%
\pgfsetfillcolor{currentfill}%
\pgfsetlinewidth{1.003750pt}%
\definecolor{currentstroke}{rgb}{0.121569,0.466667,0.705882}%
\pgfsetstrokecolor{currentstroke}%
\pgfsetdash{}{0pt}%
\pgfpathmoveto{\pgfqpoint{2.423311in}{3.203253in}}%
\pgfpathcurveto{\pgfqpoint{2.434361in}{3.203253in}}{\pgfqpoint{2.444960in}{3.207644in}}{\pgfqpoint{2.452774in}{3.215457in}}%
\pgfpathcurveto{\pgfqpoint{2.460587in}{3.223271in}}{\pgfqpoint{2.464978in}{3.233870in}}{\pgfqpoint{2.464978in}{3.244920in}}%
\pgfpathcurveto{\pgfqpoint{2.464978in}{3.255970in}}{\pgfqpoint{2.460587in}{3.266569in}}{\pgfqpoint{2.452774in}{3.274383in}}%
\pgfpathcurveto{\pgfqpoint{2.444960in}{3.282196in}}{\pgfqpoint{2.434361in}{3.286587in}}{\pgfqpoint{2.423311in}{3.286587in}}%
\pgfpathcurveto{\pgfqpoint{2.412261in}{3.286587in}}{\pgfqpoint{2.401662in}{3.282196in}}{\pgfqpoint{2.393848in}{3.274383in}}%
\pgfpathcurveto{\pgfqpoint{2.386034in}{3.266569in}}{\pgfqpoint{2.381644in}{3.255970in}}{\pgfqpoint{2.381644in}{3.244920in}}%
\pgfpathcurveto{\pgfqpoint{2.381644in}{3.233870in}}{\pgfqpoint{2.386034in}{3.223271in}}{\pgfqpoint{2.393848in}{3.215457in}}%
\pgfpathcurveto{\pgfqpoint{2.401662in}{3.207644in}}{\pgfqpoint{2.412261in}{3.203253in}}{\pgfqpoint{2.423311in}{3.203253in}}%
\pgfpathclose%
\pgfusepath{stroke,fill}%
\end{pgfscope}%
\begin{pgfscope}%
\pgfpathrectangle{\pgfqpoint{0.600000in}{0.600000in}}{\pgfqpoint{3.900000in}{3.900000in}}%
\pgfusepath{clip}%
\pgfsetbuttcap%
\pgfsetroundjoin%
\definecolor{currentfill}{rgb}{0.121569,0.466667,0.705882}%
\pgfsetfillcolor{currentfill}%
\pgfsetlinewidth{1.003750pt}%
\definecolor{currentstroke}{rgb}{0.121569,0.466667,0.705882}%
\pgfsetstrokecolor{currentstroke}%
\pgfsetdash{}{0pt}%
\pgfpathmoveto{\pgfqpoint{2.469201in}{2.546223in}}%
\pgfpathcurveto{\pgfqpoint{2.480251in}{2.546223in}}{\pgfqpoint{2.490850in}{2.550613in}}{\pgfqpoint{2.498663in}{2.558427in}}%
\pgfpathcurveto{\pgfqpoint{2.506477in}{2.566241in}}{\pgfqpoint{2.510867in}{2.576840in}}{\pgfqpoint{2.510867in}{2.587890in}}%
\pgfpathcurveto{\pgfqpoint{2.510867in}{2.598940in}}{\pgfqpoint{2.506477in}{2.609539in}}{\pgfqpoint{2.498663in}{2.617353in}}%
\pgfpathcurveto{\pgfqpoint{2.490850in}{2.625166in}}{\pgfqpoint{2.480251in}{2.629556in}}{\pgfqpoint{2.469201in}{2.629556in}}%
\pgfpathcurveto{\pgfqpoint{2.458151in}{2.629556in}}{\pgfqpoint{2.447552in}{2.625166in}}{\pgfqpoint{2.439738in}{2.617353in}}%
\pgfpathcurveto{\pgfqpoint{2.431924in}{2.609539in}}{\pgfqpoint{2.427534in}{2.598940in}}{\pgfqpoint{2.427534in}{2.587890in}}%
\pgfpathcurveto{\pgfqpoint{2.427534in}{2.576840in}}{\pgfqpoint{2.431924in}{2.566241in}}{\pgfqpoint{2.439738in}{2.558427in}}%
\pgfpathcurveto{\pgfqpoint{2.447552in}{2.550613in}}{\pgfqpoint{2.458151in}{2.546223in}}{\pgfqpoint{2.469201in}{2.546223in}}%
\pgfpathclose%
\pgfusepath{stroke,fill}%
\end{pgfscope}%
\begin{pgfscope}%
\pgfpathrectangle{\pgfqpoint{0.600000in}{0.600000in}}{\pgfqpoint{3.900000in}{3.900000in}}%
\pgfusepath{clip}%
\pgfsetbuttcap%
\pgfsetroundjoin%
\definecolor{currentfill}{rgb}{0.121569,0.466667,0.705882}%
\pgfsetfillcolor{currentfill}%
\pgfsetlinewidth{1.003750pt}%
\definecolor{currentstroke}{rgb}{0.121569,0.466667,0.705882}%
\pgfsetstrokecolor{currentstroke}%
\pgfsetdash{}{0pt}%
\pgfpathmoveto{\pgfqpoint{2.954975in}{2.685123in}}%
\pgfpathcurveto{\pgfqpoint{2.966025in}{2.685123in}}{\pgfqpoint{2.976624in}{2.689513in}}{\pgfqpoint{2.984437in}{2.697327in}}%
\pgfpathcurveto{\pgfqpoint{2.992251in}{2.705140in}}{\pgfqpoint{2.996641in}{2.715739in}}{\pgfqpoint{2.996641in}{2.726789in}}%
\pgfpathcurveto{\pgfqpoint{2.996641in}{2.737839in}}{\pgfqpoint{2.992251in}{2.748439in}}{\pgfqpoint{2.984437in}{2.756252in}}%
\pgfpathcurveto{\pgfqpoint{2.976624in}{2.764066in}}{\pgfqpoint{2.966025in}{2.768456in}}{\pgfqpoint{2.954975in}{2.768456in}}%
\pgfpathcurveto{\pgfqpoint{2.943925in}{2.768456in}}{\pgfqpoint{2.933325in}{2.764066in}}{\pgfqpoint{2.925512in}{2.756252in}}%
\pgfpathcurveto{\pgfqpoint{2.917698in}{2.748439in}}{\pgfqpoint{2.913308in}{2.737839in}}{\pgfqpoint{2.913308in}{2.726789in}}%
\pgfpathcurveto{\pgfqpoint{2.913308in}{2.715739in}}{\pgfqpoint{2.917698in}{2.705140in}}{\pgfqpoint{2.925512in}{2.697327in}}%
\pgfpathcurveto{\pgfqpoint{2.933325in}{2.689513in}}{\pgfqpoint{2.943925in}{2.685123in}}{\pgfqpoint{2.954975in}{2.685123in}}%
\pgfpathclose%
\pgfusepath{stroke,fill}%
\end{pgfscope}%
\begin{pgfscope}%
\pgfpathrectangle{\pgfqpoint{0.600000in}{0.600000in}}{\pgfqpoint{3.900000in}{3.900000in}}%
\pgfusepath{clip}%
\pgfsetbuttcap%
\pgfsetroundjoin%
\definecolor{currentfill}{rgb}{0.121569,0.466667,0.705882}%
\pgfsetfillcolor{currentfill}%
\pgfsetlinewidth{1.003750pt}%
\definecolor{currentstroke}{rgb}{0.121569,0.466667,0.705882}%
\pgfsetstrokecolor{currentstroke}%
\pgfsetdash{}{0pt}%
\pgfpathmoveto{\pgfqpoint{2.543162in}{2.897787in}}%
\pgfpathcurveto{\pgfqpoint{2.554212in}{2.897787in}}{\pgfqpoint{2.564811in}{2.902177in}}{\pgfqpoint{2.572625in}{2.909991in}}%
\pgfpathcurveto{\pgfqpoint{2.580438in}{2.917804in}}{\pgfqpoint{2.584829in}{2.928403in}}{\pgfqpoint{2.584829in}{2.939454in}}%
\pgfpathcurveto{\pgfqpoint{2.584829in}{2.950504in}}{\pgfqpoint{2.580438in}{2.961103in}}{\pgfqpoint{2.572625in}{2.968916in}}%
\pgfpathcurveto{\pgfqpoint{2.564811in}{2.976730in}}{\pgfqpoint{2.554212in}{2.981120in}}{\pgfqpoint{2.543162in}{2.981120in}}%
\pgfpathcurveto{\pgfqpoint{2.532112in}{2.981120in}}{\pgfqpoint{2.521513in}{2.976730in}}{\pgfqpoint{2.513699in}{2.968916in}}%
\pgfpathcurveto{\pgfqpoint{2.505886in}{2.961103in}}{\pgfqpoint{2.501495in}{2.950504in}}{\pgfqpoint{2.501495in}{2.939454in}}%
\pgfpathcurveto{\pgfqpoint{2.501495in}{2.928403in}}{\pgfqpoint{2.505886in}{2.917804in}}{\pgfqpoint{2.513699in}{2.909991in}}%
\pgfpathcurveto{\pgfqpoint{2.521513in}{2.902177in}}{\pgfqpoint{2.532112in}{2.897787in}}{\pgfqpoint{2.543162in}{2.897787in}}%
\pgfpathclose%
\pgfusepath{stroke,fill}%
\end{pgfscope}%
\begin{pgfscope}%
\pgfpathrectangle{\pgfqpoint{0.600000in}{0.600000in}}{\pgfqpoint{3.900000in}{3.900000in}}%
\pgfusepath{clip}%
\pgfsetbuttcap%
\pgfsetroundjoin%
\definecolor{currentfill}{rgb}{0.121569,0.466667,0.705882}%
\pgfsetfillcolor{currentfill}%
\pgfsetlinewidth{1.003750pt}%
\definecolor{currentstroke}{rgb}{0.121569,0.466667,0.705882}%
\pgfsetstrokecolor{currentstroke}%
\pgfsetdash{}{0pt}%
\pgfpathmoveto{\pgfqpoint{2.511432in}{2.119748in}}%
\pgfpathcurveto{\pgfqpoint{2.522482in}{2.119748in}}{\pgfqpoint{2.533081in}{2.124138in}}{\pgfqpoint{2.540895in}{2.131952in}}%
\pgfpathcurveto{\pgfqpoint{2.548709in}{2.139765in}}{\pgfqpoint{2.553099in}{2.150364in}}{\pgfqpoint{2.553099in}{2.161414in}}%
\pgfpathcurveto{\pgfqpoint{2.553099in}{2.172465in}}{\pgfqpoint{2.548709in}{2.183064in}}{\pgfqpoint{2.540895in}{2.190877in}}%
\pgfpathcurveto{\pgfqpoint{2.533081in}{2.198691in}}{\pgfqpoint{2.522482in}{2.203081in}}{\pgfqpoint{2.511432in}{2.203081in}}%
\pgfpathcurveto{\pgfqpoint{2.500382in}{2.203081in}}{\pgfqpoint{2.489783in}{2.198691in}}{\pgfqpoint{2.481969in}{2.190877in}}%
\pgfpathcurveto{\pgfqpoint{2.474156in}{2.183064in}}{\pgfqpoint{2.469765in}{2.172465in}}{\pgfqpoint{2.469765in}{2.161414in}}%
\pgfpathcurveto{\pgfqpoint{2.469765in}{2.150364in}}{\pgfqpoint{2.474156in}{2.139765in}}{\pgfqpoint{2.481969in}{2.131952in}}%
\pgfpathcurveto{\pgfqpoint{2.489783in}{2.124138in}}{\pgfqpoint{2.500382in}{2.119748in}}{\pgfqpoint{2.511432in}{2.119748in}}%
\pgfpathclose%
\pgfusepath{stroke,fill}%
\end{pgfscope}%
\begin{pgfscope}%
\pgfpathrectangle{\pgfqpoint{0.600000in}{0.600000in}}{\pgfqpoint{3.900000in}{3.900000in}}%
\pgfusepath{clip}%
\pgfsetbuttcap%
\pgfsetroundjoin%
\definecolor{currentfill}{rgb}{0.121569,0.466667,0.705882}%
\pgfsetfillcolor{currentfill}%
\pgfsetlinewidth{1.003750pt}%
\definecolor{currentstroke}{rgb}{0.121569,0.466667,0.705882}%
\pgfsetstrokecolor{currentstroke}%
\pgfsetdash{}{0pt}%
\pgfpathmoveto{\pgfqpoint{2.757405in}{3.329076in}}%
\pgfpathcurveto{\pgfqpoint{2.768455in}{3.329076in}}{\pgfqpoint{2.779054in}{3.333466in}}{\pgfqpoint{2.786867in}{3.341280in}}%
\pgfpathcurveto{\pgfqpoint{2.794681in}{3.349094in}}{\pgfqpoint{2.799071in}{3.359693in}}{\pgfqpoint{2.799071in}{3.370743in}}%
\pgfpathcurveto{\pgfqpoint{2.799071in}{3.381793in}}{\pgfqpoint{2.794681in}{3.392392in}}{\pgfqpoint{2.786867in}{3.400205in}}%
\pgfpathcurveto{\pgfqpoint{2.779054in}{3.408019in}}{\pgfqpoint{2.768455in}{3.412409in}}{\pgfqpoint{2.757405in}{3.412409in}}%
\pgfpathcurveto{\pgfqpoint{2.746354in}{3.412409in}}{\pgfqpoint{2.735755in}{3.408019in}}{\pgfqpoint{2.727942in}{3.400205in}}%
\pgfpathcurveto{\pgfqpoint{2.720128in}{3.392392in}}{\pgfqpoint{2.715738in}{3.381793in}}{\pgfqpoint{2.715738in}{3.370743in}}%
\pgfpathcurveto{\pgfqpoint{2.715738in}{3.359693in}}{\pgfqpoint{2.720128in}{3.349094in}}{\pgfqpoint{2.727942in}{3.341280in}}%
\pgfpathcurveto{\pgfqpoint{2.735755in}{3.333466in}}{\pgfqpoint{2.746354in}{3.329076in}}{\pgfqpoint{2.757405in}{3.329076in}}%
\pgfpathclose%
\pgfusepath{stroke,fill}%
\end{pgfscope}%
\begin{pgfscope}%
\pgfpathrectangle{\pgfqpoint{0.600000in}{0.600000in}}{\pgfqpoint{3.900000in}{3.900000in}}%
\pgfusepath{clip}%
\pgfsetbuttcap%
\pgfsetroundjoin%
\definecolor{currentfill}{rgb}{0.121569,0.466667,0.705882}%
\pgfsetfillcolor{currentfill}%
\pgfsetlinewidth{1.003750pt}%
\definecolor{currentstroke}{rgb}{0.121569,0.466667,0.705882}%
\pgfsetstrokecolor{currentstroke}%
\pgfsetdash{}{0pt}%
\pgfpathmoveto{\pgfqpoint{1.931718in}{2.162387in}}%
\pgfpathcurveto{\pgfqpoint{1.942768in}{2.162387in}}{\pgfqpoint{1.953367in}{2.166777in}}{\pgfqpoint{1.961181in}{2.174590in}}%
\pgfpathcurveto{\pgfqpoint{1.968994in}{2.182404in}}{\pgfqpoint{1.973384in}{2.193003in}}{\pgfqpoint{1.973384in}{2.204053in}}%
\pgfpathcurveto{\pgfqpoint{1.973384in}{2.215103in}}{\pgfqpoint{1.968994in}{2.225702in}}{\pgfqpoint{1.961181in}{2.233516in}}%
\pgfpathcurveto{\pgfqpoint{1.953367in}{2.241330in}}{\pgfqpoint{1.942768in}{2.245720in}}{\pgfqpoint{1.931718in}{2.245720in}}%
\pgfpathcurveto{\pgfqpoint{1.920668in}{2.245720in}}{\pgfqpoint{1.910069in}{2.241330in}}{\pgfqpoint{1.902255in}{2.233516in}}%
\pgfpathcurveto{\pgfqpoint{1.894441in}{2.225702in}}{\pgfqpoint{1.890051in}{2.215103in}}{\pgfqpoint{1.890051in}{2.204053in}}%
\pgfpathcurveto{\pgfqpoint{1.890051in}{2.193003in}}{\pgfqpoint{1.894441in}{2.182404in}}{\pgfqpoint{1.902255in}{2.174590in}}%
\pgfpathcurveto{\pgfqpoint{1.910069in}{2.166777in}}{\pgfqpoint{1.920668in}{2.162387in}}{\pgfqpoint{1.931718in}{2.162387in}}%
\pgfpathclose%
\pgfusepath{stroke,fill}%
\end{pgfscope}%
\begin{pgfscope}%
\pgfpathrectangle{\pgfqpoint{0.600000in}{0.600000in}}{\pgfqpoint{3.900000in}{3.900000in}}%
\pgfusepath{clip}%
\pgfsetbuttcap%
\pgfsetroundjoin%
\definecolor{currentfill}{rgb}{0.121569,0.466667,0.705882}%
\pgfsetfillcolor{currentfill}%
\pgfsetlinewidth{1.003750pt}%
\definecolor{currentstroke}{rgb}{0.121569,0.466667,0.705882}%
\pgfsetstrokecolor{currentstroke}%
\pgfsetdash{}{0pt}%
\pgfpathmoveto{\pgfqpoint{1.815101in}{2.909307in}}%
\pgfpathcurveto{\pgfqpoint{1.826152in}{2.909307in}}{\pgfqpoint{1.836751in}{2.913697in}}{\pgfqpoint{1.844564in}{2.921511in}}%
\pgfpathcurveto{\pgfqpoint{1.852378in}{2.929325in}}{\pgfqpoint{1.856768in}{2.939924in}}{\pgfqpoint{1.856768in}{2.950974in}}%
\pgfpathcurveto{\pgfqpoint{1.856768in}{2.962024in}}{\pgfqpoint{1.852378in}{2.972623in}}{\pgfqpoint{1.844564in}{2.980437in}}%
\pgfpathcurveto{\pgfqpoint{1.836751in}{2.988250in}}{\pgfqpoint{1.826152in}{2.992641in}}{\pgfqpoint{1.815101in}{2.992641in}}%
\pgfpathcurveto{\pgfqpoint{1.804051in}{2.992641in}}{\pgfqpoint{1.793452in}{2.988250in}}{\pgfqpoint{1.785639in}{2.980437in}}%
\pgfpathcurveto{\pgfqpoint{1.777825in}{2.972623in}}{\pgfqpoint{1.773435in}{2.962024in}}{\pgfqpoint{1.773435in}{2.950974in}}%
\pgfpathcurveto{\pgfqpoint{1.773435in}{2.939924in}}{\pgfqpoint{1.777825in}{2.929325in}}{\pgfqpoint{1.785639in}{2.921511in}}%
\pgfpathcurveto{\pgfqpoint{1.793452in}{2.913697in}}{\pgfqpoint{1.804051in}{2.909307in}}{\pgfqpoint{1.815101in}{2.909307in}}%
\pgfpathclose%
\pgfusepath{stroke,fill}%
\end{pgfscope}%
\begin{pgfscope}%
\pgfpathrectangle{\pgfqpoint{0.600000in}{0.600000in}}{\pgfqpoint{3.900000in}{3.900000in}}%
\pgfusepath{clip}%
\pgfsetbuttcap%
\pgfsetroundjoin%
\definecolor{currentfill}{rgb}{0.121569,0.466667,0.705882}%
\pgfsetfillcolor{currentfill}%
\pgfsetlinewidth{1.003750pt}%
\definecolor{currentstroke}{rgb}{0.121569,0.466667,0.705882}%
\pgfsetstrokecolor{currentstroke}%
\pgfsetdash{}{0pt}%
\pgfpathmoveto{\pgfqpoint{2.501207in}{3.121290in}}%
\pgfpathcurveto{\pgfqpoint{2.512257in}{3.121290in}}{\pgfqpoint{2.522856in}{3.125680in}}{\pgfqpoint{2.530669in}{3.133493in}}%
\pgfpathcurveto{\pgfqpoint{2.538483in}{3.141307in}}{\pgfqpoint{2.542873in}{3.151906in}}{\pgfqpoint{2.542873in}{3.162956in}}%
\pgfpathcurveto{\pgfqpoint{2.542873in}{3.174006in}}{\pgfqpoint{2.538483in}{3.184605in}}{\pgfqpoint{2.530669in}{3.192419in}}%
\pgfpathcurveto{\pgfqpoint{2.522856in}{3.200233in}}{\pgfqpoint{2.512257in}{3.204623in}}{\pgfqpoint{2.501207in}{3.204623in}}%
\pgfpathcurveto{\pgfqpoint{2.490157in}{3.204623in}}{\pgfqpoint{2.479557in}{3.200233in}}{\pgfqpoint{2.471744in}{3.192419in}}%
\pgfpathcurveto{\pgfqpoint{2.463930in}{3.184605in}}{\pgfqpoint{2.459540in}{3.174006in}}{\pgfqpoint{2.459540in}{3.162956in}}%
\pgfpathcurveto{\pgfqpoint{2.459540in}{3.151906in}}{\pgfqpoint{2.463930in}{3.141307in}}{\pgfqpoint{2.471744in}{3.133493in}}%
\pgfpathcurveto{\pgfqpoint{2.479557in}{3.125680in}}{\pgfqpoint{2.490157in}{3.121290in}}{\pgfqpoint{2.501207in}{3.121290in}}%
\pgfpathclose%
\pgfusepath{stroke,fill}%
\end{pgfscope}%
\begin{pgfscope}%
\pgfpathrectangle{\pgfqpoint{0.600000in}{0.600000in}}{\pgfqpoint{3.900000in}{3.900000in}}%
\pgfusepath{clip}%
\pgfsetbuttcap%
\pgfsetroundjoin%
\definecolor{currentfill}{rgb}{0.121569,0.466667,0.705882}%
\pgfsetfillcolor{currentfill}%
\pgfsetlinewidth{1.003750pt}%
\definecolor{currentstroke}{rgb}{0.121569,0.466667,0.705882}%
\pgfsetstrokecolor{currentstroke}%
\pgfsetdash{}{0pt}%
\pgfpathmoveto{\pgfqpoint{1.809636in}{2.079536in}}%
\pgfpathcurveto{\pgfqpoint{1.820686in}{2.079536in}}{\pgfqpoint{1.831285in}{2.083926in}}{\pgfqpoint{1.839099in}{2.091740in}}%
\pgfpathcurveto{\pgfqpoint{1.846913in}{2.099554in}}{\pgfqpoint{1.851303in}{2.110153in}}{\pgfqpoint{1.851303in}{2.121203in}}%
\pgfpathcurveto{\pgfqpoint{1.851303in}{2.132253in}}{\pgfqpoint{1.846913in}{2.142852in}}{\pgfqpoint{1.839099in}{2.150666in}}%
\pgfpathcurveto{\pgfqpoint{1.831285in}{2.158479in}}{\pgfqpoint{1.820686in}{2.162870in}}{\pgfqpoint{1.809636in}{2.162870in}}%
\pgfpathcurveto{\pgfqpoint{1.798586in}{2.162870in}}{\pgfqpoint{1.787987in}{2.158479in}}{\pgfqpoint{1.780173in}{2.150666in}}%
\pgfpathcurveto{\pgfqpoint{1.772360in}{2.142852in}}{\pgfqpoint{1.767970in}{2.132253in}}{\pgfqpoint{1.767970in}{2.121203in}}%
\pgfpathcurveto{\pgfqpoint{1.767970in}{2.110153in}}{\pgfqpoint{1.772360in}{2.099554in}}{\pgfqpoint{1.780173in}{2.091740in}}%
\pgfpathcurveto{\pgfqpoint{1.787987in}{2.083926in}}{\pgfqpoint{1.798586in}{2.079536in}}{\pgfqpoint{1.809636in}{2.079536in}}%
\pgfpathclose%
\pgfusepath{stroke,fill}%
\end{pgfscope}%
\begin{pgfscope}%
\pgfpathrectangle{\pgfqpoint{0.600000in}{0.600000in}}{\pgfqpoint{3.900000in}{3.900000in}}%
\pgfusepath{clip}%
\pgfsetbuttcap%
\pgfsetroundjoin%
\definecolor{currentfill}{rgb}{0.121569,0.466667,0.705882}%
\pgfsetfillcolor{currentfill}%
\pgfsetlinewidth{1.003750pt}%
\definecolor{currentstroke}{rgb}{0.121569,0.466667,0.705882}%
\pgfsetstrokecolor{currentstroke}%
\pgfsetdash{}{0pt}%
\pgfpathmoveto{\pgfqpoint{3.290910in}{2.710987in}}%
\pgfpathcurveto{\pgfqpoint{3.301960in}{2.710987in}}{\pgfqpoint{3.312559in}{2.715378in}}{\pgfqpoint{3.320373in}{2.723191in}}%
\pgfpathcurveto{\pgfqpoint{3.328187in}{2.731005in}}{\pgfqpoint{3.332577in}{2.741604in}}{\pgfqpoint{3.332577in}{2.752654in}}%
\pgfpathcurveto{\pgfqpoint{3.332577in}{2.763704in}}{\pgfqpoint{3.328187in}{2.774303in}}{\pgfqpoint{3.320373in}{2.782117in}}%
\pgfpathcurveto{\pgfqpoint{3.312559in}{2.789930in}}{\pgfqpoint{3.301960in}{2.794321in}}{\pgfqpoint{3.290910in}{2.794321in}}%
\pgfpathcurveto{\pgfqpoint{3.279860in}{2.794321in}}{\pgfqpoint{3.269261in}{2.789930in}}{\pgfqpoint{3.261447in}{2.782117in}}%
\pgfpathcurveto{\pgfqpoint{3.253634in}{2.774303in}}{\pgfqpoint{3.249244in}{2.763704in}}{\pgfqpoint{3.249244in}{2.752654in}}%
\pgfpathcurveto{\pgfqpoint{3.249244in}{2.741604in}}{\pgfqpoint{3.253634in}{2.731005in}}{\pgfqpoint{3.261447in}{2.723191in}}%
\pgfpathcurveto{\pgfqpoint{3.269261in}{2.715378in}}{\pgfqpoint{3.279860in}{2.710987in}}{\pgfqpoint{3.290910in}{2.710987in}}%
\pgfpathclose%
\pgfusepath{stroke,fill}%
\end{pgfscope}%
\begin{pgfscope}%
\pgfpathrectangle{\pgfqpoint{0.600000in}{0.600000in}}{\pgfqpoint{3.900000in}{3.900000in}}%
\pgfusepath{clip}%
\pgfsetbuttcap%
\pgfsetroundjoin%
\definecolor{currentfill}{rgb}{0.121569,0.466667,0.705882}%
\pgfsetfillcolor{currentfill}%
\pgfsetlinewidth{1.003750pt}%
\definecolor{currentstroke}{rgb}{0.121569,0.466667,0.705882}%
\pgfsetstrokecolor{currentstroke}%
\pgfsetdash{}{0pt}%
\pgfpathmoveto{\pgfqpoint{1.625868in}{3.347244in}}%
\pgfpathcurveto{\pgfqpoint{1.636918in}{3.347244in}}{\pgfqpoint{1.647517in}{3.351635in}}{\pgfqpoint{1.655330in}{3.359448in}}%
\pgfpathcurveto{\pgfqpoint{1.663144in}{3.367262in}}{\pgfqpoint{1.667534in}{3.377861in}}{\pgfqpoint{1.667534in}{3.388911in}}%
\pgfpathcurveto{\pgfqpoint{1.667534in}{3.399961in}}{\pgfqpoint{1.663144in}{3.410560in}}{\pgfqpoint{1.655330in}{3.418374in}}%
\pgfpathcurveto{\pgfqpoint{1.647517in}{3.426187in}}{\pgfqpoint{1.636918in}{3.430578in}}{\pgfqpoint{1.625868in}{3.430578in}}%
\pgfpathcurveto{\pgfqpoint{1.614817in}{3.430578in}}{\pgfqpoint{1.604218in}{3.426187in}}{\pgfqpoint{1.596405in}{3.418374in}}%
\pgfpathcurveto{\pgfqpoint{1.588591in}{3.410560in}}{\pgfqpoint{1.584201in}{3.399961in}}{\pgfqpoint{1.584201in}{3.388911in}}%
\pgfpathcurveto{\pgfqpoint{1.584201in}{3.377861in}}{\pgfqpoint{1.588591in}{3.367262in}}{\pgfqpoint{1.596405in}{3.359448in}}%
\pgfpathcurveto{\pgfqpoint{1.604218in}{3.351635in}}{\pgfqpoint{1.614817in}{3.347244in}}{\pgfqpoint{1.625868in}{3.347244in}}%
\pgfpathclose%
\pgfusepath{stroke,fill}%
\end{pgfscope}%
\begin{pgfscope}%
\pgfpathrectangle{\pgfqpoint{0.600000in}{0.600000in}}{\pgfqpoint{3.900000in}{3.900000in}}%
\pgfusepath{clip}%
\pgfsetbuttcap%
\pgfsetroundjoin%
\definecolor{currentfill}{rgb}{0.121569,0.466667,0.705882}%
\pgfsetfillcolor{currentfill}%
\pgfsetlinewidth{1.003750pt}%
\definecolor{currentstroke}{rgb}{0.121569,0.466667,0.705882}%
\pgfsetstrokecolor{currentstroke}%
\pgfsetdash{}{0pt}%
\pgfpathmoveto{\pgfqpoint{2.095572in}{1.694221in}}%
\pgfpathcurveto{\pgfqpoint{2.106622in}{1.694221in}}{\pgfqpoint{2.117221in}{1.698612in}}{\pgfqpoint{2.125034in}{1.706425in}}%
\pgfpathcurveto{\pgfqpoint{2.132848in}{1.714239in}}{\pgfqpoint{2.137238in}{1.724838in}}{\pgfqpoint{2.137238in}{1.735888in}}%
\pgfpathcurveto{\pgfqpoint{2.137238in}{1.746938in}}{\pgfqpoint{2.132848in}{1.757537in}}{\pgfqpoint{2.125034in}{1.765351in}}%
\pgfpathcurveto{\pgfqpoint{2.117221in}{1.773164in}}{\pgfqpoint{2.106622in}{1.777555in}}{\pgfqpoint{2.095572in}{1.777555in}}%
\pgfpathcurveto{\pgfqpoint{2.084522in}{1.777555in}}{\pgfqpoint{2.073923in}{1.773164in}}{\pgfqpoint{2.066109in}{1.765351in}}%
\pgfpathcurveto{\pgfqpoint{2.058295in}{1.757537in}}{\pgfqpoint{2.053905in}{1.746938in}}{\pgfqpoint{2.053905in}{1.735888in}}%
\pgfpathcurveto{\pgfqpoint{2.053905in}{1.724838in}}{\pgfqpoint{2.058295in}{1.714239in}}{\pgfqpoint{2.066109in}{1.706425in}}%
\pgfpathcurveto{\pgfqpoint{2.073923in}{1.698612in}}{\pgfqpoint{2.084522in}{1.694221in}}{\pgfqpoint{2.095572in}{1.694221in}}%
\pgfpathclose%
\pgfusepath{stroke,fill}%
\end{pgfscope}%
\begin{pgfscope}%
\pgfpathrectangle{\pgfqpoint{0.600000in}{0.600000in}}{\pgfqpoint{3.900000in}{3.900000in}}%
\pgfusepath{clip}%
\pgfsetbuttcap%
\pgfsetroundjoin%
\definecolor{currentfill}{rgb}{0.121569,0.466667,0.705882}%
\pgfsetfillcolor{currentfill}%
\pgfsetlinewidth{1.003750pt}%
\definecolor{currentstroke}{rgb}{0.121569,0.466667,0.705882}%
\pgfsetstrokecolor{currentstroke}%
\pgfsetdash{}{0pt}%
\pgfpathmoveto{\pgfqpoint{1.838331in}{2.868886in}}%
\pgfpathcurveto{\pgfqpoint{1.849382in}{2.868886in}}{\pgfqpoint{1.859981in}{2.873276in}}{\pgfqpoint{1.867794in}{2.881090in}}%
\pgfpathcurveto{\pgfqpoint{1.875608in}{2.888903in}}{\pgfqpoint{1.879998in}{2.899502in}}{\pgfqpoint{1.879998in}{2.910553in}}%
\pgfpathcurveto{\pgfqpoint{1.879998in}{2.921603in}}{\pgfqpoint{1.875608in}{2.932202in}}{\pgfqpoint{1.867794in}{2.940015in}}%
\pgfpathcurveto{\pgfqpoint{1.859981in}{2.947829in}}{\pgfqpoint{1.849382in}{2.952219in}}{\pgfqpoint{1.838331in}{2.952219in}}%
\pgfpathcurveto{\pgfqpoint{1.827281in}{2.952219in}}{\pgfqpoint{1.816682in}{2.947829in}}{\pgfqpoint{1.808869in}{2.940015in}}%
\pgfpathcurveto{\pgfqpoint{1.801055in}{2.932202in}}{\pgfqpoint{1.796665in}{2.921603in}}{\pgfqpoint{1.796665in}{2.910553in}}%
\pgfpathcurveto{\pgfqpoint{1.796665in}{2.899502in}}{\pgfqpoint{1.801055in}{2.888903in}}{\pgfqpoint{1.808869in}{2.881090in}}%
\pgfpathcurveto{\pgfqpoint{1.816682in}{2.873276in}}{\pgfqpoint{1.827281in}{2.868886in}}{\pgfqpoint{1.838331in}{2.868886in}}%
\pgfpathclose%
\pgfusepath{stroke,fill}%
\end{pgfscope}%
\begin{pgfscope}%
\pgfpathrectangle{\pgfqpoint{0.600000in}{0.600000in}}{\pgfqpoint{3.900000in}{3.900000in}}%
\pgfusepath{clip}%
\pgfsetbuttcap%
\pgfsetroundjoin%
\definecolor{currentfill}{rgb}{0.121569,0.466667,0.705882}%
\pgfsetfillcolor{currentfill}%
\pgfsetlinewidth{1.003750pt}%
\definecolor{currentstroke}{rgb}{0.121569,0.466667,0.705882}%
\pgfsetstrokecolor{currentstroke}%
\pgfsetdash{}{0pt}%
\pgfpathmoveto{\pgfqpoint{3.118634in}{1.641119in}}%
\pgfpathcurveto{\pgfqpoint{3.129684in}{1.641119in}}{\pgfqpoint{3.140283in}{1.645509in}}{\pgfqpoint{3.148097in}{1.653323in}}%
\pgfpathcurveto{\pgfqpoint{3.155910in}{1.661136in}}{\pgfqpoint{3.160301in}{1.671735in}}{\pgfqpoint{3.160301in}{1.682786in}}%
\pgfpathcurveto{\pgfqpoint{3.160301in}{1.693836in}}{\pgfqpoint{3.155910in}{1.704435in}}{\pgfqpoint{3.148097in}{1.712248in}}%
\pgfpathcurveto{\pgfqpoint{3.140283in}{1.720062in}}{\pgfqpoint{3.129684in}{1.724452in}}{\pgfqpoint{3.118634in}{1.724452in}}%
\pgfpathcurveto{\pgfqpoint{3.107584in}{1.724452in}}{\pgfqpoint{3.096985in}{1.720062in}}{\pgfqpoint{3.089171in}{1.712248in}}%
\pgfpathcurveto{\pgfqpoint{3.081357in}{1.704435in}}{\pgfqpoint{3.076967in}{1.693836in}}{\pgfqpoint{3.076967in}{1.682786in}}%
\pgfpathcurveto{\pgfqpoint{3.076967in}{1.671735in}}{\pgfqpoint{3.081357in}{1.661136in}}{\pgfqpoint{3.089171in}{1.653323in}}%
\pgfpathcurveto{\pgfqpoint{3.096985in}{1.645509in}}{\pgfqpoint{3.107584in}{1.641119in}}{\pgfqpoint{3.118634in}{1.641119in}}%
\pgfpathclose%
\pgfusepath{stroke,fill}%
\end{pgfscope}%
\begin{pgfscope}%
\pgfpathrectangle{\pgfqpoint{0.600000in}{0.600000in}}{\pgfqpoint{3.900000in}{3.900000in}}%
\pgfusepath{clip}%
\pgfsetbuttcap%
\pgfsetroundjoin%
\definecolor{currentfill}{rgb}{0.121569,0.466667,0.705882}%
\pgfsetfillcolor{currentfill}%
\pgfsetlinewidth{1.003750pt}%
\definecolor{currentstroke}{rgb}{0.121569,0.466667,0.705882}%
\pgfsetstrokecolor{currentstroke}%
\pgfsetdash{}{0pt}%
\pgfpathmoveto{\pgfqpoint{2.249026in}{2.533771in}}%
\pgfpathcurveto{\pgfqpoint{2.260076in}{2.533771in}}{\pgfqpoint{2.270675in}{2.538161in}}{\pgfqpoint{2.278488in}{2.545975in}}%
\pgfpathcurveto{\pgfqpoint{2.286302in}{2.553788in}}{\pgfqpoint{2.290692in}{2.564387in}}{\pgfqpoint{2.290692in}{2.575438in}}%
\pgfpathcurveto{\pgfqpoint{2.290692in}{2.586488in}}{\pgfqpoint{2.286302in}{2.597087in}}{\pgfqpoint{2.278488in}{2.604900in}}%
\pgfpathcurveto{\pgfqpoint{2.270675in}{2.612714in}}{\pgfqpoint{2.260076in}{2.617104in}}{\pgfqpoint{2.249026in}{2.617104in}}%
\pgfpathcurveto{\pgfqpoint{2.237976in}{2.617104in}}{\pgfqpoint{2.227377in}{2.612714in}}{\pgfqpoint{2.219563in}{2.604900in}}%
\pgfpathcurveto{\pgfqpoint{2.211749in}{2.597087in}}{\pgfqpoint{2.207359in}{2.586488in}}{\pgfqpoint{2.207359in}{2.575438in}}%
\pgfpathcurveto{\pgfqpoint{2.207359in}{2.564387in}}{\pgfqpoint{2.211749in}{2.553788in}}{\pgfqpoint{2.219563in}{2.545975in}}%
\pgfpathcurveto{\pgfqpoint{2.227377in}{2.538161in}}{\pgfqpoint{2.237976in}{2.533771in}}{\pgfqpoint{2.249026in}{2.533771in}}%
\pgfpathclose%
\pgfusepath{stroke,fill}%
\end{pgfscope}%
\begin{pgfscope}%
\pgfpathrectangle{\pgfqpoint{0.600000in}{0.600000in}}{\pgfqpoint{3.900000in}{3.900000in}}%
\pgfusepath{clip}%
\pgfsetbuttcap%
\pgfsetroundjoin%
\definecolor{currentfill}{rgb}{0.121569,0.466667,0.705882}%
\pgfsetfillcolor{currentfill}%
\pgfsetlinewidth{1.003750pt}%
\definecolor{currentstroke}{rgb}{0.121569,0.466667,0.705882}%
\pgfsetstrokecolor{currentstroke}%
\pgfsetdash{}{0pt}%
\pgfpathmoveto{\pgfqpoint{2.258523in}{2.354558in}}%
\pgfpathcurveto{\pgfqpoint{2.269574in}{2.354558in}}{\pgfqpoint{2.280173in}{2.358948in}}{\pgfqpoint{2.287986in}{2.366762in}}%
\pgfpathcurveto{\pgfqpoint{2.295800in}{2.374575in}}{\pgfqpoint{2.300190in}{2.385174in}}{\pgfqpoint{2.300190in}{2.396224in}}%
\pgfpathcurveto{\pgfqpoint{2.300190in}{2.407274in}}{\pgfqpoint{2.295800in}{2.417873in}}{\pgfqpoint{2.287986in}{2.425687in}}%
\pgfpathcurveto{\pgfqpoint{2.280173in}{2.433501in}}{\pgfqpoint{2.269574in}{2.437891in}}{\pgfqpoint{2.258523in}{2.437891in}}%
\pgfpathcurveto{\pgfqpoint{2.247473in}{2.437891in}}{\pgfqpoint{2.236874in}{2.433501in}}{\pgfqpoint{2.229061in}{2.425687in}}%
\pgfpathcurveto{\pgfqpoint{2.221247in}{2.417873in}}{\pgfqpoint{2.216857in}{2.407274in}}{\pgfqpoint{2.216857in}{2.396224in}}%
\pgfpathcurveto{\pgfqpoint{2.216857in}{2.385174in}}{\pgfqpoint{2.221247in}{2.374575in}}{\pgfqpoint{2.229061in}{2.366762in}}%
\pgfpathcurveto{\pgfqpoint{2.236874in}{2.358948in}}{\pgfqpoint{2.247473in}{2.354558in}}{\pgfqpoint{2.258523in}{2.354558in}}%
\pgfpathclose%
\pgfusepath{stroke,fill}%
\end{pgfscope}%
\begin{pgfscope}%
\pgfpathrectangle{\pgfqpoint{0.600000in}{0.600000in}}{\pgfqpoint{3.900000in}{3.900000in}}%
\pgfusepath{clip}%
\pgfsetbuttcap%
\pgfsetroundjoin%
\definecolor{currentfill}{rgb}{0.121569,0.466667,0.705882}%
\pgfsetfillcolor{currentfill}%
\pgfsetlinewidth{1.003750pt}%
\definecolor{currentstroke}{rgb}{0.121569,0.466667,0.705882}%
\pgfsetstrokecolor{currentstroke}%
\pgfsetdash{}{0pt}%
\pgfpathmoveto{\pgfqpoint{1.255907in}{2.030893in}}%
\pgfpathcurveto{\pgfqpoint{1.266957in}{2.030893in}}{\pgfqpoint{1.277557in}{2.035283in}}{\pgfqpoint{1.285370in}{2.043096in}}%
\pgfpathcurveto{\pgfqpoint{1.293184in}{2.050910in}}{\pgfqpoint{1.297574in}{2.061509in}}{\pgfqpoint{1.297574in}{2.072559in}}%
\pgfpathcurveto{\pgfqpoint{1.297574in}{2.083609in}}{\pgfqpoint{1.293184in}{2.094208in}}{\pgfqpoint{1.285370in}{2.102022in}}%
\pgfpathcurveto{\pgfqpoint{1.277557in}{2.109836in}}{\pgfqpoint{1.266957in}{2.114226in}}{\pgfqpoint{1.255907in}{2.114226in}}%
\pgfpathcurveto{\pgfqpoint{1.244857in}{2.114226in}}{\pgfqpoint{1.234258in}{2.109836in}}{\pgfqpoint{1.226445in}{2.102022in}}%
\pgfpathcurveto{\pgfqpoint{1.218631in}{2.094208in}}{\pgfqpoint{1.214241in}{2.083609in}}{\pgfqpoint{1.214241in}{2.072559in}}%
\pgfpathcurveto{\pgfqpoint{1.214241in}{2.061509in}}{\pgfqpoint{1.218631in}{2.050910in}}{\pgfqpoint{1.226445in}{2.043096in}}%
\pgfpathcurveto{\pgfqpoint{1.234258in}{2.035283in}}{\pgfqpoint{1.244857in}{2.030893in}}{\pgfqpoint{1.255907in}{2.030893in}}%
\pgfpathclose%
\pgfusepath{stroke,fill}%
\end{pgfscope}%
\begin{pgfscope}%
\pgfpathrectangle{\pgfqpoint{0.600000in}{0.600000in}}{\pgfqpoint{3.900000in}{3.900000in}}%
\pgfusepath{clip}%
\pgfsetbuttcap%
\pgfsetroundjoin%
\definecolor{currentfill}{rgb}{0.121569,0.466667,0.705882}%
\pgfsetfillcolor{currentfill}%
\pgfsetlinewidth{1.003750pt}%
\definecolor{currentstroke}{rgb}{0.121569,0.466667,0.705882}%
\pgfsetstrokecolor{currentstroke}%
\pgfsetdash{}{0pt}%
\pgfpathmoveto{\pgfqpoint{1.964353in}{2.326708in}}%
\pgfpathcurveto{\pgfqpoint{1.975403in}{2.326708in}}{\pgfqpoint{1.986002in}{2.331098in}}{\pgfqpoint{1.993815in}{2.338911in}}%
\pgfpathcurveto{\pgfqpoint{2.001629in}{2.346725in}}{\pgfqpoint{2.006019in}{2.357324in}}{\pgfqpoint{2.006019in}{2.368374in}}%
\pgfpathcurveto{\pgfqpoint{2.006019in}{2.379424in}}{\pgfqpoint{2.001629in}{2.390023in}}{\pgfqpoint{1.993815in}{2.397837in}}%
\pgfpathcurveto{\pgfqpoint{1.986002in}{2.405651in}}{\pgfqpoint{1.975403in}{2.410041in}}{\pgfqpoint{1.964353in}{2.410041in}}%
\pgfpathcurveto{\pgfqpoint{1.953303in}{2.410041in}}{\pgfqpoint{1.942703in}{2.405651in}}{\pgfqpoint{1.934890in}{2.397837in}}%
\pgfpathcurveto{\pgfqpoint{1.927076in}{2.390023in}}{\pgfqpoint{1.922686in}{2.379424in}}{\pgfqpoint{1.922686in}{2.368374in}}%
\pgfpathcurveto{\pgfqpoint{1.922686in}{2.357324in}}{\pgfqpoint{1.927076in}{2.346725in}}{\pgfqpoint{1.934890in}{2.338911in}}%
\pgfpathcurveto{\pgfqpoint{1.942703in}{2.331098in}}{\pgfqpoint{1.953303in}{2.326708in}}{\pgfqpoint{1.964353in}{2.326708in}}%
\pgfpathclose%
\pgfusepath{stroke,fill}%
\end{pgfscope}%
\begin{pgfscope}%
\pgfpathrectangle{\pgfqpoint{0.600000in}{0.600000in}}{\pgfqpoint{3.900000in}{3.900000in}}%
\pgfusepath{clip}%
\pgfsetbuttcap%
\pgfsetroundjoin%
\definecolor{currentfill}{rgb}{0.121569,0.466667,0.705882}%
\pgfsetfillcolor{currentfill}%
\pgfsetlinewidth{1.003750pt}%
\definecolor{currentstroke}{rgb}{0.121569,0.466667,0.705882}%
\pgfsetstrokecolor{currentstroke}%
\pgfsetdash{}{0pt}%
\pgfpathmoveto{\pgfqpoint{3.303068in}{2.666130in}}%
\pgfpathcurveto{\pgfqpoint{3.314118in}{2.666130in}}{\pgfqpoint{3.324717in}{2.670520in}}{\pgfqpoint{3.332531in}{2.678334in}}%
\pgfpathcurveto{\pgfqpoint{3.340345in}{2.686147in}}{\pgfqpoint{3.344735in}{2.696746in}}{\pgfqpoint{3.344735in}{2.707796in}}%
\pgfpathcurveto{\pgfqpoint{3.344735in}{2.718847in}}{\pgfqpoint{3.340345in}{2.729446in}}{\pgfqpoint{3.332531in}{2.737259in}}%
\pgfpathcurveto{\pgfqpoint{3.324717in}{2.745073in}}{\pgfqpoint{3.314118in}{2.749463in}}{\pgfqpoint{3.303068in}{2.749463in}}%
\pgfpathcurveto{\pgfqpoint{3.292018in}{2.749463in}}{\pgfqpoint{3.281419in}{2.745073in}}{\pgfqpoint{3.273606in}{2.737259in}}%
\pgfpathcurveto{\pgfqpoint{3.265792in}{2.729446in}}{\pgfqpoint{3.261402in}{2.718847in}}{\pgfqpoint{3.261402in}{2.707796in}}%
\pgfpathcurveto{\pgfqpoint{3.261402in}{2.696746in}}{\pgfqpoint{3.265792in}{2.686147in}}{\pgfqpoint{3.273606in}{2.678334in}}%
\pgfpathcurveto{\pgfqpoint{3.281419in}{2.670520in}}{\pgfqpoint{3.292018in}{2.666130in}}{\pgfqpoint{3.303068in}{2.666130in}}%
\pgfpathclose%
\pgfusepath{stroke,fill}%
\end{pgfscope}%
\begin{pgfscope}%
\pgfpathrectangle{\pgfqpoint{0.600000in}{0.600000in}}{\pgfqpoint{3.900000in}{3.900000in}}%
\pgfusepath{clip}%
\pgfsetbuttcap%
\pgfsetroundjoin%
\definecolor{currentfill}{rgb}{0.121569,0.466667,0.705882}%
\pgfsetfillcolor{currentfill}%
\pgfsetlinewidth{1.003750pt}%
\definecolor{currentstroke}{rgb}{0.121569,0.466667,0.705882}%
\pgfsetstrokecolor{currentstroke}%
\pgfsetdash{}{0pt}%
\pgfpathmoveto{\pgfqpoint{3.064836in}{2.056747in}}%
\pgfpathcurveto{\pgfqpoint{3.075886in}{2.056747in}}{\pgfqpoint{3.086485in}{2.061137in}}{\pgfqpoint{3.094298in}{2.068950in}}%
\pgfpathcurveto{\pgfqpoint{3.102112in}{2.076764in}}{\pgfqpoint{3.106502in}{2.087363in}}{\pgfqpoint{3.106502in}{2.098413in}}%
\pgfpathcurveto{\pgfqpoint{3.106502in}{2.109463in}}{\pgfqpoint{3.102112in}{2.120062in}}{\pgfqpoint{3.094298in}{2.127876in}}%
\pgfpathcurveto{\pgfqpoint{3.086485in}{2.135690in}}{\pgfqpoint{3.075886in}{2.140080in}}{\pgfqpoint{3.064836in}{2.140080in}}%
\pgfpathcurveto{\pgfqpoint{3.053785in}{2.140080in}}{\pgfqpoint{3.043186in}{2.135690in}}{\pgfqpoint{3.035373in}{2.127876in}}%
\pgfpathcurveto{\pgfqpoint{3.027559in}{2.120062in}}{\pgfqpoint{3.023169in}{2.109463in}}{\pgfqpoint{3.023169in}{2.098413in}}%
\pgfpathcurveto{\pgfqpoint{3.023169in}{2.087363in}}{\pgfqpoint{3.027559in}{2.076764in}}{\pgfqpoint{3.035373in}{2.068950in}}%
\pgfpathcurveto{\pgfqpoint{3.043186in}{2.061137in}}{\pgfqpoint{3.053785in}{2.056747in}}{\pgfqpoint{3.064836in}{2.056747in}}%
\pgfpathclose%
\pgfusepath{stroke,fill}%
\end{pgfscope}%
\begin{pgfscope}%
\pgfpathrectangle{\pgfqpoint{0.600000in}{0.600000in}}{\pgfqpoint{3.900000in}{3.900000in}}%
\pgfusepath{clip}%
\pgfsetbuttcap%
\pgfsetroundjoin%
\definecolor{currentfill}{rgb}{0.121569,0.466667,0.705882}%
\pgfsetfillcolor{currentfill}%
\pgfsetlinewidth{1.003750pt}%
\definecolor{currentstroke}{rgb}{0.121569,0.466667,0.705882}%
\pgfsetstrokecolor{currentstroke}%
\pgfsetdash{}{0pt}%
\pgfpathmoveto{\pgfqpoint{3.413907in}{2.075069in}}%
\pgfpathcurveto{\pgfqpoint{3.424957in}{2.075069in}}{\pgfqpoint{3.435556in}{2.079459in}}{\pgfqpoint{3.443369in}{2.087273in}}%
\pgfpathcurveto{\pgfqpoint{3.451183in}{2.095087in}}{\pgfqpoint{3.455573in}{2.105686in}}{\pgfqpoint{3.455573in}{2.116736in}}%
\pgfpathcurveto{\pgfqpoint{3.455573in}{2.127786in}}{\pgfqpoint{3.451183in}{2.138385in}}{\pgfqpoint{3.443369in}{2.146199in}}%
\pgfpathcurveto{\pgfqpoint{3.435556in}{2.154012in}}{\pgfqpoint{3.424957in}{2.158402in}}{\pgfqpoint{3.413907in}{2.158402in}}%
\pgfpathcurveto{\pgfqpoint{3.402856in}{2.158402in}}{\pgfqpoint{3.392257in}{2.154012in}}{\pgfqpoint{3.384444in}{2.146199in}}%
\pgfpathcurveto{\pgfqpoint{3.376630in}{2.138385in}}{\pgfqpoint{3.372240in}{2.127786in}}{\pgfqpoint{3.372240in}{2.116736in}}%
\pgfpathcurveto{\pgfqpoint{3.372240in}{2.105686in}}{\pgfqpoint{3.376630in}{2.095087in}}{\pgfqpoint{3.384444in}{2.087273in}}%
\pgfpathcurveto{\pgfqpoint{3.392257in}{2.079459in}}{\pgfqpoint{3.402856in}{2.075069in}}{\pgfqpoint{3.413907in}{2.075069in}}%
\pgfpathclose%
\pgfusepath{stroke,fill}%
\end{pgfscope}%
\begin{pgfscope}%
\pgfpathrectangle{\pgfqpoint{0.600000in}{0.600000in}}{\pgfqpoint{3.900000in}{3.900000in}}%
\pgfusepath{clip}%
\pgfsetbuttcap%
\pgfsetroundjoin%
\definecolor{currentfill}{rgb}{0.121569,0.466667,0.705882}%
\pgfsetfillcolor{currentfill}%
\pgfsetlinewidth{1.003750pt}%
\definecolor{currentstroke}{rgb}{0.121569,0.466667,0.705882}%
\pgfsetstrokecolor{currentstroke}%
\pgfsetdash{}{0pt}%
\pgfpathmoveto{\pgfqpoint{1.722307in}{1.541817in}}%
\pgfpathcurveto{\pgfqpoint{1.733357in}{1.541817in}}{\pgfqpoint{1.743956in}{1.546207in}}{\pgfqpoint{1.751770in}{1.554021in}}%
\pgfpathcurveto{\pgfqpoint{1.759584in}{1.561834in}}{\pgfqpoint{1.763974in}{1.572433in}}{\pgfqpoint{1.763974in}{1.583484in}}%
\pgfpathcurveto{\pgfqpoint{1.763974in}{1.594534in}}{\pgfqpoint{1.759584in}{1.605133in}}{\pgfqpoint{1.751770in}{1.612946in}}%
\pgfpathcurveto{\pgfqpoint{1.743956in}{1.620760in}}{\pgfqpoint{1.733357in}{1.625150in}}{\pgfqpoint{1.722307in}{1.625150in}}%
\pgfpathcurveto{\pgfqpoint{1.711257in}{1.625150in}}{\pgfqpoint{1.700658in}{1.620760in}}{\pgfqpoint{1.692845in}{1.612946in}}%
\pgfpathcurveto{\pgfqpoint{1.685031in}{1.605133in}}{\pgfqpoint{1.680641in}{1.594534in}}{\pgfqpoint{1.680641in}{1.583484in}}%
\pgfpathcurveto{\pgfqpoint{1.680641in}{1.572433in}}{\pgfqpoint{1.685031in}{1.561834in}}{\pgfqpoint{1.692845in}{1.554021in}}%
\pgfpathcurveto{\pgfqpoint{1.700658in}{1.546207in}}{\pgfqpoint{1.711257in}{1.541817in}}{\pgfqpoint{1.722307in}{1.541817in}}%
\pgfpathclose%
\pgfusepath{stroke,fill}%
\end{pgfscope}%
\begin{pgfscope}%
\pgfpathrectangle{\pgfqpoint{0.600000in}{0.600000in}}{\pgfqpoint{3.900000in}{3.900000in}}%
\pgfusepath{clip}%
\pgfsetbuttcap%
\pgfsetroundjoin%
\definecolor{currentfill}{rgb}{0.121569,0.466667,0.705882}%
\pgfsetfillcolor{currentfill}%
\pgfsetlinewidth{1.003750pt}%
\definecolor{currentstroke}{rgb}{0.121569,0.466667,0.705882}%
\pgfsetstrokecolor{currentstroke}%
\pgfsetdash{}{0pt}%
\pgfpathmoveto{\pgfqpoint{2.353624in}{2.403904in}}%
\pgfpathcurveto{\pgfqpoint{2.364674in}{2.403904in}}{\pgfqpoint{2.375273in}{2.408294in}}{\pgfqpoint{2.383087in}{2.416108in}}%
\pgfpathcurveto{\pgfqpoint{2.390900in}{2.423921in}}{\pgfqpoint{2.395291in}{2.434520in}}{\pgfqpoint{2.395291in}{2.445570in}}%
\pgfpathcurveto{\pgfqpoint{2.395291in}{2.456620in}}{\pgfqpoint{2.390900in}{2.467219in}}{\pgfqpoint{2.383087in}{2.475033in}}%
\pgfpathcurveto{\pgfqpoint{2.375273in}{2.482847in}}{\pgfqpoint{2.364674in}{2.487237in}}{\pgfqpoint{2.353624in}{2.487237in}}%
\pgfpathcurveto{\pgfqpoint{2.342574in}{2.487237in}}{\pgfqpoint{2.331975in}{2.482847in}}{\pgfqpoint{2.324161in}{2.475033in}}%
\pgfpathcurveto{\pgfqpoint{2.316348in}{2.467219in}}{\pgfqpoint{2.311957in}{2.456620in}}{\pgfqpoint{2.311957in}{2.445570in}}%
\pgfpathcurveto{\pgfqpoint{2.311957in}{2.434520in}}{\pgfqpoint{2.316348in}{2.423921in}}{\pgfqpoint{2.324161in}{2.416108in}}%
\pgfpathcurveto{\pgfqpoint{2.331975in}{2.408294in}}{\pgfqpoint{2.342574in}{2.403904in}}{\pgfqpoint{2.353624in}{2.403904in}}%
\pgfpathclose%
\pgfusepath{stroke,fill}%
\end{pgfscope}%
\begin{pgfscope}%
\pgfpathrectangle{\pgfqpoint{0.600000in}{0.600000in}}{\pgfqpoint{3.900000in}{3.900000in}}%
\pgfusepath{clip}%
\pgfsetbuttcap%
\pgfsetroundjoin%
\definecolor{currentfill}{rgb}{0.121569,0.466667,0.705882}%
\pgfsetfillcolor{currentfill}%
\pgfsetlinewidth{1.003750pt}%
\definecolor{currentstroke}{rgb}{0.121569,0.466667,0.705882}%
\pgfsetstrokecolor{currentstroke}%
\pgfsetdash{}{0pt}%
\pgfpathmoveto{\pgfqpoint{3.192503in}{2.750098in}}%
\pgfpathcurveto{\pgfqpoint{3.203553in}{2.750098in}}{\pgfqpoint{3.214152in}{2.754488in}}{\pgfqpoint{3.221966in}{2.762302in}}%
\pgfpathcurveto{\pgfqpoint{3.229779in}{2.770115in}}{\pgfqpoint{3.234170in}{2.780714in}}{\pgfqpoint{3.234170in}{2.791764in}}%
\pgfpathcurveto{\pgfqpoint{3.234170in}{2.802815in}}{\pgfqpoint{3.229779in}{2.813414in}}{\pgfqpoint{3.221966in}{2.821227in}}%
\pgfpathcurveto{\pgfqpoint{3.214152in}{2.829041in}}{\pgfqpoint{3.203553in}{2.833431in}}{\pgfqpoint{3.192503in}{2.833431in}}%
\pgfpathcurveto{\pgfqpoint{3.181453in}{2.833431in}}{\pgfqpoint{3.170854in}{2.829041in}}{\pgfqpoint{3.163040in}{2.821227in}}%
\pgfpathcurveto{\pgfqpoint{3.155227in}{2.813414in}}{\pgfqpoint{3.150836in}{2.802815in}}{\pgfqpoint{3.150836in}{2.791764in}}%
\pgfpathcurveto{\pgfqpoint{3.150836in}{2.780714in}}{\pgfqpoint{3.155227in}{2.770115in}}{\pgfqpoint{3.163040in}{2.762302in}}%
\pgfpathcurveto{\pgfqpoint{3.170854in}{2.754488in}}{\pgfqpoint{3.181453in}{2.750098in}}{\pgfqpoint{3.192503in}{2.750098in}}%
\pgfpathclose%
\pgfusepath{stroke,fill}%
\end{pgfscope}%
\begin{pgfscope}%
\pgfpathrectangle{\pgfqpoint{0.600000in}{0.600000in}}{\pgfqpoint{3.900000in}{3.900000in}}%
\pgfusepath{clip}%
\pgfsetbuttcap%
\pgfsetroundjoin%
\definecolor{currentfill}{rgb}{0.121569,0.466667,0.705882}%
\pgfsetfillcolor{currentfill}%
\pgfsetlinewidth{1.003750pt}%
\definecolor{currentstroke}{rgb}{0.121569,0.466667,0.705882}%
\pgfsetstrokecolor{currentstroke}%
\pgfsetdash{}{0pt}%
\pgfpathmoveto{\pgfqpoint{2.047470in}{1.926070in}}%
\pgfpathcurveto{\pgfqpoint{2.058520in}{1.926070in}}{\pgfqpoint{2.069119in}{1.930460in}}{\pgfqpoint{2.076932in}{1.938273in}}%
\pgfpathcurveto{\pgfqpoint{2.084746in}{1.946087in}}{\pgfqpoint{2.089136in}{1.956686in}}{\pgfqpoint{2.089136in}{1.967736in}}%
\pgfpathcurveto{\pgfqpoint{2.089136in}{1.978786in}}{\pgfqpoint{2.084746in}{1.989385in}}{\pgfqpoint{2.076932in}{1.997199in}}%
\pgfpathcurveto{\pgfqpoint{2.069119in}{2.005013in}}{\pgfqpoint{2.058520in}{2.009403in}}{\pgfqpoint{2.047470in}{2.009403in}}%
\pgfpathcurveto{\pgfqpoint{2.036419in}{2.009403in}}{\pgfqpoint{2.025820in}{2.005013in}}{\pgfqpoint{2.018007in}{1.997199in}}%
\pgfpathcurveto{\pgfqpoint{2.010193in}{1.989385in}}{\pgfqpoint{2.005803in}{1.978786in}}{\pgfqpoint{2.005803in}{1.967736in}}%
\pgfpathcurveto{\pgfqpoint{2.005803in}{1.956686in}}{\pgfqpoint{2.010193in}{1.946087in}}{\pgfqpoint{2.018007in}{1.938273in}}%
\pgfpathcurveto{\pgfqpoint{2.025820in}{1.930460in}}{\pgfqpoint{2.036419in}{1.926070in}}{\pgfqpoint{2.047470in}{1.926070in}}%
\pgfpathclose%
\pgfusepath{stroke,fill}%
\end{pgfscope}%
\begin{pgfscope}%
\pgfpathrectangle{\pgfqpoint{0.600000in}{0.600000in}}{\pgfqpoint{3.900000in}{3.900000in}}%
\pgfusepath{clip}%
\pgfsetbuttcap%
\pgfsetroundjoin%
\definecolor{currentfill}{rgb}{0.121569,0.466667,0.705882}%
\pgfsetfillcolor{currentfill}%
\pgfsetlinewidth{1.003750pt}%
\definecolor{currentstroke}{rgb}{0.121569,0.466667,0.705882}%
\pgfsetstrokecolor{currentstroke}%
\pgfsetdash{}{0pt}%
\pgfpathmoveto{\pgfqpoint{2.113929in}{2.087791in}}%
\pgfpathcurveto{\pgfqpoint{2.124979in}{2.087791in}}{\pgfqpoint{2.135579in}{2.092181in}}{\pgfqpoint{2.143392in}{2.099994in}}%
\pgfpathcurveto{\pgfqpoint{2.151206in}{2.107808in}}{\pgfqpoint{2.155596in}{2.118407in}}{\pgfqpoint{2.155596in}{2.129457in}}%
\pgfpathcurveto{\pgfqpoint{2.155596in}{2.140507in}}{\pgfqpoint{2.151206in}{2.151106in}}{\pgfqpoint{2.143392in}{2.158920in}}%
\pgfpathcurveto{\pgfqpoint{2.135579in}{2.166734in}}{\pgfqpoint{2.124979in}{2.171124in}}{\pgfqpoint{2.113929in}{2.171124in}}%
\pgfpathcurveto{\pgfqpoint{2.102879in}{2.171124in}}{\pgfqpoint{2.092280in}{2.166734in}}{\pgfqpoint{2.084467in}{2.158920in}}%
\pgfpathcurveto{\pgfqpoint{2.076653in}{2.151106in}}{\pgfqpoint{2.072263in}{2.140507in}}{\pgfqpoint{2.072263in}{2.129457in}}%
\pgfpathcurveto{\pgfqpoint{2.072263in}{2.118407in}}{\pgfqpoint{2.076653in}{2.107808in}}{\pgfqpoint{2.084467in}{2.099994in}}%
\pgfpathcurveto{\pgfqpoint{2.092280in}{2.092181in}}{\pgfqpoint{2.102879in}{2.087791in}}{\pgfqpoint{2.113929in}{2.087791in}}%
\pgfpathclose%
\pgfusepath{stroke,fill}%
\end{pgfscope}%
\begin{pgfscope}%
\pgfpathrectangle{\pgfqpoint{0.600000in}{0.600000in}}{\pgfqpoint{3.900000in}{3.900000in}}%
\pgfusepath{clip}%
\pgfsetbuttcap%
\pgfsetroundjoin%
\definecolor{currentfill}{rgb}{0.121569,0.466667,0.705882}%
\pgfsetfillcolor{currentfill}%
\pgfsetlinewidth{1.003750pt}%
\definecolor{currentstroke}{rgb}{0.121569,0.466667,0.705882}%
\pgfsetstrokecolor{currentstroke}%
\pgfsetdash{}{0pt}%
\pgfpathmoveto{\pgfqpoint{2.877536in}{2.718031in}}%
\pgfpathcurveto{\pgfqpoint{2.888586in}{2.718031in}}{\pgfqpoint{2.899185in}{2.722421in}}{\pgfqpoint{2.906999in}{2.730235in}}%
\pgfpathcurveto{\pgfqpoint{2.914812in}{2.738048in}}{\pgfqpoint{2.919202in}{2.748648in}}{\pgfqpoint{2.919202in}{2.759698in}}%
\pgfpathcurveto{\pgfqpoint{2.919202in}{2.770748in}}{\pgfqpoint{2.914812in}{2.781347in}}{\pgfqpoint{2.906999in}{2.789160in}}%
\pgfpathcurveto{\pgfqpoint{2.899185in}{2.796974in}}{\pgfqpoint{2.888586in}{2.801364in}}{\pgfqpoint{2.877536in}{2.801364in}}%
\pgfpathcurveto{\pgfqpoint{2.866486in}{2.801364in}}{\pgfqpoint{2.855887in}{2.796974in}}{\pgfqpoint{2.848073in}{2.789160in}}%
\pgfpathcurveto{\pgfqpoint{2.840259in}{2.781347in}}{\pgfqpoint{2.835869in}{2.770748in}}{\pgfqpoint{2.835869in}{2.759698in}}%
\pgfpathcurveto{\pgfqpoint{2.835869in}{2.748648in}}{\pgfqpoint{2.840259in}{2.738048in}}{\pgfqpoint{2.848073in}{2.730235in}}%
\pgfpathcurveto{\pgfqpoint{2.855887in}{2.722421in}}{\pgfqpoint{2.866486in}{2.718031in}}{\pgfqpoint{2.877536in}{2.718031in}}%
\pgfpathclose%
\pgfusepath{stroke,fill}%
\end{pgfscope}%
\begin{pgfscope}%
\pgfpathrectangle{\pgfqpoint{0.600000in}{0.600000in}}{\pgfqpoint{3.900000in}{3.900000in}}%
\pgfusepath{clip}%
\pgfsetbuttcap%
\pgfsetroundjoin%
\definecolor{currentfill}{rgb}{0.121569,0.466667,0.705882}%
\pgfsetfillcolor{currentfill}%
\pgfsetlinewidth{1.003750pt}%
\definecolor{currentstroke}{rgb}{0.121569,0.466667,0.705882}%
\pgfsetstrokecolor{currentstroke}%
\pgfsetdash{}{0pt}%
\pgfpathmoveto{\pgfqpoint{2.633618in}{2.167733in}}%
\pgfpathcurveto{\pgfqpoint{2.644668in}{2.167733in}}{\pgfqpoint{2.655267in}{2.172123in}}{\pgfqpoint{2.663081in}{2.179936in}}%
\pgfpathcurveto{\pgfqpoint{2.670895in}{2.187750in}}{\pgfqpoint{2.675285in}{2.198349in}}{\pgfqpoint{2.675285in}{2.209399in}}%
\pgfpathcurveto{\pgfqpoint{2.675285in}{2.220449in}}{\pgfqpoint{2.670895in}{2.231048in}}{\pgfqpoint{2.663081in}{2.238862in}}%
\pgfpathcurveto{\pgfqpoint{2.655267in}{2.246676in}}{\pgfqpoint{2.644668in}{2.251066in}}{\pgfqpoint{2.633618in}{2.251066in}}%
\pgfpathcurveto{\pgfqpoint{2.622568in}{2.251066in}}{\pgfqpoint{2.611969in}{2.246676in}}{\pgfqpoint{2.604155in}{2.238862in}}%
\pgfpathcurveto{\pgfqpoint{2.596342in}{2.231048in}}{\pgfqpoint{2.591952in}{2.220449in}}{\pgfqpoint{2.591952in}{2.209399in}}%
\pgfpathcurveto{\pgfqpoint{2.591952in}{2.198349in}}{\pgfqpoint{2.596342in}{2.187750in}}{\pgfqpoint{2.604155in}{2.179936in}}%
\pgfpathcurveto{\pgfqpoint{2.611969in}{2.172123in}}{\pgfqpoint{2.622568in}{2.167733in}}{\pgfqpoint{2.633618in}{2.167733in}}%
\pgfpathclose%
\pgfusepath{stroke,fill}%
\end{pgfscope}%
\begin{pgfscope}%
\pgfpathrectangle{\pgfqpoint{0.600000in}{0.600000in}}{\pgfqpoint{3.900000in}{3.900000in}}%
\pgfusepath{clip}%
\pgfsetbuttcap%
\pgfsetroundjoin%
\definecolor{currentfill}{rgb}{0.121569,0.466667,0.705882}%
\pgfsetfillcolor{currentfill}%
\pgfsetlinewidth{1.003750pt}%
\definecolor{currentstroke}{rgb}{0.121569,0.466667,0.705882}%
\pgfsetstrokecolor{currentstroke}%
\pgfsetdash{}{0pt}%
\pgfpathmoveto{\pgfqpoint{2.749823in}{2.388910in}}%
\pgfpathcurveto{\pgfqpoint{2.760873in}{2.388910in}}{\pgfqpoint{2.771472in}{2.393300in}}{\pgfqpoint{2.779286in}{2.401114in}}%
\pgfpathcurveto{\pgfqpoint{2.787099in}{2.408927in}}{\pgfqpoint{2.791490in}{2.419526in}}{\pgfqpoint{2.791490in}{2.430576in}}%
\pgfpathcurveto{\pgfqpoint{2.791490in}{2.441626in}}{\pgfqpoint{2.787099in}{2.452225in}}{\pgfqpoint{2.779286in}{2.460039in}}%
\pgfpathcurveto{\pgfqpoint{2.771472in}{2.467853in}}{\pgfqpoint{2.760873in}{2.472243in}}{\pgfqpoint{2.749823in}{2.472243in}}%
\pgfpathcurveto{\pgfqpoint{2.738773in}{2.472243in}}{\pgfqpoint{2.728174in}{2.467853in}}{\pgfqpoint{2.720360in}{2.460039in}}%
\pgfpathcurveto{\pgfqpoint{2.712547in}{2.452225in}}{\pgfqpoint{2.708156in}{2.441626in}}{\pgfqpoint{2.708156in}{2.430576in}}%
\pgfpathcurveto{\pgfqpoint{2.708156in}{2.419526in}}{\pgfqpoint{2.712547in}{2.408927in}}{\pgfqpoint{2.720360in}{2.401114in}}%
\pgfpathcurveto{\pgfqpoint{2.728174in}{2.393300in}}{\pgfqpoint{2.738773in}{2.388910in}}{\pgfqpoint{2.749823in}{2.388910in}}%
\pgfpathclose%
\pgfusepath{stroke,fill}%
\end{pgfscope}%
\begin{pgfscope}%
\pgfpathrectangle{\pgfqpoint{0.600000in}{0.600000in}}{\pgfqpoint{3.900000in}{3.900000in}}%
\pgfusepath{clip}%
\pgfsetbuttcap%
\pgfsetroundjoin%
\definecolor{currentfill}{rgb}{0.121569,0.466667,0.705882}%
\pgfsetfillcolor{currentfill}%
\pgfsetlinewidth{1.003750pt}%
\definecolor{currentstroke}{rgb}{0.121569,0.466667,0.705882}%
\pgfsetstrokecolor{currentstroke}%
\pgfsetdash{}{0pt}%
\pgfpathmoveto{\pgfqpoint{2.472172in}{1.739492in}}%
\pgfpathcurveto{\pgfqpoint{2.483222in}{1.739492in}}{\pgfqpoint{2.493821in}{1.743882in}}{\pgfqpoint{2.501635in}{1.751696in}}%
\pgfpathcurveto{\pgfqpoint{2.509448in}{1.759509in}}{\pgfqpoint{2.513839in}{1.770108in}}{\pgfqpoint{2.513839in}{1.781158in}}%
\pgfpathcurveto{\pgfqpoint{2.513839in}{1.792209in}}{\pgfqpoint{2.509448in}{1.802808in}}{\pgfqpoint{2.501635in}{1.810621in}}%
\pgfpathcurveto{\pgfqpoint{2.493821in}{1.818435in}}{\pgfqpoint{2.483222in}{1.822825in}}{\pgfqpoint{2.472172in}{1.822825in}}%
\pgfpathcurveto{\pgfqpoint{2.461122in}{1.822825in}}{\pgfqpoint{2.450523in}{1.818435in}}{\pgfqpoint{2.442709in}{1.810621in}}%
\pgfpathcurveto{\pgfqpoint{2.434896in}{1.802808in}}{\pgfqpoint{2.430505in}{1.792209in}}{\pgfqpoint{2.430505in}{1.781158in}}%
\pgfpathcurveto{\pgfqpoint{2.430505in}{1.770108in}}{\pgfqpoint{2.434896in}{1.759509in}}{\pgfqpoint{2.442709in}{1.751696in}}%
\pgfpathcurveto{\pgfqpoint{2.450523in}{1.743882in}}{\pgfqpoint{2.461122in}{1.739492in}}{\pgfqpoint{2.472172in}{1.739492in}}%
\pgfpathclose%
\pgfusepath{stroke,fill}%
\end{pgfscope}%
\begin{pgfscope}%
\pgfpathrectangle{\pgfqpoint{0.600000in}{0.600000in}}{\pgfqpoint{3.900000in}{3.900000in}}%
\pgfusepath{clip}%
\pgfsetbuttcap%
\pgfsetroundjoin%
\definecolor{currentfill}{rgb}{0.121569,0.466667,0.705882}%
\pgfsetfillcolor{currentfill}%
\pgfsetlinewidth{1.003750pt}%
\definecolor{currentstroke}{rgb}{0.121569,0.466667,0.705882}%
\pgfsetstrokecolor{currentstroke}%
\pgfsetdash{}{0pt}%
\pgfpathmoveto{\pgfqpoint{2.798850in}{2.424136in}}%
\pgfpathcurveto{\pgfqpoint{2.809900in}{2.424136in}}{\pgfqpoint{2.820499in}{2.428527in}}{\pgfqpoint{2.828312in}{2.436340in}}%
\pgfpathcurveto{\pgfqpoint{2.836126in}{2.444154in}}{\pgfqpoint{2.840516in}{2.454753in}}{\pgfqpoint{2.840516in}{2.465803in}}%
\pgfpathcurveto{\pgfqpoint{2.840516in}{2.476853in}}{\pgfqpoint{2.836126in}{2.487452in}}{\pgfqpoint{2.828312in}{2.495266in}}%
\pgfpathcurveto{\pgfqpoint{2.820499in}{2.503079in}}{\pgfqpoint{2.809900in}{2.507470in}}{\pgfqpoint{2.798850in}{2.507470in}}%
\pgfpathcurveto{\pgfqpoint{2.787800in}{2.507470in}}{\pgfqpoint{2.777201in}{2.503079in}}{\pgfqpoint{2.769387in}{2.495266in}}%
\pgfpathcurveto{\pgfqpoint{2.761573in}{2.487452in}}{\pgfqpoint{2.757183in}{2.476853in}}{\pgfqpoint{2.757183in}{2.465803in}}%
\pgfpathcurveto{\pgfqpoint{2.757183in}{2.454753in}}{\pgfqpoint{2.761573in}{2.444154in}}{\pgfqpoint{2.769387in}{2.436340in}}%
\pgfpathcurveto{\pgfqpoint{2.777201in}{2.428527in}}{\pgfqpoint{2.787800in}{2.424136in}}{\pgfqpoint{2.798850in}{2.424136in}}%
\pgfpathclose%
\pgfusepath{stroke,fill}%
\end{pgfscope}%
\begin{pgfscope}%
\pgfpathrectangle{\pgfqpoint{0.600000in}{0.600000in}}{\pgfqpoint{3.900000in}{3.900000in}}%
\pgfusepath{clip}%
\pgfsetbuttcap%
\pgfsetroundjoin%
\definecolor{currentfill}{rgb}{0.121569,0.466667,0.705882}%
\pgfsetfillcolor{currentfill}%
\pgfsetlinewidth{1.003750pt}%
\definecolor{currentstroke}{rgb}{0.121569,0.466667,0.705882}%
\pgfsetstrokecolor{currentstroke}%
\pgfsetdash{}{0pt}%
\pgfpathmoveto{\pgfqpoint{2.443224in}{2.317170in}}%
\pgfpathcurveto{\pgfqpoint{2.454274in}{2.317170in}}{\pgfqpoint{2.464873in}{2.321560in}}{\pgfqpoint{2.472687in}{2.329373in}}%
\pgfpathcurveto{\pgfqpoint{2.480500in}{2.337187in}}{\pgfqpoint{2.484891in}{2.347786in}}{\pgfqpoint{2.484891in}{2.358836in}}%
\pgfpathcurveto{\pgfqpoint{2.484891in}{2.369886in}}{\pgfqpoint{2.480500in}{2.380485in}}{\pgfqpoint{2.472687in}{2.388299in}}%
\pgfpathcurveto{\pgfqpoint{2.464873in}{2.396113in}}{\pgfqpoint{2.454274in}{2.400503in}}{\pgfqpoint{2.443224in}{2.400503in}}%
\pgfpathcurveto{\pgfqpoint{2.432174in}{2.400503in}}{\pgfqpoint{2.421575in}{2.396113in}}{\pgfqpoint{2.413761in}{2.388299in}}%
\pgfpathcurveto{\pgfqpoint{2.405948in}{2.380485in}}{\pgfqpoint{2.401557in}{2.369886in}}{\pgfqpoint{2.401557in}{2.358836in}}%
\pgfpathcurveto{\pgfqpoint{2.401557in}{2.347786in}}{\pgfqpoint{2.405948in}{2.337187in}}{\pgfqpoint{2.413761in}{2.329373in}}%
\pgfpathcurveto{\pgfqpoint{2.421575in}{2.321560in}}{\pgfqpoint{2.432174in}{2.317170in}}{\pgfqpoint{2.443224in}{2.317170in}}%
\pgfpathclose%
\pgfusepath{stroke,fill}%
\end{pgfscope}%
\begin{pgfscope}%
\pgfpathrectangle{\pgfqpoint{0.600000in}{0.600000in}}{\pgfqpoint{3.900000in}{3.900000in}}%
\pgfusepath{clip}%
\pgfsetbuttcap%
\pgfsetroundjoin%
\definecolor{currentfill}{rgb}{0.121569,0.466667,0.705882}%
\pgfsetfillcolor{currentfill}%
\pgfsetlinewidth{1.003750pt}%
\definecolor{currentstroke}{rgb}{0.121569,0.466667,0.705882}%
\pgfsetstrokecolor{currentstroke}%
\pgfsetdash{}{0pt}%
\pgfpathmoveto{\pgfqpoint{2.473318in}{2.991295in}}%
\pgfpathcurveto{\pgfqpoint{2.484369in}{2.991295in}}{\pgfqpoint{2.494968in}{2.995685in}}{\pgfqpoint{2.502781in}{3.003499in}}%
\pgfpathcurveto{\pgfqpoint{2.510595in}{3.011312in}}{\pgfqpoint{2.514985in}{3.021911in}}{\pgfqpoint{2.514985in}{3.032961in}}%
\pgfpathcurveto{\pgfqpoint{2.514985in}{3.044011in}}{\pgfqpoint{2.510595in}{3.054610in}}{\pgfqpoint{2.502781in}{3.062424in}}%
\pgfpathcurveto{\pgfqpoint{2.494968in}{3.070238in}}{\pgfqpoint{2.484369in}{3.074628in}}{\pgfqpoint{2.473318in}{3.074628in}}%
\pgfpathcurveto{\pgfqpoint{2.462268in}{3.074628in}}{\pgfqpoint{2.451669in}{3.070238in}}{\pgfqpoint{2.443856in}{3.062424in}}%
\pgfpathcurveto{\pgfqpoint{2.436042in}{3.054610in}}{\pgfqpoint{2.431652in}{3.044011in}}{\pgfqpoint{2.431652in}{3.032961in}}%
\pgfpathcurveto{\pgfqpoint{2.431652in}{3.021911in}}{\pgfqpoint{2.436042in}{3.011312in}}{\pgfqpoint{2.443856in}{3.003499in}}%
\pgfpathcurveto{\pgfqpoint{2.451669in}{2.995685in}}{\pgfqpoint{2.462268in}{2.991295in}}{\pgfqpoint{2.473318in}{2.991295in}}%
\pgfpathclose%
\pgfusepath{stroke,fill}%
\end{pgfscope}%
\begin{pgfscope}%
\pgfpathrectangle{\pgfqpoint{0.600000in}{0.600000in}}{\pgfqpoint{3.900000in}{3.900000in}}%
\pgfusepath{clip}%
\pgfsetbuttcap%
\pgfsetroundjoin%
\definecolor{currentfill}{rgb}{0.121569,0.466667,0.705882}%
\pgfsetfillcolor{currentfill}%
\pgfsetlinewidth{1.003750pt}%
\definecolor{currentstroke}{rgb}{0.121569,0.466667,0.705882}%
\pgfsetstrokecolor{currentstroke}%
\pgfsetdash{}{0pt}%
\pgfpathmoveto{\pgfqpoint{1.767157in}{1.533437in}}%
\pgfpathcurveto{\pgfqpoint{1.778207in}{1.533437in}}{\pgfqpoint{1.788806in}{1.537828in}}{\pgfqpoint{1.796620in}{1.545641in}}%
\pgfpathcurveto{\pgfqpoint{1.804434in}{1.553455in}}{\pgfqpoint{1.808824in}{1.564054in}}{\pgfqpoint{1.808824in}{1.575104in}}%
\pgfpathcurveto{\pgfqpoint{1.808824in}{1.586154in}}{\pgfqpoint{1.804434in}{1.596753in}}{\pgfqpoint{1.796620in}{1.604567in}}%
\pgfpathcurveto{\pgfqpoint{1.788806in}{1.612380in}}{\pgfqpoint{1.778207in}{1.616771in}}{\pgfqpoint{1.767157in}{1.616771in}}%
\pgfpathcurveto{\pgfqpoint{1.756107in}{1.616771in}}{\pgfqpoint{1.745508in}{1.612380in}}{\pgfqpoint{1.737694in}{1.604567in}}%
\pgfpathcurveto{\pgfqpoint{1.729881in}{1.596753in}}{\pgfqpoint{1.725490in}{1.586154in}}{\pgfqpoint{1.725490in}{1.575104in}}%
\pgfpathcurveto{\pgfqpoint{1.725490in}{1.564054in}}{\pgfqpoint{1.729881in}{1.553455in}}{\pgfqpoint{1.737694in}{1.545641in}}%
\pgfpathcurveto{\pgfqpoint{1.745508in}{1.537828in}}{\pgfqpoint{1.756107in}{1.533437in}}{\pgfqpoint{1.767157in}{1.533437in}}%
\pgfpathclose%
\pgfusepath{stroke,fill}%
\end{pgfscope}%
\begin{pgfscope}%
\pgfpathrectangle{\pgfqpoint{0.600000in}{0.600000in}}{\pgfqpoint{3.900000in}{3.900000in}}%
\pgfusepath{clip}%
\pgfsetbuttcap%
\pgfsetroundjoin%
\definecolor{currentfill}{rgb}{0.121569,0.466667,0.705882}%
\pgfsetfillcolor{currentfill}%
\pgfsetlinewidth{1.003750pt}%
\definecolor{currentstroke}{rgb}{0.121569,0.466667,0.705882}%
\pgfsetstrokecolor{currentstroke}%
\pgfsetdash{}{0pt}%
\pgfpathmoveto{\pgfqpoint{2.355360in}{2.716992in}}%
\pgfpathcurveto{\pgfqpoint{2.366410in}{2.716992in}}{\pgfqpoint{2.377009in}{2.721383in}}{\pgfqpoint{2.384822in}{2.729196in}}%
\pgfpathcurveto{\pgfqpoint{2.392636in}{2.737010in}}{\pgfqpoint{2.397026in}{2.747609in}}{\pgfqpoint{2.397026in}{2.758659in}}%
\pgfpathcurveto{\pgfqpoint{2.397026in}{2.769709in}}{\pgfqpoint{2.392636in}{2.780308in}}{\pgfqpoint{2.384822in}{2.788122in}}%
\pgfpathcurveto{\pgfqpoint{2.377009in}{2.795935in}}{\pgfqpoint{2.366410in}{2.800326in}}{\pgfqpoint{2.355360in}{2.800326in}}%
\pgfpathcurveto{\pgfqpoint{2.344309in}{2.800326in}}{\pgfqpoint{2.333710in}{2.795935in}}{\pgfqpoint{2.325897in}{2.788122in}}%
\pgfpathcurveto{\pgfqpoint{2.318083in}{2.780308in}}{\pgfqpoint{2.313693in}{2.769709in}}{\pgfqpoint{2.313693in}{2.758659in}}%
\pgfpathcurveto{\pgfqpoint{2.313693in}{2.747609in}}{\pgfqpoint{2.318083in}{2.737010in}}{\pgfqpoint{2.325897in}{2.729196in}}%
\pgfpathcurveto{\pgfqpoint{2.333710in}{2.721383in}}{\pgfqpoint{2.344309in}{2.716992in}}{\pgfqpoint{2.355360in}{2.716992in}}%
\pgfpathclose%
\pgfusepath{stroke,fill}%
\end{pgfscope}%
\begin{pgfscope}%
\pgfpathrectangle{\pgfqpoint{0.600000in}{0.600000in}}{\pgfqpoint{3.900000in}{3.900000in}}%
\pgfusepath{clip}%
\pgfsetbuttcap%
\pgfsetroundjoin%
\definecolor{currentfill}{rgb}{0.121569,0.466667,0.705882}%
\pgfsetfillcolor{currentfill}%
\pgfsetlinewidth{1.003750pt}%
\definecolor{currentstroke}{rgb}{0.121569,0.466667,0.705882}%
\pgfsetstrokecolor{currentstroke}%
\pgfsetdash{}{0pt}%
\pgfpathmoveto{\pgfqpoint{1.798904in}{2.516407in}}%
\pgfpathcurveto{\pgfqpoint{1.809954in}{2.516407in}}{\pgfqpoint{1.820553in}{2.520798in}}{\pgfqpoint{1.828367in}{2.528611in}}%
\pgfpathcurveto{\pgfqpoint{1.836180in}{2.536425in}}{\pgfqpoint{1.840571in}{2.547024in}}{\pgfqpoint{1.840571in}{2.558074in}}%
\pgfpathcurveto{\pgfqpoint{1.840571in}{2.569124in}}{\pgfqpoint{1.836180in}{2.579723in}}{\pgfqpoint{1.828367in}{2.587537in}}%
\pgfpathcurveto{\pgfqpoint{1.820553in}{2.595350in}}{\pgfqpoint{1.809954in}{2.599741in}}{\pgfqpoint{1.798904in}{2.599741in}}%
\pgfpathcurveto{\pgfqpoint{1.787854in}{2.599741in}}{\pgfqpoint{1.777255in}{2.595350in}}{\pgfqpoint{1.769441in}{2.587537in}}%
\pgfpathcurveto{\pgfqpoint{1.761628in}{2.579723in}}{\pgfqpoint{1.757237in}{2.569124in}}{\pgfqpoint{1.757237in}{2.558074in}}%
\pgfpathcurveto{\pgfqpoint{1.757237in}{2.547024in}}{\pgfqpoint{1.761628in}{2.536425in}}{\pgfqpoint{1.769441in}{2.528611in}}%
\pgfpathcurveto{\pgfqpoint{1.777255in}{2.520798in}}{\pgfqpoint{1.787854in}{2.516407in}}{\pgfqpoint{1.798904in}{2.516407in}}%
\pgfpathclose%
\pgfusepath{stroke,fill}%
\end{pgfscope}%
\begin{pgfscope}%
\pgfpathrectangle{\pgfqpoint{0.600000in}{0.600000in}}{\pgfqpoint{3.900000in}{3.900000in}}%
\pgfusepath{clip}%
\pgfsetbuttcap%
\pgfsetroundjoin%
\definecolor{currentfill}{rgb}{0.121569,0.466667,0.705882}%
\pgfsetfillcolor{currentfill}%
\pgfsetlinewidth{1.003750pt}%
\definecolor{currentstroke}{rgb}{0.121569,0.466667,0.705882}%
\pgfsetstrokecolor{currentstroke}%
\pgfsetdash{}{0pt}%
\pgfpathmoveto{\pgfqpoint{1.631334in}{2.748692in}}%
\pgfpathcurveto{\pgfqpoint{1.642384in}{2.748692in}}{\pgfqpoint{1.652983in}{2.753082in}}{\pgfqpoint{1.660797in}{2.760896in}}%
\pgfpathcurveto{\pgfqpoint{1.668611in}{2.768710in}}{\pgfqpoint{1.673001in}{2.779309in}}{\pgfqpoint{1.673001in}{2.790359in}}%
\pgfpathcurveto{\pgfqpoint{1.673001in}{2.801409in}}{\pgfqpoint{1.668611in}{2.812008in}}{\pgfqpoint{1.660797in}{2.819822in}}%
\pgfpathcurveto{\pgfqpoint{1.652983in}{2.827635in}}{\pgfqpoint{1.642384in}{2.832025in}}{\pgfqpoint{1.631334in}{2.832025in}}%
\pgfpathcurveto{\pgfqpoint{1.620284in}{2.832025in}}{\pgfqpoint{1.609685in}{2.827635in}}{\pgfqpoint{1.601871in}{2.819822in}}%
\pgfpathcurveto{\pgfqpoint{1.594058in}{2.812008in}}{\pgfqpoint{1.589668in}{2.801409in}}{\pgfqpoint{1.589668in}{2.790359in}}%
\pgfpathcurveto{\pgfqpoint{1.589668in}{2.779309in}}{\pgfqpoint{1.594058in}{2.768710in}}{\pgfqpoint{1.601871in}{2.760896in}}%
\pgfpathcurveto{\pgfqpoint{1.609685in}{2.753082in}}{\pgfqpoint{1.620284in}{2.748692in}}{\pgfqpoint{1.631334in}{2.748692in}}%
\pgfpathclose%
\pgfusepath{stroke,fill}%
\end{pgfscope}%
\begin{pgfscope}%
\pgfpathrectangle{\pgfqpoint{0.600000in}{0.600000in}}{\pgfqpoint{3.900000in}{3.900000in}}%
\pgfusepath{clip}%
\pgfsetbuttcap%
\pgfsetroundjoin%
\definecolor{currentfill}{rgb}{0.121569,0.466667,0.705882}%
\pgfsetfillcolor{currentfill}%
\pgfsetlinewidth{1.003750pt}%
\definecolor{currentstroke}{rgb}{0.121569,0.466667,0.705882}%
\pgfsetstrokecolor{currentstroke}%
\pgfsetdash{}{0pt}%
\pgfpathmoveto{\pgfqpoint{2.323539in}{2.689161in}}%
\pgfpathcurveto{\pgfqpoint{2.334589in}{2.689161in}}{\pgfqpoint{2.345188in}{2.693551in}}{\pgfqpoint{2.353002in}{2.701365in}}%
\pgfpathcurveto{\pgfqpoint{2.360815in}{2.709178in}}{\pgfqpoint{2.365206in}{2.719777in}}{\pgfqpoint{2.365206in}{2.730827in}}%
\pgfpathcurveto{\pgfqpoint{2.365206in}{2.741878in}}{\pgfqpoint{2.360815in}{2.752477in}}{\pgfqpoint{2.353002in}{2.760290in}}%
\pgfpathcurveto{\pgfqpoint{2.345188in}{2.768104in}}{\pgfqpoint{2.334589in}{2.772494in}}{\pgfqpoint{2.323539in}{2.772494in}}%
\pgfpathcurveto{\pgfqpoint{2.312489in}{2.772494in}}{\pgfqpoint{2.301890in}{2.768104in}}{\pgfqpoint{2.294076in}{2.760290in}}%
\pgfpathcurveto{\pgfqpoint{2.286263in}{2.752477in}}{\pgfqpoint{2.281872in}{2.741878in}}{\pgfqpoint{2.281872in}{2.730827in}}%
\pgfpathcurveto{\pgfqpoint{2.281872in}{2.719777in}}{\pgfqpoint{2.286263in}{2.709178in}}{\pgfqpoint{2.294076in}{2.701365in}}%
\pgfpathcurveto{\pgfqpoint{2.301890in}{2.693551in}}{\pgfqpoint{2.312489in}{2.689161in}}{\pgfqpoint{2.323539in}{2.689161in}}%
\pgfpathclose%
\pgfusepath{stroke,fill}%
\end{pgfscope}%
\begin{pgfscope}%
\pgfpathrectangle{\pgfqpoint{0.600000in}{0.600000in}}{\pgfqpoint{3.900000in}{3.900000in}}%
\pgfusepath{clip}%
\pgfsetbuttcap%
\pgfsetroundjoin%
\definecolor{currentfill}{rgb}{0.121569,0.466667,0.705882}%
\pgfsetfillcolor{currentfill}%
\pgfsetlinewidth{1.003750pt}%
\definecolor{currentstroke}{rgb}{0.121569,0.466667,0.705882}%
\pgfsetstrokecolor{currentstroke}%
\pgfsetdash{}{0pt}%
\pgfpathmoveto{\pgfqpoint{3.641858in}{2.725739in}}%
\pgfpathcurveto{\pgfqpoint{3.652908in}{2.725739in}}{\pgfqpoint{3.663507in}{2.730130in}}{\pgfqpoint{3.671321in}{2.737943in}}%
\pgfpathcurveto{\pgfqpoint{3.679134in}{2.745757in}}{\pgfqpoint{3.683524in}{2.756356in}}{\pgfqpoint{3.683524in}{2.767406in}}%
\pgfpathcurveto{\pgfqpoint{3.683524in}{2.778456in}}{\pgfqpoint{3.679134in}{2.789055in}}{\pgfqpoint{3.671321in}{2.796869in}}%
\pgfpathcurveto{\pgfqpoint{3.663507in}{2.804682in}}{\pgfqpoint{3.652908in}{2.809073in}}{\pgfqpoint{3.641858in}{2.809073in}}%
\pgfpathcurveto{\pgfqpoint{3.630808in}{2.809073in}}{\pgfqpoint{3.620209in}{2.804682in}}{\pgfqpoint{3.612395in}{2.796869in}}%
\pgfpathcurveto{\pgfqpoint{3.604581in}{2.789055in}}{\pgfqpoint{3.600191in}{2.778456in}}{\pgfqpoint{3.600191in}{2.767406in}}%
\pgfpathcurveto{\pgfqpoint{3.600191in}{2.756356in}}{\pgfqpoint{3.604581in}{2.745757in}}{\pgfqpoint{3.612395in}{2.737943in}}%
\pgfpathcurveto{\pgfqpoint{3.620209in}{2.730130in}}{\pgfqpoint{3.630808in}{2.725739in}}{\pgfqpoint{3.641858in}{2.725739in}}%
\pgfpathclose%
\pgfusepath{stroke,fill}%
\end{pgfscope}%
\begin{pgfscope}%
\pgfpathrectangle{\pgfqpoint{0.600000in}{0.600000in}}{\pgfqpoint{3.900000in}{3.900000in}}%
\pgfusepath{clip}%
\pgfsetbuttcap%
\pgfsetroundjoin%
\definecolor{currentfill}{rgb}{0.121569,0.466667,0.705882}%
\pgfsetfillcolor{currentfill}%
\pgfsetlinewidth{1.003750pt}%
\definecolor{currentstroke}{rgb}{0.121569,0.466667,0.705882}%
\pgfsetstrokecolor{currentstroke}%
\pgfsetdash{}{0pt}%
\pgfpathmoveto{\pgfqpoint{1.897351in}{2.699830in}}%
\pgfpathcurveto{\pgfqpoint{1.908402in}{2.699830in}}{\pgfqpoint{1.919001in}{2.704220in}}{\pgfqpoint{1.926814in}{2.712034in}}%
\pgfpathcurveto{\pgfqpoint{1.934628in}{2.719847in}}{\pgfqpoint{1.939018in}{2.730446in}}{\pgfqpoint{1.939018in}{2.741496in}}%
\pgfpathcurveto{\pgfqpoint{1.939018in}{2.752547in}}{\pgfqpoint{1.934628in}{2.763146in}}{\pgfqpoint{1.926814in}{2.770959in}}%
\pgfpathcurveto{\pgfqpoint{1.919001in}{2.778773in}}{\pgfqpoint{1.908402in}{2.783163in}}{\pgfqpoint{1.897351in}{2.783163in}}%
\pgfpathcurveto{\pgfqpoint{1.886301in}{2.783163in}}{\pgfqpoint{1.875702in}{2.778773in}}{\pgfqpoint{1.867889in}{2.770959in}}%
\pgfpathcurveto{\pgfqpoint{1.860075in}{2.763146in}}{\pgfqpoint{1.855685in}{2.752547in}}{\pgfqpoint{1.855685in}{2.741496in}}%
\pgfpathcurveto{\pgfqpoint{1.855685in}{2.730446in}}{\pgfqpoint{1.860075in}{2.719847in}}{\pgfqpoint{1.867889in}{2.712034in}}%
\pgfpathcurveto{\pgfqpoint{1.875702in}{2.704220in}}{\pgfqpoint{1.886301in}{2.699830in}}{\pgfqpoint{1.897351in}{2.699830in}}%
\pgfpathclose%
\pgfusepath{stroke,fill}%
\end{pgfscope}%
\begin{pgfscope}%
\pgfpathrectangle{\pgfqpoint{0.600000in}{0.600000in}}{\pgfqpoint{3.900000in}{3.900000in}}%
\pgfusepath{clip}%
\pgfsetbuttcap%
\pgfsetroundjoin%
\definecolor{currentfill}{rgb}{0.121569,0.466667,0.705882}%
\pgfsetfillcolor{currentfill}%
\pgfsetlinewidth{1.003750pt}%
\definecolor{currentstroke}{rgb}{0.121569,0.466667,0.705882}%
\pgfsetstrokecolor{currentstroke}%
\pgfsetdash{}{0pt}%
\pgfpathmoveto{\pgfqpoint{1.938440in}{2.703013in}}%
\pgfpathcurveto{\pgfqpoint{1.949490in}{2.703013in}}{\pgfqpoint{1.960089in}{2.707404in}}{\pgfqpoint{1.967902in}{2.715217in}}%
\pgfpathcurveto{\pgfqpoint{1.975716in}{2.723031in}}{\pgfqpoint{1.980106in}{2.733630in}}{\pgfqpoint{1.980106in}{2.744680in}}%
\pgfpathcurveto{\pgfqpoint{1.980106in}{2.755730in}}{\pgfqpoint{1.975716in}{2.766329in}}{\pgfqpoint{1.967902in}{2.774143in}}%
\pgfpathcurveto{\pgfqpoint{1.960089in}{2.781956in}}{\pgfqpoint{1.949490in}{2.786347in}}{\pgfqpoint{1.938440in}{2.786347in}}%
\pgfpathcurveto{\pgfqpoint{1.927390in}{2.786347in}}{\pgfqpoint{1.916791in}{2.781956in}}{\pgfqpoint{1.908977in}{2.774143in}}%
\pgfpathcurveto{\pgfqpoint{1.901163in}{2.766329in}}{\pgfqpoint{1.896773in}{2.755730in}}{\pgfqpoint{1.896773in}{2.744680in}}%
\pgfpathcurveto{\pgfqpoint{1.896773in}{2.733630in}}{\pgfqpoint{1.901163in}{2.723031in}}{\pgfqpoint{1.908977in}{2.715217in}}%
\pgfpathcurveto{\pgfqpoint{1.916791in}{2.707404in}}{\pgfqpoint{1.927390in}{2.703013in}}{\pgfqpoint{1.938440in}{2.703013in}}%
\pgfpathclose%
\pgfusepath{stroke,fill}%
\end{pgfscope}%
\begin{pgfscope}%
\pgfpathrectangle{\pgfqpoint{0.600000in}{0.600000in}}{\pgfqpoint{3.900000in}{3.900000in}}%
\pgfusepath{clip}%
\pgfsetbuttcap%
\pgfsetroundjoin%
\definecolor{currentfill}{rgb}{0.121569,0.466667,0.705882}%
\pgfsetfillcolor{currentfill}%
\pgfsetlinewidth{1.003750pt}%
\definecolor{currentstroke}{rgb}{0.121569,0.466667,0.705882}%
\pgfsetstrokecolor{currentstroke}%
\pgfsetdash{}{0pt}%
\pgfpathmoveto{\pgfqpoint{1.580400in}{2.055189in}}%
\pgfpathcurveto{\pgfqpoint{1.591450in}{2.055189in}}{\pgfqpoint{1.602049in}{2.059579in}}{\pgfqpoint{1.609863in}{2.067393in}}%
\pgfpathcurveto{\pgfqpoint{1.617676in}{2.075206in}}{\pgfqpoint{1.622067in}{2.085805in}}{\pgfqpoint{1.622067in}{2.096855in}}%
\pgfpathcurveto{\pgfqpoint{1.622067in}{2.107906in}}{\pgfqpoint{1.617676in}{2.118505in}}{\pgfqpoint{1.609863in}{2.126318in}}%
\pgfpathcurveto{\pgfqpoint{1.602049in}{2.134132in}}{\pgfqpoint{1.591450in}{2.138522in}}{\pgfqpoint{1.580400in}{2.138522in}}%
\pgfpathcurveto{\pgfqpoint{1.569350in}{2.138522in}}{\pgfqpoint{1.558751in}{2.134132in}}{\pgfqpoint{1.550937in}{2.126318in}}%
\pgfpathcurveto{\pgfqpoint{1.543124in}{2.118505in}}{\pgfqpoint{1.538733in}{2.107906in}}{\pgfqpoint{1.538733in}{2.096855in}}%
\pgfpathcurveto{\pgfqpoint{1.538733in}{2.085805in}}{\pgfqpoint{1.543124in}{2.075206in}}{\pgfqpoint{1.550937in}{2.067393in}}%
\pgfpathcurveto{\pgfqpoint{1.558751in}{2.059579in}}{\pgfqpoint{1.569350in}{2.055189in}}{\pgfqpoint{1.580400in}{2.055189in}}%
\pgfpathclose%
\pgfusepath{stroke,fill}%
\end{pgfscope}%
\begin{pgfscope}%
\pgfpathrectangle{\pgfqpoint{0.600000in}{0.600000in}}{\pgfqpoint{3.900000in}{3.900000in}}%
\pgfusepath{clip}%
\pgfsetbuttcap%
\pgfsetroundjoin%
\definecolor{currentfill}{rgb}{0.121569,0.466667,0.705882}%
\pgfsetfillcolor{currentfill}%
\pgfsetlinewidth{1.003750pt}%
\definecolor{currentstroke}{rgb}{0.121569,0.466667,0.705882}%
\pgfsetstrokecolor{currentstroke}%
\pgfsetdash{}{0pt}%
\pgfpathmoveto{\pgfqpoint{3.663099in}{2.966451in}}%
\pgfpathcurveto{\pgfqpoint{3.674149in}{2.966451in}}{\pgfqpoint{3.684748in}{2.970841in}}{\pgfqpoint{3.692561in}{2.978655in}}%
\pgfpathcurveto{\pgfqpoint{3.700375in}{2.986469in}}{\pgfqpoint{3.704765in}{2.997068in}}{\pgfqpoint{3.704765in}{3.008118in}}%
\pgfpathcurveto{\pgfqpoint{3.704765in}{3.019168in}}{\pgfqpoint{3.700375in}{3.029767in}}{\pgfqpoint{3.692561in}{3.037581in}}%
\pgfpathcurveto{\pgfqpoint{3.684748in}{3.045394in}}{\pgfqpoint{3.674149in}{3.049785in}}{\pgfqpoint{3.663099in}{3.049785in}}%
\pgfpathcurveto{\pgfqpoint{3.652048in}{3.049785in}}{\pgfqpoint{3.641449in}{3.045394in}}{\pgfqpoint{3.633636in}{3.037581in}}%
\pgfpathcurveto{\pgfqpoint{3.625822in}{3.029767in}}{\pgfqpoint{3.621432in}{3.019168in}}{\pgfqpoint{3.621432in}{3.008118in}}%
\pgfpathcurveto{\pgfqpoint{3.621432in}{2.997068in}}{\pgfqpoint{3.625822in}{2.986469in}}{\pgfqpoint{3.633636in}{2.978655in}}%
\pgfpathcurveto{\pgfqpoint{3.641449in}{2.970841in}}{\pgfqpoint{3.652048in}{2.966451in}}{\pgfqpoint{3.663099in}{2.966451in}}%
\pgfpathclose%
\pgfusepath{stroke,fill}%
\end{pgfscope}%
\begin{pgfscope}%
\pgfpathrectangle{\pgfqpoint{0.600000in}{0.600000in}}{\pgfqpoint{3.900000in}{3.900000in}}%
\pgfusepath{clip}%
\pgfsetbuttcap%
\pgfsetroundjoin%
\definecolor{currentfill}{rgb}{0.121569,0.466667,0.705882}%
\pgfsetfillcolor{currentfill}%
\pgfsetlinewidth{1.003750pt}%
\definecolor{currentstroke}{rgb}{0.121569,0.466667,0.705882}%
\pgfsetstrokecolor{currentstroke}%
\pgfsetdash{}{0pt}%
\pgfpathmoveto{\pgfqpoint{3.063888in}{2.675654in}}%
\pgfpathcurveto{\pgfqpoint{3.074938in}{2.675654in}}{\pgfqpoint{3.085538in}{2.680044in}}{\pgfqpoint{3.093351in}{2.687858in}}%
\pgfpathcurveto{\pgfqpoint{3.101165in}{2.695671in}}{\pgfqpoint{3.105555in}{2.706270in}}{\pgfqpoint{3.105555in}{2.717320in}}%
\pgfpathcurveto{\pgfqpoint{3.105555in}{2.728370in}}{\pgfqpoint{3.101165in}{2.738969in}}{\pgfqpoint{3.093351in}{2.746783in}}%
\pgfpathcurveto{\pgfqpoint{3.085538in}{2.754597in}}{\pgfqpoint{3.074938in}{2.758987in}}{\pgfqpoint{3.063888in}{2.758987in}}%
\pgfpathcurveto{\pgfqpoint{3.052838in}{2.758987in}}{\pgfqpoint{3.042239in}{2.754597in}}{\pgfqpoint{3.034426in}{2.746783in}}%
\pgfpathcurveto{\pgfqpoint{3.026612in}{2.738969in}}{\pgfqpoint{3.022222in}{2.728370in}}{\pgfqpoint{3.022222in}{2.717320in}}%
\pgfpathcurveto{\pgfqpoint{3.022222in}{2.706270in}}{\pgfqpoint{3.026612in}{2.695671in}}{\pgfqpoint{3.034426in}{2.687858in}}%
\pgfpathcurveto{\pgfqpoint{3.042239in}{2.680044in}}{\pgfqpoint{3.052838in}{2.675654in}}{\pgfqpoint{3.063888in}{2.675654in}}%
\pgfpathclose%
\pgfusepath{stroke,fill}%
\end{pgfscope}%
\begin{pgfscope}%
\pgfpathrectangle{\pgfqpoint{0.600000in}{0.600000in}}{\pgfqpoint{3.900000in}{3.900000in}}%
\pgfusepath{clip}%
\pgfsetbuttcap%
\pgfsetroundjoin%
\definecolor{currentfill}{rgb}{0.121569,0.466667,0.705882}%
\pgfsetfillcolor{currentfill}%
\pgfsetlinewidth{1.003750pt}%
\definecolor{currentstroke}{rgb}{0.121569,0.466667,0.705882}%
\pgfsetstrokecolor{currentstroke}%
\pgfsetdash{}{0pt}%
\pgfpathmoveto{\pgfqpoint{2.485315in}{1.699251in}}%
\pgfpathcurveto{\pgfqpoint{2.496365in}{1.699251in}}{\pgfqpoint{2.506964in}{1.703641in}}{\pgfqpoint{2.514777in}{1.711455in}}%
\pgfpathcurveto{\pgfqpoint{2.522591in}{1.719268in}}{\pgfqpoint{2.526981in}{1.729867in}}{\pgfqpoint{2.526981in}{1.740917in}}%
\pgfpathcurveto{\pgfqpoint{2.526981in}{1.751967in}}{\pgfqpoint{2.522591in}{1.762567in}}{\pgfqpoint{2.514777in}{1.770380in}}%
\pgfpathcurveto{\pgfqpoint{2.506964in}{1.778194in}}{\pgfqpoint{2.496365in}{1.782584in}}{\pgfqpoint{2.485315in}{1.782584in}}%
\pgfpathcurveto{\pgfqpoint{2.474264in}{1.782584in}}{\pgfqpoint{2.463665in}{1.778194in}}{\pgfqpoint{2.455852in}{1.770380in}}%
\pgfpathcurveto{\pgfqpoint{2.448038in}{1.762567in}}{\pgfqpoint{2.443648in}{1.751967in}}{\pgfqpoint{2.443648in}{1.740917in}}%
\pgfpathcurveto{\pgfqpoint{2.443648in}{1.729867in}}{\pgfqpoint{2.448038in}{1.719268in}}{\pgfqpoint{2.455852in}{1.711455in}}%
\pgfpathcurveto{\pgfqpoint{2.463665in}{1.703641in}}{\pgfqpoint{2.474264in}{1.699251in}}{\pgfqpoint{2.485315in}{1.699251in}}%
\pgfpathclose%
\pgfusepath{stroke,fill}%
\end{pgfscope}%
\begin{pgfscope}%
\pgfpathrectangle{\pgfqpoint{0.600000in}{0.600000in}}{\pgfqpoint{3.900000in}{3.900000in}}%
\pgfusepath{clip}%
\pgfsetbuttcap%
\pgfsetroundjoin%
\definecolor{currentfill}{rgb}{0.121569,0.466667,0.705882}%
\pgfsetfillcolor{currentfill}%
\pgfsetlinewidth{1.003750pt}%
\definecolor{currentstroke}{rgb}{0.121569,0.466667,0.705882}%
\pgfsetstrokecolor{currentstroke}%
\pgfsetdash{}{0pt}%
\pgfpathmoveto{\pgfqpoint{2.364221in}{1.319184in}}%
\pgfpathcurveto{\pgfqpoint{2.375271in}{1.319184in}}{\pgfqpoint{2.385870in}{1.323574in}}{\pgfqpoint{2.393684in}{1.331388in}}%
\pgfpathcurveto{\pgfqpoint{2.401497in}{1.339201in}}{\pgfqpoint{2.405887in}{1.349800in}}{\pgfqpoint{2.405887in}{1.360850in}}%
\pgfpathcurveto{\pgfqpoint{2.405887in}{1.371901in}}{\pgfqpoint{2.401497in}{1.382500in}}{\pgfqpoint{2.393684in}{1.390313in}}%
\pgfpathcurveto{\pgfqpoint{2.385870in}{1.398127in}}{\pgfqpoint{2.375271in}{1.402517in}}{\pgfqpoint{2.364221in}{1.402517in}}%
\pgfpathcurveto{\pgfqpoint{2.353171in}{1.402517in}}{\pgfqpoint{2.342572in}{1.398127in}}{\pgfqpoint{2.334758in}{1.390313in}}%
\pgfpathcurveto{\pgfqpoint{2.326944in}{1.382500in}}{\pgfqpoint{2.322554in}{1.371901in}}{\pgfqpoint{2.322554in}{1.360850in}}%
\pgfpathcurveto{\pgfqpoint{2.322554in}{1.349800in}}{\pgfqpoint{2.326944in}{1.339201in}}{\pgfqpoint{2.334758in}{1.331388in}}%
\pgfpathcurveto{\pgfqpoint{2.342572in}{1.323574in}}{\pgfqpoint{2.353171in}{1.319184in}}{\pgfqpoint{2.364221in}{1.319184in}}%
\pgfpathclose%
\pgfusepath{stroke,fill}%
\end{pgfscope}%
\begin{pgfscope}%
\pgfpathrectangle{\pgfqpoint{0.600000in}{0.600000in}}{\pgfqpoint{3.900000in}{3.900000in}}%
\pgfusepath{clip}%
\pgfsetbuttcap%
\pgfsetroundjoin%
\definecolor{currentfill}{rgb}{0.121569,0.466667,0.705882}%
\pgfsetfillcolor{currentfill}%
\pgfsetlinewidth{1.003750pt}%
\definecolor{currentstroke}{rgb}{0.121569,0.466667,0.705882}%
\pgfsetstrokecolor{currentstroke}%
\pgfsetdash{}{0pt}%
\pgfpathmoveto{\pgfqpoint{1.947295in}{2.482852in}}%
\pgfpathcurveto{\pgfqpoint{1.958345in}{2.482852in}}{\pgfqpoint{1.968944in}{2.487242in}}{\pgfqpoint{1.976758in}{2.495056in}}%
\pgfpathcurveto{\pgfqpoint{1.984572in}{2.502870in}}{\pgfqpoint{1.988962in}{2.513469in}}{\pgfqpoint{1.988962in}{2.524519in}}%
\pgfpathcurveto{\pgfqpoint{1.988962in}{2.535569in}}{\pgfqpoint{1.984572in}{2.546168in}}{\pgfqpoint{1.976758in}{2.553982in}}%
\pgfpathcurveto{\pgfqpoint{1.968944in}{2.561795in}}{\pgfqpoint{1.958345in}{2.566185in}}{\pgfqpoint{1.947295in}{2.566185in}}%
\pgfpathcurveto{\pgfqpoint{1.936245in}{2.566185in}}{\pgfqpoint{1.925646in}{2.561795in}}{\pgfqpoint{1.917832in}{2.553982in}}%
\pgfpathcurveto{\pgfqpoint{1.910019in}{2.546168in}}{\pgfqpoint{1.905629in}{2.535569in}}{\pgfqpoint{1.905629in}{2.524519in}}%
\pgfpathcurveto{\pgfqpoint{1.905629in}{2.513469in}}{\pgfqpoint{1.910019in}{2.502870in}}{\pgfqpoint{1.917832in}{2.495056in}}%
\pgfpathcurveto{\pgfqpoint{1.925646in}{2.487242in}}{\pgfqpoint{1.936245in}{2.482852in}}{\pgfqpoint{1.947295in}{2.482852in}}%
\pgfpathclose%
\pgfusepath{stroke,fill}%
\end{pgfscope}%
\begin{pgfscope}%
\pgfpathrectangle{\pgfqpoint{0.600000in}{0.600000in}}{\pgfqpoint{3.900000in}{3.900000in}}%
\pgfusepath{clip}%
\pgfsetbuttcap%
\pgfsetroundjoin%
\definecolor{currentfill}{rgb}{0.121569,0.466667,0.705882}%
\pgfsetfillcolor{currentfill}%
\pgfsetlinewidth{1.003750pt}%
\definecolor{currentstroke}{rgb}{0.121569,0.466667,0.705882}%
\pgfsetstrokecolor{currentstroke}%
\pgfsetdash{}{0pt}%
\pgfpathmoveto{\pgfqpoint{2.048375in}{2.400249in}}%
\pgfpathcurveto{\pgfqpoint{2.059425in}{2.400249in}}{\pgfqpoint{2.070024in}{2.404640in}}{\pgfqpoint{2.077837in}{2.412453in}}%
\pgfpathcurveto{\pgfqpoint{2.085651in}{2.420267in}}{\pgfqpoint{2.090041in}{2.430866in}}{\pgfqpoint{2.090041in}{2.441916in}}%
\pgfpathcurveto{\pgfqpoint{2.090041in}{2.452966in}}{\pgfqpoint{2.085651in}{2.463565in}}{\pgfqpoint{2.077837in}{2.471379in}}%
\pgfpathcurveto{\pgfqpoint{2.070024in}{2.479192in}}{\pgfqpoint{2.059425in}{2.483583in}}{\pgfqpoint{2.048375in}{2.483583in}}%
\pgfpathcurveto{\pgfqpoint{2.037324in}{2.483583in}}{\pgfqpoint{2.026725in}{2.479192in}}{\pgfqpoint{2.018912in}{2.471379in}}%
\pgfpathcurveto{\pgfqpoint{2.011098in}{2.463565in}}{\pgfqpoint{2.006708in}{2.452966in}}{\pgfqpoint{2.006708in}{2.441916in}}%
\pgfpathcurveto{\pgfqpoint{2.006708in}{2.430866in}}{\pgfqpoint{2.011098in}{2.420267in}}{\pgfqpoint{2.018912in}{2.412453in}}%
\pgfpathcurveto{\pgfqpoint{2.026725in}{2.404640in}}{\pgfqpoint{2.037324in}{2.400249in}}{\pgfqpoint{2.048375in}{2.400249in}}%
\pgfpathclose%
\pgfusepath{stroke,fill}%
\end{pgfscope}%
\begin{pgfscope}%
\pgfpathrectangle{\pgfqpoint{0.600000in}{0.600000in}}{\pgfqpoint{3.900000in}{3.900000in}}%
\pgfusepath{clip}%
\pgfsetbuttcap%
\pgfsetroundjoin%
\definecolor{currentfill}{rgb}{0.121569,0.466667,0.705882}%
\pgfsetfillcolor{currentfill}%
\pgfsetlinewidth{1.003750pt}%
\definecolor{currentstroke}{rgb}{0.121569,0.466667,0.705882}%
\pgfsetstrokecolor{currentstroke}%
\pgfsetdash{}{0pt}%
\pgfpathmoveto{\pgfqpoint{2.405716in}{3.564827in}}%
\pgfpathcurveto{\pgfqpoint{2.416766in}{3.564827in}}{\pgfqpoint{2.427365in}{3.569217in}}{\pgfqpoint{2.435179in}{3.577031in}}%
\pgfpathcurveto{\pgfqpoint{2.442992in}{3.584844in}}{\pgfqpoint{2.447383in}{3.595443in}}{\pgfqpoint{2.447383in}{3.606493in}}%
\pgfpathcurveto{\pgfqpoint{2.447383in}{3.617544in}}{\pgfqpoint{2.442992in}{3.628143in}}{\pgfqpoint{2.435179in}{3.635956in}}%
\pgfpathcurveto{\pgfqpoint{2.427365in}{3.643770in}}{\pgfqpoint{2.416766in}{3.648160in}}{\pgfqpoint{2.405716in}{3.648160in}}%
\pgfpathcurveto{\pgfqpoint{2.394666in}{3.648160in}}{\pgfqpoint{2.384067in}{3.643770in}}{\pgfqpoint{2.376253in}{3.635956in}}%
\pgfpathcurveto{\pgfqpoint{2.368440in}{3.628143in}}{\pgfqpoint{2.364049in}{3.617544in}}{\pgfqpoint{2.364049in}{3.606493in}}%
\pgfpathcurveto{\pgfqpoint{2.364049in}{3.595443in}}{\pgfqpoint{2.368440in}{3.584844in}}{\pgfqpoint{2.376253in}{3.577031in}}%
\pgfpathcurveto{\pgfqpoint{2.384067in}{3.569217in}}{\pgfqpoint{2.394666in}{3.564827in}}{\pgfqpoint{2.405716in}{3.564827in}}%
\pgfpathclose%
\pgfusepath{stroke,fill}%
\end{pgfscope}%
\begin{pgfscope}%
\pgfpathrectangle{\pgfqpoint{0.600000in}{0.600000in}}{\pgfqpoint{3.900000in}{3.900000in}}%
\pgfusepath{clip}%
\pgfsetbuttcap%
\pgfsetroundjoin%
\definecolor{currentfill}{rgb}{0.121569,0.466667,0.705882}%
\pgfsetfillcolor{currentfill}%
\pgfsetlinewidth{1.003750pt}%
\definecolor{currentstroke}{rgb}{0.121569,0.466667,0.705882}%
\pgfsetstrokecolor{currentstroke}%
\pgfsetdash{}{0pt}%
\pgfpathmoveto{\pgfqpoint{2.315443in}{3.376432in}}%
\pgfpathcurveto{\pgfqpoint{2.326493in}{3.376432in}}{\pgfqpoint{2.337092in}{3.380822in}}{\pgfqpoint{2.344905in}{3.388636in}}%
\pgfpathcurveto{\pgfqpoint{2.352719in}{3.396449in}}{\pgfqpoint{2.357109in}{3.407048in}}{\pgfqpoint{2.357109in}{3.418099in}}%
\pgfpathcurveto{\pgfqpoint{2.357109in}{3.429149in}}{\pgfqpoint{2.352719in}{3.439748in}}{\pgfqpoint{2.344905in}{3.447561in}}%
\pgfpathcurveto{\pgfqpoint{2.337092in}{3.455375in}}{\pgfqpoint{2.326493in}{3.459765in}}{\pgfqpoint{2.315443in}{3.459765in}}%
\pgfpathcurveto{\pgfqpoint{2.304393in}{3.459765in}}{\pgfqpoint{2.293793in}{3.455375in}}{\pgfqpoint{2.285980in}{3.447561in}}%
\pgfpathcurveto{\pgfqpoint{2.278166in}{3.439748in}}{\pgfqpoint{2.273776in}{3.429149in}}{\pgfqpoint{2.273776in}{3.418099in}}%
\pgfpathcurveto{\pgfqpoint{2.273776in}{3.407048in}}{\pgfqpoint{2.278166in}{3.396449in}}{\pgfqpoint{2.285980in}{3.388636in}}%
\pgfpathcurveto{\pgfqpoint{2.293793in}{3.380822in}}{\pgfqpoint{2.304393in}{3.376432in}}{\pgfqpoint{2.315443in}{3.376432in}}%
\pgfpathclose%
\pgfusepath{stroke,fill}%
\end{pgfscope}%
\begin{pgfscope}%
\pgfpathrectangle{\pgfqpoint{0.600000in}{0.600000in}}{\pgfqpoint{3.900000in}{3.900000in}}%
\pgfusepath{clip}%
\pgfsetbuttcap%
\pgfsetroundjoin%
\definecolor{currentfill}{rgb}{0.121569,0.466667,0.705882}%
\pgfsetfillcolor{currentfill}%
\pgfsetlinewidth{1.003750pt}%
\definecolor{currentstroke}{rgb}{0.121569,0.466667,0.705882}%
\pgfsetstrokecolor{currentstroke}%
\pgfsetdash{}{0pt}%
\pgfpathmoveto{\pgfqpoint{2.521294in}{2.542640in}}%
\pgfpathcurveto{\pgfqpoint{2.532344in}{2.542640in}}{\pgfqpoint{2.542943in}{2.547030in}}{\pgfqpoint{2.550756in}{2.554844in}}%
\pgfpathcurveto{\pgfqpoint{2.558570in}{2.562657in}}{\pgfqpoint{2.562960in}{2.573256in}}{\pgfqpoint{2.562960in}{2.584306in}}%
\pgfpathcurveto{\pgfqpoint{2.562960in}{2.595356in}}{\pgfqpoint{2.558570in}{2.605955in}}{\pgfqpoint{2.550756in}{2.613769in}}%
\pgfpathcurveto{\pgfqpoint{2.542943in}{2.621583in}}{\pgfqpoint{2.532344in}{2.625973in}}{\pgfqpoint{2.521294in}{2.625973in}}%
\pgfpathcurveto{\pgfqpoint{2.510243in}{2.625973in}}{\pgfqpoint{2.499644in}{2.621583in}}{\pgfqpoint{2.491831in}{2.613769in}}%
\pgfpathcurveto{\pgfqpoint{2.484017in}{2.605955in}}{\pgfqpoint{2.479627in}{2.595356in}}{\pgfqpoint{2.479627in}{2.584306in}}%
\pgfpathcurveto{\pgfqpoint{2.479627in}{2.573256in}}{\pgfqpoint{2.484017in}{2.562657in}}{\pgfqpoint{2.491831in}{2.554844in}}%
\pgfpathcurveto{\pgfqpoint{2.499644in}{2.547030in}}{\pgfqpoint{2.510243in}{2.542640in}}{\pgfqpoint{2.521294in}{2.542640in}}%
\pgfpathclose%
\pgfusepath{stroke,fill}%
\end{pgfscope}%
\begin{pgfscope}%
\pgfpathrectangle{\pgfqpoint{0.600000in}{0.600000in}}{\pgfqpoint{3.900000in}{3.900000in}}%
\pgfusepath{clip}%
\pgfsetbuttcap%
\pgfsetroundjoin%
\definecolor{currentfill}{rgb}{0.121569,0.466667,0.705882}%
\pgfsetfillcolor{currentfill}%
\pgfsetlinewidth{1.003750pt}%
\definecolor{currentstroke}{rgb}{0.121569,0.466667,0.705882}%
\pgfsetstrokecolor{currentstroke}%
\pgfsetdash{}{0pt}%
\pgfpathmoveto{\pgfqpoint{2.710846in}{2.199142in}}%
\pgfpathcurveto{\pgfqpoint{2.721897in}{2.199142in}}{\pgfqpoint{2.732496in}{2.203532in}}{\pgfqpoint{2.740309in}{2.211346in}}%
\pgfpathcurveto{\pgfqpoint{2.748123in}{2.219160in}}{\pgfqpoint{2.752513in}{2.229759in}}{\pgfqpoint{2.752513in}{2.240809in}}%
\pgfpathcurveto{\pgfqpoint{2.752513in}{2.251859in}}{\pgfqpoint{2.748123in}{2.262458in}}{\pgfqpoint{2.740309in}{2.270272in}}%
\pgfpathcurveto{\pgfqpoint{2.732496in}{2.278085in}}{\pgfqpoint{2.721897in}{2.282475in}}{\pgfqpoint{2.710846in}{2.282475in}}%
\pgfpathcurveto{\pgfqpoint{2.699796in}{2.282475in}}{\pgfqpoint{2.689197in}{2.278085in}}{\pgfqpoint{2.681384in}{2.270272in}}%
\pgfpathcurveto{\pgfqpoint{2.673570in}{2.262458in}}{\pgfqpoint{2.669180in}{2.251859in}}{\pgfqpoint{2.669180in}{2.240809in}}%
\pgfpathcurveto{\pgfqpoint{2.669180in}{2.229759in}}{\pgfqpoint{2.673570in}{2.219160in}}{\pgfqpoint{2.681384in}{2.211346in}}%
\pgfpathcurveto{\pgfqpoint{2.689197in}{2.203532in}}{\pgfqpoint{2.699796in}{2.199142in}}{\pgfqpoint{2.710846in}{2.199142in}}%
\pgfpathclose%
\pgfusepath{stroke,fill}%
\end{pgfscope}%
\begin{pgfscope}%
\pgfpathrectangle{\pgfqpoint{0.600000in}{0.600000in}}{\pgfqpoint{3.900000in}{3.900000in}}%
\pgfusepath{clip}%
\pgfsetbuttcap%
\pgfsetroundjoin%
\definecolor{currentfill}{rgb}{0.121569,0.466667,0.705882}%
\pgfsetfillcolor{currentfill}%
\pgfsetlinewidth{1.003750pt}%
\definecolor{currentstroke}{rgb}{0.121569,0.466667,0.705882}%
\pgfsetstrokecolor{currentstroke}%
\pgfsetdash{}{0pt}%
\pgfpathmoveto{\pgfqpoint{1.945023in}{4.151712in}}%
\pgfpathcurveto{\pgfqpoint{1.956073in}{4.151712in}}{\pgfqpoint{1.966672in}{4.156102in}}{\pgfqpoint{1.974486in}{4.163916in}}%
\pgfpathcurveto{\pgfqpoint{1.982299in}{4.171730in}}{\pgfqpoint{1.986690in}{4.182329in}}{\pgfqpoint{1.986690in}{4.193379in}}%
\pgfpathcurveto{\pgfqpoint{1.986690in}{4.204429in}}{\pgfqpoint{1.982299in}{4.215028in}}{\pgfqpoint{1.974486in}{4.222842in}}%
\pgfpathcurveto{\pgfqpoint{1.966672in}{4.230655in}}{\pgfqpoint{1.956073in}{4.235046in}}{\pgfqpoint{1.945023in}{4.235046in}}%
\pgfpathcurveto{\pgfqpoint{1.933973in}{4.235046in}}{\pgfqpoint{1.923374in}{4.230655in}}{\pgfqpoint{1.915560in}{4.222842in}}%
\pgfpathcurveto{\pgfqpoint{1.907746in}{4.215028in}}{\pgfqpoint{1.903356in}{4.204429in}}{\pgfqpoint{1.903356in}{4.193379in}}%
\pgfpathcurveto{\pgfqpoint{1.903356in}{4.182329in}}{\pgfqpoint{1.907746in}{4.171730in}}{\pgfqpoint{1.915560in}{4.163916in}}%
\pgfpathcurveto{\pgfqpoint{1.923374in}{4.156102in}}{\pgfqpoint{1.933973in}{4.151712in}}{\pgfqpoint{1.945023in}{4.151712in}}%
\pgfpathclose%
\pgfusepath{stroke,fill}%
\end{pgfscope}%
\begin{pgfscope}%
\pgfpathrectangle{\pgfqpoint{0.600000in}{0.600000in}}{\pgfqpoint{3.900000in}{3.900000in}}%
\pgfusepath{clip}%
\pgfsetbuttcap%
\pgfsetroundjoin%
\definecolor{currentfill}{rgb}{0.121569,0.466667,0.705882}%
\pgfsetfillcolor{currentfill}%
\pgfsetlinewidth{1.003750pt}%
\definecolor{currentstroke}{rgb}{0.121569,0.466667,0.705882}%
\pgfsetstrokecolor{currentstroke}%
\pgfsetdash{}{0pt}%
\pgfpathmoveto{\pgfqpoint{3.115384in}{2.040518in}}%
\pgfpathcurveto{\pgfqpoint{3.126435in}{2.040518in}}{\pgfqpoint{3.137034in}{2.044908in}}{\pgfqpoint{3.144847in}{2.052722in}}%
\pgfpathcurveto{\pgfqpoint{3.152661in}{2.060535in}}{\pgfqpoint{3.157051in}{2.071134in}}{\pgfqpoint{3.157051in}{2.082184in}}%
\pgfpathcurveto{\pgfqpoint{3.157051in}{2.093235in}}{\pgfqpoint{3.152661in}{2.103834in}}{\pgfqpoint{3.144847in}{2.111647in}}%
\pgfpathcurveto{\pgfqpoint{3.137034in}{2.119461in}}{\pgfqpoint{3.126435in}{2.123851in}}{\pgfqpoint{3.115384in}{2.123851in}}%
\pgfpathcurveto{\pgfqpoint{3.104334in}{2.123851in}}{\pgfqpoint{3.093735in}{2.119461in}}{\pgfqpoint{3.085922in}{2.111647in}}%
\pgfpathcurveto{\pgfqpoint{3.078108in}{2.103834in}}{\pgfqpoint{3.073718in}{2.093235in}}{\pgfqpoint{3.073718in}{2.082184in}}%
\pgfpathcurveto{\pgfqpoint{3.073718in}{2.071134in}}{\pgfqpoint{3.078108in}{2.060535in}}{\pgfqpoint{3.085922in}{2.052722in}}%
\pgfpathcurveto{\pgfqpoint{3.093735in}{2.044908in}}{\pgfqpoint{3.104334in}{2.040518in}}{\pgfqpoint{3.115384in}{2.040518in}}%
\pgfpathclose%
\pgfusepath{stroke,fill}%
\end{pgfscope}%
\begin{pgfscope}%
\pgfpathrectangle{\pgfqpoint{0.600000in}{0.600000in}}{\pgfqpoint{3.900000in}{3.900000in}}%
\pgfusepath{clip}%
\pgfsetbuttcap%
\pgfsetroundjoin%
\definecolor{currentfill}{rgb}{0.121569,0.466667,0.705882}%
\pgfsetfillcolor{currentfill}%
\pgfsetlinewidth{1.003750pt}%
\definecolor{currentstroke}{rgb}{0.121569,0.466667,0.705882}%
\pgfsetstrokecolor{currentstroke}%
\pgfsetdash{}{0pt}%
\pgfpathmoveto{\pgfqpoint{2.929895in}{2.960483in}}%
\pgfpathcurveto{\pgfqpoint{2.940945in}{2.960483in}}{\pgfqpoint{2.951544in}{2.964873in}}{\pgfqpoint{2.959358in}{2.972687in}}%
\pgfpathcurveto{\pgfqpoint{2.967172in}{2.980501in}}{\pgfqpoint{2.971562in}{2.991100in}}{\pgfqpoint{2.971562in}{3.002150in}}%
\pgfpathcurveto{\pgfqpoint{2.971562in}{3.013200in}}{\pgfqpoint{2.967172in}{3.023799in}}{\pgfqpoint{2.959358in}{3.031613in}}%
\pgfpathcurveto{\pgfqpoint{2.951544in}{3.039426in}}{\pgfqpoint{2.940945in}{3.043817in}}{\pgfqpoint{2.929895in}{3.043817in}}%
\pgfpathcurveto{\pgfqpoint{2.918845in}{3.043817in}}{\pgfqpoint{2.908246in}{3.039426in}}{\pgfqpoint{2.900433in}{3.031613in}}%
\pgfpathcurveto{\pgfqpoint{2.892619in}{3.023799in}}{\pgfqpoint{2.888229in}{3.013200in}}{\pgfqpoint{2.888229in}{3.002150in}}%
\pgfpathcurveto{\pgfqpoint{2.888229in}{2.991100in}}{\pgfqpoint{2.892619in}{2.980501in}}{\pgfqpoint{2.900433in}{2.972687in}}%
\pgfpathcurveto{\pgfqpoint{2.908246in}{2.964873in}}{\pgfqpoint{2.918845in}{2.960483in}}{\pgfqpoint{2.929895in}{2.960483in}}%
\pgfpathclose%
\pgfusepath{stroke,fill}%
\end{pgfscope}%
\begin{pgfscope}%
\pgfpathrectangle{\pgfqpoint{0.600000in}{0.600000in}}{\pgfqpoint{3.900000in}{3.900000in}}%
\pgfusepath{clip}%
\pgfsetbuttcap%
\pgfsetroundjoin%
\definecolor{currentfill}{rgb}{0.121569,0.466667,0.705882}%
\pgfsetfillcolor{currentfill}%
\pgfsetlinewidth{1.003750pt}%
\definecolor{currentstroke}{rgb}{0.121569,0.466667,0.705882}%
\pgfsetstrokecolor{currentstroke}%
\pgfsetdash{}{0pt}%
\pgfpathmoveto{\pgfqpoint{3.237358in}{3.186259in}}%
\pgfpathcurveto{\pgfqpoint{3.248408in}{3.186259in}}{\pgfqpoint{3.259007in}{3.190650in}}{\pgfqpoint{3.266820in}{3.198463in}}%
\pgfpathcurveto{\pgfqpoint{3.274634in}{3.206277in}}{\pgfqpoint{3.279024in}{3.216876in}}{\pgfqpoint{3.279024in}{3.227926in}}%
\pgfpathcurveto{\pgfqpoint{3.279024in}{3.238976in}}{\pgfqpoint{3.274634in}{3.249575in}}{\pgfqpoint{3.266820in}{3.257389in}}%
\pgfpathcurveto{\pgfqpoint{3.259007in}{3.265202in}}{\pgfqpoint{3.248408in}{3.269593in}}{\pgfqpoint{3.237358in}{3.269593in}}%
\pgfpathcurveto{\pgfqpoint{3.226307in}{3.269593in}}{\pgfqpoint{3.215708in}{3.265202in}}{\pgfqpoint{3.207895in}{3.257389in}}%
\pgfpathcurveto{\pgfqpoint{3.200081in}{3.249575in}}{\pgfqpoint{3.195691in}{3.238976in}}{\pgfqpoint{3.195691in}{3.227926in}}%
\pgfpathcurveto{\pgfqpoint{3.195691in}{3.216876in}}{\pgfqpoint{3.200081in}{3.206277in}}{\pgfqpoint{3.207895in}{3.198463in}}%
\pgfpathcurveto{\pgfqpoint{3.215708in}{3.190650in}}{\pgfqpoint{3.226307in}{3.186259in}}{\pgfqpoint{3.237358in}{3.186259in}}%
\pgfpathclose%
\pgfusepath{stroke,fill}%
\end{pgfscope}%
\begin{pgfscope}%
\pgfpathrectangle{\pgfqpoint{0.600000in}{0.600000in}}{\pgfqpoint{3.900000in}{3.900000in}}%
\pgfusepath{clip}%
\pgfsetbuttcap%
\pgfsetroundjoin%
\definecolor{currentfill}{rgb}{0.121569,0.466667,0.705882}%
\pgfsetfillcolor{currentfill}%
\pgfsetlinewidth{1.003750pt}%
\definecolor{currentstroke}{rgb}{0.121569,0.466667,0.705882}%
\pgfsetstrokecolor{currentstroke}%
\pgfsetdash{}{0pt}%
\pgfpathmoveto{\pgfqpoint{2.378451in}{2.077515in}}%
\pgfpathcurveto{\pgfqpoint{2.389501in}{2.077515in}}{\pgfqpoint{2.400100in}{2.081906in}}{\pgfqpoint{2.407914in}{2.089719in}}%
\pgfpathcurveto{\pgfqpoint{2.415727in}{2.097533in}}{\pgfqpoint{2.420117in}{2.108132in}}{\pgfqpoint{2.420117in}{2.119182in}}%
\pgfpathcurveto{\pgfqpoint{2.420117in}{2.130232in}}{\pgfqpoint{2.415727in}{2.140831in}}{\pgfqpoint{2.407914in}{2.148645in}}%
\pgfpathcurveto{\pgfqpoint{2.400100in}{2.156459in}}{\pgfqpoint{2.389501in}{2.160849in}}{\pgfqpoint{2.378451in}{2.160849in}}%
\pgfpathcurveto{\pgfqpoint{2.367401in}{2.160849in}}{\pgfqpoint{2.356802in}{2.156459in}}{\pgfqpoint{2.348988in}{2.148645in}}%
\pgfpathcurveto{\pgfqpoint{2.341174in}{2.140831in}}{\pgfqpoint{2.336784in}{2.130232in}}{\pgfqpoint{2.336784in}{2.119182in}}%
\pgfpathcurveto{\pgfqpoint{2.336784in}{2.108132in}}{\pgfqpoint{2.341174in}{2.097533in}}{\pgfqpoint{2.348988in}{2.089719in}}%
\pgfpathcurveto{\pgfqpoint{2.356802in}{2.081906in}}{\pgfqpoint{2.367401in}{2.077515in}}{\pgfqpoint{2.378451in}{2.077515in}}%
\pgfpathclose%
\pgfusepath{stroke,fill}%
\end{pgfscope}%
\begin{pgfscope}%
\pgfpathrectangle{\pgfqpoint{0.600000in}{0.600000in}}{\pgfqpoint{3.900000in}{3.900000in}}%
\pgfusepath{clip}%
\pgfsetbuttcap%
\pgfsetroundjoin%
\definecolor{currentfill}{rgb}{0.121569,0.466667,0.705882}%
\pgfsetfillcolor{currentfill}%
\pgfsetlinewidth{1.003750pt}%
\definecolor{currentstroke}{rgb}{0.121569,0.466667,0.705882}%
\pgfsetstrokecolor{currentstroke}%
\pgfsetdash{}{0pt}%
\pgfpathmoveto{\pgfqpoint{3.903117in}{1.881328in}}%
\pgfpathcurveto{\pgfqpoint{3.914167in}{1.881328in}}{\pgfqpoint{3.924766in}{1.885718in}}{\pgfqpoint{3.932580in}{1.893532in}}%
\pgfpathcurveto{\pgfqpoint{3.940394in}{1.901345in}}{\pgfqpoint{3.944784in}{1.911944in}}{\pgfqpoint{3.944784in}{1.922995in}}%
\pgfpathcurveto{\pgfqpoint{3.944784in}{1.934045in}}{\pgfqpoint{3.940394in}{1.944644in}}{\pgfqpoint{3.932580in}{1.952457in}}%
\pgfpathcurveto{\pgfqpoint{3.924766in}{1.960271in}}{\pgfqpoint{3.914167in}{1.964661in}}{\pgfqpoint{3.903117in}{1.964661in}}%
\pgfpathcurveto{\pgfqpoint{3.892067in}{1.964661in}}{\pgfqpoint{3.881468in}{1.960271in}}{\pgfqpoint{3.873654in}{1.952457in}}%
\pgfpathcurveto{\pgfqpoint{3.865841in}{1.944644in}}{\pgfqpoint{3.861451in}{1.934045in}}{\pgfqpoint{3.861451in}{1.922995in}}%
\pgfpathcurveto{\pgfqpoint{3.861451in}{1.911944in}}{\pgfqpoint{3.865841in}{1.901345in}}{\pgfqpoint{3.873654in}{1.893532in}}%
\pgfpathcurveto{\pgfqpoint{3.881468in}{1.885718in}}{\pgfqpoint{3.892067in}{1.881328in}}{\pgfqpoint{3.903117in}{1.881328in}}%
\pgfpathclose%
\pgfusepath{stroke,fill}%
\end{pgfscope}%
\begin{pgfscope}%
\pgfpathrectangle{\pgfqpoint{0.600000in}{0.600000in}}{\pgfqpoint{3.900000in}{3.900000in}}%
\pgfusepath{clip}%
\pgfsetbuttcap%
\pgfsetroundjoin%
\definecolor{currentfill}{rgb}{0.121569,0.466667,0.705882}%
\pgfsetfillcolor{currentfill}%
\pgfsetlinewidth{1.003750pt}%
\definecolor{currentstroke}{rgb}{0.121569,0.466667,0.705882}%
\pgfsetstrokecolor{currentstroke}%
\pgfsetdash{}{0pt}%
\pgfpathmoveto{\pgfqpoint{2.431344in}{1.743748in}}%
\pgfpathcurveto{\pgfqpoint{2.442394in}{1.743748in}}{\pgfqpoint{2.452993in}{1.748138in}}{\pgfqpoint{2.460807in}{1.755952in}}%
\pgfpathcurveto{\pgfqpoint{2.468620in}{1.763766in}}{\pgfqpoint{2.473011in}{1.774365in}}{\pgfqpoint{2.473011in}{1.785415in}}%
\pgfpathcurveto{\pgfqpoint{2.473011in}{1.796465in}}{\pgfqpoint{2.468620in}{1.807064in}}{\pgfqpoint{2.460807in}{1.814878in}}%
\pgfpathcurveto{\pgfqpoint{2.452993in}{1.822691in}}{\pgfqpoint{2.442394in}{1.827081in}}{\pgfqpoint{2.431344in}{1.827081in}}%
\pgfpathcurveto{\pgfqpoint{2.420294in}{1.827081in}}{\pgfqpoint{2.409695in}{1.822691in}}{\pgfqpoint{2.401881in}{1.814878in}}%
\pgfpathcurveto{\pgfqpoint{2.394068in}{1.807064in}}{\pgfqpoint{2.389677in}{1.796465in}}{\pgfqpoint{2.389677in}{1.785415in}}%
\pgfpathcurveto{\pgfqpoint{2.389677in}{1.774365in}}{\pgfqpoint{2.394068in}{1.763766in}}{\pgfqpoint{2.401881in}{1.755952in}}%
\pgfpathcurveto{\pgfqpoint{2.409695in}{1.748138in}}{\pgfqpoint{2.420294in}{1.743748in}}{\pgfqpoint{2.431344in}{1.743748in}}%
\pgfpathclose%
\pgfusepath{stroke,fill}%
\end{pgfscope}%
\begin{pgfscope}%
\pgfpathrectangle{\pgfqpoint{0.600000in}{0.600000in}}{\pgfqpoint{3.900000in}{3.900000in}}%
\pgfusepath{clip}%
\pgfsetbuttcap%
\pgfsetroundjoin%
\definecolor{currentfill}{rgb}{0.121569,0.466667,0.705882}%
\pgfsetfillcolor{currentfill}%
\pgfsetlinewidth{1.003750pt}%
\definecolor{currentstroke}{rgb}{0.121569,0.466667,0.705882}%
\pgfsetstrokecolor{currentstroke}%
\pgfsetdash{}{0pt}%
\pgfpathmoveto{\pgfqpoint{3.011068in}{2.126608in}}%
\pgfpathcurveto{\pgfqpoint{3.022119in}{2.126608in}}{\pgfqpoint{3.032718in}{2.130998in}}{\pgfqpoint{3.040531in}{2.138812in}}%
\pgfpathcurveto{\pgfqpoint{3.048345in}{2.146625in}}{\pgfqpoint{3.052735in}{2.157224in}}{\pgfqpoint{3.052735in}{2.168274in}}%
\pgfpathcurveto{\pgfqpoint{3.052735in}{2.179325in}}{\pgfqpoint{3.048345in}{2.189924in}}{\pgfqpoint{3.040531in}{2.197737in}}%
\pgfpathcurveto{\pgfqpoint{3.032718in}{2.205551in}}{\pgfqpoint{3.022119in}{2.209941in}}{\pgfqpoint{3.011068in}{2.209941in}}%
\pgfpathcurveto{\pgfqpoint{3.000018in}{2.209941in}}{\pgfqpoint{2.989419in}{2.205551in}}{\pgfqpoint{2.981606in}{2.197737in}}%
\pgfpathcurveto{\pgfqpoint{2.973792in}{2.189924in}}{\pgfqpoint{2.969402in}{2.179325in}}{\pgfqpoint{2.969402in}{2.168274in}}%
\pgfpathcurveto{\pgfqpoint{2.969402in}{2.157224in}}{\pgfqpoint{2.973792in}{2.146625in}}{\pgfqpoint{2.981606in}{2.138812in}}%
\pgfpathcurveto{\pgfqpoint{2.989419in}{2.130998in}}{\pgfqpoint{3.000018in}{2.126608in}}{\pgfqpoint{3.011068in}{2.126608in}}%
\pgfpathclose%
\pgfusepath{stroke,fill}%
\end{pgfscope}%
\begin{pgfscope}%
\pgfpathrectangle{\pgfqpoint{0.600000in}{0.600000in}}{\pgfqpoint{3.900000in}{3.900000in}}%
\pgfusepath{clip}%
\pgfsetbuttcap%
\pgfsetroundjoin%
\definecolor{currentfill}{rgb}{0.121569,0.466667,0.705882}%
\pgfsetfillcolor{currentfill}%
\pgfsetlinewidth{1.003750pt}%
\definecolor{currentstroke}{rgb}{0.121569,0.466667,0.705882}%
\pgfsetstrokecolor{currentstroke}%
\pgfsetdash{}{0pt}%
\pgfpathmoveto{\pgfqpoint{3.472733in}{3.346474in}}%
\pgfpathcurveto{\pgfqpoint{3.483783in}{3.346474in}}{\pgfqpoint{3.494382in}{3.350864in}}{\pgfqpoint{3.502196in}{3.358678in}}%
\pgfpathcurveto{\pgfqpoint{3.510009in}{3.366492in}}{\pgfqpoint{3.514400in}{3.377091in}}{\pgfqpoint{3.514400in}{3.388141in}}%
\pgfpathcurveto{\pgfqpoint{3.514400in}{3.399191in}}{\pgfqpoint{3.510009in}{3.409790in}}{\pgfqpoint{3.502196in}{3.417604in}}%
\pgfpathcurveto{\pgfqpoint{3.494382in}{3.425417in}}{\pgfqpoint{3.483783in}{3.429807in}}{\pgfqpoint{3.472733in}{3.429807in}}%
\pgfpathcurveto{\pgfqpoint{3.461683in}{3.429807in}}{\pgfqpoint{3.451084in}{3.425417in}}{\pgfqpoint{3.443270in}{3.417604in}}%
\pgfpathcurveto{\pgfqpoint{3.435457in}{3.409790in}}{\pgfqpoint{3.431066in}{3.399191in}}{\pgfqpoint{3.431066in}{3.388141in}}%
\pgfpathcurveto{\pgfqpoint{3.431066in}{3.377091in}}{\pgfqpoint{3.435457in}{3.366492in}}{\pgfqpoint{3.443270in}{3.358678in}}%
\pgfpathcurveto{\pgfqpoint{3.451084in}{3.350864in}}{\pgfqpoint{3.461683in}{3.346474in}}{\pgfqpoint{3.472733in}{3.346474in}}%
\pgfpathclose%
\pgfusepath{stroke,fill}%
\end{pgfscope}%
\begin{pgfscope}%
\pgfpathrectangle{\pgfqpoint{0.600000in}{0.600000in}}{\pgfqpoint{3.900000in}{3.900000in}}%
\pgfusepath{clip}%
\pgfsetbuttcap%
\pgfsetroundjoin%
\definecolor{currentfill}{rgb}{0.121569,0.466667,0.705882}%
\pgfsetfillcolor{currentfill}%
\pgfsetlinewidth{1.003750pt}%
\definecolor{currentstroke}{rgb}{0.121569,0.466667,0.705882}%
\pgfsetstrokecolor{currentstroke}%
\pgfsetdash{}{0pt}%
\pgfpathmoveto{\pgfqpoint{2.732208in}{2.282830in}}%
\pgfpathcurveto{\pgfqpoint{2.743258in}{2.282830in}}{\pgfqpoint{2.753857in}{2.287220in}}{\pgfqpoint{2.761670in}{2.295034in}}%
\pgfpathcurveto{\pgfqpoint{2.769484in}{2.302847in}}{\pgfqpoint{2.773874in}{2.313446in}}{\pgfqpoint{2.773874in}{2.324496in}}%
\pgfpathcurveto{\pgfqpoint{2.773874in}{2.335547in}}{\pgfqpoint{2.769484in}{2.346146in}}{\pgfqpoint{2.761670in}{2.353959in}}%
\pgfpathcurveto{\pgfqpoint{2.753857in}{2.361773in}}{\pgfqpoint{2.743258in}{2.366163in}}{\pgfqpoint{2.732208in}{2.366163in}}%
\pgfpathcurveto{\pgfqpoint{2.721157in}{2.366163in}}{\pgfqpoint{2.710558in}{2.361773in}}{\pgfqpoint{2.702745in}{2.353959in}}%
\pgfpathcurveto{\pgfqpoint{2.694931in}{2.346146in}}{\pgfqpoint{2.690541in}{2.335547in}}{\pgfqpoint{2.690541in}{2.324496in}}%
\pgfpathcurveto{\pgfqpoint{2.690541in}{2.313446in}}{\pgfqpoint{2.694931in}{2.302847in}}{\pgfqpoint{2.702745in}{2.295034in}}%
\pgfpathcurveto{\pgfqpoint{2.710558in}{2.287220in}}{\pgfqpoint{2.721157in}{2.282830in}}{\pgfqpoint{2.732208in}{2.282830in}}%
\pgfpathclose%
\pgfusepath{stroke,fill}%
\end{pgfscope}%
\begin{pgfscope}%
\pgfpathrectangle{\pgfqpoint{0.600000in}{0.600000in}}{\pgfqpoint{3.900000in}{3.900000in}}%
\pgfusepath{clip}%
\pgfsetbuttcap%
\pgfsetroundjoin%
\definecolor{currentfill}{rgb}{0.121569,0.466667,0.705882}%
\pgfsetfillcolor{currentfill}%
\pgfsetlinewidth{1.003750pt}%
\definecolor{currentstroke}{rgb}{0.121569,0.466667,0.705882}%
\pgfsetstrokecolor{currentstroke}%
\pgfsetdash{}{0pt}%
\pgfpathmoveto{\pgfqpoint{2.648343in}{1.683935in}}%
\pgfpathcurveto{\pgfqpoint{2.659393in}{1.683935in}}{\pgfqpoint{2.669992in}{1.688325in}}{\pgfqpoint{2.677806in}{1.696139in}}%
\pgfpathcurveto{\pgfqpoint{2.685619in}{1.703952in}}{\pgfqpoint{2.690009in}{1.714551in}}{\pgfqpoint{2.690009in}{1.725601in}}%
\pgfpathcurveto{\pgfqpoint{2.690009in}{1.736651in}}{\pgfqpoint{2.685619in}{1.747250in}}{\pgfqpoint{2.677806in}{1.755064in}}%
\pgfpathcurveto{\pgfqpoint{2.669992in}{1.762878in}}{\pgfqpoint{2.659393in}{1.767268in}}{\pgfqpoint{2.648343in}{1.767268in}}%
\pgfpathcurveto{\pgfqpoint{2.637293in}{1.767268in}}{\pgfqpoint{2.626694in}{1.762878in}}{\pgfqpoint{2.618880in}{1.755064in}}%
\pgfpathcurveto{\pgfqpoint{2.611066in}{1.747250in}}{\pgfqpoint{2.606676in}{1.736651in}}{\pgfqpoint{2.606676in}{1.725601in}}%
\pgfpathcurveto{\pgfqpoint{2.606676in}{1.714551in}}{\pgfqpoint{2.611066in}{1.703952in}}{\pgfqpoint{2.618880in}{1.696139in}}%
\pgfpathcurveto{\pgfqpoint{2.626694in}{1.688325in}}{\pgfqpoint{2.637293in}{1.683935in}}{\pgfqpoint{2.648343in}{1.683935in}}%
\pgfpathclose%
\pgfusepath{stroke,fill}%
\end{pgfscope}%
\begin{pgfscope}%
\pgfpathrectangle{\pgfqpoint{0.600000in}{0.600000in}}{\pgfqpoint{3.900000in}{3.900000in}}%
\pgfusepath{clip}%
\pgfsetbuttcap%
\pgfsetroundjoin%
\definecolor{currentfill}{rgb}{0.121569,0.466667,0.705882}%
\pgfsetfillcolor{currentfill}%
\pgfsetlinewidth{1.003750pt}%
\definecolor{currentstroke}{rgb}{0.121569,0.466667,0.705882}%
\pgfsetstrokecolor{currentstroke}%
\pgfsetdash{}{0pt}%
\pgfpathmoveto{\pgfqpoint{2.341440in}{2.402060in}}%
\pgfpathcurveto{\pgfqpoint{2.352490in}{2.402060in}}{\pgfqpoint{2.363089in}{2.406450in}}{\pgfqpoint{2.370902in}{2.414263in}}%
\pgfpathcurveto{\pgfqpoint{2.378716in}{2.422077in}}{\pgfqpoint{2.383106in}{2.432676in}}{\pgfqpoint{2.383106in}{2.443726in}}%
\pgfpathcurveto{\pgfqpoint{2.383106in}{2.454776in}}{\pgfqpoint{2.378716in}{2.465375in}}{\pgfqpoint{2.370902in}{2.473189in}}%
\pgfpathcurveto{\pgfqpoint{2.363089in}{2.481003in}}{\pgfqpoint{2.352490in}{2.485393in}}{\pgfqpoint{2.341440in}{2.485393in}}%
\pgfpathcurveto{\pgfqpoint{2.330390in}{2.485393in}}{\pgfqpoint{2.319791in}{2.481003in}}{\pgfqpoint{2.311977in}{2.473189in}}%
\pgfpathcurveto{\pgfqpoint{2.304163in}{2.465375in}}{\pgfqpoint{2.299773in}{2.454776in}}{\pgfqpoint{2.299773in}{2.443726in}}%
\pgfpathcurveto{\pgfqpoint{2.299773in}{2.432676in}}{\pgfqpoint{2.304163in}{2.422077in}}{\pgfqpoint{2.311977in}{2.414263in}}%
\pgfpathcurveto{\pgfqpoint{2.319791in}{2.406450in}}{\pgfqpoint{2.330390in}{2.402060in}}{\pgfqpoint{2.341440in}{2.402060in}}%
\pgfpathclose%
\pgfusepath{stroke,fill}%
\end{pgfscope}%
\begin{pgfscope}%
\pgfpathrectangle{\pgfqpoint{0.600000in}{0.600000in}}{\pgfqpoint{3.900000in}{3.900000in}}%
\pgfusepath{clip}%
\pgfsetbuttcap%
\pgfsetroundjoin%
\definecolor{currentfill}{rgb}{0.121569,0.466667,0.705882}%
\pgfsetfillcolor{currentfill}%
\pgfsetlinewidth{1.003750pt}%
\definecolor{currentstroke}{rgb}{0.121569,0.466667,0.705882}%
\pgfsetstrokecolor{currentstroke}%
\pgfsetdash{}{0pt}%
\pgfpathmoveto{\pgfqpoint{2.871915in}{1.672130in}}%
\pgfpathcurveto{\pgfqpoint{2.882965in}{1.672130in}}{\pgfqpoint{2.893564in}{1.676520in}}{\pgfqpoint{2.901378in}{1.684334in}}%
\pgfpathcurveto{\pgfqpoint{2.909192in}{1.692147in}}{\pgfqpoint{2.913582in}{1.702746in}}{\pgfqpoint{2.913582in}{1.713796in}}%
\pgfpathcurveto{\pgfqpoint{2.913582in}{1.724847in}}{\pgfqpoint{2.909192in}{1.735446in}}{\pgfqpoint{2.901378in}{1.743259in}}%
\pgfpathcurveto{\pgfqpoint{2.893564in}{1.751073in}}{\pgfqpoint{2.882965in}{1.755463in}}{\pgfqpoint{2.871915in}{1.755463in}}%
\pgfpathcurveto{\pgfqpoint{2.860865in}{1.755463in}}{\pgfqpoint{2.850266in}{1.751073in}}{\pgfqpoint{2.842453in}{1.743259in}}%
\pgfpathcurveto{\pgfqpoint{2.834639in}{1.735446in}}{\pgfqpoint{2.830249in}{1.724847in}}{\pgfqpoint{2.830249in}{1.713796in}}%
\pgfpathcurveto{\pgfqpoint{2.830249in}{1.702746in}}{\pgfqpoint{2.834639in}{1.692147in}}{\pgfqpoint{2.842453in}{1.684334in}}%
\pgfpathcurveto{\pgfqpoint{2.850266in}{1.676520in}}{\pgfqpoint{2.860865in}{1.672130in}}{\pgfqpoint{2.871915in}{1.672130in}}%
\pgfpathclose%
\pgfusepath{stroke,fill}%
\end{pgfscope}%
\begin{pgfscope}%
\pgfpathrectangle{\pgfqpoint{0.600000in}{0.600000in}}{\pgfqpoint{3.900000in}{3.900000in}}%
\pgfusepath{clip}%
\pgfsetbuttcap%
\pgfsetroundjoin%
\definecolor{currentfill}{rgb}{0.121569,0.466667,0.705882}%
\pgfsetfillcolor{currentfill}%
\pgfsetlinewidth{1.003750pt}%
\definecolor{currentstroke}{rgb}{0.121569,0.466667,0.705882}%
\pgfsetstrokecolor{currentstroke}%
\pgfsetdash{}{0pt}%
\pgfpathmoveto{\pgfqpoint{2.629398in}{1.959561in}}%
\pgfpathcurveto{\pgfqpoint{2.640449in}{1.959561in}}{\pgfqpoint{2.651048in}{1.963952in}}{\pgfqpoint{2.658861in}{1.971765in}}%
\pgfpathcurveto{\pgfqpoint{2.666675in}{1.979579in}}{\pgfqpoint{2.671065in}{1.990178in}}{\pgfqpoint{2.671065in}{2.001228in}}%
\pgfpathcurveto{\pgfqpoint{2.671065in}{2.012278in}}{\pgfqpoint{2.666675in}{2.022877in}}{\pgfqpoint{2.658861in}{2.030691in}}%
\pgfpathcurveto{\pgfqpoint{2.651048in}{2.038504in}}{\pgfqpoint{2.640449in}{2.042895in}}{\pgfqpoint{2.629398in}{2.042895in}}%
\pgfpathcurveto{\pgfqpoint{2.618348in}{2.042895in}}{\pgfqpoint{2.607749in}{2.038504in}}{\pgfqpoint{2.599936in}{2.030691in}}%
\pgfpathcurveto{\pgfqpoint{2.592122in}{2.022877in}}{\pgfqpoint{2.587732in}{2.012278in}}{\pgfqpoint{2.587732in}{2.001228in}}%
\pgfpathcurveto{\pgfqpoint{2.587732in}{1.990178in}}{\pgfqpoint{2.592122in}{1.979579in}}{\pgfqpoint{2.599936in}{1.971765in}}%
\pgfpathcurveto{\pgfqpoint{2.607749in}{1.963952in}}{\pgfqpoint{2.618348in}{1.959561in}}{\pgfqpoint{2.629398in}{1.959561in}}%
\pgfpathclose%
\pgfusepath{stroke,fill}%
\end{pgfscope}%
\begin{pgfscope}%
\pgfpathrectangle{\pgfqpoint{0.600000in}{0.600000in}}{\pgfqpoint{3.900000in}{3.900000in}}%
\pgfusepath{clip}%
\pgfsetbuttcap%
\pgfsetroundjoin%
\definecolor{currentfill}{rgb}{0.121569,0.466667,0.705882}%
\pgfsetfillcolor{currentfill}%
\pgfsetlinewidth{1.003750pt}%
\definecolor{currentstroke}{rgb}{0.121569,0.466667,0.705882}%
\pgfsetstrokecolor{currentstroke}%
\pgfsetdash{}{0pt}%
\pgfpathmoveto{\pgfqpoint{2.487699in}{1.833387in}}%
\pgfpathcurveto{\pgfqpoint{2.498749in}{1.833387in}}{\pgfqpoint{2.509348in}{1.837777in}}{\pgfqpoint{2.517162in}{1.845591in}}%
\pgfpathcurveto{\pgfqpoint{2.524976in}{1.853405in}}{\pgfqpoint{2.529366in}{1.864004in}}{\pgfqpoint{2.529366in}{1.875054in}}%
\pgfpathcurveto{\pgfqpoint{2.529366in}{1.886104in}}{\pgfqpoint{2.524976in}{1.896703in}}{\pgfqpoint{2.517162in}{1.904517in}}%
\pgfpathcurveto{\pgfqpoint{2.509348in}{1.912330in}}{\pgfqpoint{2.498749in}{1.916721in}}{\pgfqpoint{2.487699in}{1.916721in}}%
\pgfpathcurveto{\pgfqpoint{2.476649in}{1.916721in}}{\pgfqpoint{2.466050in}{1.912330in}}{\pgfqpoint{2.458236in}{1.904517in}}%
\pgfpathcurveto{\pgfqpoint{2.450423in}{1.896703in}}{\pgfqpoint{2.446032in}{1.886104in}}{\pgfqpoint{2.446032in}{1.875054in}}%
\pgfpathcurveto{\pgfqpoint{2.446032in}{1.864004in}}{\pgfqpoint{2.450423in}{1.853405in}}{\pgfqpoint{2.458236in}{1.845591in}}%
\pgfpathcurveto{\pgfqpoint{2.466050in}{1.837777in}}{\pgfqpoint{2.476649in}{1.833387in}}{\pgfqpoint{2.487699in}{1.833387in}}%
\pgfpathclose%
\pgfusepath{stroke,fill}%
\end{pgfscope}%
\begin{pgfscope}%
\pgfpathrectangle{\pgfqpoint{0.600000in}{0.600000in}}{\pgfqpoint{3.900000in}{3.900000in}}%
\pgfusepath{clip}%
\pgfsetbuttcap%
\pgfsetroundjoin%
\definecolor{currentfill}{rgb}{0.121569,0.466667,0.705882}%
\pgfsetfillcolor{currentfill}%
\pgfsetlinewidth{1.003750pt}%
\definecolor{currentstroke}{rgb}{0.121569,0.466667,0.705882}%
\pgfsetstrokecolor{currentstroke}%
\pgfsetdash{}{0pt}%
\pgfpathmoveto{\pgfqpoint{2.412838in}{1.701823in}}%
\pgfpathcurveto{\pgfqpoint{2.423888in}{1.701823in}}{\pgfqpoint{2.434487in}{1.706213in}}{\pgfqpoint{2.442300in}{1.714027in}}%
\pgfpathcurveto{\pgfqpoint{2.450114in}{1.721840in}}{\pgfqpoint{2.454504in}{1.732439in}}{\pgfqpoint{2.454504in}{1.743489in}}%
\pgfpathcurveto{\pgfqpoint{2.454504in}{1.754539in}}{\pgfqpoint{2.450114in}{1.765139in}}{\pgfqpoint{2.442300in}{1.772952in}}%
\pgfpathcurveto{\pgfqpoint{2.434487in}{1.780766in}}{\pgfqpoint{2.423888in}{1.785156in}}{\pgfqpoint{2.412838in}{1.785156in}}%
\pgfpathcurveto{\pgfqpoint{2.401788in}{1.785156in}}{\pgfqpoint{2.391189in}{1.780766in}}{\pgfqpoint{2.383375in}{1.772952in}}%
\pgfpathcurveto{\pgfqpoint{2.375561in}{1.765139in}}{\pgfqpoint{2.371171in}{1.754539in}}{\pgfqpoint{2.371171in}{1.743489in}}%
\pgfpathcurveto{\pgfqpoint{2.371171in}{1.732439in}}{\pgfqpoint{2.375561in}{1.721840in}}{\pgfqpoint{2.383375in}{1.714027in}}%
\pgfpathcurveto{\pgfqpoint{2.391189in}{1.706213in}}{\pgfqpoint{2.401788in}{1.701823in}}{\pgfqpoint{2.412838in}{1.701823in}}%
\pgfpathclose%
\pgfusepath{stroke,fill}%
\end{pgfscope}%
\begin{pgfscope}%
\pgfpathrectangle{\pgfqpoint{0.600000in}{0.600000in}}{\pgfqpoint{3.900000in}{3.900000in}}%
\pgfusepath{clip}%
\pgfsetbuttcap%
\pgfsetroundjoin%
\definecolor{currentfill}{rgb}{0.121569,0.466667,0.705882}%
\pgfsetfillcolor{currentfill}%
\pgfsetlinewidth{1.003750pt}%
\definecolor{currentstroke}{rgb}{0.121569,0.466667,0.705882}%
\pgfsetstrokecolor{currentstroke}%
\pgfsetdash{}{0pt}%
\pgfpathmoveto{\pgfqpoint{3.096027in}{2.661023in}}%
\pgfpathcurveto{\pgfqpoint{3.107077in}{2.661023in}}{\pgfqpoint{3.117676in}{2.665413in}}{\pgfqpoint{3.125490in}{2.673227in}}%
\pgfpathcurveto{\pgfqpoint{3.133303in}{2.681040in}}{\pgfqpoint{3.137694in}{2.691640in}}{\pgfqpoint{3.137694in}{2.702690in}}%
\pgfpathcurveto{\pgfqpoint{3.137694in}{2.713740in}}{\pgfqpoint{3.133303in}{2.724339in}}{\pgfqpoint{3.125490in}{2.732152in}}%
\pgfpathcurveto{\pgfqpoint{3.117676in}{2.739966in}}{\pgfqpoint{3.107077in}{2.744356in}}{\pgfqpoint{3.096027in}{2.744356in}}%
\pgfpathcurveto{\pgfqpoint{3.084977in}{2.744356in}}{\pgfqpoint{3.074378in}{2.739966in}}{\pgfqpoint{3.066564in}{2.732152in}}%
\pgfpathcurveto{\pgfqpoint{3.058751in}{2.724339in}}{\pgfqpoint{3.054360in}{2.713740in}}{\pgfqpoint{3.054360in}{2.702690in}}%
\pgfpathcurveto{\pgfqpoint{3.054360in}{2.691640in}}{\pgfqpoint{3.058751in}{2.681040in}}{\pgfqpoint{3.066564in}{2.673227in}}%
\pgfpathcurveto{\pgfqpoint{3.074378in}{2.665413in}}{\pgfqpoint{3.084977in}{2.661023in}}{\pgfqpoint{3.096027in}{2.661023in}}%
\pgfpathclose%
\pgfusepath{stroke,fill}%
\end{pgfscope}%
\begin{pgfscope}%
\pgfpathrectangle{\pgfqpoint{0.600000in}{0.600000in}}{\pgfqpoint{3.900000in}{3.900000in}}%
\pgfusepath{clip}%
\pgfsetbuttcap%
\pgfsetroundjoin%
\definecolor{currentfill}{rgb}{0.121569,0.466667,0.705882}%
\pgfsetfillcolor{currentfill}%
\pgfsetlinewidth{1.003750pt}%
\definecolor{currentstroke}{rgb}{0.121569,0.466667,0.705882}%
\pgfsetstrokecolor{currentstroke}%
\pgfsetdash{}{0pt}%
\pgfpathmoveto{\pgfqpoint{3.223429in}{1.856211in}}%
\pgfpathcurveto{\pgfqpoint{3.234479in}{1.856211in}}{\pgfqpoint{3.245078in}{1.860601in}}{\pgfqpoint{3.252892in}{1.868414in}}%
\pgfpathcurveto{\pgfqpoint{3.260705in}{1.876228in}}{\pgfqpoint{3.265096in}{1.886827in}}{\pgfqpoint{3.265096in}{1.897877in}}%
\pgfpathcurveto{\pgfqpoint{3.265096in}{1.908927in}}{\pgfqpoint{3.260705in}{1.919526in}}{\pgfqpoint{3.252892in}{1.927340in}}%
\pgfpathcurveto{\pgfqpoint{3.245078in}{1.935154in}}{\pgfqpoint{3.234479in}{1.939544in}}{\pgfqpoint{3.223429in}{1.939544in}}%
\pgfpathcurveto{\pgfqpoint{3.212379in}{1.939544in}}{\pgfqpoint{3.201780in}{1.935154in}}{\pgfqpoint{3.193966in}{1.927340in}}%
\pgfpathcurveto{\pgfqpoint{3.186152in}{1.919526in}}{\pgfqpoint{3.181762in}{1.908927in}}{\pgfqpoint{3.181762in}{1.897877in}}%
\pgfpathcurveto{\pgfqpoint{3.181762in}{1.886827in}}{\pgfqpoint{3.186152in}{1.876228in}}{\pgfqpoint{3.193966in}{1.868414in}}%
\pgfpathcurveto{\pgfqpoint{3.201780in}{1.860601in}}{\pgfqpoint{3.212379in}{1.856211in}}{\pgfqpoint{3.223429in}{1.856211in}}%
\pgfpathclose%
\pgfusepath{stroke,fill}%
\end{pgfscope}%
\begin{pgfscope}%
\pgfpathrectangle{\pgfqpoint{0.600000in}{0.600000in}}{\pgfqpoint{3.900000in}{3.900000in}}%
\pgfusepath{clip}%
\pgfsetbuttcap%
\pgfsetroundjoin%
\definecolor{currentfill}{rgb}{0.121569,0.466667,0.705882}%
\pgfsetfillcolor{currentfill}%
\pgfsetlinewidth{1.003750pt}%
\definecolor{currentstroke}{rgb}{0.121569,0.466667,0.705882}%
\pgfsetstrokecolor{currentstroke}%
\pgfsetdash{}{0pt}%
\pgfpathmoveto{\pgfqpoint{2.115152in}{2.883256in}}%
\pgfpathcurveto{\pgfqpoint{2.126202in}{2.883256in}}{\pgfqpoint{2.136801in}{2.887647in}}{\pgfqpoint{2.144615in}{2.895460in}}%
\pgfpathcurveto{\pgfqpoint{2.152428in}{2.903274in}}{\pgfqpoint{2.156819in}{2.913873in}}{\pgfqpoint{2.156819in}{2.924923in}}%
\pgfpathcurveto{\pgfqpoint{2.156819in}{2.935973in}}{\pgfqpoint{2.152428in}{2.946572in}}{\pgfqpoint{2.144615in}{2.954386in}}%
\pgfpathcurveto{\pgfqpoint{2.136801in}{2.962199in}}{\pgfqpoint{2.126202in}{2.966590in}}{\pgfqpoint{2.115152in}{2.966590in}}%
\pgfpathcurveto{\pgfqpoint{2.104102in}{2.966590in}}{\pgfqpoint{2.093503in}{2.962199in}}{\pgfqpoint{2.085689in}{2.954386in}}%
\pgfpathcurveto{\pgfqpoint{2.077876in}{2.946572in}}{\pgfqpoint{2.073485in}{2.935973in}}{\pgfqpoint{2.073485in}{2.924923in}}%
\pgfpathcurveto{\pgfqpoint{2.073485in}{2.913873in}}{\pgfqpoint{2.077876in}{2.903274in}}{\pgfqpoint{2.085689in}{2.895460in}}%
\pgfpathcurveto{\pgfqpoint{2.093503in}{2.887647in}}{\pgfqpoint{2.104102in}{2.883256in}}{\pgfqpoint{2.115152in}{2.883256in}}%
\pgfpathclose%
\pgfusepath{stroke,fill}%
\end{pgfscope}%
\begin{pgfscope}%
\pgfpathrectangle{\pgfqpoint{0.600000in}{0.600000in}}{\pgfqpoint{3.900000in}{3.900000in}}%
\pgfusepath{clip}%
\pgfsetbuttcap%
\pgfsetroundjoin%
\definecolor{currentfill}{rgb}{0.121569,0.466667,0.705882}%
\pgfsetfillcolor{currentfill}%
\pgfsetlinewidth{1.003750pt}%
\definecolor{currentstroke}{rgb}{0.121569,0.466667,0.705882}%
\pgfsetstrokecolor{currentstroke}%
\pgfsetdash{}{0pt}%
\pgfpathmoveto{\pgfqpoint{2.623321in}{2.596300in}}%
\pgfpathcurveto{\pgfqpoint{2.634371in}{2.596300in}}{\pgfqpoint{2.644970in}{2.600690in}}{\pgfqpoint{2.652784in}{2.608504in}}%
\pgfpathcurveto{\pgfqpoint{2.660597in}{2.616317in}}{\pgfqpoint{2.664988in}{2.626916in}}{\pgfqpoint{2.664988in}{2.637966in}}%
\pgfpathcurveto{\pgfqpoint{2.664988in}{2.649017in}}{\pgfqpoint{2.660597in}{2.659616in}}{\pgfqpoint{2.652784in}{2.667429in}}%
\pgfpathcurveto{\pgfqpoint{2.644970in}{2.675243in}}{\pgfqpoint{2.634371in}{2.679633in}}{\pgfqpoint{2.623321in}{2.679633in}}%
\pgfpathcurveto{\pgfqpoint{2.612271in}{2.679633in}}{\pgfqpoint{2.601672in}{2.675243in}}{\pgfqpoint{2.593858in}{2.667429in}}%
\pgfpathcurveto{\pgfqpoint{2.586044in}{2.659616in}}{\pgfqpoint{2.581654in}{2.649017in}}{\pgfqpoint{2.581654in}{2.637966in}}%
\pgfpathcurveto{\pgfqpoint{2.581654in}{2.626916in}}{\pgfqpoint{2.586044in}{2.616317in}}{\pgfqpoint{2.593858in}{2.608504in}}%
\pgfpathcurveto{\pgfqpoint{2.601672in}{2.600690in}}{\pgfqpoint{2.612271in}{2.596300in}}{\pgfqpoint{2.623321in}{2.596300in}}%
\pgfpathclose%
\pgfusepath{stroke,fill}%
\end{pgfscope}%
\begin{pgfscope}%
\pgfpathrectangle{\pgfqpoint{0.600000in}{0.600000in}}{\pgfqpoint{3.900000in}{3.900000in}}%
\pgfusepath{clip}%
\pgfsetbuttcap%
\pgfsetroundjoin%
\definecolor{currentfill}{rgb}{0.121569,0.466667,0.705882}%
\pgfsetfillcolor{currentfill}%
\pgfsetlinewidth{1.003750pt}%
\definecolor{currentstroke}{rgb}{0.121569,0.466667,0.705882}%
\pgfsetstrokecolor{currentstroke}%
\pgfsetdash{}{0pt}%
\pgfpathmoveto{\pgfqpoint{1.572516in}{2.808087in}}%
\pgfpathcurveto{\pgfqpoint{1.583566in}{2.808087in}}{\pgfqpoint{1.594165in}{2.812477in}}{\pgfqpoint{1.601979in}{2.820291in}}%
\pgfpathcurveto{\pgfqpoint{1.609792in}{2.828104in}}{\pgfqpoint{1.614183in}{2.838703in}}{\pgfqpoint{1.614183in}{2.849754in}}%
\pgfpathcurveto{\pgfqpoint{1.614183in}{2.860804in}}{\pgfqpoint{1.609792in}{2.871403in}}{\pgfqpoint{1.601979in}{2.879216in}}%
\pgfpathcurveto{\pgfqpoint{1.594165in}{2.887030in}}{\pgfqpoint{1.583566in}{2.891420in}}{\pgfqpoint{1.572516in}{2.891420in}}%
\pgfpathcurveto{\pgfqpoint{1.561466in}{2.891420in}}{\pgfqpoint{1.550867in}{2.887030in}}{\pgfqpoint{1.543053in}{2.879216in}}%
\pgfpathcurveto{\pgfqpoint{1.535240in}{2.871403in}}{\pgfqpoint{1.530849in}{2.860804in}}{\pgfqpoint{1.530849in}{2.849754in}}%
\pgfpathcurveto{\pgfqpoint{1.530849in}{2.838703in}}{\pgfqpoint{1.535240in}{2.828104in}}{\pgfqpoint{1.543053in}{2.820291in}}%
\pgfpathcurveto{\pgfqpoint{1.550867in}{2.812477in}}{\pgfqpoint{1.561466in}{2.808087in}}{\pgfqpoint{1.572516in}{2.808087in}}%
\pgfpathclose%
\pgfusepath{stroke,fill}%
\end{pgfscope}%
\begin{pgfscope}%
\pgfpathrectangle{\pgfqpoint{0.600000in}{0.600000in}}{\pgfqpoint{3.900000in}{3.900000in}}%
\pgfusepath{clip}%
\pgfsetbuttcap%
\pgfsetroundjoin%
\definecolor{currentfill}{rgb}{0.121569,0.466667,0.705882}%
\pgfsetfillcolor{currentfill}%
\pgfsetlinewidth{1.003750pt}%
\definecolor{currentstroke}{rgb}{0.121569,0.466667,0.705882}%
\pgfsetstrokecolor{currentstroke}%
\pgfsetdash{}{0pt}%
\pgfpathmoveto{\pgfqpoint{2.034209in}{2.253496in}}%
\pgfpathcurveto{\pgfqpoint{2.045259in}{2.253496in}}{\pgfqpoint{2.055858in}{2.257887in}}{\pgfqpoint{2.063672in}{2.265700in}}%
\pgfpathcurveto{\pgfqpoint{2.071485in}{2.273514in}}{\pgfqpoint{2.075875in}{2.284113in}}{\pgfqpoint{2.075875in}{2.295163in}}%
\pgfpathcurveto{\pgfqpoint{2.075875in}{2.306213in}}{\pgfqpoint{2.071485in}{2.316812in}}{\pgfqpoint{2.063672in}{2.324626in}}%
\pgfpathcurveto{\pgfqpoint{2.055858in}{2.332439in}}{\pgfqpoint{2.045259in}{2.336830in}}{\pgfqpoint{2.034209in}{2.336830in}}%
\pgfpathcurveto{\pgfqpoint{2.023159in}{2.336830in}}{\pgfqpoint{2.012560in}{2.332439in}}{\pgfqpoint{2.004746in}{2.324626in}}%
\pgfpathcurveto{\pgfqpoint{1.996932in}{2.316812in}}{\pgfqpoint{1.992542in}{2.306213in}}{\pgfqpoint{1.992542in}{2.295163in}}%
\pgfpathcurveto{\pgfqpoint{1.992542in}{2.284113in}}{\pgfqpoint{1.996932in}{2.273514in}}{\pgfqpoint{2.004746in}{2.265700in}}%
\pgfpathcurveto{\pgfqpoint{2.012560in}{2.257887in}}{\pgfqpoint{2.023159in}{2.253496in}}{\pgfqpoint{2.034209in}{2.253496in}}%
\pgfpathclose%
\pgfusepath{stroke,fill}%
\end{pgfscope}%
\begin{pgfscope}%
\pgfpathrectangle{\pgfqpoint{0.600000in}{0.600000in}}{\pgfqpoint{3.900000in}{3.900000in}}%
\pgfusepath{clip}%
\pgfsetbuttcap%
\pgfsetroundjoin%
\definecolor{currentfill}{rgb}{0.121569,0.466667,0.705882}%
\pgfsetfillcolor{currentfill}%
\pgfsetlinewidth{1.003750pt}%
\definecolor{currentstroke}{rgb}{0.121569,0.466667,0.705882}%
\pgfsetstrokecolor{currentstroke}%
\pgfsetdash{}{0pt}%
\pgfpathmoveto{\pgfqpoint{2.497193in}{2.921445in}}%
\pgfpathcurveto{\pgfqpoint{2.508243in}{2.921445in}}{\pgfqpoint{2.518842in}{2.925835in}}{\pgfqpoint{2.526655in}{2.933649in}}%
\pgfpathcurveto{\pgfqpoint{2.534469in}{2.941462in}}{\pgfqpoint{2.538859in}{2.952061in}}{\pgfqpoint{2.538859in}{2.963111in}}%
\pgfpathcurveto{\pgfqpoint{2.538859in}{2.974162in}}{\pgfqpoint{2.534469in}{2.984761in}}{\pgfqpoint{2.526655in}{2.992574in}}%
\pgfpathcurveto{\pgfqpoint{2.518842in}{3.000388in}}{\pgfqpoint{2.508243in}{3.004778in}}{\pgfqpoint{2.497193in}{3.004778in}}%
\pgfpathcurveto{\pgfqpoint{2.486143in}{3.004778in}}{\pgfqpoint{2.475544in}{3.000388in}}{\pgfqpoint{2.467730in}{2.992574in}}%
\pgfpathcurveto{\pgfqpoint{2.459916in}{2.984761in}}{\pgfqpoint{2.455526in}{2.974162in}}{\pgfqpoint{2.455526in}{2.963111in}}%
\pgfpathcurveto{\pgfqpoint{2.455526in}{2.952061in}}{\pgfqpoint{2.459916in}{2.941462in}}{\pgfqpoint{2.467730in}{2.933649in}}%
\pgfpathcurveto{\pgfqpoint{2.475544in}{2.925835in}}{\pgfqpoint{2.486143in}{2.921445in}}{\pgfqpoint{2.497193in}{2.921445in}}%
\pgfpathclose%
\pgfusepath{stroke,fill}%
\end{pgfscope}%
\begin{pgfscope}%
\pgfpathrectangle{\pgfqpoint{0.600000in}{0.600000in}}{\pgfqpoint{3.900000in}{3.900000in}}%
\pgfusepath{clip}%
\pgfsetbuttcap%
\pgfsetroundjoin%
\definecolor{currentfill}{rgb}{0.121569,0.466667,0.705882}%
\pgfsetfillcolor{currentfill}%
\pgfsetlinewidth{1.003750pt}%
\definecolor{currentstroke}{rgb}{0.121569,0.466667,0.705882}%
\pgfsetstrokecolor{currentstroke}%
\pgfsetdash{}{0pt}%
\pgfpathmoveto{\pgfqpoint{2.686770in}{2.839834in}}%
\pgfpathcurveto{\pgfqpoint{2.697820in}{2.839834in}}{\pgfqpoint{2.708420in}{2.844225in}}{\pgfqpoint{2.716233in}{2.852038in}}%
\pgfpathcurveto{\pgfqpoint{2.724047in}{2.859852in}}{\pgfqpoint{2.728437in}{2.870451in}}{\pgfqpoint{2.728437in}{2.881501in}}%
\pgfpathcurveto{\pgfqpoint{2.728437in}{2.892551in}}{\pgfqpoint{2.724047in}{2.903150in}}{\pgfqpoint{2.716233in}{2.910964in}}%
\pgfpathcurveto{\pgfqpoint{2.708420in}{2.918777in}}{\pgfqpoint{2.697820in}{2.923168in}}{\pgfqpoint{2.686770in}{2.923168in}}%
\pgfpathcurveto{\pgfqpoint{2.675720in}{2.923168in}}{\pgfqpoint{2.665121in}{2.918777in}}{\pgfqpoint{2.657308in}{2.910964in}}%
\pgfpathcurveto{\pgfqpoint{2.649494in}{2.903150in}}{\pgfqpoint{2.645104in}{2.892551in}}{\pgfqpoint{2.645104in}{2.881501in}}%
\pgfpathcurveto{\pgfqpoint{2.645104in}{2.870451in}}{\pgfqpoint{2.649494in}{2.859852in}}{\pgfqpoint{2.657308in}{2.852038in}}%
\pgfpathcurveto{\pgfqpoint{2.665121in}{2.844225in}}{\pgfqpoint{2.675720in}{2.839834in}}{\pgfqpoint{2.686770in}{2.839834in}}%
\pgfpathclose%
\pgfusepath{stroke,fill}%
\end{pgfscope}%
\begin{pgfscope}%
\pgfpathrectangle{\pgfqpoint{0.600000in}{0.600000in}}{\pgfqpoint{3.900000in}{3.900000in}}%
\pgfusepath{clip}%
\pgfsetbuttcap%
\pgfsetroundjoin%
\definecolor{currentfill}{rgb}{0.121569,0.466667,0.705882}%
\pgfsetfillcolor{currentfill}%
\pgfsetlinewidth{1.003750pt}%
\definecolor{currentstroke}{rgb}{0.121569,0.466667,0.705882}%
\pgfsetstrokecolor{currentstroke}%
\pgfsetdash{}{0pt}%
\pgfpathmoveto{\pgfqpoint{3.082972in}{2.096433in}}%
\pgfpathcurveto{\pgfqpoint{3.094022in}{2.096433in}}{\pgfqpoint{3.104621in}{2.100823in}}{\pgfqpoint{3.112434in}{2.108637in}}%
\pgfpathcurveto{\pgfqpoint{3.120248in}{2.116451in}}{\pgfqpoint{3.124638in}{2.127050in}}{\pgfqpoint{3.124638in}{2.138100in}}%
\pgfpathcurveto{\pgfqpoint{3.124638in}{2.149150in}}{\pgfqpoint{3.120248in}{2.159749in}}{\pgfqpoint{3.112434in}{2.167563in}}%
\pgfpathcurveto{\pgfqpoint{3.104621in}{2.175376in}}{\pgfqpoint{3.094022in}{2.179766in}}{\pgfqpoint{3.082972in}{2.179766in}}%
\pgfpathcurveto{\pgfqpoint{3.071921in}{2.179766in}}{\pgfqpoint{3.061322in}{2.175376in}}{\pgfqpoint{3.053509in}{2.167563in}}%
\pgfpathcurveto{\pgfqpoint{3.045695in}{2.159749in}}{\pgfqpoint{3.041305in}{2.149150in}}{\pgfqpoint{3.041305in}{2.138100in}}%
\pgfpathcurveto{\pgfqpoint{3.041305in}{2.127050in}}{\pgfqpoint{3.045695in}{2.116451in}}{\pgfqpoint{3.053509in}{2.108637in}}%
\pgfpathcurveto{\pgfqpoint{3.061322in}{2.100823in}}{\pgfqpoint{3.071921in}{2.096433in}}{\pgfqpoint{3.082972in}{2.096433in}}%
\pgfpathclose%
\pgfusepath{stroke,fill}%
\end{pgfscope}%
\begin{pgfscope}%
\pgfpathrectangle{\pgfqpoint{0.600000in}{0.600000in}}{\pgfqpoint{3.900000in}{3.900000in}}%
\pgfusepath{clip}%
\pgfsetbuttcap%
\pgfsetroundjoin%
\definecolor{currentfill}{rgb}{0.121569,0.466667,0.705882}%
\pgfsetfillcolor{currentfill}%
\pgfsetlinewidth{1.003750pt}%
\definecolor{currentstroke}{rgb}{0.121569,0.466667,0.705882}%
\pgfsetstrokecolor{currentstroke}%
\pgfsetdash{}{0pt}%
\pgfpathmoveto{\pgfqpoint{3.027977in}{2.737434in}}%
\pgfpathcurveto{\pgfqpoint{3.039027in}{2.737434in}}{\pgfqpoint{3.049626in}{2.741825in}}{\pgfqpoint{3.057440in}{2.749638in}}%
\pgfpathcurveto{\pgfqpoint{3.065253in}{2.757452in}}{\pgfqpoint{3.069643in}{2.768051in}}{\pgfqpoint{3.069643in}{2.779101in}}%
\pgfpathcurveto{\pgfqpoint{3.069643in}{2.790151in}}{\pgfqpoint{3.065253in}{2.800750in}}{\pgfqpoint{3.057440in}{2.808564in}}%
\pgfpathcurveto{\pgfqpoint{3.049626in}{2.816377in}}{\pgfqpoint{3.039027in}{2.820768in}}{\pgfqpoint{3.027977in}{2.820768in}}%
\pgfpathcurveto{\pgfqpoint{3.016927in}{2.820768in}}{\pgfqpoint{3.006328in}{2.816377in}}{\pgfqpoint{2.998514in}{2.808564in}}%
\pgfpathcurveto{\pgfqpoint{2.990700in}{2.800750in}}{\pgfqpoint{2.986310in}{2.790151in}}{\pgfqpoint{2.986310in}{2.779101in}}%
\pgfpathcurveto{\pgfqpoint{2.986310in}{2.768051in}}{\pgfqpoint{2.990700in}{2.757452in}}{\pgfqpoint{2.998514in}{2.749638in}}%
\pgfpathcurveto{\pgfqpoint{3.006328in}{2.741825in}}{\pgfqpoint{3.016927in}{2.737434in}}{\pgfqpoint{3.027977in}{2.737434in}}%
\pgfpathclose%
\pgfusepath{stroke,fill}%
\end{pgfscope}%
\begin{pgfscope}%
\pgfpathrectangle{\pgfqpoint{0.600000in}{0.600000in}}{\pgfqpoint{3.900000in}{3.900000in}}%
\pgfusepath{clip}%
\pgfsetbuttcap%
\pgfsetroundjoin%
\definecolor{currentfill}{rgb}{0.121569,0.466667,0.705882}%
\pgfsetfillcolor{currentfill}%
\pgfsetlinewidth{1.003750pt}%
\definecolor{currentstroke}{rgb}{0.121569,0.466667,0.705882}%
\pgfsetstrokecolor{currentstroke}%
\pgfsetdash{}{0pt}%
\pgfpathmoveto{\pgfqpoint{2.779322in}{2.309754in}}%
\pgfpathcurveto{\pgfqpoint{2.790372in}{2.309754in}}{\pgfqpoint{2.800971in}{2.314144in}}{\pgfqpoint{2.808785in}{2.321958in}}%
\pgfpathcurveto{\pgfqpoint{2.816598in}{2.329772in}}{\pgfqpoint{2.820989in}{2.340371in}}{\pgfqpoint{2.820989in}{2.351421in}}%
\pgfpathcurveto{\pgfqpoint{2.820989in}{2.362471in}}{\pgfqpoint{2.816598in}{2.373070in}}{\pgfqpoint{2.808785in}{2.380883in}}%
\pgfpathcurveto{\pgfqpoint{2.800971in}{2.388697in}}{\pgfqpoint{2.790372in}{2.393087in}}{\pgfqpoint{2.779322in}{2.393087in}}%
\pgfpathcurveto{\pgfqpoint{2.768272in}{2.393087in}}{\pgfqpoint{2.757673in}{2.388697in}}{\pgfqpoint{2.749859in}{2.380883in}}%
\pgfpathcurveto{\pgfqpoint{2.742046in}{2.373070in}}{\pgfqpoint{2.737655in}{2.362471in}}{\pgfqpoint{2.737655in}{2.351421in}}%
\pgfpathcurveto{\pgfqpoint{2.737655in}{2.340371in}}{\pgfqpoint{2.742046in}{2.329772in}}{\pgfqpoint{2.749859in}{2.321958in}}%
\pgfpathcurveto{\pgfqpoint{2.757673in}{2.314144in}}{\pgfqpoint{2.768272in}{2.309754in}}{\pgfqpoint{2.779322in}{2.309754in}}%
\pgfpathclose%
\pgfusepath{stroke,fill}%
\end{pgfscope}%
\begin{pgfscope}%
\pgfpathrectangle{\pgfqpoint{0.600000in}{0.600000in}}{\pgfqpoint{3.900000in}{3.900000in}}%
\pgfusepath{clip}%
\pgfsetbuttcap%
\pgfsetroundjoin%
\definecolor{currentfill}{rgb}{0.121569,0.466667,0.705882}%
\pgfsetfillcolor{currentfill}%
\pgfsetlinewidth{1.003750pt}%
\definecolor{currentstroke}{rgb}{0.121569,0.466667,0.705882}%
\pgfsetstrokecolor{currentstroke}%
\pgfsetdash{}{0pt}%
\pgfpathmoveto{\pgfqpoint{2.606363in}{2.317735in}}%
\pgfpathcurveto{\pgfqpoint{2.617413in}{2.317735in}}{\pgfqpoint{2.628012in}{2.322126in}}{\pgfqpoint{2.635826in}{2.329939in}}%
\pgfpathcurveto{\pgfqpoint{2.643639in}{2.337753in}}{\pgfqpoint{2.648030in}{2.348352in}}{\pgfqpoint{2.648030in}{2.359402in}}%
\pgfpathcurveto{\pgfqpoint{2.648030in}{2.370452in}}{\pgfqpoint{2.643639in}{2.381051in}}{\pgfqpoint{2.635826in}{2.388865in}}%
\pgfpathcurveto{\pgfqpoint{2.628012in}{2.396678in}}{\pgfqpoint{2.617413in}{2.401069in}}{\pgfqpoint{2.606363in}{2.401069in}}%
\pgfpathcurveto{\pgfqpoint{2.595313in}{2.401069in}}{\pgfqpoint{2.584714in}{2.396678in}}{\pgfqpoint{2.576900in}{2.388865in}}%
\pgfpathcurveto{\pgfqpoint{2.569086in}{2.381051in}}{\pgfqpoint{2.564696in}{2.370452in}}{\pgfqpoint{2.564696in}{2.359402in}}%
\pgfpathcurveto{\pgfqpoint{2.564696in}{2.348352in}}{\pgfqpoint{2.569086in}{2.337753in}}{\pgfqpoint{2.576900in}{2.329939in}}%
\pgfpathcurveto{\pgfqpoint{2.584714in}{2.322126in}}{\pgfqpoint{2.595313in}{2.317735in}}{\pgfqpoint{2.606363in}{2.317735in}}%
\pgfpathclose%
\pgfusepath{stroke,fill}%
\end{pgfscope}%
\begin{pgfscope}%
\pgfpathrectangle{\pgfqpoint{0.600000in}{0.600000in}}{\pgfqpoint{3.900000in}{3.900000in}}%
\pgfusepath{clip}%
\pgfsetbuttcap%
\pgfsetroundjoin%
\definecolor{currentfill}{rgb}{0.121569,0.466667,0.705882}%
\pgfsetfillcolor{currentfill}%
\pgfsetlinewidth{1.003750pt}%
\definecolor{currentstroke}{rgb}{0.121569,0.466667,0.705882}%
\pgfsetstrokecolor{currentstroke}%
\pgfsetdash{}{0pt}%
\pgfpathmoveto{\pgfqpoint{2.302912in}{1.876699in}}%
\pgfpathcurveto{\pgfqpoint{2.313962in}{1.876699in}}{\pgfqpoint{2.324561in}{1.881089in}}{\pgfqpoint{2.332375in}{1.888903in}}%
\pgfpathcurveto{\pgfqpoint{2.340188in}{1.896716in}}{\pgfqpoint{2.344579in}{1.907315in}}{\pgfqpoint{2.344579in}{1.918365in}}%
\pgfpathcurveto{\pgfqpoint{2.344579in}{1.929416in}}{\pgfqpoint{2.340188in}{1.940015in}}{\pgfqpoint{2.332375in}{1.947828in}}%
\pgfpathcurveto{\pgfqpoint{2.324561in}{1.955642in}}{\pgfqpoint{2.313962in}{1.960032in}}{\pgfqpoint{2.302912in}{1.960032in}}%
\pgfpathcurveto{\pgfqpoint{2.291862in}{1.960032in}}{\pgfqpoint{2.281263in}{1.955642in}}{\pgfqpoint{2.273449in}{1.947828in}}%
\pgfpathcurveto{\pgfqpoint{2.265636in}{1.940015in}}{\pgfqpoint{2.261245in}{1.929416in}}{\pgfqpoint{2.261245in}{1.918365in}}%
\pgfpathcurveto{\pgfqpoint{2.261245in}{1.907315in}}{\pgfqpoint{2.265636in}{1.896716in}}{\pgfqpoint{2.273449in}{1.888903in}}%
\pgfpathcurveto{\pgfqpoint{2.281263in}{1.881089in}}{\pgfqpoint{2.291862in}{1.876699in}}{\pgfqpoint{2.302912in}{1.876699in}}%
\pgfpathclose%
\pgfusepath{stroke,fill}%
\end{pgfscope}%
\begin{pgfscope}%
\pgfpathrectangle{\pgfqpoint{0.600000in}{0.600000in}}{\pgfqpoint{3.900000in}{3.900000in}}%
\pgfusepath{clip}%
\pgfsetbuttcap%
\pgfsetroundjoin%
\definecolor{currentfill}{rgb}{0.121569,0.466667,0.705882}%
\pgfsetfillcolor{currentfill}%
\pgfsetlinewidth{1.003750pt}%
\definecolor{currentstroke}{rgb}{0.121569,0.466667,0.705882}%
\pgfsetstrokecolor{currentstroke}%
\pgfsetdash{}{0pt}%
\pgfpathmoveto{\pgfqpoint{2.109043in}{2.492733in}}%
\pgfpathcurveto{\pgfqpoint{2.120093in}{2.492733in}}{\pgfqpoint{2.130692in}{2.497123in}}{\pgfqpoint{2.138505in}{2.504936in}}%
\pgfpathcurveto{\pgfqpoint{2.146319in}{2.512750in}}{\pgfqpoint{2.150709in}{2.523349in}}{\pgfqpoint{2.150709in}{2.534399in}}%
\pgfpathcurveto{\pgfqpoint{2.150709in}{2.545449in}}{\pgfqpoint{2.146319in}{2.556048in}}{\pgfqpoint{2.138505in}{2.563862in}}%
\pgfpathcurveto{\pgfqpoint{2.130692in}{2.571676in}}{\pgfqpoint{2.120093in}{2.576066in}}{\pgfqpoint{2.109043in}{2.576066in}}%
\pgfpathcurveto{\pgfqpoint{2.097992in}{2.576066in}}{\pgfqpoint{2.087393in}{2.571676in}}{\pgfqpoint{2.079580in}{2.563862in}}%
\pgfpathcurveto{\pgfqpoint{2.071766in}{2.556048in}}{\pgfqpoint{2.067376in}{2.545449in}}{\pgfqpoint{2.067376in}{2.534399in}}%
\pgfpathcurveto{\pgfqpoint{2.067376in}{2.523349in}}{\pgfqpoint{2.071766in}{2.512750in}}{\pgfqpoint{2.079580in}{2.504936in}}%
\pgfpathcurveto{\pgfqpoint{2.087393in}{2.497123in}}{\pgfqpoint{2.097992in}{2.492733in}}{\pgfqpoint{2.109043in}{2.492733in}}%
\pgfpathclose%
\pgfusepath{stroke,fill}%
\end{pgfscope}%
\begin{pgfscope}%
\pgfpathrectangle{\pgfqpoint{0.600000in}{0.600000in}}{\pgfqpoint{3.900000in}{3.900000in}}%
\pgfusepath{clip}%
\pgfsetbuttcap%
\pgfsetroundjoin%
\definecolor{currentfill}{rgb}{0.121569,0.466667,0.705882}%
\pgfsetfillcolor{currentfill}%
\pgfsetlinewidth{1.003750pt}%
\definecolor{currentstroke}{rgb}{0.121569,0.466667,0.705882}%
\pgfsetstrokecolor{currentstroke}%
\pgfsetdash{}{0pt}%
\pgfpathmoveto{\pgfqpoint{2.798677in}{1.649943in}}%
\pgfpathcurveto{\pgfqpoint{2.809727in}{1.649943in}}{\pgfqpoint{2.820326in}{1.654334in}}{\pgfqpoint{2.828139in}{1.662147in}}%
\pgfpathcurveto{\pgfqpoint{2.835953in}{1.669961in}}{\pgfqpoint{2.840343in}{1.680560in}}{\pgfqpoint{2.840343in}{1.691610in}}%
\pgfpathcurveto{\pgfqpoint{2.840343in}{1.702660in}}{\pgfqpoint{2.835953in}{1.713259in}}{\pgfqpoint{2.828139in}{1.721073in}}%
\pgfpathcurveto{\pgfqpoint{2.820326in}{1.728886in}}{\pgfqpoint{2.809727in}{1.733277in}}{\pgfqpoint{2.798677in}{1.733277in}}%
\pgfpathcurveto{\pgfqpoint{2.787626in}{1.733277in}}{\pgfqpoint{2.777027in}{1.728886in}}{\pgfqpoint{2.769214in}{1.721073in}}%
\pgfpathcurveto{\pgfqpoint{2.761400in}{1.713259in}}{\pgfqpoint{2.757010in}{1.702660in}}{\pgfqpoint{2.757010in}{1.691610in}}%
\pgfpathcurveto{\pgfqpoint{2.757010in}{1.680560in}}{\pgfqpoint{2.761400in}{1.669961in}}{\pgfqpoint{2.769214in}{1.662147in}}%
\pgfpathcurveto{\pgfqpoint{2.777027in}{1.654334in}}{\pgfqpoint{2.787626in}{1.649943in}}{\pgfqpoint{2.798677in}{1.649943in}}%
\pgfpathclose%
\pgfusepath{stroke,fill}%
\end{pgfscope}%
\begin{pgfscope}%
\pgfpathrectangle{\pgfqpoint{0.600000in}{0.600000in}}{\pgfqpoint{3.900000in}{3.900000in}}%
\pgfusepath{clip}%
\pgfsetbuttcap%
\pgfsetroundjoin%
\definecolor{currentfill}{rgb}{0.121569,0.466667,0.705882}%
\pgfsetfillcolor{currentfill}%
\pgfsetlinewidth{1.003750pt}%
\definecolor{currentstroke}{rgb}{0.121569,0.466667,0.705882}%
\pgfsetstrokecolor{currentstroke}%
\pgfsetdash{}{0pt}%
\pgfpathmoveto{\pgfqpoint{1.962506in}{2.289599in}}%
\pgfpathcurveto{\pgfqpoint{1.973556in}{2.289599in}}{\pgfqpoint{1.984155in}{2.293989in}}{\pgfqpoint{1.991969in}{2.301803in}}%
\pgfpathcurveto{\pgfqpoint{1.999782in}{2.309616in}}{\pgfqpoint{2.004173in}{2.320215in}}{\pgfqpoint{2.004173in}{2.331266in}}%
\pgfpathcurveto{\pgfqpoint{2.004173in}{2.342316in}}{\pgfqpoint{1.999782in}{2.352915in}}{\pgfqpoint{1.991969in}{2.360728in}}%
\pgfpathcurveto{\pgfqpoint{1.984155in}{2.368542in}}{\pgfqpoint{1.973556in}{2.372932in}}{\pgfqpoint{1.962506in}{2.372932in}}%
\pgfpathcurveto{\pgfqpoint{1.951456in}{2.372932in}}{\pgfqpoint{1.940857in}{2.368542in}}{\pgfqpoint{1.933043in}{2.360728in}}%
\pgfpathcurveto{\pgfqpoint{1.925230in}{2.352915in}}{\pgfqpoint{1.920839in}{2.342316in}}{\pgfqpoint{1.920839in}{2.331266in}}%
\pgfpathcurveto{\pgfqpoint{1.920839in}{2.320215in}}{\pgfqpoint{1.925230in}{2.309616in}}{\pgfqpoint{1.933043in}{2.301803in}}%
\pgfpathcurveto{\pgfqpoint{1.940857in}{2.293989in}}{\pgfqpoint{1.951456in}{2.289599in}}{\pgfqpoint{1.962506in}{2.289599in}}%
\pgfpathclose%
\pgfusepath{stroke,fill}%
\end{pgfscope}%
\begin{pgfscope}%
\pgfpathrectangle{\pgfqpoint{0.600000in}{0.600000in}}{\pgfqpoint{3.900000in}{3.900000in}}%
\pgfusepath{clip}%
\pgfsetbuttcap%
\pgfsetroundjoin%
\definecolor{currentfill}{rgb}{0.121569,0.466667,0.705882}%
\pgfsetfillcolor{currentfill}%
\pgfsetlinewidth{1.003750pt}%
\definecolor{currentstroke}{rgb}{0.121569,0.466667,0.705882}%
\pgfsetstrokecolor{currentstroke}%
\pgfsetdash{}{0pt}%
\pgfpathmoveto{\pgfqpoint{1.284633in}{2.090281in}}%
\pgfpathcurveto{\pgfqpoint{1.295683in}{2.090281in}}{\pgfqpoint{1.306282in}{2.094672in}}{\pgfqpoint{1.314096in}{2.102485in}}%
\pgfpathcurveto{\pgfqpoint{1.321910in}{2.110299in}}{\pgfqpoint{1.326300in}{2.120898in}}{\pgfqpoint{1.326300in}{2.131948in}}%
\pgfpathcurveto{\pgfqpoint{1.326300in}{2.142998in}}{\pgfqpoint{1.321910in}{2.153597in}}{\pgfqpoint{1.314096in}{2.161411in}}%
\pgfpathcurveto{\pgfqpoint{1.306282in}{2.169224in}}{\pgfqpoint{1.295683in}{2.173615in}}{\pgfqpoint{1.284633in}{2.173615in}}%
\pgfpathcurveto{\pgfqpoint{1.273583in}{2.173615in}}{\pgfqpoint{1.262984in}{2.169224in}}{\pgfqpoint{1.255170in}{2.161411in}}%
\pgfpathcurveto{\pgfqpoint{1.247357in}{2.153597in}}{\pgfqpoint{1.242967in}{2.142998in}}{\pgfqpoint{1.242967in}{2.131948in}}%
\pgfpathcurveto{\pgfqpoint{1.242967in}{2.120898in}}{\pgfqpoint{1.247357in}{2.110299in}}{\pgfqpoint{1.255170in}{2.102485in}}%
\pgfpathcurveto{\pgfqpoint{1.262984in}{2.094672in}}{\pgfqpoint{1.273583in}{2.090281in}}{\pgfqpoint{1.284633in}{2.090281in}}%
\pgfpathclose%
\pgfusepath{stroke,fill}%
\end{pgfscope}%
\begin{pgfscope}%
\pgfpathrectangle{\pgfqpoint{0.600000in}{0.600000in}}{\pgfqpoint{3.900000in}{3.900000in}}%
\pgfusepath{clip}%
\pgfsetbuttcap%
\pgfsetroundjoin%
\definecolor{currentfill}{rgb}{0.121569,0.466667,0.705882}%
\pgfsetfillcolor{currentfill}%
\pgfsetlinewidth{1.003750pt}%
\definecolor{currentstroke}{rgb}{0.121569,0.466667,0.705882}%
\pgfsetstrokecolor{currentstroke}%
\pgfsetdash{}{0pt}%
\pgfpathmoveto{\pgfqpoint{2.282165in}{2.725220in}}%
\pgfpathcurveto{\pgfqpoint{2.293215in}{2.725220in}}{\pgfqpoint{2.303814in}{2.729610in}}{\pgfqpoint{2.311627in}{2.737424in}}%
\pgfpathcurveto{\pgfqpoint{2.319441in}{2.745238in}}{\pgfqpoint{2.323831in}{2.755837in}}{\pgfqpoint{2.323831in}{2.766887in}}%
\pgfpathcurveto{\pgfqpoint{2.323831in}{2.777937in}}{\pgfqpoint{2.319441in}{2.788536in}}{\pgfqpoint{2.311627in}{2.796350in}}%
\pgfpathcurveto{\pgfqpoint{2.303814in}{2.804163in}}{\pgfqpoint{2.293215in}{2.808554in}}{\pgfqpoint{2.282165in}{2.808554in}}%
\pgfpathcurveto{\pgfqpoint{2.271115in}{2.808554in}}{\pgfqpoint{2.260515in}{2.804163in}}{\pgfqpoint{2.252702in}{2.796350in}}%
\pgfpathcurveto{\pgfqpoint{2.244888in}{2.788536in}}{\pgfqpoint{2.240498in}{2.777937in}}{\pgfqpoint{2.240498in}{2.766887in}}%
\pgfpathcurveto{\pgfqpoint{2.240498in}{2.755837in}}{\pgfqpoint{2.244888in}{2.745238in}}{\pgfqpoint{2.252702in}{2.737424in}}%
\pgfpathcurveto{\pgfqpoint{2.260515in}{2.729610in}}{\pgfqpoint{2.271115in}{2.725220in}}{\pgfqpoint{2.282165in}{2.725220in}}%
\pgfpathclose%
\pgfusepath{stroke,fill}%
\end{pgfscope}%
\begin{pgfscope}%
\pgfpathrectangle{\pgfqpoint{0.600000in}{0.600000in}}{\pgfqpoint{3.900000in}{3.900000in}}%
\pgfusepath{clip}%
\pgfsetbuttcap%
\pgfsetroundjoin%
\definecolor{currentfill}{rgb}{0.121569,0.466667,0.705882}%
\pgfsetfillcolor{currentfill}%
\pgfsetlinewidth{1.003750pt}%
\definecolor{currentstroke}{rgb}{0.121569,0.466667,0.705882}%
\pgfsetstrokecolor{currentstroke}%
\pgfsetdash{}{0pt}%
\pgfpathmoveto{\pgfqpoint{2.925136in}{2.891959in}}%
\pgfpathcurveto{\pgfqpoint{2.936186in}{2.891959in}}{\pgfqpoint{2.946785in}{2.896349in}}{\pgfqpoint{2.954599in}{2.904163in}}%
\pgfpathcurveto{\pgfqpoint{2.962412in}{2.911976in}}{\pgfqpoint{2.966802in}{2.922575in}}{\pgfqpoint{2.966802in}{2.933625in}}%
\pgfpathcurveto{\pgfqpoint{2.966802in}{2.944675in}}{\pgfqpoint{2.962412in}{2.955275in}}{\pgfqpoint{2.954599in}{2.963088in}}%
\pgfpathcurveto{\pgfqpoint{2.946785in}{2.970902in}}{\pgfqpoint{2.936186in}{2.975292in}}{\pgfqpoint{2.925136in}{2.975292in}}%
\pgfpathcurveto{\pgfqpoint{2.914086in}{2.975292in}}{\pgfqpoint{2.903487in}{2.970902in}}{\pgfqpoint{2.895673in}{2.963088in}}%
\pgfpathcurveto{\pgfqpoint{2.887859in}{2.955275in}}{\pgfqpoint{2.883469in}{2.944675in}}{\pgfqpoint{2.883469in}{2.933625in}}%
\pgfpathcurveto{\pgfqpoint{2.883469in}{2.922575in}}{\pgfqpoint{2.887859in}{2.911976in}}{\pgfqpoint{2.895673in}{2.904163in}}%
\pgfpathcurveto{\pgfqpoint{2.903487in}{2.896349in}}{\pgfqpoint{2.914086in}{2.891959in}}{\pgfqpoint{2.925136in}{2.891959in}}%
\pgfpathclose%
\pgfusepath{stroke,fill}%
\end{pgfscope}%
\begin{pgfscope}%
\pgfpathrectangle{\pgfqpoint{0.600000in}{0.600000in}}{\pgfqpoint{3.900000in}{3.900000in}}%
\pgfusepath{clip}%
\pgfsetbuttcap%
\pgfsetroundjoin%
\definecolor{currentfill}{rgb}{0.121569,0.466667,0.705882}%
\pgfsetfillcolor{currentfill}%
\pgfsetlinewidth{1.003750pt}%
\definecolor{currentstroke}{rgb}{0.121569,0.466667,0.705882}%
\pgfsetstrokecolor{currentstroke}%
\pgfsetdash{}{0pt}%
\pgfpathmoveto{\pgfqpoint{1.739357in}{2.675047in}}%
\pgfpathcurveto{\pgfqpoint{1.750407in}{2.675047in}}{\pgfqpoint{1.761006in}{2.679438in}}{\pgfqpoint{1.768820in}{2.687251in}}%
\pgfpathcurveto{\pgfqpoint{1.776634in}{2.695065in}}{\pgfqpoint{1.781024in}{2.705664in}}{\pgfqpoint{1.781024in}{2.716714in}}%
\pgfpathcurveto{\pgfqpoint{1.781024in}{2.727764in}}{\pgfqpoint{1.776634in}{2.738363in}}{\pgfqpoint{1.768820in}{2.746177in}}%
\pgfpathcurveto{\pgfqpoint{1.761006in}{2.753990in}}{\pgfqpoint{1.750407in}{2.758381in}}{\pgfqpoint{1.739357in}{2.758381in}}%
\pgfpathcurveto{\pgfqpoint{1.728307in}{2.758381in}}{\pgfqpoint{1.717708in}{2.753990in}}{\pgfqpoint{1.709895in}{2.746177in}}%
\pgfpathcurveto{\pgfqpoint{1.702081in}{2.738363in}}{\pgfqpoint{1.697691in}{2.727764in}}{\pgfqpoint{1.697691in}{2.716714in}}%
\pgfpathcurveto{\pgfqpoint{1.697691in}{2.705664in}}{\pgfqpoint{1.702081in}{2.695065in}}{\pgfqpoint{1.709895in}{2.687251in}}%
\pgfpathcurveto{\pgfqpoint{1.717708in}{2.679438in}}{\pgfqpoint{1.728307in}{2.675047in}}{\pgfqpoint{1.739357in}{2.675047in}}%
\pgfpathclose%
\pgfusepath{stroke,fill}%
\end{pgfscope}%
\begin{pgfscope}%
\pgfpathrectangle{\pgfqpoint{0.600000in}{0.600000in}}{\pgfqpoint{3.900000in}{3.900000in}}%
\pgfusepath{clip}%
\pgfsetbuttcap%
\pgfsetroundjoin%
\definecolor{currentfill}{rgb}{0.121569,0.466667,0.705882}%
\pgfsetfillcolor{currentfill}%
\pgfsetlinewidth{1.003750pt}%
\definecolor{currentstroke}{rgb}{0.121569,0.466667,0.705882}%
\pgfsetstrokecolor{currentstroke}%
\pgfsetdash{}{0pt}%
\pgfpathmoveto{\pgfqpoint{2.051813in}{2.684807in}}%
\pgfpathcurveto{\pgfqpoint{2.062863in}{2.684807in}}{\pgfqpoint{2.073462in}{2.689197in}}{\pgfqpoint{2.081276in}{2.697010in}}%
\pgfpathcurveto{\pgfqpoint{2.089089in}{2.704824in}}{\pgfqpoint{2.093479in}{2.715423in}}{\pgfqpoint{2.093479in}{2.726473in}}%
\pgfpathcurveto{\pgfqpoint{2.093479in}{2.737523in}}{\pgfqpoint{2.089089in}{2.748122in}}{\pgfqpoint{2.081276in}{2.755936in}}%
\pgfpathcurveto{\pgfqpoint{2.073462in}{2.763750in}}{\pgfqpoint{2.062863in}{2.768140in}}{\pgfqpoint{2.051813in}{2.768140in}}%
\pgfpathcurveto{\pgfqpoint{2.040763in}{2.768140in}}{\pgfqpoint{2.030164in}{2.763750in}}{\pgfqpoint{2.022350in}{2.755936in}}%
\pgfpathcurveto{\pgfqpoint{2.014536in}{2.748122in}}{\pgfqpoint{2.010146in}{2.737523in}}{\pgfqpoint{2.010146in}{2.726473in}}%
\pgfpathcurveto{\pgfqpoint{2.010146in}{2.715423in}}{\pgfqpoint{2.014536in}{2.704824in}}{\pgfqpoint{2.022350in}{2.697010in}}%
\pgfpathcurveto{\pgfqpoint{2.030164in}{2.689197in}}{\pgfqpoint{2.040763in}{2.684807in}}{\pgfqpoint{2.051813in}{2.684807in}}%
\pgfpathclose%
\pgfusepath{stroke,fill}%
\end{pgfscope}%
\begin{pgfscope}%
\pgfpathrectangle{\pgfqpoint{0.600000in}{0.600000in}}{\pgfqpoint{3.900000in}{3.900000in}}%
\pgfusepath{clip}%
\pgfsetbuttcap%
\pgfsetroundjoin%
\definecolor{currentfill}{rgb}{0.121569,0.466667,0.705882}%
\pgfsetfillcolor{currentfill}%
\pgfsetlinewidth{1.003750pt}%
\definecolor{currentstroke}{rgb}{0.121569,0.466667,0.705882}%
\pgfsetstrokecolor{currentstroke}%
\pgfsetdash{}{0pt}%
\pgfpathmoveto{\pgfqpoint{2.545201in}{3.107685in}}%
\pgfpathcurveto{\pgfqpoint{2.556251in}{3.107685in}}{\pgfqpoint{2.566850in}{3.112075in}}{\pgfqpoint{2.574664in}{3.119889in}}%
\pgfpathcurveto{\pgfqpoint{2.582477in}{3.127703in}}{\pgfqpoint{2.586868in}{3.138302in}}{\pgfqpoint{2.586868in}{3.149352in}}%
\pgfpathcurveto{\pgfqpoint{2.586868in}{3.160402in}}{\pgfqpoint{2.582477in}{3.171001in}}{\pgfqpoint{2.574664in}{3.178815in}}%
\pgfpathcurveto{\pgfqpoint{2.566850in}{3.186628in}}{\pgfqpoint{2.556251in}{3.191019in}}{\pgfqpoint{2.545201in}{3.191019in}}%
\pgfpathcurveto{\pgfqpoint{2.534151in}{3.191019in}}{\pgfqpoint{2.523552in}{3.186628in}}{\pgfqpoint{2.515738in}{3.178815in}}%
\pgfpathcurveto{\pgfqpoint{2.507925in}{3.171001in}}{\pgfqpoint{2.503534in}{3.160402in}}{\pgfqpoint{2.503534in}{3.149352in}}%
\pgfpathcurveto{\pgfqpoint{2.503534in}{3.138302in}}{\pgfqpoint{2.507925in}{3.127703in}}{\pgfqpoint{2.515738in}{3.119889in}}%
\pgfpathcurveto{\pgfqpoint{2.523552in}{3.112075in}}{\pgfqpoint{2.534151in}{3.107685in}}{\pgfqpoint{2.545201in}{3.107685in}}%
\pgfpathclose%
\pgfusepath{stroke,fill}%
\end{pgfscope}%
\begin{pgfscope}%
\pgfpathrectangle{\pgfqpoint{0.600000in}{0.600000in}}{\pgfqpoint{3.900000in}{3.900000in}}%
\pgfusepath{clip}%
\pgfsetbuttcap%
\pgfsetroundjoin%
\definecolor{currentfill}{rgb}{0.121569,0.466667,0.705882}%
\pgfsetfillcolor{currentfill}%
\pgfsetlinewidth{1.003750pt}%
\definecolor{currentstroke}{rgb}{0.121569,0.466667,0.705882}%
\pgfsetstrokecolor{currentstroke}%
\pgfsetdash{}{0pt}%
\pgfpathmoveto{\pgfqpoint{2.077927in}{3.064868in}}%
\pgfpathcurveto{\pgfqpoint{2.088977in}{3.064868in}}{\pgfqpoint{2.099576in}{3.069259in}}{\pgfqpoint{2.107390in}{3.077072in}}%
\pgfpathcurveto{\pgfqpoint{2.115204in}{3.084886in}}{\pgfqpoint{2.119594in}{3.095485in}}{\pgfqpoint{2.119594in}{3.106535in}}%
\pgfpathcurveto{\pgfqpoint{2.119594in}{3.117585in}}{\pgfqpoint{2.115204in}{3.128184in}}{\pgfqpoint{2.107390in}{3.135998in}}%
\pgfpathcurveto{\pgfqpoint{2.099576in}{3.143811in}}{\pgfqpoint{2.088977in}{3.148202in}}{\pgfqpoint{2.077927in}{3.148202in}}%
\pgfpathcurveto{\pgfqpoint{2.066877in}{3.148202in}}{\pgfqpoint{2.056278in}{3.143811in}}{\pgfqpoint{2.048464in}{3.135998in}}%
\pgfpathcurveto{\pgfqpoint{2.040651in}{3.128184in}}{\pgfqpoint{2.036260in}{3.117585in}}{\pgfqpoint{2.036260in}{3.106535in}}%
\pgfpathcurveto{\pgfqpoint{2.036260in}{3.095485in}}{\pgfqpoint{2.040651in}{3.084886in}}{\pgfqpoint{2.048464in}{3.077072in}}%
\pgfpathcurveto{\pgfqpoint{2.056278in}{3.069259in}}{\pgfqpoint{2.066877in}{3.064868in}}{\pgfqpoint{2.077927in}{3.064868in}}%
\pgfpathclose%
\pgfusepath{stroke,fill}%
\end{pgfscope}%
\begin{pgfscope}%
\pgfpathrectangle{\pgfqpoint{0.600000in}{0.600000in}}{\pgfqpoint{3.900000in}{3.900000in}}%
\pgfusepath{clip}%
\pgfsetbuttcap%
\pgfsetroundjoin%
\definecolor{currentfill}{rgb}{0.121569,0.466667,0.705882}%
\pgfsetfillcolor{currentfill}%
\pgfsetlinewidth{1.003750pt}%
\definecolor{currentstroke}{rgb}{0.121569,0.466667,0.705882}%
\pgfsetstrokecolor{currentstroke}%
\pgfsetdash{}{0pt}%
\pgfpathmoveto{\pgfqpoint{3.424826in}{2.344525in}}%
\pgfpathcurveto{\pgfqpoint{3.435876in}{2.344525in}}{\pgfqpoint{3.446475in}{2.348916in}}{\pgfqpoint{3.454289in}{2.356729in}}%
\pgfpathcurveto{\pgfqpoint{3.462102in}{2.364543in}}{\pgfqpoint{3.466493in}{2.375142in}}{\pgfqpoint{3.466493in}{2.386192in}}%
\pgfpathcurveto{\pgfqpoint{3.466493in}{2.397242in}}{\pgfqpoint{3.462102in}{2.407841in}}{\pgfqpoint{3.454289in}{2.415655in}}%
\pgfpathcurveto{\pgfqpoint{3.446475in}{2.423468in}}{\pgfqpoint{3.435876in}{2.427859in}}{\pgfqpoint{3.424826in}{2.427859in}}%
\pgfpathcurveto{\pgfqpoint{3.413776in}{2.427859in}}{\pgfqpoint{3.403177in}{2.423468in}}{\pgfqpoint{3.395363in}{2.415655in}}%
\pgfpathcurveto{\pgfqpoint{3.387550in}{2.407841in}}{\pgfqpoint{3.383159in}{2.397242in}}{\pgfqpoint{3.383159in}{2.386192in}}%
\pgfpathcurveto{\pgfqpoint{3.383159in}{2.375142in}}{\pgfqpoint{3.387550in}{2.364543in}}{\pgfqpoint{3.395363in}{2.356729in}}%
\pgfpathcurveto{\pgfqpoint{3.403177in}{2.348916in}}{\pgfqpoint{3.413776in}{2.344525in}}{\pgfqpoint{3.424826in}{2.344525in}}%
\pgfpathclose%
\pgfusepath{stroke,fill}%
\end{pgfscope}%
\begin{pgfscope}%
\pgfpathrectangle{\pgfqpoint{0.600000in}{0.600000in}}{\pgfqpoint{3.900000in}{3.900000in}}%
\pgfusepath{clip}%
\pgfsetbuttcap%
\pgfsetroundjoin%
\definecolor{currentfill}{rgb}{0.121569,0.466667,0.705882}%
\pgfsetfillcolor{currentfill}%
\pgfsetlinewidth{1.003750pt}%
\definecolor{currentstroke}{rgb}{0.121569,0.466667,0.705882}%
\pgfsetstrokecolor{currentstroke}%
\pgfsetdash{}{0pt}%
\pgfpathmoveto{\pgfqpoint{2.753858in}{3.197299in}}%
\pgfpathcurveto{\pgfqpoint{2.764908in}{3.197299in}}{\pgfqpoint{2.775507in}{3.201689in}}{\pgfqpoint{2.783320in}{3.209503in}}%
\pgfpathcurveto{\pgfqpoint{2.791134in}{3.217317in}}{\pgfqpoint{2.795524in}{3.227916in}}{\pgfqpoint{2.795524in}{3.238966in}}%
\pgfpathcurveto{\pgfqpoint{2.795524in}{3.250016in}}{\pgfqpoint{2.791134in}{3.260615in}}{\pgfqpoint{2.783320in}{3.268429in}}%
\pgfpathcurveto{\pgfqpoint{2.775507in}{3.276242in}}{\pgfqpoint{2.764908in}{3.280633in}}{\pgfqpoint{2.753858in}{3.280633in}}%
\pgfpathcurveto{\pgfqpoint{2.742807in}{3.280633in}}{\pgfqpoint{2.732208in}{3.276242in}}{\pgfqpoint{2.724395in}{3.268429in}}%
\pgfpathcurveto{\pgfqpoint{2.716581in}{3.260615in}}{\pgfqpoint{2.712191in}{3.250016in}}{\pgfqpoint{2.712191in}{3.238966in}}%
\pgfpathcurveto{\pgfqpoint{2.712191in}{3.227916in}}{\pgfqpoint{2.716581in}{3.217317in}}{\pgfqpoint{2.724395in}{3.209503in}}%
\pgfpathcurveto{\pgfqpoint{2.732208in}{3.201689in}}{\pgfqpoint{2.742807in}{3.197299in}}{\pgfqpoint{2.753858in}{3.197299in}}%
\pgfpathclose%
\pgfusepath{stroke,fill}%
\end{pgfscope}%
\begin{pgfscope}%
\pgfpathrectangle{\pgfqpoint{0.600000in}{0.600000in}}{\pgfqpoint{3.900000in}{3.900000in}}%
\pgfusepath{clip}%
\pgfsetbuttcap%
\pgfsetroundjoin%
\definecolor{currentfill}{rgb}{0.121569,0.466667,0.705882}%
\pgfsetfillcolor{currentfill}%
\pgfsetlinewidth{1.003750pt}%
\definecolor{currentstroke}{rgb}{0.121569,0.466667,0.705882}%
\pgfsetstrokecolor{currentstroke}%
\pgfsetdash{}{0pt}%
\pgfpathmoveto{\pgfqpoint{2.926224in}{2.704946in}}%
\pgfpathcurveto{\pgfqpoint{2.937274in}{2.704946in}}{\pgfqpoint{2.947873in}{2.709337in}}{\pgfqpoint{2.955687in}{2.717150in}}%
\pgfpathcurveto{\pgfqpoint{2.963501in}{2.724964in}}{\pgfqpoint{2.967891in}{2.735563in}}{\pgfqpoint{2.967891in}{2.746613in}}%
\pgfpathcurveto{\pgfqpoint{2.967891in}{2.757663in}}{\pgfqpoint{2.963501in}{2.768262in}}{\pgfqpoint{2.955687in}{2.776076in}}%
\pgfpathcurveto{\pgfqpoint{2.947873in}{2.783890in}}{\pgfqpoint{2.937274in}{2.788280in}}{\pgfqpoint{2.926224in}{2.788280in}}%
\pgfpathcurveto{\pgfqpoint{2.915174in}{2.788280in}}{\pgfqpoint{2.904575in}{2.783890in}}{\pgfqpoint{2.896762in}{2.776076in}}%
\pgfpathcurveto{\pgfqpoint{2.888948in}{2.768262in}}{\pgfqpoint{2.884558in}{2.757663in}}{\pgfqpoint{2.884558in}{2.746613in}}%
\pgfpathcurveto{\pgfqpoint{2.884558in}{2.735563in}}{\pgfqpoint{2.888948in}{2.724964in}}{\pgfqpoint{2.896762in}{2.717150in}}%
\pgfpathcurveto{\pgfqpoint{2.904575in}{2.709337in}}{\pgfqpoint{2.915174in}{2.704946in}}{\pgfqpoint{2.926224in}{2.704946in}}%
\pgfpathclose%
\pgfusepath{stroke,fill}%
\end{pgfscope}%
\begin{pgfscope}%
\pgfpathrectangle{\pgfqpoint{0.600000in}{0.600000in}}{\pgfqpoint{3.900000in}{3.900000in}}%
\pgfusepath{clip}%
\pgfsetbuttcap%
\pgfsetroundjoin%
\definecolor{currentfill}{rgb}{0.121569,0.466667,0.705882}%
\pgfsetfillcolor{currentfill}%
\pgfsetlinewidth{1.003750pt}%
\definecolor{currentstroke}{rgb}{0.121569,0.466667,0.705882}%
\pgfsetstrokecolor{currentstroke}%
\pgfsetdash{}{0pt}%
\pgfpathmoveto{\pgfqpoint{2.305508in}{3.172425in}}%
\pgfpathcurveto{\pgfqpoint{2.316558in}{3.172425in}}{\pgfqpoint{2.327157in}{3.176815in}}{\pgfqpoint{2.334971in}{3.184629in}}%
\pgfpathcurveto{\pgfqpoint{2.342784in}{3.192442in}}{\pgfqpoint{2.347175in}{3.203041in}}{\pgfqpoint{2.347175in}{3.214091in}}%
\pgfpathcurveto{\pgfqpoint{2.347175in}{3.225142in}}{\pgfqpoint{2.342784in}{3.235741in}}{\pgfqpoint{2.334971in}{3.243554in}}%
\pgfpathcurveto{\pgfqpoint{2.327157in}{3.251368in}}{\pgfqpoint{2.316558in}{3.255758in}}{\pgfqpoint{2.305508in}{3.255758in}}%
\pgfpathcurveto{\pgfqpoint{2.294458in}{3.255758in}}{\pgfqpoint{2.283859in}{3.251368in}}{\pgfqpoint{2.276045in}{3.243554in}}%
\pgfpathcurveto{\pgfqpoint{2.268232in}{3.235741in}}{\pgfqpoint{2.263841in}{3.225142in}}{\pgfqpoint{2.263841in}{3.214091in}}%
\pgfpathcurveto{\pgfqpoint{2.263841in}{3.203041in}}{\pgfqpoint{2.268232in}{3.192442in}}{\pgfqpoint{2.276045in}{3.184629in}}%
\pgfpathcurveto{\pgfqpoint{2.283859in}{3.176815in}}{\pgfqpoint{2.294458in}{3.172425in}}{\pgfqpoint{2.305508in}{3.172425in}}%
\pgfpathclose%
\pgfusepath{stroke,fill}%
\end{pgfscope}%
\begin{pgfscope}%
\pgfpathrectangle{\pgfqpoint{0.600000in}{0.600000in}}{\pgfqpoint{3.900000in}{3.900000in}}%
\pgfusepath{clip}%
\pgfsetbuttcap%
\pgfsetroundjoin%
\definecolor{currentfill}{rgb}{0.121569,0.466667,0.705882}%
\pgfsetfillcolor{currentfill}%
\pgfsetlinewidth{1.003750pt}%
\definecolor{currentstroke}{rgb}{0.121569,0.466667,0.705882}%
\pgfsetstrokecolor{currentstroke}%
\pgfsetdash{}{0pt}%
\pgfpathmoveto{\pgfqpoint{2.863895in}{2.102648in}}%
\pgfpathcurveto{\pgfqpoint{2.874945in}{2.102648in}}{\pgfqpoint{2.885544in}{2.107038in}}{\pgfqpoint{2.893357in}{2.114852in}}%
\pgfpathcurveto{\pgfqpoint{2.901171in}{2.122666in}}{\pgfqpoint{2.905561in}{2.133265in}}{\pgfqpoint{2.905561in}{2.144315in}}%
\pgfpathcurveto{\pgfqpoint{2.905561in}{2.155365in}}{\pgfqpoint{2.901171in}{2.165964in}}{\pgfqpoint{2.893357in}{2.173778in}}%
\pgfpathcurveto{\pgfqpoint{2.885544in}{2.181591in}}{\pgfqpoint{2.874945in}{2.185981in}}{\pgfqpoint{2.863895in}{2.185981in}}%
\pgfpathcurveto{\pgfqpoint{2.852845in}{2.185981in}}{\pgfqpoint{2.842246in}{2.181591in}}{\pgfqpoint{2.834432in}{2.173778in}}%
\pgfpathcurveto{\pgfqpoint{2.826618in}{2.165964in}}{\pgfqpoint{2.822228in}{2.155365in}}{\pgfqpoint{2.822228in}{2.144315in}}%
\pgfpathcurveto{\pgfqpoint{2.822228in}{2.133265in}}{\pgfqpoint{2.826618in}{2.122666in}}{\pgfqpoint{2.834432in}{2.114852in}}%
\pgfpathcurveto{\pgfqpoint{2.842246in}{2.107038in}}{\pgfqpoint{2.852845in}{2.102648in}}{\pgfqpoint{2.863895in}{2.102648in}}%
\pgfpathclose%
\pgfusepath{stroke,fill}%
\end{pgfscope}%
\begin{pgfscope}%
\pgfpathrectangle{\pgfqpoint{0.600000in}{0.600000in}}{\pgfqpoint{3.900000in}{3.900000in}}%
\pgfusepath{clip}%
\pgfsetbuttcap%
\pgfsetroundjoin%
\definecolor{currentfill}{rgb}{0.121569,0.466667,0.705882}%
\pgfsetfillcolor{currentfill}%
\pgfsetlinewidth{1.003750pt}%
\definecolor{currentstroke}{rgb}{0.121569,0.466667,0.705882}%
\pgfsetstrokecolor{currentstroke}%
\pgfsetdash{}{0pt}%
\pgfpathmoveto{\pgfqpoint{1.965453in}{2.375873in}}%
\pgfpathcurveto{\pgfqpoint{1.976503in}{2.375873in}}{\pgfqpoint{1.987102in}{2.380264in}}{\pgfqpoint{1.994916in}{2.388077in}}%
\pgfpathcurveto{\pgfqpoint{2.002729in}{2.395891in}}{\pgfqpoint{2.007120in}{2.406490in}}{\pgfqpoint{2.007120in}{2.417540in}}%
\pgfpathcurveto{\pgfqpoint{2.007120in}{2.428590in}}{\pgfqpoint{2.002729in}{2.439189in}}{\pgfqpoint{1.994916in}{2.447003in}}%
\pgfpathcurveto{\pgfqpoint{1.987102in}{2.454817in}}{\pgfqpoint{1.976503in}{2.459207in}}{\pgfqpoint{1.965453in}{2.459207in}}%
\pgfpathcurveto{\pgfqpoint{1.954403in}{2.459207in}}{\pgfqpoint{1.943804in}{2.454817in}}{\pgfqpoint{1.935990in}{2.447003in}}%
\pgfpathcurveto{\pgfqpoint{1.928177in}{2.439189in}}{\pgfqpoint{1.923786in}{2.428590in}}{\pgfqpoint{1.923786in}{2.417540in}}%
\pgfpathcurveto{\pgfqpoint{1.923786in}{2.406490in}}{\pgfqpoint{1.928177in}{2.395891in}}{\pgfqpoint{1.935990in}{2.388077in}}%
\pgfpathcurveto{\pgfqpoint{1.943804in}{2.380264in}}{\pgfqpoint{1.954403in}{2.375873in}}{\pgfqpoint{1.965453in}{2.375873in}}%
\pgfpathclose%
\pgfusepath{stroke,fill}%
\end{pgfscope}%
\begin{pgfscope}%
\pgfpathrectangle{\pgfqpoint{0.600000in}{0.600000in}}{\pgfqpoint{3.900000in}{3.900000in}}%
\pgfusepath{clip}%
\pgfsetbuttcap%
\pgfsetroundjoin%
\definecolor{currentfill}{rgb}{0.121569,0.466667,0.705882}%
\pgfsetfillcolor{currentfill}%
\pgfsetlinewidth{1.003750pt}%
\definecolor{currentstroke}{rgb}{0.121569,0.466667,0.705882}%
\pgfsetstrokecolor{currentstroke}%
\pgfsetdash{}{0pt}%
\pgfpathmoveto{\pgfqpoint{2.632149in}{2.091518in}}%
\pgfpathcurveto{\pgfqpoint{2.643200in}{2.091518in}}{\pgfqpoint{2.653799in}{2.095908in}}{\pgfqpoint{2.661612in}{2.103722in}}%
\pgfpathcurveto{\pgfqpoint{2.669426in}{2.111535in}}{\pgfqpoint{2.673816in}{2.122134in}}{\pgfqpoint{2.673816in}{2.133185in}}%
\pgfpathcurveto{\pgfqpoint{2.673816in}{2.144235in}}{\pgfqpoint{2.669426in}{2.154834in}}{\pgfqpoint{2.661612in}{2.162647in}}%
\pgfpathcurveto{\pgfqpoint{2.653799in}{2.170461in}}{\pgfqpoint{2.643200in}{2.174851in}}{\pgfqpoint{2.632149in}{2.174851in}}%
\pgfpathcurveto{\pgfqpoint{2.621099in}{2.174851in}}{\pgfqpoint{2.610500in}{2.170461in}}{\pgfqpoint{2.602687in}{2.162647in}}%
\pgfpathcurveto{\pgfqpoint{2.594873in}{2.154834in}}{\pgfqpoint{2.590483in}{2.144235in}}{\pgfqpoint{2.590483in}{2.133185in}}%
\pgfpathcurveto{\pgfqpoint{2.590483in}{2.122134in}}{\pgfqpoint{2.594873in}{2.111535in}}{\pgfqpoint{2.602687in}{2.103722in}}%
\pgfpathcurveto{\pgfqpoint{2.610500in}{2.095908in}}{\pgfqpoint{2.621099in}{2.091518in}}{\pgfqpoint{2.632149in}{2.091518in}}%
\pgfpathclose%
\pgfusepath{stroke,fill}%
\end{pgfscope}%
\begin{pgfscope}%
\pgfpathrectangle{\pgfqpoint{0.600000in}{0.600000in}}{\pgfqpoint{3.900000in}{3.900000in}}%
\pgfusepath{clip}%
\pgfsetbuttcap%
\pgfsetroundjoin%
\definecolor{currentfill}{rgb}{0.121569,0.466667,0.705882}%
\pgfsetfillcolor{currentfill}%
\pgfsetlinewidth{1.003750pt}%
\definecolor{currentstroke}{rgb}{0.121569,0.466667,0.705882}%
\pgfsetstrokecolor{currentstroke}%
\pgfsetdash{}{0pt}%
\pgfpathmoveto{\pgfqpoint{2.731719in}{2.457542in}}%
\pgfpathcurveto{\pgfqpoint{2.742769in}{2.457542in}}{\pgfqpoint{2.753368in}{2.461932in}}{\pgfqpoint{2.761181in}{2.469746in}}%
\pgfpathcurveto{\pgfqpoint{2.768995in}{2.477559in}}{\pgfqpoint{2.773385in}{2.488158in}}{\pgfqpoint{2.773385in}{2.499208in}}%
\pgfpathcurveto{\pgfqpoint{2.773385in}{2.510259in}}{\pgfqpoint{2.768995in}{2.520858in}}{\pgfqpoint{2.761181in}{2.528671in}}%
\pgfpathcurveto{\pgfqpoint{2.753368in}{2.536485in}}{\pgfqpoint{2.742769in}{2.540875in}}{\pgfqpoint{2.731719in}{2.540875in}}%
\pgfpathcurveto{\pgfqpoint{2.720668in}{2.540875in}}{\pgfqpoint{2.710069in}{2.536485in}}{\pgfqpoint{2.702256in}{2.528671in}}%
\pgfpathcurveto{\pgfqpoint{2.694442in}{2.520858in}}{\pgfqpoint{2.690052in}{2.510259in}}{\pgfqpoint{2.690052in}{2.499208in}}%
\pgfpathcurveto{\pgfqpoint{2.690052in}{2.488158in}}{\pgfqpoint{2.694442in}{2.477559in}}{\pgfqpoint{2.702256in}{2.469746in}}%
\pgfpathcurveto{\pgfqpoint{2.710069in}{2.461932in}}{\pgfqpoint{2.720668in}{2.457542in}}{\pgfqpoint{2.731719in}{2.457542in}}%
\pgfpathclose%
\pgfusepath{stroke,fill}%
\end{pgfscope}%
\begin{pgfscope}%
\pgfpathrectangle{\pgfqpoint{0.600000in}{0.600000in}}{\pgfqpoint{3.900000in}{3.900000in}}%
\pgfusepath{clip}%
\pgfsetbuttcap%
\pgfsetroundjoin%
\definecolor{currentfill}{rgb}{0.121569,0.466667,0.705882}%
\pgfsetfillcolor{currentfill}%
\pgfsetlinewidth{1.003750pt}%
\definecolor{currentstroke}{rgb}{0.121569,0.466667,0.705882}%
\pgfsetstrokecolor{currentstroke}%
\pgfsetdash{}{0pt}%
\pgfpathmoveto{\pgfqpoint{2.917177in}{3.064061in}}%
\pgfpathcurveto{\pgfqpoint{2.928227in}{3.064061in}}{\pgfqpoint{2.938826in}{3.068452in}}{\pgfqpoint{2.946639in}{3.076265in}}%
\pgfpathcurveto{\pgfqpoint{2.954453in}{3.084079in}}{\pgfqpoint{2.958843in}{3.094678in}}{\pgfqpoint{2.958843in}{3.105728in}}%
\pgfpathcurveto{\pgfqpoint{2.958843in}{3.116778in}}{\pgfqpoint{2.954453in}{3.127377in}}{\pgfqpoint{2.946639in}{3.135191in}}%
\pgfpathcurveto{\pgfqpoint{2.938826in}{3.143004in}}{\pgfqpoint{2.928227in}{3.147395in}}{\pgfqpoint{2.917177in}{3.147395in}}%
\pgfpathcurveto{\pgfqpoint{2.906126in}{3.147395in}}{\pgfqpoint{2.895527in}{3.143004in}}{\pgfqpoint{2.887714in}{3.135191in}}%
\pgfpathcurveto{\pgfqpoint{2.879900in}{3.127377in}}{\pgfqpoint{2.875510in}{3.116778in}}{\pgfqpoint{2.875510in}{3.105728in}}%
\pgfpathcurveto{\pgfqpoint{2.875510in}{3.094678in}}{\pgfqpoint{2.879900in}{3.084079in}}{\pgfqpoint{2.887714in}{3.076265in}}%
\pgfpathcurveto{\pgfqpoint{2.895527in}{3.068452in}}{\pgfqpoint{2.906126in}{3.064061in}}{\pgfqpoint{2.917177in}{3.064061in}}%
\pgfpathclose%
\pgfusepath{stroke,fill}%
\end{pgfscope}%
\begin{pgfscope}%
\pgfpathrectangle{\pgfqpoint{0.600000in}{0.600000in}}{\pgfqpoint{3.900000in}{3.900000in}}%
\pgfusepath{clip}%
\pgfsetbuttcap%
\pgfsetroundjoin%
\definecolor{currentfill}{rgb}{0.121569,0.466667,0.705882}%
\pgfsetfillcolor{currentfill}%
\pgfsetlinewidth{1.003750pt}%
\definecolor{currentstroke}{rgb}{0.121569,0.466667,0.705882}%
\pgfsetstrokecolor{currentstroke}%
\pgfsetdash{}{0pt}%
\pgfpathmoveto{\pgfqpoint{1.336588in}{3.511151in}}%
\pgfpathcurveto{\pgfqpoint{1.347638in}{3.511151in}}{\pgfqpoint{1.358237in}{3.515541in}}{\pgfqpoint{1.366051in}{3.523355in}}%
\pgfpathcurveto{\pgfqpoint{1.373864in}{3.531168in}}{\pgfqpoint{1.378255in}{3.541767in}}{\pgfqpoint{1.378255in}{3.552817in}}%
\pgfpathcurveto{\pgfqpoint{1.378255in}{3.563867in}}{\pgfqpoint{1.373864in}{3.574467in}}{\pgfqpoint{1.366051in}{3.582280in}}%
\pgfpathcurveto{\pgfqpoint{1.358237in}{3.590094in}}{\pgfqpoint{1.347638in}{3.594484in}}{\pgfqpoint{1.336588in}{3.594484in}}%
\pgfpathcurveto{\pgfqpoint{1.325538in}{3.594484in}}{\pgfqpoint{1.314939in}{3.590094in}}{\pgfqpoint{1.307125in}{3.582280in}}%
\pgfpathcurveto{\pgfqpoint{1.299312in}{3.574467in}}{\pgfqpoint{1.294921in}{3.563867in}}{\pgfqpoint{1.294921in}{3.552817in}}%
\pgfpathcurveto{\pgfqpoint{1.294921in}{3.541767in}}{\pgfqpoint{1.299312in}{3.531168in}}{\pgfqpoint{1.307125in}{3.523355in}}%
\pgfpathcurveto{\pgfqpoint{1.314939in}{3.515541in}}{\pgfqpoint{1.325538in}{3.511151in}}{\pgfqpoint{1.336588in}{3.511151in}}%
\pgfpathclose%
\pgfusepath{stroke,fill}%
\end{pgfscope}%
\begin{pgfscope}%
\pgfpathrectangle{\pgfqpoint{0.600000in}{0.600000in}}{\pgfqpoint{3.900000in}{3.900000in}}%
\pgfusepath{clip}%
\pgfsetbuttcap%
\pgfsetroundjoin%
\definecolor{currentfill}{rgb}{0.121569,0.466667,0.705882}%
\pgfsetfillcolor{currentfill}%
\pgfsetlinewidth{1.003750pt}%
\definecolor{currentstroke}{rgb}{0.121569,0.466667,0.705882}%
\pgfsetstrokecolor{currentstroke}%
\pgfsetdash{}{0pt}%
\pgfpathmoveto{\pgfqpoint{2.815073in}{2.653351in}}%
\pgfpathcurveto{\pgfqpoint{2.826123in}{2.653351in}}{\pgfqpoint{2.836722in}{2.657742in}}{\pgfqpoint{2.844535in}{2.665555in}}%
\pgfpathcurveto{\pgfqpoint{2.852349in}{2.673369in}}{\pgfqpoint{2.856739in}{2.683968in}}{\pgfqpoint{2.856739in}{2.695018in}}%
\pgfpathcurveto{\pgfqpoint{2.856739in}{2.706068in}}{\pgfqpoint{2.852349in}{2.716667in}}{\pgfqpoint{2.844535in}{2.724481in}}%
\pgfpathcurveto{\pgfqpoint{2.836722in}{2.732294in}}{\pgfqpoint{2.826123in}{2.736685in}}{\pgfqpoint{2.815073in}{2.736685in}}%
\pgfpathcurveto{\pgfqpoint{2.804022in}{2.736685in}}{\pgfqpoint{2.793423in}{2.732294in}}{\pgfqpoint{2.785610in}{2.724481in}}%
\pgfpathcurveto{\pgfqpoint{2.777796in}{2.716667in}}{\pgfqpoint{2.773406in}{2.706068in}}{\pgfqpoint{2.773406in}{2.695018in}}%
\pgfpathcurveto{\pgfqpoint{2.773406in}{2.683968in}}{\pgfqpoint{2.777796in}{2.673369in}}{\pgfqpoint{2.785610in}{2.665555in}}%
\pgfpathcurveto{\pgfqpoint{2.793423in}{2.657742in}}{\pgfqpoint{2.804022in}{2.653351in}}{\pgfqpoint{2.815073in}{2.653351in}}%
\pgfpathclose%
\pgfusepath{stroke,fill}%
\end{pgfscope}%
\begin{pgfscope}%
\pgfpathrectangle{\pgfqpoint{0.600000in}{0.600000in}}{\pgfqpoint{3.900000in}{3.900000in}}%
\pgfusepath{clip}%
\pgfsetbuttcap%
\pgfsetroundjoin%
\definecolor{currentfill}{rgb}{0.121569,0.466667,0.705882}%
\pgfsetfillcolor{currentfill}%
\pgfsetlinewidth{1.003750pt}%
\definecolor{currentstroke}{rgb}{0.121569,0.466667,0.705882}%
\pgfsetstrokecolor{currentstroke}%
\pgfsetdash{}{0pt}%
\pgfpathmoveto{\pgfqpoint{2.644697in}{2.090244in}}%
\pgfpathcurveto{\pgfqpoint{2.655747in}{2.090244in}}{\pgfqpoint{2.666346in}{2.094634in}}{\pgfqpoint{2.674160in}{2.102448in}}%
\pgfpathcurveto{\pgfqpoint{2.681973in}{2.110262in}}{\pgfqpoint{2.686363in}{2.120861in}}{\pgfqpoint{2.686363in}{2.131911in}}%
\pgfpathcurveto{\pgfqpoint{2.686363in}{2.142961in}}{\pgfqpoint{2.681973in}{2.153560in}}{\pgfqpoint{2.674160in}{2.161374in}}%
\pgfpathcurveto{\pgfqpoint{2.666346in}{2.169187in}}{\pgfqpoint{2.655747in}{2.173577in}}{\pgfqpoint{2.644697in}{2.173577in}}%
\pgfpathcurveto{\pgfqpoint{2.633647in}{2.173577in}}{\pgfqpoint{2.623048in}{2.169187in}}{\pgfqpoint{2.615234in}{2.161374in}}%
\pgfpathcurveto{\pgfqpoint{2.607420in}{2.153560in}}{\pgfqpoint{2.603030in}{2.142961in}}{\pgfqpoint{2.603030in}{2.131911in}}%
\pgfpathcurveto{\pgfqpoint{2.603030in}{2.120861in}}{\pgfqpoint{2.607420in}{2.110262in}}{\pgfqpoint{2.615234in}{2.102448in}}%
\pgfpathcurveto{\pgfqpoint{2.623048in}{2.094634in}}{\pgfqpoint{2.633647in}{2.090244in}}{\pgfqpoint{2.644697in}{2.090244in}}%
\pgfpathclose%
\pgfusepath{stroke,fill}%
\end{pgfscope}%
\begin{pgfscope}%
\pgfpathrectangle{\pgfqpoint{0.600000in}{0.600000in}}{\pgfqpoint{3.900000in}{3.900000in}}%
\pgfusepath{clip}%
\pgfsetbuttcap%
\pgfsetroundjoin%
\definecolor{currentfill}{rgb}{0.121569,0.466667,0.705882}%
\pgfsetfillcolor{currentfill}%
\pgfsetlinewidth{1.003750pt}%
\definecolor{currentstroke}{rgb}{0.121569,0.466667,0.705882}%
\pgfsetstrokecolor{currentstroke}%
\pgfsetdash{}{0pt}%
\pgfpathmoveto{\pgfqpoint{2.366243in}{2.777021in}}%
\pgfpathcurveto{\pgfqpoint{2.377293in}{2.777021in}}{\pgfqpoint{2.387892in}{2.781411in}}{\pgfqpoint{2.395706in}{2.789225in}}%
\pgfpathcurveto{\pgfqpoint{2.403519in}{2.797039in}}{\pgfqpoint{2.407910in}{2.807638in}}{\pgfqpoint{2.407910in}{2.818688in}}%
\pgfpathcurveto{\pgfqpoint{2.407910in}{2.829738in}}{\pgfqpoint{2.403519in}{2.840337in}}{\pgfqpoint{2.395706in}{2.848150in}}%
\pgfpathcurveto{\pgfqpoint{2.387892in}{2.855964in}}{\pgfqpoint{2.377293in}{2.860354in}}{\pgfqpoint{2.366243in}{2.860354in}}%
\pgfpathcurveto{\pgfqpoint{2.355193in}{2.860354in}}{\pgfqpoint{2.344594in}{2.855964in}}{\pgfqpoint{2.336780in}{2.848150in}}%
\pgfpathcurveto{\pgfqpoint{2.328967in}{2.840337in}}{\pgfqpoint{2.324576in}{2.829738in}}{\pgfqpoint{2.324576in}{2.818688in}}%
\pgfpathcurveto{\pgfqpoint{2.324576in}{2.807638in}}{\pgfqpoint{2.328967in}{2.797039in}}{\pgfqpoint{2.336780in}{2.789225in}}%
\pgfpathcurveto{\pgfqpoint{2.344594in}{2.781411in}}{\pgfqpoint{2.355193in}{2.777021in}}{\pgfqpoint{2.366243in}{2.777021in}}%
\pgfpathclose%
\pgfusepath{stroke,fill}%
\end{pgfscope}%
\begin{pgfscope}%
\pgfpathrectangle{\pgfqpoint{0.600000in}{0.600000in}}{\pgfqpoint{3.900000in}{3.900000in}}%
\pgfusepath{clip}%
\pgfsetbuttcap%
\pgfsetroundjoin%
\definecolor{currentfill}{rgb}{0.121569,0.466667,0.705882}%
\pgfsetfillcolor{currentfill}%
\pgfsetlinewidth{1.003750pt}%
\definecolor{currentstroke}{rgb}{0.121569,0.466667,0.705882}%
\pgfsetstrokecolor{currentstroke}%
\pgfsetdash{}{0pt}%
\pgfpathmoveto{\pgfqpoint{2.603718in}{2.076709in}}%
\pgfpathcurveto{\pgfqpoint{2.614768in}{2.076709in}}{\pgfqpoint{2.625367in}{2.081099in}}{\pgfqpoint{2.633180in}{2.088913in}}%
\pgfpathcurveto{\pgfqpoint{2.640994in}{2.096727in}}{\pgfqpoint{2.645384in}{2.107326in}}{\pgfqpoint{2.645384in}{2.118376in}}%
\pgfpathcurveto{\pgfqpoint{2.645384in}{2.129426in}}{\pgfqpoint{2.640994in}{2.140025in}}{\pgfqpoint{2.633180in}{2.147839in}}%
\pgfpathcurveto{\pgfqpoint{2.625367in}{2.155652in}}{\pgfqpoint{2.614768in}{2.160042in}}{\pgfqpoint{2.603718in}{2.160042in}}%
\pgfpathcurveto{\pgfqpoint{2.592667in}{2.160042in}}{\pgfqpoint{2.582068in}{2.155652in}}{\pgfqpoint{2.574255in}{2.147839in}}%
\pgfpathcurveto{\pgfqpoint{2.566441in}{2.140025in}}{\pgfqpoint{2.562051in}{2.129426in}}{\pgfqpoint{2.562051in}{2.118376in}}%
\pgfpathcurveto{\pgfqpoint{2.562051in}{2.107326in}}{\pgfqpoint{2.566441in}{2.096727in}}{\pgfqpoint{2.574255in}{2.088913in}}%
\pgfpathcurveto{\pgfqpoint{2.582068in}{2.081099in}}{\pgfqpoint{2.592667in}{2.076709in}}{\pgfqpoint{2.603718in}{2.076709in}}%
\pgfpathclose%
\pgfusepath{stroke,fill}%
\end{pgfscope}%
\begin{pgfscope}%
\pgfpathrectangle{\pgfqpoint{0.600000in}{0.600000in}}{\pgfqpoint{3.900000in}{3.900000in}}%
\pgfusepath{clip}%
\pgfsetbuttcap%
\pgfsetroundjoin%
\definecolor{currentfill}{rgb}{0.121569,0.466667,0.705882}%
\pgfsetfillcolor{currentfill}%
\pgfsetlinewidth{1.003750pt}%
\definecolor{currentstroke}{rgb}{0.121569,0.466667,0.705882}%
\pgfsetstrokecolor{currentstroke}%
\pgfsetdash{}{0pt}%
\pgfpathmoveto{\pgfqpoint{2.779576in}{2.667071in}}%
\pgfpathcurveto{\pgfqpoint{2.790626in}{2.667071in}}{\pgfqpoint{2.801226in}{2.671461in}}{\pgfqpoint{2.809039in}{2.679275in}}%
\pgfpathcurveto{\pgfqpoint{2.816853in}{2.687089in}}{\pgfqpoint{2.821243in}{2.697688in}}{\pgfqpoint{2.821243in}{2.708738in}}%
\pgfpathcurveto{\pgfqpoint{2.821243in}{2.719788in}}{\pgfqpoint{2.816853in}{2.730387in}}{\pgfqpoint{2.809039in}{2.738201in}}%
\pgfpathcurveto{\pgfqpoint{2.801226in}{2.746014in}}{\pgfqpoint{2.790626in}{2.750405in}}{\pgfqpoint{2.779576in}{2.750405in}}%
\pgfpathcurveto{\pgfqpoint{2.768526in}{2.750405in}}{\pgfqpoint{2.757927in}{2.746014in}}{\pgfqpoint{2.750114in}{2.738201in}}%
\pgfpathcurveto{\pgfqpoint{2.742300in}{2.730387in}}{\pgfqpoint{2.737910in}{2.719788in}}{\pgfqpoint{2.737910in}{2.708738in}}%
\pgfpathcurveto{\pgfqpoint{2.737910in}{2.697688in}}{\pgfqpoint{2.742300in}{2.687089in}}{\pgfqpoint{2.750114in}{2.679275in}}%
\pgfpathcurveto{\pgfqpoint{2.757927in}{2.671461in}}{\pgfqpoint{2.768526in}{2.667071in}}{\pgfqpoint{2.779576in}{2.667071in}}%
\pgfpathclose%
\pgfusepath{stroke,fill}%
\end{pgfscope}%
\begin{pgfscope}%
\pgfpathrectangle{\pgfqpoint{0.600000in}{0.600000in}}{\pgfqpoint{3.900000in}{3.900000in}}%
\pgfusepath{clip}%
\pgfsetbuttcap%
\pgfsetroundjoin%
\definecolor{currentfill}{rgb}{0.121569,0.466667,0.705882}%
\pgfsetfillcolor{currentfill}%
\pgfsetlinewidth{1.003750pt}%
\definecolor{currentstroke}{rgb}{0.121569,0.466667,0.705882}%
\pgfsetstrokecolor{currentstroke}%
\pgfsetdash{}{0pt}%
\pgfpathmoveto{\pgfqpoint{2.721067in}{2.763368in}}%
\pgfpathcurveto{\pgfqpoint{2.732117in}{2.763368in}}{\pgfqpoint{2.742716in}{2.767758in}}{\pgfqpoint{2.750530in}{2.775572in}}%
\pgfpathcurveto{\pgfqpoint{2.758344in}{2.783385in}}{\pgfqpoint{2.762734in}{2.793984in}}{\pgfqpoint{2.762734in}{2.805035in}}%
\pgfpathcurveto{\pgfqpoint{2.762734in}{2.816085in}}{\pgfqpoint{2.758344in}{2.826684in}}{\pgfqpoint{2.750530in}{2.834497in}}%
\pgfpathcurveto{\pgfqpoint{2.742716in}{2.842311in}}{\pgfqpoint{2.732117in}{2.846701in}}{\pgfqpoint{2.721067in}{2.846701in}}%
\pgfpathcurveto{\pgfqpoint{2.710017in}{2.846701in}}{\pgfqpoint{2.699418in}{2.842311in}}{\pgfqpoint{2.691605in}{2.834497in}}%
\pgfpathcurveto{\pgfqpoint{2.683791in}{2.826684in}}{\pgfqpoint{2.679401in}{2.816085in}}{\pgfqpoint{2.679401in}{2.805035in}}%
\pgfpathcurveto{\pgfqpoint{2.679401in}{2.793984in}}{\pgfqpoint{2.683791in}{2.783385in}}{\pgfqpoint{2.691605in}{2.775572in}}%
\pgfpathcurveto{\pgfqpoint{2.699418in}{2.767758in}}{\pgfqpoint{2.710017in}{2.763368in}}{\pgfqpoint{2.721067in}{2.763368in}}%
\pgfpathclose%
\pgfusepath{stroke,fill}%
\end{pgfscope}%
\begin{pgfscope}%
\pgfpathrectangle{\pgfqpoint{0.600000in}{0.600000in}}{\pgfqpoint{3.900000in}{3.900000in}}%
\pgfusepath{clip}%
\pgfsetbuttcap%
\pgfsetroundjoin%
\definecolor{currentfill}{rgb}{0.121569,0.466667,0.705882}%
\pgfsetfillcolor{currentfill}%
\pgfsetlinewidth{1.003750pt}%
\definecolor{currentstroke}{rgb}{0.121569,0.466667,0.705882}%
\pgfsetstrokecolor{currentstroke}%
\pgfsetdash{}{0pt}%
\pgfpathmoveto{\pgfqpoint{3.510090in}{2.231243in}}%
\pgfpathcurveto{\pgfqpoint{3.521141in}{2.231243in}}{\pgfqpoint{3.531740in}{2.235633in}}{\pgfqpoint{3.539553in}{2.243447in}}%
\pgfpathcurveto{\pgfqpoint{3.547367in}{2.251260in}}{\pgfqpoint{3.551757in}{2.261859in}}{\pgfqpoint{3.551757in}{2.272909in}}%
\pgfpathcurveto{\pgfqpoint{3.551757in}{2.283960in}}{\pgfqpoint{3.547367in}{2.294559in}}{\pgfqpoint{3.539553in}{2.302372in}}%
\pgfpathcurveto{\pgfqpoint{3.531740in}{2.310186in}}{\pgfqpoint{3.521141in}{2.314576in}}{\pgfqpoint{3.510090in}{2.314576in}}%
\pgfpathcurveto{\pgfqpoint{3.499040in}{2.314576in}}{\pgfqpoint{3.488441in}{2.310186in}}{\pgfqpoint{3.480628in}{2.302372in}}%
\pgfpathcurveto{\pgfqpoint{3.472814in}{2.294559in}}{\pgfqpoint{3.468424in}{2.283960in}}{\pgfqpoint{3.468424in}{2.272909in}}%
\pgfpathcurveto{\pgfqpoint{3.468424in}{2.261859in}}{\pgfqpoint{3.472814in}{2.251260in}}{\pgfqpoint{3.480628in}{2.243447in}}%
\pgfpathcurveto{\pgfqpoint{3.488441in}{2.235633in}}{\pgfqpoint{3.499040in}{2.231243in}}{\pgfqpoint{3.510090in}{2.231243in}}%
\pgfpathclose%
\pgfusepath{stroke,fill}%
\end{pgfscope}%
\begin{pgfscope}%
\pgfpathrectangle{\pgfqpoint{0.600000in}{0.600000in}}{\pgfqpoint{3.900000in}{3.900000in}}%
\pgfusepath{clip}%
\pgfsetbuttcap%
\pgfsetroundjoin%
\definecolor{currentfill}{rgb}{0.121569,0.466667,0.705882}%
\pgfsetfillcolor{currentfill}%
\pgfsetlinewidth{1.003750pt}%
\definecolor{currentstroke}{rgb}{0.121569,0.466667,0.705882}%
\pgfsetstrokecolor{currentstroke}%
\pgfsetdash{}{0pt}%
\pgfpathmoveto{\pgfqpoint{2.859968in}{2.352795in}}%
\pgfpathcurveto{\pgfqpoint{2.871018in}{2.352795in}}{\pgfqpoint{2.881617in}{2.357186in}}{\pgfqpoint{2.889430in}{2.364999in}}%
\pgfpathcurveto{\pgfqpoint{2.897244in}{2.372813in}}{\pgfqpoint{2.901634in}{2.383412in}}{\pgfqpoint{2.901634in}{2.394462in}}%
\pgfpathcurveto{\pgfqpoint{2.901634in}{2.405512in}}{\pgfqpoint{2.897244in}{2.416111in}}{\pgfqpoint{2.889430in}{2.423925in}}%
\pgfpathcurveto{\pgfqpoint{2.881617in}{2.431738in}}{\pgfqpoint{2.871018in}{2.436129in}}{\pgfqpoint{2.859968in}{2.436129in}}%
\pgfpathcurveto{\pgfqpoint{2.848917in}{2.436129in}}{\pgfqpoint{2.838318in}{2.431738in}}{\pgfqpoint{2.830505in}{2.423925in}}%
\pgfpathcurveto{\pgfqpoint{2.822691in}{2.416111in}}{\pgfqpoint{2.818301in}{2.405512in}}{\pgfqpoint{2.818301in}{2.394462in}}%
\pgfpathcurveto{\pgfqpoint{2.818301in}{2.383412in}}{\pgfqpoint{2.822691in}{2.372813in}}{\pgfqpoint{2.830505in}{2.364999in}}%
\pgfpathcurveto{\pgfqpoint{2.838318in}{2.357186in}}{\pgfqpoint{2.848917in}{2.352795in}}{\pgfqpoint{2.859968in}{2.352795in}}%
\pgfpathclose%
\pgfusepath{stroke,fill}%
\end{pgfscope}%
\begin{pgfscope}%
\pgfpathrectangle{\pgfqpoint{0.600000in}{0.600000in}}{\pgfqpoint{3.900000in}{3.900000in}}%
\pgfusepath{clip}%
\pgfsetbuttcap%
\pgfsetroundjoin%
\definecolor{currentfill}{rgb}{0.121569,0.466667,0.705882}%
\pgfsetfillcolor{currentfill}%
\pgfsetlinewidth{1.003750pt}%
\definecolor{currentstroke}{rgb}{0.121569,0.466667,0.705882}%
\pgfsetstrokecolor{currentstroke}%
\pgfsetdash{}{0pt}%
\pgfpathmoveto{\pgfqpoint{3.035476in}{1.928331in}}%
\pgfpathcurveto{\pgfqpoint{3.046526in}{1.928331in}}{\pgfqpoint{3.057125in}{1.932722in}}{\pgfqpoint{3.064938in}{1.940535in}}%
\pgfpathcurveto{\pgfqpoint{3.072752in}{1.948349in}}{\pgfqpoint{3.077142in}{1.958948in}}{\pgfqpoint{3.077142in}{1.969998in}}%
\pgfpathcurveto{\pgfqpoint{3.077142in}{1.981048in}}{\pgfqpoint{3.072752in}{1.991647in}}{\pgfqpoint{3.064938in}{1.999461in}}%
\pgfpathcurveto{\pgfqpoint{3.057125in}{2.007274in}}{\pgfqpoint{3.046526in}{2.011665in}}{\pgfqpoint{3.035476in}{2.011665in}}%
\pgfpathcurveto{\pgfqpoint{3.024426in}{2.011665in}}{\pgfqpoint{3.013827in}{2.007274in}}{\pgfqpoint{3.006013in}{1.999461in}}%
\pgfpathcurveto{\pgfqpoint{2.998199in}{1.991647in}}{\pgfqpoint{2.993809in}{1.981048in}}{\pgfqpoint{2.993809in}{1.969998in}}%
\pgfpathcurveto{\pgfqpoint{2.993809in}{1.958948in}}{\pgfqpoint{2.998199in}{1.948349in}}{\pgfqpoint{3.006013in}{1.940535in}}%
\pgfpathcurveto{\pgfqpoint{3.013827in}{1.932722in}}{\pgfqpoint{3.024426in}{1.928331in}}{\pgfqpoint{3.035476in}{1.928331in}}%
\pgfpathclose%
\pgfusepath{stroke,fill}%
\end{pgfscope}%
\begin{pgfscope}%
\pgfpathrectangle{\pgfqpoint{0.600000in}{0.600000in}}{\pgfqpoint{3.900000in}{3.900000in}}%
\pgfusepath{clip}%
\pgfsetbuttcap%
\pgfsetroundjoin%
\definecolor{currentfill}{rgb}{0.121569,0.466667,0.705882}%
\pgfsetfillcolor{currentfill}%
\pgfsetlinewidth{1.003750pt}%
\definecolor{currentstroke}{rgb}{0.121569,0.466667,0.705882}%
\pgfsetstrokecolor{currentstroke}%
\pgfsetdash{}{0pt}%
\pgfpathmoveto{\pgfqpoint{2.472750in}{2.072975in}}%
\pgfpathcurveto{\pgfqpoint{2.483800in}{2.072975in}}{\pgfqpoint{2.494399in}{2.077365in}}{\pgfqpoint{2.502213in}{2.085179in}}%
\pgfpathcurveto{\pgfqpoint{2.510026in}{2.092992in}}{\pgfqpoint{2.514416in}{2.103591in}}{\pgfqpoint{2.514416in}{2.114641in}}%
\pgfpathcurveto{\pgfqpoint{2.514416in}{2.125692in}}{\pgfqpoint{2.510026in}{2.136291in}}{\pgfqpoint{2.502213in}{2.144104in}}%
\pgfpathcurveto{\pgfqpoint{2.494399in}{2.151918in}}{\pgfqpoint{2.483800in}{2.156308in}}{\pgfqpoint{2.472750in}{2.156308in}}%
\pgfpathcurveto{\pgfqpoint{2.461700in}{2.156308in}}{\pgfqpoint{2.451101in}{2.151918in}}{\pgfqpoint{2.443287in}{2.144104in}}%
\pgfpathcurveto{\pgfqpoint{2.435473in}{2.136291in}}{\pgfqpoint{2.431083in}{2.125692in}}{\pgfqpoint{2.431083in}{2.114641in}}%
\pgfpathcurveto{\pgfqpoint{2.431083in}{2.103591in}}{\pgfqpoint{2.435473in}{2.092992in}}{\pgfqpoint{2.443287in}{2.085179in}}%
\pgfpathcurveto{\pgfqpoint{2.451101in}{2.077365in}}{\pgfqpoint{2.461700in}{2.072975in}}{\pgfqpoint{2.472750in}{2.072975in}}%
\pgfpathclose%
\pgfusepath{stroke,fill}%
\end{pgfscope}%
\begin{pgfscope}%
\pgfpathrectangle{\pgfqpoint{0.600000in}{0.600000in}}{\pgfqpoint{3.900000in}{3.900000in}}%
\pgfusepath{clip}%
\pgfsetbuttcap%
\pgfsetroundjoin%
\definecolor{currentfill}{rgb}{0.121569,0.466667,0.705882}%
\pgfsetfillcolor{currentfill}%
\pgfsetlinewidth{1.003750pt}%
\definecolor{currentstroke}{rgb}{0.121569,0.466667,0.705882}%
\pgfsetstrokecolor{currentstroke}%
\pgfsetdash{}{0pt}%
\pgfpathmoveto{\pgfqpoint{2.604297in}{2.662337in}}%
\pgfpathcurveto{\pgfqpoint{2.615347in}{2.662337in}}{\pgfqpoint{2.625946in}{2.666728in}}{\pgfqpoint{2.633760in}{2.674541in}}%
\pgfpathcurveto{\pgfqpoint{2.641574in}{2.682355in}}{\pgfqpoint{2.645964in}{2.692954in}}{\pgfqpoint{2.645964in}{2.704004in}}%
\pgfpathcurveto{\pgfqpoint{2.645964in}{2.715054in}}{\pgfqpoint{2.641574in}{2.725653in}}{\pgfqpoint{2.633760in}{2.733467in}}%
\pgfpathcurveto{\pgfqpoint{2.625946in}{2.741281in}}{\pgfqpoint{2.615347in}{2.745671in}}{\pgfqpoint{2.604297in}{2.745671in}}%
\pgfpathcurveto{\pgfqpoint{2.593247in}{2.745671in}}{\pgfqpoint{2.582648in}{2.741281in}}{\pgfqpoint{2.574834in}{2.733467in}}%
\pgfpathcurveto{\pgfqpoint{2.567021in}{2.725653in}}{\pgfqpoint{2.562630in}{2.715054in}}{\pgfqpoint{2.562630in}{2.704004in}}%
\pgfpathcurveto{\pgfqpoint{2.562630in}{2.692954in}}{\pgfqpoint{2.567021in}{2.682355in}}{\pgfqpoint{2.574834in}{2.674541in}}%
\pgfpathcurveto{\pgfqpoint{2.582648in}{2.666728in}}{\pgfqpoint{2.593247in}{2.662337in}}{\pgfqpoint{2.604297in}{2.662337in}}%
\pgfpathclose%
\pgfusepath{stroke,fill}%
\end{pgfscope}%
\begin{pgfscope}%
\pgfpathrectangle{\pgfqpoint{0.600000in}{0.600000in}}{\pgfqpoint{3.900000in}{3.900000in}}%
\pgfusepath{clip}%
\pgfsetbuttcap%
\pgfsetroundjoin%
\definecolor{currentfill}{rgb}{0.121569,0.466667,0.705882}%
\pgfsetfillcolor{currentfill}%
\pgfsetlinewidth{1.003750pt}%
\definecolor{currentstroke}{rgb}{0.121569,0.466667,0.705882}%
\pgfsetstrokecolor{currentstroke}%
\pgfsetdash{}{0pt}%
\pgfpathmoveto{\pgfqpoint{2.357841in}{2.738627in}}%
\pgfpathcurveto{\pgfqpoint{2.368891in}{2.738627in}}{\pgfqpoint{2.379490in}{2.743017in}}{\pgfqpoint{2.387304in}{2.750831in}}%
\pgfpathcurveto{\pgfqpoint{2.395118in}{2.758644in}}{\pgfqpoint{2.399508in}{2.769243in}}{\pgfqpoint{2.399508in}{2.780294in}}%
\pgfpathcurveto{\pgfqpoint{2.399508in}{2.791344in}}{\pgfqpoint{2.395118in}{2.801943in}}{\pgfqpoint{2.387304in}{2.809756in}}%
\pgfpathcurveto{\pgfqpoint{2.379490in}{2.817570in}}{\pgfqpoint{2.368891in}{2.821960in}}{\pgfqpoint{2.357841in}{2.821960in}}%
\pgfpathcurveto{\pgfqpoint{2.346791in}{2.821960in}}{\pgfqpoint{2.336192in}{2.817570in}}{\pgfqpoint{2.328378in}{2.809756in}}%
\pgfpathcurveto{\pgfqpoint{2.320565in}{2.801943in}}{\pgfqpoint{2.316175in}{2.791344in}}{\pgfqpoint{2.316175in}{2.780294in}}%
\pgfpathcurveto{\pgfqpoint{2.316175in}{2.769243in}}{\pgfqpoint{2.320565in}{2.758644in}}{\pgfqpoint{2.328378in}{2.750831in}}%
\pgfpathcurveto{\pgfqpoint{2.336192in}{2.743017in}}{\pgfqpoint{2.346791in}{2.738627in}}{\pgfqpoint{2.357841in}{2.738627in}}%
\pgfpathclose%
\pgfusepath{stroke,fill}%
\end{pgfscope}%
\begin{pgfscope}%
\pgfpathrectangle{\pgfqpoint{0.600000in}{0.600000in}}{\pgfqpoint{3.900000in}{3.900000in}}%
\pgfusepath{clip}%
\pgfsetbuttcap%
\pgfsetroundjoin%
\definecolor{currentfill}{rgb}{0.121569,0.466667,0.705882}%
\pgfsetfillcolor{currentfill}%
\pgfsetlinewidth{1.003750pt}%
\definecolor{currentstroke}{rgb}{0.121569,0.466667,0.705882}%
\pgfsetstrokecolor{currentstroke}%
\pgfsetdash{}{0pt}%
\pgfpathmoveto{\pgfqpoint{2.867698in}{3.085266in}}%
\pgfpathcurveto{\pgfqpoint{2.878748in}{3.085266in}}{\pgfqpoint{2.889347in}{3.089656in}}{\pgfqpoint{2.897161in}{3.097469in}}%
\pgfpathcurveto{\pgfqpoint{2.904975in}{3.105283in}}{\pgfqpoint{2.909365in}{3.115882in}}{\pgfqpoint{2.909365in}{3.126932in}}%
\pgfpathcurveto{\pgfqpoint{2.909365in}{3.137982in}}{\pgfqpoint{2.904975in}{3.148581in}}{\pgfqpoint{2.897161in}{3.156395in}}%
\pgfpathcurveto{\pgfqpoint{2.889347in}{3.164209in}}{\pgfqpoint{2.878748in}{3.168599in}}{\pgfqpoint{2.867698in}{3.168599in}}%
\pgfpathcurveto{\pgfqpoint{2.856648in}{3.168599in}}{\pgfqpoint{2.846049in}{3.164209in}}{\pgfqpoint{2.838235in}{3.156395in}}%
\pgfpathcurveto{\pgfqpoint{2.830422in}{3.148581in}}{\pgfqpoint{2.826031in}{3.137982in}}{\pgfqpoint{2.826031in}{3.126932in}}%
\pgfpathcurveto{\pgfqpoint{2.826031in}{3.115882in}}{\pgfqpoint{2.830422in}{3.105283in}}{\pgfqpoint{2.838235in}{3.097469in}}%
\pgfpathcurveto{\pgfqpoint{2.846049in}{3.089656in}}{\pgfqpoint{2.856648in}{3.085266in}}{\pgfqpoint{2.867698in}{3.085266in}}%
\pgfpathclose%
\pgfusepath{stroke,fill}%
\end{pgfscope}%
\begin{pgfscope}%
\pgfpathrectangle{\pgfqpoint{0.600000in}{0.600000in}}{\pgfqpoint{3.900000in}{3.900000in}}%
\pgfusepath{clip}%
\pgfsetbuttcap%
\pgfsetroundjoin%
\definecolor{currentfill}{rgb}{0.121569,0.466667,0.705882}%
\pgfsetfillcolor{currentfill}%
\pgfsetlinewidth{1.003750pt}%
\definecolor{currentstroke}{rgb}{0.121569,0.466667,0.705882}%
\pgfsetstrokecolor{currentstroke}%
\pgfsetdash{}{0pt}%
\pgfpathmoveto{\pgfqpoint{2.928439in}{2.125400in}}%
\pgfpathcurveto{\pgfqpoint{2.939489in}{2.125400in}}{\pgfqpoint{2.950088in}{2.129790in}}{\pgfqpoint{2.957902in}{2.137604in}}%
\pgfpathcurveto{\pgfqpoint{2.965716in}{2.145417in}}{\pgfqpoint{2.970106in}{2.156016in}}{\pgfqpoint{2.970106in}{2.167066in}}%
\pgfpathcurveto{\pgfqpoint{2.970106in}{2.178117in}}{\pgfqpoint{2.965716in}{2.188716in}}{\pgfqpoint{2.957902in}{2.196529in}}%
\pgfpathcurveto{\pgfqpoint{2.950088in}{2.204343in}}{\pgfqpoint{2.939489in}{2.208733in}}{\pgfqpoint{2.928439in}{2.208733in}}%
\pgfpathcurveto{\pgfqpoint{2.917389in}{2.208733in}}{\pgfqpoint{2.906790in}{2.204343in}}{\pgfqpoint{2.898976in}{2.196529in}}%
\pgfpathcurveto{\pgfqpoint{2.891163in}{2.188716in}}{\pgfqpoint{2.886773in}{2.178117in}}{\pgfqpoint{2.886773in}{2.167066in}}%
\pgfpathcurveto{\pgfqpoint{2.886773in}{2.156016in}}{\pgfqpoint{2.891163in}{2.145417in}}{\pgfqpoint{2.898976in}{2.137604in}}%
\pgfpathcurveto{\pgfqpoint{2.906790in}{2.129790in}}{\pgfqpoint{2.917389in}{2.125400in}}{\pgfqpoint{2.928439in}{2.125400in}}%
\pgfpathclose%
\pgfusepath{stroke,fill}%
\end{pgfscope}%
\begin{pgfscope}%
\pgfpathrectangle{\pgfqpoint{0.600000in}{0.600000in}}{\pgfqpoint{3.900000in}{3.900000in}}%
\pgfusepath{clip}%
\pgfsetbuttcap%
\pgfsetroundjoin%
\definecolor{currentfill}{rgb}{0.121569,0.466667,0.705882}%
\pgfsetfillcolor{currentfill}%
\pgfsetlinewidth{1.003750pt}%
\definecolor{currentstroke}{rgb}{0.121569,0.466667,0.705882}%
\pgfsetstrokecolor{currentstroke}%
\pgfsetdash{}{0pt}%
\pgfpathmoveto{\pgfqpoint{2.519311in}{2.528132in}}%
\pgfpathcurveto{\pgfqpoint{2.530361in}{2.528132in}}{\pgfqpoint{2.540960in}{2.532522in}}{\pgfqpoint{2.548774in}{2.540336in}}%
\pgfpathcurveto{\pgfqpoint{2.556588in}{2.548149in}}{\pgfqpoint{2.560978in}{2.558748in}}{\pgfqpoint{2.560978in}{2.569798in}}%
\pgfpathcurveto{\pgfqpoint{2.560978in}{2.580849in}}{\pgfqpoint{2.556588in}{2.591448in}}{\pgfqpoint{2.548774in}{2.599261in}}%
\pgfpathcurveto{\pgfqpoint{2.540960in}{2.607075in}}{\pgfqpoint{2.530361in}{2.611465in}}{\pgfqpoint{2.519311in}{2.611465in}}%
\pgfpathcurveto{\pgfqpoint{2.508261in}{2.611465in}}{\pgfqpoint{2.497662in}{2.607075in}}{\pgfqpoint{2.489849in}{2.599261in}}%
\pgfpathcurveto{\pgfqpoint{2.482035in}{2.591448in}}{\pgfqpoint{2.477645in}{2.580849in}}{\pgfqpoint{2.477645in}{2.569798in}}%
\pgfpathcurveto{\pgfqpoint{2.477645in}{2.558748in}}{\pgfqpoint{2.482035in}{2.548149in}}{\pgfqpoint{2.489849in}{2.540336in}}%
\pgfpathcurveto{\pgfqpoint{2.497662in}{2.532522in}}{\pgfqpoint{2.508261in}{2.528132in}}{\pgfqpoint{2.519311in}{2.528132in}}%
\pgfpathclose%
\pgfusepath{stroke,fill}%
\end{pgfscope}%
\begin{pgfscope}%
\pgfpathrectangle{\pgfqpoint{0.600000in}{0.600000in}}{\pgfqpoint{3.900000in}{3.900000in}}%
\pgfusepath{clip}%
\pgfsetbuttcap%
\pgfsetroundjoin%
\definecolor{currentfill}{rgb}{0.121569,0.466667,0.705882}%
\pgfsetfillcolor{currentfill}%
\pgfsetlinewidth{1.003750pt}%
\definecolor{currentstroke}{rgb}{0.121569,0.466667,0.705882}%
\pgfsetstrokecolor{currentstroke}%
\pgfsetdash{}{0pt}%
\pgfpathmoveto{\pgfqpoint{2.472146in}{2.154469in}}%
\pgfpathcurveto{\pgfqpoint{2.483196in}{2.154469in}}{\pgfqpoint{2.493795in}{2.158859in}}{\pgfqpoint{2.501609in}{2.166673in}}%
\pgfpathcurveto{\pgfqpoint{2.509423in}{2.174487in}}{\pgfqpoint{2.513813in}{2.185086in}}{\pgfqpoint{2.513813in}{2.196136in}}%
\pgfpathcurveto{\pgfqpoint{2.513813in}{2.207186in}}{\pgfqpoint{2.509423in}{2.217785in}}{\pgfqpoint{2.501609in}{2.225599in}}%
\pgfpathcurveto{\pgfqpoint{2.493795in}{2.233412in}}{\pgfqpoint{2.483196in}{2.237802in}}{\pgfqpoint{2.472146in}{2.237802in}}%
\pgfpathcurveto{\pgfqpoint{2.461096in}{2.237802in}}{\pgfqpoint{2.450497in}{2.233412in}}{\pgfqpoint{2.442683in}{2.225599in}}%
\pgfpathcurveto{\pgfqpoint{2.434870in}{2.217785in}}{\pgfqpoint{2.430480in}{2.207186in}}{\pgfqpoint{2.430480in}{2.196136in}}%
\pgfpathcurveto{\pgfqpoint{2.430480in}{2.185086in}}{\pgfqpoint{2.434870in}{2.174487in}}{\pgfqpoint{2.442683in}{2.166673in}}%
\pgfpathcurveto{\pgfqpoint{2.450497in}{2.158859in}}{\pgfqpoint{2.461096in}{2.154469in}}{\pgfqpoint{2.472146in}{2.154469in}}%
\pgfpathclose%
\pgfusepath{stroke,fill}%
\end{pgfscope}%
\begin{pgfscope}%
\pgfpathrectangle{\pgfqpoint{0.600000in}{0.600000in}}{\pgfqpoint{3.900000in}{3.900000in}}%
\pgfusepath{clip}%
\pgfsetbuttcap%
\pgfsetroundjoin%
\definecolor{currentfill}{rgb}{0.121569,0.466667,0.705882}%
\pgfsetfillcolor{currentfill}%
\pgfsetlinewidth{1.003750pt}%
\definecolor{currentstroke}{rgb}{0.121569,0.466667,0.705882}%
\pgfsetstrokecolor{currentstroke}%
\pgfsetdash{}{0pt}%
\pgfpathmoveto{\pgfqpoint{2.728341in}{2.900063in}}%
\pgfpathcurveto{\pgfqpoint{2.739391in}{2.900063in}}{\pgfqpoint{2.749990in}{2.904453in}}{\pgfqpoint{2.757803in}{2.912267in}}%
\pgfpathcurveto{\pgfqpoint{2.765617in}{2.920080in}}{\pgfqpoint{2.770007in}{2.930679in}}{\pgfqpoint{2.770007in}{2.941730in}}%
\pgfpathcurveto{\pgfqpoint{2.770007in}{2.952780in}}{\pgfqpoint{2.765617in}{2.963379in}}{\pgfqpoint{2.757803in}{2.971192in}}%
\pgfpathcurveto{\pgfqpoint{2.749990in}{2.979006in}}{\pgfqpoint{2.739391in}{2.983396in}}{\pgfqpoint{2.728341in}{2.983396in}}%
\pgfpathcurveto{\pgfqpoint{2.717290in}{2.983396in}}{\pgfqpoint{2.706691in}{2.979006in}}{\pgfqpoint{2.698878in}{2.971192in}}%
\pgfpathcurveto{\pgfqpoint{2.691064in}{2.963379in}}{\pgfqpoint{2.686674in}{2.952780in}}{\pgfqpoint{2.686674in}{2.941730in}}%
\pgfpathcurveto{\pgfqpoint{2.686674in}{2.930679in}}{\pgfqpoint{2.691064in}{2.920080in}}{\pgfqpoint{2.698878in}{2.912267in}}%
\pgfpathcurveto{\pgfqpoint{2.706691in}{2.904453in}}{\pgfqpoint{2.717290in}{2.900063in}}{\pgfqpoint{2.728341in}{2.900063in}}%
\pgfpathclose%
\pgfusepath{stroke,fill}%
\end{pgfscope}%
\begin{pgfscope}%
\pgfpathrectangle{\pgfqpoint{0.600000in}{0.600000in}}{\pgfqpoint{3.900000in}{3.900000in}}%
\pgfusepath{clip}%
\pgfsetbuttcap%
\pgfsetroundjoin%
\definecolor{currentfill}{rgb}{0.121569,0.466667,0.705882}%
\pgfsetfillcolor{currentfill}%
\pgfsetlinewidth{1.003750pt}%
\definecolor{currentstroke}{rgb}{0.121569,0.466667,0.705882}%
\pgfsetstrokecolor{currentstroke}%
\pgfsetdash{}{0pt}%
\pgfpathmoveto{\pgfqpoint{2.729908in}{3.015695in}}%
\pgfpathcurveto{\pgfqpoint{2.740958in}{3.015695in}}{\pgfqpoint{2.751557in}{3.020085in}}{\pgfqpoint{2.759371in}{3.027899in}}%
\pgfpathcurveto{\pgfqpoint{2.767185in}{3.035713in}}{\pgfqpoint{2.771575in}{3.046312in}}{\pgfqpoint{2.771575in}{3.057362in}}%
\pgfpathcurveto{\pgfqpoint{2.771575in}{3.068412in}}{\pgfqpoint{2.767185in}{3.079011in}}{\pgfqpoint{2.759371in}{3.086825in}}%
\pgfpathcurveto{\pgfqpoint{2.751557in}{3.094638in}}{\pgfqpoint{2.740958in}{3.099029in}}{\pgfqpoint{2.729908in}{3.099029in}}%
\pgfpathcurveto{\pgfqpoint{2.718858in}{3.099029in}}{\pgfqpoint{2.708259in}{3.094638in}}{\pgfqpoint{2.700446in}{3.086825in}}%
\pgfpathcurveto{\pgfqpoint{2.692632in}{3.079011in}}{\pgfqpoint{2.688242in}{3.068412in}}{\pgfqpoint{2.688242in}{3.057362in}}%
\pgfpathcurveto{\pgfqpoint{2.688242in}{3.046312in}}{\pgfqpoint{2.692632in}{3.035713in}}{\pgfqpoint{2.700446in}{3.027899in}}%
\pgfpathcurveto{\pgfqpoint{2.708259in}{3.020085in}}{\pgfqpoint{2.718858in}{3.015695in}}{\pgfqpoint{2.729908in}{3.015695in}}%
\pgfpathclose%
\pgfusepath{stroke,fill}%
\end{pgfscope}%
\begin{pgfscope}%
\pgfpathrectangle{\pgfqpoint{0.600000in}{0.600000in}}{\pgfqpoint{3.900000in}{3.900000in}}%
\pgfusepath{clip}%
\pgfsetbuttcap%
\pgfsetroundjoin%
\definecolor{currentfill}{rgb}{0.121569,0.466667,0.705882}%
\pgfsetfillcolor{currentfill}%
\pgfsetlinewidth{1.003750pt}%
\definecolor{currentstroke}{rgb}{0.121569,0.466667,0.705882}%
\pgfsetstrokecolor{currentstroke}%
\pgfsetdash{}{0pt}%
\pgfpathmoveto{\pgfqpoint{2.176826in}{1.863775in}}%
\pgfpathcurveto{\pgfqpoint{2.187876in}{1.863775in}}{\pgfqpoint{2.198475in}{1.868166in}}{\pgfqpoint{2.206289in}{1.875979in}}%
\pgfpathcurveto{\pgfqpoint{2.214102in}{1.883793in}}{\pgfqpoint{2.218492in}{1.894392in}}{\pgfqpoint{2.218492in}{1.905442in}}%
\pgfpathcurveto{\pgfqpoint{2.218492in}{1.916492in}}{\pgfqpoint{2.214102in}{1.927091in}}{\pgfqpoint{2.206289in}{1.934905in}}%
\pgfpathcurveto{\pgfqpoint{2.198475in}{1.942718in}}{\pgfqpoint{2.187876in}{1.947109in}}{\pgfqpoint{2.176826in}{1.947109in}}%
\pgfpathcurveto{\pgfqpoint{2.165776in}{1.947109in}}{\pgfqpoint{2.155177in}{1.942718in}}{\pgfqpoint{2.147363in}{1.934905in}}%
\pgfpathcurveto{\pgfqpoint{2.139549in}{1.927091in}}{\pgfqpoint{2.135159in}{1.916492in}}{\pgfqpoint{2.135159in}{1.905442in}}%
\pgfpathcurveto{\pgfqpoint{2.135159in}{1.894392in}}{\pgfqpoint{2.139549in}{1.883793in}}{\pgfqpoint{2.147363in}{1.875979in}}%
\pgfpathcurveto{\pgfqpoint{2.155177in}{1.868166in}}{\pgfqpoint{2.165776in}{1.863775in}}{\pgfqpoint{2.176826in}{1.863775in}}%
\pgfpathclose%
\pgfusepath{stroke,fill}%
\end{pgfscope}%
\begin{pgfscope}%
\pgfpathrectangle{\pgfqpoint{0.600000in}{0.600000in}}{\pgfqpoint{3.900000in}{3.900000in}}%
\pgfusepath{clip}%
\pgfsetbuttcap%
\pgfsetroundjoin%
\definecolor{currentfill}{rgb}{0.121569,0.466667,0.705882}%
\pgfsetfillcolor{currentfill}%
\pgfsetlinewidth{1.003750pt}%
\definecolor{currentstroke}{rgb}{0.121569,0.466667,0.705882}%
\pgfsetstrokecolor{currentstroke}%
\pgfsetdash{}{0pt}%
\pgfpathmoveto{\pgfqpoint{2.967461in}{2.160641in}}%
\pgfpathcurveto{\pgfqpoint{2.978511in}{2.160641in}}{\pgfqpoint{2.989110in}{2.165031in}}{\pgfqpoint{2.996924in}{2.172845in}}%
\pgfpathcurveto{\pgfqpoint{3.004737in}{2.180658in}}{\pgfqpoint{3.009128in}{2.191257in}}{\pgfqpoint{3.009128in}{2.202308in}}%
\pgfpathcurveto{\pgfqpoint{3.009128in}{2.213358in}}{\pgfqpoint{3.004737in}{2.223957in}}{\pgfqpoint{2.996924in}{2.231770in}}%
\pgfpathcurveto{\pgfqpoint{2.989110in}{2.239584in}}{\pgfqpoint{2.978511in}{2.243974in}}{\pgfqpoint{2.967461in}{2.243974in}}%
\pgfpathcurveto{\pgfqpoint{2.956411in}{2.243974in}}{\pgfqpoint{2.945812in}{2.239584in}}{\pgfqpoint{2.937998in}{2.231770in}}%
\pgfpathcurveto{\pgfqpoint{2.930185in}{2.223957in}}{\pgfqpoint{2.925794in}{2.213358in}}{\pgfqpoint{2.925794in}{2.202308in}}%
\pgfpathcurveto{\pgfqpoint{2.925794in}{2.191257in}}{\pgfqpoint{2.930185in}{2.180658in}}{\pgfqpoint{2.937998in}{2.172845in}}%
\pgfpathcurveto{\pgfqpoint{2.945812in}{2.165031in}}{\pgfqpoint{2.956411in}{2.160641in}}{\pgfqpoint{2.967461in}{2.160641in}}%
\pgfpathclose%
\pgfusepath{stroke,fill}%
\end{pgfscope}%
\begin{pgfscope}%
\pgfpathrectangle{\pgfqpoint{0.600000in}{0.600000in}}{\pgfqpoint{3.900000in}{3.900000in}}%
\pgfusepath{clip}%
\pgfsetbuttcap%
\pgfsetroundjoin%
\definecolor{currentfill}{rgb}{0.121569,0.466667,0.705882}%
\pgfsetfillcolor{currentfill}%
\pgfsetlinewidth{1.003750pt}%
\definecolor{currentstroke}{rgb}{0.121569,0.466667,0.705882}%
\pgfsetstrokecolor{currentstroke}%
\pgfsetdash{}{0pt}%
\pgfpathmoveto{\pgfqpoint{1.916781in}{2.013683in}}%
\pgfpathcurveto{\pgfqpoint{1.927831in}{2.013683in}}{\pgfqpoint{1.938430in}{2.018073in}}{\pgfqpoint{1.946244in}{2.025886in}}%
\pgfpathcurveto{\pgfqpoint{1.954057in}{2.033700in}}{\pgfqpoint{1.958448in}{2.044299in}}{\pgfqpoint{1.958448in}{2.055349in}}%
\pgfpathcurveto{\pgfqpoint{1.958448in}{2.066399in}}{\pgfqpoint{1.954057in}{2.076998in}}{\pgfqpoint{1.946244in}{2.084812in}}%
\pgfpathcurveto{\pgfqpoint{1.938430in}{2.092626in}}{\pgfqpoint{1.927831in}{2.097016in}}{\pgfqpoint{1.916781in}{2.097016in}}%
\pgfpathcurveto{\pgfqpoint{1.905731in}{2.097016in}}{\pgfqpoint{1.895132in}{2.092626in}}{\pgfqpoint{1.887318in}{2.084812in}}%
\pgfpathcurveto{\pgfqpoint{1.879505in}{2.076998in}}{\pgfqpoint{1.875114in}{2.066399in}}{\pgfqpoint{1.875114in}{2.055349in}}%
\pgfpathcurveto{\pgfqpoint{1.875114in}{2.044299in}}{\pgfqpoint{1.879505in}{2.033700in}}{\pgfqpoint{1.887318in}{2.025886in}}%
\pgfpathcurveto{\pgfqpoint{1.895132in}{2.018073in}}{\pgfqpoint{1.905731in}{2.013683in}}{\pgfqpoint{1.916781in}{2.013683in}}%
\pgfpathclose%
\pgfusepath{stroke,fill}%
\end{pgfscope}%
\begin{pgfscope}%
\pgfpathrectangle{\pgfqpoint{0.600000in}{0.600000in}}{\pgfqpoint{3.900000in}{3.900000in}}%
\pgfusepath{clip}%
\pgfsetbuttcap%
\pgfsetroundjoin%
\definecolor{currentfill}{rgb}{0.121569,0.466667,0.705882}%
\pgfsetfillcolor{currentfill}%
\pgfsetlinewidth{1.003750pt}%
\definecolor{currentstroke}{rgb}{0.121569,0.466667,0.705882}%
\pgfsetstrokecolor{currentstroke}%
\pgfsetdash{}{0pt}%
\pgfpathmoveto{\pgfqpoint{2.025558in}{1.873871in}}%
\pgfpathcurveto{\pgfqpoint{2.036608in}{1.873871in}}{\pgfqpoint{2.047207in}{1.878262in}}{\pgfqpoint{2.055021in}{1.886075in}}%
\pgfpathcurveto{\pgfqpoint{2.062834in}{1.893889in}}{\pgfqpoint{2.067224in}{1.904488in}}{\pgfqpoint{2.067224in}{1.915538in}}%
\pgfpathcurveto{\pgfqpoint{2.067224in}{1.926588in}}{\pgfqpoint{2.062834in}{1.937187in}}{\pgfqpoint{2.055021in}{1.945001in}}%
\pgfpathcurveto{\pgfqpoint{2.047207in}{1.952814in}}{\pgfqpoint{2.036608in}{1.957205in}}{\pgfqpoint{2.025558in}{1.957205in}}%
\pgfpathcurveto{\pgfqpoint{2.014508in}{1.957205in}}{\pgfqpoint{2.003909in}{1.952814in}}{\pgfqpoint{1.996095in}{1.945001in}}%
\pgfpathcurveto{\pgfqpoint{1.988281in}{1.937187in}}{\pgfqpoint{1.983891in}{1.926588in}}{\pgfqpoint{1.983891in}{1.915538in}}%
\pgfpathcurveto{\pgfqpoint{1.983891in}{1.904488in}}{\pgfqpoint{1.988281in}{1.893889in}}{\pgfqpoint{1.996095in}{1.886075in}}%
\pgfpathcurveto{\pgfqpoint{2.003909in}{1.878262in}}{\pgfqpoint{2.014508in}{1.873871in}}{\pgfqpoint{2.025558in}{1.873871in}}%
\pgfpathclose%
\pgfusepath{stroke,fill}%
\end{pgfscope}%
\begin{pgfscope}%
\pgfpathrectangle{\pgfqpoint{0.600000in}{0.600000in}}{\pgfqpoint{3.900000in}{3.900000in}}%
\pgfusepath{clip}%
\pgfsetbuttcap%
\pgfsetroundjoin%
\definecolor{currentfill}{rgb}{0.121569,0.466667,0.705882}%
\pgfsetfillcolor{currentfill}%
\pgfsetlinewidth{1.003750pt}%
\definecolor{currentstroke}{rgb}{0.121569,0.466667,0.705882}%
\pgfsetstrokecolor{currentstroke}%
\pgfsetdash{}{0pt}%
\pgfpathmoveto{\pgfqpoint{1.705962in}{1.634244in}}%
\pgfpathcurveto{\pgfqpoint{1.717012in}{1.634244in}}{\pgfqpoint{1.727611in}{1.638634in}}{\pgfqpoint{1.735424in}{1.646448in}}%
\pgfpathcurveto{\pgfqpoint{1.743238in}{1.654262in}}{\pgfqpoint{1.747628in}{1.664861in}}{\pgfqpoint{1.747628in}{1.675911in}}%
\pgfpathcurveto{\pgfqpoint{1.747628in}{1.686961in}}{\pgfqpoint{1.743238in}{1.697560in}}{\pgfqpoint{1.735424in}{1.705374in}}%
\pgfpathcurveto{\pgfqpoint{1.727611in}{1.713187in}}{\pgfqpoint{1.717012in}{1.717577in}}{\pgfqpoint{1.705962in}{1.717577in}}%
\pgfpathcurveto{\pgfqpoint{1.694912in}{1.717577in}}{\pgfqpoint{1.684313in}{1.713187in}}{\pgfqpoint{1.676499in}{1.705374in}}%
\pgfpathcurveto{\pgfqpoint{1.668685in}{1.697560in}}{\pgfqpoint{1.664295in}{1.686961in}}{\pgfqpoint{1.664295in}{1.675911in}}%
\pgfpathcurveto{\pgfqpoint{1.664295in}{1.664861in}}{\pgfqpoint{1.668685in}{1.654262in}}{\pgfqpoint{1.676499in}{1.646448in}}%
\pgfpathcurveto{\pgfqpoint{1.684313in}{1.638634in}}{\pgfqpoint{1.694912in}{1.634244in}}{\pgfqpoint{1.705962in}{1.634244in}}%
\pgfpathclose%
\pgfusepath{stroke,fill}%
\end{pgfscope}%
\begin{pgfscope}%
\pgfpathrectangle{\pgfqpoint{0.600000in}{0.600000in}}{\pgfqpoint{3.900000in}{3.900000in}}%
\pgfusepath{clip}%
\pgfsetbuttcap%
\pgfsetroundjoin%
\definecolor{currentfill}{rgb}{0.121569,0.466667,0.705882}%
\pgfsetfillcolor{currentfill}%
\pgfsetlinewidth{1.003750pt}%
\definecolor{currentstroke}{rgb}{0.121569,0.466667,0.705882}%
\pgfsetstrokecolor{currentstroke}%
\pgfsetdash{}{0pt}%
\pgfpathmoveto{\pgfqpoint{3.104864in}{1.901105in}}%
\pgfpathcurveto{\pgfqpoint{3.115914in}{1.901105in}}{\pgfqpoint{3.126513in}{1.905495in}}{\pgfqpoint{3.134327in}{1.913309in}}%
\pgfpathcurveto{\pgfqpoint{3.142140in}{1.921122in}}{\pgfqpoint{3.146531in}{1.931721in}}{\pgfqpoint{3.146531in}{1.942771in}}%
\pgfpathcurveto{\pgfqpoint{3.146531in}{1.953821in}}{\pgfqpoint{3.142140in}{1.964421in}}{\pgfqpoint{3.134327in}{1.972234in}}%
\pgfpathcurveto{\pgfqpoint{3.126513in}{1.980048in}}{\pgfqpoint{3.115914in}{1.984438in}}{\pgfqpoint{3.104864in}{1.984438in}}%
\pgfpathcurveto{\pgfqpoint{3.093814in}{1.984438in}}{\pgfqpoint{3.083215in}{1.980048in}}{\pgfqpoint{3.075401in}{1.972234in}}%
\pgfpathcurveto{\pgfqpoint{3.067587in}{1.964421in}}{\pgfqpoint{3.063197in}{1.953821in}}{\pgfqpoint{3.063197in}{1.942771in}}%
\pgfpathcurveto{\pgfqpoint{3.063197in}{1.931721in}}{\pgfqpoint{3.067587in}{1.921122in}}{\pgfqpoint{3.075401in}{1.913309in}}%
\pgfpathcurveto{\pgfqpoint{3.083215in}{1.905495in}}{\pgfqpoint{3.093814in}{1.901105in}}{\pgfqpoint{3.104864in}{1.901105in}}%
\pgfpathclose%
\pgfusepath{stroke,fill}%
\end{pgfscope}%
\begin{pgfscope}%
\pgfpathrectangle{\pgfqpoint{0.600000in}{0.600000in}}{\pgfqpoint{3.900000in}{3.900000in}}%
\pgfusepath{clip}%
\pgfsetbuttcap%
\pgfsetroundjoin%
\definecolor{currentfill}{rgb}{0.121569,0.466667,0.705882}%
\pgfsetfillcolor{currentfill}%
\pgfsetlinewidth{1.003750pt}%
\definecolor{currentstroke}{rgb}{0.121569,0.466667,0.705882}%
\pgfsetstrokecolor{currentstroke}%
\pgfsetdash{}{0pt}%
\pgfpathmoveto{\pgfqpoint{2.959575in}{3.447463in}}%
\pgfpathcurveto{\pgfqpoint{2.970625in}{3.447463in}}{\pgfqpoint{2.981224in}{3.451854in}}{\pgfqpoint{2.989037in}{3.459667in}}%
\pgfpathcurveto{\pgfqpoint{2.996851in}{3.467481in}}{\pgfqpoint{3.001241in}{3.478080in}}{\pgfqpoint{3.001241in}{3.489130in}}%
\pgfpathcurveto{\pgfqpoint{3.001241in}{3.500180in}}{\pgfqpoint{2.996851in}{3.510779in}}{\pgfqpoint{2.989037in}{3.518593in}}%
\pgfpathcurveto{\pgfqpoint{2.981224in}{3.526406in}}{\pgfqpoint{2.970625in}{3.530797in}}{\pgfqpoint{2.959575in}{3.530797in}}%
\pgfpathcurveto{\pgfqpoint{2.948524in}{3.530797in}}{\pgfqpoint{2.937925in}{3.526406in}}{\pgfqpoint{2.930112in}{3.518593in}}%
\pgfpathcurveto{\pgfqpoint{2.922298in}{3.510779in}}{\pgfqpoint{2.917908in}{3.500180in}}{\pgfqpoint{2.917908in}{3.489130in}}%
\pgfpathcurveto{\pgfqpoint{2.917908in}{3.478080in}}{\pgfqpoint{2.922298in}{3.467481in}}{\pgfqpoint{2.930112in}{3.459667in}}%
\pgfpathcurveto{\pgfqpoint{2.937925in}{3.451854in}}{\pgfqpoint{2.948524in}{3.447463in}}{\pgfqpoint{2.959575in}{3.447463in}}%
\pgfpathclose%
\pgfusepath{stroke,fill}%
\end{pgfscope}%
\begin{pgfscope}%
\pgfpathrectangle{\pgfqpoint{0.600000in}{0.600000in}}{\pgfqpoint{3.900000in}{3.900000in}}%
\pgfusepath{clip}%
\pgfsetbuttcap%
\pgfsetroundjoin%
\definecolor{currentfill}{rgb}{0.121569,0.466667,0.705882}%
\pgfsetfillcolor{currentfill}%
\pgfsetlinewidth{1.003750pt}%
\definecolor{currentstroke}{rgb}{0.121569,0.466667,0.705882}%
\pgfsetstrokecolor{currentstroke}%
\pgfsetdash{}{0pt}%
\pgfpathmoveto{\pgfqpoint{2.766573in}{3.047857in}}%
\pgfpathcurveto{\pgfqpoint{2.777623in}{3.047857in}}{\pgfqpoint{2.788222in}{3.052248in}}{\pgfqpoint{2.796036in}{3.060061in}}%
\pgfpathcurveto{\pgfqpoint{2.803849in}{3.067875in}}{\pgfqpoint{2.808240in}{3.078474in}}{\pgfqpoint{2.808240in}{3.089524in}}%
\pgfpathcurveto{\pgfqpoint{2.808240in}{3.100574in}}{\pgfqpoint{2.803849in}{3.111173in}}{\pgfqpoint{2.796036in}{3.118987in}}%
\pgfpathcurveto{\pgfqpoint{2.788222in}{3.126800in}}{\pgfqpoint{2.777623in}{3.131191in}}{\pgfqpoint{2.766573in}{3.131191in}}%
\pgfpathcurveto{\pgfqpoint{2.755523in}{3.131191in}}{\pgfqpoint{2.744924in}{3.126800in}}{\pgfqpoint{2.737110in}{3.118987in}}%
\pgfpathcurveto{\pgfqpoint{2.729296in}{3.111173in}}{\pgfqpoint{2.724906in}{3.100574in}}{\pgfqpoint{2.724906in}{3.089524in}}%
\pgfpathcurveto{\pgfqpoint{2.724906in}{3.078474in}}{\pgfqpoint{2.729296in}{3.067875in}}{\pgfqpoint{2.737110in}{3.060061in}}%
\pgfpathcurveto{\pgfqpoint{2.744924in}{3.052248in}}{\pgfqpoint{2.755523in}{3.047857in}}{\pgfqpoint{2.766573in}{3.047857in}}%
\pgfpathclose%
\pgfusepath{stroke,fill}%
\end{pgfscope}%
\begin{pgfscope}%
\pgfpathrectangle{\pgfqpoint{0.600000in}{0.600000in}}{\pgfqpoint{3.900000in}{3.900000in}}%
\pgfusepath{clip}%
\pgfsetbuttcap%
\pgfsetroundjoin%
\definecolor{currentfill}{rgb}{0.121569,0.466667,0.705882}%
\pgfsetfillcolor{currentfill}%
\pgfsetlinewidth{1.003750pt}%
\definecolor{currentstroke}{rgb}{0.121569,0.466667,0.705882}%
\pgfsetstrokecolor{currentstroke}%
\pgfsetdash{}{0pt}%
\pgfpathmoveto{\pgfqpoint{2.761031in}{1.652918in}}%
\pgfpathcurveto{\pgfqpoint{2.772081in}{1.652918in}}{\pgfqpoint{2.782680in}{1.657309in}}{\pgfqpoint{2.790493in}{1.665122in}}%
\pgfpathcurveto{\pgfqpoint{2.798307in}{1.672936in}}{\pgfqpoint{2.802697in}{1.683535in}}{\pgfqpoint{2.802697in}{1.694585in}}%
\pgfpathcurveto{\pgfqpoint{2.802697in}{1.705635in}}{\pgfqpoint{2.798307in}{1.716234in}}{\pgfqpoint{2.790493in}{1.724048in}}%
\pgfpathcurveto{\pgfqpoint{2.782680in}{1.731862in}}{\pgfqpoint{2.772081in}{1.736252in}}{\pgfqpoint{2.761031in}{1.736252in}}%
\pgfpathcurveto{\pgfqpoint{2.749981in}{1.736252in}}{\pgfqpoint{2.739382in}{1.731862in}}{\pgfqpoint{2.731568in}{1.724048in}}%
\pgfpathcurveto{\pgfqpoint{2.723754in}{1.716234in}}{\pgfqpoint{2.719364in}{1.705635in}}{\pgfqpoint{2.719364in}{1.694585in}}%
\pgfpathcurveto{\pgfqpoint{2.719364in}{1.683535in}}{\pgfqpoint{2.723754in}{1.672936in}}{\pgfqpoint{2.731568in}{1.665122in}}%
\pgfpathcurveto{\pgfqpoint{2.739382in}{1.657309in}}{\pgfqpoint{2.749981in}{1.652918in}}{\pgfqpoint{2.761031in}{1.652918in}}%
\pgfpathclose%
\pgfusepath{stroke,fill}%
\end{pgfscope}%
\begin{pgfscope}%
\pgfpathrectangle{\pgfqpoint{0.600000in}{0.600000in}}{\pgfqpoint{3.900000in}{3.900000in}}%
\pgfusepath{clip}%
\pgfsetbuttcap%
\pgfsetroundjoin%
\definecolor{currentfill}{rgb}{0.121569,0.466667,0.705882}%
\pgfsetfillcolor{currentfill}%
\pgfsetlinewidth{1.003750pt}%
\definecolor{currentstroke}{rgb}{0.121569,0.466667,0.705882}%
\pgfsetstrokecolor{currentstroke}%
\pgfsetdash{}{0pt}%
\pgfpathmoveto{\pgfqpoint{2.809807in}{2.504863in}}%
\pgfpathcurveto{\pgfqpoint{2.820857in}{2.504863in}}{\pgfqpoint{2.831456in}{2.509253in}}{\pgfqpoint{2.839269in}{2.517067in}}%
\pgfpathcurveto{\pgfqpoint{2.847083in}{2.524881in}}{\pgfqpoint{2.851473in}{2.535480in}}{\pgfqpoint{2.851473in}{2.546530in}}%
\pgfpathcurveto{\pgfqpoint{2.851473in}{2.557580in}}{\pgfqpoint{2.847083in}{2.568179in}}{\pgfqpoint{2.839269in}{2.575993in}}%
\pgfpathcurveto{\pgfqpoint{2.831456in}{2.583806in}}{\pgfqpoint{2.820857in}{2.588197in}}{\pgfqpoint{2.809807in}{2.588197in}}%
\pgfpathcurveto{\pgfqpoint{2.798757in}{2.588197in}}{\pgfqpoint{2.788158in}{2.583806in}}{\pgfqpoint{2.780344in}{2.575993in}}%
\pgfpathcurveto{\pgfqpoint{2.772530in}{2.568179in}}{\pgfqpoint{2.768140in}{2.557580in}}{\pgfqpoint{2.768140in}{2.546530in}}%
\pgfpathcurveto{\pgfqpoint{2.768140in}{2.535480in}}{\pgfqpoint{2.772530in}{2.524881in}}{\pgfqpoint{2.780344in}{2.517067in}}%
\pgfpathcurveto{\pgfqpoint{2.788158in}{2.509253in}}{\pgfqpoint{2.798757in}{2.504863in}}{\pgfqpoint{2.809807in}{2.504863in}}%
\pgfpathclose%
\pgfusepath{stroke,fill}%
\end{pgfscope}%
\begin{pgfscope}%
\pgfpathrectangle{\pgfqpoint{0.600000in}{0.600000in}}{\pgfqpoint{3.900000in}{3.900000in}}%
\pgfusepath{clip}%
\pgfsetbuttcap%
\pgfsetroundjoin%
\definecolor{currentfill}{rgb}{0.121569,0.466667,0.705882}%
\pgfsetfillcolor{currentfill}%
\pgfsetlinewidth{1.003750pt}%
\definecolor{currentstroke}{rgb}{0.121569,0.466667,0.705882}%
\pgfsetstrokecolor{currentstroke}%
\pgfsetdash{}{0pt}%
\pgfpathmoveto{\pgfqpoint{2.510760in}{2.553268in}}%
\pgfpathcurveto{\pgfqpoint{2.521810in}{2.553268in}}{\pgfqpoint{2.532409in}{2.557658in}}{\pgfqpoint{2.540223in}{2.565472in}}%
\pgfpathcurveto{\pgfqpoint{2.548036in}{2.573286in}}{\pgfqpoint{2.552426in}{2.583885in}}{\pgfqpoint{2.552426in}{2.594935in}}%
\pgfpathcurveto{\pgfqpoint{2.552426in}{2.605985in}}{\pgfqpoint{2.548036in}{2.616584in}}{\pgfqpoint{2.540223in}{2.624398in}}%
\pgfpathcurveto{\pgfqpoint{2.532409in}{2.632211in}}{\pgfqpoint{2.521810in}{2.636602in}}{\pgfqpoint{2.510760in}{2.636602in}}%
\pgfpathcurveto{\pgfqpoint{2.499710in}{2.636602in}}{\pgfqpoint{2.489111in}{2.632211in}}{\pgfqpoint{2.481297in}{2.624398in}}%
\pgfpathcurveto{\pgfqpoint{2.473483in}{2.616584in}}{\pgfqpoint{2.469093in}{2.605985in}}{\pgfqpoint{2.469093in}{2.594935in}}%
\pgfpathcurveto{\pgfqpoint{2.469093in}{2.583885in}}{\pgfqpoint{2.473483in}{2.573286in}}{\pgfqpoint{2.481297in}{2.565472in}}%
\pgfpathcurveto{\pgfqpoint{2.489111in}{2.557658in}}{\pgfqpoint{2.499710in}{2.553268in}}{\pgfqpoint{2.510760in}{2.553268in}}%
\pgfpathclose%
\pgfusepath{stroke,fill}%
\end{pgfscope}%
\begin{pgfscope}%
\pgfpathrectangle{\pgfqpoint{0.600000in}{0.600000in}}{\pgfqpoint{3.900000in}{3.900000in}}%
\pgfusepath{clip}%
\pgfsetbuttcap%
\pgfsetroundjoin%
\definecolor{currentfill}{rgb}{0.121569,0.466667,0.705882}%
\pgfsetfillcolor{currentfill}%
\pgfsetlinewidth{1.003750pt}%
\definecolor{currentstroke}{rgb}{0.121569,0.466667,0.705882}%
\pgfsetstrokecolor{currentstroke}%
\pgfsetdash{}{0pt}%
\pgfpathmoveto{\pgfqpoint{2.976418in}{3.412021in}}%
\pgfpathcurveto{\pgfqpoint{2.987468in}{3.412021in}}{\pgfqpoint{2.998067in}{3.416411in}}{\pgfqpoint{3.005880in}{3.424225in}}%
\pgfpathcurveto{\pgfqpoint{3.013694in}{3.432038in}}{\pgfqpoint{3.018084in}{3.442637in}}{\pgfqpoint{3.018084in}{3.453687in}}%
\pgfpathcurveto{\pgfqpoint{3.018084in}{3.464738in}}{\pgfqpoint{3.013694in}{3.475337in}}{\pgfqpoint{3.005880in}{3.483150in}}%
\pgfpathcurveto{\pgfqpoint{2.998067in}{3.490964in}}{\pgfqpoint{2.987468in}{3.495354in}}{\pgfqpoint{2.976418in}{3.495354in}}%
\pgfpathcurveto{\pgfqpoint{2.965367in}{3.495354in}}{\pgfqpoint{2.954768in}{3.490964in}}{\pgfqpoint{2.946955in}{3.483150in}}%
\pgfpathcurveto{\pgfqpoint{2.939141in}{3.475337in}}{\pgfqpoint{2.934751in}{3.464738in}}{\pgfqpoint{2.934751in}{3.453687in}}%
\pgfpathcurveto{\pgfqpoint{2.934751in}{3.442637in}}{\pgfqpoint{2.939141in}{3.432038in}}{\pgfqpoint{2.946955in}{3.424225in}}%
\pgfpathcurveto{\pgfqpoint{2.954768in}{3.416411in}}{\pgfqpoint{2.965367in}{3.412021in}}{\pgfqpoint{2.976418in}{3.412021in}}%
\pgfpathclose%
\pgfusepath{stroke,fill}%
\end{pgfscope}%
\begin{pgfscope}%
\pgfpathrectangle{\pgfqpoint{0.600000in}{0.600000in}}{\pgfqpoint{3.900000in}{3.900000in}}%
\pgfusepath{clip}%
\pgfsetbuttcap%
\pgfsetroundjoin%
\definecolor{currentfill}{rgb}{0.121569,0.466667,0.705882}%
\pgfsetfillcolor{currentfill}%
\pgfsetlinewidth{1.003750pt}%
\definecolor{currentstroke}{rgb}{0.121569,0.466667,0.705882}%
\pgfsetstrokecolor{currentstroke}%
\pgfsetdash{}{0pt}%
\pgfpathmoveto{\pgfqpoint{2.253513in}{2.350737in}}%
\pgfpathcurveto{\pgfqpoint{2.264563in}{2.350737in}}{\pgfqpoint{2.275162in}{2.355127in}}{\pgfqpoint{2.282976in}{2.362941in}}%
\pgfpathcurveto{\pgfqpoint{2.290790in}{2.370754in}}{\pgfqpoint{2.295180in}{2.381353in}}{\pgfqpoint{2.295180in}{2.392403in}}%
\pgfpathcurveto{\pgfqpoint{2.295180in}{2.403453in}}{\pgfqpoint{2.290790in}{2.414052in}}{\pgfqpoint{2.282976in}{2.421866in}}%
\pgfpathcurveto{\pgfqpoint{2.275162in}{2.429680in}}{\pgfqpoint{2.264563in}{2.434070in}}{\pgfqpoint{2.253513in}{2.434070in}}%
\pgfpathcurveto{\pgfqpoint{2.242463in}{2.434070in}}{\pgfqpoint{2.231864in}{2.429680in}}{\pgfqpoint{2.224050in}{2.421866in}}%
\pgfpathcurveto{\pgfqpoint{2.216237in}{2.414052in}}{\pgfqpoint{2.211847in}{2.403453in}}{\pgfqpoint{2.211847in}{2.392403in}}%
\pgfpathcurveto{\pgfqpoint{2.211847in}{2.381353in}}{\pgfqpoint{2.216237in}{2.370754in}}{\pgfqpoint{2.224050in}{2.362941in}}%
\pgfpathcurveto{\pgfqpoint{2.231864in}{2.355127in}}{\pgfqpoint{2.242463in}{2.350737in}}{\pgfqpoint{2.253513in}{2.350737in}}%
\pgfpathclose%
\pgfusepath{stroke,fill}%
\end{pgfscope}%
\begin{pgfscope}%
\pgfpathrectangle{\pgfqpoint{0.600000in}{0.600000in}}{\pgfqpoint{3.900000in}{3.900000in}}%
\pgfusepath{clip}%
\pgfsetbuttcap%
\pgfsetroundjoin%
\definecolor{currentfill}{rgb}{0.121569,0.466667,0.705882}%
\pgfsetfillcolor{currentfill}%
\pgfsetlinewidth{1.003750pt}%
\definecolor{currentstroke}{rgb}{0.121569,0.466667,0.705882}%
\pgfsetstrokecolor{currentstroke}%
\pgfsetdash{}{0pt}%
\pgfpathmoveto{\pgfqpoint{2.549145in}{2.161689in}}%
\pgfpathcurveto{\pgfqpoint{2.560195in}{2.161689in}}{\pgfqpoint{2.570794in}{2.166079in}}{\pgfqpoint{2.578608in}{2.173893in}}%
\pgfpathcurveto{\pgfqpoint{2.586422in}{2.181707in}}{\pgfqpoint{2.590812in}{2.192306in}}{\pgfqpoint{2.590812in}{2.203356in}}%
\pgfpathcurveto{\pgfqpoint{2.590812in}{2.214406in}}{\pgfqpoint{2.586422in}{2.225005in}}{\pgfqpoint{2.578608in}{2.232819in}}%
\pgfpathcurveto{\pgfqpoint{2.570794in}{2.240632in}}{\pgfqpoint{2.560195in}{2.245022in}}{\pgfqpoint{2.549145in}{2.245022in}}%
\pgfpathcurveto{\pgfqpoint{2.538095in}{2.245022in}}{\pgfqpoint{2.527496in}{2.240632in}}{\pgfqpoint{2.519682in}{2.232819in}}%
\pgfpathcurveto{\pgfqpoint{2.511869in}{2.225005in}}{\pgfqpoint{2.507478in}{2.214406in}}{\pgfqpoint{2.507478in}{2.203356in}}%
\pgfpathcurveto{\pgfqpoint{2.507478in}{2.192306in}}{\pgfqpoint{2.511869in}{2.181707in}}{\pgfqpoint{2.519682in}{2.173893in}}%
\pgfpathcurveto{\pgfqpoint{2.527496in}{2.166079in}}{\pgfqpoint{2.538095in}{2.161689in}}{\pgfqpoint{2.549145in}{2.161689in}}%
\pgfpathclose%
\pgfusepath{stroke,fill}%
\end{pgfscope}%
\begin{pgfscope}%
\pgfpathrectangle{\pgfqpoint{0.600000in}{0.600000in}}{\pgfqpoint{3.900000in}{3.900000in}}%
\pgfusepath{clip}%
\pgfsetbuttcap%
\pgfsetroundjoin%
\definecolor{currentfill}{rgb}{0.121569,0.466667,0.705882}%
\pgfsetfillcolor{currentfill}%
\pgfsetlinewidth{1.003750pt}%
\definecolor{currentstroke}{rgb}{0.121569,0.466667,0.705882}%
\pgfsetstrokecolor{currentstroke}%
\pgfsetdash{}{0pt}%
\pgfpathmoveto{\pgfqpoint{3.046188in}{2.150308in}}%
\pgfpathcurveto{\pgfqpoint{3.057238in}{2.150308in}}{\pgfqpoint{3.067837in}{2.154699in}}{\pgfqpoint{3.075650in}{2.162512in}}%
\pgfpathcurveto{\pgfqpoint{3.083464in}{2.170326in}}{\pgfqpoint{3.087854in}{2.180925in}}{\pgfqpoint{3.087854in}{2.191975in}}%
\pgfpathcurveto{\pgfqpoint{3.087854in}{2.203025in}}{\pgfqpoint{3.083464in}{2.213624in}}{\pgfqpoint{3.075650in}{2.221438in}}%
\pgfpathcurveto{\pgfqpoint{3.067837in}{2.229251in}}{\pgfqpoint{3.057238in}{2.233642in}}{\pgfqpoint{3.046188in}{2.233642in}}%
\pgfpathcurveto{\pgfqpoint{3.035137in}{2.233642in}}{\pgfqpoint{3.024538in}{2.229251in}}{\pgfqpoint{3.016725in}{2.221438in}}%
\pgfpathcurveto{\pgfqpoint{3.008911in}{2.213624in}}{\pgfqpoint{3.004521in}{2.203025in}}{\pgfqpoint{3.004521in}{2.191975in}}%
\pgfpathcurveto{\pgfqpoint{3.004521in}{2.180925in}}{\pgfqpoint{3.008911in}{2.170326in}}{\pgfqpoint{3.016725in}{2.162512in}}%
\pgfpathcurveto{\pgfqpoint{3.024538in}{2.154699in}}{\pgfqpoint{3.035137in}{2.150308in}}{\pgfqpoint{3.046188in}{2.150308in}}%
\pgfpathclose%
\pgfusepath{stroke,fill}%
\end{pgfscope}%
\begin{pgfscope}%
\pgfpathrectangle{\pgfqpoint{0.600000in}{0.600000in}}{\pgfqpoint{3.900000in}{3.900000in}}%
\pgfusepath{clip}%
\pgfsetbuttcap%
\pgfsetroundjoin%
\definecolor{currentfill}{rgb}{0.121569,0.466667,0.705882}%
\pgfsetfillcolor{currentfill}%
\pgfsetlinewidth{1.003750pt}%
\definecolor{currentstroke}{rgb}{0.121569,0.466667,0.705882}%
\pgfsetstrokecolor{currentstroke}%
\pgfsetdash{}{0pt}%
\pgfpathmoveto{\pgfqpoint{2.723803in}{2.385172in}}%
\pgfpathcurveto{\pgfqpoint{2.734853in}{2.385172in}}{\pgfqpoint{2.745452in}{2.389563in}}{\pgfqpoint{2.753265in}{2.397376in}}%
\pgfpathcurveto{\pgfqpoint{2.761079in}{2.405190in}}{\pgfqpoint{2.765469in}{2.415789in}}{\pgfqpoint{2.765469in}{2.426839in}}%
\pgfpathcurveto{\pgfqpoint{2.765469in}{2.437889in}}{\pgfqpoint{2.761079in}{2.448488in}}{\pgfqpoint{2.753265in}{2.456302in}}%
\pgfpathcurveto{\pgfqpoint{2.745452in}{2.464116in}}{\pgfqpoint{2.734853in}{2.468506in}}{\pgfqpoint{2.723803in}{2.468506in}}%
\pgfpathcurveto{\pgfqpoint{2.712752in}{2.468506in}}{\pgfqpoint{2.702153in}{2.464116in}}{\pgfqpoint{2.694340in}{2.456302in}}%
\pgfpathcurveto{\pgfqpoint{2.686526in}{2.448488in}}{\pgfqpoint{2.682136in}{2.437889in}}{\pgfqpoint{2.682136in}{2.426839in}}%
\pgfpathcurveto{\pgfqpoint{2.682136in}{2.415789in}}{\pgfqpoint{2.686526in}{2.405190in}}{\pgfqpoint{2.694340in}{2.397376in}}%
\pgfpathcurveto{\pgfqpoint{2.702153in}{2.389563in}}{\pgfqpoint{2.712752in}{2.385172in}}{\pgfqpoint{2.723803in}{2.385172in}}%
\pgfpathclose%
\pgfusepath{stroke,fill}%
\end{pgfscope}%
\begin{pgfscope}%
\pgfpathrectangle{\pgfqpoint{0.600000in}{0.600000in}}{\pgfqpoint{3.900000in}{3.900000in}}%
\pgfusepath{clip}%
\pgfsetbuttcap%
\pgfsetroundjoin%
\definecolor{currentfill}{rgb}{0.121569,0.466667,0.705882}%
\pgfsetfillcolor{currentfill}%
\pgfsetlinewidth{1.003750pt}%
\definecolor{currentstroke}{rgb}{0.121569,0.466667,0.705882}%
\pgfsetstrokecolor{currentstroke}%
\pgfsetdash{}{0pt}%
\pgfpathmoveto{\pgfqpoint{2.798549in}{2.739115in}}%
\pgfpathcurveto{\pgfqpoint{2.809599in}{2.739115in}}{\pgfqpoint{2.820198in}{2.743505in}}{\pgfqpoint{2.828012in}{2.751319in}}%
\pgfpathcurveto{\pgfqpoint{2.835826in}{2.759132in}}{\pgfqpoint{2.840216in}{2.769731in}}{\pgfqpoint{2.840216in}{2.780782in}}%
\pgfpathcurveto{\pgfqpoint{2.840216in}{2.791832in}}{\pgfqpoint{2.835826in}{2.802431in}}{\pgfqpoint{2.828012in}{2.810244in}}%
\pgfpathcurveto{\pgfqpoint{2.820198in}{2.818058in}}{\pgfqpoint{2.809599in}{2.822448in}}{\pgfqpoint{2.798549in}{2.822448in}}%
\pgfpathcurveto{\pgfqpoint{2.787499in}{2.822448in}}{\pgfqpoint{2.776900in}{2.818058in}}{\pgfqpoint{2.769086in}{2.810244in}}%
\pgfpathcurveto{\pgfqpoint{2.761273in}{2.802431in}}{\pgfqpoint{2.756882in}{2.791832in}}{\pgfqpoint{2.756882in}{2.780782in}}%
\pgfpathcurveto{\pgfqpoint{2.756882in}{2.769731in}}{\pgfqpoint{2.761273in}{2.759132in}}{\pgfqpoint{2.769086in}{2.751319in}}%
\pgfpathcurveto{\pgfqpoint{2.776900in}{2.743505in}}{\pgfqpoint{2.787499in}{2.739115in}}{\pgfqpoint{2.798549in}{2.739115in}}%
\pgfpathclose%
\pgfusepath{stroke,fill}%
\end{pgfscope}%
\begin{pgfscope}%
\pgfpathrectangle{\pgfqpoint{0.600000in}{0.600000in}}{\pgfqpoint{3.900000in}{3.900000in}}%
\pgfusepath{clip}%
\pgfsetbuttcap%
\pgfsetroundjoin%
\definecolor{currentfill}{rgb}{0.121569,0.466667,0.705882}%
\pgfsetfillcolor{currentfill}%
\pgfsetlinewidth{1.003750pt}%
\definecolor{currentstroke}{rgb}{0.121569,0.466667,0.705882}%
\pgfsetstrokecolor{currentstroke}%
\pgfsetdash{}{0pt}%
\pgfpathmoveto{\pgfqpoint{2.331413in}{2.328752in}}%
\pgfpathcurveto{\pgfqpoint{2.342463in}{2.328752in}}{\pgfqpoint{2.353062in}{2.333142in}}{\pgfqpoint{2.360876in}{2.340956in}}%
\pgfpathcurveto{\pgfqpoint{2.368689in}{2.348769in}}{\pgfqpoint{2.373080in}{2.359368in}}{\pgfqpoint{2.373080in}{2.370418in}}%
\pgfpathcurveto{\pgfqpoint{2.373080in}{2.381468in}}{\pgfqpoint{2.368689in}{2.392068in}}{\pgfqpoint{2.360876in}{2.399881in}}%
\pgfpathcurveto{\pgfqpoint{2.353062in}{2.407695in}}{\pgfqpoint{2.342463in}{2.412085in}}{\pgfqpoint{2.331413in}{2.412085in}}%
\pgfpathcurveto{\pgfqpoint{2.320363in}{2.412085in}}{\pgfqpoint{2.309764in}{2.407695in}}{\pgfqpoint{2.301950in}{2.399881in}}%
\pgfpathcurveto{\pgfqpoint{2.294137in}{2.392068in}}{\pgfqpoint{2.289746in}{2.381468in}}{\pgfqpoint{2.289746in}{2.370418in}}%
\pgfpathcurveto{\pgfqpoint{2.289746in}{2.359368in}}{\pgfqpoint{2.294137in}{2.348769in}}{\pgfqpoint{2.301950in}{2.340956in}}%
\pgfpathcurveto{\pgfqpoint{2.309764in}{2.333142in}}{\pgfqpoint{2.320363in}{2.328752in}}{\pgfqpoint{2.331413in}{2.328752in}}%
\pgfpathclose%
\pgfusepath{stroke,fill}%
\end{pgfscope}%
\begin{pgfscope}%
\pgfpathrectangle{\pgfqpoint{0.600000in}{0.600000in}}{\pgfqpoint{3.900000in}{3.900000in}}%
\pgfusepath{clip}%
\pgfsetbuttcap%
\pgfsetroundjoin%
\definecolor{currentfill}{rgb}{0.121569,0.466667,0.705882}%
\pgfsetfillcolor{currentfill}%
\pgfsetlinewidth{1.003750pt}%
\definecolor{currentstroke}{rgb}{0.121569,0.466667,0.705882}%
\pgfsetstrokecolor{currentstroke}%
\pgfsetdash{}{0pt}%
\pgfpathmoveto{\pgfqpoint{3.202557in}{2.854469in}}%
\pgfpathcurveto{\pgfqpoint{3.213607in}{2.854469in}}{\pgfqpoint{3.224206in}{2.858859in}}{\pgfqpoint{3.232020in}{2.866673in}}%
\pgfpathcurveto{\pgfqpoint{3.239833in}{2.874486in}}{\pgfqpoint{3.244224in}{2.885085in}}{\pgfqpoint{3.244224in}{2.896135in}}%
\pgfpathcurveto{\pgfqpoint{3.244224in}{2.907186in}}{\pgfqpoint{3.239833in}{2.917785in}}{\pgfqpoint{3.232020in}{2.925598in}}%
\pgfpathcurveto{\pgfqpoint{3.224206in}{2.933412in}}{\pgfqpoint{3.213607in}{2.937802in}}{\pgfqpoint{3.202557in}{2.937802in}}%
\pgfpathcurveto{\pgfqpoint{3.191507in}{2.937802in}}{\pgfqpoint{3.180908in}{2.933412in}}{\pgfqpoint{3.173094in}{2.925598in}}%
\pgfpathcurveto{\pgfqpoint{3.165281in}{2.917785in}}{\pgfqpoint{3.160890in}{2.907186in}}{\pgfqpoint{3.160890in}{2.896135in}}%
\pgfpathcurveto{\pgfqpoint{3.160890in}{2.885085in}}{\pgfqpoint{3.165281in}{2.874486in}}{\pgfqpoint{3.173094in}{2.866673in}}%
\pgfpathcurveto{\pgfqpoint{3.180908in}{2.858859in}}{\pgfqpoint{3.191507in}{2.854469in}}{\pgfqpoint{3.202557in}{2.854469in}}%
\pgfpathclose%
\pgfusepath{stroke,fill}%
\end{pgfscope}%
\begin{pgfscope}%
\pgfpathrectangle{\pgfqpoint{0.600000in}{0.600000in}}{\pgfqpoint{3.900000in}{3.900000in}}%
\pgfusepath{clip}%
\pgfsetbuttcap%
\pgfsetroundjoin%
\definecolor{currentfill}{rgb}{0.121569,0.466667,0.705882}%
\pgfsetfillcolor{currentfill}%
\pgfsetlinewidth{1.003750pt}%
\definecolor{currentstroke}{rgb}{0.121569,0.466667,0.705882}%
\pgfsetstrokecolor{currentstroke}%
\pgfsetdash{}{0pt}%
\pgfpathmoveto{\pgfqpoint{1.015268in}{1.528967in}}%
\pgfpathcurveto{\pgfqpoint{1.026318in}{1.528967in}}{\pgfqpoint{1.036917in}{1.533357in}}{\pgfqpoint{1.044730in}{1.541170in}}%
\pgfpathcurveto{\pgfqpoint{1.052544in}{1.548984in}}{\pgfqpoint{1.056934in}{1.559583in}}{\pgfqpoint{1.056934in}{1.570633in}}%
\pgfpathcurveto{\pgfqpoint{1.056934in}{1.581683in}}{\pgfqpoint{1.052544in}{1.592282in}}{\pgfqpoint{1.044730in}{1.600096in}}%
\pgfpathcurveto{\pgfqpoint{1.036917in}{1.607910in}}{\pgfqpoint{1.026318in}{1.612300in}}{\pgfqpoint{1.015268in}{1.612300in}}%
\pgfpathcurveto{\pgfqpoint{1.004217in}{1.612300in}}{\pgfqpoint{0.993618in}{1.607910in}}{\pgfqpoint{0.985805in}{1.600096in}}%
\pgfpathcurveto{\pgfqpoint{0.977991in}{1.592282in}}{\pgfqpoint{0.973601in}{1.581683in}}{\pgfqpoint{0.973601in}{1.570633in}}%
\pgfpathcurveto{\pgfqpoint{0.973601in}{1.559583in}}{\pgfqpoint{0.977991in}{1.548984in}}{\pgfqpoint{0.985805in}{1.541170in}}%
\pgfpathcurveto{\pgfqpoint{0.993618in}{1.533357in}}{\pgfqpoint{1.004217in}{1.528967in}}{\pgfqpoint{1.015268in}{1.528967in}}%
\pgfpathclose%
\pgfusepath{stroke,fill}%
\end{pgfscope}%
\begin{pgfscope}%
\pgfpathrectangle{\pgfqpoint{0.600000in}{0.600000in}}{\pgfqpoint{3.900000in}{3.900000in}}%
\pgfusepath{clip}%
\pgfsetbuttcap%
\pgfsetroundjoin%
\definecolor{currentfill}{rgb}{0.121569,0.466667,0.705882}%
\pgfsetfillcolor{currentfill}%
\pgfsetlinewidth{1.003750pt}%
\definecolor{currentstroke}{rgb}{0.121569,0.466667,0.705882}%
\pgfsetstrokecolor{currentstroke}%
\pgfsetdash{}{0pt}%
\pgfpathmoveto{\pgfqpoint{2.570531in}{2.233245in}}%
\pgfpathcurveto{\pgfqpoint{2.581581in}{2.233245in}}{\pgfqpoint{2.592180in}{2.237635in}}{\pgfqpoint{2.599994in}{2.245449in}}%
\pgfpathcurveto{\pgfqpoint{2.607807in}{2.253262in}}{\pgfqpoint{2.612197in}{2.263861in}}{\pgfqpoint{2.612197in}{2.274911in}}%
\pgfpathcurveto{\pgfqpoint{2.612197in}{2.285962in}}{\pgfqpoint{2.607807in}{2.296561in}}{\pgfqpoint{2.599994in}{2.304374in}}%
\pgfpathcurveto{\pgfqpoint{2.592180in}{2.312188in}}{\pgfqpoint{2.581581in}{2.316578in}}{\pgfqpoint{2.570531in}{2.316578in}}%
\pgfpathcurveto{\pgfqpoint{2.559481in}{2.316578in}}{\pgfqpoint{2.548882in}{2.312188in}}{\pgfqpoint{2.541068in}{2.304374in}}%
\pgfpathcurveto{\pgfqpoint{2.533254in}{2.296561in}}{\pgfqpoint{2.528864in}{2.285962in}}{\pgfqpoint{2.528864in}{2.274911in}}%
\pgfpathcurveto{\pgfqpoint{2.528864in}{2.263861in}}{\pgfqpoint{2.533254in}{2.253262in}}{\pgfqpoint{2.541068in}{2.245449in}}%
\pgfpathcurveto{\pgfqpoint{2.548882in}{2.237635in}}{\pgfqpoint{2.559481in}{2.233245in}}{\pgfqpoint{2.570531in}{2.233245in}}%
\pgfpathclose%
\pgfusepath{stroke,fill}%
\end{pgfscope}%
\begin{pgfscope}%
\pgfpathrectangle{\pgfqpoint{0.600000in}{0.600000in}}{\pgfqpoint{3.900000in}{3.900000in}}%
\pgfusepath{clip}%
\pgfsetbuttcap%
\pgfsetroundjoin%
\definecolor{currentfill}{rgb}{0.121569,0.466667,0.705882}%
\pgfsetfillcolor{currentfill}%
\pgfsetlinewidth{1.003750pt}%
\definecolor{currentstroke}{rgb}{0.121569,0.466667,0.705882}%
\pgfsetstrokecolor{currentstroke}%
\pgfsetdash{}{0pt}%
\pgfpathmoveto{\pgfqpoint{2.260983in}{2.635910in}}%
\pgfpathcurveto{\pgfqpoint{2.272033in}{2.635910in}}{\pgfqpoint{2.282632in}{2.640300in}}{\pgfqpoint{2.290446in}{2.648113in}}%
\pgfpathcurveto{\pgfqpoint{2.298259in}{2.655927in}}{\pgfqpoint{2.302650in}{2.666526in}}{\pgfqpoint{2.302650in}{2.677576in}}%
\pgfpathcurveto{\pgfqpoint{2.302650in}{2.688626in}}{\pgfqpoint{2.298259in}{2.699225in}}{\pgfqpoint{2.290446in}{2.707039in}}%
\pgfpathcurveto{\pgfqpoint{2.282632in}{2.714853in}}{\pgfqpoint{2.272033in}{2.719243in}}{\pgfqpoint{2.260983in}{2.719243in}}%
\pgfpathcurveto{\pgfqpoint{2.249933in}{2.719243in}}{\pgfqpoint{2.239334in}{2.714853in}}{\pgfqpoint{2.231520in}{2.707039in}}%
\pgfpathcurveto{\pgfqpoint{2.223707in}{2.699225in}}{\pgfqpoint{2.219316in}{2.688626in}}{\pgfqpoint{2.219316in}{2.677576in}}%
\pgfpathcurveto{\pgfqpoint{2.219316in}{2.666526in}}{\pgfqpoint{2.223707in}{2.655927in}}{\pgfqpoint{2.231520in}{2.648113in}}%
\pgfpathcurveto{\pgfqpoint{2.239334in}{2.640300in}}{\pgfqpoint{2.249933in}{2.635910in}}{\pgfqpoint{2.260983in}{2.635910in}}%
\pgfpathclose%
\pgfusepath{stroke,fill}%
\end{pgfscope}%
\begin{pgfscope}%
\pgfpathrectangle{\pgfqpoint{0.600000in}{0.600000in}}{\pgfqpoint{3.900000in}{3.900000in}}%
\pgfusepath{clip}%
\pgfsetbuttcap%
\pgfsetroundjoin%
\definecolor{currentfill}{rgb}{0.121569,0.466667,0.705882}%
\pgfsetfillcolor{currentfill}%
\pgfsetlinewidth{1.003750pt}%
\definecolor{currentstroke}{rgb}{0.121569,0.466667,0.705882}%
\pgfsetstrokecolor{currentstroke}%
\pgfsetdash{}{0pt}%
\pgfpathmoveto{\pgfqpoint{2.478370in}{2.229041in}}%
\pgfpathcurveto{\pgfqpoint{2.489420in}{2.229041in}}{\pgfqpoint{2.500019in}{2.233432in}}{\pgfqpoint{2.507832in}{2.241245in}}%
\pgfpathcurveto{\pgfqpoint{2.515646in}{2.249059in}}{\pgfqpoint{2.520036in}{2.259658in}}{\pgfqpoint{2.520036in}{2.270708in}}%
\pgfpathcurveto{\pgfqpoint{2.520036in}{2.281758in}}{\pgfqpoint{2.515646in}{2.292357in}}{\pgfqpoint{2.507832in}{2.300171in}}%
\pgfpathcurveto{\pgfqpoint{2.500019in}{2.307985in}}{\pgfqpoint{2.489420in}{2.312375in}}{\pgfqpoint{2.478370in}{2.312375in}}%
\pgfpathcurveto{\pgfqpoint{2.467320in}{2.312375in}}{\pgfqpoint{2.456721in}{2.307985in}}{\pgfqpoint{2.448907in}{2.300171in}}%
\pgfpathcurveto{\pgfqpoint{2.441093in}{2.292357in}}{\pgfqpoint{2.436703in}{2.281758in}}{\pgfqpoint{2.436703in}{2.270708in}}%
\pgfpathcurveto{\pgfqpoint{2.436703in}{2.259658in}}{\pgfqpoint{2.441093in}{2.249059in}}{\pgfqpoint{2.448907in}{2.241245in}}%
\pgfpathcurveto{\pgfqpoint{2.456721in}{2.233432in}}{\pgfqpoint{2.467320in}{2.229041in}}{\pgfqpoint{2.478370in}{2.229041in}}%
\pgfpathclose%
\pgfusepath{stroke,fill}%
\end{pgfscope}%
\begin{pgfscope}%
\pgfpathrectangle{\pgfqpoint{0.600000in}{0.600000in}}{\pgfqpoint{3.900000in}{3.900000in}}%
\pgfusepath{clip}%
\pgfsetbuttcap%
\pgfsetroundjoin%
\definecolor{currentfill}{rgb}{0.121569,0.466667,0.705882}%
\pgfsetfillcolor{currentfill}%
\pgfsetlinewidth{1.003750pt}%
\definecolor{currentstroke}{rgb}{0.121569,0.466667,0.705882}%
\pgfsetstrokecolor{currentstroke}%
\pgfsetdash{}{0pt}%
\pgfpathmoveto{\pgfqpoint{3.036024in}{2.631907in}}%
\pgfpathcurveto{\pgfqpoint{3.047074in}{2.631907in}}{\pgfqpoint{3.057673in}{2.636297in}}{\pgfqpoint{3.065487in}{2.644110in}}%
\pgfpathcurveto{\pgfqpoint{3.073300in}{2.651924in}}{\pgfqpoint{3.077691in}{2.662523in}}{\pgfqpoint{3.077691in}{2.673573in}}%
\pgfpathcurveto{\pgfqpoint{3.077691in}{2.684623in}}{\pgfqpoint{3.073300in}{2.695222in}}{\pgfqpoint{3.065487in}{2.703036in}}%
\pgfpathcurveto{\pgfqpoint{3.057673in}{2.710850in}}{\pgfqpoint{3.047074in}{2.715240in}}{\pgfqpoint{3.036024in}{2.715240in}}%
\pgfpathcurveto{\pgfqpoint{3.024974in}{2.715240in}}{\pgfqpoint{3.014375in}{2.710850in}}{\pgfqpoint{3.006561in}{2.703036in}}%
\pgfpathcurveto{\pgfqpoint{2.998747in}{2.695222in}}{\pgfqpoint{2.994357in}{2.684623in}}{\pgfqpoint{2.994357in}{2.673573in}}%
\pgfpathcurveto{\pgfqpoint{2.994357in}{2.662523in}}{\pgfqpoint{2.998747in}{2.651924in}}{\pgfqpoint{3.006561in}{2.644110in}}%
\pgfpathcurveto{\pgfqpoint{3.014375in}{2.636297in}}{\pgfqpoint{3.024974in}{2.631907in}}{\pgfqpoint{3.036024in}{2.631907in}}%
\pgfpathclose%
\pgfusepath{stroke,fill}%
\end{pgfscope}%
\begin{pgfscope}%
\pgfpathrectangle{\pgfqpoint{0.600000in}{0.600000in}}{\pgfqpoint{3.900000in}{3.900000in}}%
\pgfusepath{clip}%
\pgfsetbuttcap%
\pgfsetroundjoin%
\definecolor{currentfill}{rgb}{0.121569,0.466667,0.705882}%
\pgfsetfillcolor{currentfill}%
\pgfsetlinewidth{1.003750pt}%
\definecolor{currentstroke}{rgb}{0.121569,0.466667,0.705882}%
\pgfsetstrokecolor{currentstroke}%
\pgfsetdash{}{0pt}%
\pgfpathmoveto{\pgfqpoint{1.918180in}{3.109409in}}%
\pgfpathcurveto{\pgfqpoint{1.929230in}{3.109409in}}{\pgfqpoint{1.939829in}{3.113800in}}{\pgfqpoint{1.947643in}{3.121613in}}%
\pgfpathcurveto{\pgfqpoint{1.955456in}{3.129427in}}{\pgfqpoint{1.959846in}{3.140026in}}{\pgfqpoint{1.959846in}{3.151076in}}%
\pgfpathcurveto{\pgfqpoint{1.959846in}{3.162126in}}{\pgfqpoint{1.955456in}{3.172725in}}{\pgfqpoint{1.947643in}{3.180539in}}%
\pgfpathcurveto{\pgfqpoint{1.939829in}{3.188352in}}{\pgfqpoint{1.929230in}{3.192743in}}{\pgfqpoint{1.918180in}{3.192743in}}%
\pgfpathcurveto{\pgfqpoint{1.907130in}{3.192743in}}{\pgfqpoint{1.896531in}{3.188352in}}{\pgfqpoint{1.888717in}{3.180539in}}%
\pgfpathcurveto{\pgfqpoint{1.880903in}{3.172725in}}{\pgfqpoint{1.876513in}{3.162126in}}{\pgfqpoint{1.876513in}{3.151076in}}%
\pgfpathcurveto{\pgfqpoint{1.876513in}{3.140026in}}{\pgfqpoint{1.880903in}{3.129427in}}{\pgfqpoint{1.888717in}{3.121613in}}%
\pgfpathcurveto{\pgfqpoint{1.896531in}{3.113800in}}{\pgfqpoint{1.907130in}{3.109409in}}{\pgfqpoint{1.918180in}{3.109409in}}%
\pgfpathclose%
\pgfusepath{stroke,fill}%
\end{pgfscope}%
\begin{pgfscope}%
\pgfpathrectangle{\pgfqpoint{0.600000in}{0.600000in}}{\pgfqpoint{3.900000in}{3.900000in}}%
\pgfusepath{clip}%
\pgfsetbuttcap%
\pgfsetroundjoin%
\definecolor{currentfill}{rgb}{0.121569,0.466667,0.705882}%
\pgfsetfillcolor{currentfill}%
\pgfsetlinewidth{1.003750pt}%
\definecolor{currentstroke}{rgb}{0.121569,0.466667,0.705882}%
\pgfsetstrokecolor{currentstroke}%
\pgfsetdash{}{0pt}%
\pgfpathmoveto{\pgfqpoint{2.218995in}{1.777554in}}%
\pgfpathcurveto{\pgfqpoint{2.230045in}{1.777554in}}{\pgfqpoint{2.240644in}{1.781944in}}{\pgfqpoint{2.248457in}{1.789758in}}%
\pgfpathcurveto{\pgfqpoint{2.256271in}{1.797572in}}{\pgfqpoint{2.260661in}{1.808171in}}{\pgfqpoint{2.260661in}{1.819221in}}%
\pgfpathcurveto{\pgfqpoint{2.260661in}{1.830271in}}{\pgfqpoint{2.256271in}{1.840870in}}{\pgfqpoint{2.248457in}{1.848684in}}%
\pgfpathcurveto{\pgfqpoint{2.240644in}{1.856497in}}{\pgfqpoint{2.230045in}{1.860888in}}{\pgfqpoint{2.218995in}{1.860888in}}%
\pgfpathcurveto{\pgfqpoint{2.207944in}{1.860888in}}{\pgfqpoint{2.197345in}{1.856497in}}{\pgfqpoint{2.189532in}{1.848684in}}%
\pgfpathcurveto{\pgfqpoint{2.181718in}{1.840870in}}{\pgfqpoint{2.177328in}{1.830271in}}{\pgfqpoint{2.177328in}{1.819221in}}%
\pgfpathcurveto{\pgfqpoint{2.177328in}{1.808171in}}{\pgfqpoint{2.181718in}{1.797572in}}{\pgfqpoint{2.189532in}{1.789758in}}%
\pgfpathcurveto{\pgfqpoint{2.197345in}{1.781944in}}{\pgfqpoint{2.207944in}{1.777554in}}{\pgfqpoint{2.218995in}{1.777554in}}%
\pgfpathclose%
\pgfusepath{stroke,fill}%
\end{pgfscope}%
\begin{pgfscope}%
\pgfpathrectangle{\pgfqpoint{0.600000in}{0.600000in}}{\pgfqpoint{3.900000in}{3.900000in}}%
\pgfusepath{clip}%
\pgfsetbuttcap%
\pgfsetroundjoin%
\definecolor{currentfill}{rgb}{0.121569,0.466667,0.705882}%
\pgfsetfillcolor{currentfill}%
\pgfsetlinewidth{1.003750pt}%
\definecolor{currentstroke}{rgb}{0.121569,0.466667,0.705882}%
\pgfsetstrokecolor{currentstroke}%
\pgfsetdash{}{0pt}%
\pgfpathmoveto{\pgfqpoint{2.943812in}{3.135509in}}%
\pgfpathcurveto{\pgfqpoint{2.954862in}{3.135509in}}{\pgfqpoint{2.965461in}{3.139899in}}{\pgfqpoint{2.973275in}{3.147713in}}%
\pgfpathcurveto{\pgfqpoint{2.981088in}{3.155526in}}{\pgfqpoint{2.985479in}{3.166126in}}{\pgfqpoint{2.985479in}{3.177176in}}%
\pgfpathcurveto{\pgfqpoint{2.985479in}{3.188226in}}{\pgfqpoint{2.981088in}{3.198825in}}{\pgfqpoint{2.973275in}{3.206638in}}%
\pgfpathcurveto{\pgfqpoint{2.965461in}{3.214452in}}{\pgfqpoint{2.954862in}{3.218842in}}{\pgfqpoint{2.943812in}{3.218842in}}%
\pgfpathcurveto{\pgfqpoint{2.932762in}{3.218842in}}{\pgfqpoint{2.922163in}{3.214452in}}{\pgfqpoint{2.914349in}{3.206638in}}%
\pgfpathcurveto{\pgfqpoint{2.906536in}{3.198825in}}{\pgfqpoint{2.902145in}{3.188226in}}{\pgfqpoint{2.902145in}{3.177176in}}%
\pgfpathcurveto{\pgfqpoint{2.902145in}{3.166126in}}{\pgfqpoint{2.906536in}{3.155526in}}{\pgfqpoint{2.914349in}{3.147713in}}%
\pgfpathcurveto{\pgfqpoint{2.922163in}{3.139899in}}{\pgfqpoint{2.932762in}{3.135509in}}{\pgfqpoint{2.943812in}{3.135509in}}%
\pgfpathclose%
\pgfusepath{stroke,fill}%
\end{pgfscope}%
\begin{pgfscope}%
\pgfpathrectangle{\pgfqpoint{0.600000in}{0.600000in}}{\pgfqpoint{3.900000in}{3.900000in}}%
\pgfusepath{clip}%
\pgfsetbuttcap%
\pgfsetroundjoin%
\definecolor{currentfill}{rgb}{0.121569,0.466667,0.705882}%
\pgfsetfillcolor{currentfill}%
\pgfsetlinewidth{1.003750pt}%
\definecolor{currentstroke}{rgb}{0.121569,0.466667,0.705882}%
\pgfsetstrokecolor{currentstroke}%
\pgfsetdash{}{0pt}%
\pgfpathmoveto{\pgfqpoint{2.829193in}{3.438922in}}%
\pgfpathcurveto{\pgfqpoint{2.840243in}{3.438922in}}{\pgfqpoint{2.850842in}{3.443312in}}{\pgfqpoint{2.858656in}{3.451126in}}%
\pgfpathcurveto{\pgfqpoint{2.866469in}{3.458939in}}{\pgfqpoint{2.870859in}{3.469538in}}{\pgfqpoint{2.870859in}{3.480588in}}%
\pgfpathcurveto{\pgfqpoint{2.870859in}{3.491639in}}{\pgfqpoint{2.866469in}{3.502238in}}{\pgfqpoint{2.858656in}{3.510051in}}%
\pgfpathcurveto{\pgfqpoint{2.850842in}{3.517865in}}{\pgfqpoint{2.840243in}{3.522255in}}{\pgfqpoint{2.829193in}{3.522255in}}%
\pgfpathcurveto{\pgfqpoint{2.818143in}{3.522255in}}{\pgfqpoint{2.807544in}{3.517865in}}{\pgfqpoint{2.799730in}{3.510051in}}%
\pgfpathcurveto{\pgfqpoint{2.791916in}{3.502238in}}{\pgfqpoint{2.787526in}{3.491639in}}{\pgfqpoint{2.787526in}{3.480588in}}%
\pgfpathcurveto{\pgfqpoint{2.787526in}{3.469538in}}{\pgfqpoint{2.791916in}{3.458939in}}{\pgfqpoint{2.799730in}{3.451126in}}%
\pgfpathcurveto{\pgfqpoint{2.807544in}{3.443312in}}{\pgfqpoint{2.818143in}{3.438922in}}{\pgfqpoint{2.829193in}{3.438922in}}%
\pgfpathclose%
\pgfusepath{stroke,fill}%
\end{pgfscope}%
\begin{pgfscope}%
\pgfpathrectangle{\pgfqpoint{0.600000in}{0.600000in}}{\pgfqpoint{3.900000in}{3.900000in}}%
\pgfusepath{clip}%
\pgfsetbuttcap%
\pgfsetroundjoin%
\definecolor{currentfill}{rgb}{0.121569,0.466667,0.705882}%
\pgfsetfillcolor{currentfill}%
\pgfsetlinewidth{1.003750pt}%
\definecolor{currentstroke}{rgb}{0.121569,0.466667,0.705882}%
\pgfsetstrokecolor{currentstroke}%
\pgfsetdash{}{0pt}%
\pgfpathmoveto{\pgfqpoint{3.316869in}{2.448605in}}%
\pgfpathcurveto{\pgfqpoint{3.327920in}{2.448605in}}{\pgfqpoint{3.338519in}{2.452995in}}{\pgfqpoint{3.346332in}{2.460808in}}%
\pgfpathcurveto{\pgfqpoint{3.354146in}{2.468622in}}{\pgfqpoint{3.358536in}{2.479221in}}{\pgfqpoint{3.358536in}{2.490271in}}%
\pgfpathcurveto{\pgfqpoint{3.358536in}{2.501321in}}{\pgfqpoint{3.354146in}{2.511920in}}{\pgfqpoint{3.346332in}{2.519734in}}%
\pgfpathcurveto{\pgfqpoint{3.338519in}{2.527548in}}{\pgfqpoint{3.327920in}{2.531938in}}{\pgfqpoint{3.316869in}{2.531938in}}%
\pgfpathcurveto{\pgfqpoint{3.305819in}{2.531938in}}{\pgfqpoint{3.295220in}{2.527548in}}{\pgfqpoint{3.287407in}{2.519734in}}%
\pgfpathcurveto{\pgfqpoint{3.279593in}{2.511920in}}{\pgfqpoint{3.275203in}{2.501321in}}{\pgfqpoint{3.275203in}{2.490271in}}%
\pgfpathcurveto{\pgfqpoint{3.275203in}{2.479221in}}{\pgfqpoint{3.279593in}{2.468622in}}{\pgfqpoint{3.287407in}{2.460808in}}%
\pgfpathcurveto{\pgfqpoint{3.295220in}{2.452995in}}{\pgfqpoint{3.305819in}{2.448605in}}{\pgfqpoint{3.316869in}{2.448605in}}%
\pgfpathclose%
\pgfusepath{stroke,fill}%
\end{pgfscope}%
\begin{pgfscope}%
\pgfpathrectangle{\pgfqpoint{0.600000in}{0.600000in}}{\pgfqpoint{3.900000in}{3.900000in}}%
\pgfusepath{clip}%
\pgfsetbuttcap%
\pgfsetroundjoin%
\definecolor{currentfill}{rgb}{0.121569,0.466667,0.705882}%
\pgfsetfillcolor{currentfill}%
\pgfsetlinewidth{1.003750pt}%
\definecolor{currentstroke}{rgb}{0.121569,0.466667,0.705882}%
\pgfsetstrokecolor{currentstroke}%
\pgfsetdash{}{0pt}%
\pgfpathmoveto{\pgfqpoint{3.041906in}{2.711754in}}%
\pgfpathcurveto{\pgfqpoint{3.052956in}{2.711754in}}{\pgfqpoint{3.063555in}{2.716144in}}{\pgfqpoint{3.071369in}{2.723958in}}%
\pgfpathcurveto{\pgfqpoint{3.079182in}{2.731772in}}{\pgfqpoint{3.083573in}{2.742371in}}{\pgfqpoint{3.083573in}{2.753421in}}%
\pgfpathcurveto{\pgfqpoint{3.083573in}{2.764471in}}{\pgfqpoint{3.079182in}{2.775070in}}{\pgfqpoint{3.071369in}{2.782884in}}%
\pgfpathcurveto{\pgfqpoint{3.063555in}{2.790697in}}{\pgfqpoint{3.052956in}{2.795088in}}{\pgfqpoint{3.041906in}{2.795088in}}%
\pgfpathcurveto{\pgfqpoint{3.030856in}{2.795088in}}{\pgfqpoint{3.020257in}{2.790697in}}{\pgfqpoint{3.012443in}{2.782884in}}%
\pgfpathcurveto{\pgfqpoint{3.004630in}{2.775070in}}{\pgfqpoint{3.000239in}{2.764471in}}{\pgfqpoint{3.000239in}{2.753421in}}%
\pgfpathcurveto{\pgfqpoint{3.000239in}{2.742371in}}{\pgfqpoint{3.004630in}{2.731772in}}{\pgfqpoint{3.012443in}{2.723958in}}%
\pgfpathcurveto{\pgfqpoint{3.020257in}{2.716144in}}{\pgfqpoint{3.030856in}{2.711754in}}{\pgfqpoint{3.041906in}{2.711754in}}%
\pgfpathclose%
\pgfusepath{stroke,fill}%
\end{pgfscope}%
\begin{pgfscope}%
\pgfpathrectangle{\pgfqpoint{0.600000in}{0.600000in}}{\pgfqpoint{3.900000in}{3.900000in}}%
\pgfusepath{clip}%
\pgfsetbuttcap%
\pgfsetroundjoin%
\definecolor{currentfill}{rgb}{0.121569,0.466667,0.705882}%
\pgfsetfillcolor{currentfill}%
\pgfsetlinewidth{1.003750pt}%
\definecolor{currentstroke}{rgb}{0.121569,0.466667,0.705882}%
\pgfsetstrokecolor{currentstroke}%
\pgfsetdash{}{0pt}%
\pgfpathmoveto{\pgfqpoint{3.044720in}{1.837483in}}%
\pgfpathcurveto{\pgfqpoint{3.055770in}{1.837483in}}{\pgfqpoint{3.066369in}{1.841873in}}{\pgfqpoint{3.074182in}{1.849687in}}%
\pgfpathcurveto{\pgfqpoint{3.081996in}{1.857500in}}{\pgfqpoint{3.086386in}{1.868100in}}{\pgfqpoint{3.086386in}{1.879150in}}%
\pgfpathcurveto{\pgfqpoint{3.086386in}{1.890200in}}{\pgfqpoint{3.081996in}{1.900799in}}{\pgfqpoint{3.074182in}{1.908612in}}%
\pgfpathcurveto{\pgfqpoint{3.066369in}{1.916426in}}{\pgfqpoint{3.055770in}{1.920816in}}{\pgfqpoint{3.044720in}{1.920816in}}%
\pgfpathcurveto{\pgfqpoint{3.033670in}{1.920816in}}{\pgfqpoint{3.023070in}{1.916426in}}{\pgfqpoint{3.015257in}{1.908612in}}%
\pgfpathcurveto{\pgfqpoint{3.007443in}{1.900799in}}{\pgfqpoint{3.003053in}{1.890200in}}{\pgfqpoint{3.003053in}{1.879150in}}%
\pgfpathcurveto{\pgfqpoint{3.003053in}{1.868100in}}{\pgfqpoint{3.007443in}{1.857500in}}{\pgfqpoint{3.015257in}{1.849687in}}%
\pgfpathcurveto{\pgfqpoint{3.023070in}{1.841873in}}{\pgfqpoint{3.033670in}{1.837483in}}{\pgfqpoint{3.044720in}{1.837483in}}%
\pgfpathclose%
\pgfusepath{stroke,fill}%
\end{pgfscope}%
\begin{pgfscope}%
\pgfpathrectangle{\pgfqpoint{0.600000in}{0.600000in}}{\pgfqpoint{3.900000in}{3.900000in}}%
\pgfusepath{clip}%
\pgfsetbuttcap%
\pgfsetroundjoin%
\definecolor{currentfill}{rgb}{0.121569,0.466667,0.705882}%
\pgfsetfillcolor{currentfill}%
\pgfsetlinewidth{1.003750pt}%
\definecolor{currentstroke}{rgb}{0.121569,0.466667,0.705882}%
\pgfsetstrokecolor{currentstroke}%
\pgfsetdash{}{0pt}%
\pgfpathmoveto{\pgfqpoint{2.975348in}{2.862046in}}%
\pgfpathcurveto{\pgfqpoint{2.986399in}{2.862046in}}{\pgfqpoint{2.996998in}{2.866436in}}{\pgfqpoint{3.004811in}{2.874250in}}%
\pgfpathcurveto{\pgfqpoint{3.012625in}{2.882063in}}{\pgfqpoint{3.017015in}{2.892662in}}{\pgfqpoint{3.017015in}{2.903712in}}%
\pgfpathcurveto{\pgfqpoint{3.017015in}{2.914762in}}{\pgfqpoint{3.012625in}{2.925361in}}{\pgfqpoint{3.004811in}{2.933175in}}%
\pgfpathcurveto{\pgfqpoint{2.996998in}{2.940989in}}{\pgfqpoint{2.986399in}{2.945379in}}{\pgfqpoint{2.975348in}{2.945379in}}%
\pgfpathcurveto{\pgfqpoint{2.964298in}{2.945379in}}{\pgfqpoint{2.953699in}{2.940989in}}{\pgfqpoint{2.945886in}{2.933175in}}%
\pgfpathcurveto{\pgfqpoint{2.938072in}{2.925361in}}{\pgfqpoint{2.933682in}{2.914762in}}{\pgfqpoint{2.933682in}{2.903712in}}%
\pgfpathcurveto{\pgfqpoint{2.933682in}{2.892662in}}{\pgfqpoint{2.938072in}{2.882063in}}{\pgfqpoint{2.945886in}{2.874250in}}%
\pgfpathcurveto{\pgfqpoint{2.953699in}{2.866436in}}{\pgfqpoint{2.964298in}{2.862046in}}{\pgfqpoint{2.975348in}{2.862046in}}%
\pgfpathclose%
\pgfusepath{stroke,fill}%
\end{pgfscope}%
\begin{pgfscope}%
\pgfpathrectangle{\pgfqpoint{0.600000in}{0.600000in}}{\pgfqpoint{3.900000in}{3.900000in}}%
\pgfusepath{clip}%
\pgfsetbuttcap%
\pgfsetroundjoin%
\definecolor{currentfill}{rgb}{0.121569,0.466667,0.705882}%
\pgfsetfillcolor{currentfill}%
\pgfsetlinewidth{1.003750pt}%
\definecolor{currentstroke}{rgb}{0.121569,0.466667,0.705882}%
\pgfsetstrokecolor{currentstroke}%
\pgfsetdash{}{0pt}%
\pgfpathmoveto{\pgfqpoint{3.213294in}{3.243818in}}%
\pgfpathcurveto{\pgfqpoint{3.224344in}{3.243818in}}{\pgfqpoint{3.234943in}{3.248209in}}{\pgfqpoint{3.242756in}{3.256022in}}%
\pgfpathcurveto{\pgfqpoint{3.250570in}{3.263836in}}{\pgfqpoint{3.254960in}{3.274435in}}{\pgfqpoint{3.254960in}{3.285485in}}%
\pgfpathcurveto{\pgfqpoint{3.254960in}{3.296535in}}{\pgfqpoint{3.250570in}{3.307134in}}{\pgfqpoint{3.242756in}{3.314948in}}%
\pgfpathcurveto{\pgfqpoint{3.234943in}{3.322761in}}{\pgfqpoint{3.224344in}{3.327152in}}{\pgfqpoint{3.213294in}{3.327152in}}%
\pgfpathcurveto{\pgfqpoint{3.202243in}{3.327152in}}{\pgfqpoint{3.191644in}{3.322761in}}{\pgfqpoint{3.183831in}{3.314948in}}%
\pgfpathcurveto{\pgfqpoint{3.176017in}{3.307134in}}{\pgfqpoint{3.171627in}{3.296535in}}{\pgfqpoint{3.171627in}{3.285485in}}%
\pgfpathcurveto{\pgfqpoint{3.171627in}{3.274435in}}{\pgfqpoint{3.176017in}{3.263836in}}{\pgfqpoint{3.183831in}{3.256022in}}%
\pgfpathcurveto{\pgfqpoint{3.191644in}{3.248209in}}{\pgfqpoint{3.202243in}{3.243818in}}{\pgfqpoint{3.213294in}{3.243818in}}%
\pgfpathclose%
\pgfusepath{stroke,fill}%
\end{pgfscope}%
\begin{pgfscope}%
\pgfpathrectangle{\pgfqpoint{0.600000in}{0.600000in}}{\pgfqpoint{3.900000in}{3.900000in}}%
\pgfusepath{clip}%
\pgfsetbuttcap%
\pgfsetroundjoin%
\definecolor{currentfill}{rgb}{0.121569,0.466667,0.705882}%
\pgfsetfillcolor{currentfill}%
\pgfsetlinewidth{1.003750pt}%
\definecolor{currentstroke}{rgb}{0.121569,0.466667,0.705882}%
\pgfsetstrokecolor{currentstroke}%
\pgfsetdash{}{0pt}%
\pgfpathmoveto{\pgfqpoint{2.809758in}{2.455203in}}%
\pgfpathcurveto{\pgfqpoint{2.820808in}{2.455203in}}{\pgfqpoint{2.831407in}{2.459593in}}{\pgfqpoint{2.839221in}{2.467407in}}%
\pgfpathcurveto{\pgfqpoint{2.847034in}{2.475221in}}{\pgfqpoint{2.851425in}{2.485820in}}{\pgfqpoint{2.851425in}{2.496870in}}%
\pgfpathcurveto{\pgfqpoint{2.851425in}{2.507920in}}{\pgfqpoint{2.847034in}{2.518519in}}{\pgfqpoint{2.839221in}{2.526333in}}%
\pgfpathcurveto{\pgfqpoint{2.831407in}{2.534146in}}{\pgfqpoint{2.820808in}{2.538536in}}{\pgfqpoint{2.809758in}{2.538536in}}%
\pgfpathcurveto{\pgfqpoint{2.798708in}{2.538536in}}{\pgfqpoint{2.788109in}{2.534146in}}{\pgfqpoint{2.780295in}{2.526333in}}%
\pgfpathcurveto{\pgfqpoint{2.772482in}{2.518519in}}{\pgfqpoint{2.768091in}{2.507920in}}{\pgfqpoint{2.768091in}{2.496870in}}%
\pgfpathcurveto{\pgfqpoint{2.768091in}{2.485820in}}{\pgfqpoint{2.772482in}{2.475221in}}{\pgfqpoint{2.780295in}{2.467407in}}%
\pgfpathcurveto{\pgfqpoint{2.788109in}{2.459593in}}{\pgfqpoint{2.798708in}{2.455203in}}{\pgfqpoint{2.809758in}{2.455203in}}%
\pgfpathclose%
\pgfusepath{stroke,fill}%
\end{pgfscope}%
\begin{pgfscope}%
\pgfpathrectangle{\pgfqpoint{0.600000in}{0.600000in}}{\pgfqpoint{3.900000in}{3.900000in}}%
\pgfusepath{clip}%
\pgfsetbuttcap%
\pgfsetroundjoin%
\definecolor{currentfill}{rgb}{0.121569,0.466667,0.705882}%
\pgfsetfillcolor{currentfill}%
\pgfsetlinewidth{1.003750pt}%
\definecolor{currentstroke}{rgb}{0.121569,0.466667,0.705882}%
\pgfsetstrokecolor{currentstroke}%
\pgfsetdash{}{0pt}%
\pgfpathmoveto{\pgfqpoint{3.245647in}{2.436035in}}%
\pgfpathcurveto{\pgfqpoint{3.256697in}{2.436035in}}{\pgfqpoint{3.267296in}{2.440425in}}{\pgfqpoint{3.275110in}{2.448239in}}%
\pgfpathcurveto{\pgfqpoint{3.282923in}{2.456052in}}{\pgfqpoint{3.287314in}{2.466651in}}{\pgfqpoint{3.287314in}{2.477702in}}%
\pgfpathcurveto{\pgfqpoint{3.287314in}{2.488752in}}{\pgfqpoint{3.282923in}{2.499351in}}{\pgfqpoint{3.275110in}{2.507164in}}%
\pgfpathcurveto{\pgfqpoint{3.267296in}{2.514978in}}{\pgfqpoint{3.256697in}{2.519368in}}{\pgfqpoint{3.245647in}{2.519368in}}%
\pgfpathcurveto{\pgfqpoint{3.234597in}{2.519368in}}{\pgfqpoint{3.223998in}{2.514978in}}{\pgfqpoint{3.216184in}{2.507164in}}%
\pgfpathcurveto{\pgfqpoint{3.208371in}{2.499351in}}{\pgfqpoint{3.203980in}{2.488752in}}{\pgfqpoint{3.203980in}{2.477702in}}%
\pgfpathcurveto{\pgfqpoint{3.203980in}{2.466651in}}{\pgfqpoint{3.208371in}{2.456052in}}{\pgfqpoint{3.216184in}{2.448239in}}%
\pgfpathcurveto{\pgfqpoint{3.223998in}{2.440425in}}{\pgfqpoint{3.234597in}{2.436035in}}{\pgfqpoint{3.245647in}{2.436035in}}%
\pgfpathclose%
\pgfusepath{stroke,fill}%
\end{pgfscope}%
\begin{pgfscope}%
\pgfpathrectangle{\pgfqpoint{0.600000in}{0.600000in}}{\pgfqpoint{3.900000in}{3.900000in}}%
\pgfusepath{clip}%
\pgfsetbuttcap%
\pgfsetroundjoin%
\definecolor{currentfill}{rgb}{0.121569,0.466667,0.705882}%
\pgfsetfillcolor{currentfill}%
\pgfsetlinewidth{1.003750pt}%
\definecolor{currentstroke}{rgb}{0.121569,0.466667,0.705882}%
\pgfsetstrokecolor{currentstroke}%
\pgfsetdash{}{0pt}%
\pgfpathmoveto{\pgfqpoint{2.073707in}{2.715606in}}%
\pgfpathcurveto{\pgfqpoint{2.084757in}{2.715606in}}{\pgfqpoint{2.095356in}{2.719996in}}{\pgfqpoint{2.103170in}{2.727810in}}%
\pgfpathcurveto{\pgfqpoint{2.110984in}{2.735624in}}{\pgfqpoint{2.115374in}{2.746223in}}{\pgfqpoint{2.115374in}{2.757273in}}%
\pgfpathcurveto{\pgfqpoint{2.115374in}{2.768323in}}{\pgfqpoint{2.110984in}{2.778922in}}{\pgfqpoint{2.103170in}{2.786736in}}%
\pgfpathcurveto{\pgfqpoint{2.095356in}{2.794549in}}{\pgfqpoint{2.084757in}{2.798939in}}{\pgfqpoint{2.073707in}{2.798939in}}%
\pgfpathcurveto{\pgfqpoint{2.062657in}{2.798939in}}{\pgfqpoint{2.052058in}{2.794549in}}{\pgfqpoint{2.044244in}{2.786736in}}%
\pgfpathcurveto{\pgfqpoint{2.036431in}{2.778922in}}{\pgfqpoint{2.032041in}{2.768323in}}{\pgfqpoint{2.032041in}{2.757273in}}%
\pgfpathcurveto{\pgfqpoint{2.032041in}{2.746223in}}{\pgfqpoint{2.036431in}{2.735624in}}{\pgfqpoint{2.044244in}{2.727810in}}%
\pgfpathcurveto{\pgfqpoint{2.052058in}{2.719996in}}{\pgfqpoint{2.062657in}{2.715606in}}{\pgfqpoint{2.073707in}{2.715606in}}%
\pgfpathclose%
\pgfusepath{stroke,fill}%
\end{pgfscope}%
\begin{pgfscope}%
\pgfpathrectangle{\pgfqpoint{0.600000in}{0.600000in}}{\pgfqpoint{3.900000in}{3.900000in}}%
\pgfusepath{clip}%
\pgfsetbuttcap%
\pgfsetroundjoin%
\definecolor{currentfill}{rgb}{0.121569,0.466667,0.705882}%
\pgfsetfillcolor{currentfill}%
\pgfsetlinewidth{1.003750pt}%
\definecolor{currentstroke}{rgb}{0.121569,0.466667,0.705882}%
\pgfsetstrokecolor{currentstroke}%
\pgfsetdash{}{0pt}%
\pgfpathmoveto{\pgfqpoint{2.570371in}{2.460047in}}%
\pgfpathcurveto{\pgfqpoint{2.581421in}{2.460047in}}{\pgfqpoint{2.592020in}{2.464437in}}{\pgfqpoint{2.599834in}{2.472251in}}%
\pgfpathcurveto{\pgfqpoint{2.607647in}{2.480065in}}{\pgfqpoint{2.612038in}{2.490664in}}{\pgfqpoint{2.612038in}{2.501714in}}%
\pgfpathcurveto{\pgfqpoint{2.612038in}{2.512764in}}{\pgfqpoint{2.607647in}{2.523363in}}{\pgfqpoint{2.599834in}{2.531177in}}%
\pgfpathcurveto{\pgfqpoint{2.592020in}{2.538990in}}{\pgfqpoint{2.581421in}{2.543381in}}{\pgfqpoint{2.570371in}{2.543381in}}%
\pgfpathcurveto{\pgfqpoint{2.559321in}{2.543381in}}{\pgfqpoint{2.548722in}{2.538990in}}{\pgfqpoint{2.540908in}{2.531177in}}%
\pgfpathcurveto{\pgfqpoint{2.533095in}{2.523363in}}{\pgfqpoint{2.528704in}{2.512764in}}{\pgfqpoint{2.528704in}{2.501714in}}%
\pgfpathcurveto{\pgfqpoint{2.528704in}{2.490664in}}{\pgfqpoint{2.533095in}{2.480065in}}{\pgfqpoint{2.540908in}{2.472251in}}%
\pgfpathcurveto{\pgfqpoint{2.548722in}{2.464437in}}{\pgfqpoint{2.559321in}{2.460047in}}{\pgfqpoint{2.570371in}{2.460047in}}%
\pgfpathclose%
\pgfusepath{stroke,fill}%
\end{pgfscope}%
\begin{pgfscope}%
\pgfpathrectangle{\pgfqpoint{0.600000in}{0.600000in}}{\pgfqpoint{3.900000in}{3.900000in}}%
\pgfusepath{clip}%
\pgfsetbuttcap%
\pgfsetroundjoin%
\definecolor{currentfill}{rgb}{0.121569,0.466667,0.705882}%
\pgfsetfillcolor{currentfill}%
\pgfsetlinewidth{1.003750pt}%
\definecolor{currentstroke}{rgb}{0.121569,0.466667,0.705882}%
\pgfsetstrokecolor{currentstroke}%
\pgfsetdash{}{0pt}%
\pgfpathmoveto{\pgfqpoint{3.004173in}{2.419274in}}%
\pgfpathcurveto{\pgfqpoint{3.015223in}{2.419274in}}{\pgfqpoint{3.025822in}{2.423665in}}{\pgfqpoint{3.033636in}{2.431478in}}%
\pgfpathcurveto{\pgfqpoint{3.041449in}{2.439292in}}{\pgfqpoint{3.045840in}{2.449891in}}{\pgfqpoint{3.045840in}{2.460941in}}%
\pgfpathcurveto{\pgfqpoint{3.045840in}{2.471991in}}{\pgfqpoint{3.041449in}{2.482590in}}{\pgfqpoint{3.033636in}{2.490404in}}%
\pgfpathcurveto{\pgfqpoint{3.025822in}{2.498217in}}{\pgfqpoint{3.015223in}{2.502608in}}{\pgfqpoint{3.004173in}{2.502608in}}%
\pgfpathcurveto{\pgfqpoint{2.993123in}{2.502608in}}{\pgfqpoint{2.982524in}{2.498217in}}{\pgfqpoint{2.974710in}{2.490404in}}%
\pgfpathcurveto{\pgfqpoint{2.966896in}{2.482590in}}{\pgfqpoint{2.962506in}{2.471991in}}{\pgfqpoint{2.962506in}{2.460941in}}%
\pgfpathcurveto{\pgfqpoint{2.962506in}{2.449891in}}{\pgfqpoint{2.966896in}{2.439292in}}{\pgfqpoint{2.974710in}{2.431478in}}%
\pgfpathcurveto{\pgfqpoint{2.982524in}{2.423665in}}{\pgfqpoint{2.993123in}{2.419274in}}{\pgfqpoint{3.004173in}{2.419274in}}%
\pgfpathclose%
\pgfusepath{stroke,fill}%
\end{pgfscope}%
\begin{pgfscope}%
\pgfpathrectangle{\pgfqpoint{0.600000in}{0.600000in}}{\pgfqpoint{3.900000in}{3.900000in}}%
\pgfusepath{clip}%
\pgfsetbuttcap%
\pgfsetroundjoin%
\definecolor{currentfill}{rgb}{0.121569,0.466667,0.705882}%
\pgfsetfillcolor{currentfill}%
\pgfsetlinewidth{1.003750pt}%
\definecolor{currentstroke}{rgb}{0.121569,0.466667,0.705882}%
\pgfsetstrokecolor{currentstroke}%
\pgfsetdash{}{0pt}%
\pgfpathmoveto{\pgfqpoint{1.743811in}{1.973818in}}%
\pgfpathcurveto{\pgfqpoint{1.754861in}{1.973818in}}{\pgfqpoint{1.765460in}{1.978208in}}{\pgfqpoint{1.773273in}{1.986022in}}%
\pgfpathcurveto{\pgfqpoint{1.781087in}{1.993836in}}{\pgfqpoint{1.785477in}{2.004435in}}{\pgfqpoint{1.785477in}{2.015485in}}%
\pgfpathcurveto{\pgfqpoint{1.785477in}{2.026535in}}{\pgfqpoint{1.781087in}{2.037134in}}{\pgfqpoint{1.773273in}{2.044948in}}%
\pgfpathcurveto{\pgfqpoint{1.765460in}{2.052761in}}{\pgfqpoint{1.754861in}{2.057152in}}{\pgfqpoint{1.743811in}{2.057152in}}%
\pgfpathcurveto{\pgfqpoint{1.732761in}{2.057152in}}{\pgfqpoint{1.722161in}{2.052761in}}{\pgfqpoint{1.714348in}{2.044948in}}%
\pgfpathcurveto{\pgfqpoint{1.706534in}{2.037134in}}{\pgfqpoint{1.702144in}{2.026535in}}{\pgfqpoint{1.702144in}{2.015485in}}%
\pgfpathcurveto{\pgfqpoint{1.702144in}{2.004435in}}{\pgfqpoint{1.706534in}{1.993836in}}{\pgfqpoint{1.714348in}{1.986022in}}%
\pgfpathcurveto{\pgfqpoint{1.722161in}{1.978208in}}{\pgfqpoint{1.732761in}{1.973818in}}{\pgfqpoint{1.743811in}{1.973818in}}%
\pgfpathclose%
\pgfusepath{stroke,fill}%
\end{pgfscope}%
\begin{pgfscope}%
\pgfpathrectangle{\pgfqpoint{0.600000in}{0.600000in}}{\pgfqpoint{3.900000in}{3.900000in}}%
\pgfusepath{clip}%
\pgfsetbuttcap%
\pgfsetroundjoin%
\definecolor{currentfill}{rgb}{0.121569,0.466667,0.705882}%
\pgfsetfillcolor{currentfill}%
\pgfsetlinewidth{1.003750pt}%
\definecolor{currentstroke}{rgb}{0.121569,0.466667,0.705882}%
\pgfsetstrokecolor{currentstroke}%
\pgfsetdash{}{0pt}%
\pgfpathmoveto{\pgfqpoint{2.552735in}{2.114780in}}%
\pgfpathcurveto{\pgfqpoint{2.563785in}{2.114780in}}{\pgfqpoint{2.574384in}{2.119170in}}{\pgfqpoint{2.582198in}{2.126983in}}%
\pgfpathcurveto{\pgfqpoint{2.590012in}{2.134797in}}{\pgfqpoint{2.594402in}{2.145396in}}{\pgfqpoint{2.594402in}{2.156446in}}%
\pgfpathcurveto{\pgfqpoint{2.594402in}{2.167496in}}{\pgfqpoint{2.590012in}{2.178095in}}{\pgfqpoint{2.582198in}{2.185909in}}%
\pgfpathcurveto{\pgfqpoint{2.574384in}{2.193723in}}{\pgfqpoint{2.563785in}{2.198113in}}{\pgfqpoint{2.552735in}{2.198113in}}%
\pgfpathcurveto{\pgfqpoint{2.541685in}{2.198113in}}{\pgfqpoint{2.531086in}{2.193723in}}{\pgfqpoint{2.523272in}{2.185909in}}%
\pgfpathcurveto{\pgfqpoint{2.515459in}{2.178095in}}{\pgfqpoint{2.511068in}{2.167496in}}{\pgfqpoint{2.511068in}{2.156446in}}%
\pgfpathcurveto{\pgfqpoint{2.511068in}{2.145396in}}{\pgfqpoint{2.515459in}{2.134797in}}{\pgfqpoint{2.523272in}{2.126983in}}%
\pgfpathcurveto{\pgfqpoint{2.531086in}{2.119170in}}{\pgfqpoint{2.541685in}{2.114780in}}{\pgfqpoint{2.552735in}{2.114780in}}%
\pgfpathclose%
\pgfusepath{stroke,fill}%
\end{pgfscope}%
\begin{pgfscope}%
\pgfpathrectangle{\pgfqpoint{0.600000in}{0.600000in}}{\pgfqpoint{3.900000in}{3.900000in}}%
\pgfusepath{clip}%
\pgfsetbuttcap%
\pgfsetroundjoin%
\definecolor{currentfill}{rgb}{0.121569,0.466667,0.705882}%
\pgfsetfillcolor{currentfill}%
\pgfsetlinewidth{1.003750pt}%
\definecolor{currentstroke}{rgb}{0.121569,0.466667,0.705882}%
\pgfsetstrokecolor{currentstroke}%
\pgfsetdash{}{0pt}%
\pgfpathmoveto{\pgfqpoint{1.577778in}{2.247640in}}%
\pgfpathcurveto{\pgfqpoint{1.588828in}{2.247640in}}{\pgfqpoint{1.599427in}{2.252031in}}{\pgfqpoint{1.607241in}{2.259844in}}%
\pgfpathcurveto{\pgfqpoint{1.615055in}{2.267658in}}{\pgfqpoint{1.619445in}{2.278257in}}{\pgfqpoint{1.619445in}{2.289307in}}%
\pgfpathcurveto{\pgfqpoint{1.619445in}{2.300357in}}{\pgfqpoint{1.615055in}{2.310956in}}{\pgfqpoint{1.607241in}{2.318770in}}%
\pgfpathcurveto{\pgfqpoint{1.599427in}{2.326583in}}{\pgfqpoint{1.588828in}{2.330974in}}{\pgfqpoint{1.577778in}{2.330974in}}%
\pgfpathcurveto{\pgfqpoint{1.566728in}{2.330974in}}{\pgfqpoint{1.556129in}{2.326583in}}{\pgfqpoint{1.548315in}{2.318770in}}%
\pgfpathcurveto{\pgfqpoint{1.540502in}{2.310956in}}{\pgfqpoint{1.536111in}{2.300357in}}{\pgfqpoint{1.536111in}{2.289307in}}%
\pgfpathcurveto{\pgfqpoint{1.536111in}{2.278257in}}{\pgfqpoint{1.540502in}{2.267658in}}{\pgfqpoint{1.548315in}{2.259844in}}%
\pgfpathcurveto{\pgfqpoint{1.556129in}{2.252031in}}{\pgfqpoint{1.566728in}{2.247640in}}{\pgfqpoint{1.577778in}{2.247640in}}%
\pgfpathclose%
\pgfusepath{stroke,fill}%
\end{pgfscope}%
\begin{pgfscope}%
\pgfpathrectangle{\pgfqpoint{0.600000in}{0.600000in}}{\pgfqpoint{3.900000in}{3.900000in}}%
\pgfusepath{clip}%
\pgfsetbuttcap%
\pgfsetroundjoin%
\definecolor{currentfill}{rgb}{0.121569,0.466667,0.705882}%
\pgfsetfillcolor{currentfill}%
\pgfsetlinewidth{1.003750pt}%
\definecolor{currentstroke}{rgb}{0.121569,0.466667,0.705882}%
\pgfsetstrokecolor{currentstroke}%
\pgfsetdash{}{0pt}%
\pgfpathmoveto{\pgfqpoint{2.702710in}{2.159191in}}%
\pgfpathcurveto{\pgfqpoint{2.713760in}{2.159191in}}{\pgfqpoint{2.724359in}{2.163582in}}{\pgfqpoint{2.732173in}{2.171395in}}%
\pgfpathcurveto{\pgfqpoint{2.739987in}{2.179209in}}{\pgfqpoint{2.744377in}{2.189808in}}{\pgfqpoint{2.744377in}{2.200858in}}%
\pgfpathcurveto{\pgfqpoint{2.744377in}{2.211908in}}{\pgfqpoint{2.739987in}{2.222507in}}{\pgfqpoint{2.732173in}{2.230321in}}%
\pgfpathcurveto{\pgfqpoint{2.724359in}{2.238134in}}{\pgfqpoint{2.713760in}{2.242525in}}{\pgfqpoint{2.702710in}{2.242525in}}%
\pgfpathcurveto{\pgfqpoint{2.691660in}{2.242525in}}{\pgfqpoint{2.681061in}{2.238134in}}{\pgfqpoint{2.673247in}{2.230321in}}%
\pgfpathcurveto{\pgfqpoint{2.665434in}{2.222507in}}{\pgfqpoint{2.661043in}{2.211908in}}{\pgfqpoint{2.661043in}{2.200858in}}%
\pgfpathcurveto{\pgfqpoint{2.661043in}{2.189808in}}{\pgfqpoint{2.665434in}{2.179209in}}{\pgfqpoint{2.673247in}{2.171395in}}%
\pgfpathcurveto{\pgfqpoint{2.681061in}{2.163582in}}{\pgfqpoint{2.691660in}{2.159191in}}{\pgfqpoint{2.702710in}{2.159191in}}%
\pgfpathclose%
\pgfusepath{stroke,fill}%
\end{pgfscope}%
\begin{pgfscope}%
\pgfpathrectangle{\pgfqpoint{0.600000in}{0.600000in}}{\pgfqpoint{3.900000in}{3.900000in}}%
\pgfusepath{clip}%
\pgfsetbuttcap%
\pgfsetroundjoin%
\definecolor{currentfill}{rgb}{0.121569,0.466667,0.705882}%
\pgfsetfillcolor{currentfill}%
\pgfsetlinewidth{1.003750pt}%
\definecolor{currentstroke}{rgb}{0.121569,0.466667,0.705882}%
\pgfsetstrokecolor{currentstroke}%
\pgfsetdash{}{0pt}%
\pgfpathmoveto{\pgfqpoint{1.941093in}{2.365416in}}%
\pgfpathcurveto{\pgfqpoint{1.952144in}{2.365416in}}{\pgfqpoint{1.962743in}{2.369806in}}{\pgfqpoint{1.970556in}{2.377620in}}%
\pgfpathcurveto{\pgfqpoint{1.978370in}{2.385434in}}{\pgfqpoint{1.982760in}{2.396033in}}{\pgfqpoint{1.982760in}{2.407083in}}%
\pgfpathcurveto{\pgfqpoint{1.982760in}{2.418133in}}{\pgfqpoint{1.978370in}{2.428732in}}{\pgfqpoint{1.970556in}{2.436545in}}%
\pgfpathcurveto{\pgfqpoint{1.962743in}{2.444359in}}{\pgfqpoint{1.952144in}{2.448749in}}{\pgfqpoint{1.941093in}{2.448749in}}%
\pgfpathcurveto{\pgfqpoint{1.930043in}{2.448749in}}{\pgfqpoint{1.919444in}{2.444359in}}{\pgfqpoint{1.911631in}{2.436545in}}%
\pgfpathcurveto{\pgfqpoint{1.903817in}{2.428732in}}{\pgfqpoint{1.899427in}{2.418133in}}{\pgfqpoint{1.899427in}{2.407083in}}%
\pgfpathcurveto{\pgfqpoint{1.899427in}{2.396033in}}{\pgfqpoint{1.903817in}{2.385434in}}{\pgfqpoint{1.911631in}{2.377620in}}%
\pgfpathcurveto{\pgfqpoint{1.919444in}{2.369806in}}{\pgfqpoint{1.930043in}{2.365416in}}{\pgfqpoint{1.941093in}{2.365416in}}%
\pgfpathclose%
\pgfusepath{stroke,fill}%
\end{pgfscope}%
\begin{pgfscope}%
\pgfpathrectangle{\pgfqpoint{0.600000in}{0.600000in}}{\pgfqpoint{3.900000in}{3.900000in}}%
\pgfusepath{clip}%
\pgfsetbuttcap%
\pgfsetroundjoin%
\definecolor{currentfill}{rgb}{0.121569,0.466667,0.705882}%
\pgfsetfillcolor{currentfill}%
\pgfsetlinewidth{1.003750pt}%
\definecolor{currentstroke}{rgb}{0.121569,0.466667,0.705882}%
\pgfsetstrokecolor{currentstroke}%
\pgfsetdash{}{0pt}%
\pgfpathmoveto{\pgfqpoint{2.206998in}{3.404552in}}%
\pgfpathcurveto{\pgfqpoint{2.218048in}{3.404552in}}{\pgfqpoint{2.228647in}{3.408943in}}{\pgfqpoint{2.236460in}{3.416756in}}%
\pgfpathcurveto{\pgfqpoint{2.244274in}{3.424570in}}{\pgfqpoint{2.248664in}{3.435169in}}{\pgfqpoint{2.248664in}{3.446219in}}%
\pgfpathcurveto{\pgfqpoint{2.248664in}{3.457269in}}{\pgfqpoint{2.244274in}{3.467868in}}{\pgfqpoint{2.236460in}{3.475682in}}%
\pgfpathcurveto{\pgfqpoint{2.228647in}{3.483495in}}{\pgfqpoint{2.218048in}{3.487886in}}{\pgfqpoint{2.206998in}{3.487886in}}%
\pgfpathcurveto{\pgfqpoint{2.195947in}{3.487886in}}{\pgfqpoint{2.185348in}{3.483495in}}{\pgfqpoint{2.177535in}{3.475682in}}%
\pgfpathcurveto{\pgfqpoint{2.169721in}{3.467868in}}{\pgfqpoint{2.165331in}{3.457269in}}{\pgfqpoint{2.165331in}{3.446219in}}%
\pgfpathcurveto{\pgfqpoint{2.165331in}{3.435169in}}{\pgfqpoint{2.169721in}{3.424570in}}{\pgfqpoint{2.177535in}{3.416756in}}%
\pgfpathcurveto{\pgfqpoint{2.185348in}{3.408943in}}{\pgfqpoint{2.195947in}{3.404552in}}{\pgfqpoint{2.206998in}{3.404552in}}%
\pgfpathclose%
\pgfusepath{stroke,fill}%
\end{pgfscope}%
\begin{pgfscope}%
\pgfpathrectangle{\pgfqpoint{0.600000in}{0.600000in}}{\pgfqpoint{3.900000in}{3.900000in}}%
\pgfusepath{clip}%
\pgfsetbuttcap%
\pgfsetroundjoin%
\definecolor{currentfill}{rgb}{0.121569,0.466667,0.705882}%
\pgfsetfillcolor{currentfill}%
\pgfsetlinewidth{1.003750pt}%
\definecolor{currentstroke}{rgb}{0.121569,0.466667,0.705882}%
\pgfsetstrokecolor{currentstroke}%
\pgfsetdash{}{0pt}%
\pgfpathmoveto{\pgfqpoint{2.802644in}{2.705319in}}%
\pgfpathcurveto{\pgfqpoint{2.813694in}{2.705319in}}{\pgfqpoint{2.824293in}{2.709709in}}{\pgfqpoint{2.832107in}{2.717523in}}%
\pgfpathcurveto{\pgfqpoint{2.839920in}{2.725336in}}{\pgfqpoint{2.844311in}{2.735935in}}{\pgfqpoint{2.844311in}{2.746985in}}%
\pgfpathcurveto{\pgfqpoint{2.844311in}{2.758035in}}{\pgfqpoint{2.839920in}{2.768634in}}{\pgfqpoint{2.832107in}{2.776448in}}%
\pgfpathcurveto{\pgfqpoint{2.824293in}{2.784262in}}{\pgfqpoint{2.813694in}{2.788652in}}{\pgfqpoint{2.802644in}{2.788652in}}%
\pgfpathcurveto{\pgfqpoint{2.791594in}{2.788652in}}{\pgfqpoint{2.780995in}{2.784262in}}{\pgfqpoint{2.773181in}{2.776448in}}%
\pgfpathcurveto{\pgfqpoint{2.765367in}{2.768634in}}{\pgfqpoint{2.760977in}{2.758035in}}{\pgfqpoint{2.760977in}{2.746985in}}%
\pgfpathcurveto{\pgfqpoint{2.760977in}{2.735935in}}{\pgfqpoint{2.765367in}{2.725336in}}{\pgfqpoint{2.773181in}{2.717523in}}%
\pgfpathcurveto{\pgfqpoint{2.780995in}{2.709709in}}{\pgfqpoint{2.791594in}{2.705319in}}{\pgfqpoint{2.802644in}{2.705319in}}%
\pgfpathclose%
\pgfusepath{stroke,fill}%
\end{pgfscope}%
\begin{pgfscope}%
\pgfpathrectangle{\pgfqpoint{0.600000in}{0.600000in}}{\pgfqpoint{3.900000in}{3.900000in}}%
\pgfusepath{clip}%
\pgfsetbuttcap%
\pgfsetroundjoin%
\definecolor{currentfill}{rgb}{0.121569,0.466667,0.705882}%
\pgfsetfillcolor{currentfill}%
\pgfsetlinewidth{1.003750pt}%
\definecolor{currentstroke}{rgb}{0.121569,0.466667,0.705882}%
\pgfsetstrokecolor{currentstroke}%
\pgfsetdash{}{0pt}%
\pgfpathmoveto{\pgfqpoint{2.329074in}{2.474279in}}%
\pgfpathcurveto{\pgfqpoint{2.340124in}{2.474279in}}{\pgfqpoint{2.350723in}{2.478669in}}{\pgfqpoint{2.358537in}{2.486482in}}%
\pgfpathcurveto{\pgfqpoint{2.366351in}{2.494296in}}{\pgfqpoint{2.370741in}{2.504895in}}{\pgfqpoint{2.370741in}{2.515945in}}%
\pgfpathcurveto{\pgfqpoint{2.370741in}{2.526995in}}{\pgfqpoint{2.366351in}{2.537594in}}{\pgfqpoint{2.358537in}{2.545408in}}%
\pgfpathcurveto{\pgfqpoint{2.350723in}{2.553222in}}{\pgfqpoint{2.340124in}{2.557612in}}{\pgfqpoint{2.329074in}{2.557612in}}%
\pgfpathcurveto{\pgfqpoint{2.318024in}{2.557612in}}{\pgfqpoint{2.307425in}{2.553222in}}{\pgfqpoint{2.299611in}{2.545408in}}%
\pgfpathcurveto{\pgfqpoint{2.291798in}{2.537594in}}{\pgfqpoint{2.287408in}{2.526995in}}{\pgfqpoint{2.287408in}{2.515945in}}%
\pgfpathcurveto{\pgfqpoint{2.287408in}{2.504895in}}{\pgfqpoint{2.291798in}{2.494296in}}{\pgfqpoint{2.299611in}{2.486482in}}%
\pgfpathcurveto{\pgfqpoint{2.307425in}{2.478669in}}{\pgfqpoint{2.318024in}{2.474279in}}{\pgfqpoint{2.329074in}{2.474279in}}%
\pgfpathclose%
\pgfusepath{stroke,fill}%
\end{pgfscope}%
\begin{pgfscope}%
\pgfpathrectangle{\pgfqpoint{0.600000in}{0.600000in}}{\pgfqpoint{3.900000in}{3.900000in}}%
\pgfusepath{clip}%
\pgfsetbuttcap%
\pgfsetroundjoin%
\definecolor{currentfill}{rgb}{0.121569,0.466667,0.705882}%
\pgfsetfillcolor{currentfill}%
\pgfsetlinewidth{1.003750pt}%
\definecolor{currentstroke}{rgb}{0.121569,0.466667,0.705882}%
\pgfsetstrokecolor{currentstroke}%
\pgfsetdash{}{0pt}%
\pgfpathmoveto{\pgfqpoint{2.244834in}{2.856864in}}%
\pgfpathcurveto{\pgfqpoint{2.255884in}{2.856864in}}{\pgfqpoint{2.266483in}{2.861255in}}{\pgfqpoint{2.274297in}{2.869068in}}%
\pgfpathcurveto{\pgfqpoint{2.282110in}{2.876882in}}{\pgfqpoint{2.286500in}{2.887481in}}{\pgfqpoint{2.286500in}{2.898531in}}%
\pgfpathcurveto{\pgfqpoint{2.286500in}{2.909581in}}{\pgfqpoint{2.282110in}{2.920180in}}{\pgfqpoint{2.274297in}{2.927994in}}%
\pgfpathcurveto{\pgfqpoint{2.266483in}{2.935807in}}{\pgfqpoint{2.255884in}{2.940198in}}{\pgfqpoint{2.244834in}{2.940198in}}%
\pgfpathcurveto{\pgfqpoint{2.233784in}{2.940198in}}{\pgfqpoint{2.223185in}{2.935807in}}{\pgfqpoint{2.215371in}{2.927994in}}%
\pgfpathcurveto{\pgfqpoint{2.207557in}{2.920180in}}{\pgfqpoint{2.203167in}{2.909581in}}{\pgfqpoint{2.203167in}{2.898531in}}%
\pgfpathcurveto{\pgfqpoint{2.203167in}{2.887481in}}{\pgfqpoint{2.207557in}{2.876882in}}{\pgfqpoint{2.215371in}{2.869068in}}%
\pgfpathcurveto{\pgfqpoint{2.223185in}{2.861255in}}{\pgfqpoint{2.233784in}{2.856864in}}{\pgfqpoint{2.244834in}{2.856864in}}%
\pgfpathclose%
\pgfusepath{stroke,fill}%
\end{pgfscope}%
\begin{pgfscope}%
\pgfpathrectangle{\pgfqpoint{0.600000in}{0.600000in}}{\pgfqpoint{3.900000in}{3.900000in}}%
\pgfusepath{clip}%
\pgfsetbuttcap%
\pgfsetroundjoin%
\definecolor{currentfill}{rgb}{0.121569,0.466667,0.705882}%
\pgfsetfillcolor{currentfill}%
\pgfsetlinewidth{1.003750pt}%
\definecolor{currentstroke}{rgb}{0.121569,0.466667,0.705882}%
\pgfsetstrokecolor{currentstroke}%
\pgfsetdash{}{0pt}%
\pgfpathmoveto{\pgfqpoint{2.409908in}{2.610766in}}%
\pgfpathcurveto{\pgfqpoint{2.420958in}{2.610766in}}{\pgfqpoint{2.431558in}{2.615156in}}{\pgfqpoint{2.439371in}{2.622970in}}%
\pgfpathcurveto{\pgfqpoint{2.447185in}{2.630783in}}{\pgfqpoint{2.451575in}{2.641382in}}{\pgfqpoint{2.451575in}{2.652433in}}%
\pgfpathcurveto{\pgfqpoint{2.451575in}{2.663483in}}{\pgfqpoint{2.447185in}{2.674082in}}{\pgfqpoint{2.439371in}{2.681895in}}%
\pgfpathcurveto{\pgfqpoint{2.431558in}{2.689709in}}{\pgfqpoint{2.420958in}{2.694099in}}{\pgfqpoint{2.409908in}{2.694099in}}%
\pgfpathcurveto{\pgfqpoint{2.398858in}{2.694099in}}{\pgfqpoint{2.388259in}{2.689709in}}{\pgfqpoint{2.380446in}{2.681895in}}%
\pgfpathcurveto{\pgfqpoint{2.372632in}{2.674082in}}{\pgfqpoint{2.368242in}{2.663483in}}{\pgfqpoint{2.368242in}{2.652433in}}%
\pgfpathcurveto{\pgfqpoint{2.368242in}{2.641382in}}{\pgfqpoint{2.372632in}{2.630783in}}{\pgfqpoint{2.380446in}{2.622970in}}%
\pgfpathcurveto{\pgfqpoint{2.388259in}{2.615156in}}{\pgfqpoint{2.398858in}{2.610766in}}{\pgfqpoint{2.409908in}{2.610766in}}%
\pgfpathclose%
\pgfusepath{stroke,fill}%
\end{pgfscope}%
\begin{pgfscope}%
\pgfpathrectangle{\pgfqpoint{0.600000in}{0.600000in}}{\pgfqpoint{3.900000in}{3.900000in}}%
\pgfusepath{clip}%
\pgfsetbuttcap%
\pgfsetroundjoin%
\definecolor{currentfill}{rgb}{0.121569,0.466667,0.705882}%
\pgfsetfillcolor{currentfill}%
\pgfsetlinewidth{1.003750pt}%
\definecolor{currentstroke}{rgb}{0.121569,0.466667,0.705882}%
\pgfsetstrokecolor{currentstroke}%
\pgfsetdash{}{0pt}%
\pgfpathmoveto{\pgfqpoint{2.634243in}{2.421003in}}%
\pgfpathcurveto{\pgfqpoint{2.645293in}{2.421003in}}{\pgfqpoint{2.655892in}{2.425393in}}{\pgfqpoint{2.663706in}{2.433207in}}%
\pgfpathcurveto{\pgfqpoint{2.671519in}{2.441021in}}{\pgfqpoint{2.675910in}{2.451620in}}{\pgfqpoint{2.675910in}{2.462670in}}%
\pgfpathcurveto{\pgfqpoint{2.675910in}{2.473720in}}{\pgfqpoint{2.671519in}{2.484319in}}{\pgfqpoint{2.663706in}{2.492133in}}%
\pgfpathcurveto{\pgfqpoint{2.655892in}{2.499946in}}{\pgfqpoint{2.645293in}{2.504337in}}{\pgfqpoint{2.634243in}{2.504337in}}%
\pgfpathcurveto{\pgfqpoint{2.623193in}{2.504337in}}{\pgfqpoint{2.612594in}{2.499946in}}{\pgfqpoint{2.604780in}{2.492133in}}%
\pgfpathcurveto{\pgfqpoint{2.596967in}{2.484319in}}{\pgfqpoint{2.592576in}{2.473720in}}{\pgfqpoint{2.592576in}{2.462670in}}%
\pgfpathcurveto{\pgfqpoint{2.592576in}{2.451620in}}{\pgfqpoint{2.596967in}{2.441021in}}{\pgfqpoint{2.604780in}{2.433207in}}%
\pgfpathcurveto{\pgfqpoint{2.612594in}{2.425393in}}{\pgfqpoint{2.623193in}{2.421003in}}{\pgfqpoint{2.634243in}{2.421003in}}%
\pgfpathclose%
\pgfusepath{stroke,fill}%
\end{pgfscope}%
\begin{pgfscope}%
\pgfpathrectangle{\pgfqpoint{0.600000in}{0.600000in}}{\pgfqpoint{3.900000in}{3.900000in}}%
\pgfusepath{clip}%
\pgfsetbuttcap%
\pgfsetroundjoin%
\definecolor{currentfill}{rgb}{0.121569,0.466667,0.705882}%
\pgfsetfillcolor{currentfill}%
\pgfsetlinewidth{1.003750pt}%
\definecolor{currentstroke}{rgb}{0.121569,0.466667,0.705882}%
\pgfsetstrokecolor{currentstroke}%
\pgfsetdash{}{0pt}%
\pgfpathmoveto{\pgfqpoint{2.088997in}{1.870956in}}%
\pgfpathcurveto{\pgfqpoint{2.100047in}{1.870956in}}{\pgfqpoint{2.110646in}{1.875347in}}{\pgfqpoint{2.118460in}{1.883160in}}%
\pgfpathcurveto{\pgfqpoint{2.126273in}{1.890974in}}{\pgfqpoint{2.130664in}{1.901573in}}{\pgfqpoint{2.130664in}{1.912623in}}%
\pgfpathcurveto{\pgfqpoint{2.130664in}{1.923673in}}{\pgfqpoint{2.126273in}{1.934272in}}{\pgfqpoint{2.118460in}{1.942086in}}%
\pgfpathcurveto{\pgfqpoint{2.110646in}{1.949900in}}{\pgfqpoint{2.100047in}{1.954290in}}{\pgfqpoint{2.088997in}{1.954290in}}%
\pgfpathcurveto{\pgfqpoint{2.077947in}{1.954290in}}{\pgfqpoint{2.067348in}{1.949900in}}{\pgfqpoint{2.059534in}{1.942086in}}%
\pgfpathcurveto{\pgfqpoint{2.051721in}{1.934272in}}{\pgfqpoint{2.047330in}{1.923673in}}{\pgfqpoint{2.047330in}{1.912623in}}%
\pgfpathcurveto{\pgfqpoint{2.047330in}{1.901573in}}{\pgfqpoint{2.051721in}{1.890974in}}{\pgfqpoint{2.059534in}{1.883160in}}%
\pgfpathcurveto{\pgfqpoint{2.067348in}{1.875347in}}{\pgfqpoint{2.077947in}{1.870956in}}{\pgfqpoint{2.088997in}{1.870956in}}%
\pgfpathclose%
\pgfusepath{stroke,fill}%
\end{pgfscope}%
\begin{pgfscope}%
\pgfpathrectangle{\pgfqpoint{0.600000in}{0.600000in}}{\pgfqpoint{3.900000in}{3.900000in}}%
\pgfusepath{clip}%
\pgfsetbuttcap%
\pgfsetroundjoin%
\definecolor{currentfill}{rgb}{0.121569,0.466667,0.705882}%
\pgfsetfillcolor{currentfill}%
\pgfsetlinewidth{1.003750pt}%
\definecolor{currentstroke}{rgb}{0.121569,0.466667,0.705882}%
\pgfsetstrokecolor{currentstroke}%
\pgfsetdash{}{0pt}%
\pgfpathmoveto{\pgfqpoint{2.632426in}{2.958992in}}%
\pgfpathcurveto{\pgfqpoint{2.643476in}{2.958992in}}{\pgfqpoint{2.654075in}{2.963382in}}{\pgfqpoint{2.661889in}{2.971196in}}%
\pgfpathcurveto{\pgfqpoint{2.669702in}{2.979009in}}{\pgfqpoint{2.674093in}{2.989608in}}{\pgfqpoint{2.674093in}{3.000659in}}%
\pgfpathcurveto{\pgfqpoint{2.674093in}{3.011709in}}{\pgfqpoint{2.669702in}{3.022308in}}{\pgfqpoint{2.661889in}{3.030121in}}%
\pgfpathcurveto{\pgfqpoint{2.654075in}{3.037935in}}{\pgfqpoint{2.643476in}{3.042325in}}{\pgfqpoint{2.632426in}{3.042325in}}%
\pgfpathcurveto{\pgfqpoint{2.621376in}{3.042325in}}{\pgfqpoint{2.610777in}{3.037935in}}{\pgfqpoint{2.602963in}{3.030121in}}%
\pgfpathcurveto{\pgfqpoint{2.595150in}{3.022308in}}{\pgfqpoint{2.590759in}{3.011709in}}{\pgfqpoint{2.590759in}{3.000659in}}%
\pgfpathcurveto{\pgfqpoint{2.590759in}{2.989608in}}{\pgfqpoint{2.595150in}{2.979009in}}{\pgfqpoint{2.602963in}{2.971196in}}%
\pgfpathcurveto{\pgfqpoint{2.610777in}{2.963382in}}{\pgfqpoint{2.621376in}{2.958992in}}{\pgfqpoint{2.632426in}{2.958992in}}%
\pgfpathclose%
\pgfusepath{stroke,fill}%
\end{pgfscope}%
\begin{pgfscope}%
\pgfpathrectangle{\pgfqpoint{0.600000in}{0.600000in}}{\pgfqpoint{3.900000in}{3.900000in}}%
\pgfusepath{clip}%
\pgfsetbuttcap%
\pgfsetroundjoin%
\definecolor{currentfill}{rgb}{0.121569,0.466667,0.705882}%
\pgfsetfillcolor{currentfill}%
\pgfsetlinewidth{1.003750pt}%
\definecolor{currentstroke}{rgb}{0.121569,0.466667,0.705882}%
\pgfsetstrokecolor{currentstroke}%
\pgfsetdash{}{0pt}%
\pgfpathmoveto{\pgfqpoint{2.376777in}{2.516249in}}%
\pgfpathcurveto{\pgfqpoint{2.387827in}{2.516249in}}{\pgfqpoint{2.398426in}{2.520640in}}{\pgfqpoint{2.406239in}{2.528453in}}%
\pgfpathcurveto{\pgfqpoint{2.414053in}{2.536267in}}{\pgfqpoint{2.418443in}{2.546866in}}{\pgfqpoint{2.418443in}{2.557916in}}%
\pgfpathcurveto{\pgfqpoint{2.418443in}{2.568966in}}{\pgfqpoint{2.414053in}{2.579565in}}{\pgfqpoint{2.406239in}{2.587379in}}%
\pgfpathcurveto{\pgfqpoint{2.398426in}{2.595192in}}{\pgfqpoint{2.387827in}{2.599583in}}{\pgfqpoint{2.376777in}{2.599583in}}%
\pgfpathcurveto{\pgfqpoint{2.365726in}{2.599583in}}{\pgfqpoint{2.355127in}{2.595192in}}{\pgfqpoint{2.347314in}{2.587379in}}%
\pgfpathcurveto{\pgfqpoint{2.339500in}{2.579565in}}{\pgfqpoint{2.335110in}{2.568966in}}{\pgfqpoint{2.335110in}{2.557916in}}%
\pgfpathcurveto{\pgfqpoint{2.335110in}{2.546866in}}{\pgfqpoint{2.339500in}{2.536267in}}{\pgfqpoint{2.347314in}{2.528453in}}%
\pgfpathcurveto{\pgfqpoint{2.355127in}{2.520640in}}{\pgfqpoint{2.365726in}{2.516249in}}{\pgfqpoint{2.376777in}{2.516249in}}%
\pgfpathclose%
\pgfusepath{stroke,fill}%
\end{pgfscope}%
\begin{pgfscope}%
\pgfpathrectangle{\pgfqpoint{0.600000in}{0.600000in}}{\pgfqpoint{3.900000in}{3.900000in}}%
\pgfusepath{clip}%
\pgfsetbuttcap%
\pgfsetroundjoin%
\definecolor{currentfill}{rgb}{0.121569,0.466667,0.705882}%
\pgfsetfillcolor{currentfill}%
\pgfsetlinewidth{1.003750pt}%
\definecolor{currentstroke}{rgb}{0.121569,0.466667,0.705882}%
\pgfsetstrokecolor{currentstroke}%
\pgfsetdash{}{0pt}%
\pgfpathmoveto{\pgfqpoint{1.988460in}{3.214800in}}%
\pgfpathcurveto{\pgfqpoint{1.999510in}{3.214800in}}{\pgfqpoint{2.010109in}{3.219190in}}{\pgfqpoint{2.017923in}{3.227004in}}%
\pgfpathcurveto{\pgfqpoint{2.025736in}{3.234817in}}{\pgfqpoint{2.030127in}{3.245416in}}{\pgfqpoint{2.030127in}{3.256466in}}%
\pgfpathcurveto{\pgfqpoint{2.030127in}{3.267516in}}{\pgfqpoint{2.025736in}{3.278116in}}{\pgfqpoint{2.017923in}{3.285929in}}%
\pgfpathcurveto{\pgfqpoint{2.010109in}{3.293743in}}{\pgfqpoint{1.999510in}{3.298133in}}{\pgfqpoint{1.988460in}{3.298133in}}%
\pgfpathcurveto{\pgfqpoint{1.977410in}{3.298133in}}{\pgfqpoint{1.966811in}{3.293743in}}{\pgfqpoint{1.958997in}{3.285929in}}%
\pgfpathcurveto{\pgfqpoint{1.951184in}{3.278116in}}{\pgfqpoint{1.946793in}{3.267516in}}{\pgfqpoint{1.946793in}{3.256466in}}%
\pgfpathcurveto{\pgfqpoint{1.946793in}{3.245416in}}{\pgfqpoint{1.951184in}{3.234817in}}{\pgfqpoint{1.958997in}{3.227004in}}%
\pgfpathcurveto{\pgfqpoint{1.966811in}{3.219190in}}{\pgfqpoint{1.977410in}{3.214800in}}{\pgfqpoint{1.988460in}{3.214800in}}%
\pgfpathclose%
\pgfusepath{stroke,fill}%
\end{pgfscope}%
\begin{pgfscope}%
\pgfpathrectangle{\pgfqpoint{0.600000in}{0.600000in}}{\pgfqpoint{3.900000in}{3.900000in}}%
\pgfusepath{clip}%
\pgfsetbuttcap%
\pgfsetroundjoin%
\definecolor{currentfill}{rgb}{0.121569,0.466667,0.705882}%
\pgfsetfillcolor{currentfill}%
\pgfsetlinewidth{1.003750pt}%
\definecolor{currentstroke}{rgb}{0.121569,0.466667,0.705882}%
\pgfsetstrokecolor{currentstroke}%
\pgfsetdash{}{0pt}%
\pgfpathmoveto{\pgfqpoint{2.628410in}{1.722489in}}%
\pgfpathcurveto{\pgfqpoint{2.639460in}{1.722489in}}{\pgfqpoint{2.650059in}{1.726879in}}{\pgfqpoint{2.657873in}{1.734693in}}%
\pgfpathcurveto{\pgfqpoint{2.665687in}{1.742506in}}{\pgfqpoint{2.670077in}{1.753105in}}{\pgfqpoint{2.670077in}{1.764155in}}%
\pgfpathcurveto{\pgfqpoint{2.670077in}{1.775205in}}{\pgfqpoint{2.665687in}{1.785804in}}{\pgfqpoint{2.657873in}{1.793618in}}%
\pgfpathcurveto{\pgfqpoint{2.650059in}{1.801432in}}{\pgfqpoint{2.639460in}{1.805822in}}{\pgfqpoint{2.628410in}{1.805822in}}%
\pgfpathcurveto{\pgfqpoint{2.617360in}{1.805822in}}{\pgfqpoint{2.606761in}{1.801432in}}{\pgfqpoint{2.598947in}{1.793618in}}%
\pgfpathcurveto{\pgfqpoint{2.591134in}{1.785804in}}{\pgfqpoint{2.586744in}{1.775205in}}{\pgfqpoint{2.586744in}{1.764155in}}%
\pgfpathcurveto{\pgfqpoint{2.586744in}{1.753105in}}{\pgfqpoint{2.591134in}{1.742506in}}{\pgfqpoint{2.598947in}{1.734693in}}%
\pgfpathcurveto{\pgfqpoint{2.606761in}{1.726879in}}{\pgfqpoint{2.617360in}{1.722489in}}{\pgfqpoint{2.628410in}{1.722489in}}%
\pgfpathclose%
\pgfusepath{stroke,fill}%
\end{pgfscope}%
\begin{pgfscope}%
\pgfpathrectangle{\pgfqpoint{0.600000in}{0.600000in}}{\pgfqpoint{3.900000in}{3.900000in}}%
\pgfusepath{clip}%
\pgfsetbuttcap%
\pgfsetroundjoin%
\definecolor{currentfill}{rgb}{0.121569,0.466667,0.705882}%
\pgfsetfillcolor{currentfill}%
\pgfsetlinewidth{1.003750pt}%
\definecolor{currentstroke}{rgb}{0.121569,0.466667,0.705882}%
\pgfsetstrokecolor{currentstroke}%
\pgfsetdash{}{0pt}%
\pgfpathmoveto{\pgfqpoint{1.791413in}{2.175317in}}%
\pgfpathcurveto{\pgfqpoint{1.802463in}{2.175317in}}{\pgfqpoint{1.813062in}{2.179707in}}{\pgfqpoint{1.820876in}{2.187521in}}%
\pgfpathcurveto{\pgfqpoint{1.828690in}{2.195334in}}{\pgfqpoint{1.833080in}{2.205933in}}{\pgfqpoint{1.833080in}{2.216984in}}%
\pgfpathcurveto{\pgfqpoint{1.833080in}{2.228034in}}{\pgfqpoint{1.828690in}{2.238633in}}{\pgfqpoint{1.820876in}{2.246446in}}%
\pgfpathcurveto{\pgfqpoint{1.813062in}{2.254260in}}{\pgfqpoint{1.802463in}{2.258650in}}{\pgfqpoint{1.791413in}{2.258650in}}%
\pgfpathcurveto{\pgfqpoint{1.780363in}{2.258650in}}{\pgfqpoint{1.769764in}{2.254260in}}{\pgfqpoint{1.761950in}{2.246446in}}%
\pgfpathcurveto{\pgfqpoint{1.754137in}{2.238633in}}{\pgfqpoint{1.749746in}{2.228034in}}{\pgfqpoint{1.749746in}{2.216984in}}%
\pgfpathcurveto{\pgfqpoint{1.749746in}{2.205933in}}{\pgfqpoint{1.754137in}{2.195334in}}{\pgfqpoint{1.761950in}{2.187521in}}%
\pgfpathcurveto{\pgfqpoint{1.769764in}{2.179707in}}{\pgfqpoint{1.780363in}{2.175317in}}{\pgfqpoint{1.791413in}{2.175317in}}%
\pgfpathclose%
\pgfusepath{stroke,fill}%
\end{pgfscope}%
\begin{pgfscope}%
\pgfpathrectangle{\pgfqpoint{0.600000in}{0.600000in}}{\pgfqpoint{3.900000in}{3.900000in}}%
\pgfusepath{clip}%
\pgfsetbuttcap%
\pgfsetroundjoin%
\definecolor{currentfill}{rgb}{0.121569,0.466667,0.705882}%
\pgfsetfillcolor{currentfill}%
\pgfsetlinewidth{1.003750pt}%
\definecolor{currentstroke}{rgb}{0.121569,0.466667,0.705882}%
\pgfsetstrokecolor{currentstroke}%
\pgfsetdash{}{0pt}%
\pgfpathmoveto{\pgfqpoint{2.976249in}{2.377625in}}%
\pgfpathcurveto{\pgfqpoint{2.987299in}{2.377625in}}{\pgfqpoint{2.997898in}{2.382016in}}{\pgfqpoint{3.005711in}{2.389829in}}%
\pgfpathcurveto{\pgfqpoint{3.013525in}{2.397643in}}{\pgfqpoint{3.017915in}{2.408242in}}{\pgfqpoint{3.017915in}{2.419292in}}%
\pgfpathcurveto{\pgfqpoint{3.017915in}{2.430342in}}{\pgfqpoint{3.013525in}{2.440941in}}{\pgfqpoint{3.005711in}{2.448755in}}%
\pgfpathcurveto{\pgfqpoint{2.997898in}{2.456568in}}{\pgfqpoint{2.987299in}{2.460959in}}{\pgfqpoint{2.976249in}{2.460959in}}%
\pgfpathcurveto{\pgfqpoint{2.965198in}{2.460959in}}{\pgfqpoint{2.954599in}{2.456568in}}{\pgfqpoint{2.946786in}{2.448755in}}%
\pgfpathcurveto{\pgfqpoint{2.938972in}{2.440941in}}{\pgfqpoint{2.934582in}{2.430342in}}{\pgfqpoint{2.934582in}{2.419292in}}%
\pgfpathcurveto{\pgfqpoint{2.934582in}{2.408242in}}{\pgfqpoint{2.938972in}{2.397643in}}{\pgfqpoint{2.946786in}{2.389829in}}%
\pgfpathcurveto{\pgfqpoint{2.954599in}{2.382016in}}{\pgfqpoint{2.965198in}{2.377625in}}{\pgfqpoint{2.976249in}{2.377625in}}%
\pgfpathclose%
\pgfusepath{stroke,fill}%
\end{pgfscope}%
\begin{pgfscope}%
\pgfpathrectangle{\pgfqpoint{0.600000in}{0.600000in}}{\pgfqpoint{3.900000in}{3.900000in}}%
\pgfusepath{clip}%
\pgfsetbuttcap%
\pgfsetroundjoin%
\definecolor{currentfill}{rgb}{0.121569,0.466667,0.705882}%
\pgfsetfillcolor{currentfill}%
\pgfsetlinewidth{1.003750pt}%
\definecolor{currentstroke}{rgb}{0.121569,0.466667,0.705882}%
\pgfsetstrokecolor{currentstroke}%
\pgfsetdash{}{0pt}%
\pgfpathmoveto{\pgfqpoint{1.831969in}{2.836848in}}%
\pgfpathcurveto{\pgfqpoint{1.843019in}{2.836848in}}{\pgfqpoint{1.853618in}{2.841238in}}{\pgfqpoint{1.861431in}{2.849052in}}%
\pgfpathcurveto{\pgfqpoint{1.869245in}{2.856865in}}{\pgfqpoint{1.873635in}{2.867464in}}{\pgfqpoint{1.873635in}{2.878515in}}%
\pgfpathcurveto{\pgfqpoint{1.873635in}{2.889565in}}{\pgfqpoint{1.869245in}{2.900164in}}{\pgfqpoint{1.861431in}{2.907977in}}%
\pgfpathcurveto{\pgfqpoint{1.853618in}{2.915791in}}{\pgfqpoint{1.843019in}{2.920181in}}{\pgfqpoint{1.831969in}{2.920181in}}%
\pgfpathcurveto{\pgfqpoint{1.820918in}{2.920181in}}{\pgfqpoint{1.810319in}{2.915791in}}{\pgfqpoint{1.802506in}{2.907977in}}%
\pgfpathcurveto{\pgfqpoint{1.794692in}{2.900164in}}{\pgfqpoint{1.790302in}{2.889565in}}{\pgfqpoint{1.790302in}{2.878515in}}%
\pgfpathcurveto{\pgfqpoint{1.790302in}{2.867464in}}{\pgfqpoint{1.794692in}{2.856865in}}{\pgfqpoint{1.802506in}{2.849052in}}%
\pgfpathcurveto{\pgfqpoint{1.810319in}{2.841238in}}{\pgfqpoint{1.820918in}{2.836848in}}{\pgfqpoint{1.831969in}{2.836848in}}%
\pgfpathclose%
\pgfusepath{stroke,fill}%
\end{pgfscope}%
\begin{pgfscope}%
\pgfpathrectangle{\pgfqpoint{0.600000in}{0.600000in}}{\pgfqpoint{3.900000in}{3.900000in}}%
\pgfusepath{clip}%
\pgfsetbuttcap%
\pgfsetroundjoin%
\definecolor{currentfill}{rgb}{0.121569,0.466667,0.705882}%
\pgfsetfillcolor{currentfill}%
\pgfsetlinewidth{1.003750pt}%
\definecolor{currentstroke}{rgb}{0.121569,0.466667,0.705882}%
\pgfsetstrokecolor{currentstroke}%
\pgfsetdash{}{0pt}%
\pgfpathmoveto{\pgfqpoint{1.500358in}{2.732096in}}%
\pgfpathcurveto{\pgfqpoint{1.511408in}{2.732096in}}{\pgfqpoint{1.522007in}{2.736487in}}{\pgfqpoint{1.529821in}{2.744300in}}%
\pgfpathcurveto{\pgfqpoint{1.537635in}{2.752114in}}{\pgfqpoint{1.542025in}{2.762713in}}{\pgfqpoint{1.542025in}{2.773763in}}%
\pgfpathcurveto{\pgfqpoint{1.542025in}{2.784813in}}{\pgfqpoint{1.537635in}{2.795412in}}{\pgfqpoint{1.529821in}{2.803226in}}%
\pgfpathcurveto{\pgfqpoint{1.522007in}{2.811039in}}{\pgfqpoint{1.511408in}{2.815430in}}{\pgfqpoint{1.500358in}{2.815430in}}%
\pgfpathcurveto{\pgfqpoint{1.489308in}{2.815430in}}{\pgfqpoint{1.478709in}{2.811039in}}{\pgfqpoint{1.470895in}{2.803226in}}%
\pgfpathcurveto{\pgfqpoint{1.463082in}{2.795412in}}{\pgfqpoint{1.458691in}{2.784813in}}{\pgfqpoint{1.458691in}{2.773763in}}%
\pgfpathcurveto{\pgfqpoint{1.458691in}{2.762713in}}{\pgfqpoint{1.463082in}{2.752114in}}{\pgfqpoint{1.470895in}{2.744300in}}%
\pgfpathcurveto{\pgfqpoint{1.478709in}{2.736487in}}{\pgfqpoint{1.489308in}{2.732096in}}{\pgfqpoint{1.500358in}{2.732096in}}%
\pgfpathclose%
\pgfusepath{stroke,fill}%
\end{pgfscope}%
\begin{pgfscope}%
\pgfpathrectangle{\pgfqpoint{0.600000in}{0.600000in}}{\pgfqpoint{3.900000in}{3.900000in}}%
\pgfusepath{clip}%
\pgfsetbuttcap%
\pgfsetroundjoin%
\definecolor{currentfill}{rgb}{0.121569,0.466667,0.705882}%
\pgfsetfillcolor{currentfill}%
\pgfsetlinewidth{1.003750pt}%
\definecolor{currentstroke}{rgb}{0.121569,0.466667,0.705882}%
\pgfsetstrokecolor{currentstroke}%
\pgfsetdash{}{0pt}%
\pgfpathmoveto{\pgfqpoint{3.080512in}{3.065341in}}%
\pgfpathcurveto{\pgfqpoint{3.091562in}{3.065341in}}{\pgfqpoint{3.102161in}{3.069731in}}{\pgfqpoint{3.109975in}{3.077545in}}%
\pgfpathcurveto{\pgfqpoint{3.117789in}{3.085359in}}{\pgfqpoint{3.122179in}{3.095958in}}{\pgfqpoint{3.122179in}{3.107008in}}%
\pgfpathcurveto{\pgfqpoint{3.122179in}{3.118058in}}{\pgfqpoint{3.117789in}{3.128657in}}{\pgfqpoint{3.109975in}{3.136470in}}%
\pgfpathcurveto{\pgfqpoint{3.102161in}{3.144284in}}{\pgfqpoint{3.091562in}{3.148674in}}{\pgfqpoint{3.080512in}{3.148674in}}%
\pgfpathcurveto{\pgfqpoint{3.069462in}{3.148674in}}{\pgfqpoint{3.058863in}{3.144284in}}{\pgfqpoint{3.051049in}{3.136470in}}%
\pgfpathcurveto{\pgfqpoint{3.043236in}{3.128657in}}{\pgfqpoint{3.038845in}{3.118058in}}{\pgfqpoint{3.038845in}{3.107008in}}%
\pgfpathcurveto{\pgfqpoint{3.038845in}{3.095958in}}{\pgfqpoint{3.043236in}{3.085359in}}{\pgfqpoint{3.051049in}{3.077545in}}%
\pgfpathcurveto{\pgfqpoint{3.058863in}{3.069731in}}{\pgfqpoint{3.069462in}{3.065341in}}{\pgfqpoint{3.080512in}{3.065341in}}%
\pgfpathclose%
\pgfusepath{stroke,fill}%
\end{pgfscope}%
\begin{pgfscope}%
\pgfpathrectangle{\pgfqpoint{0.600000in}{0.600000in}}{\pgfqpoint{3.900000in}{3.900000in}}%
\pgfusepath{clip}%
\pgfsetbuttcap%
\pgfsetroundjoin%
\definecolor{currentfill}{rgb}{0.121569,0.466667,0.705882}%
\pgfsetfillcolor{currentfill}%
\pgfsetlinewidth{1.003750pt}%
\definecolor{currentstroke}{rgb}{0.121569,0.466667,0.705882}%
\pgfsetstrokecolor{currentstroke}%
\pgfsetdash{}{0pt}%
\pgfpathmoveto{\pgfqpoint{2.490592in}{2.783567in}}%
\pgfpathcurveto{\pgfqpoint{2.501642in}{2.783567in}}{\pgfqpoint{2.512241in}{2.787958in}}{\pgfqpoint{2.520055in}{2.795771in}}%
\pgfpathcurveto{\pgfqpoint{2.527869in}{2.803585in}}{\pgfqpoint{2.532259in}{2.814184in}}{\pgfqpoint{2.532259in}{2.825234in}}%
\pgfpathcurveto{\pgfqpoint{2.532259in}{2.836284in}}{\pgfqpoint{2.527869in}{2.846883in}}{\pgfqpoint{2.520055in}{2.854697in}}%
\pgfpathcurveto{\pgfqpoint{2.512241in}{2.862511in}}{\pgfqpoint{2.501642in}{2.866901in}}{\pgfqpoint{2.490592in}{2.866901in}}%
\pgfpathcurveto{\pgfqpoint{2.479542in}{2.866901in}}{\pgfqpoint{2.468943in}{2.862511in}}{\pgfqpoint{2.461129in}{2.854697in}}%
\pgfpathcurveto{\pgfqpoint{2.453316in}{2.846883in}}{\pgfqpoint{2.448926in}{2.836284in}}{\pgfqpoint{2.448926in}{2.825234in}}%
\pgfpathcurveto{\pgfqpoint{2.448926in}{2.814184in}}{\pgfqpoint{2.453316in}{2.803585in}}{\pgfqpoint{2.461129in}{2.795771in}}%
\pgfpathcurveto{\pgfqpoint{2.468943in}{2.787958in}}{\pgfqpoint{2.479542in}{2.783567in}}{\pgfqpoint{2.490592in}{2.783567in}}%
\pgfpathclose%
\pgfusepath{stroke,fill}%
\end{pgfscope}%
\begin{pgfscope}%
\pgfpathrectangle{\pgfqpoint{0.600000in}{0.600000in}}{\pgfqpoint{3.900000in}{3.900000in}}%
\pgfusepath{clip}%
\pgfsetbuttcap%
\pgfsetroundjoin%
\definecolor{currentfill}{rgb}{0.121569,0.466667,0.705882}%
\pgfsetfillcolor{currentfill}%
\pgfsetlinewidth{1.003750pt}%
\definecolor{currentstroke}{rgb}{0.121569,0.466667,0.705882}%
\pgfsetstrokecolor{currentstroke}%
\pgfsetdash{}{0pt}%
\pgfpathmoveto{\pgfqpoint{3.075774in}{2.607709in}}%
\pgfpathcurveto{\pgfqpoint{3.086824in}{2.607709in}}{\pgfqpoint{3.097423in}{2.612099in}}{\pgfqpoint{3.105237in}{2.619913in}}%
\pgfpathcurveto{\pgfqpoint{3.113051in}{2.627726in}}{\pgfqpoint{3.117441in}{2.638325in}}{\pgfqpoint{3.117441in}{2.649375in}}%
\pgfpathcurveto{\pgfqpoint{3.117441in}{2.660426in}}{\pgfqpoint{3.113051in}{2.671025in}}{\pgfqpoint{3.105237in}{2.678838in}}%
\pgfpathcurveto{\pgfqpoint{3.097423in}{2.686652in}}{\pgfqpoint{3.086824in}{2.691042in}}{\pgfqpoint{3.075774in}{2.691042in}}%
\pgfpathcurveto{\pgfqpoint{3.064724in}{2.691042in}}{\pgfqpoint{3.054125in}{2.686652in}}{\pgfqpoint{3.046312in}{2.678838in}}%
\pgfpathcurveto{\pgfqpoint{3.038498in}{2.671025in}}{\pgfqpoint{3.034108in}{2.660426in}}{\pgfqpoint{3.034108in}{2.649375in}}%
\pgfpathcurveto{\pgfqpoint{3.034108in}{2.638325in}}{\pgfqpoint{3.038498in}{2.627726in}}{\pgfqpoint{3.046312in}{2.619913in}}%
\pgfpathcurveto{\pgfqpoint{3.054125in}{2.612099in}}{\pgfqpoint{3.064724in}{2.607709in}}{\pgfqpoint{3.075774in}{2.607709in}}%
\pgfpathclose%
\pgfusepath{stroke,fill}%
\end{pgfscope}%
\begin{pgfscope}%
\pgfpathrectangle{\pgfqpoint{0.600000in}{0.600000in}}{\pgfqpoint{3.900000in}{3.900000in}}%
\pgfusepath{clip}%
\pgfsetbuttcap%
\pgfsetroundjoin%
\definecolor{currentfill}{rgb}{0.121569,0.466667,0.705882}%
\pgfsetfillcolor{currentfill}%
\pgfsetlinewidth{1.003750pt}%
\definecolor{currentstroke}{rgb}{0.121569,0.466667,0.705882}%
\pgfsetstrokecolor{currentstroke}%
\pgfsetdash{}{0pt}%
\pgfpathmoveto{\pgfqpoint{3.129707in}{2.659375in}}%
\pgfpathcurveto{\pgfqpoint{3.140757in}{2.659375in}}{\pgfqpoint{3.151356in}{2.663766in}}{\pgfqpoint{3.159169in}{2.671579in}}%
\pgfpathcurveto{\pgfqpoint{3.166983in}{2.679393in}}{\pgfqpoint{3.171373in}{2.689992in}}{\pgfqpoint{3.171373in}{2.701042in}}%
\pgfpathcurveto{\pgfqpoint{3.171373in}{2.712092in}}{\pgfqpoint{3.166983in}{2.722691in}}{\pgfqpoint{3.159169in}{2.730505in}}%
\pgfpathcurveto{\pgfqpoint{3.151356in}{2.738319in}}{\pgfqpoint{3.140757in}{2.742709in}}{\pgfqpoint{3.129707in}{2.742709in}}%
\pgfpathcurveto{\pgfqpoint{3.118656in}{2.742709in}}{\pgfqpoint{3.108057in}{2.738319in}}{\pgfqpoint{3.100244in}{2.730505in}}%
\pgfpathcurveto{\pgfqpoint{3.092430in}{2.722691in}}{\pgfqpoint{3.088040in}{2.712092in}}{\pgfqpoint{3.088040in}{2.701042in}}%
\pgfpathcurveto{\pgfqpoint{3.088040in}{2.689992in}}{\pgfqpoint{3.092430in}{2.679393in}}{\pgfqpoint{3.100244in}{2.671579in}}%
\pgfpathcurveto{\pgfqpoint{3.108057in}{2.663766in}}{\pgfqpoint{3.118656in}{2.659375in}}{\pgfqpoint{3.129707in}{2.659375in}}%
\pgfpathclose%
\pgfusepath{stroke,fill}%
\end{pgfscope}%
\begin{pgfscope}%
\pgfpathrectangle{\pgfqpoint{0.600000in}{0.600000in}}{\pgfqpoint{3.900000in}{3.900000in}}%
\pgfusepath{clip}%
\pgfsetbuttcap%
\pgfsetroundjoin%
\definecolor{currentfill}{rgb}{0.121569,0.466667,0.705882}%
\pgfsetfillcolor{currentfill}%
\pgfsetlinewidth{1.003750pt}%
\definecolor{currentstroke}{rgb}{0.121569,0.466667,0.705882}%
\pgfsetstrokecolor{currentstroke}%
\pgfsetdash{}{0pt}%
\pgfpathmoveto{\pgfqpoint{3.890042in}{2.283569in}}%
\pgfpathcurveto{\pgfqpoint{3.901092in}{2.283569in}}{\pgfqpoint{3.911691in}{2.287959in}}{\pgfqpoint{3.919505in}{2.295773in}}%
\pgfpathcurveto{\pgfqpoint{3.927319in}{2.303586in}}{\pgfqpoint{3.931709in}{2.314185in}}{\pgfqpoint{3.931709in}{2.325236in}}%
\pgfpathcurveto{\pgfqpoint{3.931709in}{2.336286in}}{\pgfqpoint{3.927319in}{2.346885in}}{\pgfqpoint{3.919505in}{2.354698in}}%
\pgfpathcurveto{\pgfqpoint{3.911691in}{2.362512in}}{\pgfqpoint{3.901092in}{2.366902in}}{\pgfqpoint{3.890042in}{2.366902in}}%
\pgfpathcurveto{\pgfqpoint{3.878992in}{2.366902in}}{\pgfqpoint{3.868393in}{2.362512in}}{\pgfqpoint{3.860579in}{2.354698in}}%
\pgfpathcurveto{\pgfqpoint{3.852766in}{2.346885in}}{\pgfqpoint{3.848376in}{2.336286in}}{\pgfqpoint{3.848376in}{2.325236in}}%
\pgfpathcurveto{\pgfqpoint{3.848376in}{2.314185in}}{\pgfqpoint{3.852766in}{2.303586in}}{\pgfqpoint{3.860579in}{2.295773in}}%
\pgfpathcurveto{\pgfqpoint{3.868393in}{2.287959in}}{\pgfqpoint{3.878992in}{2.283569in}}{\pgfqpoint{3.890042in}{2.283569in}}%
\pgfpathclose%
\pgfusepath{stroke,fill}%
\end{pgfscope}%
\begin{pgfscope}%
\pgfpathrectangle{\pgfqpoint{0.600000in}{0.600000in}}{\pgfqpoint{3.900000in}{3.900000in}}%
\pgfusepath{clip}%
\pgfsetbuttcap%
\pgfsetroundjoin%
\definecolor{currentfill}{rgb}{0.121569,0.466667,0.705882}%
\pgfsetfillcolor{currentfill}%
\pgfsetlinewidth{1.003750pt}%
\definecolor{currentstroke}{rgb}{0.121569,0.466667,0.705882}%
\pgfsetstrokecolor{currentstroke}%
\pgfsetdash{}{0pt}%
\pgfpathmoveto{\pgfqpoint{2.385576in}{1.141793in}}%
\pgfpathcurveto{\pgfqpoint{2.396626in}{1.141793in}}{\pgfqpoint{2.407225in}{1.146184in}}{\pgfqpoint{2.415039in}{1.153997in}}%
\pgfpathcurveto{\pgfqpoint{2.422853in}{1.161811in}}{\pgfqpoint{2.427243in}{1.172410in}}{\pgfqpoint{2.427243in}{1.183460in}}%
\pgfpathcurveto{\pgfqpoint{2.427243in}{1.194510in}}{\pgfqpoint{2.422853in}{1.205109in}}{\pgfqpoint{2.415039in}{1.212923in}}%
\pgfpathcurveto{\pgfqpoint{2.407225in}{1.220736in}}{\pgfqpoint{2.396626in}{1.225127in}}{\pgfqpoint{2.385576in}{1.225127in}}%
\pgfpathcurveto{\pgfqpoint{2.374526in}{1.225127in}}{\pgfqpoint{2.363927in}{1.220736in}}{\pgfqpoint{2.356113in}{1.212923in}}%
\pgfpathcurveto{\pgfqpoint{2.348300in}{1.205109in}}{\pgfqpoint{2.343910in}{1.194510in}}{\pgfqpoint{2.343910in}{1.183460in}}%
\pgfpathcurveto{\pgfqpoint{2.343910in}{1.172410in}}{\pgfqpoint{2.348300in}{1.161811in}}{\pgfqpoint{2.356113in}{1.153997in}}%
\pgfpathcurveto{\pgfqpoint{2.363927in}{1.146184in}}{\pgfqpoint{2.374526in}{1.141793in}}{\pgfqpoint{2.385576in}{1.141793in}}%
\pgfpathclose%
\pgfusepath{stroke,fill}%
\end{pgfscope}%
\begin{pgfscope}%
\pgfpathrectangle{\pgfqpoint{0.600000in}{0.600000in}}{\pgfqpoint{3.900000in}{3.900000in}}%
\pgfusepath{clip}%
\pgfsetbuttcap%
\pgfsetroundjoin%
\definecolor{currentfill}{rgb}{0.121569,0.466667,0.705882}%
\pgfsetfillcolor{currentfill}%
\pgfsetlinewidth{1.003750pt}%
\definecolor{currentstroke}{rgb}{0.121569,0.466667,0.705882}%
\pgfsetstrokecolor{currentstroke}%
\pgfsetdash{}{0pt}%
\pgfpathmoveto{\pgfqpoint{2.455478in}{2.846959in}}%
\pgfpathcurveto{\pgfqpoint{2.466528in}{2.846959in}}{\pgfqpoint{2.477127in}{2.851349in}}{\pgfqpoint{2.484941in}{2.859163in}}%
\pgfpathcurveto{\pgfqpoint{2.492755in}{2.866976in}}{\pgfqpoint{2.497145in}{2.877575in}}{\pgfqpoint{2.497145in}{2.888626in}}%
\pgfpathcurveto{\pgfqpoint{2.497145in}{2.899676in}}{\pgfqpoint{2.492755in}{2.910275in}}{\pgfqpoint{2.484941in}{2.918088in}}%
\pgfpathcurveto{\pgfqpoint{2.477127in}{2.925902in}}{\pgfqpoint{2.466528in}{2.930292in}}{\pgfqpoint{2.455478in}{2.930292in}}%
\pgfpathcurveto{\pgfqpoint{2.444428in}{2.930292in}}{\pgfqpoint{2.433829in}{2.925902in}}{\pgfqpoint{2.426015in}{2.918088in}}%
\pgfpathcurveto{\pgfqpoint{2.418202in}{2.910275in}}{\pgfqpoint{2.413811in}{2.899676in}}{\pgfqpoint{2.413811in}{2.888626in}}%
\pgfpathcurveto{\pgfqpoint{2.413811in}{2.877575in}}{\pgfqpoint{2.418202in}{2.866976in}}{\pgfqpoint{2.426015in}{2.859163in}}%
\pgfpathcurveto{\pgfqpoint{2.433829in}{2.851349in}}{\pgfqpoint{2.444428in}{2.846959in}}{\pgfqpoint{2.455478in}{2.846959in}}%
\pgfpathclose%
\pgfusepath{stroke,fill}%
\end{pgfscope}%
\begin{pgfscope}%
\pgfpathrectangle{\pgfqpoint{0.600000in}{0.600000in}}{\pgfqpoint{3.900000in}{3.900000in}}%
\pgfusepath{clip}%
\pgfsetbuttcap%
\pgfsetroundjoin%
\definecolor{currentfill}{rgb}{0.121569,0.466667,0.705882}%
\pgfsetfillcolor{currentfill}%
\pgfsetlinewidth{1.003750pt}%
\definecolor{currentstroke}{rgb}{0.121569,0.466667,0.705882}%
\pgfsetstrokecolor{currentstroke}%
\pgfsetdash{}{0pt}%
\pgfpathmoveto{\pgfqpoint{2.149604in}{2.927228in}}%
\pgfpathcurveto{\pgfqpoint{2.160654in}{2.927228in}}{\pgfqpoint{2.171253in}{2.931618in}}{\pgfqpoint{2.179066in}{2.939432in}}%
\pgfpathcurveto{\pgfqpoint{2.186880in}{2.947246in}}{\pgfqpoint{2.191270in}{2.957845in}}{\pgfqpoint{2.191270in}{2.968895in}}%
\pgfpathcurveto{\pgfqpoint{2.191270in}{2.979945in}}{\pgfqpoint{2.186880in}{2.990544in}}{\pgfqpoint{2.179066in}{2.998358in}}%
\pgfpathcurveto{\pgfqpoint{2.171253in}{3.006171in}}{\pgfqpoint{2.160654in}{3.010561in}}{\pgfqpoint{2.149604in}{3.010561in}}%
\pgfpathcurveto{\pgfqpoint{2.138553in}{3.010561in}}{\pgfqpoint{2.127954in}{3.006171in}}{\pgfqpoint{2.120141in}{2.998358in}}%
\pgfpathcurveto{\pgfqpoint{2.112327in}{2.990544in}}{\pgfqpoint{2.107937in}{2.979945in}}{\pgfqpoint{2.107937in}{2.968895in}}%
\pgfpathcurveto{\pgfqpoint{2.107937in}{2.957845in}}{\pgfqpoint{2.112327in}{2.947246in}}{\pgfqpoint{2.120141in}{2.939432in}}%
\pgfpathcurveto{\pgfqpoint{2.127954in}{2.931618in}}{\pgfqpoint{2.138553in}{2.927228in}}{\pgfqpoint{2.149604in}{2.927228in}}%
\pgfpathclose%
\pgfusepath{stroke,fill}%
\end{pgfscope}%
\begin{pgfscope}%
\pgfpathrectangle{\pgfqpoint{0.600000in}{0.600000in}}{\pgfqpoint{3.900000in}{3.900000in}}%
\pgfusepath{clip}%
\pgfsetbuttcap%
\pgfsetroundjoin%
\definecolor{currentfill}{rgb}{0.121569,0.466667,0.705882}%
\pgfsetfillcolor{currentfill}%
\pgfsetlinewidth{1.003750pt}%
\definecolor{currentstroke}{rgb}{0.121569,0.466667,0.705882}%
\pgfsetstrokecolor{currentstroke}%
\pgfsetdash{}{0pt}%
\pgfpathmoveto{\pgfqpoint{2.572891in}{2.887566in}}%
\pgfpathcurveto{\pgfqpoint{2.583941in}{2.887566in}}{\pgfqpoint{2.594540in}{2.891956in}}{\pgfqpoint{2.602354in}{2.899770in}}%
\pgfpathcurveto{\pgfqpoint{2.610167in}{2.907584in}}{\pgfqpoint{2.614557in}{2.918183in}}{\pgfqpoint{2.614557in}{2.929233in}}%
\pgfpathcurveto{\pgfqpoint{2.614557in}{2.940283in}}{\pgfqpoint{2.610167in}{2.950882in}}{\pgfqpoint{2.602354in}{2.958696in}}%
\pgfpathcurveto{\pgfqpoint{2.594540in}{2.966509in}}{\pgfqpoint{2.583941in}{2.970899in}}{\pgfqpoint{2.572891in}{2.970899in}}%
\pgfpathcurveto{\pgfqpoint{2.561841in}{2.970899in}}{\pgfqpoint{2.551242in}{2.966509in}}{\pgfqpoint{2.543428in}{2.958696in}}%
\pgfpathcurveto{\pgfqpoint{2.535614in}{2.950882in}}{\pgfqpoint{2.531224in}{2.940283in}}{\pgfqpoint{2.531224in}{2.929233in}}%
\pgfpathcurveto{\pgfqpoint{2.531224in}{2.918183in}}{\pgfqpoint{2.535614in}{2.907584in}}{\pgfqpoint{2.543428in}{2.899770in}}%
\pgfpathcurveto{\pgfqpoint{2.551242in}{2.891956in}}{\pgfqpoint{2.561841in}{2.887566in}}{\pgfqpoint{2.572891in}{2.887566in}}%
\pgfpathclose%
\pgfusepath{stroke,fill}%
\end{pgfscope}%
\begin{pgfscope}%
\pgfpathrectangle{\pgfqpoint{0.600000in}{0.600000in}}{\pgfqpoint{3.900000in}{3.900000in}}%
\pgfusepath{clip}%
\pgfsetbuttcap%
\pgfsetroundjoin%
\definecolor{currentfill}{rgb}{0.121569,0.466667,0.705882}%
\pgfsetfillcolor{currentfill}%
\pgfsetlinewidth{1.003750pt}%
\definecolor{currentstroke}{rgb}{0.121569,0.466667,0.705882}%
\pgfsetstrokecolor{currentstroke}%
\pgfsetdash{}{0pt}%
\pgfpathmoveto{\pgfqpoint{2.895883in}{2.142828in}}%
\pgfpathcurveto{\pgfqpoint{2.906933in}{2.142828in}}{\pgfqpoint{2.917532in}{2.147218in}}{\pgfqpoint{2.925346in}{2.155032in}}%
\pgfpathcurveto{\pgfqpoint{2.933159in}{2.162846in}}{\pgfqpoint{2.937549in}{2.173445in}}{\pgfqpoint{2.937549in}{2.184495in}}%
\pgfpathcurveto{\pgfqpoint{2.937549in}{2.195545in}}{\pgfqpoint{2.933159in}{2.206144in}}{\pgfqpoint{2.925346in}{2.213958in}}%
\pgfpathcurveto{\pgfqpoint{2.917532in}{2.221771in}}{\pgfqpoint{2.906933in}{2.226161in}}{\pgfqpoint{2.895883in}{2.226161in}}%
\pgfpathcurveto{\pgfqpoint{2.884833in}{2.226161in}}{\pgfqpoint{2.874234in}{2.221771in}}{\pgfqpoint{2.866420in}{2.213958in}}%
\pgfpathcurveto{\pgfqpoint{2.858606in}{2.206144in}}{\pgfqpoint{2.854216in}{2.195545in}}{\pgfqpoint{2.854216in}{2.184495in}}%
\pgfpathcurveto{\pgfqpoint{2.854216in}{2.173445in}}{\pgfqpoint{2.858606in}{2.162846in}}{\pgfqpoint{2.866420in}{2.155032in}}%
\pgfpathcurveto{\pgfqpoint{2.874234in}{2.147218in}}{\pgfqpoint{2.884833in}{2.142828in}}{\pgfqpoint{2.895883in}{2.142828in}}%
\pgfpathclose%
\pgfusepath{stroke,fill}%
\end{pgfscope}%
\begin{pgfscope}%
\pgfpathrectangle{\pgfqpoint{0.600000in}{0.600000in}}{\pgfqpoint{3.900000in}{3.900000in}}%
\pgfusepath{clip}%
\pgfsetbuttcap%
\pgfsetroundjoin%
\definecolor{currentfill}{rgb}{0.121569,0.466667,0.705882}%
\pgfsetfillcolor{currentfill}%
\pgfsetlinewidth{1.003750pt}%
\definecolor{currentstroke}{rgb}{0.121569,0.466667,0.705882}%
\pgfsetstrokecolor{currentstroke}%
\pgfsetdash{}{0pt}%
\pgfpathmoveto{\pgfqpoint{2.434955in}{3.227913in}}%
\pgfpathcurveto{\pgfqpoint{2.446005in}{3.227913in}}{\pgfqpoint{2.456604in}{3.232303in}}{\pgfqpoint{2.464417in}{3.240117in}}%
\pgfpathcurveto{\pgfqpoint{2.472231in}{3.247930in}}{\pgfqpoint{2.476621in}{3.258529in}}{\pgfqpoint{2.476621in}{3.269579in}}%
\pgfpathcurveto{\pgfqpoint{2.476621in}{3.280629in}}{\pgfqpoint{2.472231in}{3.291229in}}{\pgfqpoint{2.464417in}{3.299042in}}%
\pgfpathcurveto{\pgfqpoint{2.456604in}{3.306856in}}{\pgfqpoint{2.446005in}{3.311246in}}{\pgfqpoint{2.434955in}{3.311246in}}%
\pgfpathcurveto{\pgfqpoint{2.423905in}{3.311246in}}{\pgfqpoint{2.413306in}{3.306856in}}{\pgfqpoint{2.405492in}{3.299042in}}%
\pgfpathcurveto{\pgfqpoint{2.397678in}{3.291229in}}{\pgfqpoint{2.393288in}{3.280629in}}{\pgfqpoint{2.393288in}{3.269579in}}%
\pgfpathcurveto{\pgfqpoint{2.393288in}{3.258529in}}{\pgfqpoint{2.397678in}{3.247930in}}{\pgfqpoint{2.405492in}{3.240117in}}%
\pgfpathcurveto{\pgfqpoint{2.413306in}{3.232303in}}{\pgfqpoint{2.423905in}{3.227913in}}{\pgfqpoint{2.434955in}{3.227913in}}%
\pgfpathclose%
\pgfusepath{stroke,fill}%
\end{pgfscope}%
\begin{pgfscope}%
\pgfpathrectangle{\pgfqpoint{0.600000in}{0.600000in}}{\pgfqpoint{3.900000in}{3.900000in}}%
\pgfusepath{clip}%
\pgfsetbuttcap%
\pgfsetroundjoin%
\definecolor{currentfill}{rgb}{0.121569,0.466667,0.705882}%
\pgfsetfillcolor{currentfill}%
\pgfsetlinewidth{1.003750pt}%
\definecolor{currentstroke}{rgb}{0.121569,0.466667,0.705882}%
\pgfsetstrokecolor{currentstroke}%
\pgfsetdash{}{0pt}%
\pgfpathmoveto{\pgfqpoint{2.550378in}{1.706995in}}%
\pgfpathcurveto{\pgfqpoint{2.561428in}{1.706995in}}{\pgfqpoint{2.572027in}{1.711385in}}{\pgfqpoint{2.579841in}{1.719199in}}%
\pgfpathcurveto{\pgfqpoint{2.587654in}{1.727012in}}{\pgfqpoint{2.592045in}{1.737611in}}{\pgfqpoint{2.592045in}{1.748661in}}%
\pgfpathcurveto{\pgfqpoint{2.592045in}{1.759711in}}{\pgfqpoint{2.587654in}{1.770310in}}{\pgfqpoint{2.579841in}{1.778124in}}%
\pgfpathcurveto{\pgfqpoint{2.572027in}{1.785938in}}{\pgfqpoint{2.561428in}{1.790328in}}{\pgfqpoint{2.550378in}{1.790328in}}%
\pgfpathcurveto{\pgfqpoint{2.539328in}{1.790328in}}{\pgfqpoint{2.528729in}{1.785938in}}{\pgfqpoint{2.520915in}{1.778124in}}%
\pgfpathcurveto{\pgfqpoint{2.513102in}{1.770310in}}{\pgfqpoint{2.508711in}{1.759711in}}{\pgfqpoint{2.508711in}{1.748661in}}%
\pgfpathcurveto{\pgfqpoint{2.508711in}{1.737611in}}{\pgfqpoint{2.513102in}{1.727012in}}{\pgfqpoint{2.520915in}{1.719199in}}%
\pgfpathcurveto{\pgfqpoint{2.528729in}{1.711385in}}{\pgfqpoint{2.539328in}{1.706995in}}{\pgfqpoint{2.550378in}{1.706995in}}%
\pgfpathclose%
\pgfusepath{stroke,fill}%
\end{pgfscope}%
\begin{pgfscope}%
\pgfpathrectangle{\pgfqpoint{0.600000in}{0.600000in}}{\pgfqpoint{3.900000in}{3.900000in}}%
\pgfusepath{clip}%
\pgfsetbuttcap%
\pgfsetroundjoin%
\definecolor{currentfill}{rgb}{0.121569,0.466667,0.705882}%
\pgfsetfillcolor{currentfill}%
\pgfsetlinewidth{1.003750pt}%
\definecolor{currentstroke}{rgb}{0.121569,0.466667,0.705882}%
\pgfsetstrokecolor{currentstroke}%
\pgfsetdash{}{0pt}%
\pgfpathmoveto{\pgfqpoint{2.221879in}{2.193721in}}%
\pgfpathcurveto{\pgfqpoint{2.232929in}{2.193721in}}{\pgfqpoint{2.243529in}{2.198111in}}{\pgfqpoint{2.251342in}{2.205925in}}%
\pgfpathcurveto{\pgfqpoint{2.259156in}{2.213738in}}{\pgfqpoint{2.263546in}{2.224337in}}{\pgfqpoint{2.263546in}{2.235388in}}%
\pgfpathcurveto{\pgfqpoint{2.263546in}{2.246438in}}{\pgfqpoint{2.259156in}{2.257037in}}{\pgfqpoint{2.251342in}{2.264850in}}%
\pgfpathcurveto{\pgfqpoint{2.243529in}{2.272664in}}{\pgfqpoint{2.232929in}{2.277054in}}{\pgfqpoint{2.221879in}{2.277054in}}%
\pgfpathcurveto{\pgfqpoint{2.210829in}{2.277054in}}{\pgfqpoint{2.200230in}{2.272664in}}{\pgfqpoint{2.192417in}{2.264850in}}%
\pgfpathcurveto{\pgfqpoint{2.184603in}{2.257037in}}{\pgfqpoint{2.180213in}{2.246438in}}{\pgfqpoint{2.180213in}{2.235388in}}%
\pgfpathcurveto{\pgfqpoint{2.180213in}{2.224337in}}{\pgfqpoint{2.184603in}{2.213738in}}{\pgfqpoint{2.192417in}{2.205925in}}%
\pgfpathcurveto{\pgfqpoint{2.200230in}{2.198111in}}{\pgfqpoint{2.210829in}{2.193721in}}{\pgfqpoint{2.221879in}{2.193721in}}%
\pgfpathclose%
\pgfusepath{stroke,fill}%
\end{pgfscope}%
\begin{pgfscope}%
\pgfpathrectangle{\pgfqpoint{0.600000in}{0.600000in}}{\pgfqpoint{3.900000in}{3.900000in}}%
\pgfusepath{clip}%
\pgfsetbuttcap%
\pgfsetroundjoin%
\definecolor{currentfill}{rgb}{0.121569,0.466667,0.705882}%
\pgfsetfillcolor{currentfill}%
\pgfsetlinewidth{1.003750pt}%
\definecolor{currentstroke}{rgb}{0.121569,0.466667,0.705882}%
\pgfsetstrokecolor{currentstroke}%
\pgfsetdash{}{0pt}%
\pgfpathmoveto{\pgfqpoint{2.682498in}{3.065247in}}%
\pgfpathcurveto{\pgfqpoint{2.693549in}{3.065247in}}{\pgfqpoint{2.704148in}{3.069638in}}{\pgfqpoint{2.711961in}{3.077451in}}%
\pgfpathcurveto{\pgfqpoint{2.719775in}{3.085265in}}{\pgfqpoint{2.724165in}{3.095864in}}{\pgfqpoint{2.724165in}{3.106914in}}%
\pgfpathcurveto{\pgfqpoint{2.724165in}{3.117964in}}{\pgfqpoint{2.719775in}{3.128563in}}{\pgfqpoint{2.711961in}{3.136377in}}%
\pgfpathcurveto{\pgfqpoint{2.704148in}{3.144190in}}{\pgfqpoint{2.693549in}{3.148581in}}{\pgfqpoint{2.682498in}{3.148581in}}%
\pgfpathcurveto{\pgfqpoint{2.671448in}{3.148581in}}{\pgfqpoint{2.660849in}{3.144190in}}{\pgfqpoint{2.653036in}{3.136377in}}%
\pgfpathcurveto{\pgfqpoint{2.645222in}{3.128563in}}{\pgfqpoint{2.640832in}{3.117964in}}{\pgfqpoint{2.640832in}{3.106914in}}%
\pgfpathcurveto{\pgfqpoint{2.640832in}{3.095864in}}{\pgfqpoint{2.645222in}{3.085265in}}{\pgfqpoint{2.653036in}{3.077451in}}%
\pgfpathcurveto{\pgfqpoint{2.660849in}{3.069638in}}{\pgfqpoint{2.671448in}{3.065247in}}{\pgfqpoint{2.682498in}{3.065247in}}%
\pgfpathclose%
\pgfusepath{stroke,fill}%
\end{pgfscope}%
\begin{pgfscope}%
\pgfpathrectangle{\pgfqpoint{0.600000in}{0.600000in}}{\pgfqpoint{3.900000in}{3.900000in}}%
\pgfusepath{clip}%
\pgfsetbuttcap%
\pgfsetroundjoin%
\definecolor{currentfill}{rgb}{0.121569,0.466667,0.705882}%
\pgfsetfillcolor{currentfill}%
\pgfsetlinewidth{1.003750pt}%
\definecolor{currentstroke}{rgb}{0.121569,0.466667,0.705882}%
\pgfsetstrokecolor{currentstroke}%
\pgfsetdash{}{0pt}%
\pgfpathmoveto{\pgfqpoint{1.667513in}{2.268255in}}%
\pgfpathcurveto{\pgfqpoint{1.678563in}{2.268255in}}{\pgfqpoint{1.689162in}{2.272645in}}{\pgfqpoint{1.696975in}{2.280459in}}%
\pgfpathcurveto{\pgfqpoint{1.704789in}{2.288273in}}{\pgfqpoint{1.709179in}{2.298872in}}{\pgfqpoint{1.709179in}{2.309922in}}%
\pgfpathcurveto{\pgfqpoint{1.709179in}{2.320972in}}{\pgfqpoint{1.704789in}{2.331571in}}{\pgfqpoint{1.696975in}{2.339385in}}%
\pgfpathcurveto{\pgfqpoint{1.689162in}{2.347198in}}{\pgfqpoint{1.678563in}{2.351588in}}{\pgfqpoint{1.667513in}{2.351588in}}%
\pgfpathcurveto{\pgfqpoint{1.656462in}{2.351588in}}{\pgfqpoint{1.645863in}{2.347198in}}{\pgfqpoint{1.638050in}{2.339385in}}%
\pgfpathcurveto{\pgfqpoint{1.630236in}{2.331571in}}{\pgfqpoint{1.625846in}{2.320972in}}{\pgfqpoint{1.625846in}{2.309922in}}%
\pgfpathcurveto{\pgfqpoint{1.625846in}{2.298872in}}{\pgfqpoint{1.630236in}{2.288273in}}{\pgfqpoint{1.638050in}{2.280459in}}%
\pgfpathcurveto{\pgfqpoint{1.645863in}{2.272645in}}{\pgfqpoint{1.656462in}{2.268255in}}{\pgfqpoint{1.667513in}{2.268255in}}%
\pgfpathclose%
\pgfusepath{stroke,fill}%
\end{pgfscope}%
\begin{pgfscope}%
\pgfpathrectangle{\pgfqpoint{0.600000in}{0.600000in}}{\pgfqpoint{3.900000in}{3.900000in}}%
\pgfusepath{clip}%
\pgfsetbuttcap%
\pgfsetroundjoin%
\definecolor{currentfill}{rgb}{0.121569,0.466667,0.705882}%
\pgfsetfillcolor{currentfill}%
\pgfsetlinewidth{1.003750pt}%
\definecolor{currentstroke}{rgb}{0.121569,0.466667,0.705882}%
\pgfsetstrokecolor{currentstroke}%
\pgfsetdash{}{0pt}%
\pgfpathmoveto{\pgfqpoint{2.988631in}{2.584381in}}%
\pgfpathcurveto{\pgfqpoint{2.999681in}{2.584381in}}{\pgfqpoint{3.010280in}{2.588771in}}{\pgfqpoint{3.018094in}{2.596585in}}%
\pgfpathcurveto{\pgfqpoint{3.025907in}{2.604398in}}{\pgfqpoint{3.030298in}{2.614997in}}{\pgfqpoint{3.030298in}{2.626048in}}%
\pgfpathcurveto{\pgfqpoint{3.030298in}{2.637098in}}{\pgfqpoint{3.025907in}{2.647697in}}{\pgfqpoint{3.018094in}{2.655510in}}%
\pgfpathcurveto{\pgfqpoint{3.010280in}{2.663324in}}{\pgfqpoint{2.999681in}{2.667714in}}{\pgfqpoint{2.988631in}{2.667714in}}%
\pgfpathcurveto{\pgfqpoint{2.977581in}{2.667714in}}{\pgfqpoint{2.966982in}{2.663324in}}{\pgfqpoint{2.959168in}{2.655510in}}%
\pgfpathcurveto{\pgfqpoint{2.951355in}{2.647697in}}{\pgfqpoint{2.946964in}{2.637098in}}{\pgfqpoint{2.946964in}{2.626048in}}%
\pgfpathcurveto{\pgfqpoint{2.946964in}{2.614997in}}{\pgfqpoint{2.951355in}{2.604398in}}{\pgfqpoint{2.959168in}{2.596585in}}%
\pgfpathcurveto{\pgfqpoint{2.966982in}{2.588771in}}{\pgfqpoint{2.977581in}{2.584381in}}{\pgfqpoint{2.988631in}{2.584381in}}%
\pgfpathclose%
\pgfusepath{stroke,fill}%
\end{pgfscope}%
\begin{pgfscope}%
\pgfpathrectangle{\pgfqpoint{0.600000in}{0.600000in}}{\pgfqpoint{3.900000in}{3.900000in}}%
\pgfusepath{clip}%
\pgfsetbuttcap%
\pgfsetroundjoin%
\definecolor{currentfill}{rgb}{0.121569,0.466667,0.705882}%
\pgfsetfillcolor{currentfill}%
\pgfsetlinewidth{1.003750pt}%
\definecolor{currentstroke}{rgb}{0.121569,0.466667,0.705882}%
\pgfsetstrokecolor{currentstroke}%
\pgfsetdash{}{0pt}%
\pgfpathmoveto{\pgfqpoint{2.681672in}{3.300446in}}%
\pgfpathcurveto{\pgfqpoint{2.692722in}{3.300446in}}{\pgfqpoint{2.703321in}{3.304836in}}{\pgfqpoint{2.711135in}{3.312650in}}%
\pgfpathcurveto{\pgfqpoint{2.718949in}{3.320463in}}{\pgfqpoint{2.723339in}{3.331063in}}{\pgfqpoint{2.723339in}{3.342113in}}%
\pgfpathcurveto{\pgfqpoint{2.723339in}{3.353163in}}{\pgfqpoint{2.718949in}{3.363762in}}{\pgfqpoint{2.711135in}{3.371575in}}%
\pgfpathcurveto{\pgfqpoint{2.703321in}{3.379389in}}{\pgfqpoint{2.692722in}{3.383779in}}{\pgfqpoint{2.681672in}{3.383779in}}%
\pgfpathcurveto{\pgfqpoint{2.670622in}{3.383779in}}{\pgfqpoint{2.660023in}{3.379389in}}{\pgfqpoint{2.652209in}{3.371575in}}%
\pgfpathcurveto{\pgfqpoint{2.644396in}{3.363762in}}{\pgfqpoint{2.640006in}{3.353163in}}{\pgfqpoint{2.640006in}{3.342113in}}%
\pgfpathcurveto{\pgfqpoint{2.640006in}{3.331063in}}{\pgfqpoint{2.644396in}{3.320463in}}{\pgfqpoint{2.652209in}{3.312650in}}%
\pgfpathcurveto{\pgfqpoint{2.660023in}{3.304836in}}{\pgfqpoint{2.670622in}{3.300446in}}{\pgfqpoint{2.681672in}{3.300446in}}%
\pgfpathclose%
\pgfusepath{stroke,fill}%
\end{pgfscope}%
\begin{pgfscope}%
\pgfpathrectangle{\pgfqpoint{0.600000in}{0.600000in}}{\pgfqpoint{3.900000in}{3.900000in}}%
\pgfusepath{clip}%
\pgfsetbuttcap%
\pgfsetroundjoin%
\definecolor{currentfill}{rgb}{0.121569,0.466667,0.705882}%
\pgfsetfillcolor{currentfill}%
\pgfsetlinewidth{1.003750pt}%
\definecolor{currentstroke}{rgb}{0.121569,0.466667,0.705882}%
\pgfsetstrokecolor{currentstroke}%
\pgfsetdash{}{0pt}%
\pgfpathmoveto{\pgfqpoint{2.482792in}{2.762550in}}%
\pgfpathcurveto{\pgfqpoint{2.493842in}{2.762550in}}{\pgfqpoint{2.504441in}{2.766941in}}{\pgfqpoint{2.512255in}{2.774754in}}%
\pgfpathcurveto{\pgfqpoint{2.520068in}{2.782568in}}{\pgfqpoint{2.524459in}{2.793167in}}{\pgfqpoint{2.524459in}{2.804217in}}%
\pgfpathcurveto{\pgfqpoint{2.524459in}{2.815267in}}{\pgfqpoint{2.520068in}{2.825866in}}{\pgfqpoint{2.512255in}{2.833680in}}%
\pgfpathcurveto{\pgfqpoint{2.504441in}{2.841493in}}{\pgfqpoint{2.493842in}{2.845884in}}{\pgfqpoint{2.482792in}{2.845884in}}%
\pgfpathcurveto{\pgfqpoint{2.471742in}{2.845884in}}{\pgfqpoint{2.461143in}{2.841493in}}{\pgfqpoint{2.453329in}{2.833680in}}%
\pgfpathcurveto{\pgfqpoint{2.445516in}{2.825866in}}{\pgfqpoint{2.441125in}{2.815267in}}{\pgfqpoint{2.441125in}{2.804217in}}%
\pgfpathcurveto{\pgfqpoint{2.441125in}{2.793167in}}{\pgfqpoint{2.445516in}{2.782568in}}{\pgfqpoint{2.453329in}{2.774754in}}%
\pgfpathcurveto{\pgfqpoint{2.461143in}{2.766941in}}{\pgfqpoint{2.471742in}{2.762550in}}{\pgfqpoint{2.482792in}{2.762550in}}%
\pgfpathclose%
\pgfusepath{stroke,fill}%
\end{pgfscope}%
\begin{pgfscope}%
\pgfpathrectangle{\pgfqpoint{0.600000in}{0.600000in}}{\pgfqpoint{3.900000in}{3.900000in}}%
\pgfusepath{clip}%
\pgfsetbuttcap%
\pgfsetroundjoin%
\definecolor{currentfill}{rgb}{0.121569,0.466667,0.705882}%
\pgfsetfillcolor{currentfill}%
\pgfsetlinewidth{1.003750pt}%
\definecolor{currentstroke}{rgb}{0.121569,0.466667,0.705882}%
\pgfsetstrokecolor{currentstroke}%
\pgfsetdash{}{0pt}%
\pgfpathmoveto{\pgfqpoint{3.087820in}{2.721562in}}%
\pgfpathcurveto{\pgfqpoint{3.098870in}{2.721562in}}{\pgfqpoint{3.109469in}{2.725953in}}{\pgfqpoint{3.117283in}{2.733766in}}%
\pgfpathcurveto{\pgfqpoint{3.125096in}{2.741580in}}{\pgfqpoint{3.129487in}{2.752179in}}{\pgfqpoint{3.129487in}{2.763229in}}%
\pgfpathcurveto{\pgfqpoint{3.129487in}{2.774279in}}{\pgfqpoint{3.125096in}{2.784878in}}{\pgfqpoint{3.117283in}{2.792692in}}%
\pgfpathcurveto{\pgfqpoint{3.109469in}{2.800506in}}{\pgfqpoint{3.098870in}{2.804896in}}{\pgfqpoint{3.087820in}{2.804896in}}%
\pgfpathcurveto{\pgfqpoint{3.076770in}{2.804896in}}{\pgfqpoint{3.066171in}{2.800506in}}{\pgfqpoint{3.058357in}{2.792692in}}%
\pgfpathcurveto{\pgfqpoint{3.050544in}{2.784878in}}{\pgfqpoint{3.046153in}{2.774279in}}{\pgfqpoint{3.046153in}{2.763229in}}%
\pgfpathcurveto{\pgfqpoint{3.046153in}{2.752179in}}{\pgfqpoint{3.050544in}{2.741580in}}{\pgfqpoint{3.058357in}{2.733766in}}%
\pgfpathcurveto{\pgfqpoint{3.066171in}{2.725953in}}{\pgfqpoint{3.076770in}{2.721562in}}{\pgfqpoint{3.087820in}{2.721562in}}%
\pgfpathclose%
\pgfusepath{stroke,fill}%
\end{pgfscope}%
\begin{pgfscope}%
\pgfpathrectangle{\pgfqpoint{0.600000in}{0.600000in}}{\pgfqpoint{3.900000in}{3.900000in}}%
\pgfusepath{clip}%
\pgfsetbuttcap%
\pgfsetroundjoin%
\definecolor{currentfill}{rgb}{0.121569,0.466667,0.705882}%
\pgfsetfillcolor{currentfill}%
\pgfsetlinewidth{1.003750pt}%
\definecolor{currentstroke}{rgb}{0.121569,0.466667,0.705882}%
\pgfsetstrokecolor{currentstroke}%
\pgfsetdash{}{0pt}%
\pgfpathmoveto{\pgfqpoint{2.257149in}{2.208348in}}%
\pgfpathcurveto{\pgfqpoint{2.268199in}{2.208348in}}{\pgfqpoint{2.278798in}{2.212738in}}{\pgfqpoint{2.286612in}{2.220551in}}%
\pgfpathcurveto{\pgfqpoint{2.294425in}{2.228365in}}{\pgfqpoint{2.298816in}{2.238964in}}{\pgfqpoint{2.298816in}{2.250014in}}%
\pgfpathcurveto{\pgfqpoint{2.298816in}{2.261064in}}{\pgfqpoint{2.294425in}{2.271663in}}{\pgfqpoint{2.286612in}{2.279477in}}%
\pgfpathcurveto{\pgfqpoint{2.278798in}{2.287291in}}{\pgfqpoint{2.268199in}{2.291681in}}{\pgfqpoint{2.257149in}{2.291681in}}%
\pgfpathcurveto{\pgfqpoint{2.246099in}{2.291681in}}{\pgfqpoint{2.235500in}{2.287291in}}{\pgfqpoint{2.227686in}{2.279477in}}%
\pgfpathcurveto{\pgfqpoint{2.219872in}{2.271663in}}{\pgfqpoint{2.215482in}{2.261064in}}{\pgfqpoint{2.215482in}{2.250014in}}%
\pgfpathcurveto{\pgfqpoint{2.215482in}{2.238964in}}{\pgfqpoint{2.219872in}{2.228365in}}{\pgfqpoint{2.227686in}{2.220551in}}%
\pgfpathcurveto{\pgfqpoint{2.235500in}{2.212738in}}{\pgfqpoint{2.246099in}{2.208348in}}{\pgfqpoint{2.257149in}{2.208348in}}%
\pgfpathclose%
\pgfusepath{stroke,fill}%
\end{pgfscope}%
\begin{pgfscope}%
\pgfpathrectangle{\pgfqpoint{0.600000in}{0.600000in}}{\pgfqpoint{3.900000in}{3.900000in}}%
\pgfusepath{clip}%
\pgfsetbuttcap%
\pgfsetroundjoin%
\definecolor{currentfill}{rgb}{0.121569,0.466667,0.705882}%
\pgfsetfillcolor{currentfill}%
\pgfsetlinewidth{1.003750pt}%
\definecolor{currentstroke}{rgb}{0.121569,0.466667,0.705882}%
\pgfsetstrokecolor{currentstroke}%
\pgfsetdash{}{0pt}%
\pgfpathmoveto{\pgfqpoint{1.843278in}{2.660176in}}%
\pgfpathcurveto{\pgfqpoint{1.854328in}{2.660176in}}{\pgfqpoint{1.864927in}{2.664566in}}{\pgfqpoint{1.872741in}{2.672380in}}%
\pgfpathcurveto{\pgfqpoint{1.880555in}{2.680193in}}{\pgfqpoint{1.884945in}{2.690792in}}{\pgfqpoint{1.884945in}{2.701842in}}%
\pgfpathcurveto{\pgfqpoint{1.884945in}{2.712892in}}{\pgfqpoint{1.880555in}{2.723492in}}{\pgfqpoint{1.872741in}{2.731305in}}%
\pgfpathcurveto{\pgfqpoint{1.864927in}{2.739119in}}{\pgfqpoint{1.854328in}{2.743509in}}{\pgfqpoint{1.843278in}{2.743509in}}%
\pgfpathcurveto{\pgfqpoint{1.832228in}{2.743509in}}{\pgfqpoint{1.821629in}{2.739119in}}{\pgfqpoint{1.813815in}{2.731305in}}%
\pgfpathcurveto{\pgfqpoint{1.806002in}{2.723492in}}{\pgfqpoint{1.801612in}{2.712892in}}{\pgfqpoint{1.801612in}{2.701842in}}%
\pgfpathcurveto{\pgfqpoint{1.801612in}{2.690792in}}{\pgfqpoint{1.806002in}{2.680193in}}{\pgfqpoint{1.813815in}{2.672380in}}%
\pgfpathcurveto{\pgfqpoint{1.821629in}{2.664566in}}{\pgfqpoint{1.832228in}{2.660176in}}{\pgfqpoint{1.843278in}{2.660176in}}%
\pgfpathclose%
\pgfusepath{stroke,fill}%
\end{pgfscope}%
\begin{pgfscope}%
\pgfpathrectangle{\pgfqpoint{0.600000in}{0.600000in}}{\pgfqpoint{3.900000in}{3.900000in}}%
\pgfusepath{clip}%
\pgfsetbuttcap%
\pgfsetroundjoin%
\definecolor{currentfill}{rgb}{0.121569,0.466667,0.705882}%
\pgfsetfillcolor{currentfill}%
\pgfsetlinewidth{1.003750pt}%
\definecolor{currentstroke}{rgb}{0.121569,0.466667,0.705882}%
\pgfsetstrokecolor{currentstroke}%
\pgfsetdash{}{0pt}%
\pgfpathmoveto{\pgfqpoint{3.064994in}{1.985429in}}%
\pgfpathcurveto{\pgfqpoint{3.076044in}{1.985429in}}{\pgfqpoint{3.086643in}{1.989819in}}{\pgfqpoint{3.094457in}{1.997633in}}%
\pgfpathcurveto{\pgfqpoint{3.102270in}{2.005447in}}{\pgfqpoint{3.106660in}{2.016046in}}{\pgfqpoint{3.106660in}{2.027096in}}%
\pgfpathcurveto{\pgfqpoint{3.106660in}{2.038146in}}{\pgfqpoint{3.102270in}{2.048745in}}{\pgfqpoint{3.094457in}{2.056559in}}%
\pgfpathcurveto{\pgfqpoint{3.086643in}{2.064372in}}{\pgfqpoint{3.076044in}{2.068762in}}{\pgfqpoint{3.064994in}{2.068762in}}%
\pgfpathcurveto{\pgfqpoint{3.053944in}{2.068762in}}{\pgfqpoint{3.043345in}{2.064372in}}{\pgfqpoint{3.035531in}{2.056559in}}%
\pgfpathcurveto{\pgfqpoint{3.027717in}{2.048745in}}{\pgfqpoint{3.023327in}{2.038146in}}{\pgfqpoint{3.023327in}{2.027096in}}%
\pgfpathcurveto{\pgfqpoint{3.023327in}{2.016046in}}{\pgfqpoint{3.027717in}{2.005447in}}{\pgfqpoint{3.035531in}{1.997633in}}%
\pgfpathcurveto{\pgfqpoint{3.043345in}{1.989819in}}{\pgfqpoint{3.053944in}{1.985429in}}{\pgfqpoint{3.064994in}{1.985429in}}%
\pgfpathclose%
\pgfusepath{stroke,fill}%
\end{pgfscope}%
\begin{pgfscope}%
\pgfpathrectangle{\pgfqpoint{0.600000in}{0.600000in}}{\pgfqpoint{3.900000in}{3.900000in}}%
\pgfusepath{clip}%
\pgfsetbuttcap%
\pgfsetroundjoin%
\definecolor{currentfill}{rgb}{0.121569,0.466667,0.705882}%
\pgfsetfillcolor{currentfill}%
\pgfsetlinewidth{1.003750pt}%
\definecolor{currentstroke}{rgb}{0.121569,0.466667,0.705882}%
\pgfsetstrokecolor{currentstroke}%
\pgfsetdash{}{0pt}%
\pgfpathmoveto{\pgfqpoint{1.462584in}{2.009504in}}%
\pgfpathcurveto{\pgfqpoint{1.473634in}{2.009504in}}{\pgfqpoint{1.484233in}{2.013894in}}{\pgfqpoint{1.492047in}{2.021708in}}%
\pgfpathcurveto{\pgfqpoint{1.499861in}{2.029522in}}{\pgfqpoint{1.504251in}{2.040121in}}{\pgfqpoint{1.504251in}{2.051171in}}%
\pgfpathcurveto{\pgfqpoint{1.504251in}{2.062221in}}{\pgfqpoint{1.499861in}{2.072820in}}{\pgfqpoint{1.492047in}{2.080634in}}%
\pgfpathcurveto{\pgfqpoint{1.484233in}{2.088447in}}{\pgfqpoint{1.473634in}{2.092837in}}{\pgfqpoint{1.462584in}{2.092837in}}%
\pgfpathcurveto{\pgfqpoint{1.451534in}{2.092837in}}{\pgfqpoint{1.440935in}{2.088447in}}{\pgfqpoint{1.433121in}{2.080634in}}%
\pgfpathcurveto{\pgfqpoint{1.425308in}{2.072820in}}{\pgfqpoint{1.420917in}{2.062221in}}{\pgfqpoint{1.420917in}{2.051171in}}%
\pgfpathcurveto{\pgfqpoint{1.420917in}{2.040121in}}{\pgfqpoint{1.425308in}{2.029522in}}{\pgfqpoint{1.433121in}{2.021708in}}%
\pgfpathcurveto{\pgfqpoint{1.440935in}{2.013894in}}{\pgfqpoint{1.451534in}{2.009504in}}{\pgfqpoint{1.462584in}{2.009504in}}%
\pgfpathclose%
\pgfusepath{stroke,fill}%
\end{pgfscope}%
\begin{pgfscope}%
\pgfpathrectangle{\pgfqpoint{0.600000in}{0.600000in}}{\pgfqpoint{3.900000in}{3.900000in}}%
\pgfusepath{clip}%
\pgfsetbuttcap%
\pgfsetroundjoin%
\definecolor{currentfill}{rgb}{0.121569,0.466667,0.705882}%
\pgfsetfillcolor{currentfill}%
\pgfsetlinewidth{1.003750pt}%
\definecolor{currentstroke}{rgb}{0.121569,0.466667,0.705882}%
\pgfsetstrokecolor{currentstroke}%
\pgfsetdash{}{0pt}%
\pgfpathmoveto{\pgfqpoint{2.492606in}{4.017793in}}%
\pgfpathcurveto{\pgfqpoint{2.503656in}{4.017793in}}{\pgfqpoint{2.514255in}{4.022184in}}{\pgfqpoint{2.522069in}{4.029997in}}%
\pgfpathcurveto{\pgfqpoint{2.529882in}{4.037811in}}{\pgfqpoint{2.534273in}{4.048410in}}{\pgfqpoint{2.534273in}{4.059460in}}%
\pgfpathcurveto{\pgfqpoint{2.534273in}{4.070510in}}{\pgfqpoint{2.529882in}{4.081109in}}{\pgfqpoint{2.522069in}{4.088923in}}%
\pgfpathcurveto{\pgfqpoint{2.514255in}{4.096736in}}{\pgfqpoint{2.503656in}{4.101127in}}{\pgfqpoint{2.492606in}{4.101127in}}%
\pgfpathcurveto{\pgfqpoint{2.481556in}{4.101127in}}{\pgfqpoint{2.470957in}{4.096736in}}{\pgfqpoint{2.463143in}{4.088923in}}%
\pgfpathcurveto{\pgfqpoint{2.455330in}{4.081109in}}{\pgfqpoint{2.450939in}{4.070510in}}{\pgfqpoint{2.450939in}{4.059460in}}%
\pgfpathcurveto{\pgfqpoint{2.450939in}{4.048410in}}{\pgfqpoint{2.455330in}{4.037811in}}{\pgfqpoint{2.463143in}{4.029997in}}%
\pgfpathcurveto{\pgfqpoint{2.470957in}{4.022184in}}{\pgfqpoint{2.481556in}{4.017793in}}{\pgfqpoint{2.492606in}{4.017793in}}%
\pgfpathclose%
\pgfusepath{stroke,fill}%
\end{pgfscope}%
\begin{pgfscope}%
\pgfpathrectangle{\pgfqpoint{0.600000in}{0.600000in}}{\pgfqpoint{3.900000in}{3.900000in}}%
\pgfusepath{clip}%
\pgfsetbuttcap%
\pgfsetroundjoin%
\definecolor{currentfill}{rgb}{0.121569,0.466667,0.705882}%
\pgfsetfillcolor{currentfill}%
\pgfsetlinewidth{1.003750pt}%
\definecolor{currentstroke}{rgb}{0.121569,0.466667,0.705882}%
\pgfsetstrokecolor{currentstroke}%
\pgfsetdash{}{0pt}%
\pgfpathmoveto{\pgfqpoint{2.804012in}{1.894448in}}%
\pgfpathcurveto{\pgfqpoint{2.815062in}{1.894448in}}{\pgfqpoint{2.825661in}{1.898839in}}{\pgfqpoint{2.833475in}{1.906652in}}%
\pgfpathcurveto{\pgfqpoint{2.841288in}{1.914466in}}{\pgfqpoint{2.845679in}{1.925065in}}{\pgfqpoint{2.845679in}{1.936115in}}%
\pgfpathcurveto{\pgfqpoint{2.845679in}{1.947165in}}{\pgfqpoint{2.841288in}{1.957764in}}{\pgfqpoint{2.833475in}{1.965578in}}%
\pgfpathcurveto{\pgfqpoint{2.825661in}{1.973391in}}{\pgfqpoint{2.815062in}{1.977782in}}{\pgfqpoint{2.804012in}{1.977782in}}%
\pgfpathcurveto{\pgfqpoint{2.792962in}{1.977782in}}{\pgfqpoint{2.782363in}{1.973391in}}{\pgfqpoint{2.774549in}{1.965578in}}%
\pgfpathcurveto{\pgfqpoint{2.766736in}{1.957764in}}{\pgfqpoint{2.762345in}{1.947165in}}{\pgfqpoint{2.762345in}{1.936115in}}%
\pgfpathcurveto{\pgfqpoint{2.762345in}{1.925065in}}{\pgfqpoint{2.766736in}{1.914466in}}{\pgfqpoint{2.774549in}{1.906652in}}%
\pgfpathcurveto{\pgfqpoint{2.782363in}{1.898839in}}{\pgfqpoint{2.792962in}{1.894448in}}{\pgfqpoint{2.804012in}{1.894448in}}%
\pgfpathclose%
\pgfusepath{stroke,fill}%
\end{pgfscope}%
\begin{pgfscope}%
\pgfpathrectangle{\pgfqpoint{0.600000in}{0.600000in}}{\pgfqpoint{3.900000in}{3.900000in}}%
\pgfusepath{clip}%
\pgfsetbuttcap%
\pgfsetroundjoin%
\definecolor{currentfill}{rgb}{0.121569,0.466667,0.705882}%
\pgfsetfillcolor{currentfill}%
\pgfsetlinewidth{1.003750pt}%
\definecolor{currentstroke}{rgb}{0.121569,0.466667,0.705882}%
\pgfsetstrokecolor{currentstroke}%
\pgfsetdash{}{0pt}%
\pgfpathmoveto{\pgfqpoint{2.374861in}{2.180582in}}%
\pgfpathcurveto{\pgfqpoint{2.385911in}{2.180582in}}{\pgfqpoint{2.396510in}{2.184972in}}{\pgfqpoint{2.404324in}{2.192786in}}%
\pgfpathcurveto{\pgfqpoint{2.412137in}{2.200599in}}{\pgfqpoint{2.416528in}{2.211198in}}{\pgfqpoint{2.416528in}{2.222248in}}%
\pgfpathcurveto{\pgfqpoint{2.416528in}{2.233299in}}{\pgfqpoint{2.412137in}{2.243898in}}{\pgfqpoint{2.404324in}{2.251711in}}%
\pgfpathcurveto{\pgfqpoint{2.396510in}{2.259525in}}{\pgfqpoint{2.385911in}{2.263915in}}{\pgfqpoint{2.374861in}{2.263915in}}%
\pgfpathcurveto{\pgfqpoint{2.363811in}{2.263915in}}{\pgfqpoint{2.353212in}{2.259525in}}{\pgfqpoint{2.345398in}{2.251711in}}%
\pgfpathcurveto{\pgfqpoint{2.337585in}{2.243898in}}{\pgfqpoint{2.333194in}{2.233299in}}{\pgfqpoint{2.333194in}{2.222248in}}%
\pgfpathcurveto{\pgfqpoint{2.333194in}{2.211198in}}{\pgfqpoint{2.337585in}{2.200599in}}{\pgfqpoint{2.345398in}{2.192786in}}%
\pgfpathcurveto{\pgfqpoint{2.353212in}{2.184972in}}{\pgfqpoint{2.363811in}{2.180582in}}{\pgfqpoint{2.374861in}{2.180582in}}%
\pgfpathclose%
\pgfusepath{stroke,fill}%
\end{pgfscope}%
\begin{pgfscope}%
\pgfpathrectangle{\pgfqpoint{0.600000in}{0.600000in}}{\pgfqpoint{3.900000in}{3.900000in}}%
\pgfusepath{clip}%
\pgfsetbuttcap%
\pgfsetroundjoin%
\definecolor{currentfill}{rgb}{0.121569,0.466667,0.705882}%
\pgfsetfillcolor{currentfill}%
\pgfsetlinewidth{1.003750pt}%
\definecolor{currentstroke}{rgb}{0.121569,0.466667,0.705882}%
\pgfsetstrokecolor{currentstroke}%
\pgfsetdash{}{0pt}%
\pgfpathmoveto{\pgfqpoint{3.244392in}{2.828717in}}%
\pgfpathcurveto{\pgfqpoint{3.255442in}{2.828717in}}{\pgfqpoint{3.266041in}{2.833107in}}{\pgfqpoint{3.273854in}{2.840921in}}%
\pgfpathcurveto{\pgfqpoint{3.281668in}{2.848734in}}{\pgfqpoint{3.286058in}{2.859333in}}{\pgfqpoint{3.286058in}{2.870384in}}%
\pgfpathcurveto{\pgfqpoint{3.286058in}{2.881434in}}{\pgfqpoint{3.281668in}{2.892033in}}{\pgfqpoint{3.273854in}{2.899846in}}%
\pgfpathcurveto{\pgfqpoint{3.266041in}{2.907660in}}{\pgfqpoint{3.255442in}{2.912050in}}{\pgfqpoint{3.244392in}{2.912050in}}%
\pgfpathcurveto{\pgfqpoint{3.233341in}{2.912050in}}{\pgfqpoint{3.222742in}{2.907660in}}{\pgfqpoint{3.214929in}{2.899846in}}%
\pgfpathcurveto{\pgfqpoint{3.207115in}{2.892033in}}{\pgfqpoint{3.202725in}{2.881434in}}{\pgfqpoint{3.202725in}{2.870384in}}%
\pgfpathcurveto{\pgfqpoint{3.202725in}{2.859333in}}{\pgfqpoint{3.207115in}{2.848734in}}{\pgfqpoint{3.214929in}{2.840921in}}%
\pgfpathcurveto{\pgfqpoint{3.222742in}{2.833107in}}{\pgfqpoint{3.233341in}{2.828717in}}{\pgfqpoint{3.244392in}{2.828717in}}%
\pgfpathclose%
\pgfusepath{stroke,fill}%
\end{pgfscope}%
\begin{pgfscope}%
\pgfpathrectangle{\pgfqpoint{0.600000in}{0.600000in}}{\pgfqpoint{3.900000in}{3.900000in}}%
\pgfusepath{clip}%
\pgfsetbuttcap%
\pgfsetroundjoin%
\definecolor{currentfill}{rgb}{0.121569,0.466667,0.705882}%
\pgfsetfillcolor{currentfill}%
\pgfsetlinewidth{1.003750pt}%
\definecolor{currentstroke}{rgb}{0.121569,0.466667,0.705882}%
\pgfsetstrokecolor{currentstroke}%
\pgfsetdash{}{0pt}%
\pgfpathmoveto{\pgfqpoint{2.696430in}{2.143164in}}%
\pgfpathcurveto{\pgfqpoint{2.707480in}{2.143164in}}{\pgfqpoint{2.718079in}{2.147554in}}{\pgfqpoint{2.725893in}{2.155368in}}%
\pgfpathcurveto{\pgfqpoint{2.733707in}{2.163181in}}{\pgfqpoint{2.738097in}{2.173780in}}{\pgfqpoint{2.738097in}{2.184831in}}%
\pgfpathcurveto{\pgfqpoint{2.738097in}{2.195881in}}{\pgfqpoint{2.733707in}{2.206480in}}{\pgfqpoint{2.725893in}{2.214293in}}%
\pgfpathcurveto{\pgfqpoint{2.718079in}{2.222107in}}{\pgfqpoint{2.707480in}{2.226497in}}{\pgfqpoint{2.696430in}{2.226497in}}%
\pgfpathcurveto{\pgfqpoint{2.685380in}{2.226497in}}{\pgfqpoint{2.674781in}{2.222107in}}{\pgfqpoint{2.666967in}{2.214293in}}%
\pgfpathcurveto{\pgfqpoint{2.659154in}{2.206480in}}{\pgfqpoint{2.654764in}{2.195881in}}{\pgfqpoint{2.654764in}{2.184831in}}%
\pgfpathcurveto{\pgfqpoint{2.654764in}{2.173780in}}{\pgfqpoint{2.659154in}{2.163181in}}{\pgfqpoint{2.666967in}{2.155368in}}%
\pgfpathcurveto{\pgfqpoint{2.674781in}{2.147554in}}{\pgfqpoint{2.685380in}{2.143164in}}{\pgfqpoint{2.696430in}{2.143164in}}%
\pgfpathclose%
\pgfusepath{stroke,fill}%
\end{pgfscope}%
\begin{pgfscope}%
\pgfpathrectangle{\pgfqpoint{0.600000in}{0.600000in}}{\pgfqpoint{3.900000in}{3.900000in}}%
\pgfusepath{clip}%
\pgfsetbuttcap%
\pgfsetroundjoin%
\definecolor{currentfill}{rgb}{0.121569,0.466667,0.705882}%
\pgfsetfillcolor{currentfill}%
\pgfsetlinewidth{1.003750pt}%
\definecolor{currentstroke}{rgb}{0.121569,0.466667,0.705882}%
\pgfsetstrokecolor{currentstroke}%
\pgfsetdash{}{0pt}%
\pgfpathmoveto{\pgfqpoint{2.600932in}{2.499050in}}%
\pgfpathcurveto{\pgfqpoint{2.611982in}{2.499050in}}{\pgfqpoint{2.622581in}{2.503440in}}{\pgfqpoint{2.630395in}{2.511254in}}%
\pgfpathcurveto{\pgfqpoint{2.638209in}{2.519068in}}{\pgfqpoint{2.642599in}{2.529667in}}{\pgfqpoint{2.642599in}{2.540717in}}%
\pgfpathcurveto{\pgfqpoint{2.642599in}{2.551767in}}{\pgfqpoint{2.638209in}{2.562366in}}{\pgfqpoint{2.630395in}{2.570180in}}%
\pgfpathcurveto{\pgfqpoint{2.622581in}{2.577993in}}{\pgfqpoint{2.611982in}{2.582383in}}{\pgfqpoint{2.600932in}{2.582383in}}%
\pgfpathcurveto{\pgfqpoint{2.589882in}{2.582383in}}{\pgfqpoint{2.579283in}{2.577993in}}{\pgfqpoint{2.571469in}{2.570180in}}%
\pgfpathcurveto{\pgfqpoint{2.563656in}{2.562366in}}{\pgfqpoint{2.559266in}{2.551767in}}{\pgfqpoint{2.559266in}{2.540717in}}%
\pgfpathcurveto{\pgfqpoint{2.559266in}{2.529667in}}{\pgfqpoint{2.563656in}{2.519068in}}{\pgfqpoint{2.571469in}{2.511254in}}%
\pgfpathcurveto{\pgfqpoint{2.579283in}{2.503440in}}{\pgfqpoint{2.589882in}{2.499050in}}{\pgfqpoint{2.600932in}{2.499050in}}%
\pgfpathclose%
\pgfusepath{stroke,fill}%
\end{pgfscope}%
\begin{pgfscope}%
\pgfpathrectangle{\pgfqpoint{0.600000in}{0.600000in}}{\pgfqpoint{3.900000in}{3.900000in}}%
\pgfusepath{clip}%
\pgfsetbuttcap%
\pgfsetroundjoin%
\definecolor{currentfill}{rgb}{0.121569,0.466667,0.705882}%
\pgfsetfillcolor{currentfill}%
\pgfsetlinewidth{1.003750pt}%
\definecolor{currentstroke}{rgb}{0.121569,0.466667,0.705882}%
\pgfsetstrokecolor{currentstroke}%
\pgfsetdash{}{0pt}%
\pgfpathmoveto{\pgfqpoint{2.691990in}{2.487353in}}%
\pgfpathcurveto{\pgfqpoint{2.703041in}{2.487353in}}{\pgfqpoint{2.713640in}{2.491743in}}{\pgfqpoint{2.721453in}{2.499556in}}%
\pgfpathcurveto{\pgfqpoint{2.729267in}{2.507370in}}{\pgfqpoint{2.733657in}{2.517969in}}{\pgfqpoint{2.733657in}{2.529019in}}%
\pgfpathcurveto{\pgfqpoint{2.733657in}{2.540069in}}{\pgfqpoint{2.729267in}{2.550668in}}{\pgfqpoint{2.721453in}{2.558482in}}%
\pgfpathcurveto{\pgfqpoint{2.713640in}{2.566296in}}{\pgfqpoint{2.703041in}{2.570686in}}{\pgfqpoint{2.691990in}{2.570686in}}%
\pgfpathcurveto{\pgfqpoint{2.680940in}{2.570686in}}{\pgfqpoint{2.670341in}{2.566296in}}{\pgfqpoint{2.662528in}{2.558482in}}%
\pgfpathcurveto{\pgfqpoint{2.654714in}{2.550668in}}{\pgfqpoint{2.650324in}{2.540069in}}{\pgfqpoint{2.650324in}{2.529019in}}%
\pgfpathcurveto{\pgfqpoint{2.650324in}{2.517969in}}{\pgfqpoint{2.654714in}{2.507370in}}{\pgfqpoint{2.662528in}{2.499556in}}%
\pgfpathcurveto{\pgfqpoint{2.670341in}{2.491743in}}{\pgfqpoint{2.680940in}{2.487353in}}{\pgfqpoint{2.691990in}{2.487353in}}%
\pgfpathclose%
\pgfusepath{stroke,fill}%
\end{pgfscope}%
\begin{pgfscope}%
\pgfpathrectangle{\pgfqpoint{0.600000in}{0.600000in}}{\pgfqpoint{3.900000in}{3.900000in}}%
\pgfusepath{clip}%
\pgfsetbuttcap%
\pgfsetroundjoin%
\definecolor{currentfill}{rgb}{0.121569,0.466667,0.705882}%
\pgfsetfillcolor{currentfill}%
\pgfsetlinewidth{1.003750pt}%
\definecolor{currentstroke}{rgb}{0.121569,0.466667,0.705882}%
\pgfsetstrokecolor{currentstroke}%
\pgfsetdash{}{0pt}%
\pgfpathmoveto{\pgfqpoint{2.830932in}{2.136189in}}%
\pgfpathcurveto{\pgfqpoint{2.841982in}{2.136189in}}{\pgfqpoint{2.852581in}{2.140579in}}{\pgfqpoint{2.860394in}{2.148393in}}%
\pgfpathcurveto{\pgfqpoint{2.868208in}{2.156207in}}{\pgfqpoint{2.872598in}{2.166806in}}{\pgfqpoint{2.872598in}{2.177856in}}%
\pgfpathcurveto{\pgfqpoint{2.872598in}{2.188906in}}{\pgfqpoint{2.868208in}{2.199505in}}{\pgfqpoint{2.860394in}{2.207319in}}%
\pgfpathcurveto{\pgfqpoint{2.852581in}{2.215132in}}{\pgfqpoint{2.841982in}{2.219523in}}{\pgfqpoint{2.830932in}{2.219523in}}%
\pgfpathcurveto{\pgfqpoint{2.819882in}{2.219523in}}{\pgfqpoint{2.809283in}{2.215132in}}{\pgfqpoint{2.801469in}{2.207319in}}%
\pgfpathcurveto{\pgfqpoint{2.793655in}{2.199505in}}{\pgfqpoint{2.789265in}{2.188906in}}{\pgfqpoint{2.789265in}{2.177856in}}%
\pgfpathcurveto{\pgfqpoint{2.789265in}{2.166806in}}{\pgfqpoint{2.793655in}{2.156207in}}{\pgfqpoint{2.801469in}{2.148393in}}%
\pgfpathcurveto{\pgfqpoint{2.809283in}{2.140579in}}{\pgfqpoint{2.819882in}{2.136189in}}{\pgfqpoint{2.830932in}{2.136189in}}%
\pgfpathclose%
\pgfusepath{stroke,fill}%
\end{pgfscope}%
\begin{pgfscope}%
\pgfpathrectangle{\pgfqpoint{0.600000in}{0.600000in}}{\pgfqpoint{3.900000in}{3.900000in}}%
\pgfusepath{clip}%
\pgfsetbuttcap%
\pgfsetroundjoin%
\definecolor{currentfill}{rgb}{0.121569,0.466667,0.705882}%
\pgfsetfillcolor{currentfill}%
\pgfsetlinewidth{1.003750pt}%
\definecolor{currentstroke}{rgb}{0.121569,0.466667,0.705882}%
\pgfsetstrokecolor{currentstroke}%
\pgfsetdash{}{0pt}%
\pgfpathmoveto{\pgfqpoint{2.360143in}{2.347428in}}%
\pgfpathcurveto{\pgfqpoint{2.371193in}{2.347428in}}{\pgfqpoint{2.381792in}{2.351818in}}{\pgfqpoint{2.389606in}{2.359632in}}%
\pgfpathcurveto{\pgfqpoint{2.397420in}{2.367445in}}{\pgfqpoint{2.401810in}{2.378044in}}{\pgfqpoint{2.401810in}{2.389094in}}%
\pgfpathcurveto{\pgfqpoint{2.401810in}{2.400145in}}{\pgfqpoint{2.397420in}{2.410744in}}{\pgfqpoint{2.389606in}{2.418557in}}%
\pgfpathcurveto{\pgfqpoint{2.381792in}{2.426371in}}{\pgfqpoint{2.371193in}{2.430761in}}{\pgfqpoint{2.360143in}{2.430761in}}%
\pgfpathcurveto{\pgfqpoint{2.349093in}{2.430761in}}{\pgfqpoint{2.338494in}{2.426371in}}{\pgfqpoint{2.330680in}{2.418557in}}%
\pgfpathcurveto{\pgfqpoint{2.322867in}{2.410744in}}{\pgfqpoint{2.318477in}{2.400145in}}{\pgfqpoint{2.318477in}{2.389094in}}%
\pgfpathcurveto{\pgfqpoint{2.318477in}{2.378044in}}{\pgfqpoint{2.322867in}{2.367445in}}{\pgfqpoint{2.330680in}{2.359632in}}%
\pgfpathcurveto{\pgfqpoint{2.338494in}{2.351818in}}{\pgfqpoint{2.349093in}{2.347428in}}{\pgfqpoint{2.360143in}{2.347428in}}%
\pgfpathclose%
\pgfusepath{stroke,fill}%
\end{pgfscope}%
\begin{pgfscope}%
\pgfpathrectangle{\pgfqpoint{0.600000in}{0.600000in}}{\pgfqpoint{3.900000in}{3.900000in}}%
\pgfusepath{clip}%
\pgfsetbuttcap%
\pgfsetroundjoin%
\definecolor{currentfill}{rgb}{0.121569,0.466667,0.705882}%
\pgfsetfillcolor{currentfill}%
\pgfsetlinewidth{1.003750pt}%
\definecolor{currentstroke}{rgb}{0.121569,0.466667,0.705882}%
\pgfsetstrokecolor{currentstroke}%
\pgfsetdash{}{0pt}%
\pgfpathmoveto{\pgfqpoint{2.956523in}{1.707627in}}%
\pgfpathcurveto{\pgfqpoint{2.967574in}{1.707627in}}{\pgfqpoint{2.978173in}{1.712018in}}{\pgfqpoint{2.985986in}{1.719831in}}%
\pgfpathcurveto{\pgfqpoint{2.993800in}{1.727645in}}{\pgfqpoint{2.998190in}{1.738244in}}{\pgfqpoint{2.998190in}{1.749294in}}%
\pgfpathcurveto{\pgfqpoint{2.998190in}{1.760344in}}{\pgfqpoint{2.993800in}{1.770943in}}{\pgfqpoint{2.985986in}{1.778757in}}%
\pgfpathcurveto{\pgfqpoint{2.978173in}{1.786570in}}{\pgfqpoint{2.967574in}{1.790961in}}{\pgfqpoint{2.956523in}{1.790961in}}%
\pgfpathcurveto{\pgfqpoint{2.945473in}{1.790961in}}{\pgfqpoint{2.934874in}{1.786570in}}{\pgfqpoint{2.927061in}{1.778757in}}%
\pgfpathcurveto{\pgfqpoint{2.919247in}{1.770943in}}{\pgfqpoint{2.914857in}{1.760344in}}{\pgfqpoint{2.914857in}{1.749294in}}%
\pgfpathcurveto{\pgfqpoint{2.914857in}{1.738244in}}{\pgfqpoint{2.919247in}{1.727645in}}{\pgfqpoint{2.927061in}{1.719831in}}%
\pgfpathcurveto{\pgfqpoint{2.934874in}{1.712018in}}{\pgfqpoint{2.945473in}{1.707627in}}{\pgfqpoint{2.956523in}{1.707627in}}%
\pgfpathclose%
\pgfusepath{stroke,fill}%
\end{pgfscope}%
\begin{pgfscope}%
\pgfpathrectangle{\pgfqpoint{0.600000in}{0.600000in}}{\pgfqpoint{3.900000in}{3.900000in}}%
\pgfusepath{clip}%
\pgfsetbuttcap%
\pgfsetroundjoin%
\definecolor{currentfill}{rgb}{0.121569,0.466667,0.705882}%
\pgfsetfillcolor{currentfill}%
\pgfsetlinewidth{1.003750pt}%
\definecolor{currentstroke}{rgb}{0.121569,0.466667,0.705882}%
\pgfsetstrokecolor{currentstroke}%
\pgfsetdash{}{0pt}%
\pgfpathmoveto{\pgfqpoint{2.882958in}{2.744763in}}%
\pgfpathcurveto{\pgfqpoint{2.894008in}{2.744763in}}{\pgfqpoint{2.904607in}{2.749153in}}{\pgfqpoint{2.912421in}{2.756967in}}%
\pgfpathcurveto{\pgfqpoint{2.920235in}{2.764781in}}{\pgfqpoint{2.924625in}{2.775380in}}{\pgfqpoint{2.924625in}{2.786430in}}%
\pgfpathcurveto{\pgfqpoint{2.924625in}{2.797480in}}{\pgfqpoint{2.920235in}{2.808079in}}{\pgfqpoint{2.912421in}{2.815893in}}%
\pgfpathcurveto{\pgfqpoint{2.904607in}{2.823706in}}{\pgfqpoint{2.894008in}{2.828097in}}{\pgfqpoint{2.882958in}{2.828097in}}%
\pgfpathcurveto{\pgfqpoint{2.871908in}{2.828097in}}{\pgfqpoint{2.861309in}{2.823706in}}{\pgfqpoint{2.853495in}{2.815893in}}%
\pgfpathcurveto{\pgfqpoint{2.845682in}{2.808079in}}{\pgfqpoint{2.841292in}{2.797480in}}{\pgfqpoint{2.841292in}{2.786430in}}%
\pgfpathcurveto{\pgfqpoint{2.841292in}{2.775380in}}{\pgfqpoint{2.845682in}{2.764781in}}{\pgfqpoint{2.853495in}{2.756967in}}%
\pgfpathcurveto{\pgfqpoint{2.861309in}{2.749153in}}{\pgfqpoint{2.871908in}{2.744763in}}{\pgfqpoint{2.882958in}{2.744763in}}%
\pgfpathclose%
\pgfusepath{stroke,fill}%
\end{pgfscope}%
\begin{pgfscope}%
\pgfpathrectangle{\pgfqpoint{0.600000in}{0.600000in}}{\pgfqpoint{3.900000in}{3.900000in}}%
\pgfusepath{clip}%
\pgfsetbuttcap%
\pgfsetroundjoin%
\definecolor{currentfill}{rgb}{0.121569,0.466667,0.705882}%
\pgfsetfillcolor{currentfill}%
\pgfsetlinewidth{1.003750pt}%
\definecolor{currentstroke}{rgb}{0.121569,0.466667,0.705882}%
\pgfsetstrokecolor{currentstroke}%
\pgfsetdash{}{0pt}%
\pgfpathmoveto{\pgfqpoint{3.008964in}{2.261472in}}%
\pgfpathcurveto{\pgfqpoint{3.020014in}{2.261472in}}{\pgfqpoint{3.030613in}{2.265863in}}{\pgfqpoint{3.038427in}{2.273676in}}%
\pgfpathcurveto{\pgfqpoint{3.046240in}{2.281490in}}{\pgfqpoint{3.050631in}{2.292089in}}{\pgfqpoint{3.050631in}{2.303139in}}%
\pgfpathcurveto{\pgfqpoint{3.050631in}{2.314189in}}{\pgfqpoint{3.046240in}{2.324788in}}{\pgfqpoint{3.038427in}{2.332602in}}%
\pgfpathcurveto{\pgfqpoint{3.030613in}{2.340415in}}{\pgfqpoint{3.020014in}{2.344806in}}{\pgfqpoint{3.008964in}{2.344806in}}%
\pgfpathcurveto{\pgfqpoint{2.997914in}{2.344806in}}{\pgfqpoint{2.987315in}{2.340415in}}{\pgfqpoint{2.979501in}{2.332602in}}%
\pgfpathcurveto{\pgfqpoint{2.971688in}{2.324788in}}{\pgfqpoint{2.967297in}{2.314189in}}{\pgfqpoint{2.967297in}{2.303139in}}%
\pgfpathcurveto{\pgfqpoint{2.967297in}{2.292089in}}{\pgfqpoint{2.971688in}{2.281490in}}{\pgfqpoint{2.979501in}{2.273676in}}%
\pgfpathcurveto{\pgfqpoint{2.987315in}{2.265863in}}{\pgfqpoint{2.997914in}{2.261472in}}{\pgfqpoint{3.008964in}{2.261472in}}%
\pgfpathclose%
\pgfusepath{stroke,fill}%
\end{pgfscope}%
\begin{pgfscope}%
\pgfpathrectangle{\pgfqpoint{0.600000in}{0.600000in}}{\pgfqpoint{3.900000in}{3.900000in}}%
\pgfusepath{clip}%
\pgfsetbuttcap%
\pgfsetroundjoin%
\definecolor{currentfill}{rgb}{0.121569,0.466667,0.705882}%
\pgfsetfillcolor{currentfill}%
\pgfsetlinewidth{1.003750pt}%
\definecolor{currentstroke}{rgb}{0.121569,0.466667,0.705882}%
\pgfsetstrokecolor{currentstroke}%
\pgfsetdash{}{0pt}%
\pgfpathmoveto{\pgfqpoint{2.549380in}{2.154897in}}%
\pgfpathcurveto{\pgfqpoint{2.560430in}{2.154897in}}{\pgfqpoint{2.571029in}{2.159287in}}{\pgfqpoint{2.578843in}{2.167101in}}%
\pgfpathcurveto{\pgfqpoint{2.586656in}{2.174914in}}{\pgfqpoint{2.591047in}{2.185513in}}{\pgfqpoint{2.591047in}{2.196564in}}%
\pgfpathcurveto{\pgfqpoint{2.591047in}{2.207614in}}{\pgfqpoint{2.586656in}{2.218213in}}{\pgfqpoint{2.578843in}{2.226026in}}%
\pgfpathcurveto{\pgfqpoint{2.571029in}{2.233840in}}{\pgfqpoint{2.560430in}{2.238230in}}{\pgfqpoint{2.549380in}{2.238230in}}%
\pgfpathcurveto{\pgfqpoint{2.538330in}{2.238230in}}{\pgfqpoint{2.527731in}{2.233840in}}{\pgfqpoint{2.519917in}{2.226026in}}%
\pgfpathcurveto{\pgfqpoint{2.512104in}{2.218213in}}{\pgfqpoint{2.507713in}{2.207614in}}{\pgfqpoint{2.507713in}{2.196564in}}%
\pgfpathcurveto{\pgfqpoint{2.507713in}{2.185513in}}{\pgfqpoint{2.512104in}{2.174914in}}{\pgfqpoint{2.519917in}{2.167101in}}%
\pgfpathcurveto{\pgfqpoint{2.527731in}{2.159287in}}{\pgfqpoint{2.538330in}{2.154897in}}{\pgfqpoint{2.549380in}{2.154897in}}%
\pgfpathclose%
\pgfusepath{stroke,fill}%
\end{pgfscope}%
\begin{pgfscope}%
\pgfpathrectangle{\pgfqpoint{0.600000in}{0.600000in}}{\pgfqpoint{3.900000in}{3.900000in}}%
\pgfusepath{clip}%
\pgfsetbuttcap%
\pgfsetroundjoin%
\definecolor{currentfill}{rgb}{0.121569,0.466667,0.705882}%
\pgfsetfillcolor{currentfill}%
\pgfsetlinewidth{1.003750pt}%
\definecolor{currentstroke}{rgb}{0.121569,0.466667,0.705882}%
\pgfsetstrokecolor{currentstroke}%
\pgfsetdash{}{0pt}%
\pgfpathmoveto{\pgfqpoint{2.374100in}{1.933828in}}%
\pgfpathcurveto{\pgfqpoint{2.385150in}{1.933828in}}{\pgfqpoint{2.395749in}{1.938219in}}{\pgfqpoint{2.403563in}{1.946032in}}%
\pgfpathcurveto{\pgfqpoint{2.411376in}{1.953846in}}{\pgfqpoint{2.415767in}{1.964445in}}{\pgfqpoint{2.415767in}{1.975495in}}%
\pgfpathcurveto{\pgfqpoint{2.415767in}{1.986545in}}{\pgfqpoint{2.411376in}{1.997144in}}{\pgfqpoint{2.403563in}{2.004958in}}%
\pgfpathcurveto{\pgfqpoint{2.395749in}{2.012771in}}{\pgfqpoint{2.385150in}{2.017162in}}{\pgfqpoint{2.374100in}{2.017162in}}%
\pgfpathcurveto{\pgfqpoint{2.363050in}{2.017162in}}{\pgfqpoint{2.352451in}{2.012771in}}{\pgfqpoint{2.344637in}{2.004958in}}%
\pgfpathcurveto{\pgfqpoint{2.336823in}{1.997144in}}{\pgfqpoint{2.332433in}{1.986545in}}{\pgfqpoint{2.332433in}{1.975495in}}%
\pgfpathcurveto{\pgfqpoint{2.332433in}{1.964445in}}{\pgfqpoint{2.336823in}{1.953846in}}{\pgfqpoint{2.344637in}{1.946032in}}%
\pgfpathcurveto{\pgfqpoint{2.352451in}{1.938219in}}{\pgfqpoint{2.363050in}{1.933828in}}{\pgfqpoint{2.374100in}{1.933828in}}%
\pgfpathclose%
\pgfusepath{stroke,fill}%
\end{pgfscope}%
\begin{pgfscope}%
\pgfpathrectangle{\pgfqpoint{0.600000in}{0.600000in}}{\pgfqpoint{3.900000in}{3.900000in}}%
\pgfusepath{clip}%
\pgfsetbuttcap%
\pgfsetroundjoin%
\definecolor{currentfill}{rgb}{0.121569,0.466667,0.705882}%
\pgfsetfillcolor{currentfill}%
\pgfsetlinewidth{1.003750pt}%
\definecolor{currentstroke}{rgb}{0.121569,0.466667,0.705882}%
\pgfsetstrokecolor{currentstroke}%
\pgfsetdash{}{0pt}%
\pgfpathmoveto{\pgfqpoint{2.408041in}{3.857752in}}%
\pgfpathcurveto{\pgfqpoint{2.419091in}{3.857752in}}{\pgfqpoint{2.429690in}{3.862142in}}{\pgfqpoint{2.437504in}{3.869956in}}%
\pgfpathcurveto{\pgfqpoint{2.445318in}{3.877770in}}{\pgfqpoint{2.449708in}{3.888369in}}{\pgfqpoint{2.449708in}{3.899419in}}%
\pgfpathcurveto{\pgfqpoint{2.449708in}{3.910469in}}{\pgfqpoint{2.445318in}{3.921068in}}{\pgfqpoint{2.437504in}{3.928881in}}%
\pgfpathcurveto{\pgfqpoint{2.429690in}{3.936695in}}{\pgfqpoint{2.419091in}{3.941085in}}{\pgfqpoint{2.408041in}{3.941085in}}%
\pgfpathcurveto{\pgfqpoint{2.396991in}{3.941085in}}{\pgfqpoint{2.386392in}{3.936695in}}{\pgfqpoint{2.378579in}{3.928881in}}%
\pgfpathcurveto{\pgfqpoint{2.370765in}{3.921068in}}{\pgfqpoint{2.366375in}{3.910469in}}{\pgfqpoint{2.366375in}{3.899419in}}%
\pgfpathcurveto{\pgfqpoint{2.366375in}{3.888369in}}{\pgfqpoint{2.370765in}{3.877770in}}{\pgfqpoint{2.378579in}{3.869956in}}%
\pgfpathcurveto{\pgfqpoint{2.386392in}{3.862142in}}{\pgfqpoint{2.396991in}{3.857752in}}{\pgfqpoint{2.408041in}{3.857752in}}%
\pgfpathclose%
\pgfusepath{stroke,fill}%
\end{pgfscope}%
\begin{pgfscope}%
\pgfpathrectangle{\pgfqpoint{0.600000in}{0.600000in}}{\pgfqpoint{3.900000in}{3.900000in}}%
\pgfusepath{clip}%
\pgfsetbuttcap%
\pgfsetroundjoin%
\definecolor{currentfill}{rgb}{0.121569,0.466667,0.705882}%
\pgfsetfillcolor{currentfill}%
\pgfsetlinewidth{1.003750pt}%
\definecolor{currentstroke}{rgb}{0.121569,0.466667,0.705882}%
\pgfsetstrokecolor{currentstroke}%
\pgfsetdash{}{0pt}%
\pgfpathmoveto{\pgfqpoint{2.309648in}{1.803187in}}%
\pgfpathcurveto{\pgfqpoint{2.320698in}{1.803187in}}{\pgfqpoint{2.331297in}{1.807578in}}{\pgfqpoint{2.339111in}{1.815391in}}%
\pgfpathcurveto{\pgfqpoint{2.346925in}{1.823205in}}{\pgfqpoint{2.351315in}{1.833804in}}{\pgfqpoint{2.351315in}{1.844854in}}%
\pgfpathcurveto{\pgfqpoint{2.351315in}{1.855904in}}{\pgfqpoint{2.346925in}{1.866503in}}{\pgfqpoint{2.339111in}{1.874317in}}%
\pgfpathcurveto{\pgfqpoint{2.331297in}{1.882130in}}{\pgfqpoint{2.320698in}{1.886521in}}{\pgfqpoint{2.309648in}{1.886521in}}%
\pgfpathcurveto{\pgfqpoint{2.298598in}{1.886521in}}{\pgfqpoint{2.287999in}{1.882130in}}{\pgfqpoint{2.280185in}{1.874317in}}%
\pgfpathcurveto{\pgfqpoint{2.272372in}{1.866503in}}{\pgfqpoint{2.267981in}{1.855904in}}{\pgfqpoint{2.267981in}{1.844854in}}%
\pgfpathcurveto{\pgfqpoint{2.267981in}{1.833804in}}{\pgfqpoint{2.272372in}{1.823205in}}{\pgfqpoint{2.280185in}{1.815391in}}%
\pgfpathcurveto{\pgfqpoint{2.287999in}{1.807578in}}{\pgfqpoint{2.298598in}{1.803187in}}{\pgfqpoint{2.309648in}{1.803187in}}%
\pgfpathclose%
\pgfusepath{stroke,fill}%
\end{pgfscope}%
\begin{pgfscope}%
\pgfpathrectangle{\pgfqpoint{0.600000in}{0.600000in}}{\pgfqpoint{3.900000in}{3.900000in}}%
\pgfusepath{clip}%
\pgfsetbuttcap%
\pgfsetroundjoin%
\definecolor{currentfill}{rgb}{0.121569,0.466667,0.705882}%
\pgfsetfillcolor{currentfill}%
\pgfsetlinewidth{1.003750pt}%
\definecolor{currentstroke}{rgb}{0.121569,0.466667,0.705882}%
\pgfsetstrokecolor{currentstroke}%
\pgfsetdash{}{0pt}%
\pgfpathmoveto{\pgfqpoint{3.368370in}{2.064268in}}%
\pgfpathcurveto{\pgfqpoint{3.379420in}{2.064268in}}{\pgfqpoint{3.390019in}{2.068658in}}{\pgfqpoint{3.397833in}{2.076472in}}%
\pgfpathcurveto{\pgfqpoint{3.405646in}{2.084286in}}{\pgfqpoint{3.410036in}{2.094885in}}{\pgfqpoint{3.410036in}{2.105935in}}%
\pgfpathcurveto{\pgfqpoint{3.410036in}{2.116985in}}{\pgfqpoint{3.405646in}{2.127584in}}{\pgfqpoint{3.397833in}{2.135398in}}%
\pgfpathcurveto{\pgfqpoint{3.390019in}{2.143211in}}{\pgfqpoint{3.379420in}{2.147602in}}{\pgfqpoint{3.368370in}{2.147602in}}%
\pgfpathcurveto{\pgfqpoint{3.357320in}{2.147602in}}{\pgfqpoint{3.346721in}{2.143211in}}{\pgfqpoint{3.338907in}{2.135398in}}%
\pgfpathcurveto{\pgfqpoint{3.331093in}{2.127584in}}{\pgfqpoint{3.326703in}{2.116985in}}{\pgfqpoint{3.326703in}{2.105935in}}%
\pgfpathcurveto{\pgfqpoint{3.326703in}{2.094885in}}{\pgfqpoint{3.331093in}{2.084286in}}{\pgfqpoint{3.338907in}{2.076472in}}%
\pgfpathcurveto{\pgfqpoint{3.346721in}{2.068658in}}{\pgfqpoint{3.357320in}{2.064268in}}{\pgfqpoint{3.368370in}{2.064268in}}%
\pgfpathclose%
\pgfusepath{stroke,fill}%
\end{pgfscope}%
\begin{pgfscope}%
\pgfpathrectangle{\pgfqpoint{0.600000in}{0.600000in}}{\pgfqpoint{3.900000in}{3.900000in}}%
\pgfusepath{clip}%
\pgfsetbuttcap%
\pgfsetroundjoin%
\definecolor{currentfill}{rgb}{0.121569,0.466667,0.705882}%
\pgfsetfillcolor{currentfill}%
\pgfsetlinewidth{1.003750pt}%
\definecolor{currentstroke}{rgb}{0.121569,0.466667,0.705882}%
\pgfsetstrokecolor{currentstroke}%
\pgfsetdash{}{0pt}%
\pgfpathmoveto{\pgfqpoint{1.816298in}{2.618911in}}%
\pgfpathcurveto{\pgfqpoint{1.827348in}{2.618911in}}{\pgfqpoint{1.837947in}{2.623301in}}{\pgfqpoint{1.845760in}{2.631115in}}%
\pgfpathcurveto{\pgfqpoint{1.853574in}{2.638928in}}{\pgfqpoint{1.857964in}{2.649527in}}{\pgfqpoint{1.857964in}{2.660578in}}%
\pgfpathcurveto{\pgfqpoint{1.857964in}{2.671628in}}{\pgfqpoint{1.853574in}{2.682227in}}{\pgfqpoint{1.845760in}{2.690040in}}%
\pgfpathcurveto{\pgfqpoint{1.837947in}{2.697854in}}{\pgfqpoint{1.827348in}{2.702244in}}{\pgfqpoint{1.816298in}{2.702244in}}%
\pgfpathcurveto{\pgfqpoint{1.805248in}{2.702244in}}{\pgfqpoint{1.794649in}{2.697854in}}{\pgfqpoint{1.786835in}{2.690040in}}%
\pgfpathcurveto{\pgfqpoint{1.779021in}{2.682227in}}{\pgfqpoint{1.774631in}{2.671628in}}{\pgfqpoint{1.774631in}{2.660578in}}%
\pgfpathcurveto{\pgfqpoint{1.774631in}{2.649527in}}{\pgfqpoint{1.779021in}{2.638928in}}{\pgfqpoint{1.786835in}{2.631115in}}%
\pgfpathcurveto{\pgfqpoint{1.794649in}{2.623301in}}{\pgfqpoint{1.805248in}{2.618911in}}{\pgfqpoint{1.816298in}{2.618911in}}%
\pgfpathclose%
\pgfusepath{stroke,fill}%
\end{pgfscope}%
\begin{pgfscope}%
\pgfpathrectangle{\pgfqpoint{0.600000in}{0.600000in}}{\pgfqpoint{3.900000in}{3.900000in}}%
\pgfusepath{clip}%
\pgfsetbuttcap%
\pgfsetroundjoin%
\definecolor{currentfill}{rgb}{0.121569,0.466667,0.705882}%
\pgfsetfillcolor{currentfill}%
\pgfsetlinewidth{1.003750pt}%
\definecolor{currentstroke}{rgb}{0.121569,0.466667,0.705882}%
\pgfsetstrokecolor{currentstroke}%
\pgfsetdash{}{0pt}%
\pgfpathmoveto{\pgfqpoint{2.551412in}{2.270823in}}%
\pgfpathcurveto{\pgfqpoint{2.562462in}{2.270823in}}{\pgfqpoint{2.573061in}{2.275213in}}{\pgfqpoint{2.580874in}{2.283027in}}%
\pgfpathcurveto{\pgfqpoint{2.588688in}{2.290840in}}{\pgfqpoint{2.593078in}{2.301439in}}{\pgfqpoint{2.593078in}{2.312489in}}%
\pgfpathcurveto{\pgfqpoint{2.593078in}{2.323540in}}{\pgfqpoint{2.588688in}{2.334139in}}{\pgfqpoint{2.580874in}{2.341952in}}%
\pgfpathcurveto{\pgfqpoint{2.573061in}{2.349766in}}{\pgfqpoint{2.562462in}{2.354156in}}{\pgfqpoint{2.551412in}{2.354156in}}%
\pgfpathcurveto{\pgfqpoint{2.540362in}{2.354156in}}{\pgfqpoint{2.529762in}{2.349766in}}{\pgfqpoint{2.521949in}{2.341952in}}%
\pgfpathcurveto{\pgfqpoint{2.514135in}{2.334139in}}{\pgfqpoint{2.509745in}{2.323540in}}{\pgfqpoint{2.509745in}{2.312489in}}%
\pgfpathcurveto{\pgfqpoint{2.509745in}{2.301439in}}{\pgfqpoint{2.514135in}{2.290840in}}{\pgfqpoint{2.521949in}{2.283027in}}%
\pgfpathcurveto{\pgfqpoint{2.529762in}{2.275213in}}{\pgfqpoint{2.540362in}{2.270823in}}{\pgfqpoint{2.551412in}{2.270823in}}%
\pgfpathclose%
\pgfusepath{stroke,fill}%
\end{pgfscope}%
\begin{pgfscope}%
\pgfpathrectangle{\pgfqpoint{0.600000in}{0.600000in}}{\pgfqpoint{3.900000in}{3.900000in}}%
\pgfusepath{clip}%
\pgfsetbuttcap%
\pgfsetroundjoin%
\definecolor{currentfill}{rgb}{0.121569,0.466667,0.705882}%
\pgfsetfillcolor{currentfill}%
\pgfsetlinewidth{1.003750pt}%
\definecolor{currentstroke}{rgb}{0.121569,0.466667,0.705882}%
\pgfsetstrokecolor{currentstroke}%
\pgfsetdash{}{0pt}%
\pgfpathmoveto{\pgfqpoint{3.169285in}{2.757576in}}%
\pgfpathcurveto{\pgfqpoint{3.180335in}{2.757576in}}{\pgfqpoint{3.190934in}{2.761967in}}{\pgfqpoint{3.198748in}{2.769780in}}%
\pgfpathcurveto{\pgfqpoint{3.206561in}{2.777594in}}{\pgfqpoint{3.210952in}{2.788193in}}{\pgfqpoint{3.210952in}{2.799243in}}%
\pgfpathcurveto{\pgfqpoint{3.210952in}{2.810293in}}{\pgfqpoint{3.206561in}{2.820892in}}{\pgfqpoint{3.198748in}{2.828706in}}%
\pgfpathcurveto{\pgfqpoint{3.190934in}{2.836519in}}{\pgfqpoint{3.180335in}{2.840910in}}{\pgfqpoint{3.169285in}{2.840910in}}%
\pgfpathcurveto{\pgfqpoint{3.158235in}{2.840910in}}{\pgfqpoint{3.147636in}{2.836519in}}{\pgfqpoint{3.139822in}{2.828706in}}%
\pgfpathcurveto{\pgfqpoint{3.132008in}{2.820892in}}{\pgfqpoint{3.127618in}{2.810293in}}{\pgfqpoint{3.127618in}{2.799243in}}%
\pgfpathcurveto{\pgfqpoint{3.127618in}{2.788193in}}{\pgfqpoint{3.132008in}{2.777594in}}{\pgfqpoint{3.139822in}{2.769780in}}%
\pgfpathcurveto{\pgfqpoint{3.147636in}{2.761967in}}{\pgfqpoint{3.158235in}{2.757576in}}{\pgfqpoint{3.169285in}{2.757576in}}%
\pgfpathclose%
\pgfusepath{stroke,fill}%
\end{pgfscope}%
\begin{pgfscope}%
\pgfpathrectangle{\pgfqpoint{0.600000in}{0.600000in}}{\pgfqpoint{3.900000in}{3.900000in}}%
\pgfusepath{clip}%
\pgfsetbuttcap%
\pgfsetroundjoin%
\definecolor{currentfill}{rgb}{0.121569,0.466667,0.705882}%
\pgfsetfillcolor{currentfill}%
\pgfsetlinewidth{1.003750pt}%
\definecolor{currentstroke}{rgb}{0.121569,0.466667,0.705882}%
\pgfsetstrokecolor{currentstroke}%
\pgfsetdash{}{0pt}%
\pgfpathmoveto{\pgfqpoint{2.720777in}{2.345661in}}%
\pgfpathcurveto{\pgfqpoint{2.731827in}{2.345661in}}{\pgfqpoint{2.742426in}{2.350051in}}{\pgfqpoint{2.750240in}{2.357865in}}%
\pgfpathcurveto{\pgfqpoint{2.758054in}{2.365679in}}{\pgfqpoint{2.762444in}{2.376278in}}{\pgfqpoint{2.762444in}{2.387328in}}%
\pgfpathcurveto{\pgfqpoint{2.762444in}{2.398378in}}{\pgfqpoint{2.758054in}{2.408977in}}{\pgfqpoint{2.750240in}{2.416790in}}%
\pgfpathcurveto{\pgfqpoint{2.742426in}{2.424604in}}{\pgfqpoint{2.731827in}{2.428994in}}{\pgfqpoint{2.720777in}{2.428994in}}%
\pgfpathcurveto{\pgfqpoint{2.709727in}{2.428994in}}{\pgfqpoint{2.699128in}{2.424604in}}{\pgfqpoint{2.691314in}{2.416790in}}%
\pgfpathcurveto{\pgfqpoint{2.683501in}{2.408977in}}{\pgfqpoint{2.679110in}{2.398378in}}{\pgfqpoint{2.679110in}{2.387328in}}%
\pgfpathcurveto{\pgfqpoint{2.679110in}{2.376278in}}{\pgfqpoint{2.683501in}{2.365679in}}{\pgfqpoint{2.691314in}{2.357865in}}%
\pgfpathcurveto{\pgfqpoint{2.699128in}{2.350051in}}{\pgfqpoint{2.709727in}{2.345661in}}{\pgfqpoint{2.720777in}{2.345661in}}%
\pgfpathclose%
\pgfusepath{stroke,fill}%
\end{pgfscope}%
\begin{pgfscope}%
\pgfpathrectangle{\pgfqpoint{0.600000in}{0.600000in}}{\pgfqpoint{3.900000in}{3.900000in}}%
\pgfusepath{clip}%
\pgfsetbuttcap%
\pgfsetroundjoin%
\definecolor{currentfill}{rgb}{0.121569,0.466667,0.705882}%
\pgfsetfillcolor{currentfill}%
\pgfsetlinewidth{1.003750pt}%
\definecolor{currentstroke}{rgb}{0.121569,0.466667,0.705882}%
\pgfsetstrokecolor{currentstroke}%
\pgfsetdash{}{0pt}%
\pgfpathmoveto{\pgfqpoint{2.889135in}{2.327058in}}%
\pgfpathcurveto{\pgfqpoint{2.900185in}{2.327058in}}{\pgfqpoint{2.910785in}{2.331448in}}{\pgfqpoint{2.918598in}{2.339262in}}%
\pgfpathcurveto{\pgfqpoint{2.926412in}{2.347075in}}{\pgfqpoint{2.930802in}{2.357674in}}{\pgfqpoint{2.930802in}{2.368725in}}%
\pgfpathcurveto{\pgfqpoint{2.930802in}{2.379775in}}{\pgfqpoint{2.926412in}{2.390374in}}{\pgfqpoint{2.918598in}{2.398187in}}%
\pgfpathcurveto{\pgfqpoint{2.910785in}{2.406001in}}{\pgfqpoint{2.900185in}{2.410391in}}{\pgfqpoint{2.889135in}{2.410391in}}%
\pgfpathcurveto{\pgfqpoint{2.878085in}{2.410391in}}{\pgfqpoint{2.867486in}{2.406001in}}{\pgfqpoint{2.859673in}{2.398187in}}%
\pgfpathcurveto{\pgfqpoint{2.851859in}{2.390374in}}{\pgfqpoint{2.847469in}{2.379775in}}{\pgfqpoint{2.847469in}{2.368725in}}%
\pgfpathcurveto{\pgfqpoint{2.847469in}{2.357674in}}{\pgfqpoint{2.851859in}{2.347075in}}{\pgfqpoint{2.859673in}{2.339262in}}%
\pgfpathcurveto{\pgfqpoint{2.867486in}{2.331448in}}{\pgfqpoint{2.878085in}{2.327058in}}{\pgfqpoint{2.889135in}{2.327058in}}%
\pgfpathclose%
\pgfusepath{stroke,fill}%
\end{pgfscope}%
\begin{pgfscope}%
\pgfpathrectangle{\pgfqpoint{0.600000in}{0.600000in}}{\pgfqpoint{3.900000in}{3.900000in}}%
\pgfusepath{clip}%
\pgfsetbuttcap%
\pgfsetroundjoin%
\definecolor{currentfill}{rgb}{0.121569,0.466667,0.705882}%
\pgfsetfillcolor{currentfill}%
\pgfsetlinewidth{1.003750pt}%
\definecolor{currentstroke}{rgb}{0.121569,0.466667,0.705882}%
\pgfsetstrokecolor{currentstroke}%
\pgfsetdash{}{0pt}%
\pgfpathmoveto{\pgfqpoint{1.852681in}{2.887175in}}%
\pgfpathcurveto{\pgfqpoint{1.863731in}{2.887175in}}{\pgfqpoint{1.874331in}{2.891566in}}{\pgfqpoint{1.882144in}{2.899379in}}%
\pgfpathcurveto{\pgfqpoint{1.889958in}{2.907193in}}{\pgfqpoint{1.894348in}{2.917792in}}{\pgfqpoint{1.894348in}{2.928842in}}%
\pgfpathcurveto{\pgfqpoint{1.894348in}{2.939892in}}{\pgfqpoint{1.889958in}{2.950491in}}{\pgfqpoint{1.882144in}{2.958305in}}%
\pgfpathcurveto{\pgfqpoint{1.874331in}{2.966118in}}{\pgfqpoint{1.863731in}{2.970509in}}{\pgfqpoint{1.852681in}{2.970509in}}%
\pgfpathcurveto{\pgfqpoint{1.841631in}{2.970509in}}{\pgfqpoint{1.831032in}{2.966118in}}{\pgfqpoint{1.823219in}{2.958305in}}%
\pgfpathcurveto{\pgfqpoint{1.815405in}{2.950491in}}{\pgfqpoint{1.811015in}{2.939892in}}{\pgfqpoint{1.811015in}{2.928842in}}%
\pgfpathcurveto{\pgfqpoint{1.811015in}{2.917792in}}{\pgfqpoint{1.815405in}{2.907193in}}{\pgfqpoint{1.823219in}{2.899379in}}%
\pgfpathcurveto{\pgfqpoint{1.831032in}{2.891566in}}{\pgfqpoint{1.841631in}{2.887175in}}{\pgfqpoint{1.852681in}{2.887175in}}%
\pgfpathclose%
\pgfusepath{stroke,fill}%
\end{pgfscope}%
\begin{pgfscope}%
\pgfpathrectangle{\pgfqpoint{0.600000in}{0.600000in}}{\pgfqpoint{3.900000in}{3.900000in}}%
\pgfusepath{clip}%
\pgfsetbuttcap%
\pgfsetroundjoin%
\definecolor{currentfill}{rgb}{0.121569,0.466667,0.705882}%
\pgfsetfillcolor{currentfill}%
\pgfsetlinewidth{1.003750pt}%
\definecolor{currentstroke}{rgb}{0.121569,0.466667,0.705882}%
\pgfsetstrokecolor{currentstroke}%
\pgfsetdash{}{0pt}%
\pgfpathmoveto{\pgfqpoint{2.537795in}{3.510113in}}%
\pgfpathcurveto{\pgfqpoint{2.548845in}{3.510113in}}{\pgfqpoint{2.559444in}{3.514503in}}{\pgfqpoint{2.567258in}{3.522317in}}%
\pgfpathcurveto{\pgfqpoint{2.575071in}{3.530131in}}{\pgfqpoint{2.579462in}{3.540730in}}{\pgfqpoint{2.579462in}{3.551780in}}%
\pgfpathcurveto{\pgfqpoint{2.579462in}{3.562830in}}{\pgfqpoint{2.575071in}{3.573429in}}{\pgfqpoint{2.567258in}{3.581242in}}%
\pgfpathcurveto{\pgfqpoint{2.559444in}{3.589056in}}{\pgfqpoint{2.548845in}{3.593446in}}{\pgfqpoint{2.537795in}{3.593446in}}%
\pgfpathcurveto{\pgfqpoint{2.526745in}{3.593446in}}{\pgfqpoint{2.516146in}{3.589056in}}{\pgfqpoint{2.508332in}{3.581242in}}%
\pgfpathcurveto{\pgfqpoint{2.500519in}{3.573429in}}{\pgfqpoint{2.496128in}{3.562830in}}{\pgfqpoint{2.496128in}{3.551780in}}%
\pgfpathcurveto{\pgfqpoint{2.496128in}{3.540730in}}{\pgfqpoint{2.500519in}{3.530131in}}{\pgfqpoint{2.508332in}{3.522317in}}%
\pgfpathcurveto{\pgfqpoint{2.516146in}{3.514503in}}{\pgfqpoint{2.526745in}{3.510113in}}{\pgfqpoint{2.537795in}{3.510113in}}%
\pgfpathclose%
\pgfusepath{stroke,fill}%
\end{pgfscope}%
\begin{pgfscope}%
\pgfpathrectangle{\pgfqpoint{0.600000in}{0.600000in}}{\pgfqpoint{3.900000in}{3.900000in}}%
\pgfusepath{clip}%
\pgfsetbuttcap%
\pgfsetroundjoin%
\definecolor{currentfill}{rgb}{0.121569,0.466667,0.705882}%
\pgfsetfillcolor{currentfill}%
\pgfsetlinewidth{1.003750pt}%
\definecolor{currentstroke}{rgb}{0.121569,0.466667,0.705882}%
\pgfsetstrokecolor{currentstroke}%
\pgfsetdash{}{0pt}%
\pgfpathmoveto{\pgfqpoint{3.114058in}{2.076004in}}%
\pgfpathcurveto{\pgfqpoint{3.125108in}{2.076004in}}{\pgfqpoint{3.135707in}{2.080395in}}{\pgfqpoint{3.143520in}{2.088208in}}%
\pgfpathcurveto{\pgfqpoint{3.151334in}{2.096022in}}{\pgfqpoint{3.155724in}{2.106621in}}{\pgfqpoint{3.155724in}{2.117671in}}%
\pgfpathcurveto{\pgfqpoint{3.155724in}{2.128721in}}{\pgfqpoint{3.151334in}{2.139320in}}{\pgfqpoint{3.143520in}{2.147134in}}%
\pgfpathcurveto{\pgfqpoint{3.135707in}{2.154948in}}{\pgfqpoint{3.125108in}{2.159338in}}{\pgfqpoint{3.114058in}{2.159338in}}%
\pgfpathcurveto{\pgfqpoint{3.103007in}{2.159338in}}{\pgfqpoint{3.092408in}{2.154948in}}{\pgfqpoint{3.084595in}{2.147134in}}%
\pgfpathcurveto{\pgfqpoint{3.076781in}{2.139320in}}{\pgfqpoint{3.072391in}{2.128721in}}{\pgfqpoint{3.072391in}{2.117671in}}%
\pgfpathcurveto{\pgfqpoint{3.072391in}{2.106621in}}{\pgfqpoint{3.076781in}{2.096022in}}{\pgfqpoint{3.084595in}{2.088208in}}%
\pgfpathcurveto{\pgfqpoint{3.092408in}{2.080395in}}{\pgfqpoint{3.103007in}{2.076004in}}{\pgfqpoint{3.114058in}{2.076004in}}%
\pgfpathclose%
\pgfusepath{stroke,fill}%
\end{pgfscope}%
\begin{pgfscope}%
\pgfpathrectangle{\pgfqpoint{0.600000in}{0.600000in}}{\pgfqpoint{3.900000in}{3.900000in}}%
\pgfusepath{clip}%
\pgfsetbuttcap%
\pgfsetroundjoin%
\definecolor{currentfill}{rgb}{0.121569,0.466667,0.705882}%
\pgfsetfillcolor{currentfill}%
\pgfsetlinewidth{1.003750pt}%
\definecolor{currentstroke}{rgb}{0.121569,0.466667,0.705882}%
\pgfsetstrokecolor{currentstroke}%
\pgfsetdash{}{0pt}%
\pgfpathmoveto{\pgfqpoint{2.250964in}{2.226380in}}%
\pgfpathcurveto{\pgfqpoint{2.262014in}{2.226380in}}{\pgfqpoint{2.272613in}{2.230771in}}{\pgfqpoint{2.280427in}{2.238584in}}%
\pgfpathcurveto{\pgfqpoint{2.288240in}{2.246398in}}{\pgfqpoint{2.292630in}{2.256997in}}{\pgfqpoint{2.292630in}{2.268047in}}%
\pgfpathcurveto{\pgfqpoint{2.292630in}{2.279097in}}{\pgfqpoint{2.288240in}{2.289696in}}{\pgfqpoint{2.280427in}{2.297510in}}%
\pgfpathcurveto{\pgfqpoint{2.272613in}{2.305323in}}{\pgfqpoint{2.262014in}{2.309714in}}{\pgfqpoint{2.250964in}{2.309714in}}%
\pgfpathcurveto{\pgfqpoint{2.239914in}{2.309714in}}{\pgfqpoint{2.229315in}{2.305323in}}{\pgfqpoint{2.221501in}{2.297510in}}%
\pgfpathcurveto{\pgfqpoint{2.213687in}{2.289696in}}{\pgfqpoint{2.209297in}{2.279097in}}{\pgfqpoint{2.209297in}{2.268047in}}%
\pgfpathcurveto{\pgfqpoint{2.209297in}{2.256997in}}{\pgfqpoint{2.213687in}{2.246398in}}{\pgfqpoint{2.221501in}{2.238584in}}%
\pgfpathcurveto{\pgfqpoint{2.229315in}{2.230771in}}{\pgfqpoint{2.239914in}{2.226380in}}{\pgfqpoint{2.250964in}{2.226380in}}%
\pgfpathclose%
\pgfusepath{stroke,fill}%
\end{pgfscope}%
\begin{pgfscope}%
\pgfpathrectangle{\pgfqpoint{0.600000in}{0.600000in}}{\pgfqpoint{3.900000in}{3.900000in}}%
\pgfusepath{clip}%
\pgfsetbuttcap%
\pgfsetroundjoin%
\definecolor{currentfill}{rgb}{0.121569,0.466667,0.705882}%
\pgfsetfillcolor{currentfill}%
\pgfsetlinewidth{1.003750pt}%
\definecolor{currentstroke}{rgb}{0.121569,0.466667,0.705882}%
\pgfsetstrokecolor{currentstroke}%
\pgfsetdash{}{0pt}%
\pgfpathmoveto{\pgfqpoint{2.624439in}{1.795083in}}%
\pgfpathcurveto{\pgfqpoint{2.635489in}{1.795083in}}{\pgfqpoint{2.646088in}{1.799474in}}{\pgfqpoint{2.653902in}{1.807287in}}%
\pgfpathcurveto{\pgfqpoint{2.661716in}{1.815101in}}{\pgfqpoint{2.666106in}{1.825700in}}{\pgfqpoint{2.666106in}{1.836750in}}%
\pgfpathcurveto{\pgfqpoint{2.666106in}{1.847800in}}{\pgfqpoint{2.661716in}{1.858399in}}{\pgfqpoint{2.653902in}{1.866213in}}%
\pgfpathcurveto{\pgfqpoint{2.646088in}{1.874027in}}{\pgfqpoint{2.635489in}{1.878417in}}{\pgfqpoint{2.624439in}{1.878417in}}%
\pgfpathcurveto{\pgfqpoint{2.613389in}{1.878417in}}{\pgfqpoint{2.602790in}{1.874027in}}{\pgfqpoint{2.594976in}{1.866213in}}%
\pgfpathcurveto{\pgfqpoint{2.587163in}{1.858399in}}{\pgfqpoint{2.582772in}{1.847800in}}{\pgfqpoint{2.582772in}{1.836750in}}%
\pgfpathcurveto{\pgfqpoint{2.582772in}{1.825700in}}{\pgfqpoint{2.587163in}{1.815101in}}{\pgfqpoint{2.594976in}{1.807287in}}%
\pgfpathcurveto{\pgfqpoint{2.602790in}{1.799474in}}{\pgfqpoint{2.613389in}{1.795083in}}{\pgfqpoint{2.624439in}{1.795083in}}%
\pgfpathclose%
\pgfusepath{stroke,fill}%
\end{pgfscope}%
\begin{pgfscope}%
\pgfpathrectangle{\pgfqpoint{0.600000in}{0.600000in}}{\pgfqpoint{3.900000in}{3.900000in}}%
\pgfusepath{clip}%
\pgfsetbuttcap%
\pgfsetroundjoin%
\definecolor{currentfill}{rgb}{0.121569,0.466667,0.705882}%
\pgfsetfillcolor{currentfill}%
\pgfsetlinewidth{1.003750pt}%
\definecolor{currentstroke}{rgb}{0.121569,0.466667,0.705882}%
\pgfsetstrokecolor{currentstroke}%
\pgfsetdash{}{0pt}%
\pgfpathmoveto{\pgfqpoint{2.786490in}{1.858451in}}%
\pgfpathcurveto{\pgfqpoint{2.797540in}{1.858451in}}{\pgfqpoint{2.808139in}{1.862841in}}{\pgfqpoint{2.815953in}{1.870655in}}%
\pgfpathcurveto{\pgfqpoint{2.823767in}{1.878469in}}{\pgfqpoint{2.828157in}{1.889068in}}{\pgfqpoint{2.828157in}{1.900118in}}%
\pgfpathcurveto{\pgfqpoint{2.828157in}{1.911168in}}{\pgfqpoint{2.823767in}{1.921767in}}{\pgfqpoint{2.815953in}{1.929581in}}%
\pgfpathcurveto{\pgfqpoint{2.808139in}{1.937394in}}{\pgfqpoint{2.797540in}{1.941784in}}{\pgfqpoint{2.786490in}{1.941784in}}%
\pgfpathcurveto{\pgfqpoint{2.775440in}{1.941784in}}{\pgfqpoint{2.764841in}{1.937394in}}{\pgfqpoint{2.757027in}{1.929581in}}%
\pgfpathcurveto{\pgfqpoint{2.749214in}{1.921767in}}{\pgfqpoint{2.744824in}{1.911168in}}{\pgfqpoint{2.744824in}{1.900118in}}%
\pgfpathcurveto{\pgfqpoint{2.744824in}{1.889068in}}{\pgfqpoint{2.749214in}{1.878469in}}{\pgfqpoint{2.757027in}{1.870655in}}%
\pgfpathcurveto{\pgfqpoint{2.764841in}{1.862841in}}{\pgfqpoint{2.775440in}{1.858451in}}{\pgfqpoint{2.786490in}{1.858451in}}%
\pgfpathclose%
\pgfusepath{stroke,fill}%
\end{pgfscope}%
\begin{pgfscope}%
\pgfpathrectangle{\pgfqpoint{0.600000in}{0.600000in}}{\pgfqpoint{3.900000in}{3.900000in}}%
\pgfusepath{clip}%
\pgfsetbuttcap%
\pgfsetroundjoin%
\definecolor{currentfill}{rgb}{0.121569,0.466667,0.705882}%
\pgfsetfillcolor{currentfill}%
\pgfsetlinewidth{1.003750pt}%
\definecolor{currentstroke}{rgb}{0.121569,0.466667,0.705882}%
\pgfsetstrokecolor{currentstroke}%
\pgfsetdash{}{0pt}%
\pgfpathmoveto{\pgfqpoint{2.740007in}{2.464529in}}%
\pgfpathcurveto{\pgfqpoint{2.751057in}{2.464529in}}{\pgfqpoint{2.761656in}{2.468920in}}{\pgfqpoint{2.769469in}{2.476733in}}%
\pgfpathcurveto{\pgfqpoint{2.777283in}{2.484547in}}{\pgfqpoint{2.781673in}{2.495146in}}{\pgfqpoint{2.781673in}{2.506196in}}%
\pgfpathcurveto{\pgfqpoint{2.781673in}{2.517246in}}{\pgfqpoint{2.777283in}{2.527845in}}{\pgfqpoint{2.769469in}{2.535659in}}%
\pgfpathcurveto{\pgfqpoint{2.761656in}{2.543472in}}{\pgfqpoint{2.751057in}{2.547863in}}{\pgfqpoint{2.740007in}{2.547863in}}%
\pgfpathcurveto{\pgfqpoint{2.728956in}{2.547863in}}{\pgfqpoint{2.718357in}{2.543472in}}{\pgfqpoint{2.710544in}{2.535659in}}%
\pgfpathcurveto{\pgfqpoint{2.702730in}{2.527845in}}{\pgfqpoint{2.698340in}{2.517246in}}{\pgfqpoint{2.698340in}{2.506196in}}%
\pgfpathcurveto{\pgfqpoint{2.698340in}{2.495146in}}{\pgfqpoint{2.702730in}{2.484547in}}{\pgfqpoint{2.710544in}{2.476733in}}%
\pgfpathcurveto{\pgfqpoint{2.718357in}{2.468920in}}{\pgfqpoint{2.728956in}{2.464529in}}{\pgfqpoint{2.740007in}{2.464529in}}%
\pgfpathclose%
\pgfusepath{stroke,fill}%
\end{pgfscope}%
\begin{pgfscope}%
\pgfpathrectangle{\pgfqpoint{0.600000in}{0.600000in}}{\pgfqpoint{3.900000in}{3.900000in}}%
\pgfusepath{clip}%
\pgfsetbuttcap%
\pgfsetroundjoin%
\definecolor{currentfill}{rgb}{0.121569,0.466667,0.705882}%
\pgfsetfillcolor{currentfill}%
\pgfsetlinewidth{1.003750pt}%
\definecolor{currentstroke}{rgb}{0.121569,0.466667,0.705882}%
\pgfsetstrokecolor{currentstroke}%
\pgfsetdash{}{0pt}%
\pgfpathmoveto{\pgfqpoint{2.759030in}{1.669752in}}%
\pgfpathcurveto{\pgfqpoint{2.770080in}{1.669752in}}{\pgfqpoint{2.780679in}{1.674142in}}{\pgfqpoint{2.788492in}{1.681956in}}%
\pgfpathcurveto{\pgfqpoint{2.796306in}{1.689769in}}{\pgfqpoint{2.800696in}{1.700368in}}{\pgfqpoint{2.800696in}{1.711419in}}%
\pgfpathcurveto{\pgfqpoint{2.800696in}{1.722469in}}{\pgfqpoint{2.796306in}{1.733068in}}{\pgfqpoint{2.788492in}{1.740881in}}%
\pgfpathcurveto{\pgfqpoint{2.780679in}{1.748695in}}{\pgfqpoint{2.770080in}{1.753085in}}{\pgfqpoint{2.759030in}{1.753085in}}%
\pgfpathcurveto{\pgfqpoint{2.747980in}{1.753085in}}{\pgfqpoint{2.737380in}{1.748695in}}{\pgfqpoint{2.729567in}{1.740881in}}%
\pgfpathcurveto{\pgfqpoint{2.721753in}{1.733068in}}{\pgfqpoint{2.717363in}{1.722469in}}{\pgfqpoint{2.717363in}{1.711419in}}%
\pgfpathcurveto{\pgfqpoint{2.717363in}{1.700368in}}{\pgfqpoint{2.721753in}{1.689769in}}{\pgfqpoint{2.729567in}{1.681956in}}%
\pgfpathcurveto{\pgfqpoint{2.737380in}{1.674142in}}{\pgfqpoint{2.747980in}{1.669752in}}{\pgfqpoint{2.759030in}{1.669752in}}%
\pgfpathclose%
\pgfusepath{stroke,fill}%
\end{pgfscope}%
\begin{pgfscope}%
\pgfpathrectangle{\pgfqpoint{0.600000in}{0.600000in}}{\pgfqpoint{3.900000in}{3.900000in}}%
\pgfusepath{clip}%
\pgfsetbuttcap%
\pgfsetroundjoin%
\definecolor{currentfill}{rgb}{0.121569,0.466667,0.705882}%
\pgfsetfillcolor{currentfill}%
\pgfsetlinewidth{1.003750pt}%
\definecolor{currentstroke}{rgb}{0.121569,0.466667,0.705882}%
\pgfsetstrokecolor{currentstroke}%
\pgfsetdash{}{0pt}%
\pgfpathmoveto{\pgfqpoint{3.514030in}{2.721817in}}%
\pgfpathcurveto{\pgfqpoint{3.525080in}{2.721817in}}{\pgfqpoint{3.535679in}{2.726207in}}{\pgfqpoint{3.543493in}{2.734021in}}%
\pgfpathcurveto{\pgfqpoint{3.551306in}{2.741834in}}{\pgfqpoint{3.555697in}{2.752433in}}{\pgfqpoint{3.555697in}{2.763483in}}%
\pgfpathcurveto{\pgfqpoint{3.555697in}{2.774534in}}{\pgfqpoint{3.551306in}{2.785133in}}{\pgfqpoint{3.543493in}{2.792946in}}%
\pgfpathcurveto{\pgfqpoint{3.535679in}{2.800760in}}{\pgfqpoint{3.525080in}{2.805150in}}{\pgfqpoint{3.514030in}{2.805150in}}%
\pgfpathcurveto{\pgfqpoint{3.502980in}{2.805150in}}{\pgfqpoint{3.492381in}{2.800760in}}{\pgfqpoint{3.484567in}{2.792946in}}%
\pgfpathcurveto{\pgfqpoint{3.476754in}{2.785133in}}{\pgfqpoint{3.472363in}{2.774534in}}{\pgfqpoint{3.472363in}{2.763483in}}%
\pgfpathcurveto{\pgfqpoint{3.472363in}{2.752433in}}{\pgfqpoint{3.476754in}{2.741834in}}{\pgfqpoint{3.484567in}{2.734021in}}%
\pgfpathcurveto{\pgfqpoint{3.492381in}{2.726207in}}{\pgfqpoint{3.502980in}{2.721817in}}{\pgfqpoint{3.514030in}{2.721817in}}%
\pgfpathclose%
\pgfusepath{stroke,fill}%
\end{pgfscope}%
\begin{pgfscope}%
\pgfpathrectangle{\pgfqpoint{0.600000in}{0.600000in}}{\pgfqpoint{3.900000in}{3.900000in}}%
\pgfusepath{clip}%
\pgfsetbuttcap%
\pgfsetroundjoin%
\definecolor{currentfill}{rgb}{0.121569,0.466667,0.705882}%
\pgfsetfillcolor{currentfill}%
\pgfsetlinewidth{1.003750pt}%
\definecolor{currentstroke}{rgb}{0.121569,0.466667,0.705882}%
\pgfsetstrokecolor{currentstroke}%
\pgfsetdash{}{0pt}%
\pgfpathmoveto{\pgfqpoint{3.073832in}{3.129304in}}%
\pgfpathcurveto{\pgfqpoint{3.084882in}{3.129304in}}{\pgfqpoint{3.095481in}{3.133694in}}{\pgfqpoint{3.103295in}{3.141508in}}%
\pgfpathcurveto{\pgfqpoint{3.111109in}{3.149321in}}{\pgfqpoint{3.115499in}{3.159921in}}{\pgfqpoint{3.115499in}{3.170971in}}%
\pgfpathcurveto{\pgfqpoint{3.115499in}{3.182021in}}{\pgfqpoint{3.111109in}{3.192620in}}{\pgfqpoint{3.103295in}{3.200433in}}%
\pgfpathcurveto{\pgfqpoint{3.095481in}{3.208247in}}{\pgfqpoint{3.084882in}{3.212637in}}{\pgfqpoint{3.073832in}{3.212637in}}%
\pgfpathcurveto{\pgfqpoint{3.062782in}{3.212637in}}{\pgfqpoint{3.052183in}{3.208247in}}{\pgfqpoint{3.044369in}{3.200433in}}%
\pgfpathcurveto{\pgfqpoint{3.036556in}{3.192620in}}{\pgfqpoint{3.032166in}{3.182021in}}{\pgfqpoint{3.032166in}{3.170971in}}%
\pgfpathcurveto{\pgfqpoint{3.032166in}{3.159921in}}{\pgfqpoint{3.036556in}{3.149321in}}{\pgfqpoint{3.044369in}{3.141508in}}%
\pgfpathcurveto{\pgfqpoint{3.052183in}{3.133694in}}{\pgfqpoint{3.062782in}{3.129304in}}{\pgfqpoint{3.073832in}{3.129304in}}%
\pgfpathclose%
\pgfusepath{stroke,fill}%
\end{pgfscope}%
\begin{pgfscope}%
\pgfpathrectangle{\pgfqpoint{0.600000in}{0.600000in}}{\pgfqpoint{3.900000in}{3.900000in}}%
\pgfusepath{clip}%
\pgfsetbuttcap%
\pgfsetroundjoin%
\definecolor{currentfill}{rgb}{0.121569,0.466667,0.705882}%
\pgfsetfillcolor{currentfill}%
\pgfsetlinewidth{1.003750pt}%
\definecolor{currentstroke}{rgb}{0.121569,0.466667,0.705882}%
\pgfsetstrokecolor{currentstroke}%
\pgfsetdash{}{0pt}%
\pgfpathmoveto{\pgfqpoint{2.922965in}{3.360547in}}%
\pgfpathcurveto{\pgfqpoint{2.934015in}{3.360547in}}{\pgfqpoint{2.944614in}{3.364937in}}{\pgfqpoint{2.952427in}{3.372750in}}%
\pgfpathcurveto{\pgfqpoint{2.960241in}{3.380564in}}{\pgfqpoint{2.964631in}{3.391163in}}{\pgfqpoint{2.964631in}{3.402213in}}%
\pgfpathcurveto{\pgfqpoint{2.964631in}{3.413263in}}{\pgfqpoint{2.960241in}{3.423862in}}{\pgfqpoint{2.952427in}{3.431676in}}%
\pgfpathcurveto{\pgfqpoint{2.944614in}{3.439490in}}{\pgfqpoint{2.934015in}{3.443880in}}{\pgfqpoint{2.922965in}{3.443880in}}%
\pgfpathcurveto{\pgfqpoint{2.911914in}{3.443880in}}{\pgfqpoint{2.901315in}{3.439490in}}{\pgfqpoint{2.893502in}{3.431676in}}%
\pgfpathcurveto{\pgfqpoint{2.885688in}{3.423862in}}{\pgfqpoint{2.881298in}{3.413263in}}{\pgfqpoint{2.881298in}{3.402213in}}%
\pgfpathcurveto{\pgfqpoint{2.881298in}{3.391163in}}{\pgfqpoint{2.885688in}{3.380564in}}{\pgfqpoint{2.893502in}{3.372750in}}%
\pgfpathcurveto{\pgfqpoint{2.901315in}{3.364937in}}{\pgfqpoint{2.911914in}{3.360547in}}{\pgfqpoint{2.922965in}{3.360547in}}%
\pgfpathclose%
\pgfusepath{stroke,fill}%
\end{pgfscope}%
\begin{pgfscope}%
\pgfpathrectangle{\pgfqpoint{0.600000in}{0.600000in}}{\pgfqpoint{3.900000in}{3.900000in}}%
\pgfusepath{clip}%
\pgfsetbuttcap%
\pgfsetroundjoin%
\definecolor{currentfill}{rgb}{0.121569,0.466667,0.705882}%
\pgfsetfillcolor{currentfill}%
\pgfsetlinewidth{1.003750pt}%
\definecolor{currentstroke}{rgb}{0.121569,0.466667,0.705882}%
\pgfsetstrokecolor{currentstroke}%
\pgfsetdash{}{0pt}%
\pgfpathmoveto{\pgfqpoint{2.735621in}{2.731797in}}%
\pgfpathcurveto{\pgfqpoint{2.746672in}{2.731797in}}{\pgfqpoint{2.757271in}{2.736188in}}{\pgfqpoint{2.765084in}{2.744001in}}%
\pgfpathcurveto{\pgfqpoint{2.772898in}{2.751815in}}{\pgfqpoint{2.777288in}{2.762414in}}{\pgfqpoint{2.777288in}{2.773464in}}%
\pgfpathcurveto{\pgfqpoint{2.777288in}{2.784514in}}{\pgfqpoint{2.772898in}{2.795113in}}{\pgfqpoint{2.765084in}{2.802927in}}%
\pgfpathcurveto{\pgfqpoint{2.757271in}{2.810740in}}{\pgfqpoint{2.746672in}{2.815131in}}{\pgfqpoint{2.735621in}{2.815131in}}%
\pgfpathcurveto{\pgfqpoint{2.724571in}{2.815131in}}{\pgfqpoint{2.713972in}{2.810740in}}{\pgfqpoint{2.706159in}{2.802927in}}%
\pgfpathcurveto{\pgfqpoint{2.698345in}{2.795113in}}{\pgfqpoint{2.693955in}{2.784514in}}{\pgfqpoint{2.693955in}{2.773464in}}%
\pgfpathcurveto{\pgfqpoint{2.693955in}{2.762414in}}{\pgfqpoint{2.698345in}{2.751815in}}{\pgfqpoint{2.706159in}{2.744001in}}%
\pgfpathcurveto{\pgfqpoint{2.713972in}{2.736188in}}{\pgfqpoint{2.724571in}{2.731797in}}{\pgfqpoint{2.735621in}{2.731797in}}%
\pgfpathclose%
\pgfusepath{stroke,fill}%
\end{pgfscope}%
\begin{pgfscope}%
\pgfpathrectangle{\pgfqpoint{0.600000in}{0.600000in}}{\pgfqpoint{3.900000in}{3.900000in}}%
\pgfusepath{clip}%
\pgfsetbuttcap%
\pgfsetroundjoin%
\definecolor{currentfill}{rgb}{0.121569,0.466667,0.705882}%
\pgfsetfillcolor{currentfill}%
\pgfsetlinewidth{1.003750pt}%
\definecolor{currentstroke}{rgb}{0.121569,0.466667,0.705882}%
\pgfsetstrokecolor{currentstroke}%
\pgfsetdash{}{0pt}%
\pgfpathmoveto{\pgfqpoint{2.507250in}{2.689772in}}%
\pgfpathcurveto{\pgfqpoint{2.518300in}{2.689772in}}{\pgfqpoint{2.528899in}{2.694163in}}{\pgfqpoint{2.536713in}{2.701976in}}%
\pgfpathcurveto{\pgfqpoint{2.544527in}{2.709790in}}{\pgfqpoint{2.548917in}{2.720389in}}{\pgfqpoint{2.548917in}{2.731439in}}%
\pgfpathcurveto{\pgfqpoint{2.548917in}{2.742489in}}{\pgfqpoint{2.544527in}{2.753088in}}{\pgfqpoint{2.536713in}{2.760902in}}%
\pgfpathcurveto{\pgfqpoint{2.528899in}{2.768715in}}{\pgfqpoint{2.518300in}{2.773106in}}{\pgfqpoint{2.507250in}{2.773106in}}%
\pgfpathcurveto{\pgfqpoint{2.496200in}{2.773106in}}{\pgfqpoint{2.485601in}{2.768715in}}{\pgfqpoint{2.477787in}{2.760902in}}%
\pgfpathcurveto{\pgfqpoint{2.469974in}{2.753088in}}{\pgfqpoint{2.465583in}{2.742489in}}{\pgfqpoint{2.465583in}{2.731439in}}%
\pgfpathcurveto{\pgfqpoint{2.465583in}{2.720389in}}{\pgfqpoint{2.469974in}{2.709790in}}{\pgfqpoint{2.477787in}{2.701976in}}%
\pgfpathcurveto{\pgfqpoint{2.485601in}{2.694163in}}{\pgfqpoint{2.496200in}{2.689772in}}{\pgfqpoint{2.507250in}{2.689772in}}%
\pgfpathclose%
\pgfusepath{stroke,fill}%
\end{pgfscope}%
\begin{pgfscope}%
\pgfpathrectangle{\pgfqpoint{0.600000in}{0.600000in}}{\pgfqpoint{3.900000in}{3.900000in}}%
\pgfusepath{clip}%
\pgfsetbuttcap%
\pgfsetroundjoin%
\definecolor{currentfill}{rgb}{0.121569,0.466667,0.705882}%
\pgfsetfillcolor{currentfill}%
\pgfsetlinewidth{1.003750pt}%
\definecolor{currentstroke}{rgb}{0.121569,0.466667,0.705882}%
\pgfsetstrokecolor{currentstroke}%
\pgfsetdash{}{0pt}%
\pgfpathmoveto{\pgfqpoint{2.766927in}{3.307643in}}%
\pgfpathcurveto{\pgfqpoint{2.777978in}{3.307643in}}{\pgfqpoint{2.788577in}{3.312033in}}{\pgfqpoint{2.796390in}{3.319847in}}%
\pgfpathcurveto{\pgfqpoint{2.804204in}{3.327661in}}{\pgfqpoint{2.808594in}{3.338260in}}{\pgfqpoint{2.808594in}{3.349310in}}%
\pgfpathcurveto{\pgfqpoint{2.808594in}{3.360360in}}{\pgfqpoint{2.804204in}{3.370959in}}{\pgfqpoint{2.796390in}{3.378772in}}%
\pgfpathcurveto{\pgfqpoint{2.788577in}{3.386586in}}{\pgfqpoint{2.777978in}{3.390976in}}{\pgfqpoint{2.766927in}{3.390976in}}%
\pgfpathcurveto{\pgfqpoint{2.755877in}{3.390976in}}{\pgfqpoint{2.745278in}{3.386586in}}{\pgfqpoint{2.737465in}{3.378772in}}%
\pgfpathcurveto{\pgfqpoint{2.729651in}{3.370959in}}{\pgfqpoint{2.725261in}{3.360360in}}{\pgfqpoint{2.725261in}{3.349310in}}%
\pgfpathcurveto{\pgfqpoint{2.725261in}{3.338260in}}{\pgfqpoint{2.729651in}{3.327661in}}{\pgfqpoint{2.737465in}{3.319847in}}%
\pgfpathcurveto{\pgfqpoint{2.745278in}{3.312033in}}{\pgfqpoint{2.755877in}{3.307643in}}{\pgfqpoint{2.766927in}{3.307643in}}%
\pgfpathclose%
\pgfusepath{stroke,fill}%
\end{pgfscope}%
\begin{pgfscope}%
\pgfpathrectangle{\pgfqpoint{0.600000in}{0.600000in}}{\pgfqpoint{3.900000in}{3.900000in}}%
\pgfusepath{clip}%
\pgfsetbuttcap%
\pgfsetroundjoin%
\definecolor{currentfill}{rgb}{0.121569,0.466667,0.705882}%
\pgfsetfillcolor{currentfill}%
\pgfsetlinewidth{1.003750pt}%
\definecolor{currentstroke}{rgb}{0.121569,0.466667,0.705882}%
\pgfsetstrokecolor{currentstroke}%
\pgfsetdash{}{0pt}%
\pgfpathmoveto{\pgfqpoint{2.446631in}{1.949974in}}%
\pgfpathcurveto{\pgfqpoint{2.457681in}{1.949974in}}{\pgfqpoint{2.468281in}{1.954364in}}{\pgfqpoint{2.476094in}{1.962178in}}%
\pgfpathcurveto{\pgfqpoint{2.483908in}{1.969991in}}{\pgfqpoint{2.488298in}{1.980590in}}{\pgfqpoint{2.488298in}{1.991641in}}%
\pgfpathcurveto{\pgfqpoint{2.488298in}{2.002691in}}{\pgfqpoint{2.483908in}{2.013290in}}{\pgfqpoint{2.476094in}{2.021103in}}%
\pgfpathcurveto{\pgfqpoint{2.468281in}{2.028917in}}{\pgfqpoint{2.457681in}{2.033307in}}{\pgfqpoint{2.446631in}{2.033307in}}%
\pgfpathcurveto{\pgfqpoint{2.435581in}{2.033307in}}{\pgfqpoint{2.424982in}{2.028917in}}{\pgfqpoint{2.417169in}{2.021103in}}%
\pgfpathcurveto{\pgfqpoint{2.409355in}{2.013290in}}{\pgfqpoint{2.404965in}{2.002691in}}{\pgfqpoint{2.404965in}{1.991641in}}%
\pgfpathcurveto{\pgfqpoint{2.404965in}{1.980590in}}{\pgfqpoint{2.409355in}{1.969991in}}{\pgfqpoint{2.417169in}{1.962178in}}%
\pgfpathcurveto{\pgfqpoint{2.424982in}{1.954364in}}{\pgfqpoint{2.435581in}{1.949974in}}{\pgfqpoint{2.446631in}{1.949974in}}%
\pgfpathclose%
\pgfusepath{stroke,fill}%
\end{pgfscope}%
\begin{pgfscope}%
\pgfpathrectangle{\pgfqpoint{0.600000in}{0.600000in}}{\pgfqpoint{3.900000in}{3.900000in}}%
\pgfusepath{clip}%
\pgfsetbuttcap%
\pgfsetroundjoin%
\definecolor{currentfill}{rgb}{0.121569,0.466667,0.705882}%
\pgfsetfillcolor{currentfill}%
\pgfsetlinewidth{1.003750pt}%
\definecolor{currentstroke}{rgb}{0.121569,0.466667,0.705882}%
\pgfsetstrokecolor{currentstroke}%
\pgfsetdash{}{0pt}%
\pgfpathmoveto{\pgfqpoint{2.287512in}{2.567636in}}%
\pgfpathcurveto{\pgfqpoint{2.298562in}{2.567636in}}{\pgfqpoint{2.309161in}{2.572026in}}{\pgfqpoint{2.316974in}{2.579840in}}%
\pgfpathcurveto{\pgfqpoint{2.324788in}{2.587654in}}{\pgfqpoint{2.329178in}{2.598253in}}{\pgfqpoint{2.329178in}{2.609303in}}%
\pgfpathcurveto{\pgfqpoint{2.329178in}{2.620353in}}{\pgfqpoint{2.324788in}{2.630952in}}{\pgfqpoint{2.316974in}{2.638765in}}%
\pgfpathcurveto{\pgfqpoint{2.309161in}{2.646579in}}{\pgfqpoint{2.298562in}{2.650969in}}{\pgfqpoint{2.287512in}{2.650969in}}%
\pgfpathcurveto{\pgfqpoint{2.276461in}{2.650969in}}{\pgfqpoint{2.265862in}{2.646579in}}{\pgfqpoint{2.258049in}{2.638765in}}%
\pgfpathcurveto{\pgfqpoint{2.250235in}{2.630952in}}{\pgfqpoint{2.245845in}{2.620353in}}{\pgfqpoint{2.245845in}{2.609303in}}%
\pgfpathcurveto{\pgfqpoint{2.245845in}{2.598253in}}{\pgfqpoint{2.250235in}{2.587654in}}{\pgfqpoint{2.258049in}{2.579840in}}%
\pgfpathcurveto{\pgfqpoint{2.265862in}{2.572026in}}{\pgfqpoint{2.276461in}{2.567636in}}{\pgfqpoint{2.287512in}{2.567636in}}%
\pgfpathclose%
\pgfusepath{stroke,fill}%
\end{pgfscope}%
\begin{pgfscope}%
\pgfpathrectangle{\pgfqpoint{0.600000in}{0.600000in}}{\pgfqpoint{3.900000in}{3.900000in}}%
\pgfusepath{clip}%
\pgfsetbuttcap%
\pgfsetroundjoin%
\definecolor{currentfill}{rgb}{0.121569,0.466667,0.705882}%
\pgfsetfillcolor{currentfill}%
\pgfsetlinewidth{1.003750pt}%
\definecolor{currentstroke}{rgb}{0.121569,0.466667,0.705882}%
\pgfsetstrokecolor{currentstroke}%
\pgfsetdash{}{0pt}%
\pgfpathmoveto{\pgfqpoint{2.808365in}{1.766670in}}%
\pgfpathcurveto{\pgfqpoint{2.819415in}{1.766670in}}{\pgfqpoint{2.830014in}{1.771060in}}{\pgfqpoint{2.837828in}{1.778874in}}%
\pgfpathcurveto{\pgfqpoint{2.845642in}{1.786688in}}{\pgfqpoint{2.850032in}{1.797287in}}{\pgfqpoint{2.850032in}{1.808337in}}%
\pgfpathcurveto{\pgfqpoint{2.850032in}{1.819387in}}{\pgfqpoint{2.845642in}{1.829986in}}{\pgfqpoint{2.837828in}{1.837799in}}%
\pgfpathcurveto{\pgfqpoint{2.830014in}{1.845613in}}{\pgfqpoint{2.819415in}{1.850003in}}{\pgfqpoint{2.808365in}{1.850003in}}%
\pgfpathcurveto{\pgfqpoint{2.797315in}{1.850003in}}{\pgfqpoint{2.786716in}{1.845613in}}{\pgfqpoint{2.778902in}{1.837799in}}%
\pgfpathcurveto{\pgfqpoint{2.771089in}{1.829986in}}{\pgfqpoint{2.766698in}{1.819387in}}{\pgfqpoint{2.766698in}{1.808337in}}%
\pgfpathcurveto{\pgfqpoint{2.766698in}{1.797287in}}{\pgfqpoint{2.771089in}{1.786688in}}{\pgfqpoint{2.778902in}{1.778874in}}%
\pgfpathcurveto{\pgfqpoint{2.786716in}{1.771060in}}{\pgfqpoint{2.797315in}{1.766670in}}{\pgfqpoint{2.808365in}{1.766670in}}%
\pgfpathclose%
\pgfusepath{stroke,fill}%
\end{pgfscope}%
\begin{pgfscope}%
\pgfpathrectangle{\pgfqpoint{0.600000in}{0.600000in}}{\pgfqpoint{3.900000in}{3.900000in}}%
\pgfusepath{clip}%
\pgfsetbuttcap%
\pgfsetroundjoin%
\definecolor{currentfill}{rgb}{0.121569,0.466667,0.705882}%
\pgfsetfillcolor{currentfill}%
\pgfsetlinewidth{1.003750pt}%
\definecolor{currentstroke}{rgb}{0.121569,0.466667,0.705882}%
\pgfsetstrokecolor{currentstroke}%
\pgfsetdash{}{0pt}%
\pgfpathmoveto{\pgfqpoint{1.618654in}{1.680215in}}%
\pgfpathcurveto{\pgfqpoint{1.629704in}{1.680215in}}{\pgfqpoint{1.640303in}{1.684605in}}{\pgfqpoint{1.648117in}{1.692418in}}%
\pgfpathcurveto{\pgfqpoint{1.655930in}{1.700232in}}{\pgfqpoint{1.660320in}{1.710831in}}{\pgfqpoint{1.660320in}{1.721881in}}%
\pgfpathcurveto{\pgfqpoint{1.660320in}{1.732931in}}{\pgfqpoint{1.655930in}{1.743530in}}{\pgfqpoint{1.648117in}{1.751344in}}%
\pgfpathcurveto{\pgfqpoint{1.640303in}{1.759158in}}{\pgfqpoint{1.629704in}{1.763548in}}{\pgfqpoint{1.618654in}{1.763548in}}%
\pgfpathcurveto{\pgfqpoint{1.607604in}{1.763548in}}{\pgfqpoint{1.597005in}{1.759158in}}{\pgfqpoint{1.589191in}{1.751344in}}%
\pgfpathcurveto{\pgfqpoint{1.581377in}{1.743530in}}{\pgfqpoint{1.576987in}{1.732931in}}{\pgfqpoint{1.576987in}{1.721881in}}%
\pgfpathcurveto{\pgfqpoint{1.576987in}{1.710831in}}{\pgfqpoint{1.581377in}{1.700232in}}{\pgfqpoint{1.589191in}{1.692418in}}%
\pgfpathcurveto{\pgfqpoint{1.597005in}{1.684605in}}{\pgfqpoint{1.607604in}{1.680215in}}{\pgfqpoint{1.618654in}{1.680215in}}%
\pgfpathclose%
\pgfusepath{stroke,fill}%
\end{pgfscope}%
\begin{pgfscope}%
\pgfpathrectangle{\pgfqpoint{0.600000in}{0.600000in}}{\pgfqpoint{3.900000in}{3.900000in}}%
\pgfusepath{clip}%
\pgfsetbuttcap%
\pgfsetroundjoin%
\definecolor{currentfill}{rgb}{0.121569,0.466667,0.705882}%
\pgfsetfillcolor{currentfill}%
\pgfsetlinewidth{1.003750pt}%
\definecolor{currentstroke}{rgb}{0.121569,0.466667,0.705882}%
\pgfsetstrokecolor{currentstroke}%
\pgfsetdash{}{0pt}%
\pgfpathmoveto{\pgfqpoint{2.710923in}{3.037433in}}%
\pgfpathcurveto{\pgfqpoint{2.721973in}{3.037433in}}{\pgfqpoint{2.732572in}{3.041823in}}{\pgfqpoint{2.740386in}{3.049637in}}%
\pgfpathcurveto{\pgfqpoint{2.748200in}{3.057450in}}{\pgfqpoint{2.752590in}{3.068050in}}{\pgfqpoint{2.752590in}{3.079100in}}%
\pgfpathcurveto{\pgfqpoint{2.752590in}{3.090150in}}{\pgfqpoint{2.748200in}{3.100749in}}{\pgfqpoint{2.740386in}{3.108562in}}%
\pgfpathcurveto{\pgfqpoint{2.732572in}{3.116376in}}{\pgfqpoint{2.721973in}{3.120766in}}{\pgfqpoint{2.710923in}{3.120766in}}%
\pgfpathcurveto{\pgfqpoint{2.699873in}{3.120766in}}{\pgfqpoint{2.689274in}{3.116376in}}{\pgfqpoint{2.681460in}{3.108562in}}%
\pgfpathcurveto{\pgfqpoint{2.673647in}{3.100749in}}{\pgfqpoint{2.669256in}{3.090150in}}{\pgfqpoint{2.669256in}{3.079100in}}%
\pgfpathcurveto{\pgfqpoint{2.669256in}{3.068050in}}{\pgfqpoint{2.673647in}{3.057450in}}{\pgfqpoint{2.681460in}{3.049637in}}%
\pgfpathcurveto{\pgfqpoint{2.689274in}{3.041823in}}{\pgfqpoint{2.699873in}{3.037433in}}{\pgfqpoint{2.710923in}{3.037433in}}%
\pgfpathclose%
\pgfusepath{stroke,fill}%
\end{pgfscope}%
\begin{pgfscope}%
\pgfpathrectangle{\pgfqpoint{0.600000in}{0.600000in}}{\pgfqpoint{3.900000in}{3.900000in}}%
\pgfusepath{clip}%
\pgfsetbuttcap%
\pgfsetroundjoin%
\definecolor{currentfill}{rgb}{0.121569,0.466667,0.705882}%
\pgfsetfillcolor{currentfill}%
\pgfsetlinewidth{1.003750pt}%
\definecolor{currentstroke}{rgb}{0.121569,0.466667,0.705882}%
\pgfsetstrokecolor{currentstroke}%
\pgfsetdash{}{0pt}%
\pgfpathmoveto{\pgfqpoint{2.539640in}{2.073808in}}%
\pgfpathcurveto{\pgfqpoint{2.550690in}{2.073808in}}{\pgfqpoint{2.561289in}{2.078198in}}{\pgfqpoint{2.569103in}{2.086012in}}%
\pgfpathcurveto{\pgfqpoint{2.576916in}{2.093825in}}{\pgfqpoint{2.581306in}{2.104424in}}{\pgfqpoint{2.581306in}{2.115474in}}%
\pgfpathcurveto{\pgfqpoint{2.581306in}{2.126525in}}{\pgfqpoint{2.576916in}{2.137124in}}{\pgfqpoint{2.569103in}{2.144937in}}%
\pgfpathcurveto{\pgfqpoint{2.561289in}{2.152751in}}{\pgfqpoint{2.550690in}{2.157141in}}{\pgfqpoint{2.539640in}{2.157141in}}%
\pgfpathcurveto{\pgfqpoint{2.528590in}{2.157141in}}{\pgfqpoint{2.517991in}{2.152751in}}{\pgfqpoint{2.510177in}{2.144937in}}%
\pgfpathcurveto{\pgfqpoint{2.502363in}{2.137124in}}{\pgfqpoint{2.497973in}{2.126525in}}{\pgfqpoint{2.497973in}{2.115474in}}%
\pgfpathcurveto{\pgfqpoint{2.497973in}{2.104424in}}{\pgfqpoint{2.502363in}{2.093825in}}{\pgfqpoint{2.510177in}{2.086012in}}%
\pgfpathcurveto{\pgfqpoint{2.517991in}{2.078198in}}{\pgfqpoint{2.528590in}{2.073808in}}{\pgfqpoint{2.539640in}{2.073808in}}%
\pgfpathclose%
\pgfusepath{stroke,fill}%
\end{pgfscope}%
\begin{pgfscope}%
\pgfpathrectangle{\pgfqpoint{0.600000in}{0.600000in}}{\pgfqpoint{3.900000in}{3.900000in}}%
\pgfusepath{clip}%
\pgfsetbuttcap%
\pgfsetroundjoin%
\definecolor{currentfill}{rgb}{0.121569,0.466667,0.705882}%
\pgfsetfillcolor{currentfill}%
\pgfsetlinewidth{1.003750pt}%
\definecolor{currentstroke}{rgb}{0.121569,0.466667,0.705882}%
\pgfsetstrokecolor{currentstroke}%
\pgfsetdash{}{0pt}%
\pgfpathmoveto{\pgfqpoint{2.581136in}{3.528507in}}%
\pgfpathcurveto{\pgfqpoint{2.592187in}{3.528507in}}{\pgfqpoint{2.602786in}{3.532897in}}{\pgfqpoint{2.610599in}{3.540711in}}%
\pgfpathcurveto{\pgfqpoint{2.618413in}{3.548524in}}{\pgfqpoint{2.622803in}{3.559123in}}{\pgfqpoint{2.622803in}{3.570174in}}%
\pgfpathcurveto{\pgfqpoint{2.622803in}{3.581224in}}{\pgfqpoint{2.618413in}{3.591823in}}{\pgfqpoint{2.610599in}{3.599636in}}%
\pgfpathcurveto{\pgfqpoint{2.602786in}{3.607450in}}{\pgfqpoint{2.592187in}{3.611840in}}{\pgfqpoint{2.581136in}{3.611840in}}%
\pgfpathcurveto{\pgfqpoint{2.570086in}{3.611840in}}{\pgfqpoint{2.559487in}{3.607450in}}{\pgfqpoint{2.551674in}{3.599636in}}%
\pgfpathcurveto{\pgfqpoint{2.543860in}{3.591823in}}{\pgfqpoint{2.539470in}{3.581224in}}{\pgfqpoint{2.539470in}{3.570174in}}%
\pgfpathcurveto{\pgfqpoint{2.539470in}{3.559123in}}{\pgfqpoint{2.543860in}{3.548524in}}{\pgfqpoint{2.551674in}{3.540711in}}%
\pgfpathcurveto{\pgfqpoint{2.559487in}{3.532897in}}{\pgfqpoint{2.570086in}{3.528507in}}{\pgfqpoint{2.581136in}{3.528507in}}%
\pgfpathclose%
\pgfusepath{stroke,fill}%
\end{pgfscope}%
\begin{pgfscope}%
\pgfpathrectangle{\pgfqpoint{0.600000in}{0.600000in}}{\pgfqpoint{3.900000in}{3.900000in}}%
\pgfusepath{clip}%
\pgfsetbuttcap%
\pgfsetroundjoin%
\definecolor{currentfill}{rgb}{0.121569,0.466667,0.705882}%
\pgfsetfillcolor{currentfill}%
\pgfsetlinewidth{1.003750pt}%
\definecolor{currentstroke}{rgb}{0.121569,0.466667,0.705882}%
\pgfsetstrokecolor{currentstroke}%
\pgfsetdash{}{0pt}%
\pgfpathmoveto{\pgfqpoint{3.040009in}{2.434920in}}%
\pgfpathcurveto{\pgfqpoint{3.051059in}{2.434920in}}{\pgfqpoint{3.061659in}{2.439310in}}{\pgfqpoint{3.069472in}{2.447124in}}%
\pgfpathcurveto{\pgfqpoint{3.077286in}{2.454937in}}{\pgfqpoint{3.081676in}{2.465536in}}{\pgfqpoint{3.081676in}{2.476586in}}%
\pgfpathcurveto{\pgfqpoint{3.081676in}{2.487636in}}{\pgfqpoint{3.077286in}{2.498235in}}{\pgfqpoint{3.069472in}{2.506049in}}%
\pgfpathcurveto{\pgfqpoint{3.061659in}{2.513863in}}{\pgfqpoint{3.051059in}{2.518253in}}{\pgfqpoint{3.040009in}{2.518253in}}%
\pgfpathcurveto{\pgfqpoint{3.028959in}{2.518253in}}{\pgfqpoint{3.018360in}{2.513863in}}{\pgfqpoint{3.010547in}{2.506049in}}%
\pgfpathcurveto{\pgfqpoint{3.002733in}{2.498235in}}{\pgfqpoint{2.998343in}{2.487636in}}{\pgfqpoint{2.998343in}{2.476586in}}%
\pgfpathcurveto{\pgfqpoint{2.998343in}{2.465536in}}{\pgfqpoint{3.002733in}{2.454937in}}{\pgfqpoint{3.010547in}{2.447124in}}%
\pgfpathcurveto{\pgfqpoint{3.018360in}{2.439310in}}{\pgfqpoint{3.028959in}{2.434920in}}{\pgfqpoint{3.040009in}{2.434920in}}%
\pgfpathclose%
\pgfusepath{stroke,fill}%
\end{pgfscope}%
\begin{pgfscope}%
\pgfpathrectangle{\pgfqpoint{0.600000in}{0.600000in}}{\pgfqpoint{3.900000in}{3.900000in}}%
\pgfusepath{clip}%
\pgfsetbuttcap%
\pgfsetroundjoin%
\definecolor{currentfill}{rgb}{0.121569,0.466667,0.705882}%
\pgfsetfillcolor{currentfill}%
\pgfsetlinewidth{1.003750pt}%
\definecolor{currentstroke}{rgb}{0.121569,0.466667,0.705882}%
\pgfsetstrokecolor{currentstroke}%
\pgfsetdash{}{0pt}%
\pgfpathmoveto{\pgfqpoint{2.635443in}{3.132789in}}%
\pgfpathcurveto{\pgfqpoint{2.646493in}{3.132789in}}{\pgfqpoint{2.657092in}{3.137179in}}{\pgfqpoint{2.664906in}{3.144993in}}%
\pgfpathcurveto{\pgfqpoint{2.672719in}{3.152806in}}{\pgfqpoint{2.677110in}{3.163405in}}{\pgfqpoint{2.677110in}{3.174456in}}%
\pgfpathcurveto{\pgfqpoint{2.677110in}{3.185506in}}{\pgfqpoint{2.672719in}{3.196105in}}{\pgfqpoint{2.664906in}{3.203918in}}%
\pgfpathcurveto{\pgfqpoint{2.657092in}{3.211732in}}{\pgfqpoint{2.646493in}{3.216122in}}{\pgfqpoint{2.635443in}{3.216122in}}%
\pgfpathcurveto{\pgfqpoint{2.624393in}{3.216122in}}{\pgfqpoint{2.613794in}{3.211732in}}{\pgfqpoint{2.605980in}{3.203918in}}%
\pgfpathcurveto{\pgfqpoint{2.598166in}{3.196105in}}{\pgfqpoint{2.593776in}{3.185506in}}{\pgfqpoint{2.593776in}{3.174456in}}%
\pgfpathcurveto{\pgfqpoint{2.593776in}{3.163405in}}{\pgfqpoint{2.598166in}{3.152806in}}{\pgfqpoint{2.605980in}{3.144993in}}%
\pgfpathcurveto{\pgfqpoint{2.613794in}{3.137179in}}{\pgfqpoint{2.624393in}{3.132789in}}{\pgfqpoint{2.635443in}{3.132789in}}%
\pgfpathclose%
\pgfusepath{stroke,fill}%
\end{pgfscope}%
\begin{pgfscope}%
\pgfpathrectangle{\pgfqpoint{0.600000in}{0.600000in}}{\pgfqpoint{3.900000in}{3.900000in}}%
\pgfusepath{clip}%
\pgfsetbuttcap%
\pgfsetroundjoin%
\definecolor{currentfill}{rgb}{0.121569,0.466667,0.705882}%
\pgfsetfillcolor{currentfill}%
\pgfsetlinewidth{1.003750pt}%
\definecolor{currentstroke}{rgb}{0.121569,0.466667,0.705882}%
\pgfsetstrokecolor{currentstroke}%
\pgfsetdash{}{0pt}%
\pgfpathmoveto{\pgfqpoint{2.960465in}{2.026199in}}%
\pgfpathcurveto{\pgfqpoint{2.971515in}{2.026199in}}{\pgfqpoint{2.982114in}{2.030590in}}{\pgfqpoint{2.989928in}{2.038403in}}%
\pgfpathcurveto{\pgfqpoint{2.997741in}{2.046217in}}{\pgfqpoint{3.002132in}{2.056816in}}{\pgfqpoint{3.002132in}{2.067866in}}%
\pgfpathcurveto{\pgfqpoint{3.002132in}{2.078916in}}{\pgfqpoint{2.997741in}{2.089515in}}{\pgfqpoint{2.989928in}{2.097329in}}%
\pgfpathcurveto{\pgfqpoint{2.982114in}{2.105142in}}{\pgfqpoint{2.971515in}{2.109533in}}{\pgfqpoint{2.960465in}{2.109533in}}%
\pgfpathcurveto{\pgfqpoint{2.949415in}{2.109533in}}{\pgfqpoint{2.938816in}{2.105142in}}{\pgfqpoint{2.931002in}{2.097329in}}%
\pgfpathcurveto{\pgfqpoint{2.923189in}{2.089515in}}{\pgfqpoint{2.918798in}{2.078916in}}{\pgfqpoint{2.918798in}{2.067866in}}%
\pgfpathcurveto{\pgfqpoint{2.918798in}{2.056816in}}{\pgfqpoint{2.923189in}{2.046217in}}{\pgfqpoint{2.931002in}{2.038403in}}%
\pgfpathcurveto{\pgfqpoint{2.938816in}{2.030590in}}{\pgfqpoint{2.949415in}{2.026199in}}{\pgfqpoint{2.960465in}{2.026199in}}%
\pgfpathclose%
\pgfusepath{stroke,fill}%
\end{pgfscope}%
\begin{pgfscope}%
\pgfpathrectangle{\pgfqpoint{0.600000in}{0.600000in}}{\pgfqpoint{3.900000in}{3.900000in}}%
\pgfusepath{clip}%
\pgfsetbuttcap%
\pgfsetroundjoin%
\definecolor{currentfill}{rgb}{0.121569,0.466667,0.705882}%
\pgfsetfillcolor{currentfill}%
\pgfsetlinewidth{1.003750pt}%
\definecolor{currentstroke}{rgb}{0.121569,0.466667,0.705882}%
\pgfsetstrokecolor{currentstroke}%
\pgfsetdash{}{0pt}%
\pgfpathmoveto{\pgfqpoint{2.602960in}{2.278709in}}%
\pgfpathcurveto{\pgfqpoint{2.614010in}{2.278709in}}{\pgfqpoint{2.624609in}{2.283099in}}{\pgfqpoint{2.632423in}{2.290913in}}%
\pgfpathcurveto{\pgfqpoint{2.640237in}{2.298727in}}{\pgfqpoint{2.644627in}{2.309326in}}{\pgfqpoint{2.644627in}{2.320376in}}%
\pgfpathcurveto{\pgfqpoint{2.644627in}{2.331426in}}{\pgfqpoint{2.640237in}{2.342025in}}{\pgfqpoint{2.632423in}{2.349838in}}%
\pgfpathcurveto{\pgfqpoint{2.624609in}{2.357652in}}{\pgfqpoint{2.614010in}{2.362042in}}{\pgfqpoint{2.602960in}{2.362042in}}%
\pgfpathcurveto{\pgfqpoint{2.591910in}{2.362042in}}{\pgfqpoint{2.581311in}{2.357652in}}{\pgfqpoint{2.573497in}{2.349838in}}%
\pgfpathcurveto{\pgfqpoint{2.565684in}{2.342025in}}{\pgfqpoint{2.561293in}{2.331426in}}{\pgfqpoint{2.561293in}{2.320376in}}%
\pgfpathcurveto{\pgfqpoint{2.561293in}{2.309326in}}{\pgfqpoint{2.565684in}{2.298727in}}{\pgfqpoint{2.573497in}{2.290913in}}%
\pgfpathcurveto{\pgfqpoint{2.581311in}{2.283099in}}{\pgfqpoint{2.591910in}{2.278709in}}{\pgfqpoint{2.602960in}{2.278709in}}%
\pgfpathclose%
\pgfusepath{stroke,fill}%
\end{pgfscope}%
\begin{pgfscope}%
\pgfpathrectangle{\pgfqpoint{0.600000in}{0.600000in}}{\pgfqpoint{3.900000in}{3.900000in}}%
\pgfusepath{clip}%
\pgfsetbuttcap%
\pgfsetroundjoin%
\definecolor{currentfill}{rgb}{0.121569,0.466667,0.705882}%
\pgfsetfillcolor{currentfill}%
\pgfsetlinewidth{1.003750pt}%
\definecolor{currentstroke}{rgb}{0.121569,0.466667,0.705882}%
\pgfsetstrokecolor{currentstroke}%
\pgfsetdash{}{0pt}%
\pgfpathmoveto{\pgfqpoint{2.602629in}{2.231664in}}%
\pgfpathcurveto{\pgfqpoint{2.613679in}{2.231664in}}{\pgfqpoint{2.624279in}{2.236055in}}{\pgfqpoint{2.632092in}{2.243868in}}%
\pgfpathcurveto{\pgfqpoint{2.639906in}{2.251682in}}{\pgfqpoint{2.644296in}{2.262281in}}{\pgfqpoint{2.644296in}{2.273331in}}%
\pgfpathcurveto{\pgfqpoint{2.644296in}{2.284381in}}{\pgfqpoint{2.639906in}{2.294980in}}{\pgfqpoint{2.632092in}{2.302794in}}%
\pgfpathcurveto{\pgfqpoint{2.624279in}{2.310607in}}{\pgfqpoint{2.613679in}{2.314998in}}{\pgfqpoint{2.602629in}{2.314998in}}%
\pgfpathcurveto{\pgfqpoint{2.591579in}{2.314998in}}{\pgfqpoint{2.580980in}{2.310607in}}{\pgfqpoint{2.573167in}{2.302794in}}%
\pgfpathcurveto{\pgfqpoint{2.565353in}{2.294980in}}{\pgfqpoint{2.560963in}{2.284381in}}{\pgfqpoint{2.560963in}{2.273331in}}%
\pgfpathcurveto{\pgfqpoint{2.560963in}{2.262281in}}{\pgfqpoint{2.565353in}{2.251682in}}{\pgfqpoint{2.573167in}{2.243868in}}%
\pgfpathcurveto{\pgfqpoint{2.580980in}{2.236055in}}{\pgfqpoint{2.591579in}{2.231664in}}{\pgfqpoint{2.602629in}{2.231664in}}%
\pgfpathclose%
\pgfusepath{stroke,fill}%
\end{pgfscope}%
\begin{pgfscope}%
\pgfpathrectangle{\pgfqpoint{0.600000in}{0.600000in}}{\pgfqpoint{3.900000in}{3.900000in}}%
\pgfusepath{clip}%
\pgfsetbuttcap%
\pgfsetroundjoin%
\definecolor{currentfill}{rgb}{0.121569,0.466667,0.705882}%
\pgfsetfillcolor{currentfill}%
\pgfsetlinewidth{1.003750pt}%
\definecolor{currentstroke}{rgb}{0.121569,0.466667,0.705882}%
\pgfsetstrokecolor{currentstroke}%
\pgfsetdash{}{0pt}%
\pgfpathmoveto{\pgfqpoint{2.648596in}{2.424986in}}%
\pgfpathcurveto{\pgfqpoint{2.659647in}{2.424986in}}{\pgfqpoint{2.670246in}{2.429376in}}{\pgfqpoint{2.678059in}{2.437190in}}%
\pgfpathcurveto{\pgfqpoint{2.685873in}{2.445004in}}{\pgfqpoint{2.690263in}{2.455603in}}{\pgfqpoint{2.690263in}{2.466653in}}%
\pgfpathcurveto{\pgfqpoint{2.690263in}{2.477703in}}{\pgfqpoint{2.685873in}{2.488302in}}{\pgfqpoint{2.678059in}{2.496116in}}%
\pgfpathcurveto{\pgfqpoint{2.670246in}{2.503929in}}{\pgfqpoint{2.659647in}{2.508319in}}{\pgfqpoint{2.648596in}{2.508319in}}%
\pgfpathcurveto{\pgfqpoint{2.637546in}{2.508319in}}{\pgfqpoint{2.626947in}{2.503929in}}{\pgfqpoint{2.619134in}{2.496116in}}%
\pgfpathcurveto{\pgfqpoint{2.611320in}{2.488302in}}{\pgfqpoint{2.606930in}{2.477703in}}{\pgfqpoint{2.606930in}{2.466653in}}%
\pgfpathcurveto{\pgfqpoint{2.606930in}{2.455603in}}{\pgfqpoint{2.611320in}{2.445004in}}{\pgfqpoint{2.619134in}{2.437190in}}%
\pgfpathcurveto{\pgfqpoint{2.626947in}{2.429376in}}{\pgfqpoint{2.637546in}{2.424986in}}{\pgfqpoint{2.648596in}{2.424986in}}%
\pgfpathclose%
\pgfusepath{stroke,fill}%
\end{pgfscope}%
\begin{pgfscope}%
\pgfpathrectangle{\pgfqpoint{0.600000in}{0.600000in}}{\pgfqpoint{3.900000in}{3.900000in}}%
\pgfusepath{clip}%
\pgfsetbuttcap%
\pgfsetroundjoin%
\definecolor{currentfill}{rgb}{0.121569,0.466667,0.705882}%
\pgfsetfillcolor{currentfill}%
\pgfsetlinewidth{1.003750pt}%
\definecolor{currentstroke}{rgb}{0.121569,0.466667,0.705882}%
\pgfsetstrokecolor{currentstroke}%
\pgfsetdash{}{0pt}%
\pgfpathmoveto{\pgfqpoint{2.400509in}{2.109513in}}%
\pgfpathcurveto{\pgfqpoint{2.411559in}{2.109513in}}{\pgfqpoint{2.422158in}{2.113903in}}{\pgfqpoint{2.429971in}{2.121717in}}%
\pgfpathcurveto{\pgfqpoint{2.437785in}{2.129531in}}{\pgfqpoint{2.442175in}{2.140130in}}{\pgfqpoint{2.442175in}{2.151180in}}%
\pgfpathcurveto{\pgfqpoint{2.442175in}{2.162230in}}{\pgfqpoint{2.437785in}{2.172829in}}{\pgfqpoint{2.429971in}{2.180643in}}%
\pgfpathcurveto{\pgfqpoint{2.422158in}{2.188456in}}{\pgfqpoint{2.411559in}{2.192846in}}{\pgfqpoint{2.400509in}{2.192846in}}%
\pgfpathcurveto{\pgfqpoint{2.389458in}{2.192846in}}{\pgfqpoint{2.378859in}{2.188456in}}{\pgfqpoint{2.371046in}{2.180643in}}%
\pgfpathcurveto{\pgfqpoint{2.363232in}{2.172829in}}{\pgfqpoint{2.358842in}{2.162230in}}{\pgfqpoint{2.358842in}{2.151180in}}%
\pgfpathcurveto{\pgfqpoint{2.358842in}{2.140130in}}{\pgfqpoint{2.363232in}{2.129531in}}{\pgfqpoint{2.371046in}{2.121717in}}%
\pgfpathcurveto{\pgfqpoint{2.378859in}{2.113903in}}{\pgfqpoint{2.389458in}{2.109513in}}{\pgfqpoint{2.400509in}{2.109513in}}%
\pgfpathclose%
\pgfusepath{stroke,fill}%
\end{pgfscope}%
\begin{pgfscope}%
\pgfpathrectangle{\pgfqpoint{0.600000in}{0.600000in}}{\pgfqpoint{3.900000in}{3.900000in}}%
\pgfusepath{clip}%
\pgfsetbuttcap%
\pgfsetroundjoin%
\definecolor{currentfill}{rgb}{0.121569,0.466667,0.705882}%
\pgfsetfillcolor{currentfill}%
\pgfsetlinewidth{1.003750pt}%
\definecolor{currentstroke}{rgb}{0.121569,0.466667,0.705882}%
\pgfsetstrokecolor{currentstroke}%
\pgfsetdash{}{0pt}%
\pgfpathmoveto{\pgfqpoint{2.694652in}{2.767584in}}%
\pgfpathcurveto{\pgfqpoint{2.705702in}{2.767584in}}{\pgfqpoint{2.716301in}{2.771974in}}{\pgfqpoint{2.724115in}{2.779788in}}%
\pgfpathcurveto{\pgfqpoint{2.731929in}{2.787602in}}{\pgfqpoint{2.736319in}{2.798201in}}{\pgfqpoint{2.736319in}{2.809251in}}%
\pgfpathcurveto{\pgfqpoint{2.736319in}{2.820301in}}{\pgfqpoint{2.731929in}{2.830900in}}{\pgfqpoint{2.724115in}{2.838714in}}%
\pgfpathcurveto{\pgfqpoint{2.716301in}{2.846527in}}{\pgfqpoint{2.705702in}{2.850918in}}{\pgfqpoint{2.694652in}{2.850918in}}%
\pgfpathcurveto{\pgfqpoint{2.683602in}{2.850918in}}{\pgfqpoint{2.673003in}{2.846527in}}{\pgfqpoint{2.665189in}{2.838714in}}%
\pgfpathcurveto{\pgfqpoint{2.657376in}{2.830900in}}{\pgfqpoint{2.652986in}{2.820301in}}{\pgfqpoint{2.652986in}{2.809251in}}%
\pgfpathcurveto{\pgfqpoint{2.652986in}{2.798201in}}{\pgfqpoint{2.657376in}{2.787602in}}{\pgfqpoint{2.665189in}{2.779788in}}%
\pgfpathcurveto{\pgfqpoint{2.673003in}{2.771974in}}{\pgfqpoint{2.683602in}{2.767584in}}{\pgfqpoint{2.694652in}{2.767584in}}%
\pgfpathclose%
\pgfusepath{stroke,fill}%
\end{pgfscope}%
\begin{pgfscope}%
\pgfpathrectangle{\pgfqpoint{0.600000in}{0.600000in}}{\pgfqpoint{3.900000in}{3.900000in}}%
\pgfusepath{clip}%
\pgfsetbuttcap%
\pgfsetroundjoin%
\definecolor{currentfill}{rgb}{0.121569,0.466667,0.705882}%
\pgfsetfillcolor{currentfill}%
\pgfsetlinewidth{1.003750pt}%
\definecolor{currentstroke}{rgb}{0.121569,0.466667,0.705882}%
\pgfsetstrokecolor{currentstroke}%
\pgfsetdash{}{0pt}%
\pgfpathmoveto{\pgfqpoint{2.498254in}{1.706300in}}%
\pgfpathcurveto{\pgfqpoint{2.509304in}{1.706300in}}{\pgfqpoint{2.519903in}{1.710690in}}{\pgfqpoint{2.527717in}{1.718504in}}%
\pgfpathcurveto{\pgfqpoint{2.535531in}{1.726317in}}{\pgfqpoint{2.539921in}{1.736916in}}{\pgfqpoint{2.539921in}{1.747966in}}%
\pgfpathcurveto{\pgfqpoint{2.539921in}{1.759017in}}{\pgfqpoint{2.535531in}{1.769616in}}{\pgfqpoint{2.527717in}{1.777429in}}%
\pgfpathcurveto{\pgfqpoint{2.519903in}{1.785243in}}{\pgfqpoint{2.509304in}{1.789633in}}{\pgfqpoint{2.498254in}{1.789633in}}%
\pgfpathcurveto{\pgfqpoint{2.487204in}{1.789633in}}{\pgfqpoint{2.476605in}{1.785243in}}{\pgfqpoint{2.468791in}{1.777429in}}%
\pgfpathcurveto{\pgfqpoint{2.460978in}{1.769616in}}{\pgfqpoint{2.456588in}{1.759017in}}{\pgfqpoint{2.456588in}{1.747966in}}%
\pgfpathcurveto{\pgfqpoint{2.456588in}{1.736916in}}{\pgfqpoint{2.460978in}{1.726317in}}{\pgfqpoint{2.468791in}{1.718504in}}%
\pgfpathcurveto{\pgfqpoint{2.476605in}{1.710690in}}{\pgfqpoint{2.487204in}{1.706300in}}{\pgfqpoint{2.498254in}{1.706300in}}%
\pgfpathclose%
\pgfusepath{stroke,fill}%
\end{pgfscope}%
\begin{pgfscope}%
\pgfpathrectangle{\pgfqpoint{0.600000in}{0.600000in}}{\pgfqpoint{3.900000in}{3.900000in}}%
\pgfusepath{clip}%
\pgfsetbuttcap%
\pgfsetroundjoin%
\definecolor{currentfill}{rgb}{0.121569,0.466667,0.705882}%
\pgfsetfillcolor{currentfill}%
\pgfsetlinewidth{1.003750pt}%
\definecolor{currentstroke}{rgb}{0.121569,0.466667,0.705882}%
\pgfsetstrokecolor{currentstroke}%
\pgfsetdash{}{0pt}%
\pgfpathmoveto{\pgfqpoint{2.315140in}{2.745724in}}%
\pgfpathcurveto{\pgfqpoint{2.326190in}{2.745724in}}{\pgfqpoint{2.336789in}{2.750114in}}{\pgfqpoint{2.344602in}{2.757928in}}%
\pgfpathcurveto{\pgfqpoint{2.352416in}{2.765741in}}{\pgfqpoint{2.356806in}{2.776340in}}{\pgfqpoint{2.356806in}{2.787390in}}%
\pgfpathcurveto{\pgfqpoint{2.356806in}{2.798441in}}{\pgfqpoint{2.352416in}{2.809040in}}{\pgfqpoint{2.344602in}{2.816853in}}%
\pgfpathcurveto{\pgfqpoint{2.336789in}{2.824667in}}{\pgfqpoint{2.326190in}{2.829057in}}{\pgfqpoint{2.315140in}{2.829057in}}%
\pgfpathcurveto{\pgfqpoint{2.304089in}{2.829057in}}{\pgfqpoint{2.293490in}{2.824667in}}{\pgfqpoint{2.285677in}{2.816853in}}%
\pgfpathcurveto{\pgfqpoint{2.277863in}{2.809040in}}{\pgfqpoint{2.273473in}{2.798441in}}{\pgfqpoint{2.273473in}{2.787390in}}%
\pgfpathcurveto{\pgfqpoint{2.273473in}{2.776340in}}{\pgfqpoint{2.277863in}{2.765741in}}{\pgfqpoint{2.285677in}{2.757928in}}%
\pgfpathcurveto{\pgfqpoint{2.293490in}{2.750114in}}{\pgfqpoint{2.304089in}{2.745724in}}{\pgfqpoint{2.315140in}{2.745724in}}%
\pgfpathclose%
\pgfusepath{stroke,fill}%
\end{pgfscope}%
\begin{pgfscope}%
\pgfpathrectangle{\pgfqpoint{0.600000in}{0.600000in}}{\pgfqpoint{3.900000in}{3.900000in}}%
\pgfusepath{clip}%
\pgfsetbuttcap%
\pgfsetroundjoin%
\definecolor{currentfill}{rgb}{0.121569,0.466667,0.705882}%
\pgfsetfillcolor{currentfill}%
\pgfsetlinewidth{1.003750pt}%
\definecolor{currentstroke}{rgb}{0.121569,0.466667,0.705882}%
\pgfsetstrokecolor{currentstroke}%
\pgfsetdash{}{0pt}%
\pgfpathmoveto{\pgfqpoint{2.731780in}{3.226457in}}%
\pgfpathcurveto{\pgfqpoint{2.742830in}{3.226457in}}{\pgfqpoint{2.753429in}{3.230847in}}{\pgfqpoint{2.761243in}{3.238661in}}%
\pgfpathcurveto{\pgfqpoint{2.769057in}{3.246475in}}{\pgfqpoint{2.773447in}{3.257074in}}{\pgfqpoint{2.773447in}{3.268124in}}%
\pgfpathcurveto{\pgfqpoint{2.773447in}{3.279174in}}{\pgfqpoint{2.769057in}{3.289773in}}{\pgfqpoint{2.761243in}{3.297587in}}%
\pgfpathcurveto{\pgfqpoint{2.753429in}{3.305400in}}{\pgfqpoint{2.742830in}{3.309791in}}{\pgfqpoint{2.731780in}{3.309791in}}%
\pgfpathcurveto{\pgfqpoint{2.720730in}{3.309791in}}{\pgfqpoint{2.710131in}{3.305400in}}{\pgfqpoint{2.702317in}{3.297587in}}%
\pgfpathcurveto{\pgfqpoint{2.694504in}{3.289773in}}{\pgfqpoint{2.690113in}{3.279174in}}{\pgfqpoint{2.690113in}{3.268124in}}%
\pgfpathcurveto{\pgfqpoint{2.690113in}{3.257074in}}{\pgfqpoint{2.694504in}{3.246475in}}{\pgfqpoint{2.702317in}{3.238661in}}%
\pgfpathcurveto{\pgfqpoint{2.710131in}{3.230847in}}{\pgfqpoint{2.720730in}{3.226457in}}{\pgfqpoint{2.731780in}{3.226457in}}%
\pgfpathclose%
\pgfusepath{stroke,fill}%
\end{pgfscope}%
\begin{pgfscope}%
\pgfpathrectangle{\pgfqpoint{0.600000in}{0.600000in}}{\pgfqpoint{3.900000in}{3.900000in}}%
\pgfusepath{clip}%
\pgfsetbuttcap%
\pgfsetroundjoin%
\definecolor{currentfill}{rgb}{0.121569,0.466667,0.705882}%
\pgfsetfillcolor{currentfill}%
\pgfsetlinewidth{1.003750pt}%
\definecolor{currentstroke}{rgb}{0.121569,0.466667,0.705882}%
\pgfsetstrokecolor{currentstroke}%
\pgfsetdash{}{0pt}%
\pgfpathmoveto{\pgfqpoint{3.256057in}{2.203584in}}%
\pgfpathcurveto{\pgfqpoint{3.267107in}{2.203584in}}{\pgfqpoint{3.277706in}{2.207974in}}{\pgfqpoint{3.285519in}{2.215788in}}%
\pgfpathcurveto{\pgfqpoint{3.293333in}{2.223601in}}{\pgfqpoint{3.297723in}{2.234200in}}{\pgfqpoint{3.297723in}{2.245251in}}%
\pgfpathcurveto{\pgfqpoint{3.297723in}{2.256301in}}{\pgfqpoint{3.293333in}{2.266900in}}{\pgfqpoint{3.285519in}{2.274713in}}%
\pgfpathcurveto{\pgfqpoint{3.277706in}{2.282527in}}{\pgfqpoint{3.267107in}{2.286917in}}{\pgfqpoint{3.256057in}{2.286917in}}%
\pgfpathcurveto{\pgfqpoint{3.245006in}{2.286917in}}{\pgfqpoint{3.234407in}{2.282527in}}{\pgfqpoint{3.226594in}{2.274713in}}%
\pgfpathcurveto{\pgfqpoint{3.218780in}{2.266900in}}{\pgfqpoint{3.214390in}{2.256301in}}{\pgfqpoint{3.214390in}{2.245251in}}%
\pgfpathcurveto{\pgfqpoint{3.214390in}{2.234200in}}{\pgfqpoint{3.218780in}{2.223601in}}{\pgfqpoint{3.226594in}{2.215788in}}%
\pgfpathcurveto{\pgfqpoint{3.234407in}{2.207974in}}{\pgfqpoint{3.245006in}{2.203584in}}{\pgfqpoint{3.256057in}{2.203584in}}%
\pgfpathclose%
\pgfusepath{stroke,fill}%
\end{pgfscope}%
\begin{pgfscope}%
\pgfpathrectangle{\pgfqpoint{0.600000in}{0.600000in}}{\pgfqpoint{3.900000in}{3.900000in}}%
\pgfusepath{clip}%
\pgfsetbuttcap%
\pgfsetroundjoin%
\definecolor{currentfill}{rgb}{0.121569,0.466667,0.705882}%
\pgfsetfillcolor{currentfill}%
\pgfsetlinewidth{1.003750pt}%
\definecolor{currentstroke}{rgb}{0.121569,0.466667,0.705882}%
\pgfsetstrokecolor{currentstroke}%
\pgfsetdash{}{0pt}%
\pgfpathmoveto{\pgfqpoint{2.367027in}{2.430338in}}%
\pgfpathcurveto{\pgfqpoint{2.378077in}{2.430338in}}{\pgfqpoint{2.388676in}{2.434729in}}{\pgfqpoint{2.396489in}{2.442542in}}%
\pgfpathcurveto{\pgfqpoint{2.404303in}{2.450356in}}{\pgfqpoint{2.408693in}{2.460955in}}{\pgfqpoint{2.408693in}{2.472005in}}%
\pgfpathcurveto{\pgfqpoint{2.408693in}{2.483055in}}{\pgfqpoint{2.404303in}{2.493654in}}{\pgfqpoint{2.396489in}{2.501468in}}%
\pgfpathcurveto{\pgfqpoint{2.388676in}{2.509282in}}{\pgfqpoint{2.378077in}{2.513672in}}{\pgfqpoint{2.367027in}{2.513672in}}%
\pgfpathcurveto{\pgfqpoint{2.355977in}{2.513672in}}{\pgfqpoint{2.345378in}{2.509282in}}{\pgfqpoint{2.337564in}{2.501468in}}%
\pgfpathcurveto{\pgfqpoint{2.329750in}{2.493654in}}{\pgfqpoint{2.325360in}{2.483055in}}{\pgfqpoint{2.325360in}{2.472005in}}%
\pgfpathcurveto{\pgfqpoint{2.325360in}{2.460955in}}{\pgfqpoint{2.329750in}{2.450356in}}{\pgfqpoint{2.337564in}{2.442542in}}%
\pgfpathcurveto{\pgfqpoint{2.345378in}{2.434729in}}{\pgfqpoint{2.355977in}{2.430338in}}{\pgfqpoint{2.367027in}{2.430338in}}%
\pgfpathclose%
\pgfusepath{stroke,fill}%
\end{pgfscope}%
\begin{pgfscope}%
\pgfpathrectangle{\pgfqpoint{0.600000in}{0.600000in}}{\pgfqpoint{3.900000in}{3.900000in}}%
\pgfusepath{clip}%
\pgfsetbuttcap%
\pgfsetroundjoin%
\definecolor{currentfill}{rgb}{0.121569,0.466667,0.705882}%
\pgfsetfillcolor{currentfill}%
\pgfsetlinewidth{1.003750pt}%
\definecolor{currentstroke}{rgb}{0.121569,0.466667,0.705882}%
\pgfsetstrokecolor{currentstroke}%
\pgfsetdash{}{0pt}%
\pgfpathmoveto{\pgfqpoint{2.431637in}{2.403370in}}%
\pgfpathcurveto{\pgfqpoint{2.442688in}{2.403370in}}{\pgfqpoint{2.453287in}{2.407760in}}{\pgfqpoint{2.461100in}{2.415574in}}%
\pgfpathcurveto{\pgfqpoint{2.468914in}{2.423388in}}{\pgfqpoint{2.473304in}{2.433987in}}{\pgfqpoint{2.473304in}{2.445037in}}%
\pgfpathcurveto{\pgfqpoint{2.473304in}{2.456087in}}{\pgfqpoint{2.468914in}{2.466686in}}{\pgfqpoint{2.461100in}{2.474499in}}%
\pgfpathcurveto{\pgfqpoint{2.453287in}{2.482313in}}{\pgfqpoint{2.442688in}{2.486703in}}{\pgfqpoint{2.431637in}{2.486703in}}%
\pgfpathcurveto{\pgfqpoint{2.420587in}{2.486703in}}{\pgfqpoint{2.409988in}{2.482313in}}{\pgfqpoint{2.402175in}{2.474499in}}%
\pgfpathcurveto{\pgfqpoint{2.394361in}{2.466686in}}{\pgfqpoint{2.389971in}{2.456087in}}{\pgfqpoint{2.389971in}{2.445037in}}%
\pgfpathcurveto{\pgfqpoint{2.389971in}{2.433987in}}{\pgfqpoint{2.394361in}{2.423388in}}{\pgfqpoint{2.402175in}{2.415574in}}%
\pgfpathcurveto{\pgfqpoint{2.409988in}{2.407760in}}{\pgfqpoint{2.420587in}{2.403370in}}{\pgfqpoint{2.431637in}{2.403370in}}%
\pgfpathclose%
\pgfusepath{stroke,fill}%
\end{pgfscope}%
\begin{pgfscope}%
\pgfpathrectangle{\pgfqpoint{0.600000in}{0.600000in}}{\pgfqpoint{3.900000in}{3.900000in}}%
\pgfusepath{clip}%
\pgfsetbuttcap%
\pgfsetroundjoin%
\definecolor{currentfill}{rgb}{0.121569,0.466667,0.705882}%
\pgfsetfillcolor{currentfill}%
\pgfsetlinewidth{1.003750pt}%
\definecolor{currentstroke}{rgb}{0.121569,0.466667,0.705882}%
\pgfsetstrokecolor{currentstroke}%
\pgfsetdash{}{0pt}%
\pgfpathmoveto{\pgfqpoint{2.338007in}{2.203575in}}%
\pgfpathcurveto{\pgfqpoint{2.349057in}{2.203575in}}{\pgfqpoint{2.359656in}{2.207965in}}{\pgfqpoint{2.367470in}{2.215779in}}%
\pgfpathcurveto{\pgfqpoint{2.375283in}{2.223592in}}{\pgfqpoint{2.379674in}{2.234191in}}{\pgfqpoint{2.379674in}{2.245242in}}%
\pgfpathcurveto{\pgfqpoint{2.379674in}{2.256292in}}{\pgfqpoint{2.375283in}{2.266891in}}{\pgfqpoint{2.367470in}{2.274704in}}%
\pgfpathcurveto{\pgfqpoint{2.359656in}{2.282518in}}{\pgfqpoint{2.349057in}{2.286908in}}{\pgfqpoint{2.338007in}{2.286908in}}%
\pgfpathcurveto{\pgfqpoint{2.326957in}{2.286908in}}{\pgfqpoint{2.316358in}{2.282518in}}{\pgfqpoint{2.308544in}{2.274704in}}%
\pgfpathcurveto{\pgfqpoint{2.300731in}{2.266891in}}{\pgfqpoint{2.296340in}{2.256292in}}{\pgfqpoint{2.296340in}{2.245242in}}%
\pgfpathcurveto{\pgfqpoint{2.296340in}{2.234191in}}{\pgfqpoint{2.300731in}{2.223592in}}{\pgfqpoint{2.308544in}{2.215779in}}%
\pgfpathcurveto{\pgfqpoint{2.316358in}{2.207965in}}{\pgfqpoint{2.326957in}{2.203575in}}{\pgfqpoint{2.338007in}{2.203575in}}%
\pgfpathclose%
\pgfusepath{stroke,fill}%
\end{pgfscope}%
\begin{pgfscope}%
\pgfpathrectangle{\pgfqpoint{0.600000in}{0.600000in}}{\pgfqpoint{3.900000in}{3.900000in}}%
\pgfusepath{clip}%
\pgfsetbuttcap%
\pgfsetroundjoin%
\definecolor{currentfill}{rgb}{0.121569,0.466667,0.705882}%
\pgfsetfillcolor{currentfill}%
\pgfsetlinewidth{1.003750pt}%
\definecolor{currentstroke}{rgb}{0.121569,0.466667,0.705882}%
\pgfsetstrokecolor{currentstroke}%
\pgfsetdash{}{0pt}%
\pgfpathmoveto{\pgfqpoint{2.640252in}{2.679155in}}%
\pgfpathcurveto{\pgfqpoint{2.651302in}{2.679155in}}{\pgfqpoint{2.661901in}{2.683545in}}{\pgfqpoint{2.669715in}{2.691359in}}%
\pgfpathcurveto{\pgfqpoint{2.677528in}{2.699172in}}{\pgfqpoint{2.681918in}{2.709772in}}{\pgfqpoint{2.681918in}{2.720822in}}%
\pgfpathcurveto{\pgfqpoint{2.681918in}{2.731872in}}{\pgfqpoint{2.677528in}{2.742471in}}{\pgfqpoint{2.669715in}{2.750284in}}%
\pgfpathcurveto{\pgfqpoint{2.661901in}{2.758098in}}{\pgfqpoint{2.651302in}{2.762488in}}{\pgfqpoint{2.640252in}{2.762488in}}%
\pgfpathcurveto{\pgfqpoint{2.629202in}{2.762488in}}{\pgfqpoint{2.618603in}{2.758098in}}{\pgfqpoint{2.610789in}{2.750284in}}%
\pgfpathcurveto{\pgfqpoint{2.602975in}{2.742471in}}{\pgfqpoint{2.598585in}{2.731872in}}{\pgfqpoint{2.598585in}{2.720822in}}%
\pgfpathcurveto{\pgfqpoint{2.598585in}{2.709772in}}{\pgfqpoint{2.602975in}{2.699172in}}{\pgfqpoint{2.610789in}{2.691359in}}%
\pgfpathcurveto{\pgfqpoint{2.618603in}{2.683545in}}{\pgfqpoint{2.629202in}{2.679155in}}{\pgfqpoint{2.640252in}{2.679155in}}%
\pgfpathclose%
\pgfusepath{stroke,fill}%
\end{pgfscope}%
\begin{pgfscope}%
\pgfpathrectangle{\pgfqpoint{0.600000in}{0.600000in}}{\pgfqpoint{3.900000in}{3.900000in}}%
\pgfusepath{clip}%
\pgfsetbuttcap%
\pgfsetroundjoin%
\definecolor{currentfill}{rgb}{0.121569,0.466667,0.705882}%
\pgfsetfillcolor{currentfill}%
\pgfsetlinewidth{1.003750pt}%
\definecolor{currentstroke}{rgb}{0.121569,0.466667,0.705882}%
\pgfsetstrokecolor{currentstroke}%
\pgfsetdash{}{0pt}%
\pgfpathmoveto{\pgfqpoint{2.108834in}{2.990215in}}%
\pgfpathcurveto{\pgfqpoint{2.119884in}{2.990215in}}{\pgfqpoint{2.130483in}{2.994605in}}{\pgfqpoint{2.138297in}{3.002418in}}%
\pgfpathcurveto{\pgfqpoint{2.146110in}{3.010232in}}{\pgfqpoint{2.150500in}{3.020831in}}{\pgfqpoint{2.150500in}{3.031881in}}%
\pgfpathcurveto{\pgfqpoint{2.150500in}{3.042931in}}{\pgfqpoint{2.146110in}{3.053530in}}{\pgfqpoint{2.138297in}{3.061344in}}%
\pgfpathcurveto{\pgfqpoint{2.130483in}{3.069158in}}{\pgfqpoint{2.119884in}{3.073548in}}{\pgfqpoint{2.108834in}{3.073548in}}%
\pgfpathcurveto{\pgfqpoint{2.097784in}{3.073548in}}{\pgfqpoint{2.087185in}{3.069158in}}{\pgfqpoint{2.079371in}{3.061344in}}%
\pgfpathcurveto{\pgfqpoint{2.071557in}{3.053530in}}{\pgfqpoint{2.067167in}{3.042931in}}{\pgfqpoint{2.067167in}{3.031881in}}%
\pgfpathcurveto{\pgfqpoint{2.067167in}{3.020831in}}{\pgfqpoint{2.071557in}{3.010232in}}{\pgfqpoint{2.079371in}{3.002418in}}%
\pgfpathcurveto{\pgfqpoint{2.087185in}{2.994605in}}{\pgfqpoint{2.097784in}{2.990215in}}{\pgfqpoint{2.108834in}{2.990215in}}%
\pgfpathclose%
\pgfusepath{stroke,fill}%
\end{pgfscope}%
\begin{pgfscope}%
\pgfpathrectangle{\pgfqpoint{0.600000in}{0.600000in}}{\pgfqpoint{3.900000in}{3.900000in}}%
\pgfusepath{clip}%
\pgfsetbuttcap%
\pgfsetroundjoin%
\definecolor{currentfill}{rgb}{0.121569,0.466667,0.705882}%
\pgfsetfillcolor{currentfill}%
\pgfsetlinewidth{1.003750pt}%
\definecolor{currentstroke}{rgb}{0.121569,0.466667,0.705882}%
\pgfsetstrokecolor{currentstroke}%
\pgfsetdash{}{0pt}%
\pgfpathmoveto{\pgfqpoint{2.712015in}{2.752670in}}%
\pgfpathcurveto{\pgfqpoint{2.723065in}{2.752670in}}{\pgfqpoint{2.733664in}{2.757061in}}{\pgfqpoint{2.741478in}{2.764874in}}%
\pgfpathcurveto{\pgfqpoint{2.749292in}{2.772688in}}{\pgfqpoint{2.753682in}{2.783287in}}{\pgfqpoint{2.753682in}{2.794337in}}%
\pgfpathcurveto{\pgfqpoint{2.753682in}{2.805387in}}{\pgfqpoint{2.749292in}{2.815986in}}{\pgfqpoint{2.741478in}{2.823800in}}%
\pgfpathcurveto{\pgfqpoint{2.733664in}{2.831614in}}{\pgfqpoint{2.723065in}{2.836004in}}{\pgfqpoint{2.712015in}{2.836004in}}%
\pgfpathcurveto{\pgfqpoint{2.700965in}{2.836004in}}{\pgfqpoint{2.690366in}{2.831614in}}{\pgfqpoint{2.682553in}{2.823800in}}%
\pgfpathcurveto{\pgfqpoint{2.674739in}{2.815986in}}{\pgfqpoint{2.670349in}{2.805387in}}{\pgfqpoint{2.670349in}{2.794337in}}%
\pgfpathcurveto{\pgfqpoint{2.670349in}{2.783287in}}{\pgfqpoint{2.674739in}{2.772688in}}{\pgfqpoint{2.682553in}{2.764874in}}%
\pgfpathcurveto{\pgfqpoint{2.690366in}{2.757061in}}{\pgfqpoint{2.700965in}{2.752670in}}{\pgfqpoint{2.712015in}{2.752670in}}%
\pgfpathclose%
\pgfusepath{stroke,fill}%
\end{pgfscope}%
\begin{pgfscope}%
\pgfpathrectangle{\pgfqpoint{0.600000in}{0.600000in}}{\pgfqpoint{3.900000in}{3.900000in}}%
\pgfusepath{clip}%
\pgfsetbuttcap%
\pgfsetroundjoin%
\definecolor{currentfill}{rgb}{0.121569,0.466667,0.705882}%
\pgfsetfillcolor{currentfill}%
\pgfsetlinewidth{1.003750pt}%
\definecolor{currentstroke}{rgb}{0.121569,0.466667,0.705882}%
\pgfsetstrokecolor{currentstroke}%
\pgfsetdash{}{0pt}%
\pgfpathmoveto{\pgfqpoint{2.210795in}{2.799451in}}%
\pgfpathcurveto{\pgfqpoint{2.221846in}{2.799451in}}{\pgfqpoint{2.232445in}{2.803841in}}{\pgfqpoint{2.240258in}{2.811655in}}%
\pgfpathcurveto{\pgfqpoint{2.248072in}{2.819468in}}{\pgfqpoint{2.252462in}{2.830067in}}{\pgfqpoint{2.252462in}{2.841118in}}%
\pgfpathcurveto{\pgfqpoint{2.252462in}{2.852168in}}{\pgfqpoint{2.248072in}{2.862767in}}{\pgfqpoint{2.240258in}{2.870580in}}%
\pgfpathcurveto{\pgfqpoint{2.232445in}{2.878394in}}{\pgfqpoint{2.221846in}{2.882784in}}{\pgfqpoint{2.210795in}{2.882784in}}%
\pgfpathcurveto{\pgfqpoint{2.199745in}{2.882784in}}{\pgfqpoint{2.189146in}{2.878394in}}{\pgfqpoint{2.181333in}{2.870580in}}%
\pgfpathcurveto{\pgfqpoint{2.173519in}{2.862767in}}{\pgfqpoint{2.169129in}{2.852168in}}{\pgfqpoint{2.169129in}{2.841118in}}%
\pgfpathcurveto{\pgfqpoint{2.169129in}{2.830067in}}{\pgfqpoint{2.173519in}{2.819468in}}{\pgfqpoint{2.181333in}{2.811655in}}%
\pgfpathcurveto{\pgfqpoint{2.189146in}{2.803841in}}{\pgfqpoint{2.199745in}{2.799451in}}{\pgfqpoint{2.210795in}{2.799451in}}%
\pgfpathclose%
\pgfusepath{stroke,fill}%
\end{pgfscope}%
\begin{pgfscope}%
\pgfpathrectangle{\pgfqpoint{0.600000in}{0.600000in}}{\pgfqpoint{3.900000in}{3.900000in}}%
\pgfusepath{clip}%
\pgfsetbuttcap%
\pgfsetroundjoin%
\definecolor{currentfill}{rgb}{0.121569,0.466667,0.705882}%
\pgfsetfillcolor{currentfill}%
\pgfsetlinewidth{1.003750pt}%
\definecolor{currentstroke}{rgb}{0.121569,0.466667,0.705882}%
\pgfsetstrokecolor{currentstroke}%
\pgfsetdash{}{0pt}%
\pgfpathmoveto{\pgfqpoint{2.527780in}{2.679699in}}%
\pgfpathcurveto{\pgfqpoint{2.538830in}{2.679699in}}{\pgfqpoint{2.549429in}{2.684089in}}{\pgfqpoint{2.557242in}{2.691903in}}%
\pgfpathcurveto{\pgfqpoint{2.565056in}{2.699716in}}{\pgfqpoint{2.569446in}{2.710315in}}{\pgfqpoint{2.569446in}{2.721365in}}%
\pgfpathcurveto{\pgfqpoint{2.569446in}{2.732416in}}{\pgfqpoint{2.565056in}{2.743015in}}{\pgfqpoint{2.557242in}{2.750828in}}%
\pgfpathcurveto{\pgfqpoint{2.549429in}{2.758642in}}{\pgfqpoint{2.538830in}{2.763032in}}{\pgfqpoint{2.527780in}{2.763032in}}%
\pgfpathcurveto{\pgfqpoint{2.516730in}{2.763032in}}{\pgfqpoint{2.506131in}{2.758642in}}{\pgfqpoint{2.498317in}{2.750828in}}%
\pgfpathcurveto{\pgfqpoint{2.490503in}{2.743015in}}{\pgfqpoint{2.486113in}{2.732416in}}{\pgfqpoint{2.486113in}{2.721365in}}%
\pgfpathcurveto{\pgfqpoint{2.486113in}{2.710315in}}{\pgfqpoint{2.490503in}{2.699716in}}{\pgfqpoint{2.498317in}{2.691903in}}%
\pgfpathcurveto{\pgfqpoint{2.506131in}{2.684089in}}{\pgfqpoint{2.516730in}{2.679699in}}{\pgfqpoint{2.527780in}{2.679699in}}%
\pgfpathclose%
\pgfusepath{stroke,fill}%
\end{pgfscope}%
\begin{pgfscope}%
\pgfpathrectangle{\pgfqpoint{0.600000in}{0.600000in}}{\pgfqpoint{3.900000in}{3.900000in}}%
\pgfusepath{clip}%
\pgfsetbuttcap%
\pgfsetroundjoin%
\definecolor{currentfill}{rgb}{0.121569,0.466667,0.705882}%
\pgfsetfillcolor{currentfill}%
\pgfsetlinewidth{1.003750pt}%
\definecolor{currentstroke}{rgb}{0.121569,0.466667,0.705882}%
\pgfsetstrokecolor{currentstroke}%
\pgfsetdash{}{0pt}%
\pgfpathmoveto{\pgfqpoint{2.597884in}{2.659018in}}%
\pgfpathcurveto{\pgfqpoint{2.608934in}{2.659018in}}{\pgfqpoint{2.619533in}{2.663408in}}{\pgfqpoint{2.627347in}{2.671222in}}%
\pgfpathcurveto{\pgfqpoint{2.635161in}{2.679035in}}{\pgfqpoint{2.639551in}{2.689634in}}{\pgfqpoint{2.639551in}{2.700685in}}%
\pgfpathcurveto{\pgfqpoint{2.639551in}{2.711735in}}{\pgfqpoint{2.635161in}{2.722334in}}{\pgfqpoint{2.627347in}{2.730147in}}%
\pgfpathcurveto{\pgfqpoint{2.619533in}{2.737961in}}{\pgfqpoint{2.608934in}{2.742351in}}{\pgfqpoint{2.597884in}{2.742351in}}%
\pgfpathcurveto{\pgfqpoint{2.586834in}{2.742351in}}{\pgfqpoint{2.576235in}{2.737961in}}{\pgfqpoint{2.568421in}{2.730147in}}%
\pgfpathcurveto{\pgfqpoint{2.560608in}{2.722334in}}{\pgfqpoint{2.556218in}{2.711735in}}{\pgfqpoint{2.556218in}{2.700685in}}%
\pgfpathcurveto{\pgfqpoint{2.556218in}{2.689634in}}{\pgfqpoint{2.560608in}{2.679035in}}{\pgfqpoint{2.568421in}{2.671222in}}%
\pgfpathcurveto{\pgfqpoint{2.576235in}{2.663408in}}{\pgfqpoint{2.586834in}{2.659018in}}{\pgfqpoint{2.597884in}{2.659018in}}%
\pgfpathclose%
\pgfusepath{stroke,fill}%
\end{pgfscope}%
\begin{pgfscope}%
\pgfpathrectangle{\pgfqpoint{0.600000in}{0.600000in}}{\pgfqpoint{3.900000in}{3.900000in}}%
\pgfusepath{clip}%
\pgfsetbuttcap%
\pgfsetroundjoin%
\definecolor{currentfill}{rgb}{0.121569,0.466667,0.705882}%
\pgfsetfillcolor{currentfill}%
\pgfsetlinewidth{1.003750pt}%
\definecolor{currentstroke}{rgb}{0.121569,0.466667,0.705882}%
\pgfsetstrokecolor{currentstroke}%
\pgfsetdash{}{0pt}%
\pgfpathmoveto{\pgfqpoint{2.142332in}{2.933299in}}%
\pgfpathcurveto{\pgfqpoint{2.153382in}{2.933299in}}{\pgfqpoint{2.163981in}{2.937689in}}{\pgfqpoint{2.171795in}{2.945503in}}%
\pgfpathcurveto{\pgfqpoint{2.179608in}{2.953316in}}{\pgfqpoint{2.183999in}{2.963916in}}{\pgfqpoint{2.183999in}{2.974966in}}%
\pgfpathcurveto{\pgfqpoint{2.183999in}{2.986016in}}{\pgfqpoint{2.179608in}{2.996615in}}{\pgfqpoint{2.171795in}{3.004428in}}%
\pgfpathcurveto{\pgfqpoint{2.163981in}{3.012242in}}{\pgfqpoint{2.153382in}{3.016632in}}{\pgfqpoint{2.142332in}{3.016632in}}%
\pgfpathcurveto{\pgfqpoint{2.131282in}{3.016632in}}{\pgfqpoint{2.120683in}{3.012242in}}{\pgfqpoint{2.112869in}{3.004428in}}%
\pgfpathcurveto{\pgfqpoint{2.105056in}{2.996615in}}{\pgfqpoint{2.100665in}{2.986016in}}{\pgfqpoint{2.100665in}{2.974966in}}%
\pgfpathcurveto{\pgfqpoint{2.100665in}{2.963916in}}{\pgfqpoint{2.105056in}{2.953316in}}{\pgfqpoint{2.112869in}{2.945503in}}%
\pgfpathcurveto{\pgfqpoint{2.120683in}{2.937689in}}{\pgfqpoint{2.131282in}{2.933299in}}{\pgfqpoint{2.142332in}{2.933299in}}%
\pgfpathclose%
\pgfusepath{stroke,fill}%
\end{pgfscope}%
\begin{pgfscope}%
\pgfpathrectangle{\pgfqpoint{0.600000in}{0.600000in}}{\pgfqpoint{3.900000in}{3.900000in}}%
\pgfusepath{clip}%
\pgfsetbuttcap%
\pgfsetroundjoin%
\definecolor{currentfill}{rgb}{0.121569,0.466667,0.705882}%
\pgfsetfillcolor{currentfill}%
\pgfsetlinewidth{1.003750pt}%
\definecolor{currentstroke}{rgb}{0.121569,0.466667,0.705882}%
\pgfsetstrokecolor{currentstroke}%
\pgfsetdash{}{0pt}%
\pgfpathmoveto{\pgfqpoint{2.640282in}{2.521288in}}%
\pgfpathcurveto{\pgfqpoint{2.651332in}{2.521288in}}{\pgfqpoint{2.661931in}{2.525678in}}{\pgfqpoint{2.669745in}{2.533492in}}%
\pgfpathcurveto{\pgfqpoint{2.677558in}{2.541305in}}{\pgfqpoint{2.681949in}{2.551904in}}{\pgfqpoint{2.681949in}{2.562954in}}%
\pgfpathcurveto{\pgfqpoint{2.681949in}{2.574004in}}{\pgfqpoint{2.677558in}{2.584603in}}{\pgfqpoint{2.669745in}{2.592417in}}%
\pgfpathcurveto{\pgfqpoint{2.661931in}{2.600231in}}{\pgfqpoint{2.651332in}{2.604621in}}{\pgfqpoint{2.640282in}{2.604621in}}%
\pgfpathcurveto{\pgfqpoint{2.629232in}{2.604621in}}{\pgfqpoint{2.618633in}{2.600231in}}{\pgfqpoint{2.610819in}{2.592417in}}%
\pgfpathcurveto{\pgfqpoint{2.603006in}{2.584603in}}{\pgfqpoint{2.598615in}{2.574004in}}{\pgfqpoint{2.598615in}{2.562954in}}%
\pgfpathcurveto{\pgfqpoint{2.598615in}{2.551904in}}{\pgfqpoint{2.603006in}{2.541305in}}{\pgfqpoint{2.610819in}{2.533492in}}%
\pgfpathcurveto{\pgfqpoint{2.618633in}{2.525678in}}{\pgfqpoint{2.629232in}{2.521288in}}{\pgfqpoint{2.640282in}{2.521288in}}%
\pgfpathclose%
\pgfusepath{stroke,fill}%
\end{pgfscope}%
\begin{pgfscope}%
\pgfpathrectangle{\pgfqpoint{0.600000in}{0.600000in}}{\pgfqpoint{3.900000in}{3.900000in}}%
\pgfusepath{clip}%
\pgfsetbuttcap%
\pgfsetroundjoin%
\definecolor{currentfill}{rgb}{0.121569,0.466667,0.705882}%
\pgfsetfillcolor{currentfill}%
\pgfsetlinewidth{1.003750pt}%
\definecolor{currentstroke}{rgb}{0.121569,0.466667,0.705882}%
\pgfsetstrokecolor{currentstroke}%
\pgfsetdash{}{0pt}%
\pgfpathmoveto{\pgfqpoint{2.078857in}{2.665589in}}%
\pgfpathcurveto{\pgfqpoint{2.089907in}{2.665589in}}{\pgfqpoint{2.100506in}{2.669979in}}{\pgfqpoint{2.108319in}{2.677793in}}%
\pgfpathcurveto{\pgfqpoint{2.116133in}{2.685607in}}{\pgfqpoint{2.120523in}{2.696206in}}{\pgfqpoint{2.120523in}{2.707256in}}%
\pgfpathcurveto{\pgfqpoint{2.120523in}{2.718306in}}{\pgfqpoint{2.116133in}{2.728905in}}{\pgfqpoint{2.108319in}{2.736719in}}%
\pgfpathcurveto{\pgfqpoint{2.100506in}{2.744532in}}{\pgfqpoint{2.089907in}{2.748922in}}{\pgfqpoint{2.078857in}{2.748922in}}%
\pgfpathcurveto{\pgfqpoint{2.067806in}{2.748922in}}{\pgfqpoint{2.057207in}{2.744532in}}{\pgfqpoint{2.049394in}{2.736719in}}%
\pgfpathcurveto{\pgfqpoint{2.041580in}{2.728905in}}{\pgfqpoint{2.037190in}{2.718306in}}{\pgfqpoint{2.037190in}{2.707256in}}%
\pgfpathcurveto{\pgfqpoint{2.037190in}{2.696206in}}{\pgfqpoint{2.041580in}{2.685607in}}{\pgfqpoint{2.049394in}{2.677793in}}%
\pgfpathcurveto{\pgfqpoint{2.057207in}{2.669979in}}{\pgfqpoint{2.067806in}{2.665589in}}{\pgfqpoint{2.078857in}{2.665589in}}%
\pgfpathclose%
\pgfusepath{stroke,fill}%
\end{pgfscope}%
\begin{pgfscope}%
\pgfpathrectangle{\pgfqpoint{0.600000in}{0.600000in}}{\pgfqpoint{3.900000in}{3.900000in}}%
\pgfusepath{clip}%
\pgfsetbuttcap%
\pgfsetroundjoin%
\definecolor{currentfill}{rgb}{0.121569,0.466667,0.705882}%
\pgfsetfillcolor{currentfill}%
\pgfsetlinewidth{1.003750pt}%
\definecolor{currentstroke}{rgb}{0.121569,0.466667,0.705882}%
\pgfsetstrokecolor{currentstroke}%
\pgfsetdash{}{0pt}%
\pgfpathmoveto{\pgfqpoint{2.815857in}{1.642146in}}%
\pgfpathcurveto{\pgfqpoint{2.826907in}{1.642146in}}{\pgfqpoint{2.837506in}{1.646536in}}{\pgfqpoint{2.845320in}{1.654349in}}%
\pgfpathcurveto{\pgfqpoint{2.853133in}{1.662163in}}{\pgfqpoint{2.857524in}{1.672762in}}{\pgfqpoint{2.857524in}{1.683812in}}%
\pgfpathcurveto{\pgfqpoint{2.857524in}{1.694862in}}{\pgfqpoint{2.853133in}{1.705461in}}{\pgfqpoint{2.845320in}{1.713275in}}%
\pgfpathcurveto{\pgfqpoint{2.837506in}{1.721089in}}{\pgfqpoint{2.826907in}{1.725479in}}{\pgfqpoint{2.815857in}{1.725479in}}%
\pgfpathcurveto{\pgfqpoint{2.804807in}{1.725479in}}{\pgfqpoint{2.794208in}{1.721089in}}{\pgfqpoint{2.786394in}{1.713275in}}%
\pgfpathcurveto{\pgfqpoint{2.778581in}{1.705461in}}{\pgfqpoint{2.774190in}{1.694862in}}{\pgfqpoint{2.774190in}{1.683812in}}%
\pgfpathcurveto{\pgfqpoint{2.774190in}{1.672762in}}{\pgfqpoint{2.778581in}{1.662163in}}{\pgfqpoint{2.786394in}{1.654349in}}%
\pgfpathcurveto{\pgfqpoint{2.794208in}{1.646536in}}{\pgfqpoint{2.804807in}{1.642146in}}{\pgfqpoint{2.815857in}{1.642146in}}%
\pgfpathclose%
\pgfusepath{stroke,fill}%
\end{pgfscope}%
\begin{pgfscope}%
\pgfpathrectangle{\pgfqpoint{0.600000in}{0.600000in}}{\pgfqpoint{3.900000in}{3.900000in}}%
\pgfusepath{clip}%
\pgfsetbuttcap%
\pgfsetroundjoin%
\definecolor{currentfill}{rgb}{0.121569,0.466667,0.705882}%
\pgfsetfillcolor{currentfill}%
\pgfsetlinewidth{1.003750pt}%
\definecolor{currentstroke}{rgb}{0.121569,0.466667,0.705882}%
\pgfsetstrokecolor{currentstroke}%
\pgfsetdash{}{0pt}%
\pgfpathmoveto{\pgfqpoint{2.804284in}{3.065622in}}%
\pgfpathcurveto{\pgfqpoint{2.815334in}{3.065622in}}{\pgfqpoint{2.825933in}{3.070012in}}{\pgfqpoint{2.833747in}{3.077826in}}%
\pgfpathcurveto{\pgfqpoint{2.841561in}{3.085639in}}{\pgfqpoint{2.845951in}{3.096238in}}{\pgfqpoint{2.845951in}{3.107288in}}%
\pgfpathcurveto{\pgfqpoint{2.845951in}{3.118339in}}{\pgfqpoint{2.841561in}{3.128938in}}{\pgfqpoint{2.833747in}{3.136751in}}%
\pgfpathcurveto{\pgfqpoint{2.825933in}{3.144565in}}{\pgfqpoint{2.815334in}{3.148955in}}{\pgfqpoint{2.804284in}{3.148955in}}%
\pgfpathcurveto{\pgfqpoint{2.793234in}{3.148955in}}{\pgfqpoint{2.782635in}{3.144565in}}{\pgfqpoint{2.774822in}{3.136751in}}%
\pgfpathcurveto{\pgfqpoint{2.767008in}{3.128938in}}{\pgfqpoint{2.762618in}{3.118339in}}{\pgfqpoint{2.762618in}{3.107288in}}%
\pgfpathcurveto{\pgfqpoint{2.762618in}{3.096238in}}{\pgfqpoint{2.767008in}{3.085639in}}{\pgfqpoint{2.774822in}{3.077826in}}%
\pgfpathcurveto{\pgfqpoint{2.782635in}{3.070012in}}{\pgfqpoint{2.793234in}{3.065622in}}{\pgfqpoint{2.804284in}{3.065622in}}%
\pgfpathclose%
\pgfusepath{stroke,fill}%
\end{pgfscope}%
\begin{pgfscope}%
\pgfpathrectangle{\pgfqpoint{0.600000in}{0.600000in}}{\pgfqpoint{3.900000in}{3.900000in}}%
\pgfusepath{clip}%
\pgfsetbuttcap%
\pgfsetroundjoin%
\definecolor{currentfill}{rgb}{0.121569,0.466667,0.705882}%
\pgfsetfillcolor{currentfill}%
\pgfsetlinewidth{1.003750pt}%
\definecolor{currentstroke}{rgb}{0.121569,0.466667,0.705882}%
\pgfsetstrokecolor{currentstroke}%
\pgfsetdash{}{0pt}%
\pgfpathmoveto{\pgfqpoint{2.850718in}{2.858154in}}%
\pgfpathcurveto{\pgfqpoint{2.861768in}{2.858154in}}{\pgfqpoint{2.872367in}{2.862544in}}{\pgfqpoint{2.880181in}{2.870358in}}%
\pgfpathcurveto{\pgfqpoint{2.887994in}{2.878171in}}{\pgfqpoint{2.892384in}{2.888770in}}{\pgfqpoint{2.892384in}{2.899820in}}%
\pgfpathcurveto{\pgfqpoint{2.892384in}{2.910871in}}{\pgfqpoint{2.887994in}{2.921470in}}{\pgfqpoint{2.880181in}{2.929283in}}%
\pgfpathcurveto{\pgfqpoint{2.872367in}{2.937097in}}{\pgfqpoint{2.861768in}{2.941487in}}{\pgfqpoint{2.850718in}{2.941487in}}%
\pgfpathcurveto{\pgfqpoint{2.839668in}{2.941487in}}{\pgfqpoint{2.829069in}{2.937097in}}{\pgfqpoint{2.821255in}{2.929283in}}%
\pgfpathcurveto{\pgfqpoint{2.813441in}{2.921470in}}{\pgfqpoint{2.809051in}{2.910871in}}{\pgfqpoint{2.809051in}{2.899820in}}%
\pgfpathcurveto{\pgfqpoint{2.809051in}{2.888770in}}{\pgfqpoint{2.813441in}{2.878171in}}{\pgfqpoint{2.821255in}{2.870358in}}%
\pgfpathcurveto{\pgfqpoint{2.829069in}{2.862544in}}{\pgfqpoint{2.839668in}{2.858154in}}{\pgfqpoint{2.850718in}{2.858154in}}%
\pgfpathclose%
\pgfusepath{stroke,fill}%
\end{pgfscope}%
\begin{pgfscope}%
\pgfpathrectangle{\pgfqpoint{0.600000in}{0.600000in}}{\pgfqpoint{3.900000in}{3.900000in}}%
\pgfusepath{clip}%
\pgfsetbuttcap%
\pgfsetroundjoin%
\definecolor{currentfill}{rgb}{0.121569,0.466667,0.705882}%
\pgfsetfillcolor{currentfill}%
\pgfsetlinewidth{1.003750pt}%
\definecolor{currentstroke}{rgb}{0.121569,0.466667,0.705882}%
\pgfsetstrokecolor{currentstroke}%
\pgfsetdash{}{0pt}%
\pgfpathmoveto{\pgfqpoint{2.788044in}{2.893233in}}%
\pgfpathcurveto{\pgfqpoint{2.799094in}{2.893233in}}{\pgfqpoint{2.809693in}{2.897623in}}{\pgfqpoint{2.817507in}{2.905437in}}%
\pgfpathcurveto{\pgfqpoint{2.825320in}{2.913250in}}{\pgfqpoint{2.829711in}{2.923850in}}{\pgfqpoint{2.829711in}{2.934900in}}%
\pgfpathcurveto{\pgfqpoint{2.829711in}{2.945950in}}{\pgfqpoint{2.825320in}{2.956549in}}{\pgfqpoint{2.817507in}{2.964362in}}%
\pgfpathcurveto{\pgfqpoint{2.809693in}{2.972176in}}{\pgfqpoint{2.799094in}{2.976566in}}{\pgfqpoint{2.788044in}{2.976566in}}%
\pgfpathcurveto{\pgfqpoint{2.776994in}{2.976566in}}{\pgfqpoint{2.766395in}{2.972176in}}{\pgfqpoint{2.758581in}{2.964362in}}%
\pgfpathcurveto{\pgfqpoint{2.750768in}{2.956549in}}{\pgfqpoint{2.746377in}{2.945950in}}{\pgfqpoint{2.746377in}{2.934900in}}%
\pgfpathcurveto{\pgfqpoint{2.746377in}{2.923850in}}{\pgfqpoint{2.750768in}{2.913250in}}{\pgfqpoint{2.758581in}{2.905437in}}%
\pgfpathcurveto{\pgfqpoint{2.766395in}{2.897623in}}{\pgfqpoint{2.776994in}{2.893233in}}{\pgfqpoint{2.788044in}{2.893233in}}%
\pgfpathclose%
\pgfusepath{stroke,fill}%
\end{pgfscope}%
\begin{pgfscope}%
\pgfpathrectangle{\pgfqpoint{0.600000in}{0.600000in}}{\pgfqpoint{3.900000in}{3.900000in}}%
\pgfusepath{clip}%
\pgfsetbuttcap%
\pgfsetroundjoin%
\definecolor{currentfill}{rgb}{0.121569,0.466667,0.705882}%
\pgfsetfillcolor{currentfill}%
\pgfsetlinewidth{1.003750pt}%
\definecolor{currentstroke}{rgb}{0.121569,0.466667,0.705882}%
\pgfsetstrokecolor{currentstroke}%
\pgfsetdash{}{0pt}%
\pgfpathmoveto{\pgfqpoint{1.902132in}{3.121544in}}%
\pgfpathcurveto{\pgfqpoint{1.913182in}{3.121544in}}{\pgfqpoint{1.923781in}{3.125935in}}{\pgfqpoint{1.931594in}{3.133748in}}%
\pgfpathcurveto{\pgfqpoint{1.939408in}{3.141562in}}{\pgfqpoint{1.943798in}{3.152161in}}{\pgfqpoint{1.943798in}{3.163211in}}%
\pgfpathcurveto{\pgfqpoint{1.943798in}{3.174261in}}{\pgfqpoint{1.939408in}{3.184860in}}{\pgfqpoint{1.931594in}{3.192674in}}%
\pgfpathcurveto{\pgfqpoint{1.923781in}{3.200487in}}{\pgfqpoint{1.913182in}{3.204878in}}{\pgfqpoint{1.902132in}{3.204878in}}%
\pgfpathcurveto{\pgfqpoint{1.891081in}{3.204878in}}{\pgfqpoint{1.880482in}{3.200487in}}{\pgfqpoint{1.872669in}{3.192674in}}%
\pgfpathcurveto{\pgfqpoint{1.864855in}{3.184860in}}{\pgfqpoint{1.860465in}{3.174261in}}{\pgfqpoint{1.860465in}{3.163211in}}%
\pgfpathcurveto{\pgfqpoint{1.860465in}{3.152161in}}{\pgfqpoint{1.864855in}{3.141562in}}{\pgfqpoint{1.872669in}{3.133748in}}%
\pgfpathcurveto{\pgfqpoint{1.880482in}{3.125935in}}{\pgfqpoint{1.891081in}{3.121544in}}{\pgfqpoint{1.902132in}{3.121544in}}%
\pgfpathclose%
\pgfusepath{stroke,fill}%
\end{pgfscope}%
\begin{pgfscope}%
\pgfpathrectangle{\pgfqpoint{0.600000in}{0.600000in}}{\pgfqpoint{3.900000in}{3.900000in}}%
\pgfusepath{clip}%
\pgfsetbuttcap%
\pgfsetroundjoin%
\definecolor{currentfill}{rgb}{0.121569,0.466667,0.705882}%
\pgfsetfillcolor{currentfill}%
\pgfsetlinewidth{1.003750pt}%
\definecolor{currentstroke}{rgb}{0.121569,0.466667,0.705882}%
\pgfsetstrokecolor{currentstroke}%
\pgfsetdash{}{0pt}%
\pgfpathmoveto{\pgfqpoint{2.529519in}{2.745051in}}%
\pgfpathcurveto{\pgfqpoint{2.540570in}{2.745051in}}{\pgfqpoint{2.551169in}{2.749442in}}{\pgfqpoint{2.558982in}{2.757255in}}%
\pgfpathcurveto{\pgfqpoint{2.566796in}{2.765069in}}{\pgfqpoint{2.571186in}{2.775668in}}{\pgfqpoint{2.571186in}{2.786718in}}%
\pgfpathcurveto{\pgfqpoint{2.571186in}{2.797768in}}{\pgfqpoint{2.566796in}{2.808367in}}{\pgfqpoint{2.558982in}{2.816181in}}%
\pgfpathcurveto{\pgfqpoint{2.551169in}{2.823995in}}{\pgfqpoint{2.540570in}{2.828385in}}{\pgfqpoint{2.529519in}{2.828385in}}%
\pgfpathcurveto{\pgfqpoint{2.518469in}{2.828385in}}{\pgfqpoint{2.507870in}{2.823995in}}{\pgfqpoint{2.500057in}{2.816181in}}%
\pgfpathcurveto{\pgfqpoint{2.492243in}{2.808367in}}{\pgfqpoint{2.487853in}{2.797768in}}{\pgfqpoint{2.487853in}{2.786718in}}%
\pgfpathcurveto{\pgfqpoint{2.487853in}{2.775668in}}{\pgfqpoint{2.492243in}{2.765069in}}{\pgfqpoint{2.500057in}{2.757255in}}%
\pgfpathcurveto{\pgfqpoint{2.507870in}{2.749442in}}{\pgfqpoint{2.518469in}{2.745051in}}{\pgfqpoint{2.529519in}{2.745051in}}%
\pgfpathclose%
\pgfusepath{stroke,fill}%
\end{pgfscope}%
\begin{pgfscope}%
\pgfpathrectangle{\pgfqpoint{0.600000in}{0.600000in}}{\pgfqpoint{3.900000in}{3.900000in}}%
\pgfusepath{clip}%
\pgfsetbuttcap%
\pgfsetroundjoin%
\definecolor{currentfill}{rgb}{0.121569,0.466667,0.705882}%
\pgfsetfillcolor{currentfill}%
\pgfsetlinewidth{1.003750pt}%
\definecolor{currentstroke}{rgb}{0.121569,0.466667,0.705882}%
\pgfsetstrokecolor{currentstroke}%
\pgfsetdash{}{0pt}%
\pgfpathmoveto{\pgfqpoint{2.469980in}{1.982760in}}%
\pgfpathcurveto{\pgfqpoint{2.481030in}{1.982760in}}{\pgfqpoint{2.491629in}{1.987151in}}{\pgfqpoint{2.499443in}{1.994964in}}%
\pgfpathcurveto{\pgfqpoint{2.507257in}{2.002778in}}{\pgfqpoint{2.511647in}{2.013377in}}{\pgfqpoint{2.511647in}{2.024427in}}%
\pgfpathcurveto{\pgfqpoint{2.511647in}{2.035477in}}{\pgfqpoint{2.507257in}{2.046076in}}{\pgfqpoint{2.499443in}{2.053890in}}%
\pgfpathcurveto{\pgfqpoint{2.491629in}{2.061703in}}{\pgfqpoint{2.481030in}{2.066094in}}{\pgfqpoint{2.469980in}{2.066094in}}%
\pgfpathcurveto{\pgfqpoint{2.458930in}{2.066094in}}{\pgfqpoint{2.448331in}{2.061703in}}{\pgfqpoint{2.440517in}{2.053890in}}%
\pgfpathcurveto{\pgfqpoint{2.432704in}{2.046076in}}{\pgfqpoint{2.428314in}{2.035477in}}{\pgfqpoint{2.428314in}{2.024427in}}%
\pgfpathcurveto{\pgfqpoint{2.428314in}{2.013377in}}{\pgfqpoint{2.432704in}{2.002778in}}{\pgfqpoint{2.440517in}{1.994964in}}%
\pgfpathcurveto{\pgfqpoint{2.448331in}{1.987151in}}{\pgfqpoint{2.458930in}{1.982760in}}{\pgfqpoint{2.469980in}{1.982760in}}%
\pgfpathclose%
\pgfusepath{stroke,fill}%
\end{pgfscope}%
\begin{pgfscope}%
\pgfpathrectangle{\pgfqpoint{0.600000in}{0.600000in}}{\pgfqpoint{3.900000in}{3.900000in}}%
\pgfusepath{clip}%
\pgfsetbuttcap%
\pgfsetroundjoin%
\definecolor{currentfill}{rgb}{0.121569,0.466667,0.705882}%
\pgfsetfillcolor{currentfill}%
\pgfsetlinewidth{1.003750pt}%
\definecolor{currentstroke}{rgb}{0.121569,0.466667,0.705882}%
\pgfsetstrokecolor{currentstroke}%
\pgfsetdash{}{0pt}%
\pgfpathmoveto{\pgfqpoint{2.967720in}{2.037552in}}%
\pgfpathcurveto{\pgfqpoint{2.978771in}{2.037552in}}{\pgfqpoint{2.989370in}{2.041942in}}{\pgfqpoint{2.997183in}{2.049756in}}%
\pgfpathcurveto{\pgfqpoint{3.004997in}{2.057569in}}{\pgfqpoint{3.009387in}{2.068169in}}{\pgfqpoint{3.009387in}{2.079219in}}%
\pgfpathcurveto{\pgfqpoint{3.009387in}{2.090269in}}{\pgfqpoint{3.004997in}{2.100868in}}{\pgfqpoint{2.997183in}{2.108681in}}%
\pgfpathcurveto{\pgfqpoint{2.989370in}{2.116495in}}{\pgfqpoint{2.978771in}{2.120885in}}{\pgfqpoint{2.967720in}{2.120885in}}%
\pgfpathcurveto{\pgfqpoint{2.956670in}{2.120885in}}{\pgfqpoint{2.946071in}{2.116495in}}{\pgfqpoint{2.938258in}{2.108681in}}%
\pgfpathcurveto{\pgfqpoint{2.930444in}{2.100868in}}{\pgfqpoint{2.926054in}{2.090269in}}{\pgfqpoint{2.926054in}{2.079219in}}%
\pgfpathcurveto{\pgfqpoint{2.926054in}{2.068169in}}{\pgfqpoint{2.930444in}{2.057569in}}{\pgfqpoint{2.938258in}{2.049756in}}%
\pgfpathcurveto{\pgfqpoint{2.946071in}{2.041942in}}{\pgfqpoint{2.956670in}{2.037552in}}{\pgfqpoint{2.967720in}{2.037552in}}%
\pgfpathclose%
\pgfusepath{stroke,fill}%
\end{pgfscope}%
\begin{pgfscope}%
\pgfpathrectangle{\pgfqpoint{0.600000in}{0.600000in}}{\pgfqpoint{3.900000in}{3.900000in}}%
\pgfusepath{clip}%
\pgfsetbuttcap%
\pgfsetroundjoin%
\definecolor{currentfill}{rgb}{0.121569,0.466667,0.705882}%
\pgfsetfillcolor{currentfill}%
\pgfsetlinewidth{1.003750pt}%
\definecolor{currentstroke}{rgb}{0.121569,0.466667,0.705882}%
\pgfsetstrokecolor{currentstroke}%
\pgfsetdash{}{0pt}%
\pgfpathmoveto{\pgfqpoint{3.097865in}{3.341986in}}%
\pgfpathcurveto{\pgfqpoint{3.108915in}{3.341986in}}{\pgfqpoint{3.119514in}{3.346376in}}{\pgfqpoint{3.127328in}{3.354190in}}%
\pgfpathcurveto{\pgfqpoint{3.135142in}{3.362003in}}{\pgfqpoint{3.139532in}{3.372602in}}{\pgfqpoint{3.139532in}{3.383652in}}%
\pgfpathcurveto{\pgfqpoint{3.139532in}{3.394702in}}{\pgfqpoint{3.135142in}{3.405301in}}{\pgfqpoint{3.127328in}{3.413115in}}%
\pgfpathcurveto{\pgfqpoint{3.119514in}{3.420929in}}{\pgfqpoint{3.108915in}{3.425319in}}{\pgfqpoint{3.097865in}{3.425319in}}%
\pgfpathcurveto{\pgfqpoint{3.086815in}{3.425319in}}{\pgfqpoint{3.076216in}{3.420929in}}{\pgfqpoint{3.068402in}{3.413115in}}%
\pgfpathcurveto{\pgfqpoint{3.060589in}{3.405301in}}{\pgfqpoint{3.056199in}{3.394702in}}{\pgfqpoint{3.056199in}{3.383652in}}%
\pgfpathcurveto{\pgfqpoint{3.056199in}{3.372602in}}{\pgfqpoint{3.060589in}{3.362003in}}{\pgfqpoint{3.068402in}{3.354190in}}%
\pgfpathcurveto{\pgfqpoint{3.076216in}{3.346376in}}{\pgfqpoint{3.086815in}{3.341986in}}{\pgfqpoint{3.097865in}{3.341986in}}%
\pgfpathclose%
\pgfusepath{stroke,fill}%
\end{pgfscope}%
\begin{pgfscope}%
\pgfpathrectangle{\pgfqpoint{0.600000in}{0.600000in}}{\pgfqpoint{3.900000in}{3.900000in}}%
\pgfusepath{clip}%
\pgfsetbuttcap%
\pgfsetroundjoin%
\definecolor{currentfill}{rgb}{0.121569,0.466667,0.705882}%
\pgfsetfillcolor{currentfill}%
\pgfsetlinewidth{1.003750pt}%
\definecolor{currentstroke}{rgb}{0.121569,0.466667,0.705882}%
\pgfsetstrokecolor{currentstroke}%
\pgfsetdash{}{0pt}%
\pgfpathmoveto{\pgfqpoint{2.187070in}{2.914441in}}%
\pgfpathcurveto{\pgfqpoint{2.198120in}{2.914441in}}{\pgfqpoint{2.208719in}{2.918831in}}{\pgfqpoint{2.216533in}{2.926645in}}%
\pgfpathcurveto{\pgfqpoint{2.224347in}{2.934458in}}{\pgfqpoint{2.228737in}{2.945057in}}{\pgfqpoint{2.228737in}{2.956107in}}%
\pgfpathcurveto{\pgfqpoint{2.228737in}{2.967158in}}{\pgfqpoint{2.224347in}{2.977757in}}{\pgfqpoint{2.216533in}{2.985570in}}%
\pgfpathcurveto{\pgfqpoint{2.208719in}{2.993384in}}{\pgfqpoint{2.198120in}{2.997774in}}{\pgfqpoint{2.187070in}{2.997774in}}%
\pgfpathcurveto{\pgfqpoint{2.176020in}{2.997774in}}{\pgfqpoint{2.165421in}{2.993384in}}{\pgfqpoint{2.157607in}{2.985570in}}%
\pgfpathcurveto{\pgfqpoint{2.149794in}{2.977757in}}{\pgfqpoint{2.145404in}{2.967158in}}{\pgfqpoint{2.145404in}{2.956107in}}%
\pgfpathcurveto{\pgfqpoint{2.145404in}{2.945057in}}{\pgfqpoint{2.149794in}{2.934458in}}{\pgfqpoint{2.157607in}{2.926645in}}%
\pgfpathcurveto{\pgfqpoint{2.165421in}{2.918831in}}{\pgfqpoint{2.176020in}{2.914441in}}{\pgfqpoint{2.187070in}{2.914441in}}%
\pgfpathclose%
\pgfusepath{stroke,fill}%
\end{pgfscope}%
\begin{pgfscope}%
\pgfpathrectangle{\pgfqpoint{0.600000in}{0.600000in}}{\pgfqpoint{3.900000in}{3.900000in}}%
\pgfusepath{clip}%
\pgfsetbuttcap%
\pgfsetroundjoin%
\definecolor{currentfill}{rgb}{0.121569,0.466667,0.705882}%
\pgfsetfillcolor{currentfill}%
\pgfsetlinewidth{1.003750pt}%
\definecolor{currentstroke}{rgb}{0.121569,0.466667,0.705882}%
\pgfsetstrokecolor{currentstroke}%
\pgfsetdash{}{0pt}%
\pgfpathmoveto{\pgfqpoint{2.155464in}{2.800434in}}%
\pgfpathcurveto{\pgfqpoint{2.166515in}{2.800434in}}{\pgfqpoint{2.177114in}{2.804825in}}{\pgfqpoint{2.184927in}{2.812638in}}%
\pgfpathcurveto{\pgfqpoint{2.192741in}{2.820452in}}{\pgfqpoint{2.197131in}{2.831051in}}{\pgfqpoint{2.197131in}{2.842101in}}%
\pgfpathcurveto{\pgfqpoint{2.197131in}{2.853151in}}{\pgfqpoint{2.192741in}{2.863750in}}{\pgfqpoint{2.184927in}{2.871564in}}%
\pgfpathcurveto{\pgfqpoint{2.177114in}{2.879377in}}{\pgfqpoint{2.166515in}{2.883768in}}{\pgfqpoint{2.155464in}{2.883768in}}%
\pgfpathcurveto{\pgfqpoint{2.144414in}{2.883768in}}{\pgfqpoint{2.133815in}{2.879377in}}{\pgfqpoint{2.126002in}{2.871564in}}%
\pgfpathcurveto{\pgfqpoint{2.118188in}{2.863750in}}{\pgfqpoint{2.113798in}{2.853151in}}{\pgfqpoint{2.113798in}{2.842101in}}%
\pgfpathcurveto{\pgfqpoint{2.113798in}{2.831051in}}{\pgfqpoint{2.118188in}{2.820452in}}{\pgfqpoint{2.126002in}{2.812638in}}%
\pgfpathcurveto{\pgfqpoint{2.133815in}{2.804825in}}{\pgfqpoint{2.144414in}{2.800434in}}{\pgfqpoint{2.155464in}{2.800434in}}%
\pgfpathclose%
\pgfusepath{stroke,fill}%
\end{pgfscope}%
\begin{pgfscope}%
\pgfpathrectangle{\pgfqpoint{0.600000in}{0.600000in}}{\pgfqpoint{3.900000in}{3.900000in}}%
\pgfusepath{clip}%
\pgfsetbuttcap%
\pgfsetroundjoin%
\definecolor{currentfill}{rgb}{0.121569,0.466667,0.705882}%
\pgfsetfillcolor{currentfill}%
\pgfsetlinewidth{1.003750pt}%
\definecolor{currentstroke}{rgb}{0.121569,0.466667,0.705882}%
\pgfsetstrokecolor{currentstroke}%
\pgfsetdash{}{0pt}%
\pgfpathmoveto{\pgfqpoint{2.673006in}{3.343241in}}%
\pgfpathcurveto{\pgfqpoint{2.684056in}{3.343241in}}{\pgfqpoint{2.694655in}{3.347631in}}{\pgfqpoint{2.702468in}{3.355444in}}%
\pgfpathcurveto{\pgfqpoint{2.710282in}{3.363258in}}{\pgfqpoint{2.714672in}{3.373857in}}{\pgfqpoint{2.714672in}{3.384907in}}%
\pgfpathcurveto{\pgfqpoint{2.714672in}{3.395957in}}{\pgfqpoint{2.710282in}{3.406556in}}{\pgfqpoint{2.702468in}{3.414370in}}%
\pgfpathcurveto{\pgfqpoint{2.694655in}{3.422184in}}{\pgfqpoint{2.684056in}{3.426574in}}{\pgfqpoint{2.673006in}{3.426574in}}%
\pgfpathcurveto{\pgfqpoint{2.661956in}{3.426574in}}{\pgfqpoint{2.651357in}{3.422184in}}{\pgfqpoint{2.643543in}{3.414370in}}%
\pgfpathcurveto{\pgfqpoint{2.635729in}{3.406556in}}{\pgfqpoint{2.631339in}{3.395957in}}{\pgfqpoint{2.631339in}{3.384907in}}%
\pgfpathcurveto{\pgfqpoint{2.631339in}{3.373857in}}{\pgfqpoint{2.635729in}{3.363258in}}{\pgfqpoint{2.643543in}{3.355444in}}%
\pgfpathcurveto{\pgfqpoint{2.651357in}{3.347631in}}{\pgfqpoint{2.661956in}{3.343241in}}{\pgfqpoint{2.673006in}{3.343241in}}%
\pgfpathclose%
\pgfusepath{stroke,fill}%
\end{pgfscope}%
\begin{pgfscope}%
\pgfpathrectangle{\pgfqpoint{0.600000in}{0.600000in}}{\pgfqpoint{3.900000in}{3.900000in}}%
\pgfusepath{clip}%
\pgfsetbuttcap%
\pgfsetroundjoin%
\definecolor{currentfill}{rgb}{0.121569,0.466667,0.705882}%
\pgfsetfillcolor{currentfill}%
\pgfsetlinewidth{1.003750pt}%
\definecolor{currentstroke}{rgb}{0.121569,0.466667,0.705882}%
\pgfsetstrokecolor{currentstroke}%
\pgfsetdash{}{0pt}%
\pgfpathmoveto{\pgfqpoint{2.330668in}{3.106521in}}%
\pgfpathcurveto{\pgfqpoint{2.341718in}{3.106521in}}{\pgfqpoint{2.352317in}{3.110911in}}{\pgfqpoint{2.360130in}{3.118725in}}%
\pgfpathcurveto{\pgfqpoint{2.367944in}{3.126538in}}{\pgfqpoint{2.372334in}{3.137137in}}{\pgfqpoint{2.372334in}{3.148187in}}%
\pgfpathcurveto{\pgfqpoint{2.372334in}{3.159238in}}{\pgfqpoint{2.367944in}{3.169837in}}{\pgfqpoint{2.360130in}{3.177650in}}%
\pgfpathcurveto{\pgfqpoint{2.352317in}{3.185464in}}{\pgfqpoint{2.341718in}{3.189854in}}{\pgfqpoint{2.330668in}{3.189854in}}%
\pgfpathcurveto{\pgfqpoint{2.319618in}{3.189854in}}{\pgfqpoint{2.309019in}{3.185464in}}{\pgfqpoint{2.301205in}{3.177650in}}%
\pgfpathcurveto{\pgfqpoint{2.293391in}{3.169837in}}{\pgfqpoint{2.289001in}{3.159238in}}{\pgfqpoint{2.289001in}{3.148187in}}%
\pgfpathcurveto{\pgfqpoint{2.289001in}{3.137137in}}{\pgfqpoint{2.293391in}{3.126538in}}{\pgfqpoint{2.301205in}{3.118725in}}%
\pgfpathcurveto{\pgfqpoint{2.309019in}{3.110911in}}{\pgfqpoint{2.319618in}{3.106521in}}{\pgfqpoint{2.330668in}{3.106521in}}%
\pgfpathclose%
\pgfusepath{stroke,fill}%
\end{pgfscope}%
\begin{pgfscope}%
\pgfpathrectangle{\pgfqpoint{0.600000in}{0.600000in}}{\pgfqpoint{3.900000in}{3.900000in}}%
\pgfusepath{clip}%
\pgfsetbuttcap%
\pgfsetroundjoin%
\definecolor{currentfill}{rgb}{0.121569,0.466667,0.705882}%
\pgfsetfillcolor{currentfill}%
\pgfsetlinewidth{1.003750pt}%
\definecolor{currentstroke}{rgb}{0.121569,0.466667,0.705882}%
\pgfsetstrokecolor{currentstroke}%
\pgfsetdash{}{0pt}%
\pgfpathmoveto{\pgfqpoint{1.851549in}{2.738464in}}%
\pgfpathcurveto{\pgfqpoint{1.862599in}{2.738464in}}{\pgfqpoint{1.873198in}{2.742855in}}{\pgfqpoint{1.881012in}{2.750668in}}%
\pgfpathcurveto{\pgfqpoint{1.888826in}{2.758482in}}{\pgfqpoint{1.893216in}{2.769081in}}{\pgfqpoint{1.893216in}{2.780131in}}%
\pgfpathcurveto{\pgfqpoint{1.893216in}{2.791181in}}{\pgfqpoint{1.888826in}{2.801780in}}{\pgfqpoint{1.881012in}{2.809594in}}%
\pgfpathcurveto{\pgfqpoint{1.873198in}{2.817407in}}{\pgfqpoint{1.862599in}{2.821798in}}{\pgfqpoint{1.851549in}{2.821798in}}%
\pgfpathcurveto{\pgfqpoint{1.840499in}{2.821798in}}{\pgfqpoint{1.829900in}{2.817407in}}{\pgfqpoint{1.822086in}{2.809594in}}%
\pgfpathcurveto{\pgfqpoint{1.814273in}{2.801780in}}{\pgfqpoint{1.809883in}{2.791181in}}{\pgfqpoint{1.809883in}{2.780131in}}%
\pgfpathcurveto{\pgfqpoint{1.809883in}{2.769081in}}{\pgfqpoint{1.814273in}{2.758482in}}{\pgfqpoint{1.822086in}{2.750668in}}%
\pgfpathcurveto{\pgfqpoint{1.829900in}{2.742855in}}{\pgfqpoint{1.840499in}{2.738464in}}{\pgfqpoint{1.851549in}{2.738464in}}%
\pgfpathclose%
\pgfusepath{stroke,fill}%
\end{pgfscope}%
\begin{pgfscope}%
\pgfpathrectangle{\pgfqpoint{0.600000in}{0.600000in}}{\pgfqpoint{3.900000in}{3.900000in}}%
\pgfusepath{clip}%
\pgfsetbuttcap%
\pgfsetroundjoin%
\definecolor{currentfill}{rgb}{0.121569,0.466667,0.705882}%
\pgfsetfillcolor{currentfill}%
\pgfsetlinewidth{1.003750pt}%
\definecolor{currentstroke}{rgb}{0.121569,0.466667,0.705882}%
\pgfsetstrokecolor{currentstroke}%
\pgfsetdash{}{0pt}%
\pgfpathmoveto{\pgfqpoint{2.617281in}{1.430180in}}%
\pgfpathcurveto{\pgfqpoint{2.628331in}{1.430180in}}{\pgfqpoint{2.638930in}{1.434570in}}{\pgfqpoint{2.646744in}{1.442384in}}%
\pgfpathcurveto{\pgfqpoint{2.654557in}{1.450197in}}{\pgfqpoint{2.658948in}{1.460797in}}{\pgfqpoint{2.658948in}{1.471847in}}%
\pgfpathcurveto{\pgfqpoint{2.658948in}{1.482897in}}{\pgfqpoint{2.654557in}{1.493496in}}{\pgfqpoint{2.646744in}{1.501309in}}%
\pgfpathcurveto{\pgfqpoint{2.638930in}{1.509123in}}{\pgfqpoint{2.628331in}{1.513513in}}{\pgfqpoint{2.617281in}{1.513513in}}%
\pgfpathcurveto{\pgfqpoint{2.606231in}{1.513513in}}{\pgfqpoint{2.595632in}{1.509123in}}{\pgfqpoint{2.587818in}{1.501309in}}%
\pgfpathcurveto{\pgfqpoint{2.580004in}{1.493496in}}{\pgfqpoint{2.575614in}{1.482897in}}{\pgfqpoint{2.575614in}{1.471847in}}%
\pgfpathcurveto{\pgfqpoint{2.575614in}{1.460797in}}{\pgfqpoint{2.580004in}{1.450197in}}{\pgfqpoint{2.587818in}{1.442384in}}%
\pgfpathcurveto{\pgfqpoint{2.595632in}{1.434570in}}{\pgfqpoint{2.606231in}{1.430180in}}{\pgfqpoint{2.617281in}{1.430180in}}%
\pgfpathclose%
\pgfusepath{stroke,fill}%
\end{pgfscope}%
\begin{pgfscope}%
\pgfpathrectangle{\pgfqpoint{0.600000in}{0.600000in}}{\pgfqpoint{3.900000in}{3.900000in}}%
\pgfusepath{clip}%
\pgfsetbuttcap%
\pgfsetroundjoin%
\definecolor{currentfill}{rgb}{0.121569,0.466667,0.705882}%
\pgfsetfillcolor{currentfill}%
\pgfsetlinewidth{1.003750pt}%
\definecolor{currentstroke}{rgb}{0.121569,0.466667,0.705882}%
\pgfsetstrokecolor{currentstroke}%
\pgfsetdash{}{0pt}%
\pgfpathmoveto{\pgfqpoint{3.052766in}{3.023322in}}%
\pgfpathcurveto{\pgfqpoint{3.063816in}{3.023322in}}{\pgfqpoint{3.074415in}{3.027712in}}{\pgfqpoint{3.082229in}{3.035526in}}%
\pgfpathcurveto{\pgfqpoint{3.090042in}{3.043340in}}{\pgfqpoint{3.094433in}{3.053939in}}{\pgfqpoint{3.094433in}{3.064989in}}%
\pgfpathcurveto{\pgfqpoint{3.094433in}{3.076039in}}{\pgfqpoint{3.090042in}{3.086638in}}{\pgfqpoint{3.082229in}{3.094451in}}%
\pgfpathcurveto{\pgfqpoint{3.074415in}{3.102265in}}{\pgfqpoint{3.063816in}{3.106655in}}{\pgfqpoint{3.052766in}{3.106655in}}%
\pgfpathcurveto{\pgfqpoint{3.041716in}{3.106655in}}{\pgfqpoint{3.031117in}{3.102265in}}{\pgfqpoint{3.023303in}{3.094451in}}%
\pgfpathcurveto{\pgfqpoint{3.015490in}{3.086638in}}{\pgfqpoint{3.011099in}{3.076039in}}{\pgfqpoint{3.011099in}{3.064989in}}%
\pgfpathcurveto{\pgfqpoint{3.011099in}{3.053939in}}{\pgfqpoint{3.015490in}{3.043340in}}{\pgfqpoint{3.023303in}{3.035526in}}%
\pgfpathcurveto{\pgfqpoint{3.031117in}{3.027712in}}{\pgfqpoint{3.041716in}{3.023322in}}{\pgfqpoint{3.052766in}{3.023322in}}%
\pgfpathclose%
\pgfusepath{stroke,fill}%
\end{pgfscope}%
\begin{pgfscope}%
\pgfpathrectangle{\pgfqpoint{0.600000in}{0.600000in}}{\pgfqpoint{3.900000in}{3.900000in}}%
\pgfusepath{clip}%
\pgfsetbuttcap%
\pgfsetroundjoin%
\definecolor{currentfill}{rgb}{0.121569,0.466667,0.705882}%
\pgfsetfillcolor{currentfill}%
\pgfsetlinewidth{1.003750pt}%
\definecolor{currentstroke}{rgb}{0.121569,0.466667,0.705882}%
\pgfsetstrokecolor{currentstroke}%
\pgfsetdash{}{0pt}%
\pgfpathmoveto{\pgfqpoint{2.780046in}{1.236263in}}%
\pgfpathcurveto{\pgfqpoint{2.791096in}{1.236263in}}{\pgfqpoint{2.801695in}{1.240653in}}{\pgfqpoint{2.809509in}{1.248467in}}%
\pgfpathcurveto{\pgfqpoint{2.817322in}{1.256280in}}{\pgfqpoint{2.821713in}{1.266879in}}{\pgfqpoint{2.821713in}{1.277929in}}%
\pgfpathcurveto{\pgfqpoint{2.821713in}{1.288980in}}{\pgfqpoint{2.817322in}{1.299579in}}{\pgfqpoint{2.809509in}{1.307392in}}%
\pgfpathcurveto{\pgfqpoint{2.801695in}{1.315206in}}{\pgfqpoint{2.791096in}{1.319596in}}{\pgfqpoint{2.780046in}{1.319596in}}%
\pgfpathcurveto{\pgfqpoint{2.768996in}{1.319596in}}{\pgfqpoint{2.758397in}{1.315206in}}{\pgfqpoint{2.750583in}{1.307392in}}%
\pgfpathcurveto{\pgfqpoint{2.742770in}{1.299579in}}{\pgfqpoint{2.738379in}{1.288980in}}{\pgfqpoint{2.738379in}{1.277929in}}%
\pgfpathcurveto{\pgfqpoint{2.738379in}{1.266879in}}{\pgfqpoint{2.742770in}{1.256280in}}{\pgfqpoint{2.750583in}{1.248467in}}%
\pgfpathcurveto{\pgfqpoint{2.758397in}{1.240653in}}{\pgfqpoint{2.768996in}{1.236263in}}{\pgfqpoint{2.780046in}{1.236263in}}%
\pgfpathclose%
\pgfusepath{stroke,fill}%
\end{pgfscope}%
\begin{pgfscope}%
\pgfpathrectangle{\pgfqpoint{0.600000in}{0.600000in}}{\pgfqpoint{3.900000in}{3.900000in}}%
\pgfusepath{clip}%
\pgfsetbuttcap%
\pgfsetroundjoin%
\definecolor{currentfill}{rgb}{0.121569,0.466667,0.705882}%
\pgfsetfillcolor{currentfill}%
\pgfsetlinewidth{1.003750pt}%
\definecolor{currentstroke}{rgb}{0.121569,0.466667,0.705882}%
\pgfsetstrokecolor{currentstroke}%
\pgfsetdash{}{0pt}%
\pgfpathmoveto{\pgfqpoint{2.395423in}{2.620986in}}%
\pgfpathcurveto{\pgfqpoint{2.406473in}{2.620986in}}{\pgfqpoint{2.417072in}{2.625376in}}{\pgfqpoint{2.424886in}{2.633190in}}%
\pgfpathcurveto{\pgfqpoint{2.432700in}{2.641004in}}{\pgfqpoint{2.437090in}{2.651603in}}{\pgfqpoint{2.437090in}{2.662653in}}%
\pgfpathcurveto{\pgfqpoint{2.437090in}{2.673703in}}{\pgfqpoint{2.432700in}{2.684302in}}{\pgfqpoint{2.424886in}{2.692116in}}%
\pgfpathcurveto{\pgfqpoint{2.417072in}{2.699929in}}{\pgfqpoint{2.406473in}{2.704319in}}{\pgfqpoint{2.395423in}{2.704319in}}%
\pgfpathcurveto{\pgfqpoint{2.384373in}{2.704319in}}{\pgfqpoint{2.373774in}{2.699929in}}{\pgfqpoint{2.365960in}{2.692116in}}%
\pgfpathcurveto{\pgfqpoint{2.358147in}{2.684302in}}{\pgfqpoint{2.353756in}{2.673703in}}{\pgfqpoint{2.353756in}{2.662653in}}%
\pgfpathcurveto{\pgfqpoint{2.353756in}{2.651603in}}{\pgfqpoint{2.358147in}{2.641004in}}{\pgfqpoint{2.365960in}{2.633190in}}%
\pgfpathcurveto{\pgfqpoint{2.373774in}{2.625376in}}{\pgfqpoint{2.384373in}{2.620986in}}{\pgfqpoint{2.395423in}{2.620986in}}%
\pgfpathclose%
\pgfusepath{stroke,fill}%
\end{pgfscope}%
\begin{pgfscope}%
\pgfpathrectangle{\pgfqpoint{0.600000in}{0.600000in}}{\pgfqpoint{3.900000in}{3.900000in}}%
\pgfusepath{clip}%
\pgfsetbuttcap%
\pgfsetroundjoin%
\definecolor{currentfill}{rgb}{0.121569,0.466667,0.705882}%
\pgfsetfillcolor{currentfill}%
\pgfsetlinewidth{1.003750pt}%
\definecolor{currentstroke}{rgb}{0.121569,0.466667,0.705882}%
\pgfsetstrokecolor{currentstroke}%
\pgfsetdash{}{0pt}%
\pgfpathmoveto{\pgfqpoint{2.269620in}{3.076055in}}%
\pgfpathcurveto{\pgfqpoint{2.280670in}{3.076055in}}{\pgfqpoint{2.291269in}{3.080445in}}{\pgfqpoint{2.299083in}{3.088259in}}%
\pgfpathcurveto{\pgfqpoint{2.306897in}{3.096073in}}{\pgfqpoint{2.311287in}{3.106672in}}{\pgfqpoint{2.311287in}{3.117722in}}%
\pgfpathcurveto{\pgfqpoint{2.311287in}{3.128772in}}{\pgfqpoint{2.306897in}{3.139371in}}{\pgfqpoint{2.299083in}{3.147185in}}%
\pgfpathcurveto{\pgfqpoint{2.291269in}{3.154998in}}{\pgfqpoint{2.280670in}{3.159388in}}{\pgfqpoint{2.269620in}{3.159388in}}%
\pgfpathcurveto{\pgfqpoint{2.258570in}{3.159388in}}{\pgfqpoint{2.247971in}{3.154998in}}{\pgfqpoint{2.240158in}{3.147185in}}%
\pgfpathcurveto{\pgfqpoint{2.232344in}{3.139371in}}{\pgfqpoint{2.227954in}{3.128772in}}{\pgfqpoint{2.227954in}{3.117722in}}%
\pgfpathcurveto{\pgfqpoint{2.227954in}{3.106672in}}{\pgfqpoint{2.232344in}{3.096073in}}{\pgfqpoint{2.240158in}{3.088259in}}%
\pgfpathcurveto{\pgfqpoint{2.247971in}{3.080445in}}{\pgfqpoint{2.258570in}{3.076055in}}{\pgfqpoint{2.269620in}{3.076055in}}%
\pgfpathclose%
\pgfusepath{stroke,fill}%
\end{pgfscope}%
\begin{pgfscope}%
\pgfpathrectangle{\pgfqpoint{0.600000in}{0.600000in}}{\pgfqpoint{3.900000in}{3.900000in}}%
\pgfusepath{clip}%
\pgfsetbuttcap%
\pgfsetroundjoin%
\definecolor{currentfill}{rgb}{0.121569,0.466667,0.705882}%
\pgfsetfillcolor{currentfill}%
\pgfsetlinewidth{1.003750pt}%
\definecolor{currentstroke}{rgb}{0.121569,0.466667,0.705882}%
\pgfsetstrokecolor{currentstroke}%
\pgfsetdash{}{0pt}%
\pgfpathmoveto{\pgfqpoint{2.582977in}{2.811086in}}%
\pgfpathcurveto{\pgfqpoint{2.594027in}{2.811086in}}{\pgfqpoint{2.604626in}{2.815477in}}{\pgfqpoint{2.612440in}{2.823290in}}%
\pgfpathcurveto{\pgfqpoint{2.620253in}{2.831104in}}{\pgfqpoint{2.624643in}{2.841703in}}{\pgfqpoint{2.624643in}{2.852753in}}%
\pgfpathcurveto{\pgfqpoint{2.624643in}{2.863803in}}{\pgfqpoint{2.620253in}{2.874402in}}{\pgfqpoint{2.612440in}{2.882216in}}%
\pgfpathcurveto{\pgfqpoint{2.604626in}{2.890030in}}{\pgfqpoint{2.594027in}{2.894420in}}{\pgfqpoint{2.582977in}{2.894420in}}%
\pgfpathcurveto{\pgfqpoint{2.571927in}{2.894420in}}{\pgfqpoint{2.561328in}{2.890030in}}{\pgfqpoint{2.553514in}{2.882216in}}%
\pgfpathcurveto{\pgfqpoint{2.545700in}{2.874402in}}{\pgfqpoint{2.541310in}{2.863803in}}{\pgfqpoint{2.541310in}{2.852753in}}%
\pgfpathcurveto{\pgfqpoint{2.541310in}{2.841703in}}{\pgfqpoint{2.545700in}{2.831104in}}{\pgfqpoint{2.553514in}{2.823290in}}%
\pgfpathcurveto{\pgfqpoint{2.561328in}{2.815477in}}{\pgfqpoint{2.571927in}{2.811086in}}{\pgfqpoint{2.582977in}{2.811086in}}%
\pgfpathclose%
\pgfusepath{stroke,fill}%
\end{pgfscope}%
\begin{pgfscope}%
\pgfpathrectangle{\pgfqpoint{0.600000in}{0.600000in}}{\pgfqpoint{3.900000in}{3.900000in}}%
\pgfusepath{clip}%
\pgfsetbuttcap%
\pgfsetroundjoin%
\definecolor{currentfill}{rgb}{0.121569,0.466667,0.705882}%
\pgfsetfillcolor{currentfill}%
\pgfsetlinewidth{1.003750pt}%
\definecolor{currentstroke}{rgb}{0.121569,0.466667,0.705882}%
\pgfsetstrokecolor{currentstroke}%
\pgfsetdash{}{0pt}%
\pgfpathmoveto{\pgfqpoint{2.856406in}{2.825007in}}%
\pgfpathcurveto{\pgfqpoint{2.867456in}{2.825007in}}{\pgfqpoint{2.878055in}{2.829397in}}{\pgfqpoint{2.885869in}{2.837211in}}%
\pgfpathcurveto{\pgfqpoint{2.893683in}{2.845025in}}{\pgfqpoint{2.898073in}{2.855624in}}{\pgfqpoint{2.898073in}{2.866674in}}%
\pgfpathcurveto{\pgfqpoint{2.898073in}{2.877724in}}{\pgfqpoint{2.893683in}{2.888323in}}{\pgfqpoint{2.885869in}{2.896137in}}%
\pgfpathcurveto{\pgfqpoint{2.878055in}{2.903950in}}{\pgfqpoint{2.867456in}{2.908340in}}{\pgfqpoint{2.856406in}{2.908340in}}%
\pgfpathcurveto{\pgfqpoint{2.845356in}{2.908340in}}{\pgfqpoint{2.834757in}{2.903950in}}{\pgfqpoint{2.826943in}{2.896137in}}%
\pgfpathcurveto{\pgfqpoint{2.819130in}{2.888323in}}{\pgfqpoint{2.814740in}{2.877724in}}{\pgfqpoint{2.814740in}{2.866674in}}%
\pgfpathcurveto{\pgfqpoint{2.814740in}{2.855624in}}{\pgfqpoint{2.819130in}{2.845025in}}{\pgfqpoint{2.826943in}{2.837211in}}%
\pgfpathcurveto{\pgfqpoint{2.834757in}{2.829397in}}{\pgfqpoint{2.845356in}{2.825007in}}{\pgfqpoint{2.856406in}{2.825007in}}%
\pgfpathclose%
\pgfusepath{stroke,fill}%
\end{pgfscope}%
\begin{pgfscope}%
\pgfpathrectangle{\pgfqpoint{0.600000in}{0.600000in}}{\pgfqpoint{3.900000in}{3.900000in}}%
\pgfusepath{clip}%
\pgfsetbuttcap%
\pgfsetroundjoin%
\definecolor{currentfill}{rgb}{0.121569,0.466667,0.705882}%
\pgfsetfillcolor{currentfill}%
\pgfsetlinewidth{1.003750pt}%
\definecolor{currentstroke}{rgb}{0.121569,0.466667,0.705882}%
\pgfsetstrokecolor{currentstroke}%
\pgfsetdash{}{0pt}%
\pgfpathmoveto{\pgfqpoint{2.915066in}{2.706491in}}%
\pgfpathcurveto{\pgfqpoint{2.926116in}{2.706491in}}{\pgfqpoint{2.936715in}{2.710881in}}{\pgfqpoint{2.944529in}{2.718695in}}%
\pgfpathcurveto{\pgfqpoint{2.952342in}{2.726508in}}{\pgfqpoint{2.956733in}{2.737107in}}{\pgfqpoint{2.956733in}{2.748158in}}%
\pgfpathcurveto{\pgfqpoint{2.956733in}{2.759208in}}{\pgfqpoint{2.952342in}{2.769807in}}{\pgfqpoint{2.944529in}{2.777620in}}%
\pgfpathcurveto{\pgfqpoint{2.936715in}{2.785434in}}{\pgfqpoint{2.926116in}{2.789824in}}{\pgfqpoint{2.915066in}{2.789824in}}%
\pgfpathcurveto{\pgfqpoint{2.904016in}{2.789824in}}{\pgfqpoint{2.893417in}{2.785434in}}{\pgfqpoint{2.885603in}{2.777620in}}%
\pgfpathcurveto{\pgfqpoint{2.877789in}{2.769807in}}{\pgfqpoint{2.873399in}{2.759208in}}{\pgfqpoint{2.873399in}{2.748158in}}%
\pgfpathcurveto{\pgfqpoint{2.873399in}{2.737107in}}{\pgfqpoint{2.877789in}{2.726508in}}{\pgfqpoint{2.885603in}{2.718695in}}%
\pgfpathcurveto{\pgfqpoint{2.893417in}{2.710881in}}{\pgfqpoint{2.904016in}{2.706491in}}{\pgfqpoint{2.915066in}{2.706491in}}%
\pgfpathclose%
\pgfusepath{stroke,fill}%
\end{pgfscope}%
\begin{pgfscope}%
\pgfpathrectangle{\pgfqpoint{0.600000in}{0.600000in}}{\pgfqpoint{3.900000in}{3.900000in}}%
\pgfusepath{clip}%
\pgfsetbuttcap%
\pgfsetroundjoin%
\definecolor{currentfill}{rgb}{0.121569,0.466667,0.705882}%
\pgfsetfillcolor{currentfill}%
\pgfsetlinewidth{1.003750pt}%
\definecolor{currentstroke}{rgb}{0.121569,0.466667,0.705882}%
\pgfsetstrokecolor{currentstroke}%
\pgfsetdash{}{0pt}%
\pgfpathmoveto{\pgfqpoint{2.728782in}{2.502206in}}%
\pgfpathcurveto{\pgfqpoint{2.739832in}{2.502206in}}{\pgfqpoint{2.750431in}{2.506596in}}{\pgfqpoint{2.758245in}{2.514410in}}%
\pgfpathcurveto{\pgfqpoint{2.766058in}{2.522223in}}{\pgfqpoint{2.770449in}{2.532823in}}{\pgfqpoint{2.770449in}{2.543873in}}%
\pgfpathcurveto{\pgfqpoint{2.770449in}{2.554923in}}{\pgfqpoint{2.766058in}{2.565522in}}{\pgfqpoint{2.758245in}{2.573335in}}%
\pgfpathcurveto{\pgfqpoint{2.750431in}{2.581149in}}{\pgfqpoint{2.739832in}{2.585539in}}{\pgfqpoint{2.728782in}{2.585539in}}%
\pgfpathcurveto{\pgfqpoint{2.717732in}{2.585539in}}{\pgfqpoint{2.707133in}{2.581149in}}{\pgfqpoint{2.699319in}{2.573335in}}%
\pgfpathcurveto{\pgfqpoint{2.691506in}{2.565522in}}{\pgfqpoint{2.687115in}{2.554923in}}{\pgfqpoint{2.687115in}{2.543873in}}%
\pgfpathcurveto{\pgfqpoint{2.687115in}{2.532823in}}{\pgfqpoint{2.691506in}{2.522223in}}{\pgfqpoint{2.699319in}{2.514410in}}%
\pgfpathcurveto{\pgfqpoint{2.707133in}{2.506596in}}{\pgfqpoint{2.717732in}{2.502206in}}{\pgfqpoint{2.728782in}{2.502206in}}%
\pgfpathclose%
\pgfusepath{stroke,fill}%
\end{pgfscope}%
\begin{pgfscope}%
\pgfpathrectangle{\pgfqpoint{0.600000in}{0.600000in}}{\pgfqpoint{3.900000in}{3.900000in}}%
\pgfusepath{clip}%
\pgfsetbuttcap%
\pgfsetroundjoin%
\definecolor{currentfill}{rgb}{0.121569,0.466667,0.705882}%
\pgfsetfillcolor{currentfill}%
\pgfsetlinewidth{1.003750pt}%
\definecolor{currentstroke}{rgb}{0.121569,0.466667,0.705882}%
\pgfsetstrokecolor{currentstroke}%
\pgfsetdash{}{0pt}%
\pgfpathmoveto{\pgfqpoint{3.448690in}{1.945204in}}%
\pgfpathcurveto{\pgfqpoint{3.459740in}{1.945204in}}{\pgfqpoint{3.470339in}{1.949595in}}{\pgfqpoint{3.478153in}{1.957408in}}%
\pgfpathcurveto{\pgfqpoint{3.485967in}{1.965222in}}{\pgfqpoint{3.490357in}{1.975821in}}{\pgfqpoint{3.490357in}{1.986871in}}%
\pgfpathcurveto{\pgfqpoint{3.490357in}{1.997921in}}{\pgfqpoint{3.485967in}{2.008520in}}{\pgfqpoint{3.478153in}{2.016334in}}%
\pgfpathcurveto{\pgfqpoint{3.470339in}{2.024147in}}{\pgfqpoint{3.459740in}{2.028538in}}{\pgfqpoint{3.448690in}{2.028538in}}%
\pgfpathcurveto{\pgfqpoint{3.437640in}{2.028538in}}{\pgfqpoint{3.427041in}{2.024147in}}{\pgfqpoint{3.419227in}{2.016334in}}%
\pgfpathcurveto{\pgfqpoint{3.411414in}{2.008520in}}{\pgfqpoint{3.407024in}{1.997921in}}{\pgfqpoint{3.407024in}{1.986871in}}%
\pgfpathcurveto{\pgfqpoint{3.407024in}{1.975821in}}{\pgfqpoint{3.411414in}{1.965222in}}{\pgfqpoint{3.419227in}{1.957408in}}%
\pgfpathcurveto{\pgfqpoint{3.427041in}{1.949595in}}{\pgfqpoint{3.437640in}{1.945204in}}{\pgfqpoint{3.448690in}{1.945204in}}%
\pgfpathclose%
\pgfusepath{stroke,fill}%
\end{pgfscope}%
\begin{pgfscope}%
\pgfpathrectangle{\pgfqpoint{0.600000in}{0.600000in}}{\pgfqpoint{3.900000in}{3.900000in}}%
\pgfusepath{clip}%
\pgfsetbuttcap%
\pgfsetroundjoin%
\definecolor{currentfill}{rgb}{0.121569,0.466667,0.705882}%
\pgfsetfillcolor{currentfill}%
\pgfsetlinewidth{1.003750pt}%
\definecolor{currentstroke}{rgb}{0.121569,0.466667,0.705882}%
\pgfsetstrokecolor{currentstroke}%
\pgfsetdash{}{0pt}%
\pgfpathmoveto{\pgfqpoint{3.076762in}{2.534780in}}%
\pgfpathcurveto{\pgfqpoint{3.087812in}{2.534780in}}{\pgfqpoint{3.098411in}{2.539171in}}{\pgfqpoint{3.106225in}{2.546984in}}%
\pgfpathcurveto{\pgfqpoint{3.114039in}{2.554798in}}{\pgfqpoint{3.118429in}{2.565397in}}{\pgfqpoint{3.118429in}{2.576447in}}%
\pgfpathcurveto{\pgfqpoint{3.118429in}{2.587497in}}{\pgfqpoint{3.114039in}{2.598096in}}{\pgfqpoint{3.106225in}{2.605910in}}%
\pgfpathcurveto{\pgfqpoint{3.098411in}{2.613723in}}{\pgfqpoint{3.087812in}{2.618114in}}{\pgfqpoint{3.076762in}{2.618114in}}%
\pgfpathcurveto{\pgfqpoint{3.065712in}{2.618114in}}{\pgfqpoint{3.055113in}{2.613723in}}{\pgfqpoint{3.047300in}{2.605910in}}%
\pgfpathcurveto{\pgfqpoint{3.039486in}{2.598096in}}{\pgfqpoint{3.035096in}{2.587497in}}{\pgfqpoint{3.035096in}{2.576447in}}%
\pgfpathcurveto{\pgfqpoint{3.035096in}{2.565397in}}{\pgfqpoint{3.039486in}{2.554798in}}{\pgfqpoint{3.047300in}{2.546984in}}%
\pgfpathcurveto{\pgfqpoint{3.055113in}{2.539171in}}{\pgfqpoint{3.065712in}{2.534780in}}{\pgfqpoint{3.076762in}{2.534780in}}%
\pgfpathclose%
\pgfusepath{stroke,fill}%
\end{pgfscope}%
\begin{pgfscope}%
\pgfpathrectangle{\pgfqpoint{0.600000in}{0.600000in}}{\pgfqpoint{3.900000in}{3.900000in}}%
\pgfusepath{clip}%
\pgfsetbuttcap%
\pgfsetroundjoin%
\definecolor{currentfill}{rgb}{0.121569,0.466667,0.705882}%
\pgfsetfillcolor{currentfill}%
\pgfsetlinewidth{1.003750pt}%
\definecolor{currentstroke}{rgb}{0.121569,0.466667,0.705882}%
\pgfsetstrokecolor{currentstroke}%
\pgfsetdash{}{0pt}%
\pgfpathmoveto{\pgfqpoint{2.346854in}{2.903010in}}%
\pgfpathcurveto{\pgfqpoint{2.357904in}{2.903010in}}{\pgfqpoint{2.368503in}{2.907400in}}{\pgfqpoint{2.376316in}{2.915214in}}%
\pgfpathcurveto{\pgfqpoint{2.384130in}{2.923028in}}{\pgfqpoint{2.388520in}{2.933627in}}{\pgfqpoint{2.388520in}{2.944677in}}%
\pgfpathcurveto{\pgfqpoint{2.388520in}{2.955727in}}{\pgfqpoint{2.384130in}{2.966326in}}{\pgfqpoint{2.376316in}{2.974140in}}%
\pgfpathcurveto{\pgfqpoint{2.368503in}{2.981953in}}{\pgfqpoint{2.357904in}{2.986344in}}{\pgfqpoint{2.346854in}{2.986344in}}%
\pgfpathcurveto{\pgfqpoint{2.335803in}{2.986344in}}{\pgfqpoint{2.325204in}{2.981953in}}{\pgfqpoint{2.317391in}{2.974140in}}%
\pgfpathcurveto{\pgfqpoint{2.309577in}{2.966326in}}{\pgfqpoint{2.305187in}{2.955727in}}{\pgfqpoint{2.305187in}{2.944677in}}%
\pgfpathcurveto{\pgfqpoint{2.305187in}{2.933627in}}{\pgfqpoint{2.309577in}{2.923028in}}{\pgfqpoint{2.317391in}{2.915214in}}%
\pgfpathcurveto{\pgfqpoint{2.325204in}{2.907400in}}{\pgfqpoint{2.335803in}{2.903010in}}{\pgfqpoint{2.346854in}{2.903010in}}%
\pgfpathclose%
\pgfusepath{stroke,fill}%
\end{pgfscope}%
\begin{pgfscope}%
\pgfpathrectangle{\pgfqpoint{0.600000in}{0.600000in}}{\pgfqpoint{3.900000in}{3.900000in}}%
\pgfusepath{clip}%
\pgfsetbuttcap%
\pgfsetroundjoin%
\definecolor{currentfill}{rgb}{0.121569,0.466667,0.705882}%
\pgfsetfillcolor{currentfill}%
\pgfsetlinewidth{1.003750pt}%
\definecolor{currentstroke}{rgb}{0.121569,0.466667,0.705882}%
\pgfsetstrokecolor{currentstroke}%
\pgfsetdash{}{0pt}%
\pgfpathmoveto{\pgfqpoint{2.912972in}{2.371037in}}%
\pgfpathcurveto{\pgfqpoint{2.924022in}{2.371037in}}{\pgfqpoint{2.934621in}{2.375428in}}{\pgfqpoint{2.942435in}{2.383241in}}%
\pgfpathcurveto{\pgfqpoint{2.950249in}{2.391055in}}{\pgfqpoint{2.954639in}{2.401654in}}{\pgfqpoint{2.954639in}{2.412704in}}%
\pgfpathcurveto{\pgfqpoint{2.954639in}{2.423754in}}{\pgfqpoint{2.950249in}{2.434353in}}{\pgfqpoint{2.942435in}{2.442167in}}%
\pgfpathcurveto{\pgfqpoint{2.934621in}{2.449981in}}{\pgfqpoint{2.924022in}{2.454371in}}{\pgfqpoint{2.912972in}{2.454371in}}%
\pgfpathcurveto{\pgfqpoint{2.901922in}{2.454371in}}{\pgfqpoint{2.891323in}{2.449981in}}{\pgfqpoint{2.883509in}{2.442167in}}%
\pgfpathcurveto{\pgfqpoint{2.875696in}{2.434353in}}{\pgfqpoint{2.871305in}{2.423754in}}{\pgfqpoint{2.871305in}{2.412704in}}%
\pgfpathcurveto{\pgfqpoint{2.871305in}{2.401654in}}{\pgfqpoint{2.875696in}{2.391055in}}{\pgfqpoint{2.883509in}{2.383241in}}%
\pgfpathcurveto{\pgfqpoint{2.891323in}{2.375428in}}{\pgfqpoint{2.901922in}{2.371037in}}{\pgfqpoint{2.912972in}{2.371037in}}%
\pgfpathclose%
\pgfusepath{stroke,fill}%
\end{pgfscope}%
\begin{pgfscope}%
\pgfpathrectangle{\pgfqpoint{0.600000in}{0.600000in}}{\pgfqpoint{3.900000in}{3.900000in}}%
\pgfusepath{clip}%
\pgfsetbuttcap%
\pgfsetroundjoin%
\definecolor{currentfill}{rgb}{0.121569,0.466667,0.705882}%
\pgfsetfillcolor{currentfill}%
\pgfsetlinewidth{1.003750pt}%
\definecolor{currentstroke}{rgb}{0.121569,0.466667,0.705882}%
\pgfsetstrokecolor{currentstroke}%
\pgfsetdash{}{0pt}%
\pgfpathmoveto{\pgfqpoint{2.763695in}{2.912411in}}%
\pgfpathcurveto{\pgfqpoint{2.774745in}{2.912411in}}{\pgfqpoint{2.785344in}{2.916801in}}{\pgfqpoint{2.793157in}{2.924615in}}%
\pgfpathcurveto{\pgfqpoint{2.800971in}{2.932429in}}{\pgfqpoint{2.805361in}{2.943028in}}{\pgfqpoint{2.805361in}{2.954078in}}%
\pgfpathcurveto{\pgfqpoint{2.805361in}{2.965128in}}{\pgfqpoint{2.800971in}{2.975727in}}{\pgfqpoint{2.793157in}{2.983541in}}%
\pgfpathcurveto{\pgfqpoint{2.785344in}{2.991354in}}{\pgfqpoint{2.774745in}{2.995745in}}{\pgfqpoint{2.763695in}{2.995745in}}%
\pgfpathcurveto{\pgfqpoint{2.752644in}{2.995745in}}{\pgfqpoint{2.742045in}{2.991354in}}{\pgfqpoint{2.734232in}{2.983541in}}%
\pgfpathcurveto{\pgfqpoint{2.726418in}{2.975727in}}{\pgfqpoint{2.722028in}{2.965128in}}{\pgfqpoint{2.722028in}{2.954078in}}%
\pgfpathcurveto{\pgfqpoint{2.722028in}{2.943028in}}{\pgfqpoint{2.726418in}{2.932429in}}{\pgfqpoint{2.734232in}{2.924615in}}%
\pgfpathcurveto{\pgfqpoint{2.742045in}{2.916801in}}{\pgfqpoint{2.752644in}{2.912411in}}{\pgfqpoint{2.763695in}{2.912411in}}%
\pgfpathclose%
\pgfusepath{stroke,fill}%
\end{pgfscope}%
\begin{pgfscope}%
\pgfpathrectangle{\pgfqpoint{0.600000in}{0.600000in}}{\pgfqpoint{3.900000in}{3.900000in}}%
\pgfusepath{clip}%
\pgfsetbuttcap%
\pgfsetroundjoin%
\definecolor{currentfill}{rgb}{0.121569,0.466667,0.705882}%
\pgfsetfillcolor{currentfill}%
\pgfsetlinewidth{1.003750pt}%
\definecolor{currentstroke}{rgb}{0.121569,0.466667,0.705882}%
\pgfsetstrokecolor{currentstroke}%
\pgfsetdash{}{0pt}%
\pgfpathmoveto{\pgfqpoint{1.774264in}{2.618469in}}%
\pgfpathcurveto{\pgfqpoint{1.785314in}{2.618469in}}{\pgfqpoint{1.795913in}{2.622859in}}{\pgfqpoint{1.803726in}{2.630672in}}%
\pgfpathcurveto{\pgfqpoint{1.811540in}{2.638486in}}{\pgfqpoint{1.815930in}{2.649085in}}{\pgfqpoint{1.815930in}{2.660135in}}%
\pgfpathcurveto{\pgfqpoint{1.815930in}{2.671185in}}{\pgfqpoint{1.811540in}{2.681784in}}{\pgfqpoint{1.803726in}{2.689598in}}%
\pgfpathcurveto{\pgfqpoint{1.795913in}{2.697412in}}{\pgfqpoint{1.785314in}{2.701802in}}{\pgfqpoint{1.774264in}{2.701802in}}%
\pgfpathcurveto{\pgfqpoint{1.763214in}{2.701802in}}{\pgfqpoint{1.752614in}{2.697412in}}{\pgfqpoint{1.744801in}{2.689598in}}%
\pgfpathcurveto{\pgfqpoint{1.736987in}{2.681784in}}{\pgfqpoint{1.732597in}{2.671185in}}{\pgfqpoint{1.732597in}{2.660135in}}%
\pgfpathcurveto{\pgfqpoint{1.732597in}{2.649085in}}{\pgfqpoint{1.736987in}{2.638486in}}{\pgfqpoint{1.744801in}{2.630672in}}%
\pgfpathcurveto{\pgfqpoint{1.752614in}{2.622859in}}{\pgfqpoint{1.763214in}{2.618469in}}{\pgfqpoint{1.774264in}{2.618469in}}%
\pgfpathclose%
\pgfusepath{stroke,fill}%
\end{pgfscope}%
\begin{pgfscope}%
\pgfpathrectangle{\pgfqpoint{0.600000in}{0.600000in}}{\pgfqpoint{3.900000in}{3.900000in}}%
\pgfusepath{clip}%
\pgfsetbuttcap%
\pgfsetroundjoin%
\definecolor{currentfill}{rgb}{0.121569,0.466667,0.705882}%
\pgfsetfillcolor{currentfill}%
\pgfsetlinewidth{1.003750pt}%
\definecolor{currentstroke}{rgb}{0.121569,0.466667,0.705882}%
\pgfsetstrokecolor{currentstroke}%
\pgfsetdash{}{0pt}%
\pgfpathmoveto{\pgfqpoint{2.361019in}{3.046348in}}%
\pgfpathcurveto{\pgfqpoint{2.372069in}{3.046348in}}{\pgfqpoint{2.382668in}{3.050738in}}{\pgfqpoint{2.390482in}{3.058552in}}%
\pgfpathcurveto{\pgfqpoint{2.398295in}{3.066365in}}{\pgfqpoint{2.402686in}{3.076964in}}{\pgfqpoint{2.402686in}{3.088014in}}%
\pgfpathcurveto{\pgfqpoint{2.402686in}{3.099065in}}{\pgfqpoint{2.398295in}{3.109664in}}{\pgfqpoint{2.390482in}{3.117477in}}%
\pgfpathcurveto{\pgfqpoint{2.382668in}{3.125291in}}{\pgfqpoint{2.372069in}{3.129681in}}{\pgfqpoint{2.361019in}{3.129681in}}%
\pgfpathcurveto{\pgfqpoint{2.349969in}{3.129681in}}{\pgfqpoint{2.339370in}{3.125291in}}{\pgfqpoint{2.331556in}{3.117477in}}%
\pgfpathcurveto{\pgfqpoint{2.323742in}{3.109664in}}{\pgfqpoint{2.319352in}{3.099065in}}{\pgfqpoint{2.319352in}{3.088014in}}%
\pgfpathcurveto{\pgfqpoint{2.319352in}{3.076964in}}{\pgfqpoint{2.323742in}{3.066365in}}{\pgfqpoint{2.331556in}{3.058552in}}%
\pgfpathcurveto{\pgfqpoint{2.339370in}{3.050738in}}{\pgfqpoint{2.349969in}{3.046348in}}{\pgfqpoint{2.361019in}{3.046348in}}%
\pgfpathclose%
\pgfusepath{stroke,fill}%
\end{pgfscope}%
\begin{pgfscope}%
\pgfpathrectangle{\pgfqpoint{0.600000in}{0.600000in}}{\pgfqpoint{3.900000in}{3.900000in}}%
\pgfusepath{clip}%
\pgfsetbuttcap%
\pgfsetroundjoin%
\definecolor{currentfill}{rgb}{0.121569,0.466667,0.705882}%
\pgfsetfillcolor{currentfill}%
\pgfsetlinewidth{1.003750pt}%
\definecolor{currentstroke}{rgb}{0.121569,0.466667,0.705882}%
\pgfsetstrokecolor{currentstroke}%
\pgfsetdash{}{0pt}%
\pgfpathmoveto{\pgfqpoint{2.537826in}{1.583497in}}%
\pgfpathcurveto{\pgfqpoint{2.548877in}{1.583497in}}{\pgfqpoint{2.559476in}{1.587887in}}{\pgfqpoint{2.567289in}{1.595700in}}%
\pgfpathcurveto{\pgfqpoint{2.575103in}{1.603514in}}{\pgfqpoint{2.579493in}{1.614113in}}{\pgfqpoint{2.579493in}{1.625163in}}%
\pgfpathcurveto{\pgfqpoint{2.579493in}{1.636213in}}{\pgfqpoint{2.575103in}{1.646812in}}{\pgfqpoint{2.567289in}{1.654626in}}%
\pgfpathcurveto{\pgfqpoint{2.559476in}{1.662440in}}{\pgfqpoint{2.548877in}{1.666830in}}{\pgfqpoint{2.537826in}{1.666830in}}%
\pgfpathcurveto{\pgfqpoint{2.526776in}{1.666830in}}{\pgfqpoint{2.516177in}{1.662440in}}{\pgfqpoint{2.508364in}{1.654626in}}%
\pgfpathcurveto{\pgfqpoint{2.500550in}{1.646812in}}{\pgfqpoint{2.496160in}{1.636213in}}{\pgfqpoint{2.496160in}{1.625163in}}%
\pgfpathcurveto{\pgfqpoint{2.496160in}{1.614113in}}{\pgfqpoint{2.500550in}{1.603514in}}{\pgfqpoint{2.508364in}{1.595700in}}%
\pgfpathcurveto{\pgfqpoint{2.516177in}{1.587887in}}{\pgfqpoint{2.526776in}{1.583497in}}{\pgfqpoint{2.537826in}{1.583497in}}%
\pgfpathclose%
\pgfusepath{stroke,fill}%
\end{pgfscope}%
\begin{pgfscope}%
\pgfpathrectangle{\pgfqpoint{0.600000in}{0.600000in}}{\pgfqpoint{3.900000in}{3.900000in}}%
\pgfusepath{clip}%
\pgfsetbuttcap%
\pgfsetroundjoin%
\definecolor{currentfill}{rgb}{0.121569,0.466667,0.705882}%
\pgfsetfillcolor{currentfill}%
\pgfsetlinewidth{1.003750pt}%
\definecolor{currentstroke}{rgb}{0.121569,0.466667,0.705882}%
\pgfsetstrokecolor{currentstroke}%
\pgfsetdash{}{0pt}%
\pgfpathmoveto{\pgfqpoint{2.192903in}{2.707316in}}%
\pgfpathcurveto{\pgfqpoint{2.203953in}{2.707316in}}{\pgfqpoint{2.214552in}{2.711706in}}{\pgfqpoint{2.222365in}{2.719520in}}%
\pgfpathcurveto{\pgfqpoint{2.230179in}{2.727333in}}{\pgfqpoint{2.234569in}{2.737932in}}{\pgfqpoint{2.234569in}{2.748982in}}%
\pgfpathcurveto{\pgfqpoint{2.234569in}{2.760033in}}{\pgfqpoint{2.230179in}{2.770632in}}{\pgfqpoint{2.222365in}{2.778445in}}%
\pgfpathcurveto{\pgfqpoint{2.214552in}{2.786259in}}{\pgfqpoint{2.203953in}{2.790649in}}{\pgfqpoint{2.192903in}{2.790649in}}%
\pgfpathcurveto{\pgfqpoint{2.181853in}{2.790649in}}{\pgfqpoint{2.171253in}{2.786259in}}{\pgfqpoint{2.163440in}{2.778445in}}%
\pgfpathcurveto{\pgfqpoint{2.155626in}{2.770632in}}{\pgfqpoint{2.151236in}{2.760033in}}{\pgfqpoint{2.151236in}{2.748982in}}%
\pgfpathcurveto{\pgfqpoint{2.151236in}{2.737932in}}{\pgfqpoint{2.155626in}{2.727333in}}{\pgfqpoint{2.163440in}{2.719520in}}%
\pgfpathcurveto{\pgfqpoint{2.171253in}{2.711706in}}{\pgfqpoint{2.181853in}{2.707316in}}{\pgfqpoint{2.192903in}{2.707316in}}%
\pgfpathclose%
\pgfusepath{stroke,fill}%
\end{pgfscope}%
\begin{pgfscope}%
\pgfpathrectangle{\pgfqpoint{0.600000in}{0.600000in}}{\pgfqpoint{3.900000in}{3.900000in}}%
\pgfusepath{clip}%
\pgfsetbuttcap%
\pgfsetroundjoin%
\definecolor{currentfill}{rgb}{0.121569,0.466667,0.705882}%
\pgfsetfillcolor{currentfill}%
\pgfsetlinewidth{1.003750pt}%
\definecolor{currentstroke}{rgb}{0.121569,0.466667,0.705882}%
\pgfsetstrokecolor{currentstroke}%
\pgfsetdash{}{0pt}%
\pgfpathmoveto{\pgfqpoint{2.681177in}{2.586788in}}%
\pgfpathcurveto{\pgfqpoint{2.692228in}{2.586788in}}{\pgfqpoint{2.702827in}{2.591178in}}{\pgfqpoint{2.710640in}{2.598992in}}%
\pgfpathcurveto{\pgfqpoint{2.718454in}{2.606806in}}{\pgfqpoint{2.722844in}{2.617405in}}{\pgfqpoint{2.722844in}{2.628455in}}%
\pgfpathcurveto{\pgfqpoint{2.722844in}{2.639505in}}{\pgfqpoint{2.718454in}{2.650104in}}{\pgfqpoint{2.710640in}{2.657918in}}%
\pgfpathcurveto{\pgfqpoint{2.702827in}{2.665731in}}{\pgfqpoint{2.692228in}{2.670122in}}{\pgfqpoint{2.681177in}{2.670122in}}%
\pgfpathcurveto{\pgfqpoint{2.670127in}{2.670122in}}{\pgfqpoint{2.659528in}{2.665731in}}{\pgfqpoint{2.651715in}{2.657918in}}%
\pgfpathcurveto{\pgfqpoint{2.643901in}{2.650104in}}{\pgfqpoint{2.639511in}{2.639505in}}{\pgfqpoint{2.639511in}{2.628455in}}%
\pgfpathcurveto{\pgfqpoint{2.639511in}{2.617405in}}{\pgfqpoint{2.643901in}{2.606806in}}{\pgfqpoint{2.651715in}{2.598992in}}%
\pgfpathcurveto{\pgfqpoint{2.659528in}{2.591178in}}{\pgfqpoint{2.670127in}{2.586788in}}{\pgfqpoint{2.681177in}{2.586788in}}%
\pgfpathclose%
\pgfusepath{stroke,fill}%
\end{pgfscope}%
\begin{pgfscope}%
\pgfpathrectangle{\pgfqpoint{0.600000in}{0.600000in}}{\pgfqpoint{3.900000in}{3.900000in}}%
\pgfusepath{clip}%
\pgfsetbuttcap%
\pgfsetroundjoin%
\definecolor{currentfill}{rgb}{0.121569,0.466667,0.705882}%
\pgfsetfillcolor{currentfill}%
\pgfsetlinewidth{1.003750pt}%
\definecolor{currentstroke}{rgb}{0.121569,0.466667,0.705882}%
\pgfsetstrokecolor{currentstroke}%
\pgfsetdash{}{0pt}%
\pgfpathmoveto{\pgfqpoint{2.405121in}{2.905021in}}%
\pgfpathcurveto{\pgfqpoint{2.416171in}{2.905021in}}{\pgfqpoint{2.426770in}{2.909411in}}{\pgfqpoint{2.434584in}{2.917225in}}%
\pgfpathcurveto{\pgfqpoint{2.442397in}{2.925038in}}{\pgfqpoint{2.446788in}{2.935637in}}{\pgfqpoint{2.446788in}{2.946687in}}%
\pgfpathcurveto{\pgfqpoint{2.446788in}{2.957737in}}{\pgfqpoint{2.442397in}{2.968336in}}{\pgfqpoint{2.434584in}{2.976150in}}%
\pgfpathcurveto{\pgfqpoint{2.426770in}{2.983964in}}{\pgfqpoint{2.416171in}{2.988354in}}{\pgfqpoint{2.405121in}{2.988354in}}%
\pgfpathcurveto{\pgfqpoint{2.394071in}{2.988354in}}{\pgfqpoint{2.383472in}{2.983964in}}{\pgfqpoint{2.375658in}{2.976150in}}%
\pgfpathcurveto{\pgfqpoint{2.367845in}{2.968336in}}{\pgfqpoint{2.363454in}{2.957737in}}{\pgfqpoint{2.363454in}{2.946687in}}%
\pgfpathcurveto{\pgfqpoint{2.363454in}{2.935637in}}{\pgfqpoint{2.367845in}{2.925038in}}{\pgfqpoint{2.375658in}{2.917225in}}%
\pgfpathcurveto{\pgfqpoint{2.383472in}{2.909411in}}{\pgfqpoint{2.394071in}{2.905021in}}{\pgfqpoint{2.405121in}{2.905021in}}%
\pgfpathclose%
\pgfusepath{stroke,fill}%
\end{pgfscope}%
\begin{pgfscope}%
\pgfpathrectangle{\pgfqpoint{0.600000in}{0.600000in}}{\pgfqpoint{3.900000in}{3.900000in}}%
\pgfusepath{clip}%
\pgfsetbuttcap%
\pgfsetroundjoin%
\definecolor{currentfill}{rgb}{0.121569,0.466667,0.705882}%
\pgfsetfillcolor{currentfill}%
\pgfsetlinewidth{1.003750pt}%
\definecolor{currentstroke}{rgb}{0.121569,0.466667,0.705882}%
\pgfsetstrokecolor{currentstroke}%
\pgfsetdash{}{0pt}%
\pgfpathmoveto{\pgfqpoint{2.433168in}{2.357686in}}%
\pgfpathcurveto{\pgfqpoint{2.444218in}{2.357686in}}{\pgfqpoint{2.454817in}{2.362076in}}{\pgfqpoint{2.462631in}{2.369890in}}%
\pgfpathcurveto{\pgfqpoint{2.470444in}{2.377703in}}{\pgfqpoint{2.474835in}{2.388302in}}{\pgfqpoint{2.474835in}{2.399352in}}%
\pgfpathcurveto{\pgfqpoint{2.474835in}{2.410403in}}{\pgfqpoint{2.470444in}{2.421002in}}{\pgfqpoint{2.462631in}{2.428815in}}%
\pgfpathcurveto{\pgfqpoint{2.454817in}{2.436629in}}{\pgfqpoint{2.444218in}{2.441019in}}{\pgfqpoint{2.433168in}{2.441019in}}%
\pgfpathcurveto{\pgfqpoint{2.422118in}{2.441019in}}{\pgfqpoint{2.411519in}{2.436629in}}{\pgfqpoint{2.403705in}{2.428815in}}%
\pgfpathcurveto{\pgfqpoint{2.395892in}{2.421002in}}{\pgfqpoint{2.391501in}{2.410403in}}{\pgfqpoint{2.391501in}{2.399352in}}%
\pgfpathcurveto{\pgfqpoint{2.391501in}{2.388302in}}{\pgfqpoint{2.395892in}{2.377703in}}{\pgfqpoint{2.403705in}{2.369890in}}%
\pgfpathcurveto{\pgfqpoint{2.411519in}{2.362076in}}{\pgfqpoint{2.422118in}{2.357686in}}{\pgfqpoint{2.433168in}{2.357686in}}%
\pgfpathclose%
\pgfusepath{stroke,fill}%
\end{pgfscope}%
\begin{pgfscope}%
\pgfpathrectangle{\pgfqpoint{0.600000in}{0.600000in}}{\pgfqpoint{3.900000in}{3.900000in}}%
\pgfusepath{clip}%
\pgfsetbuttcap%
\pgfsetroundjoin%
\definecolor{currentfill}{rgb}{0.121569,0.466667,0.705882}%
\pgfsetfillcolor{currentfill}%
\pgfsetlinewidth{1.003750pt}%
\definecolor{currentstroke}{rgb}{0.121569,0.466667,0.705882}%
\pgfsetstrokecolor{currentstroke}%
\pgfsetdash{}{0pt}%
\pgfpathmoveto{\pgfqpoint{1.659890in}{2.566939in}}%
\pgfpathcurveto{\pgfqpoint{1.670940in}{2.566939in}}{\pgfqpoint{1.681539in}{2.571329in}}{\pgfqpoint{1.689353in}{2.579143in}}%
\pgfpathcurveto{\pgfqpoint{1.697167in}{2.586956in}}{\pgfqpoint{1.701557in}{2.597555in}}{\pgfqpoint{1.701557in}{2.608605in}}%
\pgfpathcurveto{\pgfqpoint{1.701557in}{2.619655in}}{\pgfqpoint{1.697167in}{2.630254in}}{\pgfqpoint{1.689353in}{2.638068in}}%
\pgfpathcurveto{\pgfqpoint{1.681539in}{2.645882in}}{\pgfqpoint{1.670940in}{2.650272in}}{\pgfqpoint{1.659890in}{2.650272in}}%
\pgfpathcurveto{\pgfqpoint{1.648840in}{2.650272in}}{\pgfqpoint{1.638241in}{2.645882in}}{\pgfqpoint{1.630428in}{2.638068in}}%
\pgfpathcurveto{\pgfqpoint{1.622614in}{2.630254in}}{\pgfqpoint{1.618224in}{2.619655in}}{\pgfqpoint{1.618224in}{2.608605in}}%
\pgfpathcurveto{\pgfqpoint{1.618224in}{2.597555in}}{\pgfqpoint{1.622614in}{2.586956in}}{\pgfqpoint{1.630428in}{2.579143in}}%
\pgfpathcurveto{\pgfqpoint{1.638241in}{2.571329in}}{\pgfqpoint{1.648840in}{2.566939in}}{\pgfqpoint{1.659890in}{2.566939in}}%
\pgfpathclose%
\pgfusepath{stroke,fill}%
\end{pgfscope}%
\begin{pgfscope}%
\pgfpathrectangle{\pgfqpoint{0.600000in}{0.600000in}}{\pgfqpoint{3.900000in}{3.900000in}}%
\pgfusepath{clip}%
\pgfsetbuttcap%
\pgfsetroundjoin%
\definecolor{currentfill}{rgb}{0.121569,0.466667,0.705882}%
\pgfsetfillcolor{currentfill}%
\pgfsetlinewidth{1.003750pt}%
\definecolor{currentstroke}{rgb}{0.121569,0.466667,0.705882}%
\pgfsetstrokecolor{currentstroke}%
\pgfsetdash{}{0pt}%
\pgfpathmoveto{\pgfqpoint{2.777714in}{2.374463in}}%
\pgfpathcurveto{\pgfqpoint{2.788764in}{2.374463in}}{\pgfqpoint{2.799363in}{2.378853in}}{\pgfqpoint{2.807177in}{2.386667in}}%
\pgfpathcurveto{\pgfqpoint{2.814990in}{2.394480in}}{\pgfqpoint{2.819381in}{2.405079in}}{\pgfqpoint{2.819381in}{2.416130in}}%
\pgfpathcurveto{\pgfqpoint{2.819381in}{2.427180in}}{\pgfqpoint{2.814990in}{2.437779in}}{\pgfqpoint{2.807177in}{2.445592in}}%
\pgfpathcurveto{\pgfqpoint{2.799363in}{2.453406in}}{\pgfqpoint{2.788764in}{2.457796in}}{\pgfqpoint{2.777714in}{2.457796in}}%
\pgfpathcurveto{\pgfqpoint{2.766664in}{2.457796in}}{\pgfqpoint{2.756065in}{2.453406in}}{\pgfqpoint{2.748251in}{2.445592in}}%
\pgfpathcurveto{\pgfqpoint{2.740437in}{2.437779in}}{\pgfqpoint{2.736047in}{2.427180in}}{\pgfqpoint{2.736047in}{2.416130in}}%
\pgfpathcurveto{\pgfqpoint{2.736047in}{2.405079in}}{\pgfqpoint{2.740437in}{2.394480in}}{\pgfqpoint{2.748251in}{2.386667in}}%
\pgfpathcurveto{\pgfqpoint{2.756065in}{2.378853in}}{\pgfqpoint{2.766664in}{2.374463in}}{\pgfqpoint{2.777714in}{2.374463in}}%
\pgfpathclose%
\pgfusepath{stroke,fill}%
\end{pgfscope}%
\begin{pgfscope}%
\pgfpathrectangle{\pgfqpoint{0.600000in}{0.600000in}}{\pgfqpoint{3.900000in}{3.900000in}}%
\pgfusepath{clip}%
\pgfsetbuttcap%
\pgfsetroundjoin%
\definecolor{currentfill}{rgb}{0.121569,0.466667,0.705882}%
\pgfsetfillcolor{currentfill}%
\pgfsetlinewidth{1.003750pt}%
\definecolor{currentstroke}{rgb}{0.121569,0.466667,0.705882}%
\pgfsetstrokecolor{currentstroke}%
\pgfsetdash{}{0pt}%
\pgfpathmoveto{\pgfqpoint{2.300912in}{2.913695in}}%
\pgfpathcurveto{\pgfqpoint{2.311962in}{2.913695in}}{\pgfqpoint{2.322561in}{2.918086in}}{\pgfqpoint{2.330374in}{2.925899in}}%
\pgfpathcurveto{\pgfqpoint{2.338188in}{2.933713in}}{\pgfqpoint{2.342578in}{2.944312in}}{\pgfqpoint{2.342578in}{2.955362in}}%
\pgfpathcurveto{\pgfqpoint{2.342578in}{2.966412in}}{\pgfqpoint{2.338188in}{2.977011in}}{\pgfqpoint{2.330374in}{2.984825in}}%
\pgfpathcurveto{\pgfqpoint{2.322561in}{2.992639in}}{\pgfqpoint{2.311962in}{2.997029in}}{\pgfqpoint{2.300912in}{2.997029in}}%
\pgfpathcurveto{\pgfqpoint{2.289861in}{2.997029in}}{\pgfqpoint{2.279262in}{2.992639in}}{\pgfqpoint{2.271449in}{2.984825in}}%
\pgfpathcurveto{\pgfqpoint{2.263635in}{2.977011in}}{\pgfqpoint{2.259245in}{2.966412in}}{\pgfqpoint{2.259245in}{2.955362in}}%
\pgfpathcurveto{\pgfqpoint{2.259245in}{2.944312in}}{\pgfqpoint{2.263635in}{2.933713in}}{\pgfqpoint{2.271449in}{2.925899in}}%
\pgfpathcurveto{\pgfqpoint{2.279262in}{2.918086in}}{\pgfqpoint{2.289861in}{2.913695in}}{\pgfqpoint{2.300912in}{2.913695in}}%
\pgfpathclose%
\pgfusepath{stroke,fill}%
\end{pgfscope}%
\begin{pgfscope}%
\pgfpathrectangle{\pgfqpoint{0.600000in}{0.600000in}}{\pgfqpoint{3.900000in}{3.900000in}}%
\pgfusepath{clip}%
\pgfsetbuttcap%
\pgfsetroundjoin%
\definecolor{currentfill}{rgb}{0.121569,0.466667,0.705882}%
\pgfsetfillcolor{currentfill}%
\pgfsetlinewidth{1.003750pt}%
\definecolor{currentstroke}{rgb}{0.121569,0.466667,0.705882}%
\pgfsetstrokecolor{currentstroke}%
\pgfsetdash{}{0pt}%
\pgfpathmoveto{\pgfqpoint{2.521032in}{2.618966in}}%
\pgfpathcurveto{\pgfqpoint{2.532083in}{2.618966in}}{\pgfqpoint{2.542682in}{2.623357in}}{\pgfqpoint{2.550495in}{2.631170in}}%
\pgfpathcurveto{\pgfqpoint{2.558309in}{2.638984in}}{\pgfqpoint{2.562699in}{2.649583in}}{\pgfqpoint{2.562699in}{2.660633in}}%
\pgfpathcurveto{\pgfqpoint{2.562699in}{2.671683in}}{\pgfqpoint{2.558309in}{2.682282in}}{\pgfqpoint{2.550495in}{2.690096in}}%
\pgfpathcurveto{\pgfqpoint{2.542682in}{2.697910in}}{\pgfqpoint{2.532083in}{2.702300in}}{\pgfqpoint{2.521032in}{2.702300in}}%
\pgfpathcurveto{\pgfqpoint{2.509982in}{2.702300in}}{\pgfqpoint{2.499383in}{2.697910in}}{\pgfqpoint{2.491570in}{2.690096in}}%
\pgfpathcurveto{\pgfqpoint{2.483756in}{2.682282in}}{\pgfqpoint{2.479366in}{2.671683in}}{\pgfqpoint{2.479366in}{2.660633in}}%
\pgfpathcurveto{\pgfqpoint{2.479366in}{2.649583in}}{\pgfqpoint{2.483756in}{2.638984in}}{\pgfqpoint{2.491570in}{2.631170in}}%
\pgfpathcurveto{\pgfqpoint{2.499383in}{2.623357in}}{\pgfqpoint{2.509982in}{2.618966in}}{\pgfqpoint{2.521032in}{2.618966in}}%
\pgfpathclose%
\pgfusepath{stroke,fill}%
\end{pgfscope}%
\begin{pgfscope}%
\pgfpathrectangle{\pgfqpoint{0.600000in}{0.600000in}}{\pgfqpoint{3.900000in}{3.900000in}}%
\pgfusepath{clip}%
\pgfsetbuttcap%
\pgfsetroundjoin%
\definecolor{currentfill}{rgb}{0.121569,0.466667,0.705882}%
\pgfsetfillcolor{currentfill}%
\pgfsetlinewidth{1.003750pt}%
\definecolor{currentstroke}{rgb}{0.121569,0.466667,0.705882}%
\pgfsetstrokecolor{currentstroke}%
\pgfsetdash{}{0pt}%
\pgfpathmoveto{\pgfqpoint{2.555793in}{2.370070in}}%
\pgfpathcurveto{\pgfqpoint{2.566843in}{2.370070in}}{\pgfqpoint{2.577442in}{2.374460in}}{\pgfqpoint{2.585256in}{2.382274in}}%
\pgfpathcurveto{\pgfqpoint{2.593069in}{2.390087in}}{\pgfqpoint{2.597459in}{2.400686in}}{\pgfqpoint{2.597459in}{2.411736in}}%
\pgfpathcurveto{\pgfqpoint{2.597459in}{2.422787in}}{\pgfqpoint{2.593069in}{2.433386in}}{\pgfqpoint{2.585256in}{2.441199in}}%
\pgfpathcurveto{\pgfqpoint{2.577442in}{2.449013in}}{\pgfqpoint{2.566843in}{2.453403in}}{\pgfqpoint{2.555793in}{2.453403in}}%
\pgfpathcurveto{\pgfqpoint{2.544743in}{2.453403in}}{\pgfqpoint{2.534144in}{2.449013in}}{\pgfqpoint{2.526330in}{2.441199in}}%
\pgfpathcurveto{\pgfqpoint{2.518516in}{2.433386in}}{\pgfqpoint{2.514126in}{2.422787in}}{\pgfqpoint{2.514126in}{2.411736in}}%
\pgfpathcurveto{\pgfqpoint{2.514126in}{2.400686in}}{\pgfqpoint{2.518516in}{2.390087in}}{\pgfqpoint{2.526330in}{2.382274in}}%
\pgfpathcurveto{\pgfqpoint{2.534144in}{2.374460in}}{\pgfqpoint{2.544743in}{2.370070in}}{\pgfqpoint{2.555793in}{2.370070in}}%
\pgfpathclose%
\pgfusepath{stroke,fill}%
\end{pgfscope}%
\begin{pgfscope}%
\pgfpathrectangle{\pgfqpoint{0.600000in}{0.600000in}}{\pgfqpoint{3.900000in}{3.900000in}}%
\pgfusepath{clip}%
\pgfsetbuttcap%
\pgfsetroundjoin%
\definecolor{currentfill}{rgb}{0.121569,0.466667,0.705882}%
\pgfsetfillcolor{currentfill}%
\pgfsetlinewidth{1.003750pt}%
\definecolor{currentstroke}{rgb}{0.121569,0.466667,0.705882}%
\pgfsetstrokecolor{currentstroke}%
\pgfsetdash{}{0pt}%
\pgfpathmoveto{\pgfqpoint{3.001480in}{2.354635in}}%
\pgfpathcurveto{\pgfqpoint{3.012531in}{2.354635in}}{\pgfqpoint{3.023130in}{2.359026in}}{\pgfqpoint{3.030943in}{2.366839in}}%
\pgfpathcurveto{\pgfqpoint{3.038757in}{2.374653in}}{\pgfqpoint{3.043147in}{2.385252in}}{\pgfqpoint{3.043147in}{2.396302in}}%
\pgfpathcurveto{\pgfqpoint{3.043147in}{2.407352in}}{\pgfqpoint{3.038757in}{2.417951in}}{\pgfqpoint{3.030943in}{2.425765in}}%
\pgfpathcurveto{\pgfqpoint{3.023130in}{2.433578in}}{\pgfqpoint{3.012531in}{2.437969in}}{\pgfqpoint{3.001480in}{2.437969in}}%
\pgfpathcurveto{\pgfqpoint{2.990430in}{2.437969in}}{\pgfqpoint{2.979831in}{2.433578in}}{\pgfqpoint{2.972018in}{2.425765in}}%
\pgfpathcurveto{\pgfqpoint{2.964204in}{2.417951in}}{\pgfqpoint{2.959814in}{2.407352in}}{\pgfqpoint{2.959814in}{2.396302in}}%
\pgfpathcurveto{\pgfqpoint{2.959814in}{2.385252in}}{\pgfqpoint{2.964204in}{2.374653in}}{\pgfqpoint{2.972018in}{2.366839in}}%
\pgfpathcurveto{\pgfqpoint{2.979831in}{2.359026in}}{\pgfqpoint{2.990430in}{2.354635in}}{\pgfqpoint{3.001480in}{2.354635in}}%
\pgfpathclose%
\pgfusepath{stroke,fill}%
\end{pgfscope}%
\begin{pgfscope}%
\pgfpathrectangle{\pgfqpoint{0.600000in}{0.600000in}}{\pgfqpoint{3.900000in}{3.900000in}}%
\pgfusepath{clip}%
\pgfsetbuttcap%
\pgfsetroundjoin%
\definecolor{currentfill}{rgb}{0.121569,0.466667,0.705882}%
\pgfsetfillcolor{currentfill}%
\pgfsetlinewidth{1.003750pt}%
\definecolor{currentstroke}{rgb}{0.121569,0.466667,0.705882}%
\pgfsetstrokecolor{currentstroke}%
\pgfsetdash{}{0pt}%
\pgfpathmoveto{\pgfqpoint{1.738138in}{1.448082in}}%
\pgfpathcurveto{\pgfqpoint{1.749189in}{1.448082in}}{\pgfqpoint{1.759788in}{1.452473in}}{\pgfqpoint{1.767601in}{1.460286in}}%
\pgfpathcurveto{\pgfqpoint{1.775415in}{1.468100in}}{\pgfqpoint{1.779805in}{1.478699in}}{\pgfqpoint{1.779805in}{1.489749in}}%
\pgfpathcurveto{\pgfqpoint{1.779805in}{1.500799in}}{\pgfqpoint{1.775415in}{1.511398in}}{\pgfqpoint{1.767601in}{1.519212in}}%
\pgfpathcurveto{\pgfqpoint{1.759788in}{1.527025in}}{\pgfqpoint{1.749189in}{1.531416in}}{\pgfqpoint{1.738138in}{1.531416in}}%
\pgfpathcurveto{\pgfqpoint{1.727088in}{1.531416in}}{\pgfqpoint{1.716489in}{1.527025in}}{\pgfqpoint{1.708676in}{1.519212in}}%
\pgfpathcurveto{\pgfqpoint{1.700862in}{1.511398in}}{\pgfqpoint{1.696472in}{1.500799in}}{\pgfqpoint{1.696472in}{1.489749in}}%
\pgfpathcurveto{\pgfqpoint{1.696472in}{1.478699in}}{\pgfqpoint{1.700862in}{1.468100in}}{\pgfqpoint{1.708676in}{1.460286in}}%
\pgfpathcurveto{\pgfqpoint{1.716489in}{1.452473in}}{\pgfqpoint{1.727088in}{1.448082in}}{\pgfqpoint{1.738138in}{1.448082in}}%
\pgfpathclose%
\pgfusepath{stroke,fill}%
\end{pgfscope}%
\begin{pgfscope}%
\pgfpathrectangle{\pgfqpoint{0.600000in}{0.600000in}}{\pgfqpoint{3.900000in}{3.900000in}}%
\pgfusepath{clip}%
\pgfsetbuttcap%
\pgfsetroundjoin%
\definecolor{currentfill}{rgb}{0.121569,0.466667,0.705882}%
\pgfsetfillcolor{currentfill}%
\pgfsetlinewidth{1.003750pt}%
\definecolor{currentstroke}{rgb}{0.121569,0.466667,0.705882}%
\pgfsetstrokecolor{currentstroke}%
\pgfsetdash{}{0pt}%
\pgfpathmoveto{\pgfqpoint{2.042661in}{1.188180in}}%
\pgfpathcurveto{\pgfqpoint{2.053711in}{1.188180in}}{\pgfqpoint{2.064310in}{1.192570in}}{\pgfqpoint{2.072124in}{1.200384in}}%
\pgfpathcurveto{\pgfqpoint{2.079937in}{1.208197in}}{\pgfqpoint{2.084328in}{1.218796in}}{\pgfqpoint{2.084328in}{1.229847in}}%
\pgfpathcurveto{\pgfqpoint{2.084328in}{1.240897in}}{\pgfqpoint{2.079937in}{1.251496in}}{\pgfqpoint{2.072124in}{1.259309in}}%
\pgfpathcurveto{\pgfqpoint{2.064310in}{1.267123in}}{\pgfqpoint{2.053711in}{1.271513in}}{\pgfqpoint{2.042661in}{1.271513in}}%
\pgfpathcurveto{\pgfqpoint{2.031611in}{1.271513in}}{\pgfqpoint{2.021012in}{1.267123in}}{\pgfqpoint{2.013198in}{1.259309in}}%
\pgfpathcurveto{\pgfqpoint{2.005385in}{1.251496in}}{\pgfqpoint{2.000994in}{1.240897in}}{\pgfqpoint{2.000994in}{1.229847in}}%
\pgfpathcurveto{\pgfqpoint{2.000994in}{1.218796in}}{\pgfqpoint{2.005385in}{1.208197in}}{\pgfqpoint{2.013198in}{1.200384in}}%
\pgfpathcurveto{\pgfqpoint{2.021012in}{1.192570in}}{\pgfqpoint{2.031611in}{1.188180in}}{\pgfqpoint{2.042661in}{1.188180in}}%
\pgfpathclose%
\pgfusepath{stroke,fill}%
\end{pgfscope}%
\begin{pgfscope}%
\pgfpathrectangle{\pgfqpoint{0.600000in}{0.600000in}}{\pgfqpoint{3.900000in}{3.900000in}}%
\pgfusepath{clip}%
\pgfsetbuttcap%
\pgfsetroundjoin%
\definecolor{currentfill}{rgb}{0.121569,0.466667,0.705882}%
\pgfsetfillcolor{currentfill}%
\pgfsetlinewidth{1.003750pt}%
\definecolor{currentstroke}{rgb}{0.121569,0.466667,0.705882}%
\pgfsetstrokecolor{currentstroke}%
\pgfsetdash{}{0pt}%
\pgfpathmoveto{\pgfqpoint{2.889526in}{1.551588in}}%
\pgfpathcurveto{\pgfqpoint{2.900576in}{1.551588in}}{\pgfqpoint{2.911175in}{1.555978in}}{\pgfqpoint{2.918988in}{1.563792in}}%
\pgfpathcurveto{\pgfqpoint{2.926802in}{1.571605in}}{\pgfqpoint{2.931192in}{1.582204in}}{\pgfqpoint{2.931192in}{1.593254in}}%
\pgfpathcurveto{\pgfqpoint{2.931192in}{1.604304in}}{\pgfqpoint{2.926802in}{1.614903in}}{\pgfqpoint{2.918988in}{1.622717in}}%
\pgfpathcurveto{\pgfqpoint{2.911175in}{1.630531in}}{\pgfqpoint{2.900576in}{1.634921in}}{\pgfqpoint{2.889526in}{1.634921in}}%
\pgfpathcurveto{\pgfqpoint{2.878475in}{1.634921in}}{\pgfqpoint{2.867876in}{1.630531in}}{\pgfqpoint{2.860063in}{1.622717in}}%
\pgfpathcurveto{\pgfqpoint{2.852249in}{1.614903in}}{\pgfqpoint{2.847859in}{1.604304in}}{\pgfqpoint{2.847859in}{1.593254in}}%
\pgfpathcurveto{\pgfqpoint{2.847859in}{1.582204in}}{\pgfqpoint{2.852249in}{1.571605in}}{\pgfqpoint{2.860063in}{1.563792in}}%
\pgfpathcurveto{\pgfqpoint{2.867876in}{1.555978in}}{\pgfqpoint{2.878475in}{1.551588in}}{\pgfqpoint{2.889526in}{1.551588in}}%
\pgfpathclose%
\pgfusepath{stroke,fill}%
\end{pgfscope}%
\begin{pgfscope}%
\pgfpathrectangle{\pgfqpoint{0.600000in}{0.600000in}}{\pgfqpoint{3.900000in}{3.900000in}}%
\pgfusepath{clip}%
\pgfsetbuttcap%
\pgfsetroundjoin%
\definecolor{currentfill}{rgb}{0.121569,0.466667,0.705882}%
\pgfsetfillcolor{currentfill}%
\pgfsetlinewidth{1.003750pt}%
\definecolor{currentstroke}{rgb}{0.121569,0.466667,0.705882}%
\pgfsetstrokecolor{currentstroke}%
\pgfsetdash{}{0pt}%
\pgfpathmoveto{\pgfqpoint{2.754131in}{1.617706in}}%
\pgfpathcurveto{\pgfqpoint{2.765181in}{1.617706in}}{\pgfqpoint{2.775780in}{1.622096in}}{\pgfqpoint{2.783594in}{1.629910in}}%
\pgfpathcurveto{\pgfqpoint{2.791407in}{1.637724in}}{\pgfqpoint{2.795798in}{1.648323in}}{\pgfqpoint{2.795798in}{1.659373in}}%
\pgfpathcurveto{\pgfqpoint{2.795798in}{1.670423in}}{\pgfqpoint{2.791407in}{1.681022in}}{\pgfqpoint{2.783594in}{1.688836in}}%
\pgfpathcurveto{\pgfqpoint{2.775780in}{1.696649in}}{\pgfqpoint{2.765181in}{1.701040in}}{\pgfqpoint{2.754131in}{1.701040in}}%
\pgfpathcurveto{\pgfqpoint{2.743081in}{1.701040in}}{\pgfqpoint{2.732482in}{1.696649in}}{\pgfqpoint{2.724668in}{1.688836in}}%
\pgfpathcurveto{\pgfqpoint{2.716854in}{1.681022in}}{\pgfqpoint{2.712464in}{1.670423in}}{\pgfqpoint{2.712464in}{1.659373in}}%
\pgfpathcurveto{\pgfqpoint{2.712464in}{1.648323in}}{\pgfqpoint{2.716854in}{1.637724in}}{\pgfqpoint{2.724668in}{1.629910in}}%
\pgfpathcurveto{\pgfqpoint{2.732482in}{1.622096in}}{\pgfqpoint{2.743081in}{1.617706in}}{\pgfqpoint{2.754131in}{1.617706in}}%
\pgfpathclose%
\pgfusepath{stroke,fill}%
\end{pgfscope}%
\begin{pgfscope}%
\pgfpathrectangle{\pgfqpoint{0.600000in}{0.600000in}}{\pgfqpoint{3.900000in}{3.900000in}}%
\pgfusepath{clip}%
\pgfsetbuttcap%
\pgfsetroundjoin%
\definecolor{currentfill}{rgb}{0.121569,0.466667,0.705882}%
\pgfsetfillcolor{currentfill}%
\pgfsetlinewidth{1.003750pt}%
\definecolor{currentstroke}{rgb}{0.121569,0.466667,0.705882}%
\pgfsetstrokecolor{currentstroke}%
\pgfsetdash{}{0pt}%
\pgfpathmoveto{\pgfqpoint{2.299831in}{2.813646in}}%
\pgfpathcurveto{\pgfqpoint{2.310881in}{2.813646in}}{\pgfqpoint{2.321480in}{2.818036in}}{\pgfqpoint{2.329293in}{2.825850in}}%
\pgfpathcurveto{\pgfqpoint{2.337107in}{2.833664in}}{\pgfqpoint{2.341497in}{2.844263in}}{\pgfqpoint{2.341497in}{2.855313in}}%
\pgfpathcurveto{\pgfqpoint{2.341497in}{2.866363in}}{\pgfqpoint{2.337107in}{2.876962in}}{\pgfqpoint{2.329293in}{2.884776in}}%
\pgfpathcurveto{\pgfqpoint{2.321480in}{2.892589in}}{\pgfqpoint{2.310881in}{2.896980in}}{\pgfqpoint{2.299831in}{2.896980in}}%
\pgfpathcurveto{\pgfqpoint{2.288780in}{2.896980in}}{\pgfqpoint{2.278181in}{2.892589in}}{\pgfqpoint{2.270368in}{2.884776in}}%
\pgfpathcurveto{\pgfqpoint{2.262554in}{2.876962in}}{\pgfqpoint{2.258164in}{2.866363in}}{\pgfqpoint{2.258164in}{2.855313in}}%
\pgfpathcurveto{\pgfqpoint{2.258164in}{2.844263in}}{\pgfqpoint{2.262554in}{2.833664in}}{\pgfqpoint{2.270368in}{2.825850in}}%
\pgfpathcurveto{\pgfqpoint{2.278181in}{2.818036in}}{\pgfqpoint{2.288780in}{2.813646in}}{\pgfqpoint{2.299831in}{2.813646in}}%
\pgfpathclose%
\pgfusepath{stroke,fill}%
\end{pgfscope}%
\begin{pgfscope}%
\pgfpathrectangle{\pgfqpoint{0.600000in}{0.600000in}}{\pgfqpoint{3.900000in}{3.900000in}}%
\pgfusepath{clip}%
\pgfsetbuttcap%
\pgfsetroundjoin%
\definecolor{currentfill}{rgb}{0.121569,0.466667,0.705882}%
\pgfsetfillcolor{currentfill}%
\pgfsetlinewidth{1.003750pt}%
\definecolor{currentstroke}{rgb}{0.121569,0.466667,0.705882}%
\pgfsetstrokecolor{currentstroke}%
\pgfsetdash{}{0pt}%
\pgfpathmoveto{\pgfqpoint{1.681927in}{2.524120in}}%
\pgfpathcurveto{\pgfqpoint{1.692977in}{2.524120in}}{\pgfqpoint{1.703576in}{2.528511in}}{\pgfqpoint{1.711389in}{2.536324in}}%
\pgfpathcurveto{\pgfqpoint{1.719203in}{2.544138in}}{\pgfqpoint{1.723593in}{2.554737in}}{\pgfqpoint{1.723593in}{2.565787in}}%
\pgfpathcurveto{\pgfqpoint{1.723593in}{2.576837in}}{\pgfqpoint{1.719203in}{2.587436in}}{\pgfqpoint{1.711389in}{2.595250in}}%
\pgfpathcurveto{\pgfqpoint{1.703576in}{2.603063in}}{\pgfqpoint{1.692977in}{2.607454in}}{\pgfqpoint{1.681927in}{2.607454in}}%
\pgfpathcurveto{\pgfqpoint{1.670877in}{2.607454in}}{\pgfqpoint{1.660278in}{2.603063in}}{\pgfqpoint{1.652464in}{2.595250in}}%
\pgfpathcurveto{\pgfqpoint{1.644650in}{2.587436in}}{\pgfqpoint{1.640260in}{2.576837in}}{\pgfqpoint{1.640260in}{2.565787in}}%
\pgfpathcurveto{\pgfqpoint{1.640260in}{2.554737in}}{\pgfqpoint{1.644650in}{2.544138in}}{\pgfqpoint{1.652464in}{2.536324in}}%
\pgfpathcurveto{\pgfqpoint{1.660278in}{2.528511in}}{\pgfqpoint{1.670877in}{2.524120in}}{\pgfqpoint{1.681927in}{2.524120in}}%
\pgfpathclose%
\pgfusepath{stroke,fill}%
\end{pgfscope}%
\begin{pgfscope}%
\pgfpathrectangle{\pgfqpoint{0.600000in}{0.600000in}}{\pgfqpoint{3.900000in}{3.900000in}}%
\pgfusepath{clip}%
\pgfsetbuttcap%
\pgfsetroundjoin%
\definecolor{currentfill}{rgb}{0.121569,0.466667,0.705882}%
\pgfsetfillcolor{currentfill}%
\pgfsetlinewidth{1.003750pt}%
\definecolor{currentstroke}{rgb}{0.121569,0.466667,0.705882}%
\pgfsetstrokecolor{currentstroke}%
\pgfsetdash{}{0pt}%
\pgfpathmoveto{\pgfqpoint{2.281337in}{2.303481in}}%
\pgfpathcurveto{\pgfqpoint{2.292387in}{2.303481in}}{\pgfqpoint{2.302986in}{2.307871in}}{\pgfqpoint{2.310800in}{2.315685in}}%
\pgfpathcurveto{\pgfqpoint{2.318614in}{2.323499in}}{\pgfqpoint{2.323004in}{2.334098in}}{\pgfqpoint{2.323004in}{2.345148in}}%
\pgfpathcurveto{\pgfqpoint{2.323004in}{2.356198in}}{\pgfqpoint{2.318614in}{2.366797in}}{\pgfqpoint{2.310800in}{2.374611in}}%
\pgfpathcurveto{\pgfqpoint{2.302986in}{2.382424in}}{\pgfqpoint{2.292387in}{2.386814in}}{\pgfqpoint{2.281337in}{2.386814in}}%
\pgfpathcurveto{\pgfqpoint{2.270287in}{2.386814in}}{\pgfqpoint{2.259688in}{2.382424in}}{\pgfqpoint{2.251874in}{2.374611in}}%
\pgfpathcurveto{\pgfqpoint{2.244061in}{2.366797in}}{\pgfqpoint{2.239670in}{2.356198in}}{\pgfqpoint{2.239670in}{2.345148in}}%
\pgfpathcurveto{\pgfqpoint{2.239670in}{2.334098in}}{\pgfqpoint{2.244061in}{2.323499in}}{\pgfqpoint{2.251874in}{2.315685in}}%
\pgfpathcurveto{\pgfqpoint{2.259688in}{2.307871in}}{\pgfqpoint{2.270287in}{2.303481in}}{\pgfqpoint{2.281337in}{2.303481in}}%
\pgfpathclose%
\pgfusepath{stroke,fill}%
\end{pgfscope}%
\begin{pgfscope}%
\pgfpathrectangle{\pgfqpoint{0.600000in}{0.600000in}}{\pgfqpoint{3.900000in}{3.900000in}}%
\pgfusepath{clip}%
\pgfsetbuttcap%
\pgfsetroundjoin%
\definecolor{currentfill}{rgb}{0.121569,0.466667,0.705882}%
\pgfsetfillcolor{currentfill}%
\pgfsetlinewidth{1.003750pt}%
\definecolor{currentstroke}{rgb}{0.121569,0.466667,0.705882}%
\pgfsetstrokecolor{currentstroke}%
\pgfsetdash{}{0pt}%
\pgfpathmoveto{\pgfqpoint{2.214273in}{3.112578in}}%
\pgfpathcurveto{\pgfqpoint{2.225323in}{3.112578in}}{\pgfqpoint{2.235922in}{3.116968in}}{\pgfqpoint{2.243736in}{3.124782in}}%
\pgfpathcurveto{\pgfqpoint{2.251549in}{3.132595in}}{\pgfqpoint{2.255940in}{3.143194in}}{\pgfqpoint{2.255940in}{3.154244in}}%
\pgfpathcurveto{\pgfqpoint{2.255940in}{3.165294in}}{\pgfqpoint{2.251549in}{3.175894in}}{\pgfqpoint{2.243736in}{3.183707in}}%
\pgfpathcurveto{\pgfqpoint{2.235922in}{3.191521in}}{\pgfqpoint{2.225323in}{3.195911in}}{\pgfqpoint{2.214273in}{3.195911in}}%
\pgfpathcurveto{\pgfqpoint{2.203223in}{3.195911in}}{\pgfqpoint{2.192624in}{3.191521in}}{\pgfqpoint{2.184810in}{3.183707in}}%
\pgfpathcurveto{\pgfqpoint{2.176997in}{3.175894in}}{\pgfqpoint{2.172606in}{3.165294in}}{\pgfqpoint{2.172606in}{3.154244in}}%
\pgfpathcurveto{\pgfqpoint{2.172606in}{3.143194in}}{\pgfqpoint{2.176997in}{3.132595in}}{\pgfqpoint{2.184810in}{3.124782in}}%
\pgfpathcurveto{\pgfqpoint{2.192624in}{3.116968in}}{\pgfqpoint{2.203223in}{3.112578in}}{\pgfqpoint{2.214273in}{3.112578in}}%
\pgfpathclose%
\pgfusepath{stroke,fill}%
\end{pgfscope}%
\begin{pgfscope}%
\pgfpathrectangle{\pgfqpoint{0.600000in}{0.600000in}}{\pgfqpoint{3.900000in}{3.900000in}}%
\pgfusepath{clip}%
\pgfsetbuttcap%
\pgfsetroundjoin%
\definecolor{currentfill}{rgb}{0.121569,0.466667,0.705882}%
\pgfsetfillcolor{currentfill}%
\pgfsetlinewidth{1.003750pt}%
\definecolor{currentstroke}{rgb}{0.121569,0.466667,0.705882}%
\pgfsetstrokecolor{currentstroke}%
\pgfsetdash{}{0pt}%
\pgfpathmoveto{\pgfqpoint{3.399989in}{2.761020in}}%
\pgfpathcurveto{\pgfqpoint{3.411039in}{2.761020in}}{\pgfqpoint{3.421638in}{2.765410in}}{\pgfqpoint{3.429452in}{2.773224in}}%
\pgfpathcurveto{\pgfqpoint{3.437266in}{2.781037in}}{\pgfqpoint{3.441656in}{2.791636in}}{\pgfqpoint{3.441656in}{2.802687in}}%
\pgfpathcurveto{\pgfqpoint{3.441656in}{2.813737in}}{\pgfqpoint{3.437266in}{2.824336in}}{\pgfqpoint{3.429452in}{2.832149in}}%
\pgfpathcurveto{\pgfqpoint{3.421638in}{2.839963in}}{\pgfqpoint{3.411039in}{2.844353in}}{\pgfqpoint{3.399989in}{2.844353in}}%
\pgfpathcurveto{\pgfqpoint{3.388939in}{2.844353in}}{\pgfqpoint{3.378340in}{2.839963in}}{\pgfqpoint{3.370527in}{2.832149in}}%
\pgfpathcurveto{\pgfqpoint{3.362713in}{2.824336in}}{\pgfqpoint{3.358323in}{2.813737in}}{\pgfqpoint{3.358323in}{2.802687in}}%
\pgfpathcurveto{\pgfqpoint{3.358323in}{2.791636in}}{\pgfqpoint{3.362713in}{2.781037in}}{\pgfqpoint{3.370527in}{2.773224in}}%
\pgfpathcurveto{\pgfqpoint{3.378340in}{2.765410in}}{\pgfqpoint{3.388939in}{2.761020in}}{\pgfqpoint{3.399989in}{2.761020in}}%
\pgfpathclose%
\pgfusepath{stroke,fill}%
\end{pgfscope}%
\begin{pgfscope}%
\pgfpathrectangle{\pgfqpoint{0.600000in}{0.600000in}}{\pgfqpoint{3.900000in}{3.900000in}}%
\pgfusepath{clip}%
\pgfsetbuttcap%
\pgfsetroundjoin%
\definecolor{currentfill}{rgb}{0.121569,0.466667,0.705882}%
\pgfsetfillcolor{currentfill}%
\pgfsetlinewidth{1.003750pt}%
\definecolor{currentstroke}{rgb}{0.121569,0.466667,0.705882}%
\pgfsetstrokecolor{currentstroke}%
\pgfsetdash{}{0pt}%
\pgfpathmoveto{\pgfqpoint{2.894359in}{2.433247in}}%
\pgfpathcurveto{\pgfqpoint{2.905409in}{2.433247in}}{\pgfqpoint{2.916008in}{2.437637in}}{\pgfqpoint{2.923821in}{2.445450in}}%
\pgfpathcurveto{\pgfqpoint{2.931635in}{2.453264in}}{\pgfqpoint{2.936025in}{2.463863in}}{\pgfqpoint{2.936025in}{2.474913in}}%
\pgfpathcurveto{\pgfqpoint{2.936025in}{2.485963in}}{\pgfqpoint{2.931635in}{2.496562in}}{\pgfqpoint{2.923821in}{2.504376in}}%
\pgfpathcurveto{\pgfqpoint{2.916008in}{2.512190in}}{\pgfqpoint{2.905409in}{2.516580in}}{\pgfqpoint{2.894359in}{2.516580in}}%
\pgfpathcurveto{\pgfqpoint{2.883308in}{2.516580in}}{\pgfqpoint{2.872709in}{2.512190in}}{\pgfqpoint{2.864896in}{2.504376in}}%
\pgfpathcurveto{\pgfqpoint{2.857082in}{2.496562in}}{\pgfqpoint{2.852692in}{2.485963in}}{\pgfqpoint{2.852692in}{2.474913in}}%
\pgfpathcurveto{\pgfqpoint{2.852692in}{2.463863in}}{\pgfqpoint{2.857082in}{2.453264in}}{\pgfqpoint{2.864896in}{2.445450in}}%
\pgfpathcurveto{\pgfqpoint{2.872709in}{2.437637in}}{\pgfqpoint{2.883308in}{2.433247in}}{\pgfqpoint{2.894359in}{2.433247in}}%
\pgfpathclose%
\pgfusepath{stroke,fill}%
\end{pgfscope}%
\begin{pgfscope}%
\pgfpathrectangle{\pgfqpoint{0.600000in}{0.600000in}}{\pgfqpoint{3.900000in}{3.900000in}}%
\pgfusepath{clip}%
\pgfsetbuttcap%
\pgfsetroundjoin%
\definecolor{currentfill}{rgb}{0.121569,0.466667,0.705882}%
\pgfsetfillcolor{currentfill}%
\pgfsetlinewidth{1.003750pt}%
\definecolor{currentstroke}{rgb}{0.121569,0.466667,0.705882}%
\pgfsetstrokecolor{currentstroke}%
\pgfsetdash{}{0pt}%
\pgfpathmoveto{\pgfqpoint{2.508203in}{2.435857in}}%
\pgfpathcurveto{\pgfqpoint{2.519253in}{2.435857in}}{\pgfqpoint{2.529852in}{2.440247in}}{\pgfqpoint{2.537666in}{2.448061in}}%
\pgfpathcurveto{\pgfqpoint{2.545479in}{2.455874in}}{\pgfqpoint{2.549869in}{2.466473in}}{\pgfqpoint{2.549869in}{2.477524in}}%
\pgfpathcurveto{\pgfqpoint{2.549869in}{2.488574in}}{\pgfqpoint{2.545479in}{2.499173in}}{\pgfqpoint{2.537666in}{2.506986in}}%
\pgfpathcurveto{\pgfqpoint{2.529852in}{2.514800in}}{\pgfqpoint{2.519253in}{2.519190in}}{\pgfqpoint{2.508203in}{2.519190in}}%
\pgfpathcurveto{\pgfqpoint{2.497153in}{2.519190in}}{\pgfqpoint{2.486554in}{2.514800in}}{\pgfqpoint{2.478740in}{2.506986in}}%
\pgfpathcurveto{\pgfqpoint{2.470926in}{2.499173in}}{\pgfqpoint{2.466536in}{2.488574in}}{\pgfqpoint{2.466536in}{2.477524in}}%
\pgfpathcurveto{\pgfqpoint{2.466536in}{2.466473in}}{\pgfqpoint{2.470926in}{2.455874in}}{\pgfqpoint{2.478740in}{2.448061in}}%
\pgfpathcurveto{\pgfqpoint{2.486554in}{2.440247in}}{\pgfqpoint{2.497153in}{2.435857in}}{\pgfqpoint{2.508203in}{2.435857in}}%
\pgfpathclose%
\pgfusepath{stroke,fill}%
\end{pgfscope}%
\begin{pgfscope}%
\pgfpathrectangle{\pgfqpoint{0.600000in}{0.600000in}}{\pgfqpoint{3.900000in}{3.900000in}}%
\pgfusepath{clip}%
\pgfsetbuttcap%
\pgfsetroundjoin%
\definecolor{currentfill}{rgb}{0.121569,0.466667,0.705882}%
\pgfsetfillcolor{currentfill}%
\pgfsetlinewidth{1.003750pt}%
\definecolor{currentstroke}{rgb}{0.121569,0.466667,0.705882}%
\pgfsetstrokecolor{currentstroke}%
\pgfsetdash{}{0pt}%
\pgfpathmoveto{\pgfqpoint{2.568633in}{2.381298in}}%
\pgfpathcurveto{\pgfqpoint{2.579683in}{2.381298in}}{\pgfqpoint{2.590282in}{2.385688in}}{\pgfqpoint{2.598096in}{2.393502in}}%
\pgfpathcurveto{\pgfqpoint{2.605910in}{2.401316in}}{\pgfqpoint{2.610300in}{2.411915in}}{\pgfqpoint{2.610300in}{2.422965in}}%
\pgfpathcurveto{\pgfqpoint{2.610300in}{2.434015in}}{\pgfqpoint{2.605910in}{2.444614in}}{\pgfqpoint{2.598096in}{2.452427in}}%
\pgfpathcurveto{\pgfqpoint{2.590282in}{2.460241in}}{\pgfqpoint{2.579683in}{2.464631in}}{\pgfqpoint{2.568633in}{2.464631in}}%
\pgfpathcurveto{\pgfqpoint{2.557583in}{2.464631in}}{\pgfqpoint{2.546984in}{2.460241in}}{\pgfqpoint{2.539170in}{2.452427in}}%
\pgfpathcurveto{\pgfqpoint{2.531357in}{2.444614in}}{\pgfqpoint{2.526966in}{2.434015in}}{\pgfqpoint{2.526966in}{2.422965in}}%
\pgfpathcurveto{\pgfqpoint{2.526966in}{2.411915in}}{\pgfqpoint{2.531357in}{2.401316in}}{\pgfqpoint{2.539170in}{2.393502in}}%
\pgfpathcurveto{\pgfqpoint{2.546984in}{2.385688in}}{\pgfqpoint{2.557583in}{2.381298in}}{\pgfqpoint{2.568633in}{2.381298in}}%
\pgfpathclose%
\pgfusepath{stroke,fill}%
\end{pgfscope}%
\begin{pgfscope}%
\pgfpathrectangle{\pgfqpoint{0.600000in}{0.600000in}}{\pgfqpoint{3.900000in}{3.900000in}}%
\pgfusepath{clip}%
\pgfsetbuttcap%
\pgfsetroundjoin%
\definecolor{currentfill}{rgb}{0.121569,0.466667,0.705882}%
\pgfsetfillcolor{currentfill}%
\pgfsetlinewidth{1.003750pt}%
\definecolor{currentstroke}{rgb}{0.121569,0.466667,0.705882}%
\pgfsetstrokecolor{currentstroke}%
\pgfsetdash{}{0pt}%
\pgfpathmoveto{\pgfqpoint{1.768501in}{2.783570in}}%
\pgfpathcurveto{\pgfqpoint{1.779551in}{2.783570in}}{\pgfqpoint{1.790150in}{2.787960in}}{\pgfqpoint{1.797964in}{2.795774in}}%
\pgfpathcurveto{\pgfqpoint{1.805777in}{2.803587in}}{\pgfqpoint{1.810167in}{2.814186in}}{\pgfqpoint{1.810167in}{2.825236in}}%
\pgfpathcurveto{\pgfqpoint{1.810167in}{2.836286in}}{\pgfqpoint{1.805777in}{2.846886in}}{\pgfqpoint{1.797964in}{2.854699in}}%
\pgfpathcurveto{\pgfqpoint{1.790150in}{2.862513in}}{\pgfqpoint{1.779551in}{2.866903in}}{\pgfqpoint{1.768501in}{2.866903in}}%
\pgfpathcurveto{\pgfqpoint{1.757451in}{2.866903in}}{\pgfqpoint{1.746852in}{2.862513in}}{\pgfqpoint{1.739038in}{2.854699in}}%
\pgfpathcurveto{\pgfqpoint{1.731224in}{2.846886in}}{\pgfqpoint{1.726834in}{2.836286in}}{\pgfqpoint{1.726834in}{2.825236in}}%
\pgfpathcurveto{\pgfqpoint{1.726834in}{2.814186in}}{\pgfqpoint{1.731224in}{2.803587in}}{\pgfqpoint{1.739038in}{2.795774in}}%
\pgfpathcurveto{\pgfqpoint{1.746852in}{2.787960in}}{\pgfqpoint{1.757451in}{2.783570in}}{\pgfqpoint{1.768501in}{2.783570in}}%
\pgfpathclose%
\pgfusepath{stroke,fill}%
\end{pgfscope}%
\begin{pgfscope}%
\pgfpathrectangle{\pgfqpoint{0.600000in}{0.600000in}}{\pgfqpoint{3.900000in}{3.900000in}}%
\pgfusepath{clip}%
\pgfsetbuttcap%
\pgfsetroundjoin%
\definecolor{currentfill}{rgb}{0.121569,0.466667,0.705882}%
\pgfsetfillcolor{currentfill}%
\pgfsetlinewidth{1.003750pt}%
\definecolor{currentstroke}{rgb}{0.121569,0.466667,0.705882}%
\pgfsetstrokecolor{currentstroke}%
\pgfsetdash{}{0pt}%
\pgfpathmoveto{\pgfqpoint{2.174498in}{3.361719in}}%
\pgfpathcurveto{\pgfqpoint{2.185548in}{3.361719in}}{\pgfqpoint{2.196147in}{3.366109in}}{\pgfqpoint{2.203961in}{3.373922in}}%
\pgfpathcurveto{\pgfqpoint{2.211775in}{3.381736in}}{\pgfqpoint{2.216165in}{3.392335in}}{\pgfqpoint{2.216165in}{3.403385in}}%
\pgfpathcurveto{\pgfqpoint{2.216165in}{3.414435in}}{\pgfqpoint{2.211775in}{3.425034in}}{\pgfqpoint{2.203961in}{3.432848in}}%
\pgfpathcurveto{\pgfqpoint{2.196147in}{3.440662in}}{\pgfqpoint{2.185548in}{3.445052in}}{\pgfqpoint{2.174498in}{3.445052in}}%
\pgfpathcurveto{\pgfqpoint{2.163448in}{3.445052in}}{\pgfqpoint{2.152849in}{3.440662in}}{\pgfqpoint{2.145035in}{3.432848in}}%
\pgfpathcurveto{\pgfqpoint{2.137222in}{3.425034in}}{\pgfqpoint{2.132831in}{3.414435in}}{\pgfqpoint{2.132831in}{3.403385in}}%
\pgfpathcurveto{\pgfqpoint{2.132831in}{3.392335in}}{\pgfqpoint{2.137222in}{3.381736in}}{\pgfqpoint{2.145035in}{3.373922in}}%
\pgfpathcurveto{\pgfqpoint{2.152849in}{3.366109in}}{\pgfqpoint{2.163448in}{3.361719in}}{\pgfqpoint{2.174498in}{3.361719in}}%
\pgfpathclose%
\pgfusepath{stroke,fill}%
\end{pgfscope}%
\begin{pgfscope}%
\pgfpathrectangle{\pgfqpoint{0.600000in}{0.600000in}}{\pgfqpoint{3.900000in}{3.900000in}}%
\pgfusepath{clip}%
\pgfsetbuttcap%
\pgfsetroundjoin%
\definecolor{currentfill}{rgb}{0.121569,0.466667,0.705882}%
\pgfsetfillcolor{currentfill}%
\pgfsetlinewidth{1.003750pt}%
\definecolor{currentstroke}{rgb}{0.121569,0.466667,0.705882}%
\pgfsetstrokecolor{currentstroke}%
\pgfsetdash{}{0pt}%
\pgfpathmoveto{\pgfqpoint{1.569541in}{2.787877in}}%
\pgfpathcurveto{\pgfqpoint{1.580592in}{2.787877in}}{\pgfqpoint{1.591191in}{2.792268in}}{\pgfqpoint{1.599004in}{2.800081in}}%
\pgfpathcurveto{\pgfqpoint{1.606818in}{2.807895in}}{\pgfqpoint{1.611208in}{2.818494in}}{\pgfqpoint{1.611208in}{2.829544in}}%
\pgfpathcurveto{\pgfqpoint{1.611208in}{2.840594in}}{\pgfqpoint{1.606818in}{2.851193in}}{\pgfqpoint{1.599004in}{2.859007in}}%
\pgfpathcurveto{\pgfqpoint{1.591191in}{2.866820in}}{\pgfqpoint{1.580592in}{2.871211in}}{\pgfqpoint{1.569541in}{2.871211in}}%
\pgfpathcurveto{\pgfqpoint{1.558491in}{2.871211in}}{\pgfqpoint{1.547892in}{2.866820in}}{\pgfqpoint{1.540079in}{2.859007in}}%
\pgfpathcurveto{\pgfqpoint{1.532265in}{2.851193in}}{\pgfqpoint{1.527875in}{2.840594in}}{\pgfqpoint{1.527875in}{2.829544in}}%
\pgfpathcurveto{\pgfqpoint{1.527875in}{2.818494in}}{\pgfqpoint{1.532265in}{2.807895in}}{\pgfqpoint{1.540079in}{2.800081in}}%
\pgfpathcurveto{\pgfqpoint{1.547892in}{2.792268in}}{\pgfqpoint{1.558491in}{2.787877in}}{\pgfqpoint{1.569541in}{2.787877in}}%
\pgfpathclose%
\pgfusepath{stroke,fill}%
\end{pgfscope}%
\begin{pgfscope}%
\pgfpathrectangle{\pgfqpoint{0.600000in}{0.600000in}}{\pgfqpoint{3.900000in}{3.900000in}}%
\pgfusepath{clip}%
\pgfsetbuttcap%
\pgfsetroundjoin%
\definecolor{currentfill}{rgb}{0.121569,0.466667,0.705882}%
\pgfsetfillcolor{currentfill}%
\pgfsetlinewidth{1.003750pt}%
\definecolor{currentstroke}{rgb}{0.121569,0.466667,0.705882}%
\pgfsetstrokecolor{currentstroke}%
\pgfsetdash{}{0pt}%
\pgfpathmoveto{\pgfqpoint{3.099169in}{2.549265in}}%
\pgfpathcurveto{\pgfqpoint{3.110219in}{2.549265in}}{\pgfqpoint{3.120818in}{2.553655in}}{\pgfqpoint{3.128632in}{2.561468in}}%
\pgfpathcurveto{\pgfqpoint{3.136445in}{2.569282in}}{\pgfqpoint{3.140836in}{2.579881in}}{\pgfqpoint{3.140836in}{2.590931in}}%
\pgfpathcurveto{\pgfqpoint{3.140836in}{2.601981in}}{\pgfqpoint{3.136445in}{2.612580in}}{\pgfqpoint{3.128632in}{2.620394in}}%
\pgfpathcurveto{\pgfqpoint{3.120818in}{2.628208in}}{\pgfqpoint{3.110219in}{2.632598in}}{\pgfqpoint{3.099169in}{2.632598in}}%
\pgfpathcurveto{\pgfqpoint{3.088119in}{2.632598in}}{\pgfqpoint{3.077520in}{2.628208in}}{\pgfqpoint{3.069706in}{2.620394in}}%
\pgfpathcurveto{\pgfqpoint{3.061892in}{2.612580in}}{\pgfqpoint{3.057502in}{2.601981in}}{\pgfqpoint{3.057502in}{2.590931in}}%
\pgfpathcurveto{\pgfqpoint{3.057502in}{2.579881in}}{\pgfqpoint{3.061892in}{2.569282in}}{\pgfqpoint{3.069706in}{2.561468in}}%
\pgfpathcurveto{\pgfqpoint{3.077520in}{2.553655in}}{\pgfqpoint{3.088119in}{2.549265in}}{\pgfqpoint{3.099169in}{2.549265in}}%
\pgfpathclose%
\pgfusepath{stroke,fill}%
\end{pgfscope}%
\begin{pgfscope}%
\pgfpathrectangle{\pgfqpoint{0.600000in}{0.600000in}}{\pgfqpoint{3.900000in}{3.900000in}}%
\pgfusepath{clip}%
\pgfsetbuttcap%
\pgfsetroundjoin%
\definecolor{currentfill}{rgb}{0.121569,0.466667,0.705882}%
\pgfsetfillcolor{currentfill}%
\pgfsetlinewidth{1.003750pt}%
\definecolor{currentstroke}{rgb}{0.121569,0.466667,0.705882}%
\pgfsetstrokecolor{currentstroke}%
\pgfsetdash{}{0pt}%
\pgfpathmoveto{\pgfqpoint{2.186784in}{2.391430in}}%
\pgfpathcurveto{\pgfqpoint{2.197834in}{2.391430in}}{\pgfqpoint{2.208433in}{2.395821in}}{\pgfqpoint{2.216247in}{2.403634in}}%
\pgfpathcurveto{\pgfqpoint{2.224060in}{2.411448in}}{\pgfqpoint{2.228451in}{2.422047in}}{\pgfqpoint{2.228451in}{2.433097in}}%
\pgfpathcurveto{\pgfqpoint{2.228451in}{2.444147in}}{\pgfqpoint{2.224060in}{2.454746in}}{\pgfqpoint{2.216247in}{2.462560in}}%
\pgfpathcurveto{\pgfqpoint{2.208433in}{2.470373in}}{\pgfqpoint{2.197834in}{2.474764in}}{\pgfqpoint{2.186784in}{2.474764in}}%
\pgfpathcurveto{\pgfqpoint{2.175734in}{2.474764in}}{\pgfqpoint{2.165135in}{2.470373in}}{\pgfqpoint{2.157321in}{2.462560in}}%
\pgfpathcurveto{\pgfqpoint{2.149508in}{2.454746in}}{\pgfqpoint{2.145117in}{2.444147in}}{\pgfqpoint{2.145117in}{2.433097in}}%
\pgfpathcurveto{\pgfqpoint{2.145117in}{2.422047in}}{\pgfqpoint{2.149508in}{2.411448in}}{\pgfqpoint{2.157321in}{2.403634in}}%
\pgfpathcurveto{\pgfqpoint{2.165135in}{2.395821in}}{\pgfqpoint{2.175734in}{2.391430in}}{\pgfqpoint{2.186784in}{2.391430in}}%
\pgfpathclose%
\pgfusepath{stroke,fill}%
\end{pgfscope}%
\begin{pgfscope}%
\pgfpathrectangle{\pgfqpoint{0.600000in}{0.600000in}}{\pgfqpoint{3.900000in}{3.900000in}}%
\pgfusepath{clip}%
\pgfsetbuttcap%
\pgfsetroundjoin%
\definecolor{currentfill}{rgb}{0.121569,0.466667,0.705882}%
\pgfsetfillcolor{currentfill}%
\pgfsetlinewidth{1.003750pt}%
\definecolor{currentstroke}{rgb}{0.121569,0.466667,0.705882}%
\pgfsetstrokecolor{currentstroke}%
\pgfsetdash{}{0pt}%
\pgfpathmoveto{\pgfqpoint{2.991939in}{2.067210in}}%
\pgfpathcurveto{\pgfqpoint{3.002989in}{2.067210in}}{\pgfqpoint{3.013588in}{2.071600in}}{\pgfqpoint{3.021402in}{2.079413in}}%
\pgfpathcurveto{\pgfqpoint{3.029216in}{2.087227in}}{\pgfqpoint{3.033606in}{2.097826in}}{\pgfqpoint{3.033606in}{2.108876in}}%
\pgfpathcurveto{\pgfqpoint{3.033606in}{2.119926in}}{\pgfqpoint{3.029216in}{2.130525in}}{\pgfqpoint{3.021402in}{2.138339in}}%
\pgfpathcurveto{\pgfqpoint{3.013588in}{2.146153in}}{\pgfqpoint{3.002989in}{2.150543in}}{\pgfqpoint{2.991939in}{2.150543in}}%
\pgfpathcurveto{\pgfqpoint{2.980889in}{2.150543in}}{\pgfqpoint{2.970290in}{2.146153in}}{\pgfqpoint{2.962476in}{2.138339in}}%
\pgfpathcurveto{\pgfqpoint{2.954663in}{2.130525in}}{\pgfqpoint{2.950273in}{2.119926in}}{\pgfqpoint{2.950273in}{2.108876in}}%
\pgfpathcurveto{\pgfqpoint{2.950273in}{2.097826in}}{\pgfqpoint{2.954663in}{2.087227in}}{\pgfqpoint{2.962476in}{2.079413in}}%
\pgfpathcurveto{\pgfqpoint{2.970290in}{2.071600in}}{\pgfqpoint{2.980889in}{2.067210in}}{\pgfqpoint{2.991939in}{2.067210in}}%
\pgfpathclose%
\pgfusepath{stroke,fill}%
\end{pgfscope}%
\begin{pgfscope}%
\pgfpathrectangle{\pgfqpoint{0.600000in}{0.600000in}}{\pgfqpoint{3.900000in}{3.900000in}}%
\pgfusepath{clip}%
\pgfsetbuttcap%
\pgfsetroundjoin%
\definecolor{currentfill}{rgb}{0.121569,0.466667,0.705882}%
\pgfsetfillcolor{currentfill}%
\pgfsetlinewidth{1.003750pt}%
\definecolor{currentstroke}{rgb}{0.121569,0.466667,0.705882}%
\pgfsetstrokecolor{currentstroke}%
\pgfsetdash{}{0pt}%
\pgfpathmoveto{\pgfqpoint{2.381538in}{2.361547in}}%
\pgfpathcurveto{\pgfqpoint{2.392588in}{2.361547in}}{\pgfqpoint{2.403187in}{2.365938in}}{\pgfqpoint{2.411001in}{2.373751in}}%
\pgfpathcurveto{\pgfqpoint{2.418814in}{2.381565in}}{\pgfqpoint{2.423205in}{2.392164in}}{\pgfqpoint{2.423205in}{2.403214in}}%
\pgfpathcurveto{\pgfqpoint{2.423205in}{2.414264in}}{\pgfqpoint{2.418814in}{2.424863in}}{\pgfqpoint{2.411001in}{2.432677in}}%
\pgfpathcurveto{\pgfqpoint{2.403187in}{2.440490in}}{\pgfqpoint{2.392588in}{2.444881in}}{\pgfqpoint{2.381538in}{2.444881in}}%
\pgfpathcurveto{\pgfqpoint{2.370488in}{2.444881in}}{\pgfqpoint{2.359889in}{2.440490in}}{\pgfqpoint{2.352075in}{2.432677in}}%
\pgfpathcurveto{\pgfqpoint{2.344261in}{2.424863in}}{\pgfqpoint{2.339871in}{2.414264in}}{\pgfqpoint{2.339871in}{2.403214in}}%
\pgfpathcurveto{\pgfqpoint{2.339871in}{2.392164in}}{\pgfqpoint{2.344261in}{2.381565in}}{\pgfqpoint{2.352075in}{2.373751in}}%
\pgfpathcurveto{\pgfqpoint{2.359889in}{2.365938in}}{\pgfqpoint{2.370488in}{2.361547in}}{\pgfqpoint{2.381538in}{2.361547in}}%
\pgfpathclose%
\pgfusepath{stroke,fill}%
\end{pgfscope}%
\begin{pgfscope}%
\pgfpathrectangle{\pgfqpoint{0.600000in}{0.600000in}}{\pgfqpoint{3.900000in}{3.900000in}}%
\pgfusepath{clip}%
\pgfsetbuttcap%
\pgfsetroundjoin%
\definecolor{currentfill}{rgb}{0.121569,0.466667,0.705882}%
\pgfsetfillcolor{currentfill}%
\pgfsetlinewidth{1.003750pt}%
\definecolor{currentstroke}{rgb}{0.121569,0.466667,0.705882}%
\pgfsetstrokecolor{currentstroke}%
\pgfsetdash{}{0pt}%
\pgfpathmoveto{\pgfqpoint{2.729714in}{2.475078in}}%
\pgfpathcurveto{\pgfqpoint{2.740764in}{2.475078in}}{\pgfqpoint{2.751363in}{2.479468in}}{\pgfqpoint{2.759177in}{2.487282in}}%
\pgfpathcurveto{\pgfqpoint{2.766991in}{2.495095in}}{\pgfqpoint{2.771381in}{2.505695in}}{\pgfqpoint{2.771381in}{2.516745in}}%
\pgfpathcurveto{\pgfqpoint{2.771381in}{2.527795in}}{\pgfqpoint{2.766991in}{2.538394in}}{\pgfqpoint{2.759177in}{2.546207in}}%
\pgfpathcurveto{\pgfqpoint{2.751363in}{2.554021in}}{\pgfqpoint{2.740764in}{2.558411in}}{\pgfqpoint{2.729714in}{2.558411in}}%
\pgfpathcurveto{\pgfqpoint{2.718664in}{2.558411in}}{\pgfqpoint{2.708065in}{2.554021in}}{\pgfqpoint{2.700251in}{2.546207in}}%
\pgfpathcurveto{\pgfqpoint{2.692438in}{2.538394in}}{\pgfqpoint{2.688047in}{2.527795in}}{\pgfqpoint{2.688047in}{2.516745in}}%
\pgfpathcurveto{\pgfqpoint{2.688047in}{2.505695in}}{\pgfqpoint{2.692438in}{2.495095in}}{\pgfqpoint{2.700251in}{2.487282in}}%
\pgfpathcurveto{\pgfqpoint{2.708065in}{2.479468in}}{\pgfqpoint{2.718664in}{2.475078in}}{\pgfqpoint{2.729714in}{2.475078in}}%
\pgfpathclose%
\pgfusepath{stroke,fill}%
\end{pgfscope}%
\begin{pgfscope}%
\pgfpathrectangle{\pgfqpoint{0.600000in}{0.600000in}}{\pgfqpoint{3.900000in}{3.900000in}}%
\pgfusepath{clip}%
\pgfsetbuttcap%
\pgfsetroundjoin%
\definecolor{currentfill}{rgb}{0.121569,0.466667,0.705882}%
\pgfsetfillcolor{currentfill}%
\pgfsetlinewidth{1.003750pt}%
\definecolor{currentstroke}{rgb}{0.121569,0.466667,0.705882}%
\pgfsetstrokecolor{currentstroke}%
\pgfsetdash{}{0pt}%
\pgfpathmoveto{\pgfqpoint{2.143508in}{2.484008in}}%
\pgfpathcurveto{\pgfqpoint{2.154558in}{2.484008in}}{\pgfqpoint{2.165157in}{2.488399in}}{\pgfqpoint{2.172971in}{2.496212in}}%
\pgfpathcurveto{\pgfqpoint{2.180784in}{2.504026in}}{\pgfqpoint{2.185175in}{2.514625in}}{\pgfqpoint{2.185175in}{2.525675in}}%
\pgfpathcurveto{\pgfqpoint{2.185175in}{2.536725in}}{\pgfqpoint{2.180784in}{2.547324in}}{\pgfqpoint{2.172971in}{2.555138in}}%
\pgfpathcurveto{\pgfqpoint{2.165157in}{2.562951in}}{\pgfqpoint{2.154558in}{2.567342in}}{\pgfqpoint{2.143508in}{2.567342in}}%
\pgfpathcurveto{\pgfqpoint{2.132458in}{2.567342in}}{\pgfqpoint{2.121859in}{2.562951in}}{\pgfqpoint{2.114045in}{2.555138in}}%
\pgfpathcurveto{\pgfqpoint{2.106232in}{2.547324in}}{\pgfqpoint{2.101841in}{2.536725in}}{\pgfqpoint{2.101841in}{2.525675in}}%
\pgfpathcurveto{\pgfqpoint{2.101841in}{2.514625in}}{\pgfqpoint{2.106232in}{2.504026in}}{\pgfqpoint{2.114045in}{2.496212in}}%
\pgfpathcurveto{\pgfqpoint{2.121859in}{2.488399in}}{\pgfqpoint{2.132458in}{2.484008in}}{\pgfqpoint{2.143508in}{2.484008in}}%
\pgfpathclose%
\pgfusepath{stroke,fill}%
\end{pgfscope}%
\begin{pgfscope}%
\pgfpathrectangle{\pgfqpoint{0.600000in}{0.600000in}}{\pgfqpoint{3.900000in}{3.900000in}}%
\pgfusepath{clip}%
\pgfsetbuttcap%
\pgfsetroundjoin%
\definecolor{currentfill}{rgb}{0.121569,0.466667,0.705882}%
\pgfsetfillcolor{currentfill}%
\pgfsetlinewidth{1.003750pt}%
\definecolor{currentstroke}{rgb}{0.121569,0.466667,0.705882}%
\pgfsetstrokecolor{currentstroke}%
\pgfsetdash{}{0pt}%
\pgfpathmoveto{\pgfqpoint{3.310460in}{2.041024in}}%
\pgfpathcurveto{\pgfqpoint{3.321510in}{2.041024in}}{\pgfqpoint{3.332109in}{2.045414in}}{\pgfqpoint{3.339923in}{2.053228in}}%
\pgfpathcurveto{\pgfqpoint{3.347737in}{2.061041in}}{\pgfqpoint{3.352127in}{2.071640in}}{\pgfqpoint{3.352127in}{2.082690in}}%
\pgfpathcurveto{\pgfqpoint{3.352127in}{2.093741in}}{\pgfqpoint{3.347737in}{2.104340in}}{\pgfqpoint{3.339923in}{2.112153in}}%
\pgfpathcurveto{\pgfqpoint{3.332109in}{2.119967in}}{\pgfqpoint{3.321510in}{2.124357in}}{\pgfqpoint{3.310460in}{2.124357in}}%
\pgfpathcurveto{\pgfqpoint{3.299410in}{2.124357in}}{\pgfqpoint{3.288811in}{2.119967in}}{\pgfqpoint{3.280998in}{2.112153in}}%
\pgfpathcurveto{\pgfqpoint{3.273184in}{2.104340in}}{\pgfqpoint{3.268794in}{2.093741in}}{\pgfqpoint{3.268794in}{2.082690in}}%
\pgfpathcurveto{\pgfqpoint{3.268794in}{2.071640in}}{\pgfqpoint{3.273184in}{2.061041in}}{\pgfqpoint{3.280998in}{2.053228in}}%
\pgfpathcurveto{\pgfqpoint{3.288811in}{2.045414in}}{\pgfqpoint{3.299410in}{2.041024in}}{\pgfqpoint{3.310460in}{2.041024in}}%
\pgfpathclose%
\pgfusepath{stroke,fill}%
\end{pgfscope}%
\begin{pgfscope}%
\pgfpathrectangle{\pgfqpoint{0.600000in}{0.600000in}}{\pgfqpoint{3.900000in}{3.900000in}}%
\pgfusepath{clip}%
\pgfsetbuttcap%
\pgfsetroundjoin%
\definecolor{currentfill}{rgb}{0.121569,0.466667,0.705882}%
\pgfsetfillcolor{currentfill}%
\pgfsetlinewidth{1.003750pt}%
\definecolor{currentstroke}{rgb}{0.121569,0.466667,0.705882}%
\pgfsetstrokecolor{currentstroke}%
\pgfsetdash{}{0pt}%
\pgfpathmoveto{\pgfqpoint{1.628420in}{2.669441in}}%
\pgfpathcurveto{\pgfqpoint{1.639470in}{2.669441in}}{\pgfqpoint{1.650069in}{2.673831in}}{\pgfqpoint{1.657883in}{2.681644in}}%
\pgfpathcurveto{\pgfqpoint{1.665696in}{2.689458in}}{\pgfqpoint{1.670086in}{2.700057in}}{\pgfqpoint{1.670086in}{2.711107in}}%
\pgfpathcurveto{\pgfqpoint{1.670086in}{2.722157in}}{\pgfqpoint{1.665696in}{2.732756in}}{\pgfqpoint{1.657883in}{2.740570in}}%
\pgfpathcurveto{\pgfqpoint{1.650069in}{2.748384in}}{\pgfqpoint{1.639470in}{2.752774in}}{\pgfqpoint{1.628420in}{2.752774in}}%
\pgfpathcurveto{\pgfqpoint{1.617370in}{2.752774in}}{\pgfqpoint{1.606771in}{2.748384in}}{\pgfqpoint{1.598957in}{2.740570in}}%
\pgfpathcurveto{\pgfqpoint{1.591143in}{2.732756in}}{\pgfqpoint{1.586753in}{2.722157in}}{\pgfqpoint{1.586753in}{2.711107in}}%
\pgfpathcurveto{\pgfqpoint{1.586753in}{2.700057in}}{\pgfqpoint{1.591143in}{2.689458in}}{\pgfqpoint{1.598957in}{2.681644in}}%
\pgfpathcurveto{\pgfqpoint{1.606771in}{2.673831in}}{\pgfqpoint{1.617370in}{2.669441in}}{\pgfqpoint{1.628420in}{2.669441in}}%
\pgfpathclose%
\pgfusepath{stroke,fill}%
\end{pgfscope}%
\begin{pgfscope}%
\pgfpathrectangle{\pgfqpoint{0.600000in}{0.600000in}}{\pgfqpoint{3.900000in}{3.900000in}}%
\pgfusepath{clip}%
\pgfsetbuttcap%
\pgfsetroundjoin%
\definecolor{currentfill}{rgb}{0.121569,0.466667,0.705882}%
\pgfsetfillcolor{currentfill}%
\pgfsetlinewidth{1.003750pt}%
\definecolor{currentstroke}{rgb}{0.121569,0.466667,0.705882}%
\pgfsetstrokecolor{currentstroke}%
\pgfsetdash{}{0pt}%
\pgfpathmoveto{\pgfqpoint{2.639425in}{1.866864in}}%
\pgfpathcurveto{\pgfqpoint{2.650475in}{1.866864in}}{\pgfqpoint{2.661074in}{1.871255in}}{\pgfqpoint{2.668888in}{1.879068in}}%
\pgfpathcurveto{\pgfqpoint{2.676702in}{1.886882in}}{\pgfqpoint{2.681092in}{1.897481in}}{\pgfqpoint{2.681092in}{1.908531in}}%
\pgfpathcurveto{\pgfqpoint{2.681092in}{1.919581in}}{\pgfqpoint{2.676702in}{1.930180in}}{\pgfqpoint{2.668888in}{1.937994in}}%
\pgfpathcurveto{\pgfqpoint{2.661074in}{1.945807in}}{\pgfqpoint{2.650475in}{1.950198in}}{\pgfqpoint{2.639425in}{1.950198in}}%
\pgfpathcurveto{\pgfqpoint{2.628375in}{1.950198in}}{\pgfqpoint{2.617776in}{1.945807in}}{\pgfqpoint{2.609962in}{1.937994in}}%
\pgfpathcurveto{\pgfqpoint{2.602149in}{1.930180in}}{\pgfqpoint{2.597759in}{1.919581in}}{\pgfqpoint{2.597759in}{1.908531in}}%
\pgfpathcurveto{\pgfqpoint{2.597759in}{1.897481in}}{\pgfqpoint{2.602149in}{1.886882in}}{\pgfqpoint{2.609962in}{1.879068in}}%
\pgfpathcurveto{\pgfqpoint{2.617776in}{1.871255in}}{\pgfqpoint{2.628375in}{1.866864in}}{\pgfqpoint{2.639425in}{1.866864in}}%
\pgfpathclose%
\pgfusepath{stroke,fill}%
\end{pgfscope}%
\begin{pgfscope}%
\pgfpathrectangle{\pgfqpoint{0.600000in}{0.600000in}}{\pgfqpoint{3.900000in}{3.900000in}}%
\pgfusepath{clip}%
\pgfsetbuttcap%
\pgfsetroundjoin%
\definecolor{currentfill}{rgb}{0.121569,0.466667,0.705882}%
\pgfsetfillcolor{currentfill}%
\pgfsetlinewidth{1.003750pt}%
\definecolor{currentstroke}{rgb}{0.121569,0.466667,0.705882}%
\pgfsetstrokecolor{currentstroke}%
\pgfsetdash{}{0pt}%
\pgfpathmoveto{\pgfqpoint{1.865523in}{3.485079in}}%
\pgfpathcurveto{\pgfqpoint{1.876573in}{3.485079in}}{\pgfqpoint{1.887172in}{3.489469in}}{\pgfqpoint{1.894986in}{3.497283in}}%
\pgfpathcurveto{\pgfqpoint{1.902799in}{3.505096in}}{\pgfqpoint{1.907190in}{3.515695in}}{\pgfqpoint{1.907190in}{3.526745in}}%
\pgfpathcurveto{\pgfqpoint{1.907190in}{3.537795in}}{\pgfqpoint{1.902799in}{3.548394in}}{\pgfqpoint{1.894986in}{3.556208in}}%
\pgfpathcurveto{\pgfqpoint{1.887172in}{3.564022in}}{\pgfqpoint{1.876573in}{3.568412in}}{\pgfqpoint{1.865523in}{3.568412in}}%
\pgfpathcurveto{\pgfqpoint{1.854473in}{3.568412in}}{\pgfqpoint{1.843874in}{3.564022in}}{\pgfqpoint{1.836060in}{3.556208in}}%
\pgfpathcurveto{\pgfqpoint{1.828247in}{3.548394in}}{\pgfqpoint{1.823856in}{3.537795in}}{\pgfqpoint{1.823856in}{3.526745in}}%
\pgfpathcurveto{\pgfqpoint{1.823856in}{3.515695in}}{\pgfqpoint{1.828247in}{3.505096in}}{\pgfqpoint{1.836060in}{3.497283in}}%
\pgfpathcurveto{\pgfqpoint{1.843874in}{3.489469in}}{\pgfqpoint{1.854473in}{3.485079in}}{\pgfqpoint{1.865523in}{3.485079in}}%
\pgfpathclose%
\pgfusepath{stroke,fill}%
\end{pgfscope}%
\begin{pgfscope}%
\pgfpathrectangle{\pgfqpoint{0.600000in}{0.600000in}}{\pgfqpoint{3.900000in}{3.900000in}}%
\pgfusepath{clip}%
\pgfsetbuttcap%
\pgfsetroundjoin%
\definecolor{currentfill}{rgb}{0.121569,0.466667,0.705882}%
\pgfsetfillcolor{currentfill}%
\pgfsetlinewidth{1.003750pt}%
\definecolor{currentstroke}{rgb}{0.121569,0.466667,0.705882}%
\pgfsetstrokecolor{currentstroke}%
\pgfsetdash{}{0pt}%
\pgfpathmoveto{\pgfqpoint{2.879975in}{2.242860in}}%
\pgfpathcurveto{\pgfqpoint{2.891025in}{2.242860in}}{\pgfqpoint{2.901624in}{2.247251in}}{\pgfqpoint{2.909438in}{2.255064in}}%
\pgfpathcurveto{\pgfqpoint{2.917252in}{2.262878in}}{\pgfqpoint{2.921642in}{2.273477in}}{\pgfqpoint{2.921642in}{2.284527in}}%
\pgfpathcurveto{\pgfqpoint{2.921642in}{2.295577in}}{\pgfqpoint{2.917252in}{2.306176in}}{\pgfqpoint{2.909438in}{2.313990in}}%
\pgfpathcurveto{\pgfqpoint{2.901624in}{2.321804in}}{\pgfqpoint{2.891025in}{2.326194in}}{\pgfqpoint{2.879975in}{2.326194in}}%
\pgfpathcurveto{\pgfqpoint{2.868925in}{2.326194in}}{\pgfqpoint{2.858326in}{2.321804in}}{\pgfqpoint{2.850512in}{2.313990in}}%
\pgfpathcurveto{\pgfqpoint{2.842699in}{2.306176in}}{\pgfqpoint{2.838308in}{2.295577in}}{\pgfqpoint{2.838308in}{2.284527in}}%
\pgfpathcurveto{\pgfqpoint{2.838308in}{2.273477in}}{\pgfqpoint{2.842699in}{2.262878in}}{\pgfqpoint{2.850512in}{2.255064in}}%
\pgfpathcurveto{\pgfqpoint{2.858326in}{2.247251in}}{\pgfqpoint{2.868925in}{2.242860in}}{\pgfqpoint{2.879975in}{2.242860in}}%
\pgfpathclose%
\pgfusepath{stroke,fill}%
\end{pgfscope}%
\begin{pgfscope}%
\pgfpathrectangle{\pgfqpoint{0.600000in}{0.600000in}}{\pgfqpoint{3.900000in}{3.900000in}}%
\pgfusepath{clip}%
\pgfsetbuttcap%
\pgfsetroundjoin%
\definecolor{currentfill}{rgb}{0.121569,0.466667,0.705882}%
\pgfsetfillcolor{currentfill}%
\pgfsetlinewidth{1.003750pt}%
\definecolor{currentstroke}{rgb}{0.121569,0.466667,0.705882}%
\pgfsetstrokecolor{currentstroke}%
\pgfsetdash{}{0pt}%
\pgfpathmoveto{\pgfqpoint{2.630886in}{1.823358in}}%
\pgfpathcurveto{\pgfqpoint{2.641936in}{1.823358in}}{\pgfqpoint{2.652535in}{1.827748in}}{\pgfqpoint{2.660349in}{1.835562in}}%
\pgfpathcurveto{\pgfqpoint{2.668162in}{1.843375in}}{\pgfqpoint{2.672553in}{1.853974in}}{\pgfqpoint{2.672553in}{1.865024in}}%
\pgfpathcurveto{\pgfqpoint{2.672553in}{1.876074in}}{\pgfqpoint{2.668162in}{1.886673in}}{\pgfqpoint{2.660349in}{1.894487in}}%
\pgfpathcurveto{\pgfqpoint{2.652535in}{1.902301in}}{\pgfqpoint{2.641936in}{1.906691in}}{\pgfqpoint{2.630886in}{1.906691in}}%
\pgfpathcurveto{\pgfqpoint{2.619836in}{1.906691in}}{\pgfqpoint{2.609237in}{1.902301in}}{\pgfqpoint{2.601423in}{1.894487in}}%
\pgfpathcurveto{\pgfqpoint{2.593610in}{1.886673in}}{\pgfqpoint{2.589219in}{1.876074in}}{\pgfqpoint{2.589219in}{1.865024in}}%
\pgfpathcurveto{\pgfqpoint{2.589219in}{1.853974in}}{\pgfqpoint{2.593610in}{1.843375in}}{\pgfqpoint{2.601423in}{1.835562in}}%
\pgfpathcurveto{\pgfqpoint{2.609237in}{1.827748in}}{\pgfqpoint{2.619836in}{1.823358in}}{\pgfqpoint{2.630886in}{1.823358in}}%
\pgfpathclose%
\pgfusepath{stroke,fill}%
\end{pgfscope}%
\begin{pgfscope}%
\pgfpathrectangle{\pgfqpoint{0.600000in}{0.600000in}}{\pgfqpoint{3.900000in}{3.900000in}}%
\pgfusepath{clip}%
\pgfsetbuttcap%
\pgfsetroundjoin%
\definecolor{currentfill}{rgb}{0.121569,0.466667,0.705882}%
\pgfsetfillcolor{currentfill}%
\pgfsetlinewidth{1.003750pt}%
\definecolor{currentstroke}{rgb}{0.121569,0.466667,0.705882}%
\pgfsetstrokecolor{currentstroke}%
\pgfsetdash{}{0pt}%
\pgfpathmoveto{\pgfqpoint{2.513164in}{2.215097in}}%
\pgfpathcurveto{\pgfqpoint{2.524214in}{2.215097in}}{\pgfqpoint{2.534813in}{2.219487in}}{\pgfqpoint{2.542626in}{2.227301in}}%
\pgfpathcurveto{\pgfqpoint{2.550440in}{2.235114in}}{\pgfqpoint{2.554830in}{2.245713in}}{\pgfqpoint{2.554830in}{2.256764in}}%
\pgfpathcurveto{\pgfqpoint{2.554830in}{2.267814in}}{\pgfqpoint{2.550440in}{2.278413in}}{\pgfqpoint{2.542626in}{2.286226in}}%
\pgfpathcurveto{\pgfqpoint{2.534813in}{2.294040in}}{\pgfqpoint{2.524214in}{2.298430in}}{\pgfqpoint{2.513164in}{2.298430in}}%
\pgfpathcurveto{\pgfqpoint{2.502113in}{2.298430in}}{\pgfqpoint{2.491514in}{2.294040in}}{\pgfqpoint{2.483701in}{2.286226in}}%
\pgfpathcurveto{\pgfqpoint{2.475887in}{2.278413in}}{\pgfqpoint{2.471497in}{2.267814in}}{\pgfqpoint{2.471497in}{2.256764in}}%
\pgfpathcurveto{\pgfqpoint{2.471497in}{2.245713in}}{\pgfqpoint{2.475887in}{2.235114in}}{\pgfqpoint{2.483701in}{2.227301in}}%
\pgfpathcurveto{\pgfqpoint{2.491514in}{2.219487in}}{\pgfqpoint{2.502113in}{2.215097in}}{\pgfqpoint{2.513164in}{2.215097in}}%
\pgfpathclose%
\pgfusepath{stroke,fill}%
\end{pgfscope}%
\begin{pgfscope}%
\pgfpathrectangle{\pgfqpoint{0.600000in}{0.600000in}}{\pgfqpoint{3.900000in}{3.900000in}}%
\pgfusepath{clip}%
\pgfsetbuttcap%
\pgfsetroundjoin%
\definecolor{currentfill}{rgb}{0.121569,0.466667,0.705882}%
\pgfsetfillcolor{currentfill}%
\pgfsetlinewidth{1.003750pt}%
\definecolor{currentstroke}{rgb}{0.121569,0.466667,0.705882}%
\pgfsetstrokecolor{currentstroke}%
\pgfsetdash{}{0pt}%
\pgfpathmoveto{\pgfqpoint{2.132753in}{1.929431in}}%
\pgfpathcurveto{\pgfqpoint{2.143803in}{1.929431in}}{\pgfqpoint{2.154402in}{1.933821in}}{\pgfqpoint{2.162216in}{1.941635in}}%
\pgfpathcurveto{\pgfqpoint{2.170030in}{1.949448in}}{\pgfqpoint{2.174420in}{1.960047in}}{\pgfqpoint{2.174420in}{1.971097in}}%
\pgfpathcurveto{\pgfqpoint{2.174420in}{1.982148in}}{\pgfqpoint{2.170030in}{1.992747in}}{\pgfqpoint{2.162216in}{2.000560in}}%
\pgfpathcurveto{\pgfqpoint{2.154402in}{2.008374in}}{\pgfqpoint{2.143803in}{2.012764in}}{\pgfqpoint{2.132753in}{2.012764in}}%
\pgfpathcurveto{\pgfqpoint{2.121703in}{2.012764in}}{\pgfqpoint{2.111104in}{2.008374in}}{\pgfqpoint{2.103290in}{2.000560in}}%
\pgfpathcurveto{\pgfqpoint{2.095477in}{1.992747in}}{\pgfqpoint{2.091087in}{1.982148in}}{\pgfqpoint{2.091087in}{1.971097in}}%
\pgfpathcurveto{\pgfqpoint{2.091087in}{1.960047in}}{\pgfqpoint{2.095477in}{1.949448in}}{\pgfqpoint{2.103290in}{1.941635in}}%
\pgfpathcurveto{\pgfqpoint{2.111104in}{1.933821in}}{\pgfqpoint{2.121703in}{1.929431in}}{\pgfqpoint{2.132753in}{1.929431in}}%
\pgfpathclose%
\pgfusepath{stroke,fill}%
\end{pgfscope}%
\begin{pgfscope}%
\pgfpathrectangle{\pgfqpoint{0.600000in}{0.600000in}}{\pgfqpoint{3.900000in}{3.900000in}}%
\pgfusepath{clip}%
\pgfsetbuttcap%
\pgfsetroundjoin%
\definecolor{currentfill}{rgb}{0.121569,0.466667,0.705882}%
\pgfsetfillcolor{currentfill}%
\pgfsetlinewidth{1.003750pt}%
\definecolor{currentstroke}{rgb}{0.121569,0.466667,0.705882}%
\pgfsetstrokecolor{currentstroke}%
\pgfsetdash{}{0pt}%
\pgfpathmoveto{\pgfqpoint{1.643890in}{3.124197in}}%
\pgfpathcurveto{\pgfqpoint{1.654940in}{3.124197in}}{\pgfqpoint{1.665539in}{3.128587in}}{\pgfqpoint{1.673353in}{3.136401in}}%
\pgfpathcurveto{\pgfqpoint{1.681166in}{3.144214in}}{\pgfqpoint{1.685557in}{3.154813in}}{\pgfqpoint{1.685557in}{3.165864in}}%
\pgfpathcurveto{\pgfqpoint{1.685557in}{3.176914in}}{\pgfqpoint{1.681166in}{3.187513in}}{\pgfqpoint{1.673353in}{3.195326in}}%
\pgfpathcurveto{\pgfqpoint{1.665539in}{3.203140in}}{\pgfqpoint{1.654940in}{3.207530in}}{\pgfqpoint{1.643890in}{3.207530in}}%
\pgfpathcurveto{\pgfqpoint{1.632840in}{3.207530in}}{\pgfqpoint{1.622241in}{3.203140in}}{\pgfqpoint{1.614427in}{3.195326in}}%
\pgfpathcurveto{\pgfqpoint{1.606614in}{3.187513in}}{\pgfqpoint{1.602223in}{3.176914in}}{\pgfqpoint{1.602223in}{3.165864in}}%
\pgfpathcurveto{\pgfqpoint{1.602223in}{3.154813in}}{\pgfqpoint{1.606614in}{3.144214in}}{\pgfqpoint{1.614427in}{3.136401in}}%
\pgfpathcurveto{\pgfqpoint{1.622241in}{3.128587in}}{\pgfqpoint{1.632840in}{3.124197in}}{\pgfqpoint{1.643890in}{3.124197in}}%
\pgfpathclose%
\pgfusepath{stroke,fill}%
\end{pgfscope}%
\begin{pgfscope}%
\pgfpathrectangle{\pgfqpoint{0.600000in}{0.600000in}}{\pgfqpoint{3.900000in}{3.900000in}}%
\pgfusepath{clip}%
\pgfsetbuttcap%
\pgfsetroundjoin%
\definecolor{currentfill}{rgb}{0.121569,0.466667,0.705882}%
\pgfsetfillcolor{currentfill}%
\pgfsetlinewidth{1.003750pt}%
\definecolor{currentstroke}{rgb}{0.121569,0.466667,0.705882}%
\pgfsetstrokecolor{currentstroke}%
\pgfsetdash{}{0pt}%
\pgfpathmoveto{\pgfqpoint{2.232851in}{2.066033in}}%
\pgfpathcurveto{\pgfqpoint{2.243901in}{2.066033in}}{\pgfqpoint{2.254500in}{2.070423in}}{\pgfqpoint{2.262314in}{2.078237in}}%
\pgfpathcurveto{\pgfqpoint{2.270127in}{2.086050in}}{\pgfqpoint{2.274517in}{2.096649in}}{\pgfqpoint{2.274517in}{2.107700in}}%
\pgfpathcurveto{\pgfqpoint{2.274517in}{2.118750in}}{\pgfqpoint{2.270127in}{2.129349in}}{\pgfqpoint{2.262314in}{2.137162in}}%
\pgfpathcurveto{\pgfqpoint{2.254500in}{2.144976in}}{\pgfqpoint{2.243901in}{2.149366in}}{\pgfqpoint{2.232851in}{2.149366in}}%
\pgfpathcurveto{\pgfqpoint{2.221801in}{2.149366in}}{\pgfqpoint{2.211202in}{2.144976in}}{\pgfqpoint{2.203388in}{2.137162in}}%
\pgfpathcurveto{\pgfqpoint{2.195574in}{2.129349in}}{\pgfqpoint{2.191184in}{2.118750in}}{\pgfqpoint{2.191184in}{2.107700in}}%
\pgfpathcurveto{\pgfqpoint{2.191184in}{2.096649in}}{\pgfqpoint{2.195574in}{2.086050in}}{\pgfqpoint{2.203388in}{2.078237in}}%
\pgfpathcurveto{\pgfqpoint{2.211202in}{2.070423in}}{\pgfqpoint{2.221801in}{2.066033in}}{\pgfqpoint{2.232851in}{2.066033in}}%
\pgfpathclose%
\pgfusepath{stroke,fill}%
\end{pgfscope}%
\begin{pgfscope}%
\pgfpathrectangle{\pgfqpoint{0.600000in}{0.600000in}}{\pgfqpoint{3.900000in}{3.900000in}}%
\pgfusepath{clip}%
\pgfsetbuttcap%
\pgfsetroundjoin%
\definecolor{currentfill}{rgb}{0.121569,0.466667,0.705882}%
\pgfsetfillcolor{currentfill}%
\pgfsetlinewidth{1.003750pt}%
\definecolor{currentstroke}{rgb}{0.121569,0.466667,0.705882}%
\pgfsetstrokecolor{currentstroke}%
\pgfsetdash{}{0pt}%
\pgfpathmoveto{\pgfqpoint{1.803183in}{2.109250in}}%
\pgfpathcurveto{\pgfqpoint{1.814233in}{2.109250in}}{\pgfqpoint{1.824832in}{2.113640in}}{\pgfqpoint{1.832645in}{2.121454in}}%
\pgfpathcurveto{\pgfqpoint{1.840459in}{2.129267in}}{\pgfqpoint{1.844849in}{2.139866in}}{\pgfqpoint{1.844849in}{2.150917in}}%
\pgfpathcurveto{\pgfqpoint{1.844849in}{2.161967in}}{\pgfqpoint{1.840459in}{2.172566in}}{\pgfqpoint{1.832645in}{2.180379in}}%
\pgfpathcurveto{\pgfqpoint{1.824832in}{2.188193in}}{\pgfqpoint{1.814233in}{2.192583in}}{\pgfqpoint{1.803183in}{2.192583in}}%
\pgfpathcurveto{\pgfqpoint{1.792133in}{2.192583in}}{\pgfqpoint{1.781534in}{2.188193in}}{\pgfqpoint{1.773720in}{2.180379in}}%
\pgfpathcurveto{\pgfqpoint{1.765906in}{2.172566in}}{\pgfqpoint{1.761516in}{2.161967in}}{\pgfqpoint{1.761516in}{2.150917in}}%
\pgfpathcurveto{\pgfqpoint{1.761516in}{2.139866in}}{\pgfqpoint{1.765906in}{2.129267in}}{\pgfqpoint{1.773720in}{2.121454in}}%
\pgfpathcurveto{\pgfqpoint{1.781534in}{2.113640in}}{\pgfqpoint{1.792133in}{2.109250in}}{\pgfqpoint{1.803183in}{2.109250in}}%
\pgfpathclose%
\pgfusepath{stroke,fill}%
\end{pgfscope}%
\begin{pgfscope}%
\pgfpathrectangle{\pgfqpoint{0.600000in}{0.600000in}}{\pgfqpoint{3.900000in}{3.900000in}}%
\pgfusepath{clip}%
\pgfsetbuttcap%
\pgfsetroundjoin%
\definecolor{currentfill}{rgb}{0.121569,0.466667,0.705882}%
\pgfsetfillcolor{currentfill}%
\pgfsetlinewidth{1.003750pt}%
\definecolor{currentstroke}{rgb}{0.121569,0.466667,0.705882}%
\pgfsetstrokecolor{currentstroke}%
\pgfsetdash{}{0pt}%
\pgfpathmoveto{\pgfqpoint{3.345765in}{2.605714in}}%
\pgfpathcurveto{\pgfqpoint{3.356815in}{2.605714in}}{\pgfqpoint{3.367414in}{2.610104in}}{\pgfqpoint{3.375227in}{2.617918in}}%
\pgfpathcurveto{\pgfqpoint{3.383041in}{2.625731in}}{\pgfqpoint{3.387431in}{2.636330in}}{\pgfqpoint{3.387431in}{2.647381in}}%
\pgfpathcurveto{\pgfqpoint{3.387431in}{2.658431in}}{\pgfqpoint{3.383041in}{2.669030in}}{\pgfqpoint{3.375227in}{2.676843in}}%
\pgfpathcurveto{\pgfqpoint{3.367414in}{2.684657in}}{\pgfqpoint{3.356815in}{2.689047in}}{\pgfqpoint{3.345765in}{2.689047in}}%
\pgfpathcurveto{\pgfqpoint{3.334715in}{2.689047in}}{\pgfqpoint{3.324115in}{2.684657in}}{\pgfqpoint{3.316302in}{2.676843in}}%
\pgfpathcurveto{\pgfqpoint{3.308488in}{2.669030in}}{\pgfqpoint{3.304098in}{2.658431in}}{\pgfqpoint{3.304098in}{2.647381in}}%
\pgfpathcurveto{\pgfqpoint{3.304098in}{2.636330in}}{\pgfqpoint{3.308488in}{2.625731in}}{\pgfqpoint{3.316302in}{2.617918in}}%
\pgfpathcurveto{\pgfqpoint{3.324115in}{2.610104in}}{\pgfqpoint{3.334715in}{2.605714in}}{\pgfqpoint{3.345765in}{2.605714in}}%
\pgfpathclose%
\pgfusepath{stroke,fill}%
\end{pgfscope}%
\begin{pgfscope}%
\pgfpathrectangle{\pgfqpoint{0.600000in}{0.600000in}}{\pgfqpoint{3.900000in}{3.900000in}}%
\pgfusepath{clip}%
\pgfsetbuttcap%
\pgfsetroundjoin%
\definecolor{currentfill}{rgb}{0.121569,0.466667,0.705882}%
\pgfsetfillcolor{currentfill}%
\pgfsetlinewidth{1.003750pt}%
\definecolor{currentstroke}{rgb}{0.121569,0.466667,0.705882}%
\pgfsetstrokecolor{currentstroke}%
\pgfsetdash{}{0pt}%
\pgfpathmoveto{\pgfqpoint{2.025780in}{1.414998in}}%
\pgfpathcurveto{\pgfqpoint{2.036831in}{1.414998in}}{\pgfqpoint{2.047430in}{1.419388in}}{\pgfqpoint{2.055243in}{1.427202in}}%
\pgfpathcurveto{\pgfqpoint{2.063057in}{1.435016in}}{\pgfqpoint{2.067447in}{1.445615in}}{\pgfqpoint{2.067447in}{1.456665in}}%
\pgfpathcurveto{\pgfqpoint{2.067447in}{1.467715in}}{\pgfqpoint{2.063057in}{1.478314in}}{\pgfqpoint{2.055243in}{1.486128in}}%
\pgfpathcurveto{\pgfqpoint{2.047430in}{1.493941in}}{\pgfqpoint{2.036831in}{1.498332in}}{\pgfqpoint{2.025780in}{1.498332in}}%
\pgfpathcurveto{\pgfqpoint{2.014730in}{1.498332in}}{\pgfqpoint{2.004131in}{1.493941in}}{\pgfqpoint{1.996318in}{1.486128in}}%
\pgfpathcurveto{\pgfqpoint{1.988504in}{1.478314in}}{\pgfqpoint{1.984114in}{1.467715in}}{\pgfqpoint{1.984114in}{1.456665in}}%
\pgfpathcurveto{\pgfqpoint{1.984114in}{1.445615in}}{\pgfqpoint{1.988504in}{1.435016in}}{\pgfqpoint{1.996318in}{1.427202in}}%
\pgfpathcurveto{\pgfqpoint{2.004131in}{1.419388in}}{\pgfqpoint{2.014730in}{1.414998in}}{\pgfqpoint{2.025780in}{1.414998in}}%
\pgfpathclose%
\pgfusepath{stroke,fill}%
\end{pgfscope}%
\begin{pgfscope}%
\pgfpathrectangle{\pgfqpoint{0.600000in}{0.600000in}}{\pgfqpoint{3.900000in}{3.900000in}}%
\pgfusepath{clip}%
\pgfsetbuttcap%
\pgfsetroundjoin%
\definecolor{currentfill}{rgb}{0.121569,0.466667,0.705882}%
\pgfsetfillcolor{currentfill}%
\pgfsetlinewidth{1.003750pt}%
\definecolor{currentstroke}{rgb}{0.121569,0.466667,0.705882}%
\pgfsetstrokecolor{currentstroke}%
\pgfsetdash{}{0pt}%
\pgfpathmoveto{\pgfqpoint{3.454688in}{3.735145in}}%
\pgfpathcurveto{\pgfqpoint{3.465738in}{3.735145in}}{\pgfqpoint{3.476337in}{3.739535in}}{\pgfqpoint{3.484151in}{3.747349in}}%
\pgfpathcurveto{\pgfqpoint{3.491964in}{3.755162in}}{\pgfqpoint{3.496355in}{3.765761in}}{\pgfqpoint{3.496355in}{3.776811in}}%
\pgfpathcurveto{\pgfqpoint{3.496355in}{3.787862in}}{\pgfqpoint{3.491964in}{3.798461in}}{\pgfqpoint{3.484151in}{3.806274in}}%
\pgfpathcurveto{\pgfqpoint{3.476337in}{3.814088in}}{\pgfqpoint{3.465738in}{3.818478in}}{\pgfqpoint{3.454688in}{3.818478in}}%
\pgfpathcurveto{\pgfqpoint{3.443638in}{3.818478in}}{\pgfqpoint{3.433039in}{3.814088in}}{\pgfqpoint{3.425225in}{3.806274in}}%
\pgfpathcurveto{\pgfqpoint{3.417411in}{3.798461in}}{\pgfqpoint{3.413021in}{3.787862in}}{\pgfqpoint{3.413021in}{3.776811in}}%
\pgfpathcurveto{\pgfqpoint{3.413021in}{3.765761in}}{\pgfqpoint{3.417411in}{3.755162in}}{\pgfqpoint{3.425225in}{3.747349in}}%
\pgfpathcurveto{\pgfqpoint{3.433039in}{3.739535in}}{\pgfqpoint{3.443638in}{3.735145in}}{\pgfqpoint{3.454688in}{3.735145in}}%
\pgfpathclose%
\pgfusepath{stroke,fill}%
\end{pgfscope}%
\begin{pgfscope}%
\pgfpathrectangle{\pgfqpoint{0.600000in}{0.600000in}}{\pgfqpoint{3.900000in}{3.900000in}}%
\pgfusepath{clip}%
\pgfsetbuttcap%
\pgfsetroundjoin%
\definecolor{currentfill}{rgb}{0.121569,0.466667,0.705882}%
\pgfsetfillcolor{currentfill}%
\pgfsetlinewidth{1.003750pt}%
\definecolor{currentstroke}{rgb}{0.121569,0.466667,0.705882}%
\pgfsetstrokecolor{currentstroke}%
\pgfsetdash{}{0pt}%
\pgfpathmoveto{\pgfqpoint{1.486413in}{2.852287in}}%
\pgfpathcurveto{\pgfqpoint{1.497463in}{2.852287in}}{\pgfqpoint{1.508063in}{2.856677in}}{\pgfqpoint{1.515876in}{2.864491in}}%
\pgfpathcurveto{\pgfqpoint{1.523690in}{2.872305in}}{\pgfqpoint{1.528080in}{2.882904in}}{\pgfqpoint{1.528080in}{2.893954in}}%
\pgfpathcurveto{\pgfqpoint{1.528080in}{2.905004in}}{\pgfqpoint{1.523690in}{2.915603in}}{\pgfqpoint{1.515876in}{2.923417in}}%
\pgfpathcurveto{\pgfqpoint{1.508063in}{2.931230in}}{\pgfqpoint{1.497463in}{2.935620in}}{\pgfqpoint{1.486413in}{2.935620in}}%
\pgfpathcurveto{\pgfqpoint{1.475363in}{2.935620in}}{\pgfqpoint{1.464764in}{2.931230in}}{\pgfqpoint{1.456951in}{2.923417in}}%
\pgfpathcurveto{\pgfqpoint{1.449137in}{2.915603in}}{\pgfqpoint{1.444747in}{2.905004in}}{\pgfqpoint{1.444747in}{2.893954in}}%
\pgfpathcurveto{\pgfqpoint{1.444747in}{2.882904in}}{\pgfqpoint{1.449137in}{2.872305in}}{\pgfqpoint{1.456951in}{2.864491in}}%
\pgfpathcurveto{\pgfqpoint{1.464764in}{2.856677in}}{\pgfqpoint{1.475363in}{2.852287in}}{\pgfqpoint{1.486413in}{2.852287in}}%
\pgfpathclose%
\pgfusepath{stroke,fill}%
\end{pgfscope}%
\begin{pgfscope}%
\pgfpathrectangle{\pgfqpoint{0.600000in}{0.600000in}}{\pgfqpoint{3.900000in}{3.900000in}}%
\pgfusepath{clip}%
\pgfsetbuttcap%
\pgfsetroundjoin%
\definecolor{currentfill}{rgb}{0.121569,0.466667,0.705882}%
\pgfsetfillcolor{currentfill}%
\pgfsetlinewidth{1.003750pt}%
\definecolor{currentstroke}{rgb}{0.121569,0.466667,0.705882}%
\pgfsetstrokecolor{currentstroke}%
\pgfsetdash{}{0pt}%
\pgfpathmoveto{\pgfqpoint{2.195490in}{2.067449in}}%
\pgfpathcurveto{\pgfqpoint{2.206540in}{2.067449in}}{\pgfqpoint{2.217139in}{2.071839in}}{\pgfqpoint{2.224953in}{2.079653in}}%
\pgfpathcurveto{\pgfqpoint{2.232767in}{2.087466in}}{\pgfqpoint{2.237157in}{2.098065in}}{\pgfqpoint{2.237157in}{2.109116in}}%
\pgfpathcurveto{\pgfqpoint{2.237157in}{2.120166in}}{\pgfqpoint{2.232767in}{2.130765in}}{\pgfqpoint{2.224953in}{2.138578in}}%
\pgfpathcurveto{\pgfqpoint{2.217139in}{2.146392in}}{\pgfqpoint{2.206540in}{2.150782in}}{\pgfqpoint{2.195490in}{2.150782in}}%
\pgfpathcurveto{\pgfqpoint{2.184440in}{2.150782in}}{\pgfqpoint{2.173841in}{2.146392in}}{\pgfqpoint{2.166027in}{2.138578in}}%
\pgfpathcurveto{\pgfqpoint{2.158214in}{2.130765in}}{\pgfqpoint{2.153824in}{2.120166in}}{\pgfqpoint{2.153824in}{2.109116in}}%
\pgfpathcurveto{\pgfqpoint{2.153824in}{2.098065in}}{\pgfqpoint{2.158214in}{2.087466in}}{\pgfqpoint{2.166027in}{2.079653in}}%
\pgfpathcurveto{\pgfqpoint{2.173841in}{2.071839in}}{\pgfqpoint{2.184440in}{2.067449in}}{\pgfqpoint{2.195490in}{2.067449in}}%
\pgfpathclose%
\pgfusepath{stroke,fill}%
\end{pgfscope}%
\begin{pgfscope}%
\pgfpathrectangle{\pgfqpoint{0.600000in}{0.600000in}}{\pgfqpoint{3.900000in}{3.900000in}}%
\pgfusepath{clip}%
\pgfsetbuttcap%
\pgfsetroundjoin%
\definecolor{currentfill}{rgb}{0.121569,0.466667,0.705882}%
\pgfsetfillcolor{currentfill}%
\pgfsetlinewidth{1.003750pt}%
\definecolor{currentstroke}{rgb}{0.121569,0.466667,0.705882}%
\pgfsetstrokecolor{currentstroke}%
\pgfsetdash{}{0pt}%
\pgfpathmoveto{\pgfqpoint{3.011492in}{2.086642in}}%
\pgfpathcurveto{\pgfqpoint{3.022542in}{2.086642in}}{\pgfqpoint{3.033141in}{2.091032in}}{\pgfqpoint{3.040954in}{2.098846in}}%
\pgfpathcurveto{\pgfqpoint{3.048768in}{2.106659in}}{\pgfqpoint{3.053158in}{2.117258in}}{\pgfqpoint{3.053158in}{2.128308in}}%
\pgfpathcurveto{\pgfqpoint{3.053158in}{2.139359in}}{\pgfqpoint{3.048768in}{2.149958in}}{\pgfqpoint{3.040954in}{2.157771in}}%
\pgfpathcurveto{\pgfqpoint{3.033141in}{2.165585in}}{\pgfqpoint{3.022542in}{2.169975in}}{\pgfqpoint{3.011492in}{2.169975in}}%
\pgfpathcurveto{\pgfqpoint{3.000441in}{2.169975in}}{\pgfqpoint{2.989842in}{2.165585in}}{\pgfqpoint{2.982029in}{2.157771in}}%
\pgfpathcurveto{\pgfqpoint{2.974215in}{2.149958in}}{\pgfqpoint{2.969825in}{2.139359in}}{\pgfqpoint{2.969825in}{2.128308in}}%
\pgfpathcurveto{\pgfqpoint{2.969825in}{2.117258in}}{\pgfqpoint{2.974215in}{2.106659in}}{\pgfqpoint{2.982029in}{2.098846in}}%
\pgfpathcurveto{\pgfqpoint{2.989842in}{2.091032in}}{\pgfqpoint{3.000441in}{2.086642in}}{\pgfqpoint{3.011492in}{2.086642in}}%
\pgfpathclose%
\pgfusepath{stroke,fill}%
\end{pgfscope}%
\begin{pgfscope}%
\pgfpathrectangle{\pgfqpoint{0.600000in}{0.600000in}}{\pgfqpoint{3.900000in}{3.900000in}}%
\pgfusepath{clip}%
\pgfsetbuttcap%
\pgfsetroundjoin%
\definecolor{currentfill}{rgb}{0.121569,0.466667,0.705882}%
\pgfsetfillcolor{currentfill}%
\pgfsetlinewidth{1.003750pt}%
\definecolor{currentstroke}{rgb}{0.121569,0.466667,0.705882}%
\pgfsetstrokecolor{currentstroke}%
\pgfsetdash{}{0pt}%
\pgfpathmoveto{\pgfqpoint{2.697758in}{2.309336in}}%
\pgfpathcurveto{\pgfqpoint{2.708808in}{2.309336in}}{\pgfqpoint{2.719407in}{2.313726in}}{\pgfqpoint{2.727220in}{2.321540in}}%
\pgfpathcurveto{\pgfqpoint{2.735034in}{2.329354in}}{\pgfqpoint{2.739424in}{2.339953in}}{\pgfqpoint{2.739424in}{2.351003in}}%
\pgfpathcurveto{\pgfqpoint{2.739424in}{2.362053in}}{\pgfqpoint{2.735034in}{2.372652in}}{\pgfqpoint{2.727220in}{2.380465in}}%
\pgfpathcurveto{\pgfqpoint{2.719407in}{2.388279in}}{\pgfqpoint{2.708808in}{2.392669in}}{\pgfqpoint{2.697758in}{2.392669in}}%
\pgfpathcurveto{\pgfqpoint{2.686707in}{2.392669in}}{\pgfqpoint{2.676108in}{2.388279in}}{\pgfqpoint{2.668295in}{2.380465in}}%
\pgfpathcurveto{\pgfqpoint{2.660481in}{2.372652in}}{\pgfqpoint{2.656091in}{2.362053in}}{\pgfqpoint{2.656091in}{2.351003in}}%
\pgfpathcurveto{\pgfqpoint{2.656091in}{2.339953in}}{\pgfqpoint{2.660481in}{2.329354in}}{\pgfqpoint{2.668295in}{2.321540in}}%
\pgfpathcurveto{\pgfqpoint{2.676108in}{2.313726in}}{\pgfqpoint{2.686707in}{2.309336in}}{\pgfqpoint{2.697758in}{2.309336in}}%
\pgfpathclose%
\pgfusepath{stroke,fill}%
\end{pgfscope}%
\begin{pgfscope}%
\pgfpathrectangle{\pgfqpoint{0.600000in}{0.600000in}}{\pgfqpoint{3.900000in}{3.900000in}}%
\pgfusepath{clip}%
\pgfsetbuttcap%
\pgfsetroundjoin%
\definecolor{currentfill}{rgb}{0.121569,0.466667,0.705882}%
\pgfsetfillcolor{currentfill}%
\pgfsetlinewidth{1.003750pt}%
\definecolor{currentstroke}{rgb}{0.121569,0.466667,0.705882}%
\pgfsetstrokecolor{currentstroke}%
\pgfsetdash{}{0pt}%
\pgfpathmoveto{\pgfqpoint{2.275676in}{1.913504in}}%
\pgfpathcurveto{\pgfqpoint{2.286726in}{1.913504in}}{\pgfqpoint{2.297325in}{1.917894in}}{\pgfqpoint{2.305139in}{1.925708in}}%
\pgfpathcurveto{\pgfqpoint{2.312952in}{1.933522in}}{\pgfqpoint{2.317343in}{1.944121in}}{\pgfqpoint{2.317343in}{1.955171in}}%
\pgfpathcurveto{\pgfqpoint{2.317343in}{1.966221in}}{\pgfqpoint{2.312952in}{1.976820in}}{\pgfqpoint{2.305139in}{1.984633in}}%
\pgfpathcurveto{\pgfqpoint{2.297325in}{1.992447in}}{\pgfqpoint{2.286726in}{1.996837in}}{\pgfqpoint{2.275676in}{1.996837in}}%
\pgfpathcurveto{\pgfqpoint{2.264626in}{1.996837in}}{\pgfqpoint{2.254027in}{1.992447in}}{\pgfqpoint{2.246213in}{1.984633in}}%
\pgfpathcurveto{\pgfqpoint{2.238399in}{1.976820in}}{\pgfqpoint{2.234009in}{1.966221in}}{\pgfqpoint{2.234009in}{1.955171in}}%
\pgfpathcurveto{\pgfqpoint{2.234009in}{1.944121in}}{\pgfqpoint{2.238399in}{1.933522in}}{\pgfqpoint{2.246213in}{1.925708in}}%
\pgfpathcurveto{\pgfqpoint{2.254027in}{1.917894in}}{\pgfqpoint{2.264626in}{1.913504in}}{\pgfqpoint{2.275676in}{1.913504in}}%
\pgfpathclose%
\pgfusepath{stroke,fill}%
\end{pgfscope}%
\begin{pgfscope}%
\pgfpathrectangle{\pgfqpoint{0.600000in}{0.600000in}}{\pgfqpoint{3.900000in}{3.900000in}}%
\pgfusepath{clip}%
\pgfsetbuttcap%
\pgfsetroundjoin%
\definecolor{currentfill}{rgb}{0.121569,0.466667,0.705882}%
\pgfsetfillcolor{currentfill}%
\pgfsetlinewidth{1.003750pt}%
\definecolor{currentstroke}{rgb}{0.121569,0.466667,0.705882}%
\pgfsetstrokecolor{currentstroke}%
\pgfsetdash{}{0pt}%
\pgfpathmoveto{\pgfqpoint{2.815364in}{2.918073in}}%
\pgfpathcurveto{\pgfqpoint{2.826414in}{2.918073in}}{\pgfqpoint{2.837013in}{2.922463in}}{\pgfqpoint{2.844826in}{2.930277in}}%
\pgfpathcurveto{\pgfqpoint{2.852640in}{2.938091in}}{\pgfqpoint{2.857030in}{2.948690in}}{\pgfqpoint{2.857030in}{2.959740in}}%
\pgfpathcurveto{\pgfqpoint{2.857030in}{2.970790in}}{\pgfqpoint{2.852640in}{2.981389in}}{\pgfqpoint{2.844826in}{2.989203in}}%
\pgfpathcurveto{\pgfqpoint{2.837013in}{2.997016in}}{\pgfqpoint{2.826414in}{3.001406in}}{\pgfqpoint{2.815364in}{3.001406in}}%
\pgfpathcurveto{\pgfqpoint{2.804313in}{3.001406in}}{\pgfqpoint{2.793714in}{2.997016in}}{\pgfqpoint{2.785901in}{2.989203in}}%
\pgfpathcurveto{\pgfqpoint{2.778087in}{2.981389in}}{\pgfqpoint{2.773697in}{2.970790in}}{\pgfqpoint{2.773697in}{2.959740in}}%
\pgfpathcurveto{\pgfqpoint{2.773697in}{2.948690in}}{\pgfqpoint{2.778087in}{2.938091in}}{\pgfqpoint{2.785901in}{2.930277in}}%
\pgfpathcurveto{\pgfqpoint{2.793714in}{2.922463in}}{\pgfqpoint{2.804313in}{2.918073in}}{\pgfqpoint{2.815364in}{2.918073in}}%
\pgfpathclose%
\pgfusepath{stroke,fill}%
\end{pgfscope}%
\begin{pgfscope}%
\pgfpathrectangle{\pgfqpoint{0.600000in}{0.600000in}}{\pgfqpoint{3.900000in}{3.900000in}}%
\pgfusepath{clip}%
\pgfsetbuttcap%
\pgfsetroundjoin%
\definecolor{currentfill}{rgb}{0.121569,0.466667,0.705882}%
\pgfsetfillcolor{currentfill}%
\pgfsetlinewidth{1.003750pt}%
\definecolor{currentstroke}{rgb}{0.121569,0.466667,0.705882}%
\pgfsetstrokecolor{currentstroke}%
\pgfsetdash{}{0pt}%
\pgfpathmoveto{\pgfqpoint{3.424978in}{3.018478in}}%
\pgfpathcurveto{\pgfqpoint{3.436028in}{3.018478in}}{\pgfqpoint{3.446627in}{3.022868in}}{\pgfqpoint{3.454440in}{3.030682in}}%
\pgfpathcurveto{\pgfqpoint{3.462254in}{3.038495in}}{\pgfqpoint{3.466644in}{3.049094in}}{\pgfqpoint{3.466644in}{3.060144in}}%
\pgfpathcurveto{\pgfqpoint{3.466644in}{3.071194in}}{\pgfqpoint{3.462254in}{3.081793in}}{\pgfqpoint{3.454440in}{3.089607in}}%
\pgfpathcurveto{\pgfqpoint{3.446627in}{3.097421in}}{\pgfqpoint{3.436028in}{3.101811in}}{\pgfqpoint{3.424978in}{3.101811in}}%
\pgfpathcurveto{\pgfqpoint{3.413927in}{3.101811in}}{\pgfqpoint{3.403328in}{3.097421in}}{\pgfqpoint{3.395515in}{3.089607in}}%
\pgfpathcurveto{\pgfqpoint{3.387701in}{3.081793in}}{\pgfqpoint{3.383311in}{3.071194in}}{\pgfqpoint{3.383311in}{3.060144in}}%
\pgfpathcurveto{\pgfqpoint{3.383311in}{3.049094in}}{\pgfqpoint{3.387701in}{3.038495in}}{\pgfqpoint{3.395515in}{3.030682in}}%
\pgfpathcurveto{\pgfqpoint{3.403328in}{3.022868in}}{\pgfqpoint{3.413927in}{3.018478in}}{\pgfqpoint{3.424978in}{3.018478in}}%
\pgfpathclose%
\pgfusepath{stroke,fill}%
\end{pgfscope}%
\begin{pgfscope}%
\pgfpathrectangle{\pgfqpoint{0.600000in}{0.600000in}}{\pgfqpoint{3.900000in}{3.900000in}}%
\pgfusepath{clip}%
\pgfsetbuttcap%
\pgfsetroundjoin%
\definecolor{currentfill}{rgb}{0.121569,0.466667,0.705882}%
\pgfsetfillcolor{currentfill}%
\pgfsetlinewidth{1.003750pt}%
\definecolor{currentstroke}{rgb}{0.121569,0.466667,0.705882}%
\pgfsetstrokecolor{currentstroke}%
\pgfsetdash{}{0pt}%
\pgfpathmoveto{\pgfqpoint{3.232996in}{2.324365in}}%
\pgfpathcurveto{\pgfqpoint{3.244046in}{2.324365in}}{\pgfqpoint{3.254645in}{2.328755in}}{\pgfqpoint{3.262459in}{2.336569in}}%
\pgfpathcurveto{\pgfqpoint{3.270272in}{2.344383in}}{\pgfqpoint{3.274663in}{2.354982in}}{\pgfqpoint{3.274663in}{2.366032in}}%
\pgfpathcurveto{\pgfqpoint{3.274663in}{2.377082in}}{\pgfqpoint{3.270272in}{2.387681in}}{\pgfqpoint{3.262459in}{2.395494in}}%
\pgfpathcurveto{\pgfqpoint{3.254645in}{2.403308in}}{\pgfqpoint{3.244046in}{2.407698in}}{\pgfqpoint{3.232996in}{2.407698in}}%
\pgfpathcurveto{\pgfqpoint{3.221946in}{2.407698in}}{\pgfqpoint{3.211347in}{2.403308in}}{\pgfqpoint{3.203533in}{2.395494in}}%
\pgfpathcurveto{\pgfqpoint{3.195720in}{2.387681in}}{\pgfqpoint{3.191329in}{2.377082in}}{\pgfqpoint{3.191329in}{2.366032in}}%
\pgfpathcurveto{\pgfqpoint{3.191329in}{2.354982in}}{\pgfqpoint{3.195720in}{2.344383in}}{\pgfqpoint{3.203533in}{2.336569in}}%
\pgfpathcurveto{\pgfqpoint{3.211347in}{2.328755in}}{\pgfqpoint{3.221946in}{2.324365in}}{\pgfqpoint{3.232996in}{2.324365in}}%
\pgfpathclose%
\pgfusepath{stroke,fill}%
\end{pgfscope}%
\begin{pgfscope}%
\pgfpathrectangle{\pgfqpoint{0.600000in}{0.600000in}}{\pgfqpoint{3.900000in}{3.900000in}}%
\pgfusepath{clip}%
\pgfsetbuttcap%
\pgfsetroundjoin%
\definecolor{currentfill}{rgb}{0.121569,0.466667,0.705882}%
\pgfsetfillcolor{currentfill}%
\pgfsetlinewidth{1.003750pt}%
\definecolor{currentstroke}{rgb}{0.121569,0.466667,0.705882}%
\pgfsetstrokecolor{currentstroke}%
\pgfsetdash{}{0pt}%
\pgfpathmoveto{\pgfqpoint{2.553394in}{2.528922in}}%
\pgfpathcurveto{\pgfqpoint{2.564444in}{2.528922in}}{\pgfqpoint{2.575043in}{2.533313in}}{\pgfqpoint{2.582856in}{2.541126in}}%
\pgfpathcurveto{\pgfqpoint{2.590670in}{2.548940in}}{\pgfqpoint{2.595060in}{2.559539in}}{\pgfqpoint{2.595060in}{2.570589in}}%
\pgfpathcurveto{\pgfqpoint{2.595060in}{2.581639in}}{\pgfqpoint{2.590670in}{2.592238in}}{\pgfqpoint{2.582856in}{2.600052in}}%
\pgfpathcurveto{\pgfqpoint{2.575043in}{2.607866in}}{\pgfqpoint{2.564444in}{2.612256in}}{\pgfqpoint{2.553394in}{2.612256in}}%
\pgfpathcurveto{\pgfqpoint{2.542343in}{2.612256in}}{\pgfqpoint{2.531744in}{2.607866in}}{\pgfqpoint{2.523931in}{2.600052in}}%
\pgfpathcurveto{\pgfqpoint{2.516117in}{2.592238in}}{\pgfqpoint{2.511727in}{2.581639in}}{\pgfqpoint{2.511727in}{2.570589in}}%
\pgfpathcurveto{\pgfqpoint{2.511727in}{2.559539in}}{\pgfqpoint{2.516117in}{2.548940in}}{\pgfqpoint{2.523931in}{2.541126in}}%
\pgfpathcurveto{\pgfqpoint{2.531744in}{2.533313in}}{\pgfqpoint{2.542343in}{2.528922in}}{\pgfqpoint{2.553394in}{2.528922in}}%
\pgfpathclose%
\pgfusepath{stroke,fill}%
\end{pgfscope}%
\begin{pgfscope}%
\pgfpathrectangle{\pgfqpoint{0.600000in}{0.600000in}}{\pgfqpoint{3.900000in}{3.900000in}}%
\pgfusepath{clip}%
\pgfsetbuttcap%
\pgfsetroundjoin%
\definecolor{currentfill}{rgb}{0.121569,0.466667,0.705882}%
\pgfsetfillcolor{currentfill}%
\pgfsetlinewidth{1.003750pt}%
\definecolor{currentstroke}{rgb}{0.121569,0.466667,0.705882}%
\pgfsetstrokecolor{currentstroke}%
\pgfsetdash{}{0pt}%
\pgfpathmoveto{\pgfqpoint{2.244190in}{2.601557in}}%
\pgfpathcurveto{\pgfqpoint{2.255241in}{2.601557in}}{\pgfqpoint{2.265840in}{2.605947in}}{\pgfqpoint{2.273653in}{2.613761in}}%
\pgfpathcurveto{\pgfqpoint{2.281467in}{2.621574in}}{\pgfqpoint{2.285857in}{2.632173in}}{\pgfqpoint{2.285857in}{2.643223in}}%
\pgfpathcurveto{\pgfqpoint{2.285857in}{2.654273in}}{\pgfqpoint{2.281467in}{2.664872in}}{\pgfqpoint{2.273653in}{2.672686in}}%
\pgfpathcurveto{\pgfqpoint{2.265840in}{2.680500in}}{\pgfqpoint{2.255241in}{2.684890in}}{\pgfqpoint{2.244190in}{2.684890in}}%
\pgfpathcurveto{\pgfqpoint{2.233140in}{2.684890in}}{\pgfqpoint{2.222541in}{2.680500in}}{\pgfqpoint{2.214728in}{2.672686in}}%
\pgfpathcurveto{\pgfqpoint{2.206914in}{2.664872in}}{\pgfqpoint{2.202524in}{2.654273in}}{\pgfqpoint{2.202524in}{2.643223in}}%
\pgfpathcurveto{\pgfqpoint{2.202524in}{2.632173in}}{\pgfqpoint{2.206914in}{2.621574in}}{\pgfqpoint{2.214728in}{2.613761in}}%
\pgfpathcurveto{\pgfqpoint{2.222541in}{2.605947in}}{\pgfqpoint{2.233140in}{2.601557in}}{\pgfqpoint{2.244190in}{2.601557in}}%
\pgfpathclose%
\pgfusepath{stroke,fill}%
\end{pgfscope}%
\begin{pgfscope}%
\pgfpathrectangle{\pgfqpoint{0.600000in}{0.600000in}}{\pgfqpoint{3.900000in}{3.900000in}}%
\pgfusepath{clip}%
\pgfsetbuttcap%
\pgfsetroundjoin%
\definecolor{currentfill}{rgb}{0.121569,0.466667,0.705882}%
\pgfsetfillcolor{currentfill}%
\pgfsetlinewidth{1.003750pt}%
\definecolor{currentstroke}{rgb}{0.121569,0.466667,0.705882}%
\pgfsetstrokecolor{currentstroke}%
\pgfsetdash{}{0pt}%
\pgfpathmoveto{\pgfqpoint{2.281332in}{3.408343in}}%
\pgfpathcurveto{\pgfqpoint{2.292382in}{3.408343in}}{\pgfqpoint{2.302981in}{3.412733in}}{\pgfqpoint{2.310795in}{3.420547in}}%
\pgfpathcurveto{\pgfqpoint{2.318608in}{3.428361in}}{\pgfqpoint{2.322999in}{3.438960in}}{\pgfqpoint{2.322999in}{3.450010in}}%
\pgfpathcurveto{\pgfqpoint{2.322999in}{3.461060in}}{\pgfqpoint{2.318608in}{3.471659in}}{\pgfqpoint{2.310795in}{3.479473in}}%
\pgfpathcurveto{\pgfqpoint{2.302981in}{3.487286in}}{\pgfqpoint{2.292382in}{3.491677in}}{\pgfqpoint{2.281332in}{3.491677in}}%
\pgfpathcurveto{\pgfqpoint{2.270282in}{3.491677in}}{\pgfqpoint{2.259683in}{3.487286in}}{\pgfqpoint{2.251869in}{3.479473in}}%
\pgfpathcurveto{\pgfqpoint{2.244056in}{3.471659in}}{\pgfqpoint{2.239665in}{3.461060in}}{\pgfqpoint{2.239665in}{3.450010in}}%
\pgfpathcurveto{\pgfqpoint{2.239665in}{3.438960in}}{\pgfqpoint{2.244056in}{3.428361in}}{\pgfqpoint{2.251869in}{3.420547in}}%
\pgfpathcurveto{\pgfqpoint{2.259683in}{3.412733in}}{\pgfqpoint{2.270282in}{3.408343in}}{\pgfqpoint{2.281332in}{3.408343in}}%
\pgfpathclose%
\pgfusepath{stroke,fill}%
\end{pgfscope}%
\begin{pgfscope}%
\pgfpathrectangle{\pgfqpoint{0.600000in}{0.600000in}}{\pgfqpoint{3.900000in}{3.900000in}}%
\pgfusepath{clip}%
\pgfsetbuttcap%
\pgfsetroundjoin%
\definecolor{currentfill}{rgb}{0.121569,0.466667,0.705882}%
\pgfsetfillcolor{currentfill}%
\pgfsetlinewidth{1.003750pt}%
\definecolor{currentstroke}{rgb}{0.121569,0.466667,0.705882}%
\pgfsetstrokecolor{currentstroke}%
\pgfsetdash{}{0pt}%
\pgfpathmoveto{\pgfqpoint{2.949780in}{2.013283in}}%
\pgfpathcurveto{\pgfqpoint{2.960830in}{2.013283in}}{\pgfqpoint{2.971429in}{2.017673in}}{\pgfqpoint{2.979243in}{2.025486in}}%
\pgfpathcurveto{\pgfqpoint{2.987056in}{2.033300in}}{\pgfqpoint{2.991446in}{2.043899in}}{\pgfqpoint{2.991446in}{2.054949in}}%
\pgfpathcurveto{\pgfqpoint{2.991446in}{2.065999in}}{\pgfqpoint{2.987056in}{2.076598in}}{\pgfqpoint{2.979243in}{2.084412in}}%
\pgfpathcurveto{\pgfqpoint{2.971429in}{2.092226in}}{\pgfqpoint{2.960830in}{2.096616in}}{\pgfqpoint{2.949780in}{2.096616in}}%
\pgfpathcurveto{\pgfqpoint{2.938730in}{2.096616in}}{\pgfqpoint{2.928131in}{2.092226in}}{\pgfqpoint{2.920317in}{2.084412in}}%
\pgfpathcurveto{\pgfqpoint{2.912503in}{2.076598in}}{\pgfqpoint{2.908113in}{2.065999in}}{\pgfqpoint{2.908113in}{2.054949in}}%
\pgfpathcurveto{\pgfqpoint{2.908113in}{2.043899in}}{\pgfqpoint{2.912503in}{2.033300in}}{\pgfqpoint{2.920317in}{2.025486in}}%
\pgfpathcurveto{\pgfqpoint{2.928131in}{2.017673in}}{\pgfqpoint{2.938730in}{2.013283in}}{\pgfqpoint{2.949780in}{2.013283in}}%
\pgfpathclose%
\pgfusepath{stroke,fill}%
\end{pgfscope}%
\begin{pgfscope}%
\pgfpathrectangle{\pgfqpoint{0.600000in}{0.600000in}}{\pgfqpoint{3.900000in}{3.900000in}}%
\pgfusepath{clip}%
\pgfsetbuttcap%
\pgfsetroundjoin%
\definecolor{currentfill}{rgb}{0.121569,0.466667,0.705882}%
\pgfsetfillcolor{currentfill}%
\pgfsetlinewidth{1.003750pt}%
\definecolor{currentstroke}{rgb}{0.121569,0.466667,0.705882}%
\pgfsetstrokecolor{currentstroke}%
\pgfsetdash{}{0pt}%
\pgfpathmoveto{\pgfqpoint{3.335255in}{2.715902in}}%
\pgfpathcurveto{\pgfqpoint{3.346305in}{2.715902in}}{\pgfqpoint{3.356904in}{2.720292in}}{\pgfqpoint{3.364717in}{2.728106in}}%
\pgfpathcurveto{\pgfqpoint{3.372531in}{2.735919in}}{\pgfqpoint{3.376921in}{2.746518in}}{\pgfqpoint{3.376921in}{2.757568in}}%
\pgfpathcurveto{\pgfqpoint{3.376921in}{2.768618in}}{\pgfqpoint{3.372531in}{2.779218in}}{\pgfqpoint{3.364717in}{2.787031in}}%
\pgfpathcurveto{\pgfqpoint{3.356904in}{2.794845in}}{\pgfqpoint{3.346305in}{2.799235in}}{\pgfqpoint{3.335255in}{2.799235in}}%
\pgfpathcurveto{\pgfqpoint{3.324204in}{2.799235in}}{\pgfqpoint{3.313605in}{2.794845in}}{\pgfqpoint{3.305792in}{2.787031in}}%
\pgfpathcurveto{\pgfqpoint{3.297978in}{2.779218in}}{\pgfqpoint{3.293588in}{2.768618in}}{\pgfqpoint{3.293588in}{2.757568in}}%
\pgfpathcurveto{\pgfqpoint{3.293588in}{2.746518in}}{\pgfqpoint{3.297978in}{2.735919in}}{\pgfqpoint{3.305792in}{2.728106in}}%
\pgfpathcurveto{\pgfqpoint{3.313605in}{2.720292in}}{\pgfqpoint{3.324204in}{2.715902in}}{\pgfqpoint{3.335255in}{2.715902in}}%
\pgfpathclose%
\pgfusepath{stroke,fill}%
\end{pgfscope}%
\begin{pgfscope}%
\pgfpathrectangle{\pgfqpoint{0.600000in}{0.600000in}}{\pgfqpoint{3.900000in}{3.900000in}}%
\pgfusepath{clip}%
\pgfsetbuttcap%
\pgfsetroundjoin%
\definecolor{currentfill}{rgb}{0.121569,0.466667,0.705882}%
\pgfsetfillcolor{currentfill}%
\pgfsetlinewidth{1.003750pt}%
\definecolor{currentstroke}{rgb}{0.121569,0.466667,0.705882}%
\pgfsetstrokecolor{currentstroke}%
\pgfsetdash{}{0pt}%
\pgfpathmoveto{\pgfqpoint{2.187677in}{2.998585in}}%
\pgfpathcurveto{\pgfqpoint{2.198727in}{2.998585in}}{\pgfqpoint{2.209326in}{3.002975in}}{\pgfqpoint{2.217139in}{3.010789in}}%
\pgfpathcurveto{\pgfqpoint{2.224953in}{3.018602in}}{\pgfqpoint{2.229343in}{3.029201in}}{\pgfqpoint{2.229343in}{3.040251in}}%
\pgfpathcurveto{\pgfqpoint{2.229343in}{3.051302in}}{\pgfqpoint{2.224953in}{3.061901in}}{\pgfqpoint{2.217139in}{3.069714in}}%
\pgfpathcurveto{\pgfqpoint{2.209326in}{3.077528in}}{\pgfqpoint{2.198727in}{3.081918in}}{\pgfqpoint{2.187677in}{3.081918in}}%
\pgfpathcurveto{\pgfqpoint{2.176626in}{3.081918in}}{\pgfqpoint{2.166027in}{3.077528in}}{\pgfqpoint{2.158214in}{3.069714in}}%
\pgfpathcurveto{\pgfqpoint{2.150400in}{3.061901in}}{\pgfqpoint{2.146010in}{3.051302in}}{\pgfqpoint{2.146010in}{3.040251in}}%
\pgfpathcurveto{\pgfqpoint{2.146010in}{3.029201in}}{\pgfqpoint{2.150400in}{3.018602in}}{\pgfqpoint{2.158214in}{3.010789in}}%
\pgfpathcurveto{\pgfqpoint{2.166027in}{3.002975in}}{\pgfqpoint{2.176626in}{2.998585in}}{\pgfqpoint{2.187677in}{2.998585in}}%
\pgfpathclose%
\pgfusepath{stroke,fill}%
\end{pgfscope}%
\begin{pgfscope}%
\pgfpathrectangle{\pgfqpoint{0.600000in}{0.600000in}}{\pgfqpoint{3.900000in}{3.900000in}}%
\pgfusepath{clip}%
\pgfsetbuttcap%
\pgfsetroundjoin%
\definecolor{currentfill}{rgb}{0.121569,0.466667,0.705882}%
\pgfsetfillcolor{currentfill}%
\pgfsetlinewidth{1.003750pt}%
\definecolor{currentstroke}{rgb}{0.121569,0.466667,0.705882}%
\pgfsetstrokecolor{currentstroke}%
\pgfsetdash{}{0pt}%
\pgfpathmoveto{\pgfqpoint{1.943405in}{3.063562in}}%
\pgfpathcurveto{\pgfqpoint{1.954455in}{3.063562in}}{\pgfqpoint{1.965054in}{3.067952in}}{\pgfqpoint{1.972868in}{3.075766in}}%
\pgfpathcurveto{\pgfqpoint{1.980681in}{3.083579in}}{\pgfqpoint{1.985071in}{3.094178in}}{\pgfqpoint{1.985071in}{3.105229in}}%
\pgfpathcurveto{\pgfqpoint{1.985071in}{3.116279in}}{\pgfqpoint{1.980681in}{3.126878in}}{\pgfqpoint{1.972868in}{3.134691in}}%
\pgfpathcurveto{\pgfqpoint{1.965054in}{3.142505in}}{\pgfqpoint{1.954455in}{3.146895in}}{\pgfqpoint{1.943405in}{3.146895in}}%
\pgfpathcurveto{\pgfqpoint{1.932355in}{3.146895in}}{\pgfqpoint{1.921756in}{3.142505in}}{\pgfqpoint{1.913942in}{3.134691in}}%
\pgfpathcurveto{\pgfqpoint{1.906128in}{3.126878in}}{\pgfqpoint{1.901738in}{3.116279in}}{\pgfqpoint{1.901738in}{3.105229in}}%
\pgfpathcurveto{\pgfqpoint{1.901738in}{3.094178in}}{\pgfqpoint{1.906128in}{3.083579in}}{\pgfqpoint{1.913942in}{3.075766in}}%
\pgfpathcurveto{\pgfqpoint{1.921756in}{3.067952in}}{\pgfqpoint{1.932355in}{3.063562in}}{\pgfqpoint{1.943405in}{3.063562in}}%
\pgfpathclose%
\pgfusepath{stroke,fill}%
\end{pgfscope}%
\begin{pgfscope}%
\pgfpathrectangle{\pgfqpoint{0.600000in}{0.600000in}}{\pgfqpoint{3.900000in}{3.900000in}}%
\pgfusepath{clip}%
\pgfsetbuttcap%
\pgfsetroundjoin%
\definecolor{currentfill}{rgb}{0.121569,0.466667,0.705882}%
\pgfsetfillcolor{currentfill}%
\pgfsetlinewidth{1.003750pt}%
\definecolor{currentstroke}{rgb}{0.121569,0.466667,0.705882}%
\pgfsetstrokecolor{currentstroke}%
\pgfsetdash{}{0pt}%
\pgfpathmoveto{\pgfqpoint{2.997594in}{2.222598in}}%
\pgfpathcurveto{\pgfqpoint{3.008644in}{2.222598in}}{\pgfqpoint{3.019243in}{2.226988in}}{\pgfqpoint{3.027057in}{2.234802in}}%
\pgfpathcurveto{\pgfqpoint{3.034870in}{2.242616in}}{\pgfqpoint{3.039260in}{2.253215in}}{\pgfqpoint{3.039260in}{2.264265in}}%
\pgfpathcurveto{\pgfqpoint{3.039260in}{2.275315in}}{\pgfqpoint{3.034870in}{2.285914in}}{\pgfqpoint{3.027057in}{2.293728in}}%
\pgfpathcurveto{\pgfqpoint{3.019243in}{2.301541in}}{\pgfqpoint{3.008644in}{2.305931in}}{\pgfqpoint{2.997594in}{2.305931in}}%
\pgfpathcurveto{\pgfqpoint{2.986544in}{2.305931in}}{\pgfqpoint{2.975945in}{2.301541in}}{\pgfqpoint{2.968131in}{2.293728in}}%
\pgfpathcurveto{\pgfqpoint{2.960317in}{2.285914in}}{\pgfqpoint{2.955927in}{2.275315in}}{\pgfqpoint{2.955927in}{2.264265in}}%
\pgfpathcurveto{\pgfqpoint{2.955927in}{2.253215in}}{\pgfqpoint{2.960317in}{2.242616in}}{\pgfqpoint{2.968131in}{2.234802in}}%
\pgfpathcurveto{\pgfqpoint{2.975945in}{2.226988in}}{\pgfqpoint{2.986544in}{2.222598in}}{\pgfqpoint{2.997594in}{2.222598in}}%
\pgfpathclose%
\pgfusepath{stroke,fill}%
\end{pgfscope}%
\begin{pgfscope}%
\pgfpathrectangle{\pgfqpoint{0.600000in}{0.600000in}}{\pgfqpoint{3.900000in}{3.900000in}}%
\pgfusepath{clip}%
\pgfsetbuttcap%
\pgfsetroundjoin%
\definecolor{currentfill}{rgb}{0.121569,0.466667,0.705882}%
\pgfsetfillcolor{currentfill}%
\pgfsetlinewidth{1.003750pt}%
\definecolor{currentstroke}{rgb}{0.121569,0.466667,0.705882}%
\pgfsetstrokecolor{currentstroke}%
\pgfsetdash{}{0pt}%
\pgfpathmoveto{\pgfqpoint{2.648337in}{1.353673in}}%
\pgfpathcurveto{\pgfqpoint{2.659387in}{1.353673in}}{\pgfqpoint{2.669986in}{1.358063in}}{\pgfqpoint{2.677799in}{1.365876in}}%
\pgfpathcurveto{\pgfqpoint{2.685613in}{1.373690in}}{\pgfqpoint{2.690003in}{1.384289in}}{\pgfqpoint{2.690003in}{1.395339in}}%
\pgfpathcurveto{\pgfqpoint{2.690003in}{1.406389in}}{\pgfqpoint{2.685613in}{1.416988in}}{\pgfqpoint{2.677799in}{1.424802in}}%
\pgfpathcurveto{\pgfqpoint{2.669986in}{1.432616in}}{\pgfqpoint{2.659387in}{1.437006in}}{\pgfqpoint{2.648337in}{1.437006in}}%
\pgfpathcurveto{\pgfqpoint{2.637287in}{1.437006in}}{\pgfqpoint{2.626688in}{1.432616in}}{\pgfqpoint{2.618874in}{1.424802in}}%
\pgfpathcurveto{\pgfqpoint{2.611060in}{1.416988in}}{\pgfqpoint{2.606670in}{1.406389in}}{\pgfqpoint{2.606670in}{1.395339in}}%
\pgfpathcurveto{\pgfqpoint{2.606670in}{1.384289in}}{\pgfqpoint{2.611060in}{1.373690in}}{\pgfqpoint{2.618874in}{1.365876in}}%
\pgfpathcurveto{\pgfqpoint{2.626688in}{1.358063in}}{\pgfqpoint{2.637287in}{1.353673in}}{\pgfqpoint{2.648337in}{1.353673in}}%
\pgfpathclose%
\pgfusepath{stroke,fill}%
\end{pgfscope}%
\begin{pgfscope}%
\pgfpathrectangle{\pgfqpoint{0.600000in}{0.600000in}}{\pgfqpoint{3.900000in}{3.900000in}}%
\pgfusepath{clip}%
\pgfsetbuttcap%
\pgfsetroundjoin%
\definecolor{currentfill}{rgb}{0.121569,0.466667,0.705882}%
\pgfsetfillcolor{currentfill}%
\pgfsetlinewidth{1.003750pt}%
\definecolor{currentstroke}{rgb}{0.121569,0.466667,0.705882}%
\pgfsetstrokecolor{currentstroke}%
\pgfsetdash{}{0pt}%
\pgfpathmoveto{\pgfqpoint{2.782030in}{3.501972in}}%
\pgfpathcurveto{\pgfqpoint{2.793080in}{3.501972in}}{\pgfqpoint{2.803679in}{3.506362in}}{\pgfqpoint{2.811493in}{3.514176in}}%
\pgfpathcurveto{\pgfqpoint{2.819306in}{3.521990in}}{\pgfqpoint{2.823697in}{3.532589in}}{\pgfqpoint{2.823697in}{3.543639in}}%
\pgfpathcurveto{\pgfqpoint{2.823697in}{3.554689in}}{\pgfqpoint{2.819306in}{3.565288in}}{\pgfqpoint{2.811493in}{3.573102in}}%
\pgfpathcurveto{\pgfqpoint{2.803679in}{3.580915in}}{\pgfqpoint{2.793080in}{3.585305in}}{\pgfqpoint{2.782030in}{3.585305in}}%
\pgfpathcurveto{\pgfqpoint{2.770980in}{3.585305in}}{\pgfqpoint{2.760381in}{3.580915in}}{\pgfqpoint{2.752567in}{3.573102in}}%
\pgfpathcurveto{\pgfqpoint{2.744754in}{3.565288in}}{\pgfqpoint{2.740363in}{3.554689in}}{\pgfqpoint{2.740363in}{3.543639in}}%
\pgfpathcurveto{\pgfqpoint{2.740363in}{3.532589in}}{\pgfqpoint{2.744754in}{3.521990in}}{\pgfqpoint{2.752567in}{3.514176in}}%
\pgfpathcurveto{\pgfqpoint{2.760381in}{3.506362in}}{\pgfqpoint{2.770980in}{3.501972in}}{\pgfqpoint{2.782030in}{3.501972in}}%
\pgfpathclose%
\pgfusepath{stroke,fill}%
\end{pgfscope}%
\begin{pgfscope}%
\pgfpathrectangle{\pgfqpoint{0.600000in}{0.600000in}}{\pgfqpoint{3.900000in}{3.900000in}}%
\pgfusepath{clip}%
\pgfsetbuttcap%
\pgfsetroundjoin%
\definecolor{currentfill}{rgb}{0.121569,0.466667,0.705882}%
\pgfsetfillcolor{currentfill}%
\pgfsetlinewidth{1.003750pt}%
\definecolor{currentstroke}{rgb}{0.121569,0.466667,0.705882}%
\pgfsetstrokecolor{currentstroke}%
\pgfsetdash{}{0pt}%
\pgfpathmoveto{\pgfqpoint{3.017845in}{2.018855in}}%
\pgfpathcurveto{\pgfqpoint{3.028895in}{2.018855in}}{\pgfqpoint{3.039494in}{2.023245in}}{\pgfqpoint{3.047308in}{2.031059in}}%
\pgfpathcurveto{\pgfqpoint{3.055121in}{2.038872in}}{\pgfqpoint{3.059512in}{2.049471in}}{\pgfqpoint{3.059512in}{2.060522in}}%
\pgfpathcurveto{\pgfqpoint{3.059512in}{2.071572in}}{\pgfqpoint{3.055121in}{2.082171in}}{\pgfqpoint{3.047308in}{2.089984in}}%
\pgfpathcurveto{\pgfqpoint{3.039494in}{2.097798in}}{\pgfqpoint{3.028895in}{2.102188in}}{\pgfqpoint{3.017845in}{2.102188in}}%
\pgfpathcurveto{\pgfqpoint{3.006795in}{2.102188in}}{\pgfqpoint{2.996196in}{2.097798in}}{\pgfqpoint{2.988382in}{2.089984in}}%
\pgfpathcurveto{\pgfqpoint{2.980569in}{2.082171in}}{\pgfqpoint{2.976178in}{2.071572in}}{\pgfqpoint{2.976178in}{2.060522in}}%
\pgfpathcurveto{\pgfqpoint{2.976178in}{2.049471in}}{\pgfqpoint{2.980569in}{2.038872in}}{\pgfqpoint{2.988382in}{2.031059in}}%
\pgfpathcurveto{\pgfqpoint{2.996196in}{2.023245in}}{\pgfqpoint{3.006795in}{2.018855in}}{\pgfqpoint{3.017845in}{2.018855in}}%
\pgfpathclose%
\pgfusepath{stroke,fill}%
\end{pgfscope}%
\begin{pgfscope}%
\pgfpathrectangle{\pgfqpoint{0.600000in}{0.600000in}}{\pgfqpoint{3.900000in}{3.900000in}}%
\pgfusepath{clip}%
\pgfsetbuttcap%
\pgfsetroundjoin%
\definecolor{currentfill}{rgb}{0.121569,0.466667,0.705882}%
\pgfsetfillcolor{currentfill}%
\pgfsetlinewidth{1.003750pt}%
\definecolor{currentstroke}{rgb}{0.121569,0.466667,0.705882}%
\pgfsetstrokecolor{currentstroke}%
\pgfsetdash{}{0pt}%
\pgfpathmoveto{\pgfqpoint{1.952205in}{2.029485in}}%
\pgfpathcurveto{\pgfqpoint{1.963255in}{2.029485in}}{\pgfqpoint{1.973854in}{2.033875in}}{\pgfqpoint{1.981667in}{2.041689in}}%
\pgfpathcurveto{\pgfqpoint{1.989481in}{2.049503in}}{\pgfqpoint{1.993871in}{2.060102in}}{\pgfqpoint{1.993871in}{2.071152in}}%
\pgfpathcurveto{\pgfqpoint{1.993871in}{2.082202in}}{\pgfqpoint{1.989481in}{2.092801in}}{\pgfqpoint{1.981667in}{2.100615in}}%
\pgfpathcurveto{\pgfqpoint{1.973854in}{2.108428in}}{\pgfqpoint{1.963255in}{2.112819in}}{\pgfqpoint{1.952205in}{2.112819in}}%
\pgfpathcurveto{\pgfqpoint{1.941155in}{2.112819in}}{\pgfqpoint{1.930556in}{2.108428in}}{\pgfqpoint{1.922742in}{2.100615in}}%
\pgfpathcurveto{\pgfqpoint{1.914928in}{2.092801in}}{\pgfqpoint{1.910538in}{2.082202in}}{\pgfqpoint{1.910538in}{2.071152in}}%
\pgfpathcurveto{\pgfqpoint{1.910538in}{2.060102in}}{\pgfqpoint{1.914928in}{2.049503in}}{\pgfqpoint{1.922742in}{2.041689in}}%
\pgfpathcurveto{\pgfqpoint{1.930556in}{2.033875in}}{\pgfqpoint{1.941155in}{2.029485in}}{\pgfqpoint{1.952205in}{2.029485in}}%
\pgfpathclose%
\pgfusepath{stroke,fill}%
\end{pgfscope}%
\begin{pgfscope}%
\pgfpathrectangle{\pgfqpoint{0.600000in}{0.600000in}}{\pgfqpoint{3.900000in}{3.900000in}}%
\pgfusepath{clip}%
\pgfsetbuttcap%
\pgfsetroundjoin%
\definecolor{currentfill}{rgb}{0.121569,0.466667,0.705882}%
\pgfsetfillcolor{currentfill}%
\pgfsetlinewidth{1.003750pt}%
\definecolor{currentstroke}{rgb}{0.121569,0.466667,0.705882}%
\pgfsetstrokecolor{currentstroke}%
\pgfsetdash{}{0pt}%
\pgfpathmoveto{\pgfqpoint{2.734785in}{2.749131in}}%
\pgfpathcurveto{\pgfqpoint{2.745835in}{2.749131in}}{\pgfqpoint{2.756434in}{2.753521in}}{\pgfqpoint{2.764248in}{2.761334in}}%
\pgfpathcurveto{\pgfqpoint{2.772062in}{2.769148in}}{\pgfqpoint{2.776452in}{2.779747in}}{\pgfqpoint{2.776452in}{2.790797in}}%
\pgfpathcurveto{\pgfqpoint{2.776452in}{2.801847in}}{\pgfqpoint{2.772062in}{2.812446in}}{\pgfqpoint{2.764248in}{2.820260in}}%
\pgfpathcurveto{\pgfqpoint{2.756434in}{2.828074in}}{\pgfqpoint{2.745835in}{2.832464in}}{\pgfqpoint{2.734785in}{2.832464in}}%
\pgfpathcurveto{\pgfqpoint{2.723735in}{2.832464in}}{\pgfqpoint{2.713136in}{2.828074in}}{\pgfqpoint{2.705322in}{2.820260in}}%
\pgfpathcurveto{\pgfqpoint{2.697509in}{2.812446in}}{\pgfqpoint{2.693119in}{2.801847in}}{\pgfqpoint{2.693119in}{2.790797in}}%
\pgfpathcurveto{\pgfqpoint{2.693119in}{2.779747in}}{\pgfqpoint{2.697509in}{2.769148in}}{\pgfqpoint{2.705322in}{2.761334in}}%
\pgfpathcurveto{\pgfqpoint{2.713136in}{2.753521in}}{\pgfqpoint{2.723735in}{2.749131in}}{\pgfqpoint{2.734785in}{2.749131in}}%
\pgfpathclose%
\pgfusepath{stroke,fill}%
\end{pgfscope}%
\begin{pgfscope}%
\pgfpathrectangle{\pgfqpoint{0.600000in}{0.600000in}}{\pgfqpoint{3.900000in}{3.900000in}}%
\pgfusepath{clip}%
\pgfsetbuttcap%
\pgfsetroundjoin%
\definecolor{currentfill}{rgb}{0.121569,0.466667,0.705882}%
\pgfsetfillcolor{currentfill}%
\pgfsetlinewidth{1.003750pt}%
\definecolor{currentstroke}{rgb}{0.121569,0.466667,0.705882}%
\pgfsetstrokecolor{currentstroke}%
\pgfsetdash{}{0pt}%
\pgfpathmoveto{\pgfqpoint{2.687962in}{2.185715in}}%
\pgfpathcurveto{\pgfqpoint{2.699012in}{2.185715in}}{\pgfqpoint{2.709611in}{2.190105in}}{\pgfqpoint{2.717425in}{2.197918in}}%
\pgfpathcurveto{\pgfqpoint{2.725239in}{2.205732in}}{\pgfqpoint{2.729629in}{2.216331in}}{\pgfqpoint{2.729629in}{2.227381in}}%
\pgfpathcurveto{\pgfqpoint{2.729629in}{2.238431in}}{\pgfqpoint{2.725239in}{2.249030in}}{\pgfqpoint{2.717425in}{2.256844in}}%
\pgfpathcurveto{\pgfqpoint{2.709611in}{2.264658in}}{\pgfqpoint{2.699012in}{2.269048in}}{\pgfqpoint{2.687962in}{2.269048in}}%
\pgfpathcurveto{\pgfqpoint{2.676912in}{2.269048in}}{\pgfqpoint{2.666313in}{2.264658in}}{\pgfqpoint{2.658499in}{2.256844in}}%
\pgfpathcurveto{\pgfqpoint{2.650686in}{2.249030in}}{\pgfqpoint{2.646295in}{2.238431in}}{\pgfqpoint{2.646295in}{2.227381in}}%
\pgfpathcurveto{\pgfqpoint{2.646295in}{2.216331in}}{\pgfqpoint{2.650686in}{2.205732in}}{\pgfqpoint{2.658499in}{2.197918in}}%
\pgfpathcurveto{\pgfqpoint{2.666313in}{2.190105in}}{\pgfqpoint{2.676912in}{2.185715in}}{\pgfqpoint{2.687962in}{2.185715in}}%
\pgfpathclose%
\pgfusepath{stroke,fill}%
\end{pgfscope}%
\begin{pgfscope}%
\pgfpathrectangle{\pgfqpoint{0.600000in}{0.600000in}}{\pgfqpoint{3.900000in}{3.900000in}}%
\pgfusepath{clip}%
\pgfsetbuttcap%
\pgfsetroundjoin%
\definecolor{currentfill}{rgb}{0.121569,0.466667,0.705882}%
\pgfsetfillcolor{currentfill}%
\pgfsetlinewidth{1.003750pt}%
\definecolor{currentstroke}{rgb}{0.121569,0.466667,0.705882}%
\pgfsetstrokecolor{currentstroke}%
\pgfsetdash{}{0pt}%
\pgfpathmoveto{\pgfqpoint{2.481863in}{2.338166in}}%
\pgfpathcurveto{\pgfqpoint{2.492914in}{2.338166in}}{\pgfqpoint{2.503513in}{2.342556in}}{\pgfqpoint{2.511326in}{2.350370in}}%
\pgfpathcurveto{\pgfqpoint{2.519140in}{2.358183in}}{\pgfqpoint{2.523530in}{2.368782in}}{\pgfqpoint{2.523530in}{2.379832in}}%
\pgfpathcurveto{\pgfqpoint{2.523530in}{2.390882in}}{\pgfqpoint{2.519140in}{2.401482in}}{\pgfqpoint{2.511326in}{2.409295in}}%
\pgfpathcurveto{\pgfqpoint{2.503513in}{2.417109in}}{\pgfqpoint{2.492914in}{2.421499in}}{\pgfqpoint{2.481863in}{2.421499in}}%
\pgfpathcurveto{\pgfqpoint{2.470813in}{2.421499in}}{\pgfqpoint{2.460214in}{2.417109in}}{\pgfqpoint{2.452401in}{2.409295in}}%
\pgfpathcurveto{\pgfqpoint{2.444587in}{2.401482in}}{\pgfqpoint{2.440197in}{2.390882in}}{\pgfqpoint{2.440197in}{2.379832in}}%
\pgfpathcurveto{\pgfqpoint{2.440197in}{2.368782in}}{\pgfqpoint{2.444587in}{2.358183in}}{\pgfqpoint{2.452401in}{2.350370in}}%
\pgfpathcurveto{\pgfqpoint{2.460214in}{2.342556in}}{\pgfqpoint{2.470813in}{2.338166in}}{\pgfqpoint{2.481863in}{2.338166in}}%
\pgfpathclose%
\pgfusepath{stroke,fill}%
\end{pgfscope}%
\begin{pgfscope}%
\pgfpathrectangle{\pgfqpoint{0.600000in}{0.600000in}}{\pgfqpoint{3.900000in}{3.900000in}}%
\pgfusepath{clip}%
\pgfsetbuttcap%
\pgfsetroundjoin%
\definecolor{currentfill}{rgb}{0.121569,0.466667,0.705882}%
\pgfsetfillcolor{currentfill}%
\pgfsetlinewidth{1.003750pt}%
\definecolor{currentstroke}{rgb}{0.121569,0.466667,0.705882}%
\pgfsetstrokecolor{currentstroke}%
\pgfsetdash{}{0pt}%
\pgfpathmoveto{\pgfqpoint{2.356961in}{2.409660in}}%
\pgfpathcurveto{\pgfqpoint{2.368012in}{2.409660in}}{\pgfqpoint{2.378611in}{2.414050in}}{\pgfqpoint{2.386424in}{2.421864in}}%
\pgfpathcurveto{\pgfqpoint{2.394238in}{2.429677in}}{\pgfqpoint{2.398628in}{2.440276in}}{\pgfqpoint{2.398628in}{2.451327in}}%
\pgfpathcurveto{\pgfqpoint{2.398628in}{2.462377in}}{\pgfqpoint{2.394238in}{2.472976in}}{\pgfqpoint{2.386424in}{2.480789in}}%
\pgfpathcurveto{\pgfqpoint{2.378611in}{2.488603in}}{\pgfqpoint{2.368012in}{2.492993in}}{\pgfqpoint{2.356961in}{2.492993in}}%
\pgfpathcurveto{\pgfqpoint{2.345911in}{2.492993in}}{\pgfqpoint{2.335312in}{2.488603in}}{\pgfqpoint{2.327499in}{2.480789in}}%
\pgfpathcurveto{\pgfqpoint{2.319685in}{2.472976in}}{\pgfqpoint{2.315295in}{2.462377in}}{\pgfqpoint{2.315295in}{2.451327in}}%
\pgfpathcurveto{\pgfqpoint{2.315295in}{2.440276in}}{\pgfqpoint{2.319685in}{2.429677in}}{\pgfqpoint{2.327499in}{2.421864in}}%
\pgfpathcurveto{\pgfqpoint{2.335312in}{2.414050in}}{\pgfqpoint{2.345911in}{2.409660in}}{\pgfqpoint{2.356961in}{2.409660in}}%
\pgfpathclose%
\pgfusepath{stroke,fill}%
\end{pgfscope}%
\begin{pgfscope}%
\pgfpathrectangle{\pgfqpoint{0.600000in}{0.600000in}}{\pgfqpoint{3.900000in}{3.900000in}}%
\pgfusepath{clip}%
\pgfsetbuttcap%
\pgfsetroundjoin%
\definecolor{currentfill}{rgb}{0.121569,0.466667,0.705882}%
\pgfsetfillcolor{currentfill}%
\pgfsetlinewidth{1.003750pt}%
\definecolor{currentstroke}{rgb}{0.121569,0.466667,0.705882}%
\pgfsetstrokecolor{currentstroke}%
\pgfsetdash{}{0pt}%
\pgfpathmoveto{\pgfqpoint{2.716351in}{1.998279in}}%
\pgfpathcurveto{\pgfqpoint{2.727401in}{1.998279in}}{\pgfqpoint{2.738000in}{2.002670in}}{\pgfqpoint{2.745814in}{2.010483in}}%
\pgfpathcurveto{\pgfqpoint{2.753627in}{2.018297in}}{\pgfqpoint{2.758018in}{2.028896in}}{\pgfqpoint{2.758018in}{2.039946in}}%
\pgfpathcurveto{\pgfqpoint{2.758018in}{2.050996in}}{\pgfqpoint{2.753627in}{2.061595in}}{\pgfqpoint{2.745814in}{2.069409in}}%
\pgfpathcurveto{\pgfqpoint{2.738000in}{2.077222in}}{\pgfqpoint{2.727401in}{2.081613in}}{\pgfqpoint{2.716351in}{2.081613in}}%
\pgfpathcurveto{\pgfqpoint{2.705301in}{2.081613in}}{\pgfqpoint{2.694702in}{2.077222in}}{\pgfqpoint{2.686888in}{2.069409in}}%
\pgfpathcurveto{\pgfqpoint{2.679074in}{2.061595in}}{\pgfqpoint{2.674684in}{2.050996in}}{\pgfqpoint{2.674684in}{2.039946in}}%
\pgfpathcurveto{\pgfqpoint{2.674684in}{2.028896in}}{\pgfqpoint{2.679074in}{2.018297in}}{\pgfqpoint{2.686888in}{2.010483in}}%
\pgfpathcurveto{\pgfqpoint{2.694702in}{2.002670in}}{\pgfqpoint{2.705301in}{1.998279in}}{\pgfqpoint{2.716351in}{1.998279in}}%
\pgfpathclose%
\pgfusepath{stroke,fill}%
\end{pgfscope}%
\begin{pgfscope}%
\pgfpathrectangle{\pgfqpoint{0.600000in}{0.600000in}}{\pgfqpoint{3.900000in}{3.900000in}}%
\pgfusepath{clip}%
\pgfsetbuttcap%
\pgfsetroundjoin%
\definecolor{currentfill}{rgb}{0.121569,0.466667,0.705882}%
\pgfsetfillcolor{currentfill}%
\pgfsetlinewidth{1.003750pt}%
\definecolor{currentstroke}{rgb}{0.121569,0.466667,0.705882}%
\pgfsetstrokecolor{currentstroke}%
\pgfsetdash{}{0pt}%
\pgfpathmoveto{\pgfqpoint{2.076248in}{2.547620in}}%
\pgfpathcurveto{\pgfqpoint{2.087298in}{2.547620in}}{\pgfqpoint{2.097897in}{2.552010in}}{\pgfqpoint{2.105711in}{2.559824in}}%
\pgfpathcurveto{\pgfqpoint{2.113525in}{2.567637in}}{\pgfqpoint{2.117915in}{2.578236in}}{\pgfqpoint{2.117915in}{2.589286in}}%
\pgfpathcurveto{\pgfqpoint{2.117915in}{2.600336in}}{\pgfqpoint{2.113525in}{2.610935in}}{\pgfqpoint{2.105711in}{2.618749in}}%
\pgfpathcurveto{\pgfqpoint{2.097897in}{2.626563in}}{\pgfqpoint{2.087298in}{2.630953in}}{\pgfqpoint{2.076248in}{2.630953in}}%
\pgfpathcurveto{\pgfqpoint{2.065198in}{2.630953in}}{\pgfqpoint{2.054599in}{2.626563in}}{\pgfqpoint{2.046785in}{2.618749in}}%
\pgfpathcurveto{\pgfqpoint{2.038972in}{2.610935in}}{\pgfqpoint{2.034582in}{2.600336in}}{\pgfqpoint{2.034582in}{2.589286in}}%
\pgfpathcurveto{\pgfqpoint{2.034582in}{2.578236in}}{\pgfqpoint{2.038972in}{2.567637in}}{\pgfqpoint{2.046785in}{2.559824in}}%
\pgfpathcurveto{\pgfqpoint{2.054599in}{2.552010in}}{\pgfqpoint{2.065198in}{2.547620in}}{\pgfqpoint{2.076248in}{2.547620in}}%
\pgfpathclose%
\pgfusepath{stroke,fill}%
\end{pgfscope}%
\begin{pgfscope}%
\pgfpathrectangle{\pgfqpoint{0.600000in}{0.600000in}}{\pgfqpoint{3.900000in}{3.900000in}}%
\pgfusepath{clip}%
\pgfsetbuttcap%
\pgfsetroundjoin%
\definecolor{currentfill}{rgb}{0.121569,0.466667,0.705882}%
\pgfsetfillcolor{currentfill}%
\pgfsetlinewidth{1.003750pt}%
\definecolor{currentstroke}{rgb}{0.121569,0.466667,0.705882}%
\pgfsetstrokecolor{currentstroke}%
\pgfsetdash{}{0pt}%
\pgfpathmoveto{\pgfqpoint{2.716375in}{1.968214in}}%
\pgfpathcurveto{\pgfqpoint{2.727425in}{1.968214in}}{\pgfqpoint{2.738024in}{1.972605in}}{\pgfqpoint{2.745838in}{1.980418in}}%
\pgfpathcurveto{\pgfqpoint{2.753651in}{1.988232in}}{\pgfqpoint{2.758042in}{1.998831in}}{\pgfqpoint{2.758042in}{2.009881in}}%
\pgfpathcurveto{\pgfqpoint{2.758042in}{2.020931in}}{\pgfqpoint{2.753651in}{2.031530in}}{\pgfqpoint{2.745838in}{2.039344in}}%
\pgfpathcurveto{\pgfqpoint{2.738024in}{2.047158in}}{\pgfqpoint{2.727425in}{2.051548in}}{\pgfqpoint{2.716375in}{2.051548in}}%
\pgfpathcurveto{\pgfqpoint{2.705325in}{2.051548in}}{\pgfqpoint{2.694726in}{2.047158in}}{\pgfqpoint{2.686912in}{2.039344in}}%
\pgfpathcurveto{\pgfqpoint{2.679098in}{2.031530in}}{\pgfqpoint{2.674708in}{2.020931in}}{\pgfqpoint{2.674708in}{2.009881in}}%
\pgfpathcurveto{\pgfqpoint{2.674708in}{1.998831in}}{\pgfqpoint{2.679098in}{1.988232in}}{\pgfqpoint{2.686912in}{1.980418in}}%
\pgfpathcurveto{\pgfqpoint{2.694726in}{1.972605in}}{\pgfqpoint{2.705325in}{1.968214in}}{\pgfqpoint{2.716375in}{1.968214in}}%
\pgfpathclose%
\pgfusepath{stroke,fill}%
\end{pgfscope}%
\begin{pgfscope}%
\pgfpathrectangle{\pgfqpoint{0.600000in}{0.600000in}}{\pgfqpoint{3.900000in}{3.900000in}}%
\pgfusepath{clip}%
\pgfsetbuttcap%
\pgfsetroundjoin%
\definecolor{currentfill}{rgb}{0.121569,0.466667,0.705882}%
\pgfsetfillcolor{currentfill}%
\pgfsetlinewidth{1.003750pt}%
\definecolor{currentstroke}{rgb}{0.121569,0.466667,0.705882}%
\pgfsetstrokecolor{currentstroke}%
\pgfsetdash{}{0pt}%
\pgfpathmoveto{\pgfqpoint{2.513176in}{2.155812in}}%
\pgfpathcurveto{\pgfqpoint{2.524226in}{2.155812in}}{\pgfqpoint{2.534825in}{2.160203in}}{\pgfqpoint{2.542639in}{2.168016in}}%
\pgfpathcurveto{\pgfqpoint{2.550452in}{2.175830in}}{\pgfqpoint{2.554843in}{2.186429in}}{\pgfqpoint{2.554843in}{2.197479in}}%
\pgfpathcurveto{\pgfqpoint{2.554843in}{2.208529in}}{\pgfqpoint{2.550452in}{2.219128in}}{\pgfqpoint{2.542639in}{2.226942in}}%
\pgfpathcurveto{\pgfqpoint{2.534825in}{2.234755in}}{\pgfqpoint{2.524226in}{2.239146in}}{\pgfqpoint{2.513176in}{2.239146in}}%
\pgfpathcurveto{\pgfqpoint{2.502126in}{2.239146in}}{\pgfqpoint{2.491527in}{2.234755in}}{\pgfqpoint{2.483713in}{2.226942in}}%
\pgfpathcurveto{\pgfqpoint{2.475900in}{2.219128in}}{\pgfqpoint{2.471509in}{2.208529in}}{\pgfqpoint{2.471509in}{2.197479in}}%
\pgfpathcurveto{\pgfqpoint{2.471509in}{2.186429in}}{\pgfqpoint{2.475900in}{2.175830in}}{\pgfqpoint{2.483713in}{2.168016in}}%
\pgfpathcurveto{\pgfqpoint{2.491527in}{2.160203in}}{\pgfqpoint{2.502126in}{2.155812in}}{\pgfqpoint{2.513176in}{2.155812in}}%
\pgfpathclose%
\pgfusepath{stroke,fill}%
\end{pgfscope}%
\begin{pgfscope}%
\pgfpathrectangle{\pgfqpoint{0.600000in}{0.600000in}}{\pgfqpoint{3.900000in}{3.900000in}}%
\pgfusepath{clip}%
\pgfsetbuttcap%
\pgfsetroundjoin%
\definecolor{currentfill}{rgb}{0.121569,0.466667,0.705882}%
\pgfsetfillcolor{currentfill}%
\pgfsetlinewidth{1.003750pt}%
\definecolor{currentstroke}{rgb}{0.121569,0.466667,0.705882}%
\pgfsetstrokecolor{currentstroke}%
\pgfsetdash{}{0pt}%
\pgfpathmoveto{\pgfqpoint{2.635238in}{2.433332in}}%
\pgfpathcurveto{\pgfqpoint{2.646288in}{2.433332in}}{\pgfqpoint{2.656887in}{2.437722in}}{\pgfqpoint{2.664701in}{2.445536in}}%
\pgfpathcurveto{\pgfqpoint{2.672515in}{2.453349in}}{\pgfqpoint{2.676905in}{2.463948in}}{\pgfqpoint{2.676905in}{2.474998in}}%
\pgfpathcurveto{\pgfqpoint{2.676905in}{2.486048in}}{\pgfqpoint{2.672515in}{2.496648in}}{\pgfqpoint{2.664701in}{2.504461in}}%
\pgfpathcurveto{\pgfqpoint{2.656887in}{2.512275in}}{\pgfqpoint{2.646288in}{2.516665in}}{\pgfqpoint{2.635238in}{2.516665in}}%
\pgfpathcurveto{\pgfqpoint{2.624188in}{2.516665in}}{\pgfqpoint{2.613589in}{2.512275in}}{\pgfqpoint{2.605776in}{2.504461in}}%
\pgfpathcurveto{\pgfqpoint{2.597962in}{2.496648in}}{\pgfqpoint{2.593572in}{2.486048in}}{\pgfqpoint{2.593572in}{2.474998in}}%
\pgfpathcurveto{\pgfqpoint{2.593572in}{2.463948in}}{\pgfqpoint{2.597962in}{2.453349in}}{\pgfqpoint{2.605776in}{2.445536in}}%
\pgfpathcurveto{\pgfqpoint{2.613589in}{2.437722in}}{\pgfqpoint{2.624188in}{2.433332in}}{\pgfqpoint{2.635238in}{2.433332in}}%
\pgfpathclose%
\pgfusepath{stroke,fill}%
\end{pgfscope}%
\begin{pgfscope}%
\pgfpathrectangle{\pgfqpoint{0.600000in}{0.600000in}}{\pgfqpoint{3.900000in}{3.900000in}}%
\pgfusepath{clip}%
\pgfsetbuttcap%
\pgfsetroundjoin%
\definecolor{currentfill}{rgb}{0.121569,0.466667,0.705882}%
\pgfsetfillcolor{currentfill}%
\pgfsetlinewidth{1.003750pt}%
\definecolor{currentstroke}{rgb}{0.121569,0.466667,0.705882}%
\pgfsetstrokecolor{currentstroke}%
\pgfsetdash{}{0pt}%
\pgfpathmoveto{\pgfqpoint{2.472464in}{3.197055in}}%
\pgfpathcurveto{\pgfqpoint{2.483514in}{3.197055in}}{\pgfqpoint{2.494113in}{3.201445in}}{\pgfqpoint{2.501926in}{3.209259in}}%
\pgfpathcurveto{\pgfqpoint{2.509740in}{3.217073in}}{\pgfqpoint{2.514130in}{3.227672in}}{\pgfqpoint{2.514130in}{3.238722in}}%
\pgfpathcurveto{\pgfqpoint{2.514130in}{3.249772in}}{\pgfqpoint{2.509740in}{3.260371in}}{\pgfqpoint{2.501926in}{3.268185in}}%
\pgfpathcurveto{\pgfqpoint{2.494113in}{3.275998in}}{\pgfqpoint{2.483514in}{3.280388in}}{\pgfqpoint{2.472464in}{3.280388in}}%
\pgfpathcurveto{\pgfqpoint{2.461413in}{3.280388in}}{\pgfqpoint{2.450814in}{3.275998in}}{\pgfqpoint{2.443001in}{3.268185in}}%
\pgfpathcurveto{\pgfqpoint{2.435187in}{3.260371in}}{\pgfqpoint{2.430797in}{3.249772in}}{\pgfqpoint{2.430797in}{3.238722in}}%
\pgfpathcurveto{\pgfqpoint{2.430797in}{3.227672in}}{\pgfqpoint{2.435187in}{3.217073in}}{\pgfqpoint{2.443001in}{3.209259in}}%
\pgfpathcurveto{\pgfqpoint{2.450814in}{3.201445in}}{\pgfqpoint{2.461413in}{3.197055in}}{\pgfqpoint{2.472464in}{3.197055in}}%
\pgfpathclose%
\pgfusepath{stroke,fill}%
\end{pgfscope}%
\begin{pgfscope}%
\pgfpathrectangle{\pgfqpoint{0.600000in}{0.600000in}}{\pgfqpoint{3.900000in}{3.900000in}}%
\pgfusepath{clip}%
\pgfsetbuttcap%
\pgfsetroundjoin%
\definecolor{currentfill}{rgb}{0.121569,0.466667,0.705882}%
\pgfsetfillcolor{currentfill}%
\pgfsetlinewidth{1.003750pt}%
\definecolor{currentstroke}{rgb}{0.121569,0.466667,0.705882}%
\pgfsetstrokecolor{currentstroke}%
\pgfsetdash{}{0pt}%
\pgfpathmoveto{\pgfqpoint{3.821867in}{2.072872in}}%
\pgfpathcurveto{\pgfqpoint{3.832917in}{2.072872in}}{\pgfqpoint{3.843516in}{2.077263in}}{\pgfqpoint{3.851330in}{2.085076in}}%
\pgfpathcurveto{\pgfqpoint{3.859143in}{2.092890in}}{\pgfqpoint{3.863533in}{2.103489in}}{\pgfqpoint{3.863533in}{2.114539in}}%
\pgfpathcurveto{\pgfqpoint{3.863533in}{2.125589in}}{\pgfqpoint{3.859143in}{2.136188in}}{\pgfqpoint{3.851330in}{2.144002in}}%
\pgfpathcurveto{\pgfqpoint{3.843516in}{2.151815in}}{\pgfqpoint{3.832917in}{2.156206in}}{\pgfqpoint{3.821867in}{2.156206in}}%
\pgfpathcurveto{\pgfqpoint{3.810817in}{2.156206in}}{\pgfqpoint{3.800218in}{2.151815in}}{\pgfqpoint{3.792404in}{2.144002in}}%
\pgfpathcurveto{\pgfqpoint{3.784590in}{2.136188in}}{\pgfqpoint{3.780200in}{2.125589in}}{\pgfqpoint{3.780200in}{2.114539in}}%
\pgfpathcurveto{\pgfqpoint{3.780200in}{2.103489in}}{\pgfqpoint{3.784590in}{2.092890in}}{\pgfqpoint{3.792404in}{2.085076in}}%
\pgfpathcurveto{\pgfqpoint{3.800218in}{2.077263in}}{\pgfqpoint{3.810817in}{2.072872in}}{\pgfqpoint{3.821867in}{2.072872in}}%
\pgfpathclose%
\pgfusepath{stroke,fill}%
\end{pgfscope}%
\begin{pgfscope}%
\pgfpathrectangle{\pgfqpoint{0.600000in}{0.600000in}}{\pgfqpoint{3.900000in}{3.900000in}}%
\pgfusepath{clip}%
\pgfsetbuttcap%
\pgfsetroundjoin%
\definecolor{currentfill}{rgb}{0.121569,0.466667,0.705882}%
\pgfsetfillcolor{currentfill}%
\pgfsetlinewidth{1.003750pt}%
\definecolor{currentstroke}{rgb}{0.121569,0.466667,0.705882}%
\pgfsetstrokecolor{currentstroke}%
\pgfsetdash{}{0pt}%
\pgfpathmoveto{\pgfqpoint{1.282485in}{2.823043in}}%
\pgfpathcurveto{\pgfqpoint{1.293535in}{2.823043in}}{\pgfqpoint{1.304134in}{2.827433in}}{\pgfqpoint{1.311948in}{2.835247in}}%
\pgfpathcurveto{\pgfqpoint{1.319762in}{2.843061in}}{\pgfqpoint{1.324152in}{2.853660in}}{\pgfqpoint{1.324152in}{2.864710in}}%
\pgfpathcurveto{\pgfqpoint{1.324152in}{2.875760in}}{\pgfqpoint{1.319762in}{2.886359in}}{\pgfqpoint{1.311948in}{2.894173in}}%
\pgfpathcurveto{\pgfqpoint{1.304134in}{2.901986in}}{\pgfqpoint{1.293535in}{2.906377in}}{\pgfqpoint{1.282485in}{2.906377in}}%
\pgfpathcurveto{\pgfqpoint{1.271435in}{2.906377in}}{\pgfqpoint{1.260836in}{2.901986in}}{\pgfqpoint{1.253022in}{2.894173in}}%
\pgfpathcurveto{\pgfqpoint{1.245209in}{2.886359in}}{\pgfqpoint{1.240819in}{2.875760in}}{\pgfqpoint{1.240819in}{2.864710in}}%
\pgfpathcurveto{\pgfqpoint{1.240819in}{2.853660in}}{\pgfqpoint{1.245209in}{2.843061in}}{\pgfqpoint{1.253022in}{2.835247in}}%
\pgfpathcurveto{\pgfqpoint{1.260836in}{2.827433in}}{\pgfqpoint{1.271435in}{2.823043in}}{\pgfqpoint{1.282485in}{2.823043in}}%
\pgfpathclose%
\pgfusepath{stroke,fill}%
\end{pgfscope}%
\begin{pgfscope}%
\pgfpathrectangle{\pgfqpoint{0.600000in}{0.600000in}}{\pgfqpoint{3.900000in}{3.900000in}}%
\pgfusepath{clip}%
\pgfsetbuttcap%
\pgfsetroundjoin%
\definecolor{currentfill}{rgb}{0.121569,0.466667,0.705882}%
\pgfsetfillcolor{currentfill}%
\pgfsetlinewidth{1.003750pt}%
\definecolor{currentstroke}{rgb}{0.121569,0.466667,0.705882}%
\pgfsetstrokecolor{currentstroke}%
\pgfsetdash{}{0pt}%
\pgfpathmoveto{\pgfqpoint{2.215963in}{2.465840in}}%
\pgfpathcurveto{\pgfqpoint{2.227013in}{2.465840in}}{\pgfqpoint{2.237612in}{2.470230in}}{\pgfqpoint{2.245426in}{2.478044in}}%
\pgfpathcurveto{\pgfqpoint{2.253240in}{2.485857in}}{\pgfqpoint{2.257630in}{2.496456in}}{\pgfqpoint{2.257630in}{2.507506in}}%
\pgfpathcurveto{\pgfqpoint{2.257630in}{2.518557in}}{\pgfqpoint{2.253240in}{2.529156in}}{\pgfqpoint{2.245426in}{2.536969in}}%
\pgfpathcurveto{\pgfqpoint{2.237612in}{2.544783in}}{\pgfqpoint{2.227013in}{2.549173in}}{\pgfqpoint{2.215963in}{2.549173in}}%
\pgfpathcurveto{\pgfqpoint{2.204913in}{2.549173in}}{\pgfqpoint{2.194314in}{2.544783in}}{\pgfqpoint{2.186501in}{2.536969in}}%
\pgfpathcurveto{\pgfqpoint{2.178687in}{2.529156in}}{\pgfqpoint{2.174297in}{2.518557in}}{\pgfqpoint{2.174297in}{2.507506in}}%
\pgfpathcurveto{\pgfqpoint{2.174297in}{2.496456in}}{\pgfqpoint{2.178687in}{2.485857in}}{\pgfqpoint{2.186501in}{2.478044in}}%
\pgfpathcurveto{\pgfqpoint{2.194314in}{2.470230in}}{\pgfqpoint{2.204913in}{2.465840in}}{\pgfqpoint{2.215963in}{2.465840in}}%
\pgfpathclose%
\pgfusepath{stroke,fill}%
\end{pgfscope}%
\begin{pgfscope}%
\pgfpathrectangle{\pgfqpoint{0.600000in}{0.600000in}}{\pgfqpoint{3.900000in}{3.900000in}}%
\pgfusepath{clip}%
\pgfsetbuttcap%
\pgfsetroundjoin%
\definecolor{currentfill}{rgb}{0.121569,0.466667,0.705882}%
\pgfsetfillcolor{currentfill}%
\pgfsetlinewidth{1.003750pt}%
\definecolor{currentstroke}{rgb}{0.121569,0.466667,0.705882}%
\pgfsetstrokecolor{currentstroke}%
\pgfsetdash{}{0pt}%
\pgfpathmoveto{\pgfqpoint{2.477243in}{3.454151in}}%
\pgfpathcurveto{\pgfqpoint{2.488293in}{3.454151in}}{\pgfqpoint{2.498892in}{3.458541in}}{\pgfqpoint{2.506706in}{3.466355in}}%
\pgfpathcurveto{\pgfqpoint{2.514520in}{3.474169in}}{\pgfqpoint{2.518910in}{3.484768in}}{\pgfqpoint{2.518910in}{3.495818in}}%
\pgfpathcurveto{\pgfqpoint{2.518910in}{3.506868in}}{\pgfqpoint{2.514520in}{3.517467in}}{\pgfqpoint{2.506706in}{3.525280in}}%
\pgfpathcurveto{\pgfqpoint{2.498892in}{3.533094in}}{\pgfqpoint{2.488293in}{3.537484in}}{\pgfqpoint{2.477243in}{3.537484in}}%
\pgfpathcurveto{\pgfqpoint{2.466193in}{3.537484in}}{\pgfqpoint{2.455594in}{3.533094in}}{\pgfqpoint{2.447780in}{3.525280in}}%
\pgfpathcurveto{\pgfqpoint{2.439967in}{3.517467in}}{\pgfqpoint{2.435577in}{3.506868in}}{\pgfqpoint{2.435577in}{3.495818in}}%
\pgfpathcurveto{\pgfqpoint{2.435577in}{3.484768in}}{\pgfqpoint{2.439967in}{3.474169in}}{\pgfqpoint{2.447780in}{3.466355in}}%
\pgfpathcurveto{\pgfqpoint{2.455594in}{3.458541in}}{\pgfqpoint{2.466193in}{3.454151in}}{\pgfqpoint{2.477243in}{3.454151in}}%
\pgfpathclose%
\pgfusepath{stroke,fill}%
\end{pgfscope}%
\begin{pgfscope}%
\pgfpathrectangle{\pgfqpoint{0.600000in}{0.600000in}}{\pgfqpoint{3.900000in}{3.900000in}}%
\pgfusepath{clip}%
\pgfsetbuttcap%
\pgfsetroundjoin%
\definecolor{currentfill}{rgb}{0.121569,0.466667,0.705882}%
\pgfsetfillcolor{currentfill}%
\pgfsetlinewidth{1.003750pt}%
\definecolor{currentstroke}{rgb}{0.121569,0.466667,0.705882}%
\pgfsetstrokecolor{currentstroke}%
\pgfsetdash{}{0pt}%
\pgfpathmoveto{\pgfqpoint{2.281840in}{2.204875in}}%
\pgfpathcurveto{\pgfqpoint{2.292890in}{2.204875in}}{\pgfqpoint{2.303489in}{2.209266in}}{\pgfqpoint{2.311303in}{2.217079in}}%
\pgfpathcurveto{\pgfqpoint{2.319117in}{2.224893in}}{\pgfqpoint{2.323507in}{2.235492in}}{\pgfqpoint{2.323507in}{2.246542in}}%
\pgfpathcurveto{\pgfqpoint{2.323507in}{2.257592in}}{\pgfqpoint{2.319117in}{2.268191in}}{\pgfqpoint{2.311303in}{2.276005in}}%
\pgfpathcurveto{\pgfqpoint{2.303489in}{2.283818in}}{\pgfqpoint{2.292890in}{2.288209in}}{\pgfqpoint{2.281840in}{2.288209in}}%
\pgfpathcurveto{\pgfqpoint{2.270790in}{2.288209in}}{\pgfqpoint{2.260191in}{2.283818in}}{\pgfqpoint{2.252377in}{2.276005in}}%
\pgfpathcurveto{\pgfqpoint{2.244564in}{2.268191in}}{\pgfqpoint{2.240174in}{2.257592in}}{\pgfqpoint{2.240174in}{2.246542in}}%
\pgfpathcurveto{\pgfqpoint{2.240174in}{2.235492in}}{\pgfqpoint{2.244564in}{2.224893in}}{\pgfqpoint{2.252377in}{2.217079in}}%
\pgfpathcurveto{\pgfqpoint{2.260191in}{2.209266in}}{\pgfqpoint{2.270790in}{2.204875in}}{\pgfqpoint{2.281840in}{2.204875in}}%
\pgfpathclose%
\pgfusepath{stroke,fill}%
\end{pgfscope}%
\begin{pgfscope}%
\pgfpathrectangle{\pgfqpoint{0.600000in}{0.600000in}}{\pgfqpoint{3.900000in}{3.900000in}}%
\pgfusepath{clip}%
\pgfsetbuttcap%
\pgfsetroundjoin%
\definecolor{currentfill}{rgb}{0.121569,0.466667,0.705882}%
\pgfsetfillcolor{currentfill}%
\pgfsetlinewidth{1.003750pt}%
\definecolor{currentstroke}{rgb}{0.121569,0.466667,0.705882}%
\pgfsetstrokecolor{currentstroke}%
\pgfsetdash{}{0pt}%
\pgfpathmoveto{\pgfqpoint{1.932777in}{2.354724in}}%
\pgfpathcurveto{\pgfqpoint{1.943827in}{2.354724in}}{\pgfqpoint{1.954426in}{2.359114in}}{\pgfqpoint{1.962240in}{2.366928in}}%
\pgfpathcurveto{\pgfqpoint{1.970054in}{2.374741in}}{\pgfqpoint{1.974444in}{2.385340in}}{\pgfqpoint{1.974444in}{2.396391in}}%
\pgfpathcurveto{\pgfqpoint{1.974444in}{2.407441in}}{\pgfqpoint{1.970054in}{2.418040in}}{\pgfqpoint{1.962240in}{2.425853in}}%
\pgfpathcurveto{\pgfqpoint{1.954426in}{2.433667in}}{\pgfqpoint{1.943827in}{2.438057in}}{\pgfqpoint{1.932777in}{2.438057in}}%
\pgfpathcurveto{\pgfqpoint{1.921727in}{2.438057in}}{\pgfqpoint{1.911128in}{2.433667in}}{\pgfqpoint{1.903314in}{2.425853in}}%
\pgfpathcurveto{\pgfqpoint{1.895501in}{2.418040in}}{\pgfqpoint{1.891110in}{2.407441in}}{\pgfqpoint{1.891110in}{2.396391in}}%
\pgfpathcurveto{\pgfqpoint{1.891110in}{2.385340in}}{\pgfqpoint{1.895501in}{2.374741in}}{\pgfqpoint{1.903314in}{2.366928in}}%
\pgfpathcurveto{\pgfqpoint{1.911128in}{2.359114in}}{\pgfqpoint{1.921727in}{2.354724in}}{\pgfqpoint{1.932777in}{2.354724in}}%
\pgfpathclose%
\pgfusepath{stroke,fill}%
\end{pgfscope}%
\begin{pgfscope}%
\pgfpathrectangle{\pgfqpoint{0.600000in}{0.600000in}}{\pgfqpoint{3.900000in}{3.900000in}}%
\pgfusepath{clip}%
\pgfsetbuttcap%
\pgfsetroundjoin%
\definecolor{currentfill}{rgb}{0.121569,0.466667,0.705882}%
\pgfsetfillcolor{currentfill}%
\pgfsetlinewidth{1.003750pt}%
\definecolor{currentstroke}{rgb}{0.121569,0.466667,0.705882}%
\pgfsetstrokecolor{currentstroke}%
\pgfsetdash{}{0pt}%
\pgfpathmoveto{\pgfqpoint{2.411532in}{2.579801in}}%
\pgfpathcurveto{\pgfqpoint{2.422582in}{2.579801in}}{\pgfqpoint{2.433181in}{2.584192in}}{\pgfqpoint{2.440995in}{2.592005in}}%
\pgfpathcurveto{\pgfqpoint{2.448808in}{2.599819in}}{\pgfqpoint{2.453199in}{2.610418in}}{\pgfqpoint{2.453199in}{2.621468in}}%
\pgfpathcurveto{\pgfqpoint{2.453199in}{2.632518in}}{\pgfqpoint{2.448808in}{2.643117in}}{\pgfqpoint{2.440995in}{2.650931in}}%
\pgfpathcurveto{\pgfqpoint{2.433181in}{2.658744in}}{\pgfqpoint{2.422582in}{2.663135in}}{\pgfqpoint{2.411532in}{2.663135in}}%
\pgfpathcurveto{\pgfqpoint{2.400482in}{2.663135in}}{\pgfqpoint{2.389883in}{2.658744in}}{\pgfqpoint{2.382069in}{2.650931in}}%
\pgfpathcurveto{\pgfqpoint{2.374255in}{2.643117in}}{\pgfqpoint{2.369865in}{2.632518in}}{\pgfqpoint{2.369865in}{2.621468in}}%
\pgfpathcurveto{\pgfqpoint{2.369865in}{2.610418in}}{\pgfqpoint{2.374255in}{2.599819in}}{\pgfqpoint{2.382069in}{2.592005in}}%
\pgfpathcurveto{\pgfqpoint{2.389883in}{2.584192in}}{\pgfqpoint{2.400482in}{2.579801in}}{\pgfqpoint{2.411532in}{2.579801in}}%
\pgfpathclose%
\pgfusepath{stroke,fill}%
\end{pgfscope}%
\begin{pgfscope}%
\pgfpathrectangle{\pgfqpoint{0.600000in}{0.600000in}}{\pgfqpoint{3.900000in}{3.900000in}}%
\pgfusepath{clip}%
\pgfsetbuttcap%
\pgfsetroundjoin%
\definecolor{currentfill}{rgb}{0.121569,0.466667,0.705882}%
\pgfsetfillcolor{currentfill}%
\pgfsetlinewidth{1.003750pt}%
\definecolor{currentstroke}{rgb}{0.121569,0.466667,0.705882}%
\pgfsetstrokecolor{currentstroke}%
\pgfsetdash{}{0pt}%
\pgfpathmoveto{\pgfqpoint{2.699820in}{2.592387in}}%
\pgfpathcurveto{\pgfqpoint{2.710870in}{2.592387in}}{\pgfqpoint{2.721469in}{2.596777in}}{\pgfqpoint{2.729282in}{2.604591in}}%
\pgfpathcurveto{\pgfqpoint{2.737096in}{2.612404in}}{\pgfqpoint{2.741486in}{2.623003in}}{\pgfqpoint{2.741486in}{2.634054in}}%
\pgfpathcurveto{\pgfqpoint{2.741486in}{2.645104in}}{\pgfqpoint{2.737096in}{2.655703in}}{\pgfqpoint{2.729282in}{2.663516in}}%
\pgfpathcurveto{\pgfqpoint{2.721469in}{2.671330in}}{\pgfqpoint{2.710870in}{2.675720in}}{\pgfqpoint{2.699820in}{2.675720in}}%
\pgfpathcurveto{\pgfqpoint{2.688770in}{2.675720in}}{\pgfqpoint{2.678170in}{2.671330in}}{\pgfqpoint{2.670357in}{2.663516in}}%
\pgfpathcurveto{\pgfqpoint{2.662543in}{2.655703in}}{\pgfqpoint{2.658153in}{2.645104in}}{\pgfqpoint{2.658153in}{2.634054in}}%
\pgfpathcurveto{\pgfqpoint{2.658153in}{2.623003in}}{\pgfqpoint{2.662543in}{2.612404in}}{\pgfqpoint{2.670357in}{2.604591in}}%
\pgfpathcurveto{\pgfqpoint{2.678170in}{2.596777in}}{\pgfqpoint{2.688770in}{2.592387in}}{\pgfqpoint{2.699820in}{2.592387in}}%
\pgfpathclose%
\pgfusepath{stroke,fill}%
\end{pgfscope}%
\begin{pgfscope}%
\pgfpathrectangle{\pgfqpoint{0.600000in}{0.600000in}}{\pgfqpoint{3.900000in}{3.900000in}}%
\pgfusepath{clip}%
\pgfsetbuttcap%
\pgfsetroundjoin%
\definecolor{currentfill}{rgb}{0.121569,0.466667,0.705882}%
\pgfsetfillcolor{currentfill}%
\pgfsetlinewidth{1.003750pt}%
\definecolor{currentstroke}{rgb}{0.121569,0.466667,0.705882}%
\pgfsetstrokecolor{currentstroke}%
\pgfsetdash{}{0pt}%
\pgfpathmoveto{\pgfqpoint{2.054866in}{3.233335in}}%
\pgfpathcurveto{\pgfqpoint{2.065916in}{3.233335in}}{\pgfqpoint{2.076515in}{3.237725in}}{\pgfqpoint{2.084329in}{3.245539in}}%
\pgfpathcurveto{\pgfqpoint{2.092142in}{3.253353in}}{\pgfqpoint{2.096533in}{3.263952in}}{\pgfqpoint{2.096533in}{3.275002in}}%
\pgfpathcurveto{\pgfqpoint{2.096533in}{3.286052in}}{\pgfqpoint{2.092142in}{3.296651in}}{\pgfqpoint{2.084329in}{3.304465in}}%
\pgfpathcurveto{\pgfqpoint{2.076515in}{3.312278in}}{\pgfqpoint{2.065916in}{3.316668in}}{\pgfqpoint{2.054866in}{3.316668in}}%
\pgfpathcurveto{\pgfqpoint{2.043816in}{3.316668in}}{\pgfqpoint{2.033217in}{3.312278in}}{\pgfqpoint{2.025403in}{3.304465in}}%
\pgfpathcurveto{\pgfqpoint{2.017590in}{3.296651in}}{\pgfqpoint{2.013199in}{3.286052in}}{\pgfqpoint{2.013199in}{3.275002in}}%
\pgfpathcurveto{\pgfqpoint{2.013199in}{3.263952in}}{\pgfqpoint{2.017590in}{3.253353in}}{\pgfqpoint{2.025403in}{3.245539in}}%
\pgfpathcurveto{\pgfqpoint{2.033217in}{3.237725in}}{\pgfqpoint{2.043816in}{3.233335in}}{\pgfqpoint{2.054866in}{3.233335in}}%
\pgfpathclose%
\pgfusepath{stroke,fill}%
\end{pgfscope}%
\begin{pgfscope}%
\pgfpathrectangle{\pgfqpoint{0.600000in}{0.600000in}}{\pgfqpoint{3.900000in}{3.900000in}}%
\pgfusepath{clip}%
\pgfsetbuttcap%
\pgfsetroundjoin%
\definecolor{currentfill}{rgb}{0.121569,0.466667,0.705882}%
\pgfsetfillcolor{currentfill}%
\pgfsetlinewidth{1.003750pt}%
\definecolor{currentstroke}{rgb}{0.121569,0.466667,0.705882}%
\pgfsetstrokecolor{currentstroke}%
\pgfsetdash{}{0pt}%
\pgfpathmoveto{\pgfqpoint{2.790701in}{2.961071in}}%
\pgfpathcurveto{\pgfqpoint{2.801751in}{2.961071in}}{\pgfqpoint{2.812350in}{2.965461in}}{\pgfqpoint{2.820164in}{2.973274in}}%
\pgfpathcurveto{\pgfqpoint{2.827978in}{2.981088in}}{\pgfqpoint{2.832368in}{2.991687in}}{\pgfqpoint{2.832368in}{3.002737in}}%
\pgfpathcurveto{\pgfqpoint{2.832368in}{3.013787in}}{\pgfqpoint{2.827978in}{3.024386in}}{\pgfqpoint{2.820164in}{3.032200in}}%
\pgfpathcurveto{\pgfqpoint{2.812350in}{3.040014in}}{\pgfqpoint{2.801751in}{3.044404in}}{\pgfqpoint{2.790701in}{3.044404in}}%
\pgfpathcurveto{\pgfqpoint{2.779651in}{3.044404in}}{\pgfqpoint{2.769052in}{3.040014in}}{\pgfqpoint{2.761239in}{3.032200in}}%
\pgfpathcurveto{\pgfqpoint{2.753425in}{3.024386in}}{\pgfqpoint{2.749035in}{3.013787in}}{\pgfqpoint{2.749035in}{3.002737in}}%
\pgfpathcurveto{\pgfqpoint{2.749035in}{2.991687in}}{\pgfqpoint{2.753425in}{2.981088in}}{\pgfqpoint{2.761239in}{2.973274in}}%
\pgfpathcurveto{\pgfqpoint{2.769052in}{2.965461in}}{\pgfqpoint{2.779651in}{2.961071in}}{\pgfqpoint{2.790701in}{2.961071in}}%
\pgfpathclose%
\pgfusepath{stroke,fill}%
\end{pgfscope}%
\begin{pgfscope}%
\pgfpathrectangle{\pgfqpoint{0.600000in}{0.600000in}}{\pgfqpoint{3.900000in}{3.900000in}}%
\pgfusepath{clip}%
\pgfsetbuttcap%
\pgfsetroundjoin%
\definecolor{currentfill}{rgb}{0.121569,0.466667,0.705882}%
\pgfsetfillcolor{currentfill}%
\pgfsetlinewidth{1.003750pt}%
\definecolor{currentstroke}{rgb}{0.121569,0.466667,0.705882}%
\pgfsetstrokecolor{currentstroke}%
\pgfsetdash{}{0pt}%
\pgfpathmoveto{\pgfqpoint{2.600058in}{2.661263in}}%
\pgfpathcurveto{\pgfqpoint{2.611109in}{2.661263in}}{\pgfqpoint{2.621708in}{2.665653in}}{\pgfqpoint{2.629521in}{2.673467in}}%
\pgfpathcurveto{\pgfqpoint{2.637335in}{2.681280in}}{\pgfqpoint{2.641725in}{2.691879in}}{\pgfqpoint{2.641725in}{2.702929in}}%
\pgfpathcurveto{\pgfqpoint{2.641725in}{2.713980in}}{\pgfqpoint{2.637335in}{2.724579in}}{\pgfqpoint{2.629521in}{2.732392in}}%
\pgfpathcurveto{\pgfqpoint{2.621708in}{2.740206in}}{\pgfqpoint{2.611109in}{2.744596in}}{\pgfqpoint{2.600058in}{2.744596in}}%
\pgfpathcurveto{\pgfqpoint{2.589008in}{2.744596in}}{\pgfqpoint{2.578409in}{2.740206in}}{\pgfqpoint{2.570596in}{2.732392in}}%
\pgfpathcurveto{\pgfqpoint{2.562782in}{2.724579in}}{\pgfqpoint{2.558392in}{2.713980in}}{\pgfqpoint{2.558392in}{2.702929in}}%
\pgfpathcurveto{\pgfqpoint{2.558392in}{2.691879in}}{\pgfqpoint{2.562782in}{2.681280in}}{\pgfqpoint{2.570596in}{2.673467in}}%
\pgfpathcurveto{\pgfqpoint{2.578409in}{2.665653in}}{\pgfqpoint{2.589008in}{2.661263in}}{\pgfqpoint{2.600058in}{2.661263in}}%
\pgfpathclose%
\pgfusepath{stroke,fill}%
\end{pgfscope}%
\begin{pgfscope}%
\pgfpathrectangle{\pgfqpoint{0.600000in}{0.600000in}}{\pgfqpoint{3.900000in}{3.900000in}}%
\pgfusepath{clip}%
\pgfsetbuttcap%
\pgfsetroundjoin%
\definecolor{currentfill}{rgb}{0.121569,0.466667,0.705882}%
\pgfsetfillcolor{currentfill}%
\pgfsetlinewidth{1.003750pt}%
\definecolor{currentstroke}{rgb}{0.121569,0.466667,0.705882}%
\pgfsetstrokecolor{currentstroke}%
\pgfsetdash{}{0pt}%
\pgfpathmoveto{\pgfqpoint{1.930639in}{3.443591in}}%
\pgfpathcurveto{\pgfqpoint{1.941689in}{3.443591in}}{\pgfqpoint{1.952288in}{3.447981in}}{\pgfqpoint{1.960102in}{3.455795in}}%
\pgfpathcurveto{\pgfqpoint{1.967915in}{3.463608in}}{\pgfqpoint{1.972306in}{3.474207in}}{\pgfqpoint{1.972306in}{3.485257in}}%
\pgfpathcurveto{\pgfqpoint{1.972306in}{3.496308in}}{\pgfqpoint{1.967915in}{3.506907in}}{\pgfqpoint{1.960102in}{3.514720in}}%
\pgfpathcurveto{\pgfqpoint{1.952288in}{3.522534in}}{\pgfqpoint{1.941689in}{3.526924in}}{\pgfqpoint{1.930639in}{3.526924in}}%
\pgfpathcurveto{\pgfqpoint{1.919589in}{3.526924in}}{\pgfqpoint{1.908990in}{3.522534in}}{\pgfqpoint{1.901176in}{3.514720in}}%
\pgfpathcurveto{\pgfqpoint{1.893362in}{3.506907in}}{\pgfqpoint{1.888972in}{3.496308in}}{\pgfqpoint{1.888972in}{3.485257in}}%
\pgfpathcurveto{\pgfqpoint{1.888972in}{3.474207in}}{\pgfqpoint{1.893362in}{3.463608in}}{\pgfqpoint{1.901176in}{3.455795in}}%
\pgfpathcurveto{\pgfqpoint{1.908990in}{3.447981in}}{\pgfqpoint{1.919589in}{3.443591in}}{\pgfqpoint{1.930639in}{3.443591in}}%
\pgfpathclose%
\pgfusepath{stroke,fill}%
\end{pgfscope}%
\begin{pgfscope}%
\pgfpathrectangle{\pgfqpoint{0.600000in}{0.600000in}}{\pgfqpoint{3.900000in}{3.900000in}}%
\pgfusepath{clip}%
\pgfsetbuttcap%
\pgfsetroundjoin%
\definecolor{currentfill}{rgb}{0.121569,0.466667,0.705882}%
\pgfsetfillcolor{currentfill}%
\pgfsetlinewidth{1.003750pt}%
\definecolor{currentstroke}{rgb}{0.121569,0.466667,0.705882}%
\pgfsetstrokecolor{currentstroke}%
\pgfsetdash{}{0pt}%
\pgfpathmoveto{\pgfqpoint{2.804968in}{2.803316in}}%
\pgfpathcurveto{\pgfqpoint{2.816018in}{2.803316in}}{\pgfqpoint{2.826617in}{2.807707in}}{\pgfqpoint{2.834430in}{2.815520in}}%
\pgfpathcurveto{\pgfqpoint{2.842244in}{2.823334in}}{\pgfqpoint{2.846634in}{2.833933in}}{\pgfqpoint{2.846634in}{2.844983in}}%
\pgfpathcurveto{\pgfqpoint{2.846634in}{2.856033in}}{\pgfqpoint{2.842244in}{2.866632in}}{\pgfqpoint{2.834430in}{2.874446in}}%
\pgfpathcurveto{\pgfqpoint{2.826617in}{2.882259in}}{\pgfqpoint{2.816018in}{2.886650in}}{\pgfqpoint{2.804968in}{2.886650in}}%
\pgfpathcurveto{\pgfqpoint{2.793918in}{2.886650in}}{\pgfqpoint{2.783319in}{2.882259in}}{\pgfqpoint{2.775505in}{2.874446in}}%
\pgfpathcurveto{\pgfqpoint{2.767691in}{2.866632in}}{\pgfqpoint{2.763301in}{2.856033in}}{\pgfqpoint{2.763301in}{2.844983in}}%
\pgfpathcurveto{\pgfqpoint{2.763301in}{2.833933in}}{\pgfqpoint{2.767691in}{2.823334in}}{\pgfqpoint{2.775505in}{2.815520in}}%
\pgfpathcurveto{\pgfqpoint{2.783319in}{2.807707in}}{\pgfqpoint{2.793918in}{2.803316in}}{\pgfqpoint{2.804968in}{2.803316in}}%
\pgfpathclose%
\pgfusepath{stroke,fill}%
\end{pgfscope}%
\begin{pgfscope}%
\pgfpathrectangle{\pgfqpoint{0.600000in}{0.600000in}}{\pgfqpoint{3.900000in}{3.900000in}}%
\pgfusepath{clip}%
\pgfsetbuttcap%
\pgfsetroundjoin%
\definecolor{currentfill}{rgb}{0.121569,0.466667,0.705882}%
\pgfsetfillcolor{currentfill}%
\pgfsetlinewidth{1.003750pt}%
\definecolor{currentstroke}{rgb}{0.121569,0.466667,0.705882}%
\pgfsetstrokecolor{currentstroke}%
\pgfsetdash{}{0pt}%
\pgfpathmoveto{\pgfqpoint{2.136437in}{2.980437in}}%
\pgfpathcurveto{\pgfqpoint{2.147488in}{2.980437in}}{\pgfqpoint{2.158087in}{2.984827in}}{\pgfqpoint{2.165900in}{2.992640in}}%
\pgfpathcurveto{\pgfqpoint{2.173714in}{3.000454in}}{\pgfqpoint{2.178104in}{3.011053in}}{\pgfqpoint{2.178104in}{3.022103in}}%
\pgfpathcurveto{\pgfqpoint{2.178104in}{3.033153in}}{\pgfqpoint{2.173714in}{3.043752in}}{\pgfqpoint{2.165900in}{3.051566in}}%
\pgfpathcurveto{\pgfqpoint{2.158087in}{3.059380in}}{\pgfqpoint{2.147488in}{3.063770in}}{\pgfqpoint{2.136437in}{3.063770in}}%
\pgfpathcurveto{\pgfqpoint{2.125387in}{3.063770in}}{\pgfqpoint{2.114788in}{3.059380in}}{\pgfqpoint{2.106975in}{3.051566in}}%
\pgfpathcurveto{\pgfqpoint{2.099161in}{3.043752in}}{\pgfqpoint{2.094771in}{3.033153in}}{\pgfqpoint{2.094771in}{3.022103in}}%
\pgfpathcurveto{\pgfqpoint{2.094771in}{3.011053in}}{\pgfqpoint{2.099161in}{3.000454in}}{\pgfqpoint{2.106975in}{2.992640in}}%
\pgfpathcurveto{\pgfqpoint{2.114788in}{2.984827in}}{\pgfqpoint{2.125387in}{2.980437in}}{\pgfqpoint{2.136437in}{2.980437in}}%
\pgfpathclose%
\pgfusepath{stroke,fill}%
\end{pgfscope}%
\begin{pgfscope}%
\pgfpathrectangle{\pgfqpoint{0.600000in}{0.600000in}}{\pgfqpoint{3.900000in}{3.900000in}}%
\pgfusepath{clip}%
\pgfsetbuttcap%
\pgfsetroundjoin%
\definecolor{currentfill}{rgb}{0.121569,0.466667,0.705882}%
\pgfsetfillcolor{currentfill}%
\pgfsetlinewidth{1.003750pt}%
\definecolor{currentstroke}{rgb}{0.121569,0.466667,0.705882}%
\pgfsetstrokecolor{currentstroke}%
\pgfsetdash{}{0pt}%
\pgfpathmoveto{\pgfqpoint{2.212637in}{1.447243in}}%
\pgfpathcurveto{\pgfqpoint{2.223687in}{1.447243in}}{\pgfqpoint{2.234286in}{1.451633in}}{\pgfqpoint{2.242100in}{1.459447in}}%
\pgfpathcurveto{\pgfqpoint{2.249914in}{1.467260in}}{\pgfqpoint{2.254304in}{1.477859in}}{\pgfqpoint{2.254304in}{1.488910in}}%
\pgfpathcurveto{\pgfqpoint{2.254304in}{1.499960in}}{\pgfqpoint{2.249914in}{1.510559in}}{\pgfqpoint{2.242100in}{1.518372in}}%
\pgfpathcurveto{\pgfqpoint{2.234286in}{1.526186in}}{\pgfqpoint{2.223687in}{1.530576in}}{\pgfqpoint{2.212637in}{1.530576in}}%
\pgfpathcurveto{\pgfqpoint{2.201587in}{1.530576in}}{\pgfqpoint{2.190988in}{1.526186in}}{\pgfqpoint{2.183175in}{1.518372in}}%
\pgfpathcurveto{\pgfqpoint{2.175361in}{1.510559in}}{\pgfqpoint{2.170971in}{1.499960in}}{\pgfqpoint{2.170971in}{1.488910in}}%
\pgfpathcurveto{\pgfqpoint{2.170971in}{1.477859in}}{\pgfqpoint{2.175361in}{1.467260in}}{\pgfqpoint{2.183175in}{1.459447in}}%
\pgfpathcurveto{\pgfqpoint{2.190988in}{1.451633in}}{\pgfqpoint{2.201587in}{1.447243in}}{\pgfqpoint{2.212637in}{1.447243in}}%
\pgfpathclose%
\pgfusepath{stroke,fill}%
\end{pgfscope}%
\begin{pgfscope}%
\pgfpathrectangle{\pgfqpoint{0.600000in}{0.600000in}}{\pgfqpoint{3.900000in}{3.900000in}}%
\pgfusepath{clip}%
\pgfsetbuttcap%
\pgfsetroundjoin%
\definecolor{currentfill}{rgb}{0.121569,0.466667,0.705882}%
\pgfsetfillcolor{currentfill}%
\pgfsetlinewidth{1.003750pt}%
\definecolor{currentstroke}{rgb}{0.121569,0.466667,0.705882}%
\pgfsetstrokecolor{currentstroke}%
\pgfsetdash{}{0pt}%
\pgfpathmoveto{\pgfqpoint{3.323163in}{1.602290in}}%
\pgfpathcurveto{\pgfqpoint{3.334213in}{1.602290in}}{\pgfqpoint{3.344812in}{1.606680in}}{\pgfqpoint{3.352626in}{1.614494in}}%
\pgfpathcurveto{\pgfqpoint{3.360440in}{1.622307in}}{\pgfqpoint{3.364830in}{1.632906in}}{\pgfqpoint{3.364830in}{1.643956in}}%
\pgfpathcurveto{\pgfqpoint{3.364830in}{1.655007in}}{\pgfqpoint{3.360440in}{1.665606in}}{\pgfqpoint{3.352626in}{1.673419in}}%
\pgfpathcurveto{\pgfqpoint{3.344812in}{1.681233in}}{\pgfqpoint{3.334213in}{1.685623in}}{\pgfqpoint{3.323163in}{1.685623in}}%
\pgfpathcurveto{\pgfqpoint{3.312113in}{1.685623in}}{\pgfqpoint{3.301514in}{1.681233in}}{\pgfqpoint{3.293700in}{1.673419in}}%
\pgfpathcurveto{\pgfqpoint{3.285887in}{1.665606in}}{\pgfqpoint{3.281497in}{1.655007in}}{\pgfqpoint{3.281497in}{1.643956in}}%
\pgfpathcurveto{\pgfqpoint{3.281497in}{1.632906in}}{\pgfqpoint{3.285887in}{1.622307in}}{\pgfqpoint{3.293700in}{1.614494in}}%
\pgfpathcurveto{\pgfqpoint{3.301514in}{1.606680in}}{\pgfqpoint{3.312113in}{1.602290in}}{\pgfqpoint{3.323163in}{1.602290in}}%
\pgfpathclose%
\pgfusepath{stroke,fill}%
\end{pgfscope}%
\begin{pgfscope}%
\pgfpathrectangle{\pgfqpoint{0.600000in}{0.600000in}}{\pgfqpoint{3.900000in}{3.900000in}}%
\pgfusepath{clip}%
\pgfsetbuttcap%
\pgfsetroundjoin%
\definecolor{currentfill}{rgb}{0.121569,0.466667,0.705882}%
\pgfsetfillcolor{currentfill}%
\pgfsetlinewidth{1.003750pt}%
\definecolor{currentstroke}{rgb}{0.121569,0.466667,0.705882}%
\pgfsetstrokecolor{currentstroke}%
\pgfsetdash{}{0pt}%
\pgfpathmoveto{\pgfqpoint{2.562518in}{1.826170in}}%
\pgfpathcurveto{\pgfqpoint{2.573568in}{1.826170in}}{\pgfqpoint{2.584167in}{1.830560in}}{\pgfqpoint{2.591981in}{1.838374in}}%
\pgfpathcurveto{\pgfqpoint{2.599795in}{1.846187in}}{\pgfqpoint{2.604185in}{1.856787in}}{\pgfqpoint{2.604185in}{1.867837in}}%
\pgfpathcurveto{\pgfqpoint{2.604185in}{1.878887in}}{\pgfqpoint{2.599795in}{1.889486in}}{\pgfqpoint{2.591981in}{1.897299in}}%
\pgfpathcurveto{\pgfqpoint{2.584167in}{1.905113in}}{\pgfqpoint{2.573568in}{1.909503in}}{\pgfqpoint{2.562518in}{1.909503in}}%
\pgfpathcurveto{\pgfqpoint{2.551468in}{1.909503in}}{\pgfqpoint{2.540869in}{1.905113in}}{\pgfqpoint{2.533055in}{1.897299in}}%
\pgfpathcurveto{\pgfqpoint{2.525242in}{1.889486in}}{\pgfqpoint{2.520851in}{1.878887in}}{\pgfqpoint{2.520851in}{1.867837in}}%
\pgfpathcurveto{\pgfqpoint{2.520851in}{1.856787in}}{\pgfqpoint{2.525242in}{1.846187in}}{\pgfqpoint{2.533055in}{1.838374in}}%
\pgfpathcurveto{\pgfqpoint{2.540869in}{1.830560in}}{\pgfqpoint{2.551468in}{1.826170in}}{\pgfqpoint{2.562518in}{1.826170in}}%
\pgfpathclose%
\pgfusepath{stroke,fill}%
\end{pgfscope}%
\begin{pgfscope}%
\pgfpathrectangle{\pgfqpoint{0.600000in}{0.600000in}}{\pgfqpoint{3.900000in}{3.900000in}}%
\pgfusepath{clip}%
\pgfsetbuttcap%
\pgfsetroundjoin%
\definecolor{currentfill}{rgb}{0.121569,0.466667,0.705882}%
\pgfsetfillcolor{currentfill}%
\pgfsetlinewidth{1.003750pt}%
\definecolor{currentstroke}{rgb}{0.121569,0.466667,0.705882}%
\pgfsetstrokecolor{currentstroke}%
\pgfsetdash{}{0pt}%
\pgfpathmoveto{\pgfqpoint{2.184456in}{1.938770in}}%
\pgfpathcurveto{\pgfqpoint{2.195506in}{1.938770in}}{\pgfqpoint{2.206105in}{1.943160in}}{\pgfqpoint{2.213919in}{1.950973in}}%
\pgfpathcurveto{\pgfqpoint{2.221733in}{1.958787in}}{\pgfqpoint{2.226123in}{1.969386in}}{\pgfqpoint{2.226123in}{1.980436in}}%
\pgfpathcurveto{\pgfqpoint{2.226123in}{1.991486in}}{\pgfqpoint{2.221733in}{2.002085in}}{\pgfqpoint{2.213919in}{2.009899in}}%
\pgfpathcurveto{\pgfqpoint{2.206105in}{2.017713in}}{\pgfqpoint{2.195506in}{2.022103in}}{\pgfqpoint{2.184456in}{2.022103in}}%
\pgfpathcurveto{\pgfqpoint{2.173406in}{2.022103in}}{\pgfqpoint{2.162807in}{2.017713in}}{\pgfqpoint{2.154993in}{2.009899in}}%
\pgfpathcurveto{\pgfqpoint{2.147180in}{2.002085in}}{\pgfqpoint{2.142790in}{1.991486in}}{\pgfqpoint{2.142790in}{1.980436in}}%
\pgfpathcurveto{\pgfqpoint{2.142790in}{1.969386in}}{\pgfqpoint{2.147180in}{1.958787in}}{\pgfqpoint{2.154993in}{1.950973in}}%
\pgfpathcurveto{\pgfqpoint{2.162807in}{1.943160in}}{\pgfqpoint{2.173406in}{1.938770in}}{\pgfqpoint{2.184456in}{1.938770in}}%
\pgfpathclose%
\pgfusepath{stroke,fill}%
\end{pgfscope}%
\begin{pgfscope}%
\pgfpathrectangle{\pgfqpoint{0.600000in}{0.600000in}}{\pgfqpoint{3.900000in}{3.900000in}}%
\pgfusepath{clip}%
\pgfsetbuttcap%
\pgfsetroundjoin%
\definecolor{currentfill}{rgb}{0.121569,0.466667,0.705882}%
\pgfsetfillcolor{currentfill}%
\pgfsetlinewidth{1.003750pt}%
\definecolor{currentstroke}{rgb}{0.121569,0.466667,0.705882}%
\pgfsetstrokecolor{currentstroke}%
\pgfsetdash{}{0pt}%
\pgfpathmoveto{\pgfqpoint{1.903265in}{1.833376in}}%
\pgfpathcurveto{\pgfqpoint{1.914316in}{1.833376in}}{\pgfqpoint{1.924915in}{1.837766in}}{\pgfqpoint{1.932728in}{1.845580in}}%
\pgfpathcurveto{\pgfqpoint{1.940542in}{1.853394in}}{\pgfqpoint{1.944932in}{1.863993in}}{\pgfqpoint{1.944932in}{1.875043in}}%
\pgfpathcurveto{\pgfqpoint{1.944932in}{1.886093in}}{\pgfqpoint{1.940542in}{1.896692in}}{\pgfqpoint{1.932728in}{1.904505in}}%
\pgfpathcurveto{\pgfqpoint{1.924915in}{1.912319in}}{\pgfqpoint{1.914316in}{1.916709in}}{\pgfqpoint{1.903265in}{1.916709in}}%
\pgfpathcurveto{\pgfqpoint{1.892215in}{1.916709in}}{\pgfqpoint{1.881616in}{1.912319in}}{\pgfqpoint{1.873803in}{1.904505in}}%
\pgfpathcurveto{\pgfqpoint{1.865989in}{1.896692in}}{\pgfqpoint{1.861599in}{1.886093in}}{\pgfqpoint{1.861599in}{1.875043in}}%
\pgfpathcurveto{\pgfqpoint{1.861599in}{1.863993in}}{\pgfqpoint{1.865989in}{1.853394in}}{\pgfqpoint{1.873803in}{1.845580in}}%
\pgfpathcurveto{\pgfqpoint{1.881616in}{1.837766in}}{\pgfqpoint{1.892215in}{1.833376in}}{\pgfqpoint{1.903265in}{1.833376in}}%
\pgfpathclose%
\pgfusepath{stroke,fill}%
\end{pgfscope}%
\begin{pgfscope}%
\pgfpathrectangle{\pgfqpoint{0.600000in}{0.600000in}}{\pgfqpoint{3.900000in}{3.900000in}}%
\pgfusepath{clip}%
\pgfsetbuttcap%
\pgfsetroundjoin%
\definecolor{currentfill}{rgb}{0.121569,0.466667,0.705882}%
\pgfsetfillcolor{currentfill}%
\pgfsetlinewidth{1.003750pt}%
\definecolor{currentstroke}{rgb}{0.121569,0.466667,0.705882}%
\pgfsetstrokecolor{currentstroke}%
\pgfsetdash{}{0pt}%
\pgfpathmoveto{\pgfqpoint{2.473761in}{2.737990in}}%
\pgfpathcurveto{\pgfqpoint{2.484811in}{2.737990in}}{\pgfqpoint{2.495410in}{2.742381in}}{\pgfqpoint{2.503224in}{2.750194in}}%
\pgfpathcurveto{\pgfqpoint{2.511037in}{2.758008in}}{\pgfqpoint{2.515428in}{2.768607in}}{\pgfqpoint{2.515428in}{2.779657in}}%
\pgfpathcurveto{\pgfqpoint{2.515428in}{2.790707in}}{\pgfqpoint{2.511037in}{2.801306in}}{\pgfqpoint{2.503224in}{2.809120in}}%
\pgfpathcurveto{\pgfqpoint{2.495410in}{2.816933in}}{\pgfqpoint{2.484811in}{2.821324in}}{\pgfqpoint{2.473761in}{2.821324in}}%
\pgfpathcurveto{\pgfqpoint{2.462711in}{2.821324in}}{\pgfqpoint{2.452112in}{2.816933in}}{\pgfqpoint{2.444298in}{2.809120in}}%
\pgfpathcurveto{\pgfqpoint{2.436485in}{2.801306in}}{\pgfqpoint{2.432094in}{2.790707in}}{\pgfqpoint{2.432094in}{2.779657in}}%
\pgfpathcurveto{\pgfqpoint{2.432094in}{2.768607in}}{\pgfqpoint{2.436485in}{2.758008in}}{\pgfqpoint{2.444298in}{2.750194in}}%
\pgfpathcurveto{\pgfqpoint{2.452112in}{2.742381in}}{\pgfqpoint{2.462711in}{2.737990in}}{\pgfqpoint{2.473761in}{2.737990in}}%
\pgfpathclose%
\pgfusepath{stroke,fill}%
\end{pgfscope}%
\begin{pgfscope}%
\pgfpathrectangle{\pgfqpoint{0.600000in}{0.600000in}}{\pgfqpoint{3.900000in}{3.900000in}}%
\pgfusepath{clip}%
\pgfsetbuttcap%
\pgfsetroundjoin%
\definecolor{currentfill}{rgb}{0.121569,0.466667,0.705882}%
\pgfsetfillcolor{currentfill}%
\pgfsetlinewidth{1.003750pt}%
\definecolor{currentstroke}{rgb}{0.121569,0.466667,0.705882}%
\pgfsetstrokecolor{currentstroke}%
\pgfsetdash{}{0pt}%
\pgfpathmoveto{\pgfqpoint{2.059194in}{2.291653in}}%
\pgfpathcurveto{\pgfqpoint{2.070244in}{2.291653in}}{\pgfqpoint{2.080843in}{2.296043in}}{\pgfqpoint{2.088657in}{2.303857in}}%
\pgfpathcurveto{\pgfqpoint{2.096470in}{2.311671in}}{\pgfqpoint{2.100861in}{2.322270in}}{\pgfqpoint{2.100861in}{2.333320in}}%
\pgfpathcurveto{\pgfqpoint{2.100861in}{2.344370in}}{\pgfqpoint{2.096470in}{2.354969in}}{\pgfqpoint{2.088657in}{2.362783in}}%
\pgfpathcurveto{\pgfqpoint{2.080843in}{2.370596in}}{\pgfqpoint{2.070244in}{2.374986in}}{\pgfqpoint{2.059194in}{2.374986in}}%
\pgfpathcurveto{\pgfqpoint{2.048144in}{2.374986in}}{\pgfqpoint{2.037545in}{2.370596in}}{\pgfqpoint{2.029731in}{2.362783in}}%
\pgfpathcurveto{\pgfqpoint{2.021918in}{2.354969in}}{\pgfqpoint{2.017527in}{2.344370in}}{\pgfqpoint{2.017527in}{2.333320in}}%
\pgfpathcurveto{\pgfqpoint{2.017527in}{2.322270in}}{\pgfqpoint{2.021918in}{2.311671in}}{\pgfqpoint{2.029731in}{2.303857in}}%
\pgfpathcurveto{\pgfqpoint{2.037545in}{2.296043in}}{\pgfqpoint{2.048144in}{2.291653in}}{\pgfqpoint{2.059194in}{2.291653in}}%
\pgfpathclose%
\pgfusepath{stroke,fill}%
\end{pgfscope}%
\begin{pgfscope}%
\pgfpathrectangle{\pgfqpoint{0.600000in}{0.600000in}}{\pgfqpoint{3.900000in}{3.900000in}}%
\pgfusepath{clip}%
\pgfsetbuttcap%
\pgfsetroundjoin%
\definecolor{currentfill}{rgb}{0.121569,0.466667,0.705882}%
\pgfsetfillcolor{currentfill}%
\pgfsetlinewidth{1.003750pt}%
\definecolor{currentstroke}{rgb}{0.121569,0.466667,0.705882}%
\pgfsetstrokecolor{currentstroke}%
\pgfsetdash{}{0pt}%
\pgfpathmoveto{\pgfqpoint{1.672960in}{2.635053in}}%
\pgfpathcurveto{\pgfqpoint{1.684010in}{2.635053in}}{\pgfqpoint{1.694609in}{2.639444in}}{\pgfqpoint{1.702422in}{2.647257in}}%
\pgfpathcurveto{\pgfqpoint{1.710236in}{2.655071in}}{\pgfqpoint{1.714626in}{2.665670in}}{\pgfqpoint{1.714626in}{2.676720in}}%
\pgfpathcurveto{\pgfqpoint{1.714626in}{2.687770in}}{\pgfqpoint{1.710236in}{2.698369in}}{\pgfqpoint{1.702422in}{2.706183in}}%
\pgfpathcurveto{\pgfqpoint{1.694609in}{2.713996in}}{\pgfqpoint{1.684010in}{2.718387in}}{\pgfqpoint{1.672960in}{2.718387in}}%
\pgfpathcurveto{\pgfqpoint{1.661909in}{2.718387in}}{\pgfqpoint{1.651310in}{2.713996in}}{\pgfqpoint{1.643497in}{2.706183in}}%
\pgfpathcurveto{\pgfqpoint{1.635683in}{2.698369in}}{\pgfqpoint{1.631293in}{2.687770in}}{\pgfqpoint{1.631293in}{2.676720in}}%
\pgfpathcurveto{\pgfqpoint{1.631293in}{2.665670in}}{\pgfqpoint{1.635683in}{2.655071in}}{\pgfqpoint{1.643497in}{2.647257in}}%
\pgfpathcurveto{\pgfqpoint{1.651310in}{2.639444in}}{\pgfqpoint{1.661909in}{2.635053in}}{\pgfqpoint{1.672960in}{2.635053in}}%
\pgfpathclose%
\pgfusepath{stroke,fill}%
\end{pgfscope}%
\begin{pgfscope}%
\pgfpathrectangle{\pgfqpoint{0.600000in}{0.600000in}}{\pgfqpoint{3.900000in}{3.900000in}}%
\pgfusepath{clip}%
\pgfsetbuttcap%
\pgfsetroundjoin%
\definecolor{currentfill}{rgb}{0.121569,0.466667,0.705882}%
\pgfsetfillcolor{currentfill}%
\pgfsetlinewidth{1.003750pt}%
\definecolor{currentstroke}{rgb}{0.121569,0.466667,0.705882}%
\pgfsetstrokecolor{currentstroke}%
\pgfsetdash{}{0pt}%
\pgfpathmoveto{\pgfqpoint{2.032959in}{3.518085in}}%
\pgfpathcurveto{\pgfqpoint{2.044009in}{3.518085in}}{\pgfqpoint{2.054608in}{3.522476in}}{\pgfqpoint{2.062422in}{3.530289in}}%
\pgfpathcurveto{\pgfqpoint{2.070235in}{3.538103in}}{\pgfqpoint{2.074626in}{3.548702in}}{\pgfqpoint{2.074626in}{3.559752in}}%
\pgfpathcurveto{\pgfqpoint{2.074626in}{3.570802in}}{\pgfqpoint{2.070235in}{3.581401in}}{\pgfqpoint{2.062422in}{3.589215in}}%
\pgfpathcurveto{\pgfqpoint{2.054608in}{3.597028in}}{\pgfqpoint{2.044009in}{3.601419in}}{\pgfqpoint{2.032959in}{3.601419in}}%
\pgfpathcurveto{\pgfqpoint{2.021909in}{3.601419in}}{\pgfqpoint{2.011310in}{3.597028in}}{\pgfqpoint{2.003496in}{3.589215in}}%
\pgfpathcurveto{\pgfqpoint{1.995683in}{3.581401in}}{\pgfqpoint{1.991292in}{3.570802in}}{\pgfqpoint{1.991292in}{3.559752in}}%
\pgfpathcurveto{\pgfqpoint{1.991292in}{3.548702in}}{\pgfqpoint{1.995683in}{3.538103in}}{\pgfqpoint{2.003496in}{3.530289in}}%
\pgfpathcurveto{\pgfqpoint{2.011310in}{3.522476in}}{\pgfqpoint{2.021909in}{3.518085in}}{\pgfqpoint{2.032959in}{3.518085in}}%
\pgfpathclose%
\pgfusepath{stroke,fill}%
\end{pgfscope}%
\begin{pgfscope}%
\pgfpathrectangle{\pgfqpoint{0.600000in}{0.600000in}}{\pgfqpoint{3.900000in}{3.900000in}}%
\pgfusepath{clip}%
\pgfsetbuttcap%
\pgfsetroundjoin%
\definecolor{currentfill}{rgb}{0.121569,0.466667,0.705882}%
\pgfsetfillcolor{currentfill}%
\pgfsetlinewidth{1.003750pt}%
\definecolor{currentstroke}{rgb}{0.121569,0.466667,0.705882}%
\pgfsetstrokecolor{currentstroke}%
\pgfsetdash{}{0pt}%
\pgfpathmoveto{\pgfqpoint{2.894840in}{2.944614in}}%
\pgfpathcurveto{\pgfqpoint{2.905890in}{2.944614in}}{\pgfqpoint{2.916489in}{2.949005in}}{\pgfqpoint{2.924303in}{2.956818in}}%
\pgfpathcurveto{\pgfqpoint{2.932117in}{2.964632in}}{\pgfqpoint{2.936507in}{2.975231in}}{\pgfqpoint{2.936507in}{2.986281in}}%
\pgfpathcurveto{\pgfqpoint{2.936507in}{2.997331in}}{\pgfqpoint{2.932117in}{3.007930in}}{\pgfqpoint{2.924303in}{3.015744in}}%
\pgfpathcurveto{\pgfqpoint{2.916489in}{3.023557in}}{\pgfqpoint{2.905890in}{3.027948in}}{\pgfqpoint{2.894840in}{3.027948in}}%
\pgfpathcurveto{\pgfqpoint{2.883790in}{3.027948in}}{\pgfqpoint{2.873191in}{3.023557in}}{\pgfqpoint{2.865377in}{3.015744in}}%
\pgfpathcurveto{\pgfqpoint{2.857564in}{3.007930in}}{\pgfqpoint{2.853173in}{2.997331in}}{\pgfqpoint{2.853173in}{2.986281in}}%
\pgfpathcurveto{\pgfqpoint{2.853173in}{2.975231in}}{\pgfqpoint{2.857564in}{2.964632in}}{\pgfqpoint{2.865377in}{2.956818in}}%
\pgfpathcurveto{\pgfqpoint{2.873191in}{2.949005in}}{\pgfqpoint{2.883790in}{2.944614in}}{\pgfqpoint{2.894840in}{2.944614in}}%
\pgfpathclose%
\pgfusepath{stroke,fill}%
\end{pgfscope}%
\begin{pgfscope}%
\pgfpathrectangle{\pgfqpoint{0.600000in}{0.600000in}}{\pgfqpoint{3.900000in}{3.900000in}}%
\pgfusepath{clip}%
\pgfsetbuttcap%
\pgfsetroundjoin%
\definecolor{currentfill}{rgb}{0.121569,0.466667,0.705882}%
\pgfsetfillcolor{currentfill}%
\pgfsetlinewidth{1.003750pt}%
\definecolor{currentstroke}{rgb}{0.121569,0.466667,0.705882}%
\pgfsetstrokecolor{currentstroke}%
\pgfsetdash{}{0pt}%
\pgfpathmoveto{\pgfqpoint{3.090131in}{2.950147in}}%
\pgfpathcurveto{\pgfqpoint{3.101181in}{2.950147in}}{\pgfqpoint{3.111780in}{2.954537in}}{\pgfqpoint{3.119594in}{2.962351in}}%
\pgfpathcurveto{\pgfqpoint{3.127408in}{2.970165in}}{\pgfqpoint{3.131798in}{2.980764in}}{\pgfqpoint{3.131798in}{2.991814in}}%
\pgfpathcurveto{\pgfqpoint{3.131798in}{3.002864in}}{\pgfqpoint{3.127408in}{3.013463in}}{\pgfqpoint{3.119594in}{3.021276in}}%
\pgfpathcurveto{\pgfqpoint{3.111780in}{3.029090in}}{\pgfqpoint{3.101181in}{3.033480in}}{\pgfqpoint{3.090131in}{3.033480in}}%
\pgfpathcurveto{\pgfqpoint{3.079081in}{3.033480in}}{\pgfqpoint{3.068482in}{3.029090in}}{\pgfqpoint{3.060668in}{3.021276in}}%
\pgfpathcurveto{\pgfqpoint{3.052855in}{3.013463in}}{\pgfqpoint{3.048464in}{3.002864in}}{\pgfqpoint{3.048464in}{2.991814in}}%
\pgfpathcurveto{\pgfqpoint{3.048464in}{2.980764in}}{\pgfqpoint{3.052855in}{2.970165in}}{\pgfqpoint{3.060668in}{2.962351in}}%
\pgfpathcurveto{\pgfqpoint{3.068482in}{2.954537in}}{\pgfqpoint{3.079081in}{2.950147in}}{\pgfqpoint{3.090131in}{2.950147in}}%
\pgfpathclose%
\pgfusepath{stroke,fill}%
\end{pgfscope}%
\begin{pgfscope}%
\pgfpathrectangle{\pgfqpoint{0.600000in}{0.600000in}}{\pgfqpoint{3.900000in}{3.900000in}}%
\pgfusepath{clip}%
\pgfsetbuttcap%
\pgfsetroundjoin%
\definecolor{currentfill}{rgb}{0.121569,0.466667,0.705882}%
\pgfsetfillcolor{currentfill}%
\pgfsetlinewidth{1.003750pt}%
\definecolor{currentstroke}{rgb}{0.121569,0.466667,0.705882}%
\pgfsetstrokecolor{currentstroke}%
\pgfsetdash{}{0pt}%
\pgfpathmoveto{\pgfqpoint{3.073820in}{2.207024in}}%
\pgfpathcurveto{\pgfqpoint{3.084870in}{2.207024in}}{\pgfqpoint{3.095469in}{2.211415in}}{\pgfqpoint{3.103282in}{2.219228in}}%
\pgfpathcurveto{\pgfqpoint{3.111096in}{2.227042in}}{\pgfqpoint{3.115486in}{2.237641in}}{\pgfqpoint{3.115486in}{2.248691in}}%
\pgfpathcurveto{\pgfqpoint{3.115486in}{2.259741in}}{\pgfqpoint{3.111096in}{2.270340in}}{\pgfqpoint{3.103282in}{2.278154in}}%
\pgfpathcurveto{\pgfqpoint{3.095469in}{2.285967in}}{\pgfqpoint{3.084870in}{2.290358in}}{\pgfqpoint{3.073820in}{2.290358in}}%
\pgfpathcurveto{\pgfqpoint{3.062769in}{2.290358in}}{\pgfqpoint{3.052170in}{2.285967in}}{\pgfqpoint{3.044357in}{2.278154in}}%
\pgfpathcurveto{\pgfqpoint{3.036543in}{2.270340in}}{\pgfqpoint{3.032153in}{2.259741in}}{\pgfqpoint{3.032153in}{2.248691in}}%
\pgfpathcurveto{\pgfqpoint{3.032153in}{2.237641in}}{\pgfqpoint{3.036543in}{2.227042in}}{\pgfqpoint{3.044357in}{2.219228in}}%
\pgfpathcurveto{\pgfqpoint{3.052170in}{2.211415in}}{\pgfqpoint{3.062769in}{2.207024in}}{\pgfqpoint{3.073820in}{2.207024in}}%
\pgfpathclose%
\pgfusepath{stroke,fill}%
\end{pgfscope}%
\begin{pgfscope}%
\pgfpathrectangle{\pgfqpoint{0.600000in}{0.600000in}}{\pgfqpoint{3.900000in}{3.900000in}}%
\pgfusepath{clip}%
\pgfsetbuttcap%
\pgfsetroundjoin%
\definecolor{currentfill}{rgb}{0.121569,0.466667,0.705882}%
\pgfsetfillcolor{currentfill}%
\pgfsetlinewidth{1.003750pt}%
\definecolor{currentstroke}{rgb}{0.121569,0.466667,0.705882}%
\pgfsetstrokecolor{currentstroke}%
\pgfsetdash{}{0pt}%
\pgfpathmoveto{\pgfqpoint{2.368548in}{1.958973in}}%
\pgfpathcurveto{\pgfqpoint{2.379598in}{1.958973in}}{\pgfqpoint{2.390197in}{1.963363in}}{\pgfqpoint{2.398011in}{1.971177in}}%
\pgfpathcurveto{\pgfqpoint{2.405824in}{1.978990in}}{\pgfqpoint{2.410215in}{1.989589in}}{\pgfqpoint{2.410215in}{2.000639in}}%
\pgfpathcurveto{\pgfqpoint{2.410215in}{2.011689in}}{\pgfqpoint{2.405824in}{2.022288in}}{\pgfqpoint{2.398011in}{2.030102in}}%
\pgfpathcurveto{\pgfqpoint{2.390197in}{2.037916in}}{\pgfqpoint{2.379598in}{2.042306in}}{\pgfqpoint{2.368548in}{2.042306in}}%
\pgfpathcurveto{\pgfqpoint{2.357498in}{2.042306in}}{\pgfqpoint{2.346899in}{2.037916in}}{\pgfqpoint{2.339085in}{2.030102in}}%
\pgfpathcurveto{\pgfqpoint{2.331272in}{2.022288in}}{\pgfqpoint{2.326881in}{2.011689in}}{\pgfqpoint{2.326881in}{2.000639in}}%
\pgfpathcurveto{\pgfqpoint{2.326881in}{1.989589in}}{\pgfqpoint{2.331272in}{1.978990in}}{\pgfqpoint{2.339085in}{1.971177in}}%
\pgfpathcurveto{\pgfqpoint{2.346899in}{1.963363in}}{\pgfqpoint{2.357498in}{1.958973in}}{\pgfqpoint{2.368548in}{1.958973in}}%
\pgfpathclose%
\pgfusepath{stroke,fill}%
\end{pgfscope}%
\begin{pgfscope}%
\pgfpathrectangle{\pgfqpoint{0.600000in}{0.600000in}}{\pgfqpoint{3.900000in}{3.900000in}}%
\pgfusepath{clip}%
\pgfsetbuttcap%
\pgfsetroundjoin%
\definecolor{currentfill}{rgb}{0.121569,0.466667,0.705882}%
\pgfsetfillcolor{currentfill}%
\pgfsetlinewidth{1.003750pt}%
\definecolor{currentstroke}{rgb}{0.121569,0.466667,0.705882}%
\pgfsetstrokecolor{currentstroke}%
\pgfsetdash{}{0pt}%
\pgfpathmoveto{\pgfqpoint{2.829230in}{2.394861in}}%
\pgfpathcurveto{\pgfqpoint{2.840280in}{2.394861in}}{\pgfqpoint{2.850879in}{2.399251in}}{\pgfqpoint{2.858693in}{2.407065in}}%
\pgfpathcurveto{\pgfqpoint{2.866507in}{2.414879in}}{\pgfqpoint{2.870897in}{2.425478in}}{\pgfqpoint{2.870897in}{2.436528in}}%
\pgfpathcurveto{\pgfqpoint{2.870897in}{2.447578in}}{\pgfqpoint{2.866507in}{2.458177in}}{\pgfqpoint{2.858693in}{2.465991in}}%
\pgfpathcurveto{\pgfqpoint{2.850879in}{2.473804in}}{\pgfqpoint{2.840280in}{2.478194in}}{\pgfqpoint{2.829230in}{2.478194in}}%
\pgfpathcurveto{\pgfqpoint{2.818180in}{2.478194in}}{\pgfqpoint{2.807581in}{2.473804in}}{\pgfqpoint{2.799768in}{2.465991in}}%
\pgfpathcurveto{\pgfqpoint{2.791954in}{2.458177in}}{\pgfqpoint{2.787564in}{2.447578in}}{\pgfqpoint{2.787564in}{2.436528in}}%
\pgfpathcurveto{\pgfqpoint{2.787564in}{2.425478in}}{\pgfqpoint{2.791954in}{2.414879in}}{\pgfqpoint{2.799768in}{2.407065in}}%
\pgfpathcurveto{\pgfqpoint{2.807581in}{2.399251in}}{\pgfqpoint{2.818180in}{2.394861in}}{\pgfqpoint{2.829230in}{2.394861in}}%
\pgfpathclose%
\pgfusepath{stroke,fill}%
\end{pgfscope}%
\begin{pgfscope}%
\pgfpathrectangle{\pgfqpoint{0.600000in}{0.600000in}}{\pgfqpoint{3.900000in}{3.900000in}}%
\pgfusepath{clip}%
\pgfsetbuttcap%
\pgfsetroundjoin%
\definecolor{currentfill}{rgb}{0.121569,0.466667,0.705882}%
\pgfsetfillcolor{currentfill}%
\pgfsetlinewidth{1.003750pt}%
\definecolor{currentstroke}{rgb}{0.121569,0.466667,0.705882}%
\pgfsetstrokecolor{currentstroke}%
\pgfsetdash{}{0pt}%
\pgfpathmoveto{\pgfqpoint{1.935265in}{2.830843in}}%
\pgfpathcurveto{\pgfqpoint{1.946315in}{2.830843in}}{\pgfqpoint{1.956914in}{2.835233in}}{\pgfqpoint{1.964728in}{2.843047in}}%
\pgfpathcurveto{\pgfqpoint{1.972542in}{2.850861in}}{\pgfqpoint{1.976932in}{2.861460in}}{\pgfqpoint{1.976932in}{2.872510in}}%
\pgfpathcurveto{\pgfqpoint{1.976932in}{2.883560in}}{\pgfqpoint{1.972542in}{2.894159in}}{\pgfqpoint{1.964728in}{2.901973in}}%
\pgfpathcurveto{\pgfqpoint{1.956914in}{2.909786in}}{\pgfqpoint{1.946315in}{2.914176in}}{\pgfqpoint{1.935265in}{2.914176in}}%
\pgfpathcurveto{\pgfqpoint{1.924215in}{2.914176in}}{\pgfqpoint{1.913616in}{2.909786in}}{\pgfqpoint{1.905802in}{2.901973in}}%
\pgfpathcurveto{\pgfqpoint{1.897989in}{2.894159in}}{\pgfqpoint{1.893599in}{2.883560in}}{\pgfqpoint{1.893599in}{2.872510in}}%
\pgfpathcurveto{\pgfqpoint{1.893599in}{2.861460in}}{\pgfqpoint{1.897989in}{2.850861in}}{\pgfqpoint{1.905802in}{2.843047in}}%
\pgfpathcurveto{\pgfqpoint{1.913616in}{2.835233in}}{\pgfqpoint{1.924215in}{2.830843in}}{\pgfqpoint{1.935265in}{2.830843in}}%
\pgfpathclose%
\pgfusepath{stroke,fill}%
\end{pgfscope}%
\begin{pgfscope}%
\pgfpathrectangle{\pgfqpoint{0.600000in}{0.600000in}}{\pgfqpoint{3.900000in}{3.900000in}}%
\pgfusepath{clip}%
\pgfsetbuttcap%
\pgfsetroundjoin%
\definecolor{currentfill}{rgb}{0.121569,0.466667,0.705882}%
\pgfsetfillcolor{currentfill}%
\pgfsetlinewidth{1.003750pt}%
\definecolor{currentstroke}{rgb}{0.121569,0.466667,0.705882}%
\pgfsetstrokecolor{currentstroke}%
\pgfsetdash{}{0pt}%
\pgfpathmoveto{\pgfqpoint{2.573114in}{1.807303in}}%
\pgfpathcurveto{\pgfqpoint{2.584164in}{1.807303in}}{\pgfqpoint{2.594763in}{1.811693in}}{\pgfqpoint{2.602577in}{1.819507in}}%
\pgfpathcurveto{\pgfqpoint{2.610390in}{1.827320in}}{\pgfqpoint{2.614781in}{1.837919in}}{\pgfqpoint{2.614781in}{1.848970in}}%
\pgfpathcurveto{\pgfqpoint{2.614781in}{1.860020in}}{\pgfqpoint{2.610390in}{1.870619in}}{\pgfqpoint{2.602577in}{1.878432in}}%
\pgfpathcurveto{\pgfqpoint{2.594763in}{1.886246in}}{\pgfqpoint{2.584164in}{1.890636in}}{\pgfqpoint{2.573114in}{1.890636in}}%
\pgfpathcurveto{\pgfqpoint{2.562064in}{1.890636in}}{\pgfqpoint{2.551465in}{1.886246in}}{\pgfqpoint{2.543651in}{1.878432in}}%
\pgfpathcurveto{\pgfqpoint{2.535838in}{1.870619in}}{\pgfqpoint{2.531447in}{1.860020in}}{\pgfqpoint{2.531447in}{1.848970in}}%
\pgfpathcurveto{\pgfqpoint{2.531447in}{1.837919in}}{\pgfqpoint{2.535838in}{1.827320in}}{\pgfqpoint{2.543651in}{1.819507in}}%
\pgfpathcurveto{\pgfqpoint{2.551465in}{1.811693in}}{\pgfqpoint{2.562064in}{1.807303in}}{\pgfqpoint{2.573114in}{1.807303in}}%
\pgfpathclose%
\pgfusepath{stroke,fill}%
\end{pgfscope}%
\begin{pgfscope}%
\pgfpathrectangle{\pgfqpoint{0.600000in}{0.600000in}}{\pgfqpoint{3.900000in}{3.900000in}}%
\pgfusepath{clip}%
\pgfsetbuttcap%
\pgfsetroundjoin%
\definecolor{currentfill}{rgb}{0.121569,0.466667,0.705882}%
\pgfsetfillcolor{currentfill}%
\pgfsetlinewidth{1.003750pt}%
\definecolor{currentstroke}{rgb}{0.121569,0.466667,0.705882}%
\pgfsetstrokecolor{currentstroke}%
\pgfsetdash{}{0pt}%
\pgfpathmoveto{\pgfqpoint{2.555953in}{1.617258in}}%
\pgfpathcurveto{\pgfqpoint{2.567003in}{1.617258in}}{\pgfqpoint{2.577602in}{1.621649in}}{\pgfqpoint{2.585416in}{1.629462in}}%
\pgfpathcurveto{\pgfqpoint{2.593230in}{1.637276in}}{\pgfqpoint{2.597620in}{1.647875in}}{\pgfqpoint{2.597620in}{1.658925in}}%
\pgfpathcurveto{\pgfqpoint{2.597620in}{1.669975in}}{\pgfqpoint{2.593230in}{1.680574in}}{\pgfqpoint{2.585416in}{1.688388in}}%
\pgfpathcurveto{\pgfqpoint{2.577602in}{1.696201in}}{\pgfqpoint{2.567003in}{1.700592in}}{\pgfqpoint{2.555953in}{1.700592in}}%
\pgfpathcurveto{\pgfqpoint{2.544903in}{1.700592in}}{\pgfqpoint{2.534304in}{1.696201in}}{\pgfqpoint{2.526491in}{1.688388in}}%
\pgfpathcurveto{\pgfqpoint{2.518677in}{1.680574in}}{\pgfqpoint{2.514287in}{1.669975in}}{\pgfqpoint{2.514287in}{1.658925in}}%
\pgfpathcurveto{\pgfqpoint{2.514287in}{1.647875in}}{\pgfqpoint{2.518677in}{1.637276in}}{\pgfqpoint{2.526491in}{1.629462in}}%
\pgfpathcurveto{\pgfqpoint{2.534304in}{1.621649in}}{\pgfqpoint{2.544903in}{1.617258in}}{\pgfqpoint{2.555953in}{1.617258in}}%
\pgfpathclose%
\pgfusepath{stroke,fill}%
\end{pgfscope}%
\begin{pgfscope}%
\pgfpathrectangle{\pgfqpoint{0.600000in}{0.600000in}}{\pgfqpoint{3.900000in}{3.900000in}}%
\pgfusepath{clip}%
\pgfsetbuttcap%
\pgfsetroundjoin%
\definecolor{currentfill}{rgb}{0.121569,0.466667,0.705882}%
\pgfsetfillcolor{currentfill}%
\pgfsetlinewidth{1.003750pt}%
\definecolor{currentstroke}{rgb}{0.121569,0.466667,0.705882}%
\pgfsetstrokecolor{currentstroke}%
\pgfsetdash{}{0pt}%
\pgfpathmoveto{\pgfqpoint{2.310615in}{2.539972in}}%
\pgfpathcurveto{\pgfqpoint{2.321665in}{2.539972in}}{\pgfqpoint{2.332264in}{2.544362in}}{\pgfqpoint{2.340077in}{2.552176in}}%
\pgfpathcurveto{\pgfqpoint{2.347891in}{2.559989in}}{\pgfqpoint{2.352281in}{2.570588in}}{\pgfqpoint{2.352281in}{2.581638in}}%
\pgfpathcurveto{\pgfqpoint{2.352281in}{2.592689in}}{\pgfqpoint{2.347891in}{2.603288in}}{\pgfqpoint{2.340077in}{2.611101in}}%
\pgfpathcurveto{\pgfqpoint{2.332264in}{2.618915in}}{\pgfqpoint{2.321665in}{2.623305in}}{\pgfqpoint{2.310615in}{2.623305in}}%
\pgfpathcurveto{\pgfqpoint{2.299564in}{2.623305in}}{\pgfqpoint{2.288965in}{2.618915in}}{\pgfqpoint{2.281152in}{2.611101in}}%
\pgfpathcurveto{\pgfqpoint{2.273338in}{2.603288in}}{\pgfqpoint{2.268948in}{2.592689in}}{\pgfqpoint{2.268948in}{2.581638in}}%
\pgfpathcurveto{\pgfqpoint{2.268948in}{2.570588in}}{\pgfqpoint{2.273338in}{2.559989in}}{\pgfqpoint{2.281152in}{2.552176in}}%
\pgfpathcurveto{\pgfqpoint{2.288965in}{2.544362in}}{\pgfqpoint{2.299564in}{2.539972in}}{\pgfqpoint{2.310615in}{2.539972in}}%
\pgfpathclose%
\pgfusepath{stroke,fill}%
\end{pgfscope}%
\begin{pgfscope}%
\pgfpathrectangle{\pgfqpoint{0.600000in}{0.600000in}}{\pgfqpoint{3.900000in}{3.900000in}}%
\pgfusepath{clip}%
\pgfsetbuttcap%
\pgfsetroundjoin%
\definecolor{currentfill}{rgb}{0.121569,0.466667,0.705882}%
\pgfsetfillcolor{currentfill}%
\pgfsetlinewidth{1.003750pt}%
\definecolor{currentstroke}{rgb}{0.121569,0.466667,0.705882}%
\pgfsetstrokecolor{currentstroke}%
\pgfsetdash{}{0pt}%
\pgfpathmoveto{\pgfqpoint{2.025756in}{2.821347in}}%
\pgfpathcurveto{\pgfqpoint{2.036806in}{2.821347in}}{\pgfqpoint{2.047406in}{2.825737in}}{\pgfqpoint{2.055219in}{2.833551in}}%
\pgfpathcurveto{\pgfqpoint{2.063033in}{2.841365in}}{\pgfqpoint{2.067423in}{2.851964in}}{\pgfqpoint{2.067423in}{2.863014in}}%
\pgfpathcurveto{\pgfqpoint{2.067423in}{2.874064in}}{\pgfqpoint{2.063033in}{2.884663in}}{\pgfqpoint{2.055219in}{2.892476in}}%
\pgfpathcurveto{\pgfqpoint{2.047406in}{2.900290in}}{\pgfqpoint{2.036806in}{2.904680in}}{\pgfqpoint{2.025756in}{2.904680in}}%
\pgfpathcurveto{\pgfqpoint{2.014706in}{2.904680in}}{\pgfqpoint{2.004107in}{2.900290in}}{\pgfqpoint{1.996294in}{2.892476in}}%
\pgfpathcurveto{\pgfqpoint{1.988480in}{2.884663in}}{\pgfqpoint{1.984090in}{2.874064in}}{\pgfqpoint{1.984090in}{2.863014in}}%
\pgfpathcurveto{\pgfqpoint{1.984090in}{2.851964in}}{\pgfqpoint{1.988480in}{2.841365in}}{\pgfqpoint{1.996294in}{2.833551in}}%
\pgfpathcurveto{\pgfqpoint{2.004107in}{2.825737in}}{\pgfqpoint{2.014706in}{2.821347in}}{\pgfqpoint{2.025756in}{2.821347in}}%
\pgfpathclose%
\pgfusepath{stroke,fill}%
\end{pgfscope}%
\begin{pgfscope}%
\pgfpathrectangle{\pgfqpoint{0.600000in}{0.600000in}}{\pgfqpoint{3.900000in}{3.900000in}}%
\pgfusepath{clip}%
\pgfsetbuttcap%
\pgfsetroundjoin%
\definecolor{currentfill}{rgb}{0.121569,0.466667,0.705882}%
\pgfsetfillcolor{currentfill}%
\pgfsetlinewidth{1.003750pt}%
\definecolor{currentstroke}{rgb}{0.121569,0.466667,0.705882}%
\pgfsetstrokecolor{currentstroke}%
\pgfsetdash{}{0pt}%
\pgfpathmoveto{\pgfqpoint{2.376921in}{2.465198in}}%
\pgfpathcurveto{\pgfqpoint{2.387971in}{2.465198in}}{\pgfqpoint{2.398570in}{2.469589in}}{\pgfqpoint{2.406384in}{2.477402in}}%
\pgfpathcurveto{\pgfqpoint{2.414198in}{2.485216in}}{\pgfqpoint{2.418588in}{2.495815in}}{\pgfqpoint{2.418588in}{2.506865in}}%
\pgfpathcurveto{\pgfqpoint{2.418588in}{2.517915in}}{\pgfqpoint{2.414198in}{2.528514in}}{\pgfqpoint{2.406384in}{2.536328in}}%
\pgfpathcurveto{\pgfqpoint{2.398570in}{2.544142in}}{\pgfqpoint{2.387971in}{2.548532in}}{\pgfqpoint{2.376921in}{2.548532in}}%
\pgfpathcurveto{\pgfqpoint{2.365871in}{2.548532in}}{\pgfqpoint{2.355272in}{2.544142in}}{\pgfqpoint{2.347458in}{2.536328in}}%
\pgfpathcurveto{\pgfqpoint{2.339645in}{2.528514in}}{\pgfqpoint{2.335255in}{2.517915in}}{\pgfqpoint{2.335255in}{2.506865in}}%
\pgfpathcurveto{\pgfqpoint{2.335255in}{2.495815in}}{\pgfqpoint{2.339645in}{2.485216in}}{\pgfqpoint{2.347458in}{2.477402in}}%
\pgfpathcurveto{\pgfqpoint{2.355272in}{2.469589in}}{\pgfqpoint{2.365871in}{2.465198in}}{\pgfqpoint{2.376921in}{2.465198in}}%
\pgfpathclose%
\pgfusepath{stroke,fill}%
\end{pgfscope}%
\begin{pgfscope}%
\pgfpathrectangle{\pgfqpoint{0.600000in}{0.600000in}}{\pgfqpoint{3.900000in}{3.900000in}}%
\pgfusepath{clip}%
\pgfsetbuttcap%
\pgfsetroundjoin%
\definecolor{currentfill}{rgb}{0.121569,0.466667,0.705882}%
\pgfsetfillcolor{currentfill}%
\pgfsetlinewidth{1.003750pt}%
\definecolor{currentstroke}{rgb}{0.121569,0.466667,0.705882}%
\pgfsetstrokecolor{currentstroke}%
\pgfsetdash{}{0pt}%
\pgfpathmoveto{\pgfqpoint{2.586046in}{1.588959in}}%
\pgfpathcurveto{\pgfqpoint{2.597096in}{1.588959in}}{\pgfqpoint{2.607695in}{1.593349in}}{\pgfqpoint{2.615508in}{1.601162in}}%
\pgfpathcurveto{\pgfqpoint{2.623322in}{1.608976in}}{\pgfqpoint{2.627712in}{1.619575in}}{\pgfqpoint{2.627712in}{1.630625in}}%
\pgfpathcurveto{\pgfqpoint{2.627712in}{1.641675in}}{\pgfqpoint{2.623322in}{1.652274in}}{\pgfqpoint{2.615508in}{1.660088in}}%
\pgfpathcurveto{\pgfqpoint{2.607695in}{1.667902in}}{\pgfqpoint{2.597096in}{1.672292in}}{\pgfqpoint{2.586046in}{1.672292in}}%
\pgfpathcurveto{\pgfqpoint{2.574996in}{1.672292in}}{\pgfqpoint{2.564397in}{1.667902in}}{\pgfqpoint{2.556583in}{1.660088in}}%
\pgfpathcurveto{\pgfqpoint{2.548769in}{1.652274in}}{\pgfqpoint{2.544379in}{1.641675in}}{\pgfqpoint{2.544379in}{1.630625in}}%
\pgfpathcurveto{\pgfqpoint{2.544379in}{1.619575in}}{\pgfqpoint{2.548769in}{1.608976in}}{\pgfqpoint{2.556583in}{1.601162in}}%
\pgfpathcurveto{\pgfqpoint{2.564397in}{1.593349in}}{\pgfqpoint{2.574996in}{1.588959in}}{\pgfqpoint{2.586046in}{1.588959in}}%
\pgfpathclose%
\pgfusepath{stroke,fill}%
\end{pgfscope}%
\begin{pgfscope}%
\pgfpathrectangle{\pgfqpoint{0.600000in}{0.600000in}}{\pgfqpoint{3.900000in}{3.900000in}}%
\pgfusepath{clip}%
\pgfsetbuttcap%
\pgfsetroundjoin%
\definecolor{currentfill}{rgb}{0.121569,0.466667,0.705882}%
\pgfsetfillcolor{currentfill}%
\pgfsetlinewidth{1.003750pt}%
\definecolor{currentstroke}{rgb}{0.121569,0.466667,0.705882}%
\pgfsetstrokecolor{currentstroke}%
\pgfsetdash{}{0pt}%
\pgfpathmoveto{\pgfqpoint{2.798427in}{3.283876in}}%
\pgfpathcurveto{\pgfqpoint{2.809477in}{3.283876in}}{\pgfqpoint{2.820076in}{3.288266in}}{\pgfqpoint{2.827890in}{3.296080in}}%
\pgfpathcurveto{\pgfqpoint{2.835704in}{3.303893in}}{\pgfqpoint{2.840094in}{3.314492in}}{\pgfqpoint{2.840094in}{3.325542in}}%
\pgfpathcurveto{\pgfqpoint{2.840094in}{3.336593in}}{\pgfqpoint{2.835704in}{3.347192in}}{\pgfqpoint{2.827890in}{3.355005in}}%
\pgfpathcurveto{\pgfqpoint{2.820076in}{3.362819in}}{\pgfqpoint{2.809477in}{3.367209in}}{\pgfqpoint{2.798427in}{3.367209in}}%
\pgfpathcurveto{\pgfqpoint{2.787377in}{3.367209in}}{\pgfqpoint{2.776778in}{3.362819in}}{\pgfqpoint{2.768965in}{3.355005in}}%
\pgfpathcurveto{\pgfqpoint{2.761151in}{3.347192in}}{\pgfqpoint{2.756761in}{3.336593in}}{\pgfqpoint{2.756761in}{3.325542in}}%
\pgfpathcurveto{\pgfqpoint{2.756761in}{3.314492in}}{\pgfqpoint{2.761151in}{3.303893in}}{\pgfqpoint{2.768965in}{3.296080in}}%
\pgfpathcurveto{\pgfqpoint{2.776778in}{3.288266in}}{\pgfqpoint{2.787377in}{3.283876in}}{\pgfqpoint{2.798427in}{3.283876in}}%
\pgfpathclose%
\pgfusepath{stroke,fill}%
\end{pgfscope}%
\begin{pgfscope}%
\pgfpathrectangle{\pgfqpoint{0.600000in}{0.600000in}}{\pgfqpoint{3.900000in}{3.900000in}}%
\pgfusepath{clip}%
\pgfsetbuttcap%
\pgfsetroundjoin%
\definecolor{currentfill}{rgb}{0.121569,0.466667,0.705882}%
\pgfsetfillcolor{currentfill}%
\pgfsetlinewidth{1.003750pt}%
\definecolor{currentstroke}{rgb}{0.121569,0.466667,0.705882}%
\pgfsetstrokecolor{currentstroke}%
\pgfsetdash{}{0pt}%
\pgfpathmoveto{\pgfqpoint{2.087157in}{2.212335in}}%
\pgfpathcurveto{\pgfqpoint{2.098208in}{2.212335in}}{\pgfqpoint{2.108807in}{2.216725in}}{\pgfqpoint{2.116620in}{2.224539in}}%
\pgfpathcurveto{\pgfqpoint{2.124434in}{2.232352in}}{\pgfqpoint{2.128824in}{2.242951in}}{\pgfqpoint{2.128824in}{2.254002in}}%
\pgfpathcurveto{\pgfqpoint{2.128824in}{2.265052in}}{\pgfqpoint{2.124434in}{2.275651in}}{\pgfqpoint{2.116620in}{2.283464in}}%
\pgfpathcurveto{\pgfqpoint{2.108807in}{2.291278in}}{\pgfqpoint{2.098208in}{2.295668in}}{\pgfqpoint{2.087157in}{2.295668in}}%
\pgfpathcurveto{\pgfqpoint{2.076107in}{2.295668in}}{\pgfqpoint{2.065508in}{2.291278in}}{\pgfqpoint{2.057695in}{2.283464in}}%
\pgfpathcurveto{\pgfqpoint{2.049881in}{2.275651in}}{\pgfqpoint{2.045491in}{2.265052in}}{\pgfqpoint{2.045491in}{2.254002in}}%
\pgfpathcurveto{\pgfqpoint{2.045491in}{2.242951in}}{\pgfqpoint{2.049881in}{2.232352in}}{\pgfqpoint{2.057695in}{2.224539in}}%
\pgfpathcurveto{\pgfqpoint{2.065508in}{2.216725in}}{\pgfqpoint{2.076107in}{2.212335in}}{\pgfqpoint{2.087157in}{2.212335in}}%
\pgfpathclose%
\pgfusepath{stroke,fill}%
\end{pgfscope}%
\begin{pgfscope}%
\pgfpathrectangle{\pgfqpoint{0.600000in}{0.600000in}}{\pgfqpoint{3.900000in}{3.900000in}}%
\pgfusepath{clip}%
\pgfsetbuttcap%
\pgfsetroundjoin%
\definecolor{currentfill}{rgb}{0.121569,0.466667,0.705882}%
\pgfsetfillcolor{currentfill}%
\pgfsetlinewidth{1.003750pt}%
\definecolor{currentstroke}{rgb}{0.121569,0.466667,0.705882}%
\pgfsetstrokecolor{currentstroke}%
\pgfsetdash{}{0pt}%
\pgfpathmoveto{\pgfqpoint{2.397486in}{2.164796in}}%
\pgfpathcurveto{\pgfqpoint{2.408536in}{2.164796in}}{\pgfqpoint{2.419135in}{2.169187in}}{\pgfqpoint{2.426948in}{2.177000in}}%
\pgfpathcurveto{\pgfqpoint{2.434762in}{2.184814in}}{\pgfqpoint{2.439152in}{2.195413in}}{\pgfqpoint{2.439152in}{2.206463in}}%
\pgfpathcurveto{\pgfqpoint{2.439152in}{2.217513in}}{\pgfqpoint{2.434762in}{2.228112in}}{\pgfqpoint{2.426948in}{2.235926in}}%
\pgfpathcurveto{\pgfqpoint{2.419135in}{2.243740in}}{\pgfqpoint{2.408536in}{2.248130in}}{\pgfqpoint{2.397486in}{2.248130in}}%
\pgfpathcurveto{\pgfqpoint{2.386436in}{2.248130in}}{\pgfqpoint{2.375837in}{2.243740in}}{\pgfqpoint{2.368023in}{2.235926in}}%
\pgfpathcurveto{\pgfqpoint{2.360209in}{2.228112in}}{\pgfqpoint{2.355819in}{2.217513in}}{\pgfqpoint{2.355819in}{2.206463in}}%
\pgfpathcurveto{\pgfqpoint{2.355819in}{2.195413in}}{\pgfqpoint{2.360209in}{2.184814in}}{\pgfqpoint{2.368023in}{2.177000in}}%
\pgfpathcurveto{\pgfqpoint{2.375837in}{2.169187in}}{\pgfqpoint{2.386436in}{2.164796in}}{\pgfqpoint{2.397486in}{2.164796in}}%
\pgfpathclose%
\pgfusepath{stroke,fill}%
\end{pgfscope}%
\begin{pgfscope}%
\pgfpathrectangle{\pgfqpoint{0.600000in}{0.600000in}}{\pgfqpoint{3.900000in}{3.900000in}}%
\pgfusepath{clip}%
\pgfsetbuttcap%
\pgfsetroundjoin%
\definecolor{currentfill}{rgb}{0.121569,0.466667,0.705882}%
\pgfsetfillcolor{currentfill}%
\pgfsetlinewidth{1.003750pt}%
\definecolor{currentstroke}{rgb}{0.121569,0.466667,0.705882}%
\pgfsetstrokecolor{currentstroke}%
\pgfsetdash{}{0pt}%
\pgfpathmoveto{\pgfqpoint{2.054790in}{2.299681in}}%
\pgfpathcurveto{\pgfqpoint{2.065840in}{2.299681in}}{\pgfqpoint{2.076439in}{2.304071in}}{\pgfqpoint{2.084252in}{2.311885in}}%
\pgfpathcurveto{\pgfqpoint{2.092066in}{2.319699in}}{\pgfqpoint{2.096456in}{2.330298in}}{\pgfqpoint{2.096456in}{2.341348in}}%
\pgfpathcurveto{\pgfqpoint{2.096456in}{2.352398in}}{\pgfqpoint{2.092066in}{2.362997in}}{\pgfqpoint{2.084252in}{2.370811in}}%
\pgfpathcurveto{\pgfqpoint{2.076439in}{2.378624in}}{\pgfqpoint{2.065840in}{2.383014in}}{\pgfqpoint{2.054790in}{2.383014in}}%
\pgfpathcurveto{\pgfqpoint{2.043739in}{2.383014in}}{\pgfqpoint{2.033140in}{2.378624in}}{\pgfqpoint{2.025327in}{2.370811in}}%
\pgfpathcurveto{\pgfqpoint{2.017513in}{2.362997in}}{\pgfqpoint{2.013123in}{2.352398in}}{\pgfqpoint{2.013123in}{2.341348in}}%
\pgfpathcurveto{\pgfqpoint{2.013123in}{2.330298in}}{\pgfqpoint{2.017513in}{2.319699in}}{\pgfqpoint{2.025327in}{2.311885in}}%
\pgfpathcurveto{\pgfqpoint{2.033140in}{2.304071in}}{\pgfqpoint{2.043739in}{2.299681in}}{\pgfqpoint{2.054790in}{2.299681in}}%
\pgfpathclose%
\pgfusepath{stroke,fill}%
\end{pgfscope}%
\begin{pgfscope}%
\pgfpathrectangle{\pgfqpoint{0.600000in}{0.600000in}}{\pgfqpoint{3.900000in}{3.900000in}}%
\pgfusepath{clip}%
\pgfsetbuttcap%
\pgfsetroundjoin%
\definecolor{currentfill}{rgb}{0.121569,0.466667,0.705882}%
\pgfsetfillcolor{currentfill}%
\pgfsetlinewidth{1.003750pt}%
\definecolor{currentstroke}{rgb}{0.121569,0.466667,0.705882}%
\pgfsetstrokecolor{currentstroke}%
\pgfsetdash{}{0pt}%
\pgfpathmoveto{\pgfqpoint{2.071824in}{2.181460in}}%
\pgfpathcurveto{\pgfqpoint{2.082874in}{2.181460in}}{\pgfqpoint{2.093473in}{2.185850in}}{\pgfqpoint{2.101287in}{2.193664in}}%
\pgfpathcurveto{\pgfqpoint{2.109100in}{2.201477in}}{\pgfqpoint{2.113491in}{2.212076in}}{\pgfqpoint{2.113491in}{2.223126in}}%
\pgfpathcurveto{\pgfqpoint{2.113491in}{2.234177in}}{\pgfqpoint{2.109100in}{2.244776in}}{\pgfqpoint{2.101287in}{2.252589in}}%
\pgfpathcurveto{\pgfqpoint{2.093473in}{2.260403in}}{\pgfqpoint{2.082874in}{2.264793in}}{\pgfqpoint{2.071824in}{2.264793in}}%
\pgfpathcurveto{\pgfqpoint{2.060774in}{2.264793in}}{\pgfqpoint{2.050175in}{2.260403in}}{\pgfqpoint{2.042361in}{2.252589in}}%
\pgfpathcurveto{\pgfqpoint{2.034548in}{2.244776in}}{\pgfqpoint{2.030157in}{2.234177in}}{\pgfqpoint{2.030157in}{2.223126in}}%
\pgfpathcurveto{\pgfqpoint{2.030157in}{2.212076in}}{\pgfqpoint{2.034548in}{2.201477in}}{\pgfqpoint{2.042361in}{2.193664in}}%
\pgfpathcurveto{\pgfqpoint{2.050175in}{2.185850in}}{\pgfqpoint{2.060774in}{2.181460in}}{\pgfqpoint{2.071824in}{2.181460in}}%
\pgfpathclose%
\pgfusepath{stroke,fill}%
\end{pgfscope}%
\begin{pgfscope}%
\pgfpathrectangle{\pgfqpoint{0.600000in}{0.600000in}}{\pgfqpoint{3.900000in}{3.900000in}}%
\pgfusepath{clip}%
\pgfsetbuttcap%
\pgfsetroundjoin%
\definecolor{currentfill}{rgb}{0.121569,0.466667,0.705882}%
\pgfsetfillcolor{currentfill}%
\pgfsetlinewidth{1.003750pt}%
\definecolor{currentstroke}{rgb}{0.121569,0.466667,0.705882}%
\pgfsetstrokecolor{currentstroke}%
\pgfsetdash{}{0pt}%
\pgfpathmoveto{\pgfqpoint{2.801561in}{2.335775in}}%
\pgfpathcurveto{\pgfqpoint{2.812611in}{2.335775in}}{\pgfqpoint{2.823210in}{2.340165in}}{\pgfqpoint{2.831024in}{2.347979in}}%
\pgfpathcurveto{\pgfqpoint{2.838838in}{2.355793in}}{\pgfqpoint{2.843228in}{2.366392in}}{\pgfqpoint{2.843228in}{2.377442in}}%
\pgfpathcurveto{\pgfqpoint{2.843228in}{2.388492in}}{\pgfqpoint{2.838838in}{2.399091in}}{\pgfqpoint{2.831024in}{2.406905in}}%
\pgfpathcurveto{\pgfqpoint{2.823210in}{2.414718in}}{\pgfqpoint{2.812611in}{2.419108in}}{\pgfqpoint{2.801561in}{2.419108in}}%
\pgfpathcurveto{\pgfqpoint{2.790511in}{2.419108in}}{\pgfqpoint{2.779912in}{2.414718in}}{\pgfqpoint{2.772098in}{2.406905in}}%
\pgfpathcurveto{\pgfqpoint{2.764285in}{2.399091in}}{\pgfqpoint{2.759895in}{2.388492in}}{\pgfqpoint{2.759895in}{2.377442in}}%
\pgfpathcurveto{\pgfqpoint{2.759895in}{2.366392in}}{\pgfqpoint{2.764285in}{2.355793in}}{\pgfqpoint{2.772098in}{2.347979in}}%
\pgfpathcurveto{\pgfqpoint{2.779912in}{2.340165in}}{\pgfqpoint{2.790511in}{2.335775in}}{\pgfqpoint{2.801561in}{2.335775in}}%
\pgfpathclose%
\pgfusepath{stroke,fill}%
\end{pgfscope}%
\begin{pgfscope}%
\pgfpathrectangle{\pgfqpoint{0.600000in}{0.600000in}}{\pgfqpoint{3.900000in}{3.900000in}}%
\pgfusepath{clip}%
\pgfsetbuttcap%
\pgfsetroundjoin%
\definecolor{currentfill}{rgb}{0.121569,0.466667,0.705882}%
\pgfsetfillcolor{currentfill}%
\pgfsetlinewidth{1.003750pt}%
\definecolor{currentstroke}{rgb}{0.121569,0.466667,0.705882}%
\pgfsetstrokecolor{currentstroke}%
\pgfsetdash{}{0pt}%
\pgfpathmoveto{\pgfqpoint{2.019808in}{2.312686in}}%
\pgfpathcurveto{\pgfqpoint{2.030858in}{2.312686in}}{\pgfqpoint{2.041457in}{2.317076in}}{\pgfqpoint{2.049270in}{2.324890in}}%
\pgfpathcurveto{\pgfqpoint{2.057084in}{2.332704in}}{\pgfqpoint{2.061474in}{2.343303in}}{\pgfqpoint{2.061474in}{2.354353in}}%
\pgfpathcurveto{\pgfqpoint{2.061474in}{2.365403in}}{\pgfqpoint{2.057084in}{2.376002in}}{\pgfqpoint{2.049270in}{2.383816in}}%
\pgfpathcurveto{\pgfqpoint{2.041457in}{2.391629in}}{\pgfqpoint{2.030858in}{2.396019in}}{\pgfqpoint{2.019808in}{2.396019in}}%
\pgfpathcurveto{\pgfqpoint{2.008758in}{2.396019in}}{\pgfqpoint{1.998159in}{2.391629in}}{\pgfqpoint{1.990345in}{2.383816in}}%
\pgfpathcurveto{\pgfqpoint{1.982531in}{2.376002in}}{\pgfqpoint{1.978141in}{2.365403in}}{\pgfqpoint{1.978141in}{2.354353in}}%
\pgfpathcurveto{\pgfqpoint{1.978141in}{2.343303in}}{\pgfqpoint{1.982531in}{2.332704in}}{\pgfqpoint{1.990345in}{2.324890in}}%
\pgfpathcurveto{\pgfqpoint{1.998159in}{2.317076in}}{\pgfqpoint{2.008758in}{2.312686in}}{\pgfqpoint{2.019808in}{2.312686in}}%
\pgfpathclose%
\pgfusepath{stroke,fill}%
\end{pgfscope}%
\begin{pgfscope}%
\pgfpathrectangle{\pgfqpoint{0.600000in}{0.600000in}}{\pgfqpoint{3.900000in}{3.900000in}}%
\pgfusepath{clip}%
\pgfsetbuttcap%
\pgfsetroundjoin%
\definecolor{currentfill}{rgb}{0.121569,0.466667,0.705882}%
\pgfsetfillcolor{currentfill}%
\pgfsetlinewidth{1.003750pt}%
\definecolor{currentstroke}{rgb}{0.121569,0.466667,0.705882}%
\pgfsetstrokecolor{currentstroke}%
\pgfsetdash{}{0pt}%
\pgfpathmoveto{\pgfqpoint{2.650286in}{2.710772in}}%
\pgfpathcurveto{\pgfqpoint{2.661336in}{2.710772in}}{\pgfqpoint{2.671935in}{2.715162in}}{\pgfqpoint{2.679749in}{2.722976in}}%
\pgfpathcurveto{\pgfqpoint{2.687562in}{2.730789in}}{\pgfqpoint{2.691952in}{2.741388in}}{\pgfqpoint{2.691952in}{2.752439in}}%
\pgfpathcurveto{\pgfqpoint{2.691952in}{2.763489in}}{\pgfqpoint{2.687562in}{2.774088in}}{\pgfqpoint{2.679749in}{2.781901in}}%
\pgfpathcurveto{\pgfqpoint{2.671935in}{2.789715in}}{\pgfqpoint{2.661336in}{2.794105in}}{\pgfqpoint{2.650286in}{2.794105in}}%
\pgfpathcurveto{\pgfqpoint{2.639236in}{2.794105in}}{\pgfqpoint{2.628637in}{2.789715in}}{\pgfqpoint{2.620823in}{2.781901in}}%
\pgfpathcurveto{\pgfqpoint{2.613009in}{2.774088in}}{\pgfqpoint{2.608619in}{2.763489in}}{\pgfqpoint{2.608619in}{2.752439in}}%
\pgfpathcurveto{\pgfqpoint{2.608619in}{2.741388in}}{\pgfqpoint{2.613009in}{2.730789in}}{\pgfqpoint{2.620823in}{2.722976in}}%
\pgfpathcurveto{\pgfqpoint{2.628637in}{2.715162in}}{\pgfqpoint{2.639236in}{2.710772in}}{\pgfqpoint{2.650286in}{2.710772in}}%
\pgfpathclose%
\pgfusepath{stroke,fill}%
\end{pgfscope}%
\begin{pgfscope}%
\pgfpathrectangle{\pgfqpoint{0.600000in}{0.600000in}}{\pgfqpoint{3.900000in}{3.900000in}}%
\pgfusepath{clip}%
\pgfsetbuttcap%
\pgfsetroundjoin%
\definecolor{currentfill}{rgb}{0.121569,0.466667,0.705882}%
\pgfsetfillcolor{currentfill}%
\pgfsetlinewidth{1.003750pt}%
\definecolor{currentstroke}{rgb}{0.121569,0.466667,0.705882}%
\pgfsetstrokecolor{currentstroke}%
\pgfsetdash{}{0pt}%
\pgfpathmoveto{\pgfqpoint{2.625694in}{1.554034in}}%
\pgfpathcurveto{\pgfqpoint{2.636744in}{1.554034in}}{\pgfqpoint{2.647343in}{1.558424in}}{\pgfqpoint{2.655157in}{1.566238in}}%
\pgfpathcurveto{\pgfqpoint{2.662971in}{1.574052in}}{\pgfqpoint{2.667361in}{1.584651in}}{\pgfqpoint{2.667361in}{1.595701in}}%
\pgfpathcurveto{\pgfqpoint{2.667361in}{1.606751in}}{\pgfqpoint{2.662971in}{1.617350in}}{\pgfqpoint{2.655157in}{1.625164in}}%
\pgfpathcurveto{\pgfqpoint{2.647343in}{1.632977in}}{\pgfqpoint{2.636744in}{1.637368in}}{\pgfqpoint{2.625694in}{1.637368in}}%
\pgfpathcurveto{\pgfqpoint{2.614644in}{1.637368in}}{\pgfqpoint{2.604045in}{1.632977in}}{\pgfqpoint{2.596231in}{1.625164in}}%
\pgfpathcurveto{\pgfqpoint{2.588418in}{1.617350in}}{\pgfqpoint{2.584028in}{1.606751in}}{\pgfqpoint{2.584028in}{1.595701in}}%
\pgfpathcurveto{\pgfqpoint{2.584028in}{1.584651in}}{\pgfqpoint{2.588418in}{1.574052in}}{\pgfqpoint{2.596231in}{1.566238in}}%
\pgfpathcurveto{\pgfqpoint{2.604045in}{1.558424in}}{\pgfqpoint{2.614644in}{1.554034in}}{\pgfqpoint{2.625694in}{1.554034in}}%
\pgfpathclose%
\pgfusepath{stroke,fill}%
\end{pgfscope}%
\begin{pgfscope}%
\pgfpathrectangle{\pgfqpoint{0.600000in}{0.600000in}}{\pgfqpoint{3.900000in}{3.900000in}}%
\pgfusepath{clip}%
\pgfsetbuttcap%
\pgfsetroundjoin%
\definecolor{currentfill}{rgb}{0.121569,0.466667,0.705882}%
\pgfsetfillcolor{currentfill}%
\pgfsetlinewidth{1.003750pt}%
\definecolor{currentstroke}{rgb}{0.121569,0.466667,0.705882}%
\pgfsetstrokecolor{currentstroke}%
\pgfsetdash{}{0pt}%
\pgfpathmoveto{\pgfqpoint{2.459012in}{2.381408in}}%
\pgfpathcurveto{\pgfqpoint{2.470063in}{2.381408in}}{\pgfqpoint{2.480662in}{2.385798in}}{\pgfqpoint{2.488475in}{2.393612in}}%
\pgfpathcurveto{\pgfqpoint{2.496289in}{2.401426in}}{\pgfqpoint{2.500679in}{2.412025in}}{\pgfqpoint{2.500679in}{2.423075in}}%
\pgfpathcurveto{\pgfqpoint{2.500679in}{2.434125in}}{\pgfqpoint{2.496289in}{2.444724in}}{\pgfqpoint{2.488475in}{2.452538in}}%
\pgfpathcurveto{\pgfqpoint{2.480662in}{2.460351in}}{\pgfqpoint{2.470063in}{2.464742in}}{\pgfqpoint{2.459012in}{2.464742in}}%
\pgfpathcurveto{\pgfqpoint{2.447962in}{2.464742in}}{\pgfqpoint{2.437363in}{2.460351in}}{\pgfqpoint{2.429550in}{2.452538in}}%
\pgfpathcurveto{\pgfqpoint{2.421736in}{2.444724in}}{\pgfqpoint{2.417346in}{2.434125in}}{\pgfqpoint{2.417346in}{2.423075in}}%
\pgfpathcurveto{\pgfqpoint{2.417346in}{2.412025in}}{\pgfqpoint{2.421736in}{2.401426in}}{\pgfqpoint{2.429550in}{2.393612in}}%
\pgfpathcurveto{\pgfqpoint{2.437363in}{2.385798in}}{\pgfqpoint{2.447962in}{2.381408in}}{\pgfqpoint{2.459012in}{2.381408in}}%
\pgfpathclose%
\pgfusepath{stroke,fill}%
\end{pgfscope}%
\begin{pgfscope}%
\pgfpathrectangle{\pgfqpoint{0.600000in}{0.600000in}}{\pgfqpoint{3.900000in}{3.900000in}}%
\pgfusepath{clip}%
\pgfsetbuttcap%
\pgfsetroundjoin%
\definecolor{currentfill}{rgb}{0.121569,0.466667,0.705882}%
\pgfsetfillcolor{currentfill}%
\pgfsetlinewidth{1.003750pt}%
\definecolor{currentstroke}{rgb}{0.121569,0.466667,0.705882}%
\pgfsetstrokecolor{currentstroke}%
\pgfsetdash{}{0pt}%
\pgfpathmoveto{\pgfqpoint{2.709611in}{1.876851in}}%
\pgfpathcurveto{\pgfqpoint{2.720661in}{1.876851in}}{\pgfqpoint{2.731260in}{1.881241in}}{\pgfqpoint{2.739073in}{1.889055in}}%
\pgfpathcurveto{\pgfqpoint{2.746887in}{1.896868in}}{\pgfqpoint{2.751277in}{1.907467in}}{\pgfqpoint{2.751277in}{1.918517in}}%
\pgfpathcurveto{\pgfqpoint{2.751277in}{1.929568in}}{\pgfqpoint{2.746887in}{1.940167in}}{\pgfqpoint{2.739073in}{1.947980in}}%
\pgfpathcurveto{\pgfqpoint{2.731260in}{1.955794in}}{\pgfqpoint{2.720661in}{1.960184in}}{\pgfqpoint{2.709611in}{1.960184in}}%
\pgfpathcurveto{\pgfqpoint{2.698560in}{1.960184in}}{\pgfqpoint{2.687961in}{1.955794in}}{\pgfqpoint{2.680148in}{1.947980in}}%
\pgfpathcurveto{\pgfqpoint{2.672334in}{1.940167in}}{\pgfqpoint{2.667944in}{1.929568in}}{\pgfqpoint{2.667944in}{1.918517in}}%
\pgfpathcurveto{\pgfqpoint{2.667944in}{1.907467in}}{\pgfqpoint{2.672334in}{1.896868in}}{\pgfqpoint{2.680148in}{1.889055in}}%
\pgfpathcurveto{\pgfqpoint{2.687961in}{1.881241in}}{\pgfqpoint{2.698560in}{1.876851in}}{\pgfqpoint{2.709611in}{1.876851in}}%
\pgfpathclose%
\pgfusepath{stroke,fill}%
\end{pgfscope}%
\begin{pgfscope}%
\pgfpathrectangle{\pgfqpoint{0.600000in}{0.600000in}}{\pgfqpoint{3.900000in}{3.900000in}}%
\pgfusepath{clip}%
\pgfsetbuttcap%
\pgfsetroundjoin%
\definecolor{currentfill}{rgb}{0.121569,0.466667,0.705882}%
\pgfsetfillcolor{currentfill}%
\pgfsetlinewidth{1.003750pt}%
\definecolor{currentstroke}{rgb}{0.121569,0.466667,0.705882}%
\pgfsetstrokecolor{currentstroke}%
\pgfsetdash{}{0pt}%
\pgfpathmoveto{\pgfqpoint{2.485118in}{3.047217in}}%
\pgfpathcurveto{\pgfqpoint{2.496168in}{3.047217in}}{\pgfqpoint{2.506767in}{3.051607in}}{\pgfqpoint{2.514581in}{3.059421in}}%
\pgfpathcurveto{\pgfqpoint{2.522394in}{3.067234in}}{\pgfqpoint{2.526785in}{3.077833in}}{\pgfqpoint{2.526785in}{3.088883in}}%
\pgfpathcurveto{\pgfqpoint{2.526785in}{3.099933in}}{\pgfqpoint{2.522394in}{3.110533in}}{\pgfqpoint{2.514581in}{3.118346in}}%
\pgfpathcurveto{\pgfqpoint{2.506767in}{3.126160in}}{\pgfqpoint{2.496168in}{3.130550in}}{\pgfqpoint{2.485118in}{3.130550in}}%
\pgfpathcurveto{\pgfqpoint{2.474068in}{3.130550in}}{\pgfqpoint{2.463469in}{3.126160in}}{\pgfqpoint{2.455655in}{3.118346in}}%
\pgfpathcurveto{\pgfqpoint{2.447842in}{3.110533in}}{\pgfqpoint{2.443451in}{3.099933in}}{\pgfqpoint{2.443451in}{3.088883in}}%
\pgfpathcurveto{\pgfqpoint{2.443451in}{3.077833in}}{\pgfqpoint{2.447842in}{3.067234in}}{\pgfqpoint{2.455655in}{3.059421in}}%
\pgfpathcurveto{\pgfqpoint{2.463469in}{3.051607in}}{\pgfqpoint{2.474068in}{3.047217in}}{\pgfqpoint{2.485118in}{3.047217in}}%
\pgfpathclose%
\pgfusepath{stroke,fill}%
\end{pgfscope}%
\begin{pgfscope}%
\pgfpathrectangle{\pgfqpoint{0.600000in}{0.600000in}}{\pgfqpoint{3.900000in}{3.900000in}}%
\pgfusepath{clip}%
\pgfsetbuttcap%
\pgfsetroundjoin%
\definecolor{currentfill}{rgb}{0.121569,0.466667,0.705882}%
\pgfsetfillcolor{currentfill}%
\pgfsetlinewidth{1.003750pt}%
\definecolor{currentstroke}{rgb}{0.121569,0.466667,0.705882}%
\pgfsetstrokecolor{currentstroke}%
\pgfsetdash{}{0pt}%
\pgfpathmoveto{\pgfqpoint{2.414775in}{2.708243in}}%
\pgfpathcurveto{\pgfqpoint{2.425825in}{2.708243in}}{\pgfqpoint{2.436424in}{2.712634in}}{\pgfqpoint{2.444238in}{2.720447in}}%
\pgfpathcurveto{\pgfqpoint{2.452051in}{2.728261in}}{\pgfqpoint{2.456442in}{2.738860in}}{\pgfqpoint{2.456442in}{2.749910in}}%
\pgfpathcurveto{\pgfqpoint{2.456442in}{2.760960in}}{\pgfqpoint{2.452051in}{2.771559in}}{\pgfqpoint{2.444238in}{2.779373in}}%
\pgfpathcurveto{\pgfqpoint{2.436424in}{2.787187in}}{\pgfqpoint{2.425825in}{2.791577in}}{\pgfqpoint{2.414775in}{2.791577in}}%
\pgfpathcurveto{\pgfqpoint{2.403725in}{2.791577in}}{\pgfqpoint{2.393126in}{2.787187in}}{\pgfqpoint{2.385312in}{2.779373in}}%
\pgfpathcurveto{\pgfqpoint{2.377499in}{2.771559in}}{\pgfqpoint{2.373108in}{2.760960in}}{\pgfqpoint{2.373108in}{2.749910in}}%
\pgfpathcurveto{\pgfqpoint{2.373108in}{2.738860in}}{\pgfqpoint{2.377499in}{2.728261in}}{\pgfqpoint{2.385312in}{2.720447in}}%
\pgfpathcurveto{\pgfqpoint{2.393126in}{2.712634in}}{\pgfqpoint{2.403725in}{2.708243in}}{\pgfqpoint{2.414775in}{2.708243in}}%
\pgfpathclose%
\pgfusepath{stroke,fill}%
\end{pgfscope}%
\begin{pgfscope}%
\pgfpathrectangle{\pgfqpoint{0.600000in}{0.600000in}}{\pgfqpoint{3.900000in}{3.900000in}}%
\pgfusepath{clip}%
\pgfsetbuttcap%
\pgfsetroundjoin%
\definecolor{currentfill}{rgb}{0.121569,0.466667,0.705882}%
\pgfsetfillcolor{currentfill}%
\pgfsetlinewidth{1.003750pt}%
\definecolor{currentstroke}{rgb}{0.121569,0.466667,0.705882}%
\pgfsetstrokecolor{currentstroke}%
\pgfsetdash{}{0pt}%
\pgfpathmoveto{\pgfqpoint{2.380526in}{2.900869in}}%
\pgfpathcurveto{\pgfqpoint{2.391576in}{2.900869in}}{\pgfqpoint{2.402175in}{2.905259in}}{\pgfqpoint{2.409988in}{2.913072in}}%
\pgfpathcurveto{\pgfqpoint{2.417802in}{2.920886in}}{\pgfqpoint{2.422192in}{2.931485in}}{\pgfqpoint{2.422192in}{2.942535in}}%
\pgfpathcurveto{\pgfqpoint{2.422192in}{2.953585in}}{\pgfqpoint{2.417802in}{2.964184in}}{\pgfqpoint{2.409988in}{2.971998in}}%
\pgfpathcurveto{\pgfqpoint{2.402175in}{2.979812in}}{\pgfqpoint{2.391576in}{2.984202in}}{\pgfqpoint{2.380526in}{2.984202in}}%
\pgfpathcurveto{\pgfqpoint{2.369475in}{2.984202in}}{\pgfqpoint{2.358876in}{2.979812in}}{\pgfqpoint{2.351063in}{2.971998in}}%
\pgfpathcurveto{\pgfqpoint{2.343249in}{2.964184in}}{\pgfqpoint{2.338859in}{2.953585in}}{\pgfqpoint{2.338859in}{2.942535in}}%
\pgfpathcurveto{\pgfqpoint{2.338859in}{2.931485in}}{\pgfqpoint{2.343249in}{2.920886in}}{\pgfqpoint{2.351063in}{2.913072in}}%
\pgfpathcurveto{\pgfqpoint{2.358876in}{2.905259in}}{\pgfqpoint{2.369475in}{2.900869in}}{\pgfqpoint{2.380526in}{2.900869in}}%
\pgfpathclose%
\pgfusepath{stroke,fill}%
\end{pgfscope}%
\begin{pgfscope}%
\pgfpathrectangle{\pgfqpoint{0.600000in}{0.600000in}}{\pgfqpoint{3.900000in}{3.900000in}}%
\pgfusepath{clip}%
\pgfsetbuttcap%
\pgfsetroundjoin%
\definecolor{currentfill}{rgb}{0.121569,0.466667,0.705882}%
\pgfsetfillcolor{currentfill}%
\pgfsetlinewidth{1.003750pt}%
\definecolor{currentstroke}{rgb}{0.121569,0.466667,0.705882}%
\pgfsetstrokecolor{currentstroke}%
\pgfsetdash{}{0pt}%
\pgfpathmoveto{\pgfqpoint{3.608057in}{3.284305in}}%
\pgfpathcurveto{\pgfqpoint{3.619107in}{3.284305in}}{\pgfqpoint{3.629706in}{3.288696in}}{\pgfqpoint{3.637519in}{3.296509in}}%
\pgfpathcurveto{\pgfqpoint{3.645333in}{3.304323in}}{\pgfqpoint{3.649723in}{3.314922in}}{\pgfqpoint{3.649723in}{3.325972in}}%
\pgfpathcurveto{\pgfqpoint{3.649723in}{3.337022in}}{\pgfqpoint{3.645333in}{3.347621in}}{\pgfqpoint{3.637519in}{3.355435in}}%
\pgfpathcurveto{\pgfqpoint{3.629706in}{3.363248in}}{\pgfqpoint{3.619107in}{3.367639in}}{\pgfqpoint{3.608057in}{3.367639in}}%
\pgfpathcurveto{\pgfqpoint{3.597006in}{3.367639in}}{\pgfqpoint{3.586407in}{3.363248in}}{\pgfqpoint{3.578594in}{3.355435in}}%
\pgfpathcurveto{\pgfqpoint{3.570780in}{3.347621in}}{\pgfqpoint{3.566390in}{3.337022in}}{\pgfqpoint{3.566390in}{3.325972in}}%
\pgfpathcurveto{\pgfqpoint{3.566390in}{3.314922in}}{\pgfqpoint{3.570780in}{3.304323in}}{\pgfqpoint{3.578594in}{3.296509in}}%
\pgfpathcurveto{\pgfqpoint{3.586407in}{3.288696in}}{\pgfqpoint{3.597006in}{3.284305in}}{\pgfqpoint{3.608057in}{3.284305in}}%
\pgfpathclose%
\pgfusepath{stroke,fill}%
\end{pgfscope}%
\begin{pgfscope}%
\pgfpathrectangle{\pgfqpoint{0.600000in}{0.600000in}}{\pgfqpoint{3.900000in}{3.900000in}}%
\pgfusepath{clip}%
\pgfsetbuttcap%
\pgfsetroundjoin%
\definecolor{currentfill}{rgb}{0.121569,0.466667,0.705882}%
\pgfsetfillcolor{currentfill}%
\pgfsetlinewidth{1.003750pt}%
\definecolor{currentstroke}{rgb}{0.121569,0.466667,0.705882}%
\pgfsetstrokecolor{currentstroke}%
\pgfsetdash{}{0pt}%
\pgfpathmoveto{\pgfqpoint{1.959323in}{2.526361in}}%
\pgfpathcurveto{\pgfqpoint{1.970373in}{2.526361in}}{\pgfqpoint{1.980972in}{2.530751in}}{\pgfqpoint{1.988786in}{2.538564in}}%
\pgfpathcurveto{\pgfqpoint{1.996600in}{2.546378in}}{\pgfqpoint{2.000990in}{2.556977in}}{\pgfqpoint{2.000990in}{2.568027in}}%
\pgfpathcurveto{\pgfqpoint{2.000990in}{2.579077in}}{\pgfqpoint{1.996600in}{2.589676in}}{\pgfqpoint{1.988786in}{2.597490in}}%
\pgfpathcurveto{\pgfqpoint{1.980972in}{2.605304in}}{\pgfqpoint{1.970373in}{2.609694in}}{\pgfqpoint{1.959323in}{2.609694in}}%
\pgfpathcurveto{\pgfqpoint{1.948273in}{2.609694in}}{\pgfqpoint{1.937674in}{2.605304in}}{\pgfqpoint{1.929860in}{2.597490in}}%
\pgfpathcurveto{\pgfqpoint{1.922047in}{2.589676in}}{\pgfqpoint{1.917657in}{2.579077in}}{\pgfqpoint{1.917657in}{2.568027in}}%
\pgfpathcurveto{\pgfqpoint{1.917657in}{2.556977in}}{\pgfqpoint{1.922047in}{2.546378in}}{\pgfqpoint{1.929860in}{2.538564in}}%
\pgfpathcurveto{\pgfqpoint{1.937674in}{2.530751in}}{\pgfqpoint{1.948273in}{2.526361in}}{\pgfqpoint{1.959323in}{2.526361in}}%
\pgfpathclose%
\pgfusepath{stroke,fill}%
\end{pgfscope}%
\begin{pgfscope}%
\pgfpathrectangle{\pgfqpoint{0.600000in}{0.600000in}}{\pgfqpoint{3.900000in}{3.900000in}}%
\pgfusepath{clip}%
\pgfsetbuttcap%
\pgfsetroundjoin%
\definecolor{currentfill}{rgb}{0.121569,0.466667,0.705882}%
\pgfsetfillcolor{currentfill}%
\pgfsetlinewidth{1.003750pt}%
\definecolor{currentstroke}{rgb}{0.121569,0.466667,0.705882}%
\pgfsetstrokecolor{currentstroke}%
\pgfsetdash{}{0pt}%
\pgfpathmoveto{\pgfqpoint{2.342287in}{2.970252in}}%
\pgfpathcurveto{\pgfqpoint{2.353337in}{2.970252in}}{\pgfqpoint{2.363936in}{2.974642in}}{\pgfqpoint{2.371750in}{2.982456in}}%
\pgfpathcurveto{\pgfqpoint{2.379564in}{2.990270in}}{\pgfqpoint{2.383954in}{3.000869in}}{\pgfqpoint{2.383954in}{3.011919in}}%
\pgfpathcurveto{\pgfqpoint{2.383954in}{3.022969in}}{\pgfqpoint{2.379564in}{3.033568in}}{\pgfqpoint{2.371750in}{3.041381in}}%
\pgfpathcurveto{\pgfqpoint{2.363936in}{3.049195in}}{\pgfqpoint{2.353337in}{3.053585in}}{\pgfqpoint{2.342287in}{3.053585in}}%
\pgfpathcurveto{\pgfqpoint{2.331237in}{3.053585in}}{\pgfqpoint{2.320638in}{3.049195in}}{\pgfqpoint{2.312824in}{3.041381in}}%
\pgfpathcurveto{\pgfqpoint{2.305011in}{3.033568in}}{\pgfqpoint{2.300621in}{3.022969in}}{\pgfqpoint{2.300621in}{3.011919in}}%
\pgfpathcurveto{\pgfqpoint{2.300621in}{3.000869in}}{\pgfqpoint{2.305011in}{2.990270in}}{\pgfqpoint{2.312824in}{2.982456in}}%
\pgfpathcurveto{\pgfqpoint{2.320638in}{2.974642in}}{\pgfqpoint{2.331237in}{2.970252in}}{\pgfqpoint{2.342287in}{2.970252in}}%
\pgfpathclose%
\pgfusepath{stroke,fill}%
\end{pgfscope}%
\begin{pgfscope}%
\pgfpathrectangle{\pgfqpoint{0.600000in}{0.600000in}}{\pgfqpoint{3.900000in}{3.900000in}}%
\pgfusepath{clip}%
\pgfsetbuttcap%
\pgfsetroundjoin%
\definecolor{currentfill}{rgb}{0.121569,0.466667,0.705882}%
\pgfsetfillcolor{currentfill}%
\pgfsetlinewidth{1.003750pt}%
\definecolor{currentstroke}{rgb}{0.121569,0.466667,0.705882}%
\pgfsetstrokecolor{currentstroke}%
\pgfsetdash{}{0pt}%
\pgfpathmoveto{\pgfqpoint{3.206945in}{1.633777in}}%
\pgfpathcurveto{\pgfqpoint{3.217995in}{1.633777in}}{\pgfqpoint{3.228594in}{1.638167in}}{\pgfqpoint{3.236407in}{1.645981in}}%
\pgfpathcurveto{\pgfqpoint{3.244221in}{1.653794in}}{\pgfqpoint{3.248611in}{1.664393in}}{\pgfqpoint{3.248611in}{1.675444in}}%
\pgfpathcurveto{\pgfqpoint{3.248611in}{1.686494in}}{\pgfqpoint{3.244221in}{1.697093in}}{\pgfqpoint{3.236407in}{1.704906in}}%
\pgfpathcurveto{\pgfqpoint{3.228594in}{1.712720in}}{\pgfqpoint{3.217995in}{1.717110in}}{\pgfqpoint{3.206945in}{1.717110in}}%
\pgfpathcurveto{\pgfqpoint{3.195894in}{1.717110in}}{\pgfqpoint{3.185295in}{1.712720in}}{\pgfqpoint{3.177482in}{1.704906in}}%
\pgfpathcurveto{\pgfqpoint{3.169668in}{1.697093in}}{\pgfqpoint{3.165278in}{1.686494in}}{\pgfqpoint{3.165278in}{1.675444in}}%
\pgfpathcurveto{\pgfqpoint{3.165278in}{1.664393in}}{\pgfqpoint{3.169668in}{1.653794in}}{\pgfqpoint{3.177482in}{1.645981in}}%
\pgfpathcurveto{\pgfqpoint{3.185295in}{1.638167in}}{\pgfqpoint{3.195894in}{1.633777in}}{\pgfqpoint{3.206945in}{1.633777in}}%
\pgfpathclose%
\pgfusepath{stroke,fill}%
\end{pgfscope}%
\begin{pgfscope}%
\pgfpathrectangle{\pgfqpoint{0.600000in}{0.600000in}}{\pgfqpoint{3.900000in}{3.900000in}}%
\pgfusepath{clip}%
\pgfsetbuttcap%
\pgfsetroundjoin%
\definecolor{currentfill}{rgb}{0.121569,0.466667,0.705882}%
\pgfsetfillcolor{currentfill}%
\pgfsetlinewidth{1.003750pt}%
\definecolor{currentstroke}{rgb}{0.121569,0.466667,0.705882}%
\pgfsetstrokecolor{currentstroke}%
\pgfsetdash{}{0pt}%
\pgfpathmoveto{\pgfqpoint{2.754076in}{1.919521in}}%
\pgfpathcurveto{\pgfqpoint{2.765126in}{1.919521in}}{\pgfqpoint{2.775725in}{1.923911in}}{\pgfqpoint{2.783539in}{1.931725in}}%
\pgfpathcurveto{\pgfqpoint{2.791352in}{1.939538in}}{\pgfqpoint{2.795743in}{1.950138in}}{\pgfqpoint{2.795743in}{1.961188in}}%
\pgfpathcurveto{\pgfqpoint{2.795743in}{1.972238in}}{\pgfqpoint{2.791352in}{1.982837in}}{\pgfqpoint{2.783539in}{1.990650in}}%
\pgfpathcurveto{\pgfqpoint{2.775725in}{1.998464in}}{\pgfqpoint{2.765126in}{2.002854in}}{\pgfqpoint{2.754076in}{2.002854in}}%
\pgfpathcurveto{\pgfqpoint{2.743026in}{2.002854in}}{\pgfqpoint{2.732427in}{1.998464in}}{\pgfqpoint{2.724613in}{1.990650in}}%
\pgfpathcurveto{\pgfqpoint{2.716800in}{1.982837in}}{\pgfqpoint{2.712409in}{1.972238in}}{\pgfqpoint{2.712409in}{1.961188in}}%
\pgfpathcurveto{\pgfqpoint{2.712409in}{1.950138in}}{\pgfqpoint{2.716800in}{1.939538in}}{\pgfqpoint{2.724613in}{1.931725in}}%
\pgfpathcurveto{\pgfqpoint{2.732427in}{1.923911in}}{\pgfqpoint{2.743026in}{1.919521in}}{\pgfqpoint{2.754076in}{1.919521in}}%
\pgfpathclose%
\pgfusepath{stroke,fill}%
\end{pgfscope}%
\begin{pgfscope}%
\pgfpathrectangle{\pgfqpoint{0.600000in}{0.600000in}}{\pgfqpoint{3.900000in}{3.900000in}}%
\pgfusepath{clip}%
\pgfsetbuttcap%
\pgfsetroundjoin%
\definecolor{currentfill}{rgb}{0.121569,0.466667,0.705882}%
\pgfsetfillcolor{currentfill}%
\pgfsetlinewidth{1.003750pt}%
\definecolor{currentstroke}{rgb}{0.121569,0.466667,0.705882}%
\pgfsetstrokecolor{currentstroke}%
\pgfsetdash{}{0pt}%
\pgfpathmoveto{\pgfqpoint{2.294277in}{2.645795in}}%
\pgfpathcurveto{\pgfqpoint{2.305327in}{2.645795in}}{\pgfqpoint{2.315926in}{2.650185in}}{\pgfqpoint{2.323739in}{2.657998in}}%
\pgfpathcurveto{\pgfqpoint{2.331553in}{2.665812in}}{\pgfqpoint{2.335943in}{2.676411in}}{\pgfqpoint{2.335943in}{2.687461in}}%
\pgfpathcurveto{\pgfqpoint{2.335943in}{2.698511in}}{\pgfqpoint{2.331553in}{2.709110in}}{\pgfqpoint{2.323739in}{2.716924in}}%
\pgfpathcurveto{\pgfqpoint{2.315926in}{2.724738in}}{\pgfqpoint{2.305327in}{2.729128in}}{\pgfqpoint{2.294277in}{2.729128in}}%
\pgfpathcurveto{\pgfqpoint{2.283227in}{2.729128in}}{\pgfqpoint{2.272628in}{2.724738in}}{\pgfqpoint{2.264814in}{2.716924in}}%
\pgfpathcurveto{\pgfqpoint{2.257000in}{2.709110in}}{\pgfqpoint{2.252610in}{2.698511in}}{\pgfqpoint{2.252610in}{2.687461in}}%
\pgfpathcurveto{\pgfqpoint{2.252610in}{2.676411in}}{\pgfqpoint{2.257000in}{2.665812in}}{\pgfqpoint{2.264814in}{2.657998in}}%
\pgfpathcurveto{\pgfqpoint{2.272628in}{2.650185in}}{\pgfqpoint{2.283227in}{2.645795in}}{\pgfqpoint{2.294277in}{2.645795in}}%
\pgfpathclose%
\pgfusepath{stroke,fill}%
\end{pgfscope}%
\begin{pgfscope}%
\pgfpathrectangle{\pgfqpoint{0.600000in}{0.600000in}}{\pgfqpoint{3.900000in}{3.900000in}}%
\pgfusepath{clip}%
\pgfsetbuttcap%
\pgfsetroundjoin%
\definecolor{currentfill}{rgb}{0.121569,0.466667,0.705882}%
\pgfsetfillcolor{currentfill}%
\pgfsetlinewidth{1.003750pt}%
\definecolor{currentstroke}{rgb}{0.121569,0.466667,0.705882}%
\pgfsetstrokecolor{currentstroke}%
\pgfsetdash{}{0pt}%
\pgfpathmoveto{\pgfqpoint{2.469471in}{2.623731in}}%
\pgfpathcurveto{\pgfqpoint{2.480521in}{2.623731in}}{\pgfqpoint{2.491120in}{2.628122in}}{\pgfqpoint{2.498934in}{2.635935in}}%
\pgfpathcurveto{\pgfqpoint{2.506748in}{2.643749in}}{\pgfqpoint{2.511138in}{2.654348in}}{\pgfqpoint{2.511138in}{2.665398in}}%
\pgfpathcurveto{\pgfqpoint{2.511138in}{2.676448in}}{\pgfqpoint{2.506748in}{2.687047in}}{\pgfqpoint{2.498934in}{2.694861in}}%
\pgfpathcurveto{\pgfqpoint{2.491120in}{2.702674in}}{\pgfqpoint{2.480521in}{2.707065in}}{\pgfqpoint{2.469471in}{2.707065in}}%
\pgfpathcurveto{\pgfqpoint{2.458421in}{2.707065in}}{\pgfqpoint{2.447822in}{2.702674in}}{\pgfqpoint{2.440009in}{2.694861in}}%
\pgfpathcurveto{\pgfqpoint{2.432195in}{2.687047in}}{\pgfqpoint{2.427805in}{2.676448in}}{\pgfqpoint{2.427805in}{2.665398in}}%
\pgfpathcurveto{\pgfqpoint{2.427805in}{2.654348in}}{\pgfqpoint{2.432195in}{2.643749in}}{\pgfqpoint{2.440009in}{2.635935in}}%
\pgfpathcurveto{\pgfqpoint{2.447822in}{2.628122in}}{\pgfqpoint{2.458421in}{2.623731in}}{\pgfqpoint{2.469471in}{2.623731in}}%
\pgfpathclose%
\pgfusepath{stroke,fill}%
\end{pgfscope}%
\begin{pgfscope}%
\pgfpathrectangle{\pgfqpoint{0.600000in}{0.600000in}}{\pgfqpoint{3.900000in}{3.900000in}}%
\pgfusepath{clip}%
\pgfsetbuttcap%
\pgfsetroundjoin%
\definecolor{currentfill}{rgb}{0.121569,0.466667,0.705882}%
\pgfsetfillcolor{currentfill}%
\pgfsetlinewidth{1.003750pt}%
\definecolor{currentstroke}{rgb}{0.121569,0.466667,0.705882}%
\pgfsetstrokecolor{currentstroke}%
\pgfsetdash{}{0pt}%
\pgfpathmoveto{\pgfqpoint{2.700402in}{2.838454in}}%
\pgfpathcurveto{\pgfqpoint{2.711452in}{2.838454in}}{\pgfqpoint{2.722051in}{2.842844in}}{\pgfqpoint{2.729865in}{2.850658in}}%
\pgfpathcurveto{\pgfqpoint{2.737679in}{2.858471in}}{\pgfqpoint{2.742069in}{2.869070in}}{\pgfqpoint{2.742069in}{2.880121in}}%
\pgfpathcurveto{\pgfqpoint{2.742069in}{2.891171in}}{\pgfqpoint{2.737679in}{2.901770in}}{\pgfqpoint{2.729865in}{2.909583in}}%
\pgfpathcurveto{\pgfqpoint{2.722051in}{2.917397in}}{\pgfqpoint{2.711452in}{2.921787in}}{\pgfqpoint{2.700402in}{2.921787in}}%
\pgfpathcurveto{\pgfqpoint{2.689352in}{2.921787in}}{\pgfqpoint{2.678753in}{2.917397in}}{\pgfqpoint{2.670940in}{2.909583in}}%
\pgfpathcurveto{\pgfqpoint{2.663126in}{2.901770in}}{\pgfqpoint{2.658736in}{2.891171in}}{\pgfqpoint{2.658736in}{2.880121in}}%
\pgfpathcurveto{\pgfqpoint{2.658736in}{2.869070in}}{\pgfqpoint{2.663126in}{2.858471in}}{\pgfqpoint{2.670940in}{2.850658in}}%
\pgfpathcurveto{\pgfqpoint{2.678753in}{2.842844in}}{\pgfqpoint{2.689352in}{2.838454in}}{\pgfqpoint{2.700402in}{2.838454in}}%
\pgfpathclose%
\pgfusepath{stroke,fill}%
\end{pgfscope}%
\begin{pgfscope}%
\pgfpathrectangle{\pgfqpoint{0.600000in}{0.600000in}}{\pgfqpoint{3.900000in}{3.900000in}}%
\pgfusepath{clip}%
\pgfsetbuttcap%
\pgfsetroundjoin%
\definecolor{currentfill}{rgb}{0.121569,0.466667,0.705882}%
\pgfsetfillcolor{currentfill}%
\pgfsetlinewidth{1.003750pt}%
\definecolor{currentstroke}{rgb}{0.121569,0.466667,0.705882}%
\pgfsetstrokecolor{currentstroke}%
\pgfsetdash{}{0pt}%
\pgfpathmoveto{\pgfqpoint{2.513389in}{2.360463in}}%
\pgfpathcurveto{\pgfqpoint{2.524439in}{2.360463in}}{\pgfqpoint{2.535038in}{2.364853in}}{\pgfqpoint{2.542852in}{2.372667in}}%
\pgfpathcurveto{\pgfqpoint{2.550666in}{2.380480in}}{\pgfqpoint{2.555056in}{2.391079in}}{\pgfqpoint{2.555056in}{2.402129in}}%
\pgfpathcurveto{\pgfqpoint{2.555056in}{2.413179in}}{\pgfqpoint{2.550666in}{2.423778in}}{\pgfqpoint{2.542852in}{2.431592in}}%
\pgfpathcurveto{\pgfqpoint{2.535038in}{2.439406in}}{\pgfqpoint{2.524439in}{2.443796in}}{\pgfqpoint{2.513389in}{2.443796in}}%
\pgfpathcurveto{\pgfqpoint{2.502339in}{2.443796in}}{\pgfqpoint{2.491740in}{2.439406in}}{\pgfqpoint{2.483927in}{2.431592in}}%
\pgfpathcurveto{\pgfqpoint{2.476113in}{2.423778in}}{\pgfqpoint{2.471723in}{2.413179in}}{\pgfqpoint{2.471723in}{2.402129in}}%
\pgfpathcurveto{\pgfqpoint{2.471723in}{2.391079in}}{\pgfqpoint{2.476113in}{2.380480in}}{\pgfqpoint{2.483927in}{2.372667in}}%
\pgfpathcurveto{\pgfqpoint{2.491740in}{2.364853in}}{\pgfqpoint{2.502339in}{2.360463in}}{\pgfqpoint{2.513389in}{2.360463in}}%
\pgfpathclose%
\pgfusepath{stroke,fill}%
\end{pgfscope}%
\begin{pgfscope}%
\pgfpathrectangle{\pgfqpoint{0.600000in}{0.600000in}}{\pgfqpoint{3.900000in}{3.900000in}}%
\pgfusepath{clip}%
\pgfsetbuttcap%
\pgfsetroundjoin%
\definecolor{currentfill}{rgb}{0.121569,0.466667,0.705882}%
\pgfsetfillcolor{currentfill}%
\pgfsetlinewidth{1.003750pt}%
\definecolor{currentstroke}{rgb}{0.121569,0.466667,0.705882}%
\pgfsetstrokecolor{currentstroke}%
\pgfsetdash{}{0pt}%
\pgfpathmoveto{\pgfqpoint{3.512105in}{2.733876in}}%
\pgfpathcurveto{\pgfqpoint{3.523155in}{2.733876in}}{\pgfqpoint{3.533754in}{2.738266in}}{\pgfqpoint{3.541567in}{2.746080in}}%
\pgfpathcurveto{\pgfqpoint{3.549381in}{2.753894in}}{\pgfqpoint{3.553771in}{2.764493in}}{\pgfqpoint{3.553771in}{2.775543in}}%
\pgfpathcurveto{\pgfqpoint{3.553771in}{2.786593in}}{\pgfqpoint{3.549381in}{2.797192in}}{\pgfqpoint{3.541567in}{2.805006in}}%
\pgfpathcurveto{\pgfqpoint{3.533754in}{2.812819in}}{\pgfqpoint{3.523155in}{2.817210in}}{\pgfqpoint{3.512105in}{2.817210in}}%
\pgfpathcurveto{\pgfqpoint{3.501055in}{2.817210in}}{\pgfqpoint{3.490455in}{2.812819in}}{\pgfqpoint{3.482642in}{2.805006in}}%
\pgfpathcurveto{\pgfqpoint{3.474828in}{2.797192in}}{\pgfqpoint{3.470438in}{2.786593in}}{\pgfqpoint{3.470438in}{2.775543in}}%
\pgfpathcurveto{\pgfqpoint{3.470438in}{2.764493in}}{\pgfqpoint{3.474828in}{2.753894in}}{\pgfqpoint{3.482642in}{2.746080in}}%
\pgfpathcurveto{\pgfqpoint{3.490455in}{2.738266in}}{\pgfqpoint{3.501055in}{2.733876in}}{\pgfqpoint{3.512105in}{2.733876in}}%
\pgfpathclose%
\pgfusepath{stroke,fill}%
\end{pgfscope}%
\begin{pgfscope}%
\pgfpathrectangle{\pgfqpoint{0.600000in}{0.600000in}}{\pgfqpoint{3.900000in}{3.900000in}}%
\pgfusepath{clip}%
\pgfsetbuttcap%
\pgfsetroundjoin%
\definecolor{currentfill}{rgb}{0.121569,0.466667,0.705882}%
\pgfsetfillcolor{currentfill}%
\pgfsetlinewidth{1.003750pt}%
\definecolor{currentstroke}{rgb}{0.121569,0.466667,0.705882}%
\pgfsetstrokecolor{currentstroke}%
\pgfsetdash{}{0pt}%
\pgfpathmoveto{\pgfqpoint{2.556384in}{3.352235in}}%
\pgfpathcurveto{\pgfqpoint{2.567434in}{3.352235in}}{\pgfqpoint{2.578033in}{3.356625in}}{\pgfqpoint{2.585846in}{3.364439in}}%
\pgfpathcurveto{\pgfqpoint{2.593660in}{3.372253in}}{\pgfqpoint{2.598050in}{3.382852in}}{\pgfqpoint{2.598050in}{3.393902in}}%
\pgfpathcurveto{\pgfqpoint{2.598050in}{3.404952in}}{\pgfqpoint{2.593660in}{3.415551in}}{\pgfqpoint{2.585846in}{3.423365in}}%
\pgfpathcurveto{\pgfqpoint{2.578033in}{3.431178in}}{\pgfqpoint{2.567434in}{3.435568in}}{\pgfqpoint{2.556384in}{3.435568in}}%
\pgfpathcurveto{\pgfqpoint{2.545333in}{3.435568in}}{\pgfqpoint{2.534734in}{3.431178in}}{\pgfqpoint{2.526921in}{3.423365in}}%
\pgfpathcurveto{\pgfqpoint{2.519107in}{3.415551in}}{\pgfqpoint{2.514717in}{3.404952in}}{\pgfqpoint{2.514717in}{3.393902in}}%
\pgfpathcurveto{\pgfqpoint{2.514717in}{3.382852in}}{\pgfqpoint{2.519107in}{3.372253in}}{\pgfqpoint{2.526921in}{3.364439in}}%
\pgfpathcurveto{\pgfqpoint{2.534734in}{3.356625in}}{\pgfqpoint{2.545333in}{3.352235in}}{\pgfqpoint{2.556384in}{3.352235in}}%
\pgfpathclose%
\pgfusepath{stroke,fill}%
\end{pgfscope}%
\begin{pgfscope}%
\pgfpathrectangle{\pgfqpoint{0.600000in}{0.600000in}}{\pgfqpoint{3.900000in}{3.900000in}}%
\pgfusepath{clip}%
\pgfsetbuttcap%
\pgfsetroundjoin%
\definecolor{currentfill}{rgb}{0.121569,0.466667,0.705882}%
\pgfsetfillcolor{currentfill}%
\pgfsetlinewidth{1.003750pt}%
\definecolor{currentstroke}{rgb}{0.121569,0.466667,0.705882}%
\pgfsetstrokecolor{currentstroke}%
\pgfsetdash{}{0pt}%
\pgfpathmoveto{\pgfqpoint{3.067323in}{2.298885in}}%
\pgfpathcurveto{\pgfqpoint{3.078373in}{2.298885in}}{\pgfqpoint{3.088972in}{2.303275in}}{\pgfqpoint{3.096786in}{2.311089in}}%
\pgfpathcurveto{\pgfqpoint{3.104600in}{2.318903in}}{\pgfqpoint{3.108990in}{2.329502in}}{\pgfqpoint{3.108990in}{2.340552in}}%
\pgfpathcurveto{\pgfqpoint{3.108990in}{2.351602in}}{\pgfqpoint{3.104600in}{2.362201in}}{\pgfqpoint{3.096786in}{2.370015in}}%
\pgfpathcurveto{\pgfqpoint{3.088972in}{2.377828in}}{\pgfqpoint{3.078373in}{2.382219in}}{\pgfqpoint{3.067323in}{2.382219in}}%
\pgfpathcurveto{\pgfqpoint{3.056273in}{2.382219in}}{\pgfqpoint{3.045674in}{2.377828in}}{\pgfqpoint{3.037860in}{2.370015in}}%
\pgfpathcurveto{\pgfqpoint{3.030047in}{2.362201in}}{\pgfqpoint{3.025657in}{2.351602in}}{\pgfqpoint{3.025657in}{2.340552in}}%
\pgfpathcurveto{\pgfqpoint{3.025657in}{2.329502in}}{\pgfqpoint{3.030047in}{2.318903in}}{\pgfqpoint{3.037860in}{2.311089in}}%
\pgfpathcurveto{\pgfqpoint{3.045674in}{2.303275in}}{\pgfqpoint{3.056273in}{2.298885in}}{\pgfqpoint{3.067323in}{2.298885in}}%
\pgfpathclose%
\pgfusepath{stroke,fill}%
\end{pgfscope}%
\begin{pgfscope}%
\pgfpathrectangle{\pgfqpoint{0.600000in}{0.600000in}}{\pgfqpoint{3.900000in}{3.900000in}}%
\pgfusepath{clip}%
\pgfsetbuttcap%
\pgfsetroundjoin%
\definecolor{currentfill}{rgb}{0.121569,0.466667,0.705882}%
\pgfsetfillcolor{currentfill}%
\pgfsetlinewidth{1.003750pt}%
\definecolor{currentstroke}{rgb}{0.121569,0.466667,0.705882}%
\pgfsetstrokecolor{currentstroke}%
\pgfsetdash{}{0pt}%
\pgfpathmoveto{\pgfqpoint{2.645916in}{3.357942in}}%
\pgfpathcurveto{\pgfqpoint{2.656966in}{3.357942in}}{\pgfqpoint{2.667565in}{3.362333in}}{\pgfqpoint{2.675379in}{3.370146in}}%
\pgfpathcurveto{\pgfqpoint{2.683192in}{3.377960in}}{\pgfqpoint{2.687583in}{3.388559in}}{\pgfqpoint{2.687583in}{3.399609in}}%
\pgfpathcurveto{\pgfqpoint{2.687583in}{3.410659in}}{\pgfqpoint{2.683192in}{3.421258in}}{\pgfqpoint{2.675379in}{3.429072in}}%
\pgfpathcurveto{\pgfqpoint{2.667565in}{3.436885in}}{\pgfqpoint{2.656966in}{3.441276in}}{\pgfqpoint{2.645916in}{3.441276in}}%
\pgfpathcurveto{\pgfqpoint{2.634866in}{3.441276in}}{\pgfqpoint{2.624267in}{3.436885in}}{\pgfqpoint{2.616453in}{3.429072in}}%
\pgfpathcurveto{\pgfqpoint{2.608640in}{3.421258in}}{\pgfqpoint{2.604249in}{3.410659in}}{\pgfqpoint{2.604249in}{3.399609in}}%
\pgfpathcurveto{\pgfqpoint{2.604249in}{3.388559in}}{\pgfqpoint{2.608640in}{3.377960in}}{\pgfqpoint{2.616453in}{3.370146in}}%
\pgfpathcurveto{\pgfqpoint{2.624267in}{3.362333in}}{\pgfqpoint{2.634866in}{3.357942in}}{\pgfqpoint{2.645916in}{3.357942in}}%
\pgfpathclose%
\pgfusepath{stroke,fill}%
\end{pgfscope}%
\begin{pgfscope}%
\pgfpathrectangle{\pgfqpoint{0.600000in}{0.600000in}}{\pgfqpoint{3.900000in}{3.900000in}}%
\pgfusepath{clip}%
\pgfsetbuttcap%
\pgfsetroundjoin%
\definecolor{currentfill}{rgb}{0.121569,0.466667,0.705882}%
\pgfsetfillcolor{currentfill}%
\pgfsetlinewidth{1.003750pt}%
\definecolor{currentstroke}{rgb}{0.121569,0.466667,0.705882}%
\pgfsetstrokecolor{currentstroke}%
\pgfsetdash{}{0pt}%
\pgfpathmoveto{\pgfqpoint{3.522336in}{1.849630in}}%
\pgfpathcurveto{\pgfqpoint{3.533386in}{1.849630in}}{\pgfqpoint{3.543985in}{1.854020in}}{\pgfqpoint{3.551798in}{1.861834in}}%
\pgfpathcurveto{\pgfqpoint{3.559612in}{1.869648in}}{\pgfqpoint{3.564002in}{1.880247in}}{\pgfqpoint{3.564002in}{1.891297in}}%
\pgfpathcurveto{\pgfqpoint{3.564002in}{1.902347in}}{\pgfqpoint{3.559612in}{1.912946in}}{\pgfqpoint{3.551798in}{1.920759in}}%
\pgfpathcurveto{\pgfqpoint{3.543985in}{1.928573in}}{\pgfqpoint{3.533386in}{1.932963in}}{\pgfqpoint{3.522336in}{1.932963in}}%
\pgfpathcurveto{\pgfqpoint{3.511285in}{1.932963in}}{\pgfqpoint{3.500686in}{1.928573in}}{\pgfqpoint{3.492873in}{1.920759in}}%
\pgfpathcurveto{\pgfqpoint{3.485059in}{1.912946in}}{\pgfqpoint{3.480669in}{1.902347in}}{\pgfqpoint{3.480669in}{1.891297in}}%
\pgfpathcurveto{\pgfqpoint{3.480669in}{1.880247in}}{\pgfqpoint{3.485059in}{1.869648in}}{\pgfqpoint{3.492873in}{1.861834in}}%
\pgfpathcurveto{\pgfqpoint{3.500686in}{1.854020in}}{\pgfqpoint{3.511285in}{1.849630in}}{\pgfqpoint{3.522336in}{1.849630in}}%
\pgfpathclose%
\pgfusepath{stroke,fill}%
\end{pgfscope}%
\begin{pgfscope}%
\pgfpathrectangle{\pgfqpoint{0.600000in}{0.600000in}}{\pgfqpoint{3.900000in}{3.900000in}}%
\pgfusepath{clip}%
\pgfsetbuttcap%
\pgfsetroundjoin%
\definecolor{currentfill}{rgb}{0.121569,0.466667,0.705882}%
\pgfsetfillcolor{currentfill}%
\pgfsetlinewidth{1.003750pt}%
\definecolor{currentstroke}{rgb}{0.121569,0.466667,0.705882}%
\pgfsetstrokecolor{currentstroke}%
\pgfsetdash{}{0pt}%
\pgfpathmoveto{\pgfqpoint{3.052025in}{2.414455in}}%
\pgfpathcurveto{\pgfqpoint{3.063075in}{2.414455in}}{\pgfqpoint{3.073674in}{2.418846in}}{\pgfqpoint{3.081488in}{2.426659in}}%
\pgfpathcurveto{\pgfqpoint{3.089301in}{2.434473in}}{\pgfqpoint{3.093692in}{2.445072in}}{\pgfqpoint{3.093692in}{2.456122in}}%
\pgfpathcurveto{\pgfqpoint{3.093692in}{2.467172in}}{\pgfqpoint{3.089301in}{2.477771in}}{\pgfqpoint{3.081488in}{2.485585in}}%
\pgfpathcurveto{\pgfqpoint{3.073674in}{2.493398in}}{\pgfqpoint{3.063075in}{2.497789in}}{\pgfqpoint{3.052025in}{2.497789in}}%
\pgfpathcurveto{\pgfqpoint{3.040975in}{2.497789in}}{\pgfqpoint{3.030376in}{2.493398in}}{\pgfqpoint{3.022562in}{2.485585in}}%
\pgfpathcurveto{\pgfqpoint{3.014749in}{2.477771in}}{\pgfqpoint{3.010358in}{2.467172in}}{\pgfqpoint{3.010358in}{2.456122in}}%
\pgfpathcurveto{\pgfqpoint{3.010358in}{2.445072in}}{\pgfqpoint{3.014749in}{2.434473in}}{\pgfqpoint{3.022562in}{2.426659in}}%
\pgfpathcurveto{\pgfqpoint{3.030376in}{2.418846in}}{\pgfqpoint{3.040975in}{2.414455in}}{\pgfqpoint{3.052025in}{2.414455in}}%
\pgfpathclose%
\pgfusepath{stroke,fill}%
\end{pgfscope}%
\begin{pgfscope}%
\pgfpathrectangle{\pgfqpoint{0.600000in}{0.600000in}}{\pgfqpoint{3.900000in}{3.900000in}}%
\pgfusepath{clip}%
\pgfsetbuttcap%
\pgfsetroundjoin%
\definecolor{currentfill}{rgb}{0.121569,0.466667,0.705882}%
\pgfsetfillcolor{currentfill}%
\pgfsetlinewidth{1.003750pt}%
\definecolor{currentstroke}{rgb}{0.121569,0.466667,0.705882}%
\pgfsetstrokecolor{currentstroke}%
\pgfsetdash{}{0pt}%
\pgfpathmoveto{\pgfqpoint{2.043323in}{2.409958in}}%
\pgfpathcurveto{\pgfqpoint{2.054374in}{2.409958in}}{\pgfqpoint{2.064973in}{2.414348in}}{\pgfqpoint{2.072786in}{2.422161in}}%
\pgfpathcurveto{\pgfqpoint{2.080600in}{2.429975in}}{\pgfqpoint{2.084990in}{2.440574in}}{\pgfqpoint{2.084990in}{2.451624in}}%
\pgfpathcurveto{\pgfqpoint{2.084990in}{2.462674in}}{\pgfqpoint{2.080600in}{2.473273in}}{\pgfqpoint{2.072786in}{2.481087in}}%
\pgfpathcurveto{\pgfqpoint{2.064973in}{2.488901in}}{\pgfqpoint{2.054374in}{2.493291in}}{\pgfqpoint{2.043323in}{2.493291in}}%
\pgfpathcurveto{\pgfqpoint{2.032273in}{2.493291in}}{\pgfqpoint{2.021674in}{2.488901in}}{\pgfqpoint{2.013861in}{2.481087in}}%
\pgfpathcurveto{\pgfqpoint{2.006047in}{2.473273in}}{\pgfqpoint{2.001657in}{2.462674in}}{\pgfqpoint{2.001657in}{2.451624in}}%
\pgfpathcurveto{\pgfqpoint{2.001657in}{2.440574in}}{\pgfqpoint{2.006047in}{2.429975in}}{\pgfqpoint{2.013861in}{2.422161in}}%
\pgfpathcurveto{\pgfqpoint{2.021674in}{2.414348in}}{\pgfqpoint{2.032273in}{2.409958in}}{\pgfqpoint{2.043323in}{2.409958in}}%
\pgfpathclose%
\pgfusepath{stroke,fill}%
\end{pgfscope}%
\begin{pgfscope}%
\pgfpathrectangle{\pgfqpoint{0.600000in}{0.600000in}}{\pgfqpoint{3.900000in}{3.900000in}}%
\pgfusepath{clip}%
\pgfsetbuttcap%
\pgfsetroundjoin%
\definecolor{currentfill}{rgb}{0.121569,0.466667,0.705882}%
\pgfsetfillcolor{currentfill}%
\pgfsetlinewidth{1.003750pt}%
\definecolor{currentstroke}{rgb}{0.121569,0.466667,0.705882}%
\pgfsetstrokecolor{currentstroke}%
\pgfsetdash{}{0pt}%
\pgfpathmoveto{\pgfqpoint{2.841402in}{1.945175in}}%
\pgfpathcurveto{\pgfqpoint{2.852452in}{1.945175in}}{\pgfqpoint{2.863051in}{1.949565in}}{\pgfqpoint{2.870864in}{1.957379in}}%
\pgfpathcurveto{\pgfqpoint{2.878678in}{1.965192in}}{\pgfqpoint{2.883068in}{1.975791in}}{\pgfqpoint{2.883068in}{1.986842in}}%
\pgfpathcurveto{\pgfqpoint{2.883068in}{1.997892in}}{\pgfqpoint{2.878678in}{2.008491in}}{\pgfqpoint{2.870864in}{2.016304in}}%
\pgfpathcurveto{\pgfqpoint{2.863051in}{2.024118in}}{\pgfqpoint{2.852452in}{2.028508in}}{\pgfqpoint{2.841402in}{2.028508in}}%
\pgfpathcurveto{\pgfqpoint{2.830351in}{2.028508in}}{\pgfqpoint{2.819752in}{2.024118in}}{\pgfqpoint{2.811939in}{2.016304in}}%
\pgfpathcurveto{\pgfqpoint{2.804125in}{2.008491in}}{\pgfqpoint{2.799735in}{1.997892in}}{\pgfqpoint{2.799735in}{1.986842in}}%
\pgfpathcurveto{\pgfqpoint{2.799735in}{1.975791in}}{\pgfqpoint{2.804125in}{1.965192in}}{\pgfqpoint{2.811939in}{1.957379in}}%
\pgfpathcurveto{\pgfqpoint{2.819752in}{1.949565in}}{\pgfqpoint{2.830351in}{1.945175in}}{\pgfqpoint{2.841402in}{1.945175in}}%
\pgfpathclose%
\pgfusepath{stroke,fill}%
\end{pgfscope}%
\begin{pgfscope}%
\pgfpathrectangle{\pgfqpoint{0.600000in}{0.600000in}}{\pgfqpoint{3.900000in}{3.900000in}}%
\pgfusepath{clip}%
\pgfsetbuttcap%
\pgfsetroundjoin%
\definecolor{currentfill}{rgb}{0.121569,0.466667,0.705882}%
\pgfsetfillcolor{currentfill}%
\pgfsetlinewidth{1.003750pt}%
\definecolor{currentstroke}{rgb}{0.121569,0.466667,0.705882}%
\pgfsetstrokecolor{currentstroke}%
\pgfsetdash{}{0pt}%
\pgfpathmoveto{\pgfqpoint{2.155614in}{2.352816in}}%
\pgfpathcurveto{\pgfqpoint{2.166664in}{2.352816in}}{\pgfqpoint{2.177263in}{2.357207in}}{\pgfqpoint{2.185076in}{2.365020in}}%
\pgfpathcurveto{\pgfqpoint{2.192890in}{2.372834in}}{\pgfqpoint{2.197280in}{2.383433in}}{\pgfqpoint{2.197280in}{2.394483in}}%
\pgfpathcurveto{\pgfqpoint{2.197280in}{2.405533in}}{\pgfqpoint{2.192890in}{2.416132in}}{\pgfqpoint{2.185076in}{2.423946in}}%
\pgfpathcurveto{\pgfqpoint{2.177263in}{2.431759in}}{\pgfqpoint{2.166664in}{2.436150in}}{\pgfqpoint{2.155614in}{2.436150in}}%
\pgfpathcurveto{\pgfqpoint{2.144564in}{2.436150in}}{\pgfqpoint{2.133965in}{2.431759in}}{\pgfqpoint{2.126151in}{2.423946in}}%
\pgfpathcurveto{\pgfqpoint{2.118337in}{2.416132in}}{\pgfqpoint{2.113947in}{2.405533in}}{\pgfqpoint{2.113947in}{2.394483in}}%
\pgfpathcurveto{\pgfqpoint{2.113947in}{2.383433in}}{\pgfqpoint{2.118337in}{2.372834in}}{\pgfqpoint{2.126151in}{2.365020in}}%
\pgfpathcurveto{\pgfqpoint{2.133965in}{2.357207in}}{\pgfqpoint{2.144564in}{2.352816in}}{\pgfqpoint{2.155614in}{2.352816in}}%
\pgfpathclose%
\pgfusepath{stroke,fill}%
\end{pgfscope}%
\begin{pgfscope}%
\pgfpathrectangle{\pgfqpoint{0.600000in}{0.600000in}}{\pgfqpoint{3.900000in}{3.900000in}}%
\pgfusepath{clip}%
\pgfsetbuttcap%
\pgfsetroundjoin%
\definecolor{currentfill}{rgb}{0.121569,0.466667,0.705882}%
\pgfsetfillcolor{currentfill}%
\pgfsetlinewidth{1.003750pt}%
\definecolor{currentstroke}{rgb}{0.121569,0.466667,0.705882}%
\pgfsetstrokecolor{currentstroke}%
\pgfsetdash{}{0pt}%
\pgfpathmoveto{\pgfqpoint{2.300093in}{2.323727in}}%
\pgfpathcurveto{\pgfqpoint{2.311143in}{2.323727in}}{\pgfqpoint{2.321742in}{2.328117in}}{\pgfqpoint{2.329556in}{2.335931in}}%
\pgfpathcurveto{\pgfqpoint{2.337369in}{2.343744in}}{\pgfqpoint{2.341759in}{2.354343in}}{\pgfqpoint{2.341759in}{2.365393in}}%
\pgfpathcurveto{\pgfqpoint{2.341759in}{2.376444in}}{\pgfqpoint{2.337369in}{2.387043in}}{\pgfqpoint{2.329556in}{2.394856in}}%
\pgfpathcurveto{\pgfqpoint{2.321742in}{2.402670in}}{\pgfqpoint{2.311143in}{2.407060in}}{\pgfqpoint{2.300093in}{2.407060in}}%
\pgfpathcurveto{\pgfqpoint{2.289043in}{2.407060in}}{\pgfqpoint{2.278444in}{2.402670in}}{\pgfqpoint{2.270630in}{2.394856in}}%
\pgfpathcurveto{\pgfqpoint{2.262816in}{2.387043in}}{\pgfqpoint{2.258426in}{2.376444in}}{\pgfqpoint{2.258426in}{2.365393in}}%
\pgfpathcurveto{\pgfqpoint{2.258426in}{2.354343in}}{\pgfqpoint{2.262816in}{2.343744in}}{\pgfqpoint{2.270630in}{2.335931in}}%
\pgfpathcurveto{\pgfqpoint{2.278444in}{2.328117in}}{\pgfqpoint{2.289043in}{2.323727in}}{\pgfqpoint{2.300093in}{2.323727in}}%
\pgfpathclose%
\pgfusepath{stroke,fill}%
\end{pgfscope}%
\begin{pgfscope}%
\pgfpathrectangle{\pgfqpoint{0.600000in}{0.600000in}}{\pgfqpoint{3.900000in}{3.900000in}}%
\pgfusepath{clip}%
\pgfsetbuttcap%
\pgfsetroundjoin%
\definecolor{currentfill}{rgb}{0.121569,0.466667,0.705882}%
\pgfsetfillcolor{currentfill}%
\pgfsetlinewidth{1.003750pt}%
\definecolor{currentstroke}{rgb}{0.121569,0.466667,0.705882}%
\pgfsetstrokecolor{currentstroke}%
\pgfsetdash{}{0pt}%
\pgfpathmoveto{\pgfqpoint{2.613231in}{2.074260in}}%
\pgfpathcurveto{\pgfqpoint{2.624281in}{2.074260in}}{\pgfqpoint{2.634880in}{2.078650in}}{\pgfqpoint{2.642693in}{2.086464in}}%
\pgfpathcurveto{\pgfqpoint{2.650507in}{2.094278in}}{\pgfqpoint{2.654897in}{2.104877in}}{\pgfqpoint{2.654897in}{2.115927in}}%
\pgfpathcurveto{\pgfqpoint{2.654897in}{2.126977in}}{\pgfqpoint{2.650507in}{2.137576in}}{\pgfqpoint{2.642693in}{2.145390in}}%
\pgfpathcurveto{\pgfqpoint{2.634880in}{2.153203in}}{\pgfqpoint{2.624281in}{2.157593in}}{\pgfqpoint{2.613231in}{2.157593in}}%
\pgfpathcurveto{\pgfqpoint{2.602180in}{2.157593in}}{\pgfqpoint{2.591581in}{2.153203in}}{\pgfqpoint{2.583768in}{2.145390in}}%
\pgfpathcurveto{\pgfqpoint{2.575954in}{2.137576in}}{\pgfqpoint{2.571564in}{2.126977in}}{\pgfqpoint{2.571564in}{2.115927in}}%
\pgfpathcurveto{\pgfqpoint{2.571564in}{2.104877in}}{\pgfqpoint{2.575954in}{2.094278in}}{\pgfqpoint{2.583768in}{2.086464in}}%
\pgfpathcurveto{\pgfqpoint{2.591581in}{2.078650in}}{\pgfqpoint{2.602180in}{2.074260in}}{\pgfqpoint{2.613231in}{2.074260in}}%
\pgfpathclose%
\pgfusepath{stroke,fill}%
\end{pgfscope}%
\begin{pgfscope}%
\pgfpathrectangle{\pgfqpoint{0.600000in}{0.600000in}}{\pgfqpoint{3.900000in}{3.900000in}}%
\pgfusepath{clip}%
\pgfsetbuttcap%
\pgfsetroundjoin%
\definecolor{currentfill}{rgb}{0.121569,0.466667,0.705882}%
\pgfsetfillcolor{currentfill}%
\pgfsetlinewidth{1.003750pt}%
\definecolor{currentstroke}{rgb}{0.121569,0.466667,0.705882}%
\pgfsetstrokecolor{currentstroke}%
\pgfsetdash{}{0pt}%
\pgfpathmoveto{\pgfqpoint{3.458093in}{1.743908in}}%
\pgfpathcurveto{\pgfqpoint{3.469143in}{1.743908in}}{\pgfqpoint{3.479742in}{1.748298in}}{\pgfqpoint{3.487556in}{1.756112in}}%
\pgfpathcurveto{\pgfqpoint{3.495369in}{1.763925in}}{\pgfqpoint{3.499760in}{1.774524in}}{\pgfqpoint{3.499760in}{1.785574in}}%
\pgfpathcurveto{\pgfqpoint{3.499760in}{1.796624in}}{\pgfqpoint{3.495369in}{1.807223in}}{\pgfqpoint{3.487556in}{1.815037in}}%
\pgfpathcurveto{\pgfqpoint{3.479742in}{1.822851in}}{\pgfqpoint{3.469143in}{1.827241in}}{\pgfqpoint{3.458093in}{1.827241in}}%
\pgfpathcurveto{\pgfqpoint{3.447043in}{1.827241in}}{\pgfqpoint{3.436444in}{1.822851in}}{\pgfqpoint{3.428630in}{1.815037in}}%
\pgfpathcurveto{\pgfqpoint{3.420817in}{1.807223in}}{\pgfqpoint{3.416426in}{1.796624in}}{\pgfqpoint{3.416426in}{1.785574in}}%
\pgfpathcurveto{\pgfqpoint{3.416426in}{1.774524in}}{\pgfqpoint{3.420817in}{1.763925in}}{\pgfqpoint{3.428630in}{1.756112in}}%
\pgfpathcurveto{\pgfqpoint{3.436444in}{1.748298in}}{\pgfqpoint{3.447043in}{1.743908in}}{\pgfqpoint{3.458093in}{1.743908in}}%
\pgfpathclose%
\pgfusepath{stroke,fill}%
\end{pgfscope}%
\begin{pgfscope}%
\pgfpathrectangle{\pgfqpoint{0.600000in}{0.600000in}}{\pgfqpoint{3.900000in}{3.900000in}}%
\pgfusepath{clip}%
\pgfsetbuttcap%
\pgfsetroundjoin%
\definecolor{currentfill}{rgb}{0.121569,0.466667,0.705882}%
\pgfsetfillcolor{currentfill}%
\pgfsetlinewidth{1.003750pt}%
\definecolor{currentstroke}{rgb}{0.121569,0.466667,0.705882}%
\pgfsetstrokecolor{currentstroke}%
\pgfsetdash{}{0pt}%
\pgfpathmoveto{\pgfqpoint{2.229568in}{3.138940in}}%
\pgfpathcurveto{\pgfqpoint{2.240618in}{3.138940in}}{\pgfqpoint{2.251218in}{3.143330in}}{\pgfqpoint{2.259031in}{3.151144in}}%
\pgfpathcurveto{\pgfqpoint{2.266845in}{3.158957in}}{\pgfqpoint{2.271235in}{3.169556in}}{\pgfqpoint{2.271235in}{3.180607in}}%
\pgfpathcurveto{\pgfqpoint{2.271235in}{3.191657in}}{\pgfqpoint{2.266845in}{3.202256in}}{\pgfqpoint{2.259031in}{3.210069in}}%
\pgfpathcurveto{\pgfqpoint{2.251218in}{3.217883in}}{\pgfqpoint{2.240618in}{3.222273in}}{\pgfqpoint{2.229568in}{3.222273in}}%
\pgfpathcurveto{\pgfqpoint{2.218518in}{3.222273in}}{\pgfqpoint{2.207919in}{3.217883in}}{\pgfqpoint{2.200106in}{3.210069in}}%
\pgfpathcurveto{\pgfqpoint{2.192292in}{3.202256in}}{\pgfqpoint{2.187902in}{3.191657in}}{\pgfqpoint{2.187902in}{3.180607in}}%
\pgfpathcurveto{\pgfqpoint{2.187902in}{3.169556in}}{\pgfqpoint{2.192292in}{3.158957in}}{\pgfqpoint{2.200106in}{3.151144in}}%
\pgfpathcurveto{\pgfqpoint{2.207919in}{3.143330in}}{\pgfqpoint{2.218518in}{3.138940in}}{\pgfqpoint{2.229568in}{3.138940in}}%
\pgfpathclose%
\pgfusepath{stroke,fill}%
\end{pgfscope}%
\begin{pgfscope}%
\pgfpathrectangle{\pgfqpoint{0.600000in}{0.600000in}}{\pgfqpoint{3.900000in}{3.900000in}}%
\pgfusepath{clip}%
\pgfsetbuttcap%
\pgfsetroundjoin%
\definecolor{currentfill}{rgb}{0.121569,0.466667,0.705882}%
\pgfsetfillcolor{currentfill}%
\pgfsetlinewidth{1.003750pt}%
\definecolor{currentstroke}{rgb}{0.121569,0.466667,0.705882}%
\pgfsetstrokecolor{currentstroke}%
\pgfsetdash{}{0pt}%
\pgfpathmoveto{\pgfqpoint{2.622745in}{3.227543in}}%
\pgfpathcurveto{\pgfqpoint{2.633795in}{3.227543in}}{\pgfqpoint{2.644394in}{3.231933in}}{\pgfqpoint{2.652207in}{3.239747in}}%
\pgfpathcurveto{\pgfqpoint{2.660021in}{3.247561in}}{\pgfqpoint{2.664411in}{3.258160in}}{\pgfqpoint{2.664411in}{3.269210in}}%
\pgfpathcurveto{\pgfqpoint{2.664411in}{3.280260in}}{\pgfqpoint{2.660021in}{3.290859in}}{\pgfqpoint{2.652207in}{3.298673in}}%
\pgfpathcurveto{\pgfqpoint{2.644394in}{3.306486in}}{\pgfqpoint{2.633795in}{3.310877in}}{\pgfqpoint{2.622745in}{3.310877in}}%
\pgfpathcurveto{\pgfqpoint{2.611694in}{3.310877in}}{\pgfqpoint{2.601095in}{3.306486in}}{\pgfqpoint{2.593282in}{3.298673in}}%
\pgfpathcurveto{\pgfqpoint{2.585468in}{3.290859in}}{\pgfqpoint{2.581078in}{3.280260in}}{\pgfqpoint{2.581078in}{3.269210in}}%
\pgfpathcurveto{\pgfqpoint{2.581078in}{3.258160in}}{\pgfqpoint{2.585468in}{3.247561in}}{\pgfqpoint{2.593282in}{3.239747in}}%
\pgfpathcurveto{\pgfqpoint{2.601095in}{3.231933in}}{\pgfqpoint{2.611694in}{3.227543in}}{\pgfqpoint{2.622745in}{3.227543in}}%
\pgfpathclose%
\pgfusepath{stroke,fill}%
\end{pgfscope}%
\begin{pgfscope}%
\pgfpathrectangle{\pgfqpoint{0.600000in}{0.600000in}}{\pgfqpoint{3.900000in}{3.900000in}}%
\pgfusepath{clip}%
\pgfsetbuttcap%
\pgfsetroundjoin%
\definecolor{currentfill}{rgb}{0.121569,0.466667,0.705882}%
\pgfsetfillcolor{currentfill}%
\pgfsetlinewidth{1.003750pt}%
\definecolor{currentstroke}{rgb}{0.121569,0.466667,0.705882}%
\pgfsetstrokecolor{currentstroke}%
\pgfsetdash{}{0pt}%
\pgfpathmoveto{\pgfqpoint{1.897596in}{2.036347in}}%
\pgfpathcurveto{\pgfqpoint{1.908647in}{2.036347in}}{\pgfqpoint{1.919246in}{2.040737in}}{\pgfqpoint{1.927059in}{2.048551in}}%
\pgfpathcurveto{\pgfqpoint{1.934873in}{2.056365in}}{\pgfqpoint{1.939263in}{2.066964in}}{\pgfqpoint{1.939263in}{2.078014in}}%
\pgfpathcurveto{\pgfqpoint{1.939263in}{2.089064in}}{\pgfqpoint{1.934873in}{2.099663in}}{\pgfqpoint{1.927059in}{2.107477in}}%
\pgfpathcurveto{\pgfqpoint{1.919246in}{2.115290in}}{\pgfqpoint{1.908647in}{2.119681in}}{\pgfqpoint{1.897596in}{2.119681in}}%
\pgfpathcurveto{\pgfqpoint{1.886546in}{2.119681in}}{\pgfqpoint{1.875947in}{2.115290in}}{\pgfqpoint{1.868134in}{2.107477in}}%
\pgfpathcurveto{\pgfqpoint{1.860320in}{2.099663in}}{\pgfqpoint{1.855930in}{2.089064in}}{\pgfqpoint{1.855930in}{2.078014in}}%
\pgfpathcurveto{\pgfqpoint{1.855930in}{2.066964in}}{\pgfqpoint{1.860320in}{2.056365in}}{\pgfqpoint{1.868134in}{2.048551in}}%
\pgfpathcurveto{\pgfqpoint{1.875947in}{2.040737in}}{\pgfqpoint{1.886546in}{2.036347in}}{\pgfqpoint{1.897596in}{2.036347in}}%
\pgfpathclose%
\pgfusepath{stroke,fill}%
\end{pgfscope}%
\begin{pgfscope}%
\pgfpathrectangle{\pgfqpoint{0.600000in}{0.600000in}}{\pgfqpoint{3.900000in}{3.900000in}}%
\pgfusepath{clip}%
\pgfsetbuttcap%
\pgfsetroundjoin%
\definecolor{currentfill}{rgb}{0.121569,0.466667,0.705882}%
\pgfsetfillcolor{currentfill}%
\pgfsetlinewidth{1.003750pt}%
\definecolor{currentstroke}{rgb}{0.121569,0.466667,0.705882}%
\pgfsetstrokecolor{currentstroke}%
\pgfsetdash{}{0pt}%
\pgfpathmoveto{\pgfqpoint{2.276557in}{1.621732in}}%
\pgfpathcurveto{\pgfqpoint{2.287608in}{1.621732in}}{\pgfqpoint{2.298207in}{1.626122in}}{\pgfqpoint{2.306020in}{1.633936in}}%
\pgfpathcurveto{\pgfqpoint{2.313834in}{1.641749in}}{\pgfqpoint{2.318224in}{1.652348in}}{\pgfqpoint{2.318224in}{1.663398in}}%
\pgfpathcurveto{\pgfqpoint{2.318224in}{1.674448in}}{\pgfqpoint{2.313834in}{1.685047in}}{\pgfqpoint{2.306020in}{1.692861in}}%
\pgfpathcurveto{\pgfqpoint{2.298207in}{1.700675in}}{\pgfqpoint{2.287608in}{1.705065in}}{\pgfqpoint{2.276557in}{1.705065in}}%
\pgfpathcurveto{\pgfqpoint{2.265507in}{1.705065in}}{\pgfqpoint{2.254908in}{1.700675in}}{\pgfqpoint{2.247095in}{1.692861in}}%
\pgfpathcurveto{\pgfqpoint{2.239281in}{1.685047in}}{\pgfqpoint{2.234891in}{1.674448in}}{\pgfqpoint{2.234891in}{1.663398in}}%
\pgfpathcurveto{\pgfqpoint{2.234891in}{1.652348in}}{\pgfqpoint{2.239281in}{1.641749in}}{\pgfqpoint{2.247095in}{1.633936in}}%
\pgfpathcurveto{\pgfqpoint{2.254908in}{1.626122in}}{\pgfqpoint{2.265507in}{1.621732in}}{\pgfqpoint{2.276557in}{1.621732in}}%
\pgfpathclose%
\pgfusepath{stroke,fill}%
\end{pgfscope}%
\begin{pgfscope}%
\pgfpathrectangle{\pgfqpoint{0.600000in}{0.600000in}}{\pgfqpoint{3.900000in}{3.900000in}}%
\pgfusepath{clip}%
\pgfsetbuttcap%
\pgfsetroundjoin%
\definecolor{currentfill}{rgb}{0.121569,0.466667,0.705882}%
\pgfsetfillcolor{currentfill}%
\pgfsetlinewidth{1.003750pt}%
\definecolor{currentstroke}{rgb}{0.121569,0.466667,0.705882}%
\pgfsetstrokecolor{currentstroke}%
\pgfsetdash{}{0pt}%
\pgfpathmoveto{\pgfqpoint{2.871970in}{2.394571in}}%
\pgfpathcurveto{\pgfqpoint{2.883020in}{2.394571in}}{\pgfqpoint{2.893619in}{2.398962in}}{\pgfqpoint{2.901433in}{2.406775in}}%
\pgfpathcurveto{\pgfqpoint{2.909246in}{2.414589in}}{\pgfqpoint{2.913637in}{2.425188in}}{\pgfqpoint{2.913637in}{2.436238in}}%
\pgfpathcurveto{\pgfqpoint{2.913637in}{2.447288in}}{\pgfqpoint{2.909246in}{2.457887in}}{\pgfqpoint{2.901433in}{2.465701in}}%
\pgfpathcurveto{\pgfqpoint{2.893619in}{2.473515in}}{\pgfqpoint{2.883020in}{2.477905in}}{\pgfqpoint{2.871970in}{2.477905in}}%
\pgfpathcurveto{\pgfqpoint{2.860920in}{2.477905in}}{\pgfqpoint{2.850321in}{2.473515in}}{\pgfqpoint{2.842507in}{2.465701in}}%
\pgfpathcurveto{\pgfqpoint{2.834694in}{2.457887in}}{\pgfqpoint{2.830303in}{2.447288in}}{\pgfqpoint{2.830303in}{2.436238in}}%
\pgfpathcurveto{\pgfqpoint{2.830303in}{2.425188in}}{\pgfqpoint{2.834694in}{2.414589in}}{\pgfqpoint{2.842507in}{2.406775in}}%
\pgfpathcurveto{\pgfqpoint{2.850321in}{2.398962in}}{\pgfqpoint{2.860920in}{2.394571in}}{\pgfqpoint{2.871970in}{2.394571in}}%
\pgfpathclose%
\pgfusepath{stroke,fill}%
\end{pgfscope}%
\begin{pgfscope}%
\pgfpathrectangle{\pgfqpoint{0.600000in}{0.600000in}}{\pgfqpoint{3.900000in}{3.900000in}}%
\pgfusepath{clip}%
\pgfsetbuttcap%
\pgfsetroundjoin%
\definecolor{currentfill}{rgb}{0.121569,0.466667,0.705882}%
\pgfsetfillcolor{currentfill}%
\pgfsetlinewidth{1.003750pt}%
\definecolor{currentstroke}{rgb}{0.121569,0.466667,0.705882}%
\pgfsetstrokecolor{currentstroke}%
\pgfsetdash{}{0pt}%
\pgfpathmoveto{\pgfqpoint{2.947380in}{1.705905in}}%
\pgfpathcurveto{\pgfqpoint{2.958430in}{1.705905in}}{\pgfqpoint{2.969030in}{1.710295in}}{\pgfqpoint{2.976843in}{1.718109in}}%
\pgfpathcurveto{\pgfqpoint{2.984657in}{1.725922in}}{\pgfqpoint{2.989047in}{1.736521in}}{\pgfqpoint{2.989047in}{1.747572in}}%
\pgfpathcurveto{\pgfqpoint{2.989047in}{1.758622in}}{\pgfqpoint{2.984657in}{1.769221in}}{\pgfqpoint{2.976843in}{1.777034in}}%
\pgfpathcurveto{\pgfqpoint{2.969030in}{1.784848in}}{\pgfqpoint{2.958430in}{1.789238in}}{\pgfqpoint{2.947380in}{1.789238in}}%
\pgfpathcurveto{\pgfqpoint{2.936330in}{1.789238in}}{\pgfqpoint{2.925731in}{1.784848in}}{\pgfqpoint{2.917918in}{1.777034in}}%
\pgfpathcurveto{\pgfqpoint{2.910104in}{1.769221in}}{\pgfqpoint{2.905714in}{1.758622in}}{\pgfqpoint{2.905714in}{1.747572in}}%
\pgfpathcurveto{\pgfqpoint{2.905714in}{1.736521in}}{\pgfqpoint{2.910104in}{1.725922in}}{\pgfqpoint{2.917918in}{1.718109in}}%
\pgfpathcurveto{\pgfqpoint{2.925731in}{1.710295in}}{\pgfqpoint{2.936330in}{1.705905in}}{\pgfqpoint{2.947380in}{1.705905in}}%
\pgfpathclose%
\pgfusepath{stroke,fill}%
\end{pgfscope}%
\begin{pgfscope}%
\pgfpathrectangle{\pgfqpoint{0.600000in}{0.600000in}}{\pgfqpoint{3.900000in}{3.900000in}}%
\pgfusepath{clip}%
\pgfsetbuttcap%
\pgfsetroundjoin%
\definecolor{currentfill}{rgb}{0.121569,0.466667,0.705882}%
\pgfsetfillcolor{currentfill}%
\pgfsetlinewidth{1.003750pt}%
\definecolor{currentstroke}{rgb}{0.121569,0.466667,0.705882}%
\pgfsetstrokecolor{currentstroke}%
\pgfsetdash{}{0pt}%
\pgfpathmoveto{\pgfqpoint{2.754158in}{2.304653in}}%
\pgfpathcurveto{\pgfqpoint{2.765208in}{2.304653in}}{\pgfqpoint{2.775807in}{2.309043in}}{\pgfqpoint{2.783621in}{2.316857in}}%
\pgfpathcurveto{\pgfqpoint{2.791434in}{2.324670in}}{\pgfqpoint{2.795824in}{2.335269in}}{\pgfqpoint{2.795824in}{2.346319in}}%
\pgfpathcurveto{\pgfqpoint{2.795824in}{2.357370in}}{\pgfqpoint{2.791434in}{2.367969in}}{\pgfqpoint{2.783621in}{2.375782in}}%
\pgfpathcurveto{\pgfqpoint{2.775807in}{2.383596in}}{\pgfqpoint{2.765208in}{2.387986in}}{\pgfqpoint{2.754158in}{2.387986in}}%
\pgfpathcurveto{\pgfqpoint{2.743108in}{2.387986in}}{\pgfqpoint{2.732509in}{2.383596in}}{\pgfqpoint{2.724695in}{2.375782in}}%
\pgfpathcurveto{\pgfqpoint{2.716881in}{2.367969in}}{\pgfqpoint{2.712491in}{2.357370in}}{\pgfqpoint{2.712491in}{2.346319in}}%
\pgfpathcurveto{\pgfqpoint{2.712491in}{2.335269in}}{\pgfqpoint{2.716881in}{2.324670in}}{\pgfqpoint{2.724695in}{2.316857in}}%
\pgfpathcurveto{\pgfqpoint{2.732509in}{2.309043in}}{\pgfqpoint{2.743108in}{2.304653in}}{\pgfqpoint{2.754158in}{2.304653in}}%
\pgfpathclose%
\pgfusepath{stroke,fill}%
\end{pgfscope}%
\begin{pgfscope}%
\pgfpathrectangle{\pgfqpoint{0.600000in}{0.600000in}}{\pgfqpoint{3.900000in}{3.900000in}}%
\pgfusepath{clip}%
\pgfsetbuttcap%
\pgfsetroundjoin%
\definecolor{currentfill}{rgb}{0.121569,0.466667,0.705882}%
\pgfsetfillcolor{currentfill}%
\pgfsetlinewidth{1.003750pt}%
\definecolor{currentstroke}{rgb}{0.121569,0.466667,0.705882}%
\pgfsetstrokecolor{currentstroke}%
\pgfsetdash{}{0pt}%
\pgfpathmoveto{\pgfqpoint{3.158721in}{2.879274in}}%
\pgfpathcurveto{\pgfqpoint{3.169772in}{2.879274in}}{\pgfqpoint{3.180371in}{2.883664in}}{\pgfqpoint{3.188184in}{2.891478in}}%
\pgfpathcurveto{\pgfqpoint{3.195998in}{2.899292in}}{\pgfqpoint{3.200388in}{2.909891in}}{\pgfqpoint{3.200388in}{2.920941in}}%
\pgfpathcurveto{\pgfqpoint{3.200388in}{2.931991in}}{\pgfqpoint{3.195998in}{2.942590in}}{\pgfqpoint{3.188184in}{2.950404in}}%
\pgfpathcurveto{\pgfqpoint{3.180371in}{2.958217in}}{\pgfqpoint{3.169772in}{2.962608in}}{\pgfqpoint{3.158721in}{2.962608in}}%
\pgfpathcurveto{\pgfqpoint{3.147671in}{2.962608in}}{\pgfqpoint{3.137072in}{2.958217in}}{\pgfqpoint{3.129259in}{2.950404in}}%
\pgfpathcurveto{\pgfqpoint{3.121445in}{2.942590in}}{\pgfqpoint{3.117055in}{2.931991in}}{\pgfqpoint{3.117055in}{2.920941in}}%
\pgfpathcurveto{\pgfqpoint{3.117055in}{2.909891in}}{\pgfqpoint{3.121445in}{2.899292in}}{\pgfqpoint{3.129259in}{2.891478in}}%
\pgfpathcurveto{\pgfqpoint{3.137072in}{2.883664in}}{\pgfqpoint{3.147671in}{2.879274in}}{\pgfqpoint{3.158721in}{2.879274in}}%
\pgfpathclose%
\pgfusepath{stroke,fill}%
\end{pgfscope}%
\begin{pgfscope}%
\pgfpathrectangle{\pgfqpoint{0.600000in}{0.600000in}}{\pgfqpoint{3.900000in}{3.900000in}}%
\pgfusepath{clip}%
\pgfsetbuttcap%
\pgfsetroundjoin%
\definecolor{currentfill}{rgb}{0.121569,0.466667,0.705882}%
\pgfsetfillcolor{currentfill}%
\pgfsetlinewidth{1.003750pt}%
\definecolor{currentstroke}{rgb}{0.121569,0.466667,0.705882}%
\pgfsetstrokecolor{currentstroke}%
\pgfsetdash{}{0pt}%
\pgfpathmoveto{\pgfqpoint{1.942724in}{2.399088in}}%
\pgfpathcurveto{\pgfqpoint{1.953774in}{2.399088in}}{\pgfqpoint{1.964373in}{2.403479in}}{\pgfqpoint{1.972186in}{2.411292in}}%
\pgfpathcurveto{\pgfqpoint{1.980000in}{2.419106in}}{\pgfqpoint{1.984390in}{2.429705in}}{\pgfqpoint{1.984390in}{2.440755in}}%
\pgfpathcurveto{\pgfqpoint{1.984390in}{2.451805in}}{\pgfqpoint{1.980000in}{2.462404in}}{\pgfqpoint{1.972186in}{2.470218in}}%
\pgfpathcurveto{\pgfqpoint{1.964373in}{2.478031in}}{\pgfqpoint{1.953774in}{2.482422in}}{\pgfqpoint{1.942724in}{2.482422in}}%
\pgfpathcurveto{\pgfqpoint{1.931674in}{2.482422in}}{\pgfqpoint{1.921074in}{2.478031in}}{\pgfqpoint{1.913261in}{2.470218in}}%
\pgfpathcurveto{\pgfqpoint{1.905447in}{2.462404in}}{\pgfqpoint{1.901057in}{2.451805in}}{\pgfqpoint{1.901057in}{2.440755in}}%
\pgfpathcurveto{\pgfqpoint{1.901057in}{2.429705in}}{\pgfqpoint{1.905447in}{2.419106in}}{\pgfqpoint{1.913261in}{2.411292in}}%
\pgfpathcurveto{\pgfqpoint{1.921074in}{2.403479in}}{\pgfqpoint{1.931674in}{2.399088in}}{\pgfqpoint{1.942724in}{2.399088in}}%
\pgfpathclose%
\pgfusepath{stroke,fill}%
\end{pgfscope}%
\begin{pgfscope}%
\pgfpathrectangle{\pgfqpoint{0.600000in}{0.600000in}}{\pgfqpoint{3.900000in}{3.900000in}}%
\pgfusepath{clip}%
\pgfsetbuttcap%
\pgfsetroundjoin%
\definecolor{currentfill}{rgb}{0.121569,0.466667,0.705882}%
\pgfsetfillcolor{currentfill}%
\pgfsetlinewidth{1.003750pt}%
\definecolor{currentstroke}{rgb}{0.121569,0.466667,0.705882}%
\pgfsetstrokecolor{currentstroke}%
\pgfsetdash{}{0pt}%
\pgfpathmoveto{\pgfqpoint{2.373532in}{1.989316in}}%
\pgfpathcurveto{\pgfqpoint{2.384582in}{1.989316in}}{\pgfqpoint{2.395181in}{1.993706in}}{\pgfqpoint{2.402995in}{2.001520in}}%
\pgfpathcurveto{\pgfqpoint{2.410809in}{2.009334in}}{\pgfqpoint{2.415199in}{2.019933in}}{\pgfqpoint{2.415199in}{2.030983in}}%
\pgfpathcurveto{\pgfqpoint{2.415199in}{2.042033in}}{\pgfqpoint{2.410809in}{2.052632in}}{\pgfqpoint{2.402995in}{2.060446in}}%
\pgfpathcurveto{\pgfqpoint{2.395181in}{2.068259in}}{\pgfqpoint{2.384582in}{2.072649in}}{\pgfqpoint{2.373532in}{2.072649in}}%
\pgfpathcurveto{\pgfqpoint{2.362482in}{2.072649in}}{\pgfqpoint{2.351883in}{2.068259in}}{\pgfqpoint{2.344069in}{2.060446in}}%
\pgfpathcurveto{\pgfqpoint{2.336256in}{2.052632in}}{\pgfqpoint{2.331866in}{2.042033in}}{\pgfqpoint{2.331866in}{2.030983in}}%
\pgfpathcurveto{\pgfqpoint{2.331866in}{2.019933in}}{\pgfqpoint{2.336256in}{2.009334in}}{\pgfqpoint{2.344069in}{2.001520in}}%
\pgfpathcurveto{\pgfqpoint{2.351883in}{1.993706in}}{\pgfqpoint{2.362482in}{1.989316in}}{\pgfqpoint{2.373532in}{1.989316in}}%
\pgfpathclose%
\pgfusepath{stroke,fill}%
\end{pgfscope}%
\begin{pgfscope}%
\pgfpathrectangle{\pgfqpoint{0.600000in}{0.600000in}}{\pgfqpoint{3.900000in}{3.900000in}}%
\pgfusepath{clip}%
\pgfsetbuttcap%
\pgfsetroundjoin%
\definecolor{currentfill}{rgb}{0.121569,0.466667,0.705882}%
\pgfsetfillcolor{currentfill}%
\pgfsetlinewidth{1.003750pt}%
\definecolor{currentstroke}{rgb}{0.121569,0.466667,0.705882}%
\pgfsetstrokecolor{currentstroke}%
\pgfsetdash{}{0pt}%
\pgfpathmoveto{\pgfqpoint{2.027619in}{2.563675in}}%
\pgfpathcurveto{\pgfqpoint{2.038669in}{2.563675in}}{\pgfqpoint{2.049268in}{2.568065in}}{\pgfqpoint{2.057081in}{2.575879in}}%
\pgfpathcurveto{\pgfqpoint{2.064895in}{2.583692in}}{\pgfqpoint{2.069285in}{2.594291in}}{\pgfqpoint{2.069285in}{2.605341in}}%
\pgfpathcurveto{\pgfqpoint{2.069285in}{2.616391in}}{\pgfqpoint{2.064895in}{2.626991in}}{\pgfqpoint{2.057081in}{2.634804in}}%
\pgfpathcurveto{\pgfqpoint{2.049268in}{2.642618in}}{\pgfqpoint{2.038669in}{2.647008in}}{\pgfqpoint{2.027619in}{2.647008in}}%
\pgfpathcurveto{\pgfqpoint{2.016569in}{2.647008in}}{\pgfqpoint{2.005970in}{2.642618in}}{\pgfqpoint{1.998156in}{2.634804in}}%
\pgfpathcurveto{\pgfqpoint{1.990342in}{2.626991in}}{\pgfqpoint{1.985952in}{2.616391in}}{\pgfqpoint{1.985952in}{2.605341in}}%
\pgfpathcurveto{\pgfqpoint{1.985952in}{2.594291in}}{\pgfqpoint{1.990342in}{2.583692in}}{\pgfqpoint{1.998156in}{2.575879in}}%
\pgfpathcurveto{\pgfqpoint{2.005970in}{2.568065in}}{\pgfqpoint{2.016569in}{2.563675in}}{\pgfqpoint{2.027619in}{2.563675in}}%
\pgfpathclose%
\pgfusepath{stroke,fill}%
\end{pgfscope}%
\begin{pgfscope}%
\pgfpathrectangle{\pgfqpoint{0.600000in}{0.600000in}}{\pgfqpoint{3.900000in}{3.900000in}}%
\pgfusepath{clip}%
\pgfsetbuttcap%
\pgfsetroundjoin%
\definecolor{currentfill}{rgb}{0.121569,0.466667,0.705882}%
\pgfsetfillcolor{currentfill}%
\pgfsetlinewidth{1.003750pt}%
\definecolor{currentstroke}{rgb}{0.121569,0.466667,0.705882}%
\pgfsetstrokecolor{currentstroke}%
\pgfsetdash{}{0pt}%
\pgfpathmoveto{\pgfqpoint{1.453791in}{3.031390in}}%
\pgfpathcurveto{\pgfqpoint{1.464842in}{3.031390in}}{\pgfqpoint{1.475441in}{3.035780in}}{\pgfqpoint{1.483254in}{3.043593in}}%
\pgfpathcurveto{\pgfqpoint{1.491068in}{3.051407in}}{\pgfqpoint{1.495458in}{3.062006in}}{\pgfqpoint{1.495458in}{3.073056in}}%
\pgfpathcurveto{\pgfqpoint{1.495458in}{3.084106in}}{\pgfqpoint{1.491068in}{3.094705in}}{\pgfqpoint{1.483254in}{3.102519in}}%
\pgfpathcurveto{\pgfqpoint{1.475441in}{3.110333in}}{\pgfqpoint{1.464842in}{3.114723in}}{\pgfqpoint{1.453791in}{3.114723in}}%
\pgfpathcurveto{\pgfqpoint{1.442741in}{3.114723in}}{\pgfqpoint{1.432142in}{3.110333in}}{\pgfqpoint{1.424329in}{3.102519in}}%
\pgfpathcurveto{\pgfqpoint{1.416515in}{3.094705in}}{\pgfqpoint{1.412125in}{3.084106in}}{\pgfqpoint{1.412125in}{3.073056in}}%
\pgfpathcurveto{\pgfqpoint{1.412125in}{3.062006in}}{\pgfqpoint{1.416515in}{3.051407in}}{\pgfqpoint{1.424329in}{3.043593in}}%
\pgfpathcurveto{\pgfqpoint{1.432142in}{3.035780in}}{\pgfqpoint{1.442741in}{3.031390in}}{\pgfqpoint{1.453791in}{3.031390in}}%
\pgfpathclose%
\pgfusepath{stroke,fill}%
\end{pgfscope}%
\begin{pgfscope}%
\pgfpathrectangle{\pgfqpoint{0.600000in}{0.600000in}}{\pgfqpoint{3.900000in}{3.900000in}}%
\pgfusepath{clip}%
\pgfsetbuttcap%
\pgfsetroundjoin%
\definecolor{currentfill}{rgb}{0.121569,0.466667,0.705882}%
\pgfsetfillcolor{currentfill}%
\pgfsetlinewidth{1.003750pt}%
\definecolor{currentstroke}{rgb}{0.121569,0.466667,0.705882}%
\pgfsetstrokecolor{currentstroke}%
\pgfsetdash{}{0pt}%
\pgfpathmoveto{\pgfqpoint{1.998806in}{2.381927in}}%
\pgfpathcurveto{\pgfqpoint{2.009856in}{2.381927in}}{\pgfqpoint{2.020455in}{2.386317in}}{\pgfqpoint{2.028268in}{2.394131in}}%
\pgfpathcurveto{\pgfqpoint{2.036082in}{2.401944in}}{\pgfqpoint{2.040472in}{2.412543in}}{\pgfqpoint{2.040472in}{2.423593in}}%
\pgfpathcurveto{\pgfqpoint{2.040472in}{2.434643in}}{\pgfqpoint{2.036082in}{2.445243in}}{\pgfqpoint{2.028268in}{2.453056in}}%
\pgfpathcurveto{\pgfqpoint{2.020455in}{2.460870in}}{\pgfqpoint{2.009856in}{2.465260in}}{\pgfqpoint{1.998806in}{2.465260in}}%
\pgfpathcurveto{\pgfqpoint{1.987755in}{2.465260in}}{\pgfqpoint{1.977156in}{2.460870in}}{\pgfqpoint{1.969343in}{2.453056in}}%
\pgfpathcurveto{\pgfqpoint{1.961529in}{2.445243in}}{\pgfqpoint{1.957139in}{2.434643in}}{\pgfqpoint{1.957139in}{2.423593in}}%
\pgfpathcurveto{\pgfqpoint{1.957139in}{2.412543in}}{\pgfqpoint{1.961529in}{2.401944in}}{\pgfqpoint{1.969343in}{2.394131in}}%
\pgfpathcurveto{\pgfqpoint{1.977156in}{2.386317in}}{\pgfqpoint{1.987755in}{2.381927in}}{\pgfqpoint{1.998806in}{2.381927in}}%
\pgfpathclose%
\pgfusepath{stroke,fill}%
\end{pgfscope}%
\begin{pgfscope}%
\pgfpathrectangle{\pgfqpoint{0.600000in}{0.600000in}}{\pgfqpoint{3.900000in}{3.900000in}}%
\pgfusepath{clip}%
\pgfsetbuttcap%
\pgfsetroundjoin%
\definecolor{currentfill}{rgb}{0.121569,0.466667,0.705882}%
\pgfsetfillcolor{currentfill}%
\pgfsetlinewidth{1.003750pt}%
\definecolor{currentstroke}{rgb}{0.121569,0.466667,0.705882}%
\pgfsetstrokecolor{currentstroke}%
\pgfsetdash{}{0pt}%
\pgfpathmoveto{\pgfqpoint{2.321002in}{2.744102in}}%
\pgfpathcurveto{\pgfqpoint{2.332053in}{2.744102in}}{\pgfqpoint{2.342652in}{2.748493in}}{\pgfqpoint{2.350465in}{2.756306in}}%
\pgfpathcurveto{\pgfqpoint{2.358279in}{2.764120in}}{\pgfqpoint{2.362669in}{2.774719in}}{\pgfqpoint{2.362669in}{2.785769in}}%
\pgfpathcurveto{\pgfqpoint{2.362669in}{2.796819in}}{\pgfqpoint{2.358279in}{2.807418in}}{\pgfqpoint{2.350465in}{2.815232in}}%
\pgfpathcurveto{\pgfqpoint{2.342652in}{2.823045in}}{\pgfqpoint{2.332053in}{2.827436in}}{\pgfqpoint{2.321002in}{2.827436in}}%
\pgfpathcurveto{\pgfqpoint{2.309952in}{2.827436in}}{\pgfqpoint{2.299353in}{2.823045in}}{\pgfqpoint{2.291540in}{2.815232in}}%
\pgfpathcurveto{\pgfqpoint{2.283726in}{2.807418in}}{\pgfqpoint{2.279336in}{2.796819in}}{\pgfqpoint{2.279336in}{2.785769in}}%
\pgfpathcurveto{\pgfqpoint{2.279336in}{2.774719in}}{\pgfqpoint{2.283726in}{2.764120in}}{\pgfqpoint{2.291540in}{2.756306in}}%
\pgfpathcurveto{\pgfqpoint{2.299353in}{2.748493in}}{\pgfqpoint{2.309952in}{2.744102in}}{\pgfqpoint{2.321002in}{2.744102in}}%
\pgfpathclose%
\pgfusepath{stroke,fill}%
\end{pgfscope}%
\begin{pgfscope}%
\pgfpathrectangle{\pgfqpoint{0.600000in}{0.600000in}}{\pgfqpoint{3.900000in}{3.900000in}}%
\pgfusepath{clip}%
\pgfsetbuttcap%
\pgfsetroundjoin%
\definecolor{currentfill}{rgb}{0.121569,0.466667,0.705882}%
\pgfsetfillcolor{currentfill}%
\pgfsetlinewidth{1.003750pt}%
\definecolor{currentstroke}{rgb}{0.121569,0.466667,0.705882}%
\pgfsetstrokecolor{currentstroke}%
\pgfsetdash{}{0pt}%
\pgfpathmoveto{\pgfqpoint{3.081331in}{1.149773in}}%
\pgfpathcurveto{\pgfqpoint{3.092381in}{1.149773in}}{\pgfqpoint{3.102980in}{1.154163in}}{\pgfqpoint{3.110793in}{1.161977in}}%
\pgfpathcurveto{\pgfqpoint{3.118607in}{1.169790in}}{\pgfqpoint{3.122997in}{1.180389in}}{\pgfqpoint{3.122997in}{1.191439in}}%
\pgfpathcurveto{\pgfqpoint{3.122997in}{1.202490in}}{\pgfqpoint{3.118607in}{1.213089in}}{\pgfqpoint{3.110793in}{1.220902in}}%
\pgfpathcurveto{\pgfqpoint{3.102980in}{1.228716in}}{\pgfqpoint{3.092381in}{1.233106in}}{\pgfqpoint{3.081331in}{1.233106in}}%
\pgfpathcurveto{\pgfqpoint{3.070280in}{1.233106in}}{\pgfqpoint{3.059681in}{1.228716in}}{\pgfqpoint{3.051868in}{1.220902in}}%
\pgfpathcurveto{\pgfqpoint{3.044054in}{1.213089in}}{\pgfqpoint{3.039664in}{1.202490in}}{\pgfqpoint{3.039664in}{1.191439in}}%
\pgfpathcurveto{\pgfqpoint{3.039664in}{1.180389in}}{\pgfqpoint{3.044054in}{1.169790in}}{\pgfqpoint{3.051868in}{1.161977in}}%
\pgfpathcurveto{\pgfqpoint{3.059681in}{1.154163in}}{\pgfqpoint{3.070280in}{1.149773in}}{\pgfqpoint{3.081331in}{1.149773in}}%
\pgfpathclose%
\pgfusepath{stroke,fill}%
\end{pgfscope}%
\begin{pgfscope}%
\pgfpathrectangle{\pgfqpoint{0.600000in}{0.600000in}}{\pgfqpoint{3.900000in}{3.900000in}}%
\pgfusepath{clip}%
\pgfsetbuttcap%
\pgfsetroundjoin%
\definecolor{currentfill}{rgb}{0.121569,0.466667,0.705882}%
\pgfsetfillcolor{currentfill}%
\pgfsetlinewidth{1.003750pt}%
\definecolor{currentstroke}{rgb}{0.121569,0.466667,0.705882}%
\pgfsetstrokecolor{currentstroke}%
\pgfsetdash{}{0pt}%
\pgfpathmoveto{\pgfqpoint{3.358167in}{3.321232in}}%
\pgfpathcurveto{\pgfqpoint{3.369218in}{3.321232in}}{\pgfqpoint{3.379817in}{3.325622in}}{\pgfqpoint{3.387630in}{3.333436in}}%
\pgfpathcurveto{\pgfqpoint{3.395444in}{3.341250in}}{\pgfqpoint{3.399834in}{3.351849in}}{\pgfqpoint{3.399834in}{3.362899in}}%
\pgfpathcurveto{\pgfqpoint{3.399834in}{3.373949in}}{\pgfqpoint{3.395444in}{3.384548in}}{\pgfqpoint{3.387630in}{3.392362in}}%
\pgfpathcurveto{\pgfqpoint{3.379817in}{3.400175in}}{\pgfqpoint{3.369218in}{3.404566in}}{\pgfqpoint{3.358167in}{3.404566in}}%
\pgfpathcurveto{\pgfqpoint{3.347117in}{3.404566in}}{\pgfqpoint{3.336518in}{3.400175in}}{\pgfqpoint{3.328705in}{3.392362in}}%
\pgfpathcurveto{\pgfqpoint{3.320891in}{3.384548in}}{\pgfqpoint{3.316501in}{3.373949in}}{\pgfqpoint{3.316501in}{3.362899in}}%
\pgfpathcurveto{\pgfqpoint{3.316501in}{3.351849in}}{\pgfqpoint{3.320891in}{3.341250in}}{\pgfqpoint{3.328705in}{3.333436in}}%
\pgfpathcurveto{\pgfqpoint{3.336518in}{3.325622in}}{\pgfqpoint{3.347117in}{3.321232in}}{\pgfqpoint{3.358167in}{3.321232in}}%
\pgfpathclose%
\pgfusepath{stroke,fill}%
\end{pgfscope}%
\begin{pgfscope}%
\pgfpathrectangle{\pgfqpoint{0.600000in}{0.600000in}}{\pgfqpoint{3.900000in}{3.900000in}}%
\pgfusepath{clip}%
\pgfsetbuttcap%
\pgfsetroundjoin%
\definecolor{currentfill}{rgb}{0.121569,0.466667,0.705882}%
\pgfsetfillcolor{currentfill}%
\pgfsetlinewidth{1.003750pt}%
\definecolor{currentstroke}{rgb}{0.121569,0.466667,0.705882}%
\pgfsetstrokecolor{currentstroke}%
\pgfsetdash{}{0pt}%
\pgfpathmoveto{\pgfqpoint{2.504389in}{1.954082in}}%
\pgfpathcurveto{\pgfqpoint{2.515439in}{1.954082in}}{\pgfqpoint{2.526038in}{1.958472in}}{\pgfqpoint{2.533852in}{1.966286in}}%
\pgfpathcurveto{\pgfqpoint{2.541665in}{1.974100in}}{\pgfqpoint{2.546055in}{1.984699in}}{\pgfqpoint{2.546055in}{1.995749in}}%
\pgfpathcurveto{\pgfqpoint{2.546055in}{2.006799in}}{\pgfqpoint{2.541665in}{2.017398in}}{\pgfqpoint{2.533852in}{2.025212in}}%
\pgfpathcurveto{\pgfqpoint{2.526038in}{2.033025in}}{\pgfqpoint{2.515439in}{2.037416in}}{\pgfqpoint{2.504389in}{2.037416in}}%
\pgfpathcurveto{\pgfqpoint{2.493339in}{2.037416in}}{\pgfqpoint{2.482740in}{2.033025in}}{\pgfqpoint{2.474926in}{2.025212in}}%
\pgfpathcurveto{\pgfqpoint{2.467112in}{2.017398in}}{\pgfqpoint{2.462722in}{2.006799in}}{\pgfqpoint{2.462722in}{1.995749in}}%
\pgfpathcurveto{\pgfqpoint{2.462722in}{1.984699in}}{\pgfqpoint{2.467112in}{1.974100in}}{\pgfqpoint{2.474926in}{1.966286in}}%
\pgfpathcurveto{\pgfqpoint{2.482740in}{1.958472in}}{\pgfqpoint{2.493339in}{1.954082in}}{\pgfqpoint{2.504389in}{1.954082in}}%
\pgfpathclose%
\pgfusepath{stroke,fill}%
\end{pgfscope}%
\begin{pgfscope}%
\pgfpathrectangle{\pgfqpoint{0.600000in}{0.600000in}}{\pgfqpoint{3.900000in}{3.900000in}}%
\pgfusepath{clip}%
\pgfsetbuttcap%
\pgfsetroundjoin%
\definecolor{currentfill}{rgb}{0.121569,0.466667,0.705882}%
\pgfsetfillcolor{currentfill}%
\pgfsetlinewidth{1.003750pt}%
\definecolor{currentstroke}{rgb}{0.121569,0.466667,0.705882}%
\pgfsetstrokecolor{currentstroke}%
\pgfsetdash{}{0pt}%
\pgfpathmoveto{\pgfqpoint{3.085982in}{3.186041in}}%
\pgfpathcurveto{\pgfqpoint{3.097032in}{3.186041in}}{\pgfqpoint{3.107631in}{3.190432in}}{\pgfqpoint{3.115445in}{3.198245in}}%
\pgfpathcurveto{\pgfqpoint{3.123258in}{3.206059in}}{\pgfqpoint{3.127649in}{3.216658in}}{\pgfqpoint{3.127649in}{3.227708in}}%
\pgfpathcurveto{\pgfqpoint{3.127649in}{3.238758in}}{\pgfqpoint{3.123258in}{3.249357in}}{\pgfqpoint{3.115445in}{3.257171in}}%
\pgfpathcurveto{\pgfqpoint{3.107631in}{3.264984in}}{\pgfqpoint{3.097032in}{3.269375in}}{\pgfqpoint{3.085982in}{3.269375in}}%
\pgfpathcurveto{\pgfqpoint{3.074932in}{3.269375in}}{\pgfqpoint{3.064333in}{3.264984in}}{\pgfqpoint{3.056519in}{3.257171in}}%
\pgfpathcurveto{\pgfqpoint{3.048705in}{3.249357in}}{\pgfqpoint{3.044315in}{3.238758in}}{\pgfqpoint{3.044315in}{3.227708in}}%
\pgfpathcurveto{\pgfqpoint{3.044315in}{3.216658in}}{\pgfqpoint{3.048705in}{3.206059in}}{\pgfqpoint{3.056519in}{3.198245in}}%
\pgfpathcurveto{\pgfqpoint{3.064333in}{3.190432in}}{\pgfqpoint{3.074932in}{3.186041in}}{\pgfqpoint{3.085982in}{3.186041in}}%
\pgfpathclose%
\pgfusepath{stroke,fill}%
\end{pgfscope}%
\begin{pgfscope}%
\pgfpathrectangle{\pgfqpoint{0.600000in}{0.600000in}}{\pgfqpoint{3.900000in}{3.900000in}}%
\pgfusepath{clip}%
\pgfsetbuttcap%
\pgfsetroundjoin%
\definecolor{currentfill}{rgb}{0.121569,0.466667,0.705882}%
\pgfsetfillcolor{currentfill}%
\pgfsetlinewidth{1.003750pt}%
\definecolor{currentstroke}{rgb}{0.121569,0.466667,0.705882}%
\pgfsetstrokecolor{currentstroke}%
\pgfsetdash{}{0pt}%
\pgfpathmoveto{\pgfqpoint{1.917216in}{2.034602in}}%
\pgfpathcurveto{\pgfqpoint{1.928266in}{2.034602in}}{\pgfqpoint{1.938865in}{2.038992in}}{\pgfqpoint{1.946679in}{2.046806in}}%
\pgfpathcurveto{\pgfqpoint{1.954493in}{2.054620in}}{\pgfqpoint{1.958883in}{2.065219in}}{\pgfqpoint{1.958883in}{2.076269in}}%
\pgfpathcurveto{\pgfqpoint{1.958883in}{2.087319in}}{\pgfqpoint{1.954493in}{2.097918in}}{\pgfqpoint{1.946679in}{2.105732in}}%
\pgfpathcurveto{\pgfqpoint{1.938865in}{2.113545in}}{\pgfqpoint{1.928266in}{2.117935in}}{\pgfqpoint{1.917216in}{2.117935in}}%
\pgfpathcurveto{\pgfqpoint{1.906166in}{2.117935in}}{\pgfqpoint{1.895567in}{2.113545in}}{\pgfqpoint{1.887753in}{2.105732in}}%
\pgfpathcurveto{\pgfqpoint{1.879940in}{2.097918in}}{\pgfqpoint{1.875550in}{2.087319in}}{\pgfqpoint{1.875550in}{2.076269in}}%
\pgfpathcurveto{\pgfqpoint{1.875550in}{2.065219in}}{\pgfqpoint{1.879940in}{2.054620in}}{\pgfqpoint{1.887753in}{2.046806in}}%
\pgfpathcurveto{\pgfqpoint{1.895567in}{2.038992in}}{\pgfqpoint{1.906166in}{2.034602in}}{\pgfqpoint{1.917216in}{2.034602in}}%
\pgfpathclose%
\pgfusepath{stroke,fill}%
\end{pgfscope}%
\begin{pgfscope}%
\pgfpathrectangle{\pgfqpoint{0.600000in}{0.600000in}}{\pgfqpoint{3.900000in}{3.900000in}}%
\pgfusepath{clip}%
\pgfsetbuttcap%
\pgfsetroundjoin%
\definecolor{currentfill}{rgb}{0.121569,0.466667,0.705882}%
\pgfsetfillcolor{currentfill}%
\pgfsetlinewidth{1.003750pt}%
\definecolor{currentstroke}{rgb}{0.121569,0.466667,0.705882}%
\pgfsetstrokecolor{currentstroke}%
\pgfsetdash{}{0pt}%
\pgfpathmoveto{\pgfqpoint{3.248544in}{2.695462in}}%
\pgfpathcurveto{\pgfqpoint{3.259594in}{2.695462in}}{\pgfqpoint{3.270193in}{2.699852in}}{\pgfqpoint{3.278006in}{2.707666in}}%
\pgfpathcurveto{\pgfqpoint{3.285820in}{2.715479in}}{\pgfqpoint{3.290210in}{2.726078in}}{\pgfqpoint{3.290210in}{2.737128in}}%
\pgfpathcurveto{\pgfqpoint{3.290210in}{2.748179in}}{\pgfqpoint{3.285820in}{2.758778in}}{\pgfqpoint{3.278006in}{2.766591in}}%
\pgfpathcurveto{\pgfqpoint{3.270193in}{2.774405in}}{\pgfqpoint{3.259594in}{2.778795in}}{\pgfqpoint{3.248544in}{2.778795in}}%
\pgfpathcurveto{\pgfqpoint{3.237493in}{2.778795in}}{\pgfqpoint{3.226894in}{2.774405in}}{\pgfqpoint{3.219081in}{2.766591in}}%
\pgfpathcurveto{\pgfqpoint{3.211267in}{2.758778in}}{\pgfqpoint{3.206877in}{2.748179in}}{\pgfqpoint{3.206877in}{2.737128in}}%
\pgfpathcurveto{\pgfqpoint{3.206877in}{2.726078in}}{\pgfqpoint{3.211267in}{2.715479in}}{\pgfqpoint{3.219081in}{2.707666in}}%
\pgfpathcurveto{\pgfqpoint{3.226894in}{2.699852in}}{\pgfqpoint{3.237493in}{2.695462in}}{\pgfqpoint{3.248544in}{2.695462in}}%
\pgfpathclose%
\pgfusepath{stroke,fill}%
\end{pgfscope}%
\begin{pgfscope}%
\pgfpathrectangle{\pgfqpoint{0.600000in}{0.600000in}}{\pgfqpoint{3.900000in}{3.900000in}}%
\pgfusepath{clip}%
\pgfsetbuttcap%
\pgfsetroundjoin%
\definecolor{currentfill}{rgb}{0.121569,0.466667,0.705882}%
\pgfsetfillcolor{currentfill}%
\pgfsetlinewidth{1.003750pt}%
\definecolor{currentstroke}{rgb}{0.121569,0.466667,0.705882}%
\pgfsetstrokecolor{currentstroke}%
\pgfsetdash{}{0pt}%
\pgfpathmoveto{\pgfqpoint{2.437928in}{3.502200in}}%
\pgfpathcurveto{\pgfqpoint{2.448978in}{3.502200in}}{\pgfqpoint{2.459577in}{3.506591in}}{\pgfqpoint{2.467391in}{3.514404in}}%
\pgfpathcurveto{\pgfqpoint{2.475205in}{3.522218in}}{\pgfqpoint{2.479595in}{3.532817in}}{\pgfqpoint{2.479595in}{3.543867in}}%
\pgfpathcurveto{\pgfqpoint{2.479595in}{3.554917in}}{\pgfqpoint{2.475205in}{3.565516in}}{\pgfqpoint{2.467391in}{3.573330in}}%
\pgfpathcurveto{\pgfqpoint{2.459577in}{3.581143in}}{\pgfqpoint{2.448978in}{3.585534in}}{\pgfqpoint{2.437928in}{3.585534in}}%
\pgfpathcurveto{\pgfqpoint{2.426878in}{3.585534in}}{\pgfqpoint{2.416279in}{3.581143in}}{\pgfqpoint{2.408465in}{3.573330in}}%
\pgfpathcurveto{\pgfqpoint{2.400652in}{3.565516in}}{\pgfqpoint{2.396262in}{3.554917in}}{\pgfqpoint{2.396262in}{3.543867in}}%
\pgfpathcurveto{\pgfqpoint{2.396262in}{3.532817in}}{\pgfqpoint{2.400652in}{3.522218in}}{\pgfqpoint{2.408465in}{3.514404in}}%
\pgfpathcurveto{\pgfqpoint{2.416279in}{3.506591in}}{\pgfqpoint{2.426878in}{3.502200in}}{\pgfqpoint{2.437928in}{3.502200in}}%
\pgfpathclose%
\pgfusepath{stroke,fill}%
\end{pgfscope}%
\begin{pgfscope}%
\pgfpathrectangle{\pgfqpoint{0.600000in}{0.600000in}}{\pgfqpoint{3.900000in}{3.900000in}}%
\pgfusepath{clip}%
\pgfsetbuttcap%
\pgfsetroundjoin%
\definecolor{currentfill}{rgb}{0.121569,0.466667,0.705882}%
\pgfsetfillcolor{currentfill}%
\pgfsetlinewidth{1.003750pt}%
\definecolor{currentstroke}{rgb}{0.121569,0.466667,0.705882}%
\pgfsetstrokecolor{currentstroke}%
\pgfsetdash{}{0pt}%
\pgfpathmoveto{\pgfqpoint{2.297117in}{2.238131in}}%
\pgfpathcurveto{\pgfqpoint{2.308167in}{2.238131in}}{\pgfqpoint{2.318766in}{2.242521in}}{\pgfqpoint{2.326580in}{2.250335in}}%
\pgfpathcurveto{\pgfqpoint{2.334393in}{2.258148in}}{\pgfqpoint{2.338783in}{2.268747in}}{\pgfqpoint{2.338783in}{2.279797in}}%
\pgfpathcurveto{\pgfqpoint{2.338783in}{2.290848in}}{\pgfqpoint{2.334393in}{2.301447in}}{\pgfqpoint{2.326580in}{2.309260in}}%
\pgfpathcurveto{\pgfqpoint{2.318766in}{2.317074in}}{\pgfqpoint{2.308167in}{2.321464in}}{\pgfqpoint{2.297117in}{2.321464in}}%
\pgfpathcurveto{\pgfqpoint{2.286067in}{2.321464in}}{\pgfqpoint{2.275468in}{2.317074in}}{\pgfqpoint{2.267654in}{2.309260in}}%
\pgfpathcurveto{\pgfqpoint{2.259840in}{2.301447in}}{\pgfqpoint{2.255450in}{2.290848in}}{\pgfqpoint{2.255450in}{2.279797in}}%
\pgfpathcurveto{\pgfqpoint{2.255450in}{2.268747in}}{\pgfqpoint{2.259840in}{2.258148in}}{\pgfqpoint{2.267654in}{2.250335in}}%
\pgfpathcurveto{\pgfqpoint{2.275468in}{2.242521in}}{\pgfqpoint{2.286067in}{2.238131in}}{\pgfqpoint{2.297117in}{2.238131in}}%
\pgfpathclose%
\pgfusepath{stroke,fill}%
\end{pgfscope}%
\begin{pgfscope}%
\pgfpathrectangle{\pgfqpoint{0.600000in}{0.600000in}}{\pgfqpoint{3.900000in}{3.900000in}}%
\pgfusepath{clip}%
\pgfsetbuttcap%
\pgfsetroundjoin%
\definecolor{currentfill}{rgb}{0.121569,0.466667,0.705882}%
\pgfsetfillcolor{currentfill}%
\pgfsetlinewidth{1.003750pt}%
\definecolor{currentstroke}{rgb}{0.121569,0.466667,0.705882}%
\pgfsetstrokecolor{currentstroke}%
\pgfsetdash{}{0pt}%
\pgfpathmoveto{\pgfqpoint{2.504630in}{2.638008in}}%
\pgfpathcurveto{\pgfqpoint{2.515680in}{2.638008in}}{\pgfqpoint{2.526279in}{2.642398in}}{\pgfqpoint{2.534093in}{2.650212in}}%
\pgfpathcurveto{\pgfqpoint{2.541906in}{2.658026in}}{\pgfqpoint{2.546297in}{2.668625in}}{\pgfqpoint{2.546297in}{2.679675in}}%
\pgfpathcurveto{\pgfqpoint{2.546297in}{2.690725in}}{\pgfqpoint{2.541906in}{2.701324in}}{\pgfqpoint{2.534093in}{2.709138in}}%
\pgfpathcurveto{\pgfqpoint{2.526279in}{2.716951in}}{\pgfqpoint{2.515680in}{2.721342in}}{\pgfqpoint{2.504630in}{2.721342in}}%
\pgfpathcurveto{\pgfqpoint{2.493580in}{2.721342in}}{\pgfqpoint{2.482981in}{2.716951in}}{\pgfqpoint{2.475167in}{2.709138in}}%
\pgfpathcurveto{\pgfqpoint{2.467353in}{2.701324in}}{\pgfqpoint{2.462963in}{2.690725in}}{\pgfqpoint{2.462963in}{2.679675in}}%
\pgfpathcurveto{\pgfqpoint{2.462963in}{2.668625in}}{\pgfqpoint{2.467353in}{2.658026in}}{\pgfqpoint{2.475167in}{2.650212in}}%
\pgfpathcurveto{\pgfqpoint{2.482981in}{2.642398in}}{\pgfqpoint{2.493580in}{2.638008in}}{\pgfqpoint{2.504630in}{2.638008in}}%
\pgfpathclose%
\pgfusepath{stroke,fill}%
\end{pgfscope}%
\begin{pgfscope}%
\pgfpathrectangle{\pgfqpoint{0.600000in}{0.600000in}}{\pgfqpoint{3.900000in}{3.900000in}}%
\pgfusepath{clip}%
\pgfsetbuttcap%
\pgfsetroundjoin%
\definecolor{currentfill}{rgb}{0.121569,0.466667,0.705882}%
\pgfsetfillcolor{currentfill}%
\pgfsetlinewidth{1.003750pt}%
\definecolor{currentstroke}{rgb}{0.121569,0.466667,0.705882}%
\pgfsetstrokecolor{currentstroke}%
\pgfsetdash{}{0pt}%
\pgfpathmoveto{\pgfqpoint{1.529699in}{2.213903in}}%
\pgfpathcurveto{\pgfqpoint{1.540749in}{2.213903in}}{\pgfqpoint{1.551348in}{2.218293in}}{\pgfqpoint{1.559162in}{2.226107in}}%
\pgfpathcurveto{\pgfqpoint{1.566975in}{2.233920in}}{\pgfqpoint{1.571365in}{2.244519in}}{\pgfqpoint{1.571365in}{2.255570in}}%
\pgfpathcurveto{\pgfqpoint{1.571365in}{2.266620in}}{\pgfqpoint{1.566975in}{2.277219in}}{\pgfqpoint{1.559162in}{2.285032in}}%
\pgfpathcurveto{\pgfqpoint{1.551348in}{2.292846in}}{\pgfqpoint{1.540749in}{2.297236in}}{\pgfqpoint{1.529699in}{2.297236in}}%
\pgfpathcurveto{\pgfqpoint{1.518649in}{2.297236in}}{\pgfqpoint{1.508050in}{2.292846in}}{\pgfqpoint{1.500236in}{2.285032in}}%
\pgfpathcurveto{\pgfqpoint{1.492422in}{2.277219in}}{\pgfqpoint{1.488032in}{2.266620in}}{\pgfqpoint{1.488032in}{2.255570in}}%
\pgfpathcurveto{\pgfqpoint{1.488032in}{2.244519in}}{\pgfqpoint{1.492422in}{2.233920in}}{\pgfqpoint{1.500236in}{2.226107in}}%
\pgfpathcurveto{\pgfqpoint{1.508050in}{2.218293in}}{\pgfqpoint{1.518649in}{2.213903in}}{\pgfqpoint{1.529699in}{2.213903in}}%
\pgfpathclose%
\pgfusepath{stroke,fill}%
\end{pgfscope}%
\begin{pgfscope}%
\pgfpathrectangle{\pgfqpoint{0.600000in}{0.600000in}}{\pgfqpoint{3.900000in}{3.900000in}}%
\pgfusepath{clip}%
\pgfsetbuttcap%
\pgfsetroundjoin%
\definecolor{currentfill}{rgb}{0.121569,0.466667,0.705882}%
\pgfsetfillcolor{currentfill}%
\pgfsetlinewidth{1.003750pt}%
\definecolor{currentstroke}{rgb}{0.121569,0.466667,0.705882}%
\pgfsetstrokecolor{currentstroke}%
\pgfsetdash{}{0pt}%
\pgfpathmoveto{\pgfqpoint{1.957461in}{3.550706in}}%
\pgfpathcurveto{\pgfqpoint{1.968512in}{3.550706in}}{\pgfqpoint{1.979111in}{3.555096in}}{\pgfqpoint{1.986924in}{3.562909in}}%
\pgfpathcurveto{\pgfqpoint{1.994738in}{3.570723in}}{\pgfqpoint{1.999128in}{3.581322in}}{\pgfqpoint{1.999128in}{3.592372in}}%
\pgfpathcurveto{\pgfqpoint{1.999128in}{3.603422in}}{\pgfqpoint{1.994738in}{3.614021in}}{\pgfqpoint{1.986924in}{3.621835in}}%
\pgfpathcurveto{\pgfqpoint{1.979111in}{3.629649in}}{\pgfqpoint{1.968512in}{3.634039in}}{\pgfqpoint{1.957461in}{3.634039in}}%
\pgfpathcurveto{\pgfqpoint{1.946411in}{3.634039in}}{\pgfqpoint{1.935812in}{3.629649in}}{\pgfqpoint{1.927999in}{3.621835in}}%
\pgfpathcurveto{\pgfqpoint{1.920185in}{3.614021in}}{\pgfqpoint{1.915795in}{3.603422in}}{\pgfqpoint{1.915795in}{3.592372in}}%
\pgfpathcurveto{\pgfqpoint{1.915795in}{3.581322in}}{\pgfqpoint{1.920185in}{3.570723in}}{\pgfqpoint{1.927999in}{3.562909in}}%
\pgfpathcurveto{\pgfqpoint{1.935812in}{3.555096in}}{\pgfqpoint{1.946411in}{3.550706in}}{\pgfqpoint{1.957461in}{3.550706in}}%
\pgfpathclose%
\pgfusepath{stroke,fill}%
\end{pgfscope}%
\begin{pgfscope}%
\pgfpathrectangle{\pgfqpoint{0.600000in}{0.600000in}}{\pgfqpoint{3.900000in}{3.900000in}}%
\pgfusepath{clip}%
\pgfsetbuttcap%
\pgfsetroundjoin%
\definecolor{currentfill}{rgb}{0.121569,0.466667,0.705882}%
\pgfsetfillcolor{currentfill}%
\pgfsetlinewidth{1.003750pt}%
\definecolor{currentstroke}{rgb}{0.121569,0.466667,0.705882}%
\pgfsetstrokecolor{currentstroke}%
\pgfsetdash{}{0pt}%
\pgfpathmoveto{\pgfqpoint{2.051124in}{2.335282in}}%
\pgfpathcurveto{\pgfqpoint{2.062174in}{2.335282in}}{\pgfqpoint{2.072773in}{2.339672in}}{\pgfqpoint{2.080586in}{2.347486in}}%
\pgfpathcurveto{\pgfqpoint{2.088400in}{2.355299in}}{\pgfqpoint{2.092790in}{2.365898in}}{\pgfqpoint{2.092790in}{2.376948in}}%
\pgfpathcurveto{\pgfqpoint{2.092790in}{2.387998in}}{\pgfqpoint{2.088400in}{2.398598in}}{\pgfqpoint{2.080586in}{2.406411in}}%
\pgfpathcurveto{\pgfqpoint{2.072773in}{2.414225in}}{\pgfqpoint{2.062174in}{2.418615in}}{\pgfqpoint{2.051124in}{2.418615in}}%
\pgfpathcurveto{\pgfqpoint{2.040073in}{2.418615in}}{\pgfqpoint{2.029474in}{2.414225in}}{\pgfqpoint{2.021661in}{2.406411in}}%
\pgfpathcurveto{\pgfqpoint{2.013847in}{2.398598in}}{\pgfqpoint{2.009457in}{2.387998in}}{\pgfqpoint{2.009457in}{2.376948in}}%
\pgfpathcurveto{\pgfqpoint{2.009457in}{2.365898in}}{\pgfqpoint{2.013847in}{2.355299in}}{\pgfqpoint{2.021661in}{2.347486in}}%
\pgfpathcurveto{\pgfqpoint{2.029474in}{2.339672in}}{\pgfqpoint{2.040073in}{2.335282in}}{\pgfqpoint{2.051124in}{2.335282in}}%
\pgfpathclose%
\pgfusepath{stroke,fill}%
\end{pgfscope}%
\begin{pgfscope}%
\pgfpathrectangle{\pgfqpoint{0.600000in}{0.600000in}}{\pgfqpoint{3.900000in}{3.900000in}}%
\pgfusepath{clip}%
\pgfsetbuttcap%
\pgfsetroundjoin%
\definecolor{currentfill}{rgb}{0.121569,0.466667,0.705882}%
\pgfsetfillcolor{currentfill}%
\pgfsetlinewidth{1.003750pt}%
\definecolor{currentstroke}{rgb}{0.121569,0.466667,0.705882}%
\pgfsetstrokecolor{currentstroke}%
\pgfsetdash{}{0pt}%
\pgfpathmoveto{\pgfqpoint{3.776974in}{3.115779in}}%
\pgfpathcurveto{\pgfqpoint{3.788024in}{3.115779in}}{\pgfqpoint{3.798623in}{3.120170in}}{\pgfqpoint{3.806436in}{3.127983in}}%
\pgfpathcurveto{\pgfqpoint{3.814250in}{3.135797in}}{\pgfqpoint{3.818640in}{3.146396in}}{\pgfqpoint{3.818640in}{3.157446in}}%
\pgfpathcurveto{\pgfqpoint{3.818640in}{3.168496in}}{\pgfqpoint{3.814250in}{3.179095in}}{\pgfqpoint{3.806436in}{3.186909in}}%
\pgfpathcurveto{\pgfqpoint{3.798623in}{3.194723in}}{\pgfqpoint{3.788024in}{3.199113in}}{\pgfqpoint{3.776974in}{3.199113in}}%
\pgfpathcurveto{\pgfqpoint{3.765923in}{3.199113in}}{\pgfqpoint{3.755324in}{3.194723in}}{\pgfqpoint{3.747511in}{3.186909in}}%
\pgfpathcurveto{\pgfqpoint{3.739697in}{3.179095in}}{\pgfqpoint{3.735307in}{3.168496in}}{\pgfqpoint{3.735307in}{3.157446in}}%
\pgfpathcurveto{\pgfqpoint{3.735307in}{3.146396in}}{\pgfqpoint{3.739697in}{3.135797in}}{\pgfqpoint{3.747511in}{3.127983in}}%
\pgfpathcurveto{\pgfqpoint{3.755324in}{3.120170in}}{\pgfqpoint{3.765923in}{3.115779in}}{\pgfqpoint{3.776974in}{3.115779in}}%
\pgfpathclose%
\pgfusepath{stroke,fill}%
\end{pgfscope}%
\begin{pgfscope}%
\pgfpathrectangle{\pgfqpoint{0.600000in}{0.600000in}}{\pgfqpoint{3.900000in}{3.900000in}}%
\pgfusepath{clip}%
\pgfsetbuttcap%
\pgfsetroundjoin%
\definecolor{currentfill}{rgb}{0.121569,0.466667,0.705882}%
\pgfsetfillcolor{currentfill}%
\pgfsetlinewidth{1.003750pt}%
\definecolor{currentstroke}{rgb}{0.121569,0.466667,0.705882}%
\pgfsetstrokecolor{currentstroke}%
\pgfsetdash{}{0pt}%
\pgfpathmoveto{\pgfqpoint{3.101115in}{2.169920in}}%
\pgfpathcurveto{\pgfqpoint{3.112165in}{2.169920in}}{\pgfqpoint{3.122764in}{2.174310in}}{\pgfqpoint{3.130578in}{2.182123in}}%
\pgfpathcurveto{\pgfqpoint{3.138391in}{2.189937in}}{\pgfqpoint{3.142782in}{2.200536in}}{\pgfqpoint{3.142782in}{2.211586in}}%
\pgfpathcurveto{\pgfqpoint{3.142782in}{2.222636in}}{\pgfqpoint{3.138391in}{2.233235in}}{\pgfqpoint{3.130578in}{2.241049in}}%
\pgfpathcurveto{\pgfqpoint{3.122764in}{2.248863in}}{\pgfqpoint{3.112165in}{2.253253in}}{\pgfqpoint{3.101115in}{2.253253in}}%
\pgfpathcurveto{\pgfqpoint{3.090065in}{2.253253in}}{\pgfqpoint{3.079466in}{2.248863in}}{\pgfqpoint{3.071652in}{2.241049in}}%
\pgfpathcurveto{\pgfqpoint{3.063839in}{2.233235in}}{\pgfqpoint{3.059448in}{2.222636in}}{\pgfqpoint{3.059448in}{2.211586in}}%
\pgfpathcurveto{\pgfqpoint{3.059448in}{2.200536in}}{\pgfqpoint{3.063839in}{2.189937in}}{\pgfqpoint{3.071652in}{2.182123in}}%
\pgfpathcurveto{\pgfqpoint{3.079466in}{2.174310in}}{\pgfqpoint{3.090065in}{2.169920in}}{\pgfqpoint{3.101115in}{2.169920in}}%
\pgfpathclose%
\pgfusepath{stroke,fill}%
\end{pgfscope}%
\begin{pgfscope}%
\pgfpathrectangle{\pgfqpoint{0.600000in}{0.600000in}}{\pgfqpoint{3.900000in}{3.900000in}}%
\pgfusepath{clip}%
\pgfsetbuttcap%
\pgfsetroundjoin%
\definecolor{currentfill}{rgb}{0.121569,0.466667,0.705882}%
\pgfsetfillcolor{currentfill}%
\pgfsetlinewidth{1.003750pt}%
\definecolor{currentstroke}{rgb}{0.121569,0.466667,0.705882}%
\pgfsetstrokecolor{currentstroke}%
\pgfsetdash{}{0pt}%
\pgfpathmoveto{\pgfqpoint{1.676345in}{3.135956in}}%
\pgfpathcurveto{\pgfqpoint{1.687395in}{3.135956in}}{\pgfqpoint{1.697994in}{3.140347in}}{\pgfqpoint{1.705807in}{3.148160in}}%
\pgfpathcurveto{\pgfqpoint{1.713621in}{3.155974in}}{\pgfqpoint{1.718011in}{3.166573in}}{\pgfqpoint{1.718011in}{3.177623in}}%
\pgfpathcurveto{\pgfqpoint{1.718011in}{3.188673in}}{\pgfqpoint{1.713621in}{3.199272in}}{\pgfqpoint{1.705807in}{3.207086in}}%
\pgfpathcurveto{\pgfqpoint{1.697994in}{3.214900in}}{\pgfqpoint{1.687395in}{3.219290in}}{\pgfqpoint{1.676345in}{3.219290in}}%
\pgfpathcurveto{\pgfqpoint{1.665294in}{3.219290in}}{\pgfqpoint{1.654695in}{3.214900in}}{\pgfqpoint{1.646882in}{3.207086in}}%
\pgfpathcurveto{\pgfqpoint{1.639068in}{3.199272in}}{\pgfqpoint{1.634678in}{3.188673in}}{\pgfqpoint{1.634678in}{3.177623in}}%
\pgfpathcurveto{\pgfqpoint{1.634678in}{3.166573in}}{\pgfqpoint{1.639068in}{3.155974in}}{\pgfqpoint{1.646882in}{3.148160in}}%
\pgfpathcurveto{\pgfqpoint{1.654695in}{3.140347in}}{\pgfqpoint{1.665294in}{3.135956in}}{\pgfqpoint{1.676345in}{3.135956in}}%
\pgfpathclose%
\pgfusepath{stroke,fill}%
\end{pgfscope}%
\begin{pgfscope}%
\pgfpathrectangle{\pgfqpoint{0.600000in}{0.600000in}}{\pgfqpoint{3.900000in}{3.900000in}}%
\pgfusepath{clip}%
\pgfsetbuttcap%
\pgfsetroundjoin%
\definecolor{currentfill}{rgb}{0.121569,0.466667,0.705882}%
\pgfsetfillcolor{currentfill}%
\pgfsetlinewidth{1.003750pt}%
\definecolor{currentstroke}{rgb}{0.121569,0.466667,0.705882}%
\pgfsetstrokecolor{currentstroke}%
\pgfsetdash{}{0pt}%
\pgfpathmoveto{\pgfqpoint{2.879367in}{2.265170in}}%
\pgfpathcurveto{\pgfqpoint{2.890417in}{2.265170in}}{\pgfqpoint{2.901016in}{2.269561in}}{\pgfqpoint{2.908830in}{2.277374in}}%
\pgfpathcurveto{\pgfqpoint{2.916644in}{2.285188in}}{\pgfqpoint{2.921034in}{2.295787in}}{\pgfqpoint{2.921034in}{2.306837in}}%
\pgfpathcurveto{\pgfqpoint{2.921034in}{2.317887in}}{\pgfqpoint{2.916644in}{2.328486in}}{\pgfqpoint{2.908830in}{2.336300in}}%
\pgfpathcurveto{\pgfqpoint{2.901016in}{2.344113in}}{\pgfqpoint{2.890417in}{2.348504in}}{\pgfqpoint{2.879367in}{2.348504in}}%
\pgfpathcurveto{\pgfqpoint{2.868317in}{2.348504in}}{\pgfqpoint{2.857718in}{2.344113in}}{\pgfqpoint{2.849905in}{2.336300in}}%
\pgfpathcurveto{\pgfqpoint{2.842091in}{2.328486in}}{\pgfqpoint{2.837701in}{2.317887in}}{\pgfqpoint{2.837701in}{2.306837in}}%
\pgfpathcurveto{\pgfqpoint{2.837701in}{2.295787in}}{\pgfqpoint{2.842091in}{2.285188in}}{\pgfqpoint{2.849905in}{2.277374in}}%
\pgfpathcurveto{\pgfqpoint{2.857718in}{2.269561in}}{\pgfqpoint{2.868317in}{2.265170in}}{\pgfqpoint{2.879367in}{2.265170in}}%
\pgfpathclose%
\pgfusepath{stroke,fill}%
\end{pgfscope}%
\begin{pgfscope}%
\pgfsetbuttcap%
\pgfsetroundjoin%
\definecolor{currentfill}{rgb}{0.000000,0.000000,0.000000}%
\pgfsetfillcolor{currentfill}%
\pgfsetlinewidth{0.803000pt}%
\definecolor{currentstroke}{rgb}{0.000000,0.000000,0.000000}%
\pgfsetstrokecolor{currentstroke}%
\pgfsetdash{}{0pt}%
\pgfsys@defobject{currentmarker}{\pgfqpoint{0.000000in}{0.000000in}}{\pgfqpoint{0.000000in}{0.048611in}}{%
\pgfpathmoveto{\pgfqpoint{0.000000in}{0.000000in}}%
\pgfpathlineto{\pgfqpoint{0.000000in}{0.048611in}}%
\pgfusepath{stroke,fill}%
}%
\begin{pgfscope}%
\pgfsys@transformshift{1.010526in}{0.600000in}%
\pgfsys@useobject{currentmarker}{}%
\end{pgfscope}%
\end{pgfscope}%
\begin{pgfscope}%
\pgfsetbuttcap%
\pgfsetroundjoin%
\definecolor{currentfill}{rgb}{0.000000,0.000000,0.000000}%
\pgfsetfillcolor{currentfill}%
\pgfsetlinewidth{0.803000pt}%
\definecolor{currentstroke}{rgb}{0.000000,0.000000,0.000000}%
\pgfsetstrokecolor{currentstroke}%
\pgfsetdash{}{0pt}%
\pgfsys@defobject{currentmarker}{\pgfqpoint{0.000000in}{-0.048611in}}{\pgfqpoint{0.000000in}{0.000000in}}{%
\pgfpathmoveto{\pgfqpoint{0.000000in}{0.000000in}}%
\pgfpathlineto{\pgfqpoint{0.000000in}{-0.048611in}}%
\pgfusepath{stroke,fill}%
}%
\begin{pgfscope}%
\pgfsys@transformshift{1.010526in}{4.500000in}%
\pgfsys@useobject{currentmarker}{}%
\end{pgfscope}%
\end{pgfscope}%
\begin{pgfscope}%
\definecolor{textcolor}{rgb}{0.000000,0.000000,0.000000}%
\pgfsetstrokecolor{textcolor}%
\pgfsetfillcolor{textcolor}%
\pgftext[x=1.010526in,y=0.551389in,,top]{\color{textcolor}\sffamily\fontsize{10.000000}{12.000000}\selectfont −3}%
\end{pgfscope}%
\begin{pgfscope}%
\pgfsetbuttcap%
\pgfsetroundjoin%
\definecolor{currentfill}{rgb}{0.000000,0.000000,0.000000}%
\pgfsetfillcolor{currentfill}%
\pgfsetlinewidth{0.803000pt}%
\definecolor{currentstroke}{rgb}{0.000000,0.000000,0.000000}%
\pgfsetstrokecolor{currentstroke}%
\pgfsetdash{}{0pt}%
\pgfsys@defobject{currentmarker}{\pgfqpoint{0.000000in}{0.000000in}}{\pgfqpoint{0.000000in}{0.048611in}}{%
\pgfpathmoveto{\pgfqpoint{0.000000in}{0.000000in}}%
\pgfpathlineto{\pgfqpoint{0.000000in}{0.048611in}}%
\pgfusepath{stroke,fill}%
}%
\begin{pgfscope}%
\pgfsys@transformshift{1.523684in}{0.600000in}%
\pgfsys@useobject{currentmarker}{}%
\end{pgfscope}%
\end{pgfscope}%
\begin{pgfscope}%
\pgfsetbuttcap%
\pgfsetroundjoin%
\definecolor{currentfill}{rgb}{0.000000,0.000000,0.000000}%
\pgfsetfillcolor{currentfill}%
\pgfsetlinewidth{0.803000pt}%
\definecolor{currentstroke}{rgb}{0.000000,0.000000,0.000000}%
\pgfsetstrokecolor{currentstroke}%
\pgfsetdash{}{0pt}%
\pgfsys@defobject{currentmarker}{\pgfqpoint{0.000000in}{-0.048611in}}{\pgfqpoint{0.000000in}{0.000000in}}{%
\pgfpathmoveto{\pgfqpoint{0.000000in}{0.000000in}}%
\pgfpathlineto{\pgfqpoint{0.000000in}{-0.048611in}}%
\pgfusepath{stroke,fill}%
}%
\begin{pgfscope}%
\pgfsys@transformshift{1.523684in}{4.500000in}%
\pgfsys@useobject{currentmarker}{}%
\end{pgfscope}%
\end{pgfscope}%
\begin{pgfscope}%
\definecolor{textcolor}{rgb}{0.000000,0.000000,0.000000}%
\pgfsetstrokecolor{textcolor}%
\pgfsetfillcolor{textcolor}%
\pgftext[x=1.523684in,y=0.551389in,,top]{\color{textcolor}\sffamily\fontsize{10.000000}{12.000000}\selectfont −2}%
\end{pgfscope}%
\begin{pgfscope}%
\pgfsetbuttcap%
\pgfsetroundjoin%
\definecolor{currentfill}{rgb}{0.000000,0.000000,0.000000}%
\pgfsetfillcolor{currentfill}%
\pgfsetlinewidth{0.803000pt}%
\definecolor{currentstroke}{rgb}{0.000000,0.000000,0.000000}%
\pgfsetstrokecolor{currentstroke}%
\pgfsetdash{}{0pt}%
\pgfsys@defobject{currentmarker}{\pgfqpoint{0.000000in}{0.000000in}}{\pgfqpoint{0.000000in}{0.048611in}}{%
\pgfpathmoveto{\pgfqpoint{0.000000in}{0.000000in}}%
\pgfpathlineto{\pgfqpoint{0.000000in}{0.048611in}}%
\pgfusepath{stroke,fill}%
}%
\begin{pgfscope}%
\pgfsys@transformshift{2.036842in}{0.600000in}%
\pgfsys@useobject{currentmarker}{}%
\end{pgfscope}%
\end{pgfscope}%
\begin{pgfscope}%
\pgfsetbuttcap%
\pgfsetroundjoin%
\definecolor{currentfill}{rgb}{0.000000,0.000000,0.000000}%
\pgfsetfillcolor{currentfill}%
\pgfsetlinewidth{0.803000pt}%
\definecolor{currentstroke}{rgb}{0.000000,0.000000,0.000000}%
\pgfsetstrokecolor{currentstroke}%
\pgfsetdash{}{0pt}%
\pgfsys@defobject{currentmarker}{\pgfqpoint{0.000000in}{-0.048611in}}{\pgfqpoint{0.000000in}{0.000000in}}{%
\pgfpathmoveto{\pgfqpoint{0.000000in}{0.000000in}}%
\pgfpathlineto{\pgfqpoint{0.000000in}{-0.048611in}}%
\pgfusepath{stroke,fill}%
}%
\begin{pgfscope}%
\pgfsys@transformshift{2.036842in}{4.500000in}%
\pgfsys@useobject{currentmarker}{}%
\end{pgfscope}%
\end{pgfscope}%
\begin{pgfscope}%
\definecolor{textcolor}{rgb}{0.000000,0.000000,0.000000}%
\pgfsetstrokecolor{textcolor}%
\pgfsetfillcolor{textcolor}%
\pgftext[x=2.036842in,y=0.551389in,,top]{\color{textcolor}\sffamily\fontsize{10.000000}{12.000000}\selectfont −1}%
\end{pgfscope}%
\begin{pgfscope}%
\pgfsetbuttcap%
\pgfsetroundjoin%
\definecolor{currentfill}{rgb}{0.000000,0.000000,0.000000}%
\pgfsetfillcolor{currentfill}%
\pgfsetlinewidth{0.803000pt}%
\definecolor{currentstroke}{rgb}{0.000000,0.000000,0.000000}%
\pgfsetstrokecolor{currentstroke}%
\pgfsetdash{}{0pt}%
\pgfsys@defobject{currentmarker}{\pgfqpoint{0.000000in}{0.000000in}}{\pgfqpoint{0.000000in}{0.048611in}}{%
\pgfpathmoveto{\pgfqpoint{0.000000in}{0.000000in}}%
\pgfpathlineto{\pgfqpoint{0.000000in}{0.048611in}}%
\pgfusepath{stroke,fill}%
}%
\begin{pgfscope}%
\pgfsys@transformshift{2.550000in}{0.600000in}%
\pgfsys@useobject{currentmarker}{}%
\end{pgfscope}%
\end{pgfscope}%
\begin{pgfscope}%
\pgfsetbuttcap%
\pgfsetroundjoin%
\definecolor{currentfill}{rgb}{0.000000,0.000000,0.000000}%
\pgfsetfillcolor{currentfill}%
\pgfsetlinewidth{0.803000pt}%
\definecolor{currentstroke}{rgb}{0.000000,0.000000,0.000000}%
\pgfsetstrokecolor{currentstroke}%
\pgfsetdash{}{0pt}%
\pgfsys@defobject{currentmarker}{\pgfqpoint{0.000000in}{-0.048611in}}{\pgfqpoint{0.000000in}{0.000000in}}{%
\pgfpathmoveto{\pgfqpoint{0.000000in}{0.000000in}}%
\pgfpathlineto{\pgfqpoint{0.000000in}{-0.048611in}}%
\pgfusepath{stroke,fill}%
}%
\begin{pgfscope}%
\pgfsys@transformshift{2.550000in}{4.500000in}%
\pgfsys@useobject{currentmarker}{}%
\end{pgfscope}%
\end{pgfscope}%
\begin{pgfscope}%
\definecolor{textcolor}{rgb}{0.000000,0.000000,0.000000}%
\pgfsetstrokecolor{textcolor}%
\pgfsetfillcolor{textcolor}%
\pgftext[x=2.550000in,y=0.551389in,,top]{\color{textcolor}\sffamily\fontsize{10.000000}{12.000000}\selectfont 0}%
\end{pgfscope}%
\begin{pgfscope}%
\pgfsetbuttcap%
\pgfsetroundjoin%
\definecolor{currentfill}{rgb}{0.000000,0.000000,0.000000}%
\pgfsetfillcolor{currentfill}%
\pgfsetlinewidth{0.803000pt}%
\definecolor{currentstroke}{rgb}{0.000000,0.000000,0.000000}%
\pgfsetstrokecolor{currentstroke}%
\pgfsetdash{}{0pt}%
\pgfsys@defobject{currentmarker}{\pgfqpoint{0.000000in}{0.000000in}}{\pgfqpoint{0.000000in}{0.048611in}}{%
\pgfpathmoveto{\pgfqpoint{0.000000in}{0.000000in}}%
\pgfpathlineto{\pgfqpoint{0.000000in}{0.048611in}}%
\pgfusepath{stroke,fill}%
}%
\begin{pgfscope}%
\pgfsys@transformshift{3.063158in}{0.600000in}%
\pgfsys@useobject{currentmarker}{}%
\end{pgfscope}%
\end{pgfscope}%
\begin{pgfscope}%
\pgfsetbuttcap%
\pgfsetroundjoin%
\definecolor{currentfill}{rgb}{0.000000,0.000000,0.000000}%
\pgfsetfillcolor{currentfill}%
\pgfsetlinewidth{0.803000pt}%
\definecolor{currentstroke}{rgb}{0.000000,0.000000,0.000000}%
\pgfsetstrokecolor{currentstroke}%
\pgfsetdash{}{0pt}%
\pgfsys@defobject{currentmarker}{\pgfqpoint{0.000000in}{-0.048611in}}{\pgfqpoint{0.000000in}{0.000000in}}{%
\pgfpathmoveto{\pgfqpoint{0.000000in}{0.000000in}}%
\pgfpathlineto{\pgfqpoint{0.000000in}{-0.048611in}}%
\pgfusepath{stroke,fill}%
}%
\begin{pgfscope}%
\pgfsys@transformshift{3.063158in}{4.500000in}%
\pgfsys@useobject{currentmarker}{}%
\end{pgfscope}%
\end{pgfscope}%
\begin{pgfscope}%
\definecolor{textcolor}{rgb}{0.000000,0.000000,0.000000}%
\pgfsetstrokecolor{textcolor}%
\pgfsetfillcolor{textcolor}%
\pgftext[x=3.063158in,y=0.551389in,,top]{\color{textcolor}\sffamily\fontsize{10.000000}{12.000000}\selectfont 1}%
\end{pgfscope}%
\begin{pgfscope}%
\pgfsetbuttcap%
\pgfsetroundjoin%
\definecolor{currentfill}{rgb}{0.000000,0.000000,0.000000}%
\pgfsetfillcolor{currentfill}%
\pgfsetlinewidth{0.803000pt}%
\definecolor{currentstroke}{rgb}{0.000000,0.000000,0.000000}%
\pgfsetstrokecolor{currentstroke}%
\pgfsetdash{}{0pt}%
\pgfsys@defobject{currentmarker}{\pgfqpoint{0.000000in}{0.000000in}}{\pgfqpoint{0.000000in}{0.048611in}}{%
\pgfpathmoveto{\pgfqpoint{0.000000in}{0.000000in}}%
\pgfpathlineto{\pgfqpoint{0.000000in}{0.048611in}}%
\pgfusepath{stroke,fill}%
}%
\begin{pgfscope}%
\pgfsys@transformshift{3.576316in}{0.600000in}%
\pgfsys@useobject{currentmarker}{}%
\end{pgfscope}%
\end{pgfscope}%
\begin{pgfscope}%
\pgfsetbuttcap%
\pgfsetroundjoin%
\definecolor{currentfill}{rgb}{0.000000,0.000000,0.000000}%
\pgfsetfillcolor{currentfill}%
\pgfsetlinewidth{0.803000pt}%
\definecolor{currentstroke}{rgb}{0.000000,0.000000,0.000000}%
\pgfsetstrokecolor{currentstroke}%
\pgfsetdash{}{0pt}%
\pgfsys@defobject{currentmarker}{\pgfqpoint{0.000000in}{-0.048611in}}{\pgfqpoint{0.000000in}{0.000000in}}{%
\pgfpathmoveto{\pgfqpoint{0.000000in}{0.000000in}}%
\pgfpathlineto{\pgfqpoint{0.000000in}{-0.048611in}}%
\pgfusepath{stroke,fill}%
}%
\begin{pgfscope}%
\pgfsys@transformshift{3.576316in}{4.500000in}%
\pgfsys@useobject{currentmarker}{}%
\end{pgfscope}%
\end{pgfscope}%
\begin{pgfscope}%
\definecolor{textcolor}{rgb}{0.000000,0.000000,0.000000}%
\pgfsetstrokecolor{textcolor}%
\pgfsetfillcolor{textcolor}%
\pgftext[x=3.576316in,y=0.551389in,,top]{\color{textcolor}\sffamily\fontsize{10.000000}{12.000000}\selectfont 2}%
\end{pgfscope}%
\begin{pgfscope}%
\pgfsetbuttcap%
\pgfsetroundjoin%
\definecolor{currentfill}{rgb}{0.000000,0.000000,0.000000}%
\pgfsetfillcolor{currentfill}%
\pgfsetlinewidth{0.803000pt}%
\definecolor{currentstroke}{rgb}{0.000000,0.000000,0.000000}%
\pgfsetstrokecolor{currentstroke}%
\pgfsetdash{}{0pt}%
\pgfsys@defobject{currentmarker}{\pgfqpoint{0.000000in}{0.000000in}}{\pgfqpoint{0.000000in}{0.048611in}}{%
\pgfpathmoveto{\pgfqpoint{0.000000in}{0.000000in}}%
\pgfpathlineto{\pgfqpoint{0.000000in}{0.048611in}}%
\pgfusepath{stroke,fill}%
}%
\begin{pgfscope}%
\pgfsys@transformshift{4.089474in}{0.600000in}%
\pgfsys@useobject{currentmarker}{}%
\end{pgfscope}%
\end{pgfscope}%
\begin{pgfscope}%
\pgfsetbuttcap%
\pgfsetroundjoin%
\definecolor{currentfill}{rgb}{0.000000,0.000000,0.000000}%
\pgfsetfillcolor{currentfill}%
\pgfsetlinewidth{0.803000pt}%
\definecolor{currentstroke}{rgb}{0.000000,0.000000,0.000000}%
\pgfsetstrokecolor{currentstroke}%
\pgfsetdash{}{0pt}%
\pgfsys@defobject{currentmarker}{\pgfqpoint{0.000000in}{-0.048611in}}{\pgfqpoint{0.000000in}{0.000000in}}{%
\pgfpathmoveto{\pgfqpoint{0.000000in}{0.000000in}}%
\pgfpathlineto{\pgfqpoint{0.000000in}{-0.048611in}}%
\pgfusepath{stroke,fill}%
}%
\begin{pgfscope}%
\pgfsys@transformshift{4.089474in}{4.500000in}%
\pgfsys@useobject{currentmarker}{}%
\end{pgfscope}%
\end{pgfscope}%
\begin{pgfscope}%
\definecolor{textcolor}{rgb}{0.000000,0.000000,0.000000}%
\pgfsetstrokecolor{textcolor}%
\pgfsetfillcolor{textcolor}%
\pgftext[x=4.089474in,y=0.551389in,,top]{\color{textcolor}\sffamily\fontsize{10.000000}{12.000000}\selectfont 3}%
\end{pgfscope}%
\begin{pgfscope}%
\pgfsetbuttcap%
\pgfsetroundjoin%
\definecolor{currentfill}{rgb}{0.000000,0.000000,0.000000}%
\pgfsetfillcolor{currentfill}%
\pgfsetlinewidth{0.803000pt}%
\definecolor{currentstroke}{rgb}{0.000000,0.000000,0.000000}%
\pgfsetstrokecolor{currentstroke}%
\pgfsetdash{}{0pt}%
\pgfsys@defobject{currentmarker}{\pgfqpoint{0.000000in}{0.000000in}}{\pgfqpoint{0.048611in}{0.000000in}}{%
\pgfpathmoveto{\pgfqpoint{0.000000in}{0.000000in}}%
\pgfpathlineto{\pgfqpoint{0.048611in}{0.000000in}}%
\pgfusepath{stroke,fill}%
}%
\begin{pgfscope}%
\pgfsys@transformshift{0.600000in}{1.010526in}%
\pgfsys@useobject{currentmarker}{}%
\end{pgfscope}%
\end{pgfscope}%
\begin{pgfscope}%
\pgfsetbuttcap%
\pgfsetroundjoin%
\definecolor{currentfill}{rgb}{0.000000,0.000000,0.000000}%
\pgfsetfillcolor{currentfill}%
\pgfsetlinewidth{0.803000pt}%
\definecolor{currentstroke}{rgb}{0.000000,0.000000,0.000000}%
\pgfsetstrokecolor{currentstroke}%
\pgfsetdash{}{0pt}%
\pgfsys@defobject{currentmarker}{\pgfqpoint{-0.048611in}{0.000000in}}{\pgfqpoint{0.000000in}{0.000000in}}{%
\pgfpathmoveto{\pgfqpoint{0.000000in}{0.000000in}}%
\pgfpathlineto{\pgfqpoint{-0.048611in}{0.000000in}}%
\pgfusepath{stroke,fill}%
}%
\begin{pgfscope}%
\pgfsys@transformshift{4.500000in}{1.010526in}%
\pgfsys@useobject{currentmarker}{}%
\end{pgfscope}%
\end{pgfscope}%
\begin{pgfscope}%
\definecolor{textcolor}{rgb}{0.000000,0.000000,0.000000}%
\pgfsetstrokecolor{textcolor}%
\pgfsetfillcolor{textcolor}%
\pgftext[x=0.346650in,y=0.957765in,left,base]{\color{textcolor}\sffamily\fontsize{10.000000}{12.000000}\selectfont −3}%
\end{pgfscope}%
\begin{pgfscope}%
\pgfsetbuttcap%
\pgfsetroundjoin%
\definecolor{currentfill}{rgb}{0.000000,0.000000,0.000000}%
\pgfsetfillcolor{currentfill}%
\pgfsetlinewidth{0.803000pt}%
\definecolor{currentstroke}{rgb}{0.000000,0.000000,0.000000}%
\pgfsetstrokecolor{currentstroke}%
\pgfsetdash{}{0pt}%
\pgfsys@defobject{currentmarker}{\pgfqpoint{0.000000in}{0.000000in}}{\pgfqpoint{0.048611in}{0.000000in}}{%
\pgfpathmoveto{\pgfqpoint{0.000000in}{0.000000in}}%
\pgfpathlineto{\pgfqpoint{0.048611in}{0.000000in}}%
\pgfusepath{stroke,fill}%
}%
\begin{pgfscope}%
\pgfsys@transformshift{0.600000in}{1.523684in}%
\pgfsys@useobject{currentmarker}{}%
\end{pgfscope}%
\end{pgfscope}%
\begin{pgfscope}%
\pgfsetbuttcap%
\pgfsetroundjoin%
\definecolor{currentfill}{rgb}{0.000000,0.000000,0.000000}%
\pgfsetfillcolor{currentfill}%
\pgfsetlinewidth{0.803000pt}%
\definecolor{currentstroke}{rgb}{0.000000,0.000000,0.000000}%
\pgfsetstrokecolor{currentstroke}%
\pgfsetdash{}{0pt}%
\pgfsys@defobject{currentmarker}{\pgfqpoint{-0.048611in}{0.000000in}}{\pgfqpoint{0.000000in}{0.000000in}}{%
\pgfpathmoveto{\pgfqpoint{0.000000in}{0.000000in}}%
\pgfpathlineto{\pgfqpoint{-0.048611in}{0.000000in}}%
\pgfusepath{stroke,fill}%
}%
\begin{pgfscope}%
\pgfsys@transformshift{4.500000in}{1.523684in}%
\pgfsys@useobject{currentmarker}{}%
\end{pgfscope}%
\end{pgfscope}%
\begin{pgfscope}%
\definecolor{textcolor}{rgb}{0.000000,0.000000,0.000000}%
\pgfsetstrokecolor{textcolor}%
\pgfsetfillcolor{textcolor}%
\pgftext[x=0.346650in,y=1.470923in,left,base]{\color{textcolor}\sffamily\fontsize{10.000000}{12.000000}\selectfont −2}%
\end{pgfscope}%
\begin{pgfscope}%
\pgfsetbuttcap%
\pgfsetroundjoin%
\definecolor{currentfill}{rgb}{0.000000,0.000000,0.000000}%
\pgfsetfillcolor{currentfill}%
\pgfsetlinewidth{0.803000pt}%
\definecolor{currentstroke}{rgb}{0.000000,0.000000,0.000000}%
\pgfsetstrokecolor{currentstroke}%
\pgfsetdash{}{0pt}%
\pgfsys@defobject{currentmarker}{\pgfqpoint{0.000000in}{0.000000in}}{\pgfqpoint{0.048611in}{0.000000in}}{%
\pgfpathmoveto{\pgfqpoint{0.000000in}{0.000000in}}%
\pgfpathlineto{\pgfqpoint{0.048611in}{0.000000in}}%
\pgfusepath{stroke,fill}%
}%
\begin{pgfscope}%
\pgfsys@transformshift{0.600000in}{2.036842in}%
\pgfsys@useobject{currentmarker}{}%
\end{pgfscope}%
\end{pgfscope}%
\begin{pgfscope}%
\pgfsetbuttcap%
\pgfsetroundjoin%
\definecolor{currentfill}{rgb}{0.000000,0.000000,0.000000}%
\pgfsetfillcolor{currentfill}%
\pgfsetlinewidth{0.803000pt}%
\definecolor{currentstroke}{rgb}{0.000000,0.000000,0.000000}%
\pgfsetstrokecolor{currentstroke}%
\pgfsetdash{}{0pt}%
\pgfsys@defobject{currentmarker}{\pgfqpoint{-0.048611in}{0.000000in}}{\pgfqpoint{0.000000in}{0.000000in}}{%
\pgfpathmoveto{\pgfqpoint{0.000000in}{0.000000in}}%
\pgfpathlineto{\pgfqpoint{-0.048611in}{0.000000in}}%
\pgfusepath{stroke,fill}%
}%
\begin{pgfscope}%
\pgfsys@transformshift{4.500000in}{2.036842in}%
\pgfsys@useobject{currentmarker}{}%
\end{pgfscope}%
\end{pgfscope}%
\begin{pgfscope}%
\definecolor{textcolor}{rgb}{0.000000,0.000000,0.000000}%
\pgfsetstrokecolor{textcolor}%
\pgfsetfillcolor{textcolor}%
\pgftext[x=0.346650in,y=1.984081in,left,base]{\color{textcolor}\sffamily\fontsize{10.000000}{12.000000}\selectfont −1}%
\end{pgfscope}%
\begin{pgfscope}%
\pgfsetbuttcap%
\pgfsetroundjoin%
\definecolor{currentfill}{rgb}{0.000000,0.000000,0.000000}%
\pgfsetfillcolor{currentfill}%
\pgfsetlinewidth{0.803000pt}%
\definecolor{currentstroke}{rgb}{0.000000,0.000000,0.000000}%
\pgfsetstrokecolor{currentstroke}%
\pgfsetdash{}{0pt}%
\pgfsys@defobject{currentmarker}{\pgfqpoint{0.000000in}{0.000000in}}{\pgfqpoint{0.048611in}{0.000000in}}{%
\pgfpathmoveto{\pgfqpoint{0.000000in}{0.000000in}}%
\pgfpathlineto{\pgfqpoint{0.048611in}{0.000000in}}%
\pgfusepath{stroke,fill}%
}%
\begin{pgfscope}%
\pgfsys@transformshift{0.600000in}{2.550000in}%
\pgfsys@useobject{currentmarker}{}%
\end{pgfscope}%
\end{pgfscope}%
\begin{pgfscope}%
\pgfsetbuttcap%
\pgfsetroundjoin%
\definecolor{currentfill}{rgb}{0.000000,0.000000,0.000000}%
\pgfsetfillcolor{currentfill}%
\pgfsetlinewidth{0.803000pt}%
\definecolor{currentstroke}{rgb}{0.000000,0.000000,0.000000}%
\pgfsetstrokecolor{currentstroke}%
\pgfsetdash{}{0pt}%
\pgfsys@defobject{currentmarker}{\pgfqpoint{-0.048611in}{0.000000in}}{\pgfqpoint{0.000000in}{0.000000in}}{%
\pgfpathmoveto{\pgfqpoint{0.000000in}{0.000000in}}%
\pgfpathlineto{\pgfqpoint{-0.048611in}{0.000000in}}%
\pgfusepath{stroke,fill}%
}%
\begin{pgfscope}%
\pgfsys@transformshift{4.500000in}{2.550000in}%
\pgfsys@useobject{currentmarker}{}%
\end{pgfscope}%
\end{pgfscope}%
\begin{pgfscope}%
\definecolor{textcolor}{rgb}{0.000000,0.000000,0.000000}%
\pgfsetstrokecolor{textcolor}%
\pgfsetfillcolor{textcolor}%
\pgftext[x=0.463024in,y=2.497238in,left,base]{\color{textcolor}\sffamily\fontsize{10.000000}{12.000000}\selectfont 0}%
\end{pgfscope}%
\begin{pgfscope}%
\pgfsetbuttcap%
\pgfsetroundjoin%
\definecolor{currentfill}{rgb}{0.000000,0.000000,0.000000}%
\pgfsetfillcolor{currentfill}%
\pgfsetlinewidth{0.803000pt}%
\definecolor{currentstroke}{rgb}{0.000000,0.000000,0.000000}%
\pgfsetstrokecolor{currentstroke}%
\pgfsetdash{}{0pt}%
\pgfsys@defobject{currentmarker}{\pgfqpoint{0.000000in}{0.000000in}}{\pgfqpoint{0.048611in}{0.000000in}}{%
\pgfpathmoveto{\pgfqpoint{0.000000in}{0.000000in}}%
\pgfpathlineto{\pgfqpoint{0.048611in}{0.000000in}}%
\pgfusepath{stroke,fill}%
}%
\begin{pgfscope}%
\pgfsys@transformshift{0.600000in}{3.063158in}%
\pgfsys@useobject{currentmarker}{}%
\end{pgfscope}%
\end{pgfscope}%
\begin{pgfscope}%
\pgfsetbuttcap%
\pgfsetroundjoin%
\definecolor{currentfill}{rgb}{0.000000,0.000000,0.000000}%
\pgfsetfillcolor{currentfill}%
\pgfsetlinewidth{0.803000pt}%
\definecolor{currentstroke}{rgb}{0.000000,0.000000,0.000000}%
\pgfsetstrokecolor{currentstroke}%
\pgfsetdash{}{0pt}%
\pgfsys@defobject{currentmarker}{\pgfqpoint{-0.048611in}{0.000000in}}{\pgfqpoint{0.000000in}{0.000000in}}{%
\pgfpathmoveto{\pgfqpoint{0.000000in}{0.000000in}}%
\pgfpathlineto{\pgfqpoint{-0.048611in}{0.000000in}}%
\pgfusepath{stroke,fill}%
}%
\begin{pgfscope}%
\pgfsys@transformshift{4.500000in}{3.063158in}%
\pgfsys@useobject{currentmarker}{}%
\end{pgfscope}%
\end{pgfscope}%
\begin{pgfscope}%
\definecolor{textcolor}{rgb}{0.000000,0.000000,0.000000}%
\pgfsetstrokecolor{textcolor}%
\pgfsetfillcolor{textcolor}%
\pgftext[x=0.463024in,y=3.010396in,left,base]{\color{textcolor}\sffamily\fontsize{10.000000}{12.000000}\selectfont 1}%
\end{pgfscope}%
\begin{pgfscope}%
\pgfsetbuttcap%
\pgfsetroundjoin%
\definecolor{currentfill}{rgb}{0.000000,0.000000,0.000000}%
\pgfsetfillcolor{currentfill}%
\pgfsetlinewidth{0.803000pt}%
\definecolor{currentstroke}{rgb}{0.000000,0.000000,0.000000}%
\pgfsetstrokecolor{currentstroke}%
\pgfsetdash{}{0pt}%
\pgfsys@defobject{currentmarker}{\pgfqpoint{0.000000in}{0.000000in}}{\pgfqpoint{0.048611in}{0.000000in}}{%
\pgfpathmoveto{\pgfqpoint{0.000000in}{0.000000in}}%
\pgfpathlineto{\pgfqpoint{0.048611in}{0.000000in}}%
\pgfusepath{stroke,fill}%
}%
\begin{pgfscope}%
\pgfsys@transformshift{0.600000in}{3.576316in}%
\pgfsys@useobject{currentmarker}{}%
\end{pgfscope}%
\end{pgfscope}%
\begin{pgfscope}%
\pgfsetbuttcap%
\pgfsetroundjoin%
\definecolor{currentfill}{rgb}{0.000000,0.000000,0.000000}%
\pgfsetfillcolor{currentfill}%
\pgfsetlinewidth{0.803000pt}%
\definecolor{currentstroke}{rgb}{0.000000,0.000000,0.000000}%
\pgfsetstrokecolor{currentstroke}%
\pgfsetdash{}{0pt}%
\pgfsys@defobject{currentmarker}{\pgfqpoint{-0.048611in}{0.000000in}}{\pgfqpoint{0.000000in}{0.000000in}}{%
\pgfpathmoveto{\pgfqpoint{0.000000in}{0.000000in}}%
\pgfpathlineto{\pgfqpoint{-0.048611in}{0.000000in}}%
\pgfusepath{stroke,fill}%
}%
\begin{pgfscope}%
\pgfsys@transformshift{4.500000in}{3.576316in}%
\pgfsys@useobject{currentmarker}{}%
\end{pgfscope}%
\end{pgfscope}%
\begin{pgfscope}%
\definecolor{textcolor}{rgb}{0.000000,0.000000,0.000000}%
\pgfsetstrokecolor{textcolor}%
\pgfsetfillcolor{textcolor}%
\pgftext[x=0.463024in,y=3.523554in,left,base]{\color{textcolor}\sffamily\fontsize{10.000000}{12.000000}\selectfont 2}%
\end{pgfscope}%
\begin{pgfscope}%
\pgfsetbuttcap%
\pgfsetroundjoin%
\definecolor{currentfill}{rgb}{0.000000,0.000000,0.000000}%
\pgfsetfillcolor{currentfill}%
\pgfsetlinewidth{0.803000pt}%
\definecolor{currentstroke}{rgb}{0.000000,0.000000,0.000000}%
\pgfsetstrokecolor{currentstroke}%
\pgfsetdash{}{0pt}%
\pgfsys@defobject{currentmarker}{\pgfqpoint{0.000000in}{0.000000in}}{\pgfqpoint{0.048611in}{0.000000in}}{%
\pgfpathmoveto{\pgfqpoint{0.000000in}{0.000000in}}%
\pgfpathlineto{\pgfqpoint{0.048611in}{0.000000in}}%
\pgfusepath{stroke,fill}%
}%
\begin{pgfscope}%
\pgfsys@transformshift{0.600000in}{4.089474in}%
\pgfsys@useobject{currentmarker}{}%
\end{pgfscope}%
\end{pgfscope}%
\begin{pgfscope}%
\pgfsetbuttcap%
\pgfsetroundjoin%
\definecolor{currentfill}{rgb}{0.000000,0.000000,0.000000}%
\pgfsetfillcolor{currentfill}%
\pgfsetlinewidth{0.803000pt}%
\definecolor{currentstroke}{rgb}{0.000000,0.000000,0.000000}%
\pgfsetstrokecolor{currentstroke}%
\pgfsetdash{}{0pt}%
\pgfsys@defobject{currentmarker}{\pgfqpoint{-0.048611in}{0.000000in}}{\pgfqpoint{0.000000in}{0.000000in}}{%
\pgfpathmoveto{\pgfqpoint{0.000000in}{0.000000in}}%
\pgfpathlineto{\pgfqpoint{-0.048611in}{0.000000in}}%
\pgfusepath{stroke,fill}%
}%
\begin{pgfscope}%
\pgfsys@transformshift{4.500000in}{4.089474in}%
\pgfsys@useobject{currentmarker}{}%
\end{pgfscope}%
\end{pgfscope}%
\begin{pgfscope}%
\definecolor{textcolor}{rgb}{0.000000,0.000000,0.000000}%
\pgfsetstrokecolor{textcolor}%
\pgfsetfillcolor{textcolor}%
\pgftext[x=0.463024in,y=4.036712in,left,base]{\color{textcolor}\sffamily\fontsize{10.000000}{12.000000}\selectfont 3}%
\end{pgfscope}%
\begin{pgfscope}%
\pgfsetrectcap%
\pgfsetmiterjoin%
\pgfsetlinewidth{0.803000pt}%
\definecolor{currentstroke}{rgb}{0.000000,0.000000,0.000000}%
\pgfsetstrokecolor{currentstroke}%
\pgfsetdash{}{0pt}%
\pgfpathmoveto{\pgfqpoint{0.600000in}{0.600000in}}%
\pgfpathlineto{\pgfqpoint{0.600000in}{4.500000in}}%
\pgfusepath{stroke}%
\end{pgfscope}%
\begin{pgfscope}%
\pgfsetrectcap%
\pgfsetmiterjoin%
\pgfsetlinewidth{0.803000pt}%
\definecolor{currentstroke}{rgb}{0.000000,0.000000,0.000000}%
\pgfsetstrokecolor{currentstroke}%
\pgfsetdash{}{0pt}%
\pgfpathmoveto{\pgfqpoint{4.500000in}{0.600000in}}%
\pgfpathlineto{\pgfqpoint{4.500000in}{4.500000in}}%
\pgfusepath{stroke}%
\end{pgfscope}%
\begin{pgfscope}%
\pgfsetrectcap%
\pgfsetmiterjoin%
\pgfsetlinewidth{0.803000pt}%
\definecolor{currentstroke}{rgb}{0.000000,0.000000,0.000000}%
\pgfsetstrokecolor{currentstroke}%
\pgfsetdash{}{0pt}%
\pgfpathmoveto{\pgfqpoint{0.600000in}{0.600000in}}%
\pgfpathlineto{\pgfqpoint{4.500000in}{0.600000in}}%
\pgfusepath{stroke}%
\end{pgfscope}%
\begin{pgfscope}%
\pgfsetrectcap%
\pgfsetmiterjoin%
\pgfsetlinewidth{0.803000pt}%
\definecolor{currentstroke}{rgb}{0.000000,0.000000,0.000000}%
\pgfsetstrokecolor{currentstroke}%
\pgfsetdash{}{0pt}%
\pgfpathmoveto{\pgfqpoint{0.600000in}{4.500000in}}%
\pgfpathlineto{\pgfqpoint{4.500000in}{4.500000in}}%
\pgfusepath{stroke}%
\end{pgfscope}%
\begin{pgfscope}%
\pgfsetbuttcap%
\pgfsetmiterjoin%
\definecolor{currentfill}{rgb}{1.000000,1.000000,1.000000}%
\pgfsetfillcolor{currentfill}%
\pgfsetlinewidth{0.000000pt}%
\definecolor{currentstroke}{rgb}{0.000000,0.000000,0.000000}%
\pgfsetstrokecolor{currentstroke}%
\pgfsetstrokeopacity{0.000000}%
\pgfsetdash{}{0pt}%
\pgfpathmoveto{\pgfqpoint{0.600000in}{4.530000in}}%
\pgfpathlineto{\pgfqpoint{4.500000in}{4.530000in}}%
\pgfpathlineto{\pgfqpoint{4.500000in}{5.730000in}}%
\pgfpathlineto{\pgfqpoint{0.600000in}{5.730000in}}%
\pgfpathclose%
\pgfusepath{fill}%
\end{pgfscope}%
\begin{pgfscope}%
\pgfpathrectangle{\pgfqpoint{0.600000in}{4.530000in}}{\pgfqpoint{3.900000in}{1.200000in}}%
\pgfusepath{clip}%
\pgfsetbuttcap%
\pgfsetmiterjoin%
\definecolor{currentfill}{rgb}{0.121569,0.466667,0.705882}%
\pgfsetfillcolor{currentfill}%
\pgfsetlinewidth{0.000000pt}%
\definecolor{currentstroke}{rgb}{0.000000,0.000000,0.000000}%
\pgfsetstrokecolor{currentstroke}%
\pgfsetstrokeopacity{0.000000}%
\pgfsetdash{}{0pt}%
\pgfpathmoveto{\pgfqpoint{0.600000in}{4.530000in}}%
\pgfpathlineto{\pgfqpoint{0.728289in}{4.530000in}}%
\pgfpathlineto{\pgfqpoint{0.728289in}{4.539938in}}%
\pgfpathlineto{\pgfqpoint{0.600000in}{4.539938in}}%
\pgfpathclose%
\pgfusepath{fill}%
\end{pgfscope}%
\begin{pgfscope}%
\pgfpathrectangle{\pgfqpoint{0.600000in}{4.530000in}}{\pgfqpoint{3.900000in}{1.200000in}}%
\pgfusepath{clip}%
\pgfsetbuttcap%
\pgfsetmiterjoin%
\definecolor{currentfill}{rgb}{0.121569,0.466667,0.705882}%
\pgfsetfillcolor{currentfill}%
\pgfsetlinewidth{0.000000pt}%
\definecolor{currentstroke}{rgb}{0.000000,0.000000,0.000000}%
\pgfsetstrokecolor{currentstroke}%
\pgfsetstrokeopacity{0.000000}%
\pgfsetdash{}{0pt}%
\pgfpathmoveto{\pgfqpoint{0.728289in}{4.530000in}}%
\pgfpathlineto{\pgfqpoint{0.856579in}{4.530000in}}%
\pgfpathlineto{\pgfqpoint{0.856579in}{4.530000in}}%
\pgfpathlineto{\pgfqpoint{0.728289in}{4.530000in}}%
\pgfpathclose%
\pgfusepath{fill}%
\end{pgfscope}%
\begin{pgfscope}%
\pgfpathrectangle{\pgfqpoint{0.600000in}{4.530000in}}{\pgfqpoint{3.900000in}{1.200000in}}%
\pgfusepath{clip}%
\pgfsetbuttcap%
\pgfsetmiterjoin%
\definecolor{currentfill}{rgb}{0.121569,0.466667,0.705882}%
\pgfsetfillcolor{currentfill}%
\pgfsetlinewidth{0.000000pt}%
\definecolor{currentstroke}{rgb}{0.000000,0.000000,0.000000}%
\pgfsetstrokecolor{currentstroke}%
\pgfsetstrokeopacity{0.000000}%
\pgfsetdash{}{0pt}%
\pgfpathmoveto{\pgfqpoint{0.856579in}{4.530000in}}%
\pgfpathlineto{\pgfqpoint{0.984868in}{4.530000in}}%
\pgfpathlineto{\pgfqpoint{0.984868in}{4.539938in}}%
\pgfpathlineto{\pgfqpoint{0.856579in}{4.539938in}}%
\pgfpathclose%
\pgfusepath{fill}%
\end{pgfscope}%
\begin{pgfscope}%
\pgfpathrectangle{\pgfqpoint{0.600000in}{4.530000in}}{\pgfqpoint{3.900000in}{1.200000in}}%
\pgfusepath{clip}%
\pgfsetbuttcap%
\pgfsetmiterjoin%
\definecolor{currentfill}{rgb}{0.121569,0.466667,0.705882}%
\pgfsetfillcolor{currentfill}%
\pgfsetlinewidth{0.000000pt}%
\definecolor{currentstroke}{rgb}{0.000000,0.000000,0.000000}%
\pgfsetstrokecolor{currentstroke}%
\pgfsetstrokeopacity{0.000000}%
\pgfsetdash{}{0pt}%
\pgfpathmoveto{\pgfqpoint{0.984868in}{4.530000in}}%
\pgfpathlineto{\pgfqpoint{1.113158in}{4.530000in}}%
\pgfpathlineto{\pgfqpoint{1.113158in}{4.549876in}}%
\pgfpathlineto{\pgfqpoint{0.984868in}{4.549876in}}%
\pgfpathclose%
\pgfusepath{fill}%
\end{pgfscope}%
\begin{pgfscope}%
\pgfpathrectangle{\pgfqpoint{0.600000in}{4.530000in}}{\pgfqpoint{3.900000in}{1.200000in}}%
\pgfusepath{clip}%
\pgfsetbuttcap%
\pgfsetmiterjoin%
\definecolor{currentfill}{rgb}{0.121569,0.466667,0.705882}%
\pgfsetfillcolor{currentfill}%
\pgfsetlinewidth{0.000000pt}%
\definecolor{currentstroke}{rgb}{0.000000,0.000000,0.000000}%
\pgfsetstrokecolor{currentstroke}%
\pgfsetstrokeopacity{0.000000}%
\pgfsetdash{}{0pt}%
\pgfpathmoveto{\pgfqpoint{1.113158in}{4.530000in}}%
\pgfpathlineto{\pgfqpoint{1.241447in}{4.530000in}}%
\pgfpathlineto{\pgfqpoint{1.241447in}{4.539938in}}%
\pgfpathlineto{\pgfqpoint{1.113158in}{4.539938in}}%
\pgfpathclose%
\pgfusepath{fill}%
\end{pgfscope}%
\begin{pgfscope}%
\pgfpathrectangle{\pgfqpoint{0.600000in}{4.530000in}}{\pgfqpoint{3.900000in}{1.200000in}}%
\pgfusepath{clip}%
\pgfsetbuttcap%
\pgfsetmiterjoin%
\definecolor{currentfill}{rgb}{0.121569,0.466667,0.705882}%
\pgfsetfillcolor{currentfill}%
\pgfsetlinewidth{0.000000pt}%
\definecolor{currentstroke}{rgb}{0.000000,0.000000,0.000000}%
\pgfsetstrokecolor{currentstroke}%
\pgfsetstrokeopacity{0.000000}%
\pgfsetdash{}{0pt}%
\pgfpathmoveto{\pgfqpoint{1.241447in}{4.530000in}}%
\pgfpathlineto{\pgfqpoint{1.369737in}{4.530000in}}%
\pgfpathlineto{\pgfqpoint{1.369737in}{4.579689in}}%
\pgfpathlineto{\pgfqpoint{1.241447in}{4.579689in}}%
\pgfpathclose%
\pgfusepath{fill}%
\end{pgfscope}%
\begin{pgfscope}%
\pgfpathrectangle{\pgfqpoint{0.600000in}{4.530000in}}{\pgfqpoint{3.900000in}{1.200000in}}%
\pgfusepath{clip}%
\pgfsetbuttcap%
\pgfsetmiterjoin%
\definecolor{currentfill}{rgb}{0.121569,0.466667,0.705882}%
\pgfsetfillcolor{currentfill}%
\pgfsetlinewidth{0.000000pt}%
\definecolor{currentstroke}{rgb}{0.000000,0.000000,0.000000}%
\pgfsetstrokecolor{currentstroke}%
\pgfsetstrokeopacity{0.000000}%
\pgfsetdash{}{0pt}%
\pgfpathmoveto{\pgfqpoint{1.369737in}{4.530000in}}%
\pgfpathlineto{\pgfqpoint{1.498026in}{4.530000in}}%
\pgfpathlineto{\pgfqpoint{1.498026in}{4.629379in}}%
\pgfpathlineto{\pgfqpoint{1.369737in}{4.629379in}}%
\pgfpathclose%
\pgfusepath{fill}%
\end{pgfscope}%
\begin{pgfscope}%
\pgfpathrectangle{\pgfqpoint{0.600000in}{4.530000in}}{\pgfqpoint{3.900000in}{1.200000in}}%
\pgfusepath{clip}%
\pgfsetbuttcap%
\pgfsetmiterjoin%
\definecolor{currentfill}{rgb}{0.121569,0.466667,0.705882}%
\pgfsetfillcolor{currentfill}%
\pgfsetlinewidth{0.000000pt}%
\definecolor{currentstroke}{rgb}{0.000000,0.000000,0.000000}%
\pgfsetstrokecolor{currentstroke}%
\pgfsetstrokeopacity{0.000000}%
\pgfsetdash{}{0pt}%
\pgfpathmoveto{\pgfqpoint{1.498026in}{4.530000in}}%
\pgfpathlineto{\pgfqpoint{1.626316in}{4.530000in}}%
\pgfpathlineto{\pgfqpoint{1.626316in}{4.649255in}}%
\pgfpathlineto{\pgfqpoint{1.498026in}{4.649255in}}%
\pgfpathclose%
\pgfusepath{fill}%
\end{pgfscope}%
\begin{pgfscope}%
\pgfpathrectangle{\pgfqpoint{0.600000in}{4.530000in}}{\pgfqpoint{3.900000in}{1.200000in}}%
\pgfusepath{clip}%
\pgfsetbuttcap%
\pgfsetmiterjoin%
\definecolor{currentfill}{rgb}{0.121569,0.466667,0.705882}%
\pgfsetfillcolor{currentfill}%
\pgfsetlinewidth{0.000000pt}%
\definecolor{currentstroke}{rgb}{0.000000,0.000000,0.000000}%
\pgfsetstrokecolor{currentstroke}%
\pgfsetstrokeopacity{0.000000}%
\pgfsetdash{}{0pt}%
\pgfpathmoveto{\pgfqpoint{1.626316in}{4.530000in}}%
\pgfpathlineto{\pgfqpoint{1.754605in}{4.530000in}}%
\pgfpathlineto{\pgfqpoint{1.754605in}{4.748634in}}%
\pgfpathlineto{\pgfqpoint{1.626316in}{4.748634in}}%
\pgfpathclose%
\pgfusepath{fill}%
\end{pgfscope}%
\begin{pgfscope}%
\pgfpathrectangle{\pgfqpoint{0.600000in}{4.530000in}}{\pgfqpoint{3.900000in}{1.200000in}}%
\pgfusepath{clip}%
\pgfsetbuttcap%
\pgfsetmiterjoin%
\definecolor{currentfill}{rgb}{0.121569,0.466667,0.705882}%
\pgfsetfillcolor{currentfill}%
\pgfsetlinewidth{0.000000pt}%
\definecolor{currentstroke}{rgb}{0.000000,0.000000,0.000000}%
\pgfsetstrokecolor{currentstroke}%
\pgfsetstrokeopacity{0.000000}%
\pgfsetdash{}{0pt}%
\pgfpathmoveto{\pgfqpoint{1.754605in}{4.530000in}}%
\pgfpathlineto{\pgfqpoint{1.882895in}{4.530000in}}%
\pgfpathlineto{\pgfqpoint{1.882895in}{4.937453in}}%
\pgfpathlineto{\pgfqpoint{1.754605in}{4.937453in}}%
\pgfpathclose%
\pgfusepath{fill}%
\end{pgfscope}%
\begin{pgfscope}%
\pgfpathrectangle{\pgfqpoint{0.600000in}{4.530000in}}{\pgfqpoint{3.900000in}{1.200000in}}%
\pgfusepath{clip}%
\pgfsetbuttcap%
\pgfsetmiterjoin%
\definecolor{currentfill}{rgb}{0.121569,0.466667,0.705882}%
\pgfsetfillcolor{currentfill}%
\pgfsetlinewidth{0.000000pt}%
\definecolor{currentstroke}{rgb}{0.000000,0.000000,0.000000}%
\pgfsetstrokecolor{currentstroke}%
\pgfsetstrokeopacity{0.000000}%
\pgfsetdash{}{0pt}%
\pgfpathmoveto{\pgfqpoint{1.882895in}{4.530000in}}%
\pgfpathlineto{\pgfqpoint{2.011184in}{4.530000in}}%
\pgfpathlineto{\pgfqpoint{2.011184in}{5.007019in}}%
\pgfpathlineto{\pgfqpoint{1.882895in}{5.007019in}}%
\pgfpathclose%
\pgfusepath{fill}%
\end{pgfscope}%
\begin{pgfscope}%
\pgfpathrectangle{\pgfqpoint{0.600000in}{4.530000in}}{\pgfqpoint{3.900000in}{1.200000in}}%
\pgfusepath{clip}%
\pgfsetbuttcap%
\pgfsetmiterjoin%
\definecolor{currentfill}{rgb}{0.121569,0.466667,0.705882}%
\pgfsetfillcolor{currentfill}%
\pgfsetlinewidth{0.000000pt}%
\definecolor{currentstroke}{rgb}{0.000000,0.000000,0.000000}%
\pgfsetstrokecolor{currentstroke}%
\pgfsetstrokeopacity{0.000000}%
\pgfsetdash{}{0pt}%
\pgfpathmoveto{\pgfqpoint{2.011184in}{4.530000in}}%
\pgfpathlineto{\pgfqpoint{2.139474in}{4.530000in}}%
\pgfpathlineto{\pgfqpoint{2.139474in}{5.245528in}}%
\pgfpathlineto{\pgfqpoint{2.011184in}{5.245528in}}%
\pgfpathclose%
\pgfusepath{fill}%
\end{pgfscope}%
\begin{pgfscope}%
\pgfpathrectangle{\pgfqpoint{0.600000in}{4.530000in}}{\pgfqpoint{3.900000in}{1.200000in}}%
\pgfusepath{clip}%
\pgfsetbuttcap%
\pgfsetmiterjoin%
\definecolor{currentfill}{rgb}{0.121569,0.466667,0.705882}%
\pgfsetfillcolor{currentfill}%
\pgfsetlinewidth{0.000000pt}%
\definecolor{currentstroke}{rgb}{0.000000,0.000000,0.000000}%
\pgfsetstrokecolor{currentstroke}%
\pgfsetstrokeopacity{0.000000}%
\pgfsetdash{}{0pt}%
\pgfpathmoveto{\pgfqpoint{2.139474in}{4.530000in}}%
\pgfpathlineto{\pgfqpoint{2.267763in}{4.530000in}}%
\pgfpathlineto{\pgfqpoint{2.267763in}{5.265404in}}%
\pgfpathlineto{\pgfqpoint{2.139474in}{5.265404in}}%
\pgfpathclose%
\pgfusepath{fill}%
\end{pgfscope}%
\begin{pgfscope}%
\pgfpathrectangle{\pgfqpoint{0.600000in}{4.530000in}}{\pgfqpoint{3.900000in}{1.200000in}}%
\pgfusepath{clip}%
\pgfsetbuttcap%
\pgfsetmiterjoin%
\definecolor{currentfill}{rgb}{0.121569,0.466667,0.705882}%
\pgfsetfillcolor{currentfill}%
\pgfsetlinewidth{0.000000pt}%
\definecolor{currentstroke}{rgb}{0.000000,0.000000,0.000000}%
\pgfsetstrokecolor{currentstroke}%
\pgfsetstrokeopacity{0.000000}%
\pgfsetdash{}{0pt}%
\pgfpathmoveto{\pgfqpoint{2.267763in}{4.530000in}}%
\pgfpathlineto{\pgfqpoint{2.396053in}{4.530000in}}%
\pgfpathlineto{\pgfqpoint{2.396053in}{5.444286in}}%
\pgfpathlineto{\pgfqpoint{2.267763in}{5.444286in}}%
\pgfpathclose%
\pgfusepath{fill}%
\end{pgfscope}%
\begin{pgfscope}%
\pgfpathrectangle{\pgfqpoint{0.600000in}{4.530000in}}{\pgfqpoint{3.900000in}{1.200000in}}%
\pgfusepath{clip}%
\pgfsetbuttcap%
\pgfsetmiterjoin%
\definecolor{currentfill}{rgb}{0.121569,0.466667,0.705882}%
\pgfsetfillcolor{currentfill}%
\pgfsetlinewidth{0.000000pt}%
\definecolor{currentstroke}{rgb}{0.000000,0.000000,0.000000}%
\pgfsetstrokecolor{currentstroke}%
\pgfsetstrokeopacity{0.000000}%
\pgfsetdash{}{0pt}%
\pgfpathmoveto{\pgfqpoint{2.396053in}{4.530000in}}%
\pgfpathlineto{\pgfqpoint{2.524342in}{4.530000in}}%
\pgfpathlineto{\pgfqpoint{2.524342in}{5.533727in}}%
\pgfpathlineto{\pgfqpoint{2.396053in}{5.533727in}}%
\pgfpathclose%
\pgfusepath{fill}%
\end{pgfscope}%
\begin{pgfscope}%
\pgfpathrectangle{\pgfqpoint{0.600000in}{4.530000in}}{\pgfqpoint{3.900000in}{1.200000in}}%
\pgfusepath{clip}%
\pgfsetbuttcap%
\pgfsetmiterjoin%
\definecolor{currentfill}{rgb}{0.121569,0.466667,0.705882}%
\pgfsetfillcolor{currentfill}%
\pgfsetlinewidth{0.000000pt}%
\definecolor{currentstroke}{rgb}{0.000000,0.000000,0.000000}%
\pgfsetstrokecolor{currentstroke}%
\pgfsetstrokeopacity{0.000000}%
\pgfsetdash{}{0pt}%
\pgfpathmoveto{\pgfqpoint{2.524342in}{4.530000in}}%
\pgfpathlineto{\pgfqpoint{2.652632in}{4.530000in}}%
\pgfpathlineto{\pgfqpoint{2.652632in}{5.672857in}}%
\pgfpathlineto{\pgfqpoint{2.524342in}{5.672857in}}%
\pgfpathclose%
\pgfusepath{fill}%
\end{pgfscope}%
\begin{pgfscope}%
\pgfpathrectangle{\pgfqpoint{0.600000in}{4.530000in}}{\pgfqpoint{3.900000in}{1.200000in}}%
\pgfusepath{clip}%
\pgfsetbuttcap%
\pgfsetmiterjoin%
\definecolor{currentfill}{rgb}{0.121569,0.466667,0.705882}%
\pgfsetfillcolor{currentfill}%
\pgfsetlinewidth{0.000000pt}%
\definecolor{currentstroke}{rgb}{0.000000,0.000000,0.000000}%
\pgfsetstrokecolor{currentstroke}%
\pgfsetstrokeopacity{0.000000}%
\pgfsetdash{}{0pt}%
\pgfpathmoveto{\pgfqpoint{2.652632in}{4.530000in}}%
\pgfpathlineto{\pgfqpoint{2.780921in}{4.530000in}}%
\pgfpathlineto{\pgfqpoint{2.780921in}{5.503913in}}%
\pgfpathlineto{\pgfqpoint{2.652632in}{5.503913in}}%
\pgfpathclose%
\pgfusepath{fill}%
\end{pgfscope}%
\begin{pgfscope}%
\pgfpathrectangle{\pgfqpoint{0.600000in}{4.530000in}}{\pgfqpoint{3.900000in}{1.200000in}}%
\pgfusepath{clip}%
\pgfsetbuttcap%
\pgfsetmiterjoin%
\definecolor{currentfill}{rgb}{0.121569,0.466667,0.705882}%
\pgfsetfillcolor{currentfill}%
\pgfsetlinewidth{0.000000pt}%
\definecolor{currentstroke}{rgb}{0.000000,0.000000,0.000000}%
\pgfsetstrokecolor{currentstroke}%
\pgfsetstrokeopacity{0.000000}%
\pgfsetdash{}{0pt}%
\pgfpathmoveto{\pgfqpoint{2.780921in}{4.530000in}}%
\pgfpathlineto{\pgfqpoint{2.909211in}{4.530000in}}%
\pgfpathlineto{\pgfqpoint{2.909211in}{5.305155in}}%
\pgfpathlineto{\pgfqpoint{2.780921in}{5.305155in}}%
\pgfpathclose%
\pgfusepath{fill}%
\end{pgfscope}%
\begin{pgfscope}%
\pgfpathrectangle{\pgfqpoint{0.600000in}{4.530000in}}{\pgfqpoint{3.900000in}{1.200000in}}%
\pgfusepath{clip}%
\pgfsetbuttcap%
\pgfsetmiterjoin%
\definecolor{currentfill}{rgb}{0.121569,0.466667,0.705882}%
\pgfsetfillcolor{currentfill}%
\pgfsetlinewidth{0.000000pt}%
\definecolor{currentstroke}{rgb}{0.000000,0.000000,0.000000}%
\pgfsetstrokecolor{currentstroke}%
\pgfsetstrokeopacity{0.000000}%
\pgfsetdash{}{0pt}%
\pgfpathmoveto{\pgfqpoint{2.909211in}{4.530000in}}%
\pgfpathlineto{\pgfqpoint{3.037500in}{4.530000in}}%
\pgfpathlineto{\pgfqpoint{3.037500in}{5.255466in}}%
\pgfpathlineto{\pgfqpoint{2.909211in}{5.255466in}}%
\pgfpathclose%
\pgfusepath{fill}%
\end{pgfscope}%
\begin{pgfscope}%
\pgfpathrectangle{\pgfqpoint{0.600000in}{4.530000in}}{\pgfqpoint{3.900000in}{1.200000in}}%
\pgfusepath{clip}%
\pgfsetbuttcap%
\pgfsetmiterjoin%
\definecolor{currentfill}{rgb}{0.121569,0.466667,0.705882}%
\pgfsetfillcolor{currentfill}%
\pgfsetlinewidth{0.000000pt}%
\definecolor{currentstroke}{rgb}{0.000000,0.000000,0.000000}%
\pgfsetstrokecolor{currentstroke}%
\pgfsetstrokeopacity{0.000000}%
\pgfsetdash{}{0pt}%
\pgfpathmoveto{\pgfqpoint{3.037500in}{4.530000in}}%
\pgfpathlineto{\pgfqpoint{3.165789in}{4.530000in}}%
\pgfpathlineto{\pgfqpoint{3.165789in}{5.056708in}}%
\pgfpathlineto{\pgfqpoint{3.037500in}{5.056708in}}%
\pgfpathclose%
\pgfusepath{fill}%
\end{pgfscope}%
\begin{pgfscope}%
\pgfpathrectangle{\pgfqpoint{0.600000in}{4.530000in}}{\pgfqpoint{3.900000in}{1.200000in}}%
\pgfusepath{clip}%
\pgfsetbuttcap%
\pgfsetmiterjoin%
\definecolor{currentfill}{rgb}{0.121569,0.466667,0.705882}%
\pgfsetfillcolor{currentfill}%
\pgfsetlinewidth{0.000000pt}%
\definecolor{currentstroke}{rgb}{0.000000,0.000000,0.000000}%
\pgfsetstrokecolor{currentstroke}%
\pgfsetstrokeopacity{0.000000}%
\pgfsetdash{}{0pt}%
\pgfpathmoveto{\pgfqpoint{3.165789in}{4.530000in}}%
\pgfpathlineto{\pgfqpoint{3.294079in}{4.530000in}}%
\pgfpathlineto{\pgfqpoint{3.294079in}{4.877826in}}%
\pgfpathlineto{\pgfqpoint{3.165789in}{4.877826in}}%
\pgfpathclose%
\pgfusepath{fill}%
\end{pgfscope}%
\begin{pgfscope}%
\pgfpathrectangle{\pgfqpoint{0.600000in}{4.530000in}}{\pgfqpoint{3.900000in}{1.200000in}}%
\pgfusepath{clip}%
\pgfsetbuttcap%
\pgfsetmiterjoin%
\definecolor{currentfill}{rgb}{0.121569,0.466667,0.705882}%
\pgfsetfillcolor{currentfill}%
\pgfsetlinewidth{0.000000pt}%
\definecolor{currentstroke}{rgb}{0.000000,0.000000,0.000000}%
\pgfsetstrokecolor{currentstroke}%
\pgfsetstrokeopacity{0.000000}%
\pgfsetdash{}{0pt}%
\pgfpathmoveto{\pgfqpoint{3.294079in}{4.530000in}}%
\pgfpathlineto{\pgfqpoint{3.422368in}{4.530000in}}%
\pgfpathlineto{\pgfqpoint{3.422368in}{4.778447in}}%
\pgfpathlineto{\pgfqpoint{3.294079in}{4.778447in}}%
\pgfpathclose%
\pgfusepath{fill}%
\end{pgfscope}%
\begin{pgfscope}%
\pgfpathrectangle{\pgfqpoint{0.600000in}{4.530000in}}{\pgfqpoint{3.900000in}{1.200000in}}%
\pgfusepath{clip}%
\pgfsetbuttcap%
\pgfsetmiterjoin%
\definecolor{currentfill}{rgb}{0.121569,0.466667,0.705882}%
\pgfsetfillcolor{currentfill}%
\pgfsetlinewidth{0.000000pt}%
\definecolor{currentstroke}{rgb}{0.000000,0.000000,0.000000}%
\pgfsetstrokecolor{currentstroke}%
\pgfsetstrokeopacity{0.000000}%
\pgfsetdash{}{0pt}%
\pgfpathmoveto{\pgfqpoint{3.422368in}{4.530000in}}%
\pgfpathlineto{\pgfqpoint{3.550658in}{4.530000in}}%
\pgfpathlineto{\pgfqpoint{3.550658in}{4.718820in}}%
\pgfpathlineto{\pgfqpoint{3.422368in}{4.718820in}}%
\pgfpathclose%
\pgfusepath{fill}%
\end{pgfscope}%
\begin{pgfscope}%
\pgfpathrectangle{\pgfqpoint{0.600000in}{4.530000in}}{\pgfqpoint{3.900000in}{1.200000in}}%
\pgfusepath{clip}%
\pgfsetbuttcap%
\pgfsetmiterjoin%
\definecolor{currentfill}{rgb}{0.121569,0.466667,0.705882}%
\pgfsetfillcolor{currentfill}%
\pgfsetlinewidth{0.000000pt}%
\definecolor{currentstroke}{rgb}{0.000000,0.000000,0.000000}%
\pgfsetstrokecolor{currentstroke}%
\pgfsetstrokeopacity{0.000000}%
\pgfsetdash{}{0pt}%
\pgfpathmoveto{\pgfqpoint{3.550658in}{4.530000in}}%
\pgfpathlineto{\pgfqpoint{3.678947in}{4.530000in}}%
\pgfpathlineto{\pgfqpoint{3.678947in}{4.649255in}}%
\pgfpathlineto{\pgfqpoint{3.550658in}{4.649255in}}%
\pgfpathclose%
\pgfusepath{fill}%
\end{pgfscope}%
\begin{pgfscope}%
\pgfpathrectangle{\pgfqpoint{0.600000in}{4.530000in}}{\pgfqpoint{3.900000in}{1.200000in}}%
\pgfusepath{clip}%
\pgfsetbuttcap%
\pgfsetmiterjoin%
\definecolor{currentfill}{rgb}{0.121569,0.466667,0.705882}%
\pgfsetfillcolor{currentfill}%
\pgfsetlinewidth{0.000000pt}%
\definecolor{currentstroke}{rgb}{0.000000,0.000000,0.000000}%
\pgfsetstrokecolor{currentstroke}%
\pgfsetstrokeopacity{0.000000}%
\pgfsetdash{}{0pt}%
\pgfpathmoveto{\pgfqpoint{3.678947in}{4.530000in}}%
\pgfpathlineto{\pgfqpoint{3.807237in}{4.530000in}}%
\pgfpathlineto{\pgfqpoint{3.807237in}{4.549876in}}%
\pgfpathlineto{\pgfqpoint{3.678947in}{4.549876in}}%
\pgfpathclose%
\pgfusepath{fill}%
\end{pgfscope}%
\begin{pgfscope}%
\pgfpathrectangle{\pgfqpoint{0.600000in}{4.530000in}}{\pgfqpoint{3.900000in}{1.200000in}}%
\pgfusepath{clip}%
\pgfsetbuttcap%
\pgfsetmiterjoin%
\definecolor{currentfill}{rgb}{0.121569,0.466667,0.705882}%
\pgfsetfillcolor{currentfill}%
\pgfsetlinewidth{0.000000pt}%
\definecolor{currentstroke}{rgb}{0.000000,0.000000,0.000000}%
\pgfsetstrokecolor{currentstroke}%
\pgfsetstrokeopacity{0.000000}%
\pgfsetdash{}{0pt}%
\pgfpathmoveto{\pgfqpoint{3.807237in}{4.530000in}}%
\pgfpathlineto{\pgfqpoint{3.935526in}{4.530000in}}%
\pgfpathlineto{\pgfqpoint{3.935526in}{4.589627in}}%
\pgfpathlineto{\pgfqpoint{3.807237in}{4.589627in}}%
\pgfpathclose%
\pgfusepath{fill}%
\end{pgfscope}%
\begin{pgfscope}%
\pgfpathrectangle{\pgfqpoint{0.600000in}{4.530000in}}{\pgfqpoint{3.900000in}{1.200000in}}%
\pgfusepath{clip}%
\pgfsetbuttcap%
\pgfsetmiterjoin%
\definecolor{currentfill}{rgb}{0.121569,0.466667,0.705882}%
\pgfsetfillcolor{currentfill}%
\pgfsetlinewidth{0.000000pt}%
\definecolor{currentstroke}{rgb}{0.000000,0.000000,0.000000}%
\pgfsetstrokecolor{currentstroke}%
\pgfsetstrokeopacity{0.000000}%
\pgfsetdash{}{0pt}%
\pgfpathmoveto{\pgfqpoint{3.935526in}{4.530000in}}%
\pgfpathlineto{\pgfqpoint{4.063816in}{4.530000in}}%
\pgfpathlineto{\pgfqpoint{4.063816in}{4.549876in}}%
\pgfpathlineto{\pgfqpoint{3.935526in}{4.549876in}}%
\pgfpathclose%
\pgfusepath{fill}%
\end{pgfscope}%
\begin{pgfscope}%
\pgfpathrectangle{\pgfqpoint{0.600000in}{4.530000in}}{\pgfqpoint{3.900000in}{1.200000in}}%
\pgfusepath{clip}%
\pgfsetbuttcap%
\pgfsetmiterjoin%
\definecolor{currentfill}{rgb}{0.121569,0.466667,0.705882}%
\pgfsetfillcolor{currentfill}%
\pgfsetlinewidth{0.000000pt}%
\definecolor{currentstroke}{rgb}{0.000000,0.000000,0.000000}%
\pgfsetstrokecolor{currentstroke}%
\pgfsetstrokeopacity{0.000000}%
\pgfsetdash{}{0pt}%
\pgfpathmoveto{\pgfqpoint{4.063816in}{4.530000in}}%
\pgfpathlineto{\pgfqpoint{4.192105in}{4.530000in}}%
\pgfpathlineto{\pgfqpoint{4.192105in}{4.530000in}}%
\pgfpathlineto{\pgfqpoint{4.063816in}{4.530000in}}%
\pgfpathclose%
\pgfusepath{fill}%
\end{pgfscope}%
\begin{pgfscope}%
\pgfpathrectangle{\pgfqpoint{0.600000in}{4.530000in}}{\pgfqpoint{3.900000in}{1.200000in}}%
\pgfusepath{clip}%
\pgfsetbuttcap%
\pgfsetmiterjoin%
\definecolor{currentfill}{rgb}{0.121569,0.466667,0.705882}%
\pgfsetfillcolor{currentfill}%
\pgfsetlinewidth{0.000000pt}%
\definecolor{currentstroke}{rgb}{0.000000,0.000000,0.000000}%
\pgfsetstrokecolor{currentstroke}%
\pgfsetstrokeopacity{0.000000}%
\pgfsetdash{}{0pt}%
\pgfpathmoveto{\pgfqpoint{4.192105in}{4.530000in}}%
\pgfpathlineto{\pgfqpoint{4.320395in}{4.530000in}}%
\pgfpathlineto{\pgfqpoint{4.320395in}{4.530000in}}%
\pgfpathlineto{\pgfqpoint{4.192105in}{4.530000in}}%
\pgfpathclose%
\pgfusepath{fill}%
\end{pgfscope}%
\begin{pgfscope}%
\pgfpathrectangle{\pgfqpoint{0.600000in}{4.530000in}}{\pgfqpoint{3.900000in}{1.200000in}}%
\pgfusepath{clip}%
\pgfsetbuttcap%
\pgfsetmiterjoin%
\definecolor{currentfill}{rgb}{0.121569,0.466667,0.705882}%
\pgfsetfillcolor{currentfill}%
\pgfsetlinewidth{0.000000pt}%
\definecolor{currentstroke}{rgb}{0.000000,0.000000,0.000000}%
\pgfsetstrokecolor{currentstroke}%
\pgfsetstrokeopacity{0.000000}%
\pgfsetdash{}{0pt}%
\pgfpathmoveto{\pgfqpoint{4.320395in}{4.530000in}}%
\pgfpathlineto{\pgfqpoint{4.448684in}{4.530000in}}%
\pgfpathlineto{\pgfqpoint{4.448684in}{4.530000in}}%
\pgfpathlineto{\pgfqpoint{4.320395in}{4.530000in}}%
\pgfpathclose%
\pgfusepath{fill}%
\end{pgfscope}%
\begin{pgfscope}%
\pgfpathrectangle{\pgfqpoint{0.600000in}{4.530000in}}{\pgfqpoint{3.900000in}{1.200000in}}%
\pgfusepath{clip}%
\pgfsetbuttcap%
\pgfsetmiterjoin%
\definecolor{currentfill}{rgb}{0.121569,0.466667,0.705882}%
\pgfsetfillcolor{currentfill}%
\pgfsetlinewidth{0.000000pt}%
\definecolor{currentstroke}{rgb}{0.000000,0.000000,0.000000}%
\pgfsetstrokecolor{currentstroke}%
\pgfsetstrokeopacity{0.000000}%
\pgfsetdash{}{0pt}%
\pgfpathmoveto{\pgfqpoint{4.448684in}{4.530000in}}%
\pgfpathlineto{\pgfqpoint{4.576974in}{4.530000in}}%
\pgfpathlineto{\pgfqpoint{4.576974in}{4.530000in}}%
\pgfpathlineto{\pgfqpoint{4.448684in}{4.530000in}}%
\pgfpathclose%
\pgfusepath{fill}%
\end{pgfscope}%
\begin{pgfscope}%
\pgfsetbuttcap%
\pgfsetroundjoin%
\definecolor{currentfill}{rgb}{0.000000,0.000000,0.000000}%
\pgfsetfillcolor{currentfill}%
\pgfsetlinewidth{0.803000pt}%
\definecolor{currentstroke}{rgb}{0.000000,0.000000,0.000000}%
\pgfsetstrokecolor{currentstroke}%
\pgfsetdash{}{0pt}%
\pgfsys@defobject{currentmarker}{\pgfqpoint{0.000000in}{0.000000in}}{\pgfqpoint{0.000000in}{0.048611in}}{%
\pgfpathmoveto{\pgfqpoint{0.000000in}{0.000000in}}%
\pgfpathlineto{\pgfqpoint{0.000000in}{0.048611in}}%
\pgfusepath{stroke,fill}%
}%
\begin{pgfscope}%
\pgfsys@transformshift{1.010526in}{4.530000in}%
\pgfsys@useobject{currentmarker}{}%
\end{pgfscope}%
\end{pgfscope}%
\begin{pgfscope}%
\pgfsetbuttcap%
\pgfsetroundjoin%
\definecolor{currentfill}{rgb}{0.000000,0.000000,0.000000}%
\pgfsetfillcolor{currentfill}%
\pgfsetlinewidth{0.803000pt}%
\definecolor{currentstroke}{rgb}{0.000000,0.000000,0.000000}%
\pgfsetstrokecolor{currentstroke}%
\pgfsetdash{}{0pt}%
\pgfsys@defobject{currentmarker}{\pgfqpoint{0.000000in}{0.000000in}}{\pgfqpoint{0.000000in}{0.048611in}}{%
\pgfpathmoveto{\pgfqpoint{0.000000in}{0.000000in}}%
\pgfpathlineto{\pgfqpoint{0.000000in}{0.048611in}}%
\pgfusepath{stroke,fill}%
}%
\begin{pgfscope}%
\pgfsys@transformshift{1.523684in}{4.530000in}%
\pgfsys@useobject{currentmarker}{}%
\end{pgfscope}%
\end{pgfscope}%
\begin{pgfscope}%
\pgfsetbuttcap%
\pgfsetroundjoin%
\definecolor{currentfill}{rgb}{0.000000,0.000000,0.000000}%
\pgfsetfillcolor{currentfill}%
\pgfsetlinewidth{0.803000pt}%
\definecolor{currentstroke}{rgb}{0.000000,0.000000,0.000000}%
\pgfsetstrokecolor{currentstroke}%
\pgfsetdash{}{0pt}%
\pgfsys@defobject{currentmarker}{\pgfqpoint{0.000000in}{0.000000in}}{\pgfqpoint{0.000000in}{0.048611in}}{%
\pgfpathmoveto{\pgfqpoint{0.000000in}{0.000000in}}%
\pgfpathlineto{\pgfqpoint{0.000000in}{0.048611in}}%
\pgfusepath{stroke,fill}%
}%
\begin{pgfscope}%
\pgfsys@transformshift{2.036842in}{4.530000in}%
\pgfsys@useobject{currentmarker}{}%
\end{pgfscope}%
\end{pgfscope}%
\begin{pgfscope}%
\pgfsetbuttcap%
\pgfsetroundjoin%
\definecolor{currentfill}{rgb}{0.000000,0.000000,0.000000}%
\pgfsetfillcolor{currentfill}%
\pgfsetlinewidth{0.803000pt}%
\definecolor{currentstroke}{rgb}{0.000000,0.000000,0.000000}%
\pgfsetstrokecolor{currentstroke}%
\pgfsetdash{}{0pt}%
\pgfsys@defobject{currentmarker}{\pgfqpoint{0.000000in}{0.000000in}}{\pgfqpoint{0.000000in}{0.048611in}}{%
\pgfpathmoveto{\pgfqpoint{0.000000in}{0.000000in}}%
\pgfpathlineto{\pgfqpoint{0.000000in}{0.048611in}}%
\pgfusepath{stroke,fill}%
}%
\begin{pgfscope}%
\pgfsys@transformshift{2.550000in}{4.530000in}%
\pgfsys@useobject{currentmarker}{}%
\end{pgfscope}%
\end{pgfscope}%
\begin{pgfscope}%
\pgfsetbuttcap%
\pgfsetroundjoin%
\definecolor{currentfill}{rgb}{0.000000,0.000000,0.000000}%
\pgfsetfillcolor{currentfill}%
\pgfsetlinewidth{0.803000pt}%
\definecolor{currentstroke}{rgb}{0.000000,0.000000,0.000000}%
\pgfsetstrokecolor{currentstroke}%
\pgfsetdash{}{0pt}%
\pgfsys@defobject{currentmarker}{\pgfqpoint{0.000000in}{0.000000in}}{\pgfqpoint{0.000000in}{0.048611in}}{%
\pgfpathmoveto{\pgfqpoint{0.000000in}{0.000000in}}%
\pgfpathlineto{\pgfqpoint{0.000000in}{0.048611in}}%
\pgfusepath{stroke,fill}%
}%
\begin{pgfscope}%
\pgfsys@transformshift{3.063158in}{4.530000in}%
\pgfsys@useobject{currentmarker}{}%
\end{pgfscope}%
\end{pgfscope}%
\begin{pgfscope}%
\pgfsetbuttcap%
\pgfsetroundjoin%
\definecolor{currentfill}{rgb}{0.000000,0.000000,0.000000}%
\pgfsetfillcolor{currentfill}%
\pgfsetlinewidth{0.803000pt}%
\definecolor{currentstroke}{rgb}{0.000000,0.000000,0.000000}%
\pgfsetstrokecolor{currentstroke}%
\pgfsetdash{}{0pt}%
\pgfsys@defobject{currentmarker}{\pgfqpoint{0.000000in}{0.000000in}}{\pgfqpoint{0.000000in}{0.048611in}}{%
\pgfpathmoveto{\pgfqpoint{0.000000in}{0.000000in}}%
\pgfpathlineto{\pgfqpoint{0.000000in}{0.048611in}}%
\pgfusepath{stroke,fill}%
}%
\begin{pgfscope}%
\pgfsys@transformshift{3.576316in}{4.530000in}%
\pgfsys@useobject{currentmarker}{}%
\end{pgfscope}%
\end{pgfscope}%
\begin{pgfscope}%
\pgfsetbuttcap%
\pgfsetroundjoin%
\definecolor{currentfill}{rgb}{0.000000,0.000000,0.000000}%
\pgfsetfillcolor{currentfill}%
\pgfsetlinewidth{0.803000pt}%
\definecolor{currentstroke}{rgb}{0.000000,0.000000,0.000000}%
\pgfsetstrokecolor{currentstroke}%
\pgfsetdash{}{0pt}%
\pgfsys@defobject{currentmarker}{\pgfqpoint{0.000000in}{0.000000in}}{\pgfqpoint{0.000000in}{0.048611in}}{%
\pgfpathmoveto{\pgfqpoint{0.000000in}{0.000000in}}%
\pgfpathlineto{\pgfqpoint{0.000000in}{0.048611in}}%
\pgfusepath{stroke,fill}%
}%
\begin{pgfscope}%
\pgfsys@transformshift{4.089474in}{4.530000in}%
\pgfsys@useobject{currentmarker}{}%
\end{pgfscope}%
\end{pgfscope}%
\begin{pgfscope}%
\pgfsetbuttcap%
\pgfsetroundjoin%
\definecolor{currentfill}{rgb}{0.000000,0.000000,0.000000}%
\pgfsetfillcolor{currentfill}%
\pgfsetlinewidth{0.803000pt}%
\definecolor{currentstroke}{rgb}{0.000000,0.000000,0.000000}%
\pgfsetstrokecolor{currentstroke}%
\pgfsetdash{}{0pt}%
\pgfsys@defobject{currentmarker}{\pgfqpoint{0.000000in}{0.000000in}}{\pgfqpoint{0.048611in}{0.000000in}}{%
\pgfpathmoveto{\pgfqpoint{0.000000in}{0.000000in}}%
\pgfpathlineto{\pgfqpoint{0.048611in}{0.000000in}}%
\pgfusepath{stroke,fill}%
}%
\begin{pgfscope}%
\pgfsys@transformshift{0.600000in}{4.530000in}%
\pgfsys@useobject{currentmarker}{}%
\end{pgfscope}%
\end{pgfscope}%
\begin{pgfscope}%
\definecolor{textcolor}{rgb}{0.000000,0.000000,0.000000}%
\pgfsetstrokecolor{textcolor}%
\pgfsetfillcolor{textcolor}%
\pgftext[x=0.463024in,y=4.477238in,left,base]{\color{textcolor}\sffamily\fontsize{10.000000}{12.000000}\selectfont 0}%
\end{pgfscope}%
\begin{pgfscope}%
\pgfsetbuttcap%
\pgfsetroundjoin%
\definecolor{currentfill}{rgb}{0.000000,0.000000,0.000000}%
\pgfsetfillcolor{currentfill}%
\pgfsetlinewidth{0.803000pt}%
\definecolor{currentstroke}{rgb}{0.000000,0.000000,0.000000}%
\pgfsetstrokecolor{currentstroke}%
\pgfsetdash{}{0pt}%
\pgfsys@defobject{currentmarker}{\pgfqpoint{0.000000in}{0.000000in}}{\pgfqpoint{0.048611in}{0.000000in}}{%
\pgfpathmoveto{\pgfqpoint{0.000000in}{0.000000in}}%
\pgfpathlineto{\pgfqpoint{0.048611in}{0.000000in}}%
\pgfusepath{stroke,fill}%
}%
\begin{pgfscope}%
\pgfsys@transformshift{0.600000in}{5.026894in}%
\pgfsys@useobject{currentmarker}{}%
\end{pgfscope}%
\end{pgfscope}%
\begin{pgfscope}%
\definecolor{textcolor}{rgb}{0.000000,0.000000,0.000000}%
\pgfsetstrokecolor{textcolor}%
\pgfsetfillcolor{textcolor}%
\pgftext[x=0.374658in,y=4.974133in,left,base]{\color{textcolor}\sffamily\fontsize{10.000000}{12.000000}\selectfont 50}%
\end{pgfscope}%
\begin{pgfscope}%
\pgfsetbuttcap%
\pgfsetroundjoin%
\definecolor{currentfill}{rgb}{0.000000,0.000000,0.000000}%
\pgfsetfillcolor{currentfill}%
\pgfsetlinewidth{0.803000pt}%
\definecolor{currentstroke}{rgb}{0.000000,0.000000,0.000000}%
\pgfsetstrokecolor{currentstroke}%
\pgfsetdash{}{0pt}%
\pgfsys@defobject{currentmarker}{\pgfqpoint{0.000000in}{0.000000in}}{\pgfqpoint{0.048611in}{0.000000in}}{%
\pgfpathmoveto{\pgfqpoint{0.000000in}{0.000000in}}%
\pgfpathlineto{\pgfqpoint{0.048611in}{0.000000in}}%
\pgfusepath{stroke,fill}%
}%
\begin{pgfscope}%
\pgfsys@transformshift{0.600000in}{5.523789in}%
\pgfsys@useobject{currentmarker}{}%
\end{pgfscope}%
\end{pgfscope}%
\begin{pgfscope}%
\definecolor{textcolor}{rgb}{0.000000,0.000000,0.000000}%
\pgfsetstrokecolor{textcolor}%
\pgfsetfillcolor{textcolor}%
\pgftext[x=0.286293in,y=5.471027in,left,base]{\color{textcolor}\sffamily\fontsize{10.000000}{12.000000}\selectfont 100}%
\end{pgfscope}%
\begin{pgfscope}%
\pgfsetrectcap%
\pgfsetmiterjoin%
\pgfsetlinewidth{0.803000pt}%
\definecolor{currentstroke}{rgb}{0.000000,0.000000,0.000000}%
\pgfsetstrokecolor{currentstroke}%
\pgfsetdash{}{0pt}%
\pgfpathmoveto{\pgfqpoint{0.600000in}{4.530000in}}%
\pgfpathlineto{\pgfqpoint{0.600000in}{5.730000in}}%
\pgfusepath{stroke}%
\end{pgfscope}%
\begin{pgfscope}%
\pgfsetrectcap%
\pgfsetmiterjoin%
\pgfsetlinewidth{0.803000pt}%
\definecolor{currentstroke}{rgb}{0.000000,0.000000,0.000000}%
\pgfsetstrokecolor{currentstroke}%
\pgfsetdash{}{0pt}%
\pgfpathmoveto{\pgfqpoint{4.500000in}{4.530000in}}%
\pgfpathlineto{\pgfqpoint{4.500000in}{5.730000in}}%
\pgfusepath{stroke}%
\end{pgfscope}%
\begin{pgfscope}%
\pgfsetrectcap%
\pgfsetmiterjoin%
\pgfsetlinewidth{0.803000pt}%
\definecolor{currentstroke}{rgb}{0.000000,0.000000,0.000000}%
\pgfsetstrokecolor{currentstroke}%
\pgfsetdash{}{0pt}%
\pgfpathmoveto{\pgfqpoint{0.600000in}{4.530000in}}%
\pgfpathlineto{\pgfqpoint{4.500000in}{4.530000in}}%
\pgfusepath{stroke}%
\end{pgfscope}%
\begin{pgfscope}%
\pgfsetrectcap%
\pgfsetmiterjoin%
\pgfsetlinewidth{0.803000pt}%
\definecolor{currentstroke}{rgb}{0.000000,0.000000,0.000000}%
\pgfsetstrokecolor{currentstroke}%
\pgfsetdash{}{0pt}%
\pgfpathmoveto{\pgfqpoint{0.600000in}{5.730000in}}%
\pgfpathlineto{\pgfqpoint{4.500000in}{5.730000in}}%
\pgfusepath{stroke}%
\end{pgfscope}%
\begin{pgfscope}%
\pgfsetbuttcap%
\pgfsetmiterjoin%
\definecolor{currentfill}{rgb}{1.000000,1.000000,1.000000}%
\pgfsetfillcolor{currentfill}%
\pgfsetlinewidth{0.000000pt}%
\definecolor{currentstroke}{rgb}{0.000000,0.000000,0.000000}%
\pgfsetstrokecolor{currentstroke}%
\pgfsetstrokeopacity{0.000000}%
\pgfsetdash{}{0pt}%
\pgfpathmoveto{\pgfqpoint{4.530000in}{0.600000in}}%
\pgfpathlineto{\pgfqpoint{5.730000in}{0.600000in}}%
\pgfpathlineto{\pgfqpoint{5.730000in}{4.500000in}}%
\pgfpathlineto{\pgfqpoint{4.530000in}{4.500000in}}%
\pgfpathclose%
\pgfusepath{fill}%
\end{pgfscope}%
\begin{pgfscope}%
\pgfpathrectangle{\pgfqpoint{4.530000in}{0.600000in}}{\pgfqpoint{1.200000in}{3.900000in}}%
\pgfusepath{clip}%
\pgfsetbuttcap%
\pgfsetmiterjoin%
\definecolor{currentfill}{rgb}{0.121569,0.466667,0.705882}%
\pgfsetfillcolor{currentfill}%
\pgfsetlinewidth{0.000000pt}%
\definecolor{currentstroke}{rgb}{0.000000,0.000000,0.000000}%
\pgfsetstrokecolor{currentstroke}%
\pgfsetstrokeopacity{0.000000}%
\pgfsetdash{}{0pt}%
\pgfpathmoveto{\pgfqpoint{4.530000in}{0.600000in}}%
\pgfpathlineto{\pgfqpoint{4.530000in}{0.600000in}}%
\pgfpathlineto{\pgfqpoint{4.530000in}{0.728289in}}%
\pgfpathlineto{\pgfqpoint{4.530000in}{0.728289in}}%
\pgfpathclose%
\pgfusepath{fill}%
\end{pgfscope}%
\begin{pgfscope}%
\pgfpathrectangle{\pgfqpoint{4.530000in}{0.600000in}}{\pgfqpoint{1.200000in}{3.900000in}}%
\pgfusepath{clip}%
\pgfsetbuttcap%
\pgfsetmiterjoin%
\definecolor{currentfill}{rgb}{0.121569,0.466667,0.705882}%
\pgfsetfillcolor{currentfill}%
\pgfsetlinewidth{0.000000pt}%
\definecolor{currentstroke}{rgb}{0.000000,0.000000,0.000000}%
\pgfsetstrokecolor{currentstroke}%
\pgfsetstrokeopacity{0.000000}%
\pgfsetdash{}{0pt}%
\pgfpathmoveto{\pgfqpoint{4.530000in}{0.728289in}}%
\pgfpathlineto{\pgfqpoint{4.551164in}{0.728289in}}%
\pgfpathlineto{\pgfqpoint{4.551164in}{0.856579in}}%
\pgfpathlineto{\pgfqpoint{4.530000in}{0.856579in}}%
\pgfpathclose%
\pgfusepath{fill}%
\end{pgfscope}%
\begin{pgfscope}%
\pgfpathrectangle{\pgfqpoint{4.530000in}{0.600000in}}{\pgfqpoint{1.200000in}{3.900000in}}%
\pgfusepath{clip}%
\pgfsetbuttcap%
\pgfsetmiterjoin%
\definecolor{currentfill}{rgb}{0.121569,0.466667,0.705882}%
\pgfsetfillcolor{currentfill}%
\pgfsetlinewidth{0.000000pt}%
\definecolor{currentstroke}{rgb}{0.000000,0.000000,0.000000}%
\pgfsetstrokecolor{currentstroke}%
\pgfsetstrokeopacity{0.000000}%
\pgfsetdash{}{0pt}%
\pgfpathmoveto{\pgfqpoint{4.530000in}{0.856579in}}%
\pgfpathlineto{\pgfqpoint{4.551164in}{0.856579in}}%
\pgfpathlineto{\pgfqpoint{4.551164in}{0.984868in}}%
\pgfpathlineto{\pgfqpoint{4.530000in}{0.984868in}}%
\pgfpathclose%
\pgfusepath{fill}%
\end{pgfscope}%
\begin{pgfscope}%
\pgfpathrectangle{\pgfqpoint{4.530000in}{0.600000in}}{\pgfqpoint{1.200000in}{3.900000in}}%
\pgfusepath{clip}%
\pgfsetbuttcap%
\pgfsetmiterjoin%
\definecolor{currentfill}{rgb}{0.121569,0.466667,0.705882}%
\pgfsetfillcolor{currentfill}%
\pgfsetlinewidth{0.000000pt}%
\definecolor{currentstroke}{rgb}{0.000000,0.000000,0.000000}%
\pgfsetstrokecolor{currentstroke}%
\pgfsetstrokeopacity{0.000000}%
\pgfsetdash{}{0pt}%
\pgfpathmoveto{\pgfqpoint{4.530000in}{0.984868in}}%
\pgfpathlineto{\pgfqpoint{4.530000in}{0.984868in}}%
\pgfpathlineto{\pgfqpoint{4.530000in}{1.113158in}}%
\pgfpathlineto{\pgfqpoint{4.530000in}{1.113158in}}%
\pgfpathclose%
\pgfusepath{fill}%
\end{pgfscope}%
\begin{pgfscope}%
\pgfpathrectangle{\pgfqpoint{4.530000in}{0.600000in}}{\pgfqpoint{1.200000in}{3.900000in}}%
\pgfusepath{clip}%
\pgfsetbuttcap%
\pgfsetmiterjoin%
\definecolor{currentfill}{rgb}{0.121569,0.466667,0.705882}%
\pgfsetfillcolor{currentfill}%
\pgfsetlinewidth{0.000000pt}%
\definecolor{currentstroke}{rgb}{0.000000,0.000000,0.000000}%
\pgfsetstrokecolor{currentstroke}%
\pgfsetstrokeopacity{0.000000}%
\pgfsetdash{}{0pt}%
\pgfpathmoveto{\pgfqpoint{4.530000in}{1.113158in}}%
\pgfpathlineto{\pgfqpoint{4.572328in}{1.113158in}}%
\pgfpathlineto{\pgfqpoint{4.572328in}{1.241447in}}%
\pgfpathlineto{\pgfqpoint{4.530000in}{1.241447in}}%
\pgfpathclose%
\pgfusepath{fill}%
\end{pgfscope}%
\begin{pgfscope}%
\pgfpathrectangle{\pgfqpoint{4.530000in}{0.600000in}}{\pgfqpoint{1.200000in}{3.900000in}}%
\pgfusepath{clip}%
\pgfsetbuttcap%
\pgfsetmiterjoin%
\definecolor{currentfill}{rgb}{0.121569,0.466667,0.705882}%
\pgfsetfillcolor{currentfill}%
\pgfsetlinewidth{0.000000pt}%
\definecolor{currentstroke}{rgb}{0.000000,0.000000,0.000000}%
\pgfsetstrokecolor{currentstroke}%
\pgfsetstrokeopacity{0.000000}%
\pgfsetdash{}{0pt}%
\pgfpathmoveto{\pgfqpoint{4.530000in}{1.241447in}}%
\pgfpathlineto{\pgfqpoint{4.604074in}{1.241447in}}%
\pgfpathlineto{\pgfqpoint{4.604074in}{1.369737in}}%
\pgfpathlineto{\pgfqpoint{4.530000in}{1.369737in}}%
\pgfpathclose%
\pgfusepath{fill}%
\end{pgfscope}%
\begin{pgfscope}%
\pgfpathrectangle{\pgfqpoint{4.530000in}{0.600000in}}{\pgfqpoint{1.200000in}{3.900000in}}%
\pgfusepath{clip}%
\pgfsetbuttcap%
\pgfsetmiterjoin%
\definecolor{currentfill}{rgb}{0.121569,0.466667,0.705882}%
\pgfsetfillcolor{currentfill}%
\pgfsetlinewidth{0.000000pt}%
\definecolor{currentstroke}{rgb}{0.000000,0.000000,0.000000}%
\pgfsetstrokecolor{currentstroke}%
\pgfsetstrokeopacity{0.000000}%
\pgfsetdash{}{0pt}%
\pgfpathmoveto{\pgfqpoint{4.530000in}{1.369737in}}%
\pgfpathlineto{\pgfqpoint{4.678148in}{1.369737in}}%
\pgfpathlineto{\pgfqpoint{4.678148in}{1.498026in}}%
\pgfpathlineto{\pgfqpoint{4.530000in}{1.498026in}}%
\pgfpathclose%
\pgfusepath{fill}%
\end{pgfscope}%
\begin{pgfscope}%
\pgfpathrectangle{\pgfqpoint{4.530000in}{0.600000in}}{\pgfqpoint{1.200000in}{3.900000in}}%
\pgfusepath{clip}%
\pgfsetbuttcap%
\pgfsetmiterjoin%
\definecolor{currentfill}{rgb}{0.121569,0.466667,0.705882}%
\pgfsetfillcolor{currentfill}%
\pgfsetlinewidth{0.000000pt}%
\definecolor{currentstroke}{rgb}{0.000000,0.000000,0.000000}%
\pgfsetstrokecolor{currentstroke}%
\pgfsetstrokeopacity{0.000000}%
\pgfsetdash{}{0pt}%
\pgfpathmoveto{\pgfqpoint{4.530000in}{1.498026in}}%
\pgfpathlineto{\pgfqpoint{4.699312in}{1.498026in}}%
\pgfpathlineto{\pgfqpoint{4.699312in}{1.626316in}}%
\pgfpathlineto{\pgfqpoint{4.530000in}{1.626316in}}%
\pgfpathclose%
\pgfusepath{fill}%
\end{pgfscope}%
\begin{pgfscope}%
\pgfpathrectangle{\pgfqpoint{4.530000in}{0.600000in}}{\pgfqpoint{1.200000in}{3.900000in}}%
\pgfusepath{clip}%
\pgfsetbuttcap%
\pgfsetmiterjoin%
\definecolor{currentfill}{rgb}{0.121569,0.466667,0.705882}%
\pgfsetfillcolor{currentfill}%
\pgfsetlinewidth{0.000000pt}%
\definecolor{currentstroke}{rgb}{0.000000,0.000000,0.000000}%
\pgfsetstrokecolor{currentstroke}%
\pgfsetstrokeopacity{0.000000}%
\pgfsetdash{}{0pt}%
\pgfpathmoveto{\pgfqpoint{4.530000in}{1.626316in}}%
\pgfpathlineto{\pgfqpoint{4.900370in}{1.626316in}}%
\pgfpathlineto{\pgfqpoint{4.900370in}{1.754605in}}%
\pgfpathlineto{\pgfqpoint{4.530000in}{1.754605in}}%
\pgfpathclose%
\pgfusepath{fill}%
\end{pgfscope}%
\begin{pgfscope}%
\pgfpathrectangle{\pgfqpoint{4.530000in}{0.600000in}}{\pgfqpoint{1.200000in}{3.900000in}}%
\pgfusepath{clip}%
\pgfsetbuttcap%
\pgfsetmiterjoin%
\definecolor{currentfill}{rgb}{0.121569,0.466667,0.705882}%
\pgfsetfillcolor{currentfill}%
\pgfsetlinewidth{0.000000pt}%
\definecolor{currentstroke}{rgb}{0.000000,0.000000,0.000000}%
\pgfsetstrokecolor{currentstroke}%
\pgfsetstrokeopacity{0.000000}%
\pgfsetdash{}{0pt}%
\pgfpathmoveto{\pgfqpoint{4.530000in}{1.754605in}}%
\pgfpathlineto{\pgfqpoint{4.847460in}{1.754605in}}%
\pgfpathlineto{\pgfqpoint{4.847460in}{1.882895in}}%
\pgfpathlineto{\pgfqpoint{4.530000in}{1.882895in}}%
\pgfpathclose%
\pgfusepath{fill}%
\end{pgfscope}%
\begin{pgfscope}%
\pgfpathrectangle{\pgfqpoint{4.530000in}{0.600000in}}{\pgfqpoint{1.200000in}{3.900000in}}%
\pgfusepath{clip}%
\pgfsetbuttcap%
\pgfsetmiterjoin%
\definecolor{currentfill}{rgb}{0.121569,0.466667,0.705882}%
\pgfsetfillcolor{currentfill}%
\pgfsetlinewidth{0.000000pt}%
\definecolor{currentstroke}{rgb}{0.000000,0.000000,0.000000}%
\pgfsetstrokecolor{currentstroke}%
\pgfsetstrokeopacity{0.000000}%
\pgfsetdash{}{0pt}%
\pgfpathmoveto{\pgfqpoint{4.530000in}{1.882895in}}%
\pgfpathlineto{\pgfqpoint{5.059101in}{1.882895in}}%
\pgfpathlineto{\pgfqpoint{5.059101in}{2.011184in}}%
\pgfpathlineto{\pgfqpoint{4.530000in}{2.011184in}}%
\pgfpathclose%
\pgfusepath{fill}%
\end{pgfscope}%
\begin{pgfscope}%
\pgfpathrectangle{\pgfqpoint{4.530000in}{0.600000in}}{\pgfqpoint{1.200000in}{3.900000in}}%
\pgfusepath{clip}%
\pgfsetbuttcap%
\pgfsetmiterjoin%
\definecolor{currentfill}{rgb}{0.121569,0.466667,0.705882}%
\pgfsetfillcolor{currentfill}%
\pgfsetlinewidth{0.000000pt}%
\definecolor{currentstroke}{rgb}{0.000000,0.000000,0.000000}%
\pgfsetstrokecolor{currentstroke}%
\pgfsetstrokeopacity{0.000000}%
\pgfsetdash{}{0pt}%
\pgfpathmoveto{\pgfqpoint{4.530000in}{2.011184in}}%
\pgfpathlineto{\pgfqpoint{5.291905in}{2.011184in}}%
\pgfpathlineto{\pgfqpoint{5.291905in}{2.139474in}}%
\pgfpathlineto{\pgfqpoint{4.530000in}{2.139474in}}%
\pgfpathclose%
\pgfusepath{fill}%
\end{pgfscope}%
\begin{pgfscope}%
\pgfpathrectangle{\pgfqpoint{4.530000in}{0.600000in}}{\pgfqpoint{1.200000in}{3.900000in}}%
\pgfusepath{clip}%
\pgfsetbuttcap%
\pgfsetmiterjoin%
\definecolor{currentfill}{rgb}{0.121569,0.466667,0.705882}%
\pgfsetfillcolor{currentfill}%
\pgfsetlinewidth{0.000000pt}%
\definecolor{currentstroke}{rgb}{0.000000,0.000000,0.000000}%
\pgfsetstrokecolor{currentstroke}%
\pgfsetstrokeopacity{0.000000}%
\pgfsetdash{}{0pt}%
\pgfpathmoveto{\pgfqpoint{4.530000in}{2.139474in}}%
\pgfpathlineto{\pgfqpoint{5.270741in}{2.139474in}}%
\pgfpathlineto{\pgfqpoint{5.270741in}{2.267763in}}%
\pgfpathlineto{\pgfqpoint{4.530000in}{2.267763in}}%
\pgfpathclose%
\pgfusepath{fill}%
\end{pgfscope}%
\begin{pgfscope}%
\pgfpathrectangle{\pgfqpoint{4.530000in}{0.600000in}}{\pgfqpoint{1.200000in}{3.900000in}}%
\pgfusepath{clip}%
\pgfsetbuttcap%
\pgfsetmiterjoin%
\definecolor{currentfill}{rgb}{0.121569,0.466667,0.705882}%
\pgfsetfillcolor{currentfill}%
\pgfsetlinewidth{0.000000pt}%
\definecolor{currentstroke}{rgb}{0.000000,0.000000,0.000000}%
\pgfsetstrokecolor{currentstroke}%
\pgfsetstrokeopacity{0.000000}%
\pgfsetdash{}{0pt}%
\pgfpathmoveto{\pgfqpoint{4.530000in}{2.267763in}}%
\pgfpathlineto{\pgfqpoint{5.408307in}{2.267763in}}%
\pgfpathlineto{\pgfqpoint{5.408307in}{2.396053in}}%
\pgfpathlineto{\pgfqpoint{4.530000in}{2.396053in}}%
\pgfpathclose%
\pgfusepath{fill}%
\end{pgfscope}%
\begin{pgfscope}%
\pgfpathrectangle{\pgfqpoint{4.530000in}{0.600000in}}{\pgfqpoint{1.200000in}{3.900000in}}%
\pgfusepath{clip}%
\pgfsetbuttcap%
\pgfsetmiterjoin%
\definecolor{currentfill}{rgb}{0.121569,0.466667,0.705882}%
\pgfsetfillcolor{currentfill}%
\pgfsetlinewidth{0.000000pt}%
\definecolor{currentstroke}{rgb}{0.000000,0.000000,0.000000}%
\pgfsetstrokecolor{currentstroke}%
\pgfsetstrokeopacity{0.000000}%
\pgfsetdash{}{0pt}%
\pgfpathmoveto{\pgfqpoint{4.530000in}{2.396053in}}%
\pgfpathlineto{\pgfqpoint{5.577619in}{2.396053in}}%
\pgfpathlineto{\pgfqpoint{5.577619in}{2.524342in}}%
\pgfpathlineto{\pgfqpoint{4.530000in}{2.524342in}}%
\pgfpathclose%
\pgfusepath{fill}%
\end{pgfscope}%
\begin{pgfscope}%
\pgfpathrectangle{\pgfqpoint{4.530000in}{0.600000in}}{\pgfqpoint{1.200000in}{3.900000in}}%
\pgfusepath{clip}%
\pgfsetbuttcap%
\pgfsetmiterjoin%
\definecolor{currentfill}{rgb}{0.121569,0.466667,0.705882}%
\pgfsetfillcolor{currentfill}%
\pgfsetlinewidth{0.000000pt}%
\definecolor{currentstroke}{rgb}{0.000000,0.000000,0.000000}%
\pgfsetstrokecolor{currentstroke}%
\pgfsetstrokeopacity{0.000000}%
\pgfsetdash{}{0pt}%
\pgfpathmoveto{\pgfqpoint{4.530000in}{2.524342in}}%
\pgfpathlineto{\pgfqpoint{5.313069in}{2.524342in}}%
\pgfpathlineto{\pgfqpoint{5.313069in}{2.652632in}}%
\pgfpathlineto{\pgfqpoint{4.530000in}{2.652632in}}%
\pgfpathclose%
\pgfusepath{fill}%
\end{pgfscope}%
\begin{pgfscope}%
\pgfpathrectangle{\pgfqpoint{4.530000in}{0.600000in}}{\pgfqpoint{1.200000in}{3.900000in}}%
\pgfusepath{clip}%
\pgfsetbuttcap%
\pgfsetmiterjoin%
\definecolor{currentfill}{rgb}{0.121569,0.466667,0.705882}%
\pgfsetfillcolor{currentfill}%
\pgfsetlinewidth{0.000000pt}%
\definecolor{currentstroke}{rgb}{0.000000,0.000000,0.000000}%
\pgfsetstrokecolor{currentstroke}%
\pgfsetstrokeopacity{0.000000}%
\pgfsetdash{}{0pt}%
\pgfpathmoveto{\pgfqpoint{4.530000in}{2.652632in}}%
\pgfpathlineto{\pgfqpoint{5.672857in}{2.652632in}}%
\pgfpathlineto{\pgfqpoint{5.672857in}{2.780921in}}%
\pgfpathlineto{\pgfqpoint{4.530000in}{2.780921in}}%
\pgfpathclose%
\pgfusepath{fill}%
\end{pgfscope}%
\begin{pgfscope}%
\pgfpathrectangle{\pgfqpoint{4.530000in}{0.600000in}}{\pgfqpoint{1.200000in}{3.900000in}}%
\pgfusepath{clip}%
\pgfsetbuttcap%
\pgfsetmiterjoin%
\definecolor{currentfill}{rgb}{0.121569,0.466667,0.705882}%
\pgfsetfillcolor{currentfill}%
\pgfsetlinewidth{0.000000pt}%
\definecolor{currentstroke}{rgb}{0.000000,0.000000,0.000000}%
\pgfsetstrokecolor{currentstroke}%
\pgfsetstrokeopacity{0.000000}%
\pgfsetdash{}{0pt}%
\pgfpathmoveto{\pgfqpoint{4.530000in}{2.780921in}}%
\pgfpathlineto{\pgfqpoint{5.429471in}{2.780921in}}%
\pgfpathlineto{\pgfqpoint{5.429471in}{2.909211in}}%
\pgfpathlineto{\pgfqpoint{4.530000in}{2.909211in}}%
\pgfpathclose%
\pgfusepath{fill}%
\end{pgfscope}%
\begin{pgfscope}%
\pgfpathrectangle{\pgfqpoint{4.530000in}{0.600000in}}{\pgfqpoint{1.200000in}{3.900000in}}%
\pgfusepath{clip}%
\pgfsetbuttcap%
\pgfsetmiterjoin%
\definecolor{currentfill}{rgb}{0.121569,0.466667,0.705882}%
\pgfsetfillcolor{currentfill}%
\pgfsetlinewidth{0.000000pt}%
\definecolor{currentstroke}{rgb}{0.000000,0.000000,0.000000}%
\pgfsetstrokecolor{currentstroke}%
\pgfsetstrokeopacity{0.000000}%
\pgfsetdash{}{0pt}%
\pgfpathmoveto{\pgfqpoint{4.530000in}{2.909211in}}%
\pgfpathlineto{\pgfqpoint{5.186085in}{2.909211in}}%
\pgfpathlineto{\pgfqpoint{5.186085in}{3.037500in}}%
\pgfpathlineto{\pgfqpoint{4.530000in}{3.037500in}}%
\pgfpathclose%
\pgfusepath{fill}%
\end{pgfscope}%
\begin{pgfscope}%
\pgfpathrectangle{\pgfqpoint{4.530000in}{0.600000in}}{\pgfqpoint{1.200000in}{3.900000in}}%
\pgfusepath{clip}%
\pgfsetbuttcap%
\pgfsetmiterjoin%
\definecolor{currentfill}{rgb}{0.121569,0.466667,0.705882}%
\pgfsetfillcolor{currentfill}%
\pgfsetlinewidth{0.000000pt}%
\definecolor{currentstroke}{rgb}{0.000000,0.000000,0.000000}%
\pgfsetstrokecolor{currentstroke}%
\pgfsetstrokeopacity{0.000000}%
\pgfsetdash{}{0pt}%
\pgfpathmoveto{\pgfqpoint{4.530000in}{3.037500in}}%
\pgfpathlineto{\pgfqpoint{5.175503in}{3.037500in}}%
\pgfpathlineto{\pgfqpoint{5.175503in}{3.165789in}}%
\pgfpathlineto{\pgfqpoint{4.530000in}{3.165789in}}%
\pgfpathclose%
\pgfusepath{fill}%
\end{pgfscope}%
\begin{pgfscope}%
\pgfpathrectangle{\pgfqpoint{4.530000in}{0.600000in}}{\pgfqpoint{1.200000in}{3.900000in}}%
\pgfusepath{clip}%
\pgfsetbuttcap%
\pgfsetmiterjoin%
\definecolor{currentfill}{rgb}{0.121569,0.466667,0.705882}%
\pgfsetfillcolor{currentfill}%
\pgfsetlinewidth{0.000000pt}%
\definecolor{currentstroke}{rgb}{0.000000,0.000000,0.000000}%
\pgfsetstrokecolor{currentstroke}%
\pgfsetstrokeopacity{0.000000}%
\pgfsetdash{}{0pt}%
\pgfpathmoveto{\pgfqpoint{4.530000in}{3.165789in}}%
\pgfpathlineto{\pgfqpoint{4.974444in}{3.165789in}}%
\pgfpathlineto{\pgfqpoint{4.974444in}{3.294079in}}%
\pgfpathlineto{\pgfqpoint{4.530000in}{3.294079in}}%
\pgfpathclose%
\pgfusepath{fill}%
\end{pgfscope}%
\begin{pgfscope}%
\pgfpathrectangle{\pgfqpoint{4.530000in}{0.600000in}}{\pgfqpoint{1.200000in}{3.900000in}}%
\pgfusepath{clip}%
\pgfsetbuttcap%
\pgfsetmiterjoin%
\definecolor{currentfill}{rgb}{0.121569,0.466667,0.705882}%
\pgfsetfillcolor{currentfill}%
\pgfsetlinewidth{0.000000pt}%
\definecolor{currentstroke}{rgb}{0.000000,0.000000,0.000000}%
\pgfsetstrokecolor{currentstroke}%
\pgfsetstrokeopacity{0.000000}%
\pgfsetdash{}{0pt}%
\pgfpathmoveto{\pgfqpoint{4.530000in}{3.294079in}}%
\pgfpathlineto{\pgfqpoint{4.836878in}{3.294079in}}%
\pgfpathlineto{\pgfqpoint{4.836878in}{3.422368in}}%
\pgfpathlineto{\pgfqpoint{4.530000in}{3.422368in}}%
\pgfpathclose%
\pgfusepath{fill}%
\end{pgfscope}%
\begin{pgfscope}%
\pgfpathrectangle{\pgfqpoint{4.530000in}{0.600000in}}{\pgfqpoint{1.200000in}{3.900000in}}%
\pgfusepath{clip}%
\pgfsetbuttcap%
\pgfsetmiterjoin%
\definecolor{currentfill}{rgb}{0.121569,0.466667,0.705882}%
\pgfsetfillcolor{currentfill}%
\pgfsetlinewidth{0.000000pt}%
\definecolor{currentstroke}{rgb}{0.000000,0.000000,0.000000}%
\pgfsetstrokecolor{currentstroke}%
\pgfsetstrokeopacity{0.000000}%
\pgfsetdash{}{0pt}%
\pgfpathmoveto{\pgfqpoint{4.530000in}{3.422368in}}%
\pgfpathlineto{\pgfqpoint{4.836878in}{3.422368in}}%
\pgfpathlineto{\pgfqpoint{4.836878in}{3.550658in}}%
\pgfpathlineto{\pgfqpoint{4.530000in}{3.550658in}}%
\pgfpathclose%
\pgfusepath{fill}%
\end{pgfscope}%
\begin{pgfscope}%
\pgfpathrectangle{\pgfqpoint{4.530000in}{0.600000in}}{\pgfqpoint{1.200000in}{3.900000in}}%
\pgfusepath{clip}%
\pgfsetbuttcap%
\pgfsetmiterjoin%
\definecolor{currentfill}{rgb}{0.121569,0.466667,0.705882}%
\pgfsetfillcolor{currentfill}%
\pgfsetlinewidth{0.000000pt}%
\definecolor{currentstroke}{rgb}{0.000000,0.000000,0.000000}%
\pgfsetstrokecolor{currentstroke}%
\pgfsetstrokeopacity{0.000000}%
\pgfsetdash{}{0pt}%
\pgfpathmoveto{\pgfqpoint{4.530000in}{3.550658in}}%
\pgfpathlineto{\pgfqpoint{4.667566in}{3.550658in}}%
\pgfpathlineto{\pgfqpoint{4.667566in}{3.678947in}}%
\pgfpathlineto{\pgfqpoint{4.530000in}{3.678947in}}%
\pgfpathclose%
\pgfusepath{fill}%
\end{pgfscope}%
\begin{pgfscope}%
\pgfpathrectangle{\pgfqpoint{4.530000in}{0.600000in}}{\pgfqpoint{1.200000in}{3.900000in}}%
\pgfusepath{clip}%
\pgfsetbuttcap%
\pgfsetmiterjoin%
\definecolor{currentfill}{rgb}{0.121569,0.466667,0.705882}%
\pgfsetfillcolor{currentfill}%
\pgfsetlinewidth{0.000000pt}%
\definecolor{currentstroke}{rgb}{0.000000,0.000000,0.000000}%
\pgfsetstrokecolor{currentstroke}%
\pgfsetstrokeopacity{0.000000}%
\pgfsetdash{}{0pt}%
\pgfpathmoveto{\pgfqpoint{4.530000in}{3.678947in}}%
\pgfpathlineto{\pgfqpoint{4.561746in}{3.678947in}}%
\pgfpathlineto{\pgfqpoint{4.561746in}{3.807237in}}%
\pgfpathlineto{\pgfqpoint{4.530000in}{3.807237in}}%
\pgfpathclose%
\pgfusepath{fill}%
\end{pgfscope}%
\begin{pgfscope}%
\pgfpathrectangle{\pgfqpoint{4.530000in}{0.600000in}}{\pgfqpoint{1.200000in}{3.900000in}}%
\pgfusepath{clip}%
\pgfsetbuttcap%
\pgfsetmiterjoin%
\definecolor{currentfill}{rgb}{0.121569,0.466667,0.705882}%
\pgfsetfillcolor{currentfill}%
\pgfsetlinewidth{0.000000pt}%
\definecolor{currentstroke}{rgb}{0.000000,0.000000,0.000000}%
\pgfsetstrokecolor{currentstroke}%
\pgfsetstrokeopacity{0.000000}%
\pgfsetdash{}{0pt}%
\pgfpathmoveto{\pgfqpoint{4.530000in}{3.807237in}}%
\pgfpathlineto{\pgfqpoint{4.604074in}{3.807237in}}%
\pgfpathlineto{\pgfqpoint{4.604074in}{3.935526in}}%
\pgfpathlineto{\pgfqpoint{4.530000in}{3.935526in}}%
\pgfpathclose%
\pgfusepath{fill}%
\end{pgfscope}%
\begin{pgfscope}%
\pgfpathrectangle{\pgfqpoint{4.530000in}{0.600000in}}{\pgfqpoint{1.200000in}{3.900000in}}%
\pgfusepath{clip}%
\pgfsetbuttcap%
\pgfsetmiterjoin%
\definecolor{currentfill}{rgb}{0.121569,0.466667,0.705882}%
\pgfsetfillcolor{currentfill}%
\pgfsetlinewidth{0.000000pt}%
\definecolor{currentstroke}{rgb}{0.000000,0.000000,0.000000}%
\pgfsetstrokecolor{currentstroke}%
\pgfsetstrokeopacity{0.000000}%
\pgfsetdash{}{0pt}%
\pgfpathmoveto{\pgfqpoint{4.530000in}{3.935526in}}%
\pgfpathlineto{\pgfqpoint{4.540582in}{3.935526in}}%
\pgfpathlineto{\pgfqpoint{4.540582in}{4.063816in}}%
\pgfpathlineto{\pgfqpoint{4.530000in}{4.063816in}}%
\pgfpathclose%
\pgfusepath{fill}%
\end{pgfscope}%
\begin{pgfscope}%
\pgfpathrectangle{\pgfqpoint{4.530000in}{0.600000in}}{\pgfqpoint{1.200000in}{3.900000in}}%
\pgfusepath{clip}%
\pgfsetbuttcap%
\pgfsetmiterjoin%
\definecolor{currentfill}{rgb}{0.121569,0.466667,0.705882}%
\pgfsetfillcolor{currentfill}%
\pgfsetlinewidth{0.000000pt}%
\definecolor{currentstroke}{rgb}{0.000000,0.000000,0.000000}%
\pgfsetstrokecolor{currentstroke}%
\pgfsetstrokeopacity{0.000000}%
\pgfsetdash{}{0pt}%
\pgfpathmoveto{\pgfqpoint{4.530000in}{4.063816in}}%
\pgfpathlineto{\pgfqpoint{4.530000in}{4.063816in}}%
\pgfpathlineto{\pgfqpoint{4.530000in}{4.192105in}}%
\pgfpathlineto{\pgfqpoint{4.530000in}{4.192105in}}%
\pgfpathclose%
\pgfusepath{fill}%
\end{pgfscope}%
\begin{pgfscope}%
\pgfpathrectangle{\pgfqpoint{4.530000in}{0.600000in}}{\pgfqpoint{1.200000in}{3.900000in}}%
\pgfusepath{clip}%
\pgfsetbuttcap%
\pgfsetmiterjoin%
\definecolor{currentfill}{rgb}{0.121569,0.466667,0.705882}%
\pgfsetfillcolor{currentfill}%
\pgfsetlinewidth{0.000000pt}%
\definecolor{currentstroke}{rgb}{0.000000,0.000000,0.000000}%
\pgfsetstrokecolor{currentstroke}%
\pgfsetstrokeopacity{0.000000}%
\pgfsetdash{}{0pt}%
\pgfpathmoveto{\pgfqpoint{4.530000in}{4.192105in}}%
\pgfpathlineto{\pgfqpoint{4.540582in}{4.192105in}}%
\pgfpathlineto{\pgfqpoint{4.540582in}{4.320395in}}%
\pgfpathlineto{\pgfqpoint{4.530000in}{4.320395in}}%
\pgfpathclose%
\pgfusepath{fill}%
\end{pgfscope}%
\begin{pgfscope}%
\pgfpathrectangle{\pgfqpoint{4.530000in}{0.600000in}}{\pgfqpoint{1.200000in}{3.900000in}}%
\pgfusepath{clip}%
\pgfsetbuttcap%
\pgfsetmiterjoin%
\definecolor{currentfill}{rgb}{0.121569,0.466667,0.705882}%
\pgfsetfillcolor{currentfill}%
\pgfsetlinewidth{0.000000pt}%
\definecolor{currentstroke}{rgb}{0.000000,0.000000,0.000000}%
\pgfsetstrokecolor{currentstroke}%
\pgfsetstrokeopacity{0.000000}%
\pgfsetdash{}{0pt}%
\pgfpathmoveto{\pgfqpoint{4.530000in}{4.320395in}}%
\pgfpathlineto{\pgfqpoint{4.530000in}{4.320395in}}%
\pgfpathlineto{\pgfqpoint{4.530000in}{4.448684in}}%
\pgfpathlineto{\pgfqpoint{4.530000in}{4.448684in}}%
\pgfpathclose%
\pgfusepath{fill}%
\end{pgfscope}%
\begin{pgfscope}%
\pgfpathrectangle{\pgfqpoint{4.530000in}{0.600000in}}{\pgfqpoint{1.200000in}{3.900000in}}%
\pgfusepath{clip}%
\pgfsetbuttcap%
\pgfsetmiterjoin%
\definecolor{currentfill}{rgb}{0.121569,0.466667,0.705882}%
\pgfsetfillcolor{currentfill}%
\pgfsetlinewidth{0.000000pt}%
\definecolor{currentstroke}{rgb}{0.000000,0.000000,0.000000}%
\pgfsetstrokecolor{currentstroke}%
\pgfsetstrokeopacity{0.000000}%
\pgfsetdash{}{0pt}%
\pgfpathmoveto{\pgfqpoint{4.530000in}{4.448684in}}%
\pgfpathlineto{\pgfqpoint{4.530000in}{4.448684in}}%
\pgfpathlineto{\pgfqpoint{4.530000in}{4.576974in}}%
\pgfpathlineto{\pgfqpoint{4.530000in}{4.576974in}}%
\pgfpathclose%
\pgfusepath{fill}%
\end{pgfscope}%
\begin{pgfscope}%
\pgfsetbuttcap%
\pgfsetroundjoin%
\definecolor{currentfill}{rgb}{0.000000,0.000000,0.000000}%
\pgfsetfillcolor{currentfill}%
\pgfsetlinewidth{0.803000pt}%
\definecolor{currentstroke}{rgb}{0.000000,0.000000,0.000000}%
\pgfsetstrokecolor{currentstroke}%
\pgfsetdash{}{0pt}%
\pgfsys@defobject{currentmarker}{\pgfqpoint{0.000000in}{0.000000in}}{\pgfqpoint{0.000000in}{0.048611in}}{%
\pgfpathmoveto{\pgfqpoint{0.000000in}{0.000000in}}%
\pgfpathlineto{\pgfqpoint{0.000000in}{0.048611in}}%
\pgfusepath{stroke,fill}%
}%
\begin{pgfscope}%
\pgfsys@transformshift{4.530000in}{0.600000in}%
\pgfsys@useobject{currentmarker}{}%
\end{pgfscope}%
\end{pgfscope}%
\begin{pgfscope}%
\definecolor{textcolor}{rgb}{0.000000,0.000000,0.000000}%
\pgfsetstrokecolor{textcolor}%
\pgfsetfillcolor{textcolor}%
\pgftext[x=4.530000in,y=0.551389in,,top]{\color{textcolor}\sffamily\fontsize{10.000000}{12.000000}\selectfont 0}%
\end{pgfscope}%
\begin{pgfscope}%
\pgfsetbuttcap%
\pgfsetroundjoin%
\definecolor{currentfill}{rgb}{0.000000,0.000000,0.000000}%
\pgfsetfillcolor{currentfill}%
\pgfsetlinewidth{0.803000pt}%
\definecolor{currentstroke}{rgb}{0.000000,0.000000,0.000000}%
\pgfsetstrokecolor{currentstroke}%
\pgfsetdash{}{0pt}%
\pgfsys@defobject{currentmarker}{\pgfqpoint{0.000000in}{0.000000in}}{\pgfqpoint{0.000000in}{0.048611in}}{%
\pgfpathmoveto{\pgfqpoint{0.000000in}{0.000000in}}%
\pgfpathlineto{\pgfqpoint{0.000000in}{0.048611in}}%
\pgfusepath{stroke,fill}%
}%
\begin{pgfscope}%
\pgfsys@transformshift{5.588201in}{0.600000in}%
\pgfsys@useobject{currentmarker}{}%
\end{pgfscope}%
\end{pgfscope}%
\begin{pgfscope}%
\definecolor{textcolor}{rgb}{0.000000,0.000000,0.000000}%
\pgfsetstrokecolor{textcolor}%
\pgfsetfillcolor{textcolor}%
\pgftext[x=5.588201in,y=0.551389in,,top]{\color{textcolor}\sffamily\fontsize{10.000000}{12.000000}\selectfont 100}%
\end{pgfscope}%
\begin{pgfscope}%
\pgfsetbuttcap%
\pgfsetroundjoin%
\definecolor{currentfill}{rgb}{0.000000,0.000000,0.000000}%
\pgfsetfillcolor{currentfill}%
\pgfsetlinewidth{0.803000pt}%
\definecolor{currentstroke}{rgb}{0.000000,0.000000,0.000000}%
\pgfsetstrokecolor{currentstroke}%
\pgfsetdash{}{0pt}%
\pgfsys@defobject{currentmarker}{\pgfqpoint{0.000000in}{0.000000in}}{\pgfqpoint{0.048611in}{0.000000in}}{%
\pgfpathmoveto{\pgfqpoint{0.000000in}{0.000000in}}%
\pgfpathlineto{\pgfqpoint{0.048611in}{0.000000in}}%
\pgfusepath{stroke,fill}%
}%
\begin{pgfscope}%
\pgfsys@transformshift{4.530000in}{1.010526in}%
\pgfsys@useobject{currentmarker}{}%
\end{pgfscope}%
\end{pgfscope}%
\begin{pgfscope}%
\pgfsetbuttcap%
\pgfsetroundjoin%
\definecolor{currentfill}{rgb}{0.000000,0.000000,0.000000}%
\pgfsetfillcolor{currentfill}%
\pgfsetlinewidth{0.803000pt}%
\definecolor{currentstroke}{rgb}{0.000000,0.000000,0.000000}%
\pgfsetstrokecolor{currentstroke}%
\pgfsetdash{}{0pt}%
\pgfsys@defobject{currentmarker}{\pgfqpoint{0.000000in}{0.000000in}}{\pgfqpoint{0.048611in}{0.000000in}}{%
\pgfpathmoveto{\pgfqpoint{0.000000in}{0.000000in}}%
\pgfpathlineto{\pgfqpoint{0.048611in}{0.000000in}}%
\pgfusepath{stroke,fill}%
}%
\begin{pgfscope}%
\pgfsys@transformshift{4.530000in}{1.523684in}%
\pgfsys@useobject{currentmarker}{}%
\end{pgfscope}%
\end{pgfscope}%
\begin{pgfscope}%
\pgfsetbuttcap%
\pgfsetroundjoin%
\definecolor{currentfill}{rgb}{0.000000,0.000000,0.000000}%
\pgfsetfillcolor{currentfill}%
\pgfsetlinewidth{0.803000pt}%
\definecolor{currentstroke}{rgb}{0.000000,0.000000,0.000000}%
\pgfsetstrokecolor{currentstroke}%
\pgfsetdash{}{0pt}%
\pgfsys@defobject{currentmarker}{\pgfqpoint{0.000000in}{0.000000in}}{\pgfqpoint{0.048611in}{0.000000in}}{%
\pgfpathmoveto{\pgfqpoint{0.000000in}{0.000000in}}%
\pgfpathlineto{\pgfqpoint{0.048611in}{0.000000in}}%
\pgfusepath{stroke,fill}%
}%
\begin{pgfscope}%
\pgfsys@transformshift{4.530000in}{2.036842in}%
\pgfsys@useobject{currentmarker}{}%
\end{pgfscope}%
\end{pgfscope}%
\begin{pgfscope}%
\pgfsetbuttcap%
\pgfsetroundjoin%
\definecolor{currentfill}{rgb}{0.000000,0.000000,0.000000}%
\pgfsetfillcolor{currentfill}%
\pgfsetlinewidth{0.803000pt}%
\definecolor{currentstroke}{rgb}{0.000000,0.000000,0.000000}%
\pgfsetstrokecolor{currentstroke}%
\pgfsetdash{}{0pt}%
\pgfsys@defobject{currentmarker}{\pgfqpoint{0.000000in}{0.000000in}}{\pgfqpoint{0.048611in}{0.000000in}}{%
\pgfpathmoveto{\pgfqpoint{0.000000in}{0.000000in}}%
\pgfpathlineto{\pgfqpoint{0.048611in}{0.000000in}}%
\pgfusepath{stroke,fill}%
}%
\begin{pgfscope}%
\pgfsys@transformshift{4.530000in}{2.550000in}%
\pgfsys@useobject{currentmarker}{}%
\end{pgfscope}%
\end{pgfscope}%
\begin{pgfscope}%
\pgfsetbuttcap%
\pgfsetroundjoin%
\definecolor{currentfill}{rgb}{0.000000,0.000000,0.000000}%
\pgfsetfillcolor{currentfill}%
\pgfsetlinewidth{0.803000pt}%
\definecolor{currentstroke}{rgb}{0.000000,0.000000,0.000000}%
\pgfsetstrokecolor{currentstroke}%
\pgfsetdash{}{0pt}%
\pgfsys@defobject{currentmarker}{\pgfqpoint{0.000000in}{0.000000in}}{\pgfqpoint{0.048611in}{0.000000in}}{%
\pgfpathmoveto{\pgfqpoint{0.000000in}{0.000000in}}%
\pgfpathlineto{\pgfqpoint{0.048611in}{0.000000in}}%
\pgfusepath{stroke,fill}%
}%
\begin{pgfscope}%
\pgfsys@transformshift{4.530000in}{3.063158in}%
\pgfsys@useobject{currentmarker}{}%
\end{pgfscope}%
\end{pgfscope}%
\begin{pgfscope}%
\pgfsetbuttcap%
\pgfsetroundjoin%
\definecolor{currentfill}{rgb}{0.000000,0.000000,0.000000}%
\pgfsetfillcolor{currentfill}%
\pgfsetlinewidth{0.803000pt}%
\definecolor{currentstroke}{rgb}{0.000000,0.000000,0.000000}%
\pgfsetstrokecolor{currentstroke}%
\pgfsetdash{}{0pt}%
\pgfsys@defobject{currentmarker}{\pgfqpoint{0.000000in}{0.000000in}}{\pgfqpoint{0.048611in}{0.000000in}}{%
\pgfpathmoveto{\pgfqpoint{0.000000in}{0.000000in}}%
\pgfpathlineto{\pgfqpoint{0.048611in}{0.000000in}}%
\pgfusepath{stroke,fill}%
}%
\begin{pgfscope}%
\pgfsys@transformshift{4.530000in}{3.576316in}%
\pgfsys@useobject{currentmarker}{}%
\end{pgfscope}%
\end{pgfscope}%
\begin{pgfscope}%
\pgfsetbuttcap%
\pgfsetroundjoin%
\definecolor{currentfill}{rgb}{0.000000,0.000000,0.000000}%
\pgfsetfillcolor{currentfill}%
\pgfsetlinewidth{0.803000pt}%
\definecolor{currentstroke}{rgb}{0.000000,0.000000,0.000000}%
\pgfsetstrokecolor{currentstroke}%
\pgfsetdash{}{0pt}%
\pgfsys@defobject{currentmarker}{\pgfqpoint{0.000000in}{0.000000in}}{\pgfqpoint{0.048611in}{0.000000in}}{%
\pgfpathmoveto{\pgfqpoint{0.000000in}{0.000000in}}%
\pgfpathlineto{\pgfqpoint{0.048611in}{0.000000in}}%
\pgfusepath{stroke,fill}%
}%
\begin{pgfscope}%
\pgfsys@transformshift{4.530000in}{4.089474in}%
\pgfsys@useobject{currentmarker}{}%
\end{pgfscope}%
\end{pgfscope}%
\begin{pgfscope}%
\pgfsetrectcap%
\pgfsetmiterjoin%
\pgfsetlinewidth{0.803000pt}%
\definecolor{currentstroke}{rgb}{0.000000,0.000000,0.000000}%
\pgfsetstrokecolor{currentstroke}%
\pgfsetdash{}{0pt}%
\pgfpathmoveto{\pgfqpoint{4.530000in}{0.600000in}}%
\pgfpathlineto{\pgfqpoint{4.530000in}{4.500000in}}%
\pgfusepath{stroke}%
\end{pgfscope}%
\begin{pgfscope}%
\pgfsetrectcap%
\pgfsetmiterjoin%
\pgfsetlinewidth{0.803000pt}%
\definecolor{currentstroke}{rgb}{0.000000,0.000000,0.000000}%
\pgfsetstrokecolor{currentstroke}%
\pgfsetdash{}{0pt}%
\pgfpathmoveto{\pgfqpoint{5.730000in}{0.600000in}}%
\pgfpathlineto{\pgfqpoint{5.730000in}{4.500000in}}%
\pgfusepath{stroke}%
\end{pgfscope}%
\begin{pgfscope}%
\pgfsetrectcap%
\pgfsetmiterjoin%
\pgfsetlinewidth{0.803000pt}%
\definecolor{currentstroke}{rgb}{0.000000,0.000000,0.000000}%
\pgfsetstrokecolor{currentstroke}%
\pgfsetdash{}{0pt}%
\pgfpathmoveto{\pgfqpoint{4.530000in}{0.600000in}}%
\pgfpathlineto{\pgfqpoint{5.730000in}{0.600000in}}%
\pgfusepath{stroke}%
\end{pgfscope}%
\begin{pgfscope}%
\pgfsetrectcap%
\pgfsetmiterjoin%
\pgfsetlinewidth{0.803000pt}%
\definecolor{currentstroke}{rgb}{0.000000,0.000000,0.000000}%
\pgfsetstrokecolor{currentstroke}%
\pgfsetdash{}{0pt}%
\pgfpathmoveto{\pgfqpoint{4.530000in}{4.500000in}}%
\pgfpathlineto{\pgfqpoint{5.730000in}{4.500000in}}%
\pgfusepath{stroke}%
\end{pgfscope}%
\end{pgfpicture}%
\makeatother%
\endgroup%

\caption{Scatter plot of $1000$ pairs of samples from the second algorithm}
\label{Fig:Algo2}
\end{figure}

For this problem, all the algorithms are implemented in C. To be exact, the layout is
\begin{partlist}
\item \verb"rand/gauss_box.c": Box--Muller method;
\item \verb"rand/gauss_rej.c": Rejection based method.
\end{partlist}
We write Python wrappers for C functions in \verb"rand/wrappers.c" and we invoke Python packages to summarize the numerical results and generate figures. The visualization code is placed in \verb"Problem0.py". We use \verb"icc" instead of \verb"gcc" for the compiler by default.

\textbf{Problem (Page 247 Coding Exercise 4).} We consider the 2-D and 3-D lattice Ising model for the project. We implement Metropolis--Hastings MCMC (Markov chain Monte Carlo) method to study the phase transition in the 2-D case and further implmement the kinetic Monte Carlo method for butter convergence and to study both the 2-D and 3-D case. We estimate the change of internal energy, specific heat and magnetization as indicators of phase transition. We invoke RNG (random number generator) routines from Intel MKL (math kernel library). We deploy OpenMP to introduce parallelism and compile the C source codes with \verb"icc" (Intel C compiler) to speed up the computation. The discussion are presented in the following sections.

\section{Ising model}

The Ising model corresponds to a graph $G$, where edges connecting several vertices, named sites, with value $ \sigma_i = \pm 1 $ on them. The Hamiltonian is
\begin{equation}
H \rbr{\sigma} = - J \sum_{ \rbr{ i, j } \in E \rbr{G} } \sigma_i \sigma_j - h \sum_{ i \in V \rbr{G} } \sigma_i.
\end{equation}
We consider the Gibbs measure under this Hamiltonian, say the \emph{discrete} probability measure $\mu$ satisfying
\begin{equation}
\sd \mu \rbr{\cbr{\sigma}} = \frac{1}{Z} \sum_{\sigma} \exp \rbr{ -\beta H \rbr{\sigma} }
\end{equation}
with the normalizing constant $Z$ (partition function) and
\begin{equation}
\beta = \frac{1}{ k_{\text{B}} T }
\end{equation}
where $ k_{\text{B}} = 1 $ and $T$ is the temperature.

The 2-D Ising model with $ N \times N $ square lattice and the 3-D one with $ N \times N \times N $ are considered in the following sections. The boundary conditions are set periodically.

The Gibbs measure depicts the probability of configurations under a specified temperature $T$. Hence, we may extract average or say statistical values from the measure, which reflects the nature of the system. To be exact, in the 2-D case, the internal energy is defined as
\begin{equation}
u = \frac{1}{\abs{ V \rbr{G} }} U = \frac{1}{\abs{ V \rbr{G} }} \int H \rbr{\sigma} \sd \mu \rbr{\sigma}
\end{equation}
and the specific heat is defined as
\begin{equation}
\begin{split}
c &= \frac{1}{\abs{ V \rbr{G} }} C = \frac{1}{\abs{ V \rbr{G} }} \int \rbr{ H \rbr{\sigma} - U }^2 \sd \mu \rbr{\sigma} \\
&= \frac{1}{\abs{ V \rbr{G} }} \rbr{ \int H^2 \rbr{\sigma} \sd \mu \rbr{\sigma} - \rbr{ \int H \rbr{\sigma} \sd \mu \rbr{\sigma} }^2 }.
\end{split}
\end{equation}
The magnetization represents the order and is calculated by
\begin{equation}
m = \frac{1}{\abs{ V \rbr{G} }} M = \frac{1}{\abs{ V \rbr{G} }} \int \sum_{ i \in V \rbr{G} } \abs{\sigma_i} \sd \mu \rbr{\sigma}.
\end{equation}

There is a critical temperature $T_{\text{c}}$ and the quantities above have some asymptotic behavior for $T$ near $T_{\text{c}}$.

We finally plot a figure to show the phase transition in Ising models. We use the Metropolis algorithm introduced in Section \ref{Sec:Alg} and make 8 simulations under given temperature. We take $ \mathit{ITER} = 10^8 $. Here $N$ is set to be $32$. The figure of sites is shown in Figure \ref{Fig:Sites}.

\begin{figure}[htbp]
\centering
\scalebox{0.75}
{%% Creator: Matplotlib, PGF backend
%%
%% To include the figure in your LaTeX document, write
%%   \input{<filename>.pgf}
%%
%% Make sure the required packages are loaded in your preamble
%%   \usepackage{pgf}
%%
%% Figures using additional raster images can only be included by \input if
%% they are in the same directory as the main LaTeX file. For loading figures
%% from other directories you can use the `import` package
%%   \usepackage{import}
%% and then include the figures with
%%   \import{<path to file>}{<filename>.pgf}
%%
%% Matplotlib used the following preamble
%%   \usepackage{fontspec}
%%   \setmainfont{DejaVuSerif.ttf}[Path=/home/lzh/anaconda3/envs/numana/lib/python3.7/site-packages/matplotlib/mpl-data/fonts/ttf/]
%%   \setsansfont{DejaVuSans.ttf}[Path=/home/lzh/anaconda3/envs/numana/lib/python3.7/site-packages/matplotlib/mpl-data/fonts/ttf/]
%%   \setmonofont{DejaVuSansMono.ttf}[Path=/home/lzh/anaconda3/envs/numana/lib/python3.7/site-packages/matplotlib/mpl-data/fonts/ttf/]
%%
\begingroup%
\makeatletter%
\begin{pgfpicture}%
\pgfpathrectangle{\pgfpointorigin}{\pgfqpoint{8.000000in}{11.000000in}}%
\pgfusepath{use as bounding box, clip}%
\begin{pgfscope}%
\pgfsetbuttcap%
\pgfsetmiterjoin%
\definecolor{currentfill}{rgb}{1.000000,1.000000,1.000000}%
\pgfsetfillcolor{currentfill}%
\pgfsetlinewidth{0.000000pt}%
\definecolor{currentstroke}{rgb}{1.000000,1.000000,1.000000}%
\pgfsetstrokecolor{currentstroke}%
\pgfsetdash{}{0pt}%
\pgfpathmoveto{\pgfqpoint{0.000000in}{0.000000in}}%
\pgfpathlineto{\pgfqpoint{8.000000in}{0.000000in}}%
\pgfpathlineto{\pgfqpoint{8.000000in}{11.000000in}}%
\pgfpathlineto{\pgfqpoint{0.000000in}{11.000000in}}%
\pgfpathclose%
\pgfusepath{fill}%
\end{pgfscope}%
\begin{pgfscope}%
\pgfpathrectangle{\pgfqpoint{1.000000in}{8.786611in}}{\pgfqpoint{0.885714in}{0.885714in}}%
\pgfusepath{clip}%
\pgfsys@transformshift{1.000000in}{8.786611in}%
\pgftext[left,bottom]{\pgfimage[interpolate=true,width=0.890000in,height=0.890000in]{Figure01-img0.png}}%
\end{pgfscope}%
\begin{pgfscope}%
\definecolor{textcolor}{rgb}{0.000000,0.000000,0.000000}%
\pgfsetstrokecolor{textcolor}%
\pgfsetfillcolor{textcolor}%
\pgftext[x=1.442857in,y=9.755659in,,base]{\color{textcolor}\sffamily\fontsize{12.000000}{14.400000}\selectfont \(\displaystyle  T = 1.8 \)}%
\end{pgfscope}%
\begin{pgfscope}%
\pgfpathrectangle{\pgfqpoint{1.000000in}{7.705334in}}{\pgfqpoint{0.885714in}{0.885714in}}%
\pgfusepath{clip}%
\pgfsys@transformshift{1.000000in}{7.705334in}%
\pgftext[left,bottom]{\pgfimage[interpolate=true,width=0.890000in,height=0.890000in]{Figure01-img1.png}}%
\end{pgfscope}%
\begin{pgfscope}%
\pgfpathrectangle{\pgfqpoint{1.000000in}{6.624058in}}{\pgfqpoint{0.885714in}{0.885714in}}%
\pgfusepath{clip}%
\pgfsys@transformshift{1.000000in}{6.624058in}%
\pgftext[left,bottom]{\pgfimage[interpolate=true,width=0.890000in,height=0.890000in]{Figure01-img2.png}}%
\end{pgfscope}%
\begin{pgfscope}%
\pgfpathrectangle{\pgfqpoint{1.000000in}{5.542781in}}{\pgfqpoint{0.885714in}{0.885714in}}%
\pgfusepath{clip}%
\pgfsys@transformshift{1.000000in}{5.542781in}%
\pgftext[left,bottom]{\pgfimage[interpolate=true,width=0.890000in,height=0.890000in]{Figure01-img3.png}}%
\end{pgfscope}%
\begin{pgfscope}%
\pgfpathrectangle{\pgfqpoint{1.000000in}{4.461505in}}{\pgfqpoint{0.885714in}{0.885714in}}%
\pgfusepath{clip}%
\pgfsys@transformshift{1.000000in}{4.461505in}%
\pgftext[left,bottom]{\pgfimage[interpolate=true,width=0.890000in,height=0.890000in]{Figure01-img4.png}}%
\end{pgfscope}%
\begin{pgfscope}%
\pgfpathrectangle{\pgfqpoint{1.000000in}{3.380228in}}{\pgfqpoint{0.885714in}{0.885714in}}%
\pgfusepath{clip}%
\pgfsys@transformshift{1.000000in}{3.380228in}%
\pgftext[left,bottom]{\pgfimage[interpolate=true,width=0.890000in,height=0.890000in]{Figure01-img5.png}}%
\end{pgfscope}%
\begin{pgfscope}%
\pgfpathrectangle{\pgfqpoint{1.000000in}{2.298951in}}{\pgfqpoint{0.885714in}{0.885714in}}%
\pgfusepath{clip}%
\pgfsys@transformshift{1.000000in}{2.298951in}%
\pgftext[left,bottom]{\pgfimage[interpolate=true,width=0.890000in,height=0.890000in]{Figure01-img6.png}}%
\end{pgfscope}%
\begin{pgfscope}%
\pgfpathrectangle{\pgfqpoint{1.000000in}{1.217675in}}{\pgfqpoint{0.885714in}{0.885714in}}%
\pgfusepath{clip}%
\pgfsys@transformshift{1.000000in}{1.217675in}%
\pgftext[left,bottom]{\pgfimage[interpolate=true,width=0.890000in,height=0.890000in]{Figure01-img7.png}}%
\end{pgfscope}%
\begin{pgfscope}%
\pgfpathrectangle{\pgfqpoint{2.062857in}{8.786611in}}{\pgfqpoint{0.885714in}{0.885714in}}%
\pgfusepath{clip}%
\pgfsys@transformshift{2.062857in}{8.786611in}%
\pgftext[left,bottom]{\pgfimage[interpolate=true,width=0.890000in,height=0.890000in]{Figure01-img8.png}}%
\end{pgfscope}%
\begin{pgfscope}%
\definecolor{textcolor}{rgb}{0.000000,0.000000,0.000000}%
\pgfsetstrokecolor{textcolor}%
\pgfsetfillcolor{textcolor}%
\pgftext[x=2.505714in,y=9.755659in,,base]{\color{textcolor}\sffamily\fontsize{12.000000}{14.400000}\selectfont \(\displaystyle  T = 2.0 \)}%
\end{pgfscope}%
\begin{pgfscope}%
\pgfpathrectangle{\pgfqpoint{2.062857in}{7.705334in}}{\pgfqpoint{0.885714in}{0.885714in}}%
\pgfusepath{clip}%
\pgfsys@transformshift{2.062857in}{7.705334in}%
\pgftext[left,bottom]{\pgfimage[interpolate=true,width=0.890000in,height=0.890000in]{Figure01-img9.png}}%
\end{pgfscope}%
\begin{pgfscope}%
\pgfpathrectangle{\pgfqpoint{2.062857in}{6.624058in}}{\pgfqpoint{0.885714in}{0.885714in}}%
\pgfusepath{clip}%
\pgfsys@transformshift{2.062857in}{6.624058in}%
\pgftext[left,bottom]{\pgfimage[interpolate=true,width=0.890000in,height=0.890000in]{Figure01-img10.png}}%
\end{pgfscope}%
\begin{pgfscope}%
\pgfpathrectangle{\pgfqpoint{2.062857in}{5.542781in}}{\pgfqpoint{0.885714in}{0.885714in}}%
\pgfusepath{clip}%
\pgfsys@transformshift{2.062857in}{5.542781in}%
\pgftext[left,bottom]{\pgfimage[interpolate=true,width=0.890000in,height=0.890000in]{Figure01-img11.png}}%
\end{pgfscope}%
\begin{pgfscope}%
\pgfpathrectangle{\pgfqpoint{2.062857in}{4.461505in}}{\pgfqpoint{0.885714in}{0.885714in}}%
\pgfusepath{clip}%
\pgfsys@transformshift{2.062857in}{4.461505in}%
\pgftext[left,bottom]{\pgfimage[interpolate=true,width=0.890000in,height=0.890000in]{Figure01-img12.png}}%
\end{pgfscope}%
\begin{pgfscope}%
\pgfpathrectangle{\pgfqpoint{2.062857in}{3.380228in}}{\pgfqpoint{0.885714in}{0.885714in}}%
\pgfusepath{clip}%
\pgfsys@transformshift{2.062857in}{3.380228in}%
\pgftext[left,bottom]{\pgfimage[interpolate=true,width=0.890000in,height=0.890000in]{Figure01-img13.png}}%
\end{pgfscope}%
\begin{pgfscope}%
\pgfpathrectangle{\pgfqpoint{2.062857in}{2.298951in}}{\pgfqpoint{0.885714in}{0.885714in}}%
\pgfusepath{clip}%
\pgfsys@transformshift{2.062857in}{2.298951in}%
\pgftext[left,bottom]{\pgfimage[interpolate=true,width=0.890000in,height=0.890000in]{Figure01-img14.png}}%
\end{pgfscope}%
\begin{pgfscope}%
\pgfpathrectangle{\pgfqpoint{2.062857in}{1.217675in}}{\pgfqpoint{0.885714in}{0.885714in}}%
\pgfusepath{clip}%
\pgfsys@transformshift{2.062857in}{1.217675in}%
\pgftext[left,bottom]{\pgfimage[interpolate=true,width=0.890000in,height=0.890000in]{Figure01-img15.png}}%
\end{pgfscope}%
\begin{pgfscope}%
\pgfpathrectangle{\pgfqpoint{3.125714in}{8.786611in}}{\pgfqpoint{0.885714in}{0.885714in}}%
\pgfusepath{clip}%
\pgfsys@transformshift{3.125714in}{8.786611in}%
\pgftext[left,bottom]{\pgfimage[interpolate=true,width=0.890000in,height=0.890000in]{Figure01-img16.png}}%
\end{pgfscope}%
\begin{pgfscope}%
\definecolor{textcolor}{rgb}{0.000000,0.000000,0.000000}%
\pgfsetstrokecolor{textcolor}%
\pgfsetfillcolor{textcolor}%
\pgftext[x=3.568571in,y=9.755659in,,base]{\color{textcolor}\sffamily\fontsize{12.000000}{14.400000}\selectfont \(\displaystyle  T = 2.2 \)}%
\end{pgfscope}%
\begin{pgfscope}%
\pgfpathrectangle{\pgfqpoint{3.125714in}{7.705334in}}{\pgfqpoint{0.885714in}{0.885714in}}%
\pgfusepath{clip}%
\pgfsys@transformshift{3.125714in}{7.705334in}%
\pgftext[left,bottom]{\pgfimage[interpolate=true,width=0.890000in,height=0.890000in]{Figure01-img17.png}}%
\end{pgfscope}%
\begin{pgfscope}%
\pgfpathrectangle{\pgfqpoint{3.125714in}{6.624058in}}{\pgfqpoint{0.885714in}{0.885714in}}%
\pgfusepath{clip}%
\pgfsys@transformshift{3.125714in}{6.624058in}%
\pgftext[left,bottom]{\pgfimage[interpolate=true,width=0.890000in,height=0.890000in]{Figure01-img18.png}}%
\end{pgfscope}%
\begin{pgfscope}%
\pgfpathrectangle{\pgfqpoint{3.125714in}{5.542781in}}{\pgfqpoint{0.885714in}{0.885714in}}%
\pgfusepath{clip}%
\pgfsys@transformshift{3.125714in}{5.542781in}%
\pgftext[left,bottom]{\pgfimage[interpolate=true,width=0.890000in,height=0.890000in]{Figure01-img19.png}}%
\end{pgfscope}%
\begin{pgfscope}%
\pgfpathrectangle{\pgfqpoint{3.125714in}{4.461505in}}{\pgfqpoint{0.885714in}{0.885714in}}%
\pgfusepath{clip}%
\pgfsys@transformshift{3.125714in}{4.461505in}%
\pgftext[left,bottom]{\pgfimage[interpolate=true,width=0.890000in,height=0.890000in]{Figure01-img20.png}}%
\end{pgfscope}%
\begin{pgfscope}%
\pgfpathrectangle{\pgfqpoint{3.125714in}{3.380228in}}{\pgfqpoint{0.885714in}{0.885714in}}%
\pgfusepath{clip}%
\pgfsys@transformshift{3.125714in}{3.380228in}%
\pgftext[left,bottom]{\pgfimage[interpolate=true,width=0.890000in,height=0.890000in]{Figure01-img21.png}}%
\end{pgfscope}%
\begin{pgfscope}%
\pgfpathrectangle{\pgfqpoint{3.125714in}{2.298951in}}{\pgfqpoint{0.885714in}{0.885714in}}%
\pgfusepath{clip}%
\pgfsys@transformshift{3.125714in}{2.298951in}%
\pgftext[left,bottom]{\pgfimage[interpolate=true,width=0.890000in,height=0.890000in]{Figure01-img22.png}}%
\end{pgfscope}%
\begin{pgfscope}%
\pgfpathrectangle{\pgfqpoint{3.125714in}{1.217675in}}{\pgfqpoint{0.885714in}{0.885714in}}%
\pgfusepath{clip}%
\pgfsys@transformshift{3.125714in}{1.217675in}%
\pgftext[left,bottom]{\pgfimage[interpolate=true,width=0.890000in,height=0.890000in]{Figure01-img23.png}}%
\end{pgfscope}%
\begin{pgfscope}%
\pgfpathrectangle{\pgfqpoint{4.188571in}{8.786611in}}{\pgfqpoint{0.885714in}{0.885714in}}%
\pgfusepath{clip}%
\pgfsys@transformshift{4.188571in}{8.786611in}%
\pgftext[left,bottom]{\pgfimage[interpolate=true,width=0.890000in,height=0.890000in]{Figure01-img24.png}}%
\end{pgfscope}%
\begin{pgfscope}%
\definecolor{textcolor}{rgb}{0.000000,0.000000,0.000000}%
\pgfsetstrokecolor{textcolor}%
\pgfsetfillcolor{textcolor}%
\pgftext[x=4.631429in,y=9.755659in,,base]{\color{textcolor}\sffamily\fontsize{12.000000}{14.400000}\selectfont \(\displaystyle  T = 2.4 \)}%
\end{pgfscope}%
\begin{pgfscope}%
\pgfpathrectangle{\pgfqpoint{4.188571in}{7.705334in}}{\pgfqpoint{0.885714in}{0.885714in}}%
\pgfusepath{clip}%
\pgfsys@transformshift{4.188571in}{7.705334in}%
\pgftext[left,bottom]{\pgfimage[interpolate=true,width=0.890000in,height=0.890000in]{Figure01-img25.png}}%
\end{pgfscope}%
\begin{pgfscope}%
\pgfpathrectangle{\pgfqpoint{4.188571in}{6.624058in}}{\pgfqpoint{0.885714in}{0.885714in}}%
\pgfusepath{clip}%
\pgfsys@transformshift{4.188571in}{6.624058in}%
\pgftext[left,bottom]{\pgfimage[interpolate=true,width=0.890000in,height=0.890000in]{Figure01-img26.png}}%
\end{pgfscope}%
\begin{pgfscope}%
\pgfpathrectangle{\pgfqpoint{4.188571in}{5.542781in}}{\pgfqpoint{0.885714in}{0.885714in}}%
\pgfusepath{clip}%
\pgfsys@transformshift{4.188571in}{5.542781in}%
\pgftext[left,bottom]{\pgfimage[interpolate=true,width=0.890000in,height=0.890000in]{Figure01-img27.png}}%
\end{pgfscope}%
\begin{pgfscope}%
\pgfpathrectangle{\pgfqpoint{4.188571in}{4.461505in}}{\pgfqpoint{0.885714in}{0.885714in}}%
\pgfusepath{clip}%
\pgfsys@transformshift{4.188571in}{4.461505in}%
\pgftext[left,bottom]{\pgfimage[interpolate=true,width=0.890000in,height=0.890000in]{Figure01-img28.png}}%
\end{pgfscope}%
\begin{pgfscope}%
\pgfpathrectangle{\pgfqpoint{4.188571in}{3.380228in}}{\pgfqpoint{0.885714in}{0.885714in}}%
\pgfusepath{clip}%
\pgfsys@transformshift{4.188571in}{3.380228in}%
\pgftext[left,bottom]{\pgfimage[interpolate=true,width=0.890000in,height=0.890000in]{Figure01-img29.png}}%
\end{pgfscope}%
\begin{pgfscope}%
\pgfpathrectangle{\pgfqpoint{4.188571in}{2.298951in}}{\pgfqpoint{0.885714in}{0.885714in}}%
\pgfusepath{clip}%
\pgfsys@transformshift{4.188571in}{2.298951in}%
\pgftext[left,bottom]{\pgfimage[interpolate=true,width=0.890000in,height=0.890000in]{Figure01-img30.png}}%
\end{pgfscope}%
\begin{pgfscope}%
\pgfpathrectangle{\pgfqpoint{4.188571in}{1.217675in}}{\pgfqpoint{0.885714in}{0.885714in}}%
\pgfusepath{clip}%
\pgfsys@transformshift{4.188571in}{1.217675in}%
\pgftext[left,bottom]{\pgfimage[interpolate=true,width=0.890000in,height=0.890000in]{Figure01-img31.png}}%
\end{pgfscope}%
\begin{pgfscope}%
\pgfpathrectangle{\pgfqpoint{5.251429in}{8.786611in}}{\pgfqpoint{0.885714in}{0.885714in}}%
\pgfusepath{clip}%
\pgfsys@transformshift{5.251429in}{8.786611in}%
\pgftext[left,bottom]{\pgfimage[interpolate=true,width=0.890000in,height=0.890000in]{Figure01-img32.png}}%
\end{pgfscope}%
\begin{pgfscope}%
\definecolor{textcolor}{rgb}{0.000000,0.000000,0.000000}%
\pgfsetstrokecolor{textcolor}%
\pgfsetfillcolor{textcolor}%
\pgftext[x=5.694286in,y=9.755659in,,base]{\color{textcolor}\sffamily\fontsize{12.000000}{14.400000}\selectfont \(\displaystyle  T = 2.6 \)}%
\end{pgfscope}%
\begin{pgfscope}%
\pgfpathrectangle{\pgfqpoint{5.251429in}{7.705334in}}{\pgfqpoint{0.885714in}{0.885714in}}%
\pgfusepath{clip}%
\pgfsys@transformshift{5.251429in}{7.705334in}%
\pgftext[left,bottom]{\pgfimage[interpolate=true,width=0.890000in,height=0.890000in]{Figure01-img33.png}}%
\end{pgfscope}%
\begin{pgfscope}%
\pgfpathrectangle{\pgfqpoint{5.251429in}{6.624058in}}{\pgfqpoint{0.885714in}{0.885714in}}%
\pgfusepath{clip}%
\pgfsys@transformshift{5.251429in}{6.624058in}%
\pgftext[left,bottom]{\pgfimage[interpolate=true,width=0.890000in,height=0.890000in]{Figure01-img34.png}}%
\end{pgfscope}%
\begin{pgfscope}%
\pgfpathrectangle{\pgfqpoint{5.251429in}{5.542781in}}{\pgfqpoint{0.885714in}{0.885714in}}%
\pgfusepath{clip}%
\pgfsys@transformshift{5.251429in}{5.542781in}%
\pgftext[left,bottom]{\pgfimage[interpolate=true,width=0.890000in,height=0.890000in]{Figure01-img35.png}}%
\end{pgfscope}%
\begin{pgfscope}%
\pgfpathrectangle{\pgfqpoint{5.251429in}{4.461505in}}{\pgfqpoint{0.885714in}{0.885714in}}%
\pgfusepath{clip}%
\pgfsys@transformshift{5.251429in}{4.461505in}%
\pgftext[left,bottom]{\pgfimage[interpolate=true,width=0.890000in,height=0.890000in]{Figure01-img36.png}}%
\end{pgfscope}%
\begin{pgfscope}%
\pgfpathrectangle{\pgfqpoint{5.251429in}{3.380228in}}{\pgfqpoint{0.885714in}{0.885714in}}%
\pgfusepath{clip}%
\pgfsys@transformshift{5.251429in}{3.380228in}%
\pgftext[left,bottom]{\pgfimage[interpolate=true,width=0.890000in,height=0.890000in]{Figure01-img37.png}}%
\end{pgfscope}%
\begin{pgfscope}%
\pgfpathrectangle{\pgfqpoint{5.251429in}{2.298951in}}{\pgfqpoint{0.885714in}{0.885714in}}%
\pgfusepath{clip}%
\pgfsys@transformshift{5.251429in}{2.298951in}%
\pgftext[left,bottom]{\pgfimage[interpolate=true,width=0.890000in,height=0.890000in]{Figure01-img38.png}}%
\end{pgfscope}%
\begin{pgfscope}%
\pgfpathrectangle{\pgfqpoint{5.251429in}{1.217675in}}{\pgfqpoint{0.885714in}{0.885714in}}%
\pgfusepath{clip}%
\pgfsys@transformshift{5.251429in}{1.217675in}%
\pgftext[left,bottom]{\pgfimage[interpolate=true,width=0.890000in,height=0.890000in]{Figure01-img39.png}}%
\end{pgfscope}%
\begin{pgfscope}%
\pgfpathrectangle{\pgfqpoint{6.314286in}{8.786611in}}{\pgfqpoint{0.885714in}{0.885714in}}%
\pgfusepath{clip}%
\pgfsys@transformshift{6.314286in}{8.786611in}%
\pgftext[left,bottom]{\pgfimage[interpolate=true,width=0.890000in,height=0.890000in]{Figure01-img40.png}}%
\end{pgfscope}%
\begin{pgfscope}%
\definecolor{textcolor}{rgb}{0.000000,0.000000,0.000000}%
\pgfsetstrokecolor{textcolor}%
\pgfsetfillcolor{textcolor}%
\pgftext[x=6.757143in,y=9.755659in,,base]{\color{textcolor}\sffamily\fontsize{12.000000}{14.400000}\selectfont \(\displaystyle  T = 2.8 \)}%
\end{pgfscope}%
\begin{pgfscope}%
\pgfpathrectangle{\pgfqpoint{6.314286in}{7.705334in}}{\pgfqpoint{0.885714in}{0.885714in}}%
\pgfusepath{clip}%
\pgfsys@transformshift{6.314286in}{7.705334in}%
\pgftext[left,bottom]{\pgfimage[interpolate=true,width=0.890000in,height=0.890000in]{Figure01-img41.png}}%
\end{pgfscope}%
\begin{pgfscope}%
\pgfpathrectangle{\pgfqpoint{6.314286in}{6.624058in}}{\pgfqpoint{0.885714in}{0.885714in}}%
\pgfusepath{clip}%
\pgfsys@transformshift{6.314286in}{6.624058in}%
\pgftext[left,bottom]{\pgfimage[interpolate=true,width=0.890000in,height=0.890000in]{Figure01-img42.png}}%
\end{pgfscope}%
\begin{pgfscope}%
\pgfpathrectangle{\pgfqpoint{6.314286in}{5.542781in}}{\pgfqpoint{0.885714in}{0.885714in}}%
\pgfusepath{clip}%
\pgfsys@transformshift{6.314286in}{5.542781in}%
\pgftext[left,bottom]{\pgfimage[interpolate=true,width=0.890000in,height=0.890000in]{Figure01-img43.png}}%
\end{pgfscope}%
\begin{pgfscope}%
\pgfpathrectangle{\pgfqpoint{6.314286in}{4.461505in}}{\pgfqpoint{0.885714in}{0.885714in}}%
\pgfusepath{clip}%
\pgfsys@transformshift{6.314286in}{4.461505in}%
\pgftext[left,bottom]{\pgfimage[interpolate=true,width=0.890000in,height=0.890000in]{Figure01-img44.png}}%
\end{pgfscope}%
\begin{pgfscope}%
\pgfpathrectangle{\pgfqpoint{6.314286in}{3.380228in}}{\pgfqpoint{0.885714in}{0.885714in}}%
\pgfusepath{clip}%
\pgfsys@transformshift{6.314286in}{3.380228in}%
\pgftext[left,bottom]{\pgfimage[interpolate=true,width=0.890000in,height=0.890000in]{Figure01-img45.png}}%
\end{pgfscope}%
\begin{pgfscope}%
\pgfpathrectangle{\pgfqpoint{6.314286in}{2.298951in}}{\pgfqpoint{0.885714in}{0.885714in}}%
\pgfusepath{clip}%
\pgfsys@transformshift{6.314286in}{2.298951in}%
\pgftext[left,bottom]{\pgfimage[interpolate=true,width=0.890000in,height=0.890000in]{Figure01-img46.png}}%
\end{pgfscope}%
\begin{pgfscope}%
\pgfpathrectangle{\pgfqpoint{6.314286in}{1.217675in}}{\pgfqpoint{0.885714in}{0.885714in}}%
\pgfusepath{clip}%
\pgfsys@transformshift{6.314286in}{1.217675in}%
\pgftext[left,bottom]{\pgfimage[interpolate=true,width=0.890000in,height=0.890000in]{Figure01-img47.png}}%
\end{pgfscope}%
\end{pgfpicture}%
\makeatother%
\endgroup%
}
\caption{Figure of sites}
\label{Fig:Sites}
\end{figure}

One may directly observe the transition from $ T = 1.8 $ to $ T = 2.8 $. When $ T = 1.8 $, the system are highly correlated and there are only single sites different from others. But when $ T = 2.8 $ the system is not so self-correlated and there are random patterns. The sharpest transformation perceptually is at $ 2.2 \le T \le 2.4 $. This figure provides us some intuition about the system at different temperatures.

\section{Markov Chain Monte Carlo algorithms} \label{Sec:Alg}

\subsection{Metropolis--Hastings algorithm}

In order to sample the Gibbs measure $\mu$, we apply Markov Chain Monte Carlo algorithm here. We first deploy the standard Metropolis--Hastings algorithm: the site of proposal is picked randomly from $ V \rbr{G} $ with uniform probability, and then the acceptance probability is determined by
\begin{equation}
A = \min \cbr{ \exp \rbr{ -\beta \Delta H }, 1 }.
\end{equation}
In practice, $ \Delta H $ is computed directly from the site $\sigma_i$ itself and its four or six neighbors. We sample a random number $r$ from $ \mathcal{U} \sbr{ 0, 1 } $ and decide to accept the transition if $ r < A $. Due to efficiency reasons, we apply the random number generator routines in Intel MKL to do this job. It follows from ergodicity that $\sigma$ will converge to the Gibbs distribution, which is the unique invariant distribution of the Markov Chain.

For find the value of the quantities, we take average from the trajectory, and then take average from different trajectories. To be exact, if we simulate the trajectory for $\mathit{ITER}$ iterations, and then repeat for $\mathit{TRAJ}$ trajectories, the final estimation of some value of the form
\begin{equation}
\int F \rbr{\sigma} \sd \mu \rbr{\sigma}
\end{equation}
is
\begin{equation} \label{Eq:Ave}
\hat{F} = \frac{1}{\mathit{TRAJ}} \sum_{ \textit{tr} = 1 }^{\mathit{TRAJ}} \hat{F}_{\textit{tr}},
\end{equation}
where
\begin{equation}
\hat{F}_{\textit{tr}} = \frac{1}{ \mathit{END} - \mathit{STRAT} } \sum_{ \textit{it} = \mathit{START} + 1 }^{\mathit{END}} F \rbr{\sigma_{\mathit{tr}}^{\mathit{it}}} 
\end{equation}
where $ \sigma_{\mathit{tr}}^{\mathit{it}} $ stands at the $\mathit{it}$-th iteration and the $\mathit{tr}$-th trajectory. The standard deviation can also be estimated using
\begin{equation}
\hat{S} = \sqrt{ \frac{1}{\mathit{TRAJ}^2} \rbr{ \sum_{ \mathit{tr} = 1 }^{\mathit{TRAJ}} \hat{F}_{\textit{tr}}^2 - \rbr{ \sum_{ \mathit{tr} = 1 }^{\mathit{TRAJ}} \hat{F}_{\textit{tr}} }^2 } }.
\end{equation}
Of course we constrain $ 1 \le \mathit{START} < \mathit{END} \le \mathit{ITER} $. We set $\mathit{START}$ here to ignore some of the configurations in the beginning to reduce bias, since the Markov Chain converges to equilibrium only after a number of steps. We use OpenMP to perform parallel sampling and therefore uses multiple trajectories. Again ergodicity yields convergence.

To observe its convergence, we take $ N = 16, 32, 64, 128 $ in the 2-D case and then check the curve of estimated $m$. We set $ \mathit{START} = \fbr{ \mathit{END} / 3 } $ and $ \mathit{END} = \mathit{END} $ and $ \mathit{TRAJ} = 4 $ since the machine we use have four cores. The figure is given in Figure \ref{Fig:Metro}. The shaded region stands for $ \hat{F} \pm 3 \hat{S} $ in correspondence to the $ 3 \sigma $ principle and we will use this convention all through the report. Here $ J = 1 $ and $ h = 0 $.

\begin{figure}[htbp]
\centering
\scalebox{0.666}{%% Creator: Matplotlib, PGF backend
%%
%% To include the figure in your LaTeX document, write
%%   \input{<filename>.pgf}
%%
%% Make sure the required packages are loaded in your preamble
%%   \usepackage{pgf}
%%
%% Figures using additional raster images can only be included by \input if
%% they are in the same directory as the main LaTeX file. For loading figures
%% from other directories you can use the `import` package
%%   \usepackage{import}
%% and then include the figures with
%%   \import{<path to file>}{<filename>.pgf}
%%
%% Matplotlib used the following preamble
%%   \usepackage{fontspec}
%%   \setmainfont{DejaVuSerif.ttf}[Path=/home/lzh/anaconda3/envs/numana/lib/python3.7/site-packages/matplotlib/mpl-data/fonts/ttf/]
%%   \setsansfont{DejaVuSans.ttf}[Path=/home/lzh/anaconda3/envs/numana/lib/python3.7/site-packages/matplotlib/mpl-data/fonts/ttf/]
%%   \setmonofont{DejaVuSansMono.ttf}[Path=/home/lzh/anaconda3/envs/numana/lib/python3.7/site-packages/matplotlib/mpl-data/fonts/ttf/]
%%
\begingroup%
\makeatletter%
\begin{pgfpicture}%
\pgfpathrectangle{\pgfpointorigin}{\pgfqpoint{8.000000in}{6.000000in}}%
\pgfusepath{use as bounding box, clip}%
\begin{pgfscope}%
\pgfsetbuttcap%
\pgfsetmiterjoin%
\definecolor{currentfill}{rgb}{1.000000,1.000000,1.000000}%
\pgfsetfillcolor{currentfill}%
\pgfsetlinewidth{0.000000pt}%
\definecolor{currentstroke}{rgb}{1.000000,1.000000,1.000000}%
\pgfsetstrokecolor{currentstroke}%
\pgfsetdash{}{0pt}%
\pgfpathmoveto{\pgfqpoint{0.000000in}{0.000000in}}%
\pgfpathlineto{\pgfqpoint{8.000000in}{0.000000in}}%
\pgfpathlineto{\pgfqpoint{8.000000in}{6.000000in}}%
\pgfpathlineto{\pgfqpoint{0.000000in}{6.000000in}}%
\pgfpathclose%
\pgfusepath{fill}%
\end{pgfscope}%
\begin{pgfscope}%
\pgfsetbuttcap%
\pgfsetmiterjoin%
\definecolor{currentfill}{rgb}{1.000000,1.000000,1.000000}%
\pgfsetfillcolor{currentfill}%
\pgfsetlinewidth{0.000000pt}%
\definecolor{currentstroke}{rgb}{0.000000,0.000000,0.000000}%
\pgfsetstrokecolor{currentstroke}%
\pgfsetstrokeopacity{0.000000}%
\pgfsetdash{}{0pt}%
\pgfpathmoveto{\pgfqpoint{0.557986in}{3.363889in}}%
\pgfpathlineto{\pgfqpoint{3.876389in}{3.363889in}}%
\pgfpathlineto{\pgfqpoint{3.876389in}{5.627778in}}%
\pgfpathlineto{\pgfqpoint{0.557986in}{5.627778in}}%
\pgfpathclose%
\pgfusepath{fill}%
\end{pgfscope}%
\begin{pgfscope}%
\pgfpathrectangle{\pgfqpoint{0.557986in}{3.363889in}}{\pgfqpoint{3.318403in}{2.263889in}}%
\pgfusepath{clip}%
\pgfsetbuttcap%
\pgfsetroundjoin%
\definecolor{currentfill}{rgb}{0.121569,0.466667,0.705882}%
\pgfsetfillcolor{currentfill}%
\pgfsetlinewidth{1.003750pt}%
\definecolor{currentstroke}{rgb}{0.121569,0.466667,0.705882}%
\pgfsetstrokecolor{currentstroke}%
\pgfsetdash{}{0pt}%
\pgfsys@defobject{currentmarker}{\pgfqpoint{-0.009821in}{-0.009821in}}{\pgfqpoint{0.009821in}{0.009821in}}{%
\pgfpathmoveto{\pgfqpoint{0.000000in}{-0.009821in}}%
\pgfpathcurveto{\pgfqpoint{0.002605in}{-0.009821in}}{\pgfqpoint{0.005103in}{-0.008786in}}{\pgfqpoint{0.006944in}{-0.006944in}}%
\pgfpathcurveto{\pgfqpoint{0.008786in}{-0.005103in}}{\pgfqpoint{0.009821in}{-0.002605in}}{\pgfqpoint{0.009821in}{0.000000in}}%
\pgfpathcurveto{\pgfqpoint{0.009821in}{0.002605in}}{\pgfqpoint{0.008786in}{0.005103in}}{\pgfqpoint{0.006944in}{0.006944in}}%
\pgfpathcurveto{\pgfqpoint{0.005103in}{0.008786in}}{\pgfqpoint{0.002605in}{0.009821in}}{\pgfqpoint{0.000000in}{0.009821in}}%
\pgfpathcurveto{\pgfqpoint{-0.002605in}{0.009821in}}{\pgfqpoint{-0.005103in}{0.008786in}}{\pgfqpoint{-0.006944in}{0.006944in}}%
\pgfpathcurveto{\pgfqpoint{-0.008786in}{0.005103in}}{\pgfqpoint{-0.009821in}{0.002605in}}{\pgfqpoint{-0.009821in}{0.000000in}}%
\pgfpathcurveto{\pgfqpoint{-0.009821in}{-0.002605in}}{\pgfqpoint{-0.008786in}{-0.005103in}}{\pgfqpoint{-0.006944in}{-0.006944in}}%
\pgfpathcurveto{\pgfqpoint{-0.005103in}{-0.008786in}}{\pgfqpoint{-0.002605in}{-0.009821in}}{\pgfqpoint{0.000000in}{-0.009821in}}%
\pgfpathclose%
\pgfusepath{stroke,fill}%
}%
\begin{pgfscope}%
\pgfsys@transformshift{0.708823in}{3.839424in}%
\pgfsys@useobject{currentmarker}{}%
\end{pgfscope}%
\begin{pgfscope}%
\pgfsys@transformshift{0.859609in}{3.895280in}%
\pgfsys@useobject{currentmarker}{}%
\end{pgfscope}%
\begin{pgfscope}%
\pgfsys@transformshift{1.010395in}{4.628915in}%
\pgfsys@useobject{currentmarker}{}%
\end{pgfscope}%
\begin{pgfscope}%
\pgfsys@transformshift{1.161181in}{4.801410in}%
\pgfsys@useobject{currentmarker}{}%
\end{pgfscope}%
\begin{pgfscope}%
\pgfsys@transformshift{1.311967in}{4.655646in}%
\pgfsys@useobject{currentmarker}{}%
\end{pgfscope}%
\begin{pgfscope}%
\pgfsys@transformshift{1.462753in}{4.916088in}%
\pgfsys@useobject{currentmarker}{}%
\end{pgfscope}%
\begin{pgfscope}%
\pgfsys@transformshift{1.613539in}{4.879613in}%
\pgfsys@useobject{currentmarker}{}%
\end{pgfscope}%
\begin{pgfscope}%
\pgfsys@transformshift{1.764325in}{4.690595in}%
\pgfsys@useobject{currentmarker}{}%
\end{pgfscope}%
\begin{pgfscope}%
\pgfsys@transformshift{1.915111in}{4.999268in}%
\pgfsys@useobject{currentmarker}{}%
\end{pgfscope}%
\begin{pgfscope}%
\pgfsys@transformshift{2.065897in}{5.011569in}%
\pgfsys@useobject{currentmarker}{}%
\end{pgfscope}%
\begin{pgfscope}%
\pgfsys@transformshift{2.216683in}{5.011487in}%
\pgfsys@useobject{currentmarker}{}%
\end{pgfscope}%
\begin{pgfscope}%
\pgfsys@transformshift{2.367469in}{5.011527in}%
\pgfsys@useobject{currentmarker}{}%
\end{pgfscope}%
\begin{pgfscope}%
\pgfsys@transformshift{2.518255in}{5.011511in}%
\pgfsys@useobject{currentmarker}{}%
\end{pgfscope}%
\begin{pgfscope}%
\pgfsys@transformshift{2.669041in}{5.011527in}%
\pgfsys@useobject{currentmarker}{}%
\end{pgfscope}%
\begin{pgfscope}%
\pgfsys@transformshift{2.819827in}{5.011524in}%
\pgfsys@useobject{currentmarker}{}%
\end{pgfscope}%
\begin{pgfscope}%
\pgfsys@transformshift{2.970613in}{5.011547in}%
\pgfsys@useobject{currentmarker}{}%
\end{pgfscope}%
\begin{pgfscope}%
\pgfsys@transformshift{3.121399in}{5.011541in}%
\pgfsys@useobject{currentmarker}{}%
\end{pgfscope}%
\begin{pgfscope}%
\pgfsys@transformshift{3.272185in}{5.011555in}%
\pgfsys@useobject{currentmarker}{}%
\end{pgfscope}%
\begin{pgfscope}%
\pgfsys@transformshift{3.422971in}{5.011543in}%
\pgfsys@useobject{currentmarker}{}%
\end{pgfscope}%
\begin{pgfscope}%
\pgfsys@transformshift{3.573756in}{5.011551in}%
\pgfsys@useobject{currentmarker}{}%
\end{pgfscope}%
\begin{pgfscope}%
\pgfsys@transformshift{3.724542in}{5.011551in}%
\pgfsys@useobject{currentmarker}{}%
\end{pgfscope}%
\end{pgfscope}%
\begin{pgfscope}%
\pgfpathrectangle{\pgfqpoint{0.557986in}{3.363889in}}{\pgfqpoint{3.318403in}{2.263889in}}%
\pgfusepath{clip}%
\pgfsetbuttcap%
\pgfsetroundjoin%
\definecolor{currentfill}{rgb}{0.121569,0.466667,0.705882}%
\pgfsetfillcolor{currentfill}%
\pgfsetfillopacity{0.300000}%
\pgfsetlinewidth{1.003750pt}%
\definecolor{currentstroke}{rgb}{0.121569,0.466667,0.705882}%
\pgfsetstrokecolor{currentstroke}%
\pgfsetstrokeopacity{0.300000}%
\pgfsetdash{}{0pt}%
\pgfpathmoveto{\pgfqpoint{0.708823in}{4.129527in}}%
\pgfpathlineto{\pgfqpoint{0.708823in}{3.549321in}}%
\pgfpathlineto{\pgfqpoint{0.859609in}{3.594130in}}%
\pgfpathlineto{\pgfqpoint{1.010395in}{4.125324in}}%
\pgfpathlineto{\pgfqpoint{1.161181in}{4.410400in}}%
\pgfpathlineto{\pgfqpoint{1.311967in}{3.787348in}}%
\pgfpathlineto{\pgfqpoint{1.462753in}{4.668186in}}%
\pgfpathlineto{\pgfqpoint{1.613539in}{4.536131in}}%
\pgfpathlineto{\pgfqpoint{1.764325in}{3.856317in}}%
\pgfpathlineto{\pgfqpoint{1.915111in}{4.966961in}}%
\pgfpathlineto{\pgfqpoint{2.065897in}{5.011400in}}%
\pgfpathlineto{\pgfqpoint{2.216683in}{5.011340in}}%
\pgfpathlineto{\pgfqpoint{2.367469in}{5.011483in}}%
\pgfpathlineto{\pgfqpoint{2.518255in}{5.011446in}}%
\pgfpathlineto{\pgfqpoint{2.669041in}{5.011474in}}%
\pgfpathlineto{\pgfqpoint{2.819827in}{5.011462in}}%
\pgfpathlineto{\pgfqpoint{2.970613in}{5.011511in}}%
\pgfpathlineto{\pgfqpoint{3.121399in}{5.011523in}}%
\pgfpathlineto{\pgfqpoint{3.272185in}{5.011542in}}%
\pgfpathlineto{\pgfqpoint{3.422971in}{5.011517in}}%
\pgfpathlineto{\pgfqpoint{3.573756in}{5.011537in}}%
\pgfpathlineto{\pgfqpoint{3.724542in}{5.011546in}}%
\pgfpathlineto{\pgfqpoint{3.724542in}{5.011557in}}%
\pgfpathlineto{\pgfqpoint{3.724542in}{5.011557in}}%
\pgfpathlineto{\pgfqpoint{3.573756in}{5.011564in}}%
\pgfpathlineto{\pgfqpoint{3.422971in}{5.011569in}}%
\pgfpathlineto{\pgfqpoint{3.272185in}{5.011569in}}%
\pgfpathlineto{\pgfqpoint{3.121399in}{5.011559in}}%
\pgfpathlineto{\pgfqpoint{2.970613in}{5.011582in}}%
\pgfpathlineto{\pgfqpoint{2.819827in}{5.011587in}}%
\pgfpathlineto{\pgfqpoint{2.669041in}{5.011580in}}%
\pgfpathlineto{\pgfqpoint{2.518255in}{5.011575in}}%
\pgfpathlineto{\pgfqpoint{2.367469in}{5.011572in}}%
\pgfpathlineto{\pgfqpoint{2.216683in}{5.011634in}}%
\pgfpathlineto{\pgfqpoint{2.065897in}{5.011739in}}%
\pgfpathlineto{\pgfqpoint{1.915111in}{5.031575in}}%
\pgfpathlineto{\pgfqpoint{1.764325in}{5.524874in}}%
\pgfpathlineto{\pgfqpoint{1.613539in}{5.223096in}}%
\pgfpathlineto{\pgfqpoint{1.462753in}{5.163991in}}%
\pgfpathlineto{\pgfqpoint{1.311967in}{5.523945in}}%
\pgfpathlineto{\pgfqpoint{1.161181in}{5.192420in}}%
\pgfpathlineto{\pgfqpoint{1.010395in}{5.132505in}}%
\pgfpathlineto{\pgfqpoint{0.859609in}{4.196430in}}%
\pgfpathlineto{\pgfqpoint{0.708823in}{4.129527in}}%
\pgfpathclose%
\pgfusepath{stroke,fill}%
\end{pgfscope}%
\begin{pgfscope}%
\pgfpathrectangle{\pgfqpoint{0.557986in}{3.363889in}}{\pgfqpoint{3.318403in}{2.263889in}}%
\pgfusepath{clip}%
\pgfsetbuttcap%
\pgfsetroundjoin%
\definecolor{currentfill}{rgb}{1.000000,0.498039,0.054902}%
\pgfsetfillcolor{currentfill}%
\pgfsetlinewidth{1.003750pt}%
\definecolor{currentstroke}{rgb}{1.000000,0.498039,0.054902}%
\pgfsetstrokecolor{currentstroke}%
\pgfsetdash{}{0pt}%
\pgfsys@defobject{currentmarker}{\pgfqpoint{-0.009821in}{-0.009821in}}{\pgfqpoint{0.009821in}{0.009821in}}{%
\pgfpathmoveto{\pgfqpoint{0.000000in}{-0.009821in}}%
\pgfpathcurveto{\pgfqpoint{0.002605in}{-0.009821in}}{\pgfqpoint{0.005103in}{-0.008786in}}{\pgfqpoint{0.006944in}{-0.006944in}}%
\pgfpathcurveto{\pgfqpoint{0.008786in}{-0.005103in}}{\pgfqpoint{0.009821in}{-0.002605in}}{\pgfqpoint{0.009821in}{0.000000in}}%
\pgfpathcurveto{\pgfqpoint{0.009821in}{0.002605in}}{\pgfqpoint{0.008786in}{0.005103in}}{\pgfqpoint{0.006944in}{0.006944in}}%
\pgfpathcurveto{\pgfqpoint{0.005103in}{0.008786in}}{\pgfqpoint{0.002605in}{0.009821in}}{\pgfqpoint{0.000000in}{0.009821in}}%
\pgfpathcurveto{\pgfqpoint{-0.002605in}{0.009821in}}{\pgfqpoint{-0.005103in}{0.008786in}}{\pgfqpoint{-0.006944in}{0.006944in}}%
\pgfpathcurveto{\pgfqpoint{-0.008786in}{0.005103in}}{\pgfqpoint{-0.009821in}{0.002605in}}{\pgfqpoint{-0.009821in}{0.000000in}}%
\pgfpathcurveto{\pgfqpoint{-0.009821in}{-0.002605in}}{\pgfqpoint{-0.008786in}{-0.005103in}}{\pgfqpoint{-0.006944in}{-0.006944in}}%
\pgfpathcurveto{\pgfqpoint{-0.005103in}{-0.008786in}}{\pgfqpoint{-0.002605in}{-0.009821in}}{\pgfqpoint{0.000000in}{-0.009821in}}%
\pgfpathclose%
\pgfusepath{stroke,fill}%
}%
\begin{pgfscope}%
\pgfsys@transformshift{0.708823in}{3.900071in}%
\pgfsys@useobject{currentmarker}{}%
\end{pgfscope}%
\begin{pgfscope}%
\pgfsys@transformshift{0.859609in}{4.010608in}%
\pgfsys@useobject{currentmarker}{}%
\end{pgfscope}%
\begin{pgfscope}%
\pgfsys@transformshift{1.010395in}{4.198230in}%
\pgfsys@useobject{currentmarker}{}%
\end{pgfscope}%
\begin{pgfscope}%
\pgfsys@transformshift{1.161181in}{4.861184in}%
\pgfsys@useobject{currentmarker}{}%
\end{pgfscope}%
\begin{pgfscope}%
\pgfsys@transformshift{1.311967in}{4.504774in}%
\pgfsys@useobject{currentmarker}{}%
\end{pgfscope}%
\begin{pgfscope}%
\pgfsys@transformshift{1.462753in}{4.662495in}%
\pgfsys@useobject{currentmarker}{}%
\end{pgfscope}%
\begin{pgfscope}%
\pgfsys@transformshift{1.613539in}{4.686331in}%
\pgfsys@useobject{currentmarker}{}%
\end{pgfscope}%
\begin{pgfscope}%
\pgfsys@transformshift{1.764325in}{4.993549in}%
\pgfsys@useobject{currentmarker}{}%
\end{pgfscope}%
\begin{pgfscope}%
\pgfsys@transformshift{1.915111in}{4.992256in}%
\pgfsys@useobject{currentmarker}{}%
\end{pgfscope}%
\begin{pgfscope}%
\pgfsys@transformshift{2.065897in}{4.992991in}%
\pgfsys@useobject{currentmarker}{}%
\end{pgfscope}%
\begin{pgfscope}%
\pgfsys@transformshift{2.216683in}{4.993445in}%
\pgfsys@useobject{currentmarker}{}%
\end{pgfscope}%
\begin{pgfscope}%
\pgfsys@transformshift{2.367469in}{4.992858in}%
\pgfsys@useobject{currentmarker}{}%
\end{pgfscope}%
\begin{pgfscope}%
\pgfsys@transformshift{2.518255in}{4.992895in}%
\pgfsys@useobject{currentmarker}{}%
\end{pgfscope}%
\begin{pgfscope}%
\pgfsys@transformshift{2.669041in}{4.992941in}%
\pgfsys@useobject{currentmarker}{}%
\end{pgfscope}%
\begin{pgfscope}%
\pgfsys@transformshift{2.819827in}{4.992881in}%
\pgfsys@useobject{currentmarker}{}%
\end{pgfscope}%
\begin{pgfscope}%
\pgfsys@transformshift{2.970613in}{4.992896in}%
\pgfsys@useobject{currentmarker}{}%
\end{pgfscope}%
\begin{pgfscope}%
\pgfsys@transformshift{3.121399in}{4.992917in}%
\pgfsys@useobject{currentmarker}{}%
\end{pgfscope}%
\begin{pgfscope}%
\pgfsys@transformshift{3.272185in}{4.992870in}%
\pgfsys@useobject{currentmarker}{}%
\end{pgfscope}%
\begin{pgfscope}%
\pgfsys@transformshift{3.422971in}{4.992896in}%
\pgfsys@useobject{currentmarker}{}%
\end{pgfscope}%
\begin{pgfscope}%
\pgfsys@transformshift{3.573756in}{4.992926in}%
\pgfsys@useobject{currentmarker}{}%
\end{pgfscope}%
\begin{pgfscope}%
\pgfsys@transformshift{3.724542in}{4.992870in}%
\pgfsys@useobject{currentmarker}{}%
\end{pgfscope}%
\end{pgfscope}%
\begin{pgfscope}%
\pgfpathrectangle{\pgfqpoint{0.557986in}{3.363889in}}{\pgfqpoint{3.318403in}{2.263889in}}%
\pgfusepath{clip}%
\pgfsetbuttcap%
\pgfsetroundjoin%
\definecolor{currentfill}{rgb}{1.000000,0.498039,0.054902}%
\pgfsetfillcolor{currentfill}%
\pgfsetfillopacity{0.300000}%
\pgfsetlinewidth{1.003750pt}%
\definecolor{currentstroke}{rgb}{1.000000,0.498039,0.054902}%
\pgfsetstrokecolor{currentstroke}%
\pgfsetstrokeopacity{0.300000}%
\pgfsetdash{}{0pt}%
\pgfpathmoveto{\pgfqpoint{0.708823in}{4.290679in}}%
\pgfpathlineto{\pgfqpoint{0.708823in}{3.509462in}}%
\pgfpathlineto{\pgfqpoint{0.859609in}{3.704856in}}%
\pgfpathlineto{\pgfqpoint{1.010395in}{3.653418in}}%
\pgfpathlineto{\pgfqpoint{1.161181in}{4.670864in}}%
\pgfpathlineto{\pgfqpoint{1.311967in}{3.778879in}}%
\pgfpathlineto{\pgfqpoint{1.462753in}{3.843370in}}%
\pgfpathlineto{\pgfqpoint{1.613539in}{3.889822in}}%
\pgfpathlineto{\pgfqpoint{1.764325in}{4.991041in}}%
\pgfpathlineto{\pgfqpoint{1.915111in}{4.991633in}}%
\pgfpathlineto{\pgfqpoint{2.065897in}{4.992001in}}%
\pgfpathlineto{\pgfqpoint{2.216683in}{4.992974in}}%
\pgfpathlineto{\pgfqpoint{2.367469in}{4.992336in}}%
\pgfpathlineto{\pgfqpoint{2.518255in}{4.992180in}}%
\pgfpathlineto{\pgfqpoint{2.669041in}{4.992686in}}%
\pgfpathlineto{\pgfqpoint{2.819827in}{4.992500in}}%
\pgfpathlineto{\pgfqpoint{2.970613in}{4.992681in}}%
\pgfpathlineto{\pgfqpoint{3.121399in}{4.992764in}}%
\pgfpathlineto{\pgfqpoint{3.272185in}{4.992798in}}%
\pgfpathlineto{\pgfqpoint{3.422971in}{4.992797in}}%
\pgfpathlineto{\pgfqpoint{3.573756in}{4.992876in}}%
\pgfpathlineto{\pgfqpoint{3.724542in}{4.992847in}}%
\pgfpathlineto{\pgfqpoint{3.724542in}{4.992894in}}%
\pgfpathlineto{\pgfqpoint{3.724542in}{4.992894in}}%
\pgfpathlineto{\pgfqpoint{3.573756in}{4.992976in}}%
\pgfpathlineto{\pgfqpoint{3.422971in}{4.992995in}}%
\pgfpathlineto{\pgfqpoint{3.272185in}{4.992942in}}%
\pgfpathlineto{\pgfqpoint{3.121399in}{4.993070in}}%
\pgfpathlineto{\pgfqpoint{2.970613in}{4.993111in}}%
\pgfpathlineto{\pgfqpoint{2.819827in}{4.993263in}}%
\pgfpathlineto{\pgfqpoint{2.669041in}{4.993195in}}%
\pgfpathlineto{\pgfqpoint{2.518255in}{4.993611in}}%
\pgfpathlineto{\pgfqpoint{2.367469in}{4.993380in}}%
\pgfpathlineto{\pgfqpoint{2.216683in}{4.993916in}}%
\pgfpathlineto{\pgfqpoint{2.065897in}{4.993981in}}%
\pgfpathlineto{\pgfqpoint{1.915111in}{4.992879in}}%
\pgfpathlineto{\pgfqpoint{1.764325in}{4.996057in}}%
\pgfpathlineto{\pgfqpoint{1.613539in}{5.482841in}}%
\pgfpathlineto{\pgfqpoint{1.462753in}{5.481620in}}%
\pgfpathlineto{\pgfqpoint{1.311967in}{5.230670in}}%
\pgfpathlineto{\pgfqpoint{1.161181in}{5.051504in}}%
\pgfpathlineto{\pgfqpoint{1.010395in}{4.743042in}}%
\pgfpathlineto{\pgfqpoint{0.859609in}{4.316360in}}%
\pgfpathlineto{\pgfqpoint{0.708823in}{4.290679in}}%
\pgfpathclose%
\pgfusepath{stroke,fill}%
\end{pgfscope}%
\begin{pgfscope}%
\pgfpathrectangle{\pgfqpoint{0.557986in}{3.363889in}}{\pgfqpoint{3.318403in}{2.263889in}}%
\pgfusepath{clip}%
\pgfsetbuttcap%
\pgfsetroundjoin%
\definecolor{currentfill}{rgb}{0.172549,0.627451,0.172549}%
\pgfsetfillcolor{currentfill}%
\pgfsetlinewidth{1.003750pt}%
\definecolor{currentstroke}{rgb}{0.172549,0.627451,0.172549}%
\pgfsetstrokecolor{currentstroke}%
\pgfsetdash{}{0pt}%
\pgfsys@defobject{currentmarker}{\pgfqpoint{-0.009821in}{-0.009821in}}{\pgfqpoint{0.009821in}{0.009821in}}{%
\pgfpathmoveto{\pgfqpoint{0.000000in}{-0.009821in}}%
\pgfpathcurveto{\pgfqpoint{0.002605in}{-0.009821in}}{\pgfqpoint{0.005103in}{-0.008786in}}{\pgfqpoint{0.006944in}{-0.006944in}}%
\pgfpathcurveto{\pgfqpoint{0.008786in}{-0.005103in}}{\pgfqpoint{0.009821in}{-0.002605in}}{\pgfqpoint{0.009821in}{0.000000in}}%
\pgfpathcurveto{\pgfqpoint{0.009821in}{0.002605in}}{\pgfqpoint{0.008786in}{0.005103in}}{\pgfqpoint{0.006944in}{0.006944in}}%
\pgfpathcurveto{\pgfqpoint{0.005103in}{0.008786in}}{\pgfqpoint{0.002605in}{0.009821in}}{\pgfqpoint{0.000000in}{0.009821in}}%
\pgfpathcurveto{\pgfqpoint{-0.002605in}{0.009821in}}{\pgfqpoint{-0.005103in}{0.008786in}}{\pgfqpoint{-0.006944in}{0.006944in}}%
\pgfpathcurveto{\pgfqpoint{-0.008786in}{0.005103in}}{\pgfqpoint{-0.009821in}{0.002605in}}{\pgfqpoint{-0.009821in}{0.000000in}}%
\pgfpathcurveto{\pgfqpoint{-0.009821in}{-0.002605in}}{\pgfqpoint{-0.008786in}{-0.005103in}}{\pgfqpoint{-0.006944in}{-0.006944in}}%
\pgfpathcurveto{\pgfqpoint{-0.005103in}{-0.008786in}}{\pgfqpoint{-0.002605in}{-0.009821in}}{\pgfqpoint{0.000000in}{-0.009821in}}%
\pgfpathclose%
\pgfusepath{stroke,fill}%
}%
\begin{pgfscope}%
\pgfsys@transformshift{0.708823in}{3.885386in}%
\pgfsys@useobject{currentmarker}{}%
\end{pgfscope}%
\begin{pgfscope}%
\pgfsys@transformshift{0.859609in}{3.928319in}%
\pgfsys@useobject{currentmarker}{}%
\end{pgfscope}%
\begin{pgfscope}%
\pgfsys@transformshift{1.010395in}{3.944716in}%
\pgfsys@useobject{currentmarker}{}%
\end{pgfscope}%
\begin{pgfscope}%
\pgfsys@transformshift{1.161181in}{4.209986in}%
\pgfsys@useobject{currentmarker}{}%
\end{pgfscope}%
\begin{pgfscope}%
\pgfsys@transformshift{1.311967in}{4.561953in}%
\pgfsys@useobject{currentmarker}{}%
\end{pgfscope}%
\begin{pgfscope}%
\pgfsys@transformshift{1.462753in}{4.621749in}%
\pgfsys@useobject{currentmarker}{}%
\end{pgfscope}%
\begin{pgfscope}%
\pgfsys@transformshift{1.613539in}{4.886221in}%
\pgfsys@useobject{currentmarker}{}%
\end{pgfscope}%
\begin{pgfscope}%
\pgfsys@transformshift{1.764325in}{4.889795in}%
\pgfsys@useobject{currentmarker}{}%
\end{pgfscope}%
\begin{pgfscope}%
\pgfsys@transformshift{1.915111in}{4.881060in}%
\pgfsys@useobject{currentmarker}{}%
\end{pgfscope}%
\begin{pgfscope}%
\pgfsys@transformshift{2.065897in}{4.886999in}%
\pgfsys@useobject{currentmarker}{}%
\end{pgfscope}%
\begin{pgfscope}%
\pgfsys@transformshift{2.216683in}{4.887890in}%
\pgfsys@useobject{currentmarker}{}%
\end{pgfscope}%
\begin{pgfscope}%
\pgfsys@transformshift{2.367469in}{4.879207in}%
\pgfsys@useobject{currentmarker}{}%
\end{pgfscope}%
\begin{pgfscope}%
\pgfsys@transformshift{2.518255in}{4.883891in}%
\pgfsys@useobject{currentmarker}{}%
\end{pgfscope}%
\begin{pgfscope}%
\pgfsys@transformshift{2.669041in}{4.884801in}%
\pgfsys@useobject{currentmarker}{}%
\end{pgfscope}%
\begin{pgfscope}%
\pgfsys@transformshift{2.819827in}{4.882169in}%
\pgfsys@useobject{currentmarker}{}%
\end{pgfscope}%
\begin{pgfscope}%
\pgfsys@transformshift{2.970613in}{4.882493in}%
\pgfsys@useobject{currentmarker}{}%
\end{pgfscope}%
\begin{pgfscope}%
\pgfsys@transformshift{3.121399in}{4.882825in}%
\pgfsys@useobject{currentmarker}{}%
\end{pgfscope}%
\begin{pgfscope}%
\pgfsys@transformshift{3.272185in}{4.882584in}%
\pgfsys@useobject{currentmarker}{}%
\end{pgfscope}%
\begin{pgfscope}%
\pgfsys@transformshift{3.422971in}{4.883307in}%
\pgfsys@useobject{currentmarker}{}%
\end{pgfscope}%
\begin{pgfscope}%
\pgfsys@transformshift{3.573756in}{4.883094in}%
\pgfsys@useobject{currentmarker}{}%
\end{pgfscope}%
\begin{pgfscope}%
\pgfsys@transformshift{3.724542in}{4.883098in}%
\pgfsys@useobject{currentmarker}{}%
\end{pgfscope}%
\end{pgfscope}%
\begin{pgfscope}%
\pgfpathrectangle{\pgfqpoint{0.557986in}{3.363889in}}{\pgfqpoint{3.318403in}{2.263889in}}%
\pgfusepath{clip}%
\pgfsetbuttcap%
\pgfsetroundjoin%
\definecolor{currentfill}{rgb}{0.172549,0.627451,0.172549}%
\pgfsetfillcolor{currentfill}%
\pgfsetfillopacity{0.300000}%
\pgfsetlinewidth{1.003750pt}%
\definecolor{currentstroke}{rgb}{0.172549,0.627451,0.172549}%
\pgfsetstrokecolor{currentstroke}%
\pgfsetstrokeopacity{0.300000}%
\pgfsetdash{}{0pt}%
\pgfpathmoveto{\pgfqpoint{0.708823in}{4.303980in}}%
\pgfpathlineto{\pgfqpoint{0.708823in}{3.466793in}}%
\pgfpathlineto{\pgfqpoint{0.859609in}{3.562580in}}%
\pgfpathlineto{\pgfqpoint{1.010395in}{3.474480in}}%
\pgfpathlineto{\pgfqpoint{1.161181in}{3.629183in}}%
\pgfpathlineto{\pgfqpoint{1.311967in}{3.962912in}}%
\pgfpathlineto{\pgfqpoint{1.462753in}{4.098776in}}%
\pgfpathlineto{\pgfqpoint{1.613539in}{4.867434in}}%
\pgfpathlineto{\pgfqpoint{1.764325in}{4.880684in}}%
\pgfpathlineto{\pgfqpoint{1.915111in}{4.857279in}}%
\pgfpathlineto{\pgfqpoint{2.065897in}{4.881748in}}%
\pgfpathlineto{\pgfqpoint{2.216683in}{4.879704in}}%
\pgfpathlineto{\pgfqpoint{2.367469in}{4.871829in}}%
\pgfpathlineto{\pgfqpoint{2.518255in}{4.879202in}}%
\pgfpathlineto{\pgfqpoint{2.669041in}{4.883792in}}%
\pgfpathlineto{\pgfqpoint{2.819827in}{4.879931in}}%
\pgfpathlineto{\pgfqpoint{2.970613in}{4.880667in}}%
\pgfpathlineto{\pgfqpoint{3.121399in}{4.880628in}}%
\pgfpathlineto{\pgfqpoint{3.272185in}{4.881724in}}%
\pgfpathlineto{\pgfqpoint{3.422971in}{4.882241in}}%
\pgfpathlineto{\pgfqpoint{3.573756in}{4.882439in}}%
\pgfpathlineto{\pgfqpoint{3.724542in}{4.882618in}}%
\pgfpathlineto{\pgfqpoint{3.724542in}{4.883579in}}%
\pgfpathlineto{\pgfqpoint{3.724542in}{4.883579in}}%
\pgfpathlineto{\pgfqpoint{3.573756in}{4.883749in}}%
\pgfpathlineto{\pgfqpoint{3.422971in}{4.884373in}}%
\pgfpathlineto{\pgfqpoint{3.272185in}{4.883443in}}%
\pgfpathlineto{\pgfqpoint{3.121399in}{4.885022in}}%
\pgfpathlineto{\pgfqpoint{2.970613in}{4.884320in}}%
\pgfpathlineto{\pgfqpoint{2.819827in}{4.884407in}}%
\pgfpathlineto{\pgfqpoint{2.669041in}{4.885809in}}%
\pgfpathlineto{\pgfqpoint{2.518255in}{4.888581in}}%
\pgfpathlineto{\pgfqpoint{2.367469in}{4.886584in}}%
\pgfpathlineto{\pgfqpoint{2.216683in}{4.896076in}}%
\pgfpathlineto{\pgfqpoint{2.065897in}{4.892251in}}%
\pgfpathlineto{\pgfqpoint{1.915111in}{4.904842in}}%
\pgfpathlineto{\pgfqpoint{1.764325in}{4.898906in}}%
\pgfpathlineto{\pgfqpoint{1.613539in}{4.905008in}}%
\pgfpathlineto{\pgfqpoint{1.462753in}{5.144722in}}%
\pgfpathlineto{\pgfqpoint{1.311967in}{5.160994in}}%
\pgfpathlineto{\pgfqpoint{1.161181in}{4.790788in}}%
\pgfpathlineto{\pgfqpoint{1.010395in}{4.414952in}}%
\pgfpathlineto{\pgfqpoint{0.859609in}{4.294057in}}%
\pgfpathlineto{\pgfqpoint{0.708823in}{4.303980in}}%
\pgfpathclose%
\pgfusepath{stroke,fill}%
\end{pgfscope}%
\begin{pgfscope}%
\pgfpathrectangle{\pgfqpoint{0.557986in}{3.363889in}}{\pgfqpoint{3.318403in}{2.263889in}}%
\pgfusepath{clip}%
\pgfsetbuttcap%
\pgfsetroundjoin%
\definecolor{currentfill}{rgb}{0.839216,0.152941,0.156863}%
\pgfsetfillcolor{currentfill}%
\pgfsetlinewidth{1.003750pt}%
\definecolor{currentstroke}{rgb}{0.839216,0.152941,0.156863}%
\pgfsetstrokecolor{currentstroke}%
\pgfsetdash{}{0pt}%
\pgfsys@defobject{currentmarker}{\pgfqpoint{-0.009821in}{-0.009821in}}{\pgfqpoint{0.009821in}{0.009821in}}{%
\pgfpathmoveto{\pgfqpoint{0.000000in}{-0.009821in}}%
\pgfpathcurveto{\pgfqpoint{0.002605in}{-0.009821in}}{\pgfqpoint{0.005103in}{-0.008786in}}{\pgfqpoint{0.006944in}{-0.006944in}}%
\pgfpathcurveto{\pgfqpoint{0.008786in}{-0.005103in}}{\pgfqpoint{0.009821in}{-0.002605in}}{\pgfqpoint{0.009821in}{0.000000in}}%
\pgfpathcurveto{\pgfqpoint{0.009821in}{0.002605in}}{\pgfqpoint{0.008786in}{0.005103in}}{\pgfqpoint{0.006944in}{0.006944in}}%
\pgfpathcurveto{\pgfqpoint{0.005103in}{0.008786in}}{\pgfqpoint{0.002605in}{0.009821in}}{\pgfqpoint{0.000000in}{0.009821in}}%
\pgfpathcurveto{\pgfqpoint{-0.002605in}{0.009821in}}{\pgfqpoint{-0.005103in}{0.008786in}}{\pgfqpoint{-0.006944in}{0.006944in}}%
\pgfpathcurveto{\pgfqpoint{-0.008786in}{0.005103in}}{\pgfqpoint{-0.009821in}{0.002605in}}{\pgfqpoint{-0.009821in}{0.000000in}}%
\pgfpathcurveto{\pgfqpoint{-0.009821in}{-0.002605in}}{\pgfqpoint{-0.008786in}{-0.005103in}}{\pgfqpoint{-0.006944in}{-0.006944in}}%
\pgfpathcurveto{\pgfqpoint{-0.005103in}{-0.008786in}}{\pgfqpoint{-0.002605in}{-0.009821in}}{\pgfqpoint{0.000000in}{-0.009821in}}%
\pgfpathclose%
\pgfusepath{stroke,fill}%
}%
\begin{pgfscope}%
\pgfsys@transformshift{0.708823in}{3.769642in}%
\pgfsys@useobject{currentmarker}{}%
\end{pgfscope}%
\begin{pgfscope}%
\pgfsys@transformshift{0.859609in}{4.130453in}%
\pgfsys@useobject{currentmarker}{}%
\end{pgfscope}%
\begin{pgfscope}%
\pgfsys@transformshift{1.010395in}{3.875313in}%
\pgfsys@useobject{currentmarker}{}%
\end{pgfscope}%
\begin{pgfscope}%
\pgfsys@transformshift{1.161181in}{3.988322in}%
\pgfsys@useobject{currentmarker}{}%
\end{pgfscope}%
\begin{pgfscope}%
\pgfsys@transformshift{1.311967in}{4.316108in}%
\pgfsys@useobject{currentmarker}{}%
\end{pgfscope}%
\begin{pgfscope}%
\pgfsys@transformshift{1.462753in}{4.225938in}%
\pgfsys@useobject{currentmarker}{}%
\end{pgfscope}%
\begin{pgfscope}%
\pgfsys@transformshift{1.613539in}{4.033362in}%
\pgfsys@useobject{currentmarker}{}%
\end{pgfscope}%
\begin{pgfscope}%
\pgfsys@transformshift{1.764325in}{4.175058in}%
\pgfsys@useobject{currentmarker}{}%
\end{pgfscope}%
\begin{pgfscope}%
\pgfsys@transformshift{1.915111in}{4.130992in}%
\pgfsys@useobject{currentmarker}{}%
\end{pgfscope}%
\begin{pgfscope}%
\pgfsys@transformshift{2.065897in}{4.113252in}%
\pgfsys@useobject{currentmarker}{}%
\end{pgfscope}%
\begin{pgfscope}%
\pgfsys@transformshift{2.216683in}{4.094158in}%
\pgfsys@useobject{currentmarker}{}%
\end{pgfscope}%
\begin{pgfscope}%
\pgfsys@transformshift{2.367469in}{4.153160in}%
\pgfsys@useobject{currentmarker}{}%
\end{pgfscope}%
\begin{pgfscope}%
\pgfsys@transformshift{2.518255in}{4.115981in}%
\pgfsys@useobject{currentmarker}{}%
\end{pgfscope}%
\begin{pgfscope}%
\pgfsys@transformshift{2.669041in}{4.117308in}%
\pgfsys@useobject{currentmarker}{}%
\end{pgfscope}%
\begin{pgfscope}%
\pgfsys@transformshift{2.819827in}{4.124745in}%
\pgfsys@useobject{currentmarker}{}%
\end{pgfscope}%
\begin{pgfscope}%
\pgfsys@transformshift{2.970613in}{4.113389in}%
\pgfsys@useobject{currentmarker}{}%
\end{pgfscope}%
\begin{pgfscope}%
\pgfsys@transformshift{3.121399in}{4.135437in}%
\pgfsys@useobject{currentmarker}{}%
\end{pgfscope}%
\begin{pgfscope}%
\pgfsys@transformshift{3.272185in}{4.113997in}%
\pgfsys@useobject{currentmarker}{}%
\end{pgfscope}%
\begin{pgfscope}%
\pgfsys@transformshift{3.422971in}{4.120470in}%
\pgfsys@useobject{currentmarker}{}%
\end{pgfscope}%
\begin{pgfscope}%
\pgfsys@transformshift{3.573756in}{4.118492in}%
\pgfsys@useobject{currentmarker}{}%
\end{pgfscope}%
\begin{pgfscope}%
\pgfsys@transformshift{3.724542in}{4.119344in}%
\pgfsys@useobject{currentmarker}{}%
\end{pgfscope}%
\end{pgfscope}%
\begin{pgfscope}%
\pgfpathrectangle{\pgfqpoint{0.557986in}{3.363889in}}{\pgfqpoint{3.318403in}{2.263889in}}%
\pgfusepath{clip}%
\pgfsetbuttcap%
\pgfsetroundjoin%
\definecolor{currentfill}{rgb}{0.839216,0.152941,0.156863}%
\pgfsetfillcolor{currentfill}%
\pgfsetfillopacity{0.300000}%
\pgfsetlinewidth{1.003750pt}%
\definecolor{currentstroke}{rgb}{0.839216,0.152941,0.156863}%
\pgfsetstrokecolor{currentstroke}%
\pgfsetstrokeopacity{0.300000}%
\pgfsetdash{}{0pt}%
\pgfpathmoveto{\pgfqpoint{0.708823in}{3.992717in}}%
\pgfpathlineto{\pgfqpoint{0.708823in}{3.546568in}}%
\pgfpathlineto{\pgfqpoint{0.859609in}{3.784783in}}%
\pgfpathlineto{\pgfqpoint{1.010395in}{3.648220in}}%
\pgfpathlineto{\pgfqpoint{1.161181in}{3.794873in}}%
\pgfpathlineto{\pgfqpoint{1.311967in}{4.079985in}}%
\pgfpathlineto{\pgfqpoint{1.462753in}{3.891717in}}%
\pgfpathlineto{\pgfqpoint{1.613539in}{3.852596in}}%
\pgfpathlineto{\pgfqpoint{1.764325in}{4.082865in}}%
\pgfpathlineto{\pgfqpoint{1.915111in}{4.050095in}}%
\pgfpathlineto{\pgfqpoint{2.065897in}{4.039060in}}%
\pgfpathlineto{\pgfqpoint{2.216683in}{4.048817in}}%
\pgfpathlineto{\pgfqpoint{2.367469in}{4.122324in}}%
\pgfpathlineto{\pgfqpoint{2.518255in}{4.055218in}}%
\pgfpathlineto{\pgfqpoint{2.669041in}{4.089211in}}%
\pgfpathlineto{\pgfqpoint{2.819827in}{4.107676in}}%
\pgfpathlineto{\pgfqpoint{2.970613in}{4.095718in}}%
\pgfpathlineto{\pgfqpoint{3.121399in}{4.118947in}}%
\pgfpathlineto{\pgfqpoint{3.272185in}{4.103953in}}%
\pgfpathlineto{\pgfqpoint{3.422971in}{4.114093in}}%
\pgfpathlineto{\pgfqpoint{3.573756in}{4.114423in}}%
\pgfpathlineto{\pgfqpoint{3.724542in}{4.116187in}}%
\pgfpathlineto{\pgfqpoint{3.724542in}{4.122501in}}%
\pgfpathlineto{\pgfqpoint{3.724542in}{4.122501in}}%
\pgfpathlineto{\pgfqpoint{3.573756in}{4.122561in}}%
\pgfpathlineto{\pgfqpoint{3.422971in}{4.126846in}}%
\pgfpathlineto{\pgfqpoint{3.272185in}{4.124041in}}%
\pgfpathlineto{\pgfqpoint{3.121399in}{4.151927in}}%
\pgfpathlineto{\pgfqpoint{2.970613in}{4.131061in}}%
\pgfpathlineto{\pgfqpoint{2.819827in}{4.141815in}}%
\pgfpathlineto{\pgfqpoint{2.669041in}{4.145404in}}%
\pgfpathlineto{\pgfqpoint{2.518255in}{4.176744in}}%
\pgfpathlineto{\pgfqpoint{2.367469in}{4.183996in}}%
\pgfpathlineto{\pgfqpoint{2.216683in}{4.139498in}}%
\pgfpathlineto{\pgfqpoint{2.065897in}{4.187443in}}%
\pgfpathlineto{\pgfqpoint{1.915111in}{4.211888in}}%
\pgfpathlineto{\pgfqpoint{1.764325in}{4.267250in}}%
\pgfpathlineto{\pgfqpoint{1.613539in}{4.214129in}}%
\pgfpathlineto{\pgfqpoint{1.462753in}{4.560158in}}%
\pgfpathlineto{\pgfqpoint{1.311967in}{4.552230in}}%
\pgfpathlineto{\pgfqpoint{1.161181in}{4.181772in}}%
\pgfpathlineto{\pgfqpoint{1.010395in}{4.102406in}}%
\pgfpathlineto{\pgfqpoint{0.859609in}{4.476123in}}%
\pgfpathlineto{\pgfqpoint{0.708823in}{3.992717in}}%
\pgfpathclose%
\pgfusepath{stroke,fill}%
\end{pgfscope}%
\begin{pgfscope}%
\pgfpathrectangle{\pgfqpoint{0.557986in}{3.363889in}}{\pgfqpoint{3.318403in}{2.263889in}}%
\pgfusepath{clip}%
\pgfsetbuttcap%
\pgfsetroundjoin%
\definecolor{currentfill}{rgb}{0.580392,0.403922,0.741176}%
\pgfsetfillcolor{currentfill}%
\pgfsetlinewidth{1.003750pt}%
\definecolor{currentstroke}{rgb}{0.580392,0.403922,0.741176}%
\pgfsetstrokecolor{currentstroke}%
\pgfsetdash{}{0pt}%
\pgfsys@defobject{currentmarker}{\pgfqpoint{-0.009821in}{-0.009821in}}{\pgfqpoint{0.009821in}{0.009821in}}{%
\pgfpathmoveto{\pgfqpoint{0.000000in}{-0.009821in}}%
\pgfpathcurveto{\pgfqpoint{0.002605in}{-0.009821in}}{\pgfqpoint{0.005103in}{-0.008786in}}{\pgfqpoint{0.006944in}{-0.006944in}}%
\pgfpathcurveto{\pgfqpoint{0.008786in}{-0.005103in}}{\pgfqpoint{0.009821in}{-0.002605in}}{\pgfqpoint{0.009821in}{0.000000in}}%
\pgfpathcurveto{\pgfqpoint{0.009821in}{0.002605in}}{\pgfqpoint{0.008786in}{0.005103in}}{\pgfqpoint{0.006944in}{0.006944in}}%
\pgfpathcurveto{\pgfqpoint{0.005103in}{0.008786in}}{\pgfqpoint{0.002605in}{0.009821in}}{\pgfqpoint{0.000000in}{0.009821in}}%
\pgfpathcurveto{\pgfqpoint{-0.002605in}{0.009821in}}{\pgfqpoint{-0.005103in}{0.008786in}}{\pgfqpoint{-0.006944in}{0.006944in}}%
\pgfpathcurveto{\pgfqpoint{-0.008786in}{0.005103in}}{\pgfqpoint{-0.009821in}{0.002605in}}{\pgfqpoint{-0.009821in}{0.000000in}}%
\pgfpathcurveto{\pgfqpoint{-0.009821in}{-0.002605in}}{\pgfqpoint{-0.008786in}{-0.005103in}}{\pgfqpoint{-0.006944in}{-0.006944in}}%
\pgfpathcurveto{\pgfqpoint{-0.005103in}{-0.008786in}}{\pgfqpoint{-0.002605in}{-0.009821in}}{\pgfqpoint{0.000000in}{-0.009821in}}%
\pgfpathclose%
\pgfusepath{stroke,fill}%
}%
\begin{pgfscope}%
\pgfsys@transformshift{0.708823in}{3.710452in}%
\pgfsys@useobject{currentmarker}{}%
\end{pgfscope}%
\begin{pgfscope}%
\pgfsys@transformshift{0.859609in}{3.891680in}%
\pgfsys@useobject{currentmarker}{}%
\end{pgfscope}%
\begin{pgfscope}%
\pgfsys@transformshift{1.010395in}{3.809931in}%
\pgfsys@useobject{currentmarker}{}%
\end{pgfscope}%
\begin{pgfscope}%
\pgfsys@transformshift{1.161181in}{3.822166in}%
\pgfsys@useobject{currentmarker}{}%
\end{pgfscope}%
\begin{pgfscope}%
\pgfsys@transformshift{1.311967in}{3.828266in}%
\pgfsys@useobject{currentmarker}{}%
\end{pgfscope}%
\begin{pgfscope}%
\pgfsys@transformshift{1.462753in}{3.781314in}%
\pgfsys@useobject{currentmarker}{}%
\end{pgfscope}%
\begin{pgfscope}%
\pgfsys@transformshift{1.613539in}{3.804810in}%
\pgfsys@useobject{currentmarker}{}%
\end{pgfscope}%
\begin{pgfscope}%
\pgfsys@transformshift{1.764325in}{3.803899in}%
\pgfsys@useobject{currentmarker}{}%
\end{pgfscope}%
\begin{pgfscope}%
\pgfsys@transformshift{1.915111in}{3.808713in}%
\pgfsys@useobject{currentmarker}{}%
\end{pgfscope}%
\begin{pgfscope}%
\pgfsys@transformshift{2.065897in}{3.810884in}%
\pgfsys@useobject{currentmarker}{}%
\end{pgfscope}%
\begin{pgfscope}%
\pgfsys@transformshift{2.216683in}{3.798051in}%
\pgfsys@useobject{currentmarker}{}%
\end{pgfscope}%
\begin{pgfscope}%
\pgfsys@transformshift{2.367469in}{3.794654in}%
\pgfsys@useobject{currentmarker}{}%
\end{pgfscope}%
\begin{pgfscope}%
\pgfsys@transformshift{2.518255in}{3.806475in}%
\pgfsys@useobject{currentmarker}{}%
\end{pgfscope}%
\begin{pgfscope}%
\pgfsys@transformshift{2.669041in}{3.797010in}%
\pgfsys@useobject{currentmarker}{}%
\end{pgfscope}%
\begin{pgfscope}%
\pgfsys@transformshift{2.819827in}{3.802214in}%
\pgfsys@useobject{currentmarker}{}%
\end{pgfscope}%
\begin{pgfscope}%
\pgfsys@transformshift{2.970613in}{3.805823in}%
\pgfsys@useobject{currentmarker}{}%
\end{pgfscope}%
\begin{pgfscope}%
\pgfsys@transformshift{3.121399in}{3.799601in}%
\pgfsys@useobject{currentmarker}{}%
\end{pgfscope}%
\begin{pgfscope}%
\pgfsys@transformshift{3.272185in}{3.800664in}%
\pgfsys@useobject{currentmarker}{}%
\end{pgfscope}%
\begin{pgfscope}%
\pgfsys@transformshift{3.422971in}{3.800410in}%
\pgfsys@useobject{currentmarker}{}%
\end{pgfscope}%
\begin{pgfscope}%
\pgfsys@transformshift{3.573756in}{3.801209in}%
\pgfsys@useobject{currentmarker}{}%
\end{pgfscope}%
\begin{pgfscope}%
\pgfsys@transformshift{3.724542in}{3.801497in}%
\pgfsys@useobject{currentmarker}{}%
\end{pgfscope}%
\end{pgfscope}%
\begin{pgfscope}%
\pgfpathrectangle{\pgfqpoint{0.557986in}{3.363889in}}{\pgfqpoint{3.318403in}{2.263889in}}%
\pgfusepath{clip}%
\pgfsetbuttcap%
\pgfsetroundjoin%
\definecolor{currentfill}{rgb}{0.580392,0.403922,0.741176}%
\pgfsetfillcolor{currentfill}%
\pgfsetfillopacity{0.300000}%
\pgfsetlinewidth{1.003750pt}%
\definecolor{currentstroke}{rgb}{0.580392,0.403922,0.741176}%
\pgfsetstrokecolor{currentstroke}%
\pgfsetstrokeopacity{0.300000}%
\pgfsetdash{}{0pt}%
\pgfpathmoveto{\pgfqpoint{0.708823in}{3.900924in}}%
\pgfpathlineto{\pgfqpoint{0.708823in}{3.519981in}}%
\pgfpathlineto{\pgfqpoint{0.859609in}{3.591683in}}%
\pgfpathlineto{\pgfqpoint{1.010395in}{3.667293in}}%
\pgfpathlineto{\pgfqpoint{1.161181in}{3.704313in}}%
\pgfpathlineto{\pgfqpoint{1.311967in}{3.780013in}}%
\pgfpathlineto{\pgfqpoint{1.462753in}{3.680921in}}%
\pgfpathlineto{\pgfqpoint{1.613539in}{3.771122in}}%
\pgfpathlineto{\pgfqpoint{1.764325in}{3.748663in}}%
\pgfpathlineto{\pgfqpoint{1.915111in}{3.776732in}}%
\pgfpathlineto{\pgfqpoint{2.065897in}{3.781462in}}%
\pgfpathlineto{\pgfqpoint{2.216683in}{3.778335in}}%
\pgfpathlineto{\pgfqpoint{2.367469in}{3.780351in}}%
\pgfpathlineto{\pgfqpoint{2.518255in}{3.800792in}}%
\pgfpathlineto{\pgfqpoint{2.669041in}{3.790704in}}%
\pgfpathlineto{\pgfqpoint{2.819827in}{3.799408in}}%
\pgfpathlineto{\pgfqpoint{2.970613in}{3.802528in}}%
\pgfpathlineto{\pgfqpoint{3.121399in}{3.796527in}}%
\pgfpathlineto{\pgfqpoint{3.272185in}{3.798255in}}%
\pgfpathlineto{\pgfqpoint{3.422971in}{3.798351in}}%
\pgfpathlineto{\pgfqpoint{3.573756in}{3.800696in}}%
\pgfpathlineto{\pgfqpoint{3.724542in}{3.800734in}}%
\pgfpathlineto{\pgfqpoint{3.724542in}{3.802259in}}%
\pgfpathlineto{\pgfqpoint{3.724542in}{3.802259in}}%
\pgfpathlineto{\pgfqpoint{3.573756in}{3.801722in}}%
\pgfpathlineto{\pgfqpoint{3.422971in}{3.802468in}}%
\pgfpathlineto{\pgfqpoint{3.272185in}{3.803073in}}%
\pgfpathlineto{\pgfqpoint{3.121399in}{3.802675in}}%
\pgfpathlineto{\pgfqpoint{2.970613in}{3.809117in}}%
\pgfpathlineto{\pgfqpoint{2.819827in}{3.805020in}}%
\pgfpathlineto{\pgfqpoint{2.669041in}{3.803316in}}%
\pgfpathlineto{\pgfqpoint{2.518255in}{3.812158in}}%
\pgfpathlineto{\pgfqpoint{2.367469in}{3.808957in}}%
\pgfpathlineto{\pgfqpoint{2.216683in}{3.817766in}}%
\pgfpathlineto{\pgfqpoint{2.065897in}{3.840305in}}%
\pgfpathlineto{\pgfqpoint{1.915111in}{3.840694in}}%
\pgfpathlineto{\pgfqpoint{1.764325in}{3.859135in}}%
\pgfpathlineto{\pgfqpoint{1.613539in}{3.838499in}}%
\pgfpathlineto{\pgfqpoint{1.462753in}{3.881706in}}%
\pgfpathlineto{\pgfqpoint{1.311967in}{3.876519in}}%
\pgfpathlineto{\pgfqpoint{1.161181in}{3.940019in}}%
\pgfpathlineto{\pgfqpoint{1.010395in}{3.952569in}}%
\pgfpathlineto{\pgfqpoint{0.859609in}{4.191677in}}%
\pgfpathlineto{\pgfqpoint{0.708823in}{3.900924in}}%
\pgfpathclose%
\pgfusepath{stroke,fill}%
\end{pgfscope}%
\begin{pgfscope}%
\pgfsetbuttcap%
\pgfsetroundjoin%
\definecolor{currentfill}{rgb}{0.000000,0.000000,0.000000}%
\pgfsetfillcolor{currentfill}%
\pgfsetlinewidth{0.803000pt}%
\definecolor{currentstroke}{rgb}{0.000000,0.000000,0.000000}%
\pgfsetstrokecolor{currentstroke}%
\pgfsetdash{}{0pt}%
\pgfsys@defobject{currentmarker}{\pgfqpoint{0.000000in}{-0.048611in}}{\pgfqpoint{0.000000in}{0.000000in}}{%
\pgfpathmoveto{\pgfqpoint{0.000000in}{0.000000in}}%
\pgfpathlineto{\pgfqpoint{0.000000in}{-0.048611in}}%
\pgfusepath{stroke,fill}%
}%
\begin{pgfscope}%
\pgfsys@transformshift{1.071951in}{3.363889in}%
\pgfsys@useobject{currentmarker}{}%
\end{pgfscope}%
\end{pgfscope}%
\begin{pgfscope}%
\definecolor{textcolor}{rgb}{0.000000,0.000000,0.000000}%
\pgfsetstrokecolor{textcolor}%
\pgfsetfillcolor{textcolor}%
\pgftext[x=1.071951in,y=3.266667in,,top]{\color{textcolor}\sffamily\fontsize{10.000000}{12.000000}\selectfont \(\displaystyle {10^{4}}\)}%
\end{pgfscope}%
\begin{pgfscope}%
\pgfsetbuttcap%
\pgfsetroundjoin%
\definecolor{currentfill}{rgb}{0.000000,0.000000,0.000000}%
\pgfsetfillcolor{currentfill}%
\pgfsetlinewidth{0.803000pt}%
\definecolor{currentstroke}{rgb}{0.000000,0.000000,0.000000}%
\pgfsetstrokecolor{currentstroke}%
\pgfsetdash{}{0pt}%
\pgfsys@defobject{currentmarker}{\pgfqpoint{0.000000in}{-0.048611in}}{\pgfqpoint{0.000000in}{0.000000in}}{%
\pgfpathmoveto{\pgfqpoint{0.000000in}{0.000000in}}%
\pgfpathlineto{\pgfqpoint{0.000000in}{-0.048611in}}%
\pgfusepath{stroke,fill}%
}%
\begin{pgfscope}%
\pgfsys@transformshift{1.675095in}{3.363889in}%
\pgfsys@useobject{currentmarker}{}%
\end{pgfscope}%
\end{pgfscope}%
\begin{pgfscope}%
\definecolor{textcolor}{rgb}{0.000000,0.000000,0.000000}%
\pgfsetstrokecolor{textcolor}%
\pgfsetfillcolor{textcolor}%
\pgftext[x=1.675095in,y=3.266667in,,top]{\color{textcolor}\sffamily\fontsize{10.000000}{12.000000}\selectfont \(\displaystyle {10^{5}}\)}%
\end{pgfscope}%
\begin{pgfscope}%
\pgfsetbuttcap%
\pgfsetroundjoin%
\definecolor{currentfill}{rgb}{0.000000,0.000000,0.000000}%
\pgfsetfillcolor{currentfill}%
\pgfsetlinewidth{0.803000pt}%
\definecolor{currentstroke}{rgb}{0.000000,0.000000,0.000000}%
\pgfsetstrokecolor{currentstroke}%
\pgfsetdash{}{0pt}%
\pgfsys@defobject{currentmarker}{\pgfqpoint{0.000000in}{-0.048611in}}{\pgfqpoint{0.000000in}{0.000000in}}{%
\pgfpathmoveto{\pgfqpoint{0.000000in}{0.000000in}}%
\pgfpathlineto{\pgfqpoint{0.000000in}{-0.048611in}}%
\pgfusepath{stroke,fill}%
}%
\begin{pgfscope}%
\pgfsys@transformshift{2.278239in}{3.363889in}%
\pgfsys@useobject{currentmarker}{}%
\end{pgfscope}%
\end{pgfscope}%
\begin{pgfscope}%
\definecolor{textcolor}{rgb}{0.000000,0.000000,0.000000}%
\pgfsetstrokecolor{textcolor}%
\pgfsetfillcolor{textcolor}%
\pgftext[x=2.278239in,y=3.266667in,,top]{\color{textcolor}\sffamily\fontsize{10.000000}{12.000000}\selectfont \(\displaystyle {10^{6}}\)}%
\end{pgfscope}%
\begin{pgfscope}%
\pgfsetbuttcap%
\pgfsetroundjoin%
\definecolor{currentfill}{rgb}{0.000000,0.000000,0.000000}%
\pgfsetfillcolor{currentfill}%
\pgfsetlinewidth{0.803000pt}%
\definecolor{currentstroke}{rgb}{0.000000,0.000000,0.000000}%
\pgfsetstrokecolor{currentstroke}%
\pgfsetdash{}{0pt}%
\pgfsys@defobject{currentmarker}{\pgfqpoint{0.000000in}{-0.048611in}}{\pgfqpoint{0.000000in}{0.000000in}}{%
\pgfpathmoveto{\pgfqpoint{0.000000in}{0.000000in}}%
\pgfpathlineto{\pgfqpoint{0.000000in}{-0.048611in}}%
\pgfusepath{stroke,fill}%
}%
\begin{pgfscope}%
\pgfsys@transformshift{2.881383in}{3.363889in}%
\pgfsys@useobject{currentmarker}{}%
\end{pgfscope}%
\end{pgfscope}%
\begin{pgfscope}%
\definecolor{textcolor}{rgb}{0.000000,0.000000,0.000000}%
\pgfsetstrokecolor{textcolor}%
\pgfsetfillcolor{textcolor}%
\pgftext[x=2.881383in,y=3.266667in,,top]{\color{textcolor}\sffamily\fontsize{10.000000}{12.000000}\selectfont \(\displaystyle {10^{7}}\)}%
\end{pgfscope}%
\begin{pgfscope}%
\pgfsetbuttcap%
\pgfsetroundjoin%
\definecolor{currentfill}{rgb}{0.000000,0.000000,0.000000}%
\pgfsetfillcolor{currentfill}%
\pgfsetlinewidth{0.803000pt}%
\definecolor{currentstroke}{rgb}{0.000000,0.000000,0.000000}%
\pgfsetstrokecolor{currentstroke}%
\pgfsetdash{}{0pt}%
\pgfsys@defobject{currentmarker}{\pgfqpoint{0.000000in}{-0.048611in}}{\pgfqpoint{0.000000in}{0.000000in}}{%
\pgfpathmoveto{\pgfqpoint{0.000000in}{0.000000in}}%
\pgfpathlineto{\pgfqpoint{0.000000in}{-0.048611in}}%
\pgfusepath{stroke,fill}%
}%
\begin{pgfscope}%
\pgfsys@transformshift{3.484527in}{3.363889in}%
\pgfsys@useobject{currentmarker}{}%
\end{pgfscope}%
\end{pgfscope}%
\begin{pgfscope}%
\definecolor{textcolor}{rgb}{0.000000,0.000000,0.000000}%
\pgfsetstrokecolor{textcolor}%
\pgfsetfillcolor{textcolor}%
\pgftext[x=3.484527in,y=3.266667in,,top]{\color{textcolor}\sffamily\fontsize{10.000000}{12.000000}\selectfont \(\displaystyle {10^{8}}\)}%
\end{pgfscope}%
\begin{pgfscope}%
\pgfsetbuttcap%
\pgfsetroundjoin%
\definecolor{currentfill}{rgb}{0.000000,0.000000,0.000000}%
\pgfsetfillcolor{currentfill}%
\pgfsetlinewidth{0.602250pt}%
\definecolor{currentstroke}{rgb}{0.000000,0.000000,0.000000}%
\pgfsetstrokecolor{currentstroke}%
\pgfsetdash{}{0pt}%
\pgfsys@defobject{currentmarker}{\pgfqpoint{0.000000in}{-0.027778in}}{\pgfqpoint{0.000000in}{0.000000in}}{%
\pgfpathmoveto{\pgfqpoint{0.000000in}{0.000000in}}%
\pgfpathlineto{\pgfqpoint{0.000000in}{-0.027778in}}%
\pgfusepath{stroke,fill}%
}%
\begin{pgfscope}%
\pgfsys@transformshift{0.650372in}{3.363889in}%
\pgfsys@useobject{currentmarker}{}%
\end{pgfscope}%
\end{pgfscope}%
\begin{pgfscope}%
\pgfsetbuttcap%
\pgfsetroundjoin%
\definecolor{currentfill}{rgb}{0.000000,0.000000,0.000000}%
\pgfsetfillcolor{currentfill}%
\pgfsetlinewidth{0.602250pt}%
\definecolor{currentstroke}{rgb}{0.000000,0.000000,0.000000}%
\pgfsetstrokecolor{currentstroke}%
\pgfsetdash{}{0pt}%
\pgfsys@defobject{currentmarker}{\pgfqpoint{0.000000in}{-0.027778in}}{\pgfqpoint{0.000000in}{0.000000in}}{%
\pgfpathmoveto{\pgfqpoint{0.000000in}{0.000000in}}%
\pgfpathlineto{\pgfqpoint{0.000000in}{-0.027778in}}%
\pgfusepath{stroke,fill}%
}%
\begin{pgfscope}%
\pgfsys@transformshift{0.756580in}{3.363889in}%
\pgfsys@useobject{currentmarker}{}%
\end{pgfscope}%
\end{pgfscope}%
\begin{pgfscope}%
\pgfsetbuttcap%
\pgfsetroundjoin%
\definecolor{currentfill}{rgb}{0.000000,0.000000,0.000000}%
\pgfsetfillcolor{currentfill}%
\pgfsetlinewidth{0.602250pt}%
\definecolor{currentstroke}{rgb}{0.000000,0.000000,0.000000}%
\pgfsetstrokecolor{currentstroke}%
\pgfsetdash{}{0pt}%
\pgfsys@defobject{currentmarker}{\pgfqpoint{0.000000in}{-0.027778in}}{\pgfqpoint{0.000000in}{0.000000in}}{%
\pgfpathmoveto{\pgfqpoint{0.000000in}{0.000000in}}%
\pgfpathlineto{\pgfqpoint{0.000000in}{-0.027778in}}%
\pgfusepath{stroke,fill}%
}%
\begin{pgfscope}%
\pgfsys@transformshift{0.831936in}{3.363889in}%
\pgfsys@useobject{currentmarker}{}%
\end{pgfscope}%
\end{pgfscope}%
\begin{pgfscope}%
\pgfsetbuttcap%
\pgfsetroundjoin%
\definecolor{currentfill}{rgb}{0.000000,0.000000,0.000000}%
\pgfsetfillcolor{currentfill}%
\pgfsetlinewidth{0.602250pt}%
\definecolor{currentstroke}{rgb}{0.000000,0.000000,0.000000}%
\pgfsetstrokecolor{currentstroke}%
\pgfsetdash{}{0pt}%
\pgfsys@defobject{currentmarker}{\pgfqpoint{0.000000in}{-0.027778in}}{\pgfqpoint{0.000000in}{0.000000in}}{%
\pgfpathmoveto{\pgfqpoint{0.000000in}{0.000000in}}%
\pgfpathlineto{\pgfqpoint{0.000000in}{-0.027778in}}%
\pgfusepath{stroke,fill}%
}%
\begin{pgfscope}%
\pgfsys@transformshift{0.890387in}{3.363889in}%
\pgfsys@useobject{currentmarker}{}%
\end{pgfscope}%
\end{pgfscope}%
\begin{pgfscope}%
\pgfsetbuttcap%
\pgfsetroundjoin%
\definecolor{currentfill}{rgb}{0.000000,0.000000,0.000000}%
\pgfsetfillcolor{currentfill}%
\pgfsetlinewidth{0.602250pt}%
\definecolor{currentstroke}{rgb}{0.000000,0.000000,0.000000}%
\pgfsetstrokecolor{currentstroke}%
\pgfsetdash{}{0pt}%
\pgfsys@defobject{currentmarker}{\pgfqpoint{0.000000in}{-0.027778in}}{\pgfqpoint{0.000000in}{0.000000in}}{%
\pgfpathmoveto{\pgfqpoint{0.000000in}{0.000000in}}%
\pgfpathlineto{\pgfqpoint{0.000000in}{-0.027778in}}%
\pgfusepath{stroke,fill}%
}%
\begin{pgfscope}%
\pgfsys@transformshift{0.938145in}{3.363889in}%
\pgfsys@useobject{currentmarker}{}%
\end{pgfscope}%
\end{pgfscope}%
\begin{pgfscope}%
\pgfsetbuttcap%
\pgfsetroundjoin%
\definecolor{currentfill}{rgb}{0.000000,0.000000,0.000000}%
\pgfsetfillcolor{currentfill}%
\pgfsetlinewidth{0.602250pt}%
\definecolor{currentstroke}{rgb}{0.000000,0.000000,0.000000}%
\pgfsetstrokecolor{currentstroke}%
\pgfsetdash{}{0pt}%
\pgfsys@defobject{currentmarker}{\pgfqpoint{0.000000in}{-0.027778in}}{\pgfqpoint{0.000000in}{0.000000in}}{%
\pgfpathmoveto{\pgfqpoint{0.000000in}{0.000000in}}%
\pgfpathlineto{\pgfqpoint{0.000000in}{-0.027778in}}%
\pgfusepath{stroke,fill}%
}%
\begin{pgfscope}%
\pgfsys@transformshift{0.978523in}{3.363889in}%
\pgfsys@useobject{currentmarker}{}%
\end{pgfscope}%
\end{pgfscope}%
\begin{pgfscope}%
\pgfsetbuttcap%
\pgfsetroundjoin%
\definecolor{currentfill}{rgb}{0.000000,0.000000,0.000000}%
\pgfsetfillcolor{currentfill}%
\pgfsetlinewidth{0.602250pt}%
\definecolor{currentstroke}{rgb}{0.000000,0.000000,0.000000}%
\pgfsetstrokecolor{currentstroke}%
\pgfsetdash{}{0pt}%
\pgfsys@defobject{currentmarker}{\pgfqpoint{0.000000in}{-0.027778in}}{\pgfqpoint{0.000000in}{0.000000in}}{%
\pgfpathmoveto{\pgfqpoint{0.000000in}{0.000000in}}%
\pgfpathlineto{\pgfqpoint{0.000000in}{-0.027778in}}%
\pgfusepath{stroke,fill}%
}%
\begin{pgfscope}%
\pgfsys@transformshift{1.013501in}{3.363889in}%
\pgfsys@useobject{currentmarker}{}%
\end{pgfscope}%
\end{pgfscope}%
\begin{pgfscope}%
\pgfsetbuttcap%
\pgfsetroundjoin%
\definecolor{currentfill}{rgb}{0.000000,0.000000,0.000000}%
\pgfsetfillcolor{currentfill}%
\pgfsetlinewidth{0.602250pt}%
\definecolor{currentstroke}{rgb}{0.000000,0.000000,0.000000}%
\pgfsetstrokecolor{currentstroke}%
\pgfsetdash{}{0pt}%
\pgfsys@defobject{currentmarker}{\pgfqpoint{0.000000in}{-0.027778in}}{\pgfqpoint{0.000000in}{0.000000in}}{%
\pgfpathmoveto{\pgfqpoint{0.000000in}{0.000000in}}%
\pgfpathlineto{\pgfqpoint{0.000000in}{-0.027778in}}%
\pgfusepath{stroke,fill}%
}%
\begin{pgfscope}%
\pgfsys@transformshift{1.044353in}{3.363889in}%
\pgfsys@useobject{currentmarker}{}%
\end{pgfscope}%
\end{pgfscope}%
\begin{pgfscope}%
\pgfsetbuttcap%
\pgfsetroundjoin%
\definecolor{currentfill}{rgb}{0.000000,0.000000,0.000000}%
\pgfsetfillcolor{currentfill}%
\pgfsetlinewidth{0.602250pt}%
\definecolor{currentstroke}{rgb}{0.000000,0.000000,0.000000}%
\pgfsetstrokecolor{currentstroke}%
\pgfsetdash{}{0pt}%
\pgfsys@defobject{currentmarker}{\pgfqpoint{0.000000in}{-0.027778in}}{\pgfqpoint{0.000000in}{0.000000in}}{%
\pgfpathmoveto{\pgfqpoint{0.000000in}{0.000000in}}%
\pgfpathlineto{\pgfqpoint{0.000000in}{-0.027778in}}%
\pgfusepath{stroke,fill}%
}%
\begin{pgfscope}%
\pgfsys@transformshift{1.253516in}{3.363889in}%
\pgfsys@useobject{currentmarker}{}%
\end{pgfscope}%
\end{pgfscope}%
\begin{pgfscope}%
\pgfsetbuttcap%
\pgfsetroundjoin%
\definecolor{currentfill}{rgb}{0.000000,0.000000,0.000000}%
\pgfsetfillcolor{currentfill}%
\pgfsetlinewidth{0.602250pt}%
\definecolor{currentstroke}{rgb}{0.000000,0.000000,0.000000}%
\pgfsetstrokecolor{currentstroke}%
\pgfsetdash{}{0pt}%
\pgfsys@defobject{currentmarker}{\pgfqpoint{0.000000in}{-0.027778in}}{\pgfqpoint{0.000000in}{0.000000in}}{%
\pgfpathmoveto{\pgfqpoint{0.000000in}{0.000000in}}%
\pgfpathlineto{\pgfqpoint{0.000000in}{-0.027778in}}%
\pgfusepath{stroke,fill}%
}%
\begin{pgfscope}%
\pgfsys@transformshift{1.359724in}{3.363889in}%
\pgfsys@useobject{currentmarker}{}%
\end{pgfscope}%
\end{pgfscope}%
\begin{pgfscope}%
\pgfsetbuttcap%
\pgfsetroundjoin%
\definecolor{currentfill}{rgb}{0.000000,0.000000,0.000000}%
\pgfsetfillcolor{currentfill}%
\pgfsetlinewidth{0.602250pt}%
\definecolor{currentstroke}{rgb}{0.000000,0.000000,0.000000}%
\pgfsetstrokecolor{currentstroke}%
\pgfsetdash{}{0pt}%
\pgfsys@defobject{currentmarker}{\pgfqpoint{0.000000in}{-0.027778in}}{\pgfqpoint{0.000000in}{0.000000in}}{%
\pgfpathmoveto{\pgfqpoint{0.000000in}{0.000000in}}%
\pgfpathlineto{\pgfqpoint{0.000000in}{-0.027778in}}%
\pgfusepath{stroke,fill}%
}%
\begin{pgfscope}%
\pgfsys@transformshift{1.435080in}{3.363889in}%
\pgfsys@useobject{currentmarker}{}%
\end{pgfscope}%
\end{pgfscope}%
\begin{pgfscope}%
\pgfsetbuttcap%
\pgfsetroundjoin%
\definecolor{currentfill}{rgb}{0.000000,0.000000,0.000000}%
\pgfsetfillcolor{currentfill}%
\pgfsetlinewidth{0.602250pt}%
\definecolor{currentstroke}{rgb}{0.000000,0.000000,0.000000}%
\pgfsetstrokecolor{currentstroke}%
\pgfsetdash{}{0pt}%
\pgfsys@defobject{currentmarker}{\pgfqpoint{0.000000in}{-0.027778in}}{\pgfqpoint{0.000000in}{0.000000in}}{%
\pgfpathmoveto{\pgfqpoint{0.000000in}{0.000000in}}%
\pgfpathlineto{\pgfqpoint{0.000000in}{-0.027778in}}%
\pgfusepath{stroke,fill}%
}%
\begin{pgfscope}%
\pgfsys@transformshift{1.493531in}{3.363889in}%
\pgfsys@useobject{currentmarker}{}%
\end{pgfscope}%
\end{pgfscope}%
\begin{pgfscope}%
\pgfsetbuttcap%
\pgfsetroundjoin%
\definecolor{currentfill}{rgb}{0.000000,0.000000,0.000000}%
\pgfsetfillcolor{currentfill}%
\pgfsetlinewidth{0.602250pt}%
\definecolor{currentstroke}{rgb}{0.000000,0.000000,0.000000}%
\pgfsetstrokecolor{currentstroke}%
\pgfsetdash{}{0pt}%
\pgfsys@defobject{currentmarker}{\pgfqpoint{0.000000in}{-0.027778in}}{\pgfqpoint{0.000000in}{0.000000in}}{%
\pgfpathmoveto{\pgfqpoint{0.000000in}{0.000000in}}%
\pgfpathlineto{\pgfqpoint{0.000000in}{-0.027778in}}%
\pgfusepath{stroke,fill}%
}%
\begin{pgfscope}%
\pgfsys@transformshift{1.541289in}{3.363889in}%
\pgfsys@useobject{currentmarker}{}%
\end{pgfscope}%
\end{pgfscope}%
\begin{pgfscope}%
\pgfsetbuttcap%
\pgfsetroundjoin%
\definecolor{currentfill}{rgb}{0.000000,0.000000,0.000000}%
\pgfsetfillcolor{currentfill}%
\pgfsetlinewidth{0.602250pt}%
\definecolor{currentstroke}{rgb}{0.000000,0.000000,0.000000}%
\pgfsetstrokecolor{currentstroke}%
\pgfsetdash{}{0pt}%
\pgfsys@defobject{currentmarker}{\pgfqpoint{0.000000in}{-0.027778in}}{\pgfqpoint{0.000000in}{0.000000in}}{%
\pgfpathmoveto{\pgfqpoint{0.000000in}{0.000000in}}%
\pgfpathlineto{\pgfqpoint{0.000000in}{-0.027778in}}%
\pgfusepath{stroke,fill}%
}%
\begin{pgfscope}%
\pgfsys@transformshift{1.581667in}{3.363889in}%
\pgfsys@useobject{currentmarker}{}%
\end{pgfscope}%
\end{pgfscope}%
\begin{pgfscope}%
\pgfsetbuttcap%
\pgfsetroundjoin%
\definecolor{currentfill}{rgb}{0.000000,0.000000,0.000000}%
\pgfsetfillcolor{currentfill}%
\pgfsetlinewidth{0.602250pt}%
\definecolor{currentstroke}{rgb}{0.000000,0.000000,0.000000}%
\pgfsetstrokecolor{currentstroke}%
\pgfsetdash{}{0pt}%
\pgfsys@defobject{currentmarker}{\pgfqpoint{0.000000in}{-0.027778in}}{\pgfqpoint{0.000000in}{0.000000in}}{%
\pgfpathmoveto{\pgfqpoint{0.000000in}{0.000000in}}%
\pgfpathlineto{\pgfqpoint{0.000000in}{-0.027778in}}%
\pgfusepath{stroke,fill}%
}%
\begin{pgfscope}%
\pgfsys@transformshift{1.616645in}{3.363889in}%
\pgfsys@useobject{currentmarker}{}%
\end{pgfscope}%
\end{pgfscope}%
\begin{pgfscope}%
\pgfsetbuttcap%
\pgfsetroundjoin%
\definecolor{currentfill}{rgb}{0.000000,0.000000,0.000000}%
\pgfsetfillcolor{currentfill}%
\pgfsetlinewidth{0.602250pt}%
\definecolor{currentstroke}{rgb}{0.000000,0.000000,0.000000}%
\pgfsetstrokecolor{currentstroke}%
\pgfsetdash{}{0pt}%
\pgfsys@defobject{currentmarker}{\pgfqpoint{0.000000in}{-0.027778in}}{\pgfqpoint{0.000000in}{0.000000in}}{%
\pgfpathmoveto{\pgfqpoint{0.000000in}{0.000000in}}%
\pgfpathlineto{\pgfqpoint{0.000000in}{-0.027778in}}%
\pgfusepath{stroke,fill}%
}%
\begin{pgfscope}%
\pgfsys@transformshift{1.647497in}{3.363889in}%
\pgfsys@useobject{currentmarker}{}%
\end{pgfscope}%
\end{pgfscope}%
\begin{pgfscope}%
\pgfsetbuttcap%
\pgfsetroundjoin%
\definecolor{currentfill}{rgb}{0.000000,0.000000,0.000000}%
\pgfsetfillcolor{currentfill}%
\pgfsetlinewidth{0.602250pt}%
\definecolor{currentstroke}{rgb}{0.000000,0.000000,0.000000}%
\pgfsetstrokecolor{currentstroke}%
\pgfsetdash{}{0pt}%
\pgfsys@defobject{currentmarker}{\pgfqpoint{0.000000in}{-0.027778in}}{\pgfqpoint{0.000000in}{0.000000in}}{%
\pgfpathmoveto{\pgfqpoint{0.000000in}{0.000000in}}%
\pgfpathlineto{\pgfqpoint{0.000000in}{-0.027778in}}%
\pgfusepath{stroke,fill}%
}%
\begin{pgfscope}%
\pgfsys@transformshift{1.856660in}{3.363889in}%
\pgfsys@useobject{currentmarker}{}%
\end{pgfscope}%
\end{pgfscope}%
\begin{pgfscope}%
\pgfsetbuttcap%
\pgfsetroundjoin%
\definecolor{currentfill}{rgb}{0.000000,0.000000,0.000000}%
\pgfsetfillcolor{currentfill}%
\pgfsetlinewidth{0.602250pt}%
\definecolor{currentstroke}{rgb}{0.000000,0.000000,0.000000}%
\pgfsetstrokecolor{currentstroke}%
\pgfsetdash{}{0pt}%
\pgfsys@defobject{currentmarker}{\pgfqpoint{0.000000in}{-0.027778in}}{\pgfqpoint{0.000000in}{0.000000in}}{%
\pgfpathmoveto{\pgfqpoint{0.000000in}{0.000000in}}%
\pgfpathlineto{\pgfqpoint{0.000000in}{-0.027778in}}%
\pgfusepath{stroke,fill}%
}%
\begin{pgfscope}%
\pgfsys@transformshift{1.962868in}{3.363889in}%
\pgfsys@useobject{currentmarker}{}%
\end{pgfscope}%
\end{pgfscope}%
\begin{pgfscope}%
\pgfsetbuttcap%
\pgfsetroundjoin%
\definecolor{currentfill}{rgb}{0.000000,0.000000,0.000000}%
\pgfsetfillcolor{currentfill}%
\pgfsetlinewidth{0.602250pt}%
\definecolor{currentstroke}{rgb}{0.000000,0.000000,0.000000}%
\pgfsetstrokecolor{currentstroke}%
\pgfsetdash{}{0pt}%
\pgfsys@defobject{currentmarker}{\pgfqpoint{0.000000in}{-0.027778in}}{\pgfqpoint{0.000000in}{0.000000in}}{%
\pgfpathmoveto{\pgfqpoint{0.000000in}{0.000000in}}%
\pgfpathlineto{\pgfqpoint{0.000000in}{-0.027778in}}%
\pgfusepath{stroke,fill}%
}%
\begin{pgfscope}%
\pgfsys@transformshift{2.038224in}{3.363889in}%
\pgfsys@useobject{currentmarker}{}%
\end{pgfscope}%
\end{pgfscope}%
\begin{pgfscope}%
\pgfsetbuttcap%
\pgfsetroundjoin%
\definecolor{currentfill}{rgb}{0.000000,0.000000,0.000000}%
\pgfsetfillcolor{currentfill}%
\pgfsetlinewidth{0.602250pt}%
\definecolor{currentstroke}{rgb}{0.000000,0.000000,0.000000}%
\pgfsetstrokecolor{currentstroke}%
\pgfsetdash{}{0pt}%
\pgfsys@defobject{currentmarker}{\pgfqpoint{0.000000in}{-0.027778in}}{\pgfqpoint{0.000000in}{0.000000in}}{%
\pgfpathmoveto{\pgfqpoint{0.000000in}{0.000000in}}%
\pgfpathlineto{\pgfqpoint{0.000000in}{-0.027778in}}%
\pgfusepath{stroke,fill}%
}%
\begin{pgfscope}%
\pgfsys@transformshift{2.096675in}{3.363889in}%
\pgfsys@useobject{currentmarker}{}%
\end{pgfscope}%
\end{pgfscope}%
\begin{pgfscope}%
\pgfsetbuttcap%
\pgfsetroundjoin%
\definecolor{currentfill}{rgb}{0.000000,0.000000,0.000000}%
\pgfsetfillcolor{currentfill}%
\pgfsetlinewidth{0.602250pt}%
\definecolor{currentstroke}{rgb}{0.000000,0.000000,0.000000}%
\pgfsetstrokecolor{currentstroke}%
\pgfsetdash{}{0pt}%
\pgfsys@defobject{currentmarker}{\pgfqpoint{0.000000in}{-0.027778in}}{\pgfqpoint{0.000000in}{0.000000in}}{%
\pgfpathmoveto{\pgfqpoint{0.000000in}{0.000000in}}%
\pgfpathlineto{\pgfqpoint{0.000000in}{-0.027778in}}%
\pgfusepath{stroke,fill}%
}%
\begin{pgfscope}%
\pgfsys@transformshift{2.144433in}{3.363889in}%
\pgfsys@useobject{currentmarker}{}%
\end{pgfscope}%
\end{pgfscope}%
\begin{pgfscope}%
\pgfsetbuttcap%
\pgfsetroundjoin%
\definecolor{currentfill}{rgb}{0.000000,0.000000,0.000000}%
\pgfsetfillcolor{currentfill}%
\pgfsetlinewidth{0.602250pt}%
\definecolor{currentstroke}{rgb}{0.000000,0.000000,0.000000}%
\pgfsetstrokecolor{currentstroke}%
\pgfsetdash{}{0pt}%
\pgfsys@defobject{currentmarker}{\pgfqpoint{0.000000in}{-0.027778in}}{\pgfqpoint{0.000000in}{0.000000in}}{%
\pgfpathmoveto{\pgfqpoint{0.000000in}{0.000000in}}%
\pgfpathlineto{\pgfqpoint{0.000000in}{-0.027778in}}%
\pgfusepath{stroke,fill}%
}%
\begin{pgfscope}%
\pgfsys@transformshift{2.184811in}{3.363889in}%
\pgfsys@useobject{currentmarker}{}%
\end{pgfscope}%
\end{pgfscope}%
\begin{pgfscope}%
\pgfsetbuttcap%
\pgfsetroundjoin%
\definecolor{currentfill}{rgb}{0.000000,0.000000,0.000000}%
\pgfsetfillcolor{currentfill}%
\pgfsetlinewidth{0.602250pt}%
\definecolor{currentstroke}{rgb}{0.000000,0.000000,0.000000}%
\pgfsetstrokecolor{currentstroke}%
\pgfsetdash{}{0pt}%
\pgfsys@defobject{currentmarker}{\pgfqpoint{0.000000in}{-0.027778in}}{\pgfqpoint{0.000000in}{0.000000in}}{%
\pgfpathmoveto{\pgfqpoint{0.000000in}{0.000000in}}%
\pgfpathlineto{\pgfqpoint{0.000000in}{-0.027778in}}%
\pgfusepath{stroke,fill}%
}%
\begin{pgfscope}%
\pgfsys@transformshift{2.219789in}{3.363889in}%
\pgfsys@useobject{currentmarker}{}%
\end{pgfscope}%
\end{pgfscope}%
\begin{pgfscope}%
\pgfsetbuttcap%
\pgfsetroundjoin%
\definecolor{currentfill}{rgb}{0.000000,0.000000,0.000000}%
\pgfsetfillcolor{currentfill}%
\pgfsetlinewidth{0.602250pt}%
\definecolor{currentstroke}{rgb}{0.000000,0.000000,0.000000}%
\pgfsetstrokecolor{currentstroke}%
\pgfsetdash{}{0pt}%
\pgfsys@defobject{currentmarker}{\pgfqpoint{0.000000in}{-0.027778in}}{\pgfqpoint{0.000000in}{0.000000in}}{%
\pgfpathmoveto{\pgfqpoint{0.000000in}{0.000000in}}%
\pgfpathlineto{\pgfqpoint{0.000000in}{-0.027778in}}%
\pgfusepath{stroke,fill}%
}%
\begin{pgfscope}%
\pgfsys@transformshift{2.250641in}{3.363889in}%
\pgfsys@useobject{currentmarker}{}%
\end{pgfscope}%
\end{pgfscope}%
\begin{pgfscope}%
\pgfsetbuttcap%
\pgfsetroundjoin%
\definecolor{currentfill}{rgb}{0.000000,0.000000,0.000000}%
\pgfsetfillcolor{currentfill}%
\pgfsetlinewidth{0.602250pt}%
\definecolor{currentstroke}{rgb}{0.000000,0.000000,0.000000}%
\pgfsetstrokecolor{currentstroke}%
\pgfsetdash{}{0pt}%
\pgfsys@defobject{currentmarker}{\pgfqpoint{0.000000in}{-0.027778in}}{\pgfqpoint{0.000000in}{0.000000in}}{%
\pgfpathmoveto{\pgfqpoint{0.000000in}{0.000000in}}%
\pgfpathlineto{\pgfqpoint{0.000000in}{-0.027778in}}%
\pgfusepath{stroke,fill}%
}%
\begin{pgfscope}%
\pgfsys@transformshift{2.459804in}{3.363889in}%
\pgfsys@useobject{currentmarker}{}%
\end{pgfscope}%
\end{pgfscope}%
\begin{pgfscope}%
\pgfsetbuttcap%
\pgfsetroundjoin%
\definecolor{currentfill}{rgb}{0.000000,0.000000,0.000000}%
\pgfsetfillcolor{currentfill}%
\pgfsetlinewidth{0.602250pt}%
\definecolor{currentstroke}{rgb}{0.000000,0.000000,0.000000}%
\pgfsetstrokecolor{currentstroke}%
\pgfsetdash{}{0pt}%
\pgfsys@defobject{currentmarker}{\pgfqpoint{0.000000in}{-0.027778in}}{\pgfqpoint{0.000000in}{0.000000in}}{%
\pgfpathmoveto{\pgfqpoint{0.000000in}{0.000000in}}%
\pgfpathlineto{\pgfqpoint{0.000000in}{-0.027778in}}%
\pgfusepath{stroke,fill}%
}%
\begin{pgfscope}%
\pgfsys@transformshift{2.566012in}{3.363889in}%
\pgfsys@useobject{currentmarker}{}%
\end{pgfscope}%
\end{pgfscope}%
\begin{pgfscope}%
\pgfsetbuttcap%
\pgfsetroundjoin%
\definecolor{currentfill}{rgb}{0.000000,0.000000,0.000000}%
\pgfsetfillcolor{currentfill}%
\pgfsetlinewidth{0.602250pt}%
\definecolor{currentstroke}{rgb}{0.000000,0.000000,0.000000}%
\pgfsetstrokecolor{currentstroke}%
\pgfsetdash{}{0pt}%
\pgfsys@defobject{currentmarker}{\pgfqpoint{0.000000in}{-0.027778in}}{\pgfqpoint{0.000000in}{0.000000in}}{%
\pgfpathmoveto{\pgfqpoint{0.000000in}{0.000000in}}%
\pgfpathlineto{\pgfqpoint{0.000000in}{-0.027778in}}%
\pgfusepath{stroke,fill}%
}%
\begin{pgfscope}%
\pgfsys@transformshift{2.641368in}{3.363889in}%
\pgfsys@useobject{currentmarker}{}%
\end{pgfscope}%
\end{pgfscope}%
\begin{pgfscope}%
\pgfsetbuttcap%
\pgfsetroundjoin%
\definecolor{currentfill}{rgb}{0.000000,0.000000,0.000000}%
\pgfsetfillcolor{currentfill}%
\pgfsetlinewidth{0.602250pt}%
\definecolor{currentstroke}{rgb}{0.000000,0.000000,0.000000}%
\pgfsetstrokecolor{currentstroke}%
\pgfsetdash{}{0pt}%
\pgfsys@defobject{currentmarker}{\pgfqpoint{0.000000in}{-0.027778in}}{\pgfqpoint{0.000000in}{0.000000in}}{%
\pgfpathmoveto{\pgfqpoint{0.000000in}{0.000000in}}%
\pgfpathlineto{\pgfqpoint{0.000000in}{-0.027778in}}%
\pgfusepath{stroke,fill}%
}%
\begin{pgfscope}%
\pgfsys@transformshift{2.699819in}{3.363889in}%
\pgfsys@useobject{currentmarker}{}%
\end{pgfscope}%
\end{pgfscope}%
\begin{pgfscope}%
\pgfsetbuttcap%
\pgfsetroundjoin%
\definecolor{currentfill}{rgb}{0.000000,0.000000,0.000000}%
\pgfsetfillcolor{currentfill}%
\pgfsetlinewidth{0.602250pt}%
\definecolor{currentstroke}{rgb}{0.000000,0.000000,0.000000}%
\pgfsetstrokecolor{currentstroke}%
\pgfsetdash{}{0pt}%
\pgfsys@defobject{currentmarker}{\pgfqpoint{0.000000in}{-0.027778in}}{\pgfqpoint{0.000000in}{0.000000in}}{%
\pgfpathmoveto{\pgfqpoint{0.000000in}{0.000000in}}%
\pgfpathlineto{\pgfqpoint{0.000000in}{-0.027778in}}%
\pgfusepath{stroke,fill}%
}%
\begin{pgfscope}%
\pgfsys@transformshift{2.747577in}{3.363889in}%
\pgfsys@useobject{currentmarker}{}%
\end{pgfscope}%
\end{pgfscope}%
\begin{pgfscope}%
\pgfsetbuttcap%
\pgfsetroundjoin%
\definecolor{currentfill}{rgb}{0.000000,0.000000,0.000000}%
\pgfsetfillcolor{currentfill}%
\pgfsetlinewidth{0.602250pt}%
\definecolor{currentstroke}{rgb}{0.000000,0.000000,0.000000}%
\pgfsetstrokecolor{currentstroke}%
\pgfsetdash{}{0pt}%
\pgfsys@defobject{currentmarker}{\pgfqpoint{0.000000in}{-0.027778in}}{\pgfqpoint{0.000000in}{0.000000in}}{%
\pgfpathmoveto{\pgfqpoint{0.000000in}{0.000000in}}%
\pgfpathlineto{\pgfqpoint{0.000000in}{-0.027778in}}%
\pgfusepath{stroke,fill}%
}%
\begin{pgfscope}%
\pgfsys@transformshift{2.787955in}{3.363889in}%
\pgfsys@useobject{currentmarker}{}%
\end{pgfscope}%
\end{pgfscope}%
\begin{pgfscope}%
\pgfsetbuttcap%
\pgfsetroundjoin%
\definecolor{currentfill}{rgb}{0.000000,0.000000,0.000000}%
\pgfsetfillcolor{currentfill}%
\pgfsetlinewidth{0.602250pt}%
\definecolor{currentstroke}{rgb}{0.000000,0.000000,0.000000}%
\pgfsetstrokecolor{currentstroke}%
\pgfsetdash{}{0pt}%
\pgfsys@defobject{currentmarker}{\pgfqpoint{0.000000in}{-0.027778in}}{\pgfqpoint{0.000000in}{0.000000in}}{%
\pgfpathmoveto{\pgfqpoint{0.000000in}{0.000000in}}%
\pgfpathlineto{\pgfqpoint{0.000000in}{-0.027778in}}%
\pgfusepath{stroke,fill}%
}%
\begin{pgfscope}%
\pgfsys@transformshift{2.822933in}{3.363889in}%
\pgfsys@useobject{currentmarker}{}%
\end{pgfscope}%
\end{pgfscope}%
\begin{pgfscope}%
\pgfsetbuttcap%
\pgfsetroundjoin%
\definecolor{currentfill}{rgb}{0.000000,0.000000,0.000000}%
\pgfsetfillcolor{currentfill}%
\pgfsetlinewidth{0.602250pt}%
\definecolor{currentstroke}{rgb}{0.000000,0.000000,0.000000}%
\pgfsetstrokecolor{currentstroke}%
\pgfsetdash{}{0pt}%
\pgfsys@defobject{currentmarker}{\pgfqpoint{0.000000in}{-0.027778in}}{\pgfqpoint{0.000000in}{0.000000in}}{%
\pgfpathmoveto{\pgfqpoint{0.000000in}{0.000000in}}%
\pgfpathlineto{\pgfqpoint{0.000000in}{-0.027778in}}%
\pgfusepath{stroke,fill}%
}%
\begin{pgfscope}%
\pgfsys@transformshift{2.853785in}{3.363889in}%
\pgfsys@useobject{currentmarker}{}%
\end{pgfscope}%
\end{pgfscope}%
\begin{pgfscope}%
\pgfsetbuttcap%
\pgfsetroundjoin%
\definecolor{currentfill}{rgb}{0.000000,0.000000,0.000000}%
\pgfsetfillcolor{currentfill}%
\pgfsetlinewidth{0.602250pt}%
\definecolor{currentstroke}{rgb}{0.000000,0.000000,0.000000}%
\pgfsetstrokecolor{currentstroke}%
\pgfsetdash{}{0pt}%
\pgfsys@defobject{currentmarker}{\pgfqpoint{0.000000in}{-0.027778in}}{\pgfqpoint{0.000000in}{0.000000in}}{%
\pgfpathmoveto{\pgfqpoint{0.000000in}{0.000000in}}%
\pgfpathlineto{\pgfqpoint{0.000000in}{-0.027778in}}%
\pgfusepath{stroke,fill}%
}%
\begin{pgfscope}%
\pgfsys@transformshift{3.062948in}{3.363889in}%
\pgfsys@useobject{currentmarker}{}%
\end{pgfscope}%
\end{pgfscope}%
\begin{pgfscope}%
\pgfsetbuttcap%
\pgfsetroundjoin%
\definecolor{currentfill}{rgb}{0.000000,0.000000,0.000000}%
\pgfsetfillcolor{currentfill}%
\pgfsetlinewidth{0.602250pt}%
\definecolor{currentstroke}{rgb}{0.000000,0.000000,0.000000}%
\pgfsetstrokecolor{currentstroke}%
\pgfsetdash{}{0pt}%
\pgfsys@defobject{currentmarker}{\pgfqpoint{0.000000in}{-0.027778in}}{\pgfqpoint{0.000000in}{0.000000in}}{%
\pgfpathmoveto{\pgfqpoint{0.000000in}{0.000000in}}%
\pgfpathlineto{\pgfqpoint{0.000000in}{-0.027778in}}%
\pgfusepath{stroke,fill}%
}%
\begin{pgfscope}%
\pgfsys@transformshift{3.169156in}{3.363889in}%
\pgfsys@useobject{currentmarker}{}%
\end{pgfscope}%
\end{pgfscope}%
\begin{pgfscope}%
\pgfsetbuttcap%
\pgfsetroundjoin%
\definecolor{currentfill}{rgb}{0.000000,0.000000,0.000000}%
\pgfsetfillcolor{currentfill}%
\pgfsetlinewidth{0.602250pt}%
\definecolor{currentstroke}{rgb}{0.000000,0.000000,0.000000}%
\pgfsetstrokecolor{currentstroke}%
\pgfsetdash{}{0pt}%
\pgfsys@defobject{currentmarker}{\pgfqpoint{0.000000in}{-0.027778in}}{\pgfqpoint{0.000000in}{0.000000in}}{%
\pgfpathmoveto{\pgfqpoint{0.000000in}{0.000000in}}%
\pgfpathlineto{\pgfqpoint{0.000000in}{-0.027778in}}%
\pgfusepath{stroke,fill}%
}%
\begin{pgfscope}%
\pgfsys@transformshift{3.244512in}{3.363889in}%
\pgfsys@useobject{currentmarker}{}%
\end{pgfscope}%
\end{pgfscope}%
\begin{pgfscope}%
\pgfsetbuttcap%
\pgfsetroundjoin%
\definecolor{currentfill}{rgb}{0.000000,0.000000,0.000000}%
\pgfsetfillcolor{currentfill}%
\pgfsetlinewidth{0.602250pt}%
\definecolor{currentstroke}{rgb}{0.000000,0.000000,0.000000}%
\pgfsetstrokecolor{currentstroke}%
\pgfsetdash{}{0pt}%
\pgfsys@defobject{currentmarker}{\pgfqpoint{0.000000in}{-0.027778in}}{\pgfqpoint{0.000000in}{0.000000in}}{%
\pgfpathmoveto{\pgfqpoint{0.000000in}{0.000000in}}%
\pgfpathlineto{\pgfqpoint{0.000000in}{-0.027778in}}%
\pgfusepath{stroke,fill}%
}%
\begin{pgfscope}%
\pgfsys@transformshift{3.302963in}{3.363889in}%
\pgfsys@useobject{currentmarker}{}%
\end{pgfscope}%
\end{pgfscope}%
\begin{pgfscope}%
\pgfsetbuttcap%
\pgfsetroundjoin%
\definecolor{currentfill}{rgb}{0.000000,0.000000,0.000000}%
\pgfsetfillcolor{currentfill}%
\pgfsetlinewidth{0.602250pt}%
\definecolor{currentstroke}{rgb}{0.000000,0.000000,0.000000}%
\pgfsetstrokecolor{currentstroke}%
\pgfsetdash{}{0pt}%
\pgfsys@defobject{currentmarker}{\pgfqpoint{0.000000in}{-0.027778in}}{\pgfqpoint{0.000000in}{0.000000in}}{%
\pgfpathmoveto{\pgfqpoint{0.000000in}{0.000000in}}%
\pgfpathlineto{\pgfqpoint{0.000000in}{-0.027778in}}%
\pgfusepath{stroke,fill}%
}%
\begin{pgfscope}%
\pgfsys@transformshift{3.350721in}{3.363889in}%
\pgfsys@useobject{currentmarker}{}%
\end{pgfscope}%
\end{pgfscope}%
\begin{pgfscope}%
\pgfsetbuttcap%
\pgfsetroundjoin%
\definecolor{currentfill}{rgb}{0.000000,0.000000,0.000000}%
\pgfsetfillcolor{currentfill}%
\pgfsetlinewidth{0.602250pt}%
\definecolor{currentstroke}{rgb}{0.000000,0.000000,0.000000}%
\pgfsetstrokecolor{currentstroke}%
\pgfsetdash{}{0pt}%
\pgfsys@defobject{currentmarker}{\pgfqpoint{0.000000in}{-0.027778in}}{\pgfqpoint{0.000000in}{0.000000in}}{%
\pgfpathmoveto{\pgfqpoint{0.000000in}{0.000000in}}%
\pgfpathlineto{\pgfqpoint{0.000000in}{-0.027778in}}%
\pgfusepath{stroke,fill}%
}%
\begin{pgfscope}%
\pgfsys@transformshift{3.391099in}{3.363889in}%
\pgfsys@useobject{currentmarker}{}%
\end{pgfscope}%
\end{pgfscope}%
\begin{pgfscope}%
\pgfsetbuttcap%
\pgfsetroundjoin%
\definecolor{currentfill}{rgb}{0.000000,0.000000,0.000000}%
\pgfsetfillcolor{currentfill}%
\pgfsetlinewidth{0.602250pt}%
\definecolor{currentstroke}{rgb}{0.000000,0.000000,0.000000}%
\pgfsetstrokecolor{currentstroke}%
\pgfsetdash{}{0pt}%
\pgfsys@defobject{currentmarker}{\pgfqpoint{0.000000in}{-0.027778in}}{\pgfqpoint{0.000000in}{0.000000in}}{%
\pgfpathmoveto{\pgfqpoint{0.000000in}{0.000000in}}%
\pgfpathlineto{\pgfqpoint{0.000000in}{-0.027778in}}%
\pgfusepath{stroke,fill}%
}%
\begin{pgfscope}%
\pgfsys@transformshift{3.426077in}{3.363889in}%
\pgfsys@useobject{currentmarker}{}%
\end{pgfscope}%
\end{pgfscope}%
\begin{pgfscope}%
\pgfsetbuttcap%
\pgfsetroundjoin%
\definecolor{currentfill}{rgb}{0.000000,0.000000,0.000000}%
\pgfsetfillcolor{currentfill}%
\pgfsetlinewidth{0.602250pt}%
\definecolor{currentstroke}{rgb}{0.000000,0.000000,0.000000}%
\pgfsetstrokecolor{currentstroke}%
\pgfsetdash{}{0pt}%
\pgfsys@defobject{currentmarker}{\pgfqpoint{0.000000in}{-0.027778in}}{\pgfqpoint{0.000000in}{0.000000in}}{%
\pgfpathmoveto{\pgfqpoint{0.000000in}{0.000000in}}%
\pgfpathlineto{\pgfqpoint{0.000000in}{-0.027778in}}%
\pgfusepath{stroke,fill}%
}%
\begin{pgfscope}%
\pgfsys@transformshift{3.456929in}{3.363889in}%
\pgfsys@useobject{currentmarker}{}%
\end{pgfscope}%
\end{pgfscope}%
\begin{pgfscope}%
\pgfsetbuttcap%
\pgfsetroundjoin%
\definecolor{currentfill}{rgb}{0.000000,0.000000,0.000000}%
\pgfsetfillcolor{currentfill}%
\pgfsetlinewidth{0.602250pt}%
\definecolor{currentstroke}{rgb}{0.000000,0.000000,0.000000}%
\pgfsetstrokecolor{currentstroke}%
\pgfsetdash{}{0pt}%
\pgfsys@defobject{currentmarker}{\pgfqpoint{0.000000in}{-0.027778in}}{\pgfqpoint{0.000000in}{0.000000in}}{%
\pgfpathmoveto{\pgfqpoint{0.000000in}{0.000000in}}%
\pgfpathlineto{\pgfqpoint{0.000000in}{-0.027778in}}%
\pgfusepath{stroke,fill}%
}%
\begin{pgfscope}%
\pgfsys@transformshift{3.666092in}{3.363889in}%
\pgfsys@useobject{currentmarker}{}%
\end{pgfscope}%
\end{pgfscope}%
\begin{pgfscope}%
\pgfsetbuttcap%
\pgfsetroundjoin%
\definecolor{currentfill}{rgb}{0.000000,0.000000,0.000000}%
\pgfsetfillcolor{currentfill}%
\pgfsetlinewidth{0.602250pt}%
\definecolor{currentstroke}{rgb}{0.000000,0.000000,0.000000}%
\pgfsetstrokecolor{currentstroke}%
\pgfsetdash{}{0pt}%
\pgfsys@defobject{currentmarker}{\pgfqpoint{0.000000in}{-0.027778in}}{\pgfqpoint{0.000000in}{0.000000in}}{%
\pgfpathmoveto{\pgfqpoint{0.000000in}{0.000000in}}%
\pgfpathlineto{\pgfqpoint{0.000000in}{-0.027778in}}%
\pgfusepath{stroke,fill}%
}%
\begin{pgfscope}%
\pgfsys@transformshift{3.772300in}{3.363889in}%
\pgfsys@useobject{currentmarker}{}%
\end{pgfscope}%
\end{pgfscope}%
\begin{pgfscope}%
\pgfsetbuttcap%
\pgfsetroundjoin%
\definecolor{currentfill}{rgb}{0.000000,0.000000,0.000000}%
\pgfsetfillcolor{currentfill}%
\pgfsetlinewidth{0.602250pt}%
\definecolor{currentstroke}{rgb}{0.000000,0.000000,0.000000}%
\pgfsetstrokecolor{currentstroke}%
\pgfsetdash{}{0pt}%
\pgfsys@defobject{currentmarker}{\pgfqpoint{0.000000in}{-0.027778in}}{\pgfqpoint{0.000000in}{0.000000in}}{%
\pgfpathmoveto{\pgfqpoint{0.000000in}{0.000000in}}%
\pgfpathlineto{\pgfqpoint{0.000000in}{-0.027778in}}%
\pgfusepath{stroke,fill}%
}%
\begin{pgfscope}%
\pgfsys@transformshift{3.847656in}{3.363889in}%
\pgfsys@useobject{currentmarker}{}%
\end{pgfscope}%
\end{pgfscope}%
\begin{pgfscope}%
\pgfsetbuttcap%
\pgfsetroundjoin%
\definecolor{currentfill}{rgb}{0.000000,0.000000,0.000000}%
\pgfsetfillcolor{currentfill}%
\pgfsetlinewidth{0.803000pt}%
\definecolor{currentstroke}{rgb}{0.000000,0.000000,0.000000}%
\pgfsetstrokecolor{currentstroke}%
\pgfsetdash{}{0pt}%
\pgfsys@defobject{currentmarker}{\pgfqpoint{-0.048611in}{0.000000in}}{\pgfqpoint{0.000000in}{0.000000in}}{%
\pgfpathmoveto{\pgfqpoint{0.000000in}{0.000000in}}%
\pgfpathlineto{\pgfqpoint{-0.048611in}{0.000000in}}%
\pgfusepath{stroke,fill}%
}%
\begin{pgfscope}%
\pgfsys@transformshift{0.557986in}{3.549819in}%
\pgfsys@useobject{currentmarker}{}%
\end{pgfscope}%
\end{pgfscope}%
\begin{pgfscope}%
\definecolor{textcolor}{rgb}{0.000000,0.000000,0.000000}%
\pgfsetstrokecolor{textcolor}%
\pgfsetfillcolor{textcolor}%
\pgftext[x=0.239884in,y=3.497057in,left,base]{\color{textcolor}\sffamily\fontsize{10.000000}{12.000000}\selectfont 0.0}%
\end{pgfscope}%
\begin{pgfscope}%
\pgfsetbuttcap%
\pgfsetroundjoin%
\definecolor{currentfill}{rgb}{0.000000,0.000000,0.000000}%
\pgfsetfillcolor{currentfill}%
\pgfsetlinewidth{0.803000pt}%
\definecolor{currentstroke}{rgb}{0.000000,0.000000,0.000000}%
\pgfsetstrokecolor{currentstroke}%
\pgfsetdash{}{0pt}%
\pgfsys@defobject{currentmarker}{\pgfqpoint{-0.048611in}{0.000000in}}{\pgfqpoint{0.000000in}{0.000000in}}{%
\pgfpathmoveto{\pgfqpoint{0.000000in}{0.000000in}}%
\pgfpathlineto{\pgfqpoint{-0.048611in}{0.000000in}}%
\pgfusepath{stroke,fill}%
}%
\begin{pgfscope}%
\pgfsys@transformshift{0.557986in}{3.842377in}%
\pgfsys@useobject{currentmarker}{}%
\end{pgfscope}%
\end{pgfscope}%
\begin{pgfscope}%
\definecolor{textcolor}{rgb}{0.000000,0.000000,0.000000}%
\pgfsetstrokecolor{textcolor}%
\pgfsetfillcolor{textcolor}%
\pgftext[x=0.239884in,y=3.789615in,left,base]{\color{textcolor}\sffamily\fontsize{10.000000}{12.000000}\selectfont 0.2}%
\end{pgfscope}%
\begin{pgfscope}%
\pgfsetbuttcap%
\pgfsetroundjoin%
\definecolor{currentfill}{rgb}{0.000000,0.000000,0.000000}%
\pgfsetfillcolor{currentfill}%
\pgfsetlinewidth{0.803000pt}%
\definecolor{currentstroke}{rgb}{0.000000,0.000000,0.000000}%
\pgfsetstrokecolor{currentstroke}%
\pgfsetdash{}{0pt}%
\pgfsys@defobject{currentmarker}{\pgfqpoint{-0.048611in}{0.000000in}}{\pgfqpoint{0.000000in}{0.000000in}}{%
\pgfpathmoveto{\pgfqpoint{0.000000in}{0.000000in}}%
\pgfpathlineto{\pgfqpoint{-0.048611in}{0.000000in}}%
\pgfusepath{stroke,fill}%
}%
\begin{pgfscope}%
\pgfsys@transformshift{0.557986in}{4.134935in}%
\pgfsys@useobject{currentmarker}{}%
\end{pgfscope}%
\end{pgfscope}%
\begin{pgfscope}%
\definecolor{textcolor}{rgb}{0.000000,0.000000,0.000000}%
\pgfsetstrokecolor{textcolor}%
\pgfsetfillcolor{textcolor}%
\pgftext[x=0.239884in,y=4.082173in,left,base]{\color{textcolor}\sffamily\fontsize{10.000000}{12.000000}\selectfont 0.4}%
\end{pgfscope}%
\begin{pgfscope}%
\pgfsetbuttcap%
\pgfsetroundjoin%
\definecolor{currentfill}{rgb}{0.000000,0.000000,0.000000}%
\pgfsetfillcolor{currentfill}%
\pgfsetlinewidth{0.803000pt}%
\definecolor{currentstroke}{rgb}{0.000000,0.000000,0.000000}%
\pgfsetstrokecolor{currentstroke}%
\pgfsetdash{}{0pt}%
\pgfsys@defobject{currentmarker}{\pgfqpoint{-0.048611in}{0.000000in}}{\pgfqpoint{0.000000in}{0.000000in}}{%
\pgfpathmoveto{\pgfqpoint{0.000000in}{0.000000in}}%
\pgfpathlineto{\pgfqpoint{-0.048611in}{0.000000in}}%
\pgfusepath{stroke,fill}%
}%
\begin{pgfscope}%
\pgfsys@transformshift{0.557986in}{4.427493in}%
\pgfsys@useobject{currentmarker}{}%
\end{pgfscope}%
\end{pgfscope}%
\begin{pgfscope}%
\definecolor{textcolor}{rgb}{0.000000,0.000000,0.000000}%
\pgfsetstrokecolor{textcolor}%
\pgfsetfillcolor{textcolor}%
\pgftext[x=0.239884in,y=4.374731in,left,base]{\color{textcolor}\sffamily\fontsize{10.000000}{12.000000}\selectfont 0.6}%
\end{pgfscope}%
\begin{pgfscope}%
\pgfsetbuttcap%
\pgfsetroundjoin%
\definecolor{currentfill}{rgb}{0.000000,0.000000,0.000000}%
\pgfsetfillcolor{currentfill}%
\pgfsetlinewidth{0.803000pt}%
\definecolor{currentstroke}{rgb}{0.000000,0.000000,0.000000}%
\pgfsetstrokecolor{currentstroke}%
\pgfsetdash{}{0pt}%
\pgfsys@defobject{currentmarker}{\pgfqpoint{-0.048611in}{0.000000in}}{\pgfqpoint{0.000000in}{0.000000in}}{%
\pgfpathmoveto{\pgfqpoint{0.000000in}{0.000000in}}%
\pgfpathlineto{\pgfqpoint{-0.048611in}{0.000000in}}%
\pgfusepath{stroke,fill}%
}%
\begin{pgfscope}%
\pgfsys@transformshift{0.557986in}{4.720051in}%
\pgfsys@useobject{currentmarker}{}%
\end{pgfscope}%
\end{pgfscope}%
\begin{pgfscope}%
\definecolor{textcolor}{rgb}{0.000000,0.000000,0.000000}%
\pgfsetstrokecolor{textcolor}%
\pgfsetfillcolor{textcolor}%
\pgftext[x=0.239884in,y=4.667289in,left,base]{\color{textcolor}\sffamily\fontsize{10.000000}{12.000000}\selectfont 0.8}%
\end{pgfscope}%
\begin{pgfscope}%
\pgfsetbuttcap%
\pgfsetroundjoin%
\definecolor{currentfill}{rgb}{0.000000,0.000000,0.000000}%
\pgfsetfillcolor{currentfill}%
\pgfsetlinewidth{0.803000pt}%
\definecolor{currentstroke}{rgb}{0.000000,0.000000,0.000000}%
\pgfsetstrokecolor{currentstroke}%
\pgfsetdash{}{0pt}%
\pgfsys@defobject{currentmarker}{\pgfqpoint{-0.048611in}{0.000000in}}{\pgfqpoint{0.000000in}{0.000000in}}{%
\pgfpathmoveto{\pgfqpoint{0.000000in}{0.000000in}}%
\pgfpathlineto{\pgfqpoint{-0.048611in}{0.000000in}}%
\pgfusepath{stroke,fill}%
}%
\begin{pgfscope}%
\pgfsys@transformshift{0.557986in}{5.012609in}%
\pgfsys@useobject{currentmarker}{}%
\end{pgfscope}%
\end{pgfscope}%
\begin{pgfscope}%
\definecolor{textcolor}{rgb}{0.000000,0.000000,0.000000}%
\pgfsetstrokecolor{textcolor}%
\pgfsetfillcolor{textcolor}%
\pgftext[x=0.239884in,y=4.959847in,left,base]{\color{textcolor}\sffamily\fontsize{10.000000}{12.000000}\selectfont 1.0}%
\end{pgfscope}%
\begin{pgfscope}%
\pgfsetbuttcap%
\pgfsetroundjoin%
\definecolor{currentfill}{rgb}{0.000000,0.000000,0.000000}%
\pgfsetfillcolor{currentfill}%
\pgfsetlinewidth{0.803000pt}%
\definecolor{currentstroke}{rgb}{0.000000,0.000000,0.000000}%
\pgfsetstrokecolor{currentstroke}%
\pgfsetdash{}{0pt}%
\pgfsys@defobject{currentmarker}{\pgfqpoint{-0.048611in}{0.000000in}}{\pgfqpoint{0.000000in}{0.000000in}}{%
\pgfpathmoveto{\pgfqpoint{0.000000in}{0.000000in}}%
\pgfpathlineto{\pgfqpoint{-0.048611in}{0.000000in}}%
\pgfusepath{stroke,fill}%
}%
\begin{pgfscope}%
\pgfsys@transformshift{0.557986in}{5.305167in}%
\pgfsys@useobject{currentmarker}{}%
\end{pgfscope}%
\end{pgfscope}%
\begin{pgfscope}%
\definecolor{textcolor}{rgb}{0.000000,0.000000,0.000000}%
\pgfsetstrokecolor{textcolor}%
\pgfsetfillcolor{textcolor}%
\pgftext[x=0.239884in,y=5.252405in,left,base]{\color{textcolor}\sffamily\fontsize{10.000000}{12.000000}\selectfont 1.2}%
\end{pgfscope}%
\begin{pgfscope}%
\pgfsetbuttcap%
\pgfsetroundjoin%
\definecolor{currentfill}{rgb}{0.000000,0.000000,0.000000}%
\pgfsetfillcolor{currentfill}%
\pgfsetlinewidth{0.803000pt}%
\definecolor{currentstroke}{rgb}{0.000000,0.000000,0.000000}%
\pgfsetstrokecolor{currentstroke}%
\pgfsetdash{}{0pt}%
\pgfsys@defobject{currentmarker}{\pgfqpoint{-0.048611in}{0.000000in}}{\pgfqpoint{0.000000in}{0.000000in}}{%
\pgfpathmoveto{\pgfqpoint{0.000000in}{0.000000in}}%
\pgfpathlineto{\pgfqpoint{-0.048611in}{0.000000in}}%
\pgfusepath{stroke,fill}%
}%
\begin{pgfscope}%
\pgfsys@transformshift{0.557986in}{5.597725in}%
\pgfsys@useobject{currentmarker}{}%
\end{pgfscope}%
\end{pgfscope}%
\begin{pgfscope}%
\definecolor{textcolor}{rgb}{0.000000,0.000000,0.000000}%
\pgfsetstrokecolor{textcolor}%
\pgfsetfillcolor{textcolor}%
\pgftext[x=0.239884in,y=5.544963in,left,base]{\color{textcolor}\sffamily\fontsize{10.000000}{12.000000}\selectfont 1.4}%
\end{pgfscope}%
\begin{pgfscope}%
\pgfpathrectangle{\pgfqpoint{0.557986in}{3.363889in}}{\pgfqpoint{3.318403in}{2.263889in}}%
\pgfusepath{clip}%
\pgfsetrectcap%
\pgfsetroundjoin%
\pgfsetlinewidth{1.505625pt}%
\definecolor{currentstroke}{rgb}{0.121569,0.466667,0.705882}%
\pgfsetstrokecolor{currentstroke}%
\pgfsetdash{}{0pt}%
\pgfpathmoveto{\pgfqpoint{0.708823in}{3.839424in}}%
\pgfpathlineto{\pgfqpoint{0.859609in}{3.895280in}}%
\pgfpathlineto{\pgfqpoint{1.010395in}{4.628915in}}%
\pgfpathlineto{\pgfqpoint{1.161181in}{4.801410in}}%
\pgfpathlineto{\pgfqpoint{1.311967in}{4.655646in}}%
\pgfpathlineto{\pgfqpoint{1.462753in}{4.916088in}}%
\pgfpathlineto{\pgfqpoint{1.613539in}{4.879613in}}%
\pgfpathlineto{\pgfqpoint{1.764325in}{4.690595in}}%
\pgfpathlineto{\pgfqpoint{1.915111in}{4.999268in}}%
\pgfpathlineto{\pgfqpoint{2.065897in}{5.011569in}}%
\pgfpathlineto{\pgfqpoint{2.216683in}{5.011487in}}%
\pgfpathlineto{\pgfqpoint{2.367469in}{5.011527in}}%
\pgfpathlineto{\pgfqpoint{2.518255in}{5.011511in}}%
\pgfpathlineto{\pgfqpoint{2.669041in}{5.011527in}}%
\pgfpathlineto{\pgfqpoint{2.819827in}{5.011524in}}%
\pgfpathlineto{\pgfqpoint{2.970613in}{5.011547in}}%
\pgfpathlineto{\pgfqpoint{3.121399in}{5.011541in}}%
\pgfpathlineto{\pgfqpoint{3.272185in}{5.011555in}}%
\pgfpathlineto{\pgfqpoint{3.422971in}{5.011543in}}%
\pgfpathlineto{\pgfqpoint{3.573756in}{5.011551in}}%
\pgfpathlineto{\pgfqpoint{3.724542in}{5.011551in}}%
\pgfusepath{stroke}%
\end{pgfscope}%
\begin{pgfscope}%
\pgfpathrectangle{\pgfqpoint{0.557986in}{3.363889in}}{\pgfqpoint{3.318403in}{2.263889in}}%
\pgfusepath{clip}%
\pgfsetrectcap%
\pgfsetroundjoin%
\pgfsetlinewidth{1.505625pt}%
\definecolor{currentstroke}{rgb}{1.000000,0.498039,0.054902}%
\pgfsetstrokecolor{currentstroke}%
\pgfsetdash{}{0pt}%
\pgfpathmoveto{\pgfqpoint{0.708823in}{3.900071in}}%
\pgfpathlineto{\pgfqpoint{0.859609in}{4.010608in}}%
\pgfpathlineto{\pgfqpoint{1.010395in}{4.198230in}}%
\pgfpathlineto{\pgfqpoint{1.161181in}{4.861184in}}%
\pgfpathlineto{\pgfqpoint{1.311967in}{4.504774in}}%
\pgfpathlineto{\pgfqpoint{1.462753in}{4.662495in}}%
\pgfpathlineto{\pgfqpoint{1.613539in}{4.686331in}}%
\pgfpathlineto{\pgfqpoint{1.764325in}{4.993549in}}%
\pgfpathlineto{\pgfqpoint{1.915111in}{4.992256in}}%
\pgfpathlineto{\pgfqpoint{2.065897in}{4.992991in}}%
\pgfpathlineto{\pgfqpoint{2.216683in}{4.993445in}}%
\pgfpathlineto{\pgfqpoint{2.367469in}{4.992858in}}%
\pgfpathlineto{\pgfqpoint{2.518255in}{4.992895in}}%
\pgfpathlineto{\pgfqpoint{2.669041in}{4.992941in}}%
\pgfpathlineto{\pgfqpoint{2.819827in}{4.992881in}}%
\pgfpathlineto{\pgfqpoint{2.970613in}{4.992896in}}%
\pgfpathlineto{\pgfqpoint{3.121399in}{4.992917in}}%
\pgfpathlineto{\pgfqpoint{3.272185in}{4.992870in}}%
\pgfpathlineto{\pgfqpoint{3.422971in}{4.992896in}}%
\pgfpathlineto{\pgfqpoint{3.573756in}{4.992926in}}%
\pgfpathlineto{\pgfqpoint{3.724542in}{4.992870in}}%
\pgfusepath{stroke}%
\end{pgfscope}%
\begin{pgfscope}%
\pgfpathrectangle{\pgfqpoint{0.557986in}{3.363889in}}{\pgfqpoint{3.318403in}{2.263889in}}%
\pgfusepath{clip}%
\pgfsetrectcap%
\pgfsetroundjoin%
\pgfsetlinewidth{1.505625pt}%
\definecolor{currentstroke}{rgb}{0.172549,0.627451,0.172549}%
\pgfsetstrokecolor{currentstroke}%
\pgfsetdash{}{0pt}%
\pgfpathmoveto{\pgfqpoint{0.708823in}{3.885386in}}%
\pgfpathlineto{\pgfqpoint{0.859609in}{3.928319in}}%
\pgfpathlineto{\pgfqpoint{1.010395in}{3.944716in}}%
\pgfpathlineto{\pgfqpoint{1.161181in}{4.209986in}}%
\pgfpathlineto{\pgfqpoint{1.311967in}{4.561953in}}%
\pgfpathlineto{\pgfqpoint{1.462753in}{4.621749in}}%
\pgfpathlineto{\pgfqpoint{1.613539in}{4.886221in}}%
\pgfpathlineto{\pgfqpoint{1.764325in}{4.889795in}}%
\pgfpathlineto{\pgfqpoint{1.915111in}{4.881060in}}%
\pgfpathlineto{\pgfqpoint{2.065897in}{4.886999in}}%
\pgfpathlineto{\pgfqpoint{2.216683in}{4.887890in}}%
\pgfpathlineto{\pgfqpoint{2.367469in}{4.879207in}}%
\pgfpathlineto{\pgfqpoint{2.518255in}{4.883891in}}%
\pgfpathlineto{\pgfqpoint{2.669041in}{4.884801in}}%
\pgfpathlineto{\pgfqpoint{2.819827in}{4.882169in}}%
\pgfpathlineto{\pgfqpoint{2.970613in}{4.882493in}}%
\pgfpathlineto{\pgfqpoint{3.121399in}{4.882825in}}%
\pgfpathlineto{\pgfqpoint{3.272185in}{4.882584in}}%
\pgfpathlineto{\pgfqpoint{3.422971in}{4.883307in}}%
\pgfpathlineto{\pgfqpoint{3.573756in}{4.883094in}}%
\pgfpathlineto{\pgfqpoint{3.724542in}{4.883098in}}%
\pgfusepath{stroke}%
\end{pgfscope}%
\begin{pgfscope}%
\pgfpathrectangle{\pgfqpoint{0.557986in}{3.363889in}}{\pgfqpoint{3.318403in}{2.263889in}}%
\pgfusepath{clip}%
\pgfsetrectcap%
\pgfsetroundjoin%
\pgfsetlinewidth{1.505625pt}%
\definecolor{currentstroke}{rgb}{0.839216,0.152941,0.156863}%
\pgfsetstrokecolor{currentstroke}%
\pgfsetdash{}{0pt}%
\pgfpathmoveto{\pgfqpoint{0.708823in}{3.769642in}}%
\pgfpathlineto{\pgfqpoint{0.859609in}{4.130453in}}%
\pgfpathlineto{\pgfqpoint{1.010395in}{3.875313in}}%
\pgfpathlineto{\pgfqpoint{1.161181in}{3.988322in}}%
\pgfpathlineto{\pgfqpoint{1.311967in}{4.316108in}}%
\pgfpathlineto{\pgfqpoint{1.462753in}{4.225938in}}%
\pgfpathlineto{\pgfqpoint{1.613539in}{4.033362in}}%
\pgfpathlineto{\pgfqpoint{1.764325in}{4.175058in}}%
\pgfpathlineto{\pgfqpoint{1.915111in}{4.130992in}}%
\pgfpathlineto{\pgfqpoint{2.065897in}{4.113252in}}%
\pgfpathlineto{\pgfqpoint{2.216683in}{4.094158in}}%
\pgfpathlineto{\pgfqpoint{2.367469in}{4.153160in}}%
\pgfpathlineto{\pgfqpoint{2.518255in}{4.115981in}}%
\pgfpathlineto{\pgfqpoint{2.669041in}{4.117308in}}%
\pgfpathlineto{\pgfqpoint{2.819827in}{4.124745in}}%
\pgfpathlineto{\pgfqpoint{2.970613in}{4.113389in}}%
\pgfpathlineto{\pgfqpoint{3.121399in}{4.135437in}}%
\pgfpathlineto{\pgfqpoint{3.272185in}{4.113997in}}%
\pgfpathlineto{\pgfqpoint{3.422971in}{4.120470in}}%
\pgfpathlineto{\pgfqpoint{3.573756in}{4.118492in}}%
\pgfpathlineto{\pgfqpoint{3.724542in}{4.119344in}}%
\pgfusepath{stroke}%
\end{pgfscope}%
\begin{pgfscope}%
\pgfpathrectangle{\pgfqpoint{0.557986in}{3.363889in}}{\pgfqpoint{3.318403in}{2.263889in}}%
\pgfusepath{clip}%
\pgfsetrectcap%
\pgfsetroundjoin%
\pgfsetlinewidth{1.505625pt}%
\definecolor{currentstroke}{rgb}{0.580392,0.403922,0.741176}%
\pgfsetstrokecolor{currentstroke}%
\pgfsetdash{}{0pt}%
\pgfpathmoveto{\pgfqpoint{0.708823in}{3.710452in}}%
\pgfpathlineto{\pgfqpoint{0.859609in}{3.891680in}}%
\pgfpathlineto{\pgfqpoint{1.010395in}{3.809931in}}%
\pgfpathlineto{\pgfqpoint{1.161181in}{3.822166in}}%
\pgfpathlineto{\pgfqpoint{1.311967in}{3.828266in}}%
\pgfpathlineto{\pgfqpoint{1.462753in}{3.781314in}}%
\pgfpathlineto{\pgfqpoint{1.613539in}{3.804810in}}%
\pgfpathlineto{\pgfqpoint{1.764325in}{3.803899in}}%
\pgfpathlineto{\pgfqpoint{1.915111in}{3.808713in}}%
\pgfpathlineto{\pgfqpoint{2.065897in}{3.810884in}}%
\pgfpathlineto{\pgfqpoint{2.216683in}{3.798051in}}%
\pgfpathlineto{\pgfqpoint{2.367469in}{3.794654in}}%
\pgfpathlineto{\pgfqpoint{2.518255in}{3.806475in}}%
\pgfpathlineto{\pgfqpoint{2.669041in}{3.797010in}}%
\pgfpathlineto{\pgfqpoint{2.819827in}{3.802214in}}%
\pgfpathlineto{\pgfqpoint{2.970613in}{3.805823in}}%
\pgfpathlineto{\pgfqpoint{3.121399in}{3.799601in}}%
\pgfpathlineto{\pgfqpoint{3.272185in}{3.800664in}}%
\pgfpathlineto{\pgfqpoint{3.422971in}{3.800410in}}%
\pgfpathlineto{\pgfqpoint{3.573756in}{3.801209in}}%
\pgfpathlineto{\pgfqpoint{3.724542in}{3.801497in}}%
\pgfusepath{stroke}%
\end{pgfscope}%
\begin{pgfscope}%
\pgfsetrectcap%
\pgfsetmiterjoin%
\pgfsetlinewidth{0.803000pt}%
\definecolor{currentstroke}{rgb}{0.000000,0.000000,0.000000}%
\pgfsetstrokecolor{currentstroke}%
\pgfsetdash{}{0pt}%
\pgfpathmoveto{\pgfqpoint{0.557986in}{3.363889in}}%
\pgfpathlineto{\pgfqpoint{0.557986in}{5.627778in}}%
\pgfusepath{stroke}%
\end{pgfscope}%
\begin{pgfscope}%
\pgfsetrectcap%
\pgfsetmiterjoin%
\pgfsetlinewidth{0.803000pt}%
\definecolor{currentstroke}{rgb}{0.000000,0.000000,0.000000}%
\pgfsetstrokecolor{currentstroke}%
\pgfsetdash{}{0pt}%
\pgfpathmoveto{\pgfqpoint{3.876389in}{3.363889in}}%
\pgfpathlineto{\pgfqpoint{3.876389in}{5.627778in}}%
\pgfusepath{stroke}%
\end{pgfscope}%
\begin{pgfscope}%
\pgfsetrectcap%
\pgfsetmiterjoin%
\pgfsetlinewidth{0.803000pt}%
\definecolor{currentstroke}{rgb}{0.000000,0.000000,0.000000}%
\pgfsetstrokecolor{currentstroke}%
\pgfsetdash{}{0pt}%
\pgfpathmoveto{\pgfqpoint{0.557986in}{3.363889in}}%
\pgfpathlineto{\pgfqpoint{3.876389in}{3.363889in}}%
\pgfusepath{stroke}%
\end{pgfscope}%
\begin{pgfscope}%
\pgfsetrectcap%
\pgfsetmiterjoin%
\pgfsetlinewidth{0.803000pt}%
\definecolor{currentstroke}{rgb}{0.000000,0.000000,0.000000}%
\pgfsetstrokecolor{currentstroke}%
\pgfsetdash{}{0pt}%
\pgfpathmoveto{\pgfqpoint{0.557986in}{5.627778in}}%
\pgfpathlineto{\pgfqpoint{3.876389in}{5.627778in}}%
\pgfusepath{stroke}%
\end{pgfscope}%
\begin{pgfscope}%
\definecolor{textcolor}{rgb}{0.000000,0.000000,0.000000}%
\pgfsetstrokecolor{textcolor}%
\pgfsetfillcolor{textcolor}%
\pgftext[x=2.217187in,y=5.711111in,,base]{\color{textcolor}\sffamily\fontsize{12.000000}{14.400000}\selectfont \(\displaystyle  N = 16 \)}%
\end{pgfscope}%
\begin{pgfscope}%
\pgfsetbuttcap%
\pgfsetmiterjoin%
\definecolor{currentfill}{rgb}{1.000000,1.000000,1.000000}%
\pgfsetfillcolor{currentfill}%
\pgfsetlinewidth{0.000000pt}%
\definecolor{currentstroke}{rgb}{0.000000,0.000000,0.000000}%
\pgfsetstrokecolor{currentstroke}%
\pgfsetstrokeopacity{0.000000}%
\pgfsetdash{}{0pt}%
\pgfpathmoveto{\pgfqpoint{4.482986in}{3.363889in}}%
\pgfpathlineto{\pgfqpoint{7.801389in}{3.363889in}}%
\pgfpathlineto{\pgfqpoint{7.801389in}{5.627778in}}%
\pgfpathlineto{\pgfqpoint{4.482986in}{5.627778in}}%
\pgfpathclose%
\pgfusepath{fill}%
\end{pgfscope}%
\begin{pgfscope}%
\pgfpathrectangle{\pgfqpoint{4.482986in}{3.363889in}}{\pgfqpoint{3.318403in}{2.263889in}}%
\pgfusepath{clip}%
\pgfsetbuttcap%
\pgfsetroundjoin%
\definecolor{currentfill}{rgb}{0.121569,0.466667,0.705882}%
\pgfsetfillcolor{currentfill}%
\pgfsetlinewidth{1.003750pt}%
\definecolor{currentstroke}{rgb}{0.121569,0.466667,0.705882}%
\pgfsetstrokecolor{currentstroke}%
\pgfsetdash{}{0pt}%
\pgfsys@defobject{currentmarker}{\pgfqpoint{-0.009821in}{-0.009821in}}{\pgfqpoint{0.009821in}{0.009821in}}{%
\pgfpathmoveto{\pgfqpoint{0.000000in}{-0.009821in}}%
\pgfpathcurveto{\pgfqpoint{0.002605in}{-0.009821in}}{\pgfqpoint{0.005103in}{-0.008786in}}{\pgfqpoint{0.006944in}{-0.006944in}}%
\pgfpathcurveto{\pgfqpoint{0.008786in}{-0.005103in}}{\pgfqpoint{0.009821in}{-0.002605in}}{\pgfqpoint{0.009821in}{0.000000in}}%
\pgfpathcurveto{\pgfqpoint{0.009821in}{0.002605in}}{\pgfqpoint{0.008786in}{0.005103in}}{\pgfqpoint{0.006944in}{0.006944in}}%
\pgfpathcurveto{\pgfqpoint{0.005103in}{0.008786in}}{\pgfqpoint{0.002605in}{0.009821in}}{\pgfqpoint{0.000000in}{0.009821in}}%
\pgfpathcurveto{\pgfqpoint{-0.002605in}{0.009821in}}{\pgfqpoint{-0.005103in}{0.008786in}}{\pgfqpoint{-0.006944in}{0.006944in}}%
\pgfpathcurveto{\pgfqpoint{-0.008786in}{0.005103in}}{\pgfqpoint{-0.009821in}{0.002605in}}{\pgfqpoint{-0.009821in}{0.000000in}}%
\pgfpathcurveto{\pgfqpoint{-0.009821in}{-0.002605in}}{\pgfqpoint{-0.008786in}{-0.005103in}}{\pgfqpoint{-0.006944in}{-0.006944in}}%
\pgfpathcurveto{\pgfqpoint{-0.005103in}{-0.008786in}}{\pgfqpoint{-0.002605in}{-0.009821in}}{\pgfqpoint{0.000000in}{-0.009821in}}%
\pgfpathclose%
\pgfusepath{stroke,fill}%
}%
\begin{pgfscope}%
\pgfsys@transformshift{4.633823in}{3.764997in}%
\pgfsys@useobject{currentmarker}{}%
\end{pgfscope}%
\begin{pgfscope}%
\pgfsys@transformshift{4.784609in}{3.749698in}%
\pgfsys@useobject{currentmarker}{}%
\end{pgfscope}%
\begin{pgfscope}%
\pgfsys@transformshift{4.935395in}{3.790628in}%
\pgfsys@useobject{currentmarker}{}%
\end{pgfscope}%
\begin{pgfscope}%
\pgfsys@transformshift{5.086181in}{3.894543in}%
\pgfsys@useobject{currentmarker}{}%
\end{pgfscope}%
\begin{pgfscope}%
\pgfsys@transformshift{5.236967in}{3.926825in}%
\pgfsys@useobject{currentmarker}{}%
\end{pgfscope}%
\begin{pgfscope}%
\pgfsys@transformshift{5.387753in}{4.151888in}%
\pgfsys@useobject{currentmarker}{}%
\end{pgfscope}%
\begin{pgfscope}%
\pgfsys@transformshift{5.538539in}{4.192528in}%
\pgfsys@useobject{currentmarker}{}%
\end{pgfscope}%
\begin{pgfscope}%
\pgfsys@transformshift{5.689325in}{4.633324in}%
\pgfsys@useobject{currentmarker}{}%
\end{pgfscope}%
\begin{pgfscope}%
\pgfsys@transformshift{5.840111in}{4.399581in}%
\pgfsys@useobject{currentmarker}{}%
\end{pgfscope}%
\begin{pgfscope}%
\pgfsys@transformshift{5.990897in}{4.773934in}%
\pgfsys@useobject{currentmarker}{}%
\end{pgfscope}%
\begin{pgfscope}%
\pgfsys@transformshift{6.141683in}{4.433524in}%
\pgfsys@useobject{currentmarker}{}%
\end{pgfscope}%
\begin{pgfscope}%
\pgfsys@transformshift{6.292469in}{4.762273in}%
\pgfsys@useobject{currentmarker}{}%
\end{pgfscope}%
\begin{pgfscope}%
\pgfsys@transformshift{6.443255in}{5.055804in}%
\pgfsys@useobject{currentmarker}{}%
\end{pgfscope}%
\begin{pgfscope}%
\pgfsys@transformshift{6.594041in}{4.567356in}%
\pgfsys@useobject{currentmarker}{}%
\end{pgfscope}%
\begin{pgfscope}%
\pgfsys@transformshift{6.744827in}{5.055779in}%
\pgfsys@useobject{currentmarker}{}%
\end{pgfscope}%
\begin{pgfscope}%
\pgfsys@transformshift{6.895613in}{5.055784in}%
\pgfsys@useobject{currentmarker}{}%
\end{pgfscope}%
\begin{pgfscope}%
\pgfsys@transformshift{7.046399in}{5.040685in}%
\pgfsys@useobject{currentmarker}{}%
\end{pgfscope}%
\begin{pgfscope}%
\pgfsys@transformshift{7.197185in}{5.055773in}%
\pgfsys@useobject{currentmarker}{}%
\end{pgfscope}%
\begin{pgfscope}%
\pgfsys@transformshift{7.347971in}{5.055789in}%
\pgfsys@useobject{currentmarker}{}%
\end{pgfscope}%
\begin{pgfscope}%
\pgfsys@transformshift{7.498756in}{5.055795in}%
\pgfsys@useobject{currentmarker}{}%
\end{pgfscope}%
\begin{pgfscope}%
\pgfsys@transformshift{7.649542in}{5.055788in}%
\pgfsys@useobject{currentmarker}{}%
\end{pgfscope}%
\end{pgfscope}%
\begin{pgfscope}%
\pgfpathrectangle{\pgfqpoint{4.482986in}{3.363889in}}{\pgfqpoint{3.318403in}{2.263889in}}%
\pgfusepath{clip}%
\pgfsetbuttcap%
\pgfsetroundjoin%
\definecolor{currentfill}{rgb}{0.121569,0.466667,0.705882}%
\pgfsetfillcolor{currentfill}%
\pgfsetfillopacity{0.300000}%
\pgfsetlinewidth{1.003750pt}%
\definecolor{currentstroke}{rgb}{0.121569,0.466667,0.705882}%
\pgfsetstrokecolor{currentstroke}%
\pgfsetstrokeopacity{0.300000}%
\pgfsetdash{}{0pt}%
\pgfpathmoveto{\pgfqpoint{4.633823in}{3.863167in}}%
\pgfpathlineto{\pgfqpoint{4.633823in}{3.666828in}}%
\pgfpathlineto{\pgfqpoint{4.784609in}{3.687143in}}%
\pgfpathlineto{\pgfqpoint{4.935395in}{3.632686in}}%
\pgfpathlineto{\pgfqpoint{5.086181in}{3.525142in}}%
\pgfpathlineto{\pgfqpoint{5.236967in}{3.714503in}}%
\pgfpathlineto{\pgfqpoint{5.387753in}{3.716250in}}%
\pgfpathlineto{\pgfqpoint{5.538539in}{3.466793in}}%
\pgfpathlineto{\pgfqpoint{5.689325in}{3.873495in}}%
\pgfpathlineto{\pgfqpoint{5.840111in}{3.582275in}}%
\pgfpathlineto{\pgfqpoint{5.990897in}{4.042022in}}%
\pgfpathlineto{\pgfqpoint{6.141683in}{3.497167in}}%
\pgfpathlineto{\pgfqpoint{6.292469in}{3.999672in}}%
\pgfpathlineto{\pgfqpoint{6.443255in}{5.055725in}}%
\pgfpathlineto{\pgfqpoint{6.594041in}{3.832256in}}%
\pgfpathlineto{\pgfqpoint{6.744827in}{5.055744in}}%
\pgfpathlineto{\pgfqpoint{6.895613in}{5.055756in}}%
\pgfpathlineto{\pgfqpoint{7.046399in}{5.001424in}}%
\pgfpathlineto{\pgfqpoint{7.197185in}{5.055757in}}%
\pgfpathlineto{\pgfqpoint{7.347971in}{5.055758in}}%
\pgfpathlineto{\pgfqpoint{7.498756in}{5.055791in}}%
\pgfpathlineto{\pgfqpoint{7.649542in}{5.055781in}}%
\pgfpathlineto{\pgfqpoint{7.649542in}{5.055796in}}%
\pgfpathlineto{\pgfqpoint{7.649542in}{5.055796in}}%
\pgfpathlineto{\pgfqpoint{7.498756in}{5.055800in}}%
\pgfpathlineto{\pgfqpoint{7.347971in}{5.055820in}}%
\pgfpathlineto{\pgfqpoint{7.197185in}{5.055789in}}%
\pgfpathlineto{\pgfqpoint{7.046399in}{5.079947in}}%
\pgfpathlineto{\pgfqpoint{6.895613in}{5.055812in}}%
\pgfpathlineto{\pgfqpoint{6.744827in}{5.055814in}}%
\pgfpathlineto{\pgfqpoint{6.594041in}{5.302457in}}%
\pgfpathlineto{\pgfqpoint{6.443255in}{5.055883in}}%
\pgfpathlineto{\pgfqpoint{6.292469in}{5.524874in}}%
\pgfpathlineto{\pgfqpoint{6.141683in}{5.369881in}}%
\pgfpathlineto{\pgfqpoint{5.990897in}{5.505845in}}%
\pgfpathlineto{\pgfqpoint{5.840111in}{5.216888in}}%
\pgfpathlineto{\pgfqpoint{5.689325in}{5.393152in}}%
\pgfpathlineto{\pgfqpoint{5.538539in}{4.918263in}}%
\pgfpathlineto{\pgfqpoint{5.387753in}{4.587526in}}%
\pgfpathlineto{\pgfqpoint{5.236967in}{4.139147in}}%
\pgfpathlineto{\pgfqpoint{5.086181in}{4.263944in}}%
\pgfpathlineto{\pgfqpoint{4.935395in}{3.948570in}}%
\pgfpathlineto{\pgfqpoint{4.784609in}{3.812253in}}%
\pgfpathlineto{\pgfqpoint{4.633823in}{3.863167in}}%
\pgfpathclose%
\pgfusepath{stroke,fill}%
\end{pgfscope}%
\begin{pgfscope}%
\pgfpathrectangle{\pgfqpoint{4.482986in}{3.363889in}}{\pgfqpoint{3.318403in}{2.263889in}}%
\pgfusepath{clip}%
\pgfsetbuttcap%
\pgfsetroundjoin%
\definecolor{currentfill}{rgb}{1.000000,0.498039,0.054902}%
\pgfsetfillcolor{currentfill}%
\pgfsetlinewidth{1.003750pt}%
\definecolor{currentstroke}{rgb}{1.000000,0.498039,0.054902}%
\pgfsetstrokecolor{currentstroke}%
\pgfsetdash{}{0pt}%
\pgfsys@defobject{currentmarker}{\pgfqpoint{-0.009821in}{-0.009821in}}{\pgfqpoint{0.009821in}{0.009821in}}{%
\pgfpathmoveto{\pgfqpoint{0.000000in}{-0.009821in}}%
\pgfpathcurveto{\pgfqpoint{0.002605in}{-0.009821in}}{\pgfqpoint{0.005103in}{-0.008786in}}{\pgfqpoint{0.006944in}{-0.006944in}}%
\pgfpathcurveto{\pgfqpoint{0.008786in}{-0.005103in}}{\pgfqpoint{0.009821in}{-0.002605in}}{\pgfqpoint{0.009821in}{0.000000in}}%
\pgfpathcurveto{\pgfqpoint{0.009821in}{0.002605in}}{\pgfqpoint{0.008786in}{0.005103in}}{\pgfqpoint{0.006944in}{0.006944in}}%
\pgfpathcurveto{\pgfqpoint{0.005103in}{0.008786in}}{\pgfqpoint{0.002605in}{0.009821in}}{\pgfqpoint{0.000000in}{0.009821in}}%
\pgfpathcurveto{\pgfqpoint{-0.002605in}{0.009821in}}{\pgfqpoint{-0.005103in}{0.008786in}}{\pgfqpoint{-0.006944in}{0.006944in}}%
\pgfpathcurveto{\pgfqpoint{-0.008786in}{0.005103in}}{\pgfqpoint{-0.009821in}{0.002605in}}{\pgfqpoint{-0.009821in}{0.000000in}}%
\pgfpathcurveto{\pgfqpoint{-0.009821in}{-0.002605in}}{\pgfqpoint{-0.008786in}{-0.005103in}}{\pgfqpoint{-0.006944in}{-0.006944in}}%
\pgfpathcurveto{\pgfqpoint{-0.005103in}{-0.008786in}}{\pgfqpoint{-0.002605in}{-0.009821in}}{\pgfqpoint{0.000000in}{-0.009821in}}%
\pgfpathclose%
\pgfusepath{stroke,fill}%
}%
\begin{pgfscope}%
\pgfsys@transformshift{4.633823in}{3.791985in}%
\pgfsys@useobject{currentmarker}{}%
\end{pgfscope}%
\begin{pgfscope}%
\pgfsys@transformshift{4.784609in}{3.801622in}%
\pgfsys@useobject{currentmarker}{}%
\end{pgfscope}%
\begin{pgfscope}%
\pgfsys@transformshift{4.935395in}{3.812519in}%
\pgfsys@useobject{currentmarker}{}%
\end{pgfscope}%
\begin{pgfscope}%
\pgfsys@transformshift{5.086181in}{3.806716in}%
\pgfsys@useobject{currentmarker}{}%
\end{pgfscope}%
\begin{pgfscope}%
\pgfsys@transformshift{5.236967in}{3.948761in}%
\pgfsys@useobject{currentmarker}{}%
\end{pgfscope}%
\begin{pgfscope}%
\pgfsys@transformshift{5.387753in}{3.951914in}%
\pgfsys@useobject{currentmarker}{}%
\end{pgfscope}%
\begin{pgfscope}%
\pgfsys@transformshift{5.538539in}{3.974789in}%
\pgfsys@useobject{currentmarker}{}%
\end{pgfscope}%
\begin{pgfscope}%
\pgfsys@transformshift{5.689325in}{4.399587in}%
\pgfsys@useobject{currentmarker}{}%
\end{pgfscope}%
\begin{pgfscope}%
\pgfsys@transformshift{5.840111in}{4.390414in}%
\pgfsys@useobject{currentmarker}{}%
\end{pgfscope}%
\begin{pgfscope}%
\pgfsys@transformshift{5.990897in}{4.709515in}%
\pgfsys@useobject{currentmarker}{}%
\end{pgfscope}%
\begin{pgfscope}%
\pgfsys@transformshift{6.141683in}{4.737247in}%
\pgfsys@useobject{currentmarker}{}%
\end{pgfscope}%
\begin{pgfscope}%
\pgfsys@transformshift{6.292469in}{5.036418in}%
\pgfsys@useobject{currentmarker}{}%
\end{pgfscope}%
\begin{pgfscope}%
\pgfsys@transformshift{6.443255in}{4.984157in}%
\pgfsys@useobject{currentmarker}{}%
\end{pgfscope}%
\begin{pgfscope}%
\pgfsys@transformshift{6.594041in}{4.887584in}%
\pgfsys@useobject{currentmarker}{}%
\end{pgfscope}%
\begin{pgfscope}%
\pgfsys@transformshift{6.744827in}{5.038014in}%
\pgfsys@useobject{currentmarker}{}%
\end{pgfscope}%
\begin{pgfscope}%
\pgfsys@transformshift{6.895613in}{5.037999in}%
\pgfsys@useobject{currentmarker}{}%
\end{pgfscope}%
\begin{pgfscope}%
\pgfsys@transformshift{7.046399in}{5.037887in}%
\pgfsys@useobject{currentmarker}{}%
\end{pgfscope}%
\begin{pgfscope}%
\pgfsys@transformshift{7.197185in}{5.037968in}%
\pgfsys@useobject{currentmarker}{}%
\end{pgfscope}%
\begin{pgfscope}%
\pgfsys@transformshift{7.347971in}{5.037860in}%
\pgfsys@useobject{currentmarker}{}%
\end{pgfscope}%
\begin{pgfscope}%
\pgfsys@transformshift{7.498756in}{5.037874in}%
\pgfsys@useobject{currentmarker}{}%
\end{pgfscope}%
\begin{pgfscope}%
\pgfsys@transformshift{7.649542in}{5.037922in}%
\pgfsys@useobject{currentmarker}{}%
\end{pgfscope}%
\end{pgfscope}%
\begin{pgfscope}%
\pgfpathrectangle{\pgfqpoint{4.482986in}{3.363889in}}{\pgfqpoint{3.318403in}{2.263889in}}%
\pgfusepath{clip}%
\pgfsetbuttcap%
\pgfsetroundjoin%
\definecolor{currentfill}{rgb}{1.000000,0.498039,0.054902}%
\pgfsetfillcolor{currentfill}%
\pgfsetfillopacity{0.300000}%
\pgfsetlinewidth{1.003750pt}%
\definecolor{currentstroke}{rgb}{1.000000,0.498039,0.054902}%
\pgfsetstrokecolor{currentstroke}%
\pgfsetstrokeopacity{0.300000}%
\pgfsetdash{}{0pt}%
\pgfpathmoveto{\pgfqpoint{4.633823in}{3.971940in}}%
\pgfpathlineto{\pgfqpoint{4.633823in}{3.612030in}}%
\pgfpathlineto{\pgfqpoint{4.784609in}{3.570440in}}%
\pgfpathlineto{\pgfqpoint{4.935395in}{3.714190in}}%
\pgfpathlineto{\pgfqpoint{5.086181in}{3.615441in}}%
\pgfpathlineto{\pgfqpoint{5.236967in}{3.772842in}}%
\pgfpathlineto{\pgfqpoint{5.387753in}{3.862363in}}%
\pgfpathlineto{\pgfqpoint{5.538539in}{3.598416in}}%
\pgfpathlineto{\pgfqpoint{5.689325in}{3.770700in}}%
\pgfpathlineto{\pgfqpoint{5.840111in}{3.826495in}}%
\pgfpathlineto{\pgfqpoint{5.990897in}{3.946586in}}%
\pgfpathlineto{\pgfqpoint{6.141683in}{4.042612in}}%
\pgfpathlineto{\pgfqpoint{6.292469in}{5.031927in}}%
\pgfpathlineto{\pgfqpoint{6.443255in}{4.845125in}}%
\pgfpathlineto{\pgfqpoint{6.594041in}{4.499202in}}%
\pgfpathlineto{\pgfqpoint{6.744827in}{5.037960in}}%
\pgfpathlineto{\pgfqpoint{6.895613in}{5.037894in}}%
\pgfpathlineto{\pgfqpoint{7.046399in}{5.037701in}}%
\pgfpathlineto{\pgfqpoint{7.197185in}{5.037856in}}%
\pgfpathlineto{\pgfqpoint{7.347971in}{5.037828in}}%
\pgfpathlineto{\pgfqpoint{7.498756in}{5.037814in}}%
\pgfpathlineto{\pgfqpoint{7.649542in}{5.037879in}}%
\pgfpathlineto{\pgfqpoint{7.649542in}{5.037966in}}%
\pgfpathlineto{\pgfqpoint{7.649542in}{5.037966in}}%
\pgfpathlineto{\pgfqpoint{7.498756in}{5.037933in}}%
\pgfpathlineto{\pgfqpoint{7.347971in}{5.037893in}}%
\pgfpathlineto{\pgfqpoint{7.197185in}{5.038080in}}%
\pgfpathlineto{\pgfqpoint{7.046399in}{5.038073in}}%
\pgfpathlineto{\pgfqpoint{6.895613in}{5.038104in}}%
\pgfpathlineto{\pgfqpoint{6.744827in}{5.038068in}}%
\pgfpathlineto{\pgfqpoint{6.594041in}{5.275966in}}%
\pgfpathlineto{\pgfqpoint{6.443255in}{5.123188in}}%
\pgfpathlineto{\pgfqpoint{6.292469in}{5.040910in}}%
\pgfpathlineto{\pgfqpoint{6.141683in}{5.431882in}}%
\pgfpathlineto{\pgfqpoint{5.990897in}{5.472445in}}%
\pgfpathlineto{\pgfqpoint{5.840111in}{4.954332in}}%
\pgfpathlineto{\pgfqpoint{5.689325in}{5.028473in}}%
\pgfpathlineto{\pgfqpoint{5.538539in}{4.351162in}}%
\pgfpathlineto{\pgfqpoint{5.387753in}{4.041464in}}%
\pgfpathlineto{\pgfqpoint{5.236967in}{4.124681in}}%
\pgfpathlineto{\pgfqpoint{5.086181in}{3.997990in}}%
\pgfpathlineto{\pgfqpoint{4.935395in}{3.910848in}}%
\pgfpathlineto{\pgfqpoint{4.784609in}{4.032804in}}%
\pgfpathlineto{\pgfqpoint{4.633823in}{3.971940in}}%
\pgfpathclose%
\pgfusepath{stroke,fill}%
\end{pgfscope}%
\begin{pgfscope}%
\pgfpathrectangle{\pgfqpoint{4.482986in}{3.363889in}}{\pgfqpoint{3.318403in}{2.263889in}}%
\pgfusepath{clip}%
\pgfsetbuttcap%
\pgfsetroundjoin%
\definecolor{currentfill}{rgb}{0.172549,0.627451,0.172549}%
\pgfsetfillcolor{currentfill}%
\pgfsetlinewidth{1.003750pt}%
\definecolor{currentstroke}{rgb}{0.172549,0.627451,0.172549}%
\pgfsetstrokecolor{currentstroke}%
\pgfsetdash{}{0pt}%
\pgfsys@defobject{currentmarker}{\pgfqpoint{-0.009821in}{-0.009821in}}{\pgfqpoint{0.009821in}{0.009821in}}{%
\pgfpathmoveto{\pgfqpoint{0.000000in}{-0.009821in}}%
\pgfpathcurveto{\pgfqpoint{0.002605in}{-0.009821in}}{\pgfqpoint{0.005103in}{-0.008786in}}{\pgfqpoint{0.006944in}{-0.006944in}}%
\pgfpathcurveto{\pgfqpoint{0.008786in}{-0.005103in}}{\pgfqpoint{0.009821in}{-0.002605in}}{\pgfqpoint{0.009821in}{0.000000in}}%
\pgfpathcurveto{\pgfqpoint{0.009821in}{0.002605in}}{\pgfqpoint{0.008786in}{0.005103in}}{\pgfqpoint{0.006944in}{0.006944in}}%
\pgfpathcurveto{\pgfqpoint{0.005103in}{0.008786in}}{\pgfqpoint{0.002605in}{0.009821in}}{\pgfqpoint{0.000000in}{0.009821in}}%
\pgfpathcurveto{\pgfqpoint{-0.002605in}{0.009821in}}{\pgfqpoint{-0.005103in}{0.008786in}}{\pgfqpoint{-0.006944in}{0.006944in}}%
\pgfpathcurveto{\pgfqpoint{-0.008786in}{0.005103in}}{\pgfqpoint{-0.009821in}{0.002605in}}{\pgfqpoint{-0.009821in}{0.000000in}}%
\pgfpathcurveto{\pgfqpoint{-0.009821in}{-0.002605in}}{\pgfqpoint{-0.008786in}{-0.005103in}}{\pgfqpoint{-0.006944in}{-0.006944in}}%
\pgfpathcurveto{\pgfqpoint{-0.005103in}{-0.008786in}}{\pgfqpoint{-0.002605in}{-0.009821in}}{\pgfqpoint{0.000000in}{-0.009821in}}%
\pgfpathclose%
\pgfusepath{stroke,fill}%
}%
\begin{pgfscope}%
\pgfsys@transformshift{4.633823in}{3.763868in}%
\pgfsys@useobject{currentmarker}{}%
\end{pgfscope}%
\begin{pgfscope}%
\pgfsys@transformshift{4.784609in}{3.749270in}%
\pgfsys@useobject{currentmarker}{}%
\end{pgfscope}%
\begin{pgfscope}%
\pgfsys@transformshift{4.935395in}{3.732977in}%
\pgfsys@useobject{currentmarker}{}%
\end{pgfscope}%
\begin{pgfscope}%
\pgfsys@transformshift{5.086181in}{3.873622in}%
\pgfsys@useobject{currentmarker}{}%
\end{pgfscope}%
\begin{pgfscope}%
\pgfsys@transformshift{5.236967in}{3.788222in}%
\pgfsys@useobject{currentmarker}{}%
\end{pgfscope}%
\begin{pgfscope}%
\pgfsys@transformshift{5.387753in}{4.168671in}%
\pgfsys@useobject{currentmarker}{}%
\end{pgfscope}%
\begin{pgfscope}%
\pgfsys@transformshift{5.538539in}{4.120823in}%
\pgfsys@useobject{currentmarker}{}%
\end{pgfscope}%
\begin{pgfscope}%
\pgfsys@transformshift{5.689325in}{3.919974in}%
\pgfsys@useobject{currentmarker}{}%
\end{pgfscope}%
\begin{pgfscope}%
\pgfsys@transformshift{5.840111in}{4.305093in}%
\pgfsys@useobject{currentmarker}{}%
\end{pgfscope}%
\begin{pgfscope}%
\pgfsys@transformshift{5.990897in}{4.227048in}%
\pgfsys@useobject{currentmarker}{}%
\end{pgfscope}%
\begin{pgfscope}%
\pgfsys@transformshift{6.141683in}{4.686787in}%
\pgfsys@useobject{currentmarker}{}%
\end{pgfscope}%
\begin{pgfscope}%
\pgfsys@transformshift{6.292469in}{4.932556in}%
\pgfsys@useobject{currentmarker}{}%
\end{pgfscope}%
\begin{pgfscope}%
\pgfsys@transformshift{6.443255in}{4.935088in}%
\pgfsys@useobject{currentmarker}{}%
\end{pgfscope}%
\begin{pgfscope}%
\pgfsys@transformshift{6.594041in}{4.935533in}%
\pgfsys@useobject{currentmarker}{}%
\end{pgfscope}%
\begin{pgfscope}%
\pgfsys@transformshift{6.744827in}{4.932031in}%
\pgfsys@useobject{currentmarker}{}%
\end{pgfscope}%
\begin{pgfscope}%
\pgfsys@transformshift{6.895613in}{4.934299in}%
\pgfsys@useobject{currentmarker}{}%
\end{pgfscope}%
\begin{pgfscope}%
\pgfsys@transformshift{7.046399in}{4.931718in}%
\pgfsys@useobject{currentmarker}{}%
\end{pgfscope}%
\begin{pgfscope}%
\pgfsys@transformshift{7.197185in}{4.932086in}%
\pgfsys@useobject{currentmarker}{}%
\end{pgfscope}%
\begin{pgfscope}%
\pgfsys@transformshift{7.347971in}{4.932742in}%
\pgfsys@useobject{currentmarker}{}%
\end{pgfscope}%
\begin{pgfscope}%
\pgfsys@transformshift{7.498756in}{4.932514in}%
\pgfsys@useobject{currentmarker}{}%
\end{pgfscope}%
\begin{pgfscope}%
\pgfsys@transformshift{7.649542in}{4.932821in}%
\pgfsys@useobject{currentmarker}{}%
\end{pgfscope}%
\end{pgfscope}%
\begin{pgfscope}%
\pgfpathrectangle{\pgfqpoint{4.482986in}{3.363889in}}{\pgfqpoint{3.318403in}{2.263889in}}%
\pgfusepath{clip}%
\pgfsetbuttcap%
\pgfsetroundjoin%
\definecolor{currentfill}{rgb}{0.172549,0.627451,0.172549}%
\pgfsetfillcolor{currentfill}%
\pgfsetfillopacity{0.300000}%
\pgfsetlinewidth{1.003750pt}%
\definecolor{currentstroke}{rgb}{0.172549,0.627451,0.172549}%
\pgfsetstrokecolor{currentstroke}%
\pgfsetstrokeopacity{0.300000}%
\pgfsetdash{}{0pt}%
\pgfpathmoveto{\pgfqpoint{4.633823in}{3.820673in}}%
\pgfpathlineto{\pgfqpoint{4.633823in}{3.707064in}}%
\pgfpathlineto{\pgfqpoint{4.784609in}{3.705669in}}%
\pgfpathlineto{\pgfqpoint{4.935395in}{3.680250in}}%
\pgfpathlineto{\pgfqpoint{5.086181in}{3.722318in}}%
\pgfpathlineto{\pgfqpoint{5.236967in}{3.616272in}}%
\pgfpathlineto{\pgfqpoint{5.387753in}{3.590006in}}%
\pgfpathlineto{\pgfqpoint{5.538539in}{4.006717in}}%
\pgfpathlineto{\pgfqpoint{5.689325in}{3.884588in}}%
\pgfpathlineto{\pgfqpoint{5.840111in}{3.673993in}}%
\pgfpathlineto{\pgfqpoint{5.990897in}{3.662523in}}%
\pgfpathlineto{\pgfqpoint{6.141683in}{4.291937in}}%
\pgfpathlineto{\pgfqpoint{6.292469in}{4.925906in}}%
\pgfpathlineto{\pgfqpoint{6.443255in}{4.929544in}}%
\pgfpathlineto{\pgfqpoint{6.594041in}{4.932143in}}%
\pgfpathlineto{\pgfqpoint{6.744827in}{4.927476in}}%
\pgfpathlineto{\pgfqpoint{6.895613in}{4.930481in}}%
\pgfpathlineto{\pgfqpoint{7.046399in}{4.929544in}}%
\pgfpathlineto{\pgfqpoint{7.197185in}{4.930247in}}%
\pgfpathlineto{\pgfqpoint{7.347971in}{4.932033in}}%
\pgfpathlineto{\pgfqpoint{7.498756in}{4.931961in}}%
\pgfpathlineto{\pgfqpoint{7.649542in}{4.932421in}}%
\pgfpathlineto{\pgfqpoint{7.649542in}{4.933221in}}%
\pgfpathlineto{\pgfqpoint{7.649542in}{4.933221in}}%
\pgfpathlineto{\pgfqpoint{7.498756in}{4.933068in}}%
\pgfpathlineto{\pgfqpoint{7.347971in}{4.933451in}}%
\pgfpathlineto{\pgfqpoint{7.197185in}{4.933925in}}%
\pgfpathlineto{\pgfqpoint{7.046399in}{4.933893in}}%
\pgfpathlineto{\pgfqpoint{6.895613in}{4.938118in}}%
\pgfpathlineto{\pgfqpoint{6.744827in}{4.936587in}}%
\pgfpathlineto{\pgfqpoint{6.594041in}{4.938924in}}%
\pgfpathlineto{\pgfqpoint{6.443255in}{4.940633in}}%
\pgfpathlineto{\pgfqpoint{6.292469in}{4.939207in}}%
\pgfpathlineto{\pgfqpoint{6.141683in}{5.081638in}}%
\pgfpathlineto{\pgfqpoint{5.990897in}{4.791573in}}%
\pgfpathlineto{\pgfqpoint{5.840111in}{4.936192in}}%
\pgfpathlineto{\pgfqpoint{5.689325in}{3.955359in}}%
\pgfpathlineto{\pgfqpoint{5.538539in}{4.234929in}}%
\pgfpathlineto{\pgfqpoint{5.387753in}{4.747335in}}%
\pgfpathlineto{\pgfqpoint{5.236967in}{3.960172in}}%
\pgfpathlineto{\pgfqpoint{5.086181in}{4.024926in}}%
\pgfpathlineto{\pgfqpoint{4.935395in}{3.785704in}}%
\pgfpathlineto{\pgfqpoint{4.784609in}{3.792871in}}%
\pgfpathlineto{\pgfqpoint{4.633823in}{3.820673in}}%
\pgfpathclose%
\pgfusepath{stroke,fill}%
\end{pgfscope}%
\begin{pgfscope}%
\pgfpathrectangle{\pgfqpoint{4.482986in}{3.363889in}}{\pgfqpoint{3.318403in}{2.263889in}}%
\pgfusepath{clip}%
\pgfsetbuttcap%
\pgfsetroundjoin%
\definecolor{currentfill}{rgb}{0.839216,0.152941,0.156863}%
\pgfsetfillcolor{currentfill}%
\pgfsetlinewidth{1.003750pt}%
\definecolor{currentstroke}{rgb}{0.839216,0.152941,0.156863}%
\pgfsetstrokecolor{currentstroke}%
\pgfsetdash{}{0pt}%
\pgfsys@defobject{currentmarker}{\pgfqpoint{-0.009821in}{-0.009821in}}{\pgfqpoint{0.009821in}{0.009821in}}{%
\pgfpathmoveto{\pgfqpoint{0.000000in}{-0.009821in}}%
\pgfpathcurveto{\pgfqpoint{0.002605in}{-0.009821in}}{\pgfqpoint{0.005103in}{-0.008786in}}{\pgfqpoint{0.006944in}{-0.006944in}}%
\pgfpathcurveto{\pgfqpoint{0.008786in}{-0.005103in}}{\pgfqpoint{0.009821in}{-0.002605in}}{\pgfqpoint{0.009821in}{0.000000in}}%
\pgfpathcurveto{\pgfqpoint{0.009821in}{0.002605in}}{\pgfqpoint{0.008786in}{0.005103in}}{\pgfqpoint{0.006944in}{0.006944in}}%
\pgfpathcurveto{\pgfqpoint{0.005103in}{0.008786in}}{\pgfqpoint{0.002605in}{0.009821in}}{\pgfqpoint{0.000000in}{0.009821in}}%
\pgfpathcurveto{\pgfqpoint{-0.002605in}{0.009821in}}{\pgfqpoint{-0.005103in}{0.008786in}}{\pgfqpoint{-0.006944in}{0.006944in}}%
\pgfpathcurveto{\pgfqpoint{-0.008786in}{0.005103in}}{\pgfqpoint{-0.009821in}{0.002605in}}{\pgfqpoint{-0.009821in}{0.000000in}}%
\pgfpathcurveto{\pgfqpoint{-0.009821in}{-0.002605in}}{\pgfqpoint{-0.008786in}{-0.005103in}}{\pgfqpoint{-0.006944in}{-0.006944in}}%
\pgfpathcurveto{\pgfqpoint{-0.005103in}{-0.008786in}}{\pgfqpoint{-0.002605in}{-0.009821in}}{\pgfqpoint{0.000000in}{-0.009821in}}%
\pgfpathclose%
\pgfusepath{stroke,fill}%
}%
\begin{pgfscope}%
\pgfsys@transformshift{4.633823in}{3.719885in}%
\pgfsys@useobject{currentmarker}{}%
\end{pgfscope}%
\begin{pgfscope}%
\pgfsys@transformshift{4.784609in}{3.766371in}%
\pgfsys@useobject{currentmarker}{}%
\end{pgfscope}%
\begin{pgfscope}%
\pgfsys@transformshift{4.935395in}{3.715219in}%
\pgfsys@useobject{currentmarker}{}%
\end{pgfscope}%
\begin{pgfscope}%
\pgfsys@transformshift{5.086181in}{3.801762in}%
\pgfsys@useobject{currentmarker}{}%
\end{pgfscope}%
\begin{pgfscope}%
\pgfsys@transformshift{5.236967in}{3.821271in}%
\pgfsys@useobject{currentmarker}{}%
\end{pgfscope}%
\begin{pgfscope}%
\pgfsys@transformshift{5.387753in}{3.886073in}%
\pgfsys@useobject{currentmarker}{}%
\end{pgfscope}%
\begin{pgfscope}%
\pgfsys@transformshift{5.538539in}{3.798609in}%
\pgfsys@useobject{currentmarker}{}%
\end{pgfscope}%
\begin{pgfscope}%
\pgfsys@transformshift{5.689325in}{3.894287in}%
\pgfsys@useobject{currentmarker}{}%
\end{pgfscope}%
\begin{pgfscope}%
\pgfsys@transformshift{5.840111in}{3.929800in}%
\pgfsys@useobject{currentmarker}{}%
\end{pgfscope}%
\begin{pgfscope}%
\pgfsys@transformshift{5.990897in}{4.006194in}%
\pgfsys@useobject{currentmarker}{}%
\end{pgfscope}%
\begin{pgfscope}%
\pgfsys@transformshift{6.141683in}{3.951557in}%
\pgfsys@useobject{currentmarker}{}%
\end{pgfscope}%
\begin{pgfscope}%
\pgfsys@transformshift{6.292469in}{3.943651in}%
\pgfsys@useobject{currentmarker}{}%
\end{pgfscope}%
\begin{pgfscope}%
\pgfsys@transformshift{6.443255in}{3.952953in}%
\pgfsys@useobject{currentmarker}{}%
\end{pgfscope}%
\begin{pgfscope}%
\pgfsys@transformshift{6.594041in}{3.926751in}%
\pgfsys@useobject{currentmarker}{}%
\end{pgfscope}%
\begin{pgfscope}%
\pgfsys@transformshift{6.744827in}{3.925015in}%
\pgfsys@useobject{currentmarker}{}%
\end{pgfscope}%
\begin{pgfscope}%
\pgfsys@transformshift{6.895613in}{3.935325in}%
\pgfsys@useobject{currentmarker}{}%
\end{pgfscope}%
\begin{pgfscope}%
\pgfsys@transformshift{7.046399in}{3.945034in}%
\pgfsys@useobject{currentmarker}{}%
\end{pgfscope}%
\begin{pgfscope}%
\pgfsys@transformshift{7.197185in}{3.938416in}%
\pgfsys@useobject{currentmarker}{}%
\end{pgfscope}%
\begin{pgfscope}%
\pgfsys@transformshift{7.347971in}{3.944363in}%
\pgfsys@useobject{currentmarker}{}%
\end{pgfscope}%
\begin{pgfscope}%
\pgfsys@transformshift{7.498756in}{3.941043in}%
\pgfsys@useobject{currentmarker}{}%
\end{pgfscope}%
\begin{pgfscope}%
\pgfsys@transformshift{7.649542in}{3.939452in}%
\pgfsys@useobject{currentmarker}{}%
\end{pgfscope}%
\end{pgfscope}%
\begin{pgfscope}%
\pgfpathrectangle{\pgfqpoint{4.482986in}{3.363889in}}{\pgfqpoint{3.318403in}{2.263889in}}%
\pgfusepath{clip}%
\pgfsetbuttcap%
\pgfsetroundjoin%
\definecolor{currentfill}{rgb}{0.839216,0.152941,0.156863}%
\pgfsetfillcolor{currentfill}%
\pgfsetfillopacity{0.300000}%
\pgfsetlinewidth{1.003750pt}%
\definecolor{currentstroke}{rgb}{0.839216,0.152941,0.156863}%
\pgfsetstrokecolor{currentstroke}%
\pgfsetstrokeopacity{0.300000}%
\pgfsetdash{}{0pt}%
\pgfpathmoveto{\pgfqpoint{4.633823in}{3.789954in}}%
\pgfpathlineto{\pgfqpoint{4.633823in}{3.649816in}}%
\pgfpathlineto{\pgfqpoint{4.784609in}{3.694353in}}%
\pgfpathlineto{\pgfqpoint{4.935395in}{3.695138in}}%
\pgfpathlineto{\pgfqpoint{5.086181in}{3.735947in}}%
\pgfpathlineto{\pgfqpoint{5.236967in}{3.691952in}}%
\pgfpathlineto{\pgfqpoint{5.387753in}{3.711730in}}%
\pgfpathlineto{\pgfqpoint{5.538539in}{3.679754in}}%
\pgfpathlineto{\pgfqpoint{5.689325in}{3.802766in}}%
\pgfpathlineto{\pgfqpoint{5.840111in}{3.765984in}}%
\pgfpathlineto{\pgfqpoint{5.990897in}{3.799425in}}%
\pgfpathlineto{\pgfqpoint{6.141683in}{3.885856in}}%
\pgfpathlineto{\pgfqpoint{6.292469in}{3.854209in}}%
\pgfpathlineto{\pgfqpoint{6.443255in}{3.832996in}}%
\pgfpathlineto{\pgfqpoint{6.594041in}{3.886566in}}%
\pgfpathlineto{\pgfqpoint{6.744827in}{3.881565in}}%
\pgfpathlineto{\pgfqpoint{6.895613in}{3.930521in}}%
\pgfpathlineto{\pgfqpoint{7.046399in}{3.934403in}}%
\pgfpathlineto{\pgfqpoint{7.197185in}{3.920798in}}%
\pgfpathlineto{\pgfqpoint{7.347971in}{3.936026in}}%
\pgfpathlineto{\pgfqpoint{7.498756in}{3.935992in}}%
\pgfpathlineto{\pgfqpoint{7.649542in}{3.936473in}}%
\pgfpathlineto{\pgfqpoint{7.649542in}{3.942430in}}%
\pgfpathlineto{\pgfqpoint{7.649542in}{3.942430in}}%
\pgfpathlineto{\pgfqpoint{7.498756in}{3.946094in}}%
\pgfpathlineto{\pgfqpoint{7.347971in}{3.952701in}}%
\pgfpathlineto{\pgfqpoint{7.197185in}{3.956034in}}%
\pgfpathlineto{\pgfqpoint{7.046399in}{3.955665in}}%
\pgfpathlineto{\pgfqpoint{6.895613in}{3.940128in}}%
\pgfpathlineto{\pgfqpoint{6.744827in}{3.968466in}}%
\pgfpathlineto{\pgfqpoint{6.594041in}{3.966936in}}%
\pgfpathlineto{\pgfqpoint{6.443255in}{4.072909in}}%
\pgfpathlineto{\pgfqpoint{6.292469in}{4.033094in}}%
\pgfpathlineto{\pgfqpoint{6.141683in}{4.017257in}}%
\pgfpathlineto{\pgfqpoint{5.990897in}{4.212963in}}%
\pgfpathlineto{\pgfqpoint{5.840111in}{4.093616in}}%
\pgfpathlineto{\pgfqpoint{5.689325in}{3.985809in}}%
\pgfpathlineto{\pgfqpoint{5.538539in}{3.917465in}}%
\pgfpathlineto{\pgfqpoint{5.387753in}{4.060416in}}%
\pgfpathlineto{\pgfqpoint{5.236967in}{3.950589in}}%
\pgfpathlineto{\pgfqpoint{5.086181in}{3.867578in}}%
\pgfpathlineto{\pgfqpoint{4.935395in}{3.735299in}}%
\pgfpathlineto{\pgfqpoint{4.784609in}{3.838390in}}%
\pgfpathlineto{\pgfqpoint{4.633823in}{3.789954in}}%
\pgfpathclose%
\pgfusepath{stroke,fill}%
\end{pgfscope}%
\begin{pgfscope}%
\pgfpathrectangle{\pgfqpoint{4.482986in}{3.363889in}}{\pgfqpoint{3.318403in}{2.263889in}}%
\pgfusepath{clip}%
\pgfsetbuttcap%
\pgfsetroundjoin%
\definecolor{currentfill}{rgb}{0.580392,0.403922,0.741176}%
\pgfsetfillcolor{currentfill}%
\pgfsetlinewidth{1.003750pt}%
\definecolor{currentstroke}{rgb}{0.580392,0.403922,0.741176}%
\pgfsetstrokecolor{currentstroke}%
\pgfsetdash{}{0pt}%
\pgfsys@defobject{currentmarker}{\pgfqpoint{-0.009821in}{-0.009821in}}{\pgfqpoint{0.009821in}{0.009821in}}{%
\pgfpathmoveto{\pgfqpoint{0.000000in}{-0.009821in}}%
\pgfpathcurveto{\pgfqpoint{0.002605in}{-0.009821in}}{\pgfqpoint{0.005103in}{-0.008786in}}{\pgfqpoint{0.006944in}{-0.006944in}}%
\pgfpathcurveto{\pgfqpoint{0.008786in}{-0.005103in}}{\pgfqpoint{0.009821in}{-0.002605in}}{\pgfqpoint{0.009821in}{0.000000in}}%
\pgfpathcurveto{\pgfqpoint{0.009821in}{0.002605in}}{\pgfqpoint{0.008786in}{0.005103in}}{\pgfqpoint{0.006944in}{0.006944in}}%
\pgfpathcurveto{\pgfqpoint{0.005103in}{0.008786in}}{\pgfqpoint{0.002605in}{0.009821in}}{\pgfqpoint{0.000000in}{0.009821in}}%
\pgfpathcurveto{\pgfqpoint{-0.002605in}{0.009821in}}{\pgfqpoint{-0.005103in}{0.008786in}}{\pgfqpoint{-0.006944in}{0.006944in}}%
\pgfpathcurveto{\pgfqpoint{-0.008786in}{0.005103in}}{\pgfqpoint{-0.009821in}{0.002605in}}{\pgfqpoint{-0.009821in}{0.000000in}}%
\pgfpathcurveto{\pgfqpoint{-0.009821in}{-0.002605in}}{\pgfqpoint{-0.008786in}{-0.005103in}}{\pgfqpoint{-0.006944in}{-0.006944in}}%
\pgfpathcurveto{\pgfqpoint{-0.005103in}{-0.008786in}}{\pgfqpoint{-0.002605in}{-0.009821in}}{\pgfqpoint{0.000000in}{-0.009821in}}%
\pgfpathclose%
\pgfusepath{stroke,fill}%
}%
\begin{pgfscope}%
\pgfsys@transformshift{4.633823in}{3.721978in}%
\pgfsys@useobject{currentmarker}{}%
\end{pgfscope}%
\begin{pgfscope}%
\pgfsys@transformshift{4.784609in}{3.741195in}%
\pgfsys@useobject{currentmarker}{}%
\end{pgfscope}%
\begin{pgfscope}%
\pgfsys@transformshift{4.935395in}{3.806489in}%
\pgfsys@useobject{currentmarker}{}%
\end{pgfscope}%
\begin{pgfscope}%
\pgfsys@transformshift{5.086181in}{3.837762in}%
\pgfsys@useobject{currentmarker}{}%
\end{pgfscope}%
\begin{pgfscope}%
\pgfsys@transformshift{5.236967in}{3.749054in}%
\pgfsys@useobject{currentmarker}{}%
\end{pgfscope}%
\begin{pgfscope}%
\pgfsys@transformshift{5.387753in}{3.771638in}%
\pgfsys@useobject{currentmarker}{}%
\end{pgfscope}%
\begin{pgfscope}%
\pgfsys@transformshift{5.538539in}{3.793508in}%
\pgfsys@useobject{currentmarker}{}%
\end{pgfscope}%
\begin{pgfscope}%
\pgfsys@transformshift{5.689325in}{3.782050in}%
\pgfsys@useobject{currentmarker}{}%
\end{pgfscope}%
\begin{pgfscope}%
\pgfsys@transformshift{5.840111in}{3.770886in}%
\pgfsys@useobject{currentmarker}{}%
\end{pgfscope}%
\begin{pgfscope}%
\pgfsys@transformshift{5.990897in}{3.767747in}%
\pgfsys@useobject{currentmarker}{}%
\end{pgfscope}%
\begin{pgfscope}%
\pgfsys@transformshift{6.141683in}{3.771821in}%
\pgfsys@useobject{currentmarker}{}%
\end{pgfscope}%
\begin{pgfscope}%
\pgfsys@transformshift{6.292469in}{3.776206in}%
\pgfsys@useobject{currentmarker}{}%
\end{pgfscope}%
\begin{pgfscope}%
\pgfsys@transformshift{6.443255in}{3.768919in}%
\pgfsys@useobject{currentmarker}{}%
\end{pgfscope}%
\begin{pgfscope}%
\pgfsys@transformshift{6.594041in}{3.772142in}%
\pgfsys@useobject{currentmarker}{}%
\end{pgfscope}%
\begin{pgfscope}%
\pgfsys@transformshift{6.744827in}{3.777330in}%
\pgfsys@useobject{currentmarker}{}%
\end{pgfscope}%
\begin{pgfscope}%
\pgfsys@transformshift{6.895613in}{3.775099in}%
\pgfsys@useobject{currentmarker}{}%
\end{pgfscope}%
\begin{pgfscope}%
\pgfsys@transformshift{7.046399in}{3.777212in}%
\pgfsys@useobject{currentmarker}{}%
\end{pgfscope}%
\begin{pgfscope}%
\pgfsys@transformshift{7.197185in}{3.776424in}%
\pgfsys@useobject{currentmarker}{}%
\end{pgfscope}%
\begin{pgfscope}%
\pgfsys@transformshift{7.347971in}{3.775934in}%
\pgfsys@useobject{currentmarker}{}%
\end{pgfscope}%
\begin{pgfscope}%
\pgfsys@transformshift{7.498756in}{3.776603in}%
\pgfsys@useobject{currentmarker}{}%
\end{pgfscope}%
\begin{pgfscope}%
\pgfsys@transformshift{7.649542in}{3.775983in}%
\pgfsys@useobject{currentmarker}{}%
\end{pgfscope}%
\end{pgfscope}%
\begin{pgfscope}%
\pgfpathrectangle{\pgfqpoint{4.482986in}{3.363889in}}{\pgfqpoint{3.318403in}{2.263889in}}%
\pgfusepath{clip}%
\pgfsetbuttcap%
\pgfsetroundjoin%
\definecolor{currentfill}{rgb}{0.580392,0.403922,0.741176}%
\pgfsetfillcolor{currentfill}%
\pgfsetfillopacity{0.300000}%
\pgfsetlinewidth{1.003750pt}%
\definecolor{currentstroke}{rgb}{0.580392,0.403922,0.741176}%
\pgfsetstrokecolor{currentstroke}%
\pgfsetstrokeopacity{0.300000}%
\pgfsetdash{}{0pt}%
\pgfpathmoveto{\pgfqpoint{4.633823in}{3.756832in}}%
\pgfpathlineto{\pgfqpoint{4.633823in}{3.687125in}}%
\pgfpathlineto{\pgfqpoint{4.784609in}{3.691179in}}%
\pgfpathlineto{\pgfqpoint{4.935395in}{3.710677in}}%
\pgfpathlineto{\pgfqpoint{5.086181in}{3.730258in}}%
\pgfpathlineto{\pgfqpoint{5.236967in}{3.677485in}}%
\pgfpathlineto{\pgfqpoint{5.387753in}{3.670862in}}%
\pgfpathlineto{\pgfqpoint{5.538539in}{3.717786in}}%
\pgfpathlineto{\pgfqpoint{5.689325in}{3.770289in}}%
\pgfpathlineto{\pgfqpoint{5.840111in}{3.730749in}}%
\pgfpathlineto{\pgfqpoint{5.990897in}{3.739977in}}%
\pgfpathlineto{\pgfqpoint{6.141683in}{3.757447in}}%
\pgfpathlineto{\pgfqpoint{6.292469in}{3.764237in}}%
\pgfpathlineto{\pgfqpoint{6.443255in}{3.758442in}}%
\pgfpathlineto{\pgfqpoint{6.594041in}{3.767200in}}%
\pgfpathlineto{\pgfqpoint{6.744827in}{3.771578in}}%
\pgfpathlineto{\pgfqpoint{6.895613in}{3.772848in}}%
\pgfpathlineto{\pgfqpoint{7.046399in}{3.773886in}}%
\pgfpathlineto{\pgfqpoint{7.197185in}{3.775288in}}%
\pgfpathlineto{\pgfqpoint{7.347971in}{3.775180in}}%
\pgfpathlineto{\pgfqpoint{7.498756in}{3.775434in}}%
\pgfpathlineto{\pgfqpoint{7.649542in}{3.774874in}}%
\pgfpathlineto{\pgfqpoint{7.649542in}{3.777091in}}%
\pgfpathlineto{\pgfqpoint{7.649542in}{3.777091in}}%
\pgfpathlineto{\pgfqpoint{7.498756in}{3.777773in}}%
\pgfpathlineto{\pgfqpoint{7.347971in}{3.776689in}}%
\pgfpathlineto{\pgfqpoint{7.197185in}{3.777559in}}%
\pgfpathlineto{\pgfqpoint{7.046399in}{3.780537in}}%
\pgfpathlineto{\pgfqpoint{6.895613in}{3.777350in}}%
\pgfpathlineto{\pgfqpoint{6.744827in}{3.783082in}}%
\pgfpathlineto{\pgfqpoint{6.594041in}{3.777085in}}%
\pgfpathlineto{\pgfqpoint{6.443255in}{3.779397in}}%
\pgfpathlineto{\pgfqpoint{6.292469in}{3.788175in}}%
\pgfpathlineto{\pgfqpoint{6.141683in}{3.786195in}}%
\pgfpathlineto{\pgfqpoint{5.990897in}{3.795517in}}%
\pgfpathlineto{\pgfqpoint{5.840111in}{3.811023in}}%
\pgfpathlineto{\pgfqpoint{5.689325in}{3.793810in}}%
\pgfpathlineto{\pgfqpoint{5.538539in}{3.869230in}}%
\pgfpathlineto{\pgfqpoint{5.387753in}{3.872415in}}%
\pgfpathlineto{\pgfqpoint{5.236967in}{3.820622in}}%
\pgfpathlineto{\pgfqpoint{5.086181in}{3.945266in}}%
\pgfpathlineto{\pgfqpoint{4.935395in}{3.902302in}}%
\pgfpathlineto{\pgfqpoint{4.784609in}{3.791210in}}%
\pgfpathlineto{\pgfqpoint{4.633823in}{3.756832in}}%
\pgfpathclose%
\pgfusepath{stroke,fill}%
\end{pgfscope}%
\begin{pgfscope}%
\pgfsetbuttcap%
\pgfsetroundjoin%
\definecolor{currentfill}{rgb}{0.000000,0.000000,0.000000}%
\pgfsetfillcolor{currentfill}%
\pgfsetlinewidth{0.803000pt}%
\definecolor{currentstroke}{rgb}{0.000000,0.000000,0.000000}%
\pgfsetstrokecolor{currentstroke}%
\pgfsetdash{}{0pt}%
\pgfsys@defobject{currentmarker}{\pgfqpoint{0.000000in}{-0.048611in}}{\pgfqpoint{0.000000in}{0.000000in}}{%
\pgfpathmoveto{\pgfqpoint{0.000000in}{0.000000in}}%
\pgfpathlineto{\pgfqpoint{0.000000in}{-0.048611in}}%
\pgfusepath{stroke,fill}%
}%
\begin{pgfscope}%
\pgfsys@transformshift{4.996951in}{3.363889in}%
\pgfsys@useobject{currentmarker}{}%
\end{pgfscope}%
\end{pgfscope}%
\begin{pgfscope}%
\definecolor{textcolor}{rgb}{0.000000,0.000000,0.000000}%
\pgfsetstrokecolor{textcolor}%
\pgfsetfillcolor{textcolor}%
\pgftext[x=4.996951in,y=3.266667in,,top]{\color{textcolor}\sffamily\fontsize{10.000000}{12.000000}\selectfont \(\displaystyle {10^{4}}\)}%
\end{pgfscope}%
\begin{pgfscope}%
\pgfsetbuttcap%
\pgfsetroundjoin%
\definecolor{currentfill}{rgb}{0.000000,0.000000,0.000000}%
\pgfsetfillcolor{currentfill}%
\pgfsetlinewidth{0.803000pt}%
\definecolor{currentstroke}{rgb}{0.000000,0.000000,0.000000}%
\pgfsetstrokecolor{currentstroke}%
\pgfsetdash{}{0pt}%
\pgfsys@defobject{currentmarker}{\pgfqpoint{0.000000in}{-0.048611in}}{\pgfqpoint{0.000000in}{0.000000in}}{%
\pgfpathmoveto{\pgfqpoint{0.000000in}{0.000000in}}%
\pgfpathlineto{\pgfqpoint{0.000000in}{-0.048611in}}%
\pgfusepath{stroke,fill}%
}%
\begin{pgfscope}%
\pgfsys@transformshift{5.600095in}{3.363889in}%
\pgfsys@useobject{currentmarker}{}%
\end{pgfscope}%
\end{pgfscope}%
\begin{pgfscope}%
\definecolor{textcolor}{rgb}{0.000000,0.000000,0.000000}%
\pgfsetstrokecolor{textcolor}%
\pgfsetfillcolor{textcolor}%
\pgftext[x=5.600095in,y=3.266667in,,top]{\color{textcolor}\sffamily\fontsize{10.000000}{12.000000}\selectfont \(\displaystyle {10^{5}}\)}%
\end{pgfscope}%
\begin{pgfscope}%
\pgfsetbuttcap%
\pgfsetroundjoin%
\definecolor{currentfill}{rgb}{0.000000,0.000000,0.000000}%
\pgfsetfillcolor{currentfill}%
\pgfsetlinewidth{0.803000pt}%
\definecolor{currentstroke}{rgb}{0.000000,0.000000,0.000000}%
\pgfsetstrokecolor{currentstroke}%
\pgfsetdash{}{0pt}%
\pgfsys@defobject{currentmarker}{\pgfqpoint{0.000000in}{-0.048611in}}{\pgfqpoint{0.000000in}{0.000000in}}{%
\pgfpathmoveto{\pgfqpoint{0.000000in}{0.000000in}}%
\pgfpathlineto{\pgfqpoint{0.000000in}{-0.048611in}}%
\pgfusepath{stroke,fill}%
}%
\begin{pgfscope}%
\pgfsys@transformshift{6.203239in}{3.363889in}%
\pgfsys@useobject{currentmarker}{}%
\end{pgfscope}%
\end{pgfscope}%
\begin{pgfscope}%
\definecolor{textcolor}{rgb}{0.000000,0.000000,0.000000}%
\pgfsetstrokecolor{textcolor}%
\pgfsetfillcolor{textcolor}%
\pgftext[x=6.203239in,y=3.266667in,,top]{\color{textcolor}\sffamily\fontsize{10.000000}{12.000000}\selectfont \(\displaystyle {10^{6}}\)}%
\end{pgfscope}%
\begin{pgfscope}%
\pgfsetbuttcap%
\pgfsetroundjoin%
\definecolor{currentfill}{rgb}{0.000000,0.000000,0.000000}%
\pgfsetfillcolor{currentfill}%
\pgfsetlinewidth{0.803000pt}%
\definecolor{currentstroke}{rgb}{0.000000,0.000000,0.000000}%
\pgfsetstrokecolor{currentstroke}%
\pgfsetdash{}{0pt}%
\pgfsys@defobject{currentmarker}{\pgfqpoint{0.000000in}{-0.048611in}}{\pgfqpoint{0.000000in}{0.000000in}}{%
\pgfpathmoveto{\pgfqpoint{0.000000in}{0.000000in}}%
\pgfpathlineto{\pgfqpoint{0.000000in}{-0.048611in}}%
\pgfusepath{stroke,fill}%
}%
\begin{pgfscope}%
\pgfsys@transformshift{6.806383in}{3.363889in}%
\pgfsys@useobject{currentmarker}{}%
\end{pgfscope}%
\end{pgfscope}%
\begin{pgfscope}%
\definecolor{textcolor}{rgb}{0.000000,0.000000,0.000000}%
\pgfsetstrokecolor{textcolor}%
\pgfsetfillcolor{textcolor}%
\pgftext[x=6.806383in,y=3.266667in,,top]{\color{textcolor}\sffamily\fontsize{10.000000}{12.000000}\selectfont \(\displaystyle {10^{7}}\)}%
\end{pgfscope}%
\begin{pgfscope}%
\pgfsetbuttcap%
\pgfsetroundjoin%
\definecolor{currentfill}{rgb}{0.000000,0.000000,0.000000}%
\pgfsetfillcolor{currentfill}%
\pgfsetlinewidth{0.803000pt}%
\definecolor{currentstroke}{rgb}{0.000000,0.000000,0.000000}%
\pgfsetstrokecolor{currentstroke}%
\pgfsetdash{}{0pt}%
\pgfsys@defobject{currentmarker}{\pgfqpoint{0.000000in}{-0.048611in}}{\pgfqpoint{0.000000in}{0.000000in}}{%
\pgfpathmoveto{\pgfqpoint{0.000000in}{0.000000in}}%
\pgfpathlineto{\pgfqpoint{0.000000in}{-0.048611in}}%
\pgfusepath{stroke,fill}%
}%
\begin{pgfscope}%
\pgfsys@transformshift{7.409527in}{3.363889in}%
\pgfsys@useobject{currentmarker}{}%
\end{pgfscope}%
\end{pgfscope}%
\begin{pgfscope}%
\definecolor{textcolor}{rgb}{0.000000,0.000000,0.000000}%
\pgfsetstrokecolor{textcolor}%
\pgfsetfillcolor{textcolor}%
\pgftext[x=7.409527in,y=3.266667in,,top]{\color{textcolor}\sffamily\fontsize{10.000000}{12.000000}\selectfont \(\displaystyle {10^{8}}\)}%
\end{pgfscope}%
\begin{pgfscope}%
\pgfsetbuttcap%
\pgfsetroundjoin%
\definecolor{currentfill}{rgb}{0.000000,0.000000,0.000000}%
\pgfsetfillcolor{currentfill}%
\pgfsetlinewidth{0.602250pt}%
\definecolor{currentstroke}{rgb}{0.000000,0.000000,0.000000}%
\pgfsetstrokecolor{currentstroke}%
\pgfsetdash{}{0pt}%
\pgfsys@defobject{currentmarker}{\pgfqpoint{0.000000in}{-0.027778in}}{\pgfqpoint{0.000000in}{0.000000in}}{%
\pgfpathmoveto{\pgfqpoint{0.000000in}{0.000000in}}%
\pgfpathlineto{\pgfqpoint{0.000000in}{-0.027778in}}%
\pgfusepath{stroke,fill}%
}%
\begin{pgfscope}%
\pgfsys@transformshift{4.575372in}{3.363889in}%
\pgfsys@useobject{currentmarker}{}%
\end{pgfscope}%
\end{pgfscope}%
\begin{pgfscope}%
\pgfsetbuttcap%
\pgfsetroundjoin%
\definecolor{currentfill}{rgb}{0.000000,0.000000,0.000000}%
\pgfsetfillcolor{currentfill}%
\pgfsetlinewidth{0.602250pt}%
\definecolor{currentstroke}{rgb}{0.000000,0.000000,0.000000}%
\pgfsetstrokecolor{currentstroke}%
\pgfsetdash{}{0pt}%
\pgfsys@defobject{currentmarker}{\pgfqpoint{0.000000in}{-0.027778in}}{\pgfqpoint{0.000000in}{0.000000in}}{%
\pgfpathmoveto{\pgfqpoint{0.000000in}{0.000000in}}%
\pgfpathlineto{\pgfqpoint{0.000000in}{-0.027778in}}%
\pgfusepath{stroke,fill}%
}%
\begin{pgfscope}%
\pgfsys@transformshift{4.681580in}{3.363889in}%
\pgfsys@useobject{currentmarker}{}%
\end{pgfscope}%
\end{pgfscope}%
\begin{pgfscope}%
\pgfsetbuttcap%
\pgfsetroundjoin%
\definecolor{currentfill}{rgb}{0.000000,0.000000,0.000000}%
\pgfsetfillcolor{currentfill}%
\pgfsetlinewidth{0.602250pt}%
\definecolor{currentstroke}{rgb}{0.000000,0.000000,0.000000}%
\pgfsetstrokecolor{currentstroke}%
\pgfsetdash{}{0pt}%
\pgfsys@defobject{currentmarker}{\pgfqpoint{0.000000in}{-0.027778in}}{\pgfqpoint{0.000000in}{0.000000in}}{%
\pgfpathmoveto{\pgfqpoint{0.000000in}{0.000000in}}%
\pgfpathlineto{\pgfqpoint{0.000000in}{-0.027778in}}%
\pgfusepath{stroke,fill}%
}%
\begin{pgfscope}%
\pgfsys@transformshift{4.756936in}{3.363889in}%
\pgfsys@useobject{currentmarker}{}%
\end{pgfscope}%
\end{pgfscope}%
\begin{pgfscope}%
\pgfsetbuttcap%
\pgfsetroundjoin%
\definecolor{currentfill}{rgb}{0.000000,0.000000,0.000000}%
\pgfsetfillcolor{currentfill}%
\pgfsetlinewidth{0.602250pt}%
\definecolor{currentstroke}{rgb}{0.000000,0.000000,0.000000}%
\pgfsetstrokecolor{currentstroke}%
\pgfsetdash{}{0pt}%
\pgfsys@defobject{currentmarker}{\pgfqpoint{0.000000in}{-0.027778in}}{\pgfqpoint{0.000000in}{0.000000in}}{%
\pgfpathmoveto{\pgfqpoint{0.000000in}{0.000000in}}%
\pgfpathlineto{\pgfqpoint{0.000000in}{-0.027778in}}%
\pgfusepath{stroke,fill}%
}%
\begin{pgfscope}%
\pgfsys@transformshift{4.815387in}{3.363889in}%
\pgfsys@useobject{currentmarker}{}%
\end{pgfscope}%
\end{pgfscope}%
\begin{pgfscope}%
\pgfsetbuttcap%
\pgfsetroundjoin%
\definecolor{currentfill}{rgb}{0.000000,0.000000,0.000000}%
\pgfsetfillcolor{currentfill}%
\pgfsetlinewidth{0.602250pt}%
\definecolor{currentstroke}{rgb}{0.000000,0.000000,0.000000}%
\pgfsetstrokecolor{currentstroke}%
\pgfsetdash{}{0pt}%
\pgfsys@defobject{currentmarker}{\pgfqpoint{0.000000in}{-0.027778in}}{\pgfqpoint{0.000000in}{0.000000in}}{%
\pgfpathmoveto{\pgfqpoint{0.000000in}{0.000000in}}%
\pgfpathlineto{\pgfqpoint{0.000000in}{-0.027778in}}%
\pgfusepath{stroke,fill}%
}%
\begin{pgfscope}%
\pgfsys@transformshift{4.863145in}{3.363889in}%
\pgfsys@useobject{currentmarker}{}%
\end{pgfscope}%
\end{pgfscope}%
\begin{pgfscope}%
\pgfsetbuttcap%
\pgfsetroundjoin%
\definecolor{currentfill}{rgb}{0.000000,0.000000,0.000000}%
\pgfsetfillcolor{currentfill}%
\pgfsetlinewidth{0.602250pt}%
\definecolor{currentstroke}{rgb}{0.000000,0.000000,0.000000}%
\pgfsetstrokecolor{currentstroke}%
\pgfsetdash{}{0pt}%
\pgfsys@defobject{currentmarker}{\pgfqpoint{0.000000in}{-0.027778in}}{\pgfqpoint{0.000000in}{0.000000in}}{%
\pgfpathmoveto{\pgfqpoint{0.000000in}{0.000000in}}%
\pgfpathlineto{\pgfqpoint{0.000000in}{-0.027778in}}%
\pgfusepath{stroke,fill}%
}%
\begin{pgfscope}%
\pgfsys@transformshift{4.903523in}{3.363889in}%
\pgfsys@useobject{currentmarker}{}%
\end{pgfscope}%
\end{pgfscope}%
\begin{pgfscope}%
\pgfsetbuttcap%
\pgfsetroundjoin%
\definecolor{currentfill}{rgb}{0.000000,0.000000,0.000000}%
\pgfsetfillcolor{currentfill}%
\pgfsetlinewidth{0.602250pt}%
\definecolor{currentstroke}{rgb}{0.000000,0.000000,0.000000}%
\pgfsetstrokecolor{currentstroke}%
\pgfsetdash{}{0pt}%
\pgfsys@defobject{currentmarker}{\pgfqpoint{0.000000in}{-0.027778in}}{\pgfqpoint{0.000000in}{0.000000in}}{%
\pgfpathmoveto{\pgfqpoint{0.000000in}{0.000000in}}%
\pgfpathlineto{\pgfqpoint{0.000000in}{-0.027778in}}%
\pgfusepath{stroke,fill}%
}%
\begin{pgfscope}%
\pgfsys@transformshift{4.938501in}{3.363889in}%
\pgfsys@useobject{currentmarker}{}%
\end{pgfscope}%
\end{pgfscope}%
\begin{pgfscope}%
\pgfsetbuttcap%
\pgfsetroundjoin%
\definecolor{currentfill}{rgb}{0.000000,0.000000,0.000000}%
\pgfsetfillcolor{currentfill}%
\pgfsetlinewidth{0.602250pt}%
\definecolor{currentstroke}{rgb}{0.000000,0.000000,0.000000}%
\pgfsetstrokecolor{currentstroke}%
\pgfsetdash{}{0pt}%
\pgfsys@defobject{currentmarker}{\pgfqpoint{0.000000in}{-0.027778in}}{\pgfqpoint{0.000000in}{0.000000in}}{%
\pgfpathmoveto{\pgfqpoint{0.000000in}{0.000000in}}%
\pgfpathlineto{\pgfqpoint{0.000000in}{-0.027778in}}%
\pgfusepath{stroke,fill}%
}%
\begin{pgfscope}%
\pgfsys@transformshift{4.969353in}{3.363889in}%
\pgfsys@useobject{currentmarker}{}%
\end{pgfscope}%
\end{pgfscope}%
\begin{pgfscope}%
\pgfsetbuttcap%
\pgfsetroundjoin%
\definecolor{currentfill}{rgb}{0.000000,0.000000,0.000000}%
\pgfsetfillcolor{currentfill}%
\pgfsetlinewidth{0.602250pt}%
\definecolor{currentstroke}{rgb}{0.000000,0.000000,0.000000}%
\pgfsetstrokecolor{currentstroke}%
\pgfsetdash{}{0pt}%
\pgfsys@defobject{currentmarker}{\pgfqpoint{0.000000in}{-0.027778in}}{\pgfqpoint{0.000000in}{0.000000in}}{%
\pgfpathmoveto{\pgfqpoint{0.000000in}{0.000000in}}%
\pgfpathlineto{\pgfqpoint{0.000000in}{-0.027778in}}%
\pgfusepath{stroke,fill}%
}%
\begin{pgfscope}%
\pgfsys@transformshift{5.178516in}{3.363889in}%
\pgfsys@useobject{currentmarker}{}%
\end{pgfscope}%
\end{pgfscope}%
\begin{pgfscope}%
\pgfsetbuttcap%
\pgfsetroundjoin%
\definecolor{currentfill}{rgb}{0.000000,0.000000,0.000000}%
\pgfsetfillcolor{currentfill}%
\pgfsetlinewidth{0.602250pt}%
\definecolor{currentstroke}{rgb}{0.000000,0.000000,0.000000}%
\pgfsetstrokecolor{currentstroke}%
\pgfsetdash{}{0pt}%
\pgfsys@defobject{currentmarker}{\pgfqpoint{0.000000in}{-0.027778in}}{\pgfqpoint{0.000000in}{0.000000in}}{%
\pgfpathmoveto{\pgfqpoint{0.000000in}{0.000000in}}%
\pgfpathlineto{\pgfqpoint{0.000000in}{-0.027778in}}%
\pgfusepath{stroke,fill}%
}%
\begin{pgfscope}%
\pgfsys@transformshift{5.284724in}{3.363889in}%
\pgfsys@useobject{currentmarker}{}%
\end{pgfscope}%
\end{pgfscope}%
\begin{pgfscope}%
\pgfsetbuttcap%
\pgfsetroundjoin%
\definecolor{currentfill}{rgb}{0.000000,0.000000,0.000000}%
\pgfsetfillcolor{currentfill}%
\pgfsetlinewidth{0.602250pt}%
\definecolor{currentstroke}{rgb}{0.000000,0.000000,0.000000}%
\pgfsetstrokecolor{currentstroke}%
\pgfsetdash{}{0pt}%
\pgfsys@defobject{currentmarker}{\pgfqpoint{0.000000in}{-0.027778in}}{\pgfqpoint{0.000000in}{0.000000in}}{%
\pgfpathmoveto{\pgfqpoint{0.000000in}{0.000000in}}%
\pgfpathlineto{\pgfqpoint{0.000000in}{-0.027778in}}%
\pgfusepath{stroke,fill}%
}%
\begin{pgfscope}%
\pgfsys@transformshift{5.360080in}{3.363889in}%
\pgfsys@useobject{currentmarker}{}%
\end{pgfscope}%
\end{pgfscope}%
\begin{pgfscope}%
\pgfsetbuttcap%
\pgfsetroundjoin%
\definecolor{currentfill}{rgb}{0.000000,0.000000,0.000000}%
\pgfsetfillcolor{currentfill}%
\pgfsetlinewidth{0.602250pt}%
\definecolor{currentstroke}{rgb}{0.000000,0.000000,0.000000}%
\pgfsetstrokecolor{currentstroke}%
\pgfsetdash{}{0pt}%
\pgfsys@defobject{currentmarker}{\pgfqpoint{0.000000in}{-0.027778in}}{\pgfqpoint{0.000000in}{0.000000in}}{%
\pgfpathmoveto{\pgfqpoint{0.000000in}{0.000000in}}%
\pgfpathlineto{\pgfqpoint{0.000000in}{-0.027778in}}%
\pgfusepath{stroke,fill}%
}%
\begin{pgfscope}%
\pgfsys@transformshift{5.418531in}{3.363889in}%
\pgfsys@useobject{currentmarker}{}%
\end{pgfscope}%
\end{pgfscope}%
\begin{pgfscope}%
\pgfsetbuttcap%
\pgfsetroundjoin%
\definecolor{currentfill}{rgb}{0.000000,0.000000,0.000000}%
\pgfsetfillcolor{currentfill}%
\pgfsetlinewidth{0.602250pt}%
\definecolor{currentstroke}{rgb}{0.000000,0.000000,0.000000}%
\pgfsetstrokecolor{currentstroke}%
\pgfsetdash{}{0pt}%
\pgfsys@defobject{currentmarker}{\pgfqpoint{0.000000in}{-0.027778in}}{\pgfqpoint{0.000000in}{0.000000in}}{%
\pgfpathmoveto{\pgfqpoint{0.000000in}{0.000000in}}%
\pgfpathlineto{\pgfqpoint{0.000000in}{-0.027778in}}%
\pgfusepath{stroke,fill}%
}%
\begin{pgfscope}%
\pgfsys@transformshift{5.466289in}{3.363889in}%
\pgfsys@useobject{currentmarker}{}%
\end{pgfscope}%
\end{pgfscope}%
\begin{pgfscope}%
\pgfsetbuttcap%
\pgfsetroundjoin%
\definecolor{currentfill}{rgb}{0.000000,0.000000,0.000000}%
\pgfsetfillcolor{currentfill}%
\pgfsetlinewidth{0.602250pt}%
\definecolor{currentstroke}{rgb}{0.000000,0.000000,0.000000}%
\pgfsetstrokecolor{currentstroke}%
\pgfsetdash{}{0pt}%
\pgfsys@defobject{currentmarker}{\pgfqpoint{0.000000in}{-0.027778in}}{\pgfqpoint{0.000000in}{0.000000in}}{%
\pgfpathmoveto{\pgfqpoint{0.000000in}{0.000000in}}%
\pgfpathlineto{\pgfqpoint{0.000000in}{-0.027778in}}%
\pgfusepath{stroke,fill}%
}%
\begin{pgfscope}%
\pgfsys@transformshift{5.506667in}{3.363889in}%
\pgfsys@useobject{currentmarker}{}%
\end{pgfscope}%
\end{pgfscope}%
\begin{pgfscope}%
\pgfsetbuttcap%
\pgfsetroundjoin%
\definecolor{currentfill}{rgb}{0.000000,0.000000,0.000000}%
\pgfsetfillcolor{currentfill}%
\pgfsetlinewidth{0.602250pt}%
\definecolor{currentstroke}{rgb}{0.000000,0.000000,0.000000}%
\pgfsetstrokecolor{currentstroke}%
\pgfsetdash{}{0pt}%
\pgfsys@defobject{currentmarker}{\pgfqpoint{0.000000in}{-0.027778in}}{\pgfqpoint{0.000000in}{0.000000in}}{%
\pgfpathmoveto{\pgfqpoint{0.000000in}{0.000000in}}%
\pgfpathlineto{\pgfqpoint{0.000000in}{-0.027778in}}%
\pgfusepath{stroke,fill}%
}%
\begin{pgfscope}%
\pgfsys@transformshift{5.541645in}{3.363889in}%
\pgfsys@useobject{currentmarker}{}%
\end{pgfscope}%
\end{pgfscope}%
\begin{pgfscope}%
\pgfsetbuttcap%
\pgfsetroundjoin%
\definecolor{currentfill}{rgb}{0.000000,0.000000,0.000000}%
\pgfsetfillcolor{currentfill}%
\pgfsetlinewidth{0.602250pt}%
\definecolor{currentstroke}{rgb}{0.000000,0.000000,0.000000}%
\pgfsetstrokecolor{currentstroke}%
\pgfsetdash{}{0pt}%
\pgfsys@defobject{currentmarker}{\pgfqpoint{0.000000in}{-0.027778in}}{\pgfqpoint{0.000000in}{0.000000in}}{%
\pgfpathmoveto{\pgfqpoint{0.000000in}{0.000000in}}%
\pgfpathlineto{\pgfqpoint{0.000000in}{-0.027778in}}%
\pgfusepath{stroke,fill}%
}%
\begin{pgfscope}%
\pgfsys@transformshift{5.572497in}{3.363889in}%
\pgfsys@useobject{currentmarker}{}%
\end{pgfscope}%
\end{pgfscope}%
\begin{pgfscope}%
\pgfsetbuttcap%
\pgfsetroundjoin%
\definecolor{currentfill}{rgb}{0.000000,0.000000,0.000000}%
\pgfsetfillcolor{currentfill}%
\pgfsetlinewidth{0.602250pt}%
\definecolor{currentstroke}{rgb}{0.000000,0.000000,0.000000}%
\pgfsetstrokecolor{currentstroke}%
\pgfsetdash{}{0pt}%
\pgfsys@defobject{currentmarker}{\pgfqpoint{0.000000in}{-0.027778in}}{\pgfqpoint{0.000000in}{0.000000in}}{%
\pgfpathmoveto{\pgfqpoint{0.000000in}{0.000000in}}%
\pgfpathlineto{\pgfqpoint{0.000000in}{-0.027778in}}%
\pgfusepath{stroke,fill}%
}%
\begin{pgfscope}%
\pgfsys@transformshift{5.781660in}{3.363889in}%
\pgfsys@useobject{currentmarker}{}%
\end{pgfscope}%
\end{pgfscope}%
\begin{pgfscope}%
\pgfsetbuttcap%
\pgfsetroundjoin%
\definecolor{currentfill}{rgb}{0.000000,0.000000,0.000000}%
\pgfsetfillcolor{currentfill}%
\pgfsetlinewidth{0.602250pt}%
\definecolor{currentstroke}{rgb}{0.000000,0.000000,0.000000}%
\pgfsetstrokecolor{currentstroke}%
\pgfsetdash{}{0pt}%
\pgfsys@defobject{currentmarker}{\pgfqpoint{0.000000in}{-0.027778in}}{\pgfqpoint{0.000000in}{0.000000in}}{%
\pgfpathmoveto{\pgfqpoint{0.000000in}{0.000000in}}%
\pgfpathlineto{\pgfqpoint{0.000000in}{-0.027778in}}%
\pgfusepath{stroke,fill}%
}%
\begin{pgfscope}%
\pgfsys@transformshift{5.887868in}{3.363889in}%
\pgfsys@useobject{currentmarker}{}%
\end{pgfscope}%
\end{pgfscope}%
\begin{pgfscope}%
\pgfsetbuttcap%
\pgfsetroundjoin%
\definecolor{currentfill}{rgb}{0.000000,0.000000,0.000000}%
\pgfsetfillcolor{currentfill}%
\pgfsetlinewidth{0.602250pt}%
\definecolor{currentstroke}{rgb}{0.000000,0.000000,0.000000}%
\pgfsetstrokecolor{currentstroke}%
\pgfsetdash{}{0pt}%
\pgfsys@defobject{currentmarker}{\pgfqpoint{0.000000in}{-0.027778in}}{\pgfqpoint{0.000000in}{0.000000in}}{%
\pgfpathmoveto{\pgfqpoint{0.000000in}{0.000000in}}%
\pgfpathlineto{\pgfqpoint{0.000000in}{-0.027778in}}%
\pgfusepath{stroke,fill}%
}%
\begin{pgfscope}%
\pgfsys@transformshift{5.963224in}{3.363889in}%
\pgfsys@useobject{currentmarker}{}%
\end{pgfscope}%
\end{pgfscope}%
\begin{pgfscope}%
\pgfsetbuttcap%
\pgfsetroundjoin%
\definecolor{currentfill}{rgb}{0.000000,0.000000,0.000000}%
\pgfsetfillcolor{currentfill}%
\pgfsetlinewidth{0.602250pt}%
\definecolor{currentstroke}{rgb}{0.000000,0.000000,0.000000}%
\pgfsetstrokecolor{currentstroke}%
\pgfsetdash{}{0pt}%
\pgfsys@defobject{currentmarker}{\pgfqpoint{0.000000in}{-0.027778in}}{\pgfqpoint{0.000000in}{0.000000in}}{%
\pgfpathmoveto{\pgfqpoint{0.000000in}{0.000000in}}%
\pgfpathlineto{\pgfqpoint{0.000000in}{-0.027778in}}%
\pgfusepath{stroke,fill}%
}%
\begin{pgfscope}%
\pgfsys@transformshift{6.021675in}{3.363889in}%
\pgfsys@useobject{currentmarker}{}%
\end{pgfscope}%
\end{pgfscope}%
\begin{pgfscope}%
\pgfsetbuttcap%
\pgfsetroundjoin%
\definecolor{currentfill}{rgb}{0.000000,0.000000,0.000000}%
\pgfsetfillcolor{currentfill}%
\pgfsetlinewidth{0.602250pt}%
\definecolor{currentstroke}{rgb}{0.000000,0.000000,0.000000}%
\pgfsetstrokecolor{currentstroke}%
\pgfsetdash{}{0pt}%
\pgfsys@defobject{currentmarker}{\pgfqpoint{0.000000in}{-0.027778in}}{\pgfqpoint{0.000000in}{0.000000in}}{%
\pgfpathmoveto{\pgfqpoint{0.000000in}{0.000000in}}%
\pgfpathlineto{\pgfqpoint{0.000000in}{-0.027778in}}%
\pgfusepath{stroke,fill}%
}%
\begin{pgfscope}%
\pgfsys@transformshift{6.069433in}{3.363889in}%
\pgfsys@useobject{currentmarker}{}%
\end{pgfscope}%
\end{pgfscope}%
\begin{pgfscope}%
\pgfsetbuttcap%
\pgfsetroundjoin%
\definecolor{currentfill}{rgb}{0.000000,0.000000,0.000000}%
\pgfsetfillcolor{currentfill}%
\pgfsetlinewidth{0.602250pt}%
\definecolor{currentstroke}{rgb}{0.000000,0.000000,0.000000}%
\pgfsetstrokecolor{currentstroke}%
\pgfsetdash{}{0pt}%
\pgfsys@defobject{currentmarker}{\pgfqpoint{0.000000in}{-0.027778in}}{\pgfqpoint{0.000000in}{0.000000in}}{%
\pgfpathmoveto{\pgfqpoint{0.000000in}{0.000000in}}%
\pgfpathlineto{\pgfqpoint{0.000000in}{-0.027778in}}%
\pgfusepath{stroke,fill}%
}%
\begin{pgfscope}%
\pgfsys@transformshift{6.109811in}{3.363889in}%
\pgfsys@useobject{currentmarker}{}%
\end{pgfscope}%
\end{pgfscope}%
\begin{pgfscope}%
\pgfsetbuttcap%
\pgfsetroundjoin%
\definecolor{currentfill}{rgb}{0.000000,0.000000,0.000000}%
\pgfsetfillcolor{currentfill}%
\pgfsetlinewidth{0.602250pt}%
\definecolor{currentstroke}{rgb}{0.000000,0.000000,0.000000}%
\pgfsetstrokecolor{currentstroke}%
\pgfsetdash{}{0pt}%
\pgfsys@defobject{currentmarker}{\pgfqpoint{0.000000in}{-0.027778in}}{\pgfqpoint{0.000000in}{0.000000in}}{%
\pgfpathmoveto{\pgfqpoint{0.000000in}{0.000000in}}%
\pgfpathlineto{\pgfqpoint{0.000000in}{-0.027778in}}%
\pgfusepath{stroke,fill}%
}%
\begin{pgfscope}%
\pgfsys@transformshift{6.144789in}{3.363889in}%
\pgfsys@useobject{currentmarker}{}%
\end{pgfscope}%
\end{pgfscope}%
\begin{pgfscope}%
\pgfsetbuttcap%
\pgfsetroundjoin%
\definecolor{currentfill}{rgb}{0.000000,0.000000,0.000000}%
\pgfsetfillcolor{currentfill}%
\pgfsetlinewidth{0.602250pt}%
\definecolor{currentstroke}{rgb}{0.000000,0.000000,0.000000}%
\pgfsetstrokecolor{currentstroke}%
\pgfsetdash{}{0pt}%
\pgfsys@defobject{currentmarker}{\pgfqpoint{0.000000in}{-0.027778in}}{\pgfqpoint{0.000000in}{0.000000in}}{%
\pgfpathmoveto{\pgfqpoint{0.000000in}{0.000000in}}%
\pgfpathlineto{\pgfqpoint{0.000000in}{-0.027778in}}%
\pgfusepath{stroke,fill}%
}%
\begin{pgfscope}%
\pgfsys@transformshift{6.175641in}{3.363889in}%
\pgfsys@useobject{currentmarker}{}%
\end{pgfscope}%
\end{pgfscope}%
\begin{pgfscope}%
\pgfsetbuttcap%
\pgfsetroundjoin%
\definecolor{currentfill}{rgb}{0.000000,0.000000,0.000000}%
\pgfsetfillcolor{currentfill}%
\pgfsetlinewidth{0.602250pt}%
\definecolor{currentstroke}{rgb}{0.000000,0.000000,0.000000}%
\pgfsetstrokecolor{currentstroke}%
\pgfsetdash{}{0pt}%
\pgfsys@defobject{currentmarker}{\pgfqpoint{0.000000in}{-0.027778in}}{\pgfqpoint{0.000000in}{0.000000in}}{%
\pgfpathmoveto{\pgfqpoint{0.000000in}{0.000000in}}%
\pgfpathlineto{\pgfqpoint{0.000000in}{-0.027778in}}%
\pgfusepath{stroke,fill}%
}%
\begin{pgfscope}%
\pgfsys@transformshift{6.384804in}{3.363889in}%
\pgfsys@useobject{currentmarker}{}%
\end{pgfscope}%
\end{pgfscope}%
\begin{pgfscope}%
\pgfsetbuttcap%
\pgfsetroundjoin%
\definecolor{currentfill}{rgb}{0.000000,0.000000,0.000000}%
\pgfsetfillcolor{currentfill}%
\pgfsetlinewidth{0.602250pt}%
\definecolor{currentstroke}{rgb}{0.000000,0.000000,0.000000}%
\pgfsetstrokecolor{currentstroke}%
\pgfsetdash{}{0pt}%
\pgfsys@defobject{currentmarker}{\pgfqpoint{0.000000in}{-0.027778in}}{\pgfqpoint{0.000000in}{0.000000in}}{%
\pgfpathmoveto{\pgfqpoint{0.000000in}{0.000000in}}%
\pgfpathlineto{\pgfqpoint{0.000000in}{-0.027778in}}%
\pgfusepath{stroke,fill}%
}%
\begin{pgfscope}%
\pgfsys@transformshift{6.491012in}{3.363889in}%
\pgfsys@useobject{currentmarker}{}%
\end{pgfscope}%
\end{pgfscope}%
\begin{pgfscope}%
\pgfsetbuttcap%
\pgfsetroundjoin%
\definecolor{currentfill}{rgb}{0.000000,0.000000,0.000000}%
\pgfsetfillcolor{currentfill}%
\pgfsetlinewidth{0.602250pt}%
\definecolor{currentstroke}{rgb}{0.000000,0.000000,0.000000}%
\pgfsetstrokecolor{currentstroke}%
\pgfsetdash{}{0pt}%
\pgfsys@defobject{currentmarker}{\pgfqpoint{0.000000in}{-0.027778in}}{\pgfqpoint{0.000000in}{0.000000in}}{%
\pgfpathmoveto{\pgfqpoint{0.000000in}{0.000000in}}%
\pgfpathlineto{\pgfqpoint{0.000000in}{-0.027778in}}%
\pgfusepath{stroke,fill}%
}%
\begin{pgfscope}%
\pgfsys@transformshift{6.566368in}{3.363889in}%
\pgfsys@useobject{currentmarker}{}%
\end{pgfscope}%
\end{pgfscope}%
\begin{pgfscope}%
\pgfsetbuttcap%
\pgfsetroundjoin%
\definecolor{currentfill}{rgb}{0.000000,0.000000,0.000000}%
\pgfsetfillcolor{currentfill}%
\pgfsetlinewidth{0.602250pt}%
\definecolor{currentstroke}{rgb}{0.000000,0.000000,0.000000}%
\pgfsetstrokecolor{currentstroke}%
\pgfsetdash{}{0pt}%
\pgfsys@defobject{currentmarker}{\pgfqpoint{0.000000in}{-0.027778in}}{\pgfqpoint{0.000000in}{0.000000in}}{%
\pgfpathmoveto{\pgfqpoint{0.000000in}{0.000000in}}%
\pgfpathlineto{\pgfqpoint{0.000000in}{-0.027778in}}%
\pgfusepath{stroke,fill}%
}%
\begin{pgfscope}%
\pgfsys@transformshift{6.624819in}{3.363889in}%
\pgfsys@useobject{currentmarker}{}%
\end{pgfscope}%
\end{pgfscope}%
\begin{pgfscope}%
\pgfsetbuttcap%
\pgfsetroundjoin%
\definecolor{currentfill}{rgb}{0.000000,0.000000,0.000000}%
\pgfsetfillcolor{currentfill}%
\pgfsetlinewidth{0.602250pt}%
\definecolor{currentstroke}{rgb}{0.000000,0.000000,0.000000}%
\pgfsetstrokecolor{currentstroke}%
\pgfsetdash{}{0pt}%
\pgfsys@defobject{currentmarker}{\pgfqpoint{0.000000in}{-0.027778in}}{\pgfqpoint{0.000000in}{0.000000in}}{%
\pgfpathmoveto{\pgfqpoint{0.000000in}{0.000000in}}%
\pgfpathlineto{\pgfqpoint{0.000000in}{-0.027778in}}%
\pgfusepath{stroke,fill}%
}%
\begin{pgfscope}%
\pgfsys@transformshift{6.672577in}{3.363889in}%
\pgfsys@useobject{currentmarker}{}%
\end{pgfscope}%
\end{pgfscope}%
\begin{pgfscope}%
\pgfsetbuttcap%
\pgfsetroundjoin%
\definecolor{currentfill}{rgb}{0.000000,0.000000,0.000000}%
\pgfsetfillcolor{currentfill}%
\pgfsetlinewidth{0.602250pt}%
\definecolor{currentstroke}{rgb}{0.000000,0.000000,0.000000}%
\pgfsetstrokecolor{currentstroke}%
\pgfsetdash{}{0pt}%
\pgfsys@defobject{currentmarker}{\pgfqpoint{0.000000in}{-0.027778in}}{\pgfqpoint{0.000000in}{0.000000in}}{%
\pgfpathmoveto{\pgfqpoint{0.000000in}{0.000000in}}%
\pgfpathlineto{\pgfqpoint{0.000000in}{-0.027778in}}%
\pgfusepath{stroke,fill}%
}%
\begin{pgfscope}%
\pgfsys@transformshift{6.712955in}{3.363889in}%
\pgfsys@useobject{currentmarker}{}%
\end{pgfscope}%
\end{pgfscope}%
\begin{pgfscope}%
\pgfsetbuttcap%
\pgfsetroundjoin%
\definecolor{currentfill}{rgb}{0.000000,0.000000,0.000000}%
\pgfsetfillcolor{currentfill}%
\pgfsetlinewidth{0.602250pt}%
\definecolor{currentstroke}{rgb}{0.000000,0.000000,0.000000}%
\pgfsetstrokecolor{currentstroke}%
\pgfsetdash{}{0pt}%
\pgfsys@defobject{currentmarker}{\pgfqpoint{0.000000in}{-0.027778in}}{\pgfqpoint{0.000000in}{0.000000in}}{%
\pgfpathmoveto{\pgfqpoint{0.000000in}{0.000000in}}%
\pgfpathlineto{\pgfqpoint{0.000000in}{-0.027778in}}%
\pgfusepath{stroke,fill}%
}%
\begin{pgfscope}%
\pgfsys@transformshift{6.747933in}{3.363889in}%
\pgfsys@useobject{currentmarker}{}%
\end{pgfscope}%
\end{pgfscope}%
\begin{pgfscope}%
\pgfsetbuttcap%
\pgfsetroundjoin%
\definecolor{currentfill}{rgb}{0.000000,0.000000,0.000000}%
\pgfsetfillcolor{currentfill}%
\pgfsetlinewidth{0.602250pt}%
\definecolor{currentstroke}{rgb}{0.000000,0.000000,0.000000}%
\pgfsetstrokecolor{currentstroke}%
\pgfsetdash{}{0pt}%
\pgfsys@defobject{currentmarker}{\pgfqpoint{0.000000in}{-0.027778in}}{\pgfqpoint{0.000000in}{0.000000in}}{%
\pgfpathmoveto{\pgfqpoint{0.000000in}{0.000000in}}%
\pgfpathlineto{\pgfqpoint{0.000000in}{-0.027778in}}%
\pgfusepath{stroke,fill}%
}%
\begin{pgfscope}%
\pgfsys@transformshift{6.778785in}{3.363889in}%
\pgfsys@useobject{currentmarker}{}%
\end{pgfscope}%
\end{pgfscope}%
\begin{pgfscope}%
\pgfsetbuttcap%
\pgfsetroundjoin%
\definecolor{currentfill}{rgb}{0.000000,0.000000,0.000000}%
\pgfsetfillcolor{currentfill}%
\pgfsetlinewidth{0.602250pt}%
\definecolor{currentstroke}{rgb}{0.000000,0.000000,0.000000}%
\pgfsetstrokecolor{currentstroke}%
\pgfsetdash{}{0pt}%
\pgfsys@defobject{currentmarker}{\pgfqpoint{0.000000in}{-0.027778in}}{\pgfqpoint{0.000000in}{0.000000in}}{%
\pgfpathmoveto{\pgfqpoint{0.000000in}{0.000000in}}%
\pgfpathlineto{\pgfqpoint{0.000000in}{-0.027778in}}%
\pgfusepath{stroke,fill}%
}%
\begin{pgfscope}%
\pgfsys@transformshift{6.987948in}{3.363889in}%
\pgfsys@useobject{currentmarker}{}%
\end{pgfscope}%
\end{pgfscope}%
\begin{pgfscope}%
\pgfsetbuttcap%
\pgfsetroundjoin%
\definecolor{currentfill}{rgb}{0.000000,0.000000,0.000000}%
\pgfsetfillcolor{currentfill}%
\pgfsetlinewidth{0.602250pt}%
\definecolor{currentstroke}{rgb}{0.000000,0.000000,0.000000}%
\pgfsetstrokecolor{currentstroke}%
\pgfsetdash{}{0pt}%
\pgfsys@defobject{currentmarker}{\pgfqpoint{0.000000in}{-0.027778in}}{\pgfqpoint{0.000000in}{0.000000in}}{%
\pgfpathmoveto{\pgfqpoint{0.000000in}{0.000000in}}%
\pgfpathlineto{\pgfqpoint{0.000000in}{-0.027778in}}%
\pgfusepath{stroke,fill}%
}%
\begin{pgfscope}%
\pgfsys@transformshift{7.094156in}{3.363889in}%
\pgfsys@useobject{currentmarker}{}%
\end{pgfscope}%
\end{pgfscope}%
\begin{pgfscope}%
\pgfsetbuttcap%
\pgfsetroundjoin%
\definecolor{currentfill}{rgb}{0.000000,0.000000,0.000000}%
\pgfsetfillcolor{currentfill}%
\pgfsetlinewidth{0.602250pt}%
\definecolor{currentstroke}{rgb}{0.000000,0.000000,0.000000}%
\pgfsetstrokecolor{currentstroke}%
\pgfsetdash{}{0pt}%
\pgfsys@defobject{currentmarker}{\pgfqpoint{0.000000in}{-0.027778in}}{\pgfqpoint{0.000000in}{0.000000in}}{%
\pgfpathmoveto{\pgfqpoint{0.000000in}{0.000000in}}%
\pgfpathlineto{\pgfqpoint{0.000000in}{-0.027778in}}%
\pgfusepath{stroke,fill}%
}%
\begin{pgfscope}%
\pgfsys@transformshift{7.169512in}{3.363889in}%
\pgfsys@useobject{currentmarker}{}%
\end{pgfscope}%
\end{pgfscope}%
\begin{pgfscope}%
\pgfsetbuttcap%
\pgfsetroundjoin%
\definecolor{currentfill}{rgb}{0.000000,0.000000,0.000000}%
\pgfsetfillcolor{currentfill}%
\pgfsetlinewidth{0.602250pt}%
\definecolor{currentstroke}{rgb}{0.000000,0.000000,0.000000}%
\pgfsetstrokecolor{currentstroke}%
\pgfsetdash{}{0pt}%
\pgfsys@defobject{currentmarker}{\pgfqpoint{0.000000in}{-0.027778in}}{\pgfqpoint{0.000000in}{0.000000in}}{%
\pgfpathmoveto{\pgfqpoint{0.000000in}{0.000000in}}%
\pgfpathlineto{\pgfqpoint{0.000000in}{-0.027778in}}%
\pgfusepath{stroke,fill}%
}%
\begin{pgfscope}%
\pgfsys@transformshift{7.227963in}{3.363889in}%
\pgfsys@useobject{currentmarker}{}%
\end{pgfscope}%
\end{pgfscope}%
\begin{pgfscope}%
\pgfsetbuttcap%
\pgfsetroundjoin%
\definecolor{currentfill}{rgb}{0.000000,0.000000,0.000000}%
\pgfsetfillcolor{currentfill}%
\pgfsetlinewidth{0.602250pt}%
\definecolor{currentstroke}{rgb}{0.000000,0.000000,0.000000}%
\pgfsetstrokecolor{currentstroke}%
\pgfsetdash{}{0pt}%
\pgfsys@defobject{currentmarker}{\pgfqpoint{0.000000in}{-0.027778in}}{\pgfqpoint{0.000000in}{0.000000in}}{%
\pgfpathmoveto{\pgfqpoint{0.000000in}{0.000000in}}%
\pgfpathlineto{\pgfqpoint{0.000000in}{-0.027778in}}%
\pgfusepath{stroke,fill}%
}%
\begin{pgfscope}%
\pgfsys@transformshift{7.275721in}{3.363889in}%
\pgfsys@useobject{currentmarker}{}%
\end{pgfscope}%
\end{pgfscope}%
\begin{pgfscope}%
\pgfsetbuttcap%
\pgfsetroundjoin%
\definecolor{currentfill}{rgb}{0.000000,0.000000,0.000000}%
\pgfsetfillcolor{currentfill}%
\pgfsetlinewidth{0.602250pt}%
\definecolor{currentstroke}{rgb}{0.000000,0.000000,0.000000}%
\pgfsetstrokecolor{currentstroke}%
\pgfsetdash{}{0pt}%
\pgfsys@defobject{currentmarker}{\pgfqpoint{0.000000in}{-0.027778in}}{\pgfqpoint{0.000000in}{0.000000in}}{%
\pgfpathmoveto{\pgfqpoint{0.000000in}{0.000000in}}%
\pgfpathlineto{\pgfqpoint{0.000000in}{-0.027778in}}%
\pgfusepath{stroke,fill}%
}%
\begin{pgfscope}%
\pgfsys@transformshift{7.316099in}{3.363889in}%
\pgfsys@useobject{currentmarker}{}%
\end{pgfscope}%
\end{pgfscope}%
\begin{pgfscope}%
\pgfsetbuttcap%
\pgfsetroundjoin%
\definecolor{currentfill}{rgb}{0.000000,0.000000,0.000000}%
\pgfsetfillcolor{currentfill}%
\pgfsetlinewidth{0.602250pt}%
\definecolor{currentstroke}{rgb}{0.000000,0.000000,0.000000}%
\pgfsetstrokecolor{currentstroke}%
\pgfsetdash{}{0pt}%
\pgfsys@defobject{currentmarker}{\pgfqpoint{0.000000in}{-0.027778in}}{\pgfqpoint{0.000000in}{0.000000in}}{%
\pgfpathmoveto{\pgfqpoint{0.000000in}{0.000000in}}%
\pgfpathlineto{\pgfqpoint{0.000000in}{-0.027778in}}%
\pgfusepath{stroke,fill}%
}%
\begin{pgfscope}%
\pgfsys@transformshift{7.351077in}{3.363889in}%
\pgfsys@useobject{currentmarker}{}%
\end{pgfscope}%
\end{pgfscope}%
\begin{pgfscope}%
\pgfsetbuttcap%
\pgfsetroundjoin%
\definecolor{currentfill}{rgb}{0.000000,0.000000,0.000000}%
\pgfsetfillcolor{currentfill}%
\pgfsetlinewidth{0.602250pt}%
\definecolor{currentstroke}{rgb}{0.000000,0.000000,0.000000}%
\pgfsetstrokecolor{currentstroke}%
\pgfsetdash{}{0pt}%
\pgfsys@defobject{currentmarker}{\pgfqpoint{0.000000in}{-0.027778in}}{\pgfqpoint{0.000000in}{0.000000in}}{%
\pgfpathmoveto{\pgfqpoint{0.000000in}{0.000000in}}%
\pgfpathlineto{\pgfqpoint{0.000000in}{-0.027778in}}%
\pgfusepath{stroke,fill}%
}%
\begin{pgfscope}%
\pgfsys@transformshift{7.381929in}{3.363889in}%
\pgfsys@useobject{currentmarker}{}%
\end{pgfscope}%
\end{pgfscope}%
\begin{pgfscope}%
\pgfsetbuttcap%
\pgfsetroundjoin%
\definecolor{currentfill}{rgb}{0.000000,0.000000,0.000000}%
\pgfsetfillcolor{currentfill}%
\pgfsetlinewidth{0.602250pt}%
\definecolor{currentstroke}{rgb}{0.000000,0.000000,0.000000}%
\pgfsetstrokecolor{currentstroke}%
\pgfsetdash{}{0pt}%
\pgfsys@defobject{currentmarker}{\pgfqpoint{0.000000in}{-0.027778in}}{\pgfqpoint{0.000000in}{0.000000in}}{%
\pgfpathmoveto{\pgfqpoint{0.000000in}{0.000000in}}%
\pgfpathlineto{\pgfqpoint{0.000000in}{-0.027778in}}%
\pgfusepath{stroke,fill}%
}%
\begin{pgfscope}%
\pgfsys@transformshift{7.591092in}{3.363889in}%
\pgfsys@useobject{currentmarker}{}%
\end{pgfscope}%
\end{pgfscope}%
\begin{pgfscope}%
\pgfsetbuttcap%
\pgfsetroundjoin%
\definecolor{currentfill}{rgb}{0.000000,0.000000,0.000000}%
\pgfsetfillcolor{currentfill}%
\pgfsetlinewidth{0.602250pt}%
\definecolor{currentstroke}{rgb}{0.000000,0.000000,0.000000}%
\pgfsetstrokecolor{currentstroke}%
\pgfsetdash{}{0pt}%
\pgfsys@defobject{currentmarker}{\pgfqpoint{0.000000in}{-0.027778in}}{\pgfqpoint{0.000000in}{0.000000in}}{%
\pgfpathmoveto{\pgfqpoint{0.000000in}{0.000000in}}%
\pgfpathlineto{\pgfqpoint{0.000000in}{-0.027778in}}%
\pgfusepath{stroke,fill}%
}%
\begin{pgfscope}%
\pgfsys@transformshift{7.697300in}{3.363889in}%
\pgfsys@useobject{currentmarker}{}%
\end{pgfscope}%
\end{pgfscope}%
\begin{pgfscope}%
\pgfsetbuttcap%
\pgfsetroundjoin%
\definecolor{currentfill}{rgb}{0.000000,0.000000,0.000000}%
\pgfsetfillcolor{currentfill}%
\pgfsetlinewidth{0.602250pt}%
\definecolor{currentstroke}{rgb}{0.000000,0.000000,0.000000}%
\pgfsetstrokecolor{currentstroke}%
\pgfsetdash{}{0pt}%
\pgfsys@defobject{currentmarker}{\pgfqpoint{0.000000in}{-0.027778in}}{\pgfqpoint{0.000000in}{0.000000in}}{%
\pgfpathmoveto{\pgfqpoint{0.000000in}{0.000000in}}%
\pgfpathlineto{\pgfqpoint{0.000000in}{-0.027778in}}%
\pgfusepath{stroke,fill}%
}%
\begin{pgfscope}%
\pgfsys@transformshift{7.772656in}{3.363889in}%
\pgfsys@useobject{currentmarker}{}%
\end{pgfscope}%
\end{pgfscope}%
\begin{pgfscope}%
\pgfsetbuttcap%
\pgfsetroundjoin%
\definecolor{currentfill}{rgb}{0.000000,0.000000,0.000000}%
\pgfsetfillcolor{currentfill}%
\pgfsetlinewidth{0.803000pt}%
\definecolor{currentstroke}{rgb}{0.000000,0.000000,0.000000}%
\pgfsetstrokecolor{currentstroke}%
\pgfsetdash{}{0pt}%
\pgfsys@defobject{currentmarker}{\pgfqpoint{-0.048611in}{0.000000in}}{\pgfqpoint{0.000000in}{0.000000in}}{%
\pgfpathmoveto{\pgfqpoint{0.000000in}{0.000000in}}%
\pgfpathlineto{\pgfqpoint{-0.048611in}{0.000000in}}%
\pgfusepath{stroke,fill}%
}%
\begin{pgfscope}%
\pgfsys@transformshift{4.482986in}{3.657265in}%
\pgfsys@useobject{currentmarker}{}%
\end{pgfscope}%
\end{pgfscope}%
\begin{pgfscope}%
\definecolor{textcolor}{rgb}{0.000000,0.000000,0.000000}%
\pgfsetstrokecolor{textcolor}%
\pgfsetfillcolor{textcolor}%
\pgftext[x=4.076519in,y=3.604504in,left,base]{\color{textcolor}\sffamily\fontsize{10.000000}{12.000000}\selectfont 0.00}%
\end{pgfscope}%
\begin{pgfscope}%
\pgfsetbuttcap%
\pgfsetroundjoin%
\definecolor{currentfill}{rgb}{0.000000,0.000000,0.000000}%
\pgfsetfillcolor{currentfill}%
\pgfsetlinewidth{0.803000pt}%
\definecolor{currentstroke}{rgb}{0.000000,0.000000,0.000000}%
\pgfsetstrokecolor{currentstroke}%
\pgfsetdash{}{0pt}%
\pgfsys@defobject{currentmarker}{\pgfqpoint{-0.048611in}{0.000000in}}{\pgfqpoint{0.000000in}{0.000000in}}{%
\pgfpathmoveto{\pgfqpoint{0.000000in}{0.000000in}}%
\pgfpathlineto{\pgfqpoint{-0.048611in}{0.000000in}}%
\pgfusepath{stroke,fill}%
}%
\begin{pgfscope}%
\pgfsys@transformshift{4.482986in}{4.007150in}%
\pgfsys@useobject{currentmarker}{}%
\end{pgfscope}%
\end{pgfscope}%
\begin{pgfscope}%
\definecolor{textcolor}{rgb}{0.000000,0.000000,0.000000}%
\pgfsetstrokecolor{textcolor}%
\pgfsetfillcolor{textcolor}%
\pgftext[x=4.076519in,y=3.954388in,left,base]{\color{textcolor}\sffamily\fontsize{10.000000}{12.000000}\selectfont 0.25}%
\end{pgfscope}%
\begin{pgfscope}%
\pgfsetbuttcap%
\pgfsetroundjoin%
\definecolor{currentfill}{rgb}{0.000000,0.000000,0.000000}%
\pgfsetfillcolor{currentfill}%
\pgfsetlinewidth{0.803000pt}%
\definecolor{currentstroke}{rgb}{0.000000,0.000000,0.000000}%
\pgfsetstrokecolor{currentstroke}%
\pgfsetdash{}{0pt}%
\pgfsys@defobject{currentmarker}{\pgfqpoint{-0.048611in}{0.000000in}}{\pgfqpoint{0.000000in}{0.000000in}}{%
\pgfpathmoveto{\pgfqpoint{0.000000in}{0.000000in}}%
\pgfpathlineto{\pgfqpoint{-0.048611in}{0.000000in}}%
\pgfusepath{stroke,fill}%
}%
\begin{pgfscope}%
\pgfsys@transformshift{4.482986in}{4.357034in}%
\pgfsys@useobject{currentmarker}{}%
\end{pgfscope}%
\end{pgfscope}%
\begin{pgfscope}%
\definecolor{textcolor}{rgb}{0.000000,0.000000,0.000000}%
\pgfsetstrokecolor{textcolor}%
\pgfsetfillcolor{textcolor}%
\pgftext[x=4.076519in,y=4.304273in,left,base]{\color{textcolor}\sffamily\fontsize{10.000000}{12.000000}\selectfont 0.50}%
\end{pgfscope}%
\begin{pgfscope}%
\pgfsetbuttcap%
\pgfsetroundjoin%
\definecolor{currentfill}{rgb}{0.000000,0.000000,0.000000}%
\pgfsetfillcolor{currentfill}%
\pgfsetlinewidth{0.803000pt}%
\definecolor{currentstroke}{rgb}{0.000000,0.000000,0.000000}%
\pgfsetstrokecolor{currentstroke}%
\pgfsetdash{}{0pt}%
\pgfsys@defobject{currentmarker}{\pgfqpoint{-0.048611in}{0.000000in}}{\pgfqpoint{0.000000in}{0.000000in}}{%
\pgfpathmoveto{\pgfqpoint{0.000000in}{0.000000in}}%
\pgfpathlineto{\pgfqpoint{-0.048611in}{0.000000in}}%
\pgfusepath{stroke,fill}%
}%
\begin{pgfscope}%
\pgfsys@transformshift{4.482986in}{4.706919in}%
\pgfsys@useobject{currentmarker}{}%
\end{pgfscope}%
\end{pgfscope}%
\begin{pgfscope}%
\definecolor{textcolor}{rgb}{0.000000,0.000000,0.000000}%
\pgfsetstrokecolor{textcolor}%
\pgfsetfillcolor{textcolor}%
\pgftext[x=4.076519in,y=4.654157in,left,base]{\color{textcolor}\sffamily\fontsize{10.000000}{12.000000}\selectfont 0.75}%
\end{pgfscope}%
\begin{pgfscope}%
\pgfsetbuttcap%
\pgfsetroundjoin%
\definecolor{currentfill}{rgb}{0.000000,0.000000,0.000000}%
\pgfsetfillcolor{currentfill}%
\pgfsetlinewidth{0.803000pt}%
\definecolor{currentstroke}{rgb}{0.000000,0.000000,0.000000}%
\pgfsetstrokecolor{currentstroke}%
\pgfsetdash{}{0pt}%
\pgfsys@defobject{currentmarker}{\pgfqpoint{-0.048611in}{0.000000in}}{\pgfqpoint{0.000000in}{0.000000in}}{%
\pgfpathmoveto{\pgfqpoint{0.000000in}{0.000000in}}%
\pgfpathlineto{\pgfqpoint{-0.048611in}{0.000000in}}%
\pgfusepath{stroke,fill}%
}%
\begin{pgfscope}%
\pgfsys@transformshift{4.482986in}{5.056803in}%
\pgfsys@useobject{currentmarker}{}%
\end{pgfscope}%
\end{pgfscope}%
\begin{pgfscope}%
\definecolor{textcolor}{rgb}{0.000000,0.000000,0.000000}%
\pgfsetstrokecolor{textcolor}%
\pgfsetfillcolor{textcolor}%
\pgftext[x=4.076519in,y=5.004041in,left,base]{\color{textcolor}\sffamily\fontsize{10.000000}{12.000000}\selectfont 1.00}%
\end{pgfscope}%
\begin{pgfscope}%
\pgfsetbuttcap%
\pgfsetroundjoin%
\definecolor{currentfill}{rgb}{0.000000,0.000000,0.000000}%
\pgfsetfillcolor{currentfill}%
\pgfsetlinewidth{0.803000pt}%
\definecolor{currentstroke}{rgb}{0.000000,0.000000,0.000000}%
\pgfsetstrokecolor{currentstroke}%
\pgfsetdash{}{0pt}%
\pgfsys@defobject{currentmarker}{\pgfqpoint{-0.048611in}{0.000000in}}{\pgfqpoint{0.000000in}{0.000000in}}{%
\pgfpathmoveto{\pgfqpoint{0.000000in}{0.000000in}}%
\pgfpathlineto{\pgfqpoint{-0.048611in}{0.000000in}}%
\pgfusepath{stroke,fill}%
}%
\begin{pgfscope}%
\pgfsys@transformshift{4.482986in}{5.406687in}%
\pgfsys@useobject{currentmarker}{}%
\end{pgfscope}%
\end{pgfscope}%
\begin{pgfscope}%
\definecolor{textcolor}{rgb}{0.000000,0.000000,0.000000}%
\pgfsetstrokecolor{textcolor}%
\pgfsetfillcolor{textcolor}%
\pgftext[x=4.076519in,y=5.353926in,left,base]{\color{textcolor}\sffamily\fontsize{10.000000}{12.000000}\selectfont 1.25}%
\end{pgfscope}%
\begin{pgfscope}%
\pgfpathrectangle{\pgfqpoint{4.482986in}{3.363889in}}{\pgfqpoint{3.318403in}{2.263889in}}%
\pgfusepath{clip}%
\pgfsetrectcap%
\pgfsetroundjoin%
\pgfsetlinewidth{1.505625pt}%
\definecolor{currentstroke}{rgb}{0.121569,0.466667,0.705882}%
\pgfsetstrokecolor{currentstroke}%
\pgfsetdash{}{0pt}%
\pgfpathmoveto{\pgfqpoint{4.633823in}{3.764997in}}%
\pgfpathlineto{\pgfqpoint{4.784609in}{3.749698in}}%
\pgfpathlineto{\pgfqpoint{4.935395in}{3.790628in}}%
\pgfpathlineto{\pgfqpoint{5.086181in}{3.894543in}}%
\pgfpathlineto{\pgfqpoint{5.236967in}{3.926825in}}%
\pgfpathlineto{\pgfqpoint{5.387753in}{4.151888in}}%
\pgfpathlineto{\pgfqpoint{5.538539in}{4.192528in}}%
\pgfpathlineto{\pgfqpoint{5.689325in}{4.633324in}}%
\pgfpathlineto{\pgfqpoint{5.840111in}{4.399581in}}%
\pgfpathlineto{\pgfqpoint{5.990897in}{4.773934in}}%
\pgfpathlineto{\pgfqpoint{6.141683in}{4.433524in}}%
\pgfpathlineto{\pgfqpoint{6.292469in}{4.762273in}}%
\pgfpathlineto{\pgfqpoint{6.443255in}{5.055804in}}%
\pgfpathlineto{\pgfqpoint{6.594041in}{4.567356in}}%
\pgfpathlineto{\pgfqpoint{6.744827in}{5.055779in}}%
\pgfpathlineto{\pgfqpoint{6.895613in}{5.055784in}}%
\pgfpathlineto{\pgfqpoint{7.046399in}{5.040685in}}%
\pgfpathlineto{\pgfqpoint{7.197185in}{5.055773in}}%
\pgfpathlineto{\pgfqpoint{7.347971in}{5.055789in}}%
\pgfpathlineto{\pgfqpoint{7.498756in}{5.055795in}}%
\pgfpathlineto{\pgfqpoint{7.649542in}{5.055788in}}%
\pgfusepath{stroke}%
\end{pgfscope}%
\begin{pgfscope}%
\pgfpathrectangle{\pgfqpoint{4.482986in}{3.363889in}}{\pgfqpoint{3.318403in}{2.263889in}}%
\pgfusepath{clip}%
\pgfsetrectcap%
\pgfsetroundjoin%
\pgfsetlinewidth{1.505625pt}%
\definecolor{currentstroke}{rgb}{1.000000,0.498039,0.054902}%
\pgfsetstrokecolor{currentstroke}%
\pgfsetdash{}{0pt}%
\pgfpathmoveto{\pgfqpoint{4.633823in}{3.791985in}}%
\pgfpathlineto{\pgfqpoint{4.784609in}{3.801622in}}%
\pgfpathlineto{\pgfqpoint{4.935395in}{3.812519in}}%
\pgfpathlineto{\pgfqpoint{5.086181in}{3.806716in}}%
\pgfpathlineto{\pgfqpoint{5.236967in}{3.948761in}}%
\pgfpathlineto{\pgfqpoint{5.387753in}{3.951914in}}%
\pgfpathlineto{\pgfqpoint{5.538539in}{3.974789in}}%
\pgfpathlineto{\pgfqpoint{5.689325in}{4.399587in}}%
\pgfpathlineto{\pgfqpoint{5.840111in}{4.390414in}}%
\pgfpathlineto{\pgfqpoint{5.990897in}{4.709515in}}%
\pgfpathlineto{\pgfqpoint{6.141683in}{4.737247in}}%
\pgfpathlineto{\pgfqpoint{6.292469in}{5.036418in}}%
\pgfpathlineto{\pgfqpoint{6.443255in}{4.984157in}}%
\pgfpathlineto{\pgfqpoint{6.594041in}{4.887584in}}%
\pgfpathlineto{\pgfqpoint{6.744827in}{5.038014in}}%
\pgfpathlineto{\pgfqpoint{6.895613in}{5.037999in}}%
\pgfpathlineto{\pgfqpoint{7.046399in}{5.037887in}}%
\pgfpathlineto{\pgfqpoint{7.197185in}{5.037968in}}%
\pgfpathlineto{\pgfqpoint{7.347971in}{5.037860in}}%
\pgfpathlineto{\pgfqpoint{7.498756in}{5.037874in}}%
\pgfpathlineto{\pgfqpoint{7.649542in}{5.037922in}}%
\pgfusepath{stroke}%
\end{pgfscope}%
\begin{pgfscope}%
\pgfpathrectangle{\pgfqpoint{4.482986in}{3.363889in}}{\pgfqpoint{3.318403in}{2.263889in}}%
\pgfusepath{clip}%
\pgfsetrectcap%
\pgfsetroundjoin%
\pgfsetlinewidth{1.505625pt}%
\definecolor{currentstroke}{rgb}{0.172549,0.627451,0.172549}%
\pgfsetstrokecolor{currentstroke}%
\pgfsetdash{}{0pt}%
\pgfpathmoveto{\pgfqpoint{4.633823in}{3.763868in}}%
\pgfpathlineto{\pgfqpoint{4.784609in}{3.749270in}}%
\pgfpathlineto{\pgfqpoint{4.935395in}{3.732977in}}%
\pgfpathlineto{\pgfqpoint{5.086181in}{3.873622in}}%
\pgfpathlineto{\pgfqpoint{5.236967in}{3.788222in}}%
\pgfpathlineto{\pgfqpoint{5.387753in}{4.168671in}}%
\pgfpathlineto{\pgfqpoint{5.538539in}{4.120823in}}%
\pgfpathlineto{\pgfqpoint{5.689325in}{3.919974in}}%
\pgfpathlineto{\pgfqpoint{5.840111in}{4.305093in}}%
\pgfpathlineto{\pgfqpoint{5.990897in}{4.227048in}}%
\pgfpathlineto{\pgfqpoint{6.141683in}{4.686787in}}%
\pgfpathlineto{\pgfqpoint{6.292469in}{4.932556in}}%
\pgfpathlineto{\pgfqpoint{6.443255in}{4.935088in}}%
\pgfpathlineto{\pgfqpoint{6.594041in}{4.935533in}}%
\pgfpathlineto{\pgfqpoint{6.744827in}{4.932031in}}%
\pgfpathlineto{\pgfqpoint{6.895613in}{4.934299in}}%
\pgfpathlineto{\pgfqpoint{7.046399in}{4.931718in}}%
\pgfpathlineto{\pgfqpoint{7.197185in}{4.932086in}}%
\pgfpathlineto{\pgfqpoint{7.347971in}{4.932742in}}%
\pgfpathlineto{\pgfqpoint{7.498756in}{4.932514in}}%
\pgfpathlineto{\pgfqpoint{7.649542in}{4.932821in}}%
\pgfusepath{stroke}%
\end{pgfscope}%
\begin{pgfscope}%
\pgfpathrectangle{\pgfqpoint{4.482986in}{3.363889in}}{\pgfqpoint{3.318403in}{2.263889in}}%
\pgfusepath{clip}%
\pgfsetrectcap%
\pgfsetroundjoin%
\pgfsetlinewidth{1.505625pt}%
\definecolor{currentstroke}{rgb}{0.839216,0.152941,0.156863}%
\pgfsetstrokecolor{currentstroke}%
\pgfsetdash{}{0pt}%
\pgfpathmoveto{\pgfqpoint{4.633823in}{3.719885in}}%
\pgfpathlineto{\pgfqpoint{4.784609in}{3.766371in}}%
\pgfpathlineto{\pgfqpoint{4.935395in}{3.715219in}}%
\pgfpathlineto{\pgfqpoint{5.086181in}{3.801762in}}%
\pgfpathlineto{\pgfqpoint{5.236967in}{3.821271in}}%
\pgfpathlineto{\pgfqpoint{5.387753in}{3.886073in}}%
\pgfpathlineto{\pgfqpoint{5.538539in}{3.798609in}}%
\pgfpathlineto{\pgfqpoint{5.689325in}{3.894287in}}%
\pgfpathlineto{\pgfqpoint{5.840111in}{3.929800in}}%
\pgfpathlineto{\pgfqpoint{5.990897in}{4.006194in}}%
\pgfpathlineto{\pgfqpoint{6.141683in}{3.951557in}}%
\pgfpathlineto{\pgfqpoint{6.292469in}{3.943651in}}%
\pgfpathlineto{\pgfqpoint{6.443255in}{3.952953in}}%
\pgfpathlineto{\pgfqpoint{6.594041in}{3.926751in}}%
\pgfpathlineto{\pgfqpoint{6.744827in}{3.925015in}}%
\pgfpathlineto{\pgfqpoint{6.895613in}{3.935325in}}%
\pgfpathlineto{\pgfqpoint{7.046399in}{3.945034in}}%
\pgfpathlineto{\pgfqpoint{7.197185in}{3.938416in}}%
\pgfpathlineto{\pgfqpoint{7.347971in}{3.944363in}}%
\pgfpathlineto{\pgfqpoint{7.498756in}{3.941043in}}%
\pgfpathlineto{\pgfqpoint{7.649542in}{3.939452in}}%
\pgfusepath{stroke}%
\end{pgfscope}%
\begin{pgfscope}%
\pgfpathrectangle{\pgfqpoint{4.482986in}{3.363889in}}{\pgfqpoint{3.318403in}{2.263889in}}%
\pgfusepath{clip}%
\pgfsetrectcap%
\pgfsetroundjoin%
\pgfsetlinewidth{1.505625pt}%
\definecolor{currentstroke}{rgb}{0.580392,0.403922,0.741176}%
\pgfsetstrokecolor{currentstroke}%
\pgfsetdash{}{0pt}%
\pgfpathmoveto{\pgfqpoint{4.633823in}{3.721978in}}%
\pgfpathlineto{\pgfqpoint{4.784609in}{3.741195in}}%
\pgfpathlineto{\pgfqpoint{4.935395in}{3.806489in}}%
\pgfpathlineto{\pgfqpoint{5.086181in}{3.837762in}}%
\pgfpathlineto{\pgfqpoint{5.236967in}{3.749054in}}%
\pgfpathlineto{\pgfqpoint{5.387753in}{3.771638in}}%
\pgfpathlineto{\pgfqpoint{5.538539in}{3.793508in}}%
\pgfpathlineto{\pgfqpoint{5.689325in}{3.782050in}}%
\pgfpathlineto{\pgfqpoint{5.840111in}{3.770886in}}%
\pgfpathlineto{\pgfqpoint{5.990897in}{3.767747in}}%
\pgfpathlineto{\pgfqpoint{6.141683in}{3.771821in}}%
\pgfpathlineto{\pgfqpoint{6.292469in}{3.776206in}}%
\pgfpathlineto{\pgfqpoint{6.443255in}{3.768919in}}%
\pgfpathlineto{\pgfqpoint{6.594041in}{3.772142in}}%
\pgfpathlineto{\pgfqpoint{6.744827in}{3.777330in}}%
\pgfpathlineto{\pgfqpoint{6.895613in}{3.775099in}}%
\pgfpathlineto{\pgfqpoint{7.046399in}{3.777212in}}%
\pgfpathlineto{\pgfqpoint{7.197185in}{3.776424in}}%
\pgfpathlineto{\pgfqpoint{7.347971in}{3.775934in}}%
\pgfpathlineto{\pgfqpoint{7.498756in}{3.776603in}}%
\pgfpathlineto{\pgfqpoint{7.649542in}{3.775983in}}%
\pgfusepath{stroke}%
\end{pgfscope}%
\begin{pgfscope}%
\pgfsetrectcap%
\pgfsetmiterjoin%
\pgfsetlinewidth{0.803000pt}%
\definecolor{currentstroke}{rgb}{0.000000,0.000000,0.000000}%
\pgfsetstrokecolor{currentstroke}%
\pgfsetdash{}{0pt}%
\pgfpathmoveto{\pgfqpoint{4.482986in}{3.363889in}}%
\pgfpathlineto{\pgfqpoint{4.482986in}{5.627778in}}%
\pgfusepath{stroke}%
\end{pgfscope}%
\begin{pgfscope}%
\pgfsetrectcap%
\pgfsetmiterjoin%
\pgfsetlinewidth{0.803000pt}%
\definecolor{currentstroke}{rgb}{0.000000,0.000000,0.000000}%
\pgfsetstrokecolor{currentstroke}%
\pgfsetdash{}{0pt}%
\pgfpathmoveto{\pgfqpoint{7.801389in}{3.363889in}}%
\pgfpathlineto{\pgfqpoint{7.801389in}{5.627778in}}%
\pgfusepath{stroke}%
\end{pgfscope}%
\begin{pgfscope}%
\pgfsetrectcap%
\pgfsetmiterjoin%
\pgfsetlinewidth{0.803000pt}%
\definecolor{currentstroke}{rgb}{0.000000,0.000000,0.000000}%
\pgfsetstrokecolor{currentstroke}%
\pgfsetdash{}{0pt}%
\pgfpathmoveto{\pgfqpoint{4.482986in}{3.363889in}}%
\pgfpathlineto{\pgfqpoint{7.801389in}{3.363889in}}%
\pgfusepath{stroke}%
\end{pgfscope}%
\begin{pgfscope}%
\pgfsetrectcap%
\pgfsetmiterjoin%
\pgfsetlinewidth{0.803000pt}%
\definecolor{currentstroke}{rgb}{0.000000,0.000000,0.000000}%
\pgfsetstrokecolor{currentstroke}%
\pgfsetdash{}{0pt}%
\pgfpathmoveto{\pgfqpoint{4.482986in}{5.627778in}}%
\pgfpathlineto{\pgfqpoint{7.801389in}{5.627778in}}%
\pgfusepath{stroke}%
\end{pgfscope}%
\begin{pgfscope}%
\definecolor{textcolor}{rgb}{0.000000,0.000000,0.000000}%
\pgfsetstrokecolor{textcolor}%
\pgfsetfillcolor{textcolor}%
\pgftext[x=6.142187in,y=5.711111in,,base]{\color{textcolor}\sffamily\fontsize{12.000000}{14.400000}\selectfont \(\displaystyle  N = 32 \)}%
\end{pgfscope}%
\begin{pgfscope}%
\pgfsetbuttcap%
\pgfsetmiterjoin%
\definecolor{currentfill}{rgb}{1.000000,1.000000,1.000000}%
\pgfsetfillcolor{currentfill}%
\pgfsetlinewidth{0.000000pt}%
\definecolor{currentstroke}{rgb}{0.000000,0.000000,0.000000}%
\pgfsetstrokecolor{currentstroke}%
\pgfsetstrokeopacity{0.000000}%
\pgfsetdash{}{0pt}%
\pgfpathmoveto{\pgfqpoint{0.557986in}{0.438889in}}%
\pgfpathlineto{\pgfqpoint{3.876389in}{0.438889in}}%
\pgfpathlineto{\pgfqpoint{3.876389in}{2.702778in}}%
\pgfpathlineto{\pgfqpoint{0.557986in}{2.702778in}}%
\pgfpathclose%
\pgfusepath{fill}%
\end{pgfscope}%
\begin{pgfscope}%
\pgfpathrectangle{\pgfqpoint{0.557986in}{0.438889in}}{\pgfqpoint{3.318403in}{2.263889in}}%
\pgfusepath{clip}%
\pgfsetbuttcap%
\pgfsetroundjoin%
\definecolor{currentfill}{rgb}{0.121569,0.466667,0.705882}%
\pgfsetfillcolor{currentfill}%
\pgfsetlinewidth{1.003750pt}%
\definecolor{currentstroke}{rgb}{0.121569,0.466667,0.705882}%
\pgfsetstrokecolor{currentstroke}%
\pgfsetdash{}{0pt}%
\pgfsys@defobject{currentmarker}{\pgfqpoint{-0.009821in}{-0.009821in}}{\pgfqpoint{0.009821in}{0.009821in}}{%
\pgfpathmoveto{\pgfqpoint{0.000000in}{-0.009821in}}%
\pgfpathcurveto{\pgfqpoint{0.002605in}{-0.009821in}}{\pgfqpoint{0.005103in}{-0.008786in}}{\pgfqpoint{0.006944in}{-0.006944in}}%
\pgfpathcurveto{\pgfqpoint{0.008786in}{-0.005103in}}{\pgfqpoint{0.009821in}{-0.002605in}}{\pgfqpoint{0.009821in}{0.000000in}}%
\pgfpathcurveto{\pgfqpoint{0.009821in}{0.002605in}}{\pgfqpoint{0.008786in}{0.005103in}}{\pgfqpoint{0.006944in}{0.006944in}}%
\pgfpathcurveto{\pgfqpoint{0.005103in}{0.008786in}}{\pgfqpoint{0.002605in}{0.009821in}}{\pgfqpoint{0.000000in}{0.009821in}}%
\pgfpathcurveto{\pgfqpoint{-0.002605in}{0.009821in}}{\pgfqpoint{-0.005103in}{0.008786in}}{\pgfqpoint{-0.006944in}{0.006944in}}%
\pgfpathcurveto{\pgfqpoint{-0.008786in}{0.005103in}}{\pgfqpoint{-0.009821in}{0.002605in}}{\pgfqpoint{-0.009821in}{0.000000in}}%
\pgfpathcurveto{\pgfqpoint{-0.009821in}{-0.002605in}}{\pgfqpoint{-0.008786in}{-0.005103in}}{\pgfqpoint{-0.006944in}{-0.006944in}}%
\pgfpathcurveto{\pgfqpoint{-0.005103in}{-0.008786in}}{\pgfqpoint{-0.002605in}{-0.009821in}}{\pgfqpoint{0.000000in}{-0.009821in}}%
\pgfpathclose%
\pgfusepath{stroke,fill}%
}%
\begin{pgfscope}%
\pgfsys@transformshift{0.708823in}{0.677637in}%
\pgfsys@useobject{currentmarker}{}%
\end{pgfscope}%
\begin{pgfscope}%
\pgfsys@transformshift{0.859609in}{0.665474in}%
\pgfsys@useobject{currentmarker}{}%
\end{pgfscope}%
\begin{pgfscope}%
\pgfsys@transformshift{1.010395in}{0.704744in}%
\pgfsys@useobject{currentmarker}{}%
\end{pgfscope}%
\begin{pgfscope}%
\pgfsys@transformshift{1.161181in}{0.696222in}%
\pgfsys@useobject{currentmarker}{}%
\end{pgfscope}%
\begin{pgfscope}%
\pgfsys@transformshift{1.311967in}{0.741800in}%
\pgfsys@useobject{currentmarker}{}%
\end{pgfscope}%
\begin{pgfscope}%
\pgfsys@transformshift{1.462753in}{0.754834in}%
\pgfsys@useobject{currentmarker}{}%
\end{pgfscope}%
\begin{pgfscope}%
\pgfsys@transformshift{1.613539in}{0.828864in}%
\pgfsys@useobject{currentmarker}{}%
\end{pgfscope}%
\begin{pgfscope}%
\pgfsys@transformshift{1.764325in}{0.936655in}%
\pgfsys@useobject{currentmarker}{}%
\end{pgfscope}%
\begin{pgfscope}%
\pgfsys@transformshift{1.915111in}{1.059509in}%
\pgfsys@useobject{currentmarker}{}%
\end{pgfscope}%
\begin{pgfscope}%
\pgfsys@transformshift{2.065897in}{0.890936in}%
\pgfsys@useobject{currentmarker}{}%
\end{pgfscope}%
\begin{pgfscope}%
\pgfsys@transformshift{2.216683in}{0.983823in}%
\pgfsys@useobject{currentmarker}{}%
\end{pgfscope}%
\begin{pgfscope}%
\pgfsys@transformshift{2.367469in}{1.293789in}%
\pgfsys@useobject{currentmarker}{}%
\end{pgfscope}%
\begin{pgfscope}%
\pgfsys@transformshift{2.518255in}{1.459498in}%
\pgfsys@useobject{currentmarker}{}%
\end{pgfscope}%
\begin{pgfscope}%
\pgfsys@transformshift{2.669041in}{2.050375in}%
\pgfsys@useobject{currentmarker}{}%
\end{pgfscope}%
\begin{pgfscope}%
\pgfsys@transformshift{2.819827in}{1.935330in}%
\pgfsys@useobject{currentmarker}{}%
\end{pgfscope}%
\begin{pgfscope}%
\pgfsys@transformshift{2.970613in}{1.835961in}%
\pgfsys@useobject{currentmarker}{}%
\end{pgfscope}%
\begin{pgfscope}%
\pgfsys@transformshift{3.121399in}{2.105455in}%
\pgfsys@useobject{currentmarker}{}%
\end{pgfscope}%
\begin{pgfscope}%
\pgfsys@transformshift{3.272185in}{1.841539in}%
\pgfsys@useobject{currentmarker}{}%
\end{pgfscope}%
\begin{pgfscope}%
\pgfsys@transformshift{3.422971in}{1.892115in}%
\pgfsys@useobject{currentmarker}{}%
\end{pgfscope}%
\begin{pgfscope}%
\pgfsys@transformshift{3.573756in}{2.130019in}%
\pgfsys@useobject{currentmarker}{}%
\end{pgfscope}%
\begin{pgfscope}%
\pgfsys@transformshift{3.724542in}{2.010187in}%
\pgfsys@useobject{currentmarker}{}%
\end{pgfscope}%
\end{pgfscope}%
\begin{pgfscope}%
\pgfpathrectangle{\pgfqpoint{0.557986in}{0.438889in}}{\pgfqpoint{3.318403in}{2.263889in}}%
\pgfusepath{clip}%
\pgfsetbuttcap%
\pgfsetroundjoin%
\definecolor{currentfill}{rgb}{0.121569,0.466667,0.705882}%
\pgfsetfillcolor{currentfill}%
\pgfsetfillopacity{0.300000}%
\pgfsetlinewidth{1.003750pt}%
\definecolor{currentstroke}{rgb}{0.121569,0.466667,0.705882}%
\pgfsetstrokecolor{currentstroke}%
\pgfsetstrokeopacity{0.300000}%
\pgfsetdash{}{0pt}%
\pgfpathmoveto{\pgfqpoint{0.708823in}{0.718164in}}%
\pgfpathlineto{\pgfqpoint{0.708823in}{0.637110in}}%
\pgfpathlineto{\pgfqpoint{0.859609in}{0.654559in}}%
\pgfpathlineto{\pgfqpoint{1.010395in}{0.629657in}}%
\pgfpathlineto{\pgfqpoint{1.161181in}{0.665361in}}%
\pgfpathlineto{\pgfqpoint{1.311967in}{0.707876in}}%
\pgfpathlineto{\pgfqpoint{1.462753in}{0.650313in}}%
\pgfpathlineto{\pgfqpoint{1.613539in}{0.710611in}}%
\pgfpathlineto{\pgfqpoint{1.764325in}{0.541793in}}%
\pgfpathlineto{\pgfqpoint{1.915111in}{1.043917in}}%
\pgfpathlineto{\pgfqpoint{2.065897in}{0.711899in}}%
\pgfpathlineto{\pgfqpoint{2.216683in}{0.715802in}}%
\pgfpathlineto{\pgfqpoint{2.367469in}{1.085089in}}%
\pgfpathlineto{\pgfqpoint{2.518255in}{0.549824in}}%
\pgfpathlineto{\pgfqpoint{2.669041in}{1.947977in}}%
\pgfpathlineto{\pgfqpoint{2.819827in}{1.430001in}}%
\pgfpathlineto{\pgfqpoint{2.970613in}{1.072048in}}%
\pgfpathlineto{\pgfqpoint{3.121399in}{2.041641in}}%
\pgfpathlineto{\pgfqpoint{3.272185in}{1.293992in}}%
\pgfpathlineto{\pgfqpoint{3.422971in}{1.274038in}}%
\pgfpathlineto{\pgfqpoint{3.573756in}{2.130002in}}%
\pgfpathlineto{\pgfqpoint{3.724542in}{1.698864in}}%
\pgfpathlineto{\pgfqpoint{3.724542in}{2.321509in}}%
\pgfpathlineto{\pgfqpoint{3.724542in}{2.321509in}}%
\pgfpathlineto{\pgfqpoint{3.573756in}{2.130036in}}%
\pgfpathlineto{\pgfqpoint{3.422971in}{2.510193in}}%
\pgfpathlineto{\pgfqpoint{3.272185in}{2.389086in}}%
\pgfpathlineto{\pgfqpoint{3.121399in}{2.169270in}}%
\pgfpathlineto{\pgfqpoint{2.970613in}{2.599874in}}%
\pgfpathlineto{\pgfqpoint{2.819827in}{2.440658in}}%
\pgfpathlineto{\pgfqpoint{2.669041in}{2.152773in}}%
\pgfpathlineto{\pgfqpoint{2.518255in}{2.369172in}}%
\pgfpathlineto{\pgfqpoint{2.367469in}{1.502488in}}%
\pgfpathlineto{\pgfqpoint{2.216683in}{1.251844in}}%
\pgfpathlineto{\pgfqpoint{2.065897in}{1.069972in}}%
\pgfpathlineto{\pgfqpoint{1.915111in}{1.075100in}}%
\pgfpathlineto{\pgfqpoint{1.764325in}{1.331518in}}%
\pgfpathlineto{\pgfqpoint{1.613539in}{0.947117in}}%
\pgfpathlineto{\pgfqpoint{1.462753in}{0.859356in}}%
\pgfpathlineto{\pgfqpoint{1.311967in}{0.775723in}}%
\pgfpathlineto{\pgfqpoint{1.161181in}{0.727084in}}%
\pgfpathlineto{\pgfqpoint{1.010395in}{0.779831in}}%
\pgfpathlineto{\pgfqpoint{0.859609in}{0.676388in}}%
\pgfpathlineto{\pgfqpoint{0.708823in}{0.718164in}}%
\pgfpathclose%
\pgfusepath{stroke,fill}%
\end{pgfscope}%
\begin{pgfscope}%
\pgfpathrectangle{\pgfqpoint{0.557986in}{0.438889in}}{\pgfqpoint{3.318403in}{2.263889in}}%
\pgfusepath{clip}%
\pgfsetbuttcap%
\pgfsetroundjoin%
\definecolor{currentfill}{rgb}{1.000000,0.498039,0.054902}%
\pgfsetfillcolor{currentfill}%
\pgfsetlinewidth{1.003750pt}%
\definecolor{currentstroke}{rgb}{1.000000,0.498039,0.054902}%
\pgfsetstrokecolor{currentstroke}%
\pgfsetdash{}{0pt}%
\pgfsys@defobject{currentmarker}{\pgfqpoint{-0.009821in}{-0.009821in}}{\pgfqpoint{0.009821in}{0.009821in}}{%
\pgfpathmoveto{\pgfqpoint{0.000000in}{-0.009821in}}%
\pgfpathcurveto{\pgfqpoint{0.002605in}{-0.009821in}}{\pgfqpoint{0.005103in}{-0.008786in}}{\pgfqpoint{0.006944in}{-0.006944in}}%
\pgfpathcurveto{\pgfqpoint{0.008786in}{-0.005103in}}{\pgfqpoint{0.009821in}{-0.002605in}}{\pgfqpoint{0.009821in}{0.000000in}}%
\pgfpathcurveto{\pgfqpoint{0.009821in}{0.002605in}}{\pgfqpoint{0.008786in}{0.005103in}}{\pgfqpoint{0.006944in}{0.006944in}}%
\pgfpathcurveto{\pgfqpoint{0.005103in}{0.008786in}}{\pgfqpoint{0.002605in}{0.009821in}}{\pgfqpoint{0.000000in}{0.009821in}}%
\pgfpathcurveto{\pgfqpoint{-0.002605in}{0.009821in}}{\pgfqpoint{-0.005103in}{0.008786in}}{\pgfqpoint{-0.006944in}{0.006944in}}%
\pgfpathcurveto{\pgfqpoint{-0.008786in}{0.005103in}}{\pgfqpoint{-0.009821in}{0.002605in}}{\pgfqpoint{-0.009821in}{0.000000in}}%
\pgfpathcurveto{\pgfqpoint{-0.009821in}{-0.002605in}}{\pgfqpoint{-0.008786in}{-0.005103in}}{\pgfqpoint{-0.006944in}{-0.006944in}}%
\pgfpathcurveto{\pgfqpoint{-0.005103in}{-0.008786in}}{\pgfqpoint{-0.002605in}{-0.009821in}}{\pgfqpoint{0.000000in}{-0.009821in}}%
\pgfpathclose%
\pgfusepath{stroke,fill}%
}%
\begin{pgfscope}%
\pgfsys@transformshift{0.708823in}{0.671527in}%
\pgfsys@useobject{currentmarker}{}%
\end{pgfscope}%
\begin{pgfscope}%
\pgfsys@transformshift{0.859609in}{0.678928in}%
\pgfsys@useobject{currentmarker}{}%
\end{pgfscope}%
\begin{pgfscope}%
\pgfsys@transformshift{1.010395in}{0.677951in}%
\pgfsys@useobject{currentmarker}{}%
\end{pgfscope}%
\begin{pgfscope}%
\pgfsys@transformshift{1.161181in}{0.717109in}%
\pgfsys@useobject{currentmarker}{}%
\end{pgfscope}%
\begin{pgfscope}%
\pgfsys@transformshift{1.311967in}{0.690322in}%
\pgfsys@useobject{currentmarker}{}%
\end{pgfscope}%
\begin{pgfscope}%
\pgfsys@transformshift{1.462753in}{0.709863in}%
\pgfsys@useobject{currentmarker}{}%
\end{pgfscope}%
\begin{pgfscope}%
\pgfsys@transformshift{1.613539in}{0.751513in}%
\pgfsys@useobject{currentmarker}{}%
\end{pgfscope}%
\begin{pgfscope}%
\pgfsys@transformshift{1.764325in}{0.820832in}%
\pgfsys@useobject{currentmarker}{}%
\end{pgfscope}%
\begin{pgfscope}%
\pgfsys@transformshift{1.915111in}{0.878569in}%
\pgfsys@useobject{currentmarker}{}%
\end{pgfscope}%
\begin{pgfscope}%
\pgfsys@transformshift{2.065897in}{0.914162in}%
\pgfsys@useobject{currentmarker}{}%
\end{pgfscope}%
\begin{pgfscope}%
\pgfsys@transformshift{2.216683in}{1.137951in}%
\pgfsys@useobject{currentmarker}{}%
\end{pgfscope}%
\begin{pgfscope}%
\pgfsys@transformshift{2.367469in}{1.175775in}%
\pgfsys@useobject{currentmarker}{}%
\end{pgfscope}%
\begin{pgfscope}%
\pgfsys@transformshift{2.518255in}{1.410782in}%
\pgfsys@useobject{currentmarker}{}%
\end{pgfscope}%
\begin{pgfscope}%
\pgfsys@transformshift{2.669041in}{1.700231in}%
\pgfsys@useobject{currentmarker}{}%
\end{pgfscope}%
\begin{pgfscope}%
\pgfsys@transformshift{2.819827in}{2.063352in}%
\pgfsys@useobject{currentmarker}{}%
\end{pgfscope}%
\begin{pgfscope}%
\pgfsys@transformshift{2.970613in}{2.111064in}%
\pgfsys@useobject{currentmarker}{}%
\end{pgfscope}%
\begin{pgfscope}%
\pgfsys@transformshift{3.121399in}{1.880269in}%
\pgfsys@useobject{currentmarker}{}%
\end{pgfscope}%
\begin{pgfscope}%
\pgfsys@transformshift{3.272185in}{1.837341in}%
\pgfsys@useobject{currentmarker}{}%
\end{pgfscope}%
\begin{pgfscope}%
\pgfsys@transformshift{3.422971in}{2.111034in}%
\pgfsys@useobject{currentmarker}{}%
\end{pgfscope}%
\begin{pgfscope}%
\pgfsys@transformshift{3.573756in}{2.111044in}%
\pgfsys@useobject{currentmarker}{}%
\end{pgfscope}%
\begin{pgfscope}%
\pgfsys@transformshift{3.724542in}{2.042036in}%
\pgfsys@useobject{currentmarker}{}%
\end{pgfscope}%
\end{pgfscope}%
\begin{pgfscope}%
\pgfpathrectangle{\pgfqpoint{0.557986in}{0.438889in}}{\pgfqpoint{3.318403in}{2.263889in}}%
\pgfusepath{clip}%
\pgfsetbuttcap%
\pgfsetroundjoin%
\definecolor{currentfill}{rgb}{1.000000,0.498039,0.054902}%
\pgfsetfillcolor{currentfill}%
\pgfsetfillopacity{0.300000}%
\pgfsetlinewidth{1.003750pt}%
\definecolor{currentstroke}{rgb}{1.000000,0.498039,0.054902}%
\pgfsetstrokecolor{currentstroke}%
\pgfsetstrokeopacity{0.300000}%
\pgfsetdash{}{0pt}%
\pgfpathmoveto{\pgfqpoint{0.708823in}{0.693524in}}%
\pgfpathlineto{\pgfqpoint{0.708823in}{0.649529in}}%
\pgfpathlineto{\pgfqpoint{0.859609in}{0.660819in}}%
\pgfpathlineto{\pgfqpoint{1.010395in}{0.667162in}}%
\pgfpathlineto{\pgfqpoint{1.161181in}{0.693979in}}%
\pgfpathlineto{\pgfqpoint{1.311967in}{0.623772in}}%
\pgfpathlineto{\pgfqpoint{1.462753in}{0.630075in}}%
\pgfpathlineto{\pgfqpoint{1.613539in}{0.671609in}}%
\pgfpathlineto{\pgfqpoint{1.764325in}{0.773559in}}%
\pgfpathlineto{\pgfqpoint{1.915111in}{0.641307in}}%
\pgfpathlineto{\pgfqpoint{2.065897in}{0.763457in}}%
\pgfpathlineto{\pgfqpoint{2.216683in}{0.876327in}}%
\pgfpathlineto{\pgfqpoint{2.367469in}{0.719466in}}%
\pgfpathlineto{\pgfqpoint{2.518255in}{0.615356in}}%
\pgfpathlineto{\pgfqpoint{2.669041in}{0.906592in}}%
\pgfpathlineto{\pgfqpoint{2.819827in}{1.987516in}}%
\pgfpathlineto{\pgfqpoint{2.970613in}{2.110907in}}%
\pgfpathlineto{\pgfqpoint{3.121399in}{1.280697in}}%
\pgfpathlineto{\pgfqpoint{3.272185in}{1.126399in}}%
\pgfpathlineto{\pgfqpoint{3.422971in}{2.110935in}}%
\pgfpathlineto{\pgfqpoint{3.573756in}{2.110952in}}%
\pgfpathlineto{\pgfqpoint{3.724542in}{1.862729in}}%
\pgfpathlineto{\pgfqpoint{3.724542in}{2.221343in}}%
\pgfpathlineto{\pgfqpoint{3.724542in}{2.221343in}}%
\pgfpathlineto{\pgfqpoint{3.573756in}{2.111137in}}%
\pgfpathlineto{\pgfqpoint{3.422971in}{2.111134in}}%
\pgfpathlineto{\pgfqpoint{3.272185in}{2.548283in}}%
\pgfpathlineto{\pgfqpoint{3.121399in}{2.479841in}}%
\pgfpathlineto{\pgfqpoint{2.970613in}{2.111221in}}%
\pgfpathlineto{\pgfqpoint{2.819827in}{2.139188in}}%
\pgfpathlineto{\pgfqpoint{2.669041in}{2.493871in}}%
\pgfpathlineto{\pgfqpoint{2.518255in}{2.206209in}}%
\pgfpathlineto{\pgfqpoint{2.367469in}{1.632083in}}%
\pgfpathlineto{\pgfqpoint{2.216683in}{1.399575in}}%
\pgfpathlineto{\pgfqpoint{2.065897in}{1.064867in}}%
\pgfpathlineto{\pgfqpoint{1.915111in}{1.115832in}}%
\pgfpathlineto{\pgfqpoint{1.764325in}{0.868105in}}%
\pgfpathlineto{\pgfqpoint{1.613539in}{0.831417in}}%
\pgfpathlineto{\pgfqpoint{1.462753in}{0.789651in}}%
\pgfpathlineto{\pgfqpoint{1.311967in}{0.756871in}}%
\pgfpathlineto{\pgfqpoint{1.161181in}{0.740239in}}%
\pgfpathlineto{\pgfqpoint{1.010395in}{0.688740in}}%
\pgfpathlineto{\pgfqpoint{0.859609in}{0.697037in}}%
\pgfpathlineto{\pgfqpoint{0.708823in}{0.693524in}}%
\pgfpathclose%
\pgfusepath{stroke,fill}%
\end{pgfscope}%
\begin{pgfscope}%
\pgfpathrectangle{\pgfqpoint{0.557986in}{0.438889in}}{\pgfqpoint{3.318403in}{2.263889in}}%
\pgfusepath{clip}%
\pgfsetbuttcap%
\pgfsetroundjoin%
\definecolor{currentfill}{rgb}{0.172549,0.627451,0.172549}%
\pgfsetfillcolor{currentfill}%
\pgfsetlinewidth{1.003750pt}%
\definecolor{currentstroke}{rgb}{0.172549,0.627451,0.172549}%
\pgfsetstrokecolor{currentstroke}%
\pgfsetdash{}{0pt}%
\pgfsys@defobject{currentmarker}{\pgfqpoint{-0.009821in}{-0.009821in}}{\pgfqpoint{0.009821in}{0.009821in}}{%
\pgfpathmoveto{\pgfqpoint{0.000000in}{-0.009821in}}%
\pgfpathcurveto{\pgfqpoint{0.002605in}{-0.009821in}}{\pgfqpoint{0.005103in}{-0.008786in}}{\pgfqpoint{0.006944in}{-0.006944in}}%
\pgfpathcurveto{\pgfqpoint{0.008786in}{-0.005103in}}{\pgfqpoint{0.009821in}{-0.002605in}}{\pgfqpoint{0.009821in}{0.000000in}}%
\pgfpathcurveto{\pgfqpoint{0.009821in}{0.002605in}}{\pgfqpoint{0.008786in}{0.005103in}}{\pgfqpoint{0.006944in}{0.006944in}}%
\pgfpathcurveto{\pgfqpoint{0.005103in}{0.008786in}}{\pgfqpoint{0.002605in}{0.009821in}}{\pgfqpoint{0.000000in}{0.009821in}}%
\pgfpathcurveto{\pgfqpoint{-0.002605in}{0.009821in}}{\pgfqpoint{-0.005103in}{0.008786in}}{\pgfqpoint{-0.006944in}{0.006944in}}%
\pgfpathcurveto{\pgfqpoint{-0.008786in}{0.005103in}}{\pgfqpoint{-0.009821in}{0.002605in}}{\pgfqpoint{-0.009821in}{0.000000in}}%
\pgfpathcurveto{\pgfqpoint{-0.009821in}{-0.002605in}}{\pgfqpoint{-0.008786in}{-0.005103in}}{\pgfqpoint{-0.006944in}{-0.006944in}}%
\pgfpathcurveto{\pgfqpoint{-0.005103in}{-0.008786in}}{\pgfqpoint{-0.002605in}{-0.009821in}}{\pgfqpoint{0.000000in}{-0.009821in}}%
\pgfpathclose%
\pgfusepath{stroke,fill}%
}%
\begin{pgfscope}%
\pgfsys@transformshift{0.708823in}{0.680541in}%
\pgfsys@useobject{currentmarker}{}%
\end{pgfscope}%
\begin{pgfscope}%
\pgfsys@transformshift{0.859609in}{0.679997in}%
\pgfsys@useobject{currentmarker}{}%
\end{pgfscope}%
\begin{pgfscope}%
\pgfsys@transformshift{1.010395in}{0.708723in}%
\pgfsys@useobject{currentmarker}{}%
\end{pgfscope}%
\begin{pgfscope}%
\pgfsys@transformshift{1.161181in}{0.715009in}%
\pgfsys@useobject{currentmarker}{}%
\end{pgfscope}%
\begin{pgfscope}%
\pgfsys@transformshift{1.311967in}{0.742244in}%
\pgfsys@useobject{currentmarker}{}%
\end{pgfscope}%
\begin{pgfscope}%
\pgfsys@transformshift{1.462753in}{0.765060in}%
\pgfsys@useobject{currentmarker}{}%
\end{pgfscope}%
\begin{pgfscope}%
\pgfsys@transformshift{1.613539in}{0.762023in}%
\pgfsys@useobject{currentmarker}{}%
\end{pgfscope}%
\begin{pgfscope}%
\pgfsys@transformshift{1.764325in}{0.857474in}%
\pgfsys@useobject{currentmarker}{}%
\end{pgfscope}%
\begin{pgfscope}%
\pgfsys@transformshift{1.915111in}{0.805723in}%
\pgfsys@useobject{currentmarker}{}%
\end{pgfscope}%
\begin{pgfscope}%
\pgfsys@transformshift{2.065897in}{0.949944in}%
\pgfsys@useobject{currentmarker}{}%
\end{pgfscope}%
\begin{pgfscope}%
\pgfsys@transformshift{2.216683in}{1.102171in}%
\pgfsys@useobject{currentmarker}{}%
\end{pgfscope}%
\begin{pgfscope}%
\pgfsys@transformshift{2.367469in}{0.947582in}%
\pgfsys@useobject{currentmarker}{}%
\end{pgfscope}%
\begin{pgfscope}%
\pgfsys@transformshift{2.518255in}{1.030691in}%
\pgfsys@useobject{currentmarker}{}%
\end{pgfscope}%
\begin{pgfscope}%
\pgfsys@transformshift{2.669041in}{1.292833in}%
\pgfsys@useobject{currentmarker}{}%
\end{pgfscope}%
\begin{pgfscope}%
\pgfsys@transformshift{2.819827in}{1.981429in}%
\pgfsys@useobject{currentmarker}{}%
\end{pgfscope}%
\begin{pgfscope}%
\pgfsys@transformshift{2.970613in}{1.986670in}%
\pgfsys@useobject{currentmarker}{}%
\end{pgfscope}%
\begin{pgfscope}%
\pgfsys@transformshift{3.121399in}{1.777189in}%
\pgfsys@useobject{currentmarker}{}%
\end{pgfscope}%
\begin{pgfscope}%
\pgfsys@transformshift{3.272185in}{1.930304in}%
\pgfsys@useobject{currentmarker}{}%
\end{pgfscope}%
\begin{pgfscope}%
\pgfsys@transformshift{3.422971in}{1.999769in}%
\pgfsys@useobject{currentmarker}{}%
\end{pgfscope}%
\begin{pgfscope}%
\pgfsys@transformshift{3.573756in}{1.999389in}%
\pgfsys@useobject{currentmarker}{}%
\end{pgfscope}%
\begin{pgfscope}%
\pgfsys@transformshift{3.724542in}{1.999199in}%
\pgfsys@useobject{currentmarker}{}%
\end{pgfscope}%
\end{pgfscope}%
\begin{pgfscope}%
\pgfpathrectangle{\pgfqpoint{0.557986in}{0.438889in}}{\pgfqpoint{3.318403in}{2.263889in}}%
\pgfusepath{clip}%
\pgfsetbuttcap%
\pgfsetroundjoin%
\definecolor{currentfill}{rgb}{0.172549,0.627451,0.172549}%
\pgfsetfillcolor{currentfill}%
\pgfsetfillopacity{0.300000}%
\pgfsetlinewidth{1.003750pt}%
\definecolor{currentstroke}{rgb}{0.172549,0.627451,0.172549}%
\pgfsetstrokecolor{currentstroke}%
\pgfsetstrokeopacity{0.300000}%
\pgfsetdash{}{0pt}%
\pgfpathmoveto{\pgfqpoint{0.708823in}{0.722498in}}%
\pgfpathlineto{\pgfqpoint{0.708823in}{0.638585in}}%
\pgfpathlineto{\pgfqpoint{0.859609in}{0.653403in}}%
\pgfpathlineto{\pgfqpoint{1.010395in}{0.660361in}}%
\pgfpathlineto{\pgfqpoint{1.161181in}{0.632324in}}%
\pgfpathlineto{\pgfqpoint{1.311967in}{0.689922in}}%
\pgfpathlineto{\pgfqpoint{1.462753in}{0.739996in}}%
\pgfpathlineto{\pgfqpoint{1.613539in}{0.721636in}}%
\pgfpathlineto{\pgfqpoint{1.764325in}{0.546675in}}%
\pgfpathlineto{\pgfqpoint{1.915111in}{0.641239in}}%
\pgfpathlineto{\pgfqpoint{2.065897in}{0.637163in}}%
\pgfpathlineto{\pgfqpoint{2.216683in}{0.759014in}}%
\pgfpathlineto{\pgfqpoint{2.367469in}{0.627167in}}%
\pgfpathlineto{\pgfqpoint{2.518255in}{0.848108in}}%
\pgfpathlineto{\pgfqpoint{2.669041in}{0.631905in}}%
\pgfpathlineto{\pgfqpoint{2.819827in}{1.945514in}}%
\pgfpathlineto{\pgfqpoint{2.970613in}{1.953815in}}%
\pgfpathlineto{\pgfqpoint{3.121399in}{1.201441in}}%
\pgfpathlineto{\pgfqpoint{3.272185in}{1.750920in}}%
\pgfpathlineto{\pgfqpoint{3.422971in}{1.998616in}}%
\pgfpathlineto{\pgfqpoint{3.573756in}{1.998343in}}%
\pgfpathlineto{\pgfqpoint{3.724542in}{1.998258in}}%
\pgfpathlineto{\pgfqpoint{3.724542in}{2.000139in}}%
\pgfpathlineto{\pgfqpoint{3.724542in}{2.000139in}}%
\pgfpathlineto{\pgfqpoint{3.573756in}{2.000435in}}%
\pgfpathlineto{\pgfqpoint{3.422971in}{2.000922in}}%
\pgfpathlineto{\pgfqpoint{3.272185in}{2.109688in}}%
\pgfpathlineto{\pgfqpoint{3.121399in}{2.352938in}}%
\pgfpathlineto{\pgfqpoint{2.970613in}{2.019525in}}%
\pgfpathlineto{\pgfqpoint{2.819827in}{2.017344in}}%
\pgfpathlineto{\pgfqpoint{2.669041in}{1.953761in}}%
\pgfpathlineto{\pgfqpoint{2.518255in}{1.213275in}}%
\pgfpathlineto{\pgfqpoint{2.367469in}{1.267997in}}%
\pgfpathlineto{\pgfqpoint{2.216683in}{1.445327in}}%
\pgfpathlineto{\pgfqpoint{2.065897in}{1.262724in}}%
\pgfpathlineto{\pgfqpoint{1.915111in}{0.970207in}}%
\pgfpathlineto{\pgfqpoint{1.764325in}{1.168273in}}%
\pgfpathlineto{\pgfqpoint{1.613539in}{0.802409in}}%
\pgfpathlineto{\pgfqpoint{1.462753in}{0.790125in}}%
\pgfpathlineto{\pgfqpoint{1.311967in}{0.794566in}}%
\pgfpathlineto{\pgfqpoint{1.161181in}{0.797694in}}%
\pgfpathlineto{\pgfqpoint{1.010395in}{0.757084in}}%
\pgfpathlineto{\pgfqpoint{0.859609in}{0.706590in}}%
\pgfpathlineto{\pgfqpoint{0.708823in}{0.722498in}}%
\pgfpathclose%
\pgfusepath{stroke,fill}%
\end{pgfscope}%
\begin{pgfscope}%
\pgfpathrectangle{\pgfqpoint{0.557986in}{0.438889in}}{\pgfqpoint{3.318403in}{2.263889in}}%
\pgfusepath{clip}%
\pgfsetbuttcap%
\pgfsetroundjoin%
\definecolor{currentfill}{rgb}{0.839216,0.152941,0.156863}%
\pgfsetfillcolor{currentfill}%
\pgfsetlinewidth{1.003750pt}%
\definecolor{currentstroke}{rgb}{0.839216,0.152941,0.156863}%
\pgfsetstrokecolor{currentstroke}%
\pgfsetdash{}{0pt}%
\pgfsys@defobject{currentmarker}{\pgfqpoint{-0.009821in}{-0.009821in}}{\pgfqpoint{0.009821in}{0.009821in}}{%
\pgfpathmoveto{\pgfqpoint{0.000000in}{-0.009821in}}%
\pgfpathcurveto{\pgfqpoint{0.002605in}{-0.009821in}}{\pgfqpoint{0.005103in}{-0.008786in}}{\pgfqpoint{0.006944in}{-0.006944in}}%
\pgfpathcurveto{\pgfqpoint{0.008786in}{-0.005103in}}{\pgfqpoint{0.009821in}{-0.002605in}}{\pgfqpoint{0.009821in}{0.000000in}}%
\pgfpathcurveto{\pgfqpoint{0.009821in}{0.002605in}}{\pgfqpoint{0.008786in}{0.005103in}}{\pgfqpoint{0.006944in}{0.006944in}}%
\pgfpathcurveto{\pgfqpoint{0.005103in}{0.008786in}}{\pgfqpoint{0.002605in}{0.009821in}}{\pgfqpoint{0.000000in}{0.009821in}}%
\pgfpathcurveto{\pgfqpoint{-0.002605in}{0.009821in}}{\pgfqpoint{-0.005103in}{0.008786in}}{\pgfqpoint{-0.006944in}{0.006944in}}%
\pgfpathcurveto{\pgfqpoint{-0.008786in}{0.005103in}}{\pgfqpoint{-0.009821in}{0.002605in}}{\pgfqpoint{-0.009821in}{0.000000in}}%
\pgfpathcurveto{\pgfqpoint{-0.009821in}{-0.002605in}}{\pgfqpoint{-0.008786in}{-0.005103in}}{\pgfqpoint{-0.006944in}{-0.006944in}}%
\pgfpathcurveto{\pgfqpoint{-0.005103in}{-0.008786in}}{\pgfqpoint{-0.002605in}{-0.009821in}}{\pgfqpoint{0.000000in}{-0.009821in}}%
\pgfpathclose%
\pgfusepath{stroke,fill}%
}%
\begin{pgfscope}%
\pgfsys@transformshift{0.708823in}{0.666813in}%
\pgfsys@useobject{currentmarker}{}%
\end{pgfscope}%
\begin{pgfscope}%
\pgfsys@transformshift{0.859609in}{0.674807in}%
\pgfsys@useobject{currentmarker}{}%
\end{pgfscope}%
\begin{pgfscope}%
\pgfsys@transformshift{1.010395in}{0.685725in}%
\pgfsys@useobject{currentmarker}{}%
\end{pgfscope}%
\begin{pgfscope}%
\pgfsys@transformshift{1.161181in}{0.688401in}%
\pgfsys@useobject{currentmarker}{}%
\end{pgfscope}%
\begin{pgfscope}%
\pgfsys@transformshift{1.311967in}{0.700754in}%
\pgfsys@useobject{currentmarker}{}%
\end{pgfscope}%
\begin{pgfscope}%
\pgfsys@transformshift{1.462753in}{0.708376in}%
\pgfsys@useobject{currentmarker}{}%
\end{pgfscope}%
\begin{pgfscope}%
\pgfsys@transformshift{1.613539in}{0.704879in}%
\pgfsys@useobject{currentmarker}{}%
\end{pgfscope}%
\begin{pgfscope}%
\pgfsys@transformshift{1.764325in}{0.748116in}%
\pgfsys@useobject{currentmarker}{}%
\end{pgfscope}%
\begin{pgfscope}%
\pgfsys@transformshift{1.915111in}{0.837081in}%
\pgfsys@useobject{currentmarker}{}%
\end{pgfscope}%
\begin{pgfscope}%
\pgfsys@transformshift{2.065897in}{0.831179in}%
\pgfsys@useobject{currentmarker}{}%
\end{pgfscope}%
\begin{pgfscope}%
\pgfsys@transformshift{2.216683in}{0.819256in}%
\pgfsys@useobject{currentmarker}{}%
\end{pgfscope}%
\begin{pgfscope}%
\pgfsys@transformshift{2.367469in}{0.771371in}%
\pgfsys@useobject{currentmarker}{}%
\end{pgfscope}%
\begin{pgfscope}%
\pgfsys@transformshift{2.518255in}{0.747130in}%
\pgfsys@useobject{currentmarker}{}%
\end{pgfscope}%
\begin{pgfscope}%
\pgfsys@transformshift{2.669041in}{0.775828in}%
\pgfsys@useobject{currentmarker}{}%
\end{pgfscope}%
\begin{pgfscope}%
\pgfsys@transformshift{2.819827in}{0.820274in}%
\pgfsys@useobject{currentmarker}{}%
\end{pgfscope}%
\begin{pgfscope}%
\pgfsys@transformshift{2.970613in}{0.805079in}%
\pgfsys@useobject{currentmarker}{}%
\end{pgfscope}%
\begin{pgfscope}%
\pgfsys@transformshift{3.121399in}{0.804892in}%
\pgfsys@useobject{currentmarker}{}%
\end{pgfscope}%
\begin{pgfscope}%
\pgfsys@transformshift{3.272185in}{0.798488in}%
\pgfsys@useobject{currentmarker}{}%
\end{pgfscope}%
\begin{pgfscope}%
\pgfsys@transformshift{3.422971in}{0.787199in}%
\pgfsys@useobject{currentmarker}{}%
\end{pgfscope}%
\begin{pgfscope}%
\pgfsys@transformshift{3.573756in}{0.794933in}%
\pgfsys@useobject{currentmarker}{}%
\end{pgfscope}%
\begin{pgfscope}%
\pgfsys@transformshift{3.724542in}{0.795994in}%
\pgfsys@useobject{currentmarker}{}%
\end{pgfscope}%
\end{pgfscope}%
\begin{pgfscope}%
\pgfpathrectangle{\pgfqpoint{0.557986in}{0.438889in}}{\pgfqpoint{3.318403in}{2.263889in}}%
\pgfusepath{clip}%
\pgfsetbuttcap%
\pgfsetroundjoin%
\definecolor{currentfill}{rgb}{0.839216,0.152941,0.156863}%
\pgfsetfillcolor{currentfill}%
\pgfsetfillopacity{0.300000}%
\pgfsetlinewidth{1.003750pt}%
\definecolor{currentstroke}{rgb}{0.839216,0.152941,0.156863}%
\pgfsetstrokecolor{currentstroke}%
\pgfsetstrokeopacity{0.300000}%
\pgfsetdash{}{0pt}%
\pgfpathmoveto{\pgfqpoint{0.708823in}{0.676935in}}%
\pgfpathlineto{\pgfqpoint{0.708823in}{0.656690in}}%
\pgfpathlineto{\pgfqpoint{0.859609in}{0.656720in}}%
\pgfpathlineto{\pgfqpoint{1.010395in}{0.646293in}}%
\pgfpathlineto{\pgfqpoint{1.161181in}{0.653408in}}%
\pgfpathlineto{\pgfqpoint{1.311967in}{0.657112in}}%
\pgfpathlineto{\pgfqpoint{1.462753in}{0.662949in}}%
\pgfpathlineto{\pgfqpoint{1.613539in}{0.657412in}}%
\pgfpathlineto{\pgfqpoint{1.764325in}{0.648065in}}%
\pgfpathlineto{\pgfqpoint{1.915111in}{0.783094in}}%
\pgfpathlineto{\pgfqpoint{2.065897in}{0.733145in}}%
\pgfpathlineto{\pgfqpoint{2.216683in}{0.687744in}}%
\pgfpathlineto{\pgfqpoint{2.367469in}{0.679564in}}%
\pgfpathlineto{\pgfqpoint{2.518255in}{0.707235in}}%
\pgfpathlineto{\pgfqpoint{2.669041in}{0.717788in}}%
\pgfpathlineto{\pgfqpoint{2.819827in}{0.797403in}}%
\pgfpathlineto{\pgfqpoint{2.970613in}{0.759904in}}%
\pgfpathlineto{\pgfqpoint{3.121399in}{0.783153in}}%
\pgfpathlineto{\pgfqpoint{3.272185in}{0.777641in}}%
\pgfpathlineto{\pgfqpoint{3.422971in}{0.782524in}}%
\pgfpathlineto{\pgfqpoint{3.573756in}{0.787959in}}%
\pgfpathlineto{\pgfqpoint{3.724542in}{0.790182in}}%
\pgfpathlineto{\pgfqpoint{3.724542in}{0.801806in}}%
\pgfpathlineto{\pgfqpoint{3.724542in}{0.801806in}}%
\pgfpathlineto{\pgfqpoint{3.573756in}{0.801907in}}%
\pgfpathlineto{\pgfqpoint{3.422971in}{0.791874in}}%
\pgfpathlineto{\pgfqpoint{3.272185in}{0.819336in}}%
\pgfpathlineto{\pgfqpoint{3.121399in}{0.826630in}}%
\pgfpathlineto{\pgfqpoint{2.970613in}{0.850254in}}%
\pgfpathlineto{\pgfqpoint{2.819827in}{0.843144in}}%
\pgfpathlineto{\pgfqpoint{2.669041in}{0.833867in}}%
\pgfpathlineto{\pgfqpoint{2.518255in}{0.787025in}}%
\pgfpathlineto{\pgfqpoint{2.367469in}{0.863178in}}%
\pgfpathlineto{\pgfqpoint{2.216683in}{0.950768in}}%
\pgfpathlineto{\pgfqpoint{2.065897in}{0.929213in}}%
\pgfpathlineto{\pgfqpoint{1.915111in}{0.891069in}}%
\pgfpathlineto{\pgfqpoint{1.764325in}{0.848167in}}%
\pgfpathlineto{\pgfqpoint{1.613539in}{0.752346in}}%
\pgfpathlineto{\pgfqpoint{1.462753in}{0.753802in}}%
\pgfpathlineto{\pgfqpoint{1.311967in}{0.744396in}}%
\pgfpathlineto{\pgfqpoint{1.161181in}{0.723394in}}%
\pgfpathlineto{\pgfqpoint{1.010395in}{0.725157in}}%
\pgfpathlineto{\pgfqpoint{0.859609in}{0.692895in}}%
\pgfpathlineto{\pgfqpoint{0.708823in}{0.676935in}}%
\pgfpathclose%
\pgfusepath{stroke,fill}%
\end{pgfscope}%
\begin{pgfscope}%
\pgfpathrectangle{\pgfqpoint{0.557986in}{0.438889in}}{\pgfqpoint{3.318403in}{2.263889in}}%
\pgfusepath{clip}%
\pgfsetbuttcap%
\pgfsetroundjoin%
\definecolor{currentfill}{rgb}{0.580392,0.403922,0.741176}%
\pgfsetfillcolor{currentfill}%
\pgfsetlinewidth{1.003750pt}%
\definecolor{currentstroke}{rgb}{0.580392,0.403922,0.741176}%
\pgfsetstrokecolor{currentstroke}%
\pgfsetdash{}{0pt}%
\pgfsys@defobject{currentmarker}{\pgfqpoint{-0.009821in}{-0.009821in}}{\pgfqpoint{0.009821in}{0.009821in}}{%
\pgfpathmoveto{\pgfqpoint{0.000000in}{-0.009821in}}%
\pgfpathcurveto{\pgfqpoint{0.002605in}{-0.009821in}}{\pgfqpoint{0.005103in}{-0.008786in}}{\pgfqpoint{0.006944in}{-0.006944in}}%
\pgfpathcurveto{\pgfqpoint{0.008786in}{-0.005103in}}{\pgfqpoint{0.009821in}{-0.002605in}}{\pgfqpoint{0.009821in}{0.000000in}}%
\pgfpathcurveto{\pgfqpoint{0.009821in}{0.002605in}}{\pgfqpoint{0.008786in}{0.005103in}}{\pgfqpoint{0.006944in}{0.006944in}}%
\pgfpathcurveto{\pgfqpoint{0.005103in}{0.008786in}}{\pgfqpoint{0.002605in}{0.009821in}}{\pgfqpoint{0.000000in}{0.009821in}}%
\pgfpathcurveto{\pgfqpoint{-0.002605in}{0.009821in}}{\pgfqpoint{-0.005103in}{0.008786in}}{\pgfqpoint{-0.006944in}{0.006944in}}%
\pgfpathcurveto{\pgfqpoint{-0.008786in}{0.005103in}}{\pgfqpoint{-0.009821in}{0.002605in}}{\pgfqpoint{-0.009821in}{0.000000in}}%
\pgfpathcurveto{\pgfqpoint{-0.009821in}{-0.002605in}}{\pgfqpoint{-0.008786in}{-0.005103in}}{\pgfqpoint{-0.006944in}{-0.006944in}}%
\pgfpathcurveto{\pgfqpoint{-0.005103in}{-0.008786in}}{\pgfqpoint{-0.002605in}{-0.009821in}}{\pgfqpoint{0.000000in}{-0.009821in}}%
\pgfpathclose%
\pgfusepath{stroke,fill}%
}%
\begin{pgfscope}%
\pgfsys@transformshift{0.708823in}{0.671586in}%
\pgfsys@useobject{currentmarker}{}%
\end{pgfscope}%
\begin{pgfscope}%
\pgfsys@transformshift{0.859609in}{0.678810in}%
\pgfsys@useobject{currentmarker}{}%
\end{pgfscope}%
\begin{pgfscope}%
\pgfsys@transformshift{1.010395in}{0.676971in}%
\pgfsys@useobject{currentmarker}{}%
\end{pgfscope}%
\begin{pgfscope}%
\pgfsys@transformshift{1.161181in}{0.665443in}%
\pgfsys@useobject{currentmarker}{}%
\end{pgfscope}%
\begin{pgfscope}%
\pgfsys@transformshift{1.311967in}{0.716756in}%
\pgfsys@useobject{currentmarker}{}%
\end{pgfscope}%
\begin{pgfscope}%
\pgfsys@transformshift{1.462753in}{0.711667in}%
\pgfsys@useobject{currentmarker}{}%
\end{pgfscope}%
\begin{pgfscope}%
\pgfsys@transformshift{1.613539in}{0.707580in}%
\pgfsys@useobject{currentmarker}{}%
\end{pgfscope}%
\begin{pgfscope}%
\pgfsys@transformshift{1.764325in}{0.747807in}%
\pgfsys@useobject{currentmarker}{}%
\end{pgfscope}%
\begin{pgfscope}%
\pgfsys@transformshift{1.915111in}{0.723403in}%
\pgfsys@useobject{currentmarker}{}%
\end{pgfscope}%
\begin{pgfscope}%
\pgfsys@transformshift{2.065897in}{0.703011in}%
\pgfsys@useobject{currentmarker}{}%
\end{pgfscope}%
\begin{pgfscope}%
\pgfsys@transformshift{2.216683in}{0.713746in}%
\pgfsys@useobject{currentmarker}{}%
\end{pgfscope}%
\begin{pgfscope}%
\pgfsys@transformshift{2.367469in}{0.702142in}%
\pgfsys@useobject{currentmarker}{}%
\end{pgfscope}%
\begin{pgfscope}%
\pgfsys@transformshift{2.518255in}{0.708099in}%
\pgfsys@useobject{currentmarker}{}%
\end{pgfscope}%
\begin{pgfscope}%
\pgfsys@transformshift{2.669041in}{0.714705in}%
\pgfsys@useobject{currentmarker}{}%
\end{pgfscope}%
\begin{pgfscope}%
\pgfsys@transformshift{2.819827in}{0.707268in}%
\pgfsys@useobject{currentmarker}{}%
\end{pgfscope}%
\begin{pgfscope}%
\pgfsys@transformshift{2.970613in}{0.707771in}%
\pgfsys@useobject{currentmarker}{}%
\end{pgfscope}%
\begin{pgfscope}%
\pgfsys@transformshift{3.121399in}{0.709327in}%
\pgfsys@useobject{currentmarker}{}%
\end{pgfscope}%
\begin{pgfscope}%
\pgfsys@transformshift{3.272185in}{0.709310in}%
\pgfsys@useobject{currentmarker}{}%
\end{pgfscope}%
\begin{pgfscope}%
\pgfsys@transformshift{3.422971in}{0.709991in}%
\pgfsys@useobject{currentmarker}{}%
\end{pgfscope}%
\begin{pgfscope}%
\pgfsys@transformshift{3.573756in}{0.709863in}%
\pgfsys@useobject{currentmarker}{}%
\end{pgfscope}%
\begin{pgfscope}%
\pgfsys@transformshift{3.724542in}{0.710772in}%
\pgfsys@useobject{currentmarker}{}%
\end{pgfscope}%
\end{pgfscope}%
\begin{pgfscope}%
\pgfpathrectangle{\pgfqpoint{0.557986in}{0.438889in}}{\pgfqpoint{3.318403in}{2.263889in}}%
\pgfusepath{clip}%
\pgfsetbuttcap%
\pgfsetroundjoin%
\definecolor{currentfill}{rgb}{0.580392,0.403922,0.741176}%
\pgfsetfillcolor{currentfill}%
\pgfsetfillopacity{0.300000}%
\pgfsetlinewidth{1.003750pt}%
\definecolor{currentstroke}{rgb}{0.580392,0.403922,0.741176}%
\pgfsetstrokecolor{currentstroke}%
\pgfsetstrokeopacity{0.300000}%
\pgfsetdash{}{0pt}%
\pgfpathmoveto{\pgfqpoint{0.708823in}{0.687294in}}%
\pgfpathlineto{\pgfqpoint{0.708823in}{0.655879in}}%
\pgfpathlineto{\pgfqpoint{0.859609in}{0.648051in}}%
\pgfpathlineto{\pgfqpoint{1.010395in}{0.655895in}}%
\pgfpathlineto{\pgfqpoint{1.161181in}{0.656645in}}%
\pgfpathlineto{\pgfqpoint{1.311967in}{0.660543in}}%
\pgfpathlineto{\pgfqpoint{1.462753in}{0.647994in}}%
\pgfpathlineto{\pgfqpoint{1.613539in}{0.666743in}}%
\pgfpathlineto{\pgfqpoint{1.764325in}{0.677808in}}%
\pgfpathlineto{\pgfqpoint{1.915111in}{0.658999in}}%
\pgfpathlineto{\pgfqpoint{2.065897in}{0.689818in}}%
\pgfpathlineto{\pgfqpoint{2.216683in}{0.699137in}}%
\pgfpathlineto{\pgfqpoint{2.367469in}{0.682851in}}%
\pgfpathlineto{\pgfqpoint{2.518255in}{0.697725in}}%
\pgfpathlineto{\pgfqpoint{2.669041in}{0.705201in}}%
\pgfpathlineto{\pgfqpoint{2.819827in}{0.702000in}}%
\pgfpathlineto{\pgfqpoint{2.970613in}{0.701992in}}%
\pgfpathlineto{\pgfqpoint{3.121399in}{0.707062in}}%
\pgfpathlineto{\pgfqpoint{3.272185in}{0.706681in}}%
\pgfpathlineto{\pgfqpoint{3.422971in}{0.708752in}}%
\pgfpathlineto{\pgfqpoint{3.573756in}{0.708809in}}%
\pgfpathlineto{\pgfqpoint{3.724542in}{0.710089in}}%
\pgfpathlineto{\pgfqpoint{3.724542in}{0.711455in}}%
\pgfpathlineto{\pgfqpoint{3.724542in}{0.711455in}}%
\pgfpathlineto{\pgfqpoint{3.573756in}{0.710918in}}%
\pgfpathlineto{\pgfqpoint{3.422971in}{0.711230in}}%
\pgfpathlineto{\pgfqpoint{3.272185in}{0.711940in}}%
\pgfpathlineto{\pgfqpoint{3.121399in}{0.711593in}}%
\pgfpathlineto{\pgfqpoint{2.970613in}{0.713550in}}%
\pgfpathlineto{\pgfqpoint{2.819827in}{0.712536in}}%
\pgfpathlineto{\pgfqpoint{2.669041in}{0.724210in}}%
\pgfpathlineto{\pgfqpoint{2.518255in}{0.718473in}}%
\pgfpathlineto{\pgfqpoint{2.367469in}{0.721432in}}%
\pgfpathlineto{\pgfqpoint{2.216683in}{0.728356in}}%
\pgfpathlineto{\pgfqpoint{2.065897in}{0.716204in}}%
\pgfpathlineto{\pgfqpoint{1.915111in}{0.787808in}}%
\pgfpathlineto{\pgfqpoint{1.764325in}{0.817805in}}%
\pgfpathlineto{\pgfqpoint{1.613539in}{0.748418in}}%
\pgfpathlineto{\pgfqpoint{1.462753in}{0.775339in}}%
\pgfpathlineto{\pgfqpoint{1.311967in}{0.772968in}}%
\pgfpathlineto{\pgfqpoint{1.161181in}{0.674241in}}%
\pgfpathlineto{\pgfqpoint{1.010395in}{0.698046in}}%
\pgfpathlineto{\pgfqpoint{0.859609in}{0.709569in}}%
\pgfpathlineto{\pgfqpoint{0.708823in}{0.687294in}}%
\pgfpathclose%
\pgfusepath{stroke,fill}%
\end{pgfscope}%
\begin{pgfscope}%
\pgfsetbuttcap%
\pgfsetroundjoin%
\definecolor{currentfill}{rgb}{0.000000,0.000000,0.000000}%
\pgfsetfillcolor{currentfill}%
\pgfsetlinewidth{0.803000pt}%
\definecolor{currentstroke}{rgb}{0.000000,0.000000,0.000000}%
\pgfsetstrokecolor{currentstroke}%
\pgfsetdash{}{0pt}%
\pgfsys@defobject{currentmarker}{\pgfqpoint{0.000000in}{-0.048611in}}{\pgfqpoint{0.000000in}{0.000000in}}{%
\pgfpathmoveto{\pgfqpoint{0.000000in}{0.000000in}}%
\pgfpathlineto{\pgfqpoint{0.000000in}{-0.048611in}}%
\pgfusepath{stroke,fill}%
}%
\begin{pgfscope}%
\pgfsys@transformshift{1.071951in}{0.438889in}%
\pgfsys@useobject{currentmarker}{}%
\end{pgfscope}%
\end{pgfscope}%
\begin{pgfscope}%
\definecolor{textcolor}{rgb}{0.000000,0.000000,0.000000}%
\pgfsetstrokecolor{textcolor}%
\pgfsetfillcolor{textcolor}%
\pgftext[x=1.071951in,y=0.341667in,,top]{\color{textcolor}\sffamily\fontsize{10.000000}{12.000000}\selectfont \(\displaystyle {10^{4}}\)}%
\end{pgfscope}%
\begin{pgfscope}%
\pgfsetbuttcap%
\pgfsetroundjoin%
\definecolor{currentfill}{rgb}{0.000000,0.000000,0.000000}%
\pgfsetfillcolor{currentfill}%
\pgfsetlinewidth{0.803000pt}%
\definecolor{currentstroke}{rgb}{0.000000,0.000000,0.000000}%
\pgfsetstrokecolor{currentstroke}%
\pgfsetdash{}{0pt}%
\pgfsys@defobject{currentmarker}{\pgfqpoint{0.000000in}{-0.048611in}}{\pgfqpoint{0.000000in}{0.000000in}}{%
\pgfpathmoveto{\pgfqpoint{0.000000in}{0.000000in}}%
\pgfpathlineto{\pgfqpoint{0.000000in}{-0.048611in}}%
\pgfusepath{stroke,fill}%
}%
\begin{pgfscope}%
\pgfsys@transformshift{1.675095in}{0.438889in}%
\pgfsys@useobject{currentmarker}{}%
\end{pgfscope}%
\end{pgfscope}%
\begin{pgfscope}%
\definecolor{textcolor}{rgb}{0.000000,0.000000,0.000000}%
\pgfsetstrokecolor{textcolor}%
\pgfsetfillcolor{textcolor}%
\pgftext[x=1.675095in,y=0.341667in,,top]{\color{textcolor}\sffamily\fontsize{10.000000}{12.000000}\selectfont \(\displaystyle {10^{5}}\)}%
\end{pgfscope}%
\begin{pgfscope}%
\pgfsetbuttcap%
\pgfsetroundjoin%
\definecolor{currentfill}{rgb}{0.000000,0.000000,0.000000}%
\pgfsetfillcolor{currentfill}%
\pgfsetlinewidth{0.803000pt}%
\definecolor{currentstroke}{rgb}{0.000000,0.000000,0.000000}%
\pgfsetstrokecolor{currentstroke}%
\pgfsetdash{}{0pt}%
\pgfsys@defobject{currentmarker}{\pgfqpoint{0.000000in}{-0.048611in}}{\pgfqpoint{0.000000in}{0.000000in}}{%
\pgfpathmoveto{\pgfqpoint{0.000000in}{0.000000in}}%
\pgfpathlineto{\pgfqpoint{0.000000in}{-0.048611in}}%
\pgfusepath{stroke,fill}%
}%
\begin{pgfscope}%
\pgfsys@transformshift{2.278239in}{0.438889in}%
\pgfsys@useobject{currentmarker}{}%
\end{pgfscope}%
\end{pgfscope}%
\begin{pgfscope}%
\definecolor{textcolor}{rgb}{0.000000,0.000000,0.000000}%
\pgfsetstrokecolor{textcolor}%
\pgfsetfillcolor{textcolor}%
\pgftext[x=2.278239in,y=0.341667in,,top]{\color{textcolor}\sffamily\fontsize{10.000000}{12.000000}\selectfont \(\displaystyle {10^{6}}\)}%
\end{pgfscope}%
\begin{pgfscope}%
\pgfsetbuttcap%
\pgfsetroundjoin%
\definecolor{currentfill}{rgb}{0.000000,0.000000,0.000000}%
\pgfsetfillcolor{currentfill}%
\pgfsetlinewidth{0.803000pt}%
\definecolor{currentstroke}{rgb}{0.000000,0.000000,0.000000}%
\pgfsetstrokecolor{currentstroke}%
\pgfsetdash{}{0pt}%
\pgfsys@defobject{currentmarker}{\pgfqpoint{0.000000in}{-0.048611in}}{\pgfqpoint{0.000000in}{0.000000in}}{%
\pgfpathmoveto{\pgfqpoint{0.000000in}{0.000000in}}%
\pgfpathlineto{\pgfqpoint{0.000000in}{-0.048611in}}%
\pgfusepath{stroke,fill}%
}%
\begin{pgfscope}%
\pgfsys@transformshift{2.881383in}{0.438889in}%
\pgfsys@useobject{currentmarker}{}%
\end{pgfscope}%
\end{pgfscope}%
\begin{pgfscope}%
\definecolor{textcolor}{rgb}{0.000000,0.000000,0.000000}%
\pgfsetstrokecolor{textcolor}%
\pgfsetfillcolor{textcolor}%
\pgftext[x=2.881383in,y=0.341667in,,top]{\color{textcolor}\sffamily\fontsize{10.000000}{12.000000}\selectfont \(\displaystyle {10^{7}}\)}%
\end{pgfscope}%
\begin{pgfscope}%
\pgfsetbuttcap%
\pgfsetroundjoin%
\definecolor{currentfill}{rgb}{0.000000,0.000000,0.000000}%
\pgfsetfillcolor{currentfill}%
\pgfsetlinewidth{0.803000pt}%
\definecolor{currentstroke}{rgb}{0.000000,0.000000,0.000000}%
\pgfsetstrokecolor{currentstroke}%
\pgfsetdash{}{0pt}%
\pgfsys@defobject{currentmarker}{\pgfqpoint{0.000000in}{-0.048611in}}{\pgfqpoint{0.000000in}{0.000000in}}{%
\pgfpathmoveto{\pgfqpoint{0.000000in}{0.000000in}}%
\pgfpathlineto{\pgfqpoint{0.000000in}{-0.048611in}}%
\pgfusepath{stroke,fill}%
}%
\begin{pgfscope}%
\pgfsys@transformshift{3.484527in}{0.438889in}%
\pgfsys@useobject{currentmarker}{}%
\end{pgfscope}%
\end{pgfscope}%
\begin{pgfscope}%
\definecolor{textcolor}{rgb}{0.000000,0.000000,0.000000}%
\pgfsetstrokecolor{textcolor}%
\pgfsetfillcolor{textcolor}%
\pgftext[x=3.484527in,y=0.341667in,,top]{\color{textcolor}\sffamily\fontsize{10.000000}{12.000000}\selectfont \(\displaystyle {10^{8}}\)}%
\end{pgfscope}%
\begin{pgfscope}%
\pgfsetbuttcap%
\pgfsetroundjoin%
\definecolor{currentfill}{rgb}{0.000000,0.000000,0.000000}%
\pgfsetfillcolor{currentfill}%
\pgfsetlinewidth{0.602250pt}%
\definecolor{currentstroke}{rgb}{0.000000,0.000000,0.000000}%
\pgfsetstrokecolor{currentstroke}%
\pgfsetdash{}{0pt}%
\pgfsys@defobject{currentmarker}{\pgfqpoint{0.000000in}{-0.027778in}}{\pgfqpoint{0.000000in}{0.000000in}}{%
\pgfpathmoveto{\pgfqpoint{0.000000in}{0.000000in}}%
\pgfpathlineto{\pgfqpoint{0.000000in}{-0.027778in}}%
\pgfusepath{stroke,fill}%
}%
\begin{pgfscope}%
\pgfsys@transformshift{0.650372in}{0.438889in}%
\pgfsys@useobject{currentmarker}{}%
\end{pgfscope}%
\end{pgfscope}%
\begin{pgfscope}%
\pgfsetbuttcap%
\pgfsetroundjoin%
\definecolor{currentfill}{rgb}{0.000000,0.000000,0.000000}%
\pgfsetfillcolor{currentfill}%
\pgfsetlinewidth{0.602250pt}%
\definecolor{currentstroke}{rgb}{0.000000,0.000000,0.000000}%
\pgfsetstrokecolor{currentstroke}%
\pgfsetdash{}{0pt}%
\pgfsys@defobject{currentmarker}{\pgfqpoint{0.000000in}{-0.027778in}}{\pgfqpoint{0.000000in}{0.000000in}}{%
\pgfpathmoveto{\pgfqpoint{0.000000in}{0.000000in}}%
\pgfpathlineto{\pgfqpoint{0.000000in}{-0.027778in}}%
\pgfusepath{stroke,fill}%
}%
\begin{pgfscope}%
\pgfsys@transformshift{0.756580in}{0.438889in}%
\pgfsys@useobject{currentmarker}{}%
\end{pgfscope}%
\end{pgfscope}%
\begin{pgfscope}%
\pgfsetbuttcap%
\pgfsetroundjoin%
\definecolor{currentfill}{rgb}{0.000000,0.000000,0.000000}%
\pgfsetfillcolor{currentfill}%
\pgfsetlinewidth{0.602250pt}%
\definecolor{currentstroke}{rgb}{0.000000,0.000000,0.000000}%
\pgfsetstrokecolor{currentstroke}%
\pgfsetdash{}{0pt}%
\pgfsys@defobject{currentmarker}{\pgfqpoint{0.000000in}{-0.027778in}}{\pgfqpoint{0.000000in}{0.000000in}}{%
\pgfpathmoveto{\pgfqpoint{0.000000in}{0.000000in}}%
\pgfpathlineto{\pgfqpoint{0.000000in}{-0.027778in}}%
\pgfusepath{stroke,fill}%
}%
\begin{pgfscope}%
\pgfsys@transformshift{0.831936in}{0.438889in}%
\pgfsys@useobject{currentmarker}{}%
\end{pgfscope}%
\end{pgfscope}%
\begin{pgfscope}%
\pgfsetbuttcap%
\pgfsetroundjoin%
\definecolor{currentfill}{rgb}{0.000000,0.000000,0.000000}%
\pgfsetfillcolor{currentfill}%
\pgfsetlinewidth{0.602250pt}%
\definecolor{currentstroke}{rgb}{0.000000,0.000000,0.000000}%
\pgfsetstrokecolor{currentstroke}%
\pgfsetdash{}{0pt}%
\pgfsys@defobject{currentmarker}{\pgfqpoint{0.000000in}{-0.027778in}}{\pgfqpoint{0.000000in}{0.000000in}}{%
\pgfpathmoveto{\pgfqpoint{0.000000in}{0.000000in}}%
\pgfpathlineto{\pgfqpoint{0.000000in}{-0.027778in}}%
\pgfusepath{stroke,fill}%
}%
\begin{pgfscope}%
\pgfsys@transformshift{0.890387in}{0.438889in}%
\pgfsys@useobject{currentmarker}{}%
\end{pgfscope}%
\end{pgfscope}%
\begin{pgfscope}%
\pgfsetbuttcap%
\pgfsetroundjoin%
\definecolor{currentfill}{rgb}{0.000000,0.000000,0.000000}%
\pgfsetfillcolor{currentfill}%
\pgfsetlinewidth{0.602250pt}%
\definecolor{currentstroke}{rgb}{0.000000,0.000000,0.000000}%
\pgfsetstrokecolor{currentstroke}%
\pgfsetdash{}{0pt}%
\pgfsys@defobject{currentmarker}{\pgfqpoint{0.000000in}{-0.027778in}}{\pgfqpoint{0.000000in}{0.000000in}}{%
\pgfpathmoveto{\pgfqpoint{0.000000in}{0.000000in}}%
\pgfpathlineto{\pgfqpoint{0.000000in}{-0.027778in}}%
\pgfusepath{stroke,fill}%
}%
\begin{pgfscope}%
\pgfsys@transformshift{0.938145in}{0.438889in}%
\pgfsys@useobject{currentmarker}{}%
\end{pgfscope}%
\end{pgfscope}%
\begin{pgfscope}%
\pgfsetbuttcap%
\pgfsetroundjoin%
\definecolor{currentfill}{rgb}{0.000000,0.000000,0.000000}%
\pgfsetfillcolor{currentfill}%
\pgfsetlinewidth{0.602250pt}%
\definecolor{currentstroke}{rgb}{0.000000,0.000000,0.000000}%
\pgfsetstrokecolor{currentstroke}%
\pgfsetdash{}{0pt}%
\pgfsys@defobject{currentmarker}{\pgfqpoint{0.000000in}{-0.027778in}}{\pgfqpoint{0.000000in}{0.000000in}}{%
\pgfpathmoveto{\pgfqpoint{0.000000in}{0.000000in}}%
\pgfpathlineto{\pgfqpoint{0.000000in}{-0.027778in}}%
\pgfusepath{stroke,fill}%
}%
\begin{pgfscope}%
\pgfsys@transformshift{0.978523in}{0.438889in}%
\pgfsys@useobject{currentmarker}{}%
\end{pgfscope}%
\end{pgfscope}%
\begin{pgfscope}%
\pgfsetbuttcap%
\pgfsetroundjoin%
\definecolor{currentfill}{rgb}{0.000000,0.000000,0.000000}%
\pgfsetfillcolor{currentfill}%
\pgfsetlinewidth{0.602250pt}%
\definecolor{currentstroke}{rgb}{0.000000,0.000000,0.000000}%
\pgfsetstrokecolor{currentstroke}%
\pgfsetdash{}{0pt}%
\pgfsys@defobject{currentmarker}{\pgfqpoint{0.000000in}{-0.027778in}}{\pgfqpoint{0.000000in}{0.000000in}}{%
\pgfpathmoveto{\pgfqpoint{0.000000in}{0.000000in}}%
\pgfpathlineto{\pgfqpoint{0.000000in}{-0.027778in}}%
\pgfusepath{stroke,fill}%
}%
\begin{pgfscope}%
\pgfsys@transformshift{1.013501in}{0.438889in}%
\pgfsys@useobject{currentmarker}{}%
\end{pgfscope}%
\end{pgfscope}%
\begin{pgfscope}%
\pgfsetbuttcap%
\pgfsetroundjoin%
\definecolor{currentfill}{rgb}{0.000000,0.000000,0.000000}%
\pgfsetfillcolor{currentfill}%
\pgfsetlinewidth{0.602250pt}%
\definecolor{currentstroke}{rgb}{0.000000,0.000000,0.000000}%
\pgfsetstrokecolor{currentstroke}%
\pgfsetdash{}{0pt}%
\pgfsys@defobject{currentmarker}{\pgfqpoint{0.000000in}{-0.027778in}}{\pgfqpoint{0.000000in}{0.000000in}}{%
\pgfpathmoveto{\pgfqpoint{0.000000in}{0.000000in}}%
\pgfpathlineto{\pgfqpoint{0.000000in}{-0.027778in}}%
\pgfusepath{stroke,fill}%
}%
\begin{pgfscope}%
\pgfsys@transformshift{1.044353in}{0.438889in}%
\pgfsys@useobject{currentmarker}{}%
\end{pgfscope}%
\end{pgfscope}%
\begin{pgfscope}%
\pgfsetbuttcap%
\pgfsetroundjoin%
\definecolor{currentfill}{rgb}{0.000000,0.000000,0.000000}%
\pgfsetfillcolor{currentfill}%
\pgfsetlinewidth{0.602250pt}%
\definecolor{currentstroke}{rgb}{0.000000,0.000000,0.000000}%
\pgfsetstrokecolor{currentstroke}%
\pgfsetdash{}{0pt}%
\pgfsys@defobject{currentmarker}{\pgfqpoint{0.000000in}{-0.027778in}}{\pgfqpoint{0.000000in}{0.000000in}}{%
\pgfpathmoveto{\pgfqpoint{0.000000in}{0.000000in}}%
\pgfpathlineto{\pgfqpoint{0.000000in}{-0.027778in}}%
\pgfusepath{stroke,fill}%
}%
\begin{pgfscope}%
\pgfsys@transformshift{1.253516in}{0.438889in}%
\pgfsys@useobject{currentmarker}{}%
\end{pgfscope}%
\end{pgfscope}%
\begin{pgfscope}%
\pgfsetbuttcap%
\pgfsetroundjoin%
\definecolor{currentfill}{rgb}{0.000000,0.000000,0.000000}%
\pgfsetfillcolor{currentfill}%
\pgfsetlinewidth{0.602250pt}%
\definecolor{currentstroke}{rgb}{0.000000,0.000000,0.000000}%
\pgfsetstrokecolor{currentstroke}%
\pgfsetdash{}{0pt}%
\pgfsys@defobject{currentmarker}{\pgfqpoint{0.000000in}{-0.027778in}}{\pgfqpoint{0.000000in}{0.000000in}}{%
\pgfpathmoveto{\pgfqpoint{0.000000in}{0.000000in}}%
\pgfpathlineto{\pgfqpoint{0.000000in}{-0.027778in}}%
\pgfusepath{stroke,fill}%
}%
\begin{pgfscope}%
\pgfsys@transformshift{1.359724in}{0.438889in}%
\pgfsys@useobject{currentmarker}{}%
\end{pgfscope}%
\end{pgfscope}%
\begin{pgfscope}%
\pgfsetbuttcap%
\pgfsetroundjoin%
\definecolor{currentfill}{rgb}{0.000000,0.000000,0.000000}%
\pgfsetfillcolor{currentfill}%
\pgfsetlinewidth{0.602250pt}%
\definecolor{currentstroke}{rgb}{0.000000,0.000000,0.000000}%
\pgfsetstrokecolor{currentstroke}%
\pgfsetdash{}{0pt}%
\pgfsys@defobject{currentmarker}{\pgfqpoint{0.000000in}{-0.027778in}}{\pgfqpoint{0.000000in}{0.000000in}}{%
\pgfpathmoveto{\pgfqpoint{0.000000in}{0.000000in}}%
\pgfpathlineto{\pgfqpoint{0.000000in}{-0.027778in}}%
\pgfusepath{stroke,fill}%
}%
\begin{pgfscope}%
\pgfsys@transformshift{1.435080in}{0.438889in}%
\pgfsys@useobject{currentmarker}{}%
\end{pgfscope}%
\end{pgfscope}%
\begin{pgfscope}%
\pgfsetbuttcap%
\pgfsetroundjoin%
\definecolor{currentfill}{rgb}{0.000000,0.000000,0.000000}%
\pgfsetfillcolor{currentfill}%
\pgfsetlinewidth{0.602250pt}%
\definecolor{currentstroke}{rgb}{0.000000,0.000000,0.000000}%
\pgfsetstrokecolor{currentstroke}%
\pgfsetdash{}{0pt}%
\pgfsys@defobject{currentmarker}{\pgfqpoint{0.000000in}{-0.027778in}}{\pgfqpoint{0.000000in}{0.000000in}}{%
\pgfpathmoveto{\pgfqpoint{0.000000in}{0.000000in}}%
\pgfpathlineto{\pgfqpoint{0.000000in}{-0.027778in}}%
\pgfusepath{stroke,fill}%
}%
\begin{pgfscope}%
\pgfsys@transformshift{1.493531in}{0.438889in}%
\pgfsys@useobject{currentmarker}{}%
\end{pgfscope}%
\end{pgfscope}%
\begin{pgfscope}%
\pgfsetbuttcap%
\pgfsetroundjoin%
\definecolor{currentfill}{rgb}{0.000000,0.000000,0.000000}%
\pgfsetfillcolor{currentfill}%
\pgfsetlinewidth{0.602250pt}%
\definecolor{currentstroke}{rgb}{0.000000,0.000000,0.000000}%
\pgfsetstrokecolor{currentstroke}%
\pgfsetdash{}{0pt}%
\pgfsys@defobject{currentmarker}{\pgfqpoint{0.000000in}{-0.027778in}}{\pgfqpoint{0.000000in}{0.000000in}}{%
\pgfpathmoveto{\pgfqpoint{0.000000in}{0.000000in}}%
\pgfpathlineto{\pgfqpoint{0.000000in}{-0.027778in}}%
\pgfusepath{stroke,fill}%
}%
\begin{pgfscope}%
\pgfsys@transformshift{1.541289in}{0.438889in}%
\pgfsys@useobject{currentmarker}{}%
\end{pgfscope}%
\end{pgfscope}%
\begin{pgfscope}%
\pgfsetbuttcap%
\pgfsetroundjoin%
\definecolor{currentfill}{rgb}{0.000000,0.000000,0.000000}%
\pgfsetfillcolor{currentfill}%
\pgfsetlinewidth{0.602250pt}%
\definecolor{currentstroke}{rgb}{0.000000,0.000000,0.000000}%
\pgfsetstrokecolor{currentstroke}%
\pgfsetdash{}{0pt}%
\pgfsys@defobject{currentmarker}{\pgfqpoint{0.000000in}{-0.027778in}}{\pgfqpoint{0.000000in}{0.000000in}}{%
\pgfpathmoveto{\pgfqpoint{0.000000in}{0.000000in}}%
\pgfpathlineto{\pgfqpoint{0.000000in}{-0.027778in}}%
\pgfusepath{stroke,fill}%
}%
\begin{pgfscope}%
\pgfsys@transformshift{1.581667in}{0.438889in}%
\pgfsys@useobject{currentmarker}{}%
\end{pgfscope}%
\end{pgfscope}%
\begin{pgfscope}%
\pgfsetbuttcap%
\pgfsetroundjoin%
\definecolor{currentfill}{rgb}{0.000000,0.000000,0.000000}%
\pgfsetfillcolor{currentfill}%
\pgfsetlinewidth{0.602250pt}%
\definecolor{currentstroke}{rgb}{0.000000,0.000000,0.000000}%
\pgfsetstrokecolor{currentstroke}%
\pgfsetdash{}{0pt}%
\pgfsys@defobject{currentmarker}{\pgfqpoint{0.000000in}{-0.027778in}}{\pgfqpoint{0.000000in}{0.000000in}}{%
\pgfpathmoveto{\pgfqpoint{0.000000in}{0.000000in}}%
\pgfpathlineto{\pgfqpoint{0.000000in}{-0.027778in}}%
\pgfusepath{stroke,fill}%
}%
\begin{pgfscope}%
\pgfsys@transformshift{1.616645in}{0.438889in}%
\pgfsys@useobject{currentmarker}{}%
\end{pgfscope}%
\end{pgfscope}%
\begin{pgfscope}%
\pgfsetbuttcap%
\pgfsetroundjoin%
\definecolor{currentfill}{rgb}{0.000000,0.000000,0.000000}%
\pgfsetfillcolor{currentfill}%
\pgfsetlinewidth{0.602250pt}%
\definecolor{currentstroke}{rgb}{0.000000,0.000000,0.000000}%
\pgfsetstrokecolor{currentstroke}%
\pgfsetdash{}{0pt}%
\pgfsys@defobject{currentmarker}{\pgfqpoint{0.000000in}{-0.027778in}}{\pgfqpoint{0.000000in}{0.000000in}}{%
\pgfpathmoveto{\pgfqpoint{0.000000in}{0.000000in}}%
\pgfpathlineto{\pgfqpoint{0.000000in}{-0.027778in}}%
\pgfusepath{stroke,fill}%
}%
\begin{pgfscope}%
\pgfsys@transformshift{1.647497in}{0.438889in}%
\pgfsys@useobject{currentmarker}{}%
\end{pgfscope}%
\end{pgfscope}%
\begin{pgfscope}%
\pgfsetbuttcap%
\pgfsetroundjoin%
\definecolor{currentfill}{rgb}{0.000000,0.000000,0.000000}%
\pgfsetfillcolor{currentfill}%
\pgfsetlinewidth{0.602250pt}%
\definecolor{currentstroke}{rgb}{0.000000,0.000000,0.000000}%
\pgfsetstrokecolor{currentstroke}%
\pgfsetdash{}{0pt}%
\pgfsys@defobject{currentmarker}{\pgfqpoint{0.000000in}{-0.027778in}}{\pgfqpoint{0.000000in}{0.000000in}}{%
\pgfpathmoveto{\pgfqpoint{0.000000in}{0.000000in}}%
\pgfpathlineto{\pgfqpoint{0.000000in}{-0.027778in}}%
\pgfusepath{stroke,fill}%
}%
\begin{pgfscope}%
\pgfsys@transformshift{1.856660in}{0.438889in}%
\pgfsys@useobject{currentmarker}{}%
\end{pgfscope}%
\end{pgfscope}%
\begin{pgfscope}%
\pgfsetbuttcap%
\pgfsetroundjoin%
\definecolor{currentfill}{rgb}{0.000000,0.000000,0.000000}%
\pgfsetfillcolor{currentfill}%
\pgfsetlinewidth{0.602250pt}%
\definecolor{currentstroke}{rgb}{0.000000,0.000000,0.000000}%
\pgfsetstrokecolor{currentstroke}%
\pgfsetdash{}{0pt}%
\pgfsys@defobject{currentmarker}{\pgfqpoint{0.000000in}{-0.027778in}}{\pgfqpoint{0.000000in}{0.000000in}}{%
\pgfpathmoveto{\pgfqpoint{0.000000in}{0.000000in}}%
\pgfpathlineto{\pgfqpoint{0.000000in}{-0.027778in}}%
\pgfusepath{stroke,fill}%
}%
\begin{pgfscope}%
\pgfsys@transformshift{1.962868in}{0.438889in}%
\pgfsys@useobject{currentmarker}{}%
\end{pgfscope}%
\end{pgfscope}%
\begin{pgfscope}%
\pgfsetbuttcap%
\pgfsetroundjoin%
\definecolor{currentfill}{rgb}{0.000000,0.000000,0.000000}%
\pgfsetfillcolor{currentfill}%
\pgfsetlinewidth{0.602250pt}%
\definecolor{currentstroke}{rgb}{0.000000,0.000000,0.000000}%
\pgfsetstrokecolor{currentstroke}%
\pgfsetdash{}{0pt}%
\pgfsys@defobject{currentmarker}{\pgfqpoint{0.000000in}{-0.027778in}}{\pgfqpoint{0.000000in}{0.000000in}}{%
\pgfpathmoveto{\pgfqpoint{0.000000in}{0.000000in}}%
\pgfpathlineto{\pgfqpoint{0.000000in}{-0.027778in}}%
\pgfusepath{stroke,fill}%
}%
\begin{pgfscope}%
\pgfsys@transformshift{2.038224in}{0.438889in}%
\pgfsys@useobject{currentmarker}{}%
\end{pgfscope}%
\end{pgfscope}%
\begin{pgfscope}%
\pgfsetbuttcap%
\pgfsetroundjoin%
\definecolor{currentfill}{rgb}{0.000000,0.000000,0.000000}%
\pgfsetfillcolor{currentfill}%
\pgfsetlinewidth{0.602250pt}%
\definecolor{currentstroke}{rgb}{0.000000,0.000000,0.000000}%
\pgfsetstrokecolor{currentstroke}%
\pgfsetdash{}{0pt}%
\pgfsys@defobject{currentmarker}{\pgfqpoint{0.000000in}{-0.027778in}}{\pgfqpoint{0.000000in}{0.000000in}}{%
\pgfpathmoveto{\pgfqpoint{0.000000in}{0.000000in}}%
\pgfpathlineto{\pgfqpoint{0.000000in}{-0.027778in}}%
\pgfusepath{stroke,fill}%
}%
\begin{pgfscope}%
\pgfsys@transformshift{2.096675in}{0.438889in}%
\pgfsys@useobject{currentmarker}{}%
\end{pgfscope}%
\end{pgfscope}%
\begin{pgfscope}%
\pgfsetbuttcap%
\pgfsetroundjoin%
\definecolor{currentfill}{rgb}{0.000000,0.000000,0.000000}%
\pgfsetfillcolor{currentfill}%
\pgfsetlinewidth{0.602250pt}%
\definecolor{currentstroke}{rgb}{0.000000,0.000000,0.000000}%
\pgfsetstrokecolor{currentstroke}%
\pgfsetdash{}{0pt}%
\pgfsys@defobject{currentmarker}{\pgfqpoint{0.000000in}{-0.027778in}}{\pgfqpoint{0.000000in}{0.000000in}}{%
\pgfpathmoveto{\pgfqpoint{0.000000in}{0.000000in}}%
\pgfpathlineto{\pgfqpoint{0.000000in}{-0.027778in}}%
\pgfusepath{stroke,fill}%
}%
\begin{pgfscope}%
\pgfsys@transformshift{2.144433in}{0.438889in}%
\pgfsys@useobject{currentmarker}{}%
\end{pgfscope}%
\end{pgfscope}%
\begin{pgfscope}%
\pgfsetbuttcap%
\pgfsetroundjoin%
\definecolor{currentfill}{rgb}{0.000000,0.000000,0.000000}%
\pgfsetfillcolor{currentfill}%
\pgfsetlinewidth{0.602250pt}%
\definecolor{currentstroke}{rgb}{0.000000,0.000000,0.000000}%
\pgfsetstrokecolor{currentstroke}%
\pgfsetdash{}{0pt}%
\pgfsys@defobject{currentmarker}{\pgfqpoint{0.000000in}{-0.027778in}}{\pgfqpoint{0.000000in}{0.000000in}}{%
\pgfpathmoveto{\pgfqpoint{0.000000in}{0.000000in}}%
\pgfpathlineto{\pgfqpoint{0.000000in}{-0.027778in}}%
\pgfusepath{stroke,fill}%
}%
\begin{pgfscope}%
\pgfsys@transformshift{2.184811in}{0.438889in}%
\pgfsys@useobject{currentmarker}{}%
\end{pgfscope}%
\end{pgfscope}%
\begin{pgfscope}%
\pgfsetbuttcap%
\pgfsetroundjoin%
\definecolor{currentfill}{rgb}{0.000000,0.000000,0.000000}%
\pgfsetfillcolor{currentfill}%
\pgfsetlinewidth{0.602250pt}%
\definecolor{currentstroke}{rgb}{0.000000,0.000000,0.000000}%
\pgfsetstrokecolor{currentstroke}%
\pgfsetdash{}{0pt}%
\pgfsys@defobject{currentmarker}{\pgfqpoint{0.000000in}{-0.027778in}}{\pgfqpoint{0.000000in}{0.000000in}}{%
\pgfpathmoveto{\pgfqpoint{0.000000in}{0.000000in}}%
\pgfpathlineto{\pgfqpoint{0.000000in}{-0.027778in}}%
\pgfusepath{stroke,fill}%
}%
\begin{pgfscope}%
\pgfsys@transformshift{2.219789in}{0.438889in}%
\pgfsys@useobject{currentmarker}{}%
\end{pgfscope}%
\end{pgfscope}%
\begin{pgfscope}%
\pgfsetbuttcap%
\pgfsetroundjoin%
\definecolor{currentfill}{rgb}{0.000000,0.000000,0.000000}%
\pgfsetfillcolor{currentfill}%
\pgfsetlinewidth{0.602250pt}%
\definecolor{currentstroke}{rgb}{0.000000,0.000000,0.000000}%
\pgfsetstrokecolor{currentstroke}%
\pgfsetdash{}{0pt}%
\pgfsys@defobject{currentmarker}{\pgfqpoint{0.000000in}{-0.027778in}}{\pgfqpoint{0.000000in}{0.000000in}}{%
\pgfpathmoveto{\pgfqpoint{0.000000in}{0.000000in}}%
\pgfpathlineto{\pgfqpoint{0.000000in}{-0.027778in}}%
\pgfusepath{stroke,fill}%
}%
\begin{pgfscope}%
\pgfsys@transformshift{2.250641in}{0.438889in}%
\pgfsys@useobject{currentmarker}{}%
\end{pgfscope}%
\end{pgfscope}%
\begin{pgfscope}%
\pgfsetbuttcap%
\pgfsetroundjoin%
\definecolor{currentfill}{rgb}{0.000000,0.000000,0.000000}%
\pgfsetfillcolor{currentfill}%
\pgfsetlinewidth{0.602250pt}%
\definecolor{currentstroke}{rgb}{0.000000,0.000000,0.000000}%
\pgfsetstrokecolor{currentstroke}%
\pgfsetdash{}{0pt}%
\pgfsys@defobject{currentmarker}{\pgfqpoint{0.000000in}{-0.027778in}}{\pgfqpoint{0.000000in}{0.000000in}}{%
\pgfpathmoveto{\pgfqpoint{0.000000in}{0.000000in}}%
\pgfpathlineto{\pgfqpoint{0.000000in}{-0.027778in}}%
\pgfusepath{stroke,fill}%
}%
\begin{pgfscope}%
\pgfsys@transformshift{2.459804in}{0.438889in}%
\pgfsys@useobject{currentmarker}{}%
\end{pgfscope}%
\end{pgfscope}%
\begin{pgfscope}%
\pgfsetbuttcap%
\pgfsetroundjoin%
\definecolor{currentfill}{rgb}{0.000000,0.000000,0.000000}%
\pgfsetfillcolor{currentfill}%
\pgfsetlinewidth{0.602250pt}%
\definecolor{currentstroke}{rgb}{0.000000,0.000000,0.000000}%
\pgfsetstrokecolor{currentstroke}%
\pgfsetdash{}{0pt}%
\pgfsys@defobject{currentmarker}{\pgfqpoint{0.000000in}{-0.027778in}}{\pgfqpoint{0.000000in}{0.000000in}}{%
\pgfpathmoveto{\pgfqpoint{0.000000in}{0.000000in}}%
\pgfpathlineto{\pgfqpoint{0.000000in}{-0.027778in}}%
\pgfusepath{stroke,fill}%
}%
\begin{pgfscope}%
\pgfsys@transformshift{2.566012in}{0.438889in}%
\pgfsys@useobject{currentmarker}{}%
\end{pgfscope}%
\end{pgfscope}%
\begin{pgfscope}%
\pgfsetbuttcap%
\pgfsetroundjoin%
\definecolor{currentfill}{rgb}{0.000000,0.000000,0.000000}%
\pgfsetfillcolor{currentfill}%
\pgfsetlinewidth{0.602250pt}%
\definecolor{currentstroke}{rgb}{0.000000,0.000000,0.000000}%
\pgfsetstrokecolor{currentstroke}%
\pgfsetdash{}{0pt}%
\pgfsys@defobject{currentmarker}{\pgfqpoint{0.000000in}{-0.027778in}}{\pgfqpoint{0.000000in}{0.000000in}}{%
\pgfpathmoveto{\pgfqpoint{0.000000in}{0.000000in}}%
\pgfpathlineto{\pgfqpoint{0.000000in}{-0.027778in}}%
\pgfusepath{stroke,fill}%
}%
\begin{pgfscope}%
\pgfsys@transformshift{2.641368in}{0.438889in}%
\pgfsys@useobject{currentmarker}{}%
\end{pgfscope}%
\end{pgfscope}%
\begin{pgfscope}%
\pgfsetbuttcap%
\pgfsetroundjoin%
\definecolor{currentfill}{rgb}{0.000000,0.000000,0.000000}%
\pgfsetfillcolor{currentfill}%
\pgfsetlinewidth{0.602250pt}%
\definecolor{currentstroke}{rgb}{0.000000,0.000000,0.000000}%
\pgfsetstrokecolor{currentstroke}%
\pgfsetdash{}{0pt}%
\pgfsys@defobject{currentmarker}{\pgfqpoint{0.000000in}{-0.027778in}}{\pgfqpoint{0.000000in}{0.000000in}}{%
\pgfpathmoveto{\pgfqpoint{0.000000in}{0.000000in}}%
\pgfpathlineto{\pgfqpoint{0.000000in}{-0.027778in}}%
\pgfusepath{stroke,fill}%
}%
\begin{pgfscope}%
\pgfsys@transformshift{2.699819in}{0.438889in}%
\pgfsys@useobject{currentmarker}{}%
\end{pgfscope}%
\end{pgfscope}%
\begin{pgfscope}%
\pgfsetbuttcap%
\pgfsetroundjoin%
\definecolor{currentfill}{rgb}{0.000000,0.000000,0.000000}%
\pgfsetfillcolor{currentfill}%
\pgfsetlinewidth{0.602250pt}%
\definecolor{currentstroke}{rgb}{0.000000,0.000000,0.000000}%
\pgfsetstrokecolor{currentstroke}%
\pgfsetdash{}{0pt}%
\pgfsys@defobject{currentmarker}{\pgfqpoint{0.000000in}{-0.027778in}}{\pgfqpoint{0.000000in}{0.000000in}}{%
\pgfpathmoveto{\pgfqpoint{0.000000in}{0.000000in}}%
\pgfpathlineto{\pgfqpoint{0.000000in}{-0.027778in}}%
\pgfusepath{stroke,fill}%
}%
\begin{pgfscope}%
\pgfsys@transformshift{2.747577in}{0.438889in}%
\pgfsys@useobject{currentmarker}{}%
\end{pgfscope}%
\end{pgfscope}%
\begin{pgfscope}%
\pgfsetbuttcap%
\pgfsetroundjoin%
\definecolor{currentfill}{rgb}{0.000000,0.000000,0.000000}%
\pgfsetfillcolor{currentfill}%
\pgfsetlinewidth{0.602250pt}%
\definecolor{currentstroke}{rgb}{0.000000,0.000000,0.000000}%
\pgfsetstrokecolor{currentstroke}%
\pgfsetdash{}{0pt}%
\pgfsys@defobject{currentmarker}{\pgfqpoint{0.000000in}{-0.027778in}}{\pgfqpoint{0.000000in}{0.000000in}}{%
\pgfpathmoveto{\pgfqpoint{0.000000in}{0.000000in}}%
\pgfpathlineto{\pgfqpoint{0.000000in}{-0.027778in}}%
\pgfusepath{stroke,fill}%
}%
\begin{pgfscope}%
\pgfsys@transformshift{2.787955in}{0.438889in}%
\pgfsys@useobject{currentmarker}{}%
\end{pgfscope}%
\end{pgfscope}%
\begin{pgfscope}%
\pgfsetbuttcap%
\pgfsetroundjoin%
\definecolor{currentfill}{rgb}{0.000000,0.000000,0.000000}%
\pgfsetfillcolor{currentfill}%
\pgfsetlinewidth{0.602250pt}%
\definecolor{currentstroke}{rgb}{0.000000,0.000000,0.000000}%
\pgfsetstrokecolor{currentstroke}%
\pgfsetdash{}{0pt}%
\pgfsys@defobject{currentmarker}{\pgfqpoint{0.000000in}{-0.027778in}}{\pgfqpoint{0.000000in}{0.000000in}}{%
\pgfpathmoveto{\pgfqpoint{0.000000in}{0.000000in}}%
\pgfpathlineto{\pgfqpoint{0.000000in}{-0.027778in}}%
\pgfusepath{stroke,fill}%
}%
\begin{pgfscope}%
\pgfsys@transformshift{2.822933in}{0.438889in}%
\pgfsys@useobject{currentmarker}{}%
\end{pgfscope}%
\end{pgfscope}%
\begin{pgfscope}%
\pgfsetbuttcap%
\pgfsetroundjoin%
\definecolor{currentfill}{rgb}{0.000000,0.000000,0.000000}%
\pgfsetfillcolor{currentfill}%
\pgfsetlinewidth{0.602250pt}%
\definecolor{currentstroke}{rgb}{0.000000,0.000000,0.000000}%
\pgfsetstrokecolor{currentstroke}%
\pgfsetdash{}{0pt}%
\pgfsys@defobject{currentmarker}{\pgfqpoint{0.000000in}{-0.027778in}}{\pgfqpoint{0.000000in}{0.000000in}}{%
\pgfpathmoveto{\pgfqpoint{0.000000in}{0.000000in}}%
\pgfpathlineto{\pgfqpoint{0.000000in}{-0.027778in}}%
\pgfusepath{stroke,fill}%
}%
\begin{pgfscope}%
\pgfsys@transformshift{2.853785in}{0.438889in}%
\pgfsys@useobject{currentmarker}{}%
\end{pgfscope}%
\end{pgfscope}%
\begin{pgfscope}%
\pgfsetbuttcap%
\pgfsetroundjoin%
\definecolor{currentfill}{rgb}{0.000000,0.000000,0.000000}%
\pgfsetfillcolor{currentfill}%
\pgfsetlinewidth{0.602250pt}%
\definecolor{currentstroke}{rgb}{0.000000,0.000000,0.000000}%
\pgfsetstrokecolor{currentstroke}%
\pgfsetdash{}{0pt}%
\pgfsys@defobject{currentmarker}{\pgfqpoint{0.000000in}{-0.027778in}}{\pgfqpoint{0.000000in}{0.000000in}}{%
\pgfpathmoveto{\pgfqpoint{0.000000in}{0.000000in}}%
\pgfpathlineto{\pgfqpoint{0.000000in}{-0.027778in}}%
\pgfusepath{stroke,fill}%
}%
\begin{pgfscope}%
\pgfsys@transformshift{3.062948in}{0.438889in}%
\pgfsys@useobject{currentmarker}{}%
\end{pgfscope}%
\end{pgfscope}%
\begin{pgfscope}%
\pgfsetbuttcap%
\pgfsetroundjoin%
\definecolor{currentfill}{rgb}{0.000000,0.000000,0.000000}%
\pgfsetfillcolor{currentfill}%
\pgfsetlinewidth{0.602250pt}%
\definecolor{currentstroke}{rgb}{0.000000,0.000000,0.000000}%
\pgfsetstrokecolor{currentstroke}%
\pgfsetdash{}{0pt}%
\pgfsys@defobject{currentmarker}{\pgfqpoint{0.000000in}{-0.027778in}}{\pgfqpoint{0.000000in}{0.000000in}}{%
\pgfpathmoveto{\pgfqpoint{0.000000in}{0.000000in}}%
\pgfpathlineto{\pgfqpoint{0.000000in}{-0.027778in}}%
\pgfusepath{stroke,fill}%
}%
\begin{pgfscope}%
\pgfsys@transformshift{3.169156in}{0.438889in}%
\pgfsys@useobject{currentmarker}{}%
\end{pgfscope}%
\end{pgfscope}%
\begin{pgfscope}%
\pgfsetbuttcap%
\pgfsetroundjoin%
\definecolor{currentfill}{rgb}{0.000000,0.000000,0.000000}%
\pgfsetfillcolor{currentfill}%
\pgfsetlinewidth{0.602250pt}%
\definecolor{currentstroke}{rgb}{0.000000,0.000000,0.000000}%
\pgfsetstrokecolor{currentstroke}%
\pgfsetdash{}{0pt}%
\pgfsys@defobject{currentmarker}{\pgfqpoint{0.000000in}{-0.027778in}}{\pgfqpoint{0.000000in}{0.000000in}}{%
\pgfpathmoveto{\pgfqpoint{0.000000in}{0.000000in}}%
\pgfpathlineto{\pgfqpoint{0.000000in}{-0.027778in}}%
\pgfusepath{stroke,fill}%
}%
\begin{pgfscope}%
\pgfsys@transformshift{3.244512in}{0.438889in}%
\pgfsys@useobject{currentmarker}{}%
\end{pgfscope}%
\end{pgfscope}%
\begin{pgfscope}%
\pgfsetbuttcap%
\pgfsetroundjoin%
\definecolor{currentfill}{rgb}{0.000000,0.000000,0.000000}%
\pgfsetfillcolor{currentfill}%
\pgfsetlinewidth{0.602250pt}%
\definecolor{currentstroke}{rgb}{0.000000,0.000000,0.000000}%
\pgfsetstrokecolor{currentstroke}%
\pgfsetdash{}{0pt}%
\pgfsys@defobject{currentmarker}{\pgfqpoint{0.000000in}{-0.027778in}}{\pgfqpoint{0.000000in}{0.000000in}}{%
\pgfpathmoveto{\pgfqpoint{0.000000in}{0.000000in}}%
\pgfpathlineto{\pgfqpoint{0.000000in}{-0.027778in}}%
\pgfusepath{stroke,fill}%
}%
\begin{pgfscope}%
\pgfsys@transformshift{3.302963in}{0.438889in}%
\pgfsys@useobject{currentmarker}{}%
\end{pgfscope}%
\end{pgfscope}%
\begin{pgfscope}%
\pgfsetbuttcap%
\pgfsetroundjoin%
\definecolor{currentfill}{rgb}{0.000000,0.000000,0.000000}%
\pgfsetfillcolor{currentfill}%
\pgfsetlinewidth{0.602250pt}%
\definecolor{currentstroke}{rgb}{0.000000,0.000000,0.000000}%
\pgfsetstrokecolor{currentstroke}%
\pgfsetdash{}{0pt}%
\pgfsys@defobject{currentmarker}{\pgfqpoint{0.000000in}{-0.027778in}}{\pgfqpoint{0.000000in}{0.000000in}}{%
\pgfpathmoveto{\pgfqpoint{0.000000in}{0.000000in}}%
\pgfpathlineto{\pgfqpoint{0.000000in}{-0.027778in}}%
\pgfusepath{stroke,fill}%
}%
\begin{pgfscope}%
\pgfsys@transformshift{3.350721in}{0.438889in}%
\pgfsys@useobject{currentmarker}{}%
\end{pgfscope}%
\end{pgfscope}%
\begin{pgfscope}%
\pgfsetbuttcap%
\pgfsetroundjoin%
\definecolor{currentfill}{rgb}{0.000000,0.000000,0.000000}%
\pgfsetfillcolor{currentfill}%
\pgfsetlinewidth{0.602250pt}%
\definecolor{currentstroke}{rgb}{0.000000,0.000000,0.000000}%
\pgfsetstrokecolor{currentstroke}%
\pgfsetdash{}{0pt}%
\pgfsys@defobject{currentmarker}{\pgfqpoint{0.000000in}{-0.027778in}}{\pgfqpoint{0.000000in}{0.000000in}}{%
\pgfpathmoveto{\pgfqpoint{0.000000in}{0.000000in}}%
\pgfpathlineto{\pgfqpoint{0.000000in}{-0.027778in}}%
\pgfusepath{stroke,fill}%
}%
\begin{pgfscope}%
\pgfsys@transformshift{3.391099in}{0.438889in}%
\pgfsys@useobject{currentmarker}{}%
\end{pgfscope}%
\end{pgfscope}%
\begin{pgfscope}%
\pgfsetbuttcap%
\pgfsetroundjoin%
\definecolor{currentfill}{rgb}{0.000000,0.000000,0.000000}%
\pgfsetfillcolor{currentfill}%
\pgfsetlinewidth{0.602250pt}%
\definecolor{currentstroke}{rgb}{0.000000,0.000000,0.000000}%
\pgfsetstrokecolor{currentstroke}%
\pgfsetdash{}{0pt}%
\pgfsys@defobject{currentmarker}{\pgfqpoint{0.000000in}{-0.027778in}}{\pgfqpoint{0.000000in}{0.000000in}}{%
\pgfpathmoveto{\pgfqpoint{0.000000in}{0.000000in}}%
\pgfpathlineto{\pgfqpoint{0.000000in}{-0.027778in}}%
\pgfusepath{stroke,fill}%
}%
\begin{pgfscope}%
\pgfsys@transformshift{3.426077in}{0.438889in}%
\pgfsys@useobject{currentmarker}{}%
\end{pgfscope}%
\end{pgfscope}%
\begin{pgfscope}%
\pgfsetbuttcap%
\pgfsetroundjoin%
\definecolor{currentfill}{rgb}{0.000000,0.000000,0.000000}%
\pgfsetfillcolor{currentfill}%
\pgfsetlinewidth{0.602250pt}%
\definecolor{currentstroke}{rgb}{0.000000,0.000000,0.000000}%
\pgfsetstrokecolor{currentstroke}%
\pgfsetdash{}{0pt}%
\pgfsys@defobject{currentmarker}{\pgfqpoint{0.000000in}{-0.027778in}}{\pgfqpoint{0.000000in}{0.000000in}}{%
\pgfpathmoveto{\pgfqpoint{0.000000in}{0.000000in}}%
\pgfpathlineto{\pgfqpoint{0.000000in}{-0.027778in}}%
\pgfusepath{stroke,fill}%
}%
\begin{pgfscope}%
\pgfsys@transformshift{3.456929in}{0.438889in}%
\pgfsys@useobject{currentmarker}{}%
\end{pgfscope}%
\end{pgfscope}%
\begin{pgfscope}%
\pgfsetbuttcap%
\pgfsetroundjoin%
\definecolor{currentfill}{rgb}{0.000000,0.000000,0.000000}%
\pgfsetfillcolor{currentfill}%
\pgfsetlinewidth{0.602250pt}%
\definecolor{currentstroke}{rgb}{0.000000,0.000000,0.000000}%
\pgfsetstrokecolor{currentstroke}%
\pgfsetdash{}{0pt}%
\pgfsys@defobject{currentmarker}{\pgfqpoint{0.000000in}{-0.027778in}}{\pgfqpoint{0.000000in}{0.000000in}}{%
\pgfpathmoveto{\pgfqpoint{0.000000in}{0.000000in}}%
\pgfpathlineto{\pgfqpoint{0.000000in}{-0.027778in}}%
\pgfusepath{stroke,fill}%
}%
\begin{pgfscope}%
\pgfsys@transformshift{3.666092in}{0.438889in}%
\pgfsys@useobject{currentmarker}{}%
\end{pgfscope}%
\end{pgfscope}%
\begin{pgfscope}%
\pgfsetbuttcap%
\pgfsetroundjoin%
\definecolor{currentfill}{rgb}{0.000000,0.000000,0.000000}%
\pgfsetfillcolor{currentfill}%
\pgfsetlinewidth{0.602250pt}%
\definecolor{currentstroke}{rgb}{0.000000,0.000000,0.000000}%
\pgfsetstrokecolor{currentstroke}%
\pgfsetdash{}{0pt}%
\pgfsys@defobject{currentmarker}{\pgfqpoint{0.000000in}{-0.027778in}}{\pgfqpoint{0.000000in}{0.000000in}}{%
\pgfpathmoveto{\pgfqpoint{0.000000in}{0.000000in}}%
\pgfpathlineto{\pgfqpoint{0.000000in}{-0.027778in}}%
\pgfusepath{stroke,fill}%
}%
\begin{pgfscope}%
\pgfsys@transformshift{3.772300in}{0.438889in}%
\pgfsys@useobject{currentmarker}{}%
\end{pgfscope}%
\end{pgfscope}%
\begin{pgfscope}%
\pgfsetbuttcap%
\pgfsetroundjoin%
\definecolor{currentfill}{rgb}{0.000000,0.000000,0.000000}%
\pgfsetfillcolor{currentfill}%
\pgfsetlinewidth{0.602250pt}%
\definecolor{currentstroke}{rgb}{0.000000,0.000000,0.000000}%
\pgfsetstrokecolor{currentstroke}%
\pgfsetdash{}{0pt}%
\pgfsys@defobject{currentmarker}{\pgfqpoint{0.000000in}{-0.027778in}}{\pgfqpoint{0.000000in}{0.000000in}}{%
\pgfpathmoveto{\pgfqpoint{0.000000in}{0.000000in}}%
\pgfpathlineto{\pgfqpoint{0.000000in}{-0.027778in}}%
\pgfusepath{stroke,fill}%
}%
\begin{pgfscope}%
\pgfsys@transformshift{3.847656in}{0.438889in}%
\pgfsys@useobject{currentmarker}{}%
\end{pgfscope}%
\end{pgfscope}%
\begin{pgfscope}%
\pgfsetbuttcap%
\pgfsetroundjoin%
\definecolor{currentfill}{rgb}{0.000000,0.000000,0.000000}%
\pgfsetfillcolor{currentfill}%
\pgfsetlinewidth{0.803000pt}%
\definecolor{currentstroke}{rgb}{0.000000,0.000000,0.000000}%
\pgfsetstrokecolor{currentstroke}%
\pgfsetdash{}{0pt}%
\pgfsys@defobject{currentmarker}{\pgfqpoint{-0.048611in}{0.000000in}}{\pgfqpoint{0.000000in}{0.000000in}}{%
\pgfpathmoveto{\pgfqpoint{0.000000in}{0.000000in}}%
\pgfpathlineto{\pgfqpoint{-0.048611in}{0.000000in}}%
\pgfusepath{stroke,fill}%
}%
\begin{pgfscope}%
\pgfsys@transformshift{0.557986in}{0.647407in}%
\pgfsys@useobject{currentmarker}{}%
\end{pgfscope}%
\end{pgfscope}%
\begin{pgfscope}%
\definecolor{textcolor}{rgb}{0.000000,0.000000,0.000000}%
\pgfsetstrokecolor{textcolor}%
\pgfsetfillcolor{textcolor}%
\pgftext[x=0.239884in,y=0.594645in,left,base]{\color{textcolor}\sffamily\fontsize{10.000000}{12.000000}\selectfont 0.0}%
\end{pgfscope}%
\begin{pgfscope}%
\pgfsetbuttcap%
\pgfsetroundjoin%
\definecolor{currentfill}{rgb}{0.000000,0.000000,0.000000}%
\pgfsetfillcolor{currentfill}%
\pgfsetlinewidth{0.803000pt}%
\definecolor{currentstroke}{rgb}{0.000000,0.000000,0.000000}%
\pgfsetstrokecolor{currentstroke}%
\pgfsetdash{}{0pt}%
\pgfsys@defobject{currentmarker}{\pgfqpoint{-0.048611in}{0.000000in}}{\pgfqpoint{0.000000in}{0.000000in}}{%
\pgfpathmoveto{\pgfqpoint{0.000000in}{0.000000in}}%
\pgfpathlineto{\pgfqpoint{-0.048611in}{0.000000in}}%
\pgfusepath{stroke,fill}%
}%
\begin{pgfscope}%
\pgfsys@transformshift{0.557986in}{0.944143in}%
\pgfsys@useobject{currentmarker}{}%
\end{pgfscope}%
\end{pgfscope}%
\begin{pgfscope}%
\definecolor{textcolor}{rgb}{0.000000,0.000000,0.000000}%
\pgfsetstrokecolor{textcolor}%
\pgfsetfillcolor{textcolor}%
\pgftext[x=0.239884in,y=0.891382in,left,base]{\color{textcolor}\sffamily\fontsize{10.000000}{12.000000}\selectfont 0.2}%
\end{pgfscope}%
\begin{pgfscope}%
\pgfsetbuttcap%
\pgfsetroundjoin%
\definecolor{currentfill}{rgb}{0.000000,0.000000,0.000000}%
\pgfsetfillcolor{currentfill}%
\pgfsetlinewidth{0.803000pt}%
\definecolor{currentstroke}{rgb}{0.000000,0.000000,0.000000}%
\pgfsetstrokecolor{currentstroke}%
\pgfsetdash{}{0pt}%
\pgfsys@defobject{currentmarker}{\pgfqpoint{-0.048611in}{0.000000in}}{\pgfqpoint{0.000000in}{0.000000in}}{%
\pgfpathmoveto{\pgfqpoint{0.000000in}{0.000000in}}%
\pgfpathlineto{\pgfqpoint{-0.048611in}{0.000000in}}%
\pgfusepath{stroke,fill}%
}%
\begin{pgfscope}%
\pgfsys@transformshift{0.557986in}{1.240879in}%
\pgfsys@useobject{currentmarker}{}%
\end{pgfscope}%
\end{pgfscope}%
\begin{pgfscope}%
\definecolor{textcolor}{rgb}{0.000000,0.000000,0.000000}%
\pgfsetstrokecolor{textcolor}%
\pgfsetfillcolor{textcolor}%
\pgftext[x=0.239884in,y=1.188118in,left,base]{\color{textcolor}\sffamily\fontsize{10.000000}{12.000000}\selectfont 0.4}%
\end{pgfscope}%
\begin{pgfscope}%
\pgfsetbuttcap%
\pgfsetroundjoin%
\definecolor{currentfill}{rgb}{0.000000,0.000000,0.000000}%
\pgfsetfillcolor{currentfill}%
\pgfsetlinewidth{0.803000pt}%
\definecolor{currentstroke}{rgb}{0.000000,0.000000,0.000000}%
\pgfsetstrokecolor{currentstroke}%
\pgfsetdash{}{0pt}%
\pgfsys@defobject{currentmarker}{\pgfqpoint{-0.048611in}{0.000000in}}{\pgfqpoint{0.000000in}{0.000000in}}{%
\pgfpathmoveto{\pgfqpoint{0.000000in}{0.000000in}}%
\pgfpathlineto{\pgfqpoint{-0.048611in}{0.000000in}}%
\pgfusepath{stroke,fill}%
}%
\begin{pgfscope}%
\pgfsys@transformshift{0.557986in}{1.537616in}%
\pgfsys@useobject{currentmarker}{}%
\end{pgfscope}%
\end{pgfscope}%
\begin{pgfscope}%
\definecolor{textcolor}{rgb}{0.000000,0.000000,0.000000}%
\pgfsetstrokecolor{textcolor}%
\pgfsetfillcolor{textcolor}%
\pgftext[x=0.239884in,y=1.484854in,left,base]{\color{textcolor}\sffamily\fontsize{10.000000}{12.000000}\selectfont 0.6}%
\end{pgfscope}%
\begin{pgfscope}%
\pgfsetbuttcap%
\pgfsetroundjoin%
\definecolor{currentfill}{rgb}{0.000000,0.000000,0.000000}%
\pgfsetfillcolor{currentfill}%
\pgfsetlinewidth{0.803000pt}%
\definecolor{currentstroke}{rgb}{0.000000,0.000000,0.000000}%
\pgfsetstrokecolor{currentstroke}%
\pgfsetdash{}{0pt}%
\pgfsys@defobject{currentmarker}{\pgfqpoint{-0.048611in}{0.000000in}}{\pgfqpoint{0.000000in}{0.000000in}}{%
\pgfpathmoveto{\pgfqpoint{0.000000in}{0.000000in}}%
\pgfpathlineto{\pgfqpoint{-0.048611in}{0.000000in}}%
\pgfusepath{stroke,fill}%
}%
\begin{pgfscope}%
\pgfsys@transformshift{0.557986in}{1.834352in}%
\pgfsys@useobject{currentmarker}{}%
\end{pgfscope}%
\end{pgfscope}%
\begin{pgfscope}%
\definecolor{textcolor}{rgb}{0.000000,0.000000,0.000000}%
\pgfsetstrokecolor{textcolor}%
\pgfsetfillcolor{textcolor}%
\pgftext[x=0.239884in,y=1.781591in,left,base]{\color{textcolor}\sffamily\fontsize{10.000000}{12.000000}\selectfont 0.8}%
\end{pgfscope}%
\begin{pgfscope}%
\pgfsetbuttcap%
\pgfsetroundjoin%
\definecolor{currentfill}{rgb}{0.000000,0.000000,0.000000}%
\pgfsetfillcolor{currentfill}%
\pgfsetlinewidth{0.803000pt}%
\definecolor{currentstroke}{rgb}{0.000000,0.000000,0.000000}%
\pgfsetstrokecolor{currentstroke}%
\pgfsetdash{}{0pt}%
\pgfsys@defobject{currentmarker}{\pgfqpoint{-0.048611in}{0.000000in}}{\pgfqpoint{0.000000in}{0.000000in}}{%
\pgfpathmoveto{\pgfqpoint{0.000000in}{0.000000in}}%
\pgfpathlineto{\pgfqpoint{-0.048611in}{0.000000in}}%
\pgfusepath{stroke,fill}%
}%
\begin{pgfscope}%
\pgfsys@transformshift{0.557986in}{2.131088in}%
\pgfsys@useobject{currentmarker}{}%
\end{pgfscope}%
\end{pgfscope}%
\begin{pgfscope}%
\definecolor{textcolor}{rgb}{0.000000,0.000000,0.000000}%
\pgfsetstrokecolor{textcolor}%
\pgfsetfillcolor{textcolor}%
\pgftext[x=0.239884in,y=2.078327in,left,base]{\color{textcolor}\sffamily\fontsize{10.000000}{12.000000}\selectfont 1.0}%
\end{pgfscope}%
\begin{pgfscope}%
\pgfsetbuttcap%
\pgfsetroundjoin%
\definecolor{currentfill}{rgb}{0.000000,0.000000,0.000000}%
\pgfsetfillcolor{currentfill}%
\pgfsetlinewidth{0.803000pt}%
\definecolor{currentstroke}{rgb}{0.000000,0.000000,0.000000}%
\pgfsetstrokecolor{currentstroke}%
\pgfsetdash{}{0pt}%
\pgfsys@defobject{currentmarker}{\pgfqpoint{-0.048611in}{0.000000in}}{\pgfqpoint{0.000000in}{0.000000in}}{%
\pgfpathmoveto{\pgfqpoint{0.000000in}{0.000000in}}%
\pgfpathlineto{\pgfqpoint{-0.048611in}{0.000000in}}%
\pgfusepath{stroke,fill}%
}%
\begin{pgfscope}%
\pgfsys@transformshift{0.557986in}{2.427825in}%
\pgfsys@useobject{currentmarker}{}%
\end{pgfscope}%
\end{pgfscope}%
\begin{pgfscope}%
\definecolor{textcolor}{rgb}{0.000000,0.000000,0.000000}%
\pgfsetstrokecolor{textcolor}%
\pgfsetfillcolor{textcolor}%
\pgftext[x=0.239884in,y=2.375063in,left,base]{\color{textcolor}\sffamily\fontsize{10.000000}{12.000000}\selectfont 1.2}%
\end{pgfscope}%
\begin{pgfscope}%
\pgfpathrectangle{\pgfqpoint{0.557986in}{0.438889in}}{\pgfqpoint{3.318403in}{2.263889in}}%
\pgfusepath{clip}%
\pgfsetrectcap%
\pgfsetroundjoin%
\pgfsetlinewidth{1.505625pt}%
\definecolor{currentstroke}{rgb}{0.121569,0.466667,0.705882}%
\pgfsetstrokecolor{currentstroke}%
\pgfsetdash{}{0pt}%
\pgfpathmoveto{\pgfqpoint{0.708823in}{0.677637in}}%
\pgfpathlineto{\pgfqpoint{0.859609in}{0.665474in}}%
\pgfpathlineto{\pgfqpoint{1.010395in}{0.704744in}}%
\pgfpathlineto{\pgfqpoint{1.161181in}{0.696222in}}%
\pgfpathlineto{\pgfqpoint{1.311967in}{0.741800in}}%
\pgfpathlineto{\pgfqpoint{1.462753in}{0.754834in}}%
\pgfpathlineto{\pgfqpoint{1.613539in}{0.828864in}}%
\pgfpathlineto{\pgfqpoint{1.764325in}{0.936655in}}%
\pgfpathlineto{\pgfqpoint{1.915111in}{1.059509in}}%
\pgfpathlineto{\pgfqpoint{2.065897in}{0.890936in}}%
\pgfpathlineto{\pgfqpoint{2.216683in}{0.983823in}}%
\pgfpathlineto{\pgfqpoint{2.367469in}{1.293789in}}%
\pgfpathlineto{\pgfqpoint{2.518255in}{1.459498in}}%
\pgfpathlineto{\pgfqpoint{2.669041in}{2.050375in}}%
\pgfpathlineto{\pgfqpoint{2.819827in}{1.935330in}}%
\pgfpathlineto{\pgfqpoint{2.970613in}{1.835961in}}%
\pgfpathlineto{\pgfqpoint{3.121399in}{2.105455in}}%
\pgfpathlineto{\pgfqpoint{3.272185in}{1.841539in}}%
\pgfpathlineto{\pgfqpoint{3.422971in}{1.892115in}}%
\pgfpathlineto{\pgfqpoint{3.573756in}{2.130019in}}%
\pgfpathlineto{\pgfqpoint{3.724542in}{2.010187in}}%
\pgfusepath{stroke}%
\end{pgfscope}%
\begin{pgfscope}%
\pgfpathrectangle{\pgfqpoint{0.557986in}{0.438889in}}{\pgfqpoint{3.318403in}{2.263889in}}%
\pgfusepath{clip}%
\pgfsetrectcap%
\pgfsetroundjoin%
\pgfsetlinewidth{1.505625pt}%
\definecolor{currentstroke}{rgb}{1.000000,0.498039,0.054902}%
\pgfsetstrokecolor{currentstroke}%
\pgfsetdash{}{0pt}%
\pgfpathmoveto{\pgfqpoint{0.708823in}{0.671527in}}%
\pgfpathlineto{\pgfqpoint{0.859609in}{0.678928in}}%
\pgfpathlineto{\pgfqpoint{1.010395in}{0.677951in}}%
\pgfpathlineto{\pgfqpoint{1.161181in}{0.717109in}}%
\pgfpathlineto{\pgfqpoint{1.311967in}{0.690322in}}%
\pgfpathlineto{\pgfqpoint{1.462753in}{0.709863in}}%
\pgfpathlineto{\pgfqpoint{1.613539in}{0.751513in}}%
\pgfpathlineto{\pgfqpoint{1.764325in}{0.820832in}}%
\pgfpathlineto{\pgfqpoint{1.915111in}{0.878569in}}%
\pgfpathlineto{\pgfqpoint{2.065897in}{0.914162in}}%
\pgfpathlineto{\pgfqpoint{2.216683in}{1.137951in}}%
\pgfpathlineto{\pgfqpoint{2.367469in}{1.175775in}}%
\pgfpathlineto{\pgfqpoint{2.518255in}{1.410782in}}%
\pgfpathlineto{\pgfqpoint{2.669041in}{1.700231in}}%
\pgfpathlineto{\pgfqpoint{2.819827in}{2.063352in}}%
\pgfpathlineto{\pgfqpoint{2.970613in}{2.111064in}}%
\pgfpathlineto{\pgfqpoint{3.121399in}{1.880269in}}%
\pgfpathlineto{\pgfqpoint{3.272185in}{1.837341in}}%
\pgfpathlineto{\pgfqpoint{3.422971in}{2.111034in}}%
\pgfpathlineto{\pgfqpoint{3.573756in}{2.111044in}}%
\pgfpathlineto{\pgfqpoint{3.724542in}{2.042036in}}%
\pgfusepath{stroke}%
\end{pgfscope}%
\begin{pgfscope}%
\pgfpathrectangle{\pgfqpoint{0.557986in}{0.438889in}}{\pgfqpoint{3.318403in}{2.263889in}}%
\pgfusepath{clip}%
\pgfsetrectcap%
\pgfsetroundjoin%
\pgfsetlinewidth{1.505625pt}%
\definecolor{currentstroke}{rgb}{0.172549,0.627451,0.172549}%
\pgfsetstrokecolor{currentstroke}%
\pgfsetdash{}{0pt}%
\pgfpathmoveto{\pgfqpoint{0.708823in}{0.680541in}}%
\pgfpathlineto{\pgfqpoint{0.859609in}{0.679997in}}%
\pgfpathlineto{\pgfqpoint{1.010395in}{0.708723in}}%
\pgfpathlineto{\pgfqpoint{1.161181in}{0.715009in}}%
\pgfpathlineto{\pgfqpoint{1.311967in}{0.742244in}}%
\pgfpathlineto{\pgfqpoint{1.462753in}{0.765060in}}%
\pgfpathlineto{\pgfqpoint{1.613539in}{0.762023in}}%
\pgfpathlineto{\pgfqpoint{1.764325in}{0.857474in}}%
\pgfpathlineto{\pgfqpoint{1.915111in}{0.805723in}}%
\pgfpathlineto{\pgfqpoint{2.065897in}{0.949944in}}%
\pgfpathlineto{\pgfqpoint{2.216683in}{1.102171in}}%
\pgfpathlineto{\pgfqpoint{2.367469in}{0.947582in}}%
\pgfpathlineto{\pgfqpoint{2.518255in}{1.030691in}}%
\pgfpathlineto{\pgfqpoint{2.669041in}{1.292833in}}%
\pgfpathlineto{\pgfqpoint{2.819827in}{1.981429in}}%
\pgfpathlineto{\pgfqpoint{2.970613in}{1.986670in}}%
\pgfpathlineto{\pgfqpoint{3.121399in}{1.777189in}}%
\pgfpathlineto{\pgfqpoint{3.272185in}{1.930304in}}%
\pgfpathlineto{\pgfqpoint{3.422971in}{1.999769in}}%
\pgfpathlineto{\pgfqpoint{3.573756in}{1.999389in}}%
\pgfpathlineto{\pgfqpoint{3.724542in}{1.999199in}}%
\pgfusepath{stroke}%
\end{pgfscope}%
\begin{pgfscope}%
\pgfpathrectangle{\pgfqpoint{0.557986in}{0.438889in}}{\pgfqpoint{3.318403in}{2.263889in}}%
\pgfusepath{clip}%
\pgfsetrectcap%
\pgfsetroundjoin%
\pgfsetlinewidth{1.505625pt}%
\definecolor{currentstroke}{rgb}{0.839216,0.152941,0.156863}%
\pgfsetstrokecolor{currentstroke}%
\pgfsetdash{}{0pt}%
\pgfpathmoveto{\pgfqpoint{0.708823in}{0.666813in}}%
\pgfpathlineto{\pgfqpoint{0.859609in}{0.674807in}}%
\pgfpathlineto{\pgfqpoint{1.010395in}{0.685725in}}%
\pgfpathlineto{\pgfqpoint{1.161181in}{0.688401in}}%
\pgfpathlineto{\pgfqpoint{1.311967in}{0.700754in}}%
\pgfpathlineto{\pgfqpoint{1.462753in}{0.708376in}}%
\pgfpathlineto{\pgfqpoint{1.613539in}{0.704879in}}%
\pgfpathlineto{\pgfqpoint{1.764325in}{0.748116in}}%
\pgfpathlineto{\pgfqpoint{1.915111in}{0.837081in}}%
\pgfpathlineto{\pgfqpoint{2.065897in}{0.831179in}}%
\pgfpathlineto{\pgfqpoint{2.216683in}{0.819256in}}%
\pgfpathlineto{\pgfqpoint{2.367469in}{0.771371in}}%
\pgfpathlineto{\pgfqpoint{2.518255in}{0.747130in}}%
\pgfpathlineto{\pgfqpoint{2.669041in}{0.775828in}}%
\pgfpathlineto{\pgfqpoint{2.819827in}{0.820274in}}%
\pgfpathlineto{\pgfqpoint{2.970613in}{0.805079in}}%
\pgfpathlineto{\pgfqpoint{3.121399in}{0.804892in}}%
\pgfpathlineto{\pgfqpoint{3.272185in}{0.798488in}}%
\pgfpathlineto{\pgfqpoint{3.422971in}{0.787199in}}%
\pgfpathlineto{\pgfqpoint{3.573756in}{0.794933in}}%
\pgfpathlineto{\pgfqpoint{3.724542in}{0.795994in}}%
\pgfusepath{stroke}%
\end{pgfscope}%
\begin{pgfscope}%
\pgfpathrectangle{\pgfqpoint{0.557986in}{0.438889in}}{\pgfqpoint{3.318403in}{2.263889in}}%
\pgfusepath{clip}%
\pgfsetrectcap%
\pgfsetroundjoin%
\pgfsetlinewidth{1.505625pt}%
\definecolor{currentstroke}{rgb}{0.580392,0.403922,0.741176}%
\pgfsetstrokecolor{currentstroke}%
\pgfsetdash{}{0pt}%
\pgfpathmoveto{\pgfqpoint{0.708823in}{0.671586in}}%
\pgfpathlineto{\pgfqpoint{0.859609in}{0.678810in}}%
\pgfpathlineto{\pgfqpoint{1.010395in}{0.676971in}}%
\pgfpathlineto{\pgfqpoint{1.161181in}{0.665443in}}%
\pgfpathlineto{\pgfqpoint{1.311967in}{0.716756in}}%
\pgfpathlineto{\pgfqpoint{1.462753in}{0.711667in}}%
\pgfpathlineto{\pgfqpoint{1.613539in}{0.707580in}}%
\pgfpathlineto{\pgfqpoint{1.764325in}{0.747807in}}%
\pgfpathlineto{\pgfqpoint{1.915111in}{0.723403in}}%
\pgfpathlineto{\pgfqpoint{2.065897in}{0.703011in}}%
\pgfpathlineto{\pgfqpoint{2.216683in}{0.713746in}}%
\pgfpathlineto{\pgfqpoint{2.367469in}{0.702142in}}%
\pgfpathlineto{\pgfqpoint{2.518255in}{0.708099in}}%
\pgfpathlineto{\pgfqpoint{2.669041in}{0.714705in}}%
\pgfpathlineto{\pgfqpoint{2.819827in}{0.707268in}}%
\pgfpathlineto{\pgfqpoint{2.970613in}{0.707771in}}%
\pgfpathlineto{\pgfqpoint{3.121399in}{0.709327in}}%
\pgfpathlineto{\pgfqpoint{3.272185in}{0.709310in}}%
\pgfpathlineto{\pgfqpoint{3.422971in}{0.709991in}}%
\pgfpathlineto{\pgfqpoint{3.573756in}{0.709863in}}%
\pgfpathlineto{\pgfqpoint{3.724542in}{0.710772in}}%
\pgfusepath{stroke}%
\end{pgfscope}%
\begin{pgfscope}%
\pgfsetrectcap%
\pgfsetmiterjoin%
\pgfsetlinewidth{0.803000pt}%
\definecolor{currentstroke}{rgb}{0.000000,0.000000,0.000000}%
\pgfsetstrokecolor{currentstroke}%
\pgfsetdash{}{0pt}%
\pgfpathmoveto{\pgfqpoint{0.557986in}{0.438889in}}%
\pgfpathlineto{\pgfqpoint{0.557986in}{2.702778in}}%
\pgfusepath{stroke}%
\end{pgfscope}%
\begin{pgfscope}%
\pgfsetrectcap%
\pgfsetmiterjoin%
\pgfsetlinewidth{0.803000pt}%
\definecolor{currentstroke}{rgb}{0.000000,0.000000,0.000000}%
\pgfsetstrokecolor{currentstroke}%
\pgfsetdash{}{0pt}%
\pgfpathmoveto{\pgfqpoint{3.876389in}{0.438889in}}%
\pgfpathlineto{\pgfqpoint{3.876389in}{2.702778in}}%
\pgfusepath{stroke}%
\end{pgfscope}%
\begin{pgfscope}%
\pgfsetrectcap%
\pgfsetmiterjoin%
\pgfsetlinewidth{0.803000pt}%
\definecolor{currentstroke}{rgb}{0.000000,0.000000,0.000000}%
\pgfsetstrokecolor{currentstroke}%
\pgfsetdash{}{0pt}%
\pgfpathmoveto{\pgfqpoint{0.557986in}{0.438889in}}%
\pgfpathlineto{\pgfqpoint{3.876389in}{0.438889in}}%
\pgfusepath{stroke}%
\end{pgfscope}%
\begin{pgfscope}%
\pgfsetrectcap%
\pgfsetmiterjoin%
\pgfsetlinewidth{0.803000pt}%
\definecolor{currentstroke}{rgb}{0.000000,0.000000,0.000000}%
\pgfsetstrokecolor{currentstroke}%
\pgfsetdash{}{0pt}%
\pgfpathmoveto{\pgfqpoint{0.557986in}{2.702778in}}%
\pgfpathlineto{\pgfqpoint{3.876389in}{2.702778in}}%
\pgfusepath{stroke}%
\end{pgfscope}%
\begin{pgfscope}%
\definecolor{textcolor}{rgb}{0.000000,0.000000,0.000000}%
\pgfsetstrokecolor{textcolor}%
\pgfsetfillcolor{textcolor}%
\pgftext[x=2.217187in,y=2.786111in,,base]{\color{textcolor}\sffamily\fontsize{12.000000}{14.400000}\selectfont \(\displaystyle  N = 64 \)}%
\end{pgfscope}%
\begin{pgfscope}%
\pgfsetbuttcap%
\pgfsetmiterjoin%
\definecolor{currentfill}{rgb}{1.000000,1.000000,1.000000}%
\pgfsetfillcolor{currentfill}%
\pgfsetlinewidth{0.000000pt}%
\definecolor{currentstroke}{rgb}{0.000000,0.000000,0.000000}%
\pgfsetstrokecolor{currentstroke}%
\pgfsetstrokeopacity{0.000000}%
\pgfsetdash{}{0pt}%
\pgfpathmoveto{\pgfqpoint{4.482986in}{0.438889in}}%
\pgfpathlineto{\pgfqpoint{7.801389in}{0.438889in}}%
\pgfpathlineto{\pgfqpoint{7.801389in}{2.702778in}}%
\pgfpathlineto{\pgfqpoint{4.482986in}{2.702778in}}%
\pgfpathclose%
\pgfusepath{fill}%
\end{pgfscope}%
\begin{pgfscope}%
\pgfpathrectangle{\pgfqpoint{4.482986in}{0.438889in}}{\pgfqpoint{3.318403in}{2.263889in}}%
\pgfusepath{clip}%
\pgfsetbuttcap%
\pgfsetroundjoin%
\definecolor{currentfill}{rgb}{0.121569,0.466667,0.705882}%
\pgfsetfillcolor{currentfill}%
\pgfsetlinewidth{1.003750pt}%
\definecolor{currentstroke}{rgb}{0.121569,0.466667,0.705882}%
\pgfsetstrokecolor{currentstroke}%
\pgfsetdash{}{0pt}%
\pgfsys@defobject{currentmarker}{\pgfqpoint{-0.009821in}{-0.009821in}}{\pgfqpoint{0.009821in}{0.009821in}}{%
\pgfpathmoveto{\pgfqpoint{0.000000in}{-0.009821in}}%
\pgfpathcurveto{\pgfqpoint{0.002605in}{-0.009821in}}{\pgfqpoint{0.005103in}{-0.008786in}}{\pgfqpoint{0.006944in}{-0.006944in}}%
\pgfpathcurveto{\pgfqpoint{0.008786in}{-0.005103in}}{\pgfqpoint{0.009821in}{-0.002605in}}{\pgfqpoint{0.009821in}{0.000000in}}%
\pgfpathcurveto{\pgfqpoint{0.009821in}{0.002605in}}{\pgfqpoint{0.008786in}{0.005103in}}{\pgfqpoint{0.006944in}{0.006944in}}%
\pgfpathcurveto{\pgfqpoint{0.005103in}{0.008786in}}{\pgfqpoint{0.002605in}{0.009821in}}{\pgfqpoint{0.000000in}{0.009821in}}%
\pgfpathcurveto{\pgfqpoint{-0.002605in}{0.009821in}}{\pgfqpoint{-0.005103in}{0.008786in}}{\pgfqpoint{-0.006944in}{0.006944in}}%
\pgfpathcurveto{\pgfqpoint{-0.008786in}{0.005103in}}{\pgfqpoint{-0.009821in}{0.002605in}}{\pgfqpoint{-0.009821in}{0.000000in}}%
\pgfpathcurveto{\pgfqpoint{-0.009821in}{-0.002605in}}{\pgfqpoint{-0.008786in}{-0.005103in}}{\pgfqpoint{-0.006944in}{-0.006944in}}%
\pgfpathcurveto{\pgfqpoint{-0.005103in}{-0.008786in}}{\pgfqpoint{-0.002605in}{-0.009821in}}{\pgfqpoint{0.000000in}{-0.009821in}}%
\pgfpathclose%
\pgfusepath{stroke,fill}%
}%
\begin{pgfscope}%
\pgfsys@transformshift{4.633823in}{0.764749in}%
\pgfsys@useobject{currentmarker}{}%
\end{pgfscope}%
\begin{pgfscope}%
\pgfsys@transformshift{4.784609in}{0.767561in}%
\pgfsys@useobject{currentmarker}{}%
\end{pgfscope}%
\begin{pgfscope}%
\pgfsys@transformshift{4.935395in}{0.756919in}%
\pgfsys@useobject{currentmarker}{}%
\end{pgfscope}%
\begin{pgfscope}%
\pgfsys@transformshift{5.086181in}{0.759788in}%
\pgfsys@useobject{currentmarker}{}%
\end{pgfscope}%
\begin{pgfscope}%
\pgfsys@transformshift{5.236967in}{0.768130in}%
\pgfsys@useobject{currentmarker}{}%
\end{pgfscope}%
\begin{pgfscope}%
\pgfsys@transformshift{5.387753in}{0.786866in}%
\pgfsys@useobject{currentmarker}{}%
\end{pgfscope}%
\begin{pgfscope}%
\pgfsys@transformshift{5.538539in}{0.792462in}%
\pgfsys@useobject{currentmarker}{}%
\end{pgfscope}%
\begin{pgfscope}%
\pgfsys@transformshift{5.689325in}{0.792702in}%
\pgfsys@useobject{currentmarker}{}%
\end{pgfscope}%
\begin{pgfscope}%
\pgfsys@transformshift{5.840111in}{0.835332in}%
\pgfsys@useobject{currentmarker}{}%
\end{pgfscope}%
\begin{pgfscope}%
\pgfsys@transformshift{5.990897in}{0.786811in}%
\pgfsys@useobject{currentmarker}{}%
\end{pgfscope}%
\begin{pgfscope}%
\pgfsys@transformshift{6.141683in}{0.804543in}%
\pgfsys@useobject{currentmarker}{}%
\end{pgfscope}%
\begin{pgfscope}%
\pgfsys@transformshift{6.292469in}{0.828774in}%
\pgfsys@useobject{currentmarker}{}%
\end{pgfscope}%
\begin{pgfscope}%
\pgfsys@transformshift{6.443255in}{0.864559in}%
\pgfsys@useobject{currentmarker}{}%
\end{pgfscope}%
\begin{pgfscope}%
\pgfsys@transformshift{6.594041in}{0.851787in}%
\pgfsys@useobject{currentmarker}{}%
\end{pgfscope}%
\begin{pgfscope}%
\pgfsys@transformshift{6.744827in}{1.088990in}%
\pgfsys@useobject{currentmarker}{}%
\end{pgfscope}%
\begin{pgfscope}%
\pgfsys@transformshift{6.895613in}{1.055189in}%
\pgfsys@useobject{currentmarker}{}%
\end{pgfscope}%
\begin{pgfscope}%
\pgfsys@transformshift{7.046399in}{1.487584in}%
\pgfsys@useobject{currentmarker}{}%
\end{pgfscope}%
\begin{pgfscope}%
\pgfsys@transformshift{7.197185in}{1.634492in}%
\pgfsys@useobject{currentmarker}{}%
\end{pgfscope}%
\begin{pgfscope}%
\pgfsys@transformshift{7.347971in}{0.933548in}%
\pgfsys@useobject{currentmarker}{}%
\end{pgfscope}%
\begin{pgfscope}%
\pgfsys@transformshift{7.498756in}{1.544487in}%
\pgfsys@useobject{currentmarker}{}%
\end{pgfscope}%
\begin{pgfscope}%
\pgfsys@transformshift{7.649542in}{1.363508in}%
\pgfsys@useobject{currentmarker}{}%
\end{pgfscope}%
\end{pgfscope}%
\begin{pgfscope}%
\pgfpathrectangle{\pgfqpoint{4.482986in}{0.438889in}}{\pgfqpoint{3.318403in}{2.263889in}}%
\pgfusepath{clip}%
\pgfsetbuttcap%
\pgfsetroundjoin%
\definecolor{currentfill}{rgb}{0.121569,0.466667,0.705882}%
\pgfsetfillcolor{currentfill}%
\pgfsetfillopacity{0.300000}%
\pgfsetlinewidth{1.003750pt}%
\definecolor{currentstroke}{rgb}{0.121569,0.466667,0.705882}%
\pgfsetstrokecolor{currentstroke}%
\pgfsetstrokeopacity{0.300000}%
\pgfsetdash{}{0pt}%
\pgfpathmoveto{\pgfqpoint{4.633823in}{0.774420in}}%
\pgfpathlineto{\pgfqpoint{4.633823in}{0.755078in}}%
\pgfpathlineto{\pgfqpoint{4.784609in}{0.759195in}}%
\pgfpathlineto{\pgfqpoint{4.935395in}{0.750956in}}%
\pgfpathlineto{\pgfqpoint{5.086181in}{0.748972in}}%
\pgfpathlineto{\pgfqpoint{5.236967in}{0.754222in}}%
\pgfpathlineto{\pgfqpoint{5.387753in}{0.757155in}}%
\pgfpathlineto{\pgfqpoint{5.538539in}{0.764500in}}%
\pgfpathlineto{\pgfqpoint{5.689325in}{0.756154in}}%
\pgfpathlineto{\pgfqpoint{5.840111in}{0.811698in}}%
\pgfpathlineto{\pgfqpoint{5.990897in}{0.739408in}}%
\pgfpathlineto{\pgfqpoint{6.141683in}{0.739665in}}%
\pgfpathlineto{\pgfqpoint{6.292469in}{0.772666in}}%
\pgfpathlineto{\pgfqpoint{6.443255in}{0.740392in}}%
\pgfpathlineto{\pgfqpoint{6.594041in}{0.773335in}}%
\pgfpathlineto{\pgfqpoint{6.744827in}{0.752282in}}%
\pgfpathlineto{\pgfqpoint{6.895613in}{0.791346in}}%
\pgfpathlineto{\pgfqpoint{7.046399in}{0.702963in}}%
\pgfpathlineto{\pgfqpoint{7.197185in}{0.983051in}}%
\pgfpathlineto{\pgfqpoint{7.347971in}{0.764210in}}%
\pgfpathlineto{\pgfqpoint{7.498756in}{0.900461in}}%
\pgfpathlineto{\pgfqpoint{7.649542in}{0.569181in}}%
\pgfpathlineto{\pgfqpoint{7.649542in}{2.157835in}}%
\pgfpathlineto{\pgfqpoint{7.649542in}{2.157835in}}%
\pgfpathlineto{\pgfqpoint{7.498756in}{2.188512in}}%
\pgfpathlineto{\pgfqpoint{7.347971in}{1.102886in}}%
\pgfpathlineto{\pgfqpoint{7.197185in}{2.285934in}}%
\pgfpathlineto{\pgfqpoint{7.046399in}{2.272205in}}%
\pgfpathlineto{\pgfqpoint{6.895613in}{1.319032in}}%
\pgfpathlineto{\pgfqpoint{6.744827in}{1.425698in}}%
\pgfpathlineto{\pgfqpoint{6.594041in}{0.930239in}}%
\pgfpathlineto{\pgfqpoint{6.443255in}{0.988726in}}%
\pgfpathlineto{\pgfqpoint{6.292469in}{0.884882in}}%
\pgfpathlineto{\pgfqpoint{6.141683in}{0.869421in}}%
\pgfpathlineto{\pgfqpoint{5.990897in}{0.834214in}}%
\pgfpathlineto{\pgfqpoint{5.840111in}{0.858966in}}%
\pgfpathlineto{\pgfqpoint{5.689325in}{0.829250in}}%
\pgfpathlineto{\pgfqpoint{5.538539in}{0.820425in}}%
\pgfpathlineto{\pgfqpoint{5.387753in}{0.816576in}}%
\pgfpathlineto{\pgfqpoint{5.236967in}{0.782038in}}%
\pgfpathlineto{\pgfqpoint{5.086181in}{0.770604in}}%
\pgfpathlineto{\pgfqpoint{4.935395in}{0.762881in}}%
\pgfpathlineto{\pgfqpoint{4.784609in}{0.775926in}}%
\pgfpathlineto{\pgfqpoint{4.633823in}{0.774420in}}%
\pgfpathclose%
\pgfusepath{stroke,fill}%
\end{pgfscope}%
\begin{pgfscope}%
\pgfpathrectangle{\pgfqpoint{4.482986in}{0.438889in}}{\pgfqpoint{3.318403in}{2.263889in}}%
\pgfusepath{clip}%
\pgfsetbuttcap%
\pgfsetroundjoin%
\definecolor{currentfill}{rgb}{1.000000,0.498039,0.054902}%
\pgfsetfillcolor{currentfill}%
\pgfsetlinewidth{1.003750pt}%
\definecolor{currentstroke}{rgb}{1.000000,0.498039,0.054902}%
\pgfsetstrokecolor{currentstroke}%
\pgfsetdash{}{0pt}%
\pgfsys@defobject{currentmarker}{\pgfqpoint{-0.009821in}{-0.009821in}}{\pgfqpoint{0.009821in}{0.009821in}}{%
\pgfpathmoveto{\pgfqpoint{0.000000in}{-0.009821in}}%
\pgfpathcurveto{\pgfqpoint{0.002605in}{-0.009821in}}{\pgfqpoint{0.005103in}{-0.008786in}}{\pgfqpoint{0.006944in}{-0.006944in}}%
\pgfpathcurveto{\pgfqpoint{0.008786in}{-0.005103in}}{\pgfqpoint{0.009821in}{-0.002605in}}{\pgfqpoint{0.009821in}{0.000000in}}%
\pgfpathcurveto{\pgfqpoint{0.009821in}{0.002605in}}{\pgfqpoint{0.008786in}{0.005103in}}{\pgfqpoint{0.006944in}{0.006944in}}%
\pgfpathcurveto{\pgfqpoint{0.005103in}{0.008786in}}{\pgfqpoint{0.002605in}{0.009821in}}{\pgfqpoint{0.000000in}{0.009821in}}%
\pgfpathcurveto{\pgfqpoint{-0.002605in}{0.009821in}}{\pgfqpoint{-0.005103in}{0.008786in}}{\pgfqpoint{-0.006944in}{0.006944in}}%
\pgfpathcurveto{\pgfqpoint{-0.008786in}{0.005103in}}{\pgfqpoint{-0.009821in}{0.002605in}}{\pgfqpoint{-0.009821in}{0.000000in}}%
\pgfpathcurveto{\pgfqpoint{-0.009821in}{-0.002605in}}{\pgfqpoint{-0.008786in}{-0.005103in}}{\pgfqpoint{-0.006944in}{-0.006944in}}%
\pgfpathcurveto{\pgfqpoint{-0.005103in}{-0.008786in}}{\pgfqpoint{-0.002605in}{-0.009821in}}{\pgfqpoint{0.000000in}{-0.009821in}}%
\pgfpathclose%
\pgfusepath{stroke,fill}%
}%
\begin{pgfscope}%
\pgfsys@transformshift{4.633823in}{0.761548in}%
\pgfsys@useobject{currentmarker}{}%
\end{pgfscope}%
\begin{pgfscope}%
\pgfsys@transformshift{4.784609in}{0.765046in}%
\pgfsys@useobject{currentmarker}{}%
\end{pgfscope}%
\begin{pgfscope}%
\pgfsys@transformshift{4.935395in}{0.759543in}%
\pgfsys@useobject{currentmarker}{}%
\end{pgfscope}%
\begin{pgfscope}%
\pgfsys@transformshift{5.086181in}{0.755540in}%
\pgfsys@useobject{currentmarker}{}%
\end{pgfscope}%
\begin{pgfscope}%
\pgfsys@transformshift{5.236967in}{0.763161in}%
\pgfsys@useobject{currentmarker}{}%
\end{pgfscope}%
\begin{pgfscope}%
\pgfsys@transformshift{5.387753in}{0.783863in}%
\pgfsys@useobject{currentmarker}{}%
\end{pgfscope}%
\begin{pgfscope}%
\pgfsys@transformshift{5.538539in}{0.763983in}%
\pgfsys@useobject{currentmarker}{}%
\end{pgfscope}%
\begin{pgfscope}%
\pgfsys@transformshift{5.689325in}{0.776281in}%
\pgfsys@useobject{currentmarker}{}%
\end{pgfscope}%
\begin{pgfscope}%
\pgfsys@transformshift{5.840111in}{0.786858in}%
\pgfsys@useobject{currentmarker}{}%
\end{pgfscope}%
\begin{pgfscope}%
\pgfsys@transformshift{5.990897in}{0.820554in}%
\pgfsys@useobject{currentmarker}{}%
\end{pgfscope}%
\begin{pgfscope}%
\pgfsys@transformshift{6.141683in}{0.771508in}%
\pgfsys@useobject{currentmarker}{}%
\end{pgfscope}%
\begin{pgfscope}%
\pgfsys@transformshift{6.292469in}{0.912148in}%
\pgfsys@useobject{currentmarker}{}%
\end{pgfscope}%
\begin{pgfscope}%
\pgfsys@transformshift{6.443255in}{0.856542in}%
\pgfsys@useobject{currentmarker}{}%
\end{pgfscope}%
\begin{pgfscope}%
\pgfsys@transformshift{6.594041in}{0.916649in}%
\pgfsys@useobject{currentmarker}{}%
\end{pgfscope}%
\begin{pgfscope}%
\pgfsys@transformshift{6.744827in}{1.040257in}%
\pgfsys@useobject{currentmarker}{}%
\end{pgfscope}%
\begin{pgfscope}%
\pgfsys@transformshift{6.895613in}{1.227234in}%
\pgfsys@useobject{currentmarker}{}%
\end{pgfscope}%
\begin{pgfscope}%
\pgfsys@transformshift{7.046399in}{1.061679in}%
\pgfsys@useobject{currentmarker}{}%
\end{pgfscope}%
\begin{pgfscope}%
\pgfsys@transformshift{7.197185in}{1.622452in}%
\pgfsys@useobject{currentmarker}{}%
\end{pgfscope}%
\begin{pgfscope}%
\pgfsys@transformshift{7.347971in}{1.756146in}%
\pgfsys@useobject{currentmarker}{}%
\end{pgfscope}%
\begin{pgfscope}%
\pgfsys@transformshift{7.498756in}{1.565727in}%
\pgfsys@useobject{currentmarker}{}%
\end{pgfscope}%
\begin{pgfscope}%
\pgfsys@transformshift{7.649542in}{1.672714in}%
\pgfsys@useobject{currentmarker}{}%
\end{pgfscope}%
\end{pgfscope}%
\begin{pgfscope}%
\pgfpathrectangle{\pgfqpoint{4.482986in}{0.438889in}}{\pgfqpoint{3.318403in}{2.263889in}}%
\pgfusepath{clip}%
\pgfsetbuttcap%
\pgfsetroundjoin%
\definecolor{currentfill}{rgb}{1.000000,0.498039,0.054902}%
\pgfsetfillcolor{currentfill}%
\pgfsetfillopacity{0.300000}%
\pgfsetlinewidth{1.003750pt}%
\definecolor{currentstroke}{rgb}{1.000000,0.498039,0.054902}%
\pgfsetstrokecolor{currentstroke}%
\pgfsetstrokeopacity{0.300000}%
\pgfsetdash{}{0pt}%
\pgfpathmoveto{\pgfqpoint{4.633823in}{0.772767in}}%
\pgfpathlineto{\pgfqpoint{4.633823in}{0.750329in}}%
\pgfpathlineto{\pgfqpoint{4.784609in}{0.755327in}}%
\pgfpathlineto{\pgfqpoint{4.935395in}{0.745210in}}%
\pgfpathlineto{\pgfqpoint{5.086181in}{0.747656in}}%
\pgfpathlineto{\pgfqpoint{5.236967in}{0.758111in}}%
\pgfpathlineto{\pgfqpoint{5.387753in}{0.769071in}}%
\pgfpathlineto{\pgfqpoint{5.538539in}{0.754717in}}%
\pgfpathlineto{\pgfqpoint{5.689325in}{0.750667in}}%
\pgfpathlineto{\pgfqpoint{5.840111in}{0.756280in}}%
\pgfpathlineto{\pgfqpoint{5.990897in}{0.762543in}}%
\pgfpathlineto{\pgfqpoint{6.141683in}{0.754769in}}%
\pgfpathlineto{\pgfqpoint{6.292469in}{0.779004in}}%
\pgfpathlineto{\pgfqpoint{6.443255in}{0.758541in}}%
\pgfpathlineto{\pgfqpoint{6.594041in}{0.705387in}}%
\pgfpathlineto{\pgfqpoint{6.744827in}{0.744497in}}%
\pgfpathlineto{\pgfqpoint{6.895613in}{1.011439in}}%
\pgfpathlineto{\pgfqpoint{7.046399in}{0.607297in}}%
\pgfpathlineto{\pgfqpoint{7.197185in}{0.792538in}}%
\pgfpathlineto{\pgfqpoint{7.347971in}{0.912418in}}%
\pgfpathlineto{\pgfqpoint{7.498756in}{0.541793in}}%
\pgfpathlineto{\pgfqpoint{7.649542in}{0.783542in}}%
\pgfpathlineto{\pgfqpoint{7.649542in}{2.561887in}}%
\pgfpathlineto{\pgfqpoint{7.649542in}{2.561887in}}%
\pgfpathlineto{\pgfqpoint{7.498756in}{2.589661in}}%
\pgfpathlineto{\pgfqpoint{7.347971in}{2.599874in}}%
\pgfpathlineto{\pgfqpoint{7.197185in}{2.452366in}}%
\pgfpathlineto{\pgfqpoint{7.046399in}{1.516062in}}%
\pgfpathlineto{\pgfqpoint{6.895613in}{1.443029in}}%
\pgfpathlineto{\pgfqpoint{6.744827in}{1.336016in}}%
\pgfpathlineto{\pgfqpoint{6.594041in}{1.127911in}}%
\pgfpathlineto{\pgfqpoint{6.443255in}{0.954544in}}%
\pgfpathlineto{\pgfqpoint{6.292469in}{1.045293in}}%
\pgfpathlineto{\pgfqpoint{6.141683in}{0.788246in}}%
\pgfpathlineto{\pgfqpoint{5.990897in}{0.878566in}}%
\pgfpathlineto{\pgfqpoint{5.840111in}{0.817436in}}%
\pgfpathlineto{\pgfqpoint{5.689325in}{0.801895in}}%
\pgfpathlineto{\pgfqpoint{5.538539in}{0.773248in}}%
\pgfpathlineto{\pgfqpoint{5.387753in}{0.798656in}}%
\pgfpathlineto{\pgfqpoint{5.236967in}{0.768212in}}%
\pgfpathlineto{\pgfqpoint{5.086181in}{0.763424in}}%
\pgfpathlineto{\pgfqpoint{4.935395in}{0.773875in}}%
\pgfpathlineto{\pgfqpoint{4.784609in}{0.774764in}}%
\pgfpathlineto{\pgfqpoint{4.633823in}{0.772767in}}%
\pgfpathclose%
\pgfusepath{stroke,fill}%
\end{pgfscope}%
\begin{pgfscope}%
\pgfpathrectangle{\pgfqpoint{4.482986in}{0.438889in}}{\pgfqpoint{3.318403in}{2.263889in}}%
\pgfusepath{clip}%
\pgfsetbuttcap%
\pgfsetroundjoin%
\definecolor{currentfill}{rgb}{0.172549,0.627451,0.172549}%
\pgfsetfillcolor{currentfill}%
\pgfsetlinewidth{1.003750pt}%
\definecolor{currentstroke}{rgb}{0.172549,0.627451,0.172549}%
\pgfsetstrokecolor{currentstroke}%
\pgfsetdash{}{0pt}%
\pgfsys@defobject{currentmarker}{\pgfqpoint{-0.009821in}{-0.009821in}}{\pgfqpoint{0.009821in}{0.009821in}}{%
\pgfpathmoveto{\pgfqpoint{0.000000in}{-0.009821in}}%
\pgfpathcurveto{\pgfqpoint{0.002605in}{-0.009821in}}{\pgfqpoint{0.005103in}{-0.008786in}}{\pgfqpoint{0.006944in}{-0.006944in}}%
\pgfpathcurveto{\pgfqpoint{0.008786in}{-0.005103in}}{\pgfqpoint{0.009821in}{-0.002605in}}{\pgfqpoint{0.009821in}{0.000000in}}%
\pgfpathcurveto{\pgfqpoint{0.009821in}{0.002605in}}{\pgfqpoint{0.008786in}{0.005103in}}{\pgfqpoint{0.006944in}{0.006944in}}%
\pgfpathcurveto{\pgfqpoint{0.005103in}{0.008786in}}{\pgfqpoint{0.002605in}{0.009821in}}{\pgfqpoint{0.000000in}{0.009821in}}%
\pgfpathcurveto{\pgfqpoint{-0.002605in}{0.009821in}}{\pgfqpoint{-0.005103in}{0.008786in}}{\pgfqpoint{-0.006944in}{0.006944in}}%
\pgfpathcurveto{\pgfqpoint{-0.008786in}{0.005103in}}{\pgfqpoint{-0.009821in}{0.002605in}}{\pgfqpoint{-0.009821in}{0.000000in}}%
\pgfpathcurveto{\pgfqpoint{-0.009821in}{-0.002605in}}{\pgfqpoint{-0.008786in}{-0.005103in}}{\pgfqpoint{-0.006944in}{-0.006944in}}%
\pgfpathcurveto{\pgfqpoint{-0.005103in}{-0.008786in}}{\pgfqpoint{-0.002605in}{-0.009821in}}{\pgfqpoint{0.000000in}{-0.009821in}}%
\pgfpathclose%
\pgfusepath{stroke,fill}%
}%
\begin{pgfscope}%
\pgfsys@transformshift{4.633823in}{0.757559in}%
\pgfsys@useobject{currentmarker}{}%
\end{pgfscope}%
\begin{pgfscope}%
\pgfsys@transformshift{4.784609in}{0.762641in}%
\pgfsys@useobject{currentmarker}{}%
\end{pgfscope}%
\begin{pgfscope}%
\pgfsys@transformshift{4.935395in}{0.754460in}%
\pgfsys@useobject{currentmarker}{}%
\end{pgfscope}%
\begin{pgfscope}%
\pgfsys@transformshift{5.086181in}{0.770158in}%
\pgfsys@useobject{currentmarker}{}%
\end{pgfscope}%
\begin{pgfscope}%
\pgfsys@transformshift{5.236967in}{0.769649in}%
\pgfsys@useobject{currentmarker}{}%
\end{pgfscope}%
\begin{pgfscope}%
\pgfsys@transformshift{5.387753in}{0.765221in}%
\pgfsys@useobject{currentmarker}{}%
\end{pgfscope}%
\begin{pgfscope}%
\pgfsys@transformshift{5.538539in}{0.786852in}%
\pgfsys@useobject{currentmarker}{}%
\end{pgfscope}%
\begin{pgfscope}%
\pgfsys@transformshift{5.689325in}{0.773442in}%
\pgfsys@useobject{currentmarker}{}%
\end{pgfscope}%
\begin{pgfscope}%
\pgfsys@transformshift{5.840111in}{0.857947in}%
\pgfsys@useobject{currentmarker}{}%
\end{pgfscope}%
\begin{pgfscope}%
\pgfsys@transformshift{5.990897in}{0.789164in}%
\pgfsys@useobject{currentmarker}{}%
\end{pgfscope}%
\begin{pgfscope}%
\pgfsys@transformshift{6.141683in}{0.799222in}%
\pgfsys@useobject{currentmarker}{}%
\end{pgfscope}%
\begin{pgfscope}%
\pgfsys@transformshift{6.292469in}{0.911399in}%
\pgfsys@useobject{currentmarker}{}%
\end{pgfscope}%
\begin{pgfscope}%
\pgfsys@transformshift{6.443255in}{1.002522in}%
\pgfsys@useobject{currentmarker}{}%
\end{pgfscope}%
\begin{pgfscope}%
\pgfsys@transformshift{6.594041in}{1.152909in}%
\pgfsys@useobject{currentmarker}{}%
\end{pgfscope}%
\begin{pgfscope}%
\pgfsys@transformshift{6.744827in}{1.142144in}%
\pgfsys@useobject{currentmarker}{}%
\end{pgfscope}%
\begin{pgfscope}%
\pgfsys@transformshift{6.895613in}{0.841196in}%
\pgfsys@useobject{currentmarker}{}%
\end{pgfscope}%
\begin{pgfscope}%
\pgfsys@transformshift{7.046399in}{1.382422in}%
\pgfsys@useobject{currentmarker}{}%
\end{pgfscope}%
\begin{pgfscope}%
\pgfsys@transformshift{7.197185in}{1.793381in}%
\pgfsys@useobject{currentmarker}{}%
\end{pgfscope}%
\begin{pgfscope}%
\pgfsys@transformshift{7.347971in}{1.474619in}%
\pgfsys@useobject{currentmarker}{}%
\end{pgfscope}%
\begin{pgfscope}%
\pgfsys@transformshift{7.498756in}{2.110144in}%
\pgfsys@useobject{currentmarker}{}%
\end{pgfscope}%
\begin{pgfscope}%
\pgfsys@transformshift{7.649542in}{1.870352in}%
\pgfsys@useobject{currentmarker}{}%
\end{pgfscope}%
\end{pgfscope}%
\begin{pgfscope}%
\pgfpathrectangle{\pgfqpoint{4.482986in}{0.438889in}}{\pgfqpoint{3.318403in}{2.263889in}}%
\pgfusepath{clip}%
\pgfsetbuttcap%
\pgfsetroundjoin%
\definecolor{currentfill}{rgb}{0.172549,0.627451,0.172549}%
\pgfsetfillcolor{currentfill}%
\pgfsetfillopacity{0.300000}%
\pgfsetlinewidth{1.003750pt}%
\definecolor{currentstroke}{rgb}{0.172549,0.627451,0.172549}%
\pgfsetstrokecolor{currentstroke}%
\pgfsetstrokeopacity{0.300000}%
\pgfsetdash{}{0pt}%
\pgfpathmoveto{\pgfqpoint{4.633823in}{0.773351in}}%
\pgfpathlineto{\pgfqpoint{4.633823in}{0.741767in}}%
\pgfpathlineto{\pgfqpoint{4.784609in}{0.747902in}}%
\pgfpathlineto{\pgfqpoint{4.935395in}{0.751095in}}%
\pgfpathlineto{\pgfqpoint{5.086181in}{0.752537in}}%
\pgfpathlineto{\pgfqpoint{5.236967in}{0.745467in}}%
\pgfpathlineto{\pgfqpoint{5.387753in}{0.754114in}}%
\pgfpathlineto{\pgfqpoint{5.538539in}{0.750579in}}%
\pgfpathlineto{\pgfqpoint{5.689325in}{0.758323in}}%
\pgfpathlineto{\pgfqpoint{5.840111in}{0.788052in}}%
\pgfpathlineto{\pgfqpoint{5.990897in}{0.738447in}}%
\pgfpathlineto{\pgfqpoint{6.141683in}{0.715985in}}%
\pgfpathlineto{\pgfqpoint{6.292469in}{0.870381in}}%
\pgfpathlineto{\pgfqpoint{6.443255in}{0.752686in}}%
\pgfpathlineto{\pgfqpoint{6.594041in}{0.874066in}}%
\pgfpathlineto{\pgfqpoint{6.744827in}{0.673743in}}%
\pgfpathlineto{\pgfqpoint{6.895613in}{0.769416in}}%
\pgfpathlineto{\pgfqpoint{7.046399in}{0.654641in}}%
\pgfpathlineto{\pgfqpoint{7.197185in}{1.101894in}}%
\pgfpathlineto{\pgfqpoint{7.347971in}{0.835487in}}%
\pgfpathlineto{\pgfqpoint{7.498756in}{2.064803in}}%
\pgfpathlineto{\pgfqpoint{7.649542in}{1.155846in}}%
\pgfpathlineto{\pgfqpoint{7.649542in}{2.584858in}}%
\pgfpathlineto{\pgfqpoint{7.649542in}{2.584858in}}%
\pgfpathlineto{\pgfqpoint{7.498756in}{2.155484in}}%
\pgfpathlineto{\pgfqpoint{7.347971in}{2.113751in}}%
\pgfpathlineto{\pgfqpoint{7.197185in}{2.484868in}}%
\pgfpathlineto{\pgfqpoint{7.046399in}{2.110203in}}%
\pgfpathlineto{\pgfqpoint{6.895613in}{0.912975in}}%
\pgfpathlineto{\pgfqpoint{6.744827in}{1.610545in}}%
\pgfpathlineto{\pgfqpoint{6.594041in}{1.431753in}}%
\pgfpathlineto{\pgfqpoint{6.443255in}{1.252358in}}%
\pgfpathlineto{\pgfqpoint{6.292469in}{0.952418in}}%
\pgfpathlineto{\pgfqpoint{6.141683in}{0.882460in}}%
\pgfpathlineto{\pgfqpoint{5.990897in}{0.839881in}}%
\pgfpathlineto{\pgfqpoint{5.840111in}{0.927843in}}%
\pgfpathlineto{\pgfqpoint{5.689325in}{0.788562in}}%
\pgfpathlineto{\pgfqpoint{5.538539in}{0.823126in}}%
\pgfpathlineto{\pgfqpoint{5.387753in}{0.776328in}}%
\pgfpathlineto{\pgfqpoint{5.236967in}{0.793832in}}%
\pgfpathlineto{\pgfqpoint{5.086181in}{0.787779in}}%
\pgfpathlineto{\pgfqpoint{4.935395in}{0.757826in}}%
\pgfpathlineto{\pgfqpoint{4.784609in}{0.777380in}}%
\pgfpathlineto{\pgfqpoint{4.633823in}{0.773351in}}%
\pgfpathclose%
\pgfusepath{stroke,fill}%
\end{pgfscope}%
\begin{pgfscope}%
\pgfpathrectangle{\pgfqpoint{4.482986in}{0.438889in}}{\pgfqpoint{3.318403in}{2.263889in}}%
\pgfusepath{clip}%
\pgfsetbuttcap%
\pgfsetroundjoin%
\definecolor{currentfill}{rgb}{0.839216,0.152941,0.156863}%
\pgfsetfillcolor{currentfill}%
\pgfsetlinewidth{1.003750pt}%
\definecolor{currentstroke}{rgb}{0.839216,0.152941,0.156863}%
\pgfsetstrokecolor{currentstroke}%
\pgfsetdash{}{0pt}%
\pgfsys@defobject{currentmarker}{\pgfqpoint{-0.009821in}{-0.009821in}}{\pgfqpoint{0.009821in}{0.009821in}}{%
\pgfpathmoveto{\pgfqpoint{0.000000in}{-0.009821in}}%
\pgfpathcurveto{\pgfqpoint{0.002605in}{-0.009821in}}{\pgfqpoint{0.005103in}{-0.008786in}}{\pgfqpoint{0.006944in}{-0.006944in}}%
\pgfpathcurveto{\pgfqpoint{0.008786in}{-0.005103in}}{\pgfqpoint{0.009821in}{-0.002605in}}{\pgfqpoint{0.009821in}{0.000000in}}%
\pgfpathcurveto{\pgfqpoint{0.009821in}{0.002605in}}{\pgfqpoint{0.008786in}{0.005103in}}{\pgfqpoint{0.006944in}{0.006944in}}%
\pgfpathcurveto{\pgfqpoint{0.005103in}{0.008786in}}{\pgfqpoint{0.002605in}{0.009821in}}{\pgfqpoint{0.000000in}{0.009821in}}%
\pgfpathcurveto{\pgfqpoint{-0.002605in}{0.009821in}}{\pgfqpoint{-0.005103in}{0.008786in}}{\pgfqpoint{-0.006944in}{0.006944in}}%
\pgfpathcurveto{\pgfqpoint{-0.008786in}{0.005103in}}{\pgfqpoint{-0.009821in}{0.002605in}}{\pgfqpoint{-0.009821in}{0.000000in}}%
\pgfpathcurveto{\pgfqpoint{-0.009821in}{-0.002605in}}{\pgfqpoint{-0.008786in}{-0.005103in}}{\pgfqpoint{-0.006944in}{-0.006944in}}%
\pgfpathcurveto{\pgfqpoint{-0.005103in}{-0.008786in}}{\pgfqpoint{-0.002605in}{-0.009821in}}{\pgfqpoint{0.000000in}{-0.009821in}}%
\pgfpathclose%
\pgfusepath{stroke,fill}%
}%
\begin{pgfscope}%
\pgfsys@transformshift{4.633823in}{0.757423in}%
\pgfsys@useobject{currentmarker}{}%
\end{pgfscope}%
\begin{pgfscope}%
\pgfsys@transformshift{4.784609in}{0.758159in}%
\pgfsys@useobject{currentmarker}{}%
\end{pgfscope}%
\begin{pgfscope}%
\pgfsys@transformshift{4.935395in}{0.759199in}%
\pgfsys@useobject{currentmarker}{}%
\end{pgfscope}%
\begin{pgfscope}%
\pgfsys@transformshift{5.086181in}{0.766876in}%
\pgfsys@useobject{currentmarker}{}%
\end{pgfscope}%
\begin{pgfscope}%
\pgfsys@transformshift{5.236967in}{0.756045in}%
\pgfsys@useobject{currentmarker}{}%
\end{pgfscope}%
\begin{pgfscope}%
\pgfsys@transformshift{5.387753in}{0.768701in}%
\pgfsys@useobject{currentmarker}{}%
\end{pgfscope}%
\begin{pgfscope}%
\pgfsys@transformshift{5.538539in}{0.763696in}%
\pgfsys@useobject{currentmarker}{}%
\end{pgfscope}%
\begin{pgfscope}%
\pgfsys@transformshift{5.689325in}{0.773673in}%
\pgfsys@useobject{currentmarker}{}%
\end{pgfscope}%
\begin{pgfscope}%
\pgfsys@transformshift{5.840111in}{0.802583in}%
\pgfsys@useobject{currentmarker}{}%
\end{pgfscope}%
\begin{pgfscope}%
\pgfsys@transformshift{5.990897in}{0.803897in}%
\pgfsys@useobject{currentmarker}{}%
\end{pgfscope}%
\begin{pgfscope}%
\pgfsys@transformshift{6.141683in}{0.800905in}%
\pgfsys@useobject{currentmarker}{}%
\end{pgfscope}%
\begin{pgfscope}%
\pgfsys@transformshift{6.292469in}{0.784364in}%
\pgfsys@useobject{currentmarker}{}%
\end{pgfscope}%
\begin{pgfscope}%
\pgfsys@transformshift{6.443255in}{0.802315in}%
\pgfsys@useobject{currentmarker}{}%
\end{pgfscope}%
\begin{pgfscope}%
\pgfsys@transformshift{6.594041in}{0.814135in}%
\pgfsys@useobject{currentmarker}{}%
\end{pgfscope}%
\begin{pgfscope}%
\pgfsys@transformshift{6.744827in}{0.808637in}%
\pgfsys@useobject{currentmarker}{}%
\end{pgfscope}%
\begin{pgfscope}%
\pgfsys@transformshift{6.895613in}{0.841750in}%
\pgfsys@useobject{currentmarker}{}%
\end{pgfscope}%
\begin{pgfscope}%
\pgfsys@transformshift{7.046399in}{0.819765in}%
\pgfsys@useobject{currentmarker}{}%
\end{pgfscope}%
\begin{pgfscope}%
\pgfsys@transformshift{7.197185in}{0.831620in}%
\pgfsys@useobject{currentmarker}{}%
\end{pgfscope}%
\begin{pgfscope}%
\pgfsys@transformshift{7.347971in}{0.823799in}%
\pgfsys@useobject{currentmarker}{}%
\end{pgfscope}%
\begin{pgfscope}%
\pgfsys@transformshift{7.498756in}{0.825166in}%
\pgfsys@useobject{currentmarker}{}%
\end{pgfscope}%
\begin{pgfscope}%
\pgfsys@transformshift{7.649542in}{0.820728in}%
\pgfsys@useobject{currentmarker}{}%
\end{pgfscope}%
\end{pgfscope}%
\begin{pgfscope}%
\pgfpathrectangle{\pgfqpoint{4.482986in}{0.438889in}}{\pgfqpoint{3.318403in}{2.263889in}}%
\pgfusepath{clip}%
\pgfsetbuttcap%
\pgfsetroundjoin%
\definecolor{currentfill}{rgb}{0.839216,0.152941,0.156863}%
\pgfsetfillcolor{currentfill}%
\pgfsetfillopacity{0.300000}%
\pgfsetlinewidth{1.003750pt}%
\definecolor{currentstroke}{rgb}{0.839216,0.152941,0.156863}%
\pgfsetstrokecolor{currentstroke}%
\pgfsetstrokeopacity{0.300000}%
\pgfsetdash{}{0pt}%
\pgfpathmoveto{\pgfqpoint{4.633823in}{0.764049in}}%
\pgfpathlineto{\pgfqpoint{4.633823in}{0.750798in}}%
\pgfpathlineto{\pgfqpoint{4.784609in}{0.747831in}}%
\pgfpathlineto{\pgfqpoint{4.935395in}{0.755863in}}%
\pgfpathlineto{\pgfqpoint{5.086181in}{0.751642in}}%
\pgfpathlineto{\pgfqpoint{5.236967in}{0.750076in}}%
\pgfpathlineto{\pgfqpoint{5.387753in}{0.750186in}}%
\pgfpathlineto{\pgfqpoint{5.538539in}{0.751421in}}%
\pgfpathlineto{\pgfqpoint{5.689325in}{0.758961in}}%
\pgfpathlineto{\pgfqpoint{5.840111in}{0.751167in}}%
\pgfpathlineto{\pgfqpoint{5.990897in}{0.760037in}}%
\pgfpathlineto{\pgfqpoint{6.141683in}{0.746462in}}%
\pgfpathlineto{\pgfqpoint{6.292469in}{0.759034in}}%
\pgfpathlineto{\pgfqpoint{6.443255in}{0.766670in}}%
\pgfpathlineto{\pgfqpoint{6.594041in}{0.781331in}}%
\pgfpathlineto{\pgfqpoint{6.744827in}{0.789138in}}%
\pgfpathlineto{\pgfqpoint{6.895613in}{0.798782in}}%
\pgfpathlineto{\pgfqpoint{7.046399in}{0.800903in}}%
\pgfpathlineto{\pgfqpoint{7.197185in}{0.812031in}}%
\pgfpathlineto{\pgfqpoint{7.347971in}{0.820750in}}%
\pgfpathlineto{\pgfqpoint{7.498756in}{0.813710in}}%
\pgfpathlineto{\pgfqpoint{7.649542in}{0.818159in}}%
\pgfpathlineto{\pgfqpoint{7.649542in}{0.823297in}}%
\pgfpathlineto{\pgfqpoint{7.649542in}{0.823297in}}%
\pgfpathlineto{\pgfqpoint{7.498756in}{0.836623in}}%
\pgfpathlineto{\pgfqpoint{7.347971in}{0.826849in}}%
\pgfpathlineto{\pgfqpoint{7.197185in}{0.851210in}}%
\pgfpathlineto{\pgfqpoint{7.046399in}{0.838628in}}%
\pgfpathlineto{\pgfqpoint{6.895613in}{0.884718in}}%
\pgfpathlineto{\pgfqpoint{6.744827in}{0.828136in}}%
\pgfpathlineto{\pgfqpoint{6.594041in}{0.846939in}}%
\pgfpathlineto{\pgfqpoint{6.443255in}{0.837960in}}%
\pgfpathlineto{\pgfqpoint{6.292469in}{0.809693in}}%
\pgfpathlineto{\pgfqpoint{6.141683in}{0.855349in}}%
\pgfpathlineto{\pgfqpoint{5.990897in}{0.847757in}}%
\pgfpathlineto{\pgfqpoint{5.840111in}{0.853999in}}%
\pgfpathlineto{\pgfqpoint{5.689325in}{0.788384in}}%
\pgfpathlineto{\pgfqpoint{5.538539in}{0.775971in}}%
\pgfpathlineto{\pgfqpoint{5.387753in}{0.787216in}}%
\pgfpathlineto{\pgfqpoint{5.236967in}{0.762015in}}%
\pgfpathlineto{\pgfqpoint{5.086181in}{0.782110in}}%
\pgfpathlineto{\pgfqpoint{4.935395in}{0.762535in}}%
\pgfpathlineto{\pgfqpoint{4.784609in}{0.768487in}}%
\pgfpathlineto{\pgfqpoint{4.633823in}{0.764049in}}%
\pgfpathclose%
\pgfusepath{stroke,fill}%
\end{pgfscope}%
\begin{pgfscope}%
\pgfpathrectangle{\pgfqpoint{4.482986in}{0.438889in}}{\pgfqpoint{3.318403in}{2.263889in}}%
\pgfusepath{clip}%
\pgfsetbuttcap%
\pgfsetroundjoin%
\definecolor{currentfill}{rgb}{0.580392,0.403922,0.741176}%
\pgfsetfillcolor{currentfill}%
\pgfsetlinewidth{1.003750pt}%
\definecolor{currentstroke}{rgb}{0.580392,0.403922,0.741176}%
\pgfsetstrokecolor{currentstroke}%
\pgfsetdash{}{0pt}%
\pgfsys@defobject{currentmarker}{\pgfqpoint{-0.009821in}{-0.009821in}}{\pgfqpoint{0.009821in}{0.009821in}}{%
\pgfpathmoveto{\pgfqpoint{0.000000in}{-0.009821in}}%
\pgfpathcurveto{\pgfqpoint{0.002605in}{-0.009821in}}{\pgfqpoint{0.005103in}{-0.008786in}}{\pgfqpoint{0.006944in}{-0.006944in}}%
\pgfpathcurveto{\pgfqpoint{0.008786in}{-0.005103in}}{\pgfqpoint{0.009821in}{-0.002605in}}{\pgfqpoint{0.009821in}{0.000000in}}%
\pgfpathcurveto{\pgfqpoint{0.009821in}{0.002605in}}{\pgfqpoint{0.008786in}{0.005103in}}{\pgfqpoint{0.006944in}{0.006944in}}%
\pgfpathcurveto{\pgfqpoint{0.005103in}{0.008786in}}{\pgfqpoint{0.002605in}{0.009821in}}{\pgfqpoint{0.000000in}{0.009821in}}%
\pgfpathcurveto{\pgfqpoint{-0.002605in}{0.009821in}}{\pgfqpoint{-0.005103in}{0.008786in}}{\pgfqpoint{-0.006944in}{0.006944in}}%
\pgfpathcurveto{\pgfqpoint{-0.008786in}{0.005103in}}{\pgfqpoint{-0.009821in}{0.002605in}}{\pgfqpoint{-0.009821in}{0.000000in}}%
\pgfpathcurveto{\pgfqpoint{-0.009821in}{-0.002605in}}{\pgfqpoint{-0.008786in}{-0.005103in}}{\pgfqpoint{-0.006944in}{-0.006944in}}%
\pgfpathcurveto{\pgfqpoint{-0.005103in}{-0.008786in}}{\pgfqpoint{-0.002605in}{-0.009821in}}{\pgfqpoint{0.000000in}{-0.009821in}}%
\pgfpathclose%
\pgfusepath{stroke,fill}%
}%
\begin{pgfscope}%
\pgfsys@transformshift{4.633823in}{0.762455in}%
\pgfsys@useobject{currentmarker}{}%
\end{pgfscope}%
\begin{pgfscope}%
\pgfsys@transformshift{4.784609in}{0.759105in}%
\pgfsys@useobject{currentmarker}{}%
\end{pgfscope}%
\begin{pgfscope}%
\pgfsys@transformshift{4.935395in}{0.770185in}%
\pgfsys@useobject{currentmarker}{}%
\end{pgfscope}%
\begin{pgfscope}%
\pgfsys@transformshift{5.086181in}{0.759677in}%
\pgfsys@useobject{currentmarker}{}%
\end{pgfscope}%
\begin{pgfscope}%
\pgfsys@transformshift{5.236967in}{0.771885in}%
\pgfsys@useobject{currentmarker}{}%
\end{pgfscope}%
\begin{pgfscope}%
\pgfsys@transformshift{5.387753in}{0.768289in}%
\pgfsys@useobject{currentmarker}{}%
\end{pgfscope}%
\begin{pgfscope}%
\pgfsys@transformshift{5.538539in}{0.780615in}%
\pgfsys@useobject{currentmarker}{}%
\end{pgfscope}%
\begin{pgfscope}%
\pgfsys@transformshift{5.689325in}{0.786678in}%
\pgfsys@useobject{currentmarker}{}%
\end{pgfscope}%
\begin{pgfscope}%
\pgfsys@transformshift{5.840111in}{0.792690in}%
\pgfsys@useobject{currentmarker}{}%
\end{pgfscope}%
\begin{pgfscope}%
\pgfsys@transformshift{5.990897in}{0.770196in}%
\pgfsys@useobject{currentmarker}{}%
\end{pgfscope}%
\begin{pgfscope}%
\pgfsys@transformshift{6.141683in}{0.777113in}%
\pgfsys@useobject{currentmarker}{}%
\end{pgfscope}%
\begin{pgfscope}%
\pgfsys@transformshift{6.292469in}{0.774979in}%
\pgfsys@useobject{currentmarker}{}%
\end{pgfscope}%
\begin{pgfscope}%
\pgfsys@transformshift{6.443255in}{0.779010in}%
\pgfsys@useobject{currentmarker}{}%
\end{pgfscope}%
\begin{pgfscope}%
\pgfsys@transformshift{6.594041in}{0.781131in}%
\pgfsys@useobject{currentmarker}{}%
\end{pgfscope}%
\begin{pgfscope}%
\pgfsys@transformshift{6.744827in}{0.778297in}%
\pgfsys@useobject{currentmarker}{}%
\end{pgfscope}%
\begin{pgfscope}%
\pgfsys@transformshift{6.895613in}{0.777872in}%
\pgfsys@useobject{currentmarker}{}%
\end{pgfscope}%
\begin{pgfscope}%
\pgfsys@transformshift{7.046399in}{0.781578in}%
\pgfsys@useobject{currentmarker}{}%
\end{pgfscope}%
\begin{pgfscope}%
\pgfsys@transformshift{7.197185in}{0.780584in}%
\pgfsys@useobject{currentmarker}{}%
\end{pgfscope}%
\begin{pgfscope}%
\pgfsys@transformshift{7.347971in}{0.780324in}%
\pgfsys@useobject{currentmarker}{}%
\end{pgfscope}%
\begin{pgfscope}%
\pgfsys@transformshift{7.498756in}{0.780137in}%
\pgfsys@useobject{currentmarker}{}%
\end{pgfscope}%
\begin{pgfscope}%
\pgfsys@transformshift{7.649542in}{0.779765in}%
\pgfsys@useobject{currentmarker}{}%
\end{pgfscope}%
\end{pgfscope}%
\begin{pgfscope}%
\pgfpathrectangle{\pgfqpoint{4.482986in}{0.438889in}}{\pgfqpoint{3.318403in}{2.263889in}}%
\pgfusepath{clip}%
\pgfsetbuttcap%
\pgfsetroundjoin%
\definecolor{currentfill}{rgb}{0.580392,0.403922,0.741176}%
\pgfsetfillcolor{currentfill}%
\pgfsetfillopacity{0.300000}%
\pgfsetlinewidth{1.003750pt}%
\definecolor{currentstroke}{rgb}{0.580392,0.403922,0.741176}%
\pgfsetstrokecolor{currentstroke}%
\pgfsetstrokeopacity{0.300000}%
\pgfsetdash{}{0pt}%
\pgfpathmoveto{\pgfqpoint{4.633823in}{0.764615in}}%
\pgfpathlineto{\pgfqpoint{4.633823in}{0.760294in}}%
\pgfpathlineto{\pgfqpoint{4.784609in}{0.750271in}}%
\pgfpathlineto{\pgfqpoint{4.935395in}{0.746419in}}%
\pgfpathlineto{\pgfqpoint{5.086181in}{0.747172in}}%
\pgfpathlineto{\pgfqpoint{5.236967in}{0.756207in}}%
\pgfpathlineto{\pgfqpoint{5.387753in}{0.752301in}}%
\pgfpathlineto{\pgfqpoint{5.538539in}{0.746237in}}%
\pgfpathlineto{\pgfqpoint{5.689325in}{0.764218in}}%
\pgfpathlineto{\pgfqpoint{5.840111in}{0.757663in}}%
\pgfpathlineto{\pgfqpoint{5.990897in}{0.763530in}}%
\pgfpathlineto{\pgfqpoint{6.141683in}{0.758254in}}%
\pgfpathlineto{\pgfqpoint{6.292469in}{0.764674in}}%
\pgfpathlineto{\pgfqpoint{6.443255in}{0.767998in}}%
\pgfpathlineto{\pgfqpoint{6.594041in}{0.768310in}}%
\pgfpathlineto{\pgfqpoint{6.744827in}{0.770953in}}%
\pgfpathlineto{\pgfqpoint{6.895613in}{0.773385in}}%
\pgfpathlineto{\pgfqpoint{7.046399in}{0.777636in}}%
\pgfpathlineto{\pgfqpoint{7.197185in}{0.776734in}}%
\pgfpathlineto{\pgfqpoint{7.347971in}{0.779063in}}%
\pgfpathlineto{\pgfqpoint{7.498756in}{0.778250in}}%
\pgfpathlineto{\pgfqpoint{7.649542in}{0.778948in}}%
\pgfpathlineto{\pgfqpoint{7.649542in}{0.780582in}}%
\pgfpathlineto{\pgfqpoint{7.649542in}{0.780582in}}%
\pgfpathlineto{\pgfqpoint{7.498756in}{0.782024in}}%
\pgfpathlineto{\pgfqpoint{7.347971in}{0.781585in}}%
\pgfpathlineto{\pgfqpoint{7.197185in}{0.784434in}}%
\pgfpathlineto{\pgfqpoint{7.046399in}{0.785521in}}%
\pgfpathlineto{\pgfqpoint{6.895613in}{0.782359in}}%
\pgfpathlineto{\pgfqpoint{6.744827in}{0.785641in}}%
\pgfpathlineto{\pgfqpoint{6.594041in}{0.793952in}}%
\pgfpathlineto{\pgfqpoint{6.443255in}{0.790022in}}%
\pgfpathlineto{\pgfqpoint{6.292469in}{0.785285in}}%
\pgfpathlineto{\pgfqpoint{6.141683in}{0.795971in}}%
\pgfpathlineto{\pgfqpoint{5.990897in}{0.776863in}}%
\pgfpathlineto{\pgfqpoint{5.840111in}{0.827718in}}%
\pgfpathlineto{\pgfqpoint{5.689325in}{0.809138in}}%
\pgfpathlineto{\pgfqpoint{5.538539in}{0.814994in}}%
\pgfpathlineto{\pgfqpoint{5.387753in}{0.784276in}}%
\pgfpathlineto{\pgfqpoint{5.236967in}{0.787564in}}%
\pgfpathlineto{\pgfqpoint{5.086181in}{0.772181in}}%
\pgfpathlineto{\pgfqpoint{4.935395in}{0.793952in}}%
\pgfpathlineto{\pgfqpoint{4.784609in}{0.767938in}}%
\pgfpathlineto{\pgfqpoint{4.633823in}{0.764615in}}%
\pgfpathclose%
\pgfusepath{stroke,fill}%
\end{pgfscope}%
\begin{pgfscope}%
\pgfsetbuttcap%
\pgfsetroundjoin%
\definecolor{currentfill}{rgb}{0.000000,0.000000,0.000000}%
\pgfsetfillcolor{currentfill}%
\pgfsetlinewidth{0.803000pt}%
\definecolor{currentstroke}{rgb}{0.000000,0.000000,0.000000}%
\pgfsetstrokecolor{currentstroke}%
\pgfsetdash{}{0pt}%
\pgfsys@defobject{currentmarker}{\pgfqpoint{0.000000in}{-0.048611in}}{\pgfqpoint{0.000000in}{0.000000in}}{%
\pgfpathmoveto{\pgfqpoint{0.000000in}{0.000000in}}%
\pgfpathlineto{\pgfqpoint{0.000000in}{-0.048611in}}%
\pgfusepath{stroke,fill}%
}%
\begin{pgfscope}%
\pgfsys@transformshift{4.996951in}{0.438889in}%
\pgfsys@useobject{currentmarker}{}%
\end{pgfscope}%
\end{pgfscope}%
\begin{pgfscope}%
\definecolor{textcolor}{rgb}{0.000000,0.000000,0.000000}%
\pgfsetstrokecolor{textcolor}%
\pgfsetfillcolor{textcolor}%
\pgftext[x=4.996951in,y=0.341667in,,top]{\color{textcolor}\sffamily\fontsize{10.000000}{12.000000}\selectfont \(\displaystyle {10^{4}}\)}%
\end{pgfscope}%
\begin{pgfscope}%
\pgfsetbuttcap%
\pgfsetroundjoin%
\definecolor{currentfill}{rgb}{0.000000,0.000000,0.000000}%
\pgfsetfillcolor{currentfill}%
\pgfsetlinewidth{0.803000pt}%
\definecolor{currentstroke}{rgb}{0.000000,0.000000,0.000000}%
\pgfsetstrokecolor{currentstroke}%
\pgfsetdash{}{0pt}%
\pgfsys@defobject{currentmarker}{\pgfqpoint{0.000000in}{-0.048611in}}{\pgfqpoint{0.000000in}{0.000000in}}{%
\pgfpathmoveto{\pgfqpoint{0.000000in}{0.000000in}}%
\pgfpathlineto{\pgfqpoint{0.000000in}{-0.048611in}}%
\pgfusepath{stroke,fill}%
}%
\begin{pgfscope}%
\pgfsys@transformshift{5.600095in}{0.438889in}%
\pgfsys@useobject{currentmarker}{}%
\end{pgfscope}%
\end{pgfscope}%
\begin{pgfscope}%
\definecolor{textcolor}{rgb}{0.000000,0.000000,0.000000}%
\pgfsetstrokecolor{textcolor}%
\pgfsetfillcolor{textcolor}%
\pgftext[x=5.600095in,y=0.341667in,,top]{\color{textcolor}\sffamily\fontsize{10.000000}{12.000000}\selectfont \(\displaystyle {10^{5}}\)}%
\end{pgfscope}%
\begin{pgfscope}%
\pgfsetbuttcap%
\pgfsetroundjoin%
\definecolor{currentfill}{rgb}{0.000000,0.000000,0.000000}%
\pgfsetfillcolor{currentfill}%
\pgfsetlinewidth{0.803000pt}%
\definecolor{currentstroke}{rgb}{0.000000,0.000000,0.000000}%
\pgfsetstrokecolor{currentstroke}%
\pgfsetdash{}{0pt}%
\pgfsys@defobject{currentmarker}{\pgfqpoint{0.000000in}{-0.048611in}}{\pgfqpoint{0.000000in}{0.000000in}}{%
\pgfpathmoveto{\pgfqpoint{0.000000in}{0.000000in}}%
\pgfpathlineto{\pgfqpoint{0.000000in}{-0.048611in}}%
\pgfusepath{stroke,fill}%
}%
\begin{pgfscope}%
\pgfsys@transformshift{6.203239in}{0.438889in}%
\pgfsys@useobject{currentmarker}{}%
\end{pgfscope}%
\end{pgfscope}%
\begin{pgfscope}%
\definecolor{textcolor}{rgb}{0.000000,0.000000,0.000000}%
\pgfsetstrokecolor{textcolor}%
\pgfsetfillcolor{textcolor}%
\pgftext[x=6.203239in,y=0.341667in,,top]{\color{textcolor}\sffamily\fontsize{10.000000}{12.000000}\selectfont \(\displaystyle {10^{6}}\)}%
\end{pgfscope}%
\begin{pgfscope}%
\pgfsetbuttcap%
\pgfsetroundjoin%
\definecolor{currentfill}{rgb}{0.000000,0.000000,0.000000}%
\pgfsetfillcolor{currentfill}%
\pgfsetlinewidth{0.803000pt}%
\definecolor{currentstroke}{rgb}{0.000000,0.000000,0.000000}%
\pgfsetstrokecolor{currentstroke}%
\pgfsetdash{}{0pt}%
\pgfsys@defobject{currentmarker}{\pgfqpoint{0.000000in}{-0.048611in}}{\pgfqpoint{0.000000in}{0.000000in}}{%
\pgfpathmoveto{\pgfqpoint{0.000000in}{0.000000in}}%
\pgfpathlineto{\pgfqpoint{0.000000in}{-0.048611in}}%
\pgfusepath{stroke,fill}%
}%
\begin{pgfscope}%
\pgfsys@transformshift{6.806383in}{0.438889in}%
\pgfsys@useobject{currentmarker}{}%
\end{pgfscope}%
\end{pgfscope}%
\begin{pgfscope}%
\definecolor{textcolor}{rgb}{0.000000,0.000000,0.000000}%
\pgfsetstrokecolor{textcolor}%
\pgfsetfillcolor{textcolor}%
\pgftext[x=6.806383in,y=0.341667in,,top]{\color{textcolor}\sffamily\fontsize{10.000000}{12.000000}\selectfont \(\displaystyle {10^{7}}\)}%
\end{pgfscope}%
\begin{pgfscope}%
\pgfsetbuttcap%
\pgfsetroundjoin%
\definecolor{currentfill}{rgb}{0.000000,0.000000,0.000000}%
\pgfsetfillcolor{currentfill}%
\pgfsetlinewidth{0.803000pt}%
\definecolor{currentstroke}{rgb}{0.000000,0.000000,0.000000}%
\pgfsetstrokecolor{currentstroke}%
\pgfsetdash{}{0pt}%
\pgfsys@defobject{currentmarker}{\pgfqpoint{0.000000in}{-0.048611in}}{\pgfqpoint{0.000000in}{0.000000in}}{%
\pgfpathmoveto{\pgfqpoint{0.000000in}{0.000000in}}%
\pgfpathlineto{\pgfqpoint{0.000000in}{-0.048611in}}%
\pgfusepath{stroke,fill}%
}%
\begin{pgfscope}%
\pgfsys@transformshift{7.409527in}{0.438889in}%
\pgfsys@useobject{currentmarker}{}%
\end{pgfscope}%
\end{pgfscope}%
\begin{pgfscope}%
\definecolor{textcolor}{rgb}{0.000000,0.000000,0.000000}%
\pgfsetstrokecolor{textcolor}%
\pgfsetfillcolor{textcolor}%
\pgftext[x=7.409527in,y=0.341667in,,top]{\color{textcolor}\sffamily\fontsize{10.000000}{12.000000}\selectfont \(\displaystyle {10^{8}}\)}%
\end{pgfscope}%
\begin{pgfscope}%
\pgfsetbuttcap%
\pgfsetroundjoin%
\definecolor{currentfill}{rgb}{0.000000,0.000000,0.000000}%
\pgfsetfillcolor{currentfill}%
\pgfsetlinewidth{0.602250pt}%
\definecolor{currentstroke}{rgb}{0.000000,0.000000,0.000000}%
\pgfsetstrokecolor{currentstroke}%
\pgfsetdash{}{0pt}%
\pgfsys@defobject{currentmarker}{\pgfqpoint{0.000000in}{-0.027778in}}{\pgfqpoint{0.000000in}{0.000000in}}{%
\pgfpathmoveto{\pgfqpoint{0.000000in}{0.000000in}}%
\pgfpathlineto{\pgfqpoint{0.000000in}{-0.027778in}}%
\pgfusepath{stroke,fill}%
}%
\begin{pgfscope}%
\pgfsys@transformshift{4.575372in}{0.438889in}%
\pgfsys@useobject{currentmarker}{}%
\end{pgfscope}%
\end{pgfscope}%
\begin{pgfscope}%
\pgfsetbuttcap%
\pgfsetroundjoin%
\definecolor{currentfill}{rgb}{0.000000,0.000000,0.000000}%
\pgfsetfillcolor{currentfill}%
\pgfsetlinewidth{0.602250pt}%
\definecolor{currentstroke}{rgb}{0.000000,0.000000,0.000000}%
\pgfsetstrokecolor{currentstroke}%
\pgfsetdash{}{0pt}%
\pgfsys@defobject{currentmarker}{\pgfqpoint{0.000000in}{-0.027778in}}{\pgfqpoint{0.000000in}{0.000000in}}{%
\pgfpathmoveto{\pgfqpoint{0.000000in}{0.000000in}}%
\pgfpathlineto{\pgfqpoint{0.000000in}{-0.027778in}}%
\pgfusepath{stroke,fill}%
}%
\begin{pgfscope}%
\pgfsys@transformshift{4.681580in}{0.438889in}%
\pgfsys@useobject{currentmarker}{}%
\end{pgfscope}%
\end{pgfscope}%
\begin{pgfscope}%
\pgfsetbuttcap%
\pgfsetroundjoin%
\definecolor{currentfill}{rgb}{0.000000,0.000000,0.000000}%
\pgfsetfillcolor{currentfill}%
\pgfsetlinewidth{0.602250pt}%
\definecolor{currentstroke}{rgb}{0.000000,0.000000,0.000000}%
\pgfsetstrokecolor{currentstroke}%
\pgfsetdash{}{0pt}%
\pgfsys@defobject{currentmarker}{\pgfqpoint{0.000000in}{-0.027778in}}{\pgfqpoint{0.000000in}{0.000000in}}{%
\pgfpathmoveto{\pgfqpoint{0.000000in}{0.000000in}}%
\pgfpathlineto{\pgfqpoint{0.000000in}{-0.027778in}}%
\pgfusepath{stroke,fill}%
}%
\begin{pgfscope}%
\pgfsys@transformshift{4.756936in}{0.438889in}%
\pgfsys@useobject{currentmarker}{}%
\end{pgfscope}%
\end{pgfscope}%
\begin{pgfscope}%
\pgfsetbuttcap%
\pgfsetroundjoin%
\definecolor{currentfill}{rgb}{0.000000,0.000000,0.000000}%
\pgfsetfillcolor{currentfill}%
\pgfsetlinewidth{0.602250pt}%
\definecolor{currentstroke}{rgb}{0.000000,0.000000,0.000000}%
\pgfsetstrokecolor{currentstroke}%
\pgfsetdash{}{0pt}%
\pgfsys@defobject{currentmarker}{\pgfqpoint{0.000000in}{-0.027778in}}{\pgfqpoint{0.000000in}{0.000000in}}{%
\pgfpathmoveto{\pgfqpoint{0.000000in}{0.000000in}}%
\pgfpathlineto{\pgfqpoint{0.000000in}{-0.027778in}}%
\pgfusepath{stroke,fill}%
}%
\begin{pgfscope}%
\pgfsys@transformshift{4.815387in}{0.438889in}%
\pgfsys@useobject{currentmarker}{}%
\end{pgfscope}%
\end{pgfscope}%
\begin{pgfscope}%
\pgfsetbuttcap%
\pgfsetroundjoin%
\definecolor{currentfill}{rgb}{0.000000,0.000000,0.000000}%
\pgfsetfillcolor{currentfill}%
\pgfsetlinewidth{0.602250pt}%
\definecolor{currentstroke}{rgb}{0.000000,0.000000,0.000000}%
\pgfsetstrokecolor{currentstroke}%
\pgfsetdash{}{0pt}%
\pgfsys@defobject{currentmarker}{\pgfqpoint{0.000000in}{-0.027778in}}{\pgfqpoint{0.000000in}{0.000000in}}{%
\pgfpathmoveto{\pgfqpoint{0.000000in}{0.000000in}}%
\pgfpathlineto{\pgfqpoint{0.000000in}{-0.027778in}}%
\pgfusepath{stroke,fill}%
}%
\begin{pgfscope}%
\pgfsys@transformshift{4.863145in}{0.438889in}%
\pgfsys@useobject{currentmarker}{}%
\end{pgfscope}%
\end{pgfscope}%
\begin{pgfscope}%
\pgfsetbuttcap%
\pgfsetroundjoin%
\definecolor{currentfill}{rgb}{0.000000,0.000000,0.000000}%
\pgfsetfillcolor{currentfill}%
\pgfsetlinewidth{0.602250pt}%
\definecolor{currentstroke}{rgb}{0.000000,0.000000,0.000000}%
\pgfsetstrokecolor{currentstroke}%
\pgfsetdash{}{0pt}%
\pgfsys@defobject{currentmarker}{\pgfqpoint{0.000000in}{-0.027778in}}{\pgfqpoint{0.000000in}{0.000000in}}{%
\pgfpathmoveto{\pgfqpoint{0.000000in}{0.000000in}}%
\pgfpathlineto{\pgfqpoint{0.000000in}{-0.027778in}}%
\pgfusepath{stroke,fill}%
}%
\begin{pgfscope}%
\pgfsys@transformshift{4.903523in}{0.438889in}%
\pgfsys@useobject{currentmarker}{}%
\end{pgfscope}%
\end{pgfscope}%
\begin{pgfscope}%
\pgfsetbuttcap%
\pgfsetroundjoin%
\definecolor{currentfill}{rgb}{0.000000,0.000000,0.000000}%
\pgfsetfillcolor{currentfill}%
\pgfsetlinewidth{0.602250pt}%
\definecolor{currentstroke}{rgb}{0.000000,0.000000,0.000000}%
\pgfsetstrokecolor{currentstroke}%
\pgfsetdash{}{0pt}%
\pgfsys@defobject{currentmarker}{\pgfqpoint{0.000000in}{-0.027778in}}{\pgfqpoint{0.000000in}{0.000000in}}{%
\pgfpathmoveto{\pgfqpoint{0.000000in}{0.000000in}}%
\pgfpathlineto{\pgfqpoint{0.000000in}{-0.027778in}}%
\pgfusepath{stroke,fill}%
}%
\begin{pgfscope}%
\pgfsys@transformshift{4.938501in}{0.438889in}%
\pgfsys@useobject{currentmarker}{}%
\end{pgfscope}%
\end{pgfscope}%
\begin{pgfscope}%
\pgfsetbuttcap%
\pgfsetroundjoin%
\definecolor{currentfill}{rgb}{0.000000,0.000000,0.000000}%
\pgfsetfillcolor{currentfill}%
\pgfsetlinewidth{0.602250pt}%
\definecolor{currentstroke}{rgb}{0.000000,0.000000,0.000000}%
\pgfsetstrokecolor{currentstroke}%
\pgfsetdash{}{0pt}%
\pgfsys@defobject{currentmarker}{\pgfqpoint{0.000000in}{-0.027778in}}{\pgfqpoint{0.000000in}{0.000000in}}{%
\pgfpathmoveto{\pgfqpoint{0.000000in}{0.000000in}}%
\pgfpathlineto{\pgfqpoint{0.000000in}{-0.027778in}}%
\pgfusepath{stroke,fill}%
}%
\begin{pgfscope}%
\pgfsys@transformshift{4.969353in}{0.438889in}%
\pgfsys@useobject{currentmarker}{}%
\end{pgfscope}%
\end{pgfscope}%
\begin{pgfscope}%
\pgfsetbuttcap%
\pgfsetroundjoin%
\definecolor{currentfill}{rgb}{0.000000,0.000000,0.000000}%
\pgfsetfillcolor{currentfill}%
\pgfsetlinewidth{0.602250pt}%
\definecolor{currentstroke}{rgb}{0.000000,0.000000,0.000000}%
\pgfsetstrokecolor{currentstroke}%
\pgfsetdash{}{0pt}%
\pgfsys@defobject{currentmarker}{\pgfqpoint{0.000000in}{-0.027778in}}{\pgfqpoint{0.000000in}{0.000000in}}{%
\pgfpathmoveto{\pgfqpoint{0.000000in}{0.000000in}}%
\pgfpathlineto{\pgfqpoint{0.000000in}{-0.027778in}}%
\pgfusepath{stroke,fill}%
}%
\begin{pgfscope}%
\pgfsys@transformshift{5.178516in}{0.438889in}%
\pgfsys@useobject{currentmarker}{}%
\end{pgfscope}%
\end{pgfscope}%
\begin{pgfscope}%
\pgfsetbuttcap%
\pgfsetroundjoin%
\definecolor{currentfill}{rgb}{0.000000,0.000000,0.000000}%
\pgfsetfillcolor{currentfill}%
\pgfsetlinewidth{0.602250pt}%
\definecolor{currentstroke}{rgb}{0.000000,0.000000,0.000000}%
\pgfsetstrokecolor{currentstroke}%
\pgfsetdash{}{0pt}%
\pgfsys@defobject{currentmarker}{\pgfqpoint{0.000000in}{-0.027778in}}{\pgfqpoint{0.000000in}{0.000000in}}{%
\pgfpathmoveto{\pgfqpoint{0.000000in}{0.000000in}}%
\pgfpathlineto{\pgfqpoint{0.000000in}{-0.027778in}}%
\pgfusepath{stroke,fill}%
}%
\begin{pgfscope}%
\pgfsys@transformshift{5.284724in}{0.438889in}%
\pgfsys@useobject{currentmarker}{}%
\end{pgfscope}%
\end{pgfscope}%
\begin{pgfscope}%
\pgfsetbuttcap%
\pgfsetroundjoin%
\definecolor{currentfill}{rgb}{0.000000,0.000000,0.000000}%
\pgfsetfillcolor{currentfill}%
\pgfsetlinewidth{0.602250pt}%
\definecolor{currentstroke}{rgb}{0.000000,0.000000,0.000000}%
\pgfsetstrokecolor{currentstroke}%
\pgfsetdash{}{0pt}%
\pgfsys@defobject{currentmarker}{\pgfqpoint{0.000000in}{-0.027778in}}{\pgfqpoint{0.000000in}{0.000000in}}{%
\pgfpathmoveto{\pgfqpoint{0.000000in}{0.000000in}}%
\pgfpathlineto{\pgfqpoint{0.000000in}{-0.027778in}}%
\pgfusepath{stroke,fill}%
}%
\begin{pgfscope}%
\pgfsys@transformshift{5.360080in}{0.438889in}%
\pgfsys@useobject{currentmarker}{}%
\end{pgfscope}%
\end{pgfscope}%
\begin{pgfscope}%
\pgfsetbuttcap%
\pgfsetroundjoin%
\definecolor{currentfill}{rgb}{0.000000,0.000000,0.000000}%
\pgfsetfillcolor{currentfill}%
\pgfsetlinewidth{0.602250pt}%
\definecolor{currentstroke}{rgb}{0.000000,0.000000,0.000000}%
\pgfsetstrokecolor{currentstroke}%
\pgfsetdash{}{0pt}%
\pgfsys@defobject{currentmarker}{\pgfqpoint{0.000000in}{-0.027778in}}{\pgfqpoint{0.000000in}{0.000000in}}{%
\pgfpathmoveto{\pgfqpoint{0.000000in}{0.000000in}}%
\pgfpathlineto{\pgfqpoint{0.000000in}{-0.027778in}}%
\pgfusepath{stroke,fill}%
}%
\begin{pgfscope}%
\pgfsys@transformshift{5.418531in}{0.438889in}%
\pgfsys@useobject{currentmarker}{}%
\end{pgfscope}%
\end{pgfscope}%
\begin{pgfscope}%
\pgfsetbuttcap%
\pgfsetroundjoin%
\definecolor{currentfill}{rgb}{0.000000,0.000000,0.000000}%
\pgfsetfillcolor{currentfill}%
\pgfsetlinewidth{0.602250pt}%
\definecolor{currentstroke}{rgb}{0.000000,0.000000,0.000000}%
\pgfsetstrokecolor{currentstroke}%
\pgfsetdash{}{0pt}%
\pgfsys@defobject{currentmarker}{\pgfqpoint{0.000000in}{-0.027778in}}{\pgfqpoint{0.000000in}{0.000000in}}{%
\pgfpathmoveto{\pgfqpoint{0.000000in}{0.000000in}}%
\pgfpathlineto{\pgfqpoint{0.000000in}{-0.027778in}}%
\pgfusepath{stroke,fill}%
}%
\begin{pgfscope}%
\pgfsys@transformshift{5.466289in}{0.438889in}%
\pgfsys@useobject{currentmarker}{}%
\end{pgfscope}%
\end{pgfscope}%
\begin{pgfscope}%
\pgfsetbuttcap%
\pgfsetroundjoin%
\definecolor{currentfill}{rgb}{0.000000,0.000000,0.000000}%
\pgfsetfillcolor{currentfill}%
\pgfsetlinewidth{0.602250pt}%
\definecolor{currentstroke}{rgb}{0.000000,0.000000,0.000000}%
\pgfsetstrokecolor{currentstroke}%
\pgfsetdash{}{0pt}%
\pgfsys@defobject{currentmarker}{\pgfqpoint{0.000000in}{-0.027778in}}{\pgfqpoint{0.000000in}{0.000000in}}{%
\pgfpathmoveto{\pgfqpoint{0.000000in}{0.000000in}}%
\pgfpathlineto{\pgfqpoint{0.000000in}{-0.027778in}}%
\pgfusepath{stroke,fill}%
}%
\begin{pgfscope}%
\pgfsys@transformshift{5.506667in}{0.438889in}%
\pgfsys@useobject{currentmarker}{}%
\end{pgfscope}%
\end{pgfscope}%
\begin{pgfscope}%
\pgfsetbuttcap%
\pgfsetroundjoin%
\definecolor{currentfill}{rgb}{0.000000,0.000000,0.000000}%
\pgfsetfillcolor{currentfill}%
\pgfsetlinewidth{0.602250pt}%
\definecolor{currentstroke}{rgb}{0.000000,0.000000,0.000000}%
\pgfsetstrokecolor{currentstroke}%
\pgfsetdash{}{0pt}%
\pgfsys@defobject{currentmarker}{\pgfqpoint{0.000000in}{-0.027778in}}{\pgfqpoint{0.000000in}{0.000000in}}{%
\pgfpathmoveto{\pgfqpoint{0.000000in}{0.000000in}}%
\pgfpathlineto{\pgfqpoint{0.000000in}{-0.027778in}}%
\pgfusepath{stroke,fill}%
}%
\begin{pgfscope}%
\pgfsys@transformshift{5.541645in}{0.438889in}%
\pgfsys@useobject{currentmarker}{}%
\end{pgfscope}%
\end{pgfscope}%
\begin{pgfscope}%
\pgfsetbuttcap%
\pgfsetroundjoin%
\definecolor{currentfill}{rgb}{0.000000,0.000000,0.000000}%
\pgfsetfillcolor{currentfill}%
\pgfsetlinewidth{0.602250pt}%
\definecolor{currentstroke}{rgb}{0.000000,0.000000,0.000000}%
\pgfsetstrokecolor{currentstroke}%
\pgfsetdash{}{0pt}%
\pgfsys@defobject{currentmarker}{\pgfqpoint{0.000000in}{-0.027778in}}{\pgfqpoint{0.000000in}{0.000000in}}{%
\pgfpathmoveto{\pgfqpoint{0.000000in}{0.000000in}}%
\pgfpathlineto{\pgfqpoint{0.000000in}{-0.027778in}}%
\pgfusepath{stroke,fill}%
}%
\begin{pgfscope}%
\pgfsys@transformshift{5.572497in}{0.438889in}%
\pgfsys@useobject{currentmarker}{}%
\end{pgfscope}%
\end{pgfscope}%
\begin{pgfscope}%
\pgfsetbuttcap%
\pgfsetroundjoin%
\definecolor{currentfill}{rgb}{0.000000,0.000000,0.000000}%
\pgfsetfillcolor{currentfill}%
\pgfsetlinewidth{0.602250pt}%
\definecolor{currentstroke}{rgb}{0.000000,0.000000,0.000000}%
\pgfsetstrokecolor{currentstroke}%
\pgfsetdash{}{0pt}%
\pgfsys@defobject{currentmarker}{\pgfqpoint{0.000000in}{-0.027778in}}{\pgfqpoint{0.000000in}{0.000000in}}{%
\pgfpathmoveto{\pgfqpoint{0.000000in}{0.000000in}}%
\pgfpathlineto{\pgfqpoint{0.000000in}{-0.027778in}}%
\pgfusepath{stroke,fill}%
}%
\begin{pgfscope}%
\pgfsys@transformshift{5.781660in}{0.438889in}%
\pgfsys@useobject{currentmarker}{}%
\end{pgfscope}%
\end{pgfscope}%
\begin{pgfscope}%
\pgfsetbuttcap%
\pgfsetroundjoin%
\definecolor{currentfill}{rgb}{0.000000,0.000000,0.000000}%
\pgfsetfillcolor{currentfill}%
\pgfsetlinewidth{0.602250pt}%
\definecolor{currentstroke}{rgb}{0.000000,0.000000,0.000000}%
\pgfsetstrokecolor{currentstroke}%
\pgfsetdash{}{0pt}%
\pgfsys@defobject{currentmarker}{\pgfqpoint{0.000000in}{-0.027778in}}{\pgfqpoint{0.000000in}{0.000000in}}{%
\pgfpathmoveto{\pgfqpoint{0.000000in}{0.000000in}}%
\pgfpathlineto{\pgfqpoint{0.000000in}{-0.027778in}}%
\pgfusepath{stroke,fill}%
}%
\begin{pgfscope}%
\pgfsys@transformshift{5.887868in}{0.438889in}%
\pgfsys@useobject{currentmarker}{}%
\end{pgfscope}%
\end{pgfscope}%
\begin{pgfscope}%
\pgfsetbuttcap%
\pgfsetroundjoin%
\definecolor{currentfill}{rgb}{0.000000,0.000000,0.000000}%
\pgfsetfillcolor{currentfill}%
\pgfsetlinewidth{0.602250pt}%
\definecolor{currentstroke}{rgb}{0.000000,0.000000,0.000000}%
\pgfsetstrokecolor{currentstroke}%
\pgfsetdash{}{0pt}%
\pgfsys@defobject{currentmarker}{\pgfqpoint{0.000000in}{-0.027778in}}{\pgfqpoint{0.000000in}{0.000000in}}{%
\pgfpathmoveto{\pgfqpoint{0.000000in}{0.000000in}}%
\pgfpathlineto{\pgfqpoint{0.000000in}{-0.027778in}}%
\pgfusepath{stroke,fill}%
}%
\begin{pgfscope}%
\pgfsys@transformshift{5.963224in}{0.438889in}%
\pgfsys@useobject{currentmarker}{}%
\end{pgfscope}%
\end{pgfscope}%
\begin{pgfscope}%
\pgfsetbuttcap%
\pgfsetroundjoin%
\definecolor{currentfill}{rgb}{0.000000,0.000000,0.000000}%
\pgfsetfillcolor{currentfill}%
\pgfsetlinewidth{0.602250pt}%
\definecolor{currentstroke}{rgb}{0.000000,0.000000,0.000000}%
\pgfsetstrokecolor{currentstroke}%
\pgfsetdash{}{0pt}%
\pgfsys@defobject{currentmarker}{\pgfqpoint{0.000000in}{-0.027778in}}{\pgfqpoint{0.000000in}{0.000000in}}{%
\pgfpathmoveto{\pgfqpoint{0.000000in}{0.000000in}}%
\pgfpathlineto{\pgfqpoint{0.000000in}{-0.027778in}}%
\pgfusepath{stroke,fill}%
}%
\begin{pgfscope}%
\pgfsys@transformshift{6.021675in}{0.438889in}%
\pgfsys@useobject{currentmarker}{}%
\end{pgfscope}%
\end{pgfscope}%
\begin{pgfscope}%
\pgfsetbuttcap%
\pgfsetroundjoin%
\definecolor{currentfill}{rgb}{0.000000,0.000000,0.000000}%
\pgfsetfillcolor{currentfill}%
\pgfsetlinewidth{0.602250pt}%
\definecolor{currentstroke}{rgb}{0.000000,0.000000,0.000000}%
\pgfsetstrokecolor{currentstroke}%
\pgfsetdash{}{0pt}%
\pgfsys@defobject{currentmarker}{\pgfqpoint{0.000000in}{-0.027778in}}{\pgfqpoint{0.000000in}{0.000000in}}{%
\pgfpathmoveto{\pgfqpoint{0.000000in}{0.000000in}}%
\pgfpathlineto{\pgfqpoint{0.000000in}{-0.027778in}}%
\pgfusepath{stroke,fill}%
}%
\begin{pgfscope}%
\pgfsys@transformshift{6.069433in}{0.438889in}%
\pgfsys@useobject{currentmarker}{}%
\end{pgfscope}%
\end{pgfscope}%
\begin{pgfscope}%
\pgfsetbuttcap%
\pgfsetroundjoin%
\definecolor{currentfill}{rgb}{0.000000,0.000000,0.000000}%
\pgfsetfillcolor{currentfill}%
\pgfsetlinewidth{0.602250pt}%
\definecolor{currentstroke}{rgb}{0.000000,0.000000,0.000000}%
\pgfsetstrokecolor{currentstroke}%
\pgfsetdash{}{0pt}%
\pgfsys@defobject{currentmarker}{\pgfqpoint{0.000000in}{-0.027778in}}{\pgfqpoint{0.000000in}{0.000000in}}{%
\pgfpathmoveto{\pgfqpoint{0.000000in}{0.000000in}}%
\pgfpathlineto{\pgfqpoint{0.000000in}{-0.027778in}}%
\pgfusepath{stroke,fill}%
}%
\begin{pgfscope}%
\pgfsys@transformshift{6.109811in}{0.438889in}%
\pgfsys@useobject{currentmarker}{}%
\end{pgfscope}%
\end{pgfscope}%
\begin{pgfscope}%
\pgfsetbuttcap%
\pgfsetroundjoin%
\definecolor{currentfill}{rgb}{0.000000,0.000000,0.000000}%
\pgfsetfillcolor{currentfill}%
\pgfsetlinewidth{0.602250pt}%
\definecolor{currentstroke}{rgb}{0.000000,0.000000,0.000000}%
\pgfsetstrokecolor{currentstroke}%
\pgfsetdash{}{0pt}%
\pgfsys@defobject{currentmarker}{\pgfqpoint{0.000000in}{-0.027778in}}{\pgfqpoint{0.000000in}{0.000000in}}{%
\pgfpathmoveto{\pgfqpoint{0.000000in}{0.000000in}}%
\pgfpathlineto{\pgfqpoint{0.000000in}{-0.027778in}}%
\pgfusepath{stroke,fill}%
}%
\begin{pgfscope}%
\pgfsys@transformshift{6.144789in}{0.438889in}%
\pgfsys@useobject{currentmarker}{}%
\end{pgfscope}%
\end{pgfscope}%
\begin{pgfscope}%
\pgfsetbuttcap%
\pgfsetroundjoin%
\definecolor{currentfill}{rgb}{0.000000,0.000000,0.000000}%
\pgfsetfillcolor{currentfill}%
\pgfsetlinewidth{0.602250pt}%
\definecolor{currentstroke}{rgb}{0.000000,0.000000,0.000000}%
\pgfsetstrokecolor{currentstroke}%
\pgfsetdash{}{0pt}%
\pgfsys@defobject{currentmarker}{\pgfqpoint{0.000000in}{-0.027778in}}{\pgfqpoint{0.000000in}{0.000000in}}{%
\pgfpathmoveto{\pgfqpoint{0.000000in}{0.000000in}}%
\pgfpathlineto{\pgfqpoint{0.000000in}{-0.027778in}}%
\pgfusepath{stroke,fill}%
}%
\begin{pgfscope}%
\pgfsys@transformshift{6.175641in}{0.438889in}%
\pgfsys@useobject{currentmarker}{}%
\end{pgfscope}%
\end{pgfscope}%
\begin{pgfscope}%
\pgfsetbuttcap%
\pgfsetroundjoin%
\definecolor{currentfill}{rgb}{0.000000,0.000000,0.000000}%
\pgfsetfillcolor{currentfill}%
\pgfsetlinewidth{0.602250pt}%
\definecolor{currentstroke}{rgb}{0.000000,0.000000,0.000000}%
\pgfsetstrokecolor{currentstroke}%
\pgfsetdash{}{0pt}%
\pgfsys@defobject{currentmarker}{\pgfqpoint{0.000000in}{-0.027778in}}{\pgfqpoint{0.000000in}{0.000000in}}{%
\pgfpathmoveto{\pgfqpoint{0.000000in}{0.000000in}}%
\pgfpathlineto{\pgfqpoint{0.000000in}{-0.027778in}}%
\pgfusepath{stroke,fill}%
}%
\begin{pgfscope}%
\pgfsys@transformshift{6.384804in}{0.438889in}%
\pgfsys@useobject{currentmarker}{}%
\end{pgfscope}%
\end{pgfscope}%
\begin{pgfscope}%
\pgfsetbuttcap%
\pgfsetroundjoin%
\definecolor{currentfill}{rgb}{0.000000,0.000000,0.000000}%
\pgfsetfillcolor{currentfill}%
\pgfsetlinewidth{0.602250pt}%
\definecolor{currentstroke}{rgb}{0.000000,0.000000,0.000000}%
\pgfsetstrokecolor{currentstroke}%
\pgfsetdash{}{0pt}%
\pgfsys@defobject{currentmarker}{\pgfqpoint{0.000000in}{-0.027778in}}{\pgfqpoint{0.000000in}{0.000000in}}{%
\pgfpathmoveto{\pgfqpoint{0.000000in}{0.000000in}}%
\pgfpathlineto{\pgfqpoint{0.000000in}{-0.027778in}}%
\pgfusepath{stroke,fill}%
}%
\begin{pgfscope}%
\pgfsys@transformshift{6.491012in}{0.438889in}%
\pgfsys@useobject{currentmarker}{}%
\end{pgfscope}%
\end{pgfscope}%
\begin{pgfscope}%
\pgfsetbuttcap%
\pgfsetroundjoin%
\definecolor{currentfill}{rgb}{0.000000,0.000000,0.000000}%
\pgfsetfillcolor{currentfill}%
\pgfsetlinewidth{0.602250pt}%
\definecolor{currentstroke}{rgb}{0.000000,0.000000,0.000000}%
\pgfsetstrokecolor{currentstroke}%
\pgfsetdash{}{0pt}%
\pgfsys@defobject{currentmarker}{\pgfqpoint{0.000000in}{-0.027778in}}{\pgfqpoint{0.000000in}{0.000000in}}{%
\pgfpathmoveto{\pgfqpoint{0.000000in}{0.000000in}}%
\pgfpathlineto{\pgfqpoint{0.000000in}{-0.027778in}}%
\pgfusepath{stroke,fill}%
}%
\begin{pgfscope}%
\pgfsys@transformshift{6.566368in}{0.438889in}%
\pgfsys@useobject{currentmarker}{}%
\end{pgfscope}%
\end{pgfscope}%
\begin{pgfscope}%
\pgfsetbuttcap%
\pgfsetroundjoin%
\definecolor{currentfill}{rgb}{0.000000,0.000000,0.000000}%
\pgfsetfillcolor{currentfill}%
\pgfsetlinewidth{0.602250pt}%
\definecolor{currentstroke}{rgb}{0.000000,0.000000,0.000000}%
\pgfsetstrokecolor{currentstroke}%
\pgfsetdash{}{0pt}%
\pgfsys@defobject{currentmarker}{\pgfqpoint{0.000000in}{-0.027778in}}{\pgfqpoint{0.000000in}{0.000000in}}{%
\pgfpathmoveto{\pgfqpoint{0.000000in}{0.000000in}}%
\pgfpathlineto{\pgfqpoint{0.000000in}{-0.027778in}}%
\pgfusepath{stroke,fill}%
}%
\begin{pgfscope}%
\pgfsys@transformshift{6.624819in}{0.438889in}%
\pgfsys@useobject{currentmarker}{}%
\end{pgfscope}%
\end{pgfscope}%
\begin{pgfscope}%
\pgfsetbuttcap%
\pgfsetroundjoin%
\definecolor{currentfill}{rgb}{0.000000,0.000000,0.000000}%
\pgfsetfillcolor{currentfill}%
\pgfsetlinewidth{0.602250pt}%
\definecolor{currentstroke}{rgb}{0.000000,0.000000,0.000000}%
\pgfsetstrokecolor{currentstroke}%
\pgfsetdash{}{0pt}%
\pgfsys@defobject{currentmarker}{\pgfqpoint{0.000000in}{-0.027778in}}{\pgfqpoint{0.000000in}{0.000000in}}{%
\pgfpathmoveto{\pgfqpoint{0.000000in}{0.000000in}}%
\pgfpathlineto{\pgfqpoint{0.000000in}{-0.027778in}}%
\pgfusepath{stroke,fill}%
}%
\begin{pgfscope}%
\pgfsys@transformshift{6.672577in}{0.438889in}%
\pgfsys@useobject{currentmarker}{}%
\end{pgfscope}%
\end{pgfscope}%
\begin{pgfscope}%
\pgfsetbuttcap%
\pgfsetroundjoin%
\definecolor{currentfill}{rgb}{0.000000,0.000000,0.000000}%
\pgfsetfillcolor{currentfill}%
\pgfsetlinewidth{0.602250pt}%
\definecolor{currentstroke}{rgb}{0.000000,0.000000,0.000000}%
\pgfsetstrokecolor{currentstroke}%
\pgfsetdash{}{0pt}%
\pgfsys@defobject{currentmarker}{\pgfqpoint{0.000000in}{-0.027778in}}{\pgfqpoint{0.000000in}{0.000000in}}{%
\pgfpathmoveto{\pgfqpoint{0.000000in}{0.000000in}}%
\pgfpathlineto{\pgfqpoint{0.000000in}{-0.027778in}}%
\pgfusepath{stroke,fill}%
}%
\begin{pgfscope}%
\pgfsys@transformshift{6.712955in}{0.438889in}%
\pgfsys@useobject{currentmarker}{}%
\end{pgfscope}%
\end{pgfscope}%
\begin{pgfscope}%
\pgfsetbuttcap%
\pgfsetroundjoin%
\definecolor{currentfill}{rgb}{0.000000,0.000000,0.000000}%
\pgfsetfillcolor{currentfill}%
\pgfsetlinewidth{0.602250pt}%
\definecolor{currentstroke}{rgb}{0.000000,0.000000,0.000000}%
\pgfsetstrokecolor{currentstroke}%
\pgfsetdash{}{0pt}%
\pgfsys@defobject{currentmarker}{\pgfqpoint{0.000000in}{-0.027778in}}{\pgfqpoint{0.000000in}{0.000000in}}{%
\pgfpathmoveto{\pgfqpoint{0.000000in}{0.000000in}}%
\pgfpathlineto{\pgfqpoint{0.000000in}{-0.027778in}}%
\pgfusepath{stroke,fill}%
}%
\begin{pgfscope}%
\pgfsys@transformshift{6.747933in}{0.438889in}%
\pgfsys@useobject{currentmarker}{}%
\end{pgfscope}%
\end{pgfscope}%
\begin{pgfscope}%
\pgfsetbuttcap%
\pgfsetroundjoin%
\definecolor{currentfill}{rgb}{0.000000,0.000000,0.000000}%
\pgfsetfillcolor{currentfill}%
\pgfsetlinewidth{0.602250pt}%
\definecolor{currentstroke}{rgb}{0.000000,0.000000,0.000000}%
\pgfsetstrokecolor{currentstroke}%
\pgfsetdash{}{0pt}%
\pgfsys@defobject{currentmarker}{\pgfqpoint{0.000000in}{-0.027778in}}{\pgfqpoint{0.000000in}{0.000000in}}{%
\pgfpathmoveto{\pgfqpoint{0.000000in}{0.000000in}}%
\pgfpathlineto{\pgfqpoint{0.000000in}{-0.027778in}}%
\pgfusepath{stroke,fill}%
}%
\begin{pgfscope}%
\pgfsys@transformshift{6.778785in}{0.438889in}%
\pgfsys@useobject{currentmarker}{}%
\end{pgfscope}%
\end{pgfscope}%
\begin{pgfscope}%
\pgfsetbuttcap%
\pgfsetroundjoin%
\definecolor{currentfill}{rgb}{0.000000,0.000000,0.000000}%
\pgfsetfillcolor{currentfill}%
\pgfsetlinewidth{0.602250pt}%
\definecolor{currentstroke}{rgb}{0.000000,0.000000,0.000000}%
\pgfsetstrokecolor{currentstroke}%
\pgfsetdash{}{0pt}%
\pgfsys@defobject{currentmarker}{\pgfqpoint{0.000000in}{-0.027778in}}{\pgfqpoint{0.000000in}{0.000000in}}{%
\pgfpathmoveto{\pgfqpoint{0.000000in}{0.000000in}}%
\pgfpathlineto{\pgfqpoint{0.000000in}{-0.027778in}}%
\pgfusepath{stroke,fill}%
}%
\begin{pgfscope}%
\pgfsys@transformshift{6.987948in}{0.438889in}%
\pgfsys@useobject{currentmarker}{}%
\end{pgfscope}%
\end{pgfscope}%
\begin{pgfscope}%
\pgfsetbuttcap%
\pgfsetroundjoin%
\definecolor{currentfill}{rgb}{0.000000,0.000000,0.000000}%
\pgfsetfillcolor{currentfill}%
\pgfsetlinewidth{0.602250pt}%
\definecolor{currentstroke}{rgb}{0.000000,0.000000,0.000000}%
\pgfsetstrokecolor{currentstroke}%
\pgfsetdash{}{0pt}%
\pgfsys@defobject{currentmarker}{\pgfqpoint{0.000000in}{-0.027778in}}{\pgfqpoint{0.000000in}{0.000000in}}{%
\pgfpathmoveto{\pgfqpoint{0.000000in}{0.000000in}}%
\pgfpathlineto{\pgfqpoint{0.000000in}{-0.027778in}}%
\pgfusepath{stroke,fill}%
}%
\begin{pgfscope}%
\pgfsys@transformshift{7.094156in}{0.438889in}%
\pgfsys@useobject{currentmarker}{}%
\end{pgfscope}%
\end{pgfscope}%
\begin{pgfscope}%
\pgfsetbuttcap%
\pgfsetroundjoin%
\definecolor{currentfill}{rgb}{0.000000,0.000000,0.000000}%
\pgfsetfillcolor{currentfill}%
\pgfsetlinewidth{0.602250pt}%
\definecolor{currentstroke}{rgb}{0.000000,0.000000,0.000000}%
\pgfsetstrokecolor{currentstroke}%
\pgfsetdash{}{0pt}%
\pgfsys@defobject{currentmarker}{\pgfqpoint{0.000000in}{-0.027778in}}{\pgfqpoint{0.000000in}{0.000000in}}{%
\pgfpathmoveto{\pgfqpoint{0.000000in}{0.000000in}}%
\pgfpathlineto{\pgfqpoint{0.000000in}{-0.027778in}}%
\pgfusepath{stroke,fill}%
}%
\begin{pgfscope}%
\pgfsys@transformshift{7.169512in}{0.438889in}%
\pgfsys@useobject{currentmarker}{}%
\end{pgfscope}%
\end{pgfscope}%
\begin{pgfscope}%
\pgfsetbuttcap%
\pgfsetroundjoin%
\definecolor{currentfill}{rgb}{0.000000,0.000000,0.000000}%
\pgfsetfillcolor{currentfill}%
\pgfsetlinewidth{0.602250pt}%
\definecolor{currentstroke}{rgb}{0.000000,0.000000,0.000000}%
\pgfsetstrokecolor{currentstroke}%
\pgfsetdash{}{0pt}%
\pgfsys@defobject{currentmarker}{\pgfqpoint{0.000000in}{-0.027778in}}{\pgfqpoint{0.000000in}{0.000000in}}{%
\pgfpathmoveto{\pgfqpoint{0.000000in}{0.000000in}}%
\pgfpathlineto{\pgfqpoint{0.000000in}{-0.027778in}}%
\pgfusepath{stroke,fill}%
}%
\begin{pgfscope}%
\pgfsys@transformshift{7.227963in}{0.438889in}%
\pgfsys@useobject{currentmarker}{}%
\end{pgfscope}%
\end{pgfscope}%
\begin{pgfscope}%
\pgfsetbuttcap%
\pgfsetroundjoin%
\definecolor{currentfill}{rgb}{0.000000,0.000000,0.000000}%
\pgfsetfillcolor{currentfill}%
\pgfsetlinewidth{0.602250pt}%
\definecolor{currentstroke}{rgb}{0.000000,0.000000,0.000000}%
\pgfsetstrokecolor{currentstroke}%
\pgfsetdash{}{0pt}%
\pgfsys@defobject{currentmarker}{\pgfqpoint{0.000000in}{-0.027778in}}{\pgfqpoint{0.000000in}{0.000000in}}{%
\pgfpathmoveto{\pgfqpoint{0.000000in}{0.000000in}}%
\pgfpathlineto{\pgfqpoint{0.000000in}{-0.027778in}}%
\pgfusepath{stroke,fill}%
}%
\begin{pgfscope}%
\pgfsys@transformshift{7.275721in}{0.438889in}%
\pgfsys@useobject{currentmarker}{}%
\end{pgfscope}%
\end{pgfscope}%
\begin{pgfscope}%
\pgfsetbuttcap%
\pgfsetroundjoin%
\definecolor{currentfill}{rgb}{0.000000,0.000000,0.000000}%
\pgfsetfillcolor{currentfill}%
\pgfsetlinewidth{0.602250pt}%
\definecolor{currentstroke}{rgb}{0.000000,0.000000,0.000000}%
\pgfsetstrokecolor{currentstroke}%
\pgfsetdash{}{0pt}%
\pgfsys@defobject{currentmarker}{\pgfqpoint{0.000000in}{-0.027778in}}{\pgfqpoint{0.000000in}{0.000000in}}{%
\pgfpathmoveto{\pgfqpoint{0.000000in}{0.000000in}}%
\pgfpathlineto{\pgfqpoint{0.000000in}{-0.027778in}}%
\pgfusepath{stroke,fill}%
}%
\begin{pgfscope}%
\pgfsys@transformshift{7.316099in}{0.438889in}%
\pgfsys@useobject{currentmarker}{}%
\end{pgfscope}%
\end{pgfscope}%
\begin{pgfscope}%
\pgfsetbuttcap%
\pgfsetroundjoin%
\definecolor{currentfill}{rgb}{0.000000,0.000000,0.000000}%
\pgfsetfillcolor{currentfill}%
\pgfsetlinewidth{0.602250pt}%
\definecolor{currentstroke}{rgb}{0.000000,0.000000,0.000000}%
\pgfsetstrokecolor{currentstroke}%
\pgfsetdash{}{0pt}%
\pgfsys@defobject{currentmarker}{\pgfqpoint{0.000000in}{-0.027778in}}{\pgfqpoint{0.000000in}{0.000000in}}{%
\pgfpathmoveto{\pgfqpoint{0.000000in}{0.000000in}}%
\pgfpathlineto{\pgfqpoint{0.000000in}{-0.027778in}}%
\pgfusepath{stroke,fill}%
}%
\begin{pgfscope}%
\pgfsys@transformshift{7.351077in}{0.438889in}%
\pgfsys@useobject{currentmarker}{}%
\end{pgfscope}%
\end{pgfscope}%
\begin{pgfscope}%
\pgfsetbuttcap%
\pgfsetroundjoin%
\definecolor{currentfill}{rgb}{0.000000,0.000000,0.000000}%
\pgfsetfillcolor{currentfill}%
\pgfsetlinewidth{0.602250pt}%
\definecolor{currentstroke}{rgb}{0.000000,0.000000,0.000000}%
\pgfsetstrokecolor{currentstroke}%
\pgfsetdash{}{0pt}%
\pgfsys@defobject{currentmarker}{\pgfqpoint{0.000000in}{-0.027778in}}{\pgfqpoint{0.000000in}{0.000000in}}{%
\pgfpathmoveto{\pgfqpoint{0.000000in}{0.000000in}}%
\pgfpathlineto{\pgfqpoint{0.000000in}{-0.027778in}}%
\pgfusepath{stroke,fill}%
}%
\begin{pgfscope}%
\pgfsys@transformshift{7.381929in}{0.438889in}%
\pgfsys@useobject{currentmarker}{}%
\end{pgfscope}%
\end{pgfscope}%
\begin{pgfscope}%
\pgfsetbuttcap%
\pgfsetroundjoin%
\definecolor{currentfill}{rgb}{0.000000,0.000000,0.000000}%
\pgfsetfillcolor{currentfill}%
\pgfsetlinewidth{0.602250pt}%
\definecolor{currentstroke}{rgb}{0.000000,0.000000,0.000000}%
\pgfsetstrokecolor{currentstroke}%
\pgfsetdash{}{0pt}%
\pgfsys@defobject{currentmarker}{\pgfqpoint{0.000000in}{-0.027778in}}{\pgfqpoint{0.000000in}{0.000000in}}{%
\pgfpathmoveto{\pgfqpoint{0.000000in}{0.000000in}}%
\pgfpathlineto{\pgfqpoint{0.000000in}{-0.027778in}}%
\pgfusepath{stroke,fill}%
}%
\begin{pgfscope}%
\pgfsys@transformshift{7.591092in}{0.438889in}%
\pgfsys@useobject{currentmarker}{}%
\end{pgfscope}%
\end{pgfscope}%
\begin{pgfscope}%
\pgfsetbuttcap%
\pgfsetroundjoin%
\definecolor{currentfill}{rgb}{0.000000,0.000000,0.000000}%
\pgfsetfillcolor{currentfill}%
\pgfsetlinewidth{0.602250pt}%
\definecolor{currentstroke}{rgb}{0.000000,0.000000,0.000000}%
\pgfsetstrokecolor{currentstroke}%
\pgfsetdash{}{0pt}%
\pgfsys@defobject{currentmarker}{\pgfqpoint{0.000000in}{-0.027778in}}{\pgfqpoint{0.000000in}{0.000000in}}{%
\pgfpathmoveto{\pgfqpoint{0.000000in}{0.000000in}}%
\pgfpathlineto{\pgfqpoint{0.000000in}{-0.027778in}}%
\pgfusepath{stroke,fill}%
}%
\begin{pgfscope}%
\pgfsys@transformshift{7.697300in}{0.438889in}%
\pgfsys@useobject{currentmarker}{}%
\end{pgfscope}%
\end{pgfscope}%
\begin{pgfscope}%
\pgfsetbuttcap%
\pgfsetroundjoin%
\definecolor{currentfill}{rgb}{0.000000,0.000000,0.000000}%
\pgfsetfillcolor{currentfill}%
\pgfsetlinewidth{0.602250pt}%
\definecolor{currentstroke}{rgb}{0.000000,0.000000,0.000000}%
\pgfsetstrokecolor{currentstroke}%
\pgfsetdash{}{0pt}%
\pgfsys@defobject{currentmarker}{\pgfqpoint{0.000000in}{-0.027778in}}{\pgfqpoint{0.000000in}{0.000000in}}{%
\pgfpathmoveto{\pgfqpoint{0.000000in}{0.000000in}}%
\pgfpathlineto{\pgfqpoint{0.000000in}{-0.027778in}}%
\pgfusepath{stroke,fill}%
}%
\begin{pgfscope}%
\pgfsys@transformshift{7.772656in}{0.438889in}%
\pgfsys@useobject{currentmarker}{}%
\end{pgfscope}%
\end{pgfscope}%
\begin{pgfscope}%
\pgfsetbuttcap%
\pgfsetroundjoin%
\definecolor{currentfill}{rgb}{0.000000,0.000000,0.000000}%
\pgfsetfillcolor{currentfill}%
\pgfsetlinewidth{0.803000pt}%
\definecolor{currentstroke}{rgb}{0.000000,0.000000,0.000000}%
\pgfsetstrokecolor{currentstroke}%
\pgfsetdash{}{0pt}%
\pgfsys@defobject{currentmarker}{\pgfqpoint{-0.048611in}{0.000000in}}{\pgfqpoint{0.000000in}{0.000000in}}{%
\pgfpathmoveto{\pgfqpoint{0.000000in}{0.000000in}}%
\pgfpathlineto{\pgfqpoint{-0.048611in}{0.000000in}}%
\pgfusepath{stroke,fill}%
}%
\begin{pgfscope}%
\pgfsys@transformshift{4.482986in}{0.440982in}%
\pgfsys@useobject{currentmarker}{}%
\end{pgfscope}%
\end{pgfscope}%
\begin{pgfscope}%
\definecolor{textcolor}{rgb}{0.000000,0.000000,0.000000}%
\pgfsetstrokecolor{textcolor}%
\pgfsetfillcolor{textcolor}%
\pgftext[x=4.048511in,y=0.388220in,left,base]{\color{textcolor}\sffamily\fontsize{10.000000}{12.000000}\selectfont −0.2}%
\end{pgfscope}%
\begin{pgfscope}%
\pgfsetbuttcap%
\pgfsetroundjoin%
\definecolor{currentfill}{rgb}{0.000000,0.000000,0.000000}%
\pgfsetfillcolor{currentfill}%
\pgfsetlinewidth{0.803000pt}%
\definecolor{currentstroke}{rgb}{0.000000,0.000000,0.000000}%
\pgfsetstrokecolor{currentstroke}%
\pgfsetdash{}{0pt}%
\pgfsys@defobject{currentmarker}{\pgfqpoint{-0.048611in}{0.000000in}}{\pgfqpoint{0.000000in}{0.000000in}}{%
\pgfpathmoveto{\pgfqpoint{0.000000in}{0.000000in}}%
\pgfpathlineto{\pgfqpoint{-0.048611in}{0.000000in}}%
\pgfusepath{stroke,fill}%
}%
\begin{pgfscope}%
\pgfsys@transformshift{4.482986in}{0.747697in}%
\pgfsys@useobject{currentmarker}{}%
\end{pgfscope}%
\end{pgfscope}%
\begin{pgfscope}%
\definecolor{textcolor}{rgb}{0.000000,0.000000,0.000000}%
\pgfsetstrokecolor{textcolor}%
\pgfsetfillcolor{textcolor}%
\pgftext[x=4.164884in,y=0.694935in,left,base]{\color{textcolor}\sffamily\fontsize{10.000000}{12.000000}\selectfont 0.0}%
\end{pgfscope}%
\begin{pgfscope}%
\pgfsetbuttcap%
\pgfsetroundjoin%
\definecolor{currentfill}{rgb}{0.000000,0.000000,0.000000}%
\pgfsetfillcolor{currentfill}%
\pgfsetlinewidth{0.803000pt}%
\definecolor{currentstroke}{rgb}{0.000000,0.000000,0.000000}%
\pgfsetstrokecolor{currentstroke}%
\pgfsetdash{}{0pt}%
\pgfsys@defobject{currentmarker}{\pgfqpoint{-0.048611in}{0.000000in}}{\pgfqpoint{0.000000in}{0.000000in}}{%
\pgfpathmoveto{\pgfqpoint{0.000000in}{0.000000in}}%
\pgfpathlineto{\pgfqpoint{-0.048611in}{0.000000in}}%
\pgfusepath{stroke,fill}%
}%
\begin{pgfscope}%
\pgfsys@transformshift{4.482986in}{1.054412in}%
\pgfsys@useobject{currentmarker}{}%
\end{pgfscope}%
\end{pgfscope}%
\begin{pgfscope}%
\definecolor{textcolor}{rgb}{0.000000,0.000000,0.000000}%
\pgfsetstrokecolor{textcolor}%
\pgfsetfillcolor{textcolor}%
\pgftext[x=4.164884in,y=1.001650in,left,base]{\color{textcolor}\sffamily\fontsize{10.000000}{12.000000}\selectfont 0.2}%
\end{pgfscope}%
\begin{pgfscope}%
\pgfsetbuttcap%
\pgfsetroundjoin%
\definecolor{currentfill}{rgb}{0.000000,0.000000,0.000000}%
\pgfsetfillcolor{currentfill}%
\pgfsetlinewidth{0.803000pt}%
\definecolor{currentstroke}{rgb}{0.000000,0.000000,0.000000}%
\pgfsetstrokecolor{currentstroke}%
\pgfsetdash{}{0pt}%
\pgfsys@defobject{currentmarker}{\pgfqpoint{-0.048611in}{0.000000in}}{\pgfqpoint{0.000000in}{0.000000in}}{%
\pgfpathmoveto{\pgfqpoint{0.000000in}{0.000000in}}%
\pgfpathlineto{\pgfqpoint{-0.048611in}{0.000000in}}%
\pgfusepath{stroke,fill}%
}%
\begin{pgfscope}%
\pgfsys@transformshift{4.482986in}{1.361127in}%
\pgfsys@useobject{currentmarker}{}%
\end{pgfscope}%
\end{pgfscope}%
\begin{pgfscope}%
\definecolor{textcolor}{rgb}{0.000000,0.000000,0.000000}%
\pgfsetstrokecolor{textcolor}%
\pgfsetfillcolor{textcolor}%
\pgftext[x=4.164884in,y=1.308365in,left,base]{\color{textcolor}\sffamily\fontsize{10.000000}{12.000000}\selectfont 0.4}%
\end{pgfscope}%
\begin{pgfscope}%
\pgfsetbuttcap%
\pgfsetroundjoin%
\definecolor{currentfill}{rgb}{0.000000,0.000000,0.000000}%
\pgfsetfillcolor{currentfill}%
\pgfsetlinewidth{0.803000pt}%
\definecolor{currentstroke}{rgb}{0.000000,0.000000,0.000000}%
\pgfsetstrokecolor{currentstroke}%
\pgfsetdash{}{0pt}%
\pgfsys@defobject{currentmarker}{\pgfqpoint{-0.048611in}{0.000000in}}{\pgfqpoint{0.000000in}{0.000000in}}{%
\pgfpathmoveto{\pgfqpoint{0.000000in}{0.000000in}}%
\pgfpathlineto{\pgfqpoint{-0.048611in}{0.000000in}}%
\pgfusepath{stroke,fill}%
}%
\begin{pgfscope}%
\pgfsys@transformshift{4.482986in}{1.667842in}%
\pgfsys@useobject{currentmarker}{}%
\end{pgfscope}%
\end{pgfscope}%
\begin{pgfscope}%
\definecolor{textcolor}{rgb}{0.000000,0.000000,0.000000}%
\pgfsetstrokecolor{textcolor}%
\pgfsetfillcolor{textcolor}%
\pgftext[x=4.164884in,y=1.615080in,left,base]{\color{textcolor}\sffamily\fontsize{10.000000}{12.000000}\selectfont 0.6}%
\end{pgfscope}%
\begin{pgfscope}%
\pgfsetbuttcap%
\pgfsetroundjoin%
\definecolor{currentfill}{rgb}{0.000000,0.000000,0.000000}%
\pgfsetfillcolor{currentfill}%
\pgfsetlinewidth{0.803000pt}%
\definecolor{currentstroke}{rgb}{0.000000,0.000000,0.000000}%
\pgfsetstrokecolor{currentstroke}%
\pgfsetdash{}{0pt}%
\pgfsys@defobject{currentmarker}{\pgfqpoint{-0.048611in}{0.000000in}}{\pgfqpoint{0.000000in}{0.000000in}}{%
\pgfpathmoveto{\pgfqpoint{0.000000in}{0.000000in}}%
\pgfpathlineto{\pgfqpoint{-0.048611in}{0.000000in}}%
\pgfusepath{stroke,fill}%
}%
\begin{pgfscope}%
\pgfsys@transformshift{4.482986in}{1.974557in}%
\pgfsys@useobject{currentmarker}{}%
\end{pgfscope}%
\end{pgfscope}%
\begin{pgfscope}%
\definecolor{textcolor}{rgb}{0.000000,0.000000,0.000000}%
\pgfsetstrokecolor{textcolor}%
\pgfsetfillcolor{textcolor}%
\pgftext[x=4.164884in,y=1.921795in,left,base]{\color{textcolor}\sffamily\fontsize{10.000000}{12.000000}\selectfont 0.8}%
\end{pgfscope}%
\begin{pgfscope}%
\pgfsetbuttcap%
\pgfsetroundjoin%
\definecolor{currentfill}{rgb}{0.000000,0.000000,0.000000}%
\pgfsetfillcolor{currentfill}%
\pgfsetlinewidth{0.803000pt}%
\definecolor{currentstroke}{rgb}{0.000000,0.000000,0.000000}%
\pgfsetstrokecolor{currentstroke}%
\pgfsetdash{}{0pt}%
\pgfsys@defobject{currentmarker}{\pgfqpoint{-0.048611in}{0.000000in}}{\pgfqpoint{0.000000in}{0.000000in}}{%
\pgfpathmoveto{\pgfqpoint{0.000000in}{0.000000in}}%
\pgfpathlineto{\pgfqpoint{-0.048611in}{0.000000in}}%
\pgfusepath{stroke,fill}%
}%
\begin{pgfscope}%
\pgfsys@transformshift{4.482986in}{2.281272in}%
\pgfsys@useobject{currentmarker}{}%
\end{pgfscope}%
\end{pgfscope}%
\begin{pgfscope}%
\definecolor{textcolor}{rgb}{0.000000,0.000000,0.000000}%
\pgfsetstrokecolor{textcolor}%
\pgfsetfillcolor{textcolor}%
\pgftext[x=4.164884in,y=2.228510in,left,base]{\color{textcolor}\sffamily\fontsize{10.000000}{12.000000}\selectfont 1.0}%
\end{pgfscope}%
\begin{pgfscope}%
\pgfsetbuttcap%
\pgfsetroundjoin%
\definecolor{currentfill}{rgb}{0.000000,0.000000,0.000000}%
\pgfsetfillcolor{currentfill}%
\pgfsetlinewidth{0.803000pt}%
\definecolor{currentstroke}{rgb}{0.000000,0.000000,0.000000}%
\pgfsetstrokecolor{currentstroke}%
\pgfsetdash{}{0pt}%
\pgfsys@defobject{currentmarker}{\pgfqpoint{-0.048611in}{0.000000in}}{\pgfqpoint{0.000000in}{0.000000in}}{%
\pgfpathmoveto{\pgfqpoint{0.000000in}{0.000000in}}%
\pgfpathlineto{\pgfqpoint{-0.048611in}{0.000000in}}%
\pgfusepath{stroke,fill}%
}%
\begin{pgfscope}%
\pgfsys@transformshift{4.482986in}{2.587987in}%
\pgfsys@useobject{currentmarker}{}%
\end{pgfscope}%
\end{pgfscope}%
\begin{pgfscope}%
\definecolor{textcolor}{rgb}{0.000000,0.000000,0.000000}%
\pgfsetstrokecolor{textcolor}%
\pgfsetfillcolor{textcolor}%
\pgftext[x=4.164884in,y=2.535225in,left,base]{\color{textcolor}\sffamily\fontsize{10.000000}{12.000000}\selectfont 1.2}%
\end{pgfscope}%
\begin{pgfscope}%
\pgfpathrectangle{\pgfqpoint{4.482986in}{0.438889in}}{\pgfqpoint{3.318403in}{2.263889in}}%
\pgfusepath{clip}%
\pgfsetrectcap%
\pgfsetroundjoin%
\pgfsetlinewidth{1.505625pt}%
\definecolor{currentstroke}{rgb}{0.121569,0.466667,0.705882}%
\pgfsetstrokecolor{currentstroke}%
\pgfsetdash{}{0pt}%
\pgfpathmoveto{\pgfqpoint{4.633823in}{0.764749in}}%
\pgfpathlineto{\pgfqpoint{4.784609in}{0.767561in}}%
\pgfpathlineto{\pgfqpoint{4.935395in}{0.756919in}}%
\pgfpathlineto{\pgfqpoint{5.086181in}{0.759788in}}%
\pgfpathlineto{\pgfqpoint{5.236967in}{0.768130in}}%
\pgfpathlineto{\pgfqpoint{5.387753in}{0.786866in}}%
\pgfpathlineto{\pgfqpoint{5.538539in}{0.792462in}}%
\pgfpathlineto{\pgfqpoint{5.689325in}{0.792702in}}%
\pgfpathlineto{\pgfqpoint{5.840111in}{0.835332in}}%
\pgfpathlineto{\pgfqpoint{5.990897in}{0.786811in}}%
\pgfpathlineto{\pgfqpoint{6.141683in}{0.804543in}}%
\pgfpathlineto{\pgfqpoint{6.292469in}{0.828774in}}%
\pgfpathlineto{\pgfqpoint{6.443255in}{0.864559in}}%
\pgfpathlineto{\pgfqpoint{6.594041in}{0.851787in}}%
\pgfpathlineto{\pgfqpoint{6.744827in}{1.088990in}}%
\pgfpathlineto{\pgfqpoint{6.895613in}{1.055189in}}%
\pgfpathlineto{\pgfqpoint{7.046399in}{1.487584in}}%
\pgfpathlineto{\pgfqpoint{7.197185in}{1.634492in}}%
\pgfpathlineto{\pgfqpoint{7.347971in}{0.933548in}}%
\pgfpathlineto{\pgfqpoint{7.498756in}{1.544487in}}%
\pgfpathlineto{\pgfqpoint{7.649542in}{1.363508in}}%
\pgfusepath{stroke}%
\end{pgfscope}%
\begin{pgfscope}%
\pgfpathrectangle{\pgfqpoint{4.482986in}{0.438889in}}{\pgfqpoint{3.318403in}{2.263889in}}%
\pgfusepath{clip}%
\pgfsetrectcap%
\pgfsetroundjoin%
\pgfsetlinewidth{1.505625pt}%
\definecolor{currentstroke}{rgb}{1.000000,0.498039,0.054902}%
\pgfsetstrokecolor{currentstroke}%
\pgfsetdash{}{0pt}%
\pgfpathmoveto{\pgfqpoint{4.633823in}{0.761548in}}%
\pgfpathlineto{\pgfqpoint{4.784609in}{0.765046in}}%
\pgfpathlineto{\pgfqpoint{4.935395in}{0.759543in}}%
\pgfpathlineto{\pgfqpoint{5.086181in}{0.755540in}}%
\pgfpathlineto{\pgfqpoint{5.236967in}{0.763161in}}%
\pgfpathlineto{\pgfqpoint{5.387753in}{0.783863in}}%
\pgfpathlineto{\pgfqpoint{5.538539in}{0.763983in}}%
\pgfpathlineto{\pgfqpoint{5.689325in}{0.776281in}}%
\pgfpathlineto{\pgfqpoint{5.840111in}{0.786858in}}%
\pgfpathlineto{\pgfqpoint{5.990897in}{0.820554in}}%
\pgfpathlineto{\pgfqpoint{6.141683in}{0.771508in}}%
\pgfpathlineto{\pgfqpoint{6.292469in}{0.912148in}}%
\pgfpathlineto{\pgfqpoint{6.443255in}{0.856542in}}%
\pgfpathlineto{\pgfqpoint{6.594041in}{0.916649in}}%
\pgfpathlineto{\pgfqpoint{6.744827in}{1.040257in}}%
\pgfpathlineto{\pgfqpoint{6.895613in}{1.227234in}}%
\pgfpathlineto{\pgfqpoint{7.046399in}{1.061679in}}%
\pgfpathlineto{\pgfqpoint{7.197185in}{1.622452in}}%
\pgfpathlineto{\pgfqpoint{7.347971in}{1.756146in}}%
\pgfpathlineto{\pgfqpoint{7.498756in}{1.565727in}}%
\pgfpathlineto{\pgfqpoint{7.649542in}{1.672714in}}%
\pgfusepath{stroke}%
\end{pgfscope}%
\begin{pgfscope}%
\pgfpathrectangle{\pgfqpoint{4.482986in}{0.438889in}}{\pgfqpoint{3.318403in}{2.263889in}}%
\pgfusepath{clip}%
\pgfsetrectcap%
\pgfsetroundjoin%
\pgfsetlinewidth{1.505625pt}%
\definecolor{currentstroke}{rgb}{0.172549,0.627451,0.172549}%
\pgfsetstrokecolor{currentstroke}%
\pgfsetdash{}{0pt}%
\pgfpathmoveto{\pgfqpoint{4.633823in}{0.757559in}}%
\pgfpathlineto{\pgfqpoint{4.784609in}{0.762641in}}%
\pgfpathlineto{\pgfqpoint{4.935395in}{0.754460in}}%
\pgfpathlineto{\pgfqpoint{5.086181in}{0.770158in}}%
\pgfpathlineto{\pgfqpoint{5.236967in}{0.769649in}}%
\pgfpathlineto{\pgfqpoint{5.387753in}{0.765221in}}%
\pgfpathlineto{\pgfqpoint{5.538539in}{0.786852in}}%
\pgfpathlineto{\pgfqpoint{5.689325in}{0.773442in}}%
\pgfpathlineto{\pgfqpoint{5.840111in}{0.857947in}}%
\pgfpathlineto{\pgfqpoint{5.990897in}{0.789164in}}%
\pgfpathlineto{\pgfqpoint{6.141683in}{0.799222in}}%
\pgfpathlineto{\pgfqpoint{6.292469in}{0.911399in}}%
\pgfpathlineto{\pgfqpoint{6.443255in}{1.002522in}}%
\pgfpathlineto{\pgfqpoint{6.594041in}{1.152909in}}%
\pgfpathlineto{\pgfqpoint{6.744827in}{1.142144in}}%
\pgfpathlineto{\pgfqpoint{6.895613in}{0.841196in}}%
\pgfpathlineto{\pgfqpoint{7.046399in}{1.382422in}}%
\pgfpathlineto{\pgfqpoint{7.197185in}{1.793381in}}%
\pgfpathlineto{\pgfqpoint{7.347971in}{1.474619in}}%
\pgfpathlineto{\pgfqpoint{7.498756in}{2.110144in}}%
\pgfpathlineto{\pgfqpoint{7.649542in}{1.870352in}}%
\pgfusepath{stroke}%
\end{pgfscope}%
\begin{pgfscope}%
\pgfpathrectangle{\pgfqpoint{4.482986in}{0.438889in}}{\pgfqpoint{3.318403in}{2.263889in}}%
\pgfusepath{clip}%
\pgfsetrectcap%
\pgfsetroundjoin%
\pgfsetlinewidth{1.505625pt}%
\definecolor{currentstroke}{rgb}{0.839216,0.152941,0.156863}%
\pgfsetstrokecolor{currentstroke}%
\pgfsetdash{}{0pt}%
\pgfpathmoveto{\pgfqpoint{4.633823in}{0.757423in}}%
\pgfpathlineto{\pgfqpoint{4.784609in}{0.758159in}}%
\pgfpathlineto{\pgfqpoint{4.935395in}{0.759199in}}%
\pgfpathlineto{\pgfqpoint{5.086181in}{0.766876in}}%
\pgfpathlineto{\pgfqpoint{5.236967in}{0.756045in}}%
\pgfpathlineto{\pgfqpoint{5.387753in}{0.768701in}}%
\pgfpathlineto{\pgfqpoint{5.538539in}{0.763696in}}%
\pgfpathlineto{\pgfqpoint{5.689325in}{0.773673in}}%
\pgfpathlineto{\pgfqpoint{5.840111in}{0.802583in}}%
\pgfpathlineto{\pgfqpoint{5.990897in}{0.803897in}}%
\pgfpathlineto{\pgfqpoint{6.141683in}{0.800905in}}%
\pgfpathlineto{\pgfqpoint{6.292469in}{0.784364in}}%
\pgfpathlineto{\pgfqpoint{6.443255in}{0.802315in}}%
\pgfpathlineto{\pgfqpoint{6.594041in}{0.814135in}}%
\pgfpathlineto{\pgfqpoint{6.744827in}{0.808637in}}%
\pgfpathlineto{\pgfqpoint{6.895613in}{0.841750in}}%
\pgfpathlineto{\pgfqpoint{7.046399in}{0.819765in}}%
\pgfpathlineto{\pgfqpoint{7.197185in}{0.831620in}}%
\pgfpathlineto{\pgfqpoint{7.347971in}{0.823799in}}%
\pgfpathlineto{\pgfqpoint{7.498756in}{0.825166in}}%
\pgfpathlineto{\pgfqpoint{7.649542in}{0.820728in}}%
\pgfusepath{stroke}%
\end{pgfscope}%
\begin{pgfscope}%
\pgfpathrectangle{\pgfqpoint{4.482986in}{0.438889in}}{\pgfqpoint{3.318403in}{2.263889in}}%
\pgfusepath{clip}%
\pgfsetrectcap%
\pgfsetroundjoin%
\pgfsetlinewidth{1.505625pt}%
\definecolor{currentstroke}{rgb}{0.580392,0.403922,0.741176}%
\pgfsetstrokecolor{currentstroke}%
\pgfsetdash{}{0pt}%
\pgfpathmoveto{\pgfqpoint{4.633823in}{0.762455in}}%
\pgfpathlineto{\pgfqpoint{4.784609in}{0.759105in}}%
\pgfpathlineto{\pgfqpoint{4.935395in}{0.770185in}}%
\pgfpathlineto{\pgfqpoint{5.086181in}{0.759677in}}%
\pgfpathlineto{\pgfqpoint{5.236967in}{0.771885in}}%
\pgfpathlineto{\pgfqpoint{5.387753in}{0.768289in}}%
\pgfpathlineto{\pgfqpoint{5.538539in}{0.780615in}}%
\pgfpathlineto{\pgfqpoint{5.689325in}{0.786678in}}%
\pgfpathlineto{\pgfqpoint{5.840111in}{0.792690in}}%
\pgfpathlineto{\pgfqpoint{5.990897in}{0.770196in}}%
\pgfpathlineto{\pgfqpoint{6.141683in}{0.777113in}}%
\pgfpathlineto{\pgfqpoint{6.292469in}{0.774979in}}%
\pgfpathlineto{\pgfqpoint{6.443255in}{0.779010in}}%
\pgfpathlineto{\pgfqpoint{6.594041in}{0.781131in}}%
\pgfpathlineto{\pgfqpoint{6.744827in}{0.778297in}}%
\pgfpathlineto{\pgfqpoint{6.895613in}{0.777872in}}%
\pgfpathlineto{\pgfqpoint{7.046399in}{0.781578in}}%
\pgfpathlineto{\pgfqpoint{7.197185in}{0.780584in}}%
\pgfpathlineto{\pgfqpoint{7.347971in}{0.780324in}}%
\pgfpathlineto{\pgfqpoint{7.498756in}{0.780137in}}%
\pgfpathlineto{\pgfqpoint{7.649542in}{0.779765in}}%
\pgfusepath{stroke}%
\end{pgfscope}%
\begin{pgfscope}%
\pgfsetrectcap%
\pgfsetmiterjoin%
\pgfsetlinewidth{0.803000pt}%
\definecolor{currentstroke}{rgb}{0.000000,0.000000,0.000000}%
\pgfsetstrokecolor{currentstroke}%
\pgfsetdash{}{0pt}%
\pgfpathmoveto{\pgfqpoint{4.482986in}{0.438889in}}%
\pgfpathlineto{\pgfqpoint{4.482986in}{2.702778in}}%
\pgfusepath{stroke}%
\end{pgfscope}%
\begin{pgfscope}%
\pgfsetrectcap%
\pgfsetmiterjoin%
\pgfsetlinewidth{0.803000pt}%
\definecolor{currentstroke}{rgb}{0.000000,0.000000,0.000000}%
\pgfsetstrokecolor{currentstroke}%
\pgfsetdash{}{0pt}%
\pgfpathmoveto{\pgfqpoint{7.801389in}{0.438889in}}%
\pgfpathlineto{\pgfqpoint{7.801389in}{2.702778in}}%
\pgfusepath{stroke}%
\end{pgfscope}%
\begin{pgfscope}%
\pgfsetrectcap%
\pgfsetmiterjoin%
\pgfsetlinewidth{0.803000pt}%
\definecolor{currentstroke}{rgb}{0.000000,0.000000,0.000000}%
\pgfsetstrokecolor{currentstroke}%
\pgfsetdash{}{0pt}%
\pgfpathmoveto{\pgfqpoint{4.482986in}{0.438889in}}%
\pgfpathlineto{\pgfqpoint{7.801389in}{0.438889in}}%
\pgfusepath{stroke}%
\end{pgfscope}%
\begin{pgfscope}%
\pgfsetrectcap%
\pgfsetmiterjoin%
\pgfsetlinewidth{0.803000pt}%
\definecolor{currentstroke}{rgb}{0.000000,0.000000,0.000000}%
\pgfsetstrokecolor{currentstroke}%
\pgfsetdash{}{0pt}%
\pgfpathmoveto{\pgfqpoint{4.482986in}{2.702778in}}%
\pgfpathlineto{\pgfqpoint{7.801389in}{2.702778in}}%
\pgfusepath{stroke}%
\end{pgfscope}%
\begin{pgfscope}%
\definecolor{textcolor}{rgb}{0.000000,0.000000,0.000000}%
\pgfsetstrokecolor{textcolor}%
\pgfsetfillcolor{textcolor}%
\pgftext[x=6.142187in,y=2.786111in,,base]{\color{textcolor}\sffamily\fontsize{12.000000}{14.400000}\selectfont \(\displaystyle  N = 128 \)}%
\end{pgfscope}%
\begin{pgfscope}%
\pgfsetbuttcap%
\pgfsetmiterjoin%
\definecolor{currentfill}{rgb}{1.000000,1.000000,1.000000}%
\pgfsetfillcolor{currentfill}%
\pgfsetfillopacity{0.800000}%
\pgfsetlinewidth{1.003750pt}%
\definecolor{currentstroke}{rgb}{0.800000,0.800000,0.800000}%
\pgfsetstrokecolor{currentstroke}%
\pgfsetstrokeopacity{0.800000}%
\pgfsetdash{}{0pt}%
\pgfpathmoveto{\pgfqpoint{4.580208in}{1.572381in}}%
\pgfpathlineto{\pgfqpoint{5.487760in}{1.572381in}}%
\pgfpathquadraticcurveto{\pgfqpoint{5.515537in}{1.572381in}}{\pgfqpoint{5.515537in}{1.600158in}}%
\pgfpathlineto{\pgfqpoint{5.515537in}{2.605556in}}%
\pgfpathquadraticcurveto{\pgfqpoint{5.515537in}{2.633333in}}{\pgfqpoint{5.487760in}{2.633333in}}%
\pgfpathlineto{\pgfqpoint{4.580208in}{2.633333in}}%
\pgfpathquadraticcurveto{\pgfqpoint{4.552431in}{2.633333in}}{\pgfqpoint{4.552431in}{2.605556in}}%
\pgfpathlineto{\pgfqpoint{4.552431in}{1.600158in}}%
\pgfpathquadraticcurveto{\pgfqpoint{4.552431in}{1.572381in}}{\pgfqpoint{4.580208in}{1.572381in}}%
\pgfpathclose%
\pgfusepath{stroke,fill}%
\end{pgfscope}%
\begin{pgfscope}%
\pgfsetrectcap%
\pgfsetroundjoin%
\pgfsetlinewidth{1.505625pt}%
\definecolor{currentstroke}{rgb}{0.121569,0.466667,0.705882}%
\pgfsetstrokecolor{currentstroke}%
\pgfsetdash{}{0pt}%
\pgfpathmoveto{\pgfqpoint{4.607986in}{2.520866in}}%
\pgfpathlineto{\pgfqpoint{4.885764in}{2.520866in}}%
\pgfusepath{stroke}%
\end{pgfscope}%
\begin{pgfscope}%
\definecolor{textcolor}{rgb}{0.000000,0.000000,0.000000}%
\pgfsetstrokecolor{textcolor}%
\pgfsetfillcolor{textcolor}%
\pgftext[x=4.996875in,y=2.472255in,left,base]{\color{textcolor}\sffamily\fontsize{10.000000}{12.000000}\selectfont \(\displaystyle  T = 1.0 \)}%
\end{pgfscope}%
\begin{pgfscope}%
\pgfsetrectcap%
\pgfsetroundjoin%
\pgfsetlinewidth{1.505625pt}%
\definecolor{currentstroke}{rgb}{1.000000,0.498039,0.054902}%
\pgfsetstrokecolor{currentstroke}%
\pgfsetdash{}{0pt}%
\pgfpathmoveto{\pgfqpoint{4.607986in}{2.317009in}}%
\pgfpathlineto{\pgfqpoint{4.885764in}{2.317009in}}%
\pgfusepath{stroke}%
\end{pgfscope}%
\begin{pgfscope}%
\definecolor{textcolor}{rgb}{0.000000,0.000000,0.000000}%
\pgfsetstrokecolor{textcolor}%
\pgfsetfillcolor{textcolor}%
\pgftext[x=4.996875in,y=2.268397in,left,base]{\color{textcolor}\sffamily\fontsize{10.000000}{12.000000}\selectfont \(\displaystyle  T = 1.5 \)}%
\end{pgfscope}%
\begin{pgfscope}%
\pgfsetrectcap%
\pgfsetroundjoin%
\pgfsetlinewidth{1.505625pt}%
\definecolor{currentstroke}{rgb}{0.172549,0.627451,0.172549}%
\pgfsetstrokecolor{currentstroke}%
\pgfsetdash{}{0pt}%
\pgfpathmoveto{\pgfqpoint{4.607986in}{2.113151in}}%
\pgfpathlineto{\pgfqpoint{4.885764in}{2.113151in}}%
\pgfusepath{stroke}%
\end{pgfscope}%
\begin{pgfscope}%
\definecolor{textcolor}{rgb}{0.000000,0.000000,0.000000}%
\pgfsetstrokecolor{textcolor}%
\pgfsetfillcolor{textcolor}%
\pgftext[x=4.996875in,y=2.064540in,left,base]{\color{textcolor}\sffamily\fontsize{10.000000}{12.000000}\selectfont \(\displaystyle  T = 2.0 \)}%
\end{pgfscope}%
\begin{pgfscope}%
\pgfsetrectcap%
\pgfsetroundjoin%
\pgfsetlinewidth{1.505625pt}%
\definecolor{currentstroke}{rgb}{0.839216,0.152941,0.156863}%
\pgfsetstrokecolor{currentstroke}%
\pgfsetdash{}{0pt}%
\pgfpathmoveto{\pgfqpoint{4.607986in}{1.909294in}}%
\pgfpathlineto{\pgfqpoint{4.885764in}{1.909294in}}%
\pgfusepath{stroke}%
\end{pgfscope}%
\begin{pgfscope}%
\definecolor{textcolor}{rgb}{0.000000,0.000000,0.000000}%
\pgfsetstrokecolor{textcolor}%
\pgfsetfillcolor{textcolor}%
\pgftext[x=4.996875in,y=1.860683in,left,base]{\color{textcolor}\sffamily\fontsize{10.000000}{12.000000}\selectfont \(\displaystyle  T = 2.5 \)}%
\end{pgfscope}%
\begin{pgfscope}%
\pgfsetrectcap%
\pgfsetroundjoin%
\pgfsetlinewidth{1.505625pt}%
\definecolor{currentstroke}{rgb}{0.580392,0.403922,0.741176}%
\pgfsetstrokecolor{currentstroke}%
\pgfsetdash{}{0pt}%
\pgfpathmoveto{\pgfqpoint{4.607986in}{1.705437in}}%
\pgfpathlineto{\pgfqpoint{4.885764in}{1.705437in}}%
\pgfusepath{stroke}%
\end{pgfscope}%
\begin{pgfscope}%
\definecolor{textcolor}{rgb}{0.000000,0.000000,0.000000}%
\pgfsetstrokecolor{textcolor}%
\pgfsetfillcolor{textcolor}%
\pgftext[x=4.996875in,y=1.656826in,left,base]{\color{textcolor}\sffamily\fontsize{10.000000}{12.000000}\selectfont \(\displaystyle  T = 3.0 \)}%
\end{pgfscope}%
\end{pgfpicture}%
\makeatother%
\endgroup%
}
\caption{Magnetization $m$ with respect to iterations using Metropolis algorithm}
\label{Fig:Metro}
\end{figure}

We can see generally that the sampling of $ T = 2.5, 3.0 $ converges in $ 2.5 \times 10^8 $ iterations, and the convergence speed is faster as $N$ increases since the absolute value of $m$ decreases. However, for $ T = 1.0, 1.5, 2.0 $, the convergence speed gets slower and slower and finally fails to convergence in $ 2.5 \times 10^8 $ iterations. Generally speaking, using Metropolis algorithm, we need $10^6$ iterations to establish equilibrium at $ N = 16 $ and $10^7$, $10^8$ and $ > 10^8 $ (we guess $10^9$) iterations at $ N = 32, 64, 128 $ respectively.

\subsection{Kinetic Monte Carlo algorithm}

The failure at smaller $T$ indicates the growth of rejection rate: for smaller $T$, the system gets correlated, and therefore a flip always leads to $ \Delta H > 0 $, which are more frequently rejected. Hence, we try and use the kinetic Monte Carlo algorithm. In this algorithm, the sites are classified according to the value of $\sigma_i$ and its neighbors, since $ \Delta H $ depends only on these sites. In the 2-D case, there are 10 classes, and we calculate the acceptance rate
\begin{equation}
A_c = \min \cbr{ \exp \rbr{ -\beta \Delta H_c }, 1 }
\end{equation}
where $ \Delta H_c $ is the increment of Hamiltonian in class $c$ for $ 1 \le c \le 10 $. We also count the number of sites in each class as $n_c$. As a result, we remove the possibility of rejection and sample class a single $c$ with probability
\begin{equation}
P_c \propto A_c n_c.
\end{equation}
We eventually flip a random site in class $c$. To calculate $\hat{F}$, we need to plug in \eqref{Eq:Ave} by
\begin{equation}
\hat{F}_{\mathit{tr}} = \bfrac{ \sum_{ \textit{it} = \textit{START} + 1 }^{\textit{END}} w_{\textit{tr}}^{\textit{it}} F \rbr{\sigma_{\mathit{tr}}^{\mathit{it}} } }{ \sum_{ \textit{it} = \textit{START} + 1 }^{\textit{END}} w_{\textit{tr}}^{\textit{it}} }
\end{equation}
and
\begin{equation}
w_{\textit{tr}}^{\textit{it}} = \bfrac{n^2}{ \sum_{ c = 1 }^{10} \rbr{ n_c A_c }_{\textit{tr}}^{\textit{it}} }
\end{equation}
is the expectation of iterations until next flip using Metropolis--Hastings algorithm. Again ergodicity yields the convergence.

We also test the efficiency of kinetic Monte Carlo using identical settings. The figure is shown in Figure \ref{Fig:KMC}. Note that we use a stack to maintain the data structure such that each transition only takes $ O \rbr{1} $ time complexity. This heavily increase the expense of coding.

\begin{figure}[htbp]
\centering
\scalebox{0.666}{%% Creator: Matplotlib, PGF backend
%%
%% To include the figure in your LaTeX document, write
%%   \input{<filename>.pgf}
%%
%% Make sure the required packages are loaded in your preamble
%%   \usepackage{pgf}
%%
%% Figures using additional raster images can only be included by \input if
%% they are in the same directory as the main LaTeX file. For loading figures
%% from other directories you can use the `import` package
%%   \usepackage{import}
%% and then include the figures with
%%   \import{<path to file>}{<filename>.pgf}
%%
%% Matplotlib used the following preamble
%%   \usepackage{fontspec}
%%   \setmainfont{DejaVuSerif.ttf}[Path=/home/lzh/anaconda3/envs/numana/lib/python3.7/site-packages/matplotlib/mpl-data/fonts/ttf/]
%%   \setsansfont{DejaVuSans.ttf}[Path=/home/lzh/anaconda3/envs/numana/lib/python3.7/site-packages/matplotlib/mpl-data/fonts/ttf/]
%%   \setmonofont{DejaVuSansMono.ttf}[Path=/home/lzh/anaconda3/envs/numana/lib/python3.7/site-packages/matplotlib/mpl-data/fonts/ttf/]
%%
\begingroup%
\makeatletter%
\begin{pgfpicture}%
\pgfpathrectangle{\pgfpointorigin}{\pgfqpoint{8.000000in}{6.000000in}}%
\pgfusepath{use as bounding box, clip}%
\begin{pgfscope}%
\pgfsetbuttcap%
\pgfsetmiterjoin%
\definecolor{currentfill}{rgb}{1.000000,1.000000,1.000000}%
\pgfsetfillcolor{currentfill}%
\pgfsetlinewidth{0.000000pt}%
\definecolor{currentstroke}{rgb}{1.000000,1.000000,1.000000}%
\pgfsetstrokecolor{currentstroke}%
\pgfsetdash{}{0pt}%
\pgfpathmoveto{\pgfqpoint{0.000000in}{0.000000in}}%
\pgfpathlineto{\pgfqpoint{8.000000in}{0.000000in}}%
\pgfpathlineto{\pgfqpoint{8.000000in}{6.000000in}}%
\pgfpathlineto{\pgfqpoint{0.000000in}{6.000000in}}%
\pgfpathclose%
\pgfusepath{fill}%
\end{pgfscope}%
\begin{pgfscope}%
\pgfsetbuttcap%
\pgfsetmiterjoin%
\definecolor{currentfill}{rgb}{1.000000,1.000000,1.000000}%
\pgfsetfillcolor{currentfill}%
\pgfsetlinewidth{0.000000pt}%
\definecolor{currentstroke}{rgb}{0.000000,0.000000,0.000000}%
\pgfsetstrokecolor{currentstroke}%
\pgfsetstrokeopacity{0.000000}%
\pgfsetdash{}{0pt}%
\pgfpathmoveto{\pgfqpoint{0.672569in}{3.363889in}}%
\pgfpathlineto{\pgfqpoint{3.876389in}{3.363889in}}%
\pgfpathlineto{\pgfqpoint{3.876389in}{5.627778in}}%
\pgfpathlineto{\pgfqpoint{0.672569in}{5.627778in}}%
\pgfpathclose%
\pgfusepath{fill}%
\end{pgfscope}%
\begin{pgfscope}%
\pgfpathrectangle{\pgfqpoint{0.672569in}{3.363889in}}{\pgfqpoint{3.203819in}{2.263889in}}%
\pgfusepath{clip}%
\pgfsetbuttcap%
\pgfsetroundjoin%
\definecolor{currentfill}{rgb}{0.121569,0.466667,0.705882}%
\pgfsetfillcolor{currentfill}%
\pgfsetlinewidth{1.003750pt}%
\definecolor{currentstroke}{rgb}{0.121569,0.466667,0.705882}%
\pgfsetstrokecolor{currentstroke}%
\pgfsetdash{}{0pt}%
\pgfsys@defobject{currentmarker}{\pgfqpoint{-0.009821in}{-0.009821in}}{\pgfqpoint{0.009821in}{0.009821in}}{%
\pgfpathmoveto{\pgfqpoint{0.000000in}{-0.009821in}}%
\pgfpathcurveto{\pgfqpoint{0.002605in}{-0.009821in}}{\pgfqpoint{0.005103in}{-0.008786in}}{\pgfqpoint{0.006944in}{-0.006944in}}%
\pgfpathcurveto{\pgfqpoint{0.008786in}{-0.005103in}}{\pgfqpoint{0.009821in}{-0.002605in}}{\pgfqpoint{0.009821in}{0.000000in}}%
\pgfpathcurveto{\pgfqpoint{0.009821in}{0.002605in}}{\pgfqpoint{0.008786in}{0.005103in}}{\pgfqpoint{0.006944in}{0.006944in}}%
\pgfpathcurveto{\pgfqpoint{0.005103in}{0.008786in}}{\pgfqpoint{0.002605in}{0.009821in}}{\pgfqpoint{0.000000in}{0.009821in}}%
\pgfpathcurveto{\pgfqpoint{-0.002605in}{0.009821in}}{\pgfqpoint{-0.005103in}{0.008786in}}{\pgfqpoint{-0.006944in}{0.006944in}}%
\pgfpathcurveto{\pgfqpoint{-0.008786in}{0.005103in}}{\pgfqpoint{-0.009821in}{0.002605in}}{\pgfqpoint{-0.009821in}{0.000000in}}%
\pgfpathcurveto{\pgfqpoint{-0.009821in}{-0.002605in}}{\pgfqpoint{-0.008786in}{-0.005103in}}{\pgfqpoint{-0.006944in}{-0.006944in}}%
\pgfpathcurveto{\pgfqpoint{-0.005103in}{-0.008786in}}{\pgfqpoint{-0.002605in}{-0.009821in}}{\pgfqpoint{0.000000in}{-0.009821in}}%
\pgfpathclose%
\pgfusepath{stroke,fill}%
}%
\begin{pgfscope}%
\pgfsys@transformshift{0.818198in}{4.316983in}%
\pgfsys@useobject{currentmarker}{}%
\end{pgfscope}%
\begin{pgfscope}%
\pgfsys@transformshift{0.963777in}{5.282805in}%
\pgfsys@useobject{currentmarker}{}%
\end{pgfscope}%
\begin{pgfscope}%
\pgfsys@transformshift{1.109356in}{5.286117in}%
\pgfsys@useobject{currentmarker}{}%
\end{pgfscope}%
\begin{pgfscope}%
\pgfsys@transformshift{1.254936in}{5.287603in}%
\pgfsys@useobject{currentmarker}{}%
\end{pgfscope}%
\begin{pgfscope}%
\pgfsys@transformshift{1.400515in}{5.287631in}%
\pgfsys@useobject{currentmarker}{}%
\end{pgfscope}%
\begin{pgfscope}%
\pgfsys@transformshift{1.546095in}{5.288392in}%
\pgfsys@useobject{currentmarker}{}%
\end{pgfscope}%
\begin{pgfscope}%
\pgfsys@transformshift{1.691674in}{5.288394in}%
\pgfsys@useobject{currentmarker}{}%
\end{pgfscope}%
\begin{pgfscope}%
\pgfsys@transformshift{1.837253in}{5.288391in}%
\pgfsys@useobject{currentmarker}{}%
\end{pgfscope}%
\begin{pgfscope}%
\pgfsys@transformshift{1.982833in}{5.288391in}%
\pgfsys@useobject{currentmarker}{}%
\end{pgfscope}%
\begin{pgfscope}%
\pgfsys@transformshift{2.128412in}{5.288392in}%
\pgfsys@useobject{currentmarker}{}%
\end{pgfscope}%
\begin{pgfscope}%
\pgfsys@transformshift{2.273992in}{5.288391in}%
\pgfsys@useobject{currentmarker}{}%
\end{pgfscope}%
\begin{pgfscope}%
\pgfsys@transformshift{2.419571in}{5.288391in}%
\pgfsys@useobject{currentmarker}{}%
\end{pgfscope}%
\begin{pgfscope}%
\pgfsys@transformshift{2.565150in}{5.288392in}%
\pgfsys@useobject{currentmarker}{}%
\end{pgfscope}%
\begin{pgfscope}%
\pgfsys@transformshift{2.710730in}{5.288392in}%
\pgfsys@useobject{currentmarker}{}%
\end{pgfscope}%
\begin{pgfscope}%
\pgfsys@transformshift{2.856309in}{5.288392in}%
\pgfsys@useobject{currentmarker}{}%
\end{pgfscope}%
\begin{pgfscope}%
\pgfsys@transformshift{3.001889in}{5.288392in}%
\pgfsys@useobject{currentmarker}{}%
\end{pgfscope}%
\begin{pgfscope}%
\pgfsys@transformshift{3.147468in}{5.288392in}%
\pgfsys@useobject{currentmarker}{}%
\end{pgfscope}%
\begin{pgfscope}%
\pgfsys@transformshift{3.293047in}{5.288392in}%
\pgfsys@useobject{currentmarker}{}%
\end{pgfscope}%
\begin{pgfscope}%
\pgfsys@transformshift{3.438627in}{5.288392in}%
\pgfsys@useobject{currentmarker}{}%
\end{pgfscope}%
\begin{pgfscope}%
\pgfsys@transformshift{3.584206in}{5.288392in}%
\pgfsys@useobject{currentmarker}{}%
\end{pgfscope}%
\begin{pgfscope}%
\pgfsys@transformshift{3.729786in}{5.288392in}%
\pgfsys@useobject{currentmarker}{}%
\end{pgfscope}%
\end{pgfscope}%
\begin{pgfscope}%
\pgfpathrectangle{\pgfqpoint{0.672569in}{3.363889in}}{\pgfqpoint{3.203819in}{2.263889in}}%
\pgfusepath{clip}%
\pgfsetbuttcap%
\pgfsetroundjoin%
\definecolor{currentfill}{rgb}{0.121569,0.466667,0.705882}%
\pgfsetfillcolor{currentfill}%
\pgfsetfillopacity{0.300000}%
\pgfsetlinewidth{1.003750pt}%
\definecolor{currentstroke}{rgb}{0.121569,0.466667,0.705882}%
\pgfsetstrokecolor{currentstroke}%
\pgfsetstrokeopacity{0.300000}%
\pgfsetdash{}{0pt}%
\pgfpathmoveto{\pgfqpoint{0.818198in}{5.167173in}}%
\pgfpathlineto{\pgfqpoint{0.818198in}{3.466793in}}%
\pgfpathlineto{\pgfqpoint{0.963777in}{5.268307in}}%
\pgfpathlineto{\pgfqpoint{1.109356in}{5.281237in}}%
\pgfpathlineto{\pgfqpoint{1.254936in}{5.285544in}}%
\pgfpathlineto{\pgfqpoint{1.400515in}{5.285990in}}%
\pgfpathlineto{\pgfqpoint{1.546095in}{5.288386in}}%
\pgfpathlineto{\pgfqpoint{1.691674in}{5.288391in}}%
\pgfpathlineto{\pgfqpoint{1.837253in}{5.288386in}}%
\pgfpathlineto{\pgfqpoint{1.982833in}{5.288391in}}%
\pgfpathlineto{\pgfqpoint{2.128412in}{5.288390in}}%
\pgfpathlineto{\pgfqpoint{2.273992in}{5.288389in}}%
\pgfpathlineto{\pgfqpoint{2.419571in}{5.288390in}}%
\pgfpathlineto{\pgfqpoint{2.565150in}{5.288391in}}%
\pgfpathlineto{\pgfqpoint{2.710730in}{5.288392in}}%
\pgfpathlineto{\pgfqpoint{2.856309in}{5.288392in}}%
\pgfpathlineto{\pgfqpoint{3.001889in}{5.288391in}}%
\pgfpathlineto{\pgfqpoint{3.147468in}{5.288392in}}%
\pgfpathlineto{\pgfqpoint{3.293047in}{5.288392in}}%
\pgfpathlineto{\pgfqpoint{3.438627in}{5.288392in}}%
\pgfpathlineto{\pgfqpoint{3.584206in}{5.288392in}}%
\pgfpathlineto{\pgfqpoint{3.729786in}{5.288392in}}%
\pgfpathlineto{\pgfqpoint{3.729786in}{5.288392in}}%
\pgfpathlineto{\pgfqpoint{3.729786in}{5.288392in}}%
\pgfpathlineto{\pgfqpoint{3.584206in}{5.288392in}}%
\pgfpathlineto{\pgfqpoint{3.438627in}{5.288392in}}%
\pgfpathlineto{\pgfqpoint{3.293047in}{5.288392in}}%
\pgfpathlineto{\pgfqpoint{3.147468in}{5.288392in}}%
\pgfpathlineto{\pgfqpoint{3.001889in}{5.288392in}}%
\pgfpathlineto{\pgfqpoint{2.856309in}{5.288392in}}%
\pgfpathlineto{\pgfqpoint{2.710730in}{5.288392in}}%
\pgfpathlineto{\pgfqpoint{2.565150in}{5.288393in}}%
\pgfpathlineto{\pgfqpoint{2.419571in}{5.288392in}}%
\pgfpathlineto{\pgfqpoint{2.273992in}{5.288393in}}%
\pgfpathlineto{\pgfqpoint{2.128412in}{5.288393in}}%
\pgfpathlineto{\pgfqpoint{1.982833in}{5.288392in}}%
\pgfpathlineto{\pgfqpoint{1.837253in}{5.288396in}}%
\pgfpathlineto{\pgfqpoint{1.691674in}{5.288397in}}%
\pgfpathlineto{\pgfqpoint{1.546095in}{5.288397in}}%
\pgfpathlineto{\pgfqpoint{1.400515in}{5.289272in}}%
\pgfpathlineto{\pgfqpoint{1.254936in}{5.289662in}}%
\pgfpathlineto{\pgfqpoint{1.109356in}{5.290998in}}%
\pgfpathlineto{\pgfqpoint{0.963777in}{5.297302in}}%
\pgfpathlineto{\pgfqpoint{0.818198in}{5.167173in}}%
\pgfpathclose%
\pgfusepath{stroke,fill}%
\end{pgfscope}%
\begin{pgfscope}%
\pgfpathrectangle{\pgfqpoint{0.672569in}{3.363889in}}{\pgfqpoint{3.203819in}{2.263889in}}%
\pgfusepath{clip}%
\pgfsetbuttcap%
\pgfsetroundjoin%
\definecolor{currentfill}{rgb}{1.000000,0.498039,0.054902}%
\pgfsetfillcolor{currentfill}%
\pgfsetlinewidth{1.003750pt}%
\definecolor{currentstroke}{rgb}{1.000000,0.498039,0.054902}%
\pgfsetstrokecolor{currentstroke}%
\pgfsetdash{}{0pt}%
\pgfsys@defobject{currentmarker}{\pgfqpoint{-0.009821in}{-0.009821in}}{\pgfqpoint{0.009821in}{0.009821in}}{%
\pgfpathmoveto{\pgfqpoint{0.000000in}{-0.009821in}}%
\pgfpathcurveto{\pgfqpoint{0.002605in}{-0.009821in}}{\pgfqpoint{0.005103in}{-0.008786in}}{\pgfqpoint{0.006944in}{-0.006944in}}%
\pgfpathcurveto{\pgfqpoint{0.008786in}{-0.005103in}}{\pgfqpoint{0.009821in}{-0.002605in}}{\pgfqpoint{0.009821in}{0.000000in}}%
\pgfpathcurveto{\pgfqpoint{0.009821in}{0.002605in}}{\pgfqpoint{0.008786in}{0.005103in}}{\pgfqpoint{0.006944in}{0.006944in}}%
\pgfpathcurveto{\pgfqpoint{0.005103in}{0.008786in}}{\pgfqpoint{0.002605in}{0.009821in}}{\pgfqpoint{0.000000in}{0.009821in}}%
\pgfpathcurveto{\pgfqpoint{-0.002605in}{0.009821in}}{\pgfqpoint{-0.005103in}{0.008786in}}{\pgfqpoint{-0.006944in}{0.006944in}}%
\pgfpathcurveto{\pgfqpoint{-0.008786in}{0.005103in}}{\pgfqpoint{-0.009821in}{0.002605in}}{\pgfqpoint{-0.009821in}{0.000000in}}%
\pgfpathcurveto{\pgfqpoint{-0.009821in}{-0.002605in}}{\pgfqpoint{-0.008786in}{-0.005103in}}{\pgfqpoint{-0.006944in}{-0.006944in}}%
\pgfpathcurveto{\pgfqpoint{-0.005103in}{-0.008786in}}{\pgfqpoint{-0.002605in}{-0.009821in}}{\pgfqpoint{0.000000in}{-0.009821in}}%
\pgfpathclose%
\pgfusepath{stroke,fill}%
}%
\begin{pgfscope}%
\pgfsys@transformshift{0.818198in}{4.616060in}%
\pgfsys@useobject{currentmarker}{}%
\end{pgfscope}%
\begin{pgfscope}%
\pgfsys@transformshift{0.963777in}{5.254836in}%
\pgfsys@useobject{currentmarker}{}%
\end{pgfscope}%
\begin{pgfscope}%
\pgfsys@transformshift{1.109356in}{5.266997in}%
\pgfsys@useobject{currentmarker}{}%
\end{pgfscope}%
\begin{pgfscope}%
\pgfsys@transformshift{1.254936in}{5.273411in}%
\pgfsys@useobject{currentmarker}{}%
\end{pgfscope}%
\begin{pgfscope}%
\pgfsys@transformshift{1.400515in}{5.276051in}%
\pgfsys@useobject{currentmarker}{}%
\end{pgfscope}%
\begin{pgfscope}%
\pgfsys@transformshift{1.546095in}{5.275714in}%
\pgfsys@useobject{currentmarker}{}%
\end{pgfscope}%
\begin{pgfscope}%
\pgfsys@transformshift{1.691674in}{5.275936in}%
\pgfsys@useobject{currentmarker}{}%
\end{pgfscope}%
\begin{pgfscope}%
\pgfsys@transformshift{1.837253in}{5.275842in}%
\pgfsys@useobject{currentmarker}{}%
\end{pgfscope}%
\begin{pgfscope}%
\pgfsys@transformshift{1.982833in}{5.275911in}%
\pgfsys@useobject{currentmarker}{}%
\end{pgfscope}%
\begin{pgfscope}%
\pgfsys@transformshift{2.128412in}{5.276057in}%
\pgfsys@useobject{currentmarker}{}%
\end{pgfscope}%
\begin{pgfscope}%
\pgfsys@transformshift{2.273992in}{5.275961in}%
\pgfsys@useobject{currentmarker}{}%
\end{pgfscope}%
\begin{pgfscope}%
\pgfsys@transformshift{2.419571in}{5.276023in}%
\pgfsys@useobject{currentmarker}{}%
\end{pgfscope}%
\begin{pgfscope}%
\pgfsys@transformshift{2.565150in}{5.276019in}%
\pgfsys@useobject{currentmarker}{}%
\end{pgfscope}%
\begin{pgfscope}%
\pgfsys@transformshift{2.710730in}{5.276009in}%
\pgfsys@useobject{currentmarker}{}%
\end{pgfscope}%
\begin{pgfscope}%
\pgfsys@transformshift{2.856309in}{5.275978in}%
\pgfsys@useobject{currentmarker}{}%
\end{pgfscope}%
\begin{pgfscope}%
\pgfsys@transformshift{3.001889in}{5.275988in}%
\pgfsys@useobject{currentmarker}{}%
\end{pgfscope}%
\begin{pgfscope}%
\pgfsys@transformshift{3.147468in}{5.275985in}%
\pgfsys@useobject{currentmarker}{}%
\end{pgfscope}%
\begin{pgfscope}%
\pgfsys@transformshift{3.293047in}{5.275997in}%
\pgfsys@useobject{currentmarker}{}%
\end{pgfscope}%
\begin{pgfscope}%
\pgfsys@transformshift{3.438627in}{5.275991in}%
\pgfsys@useobject{currentmarker}{}%
\end{pgfscope}%
\begin{pgfscope}%
\pgfsys@transformshift{3.584206in}{5.275984in}%
\pgfsys@useobject{currentmarker}{}%
\end{pgfscope}%
\begin{pgfscope}%
\pgfsys@transformshift{3.729786in}{5.275988in}%
\pgfsys@useobject{currentmarker}{}%
\end{pgfscope}%
\end{pgfscope}%
\begin{pgfscope}%
\pgfpathrectangle{\pgfqpoint{0.672569in}{3.363889in}}{\pgfqpoint{3.203819in}{2.263889in}}%
\pgfusepath{clip}%
\pgfsetbuttcap%
\pgfsetroundjoin%
\definecolor{currentfill}{rgb}{1.000000,0.498039,0.054902}%
\pgfsetfillcolor{currentfill}%
\pgfsetfillopacity{0.300000}%
\pgfsetlinewidth{1.003750pt}%
\definecolor{currentstroke}{rgb}{1.000000,0.498039,0.054902}%
\pgfsetstrokecolor{currentstroke}%
\pgfsetstrokeopacity{0.300000}%
\pgfsetdash{}{0pt}%
\pgfpathmoveto{\pgfqpoint{0.818198in}{5.524874in}}%
\pgfpathlineto{\pgfqpoint{0.818198in}{3.707247in}}%
\pgfpathlineto{\pgfqpoint{0.963777in}{5.196656in}}%
\pgfpathlineto{\pgfqpoint{1.109356in}{5.245687in}}%
\pgfpathlineto{\pgfqpoint{1.254936in}{5.265765in}}%
\pgfpathlineto{\pgfqpoint{1.400515in}{5.275227in}}%
\pgfpathlineto{\pgfqpoint{1.546095in}{5.275287in}}%
\pgfpathlineto{\pgfqpoint{1.691674in}{5.275305in}}%
\pgfpathlineto{\pgfqpoint{1.837253in}{5.275590in}}%
\pgfpathlineto{\pgfqpoint{1.982833in}{5.275831in}}%
\pgfpathlineto{\pgfqpoint{2.128412in}{5.275917in}}%
\pgfpathlineto{\pgfqpoint{2.273992in}{5.275883in}}%
\pgfpathlineto{\pgfqpoint{2.419571in}{5.275933in}}%
\pgfpathlineto{\pgfqpoint{2.565150in}{5.275964in}}%
\pgfpathlineto{\pgfqpoint{2.710730in}{5.275982in}}%
\pgfpathlineto{\pgfqpoint{2.856309in}{5.275949in}}%
\pgfpathlineto{\pgfqpoint{3.001889in}{5.275977in}}%
\pgfpathlineto{\pgfqpoint{3.147468in}{5.275969in}}%
\pgfpathlineto{\pgfqpoint{3.293047in}{5.275977in}}%
\pgfpathlineto{\pgfqpoint{3.438627in}{5.275969in}}%
\pgfpathlineto{\pgfqpoint{3.584206in}{5.275976in}}%
\pgfpathlineto{\pgfqpoint{3.729786in}{5.275978in}}%
\pgfpathlineto{\pgfqpoint{3.729786in}{5.275999in}}%
\pgfpathlineto{\pgfqpoint{3.729786in}{5.275999in}}%
\pgfpathlineto{\pgfqpoint{3.584206in}{5.275991in}}%
\pgfpathlineto{\pgfqpoint{3.438627in}{5.276014in}}%
\pgfpathlineto{\pgfqpoint{3.293047in}{5.276016in}}%
\pgfpathlineto{\pgfqpoint{3.147468in}{5.276000in}}%
\pgfpathlineto{\pgfqpoint{3.001889in}{5.275998in}}%
\pgfpathlineto{\pgfqpoint{2.856309in}{5.276007in}}%
\pgfpathlineto{\pgfqpoint{2.710730in}{5.276036in}}%
\pgfpathlineto{\pgfqpoint{2.565150in}{5.276074in}}%
\pgfpathlineto{\pgfqpoint{2.419571in}{5.276113in}}%
\pgfpathlineto{\pgfqpoint{2.273992in}{5.276038in}}%
\pgfpathlineto{\pgfqpoint{2.128412in}{5.276197in}}%
\pgfpathlineto{\pgfqpoint{1.982833in}{5.275992in}}%
\pgfpathlineto{\pgfqpoint{1.837253in}{5.276095in}}%
\pgfpathlineto{\pgfqpoint{1.691674in}{5.276568in}}%
\pgfpathlineto{\pgfqpoint{1.546095in}{5.276141in}}%
\pgfpathlineto{\pgfqpoint{1.400515in}{5.276875in}}%
\pgfpathlineto{\pgfqpoint{1.254936in}{5.281057in}}%
\pgfpathlineto{\pgfqpoint{1.109356in}{5.288307in}}%
\pgfpathlineto{\pgfqpoint{0.963777in}{5.313017in}}%
\pgfpathlineto{\pgfqpoint{0.818198in}{5.524874in}}%
\pgfpathclose%
\pgfusepath{stroke,fill}%
\end{pgfscope}%
\begin{pgfscope}%
\pgfpathrectangle{\pgfqpoint{0.672569in}{3.363889in}}{\pgfqpoint{3.203819in}{2.263889in}}%
\pgfusepath{clip}%
\pgfsetbuttcap%
\pgfsetroundjoin%
\definecolor{currentfill}{rgb}{0.172549,0.627451,0.172549}%
\pgfsetfillcolor{currentfill}%
\pgfsetlinewidth{1.003750pt}%
\definecolor{currentstroke}{rgb}{0.172549,0.627451,0.172549}%
\pgfsetstrokecolor{currentstroke}%
\pgfsetdash{}{0pt}%
\pgfsys@defobject{currentmarker}{\pgfqpoint{-0.009821in}{-0.009821in}}{\pgfqpoint{0.009821in}{0.009821in}}{%
\pgfpathmoveto{\pgfqpoint{0.000000in}{-0.009821in}}%
\pgfpathcurveto{\pgfqpoint{0.002605in}{-0.009821in}}{\pgfqpoint{0.005103in}{-0.008786in}}{\pgfqpoint{0.006944in}{-0.006944in}}%
\pgfpathcurveto{\pgfqpoint{0.008786in}{-0.005103in}}{\pgfqpoint{0.009821in}{-0.002605in}}{\pgfqpoint{0.009821in}{0.000000in}}%
\pgfpathcurveto{\pgfqpoint{0.009821in}{0.002605in}}{\pgfqpoint{0.008786in}{0.005103in}}{\pgfqpoint{0.006944in}{0.006944in}}%
\pgfpathcurveto{\pgfqpoint{0.005103in}{0.008786in}}{\pgfqpoint{0.002605in}{0.009821in}}{\pgfqpoint{0.000000in}{0.009821in}}%
\pgfpathcurveto{\pgfqpoint{-0.002605in}{0.009821in}}{\pgfqpoint{-0.005103in}{0.008786in}}{\pgfqpoint{-0.006944in}{0.006944in}}%
\pgfpathcurveto{\pgfqpoint{-0.008786in}{0.005103in}}{\pgfqpoint{-0.009821in}{0.002605in}}{\pgfqpoint{-0.009821in}{0.000000in}}%
\pgfpathcurveto{\pgfqpoint{-0.009821in}{-0.002605in}}{\pgfqpoint{-0.008786in}{-0.005103in}}{\pgfqpoint{-0.006944in}{-0.006944in}}%
\pgfpathcurveto{\pgfqpoint{-0.005103in}{-0.008786in}}{\pgfqpoint{-0.002605in}{-0.009821in}}{\pgfqpoint{0.000000in}{-0.009821in}}%
\pgfpathclose%
\pgfusepath{stroke,fill}%
}%
\begin{pgfscope}%
\pgfsys@transformshift{0.818198in}{4.210018in}%
\pgfsys@useobject{currentmarker}{}%
\end{pgfscope}%
\begin{pgfscope}%
\pgfsys@transformshift{0.963777in}{5.101094in}%
\pgfsys@useobject{currentmarker}{}%
\end{pgfscope}%
\begin{pgfscope}%
\pgfsys@transformshift{1.109356in}{5.081234in}%
\pgfsys@useobject{currentmarker}{}%
\end{pgfscope}%
\begin{pgfscope}%
\pgfsys@transformshift{1.254936in}{5.136413in}%
\pgfsys@useobject{currentmarker}{}%
\end{pgfscope}%
\begin{pgfscope}%
\pgfsys@transformshift{1.400515in}{5.161371in}%
\pgfsys@useobject{currentmarker}{}%
\end{pgfscope}%
\begin{pgfscope}%
\pgfsys@transformshift{1.546095in}{5.160666in}%
\pgfsys@useobject{currentmarker}{}%
\end{pgfscope}%
\begin{pgfscope}%
\pgfsys@transformshift{1.691674in}{5.157862in}%
\pgfsys@useobject{currentmarker}{}%
\end{pgfscope}%
\begin{pgfscope}%
\pgfsys@transformshift{1.837253in}{5.162953in}%
\pgfsys@useobject{currentmarker}{}%
\end{pgfscope}%
\begin{pgfscope}%
\pgfsys@transformshift{1.982833in}{5.156626in}%
\pgfsys@useobject{currentmarker}{}%
\end{pgfscope}%
\begin{pgfscope}%
\pgfsys@transformshift{2.128412in}{5.157059in}%
\pgfsys@useobject{currentmarker}{}%
\end{pgfscope}%
\begin{pgfscope}%
\pgfsys@transformshift{2.273992in}{5.158767in}%
\pgfsys@useobject{currentmarker}{}%
\end{pgfscope}%
\begin{pgfscope}%
\pgfsys@transformshift{2.419571in}{5.160370in}%
\pgfsys@useobject{currentmarker}{}%
\end{pgfscope}%
\begin{pgfscope}%
\pgfsys@transformshift{2.565150in}{5.158807in}%
\pgfsys@useobject{currentmarker}{}%
\end{pgfscope}%
\begin{pgfscope}%
\pgfsys@transformshift{2.710730in}{5.158851in}%
\pgfsys@useobject{currentmarker}{}%
\end{pgfscope}%
\begin{pgfscope}%
\pgfsys@transformshift{2.856309in}{5.159179in}%
\pgfsys@useobject{currentmarker}{}%
\end{pgfscope}%
\begin{pgfscope}%
\pgfsys@transformshift{3.001889in}{5.158984in}%
\pgfsys@useobject{currentmarker}{}%
\end{pgfscope}%
\begin{pgfscope}%
\pgfsys@transformshift{3.147468in}{5.159094in}%
\pgfsys@useobject{currentmarker}{}%
\end{pgfscope}%
\begin{pgfscope}%
\pgfsys@transformshift{3.293047in}{5.158780in}%
\pgfsys@useobject{currentmarker}{}%
\end{pgfscope}%
\begin{pgfscope}%
\pgfsys@transformshift{3.438627in}{5.159157in}%
\pgfsys@useobject{currentmarker}{}%
\end{pgfscope}%
\begin{pgfscope}%
\pgfsys@transformshift{3.584206in}{5.159001in}%
\pgfsys@useobject{currentmarker}{}%
\end{pgfscope}%
\begin{pgfscope}%
\pgfsys@transformshift{3.729786in}{5.159114in}%
\pgfsys@useobject{currentmarker}{}%
\end{pgfscope}%
\end{pgfscope}%
\begin{pgfscope}%
\pgfpathrectangle{\pgfqpoint{0.672569in}{3.363889in}}{\pgfqpoint{3.203819in}{2.263889in}}%
\pgfusepath{clip}%
\pgfsetbuttcap%
\pgfsetroundjoin%
\definecolor{currentfill}{rgb}{0.172549,0.627451,0.172549}%
\pgfsetfillcolor{currentfill}%
\pgfsetfillopacity{0.300000}%
\pgfsetlinewidth{1.003750pt}%
\definecolor{currentstroke}{rgb}{0.172549,0.627451,0.172549}%
\pgfsetstrokecolor{currentstroke}%
\pgfsetstrokeopacity{0.300000}%
\pgfsetdash{}{0pt}%
\pgfpathmoveto{\pgfqpoint{0.818198in}{4.864107in}}%
\pgfpathlineto{\pgfqpoint{0.818198in}{3.555929in}}%
\pgfpathlineto{\pgfqpoint{0.963777in}{5.052159in}}%
\pgfpathlineto{\pgfqpoint{1.109356in}{4.966679in}}%
\pgfpathlineto{\pgfqpoint{1.254936in}{5.093987in}}%
\pgfpathlineto{\pgfqpoint{1.400515in}{5.152373in}}%
\pgfpathlineto{\pgfqpoint{1.546095in}{5.146473in}}%
\pgfpathlineto{\pgfqpoint{1.691674in}{5.151364in}}%
\pgfpathlineto{\pgfqpoint{1.837253in}{5.155811in}}%
\pgfpathlineto{\pgfqpoint{1.982833in}{5.147296in}}%
\pgfpathlineto{\pgfqpoint{2.128412in}{5.153267in}}%
\pgfpathlineto{\pgfqpoint{2.273992in}{5.155849in}}%
\pgfpathlineto{\pgfqpoint{2.419571in}{5.158212in}}%
\pgfpathlineto{\pgfqpoint{2.565150in}{5.157597in}}%
\pgfpathlineto{\pgfqpoint{2.710730in}{5.158065in}}%
\pgfpathlineto{\pgfqpoint{2.856309in}{5.157845in}}%
\pgfpathlineto{\pgfqpoint{3.001889in}{5.157989in}}%
\pgfpathlineto{\pgfqpoint{3.147468in}{5.158387in}}%
\pgfpathlineto{\pgfqpoint{3.293047in}{5.158508in}}%
\pgfpathlineto{\pgfqpoint{3.438627in}{5.158859in}}%
\pgfpathlineto{\pgfqpoint{3.584206in}{5.158804in}}%
\pgfpathlineto{\pgfqpoint{3.729786in}{5.159003in}}%
\pgfpathlineto{\pgfqpoint{3.729786in}{5.159225in}}%
\pgfpathlineto{\pgfqpoint{3.729786in}{5.159225in}}%
\pgfpathlineto{\pgfqpoint{3.584206in}{5.159199in}}%
\pgfpathlineto{\pgfqpoint{3.438627in}{5.159455in}}%
\pgfpathlineto{\pgfqpoint{3.293047in}{5.159051in}}%
\pgfpathlineto{\pgfqpoint{3.147468in}{5.159801in}}%
\pgfpathlineto{\pgfqpoint{3.001889in}{5.159978in}}%
\pgfpathlineto{\pgfqpoint{2.856309in}{5.160513in}}%
\pgfpathlineto{\pgfqpoint{2.710730in}{5.159638in}}%
\pgfpathlineto{\pgfqpoint{2.565150in}{5.160016in}}%
\pgfpathlineto{\pgfqpoint{2.419571in}{5.162529in}}%
\pgfpathlineto{\pgfqpoint{2.273992in}{5.161684in}}%
\pgfpathlineto{\pgfqpoint{2.128412in}{5.160852in}}%
\pgfpathlineto{\pgfqpoint{1.982833in}{5.165957in}}%
\pgfpathlineto{\pgfqpoint{1.837253in}{5.170095in}}%
\pgfpathlineto{\pgfqpoint{1.691674in}{5.164361in}}%
\pgfpathlineto{\pgfqpoint{1.546095in}{5.174860in}}%
\pgfpathlineto{\pgfqpoint{1.400515in}{5.170370in}}%
\pgfpathlineto{\pgfqpoint{1.254936in}{5.178838in}}%
\pgfpathlineto{\pgfqpoint{1.109356in}{5.195790in}}%
\pgfpathlineto{\pgfqpoint{0.963777in}{5.150029in}}%
\pgfpathlineto{\pgfqpoint{0.818198in}{4.864107in}}%
\pgfpathclose%
\pgfusepath{stroke,fill}%
\end{pgfscope}%
\begin{pgfscope}%
\pgfpathrectangle{\pgfqpoint{0.672569in}{3.363889in}}{\pgfqpoint{3.203819in}{2.263889in}}%
\pgfusepath{clip}%
\pgfsetbuttcap%
\pgfsetroundjoin%
\definecolor{currentfill}{rgb}{0.839216,0.152941,0.156863}%
\pgfsetfillcolor{currentfill}%
\pgfsetlinewidth{1.003750pt}%
\definecolor{currentstroke}{rgb}{0.839216,0.152941,0.156863}%
\pgfsetstrokecolor{currentstroke}%
\pgfsetdash{}{0pt}%
\pgfsys@defobject{currentmarker}{\pgfqpoint{-0.009821in}{-0.009821in}}{\pgfqpoint{0.009821in}{0.009821in}}{%
\pgfpathmoveto{\pgfqpoint{0.000000in}{-0.009821in}}%
\pgfpathcurveto{\pgfqpoint{0.002605in}{-0.009821in}}{\pgfqpoint{0.005103in}{-0.008786in}}{\pgfqpoint{0.006944in}{-0.006944in}}%
\pgfpathcurveto{\pgfqpoint{0.008786in}{-0.005103in}}{\pgfqpoint{0.009821in}{-0.002605in}}{\pgfqpoint{0.009821in}{0.000000in}}%
\pgfpathcurveto{\pgfqpoint{0.009821in}{0.002605in}}{\pgfqpoint{0.008786in}{0.005103in}}{\pgfqpoint{0.006944in}{0.006944in}}%
\pgfpathcurveto{\pgfqpoint{0.005103in}{0.008786in}}{\pgfqpoint{0.002605in}{0.009821in}}{\pgfqpoint{0.000000in}{0.009821in}}%
\pgfpathcurveto{\pgfqpoint{-0.002605in}{0.009821in}}{\pgfqpoint{-0.005103in}{0.008786in}}{\pgfqpoint{-0.006944in}{0.006944in}}%
\pgfpathcurveto{\pgfqpoint{-0.008786in}{0.005103in}}{\pgfqpoint{-0.009821in}{0.002605in}}{\pgfqpoint{-0.009821in}{0.000000in}}%
\pgfpathcurveto{\pgfqpoint{-0.009821in}{-0.002605in}}{\pgfqpoint{-0.008786in}{-0.005103in}}{\pgfqpoint{-0.006944in}{-0.006944in}}%
\pgfpathcurveto{\pgfqpoint{-0.005103in}{-0.008786in}}{\pgfqpoint{-0.002605in}{-0.009821in}}{\pgfqpoint{0.000000in}{-0.009821in}}%
\pgfpathclose%
\pgfusepath{stroke,fill}%
}%
\begin{pgfscope}%
\pgfsys@transformshift{0.818198in}{4.228213in}%
\pgfsys@useobject{currentmarker}{}%
\end{pgfscope}%
\begin{pgfscope}%
\pgfsys@transformshift{0.963777in}{4.398895in}%
\pgfsys@useobject{currentmarker}{}%
\end{pgfscope}%
\begin{pgfscope}%
\pgfsys@transformshift{1.109356in}{4.420263in}%
\pgfsys@useobject{currentmarker}{}%
\end{pgfscope}%
\begin{pgfscope}%
\pgfsys@transformshift{1.254936in}{4.208862in}%
\pgfsys@useobject{currentmarker}{}%
\end{pgfscope}%
\begin{pgfscope}%
\pgfsys@transformshift{1.400515in}{4.308323in}%
\pgfsys@useobject{currentmarker}{}%
\end{pgfscope}%
\begin{pgfscope}%
\pgfsys@transformshift{1.546095in}{4.311813in}%
\pgfsys@useobject{currentmarker}{}%
\end{pgfscope}%
\begin{pgfscope}%
\pgfsys@transformshift{1.691674in}{4.279686in}%
\pgfsys@useobject{currentmarker}{}%
\end{pgfscope}%
\begin{pgfscope}%
\pgfsys@transformshift{1.837253in}{4.396138in}%
\pgfsys@useobject{currentmarker}{}%
\end{pgfscope}%
\begin{pgfscope}%
\pgfsys@transformshift{1.982833in}{4.310469in}%
\pgfsys@useobject{currentmarker}{}%
\end{pgfscope}%
\begin{pgfscope}%
\pgfsys@transformshift{2.128412in}{4.332981in}%
\pgfsys@useobject{currentmarker}{}%
\end{pgfscope}%
\begin{pgfscope}%
\pgfsys@transformshift{2.273992in}{4.316087in}%
\pgfsys@useobject{currentmarker}{}%
\end{pgfscope}%
\begin{pgfscope}%
\pgfsys@transformshift{2.419571in}{4.331433in}%
\pgfsys@useobject{currentmarker}{}%
\end{pgfscope}%
\begin{pgfscope}%
\pgfsys@transformshift{2.565150in}{4.337534in}%
\pgfsys@useobject{currentmarker}{}%
\end{pgfscope}%
\begin{pgfscope}%
\pgfsys@transformshift{2.710730in}{4.327943in}%
\pgfsys@useobject{currentmarker}{}%
\end{pgfscope}%
\begin{pgfscope}%
\pgfsys@transformshift{2.856309in}{4.327158in}%
\pgfsys@useobject{currentmarker}{}%
\end{pgfscope}%
\begin{pgfscope}%
\pgfsys@transformshift{3.001889in}{4.324664in}%
\pgfsys@useobject{currentmarker}{}%
\end{pgfscope}%
\begin{pgfscope}%
\pgfsys@transformshift{3.147468in}{4.330845in}%
\pgfsys@useobject{currentmarker}{}%
\end{pgfscope}%
\begin{pgfscope}%
\pgfsys@transformshift{3.293047in}{4.334590in}%
\pgfsys@useobject{currentmarker}{}%
\end{pgfscope}%
\begin{pgfscope}%
\pgfsys@transformshift{3.438627in}{4.331444in}%
\pgfsys@useobject{currentmarker}{}%
\end{pgfscope}%
\begin{pgfscope}%
\pgfsys@transformshift{3.584206in}{4.334423in}%
\pgfsys@useobject{currentmarker}{}%
\end{pgfscope}%
\begin{pgfscope}%
\pgfsys@transformshift{3.729786in}{4.332055in}%
\pgfsys@useobject{currentmarker}{}%
\end{pgfscope}%
\end{pgfscope}%
\begin{pgfscope}%
\pgfpathrectangle{\pgfqpoint{0.672569in}{3.363889in}}{\pgfqpoint{3.203819in}{2.263889in}}%
\pgfusepath{clip}%
\pgfsetbuttcap%
\pgfsetroundjoin%
\definecolor{currentfill}{rgb}{0.839216,0.152941,0.156863}%
\pgfsetfillcolor{currentfill}%
\pgfsetfillopacity{0.300000}%
\pgfsetlinewidth{1.003750pt}%
\definecolor{currentstroke}{rgb}{0.839216,0.152941,0.156863}%
\pgfsetstrokecolor{currentstroke}%
\pgfsetstrokeopacity{0.300000}%
\pgfsetdash{}{0pt}%
\pgfpathmoveto{\pgfqpoint{0.818198in}{4.632997in}}%
\pgfpathlineto{\pgfqpoint{0.818198in}{3.823429in}}%
\pgfpathlineto{\pgfqpoint{0.963777in}{4.025557in}}%
\pgfpathlineto{\pgfqpoint{1.109356in}{4.187417in}}%
\pgfpathlineto{\pgfqpoint{1.254936in}{3.924755in}}%
\pgfpathlineto{\pgfqpoint{1.400515in}{4.027147in}}%
\pgfpathlineto{\pgfqpoint{1.546095in}{4.150836in}}%
\pgfpathlineto{\pgfqpoint{1.691674in}{4.115324in}}%
\pgfpathlineto{\pgfqpoint{1.837253in}{4.320828in}}%
\pgfpathlineto{\pgfqpoint{1.982833in}{4.221947in}}%
\pgfpathlineto{\pgfqpoint{2.128412in}{4.296212in}}%
\pgfpathlineto{\pgfqpoint{2.273992in}{4.286930in}}%
\pgfpathlineto{\pgfqpoint{2.419571in}{4.322683in}}%
\pgfpathlineto{\pgfqpoint{2.565150in}{4.305039in}}%
\pgfpathlineto{\pgfqpoint{2.710730in}{4.294802in}}%
\pgfpathlineto{\pgfqpoint{2.856309in}{4.310089in}}%
\pgfpathlineto{\pgfqpoint{3.001889in}{4.317875in}}%
\pgfpathlineto{\pgfqpoint{3.147468in}{4.314177in}}%
\pgfpathlineto{\pgfqpoint{3.293047in}{4.327193in}}%
\pgfpathlineto{\pgfqpoint{3.438627in}{4.328167in}}%
\pgfpathlineto{\pgfqpoint{3.584206in}{4.332070in}}%
\pgfpathlineto{\pgfqpoint{3.729786in}{4.330485in}}%
\pgfpathlineto{\pgfqpoint{3.729786in}{4.333624in}}%
\pgfpathlineto{\pgfqpoint{3.729786in}{4.333624in}}%
\pgfpathlineto{\pgfqpoint{3.584206in}{4.336776in}}%
\pgfpathlineto{\pgfqpoint{3.438627in}{4.334721in}}%
\pgfpathlineto{\pgfqpoint{3.293047in}{4.341987in}}%
\pgfpathlineto{\pgfqpoint{3.147468in}{4.347512in}}%
\pgfpathlineto{\pgfqpoint{3.001889in}{4.331454in}}%
\pgfpathlineto{\pgfqpoint{2.856309in}{4.344227in}}%
\pgfpathlineto{\pgfqpoint{2.710730in}{4.361084in}}%
\pgfpathlineto{\pgfqpoint{2.565150in}{4.370029in}}%
\pgfpathlineto{\pgfqpoint{2.419571in}{4.340184in}}%
\pgfpathlineto{\pgfqpoint{2.273992in}{4.345245in}}%
\pgfpathlineto{\pgfqpoint{2.128412in}{4.369751in}}%
\pgfpathlineto{\pgfqpoint{1.982833in}{4.398991in}}%
\pgfpathlineto{\pgfqpoint{1.837253in}{4.471448in}}%
\pgfpathlineto{\pgfqpoint{1.691674in}{4.444048in}}%
\pgfpathlineto{\pgfqpoint{1.546095in}{4.472790in}}%
\pgfpathlineto{\pgfqpoint{1.400515in}{4.589499in}}%
\pgfpathlineto{\pgfqpoint{1.254936in}{4.492968in}}%
\pgfpathlineto{\pgfqpoint{1.109356in}{4.653109in}}%
\pgfpathlineto{\pgfqpoint{0.963777in}{4.772233in}}%
\pgfpathlineto{\pgfqpoint{0.818198in}{4.632997in}}%
\pgfpathclose%
\pgfusepath{stroke,fill}%
\end{pgfscope}%
\begin{pgfscope}%
\pgfpathrectangle{\pgfqpoint{0.672569in}{3.363889in}}{\pgfqpoint{3.203819in}{2.263889in}}%
\pgfusepath{clip}%
\pgfsetbuttcap%
\pgfsetroundjoin%
\definecolor{currentfill}{rgb}{0.580392,0.403922,0.741176}%
\pgfsetfillcolor{currentfill}%
\pgfsetlinewidth{1.003750pt}%
\definecolor{currentstroke}{rgb}{0.580392,0.403922,0.741176}%
\pgfsetstrokecolor{currentstroke}%
\pgfsetdash{}{0pt}%
\pgfsys@defobject{currentmarker}{\pgfqpoint{-0.009821in}{-0.009821in}}{\pgfqpoint{0.009821in}{0.009821in}}{%
\pgfpathmoveto{\pgfqpoint{0.000000in}{-0.009821in}}%
\pgfpathcurveto{\pgfqpoint{0.002605in}{-0.009821in}}{\pgfqpoint{0.005103in}{-0.008786in}}{\pgfqpoint{0.006944in}{-0.006944in}}%
\pgfpathcurveto{\pgfqpoint{0.008786in}{-0.005103in}}{\pgfqpoint{0.009821in}{-0.002605in}}{\pgfqpoint{0.009821in}{0.000000in}}%
\pgfpathcurveto{\pgfqpoint{0.009821in}{0.002605in}}{\pgfqpoint{0.008786in}{0.005103in}}{\pgfqpoint{0.006944in}{0.006944in}}%
\pgfpathcurveto{\pgfqpoint{0.005103in}{0.008786in}}{\pgfqpoint{0.002605in}{0.009821in}}{\pgfqpoint{0.000000in}{0.009821in}}%
\pgfpathcurveto{\pgfqpoint{-0.002605in}{0.009821in}}{\pgfqpoint{-0.005103in}{0.008786in}}{\pgfqpoint{-0.006944in}{0.006944in}}%
\pgfpathcurveto{\pgfqpoint{-0.008786in}{0.005103in}}{\pgfqpoint{-0.009821in}{0.002605in}}{\pgfqpoint{-0.009821in}{0.000000in}}%
\pgfpathcurveto{\pgfqpoint{-0.009821in}{-0.002605in}}{\pgfqpoint{-0.008786in}{-0.005103in}}{\pgfqpoint{-0.006944in}{-0.006944in}}%
\pgfpathcurveto{\pgfqpoint{-0.005103in}{-0.008786in}}{\pgfqpoint{-0.002605in}{-0.009821in}}{\pgfqpoint{0.000000in}{-0.009821in}}%
\pgfpathclose%
\pgfusepath{stroke,fill}%
}%
\begin{pgfscope}%
\pgfsys@transformshift{0.818198in}{4.109514in}%
\pgfsys@useobject{currentmarker}{}%
\end{pgfscope}%
\begin{pgfscope}%
\pgfsys@transformshift{0.963777in}{3.982178in}%
\pgfsys@useobject{currentmarker}{}%
\end{pgfscope}%
\begin{pgfscope}%
\pgfsys@transformshift{1.109356in}{3.945122in}%
\pgfsys@useobject{currentmarker}{}%
\end{pgfscope}%
\begin{pgfscope}%
\pgfsys@transformshift{1.254936in}{3.957958in}%
\pgfsys@useobject{currentmarker}{}%
\end{pgfscope}%
\begin{pgfscope}%
\pgfsys@transformshift{1.400515in}{4.012700in}%
\pgfsys@useobject{currentmarker}{}%
\end{pgfscope}%
\begin{pgfscope}%
\pgfsys@transformshift{1.546095in}{4.027976in}%
\pgfsys@useobject{currentmarker}{}%
\end{pgfscope}%
\begin{pgfscope}%
\pgfsys@transformshift{1.691674in}{3.988955in}%
\pgfsys@useobject{currentmarker}{}%
\end{pgfscope}%
\begin{pgfscope}%
\pgfsys@transformshift{1.837253in}{3.983187in}%
\pgfsys@useobject{currentmarker}{}%
\end{pgfscope}%
\begin{pgfscope}%
\pgfsys@transformshift{1.982833in}{3.981111in}%
\pgfsys@useobject{currentmarker}{}%
\end{pgfscope}%
\begin{pgfscope}%
\pgfsys@transformshift{2.128412in}{3.975360in}%
\pgfsys@useobject{currentmarker}{}%
\end{pgfscope}%
\begin{pgfscope}%
\pgfsys@transformshift{2.273992in}{3.981842in}%
\pgfsys@useobject{currentmarker}{}%
\end{pgfscope}%
\begin{pgfscope}%
\pgfsys@transformshift{2.419571in}{3.991072in}%
\pgfsys@useobject{currentmarker}{}%
\end{pgfscope}%
\begin{pgfscope}%
\pgfsys@transformshift{2.565150in}{3.987831in}%
\pgfsys@useobject{currentmarker}{}%
\end{pgfscope}%
\begin{pgfscope}%
\pgfsys@transformshift{2.710730in}{3.989218in}%
\pgfsys@useobject{currentmarker}{}%
\end{pgfscope}%
\begin{pgfscope}%
\pgfsys@transformshift{2.856309in}{3.987588in}%
\pgfsys@useobject{currentmarker}{}%
\end{pgfscope}%
\begin{pgfscope}%
\pgfsys@transformshift{3.001889in}{3.991650in}%
\pgfsys@useobject{currentmarker}{}%
\end{pgfscope}%
\begin{pgfscope}%
\pgfsys@transformshift{3.147468in}{3.988983in}%
\pgfsys@useobject{currentmarker}{}%
\end{pgfscope}%
\begin{pgfscope}%
\pgfsys@transformshift{3.293047in}{3.988435in}%
\pgfsys@useobject{currentmarker}{}%
\end{pgfscope}%
\begin{pgfscope}%
\pgfsys@transformshift{3.438627in}{3.989009in}%
\pgfsys@useobject{currentmarker}{}%
\end{pgfscope}%
\begin{pgfscope}%
\pgfsys@transformshift{3.584206in}{3.988020in}%
\pgfsys@useobject{currentmarker}{}%
\end{pgfscope}%
\begin{pgfscope}%
\pgfsys@transformshift{3.729786in}{3.988472in}%
\pgfsys@useobject{currentmarker}{}%
\end{pgfscope}%
\end{pgfscope}%
\begin{pgfscope}%
\pgfpathrectangle{\pgfqpoint{0.672569in}{3.363889in}}{\pgfqpoint{3.203819in}{2.263889in}}%
\pgfusepath{clip}%
\pgfsetbuttcap%
\pgfsetroundjoin%
\definecolor{currentfill}{rgb}{0.580392,0.403922,0.741176}%
\pgfsetfillcolor{currentfill}%
\pgfsetfillopacity{0.300000}%
\pgfsetlinewidth{1.003750pt}%
\definecolor{currentstroke}{rgb}{0.580392,0.403922,0.741176}%
\pgfsetstrokecolor{currentstroke}%
\pgfsetstrokeopacity{0.300000}%
\pgfsetdash{}{0pt}%
\pgfpathmoveto{\pgfqpoint{0.818198in}{4.360825in}}%
\pgfpathlineto{\pgfqpoint{0.818198in}{3.858202in}}%
\pgfpathlineto{\pgfqpoint{0.963777in}{3.801460in}}%
\pgfpathlineto{\pgfqpoint{1.109356in}{3.787640in}}%
\pgfpathlineto{\pgfqpoint{1.254936in}{3.848330in}}%
\pgfpathlineto{\pgfqpoint{1.400515in}{3.978172in}}%
\pgfpathlineto{\pgfqpoint{1.546095in}{3.957815in}}%
\pgfpathlineto{\pgfqpoint{1.691674in}{3.949341in}}%
\pgfpathlineto{\pgfqpoint{1.837253in}{3.946819in}}%
\pgfpathlineto{\pgfqpoint{1.982833in}{3.967830in}}%
\pgfpathlineto{\pgfqpoint{2.128412in}{3.966539in}}%
\pgfpathlineto{\pgfqpoint{2.273992in}{3.973907in}}%
\pgfpathlineto{\pgfqpoint{2.419571in}{3.989063in}}%
\pgfpathlineto{\pgfqpoint{2.565150in}{3.979059in}}%
\pgfpathlineto{\pgfqpoint{2.710730in}{3.981419in}}%
\pgfpathlineto{\pgfqpoint{2.856309in}{3.981452in}}%
\pgfpathlineto{\pgfqpoint{3.001889in}{3.989107in}}%
\pgfpathlineto{\pgfqpoint{3.147468in}{3.986660in}}%
\pgfpathlineto{\pgfqpoint{3.293047in}{3.986880in}}%
\pgfpathlineto{\pgfqpoint{3.438627in}{3.987539in}}%
\pgfpathlineto{\pgfqpoint{3.584206in}{3.987221in}}%
\pgfpathlineto{\pgfqpoint{3.729786in}{3.987238in}}%
\pgfpathlineto{\pgfqpoint{3.729786in}{3.989706in}}%
\pgfpathlineto{\pgfqpoint{3.729786in}{3.989706in}}%
\pgfpathlineto{\pgfqpoint{3.584206in}{3.988820in}}%
\pgfpathlineto{\pgfqpoint{3.438627in}{3.990480in}}%
\pgfpathlineto{\pgfqpoint{3.293047in}{3.989990in}}%
\pgfpathlineto{\pgfqpoint{3.147468in}{3.991305in}}%
\pgfpathlineto{\pgfqpoint{3.001889in}{3.994193in}}%
\pgfpathlineto{\pgfqpoint{2.856309in}{3.993725in}}%
\pgfpathlineto{\pgfqpoint{2.710730in}{3.997016in}}%
\pgfpathlineto{\pgfqpoint{2.565150in}{3.996604in}}%
\pgfpathlineto{\pgfqpoint{2.419571in}{3.993081in}}%
\pgfpathlineto{\pgfqpoint{2.273992in}{3.989777in}}%
\pgfpathlineto{\pgfqpoint{2.128412in}{3.984180in}}%
\pgfpathlineto{\pgfqpoint{1.982833in}{3.994392in}}%
\pgfpathlineto{\pgfqpoint{1.837253in}{4.019554in}}%
\pgfpathlineto{\pgfqpoint{1.691674in}{4.028569in}}%
\pgfpathlineto{\pgfqpoint{1.546095in}{4.098136in}}%
\pgfpathlineto{\pgfqpoint{1.400515in}{4.047228in}}%
\pgfpathlineto{\pgfqpoint{1.254936in}{4.067587in}}%
\pgfpathlineto{\pgfqpoint{1.109356in}{4.102603in}}%
\pgfpathlineto{\pgfqpoint{0.963777in}{4.162896in}}%
\pgfpathlineto{\pgfqpoint{0.818198in}{4.360825in}}%
\pgfpathclose%
\pgfusepath{stroke,fill}%
\end{pgfscope}%
\begin{pgfscope}%
\pgfsetbuttcap%
\pgfsetroundjoin%
\definecolor{currentfill}{rgb}{0.000000,0.000000,0.000000}%
\pgfsetfillcolor{currentfill}%
\pgfsetlinewidth{0.803000pt}%
\definecolor{currentstroke}{rgb}{0.000000,0.000000,0.000000}%
\pgfsetstrokecolor{currentstroke}%
\pgfsetdash{}{0pt}%
\pgfsys@defobject{currentmarker}{\pgfqpoint{0.000000in}{-0.048611in}}{\pgfqpoint{0.000000in}{0.000000in}}{%
\pgfpathmoveto{\pgfqpoint{0.000000in}{0.000000in}}%
\pgfpathlineto{\pgfqpoint{0.000000in}{-0.048611in}}%
\pgfusepath{stroke,fill}%
}%
\begin{pgfscope}%
\pgfsys@transformshift{1.168788in}{3.363889in}%
\pgfsys@useobject{currentmarker}{}%
\end{pgfscope}%
\end{pgfscope}%
\begin{pgfscope}%
\definecolor{textcolor}{rgb}{0.000000,0.000000,0.000000}%
\pgfsetstrokecolor{textcolor}%
\pgfsetfillcolor{textcolor}%
\pgftext[x=1.168788in,y=3.266667in,,top]{\color{textcolor}\sffamily\fontsize{10.000000}{12.000000}\selectfont \(\displaystyle {10^{4}}\)}%
\end{pgfscope}%
\begin{pgfscope}%
\pgfsetbuttcap%
\pgfsetroundjoin%
\definecolor{currentfill}{rgb}{0.000000,0.000000,0.000000}%
\pgfsetfillcolor{currentfill}%
\pgfsetlinewidth{0.803000pt}%
\definecolor{currentstroke}{rgb}{0.000000,0.000000,0.000000}%
\pgfsetstrokecolor{currentstroke}%
\pgfsetdash{}{0pt}%
\pgfsys@defobject{currentmarker}{\pgfqpoint{0.000000in}{-0.048611in}}{\pgfqpoint{0.000000in}{0.000000in}}{%
\pgfpathmoveto{\pgfqpoint{0.000000in}{0.000000in}}%
\pgfpathlineto{\pgfqpoint{0.000000in}{-0.048611in}}%
\pgfusepath{stroke,fill}%
}%
\begin{pgfscope}%
\pgfsys@transformshift{1.751105in}{3.363889in}%
\pgfsys@useobject{currentmarker}{}%
\end{pgfscope}%
\end{pgfscope}%
\begin{pgfscope}%
\definecolor{textcolor}{rgb}{0.000000,0.000000,0.000000}%
\pgfsetstrokecolor{textcolor}%
\pgfsetfillcolor{textcolor}%
\pgftext[x=1.751105in,y=3.266667in,,top]{\color{textcolor}\sffamily\fontsize{10.000000}{12.000000}\selectfont \(\displaystyle {10^{5}}\)}%
\end{pgfscope}%
\begin{pgfscope}%
\pgfsetbuttcap%
\pgfsetroundjoin%
\definecolor{currentfill}{rgb}{0.000000,0.000000,0.000000}%
\pgfsetfillcolor{currentfill}%
\pgfsetlinewidth{0.803000pt}%
\definecolor{currentstroke}{rgb}{0.000000,0.000000,0.000000}%
\pgfsetstrokecolor{currentstroke}%
\pgfsetdash{}{0pt}%
\pgfsys@defobject{currentmarker}{\pgfqpoint{0.000000in}{-0.048611in}}{\pgfqpoint{0.000000in}{0.000000in}}{%
\pgfpathmoveto{\pgfqpoint{0.000000in}{0.000000in}}%
\pgfpathlineto{\pgfqpoint{0.000000in}{-0.048611in}}%
\pgfusepath{stroke,fill}%
}%
\begin{pgfscope}%
\pgfsys@transformshift{2.333423in}{3.363889in}%
\pgfsys@useobject{currentmarker}{}%
\end{pgfscope}%
\end{pgfscope}%
\begin{pgfscope}%
\definecolor{textcolor}{rgb}{0.000000,0.000000,0.000000}%
\pgfsetstrokecolor{textcolor}%
\pgfsetfillcolor{textcolor}%
\pgftext[x=2.333423in,y=3.266667in,,top]{\color{textcolor}\sffamily\fontsize{10.000000}{12.000000}\selectfont \(\displaystyle {10^{6}}\)}%
\end{pgfscope}%
\begin{pgfscope}%
\pgfsetbuttcap%
\pgfsetroundjoin%
\definecolor{currentfill}{rgb}{0.000000,0.000000,0.000000}%
\pgfsetfillcolor{currentfill}%
\pgfsetlinewidth{0.803000pt}%
\definecolor{currentstroke}{rgb}{0.000000,0.000000,0.000000}%
\pgfsetstrokecolor{currentstroke}%
\pgfsetdash{}{0pt}%
\pgfsys@defobject{currentmarker}{\pgfqpoint{0.000000in}{-0.048611in}}{\pgfqpoint{0.000000in}{0.000000in}}{%
\pgfpathmoveto{\pgfqpoint{0.000000in}{0.000000in}}%
\pgfpathlineto{\pgfqpoint{0.000000in}{-0.048611in}}%
\pgfusepath{stroke,fill}%
}%
\begin{pgfscope}%
\pgfsys@transformshift{2.915741in}{3.363889in}%
\pgfsys@useobject{currentmarker}{}%
\end{pgfscope}%
\end{pgfscope}%
\begin{pgfscope}%
\definecolor{textcolor}{rgb}{0.000000,0.000000,0.000000}%
\pgfsetstrokecolor{textcolor}%
\pgfsetfillcolor{textcolor}%
\pgftext[x=2.915741in,y=3.266667in,,top]{\color{textcolor}\sffamily\fontsize{10.000000}{12.000000}\selectfont \(\displaystyle {10^{7}}\)}%
\end{pgfscope}%
\begin{pgfscope}%
\pgfsetbuttcap%
\pgfsetroundjoin%
\definecolor{currentfill}{rgb}{0.000000,0.000000,0.000000}%
\pgfsetfillcolor{currentfill}%
\pgfsetlinewidth{0.803000pt}%
\definecolor{currentstroke}{rgb}{0.000000,0.000000,0.000000}%
\pgfsetstrokecolor{currentstroke}%
\pgfsetdash{}{0pt}%
\pgfsys@defobject{currentmarker}{\pgfqpoint{0.000000in}{-0.048611in}}{\pgfqpoint{0.000000in}{0.000000in}}{%
\pgfpathmoveto{\pgfqpoint{0.000000in}{0.000000in}}%
\pgfpathlineto{\pgfqpoint{0.000000in}{-0.048611in}}%
\pgfusepath{stroke,fill}%
}%
\begin{pgfscope}%
\pgfsys@transformshift{3.498058in}{3.363889in}%
\pgfsys@useobject{currentmarker}{}%
\end{pgfscope}%
\end{pgfscope}%
\begin{pgfscope}%
\definecolor{textcolor}{rgb}{0.000000,0.000000,0.000000}%
\pgfsetstrokecolor{textcolor}%
\pgfsetfillcolor{textcolor}%
\pgftext[x=3.498058in,y=3.266667in,,top]{\color{textcolor}\sffamily\fontsize{10.000000}{12.000000}\selectfont \(\displaystyle {10^{8}}\)}%
\end{pgfscope}%
\begin{pgfscope}%
\pgfsetbuttcap%
\pgfsetroundjoin%
\definecolor{currentfill}{rgb}{0.000000,0.000000,0.000000}%
\pgfsetfillcolor{currentfill}%
\pgfsetlinewidth{0.602250pt}%
\definecolor{currentstroke}{rgb}{0.000000,0.000000,0.000000}%
\pgfsetstrokecolor{currentstroke}%
\pgfsetdash{}{0pt}%
\pgfsys@defobject{currentmarker}{\pgfqpoint{0.000000in}{-0.027778in}}{\pgfqpoint{0.000000in}{0.000000in}}{%
\pgfpathmoveto{\pgfqpoint{0.000000in}{0.000000in}}%
\pgfpathlineto{\pgfqpoint{0.000000in}{-0.027778in}}%
\pgfusepath{stroke,fill}%
}%
\begin{pgfscope}%
\pgfsys@transformshift{0.761765in}{3.363889in}%
\pgfsys@useobject{currentmarker}{}%
\end{pgfscope}%
\end{pgfscope}%
\begin{pgfscope}%
\pgfsetbuttcap%
\pgfsetroundjoin%
\definecolor{currentfill}{rgb}{0.000000,0.000000,0.000000}%
\pgfsetfillcolor{currentfill}%
\pgfsetlinewidth{0.602250pt}%
\definecolor{currentstroke}{rgb}{0.000000,0.000000,0.000000}%
\pgfsetstrokecolor{currentstroke}%
\pgfsetdash{}{0pt}%
\pgfsys@defobject{currentmarker}{\pgfqpoint{0.000000in}{-0.027778in}}{\pgfqpoint{0.000000in}{0.000000in}}{%
\pgfpathmoveto{\pgfqpoint{0.000000in}{0.000000in}}%
\pgfpathlineto{\pgfqpoint{0.000000in}{-0.027778in}}%
\pgfusepath{stroke,fill}%
}%
\begin{pgfscope}%
\pgfsys@transformshift{0.864306in}{3.363889in}%
\pgfsys@useobject{currentmarker}{}%
\end{pgfscope}%
\end{pgfscope}%
\begin{pgfscope}%
\pgfsetbuttcap%
\pgfsetroundjoin%
\definecolor{currentfill}{rgb}{0.000000,0.000000,0.000000}%
\pgfsetfillcolor{currentfill}%
\pgfsetlinewidth{0.602250pt}%
\definecolor{currentstroke}{rgb}{0.000000,0.000000,0.000000}%
\pgfsetstrokecolor{currentstroke}%
\pgfsetdash{}{0pt}%
\pgfsys@defobject{currentmarker}{\pgfqpoint{0.000000in}{-0.027778in}}{\pgfqpoint{0.000000in}{0.000000in}}{%
\pgfpathmoveto{\pgfqpoint{0.000000in}{0.000000in}}%
\pgfpathlineto{\pgfqpoint{0.000000in}{-0.027778in}}%
\pgfusepath{stroke,fill}%
}%
\begin{pgfscope}%
\pgfsys@transformshift{0.937060in}{3.363889in}%
\pgfsys@useobject{currentmarker}{}%
\end{pgfscope}%
\end{pgfscope}%
\begin{pgfscope}%
\pgfsetbuttcap%
\pgfsetroundjoin%
\definecolor{currentfill}{rgb}{0.000000,0.000000,0.000000}%
\pgfsetfillcolor{currentfill}%
\pgfsetlinewidth{0.602250pt}%
\definecolor{currentstroke}{rgb}{0.000000,0.000000,0.000000}%
\pgfsetstrokecolor{currentstroke}%
\pgfsetdash{}{0pt}%
\pgfsys@defobject{currentmarker}{\pgfqpoint{0.000000in}{-0.027778in}}{\pgfqpoint{0.000000in}{0.000000in}}{%
\pgfpathmoveto{\pgfqpoint{0.000000in}{0.000000in}}%
\pgfpathlineto{\pgfqpoint{0.000000in}{-0.027778in}}%
\pgfusepath{stroke,fill}%
}%
\begin{pgfscope}%
\pgfsys@transformshift{0.993493in}{3.363889in}%
\pgfsys@useobject{currentmarker}{}%
\end{pgfscope}%
\end{pgfscope}%
\begin{pgfscope}%
\pgfsetbuttcap%
\pgfsetroundjoin%
\definecolor{currentfill}{rgb}{0.000000,0.000000,0.000000}%
\pgfsetfillcolor{currentfill}%
\pgfsetlinewidth{0.602250pt}%
\definecolor{currentstroke}{rgb}{0.000000,0.000000,0.000000}%
\pgfsetstrokecolor{currentstroke}%
\pgfsetdash{}{0pt}%
\pgfsys@defobject{currentmarker}{\pgfqpoint{0.000000in}{-0.027778in}}{\pgfqpoint{0.000000in}{0.000000in}}{%
\pgfpathmoveto{\pgfqpoint{0.000000in}{0.000000in}}%
\pgfpathlineto{\pgfqpoint{0.000000in}{-0.027778in}}%
\pgfusepath{stroke,fill}%
}%
\begin{pgfscope}%
\pgfsys@transformshift{1.039601in}{3.363889in}%
\pgfsys@useobject{currentmarker}{}%
\end{pgfscope}%
\end{pgfscope}%
\begin{pgfscope}%
\pgfsetbuttcap%
\pgfsetroundjoin%
\definecolor{currentfill}{rgb}{0.000000,0.000000,0.000000}%
\pgfsetfillcolor{currentfill}%
\pgfsetlinewidth{0.602250pt}%
\definecolor{currentstroke}{rgb}{0.000000,0.000000,0.000000}%
\pgfsetstrokecolor{currentstroke}%
\pgfsetdash{}{0pt}%
\pgfsys@defobject{currentmarker}{\pgfqpoint{0.000000in}{-0.027778in}}{\pgfqpoint{0.000000in}{0.000000in}}{%
\pgfpathmoveto{\pgfqpoint{0.000000in}{0.000000in}}%
\pgfpathlineto{\pgfqpoint{0.000000in}{-0.027778in}}%
\pgfusepath{stroke,fill}%
}%
\begin{pgfscope}%
\pgfsys@transformshift{1.078586in}{3.363889in}%
\pgfsys@useobject{currentmarker}{}%
\end{pgfscope}%
\end{pgfscope}%
\begin{pgfscope}%
\pgfsetbuttcap%
\pgfsetroundjoin%
\definecolor{currentfill}{rgb}{0.000000,0.000000,0.000000}%
\pgfsetfillcolor{currentfill}%
\pgfsetlinewidth{0.602250pt}%
\definecolor{currentstroke}{rgb}{0.000000,0.000000,0.000000}%
\pgfsetstrokecolor{currentstroke}%
\pgfsetdash{}{0pt}%
\pgfsys@defobject{currentmarker}{\pgfqpoint{0.000000in}{-0.027778in}}{\pgfqpoint{0.000000in}{0.000000in}}{%
\pgfpathmoveto{\pgfqpoint{0.000000in}{0.000000in}}%
\pgfpathlineto{\pgfqpoint{0.000000in}{-0.027778in}}%
\pgfusepath{stroke,fill}%
}%
\begin{pgfscope}%
\pgfsys@transformshift{1.112355in}{3.363889in}%
\pgfsys@useobject{currentmarker}{}%
\end{pgfscope}%
\end{pgfscope}%
\begin{pgfscope}%
\pgfsetbuttcap%
\pgfsetroundjoin%
\definecolor{currentfill}{rgb}{0.000000,0.000000,0.000000}%
\pgfsetfillcolor{currentfill}%
\pgfsetlinewidth{0.602250pt}%
\definecolor{currentstroke}{rgb}{0.000000,0.000000,0.000000}%
\pgfsetstrokecolor{currentstroke}%
\pgfsetdash{}{0pt}%
\pgfsys@defobject{currentmarker}{\pgfqpoint{0.000000in}{-0.027778in}}{\pgfqpoint{0.000000in}{0.000000in}}{%
\pgfpathmoveto{\pgfqpoint{0.000000in}{0.000000in}}%
\pgfpathlineto{\pgfqpoint{0.000000in}{-0.027778in}}%
\pgfusepath{stroke,fill}%
}%
\begin{pgfscope}%
\pgfsys@transformshift{1.142142in}{3.363889in}%
\pgfsys@useobject{currentmarker}{}%
\end{pgfscope}%
\end{pgfscope}%
\begin{pgfscope}%
\pgfsetbuttcap%
\pgfsetroundjoin%
\definecolor{currentfill}{rgb}{0.000000,0.000000,0.000000}%
\pgfsetfillcolor{currentfill}%
\pgfsetlinewidth{0.602250pt}%
\definecolor{currentstroke}{rgb}{0.000000,0.000000,0.000000}%
\pgfsetstrokecolor{currentstroke}%
\pgfsetdash{}{0pt}%
\pgfsys@defobject{currentmarker}{\pgfqpoint{0.000000in}{-0.027778in}}{\pgfqpoint{0.000000in}{0.000000in}}{%
\pgfpathmoveto{\pgfqpoint{0.000000in}{0.000000in}}%
\pgfpathlineto{\pgfqpoint{0.000000in}{-0.027778in}}%
\pgfusepath{stroke,fill}%
}%
\begin{pgfscope}%
\pgfsys@transformshift{1.344083in}{3.363889in}%
\pgfsys@useobject{currentmarker}{}%
\end{pgfscope}%
\end{pgfscope}%
\begin{pgfscope}%
\pgfsetbuttcap%
\pgfsetroundjoin%
\definecolor{currentfill}{rgb}{0.000000,0.000000,0.000000}%
\pgfsetfillcolor{currentfill}%
\pgfsetlinewidth{0.602250pt}%
\definecolor{currentstroke}{rgb}{0.000000,0.000000,0.000000}%
\pgfsetstrokecolor{currentstroke}%
\pgfsetdash{}{0pt}%
\pgfsys@defobject{currentmarker}{\pgfqpoint{0.000000in}{-0.027778in}}{\pgfqpoint{0.000000in}{0.000000in}}{%
\pgfpathmoveto{\pgfqpoint{0.000000in}{0.000000in}}%
\pgfpathlineto{\pgfqpoint{0.000000in}{-0.027778in}}%
\pgfusepath{stroke,fill}%
}%
\begin{pgfscope}%
\pgfsys@transformshift{1.446624in}{3.363889in}%
\pgfsys@useobject{currentmarker}{}%
\end{pgfscope}%
\end{pgfscope}%
\begin{pgfscope}%
\pgfsetbuttcap%
\pgfsetroundjoin%
\definecolor{currentfill}{rgb}{0.000000,0.000000,0.000000}%
\pgfsetfillcolor{currentfill}%
\pgfsetlinewidth{0.602250pt}%
\definecolor{currentstroke}{rgb}{0.000000,0.000000,0.000000}%
\pgfsetstrokecolor{currentstroke}%
\pgfsetdash{}{0pt}%
\pgfsys@defobject{currentmarker}{\pgfqpoint{0.000000in}{-0.027778in}}{\pgfqpoint{0.000000in}{0.000000in}}{%
\pgfpathmoveto{\pgfqpoint{0.000000in}{0.000000in}}%
\pgfpathlineto{\pgfqpoint{0.000000in}{-0.027778in}}%
\pgfusepath{stroke,fill}%
}%
\begin{pgfscope}%
\pgfsys@transformshift{1.519378in}{3.363889in}%
\pgfsys@useobject{currentmarker}{}%
\end{pgfscope}%
\end{pgfscope}%
\begin{pgfscope}%
\pgfsetbuttcap%
\pgfsetroundjoin%
\definecolor{currentfill}{rgb}{0.000000,0.000000,0.000000}%
\pgfsetfillcolor{currentfill}%
\pgfsetlinewidth{0.602250pt}%
\definecolor{currentstroke}{rgb}{0.000000,0.000000,0.000000}%
\pgfsetstrokecolor{currentstroke}%
\pgfsetdash{}{0pt}%
\pgfsys@defobject{currentmarker}{\pgfqpoint{0.000000in}{-0.027778in}}{\pgfqpoint{0.000000in}{0.000000in}}{%
\pgfpathmoveto{\pgfqpoint{0.000000in}{0.000000in}}%
\pgfpathlineto{\pgfqpoint{0.000000in}{-0.027778in}}%
\pgfusepath{stroke,fill}%
}%
\begin{pgfscope}%
\pgfsys@transformshift{1.575810in}{3.363889in}%
\pgfsys@useobject{currentmarker}{}%
\end{pgfscope}%
\end{pgfscope}%
\begin{pgfscope}%
\pgfsetbuttcap%
\pgfsetroundjoin%
\definecolor{currentfill}{rgb}{0.000000,0.000000,0.000000}%
\pgfsetfillcolor{currentfill}%
\pgfsetlinewidth{0.602250pt}%
\definecolor{currentstroke}{rgb}{0.000000,0.000000,0.000000}%
\pgfsetstrokecolor{currentstroke}%
\pgfsetdash{}{0pt}%
\pgfsys@defobject{currentmarker}{\pgfqpoint{0.000000in}{-0.027778in}}{\pgfqpoint{0.000000in}{0.000000in}}{%
\pgfpathmoveto{\pgfqpoint{0.000000in}{0.000000in}}%
\pgfpathlineto{\pgfqpoint{0.000000in}{-0.027778in}}%
\pgfusepath{stroke,fill}%
}%
\begin{pgfscope}%
\pgfsys@transformshift{1.621919in}{3.363889in}%
\pgfsys@useobject{currentmarker}{}%
\end{pgfscope}%
\end{pgfscope}%
\begin{pgfscope}%
\pgfsetbuttcap%
\pgfsetroundjoin%
\definecolor{currentfill}{rgb}{0.000000,0.000000,0.000000}%
\pgfsetfillcolor{currentfill}%
\pgfsetlinewidth{0.602250pt}%
\definecolor{currentstroke}{rgb}{0.000000,0.000000,0.000000}%
\pgfsetstrokecolor{currentstroke}%
\pgfsetdash{}{0pt}%
\pgfsys@defobject{currentmarker}{\pgfqpoint{0.000000in}{-0.027778in}}{\pgfqpoint{0.000000in}{0.000000in}}{%
\pgfpathmoveto{\pgfqpoint{0.000000in}{0.000000in}}%
\pgfpathlineto{\pgfqpoint{0.000000in}{-0.027778in}}%
\pgfusepath{stroke,fill}%
}%
\begin{pgfscope}%
\pgfsys@transformshift{1.660903in}{3.363889in}%
\pgfsys@useobject{currentmarker}{}%
\end{pgfscope}%
\end{pgfscope}%
\begin{pgfscope}%
\pgfsetbuttcap%
\pgfsetroundjoin%
\definecolor{currentfill}{rgb}{0.000000,0.000000,0.000000}%
\pgfsetfillcolor{currentfill}%
\pgfsetlinewidth{0.602250pt}%
\definecolor{currentstroke}{rgb}{0.000000,0.000000,0.000000}%
\pgfsetstrokecolor{currentstroke}%
\pgfsetdash{}{0pt}%
\pgfsys@defobject{currentmarker}{\pgfqpoint{0.000000in}{-0.027778in}}{\pgfqpoint{0.000000in}{0.000000in}}{%
\pgfpathmoveto{\pgfqpoint{0.000000in}{0.000000in}}%
\pgfpathlineto{\pgfqpoint{0.000000in}{-0.027778in}}%
\pgfusepath{stroke,fill}%
}%
\begin{pgfscope}%
\pgfsys@transformshift{1.694673in}{3.363889in}%
\pgfsys@useobject{currentmarker}{}%
\end{pgfscope}%
\end{pgfscope}%
\begin{pgfscope}%
\pgfsetbuttcap%
\pgfsetroundjoin%
\definecolor{currentfill}{rgb}{0.000000,0.000000,0.000000}%
\pgfsetfillcolor{currentfill}%
\pgfsetlinewidth{0.602250pt}%
\definecolor{currentstroke}{rgb}{0.000000,0.000000,0.000000}%
\pgfsetstrokecolor{currentstroke}%
\pgfsetdash{}{0pt}%
\pgfsys@defobject{currentmarker}{\pgfqpoint{0.000000in}{-0.027778in}}{\pgfqpoint{0.000000in}{0.000000in}}{%
\pgfpathmoveto{\pgfqpoint{0.000000in}{0.000000in}}%
\pgfpathlineto{\pgfqpoint{0.000000in}{-0.027778in}}%
\pgfusepath{stroke,fill}%
}%
\begin{pgfscope}%
\pgfsys@transformshift{1.724460in}{3.363889in}%
\pgfsys@useobject{currentmarker}{}%
\end{pgfscope}%
\end{pgfscope}%
\begin{pgfscope}%
\pgfsetbuttcap%
\pgfsetroundjoin%
\definecolor{currentfill}{rgb}{0.000000,0.000000,0.000000}%
\pgfsetfillcolor{currentfill}%
\pgfsetlinewidth{0.602250pt}%
\definecolor{currentstroke}{rgb}{0.000000,0.000000,0.000000}%
\pgfsetstrokecolor{currentstroke}%
\pgfsetdash{}{0pt}%
\pgfsys@defobject{currentmarker}{\pgfqpoint{0.000000in}{-0.027778in}}{\pgfqpoint{0.000000in}{0.000000in}}{%
\pgfpathmoveto{\pgfqpoint{0.000000in}{0.000000in}}%
\pgfpathlineto{\pgfqpoint{0.000000in}{-0.027778in}}%
\pgfusepath{stroke,fill}%
}%
\begin{pgfscope}%
\pgfsys@transformshift{1.926400in}{3.363889in}%
\pgfsys@useobject{currentmarker}{}%
\end{pgfscope}%
\end{pgfscope}%
\begin{pgfscope}%
\pgfsetbuttcap%
\pgfsetroundjoin%
\definecolor{currentfill}{rgb}{0.000000,0.000000,0.000000}%
\pgfsetfillcolor{currentfill}%
\pgfsetlinewidth{0.602250pt}%
\definecolor{currentstroke}{rgb}{0.000000,0.000000,0.000000}%
\pgfsetstrokecolor{currentstroke}%
\pgfsetdash{}{0pt}%
\pgfsys@defobject{currentmarker}{\pgfqpoint{0.000000in}{-0.027778in}}{\pgfqpoint{0.000000in}{0.000000in}}{%
\pgfpathmoveto{\pgfqpoint{0.000000in}{0.000000in}}%
\pgfpathlineto{\pgfqpoint{0.000000in}{-0.027778in}}%
\pgfusepath{stroke,fill}%
}%
\begin{pgfscope}%
\pgfsys@transformshift{2.028941in}{3.363889in}%
\pgfsys@useobject{currentmarker}{}%
\end{pgfscope}%
\end{pgfscope}%
\begin{pgfscope}%
\pgfsetbuttcap%
\pgfsetroundjoin%
\definecolor{currentfill}{rgb}{0.000000,0.000000,0.000000}%
\pgfsetfillcolor{currentfill}%
\pgfsetlinewidth{0.602250pt}%
\definecolor{currentstroke}{rgb}{0.000000,0.000000,0.000000}%
\pgfsetstrokecolor{currentstroke}%
\pgfsetdash{}{0pt}%
\pgfsys@defobject{currentmarker}{\pgfqpoint{0.000000in}{-0.027778in}}{\pgfqpoint{0.000000in}{0.000000in}}{%
\pgfpathmoveto{\pgfqpoint{0.000000in}{0.000000in}}%
\pgfpathlineto{\pgfqpoint{0.000000in}{-0.027778in}}%
\pgfusepath{stroke,fill}%
}%
\begin{pgfscope}%
\pgfsys@transformshift{2.101696in}{3.363889in}%
\pgfsys@useobject{currentmarker}{}%
\end{pgfscope}%
\end{pgfscope}%
\begin{pgfscope}%
\pgfsetbuttcap%
\pgfsetroundjoin%
\definecolor{currentfill}{rgb}{0.000000,0.000000,0.000000}%
\pgfsetfillcolor{currentfill}%
\pgfsetlinewidth{0.602250pt}%
\definecolor{currentstroke}{rgb}{0.000000,0.000000,0.000000}%
\pgfsetstrokecolor{currentstroke}%
\pgfsetdash{}{0pt}%
\pgfsys@defobject{currentmarker}{\pgfqpoint{0.000000in}{-0.027778in}}{\pgfqpoint{0.000000in}{0.000000in}}{%
\pgfpathmoveto{\pgfqpoint{0.000000in}{0.000000in}}%
\pgfpathlineto{\pgfqpoint{0.000000in}{-0.027778in}}%
\pgfusepath{stroke,fill}%
}%
\begin{pgfscope}%
\pgfsys@transformshift{2.158128in}{3.363889in}%
\pgfsys@useobject{currentmarker}{}%
\end{pgfscope}%
\end{pgfscope}%
\begin{pgfscope}%
\pgfsetbuttcap%
\pgfsetroundjoin%
\definecolor{currentfill}{rgb}{0.000000,0.000000,0.000000}%
\pgfsetfillcolor{currentfill}%
\pgfsetlinewidth{0.602250pt}%
\definecolor{currentstroke}{rgb}{0.000000,0.000000,0.000000}%
\pgfsetstrokecolor{currentstroke}%
\pgfsetdash{}{0pt}%
\pgfsys@defobject{currentmarker}{\pgfqpoint{0.000000in}{-0.027778in}}{\pgfqpoint{0.000000in}{0.000000in}}{%
\pgfpathmoveto{\pgfqpoint{0.000000in}{0.000000in}}%
\pgfpathlineto{\pgfqpoint{0.000000in}{-0.027778in}}%
\pgfusepath{stroke,fill}%
}%
\begin{pgfscope}%
\pgfsys@transformshift{2.204237in}{3.363889in}%
\pgfsys@useobject{currentmarker}{}%
\end{pgfscope}%
\end{pgfscope}%
\begin{pgfscope}%
\pgfsetbuttcap%
\pgfsetroundjoin%
\definecolor{currentfill}{rgb}{0.000000,0.000000,0.000000}%
\pgfsetfillcolor{currentfill}%
\pgfsetlinewidth{0.602250pt}%
\definecolor{currentstroke}{rgb}{0.000000,0.000000,0.000000}%
\pgfsetstrokecolor{currentstroke}%
\pgfsetdash{}{0pt}%
\pgfsys@defobject{currentmarker}{\pgfqpoint{0.000000in}{-0.027778in}}{\pgfqpoint{0.000000in}{0.000000in}}{%
\pgfpathmoveto{\pgfqpoint{0.000000in}{0.000000in}}%
\pgfpathlineto{\pgfqpoint{0.000000in}{-0.027778in}}%
\pgfusepath{stroke,fill}%
}%
\begin{pgfscope}%
\pgfsys@transformshift{2.243221in}{3.363889in}%
\pgfsys@useobject{currentmarker}{}%
\end{pgfscope}%
\end{pgfscope}%
\begin{pgfscope}%
\pgfsetbuttcap%
\pgfsetroundjoin%
\definecolor{currentfill}{rgb}{0.000000,0.000000,0.000000}%
\pgfsetfillcolor{currentfill}%
\pgfsetlinewidth{0.602250pt}%
\definecolor{currentstroke}{rgb}{0.000000,0.000000,0.000000}%
\pgfsetstrokecolor{currentstroke}%
\pgfsetdash{}{0pt}%
\pgfsys@defobject{currentmarker}{\pgfqpoint{0.000000in}{-0.027778in}}{\pgfqpoint{0.000000in}{0.000000in}}{%
\pgfpathmoveto{\pgfqpoint{0.000000in}{0.000000in}}%
\pgfpathlineto{\pgfqpoint{0.000000in}{-0.027778in}}%
\pgfusepath{stroke,fill}%
}%
\begin{pgfscope}%
\pgfsys@transformshift{2.276991in}{3.363889in}%
\pgfsys@useobject{currentmarker}{}%
\end{pgfscope}%
\end{pgfscope}%
\begin{pgfscope}%
\pgfsetbuttcap%
\pgfsetroundjoin%
\definecolor{currentfill}{rgb}{0.000000,0.000000,0.000000}%
\pgfsetfillcolor{currentfill}%
\pgfsetlinewidth{0.602250pt}%
\definecolor{currentstroke}{rgb}{0.000000,0.000000,0.000000}%
\pgfsetstrokecolor{currentstroke}%
\pgfsetdash{}{0pt}%
\pgfsys@defobject{currentmarker}{\pgfqpoint{0.000000in}{-0.027778in}}{\pgfqpoint{0.000000in}{0.000000in}}{%
\pgfpathmoveto{\pgfqpoint{0.000000in}{0.000000in}}%
\pgfpathlineto{\pgfqpoint{0.000000in}{-0.027778in}}%
\pgfusepath{stroke,fill}%
}%
\begin{pgfscope}%
\pgfsys@transformshift{2.306778in}{3.363889in}%
\pgfsys@useobject{currentmarker}{}%
\end{pgfscope}%
\end{pgfscope}%
\begin{pgfscope}%
\pgfsetbuttcap%
\pgfsetroundjoin%
\definecolor{currentfill}{rgb}{0.000000,0.000000,0.000000}%
\pgfsetfillcolor{currentfill}%
\pgfsetlinewidth{0.602250pt}%
\definecolor{currentstroke}{rgb}{0.000000,0.000000,0.000000}%
\pgfsetstrokecolor{currentstroke}%
\pgfsetdash{}{0pt}%
\pgfsys@defobject{currentmarker}{\pgfqpoint{0.000000in}{-0.027778in}}{\pgfqpoint{0.000000in}{0.000000in}}{%
\pgfpathmoveto{\pgfqpoint{0.000000in}{0.000000in}}%
\pgfpathlineto{\pgfqpoint{0.000000in}{-0.027778in}}%
\pgfusepath{stroke,fill}%
}%
\begin{pgfscope}%
\pgfsys@transformshift{2.508718in}{3.363889in}%
\pgfsys@useobject{currentmarker}{}%
\end{pgfscope}%
\end{pgfscope}%
\begin{pgfscope}%
\pgfsetbuttcap%
\pgfsetroundjoin%
\definecolor{currentfill}{rgb}{0.000000,0.000000,0.000000}%
\pgfsetfillcolor{currentfill}%
\pgfsetlinewidth{0.602250pt}%
\definecolor{currentstroke}{rgb}{0.000000,0.000000,0.000000}%
\pgfsetstrokecolor{currentstroke}%
\pgfsetdash{}{0pt}%
\pgfsys@defobject{currentmarker}{\pgfqpoint{0.000000in}{-0.027778in}}{\pgfqpoint{0.000000in}{0.000000in}}{%
\pgfpathmoveto{\pgfqpoint{0.000000in}{0.000000in}}%
\pgfpathlineto{\pgfqpoint{0.000000in}{-0.027778in}}%
\pgfusepath{stroke,fill}%
}%
\begin{pgfscope}%
\pgfsys@transformshift{2.611259in}{3.363889in}%
\pgfsys@useobject{currentmarker}{}%
\end{pgfscope}%
\end{pgfscope}%
\begin{pgfscope}%
\pgfsetbuttcap%
\pgfsetroundjoin%
\definecolor{currentfill}{rgb}{0.000000,0.000000,0.000000}%
\pgfsetfillcolor{currentfill}%
\pgfsetlinewidth{0.602250pt}%
\definecolor{currentstroke}{rgb}{0.000000,0.000000,0.000000}%
\pgfsetstrokecolor{currentstroke}%
\pgfsetdash{}{0pt}%
\pgfsys@defobject{currentmarker}{\pgfqpoint{0.000000in}{-0.027778in}}{\pgfqpoint{0.000000in}{0.000000in}}{%
\pgfpathmoveto{\pgfqpoint{0.000000in}{0.000000in}}%
\pgfpathlineto{\pgfqpoint{0.000000in}{-0.027778in}}%
\pgfusepath{stroke,fill}%
}%
\begin{pgfscope}%
\pgfsys@transformshift{2.684013in}{3.363889in}%
\pgfsys@useobject{currentmarker}{}%
\end{pgfscope}%
\end{pgfscope}%
\begin{pgfscope}%
\pgfsetbuttcap%
\pgfsetroundjoin%
\definecolor{currentfill}{rgb}{0.000000,0.000000,0.000000}%
\pgfsetfillcolor{currentfill}%
\pgfsetlinewidth{0.602250pt}%
\definecolor{currentstroke}{rgb}{0.000000,0.000000,0.000000}%
\pgfsetstrokecolor{currentstroke}%
\pgfsetdash{}{0pt}%
\pgfsys@defobject{currentmarker}{\pgfqpoint{0.000000in}{-0.027778in}}{\pgfqpoint{0.000000in}{0.000000in}}{%
\pgfpathmoveto{\pgfqpoint{0.000000in}{0.000000in}}%
\pgfpathlineto{\pgfqpoint{0.000000in}{-0.027778in}}%
\pgfusepath{stroke,fill}%
}%
\begin{pgfscope}%
\pgfsys@transformshift{2.740446in}{3.363889in}%
\pgfsys@useobject{currentmarker}{}%
\end{pgfscope}%
\end{pgfscope}%
\begin{pgfscope}%
\pgfsetbuttcap%
\pgfsetroundjoin%
\definecolor{currentfill}{rgb}{0.000000,0.000000,0.000000}%
\pgfsetfillcolor{currentfill}%
\pgfsetlinewidth{0.602250pt}%
\definecolor{currentstroke}{rgb}{0.000000,0.000000,0.000000}%
\pgfsetstrokecolor{currentstroke}%
\pgfsetdash{}{0pt}%
\pgfsys@defobject{currentmarker}{\pgfqpoint{0.000000in}{-0.027778in}}{\pgfqpoint{0.000000in}{0.000000in}}{%
\pgfpathmoveto{\pgfqpoint{0.000000in}{0.000000in}}%
\pgfpathlineto{\pgfqpoint{0.000000in}{-0.027778in}}%
\pgfusepath{stroke,fill}%
}%
\begin{pgfscope}%
\pgfsys@transformshift{2.786554in}{3.363889in}%
\pgfsys@useobject{currentmarker}{}%
\end{pgfscope}%
\end{pgfscope}%
\begin{pgfscope}%
\pgfsetbuttcap%
\pgfsetroundjoin%
\definecolor{currentfill}{rgb}{0.000000,0.000000,0.000000}%
\pgfsetfillcolor{currentfill}%
\pgfsetlinewidth{0.602250pt}%
\definecolor{currentstroke}{rgb}{0.000000,0.000000,0.000000}%
\pgfsetstrokecolor{currentstroke}%
\pgfsetdash{}{0pt}%
\pgfsys@defobject{currentmarker}{\pgfqpoint{0.000000in}{-0.027778in}}{\pgfqpoint{0.000000in}{0.000000in}}{%
\pgfpathmoveto{\pgfqpoint{0.000000in}{0.000000in}}%
\pgfpathlineto{\pgfqpoint{0.000000in}{-0.027778in}}%
\pgfusepath{stroke,fill}%
}%
\begin{pgfscope}%
\pgfsys@transformshift{2.825538in}{3.363889in}%
\pgfsys@useobject{currentmarker}{}%
\end{pgfscope}%
\end{pgfscope}%
\begin{pgfscope}%
\pgfsetbuttcap%
\pgfsetroundjoin%
\definecolor{currentfill}{rgb}{0.000000,0.000000,0.000000}%
\pgfsetfillcolor{currentfill}%
\pgfsetlinewidth{0.602250pt}%
\definecolor{currentstroke}{rgb}{0.000000,0.000000,0.000000}%
\pgfsetstrokecolor{currentstroke}%
\pgfsetdash{}{0pt}%
\pgfsys@defobject{currentmarker}{\pgfqpoint{0.000000in}{-0.027778in}}{\pgfqpoint{0.000000in}{0.000000in}}{%
\pgfpathmoveto{\pgfqpoint{0.000000in}{0.000000in}}%
\pgfpathlineto{\pgfqpoint{0.000000in}{-0.027778in}}%
\pgfusepath{stroke,fill}%
}%
\begin{pgfscope}%
\pgfsys@transformshift{2.859308in}{3.363889in}%
\pgfsys@useobject{currentmarker}{}%
\end{pgfscope}%
\end{pgfscope}%
\begin{pgfscope}%
\pgfsetbuttcap%
\pgfsetroundjoin%
\definecolor{currentfill}{rgb}{0.000000,0.000000,0.000000}%
\pgfsetfillcolor{currentfill}%
\pgfsetlinewidth{0.602250pt}%
\definecolor{currentstroke}{rgb}{0.000000,0.000000,0.000000}%
\pgfsetstrokecolor{currentstroke}%
\pgfsetdash{}{0pt}%
\pgfsys@defobject{currentmarker}{\pgfqpoint{0.000000in}{-0.027778in}}{\pgfqpoint{0.000000in}{0.000000in}}{%
\pgfpathmoveto{\pgfqpoint{0.000000in}{0.000000in}}%
\pgfpathlineto{\pgfqpoint{0.000000in}{-0.027778in}}%
\pgfusepath{stroke,fill}%
}%
\begin{pgfscope}%
\pgfsys@transformshift{2.889095in}{3.363889in}%
\pgfsys@useobject{currentmarker}{}%
\end{pgfscope}%
\end{pgfscope}%
\begin{pgfscope}%
\pgfsetbuttcap%
\pgfsetroundjoin%
\definecolor{currentfill}{rgb}{0.000000,0.000000,0.000000}%
\pgfsetfillcolor{currentfill}%
\pgfsetlinewidth{0.602250pt}%
\definecolor{currentstroke}{rgb}{0.000000,0.000000,0.000000}%
\pgfsetstrokecolor{currentstroke}%
\pgfsetdash{}{0pt}%
\pgfsys@defobject{currentmarker}{\pgfqpoint{0.000000in}{-0.027778in}}{\pgfqpoint{0.000000in}{0.000000in}}{%
\pgfpathmoveto{\pgfqpoint{0.000000in}{0.000000in}}%
\pgfpathlineto{\pgfqpoint{0.000000in}{-0.027778in}}%
\pgfusepath{stroke,fill}%
}%
\begin{pgfscope}%
\pgfsys@transformshift{3.091036in}{3.363889in}%
\pgfsys@useobject{currentmarker}{}%
\end{pgfscope}%
\end{pgfscope}%
\begin{pgfscope}%
\pgfsetbuttcap%
\pgfsetroundjoin%
\definecolor{currentfill}{rgb}{0.000000,0.000000,0.000000}%
\pgfsetfillcolor{currentfill}%
\pgfsetlinewidth{0.602250pt}%
\definecolor{currentstroke}{rgb}{0.000000,0.000000,0.000000}%
\pgfsetstrokecolor{currentstroke}%
\pgfsetdash{}{0pt}%
\pgfsys@defobject{currentmarker}{\pgfqpoint{0.000000in}{-0.027778in}}{\pgfqpoint{0.000000in}{0.000000in}}{%
\pgfpathmoveto{\pgfqpoint{0.000000in}{0.000000in}}%
\pgfpathlineto{\pgfqpoint{0.000000in}{-0.027778in}}%
\pgfusepath{stroke,fill}%
}%
\begin{pgfscope}%
\pgfsys@transformshift{3.193577in}{3.363889in}%
\pgfsys@useobject{currentmarker}{}%
\end{pgfscope}%
\end{pgfscope}%
\begin{pgfscope}%
\pgfsetbuttcap%
\pgfsetroundjoin%
\definecolor{currentfill}{rgb}{0.000000,0.000000,0.000000}%
\pgfsetfillcolor{currentfill}%
\pgfsetlinewidth{0.602250pt}%
\definecolor{currentstroke}{rgb}{0.000000,0.000000,0.000000}%
\pgfsetstrokecolor{currentstroke}%
\pgfsetdash{}{0pt}%
\pgfsys@defobject{currentmarker}{\pgfqpoint{0.000000in}{-0.027778in}}{\pgfqpoint{0.000000in}{0.000000in}}{%
\pgfpathmoveto{\pgfqpoint{0.000000in}{0.000000in}}%
\pgfpathlineto{\pgfqpoint{0.000000in}{-0.027778in}}%
\pgfusepath{stroke,fill}%
}%
\begin{pgfscope}%
\pgfsys@transformshift{3.266331in}{3.363889in}%
\pgfsys@useobject{currentmarker}{}%
\end{pgfscope}%
\end{pgfscope}%
\begin{pgfscope}%
\pgfsetbuttcap%
\pgfsetroundjoin%
\definecolor{currentfill}{rgb}{0.000000,0.000000,0.000000}%
\pgfsetfillcolor{currentfill}%
\pgfsetlinewidth{0.602250pt}%
\definecolor{currentstroke}{rgb}{0.000000,0.000000,0.000000}%
\pgfsetstrokecolor{currentstroke}%
\pgfsetdash{}{0pt}%
\pgfsys@defobject{currentmarker}{\pgfqpoint{0.000000in}{-0.027778in}}{\pgfqpoint{0.000000in}{0.000000in}}{%
\pgfpathmoveto{\pgfqpoint{0.000000in}{0.000000in}}%
\pgfpathlineto{\pgfqpoint{0.000000in}{-0.027778in}}%
\pgfusepath{stroke,fill}%
}%
\begin{pgfscope}%
\pgfsys@transformshift{3.322763in}{3.363889in}%
\pgfsys@useobject{currentmarker}{}%
\end{pgfscope}%
\end{pgfscope}%
\begin{pgfscope}%
\pgfsetbuttcap%
\pgfsetroundjoin%
\definecolor{currentfill}{rgb}{0.000000,0.000000,0.000000}%
\pgfsetfillcolor{currentfill}%
\pgfsetlinewidth{0.602250pt}%
\definecolor{currentstroke}{rgb}{0.000000,0.000000,0.000000}%
\pgfsetstrokecolor{currentstroke}%
\pgfsetdash{}{0pt}%
\pgfsys@defobject{currentmarker}{\pgfqpoint{0.000000in}{-0.027778in}}{\pgfqpoint{0.000000in}{0.000000in}}{%
\pgfpathmoveto{\pgfqpoint{0.000000in}{0.000000in}}%
\pgfpathlineto{\pgfqpoint{0.000000in}{-0.027778in}}%
\pgfusepath{stroke,fill}%
}%
\begin{pgfscope}%
\pgfsys@transformshift{3.368872in}{3.363889in}%
\pgfsys@useobject{currentmarker}{}%
\end{pgfscope}%
\end{pgfscope}%
\begin{pgfscope}%
\pgfsetbuttcap%
\pgfsetroundjoin%
\definecolor{currentfill}{rgb}{0.000000,0.000000,0.000000}%
\pgfsetfillcolor{currentfill}%
\pgfsetlinewidth{0.602250pt}%
\definecolor{currentstroke}{rgb}{0.000000,0.000000,0.000000}%
\pgfsetstrokecolor{currentstroke}%
\pgfsetdash{}{0pt}%
\pgfsys@defobject{currentmarker}{\pgfqpoint{0.000000in}{-0.027778in}}{\pgfqpoint{0.000000in}{0.000000in}}{%
\pgfpathmoveto{\pgfqpoint{0.000000in}{0.000000in}}%
\pgfpathlineto{\pgfqpoint{0.000000in}{-0.027778in}}%
\pgfusepath{stroke,fill}%
}%
\begin{pgfscope}%
\pgfsys@transformshift{3.407856in}{3.363889in}%
\pgfsys@useobject{currentmarker}{}%
\end{pgfscope}%
\end{pgfscope}%
\begin{pgfscope}%
\pgfsetbuttcap%
\pgfsetroundjoin%
\definecolor{currentfill}{rgb}{0.000000,0.000000,0.000000}%
\pgfsetfillcolor{currentfill}%
\pgfsetlinewidth{0.602250pt}%
\definecolor{currentstroke}{rgb}{0.000000,0.000000,0.000000}%
\pgfsetstrokecolor{currentstroke}%
\pgfsetdash{}{0pt}%
\pgfsys@defobject{currentmarker}{\pgfqpoint{0.000000in}{-0.027778in}}{\pgfqpoint{0.000000in}{0.000000in}}{%
\pgfpathmoveto{\pgfqpoint{0.000000in}{0.000000in}}%
\pgfpathlineto{\pgfqpoint{0.000000in}{-0.027778in}}%
\pgfusepath{stroke,fill}%
}%
\begin{pgfscope}%
\pgfsys@transformshift{3.441626in}{3.363889in}%
\pgfsys@useobject{currentmarker}{}%
\end{pgfscope}%
\end{pgfscope}%
\begin{pgfscope}%
\pgfsetbuttcap%
\pgfsetroundjoin%
\definecolor{currentfill}{rgb}{0.000000,0.000000,0.000000}%
\pgfsetfillcolor{currentfill}%
\pgfsetlinewidth{0.602250pt}%
\definecolor{currentstroke}{rgb}{0.000000,0.000000,0.000000}%
\pgfsetstrokecolor{currentstroke}%
\pgfsetdash{}{0pt}%
\pgfsys@defobject{currentmarker}{\pgfqpoint{0.000000in}{-0.027778in}}{\pgfqpoint{0.000000in}{0.000000in}}{%
\pgfpathmoveto{\pgfqpoint{0.000000in}{0.000000in}}%
\pgfpathlineto{\pgfqpoint{0.000000in}{-0.027778in}}%
\pgfusepath{stroke,fill}%
}%
\begin{pgfscope}%
\pgfsys@transformshift{3.471413in}{3.363889in}%
\pgfsys@useobject{currentmarker}{}%
\end{pgfscope}%
\end{pgfscope}%
\begin{pgfscope}%
\pgfsetbuttcap%
\pgfsetroundjoin%
\definecolor{currentfill}{rgb}{0.000000,0.000000,0.000000}%
\pgfsetfillcolor{currentfill}%
\pgfsetlinewidth{0.602250pt}%
\definecolor{currentstroke}{rgb}{0.000000,0.000000,0.000000}%
\pgfsetstrokecolor{currentstroke}%
\pgfsetdash{}{0pt}%
\pgfsys@defobject{currentmarker}{\pgfqpoint{0.000000in}{-0.027778in}}{\pgfqpoint{0.000000in}{0.000000in}}{%
\pgfpathmoveto{\pgfqpoint{0.000000in}{0.000000in}}%
\pgfpathlineto{\pgfqpoint{0.000000in}{-0.027778in}}%
\pgfusepath{stroke,fill}%
}%
\begin{pgfscope}%
\pgfsys@transformshift{3.673353in}{3.363889in}%
\pgfsys@useobject{currentmarker}{}%
\end{pgfscope}%
\end{pgfscope}%
\begin{pgfscope}%
\pgfsetbuttcap%
\pgfsetroundjoin%
\definecolor{currentfill}{rgb}{0.000000,0.000000,0.000000}%
\pgfsetfillcolor{currentfill}%
\pgfsetlinewidth{0.602250pt}%
\definecolor{currentstroke}{rgb}{0.000000,0.000000,0.000000}%
\pgfsetstrokecolor{currentstroke}%
\pgfsetdash{}{0pt}%
\pgfsys@defobject{currentmarker}{\pgfqpoint{0.000000in}{-0.027778in}}{\pgfqpoint{0.000000in}{0.000000in}}{%
\pgfpathmoveto{\pgfqpoint{0.000000in}{0.000000in}}%
\pgfpathlineto{\pgfqpoint{0.000000in}{-0.027778in}}%
\pgfusepath{stroke,fill}%
}%
\begin{pgfscope}%
\pgfsys@transformshift{3.775894in}{3.363889in}%
\pgfsys@useobject{currentmarker}{}%
\end{pgfscope}%
\end{pgfscope}%
\begin{pgfscope}%
\pgfsetbuttcap%
\pgfsetroundjoin%
\definecolor{currentfill}{rgb}{0.000000,0.000000,0.000000}%
\pgfsetfillcolor{currentfill}%
\pgfsetlinewidth{0.602250pt}%
\definecolor{currentstroke}{rgb}{0.000000,0.000000,0.000000}%
\pgfsetstrokecolor{currentstroke}%
\pgfsetdash{}{0pt}%
\pgfsys@defobject{currentmarker}{\pgfqpoint{0.000000in}{-0.027778in}}{\pgfqpoint{0.000000in}{0.000000in}}{%
\pgfpathmoveto{\pgfqpoint{0.000000in}{0.000000in}}%
\pgfpathlineto{\pgfqpoint{0.000000in}{-0.027778in}}%
\pgfusepath{stroke,fill}%
}%
\begin{pgfscope}%
\pgfsys@transformshift{3.848648in}{3.363889in}%
\pgfsys@useobject{currentmarker}{}%
\end{pgfscope}%
\end{pgfscope}%
\begin{pgfscope}%
\pgfsetbuttcap%
\pgfsetroundjoin%
\definecolor{currentfill}{rgb}{0.000000,0.000000,0.000000}%
\pgfsetfillcolor{currentfill}%
\pgfsetlinewidth{0.803000pt}%
\definecolor{currentstroke}{rgb}{0.000000,0.000000,0.000000}%
\pgfsetstrokecolor{currentstroke}%
\pgfsetdash{}{0pt}%
\pgfsys@defobject{currentmarker}{\pgfqpoint{-0.048611in}{0.000000in}}{\pgfqpoint{0.000000in}{0.000000in}}{%
\pgfpathmoveto{\pgfqpoint{0.000000in}{0.000000in}}%
\pgfpathlineto{\pgfqpoint{-0.048611in}{0.000000in}}%
\pgfusepath{stroke,fill}%
}%
\begin{pgfscope}%
\pgfsys@transformshift{0.672569in}{3.400008in}%
\pgfsys@useobject{currentmarker}{}%
\end{pgfscope}%
\end{pgfscope}%
\begin{pgfscope}%
\definecolor{textcolor}{rgb}{0.000000,0.000000,0.000000}%
\pgfsetstrokecolor{textcolor}%
\pgfsetfillcolor{textcolor}%
\pgftext[x=0.238094in,y=3.347247in,left,base]{\color{textcolor}\sffamily\fontsize{10.000000}{12.000000}\selectfont −0.2}%
\end{pgfscope}%
\begin{pgfscope}%
\pgfsetbuttcap%
\pgfsetroundjoin%
\definecolor{currentfill}{rgb}{0.000000,0.000000,0.000000}%
\pgfsetfillcolor{currentfill}%
\pgfsetlinewidth{0.803000pt}%
\definecolor{currentstroke}{rgb}{0.000000,0.000000,0.000000}%
\pgfsetstrokecolor{currentstroke}%
\pgfsetdash{}{0pt}%
\pgfsys@defobject{currentmarker}{\pgfqpoint{-0.048611in}{0.000000in}}{\pgfqpoint{0.000000in}{0.000000in}}{%
\pgfpathmoveto{\pgfqpoint{0.000000in}{0.000000in}}%
\pgfpathlineto{\pgfqpoint{-0.048611in}{0.000000in}}%
\pgfusepath{stroke,fill}%
}%
\begin{pgfscope}%
\pgfsys@transformshift{0.672569in}{3.716687in}%
\pgfsys@useobject{currentmarker}{}%
\end{pgfscope}%
\end{pgfscope}%
\begin{pgfscope}%
\definecolor{textcolor}{rgb}{0.000000,0.000000,0.000000}%
\pgfsetstrokecolor{textcolor}%
\pgfsetfillcolor{textcolor}%
\pgftext[x=0.354468in,y=3.663926in,left,base]{\color{textcolor}\sffamily\fontsize{10.000000}{12.000000}\selectfont 0.0}%
\end{pgfscope}%
\begin{pgfscope}%
\pgfsetbuttcap%
\pgfsetroundjoin%
\definecolor{currentfill}{rgb}{0.000000,0.000000,0.000000}%
\pgfsetfillcolor{currentfill}%
\pgfsetlinewidth{0.803000pt}%
\definecolor{currentstroke}{rgb}{0.000000,0.000000,0.000000}%
\pgfsetstrokecolor{currentstroke}%
\pgfsetdash{}{0pt}%
\pgfsys@defobject{currentmarker}{\pgfqpoint{-0.048611in}{0.000000in}}{\pgfqpoint{0.000000in}{0.000000in}}{%
\pgfpathmoveto{\pgfqpoint{0.000000in}{0.000000in}}%
\pgfpathlineto{\pgfqpoint{-0.048611in}{0.000000in}}%
\pgfusepath{stroke,fill}%
}%
\begin{pgfscope}%
\pgfsys@transformshift{0.672569in}{4.033366in}%
\pgfsys@useobject{currentmarker}{}%
\end{pgfscope}%
\end{pgfscope}%
\begin{pgfscope}%
\definecolor{textcolor}{rgb}{0.000000,0.000000,0.000000}%
\pgfsetstrokecolor{textcolor}%
\pgfsetfillcolor{textcolor}%
\pgftext[x=0.354468in,y=3.980604in,left,base]{\color{textcolor}\sffamily\fontsize{10.000000}{12.000000}\selectfont 0.2}%
\end{pgfscope}%
\begin{pgfscope}%
\pgfsetbuttcap%
\pgfsetroundjoin%
\definecolor{currentfill}{rgb}{0.000000,0.000000,0.000000}%
\pgfsetfillcolor{currentfill}%
\pgfsetlinewidth{0.803000pt}%
\definecolor{currentstroke}{rgb}{0.000000,0.000000,0.000000}%
\pgfsetstrokecolor{currentstroke}%
\pgfsetdash{}{0pt}%
\pgfsys@defobject{currentmarker}{\pgfqpoint{-0.048611in}{0.000000in}}{\pgfqpoint{0.000000in}{0.000000in}}{%
\pgfpathmoveto{\pgfqpoint{0.000000in}{0.000000in}}%
\pgfpathlineto{\pgfqpoint{-0.048611in}{0.000000in}}%
\pgfusepath{stroke,fill}%
}%
\begin{pgfscope}%
\pgfsys@transformshift{0.672569in}{4.350044in}%
\pgfsys@useobject{currentmarker}{}%
\end{pgfscope}%
\end{pgfscope}%
\begin{pgfscope}%
\definecolor{textcolor}{rgb}{0.000000,0.000000,0.000000}%
\pgfsetstrokecolor{textcolor}%
\pgfsetfillcolor{textcolor}%
\pgftext[x=0.354468in,y=4.297283in,left,base]{\color{textcolor}\sffamily\fontsize{10.000000}{12.000000}\selectfont 0.4}%
\end{pgfscope}%
\begin{pgfscope}%
\pgfsetbuttcap%
\pgfsetroundjoin%
\definecolor{currentfill}{rgb}{0.000000,0.000000,0.000000}%
\pgfsetfillcolor{currentfill}%
\pgfsetlinewidth{0.803000pt}%
\definecolor{currentstroke}{rgb}{0.000000,0.000000,0.000000}%
\pgfsetstrokecolor{currentstroke}%
\pgfsetdash{}{0pt}%
\pgfsys@defobject{currentmarker}{\pgfqpoint{-0.048611in}{0.000000in}}{\pgfqpoint{0.000000in}{0.000000in}}{%
\pgfpathmoveto{\pgfqpoint{0.000000in}{0.000000in}}%
\pgfpathlineto{\pgfqpoint{-0.048611in}{0.000000in}}%
\pgfusepath{stroke,fill}%
}%
\begin{pgfscope}%
\pgfsys@transformshift{0.672569in}{4.666723in}%
\pgfsys@useobject{currentmarker}{}%
\end{pgfscope}%
\end{pgfscope}%
\begin{pgfscope}%
\definecolor{textcolor}{rgb}{0.000000,0.000000,0.000000}%
\pgfsetstrokecolor{textcolor}%
\pgfsetfillcolor{textcolor}%
\pgftext[x=0.354468in,y=4.613961in,left,base]{\color{textcolor}\sffamily\fontsize{10.000000}{12.000000}\selectfont 0.6}%
\end{pgfscope}%
\begin{pgfscope}%
\pgfsetbuttcap%
\pgfsetroundjoin%
\definecolor{currentfill}{rgb}{0.000000,0.000000,0.000000}%
\pgfsetfillcolor{currentfill}%
\pgfsetlinewidth{0.803000pt}%
\definecolor{currentstroke}{rgb}{0.000000,0.000000,0.000000}%
\pgfsetstrokecolor{currentstroke}%
\pgfsetdash{}{0pt}%
\pgfsys@defobject{currentmarker}{\pgfqpoint{-0.048611in}{0.000000in}}{\pgfqpoint{0.000000in}{0.000000in}}{%
\pgfpathmoveto{\pgfqpoint{0.000000in}{0.000000in}}%
\pgfpathlineto{\pgfqpoint{-0.048611in}{0.000000in}}%
\pgfusepath{stroke,fill}%
}%
\begin{pgfscope}%
\pgfsys@transformshift{0.672569in}{4.983402in}%
\pgfsys@useobject{currentmarker}{}%
\end{pgfscope}%
\end{pgfscope}%
\begin{pgfscope}%
\definecolor{textcolor}{rgb}{0.000000,0.000000,0.000000}%
\pgfsetstrokecolor{textcolor}%
\pgfsetfillcolor{textcolor}%
\pgftext[x=0.354468in,y=4.930640in,left,base]{\color{textcolor}\sffamily\fontsize{10.000000}{12.000000}\selectfont 0.8}%
\end{pgfscope}%
\begin{pgfscope}%
\pgfsetbuttcap%
\pgfsetroundjoin%
\definecolor{currentfill}{rgb}{0.000000,0.000000,0.000000}%
\pgfsetfillcolor{currentfill}%
\pgfsetlinewidth{0.803000pt}%
\definecolor{currentstroke}{rgb}{0.000000,0.000000,0.000000}%
\pgfsetstrokecolor{currentstroke}%
\pgfsetdash{}{0pt}%
\pgfsys@defobject{currentmarker}{\pgfqpoint{-0.048611in}{0.000000in}}{\pgfqpoint{0.000000in}{0.000000in}}{%
\pgfpathmoveto{\pgfqpoint{0.000000in}{0.000000in}}%
\pgfpathlineto{\pgfqpoint{-0.048611in}{0.000000in}}%
\pgfusepath{stroke,fill}%
}%
\begin{pgfscope}%
\pgfsys@transformshift{0.672569in}{5.300080in}%
\pgfsys@useobject{currentmarker}{}%
\end{pgfscope}%
\end{pgfscope}%
\begin{pgfscope}%
\definecolor{textcolor}{rgb}{0.000000,0.000000,0.000000}%
\pgfsetstrokecolor{textcolor}%
\pgfsetfillcolor{textcolor}%
\pgftext[x=0.354468in,y=5.247319in,left,base]{\color{textcolor}\sffamily\fontsize{10.000000}{12.000000}\selectfont 1.0}%
\end{pgfscope}%
\begin{pgfscope}%
\pgfsetbuttcap%
\pgfsetroundjoin%
\definecolor{currentfill}{rgb}{0.000000,0.000000,0.000000}%
\pgfsetfillcolor{currentfill}%
\pgfsetlinewidth{0.803000pt}%
\definecolor{currentstroke}{rgb}{0.000000,0.000000,0.000000}%
\pgfsetstrokecolor{currentstroke}%
\pgfsetdash{}{0pt}%
\pgfsys@defobject{currentmarker}{\pgfqpoint{-0.048611in}{0.000000in}}{\pgfqpoint{0.000000in}{0.000000in}}{%
\pgfpathmoveto{\pgfqpoint{0.000000in}{0.000000in}}%
\pgfpathlineto{\pgfqpoint{-0.048611in}{0.000000in}}%
\pgfusepath{stroke,fill}%
}%
\begin{pgfscope}%
\pgfsys@transformshift{0.672569in}{5.616759in}%
\pgfsys@useobject{currentmarker}{}%
\end{pgfscope}%
\end{pgfscope}%
\begin{pgfscope}%
\definecolor{textcolor}{rgb}{0.000000,0.000000,0.000000}%
\pgfsetstrokecolor{textcolor}%
\pgfsetfillcolor{textcolor}%
\pgftext[x=0.354468in,y=5.563997in,left,base]{\color{textcolor}\sffamily\fontsize{10.000000}{12.000000}\selectfont 1.2}%
\end{pgfscope}%
\begin{pgfscope}%
\pgfpathrectangle{\pgfqpoint{0.672569in}{3.363889in}}{\pgfqpoint{3.203819in}{2.263889in}}%
\pgfusepath{clip}%
\pgfsetrectcap%
\pgfsetroundjoin%
\pgfsetlinewidth{1.505625pt}%
\definecolor{currentstroke}{rgb}{0.121569,0.466667,0.705882}%
\pgfsetstrokecolor{currentstroke}%
\pgfsetdash{}{0pt}%
\pgfpathmoveto{\pgfqpoint{0.818198in}{4.316983in}}%
\pgfpathlineto{\pgfqpoint{0.963777in}{5.282805in}}%
\pgfpathlineto{\pgfqpoint{1.109356in}{5.286117in}}%
\pgfpathlineto{\pgfqpoint{1.254936in}{5.287603in}}%
\pgfpathlineto{\pgfqpoint{1.400515in}{5.287631in}}%
\pgfpathlineto{\pgfqpoint{1.546095in}{5.288392in}}%
\pgfpathlineto{\pgfqpoint{1.691674in}{5.288394in}}%
\pgfpathlineto{\pgfqpoint{1.837253in}{5.288391in}}%
\pgfpathlineto{\pgfqpoint{1.982833in}{5.288391in}}%
\pgfpathlineto{\pgfqpoint{2.128412in}{5.288392in}}%
\pgfpathlineto{\pgfqpoint{2.273992in}{5.288391in}}%
\pgfpathlineto{\pgfqpoint{2.419571in}{5.288391in}}%
\pgfpathlineto{\pgfqpoint{2.565150in}{5.288392in}}%
\pgfpathlineto{\pgfqpoint{2.710730in}{5.288392in}}%
\pgfpathlineto{\pgfqpoint{2.856309in}{5.288392in}}%
\pgfpathlineto{\pgfqpoint{3.001889in}{5.288392in}}%
\pgfpathlineto{\pgfqpoint{3.147468in}{5.288392in}}%
\pgfpathlineto{\pgfqpoint{3.293047in}{5.288392in}}%
\pgfpathlineto{\pgfqpoint{3.438627in}{5.288392in}}%
\pgfpathlineto{\pgfqpoint{3.584206in}{5.288392in}}%
\pgfpathlineto{\pgfqpoint{3.729786in}{5.288392in}}%
\pgfusepath{stroke}%
\end{pgfscope}%
\begin{pgfscope}%
\pgfpathrectangle{\pgfqpoint{0.672569in}{3.363889in}}{\pgfqpoint{3.203819in}{2.263889in}}%
\pgfusepath{clip}%
\pgfsetrectcap%
\pgfsetroundjoin%
\pgfsetlinewidth{1.505625pt}%
\definecolor{currentstroke}{rgb}{1.000000,0.498039,0.054902}%
\pgfsetstrokecolor{currentstroke}%
\pgfsetdash{}{0pt}%
\pgfpathmoveto{\pgfqpoint{0.818198in}{4.616060in}}%
\pgfpathlineto{\pgfqpoint{0.963777in}{5.254836in}}%
\pgfpathlineto{\pgfqpoint{1.109356in}{5.266997in}}%
\pgfpathlineto{\pgfqpoint{1.254936in}{5.273411in}}%
\pgfpathlineto{\pgfqpoint{1.400515in}{5.276051in}}%
\pgfpathlineto{\pgfqpoint{1.546095in}{5.275714in}}%
\pgfpathlineto{\pgfqpoint{1.691674in}{5.275936in}}%
\pgfpathlineto{\pgfqpoint{1.837253in}{5.275842in}}%
\pgfpathlineto{\pgfqpoint{1.982833in}{5.275911in}}%
\pgfpathlineto{\pgfqpoint{2.128412in}{5.276057in}}%
\pgfpathlineto{\pgfqpoint{2.273992in}{5.275961in}}%
\pgfpathlineto{\pgfqpoint{2.419571in}{5.276023in}}%
\pgfpathlineto{\pgfqpoint{2.565150in}{5.276019in}}%
\pgfpathlineto{\pgfqpoint{2.710730in}{5.276009in}}%
\pgfpathlineto{\pgfqpoint{2.856309in}{5.275978in}}%
\pgfpathlineto{\pgfqpoint{3.001889in}{5.275988in}}%
\pgfpathlineto{\pgfqpoint{3.147468in}{5.275985in}}%
\pgfpathlineto{\pgfqpoint{3.293047in}{5.275997in}}%
\pgfpathlineto{\pgfqpoint{3.438627in}{5.275991in}}%
\pgfpathlineto{\pgfqpoint{3.584206in}{5.275984in}}%
\pgfpathlineto{\pgfqpoint{3.729786in}{5.275988in}}%
\pgfusepath{stroke}%
\end{pgfscope}%
\begin{pgfscope}%
\pgfpathrectangle{\pgfqpoint{0.672569in}{3.363889in}}{\pgfqpoint{3.203819in}{2.263889in}}%
\pgfusepath{clip}%
\pgfsetrectcap%
\pgfsetroundjoin%
\pgfsetlinewidth{1.505625pt}%
\definecolor{currentstroke}{rgb}{0.172549,0.627451,0.172549}%
\pgfsetstrokecolor{currentstroke}%
\pgfsetdash{}{0pt}%
\pgfpathmoveto{\pgfqpoint{0.818198in}{4.210018in}}%
\pgfpathlineto{\pgfqpoint{0.963777in}{5.101094in}}%
\pgfpathlineto{\pgfqpoint{1.109356in}{5.081234in}}%
\pgfpathlineto{\pgfqpoint{1.254936in}{5.136413in}}%
\pgfpathlineto{\pgfqpoint{1.400515in}{5.161371in}}%
\pgfpathlineto{\pgfqpoint{1.546095in}{5.160666in}}%
\pgfpathlineto{\pgfqpoint{1.691674in}{5.157862in}}%
\pgfpathlineto{\pgfqpoint{1.837253in}{5.162953in}}%
\pgfpathlineto{\pgfqpoint{1.982833in}{5.156626in}}%
\pgfpathlineto{\pgfqpoint{2.128412in}{5.157059in}}%
\pgfpathlineto{\pgfqpoint{2.273992in}{5.158767in}}%
\pgfpathlineto{\pgfqpoint{2.419571in}{5.160370in}}%
\pgfpathlineto{\pgfqpoint{2.565150in}{5.158807in}}%
\pgfpathlineto{\pgfqpoint{2.710730in}{5.158851in}}%
\pgfpathlineto{\pgfqpoint{2.856309in}{5.159179in}}%
\pgfpathlineto{\pgfqpoint{3.001889in}{5.158984in}}%
\pgfpathlineto{\pgfqpoint{3.147468in}{5.159094in}}%
\pgfpathlineto{\pgfqpoint{3.293047in}{5.158780in}}%
\pgfpathlineto{\pgfqpoint{3.438627in}{5.159157in}}%
\pgfpathlineto{\pgfqpoint{3.584206in}{5.159001in}}%
\pgfpathlineto{\pgfqpoint{3.729786in}{5.159114in}}%
\pgfusepath{stroke}%
\end{pgfscope}%
\begin{pgfscope}%
\pgfpathrectangle{\pgfqpoint{0.672569in}{3.363889in}}{\pgfqpoint{3.203819in}{2.263889in}}%
\pgfusepath{clip}%
\pgfsetrectcap%
\pgfsetroundjoin%
\pgfsetlinewidth{1.505625pt}%
\definecolor{currentstroke}{rgb}{0.839216,0.152941,0.156863}%
\pgfsetstrokecolor{currentstroke}%
\pgfsetdash{}{0pt}%
\pgfpathmoveto{\pgfqpoint{0.818198in}{4.228213in}}%
\pgfpathlineto{\pgfqpoint{0.963777in}{4.398895in}}%
\pgfpathlineto{\pgfqpoint{1.109356in}{4.420263in}}%
\pgfpathlineto{\pgfqpoint{1.254936in}{4.208862in}}%
\pgfpathlineto{\pgfqpoint{1.400515in}{4.308323in}}%
\pgfpathlineto{\pgfqpoint{1.546095in}{4.311813in}}%
\pgfpathlineto{\pgfqpoint{1.691674in}{4.279686in}}%
\pgfpathlineto{\pgfqpoint{1.837253in}{4.396138in}}%
\pgfpathlineto{\pgfqpoint{1.982833in}{4.310469in}}%
\pgfpathlineto{\pgfqpoint{2.128412in}{4.332981in}}%
\pgfpathlineto{\pgfqpoint{2.273992in}{4.316087in}}%
\pgfpathlineto{\pgfqpoint{2.419571in}{4.331433in}}%
\pgfpathlineto{\pgfqpoint{2.565150in}{4.337534in}}%
\pgfpathlineto{\pgfqpoint{2.710730in}{4.327943in}}%
\pgfpathlineto{\pgfqpoint{2.856309in}{4.327158in}}%
\pgfpathlineto{\pgfqpoint{3.001889in}{4.324664in}}%
\pgfpathlineto{\pgfqpoint{3.147468in}{4.330845in}}%
\pgfpathlineto{\pgfqpoint{3.293047in}{4.334590in}}%
\pgfpathlineto{\pgfqpoint{3.438627in}{4.331444in}}%
\pgfpathlineto{\pgfqpoint{3.584206in}{4.334423in}}%
\pgfpathlineto{\pgfqpoint{3.729786in}{4.332055in}}%
\pgfusepath{stroke}%
\end{pgfscope}%
\begin{pgfscope}%
\pgfpathrectangle{\pgfqpoint{0.672569in}{3.363889in}}{\pgfqpoint{3.203819in}{2.263889in}}%
\pgfusepath{clip}%
\pgfsetrectcap%
\pgfsetroundjoin%
\pgfsetlinewidth{1.505625pt}%
\definecolor{currentstroke}{rgb}{0.580392,0.403922,0.741176}%
\pgfsetstrokecolor{currentstroke}%
\pgfsetdash{}{0pt}%
\pgfpathmoveto{\pgfqpoint{0.818198in}{4.109514in}}%
\pgfpathlineto{\pgfqpoint{0.963777in}{3.982178in}}%
\pgfpathlineto{\pgfqpoint{1.109356in}{3.945122in}}%
\pgfpathlineto{\pgfqpoint{1.254936in}{3.957958in}}%
\pgfpathlineto{\pgfqpoint{1.400515in}{4.012700in}}%
\pgfpathlineto{\pgfqpoint{1.546095in}{4.027976in}}%
\pgfpathlineto{\pgfqpoint{1.691674in}{3.988955in}}%
\pgfpathlineto{\pgfqpoint{1.837253in}{3.983187in}}%
\pgfpathlineto{\pgfqpoint{1.982833in}{3.981111in}}%
\pgfpathlineto{\pgfqpoint{2.128412in}{3.975360in}}%
\pgfpathlineto{\pgfqpoint{2.273992in}{3.981842in}}%
\pgfpathlineto{\pgfqpoint{2.419571in}{3.991072in}}%
\pgfpathlineto{\pgfqpoint{2.565150in}{3.987831in}}%
\pgfpathlineto{\pgfqpoint{2.710730in}{3.989218in}}%
\pgfpathlineto{\pgfqpoint{2.856309in}{3.987588in}}%
\pgfpathlineto{\pgfqpoint{3.001889in}{3.991650in}}%
\pgfpathlineto{\pgfqpoint{3.147468in}{3.988983in}}%
\pgfpathlineto{\pgfqpoint{3.293047in}{3.988435in}}%
\pgfpathlineto{\pgfqpoint{3.438627in}{3.989009in}}%
\pgfpathlineto{\pgfqpoint{3.584206in}{3.988020in}}%
\pgfpathlineto{\pgfqpoint{3.729786in}{3.988472in}}%
\pgfusepath{stroke}%
\end{pgfscope}%
\begin{pgfscope}%
\pgfsetrectcap%
\pgfsetmiterjoin%
\pgfsetlinewidth{0.803000pt}%
\definecolor{currentstroke}{rgb}{0.000000,0.000000,0.000000}%
\pgfsetstrokecolor{currentstroke}%
\pgfsetdash{}{0pt}%
\pgfpathmoveto{\pgfqpoint{0.672569in}{3.363889in}}%
\pgfpathlineto{\pgfqpoint{0.672569in}{5.627778in}}%
\pgfusepath{stroke}%
\end{pgfscope}%
\begin{pgfscope}%
\pgfsetrectcap%
\pgfsetmiterjoin%
\pgfsetlinewidth{0.803000pt}%
\definecolor{currentstroke}{rgb}{0.000000,0.000000,0.000000}%
\pgfsetstrokecolor{currentstroke}%
\pgfsetdash{}{0pt}%
\pgfpathmoveto{\pgfqpoint{3.876389in}{3.363889in}}%
\pgfpathlineto{\pgfqpoint{3.876389in}{5.627778in}}%
\pgfusepath{stroke}%
\end{pgfscope}%
\begin{pgfscope}%
\pgfsetrectcap%
\pgfsetmiterjoin%
\pgfsetlinewidth{0.803000pt}%
\definecolor{currentstroke}{rgb}{0.000000,0.000000,0.000000}%
\pgfsetstrokecolor{currentstroke}%
\pgfsetdash{}{0pt}%
\pgfpathmoveto{\pgfqpoint{0.672569in}{3.363889in}}%
\pgfpathlineto{\pgfqpoint{3.876389in}{3.363889in}}%
\pgfusepath{stroke}%
\end{pgfscope}%
\begin{pgfscope}%
\pgfsetrectcap%
\pgfsetmiterjoin%
\pgfsetlinewidth{0.803000pt}%
\definecolor{currentstroke}{rgb}{0.000000,0.000000,0.000000}%
\pgfsetstrokecolor{currentstroke}%
\pgfsetdash{}{0pt}%
\pgfpathmoveto{\pgfqpoint{0.672569in}{5.627778in}}%
\pgfpathlineto{\pgfqpoint{3.876389in}{5.627778in}}%
\pgfusepath{stroke}%
\end{pgfscope}%
\begin{pgfscope}%
\definecolor{textcolor}{rgb}{0.000000,0.000000,0.000000}%
\pgfsetstrokecolor{textcolor}%
\pgfsetfillcolor{textcolor}%
\pgftext[x=2.274479in,y=5.711111in,,base]{\color{textcolor}\sffamily\fontsize{12.000000}{14.400000}\selectfont \(\displaystyle  N = 16 \)}%
\end{pgfscope}%
\begin{pgfscope}%
\pgfsetbuttcap%
\pgfsetmiterjoin%
\definecolor{currentfill}{rgb}{1.000000,1.000000,1.000000}%
\pgfsetfillcolor{currentfill}%
\pgfsetlinewidth{0.000000pt}%
\definecolor{currentstroke}{rgb}{0.000000,0.000000,0.000000}%
\pgfsetstrokecolor{currentstroke}%
\pgfsetstrokeopacity{0.000000}%
\pgfsetdash{}{0pt}%
\pgfpathmoveto{\pgfqpoint{4.597569in}{3.363889in}}%
\pgfpathlineto{\pgfqpoint{7.801389in}{3.363889in}}%
\pgfpathlineto{\pgfqpoint{7.801389in}{5.627778in}}%
\pgfpathlineto{\pgfqpoint{4.597569in}{5.627778in}}%
\pgfpathclose%
\pgfusepath{fill}%
\end{pgfscope}%
\begin{pgfscope}%
\pgfpathrectangle{\pgfqpoint{4.597569in}{3.363889in}}{\pgfqpoint{3.203819in}{2.263889in}}%
\pgfusepath{clip}%
\pgfsetbuttcap%
\pgfsetroundjoin%
\definecolor{currentfill}{rgb}{0.121569,0.466667,0.705882}%
\pgfsetfillcolor{currentfill}%
\pgfsetlinewidth{1.003750pt}%
\definecolor{currentstroke}{rgb}{0.121569,0.466667,0.705882}%
\pgfsetstrokecolor{currentstroke}%
\pgfsetdash{}{0pt}%
\pgfsys@defobject{currentmarker}{\pgfqpoint{-0.009821in}{-0.009821in}}{\pgfqpoint{0.009821in}{0.009821in}}{%
\pgfpathmoveto{\pgfqpoint{0.000000in}{-0.009821in}}%
\pgfpathcurveto{\pgfqpoint{0.002605in}{-0.009821in}}{\pgfqpoint{0.005103in}{-0.008786in}}{\pgfqpoint{0.006944in}{-0.006944in}}%
\pgfpathcurveto{\pgfqpoint{0.008786in}{-0.005103in}}{\pgfqpoint{0.009821in}{-0.002605in}}{\pgfqpoint{0.009821in}{0.000000in}}%
\pgfpathcurveto{\pgfqpoint{0.009821in}{0.002605in}}{\pgfqpoint{0.008786in}{0.005103in}}{\pgfqpoint{0.006944in}{0.006944in}}%
\pgfpathcurveto{\pgfqpoint{0.005103in}{0.008786in}}{\pgfqpoint{0.002605in}{0.009821in}}{\pgfqpoint{0.000000in}{0.009821in}}%
\pgfpathcurveto{\pgfqpoint{-0.002605in}{0.009821in}}{\pgfqpoint{-0.005103in}{0.008786in}}{\pgfqpoint{-0.006944in}{0.006944in}}%
\pgfpathcurveto{\pgfqpoint{-0.008786in}{0.005103in}}{\pgfqpoint{-0.009821in}{0.002605in}}{\pgfqpoint{-0.009821in}{0.000000in}}%
\pgfpathcurveto{\pgfqpoint{-0.009821in}{-0.002605in}}{\pgfqpoint{-0.008786in}{-0.005103in}}{\pgfqpoint{-0.006944in}{-0.006944in}}%
\pgfpathcurveto{\pgfqpoint{-0.005103in}{-0.008786in}}{\pgfqpoint{-0.002605in}{-0.009821in}}{\pgfqpoint{0.000000in}{-0.009821in}}%
\pgfpathclose%
\pgfusepath{stroke,fill}%
}%
\begin{pgfscope}%
\pgfsys@transformshift{4.743198in}{3.789285in}%
\pgfsys@useobject{currentmarker}{}%
\end{pgfscope}%
\begin{pgfscope}%
\pgfsys@transformshift{4.888777in}{4.181473in}%
\pgfsys@useobject{currentmarker}{}%
\end{pgfscope}%
\begin{pgfscope}%
\pgfsys@transformshift{5.034356in}{4.235616in}%
\pgfsys@useobject{currentmarker}{}%
\end{pgfscope}%
\begin{pgfscope}%
\pgfsys@transformshift{5.179936in}{4.722540in}%
\pgfsys@useobject{currentmarker}{}%
\end{pgfscope}%
\begin{pgfscope}%
\pgfsys@transformshift{5.325515in}{4.508485in}%
\pgfsys@useobject{currentmarker}{}%
\end{pgfscope}%
\begin{pgfscope}%
\pgfsys@transformshift{5.471095in}{4.730858in}%
\pgfsys@useobject{currentmarker}{}%
\end{pgfscope}%
\begin{pgfscope}%
\pgfsys@transformshift{5.616674in}{4.708252in}%
\pgfsys@useobject{currentmarker}{}%
\end{pgfscope}%
\begin{pgfscope}%
\pgfsys@transformshift{5.762253in}{4.704728in}%
\pgfsys@useobject{currentmarker}{}%
\end{pgfscope}%
\begin{pgfscope}%
\pgfsys@transformshift{5.907833in}{5.020398in}%
\pgfsys@useobject{currentmarker}{}%
\end{pgfscope}%
\begin{pgfscope}%
\pgfsys@transformshift{6.053412in}{5.020400in}%
\pgfsys@useobject{currentmarker}{}%
\end{pgfscope}%
\begin{pgfscope}%
\pgfsys@transformshift{6.198992in}{5.020401in}%
\pgfsys@useobject{currentmarker}{}%
\end{pgfscope}%
\begin{pgfscope}%
\pgfsys@transformshift{6.344571in}{5.020401in}%
\pgfsys@useobject{currentmarker}{}%
\end{pgfscope}%
\begin{pgfscope}%
\pgfsys@transformshift{6.490150in}{5.020402in}%
\pgfsys@useobject{currentmarker}{}%
\end{pgfscope}%
\begin{pgfscope}%
\pgfsys@transformshift{6.635730in}{5.020401in}%
\pgfsys@useobject{currentmarker}{}%
\end{pgfscope}%
\begin{pgfscope}%
\pgfsys@transformshift{6.781309in}{5.020402in}%
\pgfsys@useobject{currentmarker}{}%
\end{pgfscope}%
\begin{pgfscope}%
\pgfsys@transformshift{6.926889in}{5.020401in}%
\pgfsys@useobject{currentmarker}{}%
\end{pgfscope}%
\begin{pgfscope}%
\pgfsys@transformshift{7.072468in}{5.020401in}%
\pgfsys@useobject{currentmarker}{}%
\end{pgfscope}%
\begin{pgfscope}%
\pgfsys@transformshift{7.218047in}{5.020401in}%
\pgfsys@useobject{currentmarker}{}%
\end{pgfscope}%
\begin{pgfscope}%
\pgfsys@transformshift{7.363627in}{5.020401in}%
\pgfsys@useobject{currentmarker}{}%
\end{pgfscope}%
\begin{pgfscope}%
\pgfsys@transformshift{7.509206in}{5.020401in}%
\pgfsys@useobject{currentmarker}{}%
\end{pgfscope}%
\begin{pgfscope}%
\pgfsys@transformshift{7.654786in}{5.020401in}%
\pgfsys@useobject{currentmarker}{}%
\end{pgfscope}%
\end{pgfscope}%
\begin{pgfscope}%
\pgfpathrectangle{\pgfqpoint{4.597569in}{3.363889in}}{\pgfqpoint{3.203819in}{2.263889in}}%
\pgfusepath{clip}%
\pgfsetbuttcap%
\pgfsetroundjoin%
\definecolor{currentfill}{rgb}{0.121569,0.466667,0.705882}%
\pgfsetfillcolor{currentfill}%
\pgfsetfillopacity{0.300000}%
\pgfsetlinewidth{1.003750pt}%
\definecolor{currentstroke}{rgb}{0.121569,0.466667,0.705882}%
\pgfsetstrokecolor{currentstroke}%
\pgfsetstrokeopacity{0.300000}%
\pgfsetdash{}{0pt}%
\pgfpathmoveto{\pgfqpoint{4.743198in}{3.937315in}}%
\pgfpathlineto{\pgfqpoint{4.743198in}{3.641256in}}%
\pgfpathlineto{\pgfqpoint{4.888777in}{3.750010in}}%
\pgfpathlineto{\pgfqpoint{5.034356in}{3.466793in}}%
\pgfpathlineto{\pgfqpoint{5.179936in}{3.961933in}}%
\pgfpathlineto{\pgfqpoint{5.325515in}{3.702879in}}%
\pgfpathlineto{\pgfqpoint{5.471095in}{3.978610in}}%
\pgfpathlineto{\pgfqpoint{5.616674in}{3.897269in}}%
\pgfpathlineto{\pgfqpoint{5.762253in}{3.884582in}}%
\pgfpathlineto{\pgfqpoint{5.907833in}{5.020395in}}%
\pgfpathlineto{\pgfqpoint{6.053412in}{5.020396in}}%
\pgfpathlineto{\pgfqpoint{6.198992in}{5.020398in}}%
\pgfpathlineto{\pgfqpoint{6.344571in}{5.020398in}}%
\pgfpathlineto{\pgfqpoint{6.490150in}{5.020400in}}%
\pgfpathlineto{\pgfqpoint{6.635730in}{5.020401in}}%
\pgfpathlineto{\pgfqpoint{6.781309in}{5.020401in}}%
\pgfpathlineto{\pgfqpoint{6.926889in}{5.020400in}}%
\pgfpathlineto{\pgfqpoint{7.072468in}{5.020401in}}%
\pgfpathlineto{\pgfqpoint{7.218047in}{5.020401in}}%
\pgfpathlineto{\pgfqpoint{7.363627in}{5.020401in}}%
\pgfpathlineto{\pgfqpoint{7.509206in}{5.020401in}}%
\pgfpathlineto{\pgfqpoint{7.654786in}{5.020401in}}%
\pgfpathlineto{\pgfqpoint{7.654786in}{5.020401in}}%
\pgfpathlineto{\pgfqpoint{7.654786in}{5.020401in}}%
\pgfpathlineto{\pgfqpoint{7.509206in}{5.020401in}}%
\pgfpathlineto{\pgfqpoint{7.363627in}{5.020402in}}%
\pgfpathlineto{\pgfqpoint{7.218047in}{5.020402in}}%
\pgfpathlineto{\pgfqpoint{7.072468in}{5.020402in}}%
\pgfpathlineto{\pgfqpoint{6.926889in}{5.020401in}}%
\pgfpathlineto{\pgfqpoint{6.781309in}{5.020402in}}%
\pgfpathlineto{\pgfqpoint{6.635730in}{5.020402in}}%
\pgfpathlineto{\pgfqpoint{6.490150in}{5.020403in}}%
\pgfpathlineto{\pgfqpoint{6.344571in}{5.020404in}}%
\pgfpathlineto{\pgfqpoint{6.198992in}{5.020403in}}%
\pgfpathlineto{\pgfqpoint{6.053412in}{5.020404in}}%
\pgfpathlineto{\pgfqpoint{5.907833in}{5.020402in}}%
\pgfpathlineto{\pgfqpoint{5.762253in}{5.524874in}}%
\pgfpathlineto{\pgfqpoint{5.616674in}{5.519236in}}%
\pgfpathlineto{\pgfqpoint{5.471095in}{5.483106in}}%
\pgfpathlineto{\pgfqpoint{5.325515in}{5.314091in}}%
\pgfpathlineto{\pgfqpoint{5.179936in}{5.483147in}}%
\pgfpathlineto{\pgfqpoint{5.034356in}{5.004439in}}%
\pgfpathlineto{\pgfqpoint{4.888777in}{4.612936in}}%
\pgfpathlineto{\pgfqpoint{4.743198in}{3.937315in}}%
\pgfpathclose%
\pgfusepath{stroke,fill}%
\end{pgfscope}%
\begin{pgfscope}%
\pgfpathrectangle{\pgfqpoint{4.597569in}{3.363889in}}{\pgfqpoint{3.203819in}{2.263889in}}%
\pgfusepath{clip}%
\pgfsetbuttcap%
\pgfsetroundjoin%
\definecolor{currentfill}{rgb}{1.000000,0.498039,0.054902}%
\pgfsetfillcolor{currentfill}%
\pgfsetlinewidth{1.003750pt}%
\definecolor{currentstroke}{rgb}{1.000000,0.498039,0.054902}%
\pgfsetstrokecolor{currentstroke}%
\pgfsetdash{}{0pt}%
\pgfsys@defobject{currentmarker}{\pgfqpoint{-0.009821in}{-0.009821in}}{\pgfqpoint{0.009821in}{0.009821in}}{%
\pgfpathmoveto{\pgfqpoint{0.000000in}{-0.009821in}}%
\pgfpathcurveto{\pgfqpoint{0.002605in}{-0.009821in}}{\pgfqpoint{0.005103in}{-0.008786in}}{\pgfqpoint{0.006944in}{-0.006944in}}%
\pgfpathcurveto{\pgfqpoint{0.008786in}{-0.005103in}}{\pgfqpoint{0.009821in}{-0.002605in}}{\pgfqpoint{0.009821in}{0.000000in}}%
\pgfpathcurveto{\pgfqpoint{0.009821in}{0.002605in}}{\pgfqpoint{0.008786in}{0.005103in}}{\pgfqpoint{0.006944in}{0.006944in}}%
\pgfpathcurveto{\pgfqpoint{0.005103in}{0.008786in}}{\pgfqpoint{0.002605in}{0.009821in}}{\pgfqpoint{0.000000in}{0.009821in}}%
\pgfpathcurveto{\pgfqpoint{-0.002605in}{0.009821in}}{\pgfqpoint{-0.005103in}{0.008786in}}{\pgfqpoint{-0.006944in}{0.006944in}}%
\pgfpathcurveto{\pgfqpoint{-0.008786in}{0.005103in}}{\pgfqpoint{-0.009821in}{0.002605in}}{\pgfqpoint{-0.009821in}{0.000000in}}%
\pgfpathcurveto{\pgfqpoint{-0.009821in}{-0.002605in}}{\pgfqpoint{-0.008786in}{-0.005103in}}{\pgfqpoint{-0.006944in}{-0.006944in}}%
\pgfpathcurveto{\pgfqpoint{-0.005103in}{-0.008786in}}{\pgfqpoint{-0.002605in}{-0.009821in}}{\pgfqpoint{0.000000in}{-0.009821in}}%
\pgfpathclose%
\pgfusepath{stroke,fill}%
}%
\begin{pgfscope}%
\pgfsys@transformshift{4.743198in}{3.714267in}%
\pgfsys@useobject{currentmarker}{}%
\end{pgfscope}%
\begin{pgfscope}%
\pgfsys@transformshift{4.888777in}{3.804735in}%
\pgfsys@useobject{currentmarker}{}%
\end{pgfscope}%
\begin{pgfscope}%
\pgfsys@transformshift{5.034356in}{4.207125in}%
\pgfsys@useobject{currentmarker}{}%
\end{pgfscope}%
\begin{pgfscope}%
\pgfsys@transformshift{5.179936in}{4.651449in}%
\pgfsys@useobject{currentmarker}{}%
\end{pgfscope}%
\begin{pgfscope}%
\pgfsys@transformshift{5.325515in}{4.688843in}%
\pgfsys@useobject{currentmarker}{}%
\end{pgfscope}%
\begin{pgfscope}%
\pgfsys@transformshift{5.471095in}{4.397384in}%
\pgfsys@useobject{currentmarker}{}%
\end{pgfscope}%
\begin{pgfscope}%
\pgfsys@transformshift{5.616674in}{4.979136in}%
\pgfsys@useobject{currentmarker}{}%
\end{pgfscope}%
\begin{pgfscope}%
\pgfsys@transformshift{5.762253in}{4.874560in}%
\pgfsys@useobject{currentmarker}{}%
\end{pgfscope}%
\begin{pgfscope}%
\pgfsys@transformshift{5.907833in}{5.003656in}%
\pgfsys@useobject{currentmarker}{}%
\end{pgfscope}%
\begin{pgfscope}%
\pgfsys@transformshift{6.053412in}{5.003720in}%
\pgfsys@useobject{currentmarker}{}%
\end{pgfscope}%
\begin{pgfscope}%
\pgfsys@transformshift{6.198992in}{5.003717in}%
\pgfsys@useobject{currentmarker}{}%
\end{pgfscope}%
\begin{pgfscope}%
\pgfsys@transformshift{6.344571in}{5.003742in}%
\pgfsys@useobject{currentmarker}{}%
\end{pgfscope}%
\begin{pgfscope}%
\pgfsys@transformshift{6.490150in}{5.003742in}%
\pgfsys@useobject{currentmarker}{}%
\end{pgfscope}%
\begin{pgfscope}%
\pgfsys@transformshift{6.635730in}{5.003711in}%
\pgfsys@useobject{currentmarker}{}%
\end{pgfscope}%
\begin{pgfscope}%
\pgfsys@transformshift{6.781309in}{5.003748in}%
\pgfsys@useobject{currentmarker}{}%
\end{pgfscope}%
\begin{pgfscope}%
\pgfsys@transformshift{6.926889in}{5.003744in}%
\pgfsys@useobject{currentmarker}{}%
\end{pgfscope}%
\begin{pgfscope}%
\pgfsys@transformshift{7.072468in}{5.003735in}%
\pgfsys@useobject{currentmarker}{}%
\end{pgfscope}%
\begin{pgfscope}%
\pgfsys@transformshift{7.218047in}{5.003743in}%
\pgfsys@useobject{currentmarker}{}%
\end{pgfscope}%
\begin{pgfscope}%
\pgfsys@transformshift{7.363627in}{5.003737in}%
\pgfsys@useobject{currentmarker}{}%
\end{pgfscope}%
\begin{pgfscope}%
\pgfsys@transformshift{7.509206in}{5.003734in}%
\pgfsys@useobject{currentmarker}{}%
\end{pgfscope}%
\begin{pgfscope}%
\pgfsys@transformshift{7.654786in}{5.003739in}%
\pgfsys@useobject{currentmarker}{}%
\end{pgfscope}%
\end{pgfscope}%
\begin{pgfscope}%
\pgfpathrectangle{\pgfqpoint{4.597569in}{3.363889in}}{\pgfqpoint{3.203819in}{2.263889in}}%
\pgfusepath{clip}%
\pgfsetbuttcap%
\pgfsetroundjoin%
\definecolor{currentfill}{rgb}{1.000000,0.498039,0.054902}%
\pgfsetfillcolor{currentfill}%
\pgfsetfillopacity{0.300000}%
\pgfsetlinewidth{1.003750pt}%
\definecolor{currentstroke}{rgb}{1.000000,0.498039,0.054902}%
\pgfsetstrokecolor{currentstroke}%
\pgfsetstrokeopacity{0.300000}%
\pgfsetdash{}{0pt}%
\pgfpathmoveto{\pgfqpoint{4.743198in}{3.814675in}}%
\pgfpathlineto{\pgfqpoint{4.743198in}{3.613859in}}%
\pgfpathlineto{\pgfqpoint{4.888777in}{3.728039in}}%
\pgfpathlineto{\pgfqpoint{5.034356in}{3.524229in}}%
\pgfpathlineto{\pgfqpoint{5.179936in}{3.806981in}}%
\pgfpathlineto{\pgfqpoint{5.325515in}{3.898391in}}%
\pgfpathlineto{\pgfqpoint{5.471095in}{3.487303in}}%
\pgfpathlineto{\pgfqpoint{5.616674in}{4.915096in}}%
\pgfpathlineto{\pgfqpoint{5.762253in}{4.538876in}}%
\pgfpathlineto{\pgfqpoint{5.907833in}{5.003423in}}%
\pgfpathlineto{\pgfqpoint{6.053412in}{5.003632in}}%
\pgfpathlineto{\pgfqpoint{6.198992in}{5.003606in}}%
\pgfpathlineto{\pgfqpoint{6.344571in}{5.003674in}}%
\pgfpathlineto{\pgfqpoint{6.490150in}{5.003696in}}%
\pgfpathlineto{\pgfqpoint{6.635730in}{5.003653in}}%
\pgfpathlineto{\pgfqpoint{6.781309in}{5.003706in}}%
\pgfpathlineto{\pgfqpoint{6.926889in}{5.003713in}}%
\pgfpathlineto{\pgfqpoint{7.072468in}{5.003723in}}%
\pgfpathlineto{\pgfqpoint{7.218047in}{5.003725in}}%
\pgfpathlineto{\pgfqpoint{7.363627in}{5.003733in}}%
\pgfpathlineto{\pgfqpoint{7.509206in}{5.003725in}}%
\pgfpathlineto{\pgfqpoint{7.654786in}{5.003734in}}%
\pgfpathlineto{\pgfqpoint{7.654786in}{5.003744in}}%
\pgfpathlineto{\pgfqpoint{7.654786in}{5.003744in}}%
\pgfpathlineto{\pgfqpoint{7.509206in}{5.003743in}}%
\pgfpathlineto{\pgfqpoint{7.363627in}{5.003741in}}%
\pgfpathlineto{\pgfqpoint{7.218047in}{5.003760in}}%
\pgfpathlineto{\pgfqpoint{7.072468in}{5.003748in}}%
\pgfpathlineto{\pgfqpoint{6.926889in}{5.003775in}}%
\pgfpathlineto{\pgfqpoint{6.781309in}{5.003790in}}%
\pgfpathlineto{\pgfqpoint{6.635730in}{5.003769in}}%
\pgfpathlineto{\pgfqpoint{6.490150in}{5.003787in}}%
\pgfpathlineto{\pgfqpoint{6.344571in}{5.003809in}}%
\pgfpathlineto{\pgfqpoint{6.198992in}{5.003828in}}%
\pgfpathlineto{\pgfqpoint{6.053412in}{5.003809in}}%
\pgfpathlineto{\pgfqpoint{5.907833in}{5.003888in}}%
\pgfpathlineto{\pgfqpoint{5.762253in}{5.210245in}}%
\pgfpathlineto{\pgfqpoint{5.616674in}{5.043176in}}%
\pgfpathlineto{\pgfqpoint{5.471095in}{5.307465in}}%
\pgfpathlineto{\pgfqpoint{5.325515in}{5.479296in}}%
\pgfpathlineto{\pgfqpoint{5.179936in}{5.495918in}}%
\pgfpathlineto{\pgfqpoint{5.034356in}{4.890021in}}%
\pgfpathlineto{\pgfqpoint{4.888777in}{3.881431in}}%
\pgfpathlineto{\pgfqpoint{4.743198in}{3.814675in}}%
\pgfpathclose%
\pgfusepath{stroke,fill}%
\end{pgfscope}%
\begin{pgfscope}%
\pgfpathrectangle{\pgfqpoint{4.597569in}{3.363889in}}{\pgfqpoint{3.203819in}{2.263889in}}%
\pgfusepath{clip}%
\pgfsetbuttcap%
\pgfsetroundjoin%
\definecolor{currentfill}{rgb}{0.172549,0.627451,0.172549}%
\pgfsetfillcolor{currentfill}%
\pgfsetlinewidth{1.003750pt}%
\definecolor{currentstroke}{rgb}{0.172549,0.627451,0.172549}%
\pgfsetstrokecolor{currentstroke}%
\pgfsetdash{}{0pt}%
\pgfsys@defobject{currentmarker}{\pgfqpoint{-0.009821in}{-0.009821in}}{\pgfqpoint{0.009821in}{0.009821in}}{%
\pgfpathmoveto{\pgfqpoint{0.000000in}{-0.009821in}}%
\pgfpathcurveto{\pgfqpoint{0.002605in}{-0.009821in}}{\pgfqpoint{0.005103in}{-0.008786in}}{\pgfqpoint{0.006944in}{-0.006944in}}%
\pgfpathcurveto{\pgfqpoint{0.008786in}{-0.005103in}}{\pgfqpoint{0.009821in}{-0.002605in}}{\pgfqpoint{0.009821in}{0.000000in}}%
\pgfpathcurveto{\pgfqpoint{0.009821in}{0.002605in}}{\pgfqpoint{0.008786in}{0.005103in}}{\pgfqpoint{0.006944in}{0.006944in}}%
\pgfpathcurveto{\pgfqpoint{0.005103in}{0.008786in}}{\pgfqpoint{0.002605in}{0.009821in}}{\pgfqpoint{0.000000in}{0.009821in}}%
\pgfpathcurveto{\pgfqpoint{-0.002605in}{0.009821in}}{\pgfqpoint{-0.005103in}{0.008786in}}{\pgfqpoint{-0.006944in}{0.006944in}}%
\pgfpathcurveto{\pgfqpoint{-0.008786in}{0.005103in}}{\pgfqpoint{-0.009821in}{0.002605in}}{\pgfqpoint{-0.009821in}{0.000000in}}%
\pgfpathcurveto{\pgfqpoint{-0.009821in}{-0.002605in}}{\pgfqpoint{-0.008786in}{-0.005103in}}{\pgfqpoint{-0.006944in}{-0.006944in}}%
\pgfpathcurveto{\pgfqpoint{-0.005103in}{-0.008786in}}{\pgfqpoint{-0.002605in}{-0.009821in}}{\pgfqpoint{0.000000in}{-0.009821in}}%
\pgfpathclose%
\pgfusepath{stroke,fill}%
}%
\begin{pgfscope}%
\pgfsys@transformshift{4.743198in}{3.784815in}%
\pgfsys@useobject{currentmarker}{}%
\end{pgfscope}%
\begin{pgfscope}%
\pgfsys@transformshift{4.888777in}{3.727902in}%
\pgfsys@useobject{currentmarker}{}%
\end{pgfscope}%
\begin{pgfscope}%
\pgfsys@transformshift{5.034356in}{4.075821in}%
\pgfsys@useobject{currentmarker}{}%
\end{pgfscope}%
\begin{pgfscope}%
\pgfsys@transformshift{5.179936in}{4.411848in}%
\pgfsys@useobject{currentmarker}{}%
\end{pgfscope}%
\begin{pgfscope}%
\pgfsys@transformshift{5.325515in}{4.093955in}%
\pgfsys@useobject{currentmarker}{}%
\end{pgfscope}%
\begin{pgfscope}%
\pgfsys@transformshift{5.471095in}{4.323982in}%
\pgfsys@useobject{currentmarker}{}%
\end{pgfscope}%
\begin{pgfscope}%
\pgfsys@transformshift{5.616674in}{4.843650in}%
\pgfsys@useobject{currentmarker}{}%
\end{pgfscope}%
\begin{pgfscope}%
\pgfsys@transformshift{5.762253in}{4.899680in}%
\pgfsys@useobject{currentmarker}{}%
\end{pgfscope}%
\begin{pgfscope}%
\pgfsys@transformshift{5.907833in}{4.897408in}%
\pgfsys@useobject{currentmarker}{}%
\end{pgfscope}%
\begin{pgfscope}%
\pgfsys@transformshift{6.053412in}{4.894732in}%
\pgfsys@useobject{currentmarker}{}%
\end{pgfscope}%
\begin{pgfscope}%
\pgfsys@transformshift{6.198992in}{4.896714in}%
\pgfsys@useobject{currentmarker}{}%
\end{pgfscope}%
\begin{pgfscope}%
\pgfsys@transformshift{6.344571in}{4.897550in}%
\pgfsys@useobject{currentmarker}{}%
\end{pgfscope}%
\begin{pgfscope}%
\pgfsys@transformshift{6.490150in}{4.897269in}%
\pgfsys@useobject{currentmarker}{}%
\end{pgfscope}%
\begin{pgfscope}%
\pgfsys@transformshift{6.635730in}{4.897600in}%
\pgfsys@useobject{currentmarker}{}%
\end{pgfscope}%
\begin{pgfscope}%
\pgfsys@transformshift{6.781309in}{4.897312in}%
\pgfsys@useobject{currentmarker}{}%
\end{pgfscope}%
\begin{pgfscope}%
\pgfsys@transformshift{6.926889in}{4.897467in}%
\pgfsys@useobject{currentmarker}{}%
\end{pgfscope}%
\begin{pgfscope}%
\pgfsys@transformshift{7.072468in}{4.897223in}%
\pgfsys@useobject{currentmarker}{}%
\end{pgfscope}%
\begin{pgfscope}%
\pgfsys@transformshift{7.218047in}{4.897223in}%
\pgfsys@useobject{currentmarker}{}%
\end{pgfscope}%
\begin{pgfscope}%
\pgfsys@transformshift{7.363627in}{4.897436in}%
\pgfsys@useobject{currentmarker}{}%
\end{pgfscope}%
\begin{pgfscope}%
\pgfsys@transformshift{7.509206in}{4.897410in}%
\pgfsys@useobject{currentmarker}{}%
\end{pgfscope}%
\begin{pgfscope}%
\pgfsys@transformshift{7.654786in}{4.897379in}%
\pgfsys@useobject{currentmarker}{}%
\end{pgfscope}%
\end{pgfscope}%
\begin{pgfscope}%
\pgfpathrectangle{\pgfqpoint{4.597569in}{3.363889in}}{\pgfqpoint{3.203819in}{2.263889in}}%
\pgfusepath{clip}%
\pgfsetbuttcap%
\pgfsetroundjoin%
\definecolor{currentfill}{rgb}{0.172549,0.627451,0.172549}%
\pgfsetfillcolor{currentfill}%
\pgfsetfillopacity{0.300000}%
\pgfsetlinewidth{1.003750pt}%
\definecolor{currentstroke}{rgb}{0.172549,0.627451,0.172549}%
\pgfsetstrokecolor{currentstroke}%
\pgfsetstrokeopacity{0.300000}%
\pgfsetdash{}{0pt}%
\pgfpathmoveto{\pgfqpoint{4.743198in}{3.897746in}}%
\pgfpathlineto{\pgfqpoint{4.743198in}{3.671883in}}%
\pgfpathlineto{\pgfqpoint{4.888777in}{3.549823in}}%
\pgfpathlineto{\pgfqpoint{5.034356in}{3.961455in}}%
\pgfpathlineto{\pgfqpoint{5.179936in}{3.846558in}}%
\pgfpathlineto{\pgfqpoint{5.325515in}{3.492902in}}%
\pgfpathlineto{\pgfqpoint{5.471095in}{3.800993in}}%
\pgfpathlineto{\pgfqpoint{5.616674in}{4.718087in}}%
\pgfpathlineto{\pgfqpoint{5.762253in}{4.894706in}}%
\pgfpathlineto{\pgfqpoint{5.907833in}{4.892848in}}%
\pgfpathlineto{\pgfqpoint{6.053412in}{4.892024in}}%
\pgfpathlineto{\pgfqpoint{6.198992in}{4.894109in}}%
\pgfpathlineto{\pgfqpoint{6.344571in}{4.896828in}}%
\pgfpathlineto{\pgfqpoint{6.490150in}{4.895892in}}%
\pgfpathlineto{\pgfqpoint{6.635730in}{4.896276in}}%
\pgfpathlineto{\pgfqpoint{6.781309in}{4.896568in}}%
\pgfpathlineto{\pgfqpoint{6.926889in}{4.896299in}}%
\pgfpathlineto{\pgfqpoint{7.072468in}{4.896959in}}%
\pgfpathlineto{\pgfqpoint{7.218047in}{4.896799in}}%
\pgfpathlineto{\pgfqpoint{7.363627in}{4.897139in}}%
\pgfpathlineto{\pgfqpoint{7.509206in}{4.897229in}}%
\pgfpathlineto{\pgfqpoint{7.654786in}{4.897243in}}%
\pgfpathlineto{\pgfqpoint{7.654786in}{4.897514in}}%
\pgfpathlineto{\pgfqpoint{7.654786in}{4.897514in}}%
\pgfpathlineto{\pgfqpoint{7.509206in}{4.897590in}}%
\pgfpathlineto{\pgfqpoint{7.363627in}{4.897732in}}%
\pgfpathlineto{\pgfqpoint{7.218047in}{4.897647in}}%
\pgfpathlineto{\pgfqpoint{7.072468in}{4.897487in}}%
\pgfpathlineto{\pgfqpoint{6.926889in}{4.898635in}}%
\pgfpathlineto{\pgfqpoint{6.781309in}{4.898056in}}%
\pgfpathlineto{\pgfqpoint{6.635730in}{4.898924in}}%
\pgfpathlineto{\pgfqpoint{6.490150in}{4.898646in}}%
\pgfpathlineto{\pgfqpoint{6.344571in}{4.898273in}}%
\pgfpathlineto{\pgfqpoint{6.198992in}{4.899318in}}%
\pgfpathlineto{\pgfqpoint{6.053412in}{4.897440in}}%
\pgfpathlineto{\pgfqpoint{5.907833in}{4.901968in}}%
\pgfpathlineto{\pgfqpoint{5.762253in}{4.904654in}}%
\pgfpathlineto{\pgfqpoint{5.616674in}{4.969213in}}%
\pgfpathlineto{\pgfqpoint{5.471095in}{4.846971in}}%
\pgfpathlineto{\pgfqpoint{5.325515in}{4.695008in}}%
\pgfpathlineto{\pgfqpoint{5.179936in}{4.977138in}}%
\pgfpathlineto{\pgfqpoint{5.034356in}{4.190187in}}%
\pgfpathlineto{\pgfqpoint{4.888777in}{3.905981in}}%
\pgfpathlineto{\pgfqpoint{4.743198in}{3.897746in}}%
\pgfpathclose%
\pgfusepath{stroke,fill}%
\end{pgfscope}%
\begin{pgfscope}%
\pgfpathrectangle{\pgfqpoint{4.597569in}{3.363889in}}{\pgfqpoint{3.203819in}{2.263889in}}%
\pgfusepath{clip}%
\pgfsetbuttcap%
\pgfsetroundjoin%
\definecolor{currentfill}{rgb}{0.839216,0.152941,0.156863}%
\pgfsetfillcolor{currentfill}%
\pgfsetlinewidth{1.003750pt}%
\definecolor{currentstroke}{rgb}{0.839216,0.152941,0.156863}%
\pgfsetstrokecolor{currentstroke}%
\pgfsetdash{}{0pt}%
\pgfsys@defobject{currentmarker}{\pgfqpoint{-0.009821in}{-0.009821in}}{\pgfqpoint{0.009821in}{0.009821in}}{%
\pgfpathmoveto{\pgfqpoint{0.000000in}{-0.009821in}}%
\pgfpathcurveto{\pgfqpoint{0.002605in}{-0.009821in}}{\pgfqpoint{0.005103in}{-0.008786in}}{\pgfqpoint{0.006944in}{-0.006944in}}%
\pgfpathcurveto{\pgfqpoint{0.008786in}{-0.005103in}}{\pgfqpoint{0.009821in}{-0.002605in}}{\pgfqpoint{0.009821in}{0.000000in}}%
\pgfpathcurveto{\pgfqpoint{0.009821in}{0.002605in}}{\pgfqpoint{0.008786in}{0.005103in}}{\pgfqpoint{0.006944in}{0.006944in}}%
\pgfpathcurveto{\pgfqpoint{0.005103in}{0.008786in}}{\pgfqpoint{0.002605in}{0.009821in}}{\pgfqpoint{0.000000in}{0.009821in}}%
\pgfpathcurveto{\pgfqpoint{-0.002605in}{0.009821in}}{\pgfqpoint{-0.005103in}{0.008786in}}{\pgfqpoint{-0.006944in}{0.006944in}}%
\pgfpathcurveto{\pgfqpoint{-0.008786in}{0.005103in}}{\pgfqpoint{-0.009821in}{0.002605in}}{\pgfqpoint{-0.009821in}{0.000000in}}%
\pgfpathcurveto{\pgfqpoint{-0.009821in}{-0.002605in}}{\pgfqpoint{-0.008786in}{-0.005103in}}{\pgfqpoint{-0.006944in}{-0.006944in}}%
\pgfpathcurveto{\pgfqpoint{-0.005103in}{-0.008786in}}{\pgfqpoint{-0.002605in}{-0.009821in}}{\pgfqpoint{0.000000in}{-0.009821in}}%
\pgfpathclose%
\pgfusepath{stroke,fill}%
}%
\begin{pgfscope}%
\pgfsys@transformshift{4.743198in}{3.711585in}%
\pgfsys@useobject{currentmarker}{}%
\end{pgfscope}%
\begin{pgfscope}%
\pgfsys@transformshift{4.888777in}{3.825136in}%
\pgfsys@useobject{currentmarker}{}%
\end{pgfscope}%
\begin{pgfscope}%
\pgfsys@transformshift{5.034356in}{3.780969in}%
\pgfsys@useobject{currentmarker}{}%
\end{pgfscope}%
\begin{pgfscope}%
\pgfsys@transformshift{5.179936in}{3.687404in}%
\pgfsys@useobject{currentmarker}{}%
\end{pgfscope}%
\begin{pgfscope}%
\pgfsys@transformshift{5.325515in}{3.899836in}%
\pgfsys@useobject{currentmarker}{}%
\end{pgfscope}%
\begin{pgfscope}%
\pgfsys@transformshift{5.471095in}{3.761729in}%
\pgfsys@useobject{currentmarker}{}%
\end{pgfscope}%
\begin{pgfscope}%
\pgfsys@transformshift{5.616674in}{3.923507in}%
\pgfsys@useobject{currentmarker}{}%
\end{pgfscope}%
\begin{pgfscope}%
\pgfsys@transformshift{5.762253in}{3.898315in}%
\pgfsys@useobject{currentmarker}{}%
\end{pgfscope}%
\begin{pgfscope}%
\pgfsys@transformshift{5.907833in}{3.885477in}%
\pgfsys@useobject{currentmarker}{}%
\end{pgfscope}%
\begin{pgfscope}%
\pgfsys@transformshift{6.053412in}{3.923710in}%
\pgfsys@useobject{currentmarker}{}%
\end{pgfscope}%
\begin{pgfscope}%
\pgfsys@transformshift{6.198992in}{3.888030in}%
\pgfsys@useobject{currentmarker}{}%
\end{pgfscope}%
\begin{pgfscope}%
\pgfsys@transformshift{6.344571in}{3.883960in}%
\pgfsys@useobject{currentmarker}{}%
\end{pgfscope}%
\begin{pgfscope}%
\pgfsys@transformshift{6.490150in}{3.898700in}%
\pgfsys@useobject{currentmarker}{}%
\end{pgfscope}%
\begin{pgfscope}%
\pgfsys@transformshift{6.635730in}{3.893700in}%
\pgfsys@useobject{currentmarker}{}%
\end{pgfscope}%
\begin{pgfscope}%
\pgfsys@transformshift{6.781309in}{3.885673in}%
\pgfsys@useobject{currentmarker}{}%
\end{pgfscope}%
\begin{pgfscope}%
\pgfsys@transformshift{6.926889in}{3.889178in}%
\pgfsys@useobject{currentmarker}{}%
\end{pgfscope}%
\begin{pgfscope}%
\pgfsys@transformshift{7.072468in}{3.896434in}%
\pgfsys@useobject{currentmarker}{}%
\end{pgfscope}%
\begin{pgfscope}%
\pgfsys@transformshift{7.218047in}{3.888814in}%
\pgfsys@useobject{currentmarker}{}%
\end{pgfscope}%
\begin{pgfscope}%
\pgfsys@transformshift{7.363627in}{3.894636in}%
\pgfsys@useobject{currentmarker}{}%
\end{pgfscope}%
\begin{pgfscope}%
\pgfsys@transformshift{7.509206in}{3.890218in}%
\pgfsys@useobject{currentmarker}{}%
\end{pgfscope}%
\begin{pgfscope}%
\pgfsys@transformshift{7.654786in}{3.892888in}%
\pgfsys@useobject{currentmarker}{}%
\end{pgfscope}%
\end{pgfscope}%
\begin{pgfscope}%
\pgfpathrectangle{\pgfqpoint{4.597569in}{3.363889in}}{\pgfqpoint{3.203819in}{2.263889in}}%
\pgfusepath{clip}%
\pgfsetbuttcap%
\pgfsetroundjoin%
\definecolor{currentfill}{rgb}{0.839216,0.152941,0.156863}%
\pgfsetfillcolor{currentfill}%
\pgfsetfillopacity{0.300000}%
\pgfsetlinewidth{1.003750pt}%
\definecolor{currentstroke}{rgb}{0.839216,0.152941,0.156863}%
\pgfsetstrokecolor{currentstroke}%
\pgfsetstrokeopacity{0.300000}%
\pgfsetdash{}{0pt}%
\pgfpathmoveto{\pgfqpoint{4.743198in}{3.778202in}}%
\pgfpathlineto{\pgfqpoint{4.743198in}{3.644968in}}%
\pgfpathlineto{\pgfqpoint{4.888777in}{3.645372in}}%
\pgfpathlineto{\pgfqpoint{5.034356in}{3.550770in}}%
\pgfpathlineto{\pgfqpoint{5.179936in}{3.658330in}}%
\pgfpathlineto{\pgfqpoint{5.325515in}{3.658381in}}%
\pgfpathlineto{\pgfqpoint{5.471095in}{3.630778in}}%
\pgfpathlineto{\pgfqpoint{5.616674in}{3.763914in}}%
\pgfpathlineto{\pgfqpoint{5.762253in}{3.661719in}}%
\pgfpathlineto{\pgfqpoint{5.907833in}{3.828174in}}%
\pgfpathlineto{\pgfqpoint{6.053412in}{3.880339in}}%
\pgfpathlineto{\pgfqpoint{6.198992in}{3.827835in}}%
\pgfpathlineto{\pgfqpoint{6.344571in}{3.824290in}}%
\pgfpathlineto{\pgfqpoint{6.490150in}{3.859757in}}%
\pgfpathlineto{\pgfqpoint{6.635730in}{3.862608in}}%
\pgfpathlineto{\pgfqpoint{6.781309in}{3.860921in}}%
\pgfpathlineto{\pgfqpoint{6.926889in}{3.873217in}}%
\pgfpathlineto{\pgfqpoint{7.072468in}{3.881696in}}%
\pgfpathlineto{\pgfqpoint{7.218047in}{3.880550in}}%
\pgfpathlineto{\pgfqpoint{7.363627in}{3.889417in}}%
\pgfpathlineto{\pgfqpoint{7.509206in}{3.884674in}}%
\pgfpathlineto{\pgfqpoint{7.654786in}{3.887691in}}%
\pgfpathlineto{\pgfqpoint{7.654786in}{3.898085in}}%
\pgfpathlineto{\pgfqpoint{7.654786in}{3.898085in}}%
\pgfpathlineto{\pgfqpoint{7.509206in}{3.895761in}}%
\pgfpathlineto{\pgfqpoint{7.363627in}{3.899855in}}%
\pgfpathlineto{\pgfqpoint{7.218047in}{3.897079in}}%
\pgfpathlineto{\pgfqpoint{7.072468in}{3.911173in}}%
\pgfpathlineto{\pgfqpoint{6.926889in}{3.905139in}}%
\pgfpathlineto{\pgfqpoint{6.781309in}{3.910426in}}%
\pgfpathlineto{\pgfqpoint{6.635730in}{3.924793in}}%
\pgfpathlineto{\pgfqpoint{6.490150in}{3.937643in}}%
\pgfpathlineto{\pgfqpoint{6.344571in}{3.943631in}}%
\pgfpathlineto{\pgfqpoint{6.198992in}{3.948225in}}%
\pgfpathlineto{\pgfqpoint{6.053412in}{3.967081in}}%
\pgfpathlineto{\pgfqpoint{5.907833in}{3.942779in}}%
\pgfpathlineto{\pgfqpoint{5.762253in}{4.134912in}}%
\pgfpathlineto{\pgfqpoint{5.616674in}{4.083100in}}%
\pgfpathlineto{\pgfqpoint{5.471095in}{3.892680in}}%
\pgfpathlineto{\pgfqpoint{5.325515in}{4.141291in}}%
\pgfpathlineto{\pgfqpoint{5.179936in}{3.716479in}}%
\pgfpathlineto{\pgfqpoint{5.034356in}{4.011167in}}%
\pgfpathlineto{\pgfqpoint{4.888777in}{4.004899in}}%
\pgfpathlineto{\pgfqpoint{4.743198in}{3.778202in}}%
\pgfpathclose%
\pgfusepath{stroke,fill}%
\end{pgfscope}%
\begin{pgfscope}%
\pgfpathrectangle{\pgfqpoint{4.597569in}{3.363889in}}{\pgfqpoint{3.203819in}{2.263889in}}%
\pgfusepath{clip}%
\pgfsetbuttcap%
\pgfsetroundjoin%
\definecolor{currentfill}{rgb}{0.580392,0.403922,0.741176}%
\pgfsetfillcolor{currentfill}%
\pgfsetlinewidth{1.003750pt}%
\definecolor{currentstroke}{rgb}{0.580392,0.403922,0.741176}%
\pgfsetstrokecolor{currentstroke}%
\pgfsetdash{}{0pt}%
\pgfsys@defobject{currentmarker}{\pgfqpoint{-0.009821in}{-0.009821in}}{\pgfqpoint{0.009821in}{0.009821in}}{%
\pgfpathmoveto{\pgfqpoint{0.000000in}{-0.009821in}}%
\pgfpathcurveto{\pgfqpoint{0.002605in}{-0.009821in}}{\pgfqpoint{0.005103in}{-0.008786in}}{\pgfqpoint{0.006944in}{-0.006944in}}%
\pgfpathcurveto{\pgfqpoint{0.008786in}{-0.005103in}}{\pgfqpoint{0.009821in}{-0.002605in}}{\pgfqpoint{0.009821in}{0.000000in}}%
\pgfpathcurveto{\pgfqpoint{0.009821in}{0.002605in}}{\pgfqpoint{0.008786in}{0.005103in}}{\pgfqpoint{0.006944in}{0.006944in}}%
\pgfpathcurveto{\pgfqpoint{0.005103in}{0.008786in}}{\pgfqpoint{0.002605in}{0.009821in}}{\pgfqpoint{0.000000in}{0.009821in}}%
\pgfpathcurveto{\pgfqpoint{-0.002605in}{0.009821in}}{\pgfqpoint{-0.005103in}{0.008786in}}{\pgfqpoint{-0.006944in}{0.006944in}}%
\pgfpathcurveto{\pgfqpoint{-0.008786in}{0.005103in}}{\pgfqpoint{-0.009821in}{0.002605in}}{\pgfqpoint{-0.009821in}{0.000000in}}%
\pgfpathcurveto{\pgfqpoint{-0.009821in}{-0.002605in}}{\pgfqpoint{-0.008786in}{-0.005103in}}{\pgfqpoint{-0.006944in}{-0.006944in}}%
\pgfpathcurveto{\pgfqpoint{-0.005103in}{-0.008786in}}{\pgfqpoint{-0.002605in}{-0.009821in}}{\pgfqpoint{0.000000in}{-0.009821in}}%
\pgfpathclose%
\pgfusepath{stroke,fill}%
}%
\begin{pgfscope}%
\pgfsys@transformshift{4.743198in}{3.672771in}%
\pgfsys@useobject{currentmarker}{}%
\end{pgfscope}%
\begin{pgfscope}%
\pgfsys@transformshift{4.888777in}{3.692738in}%
\pgfsys@useobject{currentmarker}{}%
\end{pgfscope}%
\begin{pgfscope}%
\pgfsys@transformshift{5.034356in}{3.692571in}%
\pgfsys@useobject{currentmarker}{}%
\end{pgfscope}%
\begin{pgfscope}%
\pgfsys@transformshift{5.179936in}{3.698854in}%
\pgfsys@useobject{currentmarker}{}%
\end{pgfscope}%
\begin{pgfscope}%
\pgfsys@transformshift{5.325515in}{3.765937in}%
\pgfsys@useobject{currentmarker}{}%
\end{pgfscope}%
\begin{pgfscope}%
\pgfsys@transformshift{5.471095in}{3.735452in}%
\pgfsys@useobject{currentmarker}{}%
\end{pgfscope}%
\begin{pgfscope}%
\pgfsys@transformshift{5.616674in}{3.698868in}%
\pgfsys@useobject{currentmarker}{}%
\end{pgfscope}%
\begin{pgfscope}%
\pgfsys@transformshift{5.762253in}{3.735253in}%
\pgfsys@useobject{currentmarker}{}%
\end{pgfscope}%
\begin{pgfscope}%
\pgfsys@transformshift{5.907833in}{3.727262in}%
\pgfsys@useobject{currentmarker}{}%
\end{pgfscope}%
\begin{pgfscope}%
\pgfsys@transformshift{6.053412in}{3.733837in}%
\pgfsys@useobject{currentmarker}{}%
\end{pgfscope}%
\begin{pgfscope}%
\pgfsys@transformshift{6.198992in}{3.729804in}%
\pgfsys@useobject{currentmarker}{}%
\end{pgfscope}%
\begin{pgfscope}%
\pgfsys@transformshift{6.344571in}{3.730395in}%
\pgfsys@useobject{currentmarker}{}%
\end{pgfscope}%
\begin{pgfscope}%
\pgfsys@transformshift{6.490150in}{3.725007in}%
\pgfsys@useobject{currentmarker}{}%
\end{pgfscope}%
\begin{pgfscope}%
\pgfsys@transformshift{6.635730in}{3.730992in}%
\pgfsys@useobject{currentmarker}{}%
\end{pgfscope}%
\begin{pgfscope}%
\pgfsys@transformshift{6.781309in}{3.727099in}%
\pgfsys@useobject{currentmarker}{}%
\end{pgfscope}%
\begin{pgfscope}%
\pgfsys@transformshift{6.926889in}{3.729625in}%
\pgfsys@useobject{currentmarker}{}%
\end{pgfscope}%
\begin{pgfscope}%
\pgfsys@transformshift{7.072468in}{3.728683in}%
\pgfsys@useobject{currentmarker}{}%
\end{pgfscope}%
\begin{pgfscope}%
\pgfsys@transformshift{7.218047in}{3.728440in}%
\pgfsys@useobject{currentmarker}{}%
\end{pgfscope}%
\begin{pgfscope}%
\pgfsys@transformshift{7.363627in}{3.727713in}%
\pgfsys@useobject{currentmarker}{}%
\end{pgfscope}%
\begin{pgfscope}%
\pgfsys@transformshift{7.509206in}{3.727808in}%
\pgfsys@useobject{currentmarker}{}%
\end{pgfscope}%
\begin{pgfscope}%
\pgfsys@transformshift{7.654786in}{3.727714in}%
\pgfsys@useobject{currentmarker}{}%
\end{pgfscope}%
\end{pgfscope}%
\begin{pgfscope}%
\pgfpathrectangle{\pgfqpoint{4.597569in}{3.363889in}}{\pgfqpoint{3.203819in}{2.263889in}}%
\pgfusepath{clip}%
\pgfsetbuttcap%
\pgfsetroundjoin%
\definecolor{currentfill}{rgb}{0.580392,0.403922,0.741176}%
\pgfsetfillcolor{currentfill}%
\pgfsetfillopacity{0.300000}%
\pgfsetlinewidth{1.003750pt}%
\definecolor{currentstroke}{rgb}{0.580392,0.403922,0.741176}%
\pgfsetstrokecolor{currentstroke}%
\pgfsetstrokeopacity{0.300000}%
\pgfsetdash{}{0pt}%
\pgfpathmoveto{\pgfqpoint{4.743198in}{3.713626in}}%
\pgfpathlineto{\pgfqpoint{4.743198in}{3.631916in}}%
\pgfpathlineto{\pgfqpoint{4.888777in}{3.640697in}}%
\pgfpathlineto{\pgfqpoint{5.034356in}{3.644539in}}%
\pgfpathlineto{\pgfqpoint{5.179936in}{3.649956in}}%
\pgfpathlineto{\pgfqpoint{5.325515in}{3.694731in}}%
\pgfpathlineto{\pgfqpoint{5.471095in}{3.645044in}}%
\pgfpathlineto{\pgfqpoint{5.616674in}{3.671728in}}%
\pgfpathlineto{\pgfqpoint{5.762253in}{3.693401in}}%
\pgfpathlineto{\pgfqpoint{5.907833in}{3.691721in}}%
\pgfpathlineto{\pgfqpoint{6.053412in}{3.718406in}}%
\pgfpathlineto{\pgfqpoint{6.198992in}{3.722876in}}%
\pgfpathlineto{\pgfqpoint{6.344571in}{3.720005in}}%
\pgfpathlineto{\pgfqpoint{6.490150in}{3.715918in}}%
\pgfpathlineto{\pgfqpoint{6.635730in}{3.725503in}}%
\pgfpathlineto{\pgfqpoint{6.781309in}{3.722897in}}%
\pgfpathlineto{\pgfqpoint{6.926889in}{3.728483in}}%
\pgfpathlineto{\pgfqpoint{7.072468in}{3.728089in}}%
\pgfpathlineto{\pgfqpoint{7.218047in}{3.726601in}}%
\pgfpathlineto{\pgfqpoint{7.363627in}{3.727076in}}%
\pgfpathlineto{\pgfqpoint{7.509206in}{3.726691in}}%
\pgfpathlineto{\pgfqpoint{7.654786in}{3.727065in}}%
\pgfpathlineto{\pgfqpoint{7.654786in}{3.728363in}}%
\pgfpathlineto{\pgfqpoint{7.654786in}{3.728363in}}%
\pgfpathlineto{\pgfqpoint{7.509206in}{3.728926in}}%
\pgfpathlineto{\pgfqpoint{7.363627in}{3.728350in}}%
\pgfpathlineto{\pgfqpoint{7.218047in}{3.730278in}}%
\pgfpathlineto{\pgfqpoint{7.072468in}{3.729276in}}%
\pgfpathlineto{\pgfqpoint{6.926889in}{3.730767in}}%
\pgfpathlineto{\pgfqpoint{6.781309in}{3.731301in}}%
\pgfpathlineto{\pgfqpoint{6.635730in}{3.736480in}}%
\pgfpathlineto{\pgfqpoint{6.490150in}{3.734096in}}%
\pgfpathlineto{\pgfqpoint{6.344571in}{3.740784in}}%
\pgfpathlineto{\pgfqpoint{6.198992in}{3.736732in}}%
\pgfpathlineto{\pgfqpoint{6.053412in}{3.749269in}}%
\pgfpathlineto{\pgfqpoint{5.907833in}{3.762803in}}%
\pgfpathlineto{\pgfqpoint{5.762253in}{3.777105in}}%
\pgfpathlineto{\pgfqpoint{5.616674in}{3.726008in}}%
\pgfpathlineto{\pgfqpoint{5.471095in}{3.825861in}}%
\pgfpathlineto{\pgfqpoint{5.325515in}{3.837144in}}%
\pgfpathlineto{\pgfqpoint{5.179936in}{3.747752in}}%
\pgfpathlineto{\pgfqpoint{5.034356in}{3.740602in}}%
\pgfpathlineto{\pgfqpoint{4.888777in}{3.744779in}}%
\pgfpathlineto{\pgfqpoint{4.743198in}{3.713626in}}%
\pgfpathclose%
\pgfusepath{stroke,fill}%
\end{pgfscope}%
\begin{pgfscope}%
\pgfsetbuttcap%
\pgfsetroundjoin%
\definecolor{currentfill}{rgb}{0.000000,0.000000,0.000000}%
\pgfsetfillcolor{currentfill}%
\pgfsetlinewidth{0.803000pt}%
\definecolor{currentstroke}{rgb}{0.000000,0.000000,0.000000}%
\pgfsetstrokecolor{currentstroke}%
\pgfsetdash{}{0pt}%
\pgfsys@defobject{currentmarker}{\pgfqpoint{0.000000in}{-0.048611in}}{\pgfqpoint{0.000000in}{0.000000in}}{%
\pgfpathmoveto{\pgfqpoint{0.000000in}{0.000000in}}%
\pgfpathlineto{\pgfqpoint{0.000000in}{-0.048611in}}%
\pgfusepath{stroke,fill}%
}%
\begin{pgfscope}%
\pgfsys@transformshift{5.093788in}{3.363889in}%
\pgfsys@useobject{currentmarker}{}%
\end{pgfscope}%
\end{pgfscope}%
\begin{pgfscope}%
\definecolor{textcolor}{rgb}{0.000000,0.000000,0.000000}%
\pgfsetstrokecolor{textcolor}%
\pgfsetfillcolor{textcolor}%
\pgftext[x=5.093788in,y=3.266667in,,top]{\color{textcolor}\sffamily\fontsize{10.000000}{12.000000}\selectfont \(\displaystyle {10^{4}}\)}%
\end{pgfscope}%
\begin{pgfscope}%
\pgfsetbuttcap%
\pgfsetroundjoin%
\definecolor{currentfill}{rgb}{0.000000,0.000000,0.000000}%
\pgfsetfillcolor{currentfill}%
\pgfsetlinewidth{0.803000pt}%
\definecolor{currentstroke}{rgb}{0.000000,0.000000,0.000000}%
\pgfsetstrokecolor{currentstroke}%
\pgfsetdash{}{0pt}%
\pgfsys@defobject{currentmarker}{\pgfqpoint{0.000000in}{-0.048611in}}{\pgfqpoint{0.000000in}{0.000000in}}{%
\pgfpathmoveto{\pgfqpoint{0.000000in}{0.000000in}}%
\pgfpathlineto{\pgfqpoint{0.000000in}{-0.048611in}}%
\pgfusepath{stroke,fill}%
}%
\begin{pgfscope}%
\pgfsys@transformshift{5.676105in}{3.363889in}%
\pgfsys@useobject{currentmarker}{}%
\end{pgfscope}%
\end{pgfscope}%
\begin{pgfscope}%
\definecolor{textcolor}{rgb}{0.000000,0.000000,0.000000}%
\pgfsetstrokecolor{textcolor}%
\pgfsetfillcolor{textcolor}%
\pgftext[x=5.676105in,y=3.266667in,,top]{\color{textcolor}\sffamily\fontsize{10.000000}{12.000000}\selectfont \(\displaystyle {10^{5}}\)}%
\end{pgfscope}%
\begin{pgfscope}%
\pgfsetbuttcap%
\pgfsetroundjoin%
\definecolor{currentfill}{rgb}{0.000000,0.000000,0.000000}%
\pgfsetfillcolor{currentfill}%
\pgfsetlinewidth{0.803000pt}%
\definecolor{currentstroke}{rgb}{0.000000,0.000000,0.000000}%
\pgfsetstrokecolor{currentstroke}%
\pgfsetdash{}{0pt}%
\pgfsys@defobject{currentmarker}{\pgfqpoint{0.000000in}{-0.048611in}}{\pgfqpoint{0.000000in}{0.000000in}}{%
\pgfpathmoveto{\pgfqpoint{0.000000in}{0.000000in}}%
\pgfpathlineto{\pgfqpoint{0.000000in}{-0.048611in}}%
\pgfusepath{stroke,fill}%
}%
\begin{pgfscope}%
\pgfsys@transformshift{6.258423in}{3.363889in}%
\pgfsys@useobject{currentmarker}{}%
\end{pgfscope}%
\end{pgfscope}%
\begin{pgfscope}%
\definecolor{textcolor}{rgb}{0.000000,0.000000,0.000000}%
\pgfsetstrokecolor{textcolor}%
\pgfsetfillcolor{textcolor}%
\pgftext[x=6.258423in,y=3.266667in,,top]{\color{textcolor}\sffamily\fontsize{10.000000}{12.000000}\selectfont \(\displaystyle {10^{6}}\)}%
\end{pgfscope}%
\begin{pgfscope}%
\pgfsetbuttcap%
\pgfsetroundjoin%
\definecolor{currentfill}{rgb}{0.000000,0.000000,0.000000}%
\pgfsetfillcolor{currentfill}%
\pgfsetlinewidth{0.803000pt}%
\definecolor{currentstroke}{rgb}{0.000000,0.000000,0.000000}%
\pgfsetstrokecolor{currentstroke}%
\pgfsetdash{}{0pt}%
\pgfsys@defobject{currentmarker}{\pgfqpoint{0.000000in}{-0.048611in}}{\pgfqpoint{0.000000in}{0.000000in}}{%
\pgfpathmoveto{\pgfqpoint{0.000000in}{0.000000in}}%
\pgfpathlineto{\pgfqpoint{0.000000in}{-0.048611in}}%
\pgfusepath{stroke,fill}%
}%
\begin{pgfscope}%
\pgfsys@transformshift{6.840741in}{3.363889in}%
\pgfsys@useobject{currentmarker}{}%
\end{pgfscope}%
\end{pgfscope}%
\begin{pgfscope}%
\definecolor{textcolor}{rgb}{0.000000,0.000000,0.000000}%
\pgfsetstrokecolor{textcolor}%
\pgfsetfillcolor{textcolor}%
\pgftext[x=6.840741in,y=3.266667in,,top]{\color{textcolor}\sffamily\fontsize{10.000000}{12.000000}\selectfont \(\displaystyle {10^{7}}\)}%
\end{pgfscope}%
\begin{pgfscope}%
\pgfsetbuttcap%
\pgfsetroundjoin%
\definecolor{currentfill}{rgb}{0.000000,0.000000,0.000000}%
\pgfsetfillcolor{currentfill}%
\pgfsetlinewidth{0.803000pt}%
\definecolor{currentstroke}{rgb}{0.000000,0.000000,0.000000}%
\pgfsetstrokecolor{currentstroke}%
\pgfsetdash{}{0pt}%
\pgfsys@defobject{currentmarker}{\pgfqpoint{0.000000in}{-0.048611in}}{\pgfqpoint{0.000000in}{0.000000in}}{%
\pgfpathmoveto{\pgfqpoint{0.000000in}{0.000000in}}%
\pgfpathlineto{\pgfqpoint{0.000000in}{-0.048611in}}%
\pgfusepath{stroke,fill}%
}%
\begin{pgfscope}%
\pgfsys@transformshift{7.423058in}{3.363889in}%
\pgfsys@useobject{currentmarker}{}%
\end{pgfscope}%
\end{pgfscope}%
\begin{pgfscope}%
\definecolor{textcolor}{rgb}{0.000000,0.000000,0.000000}%
\pgfsetstrokecolor{textcolor}%
\pgfsetfillcolor{textcolor}%
\pgftext[x=7.423058in,y=3.266667in,,top]{\color{textcolor}\sffamily\fontsize{10.000000}{12.000000}\selectfont \(\displaystyle {10^{8}}\)}%
\end{pgfscope}%
\begin{pgfscope}%
\pgfsetbuttcap%
\pgfsetroundjoin%
\definecolor{currentfill}{rgb}{0.000000,0.000000,0.000000}%
\pgfsetfillcolor{currentfill}%
\pgfsetlinewidth{0.602250pt}%
\definecolor{currentstroke}{rgb}{0.000000,0.000000,0.000000}%
\pgfsetstrokecolor{currentstroke}%
\pgfsetdash{}{0pt}%
\pgfsys@defobject{currentmarker}{\pgfqpoint{0.000000in}{-0.027778in}}{\pgfqpoint{0.000000in}{0.000000in}}{%
\pgfpathmoveto{\pgfqpoint{0.000000in}{0.000000in}}%
\pgfpathlineto{\pgfqpoint{0.000000in}{-0.027778in}}%
\pgfusepath{stroke,fill}%
}%
\begin{pgfscope}%
\pgfsys@transformshift{4.686765in}{3.363889in}%
\pgfsys@useobject{currentmarker}{}%
\end{pgfscope}%
\end{pgfscope}%
\begin{pgfscope}%
\pgfsetbuttcap%
\pgfsetroundjoin%
\definecolor{currentfill}{rgb}{0.000000,0.000000,0.000000}%
\pgfsetfillcolor{currentfill}%
\pgfsetlinewidth{0.602250pt}%
\definecolor{currentstroke}{rgb}{0.000000,0.000000,0.000000}%
\pgfsetstrokecolor{currentstroke}%
\pgfsetdash{}{0pt}%
\pgfsys@defobject{currentmarker}{\pgfqpoint{0.000000in}{-0.027778in}}{\pgfqpoint{0.000000in}{0.000000in}}{%
\pgfpathmoveto{\pgfqpoint{0.000000in}{0.000000in}}%
\pgfpathlineto{\pgfqpoint{0.000000in}{-0.027778in}}%
\pgfusepath{stroke,fill}%
}%
\begin{pgfscope}%
\pgfsys@transformshift{4.789306in}{3.363889in}%
\pgfsys@useobject{currentmarker}{}%
\end{pgfscope}%
\end{pgfscope}%
\begin{pgfscope}%
\pgfsetbuttcap%
\pgfsetroundjoin%
\definecolor{currentfill}{rgb}{0.000000,0.000000,0.000000}%
\pgfsetfillcolor{currentfill}%
\pgfsetlinewidth{0.602250pt}%
\definecolor{currentstroke}{rgb}{0.000000,0.000000,0.000000}%
\pgfsetstrokecolor{currentstroke}%
\pgfsetdash{}{0pt}%
\pgfsys@defobject{currentmarker}{\pgfqpoint{0.000000in}{-0.027778in}}{\pgfqpoint{0.000000in}{0.000000in}}{%
\pgfpathmoveto{\pgfqpoint{0.000000in}{0.000000in}}%
\pgfpathlineto{\pgfqpoint{0.000000in}{-0.027778in}}%
\pgfusepath{stroke,fill}%
}%
\begin{pgfscope}%
\pgfsys@transformshift{4.862060in}{3.363889in}%
\pgfsys@useobject{currentmarker}{}%
\end{pgfscope}%
\end{pgfscope}%
\begin{pgfscope}%
\pgfsetbuttcap%
\pgfsetroundjoin%
\definecolor{currentfill}{rgb}{0.000000,0.000000,0.000000}%
\pgfsetfillcolor{currentfill}%
\pgfsetlinewidth{0.602250pt}%
\definecolor{currentstroke}{rgb}{0.000000,0.000000,0.000000}%
\pgfsetstrokecolor{currentstroke}%
\pgfsetdash{}{0pt}%
\pgfsys@defobject{currentmarker}{\pgfqpoint{0.000000in}{-0.027778in}}{\pgfqpoint{0.000000in}{0.000000in}}{%
\pgfpathmoveto{\pgfqpoint{0.000000in}{0.000000in}}%
\pgfpathlineto{\pgfqpoint{0.000000in}{-0.027778in}}%
\pgfusepath{stroke,fill}%
}%
\begin{pgfscope}%
\pgfsys@transformshift{4.918493in}{3.363889in}%
\pgfsys@useobject{currentmarker}{}%
\end{pgfscope}%
\end{pgfscope}%
\begin{pgfscope}%
\pgfsetbuttcap%
\pgfsetroundjoin%
\definecolor{currentfill}{rgb}{0.000000,0.000000,0.000000}%
\pgfsetfillcolor{currentfill}%
\pgfsetlinewidth{0.602250pt}%
\definecolor{currentstroke}{rgb}{0.000000,0.000000,0.000000}%
\pgfsetstrokecolor{currentstroke}%
\pgfsetdash{}{0pt}%
\pgfsys@defobject{currentmarker}{\pgfqpoint{0.000000in}{-0.027778in}}{\pgfqpoint{0.000000in}{0.000000in}}{%
\pgfpathmoveto{\pgfqpoint{0.000000in}{0.000000in}}%
\pgfpathlineto{\pgfqpoint{0.000000in}{-0.027778in}}%
\pgfusepath{stroke,fill}%
}%
\begin{pgfscope}%
\pgfsys@transformshift{4.964601in}{3.363889in}%
\pgfsys@useobject{currentmarker}{}%
\end{pgfscope}%
\end{pgfscope}%
\begin{pgfscope}%
\pgfsetbuttcap%
\pgfsetroundjoin%
\definecolor{currentfill}{rgb}{0.000000,0.000000,0.000000}%
\pgfsetfillcolor{currentfill}%
\pgfsetlinewidth{0.602250pt}%
\definecolor{currentstroke}{rgb}{0.000000,0.000000,0.000000}%
\pgfsetstrokecolor{currentstroke}%
\pgfsetdash{}{0pt}%
\pgfsys@defobject{currentmarker}{\pgfqpoint{0.000000in}{-0.027778in}}{\pgfqpoint{0.000000in}{0.000000in}}{%
\pgfpathmoveto{\pgfqpoint{0.000000in}{0.000000in}}%
\pgfpathlineto{\pgfqpoint{0.000000in}{-0.027778in}}%
\pgfusepath{stroke,fill}%
}%
\begin{pgfscope}%
\pgfsys@transformshift{5.003586in}{3.363889in}%
\pgfsys@useobject{currentmarker}{}%
\end{pgfscope}%
\end{pgfscope}%
\begin{pgfscope}%
\pgfsetbuttcap%
\pgfsetroundjoin%
\definecolor{currentfill}{rgb}{0.000000,0.000000,0.000000}%
\pgfsetfillcolor{currentfill}%
\pgfsetlinewidth{0.602250pt}%
\definecolor{currentstroke}{rgb}{0.000000,0.000000,0.000000}%
\pgfsetstrokecolor{currentstroke}%
\pgfsetdash{}{0pt}%
\pgfsys@defobject{currentmarker}{\pgfqpoint{0.000000in}{-0.027778in}}{\pgfqpoint{0.000000in}{0.000000in}}{%
\pgfpathmoveto{\pgfqpoint{0.000000in}{0.000000in}}%
\pgfpathlineto{\pgfqpoint{0.000000in}{-0.027778in}}%
\pgfusepath{stroke,fill}%
}%
\begin{pgfscope}%
\pgfsys@transformshift{5.037355in}{3.363889in}%
\pgfsys@useobject{currentmarker}{}%
\end{pgfscope}%
\end{pgfscope}%
\begin{pgfscope}%
\pgfsetbuttcap%
\pgfsetroundjoin%
\definecolor{currentfill}{rgb}{0.000000,0.000000,0.000000}%
\pgfsetfillcolor{currentfill}%
\pgfsetlinewidth{0.602250pt}%
\definecolor{currentstroke}{rgb}{0.000000,0.000000,0.000000}%
\pgfsetstrokecolor{currentstroke}%
\pgfsetdash{}{0pt}%
\pgfsys@defobject{currentmarker}{\pgfqpoint{0.000000in}{-0.027778in}}{\pgfqpoint{0.000000in}{0.000000in}}{%
\pgfpathmoveto{\pgfqpoint{0.000000in}{0.000000in}}%
\pgfpathlineto{\pgfqpoint{0.000000in}{-0.027778in}}%
\pgfusepath{stroke,fill}%
}%
\begin{pgfscope}%
\pgfsys@transformshift{5.067142in}{3.363889in}%
\pgfsys@useobject{currentmarker}{}%
\end{pgfscope}%
\end{pgfscope}%
\begin{pgfscope}%
\pgfsetbuttcap%
\pgfsetroundjoin%
\definecolor{currentfill}{rgb}{0.000000,0.000000,0.000000}%
\pgfsetfillcolor{currentfill}%
\pgfsetlinewidth{0.602250pt}%
\definecolor{currentstroke}{rgb}{0.000000,0.000000,0.000000}%
\pgfsetstrokecolor{currentstroke}%
\pgfsetdash{}{0pt}%
\pgfsys@defobject{currentmarker}{\pgfqpoint{0.000000in}{-0.027778in}}{\pgfqpoint{0.000000in}{0.000000in}}{%
\pgfpathmoveto{\pgfqpoint{0.000000in}{0.000000in}}%
\pgfpathlineto{\pgfqpoint{0.000000in}{-0.027778in}}%
\pgfusepath{stroke,fill}%
}%
\begin{pgfscope}%
\pgfsys@transformshift{5.269083in}{3.363889in}%
\pgfsys@useobject{currentmarker}{}%
\end{pgfscope}%
\end{pgfscope}%
\begin{pgfscope}%
\pgfsetbuttcap%
\pgfsetroundjoin%
\definecolor{currentfill}{rgb}{0.000000,0.000000,0.000000}%
\pgfsetfillcolor{currentfill}%
\pgfsetlinewidth{0.602250pt}%
\definecolor{currentstroke}{rgb}{0.000000,0.000000,0.000000}%
\pgfsetstrokecolor{currentstroke}%
\pgfsetdash{}{0pt}%
\pgfsys@defobject{currentmarker}{\pgfqpoint{0.000000in}{-0.027778in}}{\pgfqpoint{0.000000in}{0.000000in}}{%
\pgfpathmoveto{\pgfqpoint{0.000000in}{0.000000in}}%
\pgfpathlineto{\pgfqpoint{0.000000in}{-0.027778in}}%
\pgfusepath{stroke,fill}%
}%
\begin{pgfscope}%
\pgfsys@transformshift{5.371624in}{3.363889in}%
\pgfsys@useobject{currentmarker}{}%
\end{pgfscope}%
\end{pgfscope}%
\begin{pgfscope}%
\pgfsetbuttcap%
\pgfsetroundjoin%
\definecolor{currentfill}{rgb}{0.000000,0.000000,0.000000}%
\pgfsetfillcolor{currentfill}%
\pgfsetlinewidth{0.602250pt}%
\definecolor{currentstroke}{rgb}{0.000000,0.000000,0.000000}%
\pgfsetstrokecolor{currentstroke}%
\pgfsetdash{}{0pt}%
\pgfsys@defobject{currentmarker}{\pgfqpoint{0.000000in}{-0.027778in}}{\pgfqpoint{0.000000in}{0.000000in}}{%
\pgfpathmoveto{\pgfqpoint{0.000000in}{0.000000in}}%
\pgfpathlineto{\pgfqpoint{0.000000in}{-0.027778in}}%
\pgfusepath{stroke,fill}%
}%
\begin{pgfscope}%
\pgfsys@transformshift{5.444378in}{3.363889in}%
\pgfsys@useobject{currentmarker}{}%
\end{pgfscope}%
\end{pgfscope}%
\begin{pgfscope}%
\pgfsetbuttcap%
\pgfsetroundjoin%
\definecolor{currentfill}{rgb}{0.000000,0.000000,0.000000}%
\pgfsetfillcolor{currentfill}%
\pgfsetlinewidth{0.602250pt}%
\definecolor{currentstroke}{rgb}{0.000000,0.000000,0.000000}%
\pgfsetstrokecolor{currentstroke}%
\pgfsetdash{}{0pt}%
\pgfsys@defobject{currentmarker}{\pgfqpoint{0.000000in}{-0.027778in}}{\pgfqpoint{0.000000in}{0.000000in}}{%
\pgfpathmoveto{\pgfqpoint{0.000000in}{0.000000in}}%
\pgfpathlineto{\pgfqpoint{0.000000in}{-0.027778in}}%
\pgfusepath{stroke,fill}%
}%
\begin{pgfscope}%
\pgfsys@transformshift{5.500810in}{3.363889in}%
\pgfsys@useobject{currentmarker}{}%
\end{pgfscope}%
\end{pgfscope}%
\begin{pgfscope}%
\pgfsetbuttcap%
\pgfsetroundjoin%
\definecolor{currentfill}{rgb}{0.000000,0.000000,0.000000}%
\pgfsetfillcolor{currentfill}%
\pgfsetlinewidth{0.602250pt}%
\definecolor{currentstroke}{rgb}{0.000000,0.000000,0.000000}%
\pgfsetstrokecolor{currentstroke}%
\pgfsetdash{}{0pt}%
\pgfsys@defobject{currentmarker}{\pgfqpoint{0.000000in}{-0.027778in}}{\pgfqpoint{0.000000in}{0.000000in}}{%
\pgfpathmoveto{\pgfqpoint{0.000000in}{0.000000in}}%
\pgfpathlineto{\pgfqpoint{0.000000in}{-0.027778in}}%
\pgfusepath{stroke,fill}%
}%
\begin{pgfscope}%
\pgfsys@transformshift{5.546919in}{3.363889in}%
\pgfsys@useobject{currentmarker}{}%
\end{pgfscope}%
\end{pgfscope}%
\begin{pgfscope}%
\pgfsetbuttcap%
\pgfsetroundjoin%
\definecolor{currentfill}{rgb}{0.000000,0.000000,0.000000}%
\pgfsetfillcolor{currentfill}%
\pgfsetlinewidth{0.602250pt}%
\definecolor{currentstroke}{rgb}{0.000000,0.000000,0.000000}%
\pgfsetstrokecolor{currentstroke}%
\pgfsetdash{}{0pt}%
\pgfsys@defobject{currentmarker}{\pgfqpoint{0.000000in}{-0.027778in}}{\pgfqpoint{0.000000in}{0.000000in}}{%
\pgfpathmoveto{\pgfqpoint{0.000000in}{0.000000in}}%
\pgfpathlineto{\pgfqpoint{0.000000in}{-0.027778in}}%
\pgfusepath{stroke,fill}%
}%
\begin{pgfscope}%
\pgfsys@transformshift{5.585903in}{3.363889in}%
\pgfsys@useobject{currentmarker}{}%
\end{pgfscope}%
\end{pgfscope}%
\begin{pgfscope}%
\pgfsetbuttcap%
\pgfsetroundjoin%
\definecolor{currentfill}{rgb}{0.000000,0.000000,0.000000}%
\pgfsetfillcolor{currentfill}%
\pgfsetlinewidth{0.602250pt}%
\definecolor{currentstroke}{rgb}{0.000000,0.000000,0.000000}%
\pgfsetstrokecolor{currentstroke}%
\pgfsetdash{}{0pt}%
\pgfsys@defobject{currentmarker}{\pgfqpoint{0.000000in}{-0.027778in}}{\pgfqpoint{0.000000in}{0.000000in}}{%
\pgfpathmoveto{\pgfqpoint{0.000000in}{0.000000in}}%
\pgfpathlineto{\pgfqpoint{0.000000in}{-0.027778in}}%
\pgfusepath{stroke,fill}%
}%
\begin{pgfscope}%
\pgfsys@transformshift{5.619673in}{3.363889in}%
\pgfsys@useobject{currentmarker}{}%
\end{pgfscope}%
\end{pgfscope}%
\begin{pgfscope}%
\pgfsetbuttcap%
\pgfsetroundjoin%
\definecolor{currentfill}{rgb}{0.000000,0.000000,0.000000}%
\pgfsetfillcolor{currentfill}%
\pgfsetlinewidth{0.602250pt}%
\definecolor{currentstroke}{rgb}{0.000000,0.000000,0.000000}%
\pgfsetstrokecolor{currentstroke}%
\pgfsetdash{}{0pt}%
\pgfsys@defobject{currentmarker}{\pgfqpoint{0.000000in}{-0.027778in}}{\pgfqpoint{0.000000in}{0.000000in}}{%
\pgfpathmoveto{\pgfqpoint{0.000000in}{0.000000in}}%
\pgfpathlineto{\pgfqpoint{0.000000in}{-0.027778in}}%
\pgfusepath{stroke,fill}%
}%
\begin{pgfscope}%
\pgfsys@transformshift{5.649460in}{3.363889in}%
\pgfsys@useobject{currentmarker}{}%
\end{pgfscope}%
\end{pgfscope}%
\begin{pgfscope}%
\pgfsetbuttcap%
\pgfsetroundjoin%
\definecolor{currentfill}{rgb}{0.000000,0.000000,0.000000}%
\pgfsetfillcolor{currentfill}%
\pgfsetlinewidth{0.602250pt}%
\definecolor{currentstroke}{rgb}{0.000000,0.000000,0.000000}%
\pgfsetstrokecolor{currentstroke}%
\pgfsetdash{}{0pt}%
\pgfsys@defobject{currentmarker}{\pgfqpoint{0.000000in}{-0.027778in}}{\pgfqpoint{0.000000in}{0.000000in}}{%
\pgfpathmoveto{\pgfqpoint{0.000000in}{0.000000in}}%
\pgfpathlineto{\pgfqpoint{0.000000in}{-0.027778in}}%
\pgfusepath{stroke,fill}%
}%
\begin{pgfscope}%
\pgfsys@transformshift{5.851400in}{3.363889in}%
\pgfsys@useobject{currentmarker}{}%
\end{pgfscope}%
\end{pgfscope}%
\begin{pgfscope}%
\pgfsetbuttcap%
\pgfsetroundjoin%
\definecolor{currentfill}{rgb}{0.000000,0.000000,0.000000}%
\pgfsetfillcolor{currentfill}%
\pgfsetlinewidth{0.602250pt}%
\definecolor{currentstroke}{rgb}{0.000000,0.000000,0.000000}%
\pgfsetstrokecolor{currentstroke}%
\pgfsetdash{}{0pt}%
\pgfsys@defobject{currentmarker}{\pgfqpoint{0.000000in}{-0.027778in}}{\pgfqpoint{0.000000in}{0.000000in}}{%
\pgfpathmoveto{\pgfqpoint{0.000000in}{0.000000in}}%
\pgfpathlineto{\pgfqpoint{0.000000in}{-0.027778in}}%
\pgfusepath{stroke,fill}%
}%
\begin{pgfscope}%
\pgfsys@transformshift{5.953941in}{3.363889in}%
\pgfsys@useobject{currentmarker}{}%
\end{pgfscope}%
\end{pgfscope}%
\begin{pgfscope}%
\pgfsetbuttcap%
\pgfsetroundjoin%
\definecolor{currentfill}{rgb}{0.000000,0.000000,0.000000}%
\pgfsetfillcolor{currentfill}%
\pgfsetlinewidth{0.602250pt}%
\definecolor{currentstroke}{rgb}{0.000000,0.000000,0.000000}%
\pgfsetstrokecolor{currentstroke}%
\pgfsetdash{}{0pt}%
\pgfsys@defobject{currentmarker}{\pgfqpoint{0.000000in}{-0.027778in}}{\pgfqpoint{0.000000in}{0.000000in}}{%
\pgfpathmoveto{\pgfqpoint{0.000000in}{0.000000in}}%
\pgfpathlineto{\pgfqpoint{0.000000in}{-0.027778in}}%
\pgfusepath{stroke,fill}%
}%
\begin{pgfscope}%
\pgfsys@transformshift{6.026696in}{3.363889in}%
\pgfsys@useobject{currentmarker}{}%
\end{pgfscope}%
\end{pgfscope}%
\begin{pgfscope}%
\pgfsetbuttcap%
\pgfsetroundjoin%
\definecolor{currentfill}{rgb}{0.000000,0.000000,0.000000}%
\pgfsetfillcolor{currentfill}%
\pgfsetlinewidth{0.602250pt}%
\definecolor{currentstroke}{rgb}{0.000000,0.000000,0.000000}%
\pgfsetstrokecolor{currentstroke}%
\pgfsetdash{}{0pt}%
\pgfsys@defobject{currentmarker}{\pgfqpoint{0.000000in}{-0.027778in}}{\pgfqpoint{0.000000in}{0.000000in}}{%
\pgfpathmoveto{\pgfqpoint{0.000000in}{0.000000in}}%
\pgfpathlineto{\pgfqpoint{0.000000in}{-0.027778in}}%
\pgfusepath{stroke,fill}%
}%
\begin{pgfscope}%
\pgfsys@transformshift{6.083128in}{3.363889in}%
\pgfsys@useobject{currentmarker}{}%
\end{pgfscope}%
\end{pgfscope}%
\begin{pgfscope}%
\pgfsetbuttcap%
\pgfsetroundjoin%
\definecolor{currentfill}{rgb}{0.000000,0.000000,0.000000}%
\pgfsetfillcolor{currentfill}%
\pgfsetlinewidth{0.602250pt}%
\definecolor{currentstroke}{rgb}{0.000000,0.000000,0.000000}%
\pgfsetstrokecolor{currentstroke}%
\pgfsetdash{}{0pt}%
\pgfsys@defobject{currentmarker}{\pgfqpoint{0.000000in}{-0.027778in}}{\pgfqpoint{0.000000in}{0.000000in}}{%
\pgfpathmoveto{\pgfqpoint{0.000000in}{0.000000in}}%
\pgfpathlineto{\pgfqpoint{0.000000in}{-0.027778in}}%
\pgfusepath{stroke,fill}%
}%
\begin{pgfscope}%
\pgfsys@transformshift{6.129237in}{3.363889in}%
\pgfsys@useobject{currentmarker}{}%
\end{pgfscope}%
\end{pgfscope}%
\begin{pgfscope}%
\pgfsetbuttcap%
\pgfsetroundjoin%
\definecolor{currentfill}{rgb}{0.000000,0.000000,0.000000}%
\pgfsetfillcolor{currentfill}%
\pgfsetlinewidth{0.602250pt}%
\definecolor{currentstroke}{rgb}{0.000000,0.000000,0.000000}%
\pgfsetstrokecolor{currentstroke}%
\pgfsetdash{}{0pt}%
\pgfsys@defobject{currentmarker}{\pgfqpoint{0.000000in}{-0.027778in}}{\pgfqpoint{0.000000in}{0.000000in}}{%
\pgfpathmoveto{\pgfqpoint{0.000000in}{0.000000in}}%
\pgfpathlineto{\pgfqpoint{0.000000in}{-0.027778in}}%
\pgfusepath{stroke,fill}%
}%
\begin{pgfscope}%
\pgfsys@transformshift{6.168221in}{3.363889in}%
\pgfsys@useobject{currentmarker}{}%
\end{pgfscope}%
\end{pgfscope}%
\begin{pgfscope}%
\pgfsetbuttcap%
\pgfsetroundjoin%
\definecolor{currentfill}{rgb}{0.000000,0.000000,0.000000}%
\pgfsetfillcolor{currentfill}%
\pgfsetlinewidth{0.602250pt}%
\definecolor{currentstroke}{rgb}{0.000000,0.000000,0.000000}%
\pgfsetstrokecolor{currentstroke}%
\pgfsetdash{}{0pt}%
\pgfsys@defobject{currentmarker}{\pgfqpoint{0.000000in}{-0.027778in}}{\pgfqpoint{0.000000in}{0.000000in}}{%
\pgfpathmoveto{\pgfqpoint{0.000000in}{0.000000in}}%
\pgfpathlineto{\pgfqpoint{0.000000in}{-0.027778in}}%
\pgfusepath{stroke,fill}%
}%
\begin{pgfscope}%
\pgfsys@transformshift{6.201991in}{3.363889in}%
\pgfsys@useobject{currentmarker}{}%
\end{pgfscope}%
\end{pgfscope}%
\begin{pgfscope}%
\pgfsetbuttcap%
\pgfsetroundjoin%
\definecolor{currentfill}{rgb}{0.000000,0.000000,0.000000}%
\pgfsetfillcolor{currentfill}%
\pgfsetlinewidth{0.602250pt}%
\definecolor{currentstroke}{rgb}{0.000000,0.000000,0.000000}%
\pgfsetstrokecolor{currentstroke}%
\pgfsetdash{}{0pt}%
\pgfsys@defobject{currentmarker}{\pgfqpoint{0.000000in}{-0.027778in}}{\pgfqpoint{0.000000in}{0.000000in}}{%
\pgfpathmoveto{\pgfqpoint{0.000000in}{0.000000in}}%
\pgfpathlineto{\pgfqpoint{0.000000in}{-0.027778in}}%
\pgfusepath{stroke,fill}%
}%
\begin{pgfscope}%
\pgfsys@transformshift{6.231778in}{3.363889in}%
\pgfsys@useobject{currentmarker}{}%
\end{pgfscope}%
\end{pgfscope}%
\begin{pgfscope}%
\pgfsetbuttcap%
\pgfsetroundjoin%
\definecolor{currentfill}{rgb}{0.000000,0.000000,0.000000}%
\pgfsetfillcolor{currentfill}%
\pgfsetlinewidth{0.602250pt}%
\definecolor{currentstroke}{rgb}{0.000000,0.000000,0.000000}%
\pgfsetstrokecolor{currentstroke}%
\pgfsetdash{}{0pt}%
\pgfsys@defobject{currentmarker}{\pgfqpoint{0.000000in}{-0.027778in}}{\pgfqpoint{0.000000in}{0.000000in}}{%
\pgfpathmoveto{\pgfqpoint{0.000000in}{0.000000in}}%
\pgfpathlineto{\pgfqpoint{0.000000in}{-0.027778in}}%
\pgfusepath{stroke,fill}%
}%
\begin{pgfscope}%
\pgfsys@transformshift{6.433718in}{3.363889in}%
\pgfsys@useobject{currentmarker}{}%
\end{pgfscope}%
\end{pgfscope}%
\begin{pgfscope}%
\pgfsetbuttcap%
\pgfsetroundjoin%
\definecolor{currentfill}{rgb}{0.000000,0.000000,0.000000}%
\pgfsetfillcolor{currentfill}%
\pgfsetlinewidth{0.602250pt}%
\definecolor{currentstroke}{rgb}{0.000000,0.000000,0.000000}%
\pgfsetstrokecolor{currentstroke}%
\pgfsetdash{}{0pt}%
\pgfsys@defobject{currentmarker}{\pgfqpoint{0.000000in}{-0.027778in}}{\pgfqpoint{0.000000in}{0.000000in}}{%
\pgfpathmoveto{\pgfqpoint{0.000000in}{0.000000in}}%
\pgfpathlineto{\pgfqpoint{0.000000in}{-0.027778in}}%
\pgfusepath{stroke,fill}%
}%
\begin{pgfscope}%
\pgfsys@transformshift{6.536259in}{3.363889in}%
\pgfsys@useobject{currentmarker}{}%
\end{pgfscope}%
\end{pgfscope}%
\begin{pgfscope}%
\pgfsetbuttcap%
\pgfsetroundjoin%
\definecolor{currentfill}{rgb}{0.000000,0.000000,0.000000}%
\pgfsetfillcolor{currentfill}%
\pgfsetlinewidth{0.602250pt}%
\definecolor{currentstroke}{rgb}{0.000000,0.000000,0.000000}%
\pgfsetstrokecolor{currentstroke}%
\pgfsetdash{}{0pt}%
\pgfsys@defobject{currentmarker}{\pgfqpoint{0.000000in}{-0.027778in}}{\pgfqpoint{0.000000in}{0.000000in}}{%
\pgfpathmoveto{\pgfqpoint{0.000000in}{0.000000in}}%
\pgfpathlineto{\pgfqpoint{0.000000in}{-0.027778in}}%
\pgfusepath{stroke,fill}%
}%
\begin{pgfscope}%
\pgfsys@transformshift{6.609013in}{3.363889in}%
\pgfsys@useobject{currentmarker}{}%
\end{pgfscope}%
\end{pgfscope}%
\begin{pgfscope}%
\pgfsetbuttcap%
\pgfsetroundjoin%
\definecolor{currentfill}{rgb}{0.000000,0.000000,0.000000}%
\pgfsetfillcolor{currentfill}%
\pgfsetlinewidth{0.602250pt}%
\definecolor{currentstroke}{rgb}{0.000000,0.000000,0.000000}%
\pgfsetstrokecolor{currentstroke}%
\pgfsetdash{}{0pt}%
\pgfsys@defobject{currentmarker}{\pgfqpoint{0.000000in}{-0.027778in}}{\pgfqpoint{0.000000in}{0.000000in}}{%
\pgfpathmoveto{\pgfqpoint{0.000000in}{0.000000in}}%
\pgfpathlineto{\pgfqpoint{0.000000in}{-0.027778in}}%
\pgfusepath{stroke,fill}%
}%
\begin{pgfscope}%
\pgfsys@transformshift{6.665446in}{3.363889in}%
\pgfsys@useobject{currentmarker}{}%
\end{pgfscope}%
\end{pgfscope}%
\begin{pgfscope}%
\pgfsetbuttcap%
\pgfsetroundjoin%
\definecolor{currentfill}{rgb}{0.000000,0.000000,0.000000}%
\pgfsetfillcolor{currentfill}%
\pgfsetlinewidth{0.602250pt}%
\definecolor{currentstroke}{rgb}{0.000000,0.000000,0.000000}%
\pgfsetstrokecolor{currentstroke}%
\pgfsetdash{}{0pt}%
\pgfsys@defobject{currentmarker}{\pgfqpoint{0.000000in}{-0.027778in}}{\pgfqpoint{0.000000in}{0.000000in}}{%
\pgfpathmoveto{\pgfqpoint{0.000000in}{0.000000in}}%
\pgfpathlineto{\pgfqpoint{0.000000in}{-0.027778in}}%
\pgfusepath{stroke,fill}%
}%
\begin{pgfscope}%
\pgfsys@transformshift{6.711554in}{3.363889in}%
\pgfsys@useobject{currentmarker}{}%
\end{pgfscope}%
\end{pgfscope}%
\begin{pgfscope}%
\pgfsetbuttcap%
\pgfsetroundjoin%
\definecolor{currentfill}{rgb}{0.000000,0.000000,0.000000}%
\pgfsetfillcolor{currentfill}%
\pgfsetlinewidth{0.602250pt}%
\definecolor{currentstroke}{rgb}{0.000000,0.000000,0.000000}%
\pgfsetstrokecolor{currentstroke}%
\pgfsetdash{}{0pt}%
\pgfsys@defobject{currentmarker}{\pgfqpoint{0.000000in}{-0.027778in}}{\pgfqpoint{0.000000in}{0.000000in}}{%
\pgfpathmoveto{\pgfqpoint{0.000000in}{0.000000in}}%
\pgfpathlineto{\pgfqpoint{0.000000in}{-0.027778in}}%
\pgfusepath{stroke,fill}%
}%
\begin{pgfscope}%
\pgfsys@transformshift{6.750538in}{3.363889in}%
\pgfsys@useobject{currentmarker}{}%
\end{pgfscope}%
\end{pgfscope}%
\begin{pgfscope}%
\pgfsetbuttcap%
\pgfsetroundjoin%
\definecolor{currentfill}{rgb}{0.000000,0.000000,0.000000}%
\pgfsetfillcolor{currentfill}%
\pgfsetlinewidth{0.602250pt}%
\definecolor{currentstroke}{rgb}{0.000000,0.000000,0.000000}%
\pgfsetstrokecolor{currentstroke}%
\pgfsetdash{}{0pt}%
\pgfsys@defobject{currentmarker}{\pgfqpoint{0.000000in}{-0.027778in}}{\pgfqpoint{0.000000in}{0.000000in}}{%
\pgfpathmoveto{\pgfqpoint{0.000000in}{0.000000in}}%
\pgfpathlineto{\pgfqpoint{0.000000in}{-0.027778in}}%
\pgfusepath{stroke,fill}%
}%
\begin{pgfscope}%
\pgfsys@transformshift{6.784308in}{3.363889in}%
\pgfsys@useobject{currentmarker}{}%
\end{pgfscope}%
\end{pgfscope}%
\begin{pgfscope}%
\pgfsetbuttcap%
\pgfsetroundjoin%
\definecolor{currentfill}{rgb}{0.000000,0.000000,0.000000}%
\pgfsetfillcolor{currentfill}%
\pgfsetlinewidth{0.602250pt}%
\definecolor{currentstroke}{rgb}{0.000000,0.000000,0.000000}%
\pgfsetstrokecolor{currentstroke}%
\pgfsetdash{}{0pt}%
\pgfsys@defobject{currentmarker}{\pgfqpoint{0.000000in}{-0.027778in}}{\pgfqpoint{0.000000in}{0.000000in}}{%
\pgfpathmoveto{\pgfqpoint{0.000000in}{0.000000in}}%
\pgfpathlineto{\pgfqpoint{0.000000in}{-0.027778in}}%
\pgfusepath{stroke,fill}%
}%
\begin{pgfscope}%
\pgfsys@transformshift{6.814095in}{3.363889in}%
\pgfsys@useobject{currentmarker}{}%
\end{pgfscope}%
\end{pgfscope}%
\begin{pgfscope}%
\pgfsetbuttcap%
\pgfsetroundjoin%
\definecolor{currentfill}{rgb}{0.000000,0.000000,0.000000}%
\pgfsetfillcolor{currentfill}%
\pgfsetlinewidth{0.602250pt}%
\definecolor{currentstroke}{rgb}{0.000000,0.000000,0.000000}%
\pgfsetstrokecolor{currentstroke}%
\pgfsetdash{}{0pt}%
\pgfsys@defobject{currentmarker}{\pgfqpoint{0.000000in}{-0.027778in}}{\pgfqpoint{0.000000in}{0.000000in}}{%
\pgfpathmoveto{\pgfqpoint{0.000000in}{0.000000in}}%
\pgfpathlineto{\pgfqpoint{0.000000in}{-0.027778in}}%
\pgfusepath{stroke,fill}%
}%
\begin{pgfscope}%
\pgfsys@transformshift{7.016036in}{3.363889in}%
\pgfsys@useobject{currentmarker}{}%
\end{pgfscope}%
\end{pgfscope}%
\begin{pgfscope}%
\pgfsetbuttcap%
\pgfsetroundjoin%
\definecolor{currentfill}{rgb}{0.000000,0.000000,0.000000}%
\pgfsetfillcolor{currentfill}%
\pgfsetlinewidth{0.602250pt}%
\definecolor{currentstroke}{rgb}{0.000000,0.000000,0.000000}%
\pgfsetstrokecolor{currentstroke}%
\pgfsetdash{}{0pt}%
\pgfsys@defobject{currentmarker}{\pgfqpoint{0.000000in}{-0.027778in}}{\pgfqpoint{0.000000in}{0.000000in}}{%
\pgfpathmoveto{\pgfqpoint{0.000000in}{0.000000in}}%
\pgfpathlineto{\pgfqpoint{0.000000in}{-0.027778in}}%
\pgfusepath{stroke,fill}%
}%
\begin{pgfscope}%
\pgfsys@transformshift{7.118577in}{3.363889in}%
\pgfsys@useobject{currentmarker}{}%
\end{pgfscope}%
\end{pgfscope}%
\begin{pgfscope}%
\pgfsetbuttcap%
\pgfsetroundjoin%
\definecolor{currentfill}{rgb}{0.000000,0.000000,0.000000}%
\pgfsetfillcolor{currentfill}%
\pgfsetlinewidth{0.602250pt}%
\definecolor{currentstroke}{rgb}{0.000000,0.000000,0.000000}%
\pgfsetstrokecolor{currentstroke}%
\pgfsetdash{}{0pt}%
\pgfsys@defobject{currentmarker}{\pgfqpoint{0.000000in}{-0.027778in}}{\pgfqpoint{0.000000in}{0.000000in}}{%
\pgfpathmoveto{\pgfqpoint{0.000000in}{0.000000in}}%
\pgfpathlineto{\pgfqpoint{0.000000in}{-0.027778in}}%
\pgfusepath{stroke,fill}%
}%
\begin{pgfscope}%
\pgfsys@transformshift{7.191331in}{3.363889in}%
\pgfsys@useobject{currentmarker}{}%
\end{pgfscope}%
\end{pgfscope}%
\begin{pgfscope}%
\pgfsetbuttcap%
\pgfsetroundjoin%
\definecolor{currentfill}{rgb}{0.000000,0.000000,0.000000}%
\pgfsetfillcolor{currentfill}%
\pgfsetlinewidth{0.602250pt}%
\definecolor{currentstroke}{rgb}{0.000000,0.000000,0.000000}%
\pgfsetstrokecolor{currentstroke}%
\pgfsetdash{}{0pt}%
\pgfsys@defobject{currentmarker}{\pgfqpoint{0.000000in}{-0.027778in}}{\pgfqpoint{0.000000in}{0.000000in}}{%
\pgfpathmoveto{\pgfqpoint{0.000000in}{0.000000in}}%
\pgfpathlineto{\pgfqpoint{0.000000in}{-0.027778in}}%
\pgfusepath{stroke,fill}%
}%
\begin{pgfscope}%
\pgfsys@transformshift{7.247763in}{3.363889in}%
\pgfsys@useobject{currentmarker}{}%
\end{pgfscope}%
\end{pgfscope}%
\begin{pgfscope}%
\pgfsetbuttcap%
\pgfsetroundjoin%
\definecolor{currentfill}{rgb}{0.000000,0.000000,0.000000}%
\pgfsetfillcolor{currentfill}%
\pgfsetlinewidth{0.602250pt}%
\definecolor{currentstroke}{rgb}{0.000000,0.000000,0.000000}%
\pgfsetstrokecolor{currentstroke}%
\pgfsetdash{}{0pt}%
\pgfsys@defobject{currentmarker}{\pgfqpoint{0.000000in}{-0.027778in}}{\pgfqpoint{0.000000in}{0.000000in}}{%
\pgfpathmoveto{\pgfqpoint{0.000000in}{0.000000in}}%
\pgfpathlineto{\pgfqpoint{0.000000in}{-0.027778in}}%
\pgfusepath{stroke,fill}%
}%
\begin{pgfscope}%
\pgfsys@transformshift{7.293872in}{3.363889in}%
\pgfsys@useobject{currentmarker}{}%
\end{pgfscope}%
\end{pgfscope}%
\begin{pgfscope}%
\pgfsetbuttcap%
\pgfsetroundjoin%
\definecolor{currentfill}{rgb}{0.000000,0.000000,0.000000}%
\pgfsetfillcolor{currentfill}%
\pgfsetlinewidth{0.602250pt}%
\definecolor{currentstroke}{rgb}{0.000000,0.000000,0.000000}%
\pgfsetstrokecolor{currentstroke}%
\pgfsetdash{}{0pt}%
\pgfsys@defobject{currentmarker}{\pgfqpoint{0.000000in}{-0.027778in}}{\pgfqpoint{0.000000in}{0.000000in}}{%
\pgfpathmoveto{\pgfqpoint{0.000000in}{0.000000in}}%
\pgfpathlineto{\pgfqpoint{0.000000in}{-0.027778in}}%
\pgfusepath{stroke,fill}%
}%
\begin{pgfscope}%
\pgfsys@transformshift{7.332856in}{3.363889in}%
\pgfsys@useobject{currentmarker}{}%
\end{pgfscope}%
\end{pgfscope}%
\begin{pgfscope}%
\pgfsetbuttcap%
\pgfsetroundjoin%
\definecolor{currentfill}{rgb}{0.000000,0.000000,0.000000}%
\pgfsetfillcolor{currentfill}%
\pgfsetlinewidth{0.602250pt}%
\definecolor{currentstroke}{rgb}{0.000000,0.000000,0.000000}%
\pgfsetstrokecolor{currentstroke}%
\pgfsetdash{}{0pt}%
\pgfsys@defobject{currentmarker}{\pgfqpoint{0.000000in}{-0.027778in}}{\pgfqpoint{0.000000in}{0.000000in}}{%
\pgfpathmoveto{\pgfqpoint{0.000000in}{0.000000in}}%
\pgfpathlineto{\pgfqpoint{0.000000in}{-0.027778in}}%
\pgfusepath{stroke,fill}%
}%
\begin{pgfscope}%
\pgfsys@transformshift{7.366626in}{3.363889in}%
\pgfsys@useobject{currentmarker}{}%
\end{pgfscope}%
\end{pgfscope}%
\begin{pgfscope}%
\pgfsetbuttcap%
\pgfsetroundjoin%
\definecolor{currentfill}{rgb}{0.000000,0.000000,0.000000}%
\pgfsetfillcolor{currentfill}%
\pgfsetlinewidth{0.602250pt}%
\definecolor{currentstroke}{rgb}{0.000000,0.000000,0.000000}%
\pgfsetstrokecolor{currentstroke}%
\pgfsetdash{}{0pt}%
\pgfsys@defobject{currentmarker}{\pgfqpoint{0.000000in}{-0.027778in}}{\pgfqpoint{0.000000in}{0.000000in}}{%
\pgfpathmoveto{\pgfqpoint{0.000000in}{0.000000in}}%
\pgfpathlineto{\pgfqpoint{0.000000in}{-0.027778in}}%
\pgfusepath{stroke,fill}%
}%
\begin{pgfscope}%
\pgfsys@transformshift{7.396413in}{3.363889in}%
\pgfsys@useobject{currentmarker}{}%
\end{pgfscope}%
\end{pgfscope}%
\begin{pgfscope}%
\pgfsetbuttcap%
\pgfsetroundjoin%
\definecolor{currentfill}{rgb}{0.000000,0.000000,0.000000}%
\pgfsetfillcolor{currentfill}%
\pgfsetlinewidth{0.602250pt}%
\definecolor{currentstroke}{rgb}{0.000000,0.000000,0.000000}%
\pgfsetstrokecolor{currentstroke}%
\pgfsetdash{}{0pt}%
\pgfsys@defobject{currentmarker}{\pgfqpoint{0.000000in}{-0.027778in}}{\pgfqpoint{0.000000in}{0.000000in}}{%
\pgfpathmoveto{\pgfqpoint{0.000000in}{0.000000in}}%
\pgfpathlineto{\pgfqpoint{0.000000in}{-0.027778in}}%
\pgfusepath{stroke,fill}%
}%
\begin{pgfscope}%
\pgfsys@transformshift{7.598353in}{3.363889in}%
\pgfsys@useobject{currentmarker}{}%
\end{pgfscope}%
\end{pgfscope}%
\begin{pgfscope}%
\pgfsetbuttcap%
\pgfsetroundjoin%
\definecolor{currentfill}{rgb}{0.000000,0.000000,0.000000}%
\pgfsetfillcolor{currentfill}%
\pgfsetlinewidth{0.602250pt}%
\definecolor{currentstroke}{rgb}{0.000000,0.000000,0.000000}%
\pgfsetstrokecolor{currentstroke}%
\pgfsetdash{}{0pt}%
\pgfsys@defobject{currentmarker}{\pgfqpoint{0.000000in}{-0.027778in}}{\pgfqpoint{0.000000in}{0.000000in}}{%
\pgfpathmoveto{\pgfqpoint{0.000000in}{0.000000in}}%
\pgfpathlineto{\pgfqpoint{0.000000in}{-0.027778in}}%
\pgfusepath{stroke,fill}%
}%
\begin{pgfscope}%
\pgfsys@transformshift{7.700894in}{3.363889in}%
\pgfsys@useobject{currentmarker}{}%
\end{pgfscope}%
\end{pgfscope}%
\begin{pgfscope}%
\pgfsetbuttcap%
\pgfsetroundjoin%
\definecolor{currentfill}{rgb}{0.000000,0.000000,0.000000}%
\pgfsetfillcolor{currentfill}%
\pgfsetlinewidth{0.602250pt}%
\definecolor{currentstroke}{rgb}{0.000000,0.000000,0.000000}%
\pgfsetstrokecolor{currentstroke}%
\pgfsetdash{}{0pt}%
\pgfsys@defobject{currentmarker}{\pgfqpoint{0.000000in}{-0.027778in}}{\pgfqpoint{0.000000in}{0.000000in}}{%
\pgfpathmoveto{\pgfqpoint{0.000000in}{0.000000in}}%
\pgfpathlineto{\pgfqpoint{0.000000in}{-0.027778in}}%
\pgfusepath{stroke,fill}%
}%
\begin{pgfscope}%
\pgfsys@transformshift{7.773648in}{3.363889in}%
\pgfsys@useobject{currentmarker}{}%
\end{pgfscope}%
\end{pgfscope}%
\begin{pgfscope}%
\pgfsetbuttcap%
\pgfsetroundjoin%
\definecolor{currentfill}{rgb}{0.000000,0.000000,0.000000}%
\pgfsetfillcolor{currentfill}%
\pgfsetlinewidth{0.803000pt}%
\definecolor{currentstroke}{rgb}{0.000000,0.000000,0.000000}%
\pgfsetstrokecolor{currentstroke}%
\pgfsetdash{}{0pt}%
\pgfsys@defobject{currentmarker}{\pgfqpoint{-0.048611in}{0.000000in}}{\pgfqpoint{0.000000in}{0.000000in}}{%
\pgfpathmoveto{\pgfqpoint{0.000000in}{0.000000in}}%
\pgfpathlineto{\pgfqpoint{-0.048611in}{0.000000in}}%
\pgfusepath{stroke,fill}%
}%
\begin{pgfscope}%
\pgfsys@transformshift{4.597569in}{3.607430in}%
\pgfsys@useobject{currentmarker}{}%
\end{pgfscope}%
\end{pgfscope}%
\begin{pgfscope}%
\definecolor{textcolor}{rgb}{0.000000,0.000000,0.000000}%
\pgfsetstrokecolor{textcolor}%
\pgfsetfillcolor{textcolor}%
\pgftext[x=4.279468in,y=3.554668in,left,base]{\color{textcolor}\sffamily\fontsize{10.000000}{12.000000}\selectfont 0.0}%
\end{pgfscope}%
\begin{pgfscope}%
\pgfsetbuttcap%
\pgfsetroundjoin%
\definecolor{currentfill}{rgb}{0.000000,0.000000,0.000000}%
\pgfsetfillcolor{currentfill}%
\pgfsetlinewidth{0.803000pt}%
\definecolor{currentstroke}{rgb}{0.000000,0.000000,0.000000}%
\pgfsetstrokecolor{currentstroke}%
\pgfsetdash{}{0pt}%
\pgfsys@defobject{currentmarker}{\pgfqpoint{-0.048611in}{0.000000in}}{\pgfqpoint{0.000000in}{0.000000in}}{%
\pgfpathmoveto{\pgfqpoint{0.000000in}{0.000000in}}%
\pgfpathlineto{\pgfqpoint{-0.048611in}{0.000000in}}%
\pgfusepath{stroke,fill}%
}%
\begin{pgfscope}%
\pgfsys@transformshift{4.597569in}{3.890541in}%
\pgfsys@useobject{currentmarker}{}%
\end{pgfscope}%
\end{pgfscope}%
\begin{pgfscope}%
\definecolor{textcolor}{rgb}{0.000000,0.000000,0.000000}%
\pgfsetstrokecolor{textcolor}%
\pgfsetfillcolor{textcolor}%
\pgftext[x=4.279468in,y=3.837779in,left,base]{\color{textcolor}\sffamily\fontsize{10.000000}{12.000000}\selectfont 0.2}%
\end{pgfscope}%
\begin{pgfscope}%
\pgfsetbuttcap%
\pgfsetroundjoin%
\definecolor{currentfill}{rgb}{0.000000,0.000000,0.000000}%
\pgfsetfillcolor{currentfill}%
\pgfsetlinewidth{0.803000pt}%
\definecolor{currentstroke}{rgb}{0.000000,0.000000,0.000000}%
\pgfsetstrokecolor{currentstroke}%
\pgfsetdash{}{0pt}%
\pgfsys@defobject{currentmarker}{\pgfqpoint{-0.048611in}{0.000000in}}{\pgfqpoint{0.000000in}{0.000000in}}{%
\pgfpathmoveto{\pgfqpoint{0.000000in}{0.000000in}}%
\pgfpathlineto{\pgfqpoint{-0.048611in}{0.000000in}}%
\pgfusepath{stroke,fill}%
}%
\begin{pgfscope}%
\pgfsys@transformshift{4.597569in}{4.173652in}%
\pgfsys@useobject{currentmarker}{}%
\end{pgfscope}%
\end{pgfscope}%
\begin{pgfscope}%
\definecolor{textcolor}{rgb}{0.000000,0.000000,0.000000}%
\pgfsetstrokecolor{textcolor}%
\pgfsetfillcolor{textcolor}%
\pgftext[x=4.279468in,y=4.120890in,left,base]{\color{textcolor}\sffamily\fontsize{10.000000}{12.000000}\selectfont 0.4}%
\end{pgfscope}%
\begin{pgfscope}%
\pgfsetbuttcap%
\pgfsetroundjoin%
\definecolor{currentfill}{rgb}{0.000000,0.000000,0.000000}%
\pgfsetfillcolor{currentfill}%
\pgfsetlinewidth{0.803000pt}%
\definecolor{currentstroke}{rgb}{0.000000,0.000000,0.000000}%
\pgfsetstrokecolor{currentstroke}%
\pgfsetdash{}{0pt}%
\pgfsys@defobject{currentmarker}{\pgfqpoint{-0.048611in}{0.000000in}}{\pgfqpoint{0.000000in}{0.000000in}}{%
\pgfpathmoveto{\pgfqpoint{0.000000in}{0.000000in}}%
\pgfpathlineto{\pgfqpoint{-0.048611in}{0.000000in}}%
\pgfusepath{stroke,fill}%
}%
\begin{pgfscope}%
\pgfsys@transformshift{4.597569in}{4.456763in}%
\pgfsys@useobject{currentmarker}{}%
\end{pgfscope}%
\end{pgfscope}%
\begin{pgfscope}%
\definecolor{textcolor}{rgb}{0.000000,0.000000,0.000000}%
\pgfsetstrokecolor{textcolor}%
\pgfsetfillcolor{textcolor}%
\pgftext[x=4.279468in,y=4.404001in,left,base]{\color{textcolor}\sffamily\fontsize{10.000000}{12.000000}\selectfont 0.6}%
\end{pgfscope}%
\begin{pgfscope}%
\pgfsetbuttcap%
\pgfsetroundjoin%
\definecolor{currentfill}{rgb}{0.000000,0.000000,0.000000}%
\pgfsetfillcolor{currentfill}%
\pgfsetlinewidth{0.803000pt}%
\definecolor{currentstroke}{rgb}{0.000000,0.000000,0.000000}%
\pgfsetstrokecolor{currentstroke}%
\pgfsetdash{}{0pt}%
\pgfsys@defobject{currentmarker}{\pgfqpoint{-0.048611in}{0.000000in}}{\pgfqpoint{0.000000in}{0.000000in}}{%
\pgfpathmoveto{\pgfqpoint{0.000000in}{0.000000in}}%
\pgfpathlineto{\pgfqpoint{-0.048611in}{0.000000in}}%
\pgfusepath{stroke,fill}%
}%
\begin{pgfscope}%
\pgfsys@transformshift{4.597569in}{4.739874in}%
\pgfsys@useobject{currentmarker}{}%
\end{pgfscope}%
\end{pgfscope}%
\begin{pgfscope}%
\definecolor{textcolor}{rgb}{0.000000,0.000000,0.000000}%
\pgfsetstrokecolor{textcolor}%
\pgfsetfillcolor{textcolor}%
\pgftext[x=4.279468in,y=4.687113in,left,base]{\color{textcolor}\sffamily\fontsize{10.000000}{12.000000}\selectfont 0.8}%
\end{pgfscope}%
\begin{pgfscope}%
\pgfsetbuttcap%
\pgfsetroundjoin%
\definecolor{currentfill}{rgb}{0.000000,0.000000,0.000000}%
\pgfsetfillcolor{currentfill}%
\pgfsetlinewidth{0.803000pt}%
\definecolor{currentstroke}{rgb}{0.000000,0.000000,0.000000}%
\pgfsetstrokecolor{currentstroke}%
\pgfsetdash{}{0pt}%
\pgfsys@defobject{currentmarker}{\pgfqpoint{-0.048611in}{0.000000in}}{\pgfqpoint{0.000000in}{0.000000in}}{%
\pgfpathmoveto{\pgfqpoint{0.000000in}{0.000000in}}%
\pgfpathlineto{\pgfqpoint{-0.048611in}{0.000000in}}%
\pgfusepath{stroke,fill}%
}%
\begin{pgfscope}%
\pgfsys@transformshift{4.597569in}{5.022985in}%
\pgfsys@useobject{currentmarker}{}%
\end{pgfscope}%
\end{pgfscope}%
\begin{pgfscope}%
\definecolor{textcolor}{rgb}{0.000000,0.000000,0.000000}%
\pgfsetstrokecolor{textcolor}%
\pgfsetfillcolor{textcolor}%
\pgftext[x=4.279468in,y=4.970224in,left,base]{\color{textcolor}\sffamily\fontsize{10.000000}{12.000000}\selectfont 1.0}%
\end{pgfscope}%
\begin{pgfscope}%
\pgfsetbuttcap%
\pgfsetroundjoin%
\definecolor{currentfill}{rgb}{0.000000,0.000000,0.000000}%
\pgfsetfillcolor{currentfill}%
\pgfsetlinewidth{0.803000pt}%
\definecolor{currentstroke}{rgb}{0.000000,0.000000,0.000000}%
\pgfsetstrokecolor{currentstroke}%
\pgfsetdash{}{0pt}%
\pgfsys@defobject{currentmarker}{\pgfqpoint{-0.048611in}{0.000000in}}{\pgfqpoint{0.000000in}{0.000000in}}{%
\pgfpathmoveto{\pgfqpoint{0.000000in}{0.000000in}}%
\pgfpathlineto{\pgfqpoint{-0.048611in}{0.000000in}}%
\pgfusepath{stroke,fill}%
}%
\begin{pgfscope}%
\pgfsys@transformshift{4.597569in}{5.306096in}%
\pgfsys@useobject{currentmarker}{}%
\end{pgfscope}%
\end{pgfscope}%
\begin{pgfscope}%
\definecolor{textcolor}{rgb}{0.000000,0.000000,0.000000}%
\pgfsetstrokecolor{textcolor}%
\pgfsetfillcolor{textcolor}%
\pgftext[x=4.279468in,y=5.253335in,left,base]{\color{textcolor}\sffamily\fontsize{10.000000}{12.000000}\selectfont 1.2}%
\end{pgfscope}%
\begin{pgfscope}%
\pgfsetbuttcap%
\pgfsetroundjoin%
\definecolor{currentfill}{rgb}{0.000000,0.000000,0.000000}%
\pgfsetfillcolor{currentfill}%
\pgfsetlinewidth{0.803000pt}%
\definecolor{currentstroke}{rgb}{0.000000,0.000000,0.000000}%
\pgfsetstrokecolor{currentstroke}%
\pgfsetdash{}{0pt}%
\pgfsys@defobject{currentmarker}{\pgfqpoint{-0.048611in}{0.000000in}}{\pgfqpoint{0.000000in}{0.000000in}}{%
\pgfpathmoveto{\pgfqpoint{0.000000in}{0.000000in}}%
\pgfpathlineto{\pgfqpoint{-0.048611in}{0.000000in}}%
\pgfusepath{stroke,fill}%
}%
\begin{pgfscope}%
\pgfsys@transformshift{4.597569in}{5.589207in}%
\pgfsys@useobject{currentmarker}{}%
\end{pgfscope}%
\end{pgfscope}%
\begin{pgfscope}%
\definecolor{textcolor}{rgb}{0.000000,0.000000,0.000000}%
\pgfsetstrokecolor{textcolor}%
\pgfsetfillcolor{textcolor}%
\pgftext[x=4.279468in,y=5.536446in,left,base]{\color{textcolor}\sffamily\fontsize{10.000000}{12.000000}\selectfont 1.4}%
\end{pgfscope}%
\begin{pgfscope}%
\pgfpathrectangle{\pgfqpoint{4.597569in}{3.363889in}}{\pgfqpoint{3.203819in}{2.263889in}}%
\pgfusepath{clip}%
\pgfsetrectcap%
\pgfsetroundjoin%
\pgfsetlinewidth{1.505625pt}%
\definecolor{currentstroke}{rgb}{0.121569,0.466667,0.705882}%
\pgfsetstrokecolor{currentstroke}%
\pgfsetdash{}{0pt}%
\pgfpathmoveto{\pgfqpoint{4.743198in}{3.789285in}}%
\pgfpathlineto{\pgfqpoint{4.888777in}{4.181473in}}%
\pgfpathlineto{\pgfqpoint{5.034356in}{4.235616in}}%
\pgfpathlineto{\pgfqpoint{5.179936in}{4.722540in}}%
\pgfpathlineto{\pgfqpoint{5.325515in}{4.508485in}}%
\pgfpathlineto{\pgfqpoint{5.471095in}{4.730858in}}%
\pgfpathlineto{\pgfqpoint{5.616674in}{4.708252in}}%
\pgfpathlineto{\pgfqpoint{5.762253in}{4.704728in}}%
\pgfpathlineto{\pgfqpoint{5.907833in}{5.020398in}}%
\pgfpathlineto{\pgfqpoint{6.053412in}{5.020400in}}%
\pgfpathlineto{\pgfqpoint{6.198992in}{5.020401in}}%
\pgfpathlineto{\pgfqpoint{6.344571in}{5.020401in}}%
\pgfpathlineto{\pgfqpoint{6.490150in}{5.020402in}}%
\pgfpathlineto{\pgfqpoint{6.635730in}{5.020401in}}%
\pgfpathlineto{\pgfqpoint{6.781309in}{5.020402in}}%
\pgfpathlineto{\pgfqpoint{6.926889in}{5.020401in}}%
\pgfpathlineto{\pgfqpoint{7.072468in}{5.020401in}}%
\pgfpathlineto{\pgfqpoint{7.218047in}{5.020401in}}%
\pgfpathlineto{\pgfqpoint{7.363627in}{5.020401in}}%
\pgfpathlineto{\pgfqpoint{7.509206in}{5.020401in}}%
\pgfpathlineto{\pgfqpoint{7.654786in}{5.020401in}}%
\pgfusepath{stroke}%
\end{pgfscope}%
\begin{pgfscope}%
\pgfpathrectangle{\pgfqpoint{4.597569in}{3.363889in}}{\pgfqpoint{3.203819in}{2.263889in}}%
\pgfusepath{clip}%
\pgfsetrectcap%
\pgfsetroundjoin%
\pgfsetlinewidth{1.505625pt}%
\definecolor{currentstroke}{rgb}{1.000000,0.498039,0.054902}%
\pgfsetstrokecolor{currentstroke}%
\pgfsetdash{}{0pt}%
\pgfpathmoveto{\pgfqpoint{4.743198in}{3.714267in}}%
\pgfpathlineto{\pgfqpoint{4.888777in}{3.804735in}}%
\pgfpathlineto{\pgfqpoint{5.034356in}{4.207125in}}%
\pgfpathlineto{\pgfqpoint{5.179936in}{4.651449in}}%
\pgfpathlineto{\pgfqpoint{5.325515in}{4.688843in}}%
\pgfpathlineto{\pgfqpoint{5.471095in}{4.397384in}}%
\pgfpathlineto{\pgfqpoint{5.616674in}{4.979136in}}%
\pgfpathlineto{\pgfqpoint{5.762253in}{4.874560in}}%
\pgfpathlineto{\pgfqpoint{5.907833in}{5.003656in}}%
\pgfpathlineto{\pgfqpoint{6.053412in}{5.003720in}}%
\pgfpathlineto{\pgfqpoint{6.198992in}{5.003717in}}%
\pgfpathlineto{\pgfqpoint{6.344571in}{5.003742in}}%
\pgfpathlineto{\pgfqpoint{6.490150in}{5.003742in}}%
\pgfpathlineto{\pgfqpoint{6.635730in}{5.003711in}}%
\pgfpathlineto{\pgfqpoint{6.781309in}{5.003748in}}%
\pgfpathlineto{\pgfqpoint{6.926889in}{5.003744in}}%
\pgfpathlineto{\pgfqpoint{7.072468in}{5.003735in}}%
\pgfpathlineto{\pgfqpoint{7.218047in}{5.003743in}}%
\pgfpathlineto{\pgfqpoint{7.363627in}{5.003737in}}%
\pgfpathlineto{\pgfqpoint{7.509206in}{5.003734in}}%
\pgfpathlineto{\pgfqpoint{7.654786in}{5.003739in}}%
\pgfusepath{stroke}%
\end{pgfscope}%
\begin{pgfscope}%
\pgfpathrectangle{\pgfqpoint{4.597569in}{3.363889in}}{\pgfqpoint{3.203819in}{2.263889in}}%
\pgfusepath{clip}%
\pgfsetrectcap%
\pgfsetroundjoin%
\pgfsetlinewidth{1.505625pt}%
\definecolor{currentstroke}{rgb}{0.172549,0.627451,0.172549}%
\pgfsetstrokecolor{currentstroke}%
\pgfsetdash{}{0pt}%
\pgfpathmoveto{\pgfqpoint{4.743198in}{3.784815in}}%
\pgfpathlineto{\pgfqpoint{4.888777in}{3.727902in}}%
\pgfpathlineto{\pgfqpoint{5.034356in}{4.075821in}}%
\pgfpathlineto{\pgfqpoint{5.179936in}{4.411848in}}%
\pgfpathlineto{\pgfqpoint{5.325515in}{4.093955in}}%
\pgfpathlineto{\pgfqpoint{5.471095in}{4.323982in}}%
\pgfpathlineto{\pgfqpoint{5.616674in}{4.843650in}}%
\pgfpathlineto{\pgfqpoint{5.762253in}{4.899680in}}%
\pgfpathlineto{\pgfqpoint{5.907833in}{4.897408in}}%
\pgfpathlineto{\pgfqpoint{6.053412in}{4.894732in}}%
\pgfpathlineto{\pgfqpoint{6.198992in}{4.896714in}}%
\pgfpathlineto{\pgfqpoint{6.344571in}{4.897550in}}%
\pgfpathlineto{\pgfqpoint{6.490150in}{4.897269in}}%
\pgfpathlineto{\pgfqpoint{6.635730in}{4.897600in}}%
\pgfpathlineto{\pgfqpoint{6.781309in}{4.897312in}}%
\pgfpathlineto{\pgfqpoint{6.926889in}{4.897467in}}%
\pgfpathlineto{\pgfqpoint{7.072468in}{4.897223in}}%
\pgfpathlineto{\pgfqpoint{7.218047in}{4.897223in}}%
\pgfpathlineto{\pgfqpoint{7.363627in}{4.897436in}}%
\pgfpathlineto{\pgfqpoint{7.509206in}{4.897410in}}%
\pgfpathlineto{\pgfqpoint{7.654786in}{4.897379in}}%
\pgfusepath{stroke}%
\end{pgfscope}%
\begin{pgfscope}%
\pgfpathrectangle{\pgfqpoint{4.597569in}{3.363889in}}{\pgfqpoint{3.203819in}{2.263889in}}%
\pgfusepath{clip}%
\pgfsetrectcap%
\pgfsetroundjoin%
\pgfsetlinewidth{1.505625pt}%
\definecolor{currentstroke}{rgb}{0.839216,0.152941,0.156863}%
\pgfsetstrokecolor{currentstroke}%
\pgfsetdash{}{0pt}%
\pgfpathmoveto{\pgfqpoint{4.743198in}{3.711585in}}%
\pgfpathlineto{\pgfqpoint{4.888777in}{3.825136in}}%
\pgfpathlineto{\pgfqpoint{5.034356in}{3.780969in}}%
\pgfpathlineto{\pgfqpoint{5.179936in}{3.687404in}}%
\pgfpathlineto{\pgfqpoint{5.325515in}{3.899836in}}%
\pgfpathlineto{\pgfqpoint{5.471095in}{3.761729in}}%
\pgfpathlineto{\pgfqpoint{5.616674in}{3.923507in}}%
\pgfpathlineto{\pgfqpoint{5.762253in}{3.898315in}}%
\pgfpathlineto{\pgfqpoint{5.907833in}{3.885477in}}%
\pgfpathlineto{\pgfqpoint{6.053412in}{3.923710in}}%
\pgfpathlineto{\pgfqpoint{6.198992in}{3.888030in}}%
\pgfpathlineto{\pgfqpoint{6.344571in}{3.883960in}}%
\pgfpathlineto{\pgfqpoint{6.490150in}{3.898700in}}%
\pgfpathlineto{\pgfqpoint{6.635730in}{3.893700in}}%
\pgfpathlineto{\pgfqpoint{6.781309in}{3.885673in}}%
\pgfpathlineto{\pgfqpoint{6.926889in}{3.889178in}}%
\pgfpathlineto{\pgfqpoint{7.072468in}{3.896434in}}%
\pgfpathlineto{\pgfqpoint{7.218047in}{3.888814in}}%
\pgfpathlineto{\pgfqpoint{7.363627in}{3.894636in}}%
\pgfpathlineto{\pgfqpoint{7.509206in}{3.890218in}}%
\pgfpathlineto{\pgfqpoint{7.654786in}{3.892888in}}%
\pgfusepath{stroke}%
\end{pgfscope}%
\begin{pgfscope}%
\pgfpathrectangle{\pgfqpoint{4.597569in}{3.363889in}}{\pgfqpoint{3.203819in}{2.263889in}}%
\pgfusepath{clip}%
\pgfsetrectcap%
\pgfsetroundjoin%
\pgfsetlinewidth{1.505625pt}%
\definecolor{currentstroke}{rgb}{0.580392,0.403922,0.741176}%
\pgfsetstrokecolor{currentstroke}%
\pgfsetdash{}{0pt}%
\pgfpathmoveto{\pgfqpoint{4.743198in}{3.672771in}}%
\pgfpathlineto{\pgfqpoint{4.888777in}{3.692738in}}%
\pgfpathlineto{\pgfqpoint{5.034356in}{3.692571in}}%
\pgfpathlineto{\pgfqpoint{5.179936in}{3.698854in}}%
\pgfpathlineto{\pgfqpoint{5.325515in}{3.765937in}}%
\pgfpathlineto{\pgfqpoint{5.471095in}{3.735452in}}%
\pgfpathlineto{\pgfqpoint{5.616674in}{3.698868in}}%
\pgfpathlineto{\pgfqpoint{5.762253in}{3.735253in}}%
\pgfpathlineto{\pgfqpoint{5.907833in}{3.727262in}}%
\pgfpathlineto{\pgfqpoint{6.053412in}{3.733837in}}%
\pgfpathlineto{\pgfqpoint{6.198992in}{3.729804in}}%
\pgfpathlineto{\pgfqpoint{6.344571in}{3.730395in}}%
\pgfpathlineto{\pgfqpoint{6.490150in}{3.725007in}}%
\pgfpathlineto{\pgfqpoint{6.635730in}{3.730992in}}%
\pgfpathlineto{\pgfqpoint{6.781309in}{3.727099in}}%
\pgfpathlineto{\pgfqpoint{6.926889in}{3.729625in}}%
\pgfpathlineto{\pgfqpoint{7.072468in}{3.728683in}}%
\pgfpathlineto{\pgfqpoint{7.218047in}{3.728440in}}%
\pgfpathlineto{\pgfqpoint{7.363627in}{3.727713in}}%
\pgfpathlineto{\pgfqpoint{7.509206in}{3.727808in}}%
\pgfpathlineto{\pgfqpoint{7.654786in}{3.727714in}}%
\pgfusepath{stroke}%
\end{pgfscope}%
\begin{pgfscope}%
\pgfsetrectcap%
\pgfsetmiterjoin%
\pgfsetlinewidth{0.803000pt}%
\definecolor{currentstroke}{rgb}{0.000000,0.000000,0.000000}%
\pgfsetstrokecolor{currentstroke}%
\pgfsetdash{}{0pt}%
\pgfpathmoveto{\pgfqpoint{4.597569in}{3.363889in}}%
\pgfpathlineto{\pgfqpoint{4.597569in}{5.627778in}}%
\pgfusepath{stroke}%
\end{pgfscope}%
\begin{pgfscope}%
\pgfsetrectcap%
\pgfsetmiterjoin%
\pgfsetlinewidth{0.803000pt}%
\definecolor{currentstroke}{rgb}{0.000000,0.000000,0.000000}%
\pgfsetstrokecolor{currentstroke}%
\pgfsetdash{}{0pt}%
\pgfpathmoveto{\pgfqpoint{7.801389in}{3.363889in}}%
\pgfpathlineto{\pgfqpoint{7.801389in}{5.627778in}}%
\pgfusepath{stroke}%
\end{pgfscope}%
\begin{pgfscope}%
\pgfsetrectcap%
\pgfsetmiterjoin%
\pgfsetlinewidth{0.803000pt}%
\definecolor{currentstroke}{rgb}{0.000000,0.000000,0.000000}%
\pgfsetstrokecolor{currentstroke}%
\pgfsetdash{}{0pt}%
\pgfpathmoveto{\pgfqpoint{4.597569in}{3.363889in}}%
\pgfpathlineto{\pgfqpoint{7.801389in}{3.363889in}}%
\pgfusepath{stroke}%
\end{pgfscope}%
\begin{pgfscope}%
\pgfsetrectcap%
\pgfsetmiterjoin%
\pgfsetlinewidth{0.803000pt}%
\definecolor{currentstroke}{rgb}{0.000000,0.000000,0.000000}%
\pgfsetstrokecolor{currentstroke}%
\pgfsetdash{}{0pt}%
\pgfpathmoveto{\pgfqpoint{4.597569in}{5.627778in}}%
\pgfpathlineto{\pgfqpoint{7.801389in}{5.627778in}}%
\pgfusepath{stroke}%
\end{pgfscope}%
\begin{pgfscope}%
\definecolor{textcolor}{rgb}{0.000000,0.000000,0.000000}%
\pgfsetstrokecolor{textcolor}%
\pgfsetfillcolor{textcolor}%
\pgftext[x=6.199479in,y=5.711111in,,base]{\color{textcolor}\sffamily\fontsize{12.000000}{14.400000}\selectfont \(\displaystyle  N = 32 \)}%
\end{pgfscope}%
\begin{pgfscope}%
\pgfsetbuttcap%
\pgfsetmiterjoin%
\definecolor{currentfill}{rgb}{1.000000,1.000000,1.000000}%
\pgfsetfillcolor{currentfill}%
\pgfsetlinewidth{0.000000pt}%
\definecolor{currentstroke}{rgb}{0.000000,0.000000,0.000000}%
\pgfsetstrokecolor{currentstroke}%
\pgfsetstrokeopacity{0.000000}%
\pgfsetdash{}{0pt}%
\pgfpathmoveto{\pgfqpoint{0.672569in}{0.438889in}}%
\pgfpathlineto{\pgfqpoint{3.876389in}{0.438889in}}%
\pgfpathlineto{\pgfqpoint{3.876389in}{2.702778in}}%
\pgfpathlineto{\pgfqpoint{0.672569in}{2.702778in}}%
\pgfpathclose%
\pgfusepath{fill}%
\end{pgfscope}%
\begin{pgfscope}%
\pgfpathrectangle{\pgfqpoint{0.672569in}{0.438889in}}{\pgfqpoint{3.203819in}{2.263889in}}%
\pgfusepath{clip}%
\pgfsetbuttcap%
\pgfsetroundjoin%
\definecolor{currentfill}{rgb}{0.121569,0.466667,0.705882}%
\pgfsetfillcolor{currentfill}%
\pgfsetlinewidth{1.003750pt}%
\definecolor{currentstroke}{rgb}{0.121569,0.466667,0.705882}%
\pgfsetstrokecolor{currentstroke}%
\pgfsetdash{}{0pt}%
\pgfsys@defobject{currentmarker}{\pgfqpoint{-0.009821in}{-0.009821in}}{\pgfqpoint{0.009821in}{0.009821in}}{%
\pgfpathmoveto{\pgfqpoint{0.000000in}{-0.009821in}}%
\pgfpathcurveto{\pgfqpoint{0.002605in}{-0.009821in}}{\pgfqpoint{0.005103in}{-0.008786in}}{\pgfqpoint{0.006944in}{-0.006944in}}%
\pgfpathcurveto{\pgfqpoint{0.008786in}{-0.005103in}}{\pgfqpoint{0.009821in}{-0.002605in}}{\pgfqpoint{0.009821in}{0.000000in}}%
\pgfpathcurveto{\pgfqpoint{0.009821in}{0.002605in}}{\pgfqpoint{0.008786in}{0.005103in}}{\pgfqpoint{0.006944in}{0.006944in}}%
\pgfpathcurveto{\pgfqpoint{0.005103in}{0.008786in}}{\pgfqpoint{0.002605in}{0.009821in}}{\pgfqpoint{0.000000in}{0.009821in}}%
\pgfpathcurveto{\pgfqpoint{-0.002605in}{0.009821in}}{\pgfqpoint{-0.005103in}{0.008786in}}{\pgfqpoint{-0.006944in}{0.006944in}}%
\pgfpathcurveto{\pgfqpoint{-0.008786in}{0.005103in}}{\pgfqpoint{-0.009821in}{0.002605in}}{\pgfqpoint{-0.009821in}{0.000000in}}%
\pgfpathcurveto{\pgfqpoint{-0.009821in}{-0.002605in}}{\pgfqpoint{-0.008786in}{-0.005103in}}{\pgfqpoint{-0.006944in}{-0.006944in}}%
\pgfpathcurveto{\pgfqpoint{-0.005103in}{-0.008786in}}{\pgfqpoint{-0.002605in}{-0.009821in}}{\pgfqpoint{0.000000in}{-0.009821in}}%
\pgfpathclose%
\pgfusepath{stroke,fill}%
}%
\begin{pgfscope}%
\pgfsys@transformshift{0.818198in}{0.803175in}%
\pgfsys@useobject{currentmarker}{}%
\end{pgfscope}%
\begin{pgfscope}%
\pgfsys@transformshift{0.963777in}{0.834136in}%
\pgfsys@useobject{currentmarker}{}%
\end{pgfscope}%
\begin{pgfscope}%
\pgfsys@transformshift{1.109356in}{0.877730in}%
\pgfsys@useobject{currentmarker}{}%
\end{pgfscope}%
\begin{pgfscope}%
\pgfsys@transformshift{1.254936in}{0.842812in}%
\pgfsys@useobject{currentmarker}{}%
\end{pgfscope}%
\begin{pgfscope}%
\pgfsys@transformshift{1.400515in}{1.173191in}%
\pgfsys@useobject{currentmarker}{}%
\end{pgfscope}%
\begin{pgfscope}%
\pgfsys@transformshift{1.546095in}{1.401194in}%
\pgfsys@useobject{currentmarker}{}%
\end{pgfscope}%
\begin{pgfscope}%
\pgfsys@transformshift{1.691674in}{1.663434in}%
\pgfsys@useobject{currentmarker}{}%
\end{pgfscope}%
\begin{pgfscope}%
\pgfsys@transformshift{1.837253in}{1.865421in}%
\pgfsys@useobject{currentmarker}{}%
\end{pgfscope}%
\begin{pgfscope}%
\pgfsys@transformshift{1.982833in}{1.541521in}%
\pgfsys@useobject{currentmarker}{}%
\end{pgfscope}%
\begin{pgfscope}%
\pgfsys@transformshift{2.128412in}{1.501193in}%
\pgfsys@useobject{currentmarker}{}%
\end{pgfscope}%
\begin{pgfscope}%
\pgfsys@transformshift{2.273992in}{1.782931in}%
\pgfsys@useobject{currentmarker}{}%
\end{pgfscope}%
\begin{pgfscope}%
\pgfsys@transformshift{2.419571in}{1.528804in}%
\pgfsys@useobject{currentmarker}{}%
\end{pgfscope}%
\begin{pgfscope}%
\pgfsys@transformshift{2.565150in}{1.852812in}%
\pgfsys@useobject{currentmarker}{}%
\end{pgfscope}%
\begin{pgfscope}%
\pgfsys@transformshift{2.710730in}{2.134260in}%
\pgfsys@useobject{currentmarker}{}%
\end{pgfscope}%
\begin{pgfscope}%
\pgfsys@transformshift{2.856309in}{2.140356in}%
\pgfsys@useobject{currentmarker}{}%
\end{pgfscope}%
\begin{pgfscope}%
\pgfsys@transformshift{3.001889in}{2.140356in}%
\pgfsys@useobject{currentmarker}{}%
\end{pgfscope}%
\begin{pgfscope}%
\pgfsys@transformshift{3.147468in}{2.140356in}%
\pgfsys@useobject{currentmarker}{}%
\end{pgfscope}%
\begin{pgfscope}%
\pgfsys@transformshift{3.293047in}{2.140356in}%
\pgfsys@useobject{currentmarker}{}%
\end{pgfscope}%
\begin{pgfscope}%
\pgfsys@transformshift{3.438627in}{2.140356in}%
\pgfsys@useobject{currentmarker}{}%
\end{pgfscope}%
\begin{pgfscope}%
\pgfsys@transformshift{3.584206in}{2.140356in}%
\pgfsys@useobject{currentmarker}{}%
\end{pgfscope}%
\begin{pgfscope}%
\pgfsys@transformshift{3.729786in}{2.140356in}%
\pgfsys@useobject{currentmarker}{}%
\end{pgfscope}%
\end{pgfscope}%
\begin{pgfscope}%
\pgfpathrectangle{\pgfqpoint{0.672569in}{0.438889in}}{\pgfqpoint{3.203819in}{2.263889in}}%
\pgfusepath{clip}%
\pgfsetbuttcap%
\pgfsetroundjoin%
\definecolor{currentfill}{rgb}{0.121569,0.466667,0.705882}%
\pgfsetfillcolor{currentfill}%
\pgfsetfillopacity{0.300000}%
\pgfsetlinewidth{1.003750pt}%
\definecolor{currentstroke}{rgb}{0.121569,0.466667,0.705882}%
\pgfsetstrokecolor{currentstroke}%
\pgfsetstrokeopacity{0.300000}%
\pgfsetdash{}{0pt}%
\pgfpathmoveto{\pgfqpoint{0.818198in}{0.813127in}}%
\pgfpathlineto{\pgfqpoint{0.818198in}{0.793222in}}%
\pgfpathlineto{\pgfqpoint{0.963777in}{0.789455in}}%
\pgfpathlineto{\pgfqpoint{1.109356in}{0.713046in}}%
\pgfpathlineto{\pgfqpoint{1.254936in}{0.774061in}}%
\pgfpathlineto{\pgfqpoint{1.400515in}{0.848874in}}%
\pgfpathlineto{\pgfqpoint{1.546095in}{0.758350in}}%
\pgfpathlineto{\pgfqpoint{1.691674in}{0.889414in}}%
\pgfpathlineto{\pgfqpoint{1.837253in}{1.169552in}}%
\pgfpathlineto{\pgfqpoint{1.982833in}{0.638301in}}%
\pgfpathlineto{\pgfqpoint{2.128412in}{0.541793in}}%
\pgfpathlineto{\pgfqpoint{2.273992in}{1.222927in}}%
\pgfpathlineto{\pgfqpoint{2.419571in}{0.754964in}}%
\pgfpathlineto{\pgfqpoint{2.565150in}{1.105750in}}%
\pgfpathlineto{\pgfqpoint{2.710730in}{2.123519in}}%
\pgfpathlineto{\pgfqpoint{2.856309in}{2.140356in}}%
\pgfpathlineto{\pgfqpoint{3.001889in}{2.140355in}}%
\pgfpathlineto{\pgfqpoint{3.147468in}{2.140356in}}%
\pgfpathlineto{\pgfqpoint{3.293047in}{2.140356in}}%
\pgfpathlineto{\pgfqpoint{3.438627in}{2.140356in}}%
\pgfpathlineto{\pgfqpoint{3.584206in}{2.140356in}}%
\pgfpathlineto{\pgfqpoint{3.729786in}{2.140356in}}%
\pgfpathlineto{\pgfqpoint{3.729786in}{2.140356in}}%
\pgfpathlineto{\pgfqpoint{3.729786in}{2.140356in}}%
\pgfpathlineto{\pgfqpoint{3.584206in}{2.140356in}}%
\pgfpathlineto{\pgfqpoint{3.438627in}{2.140357in}}%
\pgfpathlineto{\pgfqpoint{3.293047in}{2.140356in}}%
\pgfpathlineto{\pgfqpoint{3.147468in}{2.140356in}}%
\pgfpathlineto{\pgfqpoint{3.001889in}{2.140356in}}%
\pgfpathlineto{\pgfqpoint{2.856309in}{2.140357in}}%
\pgfpathlineto{\pgfqpoint{2.710730in}{2.145001in}}%
\pgfpathlineto{\pgfqpoint{2.565150in}{2.599874in}}%
\pgfpathlineto{\pgfqpoint{2.419571in}{2.302643in}}%
\pgfpathlineto{\pgfqpoint{2.273992in}{2.342935in}}%
\pgfpathlineto{\pgfqpoint{2.128412in}{2.460594in}}%
\pgfpathlineto{\pgfqpoint{1.982833in}{2.444741in}}%
\pgfpathlineto{\pgfqpoint{1.837253in}{2.561291in}}%
\pgfpathlineto{\pgfqpoint{1.691674in}{2.437454in}}%
\pgfpathlineto{\pgfqpoint{1.546095in}{2.044038in}}%
\pgfpathlineto{\pgfqpoint{1.400515in}{1.497507in}}%
\pgfpathlineto{\pgfqpoint{1.254936in}{0.911562in}}%
\pgfpathlineto{\pgfqpoint{1.109356in}{1.042414in}}%
\pgfpathlineto{\pgfqpoint{0.963777in}{0.878816in}}%
\pgfpathlineto{\pgfqpoint{0.818198in}{0.813127in}}%
\pgfpathclose%
\pgfusepath{stroke,fill}%
\end{pgfscope}%
\begin{pgfscope}%
\pgfpathrectangle{\pgfqpoint{0.672569in}{0.438889in}}{\pgfqpoint{3.203819in}{2.263889in}}%
\pgfusepath{clip}%
\pgfsetbuttcap%
\pgfsetroundjoin%
\definecolor{currentfill}{rgb}{1.000000,0.498039,0.054902}%
\pgfsetfillcolor{currentfill}%
\pgfsetlinewidth{1.003750pt}%
\definecolor{currentstroke}{rgb}{1.000000,0.498039,0.054902}%
\pgfsetstrokecolor{currentstroke}%
\pgfsetdash{}{0pt}%
\pgfsys@defobject{currentmarker}{\pgfqpoint{-0.009821in}{-0.009821in}}{\pgfqpoint{0.009821in}{0.009821in}}{%
\pgfpathmoveto{\pgfqpoint{0.000000in}{-0.009821in}}%
\pgfpathcurveto{\pgfqpoint{0.002605in}{-0.009821in}}{\pgfqpoint{0.005103in}{-0.008786in}}{\pgfqpoint{0.006944in}{-0.006944in}}%
\pgfpathcurveto{\pgfqpoint{0.008786in}{-0.005103in}}{\pgfqpoint{0.009821in}{-0.002605in}}{\pgfqpoint{0.009821in}{0.000000in}}%
\pgfpathcurveto{\pgfqpoint{0.009821in}{0.002605in}}{\pgfqpoint{0.008786in}{0.005103in}}{\pgfqpoint{0.006944in}{0.006944in}}%
\pgfpathcurveto{\pgfqpoint{0.005103in}{0.008786in}}{\pgfqpoint{0.002605in}{0.009821in}}{\pgfqpoint{0.000000in}{0.009821in}}%
\pgfpathcurveto{\pgfqpoint{-0.002605in}{0.009821in}}{\pgfqpoint{-0.005103in}{0.008786in}}{\pgfqpoint{-0.006944in}{0.006944in}}%
\pgfpathcurveto{\pgfqpoint{-0.008786in}{0.005103in}}{\pgfqpoint{-0.009821in}{0.002605in}}{\pgfqpoint{-0.009821in}{0.000000in}}%
\pgfpathcurveto{\pgfqpoint{-0.009821in}{-0.002605in}}{\pgfqpoint{-0.008786in}{-0.005103in}}{\pgfqpoint{-0.006944in}{-0.006944in}}%
\pgfpathcurveto{\pgfqpoint{-0.005103in}{-0.008786in}}{\pgfqpoint{-0.002605in}{-0.009821in}}{\pgfqpoint{0.000000in}{-0.009821in}}%
\pgfpathclose%
\pgfusepath{stroke,fill}%
}%
\begin{pgfscope}%
\pgfsys@transformshift{0.818198in}{0.823177in}%
\pgfsys@useobject{currentmarker}{}%
\end{pgfscope}%
\begin{pgfscope}%
\pgfsys@transformshift{0.963777in}{0.840950in}%
\pgfsys@useobject{currentmarker}{}%
\end{pgfscope}%
\begin{pgfscope}%
\pgfsys@transformshift{1.109356in}{0.872758in}%
\pgfsys@useobject{currentmarker}{}%
\end{pgfscope}%
\begin{pgfscope}%
\pgfsys@transformshift{1.254936in}{0.858014in}%
\pgfsys@useobject{currentmarker}{}%
\end{pgfscope}%
\begin{pgfscope}%
\pgfsys@transformshift{1.400515in}{0.973139in}%
\pgfsys@useobject{currentmarker}{}%
\end{pgfscope}%
\begin{pgfscope}%
\pgfsys@transformshift{1.546095in}{1.110069in}%
\pgfsys@useobject{currentmarker}{}%
\end{pgfscope}%
\begin{pgfscope}%
\pgfsys@transformshift{1.691674in}{1.192576in}%
\pgfsys@useobject{currentmarker}{}%
\end{pgfscope}%
\begin{pgfscope}%
\pgfsys@transformshift{1.837253in}{1.497286in}%
\pgfsys@useobject{currentmarker}{}%
\end{pgfscope}%
\begin{pgfscope}%
\pgfsys@transformshift{1.982833in}{1.486471in}%
\pgfsys@useobject{currentmarker}{}%
\end{pgfscope}%
\begin{pgfscope}%
\pgfsys@transformshift{2.128412in}{1.832282in}%
\pgfsys@useobject{currentmarker}{}%
\end{pgfscope}%
\begin{pgfscope}%
\pgfsys@transformshift{2.273992in}{2.123094in}%
\pgfsys@useobject{currentmarker}{}%
\end{pgfscope}%
\begin{pgfscope}%
\pgfsys@transformshift{2.419571in}{2.062192in}%
\pgfsys@useobject{currentmarker}{}%
\end{pgfscope}%
\begin{pgfscope}%
\pgfsys@transformshift{2.565150in}{1.825785in}%
\pgfsys@useobject{currentmarker}{}%
\end{pgfscope}%
\begin{pgfscope}%
\pgfsys@transformshift{2.710730in}{2.123137in}%
\pgfsys@useobject{currentmarker}{}%
\end{pgfscope}%
\begin{pgfscope}%
\pgfsys@transformshift{2.856309in}{2.123133in}%
\pgfsys@useobject{currentmarker}{}%
\end{pgfscope}%
\begin{pgfscope}%
\pgfsys@transformshift{3.001889in}{2.123119in}%
\pgfsys@useobject{currentmarker}{}%
\end{pgfscope}%
\begin{pgfscope}%
\pgfsys@transformshift{3.147468in}{2.123129in}%
\pgfsys@useobject{currentmarker}{}%
\end{pgfscope}%
\begin{pgfscope}%
\pgfsys@transformshift{3.293047in}{2.123131in}%
\pgfsys@useobject{currentmarker}{}%
\end{pgfscope}%
\begin{pgfscope}%
\pgfsys@transformshift{3.438627in}{2.123134in}%
\pgfsys@useobject{currentmarker}{}%
\end{pgfscope}%
\begin{pgfscope}%
\pgfsys@transformshift{3.584206in}{2.123137in}%
\pgfsys@useobject{currentmarker}{}%
\end{pgfscope}%
\begin{pgfscope}%
\pgfsys@transformshift{3.729786in}{2.123138in}%
\pgfsys@useobject{currentmarker}{}%
\end{pgfscope}%
\end{pgfscope}%
\begin{pgfscope}%
\pgfpathrectangle{\pgfqpoint{0.672569in}{0.438889in}}{\pgfqpoint{3.203819in}{2.263889in}}%
\pgfusepath{clip}%
\pgfsetbuttcap%
\pgfsetroundjoin%
\definecolor{currentfill}{rgb}{1.000000,0.498039,0.054902}%
\pgfsetfillcolor{currentfill}%
\pgfsetfillopacity{0.300000}%
\pgfsetlinewidth{1.003750pt}%
\definecolor{currentstroke}{rgb}{1.000000,0.498039,0.054902}%
\pgfsetstrokecolor{currentstroke}%
\pgfsetstrokeopacity{0.300000}%
\pgfsetdash{}{0pt}%
\pgfpathmoveto{\pgfqpoint{0.818198in}{0.859656in}}%
\pgfpathlineto{\pgfqpoint{0.818198in}{0.786697in}}%
\pgfpathlineto{\pgfqpoint{0.963777in}{0.806402in}}%
\pgfpathlineto{\pgfqpoint{1.109356in}{0.811055in}}%
\pgfpathlineto{\pgfqpoint{1.254936in}{0.807811in}}%
\pgfpathlineto{\pgfqpoint{1.400515in}{0.914822in}}%
\pgfpathlineto{\pgfqpoint{1.546095in}{0.630252in}}%
\pgfpathlineto{\pgfqpoint{1.691674in}{0.972250in}}%
\pgfpathlineto{\pgfqpoint{1.837253in}{0.893148in}}%
\pgfpathlineto{\pgfqpoint{1.982833in}{0.627907in}}%
\pgfpathlineto{\pgfqpoint{2.128412in}{1.076855in}}%
\pgfpathlineto{\pgfqpoint{2.273992in}{2.122977in}}%
\pgfpathlineto{\pgfqpoint{2.419571in}{1.903861in}}%
\pgfpathlineto{\pgfqpoint{2.565150in}{1.053370in}}%
\pgfpathlineto{\pgfqpoint{2.710730in}{2.123114in}}%
\pgfpathlineto{\pgfqpoint{2.856309in}{2.123094in}}%
\pgfpathlineto{\pgfqpoint{3.001889in}{2.123096in}}%
\pgfpathlineto{\pgfqpoint{3.147468in}{2.123115in}}%
\pgfpathlineto{\pgfqpoint{3.293047in}{2.123123in}}%
\pgfpathlineto{\pgfqpoint{3.438627in}{2.123122in}}%
\pgfpathlineto{\pgfqpoint{3.584206in}{2.123130in}}%
\pgfpathlineto{\pgfqpoint{3.729786in}{2.123134in}}%
\pgfpathlineto{\pgfqpoint{3.729786in}{2.123141in}}%
\pgfpathlineto{\pgfqpoint{3.729786in}{2.123141in}}%
\pgfpathlineto{\pgfqpoint{3.584206in}{2.123145in}}%
\pgfpathlineto{\pgfqpoint{3.438627in}{2.123146in}}%
\pgfpathlineto{\pgfqpoint{3.293047in}{2.123138in}}%
\pgfpathlineto{\pgfqpoint{3.147468in}{2.123143in}}%
\pgfpathlineto{\pgfqpoint{3.001889in}{2.123143in}}%
\pgfpathlineto{\pgfqpoint{2.856309in}{2.123172in}}%
\pgfpathlineto{\pgfqpoint{2.710730in}{2.123160in}}%
\pgfpathlineto{\pgfqpoint{2.565150in}{2.598201in}}%
\pgfpathlineto{\pgfqpoint{2.419571in}{2.220522in}}%
\pgfpathlineto{\pgfqpoint{2.273992in}{2.123210in}}%
\pgfpathlineto{\pgfqpoint{2.128412in}{2.587709in}}%
\pgfpathlineto{\pgfqpoint{1.982833in}{2.345035in}}%
\pgfpathlineto{\pgfqpoint{1.837253in}{2.101423in}}%
\pgfpathlineto{\pgfqpoint{1.691674in}{1.412902in}}%
\pgfpathlineto{\pgfqpoint{1.546095in}{1.589887in}}%
\pgfpathlineto{\pgfqpoint{1.400515in}{1.031456in}}%
\pgfpathlineto{\pgfqpoint{1.254936in}{0.908217in}}%
\pgfpathlineto{\pgfqpoint{1.109356in}{0.934462in}}%
\pgfpathlineto{\pgfqpoint{0.963777in}{0.875499in}}%
\pgfpathlineto{\pgfqpoint{0.818198in}{0.859656in}}%
\pgfpathclose%
\pgfusepath{stroke,fill}%
\end{pgfscope}%
\begin{pgfscope}%
\pgfpathrectangle{\pgfqpoint{0.672569in}{0.438889in}}{\pgfqpoint{3.203819in}{2.263889in}}%
\pgfusepath{clip}%
\pgfsetbuttcap%
\pgfsetroundjoin%
\definecolor{currentfill}{rgb}{0.172549,0.627451,0.172549}%
\pgfsetfillcolor{currentfill}%
\pgfsetlinewidth{1.003750pt}%
\definecolor{currentstroke}{rgb}{0.172549,0.627451,0.172549}%
\pgfsetstrokecolor{currentstroke}%
\pgfsetdash{}{0pt}%
\pgfsys@defobject{currentmarker}{\pgfqpoint{-0.009821in}{-0.009821in}}{\pgfqpoint{0.009821in}{0.009821in}}{%
\pgfpathmoveto{\pgfqpoint{0.000000in}{-0.009821in}}%
\pgfpathcurveto{\pgfqpoint{0.002605in}{-0.009821in}}{\pgfqpoint{0.005103in}{-0.008786in}}{\pgfqpoint{0.006944in}{-0.006944in}}%
\pgfpathcurveto{\pgfqpoint{0.008786in}{-0.005103in}}{\pgfqpoint{0.009821in}{-0.002605in}}{\pgfqpoint{0.009821in}{0.000000in}}%
\pgfpathcurveto{\pgfqpoint{0.009821in}{0.002605in}}{\pgfqpoint{0.008786in}{0.005103in}}{\pgfqpoint{0.006944in}{0.006944in}}%
\pgfpathcurveto{\pgfqpoint{0.005103in}{0.008786in}}{\pgfqpoint{0.002605in}{0.009821in}}{\pgfqpoint{0.000000in}{0.009821in}}%
\pgfpathcurveto{\pgfqpoint{-0.002605in}{0.009821in}}{\pgfqpoint{-0.005103in}{0.008786in}}{\pgfqpoint{-0.006944in}{0.006944in}}%
\pgfpathcurveto{\pgfqpoint{-0.008786in}{0.005103in}}{\pgfqpoint{-0.009821in}{0.002605in}}{\pgfqpoint{-0.009821in}{0.000000in}}%
\pgfpathcurveto{\pgfqpoint{-0.009821in}{-0.002605in}}{\pgfqpoint{-0.008786in}{-0.005103in}}{\pgfqpoint{-0.006944in}{-0.006944in}}%
\pgfpathcurveto{\pgfqpoint{-0.005103in}{-0.008786in}}{\pgfqpoint{-0.002605in}{-0.009821in}}{\pgfqpoint{0.000000in}{-0.009821in}}%
\pgfpathclose%
\pgfusepath{stroke,fill}%
}%
\begin{pgfscope}%
\pgfsys@transformshift{0.818198in}{0.801687in}%
\pgfsys@useobject{currentmarker}{}%
\end{pgfscope}%
\begin{pgfscope}%
\pgfsys@transformshift{0.963777in}{0.842016in}%
\pgfsys@useobject{currentmarker}{}%
\end{pgfscope}%
\begin{pgfscope}%
\pgfsys@transformshift{1.109356in}{0.879562in}%
\pgfsys@useobject{currentmarker}{}%
\end{pgfscope}%
\begin{pgfscope}%
\pgfsys@transformshift{1.254936in}{0.863607in}%
\pgfsys@useobject{currentmarker}{}%
\end{pgfscope}%
\begin{pgfscope}%
\pgfsys@transformshift{1.400515in}{0.956933in}%
\pgfsys@useobject{currentmarker}{}%
\end{pgfscope}%
\begin{pgfscope}%
\pgfsys@transformshift{1.546095in}{0.954558in}%
\pgfsys@useobject{currentmarker}{}%
\end{pgfscope}%
\begin{pgfscope}%
\pgfsys@transformshift{1.691674in}{0.958592in}%
\pgfsys@useobject{currentmarker}{}%
\end{pgfscope}%
\begin{pgfscope}%
\pgfsys@transformshift{1.837253in}{1.228369in}%
\pgfsys@useobject{currentmarker}{}%
\end{pgfscope}%
\begin{pgfscope}%
\pgfsys@transformshift{1.982833in}{1.063034in}%
\pgfsys@useobject{currentmarker}{}%
\end{pgfscope}%
\begin{pgfscope}%
\pgfsys@transformshift{2.128412in}{1.349350in}%
\pgfsys@useobject{currentmarker}{}%
\end{pgfscope}%
\begin{pgfscope}%
\pgfsys@transformshift{2.273992in}{1.447769in}%
\pgfsys@useobject{currentmarker}{}%
\end{pgfscope}%
\begin{pgfscope}%
\pgfsys@transformshift{2.419571in}{2.018938in}%
\pgfsys@useobject{currentmarker}{}%
\end{pgfscope}%
\begin{pgfscope}%
\pgfsys@transformshift{2.565150in}{1.756945in}%
\pgfsys@useobject{currentmarker}{}%
\end{pgfscope}%
\begin{pgfscope}%
\pgfsys@transformshift{2.710730in}{2.021722in}%
\pgfsys@useobject{currentmarker}{}%
\end{pgfscope}%
\begin{pgfscope}%
\pgfsys@transformshift{2.856309in}{1.981424in}%
\pgfsys@useobject{currentmarker}{}%
\end{pgfscope}%
\begin{pgfscope}%
\pgfsys@transformshift{3.001889in}{2.020663in}%
\pgfsys@useobject{currentmarker}{}%
\end{pgfscope}%
\begin{pgfscope}%
\pgfsys@transformshift{3.147468in}{2.021108in}%
\pgfsys@useobject{currentmarker}{}%
\end{pgfscope}%
\begin{pgfscope}%
\pgfsys@transformshift{3.293047in}{2.021179in}%
\pgfsys@useobject{currentmarker}{}%
\end{pgfscope}%
\begin{pgfscope}%
\pgfsys@transformshift{3.438627in}{2.021153in}%
\pgfsys@useobject{currentmarker}{}%
\end{pgfscope}%
\begin{pgfscope}%
\pgfsys@transformshift{3.584206in}{2.021047in}%
\pgfsys@useobject{currentmarker}{}%
\end{pgfscope}%
\begin{pgfscope}%
\pgfsys@transformshift{3.729786in}{2.021076in}%
\pgfsys@useobject{currentmarker}{}%
\end{pgfscope}%
\end{pgfscope}%
\begin{pgfscope}%
\pgfpathrectangle{\pgfqpoint{0.672569in}{0.438889in}}{\pgfqpoint{3.203819in}{2.263889in}}%
\pgfusepath{clip}%
\pgfsetbuttcap%
\pgfsetroundjoin%
\definecolor{currentfill}{rgb}{0.172549,0.627451,0.172549}%
\pgfsetfillcolor{currentfill}%
\pgfsetfillopacity{0.300000}%
\pgfsetlinewidth{1.003750pt}%
\definecolor{currentstroke}{rgb}{0.172549,0.627451,0.172549}%
\pgfsetstrokecolor{currentstroke}%
\pgfsetstrokeopacity{0.300000}%
\pgfsetdash{}{0pt}%
\pgfpathmoveto{\pgfqpoint{0.818198in}{0.818608in}}%
\pgfpathlineto{\pgfqpoint{0.818198in}{0.784765in}}%
\pgfpathlineto{\pgfqpoint{0.963777in}{0.786594in}}%
\pgfpathlineto{\pgfqpoint{1.109356in}{0.814145in}}%
\pgfpathlineto{\pgfqpoint{1.254936in}{0.846879in}}%
\pgfpathlineto{\pgfqpoint{1.400515in}{0.902663in}}%
\pgfpathlineto{\pgfqpoint{1.546095in}{0.768727in}}%
\pgfpathlineto{\pgfqpoint{1.691674in}{0.934387in}}%
\pgfpathlineto{\pgfqpoint{1.837253in}{0.667576in}}%
\pgfpathlineto{\pgfqpoint{1.982833in}{0.701766in}}%
\pgfpathlineto{\pgfqpoint{2.128412in}{0.869388in}}%
\pgfpathlineto{\pgfqpoint{2.273992in}{1.141560in}}%
\pgfpathlineto{\pgfqpoint{2.419571in}{2.013707in}}%
\pgfpathlineto{\pgfqpoint{2.565150in}{1.067748in}}%
\pgfpathlineto{\pgfqpoint{2.710730in}{2.021063in}}%
\pgfpathlineto{\pgfqpoint{2.856309in}{1.877306in}}%
\pgfpathlineto{\pgfqpoint{3.001889in}{2.020347in}}%
\pgfpathlineto{\pgfqpoint{3.147468in}{2.020433in}}%
\pgfpathlineto{\pgfqpoint{3.293047in}{2.020684in}}%
\pgfpathlineto{\pgfqpoint{3.438627in}{2.021009in}}%
\pgfpathlineto{\pgfqpoint{3.584206in}{2.020770in}}%
\pgfpathlineto{\pgfqpoint{3.729786in}{2.021010in}}%
\pgfpathlineto{\pgfqpoint{3.729786in}{2.021142in}}%
\pgfpathlineto{\pgfqpoint{3.729786in}{2.021142in}}%
\pgfpathlineto{\pgfqpoint{3.584206in}{2.021325in}}%
\pgfpathlineto{\pgfqpoint{3.438627in}{2.021297in}}%
\pgfpathlineto{\pgfqpoint{3.293047in}{2.021675in}}%
\pgfpathlineto{\pgfqpoint{3.147468in}{2.021783in}}%
\pgfpathlineto{\pgfqpoint{3.001889in}{2.020979in}}%
\pgfpathlineto{\pgfqpoint{2.856309in}{2.085543in}}%
\pgfpathlineto{\pgfqpoint{2.710730in}{2.022381in}}%
\pgfpathlineto{\pgfqpoint{2.565150in}{2.446143in}}%
\pgfpathlineto{\pgfqpoint{2.419571in}{2.024169in}}%
\pgfpathlineto{\pgfqpoint{2.273992in}{1.753979in}}%
\pgfpathlineto{\pgfqpoint{2.128412in}{1.829311in}}%
\pgfpathlineto{\pgfqpoint{1.982833in}{1.424303in}}%
\pgfpathlineto{\pgfqpoint{1.837253in}{1.789162in}}%
\pgfpathlineto{\pgfqpoint{1.691674in}{0.982797in}}%
\pgfpathlineto{\pgfqpoint{1.546095in}{1.140388in}}%
\pgfpathlineto{\pgfqpoint{1.400515in}{1.011204in}}%
\pgfpathlineto{\pgfqpoint{1.254936in}{0.880334in}}%
\pgfpathlineto{\pgfqpoint{1.109356in}{0.944979in}}%
\pgfpathlineto{\pgfqpoint{0.963777in}{0.897439in}}%
\pgfpathlineto{\pgfqpoint{0.818198in}{0.818608in}}%
\pgfpathclose%
\pgfusepath{stroke,fill}%
\end{pgfscope}%
\begin{pgfscope}%
\pgfpathrectangle{\pgfqpoint{0.672569in}{0.438889in}}{\pgfqpoint{3.203819in}{2.263889in}}%
\pgfusepath{clip}%
\pgfsetbuttcap%
\pgfsetroundjoin%
\definecolor{currentfill}{rgb}{0.839216,0.152941,0.156863}%
\pgfsetfillcolor{currentfill}%
\pgfsetlinewidth{1.003750pt}%
\definecolor{currentstroke}{rgb}{0.839216,0.152941,0.156863}%
\pgfsetstrokecolor{currentstroke}%
\pgfsetdash{}{0pt}%
\pgfsys@defobject{currentmarker}{\pgfqpoint{-0.009821in}{-0.009821in}}{\pgfqpoint{0.009821in}{0.009821in}}{%
\pgfpathmoveto{\pgfqpoint{0.000000in}{-0.009821in}}%
\pgfpathcurveto{\pgfqpoint{0.002605in}{-0.009821in}}{\pgfqpoint{0.005103in}{-0.008786in}}{\pgfqpoint{0.006944in}{-0.006944in}}%
\pgfpathcurveto{\pgfqpoint{0.008786in}{-0.005103in}}{\pgfqpoint{0.009821in}{-0.002605in}}{\pgfqpoint{0.009821in}{0.000000in}}%
\pgfpathcurveto{\pgfqpoint{0.009821in}{0.002605in}}{\pgfqpoint{0.008786in}{0.005103in}}{\pgfqpoint{0.006944in}{0.006944in}}%
\pgfpathcurveto{\pgfqpoint{0.005103in}{0.008786in}}{\pgfqpoint{0.002605in}{0.009821in}}{\pgfqpoint{0.000000in}{0.009821in}}%
\pgfpathcurveto{\pgfqpoint{-0.002605in}{0.009821in}}{\pgfqpoint{-0.005103in}{0.008786in}}{\pgfqpoint{-0.006944in}{0.006944in}}%
\pgfpathcurveto{\pgfqpoint{-0.008786in}{0.005103in}}{\pgfqpoint{-0.009821in}{0.002605in}}{\pgfqpoint{-0.009821in}{0.000000in}}%
\pgfpathcurveto{\pgfqpoint{-0.009821in}{-0.002605in}}{\pgfqpoint{-0.008786in}{-0.005103in}}{\pgfqpoint{-0.006944in}{-0.006944in}}%
\pgfpathcurveto{\pgfqpoint{-0.005103in}{-0.008786in}}{\pgfqpoint{-0.002605in}{-0.009821in}}{\pgfqpoint{0.000000in}{-0.009821in}}%
\pgfpathclose%
\pgfusepath{stroke,fill}%
}%
\begin{pgfscope}%
\pgfsys@transformshift{0.818198in}{0.824772in}%
\pgfsys@useobject{currentmarker}{}%
\end{pgfscope}%
\begin{pgfscope}%
\pgfsys@transformshift{0.963777in}{0.843096in}%
\pgfsys@useobject{currentmarker}{}%
\end{pgfscope}%
\begin{pgfscope}%
\pgfsys@transformshift{1.109356in}{0.819285in}%
\pgfsys@useobject{currentmarker}{}%
\end{pgfscope}%
\begin{pgfscope}%
\pgfsys@transformshift{1.254936in}{0.840558in}%
\pgfsys@useobject{currentmarker}{}%
\end{pgfscope}%
\begin{pgfscope}%
\pgfsys@transformshift{1.400515in}{0.841659in}%
\pgfsys@useobject{currentmarker}{}%
\end{pgfscope}%
\begin{pgfscope}%
\pgfsys@transformshift{1.546095in}{0.891934in}%
\pgfsys@useobject{currentmarker}{}%
\end{pgfscope}%
\begin{pgfscope}%
\pgfsys@transformshift{1.691674in}{0.924812in}%
\pgfsys@useobject{currentmarker}{}%
\end{pgfscope}%
\begin{pgfscope}%
\pgfsys@transformshift{1.837253in}{0.943542in}%
\pgfsys@useobject{currentmarker}{}%
\end{pgfscope}%
\begin{pgfscope}%
\pgfsys@transformshift{1.982833in}{0.877138in}%
\pgfsys@useobject{currentmarker}{}%
\end{pgfscope}%
\begin{pgfscope}%
\pgfsys@transformshift{2.128412in}{0.936012in}%
\pgfsys@useobject{currentmarker}{}%
\end{pgfscope}%
\begin{pgfscope}%
\pgfsys@transformshift{2.273992in}{0.958085in}%
\pgfsys@useobject{currentmarker}{}%
\end{pgfscope}%
\begin{pgfscope}%
\pgfsys@transformshift{2.419571in}{0.928996in}%
\pgfsys@useobject{currentmarker}{}%
\end{pgfscope}%
\begin{pgfscope}%
\pgfsys@transformshift{2.565150in}{0.899330in}%
\pgfsys@useobject{currentmarker}{}%
\end{pgfscope}%
\begin{pgfscope}%
\pgfsys@transformshift{2.710730in}{0.918372in}%
\pgfsys@useobject{currentmarker}{}%
\end{pgfscope}%
\begin{pgfscope}%
\pgfsys@transformshift{2.856309in}{0.921586in}%
\pgfsys@useobject{currentmarker}{}%
\end{pgfscope}%
\begin{pgfscope}%
\pgfsys@transformshift{3.001889in}{0.924171in}%
\pgfsys@useobject{currentmarker}{}%
\end{pgfscope}%
\begin{pgfscope}%
\pgfsys@transformshift{3.147468in}{0.920232in}%
\pgfsys@useobject{currentmarker}{}%
\end{pgfscope}%
\begin{pgfscope}%
\pgfsys@transformshift{3.293047in}{0.922115in}%
\pgfsys@useobject{currentmarker}{}%
\end{pgfscope}%
\begin{pgfscope}%
\pgfsys@transformshift{3.438627in}{0.915186in}%
\pgfsys@useobject{currentmarker}{}%
\end{pgfscope}%
\begin{pgfscope}%
\pgfsys@transformshift{3.584206in}{0.919125in}%
\pgfsys@useobject{currentmarker}{}%
\end{pgfscope}%
\begin{pgfscope}%
\pgfsys@transformshift{3.729786in}{0.919441in}%
\pgfsys@useobject{currentmarker}{}%
\end{pgfscope}%
\end{pgfscope}%
\begin{pgfscope}%
\pgfpathrectangle{\pgfqpoint{0.672569in}{0.438889in}}{\pgfqpoint{3.203819in}{2.263889in}}%
\pgfusepath{clip}%
\pgfsetbuttcap%
\pgfsetroundjoin%
\definecolor{currentfill}{rgb}{0.839216,0.152941,0.156863}%
\pgfsetfillcolor{currentfill}%
\pgfsetfillopacity{0.300000}%
\pgfsetlinewidth{1.003750pt}%
\definecolor{currentstroke}{rgb}{0.839216,0.152941,0.156863}%
\pgfsetstrokecolor{currentstroke}%
\pgfsetstrokeopacity{0.300000}%
\pgfsetdash{}{0pt}%
\pgfpathmoveto{\pgfqpoint{0.818198in}{0.846963in}}%
\pgfpathlineto{\pgfqpoint{0.818198in}{0.802581in}}%
\pgfpathlineto{\pgfqpoint{0.963777in}{0.786175in}}%
\pgfpathlineto{\pgfqpoint{1.109356in}{0.793160in}}%
\pgfpathlineto{\pgfqpoint{1.254936in}{0.792266in}}%
\pgfpathlineto{\pgfqpoint{1.400515in}{0.818753in}}%
\pgfpathlineto{\pgfqpoint{1.546095in}{0.869597in}}%
\pgfpathlineto{\pgfqpoint{1.691674in}{0.805795in}}%
\pgfpathlineto{\pgfqpoint{1.837253in}{0.870132in}}%
\pgfpathlineto{\pgfqpoint{1.982833in}{0.820398in}}%
\pgfpathlineto{\pgfqpoint{2.128412in}{0.824237in}}%
\pgfpathlineto{\pgfqpoint{2.273992in}{0.877589in}}%
\pgfpathlineto{\pgfqpoint{2.419571in}{0.910305in}}%
\pgfpathlineto{\pgfqpoint{2.565150in}{0.890177in}}%
\pgfpathlineto{\pgfqpoint{2.710730in}{0.881104in}}%
\pgfpathlineto{\pgfqpoint{2.856309in}{0.907589in}}%
\pgfpathlineto{\pgfqpoint{3.001889in}{0.917097in}}%
\pgfpathlineto{\pgfqpoint{3.147468in}{0.905854in}}%
\pgfpathlineto{\pgfqpoint{3.293047in}{0.918192in}}%
\pgfpathlineto{\pgfqpoint{3.438627in}{0.907268in}}%
\pgfpathlineto{\pgfqpoint{3.584206in}{0.914431in}}%
\pgfpathlineto{\pgfqpoint{3.729786in}{0.913760in}}%
\pgfpathlineto{\pgfqpoint{3.729786in}{0.925121in}}%
\pgfpathlineto{\pgfqpoint{3.729786in}{0.925121in}}%
\pgfpathlineto{\pgfqpoint{3.584206in}{0.923819in}}%
\pgfpathlineto{\pgfqpoint{3.438627in}{0.923104in}}%
\pgfpathlineto{\pgfqpoint{3.293047in}{0.926037in}}%
\pgfpathlineto{\pgfqpoint{3.147468in}{0.934609in}}%
\pgfpathlineto{\pgfqpoint{3.001889in}{0.931244in}}%
\pgfpathlineto{\pgfqpoint{2.856309in}{0.935584in}}%
\pgfpathlineto{\pgfqpoint{2.710730in}{0.955639in}}%
\pgfpathlineto{\pgfqpoint{2.565150in}{0.908483in}}%
\pgfpathlineto{\pgfqpoint{2.419571in}{0.947686in}}%
\pgfpathlineto{\pgfqpoint{2.273992in}{1.038580in}}%
\pgfpathlineto{\pgfqpoint{2.128412in}{1.047787in}}%
\pgfpathlineto{\pgfqpoint{1.982833in}{0.933879in}}%
\pgfpathlineto{\pgfqpoint{1.837253in}{1.016952in}}%
\pgfpathlineto{\pgfqpoint{1.691674in}{1.043829in}}%
\pgfpathlineto{\pgfqpoint{1.546095in}{0.914271in}}%
\pgfpathlineto{\pgfqpoint{1.400515in}{0.864564in}}%
\pgfpathlineto{\pgfqpoint{1.254936in}{0.888850in}}%
\pgfpathlineto{\pgfqpoint{1.109356in}{0.845410in}}%
\pgfpathlineto{\pgfqpoint{0.963777in}{0.900016in}}%
\pgfpathlineto{\pgfqpoint{0.818198in}{0.846963in}}%
\pgfpathclose%
\pgfusepath{stroke,fill}%
\end{pgfscope}%
\begin{pgfscope}%
\pgfpathrectangle{\pgfqpoint{0.672569in}{0.438889in}}{\pgfqpoint{3.203819in}{2.263889in}}%
\pgfusepath{clip}%
\pgfsetbuttcap%
\pgfsetroundjoin%
\definecolor{currentfill}{rgb}{0.580392,0.403922,0.741176}%
\pgfsetfillcolor{currentfill}%
\pgfsetlinewidth{1.003750pt}%
\definecolor{currentstroke}{rgb}{0.580392,0.403922,0.741176}%
\pgfsetstrokecolor{currentstroke}%
\pgfsetdash{}{0pt}%
\pgfsys@defobject{currentmarker}{\pgfqpoint{-0.009821in}{-0.009821in}}{\pgfqpoint{0.009821in}{0.009821in}}{%
\pgfpathmoveto{\pgfqpoint{0.000000in}{-0.009821in}}%
\pgfpathcurveto{\pgfqpoint{0.002605in}{-0.009821in}}{\pgfqpoint{0.005103in}{-0.008786in}}{\pgfqpoint{0.006944in}{-0.006944in}}%
\pgfpathcurveto{\pgfqpoint{0.008786in}{-0.005103in}}{\pgfqpoint{0.009821in}{-0.002605in}}{\pgfqpoint{0.009821in}{0.000000in}}%
\pgfpathcurveto{\pgfqpoint{0.009821in}{0.002605in}}{\pgfqpoint{0.008786in}{0.005103in}}{\pgfqpoint{0.006944in}{0.006944in}}%
\pgfpathcurveto{\pgfqpoint{0.005103in}{0.008786in}}{\pgfqpoint{0.002605in}{0.009821in}}{\pgfqpoint{0.000000in}{0.009821in}}%
\pgfpathcurveto{\pgfqpoint{-0.002605in}{0.009821in}}{\pgfqpoint{-0.005103in}{0.008786in}}{\pgfqpoint{-0.006944in}{0.006944in}}%
\pgfpathcurveto{\pgfqpoint{-0.008786in}{0.005103in}}{\pgfqpoint{-0.009821in}{0.002605in}}{\pgfqpoint{-0.009821in}{0.000000in}}%
\pgfpathcurveto{\pgfqpoint{-0.009821in}{-0.002605in}}{\pgfqpoint{-0.008786in}{-0.005103in}}{\pgfqpoint{-0.006944in}{-0.006944in}}%
\pgfpathcurveto{\pgfqpoint{-0.005103in}{-0.008786in}}{\pgfqpoint{-0.002605in}{-0.009821in}}{\pgfqpoint{0.000000in}{-0.009821in}}%
\pgfpathclose%
\pgfusepath{stroke,fill}%
}%
\begin{pgfscope}%
\pgfsys@transformshift{0.818198in}{0.811714in}%
\pgfsys@useobject{currentmarker}{}%
\end{pgfscope}%
\begin{pgfscope}%
\pgfsys@transformshift{0.963777in}{0.826335in}%
\pgfsys@useobject{currentmarker}{}%
\end{pgfscope}%
\begin{pgfscope}%
\pgfsys@transformshift{1.109356in}{0.829172in}%
\pgfsys@useobject{currentmarker}{}%
\end{pgfscope}%
\begin{pgfscope}%
\pgfsys@transformshift{1.254936in}{0.814868in}%
\pgfsys@useobject{currentmarker}{}%
\end{pgfscope}%
\begin{pgfscope}%
\pgfsys@transformshift{1.400515in}{0.823229in}%
\pgfsys@useobject{currentmarker}{}%
\end{pgfscope}%
\begin{pgfscope}%
\pgfsys@transformshift{1.546095in}{0.836563in}%
\pgfsys@useobject{currentmarker}{}%
\end{pgfscope}%
\begin{pgfscope}%
\pgfsys@transformshift{1.691674in}{0.848082in}%
\pgfsys@useobject{currentmarker}{}%
\end{pgfscope}%
\begin{pgfscope}%
\pgfsys@transformshift{1.837253in}{0.837108in}%
\pgfsys@useobject{currentmarker}{}%
\end{pgfscope}%
\begin{pgfscope}%
\pgfsys@transformshift{1.982833in}{0.827765in}%
\pgfsys@useobject{currentmarker}{}%
\end{pgfscope}%
\begin{pgfscope}%
\pgfsys@transformshift{2.128412in}{0.838438in}%
\pgfsys@useobject{currentmarker}{}%
\end{pgfscope}%
\begin{pgfscope}%
\pgfsys@transformshift{2.273992in}{0.833169in}%
\pgfsys@useobject{currentmarker}{}%
\end{pgfscope}%
\begin{pgfscope}%
\pgfsys@transformshift{2.419571in}{0.842866in}%
\pgfsys@useobject{currentmarker}{}%
\end{pgfscope}%
\begin{pgfscope}%
\pgfsys@transformshift{2.565150in}{0.840421in}%
\pgfsys@useobject{currentmarker}{}%
\end{pgfscope}%
\begin{pgfscope}%
\pgfsys@transformshift{2.710730in}{0.842247in}%
\pgfsys@useobject{currentmarker}{}%
\end{pgfscope}%
\begin{pgfscope}%
\pgfsys@transformshift{2.856309in}{0.842537in}%
\pgfsys@useobject{currentmarker}{}%
\end{pgfscope}%
\begin{pgfscope}%
\pgfsys@transformshift{3.001889in}{0.844327in}%
\pgfsys@useobject{currentmarker}{}%
\end{pgfscope}%
\begin{pgfscope}%
\pgfsys@transformshift{3.147468in}{0.841436in}%
\pgfsys@useobject{currentmarker}{}%
\end{pgfscope}%
\begin{pgfscope}%
\pgfsys@transformshift{3.293047in}{0.842469in}%
\pgfsys@useobject{currentmarker}{}%
\end{pgfscope}%
\begin{pgfscope}%
\pgfsys@transformshift{3.438627in}{0.841353in}%
\pgfsys@useobject{currentmarker}{}%
\end{pgfscope}%
\begin{pgfscope}%
\pgfsys@transformshift{3.584206in}{0.840838in}%
\pgfsys@useobject{currentmarker}{}%
\end{pgfscope}%
\begin{pgfscope}%
\pgfsys@transformshift{3.729786in}{0.841038in}%
\pgfsys@useobject{currentmarker}{}%
\end{pgfscope}%
\end{pgfscope}%
\begin{pgfscope}%
\pgfpathrectangle{\pgfqpoint{0.672569in}{0.438889in}}{\pgfqpoint{3.203819in}{2.263889in}}%
\pgfusepath{clip}%
\pgfsetbuttcap%
\pgfsetroundjoin%
\definecolor{currentfill}{rgb}{0.580392,0.403922,0.741176}%
\pgfsetfillcolor{currentfill}%
\pgfsetfillopacity{0.300000}%
\pgfsetlinewidth{1.003750pt}%
\definecolor{currentstroke}{rgb}{0.580392,0.403922,0.741176}%
\pgfsetstrokecolor{currentstroke}%
\pgfsetstrokeopacity{0.300000}%
\pgfsetdash{}{0pt}%
\pgfpathmoveto{\pgfqpoint{0.818198in}{0.835417in}}%
\pgfpathlineto{\pgfqpoint{0.818198in}{0.788012in}}%
\pgfpathlineto{\pgfqpoint{0.963777in}{0.785924in}}%
\pgfpathlineto{\pgfqpoint{1.109356in}{0.804564in}}%
\pgfpathlineto{\pgfqpoint{1.254936in}{0.808039in}}%
\pgfpathlineto{\pgfqpoint{1.400515in}{0.803574in}}%
\pgfpathlineto{\pgfqpoint{1.546095in}{0.795840in}}%
\pgfpathlineto{\pgfqpoint{1.691674in}{0.811507in}}%
\pgfpathlineto{\pgfqpoint{1.837253in}{0.805030in}}%
\pgfpathlineto{\pgfqpoint{1.982833in}{0.810406in}}%
\pgfpathlineto{\pgfqpoint{2.128412in}{0.833589in}}%
\pgfpathlineto{\pgfqpoint{2.273992in}{0.831237in}}%
\pgfpathlineto{\pgfqpoint{2.419571in}{0.836603in}}%
\pgfpathlineto{\pgfqpoint{2.565150in}{0.835182in}}%
\pgfpathlineto{\pgfqpoint{2.710730in}{0.836882in}}%
\pgfpathlineto{\pgfqpoint{2.856309in}{0.837738in}}%
\pgfpathlineto{\pgfqpoint{3.001889in}{0.841074in}}%
\pgfpathlineto{\pgfqpoint{3.147468in}{0.839871in}}%
\pgfpathlineto{\pgfqpoint{3.293047in}{0.842120in}}%
\pgfpathlineto{\pgfqpoint{3.438627in}{0.840044in}}%
\pgfpathlineto{\pgfqpoint{3.584206in}{0.839632in}}%
\pgfpathlineto{\pgfqpoint{3.729786in}{0.840374in}}%
\pgfpathlineto{\pgfqpoint{3.729786in}{0.841703in}}%
\pgfpathlineto{\pgfqpoint{3.729786in}{0.841703in}}%
\pgfpathlineto{\pgfqpoint{3.584206in}{0.842044in}}%
\pgfpathlineto{\pgfqpoint{3.438627in}{0.842662in}}%
\pgfpathlineto{\pgfqpoint{3.293047in}{0.842819in}}%
\pgfpathlineto{\pgfqpoint{3.147468in}{0.843001in}}%
\pgfpathlineto{\pgfqpoint{3.001889in}{0.847579in}}%
\pgfpathlineto{\pgfqpoint{2.856309in}{0.847336in}}%
\pgfpathlineto{\pgfqpoint{2.710730in}{0.847611in}}%
\pgfpathlineto{\pgfqpoint{2.565150in}{0.845660in}}%
\pgfpathlineto{\pgfqpoint{2.419571in}{0.849128in}}%
\pgfpathlineto{\pgfqpoint{2.273992in}{0.835101in}}%
\pgfpathlineto{\pgfqpoint{2.128412in}{0.843288in}}%
\pgfpathlineto{\pgfqpoint{1.982833in}{0.845123in}}%
\pgfpathlineto{\pgfqpoint{1.837253in}{0.869185in}}%
\pgfpathlineto{\pgfqpoint{1.691674in}{0.884656in}}%
\pgfpathlineto{\pgfqpoint{1.546095in}{0.877286in}}%
\pgfpathlineto{\pgfqpoint{1.400515in}{0.842884in}}%
\pgfpathlineto{\pgfqpoint{1.254936in}{0.821696in}}%
\pgfpathlineto{\pgfqpoint{1.109356in}{0.853781in}}%
\pgfpathlineto{\pgfqpoint{0.963777in}{0.866746in}}%
\pgfpathlineto{\pgfqpoint{0.818198in}{0.835417in}}%
\pgfpathclose%
\pgfusepath{stroke,fill}%
\end{pgfscope}%
\begin{pgfscope}%
\pgfsetbuttcap%
\pgfsetroundjoin%
\definecolor{currentfill}{rgb}{0.000000,0.000000,0.000000}%
\pgfsetfillcolor{currentfill}%
\pgfsetlinewidth{0.803000pt}%
\definecolor{currentstroke}{rgb}{0.000000,0.000000,0.000000}%
\pgfsetstrokecolor{currentstroke}%
\pgfsetdash{}{0pt}%
\pgfsys@defobject{currentmarker}{\pgfqpoint{0.000000in}{-0.048611in}}{\pgfqpoint{0.000000in}{0.000000in}}{%
\pgfpathmoveto{\pgfqpoint{0.000000in}{0.000000in}}%
\pgfpathlineto{\pgfqpoint{0.000000in}{-0.048611in}}%
\pgfusepath{stroke,fill}%
}%
\begin{pgfscope}%
\pgfsys@transformshift{1.168788in}{0.438889in}%
\pgfsys@useobject{currentmarker}{}%
\end{pgfscope}%
\end{pgfscope}%
\begin{pgfscope}%
\definecolor{textcolor}{rgb}{0.000000,0.000000,0.000000}%
\pgfsetstrokecolor{textcolor}%
\pgfsetfillcolor{textcolor}%
\pgftext[x=1.168788in,y=0.341667in,,top]{\color{textcolor}\sffamily\fontsize{10.000000}{12.000000}\selectfont \(\displaystyle {10^{4}}\)}%
\end{pgfscope}%
\begin{pgfscope}%
\pgfsetbuttcap%
\pgfsetroundjoin%
\definecolor{currentfill}{rgb}{0.000000,0.000000,0.000000}%
\pgfsetfillcolor{currentfill}%
\pgfsetlinewidth{0.803000pt}%
\definecolor{currentstroke}{rgb}{0.000000,0.000000,0.000000}%
\pgfsetstrokecolor{currentstroke}%
\pgfsetdash{}{0pt}%
\pgfsys@defobject{currentmarker}{\pgfqpoint{0.000000in}{-0.048611in}}{\pgfqpoint{0.000000in}{0.000000in}}{%
\pgfpathmoveto{\pgfqpoint{0.000000in}{0.000000in}}%
\pgfpathlineto{\pgfqpoint{0.000000in}{-0.048611in}}%
\pgfusepath{stroke,fill}%
}%
\begin{pgfscope}%
\pgfsys@transformshift{1.751105in}{0.438889in}%
\pgfsys@useobject{currentmarker}{}%
\end{pgfscope}%
\end{pgfscope}%
\begin{pgfscope}%
\definecolor{textcolor}{rgb}{0.000000,0.000000,0.000000}%
\pgfsetstrokecolor{textcolor}%
\pgfsetfillcolor{textcolor}%
\pgftext[x=1.751105in,y=0.341667in,,top]{\color{textcolor}\sffamily\fontsize{10.000000}{12.000000}\selectfont \(\displaystyle {10^{5}}\)}%
\end{pgfscope}%
\begin{pgfscope}%
\pgfsetbuttcap%
\pgfsetroundjoin%
\definecolor{currentfill}{rgb}{0.000000,0.000000,0.000000}%
\pgfsetfillcolor{currentfill}%
\pgfsetlinewidth{0.803000pt}%
\definecolor{currentstroke}{rgb}{0.000000,0.000000,0.000000}%
\pgfsetstrokecolor{currentstroke}%
\pgfsetdash{}{0pt}%
\pgfsys@defobject{currentmarker}{\pgfqpoint{0.000000in}{-0.048611in}}{\pgfqpoint{0.000000in}{0.000000in}}{%
\pgfpathmoveto{\pgfqpoint{0.000000in}{0.000000in}}%
\pgfpathlineto{\pgfqpoint{0.000000in}{-0.048611in}}%
\pgfusepath{stroke,fill}%
}%
\begin{pgfscope}%
\pgfsys@transformshift{2.333423in}{0.438889in}%
\pgfsys@useobject{currentmarker}{}%
\end{pgfscope}%
\end{pgfscope}%
\begin{pgfscope}%
\definecolor{textcolor}{rgb}{0.000000,0.000000,0.000000}%
\pgfsetstrokecolor{textcolor}%
\pgfsetfillcolor{textcolor}%
\pgftext[x=2.333423in,y=0.341667in,,top]{\color{textcolor}\sffamily\fontsize{10.000000}{12.000000}\selectfont \(\displaystyle {10^{6}}\)}%
\end{pgfscope}%
\begin{pgfscope}%
\pgfsetbuttcap%
\pgfsetroundjoin%
\definecolor{currentfill}{rgb}{0.000000,0.000000,0.000000}%
\pgfsetfillcolor{currentfill}%
\pgfsetlinewidth{0.803000pt}%
\definecolor{currentstroke}{rgb}{0.000000,0.000000,0.000000}%
\pgfsetstrokecolor{currentstroke}%
\pgfsetdash{}{0pt}%
\pgfsys@defobject{currentmarker}{\pgfqpoint{0.000000in}{-0.048611in}}{\pgfqpoint{0.000000in}{0.000000in}}{%
\pgfpathmoveto{\pgfqpoint{0.000000in}{0.000000in}}%
\pgfpathlineto{\pgfqpoint{0.000000in}{-0.048611in}}%
\pgfusepath{stroke,fill}%
}%
\begin{pgfscope}%
\pgfsys@transformshift{2.915741in}{0.438889in}%
\pgfsys@useobject{currentmarker}{}%
\end{pgfscope}%
\end{pgfscope}%
\begin{pgfscope}%
\definecolor{textcolor}{rgb}{0.000000,0.000000,0.000000}%
\pgfsetstrokecolor{textcolor}%
\pgfsetfillcolor{textcolor}%
\pgftext[x=2.915741in,y=0.341667in,,top]{\color{textcolor}\sffamily\fontsize{10.000000}{12.000000}\selectfont \(\displaystyle {10^{7}}\)}%
\end{pgfscope}%
\begin{pgfscope}%
\pgfsetbuttcap%
\pgfsetroundjoin%
\definecolor{currentfill}{rgb}{0.000000,0.000000,0.000000}%
\pgfsetfillcolor{currentfill}%
\pgfsetlinewidth{0.803000pt}%
\definecolor{currentstroke}{rgb}{0.000000,0.000000,0.000000}%
\pgfsetstrokecolor{currentstroke}%
\pgfsetdash{}{0pt}%
\pgfsys@defobject{currentmarker}{\pgfqpoint{0.000000in}{-0.048611in}}{\pgfqpoint{0.000000in}{0.000000in}}{%
\pgfpathmoveto{\pgfqpoint{0.000000in}{0.000000in}}%
\pgfpathlineto{\pgfqpoint{0.000000in}{-0.048611in}}%
\pgfusepath{stroke,fill}%
}%
\begin{pgfscope}%
\pgfsys@transformshift{3.498058in}{0.438889in}%
\pgfsys@useobject{currentmarker}{}%
\end{pgfscope}%
\end{pgfscope}%
\begin{pgfscope}%
\definecolor{textcolor}{rgb}{0.000000,0.000000,0.000000}%
\pgfsetstrokecolor{textcolor}%
\pgfsetfillcolor{textcolor}%
\pgftext[x=3.498058in,y=0.341667in,,top]{\color{textcolor}\sffamily\fontsize{10.000000}{12.000000}\selectfont \(\displaystyle {10^{8}}\)}%
\end{pgfscope}%
\begin{pgfscope}%
\pgfsetbuttcap%
\pgfsetroundjoin%
\definecolor{currentfill}{rgb}{0.000000,0.000000,0.000000}%
\pgfsetfillcolor{currentfill}%
\pgfsetlinewidth{0.602250pt}%
\definecolor{currentstroke}{rgb}{0.000000,0.000000,0.000000}%
\pgfsetstrokecolor{currentstroke}%
\pgfsetdash{}{0pt}%
\pgfsys@defobject{currentmarker}{\pgfqpoint{0.000000in}{-0.027778in}}{\pgfqpoint{0.000000in}{0.000000in}}{%
\pgfpathmoveto{\pgfqpoint{0.000000in}{0.000000in}}%
\pgfpathlineto{\pgfqpoint{0.000000in}{-0.027778in}}%
\pgfusepath{stroke,fill}%
}%
\begin{pgfscope}%
\pgfsys@transformshift{0.761765in}{0.438889in}%
\pgfsys@useobject{currentmarker}{}%
\end{pgfscope}%
\end{pgfscope}%
\begin{pgfscope}%
\pgfsetbuttcap%
\pgfsetroundjoin%
\definecolor{currentfill}{rgb}{0.000000,0.000000,0.000000}%
\pgfsetfillcolor{currentfill}%
\pgfsetlinewidth{0.602250pt}%
\definecolor{currentstroke}{rgb}{0.000000,0.000000,0.000000}%
\pgfsetstrokecolor{currentstroke}%
\pgfsetdash{}{0pt}%
\pgfsys@defobject{currentmarker}{\pgfqpoint{0.000000in}{-0.027778in}}{\pgfqpoint{0.000000in}{0.000000in}}{%
\pgfpathmoveto{\pgfqpoint{0.000000in}{0.000000in}}%
\pgfpathlineto{\pgfqpoint{0.000000in}{-0.027778in}}%
\pgfusepath{stroke,fill}%
}%
\begin{pgfscope}%
\pgfsys@transformshift{0.864306in}{0.438889in}%
\pgfsys@useobject{currentmarker}{}%
\end{pgfscope}%
\end{pgfscope}%
\begin{pgfscope}%
\pgfsetbuttcap%
\pgfsetroundjoin%
\definecolor{currentfill}{rgb}{0.000000,0.000000,0.000000}%
\pgfsetfillcolor{currentfill}%
\pgfsetlinewidth{0.602250pt}%
\definecolor{currentstroke}{rgb}{0.000000,0.000000,0.000000}%
\pgfsetstrokecolor{currentstroke}%
\pgfsetdash{}{0pt}%
\pgfsys@defobject{currentmarker}{\pgfqpoint{0.000000in}{-0.027778in}}{\pgfqpoint{0.000000in}{0.000000in}}{%
\pgfpathmoveto{\pgfqpoint{0.000000in}{0.000000in}}%
\pgfpathlineto{\pgfqpoint{0.000000in}{-0.027778in}}%
\pgfusepath{stroke,fill}%
}%
\begin{pgfscope}%
\pgfsys@transformshift{0.937060in}{0.438889in}%
\pgfsys@useobject{currentmarker}{}%
\end{pgfscope}%
\end{pgfscope}%
\begin{pgfscope}%
\pgfsetbuttcap%
\pgfsetroundjoin%
\definecolor{currentfill}{rgb}{0.000000,0.000000,0.000000}%
\pgfsetfillcolor{currentfill}%
\pgfsetlinewidth{0.602250pt}%
\definecolor{currentstroke}{rgb}{0.000000,0.000000,0.000000}%
\pgfsetstrokecolor{currentstroke}%
\pgfsetdash{}{0pt}%
\pgfsys@defobject{currentmarker}{\pgfqpoint{0.000000in}{-0.027778in}}{\pgfqpoint{0.000000in}{0.000000in}}{%
\pgfpathmoveto{\pgfqpoint{0.000000in}{0.000000in}}%
\pgfpathlineto{\pgfqpoint{0.000000in}{-0.027778in}}%
\pgfusepath{stroke,fill}%
}%
\begin{pgfscope}%
\pgfsys@transformshift{0.993493in}{0.438889in}%
\pgfsys@useobject{currentmarker}{}%
\end{pgfscope}%
\end{pgfscope}%
\begin{pgfscope}%
\pgfsetbuttcap%
\pgfsetroundjoin%
\definecolor{currentfill}{rgb}{0.000000,0.000000,0.000000}%
\pgfsetfillcolor{currentfill}%
\pgfsetlinewidth{0.602250pt}%
\definecolor{currentstroke}{rgb}{0.000000,0.000000,0.000000}%
\pgfsetstrokecolor{currentstroke}%
\pgfsetdash{}{0pt}%
\pgfsys@defobject{currentmarker}{\pgfqpoint{0.000000in}{-0.027778in}}{\pgfqpoint{0.000000in}{0.000000in}}{%
\pgfpathmoveto{\pgfqpoint{0.000000in}{0.000000in}}%
\pgfpathlineto{\pgfqpoint{0.000000in}{-0.027778in}}%
\pgfusepath{stroke,fill}%
}%
\begin{pgfscope}%
\pgfsys@transformshift{1.039601in}{0.438889in}%
\pgfsys@useobject{currentmarker}{}%
\end{pgfscope}%
\end{pgfscope}%
\begin{pgfscope}%
\pgfsetbuttcap%
\pgfsetroundjoin%
\definecolor{currentfill}{rgb}{0.000000,0.000000,0.000000}%
\pgfsetfillcolor{currentfill}%
\pgfsetlinewidth{0.602250pt}%
\definecolor{currentstroke}{rgb}{0.000000,0.000000,0.000000}%
\pgfsetstrokecolor{currentstroke}%
\pgfsetdash{}{0pt}%
\pgfsys@defobject{currentmarker}{\pgfqpoint{0.000000in}{-0.027778in}}{\pgfqpoint{0.000000in}{0.000000in}}{%
\pgfpathmoveto{\pgfqpoint{0.000000in}{0.000000in}}%
\pgfpathlineto{\pgfqpoint{0.000000in}{-0.027778in}}%
\pgfusepath{stroke,fill}%
}%
\begin{pgfscope}%
\pgfsys@transformshift{1.078586in}{0.438889in}%
\pgfsys@useobject{currentmarker}{}%
\end{pgfscope}%
\end{pgfscope}%
\begin{pgfscope}%
\pgfsetbuttcap%
\pgfsetroundjoin%
\definecolor{currentfill}{rgb}{0.000000,0.000000,0.000000}%
\pgfsetfillcolor{currentfill}%
\pgfsetlinewidth{0.602250pt}%
\definecolor{currentstroke}{rgb}{0.000000,0.000000,0.000000}%
\pgfsetstrokecolor{currentstroke}%
\pgfsetdash{}{0pt}%
\pgfsys@defobject{currentmarker}{\pgfqpoint{0.000000in}{-0.027778in}}{\pgfqpoint{0.000000in}{0.000000in}}{%
\pgfpathmoveto{\pgfqpoint{0.000000in}{0.000000in}}%
\pgfpathlineto{\pgfqpoint{0.000000in}{-0.027778in}}%
\pgfusepath{stroke,fill}%
}%
\begin{pgfscope}%
\pgfsys@transformshift{1.112355in}{0.438889in}%
\pgfsys@useobject{currentmarker}{}%
\end{pgfscope}%
\end{pgfscope}%
\begin{pgfscope}%
\pgfsetbuttcap%
\pgfsetroundjoin%
\definecolor{currentfill}{rgb}{0.000000,0.000000,0.000000}%
\pgfsetfillcolor{currentfill}%
\pgfsetlinewidth{0.602250pt}%
\definecolor{currentstroke}{rgb}{0.000000,0.000000,0.000000}%
\pgfsetstrokecolor{currentstroke}%
\pgfsetdash{}{0pt}%
\pgfsys@defobject{currentmarker}{\pgfqpoint{0.000000in}{-0.027778in}}{\pgfqpoint{0.000000in}{0.000000in}}{%
\pgfpathmoveto{\pgfqpoint{0.000000in}{0.000000in}}%
\pgfpathlineto{\pgfqpoint{0.000000in}{-0.027778in}}%
\pgfusepath{stroke,fill}%
}%
\begin{pgfscope}%
\pgfsys@transformshift{1.142142in}{0.438889in}%
\pgfsys@useobject{currentmarker}{}%
\end{pgfscope}%
\end{pgfscope}%
\begin{pgfscope}%
\pgfsetbuttcap%
\pgfsetroundjoin%
\definecolor{currentfill}{rgb}{0.000000,0.000000,0.000000}%
\pgfsetfillcolor{currentfill}%
\pgfsetlinewidth{0.602250pt}%
\definecolor{currentstroke}{rgb}{0.000000,0.000000,0.000000}%
\pgfsetstrokecolor{currentstroke}%
\pgfsetdash{}{0pt}%
\pgfsys@defobject{currentmarker}{\pgfqpoint{0.000000in}{-0.027778in}}{\pgfqpoint{0.000000in}{0.000000in}}{%
\pgfpathmoveto{\pgfqpoint{0.000000in}{0.000000in}}%
\pgfpathlineto{\pgfqpoint{0.000000in}{-0.027778in}}%
\pgfusepath{stroke,fill}%
}%
\begin{pgfscope}%
\pgfsys@transformshift{1.344083in}{0.438889in}%
\pgfsys@useobject{currentmarker}{}%
\end{pgfscope}%
\end{pgfscope}%
\begin{pgfscope}%
\pgfsetbuttcap%
\pgfsetroundjoin%
\definecolor{currentfill}{rgb}{0.000000,0.000000,0.000000}%
\pgfsetfillcolor{currentfill}%
\pgfsetlinewidth{0.602250pt}%
\definecolor{currentstroke}{rgb}{0.000000,0.000000,0.000000}%
\pgfsetstrokecolor{currentstroke}%
\pgfsetdash{}{0pt}%
\pgfsys@defobject{currentmarker}{\pgfqpoint{0.000000in}{-0.027778in}}{\pgfqpoint{0.000000in}{0.000000in}}{%
\pgfpathmoveto{\pgfqpoint{0.000000in}{0.000000in}}%
\pgfpathlineto{\pgfqpoint{0.000000in}{-0.027778in}}%
\pgfusepath{stroke,fill}%
}%
\begin{pgfscope}%
\pgfsys@transformshift{1.446624in}{0.438889in}%
\pgfsys@useobject{currentmarker}{}%
\end{pgfscope}%
\end{pgfscope}%
\begin{pgfscope}%
\pgfsetbuttcap%
\pgfsetroundjoin%
\definecolor{currentfill}{rgb}{0.000000,0.000000,0.000000}%
\pgfsetfillcolor{currentfill}%
\pgfsetlinewidth{0.602250pt}%
\definecolor{currentstroke}{rgb}{0.000000,0.000000,0.000000}%
\pgfsetstrokecolor{currentstroke}%
\pgfsetdash{}{0pt}%
\pgfsys@defobject{currentmarker}{\pgfqpoint{0.000000in}{-0.027778in}}{\pgfqpoint{0.000000in}{0.000000in}}{%
\pgfpathmoveto{\pgfqpoint{0.000000in}{0.000000in}}%
\pgfpathlineto{\pgfqpoint{0.000000in}{-0.027778in}}%
\pgfusepath{stroke,fill}%
}%
\begin{pgfscope}%
\pgfsys@transformshift{1.519378in}{0.438889in}%
\pgfsys@useobject{currentmarker}{}%
\end{pgfscope}%
\end{pgfscope}%
\begin{pgfscope}%
\pgfsetbuttcap%
\pgfsetroundjoin%
\definecolor{currentfill}{rgb}{0.000000,0.000000,0.000000}%
\pgfsetfillcolor{currentfill}%
\pgfsetlinewidth{0.602250pt}%
\definecolor{currentstroke}{rgb}{0.000000,0.000000,0.000000}%
\pgfsetstrokecolor{currentstroke}%
\pgfsetdash{}{0pt}%
\pgfsys@defobject{currentmarker}{\pgfqpoint{0.000000in}{-0.027778in}}{\pgfqpoint{0.000000in}{0.000000in}}{%
\pgfpathmoveto{\pgfqpoint{0.000000in}{0.000000in}}%
\pgfpathlineto{\pgfqpoint{0.000000in}{-0.027778in}}%
\pgfusepath{stroke,fill}%
}%
\begin{pgfscope}%
\pgfsys@transformshift{1.575810in}{0.438889in}%
\pgfsys@useobject{currentmarker}{}%
\end{pgfscope}%
\end{pgfscope}%
\begin{pgfscope}%
\pgfsetbuttcap%
\pgfsetroundjoin%
\definecolor{currentfill}{rgb}{0.000000,0.000000,0.000000}%
\pgfsetfillcolor{currentfill}%
\pgfsetlinewidth{0.602250pt}%
\definecolor{currentstroke}{rgb}{0.000000,0.000000,0.000000}%
\pgfsetstrokecolor{currentstroke}%
\pgfsetdash{}{0pt}%
\pgfsys@defobject{currentmarker}{\pgfqpoint{0.000000in}{-0.027778in}}{\pgfqpoint{0.000000in}{0.000000in}}{%
\pgfpathmoveto{\pgfqpoint{0.000000in}{0.000000in}}%
\pgfpathlineto{\pgfqpoint{0.000000in}{-0.027778in}}%
\pgfusepath{stroke,fill}%
}%
\begin{pgfscope}%
\pgfsys@transformshift{1.621919in}{0.438889in}%
\pgfsys@useobject{currentmarker}{}%
\end{pgfscope}%
\end{pgfscope}%
\begin{pgfscope}%
\pgfsetbuttcap%
\pgfsetroundjoin%
\definecolor{currentfill}{rgb}{0.000000,0.000000,0.000000}%
\pgfsetfillcolor{currentfill}%
\pgfsetlinewidth{0.602250pt}%
\definecolor{currentstroke}{rgb}{0.000000,0.000000,0.000000}%
\pgfsetstrokecolor{currentstroke}%
\pgfsetdash{}{0pt}%
\pgfsys@defobject{currentmarker}{\pgfqpoint{0.000000in}{-0.027778in}}{\pgfqpoint{0.000000in}{0.000000in}}{%
\pgfpathmoveto{\pgfqpoint{0.000000in}{0.000000in}}%
\pgfpathlineto{\pgfqpoint{0.000000in}{-0.027778in}}%
\pgfusepath{stroke,fill}%
}%
\begin{pgfscope}%
\pgfsys@transformshift{1.660903in}{0.438889in}%
\pgfsys@useobject{currentmarker}{}%
\end{pgfscope}%
\end{pgfscope}%
\begin{pgfscope}%
\pgfsetbuttcap%
\pgfsetroundjoin%
\definecolor{currentfill}{rgb}{0.000000,0.000000,0.000000}%
\pgfsetfillcolor{currentfill}%
\pgfsetlinewidth{0.602250pt}%
\definecolor{currentstroke}{rgb}{0.000000,0.000000,0.000000}%
\pgfsetstrokecolor{currentstroke}%
\pgfsetdash{}{0pt}%
\pgfsys@defobject{currentmarker}{\pgfqpoint{0.000000in}{-0.027778in}}{\pgfqpoint{0.000000in}{0.000000in}}{%
\pgfpathmoveto{\pgfqpoint{0.000000in}{0.000000in}}%
\pgfpathlineto{\pgfqpoint{0.000000in}{-0.027778in}}%
\pgfusepath{stroke,fill}%
}%
\begin{pgfscope}%
\pgfsys@transformshift{1.694673in}{0.438889in}%
\pgfsys@useobject{currentmarker}{}%
\end{pgfscope}%
\end{pgfscope}%
\begin{pgfscope}%
\pgfsetbuttcap%
\pgfsetroundjoin%
\definecolor{currentfill}{rgb}{0.000000,0.000000,0.000000}%
\pgfsetfillcolor{currentfill}%
\pgfsetlinewidth{0.602250pt}%
\definecolor{currentstroke}{rgb}{0.000000,0.000000,0.000000}%
\pgfsetstrokecolor{currentstroke}%
\pgfsetdash{}{0pt}%
\pgfsys@defobject{currentmarker}{\pgfqpoint{0.000000in}{-0.027778in}}{\pgfqpoint{0.000000in}{0.000000in}}{%
\pgfpathmoveto{\pgfqpoint{0.000000in}{0.000000in}}%
\pgfpathlineto{\pgfqpoint{0.000000in}{-0.027778in}}%
\pgfusepath{stroke,fill}%
}%
\begin{pgfscope}%
\pgfsys@transformshift{1.724460in}{0.438889in}%
\pgfsys@useobject{currentmarker}{}%
\end{pgfscope}%
\end{pgfscope}%
\begin{pgfscope}%
\pgfsetbuttcap%
\pgfsetroundjoin%
\definecolor{currentfill}{rgb}{0.000000,0.000000,0.000000}%
\pgfsetfillcolor{currentfill}%
\pgfsetlinewidth{0.602250pt}%
\definecolor{currentstroke}{rgb}{0.000000,0.000000,0.000000}%
\pgfsetstrokecolor{currentstroke}%
\pgfsetdash{}{0pt}%
\pgfsys@defobject{currentmarker}{\pgfqpoint{0.000000in}{-0.027778in}}{\pgfqpoint{0.000000in}{0.000000in}}{%
\pgfpathmoveto{\pgfqpoint{0.000000in}{0.000000in}}%
\pgfpathlineto{\pgfqpoint{0.000000in}{-0.027778in}}%
\pgfusepath{stroke,fill}%
}%
\begin{pgfscope}%
\pgfsys@transformshift{1.926400in}{0.438889in}%
\pgfsys@useobject{currentmarker}{}%
\end{pgfscope}%
\end{pgfscope}%
\begin{pgfscope}%
\pgfsetbuttcap%
\pgfsetroundjoin%
\definecolor{currentfill}{rgb}{0.000000,0.000000,0.000000}%
\pgfsetfillcolor{currentfill}%
\pgfsetlinewidth{0.602250pt}%
\definecolor{currentstroke}{rgb}{0.000000,0.000000,0.000000}%
\pgfsetstrokecolor{currentstroke}%
\pgfsetdash{}{0pt}%
\pgfsys@defobject{currentmarker}{\pgfqpoint{0.000000in}{-0.027778in}}{\pgfqpoint{0.000000in}{0.000000in}}{%
\pgfpathmoveto{\pgfqpoint{0.000000in}{0.000000in}}%
\pgfpathlineto{\pgfqpoint{0.000000in}{-0.027778in}}%
\pgfusepath{stroke,fill}%
}%
\begin{pgfscope}%
\pgfsys@transformshift{2.028941in}{0.438889in}%
\pgfsys@useobject{currentmarker}{}%
\end{pgfscope}%
\end{pgfscope}%
\begin{pgfscope}%
\pgfsetbuttcap%
\pgfsetroundjoin%
\definecolor{currentfill}{rgb}{0.000000,0.000000,0.000000}%
\pgfsetfillcolor{currentfill}%
\pgfsetlinewidth{0.602250pt}%
\definecolor{currentstroke}{rgb}{0.000000,0.000000,0.000000}%
\pgfsetstrokecolor{currentstroke}%
\pgfsetdash{}{0pt}%
\pgfsys@defobject{currentmarker}{\pgfqpoint{0.000000in}{-0.027778in}}{\pgfqpoint{0.000000in}{0.000000in}}{%
\pgfpathmoveto{\pgfqpoint{0.000000in}{0.000000in}}%
\pgfpathlineto{\pgfqpoint{0.000000in}{-0.027778in}}%
\pgfusepath{stroke,fill}%
}%
\begin{pgfscope}%
\pgfsys@transformshift{2.101696in}{0.438889in}%
\pgfsys@useobject{currentmarker}{}%
\end{pgfscope}%
\end{pgfscope}%
\begin{pgfscope}%
\pgfsetbuttcap%
\pgfsetroundjoin%
\definecolor{currentfill}{rgb}{0.000000,0.000000,0.000000}%
\pgfsetfillcolor{currentfill}%
\pgfsetlinewidth{0.602250pt}%
\definecolor{currentstroke}{rgb}{0.000000,0.000000,0.000000}%
\pgfsetstrokecolor{currentstroke}%
\pgfsetdash{}{0pt}%
\pgfsys@defobject{currentmarker}{\pgfqpoint{0.000000in}{-0.027778in}}{\pgfqpoint{0.000000in}{0.000000in}}{%
\pgfpathmoveto{\pgfqpoint{0.000000in}{0.000000in}}%
\pgfpathlineto{\pgfqpoint{0.000000in}{-0.027778in}}%
\pgfusepath{stroke,fill}%
}%
\begin{pgfscope}%
\pgfsys@transformshift{2.158128in}{0.438889in}%
\pgfsys@useobject{currentmarker}{}%
\end{pgfscope}%
\end{pgfscope}%
\begin{pgfscope}%
\pgfsetbuttcap%
\pgfsetroundjoin%
\definecolor{currentfill}{rgb}{0.000000,0.000000,0.000000}%
\pgfsetfillcolor{currentfill}%
\pgfsetlinewidth{0.602250pt}%
\definecolor{currentstroke}{rgb}{0.000000,0.000000,0.000000}%
\pgfsetstrokecolor{currentstroke}%
\pgfsetdash{}{0pt}%
\pgfsys@defobject{currentmarker}{\pgfqpoint{0.000000in}{-0.027778in}}{\pgfqpoint{0.000000in}{0.000000in}}{%
\pgfpathmoveto{\pgfqpoint{0.000000in}{0.000000in}}%
\pgfpathlineto{\pgfqpoint{0.000000in}{-0.027778in}}%
\pgfusepath{stroke,fill}%
}%
\begin{pgfscope}%
\pgfsys@transformshift{2.204237in}{0.438889in}%
\pgfsys@useobject{currentmarker}{}%
\end{pgfscope}%
\end{pgfscope}%
\begin{pgfscope}%
\pgfsetbuttcap%
\pgfsetroundjoin%
\definecolor{currentfill}{rgb}{0.000000,0.000000,0.000000}%
\pgfsetfillcolor{currentfill}%
\pgfsetlinewidth{0.602250pt}%
\definecolor{currentstroke}{rgb}{0.000000,0.000000,0.000000}%
\pgfsetstrokecolor{currentstroke}%
\pgfsetdash{}{0pt}%
\pgfsys@defobject{currentmarker}{\pgfqpoint{0.000000in}{-0.027778in}}{\pgfqpoint{0.000000in}{0.000000in}}{%
\pgfpathmoveto{\pgfqpoint{0.000000in}{0.000000in}}%
\pgfpathlineto{\pgfqpoint{0.000000in}{-0.027778in}}%
\pgfusepath{stroke,fill}%
}%
\begin{pgfscope}%
\pgfsys@transformshift{2.243221in}{0.438889in}%
\pgfsys@useobject{currentmarker}{}%
\end{pgfscope}%
\end{pgfscope}%
\begin{pgfscope}%
\pgfsetbuttcap%
\pgfsetroundjoin%
\definecolor{currentfill}{rgb}{0.000000,0.000000,0.000000}%
\pgfsetfillcolor{currentfill}%
\pgfsetlinewidth{0.602250pt}%
\definecolor{currentstroke}{rgb}{0.000000,0.000000,0.000000}%
\pgfsetstrokecolor{currentstroke}%
\pgfsetdash{}{0pt}%
\pgfsys@defobject{currentmarker}{\pgfqpoint{0.000000in}{-0.027778in}}{\pgfqpoint{0.000000in}{0.000000in}}{%
\pgfpathmoveto{\pgfqpoint{0.000000in}{0.000000in}}%
\pgfpathlineto{\pgfqpoint{0.000000in}{-0.027778in}}%
\pgfusepath{stroke,fill}%
}%
\begin{pgfscope}%
\pgfsys@transformshift{2.276991in}{0.438889in}%
\pgfsys@useobject{currentmarker}{}%
\end{pgfscope}%
\end{pgfscope}%
\begin{pgfscope}%
\pgfsetbuttcap%
\pgfsetroundjoin%
\definecolor{currentfill}{rgb}{0.000000,0.000000,0.000000}%
\pgfsetfillcolor{currentfill}%
\pgfsetlinewidth{0.602250pt}%
\definecolor{currentstroke}{rgb}{0.000000,0.000000,0.000000}%
\pgfsetstrokecolor{currentstroke}%
\pgfsetdash{}{0pt}%
\pgfsys@defobject{currentmarker}{\pgfqpoint{0.000000in}{-0.027778in}}{\pgfqpoint{0.000000in}{0.000000in}}{%
\pgfpathmoveto{\pgfqpoint{0.000000in}{0.000000in}}%
\pgfpathlineto{\pgfqpoint{0.000000in}{-0.027778in}}%
\pgfusepath{stroke,fill}%
}%
\begin{pgfscope}%
\pgfsys@transformshift{2.306778in}{0.438889in}%
\pgfsys@useobject{currentmarker}{}%
\end{pgfscope}%
\end{pgfscope}%
\begin{pgfscope}%
\pgfsetbuttcap%
\pgfsetroundjoin%
\definecolor{currentfill}{rgb}{0.000000,0.000000,0.000000}%
\pgfsetfillcolor{currentfill}%
\pgfsetlinewidth{0.602250pt}%
\definecolor{currentstroke}{rgb}{0.000000,0.000000,0.000000}%
\pgfsetstrokecolor{currentstroke}%
\pgfsetdash{}{0pt}%
\pgfsys@defobject{currentmarker}{\pgfqpoint{0.000000in}{-0.027778in}}{\pgfqpoint{0.000000in}{0.000000in}}{%
\pgfpathmoveto{\pgfqpoint{0.000000in}{0.000000in}}%
\pgfpathlineto{\pgfqpoint{0.000000in}{-0.027778in}}%
\pgfusepath{stroke,fill}%
}%
\begin{pgfscope}%
\pgfsys@transformshift{2.508718in}{0.438889in}%
\pgfsys@useobject{currentmarker}{}%
\end{pgfscope}%
\end{pgfscope}%
\begin{pgfscope}%
\pgfsetbuttcap%
\pgfsetroundjoin%
\definecolor{currentfill}{rgb}{0.000000,0.000000,0.000000}%
\pgfsetfillcolor{currentfill}%
\pgfsetlinewidth{0.602250pt}%
\definecolor{currentstroke}{rgb}{0.000000,0.000000,0.000000}%
\pgfsetstrokecolor{currentstroke}%
\pgfsetdash{}{0pt}%
\pgfsys@defobject{currentmarker}{\pgfqpoint{0.000000in}{-0.027778in}}{\pgfqpoint{0.000000in}{0.000000in}}{%
\pgfpathmoveto{\pgfqpoint{0.000000in}{0.000000in}}%
\pgfpathlineto{\pgfqpoint{0.000000in}{-0.027778in}}%
\pgfusepath{stroke,fill}%
}%
\begin{pgfscope}%
\pgfsys@transformshift{2.611259in}{0.438889in}%
\pgfsys@useobject{currentmarker}{}%
\end{pgfscope}%
\end{pgfscope}%
\begin{pgfscope}%
\pgfsetbuttcap%
\pgfsetroundjoin%
\definecolor{currentfill}{rgb}{0.000000,0.000000,0.000000}%
\pgfsetfillcolor{currentfill}%
\pgfsetlinewidth{0.602250pt}%
\definecolor{currentstroke}{rgb}{0.000000,0.000000,0.000000}%
\pgfsetstrokecolor{currentstroke}%
\pgfsetdash{}{0pt}%
\pgfsys@defobject{currentmarker}{\pgfqpoint{0.000000in}{-0.027778in}}{\pgfqpoint{0.000000in}{0.000000in}}{%
\pgfpathmoveto{\pgfqpoint{0.000000in}{0.000000in}}%
\pgfpathlineto{\pgfqpoint{0.000000in}{-0.027778in}}%
\pgfusepath{stroke,fill}%
}%
\begin{pgfscope}%
\pgfsys@transformshift{2.684013in}{0.438889in}%
\pgfsys@useobject{currentmarker}{}%
\end{pgfscope}%
\end{pgfscope}%
\begin{pgfscope}%
\pgfsetbuttcap%
\pgfsetroundjoin%
\definecolor{currentfill}{rgb}{0.000000,0.000000,0.000000}%
\pgfsetfillcolor{currentfill}%
\pgfsetlinewidth{0.602250pt}%
\definecolor{currentstroke}{rgb}{0.000000,0.000000,0.000000}%
\pgfsetstrokecolor{currentstroke}%
\pgfsetdash{}{0pt}%
\pgfsys@defobject{currentmarker}{\pgfqpoint{0.000000in}{-0.027778in}}{\pgfqpoint{0.000000in}{0.000000in}}{%
\pgfpathmoveto{\pgfqpoint{0.000000in}{0.000000in}}%
\pgfpathlineto{\pgfqpoint{0.000000in}{-0.027778in}}%
\pgfusepath{stroke,fill}%
}%
\begin{pgfscope}%
\pgfsys@transformshift{2.740446in}{0.438889in}%
\pgfsys@useobject{currentmarker}{}%
\end{pgfscope}%
\end{pgfscope}%
\begin{pgfscope}%
\pgfsetbuttcap%
\pgfsetroundjoin%
\definecolor{currentfill}{rgb}{0.000000,0.000000,0.000000}%
\pgfsetfillcolor{currentfill}%
\pgfsetlinewidth{0.602250pt}%
\definecolor{currentstroke}{rgb}{0.000000,0.000000,0.000000}%
\pgfsetstrokecolor{currentstroke}%
\pgfsetdash{}{0pt}%
\pgfsys@defobject{currentmarker}{\pgfqpoint{0.000000in}{-0.027778in}}{\pgfqpoint{0.000000in}{0.000000in}}{%
\pgfpathmoveto{\pgfqpoint{0.000000in}{0.000000in}}%
\pgfpathlineto{\pgfqpoint{0.000000in}{-0.027778in}}%
\pgfusepath{stroke,fill}%
}%
\begin{pgfscope}%
\pgfsys@transformshift{2.786554in}{0.438889in}%
\pgfsys@useobject{currentmarker}{}%
\end{pgfscope}%
\end{pgfscope}%
\begin{pgfscope}%
\pgfsetbuttcap%
\pgfsetroundjoin%
\definecolor{currentfill}{rgb}{0.000000,0.000000,0.000000}%
\pgfsetfillcolor{currentfill}%
\pgfsetlinewidth{0.602250pt}%
\definecolor{currentstroke}{rgb}{0.000000,0.000000,0.000000}%
\pgfsetstrokecolor{currentstroke}%
\pgfsetdash{}{0pt}%
\pgfsys@defobject{currentmarker}{\pgfqpoint{0.000000in}{-0.027778in}}{\pgfqpoint{0.000000in}{0.000000in}}{%
\pgfpathmoveto{\pgfqpoint{0.000000in}{0.000000in}}%
\pgfpathlineto{\pgfqpoint{0.000000in}{-0.027778in}}%
\pgfusepath{stroke,fill}%
}%
\begin{pgfscope}%
\pgfsys@transformshift{2.825538in}{0.438889in}%
\pgfsys@useobject{currentmarker}{}%
\end{pgfscope}%
\end{pgfscope}%
\begin{pgfscope}%
\pgfsetbuttcap%
\pgfsetroundjoin%
\definecolor{currentfill}{rgb}{0.000000,0.000000,0.000000}%
\pgfsetfillcolor{currentfill}%
\pgfsetlinewidth{0.602250pt}%
\definecolor{currentstroke}{rgb}{0.000000,0.000000,0.000000}%
\pgfsetstrokecolor{currentstroke}%
\pgfsetdash{}{0pt}%
\pgfsys@defobject{currentmarker}{\pgfqpoint{0.000000in}{-0.027778in}}{\pgfqpoint{0.000000in}{0.000000in}}{%
\pgfpathmoveto{\pgfqpoint{0.000000in}{0.000000in}}%
\pgfpathlineto{\pgfqpoint{0.000000in}{-0.027778in}}%
\pgfusepath{stroke,fill}%
}%
\begin{pgfscope}%
\pgfsys@transformshift{2.859308in}{0.438889in}%
\pgfsys@useobject{currentmarker}{}%
\end{pgfscope}%
\end{pgfscope}%
\begin{pgfscope}%
\pgfsetbuttcap%
\pgfsetroundjoin%
\definecolor{currentfill}{rgb}{0.000000,0.000000,0.000000}%
\pgfsetfillcolor{currentfill}%
\pgfsetlinewidth{0.602250pt}%
\definecolor{currentstroke}{rgb}{0.000000,0.000000,0.000000}%
\pgfsetstrokecolor{currentstroke}%
\pgfsetdash{}{0pt}%
\pgfsys@defobject{currentmarker}{\pgfqpoint{0.000000in}{-0.027778in}}{\pgfqpoint{0.000000in}{0.000000in}}{%
\pgfpathmoveto{\pgfqpoint{0.000000in}{0.000000in}}%
\pgfpathlineto{\pgfqpoint{0.000000in}{-0.027778in}}%
\pgfusepath{stroke,fill}%
}%
\begin{pgfscope}%
\pgfsys@transformshift{2.889095in}{0.438889in}%
\pgfsys@useobject{currentmarker}{}%
\end{pgfscope}%
\end{pgfscope}%
\begin{pgfscope}%
\pgfsetbuttcap%
\pgfsetroundjoin%
\definecolor{currentfill}{rgb}{0.000000,0.000000,0.000000}%
\pgfsetfillcolor{currentfill}%
\pgfsetlinewidth{0.602250pt}%
\definecolor{currentstroke}{rgb}{0.000000,0.000000,0.000000}%
\pgfsetstrokecolor{currentstroke}%
\pgfsetdash{}{0pt}%
\pgfsys@defobject{currentmarker}{\pgfqpoint{0.000000in}{-0.027778in}}{\pgfqpoint{0.000000in}{0.000000in}}{%
\pgfpathmoveto{\pgfqpoint{0.000000in}{0.000000in}}%
\pgfpathlineto{\pgfqpoint{0.000000in}{-0.027778in}}%
\pgfusepath{stroke,fill}%
}%
\begin{pgfscope}%
\pgfsys@transformshift{3.091036in}{0.438889in}%
\pgfsys@useobject{currentmarker}{}%
\end{pgfscope}%
\end{pgfscope}%
\begin{pgfscope}%
\pgfsetbuttcap%
\pgfsetroundjoin%
\definecolor{currentfill}{rgb}{0.000000,0.000000,0.000000}%
\pgfsetfillcolor{currentfill}%
\pgfsetlinewidth{0.602250pt}%
\definecolor{currentstroke}{rgb}{0.000000,0.000000,0.000000}%
\pgfsetstrokecolor{currentstroke}%
\pgfsetdash{}{0pt}%
\pgfsys@defobject{currentmarker}{\pgfqpoint{0.000000in}{-0.027778in}}{\pgfqpoint{0.000000in}{0.000000in}}{%
\pgfpathmoveto{\pgfqpoint{0.000000in}{0.000000in}}%
\pgfpathlineto{\pgfqpoint{0.000000in}{-0.027778in}}%
\pgfusepath{stroke,fill}%
}%
\begin{pgfscope}%
\pgfsys@transformshift{3.193577in}{0.438889in}%
\pgfsys@useobject{currentmarker}{}%
\end{pgfscope}%
\end{pgfscope}%
\begin{pgfscope}%
\pgfsetbuttcap%
\pgfsetroundjoin%
\definecolor{currentfill}{rgb}{0.000000,0.000000,0.000000}%
\pgfsetfillcolor{currentfill}%
\pgfsetlinewidth{0.602250pt}%
\definecolor{currentstroke}{rgb}{0.000000,0.000000,0.000000}%
\pgfsetstrokecolor{currentstroke}%
\pgfsetdash{}{0pt}%
\pgfsys@defobject{currentmarker}{\pgfqpoint{0.000000in}{-0.027778in}}{\pgfqpoint{0.000000in}{0.000000in}}{%
\pgfpathmoveto{\pgfqpoint{0.000000in}{0.000000in}}%
\pgfpathlineto{\pgfqpoint{0.000000in}{-0.027778in}}%
\pgfusepath{stroke,fill}%
}%
\begin{pgfscope}%
\pgfsys@transformshift{3.266331in}{0.438889in}%
\pgfsys@useobject{currentmarker}{}%
\end{pgfscope}%
\end{pgfscope}%
\begin{pgfscope}%
\pgfsetbuttcap%
\pgfsetroundjoin%
\definecolor{currentfill}{rgb}{0.000000,0.000000,0.000000}%
\pgfsetfillcolor{currentfill}%
\pgfsetlinewidth{0.602250pt}%
\definecolor{currentstroke}{rgb}{0.000000,0.000000,0.000000}%
\pgfsetstrokecolor{currentstroke}%
\pgfsetdash{}{0pt}%
\pgfsys@defobject{currentmarker}{\pgfqpoint{0.000000in}{-0.027778in}}{\pgfqpoint{0.000000in}{0.000000in}}{%
\pgfpathmoveto{\pgfqpoint{0.000000in}{0.000000in}}%
\pgfpathlineto{\pgfqpoint{0.000000in}{-0.027778in}}%
\pgfusepath{stroke,fill}%
}%
\begin{pgfscope}%
\pgfsys@transformshift{3.322763in}{0.438889in}%
\pgfsys@useobject{currentmarker}{}%
\end{pgfscope}%
\end{pgfscope}%
\begin{pgfscope}%
\pgfsetbuttcap%
\pgfsetroundjoin%
\definecolor{currentfill}{rgb}{0.000000,0.000000,0.000000}%
\pgfsetfillcolor{currentfill}%
\pgfsetlinewidth{0.602250pt}%
\definecolor{currentstroke}{rgb}{0.000000,0.000000,0.000000}%
\pgfsetstrokecolor{currentstroke}%
\pgfsetdash{}{0pt}%
\pgfsys@defobject{currentmarker}{\pgfqpoint{0.000000in}{-0.027778in}}{\pgfqpoint{0.000000in}{0.000000in}}{%
\pgfpathmoveto{\pgfqpoint{0.000000in}{0.000000in}}%
\pgfpathlineto{\pgfqpoint{0.000000in}{-0.027778in}}%
\pgfusepath{stroke,fill}%
}%
\begin{pgfscope}%
\pgfsys@transformshift{3.368872in}{0.438889in}%
\pgfsys@useobject{currentmarker}{}%
\end{pgfscope}%
\end{pgfscope}%
\begin{pgfscope}%
\pgfsetbuttcap%
\pgfsetroundjoin%
\definecolor{currentfill}{rgb}{0.000000,0.000000,0.000000}%
\pgfsetfillcolor{currentfill}%
\pgfsetlinewidth{0.602250pt}%
\definecolor{currentstroke}{rgb}{0.000000,0.000000,0.000000}%
\pgfsetstrokecolor{currentstroke}%
\pgfsetdash{}{0pt}%
\pgfsys@defobject{currentmarker}{\pgfqpoint{0.000000in}{-0.027778in}}{\pgfqpoint{0.000000in}{0.000000in}}{%
\pgfpathmoveto{\pgfqpoint{0.000000in}{0.000000in}}%
\pgfpathlineto{\pgfqpoint{0.000000in}{-0.027778in}}%
\pgfusepath{stroke,fill}%
}%
\begin{pgfscope}%
\pgfsys@transformshift{3.407856in}{0.438889in}%
\pgfsys@useobject{currentmarker}{}%
\end{pgfscope}%
\end{pgfscope}%
\begin{pgfscope}%
\pgfsetbuttcap%
\pgfsetroundjoin%
\definecolor{currentfill}{rgb}{0.000000,0.000000,0.000000}%
\pgfsetfillcolor{currentfill}%
\pgfsetlinewidth{0.602250pt}%
\definecolor{currentstroke}{rgb}{0.000000,0.000000,0.000000}%
\pgfsetstrokecolor{currentstroke}%
\pgfsetdash{}{0pt}%
\pgfsys@defobject{currentmarker}{\pgfqpoint{0.000000in}{-0.027778in}}{\pgfqpoint{0.000000in}{0.000000in}}{%
\pgfpathmoveto{\pgfqpoint{0.000000in}{0.000000in}}%
\pgfpathlineto{\pgfqpoint{0.000000in}{-0.027778in}}%
\pgfusepath{stroke,fill}%
}%
\begin{pgfscope}%
\pgfsys@transformshift{3.441626in}{0.438889in}%
\pgfsys@useobject{currentmarker}{}%
\end{pgfscope}%
\end{pgfscope}%
\begin{pgfscope}%
\pgfsetbuttcap%
\pgfsetroundjoin%
\definecolor{currentfill}{rgb}{0.000000,0.000000,0.000000}%
\pgfsetfillcolor{currentfill}%
\pgfsetlinewidth{0.602250pt}%
\definecolor{currentstroke}{rgb}{0.000000,0.000000,0.000000}%
\pgfsetstrokecolor{currentstroke}%
\pgfsetdash{}{0pt}%
\pgfsys@defobject{currentmarker}{\pgfqpoint{0.000000in}{-0.027778in}}{\pgfqpoint{0.000000in}{0.000000in}}{%
\pgfpathmoveto{\pgfqpoint{0.000000in}{0.000000in}}%
\pgfpathlineto{\pgfqpoint{0.000000in}{-0.027778in}}%
\pgfusepath{stroke,fill}%
}%
\begin{pgfscope}%
\pgfsys@transformshift{3.471413in}{0.438889in}%
\pgfsys@useobject{currentmarker}{}%
\end{pgfscope}%
\end{pgfscope}%
\begin{pgfscope}%
\pgfsetbuttcap%
\pgfsetroundjoin%
\definecolor{currentfill}{rgb}{0.000000,0.000000,0.000000}%
\pgfsetfillcolor{currentfill}%
\pgfsetlinewidth{0.602250pt}%
\definecolor{currentstroke}{rgb}{0.000000,0.000000,0.000000}%
\pgfsetstrokecolor{currentstroke}%
\pgfsetdash{}{0pt}%
\pgfsys@defobject{currentmarker}{\pgfqpoint{0.000000in}{-0.027778in}}{\pgfqpoint{0.000000in}{0.000000in}}{%
\pgfpathmoveto{\pgfqpoint{0.000000in}{0.000000in}}%
\pgfpathlineto{\pgfqpoint{0.000000in}{-0.027778in}}%
\pgfusepath{stroke,fill}%
}%
\begin{pgfscope}%
\pgfsys@transformshift{3.673353in}{0.438889in}%
\pgfsys@useobject{currentmarker}{}%
\end{pgfscope}%
\end{pgfscope}%
\begin{pgfscope}%
\pgfsetbuttcap%
\pgfsetroundjoin%
\definecolor{currentfill}{rgb}{0.000000,0.000000,0.000000}%
\pgfsetfillcolor{currentfill}%
\pgfsetlinewidth{0.602250pt}%
\definecolor{currentstroke}{rgb}{0.000000,0.000000,0.000000}%
\pgfsetstrokecolor{currentstroke}%
\pgfsetdash{}{0pt}%
\pgfsys@defobject{currentmarker}{\pgfqpoint{0.000000in}{-0.027778in}}{\pgfqpoint{0.000000in}{0.000000in}}{%
\pgfpathmoveto{\pgfqpoint{0.000000in}{0.000000in}}%
\pgfpathlineto{\pgfqpoint{0.000000in}{-0.027778in}}%
\pgfusepath{stroke,fill}%
}%
\begin{pgfscope}%
\pgfsys@transformshift{3.775894in}{0.438889in}%
\pgfsys@useobject{currentmarker}{}%
\end{pgfscope}%
\end{pgfscope}%
\begin{pgfscope}%
\pgfsetbuttcap%
\pgfsetroundjoin%
\definecolor{currentfill}{rgb}{0.000000,0.000000,0.000000}%
\pgfsetfillcolor{currentfill}%
\pgfsetlinewidth{0.602250pt}%
\definecolor{currentstroke}{rgb}{0.000000,0.000000,0.000000}%
\pgfsetstrokecolor{currentstroke}%
\pgfsetdash{}{0pt}%
\pgfsys@defobject{currentmarker}{\pgfqpoint{0.000000in}{-0.027778in}}{\pgfqpoint{0.000000in}{0.000000in}}{%
\pgfpathmoveto{\pgfqpoint{0.000000in}{0.000000in}}%
\pgfpathlineto{\pgfqpoint{0.000000in}{-0.027778in}}%
\pgfusepath{stroke,fill}%
}%
\begin{pgfscope}%
\pgfsys@transformshift{3.848648in}{0.438889in}%
\pgfsys@useobject{currentmarker}{}%
\end{pgfscope}%
\end{pgfscope}%
\begin{pgfscope}%
\pgfsetbuttcap%
\pgfsetroundjoin%
\definecolor{currentfill}{rgb}{0.000000,0.000000,0.000000}%
\pgfsetfillcolor{currentfill}%
\pgfsetlinewidth{0.803000pt}%
\definecolor{currentstroke}{rgb}{0.000000,0.000000,0.000000}%
\pgfsetstrokecolor{currentstroke}%
\pgfsetdash{}{0pt}%
\pgfsys@defobject{currentmarker}{\pgfqpoint{-0.048611in}{0.000000in}}{\pgfqpoint{0.000000in}{0.000000in}}{%
\pgfpathmoveto{\pgfqpoint{0.000000in}{0.000000in}}%
\pgfpathlineto{\pgfqpoint{-0.048611in}{0.000000in}}%
\pgfusepath{stroke,fill}%
}%
\begin{pgfscope}%
\pgfsys@transformshift{0.672569in}{0.444194in}%
\pgfsys@useobject{currentmarker}{}%
\end{pgfscope}%
\end{pgfscope}%
\begin{pgfscope}%
\definecolor{textcolor}{rgb}{0.000000,0.000000,0.000000}%
\pgfsetstrokecolor{textcolor}%
\pgfsetfillcolor{textcolor}%
\pgftext[x=0.149729in,y=0.391433in,left,base]{\color{textcolor}\sffamily\fontsize{10.000000}{12.000000}\selectfont −0.25}%
\end{pgfscope}%
\begin{pgfscope}%
\pgfsetbuttcap%
\pgfsetroundjoin%
\definecolor{currentfill}{rgb}{0.000000,0.000000,0.000000}%
\pgfsetfillcolor{currentfill}%
\pgfsetlinewidth{0.803000pt}%
\definecolor{currentstroke}{rgb}{0.000000,0.000000,0.000000}%
\pgfsetstrokecolor{currentstroke}%
\pgfsetdash{}{0pt}%
\pgfsys@defobject{currentmarker}{\pgfqpoint{-0.048611in}{0.000000in}}{\pgfqpoint{0.000000in}{0.000000in}}{%
\pgfpathmoveto{\pgfqpoint{0.000000in}{0.000000in}}%
\pgfpathlineto{\pgfqpoint{-0.048611in}{0.000000in}}%
\pgfusepath{stroke,fill}%
}%
\begin{pgfscope}%
\pgfsys@transformshift{0.672569in}{0.783650in}%
\pgfsys@useobject{currentmarker}{}%
\end{pgfscope}%
\end{pgfscope}%
\begin{pgfscope}%
\definecolor{textcolor}{rgb}{0.000000,0.000000,0.000000}%
\pgfsetstrokecolor{textcolor}%
\pgfsetfillcolor{textcolor}%
\pgftext[x=0.266102in,y=0.730888in,left,base]{\color{textcolor}\sffamily\fontsize{10.000000}{12.000000}\selectfont 0.00}%
\end{pgfscope}%
\begin{pgfscope}%
\pgfsetbuttcap%
\pgfsetroundjoin%
\definecolor{currentfill}{rgb}{0.000000,0.000000,0.000000}%
\pgfsetfillcolor{currentfill}%
\pgfsetlinewidth{0.803000pt}%
\definecolor{currentstroke}{rgb}{0.000000,0.000000,0.000000}%
\pgfsetstrokecolor{currentstroke}%
\pgfsetdash{}{0pt}%
\pgfsys@defobject{currentmarker}{\pgfqpoint{-0.048611in}{0.000000in}}{\pgfqpoint{0.000000in}{0.000000in}}{%
\pgfpathmoveto{\pgfqpoint{0.000000in}{0.000000in}}%
\pgfpathlineto{\pgfqpoint{-0.048611in}{0.000000in}}%
\pgfusepath{stroke,fill}%
}%
\begin{pgfscope}%
\pgfsys@transformshift{0.672569in}{1.123105in}%
\pgfsys@useobject{currentmarker}{}%
\end{pgfscope}%
\end{pgfscope}%
\begin{pgfscope}%
\definecolor{textcolor}{rgb}{0.000000,0.000000,0.000000}%
\pgfsetstrokecolor{textcolor}%
\pgfsetfillcolor{textcolor}%
\pgftext[x=0.266102in,y=1.070344in,left,base]{\color{textcolor}\sffamily\fontsize{10.000000}{12.000000}\selectfont 0.25}%
\end{pgfscope}%
\begin{pgfscope}%
\pgfsetbuttcap%
\pgfsetroundjoin%
\definecolor{currentfill}{rgb}{0.000000,0.000000,0.000000}%
\pgfsetfillcolor{currentfill}%
\pgfsetlinewidth{0.803000pt}%
\definecolor{currentstroke}{rgb}{0.000000,0.000000,0.000000}%
\pgfsetstrokecolor{currentstroke}%
\pgfsetdash{}{0pt}%
\pgfsys@defobject{currentmarker}{\pgfqpoint{-0.048611in}{0.000000in}}{\pgfqpoint{0.000000in}{0.000000in}}{%
\pgfpathmoveto{\pgfqpoint{0.000000in}{0.000000in}}%
\pgfpathlineto{\pgfqpoint{-0.048611in}{0.000000in}}%
\pgfusepath{stroke,fill}%
}%
\begin{pgfscope}%
\pgfsys@transformshift{0.672569in}{1.462561in}%
\pgfsys@useobject{currentmarker}{}%
\end{pgfscope}%
\end{pgfscope}%
\begin{pgfscope}%
\definecolor{textcolor}{rgb}{0.000000,0.000000,0.000000}%
\pgfsetstrokecolor{textcolor}%
\pgfsetfillcolor{textcolor}%
\pgftext[x=0.266102in,y=1.409799in,left,base]{\color{textcolor}\sffamily\fontsize{10.000000}{12.000000}\selectfont 0.50}%
\end{pgfscope}%
\begin{pgfscope}%
\pgfsetbuttcap%
\pgfsetroundjoin%
\definecolor{currentfill}{rgb}{0.000000,0.000000,0.000000}%
\pgfsetfillcolor{currentfill}%
\pgfsetlinewidth{0.803000pt}%
\definecolor{currentstroke}{rgb}{0.000000,0.000000,0.000000}%
\pgfsetstrokecolor{currentstroke}%
\pgfsetdash{}{0pt}%
\pgfsys@defobject{currentmarker}{\pgfqpoint{-0.048611in}{0.000000in}}{\pgfqpoint{0.000000in}{0.000000in}}{%
\pgfpathmoveto{\pgfqpoint{0.000000in}{0.000000in}}%
\pgfpathlineto{\pgfqpoint{-0.048611in}{0.000000in}}%
\pgfusepath{stroke,fill}%
}%
\begin{pgfscope}%
\pgfsys@transformshift{0.672569in}{1.802016in}%
\pgfsys@useobject{currentmarker}{}%
\end{pgfscope}%
\end{pgfscope}%
\begin{pgfscope}%
\definecolor{textcolor}{rgb}{0.000000,0.000000,0.000000}%
\pgfsetstrokecolor{textcolor}%
\pgfsetfillcolor{textcolor}%
\pgftext[x=0.266102in,y=1.749255in,left,base]{\color{textcolor}\sffamily\fontsize{10.000000}{12.000000}\selectfont 0.75}%
\end{pgfscope}%
\begin{pgfscope}%
\pgfsetbuttcap%
\pgfsetroundjoin%
\definecolor{currentfill}{rgb}{0.000000,0.000000,0.000000}%
\pgfsetfillcolor{currentfill}%
\pgfsetlinewidth{0.803000pt}%
\definecolor{currentstroke}{rgb}{0.000000,0.000000,0.000000}%
\pgfsetstrokecolor{currentstroke}%
\pgfsetdash{}{0pt}%
\pgfsys@defobject{currentmarker}{\pgfqpoint{-0.048611in}{0.000000in}}{\pgfqpoint{0.000000in}{0.000000in}}{%
\pgfpathmoveto{\pgfqpoint{0.000000in}{0.000000in}}%
\pgfpathlineto{\pgfqpoint{-0.048611in}{0.000000in}}%
\pgfusepath{stroke,fill}%
}%
\begin{pgfscope}%
\pgfsys@transformshift{0.672569in}{2.141472in}%
\pgfsys@useobject{currentmarker}{}%
\end{pgfscope}%
\end{pgfscope}%
\begin{pgfscope}%
\definecolor{textcolor}{rgb}{0.000000,0.000000,0.000000}%
\pgfsetstrokecolor{textcolor}%
\pgfsetfillcolor{textcolor}%
\pgftext[x=0.266102in,y=2.088710in,left,base]{\color{textcolor}\sffamily\fontsize{10.000000}{12.000000}\selectfont 1.00}%
\end{pgfscope}%
\begin{pgfscope}%
\pgfsetbuttcap%
\pgfsetroundjoin%
\definecolor{currentfill}{rgb}{0.000000,0.000000,0.000000}%
\pgfsetfillcolor{currentfill}%
\pgfsetlinewidth{0.803000pt}%
\definecolor{currentstroke}{rgb}{0.000000,0.000000,0.000000}%
\pgfsetstrokecolor{currentstroke}%
\pgfsetdash{}{0pt}%
\pgfsys@defobject{currentmarker}{\pgfqpoint{-0.048611in}{0.000000in}}{\pgfqpoint{0.000000in}{0.000000in}}{%
\pgfpathmoveto{\pgfqpoint{0.000000in}{0.000000in}}%
\pgfpathlineto{\pgfqpoint{-0.048611in}{0.000000in}}%
\pgfusepath{stroke,fill}%
}%
\begin{pgfscope}%
\pgfsys@transformshift{0.672569in}{2.480927in}%
\pgfsys@useobject{currentmarker}{}%
\end{pgfscope}%
\end{pgfscope}%
\begin{pgfscope}%
\definecolor{textcolor}{rgb}{0.000000,0.000000,0.000000}%
\pgfsetstrokecolor{textcolor}%
\pgfsetfillcolor{textcolor}%
\pgftext[x=0.266102in,y=2.428166in,left,base]{\color{textcolor}\sffamily\fontsize{10.000000}{12.000000}\selectfont 1.25}%
\end{pgfscope}%
\begin{pgfscope}%
\pgfpathrectangle{\pgfqpoint{0.672569in}{0.438889in}}{\pgfqpoint{3.203819in}{2.263889in}}%
\pgfusepath{clip}%
\pgfsetrectcap%
\pgfsetroundjoin%
\pgfsetlinewidth{1.505625pt}%
\definecolor{currentstroke}{rgb}{0.121569,0.466667,0.705882}%
\pgfsetstrokecolor{currentstroke}%
\pgfsetdash{}{0pt}%
\pgfpathmoveto{\pgfqpoint{0.818198in}{0.803175in}}%
\pgfpathlineto{\pgfqpoint{0.963777in}{0.834136in}}%
\pgfpathlineto{\pgfqpoint{1.109356in}{0.877730in}}%
\pgfpathlineto{\pgfqpoint{1.254936in}{0.842812in}}%
\pgfpathlineto{\pgfqpoint{1.400515in}{1.173191in}}%
\pgfpathlineto{\pgfqpoint{1.546095in}{1.401194in}}%
\pgfpathlineto{\pgfqpoint{1.691674in}{1.663434in}}%
\pgfpathlineto{\pgfqpoint{1.837253in}{1.865421in}}%
\pgfpathlineto{\pgfqpoint{1.982833in}{1.541521in}}%
\pgfpathlineto{\pgfqpoint{2.128412in}{1.501193in}}%
\pgfpathlineto{\pgfqpoint{2.273992in}{1.782931in}}%
\pgfpathlineto{\pgfqpoint{2.419571in}{1.528804in}}%
\pgfpathlineto{\pgfqpoint{2.565150in}{1.852812in}}%
\pgfpathlineto{\pgfqpoint{2.710730in}{2.134260in}}%
\pgfpathlineto{\pgfqpoint{2.856309in}{2.140356in}}%
\pgfpathlineto{\pgfqpoint{3.001889in}{2.140356in}}%
\pgfpathlineto{\pgfqpoint{3.147468in}{2.140356in}}%
\pgfpathlineto{\pgfqpoint{3.293047in}{2.140356in}}%
\pgfpathlineto{\pgfqpoint{3.438627in}{2.140356in}}%
\pgfpathlineto{\pgfqpoint{3.584206in}{2.140356in}}%
\pgfpathlineto{\pgfqpoint{3.729786in}{2.140356in}}%
\pgfusepath{stroke}%
\end{pgfscope}%
\begin{pgfscope}%
\pgfpathrectangle{\pgfqpoint{0.672569in}{0.438889in}}{\pgfqpoint{3.203819in}{2.263889in}}%
\pgfusepath{clip}%
\pgfsetrectcap%
\pgfsetroundjoin%
\pgfsetlinewidth{1.505625pt}%
\definecolor{currentstroke}{rgb}{1.000000,0.498039,0.054902}%
\pgfsetstrokecolor{currentstroke}%
\pgfsetdash{}{0pt}%
\pgfpathmoveto{\pgfqpoint{0.818198in}{0.823177in}}%
\pgfpathlineto{\pgfqpoint{0.963777in}{0.840950in}}%
\pgfpathlineto{\pgfqpoint{1.109356in}{0.872758in}}%
\pgfpathlineto{\pgfqpoint{1.254936in}{0.858014in}}%
\pgfpathlineto{\pgfqpoint{1.400515in}{0.973139in}}%
\pgfpathlineto{\pgfqpoint{1.546095in}{1.110069in}}%
\pgfpathlineto{\pgfqpoint{1.691674in}{1.192576in}}%
\pgfpathlineto{\pgfqpoint{1.837253in}{1.497286in}}%
\pgfpathlineto{\pgfqpoint{1.982833in}{1.486471in}}%
\pgfpathlineto{\pgfqpoint{2.128412in}{1.832282in}}%
\pgfpathlineto{\pgfqpoint{2.273992in}{2.123094in}}%
\pgfpathlineto{\pgfqpoint{2.419571in}{2.062192in}}%
\pgfpathlineto{\pgfqpoint{2.565150in}{1.825785in}}%
\pgfpathlineto{\pgfqpoint{2.710730in}{2.123137in}}%
\pgfpathlineto{\pgfqpoint{2.856309in}{2.123133in}}%
\pgfpathlineto{\pgfqpoint{3.001889in}{2.123119in}}%
\pgfpathlineto{\pgfqpoint{3.147468in}{2.123129in}}%
\pgfpathlineto{\pgfqpoint{3.293047in}{2.123131in}}%
\pgfpathlineto{\pgfqpoint{3.438627in}{2.123134in}}%
\pgfpathlineto{\pgfqpoint{3.584206in}{2.123137in}}%
\pgfpathlineto{\pgfqpoint{3.729786in}{2.123138in}}%
\pgfusepath{stroke}%
\end{pgfscope}%
\begin{pgfscope}%
\pgfpathrectangle{\pgfqpoint{0.672569in}{0.438889in}}{\pgfqpoint{3.203819in}{2.263889in}}%
\pgfusepath{clip}%
\pgfsetrectcap%
\pgfsetroundjoin%
\pgfsetlinewidth{1.505625pt}%
\definecolor{currentstroke}{rgb}{0.172549,0.627451,0.172549}%
\pgfsetstrokecolor{currentstroke}%
\pgfsetdash{}{0pt}%
\pgfpathmoveto{\pgfqpoint{0.818198in}{0.801687in}}%
\pgfpathlineto{\pgfqpoint{0.963777in}{0.842016in}}%
\pgfpathlineto{\pgfqpoint{1.109356in}{0.879562in}}%
\pgfpathlineto{\pgfqpoint{1.254936in}{0.863607in}}%
\pgfpathlineto{\pgfqpoint{1.400515in}{0.956933in}}%
\pgfpathlineto{\pgfqpoint{1.546095in}{0.954558in}}%
\pgfpathlineto{\pgfqpoint{1.691674in}{0.958592in}}%
\pgfpathlineto{\pgfqpoint{1.837253in}{1.228369in}}%
\pgfpathlineto{\pgfqpoint{1.982833in}{1.063034in}}%
\pgfpathlineto{\pgfqpoint{2.128412in}{1.349350in}}%
\pgfpathlineto{\pgfqpoint{2.273992in}{1.447769in}}%
\pgfpathlineto{\pgfqpoint{2.419571in}{2.018938in}}%
\pgfpathlineto{\pgfqpoint{2.565150in}{1.756945in}}%
\pgfpathlineto{\pgfqpoint{2.710730in}{2.021722in}}%
\pgfpathlineto{\pgfqpoint{2.856309in}{1.981424in}}%
\pgfpathlineto{\pgfqpoint{3.001889in}{2.020663in}}%
\pgfpathlineto{\pgfqpoint{3.147468in}{2.021108in}}%
\pgfpathlineto{\pgfqpoint{3.293047in}{2.021179in}}%
\pgfpathlineto{\pgfqpoint{3.438627in}{2.021153in}}%
\pgfpathlineto{\pgfqpoint{3.584206in}{2.021047in}}%
\pgfpathlineto{\pgfqpoint{3.729786in}{2.021076in}}%
\pgfusepath{stroke}%
\end{pgfscope}%
\begin{pgfscope}%
\pgfpathrectangle{\pgfqpoint{0.672569in}{0.438889in}}{\pgfqpoint{3.203819in}{2.263889in}}%
\pgfusepath{clip}%
\pgfsetrectcap%
\pgfsetroundjoin%
\pgfsetlinewidth{1.505625pt}%
\definecolor{currentstroke}{rgb}{0.839216,0.152941,0.156863}%
\pgfsetstrokecolor{currentstroke}%
\pgfsetdash{}{0pt}%
\pgfpathmoveto{\pgfqpoint{0.818198in}{0.824772in}}%
\pgfpathlineto{\pgfqpoint{0.963777in}{0.843096in}}%
\pgfpathlineto{\pgfqpoint{1.109356in}{0.819285in}}%
\pgfpathlineto{\pgfqpoint{1.254936in}{0.840558in}}%
\pgfpathlineto{\pgfqpoint{1.400515in}{0.841659in}}%
\pgfpathlineto{\pgfqpoint{1.546095in}{0.891934in}}%
\pgfpathlineto{\pgfqpoint{1.691674in}{0.924812in}}%
\pgfpathlineto{\pgfqpoint{1.837253in}{0.943542in}}%
\pgfpathlineto{\pgfqpoint{1.982833in}{0.877138in}}%
\pgfpathlineto{\pgfqpoint{2.128412in}{0.936012in}}%
\pgfpathlineto{\pgfqpoint{2.273992in}{0.958085in}}%
\pgfpathlineto{\pgfqpoint{2.419571in}{0.928996in}}%
\pgfpathlineto{\pgfqpoint{2.565150in}{0.899330in}}%
\pgfpathlineto{\pgfqpoint{2.710730in}{0.918372in}}%
\pgfpathlineto{\pgfqpoint{2.856309in}{0.921586in}}%
\pgfpathlineto{\pgfqpoint{3.001889in}{0.924171in}}%
\pgfpathlineto{\pgfqpoint{3.147468in}{0.920232in}}%
\pgfpathlineto{\pgfqpoint{3.293047in}{0.922115in}}%
\pgfpathlineto{\pgfqpoint{3.438627in}{0.915186in}}%
\pgfpathlineto{\pgfqpoint{3.584206in}{0.919125in}}%
\pgfpathlineto{\pgfqpoint{3.729786in}{0.919441in}}%
\pgfusepath{stroke}%
\end{pgfscope}%
\begin{pgfscope}%
\pgfpathrectangle{\pgfqpoint{0.672569in}{0.438889in}}{\pgfqpoint{3.203819in}{2.263889in}}%
\pgfusepath{clip}%
\pgfsetrectcap%
\pgfsetroundjoin%
\pgfsetlinewidth{1.505625pt}%
\definecolor{currentstroke}{rgb}{0.580392,0.403922,0.741176}%
\pgfsetstrokecolor{currentstroke}%
\pgfsetdash{}{0pt}%
\pgfpathmoveto{\pgfqpoint{0.818198in}{0.811714in}}%
\pgfpathlineto{\pgfqpoint{0.963777in}{0.826335in}}%
\pgfpathlineto{\pgfqpoint{1.109356in}{0.829172in}}%
\pgfpathlineto{\pgfqpoint{1.254936in}{0.814868in}}%
\pgfpathlineto{\pgfqpoint{1.400515in}{0.823229in}}%
\pgfpathlineto{\pgfqpoint{1.546095in}{0.836563in}}%
\pgfpathlineto{\pgfqpoint{1.691674in}{0.848082in}}%
\pgfpathlineto{\pgfqpoint{1.837253in}{0.837108in}}%
\pgfpathlineto{\pgfqpoint{1.982833in}{0.827765in}}%
\pgfpathlineto{\pgfqpoint{2.128412in}{0.838438in}}%
\pgfpathlineto{\pgfqpoint{2.273992in}{0.833169in}}%
\pgfpathlineto{\pgfqpoint{2.419571in}{0.842866in}}%
\pgfpathlineto{\pgfqpoint{2.565150in}{0.840421in}}%
\pgfpathlineto{\pgfqpoint{2.710730in}{0.842247in}}%
\pgfpathlineto{\pgfqpoint{2.856309in}{0.842537in}}%
\pgfpathlineto{\pgfqpoint{3.001889in}{0.844327in}}%
\pgfpathlineto{\pgfqpoint{3.147468in}{0.841436in}}%
\pgfpathlineto{\pgfqpoint{3.293047in}{0.842469in}}%
\pgfpathlineto{\pgfqpoint{3.438627in}{0.841353in}}%
\pgfpathlineto{\pgfqpoint{3.584206in}{0.840838in}}%
\pgfpathlineto{\pgfqpoint{3.729786in}{0.841038in}}%
\pgfusepath{stroke}%
\end{pgfscope}%
\begin{pgfscope}%
\pgfsetrectcap%
\pgfsetmiterjoin%
\pgfsetlinewidth{0.803000pt}%
\definecolor{currentstroke}{rgb}{0.000000,0.000000,0.000000}%
\pgfsetstrokecolor{currentstroke}%
\pgfsetdash{}{0pt}%
\pgfpathmoveto{\pgfqpoint{0.672569in}{0.438889in}}%
\pgfpathlineto{\pgfqpoint{0.672569in}{2.702778in}}%
\pgfusepath{stroke}%
\end{pgfscope}%
\begin{pgfscope}%
\pgfsetrectcap%
\pgfsetmiterjoin%
\pgfsetlinewidth{0.803000pt}%
\definecolor{currentstroke}{rgb}{0.000000,0.000000,0.000000}%
\pgfsetstrokecolor{currentstroke}%
\pgfsetdash{}{0pt}%
\pgfpathmoveto{\pgfqpoint{3.876389in}{0.438889in}}%
\pgfpathlineto{\pgfqpoint{3.876389in}{2.702778in}}%
\pgfusepath{stroke}%
\end{pgfscope}%
\begin{pgfscope}%
\pgfsetrectcap%
\pgfsetmiterjoin%
\pgfsetlinewidth{0.803000pt}%
\definecolor{currentstroke}{rgb}{0.000000,0.000000,0.000000}%
\pgfsetstrokecolor{currentstroke}%
\pgfsetdash{}{0pt}%
\pgfpathmoveto{\pgfqpoint{0.672569in}{0.438889in}}%
\pgfpathlineto{\pgfqpoint{3.876389in}{0.438889in}}%
\pgfusepath{stroke}%
\end{pgfscope}%
\begin{pgfscope}%
\pgfsetrectcap%
\pgfsetmiterjoin%
\pgfsetlinewidth{0.803000pt}%
\definecolor{currentstroke}{rgb}{0.000000,0.000000,0.000000}%
\pgfsetstrokecolor{currentstroke}%
\pgfsetdash{}{0pt}%
\pgfpathmoveto{\pgfqpoint{0.672569in}{2.702778in}}%
\pgfpathlineto{\pgfqpoint{3.876389in}{2.702778in}}%
\pgfusepath{stroke}%
\end{pgfscope}%
\begin{pgfscope}%
\definecolor{textcolor}{rgb}{0.000000,0.000000,0.000000}%
\pgfsetstrokecolor{textcolor}%
\pgfsetfillcolor{textcolor}%
\pgftext[x=2.274479in,y=2.786111in,,base]{\color{textcolor}\sffamily\fontsize{12.000000}{14.400000}\selectfont \(\displaystyle  N = 64 \)}%
\end{pgfscope}%
\begin{pgfscope}%
\pgfsetbuttcap%
\pgfsetmiterjoin%
\definecolor{currentfill}{rgb}{1.000000,1.000000,1.000000}%
\pgfsetfillcolor{currentfill}%
\pgfsetlinewidth{0.000000pt}%
\definecolor{currentstroke}{rgb}{0.000000,0.000000,0.000000}%
\pgfsetstrokecolor{currentstroke}%
\pgfsetstrokeopacity{0.000000}%
\pgfsetdash{}{0pt}%
\pgfpathmoveto{\pgfqpoint{4.597569in}{0.438889in}}%
\pgfpathlineto{\pgfqpoint{7.801389in}{0.438889in}}%
\pgfpathlineto{\pgfqpoint{7.801389in}{2.702778in}}%
\pgfpathlineto{\pgfqpoint{4.597569in}{2.702778in}}%
\pgfpathclose%
\pgfusepath{fill}%
\end{pgfscope}%
\begin{pgfscope}%
\pgfpathrectangle{\pgfqpoint{4.597569in}{0.438889in}}{\pgfqpoint{3.203819in}{2.263889in}}%
\pgfusepath{clip}%
\pgfsetbuttcap%
\pgfsetroundjoin%
\definecolor{currentfill}{rgb}{0.121569,0.466667,0.705882}%
\pgfsetfillcolor{currentfill}%
\pgfsetlinewidth{1.003750pt}%
\definecolor{currentstroke}{rgb}{0.121569,0.466667,0.705882}%
\pgfsetstrokecolor{currentstroke}%
\pgfsetdash{}{0pt}%
\pgfsys@defobject{currentmarker}{\pgfqpoint{-0.009821in}{-0.009821in}}{\pgfqpoint{0.009821in}{0.009821in}}{%
\pgfpathmoveto{\pgfqpoint{0.000000in}{-0.009821in}}%
\pgfpathcurveto{\pgfqpoint{0.002605in}{-0.009821in}}{\pgfqpoint{0.005103in}{-0.008786in}}{\pgfqpoint{0.006944in}{-0.006944in}}%
\pgfpathcurveto{\pgfqpoint{0.008786in}{-0.005103in}}{\pgfqpoint{0.009821in}{-0.002605in}}{\pgfqpoint{0.009821in}{0.000000in}}%
\pgfpathcurveto{\pgfqpoint{0.009821in}{0.002605in}}{\pgfqpoint{0.008786in}{0.005103in}}{\pgfqpoint{0.006944in}{0.006944in}}%
\pgfpathcurveto{\pgfqpoint{0.005103in}{0.008786in}}{\pgfqpoint{0.002605in}{0.009821in}}{\pgfqpoint{0.000000in}{0.009821in}}%
\pgfpathcurveto{\pgfqpoint{-0.002605in}{0.009821in}}{\pgfqpoint{-0.005103in}{0.008786in}}{\pgfqpoint{-0.006944in}{0.006944in}}%
\pgfpathcurveto{\pgfqpoint{-0.008786in}{0.005103in}}{\pgfqpoint{-0.009821in}{0.002605in}}{\pgfqpoint{-0.009821in}{0.000000in}}%
\pgfpathcurveto{\pgfqpoint{-0.009821in}{-0.002605in}}{\pgfqpoint{-0.008786in}{-0.005103in}}{\pgfqpoint{-0.006944in}{-0.006944in}}%
\pgfpathcurveto{\pgfqpoint{-0.005103in}{-0.008786in}}{\pgfqpoint{-0.002605in}{-0.009821in}}{\pgfqpoint{0.000000in}{-0.009821in}}%
\pgfpathclose%
\pgfusepath{stroke,fill}%
}%
\begin{pgfscope}%
\pgfsys@transformshift{4.743198in}{0.794575in}%
\pgfsys@useobject{currentmarker}{}%
\end{pgfscope}%
\begin{pgfscope}%
\pgfsys@transformshift{4.888777in}{0.792780in}%
\pgfsys@useobject{currentmarker}{}%
\end{pgfscope}%
\begin{pgfscope}%
\pgfsys@transformshift{5.034356in}{0.795380in}%
\pgfsys@useobject{currentmarker}{}%
\end{pgfscope}%
\begin{pgfscope}%
\pgfsys@transformshift{5.179936in}{0.797858in}%
\pgfsys@useobject{currentmarker}{}%
\end{pgfscope}%
\begin{pgfscope}%
\pgfsys@transformshift{5.325515in}{0.822146in}%
\pgfsys@useobject{currentmarker}{}%
\end{pgfscope}%
\begin{pgfscope}%
\pgfsys@transformshift{5.471095in}{0.833375in}%
\pgfsys@useobject{currentmarker}{}%
\end{pgfscope}%
\begin{pgfscope}%
\pgfsys@transformshift{5.616674in}{0.846930in}%
\pgfsys@useobject{currentmarker}{}%
\end{pgfscope}%
\begin{pgfscope}%
\pgfsys@transformshift{5.762253in}{0.833145in}%
\pgfsys@useobject{currentmarker}{}%
\end{pgfscope}%
\begin{pgfscope}%
\pgfsys@transformshift{5.907833in}{1.030952in}%
\pgfsys@useobject{currentmarker}{}%
\end{pgfscope}%
\begin{pgfscope}%
\pgfsys@transformshift{6.053412in}{1.767985in}%
\pgfsys@useobject{currentmarker}{}%
\end{pgfscope}%
\begin{pgfscope}%
\pgfsys@transformshift{6.198992in}{1.786628in}%
\pgfsys@useobject{currentmarker}{}%
\end{pgfscope}%
\begin{pgfscope}%
\pgfsys@transformshift{6.344571in}{1.573184in}%
\pgfsys@useobject{currentmarker}{}%
\end{pgfscope}%
\begin{pgfscope}%
\pgfsys@transformshift{6.490150in}{2.127317in}%
\pgfsys@useobject{currentmarker}{}%
\end{pgfscope}%
\begin{pgfscope}%
\pgfsys@transformshift{6.635730in}{2.127315in}%
\pgfsys@useobject{currentmarker}{}%
\end{pgfscope}%
\begin{pgfscope}%
\pgfsys@transformshift{6.781309in}{1.833041in}%
\pgfsys@useobject{currentmarker}{}%
\end{pgfscope}%
\begin{pgfscope}%
\pgfsys@transformshift{6.926889in}{1.620308in}%
\pgfsys@useobject{currentmarker}{}%
\end{pgfscope}%
\begin{pgfscope}%
\pgfsys@transformshift{7.072468in}{1.544095in}%
\pgfsys@useobject{currentmarker}{}%
\end{pgfscope}%
\begin{pgfscope}%
\pgfsys@transformshift{7.218047in}{2.123929in}%
\pgfsys@useobject{currentmarker}{}%
\end{pgfscope}%
\begin{pgfscope}%
\pgfsys@transformshift{7.363627in}{2.127009in}%
\pgfsys@useobject{currentmarker}{}%
\end{pgfscope}%
\begin{pgfscope}%
\pgfsys@transformshift{7.509206in}{1.990203in}%
\pgfsys@useobject{currentmarker}{}%
\end{pgfscope}%
\begin{pgfscope}%
\pgfsys@transformshift{7.654786in}{2.127315in}%
\pgfsys@useobject{currentmarker}{}%
\end{pgfscope}%
\end{pgfscope}%
\begin{pgfscope}%
\pgfpathrectangle{\pgfqpoint{4.597569in}{0.438889in}}{\pgfqpoint{3.203819in}{2.263889in}}%
\pgfusepath{clip}%
\pgfsetbuttcap%
\pgfsetroundjoin%
\definecolor{currentfill}{rgb}{0.121569,0.466667,0.705882}%
\pgfsetfillcolor{currentfill}%
\pgfsetfillopacity{0.300000}%
\pgfsetlinewidth{1.003750pt}%
\definecolor{currentstroke}{rgb}{0.121569,0.466667,0.705882}%
\pgfsetstrokecolor{currentstroke}%
\pgfsetstrokeopacity{0.300000}%
\pgfsetdash{}{0pt}%
\pgfpathmoveto{\pgfqpoint{4.743198in}{0.804433in}}%
\pgfpathlineto{\pgfqpoint{4.743198in}{0.784717in}}%
\pgfpathlineto{\pgfqpoint{4.888777in}{0.786444in}}%
\pgfpathlineto{\pgfqpoint{5.034356in}{0.784673in}}%
\pgfpathlineto{\pgfqpoint{5.179936in}{0.788376in}}%
\pgfpathlineto{\pgfqpoint{5.325515in}{0.793438in}}%
\pgfpathlineto{\pgfqpoint{5.471095in}{0.774904in}}%
\pgfpathlineto{\pgfqpoint{5.616674in}{0.770238in}}%
\pgfpathlineto{\pgfqpoint{5.762253in}{0.799483in}}%
\pgfpathlineto{\pgfqpoint{5.907833in}{0.647727in}}%
\pgfpathlineto{\pgfqpoint{6.053412in}{0.936097in}}%
\pgfpathlineto{\pgfqpoint{6.198992in}{0.976370in}}%
\pgfpathlineto{\pgfqpoint{6.344571in}{0.738291in}}%
\pgfpathlineto{\pgfqpoint{6.490150in}{2.127316in}}%
\pgfpathlineto{\pgfqpoint{6.635730in}{2.127314in}}%
\pgfpathlineto{\pgfqpoint{6.781309in}{1.068496in}}%
\pgfpathlineto{\pgfqpoint{6.926889in}{0.857083in}}%
\pgfpathlineto{\pgfqpoint{7.072468in}{0.668978in}}%
\pgfpathlineto{\pgfqpoint{7.218047in}{2.115131in}}%
\pgfpathlineto{\pgfqpoint{7.363627in}{2.126212in}}%
\pgfpathlineto{\pgfqpoint{7.509206in}{1.680123in}}%
\pgfpathlineto{\pgfqpoint{7.654786in}{2.127315in}}%
\pgfpathlineto{\pgfqpoint{7.654786in}{2.127316in}}%
\pgfpathlineto{\pgfqpoint{7.654786in}{2.127316in}}%
\pgfpathlineto{\pgfqpoint{7.509206in}{2.300283in}}%
\pgfpathlineto{\pgfqpoint{7.363627in}{2.127805in}}%
\pgfpathlineto{\pgfqpoint{7.218047in}{2.132727in}}%
\pgfpathlineto{\pgfqpoint{7.072468in}{2.419212in}}%
\pgfpathlineto{\pgfqpoint{6.926889in}{2.383532in}}%
\pgfpathlineto{\pgfqpoint{6.781309in}{2.597586in}}%
\pgfpathlineto{\pgfqpoint{6.635730in}{2.127315in}}%
\pgfpathlineto{\pgfqpoint{6.490150in}{2.127319in}}%
\pgfpathlineto{\pgfqpoint{6.344571in}{2.408078in}}%
\pgfpathlineto{\pgfqpoint{6.198992in}{2.596886in}}%
\pgfpathlineto{\pgfqpoint{6.053412in}{2.599874in}}%
\pgfpathlineto{\pgfqpoint{5.907833in}{1.414177in}}%
\pgfpathlineto{\pgfqpoint{5.762253in}{0.866807in}}%
\pgfpathlineto{\pgfqpoint{5.616674in}{0.923622in}}%
\pgfpathlineto{\pgfqpoint{5.471095in}{0.891845in}}%
\pgfpathlineto{\pgfqpoint{5.325515in}{0.850853in}}%
\pgfpathlineto{\pgfqpoint{5.179936in}{0.807340in}}%
\pgfpathlineto{\pgfqpoint{5.034356in}{0.806087in}}%
\pgfpathlineto{\pgfqpoint{4.888777in}{0.799115in}}%
\pgfpathlineto{\pgfqpoint{4.743198in}{0.804433in}}%
\pgfpathclose%
\pgfusepath{stroke,fill}%
\end{pgfscope}%
\begin{pgfscope}%
\pgfpathrectangle{\pgfqpoint{4.597569in}{0.438889in}}{\pgfqpoint{3.203819in}{2.263889in}}%
\pgfusepath{clip}%
\pgfsetbuttcap%
\pgfsetroundjoin%
\definecolor{currentfill}{rgb}{1.000000,0.498039,0.054902}%
\pgfsetfillcolor{currentfill}%
\pgfsetlinewidth{1.003750pt}%
\definecolor{currentstroke}{rgb}{1.000000,0.498039,0.054902}%
\pgfsetstrokecolor{currentstroke}%
\pgfsetdash{}{0pt}%
\pgfsys@defobject{currentmarker}{\pgfqpoint{-0.009821in}{-0.009821in}}{\pgfqpoint{0.009821in}{0.009821in}}{%
\pgfpathmoveto{\pgfqpoint{0.000000in}{-0.009821in}}%
\pgfpathcurveto{\pgfqpoint{0.002605in}{-0.009821in}}{\pgfqpoint{0.005103in}{-0.008786in}}{\pgfqpoint{0.006944in}{-0.006944in}}%
\pgfpathcurveto{\pgfqpoint{0.008786in}{-0.005103in}}{\pgfqpoint{0.009821in}{-0.002605in}}{\pgfqpoint{0.009821in}{0.000000in}}%
\pgfpathcurveto{\pgfqpoint{0.009821in}{0.002605in}}{\pgfqpoint{0.008786in}{0.005103in}}{\pgfqpoint{0.006944in}{0.006944in}}%
\pgfpathcurveto{\pgfqpoint{0.005103in}{0.008786in}}{\pgfqpoint{0.002605in}{0.009821in}}{\pgfqpoint{0.000000in}{0.009821in}}%
\pgfpathcurveto{\pgfqpoint{-0.002605in}{0.009821in}}{\pgfqpoint{-0.005103in}{0.008786in}}{\pgfqpoint{-0.006944in}{0.006944in}}%
\pgfpathcurveto{\pgfqpoint{-0.008786in}{0.005103in}}{\pgfqpoint{-0.009821in}{0.002605in}}{\pgfqpoint{-0.009821in}{0.000000in}}%
\pgfpathcurveto{\pgfqpoint{-0.009821in}{-0.002605in}}{\pgfqpoint{-0.008786in}{-0.005103in}}{\pgfqpoint{-0.006944in}{-0.006944in}}%
\pgfpathcurveto{\pgfqpoint{-0.005103in}{-0.008786in}}{\pgfqpoint{-0.002605in}{-0.009821in}}{\pgfqpoint{0.000000in}{-0.009821in}}%
\pgfpathclose%
\pgfusepath{stroke,fill}%
}%
\begin{pgfscope}%
\pgfsys@transformshift{4.743198in}{0.794555in}%
\pgfsys@useobject{currentmarker}{}%
\end{pgfscope}%
\begin{pgfscope}%
\pgfsys@transformshift{4.888777in}{0.801504in}%
\pgfsys@useobject{currentmarker}{}%
\end{pgfscope}%
\begin{pgfscope}%
\pgfsys@transformshift{5.034356in}{0.803876in}%
\pgfsys@useobject{currentmarker}{}%
\end{pgfscope}%
\begin{pgfscope}%
\pgfsys@transformshift{5.179936in}{0.806776in}%
\pgfsys@useobject{currentmarker}{}%
\end{pgfscope}%
\begin{pgfscope}%
\pgfsys@transformshift{5.325515in}{0.798437in}%
\pgfsys@useobject{currentmarker}{}%
\end{pgfscope}%
\begin{pgfscope}%
\pgfsys@transformshift{5.471095in}{0.876244in}%
\pgfsys@useobject{currentmarker}{}%
\end{pgfscope}%
\begin{pgfscope}%
\pgfsys@transformshift{5.616674in}{0.845025in}%
\pgfsys@useobject{currentmarker}{}%
\end{pgfscope}%
\begin{pgfscope}%
\pgfsys@transformshift{5.762253in}{0.974234in}%
\pgfsys@useobject{currentmarker}{}%
\end{pgfscope}%
\begin{pgfscope}%
\pgfsys@transformshift{5.907833in}{0.958374in}%
\pgfsys@useobject{currentmarker}{}%
\end{pgfscope}%
\begin{pgfscope}%
\pgfsys@transformshift{6.053412in}{0.991935in}%
\pgfsys@useobject{currentmarker}{}%
\end{pgfscope}%
\begin{pgfscope}%
\pgfsys@transformshift{6.198992in}{1.328694in}%
\pgfsys@useobject{currentmarker}{}%
\end{pgfscope}%
\begin{pgfscope}%
\pgfsys@transformshift{6.344571in}{1.392320in}%
\pgfsys@useobject{currentmarker}{}%
\end{pgfscope}%
\begin{pgfscope}%
\pgfsys@transformshift{6.490150in}{1.888858in}%
\pgfsys@useobject{currentmarker}{}%
\end{pgfscope}%
\begin{pgfscope}%
\pgfsys@transformshift{6.635730in}{2.110121in}%
\pgfsys@useobject{currentmarker}{}%
\end{pgfscope}%
\begin{pgfscope}%
\pgfsys@transformshift{6.781309in}{2.110130in}%
\pgfsys@useobject{currentmarker}{}%
\end{pgfscope}%
\begin{pgfscope}%
\pgfsys@transformshift{6.926889in}{1.196342in}%
\pgfsys@useobject{currentmarker}{}%
\end{pgfscope}%
\begin{pgfscope}%
\pgfsys@transformshift{7.072468in}{1.854816in}%
\pgfsys@useobject{currentmarker}{}%
\end{pgfscope}%
\begin{pgfscope}%
\pgfsys@transformshift{7.218047in}{2.033369in}%
\pgfsys@useobject{currentmarker}{}%
\end{pgfscope}%
\begin{pgfscope}%
\pgfsys@transformshift{7.363627in}{2.062848in}%
\pgfsys@useobject{currentmarker}{}%
\end{pgfscope}%
\begin{pgfscope}%
\pgfsys@transformshift{7.509206in}{2.110146in}%
\pgfsys@useobject{currentmarker}{}%
\end{pgfscope}%
\begin{pgfscope}%
\pgfsys@transformshift{7.654786in}{2.110145in}%
\pgfsys@useobject{currentmarker}{}%
\end{pgfscope}%
\end{pgfscope}%
\begin{pgfscope}%
\pgfpathrectangle{\pgfqpoint{4.597569in}{0.438889in}}{\pgfqpoint{3.203819in}{2.263889in}}%
\pgfusepath{clip}%
\pgfsetbuttcap%
\pgfsetroundjoin%
\definecolor{currentfill}{rgb}{1.000000,0.498039,0.054902}%
\pgfsetfillcolor{currentfill}%
\pgfsetfillopacity{0.300000}%
\pgfsetlinewidth{1.003750pt}%
\definecolor{currentstroke}{rgb}{1.000000,0.498039,0.054902}%
\pgfsetstrokecolor{currentstroke}%
\pgfsetstrokeopacity{0.300000}%
\pgfsetdash{}{0pt}%
\pgfpathmoveto{\pgfqpoint{4.743198in}{0.802661in}}%
\pgfpathlineto{\pgfqpoint{4.743198in}{0.786448in}}%
\pgfpathlineto{\pgfqpoint{4.888777in}{0.797201in}}%
\pgfpathlineto{\pgfqpoint{5.034356in}{0.790153in}}%
\pgfpathlineto{\pgfqpoint{5.179936in}{0.776646in}}%
\pgfpathlineto{\pgfqpoint{5.325515in}{0.783579in}}%
\pgfpathlineto{\pgfqpoint{5.471095in}{0.772505in}}%
\pgfpathlineto{\pgfqpoint{5.616674in}{0.820642in}}%
\pgfpathlineto{\pgfqpoint{5.762253in}{0.826392in}}%
\pgfpathlineto{\pgfqpoint{5.907833in}{0.586274in}}%
\pgfpathlineto{\pgfqpoint{6.053412in}{0.799565in}}%
\pgfpathlineto{\pgfqpoint{6.198992in}{0.541793in}}%
\pgfpathlineto{\pgfqpoint{6.344571in}{0.888358in}}%
\pgfpathlineto{\pgfqpoint{6.490150in}{1.348899in}}%
\pgfpathlineto{\pgfqpoint{6.635730in}{2.110045in}}%
\pgfpathlineto{\pgfqpoint{6.781309in}{2.110110in}}%
\pgfpathlineto{\pgfqpoint{6.926889in}{0.620363in}}%
\pgfpathlineto{\pgfqpoint{7.072468in}{1.191479in}}%
\pgfpathlineto{\pgfqpoint{7.218047in}{1.833919in}}%
\pgfpathlineto{\pgfqpoint{7.363627in}{1.939976in}}%
\pgfpathlineto{\pgfqpoint{7.509206in}{2.110137in}}%
\pgfpathlineto{\pgfqpoint{7.654786in}{2.110141in}}%
\pgfpathlineto{\pgfqpoint{7.654786in}{2.110149in}}%
\pgfpathlineto{\pgfqpoint{7.654786in}{2.110149in}}%
\pgfpathlineto{\pgfqpoint{7.509206in}{2.110155in}}%
\pgfpathlineto{\pgfqpoint{7.363627in}{2.185720in}}%
\pgfpathlineto{\pgfqpoint{7.218047in}{2.232819in}}%
\pgfpathlineto{\pgfqpoint{7.072468in}{2.518152in}}%
\pgfpathlineto{\pgfqpoint{6.926889in}{1.772320in}}%
\pgfpathlineto{\pgfqpoint{6.781309in}{2.110150in}}%
\pgfpathlineto{\pgfqpoint{6.635730in}{2.110197in}}%
\pgfpathlineto{\pgfqpoint{6.490150in}{2.428817in}}%
\pgfpathlineto{\pgfqpoint{6.344571in}{1.896283in}}%
\pgfpathlineto{\pgfqpoint{6.198992in}{2.115596in}}%
\pgfpathlineto{\pgfqpoint{6.053412in}{1.184306in}}%
\pgfpathlineto{\pgfqpoint{5.907833in}{1.330473in}}%
\pgfpathlineto{\pgfqpoint{5.762253in}{1.122075in}}%
\pgfpathlineto{\pgfqpoint{5.616674in}{0.869408in}}%
\pgfpathlineto{\pgfqpoint{5.471095in}{0.979982in}}%
\pgfpathlineto{\pgfqpoint{5.325515in}{0.813296in}}%
\pgfpathlineto{\pgfqpoint{5.179936in}{0.836906in}}%
\pgfpathlineto{\pgfqpoint{5.034356in}{0.817599in}}%
\pgfpathlineto{\pgfqpoint{4.888777in}{0.805807in}}%
\pgfpathlineto{\pgfqpoint{4.743198in}{0.802661in}}%
\pgfpathclose%
\pgfusepath{stroke,fill}%
\end{pgfscope}%
\begin{pgfscope}%
\pgfpathrectangle{\pgfqpoint{4.597569in}{0.438889in}}{\pgfqpoint{3.203819in}{2.263889in}}%
\pgfusepath{clip}%
\pgfsetbuttcap%
\pgfsetroundjoin%
\definecolor{currentfill}{rgb}{0.172549,0.627451,0.172549}%
\pgfsetfillcolor{currentfill}%
\pgfsetlinewidth{1.003750pt}%
\definecolor{currentstroke}{rgb}{0.172549,0.627451,0.172549}%
\pgfsetstrokecolor{currentstroke}%
\pgfsetdash{}{0pt}%
\pgfsys@defobject{currentmarker}{\pgfqpoint{-0.009821in}{-0.009821in}}{\pgfqpoint{0.009821in}{0.009821in}}{%
\pgfpathmoveto{\pgfqpoint{0.000000in}{-0.009821in}}%
\pgfpathcurveto{\pgfqpoint{0.002605in}{-0.009821in}}{\pgfqpoint{0.005103in}{-0.008786in}}{\pgfqpoint{0.006944in}{-0.006944in}}%
\pgfpathcurveto{\pgfqpoint{0.008786in}{-0.005103in}}{\pgfqpoint{0.009821in}{-0.002605in}}{\pgfqpoint{0.009821in}{0.000000in}}%
\pgfpathcurveto{\pgfqpoint{0.009821in}{0.002605in}}{\pgfqpoint{0.008786in}{0.005103in}}{\pgfqpoint{0.006944in}{0.006944in}}%
\pgfpathcurveto{\pgfqpoint{0.005103in}{0.008786in}}{\pgfqpoint{0.002605in}{0.009821in}}{\pgfqpoint{0.000000in}{0.009821in}}%
\pgfpathcurveto{\pgfqpoint{-0.002605in}{0.009821in}}{\pgfqpoint{-0.005103in}{0.008786in}}{\pgfqpoint{-0.006944in}{0.006944in}}%
\pgfpathcurveto{\pgfqpoint{-0.008786in}{0.005103in}}{\pgfqpoint{-0.009821in}{0.002605in}}{\pgfqpoint{-0.009821in}{0.000000in}}%
\pgfpathcurveto{\pgfqpoint{-0.009821in}{-0.002605in}}{\pgfqpoint{-0.008786in}{-0.005103in}}{\pgfqpoint{-0.006944in}{-0.006944in}}%
\pgfpathcurveto{\pgfqpoint{-0.005103in}{-0.008786in}}{\pgfqpoint{-0.002605in}{-0.009821in}}{\pgfqpoint{0.000000in}{-0.009821in}}%
\pgfpathclose%
\pgfusepath{stroke,fill}%
}%
\begin{pgfscope}%
\pgfsys@transformshift{4.743198in}{0.794607in}%
\pgfsys@useobject{currentmarker}{}%
\end{pgfscope}%
\begin{pgfscope}%
\pgfsys@transformshift{4.888777in}{0.794424in}%
\pgfsys@useobject{currentmarker}{}%
\end{pgfscope}%
\begin{pgfscope}%
\pgfsys@transformshift{5.034356in}{0.794160in}%
\pgfsys@useobject{currentmarker}{}%
\end{pgfscope}%
\begin{pgfscope}%
\pgfsys@transformshift{5.179936in}{0.807612in}%
\pgfsys@useobject{currentmarker}{}%
\end{pgfscope}%
\begin{pgfscope}%
\pgfsys@transformshift{5.325515in}{0.792327in}%
\pgfsys@useobject{currentmarker}{}%
\end{pgfscope}%
\begin{pgfscope}%
\pgfsys@transformshift{5.471095in}{0.818182in}%
\pgfsys@useobject{currentmarker}{}%
\end{pgfscope}%
\begin{pgfscope}%
\pgfsys@transformshift{5.616674in}{0.838357in}%
\pgfsys@useobject{currentmarker}{}%
\end{pgfscope}%
\begin{pgfscope}%
\pgfsys@transformshift{5.762253in}{0.868646in}%
\pgfsys@useobject{currentmarker}{}%
\end{pgfscope}%
\begin{pgfscope}%
\pgfsys@transformshift{5.907833in}{0.864188in}%
\pgfsys@useobject{currentmarker}{}%
\end{pgfscope}%
\begin{pgfscope}%
\pgfsys@transformshift{6.053412in}{0.890527in}%
\pgfsys@useobject{currentmarker}{}%
\end{pgfscope}%
\begin{pgfscope}%
\pgfsys@transformshift{6.198992in}{0.870319in}%
\pgfsys@useobject{currentmarker}{}%
\end{pgfscope}%
\begin{pgfscope}%
\pgfsys@transformshift{6.344571in}{1.160606in}%
\pgfsys@useobject{currentmarker}{}%
\end{pgfscope}%
\begin{pgfscope}%
\pgfsys@transformshift{6.490150in}{1.074847in}%
\pgfsys@useobject{currentmarker}{}%
\end{pgfscope}%
\begin{pgfscope}%
\pgfsys@transformshift{6.635730in}{1.512983in}%
\pgfsys@useobject{currentmarker}{}%
\end{pgfscope}%
\begin{pgfscope}%
\pgfsys@transformshift{6.781309in}{1.590899in}%
\pgfsys@useobject{currentmarker}{}%
\end{pgfscope}%
\begin{pgfscope}%
\pgfsys@transformshift{6.926889in}{1.731890in}%
\pgfsys@useobject{currentmarker}{}%
\end{pgfscope}%
\begin{pgfscope}%
\pgfsys@transformshift{7.072468in}{1.733244in}%
\pgfsys@useobject{currentmarker}{}%
\end{pgfscope}%
\begin{pgfscope}%
\pgfsys@transformshift{7.218047in}{1.588091in}%
\pgfsys@useobject{currentmarker}{}%
\end{pgfscope}%
\begin{pgfscope}%
\pgfsys@transformshift{7.363627in}{2.007572in}%
\pgfsys@useobject{currentmarker}{}%
\end{pgfscope}%
\begin{pgfscope}%
\pgfsys@transformshift{7.509206in}{2.009131in}%
\pgfsys@useobject{currentmarker}{}%
\end{pgfscope}%
\begin{pgfscope}%
\pgfsys@transformshift{7.654786in}{2.009065in}%
\pgfsys@useobject{currentmarker}{}%
\end{pgfscope}%
\end{pgfscope}%
\begin{pgfscope}%
\pgfpathrectangle{\pgfqpoint{4.597569in}{0.438889in}}{\pgfqpoint{3.203819in}{2.263889in}}%
\pgfusepath{clip}%
\pgfsetbuttcap%
\pgfsetroundjoin%
\definecolor{currentfill}{rgb}{0.172549,0.627451,0.172549}%
\pgfsetfillcolor{currentfill}%
\pgfsetfillopacity{0.300000}%
\pgfsetlinewidth{1.003750pt}%
\definecolor{currentstroke}{rgb}{0.172549,0.627451,0.172549}%
\pgfsetstrokecolor{currentstroke}%
\pgfsetstrokeopacity{0.300000}%
\pgfsetdash{}{0pt}%
\pgfpathmoveto{\pgfqpoint{4.743198in}{0.801254in}}%
\pgfpathlineto{\pgfqpoint{4.743198in}{0.787960in}}%
\pgfpathlineto{\pgfqpoint{4.888777in}{0.782140in}}%
\pgfpathlineto{\pgfqpoint{5.034356in}{0.784031in}}%
\pgfpathlineto{\pgfqpoint{5.179936in}{0.783594in}}%
\pgfpathlineto{\pgfqpoint{5.325515in}{0.786978in}}%
\pgfpathlineto{\pgfqpoint{5.471095in}{0.773471in}}%
\pgfpathlineto{\pgfqpoint{5.616674in}{0.768755in}}%
\pgfpathlineto{\pgfqpoint{5.762253in}{0.795611in}}%
\pgfpathlineto{\pgfqpoint{5.907833in}{0.710184in}}%
\pgfpathlineto{\pgfqpoint{6.053412in}{0.737661in}}%
\pgfpathlineto{\pgfqpoint{6.198992in}{0.759744in}}%
\pgfpathlineto{\pgfqpoint{6.344571in}{0.633175in}}%
\pgfpathlineto{\pgfqpoint{6.490150in}{0.870342in}}%
\pgfpathlineto{\pgfqpoint{6.635730in}{1.168766in}}%
\pgfpathlineto{\pgfqpoint{6.781309in}{0.945780in}}%
\pgfpathlineto{\pgfqpoint{6.926889in}{1.069067in}}%
\pgfpathlineto{\pgfqpoint{7.072468in}{1.016644in}}%
\pgfpathlineto{\pgfqpoint{7.218047in}{0.925902in}}%
\pgfpathlineto{\pgfqpoint{7.363627in}{2.003584in}}%
\pgfpathlineto{\pgfqpoint{7.509206in}{2.008948in}}%
\pgfpathlineto{\pgfqpoint{7.654786in}{2.008877in}}%
\pgfpathlineto{\pgfqpoint{7.654786in}{2.009253in}}%
\pgfpathlineto{\pgfqpoint{7.654786in}{2.009253in}}%
\pgfpathlineto{\pgfqpoint{7.509206in}{2.009315in}}%
\pgfpathlineto{\pgfqpoint{7.363627in}{2.011561in}}%
\pgfpathlineto{\pgfqpoint{7.218047in}{2.250279in}}%
\pgfpathlineto{\pgfqpoint{7.072468in}{2.449845in}}%
\pgfpathlineto{\pgfqpoint{6.926889in}{2.394713in}}%
\pgfpathlineto{\pgfqpoint{6.781309in}{2.236019in}}%
\pgfpathlineto{\pgfqpoint{6.635730in}{1.857200in}}%
\pgfpathlineto{\pgfqpoint{6.490150in}{1.279352in}}%
\pgfpathlineto{\pgfqpoint{6.344571in}{1.688037in}}%
\pgfpathlineto{\pgfqpoint{6.198992in}{0.980894in}}%
\pgfpathlineto{\pgfqpoint{6.053412in}{1.043394in}}%
\pgfpathlineto{\pgfqpoint{5.907833in}{1.018193in}}%
\pgfpathlineto{\pgfqpoint{5.762253in}{0.941682in}}%
\pgfpathlineto{\pgfqpoint{5.616674in}{0.907959in}}%
\pgfpathlineto{\pgfqpoint{5.471095in}{0.862894in}}%
\pgfpathlineto{\pgfqpoint{5.325515in}{0.797676in}}%
\pgfpathlineto{\pgfqpoint{5.179936in}{0.831629in}}%
\pgfpathlineto{\pgfqpoint{5.034356in}{0.804288in}}%
\pgfpathlineto{\pgfqpoint{4.888777in}{0.806709in}}%
\pgfpathlineto{\pgfqpoint{4.743198in}{0.801254in}}%
\pgfpathclose%
\pgfusepath{stroke,fill}%
\end{pgfscope}%
\begin{pgfscope}%
\pgfpathrectangle{\pgfqpoint{4.597569in}{0.438889in}}{\pgfqpoint{3.203819in}{2.263889in}}%
\pgfusepath{clip}%
\pgfsetbuttcap%
\pgfsetroundjoin%
\definecolor{currentfill}{rgb}{0.839216,0.152941,0.156863}%
\pgfsetfillcolor{currentfill}%
\pgfsetlinewidth{1.003750pt}%
\definecolor{currentstroke}{rgb}{0.839216,0.152941,0.156863}%
\pgfsetstrokecolor{currentstroke}%
\pgfsetdash{}{0pt}%
\pgfsys@defobject{currentmarker}{\pgfqpoint{-0.009821in}{-0.009821in}}{\pgfqpoint{0.009821in}{0.009821in}}{%
\pgfpathmoveto{\pgfqpoint{0.000000in}{-0.009821in}}%
\pgfpathcurveto{\pgfqpoint{0.002605in}{-0.009821in}}{\pgfqpoint{0.005103in}{-0.008786in}}{\pgfqpoint{0.006944in}{-0.006944in}}%
\pgfpathcurveto{\pgfqpoint{0.008786in}{-0.005103in}}{\pgfqpoint{0.009821in}{-0.002605in}}{\pgfqpoint{0.009821in}{0.000000in}}%
\pgfpathcurveto{\pgfqpoint{0.009821in}{0.002605in}}{\pgfqpoint{0.008786in}{0.005103in}}{\pgfqpoint{0.006944in}{0.006944in}}%
\pgfpathcurveto{\pgfqpoint{0.005103in}{0.008786in}}{\pgfqpoint{0.002605in}{0.009821in}}{\pgfqpoint{0.000000in}{0.009821in}}%
\pgfpathcurveto{\pgfqpoint{-0.002605in}{0.009821in}}{\pgfqpoint{-0.005103in}{0.008786in}}{\pgfqpoint{-0.006944in}{0.006944in}}%
\pgfpathcurveto{\pgfqpoint{-0.008786in}{0.005103in}}{\pgfqpoint{-0.009821in}{0.002605in}}{\pgfqpoint{-0.009821in}{0.000000in}}%
\pgfpathcurveto{\pgfqpoint{-0.009821in}{-0.002605in}}{\pgfqpoint{-0.008786in}{-0.005103in}}{\pgfqpoint{-0.006944in}{-0.006944in}}%
\pgfpathcurveto{\pgfqpoint{-0.005103in}{-0.008786in}}{\pgfqpoint{-0.002605in}{-0.009821in}}{\pgfqpoint{0.000000in}{-0.009821in}}%
\pgfpathclose%
\pgfusepath{stroke,fill}%
}%
\begin{pgfscope}%
\pgfsys@transformshift{4.743198in}{0.790717in}%
\pgfsys@useobject{currentmarker}{}%
\end{pgfscope}%
\begin{pgfscope}%
\pgfsys@transformshift{4.888777in}{0.795658in}%
\pgfsys@useobject{currentmarker}{}%
\end{pgfscope}%
\begin{pgfscope}%
\pgfsys@transformshift{5.034356in}{0.792128in}%
\pgfsys@useobject{currentmarker}{}%
\end{pgfscope}%
\begin{pgfscope}%
\pgfsys@transformshift{5.179936in}{0.794874in}%
\pgfsys@useobject{currentmarker}{}%
\end{pgfscope}%
\begin{pgfscope}%
\pgfsys@transformshift{5.325515in}{0.799703in}%
\pgfsys@useobject{currentmarker}{}%
\end{pgfscope}%
\begin{pgfscope}%
\pgfsys@transformshift{5.471095in}{0.804125in}%
\pgfsys@useobject{currentmarker}{}%
\end{pgfscope}%
\begin{pgfscope}%
\pgfsys@transformshift{5.616674in}{0.815507in}%
\pgfsys@useobject{currentmarker}{}%
\end{pgfscope}%
\begin{pgfscope}%
\pgfsys@transformshift{5.762253in}{0.806506in}%
\pgfsys@useobject{currentmarker}{}%
\end{pgfscope}%
\begin{pgfscope}%
\pgfsys@transformshift{5.907833in}{0.823500in}%
\pgfsys@useobject{currentmarker}{}%
\end{pgfscope}%
\begin{pgfscope}%
\pgfsys@transformshift{6.053412in}{0.854539in}%
\pgfsys@useobject{currentmarker}{}%
\end{pgfscope}%
\begin{pgfscope}%
\pgfsys@transformshift{6.198992in}{0.853238in}%
\pgfsys@useobject{currentmarker}{}%
\end{pgfscope}%
\begin{pgfscope}%
\pgfsys@transformshift{6.344571in}{0.846594in}%
\pgfsys@useobject{currentmarker}{}%
\end{pgfscope}%
\begin{pgfscope}%
\pgfsys@transformshift{6.490150in}{0.839820in}%
\pgfsys@useobject{currentmarker}{}%
\end{pgfscope}%
\begin{pgfscope}%
\pgfsys@transformshift{6.635730in}{0.852733in}%
\pgfsys@useobject{currentmarker}{}%
\end{pgfscope}%
\begin{pgfscope}%
\pgfsys@transformshift{6.781309in}{0.847439in}%
\pgfsys@useobject{currentmarker}{}%
\end{pgfscope}%
\begin{pgfscope}%
\pgfsys@transformshift{6.926889in}{0.838671in}%
\pgfsys@useobject{currentmarker}{}%
\end{pgfscope}%
\begin{pgfscope}%
\pgfsys@transformshift{7.072468in}{0.844658in}%
\pgfsys@useobject{currentmarker}{}%
\end{pgfscope}%
\begin{pgfscope}%
\pgfsys@transformshift{7.218047in}{0.856589in}%
\pgfsys@useobject{currentmarker}{}%
\end{pgfscope}%
\begin{pgfscope}%
\pgfsys@transformshift{7.363627in}{0.847952in}%
\pgfsys@useobject{currentmarker}{}%
\end{pgfscope}%
\begin{pgfscope}%
\pgfsys@transformshift{7.509206in}{0.849700in}%
\pgfsys@useobject{currentmarker}{}%
\end{pgfscope}%
\begin{pgfscope}%
\pgfsys@transformshift{7.654786in}{0.849308in}%
\pgfsys@useobject{currentmarker}{}%
\end{pgfscope}%
\end{pgfscope}%
\begin{pgfscope}%
\pgfpathrectangle{\pgfqpoint{4.597569in}{0.438889in}}{\pgfqpoint{3.203819in}{2.263889in}}%
\pgfusepath{clip}%
\pgfsetbuttcap%
\pgfsetroundjoin%
\definecolor{currentfill}{rgb}{0.839216,0.152941,0.156863}%
\pgfsetfillcolor{currentfill}%
\pgfsetfillopacity{0.300000}%
\pgfsetlinewidth{1.003750pt}%
\definecolor{currentstroke}{rgb}{0.839216,0.152941,0.156863}%
\pgfsetstrokecolor{currentstroke}%
\pgfsetstrokeopacity{0.300000}%
\pgfsetdash{}{0pt}%
\pgfpathmoveto{\pgfqpoint{4.743198in}{0.795147in}}%
\pgfpathlineto{\pgfqpoint{4.743198in}{0.786286in}}%
\pgfpathlineto{\pgfqpoint{4.888777in}{0.783836in}}%
\pgfpathlineto{\pgfqpoint{5.034356in}{0.788105in}}%
\pgfpathlineto{\pgfqpoint{5.179936in}{0.788496in}}%
\pgfpathlineto{\pgfqpoint{5.325515in}{0.787620in}}%
\pgfpathlineto{\pgfqpoint{5.471095in}{0.790040in}}%
\pgfpathlineto{\pgfqpoint{5.616674in}{0.769609in}}%
\pgfpathlineto{\pgfqpoint{5.762253in}{0.789255in}}%
\pgfpathlineto{\pgfqpoint{5.907833in}{0.803689in}}%
\pgfpathlineto{\pgfqpoint{6.053412in}{0.806022in}}%
\pgfpathlineto{\pgfqpoint{6.198992in}{0.823232in}}%
\pgfpathlineto{\pgfqpoint{6.344571in}{0.810769in}}%
\pgfpathlineto{\pgfqpoint{6.490150in}{0.818278in}}%
\pgfpathlineto{\pgfqpoint{6.635730in}{0.838833in}}%
\pgfpathlineto{\pgfqpoint{6.781309in}{0.823289in}}%
\pgfpathlineto{\pgfqpoint{6.926889in}{0.833054in}}%
\pgfpathlineto{\pgfqpoint{7.072468in}{0.836661in}}%
\pgfpathlineto{\pgfqpoint{7.218047in}{0.850110in}}%
\pgfpathlineto{\pgfqpoint{7.363627in}{0.839697in}}%
\pgfpathlineto{\pgfqpoint{7.509206in}{0.844430in}}%
\pgfpathlineto{\pgfqpoint{7.654786in}{0.847166in}}%
\pgfpathlineto{\pgfqpoint{7.654786in}{0.851451in}}%
\pgfpathlineto{\pgfqpoint{7.654786in}{0.851451in}}%
\pgfpathlineto{\pgfqpoint{7.509206in}{0.854969in}}%
\pgfpathlineto{\pgfqpoint{7.363627in}{0.856206in}}%
\pgfpathlineto{\pgfqpoint{7.218047in}{0.863068in}}%
\pgfpathlineto{\pgfqpoint{7.072468in}{0.852656in}}%
\pgfpathlineto{\pgfqpoint{6.926889in}{0.844288in}}%
\pgfpathlineto{\pgfqpoint{6.781309in}{0.871589in}}%
\pgfpathlineto{\pgfqpoint{6.635730in}{0.866634in}}%
\pgfpathlineto{\pgfqpoint{6.490150in}{0.861362in}}%
\pgfpathlineto{\pgfqpoint{6.344571in}{0.882420in}}%
\pgfpathlineto{\pgfqpoint{6.198992in}{0.883245in}}%
\pgfpathlineto{\pgfqpoint{6.053412in}{0.903055in}}%
\pgfpathlineto{\pgfqpoint{5.907833in}{0.843311in}}%
\pgfpathlineto{\pgfqpoint{5.762253in}{0.823758in}}%
\pgfpathlineto{\pgfqpoint{5.616674in}{0.861405in}}%
\pgfpathlineto{\pgfqpoint{5.471095in}{0.818211in}}%
\pgfpathlineto{\pgfqpoint{5.325515in}{0.811787in}}%
\pgfpathlineto{\pgfqpoint{5.179936in}{0.801251in}}%
\pgfpathlineto{\pgfqpoint{5.034356in}{0.796152in}}%
\pgfpathlineto{\pgfqpoint{4.888777in}{0.807480in}}%
\pgfpathlineto{\pgfqpoint{4.743198in}{0.795147in}}%
\pgfpathclose%
\pgfusepath{stroke,fill}%
\end{pgfscope}%
\begin{pgfscope}%
\pgfpathrectangle{\pgfqpoint{4.597569in}{0.438889in}}{\pgfqpoint{3.203819in}{2.263889in}}%
\pgfusepath{clip}%
\pgfsetbuttcap%
\pgfsetroundjoin%
\definecolor{currentfill}{rgb}{0.580392,0.403922,0.741176}%
\pgfsetfillcolor{currentfill}%
\pgfsetlinewidth{1.003750pt}%
\definecolor{currentstroke}{rgb}{0.580392,0.403922,0.741176}%
\pgfsetstrokecolor{currentstroke}%
\pgfsetdash{}{0pt}%
\pgfsys@defobject{currentmarker}{\pgfqpoint{-0.009821in}{-0.009821in}}{\pgfqpoint{0.009821in}{0.009821in}}{%
\pgfpathmoveto{\pgfqpoint{0.000000in}{-0.009821in}}%
\pgfpathcurveto{\pgfqpoint{0.002605in}{-0.009821in}}{\pgfqpoint{0.005103in}{-0.008786in}}{\pgfqpoint{0.006944in}{-0.006944in}}%
\pgfpathcurveto{\pgfqpoint{0.008786in}{-0.005103in}}{\pgfqpoint{0.009821in}{-0.002605in}}{\pgfqpoint{0.009821in}{0.000000in}}%
\pgfpathcurveto{\pgfqpoint{0.009821in}{0.002605in}}{\pgfqpoint{0.008786in}{0.005103in}}{\pgfqpoint{0.006944in}{0.006944in}}%
\pgfpathcurveto{\pgfqpoint{0.005103in}{0.008786in}}{\pgfqpoint{0.002605in}{0.009821in}}{\pgfqpoint{0.000000in}{0.009821in}}%
\pgfpathcurveto{\pgfqpoint{-0.002605in}{0.009821in}}{\pgfqpoint{-0.005103in}{0.008786in}}{\pgfqpoint{-0.006944in}{0.006944in}}%
\pgfpathcurveto{\pgfqpoint{-0.008786in}{0.005103in}}{\pgfqpoint{-0.009821in}{0.002605in}}{\pgfqpoint{-0.009821in}{0.000000in}}%
\pgfpathcurveto{\pgfqpoint{-0.009821in}{-0.002605in}}{\pgfqpoint{-0.008786in}{-0.005103in}}{\pgfqpoint{-0.006944in}{-0.006944in}}%
\pgfpathcurveto{\pgfqpoint{-0.005103in}{-0.008786in}}{\pgfqpoint{-0.002605in}{-0.009821in}}{\pgfqpoint{0.000000in}{-0.009821in}}%
\pgfpathclose%
\pgfusepath{stroke,fill}%
}%
\begin{pgfscope}%
\pgfsys@transformshift{4.743198in}{0.793191in}%
\pgfsys@useobject{currentmarker}{}%
\end{pgfscope}%
\begin{pgfscope}%
\pgfsys@transformshift{4.888777in}{0.795370in}%
\pgfsys@useobject{currentmarker}{}%
\end{pgfscope}%
\begin{pgfscope}%
\pgfsys@transformshift{5.034356in}{0.800660in}%
\pgfsys@useobject{currentmarker}{}%
\end{pgfscope}%
\begin{pgfscope}%
\pgfsys@transformshift{5.179936in}{0.799883in}%
\pgfsys@useobject{currentmarker}{}%
\end{pgfscope}%
\begin{pgfscope}%
\pgfsys@transformshift{5.325515in}{0.794280in}%
\pgfsys@useobject{currentmarker}{}%
\end{pgfscope}%
\begin{pgfscope}%
\pgfsys@transformshift{5.471095in}{0.806989in}%
\pgfsys@useobject{currentmarker}{}%
\end{pgfscope}%
\begin{pgfscope}%
\pgfsys@transformshift{5.616674in}{0.804853in}%
\pgfsys@useobject{currentmarker}{}%
\end{pgfscope}%
\begin{pgfscope}%
\pgfsys@transformshift{5.762253in}{0.821082in}%
\pgfsys@useobject{currentmarker}{}%
\end{pgfscope}%
\begin{pgfscope}%
\pgfsys@transformshift{5.907833in}{0.813351in}%
\pgfsys@useobject{currentmarker}{}%
\end{pgfscope}%
\begin{pgfscope}%
\pgfsys@transformshift{6.053412in}{0.810823in}%
\pgfsys@useobject{currentmarker}{}%
\end{pgfscope}%
\begin{pgfscope}%
\pgfsys@transformshift{6.198992in}{0.812176in}%
\pgfsys@useobject{currentmarker}{}%
\end{pgfscope}%
\begin{pgfscope}%
\pgfsys@transformshift{6.344571in}{0.811001in}%
\pgfsys@useobject{currentmarker}{}%
\end{pgfscope}%
\begin{pgfscope}%
\pgfsys@transformshift{6.490150in}{0.813398in}%
\pgfsys@useobject{currentmarker}{}%
\end{pgfscope}%
\begin{pgfscope}%
\pgfsys@transformshift{6.635730in}{0.814203in}%
\pgfsys@useobject{currentmarker}{}%
\end{pgfscope}%
\begin{pgfscope}%
\pgfsys@transformshift{6.781309in}{0.811183in}%
\pgfsys@useobject{currentmarker}{}%
\end{pgfscope}%
\begin{pgfscope}%
\pgfsys@transformshift{6.926889in}{0.810360in}%
\pgfsys@useobject{currentmarker}{}%
\end{pgfscope}%
\begin{pgfscope}%
\pgfsys@transformshift{7.072468in}{0.812024in}%
\pgfsys@useobject{currentmarker}{}%
\end{pgfscope}%
\begin{pgfscope}%
\pgfsys@transformshift{7.218047in}{0.810785in}%
\pgfsys@useobject{currentmarker}{}%
\end{pgfscope}%
\begin{pgfscope}%
\pgfsys@transformshift{7.363627in}{0.812355in}%
\pgfsys@useobject{currentmarker}{}%
\end{pgfscope}%
\begin{pgfscope}%
\pgfsys@transformshift{7.509206in}{0.812138in}%
\pgfsys@useobject{currentmarker}{}%
\end{pgfscope}%
\begin{pgfscope}%
\pgfsys@transformshift{7.654786in}{0.812141in}%
\pgfsys@useobject{currentmarker}{}%
\end{pgfscope}%
\end{pgfscope}%
\begin{pgfscope}%
\pgfpathrectangle{\pgfqpoint{4.597569in}{0.438889in}}{\pgfqpoint{3.203819in}{2.263889in}}%
\pgfusepath{clip}%
\pgfsetbuttcap%
\pgfsetroundjoin%
\definecolor{currentfill}{rgb}{0.580392,0.403922,0.741176}%
\pgfsetfillcolor{currentfill}%
\pgfsetfillopacity{0.300000}%
\pgfsetlinewidth{1.003750pt}%
\definecolor{currentstroke}{rgb}{0.580392,0.403922,0.741176}%
\pgfsetstrokecolor{currentstroke}%
\pgfsetstrokeopacity{0.300000}%
\pgfsetdash{}{0pt}%
\pgfpathmoveto{\pgfqpoint{4.743198in}{0.805616in}}%
\pgfpathlineto{\pgfqpoint{4.743198in}{0.780767in}}%
\pgfpathlineto{\pgfqpoint{4.888777in}{0.784120in}}%
\pgfpathlineto{\pgfqpoint{5.034356in}{0.783118in}}%
\pgfpathlineto{\pgfqpoint{5.179936in}{0.787947in}}%
\pgfpathlineto{\pgfqpoint{5.325515in}{0.786960in}}%
\pgfpathlineto{\pgfqpoint{5.471095in}{0.784913in}}%
\pgfpathlineto{\pgfqpoint{5.616674in}{0.799338in}}%
\pgfpathlineto{\pgfqpoint{5.762253in}{0.800627in}}%
\pgfpathlineto{\pgfqpoint{5.907833in}{0.780637in}}%
\pgfpathlineto{\pgfqpoint{6.053412in}{0.798724in}}%
\pgfpathlineto{\pgfqpoint{6.198992in}{0.797236in}}%
\pgfpathlineto{\pgfqpoint{6.344571in}{0.802441in}}%
\pgfpathlineto{\pgfqpoint{6.490150in}{0.806158in}}%
\pgfpathlineto{\pgfqpoint{6.635730in}{0.809940in}}%
\pgfpathlineto{\pgfqpoint{6.781309in}{0.808617in}}%
\pgfpathlineto{\pgfqpoint{6.926889in}{0.807416in}}%
\pgfpathlineto{\pgfqpoint{7.072468in}{0.810514in}}%
\pgfpathlineto{\pgfqpoint{7.218047in}{0.808826in}}%
\pgfpathlineto{\pgfqpoint{7.363627in}{0.811663in}}%
\pgfpathlineto{\pgfqpoint{7.509206in}{0.811338in}}%
\pgfpathlineto{\pgfqpoint{7.654786in}{0.811600in}}%
\pgfpathlineto{\pgfqpoint{7.654786in}{0.812682in}}%
\pgfpathlineto{\pgfqpoint{7.654786in}{0.812682in}}%
\pgfpathlineto{\pgfqpoint{7.509206in}{0.812937in}}%
\pgfpathlineto{\pgfqpoint{7.363627in}{0.813046in}}%
\pgfpathlineto{\pgfqpoint{7.218047in}{0.812745in}}%
\pgfpathlineto{\pgfqpoint{7.072468in}{0.813534in}}%
\pgfpathlineto{\pgfqpoint{6.926889in}{0.813303in}}%
\pgfpathlineto{\pgfqpoint{6.781309in}{0.813749in}}%
\pgfpathlineto{\pgfqpoint{6.635730in}{0.818465in}}%
\pgfpathlineto{\pgfqpoint{6.490150in}{0.820639in}}%
\pgfpathlineto{\pgfqpoint{6.344571in}{0.819561in}}%
\pgfpathlineto{\pgfqpoint{6.198992in}{0.827116in}}%
\pgfpathlineto{\pgfqpoint{6.053412in}{0.822923in}}%
\pgfpathlineto{\pgfqpoint{5.907833in}{0.846066in}}%
\pgfpathlineto{\pgfqpoint{5.762253in}{0.841536in}}%
\pgfpathlineto{\pgfqpoint{5.616674in}{0.810368in}}%
\pgfpathlineto{\pgfqpoint{5.471095in}{0.829064in}}%
\pgfpathlineto{\pgfqpoint{5.325515in}{0.801600in}}%
\pgfpathlineto{\pgfqpoint{5.179936in}{0.811819in}}%
\pgfpathlineto{\pgfqpoint{5.034356in}{0.818201in}}%
\pgfpathlineto{\pgfqpoint{4.888777in}{0.806619in}}%
\pgfpathlineto{\pgfqpoint{4.743198in}{0.805616in}}%
\pgfpathclose%
\pgfusepath{stroke,fill}%
\end{pgfscope}%
\begin{pgfscope}%
\pgfsetbuttcap%
\pgfsetroundjoin%
\definecolor{currentfill}{rgb}{0.000000,0.000000,0.000000}%
\pgfsetfillcolor{currentfill}%
\pgfsetlinewidth{0.803000pt}%
\definecolor{currentstroke}{rgb}{0.000000,0.000000,0.000000}%
\pgfsetstrokecolor{currentstroke}%
\pgfsetdash{}{0pt}%
\pgfsys@defobject{currentmarker}{\pgfqpoint{0.000000in}{-0.048611in}}{\pgfqpoint{0.000000in}{0.000000in}}{%
\pgfpathmoveto{\pgfqpoint{0.000000in}{0.000000in}}%
\pgfpathlineto{\pgfqpoint{0.000000in}{-0.048611in}}%
\pgfusepath{stroke,fill}%
}%
\begin{pgfscope}%
\pgfsys@transformshift{5.093788in}{0.438889in}%
\pgfsys@useobject{currentmarker}{}%
\end{pgfscope}%
\end{pgfscope}%
\begin{pgfscope}%
\definecolor{textcolor}{rgb}{0.000000,0.000000,0.000000}%
\pgfsetstrokecolor{textcolor}%
\pgfsetfillcolor{textcolor}%
\pgftext[x=5.093788in,y=0.341667in,,top]{\color{textcolor}\sffamily\fontsize{10.000000}{12.000000}\selectfont \(\displaystyle {10^{4}}\)}%
\end{pgfscope}%
\begin{pgfscope}%
\pgfsetbuttcap%
\pgfsetroundjoin%
\definecolor{currentfill}{rgb}{0.000000,0.000000,0.000000}%
\pgfsetfillcolor{currentfill}%
\pgfsetlinewidth{0.803000pt}%
\definecolor{currentstroke}{rgb}{0.000000,0.000000,0.000000}%
\pgfsetstrokecolor{currentstroke}%
\pgfsetdash{}{0pt}%
\pgfsys@defobject{currentmarker}{\pgfqpoint{0.000000in}{-0.048611in}}{\pgfqpoint{0.000000in}{0.000000in}}{%
\pgfpathmoveto{\pgfqpoint{0.000000in}{0.000000in}}%
\pgfpathlineto{\pgfqpoint{0.000000in}{-0.048611in}}%
\pgfusepath{stroke,fill}%
}%
\begin{pgfscope}%
\pgfsys@transformshift{5.676105in}{0.438889in}%
\pgfsys@useobject{currentmarker}{}%
\end{pgfscope}%
\end{pgfscope}%
\begin{pgfscope}%
\definecolor{textcolor}{rgb}{0.000000,0.000000,0.000000}%
\pgfsetstrokecolor{textcolor}%
\pgfsetfillcolor{textcolor}%
\pgftext[x=5.676105in,y=0.341667in,,top]{\color{textcolor}\sffamily\fontsize{10.000000}{12.000000}\selectfont \(\displaystyle {10^{5}}\)}%
\end{pgfscope}%
\begin{pgfscope}%
\pgfsetbuttcap%
\pgfsetroundjoin%
\definecolor{currentfill}{rgb}{0.000000,0.000000,0.000000}%
\pgfsetfillcolor{currentfill}%
\pgfsetlinewidth{0.803000pt}%
\definecolor{currentstroke}{rgb}{0.000000,0.000000,0.000000}%
\pgfsetstrokecolor{currentstroke}%
\pgfsetdash{}{0pt}%
\pgfsys@defobject{currentmarker}{\pgfqpoint{0.000000in}{-0.048611in}}{\pgfqpoint{0.000000in}{0.000000in}}{%
\pgfpathmoveto{\pgfqpoint{0.000000in}{0.000000in}}%
\pgfpathlineto{\pgfqpoint{0.000000in}{-0.048611in}}%
\pgfusepath{stroke,fill}%
}%
\begin{pgfscope}%
\pgfsys@transformshift{6.258423in}{0.438889in}%
\pgfsys@useobject{currentmarker}{}%
\end{pgfscope}%
\end{pgfscope}%
\begin{pgfscope}%
\definecolor{textcolor}{rgb}{0.000000,0.000000,0.000000}%
\pgfsetstrokecolor{textcolor}%
\pgfsetfillcolor{textcolor}%
\pgftext[x=6.258423in,y=0.341667in,,top]{\color{textcolor}\sffamily\fontsize{10.000000}{12.000000}\selectfont \(\displaystyle {10^{6}}\)}%
\end{pgfscope}%
\begin{pgfscope}%
\pgfsetbuttcap%
\pgfsetroundjoin%
\definecolor{currentfill}{rgb}{0.000000,0.000000,0.000000}%
\pgfsetfillcolor{currentfill}%
\pgfsetlinewidth{0.803000pt}%
\definecolor{currentstroke}{rgb}{0.000000,0.000000,0.000000}%
\pgfsetstrokecolor{currentstroke}%
\pgfsetdash{}{0pt}%
\pgfsys@defobject{currentmarker}{\pgfqpoint{0.000000in}{-0.048611in}}{\pgfqpoint{0.000000in}{0.000000in}}{%
\pgfpathmoveto{\pgfqpoint{0.000000in}{0.000000in}}%
\pgfpathlineto{\pgfqpoint{0.000000in}{-0.048611in}}%
\pgfusepath{stroke,fill}%
}%
\begin{pgfscope}%
\pgfsys@transformshift{6.840741in}{0.438889in}%
\pgfsys@useobject{currentmarker}{}%
\end{pgfscope}%
\end{pgfscope}%
\begin{pgfscope}%
\definecolor{textcolor}{rgb}{0.000000,0.000000,0.000000}%
\pgfsetstrokecolor{textcolor}%
\pgfsetfillcolor{textcolor}%
\pgftext[x=6.840741in,y=0.341667in,,top]{\color{textcolor}\sffamily\fontsize{10.000000}{12.000000}\selectfont \(\displaystyle {10^{7}}\)}%
\end{pgfscope}%
\begin{pgfscope}%
\pgfsetbuttcap%
\pgfsetroundjoin%
\definecolor{currentfill}{rgb}{0.000000,0.000000,0.000000}%
\pgfsetfillcolor{currentfill}%
\pgfsetlinewidth{0.803000pt}%
\definecolor{currentstroke}{rgb}{0.000000,0.000000,0.000000}%
\pgfsetstrokecolor{currentstroke}%
\pgfsetdash{}{0pt}%
\pgfsys@defobject{currentmarker}{\pgfqpoint{0.000000in}{-0.048611in}}{\pgfqpoint{0.000000in}{0.000000in}}{%
\pgfpathmoveto{\pgfqpoint{0.000000in}{0.000000in}}%
\pgfpathlineto{\pgfqpoint{0.000000in}{-0.048611in}}%
\pgfusepath{stroke,fill}%
}%
\begin{pgfscope}%
\pgfsys@transformshift{7.423058in}{0.438889in}%
\pgfsys@useobject{currentmarker}{}%
\end{pgfscope}%
\end{pgfscope}%
\begin{pgfscope}%
\definecolor{textcolor}{rgb}{0.000000,0.000000,0.000000}%
\pgfsetstrokecolor{textcolor}%
\pgfsetfillcolor{textcolor}%
\pgftext[x=7.423058in,y=0.341667in,,top]{\color{textcolor}\sffamily\fontsize{10.000000}{12.000000}\selectfont \(\displaystyle {10^{8}}\)}%
\end{pgfscope}%
\begin{pgfscope}%
\pgfsetbuttcap%
\pgfsetroundjoin%
\definecolor{currentfill}{rgb}{0.000000,0.000000,0.000000}%
\pgfsetfillcolor{currentfill}%
\pgfsetlinewidth{0.602250pt}%
\definecolor{currentstroke}{rgb}{0.000000,0.000000,0.000000}%
\pgfsetstrokecolor{currentstroke}%
\pgfsetdash{}{0pt}%
\pgfsys@defobject{currentmarker}{\pgfqpoint{0.000000in}{-0.027778in}}{\pgfqpoint{0.000000in}{0.000000in}}{%
\pgfpathmoveto{\pgfqpoint{0.000000in}{0.000000in}}%
\pgfpathlineto{\pgfqpoint{0.000000in}{-0.027778in}}%
\pgfusepath{stroke,fill}%
}%
\begin{pgfscope}%
\pgfsys@transformshift{4.686765in}{0.438889in}%
\pgfsys@useobject{currentmarker}{}%
\end{pgfscope}%
\end{pgfscope}%
\begin{pgfscope}%
\pgfsetbuttcap%
\pgfsetroundjoin%
\definecolor{currentfill}{rgb}{0.000000,0.000000,0.000000}%
\pgfsetfillcolor{currentfill}%
\pgfsetlinewidth{0.602250pt}%
\definecolor{currentstroke}{rgb}{0.000000,0.000000,0.000000}%
\pgfsetstrokecolor{currentstroke}%
\pgfsetdash{}{0pt}%
\pgfsys@defobject{currentmarker}{\pgfqpoint{0.000000in}{-0.027778in}}{\pgfqpoint{0.000000in}{0.000000in}}{%
\pgfpathmoveto{\pgfqpoint{0.000000in}{0.000000in}}%
\pgfpathlineto{\pgfqpoint{0.000000in}{-0.027778in}}%
\pgfusepath{stroke,fill}%
}%
\begin{pgfscope}%
\pgfsys@transformshift{4.789306in}{0.438889in}%
\pgfsys@useobject{currentmarker}{}%
\end{pgfscope}%
\end{pgfscope}%
\begin{pgfscope}%
\pgfsetbuttcap%
\pgfsetroundjoin%
\definecolor{currentfill}{rgb}{0.000000,0.000000,0.000000}%
\pgfsetfillcolor{currentfill}%
\pgfsetlinewidth{0.602250pt}%
\definecolor{currentstroke}{rgb}{0.000000,0.000000,0.000000}%
\pgfsetstrokecolor{currentstroke}%
\pgfsetdash{}{0pt}%
\pgfsys@defobject{currentmarker}{\pgfqpoint{0.000000in}{-0.027778in}}{\pgfqpoint{0.000000in}{0.000000in}}{%
\pgfpathmoveto{\pgfqpoint{0.000000in}{0.000000in}}%
\pgfpathlineto{\pgfqpoint{0.000000in}{-0.027778in}}%
\pgfusepath{stroke,fill}%
}%
\begin{pgfscope}%
\pgfsys@transformshift{4.862060in}{0.438889in}%
\pgfsys@useobject{currentmarker}{}%
\end{pgfscope}%
\end{pgfscope}%
\begin{pgfscope}%
\pgfsetbuttcap%
\pgfsetroundjoin%
\definecolor{currentfill}{rgb}{0.000000,0.000000,0.000000}%
\pgfsetfillcolor{currentfill}%
\pgfsetlinewidth{0.602250pt}%
\definecolor{currentstroke}{rgb}{0.000000,0.000000,0.000000}%
\pgfsetstrokecolor{currentstroke}%
\pgfsetdash{}{0pt}%
\pgfsys@defobject{currentmarker}{\pgfqpoint{0.000000in}{-0.027778in}}{\pgfqpoint{0.000000in}{0.000000in}}{%
\pgfpathmoveto{\pgfqpoint{0.000000in}{0.000000in}}%
\pgfpathlineto{\pgfqpoint{0.000000in}{-0.027778in}}%
\pgfusepath{stroke,fill}%
}%
\begin{pgfscope}%
\pgfsys@transformshift{4.918493in}{0.438889in}%
\pgfsys@useobject{currentmarker}{}%
\end{pgfscope}%
\end{pgfscope}%
\begin{pgfscope}%
\pgfsetbuttcap%
\pgfsetroundjoin%
\definecolor{currentfill}{rgb}{0.000000,0.000000,0.000000}%
\pgfsetfillcolor{currentfill}%
\pgfsetlinewidth{0.602250pt}%
\definecolor{currentstroke}{rgb}{0.000000,0.000000,0.000000}%
\pgfsetstrokecolor{currentstroke}%
\pgfsetdash{}{0pt}%
\pgfsys@defobject{currentmarker}{\pgfqpoint{0.000000in}{-0.027778in}}{\pgfqpoint{0.000000in}{0.000000in}}{%
\pgfpathmoveto{\pgfqpoint{0.000000in}{0.000000in}}%
\pgfpathlineto{\pgfqpoint{0.000000in}{-0.027778in}}%
\pgfusepath{stroke,fill}%
}%
\begin{pgfscope}%
\pgfsys@transformshift{4.964601in}{0.438889in}%
\pgfsys@useobject{currentmarker}{}%
\end{pgfscope}%
\end{pgfscope}%
\begin{pgfscope}%
\pgfsetbuttcap%
\pgfsetroundjoin%
\definecolor{currentfill}{rgb}{0.000000,0.000000,0.000000}%
\pgfsetfillcolor{currentfill}%
\pgfsetlinewidth{0.602250pt}%
\definecolor{currentstroke}{rgb}{0.000000,0.000000,0.000000}%
\pgfsetstrokecolor{currentstroke}%
\pgfsetdash{}{0pt}%
\pgfsys@defobject{currentmarker}{\pgfqpoint{0.000000in}{-0.027778in}}{\pgfqpoint{0.000000in}{0.000000in}}{%
\pgfpathmoveto{\pgfqpoint{0.000000in}{0.000000in}}%
\pgfpathlineto{\pgfqpoint{0.000000in}{-0.027778in}}%
\pgfusepath{stroke,fill}%
}%
\begin{pgfscope}%
\pgfsys@transformshift{5.003586in}{0.438889in}%
\pgfsys@useobject{currentmarker}{}%
\end{pgfscope}%
\end{pgfscope}%
\begin{pgfscope}%
\pgfsetbuttcap%
\pgfsetroundjoin%
\definecolor{currentfill}{rgb}{0.000000,0.000000,0.000000}%
\pgfsetfillcolor{currentfill}%
\pgfsetlinewidth{0.602250pt}%
\definecolor{currentstroke}{rgb}{0.000000,0.000000,0.000000}%
\pgfsetstrokecolor{currentstroke}%
\pgfsetdash{}{0pt}%
\pgfsys@defobject{currentmarker}{\pgfqpoint{0.000000in}{-0.027778in}}{\pgfqpoint{0.000000in}{0.000000in}}{%
\pgfpathmoveto{\pgfqpoint{0.000000in}{0.000000in}}%
\pgfpathlineto{\pgfqpoint{0.000000in}{-0.027778in}}%
\pgfusepath{stroke,fill}%
}%
\begin{pgfscope}%
\pgfsys@transformshift{5.037355in}{0.438889in}%
\pgfsys@useobject{currentmarker}{}%
\end{pgfscope}%
\end{pgfscope}%
\begin{pgfscope}%
\pgfsetbuttcap%
\pgfsetroundjoin%
\definecolor{currentfill}{rgb}{0.000000,0.000000,0.000000}%
\pgfsetfillcolor{currentfill}%
\pgfsetlinewidth{0.602250pt}%
\definecolor{currentstroke}{rgb}{0.000000,0.000000,0.000000}%
\pgfsetstrokecolor{currentstroke}%
\pgfsetdash{}{0pt}%
\pgfsys@defobject{currentmarker}{\pgfqpoint{0.000000in}{-0.027778in}}{\pgfqpoint{0.000000in}{0.000000in}}{%
\pgfpathmoveto{\pgfqpoint{0.000000in}{0.000000in}}%
\pgfpathlineto{\pgfqpoint{0.000000in}{-0.027778in}}%
\pgfusepath{stroke,fill}%
}%
\begin{pgfscope}%
\pgfsys@transformshift{5.067142in}{0.438889in}%
\pgfsys@useobject{currentmarker}{}%
\end{pgfscope}%
\end{pgfscope}%
\begin{pgfscope}%
\pgfsetbuttcap%
\pgfsetroundjoin%
\definecolor{currentfill}{rgb}{0.000000,0.000000,0.000000}%
\pgfsetfillcolor{currentfill}%
\pgfsetlinewidth{0.602250pt}%
\definecolor{currentstroke}{rgb}{0.000000,0.000000,0.000000}%
\pgfsetstrokecolor{currentstroke}%
\pgfsetdash{}{0pt}%
\pgfsys@defobject{currentmarker}{\pgfqpoint{0.000000in}{-0.027778in}}{\pgfqpoint{0.000000in}{0.000000in}}{%
\pgfpathmoveto{\pgfqpoint{0.000000in}{0.000000in}}%
\pgfpathlineto{\pgfqpoint{0.000000in}{-0.027778in}}%
\pgfusepath{stroke,fill}%
}%
\begin{pgfscope}%
\pgfsys@transformshift{5.269083in}{0.438889in}%
\pgfsys@useobject{currentmarker}{}%
\end{pgfscope}%
\end{pgfscope}%
\begin{pgfscope}%
\pgfsetbuttcap%
\pgfsetroundjoin%
\definecolor{currentfill}{rgb}{0.000000,0.000000,0.000000}%
\pgfsetfillcolor{currentfill}%
\pgfsetlinewidth{0.602250pt}%
\definecolor{currentstroke}{rgb}{0.000000,0.000000,0.000000}%
\pgfsetstrokecolor{currentstroke}%
\pgfsetdash{}{0pt}%
\pgfsys@defobject{currentmarker}{\pgfqpoint{0.000000in}{-0.027778in}}{\pgfqpoint{0.000000in}{0.000000in}}{%
\pgfpathmoveto{\pgfqpoint{0.000000in}{0.000000in}}%
\pgfpathlineto{\pgfqpoint{0.000000in}{-0.027778in}}%
\pgfusepath{stroke,fill}%
}%
\begin{pgfscope}%
\pgfsys@transformshift{5.371624in}{0.438889in}%
\pgfsys@useobject{currentmarker}{}%
\end{pgfscope}%
\end{pgfscope}%
\begin{pgfscope}%
\pgfsetbuttcap%
\pgfsetroundjoin%
\definecolor{currentfill}{rgb}{0.000000,0.000000,0.000000}%
\pgfsetfillcolor{currentfill}%
\pgfsetlinewidth{0.602250pt}%
\definecolor{currentstroke}{rgb}{0.000000,0.000000,0.000000}%
\pgfsetstrokecolor{currentstroke}%
\pgfsetdash{}{0pt}%
\pgfsys@defobject{currentmarker}{\pgfqpoint{0.000000in}{-0.027778in}}{\pgfqpoint{0.000000in}{0.000000in}}{%
\pgfpathmoveto{\pgfqpoint{0.000000in}{0.000000in}}%
\pgfpathlineto{\pgfqpoint{0.000000in}{-0.027778in}}%
\pgfusepath{stroke,fill}%
}%
\begin{pgfscope}%
\pgfsys@transformshift{5.444378in}{0.438889in}%
\pgfsys@useobject{currentmarker}{}%
\end{pgfscope}%
\end{pgfscope}%
\begin{pgfscope}%
\pgfsetbuttcap%
\pgfsetroundjoin%
\definecolor{currentfill}{rgb}{0.000000,0.000000,0.000000}%
\pgfsetfillcolor{currentfill}%
\pgfsetlinewidth{0.602250pt}%
\definecolor{currentstroke}{rgb}{0.000000,0.000000,0.000000}%
\pgfsetstrokecolor{currentstroke}%
\pgfsetdash{}{0pt}%
\pgfsys@defobject{currentmarker}{\pgfqpoint{0.000000in}{-0.027778in}}{\pgfqpoint{0.000000in}{0.000000in}}{%
\pgfpathmoveto{\pgfqpoint{0.000000in}{0.000000in}}%
\pgfpathlineto{\pgfqpoint{0.000000in}{-0.027778in}}%
\pgfusepath{stroke,fill}%
}%
\begin{pgfscope}%
\pgfsys@transformshift{5.500810in}{0.438889in}%
\pgfsys@useobject{currentmarker}{}%
\end{pgfscope}%
\end{pgfscope}%
\begin{pgfscope}%
\pgfsetbuttcap%
\pgfsetroundjoin%
\definecolor{currentfill}{rgb}{0.000000,0.000000,0.000000}%
\pgfsetfillcolor{currentfill}%
\pgfsetlinewidth{0.602250pt}%
\definecolor{currentstroke}{rgb}{0.000000,0.000000,0.000000}%
\pgfsetstrokecolor{currentstroke}%
\pgfsetdash{}{0pt}%
\pgfsys@defobject{currentmarker}{\pgfqpoint{0.000000in}{-0.027778in}}{\pgfqpoint{0.000000in}{0.000000in}}{%
\pgfpathmoveto{\pgfqpoint{0.000000in}{0.000000in}}%
\pgfpathlineto{\pgfqpoint{0.000000in}{-0.027778in}}%
\pgfusepath{stroke,fill}%
}%
\begin{pgfscope}%
\pgfsys@transformshift{5.546919in}{0.438889in}%
\pgfsys@useobject{currentmarker}{}%
\end{pgfscope}%
\end{pgfscope}%
\begin{pgfscope}%
\pgfsetbuttcap%
\pgfsetroundjoin%
\definecolor{currentfill}{rgb}{0.000000,0.000000,0.000000}%
\pgfsetfillcolor{currentfill}%
\pgfsetlinewidth{0.602250pt}%
\definecolor{currentstroke}{rgb}{0.000000,0.000000,0.000000}%
\pgfsetstrokecolor{currentstroke}%
\pgfsetdash{}{0pt}%
\pgfsys@defobject{currentmarker}{\pgfqpoint{0.000000in}{-0.027778in}}{\pgfqpoint{0.000000in}{0.000000in}}{%
\pgfpathmoveto{\pgfqpoint{0.000000in}{0.000000in}}%
\pgfpathlineto{\pgfqpoint{0.000000in}{-0.027778in}}%
\pgfusepath{stroke,fill}%
}%
\begin{pgfscope}%
\pgfsys@transformshift{5.585903in}{0.438889in}%
\pgfsys@useobject{currentmarker}{}%
\end{pgfscope}%
\end{pgfscope}%
\begin{pgfscope}%
\pgfsetbuttcap%
\pgfsetroundjoin%
\definecolor{currentfill}{rgb}{0.000000,0.000000,0.000000}%
\pgfsetfillcolor{currentfill}%
\pgfsetlinewidth{0.602250pt}%
\definecolor{currentstroke}{rgb}{0.000000,0.000000,0.000000}%
\pgfsetstrokecolor{currentstroke}%
\pgfsetdash{}{0pt}%
\pgfsys@defobject{currentmarker}{\pgfqpoint{0.000000in}{-0.027778in}}{\pgfqpoint{0.000000in}{0.000000in}}{%
\pgfpathmoveto{\pgfqpoint{0.000000in}{0.000000in}}%
\pgfpathlineto{\pgfqpoint{0.000000in}{-0.027778in}}%
\pgfusepath{stroke,fill}%
}%
\begin{pgfscope}%
\pgfsys@transformshift{5.619673in}{0.438889in}%
\pgfsys@useobject{currentmarker}{}%
\end{pgfscope}%
\end{pgfscope}%
\begin{pgfscope}%
\pgfsetbuttcap%
\pgfsetroundjoin%
\definecolor{currentfill}{rgb}{0.000000,0.000000,0.000000}%
\pgfsetfillcolor{currentfill}%
\pgfsetlinewidth{0.602250pt}%
\definecolor{currentstroke}{rgb}{0.000000,0.000000,0.000000}%
\pgfsetstrokecolor{currentstroke}%
\pgfsetdash{}{0pt}%
\pgfsys@defobject{currentmarker}{\pgfqpoint{0.000000in}{-0.027778in}}{\pgfqpoint{0.000000in}{0.000000in}}{%
\pgfpathmoveto{\pgfqpoint{0.000000in}{0.000000in}}%
\pgfpathlineto{\pgfqpoint{0.000000in}{-0.027778in}}%
\pgfusepath{stroke,fill}%
}%
\begin{pgfscope}%
\pgfsys@transformshift{5.649460in}{0.438889in}%
\pgfsys@useobject{currentmarker}{}%
\end{pgfscope}%
\end{pgfscope}%
\begin{pgfscope}%
\pgfsetbuttcap%
\pgfsetroundjoin%
\definecolor{currentfill}{rgb}{0.000000,0.000000,0.000000}%
\pgfsetfillcolor{currentfill}%
\pgfsetlinewidth{0.602250pt}%
\definecolor{currentstroke}{rgb}{0.000000,0.000000,0.000000}%
\pgfsetstrokecolor{currentstroke}%
\pgfsetdash{}{0pt}%
\pgfsys@defobject{currentmarker}{\pgfqpoint{0.000000in}{-0.027778in}}{\pgfqpoint{0.000000in}{0.000000in}}{%
\pgfpathmoveto{\pgfqpoint{0.000000in}{0.000000in}}%
\pgfpathlineto{\pgfqpoint{0.000000in}{-0.027778in}}%
\pgfusepath{stroke,fill}%
}%
\begin{pgfscope}%
\pgfsys@transformshift{5.851400in}{0.438889in}%
\pgfsys@useobject{currentmarker}{}%
\end{pgfscope}%
\end{pgfscope}%
\begin{pgfscope}%
\pgfsetbuttcap%
\pgfsetroundjoin%
\definecolor{currentfill}{rgb}{0.000000,0.000000,0.000000}%
\pgfsetfillcolor{currentfill}%
\pgfsetlinewidth{0.602250pt}%
\definecolor{currentstroke}{rgb}{0.000000,0.000000,0.000000}%
\pgfsetstrokecolor{currentstroke}%
\pgfsetdash{}{0pt}%
\pgfsys@defobject{currentmarker}{\pgfqpoint{0.000000in}{-0.027778in}}{\pgfqpoint{0.000000in}{0.000000in}}{%
\pgfpathmoveto{\pgfqpoint{0.000000in}{0.000000in}}%
\pgfpathlineto{\pgfqpoint{0.000000in}{-0.027778in}}%
\pgfusepath{stroke,fill}%
}%
\begin{pgfscope}%
\pgfsys@transformshift{5.953941in}{0.438889in}%
\pgfsys@useobject{currentmarker}{}%
\end{pgfscope}%
\end{pgfscope}%
\begin{pgfscope}%
\pgfsetbuttcap%
\pgfsetroundjoin%
\definecolor{currentfill}{rgb}{0.000000,0.000000,0.000000}%
\pgfsetfillcolor{currentfill}%
\pgfsetlinewidth{0.602250pt}%
\definecolor{currentstroke}{rgb}{0.000000,0.000000,0.000000}%
\pgfsetstrokecolor{currentstroke}%
\pgfsetdash{}{0pt}%
\pgfsys@defobject{currentmarker}{\pgfqpoint{0.000000in}{-0.027778in}}{\pgfqpoint{0.000000in}{0.000000in}}{%
\pgfpathmoveto{\pgfqpoint{0.000000in}{0.000000in}}%
\pgfpathlineto{\pgfqpoint{0.000000in}{-0.027778in}}%
\pgfusepath{stroke,fill}%
}%
\begin{pgfscope}%
\pgfsys@transformshift{6.026696in}{0.438889in}%
\pgfsys@useobject{currentmarker}{}%
\end{pgfscope}%
\end{pgfscope}%
\begin{pgfscope}%
\pgfsetbuttcap%
\pgfsetroundjoin%
\definecolor{currentfill}{rgb}{0.000000,0.000000,0.000000}%
\pgfsetfillcolor{currentfill}%
\pgfsetlinewidth{0.602250pt}%
\definecolor{currentstroke}{rgb}{0.000000,0.000000,0.000000}%
\pgfsetstrokecolor{currentstroke}%
\pgfsetdash{}{0pt}%
\pgfsys@defobject{currentmarker}{\pgfqpoint{0.000000in}{-0.027778in}}{\pgfqpoint{0.000000in}{0.000000in}}{%
\pgfpathmoveto{\pgfqpoint{0.000000in}{0.000000in}}%
\pgfpathlineto{\pgfqpoint{0.000000in}{-0.027778in}}%
\pgfusepath{stroke,fill}%
}%
\begin{pgfscope}%
\pgfsys@transformshift{6.083128in}{0.438889in}%
\pgfsys@useobject{currentmarker}{}%
\end{pgfscope}%
\end{pgfscope}%
\begin{pgfscope}%
\pgfsetbuttcap%
\pgfsetroundjoin%
\definecolor{currentfill}{rgb}{0.000000,0.000000,0.000000}%
\pgfsetfillcolor{currentfill}%
\pgfsetlinewidth{0.602250pt}%
\definecolor{currentstroke}{rgb}{0.000000,0.000000,0.000000}%
\pgfsetstrokecolor{currentstroke}%
\pgfsetdash{}{0pt}%
\pgfsys@defobject{currentmarker}{\pgfqpoint{0.000000in}{-0.027778in}}{\pgfqpoint{0.000000in}{0.000000in}}{%
\pgfpathmoveto{\pgfqpoint{0.000000in}{0.000000in}}%
\pgfpathlineto{\pgfqpoint{0.000000in}{-0.027778in}}%
\pgfusepath{stroke,fill}%
}%
\begin{pgfscope}%
\pgfsys@transformshift{6.129237in}{0.438889in}%
\pgfsys@useobject{currentmarker}{}%
\end{pgfscope}%
\end{pgfscope}%
\begin{pgfscope}%
\pgfsetbuttcap%
\pgfsetroundjoin%
\definecolor{currentfill}{rgb}{0.000000,0.000000,0.000000}%
\pgfsetfillcolor{currentfill}%
\pgfsetlinewidth{0.602250pt}%
\definecolor{currentstroke}{rgb}{0.000000,0.000000,0.000000}%
\pgfsetstrokecolor{currentstroke}%
\pgfsetdash{}{0pt}%
\pgfsys@defobject{currentmarker}{\pgfqpoint{0.000000in}{-0.027778in}}{\pgfqpoint{0.000000in}{0.000000in}}{%
\pgfpathmoveto{\pgfqpoint{0.000000in}{0.000000in}}%
\pgfpathlineto{\pgfqpoint{0.000000in}{-0.027778in}}%
\pgfusepath{stroke,fill}%
}%
\begin{pgfscope}%
\pgfsys@transformshift{6.168221in}{0.438889in}%
\pgfsys@useobject{currentmarker}{}%
\end{pgfscope}%
\end{pgfscope}%
\begin{pgfscope}%
\pgfsetbuttcap%
\pgfsetroundjoin%
\definecolor{currentfill}{rgb}{0.000000,0.000000,0.000000}%
\pgfsetfillcolor{currentfill}%
\pgfsetlinewidth{0.602250pt}%
\definecolor{currentstroke}{rgb}{0.000000,0.000000,0.000000}%
\pgfsetstrokecolor{currentstroke}%
\pgfsetdash{}{0pt}%
\pgfsys@defobject{currentmarker}{\pgfqpoint{0.000000in}{-0.027778in}}{\pgfqpoint{0.000000in}{0.000000in}}{%
\pgfpathmoveto{\pgfqpoint{0.000000in}{0.000000in}}%
\pgfpathlineto{\pgfqpoint{0.000000in}{-0.027778in}}%
\pgfusepath{stroke,fill}%
}%
\begin{pgfscope}%
\pgfsys@transformshift{6.201991in}{0.438889in}%
\pgfsys@useobject{currentmarker}{}%
\end{pgfscope}%
\end{pgfscope}%
\begin{pgfscope}%
\pgfsetbuttcap%
\pgfsetroundjoin%
\definecolor{currentfill}{rgb}{0.000000,0.000000,0.000000}%
\pgfsetfillcolor{currentfill}%
\pgfsetlinewidth{0.602250pt}%
\definecolor{currentstroke}{rgb}{0.000000,0.000000,0.000000}%
\pgfsetstrokecolor{currentstroke}%
\pgfsetdash{}{0pt}%
\pgfsys@defobject{currentmarker}{\pgfqpoint{0.000000in}{-0.027778in}}{\pgfqpoint{0.000000in}{0.000000in}}{%
\pgfpathmoveto{\pgfqpoint{0.000000in}{0.000000in}}%
\pgfpathlineto{\pgfqpoint{0.000000in}{-0.027778in}}%
\pgfusepath{stroke,fill}%
}%
\begin{pgfscope}%
\pgfsys@transformshift{6.231778in}{0.438889in}%
\pgfsys@useobject{currentmarker}{}%
\end{pgfscope}%
\end{pgfscope}%
\begin{pgfscope}%
\pgfsetbuttcap%
\pgfsetroundjoin%
\definecolor{currentfill}{rgb}{0.000000,0.000000,0.000000}%
\pgfsetfillcolor{currentfill}%
\pgfsetlinewidth{0.602250pt}%
\definecolor{currentstroke}{rgb}{0.000000,0.000000,0.000000}%
\pgfsetstrokecolor{currentstroke}%
\pgfsetdash{}{0pt}%
\pgfsys@defobject{currentmarker}{\pgfqpoint{0.000000in}{-0.027778in}}{\pgfqpoint{0.000000in}{0.000000in}}{%
\pgfpathmoveto{\pgfqpoint{0.000000in}{0.000000in}}%
\pgfpathlineto{\pgfqpoint{0.000000in}{-0.027778in}}%
\pgfusepath{stroke,fill}%
}%
\begin{pgfscope}%
\pgfsys@transformshift{6.433718in}{0.438889in}%
\pgfsys@useobject{currentmarker}{}%
\end{pgfscope}%
\end{pgfscope}%
\begin{pgfscope}%
\pgfsetbuttcap%
\pgfsetroundjoin%
\definecolor{currentfill}{rgb}{0.000000,0.000000,0.000000}%
\pgfsetfillcolor{currentfill}%
\pgfsetlinewidth{0.602250pt}%
\definecolor{currentstroke}{rgb}{0.000000,0.000000,0.000000}%
\pgfsetstrokecolor{currentstroke}%
\pgfsetdash{}{0pt}%
\pgfsys@defobject{currentmarker}{\pgfqpoint{0.000000in}{-0.027778in}}{\pgfqpoint{0.000000in}{0.000000in}}{%
\pgfpathmoveto{\pgfqpoint{0.000000in}{0.000000in}}%
\pgfpathlineto{\pgfqpoint{0.000000in}{-0.027778in}}%
\pgfusepath{stroke,fill}%
}%
\begin{pgfscope}%
\pgfsys@transformshift{6.536259in}{0.438889in}%
\pgfsys@useobject{currentmarker}{}%
\end{pgfscope}%
\end{pgfscope}%
\begin{pgfscope}%
\pgfsetbuttcap%
\pgfsetroundjoin%
\definecolor{currentfill}{rgb}{0.000000,0.000000,0.000000}%
\pgfsetfillcolor{currentfill}%
\pgfsetlinewidth{0.602250pt}%
\definecolor{currentstroke}{rgb}{0.000000,0.000000,0.000000}%
\pgfsetstrokecolor{currentstroke}%
\pgfsetdash{}{0pt}%
\pgfsys@defobject{currentmarker}{\pgfqpoint{0.000000in}{-0.027778in}}{\pgfqpoint{0.000000in}{0.000000in}}{%
\pgfpathmoveto{\pgfqpoint{0.000000in}{0.000000in}}%
\pgfpathlineto{\pgfqpoint{0.000000in}{-0.027778in}}%
\pgfusepath{stroke,fill}%
}%
\begin{pgfscope}%
\pgfsys@transformshift{6.609013in}{0.438889in}%
\pgfsys@useobject{currentmarker}{}%
\end{pgfscope}%
\end{pgfscope}%
\begin{pgfscope}%
\pgfsetbuttcap%
\pgfsetroundjoin%
\definecolor{currentfill}{rgb}{0.000000,0.000000,0.000000}%
\pgfsetfillcolor{currentfill}%
\pgfsetlinewidth{0.602250pt}%
\definecolor{currentstroke}{rgb}{0.000000,0.000000,0.000000}%
\pgfsetstrokecolor{currentstroke}%
\pgfsetdash{}{0pt}%
\pgfsys@defobject{currentmarker}{\pgfqpoint{0.000000in}{-0.027778in}}{\pgfqpoint{0.000000in}{0.000000in}}{%
\pgfpathmoveto{\pgfqpoint{0.000000in}{0.000000in}}%
\pgfpathlineto{\pgfqpoint{0.000000in}{-0.027778in}}%
\pgfusepath{stroke,fill}%
}%
\begin{pgfscope}%
\pgfsys@transformshift{6.665446in}{0.438889in}%
\pgfsys@useobject{currentmarker}{}%
\end{pgfscope}%
\end{pgfscope}%
\begin{pgfscope}%
\pgfsetbuttcap%
\pgfsetroundjoin%
\definecolor{currentfill}{rgb}{0.000000,0.000000,0.000000}%
\pgfsetfillcolor{currentfill}%
\pgfsetlinewidth{0.602250pt}%
\definecolor{currentstroke}{rgb}{0.000000,0.000000,0.000000}%
\pgfsetstrokecolor{currentstroke}%
\pgfsetdash{}{0pt}%
\pgfsys@defobject{currentmarker}{\pgfqpoint{0.000000in}{-0.027778in}}{\pgfqpoint{0.000000in}{0.000000in}}{%
\pgfpathmoveto{\pgfqpoint{0.000000in}{0.000000in}}%
\pgfpathlineto{\pgfqpoint{0.000000in}{-0.027778in}}%
\pgfusepath{stroke,fill}%
}%
\begin{pgfscope}%
\pgfsys@transformshift{6.711554in}{0.438889in}%
\pgfsys@useobject{currentmarker}{}%
\end{pgfscope}%
\end{pgfscope}%
\begin{pgfscope}%
\pgfsetbuttcap%
\pgfsetroundjoin%
\definecolor{currentfill}{rgb}{0.000000,0.000000,0.000000}%
\pgfsetfillcolor{currentfill}%
\pgfsetlinewidth{0.602250pt}%
\definecolor{currentstroke}{rgb}{0.000000,0.000000,0.000000}%
\pgfsetstrokecolor{currentstroke}%
\pgfsetdash{}{0pt}%
\pgfsys@defobject{currentmarker}{\pgfqpoint{0.000000in}{-0.027778in}}{\pgfqpoint{0.000000in}{0.000000in}}{%
\pgfpathmoveto{\pgfqpoint{0.000000in}{0.000000in}}%
\pgfpathlineto{\pgfqpoint{0.000000in}{-0.027778in}}%
\pgfusepath{stroke,fill}%
}%
\begin{pgfscope}%
\pgfsys@transformshift{6.750538in}{0.438889in}%
\pgfsys@useobject{currentmarker}{}%
\end{pgfscope}%
\end{pgfscope}%
\begin{pgfscope}%
\pgfsetbuttcap%
\pgfsetroundjoin%
\definecolor{currentfill}{rgb}{0.000000,0.000000,0.000000}%
\pgfsetfillcolor{currentfill}%
\pgfsetlinewidth{0.602250pt}%
\definecolor{currentstroke}{rgb}{0.000000,0.000000,0.000000}%
\pgfsetstrokecolor{currentstroke}%
\pgfsetdash{}{0pt}%
\pgfsys@defobject{currentmarker}{\pgfqpoint{0.000000in}{-0.027778in}}{\pgfqpoint{0.000000in}{0.000000in}}{%
\pgfpathmoveto{\pgfqpoint{0.000000in}{0.000000in}}%
\pgfpathlineto{\pgfqpoint{0.000000in}{-0.027778in}}%
\pgfusepath{stroke,fill}%
}%
\begin{pgfscope}%
\pgfsys@transformshift{6.784308in}{0.438889in}%
\pgfsys@useobject{currentmarker}{}%
\end{pgfscope}%
\end{pgfscope}%
\begin{pgfscope}%
\pgfsetbuttcap%
\pgfsetroundjoin%
\definecolor{currentfill}{rgb}{0.000000,0.000000,0.000000}%
\pgfsetfillcolor{currentfill}%
\pgfsetlinewidth{0.602250pt}%
\definecolor{currentstroke}{rgb}{0.000000,0.000000,0.000000}%
\pgfsetstrokecolor{currentstroke}%
\pgfsetdash{}{0pt}%
\pgfsys@defobject{currentmarker}{\pgfqpoint{0.000000in}{-0.027778in}}{\pgfqpoint{0.000000in}{0.000000in}}{%
\pgfpathmoveto{\pgfqpoint{0.000000in}{0.000000in}}%
\pgfpathlineto{\pgfqpoint{0.000000in}{-0.027778in}}%
\pgfusepath{stroke,fill}%
}%
\begin{pgfscope}%
\pgfsys@transformshift{6.814095in}{0.438889in}%
\pgfsys@useobject{currentmarker}{}%
\end{pgfscope}%
\end{pgfscope}%
\begin{pgfscope}%
\pgfsetbuttcap%
\pgfsetroundjoin%
\definecolor{currentfill}{rgb}{0.000000,0.000000,0.000000}%
\pgfsetfillcolor{currentfill}%
\pgfsetlinewidth{0.602250pt}%
\definecolor{currentstroke}{rgb}{0.000000,0.000000,0.000000}%
\pgfsetstrokecolor{currentstroke}%
\pgfsetdash{}{0pt}%
\pgfsys@defobject{currentmarker}{\pgfqpoint{0.000000in}{-0.027778in}}{\pgfqpoint{0.000000in}{0.000000in}}{%
\pgfpathmoveto{\pgfqpoint{0.000000in}{0.000000in}}%
\pgfpathlineto{\pgfqpoint{0.000000in}{-0.027778in}}%
\pgfusepath{stroke,fill}%
}%
\begin{pgfscope}%
\pgfsys@transformshift{7.016036in}{0.438889in}%
\pgfsys@useobject{currentmarker}{}%
\end{pgfscope}%
\end{pgfscope}%
\begin{pgfscope}%
\pgfsetbuttcap%
\pgfsetroundjoin%
\definecolor{currentfill}{rgb}{0.000000,0.000000,0.000000}%
\pgfsetfillcolor{currentfill}%
\pgfsetlinewidth{0.602250pt}%
\definecolor{currentstroke}{rgb}{0.000000,0.000000,0.000000}%
\pgfsetstrokecolor{currentstroke}%
\pgfsetdash{}{0pt}%
\pgfsys@defobject{currentmarker}{\pgfqpoint{0.000000in}{-0.027778in}}{\pgfqpoint{0.000000in}{0.000000in}}{%
\pgfpathmoveto{\pgfqpoint{0.000000in}{0.000000in}}%
\pgfpathlineto{\pgfqpoint{0.000000in}{-0.027778in}}%
\pgfusepath{stroke,fill}%
}%
\begin{pgfscope}%
\pgfsys@transformshift{7.118577in}{0.438889in}%
\pgfsys@useobject{currentmarker}{}%
\end{pgfscope}%
\end{pgfscope}%
\begin{pgfscope}%
\pgfsetbuttcap%
\pgfsetroundjoin%
\definecolor{currentfill}{rgb}{0.000000,0.000000,0.000000}%
\pgfsetfillcolor{currentfill}%
\pgfsetlinewidth{0.602250pt}%
\definecolor{currentstroke}{rgb}{0.000000,0.000000,0.000000}%
\pgfsetstrokecolor{currentstroke}%
\pgfsetdash{}{0pt}%
\pgfsys@defobject{currentmarker}{\pgfqpoint{0.000000in}{-0.027778in}}{\pgfqpoint{0.000000in}{0.000000in}}{%
\pgfpathmoveto{\pgfqpoint{0.000000in}{0.000000in}}%
\pgfpathlineto{\pgfqpoint{0.000000in}{-0.027778in}}%
\pgfusepath{stroke,fill}%
}%
\begin{pgfscope}%
\pgfsys@transformshift{7.191331in}{0.438889in}%
\pgfsys@useobject{currentmarker}{}%
\end{pgfscope}%
\end{pgfscope}%
\begin{pgfscope}%
\pgfsetbuttcap%
\pgfsetroundjoin%
\definecolor{currentfill}{rgb}{0.000000,0.000000,0.000000}%
\pgfsetfillcolor{currentfill}%
\pgfsetlinewidth{0.602250pt}%
\definecolor{currentstroke}{rgb}{0.000000,0.000000,0.000000}%
\pgfsetstrokecolor{currentstroke}%
\pgfsetdash{}{0pt}%
\pgfsys@defobject{currentmarker}{\pgfqpoint{0.000000in}{-0.027778in}}{\pgfqpoint{0.000000in}{0.000000in}}{%
\pgfpathmoveto{\pgfqpoint{0.000000in}{0.000000in}}%
\pgfpathlineto{\pgfqpoint{0.000000in}{-0.027778in}}%
\pgfusepath{stroke,fill}%
}%
\begin{pgfscope}%
\pgfsys@transformshift{7.247763in}{0.438889in}%
\pgfsys@useobject{currentmarker}{}%
\end{pgfscope}%
\end{pgfscope}%
\begin{pgfscope}%
\pgfsetbuttcap%
\pgfsetroundjoin%
\definecolor{currentfill}{rgb}{0.000000,0.000000,0.000000}%
\pgfsetfillcolor{currentfill}%
\pgfsetlinewidth{0.602250pt}%
\definecolor{currentstroke}{rgb}{0.000000,0.000000,0.000000}%
\pgfsetstrokecolor{currentstroke}%
\pgfsetdash{}{0pt}%
\pgfsys@defobject{currentmarker}{\pgfqpoint{0.000000in}{-0.027778in}}{\pgfqpoint{0.000000in}{0.000000in}}{%
\pgfpathmoveto{\pgfqpoint{0.000000in}{0.000000in}}%
\pgfpathlineto{\pgfqpoint{0.000000in}{-0.027778in}}%
\pgfusepath{stroke,fill}%
}%
\begin{pgfscope}%
\pgfsys@transformshift{7.293872in}{0.438889in}%
\pgfsys@useobject{currentmarker}{}%
\end{pgfscope}%
\end{pgfscope}%
\begin{pgfscope}%
\pgfsetbuttcap%
\pgfsetroundjoin%
\definecolor{currentfill}{rgb}{0.000000,0.000000,0.000000}%
\pgfsetfillcolor{currentfill}%
\pgfsetlinewidth{0.602250pt}%
\definecolor{currentstroke}{rgb}{0.000000,0.000000,0.000000}%
\pgfsetstrokecolor{currentstroke}%
\pgfsetdash{}{0pt}%
\pgfsys@defobject{currentmarker}{\pgfqpoint{0.000000in}{-0.027778in}}{\pgfqpoint{0.000000in}{0.000000in}}{%
\pgfpathmoveto{\pgfqpoint{0.000000in}{0.000000in}}%
\pgfpathlineto{\pgfqpoint{0.000000in}{-0.027778in}}%
\pgfusepath{stroke,fill}%
}%
\begin{pgfscope}%
\pgfsys@transformshift{7.332856in}{0.438889in}%
\pgfsys@useobject{currentmarker}{}%
\end{pgfscope}%
\end{pgfscope}%
\begin{pgfscope}%
\pgfsetbuttcap%
\pgfsetroundjoin%
\definecolor{currentfill}{rgb}{0.000000,0.000000,0.000000}%
\pgfsetfillcolor{currentfill}%
\pgfsetlinewidth{0.602250pt}%
\definecolor{currentstroke}{rgb}{0.000000,0.000000,0.000000}%
\pgfsetstrokecolor{currentstroke}%
\pgfsetdash{}{0pt}%
\pgfsys@defobject{currentmarker}{\pgfqpoint{0.000000in}{-0.027778in}}{\pgfqpoint{0.000000in}{0.000000in}}{%
\pgfpathmoveto{\pgfqpoint{0.000000in}{0.000000in}}%
\pgfpathlineto{\pgfqpoint{0.000000in}{-0.027778in}}%
\pgfusepath{stroke,fill}%
}%
\begin{pgfscope}%
\pgfsys@transformshift{7.366626in}{0.438889in}%
\pgfsys@useobject{currentmarker}{}%
\end{pgfscope}%
\end{pgfscope}%
\begin{pgfscope}%
\pgfsetbuttcap%
\pgfsetroundjoin%
\definecolor{currentfill}{rgb}{0.000000,0.000000,0.000000}%
\pgfsetfillcolor{currentfill}%
\pgfsetlinewidth{0.602250pt}%
\definecolor{currentstroke}{rgb}{0.000000,0.000000,0.000000}%
\pgfsetstrokecolor{currentstroke}%
\pgfsetdash{}{0pt}%
\pgfsys@defobject{currentmarker}{\pgfqpoint{0.000000in}{-0.027778in}}{\pgfqpoint{0.000000in}{0.000000in}}{%
\pgfpathmoveto{\pgfqpoint{0.000000in}{0.000000in}}%
\pgfpathlineto{\pgfqpoint{0.000000in}{-0.027778in}}%
\pgfusepath{stroke,fill}%
}%
\begin{pgfscope}%
\pgfsys@transformshift{7.396413in}{0.438889in}%
\pgfsys@useobject{currentmarker}{}%
\end{pgfscope}%
\end{pgfscope}%
\begin{pgfscope}%
\pgfsetbuttcap%
\pgfsetroundjoin%
\definecolor{currentfill}{rgb}{0.000000,0.000000,0.000000}%
\pgfsetfillcolor{currentfill}%
\pgfsetlinewidth{0.602250pt}%
\definecolor{currentstroke}{rgb}{0.000000,0.000000,0.000000}%
\pgfsetstrokecolor{currentstroke}%
\pgfsetdash{}{0pt}%
\pgfsys@defobject{currentmarker}{\pgfqpoint{0.000000in}{-0.027778in}}{\pgfqpoint{0.000000in}{0.000000in}}{%
\pgfpathmoveto{\pgfqpoint{0.000000in}{0.000000in}}%
\pgfpathlineto{\pgfqpoint{0.000000in}{-0.027778in}}%
\pgfusepath{stroke,fill}%
}%
\begin{pgfscope}%
\pgfsys@transformshift{7.598353in}{0.438889in}%
\pgfsys@useobject{currentmarker}{}%
\end{pgfscope}%
\end{pgfscope}%
\begin{pgfscope}%
\pgfsetbuttcap%
\pgfsetroundjoin%
\definecolor{currentfill}{rgb}{0.000000,0.000000,0.000000}%
\pgfsetfillcolor{currentfill}%
\pgfsetlinewidth{0.602250pt}%
\definecolor{currentstroke}{rgb}{0.000000,0.000000,0.000000}%
\pgfsetstrokecolor{currentstroke}%
\pgfsetdash{}{0pt}%
\pgfsys@defobject{currentmarker}{\pgfqpoint{0.000000in}{-0.027778in}}{\pgfqpoint{0.000000in}{0.000000in}}{%
\pgfpathmoveto{\pgfqpoint{0.000000in}{0.000000in}}%
\pgfpathlineto{\pgfqpoint{0.000000in}{-0.027778in}}%
\pgfusepath{stroke,fill}%
}%
\begin{pgfscope}%
\pgfsys@transformshift{7.700894in}{0.438889in}%
\pgfsys@useobject{currentmarker}{}%
\end{pgfscope}%
\end{pgfscope}%
\begin{pgfscope}%
\pgfsetbuttcap%
\pgfsetroundjoin%
\definecolor{currentfill}{rgb}{0.000000,0.000000,0.000000}%
\pgfsetfillcolor{currentfill}%
\pgfsetlinewidth{0.602250pt}%
\definecolor{currentstroke}{rgb}{0.000000,0.000000,0.000000}%
\pgfsetstrokecolor{currentstroke}%
\pgfsetdash{}{0pt}%
\pgfsys@defobject{currentmarker}{\pgfqpoint{0.000000in}{-0.027778in}}{\pgfqpoint{0.000000in}{0.000000in}}{%
\pgfpathmoveto{\pgfqpoint{0.000000in}{0.000000in}}%
\pgfpathlineto{\pgfqpoint{0.000000in}{-0.027778in}}%
\pgfusepath{stroke,fill}%
}%
\begin{pgfscope}%
\pgfsys@transformshift{7.773648in}{0.438889in}%
\pgfsys@useobject{currentmarker}{}%
\end{pgfscope}%
\end{pgfscope}%
\begin{pgfscope}%
\pgfsetbuttcap%
\pgfsetroundjoin%
\definecolor{currentfill}{rgb}{0.000000,0.000000,0.000000}%
\pgfsetfillcolor{currentfill}%
\pgfsetlinewidth{0.803000pt}%
\definecolor{currentstroke}{rgb}{0.000000,0.000000,0.000000}%
\pgfsetstrokecolor{currentstroke}%
\pgfsetdash{}{0pt}%
\pgfsys@defobject{currentmarker}{\pgfqpoint{-0.048611in}{0.000000in}}{\pgfqpoint{0.000000in}{0.000000in}}{%
\pgfpathmoveto{\pgfqpoint{0.000000in}{0.000000in}}%
\pgfpathlineto{\pgfqpoint{-0.048611in}{0.000000in}}%
\pgfusepath{stroke,fill}%
}%
\begin{pgfscope}%
\pgfsys@transformshift{4.597569in}{0.447262in}%
\pgfsys@useobject{currentmarker}{}%
\end{pgfscope}%
\end{pgfscope}%
\begin{pgfscope}%
\definecolor{textcolor}{rgb}{0.000000,0.000000,0.000000}%
\pgfsetstrokecolor{textcolor}%
\pgfsetfillcolor{textcolor}%
\pgftext[x=4.074729in,y=0.394501in,left,base]{\color{textcolor}\sffamily\fontsize{10.000000}{12.000000}\selectfont −0.25}%
\end{pgfscope}%
\begin{pgfscope}%
\pgfsetbuttcap%
\pgfsetroundjoin%
\definecolor{currentfill}{rgb}{0.000000,0.000000,0.000000}%
\pgfsetfillcolor{currentfill}%
\pgfsetlinewidth{0.803000pt}%
\definecolor{currentstroke}{rgb}{0.000000,0.000000,0.000000}%
\pgfsetstrokecolor{currentstroke}%
\pgfsetdash{}{0pt}%
\pgfsys@defobject{currentmarker}{\pgfqpoint{-0.048611in}{0.000000in}}{\pgfqpoint{0.000000in}{0.000000in}}{%
\pgfpathmoveto{\pgfqpoint{0.000000in}{0.000000in}}%
\pgfpathlineto{\pgfqpoint{-0.048611in}{0.000000in}}%
\pgfusepath{stroke,fill}%
}%
\begin{pgfscope}%
\pgfsys@transformshift{4.597569in}{0.783469in}%
\pgfsys@useobject{currentmarker}{}%
\end{pgfscope}%
\end{pgfscope}%
\begin{pgfscope}%
\definecolor{textcolor}{rgb}{0.000000,0.000000,0.000000}%
\pgfsetstrokecolor{textcolor}%
\pgfsetfillcolor{textcolor}%
\pgftext[x=4.191102in,y=0.730708in,left,base]{\color{textcolor}\sffamily\fontsize{10.000000}{12.000000}\selectfont 0.00}%
\end{pgfscope}%
\begin{pgfscope}%
\pgfsetbuttcap%
\pgfsetroundjoin%
\definecolor{currentfill}{rgb}{0.000000,0.000000,0.000000}%
\pgfsetfillcolor{currentfill}%
\pgfsetlinewidth{0.803000pt}%
\definecolor{currentstroke}{rgb}{0.000000,0.000000,0.000000}%
\pgfsetstrokecolor{currentstroke}%
\pgfsetdash{}{0pt}%
\pgfsys@defobject{currentmarker}{\pgfqpoint{-0.048611in}{0.000000in}}{\pgfqpoint{0.000000in}{0.000000in}}{%
\pgfpathmoveto{\pgfqpoint{0.000000in}{0.000000in}}%
\pgfpathlineto{\pgfqpoint{-0.048611in}{0.000000in}}%
\pgfusepath{stroke,fill}%
}%
\begin{pgfscope}%
\pgfsys@transformshift{4.597569in}{1.119676in}%
\pgfsys@useobject{currentmarker}{}%
\end{pgfscope}%
\end{pgfscope}%
\begin{pgfscope}%
\definecolor{textcolor}{rgb}{0.000000,0.000000,0.000000}%
\pgfsetstrokecolor{textcolor}%
\pgfsetfillcolor{textcolor}%
\pgftext[x=4.191102in,y=1.066915in,left,base]{\color{textcolor}\sffamily\fontsize{10.000000}{12.000000}\selectfont 0.25}%
\end{pgfscope}%
\begin{pgfscope}%
\pgfsetbuttcap%
\pgfsetroundjoin%
\definecolor{currentfill}{rgb}{0.000000,0.000000,0.000000}%
\pgfsetfillcolor{currentfill}%
\pgfsetlinewidth{0.803000pt}%
\definecolor{currentstroke}{rgb}{0.000000,0.000000,0.000000}%
\pgfsetstrokecolor{currentstroke}%
\pgfsetdash{}{0pt}%
\pgfsys@defobject{currentmarker}{\pgfqpoint{-0.048611in}{0.000000in}}{\pgfqpoint{0.000000in}{0.000000in}}{%
\pgfpathmoveto{\pgfqpoint{0.000000in}{0.000000in}}%
\pgfpathlineto{\pgfqpoint{-0.048611in}{0.000000in}}%
\pgfusepath{stroke,fill}%
}%
\begin{pgfscope}%
\pgfsys@transformshift{4.597569in}{1.455883in}%
\pgfsys@useobject{currentmarker}{}%
\end{pgfscope}%
\end{pgfscope}%
\begin{pgfscope}%
\definecolor{textcolor}{rgb}{0.000000,0.000000,0.000000}%
\pgfsetstrokecolor{textcolor}%
\pgfsetfillcolor{textcolor}%
\pgftext[x=4.191102in,y=1.403122in,left,base]{\color{textcolor}\sffamily\fontsize{10.000000}{12.000000}\selectfont 0.50}%
\end{pgfscope}%
\begin{pgfscope}%
\pgfsetbuttcap%
\pgfsetroundjoin%
\definecolor{currentfill}{rgb}{0.000000,0.000000,0.000000}%
\pgfsetfillcolor{currentfill}%
\pgfsetlinewidth{0.803000pt}%
\definecolor{currentstroke}{rgb}{0.000000,0.000000,0.000000}%
\pgfsetstrokecolor{currentstroke}%
\pgfsetdash{}{0pt}%
\pgfsys@defobject{currentmarker}{\pgfqpoint{-0.048611in}{0.000000in}}{\pgfqpoint{0.000000in}{0.000000in}}{%
\pgfpathmoveto{\pgfqpoint{0.000000in}{0.000000in}}%
\pgfpathlineto{\pgfqpoint{-0.048611in}{0.000000in}}%
\pgfusepath{stroke,fill}%
}%
\begin{pgfscope}%
\pgfsys@transformshift{4.597569in}{1.792090in}%
\pgfsys@useobject{currentmarker}{}%
\end{pgfscope}%
\end{pgfscope}%
\begin{pgfscope}%
\definecolor{textcolor}{rgb}{0.000000,0.000000,0.000000}%
\pgfsetstrokecolor{textcolor}%
\pgfsetfillcolor{textcolor}%
\pgftext[x=4.191102in,y=1.739329in,left,base]{\color{textcolor}\sffamily\fontsize{10.000000}{12.000000}\selectfont 0.75}%
\end{pgfscope}%
\begin{pgfscope}%
\pgfsetbuttcap%
\pgfsetroundjoin%
\definecolor{currentfill}{rgb}{0.000000,0.000000,0.000000}%
\pgfsetfillcolor{currentfill}%
\pgfsetlinewidth{0.803000pt}%
\definecolor{currentstroke}{rgb}{0.000000,0.000000,0.000000}%
\pgfsetstrokecolor{currentstroke}%
\pgfsetdash{}{0pt}%
\pgfsys@defobject{currentmarker}{\pgfqpoint{-0.048611in}{0.000000in}}{\pgfqpoint{0.000000in}{0.000000in}}{%
\pgfpathmoveto{\pgfqpoint{0.000000in}{0.000000in}}%
\pgfpathlineto{\pgfqpoint{-0.048611in}{0.000000in}}%
\pgfusepath{stroke,fill}%
}%
\begin{pgfscope}%
\pgfsys@transformshift{4.597569in}{2.128297in}%
\pgfsys@useobject{currentmarker}{}%
\end{pgfscope}%
\end{pgfscope}%
\begin{pgfscope}%
\definecolor{textcolor}{rgb}{0.000000,0.000000,0.000000}%
\pgfsetstrokecolor{textcolor}%
\pgfsetfillcolor{textcolor}%
\pgftext[x=4.191102in,y=2.075536in,left,base]{\color{textcolor}\sffamily\fontsize{10.000000}{12.000000}\selectfont 1.00}%
\end{pgfscope}%
\begin{pgfscope}%
\pgfsetbuttcap%
\pgfsetroundjoin%
\definecolor{currentfill}{rgb}{0.000000,0.000000,0.000000}%
\pgfsetfillcolor{currentfill}%
\pgfsetlinewidth{0.803000pt}%
\definecolor{currentstroke}{rgb}{0.000000,0.000000,0.000000}%
\pgfsetstrokecolor{currentstroke}%
\pgfsetdash{}{0pt}%
\pgfsys@defobject{currentmarker}{\pgfqpoint{-0.048611in}{0.000000in}}{\pgfqpoint{0.000000in}{0.000000in}}{%
\pgfpathmoveto{\pgfqpoint{0.000000in}{0.000000in}}%
\pgfpathlineto{\pgfqpoint{-0.048611in}{0.000000in}}%
\pgfusepath{stroke,fill}%
}%
\begin{pgfscope}%
\pgfsys@transformshift{4.597569in}{2.464504in}%
\pgfsys@useobject{currentmarker}{}%
\end{pgfscope}%
\end{pgfscope}%
\begin{pgfscope}%
\definecolor{textcolor}{rgb}{0.000000,0.000000,0.000000}%
\pgfsetstrokecolor{textcolor}%
\pgfsetfillcolor{textcolor}%
\pgftext[x=4.191102in,y=2.411743in,left,base]{\color{textcolor}\sffamily\fontsize{10.000000}{12.000000}\selectfont 1.25}%
\end{pgfscope}%
\begin{pgfscope}%
\pgfpathrectangle{\pgfqpoint{4.597569in}{0.438889in}}{\pgfqpoint{3.203819in}{2.263889in}}%
\pgfusepath{clip}%
\pgfsetrectcap%
\pgfsetroundjoin%
\pgfsetlinewidth{1.505625pt}%
\definecolor{currentstroke}{rgb}{0.121569,0.466667,0.705882}%
\pgfsetstrokecolor{currentstroke}%
\pgfsetdash{}{0pt}%
\pgfpathmoveto{\pgfqpoint{4.743198in}{0.794575in}}%
\pgfpathlineto{\pgfqpoint{4.888777in}{0.792780in}}%
\pgfpathlineto{\pgfqpoint{5.034356in}{0.795380in}}%
\pgfpathlineto{\pgfqpoint{5.179936in}{0.797858in}}%
\pgfpathlineto{\pgfqpoint{5.325515in}{0.822146in}}%
\pgfpathlineto{\pgfqpoint{5.471095in}{0.833375in}}%
\pgfpathlineto{\pgfqpoint{5.616674in}{0.846930in}}%
\pgfpathlineto{\pgfqpoint{5.762253in}{0.833145in}}%
\pgfpathlineto{\pgfqpoint{5.907833in}{1.030952in}}%
\pgfpathlineto{\pgfqpoint{6.053412in}{1.767985in}}%
\pgfpathlineto{\pgfqpoint{6.198992in}{1.786628in}}%
\pgfpathlineto{\pgfqpoint{6.344571in}{1.573184in}}%
\pgfpathlineto{\pgfqpoint{6.490150in}{2.127317in}}%
\pgfpathlineto{\pgfqpoint{6.635730in}{2.127315in}}%
\pgfpathlineto{\pgfqpoint{6.781309in}{1.833041in}}%
\pgfpathlineto{\pgfqpoint{6.926889in}{1.620308in}}%
\pgfpathlineto{\pgfqpoint{7.072468in}{1.544095in}}%
\pgfpathlineto{\pgfqpoint{7.218047in}{2.123929in}}%
\pgfpathlineto{\pgfqpoint{7.363627in}{2.127009in}}%
\pgfpathlineto{\pgfqpoint{7.509206in}{1.990203in}}%
\pgfpathlineto{\pgfqpoint{7.654786in}{2.127315in}}%
\pgfusepath{stroke}%
\end{pgfscope}%
\begin{pgfscope}%
\pgfpathrectangle{\pgfqpoint{4.597569in}{0.438889in}}{\pgfqpoint{3.203819in}{2.263889in}}%
\pgfusepath{clip}%
\pgfsetrectcap%
\pgfsetroundjoin%
\pgfsetlinewidth{1.505625pt}%
\definecolor{currentstroke}{rgb}{1.000000,0.498039,0.054902}%
\pgfsetstrokecolor{currentstroke}%
\pgfsetdash{}{0pt}%
\pgfpathmoveto{\pgfqpoint{4.743198in}{0.794555in}}%
\pgfpathlineto{\pgfqpoint{4.888777in}{0.801504in}}%
\pgfpathlineto{\pgfqpoint{5.034356in}{0.803876in}}%
\pgfpathlineto{\pgfqpoint{5.179936in}{0.806776in}}%
\pgfpathlineto{\pgfqpoint{5.325515in}{0.798437in}}%
\pgfpathlineto{\pgfqpoint{5.471095in}{0.876244in}}%
\pgfpathlineto{\pgfqpoint{5.616674in}{0.845025in}}%
\pgfpathlineto{\pgfqpoint{5.762253in}{0.974234in}}%
\pgfpathlineto{\pgfqpoint{5.907833in}{0.958374in}}%
\pgfpathlineto{\pgfqpoint{6.053412in}{0.991935in}}%
\pgfpathlineto{\pgfqpoint{6.198992in}{1.328694in}}%
\pgfpathlineto{\pgfqpoint{6.344571in}{1.392320in}}%
\pgfpathlineto{\pgfqpoint{6.490150in}{1.888858in}}%
\pgfpathlineto{\pgfqpoint{6.635730in}{2.110121in}}%
\pgfpathlineto{\pgfqpoint{6.781309in}{2.110130in}}%
\pgfpathlineto{\pgfqpoint{6.926889in}{1.196342in}}%
\pgfpathlineto{\pgfqpoint{7.072468in}{1.854816in}}%
\pgfpathlineto{\pgfqpoint{7.218047in}{2.033369in}}%
\pgfpathlineto{\pgfqpoint{7.363627in}{2.062848in}}%
\pgfpathlineto{\pgfqpoint{7.509206in}{2.110146in}}%
\pgfpathlineto{\pgfqpoint{7.654786in}{2.110145in}}%
\pgfusepath{stroke}%
\end{pgfscope}%
\begin{pgfscope}%
\pgfpathrectangle{\pgfqpoint{4.597569in}{0.438889in}}{\pgfqpoint{3.203819in}{2.263889in}}%
\pgfusepath{clip}%
\pgfsetrectcap%
\pgfsetroundjoin%
\pgfsetlinewidth{1.505625pt}%
\definecolor{currentstroke}{rgb}{0.172549,0.627451,0.172549}%
\pgfsetstrokecolor{currentstroke}%
\pgfsetdash{}{0pt}%
\pgfpathmoveto{\pgfqpoint{4.743198in}{0.794607in}}%
\pgfpathlineto{\pgfqpoint{4.888777in}{0.794424in}}%
\pgfpathlineto{\pgfqpoint{5.034356in}{0.794160in}}%
\pgfpathlineto{\pgfqpoint{5.179936in}{0.807612in}}%
\pgfpathlineto{\pgfqpoint{5.325515in}{0.792327in}}%
\pgfpathlineto{\pgfqpoint{5.471095in}{0.818182in}}%
\pgfpathlineto{\pgfqpoint{5.616674in}{0.838357in}}%
\pgfpathlineto{\pgfqpoint{5.762253in}{0.868646in}}%
\pgfpathlineto{\pgfqpoint{5.907833in}{0.864188in}}%
\pgfpathlineto{\pgfqpoint{6.053412in}{0.890527in}}%
\pgfpathlineto{\pgfqpoint{6.198992in}{0.870319in}}%
\pgfpathlineto{\pgfqpoint{6.344571in}{1.160606in}}%
\pgfpathlineto{\pgfqpoint{6.490150in}{1.074847in}}%
\pgfpathlineto{\pgfqpoint{6.635730in}{1.512983in}}%
\pgfpathlineto{\pgfqpoint{6.781309in}{1.590899in}}%
\pgfpathlineto{\pgfqpoint{6.926889in}{1.731890in}}%
\pgfpathlineto{\pgfqpoint{7.072468in}{1.733244in}}%
\pgfpathlineto{\pgfqpoint{7.218047in}{1.588091in}}%
\pgfpathlineto{\pgfqpoint{7.363627in}{2.007572in}}%
\pgfpathlineto{\pgfqpoint{7.509206in}{2.009131in}}%
\pgfpathlineto{\pgfqpoint{7.654786in}{2.009065in}}%
\pgfusepath{stroke}%
\end{pgfscope}%
\begin{pgfscope}%
\pgfpathrectangle{\pgfqpoint{4.597569in}{0.438889in}}{\pgfqpoint{3.203819in}{2.263889in}}%
\pgfusepath{clip}%
\pgfsetrectcap%
\pgfsetroundjoin%
\pgfsetlinewidth{1.505625pt}%
\definecolor{currentstroke}{rgb}{0.839216,0.152941,0.156863}%
\pgfsetstrokecolor{currentstroke}%
\pgfsetdash{}{0pt}%
\pgfpathmoveto{\pgfqpoint{4.743198in}{0.790717in}}%
\pgfpathlineto{\pgfqpoint{4.888777in}{0.795658in}}%
\pgfpathlineto{\pgfqpoint{5.034356in}{0.792128in}}%
\pgfpathlineto{\pgfqpoint{5.179936in}{0.794874in}}%
\pgfpathlineto{\pgfqpoint{5.325515in}{0.799703in}}%
\pgfpathlineto{\pgfqpoint{5.471095in}{0.804125in}}%
\pgfpathlineto{\pgfqpoint{5.616674in}{0.815507in}}%
\pgfpathlineto{\pgfqpoint{5.762253in}{0.806506in}}%
\pgfpathlineto{\pgfqpoint{5.907833in}{0.823500in}}%
\pgfpathlineto{\pgfqpoint{6.053412in}{0.854539in}}%
\pgfpathlineto{\pgfqpoint{6.198992in}{0.853238in}}%
\pgfpathlineto{\pgfqpoint{6.344571in}{0.846594in}}%
\pgfpathlineto{\pgfqpoint{6.490150in}{0.839820in}}%
\pgfpathlineto{\pgfqpoint{6.635730in}{0.852733in}}%
\pgfpathlineto{\pgfqpoint{6.781309in}{0.847439in}}%
\pgfpathlineto{\pgfqpoint{6.926889in}{0.838671in}}%
\pgfpathlineto{\pgfqpoint{7.072468in}{0.844658in}}%
\pgfpathlineto{\pgfqpoint{7.218047in}{0.856589in}}%
\pgfpathlineto{\pgfqpoint{7.363627in}{0.847952in}}%
\pgfpathlineto{\pgfqpoint{7.509206in}{0.849700in}}%
\pgfpathlineto{\pgfqpoint{7.654786in}{0.849308in}}%
\pgfusepath{stroke}%
\end{pgfscope}%
\begin{pgfscope}%
\pgfpathrectangle{\pgfqpoint{4.597569in}{0.438889in}}{\pgfqpoint{3.203819in}{2.263889in}}%
\pgfusepath{clip}%
\pgfsetrectcap%
\pgfsetroundjoin%
\pgfsetlinewidth{1.505625pt}%
\definecolor{currentstroke}{rgb}{0.580392,0.403922,0.741176}%
\pgfsetstrokecolor{currentstroke}%
\pgfsetdash{}{0pt}%
\pgfpathmoveto{\pgfqpoint{4.743198in}{0.793191in}}%
\pgfpathlineto{\pgfqpoint{4.888777in}{0.795370in}}%
\pgfpathlineto{\pgfqpoint{5.034356in}{0.800660in}}%
\pgfpathlineto{\pgfqpoint{5.179936in}{0.799883in}}%
\pgfpathlineto{\pgfqpoint{5.325515in}{0.794280in}}%
\pgfpathlineto{\pgfqpoint{5.471095in}{0.806989in}}%
\pgfpathlineto{\pgfqpoint{5.616674in}{0.804853in}}%
\pgfpathlineto{\pgfqpoint{5.762253in}{0.821082in}}%
\pgfpathlineto{\pgfqpoint{5.907833in}{0.813351in}}%
\pgfpathlineto{\pgfqpoint{6.053412in}{0.810823in}}%
\pgfpathlineto{\pgfqpoint{6.198992in}{0.812176in}}%
\pgfpathlineto{\pgfqpoint{6.344571in}{0.811001in}}%
\pgfpathlineto{\pgfqpoint{6.490150in}{0.813398in}}%
\pgfpathlineto{\pgfqpoint{6.635730in}{0.814203in}}%
\pgfpathlineto{\pgfqpoint{6.781309in}{0.811183in}}%
\pgfpathlineto{\pgfqpoint{6.926889in}{0.810360in}}%
\pgfpathlineto{\pgfqpoint{7.072468in}{0.812024in}}%
\pgfpathlineto{\pgfqpoint{7.218047in}{0.810785in}}%
\pgfpathlineto{\pgfqpoint{7.363627in}{0.812355in}}%
\pgfpathlineto{\pgfqpoint{7.509206in}{0.812138in}}%
\pgfpathlineto{\pgfqpoint{7.654786in}{0.812141in}}%
\pgfusepath{stroke}%
\end{pgfscope}%
\begin{pgfscope}%
\pgfsetrectcap%
\pgfsetmiterjoin%
\pgfsetlinewidth{0.803000pt}%
\definecolor{currentstroke}{rgb}{0.000000,0.000000,0.000000}%
\pgfsetstrokecolor{currentstroke}%
\pgfsetdash{}{0pt}%
\pgfpathmoveto{\pgfqpoint{4.597569in}{0.438889in}}%
\pgfpathlineto{\pgfqpoint{4.597569in}{2.702778in}}%
\pgfusepath{stroke}%
\end{pgfscope}%
\begin{pgfscope}%
\pgfsetrectcap%
\pgfsetmiterjoin%
\pgfsetlinewidth{0.803000pt}%
\definecolor{currentstroke}{rgb}{0.000000,0.000000,0.000000}%
\pgfsetstrokecolor{currentstroke}%
\pgfsetdash{}{0pt}%
\pgfpathmoveto{\pgfqpoint{7.801389in}{0.438889in}}%
\pgfpathlineto{\pgfqpoint{7.801389in}{2.702778in}}%
\pgfusepath{stroke}%
\end{pgfscope}%
\begin{pgfscope}%
\pgfsetrectcap%
\pgfsetmiterjoin%
\pgfsetlinewidth{0.803000pt}%
\definecolor{currentstroke}{rgb}{0.000000,0.000000,0.000000}%
\pgfsetstrokecolor{currentstroke}%
\pgfsetdash{}{0pt}%
\pgfpathmoveto{\pgfqpoint{4.597569in}{0.438889in}}%
\pgfpathlineto{\pgfqpoint{7.801389in}{0.438889in}}%
\pgfusepath{stroke}%
\end{pgfscope}%
\begin{pgfscope}%
\pgfsetrectcap%
\pgfsetmiterjoin%
\pgfsetlinewidth{0.803000pt}%
\definecolor{currentstroke}{rgb}{0.000000,0.000000,0.000000}%
\pgfsetstrokecolor{currentstroke}%
\pgfsetdash{}{0pt}%
\pgfpathmoveto{\pgfqpoint{4.597569in}{2.702778in}}%
\pgfpathlineto{\pgfqpoint{7.801389in}{2.702778in}}%
\pgfusepath{stroke}%
\end{pgfscope}%
\begin{pgfscope}%
\definecolor{textcolor}{rgb}{0.000000,0.000000,0.000000}%
\pgfsetstrokecolor{textcolor}%
\pgfsetfillcolor{textcolor}%
\pgftext[x=6.199479in,y=2.786111in,,base]{\color{textcolor}\sffamily\fontsize{12.000000}{14.400000}\selectfont \(\displaystyle  N = 128 \)}%
\end{pgfscope}%
\begin{pgfscope}%
\pgfsetbuttcap%
\pgfsetmiterjoin%
\definecolor{currentfill}{rgb}{1.000000,1.000000,1.000000}%
\pgfsetfillcolor{currentfill}%
\pgfsetfillopacity{0.800000}%
\pgfsetlinewidth{1.003750pt}%
\definecolor{currentstroke}{rgb}{0.800000,0.800000,0.800000}%
\pgfsetstrokecolor{currentstroke}%
\pgfsetstrokeopacity{0.800000}%
\pgfsetdash{}{0pt}%
\pgfpathmoveto{\pgfqpoint{4.694792in}{1.572381in}}%
\pgfpathlineto{\pgfqpoint{5.602343in}{1.572381in}}%
\pgfpathquadraticcurveto{\pgfqpoint{5.630121in}{1.572381in}}{\pgfqpoint{5.630121in}{1.600158in}}%
\pgfpathlineto{\pgfqpoint{5.630121in}{2.605556in}}%
\pgfpathquadraticcurveto{\pgfqpoint{5.630121in}{2.633333in}}{\pgfqpoint{5.602343in}{2.633333in}}%
\pgfpathlineto{\pgfqpoint{4.694792in}{2.633333in}}%
\pgfpathquadraticcurveto{\pgfqpoint{4.667014in}{2.633333in}}{\pgfqpoint{4.667014in}{2.605556in}}%
\pgfpathlineto{\pgfqpoint{4.667014in}{1.600158in}}%
\pgfpathquadraticcurveto{\pgfqpoint{4.667014in}{1.572381in}}{\pgfqpoint{4.694792in}{1.572381in}}%
\pgfpathclose%
\pgfusepath{stroke,fill}%
\end{pgfscope}%
\begin{pgfscope}%
\pgfsetrectcap%
\pgfsetroundjoin%
\pgfsetlinewidth{1.505625pt}%
\definecolor{currentstroke}{rgb}{0.121569,0.466667,0.705882}%
\pgfsetstrokecolor{currentstroke}%
\pgfsetdash{}{0pt}%
\pgfpathmoveto{\pgfqpoint{4.722569in}{2.520866in}}%
\pgfpathlineto{\pgfqpoint{5.000347in}{2.520866in}}%
\pgfusepath{stroke}%
\end{pgfscope}%
\begin{pgfscope}%
\definecolor{textcolor}{rgb}{0.000000,0.000000,0.000000}%
\pgfsetstrokecolor{textcolor}%
\pgfsetfillcolor{textcolor}%
\pgftext[x=5.111458in,y=2.472255in,left,base]{\color{textcolor}\sffamily\fontsize{10.000000}{12.000000}\selectfont \(\displaystyle  T = 1.0 \)}%
\end{pgfscope}%
\begin{pgfscope}%
\pgfsetrectcap%
\pgfsetroundjoin%
\pgfsetlinewidth{1.505625pt}%
\definecolor{currentstroke}{rgb}{1.000000,0.498039,0.054902}%
\pgfsetstrokecolor{currentstroke}%
\pgfsetdash{}{0pt}%
\pgfpathmoveto{\pgfqpoint{4.722569in}{2.317009in}}%
\pgfpathlineto{\pgfqpoint{5.000347in}{2.317009in}}%
\pgfusepath{stroke}%
\end{pgfscope}%
\begin{pgfscope}%
\definecolor{textcolor}{rgb}{0.000000,0.000000,0.000000}%
\pgfsetstrokecolor{textcolor}%
\pgfsetfillcolor{textcolor}%
\pgftext[x=5.111458in,y=2.268397in,left,base]{\color{textcolor}\sffamily\fontsize{10.000000}{12.000000}\selectfont \(\displaystyle  T = 1.5 \)}%
\end{pgfscope}%
\begin{pgfscope}%
\pgfsetrectcap%
\pgfsetroundjoin%
\pgfsetlinewidth{1.505625pt}%
\definecolor{currentstroke}{rgb}{0.172549,0.627451,0.172549}%
\pgfsetstrokecolor{currentstroke}%
\pgfsetdash{}{0pt}%
\pgfpathmoveto{\pgfqpoint{4.722569in}{2.113151in}}%
\pgfpathlineto{\pgfqpoint{5.000347in}{2.113151in}}%
\pgfusepath{stroke}%
\end{pgfscope}%
\begin{pgfscope}%
\definecolor{textcolor}{rgb}{0.000000,0.000000,0.000000}%
\pgfsetstrokecolor{textcolor}%
\pgfsetfillcolor{textcolor}%
\pgftext[x=5.111458in,y=2.064540in,left,base]{\color{textcolor}\sffamily\fontsize{10.000000}{12.000000}\selectfont \(\displaystyle  T = 2.0 \)}%
\end{pgfscope}%
\begin{pgfscope}%
\pgfsetrectcap%
\pgfsetroundjoin%
\pgfsetlinewidth{1.505625pt}%
\definecolor{currentstroke}{rgb}{0.839216,0.152941,0.156863}%
\pgfsetstrokecolor{currentstroke}%
\pgfsetdash{}{0pt}%
\pgfpathmoveto{\pgfqpoint{4.722569in}{1.909294in}}%
\pgfpathlineto{\pgfqpoint{5.000347in}{1.909294in}}%
\pgfusepath{stroke}%
\end{pgfscope}%
\begin{pgfscope}%
\definecolor{textcolor}{rgb}{0.000000,0.000000,0.000000}%
\pgfsetstrokecolor{textcolor}%
\pgfsetfillcolor{textcolor}%
\pgftext[x=5.111458in,y=1.860683in,left,base]{\color{textcolor}\sffamily\fontsize{10.000000}{12.000000}\selectfont \(\displaystyle  T = 2.5 \)}%
\end{pgfscope}%
\begin{pgfscope}%
\pgfsetrectcap%
\pgfsetroundjoin%
\pgfsetlinewidth{1.505625pt}%
\definecolor{currentstroke}{rgb}{0.580392,0.403922,0.741176}%
\pgfsetstrokecolor{currentstroke}%
\pgfsetdash{}{0pt}%
\pgfpathmoveto{\pgfqpoint{4.722569in}{1.705437in}}%
\pgfpathlineto{\pgfqpoint{5.000347in}{1.705437in}}%
\pgfusepath{stroke}%
\end{pgfscope}%
\begin{pgfscope}%
\definecolor{textcolor}{rgb}{0.000000,0.000000,0.000000}%
\pgfsetstrokecolor{textcolor}%
\pgfsetfillcolor{textcolor}%
\pgftext[x=5.111458in,y=1.656826in,left,base]{\color{textcolor}\sffamily\fontsize{10.000000}{12.000000}\selectfont \(\displaystyle  T = 3.0 \)}%
\end{pgfscope}%
\end{pgfpicture}%
\makeatother%
\endgroup%
}
\caption{Magnetization $m$ with respect to iterations using kinetic Monte Carlo}
\label{Fig:KMC}
\end{figure}

As we stated before, the advantage of kinetic Monte Carlo is that it completely remove the possibility of rejection. One may observe that the curve at $ T = 2.5, 3.0 $ does not change much since the error main stems from sampling instead of rejection. However, the convergence at $ T = 1.0, 1.5, 2.5 $ are dramatically boosted. Generally speaking, we can save $ 9 / 10 $ of the iterations, since the number of iterations of convergence at $ N = 16, 32, 64, 128 $ are about $ 10^5, 10^6, 10^7, 10^8 $ respectively.

\section{Numerical results}

All the algorithms are implemented in C. To be exact, the layout is
\begin{partlist}
\item \verb"samp/etr_2d.c": Metropolis--Hastings sampler for the 2-D case;
\item \verb"samp/kin_2d.c": kinetic Monte Carlo sampler for the 2-D case;
\item \verb"samp/kin_dd.c": kinetic Monte Carlo sampler for the 3-D case;
\item \verb"samp/sing_2d.c": single trajectory sampler using Metropolis--Hasting algorithm, for visualization;
\item \verb"samp/utils.c": Miscellaneous routines.
\end{partlist}
We write Python wrappers for C functions in \verb"samp/wrappers.c" and we invoke Python packages to summarize the numerical results and generate figures. The visualization code is placed in \verb"Problem*.py" and \verb"Plot.py". We use \verb"icc" instead of \verb"gcc" for the compiler by default. We link against Intel MKL and OpenMP libraries.


\subsection{2-D case using Metropolis--Hastings algorithm}

We then proceed to check the quantities. We first use the Metropolis--Hastings method first. We first test a wide range of temperature $T$ with $ \mathit{TRAJ} = 4 $ and $ \mathit{ITER} = 2.5 \times 10^8 $. We produce the numerical result on a parallel 4-core machine. The 2-D case is given by Figure \ref{Fig:HeatSmall}, \ref{Fig:CapSmall} and \ref{Fig:MagSmall}. Here $J$ is set to be $1$ and $ h = 0 $.

\begin{figure}[htbp]
\centering
\scalebox{0.666}{%% Creator: Matplotlib, PGF backend
%%
%% To include the figure in your LaTeX document, write
%%   \input{<filename>.pgf}
%%
%% Make sure the required packages are loaded in your preamble
%%   \usepackage{pgf}
%%
%% Figures using additional raster images can only be included by \input if
%% they are in the same directory as the main LaTeX file. For loading figures
%% from other directories you can use the `import` package
%%   \usepackage{import}
%% and then include the figures with
%%   \import{<path to file>}{<filename>.pgf}
%%
%% Matplotlib used the following preamble
%%   \usepackage{fontspec}
%%   \setmainfont{DejaVuSerif.ttf}[Path=/home/lzh/anaconda3/envs/numana/lib/python3.7/site-packages/matplotlib/mpl-data/fonts/ttf/]
%%   \setsansfont{DejaVuSans.ttf}[Path=/home/lzh/anaconda3/envs/numana/lib/python3.7/site-packages/matplotlib/mpl-data/fonts/ttf/]
%%   \setmonofont{DejaVuSansMono.ttf}[Path=/home/lzh/anaconda3/envs/numana/lib/python3.7/site-packages/matplotlib/mpl-data/fonts/ttf/]
%%
\begingroup%
\makeatletter%
\begin{pgfpicture}%
\pgfpathrectangle{\pgfpointorigin}{\pgfqpoint{4.000000in}{4.000000in}}%
\pgfusepath{use as bounding box, clip}%
\begin{pgfscope}%
\pgfsetbuttcap%
\pgfsetmiterjoin%
\definecolor{currentfill}{rgb}{1.000000,1.000000,1.000000}%
\pgfsetfillcolor{currentfill}%
\pgfsetlinewidth{0.000000pt}%
\definecolor{currentstroke}{rgb}{1.000000,1.000000,1.000000}%
\pgfsetstrokecolor{currentstroke}%
\pgfsetdash{}{0pt}%
\pgfpathmoveto{\pgfqpoint{0.000000in}{0.000000in}}%
\pgfpathlineto{\pgfqpoint{4.000000in}{0.000000in}}%
\pgfpathlineto{\pgfqpoint{4.000000in}{4.000000in}}%
\pgfpathlineto{\pgfqpoint{0.000000in}{4.000000in}}%
\pgfpathclose%
\pgfusepath{fill}%
\end{pgfscope}%
\begin{pgfscope}%
\pgfsetbuttcap%
\pgfsetmiterjoin%
\definecolor{currentfill}{rgb}{1.000000,1.000000,1.000000}%
\pgfsetfillcolor{currentfill}%
\pgfsetlinewidth{0.000000pt}%
\definecolor{currentstroke}{rgb}{0.000000,0.000000,0.000000}%
\pgfsetstrokecolor{currentstroke}%
\pgfsetstrokeopacity{0.000000}%
\pgfsetdash{}{0pt}%
\pgfpathmoveto{\pgfqpoint{0.510000in}{0.440000in}}%
\pgfpathlineto{\pgfqpoint{3.590000in}{0.440000in}}%
\pgfpathlineto{\pgfqpoint{3.590000in}{3.520000in}}%
\pgfpathlineto{\pgfqpoint{0.510000in}{3.520000in}}%
\pgfpathclose%
\pgfusepath{fill}%
\end{pgfscope}%
\begin{pgfscope}%
\pgfpathrectangle{\pgfqpoint{0.510000in}{0.440000in}}{\pgfqpoint{3.080000in}{3.080000in}}%
\pgfusepath{clip}%
\pgfsetbuttcap%
\pgfsetroundjoin%
\definecolor{currentfill}{rgb}{0.121569,0.466667,0.705882}%
\pgfsetfillcolor{currentfill}%
\pgfsetlinewidth{1.003750pt}%
\definecolor{currentstroke}{rgb}{0.121569,0.466667,0.705882}%
\pgfsetstrokecolor{currentstroke}%
\pgfsetdash{}{0pt}%
\pgfpathmoveto{\pgfqpoint{1.293201in}{1.248684in}}%
\pgfpathcurveto{\pgfqpoint{1.299025in}{1.248684in}}{\pgfqpoint{1.304611in}{1.250998in}}{\pgfqpoint{1.308729in}{1.255116in}}%
\pgfpathcurveto{\pgfqpoint{1.312847in}{1.259234in}}{\pgfqpoint{1.315161in}{1.264820in}}{\pgfqpoint{1.315161in}{1.270644in}}%
\pgfpathcurveto{\pgfqpoint{1.315161in}{1.276468in}}{\pgfqpoint{1.312847in}{1.282054in}}{\pgfqpoint{1.308729in}{1.286172in}}%
\pgfpathcurveto{\pgfqpoint{1.304611in}{1.290291in}}{\pgfqpoint{1.299025in}{1.292604in}}{\pgfqpoint{1.293201in}{1.292604in}}%
\pgfpathcurveto{\pgfqpoint{1.287377in}{1.292604in}}{\pgfqpoint{1.281791in}{1.290291in}}{\pgfqpoint{1.277673in}{1.286172in}}%
\pgfpathcurveto{\pgfqpoint{1.273554in}{1.282054in}}{\pgfqpoint{1.271241in}{1.276468in}}{\pgfqpoint{1.271241in}{1.270644in}}%
\pgfpathcurveto{\pgfqpoint{1.271241in}{1.264820in}}{\pgfqpoint{1.273554in}{1.259234in}}{\pgfqpoint{1.277673in}{1.255116in}}%
\pgfpathcurveto{\pgfqpoint{1.281791in}{1.250998in}}{\pgfqpoint{1.287377in}{1.248684in}}{\pgfqpoint{1.293201in}{1.248684in}}%
\pgfpathclose%
\pgfusepath{stroke,fill}%
\end{pgfscope}%
\begin{pgfscope}%
\pgfpathrectangle{\pgfqpoint{0.510000in}{0.440000in}}{\pgfqpoint{3.080000in}{3.080000in}}%
\pgfusepath{clip}%
\pgfsetbuttcap%
\pgfsetroundjoin%
\definecolor{currentfill}{rgb}{1.000000,0.498039,0.054902}%
\pgfsetfillcolor{currentfill}%
\pgfsetlinewidth{1.003750pt}%
\definecolor{currentstroke}{rgb}{1.000000,0.498039,0.054902}%
\pgfsetstrokecolor{currentstroke}%
\pgfsetdash{}{0pt}%
\pgfpathmoveto{\pgfqpoint{2.416155in}{1.201241in}}%
\pgfpathcurveto{\pgfqpoint{2.421979in}{1.201241in}}{\pgfqpoint{2.427565in}{1.203554in}}{\pgfqpoint{2.431683in}{1.207673in}}%
\pgfpathcurveto{\pgfqpoint{2.435801in}{1.211791in}}{\pgfqpoint{2.438115in}{1.217377in}}{\pgfqpoint{2.438115in}{1.223201in}}%
\pgfpathcurveto{\pgfqpoint{2.438115in}{1.229025in}}{\pgfqpoint{2.435801in}{1.234611in}}{\pgfqpoint{2.431683in}{1.238729in}}%
\pgfpathcurveto{\pgfqpoint{2.427565in}{1.242847in}}{\pgfqpoint{2.421979in}{1.245161in}}{\pgfqpoint{2.416155in}{1.245161in}}%
\pgfpathcurveto{\pgfqpoint{2.410331in}{1.245161in}}{\pgfqpoint{2.404745in}{1.242847in}}{\pgfqpoint{2.400627in}{1.238729in}}%
\pgfpathcurveto{\pgfqpoint{2.396509in}{1.234611in}}{\pgfqpoint{2.394195in}{1.229025in}}{\pgfqpoint{2.394195in}{1.223201in}}%
\pgfpathcurveto{\pgfqpoint{2.394195in}{1.217377in}}{\pgfqpoint{2.396509in}{1.211791in}}{\pgfqpoint{2.400627in}{1.207673in}}%
\pgfpathcurveto{\pgfqpoint{2.404745in}{1.203554in}}{\pgfqpoint{2.410331in}{1.201241in}}{\pgfqpoint{2.416155in}{1.201241in}}%
\pgfpathclose%
\pgfusepath{stroke,fill}%
\end{pgfscope}%
\begin{pgfscope}%
\pgfpathrectangle{\pgfqpoint{0.510000in}{0.440000in}}{\pgfqpoint{3.080000in}{3.080000in}}%
\pgfusepath{clip}%
\pgfsetbuttcap%
\pgfsetroundjoin%
\definecolor{currentfill}{rgb}{0.172549,0.627451,0.172549}%
\pgfsetfillcolor{currentfill}%
\pgfsetlinewidth{1.003750pt}%
\definecolor{currentstroke}{rgb}{0.172549,0.627451,0.172549}%
\pgfsetstrokecolor{currentstroke}%
\pgfsetdash{}{0pt}%
\pgfpathmoveto{\pgfqpoint{1.340644in}{2.324195in}}%
\pgfpathcurveto{\pgfqpoint{1.346468in}{2.324195in}}{\pgfqpoint{1.352054in}{2.326509in}}{\pgfqpoint{1.356172in}{2.330627in}}%
\pgfpathcurveto{\pgfqpoint{1.360291in}{2.334745in}}{\pgfqpoint{1.362604in}{2.340331in}}{\pgfqpoint{1.362604in}{2.346155in}}%
\pgfpathcurveto{\pgfqpoint{1.362604in}{2.351979in}}{\pgfqpoint{1.360291in}{2.357565in}}{\pgfqpoint{1.356172in}{2.361683in}}%
\pgfpathcurveto{\pgfqpoint{1.352054in}{2.365801in}}{\pgfqpoint{1.346468in}{2.368115in}}{\pgfqpoint{1.340644in}{2.368115in}}%
\pgfpathcurveto{\pgfqpoint{1.334820in}{2.368115in}}{\pgfqpoint{1.329234in}{2.365801in}}{\pgfqpoint{1.325116in}{2.361683in}}%
\pgfpathcurveto{\pgfqpoint{1.320998in}{2.357565in}}{\pgfqpoint{1.318684in}{2.351979in}}{\pgfqpoint{1.318684in}{2.346155in}}%
\pgfpathcurveto{\pgfqpoint{1.318684in}{2.340331in}}{\pgfqpoint{1.320998in}{2.334745in}}{\pgfqpoint{1.325116in}{2.330627in}}%
\pgfpathcurveto{\pgfqpoint{1.329234in}{2.326509in}}{\pgfqpoint{1.334820in}{2.324195in}}{\pgfqpoint{1.340644in}{2.324195in}}%
\pgfpathclose%
\pgfusepath{stroke,fill}%
\end{pgfscope}%
\begin{pgfscope}%
\pgfpathrectangle{\pgfqpoint{0.510000in}{0.440000in}}{\pgfqpoint{3.080000in}{3.080000in}}%
\pgfusepath{clip}%
\pgfsetbuttcap%
\pgfsetmiterjoin%
\definecolor{currentfill}{rgb}{0.100000,0.100000,0.100000}%
\pgfsetfillcolor{currentfill}%
\pgfsetfillopacity{0.100000}%
\pgfsetlinewidth{0.301125pt}%
\definecolor{currentstroke}{rgb}{0.000000,0.000000,0.000000}%
\pgfsetstrokecolor{currentstroke}%
\pgfsetdash{}{0pt}%
\pgfpathmoveto{\pgfqpoint{0.950000in}{0.880000in}}%
\pgfpathlineto{\pgfqpoint{0.950000in}{3.080000in}}%
\pgfpathlineto{\pgfqpoint{3.150000in}{0.880000in}}%
\pgfpathclose%
\pgfusepath{stroke,fill}%
\end{pgfscope}%
\begin{pgfscope}%
\pgfpathrectangle{\pgfqpoint{0.510000in}{0.440000in}}{\pgfqpoint{3.080000in}{3.080000in}}%
\pgfusepath{clip}%
\pgfsetrectcap%
\pgfsetroundjoin%
\pgfsetlinewidth{0.803000pt}%
\definecolor{currentstroke}{rgb}{0.690196,0.690196,0.690196}%
\pgfsetstrokecolor{currentstroke}%
\pgfsetdash{}{0pt}%
\pgfpathmoveto{\pgfqpoint{0.510000in}{0.440000in}}%
\pgfpathlineto{\pgfqpoint{0.510000in}{3.520000in}}%
\pgfusepath{stroke}%
\end{pgfscope}%
\begin{pgfscope}%
\pgfsetbuttcap%
\pgfsetroundjoin%
\definecolor{currentfill}{rgb}{0.000000,0.000000,0.000000}%
\pgfsetfillcolor{currentfill}%
\pgfsetlinewidth{0.803000pt}%
\definecolor{currentstroke}{rgb}{0.000000,0.000000,0.000000}%
\pgfsetstrokecolor{currentstroke}%
\pgfsetdash{}{0pt}%
\pgfsys@defobject{currentmarker}{\pgfqpoint{0.000000in}{-0.048611in}}{\pgfqpoint{0.000000in}{0.000000in}}{%
\pgfpathmoveto{\pgfqpoint{0.000000in}{0.000000in}}%
\pgfpathlineto{\pgfqpoint{0.000000in}{-0.048611in}}%
\pgfusepath{stroke,fill}%
}%
\begin{pgfscope}%
\pgfsys@transformshift{0.510000in}{0.440000in}%
\pgfsys@useobject{currentmarker}{}%
\end{pgfscope}%
\end{pgfscope}%
\begin{pgfscope}%
\definecolor{textcolor}{rgb}{0.000000,0.000000,0.000000}%
\pgfsetstrokecolor{textcolor}%
\pgfsetfillcolor{textcolor}%
\pgftext[x=0.510000in,y=0.342778in,,top]{\color{textcolor}\sffamily\fontsize{10.000000}{12.000000}\selectfont -0.2}%
\end{pgfscope}%
\begin{pgfscope}%
\pgfpathrectangle{\pgfqpoint{0.510000in}{0.440000in}}{\pgfqpoint{3.080000in}{3.080000in}}%
\pgfusepath{clip}%
\pgfsetrectcap%
\pgfsetroundjoin%
\pgfsetlinewidth{0.803000pt}%
\definecolor{currentstroke}{rgb}{0.690196,0.690196,0.690196}%
\pgfsetstrokecolor{currentstroke}%
\pgfsetdash{}{0pt}%
\pgfpathmoveto{\pgfqpoint{0.950000in}{0.440000in}}%
\pgfpathlineto{\pgfqpoint{0.950000in}{3.520000in}}%
\pgfusepath{stroke}%
\end{pgfscope}%
\begin{pgfscope}%
\pgfsetbuttcap%
\pgfsetroundjoin%
\definecolor{currentfill}{rgb}{0.000000,0.000000,0.000000}%
\pgfsetfillcolor{currentfill}%
\pgfsetlinewidth{0.803000pt}%
\definecolor{currentstroke}{rgb}{0.000000,0.000000,0.000000}%
\pgfsetstrokecolor{currentstroke}%
\pgfsetdash{}{0pt}%
\pgfsys@defobject{currentmarker}{\pgfqpoint{0.000000in}{-0.048611in}}{\pgfqpoint{0.000000in}{0.000000in}}{%
\pgfpathmoveto{\pgfqpoint{0.000000in}{0.000000in}}%
\pgfpathlineto{\pgfqpoint{0.000000in}{-0.048611in}}%
\pgfusepath{stroke,fill}%
}%
\begin{pgfscope}%
\pgfsys@transformshift{0.950000in}{0.440000in}%
\pgfsys@useobject{currentmarker}{}%
\end{pgfscope}%
\end{pgfscope}%
\begin{pgfscope}%
\definecolor{textcolor}{rgb}{0.000000,0.000000,0.000000}%
\pgfsetstrokecolor{textcolor}%
\pgfsetfillcolor{textcolor}%
\pgftext[x=0.950000in,y=0.342778in,,top]{\color{textcolor}\sffamily\fontsize{10.000000}{12.000000}\selectfont 0.0}%
\end{pgfscope}%
\begin{pgfscope}%
\pgfpathrectangle{\pgfqpoint{0.510000in}{0.440000in}}{\pgfqpoint{3.080000in}{3.080000in}}%
\pgfusepath{clip}%
\pgfsetrectcap%
\pgfsetroundjoin%
\pgfsetlinewidth{0.803000pt}%
\definecolor{currentstroke}{rgb}{0.690196,0.690196,0.690196}%
\pgfsetstrokecolor{currentstroke}%
\pgfsetdash{}{0pt}%
\pgfpathmoveto{\pgfqpoint{1.390000in}{0.440000in}}%
\pgfpathlineto{\pgfqpoint{1.390000in}{3.520000in}}%
\pgfusepath{stroke}%
\end{pgfscope}%
\begin{pgfscope}%
\pgfsetbuttcap%
\pgfsetroundjoin%
\definecolor{currentfill}{rgb}{0.000000,0.000000,0.000000}%
\pgfsetfillcolor{currentfill}%
\pgfsetlinewidth{0.803000pt}%
\definecolor{currentstroke}{rgb}{0.000000,0.000000,0.000000}%
\pgfsetstrokecolor{currentstroke}%
\pgfsetdash{}{0pt}%
\pgfsys@defobject{currentmarker}{\pgfqpoint{0.000000in}{-0.048611in}}{\pgfqpoint{0.000000in}{0.000000in}}{%
\pgfpathmoveto{\pgfqpoint{0.000000in}{0.000000in}}%
\pgfpathlineto{\pgfqpoint{0.000000in}{-0.048611in}}%
\pgfusepath{stroke,fill}%
}%
\begin{pgfscope}%
\pgfsys@transformshift{1.390000in}{0.440000in}%
\pgfsys@useobject{currentmarker}{}%
\end{pgfscope}%
\end{pgfscope}%
\begin{pgfscope}%
\definecolor{textcolor}{rgb}{0.000000,0.000000,0.000000}%
\pgfsetstrokecolor{textcolor}%
\pgfsetfillcolor{textcolor}%
\pgftext[x=1.390000in,y=0.342778in,,top]{\color{textcolor}\sffamily\fontsize{10.000000}{12.000000}\selectfont 0.2}%
\end{pgfscope}%
\begin{pgfscope}%
\pgfpathrectangle{\pgfqpoint{0.510000in}{0.440000in}}{\pgfqpoint{3.080000in}{3.080000in}}%
\pgfusepath{clip}%
\pgfsetrectcap%
\pgfsetroundjoin%
\pgfsetlinewidth{0.803000pt}%
\definecolor{currentstroke}{rgb}{0.690196,0.690196,0.690196}%
\pgfsetstrokecolor{currentstroke}%
\pgfsetdash{}{0pt}%
\pgfpathmoveto{\pgfqpoint{1.830000in}{0.440000in}}%
\pgfpathlineto{\pgfqpoint{1.830000in}{3.520000in}}%
\pgfusepath{stroke}%
\end{pgfscope}%
\begin{pgfscope}%
\pgfsetbuttcap%
\pgfsetroundjoin%
\definecolor{currentfill}{rgb}{0.000000,0.000000,0.000000}%
\pgfsetfillcolor{currentfill}%
\pgfsetlinewidth{0.803000pt}%
\definecolor{currentstroke}{rgb}{0.000000,0.000000,0.000000}%
\pgfsetstrokecolor{currentstroke}%
\pgfsetdash{}{0pt}%
\pgfsys@defobject{currentmarker}{\pgfqpoint{0.000000in}{-0.048611in}}{\pgfqpoint{0.000000in}{0.000000in}}{%
\pgfpathmoveto{\pgfqpoint{0.000000in}{0.000000in}}%
\pgfpathlineto{\pgfqpoint{0.000000in}{-0.048611in}}%
\pgfusepath{stroke,fill}%
}%
\begin{pgfscope}%
\pgfsys@transformshift{1.830000in}{0.440000in}%
\pgfsys@useobject{currentmarker}{}%
\end{pgfscope}%
\end{pgfscope}%
\begin{pgfscope}%
\definecolor{textcolor}{rgb}{0.000000,0.000000,0.000000}%
\pgfsetstrokecolor{textcolor}%
\pgfsetfillcolor{textcolor}%
\pgftext[x=1.830000in,y=0.342778in,,top]{\color{textcolor}\sffamily\fontsize{10.000000}{12.000000}\selectfont 0.4}%
\end{pgfscope}%
\begin{pgfscope}%
\pgfpathrectangle{\pgfqpoint{0.510000in}{0.440000in}}{\pgfqpoint{3.080000in}{3.080000in}}%
\pgfusepath{clip}%
\pgfsetrectcap%
\pgfsetroundjoin%
\pgfsetlinewidth{0.803000pt}%
\definecolor{currentstroke}{rgb}{0.690196,0.690196,0.690196}%
\pgfsetstrokecolor{currentstroke}%
\pgfsetdash{}{0pt}%
\pgfpathmoveto{\pgfqpoint{2.270000in}{0.440000in}}%
\pgfpathlineto{\pgfqpoint{2.270000in}{3.520000in}}%
\pgfusepath{stroke}%
\end{pgfscope}%
\begin{pgfscope}%
\pgfsetbuttcap%
\pgfsetroundjoin%
\definecolor{currentfill}{rgb}{0.000000,0.000000,0.000000}%
\pgfsetfillcolor{currentfill}%
\pgfsetlinewidth{0.803000pt}%
\definecolor{currentstroke}{rgb}{0.000000,0.000000,0.000000}%
\pgfsetstrokecolor{currentstroke}%
\pgfsetdash{}{0pt}%
\pgfsys@defobject{currentmarker}{\pgfqpoint{0.000000in}{-0.048611in}}{\pgfqpoint{0.000000in}{0.000000in}}{%
\pgfpathmoveto{\pgfqpoint{0.000000in}{0.000000in}}%
\pgfpathlineto{\pgfqpoint{0.000000in}{-0.048611in}}%
\pgfusepath{stroke,fill}%
}%
\begin{pgfscope}%
\pgfsys@transformshift{2.270000in}{0.440000in}%
\pgfsys@useobject{currentmarker}{}%
\end{pgfscope}%
\end{pgfscope}%
\begin{pgfscope}%
\definecolor{textcolor}{rgb}{0.000000,0.000000,0.000000}%
\pgfsetstrokecolor{textcolor}%
\pgfsetfillcolor{textcolor}%
\pgftext[x=2.270000in,y=0.342778in,,top]{\color{textcolor}\sffamily\fontsize{10.000000}{12.000000}\selectfont 0.6}%
\end{pgfscope}%
\begin{pgfscope}%
\pgfpathrectangle{\pgfqpoint{0.510000in}{0.440000in}}{\pgfqpoint{3.080000in}{3.080000in}}%
\pgfusepath{clip}%
\pgfsetrectcap%
\pgfsetroundjoin%
\pgfsetlinewidth{0.803000pt}%
\definecolor{currentstroke}{rgb}{0.690196,0.690196,0.690196}%
\pgfsetstrokecolor{currentstroke}%
\pgfsetdash{}{0pt}%
\pgfpathmoveto{\pgfqpoint{2.710000in}{0.440000in}}%
\pgfpathlineto{\pgfqpoint{2.710000in}{3.520000in}}%
\pgfusepath{stroke}%
\end{pgfscope}%
\begin{pgfscope}%
\pgfsetbuttcap%
\pgfsetroundjoin%
\definecolor{currentfill}{rgb}{0.000000,0.000000,0.000000}%
\pgfsetfillcolor{currentfill}%
\pgfsetlinewidth{0.803000pt}%
\definecolor{currentstroke}{rgb}{0.000000,0.000000,0.000000}%
\pgfsetstrokecolor{currentstroke}%
\pgfsetdash{}{0pt}%
\pgfsys@defobject{currentmarker}{\pgfqpoint{0.000000in}{-0.048611in}}{\pgfqpoint{0.000000in}{0.000000in}}{%
\pgfpathmoveto{\pgfqpoint{0.000000in}{0.000000in}}%
\pgfpathlineto{\pgfqpoint{0.000000in}{-0.048611in}}%
\pgfusepath{stroke,fill}%
}%
\begin{pgfscope}%
\pgfsys@transformshift{2.710000in}{0.440000in}%
\pgfsys@useobject{currentmarker}{}%
\end{pgfscope}%
\end{pgfscope}%
\begin{pgfscope}%
\definecolor{textcolor}{rgb}{0.000000,0.000000,0.000000}%
\pgfsetstrokecolor{textcolor}%
\pgfsetfillcolor{textcolor}%
\pgftext[x=2.710000in,y=0.342778in,,top]{\color{textcolor}\sffamily\fontsize{10.000000}{12.000000}\selectfont 0.8}%
\end{pgfscope}%
\begin{pgfscope}%
\pgfpathrectangle{\pgfqpoint{0.510000in}{0.440000in}}{\pgfqpoint{3.080000in}{3.080000in}}%
\pgfusepath{clip}%
\pgfsetrectcap%
\pgfsetroundjoin%
\pgfsetlinewidth{0.803000pt}%
\definecolor{currentstroke}{rgb}{0.690196,0.690196,0.690196}%
\pgfsetstrokecolor{currentstroke}%
\pgfsetdash{}{0pt}%
\pgfpathmoveto{\pgfqpoint{3.150000in}{0.440000in}}%
\pgfpathlineto{\pgfqpoint{3.150000in}{3.520000in}}%
\pgfusepath{stroke}%
\end{pgfscope}%
\begin{pgfscope}%
\pgfsetbuttcap%
\pgfsetroundjoin%
\definecolor{currentfill}{rgb}{0.000000,0.000000,0.000000}%
\pgfsetfillcolor{currentfill}%
\pgfsetlinewidth{0.803000pt}%
\definecolor{currentstroke}{rgb}{0.000000,0.000000,0.000000}%
\pgfsetstrokecolor{currentstroke}%
\pgfsetdash{}{0pt}%
\pgfsys@defobject{currentmarker}{\pgfqpoint{0.000000in}{-0.048611in}}{\pgfqpoint{0.000000in}{0.000000in}}{%
\pgfpathmoveto{\pgfqpoint{0.000000in}{0.000000in}}%
\pgfpathlineto{\pgfqpoint{0.000000in}{-0.048611in}}%
\pgfusepath{stroke,fill}%
}%
\begin{pgfscope}%
\pgfsys@transformshift{3.150000in}{0.440000in}%
\pgfsys@useobject{currentmarker}{}%
\end{pgfscope}%
\end{pgfscope}%
\begin{pgfscope}%
\definecolor{textcolor}{rgb}{0.000000,0.000000,0.000000}%
\pgfsetstrokecolor{textcolor}%
\pgfsetfillcolor{textcolor}%
\pgftext[x=3.150000in,y=0.342778in,,top]{\color{textcolor}\sffamily\fontsize{10.000000}{12.000000}\selectfont 1.0}%
\end{pgfscope}%
\begin{pgfscope}%
\pgfpathrectangle{\pgfqpoint{0.510000in}{0.440000in}}{\pgfqpoint{3.080000in}{3.080000in}}%
\pgfusepath{clip}%
\pgfsetrectcap%
\pgfsetroundjoin%
\pgfsetlinewidth{0.803000pt}%
\definecolor{currentstroke}{rgb}{0.690196,0.690196,0.690196}%
\pgfsetstrokecolor{currentstroke}%
\pgfsetdash{}{0pt}%
\pgfpathmoveto{\pgfqpoint{3.590000in}{0.440000in}}%
\pgfpathlineto{\pgfqpoint{3.590000in}{3.520000in}}%
\pgfusepath{stroke}%
\end{pgfscope}%
\begin{pgfscope}%
\pgfsetbuttcap%
\pgfsetroundjoin%
\definecolor{currentfill}{rgb}{0.000000,0.000000,0.000000}%
\pgfsetfillcolor{currentfill}%
\pgfsetlinewidth{0.803000pt}%
\definecolor{currentstroke}{rgb}{0.000000,0.000000,0.000000}%
\pgfsetstrokecolor{currentstroke}%
\pgfsetdash{}{0pt}%
\pgfsys@defobject{currentmarker}{\pgfqpoint{0.000000in}{-0.048611in}}{\pgfqpoint{0.000000in}{0.000000in}}{%
\pgfpathmoveto{\pgfqpoint{0.000000in}{0.000000in}}%
\pgfpathlineto{\pgfqpoint{0.000000in}{-0.048611in}}%
\pgfusepath{stroke,fill}%
}%
\begin{pgfscope}%
\pgfsys@transformshift{3.590000in}{0.440000in}%
\pgfsys@useobject{currentmarker}{}%
\end{pgfscope}%
\end{pgfscope}%
\begin{pgfscope}%
\definecolor{textcolor}{rgb}{0.000000,0.000000,0.000000}%
\pgfsetstrokecolor{textcolor}%
\pgfsetfillcolor{textcolor}%
\pgftext[x=3.590000in,y=0.342778in,,top]{\color{textcolor}\sffamily\fontsize{10.000000}{12.000000}\selectfont 1.2}%
\end{pgfscope}%
\begin{pgfscope}%
\pgfpathrectangle{\pgfqpoint{0.510000in}{0.440000in}}{\pgfqpoint{3.080000in}{3.080000in}}%
\pgfusepath{clip}%
\pgfsetrectcap%
\pgfsetroundjoin%
\pgfsetlinewidth{0.803000pt}%
\definecolor{currentstroke}{rgb}{0.690196,0.690196,0.690196}%
\pgfsetstrokecolor{currentstroke}%
\pgfsetdash{}{0pt}%
\pgfpathmoveto{\pgfqpoint{0.510000in}{0.440000in}}%
\pgfpathlineto{\pgfqpoint{3.590000in}{0.440000in}}%
\pgfusepath{stroke}%
\end{pgfscope}%
\begin{pgfscope}%
\pgfsetbuttcap%
\pgfsetroundjoin%
\definecolor{currentfill}{rgb}{0.000000,0.000000,0.000000}%
\pgfsetfillcolor{currentfill}%
\pgfsetlinewidth{0.803000pt}%
\definecolor{currentstroke}{rgb}{0.000000,0.000000,0.000000}%
\pgfsetstrokecolor{currentstroke}%
\pgfsetdash{}{0pt}%
\pgfsys@defobject{currentmarker}{\pgfqpoint{-0.048611in}{0.000000in}}{\pgfqpoint{0.000000in}{0.000000in}}{%
\pgfpathmoveto{\pgfqpoint{0.000000in}{0.000000in}}%
\pgfpathlineto{\pgfqpoint{-0.048611in}{0.000000in}}%
\pgfusepath{stroke,fill}%
}%
\begin{pgfscope}%
\pgfsys@transformshift{0.510000in}{0.440000in}%
\pgfsys@useobject{currentmarker}{}%
\end{pgfscope}%
\end{pgfscope}%
\begin{pgfscope}%
\definecolor{textcolor}{rgb}{0.000000,0.000000,0.000000}%
\pgfsetstrokecolor{textcolor}%
\pgfsetfillcolor{textcolor}%
\pgftext[x=0.075525in,y=0.387238in,left,base]{\color{textcolor}\sffamily\fontsize{10.000000}{12.000000}\selectfont -0.2}%
\end{pgfscope}%
\begin{pgfscope}%
\pgfpathrectangle{\pgfqpoint{0.510000in}{0.440000in}}{\pgfqpoint{3.080000in}{3.080000in}}%
\pgfusepath{clip}%
\pgfsetrectcap%
\pgfsetroundjoin%
\pgfsetlinewidth{0.803000pt}%
\definecolor{currentstroke}{rgb}{0.690196,0.690196,0.690196}%
\pgfsetstrokecolor{currentstroke}%
\pgfsetdash{}{0pt}%
\pgfpathmoveto{\pgfqpoint{0.510000in}{0.880000in}}%
\pgfpathlineto{\pgfqpoint{3.590000in}{0.880000in}}%
\pgfusepath{stroke}%
\end{pgfscope}%
\begin{pgfscope}%
\pgfsetbuttcap%
\pgfsetroundjoin%
\definecolor{currentfill}{rgb}{0.000000,0.000000,0.000000}%
\pgfsetfillcolor{currentfill}%
\pgfsetlinewidth{0.803000pt}%
\definecolor{currentstroke}{rgb}{0.000000,0.000000,0.000000}%
\pgfsetstrokecolor{currentstroke}%
\pgfsetdash{}{0pt}%
\pgfsys@defobject{currentmarker}{\pgfqpoint{-0.048611in}{0.000000in}}{\pgfqpoint{0.000000in}{0.000000in}}{%
\pgfpathmoveto{\pgfqpoint{0.000000in}{0.000000in}}%
\pgfpathlineto{\pgfqpoint{-0.048611in}{0.000000in}}%
\pgfusepath{stroke,fill}%
}%
\begin{pgfscope}%
\pgfsys@transformshift{0.510000in}{0.880000in}%
\pgfsys@useobject{currentmarker}{}%
\end{pgfscope}%
\end{pgfscope}%
\begin{pgfscope}%
\definecolor{textcolor}{rgb}{0.000000,0.000000,0.000000}%
\pgfsetstrokecolor{textcolor}%
\pgfsetfillcolor{textcolor}%
\pgftext[x=0.191898in,y=0.827238in,left,base]{\color{textcolor}\sffamily\fontsize{10.000000}{12.000000}\selectfont 0.0}%
\end{pgfscope}%
\begin{pgfscope}%
\pgfpathrectangle{\pgfqpoint{0.510000in}{0.440000in}}{\pgfqpoint{3.080000in}{3.080000in}}%
\pgfusepath{clip}%
\pgfsetrectcap%
\pgfsetroundjoin%
\pgfsetlinewidth{0.803000pt}%
\definecolor{currentstroke}{rgb}{0.690196,0.690196,0.690196}%
\pgfsetstrokecolor{currentstroke}%
\pgfsetdash{}{0pt}%
\pgfpathmoveto{\pgfqpoint{0.510000in}{1.320000in}}%
\pgfpathlineto{\pgfqpoint{3.590000in}{1.320000in}}%
\pgfusepath{stroke}%
\end{pgfscope}%
\begin{pgfscope}%
\pgfsetbuttcap%
\pgfsetroundjoin%
\definecolor{currentfill}{rgb}{0.000000,0.000000,0.000000}%
\pgfsetfillcolor{currentfill}%
\pgfsetlinewidth{0.803000pt}%
\definecolor{currentstroke}{rgb}{0.000000,0.000000,0.000000}%
\pgfsetstrokecolor{currentstroke}%
\pgfsetdash{}{0pt}%
\pgfsys@defobject{currentmarker}{\pgfqpoint{-0.048611in}{0.000000in}}{\pgfqpoint{0.000000in}{0.000000in}}{%
\pgfpathmoveto{\pgfqpoint{0.000000in}{0.000000in}}%
\pgfpathlineto{\pgfqpoint{-0.048611in}{0.000000in}}%
\pgfusepath{stroke,fill}%
}%
\begin{pgfscope}%
\pgfsys@transformshift{0.510000in}{1.320000in}%
\pgfsys@useobject{currentmarker}{}%
\end{pgfscope}%
\end{pgfscope}%
\begin{pgfscope}%
\definecolor{textcolor}{rgb}{0.000000,0.000000,0.000000}%
\pgfsetstrokecolor{textcolor}%
\pgfsetfillcolor{textcolor}%
\pgftext[x=0.191898in,y=1.267238in,left,base]{\color{textcolor}\sffamily\fontsize{10.000000}{12.000000}\selectfont 0.2}%
\end{pgfscope}%
\begin{pgfscope}%
\pgfpathrectangle{\pgfqpoint{0.510000in}{0.440000in}}{\pgfqpoint{3.080000in}{3.080000in}}%
\pgfusepath{clip}%
\pgfsetrectcap%
\pgfsetroundjoin%
\pgfsetlinewidth{0.803000pt}%
\definecolor{currentstroke}{rgb}{0.690196,0.690196,0.690196}%
\pgfsetstrokecolor{currentstroke}%
\pgfsetdash{}{0pt}%
\pgfpathmoveto{\pgfqpoint{0.510000in}{1.760000in}}%
\pgfpathlineto{\pgfqpoint{3.590000in}{1.760000in}}%
\pgfusepath{stroke}%
\end{pgfscope}%
\begin{pgfscope}%
\pgfsetbuttcap%
\pgfsetroundjoin%
\definecolor{currentfill}{rgb}{0.000000,0.000000,0.000000}%
\pgfsetfillcolor{currentfill}%
\pgfsetlinewidth{0.803000pt}%
\definecolor{currentstroke}{rgb}{0.000000,0.000000,0.000000}%
\pgfsetstrokecolor{currentstroke}%
\pgfsetdash{}{0pt}%
\pgfsys@defobject{currentmarker}{\pgfqpoint{-0.048611in}{0.000000in}}{\pgfqpoint{0.000000in}{0.000000in}}{%
\pgfpathmoveto{\pgfqpoint{0.000000in}{0.000000in}}%
\pgfpathlineto{\pgfqpoint{-0.048611in}{0.000000in}}%
\pgfusepath{stroke,fill}%
}%
\begin{pgfscope}%
\pgfsys@transformshift{0.510000in}{1.760000in}%
\pgfsys@useobject{currentmarker}{}%
\end{pgfscope}%
\end{pgfscope}%
\begin{pgfscope}%
\definecolor{textcolor}{rgb}{0.000000,0.000000,0.000000}%
\pgfsetstrokecolor{textcolor}%
\pgfsetfillcolor{textcolor}%
\pgftext[x=0.191898in,y=1.707238in,left,base]{\color{textcolor}\sffamily\fontsize{10.000000}{12.000000}\selectfont 0.4}%
\end{pgfscope}%
\begin{pgfscope}%
\pgfpathrectangle{\pgfqpoint{0.510000in}{0.440000in}}{\pgfqpoint{3.080000in}{3.080000in}}%
\pgfusepath{clip}%
\pgfsetrectcap%
\pgfsetroundjoin%
\pgfsetlinewidth{0.803000pt}%
\definecolor{currentstroke}{rgb}{0.690196,0.690196,0.690196}%
\pgfsetstrokecolor{currentstroke}%
\pgfsetdash{}{0pt}%
\pgfpathmoveto{\pgfqpoint{0.510000in}{2.200000in}}%
\pgfpathlineto{\pgfqpoint{3.590000in}{2.200000in}}%
\pgfusepath{stroke}%
\end{pgfscope}%
\begin{pgfscope}%
\pgfsetbuttcap%
\pgfsetroundjoin%
\definecolor{currentfill}{rgb}{0.000000,0.000000,0.000000}%
\pgfsetfillcolor{currentfill}%
\pgfsetlinewidth{0.803000pt}%
\definecolor{currentstroke}{rgb}{0.000000,0.000000,0.000000}%
\pgfsetstrokecolor{currentstroke}%
\pgfsetdash{}{0pt}%
\pgfsys@defobject{currentmarker}{\pgfqpoint{-0.048611in}{0.000000in}}{\pgfqpoint{0.000000in}{0.000000in}}{%
\pgfpathmoveto{\pgfqpoint{0.000000in}{0.000000in}}%
\pgfpathlineto{\pgfqpoint{-0.048611in}{0.000000in}}%
\pgfusepath{stroke,fill}%
}%
\begin{pgfscope}%
\pgfsys@transformshift{0.510000in}{2.200000in}%
\pgfsys@useobject{currentmarker}{}%
\end{pgfscope}%
\end{pgfscope}%
\begin{pgfscope}%
\definecolor{textcolor}{rgb}{0.000000,0.000000,0.000000}%
\pgfsetstrokecolor{textcolor}%
\pgfsetfillcolor{textcolor}%
\pgftext[x=0.191898in,y=2.147238in,left,base]{\color{textcolor}\sffamily\fontsize{10.000000}{12.000000}\selectfont 0.6}%
\end{pgfscope}%
\begin{pgfscope}%
\pgfpathrectangle{\pgfqpoint{0.510000in}{0.440000in}}{\pgfqpoint{3.080000in}{3.080000in}}%
\pgfusepath{clip}%
\pgfsetrectcap%
\pgfsetroundjoin%
\pgfsetlinewidth{0.803000pt}%
\definecolor{currentstroke}{rgb}{0.690196,0.690196,0.690196}%
\pgfsetstrokecolor{currentstroke}%
\pgfsetdash{}{0pt}%
\pgfpathmoveto{\pgfqpoint{0.510000in}{2.640000in}}%
\pgfpathlineto{\pgfqpoint{3.590000in}{2.640000in}}%
\pgfusepath{stroke}%
\end{pgfscope}%
\begin{pgfscope}%
\pgfsetbuttcap%
\pgfsetroundjoin%
\definecolor{currentfill}{rgb}{0.000000,0.000000,0.000000}%
\pgfsetfillcolor{currentfill}%
\pgfsetlinewidth{0.803000pt}%
\definecolor{currentstroke}{rgb}{0.000000,0.000000,0.000000}%
\pgfsetstrokecolor{currentstroke}%
\pgfsetdash{}{0pt}%
\pgfsys@defobject{currentmarker}{\pgfqpoint{-0.048611in}{0.000000in}}{\pgfqpoint{0.000000in}{0.000000in}}{%
\pgfpathmoveto{\pgfqpoint{0.000000in}{0.000000in}}%
\pgfpathlineto{\pgfqpoint{-0.048611in}{0.000000in}}%
\pgfusepath{stroke,fill}%
}%
\begin{pgfscope}%
\pgfsys@transformshift{0.510000in}{2.640000in}%
\pgfsys@useobject{currentmarker}{}%
\end{pgfscope}%
\end{pgfscope}%
\begin{pgfscope}%
\definecolor{textcolor}{rgb}{0.000000,0.000000,0.000000}%
\pgfsetstrokecolor{textcolor}%
\pgfsetfillcolor{textcolor}%
\pgftext[x=0.191898in,y=2.587238in,left,base]{\color{textcolor}\sffamily\fontsize{10.000000}{12.000000}\selectfont 0.8}%
\end{pgfscope}%
\begin{pgfscope}%
\pgfpathrectangle{\pgfqpoint{0.510000in}{0.440000in}}{\pgfqpoint{3.080000in}{3.080000in}}%
\pgfusepath{clip}%
\pgfsetrectcap%
\pgfsetroundjoin%
\pgfsetlinewidth{0.803000pt}%
\definecolor{currentstroke}{rgb}{0.690196,0.690196,0.690196}%
\pgfsetstrokecolor{currentstroke}%
\pgfsetdash{}{0pt}%
\pgfpathmoveto{\pgfqpoint{0.510000in}{3.080000in}}%
\pgfpathlineto{\pgfqpoint{3.590000in}{3.080000in}}%
\pgfusepath{stroke}%
\end{pgfscope}%
\begin{pgfscope}%
\pgfsetbuttcap%
\pgfsetroundjoin%
\definecolor{currentfill}{rgb}{0.000000,0.000000,0.000000}%
\pgfsetfillcolor{currentfill}%
\pgfsetlinewidth{0.803000pt}%
\definecolor{currentstroke}{rgb}{0.000000,0.000000,0.000000}%
\pgfsetstrokecolor{currentstroke}%
\pgfsetdash{}{0pt}%
\pgfsys@defobject{currentmarker}{\pgfqpoint{-0.048611in}{0.000000in}}{\pgfqpoint{0.000000in}{0.000000in}}{%
\pgfpathmoveto{\pgfqpoint{0.000000in}{0.000000in}}%
\pgfpathlineto{\pgfqpoint{-0.048611in}{0.000000in}}%
\pgfusepath{stroke,fill}%
}%
\begin{pgfscope}%
\pgfsys@transformshift{0.510000in}{3.080000in}%
\pgfsys@useobject{currentmarker}{}%
\end{pgfscope}%
\end{pgfscope}%
\begin{pgfscope}%
\definecolor{textcolor}{rgb}{0.000000,0.000000,0.000000}%
\pgfsetstrokecolor{textcolor}%
\pgfsetfillcolor{textcolor}%
\pgftext[x=0.191898in,y=3.027238in,left,base]{\color{textcolor}\sffamily\fontsize{10.000000}{12.000000}\selectfont 1.0}%
\end{pgfscope}%
\begin{pgfscope}%
\pgfpathrectangle{\pgfqpoint{0.510000in}{0.440000in}}{\pgfqpoint{3.080000in}{3.080000in}}%
\pgfusepath{clip}%
\pgfsetrectcap%
\pgfsetroundjoin%
\pgfsetlinewidth{0.803000pt}%
\definecolor{currentstroke}{rgb}{0.690196,0.690196,0.690196}%
\pgfsetstrokecolor{currentstroke}%
\pgfsetdash{}{0pt}%
\pgfpathmoveto{\pgfqpoint{0.510000in}{3.520000in}}%
\pgfpathlineto{\pgfqpoint{3.590000in}{3.520000in}}%
\pgfusepath{stroke}%
\end{pgfscope}%
\begin{pgfscope}%
\pgfsetbuttcap%
\pgfsetroundjoin%
\definecolor{currentfill}{rgb}{0.000000,0.000000,0.000000}%
\pgfsetfillcolor{currentfill}%
\pgfsetlinewidth{0.803000pt}%
\definecolor{currentstroke}{rgb}{0.000000,0.000000,0.000000}%
\pgfsetstrokecolor{currentstroke}%
\pgfsetdash{}{0pt}%
\pgfsys@defobject{currentmarker}{\pgfqpoint{-0.048611in}{0.000000in}}{\pgfqpoint{0.000000in}{0.000000in}}{%
\pgfpathmoveto{\pgfqpoint{0.000000in}{0.000000in}}%
\pgfpathlineto{\pgfqpoint{-0.048611in}{0.000000in}}%
\pgfusepath{stroke,fill}%
}%
\begin{pgfscope}%
\pgfsys@transformshift{0.510000in}{3.520000in}%
\pgfsys@useobject{currentmarker}{}%
\end{pgfscope}%
\end{pgfscope}%
\begin{pgfscope}%
\definecolor{textcolor}{rgb}{0.000000,0.000000,0.000000}%
\pgfsetstrokecolor{textcolor}%
\pgfsetfillcolor{textcolor}%
\pgftext[x=0.191898in,y=3.467238in,left,base]{\color{textcolor}\sffamily\fontsize{10.000000}{12.000000}\selectfont 1.2}%
\end{pgfscope}%
\begin{pgfscope}%
\pgfsetrectcap%
\pgfsetmiterjoin%
\pgfsetlinewidth{0.803000pt}%
\definecolor{currentstroke}{rgb}{0.000000,0.000000,0.000000}%
\pgfsetstrokecolor{currentstroke}%
\pgfsetdash{}{0pt}%
\pgfpathmoveto{\pgfqpoint{0.510000in}{0.440000in}}%
\pgfpathlineto{\pgfqpoint{0.510000in}{3.520000in}}%
\pgfusepath{stroke}%
\end{pgfscope}%
\begin{pgfscope}%
\pgfsetrectcap%
\pgfsetmiterjoin%
\pgfsetlinewidth{0.803000pt}%
\definecolor{currentstroke}{rgb}{0.000000,0.000000,0.000000}%
\pgfsetstrokecolor{currentstroke}%
\pgfsetdash{}{0pt}%
\pgfpathmoveto{\pgfqpoint{3.590000in}{0.440000in}}%
\pgfpathlineto{\pgfqpoint{3.590000in}{3.520000in}}%
\pgfusepath{stroke}%
\end{pgfscope}%
\begin{pgfscope}%
\pgfsetrectcap%
\pgfsetmiterjoin%
\pgfsetlinewidth{0.803000pt}%
\definecolor{currentstroke}{rgb}{0.000000,0.000000,0.000000}%
\pgfsetstrokecolor{currentstroke}%
\pgfsetdash{}{0pt}%
\pgfpathmoveto{\pgfqpoint{0.510000in}{0.440000in}}%
\pgfpathlineto{\pgfqpoint{3.590000in}{0.440000in}}%
\pgfusepath{stroke}%
\end{pgfscope}%
\begin{pgfscope}%
\pgfsetrectcap%
\pgfsetmiterjoin%
\pgfsetlinewidth{0.803000pt}%
\definecolor{currentstroke}{rgb}{0.000000,0.000000,0.000000}%
\pgfsetstrokecolor{currentstroke}%
\pgfsetdash{}{0pt}%
\pgfpathmoveto{\pgfqpoint{0.510000in}{3.520000in}}%
\pgfpathlineto{\pgfqpoint{3.590000in}{3.520000in}}%
\pgfusepath{stroke}%
\end{pgfscope}%
\begin{pgfscope}%
\definecolor{textcolor}{rgb}{0.000000,0.000000,0.000000}%
\pgfsetstrokecolor{textcolor}%
\pgfsetfillcolor{textcolor}%
\pgftext[x=1.293201in,y=1.270644in,left,base]{\color{textcolor}\sffamily\fontsize{10.000000}{12.000000}\selectfont 0.16667}%
\end{pgfscope}%
\begin{pgfscope}%
\definecolor{textcolor}{rgb}{0.000000,0.000000,0.000000}%
\pgfsetstrokecolor{textcolor}%
\pgfsetfillcolor{textcolor}%
\pgftext[x=2.416155in,y=1.223201in,left,base]{\color{textcolor}\sffamily\fontsize{10.000000}{12.000000}\selectfont 0.16667}%
\end{pgfscope}%
\begin{pgfscope}%
\definecolor{textcolor}{rgb}{0.000000,0.000000,0.000000}%
\pgfsetstrokecolor{textcolor}%
\pgfsetfillcolor{textcolor}%
\pgftext[x=1.340644in,y=2.346155in,left,base]{\color{textcolor}\sffamily\fontsize{10.000000}{12.000000}\selectfont 0.16667}%
\end{pgfscope}%
\end{pgfpicture}%
\makeatother%
\endgroup%
}
\caption{Internal energy $u$ for different temperatures $T$ using Metropolis--Hastings algorithm}
\label{Fig:HeatSmall}
\end{figure}

\begin{figure}[htbp]
\centering
\scalebox{0.666}{%% Creator: Matplotlib, PGF backend
%%
%% To include the figure in your LaTeX document, write
%%   \input{<filename>.pgf}
%%
%% Make sure the required packages are loaded in your preamble
%%   \usepackage{pgf}
%%
%% Figures using additional raster images can only be included by \input if
%% they are in the same directory as the main LaTeX file. For loading figures
%% from other directories you can use the `import` package
%%   \usepackage{import}
%% and then include the figures with
%%   \import{<path to file>}{<filename>.pgf}
%%
%% Matplotlib used the following preamble
%%   \usepackage{fontspec}
%%   \setmainfont{DejaVuSerif.ttf}[Path=/home/lzh/anaconda3/envs/numana/lib/python3.7/site-packages/matplotlib/mpl-data/fonts/ttf/]
%%   \setsansfont{DejaVuSans.ttf}[Path=/home/lzh/anaconda3/envs/numana/lib/python3.7/site-packages/matplotlib/mpl-data/fonts/ttf/]
%%   \setmonofont{DejaVuSansMono.ttf}[Path=/home/lzh/anaconda3/envs/numana/lib/python3.7/site-packages/matplotlib/mpl-data/fonts/ttf/]
%%
\begingroup%
\makeatletter%
\begin{pgfpicture}%
\pgfpathrectangle{\pgfpointorigin}{\pgfqpoint{8.000000in}{6.000000in}}%
\pgfusepath{use as bounding box, clip}%
\begin{pgfscope}%
\pgfsetbuttcap%
\pgfsetmiterjoin%
\definecolor{currentfill}{rgb}{1.000000,1.000000,1.000000}%
\pgfsetfillcolor{currentfill}%
\pgfsetlinewidth{0.000000pt}%
\definecolor{currentstroke}{rgb}{1.000000,1.000000,1.000000}%
\pgfsetstrokecolor{currentstroke}%
\pgfsetdash{}{0pt}%
\pgfpathmoveto{\pgfqpoint{0.000000in}{0.000000in}}%
\pgfpathlineto{\pgfqpoint{8.000000in}{0.000000in}}%
\pgfpathlineto{\pgfqpoint{8.000000in}{6.000000in}}%
\pgfpathlineto{\pgfqpoint{0.000000in}{6.000000in}}%
\pgfpathclose%
\pgfusepath{fill}%
\end{pgfscope}%
\begin{pgfscope}%
\pgfsetbuttcap%
\pgfsetmiterjoin%
\definecolor{currentfill}{rgb}{1.000000,1.000000,1.000000}%
\pgfsetfillcolor{currentfill}%
\pgfsetlinewidth{0.000000pt}%
\definecolor{currentstroke}{rgb}{0.000000,0.000000,0.000000}%
\pgfsetstrokecolor{currentstroke}%
\pgfsetstrokeopacity{0.000000}%
\pgfsetdash{}{0pt}%
\pgfpathmoveto{\pgfqpoint{0.555972in}{3.312222in}}%
\pgfpathlineto{\pgfqpoint{3.933750in}{3.312222in}}%
\pgfpathlineto{\pgfqpoint{3.933750in}{5.636667in}}%
\pgfpathlineto{\pgfqpoint{0.555972in}{5.636667in}}%
\pgfpathclose%
\pgfusepath{fill}%
\end{pgfscope}%
\begin{pgfscope}%
\pgfpathrectangle{\pgfqpoint{0.555972in}{3.312222in}}{\pgfqpoint{3.377778in}{2.324444in}}%
\pgfusepath{clip}%
\pgfsetbuttcap%
\pgfsetroundjoin%
\definecolor{currentfill}{rgb}{0.121569,0.466667,0.705882}%
\pgfsetfillcolor{currentfill}%
\pgfsetfillopacity{0.300000}%
\pgfsetlinewidth{1.003750pt}%
\definecolor{currentstroke}{rgb}{0.121569,0.466667,0.705882}%
\pgfsetstrokecolor{currentstroke}%
\pgfsetstrokeopacity{0.300000}%
\pgfsetdash{}{0pt}%
\pgfpathmoveto{\pgfqpoint{0.717936in}{3.425698in}}%
\pgfpathlineto{\pgfqpoint{0.717936in}{3.425655in}}%
\pgfpathlineto{\pgfqpoint{0.794282in}{3.426017in}}%
\pgfpathlineto{\pgfqpoint{0.870629in}{3.427567in}}%
\pgfpathlineto{\pgfqpoint{0.946975in}{3.431918in}}%
\pgfpathlineto{\pgfqpoint{1.023321in}{3.441084in}}%
\pgfpathlineto{\pgfqpoint{1.099667in}{3.457252in}}%
\pgfpathlineto{\pgfqpoint{1.176014in}{3.481840in}}%
\pgfpathlineto{\pgfqpoint{1.252360in}{3.516407in}}%
\pgfpathlineto{\pgfqpoint{1.328706in}{3.562416in}}%
\pgfpathlineto{\pgfqpoint{1.405052in}{3.621772in}}%
\pgfpathlineto{\pgfqpoint{1.481399in}{3.692245in}}%
\pgfpathlineto{\pgfqpoint{1.557745in}{3.781062in}}%
\pgfpathlineto{\pgfqpoint{1.634091in}{3.888939in}}%
\pgfpathlineto{\pgfqpoint{1.710437in}{4.021209in}}%
\pgfpathlineto{\pgfqpoint{1.786784in}{4.187475in}}%
\pgfpathlineto{\pgfqpoint{1.863130in}{4.402393in}}%
\pgfpathlineto{\pgfqpoint{1.939476in}{4.727603in}}%
\pgfpathlineto{\pgfqpoint{2.015822in}{5.154702in}}%
\pgfpathlineto{\pgfqpoint{2.092169in}{5.508824in}}%
\pgfpathlineto{\pgfqpoint{2.168515in}{5.326882in}}%
\pgfpathlineto{\pgfqpoint{2.244861in}{4.859046in}}%
\pgfpathlineto{\pgfqpoint{2.321207in}{4.510171in}}%
\pgfpathlineto{\pgfqpoint{2.397554in}{4.283031in}}%
\pgfpathlineto{\pgfqpoint{2.473900in}{4.144978in}}%
\pgfpathlineto{\pgfqpoint{2.550246in}{4.047078in}}%
\pgfpathlineto{\pgfqpoint{2.626592in}{3.975828in}}%
\pgfpathlineto{\pgfqpoint{2.702939in}{3.914389in}}%
\pgfpathlineto{\pgfqpoint{2.779285in}{3.866545in}}%
\pgfpathlineto{\pgfqpoint{2.855631in}{3.827140in}}%
\pgfpathlineto{\pgfqpoint{2.931977in}{3.791518in}}%
\pgfpathlineto{\pgfqpoint{3.008324in}{3.761806in}}%
\pgfpathlineto{\pgfqpoint{3.084670in}{3.735893in}}%
\pgfpathlineto{\pgfqpoint{3.161016in}{3.712437in}}%
\pgfpathlineto{\pgfqpoint{3.237362in}{3.692855in}}%
\pgfpathlineto{\pgfqpoint{3.313709in}{3.674578in}}%
\pgfpathlineto{\pgfqpoint{3.390055in}{3.657876in}}%
\pgfpathlineto{\pgfqpoint{3.466401in}{3.643904in}}%
\pgfpathlineto{\pgfqpoint{3.542747in}{3.630210in}}%
\pgfpathlineto{\pgfqpoint{3.619094in}{3.618258in}}%
\pgfpathlineto{\pgfqpoint{3.695440in}{3.606961in}}%
\pgfpathlineto{\pgfqpoint{3.771786in}{3.597166in}}%
\pgfpathlineto{\pgfqpoint{3.771786in}{3.597814in}}%
\pgfpathlineto{\pgfqpoint{3.771786in}{3.597814in}}%
\pgfpathlineto{\pgfqpoint{3.695440in}{3.607729in}}%
\pgfpathlineto{\pgfqpoint{3.619094in}{3.618719in}}%
\pgfpathlineto{\pgfqpoint{3.542747in}{3.631021in}}%
\pgfpathlineto{\pgfqpoint{3.466401in}{3.643962in}}%
\pgfpathlineto{\pgfqpoint{3.390055in}{3.659009in}}%
\pgfpathlineto{\pgfqpoint{3.313709in}{3.675354in}}%
\pgfpathlineto{\pgfqpoint{3.237362in}{3.693404in}}%
\pgfpathlineto{\pgfqpoint{3.161016in}{3.713834in}}%
\pgfpathlineto{\pgfqpoint{3.084670in}{3.737272in}}%
\pgfpathlineto{\pgfqpoint{3.008324in}{3.762571in}}%
\pgfpathlineto{\pgfqpoint{2.931977in}{3.794789in}}%
\pgfpathlineto{\pgfqpoint{2.855631in}{3.828150in}}%
\pgfpathlineto{\pgfqpoint{2.779285in}{3.869921in}}%
\pgfpathlineto{\pgfqpoint{2.702939in}{3.917351in}}%
\pgfpathlineto{\pgfqpoint{2.626592in}{3.976721in}}%
\pgfpathlineto{\pgfqpoint{2.550246in}{4.050506in}}%
\pgfpathlineto{\pgfqpoint{2.473900in}{4.150940in}}%
\pgfpathlineto{\pgfqpoint{2.397554in}{4.290344in}}%
\pgfpathlineto{\pgfqpoint{2.321207in}{4.520253in}}%
\pgfpathlineto{\pgfqpoint{2.244861in}{4.895435in}}%
\pgfpathlineto{\pgfqpoint{2.168515in}{5.353315in}}%
\pgfpathlineto{\pgfqpoint{2.092169in}{5.531010in}}%
\pgfpathlineto{\pgfqpoint{2.015822in}{5.191219in}}%
\pgfpathlineto{\pgfqpoint{1.939476in}{4.742491in}}%
\pgfpathlineto{\pgfqpoint{1.863130in}{4.414088in}}%
\pgfpathlineto{\pgfqpoint{1.786784in}{4.197908in}}%
\pgfpathlineto{\pgfqpoint{1.710437in}{4.024190in}}%
\pgfpathlineto{\pgfqpoint{1.634091in}{3.892507in}}%
\pgfpathlineto{\pgfqpoint{1.557745in}{3.781837in}}%
\pgfpathlineto{\pgfqpoint{1.481399in}{3.694190in}}%
\pgfpathlineto{\pgfqpoint{1.405052in}{3.622252in}}%
\pgfpathlineto{\pgfqpoint{1.328706in}{3.563097in}}%
\pgfpathlineto{\pgfqpoint{1.252360in}{3.517137in}}%
\pgfpathlineto{\pgfqpoint{1.176014in}{3.482639in}}%
\pgfpathlineto{\pgfqpoint{1.099667in}{3.457687in}}%
\pgfpathlineto{\pgfqpoint{1.023321in}{3.441407in}}%
\pgfpathlineto{\pgfqpoint{0.946975in}{3.432025in}}%
\pgfpathlineto{\pgfqpoint{0.870629in}{3.427627in}}%
\pgfpathlineto{\pgfqpoint{0.794282in}{3.426082in}}%
\pgfpathlineto{\pgfqpoint{0.717936in}{3.425698in}}%
\pgfpathclose%
\pgfusepath{stroke,fill}%
\end{pgfscope}%
\begin{pgfscope}%
\pgfpathrectangle{\pgfqpoint{0.555972in}{3.312222in}}{\pgfqpoint{3.377778in}{2.324444in}}%
\pgfusepath{clip}%
\pgfsetbuttcap%
\pgfsetroundjoin%
\definecolor{currentfill}{rgb}{0.121569,0.466667,0.705882}%
\pgfsetfillcolor{currentfill}%
\pgfsetlinewidth{1.003750pt}%
\definecolor{currentstroke}{rgb}{0.121569,0.466667,0.705882}%
\pgfsetstrokecolor{currentstroke}%
\pgfsetdash{}{0pt}%
\pgfsys@defobject{currentmarker}{\pgfqpoint{-0.009821in}{-0.009821in}}{\pgfqpoint{0.009821in}{0.009821in}}{%
\pgfpathmoveto{\pgfqpoint{0.000000in}{-0.009821in}}%
\pgfpathcurveto{\pgfqpoint{0.002605in}{-0.009821in}}{\pgfqpoint{0.005103in}{-0.008786in}}{\pgfqpoint{0.006944in}{-0.006944in}}%
\pgfpathcurveto{\pgfqpoint{0.008786in}{-0.005103in}}{\pgfqpoint{0.009821in}{-0.002605in}}{\pgfqpoint{0.009821in}{0.000000in}}%
\pgfpathcurveto{\pgfqpoint{0.009821in}{0.002605in}}{\pgfqpoint{0.008786in}{0.005103in}}{\pgfqpoint{0.006944in}{0.006944in}}%
\pgfpathcurveto{\pgfqpoint{0.005103in}{0.008786in}}{\pgfqpoint{0.002605in}{0.009821in}}{\pgfqpoint{0.000000in}{0.009821in}}%
\pgfpathcurveto{\pgfqpoint{-0.002605in}{0.009821in}}{\pgfqpoint{-0.005103in}{0.008786in}}{\pgfqpoint{-0.006944in}{0.006944in}}%
\pgfpathcurveto{\pgfqpoint{-0.008786in}{0.005103in}}{\pgfqpoint{-0.009821in}{0.002605in}}{\pgfqpoint{-0.009821in}{0.000000in}}%
\pgfpathcurveto{\pgfqpoint{-0.009821in}{-0.002605in}}{\pgfqpoint{-0.008786in}{-0.005103in}}{\pgfqpoint{-0.006944in}{-0.006944in}}%
\pgfpathcurveto{\pgfqpoint{-0.005103in}{-0.008786in}}{\pgfqpoint{-0.002605in}{-0.009821in}}{\pgfqpoint{0.000000in}{-0.009821in}}%
\pgfpathclose%
\pgfusepath{stroke,fill}%
}%
\begin{pgfscope}%
\pgfsys@transformshift{0.717936in}{3.425677in}%
\pgfsys@useobject{currentmarker}{}%
\end{pgfscope}%
\begin{pgfscope}%
\pgfsys@transformshift{0.794282in}{3.426050in}%
\pgfsys@useobject{currentmarker}{}%
\end{pgfscope}%
\begin{pgfscope}%
\pgfsys@transformshift{0.870629in}{3.427597in}%
\pgfsys@useobject{currentmarker}{}%
\end{pgfscope}%
\begin{pgfscope}%
\pgfsys@transformshift{0.946975in}{3.431972in}%
\pgfsys@useobject{currentmarker}{}%
\end{pgfscope}%
\begin{pgfscope}%
\pgfsys@transformshift{1.023321in}{3.441245in}%
\pgfsys@useobject{currentmarker}{}%
\end{pgfscope}%
\begin{pgfscope}%
\pgfsys@transformshift{1.099667in}{3.457470in}%
\pgfsys@useobject{currentmarker}{}%
\end{pgfscope}%
\begin{pgfscope}%
\pgfsys@transformshift{1.176014in}{3.482239in}%
\pgfsys@useobject{currentmarker}{}%
\end{pgfscope}%
\begin{pgfscope}%
\pgfsys@transformshift{1.252360in}{3.516772in}%
\pgfsys@useobject{currentmarker}{}%
\end{pgfscope}%
\begin{pgfscope}%
\pgfsys@transformshift{1.328706in}{3.562756in}%
\pgfsys@useobject{currentmarker}{}%
\end{pgfscope}%
\begin{pgfscope}%
\pgfsys@transformshift{1.405052in}{3.622012in}%
\pgfsys@useobject{currentmarker}{}%
\end{pgfscope}%
\begin{pgfscope}%
\pgfsys@transformshift{1.481399in}{3.693217in}%
\pgfsys@useobject{currentmarker}{}%
\end{pgfscope}%
\begin{pgfscope}%
\pgfsys@transformshift{1.557745in}{3.781449in}%
\pgfsys@useobject{currentmarker}{}%
\end{pgfscope}%
\begin{pgfscope}%
\pgfsys@transformshift{1.634091in}{3.890723in}%
\pgfsys@useobject{currentmarker}{}%
\end{pgfscope}%
\begin{pgfscope}%
\pgfsys@transformshift{1.710437in}{4.022699in}%
\pgfsys@useobject{currentmarker}{}%
\end{pgfscope}%
\begin{pgfscope}%
\pgfsys@transformshift{1.786784in}{4.192691in}%
\pgfsys@useobject{currentmarker}{}%
\end{pgfscope}%
\begin{pgfscope}%
\pgfsys@transformshift{1.863130in}{4.408240in}%
\pgfsys@useobject{currentmarker}{}%
\end{pgfscope}%
\begin{pgfscope}%
\pgfsys@transformshift{1.939476in}{4.735047in}%
\pgfsys@useobject{currentmarker}{}%
\end{pgfscope}%
\begin{pgfscope}%
\pgfsys@transformshift{2.015822in}{5.172961in}%
\pgfsys@useobject{currentmarker}{}%
\end{pgfscope}%
\begin{pgfscope}%
\pgfsys@transformshift{2.092169in}{5.519917in}%
\pgfsys@useobject{currentmarker}{}%
\end{pgfscope}%
\begin{pgfscope}%
\pgfsys@transformshift{2.168515in}{5.340099in}%
\pgfsys@useobject{currentmarker}{}%
\end{pgfscope}%
\begin{pgfscope}%
\pgfsys@transformshift{2.244861in}{4.877241in}%
\pgfsys@useobject{currentmarker}{}%
\end{pgfscope}%
\begin{pgfscope}%
\pgfsys@transformshift{2.321207in}{4.515212in}%
\pgfsys@useobject{currentmarker}{}%
\end{pgfscope}%
\begin{pgfscope}%
\pgfsys@transformshift{2.397554in}{4.286688in}%
\pgfsys@useobject{currentmarker}{}%
\end{pgfscope}%
\begin{pgfscope}%
\pgfsys@transformshift{2.473900in}{4.147959in}%
\pgfsys@useobject{currentmarker}{}%
\end{pgfscope}%
\begin{pgfscope}%
\pgfsys@transformshift{2.550246in}{4.048792in}%
\pgfsys@useobject{currentmarker}{}%
\end{pgfscope}%
\begin{pgfscope}%
\pgfsys@transformshift{2.626592in}{3.976275in}%
\pgfsys@useobject{currentmarker}{}%
\end{pgfscope}%
\begin{pgfscope}%
\pgfsys@transformshift{2.702939in}{3.915870in}%
\pgfsys@useobject{currentmarker}{}%
\end{pgfscope}%
\begin{pgfscope}%
\pgfsys@transformshift{2.779285in}{3.868233in}%
\pgfsys@useobject{currentmarker}{}%
\end{pgfscope}%
\begin{pgfscope}%
\pgfsys@transformshift{2.855631in}{3.827645in}%
\pgfsys@useobject{currentmarker}{}%
\end{pgfscope}%
\begin{pgfscope}%
\pgfsys@transformshift{2.931977in}{3.793153in}%
\pgfsys@useobject{currentmarker}{}%
\end{pgfscope}%
\begin{pgfscope}%
\pgfsys@transformshift{3.008324in}{3.762188in}%
\pgfsys@useobject{currentmarker}{}%
\end{pgfscope}%
\begin{pgfscope}%
\pgfsys@transformshift{3.084670in}{3.736582in}%
\pgfsys@useobject{currentmarker}{}%
\end{pgfscope}%
\begin{pgfscope}%
\pgfsys@transformshift{3.161016in}{3.713135in}%
\pgfsys@useobject{currentmarker}{}%
\end{pgfscope}%
\begin{pgfscope}%
\pgfsys@transformshift{3.237362in}{3.693130in}%
\pgfsys@useobject{currentmarker}{}%
\end{pgfscope}%
\begin{pgfscope}%
\pgfsys@transformshift{3.313709in}{3.674966in}%
\pgfsys@useobject{currentmarker}{}%
\end{pgfscope}%
\begin{pgfscope}%
\pgfsys@transformshift{3.390055in}{3.658442in}%
\pgfsys@useobject{currentmarker}{}%
\end{pgfscope}%
\begin{pgfscope}%
\pgfsys@transformshift{3.466401in}{3.643933in}%
\pgfsys@useobject{currentmarker}{}%
\end{pgfscope}%
\begin{pgfscope}%
\pgfsys@transformshift{3.542747in}{3.630616in}%
\pgfsys@useobject{currentmarker}{}%
\end{pgfscope}%
\begin{pgfscope}%
\pgfsys@transformshift{3.619094in}{3.618488in}%
\pgfsys@useobject{currentmarker}{}%
\end{pgfscope}%
\begin{pgfscope}%
\pgfsys@transformshift{3.695440in}{3.607345in}%
\pgfsys@useobject{currentmarker}{}%
\end{pgfscope}%
\begin{pgfscope}%
\pgfsys@transformshift{3.771786in}{3.597490in}%
\pgfsys@useobject{currentmarker}{}%
\end{pgfscope}%
\end{pgfscope}%
\begin{pgfscope}%
\pgfsetbuttcap%
\pgfsetroundjoin%
\definecolor{currentfill}{rgb}{0.000000,0.000000,0.000000}%
\pgfsetfillcolor{currentfill}%
\pgfsetlinewidth{0.803000pt}%
\definecolor{currentstroke}{rgb}{0.000000,0.000000,0.000000}%
\pgfsetstrokecolor{currentstroke}%
\pgfsetdash{}{0pt}%
\pgfsys@defobject{currentmarker}{\pgfqpoint{0.000000in}{-0.048611in}}{\pgfqpoint{0.000000in}{0.000000in}}{%
\pgfpathmoveto{\pgfqpoint{0.000000in}{0.000000in}}%
\pgfpathlineto{\pgfqpoint{0.000000in}{-0.048611in}}%
\pgfusepath{stroke,fill}%
}%
\begin{pgfscope}%
\pgfsys@transformshift{1.099667in}{3.312222in}%
\pgfsys@useobject{currentmarker}{}%
\end{pgfscope}%
\end{pgfscope}%
\begin{pgfscope}%
\definecolor{textcolor}{rgb}{0.000000,0.000000,0.000000}%
\pgfsetstrokecolor{textcolor}%
\pgfsetfillcolor{textcolor}%
\pgftext[x=1.099667in,y=3.215000in,,top]{\color{textcolor}\sffamily\fontsize{10.000000}{12.000000}\selectfont 1}%
\end{pgfscope}%
\begin{pgfscope}%
\pgfsetbuttcap%
\pgfsetroundjoin%
\definecolor{currentfill}{rgb}{0.000000,0.000000,0.000000}%
\pgfsetfillcolor{currentfill}%
\pgfsetlinewidth{0.803000pt}%
\definecolor{currentstroke}{rgb}{0.000000,0.000000,0.000000}%
\pgfsetstrokecolor{currentstroke}%
\pgfsetdash{}{0pt}%
\pgfsys@defobject{currentmarker}{\pgfqpoint{0.000000in}{-0.048611in}}{\pgfqpoint{0.000000in}{0.000000in}}{%
\pgfpathmoveto{\pgfqpoint{0.000000in}{0.000000in}}%
\pgfpathlineto{\pgfqpoint{0.000000in}{-0.048611in}}%
\pgfusepath{stroke,fill}%
}%
\begin{pgfscope}%
\pgfsys@transformshift{1.863130in}{3.312222in}%
\pgfsys@useobject{currentmarker}{}%
\end{pgfscope}%
\end{pgfscope}%
\begin{pgfscope}%
\definecolor{textcolor}{rgb}{0.000000,0.000000,0.000000}%
\pgfsetstrokecolor{textcolor}%
\pgfsetfillcolor{textcolor}%
\pgftext[x=1.863130in,y=3.215000in,,top]{\color{textcolor}\sffamily\fontsize{10.000000}{12.000000}\selectfont 2}%
\end{pgfscope}%
\begin{pgfscope}%
\pgfsetbuttcap%
\pgfsetroundjoin%
\definecolor{currentfill}{rgb}{0.000000,0.000000,0.000000}%
\pgfsetfillcolor{currentfill}%
\pgfsetlinewidth{0.803000pt}%
\definecolor{currentstroke}{rgb}{0.000000,0.000000,0.000000}%
\pgfsetstrokecolor{currentstroke}%
\pgfsetdash{}{0pt}%
\pgfsys@defobject{currentmarker}{\pgfqpoint{0.000000in}{-0.048611in}}{\pgfqpoint{0.000000in}{0.000000in}}{%
\pgfpathmoveto{\pgfqpoint{0.000000in}{0.000000in}}%
\pgfpathlineto{\pgfqpoint{0.000000in}{-0.048611in}}%
\pgfusepath{stroke,fill}%
}%
\begin{pgfscope}%
\pgfsys@transformshift{2.626592in}{3.312222in}%
\pgfsys@useobject{currentmarker}{}%
\end{pgfscope}%
\end{pgfscope}%
\begin{pgfscope}%
\definecolor{textcolor}{rgb}{0.000000,0.000000,0.000000}%
\pgfsetstrokecolor{textcolor}%
\pgfsetfillcolor{textcolor}%
\pgftext[x=2.626592in,y=3.215000in,,top]{\color{textcolor}\sffamily\fontsize{10.000000}{12.000000}\selectfont 3}%
\end{pgfscope}%
\begin{pgfscope}%
\pgfsetbuttcap%
\pgfsetroundjoin%
\definecolor{currentfill}{rgb}{0.000000,0.000000,0.000000}%
\pgfsetfillcolor{currentfill}%
\pgfsetlinewidth{0.803000pt}%
\definecolor{currentstroke}{rgb}{0.000000,0.000000,0.000000}%
\pgfsetstrokecolor{currentstroke}%
\pgfsetdash{}{0pt}%
\pgfsys@defobject{currentmarker}{\pgfqpoint{0.000000in}{-0.048611in}}{\pgfqpoint{0.000000in}{0.000000in}}{%
\pgfpathmoveto{\pgfqpoint{0.000000in}{0.000000in}}%
\pgfpathlineto{\pgfqpoint{0.000000in}{-0.048611in}}%
\pgfusepath{stroke,fill}%
}%
\begin{pgfscope}%
\pgfsys@transformshift{3.390055in}{3.312222in}%
\pgfsys@useobject{currentmarker}{}%
\end{pgfscope}%
\end{pgfscope}%
\begin{pgfscope}%
\definecolor{textcolor}{rgb}{0.000000,0.000000,0.000000}%
\pgfsetstrokecolor{textcolor}%
\pgfsetfillcolor{textcolor}%
\pgftext[x=3.390055in,y=3.215000in,,top]{\color{textcolor}\sffamily\fontsize{10.000000}{12.000000}\selectfont 4}%
\end{pgfscope}%
\begin{pgfscope}%
\pgfsetbuttcap%
\pgfsetroundjoin%
\definecolor{currentfill}{rgb}{0.000000,0.000000,0.000000}%
\pgfsetfillcolor{currentfill}%
\pgfsetlinewidth{0.803000pt}%
\definecolor{currentstroke}{rgb}{0.000000,0.000000,0.000000}%
\pgfsetstrokecolor{currentstroke}%
\pgfsetdash{}{0pt}%
\pgfsys@defobject{currentmarker}{\pgfqpoint{-0.048611in}{0.000000in}}{\pgfqpoint{0.000000in}{0.000000in}}{%
\pgfpathmoveto{\pgfqpoint{0.000000in}{0.000000in}}%
\pgfpathlineto{\pgfqpoint{-0.048611in}{0.000000in}}%
\pgfusepath{stroke,fill}%
}%
\begin{pgfscope}%
\pgfsys@transformshift{0.555972in}{3.425636in}%
\pgfsys@useobject{currentmarker}{}%
\end{pgfscope}%
\end{pgfscope}%
\begin{pgfscope}%
\definecolor{textcolor}{rgb}{0.000000,0.000000,0.000000}%
\pgfsetstrokecolor{textcolor}%
\pgfsetfillcolor{textcolor}%
\pgftext[x=0.149505in,y=3.372874in,left,base]{\color{textcolor}\sffamily\fontsize{10.000000}{12.000000}\selectfont 0.00}%
\end{pgfscope}%
\begin{pgfscope}%
\pgfsetbuttcap%
\pgfsetroundjoin%
\definecolor{currentfill}{rgb}{0.000000,0.000000,0.000000}%
\pgfsetfillcolor{currentfill}%
\pgfsetlinewidth{0.803000pt}%
\definecolor{currentstroke}{rgb}{0.000000,0.000000,0.000000}%
\pgfsetstrokecolor{currentstroke}%
\pgfsetdash{}{0pt}%
\pgfsys@defobject{currentmarker}{\pgfqpoint{-0.048611in}{0.000000in}}{\pgfqpoint{0.000000in}{0.000000in}}{%
\pgfpathmoveto{\pgfqpoint{0.000000in}{0.000000in}}%
\pgfpathlineto{\pgfqpoint{-0.048611in}{0.000000in}}%
\pgfusepath{stroke,fill}%
}%
\begin{pgfscope}%
\pgfsys@transformshift{0.555972in}{3.765552in}%
\pgfsys@useobject{currentmarker}{}%
\end{pgfscope}%
\end{pgfscope}%
\begin{pgfscope}%
\definecolor{textcolor}{rgb}{0.000000,0.000000,0.000000}%
\pgfsetstrokecolor{textcolor}%
\pgfsetfillcolor{textcolor}%
\pgftext[x=0.149505in,y=3.712791in,left,base]{\color{textcolor}\sffamily\fontsize{10.000000}{12.000000}\selectfont 0.25}%
\end{pgfscope}%
\begin{pgfscope}%
\pgfsetbuttcap%
\pgfsetroundjoin%
\definecolor{currentfill}{rgb}{0.000000,0.000000,0.000000}%
\pgfsetfillcolor{currentfill}%
\pgfsetlinewidth{0.803000pt}%
\definecolor{currentstroke}{rgb}{0.000000,0.000000,0.000000}%
\pgfsetstrokecolor{currentstroke}%
\pgfsetdash{}{0pt}%
\pgfsys@defobject{currentmarker}{\pgfqpoint{-0.048611in}{0.000000in}}{\pgfqpoint{0.000000in}{0.000000in}}{%
\pgfpathmoveto{\pgfqpoint{0.000000in}{0.000000in}}%
\pgfpathlineto{\pgfqpoint{-0.048611in}{0.000000in}}%
\pgfusepath{stroke,fill}%
}%
\begin{pgfscope}%
\pgfsys@transformshift{0.555972in}{4.105469in}%
\pgfsys@useobject{currentmarker}{}%
\end{pgfscope}%
\end{pgfscope}%
\begin{pgfscope}%
\definecolor{textcolor}{rgb}{0.000000,0.000000,0.000000}%
\pgfsetstrokecolor{textcolor}%
\pgfsetfillcolor{textcolor}%
\pgftext[x=0.149505in,y=4.052707in,left,base]{\color{textcolor}\sffamily\fontsize{10.000000}{12.000000}\selectfont 0.50}%
\end{pgfscope}%
\begin{pgfscope}%
\pgfsetbuttcap%
\pgfsetroundjoin%
\definecolor{currentfill}{rgb}{0.000000,0.000000,0.000000}%
\pgfsetfillcolor{currentfill}%
\pgfsetlinewidth{0.803000pt}%
\definecolor{currentstroke}{rgb}{0.000000,0.000000,0.000000}%
\pgfsetstrokecolor{currentstroke}%
\pgfsetdash{}{0pt}%
\pgfsys@defobject{currentmarker}{\pgfqpoint{-0.048611in}{0.000000in}}{\pgfqpoint{0.000000in}{0.000000in}}{%
\pgfpathmoveto{\pgfqpoint{0.000000in}{0.000000in}}%
\pgfpathlineto{\pgfqpoint{-0.048611in}{0.000000in}}%
\pgfusepath{stroke,fill}%
}%
\begin{pgfscope}%
\pgfsys@transformshift{0.555972in}{4.445385in}%
\pgfsys@useobject{currentmarker}{}%
\end{pgfscope}%
\end{pgfscope}%
\begin{pgfscope}%
\definecolor{textcolor}{rgb}{0.000000,0.000000,0.000000}%
\pgfsetstrokecolor{textcolor}%
\pgfsetfillcolor{textcolor}%
\pgftext[x=0.149505in,y=4.392623in,left,base]{\color{textcolor}\sffamily\fontsize{10.000000}{12.000000}\selectfont 0.75}%
\end{pgfscope}%
\begin{pgfscope}%
\pgfsetbuttcap%
\pgfsetroundjoin%
\definecolor{currentfill}{rgb}{0.000000,0.000000,0.000000}%
\pgfsetfillcolor{currentfill}%
\pgfsetlinewidth{0.803000pt}%
\definecolor{currentstroke}{rgb}{0.000000,0.000000,0.000000}%
\pgfsetstrokecolor{currentstroke}%
\pgfsetdash{}{0pt}%
\pgfsys@defobject{currentmarker}{\pgfqpoint{-0.048611in}{0.000000in}}{\pgfqpoint{0.000000in}{0.000000in}}{%
\pgfpathmoveto{\pgfqpoint{0.000000in}{0.000000in}}%
\pgfpathlineto{\pgfqpoint{-0.048611in}{0.000000in}}%
\pgfusepath{stroke,fill}%
}%
\begin{pgfscope}%
\pgfsys@transformshift{0.555972in}{4.785301in}%
\pgfsys@useobject{currentmarker}{}%
\end{pgfscope}%
\end{pgfscope}%
\begin{pgfscope}%
\definecolor{textcolor}{rgb}{0.000000,0.000000,0.000000}%
\pgfsetstrokecolor{textcolor}%
\pgfsetfillcolor{textcolor}%
\pgftext[x=0.149505in,y=4.732540in,left,base]{\color{textcolor}\sffamily\fontsize{10.000000}{12.000000}\selectfont 1.00}%
\end{pgfscope}%
\begin{pgfscope}%
\pgfsetbuttcap%
\pgfsetroundjoin%
\definecolor{currentfill}{rgb}{0.000000,0.000000,0.000000}%
\pgfsetfillcolor{currentfill}%
\pgfsetlinewidth{0.803000pt}%
\definecolor{currentstroke}{rgb}{0.000000,0.000000,0.000000}%
\pgfsetstrokecolor{currentstroke}%
\pgfsetdash{}{0pt}%
\pgfsys@defobject{currentmarker}{\pgfqpoint{-0.048611in}{0.000000in}}{\pgfqpoint{0.000000in}{0.000000in}}{%
\pgfpathmoveto{\pgfqpoint{0.000000in}{0.000000in}}%
\pgfpathlineto{\pgfqpoint{-0.048611in}{0.000000in}}%
\pgfusepath{stroke,fill}%
}%
\begin{pgfscope}%
\pgfsys@transformshift{0.555972in}{5.125218in}%
\pgfsys@useobject{currentmarker}{}%
\end{pgfscope}%
\end{pgfscope}%
\begin{pgfscope}%
\definecolor{textcolor}{rgb}{0.000000,0.000000,0.000000}%
\pgfsetstrokecolor{textcolor}%
\pgfsetfillcolor{textcolor}%
\pgftext[x=0.149505in,y=5.072456in,left,base]{\color{textcolor}\sffamily\fontsize{10.000000}{12.000000}\selectfont 1.25}%
\end{pgfscope}%
\begin{pgfscope}%
\pgfsetbuttcap%
\pgfsetroundjoin%
\definecolor{currentfill}{rgb}{0.000000,0.000000,0.000000}%
\pgfsetfillcolor{currentfill}%
\pgfsetlinewidth{0.803000pt}%
\definecolor{currentstroke}{rgb}{0.000000,0.000000,0.000000}%
\pgfsetstrokecolor{currentstroke}%
\pgfsetdash{}{0pt}%
\pgfsys@defobject{currentmarker}{\pgfqpoint{-0.048611in}{0.000000in}}{\pgfqpoint{0.000000in}{0.000000in}}{%
\pgfpathmoveto{\pgfqpoint{0.000000in}{0.000000in}}%
\pgfpathlineto{\pgfqpoint{-0.048611in}{0.000000in}}%
\pgfusepath{stroke,fill}%
}%
\begin{pgfscope}%
\pgfsys@transformshift{0.555972in}{5.465134in}%
\pgfsys@useobject{currentmarker}{}%
\end{pgfscope}%
\end{pgfscope}%
\begin{pgfscope}%
\definecolor{textcolor}{rgb}{0.000000,0.000000,0.000000}%
\pgfsetstrokecolor{textcolor}%
\pgfsetfillcolor{textcolor}%
\pgftext[x=0.149505in,y=5.412372in,left,base]{\color{textcolor}\sffamily\fontsize{10.000000}{12.000000}\selectfont 1.50}%
\end{pgfscope}%
\begin{pgfscope}%
\pgfpathrectangle{\pgfqpoint{0.555972in}{3.312222in}}{\pgfqpoint{3.377778in}{2.324444in}}%
\pgfusepath{clip}%
\pgfsetrectcap%
\pgfsetroundjoin%
\pgfsetlinewidth{1.505625pt}%
\definecolor{currentstroke}{rgb}{0.121569,0.466667,0.705882}%
\pgfsetstrokecolor{currentstroke}%
\pgfsetdash{}{0pt}%
\pgfpathmoveto{\pgfqpoint{0.717936in}{3.425677in}}%
\pgfpathlineto{\pgfqpoint{0.794282in}{3.426050in}}%
\pgfpathlineto{\pgfqpoint{0.870629in}{3.427597in}}%
\pgfpathlineto{\pgfqpoint{0.946975in}{3.431972in}}%
\pgfpathlineto{\pgfqpoint{1.023321in}{3.441245in}}%
\pgfpathlineto{\pgfqpoint{1.099667in}{3.457470in}}%
\pgfpathlineto{\pgfqpoint{1.176014in}{3.482239in}}%
\pgfpathlineto{\pgfqpoint{1.252360in}{3.516772in}}%
\pgfpathlineto{\pgfqpoint{1.328706in}{3.562756in}}%
\pgfpathlineto{\pgfqpoint{1.405052in}{3.622012in}}%
\pgfpathlineto{\pgfqpoint{1.481399in}{3.693217in}}%
\pgfpathlineto{\pgfqpoint{1.557745in}{3.781449in}}%
\pgfpathlineto{\pgfqpoint{1.634091in}{3.890723in}}%
\pgfpathlineto{\pgfqpoint{1.710437in}{4.022699in}}%
\pgfpathlineto{\pgfqpoint{1.786784in}{4.192691in}}%
\pgfpathlineto{\pgfqpoint{1.863130in}{4.408240in}}%
\pgfpathlineto{\pgfqpoint{1.939476in}{4.735047in}}%
\pgfpathlineto{\pgfqpoint{2.015822in}{5.172961in}}%
\pgfpathlineto{\pgfqpoint{2.092169in}{5.519917in}}%
\pgfpathlineto{\pgfqpoint{2.168515in}{5.340099in}}%
\pgfpathlineto{\pgfqpoint{2.244861in}{4.877241in}}%
\pgfpathlineto{\pgfqpoint{2.321207in}{4.515212in}}%
\pgfpathlineto{\pgfqpoint{2.397554in}{4.286688in}}%
\pgfpathlineto{\pgfqpoint{2.473900in}{4.147959in}}%
\pgfpathlineto{\pgfqpoint{2.550246in}{4.048792in}}%
\pgfpathlineto{\pgfqpoint{2.626592in}{3.976275in}}%
\pgfpathlineto{\pgfqpoint{2.702939in}{3.915870in}}%
\pgfpathlineto{\pgfqpoint{2.779285in}{3.868233in}}%
\pgfpathlineto{\pgfqpoint{2.855631in}{3.827645in}}%
\pgfpathlineto{\pgfqpoint{2.931977in}{3.793153in}}%
\pgfpathlineto{\pgfqpoint{3.008324in}{3.762188in}}%
\pgfpathlineto{\pgfqpoint{3.084670in}{3.736582in}}%
\pgfpathlineto{\pgfqpoint{3.161016in}{3.713135in}}%
\pgfpathlineto{\pgfqpoint{3.237362in}{3.693130in}}%
\pgfpathlineto{\pgfqpoint{3.313709in}{3.674966in}}%
\pgfpathlineto{\pgfqpoint{3.390055in}{3.658442in}}%
\pgfpathlineto{\pgfqpoint{3.466401in}{3.643933in}}%
\pgfpathlineto{\pgfqpoint{3.542747in}{3.630616in}}%
\pgfpathlineto{\pgfqpoint{3.619094in}{3.618488in}}%
\pgfpathlineto{\pgfqpoint{3.695440in}{3.607345in}}%
\pgfpathlineto{\pgfqpoint{3.771786in}{3.597490in}}%
\pgfusepath{stroke}%
\end{pgfscope}%
\begin{pgfscope}%
\pgfsetrectcap%
\pgfsetmiterjoin%
\pgfsetlinewidth{0.803000pt}%
\definecolor{currentstroke}{rgb}{0.000000,0.000000,0.000000}%
\pgfsetstrokecolor{currentstroke}%
\pgfsetdash{}{0pt}%
\pgfpathmoveto{\pgfqpoint{0.555972in}{3.312222in}}%
\pgfpathlineto{\pgfqpoint{0.555972in}{5.636667in}}%
\pgfusepath{stroke}%
\end{pgfscope}%
\begin{pgfscope}%
\pgfsetrectcap%
\pgfsetmiterjoin%
\pgfsetlinewidth{0.803000pt}%
\definecolor{currentstroke}{rgb}{0.000000,0.000000,0.000000}%
\pgfsetstrokecolor{currentstroke}%
\pgfsetdash{}{0pt}%
\pgfpathmoveto{\pgfqpoint{3.933750in}{3.312222in}}%
\pgfpathlineto{\pgfqpoint{3.933750in}{5.636667in}}%
\pgfusepath{stroke}%
\end{pgfscope}%
\begin{pgfscope}%
\pgfsetrectcap%
\pgfsetmiterjoin%
\pgfsetlinewidth{0.803000pt}%
\definecolor{currentstroke}{rgb}{0.000000,0.000000,0.000000}%
\pgfsetstrokecolor{currentstroke}%
\pgfsetdash{}{0pt}%
\pgfpathmoveto{\pgfqpoint{0.555972in}{3.312222in}}%
\pgfpathlineto{\pgfqpoint{3.933750in}{3.312222in}}%
\pgfusepath{stroke}%
\end{pgfscope}%
\begin{pgfscope}%
\pgfsetrectcap%
\pgfsetmiterjoin%
\pgfsetlinewidth{0.803000pt}%
\definecolor{currentstroke}{rgb}{0.000000,0.000000,0.000000}%
\pgfsetstrokecolor{currentstroke}%
\pgfsetdash{}{0pt}%
\pgfpathmoveto{\pgfqpoint{0.555972in}{5.636667in}}%
\pgfpathlineto{\pgfqpoint{3.933750in}{5.636667in}}%
\pgfusepath{stroke}%
\end{pgfscope}%
\begin{pgfscope}%
\definecolor{textcolor}{rgb}{0.000000,0.000000,0.000000}%
\pgfsetstrokecolor{textcolor}%
\pgfsetfillcolor{textcolor}%
\pgftext[x=2.244861in,y=5.720000in,,base]{\color{textcolor}\sffamily\fontsize{12.000000}{14.400000}\selectfont \(\displaystyle  N = 16 \)}%
\end{pgfscope}%
\begin{pgfscope}%
\pgfsetbuttcap%
\pgfsetmiterjoin%
\definecolor{currentfill}{rgb}{1.000000,1.000000,1.000000}%
\pgfsetfillcolor{currentfill}%
\pgfsetlinewidth{0.000000pt}%
\definecolor{currentstroke}{rgb}{0.000000,0.000000,0.000000}%
\pgfsetstrokecolor{currentstroke}%
\pgfsetstrokeopacity{0.000000}%
\pgfsetdash{}{0pt}%
\pgfpathmoveto{\pgfqpoint{4.437222in}{3.312222in}}%
\pgfpathlineto{\pgfqpoint{7.815000in}{3.312222in}}%
\pgfpathlineto{\pgfqpoint{7.815000in}{5.636667in}}%
\pgfpathlineto{\pgfqpoint{4.437222in}{5.636667in}}%
\pgfpathclose%
\pgfusepath{fill}%
\end{pgfscope}%
\begin{pgfscope}%
\pgfpathrectangle{\pgfqpoint{4.437222in}{3.312222in}}{\pgfqpoint{3.377778in}{2.324444in}}%
\pgfusepath{clip}%
\pgfsetbuttcap%
\pgfsetroundjoin%
\definecolor{currentfill}{rgb}{0.121569,0.466667,0.705882}%
\pgfsetfillcolor{currentfill}%
\pgfsetfillopacity{0.300000}%
\pgfsetlinewidth{1.003750pt}%
\definecolor{currentstroke}{rgb}{0.121569,0.466667,0.705882}%
\pgfsetstrokecolor{currentstroke}%
\pgfsetstrokeopacity{0.300000}%
\pgfsetdash{}{0pt}%
\pgfpathmoveto{\pgfqpoint{4.599186in}{3.425689in}}%
\pgfpathlineto{\pgfqpoint{4.599186in}{3.425665in}}%
\pgfpathlineto{\pgfqpoint{4.675532in}{3.425944in}}%
\pgfpathlineto{\pgfqpoint{4.751879in}{3.427170in}}%
\pgfpathlineto{\pgfqpoint{4.828225in}{3.430506in}}%
\pgfpathlineto{\pgfqpoint{4.904571in}{3.437714in}}%
\pgfpathlineto{\pgfqpoint{4.980917in}{3.450282in}}%
\pgfpathlineto{\pgfqpoint{5.057264in}{3.469400in}}%
\pgfpathlineto{\pgfqpoint{5.133610in}{3.496155in}}%
\pgfpathlineto{\pgfqpoint{5.209956in}{3.531714in}}%
\pgfpathlineto{\pgfqpoint{5.286302in}{3.577873in}}%
\pgfpathlineto{\pgfqpoint{5.362649in}{3.633825in}}%
\pgfpathlineto{\pgfqpoint{5.438995in}{3.701892in}}%
\pgfpathlineto{\pgfqpoint{5.515341in}{3.785830in}}%
\pgfpathlineto{\pgfqpoint{5.591687in}{3.891394in}}%
\pgfpathlineto{\pgfqpoint{5.668034in}{4.015454in}}%
\pgfpathlineto{\pgfqpoint{5.744380in}{4.183352in}}%
\pgfpathlineto{\pgfqpoint{5.820726in}{4.440155in}}%
\pgfpathlineto{\pgfqpoint{5.897072in}{4.864631in}}%
\pgfpathlineto{\pgfqpoint{5.973419in}{5.410832in}}%
\pgfpathlineto{\pgfqpoint{6.049765in}{4.766220in}}%
\pgfpathlineto{\pgfqpoint{6.126111in}{4.359934in}}%
\pgfpathlineto{\pgfqpoint{6.202457in}{4.168319in}}%
\pgfpathlineto{\pgfqpoint{6.278804in}{4.057337in}}%
\pgfpathlineto{\pgfqpoint{6.355150in}{3.968664in}}%
\pgfpathlineto{\pgfqpoint{6.431496in}{3.901629in}}%
\pgfpathlineto{\pgfqpoint{6.507842in}{3.847316in}}%
\pgfpathlineto{\pgfqpoint{6.584189in}{3.805610in}}%
\pgfpathlineto{\pgfqpoint{6.660535in}{3.768730in}}%
\pgfpathlineto{\pgfqpoint{6.736881in}{3.735608in}}%
\pgfpathlineto{\pgfqpoint{6.813227in}{3.710005in}}%
\pgfpathlineto{\pgfqpoint{6.889574in}{3.687215in}}%
\pgfpathlineto{\pgfqpoint{6.965920in}{3.666447in}}%
\pgfpathlineto{\pgfqpoint{7.042266in}{3.649460in}}%
\pgfpathlineto{\pgfqpoint{7.118612in}{3.632544in}}%
\pgfpathlineto{\pgfqpoint{7.194959in}{3.619071in}}%
\pgfpathlineto{\pgfqpoint{7.271305in}{3.606682in}}%
\pgfpathlineto{\pgfqpoint{7.347651in}{3.595236in}}%
\pgfpathlineto{\pgfqpoint{7.423997in}{3.584247in}}%
\pgfpathlineto{\pgfqpoint{7.500344in}{3.575546in}}%
\pgfpathlineto{\pgfqpoint{7.576690in}{3.565937in}}%
\pgfpathlineto{\pgfqpoint{7.653036in}{3.559118in}}%
\pgfpathlineto{\pgfqpoint{7.653036in}{3.559821in}}%
\pgfpathlineto{\pgfqpoint{7.653036in}{3.559821in}}%
\pgfpathlineto{\pgfqpoint{7.576690in}{3.567626in}}%
\pgfpathlineto{\pgfqpoint{7.500344in}{3.575943in}}%
\pgfpathlineto{\pgfqpoint{7.423997in}{3.585426in}}%
\pgfpathlineto{\pgfqpoint{7.347651in}{3.596561in}}%
\pgfpathlineto{\pgfqpoint{7.271305in}{3.607438in}}%
\pgfpathlineto{\pgfqpoint{7.194959in}{3.619895in}}%
\pgfpathlineto{\pgfqpoint{7.118612in}{3.635053in}}%
\pgfpathlineto{\pgfqpoint{7.042266in}{3.650594in}}%
\pgfpathlineto{\pgfqpoint{6.965920in}{3.669272in}}%
\pgfpathlineto{\pgfqpoint{6.889574in}{3.689036in}}%
\pgfpathlineto{\pgfqpoint{6.813227in}{3.712922in}}%
\pgfpathlineto{\pgfqpoint{6.736881in}{3.741014in}}%
\pgfpathlineto{\pgfqpoint{6.660535in}{3.770592in}}%
\pgfpathlineto{\pgfqpoint{6.584189in}{3.809491in}}%
\pgfpathlineto{\pgfqpoint{6.507842in}{3.851276in}}%
\pgfpathlineto{\pgfqpoint{6.431496in}{3.907246in}}%
\pgfpathlineto{\pgfqpoint{6.355150in}{3.975005in}}%
\pgfpathlineto{\pgfqpoint{6.278804in}{4.062718in}}%
\pgfpathlineto{\pgfqpoint{6.202457in}{4.183350in}}%
\pgfpathlineto{\pgfqpoint{6.126111in}{4.381700in}}%
\pgfpathlineto{\pgfqpoint{6.049765in}{4.815445in}}%
\pgfpathlineto{\pgfqpoint{5.973419in}{5.531010in}}%
\pgfpathlineto{\pgfqpoint{5.897072in}{4.914749in}}%
\pgfpathlineto{\pgfqpoint{5.820726in}{4.466961in}}%
\pgfpathlineto{\pgfqpoint{5.744380in}{4.202666in}}%
\pgfpathlineto{\pgfqpoint{5.668034in}{4.025565in}}%
\pgfpathlineto{\pgfqpoint{5.591687in}{3.895271in}}%
\pgfpathlineto{\pgfqpoint{5.515341in}{3.791573in}}%
\pgfpathlineto{\pgfqpoint{5.438995in}{3.706623in}}%
\pgfpathlineto{\pgfqpoint{5.362649in}{3.635408in}}%
\pgfpathlineto{\pgfqpoint{5.286302in}{3.579049in}}%
\pgfpathlineto{\pgfqpoint{5.209956in}{3.533136in}}%
\pgfpathlineto{\pgfqpoint{5.133610in}{3.497145in}}%
\pgfpathlineto{\pgfqpoint{5.057264in}{3.469869in}}%
\pgfpathlineto{\pgfqpoint{4.980917in}{3.450659in}}%
\pgfpathlineto{\pgfqpoint{4.904571in}{3.438019in}}%
\pgfpathlineto{\pgfqpoint{4.828225in}{3.430685in}}%
\pgfpathlineto{\pgfqpoint{4.751879in}{3.427234in}}%
\pgfpathlineto{\pgfqpoint{4.675532in}{3.425972in}}%
\pgfpathlineto{\pgfqpoint{4.599186in}{3.425689in}}%
\pgfpathclose%
\pgfusepath{stroke,fill}%
\end{pgfscope}%
\begin{pgfscope}%
\pgfpathrectangle{\pgfqpoint{4.437222in}{3.312222in}}{\pgfqpoint{3.377778in}{2.324444in}}%
\pgfusepath{clip}%
\pgfsetbuttcap%
\pgfsetroundjoin%
\definecolor{currentfill}{rgb}{0.121569,0.466667,0.705882}%
\pgfsetfillcolor{currentfill}%
\pgfsetlinewidth{1.003750pt}%
\definecolor{currentstroke}{rgb}{0.121569,0.466667,0.705882}%
\pgfsetstrokecolor{currentstroke}%
\pgfsetdash{}{0pt}%
\pgfsys@defobject{currentmarker}{\pgfqpoint{-0.009821in}{-0.009821in}}{\pgfqpoint{0.009821in}{0.009821in}}{%
\pgfpathmoveto{\pgfqpoint{0.000000in}{-0.009821in}}%
\pgfpathcurveto{\pgfqpoint{0.002605in}{-0.009821in}}{\pgfqpoint{0.005103in}{-0.008786in}}{\pgfqpoint{0.006944in}{-0.006944in}}%
\pgfpathcurveto{\pgfqpoint{0.008786in}{-0.005103in}}{\pgfqpoint{0.009821in}{-0.002605in}}{\pgfqpoint{0.009821in}{0.000000in}}%
\pgfpathcurveto{\pgfqpoint{0.009821in}{0.002605in}}{\pgfqpoint{0.008786in}{0.005103in}}{\pgfqpoint{0.006944in}{0.006944in}}%
\pgfpathcurveto{\pgfqpoint{0.005103in}{0.008786in}}{\pgfqpoint{0.002605in}{0.009821in}}{\pgfqpoint{0.000000in}{0.009821in}}%
\pgfpathcurveto{\pgfqpoint{-0.002605in}{0.009821in}}{\pgfqpoint{-0.005103in}{0.008786in}}{\pgfqpoint{-0.006944in}{0.006944in}}%
\pgfpathcurveto{\pgfqpoint{-0.008786in}{0.005103in}}{\pgfqpoint{-0.009821in}{0.002605in}}{\pgfqpoint{-0.009821in}{0.000000in}}%
\pgfpathcurveto{\pgfqpoint{-0.009821in}{-0.002605in}}{\pgfqpoint{-0.008786in}{-0.005103in}}{\pgfqpoint{-0.006944in}{-0.006944in}}%
\pgfpathcurveto{\pgfqpoint{-0.005103in}{-0.008786in}}{\pgfqpoint{-0.002605in}{-0.009821in}}{\pgfqpoint{0.000000in}{-0.009821in}}%
\pgfpathclose%
\pgfusepath{stroke,fill}%
}%
\begin{pgfscope}%
\pgfsys@transformshift{4.599186in}{3.425677in}%
\pgfsys@useobject{currentmarker}{}%
\end{pgfscope}%
\begin{pgfscope}%
\pgfsys@transformshift{4.675532in}{3.425958in}%
\pgfsys@useobject{currentmarker}{}%
\end{pgfscope}%
\begin{pgfscope}%
\pgfsys@transformshift{4.751879in}{3.427202in}%
\pgfsys@useobject{currentmarker}{}%
\end{pgfscope}%
\begin{pgfscope}%
\pgfsys@transformshift{4.828225in}{3.430595in}%
\pgfsys@useobject{currentmarker}{}%
\end{pgfscope}%
\begin{pgfscope}%
\pgfsys@transformshift{4.904571in}{3.437866in}%
\pgfsys@useobject{currentmarker}{}%
\end{pgfscope}%
\begin{pgfscope}%
\pgfsys@transformshift{4.980917in}{3.450471in}%
\pgfsys@useobject{currentmarker}{}%
\end{pgfscope}%
\begin{pgfscope}%
\pgfsys@transformshift{5.057264in}{3.469635in}%
\pgfsys@useobject{currentmarker}{}%
\end{pgfscope}%
\begin{pgfscope}%
\pgfsys@transformshift{5.133610in}{3.496650in}%
\pgfsys@useobject{currentmarker}{}%
\end{pgfscope}%
\begin{pgfscope}%
\pgfsys@transformshift{5.209956in}{3.532425in}%
\pgfsys@useobject{currentmarker}{}%
\end{pgfscope}%
\begin{pgfscope}%
\pgfsys@transformshift{5.286302in}{3.578461in}%
\pgfsys@useobject{currentmarker}{}%
\end{pgfscope}%
\begin{pgfscope}%
\pgfsys@transformshift{5.362649in}{3.634616in}%
\pgfsys@useobject{currentmarker}{}%
\end{pgfscope}%
\begin{pgfscope}%
\pgfsys@transformshift{5.438995in}{3.704258in}%
\pgfsys@useobject{currentmarker}{}%
\end{pgfscope}%
\begin{pgfscope}%
\pgfsys@transformshift{5.515341in}{3.788702in}%
\pgfsys@useobject{currentmarker}{}%
\end{pgfscope}%
\begin{pgfscope}%
\pgfsys@transformshift{5.591687in}{3.893333in}%
\pgfsys@useobject{currentmarker}{}%
\end{pgfscope}%
\begin{pgfscope}%
\pgfsys@transformshift{5.668034in}{4.020510in}%
\pgfsys@useobject{currentmarker}{}%
\end{pgfscope}%
\begin{pgfscope}%
\pgfsys@transformshift{5.744380in}{4.193009in}%
\pgfsys@useobject{currentmarker}{}%
\end{pgfscope}%
\begin{pgfscope}%
\pgfsys@transformshift{5.820726in}{4.453558in}%
\pgfsys@useobject{currentmarker}{}%
\end{pgfscope}%
\begin{pgfscope}%
\pgfsys@transformshift{5.897072in}{4.889690in}%
\pgfsys@useobject{currentmarker}{}%
\end{pgfscope}%
\begin{pgfscope}%
\pgfsys@transformshift{5.973419in}{5.470921in}%
\pgfsys@useobject{currentmarker}{}%
\end{pgfscope}%
\begin{pgfscope}%
\pgfsys@transformshift{6.049765in}{4.790832in}%
\pgfsys@useobject{currentmarker}{}%
\end{pgfscope}%
\begin{pgfscope}%
\pgfsys@transformshift{6.126111in}{4.370817in}%
\pgfsys@useobject{currentmarker}{}%
\end{pgfscope}%
\begin{pgfscope}%
\pgfsys@transformshift{6.202457in}{4.175834in}%
\pgfsys@useobject{currentmarker}{}%
\end{pgfscope}%
\begin{pgfscope}%
\pgfsys@transformshift{6.278804in}{4.060028in}%
\pgfsys@useobject{currentmarker}{}%
\end{pgfscope}%
\begin{pgfscope}%
\pgfsys@transformshift{6.355150in}{3.971834in}%
\pgfsys@useobject{currentmarker}{}%
\end{pgfscope}%
\begin{pgfscope}%
\pgfsys@transformshift{6.431496in}{3.904437in}%
\pgfsys@useobject{currentmarker}{}%
\end{pgfscope}%
\begin{pgfscope}%
\pgfsys@transformshift{6.507842in}{3.849296in}%
\pgfsys@useobject{currentmarker}{}%
\end{pgfscope}%
\begin{pgfscope}%
\pgfsys@transformshift{6.584189in}{3.807550in}%
\pgfsys@useobject{currentmarker}{}%
\end{pgfscope}%
\begin{pgfscope}%
\pgfsys@transformshift{6.660535in}{3.769661in}%
\pgfsys@useobject{currentmarker}{}%
\end{pgfscope}%
\begin{pgfscope}%
\pgfsys@transformshift{6.736881in}{3.738311in}%
\pgfsys@useobject{currentmarker}{}%
\end{pgfscope}%
\begin{pgfscope}%
\pgfsys@transformshift{6.813227in}{3.711464in}%
\pgfsys@useobject{currentmarker}{}%
\end{pgfscope}%
\begin{pgfscope}%
\pgfsys@transformshift{6.889574in}{3.688126in}%
\pgfsys@useobject{currentmarker}{}%
\end{pgfscope}%
\begin{pgfscope}%
\pgfsys@transformshift{6.965920in}{3.667859in}%
\pgfsys@useobject{currentmarker}{}%
\end{pgfscope}%
\begin{pgfscope}%
\pgfsys@transformshift{7.042266in}{3.650027in}%
\pgfsys@useobject{currentmarker}{}%
\end{pgfscope}%
\begin{pgfscope}%
\pgfsys@transformshift{7.118612in}{3.633799in}%
\pgfsys@useobject{currentmarker}{}%
\end{pgfscope}%
\begin{pgfscope}%
\pgfsys@transformshift{7.194959in}{3.619483in}%
\pgfsys@useobject{currentmarker}{}%
\end{pgfscope}%
\begin{pgfscope}%
\pgfsys@transformshift{7.271305in}{3.607060in}%
\pgfsys@useobject{currentmarker}{}%
\end{pgfscope}%
\begin{pgfscope}%
\pgfsys@transformshift{7.347651in}{3.595899in}%
\pgfsys@useobject{currentmarker}{}%
\end{pgfscope}%
\begin{pgfscope}%
\pgfsys@transformshift{7.423997in}{3.584837in}%
\pgfsys@useobject{currentmarker}{}%
\end{pgfscope}%
\begin{pgfscope}%
\pgfsys@transformshift{7.500344in}{3.575745in}%
\pgfsys@useobject{currentmarker}{}%
\end{pgfscope}%
\begin{pgfscope}%
\pgfsys@transformshift{7.576690in}{3.566781in}%
\pgfsys@useobject{currentmarker}{}%
\end{pgfscope}%
\begin{pgfscope}%
\pgfsys@transformshift{7.653036in}{3.559469in}%
\pgfsys@useobject{currentmarker}{}%
\end{pgfscope}%
\end{pgfscope}%
\begin{pgfscope}%
\pgfsetbuttcap%
\pgfsetroundjoin%
\definecolor{currentfill}{rgb}{0.000000,0.000000,0.000000}%
\pgfsetfillcolor{currentfill}%
\pgfsetlinewidth{0.803000pt}%
\definecolor{currentstroke}{rgb}{0.000000,0.000000,0.000000}%
\pgfsetstrokecolor{currentstroke}%
\pgfsetdash{}{0pt}%
\pgfsys@defobject{currentmarker}{\pgfqpoint{0.000000in}{-0.048611in}}{\pgfqpoint{0.000000in}{0.000000in}}{%
\pgfpathmoveto{\pgfqpoint{0.000000in}{0.000000in}}%
\pgfpathlineto{\pgfqpoint{0.000000in}{-0.048611in}}%
\pgfusepath{stroke,fill}%
}%
\begin{pgfscope}%
\pgfsys@transformshift{4.980917in}{3.312222in}%
\pgfsys@useobject{currentmarker}{}%
\end{pgfscope}%
\end{pgfscope}%
\begin{pgfscope}%
\definecolor{textcolor}{rgb}{0.000000,0.000000,0.000000}%
\pgfsetstrokecolor{textcolor}%
\pgfsetfillcolor{textcolor}%
\pgftext[x=4.980917in,y=3.215000in,,top]{\color{textcolor}\sffamily\fontsize{10.000000}{12.000000}\selectfont 1}%
\end{pgfscope}%
\begin{pgfscope}%
\pgfsetbuttcap%
\pgfsetroundjoin%
\definecolor{currentfill}{rgb}{0.000000,0.000000,0.000000}%
\pgfsetfillcolor{currentfill}%
\pgfsetlinewidth{0.803000pt}%
\definecolor{currentstroke}{rgb}{0.000000,0.000000,0.000000}%
\pgfsetstrokecolor{currentstroke}%
\pgfsetdash{}{0pt}%
\pgfsys@defobject{currentmarker}{\pgfqpoint{0.000000in}{-0.048611in}}{\pgfqpoint{0.000000in}{0.000000in}}{%
\pgfpathmoveto{\pgfqpoint{0.000000in}{0.000000in}}%
\pgfpathlineto{\pgfqpoint{0.000000in}{-0.048611in}}%
\pgfusepath{stroke,fill}%
}%
\begin{pgfscope}%
\pgfsys@transformshift{5.744380in}{3.312222in}%
\pgfsys@useobject{currentmarker}{}%
\end{pgfscope}%
\end{pgfscope}%
\begin{pgfscope}%
\definecolor{textcolor}{rgb}{0.000000,0.000000,0.000000}%
\pgfsetstrokecolor{textcolor}%
\pgfsetfillcolor{textcolor}%
\pgftext[x=5.744380in,y=3.215000in,,top]{\color{textcolor}\sffamily\fontsize{10.000000}{12.000000}\selectfont 2}%
\end{pgfscope}%
\begin{pgfscope}%
\pgfsetbuttcap%
\pgfsetroundjoin%
\definecolor{currentfill}{rgb}{0.000000,0.000000,0.000000}%
\pgfsetfillcolor{currentfill}%
\pgfsetlinewidth{0.803000pt}%
\definecolor{currentstroke}{rgb}{0.000000,0.000000,0.000000}%
\pgfsetstrokecolor{currentstroke}%
\pgfsetdash{}{0pt}%
\pgfsys@defobject{currentmarker}{\pgfqpoint{0.000000in}{-0.048611in}}{\pgfqpoint{0.000000in}{0.000000in}}{%
\pgfpathmoveto{\pgfqpoint{0.000000in}{0.000000in}}%
\pgfpathlineto{\pgfqpoint{0.000000in}{-0.048611in}}%
\pgfusepath{stroke,fill}%
}%
\begin{pgfscope}%
\pgfsys@transformshift{6.507842in}{3.312222in}%
\pgfsys@useobject{currentmarker}{}%
\end{pgfscope}%
\end{pgfscope}%
\begin{pgfscope}%
\definecolor{textcolor}{rgb}{0.000000,0.000000,0.000000}%
\pgfsetstrokecolor{textcolor}%
\pgfsetfillcolor{textcolor}%
\pgftext[x=6.507842in,y=3.215000in,,top]{\color{textcolor}\sffamily\fontsize{10.000000}{12.000000}\selectfont 3}%
\end{pgfscope}%
\begin{pgfscope}%
\pgfsetbuttcap%
\pgfsetroundjoin%
\definecolor{currentfill}{rgb}{0.000000,0.000000,0.000000}%
\pgfsetfillcolor{currentfill}%
\pgfsetlinewidth{0.803000pt}%
\definecolor{currentstroke}{rgb}{0.000000,0.000000,0.000000}%
\pgfsetstrokecolor{currentstroke}%
\pgfsetdash{}{0pt}%
\pgfsys@defobject{currentmarker}{\pgfqpoint{0.000000in}{-0.048611in}}{\pgfqpoint{0.000000in}{0.000000in}}{%
\pgfpathmoveto{\pgfqpoint{0.000000in}{0.000000in}}%
\pgfpathlineto{\pgfqpoint{0.000000in}{-0.048611in}}%
\pgfusepath{stroke,fill}%
}%
\begin{pgfscope}%
\pgfsys@transformshift{7.271305in}{3.312222in}%
\pgfsys@useobject{currentmarker}{}%
\end{pgfscope}%
\end{pgfscope}%
\begin{pgfscope}%
\definecolor{textcolor}{rgb}{0.000000,0.000000,0.000000}%
\pgfsetstrokecolor{textcolor}%
\pgfsetfillcolor{textcolor}%
\pgftext[x=7.271305in,y=3.215000in,,top]{\color{textcolor}\sffamily\fontsize{10.000000}{12.000000}\selectfont 4}%
\end{pgfscope}%
\begin{pgfscope}%
\pgfsetbuttcap%
\pgfsetroundjoin%
\definecolor{currentfill}{rgb}{0.000000,0.000000,0.000000}%
\pgfsetfillcolor{currentfill}%
\pgfsetlinewidth{0.803000pt}%
\definecolor{currentstroke}{rgb}{0.000000,0.000000,0.000000}%
\pgfsetstrokecolor{currentstroke}%
\pgfsetdash{}{0pt}%
\pgfsys@defobject{currentmarker}{\pgfqpoint{-0.048611in}{0.000000in}}{\pgfqpoint{0.000000in}{0.000000in}}{%
\pgfpathmoveto{\pgfqpoint{0.000000in}{0.000000in}}%
\pgfpathlineto{\pgfqpoint{-0.048611in}{0.000000in}}%
\pgfusepath{stroke,fill}%
}%
\begin{pgfscope}%
\pgfsys@transformshift{4.437222in}{3.425649in}%
\pgfsys@useobject{currentmarker}{}%
\end{pgfscope}%
\end{pgfscope}%
\begin{pgfscope}%
\definecolor{textcolor}{rgb}{0.000000,0.000000,0.000000}%
\pgfsetstrokecolor{textcolor}%
\pgfsetfillcolor{textcolor}%
\pgftext[x=4.119121in,y=3.372888in,left,base]{\color{textcolor}\sffamily\fontsize{10.000000}{12.000000}\selectfont 0.0}%
\end{pgfscope}%
\begin{pgfscope}%
\pgfsetbuttcap%
\pgfsetroundjoin%
\definecolor{currentfill}{rgb}{0.000000,0.000000,0.000000}%
\pgfsetfillcolor{currentfill}%
\pgfsetlinewidth{0.803000pt}%
\definecolor{currentstroke}{rgb}{0.000000,0.000000,0.000000}%
\pgfsetstrokecolor{currentstroke}%
\pgfsetdash{}{0pt}%
\pgfsys@defobject{currentmarker}{\pgfqpoint{-0.048611in}{0.000000in}}{\pgfqpoint{0.000000in}{0.000000in}}{%
\pgfpathmoveto{\pgfqpoint{0.000000in}{0.000000in}}%
\pgfpathlineto{\pgfqpoint{-0.048611in}{0.000000in}}%
\pgfusepath{stroke,fill}%
}%
\begin{pgfscope}%
\pgfsys@transformshift{4.437222in}{3.955101in}%
\pgfsys@useobject{currentmarker}{}%
\end{pgfscope}%
\end{pgfscope}%
\begin{pgfscope}%
\definecolor{textcolor}{rgb}{0.000000,0.000000,0.000000}%
\pgfsetstrokecolor{textcolor}%
\pgfsetfillcolor{textcolor}%
\pgftext[x=4.119121in,y=3.902339in,left,base]{\color{textcolor}\sffamily\fontsize{10.000000}{12.000000}\selectfont 0.5}%
\end{pgfscope}%
\begin{pgfscope}%
\pgfsetbuttcap%
\pgfsetroundjoin%
\definecolor{currentfill}{rgb}{0.000000,0.000000,0.000000}%
\pgfsetfillcolor{currentfill}%
\pgfsetlinewidth{0.803000pt}%
\definecolor{currentstroke}{rgb}{0.000000,0.000000,0.000000}%
\pgfsetstrokecolor{currentstroke}%
\pgfsetdash{}{0pt}%
\pgfsys@defobject{currentmarker}{\pgfqpoint{-0.048611in}{0.000000in}}{\pgfqpoint{0.000000in}{0.000000in}}{%
\pgfpathmoveto{\pgfqpoint{0.000000in}{0.000000in}}%
\pgfpathlineto{\pgfqpoint{-0.048611in}{0.000000in}}%
\pgfusepath{stroke,fill}%
}%
\begin{pgfscope}%
\pgfsys@transformshift{4.437222in}{4.484552in}%
\pgfsys@useobject{currentmarker}{}%
\end{pgfscope}%
\end{pgfscope}%
\begin{pgfscope}%
\definecolor{textcolor}{rgb}{0.000000,0.000000,0.000000}%
\pgfsetstrokecolor{textcolor}%
\pgfsetfillcolor{textcolor}%
\pgftext[x=4.119121in,y=4.431791in,left,base]{\color{textcolor}\sffamily\fontsize{10.000000}{12.000000}\selectfont 1.0}%
\end{pgfscope}%
\begin{pgfscope}%
\pgfsetbuttcap%
\pgfsetroundjoin%
\definecolor{currentfill}{rgb}{0.000000,0.000000,0.000000}%
\pgfsetfillcolor{currentfill}%
\pgfsetlinewidth{0.803000pt}%
\definecolor{currentstroke}{rgb}{0.000000,0.000000,0.000000}%
\pgfsetstrokecolor{currentstroke}%
\pgfsetdash{}{0pt}%
\pgfsys@defobject{currentmarker}{\pgfqpoint{-0.048611in}{0.000000in}}{\pgfqpoint{0.000000in}{0.000000in}}{%
\pgfpathmoveto{\pgfqpoint{0.000000in}{0.000000in}}%
\pgfpathlineto{\pgfqpoint{-0.048611in}{0.000000in}}%
\pgfusepath{stroke,fill}%
}%
\begin{pgfscope}%
\pgfsys@transformshift{4.437222in}{5.014004in}%
\pgfsys@useobject{currentmarker}{}%
\end{pgfscope}%
\end{pgfscope}%
\begin{pgfscope}%
\definecolor{textcolor}{rgb}{0.000000,0.000000,0.000000}%
\pgfsetstrokecolor{textcolor}%
\pgfsetfillcolor{textcolor}%
\pgftext[x=4.119121in,y=4.961242in,left,base]{\color{textcolor}\sffamily\fontsize{10.000000}{12.000000}\selectfont 1.5}%
\end{pgfscope}%
\begin{pgfscope}%
\pgfsetbuttcap%
\pgfsetroundjoin%
\definecolor{currentfill}{rgb}{0.000000,0.000000,0.000000}%
\pgfsetfillcolor{currentfill}%
\pgfsetlinewidth{0.803000pt}%
\definecolor{currentstroke}{rgb}{0.000000,0.000000,0.000000}%
\pgfsetstrokecolor{currentstroke}%
\pgfsetdash{}{0pt}%
\pgfsys@defobject{currentmarker}{\pgfqpoint{-0.048611in}{0.000000in}}{\pgfqpoint{0.000000in}{0.000000in}}{%
\pgfpathmoveto{\pgfqpoint{0.000000in}{0.000000in}}%
\pgfpathlineto{\pgfqpoint{-0.048611in}{0.000000in}}%
\pgfusepath{stroke,fill}%
}%
\begin{pgfscope}%
\pgfsys@transformshift{4.437222in}{5.543455in}%
\pgfsys@useobject{currentmarker}{}%
\end{pgfscope}%
\end{pgfscope}%
\begin{pgfscope}%
\definecolor{textcolor}{rgb}{0.000000,0.000000,0.000000}%
\pgfsetstrokecolor{textcolor}%
\pgfsetfillcolor{textcolor}%
\pgftext[x=4.119121in,y=5.490694in,left,base]{\color{textcolor}\sffamily\fontsize{10.000000}{12.000000}\selectfont 2.0}%
\end{pgfscope}%
\begin{pgfscope}%
\pgfpathrectangle{\pgfqpoint{4.437222in}{3.312222in}}{\pgfqpoint{3.377778in}{2.324444in}}%
\pgfusepath{clip}%
\pgfsetrectcap%
\pgfsetroundjoin%
\pgfsetlinewidth{1.505625pt}%
\definecolor{currentstroke}{rgb}{0.121569,0.466667,0.705882}%
\pgfsetstrokecolor{currentstroke}%
\pgfsetdash{}{0pt}%
\pgfpathmoveto{\pgfqpoint{4.599186in}{3.425677in}}%
\pgfpathlineto{\pgfqpoint{4.675532in}{3.425958in}}%
\pgfpathlineto{\pgfqpoint{4.751879in}{3.427202in}}%
\pgfpathlineto{\pgfqpoint{4.828225in}{3.430595in}}%
\pgfpathlineto{\pgfqpoint{4.904571in}{3.437866in}}%
\pgfpathlineto{\pgfqpoint{4.980917in}{3.450471in}}%
\pgfpathlineto{\pgfqpoint{5.057264in}{3.469635in}}%
\pgfpathlineto{\pgfqpoint{5.133610in}{3.496650in}}%
\pgfpathlineto{\pgfqpoint{5.209956in}{3.532425in}}%
\pgfpathlineto{\pgfqpoint{5.286302in}{3.578461in}}%
\pgfpathlineto{\pgfqpoint{5.362649in}{3.634616in}}%
\pgfpathlineto{\pgfqpoint{5.438995in}{3.704258in}}%
\pgfpathlineto{\pgfqpoint{5.515341in}{3.788702in}}%
\pgfpathlineto{\pgfqpoint{5.591687in}{3.893333in}}%
\pgfpathlineto{\pgfqpoint{5.668034in}{4.020510in}}%
\pgfpathlineto{\pgfqpoint{5.744380in}{4.193009in}}%
\pgfpathlineto{\pgfqpoint{5.820726in}{4.453558in}}%
\pgfpathlineto{\pgfqpoint{5.897072in}{4.889690in}}%
\pgfpathlineto{\pgfqpoint{5.973419in}{5.470921in}}%
\pgfpathlineto{\pgfqpoint{6.049765in}{4.790832in}}%
\pgfpathlineto{\pgfqpoint{6.126111in}{4.370817in}}%
\pgfpathlineto{\pgfqpoint{6.202457in}{4.175834in}}%
\pgfpathlineto{\pgfqpoint{6.278804in}{4.060028in}}%
\pgfpathlineto{\pgfqpoint{6.355150in}{3.971834in}}%
\pgfpathlineto{\pgfqpoint{6.431496in}{3.904437in}}%
\pgfpathlineto{\pgfqpoint{6.507842in}{3.849296in}}%
\pgfpathlineto{\pgfqpoint{6.584189in}{3.807550in}}%
\pgfpathlineto{\pgfqpoint{6.660535in}{3.769661in}}%
\pgfpathlineto{\pgfqpoint{6.736881in}{3.738311in}}%
\pgfpathlineto{\pgfqpoint{6.813227in}{3.711464in}}%
\pgfpathlineto{\pgfqpoint{6.889574in}{3.688126in}}%
\pgfpathlineto{\pgfqpoint{6.965920in}{3.667859in}}%
\pgfpathlineto{\pgfqpoint{7.042266in}{3.650027in}}%
\pgfpathlineto{\pgfqpoint{7.118612in}{3.633799in}}%
\pgfpathlineto{\pgfqpoint{7.194959in}{3.619483in}}%
\pgfpathlineto{\pgfqpoint{7.271305in}{3.607060in}}%
\pgfpathlineto{\pgfqpoint{7.347651in}{3.595899in}}%
\pgfpathlineto{\pgfqpoint{7.423997in}{3.584837in}}%
\pgfpathlineto{\pgfqpoint{7.500344in}{3.575745in}}%
\pgfpathlineto{\pgfqpoint{7.576690in}{3.566781in}}%
\pgfpathlineto{\pgfqpoint{7.653036in}{3.559469in}}%
\pgfusepath{stroke}%
\end{pgfscope}%
\begin{pgfscope}%
\pgfsetrectcap%
\pgfsetmiterjoin%
\pgfsetlinewidth{0.803000pt}%
\definecolor{currentstroke}{rgb}{0.000000,0.000000,0.000000}%
\pgfsetstrokecolor{currentstroke}%
\pgfsetdash{}{0pt}%
\pgfpathmoveto{\pgfqpoint{4.437222in}{3.312222in}}%
\pgfpathlineto{\pgfqpoint{4.437222in}{5.636667in}}%
\pgfusepath{stroke}%
\end{pgfscope}%
\begin{pgfscope}%
\pgfsetrectcap%
\pgfsetmiterjoin%
\pgfsetlinewidth{0.803000pt}%
\definecolor{currentstroke}{rgb}{0.000000,0.000000,0.000000}%
\pgfsetstrokecolor{currentstroke}%
\pgfsetdash{}{0pt}%
\pgfpathmoveto{\pgfqpoint{7.815000in}{3.312222in}}%
\pgfpathlineto{\pgfqpoint{7.815000in}{5.636667in}}%
\pgfusepath{stroke}%
\end{pgfscope}%
\begin{pgfscope}%
\pgfsetrectcap%
\pgfsetmiterjoin%
\pgfsetlinewidth{0.803000pt}%
\definecolor{currentstroke}{rgb}{0.000000,0.000000,0.000000}%
\pgfsetstrokecolor{currentstroke}%
\pgfsetdash{}{0pt}%
\pgfpathmoveto{\pgfqpoint{4.437222in}{3.312222in}}%
\pgfpathlineto{\pgfqpoint{7.815000in}{3.312222in}}%
\pgfusepath{stroke}%
\end{pgfscope}%
\begin{pgfscope}%
\pgfsetrectcap%
\pgfsetmiterjoin%
\pgfsetlinewidth{0.803000pt}%
\definecolor{currentstroke}{rgb}{0.000000,0.000000,0.000000}%
\pgfsetstrokecolor{currentstroke}%
\pgfsetdash{}{0pt}%
\pgfpathmoveto{\pgfqpoint{4.437222in}{5.636667in}}%
\pgfpathlineto{\pgfqpoint{7.815000in}{5.636667in}}%
\pgfusepath{stroke}%
\end{pgfscope}%
\begin{pgfscope}%
\definecolor{textcolor}{rgb}{0.000000,0.000000,0.000000}%
\pgfsetstrokecolor{textcolor}%
\pgfsetfillcolor{textcolor}%
\pgftext[x=6.126111in,y=5.720000in,,base]{\color{textcolor}\sffamily\fontsize{12.000000}{14.400000}\selectfont \(\displaystyle  N = 32 \)}%
\end{pgfscope}%
\begin{pgfscope}%
\pgfsetbuttcap%
\pgfsetmiterjoin%
\definecolor{currentfill}{rgb}{1.000000,1.000000,1.000000}%
\pgfsetfillcolor{currentfill}%
\pgfsetlinewidth{0.000000pt}%
\definecolor{currentstroke}{rgb}{0.000000,0.000000,0.000000}%
\pgfsetstrokecolor{currentstroke}%
\pgfsetstrokeopacity{0.000000}%
\pgfsetdash{}{0pt}%
\pgfpathmoveto{\pgfqpoint{0.555972in}{0.387222in}}%
\pgfpathlineto{\pgfqpoint{3.933750in}{0.387222in}}%
\pgfpathlineto{\pgfqpoint{3.933750in}{2.711667in}}%
\pgfpathlineto{\pgfqpoint{0.555972in}{2.711667in}}%
\pgfpathclose%
\pgfusepath{fill}%
\end{pgfscope}%
\begin{pgfscope}%
\pgfpathrectangle{\pgfqpoint{0.555972in}{0.387222in}}{\pgfqpoint{3.377778in}{2.324444in}}%
\pgfusepath{clip}%
\pgfsetbuttcap%
\pgfsetroundjoin%
\definecolor{currentfill}{rgb}{0.121569,0.466667,0.705882}%
\pgfsetfillcolor{currentfill}%
\pgfsetfillopacity{0.300000}%
\pgfsetlinewidth{1.003750pt}%
\definecolor{currentstroke}{rgb}{0.121569,0.466667,0.705882}%
\pgfsetstrokecolor{currentstroke}%
\pgfsetstrokeopacity{0.300000}%
\pgfsetdash{}{0pt}%
\pgfpathmoveto{\pgfqpoint{0.717936in}{1.145294in}}%
\pgfpathlineto{\pgfqpoint{0.717936in}{1.141600in}}%
\pgfpathlineto{\pgfqpoint{0.794282in}{0.492879in}}%
\pgfpathlineto{\pgfqpoint{0.870629in}{1.141405in}}%
\pgfpathlineto{\pgfqpoint{0.946975in}{1.141717in}}%
\pgfpathlineto{\pgfqpoint{1.023321in}{0.819603in}}%
\pgfpathlineto{\pgfqpoint{1.099667in}{0.865642in}}%
\pgfpathlineto{\pgfqpoint{1.176014in}{1.148225in}}%
\pgfpathlineto{\pgfqpoint{1.252360in}{1.151419in}}%
\pgfpathlineto{\pgfqpoint{1.328706in}{1.155894in}}%
\pgfpathlineto{\pgfqpoint{1.405052in}{1.161531in}}%
\pgfpathlineto{\pgfqpoint{1.481399in}{1.168476in}}%
\pgfpathlineto{\pgfqpoint{1.557745in}{1.176884in}}%
\pgfpathlineto{\pgfqpoint{1.634091in}{1.187201in}}%
\pgfpathlineto{\pgfqpoint{1.710437in}{1.200326in}}%
\pgfpathlineto{\pgfqpoint{1.786784in}{1.215600in}}%
\pgfpathlineto{\pgfqpoint{1.863130in}{1.237329in}}%
\pgfpathlineto{\pgfqpoint{1.939476in}{1.265208in}}%
\pgfpathlineto{\pgfqpoint{2.015822in}{1.311445in}}%
\pgfpathlineto{\pgfqpoint{2.092169in}{1.391548in}}%
\pgfpathlineto{\pgfqpoint{2.168515in}{1.288717in}}%
\pgfpathlineto{\pgfqpoint{2.244861in}{1.254730in}}%
\pgfpathlineto{\pgfqpoint{2.321207in}{1.233285in}}%
\pgfpathlineto{\pgfqpoint{2.397554in}{1.219013in}}%
\pgfpathlineto{\pgfqpoint{2.473900in}{1.209803in}}%
\pgfpathlineto{\pgfqpoint{2.550246in}{1.201662in}}%
\pgfpathlineto{\pgfqpoint{2.626592in}{1.195107in}}%
\pgfpathlineto{\pgfqpoint{2.702939in}{1.189269in}}%
\pgfpathlineto{\pgfqpoint{2.779285in}{1.185144in}}%
\pgfpathlineto{\pgfqpoint{2.855631in}{1.180665in}}%
\pgfpathlineto{\pgfqpoint{2.931977in}{1.177779in}}%
\pgfpathlineto{\pgfqpoint{3.008324in}{1.174669in}}%
\pgfpathlineto{\pgfqpoint{3.084670in}{1.171984in}}%
\pgfpathlineto{\pgfqpoint{3.161016in}{1.170340in}}%
\pgfpathlineto{\pgfqpoint{3.237362in}{1.168202in}}%
\pgfpathlineto{\pgfqpoint{3.313709in}{1.166554in}}%
\pgfpathlineto{\pgfqpoint{3.390055in}{1.165048in}}%
\pgfpathlineto{\pgfqpoint{3.466401in}{1.163504in}}%
\pgfpathlineto{\pgfqpoint{3.542747in}{1.162299in}}%
\pgfpathlineto{\pgfqpoint{3.619094in}{1.161198in}}%
\pgfpathlineto{\pgfqpoint{3.695440in}{1.160168in}}%
\pgfpathlineto{\pgfqpoint{3.771786in}{1.159169in}}%
\pgfpathlineto{\pgfqpoint{3.771786in}{1.159443in}}%
\pgfpathlineto{\pgfqpoint{3.771786in}{1.159443in}}%
\pgfpathlineto{\pgfqpoint{3.695440in}{1.160462in}}%
\pgfpathlineto{\pgfqpoint{3.619094in}{1.161384in}}%
\pgfpathlineto{\pgfqpoint{3.542747in}{1.162564in}}%
\pgfpathlineto{\pgfqpoint{3.466401in}{1.164011in}}%
\pgfpathlineto{\pgfqpoint{3.390055in}{1.165329in}}%
\pgfpathlineto{\pgfqpoint{3.313709in}{1.167116in}}%
\pgfpathlineto{\pgfqpoint{3.237362in}{1.168717in}}%
\pgfpathlineto{\pgfqpoint{3.161016in}{1.170707in}}%
\pgfpathlineto{\pgfqpoint{3.084670in}{1.173042in}}%
\pgfpathlineto{\pgfqpoint{3.008324in}{1.175220in}}%
\pgfpathlineto{\pgfqpoint{2.931977in}{1.178416in}}%
\pgfpathlineto{\pgfqpoint{2.855631in}{1.181670in}}%
\pgfpathlineto{\pgfqpoint{2.779285in}{1.185718in}}%
\pgfpathlineto{\pgfqpoint{2.702939in}{1.190298in}}%
\pgfpathlineto{\pgfqpoint{2.626592in}{1.195909in}}%
\pgfpathlineto{\pgfqpoint{2.550246in}{1.201908in}}%
\pgfpathlineto{\pgfqpoint{2.473900in}{1.212226in}}%
\pgfpathlineto{\pgfqpoint{2.397554in}{1.222583in}}%
\pgfpathlineto{\pgfqpoint{2.321207in}{1.236647in}}%
\pgfpathlineto{\pgfqpoint{2.244861in}{1.256695in}}%
\pgfpathlineto{\pgfqpoint{2.168515in}{1.297697in}}%
\pgfpathlineto{\pgfqpoint{2.092169in}{1.438777in}}%
\pgfpathlineto{\pgfqpoint{2.015822in}{1.326155in}}%
\pgfpathlineto{\pgfqpoint{1.939476in}{1.272777in}}%
\pgfpathlineto{\pgfqpoint{1.863130in}{1.239229in}}%
\pgfpathlineto{\pgfqpoint{1.786784in}{1.217732in}}%
\pgfpathlineto{\pgfqpoint{1.710437in}{1.200625in}}%
\pgfpathlineto{\pgfqpoint{1.634091in}{1.188188in}}%
\pgfpathlineto{\pgfqpoint{1.557745in}{1.177290in}}%
\pgfpathlineto{\pgfqpoint{1.481399in}{1.169121in}}%
\pgfpathlineto{\pgfqpoint{1.405052in}{1.161776in}}%
\pgfpathlineto{\pgfqpoint{1.328706in}{1.156064in}}%
\pgfpathlineto{\pgfqpoint{1.252360in}{1.151705in}}%
\pgfpathlineto{\pgfqpoint{1.176014in}{1.148289in}}%
\pgfpathlineto{\pgfqpoint{1.099667in}{1.776578in}}%
\pgfpathlineto{\pgfqpoint{1.023321in}{1.884856in}}%
\pgfpathlineto{\pgfqpoint{0.946975in}{1.147025in}}%
\pgfpathlineto{\pgfqpoint{0.870629in}{1.146353in}}%
\pgfpathlineto{\pgfqpoint{0.794282in}{2.606010in}}%
\pgfpathlineto{\pgfqpoint{0.717936in}{1.145294in}}%
\pgfpathclose%
\pgfusepath{stroke,fill}%
\end{pgfscope}%
\begin{pgfscope}%
\pgfpathrectangle{\pgfqpoint{0.555972in}{0.387222in}}{\pgfqpoint{3.377778in}{2.324444in}}%
\pgfusepath{clip}%
\pgfsetbuttcap%
\pgfsetroundjoin%
\definecolor{currentfill}{rgb}{0.121569,0.466667,0.705882}%
\pgfsetfillcolor{currentfill}%
\pgfsetlinewidth{1.003750pt}%
\definecolor{currentstroke}{rgb}{0.121569,0.466667,0.705882}%
\pgfsetstrokecolor{currentstroke}%
\pgfsetdash{}{0pt}%
\pgfsys@defobject{currentmarker}{\pgfqpoint{-0.009821in}{-0.009821in}}{\pgfqpoint{0.009821in}{0.009821in}}{%
\pgfpathmoveto{\pgfqpoint{0.000000in}{-0.009821in}}%
\pgfpathcurveto{\pgfqpoint{0.002605in}{-0.009821in}}{\pgfqpoint{0.005103in}{-0.008786in}}{\pgfqpoint{0.006944in}{-0.006944in}}%
\pgfpathcurveto{\pgfqpoint{0.008786in}{-0.005103in}}{\pgfqpoint{0.009821in}{-0.002605in}}{\pgfqpoint{0.009821in}{0.000000in}}%
\pgfpathcurveto{\pgfqpoint{0.009821in}{0.002605in}}{\pgfqpoint{0.008786in}{0.005103in}}{\pgfqpoint{0.006944in}{0.006944in}}%
\pgfpathcurveto{\pgfqpoint{0.005103in}{0.008786in}}{\pgfqpoint{0.002605in}{0.009821in}}{\pgfqpoint{0.000000in}{0.009821in}}%
\pgfpathcurveto{\pgfqpoint{-0.002605in}{0.009821in}}{\pgfqpoint{-0.005103in}{0.008786in}}{\pgfqpoint{-0.006944in}{0.006944in}}%
\pgfpathcurveto{\pgfqpoint{-0.008786in}{0.005103in}}{\pgfqpoint{-0.009821in}{0.002605in}}{\pgfqpoint{-0.009821in}{0.000000in}}%
\pgfpathcurveto{\pgfqpoint{-0.009821in}{-0.002605in}}{\pgfqpoint{-0.008786in}{-0.005103in}}{\pgfqpoint{-0.006944in}{-0.006944in}}%
\pgfpathcurveto{\pgfqpoint{-0.005103in}{-0.008786in}}{\pgfqpoint{-0.002605in}{-0.009821in}}{\pgfqpoint{0.000000in}{-0.009821in}}%
\pgfpathclose%
\pgfusepath{stroke,fill}%
}%
\begin{pgfscope}%
\pgfsys@transformshift{0.717936in}{1.143447in}%
\pgfsys@useobject{currentmarker}{}%
\end{pgfscope}%
\begin{pgfscope}%
\pgfsys@transformshift{0.794282in}{1.549444in}%
\pgfsys@useobject{currentmarker}{}%
\end{pgfscope}%
\begin{pgfscope}%
\pgfsys@transformshift{0.870629in}{1.143879in}%
\pgfsys@useobject{currentmarker}{}%
\end{pgfscope}%
\begin{pgfscope}%
\pgfsys@transformshift{0.946975in}{1.144371in}%
\pgfsys@useobject{currentmarker}{}%
\end{pgfscope}%
\begin{pgfscope}%
\pgfsys@transformshift{1.023321in}{1.352230in}%
\pgfsys@useobject{currentmarker}{}%
\end{pgfscope}%
\begin{pgfscope}%
\pgfsys@transformshift{1.099667in}{1.321110in}%
\pgfsys@useobject{currentmarker}{}%
\end{pgfscope}%
\begin{pgfscope}%
\pgfsys@transformshift{1.176014in}{1.148257in}%
\pgfsys@useobject{currentmarker}{}%
\end{pgfscope}%
\begin{pgfscope}%
\pgfsys@transformshift{1.252360in}{1.151562in}%
\pgfsys@useobject{currentmarker}{}%
\end{pgfscope}%
\begin{pgfscope}%
\pgfsys@transformshift{1.328706in}{1.155979in}%
\pgfsys@useobject{currentmarker}{}%
\end{pgfscope}%
\begin{pgfscope}%
\pgfsys@transformshift{1.405052in}{1.161653in}%
\pgfsys@useobject{currentmarker}{}%
\end{pgfscope}%
\begin{pgfscope}%
\pgfsys@transformshift{1.481399in}{1.168799in}%
\pgfsys@useobject{currentmarker}{}%
\end{pgfscope}%
\begin{pgfscope}%
\pgfsys@transformshift{1.557745in}{1.177087in}%
\pgfsys@useobject{currentmarker}{}%
\end{pgfscope}%
\begin{pgfscope}%
\pgfsys@transformshift{1.634091in}{1.187694in}%
\pgfsys@useobject{currentmarker}{}%
\end{pgfscope}%
\begin{pgfscope}%
\pgfsys@transformshift{1.710437in}{1.200475in}%
\pgfsys@useobject{currentmarker}{}%
\end{pgfscope}%
\begin{pgfscope}%
\pgfsys@transformshift{1.786784in}{1.216666in}%
\pgfsys@useobject{currentmarker}{}%
\end{pgfscope}%
\begin{pgfscope}%
\pgfsys@transformshift{1.863130in}{1.238279in}%
\pgfsys@useobject{currentmarker}{}%
\end{pgfscope}%
\begin{pgfscope}%
\pgfsys@transformshift{1.939476in}{1.268992in}%
\pgfsys@useobject{currentmarker}{}%
\end{pgfscope}%
\begin{pgfscope}%
\pgfsys@transformshift{2.015822in}{1.318800in}%
\pgfsys@useobject{currentmarker}{}%
\end{pgfscope}%
\begin{pgfscope}%
\pgfsys@transformshift{2.092169in}{1.415163in}%
\pgfsys@useobject{currentmarker}{}%
\end{pgfscope}%
\begin{pgfscope}%
\pgfsys@transformshift{2.168515in}{1.293207in}%
\pgfsys@useobject{currentmarker}{}%
\end{pgfscope}%
\begin{pgfscope}%
\pgfsys@transformshift{2.244861in}{1.255712in}%
\pgfsys@useobject{currentmarker}{}%
\end{pgfscope}%
\begin{pgfscope}%
\pgfsys@transformshift{2.321207in}{1.234966in}%
\pgfsys@useobject{currentmarker}{}%
\end{pgfscope}%
\begin{pgfscope}%
\pgfsys@transformshift{2.397554in}{1.220798in}%
\pgfsys@useobject{currentmarker}{}%
\end{pgfscope}%
\begin{pgfscope}%
\pgfsys@transformshift{2.473900in}{1.211015in}%
\pgfsys@useobject{currentmarker}{}%
\end{pgfscope}%
\begin{pgfscope}%
\pgfsys@transformshift{2.550246in}{1.201785in}%
\pgfsys@useobject{currentmarker}{}%
\end{pgfscope}%
\begin{pgfscope}%
\pgfsys@transformshift{2.626592in}{1.195508in}%
\pgfsys@useobject{currentmarker}{}%
\end{pgfscope}%
\begin{pgfscope}%
\pgfsys@transformshift{2.702939in}{1.189784in}%
\pgfsys@useobject{currentmarker}{}%
\end{pgfscope}%
\begin{pgfscope}%
\pgfsys@transformshift{2.779285in}{1.185431in}%
\pgfsys@useobject{currentmarker}{}%
\end{pgfscope}%
\begin{pgfscope}%
\pgfsys@transformshift{2.855631in}{1.181168in}%
\pgfsys@useobject{currentmarker}{}%
\end{pgfscope}%
\begin{pgfscope}%
\pgfsys@transformshift{2.931977in}{1.178098in}%
\pgfsys@useobject{currentmarker}{}%
\end{pgfscope}%
\begin{pgfscope}%
\pgfsys@transformshift{3.008324in}{1.174945in}%
\pgfsys@useobject{currentmarker}{}%
\end{pgfscope}%
\begin{pgfscope}%
\pgfsys@transformshift{3.084670in}{1.172513in}%
\pgfsys@useobject{currentmarker}{}%
\end{pgfscope}%
\begin{pgfscope}%
\pgfsys@transformshift{3.161016in}{1.170524in}%
\pgfsys@useobject{currentmarker}{}%
\end{pgfscope}%
\begin{pgfscope}%
\pgfsys@transformshift{3.237362in}{1.168459in}%
\pgfsys@useobject{currentmarker}{}%
\end{pgfscope}%
\begin{pgfscope}%
\pgfsys@transformshift{3.313709in}{1.166835in}%
\pgfsys@useobject{currentmarker}{}%
\end{pgfscope}%
\begin{pgfscope}%
\pgfsys@transformshift{3.390055in}{1.165188in}%
\pgfsys@useobject{currentmarker}{}%
\end{pgfscope}%
\begin{pgfscope}%
\pgfsys@transformshift{3.466401in}{1.163757in}%
\pgfsys@useobject{currentmarker}{}%
\end{pgfscope}%
\begin{pgfscope}%
\pgfsys@transformshift{3.542747in}{1.162432in}%
\pgfsys@useobject{currentmarker}{}%
\end{pgfscope}%
\begin{pgfscope}%
\pgfsys@transformshift{3.619094in}{1.161291in}%
\pgfsys@useobject{currentmarker}{}%
\end{pgfscope}%
\begin{pgfscope}%
\pgfsys@transformshift{3.695440in}{1.160315in}%
\pgfsys@useobject{currentmarker}{}%
\end{pgfscope}%
\begin{pgfscope}%
\pgfsys@transformshift{3.771786in}{1.159306in}%
\pgfsys@useobject{currentmarker}{}%
\end{pgfscope}%
\end{pgfscope}%
\begin{pgfscope}%
\pgfsetbuttcap%
\pgfsetroundjoin%
\definecolor{currentfill}{rgb}{0.000000,0.000000,0.000000}%
\pgfsetfillcolor{currentfill}%
\pgfsetlinewidth{0.803000pt}%
\definecolor{currentstroke}{rgb}{0.000000,0.000000,0.000000}%
\pgfsetstrokecolor{currentstroke}%
\pgfsetdash{}{0pt}%
\pgfsys@defobject{currentmarker}{\pgfqpoint{0.000000in}{-0.048611in}}{\pgfqpoint{0.000000in}{0.000000in}}{%
\pgfpathmoveto{\pgfqpoint{0.000000in}{0.000000in}}%
\pgfpathlineto{\pgfqpoint{0.000000in}{-0.048611in}}%
\pgfusepath{stroke,fill}%
}%
\begin{pgfscope}%
\pgfsys@transformshift{1.099667in}{0.387222in}%
\pgfsys@useobject{currentmarker}{}%
\end{pgfscope}%
\end{pgfscope}%
\begin{pgfscope}%
\definecolor{textcolor}{rgb}{0.000000,0.000000,0.000000}%
\pgfsetstrokecolor{textcolor}%
\pgfsetfillcolor{textcolor}%
\pgftext[x=1.099667in,y=0.290000in,,top]{\color{textcolor}\sffamily\fontsize{10.000000}{12.000000}\selectfont 1}%
\end{pgfscope}%
\begin{pgfscope}%
\pgfsetbuttcap%
\pgfsetroundjoin%
\definecolor{currentfill}{rgb}{0.000000,0.000000,0.000000}%
\pgfsetfillcolor{currentfill}%
\pgfsetlinewidth{0.803000pt}%
\definecolor{currentstroke}{rgb}{0.000000,0.000000,0.000000}%
\pgfsetstrokecolor{currentstroke}%
\pgfsetdash{}{0pt}%
\pgfsys@defobject{currentmarker}{\pgfqpoint{0.000000in}{-0.048611in}}{\pgfqpoint{0.000000in}{0.000000in}}{%
\pgfpathmoveto{\pgfqpoint{0.000000in}{0.000000in}}%
\pgfpathlineto{\pgfqpoint{0.000000in}{-0.048611in}}%
\pgfusepath{stroke,fill}%
}%
\begin{pgfscope}%
\pgfsys@transformshift{1.863130in}{0.387222in}%
\pgfsys@useobject{currentmarker}{}%
\end{pgfscope}%
\end{pgfscope}%
\begin{pgfscope}%
\definecolor{textcolor}{rgb}{0.000000,0.000000,0.000000}%
\pgfsetstrokecolor{textcolor}%
\pgfsetfillcolor{textcolor}%
\pgftext[x=1.863130in,y=0.290000in,,top]{\color{textcolor}\sffamily\fontsize{10.000000}{12.000000}\selectfont 2}%
\end{pgfscope}%
\begin{pgfscope}%
\pgfsetbuttcap%
\pgfsetroundjoin%
\definecolor{currentfill}{rgb}{0.000000,0.000000,0.000000}%
\pgfsetfillcolor{currentfill}%
\pgfsetlinewidth{0.803000pt}%
\definecolor{currentstroke}{rgb}{0.000000,0.000000,0.000000}%
\pgfsetstrokecolor{currentstroke}%
\pgfsetdash{}{0pt}%
\pgfsys@defobject{currentmarker}{\pgfqpoint{0.000000in}{-0.048611in}}{\pgfqpoint{0.000000in}{0.000000in}}{%
\pgfpathmoveto{\pgfqpoint{0.000000in}{0.000000in}}%
\pgfpathlineto{\pgfqpoint{0.000000in}{-0.048611in}}%
\pgfusepath{stroke,fill}%
}%
\begin{pgfscope}%
\pgfsys@transformshift{2.626592in}{0.387222in}%
\pgfsys@useobject{currentmarker}{}%
\end{pgfscope}%
\end{pgfscope}%
\begin{pgfscope}%
\definecolor{textcolor}{rgb}{0.000000,0.000000,0.000000}%
\pgfsetstrokecolor{textcolor}%
\pgfsetfillcolor{textcolor}%
\pgftext[x=2.626592in,y=0.290000in,,top]{\color{textcolor}\sffamily\fontsize{10.000000}{12.000000}\selectfont 3}%
\end{pgfscope}%
\begin{pgfscope}%
\pgfsetbuttcap%
\pgfsetroundjoin%
\definecolor{currentfill}{rgb}{0.000000,0.000000,0.000000}%
\pgfsetfillcolor{currentfill}%
\pgfsetlinewidth{0.803000pt}%
\definecolor{currentstroke}{rgb}{0.000000,0.000000,0.000000}%
\pgfsetstrokecolor{currentstroke}%
\pgfsetdash{}{0pt}%
\pgfsys@defobject{currentmarker}{\pgfqpoint{0.000000in}{-0.048611in}}{\pgfqpoint{0.000000in}{0.000000in}}{%
\pgfpathmoveto{\pgfqpoint{0.000000in}{0.000000in}}%
\pgfpathlineto{\pgfqpoint{0.000000in}{-0.048611in}}%
\pgfusepath{stroke,fill}%
}%
\begin{pgfscope}%
\pgfsys@transformshift{3.390055in}{0.387222in}%
\pgfsys@useobject{currentmarker}{}%
\end{pgfscope}%
\end{pgfscope}%
\begin{pgfscope}%
\definecolor{textcolor}{rgb}{0.000000,0.000000,0.000000}%
\pgfsetstrokecolor{textcolor}%
\pgfsetfillcolor{textcolor}%
\pgftext[x=3.390055in,y=0.290000in,,top]{\color{textcolor}\sffamily\fontsize{10.000000}{12.000000}\selectfont 4}%
\end{pgfscope}%
\begin{pgfscope}%
\pgfsetbuttcap%
\pgfsetroundjoin%
\definecolor{currentfill}{rgb}{0.000000,0.000000,0.000000}%
\pgfsetfillcolor{currentfill}%
\pgfsetlinewidth{0.803000pt}%
\definecolor{currentstroke}{rgb}{0.000000,0.000000,0.000000}%
\pgfsetstrokecolor{currentstroke}%
\pgfsetdash{}{0pt}%
\pgfsys@defobject{currentmarker}{\pgfqpoint{-0.048611in}{0.000000in}}{\pgfqpoint{0.000000in}{0.000000in}}{%
\pgfpathmoveto{\pgfqpoint{0.000000in}{0.000000in}}%
\pgfpathlineto{\pgfqpoint{-0.048611in}{0.000000in}}%
\pgfusepath{stroke,fill}%
}%
\begin{pgfscope}%
\pgfsys@transformshift{0.555972in}{0.487478in}%
\pgfsys@useobject{currentmarker}{}%
\end{pgfscope}%
\end{pgfscope}%
\begin{pgfscope}%
\definecolor{textcolor}{rgb}{0.000000,0.000000,0.000000}%
\pgfsetstrokecolor{textcolor}%
\pgfsetfillcolor{textcolor}%
\pgftext[x=0.121497in,y=0.434717in,left,base]{\color{textcolor}\sffamily\fontsize{10.000000}{12.000000}\selectfont −5.0}%
\end{pgfscope}%
\begin{pgfscope}%
\pgfsetbuttcap%
\pgfsetroundjoin%
\definecolor{currentfill}{rgb}{0.000000,0.000000,0.000000}%
\pgfsetfillcolor{currentfill}%
\pgfsetlinewidth{0.803000pt}%
\definecolor{currentstroke}{rgb}{0.000000,0.000000,0.000000}%
\pgfsetstrokecolor{currentstroke}%
\pgfsetdash{}{0pt}%
\pgfsys@defobject{currentmarker}{\pgfqpoint{-0.048611in}{0.000000in}}{\pgfqpoint{0.000000in}{0.000000in}}{%
\pgfpathmoveto{\pgfqpoint{0.000000in}{0.000000in}}%
\pgfpathlineto{\pgfqpoint{-0.048611in}{0.000000in}}%
\pgfusepath{stroke,fill}%
}%
\begin{pgfscope}%
\pgfsys@transformshift{0.555972in}{0.815105in}%
\pgfsys@useobject{currentmarker}{}%
\end{pgfscope}%
\end{pgfscope}%
\begin{pgfscope}%
\definecolor{textcolor}{rgb}{0.000000,0.000000,0.000000}%
\pgfsetstrokecolor{textcolor}%
\pgfsetfillcolor{textcolor}%
\pgftext[x=0.121497in,y=0.762344in,left,base]{\color{textcolor}\sffamily\fontsize{10.000000}{12.000000}\selectfont −2.5}%
\end{pgfscope}%
\begin{pgfscope}%
\pgfsetbuttcap%
\pgfsetroundjoin%
\definecolor{currentfill}{rgb}{0.000000,0.000000,0.000000}%
\pgfsetfillcolor{currentfill}%
\pgfsetlinewidth{0.803000pt}%
\definecolor{currentstroke}{rgb}{0.000000,0.000000,0.000000}%
\pgfsetstrokecolor{currentstroke}%
\pgfsetdash{}{0pt}%
\pgfsys@defobject{currentmarker}{\pgfqpoint{-0.048611in}{0.000000in}}{\pgfqpoint{0.000000in}{0.000000in}}{%
\pgfpathmoveto{\pgfqpoint{0.000000in}{0.000000in}}%
\pgfpathlineto{\pgfqpoint{-0.048611in}{0.000000in}}%
\pgfusepath{stroke,fill}%
}%
\begin{pgfscope}%
\pgfsys@transformshift{0.555972in}{1.142732in}%
\pgfsys@useobject{currentmarker}{}%
\end{pgfscope}%
\end{pgfscope}%
\begin{pgfscope}%
\definecolor{textcolor}{rgb}{0.000000,0.000000,0.000000}%
\pgfsetstrokecolor{textcolor}%
\pgfsetfillcolor{textcolor}%
\pgftext[x=0.237871in,y=1.089971in,left,base]{\color{textcolor}\sffamily\fontsize{10.000000}{12.000000}\selectfont 0.0}%
\end{pgfscope}%
\begin{pgfscope}%
\pgfsetbuttcap%
\pgfsetroundjoin%
\definecolor{currentfill}{rgb}{0.000000,0.000000,0.000000}%
\pgfsetfillcolor{currentfill}%
\pgfsetlinewidth{0.803000pt}%
\definecolor{currentstroke}{rgb}{0.000000,0.000000,0.000000}%
\pgfsetstrokecolor{currentstroke}%
\pgfsetdash{}{0pt}%
\pgfsys@defobject{currentmarker}{\pgfqpoint{-0.048611in}{0.000000in}}{\pgfqpoint{0.000000in}{0.000000in}}{%
\pgfpathmoveto{\pgfqpoint{0.000000in}{0.000000in}}%
\pgfpathlineto{\pgfqpoint{-0.048611in}{0.000000in}}%
\pgfusepath{stroke,fill}%
}%
\begin{pgfscope}%
\pgfsys@transformshift{0.555972in}{1.470359in}%
\pgfsys@useobject{currentmarker}{}%
\end{pgfscope}%
\end{pgfscope}%
\begin{pgfscope}%
\definecolor{textcolor}{rgb}{0.000000,0.000000,0.000000}%
\pgfsetstrokecolor{textcolor}%
\pgfsetfillcolor{textcolor}%
\pgftext[x=0.237871in,y=1.417598in,left,base]{\color{textcolor}\sffamily\fontsize{10.000000}{12.000000}\selectfont 2.5}%
\end{pgfscope}%
\begin{pgfscope}%
\pgfsetbuttcap%
\pgfsetroundjoin%
\definecolor{currentfill}{rgb}{0.000000,0.000000,0.000000}%
\pgfsetfillcolor{currentfill}%
\pgfsetlinewidth{0.803000pt}%
\definecolor{currentstroke}{rgb}{0.000000,0.000000,0.000000}%
\pgfsetstrokecolor{currentstroke}%
\pgfsetdash{}{0pt}%
\pgfsys@defobject{currentmarker}{\pgfqpoint{-0.048611in}{0.000000in}}{\pgfqpoint{0.000000in}{0.000000in}}{%
\pgfpathmoveto{\pgfqpoint{0.000000in}{0.000000in}}%
\pgfpathlineto{\pgfqpoint{-0.048611in}{0.000000in}}%
\pgfusepath{stroke,fill}%
}%
\begin{pgfscope}%
\pgfsys@transformshift{0.555972in}{1.797986in}%
\pgfsys@useobject{currentmarker}{}%
\end{pgfscope}%
\end{pgfscope}%
\begin{pgfscope}%
\definecolor{textcolor}{rgb}{0.000000,0.000000,0.000000}%
\pgfsetstrokecolor{textcolor}%
\pgfsetfillcolor{textcolor}%
\pgftext[x=0.237871in,y=1.745225in,left,base]{\color{textcolor}\sffamily\fontsize{10.000000}{12.000000}\selectfont 5.0}%
\end{pgfscope}%
\begin{pgfscope}%
\pgfsetbuttcap%
\pgfsetroundjoin%
\definecolor{currentfill}{rgb}{0.000000,0.000000,0.000000}%
\pgfsetfillcolor{currentfill}%
\pgfsetlinewidth{0.803000pt}%
\definecolor{currentstroke}{rgb}{0.000000,0.000000,0.000000}%
\pgfsetstrokecolor{currentstroke}%
\pgfsetdash{}{0pt}%
\pgfsys@defobject{currentmarker}{\pgfqpoint{-0.048611in}{0.000000in}}{\pgfqpoint{0.000000in}{0.000000in}}{%
\pgfpathmoveto{\pgfqpoint{0.000000in}{0.000000in}}%
\pgfpathlineto{\pgfqpoint{-0.048611in}{0.000000in}}%
\pgfusepath{stroke,fill}%
}%
\begin{pgfscope}%
\pgfsys@transformshift{0.555972in}{2.125614in}%
\pgfsys@useobject{currentmarker}{}%
\end{pgfscope}%
\end{pgfscope}%
\begin{pgfscope}%
\definecolor{textcolor}{rgb}{0.000000,0.000000,0.000000}%
\pgfsetstrokecolor{textcolor}%
\pgfsetfillcolor{textcolor}%
\pgftext[x=0.237871in,y=2.072852in,left,base]{\color{textcolor}\sffamily\fontsize{10.000000}{12.000000}\selectfont 7.5}%
\end{pgfscope}%
\begin{pgfscope}%
\pgfsetbuttcap%
\pgfsetroundjoin%
\definecolor{currentfill}{rgb}{0.000000,0.000000,0.000000}%
\pgfsetfillcolor{currentfill}%
\pgfsetlinewidth{0.803000pt}%
\definecolor{currentstroke}{rgb}{0.000000,0.000000,0.000000}%
\pgfsetstrokecolor{currentstroke}%
\pgfsetdash{}{0pt}%
\pgfsys@defobject{currentmarker}{\pgfqpoint{-0.048611in}{0.000000in}}{\pgfqpoint{0.000000in}{0.000000in}}{%
\pgfpathmoveto{\pgfqpoint{0.000000in}{0.000000in}}%
\pgfpathlineto{\pgfqpoint{-0.048611in}{0.000000in}}%
\pgfusepath{stroke,fill}%
}%
\begin{pgfscope}%
\pgfsys@transformshift{0.555972in}{2.453241in}%
\pgfsys@useobject{currentmarker}{}%
\end{pgfscope}%
\end{pgfscope}%
\begin{pgfscope}%
\definecolor{textcolor}{rgb}{0.000000,0.000000,0.000000}%
\pgfsetstrokecolor{textcolor}%
\pgfsetfillcolor{textcolor}%
\pgftext[x=0.149505in,y=2.400479in,left,base]{\color{textcolor}\sffamily\fontsize{10.000000}{12.000000}\selectfont 10.0}%
\end{pgfscope}%
\begin{pgfscope}%
\pgfpathrectangle{\pgfqpoint{0.555972in}{0.387222in}}{\pgfqpoint{3.377778in}{2.324444in}}%
\pgfusepath{clip}%
\pgfsetrectcap%
\pgfsetroundjoin%
\pgfsetlinewidth{1.505625pt}%
\definecolor{currentstroke}{rgb}{0.121569,0.466667,0.705882}%
\pgfsetstrokecolor{currentstroke}%
\pgfsetdash{}{0pt}%
\pgfpathmoveto{\pgfqpoint{0.717936in}{1.143447in}}%
\pgfpathlineto{\pgfqpoint{0.794282in}{1.549444in}}%
\pgfpathlineto{\pgfqpoint{0.870629in}{1.143879in}}%
\pgfpathlineto{\pgfqpoint{0.946975in}{1.144371in}}%
\pgfpathlineto{\pgfqpoint{1.023321in}{1.352230in}}%
\pgfpathlineto{\pgfqpoint{1.099667in}{1.321110in}}%
\pgfpathlineto{\pgfqpoint{1.176014in}{1.148257in}}%
\pgfpathlineto{\pgfqpoint{1.252360in}{1.151562in}}%
\pgfpathlineto{\pgfqpoint{1.328706in}{1.155979in}}%
\pgfpathlineto{\pgfqpoint{1.405052in}{1.161653in}}%
\pgfpathlineto{\pgfqpoint{1.481399in}{1.168799in}}%
\pgfpathlineto{\pgfqpoint{1.557745in}{1.177087in}}%
\pgfpathlineto{\pgfqpoint{1.634091in}{1.187694in}}%
\pgfpathlineto{\pgfqpoint{1.710437in}{1.200475in}}%
\pgfpathlineto{\pgfqpoint{1.786784in}{1.216666in}}%
\pgfpathlineto{\pgfqpoint{1.863130in}{1.238279in}}%
\pgfpathlineto{\pgfqpoint{1.939476in}{1.268992in}}%
\pgfpathlineto{\pgfqpoint{2.015822in}{1.318800in}}%
\pgfpathlineto{\pgfqpoint{2.092169in}{1.415163in}}%
\pgfpathlineto{\pgfqpoint{2.168515in}{1.293207in}}%
\pgfpathlineto{\pgfqpoint{2.244861in}{1.255712in}}%
\pgfpathlineto{\pgfqpoint{2.321207in}{1.234966in}}%
\pgfpathlineto{\pgfqpoint{2.397554in}{1.220798in}}%
\pgfpathlineto{\pgfqpoint{2.473900in}{1.211015in}}%
\pgfpathlineto{\pgfqpoint{2.550246in}{1.201785in}}%
\pgfpathlineto{\pgfqpoint{2.626592in}{1.195508in}}%
\pgfpathlineto{\pgfqpoint{2.702939in}{1.189784in}}%
\pgfpathlineto{\pgfqpoint{2.779285in}{1.185431in}}%
\pgfpathlineto{\pgfqpoint{2.855631in}{1.181168in}}%
\pgfpathlineto{\pgfqpoint{2.931977in}{1.178098in}}%
\pgfpathlineto{\pgfqpoint{3.008324in}{1.174945in}}%
\pgfpathlineto{\pgfqpoint{3.084670in}{1.172513in}}%
\pgfpathlineto{\pgfqpoint{3.161016in}{1.170524in}}%
\pgfpathlineto{\pgfqpoint{3.237362in}{1.168459in}}%
\pgfpathlineto{\pgfqpoint{3.313709in}{1.166835in}}%
\pgfpathlineto{\pgfqpoint{3.390055in}{1.165188in}}%
\pgfpathlineto{\pgfqpoint{3.466401in}{1.163757in}}%
\pgfpathlineto{\pgfqpoint{3.542747in}{1.162432in}}%
\pgfpathlineto{\pgfqpoint{3.619094in}{1.161291in}}%
\pgfpathlineto{\pgfqpoint{3.695440in}{1.160315in}}%
\pgfpathlineto{\pgfqpoint{3.771786in}{1.159306in}}%
\pgfusepath{stroke}%
\end{pgfscope}%
\begin{pgfscope}%
\pgfsetrectcap%
\pgfsetmiterjoin%
\pgfsetlinewidth{0.803000pt}%
\definecolor{currentstroke}{rgb}{0.000000,0.000000,0.000000}%
\pgfsetstrokecolor{currentstroke}%
\pgfsetdash{}{0pt}%
\pgfpathmoveto{\pgfqpoint{0.555972in}{0.387222in}}%
\pgfpathlineto{\pgfqpoint{0.555972in}{2.711667in}}%
\pgfusepath{stroke}%
\end{pgfscope}%
\begin{pgfscope}%
\pgfsetrectcap%
\pgfsetmiterjoin%
\pgfsetlinewidth{0.803000pt}%
\definecolor{currentstroke}{rgb}{0.000000,0.000000,0.000000}%
\pgfsetstrokecolor{currentstroke}%
\pgfsetdash{}{0pt}%
\pgfpathmoveto{\pgfqpoint{3.933750in}{0.387222in}}%
\pgfpathlineto{\pgfqpoint{3.933750in}{2.711667in}}%
\pgfusepath{stroke}%
\end{pgfscope}%
\begin{pgfscope}%
\pgfsetrectcap%
\pgfsetmiterjoin%
\pgfsetlinewidth{0.803000pt}%
\definecolor{currentstroke}{rgb}{0.000000,0.000000,0.000000}%
\pgfsetstrokecolor{currentstroke}%
\pgfsetdash{}{0pt}%
\pgfpathmoveto{\pgfqpoint{0.555972in}{0.387222in}}%
\pgfpathlineto{\pgfqpoint{3.933750in}{0.387222in}}%
\pgfusepath{stroke}%
\end{pgfscope}%
\begin{pgfscope}%
\pgfsetrectcap%
\pgfsetmiterjoin%
\pgfsetlinewidth{0.803000pt}%
\definecolor{currentstroke}{rgb}{0.000000,0.000000,0.000000}%
\pgfsetstrokecolor{currentstroke}%
\pgfsetdash{}{0pt}%
\pgfpathmoveto{\pgfqpoint{0.555972in}{2.711667in}}%
\pgfpathlineto{\pgfqpoint{3.933750in}{2.711667in}}%
\pgfusepath{stroke}%
\end{pgfscope}%
\begin{pgfscope}%
\definecolor{textcolor}{rgb}{0.000000,0.000000,0.000000}%
\pgfsetstrokecolor{textcolor}%
\pgfsetfillcolor{textcolor}%
\pgftext[x=2.244861in,y=2.795000in,,base]{\color{textcolor}\sffamily\fontsize{12.000000}{14.400000}\selectfont \(\displaystyle  N = 64 \)}%
\end{pgfscope}%
\begin{pgfscope}%
\pgfsetbuttcap%
\pgfsetmiterjoin%
\definecolor{currentfill}{rgb}{1.000000,1.000000,1.000000}%
\pgfsetfillcolor{currentfill}%
\pgfsetlinewidth{0.000000pt}%
\definecolor{currentstroke}{rgb}{0.000000,0.000000,0.000000}%
\pgfsetstrokecolor{currentstroke}%
\pgfsetstrokeopacity{0.000000}%
\pgfsetdash{}{0pt}%
\pgfpathmoveto{\pgfqpoint{4.437222in}{0.387222in}}%
\pgfpathlineto{\pgfqpoint{7.815000in}{0.387222in}}%
\pgfpathlineto{\pgfqpoint{7.815000in}{2.711667in}}%
\pgfpathlineto{\pgfqpoint{4.437222in}{2.711667in}}%
\pgfpathclose%
\pgfusepath{fill}%
\end{pgfscope}%
\begin{pgfscope}%
\pgfpathrectangle{\pgfqpoint{4.437222in}{0.387222in}}{\pgfqpoint{3.377778in}{2.324444in}}%
\pgfusepath{clip}%
\pgfsetbuttcap%
\pgfsetroundjoin%
\definecolor{currentfill}{rgb}{0.121569,0.466667,0.705882}%
\pgfsetfillcolor{currentfill}%
\pgfsetfillopacity{0.300000}%
\pgfsetlinewidth{1.003750pt}%
\definecolor{currentstroke}{rgb}{0.121569,0.466667,0.705882}%
\pgfsetstrokecolor{currentstroke}%
\pgfsetstrokeopacity{0.300000}%
\pgfsetdash{}{0pt}%
\pgfpathmoveto{\pgfqpoint{4.599186in}{0.459896in}}%
\pgfpathlineto{\pgfqpoint{4.599186in}{0.459876in}}%
\pgfpathlineto{\pgfqpoint{4.675532in}{0.460943in}}%
\pgfpathlineto{\pgfqpoint{4.751879in}{0.457212in}}%
\pgfpathlineto{\pgfqpoint{4.828225in}{0.460109in}}%
\pgfpathlineto{\pgfqpoint{4.904571in}{0.465086in}}%
\pgfpathlineto{\pgfqpoint{4.980917in}{0.478533in}}%
\pgfpathlineto{\pgfqpoint{5.057264in}{0.485175in}}%
\pgfpathlineto{\pgfqpoint{5.133610in}{0.506839in}}%
\pgfpathlineto{\pgfqpoint{5.209956in}{0.527027in}}%
\pgfpathlineto{\pgfqpoint{5.286302in}{0.556847in}}%
\pgfpathlineto{\pgfqpoint{5.362649in}{0.599925in}}%
\pgfpathlineto{\pgfqpoint{5.438995in}{0.649042in}}%
\pgfpathlineto{\pgfqpoint{5.515341in}{0.705853in}}%
\pgfpathlineto{\pgfqpoint{5.591687in}{0.767477in}}%
\pgfpathlineto{\pgfqpoint{5.668034in}{0.844921in}}%
\pgfpathlineto{\pgfqpoint{5.744380in}{0.963089in}}%
\pgfpathlineto{\pgfqpoint{5.820726in}{0.728954in}}%
\pgfpathlineto{\pgfqpoint{5.897072in}{1.431885in}}%
\pgfpathlineto{\pgfqpoint{5.973419in}{1.511463in}}%
\pgfpathlineto{\pgfqpoint{6.049765in}{1.237576in}}%
\pgfpathlineto{\pgfqpoint{6.126111in}{1.058096in}}%
\pgfpathlineto{\pgfqpoint{6.202457in}{0.972830in}}%
\pgfpathlineto{\pgfqpoint{6.278804in}{0.887753in}}%
\pgfpathlineto{\pgfqpoint{6.355150in}{0.823643in}}%
\pgfpathlineto{\pgfqpoint{6.431496in}{0.788405in}}%
\pgfpathlineto{\pgfqpoint{6.507842in}{0.741913in}}%
\pgfpathlineto{\pgfqpoint{6.584189in}{0.713403in}}%
\pgfpathlineto{\pgfqpoint{6.660535in}{0.689592in}}%
\pgfpathlineto{\pgfqpoint{6.736881in}{0.671518in}}%
\pgfpathlineto{\pgfqpoint{6.813227in}{0.652141in}}%
\pgfpathlineto{\pgfqpoint{6.889574in}{0.634504in}}%
\pgfpathlineto{\pgfqpoint{6.965920in}{0.621207in}}%
\pgfpathlineto{\pgfqpoint{7.042266in}{0.611891in}}%
\pgfpathlineto{\pgfqpoint{7.118612in}{0.599770in}}%
\pgfpathlineto{\pgfqpoint{7.194959in}{0.593823in}}%
\pgfpathlineto{\pgfqpoint{7.271305in}{0.582766in}}%
\pgfpathlineto{\pgfqpoint{7.347651in}{0.574257in}}%
\pgfpathlineto{\pgfqpoint{7.423997in}{0.567914in}}%
\pgfpathlineto{\pgfqpoint{7.500344in}{0.561715in}}%
\pgfpathlineto{\pgfqpoint{7.576690in}{0.556145in}}%
\pgfpathlineto{\pgfqpoint{7.653036in}{0.550749in}}%
\pgfpathlineto{\pgfqpoint{7.653036in}{0.552227in}}%
\pgfpathlineto{\pgfqpoint{7.653036in}{0.552227in}}%
\pgfpathlineto{\pgfqpoint{7.576690in}{0.557751in}}%
\pgfpathlineto{\pgfqpoint{7.500344in}{0.564816in}}%
\pgfpathlineto{\pgfqpoint{7.423997in}{0.570854in}}%
\pgfpathlineto{\pgfqpoint{7.347651in}{0.579201in}}%
\pgfpathlineto{\pgfqpoint{7.271305in}{0.585570in}}%
\pgfpathlineto{\pgfqpoint{7.194959in}{0.594382in}}%
\pgfpathlineto{\pgfqpoint{7.118612in}{0.604302in}}%
\pgfpathlineto{\pgfqpoint{7.042266in}{0.616505in}}%
\pgfpathlineto{\pgfqpoint{6.965920in}{0.628122in}}%
\pgfpathlineto{\pgfqpoint{6.889574in}{0.639630in}}%
\pgfpathlineto{\pgfqpoint{6.813227in}{0.659446in}}%
\pgfpathlineto{\pgfqpoint{6.736881in}{0.676756in}}%
\pgfpathlineto{\pgfqpoint{6.660535in}{0.704438in}}%
\pgfpathlineto{\pgfqpoint{6.584189in}{0.730791in}}%
\pgfpathlineto{\pgfqpoint{6.507842in}{0.757234in}}%
\pgfpathlineto{\pgfqpoint{6.431496in}{0.799979in}}%
\pgfpathlineto{\pgfqpoint{6.355150in}{0.838622in}}%
\pgfpathlineto{\pgfqpoint{6.278804in}{0.904323in}}%
\pgfpathlineto{\pgfqpoint{6.202457in}{0.996607in}}%
\pgfpathlineto{\pgfqpoint{6.126111in}{1.141701in}}%
\pgfpathlineto{\pgfqpoint{6.049765in}{1.315095in}}%
\pgfpathlineto{\pgfqpoint{5.973419in}{2.514066in}}%
\pgfpathlineto{\pgfqpoint{5.897072in}{2.406836in}}%
\pgfpathlineto{\pgfqpoint{5.820726in}{2.134922in}}%
\pgfpathlineto{\pgfqpoint{5.744380in}{0.996471in}}%
\pgfpathlineto{\pgfqpoint{5.668034in}{0.879519in}}%
\pgfpathlineto{\pgfqpoint{5.591687in}{0.790874in}}%
\pgfpathlineto{\pgfqpoint{5.515341in}{0.727937in}}%
\pgfpathlineto{\pgfqpoint{5.438995in}{0.664908in}}%
\pgfpathlineto{\pgfqpoint{5.362649in}{0.620364in}}%
\pgfpathlineto{\pgfqpoint{5.286302in}{0.578051in}}%
\pgfpathlineto{\pgfqpoint{5.209956in}{0.546101in}}%
\pgfpathlineto{\pgfqpoint{5.133610in}{0.520982in}}%
\pgfpathlineto{\pgfqpoint{5.057264in}{0.502082in}}%
\pgfpathlineto{\pgfqpoint{4.980917in}{0.494447in}}%
\pgfpathlineto{\pgfqpoint{4.904571in}{0.483411in}}%
\pgfpathlineto{\pgfqpoint{4.828225in}{0.473968in}}%
\pgfpathlineto{\pgfqpoint{4.751879in}{0.469316in}}%
\pgfpathlineto{\pgfqpoint{4.675532in}{0.472446in}}%
\pgfpathlineto{\pgfqpoint{4.599186in}{0.459896in}}%
\pgfpathclose%
\pgfusepath{stroke,fill}%
\end{pgfscope}%
\begin{pgfscope}%
\pgfpathrectangle{\pgfqpoint{4.437222in}{0.387222in}}{\pgfqpoint{3.377778in}{2.324444in}}%
\pgfusepath{clip}%
\pgfsetbuttcap%
\pgfsetroundjoin%
\definecolor{currentfill}{rgb}{0.121569,0.466667,0.705882}%
\pgfsetfillcolor{currentfill}%
\pgfsetlinewidth{1.003750pt}%
\definecolor{currentstroke}{rgb}{0.121569,0.466667,0.705882}%
\pgfsetstrokecolor{currentstroke}%
\pgfsetdash{}{0pt}%
\pgfsys@defobject{currentmarker}{\pgfqpoint{-0.009821in}{-0.009821in}}{\pgfqpoint{0.009821in}{0.009821in}}{%
\pgfpathmoveto{\pgfqpoint{0.000000in}{-0.009821in}}%
\pgfpathcurveto{\pgfqpoint{0.002605in}{-0.009821in}}{\pgfqpoint{0.005103in}{-0.008786in}}{\pgfqpoint{0.006944in}{-0.006944in}}%
\pgfpathcurveto{\pgfqpoint{0.008786in}{-0.005103in}}{\pgfqpoint{0.009821in}{-0.002605in}}{\pgfqpoint{0.009821in}{0.000000in}}%
\pgfpathcurveto{\pgfqpoint{0.009821in}{0.002605in}}{\pgfqpoint{0.008786in}{0.005103in}}{\pgfqpoint{0.006944in}{0.006944in}}%
\pgfpathcurveto{\pgfqpoint{0.005103in}{0.008786in}}{\pgfqpoint{0.002605in}{0.009821in}}{\pgfqpoint{0.000000in}{0.009821in}}%
\pgfpathcurveto{\pgfqpoint{-0.002605in}{0.009821in}}{\pgfqpoint{-0.005103in}{0.008786in}}{\pgfqpoint{-0.006944in}{0.006944in}}%
\pgfpathcurveto{\pgfqpoint{-0.008786in}{0.005103in}}{\pgfqpoint{-0.009821in}{0.002605in}}{\pgfqpoint{-0.009821in}{0.000000in}}%
\pgfpathcurveto{\pgfqpoint{-0.009821in}{-0.002605in}}{\pgfqpoint{-0.008786in}{-0.005103in}}{\pgfqpoint{-0.006944in}{-0.006944in}}%
\pgfpathcurveto{\pgfqpoint{-0.005103in}{-0.008786in}}{\pgfqpoint{-0.002605in}{-0.009821in}}{\pgfqpoint{0.000000in}{-0.009821in}}%
\pgfpathclose%
\pgfusepath{stroke,fill}%
}%
\begin{pgfscope}%
\pgfsys@transformshift{4.599186in}{0.459886in}%
\pgfsys@useobject{currentmarker}{}%
\end{pgfscope}%
\begin{pgfscope}%
\pgfsys@transformshift{4.675532in}{0.466695in}%
\pgfsys@useobject{currentmarker}{}%
\end{pgfscope}%
\begin{pgfscope}%
\pgfsys@transformshift{4.751879in}{0.463264in}%
\pgfsys@useobject{currentmarker}{}%
\end{pgfscope}%
\begin{pgfscope}%
\pgfsys@transformshift{4.828225in}{0.467038in}%
\pgfsys@useobject{currentmarker}{}%
\end{pgfscope}%
\begin{pgfscope}%
\pgfsys@transformshift{4.904571in}{0.474249in}%
\pgfsys@useobject{currentmarker}{}%
\end{pgfscope}%
\begin{pgfscope}%
\pgfsys@transformshift{4.980917in}{0.486490in}%
\pgfsys@useobject{currentmarker}{}%
\end{pgfscope}%
\begin{pgfscope}%
\pgfsys@transformshift{5.057264in}{0.493628in}%
\pgfsys@useobject{currentmarker}{}%
\end{pgfscope}%
\begin{pgfscope}%
\pgfsys@transformshift{5.133610in}{0.513911in}%
\pgfsys@useobject{currentmarker}{}%
\end{pgfscope}%
\begin{pgfscope}%
\pgfsys@transformshift{5.209956in}{0.536564in}%
\pgfsys@useobject{currentmarker}{}%
\end{pgfscope}%
\begin{pgfscope}%
\pgfsys@transformshift{5.286302in}{0.567449in}%
\pgfsys@useobject{currentmarker}{}%
\end{pgfscope}%
\begin{pgfscope}%
\pgfsys@transformshift{5.362649in}{0.610144in}%
\pgfsys@useobject{currentmarker}{}%
\end{pgfscope}%
\begin{pgfscope}%
\pgfsys@transformshift{5.438995in}{0.656975in}%
\pgfsys@useobject{currentmarker}{}%
\end{pgfscope}%
\begin{pgfscope}%
\pgfsys@transformshift{5.515341in}{0.716895in}%
\pgfsys@useobject{currentmarker}{}%
\end{pgfscope}%
\begin{pgfscope}%
\pgfsys@transformshift{5.591687in}{0.779176in}%
\pgfsys@useobject{currentmarker}{}%
\end{pgfscope}%
\begin{pgfscope}%
\pgfsys@transformshift{5.668034in}{0.862220in}%
\pgfsys@useobject{currentmarker}{}%
\end{pgfscope}%
\begin{pgfscope}%
\pgfsys@transformshift{5.744380in}{0.979780in}%
\pgfsys@useobject{currentmarker}{}%
\end{pgfscope}%
\begin{pgfscope}%
\pgfsys@transformshift{5.820726in}{1.431938in}%
\pgfsys@useobject{currentmarker}{}%
\end{pgfscope}%
\begin{pgfscope}%
\pgfsys@transformshift{5.897072in}{1.919360in}%
\pgfsys@useobject{currentmarker}{}%
\end{pgfscope}%
\begin{pgfscope}%
\pgfsys@transformshift{5.973419in}{2.012764in}%
\pgfsys@useobject{currentmarker}{}%
\end{pgfscope}%
\begin{pgfscope}%
\pgfsys@transformshift{6.049765in}{1.276336in}%
\pgfsys@useobject{currentmarker}{}%
\end{pgfscope}%
\begin{pgfscope}%
\pgfsys@transformshift{6.126111in}{1.099898in}%
\pgfsys@useobject{currentmarker}{}%
\end{pgfscope}%
\begin{pgfscope}%
\pgfsys@transformshift{6.202457in}{0.984718in}%
\pgfsys@useobject{currentmarker}{}%
\end{pgfscope}%
\begin{pgfscope}%
\pgfsys@transformshift{6.278804in}{0.896038in}%
\pgfsys@useobject{currentmarker}{}%
\end{pgfscope}%
\begin{pgfscope}%
\pgfsys@transformshift{6.355150in}{0.831133in}%
\pgfsys@useobject{currentmarker}{}%
\end{pgfscope}%
\begin{pgfscope}%
\pgfsys@transformshift{6.431496in}{0.794192in}%
\pgfsys@useobject{currentmarker}{}%
\end{pgfscope}%
\begin{pgfscope}%
\pgfsys@transformshift{6.507842in}{0.749574in}%
\pgfsys@useobject{currentmarker}{}%
\end{pgfscope}%
\begin{pgfscope}%
\pgfsys@transformshift{6.584189in}{0.722097in}%
\pgfsys@useobject{currentmarker}{}%
\end{pgfscope}%
\begin{pgfscope}%
\pgfsys@transformshift{6.660535in}{0.697015in}%
\pgfsys@useobject{currentmarker}{}%
\end{pgfscope}%
\begin{pgfscope}%
\pgfsys@transformshift{6.736881in}{0.674137in}%
\pgfsys@useobject{currentmarker}{}%
\end{pgfscope}%
\begin{pgfscope}%
\pgfsys@transformshift{6.813227in}{0.655793in}%
\pgfsys@useobject{currentmarker}{}%
\end{pgfscope}%
\begin{pgfscope}%
\pgfsys@transformshift{6.889574in}{0.637067in}%
\pgfsys@useobject{currentmarker}{}%
\end{pgfscope}%
\begin{pgfscope}%
\pgfsys@transformshift{6.965920in}{0.624664in}%
\pgfsys@useobject{currentmarker}{}%
\end{pgfscope}%
\begin{pgfscope}%
\pgfsys@transformshift{7.042266in}{0.614198in}%
\pgfsys@useobject{currentmarker}{}%
\end{pgfscope}%
\begin{pgfscope}%
\pgfsys@transformshift{7.118612in}{0.602036in}%
\pgfsys@useobject{currentmarker}{}%
\end{pgfscope}%
\begin{pgfscope}%
\pgfsys@transformshift{7.194959in}{0.594103in}%
\pgfsys@useobject{currentmarker}{}%
\end{pgfscope}%
\begin{pgfscope}%
\pgfsys@transformshift{7.271305in}{0.584168in}%
\pgfsys@useobject{currentmarker}{}%
\end{pgfscope}%
\begin{pgfscope}%
\pgfsys@transformshift{7.347651in}{0.576729in}%
\pgfsys@useobject{currentmarker}{}%
\end{pgfscope}%
\begin{pgfscope}%
\pgfsys@transformshift{7.423997in}{0.569384in}%
\pgfsys@useobject{currentmarker}{}%
\end{pgfscope}%
\begin{pgfscope}%
\pgfsys@transformshift{7.500344in}{0.563266in}%
\pgfsys@useobject{currentmarker}{}%
\end{pgfscope}%
\begin{pgfscope}%
\pgfsys@transformshift{7.576690in}{0.556948in}%
\pgfsys@useobject{currentmarker}{}%
\end{pgfscope}%
\begin{pgfscope}%
\pgfsys@transformshift{7.653036in}{0.551488in}%
\pgfsys@useobject{currentmarker}{}%
\end{pgfscope}%
\end{pgfscope}%
\begin{pgfscope}%
\pgfsetbuttcap%
\pgfsetroundjoin%
\definecolor{currentfill}{rgb}{0.000000,0.000000,0.000000}%
\pgfsetfillcolor{currentfill}%
\pgfsetlinewidth{0.803000pt}%
\definecolor{currentstroke}{rgb}{0.000000,0.000000,0.000000}%
\pgfsetstrokecolor{currentstroke}%
\pgfsetdash{}{0pt}%
\pgfsys@defobject{currentmarker}{\pgfqpoint{0.000000in}{-0.048611in}}{\pgfqpoint{0.000000in}{0.000000in}}{%
\pgfpathmoveto{\pgfqpoint{0.000000in}{0.000000in}}%
\pgfpathlineto{\pgfqpoint{0.000000in}{-0.048611in}}%
\pgfusepath{stroke,fill}%
}%
\begin{pgfscope}%
\pgfsys@transformshift{4.980917in}{0.387222in}%
\pgfsys@useobject{currentmarker}{}%
\end{pgfscope}%
\end{pgfscope}%
\begin{pgfscope}%
\definecolor{textcolor}{rgb}{0.000000,0.000000,0.000000}%
\pgfsetstrokecolor{textcolor}%
\pgfsetfillcolor{textcolor}%
\pgftext[x=4.980917in,y=0.290000in,,top]{\color{textcolor}\sffamily\fontsize{10.000000}{12.000000}\selectfont 1}%
\end{pgfscope}%
\begin{pgfscope}%
\pgfsetbuttcap%
\pgfsetroundjoin%
\definecolor{currentfill}{rgb}{0.000000,0.000000,0.000000}%
\pgfsetfillcolor{currentfill}%
\pgfsetlinewidth{0.803000pt}%
\definecolor{currentstroke}{rgb}{0.000000,0.000000,0.000000}%
\pgfsetstrokecolor{currentstroke}%
\pgfsetdash{}{0pt}%
\pgfsys@defobject{currentmarker}{\pgfqpoint{0.000000in}{-0.048611in}}{\pgfqpoint{0.000000in}{0.000000in}}{%
\pgfpathmoveto{\pgfqpoint{0.000000in}{0.000000in}}%
\pgfpathlineto{\pgfqpoint{0.000000in}{-0.048611in}}%
\pgfusepath{stroke,fill}%
}%
\begin{pgfscope}%
\pgfsys@transformshift{5.744380in}{0.387222in}%
\pgfsys@useobject{currentmarker}{}%
\end{pgfscope}%
\end{pgfscope}%
\begin{pgfscope}%
\definecolor{textcolor}{rgb}{0.000000,0.000000,0.000000}%
\pgfsetstrokecolor{textcolor}%
\pgfsetfillcolor{textcolor}%
\pgftext[x=5.744380in,y=0.290000in,,top]{\color{textcolor}\sffamily\fontsize{10.000000}{12.000000}\selectfont 2}%
\end{pgfscope}%
\begin{pgfscope}%
\pgfsetbuttcap%
\pgfsetroundjoin%
\definecolor{currentfill}{rgb}{0.000000,0.000000,0.000000}%
\pgfsetfillcolor{currentfill}%
\pgfsetlinewidth{0.803000pt}%
\definecolor{currentstroke}{rgb}{0.000000,0.000000,0.000000}%
\pgfsetstrokecolor{currentstroke}%
\pgfsetdash{}{0pt}%
\pgfsys@defobject{currentmarker}{\pgfqpoint{0.000000in}{-0.048611in}}{\pgfqpoint{0.000000in}{0.000000in}}{%
\pgfpathmoveto{\pgfqpoint{0.000000in}{0.000000in}}%
\pgfpathlineto{\pgfqpoint{0.000000in}{-0.048611in}}%
\pgfusepath{stroke,fill}%
}%
\begin{pgfscope}%
\pgfsys@transformshift{6.507842in}{0.387222in}%
\pgfsys@useobject{currentmarker}{}%
\end{pgfscope}%
\end{pgfscope}%
\begin{pgfscope}%
\definecolor{textcolor}{rgb}{0.000000,0.000000,0.000000}%
\pgfsetstrokecolor{textcolor}%
\pgfsetfillcolor{textcolor}%
\pgftext[x=6.507842in,y=0.290000in,,top]{\color{textcolor}\sffamily\fontsize{10.000000}{12.000000}\selectfont 3}%
\end{pgfscope}%
\begin{pgfscope}%
\pgfsetbuttcap%
\pgfsetroundjoin%
\definecolor{currentfill}{rgb}{0.000000,0.000000,0.000000}%
\pgfsetfillcolor{currentfill}%
\pgfsetlinewidth{0.803000pt}%
\definecolor{currentstroke}{rgb}{0.000000,0.000000,0.000000}%
\pgfsetstrokecolor{currentstroke}%
\pgfsetdash{}{0pt}%
\pgfsys@defobject{currentmarker}{\pgfqpoint{0.000000in}{-0.048611in}}{\pgfqpoint{0.000000in}{0.000000in}}{%
\pgfpathmoveto{\pgfqpoint{0.000000in}{0.000000in}}%
\pgfpathlineto{\pgfqpoint{0.000000in}{-0.048611in}}%
\pgfusepath{stroke,fill}%
}%
\begin{pgfscope}%
\pgfsys@transformshift{7.271305in}{0.387222in}%
\pgfsys@useobject{currentmarker}{}%
\end{pgfscope}%
\end{pgfscope}%
\begin{pgfscope}%
\definecolor{textcolor}{rgb}{0.000000,0.000000,0.000000}%
\pgfsetstrokecolor{textcolor}%
\pgfsetfillcolor{textcolor}%
\pgftext[x=7.271305in,y=0.290000in,,top]{\color{textcolor}\sffamily\fontsize{10.000000}{12.000000}\selectfont 4}%
\end{pgfscope}%
\begin{pgfscope}%
\pgfsetbuttcap%
\pgfsetroundjoin%
\definecolor{currentfill}{rgb}{0.000000,0.000000,0.000000}%
\pgfsetfillcolor{currentfill}%
\pgfsetlinewidth{0.803000pt}%
\definecolor{currentstroke}{rgb}{0.000000,0.000000,0.000000}%
\pgfsetstrokecolor{currentstroke}%
\pgfsetdash{}{0pt}%
\pgfsys@defobject{currentmarker}{\pgfqpoint{-0.048611in}{0.000000in}}{\pgfqpoint{0.000000in}{0.000000in}}{%
\pgfpathmoveto{\pgfqpoint{0.000000in}{0.000000in}}%
\pgfpathlineto{\pgfqpoint{-0.048611in}{0.000000in}}%
\pgfusepath{stroke,fill}%
}%
\begin{pgfscope}%
\pgfsys@transformshift{4.437222in}{0.459861in}%
\pgfsys@useobject{currentmarker}{}%
\end{pgfscope}%
\end{pgfscope}%
\begin{pgfscope}%
\definecolor{textcolor}{rgb}{0.000000,0.000000,0.000000}%
\pgfsetstrokecolor{textcolor}%
\pgfsetfillcolor{textcolor}%
\pgftext[x=4.119121in,y=0.407100in,left,base]{\color{textcolor}\sffamily\fontsize{10.000000}{12.000000}\selectfont 0.0}%
\end{pgfscope}%
\begin{pgfscope}%
\pgfsetbuttcap%
\pgfsetroundjoin%
\definecolor{currentfill}{rgb}{0.000000,0.000000,0.000000}%
\pgfsetfillcolor{currentfill}%
\pgfsetlinewidth{0.803000pt}%
\definecolor{currentstroke}{rgb}{0.000000,0.000000,0.000000}%
\pgfsetstrokecolor{currentstroke}%
\pgfsetdash{}{0pt}%
\pgfsys@defobject{currentmarker}{\pgfqpoint{-0.048611in}{0.000000in}}{\pgfqpoint{0.000000in}{0.000000in}}{%
\pgfpathmoveto{\pgfqpoint{0.000000in}{0.000000in}}%
\pgfpathlineto{\pgfqpoint{-0.048611in}{0.000000in}}%
\pgfusepath{stroke,fill}%
}%
\begin{pgfscope}%
\pgfsys@transformshift{4.437222in}{0.823056in}%
\pgfsys@useobject{currentmarker}{}%
\end{pgfscope}%
\end{pgfscope}%
\begin{pgfscope}%
\definecolor{textcolor}{rgb}{0.000000,0.000000,0.000000}%
\pgfsetstrokecolor{textcolor}%
\pgfsetfillcolor{textcolor}%
\pgftext[x=4.119121in,y=0.770294in,left,base]{\color{textcolor}\sffamily\fontsize{10.000000}{12.000000}\selectfont 0.5}%
\end{pgfscope}%
\begin{pgfscope}%
\pgfsetbuttcap%
\pgfsetroundjoin%
\definecolor{currentfill}{rgb}{0.000000,0.000000,0.000000}%
\pgfsetfillcolor{currentfill}%
\pgfsetlinewidth{0.803000pt}%
\definecolor{currentstroke}{rgb}{0.000000,0.000000,0.000000}%
\pgfsetstrokecolor{currentstroke}%
\pgfsetdash{}{0pt}%
\pgfsys@defobject{currentmarker}{\pgfqpoint{-0.048611in}{0.000000in}}{\pgfqpoint{0.000000in}{0.000000in}}{%
\pgfpathmoveto{\pgfqpoint{0.000000in}{0.000000in}}%
\pgfpathlineto{\pgfqpoint{-0.048611in}{0.000000in}}%
\pgfusepath{stroke,fill}%
}%
\begin{pgfscope}%
\pgfsys@transformshift{4.437222in}{1.186250in}%
\pgfsys@useobject{currentmarker}{}%
\end{pgfscope}%
\end{pgfscope}%
\begin{pgfscope}%
\definecolor{textcolor}{rgb}{0.000000,0.000000,0.000000}%
\pgfsetstrokecolor{textcolor}%
\pgfsetfillcolor{textcolor}%
\pgftext[x=4.119121in,y=1.133488in,left,base]{\color{textcolor}\sffamily\fontsize{10.000000}{12.000000}\selectfont 1.0}%
\end{pgfscope}%
\begin{pgfscope}%
\pgfsetbuttcap%
\pgfsetroundjoin%
\definecolor{currentfill}{rgb}{0.000000,0.000000,0.000000}%
\pgfsetfillcolor{currentfill}%
\pgfsetlinewidth{0.803000pt}%
\definecolor{currentstroke}{rgb}{0.000000,0.000000,0.000000}%
\pgfsetstrokecolor{currentstroke}%
\pgfsetdash{}{0pt}%
\pgfsys@defobject{currentmarker}{\pgfqpoint{-0.048611in}{0.000000in}}{\pgfqpoint{0.000000in}{0.000000in}}{%
\pgfpathmoveto{\pgfqpoint{0.000000in}{0.000000in}}%
\pgfpathlineto{\pgfqpoint{-0.048611in}{0.000000in}}%
\pgfusepath{stroke,fill}%
}%
\begin{pgfscope}%
\pgfsys@transformshift{4.437222in}{1.549444in}%
\pgfsys@useobject{currentmarker}{}%
\end{pgfscope}%
\end{pgfscope}%
\begin{pgfscope}%
\definecolor{textcolor}{rgb}{0.000000,0.000000,0.000000}%
\pgfsetstrokecolor{textcolor}%
\pgfsetfillcolor{textcolor}%
\pgftext[x=4.119121in,y=1.496683in,left,base]{\color{textcolor}\sffamily\fontsize{10.000000}{12.000000}\selectfont 1.5}%
\end{pgfscope}%
\begin{pgfscope}%
\pgfsetbuttcap%
\pgfsetroundjoin%
\definecolor{currentfill}{rgb}{0.000000,0.000000,0.000000}%
\pgfsetfillcolor{currentfill}%
\pgfsetlinewidth{0.803000pt}%
\definecolor{currentstroke}{rgb}{0.000000,0.000000,0.000000}%
\pgfsetstrokecolor{currentstroke}%
\pgfsetdash{}{0pt}%
\pgfsys@defobject{currentmarker}{\pgfqpoint{-0.048611in}{0.000000in}}{\pgfqpoint{0.000000in}{0.000000in}}{%
\pgfpathmoveto{\pgfqpoint{0.000000in}{0.000000in}}%
\pgfpathlineto{\pgfqpoint{-0.048611in}{0.000000in}}%
\pgfusepath{stroke,fill}%
}%
\begin{pgfscope}%
\pgfsys@transformshift{4.437222in}{1.912639in}%
\pgfsys@useobject{currentmarker}{}%
\end{pgfscope}%
\end{pgfscope}%
\begin{pgfscope}%
\definecolor{textcolor}{rgb}{0.000000,0.000000,0.000000}%
\pgfsetstrokecolor{textcolor}%
\pgfsetfillcolor{textcolor}%
\pgftext[x=4.119121in,y=1.859877in,left,base]{\color{textcolor}\sffamily\fontsize{10.000000}{12.000000}\selectfont 2.0}%
\end{pgfscope}%
\begin{pgfscope}%
\pgfsetbuttcap%
\pgfsetroundjoin%
\definecolor{currentfill}{rgb}{0.000000,0.000000,0.000000}%
\pgfsetfillcolor{currentfill}%
\pgfsetlinewidth{0.803000pt}%
\definecolor{currentstroke}{rgb}{0.000000,0.000000,0.000000}%
\pgfsetstrokecolor{currentstroke}%
\pgfsetdash{}{0pt}%
\pgfsys@defobject{currentmarker}{\pgfqpoint{-0.048611in}{0.000000in}}{\pgfqpoint{0.000000in}{0.000000in}}{%
\pgfpathmoveto{\pgfqpoint{0.000000in}{0.000000in}}%
\pgfpathlineto{\pgfqpoint{-0.048611in}{0.000000in}}%
\pgfusepath{stroke,fill}%
}%
\begin{pgfscope}%
\pgfsys@transformshift{4.437222in}{2.275833in}%
\pgfsys@useobject{currentmarker}{}%
\end{pgfscope}%
\end{pgfscope}%
\begin{pgfscope}%
\definecolor{textcolor}{rgb}{0.000000,0.000000,0.000000}%
\pgfsetstrokecolor{textcolor}%
\pgfsetfillcolor{textcolor}%
\pgftext[x=4.119121in,y=2.223072in,left,base]{\color{textcolor}\sffamily\fontsize{10.000000}{12.000000}\selectfont 2.5}%
\end{pgfscope}%
\begin{pgfscope}%
\pgfsetbuttcap%
\pgfsetroundjoin%
\definecolor{currentfill}{rgb}{0.000000,0.000000,0.000000}%
\pgfsetfillcolor{currentfill}%
\pgfsetlinewidth{0.803000pt}%
\definecolor{currentstroke}{rgb}{0.000000,0.000000,0.000000}%
\pgfsetstrokecolor{currentstroke}%
\pgfsetdash{}{0pt}%
\pgfsys@defobject{currentmarker}{\pgfqpoint{-0.048611in}{0.000000in}}{\pgfqpoint{0.000000in}{0.000000in}}{%
\pgfpathmoveto{\pgfqpoint{0.000000in}{0.000000in}}%
\pgfpathlineto{\pgfqpoint{-0.048611in}{0.000000in}}%
\pgfusepath{stroke,fill}%
}%
\begin{pgfscope}%
\pgfsys@transformshift{4.437222in}{2.639028in}%
\pgfsys@useobject{currentmarker}{}%
\end{pgfscope}%
\end{pgfscope}%
\begin{pgfscope}%
\definecolor{textcolor}{rgb}{0.000000,0.000000,0.000000}%
\pgfsetstrokecolor{textcolor}%
\pgfsetfillcolor{textcolor}%
\pgftext[x=4.119121in,y=2.586266in,left,base]{\color{textcolor}\sffamily\fontsize{10.000000}{12.000000}\selectfont 3.0}%
\end{pgfscope}%
\begin{pgfscope}%
\pgfpathrectangle{\pgfqpoint{4.437222in}{0.387222in}}{\pgfqpoint{3.377778in}{2.324444in}}%
\pgfusepath{clip}%
\pgfsetrectcap%
\pgfsetroundjoin%
\pgfsetlinewidth{1.505625pt}%
\definecolor{currentstroke}{rgb}{0.121569,0.466667,0.705882}%
\pgfsetstrokecolor{currentstroke}%
\pgfsetdash{}{0pt}%
\pgfpathmoveto{\pgfqpoint{4.599186in}{0.459886in}}%
\pgfpathlineto{\pgfqpoint{4.675532in}{0.466695in}}%
\pgfpathlineto{\pgfqpoint{4.751879in}{0.463264in}}%
\pgfpathlineto{\pgfqpoint{4.828225in}{0.467038in}}%
\pgfpathlineto{\pgfqpoint{4.904571in}{0.474249in}}%
\pgfpathlineto{\pgfqpoint{4.980917in}{0.486490in}}%
\pgfpathlineto{\pgfqpoint{5.057264in}{0.493628in}}%
\pgfpathlineto{\pgfqpoint{5.133610in}{0.513911in}}%
\pgfpathlineto{\pgfqpoint{5.209956in}{0.536564in}}%
\pgfpathlineto{\pgfqpoint{5.286302in}{0.567449in}}%
\pgfpathlineto{\pgfqpoint{5.362649in}{0.610144in}}%
\pgfpathlineto{\pgfqpoint{5.438995in}{0.656975in}}%
\pgfpathlineto{\pgfqpoint{5.515341in}{0.716895in}}%
\pgfpathlineto{\pgfqpoint{5.591687in}{0.779176in}}%
\pgfpathlineto{\pgfqpoint{5.668034in}{0.862220in}}%
\pgfpathlineto{\pgfqpoint{5.744380in}{0.979780in}}%
\pgfpathlineto{\pgfqpoint{5.820726in}{1.431938in}}%
\pgfpathlineto{\pgfqpoint{5.897072in}{1.919360in}}%
\pgfpathlineto{\pgfqpoint{5.973419in}{2.012764in}}%
\pgfpathlineto{\pgfqpoint{6.049765in}{1.276336in}}%
\pgfpathlineto{\pgfqpoint{6.126111in}{1.099898in}}%
\pgfpathlineto{\pgfqpoint{6.202457in}{0.984718in}}%
\pgfpathlineto{\pgfqpoint{6.278804in}{0.896038in}}%
\pgfpathlineto{\pgfqpoint{6.355150in}{0.831133in}}%
\pgfpathlineto{\pgfqpoint{6.431496in}{0.794192in}}%
\pgfpathlineto{\pgfqpoint{6.507842in}{0.749574in}}%
\pgfpathlineto{\pgfqpoint{6.584189in}{0.722097in}}%
\pgfpathlineto{\pgfqpoint{6.660535in}{0.697015in}}%
\pgfpathlineto{\pgfqpoint{6.736881in}{0.674137in}}%
\pgfpathlineto{\pgfqpoint{6.813227in}{0.655793in}}%
\pgfpathlineto{\pgfqpoint{6.889574in}{0.637067in}}%
\pgfpathlineto{\pgfqpoint{6.965920in}{0.624664in}}%
\pgfpathlineto{\pgfqpoint{7.042266in}{0.614198in}}%
\pgfpathlineto{\pgfqpoint{7.118612in}{0.602036in}}%
\pgfpathlineto{\pgfqpoint{7.194959in}{0.594103in}}%
\pgfpathlineto{\pgfqpoint{7.271305in}{0.584168in}}%
\pgfpathlineto{\pgfqpoint{7.347651in}{0.576729in}}%
\pgfpathlineto{\pgfqpoint{7.423997in}{0.569384in}}%
\pgfpathlineto{\pgfqpoint{7.500344in}{0.563266in}}%
\pgfpathlineto{\pgfqpoint{7.576690in}{0.556948in}}%
\pgfpathlineto{\pgfqpoint{7.653036in}{0.551488in}}%
\pgfusepath{stroke}%
\end{pgfscope}%
\begin{pgfscope}%
\pgfsetrectcap%
\pgfsetmiterjoin%
\pgfsetlinewidth{0.803000pt}%
\definecolor{currentstroke}{rgb}{0.000000,0.000000,0.000000}%
\pgfsetstrokecolor{currentstroke}%
\pgfsetdash{}{0pt}%
\pgfpathmoveto{\pgfqpoint{4.437222in}{0.387222in}}%
\pgfpathlineto{\pgfqpoint{4.437222in}{2.711667in}}%
\pgfusepath{stroke}%
\end{pgfscope}%
\begin{pgfscope}%
\pgfsetrectcap%
\pgfsetmiterjoin%
\pgfsetlinewidth{0.803000pt}%
\definecolor{currentstroke}{rgb}{0.000000,0.000000,0.000000}%
\pgfsetstrokecolor{currentstroke}%
\pgfsetdash{}{0pt}%
\pgfpathmoveto{\pgfqpoint{7.815000in}{0.387222in}}%
\pgfpathlineto{\pgfqpoint{7.815000in}{2.711667in}}%
\pgfusepath{stroke}%
\end{pgfscope}%
\begin{pgfscope}%
\pgfsetrectcap%
\pgfsetmiterjoin%
\pgfsetlinewidth{0.803000pt}%
\definecolor{currentstroke}{rgb}{0.000000,0.000000,0.000000}%
\pgfsetstrokecolor{currentstroke}%
\pgfsetdash{}{0pt}%
\pgfpathmoveto{\pgfqpoint{4.437222in}{0.387222in}}%
\pgfpathlineto{\pgfqpoint{7.815000in}{0.387222in}}%
\pgfusepath{stroke}%
\end{pgfscope}%
\begin{pgfscope}%
\pgfsetrectcap%
\pgfsetmiterjoin%
\pgfsetlinewidth{0.803000pt}%
\definecolor{currentstroke}{rgb}{0.000000,0.000000,0.000000}%
\pgfsetstrokecolor{currentstroke}%
\pgfsetdash{}{0pt}%
\pgfpathmoveto{\pgfqpoint{4.437222in}{2.711667in}}%
\pgfpathlineto{\pgfqpoint{7.815000in}{2.711667in}}%
\pgfusepath{stroke}%
\end{pgfscope}%
\begin{pgfscope}%
\definecolor{textcolor}{rgb}{0.000000,0.000000,0.000000}%
\pgfsetstrokecolor{textcolor}%
\pgfsetfillcolor{textcolor}%
\pgftext[x=6.126111in,y=2.795000in,,base]{\color{textcolor}\sffamily\fontsize{12.000000}{14.400000}\selectfont \(\displaystyle  N = 128 \)}%
\end{pgfscope}%
\end{pgfpicture}%
\makeatother%
\endgroup%
}
\caption{Specific heat $c$ for different temperatures $T$ using Metropolis--Hastings algorithm}
\label{Fig:CapSmall}
\end{figure}

\begin{figure}[htbp]
\centering
\scalebox{0.666}{%% Creator: Matplotlib, PGF backend
%%
%% To include the figure in your LaTeX document, write
%%   \input{<filename>.pgf}
%%
%% Make sure the required packages are loaded in your preamble
%%   \usepackage{pgf}
%%
%% Figures using additional raster images can only be included by \input if
%% they are in the same directory as the main LaTeX file. For loading figures
%% from other directories you can use the `import` package
%%   \usepackage{import}
%% and then include the figures with
%%   \import{<path to file>}{<filename>.pgf}
%%
%% Matplotlib used the following preamble
%%   \usepackage{fontspec}
%%   \setmainfont{DejaVuSerif.ttf}[Path=/home/lzh/anaconda3/envs/numana/lib/python3.7/site-packages/matplotlib/mpl-data/fonts/ttf/]
%%   \setsansfont{DejaVuSans.ttf}[Path=/home/lzh/anaconda3/envs/numana/lib/python3.7/site-packages/matplotlib/mpl-data/fonts/ttf/]
%%   \setmonofont{DejaVuSansMono.ttf}[Path=/home/lzh/anaconda3/envs/numana/lib/python3.7/site-packages/matplotlib/mpl-data/fonts/ttf/]
%%
\begingroup%
\makeatletter%
\begin{pgfpicture}%
\pgfpathrectangle{\pgfpointorigin}{\pgfqpoint{4.000000in}{4.000000in}}%
\pgfusepath{use as bounding box, clip}%
\begin{pgfscope}%
\pgfsetbuttcap%
\pgfsetmiterjoin%
\definecolor{currentfill}{rgb}{1.000000,1.000000,1.000000}%
\pgfsetfillcolor{currentfill}%
\pgfsetlinewidth{0.000000pt}%
\definecolor{currentstroke}{rgb}{1.000000,1.000000,1.000000}%
\pgfsetstrokecolor{currentstroke}%
\pgfsetdash{}{0pt}%
\pgfpathmoveto{\pgfqpoint{0.000000in}{0.000000in}}%
\pgfpathlineto{\pgfqpoint{4.000000in}{0.000000in}}%
\pgfpathlineto{\pgfqpoint{4.000000in}{4.000000in}}%
\pgfpathlineto{\pgfqpoint{0.000000in}{4.000000in}}%
\pgfpathclose%
\pgfusepath{fill}%
\end{pgfscope}%
\begin{pgfscope}%
\pgfsetbuttcap%
\pgfsetmiterjoin%
\definecolor{currentfill}{rgb}{1.000000,1.000000,1.000000}%
\pgfsetfillcolor{currentfill}%
\pgfsetlinewidth{0.000000pt}%
\definecolor{currentstroke}{rgb}{0.000000,0.000000,0.000000}%
\pgfsetstrokecolor{currentstroke}%
\pgfsetstrokeopacity{0.000000}%
\pgfsetdash{}{0pt}%
\pgfpathmoveto{\pgfqpoint{0.510000in}{0.440000in}}%
\pgfpathlineto{\pgfqpoint{3.590000in}{0.440000in}}%
\pgfpathlineto{\pgfqpoint{3.590000in}{3.520000in}}%
\pgfpathlineto{\pgfqpoint{0.510000in}{3.520000in}}%
\pgfpathclose%
\pgfusepath{fill}%
\end{pgfscope}%
\begin{pgfscope}%
\pgfpathrectangle{\pgfqpoint{0.510000in}{0.440000in}}{\pgfqpoint{3.080000in}{3.080000in}}%
\pgfusepath{clip}%
\pgfsetbuttcap%
\pgfsetroundjoin%
\definecolor{currentfill}{rgb}{0.121569,0.466667,0.705882}%
\pgfsetfillcolor{currentfill}%
\pgfsetlinewidth{1.003750pt}%
\definecolor{currentstroke}{rgb}{0.121569,0.466667,0.705882}%
\pgfsetstrokecolor{currentstroke}%
\pgfsetdash{}{0pt}%
\pgfpathmoveto{\pgfqpoint{1.342829in}{2.324098in}}%
\pgfpathcurveto{\pgfqpoint{1.348653in}{2.324098in}}{\pgfqpoint{1.354239in}{2.326412in}}{\pgfqpoint{1.358357in}{2.330530in}}%
\pgfpathcurveto{\pgfqpoint{1.362476in}{2.334648in}}{\pgfqpoint{1.364789in}{2.340235in}}{\pgfqpoint{1.364789in}{2.346059in}}%
\pgfpathcurveto{\pgfqpoint{1.364789in}{2.351882in}}{\pgfqpoint{1.362476in}{2.357469in}}{\pgfqpoint{1.358357in}{2.361587in}}%
\pgfpathcurveto{\pgfqpoint{1.354239in}{2.365705in}}{\pgfqpoint{1.348653in}{2.368019in}}{\pgfqpoint{1.342829in}{2.368019in}}%
\pgfpathcurveto{\pgfqpoint{1.337005in}{2.368019in}}{\pgfqpoint{1.331419in}{2.365705in}}{\pgfqpoint{1.327301in}{2.361587in}}%
\pgfpathcurveto{\pgfqpoint{1.323183in}{2.357469in}}{\pgfqpoint{1.320869in}{2.351882in}}{\pgfqpoint{1.320869in}{2.346059in}}%
\pgfpathcurveto{\pgfqpoint{1.320869in}{2.340235in}}{\pgfqpoint{1.323183in}{2.334648in}}{\pgfqpoint{1.327301in}{2.330530in}}%
\pgfpathcurveto{\pgfqpoint{1.331419in}{2.326412in}}{\pgfqpoint{1.337005in}{2.324098in}}{\pgfqpoint{1.342829in}{2.324098in}}%
\pgfpathclose%
\pgfusepath{stroke,fill}%
\end{pgfscope}%
\begin{pgfscope}%
\pgfpathrectangle{\pgfqpoint{0.510000in}{0.440000in}}{\pgfqpoint{3.080000in}{3.080000in}}%
\pgfusepath{clip}%
\pgfsetbuttcap%
\pgfsetroundjoin%
\definecolor{currentfill}{rgb}{1.000000,0.498039,0.054902}%
\pgfsetfillcolor{currentfill}%
\pgfsetlinewidth{1.003750pt}%
\definecolor{currentstroke}{rgb}{1.000000,0.498039,0.054902}%
\pgfsetstrokecolor{currentstroke}%
\pgfsetdash{}{0pt}%
\pgfpathmoveto{\pgfqpoint{2.416059in}{1.250869in}}%
\pgfpathcurveto{\pgfqpoint{2.421882in}{1.250869in}}{\pgfqpoint{2.427469in}{1.253183in}}{\pgfqpoint{2.431587in}{1.257301in}}%
\pgfpathcurveto{\pgfqpoint{2.435705in}{1.261419in}}{\pgfqpoint{2.438019in}{1.267005in}}{\pgfqpoint{2.438019in}{1.272829in}}%
\pgfpathcurveto{\pgfqpoint{2.438019in}{1.278653in}}{\pgfqpoint{2.435705in}{1.284239in}}{\pgfqpoint{2.431587in}{1.288357in}}%
\pgfpathcurveto{\pgfqpoint{2.427469in}{1.292476in}}{\pgfqpoint{2.421882in}{1.294789in}}{\pgfqpoint{2.416059in}{1.294789in}}%
\pgfpathcurveto{\pgfqpoint{2.410235in}{1.294789in}}{\pgfqpoint{2.404648in}{1.292476in}}{\pgfqpoint{2.400530in}{1.288357in}}%
\pgfpathcurveto{\pgfqpoint{2.396412in}{1.284239in}}{\pgfqpoint{2.394098in}{1.278653in}}{\pgfqpoint{2.394098in}{1.272829in}}%
\pgfpathcurveto{\pgfqpoint{2.394098in}{1.267005in}}{\pgfqpoint{2.396412in}{1.261419in}}{\pgfqpoint{2.400530in}{1.257301in}}%
\pgfpathcurveto{\pgfqpoint{2.404648in}{1.253183in}}{\pgfqpoint{2.410235in}{1.250869in}}{\pgfqpoint{2.416059in}{1.250869in}}%
\pgfpathclose%
\pgfusepath{stroke,fill}%
\end{pgfscope}%
\begin{pgfscope}%
\pgfpathrectangle{\pgfqpoint{0.510000in}{0.440000in}}{\pgfqpoint{3.080000in}{3.080000in}}%
\pgfusepath{clip}%
\pgfsetbuttcap%
\pgfsetroundjoin%
\definecolor{currentfill}{rgb}{0.172549,0.627451,0.172549}%
\pgfsetfillcolor{currentfill}%
\pgfsetlinewidth{1.003750pt}%
\definecolor{currentstroke}{rgb}{0.172549,0.627451,0.172549}%
\pgfsetstrokecolor{currentstroke}%
\pgfsetdash{}{0pt}%
\pgfpathmoveto{\pgfqpoint{1.566044in}{1.023108in}}%
\pgfpathcurveto{\pgfqpoint{1.571868in}{1.023108in}}{\pgfqpoint{1.577454in}{1.025422in}}{\pgfqpoint{1.581572in}{1.029540in}}%
\pgfpathcurveto{\pgfqpoint{1.585690in}{1.033658in}}{\pgfqpoint{1.588004in}{1.039245in}}{\pgfqpoint{1.588004in}{1.045068in}}%
\pgfpathcurveto{\pgfqpoint{1.588004in}{1.050892in}}{\pgfqpoint{1.585690in}{1.056479in}}{\pgfqpoint{1.581572in}{1.060597in}}%
\pgfpathcurveto{\pgfqpoint{1.577454in}{1.064715in}}{\pgfqpoint{1.571868in}{1.067029in}}{\pgfqpoint{1.566044in}{1.067029in}}%
\pgfpathcurveto{\pgfqpoint{1.560220in}{1.067029in}}{\pgfqpoint{1.554634in}{1.064715in}}{\pgfqpoint{1.550516in}{1.060597in}}%
\pgfpathcurveto{\pgfqpoint{1.546397in}{1.056479in}}{\pgfqpoint{1.544084in}{1.050892in}}{\pgfqpoint{1.544084in}{1.045068in}}%
\pgfpathcurveto{\pgfqpoint{1.544084in}{1.039245in}}{\pgfqpoint{1.546397in}{1.033658in}}{\pgfqpoint{1.550516in}{1.029540in}}%
\pgfpathcurveto{\pgfqpoint{1.554634in}{1.025422in}}{\pgfqpoint{1.560220in}{1.023108in}}{\pgfqpoint{1.566044in}{1.023108in}}%
\pgfpathclose%
\pgfusepath{stroke,fill}%
\end{pgfscope}%
\begin{pgfscope}%
\pgfpathrectangle{\pgfqpoint{0.510000in}{0.440000in}}{\pgfqpoint{3.080000in}{3.080000in}}%
\pgfusepath{clip}%
\pgfsetbuttcap%
\pgfsetroundjoin%
\definecolor{currentfill}{rgb}{0.839216,0.152941,0.156863}%
\pgfsetfillcolor{currentfill}%
\pgfsetlinewidth{1.003750pt}%
\definecolor{currentstroke}{rgb}{0.839216,0.152941,0.156863}%
\pgfsetstrokecolor{currentstroke}%
\pgfsetdash{}{0pt}%
\pgfpathmoveto{\pgfqpoint{1.115068in}{1.474084in}}%
\pgfpathcurveto{\pgfqpoint{1.120892in}{1.474084in}}{\pgfqpoint{1.126479in}{1.476397in}}{\pgfqpoint{1.130597in}{1.480516in}}%
\pgfpathcurveto{\pgfqpoint{1.134715in}{1.484634in}}{\pgfqpoint{1.137029in}{1.490220in}}{\pgfqpoint{1.137029in}{1.496044in}}%
\pgfpathcurveto{\pgfqpoint{1.137029in}{1.501868in}}{\pgfqpoint{1.134715in}{1.507454in}}{\pgfqpoint{1.130597in}{1.511572in}}%
\pgfpathcurveto{\pgfqpoint{1.126479in}{1.515690in}}{\pgfqpoint{1.120892in}{1.518004in}}{\pgfqpoint{1.115068in}{1.518004in}}%
\pgfpathcurveto{\pgfqpoint{1.109245in}{1.518004in}}{\pgfqpoint{1.103658in}{1.515690in}}{\pgfqpoint{1.099540in}{1.511572in}}%
\pgfpathcurveto{\pgfqpoint{1.095422in}{1.507454in}}{\pgfqpoint{1.093108in}{1.501868in}}{\pgfqpoint{1.093108in}{1.496044in}}%
\pgfpathcurveto{\pgfqpoint{1.093108in}{1.490220in}}{\pgfqpoint{1.095422in}{1.484634in}}{\pgfqpoint{1.099540in}{1.480516in}}%
\pgfpathcurveto{\pgfqpoint{1.103658in}{1.476397in}}{\pgfqpoint{1.109245in}{1.474084in}}{\pgfqpoint{1.115068in}{1.474084in}}%
\pgfpathclose%
\pgfusepath{stroke,fill}%
\end{pgfscope}%
\begin{pgfscope}%
\pgfpathrectangle{\pgfqpoint{0.510000in}{0.440000in}}{\pgfqpoint{3.080000in}{3.080000in}}%
\pgfusepath{clip}%
\pgfsetbuttcap%
\pgfsetmiterjoin%
\definecolor{currentfill}{rgb}{0.100000,0.100000,0.100000}%
\pgfsetfillcolor{currentfill}%
\pgfsetfillopacity{0.100000}%
\pgfsetlinewidth{0.301125pt}%
\definecolor{currentstroke}{rgb}{0.000000,0.000000,0.000000}%
\pgfsetstrokecolor{currentstroke}%
\pgfsetdash{}{0pt}%
\pgfpathmoveto{\pgfqpoint{0.950000in}{0.880000in}}%
\pgfpathlineto{\pgfqpoint{0.950000in}{3.080000in}}%
\pgfpathlineto{\pgfqpoint{3.150000in}{0.880000in}}%
\pgfpathclose%
\pgfusepath{stroke,fill}%
\end{pgfscope}%
\begin{pgfscope}%
\pgfpathrectangle{\pgfqpoint{0.510000in}{0.440000in}}{\pgfqpoint{3.080000in}{3.080000in}}%
\pgfusepath{clip}%
\pgfsetrectcap%
\pgfsetroundjoin%
\pgfsetlinewidth{0.803000pt}%
\definecolor{currentstroke}{rgb}{0.690196,0.690196,0.690196}%
\pgfsetstrokecolor{currentstroke}%
\pgfsetdash{}{0pt}%
\pgfpathmoveto{\pgfqpoint{0.510000in}{0.440000in}}%
\pgfpathlineto{\pgfqpoint{0.510000in}{3.520000in}}%
\pgfusepath{stroke}%
\end{pgfscope}%
\begin{pgfscope}%
\pgfsetbuttcap%
\pgfsetroundjoin%
\definecolor{currentfill}{rgb}{0.000000,0.000000,0.000000}%
\pgfsetfillcolor{currentfill}%
\pgfsetlinewidth{0.803000pt}%
\definecolor{currentstroke}{rgb}{0.000000,0.000000,0.000000}%
\pgfsetstrokecolor{currentstroke}%
\pgfsetdash{}{0pt}%
\pgfsys@defobject{currentmarker}{\pgfqpoint{0.000000in}{-0.048611in}}{\pgfqpoint{0.000000in}{0.000000in}}{%
\pgfpathmoveto{\pgfqpoint{0.000000in}{0.000000in}}%
\pgfpathlineto{\pgfqpoint{0.000000in}{-0.048611in}}%
\pgfusepath{stroke,fill}%
}%
\begin{pgfscope}%
\pgfsys@transformshift{0.510000in}{0.440000in}%
\pgfsys@useobject{currentmarker}{}%
\end{pgfscope}%
\end{pgfscope}%
\begin{pgfscope}%
\definecolor{textcolor}{rgb}{0.000000,0.000000,0.000000}%
\pgfsetstrokecolor{textcolor}%
\pgfsetfillcolor{textcolor}%
\pgftext[x=0.510000in,y=0.342778in,,top]{\color{textcolor}\sffamily\fontsize{10.000000}{12.000000}\selectfont -0.2}%
\end{pgfscope}%
\begin{pgfscope}%
\pgfpathrectangle{\pgfqpoint{0.510000in}{0.440000in}}{\pgfqpoint{3.080000in}{3.080000in}}%
\pgfusepath{clip}%
\pgfsetrectcap%
\pgfsetroundjoin%
\pgfsetlinewidth{0.803000pt}%
\definecolor{currentstroke}{rgb}{0.690196,0.690196,0.690196}%
\pgfsetstrokecolor{currentstroke}%
\pgfsetdash{}{0pt}%
\pgfpathmoveto{\pgfqpoint{0.950000in}{0.440000in}}%
\pgfpathlineto{\pgfqpoint{0.950000in}{3.520000in}}%
\pgfusepath{stroke}%
\end{pgfscope}%
\begin{pgfscope}%
\pgfsetbuttcap%
\pgfsetroundjoin%
\definecolor{currentfill}{rgb}{0.000000,0.000000,0.000000}%
\pgfsetfillcolor{currentfill}%
\pgfsetlinewidth{0.803000pt}%
\definecolor{currentstroke}{rgb}{0.000000,0.000000,0.000000}%
\pgfsetstrokecolor{currentstroke}%
\pgfsetdash{}{0pt}%
\pgfsys@defobject{currentmarker}{\pgfqpoint{0.000000in}{-0.048611in}}{\pgfqpoint{0.000000in}{0.000000in}}{%
\pgfpathmoveto{\pgfqpoint{0.000000in}{0.000000in}}%
\pgfpathlineto{\pgfqpoint{0.000000in}{-0.048611in}}%
\pgfusepath{stroke,fill}%
}%
\begin{pgfscope}%
\pgfsys@transformshift{0.950000in}{0.440000in}%
\pgfsys@useobject{currentmarker}{}%
\end{pgfscope}%
\end{pgfscope}%
\begin{pgfscope}%
\definecolor{textcolor}{rgb}{0.000000,0.000000,0.000000}%
\pgfsetstrokecolor{textcolor}%
\pgfsetfillcolor{textcolor}%
\pgftext[x=0.950000in,y=0.342778in,,top]{\color{textcolor}\sffamily\fontsize{10.000000}{12.000000}\selectfont 0.0}%
\end{pgfscope}%
\begin{pgfscope}%
\pgfpathrectangle{\pgfqpoint{0.510000in}{0.440000in}}{\pgfqpoint{3.080000in}{3.080000in}}%
\pgfusepath{clip}%
\pgfsetrectcap%
\pgfsetroundjoin%
\pgfsetlinewidth{0.803000pt}%
\definecolor{currentstroke}{rgb}{0.690196,0.690196,0.690196}%
\pgfsetstrokecolor{currentstroke}%
\pgfsetdash{}{0pt}%
\pgfpathmoveto{\pgfqpoint{1.390000in}{0.440000in}}%
\pgfpathlineto{\pgfqpoint{1.390000in}{3.520000in}}%
\pgfusepath{stroke}%
\end{pgfscope}%
\begin{pgfscope}%
\pgfsetbuttcap%
\pgfsetroundjoin%
\definecolor{currentfill}{rgb}{0.000000,0.000000,0.000000}%
\pgfsetfillcolor{currentfill}%
\pgfsetlinewidth{0.803000pt}%
\definecolor{currentstroke}{rgb}{0.000000,0.000000,0.000000}%
\pgfsetstrokecolor{currentstroke}%
\pgfsetdash{}{0pt}%
\pgfsys@defobject{currentmarker}{\pgfqpoint{0.000000in}{-0.048611in}}{\pgfqpoint{0.000000in}{0.000000in}}{%
\pgfpathmoveto{\pgfqpoint{0.000000in}{0.000000in}}%
\pgfpathlineto{\pgfqpoint{0.000000in}{-0.048611in}}%
\pgfusepath{stroke,fill}%
}%
\begin{pgfscope}%
\pgfsys@transformshift{1.390000in}{0.440000in}%
\pgfsys@useobject{currentmarker}{}%
\end{pgfscope}%
\end{pgfscope}%
\begin{pgfscope}%
\definecolor{textcolor}{rgb}{0.000000,0.000000,0.000000}%
\pgfsetstrokecolor{textcolor}%
\pgfsetfillcolor{textcolor}%
\pgftext[x=1.390000in,y=0.342778in,,top]{\color{textcolor}\sffamily\fontsize{10.000000}{12.000000}\selectfont 0.2}%
\end{pgfscope}%
\begin{pgfscope}%
\pgfpathrectangle{\pgfqpoint{0.510000in}{0.440000in}}{\pgfqpoint{3.080000in}{3.080000in}}%
\pgfusepath{clip}%
\pgfsetrectcap%
\pgfsetroundjoin%
\pgfsetlinewidth{0.803000pt}%
\definecolor{currentstroke}{rgb}{0.690196,0.690196,0.690196}%
\pgfsetstrokecolor{currentstroke}%
\pgfsetdash{}{0pt}%
\pgfpathmoveto{\pgfqpoint{1.830000in}{0.440000in}}%
\pgfpathlineto{\pgfqpoint{1.830000in}{3.520000in}}%
\pgfusepath{stroke}%
\end{pgfscope}%
\begin{pgfscope}%
\pgfsetbuttcap%
\pgfsetroundjoin%
\definecolor{currentfill}{rgb}{0.000000,0.000000,0.000000}%
\pgfsetfillcolor{currentfill}%
\pgfsetlinewidth{0.803000pt}%
\definecolor{currentstroke}{rgb}{0.000000,0.000000,0.000000}%
\pgfsetstrokecolor{currentstroke}%
\pgfsetdash{}{0pt}%
\pgfsys@defobject{currentmarker}{\pgfqpoint{0.000000in}{-0.048611in}}{\pgfqpoint{0.000000in}{0.000000in}}{%
\pgfpathmoveto{\pgfqpoint{0.000000in}{0.000000in}}%
\pgfpathlineto{\pgfqpoint{0.000000in}{-0.048611in}}%
\pgfusepath{stroke,fill}%
}%
\begin{pgfscope}%
\pgfsys@transformshift{1.830000in}{0.440000in}%
\pgfsys@useobject{currentmarker}{}%
\end{pgfscope}%
\end{pgfscope}%
\begin{pgfscope}%
\definecolor{textcolor}{rgb}{0.000000,0.000000,0.000000}%
\pgfsetstrokecolor{textcolor}%
\pgfsetfillcolor{textcolor}%
\pgftext[x=1.830000in,y=0.342778in,,top]{\color{textcolor}\sffamily\fontsize{10.000000}{12.000000}\selectfont 0.4}%
\end{pgfscope}%
\begin{pgfscope}%
\pgfpathrectangle{\pgfqpoint{0.510000in}{0.440000in}}{\pgfqpoint{3.080000in}{3.080000in}}%
\pgfusepath{clip}%
\pgfsetrectcap%
\pgfsetroundjoin%
\pgfsetlinewidth{0.803000pt}%
\definecolor{currentstroke}{rgb}{0.690196,0.690196,0.690196}%
\pgfsetstrokecolor{currentstroke}%
\pgfsetdash{}{0pt}%
\pgfpathmoveto{\pgfqpoint{2.270000in}{0.440000in}}%
\pgfpathlineto{\pgfqpoint{2.270000in}{3.520000in}}%
\pgfusepath{stroke}%
\end{pgfscope}%
\begin{pgfscope}%
\pgfsetbuttcap%
\pgfsetroundjoin%
\definecolor{currentfill}{rgb}{0.000000,0.000000,0.000000}%
\pgfsetfillcolor{currentfill}%
\pgfsetlinewidth{0.803000pt}%
\definecolor{currentstroke}{rgb}{0.000000,0.000000,0.000000}%
\pgfsetstrokecolor{currentstroke}%
\pgfsetdash{}{0pt}%
\pgfsys@defobject{currentmarker}{\pgfqpoint{0.000000in}{-0.048611in}}{\pgfqpoint{0.000000in}{0.000000in}}{%
\pgfpathmoveto{\pgfqpoint{0.000000in}{0.000000in}}%
\pgfpathlineto{\pgfqpoint{0.000000in}{-0.048611in}}%
\pgfusepath{stroke,fill}%
}%
\begin{pgfscope}%
\pgfsys@transformshift{2.270000in}{0.440000in}%
\pgfsys@useobject{currentmarker}{}%
\end{pgfscope}%
\end{pgfscope}%
\begin{pgfscope}%
\definecolor{textcolor}{rgb}{0.000000,0.000000,0.000000}%
\pgfsetstrokecolor{textcolor}%
\pgfsetfillcolor{textcolor}%
\pgftext[x=2.270000in,y=0.342778in,,top]{\color{textcolor}\sffamily\fontsize{10.000000}{12.000000}\selectfont 0.6}%
\end{pgfscope}%
\begin{pgfscope}%
\pgfpathrectangle{\pgfqpoint{0.510000in}{0.440000in}}{\pgfqpoint{3.080000in}{3.080000in}}%
\pgfusepath{clip}%
\pgfsetrectcap%
\pgfsetroundjoin%
\pgfsetlinewidth{0.803000pt}%
\definecolor{currentstroke}{rgb}{0.690196,0.690196,0.690196}%
\pgfsetstrokecolor{currentstroke}%
\pgfsetdash{}{0pt}%
\pgfpathmoveto{\pgfqpoint{2.710000in}{0.440000in}}%
\pgfpathlineto{\pgfqpoint{2.710000in}{3.520000in}}%
\pgfusepath{stroke}%
\end{pgfscope}%
\begin{pgfscope}%
\pgfsetbuttcap%
\pgfsetroundjoin%
\definecolor{currentfill}{rgb}{0.000000,0.000000,0.000000}%
\pgfsetfillcolor{currentfill}%
\pgfsetlinewidth{0.803000pt}%
\definecolor{currentstroke}{rgb}{0.000000,0.000000,0.000000}%
\pgfsetstrokecolor{currentstroke}%
\pgfsetdash{}{0pt}%
\pgfsys@defobject{currentmarker}{\pgfqpoint{0.000000in}{-0.048611in}}{\pgfqpoint{0.000000in}{0.000000in}}{%
\pgfpathmoveto{\pgfqpoint{0.000000in}{0.000000in}}%
\pgfpathlineto{\pgfqpoint{0.000000in}{-0.048611in}}%
\pgfusepath{stroke,fill}%
}%
\begin{pgfscope}%
\pgfsys@transformshift{2.710000in}{0.440000in}%
\pgfsys@useobject{currentmarker}{}%
\end{pgfscope}%
\end{pgfscope}%
\begin{pgfscope}%
\definecolor{textcolor}{rgb}{0.000000,0.000000,0.000000}%
\pgfsetstrokecolor{textcolor}%
\pgfsetfillcolor{textcolor}%
\pgftext[x=2.710000in,y=0.342778in,,top]{\color{textcolor}\sffamily\fontsize{10.000000}{12.000000}\selectfont 0.8}%
\end{pgfscope}%
\begin{pgfscope}%
\pgfpathrectangle{\pgfqpoint{0.510000in}{0.440000in}}{\pgfqpoint{3.080000in}{3.080000in}}%
\pgfusepath{clip}%
\pgfsetrectcap%
\pgfsetroundjoin%
\pgfsetlinewidth{0.803000pt}%
\definecolor{currentstroke}{rgb}{0.690196,0.690196,0.690196}%
\pgfsetstrokecolor{currentstroke}%
\pgfsetdash{}{0pt}%
\pgfpathmoveto{\pgfqpoint{3.150000in}{0.440000in}}%
\pgfpathlineto{\pgfqpoint{3.150000in}{3.520000in}}%
\pgfusepath{stroke}%
\end{pgfscope}%
\begin{pgfscope}%
\pgfsetbuttcap%
\pgfsetroundjoin%
\definecolor{currentfill}{rgb}{0.000000,0.000000,0.000000}%
\pgfsetfillcolor{currentfill}%
\pgfsetlinewidth{0.803000pt}%
\definecolor{currentstroke}{rgb}{0.000000,0.000000,0.000000}%
\pgfsetstrokecolor{currentstroke}%
\pgfsetdash{}{0pt}%
\pgfsys@defobject{currentmarker}{\pgfqpoint{0.000000in}{-0.048611in}}{\pgfqpoint{0.000000in}{0.000000in}}{%
\pgfpathmoveto{\pgfqpoint{0.000000in}{0.000000in}}%
\pgfpathlineto{\pgfqpoint{0.000000in}{-0.048611in}}%
\pgfusepath{stroke,fill}%
}%
\begin{pgfscope}%
\pgfsys@transformshift{3.150000in}{0.440000in}%
\pgfsys@useobject{currentmarker}{}%
\end{pgfscope}%
\end{pgfscope}%
\begin{pgfscope}%
\definecolor{textcolor}{rgb}{0.000000,0.000000,0.000000}%
\pgfsetstrokecolor{textcolor}%
\pgfsetfillcolor{textcolor}%
\pgftext[x=3.150000in,y=0.342778in,,top]{\color{textcolor}\sffamily\fontsize{10.000000}{12.000000}\selectfont 1.0}%
\end{pgfscope}%
\begin{pgfscope}%
\pgfpathrectangle{\pgfqpoint{0.510000in}{0.440000in}}{\pgfqpoint{3.080000in}{3.080000in}}%
\pgfusepath{clip}%
\pgfsetrectcap%
\pgfsetroundjoin%
\pgfsetlinewidth{0.803000pt}%
\definecolor{currentstroke}{rgb}{0.690196,0.690196,0.690196}%
\pgfsetstrokecolor{currentstroke}%
\pgfsetdash{}{0pt}%
\pgfpathmoveto{\pgfqpoint{3.590000in}{0.440000in}}%
\pgfpathlineto{\pgfqpoint{3.590000in}{3.520000in}}%
\pgfusepath{stroke}%
\end{pgfscope}%
\begin{pgfscope}%
\pgfsetbuttcap%
\pgfsetroundjoin%
\definecolor{currentfill}{rgb}{0.000000,0.000000,0.000000}%
\pgfsetfillcolor{currentfill}%
\pgfsetlinewidth{0.803000pt}%
\definecolor{currentstroke}{rgb}{0.000000,0.000000,0.000000}%
\pgfsetstrokecolor{currentstroke}%
\pgfsetdash{}{0pt}%
\pgfsys@defobject{currentmarker}{\pgfqpoint{0.000000in}{-0.048611in}}{\pgfqpoint{0.000000in}{0.000000in}}{%
\pgfpathmoveto{\pgfqpoint{0.000000in}{0.000000in}}%
\pgfpathlineto{\pgfqpoint{0.000000in}{-0.048611in}}%
\pgfusepath{stroke,fill}%
}%
\begin{pgfscope}%
\pgfsys@transformshift{3.590000in}{0.440000in}%
\pgfsys@useobject{currentmarker}{}%
\end{pgfscope}%
\end{pgfscope}%
\begin{pgfscope}%
\definecolor{textcolor}{rgb}{0.000000,0.000000,0.000000}%
\pgfsetstrokecolor{textcolor}%
\pgfsetfillcolor{textcolor}%
\pgftext[x=3.590000in,y=0.342778in,,top]{\color{textcolor}\sffamily\fontsize{10.000000}{12.000000}\selectfont 1.2}%
\end{pgfscope}%
\begin{pgfscope}%
\pgfpathrectangle{\pgfqpoint{0.510000in}{0.440000in}}{\pgfqpoint{3.080000in}{3.080000in}}%
\pgfusepath{clip}%
\pgfsetrectcap%
\pgfsetroundjoin%
\pgfsetlinewidth{0.803000pt}%
\definecolor{currentstroke}{rgb}{0.690196,0.690196,0.690196}%
\pgfsetstrokecolor{currentstroke}%
\pgfsetdash{}{0pt}%
\pgfpathmoveto{\pgfqpoint{0.510000in}{0.440000in}}%
\pgfpathlineto{\pgfqpoint{3.590000in}{0.440000in}}%
\pgfusepath{stroke}%
\end{pgfscope}%
\begin{pgfscope}%
\pgfsetbuttcap%
\pgfsetroundjoin%
\definecolor{currentfill}{rgb}{0.000000,0.000000,0.000000}%
\pgfsetfillcolor{currentfill}%
\pgfsetlinewidth{0.803000pt}%
\definecolor{currentstroke}{rgb}{0.000000,0.000000,0.000000}%
\pgfsetstrokecolor{currentstroke}%
\pgfsetdash{}{0pt}%
\pgfsys@defobject{currentmarker}{\pgfqpoint{-0.048611in}{0.000000in}}{\pgfqpoint{0.000000in}{0.000000in}}{%
\pgfpathmoveto{\pgfqpoint{0.000000in}{0.000000in}}%
\pgfpathlineto{\pgfqpoint{-0.048611in}{0.000000in}}%
\pgfusepath{stroke,fill}%
}%
\begin{pgfscope}%
\pgfsys@transformshift{0.510000in}{0.440000in}%
\pgfsys@useobject{currentmarker}{}%
\end{pgfscope}%
\end{pgfscope}%
\begin{pgfscope}%
\definecolor{textcolor}{rgb}{0.000000,0.000000,0.000000}%
\pgfsetstrokecolor{textcolor}%
\pgfsetfillcolor{textcolor}%
\pgftext[x=0.075525in,y=0.387238in,left,base]{\color{textcolor}\sffamily\fontsize{10.000000}{12.000000}\selectfont -0.2}%
\end{pgfscope}%
\begin{pgfscope}%
\pgfpathrectangle{\pgfqpoint{0.510000in}{0.440000in}}{\pgfqpoint{3.080000in}{3.080000in}}%
\pgfusepath{clip}%
\pgfsetrectcap%
\pgfsetroundjoin%
\pgfsetlinewidth{0.803000pt}%
\definecolor{currentstroke}{rgb}{0.690196,0.690196,0.690196}%
\pgfsetstrokecolor{currentstroke}%
\pgfsetdash{}{0pt}%
\pgfpathmoveto{\pgfqpoint{0.510000in}{0.880000in}}%
\pgfpathlineto{\pgfqpoint{3.590000in}{0.880000in}}%
\pgfusepath{stroke}%
\end{pgfscope}%
\begin{pgfscope}%
\pgfsetbuttcap%
\pgfsetroundjoin%
\definecolor{currentfill}{rgb}{0.000000,0.000000,0.000000}%
\pgfsetfillcolor{currentfill}%
\pgfsetlinewidth{0.803000pt}%
\definecolor{currentstroke}{rgb}{0.000000,0.000000,0.000000}%
\pgfsetstrokecolor{currentstroke}%
\pgfsetdash{}{0pt}%
\pgfsys@defobject{currentmarker}{\pgfqpoint{-0.048611in}{0.000000in}}{\pgfqpoint{0.000000in}{0.000000in}}{%
\pgfpathmoveto{\pgfqpoint{0.000000in}{0.000000in}}%
\pgfpathlineto{\pgfqpoint{-0.048611in}{0.000000in}}%
\pgfusepath{stroke,fill}%
}%
\begin{pgfscope}%
\pgfsys@transformshift{0.510000in}{0.880000in}%
\pgfsys@useobject{currentmarker}{}%
\end{pgfscope}%
\end{pgfscope}%
\begin{pgfscope}%
\definecolor{textcolor}{rgb}{0.000000,0.000000,0.000000}%
\pgfsetstrokecolor{textcolor}%
\pgfsetfillcolor{textcolor}%
\pgftext[x=0.191898in,y=0.827238in,left,base]{\color{textcolor}\sffamily\fontsize{10.000000}{12.000000}\selectfont 0.0}%
\end{pgfscope}%
\begin{pgfscope}%
\pgfpathrectangle{\pgfqpoint{0.510000in}{0.440000in}}{\pgfqpoint{3.080000in}{3.080000in}}%
\pgfusepath{clip}%
\pgfsetrectcap%
\pgfsetroundjoin%
\pgfsetlinewidth{0.803000pt}%
\definecolor{currentstroke}{rgb}{0.690196,0.690196,0.690196}%
\pgfsetstrokecolor{currentstroke}%
\pgfsetdash{}{0pt}%
\pgfpathmoveto{\pgfqpoint{0.510000in}{1.320000in}}%
\pgfpathlineto{\pgfqpoint{3.590000in}{1.320000in}}%
\pgfusepath{stroke}%
\end{pgfscope}%
\begin{pgfscope}%
\pgfsetbuttcap%
\pgfsetroundjoin%
\definecolor{currentfill}{rgb}{0.000000,0.000000,0.000000}%
\pgfsetfillcolor{currentfill}%
\pgfsetlinewidth{0.803000pt}%
\definecolor{currentstroke}{rgb}{0.000000,0.000000,0.000000}%
\pgfsetstrokecolor{currentstroke}%
\pgfsetdash{}{0pt}%
\pgfsys@defobject{currentmarker}{\pgfqpoint{-0.048611in}{0.000000in}}{\pgfqpoint{0.000000in}{0.000000in}}{%
\pgfpathmoveto{\pgfqpoint{0.000000in}{0.000000in}}%
\pgfpathlineto{\pgfqpoint{-0.048611in}{0.000000in}}%
\pgfusepath{stroke,fill}%
}%
\begin{pgfscope}%
\pgfsys@transformshift{0.510000in}{1.320000in}%
\pgfsys@useobject{currentmarker}{}%
\end{pgfscope}%
\end{pgfscope}%
\begin{pgfscope}%
\definecolor{textcolor}{rgb}{0.000000,0.000000,0.000000}%
\pgfsetstrokecolor{textcolor}%
\pgfsetfillcolor{textcolor}%
\pgftext[x=0.191898in,y=1.267238in,left,base]{\color{textcolor}\sffamily\fontsize{10.000000}{12.000000}\selectfont 0.2}%
\end{pgfscope}%
\begin{pgfscope}%
\pgfpathrectangle{\pgfqpoint{0.510000in}{0.440000in}}{\pgfqpoint{3.080000in}{3.080000in}}%
\pgfusepath{clip}%
\pgfsetrectcap%
\pgfsetroundjoin%
\pgfsetlinewidth{0.803000pt}%
\definecolor{currentstroke}{rgb}{0.690196,0.690196,0.690196}%
\pgfsetstrokecolor{currentstroke}%
\pgfsetdash{}{0pt}%
\pgfpathmoveto{\pgfqpoint{0.510000in}{1.760000in}}%
\pgfpathlineto{\pgfqpoint{3.590000in}{1.760000in}}%
\pgfusepath{stroke}%
\end{pgfscope}%
\begin{pgfscope}%
\pgfsetbuttcap%
\pgfsetroundjoin%
\definecolor{currentfill}{rgb}{0.000000,0.000000,0.000000}%
\pgfsetfillcolor{currentfill}%
\pgfsetlinewidth{0.803000pt}%
\definecolor{currentstroke}{rgb}{0.000000,0.000000,0.000000}%
\pgfsetstrokecolor{currentstroke}%
\pgfsetdash{}{0pt}%
\pgfsys@defobject{currentmarker}{\pgfqpoint{-0.048611in}{0.000000in}}{\pgfqpoint{0.000000in}{0.000000in}}{%
\pgfpathmoveto{\pgfqpoint{0.000000in}{0.000000in}}%
\pgfpathlineto{\pgfqpoint{-0.048611in}{0.000000in}}%
\pgfusepath{stroke,fill}%
}%
\begin{pgfscope}%
\pgfsys@transformshift{0.510000in}{1.760000in}%
\pgfsys@useobject{currentmarker}{}%
\end{pgfscope}%
\end{pgfscope}%
\begin{pgfscope}%
\definecolor{textcolor}{rgb}{0.000000,0.000000,0.000000}%
\pgfsetstrokecolor{textcolor}%
\pgfsetfillcolor{textcolor}%
\pgftext[x=0.191898in,y=1.707238in,left,base]{\color{textcolor}\sffamily\fontsize{10.000000}{12.000000}\selectfont 0.4}%
\end{pgfscope}%
\begin{pgfscope}%
\pgfpathrectangle{\pgfqpoint{0.510000in}{0.440000in}}{\pgfqpoint{3.080000in}{3.080000in}}%
\pgfusepath{clip}%
\pgfsetrectcap%
\pgfsetroundjoin%
\pgfsetlinewidth{0.803000pt}%
\definecolor{currentstroke}{rgb}{0.690196,0.690196,0.690196}%
\pgfsetstrokecolor{currentstroke}%
\pgfsetdash{}{0pt}%
\pgfpathmoveto{\pgfqpoint{0.510000in}{2.200000in}}%
\pgfpathlineto{\pgfqpoint{3.590000in}{2.200000in}}%
\pgfusepath{stroke}%
\end{pgfscope}%
\begin{pgfscope}%
\pgfsetbuttcap%
\pgfsetroundjoin%
\definecolor{currentfill}{rgb}{0.000000,0.000000,0.000000}%
\pgfsetfillcolor{currentfill}%
\pgfsetlinewidth{0.803000pt}%
\definecolor{currentstroke}{rgb}{0.000000,0.000000,0.000000}%
\pgfsetstrokecolor{currentstroke}%
\pgfsetdash{}{0pt}%
\pgfsys@defobject{currentmarker}{\pgfqpoint{-0.048611in}{0.000000in}}{\pgfqpoint{0.000000in}{0.000000in}}{%
\pgfpathmoveto{\pgfqpoint{0.000000in}{0.000000in}}%
\pgfpathlineto{\pgfqpoint{-0.048611in}{0.000000in}}%
\pgfusepath{stroke,fill}%
}%
\begin{pgfscope}%
\pgfsys@transformshift{0.510000in}{2.200000in}%
\pgfsys@useobject{currentmarker}{}%
\end{pgfscope}%
\end{pgfscope}%
\begin{pgfscope}%
\definecolor{textcolor}{rgb}{0.000000,0.000000,0.000000}%
\pgfsetstrokecolor{textcolor}%
\pgfsetfillcolor{textcolor}%
\pgftext[x=0.191898in,y=2.147238in,left,base]{\color{textcolor}\sffamily\fontsize{10.000000}{12.000000}\selectfont 0.6}%
\end{pgfscope}%
\begin{pgfscope}%
\pgfpathrectangle{\pgfqpoint{0.510000in}{0.440000in}}{\pgfqpoint{3.080000in}{3.080000in}}%
\pgfusepath{clip}%
\pgfsetrectcap%
\pgfsetroundjoin%
\pgfsetlinewidth{0.803000pt}%
\definecolor{currentstroke}{rgb}{0.690196,0.690196,0.690196}%
\pgfsetstrokecolor{currentstroke}%
\pgfsetdash{}{0pt}%
\pgfpathmoveto{\pgfqpoint{0.510000in}{2.640000in}}%
\pgfpathlineto{\pgfqpoint{3.590000in}{2.640000in}}%
\pgfusepath{stroke}%
\end{pgfscope}%
\begin{pgfscope}%
\pgfsetbuttcap%
\pgfsetroundjoin%
\definecolor{currentfill}{rgb}{0.000000,0.000000,0.000000}%
\pgfsetfillcolor{currentfill}%
\pgfsetlinewidth{0.803000pt}%
\definecolor{currentstroke}{rgb}{0.000000,0.000000,0.000000}%
\pgfsetstrokecolor{currentstroke}%
\pgfsetdash{}{0pt}%
\pgfsys@defobject{currentmarker}{\pgfqpoint{-0.048611in}{0.000000in}}{\pgfqpoint{0.000000in}{0.000000in}}{%
\pgfpathmoveto{\pgfqpoint{0.000000in}{0.000000in}}%
\pgfpathlineto{\pgfqpoint{-0.048611in}{0.000000in}}%
\pgfusepath{stroke,fill}%
}%
\begin{pgfscope}%
\pgfsys@transformshift{0.510000in}{2.640000in}%
\pgfsys@useobject{currentmarker}{}%
\end{pgfscope}%
\end{pgfscope}%
\begin{pgfscope}%
\definecolor{textcolor}{rgb}{0.000000,0.000000,0.000000}%
\pgfsetstrokecolor{textcolor}%
\pgfsetfillcolor{textcolor}%
\pgftext[x=0.191898in,y=2.587238in,left,base]{\color{textcolor}\sffamily\fontsize{10.000000}{12.000000}\selectfont 0.8}%
\end{pgfscope}%
\begin{pgfscope}%
\pgfpathrectangle{\pgfqpoint{0.510000in}{0.440000in}}{\pgfqpoint{3.080000in}{3.080000in}}%
\pgfusepath{clip}%
\pgfsetrectcap%
\pgfsetroundjoin%
\pgfsetlinewidth{0.803000pt}%
\definecolor{currentstroke}{rgb}{0.690196,0.690196,0.690196}%
\pgfsetstrokecolor{currentstroke}%
\pgfsetdash{}{0pt}%
\pgfpathmoveto{\pgfqpoint{0.510000in}{3.080000in}}%
\pgfpathlineto{\pgfqpoint{3.590000in}{3.080000in}}%
\pgfusepath{stroke}%
\end{pgfscope}%
\begin{pgfscope}%
\pgfsetbuttcap%
\pgfsetroundjoin%
\definecolor{currentfill}{rgb}{0.000000,0.000000,0.000000}%
\pgfsetfillcolor{currentfill}%
\pgfsetlinewidth{0.803000pt}%
\definecolor{currentstroke}{rgb}{0.000000,0.000000,0.000000}%
\pgfsetstrokecolor{currentstroke}%
\pgfsetdash{}{0pt}%
\pgfsys@defobject{currentmarker}{\pgfqpoint{-0.048611in}{0.000000in}}{\pgfqpoint{0.000000in}{0.000000in}}{%
\pgfpathmoveto{\pgfqpoint{0.000000in}{0.000000in}}%
\pgfpathlineto{\pgfqpoint{-0.048611in}{0.000000in}}%
\pgfusepath{stroke,fill}%
}%
\begin{pgfscope}%
\pgfsys@transformshift{0.510000in}{3.080000in}%
\pgfsys@useobject{currentmarker}{}%
\end{pgfscope}%
\end{pgfscope}%
\begin{pgfscope}%
\definecolor{textcolor}{rgb}{0.000000,0.000000,0.000000}%
\pgfsetstrokecolor{textcolor}%
\pgfsetfillcolor{textcolor}%
\pgftext[x=0.191898in,y=3.027238in,left,base]{\color{textcolor}\sffamily\fontsize{10.000000}{12.000000}\selectfont 1.0}%
\end{pgfscope}%
\begin{pgfscope}%
\pgfpathrectangle{\pgfqpoint{0.510000in}{0.440000in}}{\pgfqpoint{3.080000in}{3.080000in}}%
\pgfusepath{clip}%
\pgfsetrectcap%
\pgfsetroundjoin%
\pgfsetlinewidth{0.803000pt}%
\definecolor{currentstroke}{rgb}{0.690196,0.690196,0.690196}%
\pgfsetstrokecolor{currentstroke}%
\pgfsetdash{}{0pt}%
\pgfpathmoveto{\pgfqpoint{0.510000in}{3.520000in}}%
\pgfpathlineto{\pgfqpoint{3.590000in}{3.520000in}}%
\pgfusepath{stroke}%
\end{pgfscope}%
\begin{pgfscope}%
\pgfsetbuttcap%
\pgfsetroundjoin%
\definecolor{currentfill}{rgb}{0.000000,0.000000,0.000000}%
\pgfsetfillcolor{currentfill}%
\pgfsetlinewidth{0.803000pt}%
\definecolor{currentstroke}{rgb}{0.000000,0.000000,0.000000}%
\pgfsetstrokecolor{currentstroke}%
\pgfsetdash{}{0pt}%
\pgfsys@defobject{currentmarker}{\pgfqpoint{-0.048611in}{0.000000in}}{\pgfqpoint{0.000000in}{0.000000in}}{%
\pgfpathmoveto{\pgfqpoint{0.000000in}{0.000000in}}%
\pgfpathlineto{\pgfqpoint{-0.048611in}{0.000000in}}%
\pgfusepath{stroke,fill}%
}%
\begin{pgfscope}%
\pgfsys@transformshift{0.510000in}{3.520000in}%
\pgfsys@useobject{currentmarker}{}%
\end{pgfscope}%
\end{pgfscope}%
\begin{pgfscope}%
\definecolor{textcolor}{rgb}{0.000000,0.000000,0.000000}%
\pgfsetstrokecolor{textcolor}%
\pgfsetfillcolor{textcolor}%
\pgftext[x=0.191898in,y=3.467238in,left,base]{\color{textcolor}\sffamily\fontsize{10.000000}{12.000000}\selectfont 1.2}%
\end{pgfscope}%
\begin{pgfscope}%
\pgfsetrectcap%
\pgfsetmiterjoin%
\pgfsetlinewidth{0.803000pt}%
\definecolor{currentstroke}{rgb}{0.000000,0.000000,0.000000}%
\pgfsetstrokecolor{currentstroke}%
\pgfsetdash{}{0pt}%
\pgfpathmoveto{\pgfqpoint{0.510000in}{0.440000in}}%
\pgfpathlineto{\pgfqpoint{0.510000in}{3.520000in}}%
\pgfusepath{stroke}%
\end{pgfscope}%
\begin{pgfscope}%
\pgfsetrectcap%
\pgfsetmiterjoin%
\pgfsetlinewidth{0.803000pt}%
\definecolor{currentstroke}{rgb}{0.000000,0.000000,0.000000}%
\pgfsetstrokecolor{currentstroke}%
\pgfsetdash{}{0pt}%
\pgfpathmoveto{\pgfqpoint{3.590000in}{0.440000in}}%
\pgfpathlineto{\pgfqpoint{3.590000in}{3.520000in}}%
\pgfusepath{stroke}%
\end{pgfscope}%
\begin{pgfscope}%
\pgfsetrectcap%
\pgfsetmiterjoin%
\pgfsetlinewidth{0.803000pt}%
\definecolor{currentstroke}{rgb}{0.000000,0.000000,0.000000}%
\pgfsetstrokecolor{currentstroke}%
\pgfsetdash{}{0pt}%
\pgfpathmoveto{\pgfqpoint{0.510000in}{0.440000in}}%
\pgfpathlineto{\pgfqpoint{3.590000in}{0.440000in}}%
\pgfusepath{stroke}%
\end{pgfscope}%
\begin{pgfscope}%
\pgfsetrectcap%
\pgfsetmiterjoin%
\pgfsetlinewidth{0.803000pt}%
\definecolor{currentstroke}{rgb}{0.000000,0.000000,0.000000}%
\pgfsetstrokecolor{currentstroke}%
\pgfsetdash{}{0pt}%
\pgfpathmoveto{\pgfqpoint{0.510000in}{3.520000in}}%
\pgfpathlineto{\pgfqpoint{3.590000in}{3.520000in}}%
\pgfusepath{stroke}%
\end{pgfscope}%
\begin{pgfscope}%
\definecolor{textcolor}{rgb}{0.000000,0.000000,0.000000}%
\pgfsetstrokecolor{textcolor}%
\pgfsetfillcolor{textcolor}%
\pgftext[x=1.342829in,y=2.346059in,left,base]{\color{textcolor}\sffamily\fontsize{10.000000}{12.000000}\selectfont 0.15902}%
\end{pgfscope}%
\begin{pgfscope}%
\definecolor{textcolor}{rgb}{0.000000,0.000000,0.000000}%
\pgfsetstrokecolor{textcolor}%
\pgfsetfillcolor{textcolor}%
\pgftext[x=2.416059in,y=1.272829in,left,base]{\color{textcolor}\sffamily\fontsize{10.000000}{12.000000}\selectfont 0.15902}%
\end{pgfscope}%
\begin{pgfscope}%
\definecolor{textcolor}{rgb}{0.000000,0.000000,0.000000}%
\pgfsetstrokecolor{textcolor}%
\pgfsetfillcolor{textcolor}%
\pgftext[x=1.566044in,y=1.045068in,left,base]{\color{textcolor}\sffamily\fontsize{10.000000}{12.000000}\selectfont 0.09098}%
\end{pgfscope}%
\begin{pgfscope}%
\definecolor{textcolor}{rgb}{0.000000,0.000000,0.000000}%
\pgfsetstrokecolor{textcolor}%
\pgfsetfillcolor{textcolor}%
\pgftext[x=1.115068in,y=1.496044in,left,base]{\color{textcolor}\sffamily\fontsize{10.000000}{12.000000}\selectfont 0.09098}%
\end{pgfscope}%
\end{pgfpicture}%
\makeatother%
\endgroup%
}
\caption{Magnetization $m$ for different temperatures $T$ using Metropolis--Hastings algorithm}
\label{Fig:MagSmall}
\end{figure}

We can see from the figures that the internal energy $u$ has a steepest growth at about $ T = 2.3 $ and the specific heat has a sharp maximum point at the same temperature, which indicates the phase transition. Combined with Figure \ref{Fig:Sites}, this indicates that there is kind of phase transition near this temperature. (Convergence issues accounts for the failure of $ N = 128 $ at low temperatures.) As a result, we zoom in and perform numerical experiments in the interval $ \sbr{ 2.22, 2.32 } $. The numerical result is shown in Figure \ref{Fig:HeatBig}, \ref{Fig:CapBig} and \ref{Fig:MagBig}.

\begin{figure}[htbp]
\centering
\scalebox{0.666}{%% Creator: Matplotlib, PGF backend
%%
%% To include the figure in your LaTeX document, write
%%   \input{<filename>.pgf}
%%
%% Make sure the required packages are loaded in your preamble
%%   \usepackage{pgf}
%%
%% Figures using additional raster images can only be included by \input if
%% they are in the same directory as the main LaTeX file. For loading figures
%% from other directories you can use the `import` package
%%   \usepackage{import}
%% and then include the figures with
%%   \import{<path to file>}{<filename>.pgf}
%%
%% Matplotlib used the following preamble
%%   \usepackage{fontspec}
%%   \setmainfont{DejaVuSerif.ttf}[Path=/home/lzh/anaconda3/envs/numana/lib/python3.7/site-packages/matplotlib/mpl-data/fonts/ttf/]
%%   \setsansfont{DejaVuSans.ttf}[Path=/home/lzh/anaconda3/envs/numana/lib/python3.7/site-packages/matplotlib/mpl-data/fonts/ttf/]
%%   \setmonofont{DejaVuSansMono.ttf}[Path=/home/lzh/anaconda3/envs/numana/lib/python3.7/site-packages/matplotlib/mpl-data/fonts/ttf/]
%%
\begingroup%
\makeatletter%
\begin{pgfpicture}%
\pgfpathrectangle{\pgfpointorigin}{\pgfqpoint{8.000000in}{6.000000in}}%
\pgfusepath{use as bounding box, clip}%
\begin{pgfscope}%
\pgfsetbuttcap%
\pgfsetmiterjoin%
\definecolor{currentfill}{rgb}{1.000000,1.000000,1.000000}%
\pgfsetfillcolor{currentfill}%
\pgfsetlinewidth{0.000000pt}%
\definecolor{currentstroke}{rgb}{1.000000,1.000000,1.000000}%
\pgfsetstrokecolor{currentstroke}%
\pgfsetdash{}{0pt}%
\pgfpathmoveto{\pgfqpoint{0.000000in}{0.000000in}}%
\pgfpathlineto{\pgfqpoint{8.000000in}{0.000000in}}%
\pgfpathlineto{\pgfqpoint{8.000000in}{6.000000in}}%
\pgfpathlineto{\pgfqpoint{0.000000in}{6.000000in}}%
\pgfpathclose%
\pgfusepath{fill}%
\end{pgfscope}%
\begin{pgfscope}%
\pgfsetbuttcap%
\pgfsetmiterjoin%
\definecolor{currentfill}{rgb}{1.000000,1.000000,1.000000}%
\pgfsetfillcolor{currentfill}%
\pgfsetlinewidth{0.000000pt}%
\definecolor{currentstroke}{rgb}{0.000000,0.000000,0.000000}%
\pgfsetstrokecolor{currentstroke}%
\pgfsetstrokeopacity{0.000000}%
\pgfsetdash{}{0pt}%
\pgfpathmoveto{\pgfqpoint{0.672222in}{3.312222in}}%
\pgfpathlineto{\pgfqpoint{3.878555in}{3.312222in}}%
\pgfpathlineto{\pgfqpoint{3.878555in}{5.636667in}}%
\pgfpathlineto{\pgfqpoint{0.672222in}{5.636667in}}%
\pgfpathclose%
\pgfusepath{fill}%
\end{pgfscope}%
\begin{pgfscope}%
\pgfpathrectangle{\pgfqpoint{0.672222in}{3.312222in}}{\pgfqpoint{3.206333in}{2.324444in}}%
\pgfusepath{clip}%
\pgfsetbuttcap%
\pgfsetroundjoin%
\definecolor{currentfill}{rgb}{0.121569,0.466667,0.705882}%
\pgfsetfillcolor{currentfill}%
\pgfsetfillopacity{0.300000}%
\pgfsetlinewidth{1.003750pt}%
\definecolor{currentstroke}{rgb}{0.121569,0.466667,0.705882}%
\pgfsetstrokecolor{currentstroke}%
\pgfsetstrokeopacity{0.300000}%
\pgfsetdash{}{0pt}%
\pgfpathmoveto{\pgfqpoint{0.825965in}{3.437041in}}%
\pgfpathlineto{\pgfqpoint{0.825965in}{3.417879in}}%
\pgfpathlineto{\pgfqpoint{0.970908in}{3.500209in}}%
\pgfpathlineto{\pgfqpoint{1.115850in}{3.607983in}}%
\pgfpathlineto{\pgfqpoint{1.260792in}{3.702210in}}%
\pgfpathlineto{\pgfqpoint{1.405735in}{3.817192in}}%
\pgfpathlineto{\pgfqpoint{1.550677in}{3.912772in}}%
\pgfpathlineto{\pgfqpoint{1.695619in}{4.015408in}}%
\pgfpathlineto{\pgfqpoint{1.840562in}{4.132399in}}%
\pgfpathlineto{\pgfqpoint{1.985504in}{4.255669in}}%
\pgfpathlineto{\pgfqpoint{2.130446in}{4.349757in}}%
\pgfpathlineto{\pgfqpoint{2.275389in}{4.467654in}}%
\pgfpathlineto{\pgfqpoint{2.420331in}{4.563005in}}%
\pgfpathlineto{\pgfqpoint{2.565273in}{4.675889in}}%
\pgfpathlineto{\pgfqpoint{2.710216in}{4.795109in}}%
\pgfpathlineto{\pgfqpoint{2.855158in}{4.902481in}}%
\pgfpathlineto{\pgfqpoint{3.000100in}{5.013797in}}%
\pgfpathlineto{\pgfqpoint{3.145043in}{5.122730in}}%
\pgfpathlineto{\pgfqpoint{3.289985in}{5.220071in}}%
\pgfpathlineto{\pgfqpoint{3.434927in}{5.326158in}}%
\pgfpathlineto{\pgfqpoint{3.579870in}{5.415819in}}%
\pgfpathlineto{\pgfqpoint{3.724812in}{5.512354in}}%
\pgfpathlineto{\pgfqpoint{3.724812in}{5.531010in}}%
\pgfpathlineto{\pgfqpoint{3.724812in}{5.531010in}}%
\pgfpathlineto{\pgfqpoint{3.579870in}{5.447506in}}%
\pgfpathlineto{\pgfqpoint{3.434927in}{5.356013in}}%
\pgfpathlineto{\pgfqpoint{3.289985in}{5.226820in}}%
\pgfpathlineto{\pgfqpoint{3.145043in}{5.142404in}}%
\pgfpathlineto{\pgfqpoint{3.000100in}{5.025410in}}%
\pgfpathlineto{\pgfqpoint{2.855158in}{4.910163in}}%
\pgfpathlineto{\pgfqpoint{2.710216in}{4.800566in}}%
\pgfpathlineto{\pgfqpoint{2.565273in}{4.707840in}}%
\pgfpathlineto{\pgfqpoint{2.420331in}{4.598021in}}%
\pgfpathlineto{\pgfqpoint{2.275389in}{4.485465in}}%
\pgfpathlineto{\pgfqpoint{2.130446in}{4.375618in}}%
\pgfpathlineto{\pgfqpoint{1.985504in}{4.265316in}}%
\pgfpathlineto{\pgfqpoint{1.840562in}{4.163644in}}%
\pgfpathlineto{\pgfqpoint{1.695619in}{4.036108in}}%
\pgfpathlineto{\pgfqpoint{1.550677in}{3.937898in}}%
\pgfpathlineto{\pgfqpoint{1.405735in}{3.840984in}}%
\pgfpathlineto{\pgfqpoint{1.260792in}{3.725046in}}%
\pgfpathlineto{\pgfqpoint{1.115850in}{3.617685in}}%
\pgfpathlineto{\pgfqpoint{0.970908in}{3.543278in}}%
\pgfpathlineto{\pgfqpoint{0.825965in}{3.437041in}}%
\pgfpathclose%
\pgfusepath{stroke,fill}%
\end{pgfscope}%
\begin{pgfscope}%
\pgfpathrectangle{\pgfqpoint{0.672222in}{3.312222in}}{\pgfqpoint{3.206333in}{2.324444in}}%
\pgfusepath{clip}%
\pgfsetbuttcap%
\pgfsetroundjoin%
\definecolor{currentfill}{rgb}{0.121569,0.466667,0.705882}%
\pgfsetfillcolor{currentfill}%
\pgfsetlinewidth{1.003750pt}%
\definecolor{currentstroke}{rgb}{0.121569,0.466667,0.705882}%
\pgfsetstrokecolor{currentstroke}%
\pgfsetdash{}{0pt}%
\pgfsys@defobject{currentmarker}{\pgfqpoint{-0.009821in}{-0.009821in}}{\pgfqpoint{0.009821in}{0.009821in}}{%
\pgfpathmoveto{\pgfqpoint{0.000000in}{-0.009821in}}%
\pgfpathcurveto{\pgfqpoint{0.002605in}{-0.009821in}}{\pgfqpoint{0.005103in}{-0.008786in}}{\pgfqpoint{0.006944in}{-0.006944in}}%
\pgfpathcurveto{\pgfqpoint{0.008786in}{-0.005103in}}{\pgfqpoint{0.009821in}{-0.002605in}}{\pgfqpoint{0.009821in}{0.000000in}}%
\pgfpathcurveto{\pgfqpoint{0.009821in}{0.002605in}}{\pgfqpoint{0.008786in}{0.005103in}}{\pgfqpoint{0.006944in}{0.006944in}}%
\pgfpathcurveto{\pgfqpoint{0.005103in}{0.008786in}}{\pgfqpoint{0.002605in}{0.009821in}}{\pgfqpoint{0.000000in}{0.009821in}}%
\pgfpathcurveto{\pgfqpoint{-0.002605in}{0.009821in}}{\pgfqpoint{-0.005103in}{0.008786in}}{\pgfqpoint{-0.006944in}{0.006944in}}%
\pgfpathcurveto{\pgfqpoint{-0.008786in}{0.005103in}}{\pgfqpoint{-0.009821in}{0.002605in}}{\pgfqpoint{-0.009821in}{0.000000in}}%
\pgfpathcurveto{\pgfqpoint{-0.009821in}{-0.002605in}}{\pgfqpoint{-0.008786in}{-0.005103in}}{\pgfqpoint{-0.006944in}{-0.006944in}}%
\pgfpathcurveto{\pgfqpoint{-0.005103in}{-0.008786in}}{\pgfqpoint{-0.002605in}{-0.009821in}}{\pgfqpoint{0.000000in}{-0.009821in}}%
\pgfpathclose%
\pgfusepath{stroke,fill}%
}%
\begin{pgfscope}%
\pgfsys@transformshift{0.825965in}{3.427460in}%
\pgfsys@useobject{currentmarker}{}%
\end{pgfscope}%
\begin{pgfscope}%
\pgfsys@transformshift{0.970908in}{3.521743in}%
\pgfsys@useobject{currentmarker}{}%
\end{pgfscope}%
\begin{pgfscope}%
\pgfsys@transformshift{1.115850in}{3.612834in}%
\pgfsys@useobject{currentmarker}{}%
\end{pgfscope}%
\begin{pgfscope}%
\pgfsys@transformshift{1.260792in}{3.713628in}%
\pgfsys@useobject{currentmarker}{}%
\end{pgfscope}%
\begin{pgfscope}%
\pgfsys@transformshift{1.405735in}{3.829088in}%
\pgfsys@useobject{currentmarker}{}%
\end{pgfscope}%
\begin{pgfscope}%
\pgfsys@transformshift{1.550677in}{3.925335in}%
\pgfsys@useobject{currentmarker}{}%
\end{pgfscope}%
\begin{pgfscope}%
\pgfsys@transformshift{1.695619in}{4.025758in}%
\pgfsys@useobject{currentmarker}{}%
\end{pgfscope}%
\begin{pgfscope}%
\pgfsys@transformshift{1.840562in}{4.148021in}%
\pgfsys@useobject{currentmarker}{}%
\end{pgfscope}%
\begin{pgfscope}%
\pgfsys@transformshift{1.985504in}{4.260493in}%
\pgfsys@useobject{currentmarker}{}%
\end{pgfscope}%
\begin{pgfscope}%
\pgfsys@transformshift{2.130446in}{4.362687in}%
\pgfsys@useobject{currentmarker}{}%
\end{pgfscope}%
\begin{pgfscope}%
\pgfsys@transformshift{2.275389in}{4.476560in}%
\pgfsys@useobject{currentmarker}{}%
\end{pgfscope}%
\begin{pgfscope}%
\pgfsys@transformshift{2.420331in}{4.580513in}%
\pgfsys@useobject{currentmarker}{}%
\end{pgfscope}%
\begin{pgfscope}%
\pgfsys@transformshift{2.565273in}{4.691865in}%
\pgfsys@useobject{currentmarker}{}%
\end{pgfscope}%
\begin{pgfscope}%
\pgfsys@transformshift{2.710216in}{4.797837in}%
\pgfsys@useobject{currentmarker}{}%
\end{pgfscope}%
\begin{pgfscope}%
\pgfsys@transformshift{2.855158in}{4.906322in}%
\pgfsys@useobject{currentmarker}{}%
\end{pgfscope}%
\begin{pgfscope}%
\pgfsys@transformshift{3.000100in}{5.019604in}%
\pgfsys@useobject{currentmarker}{}%
\end{pgfscope}%
\begin{pgfscope}%
\pgfsys@transformshift{3.145043in}{5.132567in}%
\pgfsys@useobject{currentmarker}{}%
\end{pgfscope}%
\begin{pgfscope}%
\pgfsys@transformshift{3.289985in}{5.223446in}%
\pgfsys@useobject{currentmarker}{}%
\end{pgfscope}%
\begin{pgfscope}%
\pgfsys@transformshift{3.434927in}{5.341086in}%
\pgfsys@useobject{currentmarker}{}%
\end{pgfscope}%
\begin{pgfscope}%
\pgfsys@transformshift{3.579870in}{5.431662in}%
\pgfsys@useobject{currentmarker}{}%
\end{pgfscope}%
\begin{pgfscope}%
\pgfsys@transformshift{3.724812in}{5.521682in}%
\pgfsys@useobject{currentmarker}{}%
\end{pgfscope}%
\end{pgfscope}%
\begin{pgfscope}%
\pgfsetbuttcap%
\pgfsetroundjoin%
\definecolor{currentfill}{rgb}{0.000000,0.000000,0.000000}%
\pgfsetfillcolor{currentfill}%
\pgfsetlinewidth{0.803000pt}%
\definecolor{currentstroke}{rgb}{0.000000,0.000000,0.000000}%
\pgfsetstrokecolor{currentstroke}%
\pgfsetdash{}{0pt}%
\pgfsys@defobject{currentmarker}{\pgfqpoint{0.000000in}{-0.048611in}}{\pgfqpoint{0.000000in}{0.000000in}}{%
\pgfpathmoveto{\pgfqpoint{0.000000in}{0.000000in}}%
\pgfpathlineto{\pgfqpoint{0.000000in}{-0.048611in}}%
\pgfusepath{stroke,fill}%
}%
\begin{pgfscope}%
\pgfsys@transformshift{1.260792in}{3.312222in}%
\pgfsys@useobject{currentmarker}{}%
\end{pgfscope}%
\end{pgfscope}%
\begin{pgfscope}%
\definecolor{textcolor}{rgb}{0.000000,0.000000,0.000000}%
\pgfsetstrokecolor{textcolor}%
\pgfsetfillcolor{textcolor}%
\pgftext[x=1.260792in,y=3.215000in,,top]{\color{textcolor}\sffamily\fontsize{10.000000}{12.000000}\selectfont 2.25}%
\end{pgfscope}%
\begin{pgfscope}%
\pgfsetbuttcap%
\pgfsetroundjoin%
\definecolor{currentfill}{rgb}{0.000000,0.000000,0.000000}%
\pgfsetfillcolor{currentfill}%
\pgfsetlinewidth{0.803000pt}%
\definecolor{currentstroke}{rgb}{0.000000,0.000000,0.000000}%
\pgfsetstrokecolor{currentstroke}%
\pgfsetdash{}{0pt}%
\pgfsys@defobject{currentmarker}{\pgfqpoint{0.000000in}{-0.048611in}}{\pgfqpoint{0.000000in}{0.000000in}}{%
\pgfpathmoveto{\pgfqpoint{0.000000in}{0.000000in}}%
\pgfpathlineto{\pgfqpoint{0.000000in}{-0.048611in}}%
\pgfusepath{stroke,fill}%
}%
\begin{pgfscope}%
\pgfsys@transformshift{1.985504in}{3.312222in}%
\pgfsys@useobject{currentmarker}{}%
\end{pgfscope}%
\end{pgfscope}%
\begin{pgfscope}%
\definecolor{textcolor}{rgb}{0.000000,0.000000,0.000000}%
\pgfsetstrokecolor{textcolor}%
\pgfsetfillcolor{textcolor}%
\pgftext[x=1.985504in,y=3.215000in,,top]{\color{textcolor}\sffamily\fontsize{10.000000}{12.000000}\selectfont 2.30}%
\end{pgfscope}%
\begin{pgfscope}%
\pgfsetbuttcap%
\pgfsetroundjoin%
\definecolor{currentfill}{rgb}{0.000000,0.000000,0.000000}%
\pgfsetfillcolor{currentfill}%
\pgfsetlinewidth{0.803000pt}%
\definecolor{currentstroke}{rgb}{0.000000,0.000000,0.000000}%
\pgfsetstrokecolor{currentstroke}%
\pgfsetdash{}{0pt}%
\pgfsys@defobject{currentmarker}{\pgfqpoint{0.000000in}{-0.048611in}}{\pgfqpoint{0.000000in}{0.000000in}}{%
\pgfpathmoveto{\pgfqpoint{0.000000in}{0.000000in}}%
\pgfpathlineto{\pgfqpoint{0.000000in}{-0.048611in}}%
\pgfusepath{stroke,fill}%
}%
\begin{pgfscope}%
\pgfsys@transformshift{2.710216in}{3.312222in}%
\pgfsys@useobject{currentmarker}{}%
\end{pgfscope}%
\end{pgfscope}%
\begin{pgfscope}%
\definecolor{textcolor}{rgb}{0.000000,0.000000,0.000000}%
\pgfsetstrokecolor{textcolor}%
\pgfsetfillcolor{textcolor}%
\pgftext[x=2.710216in,y=3.215000in,,top]{\color{textcolor}\sffamily\fontsize{10.000000}{12.000000}\selectfont 2.35}%
\end{pgfscope}%
\begin{pgfscope}%
\pgfsetbuttcap%
\pgfsetroundjoin%
\definecolor{currentfill}{rgb}{0.000000,0.000000,0.000000}%
\pgfsetfillcolor{currentfill}%
\pgfsetlinewidth{0.803000pt}%
\definecolor{currentstroke}{rgb}{0.000000,0.000000,0.000000}%
\pgfsetstrokecolor{currentstroke}%
\pgfsetdash{}{0pt}%
\pgfsys@defobject{currentmarker}{\pgfqpoint{0.000000in}{-0.048611in}}{\pgfqpoint{0.000000in}{0.000000in}}{%
\pgfpathmoveto{\pgfqpoint{0.000000in}{0.000000in}}%
\pgfpathlineto{\pgfqpoint{0.000000in}{-0.048611in}}%
\pgfusepath{stroke,fill}%
}%
\begin{pgfscope}%
\pgfsys@transformshift{3.434927in}{3.312222in}%
\pgfsys@useobject{currentmarker}{}%
\end{pgfscope}%
\end{pgfscope}%
\begin{pgfscope}%
\definecolor{textcolor}{rgb}{0.000000,0.000000,0.000000}%
\pgfsetstrokecolor{textcolor}%
\pgfsetfillcolor{textcolor}%
\pgftext[x=3.434927in,y=3.215000in,,top]{\color{textcolor}\sffamily\fontsize{10.000000}{12.000000}\selectfont 2.40}%
\end{pgfscope}%
\begin{pgfscope}%
\pgfsetbuttcap%
\pgfsetroundjoin%
\definecolor{currentfill}{rgb}{0.000000,0.000000,0.000000}%
\pgfsetfillcolor{currentfill}%
\pgfsetlinewidth{0.803000pt}%
\definecolor{currentstroke}{rgb}{0.000000,0.000000,0.000000}%
\pgfsetstrokecolor{currentstroke}%
\pgfsetdash{}{0pt}%
\pgfsys@defobject{currentmarker}{\pgfqpoint{-0.048611in}{0.000000in}}{\pgfqpoint{0.000000in}{0.000000in}}{%
\pgfpathmoveto{\pgfqpoint{0.000000in}{0.000000in}}%
\pgfpathlineto{\pgfqpoint{-0.048611in}{0.000000in}}%
\pgfusepath{stroke,fill}%
}%
\begin{pgfscope}%
\pgfsys@transformshift{0.672222in}{3.586181in}%
\pgfsys@useobject{currentmarker}{}%
\end{pgfscope}%
\end{pgfscope}%
\begin{pgfscope}%
\definecolor{textcolor}{rgb}{0.000000,0.000000,0.000000}%
\pgfsetstrokecolor{textcolor}%
\pgfsetfillcolor{textcolor}%
\pgftext[x=0.149382in,y=3.533419in,left,base]{\color{textcolor}\sffamily\fontsize{10.000000}{12.000000}\selectfont −1.50}%
\end{pgfscope}%
\begin{pgfscope}%
\pgfsetbuttcap%
\pgfsetroundjoin%
\definecolor{currentfill}{rgb}{0.000000,0.000000,0.000000}%
\pgfsetfillcolor{currentfill}%
\pgfsetlinewidth{0.803000pt}%
\definecolor{currentstroke}{rgb}{0.000000,0.000000,0.000000}%
\pgfsetstrokecolor{currentstroke}%
\pgfsetdash{}{0pt}%
\pgfsys@defobject{currentmarker}{\pgfqpoint{-0.048611in}{0.000000in}}{\pgfqpoint{0.000000in}{0.000000in}}{%
\pgfpathmoveto{\pgfqpoint{0.000000in}{0.000000in}}%
\pgfpathlineto{\pgfqpoint{-0.048611in}{0.000000in}}%
\pgfusepath{stroke,fill}%
}%
\begin{pgfscope}%
\pgfsys@transformshift{0.672222in}{3.942430in}%
\pgfsys@useobject{currentmarker}{}%
\end{pgfscope}%
\end{pgfscope}%
\begin{pgfscope}%
\definecolor{textcolor}{rgb}{0.000000,0.000000,0.000000}%
\pgfsetstrokecolor{textcolor}%
\pgfsetfillcolor{textcolor}%
\pgftext[x=0.149382in,y=3.889668in,left,base]{\color{textcolor}\sffamily\fontsize{10.000000}{12.000000}\selectfont −1.45}%
\end{pgfscope}%
\begin{pgfscope}%
\pgfsetbuttcap%
\pgfsetroundjoin%
\definecolor{currentfill}{rgb}{0.000000,0.000000,0.000000}%
\pgfsetfillcolor{currentfill}%
\pgfsetlinewidth{0.803000pt}%
\definecolor{currentstroke}{rgb}{0.000000,0.000000,0.000000}%
\pgfsetstrokecolor{currentstroke}%
\pgfsetdash{}{0pt}%
\pgfsys@defobject{currentmarker}{\pgfqpoint{-0.048611in}{0.000000in}}{\pgfqpoint{0.000000in}{0.000000in}}{%
\pgfpathmoveto{\pgfqpoint{0.000000in}{0.000000in}}%
\pgfpathlineto{\pgfqpoint{-0.048611in}{0.000000in}}%
\pgfusepath{stroke,fill}%
}%
\begin{pgfscope}%
\pgfsys@transformshift{0.672222in}{4.298678in}%
\pgfsys@useobject{currentmarker}{}%
\end{pgfscope}%
\end{pgfscope}%
\begin{pgfscope}%
\definecolor{textcolor}{rgb}{0.000000,0.000000,0.000000}%
\pgfsetstrokecolor{textcolor}%
\pgfsetfillcolor{textcolor}%
\pgftext[x=0.149382in,y=4.245917in,left,base]{\color{textcolor}\sffamily\fontsize{10.000000}{12.000000}\selectfont −1.40}%
\end{pgfscope}%
\begin{pgfscope}%
\pgfsetbuttcap%
\pgfsetroundjoin%
\definecolor{currentfill}{rgb}{0.000000,0.000000,0.000000}%
\pgfsetfillcolor{currentfill}%
\pgfsetlinewidth{0.803000pt}%
\definecolor{currentstroke}{rgb}{0.000000,0.000000,0.000000}%
\pgfsetstrokecolor{currentstroke}%
\pgfsetdash{}{0pt}%
\pgfsys@defobject{currentmarker}{\pgfqpoint{-0.048611in}{0.000000in}}{\pgfqpoint{0.000000in}{0.000000in}}{%
\pgfpathmoveto{\pgfqpoint{0.000000in}{0.000000in}}%
\pgfpathlineto{\pgfqpoint{-0.048611in}{0.000000in}}%
\pgfusepath{stroke,fill}%
}%
\begin{pgfscope}%
\pgfsys@transformshift{0.672222in}{4.654927in}%
\pgfsys@useobject{currentmarker}{}%
\end{pgfscope}%
\end{pgfscope}%
\begin{pgfscope}%
\definecolor{textcolor}{rgb}{0.000000,0.000000,0.000000}%
\pgfsetstrokecolor{textcolor}%
\pgfsetfillcolor{textcolor}%
\pgftext[x=0.149382in,y=4.602165in,left,base]{\color{textcolor}\sffamily\fontsize{10.000000}{12.000000}\selectfont −1.35}%
\end{pgfscope}%
\begin{pgfscope}%
\pgfsetbuttcap%
\pgfsetroundjoin%
\definecolor{currentfill}{rgb}{0.000000,0.000000,0.000000}%
\pgfsetfillcolor{currentfill}%
\pgfsetlinewidth{0.803000pt}%
\definecolor{currentstroke}{rgb}{0.000000,0.000000,0.000000}%
\pgfsetstrokecolor{currentstroke}%
\pgfsetdash{}{0pt}%
\pgfsys@defobject{currentmarker}{\pgfqpoint{-0.048611in}{0.000000in}}{\pgfqpoint{0.000000in}{0.000000in}}{%
\pgfpathmoveto{\pgfqpoint{0.000000in}{0.000000in}}%
\pgfpathlineto{\pgfqpoint{-0.048611in}{0.000000in}}%
\pgfusepath{stroke,fill}%
}%
\begin{pgfscope}%
\pgfsys@transformshift{0.672222in}{5.011176in}%
\pgfsys@useobject{currentmarker}{}%
\end{pgfscope}%
\end{pgfscope}%
\begin{pgfscope}%
\definecolor{textcolor}{rgb}{0.000000,0.000000,0.000000}%
\pgfsetstrokecolor{textcolor}%
\pgfsetfillcolor{textcolor}%
\pgftext[x=0.149382in,y=4.958414in,left,base]{\color{textcolor}\sffamily\fontsize{10.000000}{12.000000}\selectfont −1.30}%
\end{pgfscope}%
\begin{pgfscope}%
\pgfsetbuttcap%
\pgfsetroundjoin%
\definecolor{currentfill}{rgb}{0.000000,0.000000,0.000000}%
\pgfsetfillcolor{currentfill}%
\pgfsetlinewidth{0.803000pt}%
\definecolor{currentstroke}{rgb}{0.000000,0.000000,0.000000}%
\pgfsetstrokecolor{currentstroke}%
\pgfsetdash{}{0pt}%
\pgfsys@defobject{currentmarker}{\pgfqpoint{-0.048611in}{0.000000in}}{\pgfqpoint{0.000000in}{0.000000in}}{%
\pgfpathmoveto{\pgfqpoint{0.000000in}{0.000000in}}%
\pgfpathlineto{\pgfqpoint{-0.048611in}{0.000000in}}%
\pgfusepath{stroke,fill}%
}%
\begin{pgfscope}%
\pgfsys@transformshift{0.672222in}{5.367424in}%
\pgfsys@useobject{currentmarker}{}%
\end{pgfscope}%
\end{pgfscope}%
\begin{pgfscope}%
\definecolor{textcolor}{rgb}{0.000000,0.000000,0.000000}%
\pgfsetstrokecolor{textcolor}%
\pgfsetfillcolor{textcolor}%
\pgftext[x=0.149382in,y=5.314663in,left,base]{\color{textcolor}\sffamily\fontsize{10.000000}{12.000000}\selectfont −1.25}%
\end{pgfscope}%
\begin{pgfscope}%
\pgfpathrectangle{\pgfqpoint{0.672222in}{3.312222in}}{\pgfqpoint{3.206333in}{2.324444in}}%
\pgfusepath{clip}%
\pgfsetrectcap%
\pgfsetroundjoin%
\pgfsetlinewidth{1.505625pt}%
\definecolor{currentstroke}{rgb}{0.121569,0.466667,0.705882}%
\pgfsetstrokecolor{currentstroke}%
\pgfsetdash{}{0pt}%
\pgfpathmoveto{\pgfqpoint{0.825965in}{3.427460in}}%
\pgfpathlineto{\pgfqpoint{0.970908in}{3.521743in}}%
\pgfpathlineto{\pgfqpoint{1.115850in}{3.612834in}}%
\pgfpathlineto{\pgfqpoint{1.260792in}{3.713628in}}%
\pgfpathlineto{\pgfqpoint{1.405735in}{3.829088in}}%
\pgfpathlineto{\pgfqpoint{1.550677in}{3.925335in}}%
\pgfpathlineto{\pgfqpoint{1.695619in}{4.025758in}}%
\pgfpathlineto{\pgfqpoint{1.840562in}{4.148021in}}%
\pgfpathlineto{\pgfqpoint{1.985504in}{4.260493in}}%
\pgfpathlineto{\pgfqpoint{2.130446in}{4.362687in}}%
\pgfpathlineto{\pgfqpoint{2.275389in}{4.476560in}}%
\pgfpathlineto{\pgfqpoint{2.420331in}{4.580513in}}%
\pgfpathlineto{\pgfqpoint{2.565273in}{4.691865in}}%
\pgfpathlineto{\pgfqpoint{2.710216in}{4.797837in}}%
\pgfpathlineto{\pgfqpoint{2.855158in}{4.906322in}}%
\pgfpathlineto{\pgfqpoint{3.000100in}{5.019604in}}%
\pgfpathlineto{\pgfqpoint{3.145043in}{5.132567in}}%
\pgfpathlineto{\pgfqpoint{3.289985in}{5.223446in}}%
\pgfpathlineto{\pgfqpoint{3.434927in}{5.341086in}}%
\pgfpathlineto{\pgfqpoint{3.579870in}{5.431662in}}%
\pgfpathlineto{\pgfqpoint{3.724812in}{5.521682in}}%
\pgfusepath{stroke}%
\end{pgfscope}%
\begin{pgfscope}%
\pgfsetrectcap%
\pgfsetmiterjoin%
\pgfsetlinewidth{0.803000pt}%
\definecolor{currentstroke}{rgb}{0.000000,0.000000,0.000000}%
\pgfsetstrokecolor{currentstroke}%
\pgfsetdash{}{0pt}%
\pgfpathmoveto{\pgfqpoint{0.672222in}{3.312222in}}%
\pgfpathlineto{\pgfqpoint{0.672222in}{5.636667in}}%
\pgfusepath{stroke}%
\end{pgfscope}%
\begin{pgfscope}%
\pgfsetrectcap%
\pgfsetmiterjoin%
\pgfsetlinewidth{0.803000pt}%
\definecolor{currentstroke}{rgb}{0.000000,0.000000,0.000000}%
\pgfsetstrokecolor{currentstroke}%
\pgfsetdash{}{0pt}%
\pgfpathmoveto{\pgfqpoint{3.878555in}{3.312222in}}%
\pgfpathlineto{\pgfqpoint{3.878555in}{5.636667in}}%
\pgfusepath{stroke}%
\end{pgfscope}%
\begin{pgfscope}%
\pgfsetrectcap%
\pgfsetmiterjoin%
\pgfsetlinewidth{0.803000pt}%
\definecolor{currentstroke}{rgb}{0.000000,0.000000,0.000000}%
\pgfsetstrokecolor{currentstroke}%
\pgfsetdash{}{0pt}%
\pgfpathmoveto{\pgfqpoint{0.672222in}{3.312222in}}%
\pgfpathlineto{\pgfqpoint{3.878555in}{3.312222in}}%
\pgfusepath{stroke}%
\end{pgfscope}%
\begin{pgfscope}%
\pgfsetrectcap%
\pgfsetmiterjoin%
\pgfsetlinewidth{0.803000pt}%
\definecolor{currentstroke}{rgb}{0.000000,0.000000,0.000000}%
\pgfsetstrokecolor{currentstroke}%
\pgfsetdash{}{0pt}%
\pgfpathmoveto{\pgfqpoint{0.672222in}{5.636667in}}%
\pgfpathlineto{\pgfqpoint{3.878555in}{5.636667in}}%
\pgfusepath{stroke}%
\end{pgfscope}%
\begin{pgfscope}%
\definecolor{textcolor}{rgb}{0.000000,0.000000,0.000000}%
\pgfsetstrokecolor{textcolor}%
\pgfsetfillcolor{textcolor}%
\pgftext[x=2.275389in,y=5.720000in,,base]{\color{textcolor}\sffamily\fontsize{12.000000}{14.400000}\selectfont \(\displaystyle  N = 16 \)}%
\end{pgfscope}%
\begin{pgfscope}%
\pgfsetbuttcap%
\pgfsetmiterjoin%
\definecolor{currentfill}{rgb}{1.000000,1.000000,1.000000}%
\pgfsetfillcolor{currentfill}%
\pgfsetlinewidth{0.000000pt}%
\definecolor{currentstroke}{rgb}{0.000000,0.000000,0.000000}%
\pgfsetstrokecolor{currentstroke}%
\pgfsetstrokeopacity{0.000000}%
\pgfsetdash{}{0pt}%
\pgfpathmoveto{\pgfqpoint{4.498277in}{3.312222in}}%
\pgfpathlineto{\pgfqpoint{7.704610in}{3.312222in}}%
\pgfpathlineto{\pgfqpoint{7.704610in}{5.636667in}}%
\pgfpathlineto{\pgfqpoint{4.498277in}{5.636667in}}%
\pgfpathclose%
\pgfusepath{fill}%
\end{pgfscope}%
\begin{pgfscope}%
\pgfpathrectangle{\pgfqpoint{4.498277in}{3.312222in}}{\pgfqpoint{3.206333in}{2.324444in}}%
\pgfusepath{clip}%
\pgfsetbuttcap%
\pgfsetroundjoin%
\definecolor{currentfill}{rgb}{0.121569,0.466667,0.705882}%
\pgfsetfillcolor{currentfill}%
\pgfsetfillopacity{0.300000}%
\pgfsetlinewidth{1.003750pt}%
\definecolor{currentstroke}{rgb}{0.121569,0.466667,0.705882}%
\pgfsetstrokecolor{currentstroke}%
\pgfsetstrokeopacity{0.300000}%
\pgfsetdash{}{0pt}%
\pgfpathmoveto{\pgfqpoint{4.652021in}{3.435227in}}%
\pgfpathlineto{\pgfqpoint{4.652021in}{3.417879in}}%
\pgfpathlineto{\pgfqpoint{4.796963in}{3.503122in}}%
\pgfpathlineto{\pgfqpoint{4.941905in}{3.583782in}}%
\pgfpathlineto{\pgfqpoint{5.086848in}{3.676787in}}%
\pgfpathlineto{\pgfqpoint{5.231790in}{3.773916in}}%
\pgfpathlineto{\pgfqpoint{5.376732in}{3.882559in}}%
\pgfpathlineto{\pgfqpoint{5.521675in}{3.982814in}}%
\pgfpathlineto{\pgfqpoint{5.666617in}{4.118524in}}%
\pgfpathlineto{\pgfqpoint{5.811559in}{4.191846in}}%
\pgfpathlineto{\pgfqpoint{5.956502in}{4.308289in}}%
\pgfpathlineto{\pgfqpoint{6.101444in}{4.444567in}}%
\pgfpathlineto{\pgfqpoint{6.246386in}{4.589884in}}%
\pgfpathlineto{\pgfqpoint{6.391329in}{4.678251in}}%
\pgfpathlineto{\pgfqpoint{6.536271in}{4.810423in}}%
\pgfpathlineto{\pgfqpoint{6.681213in}{4.923460in}}%
\pgfpathlineto{\pgfqpoint{6.826156in}{5.038548in}}%
\pgfpathlineto{\pgfqpoint{6.971098in}{5.130711in}}%
\pgfpathlineto{\pgfqpoint{7.116040in}{5.216541in}}%
\pgfpathlineto{\pgfqpoint{7.260983in}{5.335192in}}%
\pgfpathlineto{\pgfqpoint{7.405925in}{5.430760in}}%
\pgfpathlineto{\pgfqpoint{7.550867in}{5.494550in}}%
\pgfpathlineto{\pgfqpoint{7.550867in}{5.531010in}}%
\pgfpathlineto{\pgfqpoint{7.550867in}{5.531010in}}%
\pgfpathlineto{\pgfqpoint{7.405925in}{5.438205in}}%
\pgfpathlineto{\pgfqpoint{7.260983in}{5.364024in}}%
\pgfpathlineto{\pgfqpoint{7.116040in}{5.252914in}}%
\pgfpathlineto{\pgfqpoint{6.971098in}{5.144982in}}%
\pgfpathlineto{\pgfqpoint{6.826156in}{5.051843in}}%
\pgfpathlineto{\pgfqpoint{6.681213in}{4.958414in}}%
\pgfpathlineto{\pgfqpoint{6.536271in}{4.850257in}}%
\pgfpathlineto{\pgfqpoint{6.391329in}{4.756923in}}%
\pgfpathlineto{\pgfqpoint{6.246386in}{4.605174in}}%
\pgfpathlineto{\pgfqpoint{6.101444in}{4.492312in}}%
\pgfpathlineto{\pgfqpoint{5.956502in}{4.339511in}}%
\pgfpathlineto{\pgfqpoint{5.811559in}{4.253454in}}%
\pgfpathlineto{\pgfqpoint{5.666617in}{4.126988in}}%
\pgfpathlineto{\pgfqpoint{5.521675in}{3.995375in}}%
\pgfpathlineto{\pgfqpoint{5.376732in}{3.943286in}}%
\pgfpathlineto{\pgfqpoint{5.231790in}{3.793102in}}%
\pgfpathlineto{\pgfqpoint{5.086848in}{3.708820in}}%
\pgfpathlineto{\pgfqpoint{4.941905in}{3.609716in}}%
\pgfpathlineto{\pgfqpoint{4.796963in}{3.516439in}}%
\pgfpathlineto{\pgfqpoint{4.652021in}{3.435227in}}%
\pgfpathclose%
\pgfusepath{stroke,fill}%
\end{pgfscope}%
\begin{pgfscope}%
\pgfpathrectangle{\pgfqpoint{4.498277in}{3.312222in}}{\pgfqpoint{3.206333in}{2.324444in}}%
\pgfusepath{clip}%
\pgfsetbuttcap%
\pgfsetroundjoin%
\definecolor{currentfill}{rgb}{0.121569,0.466667,0.705882}%
\pgfsetfillcolor{currentfill}%
\pgfsetlinewidth{1.003750pt}%
\definecolor{currentstroke}{rgb}{0.121569,0.466667,0.705882}%
\pgfsetstrokecolor{currentstroke}%
\pgfsetdash{}{0pt}%
\pgfsys@defobject{currentmarker}{\pgfqpoint{-0.009821in}{-0.009821in}}{\pgfqpoint{0.009821in}{0.009821in}}{%
\pgfpathmoveto{\pgfqpoint{0.000000in}{-0.009821in}}%
\pgfpathcurveto{\pgfqpoint{0.002605in}{-0.009821in}}{\pgfqpoint{0.005103in}{-0.008786in}}{\pgfqpoint{0.006944in}{-0.006944in}}%
\pgfpathcurveto{\pgfqpoint{0.008786in}{-0.005103in}}{\pgfqpoint{0.009821in}{-0.002605in}}{\pgfqpoint{0.009821in}{0.000000in}}%
\pgfpathcurveto{\pgfqpoint{0.009821in}{0.002605in}}{\pgfqpoint{0.008786in}{0.005103in}}{\pgfqpoint{0.006944in}{0.006944in}}%
\pgfpathcurveto{\pgfqpoint{0.005103in}{0.008786in}}{\pgfqpoint{0.002605in}{0.009821in}}{\pgfqpoint{0.000000in}{0.009821in}}%
\pgfpathcurveto{\pgfqpoint{-0.002605in}{0.009821in}}{\pgfqpoint{-0.005103in}{0.008786in}}{\pgfqpoint{-0.006944in}{0.006944in}}%
\pgfpathcurveto{\pgfqpoint{-0.008786in}{0.005103in}}{\pgfqpoint{-0.009821in}{0.002605in}}{\pgfqpoint{-0.009821in}{0.000000in}}%
\pgfpathcurveto{\pgfqpoint{-0.009821in}{-0.002605in}}{\pgfqpoint{-0.008786in}{-0.005103in}}{\pgfqpoint{-0.006944in}{-0.006944in}}%
\pgfpathcurveto{\pgfqpoint{-0.005103in}{-0.008786in}}{\pgfqpoint{-0.002605in}{-0.009821in}}{\pgfqpoint{0.000000in}{-0.009821in}}%
\pgfpathclose%
\pgfusepath{stroke,fill}%
}%
\begin{pgfscope}%
\pgfsys@transformshift{4.652021in}{3.426553in}%
\pgfsys@useobject{currentmarker}{}%
\end{pgfscope}%
\begin{pgfscope}%
\pgfsys@transformshift{4.796963in}{3.509781in}%
\pgfsys@useobject{currentmarker}{}%
\end{pgfscope}%
\begin{pgfscope}%
\pgfsys@transformshift{4.941905in}{3.596749in}%
\pgfsys@useobject{currentmarker}{}%
\end{pgfscope}%
\begin{pgfscope}%
\pgfsys@transformshift{5.086848in}{3.692804in}%
\pgfsys@useobject{currentmarker}{}%
\end{pgfscope}%
\begin{pgfscope}%
\pgfsys@transformshift{5.231790in}{3.783509in}%
\pgfsys@useobject{currentmarker}{}%
\end{pgfscope}%
\begin{pgfscope}%
\pgfsys@transformshift{5.376732in}{3.912923in}%
\pgfsys@useobject{currentmarker}{}%
\end{pgfscope}%
\begin{pgfscope}%
\pgfsys@transformshift{5.521675in}{3.989095in}%
\pgfsys@useobject{currentmarker}{}%
\end{pgfscope}%
\begin{pgfscope}%
\pgfsys@transformshift{5.666617in}{4.122756in}%
\pgfsys@useobject{currentmarker}{}%
\end{pgfscope}%
\begin{pgfscope}%
\pgfsys@transformshift{5.811559in}{4.222650in}%
\pgfsys@useobject{currentmarker}{}%
\end{pgfscope}%
\begin{pgfscope}%
\pgfsys@transformshift{5.956502in}{4.323900in}%
\pgfsys@useobject{currentmarker}{}%
\end{pgfscope}%
\begin{pgfscope}%
\pgfsys@transformshift{6.101444in}{4.468439in}%
\pgfsys@useobject{currentmarker}{}%
\end{pgfscope}%
\begin{pgfscope}%
\pgfsys@transformshift{6.246386in}{4.597529in}%
\pgfsys@useobject{currentmarker}{}%
\end{pgfscope}%
\begin{pgfscope}%
\pgfsys@transformshift{6.391329in}{4.717587in}%
\pgfsys@useobject{currentmarker}{}%
\end{pgfscope}%
\begin{pgfscope}%
\pgfsys@transformshift{6.536271in}{4.830340in}%
\pgfsys@useobject{currentmarker}{}%
\end{pgfscope}%
\begin{pgfscope}%
\pgfsys@transformshift{6.681213in}{4.940937in}%
\pgfsys@useobject{currentmarker}{}%
\end{pgfscope}%
\begin{pgfscope}%
\pgfsys@transformshift{6.826156in}{5.045196in}%
\pgfsys@useobject{currentmarker}{}%
\end{pgfscope}%
\begin{pgfscope}%
\pgfsys@transformshift{6.971098in}{5.137847in}%
\pgfsys@useobject{currentmarker}{}%
\end{pgfscope}%
\begin{pgfscope}%
\pgfsys@transformshift{7.116040in}{5.234727in}%
\pgfsys@useobject{currentmarker}{}%
\end{pgfscope}%
\begin{pgfscope}%
\pgfsys@transformshift{7.260983in}{5.349608in}%
\pgfsys@useobject{currentmarker}{}%
\end{pgfscope}%
\begin{pgfscope}%
\pgfsys@transformshift{7.405925in}{5.434483in}%
\pgfsys@useobject{currentmarker}{}%
\end{pgfscope}%
\begin{pgfscope}%
\pgfsys@transformshift{7.550867in}{5.512780in}%
\pgfsys@useobject{currentmarker}{}%
\end{pgfscope}%
\end{pgfscope}%
\begin{pgfscope}%
\pgfsetbuttcap%
\pgfsetroundjoin%
\definecolor{currentfill}{rgb}{0.000000,0.000000,0.000000}%
\pgfsetfillcolor{currentfill}%
\pgfsetlinewidth{0.803000pt}%
\definecolor{currentstroke}{rgb}{0.000000,0.000000,0.000000}%
\pgfsetstrokecolor{currentstroke}%
\pgfsetdash{}{0pt}%
\pgfsys@defobject{currentmarker}{\pgfqpoint{0.000000in}{-0.048611in}}{\pgfqpoint{0.000000in}{0.000000in}}{%
\pgfpathmoveto{\pgfqpoint{0.000000in}{0.000000in}}%
\pgfpathlineto{\pgfqpoint{0.000000in}{-0.048611in}}%
\pgfusepath{stroke,fill}%
}%
\begin{pgfscope}%
\pgfsys@transformshift{4.796963in}{3.312222in}%
\pgfsys@useobject{currentmarker}{}%
\end{pgfscope}%
\end{pgfscope}%
\begin{pgfscope}%
\definecolor{textcolor}{rgb}{0.000000,0.000000,0.000000}%
\pgfsetstrokecolor{textcolor}%
\pgfsetfillcolor{textcolor}%
\pgftext[x=4.796963in,y=3.215000in,,top]{\color{textcolor}\sffamily\fontsize{10.000000}{12.000000}\selectfont 2.20}%
\end{pgfscope}%
\begin{pgfscope}%
\pgfsetbuttcap%
\pgfsetroundjoin%
\definecolor{currentfill}{rgb}{0.000000,0.000000,0.000000}%
\pgfsetfillcolor{currentfill}%
\pgfsetlinewidth{0.803000pt}%
\definecolor{currentstroke}{rgb}{0.000000,0.000000,0.000000}%
\pgfsetstrokecolor{currentstroke}%
\pgfsetdash{}{0pt}%
\pgfsys@defobject{currentmarker}{\pgfqpoint{0.000000in}{-0.048611in}}{\pgfqpoint{0.000000in}{0.000000in}}{%
\pgfpathmoveto{\pgfqpoint{0.000000in}{0.000000in}}%
\pgfpathlineto{\pgfqpoint{0.000000in}{-0.048611in}}%
\pgfusepath{stroke,fill}%
}%
\begin{pgfscope}%
\pgfsys@transformshift{5.521675in}{3.312222in}%
\pgfsys@useobject{currentmarker}{}%
\end{pgfscope}%
\end{pgfscope}%
\begin{pgfscope}%
\definecolor{textcolor}{rgb}{0.000000,0.000000,0.000000}%
\pgfsetstrokecolor{textcolor}%
\pgfsetfillcolor{textcolor}%
\pgftext[x=5.521675in,y=3.215000in,,top]{\color{textcolor}\sffamily\fontsize{10.000000}{12.000000}\selectfont 2.25}%
\end{pgfscope}%
\begin{pgfscope}%
\pgfsetbuttcap%
\pgfsetroundjoin%
\definecolor{currentfill}{rgb}{0.000000,0.000000,0.000000}%
\pgfsetfillcolor{currentfill}%
\pgfsetlinewidth{0.803000pt}%
\definecolor{currentstroke}{rgb}{0.000000,0.000000,0.000000}%
\pgfsetstrokecolor{currentstroke}%
\pgfsetdash{}{0pt}%
\pgfsys@defobject{currentmarker}{\pgfqpoint{0.000000in}{-0.048611in}}{\pgfqpoint{0.000000in}{0.000000in}}{%
\pgfpathmoveto{\pgfqpoint{0.000000in}{0.000000in}}%
\pgfpathlineto{\pgfqpoint{0.000000in}{-0.048611in}}%
\pgfusepath{stroke,fill}%
}%
\begin{pgfscope}%
\pgfsys@transformshift{6.246386in}{3.312222in}%
\pgfsys@useobject{currentmarker}{}%
\end{pgfscope}%
\end{pgfscope}%
\begin{pgfscope}%
\definecolor{textcolor}{rgb}{0.000000,0.000000,0.000000}%
\pgfsetstrokecolor{textcolor}%
\pgfsetfillcolor{textcolor}%
\pgftext[x=6.246386in,y=3.215000in,,top]{\color{textcolor}\sffamily\fontsize{10.000000}{12.000000}\selectfont 2.30}%
\end{pgfscope}%
\begin{pgfscope}%
\pgfsetbuttcap%
\pgfsetroundjoin%
\definecolor{currentfill}{rgb}{0.000000,0.000000,0.000000}%
\pgfsetfillcolor{currentfill}%
\pgfsetlinewidth{0.803000pt}%
\definecolor{currentstroke}{rgb}{0.000000,0.000000,0.000000}%
\pgfsetstrokecolor{currentstroke}%
\pgfsetdash{}{0pt}%
\pgfsys@defobject{currentmarker}{\pgfqpoint{0.000000in}{-0.048611in}}{\pgfqpoint{0.000000in}{0.000000in}}{%
\pgfpathmoveto{\pgfqpoint{0.000000in}{0.000000in}}%
\pgfpathlineto{\pgfqpoint{0.000000in}{-0.048611in}}%
\pgfusepath{stroke,fill}%
}%
\begin{pgfscope}%
\pgfsys@transformshift{6.971098in}{3.312222in}%
\pgfsys@useobject{currentmarker}{}%
\end{pgfscope}%
\end{pgfscope}%
\begin{pgfscope}%
\definecolor{textcolor}{rgb}{0.000000,0.000000,0.000000}%
\pgfsetstrokecolor{textcolor}%
\pgfsetfillcolor{textcolor}%
\pgftext[x=6.971098in,y=3.215000in,,top]{\color{textcolor}\sffamily\fontsize{10.000000}{12.000000}\selectfont 2.35}%
\end{pgfscope}%
\begin{pgfscope}%
\pgfsetbuttcap%
\pgfsetroundjoin%
\definecolor{currentfill}{rgb}{0.000000,0.000000,0.000000}%
\pgfsetfillcolor{currentfill}%
\pgfsetlinewidth{0.803000pt}%
\definecolor{currentstroke}{rgb}{0.000000,0.000000,0.000000}%
\pgfsetstrokecolor{currentstroke}%
\pgfsetdash{}{0pt}%
\pgfsys@defobject{currentmarker}{\pgfqpoint{0.000000in}{-0.048611in}}{\pgfqpoint{0.000000in}{0.000000in}}{%
\pgfpathmoveto{\pgfqpoint{0.000000in}{0.000000in}}%
\pgfpathlineto{\pgfqpoint{0.000000in}{-0.048611in}}%
\pgfusepath{stroke,fill}%
}%
\begin{pgfscope}%
\pgfsys@transformshift{7.695810in}{3.312222in}%
\pgfsys@useobject{currentmarker}{}%
\end{pgfscope}%
\end{pgfscope}%
\begin{pgfscope}%
\definecolor{textcolor}{rgb}{0.000000,0.000000,0.000000}%
\pgfsetstrokecolor{textcolor}%
\pgfsetfillcolor{textcolor}%
\pgftext[x=7.695810in,y=3.215000in,,top]{\color{textcolor}\sffamily\fontsize{10.000000}{12.000000}\selectfont 2.40}%
\end{pgfscope}%
\begin{pgfscope}%
\pgfsetbuttcap%
\pgfsetroundjoin%
\definecolor{currentfill}{rgb}{0.000000,0.000000,0.000000}%
\pgfsetfillcolor{currentfill}%
\pgfsetlinewidth{0.803000pt}%
\definecolor{currentstroke}{rgb}{0.000000,0.000000,0.000000}%
\pgfsetstrokecolor{currentstroke}%
\pgfsetdash{}{0pt}%
\pgfsys@defobject{currentmarker}{\pgfqpoint{-0.048611in}{0.000000in}}{\pgfqpoint{0.000000in}{0.000000in}}{%
\pgfpathmoveto{\pgfqpoint{0.000000in}{0.000000in}}%
\pgfpathlineto{\pgfqpoint{-0.048611in}{0.000000in}}%
\pgfusepath{stroke,fill}%
}%
\begin{pgfscope}%
\pgfsys@transformshift{4.498277in}{3.492932in}%
\pgfsys@useobject{currentmarker}{}%
\end{pgfscope}%
\end{pgfscope}%
\begin{pgfscope}%
\definecolor{textcolor}{rgb}{0.000000,0.000000,0.000000}%
\pgfsetstrokecolor{textcolor}%
\pgfsetfillcolor{textcolor}%
\pgftext[x=3.975437in,y=3.440171in,left,base]{\color{textcolor}\sffamily\fontsize{10.000000}{12.000000}\selectfont −1.55}%
\end{pgfscope}%
\begin{pgfscope}%
\pgfsetbuttcap%
\pgfsetroundjoin%
\definecolor{currentfill}{rgb}{0.000000,0.000000,0.000000}%
\pgfsetfillcolor{currentfill}%
\pgfsetlinewidth{0.803000pt}%
\definecolor{currentstroke}{rgb}{0.000000,0.000000,0.000000}%
\pgfsetstrokecolor{currentstroke}%
\pgfsetdash{}{0pt}%
\pgfsys@defobject{currentmarker}{\pgfqpoint{-0.048611in}{0.000000in}}{\pgfqpoint{0.000000in}{0.000000in}}{%
\pgfpathmoveto{\pgfqpoint{0.000000in}{0.000000in}}%
\pgfpathlineto{\pgfqpoint{-0.048611in}{0.000000in}}%
\pgfusepath{stroke,fill}%
}%
\begin{pgfscope}%
\pgfsys@transformshift{4.498277in}{3.804448in}%
\pgfsys@useobject{currentmarker}{}%
\end{pgfscope}%
\end{pgfscope}%
\begin{pgfscope}%
\definecolor{textcolor}{rgb}{0.000000,0.000000,0.000000}%
\pgfsetstrokecolor{textcolor}%
\pgfsetfillcolor{textcolor}%
\pgftext[x=3.975437in,y=3.751687in,left,base]{\color{textcolor}\sffamily\fontsize{10.000000}{12.000000}\selectfont −1.50}%
\end{pgfscope}%
\begin{pgfscope}%
\pgfsetbuttcap%
\pgfsetroundjoin%
\definecolor{currentfill}{rgb}{0.000000,0.000000,0.000000}%
\pgfsetfillcolor{currentfill}%
\pgfsetlinewidth{0.803000pt}%
\definecolor{currentstroke}{rgb}{0.000000,0.000000,0.000000}%
\pgfsetstrokecolor{currentstroke}%
\pgfsetdash{}{0pt}%
\pgfsys@defobject{currentmarker}{\pgfqpoint{-0.048611in}{0.000000in}}{\pgfqpoint{0.000000in}{0.000000in}}{%
\pgfpathmoveto{\pgfqpoint{0.000000in}{0.000000in}}%
\pgfpathlineto{\pgfqpoint{-0.048611in}{0.000000in}}%
\pgfusepath{stroke,fill}%
}%
\begin{pgfscope}%
\pgfsys@transformshift{4.498277in}{4.115964in}%
\pgfsys@useobject{currentmarker}{}%
\end{pgfscope}%
\end{pgfscope}%
\begin{pgfscope}%
\definecolor{textcolor}{rgb}{0.000000,0.000000,0.000000}%
\pgfsetstrokecolor{textcolor}%
\pgfsetfillcolor{textcolor}%
\pgftext[x=3.975437in,y=4.063203in,left,base]{\color{textcolor}\sffamily\fontsize{10.000000}{12.000000}\selectfont −1.45}%
\end{pgfscope}%
\begin{pgfscope}%
\pgfsetbuttcap%
\pgfsetroundjoin%
\definecolor{currentfill}{rgb}{0.000000,0.000000,0.000000}%
\pgfsetfillcolor{currentfill}%
\pgfsetlinewidth{0.803000pt}%
\definecolor{currentstroke}{rgb}{0.000000,0.000000,0.000000}%
\pgfsetstrokecolor{currentstroke}%
\pgfsetdash{}{0pt}%
\pgfsys@defobject{currentmarker}{\pgfqpoint{-0.048611in}{0.000000in}}{\pgfqpoint{0.000000in}{0.000000in}}{%
\pgfpathmoveto{\pgfqpoint{0.000000in}{0.000000in}}%
\pgfpathlineto{\pgfqpoint{-0.048611in}{0.000000in}}%
\pgfusepath{stroke,fill}%
}%
\begin{pgfscope}%
\pgfsys@transformshift{4.498277in}{4.427480in}%
\pgfsys@useobject{currentmarker}{}%
\end{pgfscope}%
\end{pgfscope}%
\begin{pgfscope}%
\definecolor{textcolor}{rgb}{0.000000,0.000000,0.000000}%
\pgfsetstrokecolor{textcolor}%
\pgfsetfillcolor{textcolor}%
\pgftext[x=3.975437in,y=4.374719in,left,base]{\color{textcolor}\sffamily\fontsize{10.000000}{12.000000}\selectfont −1.40}%
\end{pgfscope}%
\begin{pgfscope}%
\pgfsetbuttcap%
\pgfsetroundjoin%
\definecolor{currentfill}{rgb}{0.000000,0.000000,0.000000}%
\pgfsetfillcolor{currentfill}%
\pgfsetlinewidth{0.803000pt}%
\definecolor{currentstroke}{rgb}{0.000000,0.000000,0.000000}%
\pgfsetstrokecolor{currentstroke}%
\pgfsetdash{}{0pt}%
\pgfsys@defobject{currentmarker}{\pgfqpoint{-0.048611in}{0.000000in}}{\pgfqpoint{0.000000in}{0.000000in}}{%
\pgfpathmoveto{\pgfqpoint{0.000000in}{0.000000in}}%
\pgfpathlineto{\pgfqpoint{-0.048611in}{0.000000in}}%
\pgfusepath{stroke,fill}%
}%
\begin{pgfscope}%
\pgfsys@transformshift{4.498277in}{4.738996in}%
\pgfsys@useobject{currentmarker}{}%
\end{pgfscope}%
\end{pgfscope}%
\begin{pgfscope}%
\definecolor{textcolor}{rgb}{0.000000,0.000000,0.000000}%
\pgfsetstrokecolor{textcolor}%
\pgfsetfillcolor{textcolor}%
\pgftext[x=3.975437in,y=4.686235in,left,base]{\color{textcolor}\sffamily\fontsize{10.000000}{12.000000}\selectfont −1.35}%
\end{pgfscope}%
\begin{pgfscope}%
\pgfsetbuttcap%
\pgfsetroundjoin%
\definecolor{currentfill}{rgb}{0.000000,0.000000,0.000000}%
\pgfsetfillcolor{currentfill}%
\pgfsetlinewidth{0.803000pt}%
\definecolor{currentstroke}{rgb}{0.000000,0.000000,0.000000}%
\pgfsetstrokecolor{currentstroke}%
\pgfsetdash{}{0pt}%
\pgfsys@defobject{currentmarker}{\pgfqpoint{-0.048611in}{0.000000in}}{\pgfqpoint{0.000000in}{0.000000in}}{%
\pgfpathmoveto{\pgfqpoint{0.000000in}{0.000000in}}%
\pgfpathlineto{\pgfqpoint{-0.048611in}{0.000000in}}%
\pgfusepath{stroke,fill}%
}%
\begin{pgfscope}%
\pgfsys@transformshift{4.498277in}{5.050512in}%
\pgfsys@useobject{currentmarker}{}%
\end{pgfscope}%
\end{pgfscope}%
\begin{pgfscope}%
\definecolor{textcolor}{rgb}{0.000000,0.000000,0.000000}%
\pgfsetstrokecolor{textcolor}%
\pgfsetfillcolor{textcolor}%
\pgftext[x=3.975437in,y=4.997751in,left,base]{\color{textcolor}\sffamily\fontsize{10.000000}{12.000000}\selectfont −1.30}%
\end{pgfscope}%
\begin{pgfscope}%
\pgfsetbuttcap%
\pgfsetroundjoin%
\definecolor{currentfill}{rgb}{0.000000,0.000000,0.000000}%
\pgfsetfillcolor{currentfill}%
\pgfsetlinewidth{0.803000pt}%
\definecolor{currentstroke}{rgb}{0.000000,0.000000,0.000000}%
\pgfsetstrokecolor{currentstroke}%
\pgfsetdash{}{0pt}%
\pgfsys@defobject{currentmarker}{\pgfqpoint{-0.048611in}{0.000000in}}{\pgfqpoint{0.000000in}{0.000000in}}{%
\pgfpathmoveto{\pgfqpoint{0.000000in}{0.000000in}}%
\pgfpathlineto{\pgfqpoint{-0.048611in}{0.000000in}}%
\pgfusepath{stroke,fill}%
}%
\begin{pgfscope}%
\pgfsys@transformshift{4.498277in}{5.362029in}%
\pgfsys@useobject{currentmarker}{}%
\end{pgfscope}%
\end{pgfscope}%
\begin{pgfscope}%
\definecolor{textcolor}{rgb}{0.000000,0.000000,0.000000}%
\pgfsetstrokecolor{textcolor}%
\pgfsetfillcolor{textcolor}%
\pgftext[x=3.975437in,y=5.309267in,left,base]{\color{textcolor}\sffamily\fontsize{10.000000}{12.000000}\selectfont −1.25}%
\end{pgfscope}%
\begin{pgfscope}%
\pgfpathrectangle{\pgfqpoint{4.498277in}{3.312222in}}{\pgfqpoint{3.206333in}{2.324444in}}%
\pgfusepath{clip}%
\pgfsetrectcap%
\pgfsetroundjoin%
\pgfsetlinewidth{1.505625pt}%
\definecolor{currentstroke}{rgb}{0.121569,0.466667,0.705882}%
\pgfsetstrokecolor{currentstroke}%
\pgfsetdash{}{0pt}%
\pgfpathmoveto{\pgfqpoint{4.652021in}{3.426553in}}%
\pgfpathlineto{\pgfqpoint{4.796963in}{3.509781in}}%
\pgfpathlineto{\pgfqpoint{4.941905in}{3.596749in}}%
\pgfpathlineto{\pgfqpoint{5.086848in}{3.692804in}}%
\pgfpathlineto{\pgfqpoint{5.231790in}{3.783509in}}%
\pgfpathlineto{\pgfqpoint{5.376732in}{3.912923in}}%
\pgfpathlineto{\pgfqpoint{5.521675in}{3.989095in}}%
\pgfpathlineto{\pgfqpoint{5.666617in}{4.122756in}}%
\pgfpathlineto{\pgfqpoint{5.811559in}{4.222650in}}%
\pgfpathlineto{\pgfqpoint{5.956502in}{4.323900in}}%
\pgfpathlineto{\pgfqpoint{6.101444in}{4.468439in}}%
\pgfpathlineto{\pgfqpoint{6.246386in}{4.597529in}}%
\pgfpathlineto{\pgfqpoint{6.391329in}{4.717587in}}%
\pgfpathlineto{\pgfqpoint{6.536271in}{4.830340in}}%
\pgfpathlineto{\pgfqpoint{6.681213in}{4.940937in}}%
\pgfpathlineto{\pgfqpoint{6.826156in}{5.045196in}}%
\pgfpathlineto{\pgfqpoint{6.971098in}{5.137847in}}%
\pgfpathlineto{\pgfqpoint{7.116040in}{5.234727in}}%
\pgfpathlineto{\pgfqpoint{7.260983in}{5.349608in}}%
\pgfpathlineto{\pgfqpoint{7.405925in}{5.434483in}}%
\pgfpathlineto{\pgfqpoint{7.550867in}{5.512780in}}%
\pgfusepath{stroke}%
\end{pgfscope}%
\begin{pgfscope}%
\pgfsetrectcap%
\pgfsetmiterjoin%
\pgfsetlinewidth{0.803000pt}%
\definecolor{currentstroke}{rgb}{0.000000,0.000000,0.000000}%
\pgfsetstrokecolor{currentstroke}%
\pgfsetdash{}{0pt}%
\pgfpathmoveto{\pgfqpoint{4.498277in}{3.312222in}}%
\pgfpathlineto{\pgfqpoint{4.498277in}{5.636667in}}%
\pgfusepath{stroke}%
\end{pgfscope}%
\begin{pgfscope}%
\pgfsetrectcap%
\pgfsetmiterjoin%
\pgfsetlinewidth{0.803000pt}%
\definecolor{currentstroke}{rgb}{0.000000,0.000000,0.000000}%
\pgfsetstrokecolor{currentstroke}%
\pgfsetdash{}{0pt}%
\pgfpathmoveto{\pgfqpoint{7.704610in}{3.312222in}}%
\pgfpathlineto{\pgfqpoint{7.704610in}{5.636667in}}%
\pgfusepath{stroke}%
\end{pgfscope}%
\begin{pgfscope}%
\pgfsetrectcap%
\pgfsetmiterjoin%
\pgfsetlinewidth{0.803000pt}%
\definecolor{currentstroke}{rgb}{0.000000,0.000000,0.000000}%
\pgfsetstrokecolor{currentstroke}%
\pgfsetdash{}{0pt}%
\pgfpathmoveto{\pgfqpoint{4.498277in}{3.312222in}}%
\pgfpathlineto{\pgfqpoint{7.704610in}{3.312222in}}%
\pgfusepath{stroke}%
\end{pgfscope}%
\begin{pgfscope}%
\pgfsetrectcap%
\pgfsetmiterjoin%
\pgfsetlinewidth{0.803000pt}%
\definecolor{currentstroke}{rgb}{0.000000,0.000000,0.000000}%
\pgfsetstrokecolor{currentstroke}%
\pgfsetdash{}{0pt}%
\pgfpathmoveto{\pgfqpoint{4.498277in}{5.636667in}}%
\pgfpathlineto{\pgfqpoint{7.704610in}{5.636667in}}%
\pgfusepath{stroke}%
\end{pgfscope}%
\begin{pgfscope}%
\definecolor{textcolor}{rgb}{0.000000,0.000000,0.000000}%
\pgfsetstrokecolor{textcolor}%
\pgfsetfillcolor{textcolor}%
\pgftext[x=6.101444in,y=5.720000in,,base]{\color{textcolor}\sffamily\fontsize{12.000000}{14.400000}\selectfont \(\displaystyle  N = 32 \)}%
\end{pgfscope}%
\begin{pgfscope}%
\pgfsetbuttcap%
\pgfsetmiterjoin%
\definecolor{currentfill}{rgb}{1.000000,1.000000,1.000000}%
\pgfsetfillcolor{currentfill}%
\pgfsetlinewidth{0.000000pt}%
\definecolor{currentstroke}{rgb}{0.000000,0.000000,0.000000}%
\pgfsetstrokecolor{currentstroke}%
\pgfsetstrokeopacity{0.000000}%
\pgfsetdash{}{0pt}%
\pgfpathmoveto{\pgfqpoint{0.672222in}{0.387222in}}%
\pgfpathlineto{\pgfqpoint{3.878555in}{0.387222in}}%
\pgfpathlineto{\pgfqpoint{3.878555in}{2.711667in}}%
\pgfpathlineto{\pgfqpoint{0.672222in}{2.711667in}}%
\pgfpathclose%
\pgfusepath{fill}%
\end{pgfscope}%
\begin{pgfscope}%
\pgfpathrectangle{\pgfqpoint{0.672222in}{0.387222in}}{\pgfqpoint{3.206333in}{2.324444in}}%
\pgfusepath{clip}%
\pgfsetbuttcap%
\pgfsetroundjoin%
\definecolor{currentfill}{rgb}{0.121569,0.466667,0.705882}%
\pgfsetfillcolor{currentfill}%
\pgfsetfillopacity{0.300000}%
\pgfsetlinewidth{1.003750pt}%
\definecolor{currentstroke}{rgb}{0.121569,0.466667,0.705882}%
\pgfsetstrokecolor{currentstroke}%
\pgfsetstrokeopacity{0.300000}%
\pgfsetdash{}{0pt}%
\pgfpathmoveto{\pgfqpoint{0.825965in}{0.505013in}}%
\pgfpathlineto{\pgfqpoint{0.825965in}{0.496372in}}%
\pgfpathlineto{\pgfqpoint{0.970908in}{0.570115in}}%
\pgfpathlineto{\pgfqpoint{1.115850in}{0.629871in}}%
\pgfpathlineto{\pgfqpoint{1.260792in}{0.716258in}}%
\pgfpathlineto{\pgfqpoint{1.405735in}{0.826716in}}%
\pgfpathlineto{\pgfqpoint{1.550677in}{0.912296in}}%
\pgfpathlineto{\pgfqpoint{1.695619in}{1.016842in}}%
\pgfpathlineto{\pgfqpoint{1.840562in}{1.143860in}}%
\pgfpathlineto{\pgfqpoint{1.985504in}{1.244689in}}%
\pgfpathlineto{\pgfqpoint{2.130446in}{1.343996in}}%
\pgfpathlineto{\pgfqpoint{2.275389in}{1.464678in}}%
\pgfpathlineto{\pgfqpoint{2.420331in}{1.646044in}}%
\pgfpathlineto{\pgfqpoint{2.565273in}{1.752673in}}%
\pgfpathlineto{\pgfqpoint{2.710216in}{1.869435in}}%
\pgfpathlineto{\pgfqpoint{2.855158in}{2.011349in}}%
\pgfpathlineto{\pgfqpoint{3.000100in}{2.119090in}}%
\pgfpathlineto{\pgfqpoint{3.145043in}{2.221419in}}%
\pgfpathlineto{\pgfqpoint{3.289985in}{2.339932in}}%
\pgfpathlineto{\pgfqpoint{3.434927in}{2.411767in}}%
\pgfpathlineto{\pgfqpoint{3.579870in}{2.497522in}}%
\pgfpathlineto{\pgfqpoint{3.724812in}{2.576536in}}%
\pgfpathlineto{\pgfqpoint{3.724812in}{2.606010in}}%
\pgfpathlineto{\pgfqpoint{3.724812in}{2.606010in}}%
\pgfpathlineto{\pgfqpoint{3.579870in}{2.534355in}}%
\pgfpathlineto{\pgfqpoint{3.434927in}{2.437615in}}%
\pgfpathlineto{\pgfqpoint{3.289985in}{2.371508in}}%
\pgfpathlineto{\pgfqpoint{3.145043in}{2.287097in}}%
\pgfpathlineto{\pgfqpoint{3.000100in}{2.199825in}}%
\pgfpathlineto{\pgfqpoint{2.855158in}{2.139773in}}%
\pgfpathlineto{\pgfqpoint{2.710216in}{1.989117in}}%
\pgfpathlineto{\pgfqpoint{2.565273in}{1.838909in}}%
\pgfpathlineto{\pgfqpoint{2.420331in}{1.702496in}}%
\pgfpathlineto{\pgfqpoint{2.275389in}{1.607942in}}%
\pgfpathlineto{\pgfqpoint{2.130446in}{1.524749in}}%
\pgfpathlineto{\pgfqpoint{1.985504in}{1.396365in}}%
\pgfpathlineto{\pgfqpoint{1.840562in}{1.168172in}}%
\pgfpathlineto{\pgfqpoint{1.695619in}{1.068660in}}%
\pgfpathlineto{\pgfqpoint{1.550677in}{0.960257in}}%
\pgfpathlineto{\pgfqpoint{1.405735in}{0.885210in}}%
\pgfpathlineto{\pgfqpoint{1.260792in}{0.756185in}}%
\pgfpathlineto{\pgfqpoint{1.115850in}{0.691873in}}%
\pgfpathlineto{\pgfqpoint{0.970908in}{0.586490in}}%
\pgfpathlineto{\pgfqpoint{0.825965in}{0.505013in}}%
\pgfpathclose%
\pgfusepath{stroke,fill}%
\end{pgfscope}%
\begin{pgfscope}%
\pgfpathrectangle{\pgfqpoint{0.672222in}{0.387222in}}{\pgfqpoint{3.206333in}{2.324444in}}%
\pgfusepath{clip}%
\pgfsetbuttcap%
\pgfsetroundjoin%
\definecolor{currentfill}{rgb}{0.121569,0.466667,0.705882}%
\pgfsetfillcolor{currentfill}%
\pgfsetlinewidth{1.003750pt}%
\definecolor{currentstroke}{rgb}{0.121569,0.466667,0.705882}%
\pgfsetstrokecolor{currentstroke}%
\pgfsetdash{}{0pt}%
\pgfsys@defobject{currentmarker}{\pgfqpoint{-0.009821in}{-0.009821in}}{\pgfqpoint{0.009821in}{0.009821in}}{%
\pgfpathmoveto{\pgfqpoint{0.000000in}{-0.009821in}}%
\pgfpathcurveto{\pgfqpoint{0.002605in}{-0.009821in}}{\pgfqpoint{0.005103in}{-0.008786in}}{\pgfqpoint{0.006944in}{-0.006944in}}%
\pgfpathcurveto{\pgfqpoint{0.008786in}{-0.005103in}}{\pgfqpoint{0.009821in}{-0.002605in}}{\pgfqpoint{0.009821in}{0.000000in}}%
\pgfpathcurveto{\pgfqpoint{0.009821in}{0.002605in}}{\pgfqpoint{0.008786in}{0.005103in}}{\pgfqpoint{0.006944in}{0.006944in}}%
\pgfpathcurveto{\pgfqpoint{0.005103in}{0.008786in}}{\pgfqpoint{0.002605in}{0.009821in}}{\pgfqpoint{0.000000in}{0.009821in}}%
\pgfpathcurveto{\pgfqpoint{-0.002605in}{0.009821in}}{\pgfqpoint{-0.005103in}{0.008786in}}{\pgfqpoint{-0.006944in}{0.006944in}}%
\pgfpathcurveto{\pgfqpoint{-0.008786in}{0.005103in}}{\pgfqpoint{-0.009821in}{0.002605in}}{\pgfqpoint{-0.009821in}{0.000000in}}%
\pgfpathcurveto{\pgfqpoint{-0.009821in}{-0.002605in}}{\pgfqpoint{-0.008786in}{-0.005103in}}{\pgfqpoint{-0.006944in}{-0.006944in}}%
\pgfpathcurveto{\pgfqpoint{-0.005103in}{-0.008786in}}{\pgfqpoint{-0.002605in}{-0.009821in}}{\pgfqpoint{0.000000in}{-0.009821in}}%
\pgfpathclose%
\pgfusepath{stroke,fill}%
}%
\begin{pgfscope}%
\pgfsys@transformshift{0.825965in}{0.500693in}%
\pgfsys@useobject{currentmarker}{}%
\end{pgfscope}%
\begin{pgfscope}%
\pgfsys@transformshift{0.970908in}{0.578303in}%
\pgfsys@useobject{currentmarker}{}%
\end{pgfscope}%
\begin{pgfscope}%
\pgfsys@transformshift{1.115850in}{0.660872in}%
\pgfsys@useobject{currentmarker}{}%
\end{pgfscope}%
\begin{pgfscope}%
\pgfsys@transformshift{1.260792in}{0.736222in}%
\pgfsys@useobject{currentmarker}{}%
\end{pgfscope}%
\begin{pgfscope}%
\pgfsys@transformshift{1.405735in}{0.855963in}%
\pgfsys@useobject{currentmarker}{}%
\end{pgfscope}%
\begin{pgfscope}%
\pgfsys@transformshift{1.550677in}{0.936276in}%
\pgfsys@useobject{currentmarker}{}%
\end{pgfscope}%
\begin{pgfscope}%
\pgfsys@transformshift{1.695619in}{1.042751in}%
\pgfsys@useobject{currentmarker}{}%
\end{pgfscope}%
\begin{pgfscope}%
\pgfsys@transformshift{1.840562in}{1.156016in}%
\pgfsys@useobject{currentmarker}{}%
\end{pgfscope}%
\begin{pgfscope}%
\pgfsys@transformshift{1.985504in}{1.320527in}%
\pgfsys@useobject{currentmarker}{}%
\end{pgfscope}%
\begin{pgfscope}%
\pgfsys@transformshift{2.130446in}{1.434372in}%
\pgfsys@useobject{currentmarker}{}%
\end{pgfscope}%
\begin{pgfscope}%
\pgfsys@transformshift{2.275389in}{1.536310in}%
\pgfsys@useobject{currentmarker}{}%
\end{pgfscope}%
\begin{pgfscope}%
\pgfsys@transformshift{2.420331in}{1.674270in}%
\pgfsys@useobject{currentmarker}{}%
\end{pgfscope}%
\begin{pgfscope}%
\pgfsys@transformshift{2.565273in}{1.795791in}%
\pgfsys@useobject{currentmarker}{}%
\end{pgfscope}%
\begin{pgfscope}%
\pgfsys@transformshift{2.710216in}{1.929276in}%
\pgfsys@useobject{currentmarker}{}%
\end{pgfscope}%
\begin{pgfscope}%
\pgfsys@transformshift{2.855158in}{2.075561in}%
\pgfsys@useobject{currentmarker}{}%
\end{pgfscope}%
\begin{pgfscope}%
\pgfsys@transformshift{3.000100in}{2.159457in}%
\pgfsys@useobject{currentmarker}{}%
\end{pgfscope}%
\begin{pgfscope}%
\pgfsys@transformshift{3.145043in}{2.254258in}%
\pgfsys@useobject{currentmarker}{}%
\end{pgfscope}%
\begin{pgfscope}%
\pgfsys@transformshift{3.289985in}{2.355720in}%
\pgfsys@useobject{currentmarker}{}%
\end{pgfscope}%
\begin{pgfscope}%
\pgfsys@transformshift{3.434927in}{2.424691in}%
\pgfsys@useobject{currentmarker}{}%
\end{pgfscope}%
\begin{pgfscope}%
\pgfsys@transformshift{3.579870in}{2.515938in}%
\pgfsys@useobject{currentmarker}{}%
\end{pgfscope}%
\begin{pgfscope}%
\pgfsys@transformshift{3.724812in}{2.591273in}%
\pgfsys@useobject{currentmarker}{}%
\end{pgfscope}%
\end{pgfscope}%
\begin{pgfscope}%
\pgfsetbuttcap%
\pgfsetroundjoin%
\definecolor{currentfill}{rgb}{0.000000,0.000000,0.000000}%
\pgfsetfillcolor{currentfill}%
\pgfsetlinewidth{0.803000pt}%
\definecolor{currentstroke}{rgb}{0.000000,0.000000,0.000000}%
\pgfsetstrokecolor{currentstroke}%
\pgfsetdash{}{0pt}%
\pgfsys@defobject{currentmarker}{\pgfqpoint{0.000000in}{-0.048611in}}{\pgfqpoint{0.000000in}{0.000000in}}{%
\pgfpathmoveto{\pgfqpoint{0.000000in}{0.000000in}}%
\pgfpathlineto{\pgfqpoint{0.000000in}{-0.048611in}}%
\pgfusepath{stroke,fill}%
}%
\begin{pgfscope}%
\pgfsys@transformshift{1.115850in}{0.387222in}%
\pgfsys@useobject{currentmarker}{}%
\end{pgfscope}%
\end{pgfscope}%
\begin{pgfscope}%
\definecolor{textcolor}{rgb}{0.000000,0.000000,0.000000}%
\pgfsetstrokecolor{textcolor}%
\pgfsetfillcolor{textcolor}%
\pgftext[x=1.115850in,y=0.290000in,,top]{\color{textcolor}\sffamily\fontsize{10.000000}{12.000000}\selectfont 2.20}%
\end{pgfscope}%
\begin{pgfscope}%
\pgfsetbuttcap%
\pgfsetroundjoin%
\definecolor{currentfill}{rgb}{0.000000,0.000000,0.000000}%
\pgfsetfillcolor{currentfill}%
\pgfsetlinewidth{0.803000pt}%
\definecolor{currentstroke}{rgb}{0.000000,0.000000,0.000000}%
\pgfsetstrokecolor{currentstroke}%
\pgfsetdash{}{0pt}%
\pgfsys@defobject{currentmarker}{\pgfqpoint{0.000000in}{-0.048611in}}{\pgfqpoint{0.000000in}{0.000000in}}{%
\pgfpathmoveto{\pgfqpoint{0.000000in}{0.000000in}}%
\pgfpathlineto{\pgfqpoint{0.000000in}{-0.048611in}}%
\pgfusepath{stroke,fill}%
}%
\begin{pgfscope}%
\pgfsys@transformshift{1.840562in}{0.387222in}%
\pgfsys@useobject{currentmarker}{}%
\end{pgfscope}%
\end{pgfscope}%
\begin{pgfscope}%
\definecolor{textcolor}{rgb}{0.000000,0.000000,0.000000}%
\pgfsetstrokecolor{textcolor}%
\pgfsetfillcolor{textcolor}%
\pgftext[x=1.840562in,y=0.290000in,,top]{\color{textcolor}\sffamily\fontsize{10.000000}{12.000000}\selectfont 2.25}%
\end{pgfscope}%
\begin{pgfscope}%
\pgfsetbuttcap%
\pgfsetroundjoin%
\definecolor{currentfill}{rgb}{0.000000,0.000000,0.000000}%
\pgfsetfillcolor{currentfill}%
\pgfsetlinewidth{0.803000pt}%
\definecolor{currentstroke}{rgb}{0.000000,0.000000,0.000000}%
\pgfsetstrokecolor{currentstroke}%
\pgfsetdash{}{0pt}%
\pgfsys@defobject{currentmarker}{\pgfqpoint{0.000000in}{-0.048611in}}{\pgfqpoint{0.000000in}{0.000000in}}{%
\pgfpathmoveto{\pgfqpoint{0.000000in}{0.000000in}}%
\pgfpathlineto{\pgfqpoint{0.000000in}{-0.048611in}}%
\pgfusepath{stroke,fill}%
}%
\begin{pgfscope}%
\pgfsys@transformshift{2.565273in}{0.387222in}%
\pgfsys@useobject{currentmarker}{}%
\end{pgfscope}%
\end{pgfscope}%
\begin{pgfscope}%
\definecolor{textcolor}{rgb}{0.000000,0.000000,0.000000}%
\pgfsetstrokecolor{textcolor}%
\pgfsetfillcolor{textcolor}%
\pgftext[x=2.565273in,y=0.290000in,,top]{\color{textcolor}\sffamily\fontsize{10.000000}{12.000000}\selectfont 2.30}%
\end{pgfscope}%
\begin{pgfscope}%
\pgfsetbuttcap%
\pgfsetroundjoin%
\definecolor{currentfill}{rgb}{0.000000,0.000000,0.000000}%
\pgfsetfillcolor{currentfill}%
\pgfsetlinewidth{0.803000pt}%
\definecolor{currentstroke}{rgb}{0.000000,0.000000,0.000000}%
\pgfsetstrokecolor{currentstroke}%
\pgfsetdash{}{0pt}%
\pgfsys@defobject{currentmarker}{\pgfqpoint{0.000000in}{-0.048611in}}{\pgfqpoint{0.000000in}{0.000000in}}{%
\pgfpathmoveto{\pgfqpoint{0.000000in}{0.000000in}}%
\pgfpathlineto{\pgfqpoint{0.000000in}{-0.048611in}}%
\pgfusepath{stroke,fill}%
}%
\begin{pgfscope}%
\pgfsys@transformshift{3.289985in}{0.387222in}%
\pgfsys@useobject{currentmarker}{}%
\end{pgfscope}%
\end{pgfscope}%
\begin{pgfscope}%
\definecolor{textcolor}{rgb}{0.000000,0.000000,0.000000}%
\pgfsetstrokecolor{textcolor}%
\pgfsetfillcolor{textcolor}%
\pgftext[x=3.289985in,y=0.290000in,,top]{\color{textcolor}\sffamily\fontsize{10.000000}{12.000000}\selectfont 2.35}%
\end{pgfscope}%
\begin{pgfscope}%
\pgfsetbuttcap%
\pgfsetroundjoin%
\definecolor{currentfill}{rgb}{0.000000,0.000000,0.000000}%
\pgfsetfillcolor{currentfill}%
\pgfsetlinewidth{0.803000pt}%
\definecolor{currentstroke}{rgb}{0.000000,0.000000,0.000000}%
\pgfsetstrokecolor{currentstroke}%
\pgfsetdash{}{0pt}%
\pgfsys@defobject{currentmarker}{\pgfqpoint{-0.048611in}{0.000000in}}{\pgfqpoint{0.000000in}{0.000000in}}{%
\pgfpathmoveto{\pgfqpoint{0.000000in}{0.000000in}}%
\pgfpathlineto{\pgfqpoint{-0.048611in}{0.000000in}}%
\pgfusepath{stroke,fill}%
}%
\begin{pgfscope}%
\pgfsys@transformshift{0.672222in}{0.633145in}%
\pgfsys@useobject{currentmarker}{}%
\end{pgfscope}%
\end{pgfscope}%
\begin{pgfscope}%
\definecolor{textcolor}{rgb}{0.000000,0.000000,0.000000}%
\pgfsetstrokecolor{textcolor}%
\pgfsetfillcolor{textcolor}%
\pgftext[x=0.149382in,y=0.580383in,left,base]{\color{textcolor}\sffamily\fontsize{10.000000}{12.000000}\selectfont −1.55}%
\end{pgfscope}%
\begin{pgfscope}%
\pgfsetbuttcap%
\pgfsetroundjoin%
\definecolor{currentfill}{rgb}{0.000000,0.000000,0.000000}%
\pgfsetfillcolor{currentfill}%
\pgfsetlinewidth{0.803000pt}%
\definecolor{currentstroke}{rgb}{0.000000,0.000000,0.000000}%
\pgfsetstrokecolor{currentstroke}%
\pgfsetdash{}{0pt}%
\pgfsys@defobject{currentmarker}{\pgfqpoint{-0.048611in}{0.000000in}}{\pgfqpoint{0.000000in}{0.000000in}}{%
\pgfpathmoveto{\pgfqpoint{0.000000in}{0.000000in}}%
\pgfpathlineto{\pgfqpoint{-0.048611in}{0.000000in}}%
\pgfusepath{stroke,fill}%
}%
\begin{pgfscope}%
\pgfsys@transformshift{0.672222in}{0.937072in}%
\pgfsys@useobject{currentmarker}{}%
\end{pgfscope}%
\end{pgfscope}%
\begin{pgfscope}%
\definecolor{textcolor}{rgb}{0.000000,0.000000,0.000000}%
\pgfsetstrokecolor{textcolor}%
\pgfsetfillcolor{textcolor}%
\pgftext[x=0.149382in,y=0.884311in,left,base]{\color{textcolor}\sffamily\fontsize{10.000000}{12.000000}\selectfont −1.50}%
\end{pgfscope}%
\begin{pgfscope}%
\pgfsetbuttcap%
\pgfsetroundjoin%
\definecolor{currentfill}{rgb}{0.000000,0.000000,0.000000}%
\pgfsetfillcolor{currentfill}%
\pgfsetlinewidth{0.803000pt}%
\definecolor{currentstroke}{rgb}{0.000000,0.000000,0.000000}%
\pgfsetstrokecolor{currentstroke}%
\pgfsetdash{}{0pt}%
\pgfsys@defobject{currentmarker}{\pgfqpoint{-0.048611in}{0.000000in}}{\pgfqpoint{0.000000in}{0.000000in}}{%
\pgfpathmoveto{\pgfqpoint{0.000000in}{0.000000in}}%
\pgfpathlineto{\pgfqpoint{-0.048611in}{0.000000in}}%
\pgfusepath{stroke,fill}%
}%
\begin{pgfscope}%
\pgfsys@transformshift{0.672222in}{1.241000in}%
\pgfsys@useobject{currentmarker}{}%
\end{pgfscope}%
\end{pgfscope}%
\begin{pgfscope}%
\definecolor{textcolor}{rgb}{0.000000,0.000000,0.000000}%
\pgfsetstrokecolor{textcolor}%
\pgfsetfillcolor{textcolor}%
\pgftext[x=0.149382in,y=1.188238in,left,base]{\color{textcolor}\sffamily\fontsize{10.000000}{12.000000}\selectfont −1.45}%
\end{pgfscope}%
\begin{pgfscope}%
\pgfsetbuttcap%
\pgfsetroundjoin%
\definecolor{currentfill}{rgb}{0.000000,0.000000,0.000000}%
\pgfsetfillcolor{currentfill}%
\pgfsetlinewidth{0.803000pt}%
\definecolor{currentstroke}{rgb}{0.000000,0.000000,0.000000}%
\pgfsetstrokecolor{currentstroke}%
\pgfsetdash{}{0pt}%
\pgfsys@defobject{currentmarker}{\pgfqpoint{-0.048611in}{0.000000in}}{\pgfqpoint{0.000000in}{0.000000in}}{%
\pgfpathmoveto{\pgfqpoint{0.000000in}{0.000000in}}%
\pgfpathlineto{\pgfqpoint{-0.048611in}{0.000000in}}%
\pgfusepath{stroke,fill}%
}%
\begin{pgfscope}%
\pgfsys@transformshift{0.672222in}{1.544928in}%
\pgfsys@useobject{currentmarker}{}%
\end{pgfscope}%
\end{pgfscope}%
\begin{pgfscope}%
\definecolor{textcolor}{rgb}{0.000000,0.000000,0.000000}%
\pgfsetstrokecolor{textcolor}%
\pgfsetfillcolor{textcolor}%
\pgftext[x=0.149382in,y=1.492166in,left,base]{\color{textcolor}\sffamily\fontsize{10.000000}{12.000000}\selectfont −1.40}%
\end{pgfscope}%
\begin{pgfscope}%
\pgfsetbuttcap%
\pgfsetroundjoin%
\definecolor{currentfill}{rgb}{0.000000,0.000000,0.000000}%
\pgfsetfillcolor{currentfill}%
\pgfsetlinewidth{0.803000pt}%
\definecolor{currentstroke}{rgb}{0.000000,0.000000,0.000000}%
\pgfsetstrokecolor{currentstroke}%
\pgfsetdash{}{0pt}%
\pgfsys@defobject{currentmarker}{\pgfqpoint{-0.048611in}{0.000000in}}{\pgfqpoint{0.000000in}{0.000000in}}{%
\pgfpathmoveto{\pgfqpoint{0.000000in}{0.000000in}}%
\pgfpathlineto{\pgfqpoint{-0.048611in}{0.000000in}}%
\pgfusepath{stroke,fill}%
}%
\begin{pgfscope}%
\pgfsys@transformshift{0.672222in}{1.848855in}%
\pgfsys@useobject{currentmarker}{}%
\end{pgfscope}%
\end{pgfscope}%
\begin{pgfscope}%
\definecolor{textcolor}{rgb}{0.000000,0.000000,0.000000}%
\pgfsetstrokecolor{textcolor}%
\pgfsetfillcolor{textcolor}%
\pgftext[x=0.149382in,y=1.796094in,left,base]{\color{textcolor}\sffamily\fontsize{10.000000}{12.000000}\selectfont −1.35}%
\end{pgfscope}%
\begin{pgfscope}%
\pgfsetbuttcap%
\pgfsetroundjoin%
\definecolor{currentfill}{rgb}{0.000000,0.000000,0.000000}%
\pgfsetfillcolor{currentfill}%
\pgfsetlinewidth{0.803000pt}%
\definecolor{currentstroke}{rgb}{0.000000,0.000000,0.000000}%
\pgfsetstrokecolor{currentstroke}%
\pgfsetdash{}{0pt}%
\pgfsys@defobject{currentmarker}{\pgfqpoint{-0.048611in}{0.000000in}}{\pgfqpoint{0.000000in}{0.000000in}}{%
\pgfpathmoveto{\pgfqpoint{0.000000in}{0.000000in}}%
\pgfpathlineto{\pgfqpoint{-0.048611in}{0.000000in}}%
\pgfusepath{stroke,fill}%
}%
\begin{pgfscope}%
\pgfsys@transformshift{0.672222in}{2.152783in}%
\pgfsys@useobject{currentmarker}{}%
\end{pgfscope}%
\end{pgfscope}%
\begin{pgfscope}%
\definecolor{textcolor}{rgb}{0.000000,0.000000,0.000000}%
\pgfsetstrokecolor{textcolor}%
\pgfsetfillcolor{textcolor}%
\pgftext[x=0.149382in,y=2.100021in,left,base]{\color{textcolor}\sffamily\fontsize{10.000000}{12.000000}\selectfont −1.30}%
\end{pgfscope}%
\begin{pgfscope}%
\pgfsetbuttcap%
\pgfsetroundjoin%
\definecolor{currentfill}{rgb}{0.000000,0.000000,0.000000}%
\pgfsetfillcolor{currentfill}%
\pgfsetlinewidth{0.803000pt}%
\definecolor{currentstroke}{rgb}{0.000000,0.000000,0.000000}%
\pgfsetstrokecolor{currentstroke}%
\pgfsetdash{}{0pt}%
\pgfsys@defobject{currentmarker}{\pgfqpoint{-0.048611in}{0.000000in}}{\pgfqpoint{0.000000in}{0.000000in}}{%
\pgfpathmoveto{\pgfqpoint{0.000000in}{0.000000in}}%
\pgfpathlineto{\pgfqpoint{-0.048611in}{0.000000in}}%
\pgfusepath{stroke,fill}%
}%
\begin{pgfscope}%
\pgfsys@transformshift{0.672222in}{2.456710in}%
\pgfsys@useobject{currentmarker}{}%
\end{pgfscope}%
\end{pgfscope}%
\begin{pgfscope}%
\definecolor{textcolor}{rgb}{0.000000,0.000000,0.000000}%
\pgfsetstrokecolor{textcolor}%
\pgfsetfillcolor{textcolor}%
\pgftext[x=0.149382in,y=2.403949in,left,base]{\color{textcolor}\sffamily\fontsize{10.000000}{12.000000}\selectfont −1.25}%
\end{pgfscope}%
\begin{pgfscope}%
\pgfpathrectangle{\pgfqpoint{0.672222in}{0.387222in}}{\pgfqpoint{3.206333in}{2.324444in}}%
\pgfusepath{clip}%
\pgfsetrectcap%
\pgfsetroundjoin%
\pgfsetlinewidth{1.505625pt}%
\definecolor{currentstroke}{rgb}{0.121569,0.466667,0.705882}%
\pgfsetstrokecolor{currentstroke}%
\pgfsetdash{}{0pt}%
\pgfpathmoveto{\pgfqpoint{0.825965in}{0.500693in}}%
\pgfpathlineto{\pgfqpoint{0.970908in}{0.578303in}}%
\pgfpathlineto{\pgfqpoint{1.115850in}{0.660872in}}%
\pgfpathlineto{\pgfqpoint{1.260792in}{0.736222in}}%
\pgfpathlineto{\pgfqpoint{1.405735in}{0.855963in}}%
\pgfpathlineto{\pgfqpoint{1.550677in}{0.936276in}}%
\pgfpathlineto{\pgfqpoint{1.695619in}{1.042751in}}%
\pgfpathlineto{\pgfqpoint{1.840562in}{1.156016in}}%
\pgfpathlineto{\pgfqpoint{1.985504in}{1.320527in}}%
\pgfpathlineto{\pgfqpoint{2.130446in}{1.434372in}}%
\pgfpathlineto{\pgfqpoint{2.275389in}{1.536310in}}%
\pgfpathlineto{\pgfqpoint{2.420331in}{1.674270in}}%
\pgfpathlineto{\pgfqpoint{2.565273in}{1.795791in}}%
\pgfpathlineto{\pgfqpoint{2.710216in}{1.929276in}}%
\pgfpathlineto{\pgfqpoint{2.855158in}{2.075561in}}%
\pgfpathlineto{\pgfqpoint{3.000100in}{2.159457in}}%
\pgfpathlineto{\pgfqpoint{3.145043in}{2.254258in}}%
\pgfpathlineto{\pgfqpoint{3.289985in}{2.355720in}}%
\pgfpathlineto{\pgfqpoint{3.434927in}{2.424691in}}%
\pgfpathlineto{\pgfqpoint{3.579870in}{2.515938in}}%
\pgfpathlineto{\pgfqpoint{3.724812in}{2.591273in}}%
\pgfusepath{stroke}%
\end{pgfscope}%
\begin{pgfscope}%
\pgfsetrectcap%
\pgfsetmiterjoin%
\pgfsetlinewidth{0.803000pt}%
\definecolor{currentstroke}{rgb}{0.000000,0.000000,0.000000}%
\pgfsetstrokecolor{currentstroke}%
\pgfsetdash{}{0pt}%
\pgfpathmoveto{\pgfqpoint{0.672222in}{0.387222in}}%
\pgfpathlineto{\pgfqpoint{0.672222in}{2.711667in}}%
\pgfusepath{stroke}%
\end{pgfscope}%
\begin{pgfscope}%
\pgfsetrectcap%
\pgfsetmiterjoin%
\pgfsetlinewidth{0.803000pt}%
\definecolor{currentstroke}{rgb}{0.000000,0.000000,0.000000}%
\pgfsetstrokecolor{currentstroke}%
\pgfsetdash{}{0pt}%
\pgfpathmoveto{\pgfqpoint{3.878555in}{0.387222in}}%
\pgfpathlineto{\pgfqpoint{3.878555in}{2.711667in}}%
\pgfusepath{stroke}%
\end{pgfscope}%
\begin{pgfscope}%
\pgfsetrectcap%
\pgfsetmiterjoin%
\pgfsetlinewidth{0.803000pt}%
\definecolor{currentstroke}{rgb}{0.000000,0.000000,0.000000}%
\pgfsetstrokecolor{currentstroke}%
\pgfsetdash{}{0pt}%
\pgfpathmoveto{\pgfqpoint{0.672222in}{0.387222in}}%
\pgfpathlineto{\pgfqpoint{3.878555in}{0.387222in}}%
\pgfusepath{stroke}%
\end{pgfscope}%
\begin{pgfscope}%
\pgfsetrectcap%
\pgfsetmiterjoin%
\pgfsetlinewidth{0.803000pt}%
\definecolor{currentstroke}{rgb}{0.000000,0.000000,0.000000}%
\pgfsetstrokecolor{currentstroke}%
\pgfsetdash{}{0pt}%
\pgfpathmoveto{\pgfqpoint{0.672222in}{2.711667in}}%
\pgfpathlineto{\pgfqpoint{3.878555in}{2.711667in}}%
\pgfusepath{stroke}%
\end{pgfscope}%
\begin{pgfscope}%
\definecolor{textcolor}{rgb}{0.000000,0.000000,0.000000}%
\pgfsetstrokecolor{textcolor}%
\pgfsetfillcolor{textcolor}%
\pgftext[x=2.275389in,y=2.795000in,,base]{\color{textcolor}\sffamily\fontsize{12.000000}{14.400000}\selectfont \(\displaystyle  N = 64 \)}%
\end{pgfscope}%
\begin{pgfscope}%
\pgfsetbuttcap%
\pgfsetmiterjoin%
\definecolor{currentfill}{rgb}{1.000000,1.000000,1.000000}%
\pgfsetfillcolor{currentfill}%
\pgfsetlinewidth{0.000000pt}%
\definecolor{currentstroke}{rgb}{0.000000,0.000000,0.000000}%
\pgfsetstrokecolor{currentstroke}%
\pgfsetstrokeopacity{0.000000}%
\pgfsetdash{}{0pt}%
\pgfpathmoveto{\pgfqpoint{4.498277in}{0.387222in}}%
\pgfpathlineto{\pgfqpoint{7.704610in}{0.387222in}}%
\pgfpathlineto{\pgfqpoint{7.704610in}{2.711667in}}%
\pgfpathlineto{\pgfqpoint{4.498277in}{2.711667in}}%
\pgfpathclose%
\pgfusepath{fill}%
\end{pgfscope}%
\begin{pgfscope}%
\pgfpathrectangle{\pgfqpoint{4.498277in}{0.387222in}}{\pgfqpoint{3.206333in}{2.324444in}}%
\pgfusepath{clip}%
\pgfsetbuttcap%
\pgfsetroundjoin%
\definecolor{currentfill}{rgb}{0.121569,0.466667,0.705882}%
\pgfsetfillcolor{currentfill}%
\pgfsetfillopacity{0.300000}%
\pgfsetlinewidth{1.003750pt}%
\definecolor{currentstroke}{rgb}{0.121569,0.466667,0.705882}%
\pgfsetstrokecolor{currentstroke}%
\pgfsetstrokeopacity{0.300000}%
\pgfsetdash{}{0pt}%
\pgfpathmoveto{\pgfqpoint{4.652021in}{0.515976in}}%
\pgfpathlineto{\pgfqpoint{4.652021in}{0.492879in}}%
\pgfpathlineto{\pgfqpoint{4.796963in}{0.573793in}}%
\pgfpathlineto{\pgfqpoint{4.941905in}{0.724675in}}%
\pgfpathlineto{\pgfqpoint{5.086848in}{0.742556in}}%
\pgfpathlineto{\pgfqpoint{5.231790in}{0.738306in}}%
\pgfpathlineto{\pgfqpoint{5.376732in}{0.853266in}}%
\pgfpathlineto{\pgfqpoint{5.521675in}{0.931648in}}%
\pgfpathlineto{\pgfqpoint{5.666617in}{1.136673in}}%
\pgfpathlineto{\pgfqpoint{5.811559in}{1.221992in}}%
\pgfpathlineto{\pgfqpoint{5.956502in}{1.247960in}}%
\pgfpathlineto{\pgfqpoint{6.101444in}{1.468028in}}%
\pgfpathlineto{\pgfqpoint{6.246386in}{1.578868in}}%
\pgfpathlineto{\pgfqpoint{6.391329in}{1.799283in}}%
\pgfpathlineto{\pgfqpoint{6.536271in}{1.886702in}}%
\pgfpathlineto{\pgfqpoint{6.681213in}{2.054477in}}%
\pgfpathlineto{\pgfqpoint{6.826156in}{2.156592in}}%
\pgfpathlineto{\pgfqpoint{6.971098in}{2.236641in}}%
\pgfpathlineto{\pgfqpoint{7.116040in}{2.334760in}}%
\pgfpathlineto{\pgfqpoint{7.260983in}{2.424897in}}%
\pgfpathlineto{\pgfqpoint{7.405925in}{2.501313in}}%
\pgfpathlineto{\pgfqpoint{7.550867in}{2.585953in}}%
\pgfpathlineto{\pgfqpoint{7.550867in}{2.606010in}}%
\pgfpathlineto{\pgfqpoint{7.550867in}{2.606010in}}%
\pgfpathlineto{\pgfqpoint{7.405925in}{2.556778in}}%
\pgfpathlineto{\pgfqpoint{7.260983in}{2.455479in}}%
\pgfpathlineto{\pgfqpoint{7.116040in}{2.366723in}}%
\pgfpathlineto{\pgfqpoint{6.971098in}{2.299863in}}%
\pgfpathlineto{\pgfqpoint{6.826156in}{2.201763in}}%
\pgfpathlineto{\pgfqpoint{6.681213in}{2.088589in}}%
\pgfpathlineto{\pgfqpoint{6.536271in}{2.007808in}}%
\pgfpathlineto{\pgfqpoint{6.391329in}{1.916915in}}%
\pgfpathlineto{\pgfqpoint{6.246386in}{1.677787in}}%
\pgfpathlineto{\pgfqpoint{6.101444in}{1.604918in}}%
\pgfpathlineto{\pgfqpoint{5.956502in}{1.568503in}}%
\pgfpathlineto{\pgfqpoint{5.811559in}{1.387616in}}%
\pgfpathlineto{\pgfqpoint{5.666617in}{1.212358in}}%
\pgfpathlineto{\pgfqpoint{5.521675in}{1.196647in}}%
\pgfpathlineto{\pgfqpoint{5.376732in}{1.166506in}}%
\pgfpathlineto{\pgfqpoint{5.231790in}{1.105677in}}%
\pgfpathlineto{\pgfqpoint{5.086848in}{0.773287in}}%
\pgfpathlineto{\pgfqpoint{4.941905in}{1.003794in}}%
\pgfpathlineto{\pgfqpoint{4.796963in}{0.614247in}}%
\pgfpathlineto{\pgfqpoint{4.652021in}{0.515976in}}%
\pgfpathclose%
\pgfusepath{stroke,fill}%
\end{pgfscope}%
\begin{pgfscope}%
\pgfpathrectangle{\pgfqpoint{4.498277in}{0.387222in}}{\pgfqpoint{3.206333in}{2.324444in}}%
\pgfusepath{clip}%
\pgfsetbuttcap%
\pgfsetroundjoin%
\definecolor{currentfill}{rgb}{0.121569,0.466667,0.705882}%
\pgfsetfillcolor{currentfill}%
\pgfsetlinewidth{1.003750pt}%
\definecolor{currentstroke}{rgb}{0.121569,0.466667,0.705882}%
\pgfsetstrokecolor{currentstroke}%
\pgfsetdash{}{0pt}%
\pgfsys@defobject{currentmarker}{\pgfqpoint{-0.009821in}{-0.009821in}}{\pgfqpoint{0.009821in}{0.009821in}}{%
\pgfpathmoveto{\pgfqpoint{0.000000in}{-0.009821in}}%
\pgfpathcurveto{\pgfqpoint{0.002605in}{-0.009821in}}{\pgfqpoint{0.005103in}{-0.008786in}}{\pgfqpoint{0.006944in}{-0.006944in}}%
\pgfpathcurveto{\pgfqpoint{0.008786in}{-0.005103in}}{\pgfqpoint{0.009821in}{-0.002605in}}{\pgfqpoint{0.009821in}{0.000000in}}%
\pgfpathcurveto{\pgfqpoint{0.009821in}{0.002605in}}{\pgfqpoint{0.008786in}{0.005103in}}{\pgfqpoint{0.006944in}{0.006944in}}%
\pgfpathcurveto{\pgfqpoint{0.005103in}{0.008786in}}{\pgfqpoint{0.002605in}{0.009821in}}{\pgfqpoint{0.000000in}{0.009821in}}%
\pgfpathcurveto{\pgfqpoint{-0.002605in}{0.009821in}}{\pgfqpoint{-0.005103in}{0.008786in}}{\pgfqpoint{-0.006944in}{0.006944in}}%
\pgfpathcurveto{\pgfqpoint{-0.008786in}{0.005103in}}{\pgfqpoint{-0.009821in}{0.002605in}}{\pgfqpoint{-0.009821in}{0.000000in}}%
\pgfpathcurveto{\pgfqpoint{-0.009821in}{-0.002605in}}{\pgfqpoint{-0.008786in}{-0.005103in}}{\pgfqpoint{-0.006944in}{-0.006944in}}%
\pgfpathcurveto{\pgfqpoint{-0.005103in}{-0.008786in}}{\pgfqpoint{-0.002605in}{-0.009821in}}{\pgfqpoint{0.000000in}{-0.009821in}}%
\pgfpathclose%
\pgfusepath{stroke,fill}%
}%
\begin{pgfscope}%
\pgfsys@transformshift{4.652021in}{0.504428in}%
\pgfsys@useobject{currentmarker}{}%
\end{pgfscope}%
\begin{pgfscope}%
\pgfsys@transformshift{4.796963in}{0.594020in}%
\pgfsys@useobject{currentmarker}{}%
\end{pgfscope}%
\begin{pgfscope}%
\pgfsys@transformshift{4.941905in}{0.864234in}%
\pgfsys@useobject{currentmarker}{}%
\end{pgfscope}%
\begin{pgfscope}%
\pgfsys@transformshift{5.086848in}{0.757921in}%
\pgfsys@useobject{currentmarker}{}%
\end{pgfscope}%
\begin{pgfscope}%
\pgfsys@transformshift{5.231790in}{0.921991in}%
\pgfsys@useobject{currentmarker}{}%
\end{pgfscope}%
\begin{pgfscope}%
\pgfsys@transformshift{5.376732in}{1.009886in}%
\pgfsys@useobject{currentmarker}{}%
\end{pgfscope}%
\begin{pgfscope}%
\pgfsys@transformshift{5.521675in}{1.064148in}%
\pgfsys@useobject{currentmarker}{}%
\end{pgfscope}%
\begin{pgfscope}%
\pgfsys@transformshift{5.666617in}{1.174515in}%
\pgfsys@useobject{currentmarker}{}%
\end{pgfscope}%
\begin{pgfscope}%
\pgfsys@transformshift{5.811559in}{1.304804in}%
\pgfsys@useobject{currentmarker}{}%
\end{pgfscope}%
\begin{pgfscope}%
\pgfsys@transformshift{5.956502in}{1.408231in}%
\pgfsys@useobject{currentmarker}{}%
\end{pgfscope}%
\begin{pgfscope}%
\pgfsys@transformshift{6.101444in}{1.536473in}%
\pgfsys@useobject{currentmarker}{}%
\end{pgfscope}%
\begin{pgfscope}%
\pgfsys@transformshift{6.246386in}{1.628328in}%
\pgfsys@useobject{currentmarker}{}%
\end{pgfscope}%
\begin{pgfscope}%
\pgfsys@transformshift{6.391329in}{1.858099in}%
\pgfsys@useobject{currentmarker}{}%
\end{pgfscope}%
\begin{pgfscope}%
\pgfsys@transformshift{6.536271in}{1.947255in}%
\pgfsys@useobject{currentmarker}{}%
\end{pgfscope}%
\begin{pgfscope}%
\pgfsys@transformshift{6.681213in}{2.071533in}%
\pgfsys@useobject{currentmarker}{}%
\end{pgfscope}%
\begin{pgfscope}%
\pgfsys@transformshift{6.826156in}{2.179177in}%
\pgfsys@useobject{currentmarker}{}%
\end{pgfscope}%
\begin{pgfscope}%
\pgfsys@transformshift{6.971098in}{2.268252in}%
\pgfsys@useobject{currentmarker}{}%
\end{pgfscope}%
\begin{pgfscope}%
\pgfsys@transformshift{7.116040in}{2.350742in}%
\pgfsys@useobject{currentmarker}{}%
\end{pgfscope}%
\begin{pgfscope}%
\pgfsys@transformshift{7.260983in}{2.440188in}%
\pgfsys@useobject{currentmarker}{}%
\end{pgfscope}%
\begin{pgfscope}%
\pgfsys@transformshift{7.405925in}{2.529045in}%
\pgfsys@useobject{currentmarker}{}%
\end{pgfscope}%
\begin{pgfscope}%
\pgfsys@transformshift{7.550867in}{2.595982in}%
\pgfsys@useobject{currentmarker}{}%
\end{pgfscope}%
\end{pgfscope}%
\begin{pgfscope}%
\pgfsetbuttcap%
\pgfsetroundjoin%
\definecolor{currentfill}{rgb}{0.000000,0.000000,0.000000}%
\pgfsetfillcolor{currentfill}%
\pgfsetlinewidth{0.803000pt}%
\definecolor{currentstroke}{rgb}{0.000000,0.000000,0.000000}%
\pgfsetstrokecolor{currentstroke}%
\pgfsetdash{}{0pt}%
\pgfsys@defobject{currentmarker}{\pgfqpoint{0.000000in}{-0.048611in}}{\pgfqpoint{0.000000in}{0.000000in}}{%
\pgfpathmoveto{\pgfqpoint{0.000000in}{0.000000in}}%
\pgfpathlineto{\pgfqpoint{0.000000in}{-0.048611in}}%
\pgfusepath{stroke,fill}%
}%
\begin{pgfscope}%
\pgfsys@transformshift{5.086848in}{0.387222in}%
\pgfsys@useobject{currentmarker}{}%
\end{pgfscope}%
\end{pgfscope}%
\begin{pgfscope}%
\definecolor{textcolor}{rgb}{0.000000,0.000000,0.000000}%
\pgfsetstrokecolor{textcolor}%
\pgfsetfillcolor{textcolor}%
\pgftext[x=5.086848in,y=0.290000in,,top]{\color{textcolor}\sffamily\fontsize{10.000000}{12.000000}\selectfont 2.20}%
\end{pgfscope}%
\begin{pgfscope}%
\pgfsetbuttcap%
\pgfsetroundjoin%
\definecolor{currentfill}{rgb}{0.000000,0.000000,0.000000}%
\pgfsetfillcolor{currentfill}%
\pgfsetlinewidth{0.803000pt}%
\definecolor{currentstroke}{rgb}{0.000000,0.000000,0.000000}%
\pgfsetstrokecolor{currentstroke}%
\pgfsetdash{}{0pt}%
\pgfsys@defobject{currentmarker}{\pgfqpoint{0.000000in}{-0.048611in}}{\pgfqpoint{0.000000in}{0.000000in}}{%
\pgfpathmoveto{\pgfqpoint{0.000000in}{0.000000in}}%
\pgfpathlineto{\pgfqpoint{0.000000in}{-0.048611in}}%
\pgfusepath{stroke,fill}%
}%
\begin{pgfscope}%
\pgfsys@transformshift{5.811559in}{0.387222in}%
\pgfsys@useobject{currentmarker}{}%
\end{pgfscope}%
\end{pgfscope}%
\begin{pgfscope}%
\definecolor{textcolor}{rgb}{0.000000,0.000000,0.000000}%
\pgfsetstrokecolor{textcolor}%
\pgfsetfillcolor{textcolor}%
\pgftext[x=5.811559in,y=0.290000in,,top]{\color{textcolor}\sffamily\fontsize{10.000000}{12.000000}\selectfont 2.25}%
\end{pgfscope}%
\begin{pgfscope}%
\pgfsetbuttcap%
\pgfsetroundjoin%
\definecolor{currentfill}{rgb}{0.000000,0.000000,0.000000}%
\pgfsetfillcolor{currentfill}%
\pgfsetlinewidth{0.803000pt}%
\definecolor{currentstroke}{rgb}{0.000000,0.000000,0.000000}%
\pgfsetstrokecolor{currentstroke}%
\pgfsetdash{}{0pt}%
\pgfsys@defobject{currentmarker}{\pgfqpoint{0.000000in}{-0.048611in}}{\pgfqpoint{0.000000in}{0.000000in}}{%
\pgfpathmoveto{\pgfqpoint{0.000000in}{0.000000in}}%
\pgfpathlineto{\pgfqpoint{0.000000in}{-0.048611in}}%
\pgfusepath{stroke,fill}%
}%
\begin{pgfscope}%
\pgfsys@transformshift{6.536271in}{0.387222in}%
\pgfsys@useobject{currentmarker}{}%
\end{pgfscope}%
\end{pgfscope}%
\begin{pgfscope}%
\definecolor{textcolor}{rgb}{0.000000,0.000000,0.000000}%
\pgfsetstrokecolor{textcolor}%
\pgfsetfillcolor{textcolor}%
\pgftext[x=6.536271in,y=0.290000in,,top]{\color{textcolor}\sffamily\fontsize{10.000000}{12.000000}\selectfont 2.30}%
\end{pgfscope}%
\begin{pgfscope}%
\pgfsetbuttcap%
\pgfsetroundjoin%
\definecolor{currentfill}{rgb}{0.000000,0.000000,0.000000}%
\pgfsetfillcolor{currentfill}%
\pgfsetlinewidth{0.803000pt}%
\definecolor{currentstroke}{rgb}{0.000000,0.000000,0.000000}%
\pgfsetstrokecolor{currentstroke}%
\pgfsetdash{}{0pt}%
\pgfsys@defobject{currentmarker}{\pgfqpoint{0.000000in}{-0.048611in}}{\pgfqpoint{0.000000in}{0.000000in}}{%
\pgfpathmoveto{\pgfqpoint{0.000000in}{0.000000in}}%
\pgfpathlineto{\pgfqpoint{0.000000in}{-0.048611in}}%
\pgfusepath{stroke,fill}%
}%
\begin{pgfscope}%
\pgfsys@transformshift{7.260983in}{0.387222in}%
\pgfsys@useobject{currentmarker}{}%
\end{pgfscope}%
\end{pgfscope}%
\begin{pgfscope}%
\definecolor{textcolor}{rgb}{0.000000,0.000000,0.000000}%
\pgfsetstrokecolor{textcolor}%
\pgfsetfillcolor{textcolor}%
\pgftext[x=7.260983in,y=0.290000in,,top]{\color{textcolor}\sffamily\fontsize{10.000000}{12.000000}\selectfont 2.35}%
\end{pgfscope}%
\begin{pgfscope}%
\pgfsetbuttcap%
\pgfsetroundjoin%
\definecolor{currentfill}{rgb}{0.000000,0.000000,0.000000}%
\pgfsetfillcolor{currentfill}%
\pgfsetlinewidth{0.803000pt}%
\definecolor{currentstroke}{rgb}{0.000000,0.000000,0.000000}%
\pgfsetstrokecolor{currentstroke}%
\pgfsetdash{}{0pt}%
\pgfsys@defobject{currentmarker}{\pgfqpoint{-0.048611in}{0.000000in}}{\pgfqpoint{0.000000in}{0.000000in}}{%
\pgfpathmoveto{\pgfqpoint{0.000000in}{0.000000in}}%
\pgfpathlineto{\pgfqpoint{-0.048611in}{0.000000in}}%
\pgfusepath{stroke,fill}%
}%
\begin{pgfscope}%
\pgfsys@transformshift{4.498277in}{0.423647in}%
\pgfsys@useobject{currentmarker}{}%
\end{pgfscope}%
\end{pgfscope}%
\begin{pgfscope}%
\definecolor{textcolor}{rgb}{0.000000,0.000000,0.000000}%
\pgfsetstrokecolor{textcolor}%
\pgfsetfillcolor{textcolor}%
\pgftext[x=3.975437in,y=0.370886in,left,base]{\color{textcolor}\sffamily\fontsize{10.000000}{12.000000}\selectfont −1.60}%
\end{pgfscope}%
\begin{pgfscope}%
\pgfsetbuttcap%
\pgfsetroundjoin%
\definecolor{currentfill}{rgb}{0.000000,0.000000,0.000000}%
\pgfsetfillcolor{currentfill}%
\pgfsetlinewidth{0.803000pt}%
\definecolor{currentstroke}{rgb}{0.000000,0.000000,0.000000}%
\pgfsetstrokecolor{currentstroke}%
\pgfsetdash{}{0pt}%
\pgfsys@defobject{currentmarker}{\pgfqpoint{-0.048611in}{0.000000in}}{\pgfqpoint{0.000000in}{0.000000in}}{%
\pgfpathmoveto{\pgfqpoint{0.000000in}{0.000000in}}%
\pgfpathlineto{\pgfqpoint{-0.048611in}{0.000000in}}%
\pgfusepath{stroke,fill}%
}%
\begin{pgfscope}%
\pgfsys@transformshift{4.498277in}{0.724676in}%
\pgfsys@useobject{currentmarker}{}%
\end{pgfscope}%
\end{pgfscope}%
\begin{pgfscope}%
\definecolor{textcolor}{rgb}{0.000000,0.000000,0.000000}%
\pgfsetstrokecolor{textcolor}%
\pgfsetfillcolor{textcolor}%
\pgftext[x=3.975437in,y=0.671914in,left,base]{\color{textcolor}\sffamily\fontsize{10.000000}{12.000000}\selectfont −1.55}%
\end{pgfscope}%
\begin{pgfscope}%
\pgfsetbuttcap%
\pgfsetroundjoin%
\definecolor{currentfill}{rgb}{0.000000,0.000000,0.000000}%
\pgfsetfillcolor{currentfill}%
\pgfsetlinewidth{0.803000pt}%
\definecolor{currentstroke}{rgb}{0.000000,0.000000,0.000000}%
\pgfsetstrokecolor{currentstroke}%
\pgfsetdash{}{0pt}%
\pgfsys@defobject{currentmarker}{\pgfqpoint{-0.048611in}{0.000000in}}{\pgfqpoint{0.000000in}{0.000000in}}{%
\pgfpathmoveto{\pgfqpoint{0.000000in}{0.000000in}}%
\pgfpathlineto{\pgfqpoint{-0.048611in}{0.000000in}}%
\pgfusepath{stroke,fill}%
}%
\begin{pgfscope}%
\pgfsys@transformshift{4.498277in}{1.025704in}%
\pgfsys@useobject{currentmarker}{}%
\end{pgfscope}%
\end{pgfscope}%
\begin{pgfscope}%
\definecolor{textcolor}{rgb}{0.000000,0.000000,0.000000}%
\pgfsetstrokecolor{textcolor}%
\pgfsetfillcolor{textcolor}%
\pgftext[x=3.975437in,y=0.972943in,left,base]{\color{textcolor}\sffamily\fontsize{10.000000}{12.000000}\selectfont −1.50}%
\end{pgfscope}%
\begin{pgfscope}%
\pgfsetbuttcap%
\pgfsetroundjoin%
\definecolor{currentfill}{rgb}{0.000000,0.000000,0.000000}%
\pgfsetfillcolor{currentfill}%
\pgfsetlinewidth{0.803000pt}%
\definecolor{currentstroke}{rgb}{0.000000,0.000000,0.000000}%
\pgfsetstrokecolor{currentstroke}%
\pgfsetdash{}{0pt}%
\pgfsys@defobject{currentmarker}{\pgfqpoint{-0.048611in}{0.000000in}}{\pgfqpoint{0.000000in}{0.000000in}}{%
\pgfpathmoveto{\pgfqpoint{0.000000in}{0.000000in}}%
\pgfpathlineto{\pgfqpoint{-0.048611in}{0.000000in}}%
\pgfusepath{stroke,fill}%
}%
\begin{pgfscope}%
\pgfsys@transformshift{4.498277in}{1.326733in}%
\pgfsys@useobject{currentmarker}{}%
\end{pgfscope}%
\end{pgfscope}%
\begin{pgfscope}%
\definecolor{textcolor}{rgb}{0.000000,0.000000,0.000000}%
\pgfsetstrokecolor{textcolor}%
\pgfsetfillcolor{textcolor}%
\pgftext[x=3.975437in,y=1.273972in,left,base]{\color{textcolor}\sffamily\fontsize{10.000000}{12.000000}\selectfont −1.45}%
\end{pgfscope}%
\begin{pgfscope}%
\pgfsetbuttcap%
\pgfsetroundjoin%
\definecolor{currentfill}{rgb}{0.000000,0.000000,0.000000}%
\pgfsetfillcolor{currentfill}%
\pgfsetlinewidth{0.803000pt}%
\definecolor{currentstroke}{rgb}{0.000000,0.000000,0.000000}%
\pgfsetstrokecolor{currentstroke}%
\pgfsetdash{}{0pt}%
\pgfsys@defobject{currentmarker}{\pgfqpoint{-0.048611in}{0.000000in}}{\pgfqpoint{0.000000in}{0.000000in}}{%
\pgfpathmoveto{\pgfqpoint{0.000000in}{0.000000in}}%
\pgfpathlineto{\pgfqpoint{-0.048611in}{0.000000in}}%
\pgfusepath{stroke,fill}%
}%
\begin{pgfscope}%
\pgfsys@transformshift{4.498277in}{1.627762in}%
\pgfsys@useobject{currentmarker}{}%
\end{pgfscope}%
\end{pgfscope}%
\begin{pgfscope}%
\definecolor{textcolor}{rgb}{0.000000,0.000000,0.000000}%
\pgfsetstrokecolor{textcolor}%
\pgfsetfillcolor{textcolor}%
\pgftext[x=3.975437in,y=1.575000in,left,base]{\color{textcolor}\sffamily\fontsize{10.000000}{12.000000}\selectfont −1.40}%
\end{pgfscope}%
\begin{pgfscope}%
\pgfsetbuttcap%
\pgfsetroundjoin%
\definecolor{currentfill}{rgb}{0.000000,0.000000,0.000000}%
\pgfsetfillcolor{currentfill}%
\pgfsetlinewidth{0.803000pt}%
\definecolor{currentstroke}{rgb}{0.000000,0.000000,0.000000}%
\pgfsetstrokecolor{currentstroke}%
\pgfsetdash{}{0pt}%
\pgfsys@defobject{currentmarker}{\pgfqpoint{-0.048611in}{0.000000in}}{\pgfqpoint{0.000000in}{0.000000in}}{%
\pgfpathmoveto{\pgfqpoint{0.000000in}{0.000000in}}%
\pgfpathlineto{\pgfqpoint{-0.048611in}{0.000000in}}%
\pgfusepath{stroke,fill}%
}%
\begin{pgfscope}%
\pgfsys@transformshift{4.498277in}{1.928790in}%
\pgfsys@useobject{currentmarker}{}%
\end{pgfscope}%
\end{pgfscope}%
\begin{pgfscope}%
\definecolor{textcolor}{rgb}{0.000000,0.000000,0.000000}%
\pgfsetstrokecolor{textcolor}%
\pgfsetfillcolor{textcolor}%
\pgftext[x=3.975437in,y=1.876029in,left,base]{\color{textcolor}\sffamily\fontsize{10.000000}{12.000000}\selectfont −1.35}%
\end{pgfscope}%
\begin{pgfscope}%
\pgfsetbuttcap%
\pgfsetroundjoin%
\definecolor{currentfill}{rgb}{0.000000,0.000000,0.000000}%
\pgfsetfillcolor{currentfill}%
\pgfsetlinewidth{0.803000pt}%
\definecolor{currentstroke}{rgb}{0.000000,0.000000,0.000000}%
\pgfsetstrokecolor{currentstroke}%
\pgfsetdash{}{0pt}%
\pgfsys@defobject{currentmarker}{\pgfqpoint{-0.048611in}{0.000000in}}{\pgfqpoint{0.000000in}{0.000000in}}{%
\pgfpathmoveto{\pgfqpoint{0.000000in}{0.000000in}}%
\pgfpathlineto{\pgfqpoint{-0.048611in}{0.000000in}}%
\pgfusepath{stroke,fill}%
}%
\begin{pgfscope}%
\pgfsys@transformshift{4.498277in}{2.229819in}%
\pgfsys@useobject{currentmarker}{}%
\end{pgfscope}%
\end{pgfscope}%
\begin{pgfscope}%
\definecolor{textcolor}{rgb}{0.000000,0.000000,0.000000}%
\pgfsetstrokecolor{textcolor}%
\pgfsetfillcolor{textcolor}%
\pgftext[x=3.975437in,y=2.177057in,left,base]{\color{textcolor}\sffamily\fontsize{10.000000}{12.000000}\selectfont −1.30}%
\end{pgfscope}%
\begin{pgfscope}%
\pgfsetbuttcap%
\pgfsetroundjoin%
\definecolor{currentfill}{rgb}{0.000000,0.000000,0.000000}%
\pgfsetfillcolor{currentfill}%
\pgfsetlinewidth{0.803000pt}%
\definecolor{currentstroke}{rgb}{0.000000,0.000000,0.000000}%
\pgfsetstrokecolor{currentstroke}%
\pgfsetdash{}{0pt}%
\pgfsys@defobject{currentmarker}{\pgfqpoint{-0.048611in}{0.000000in}}{\pgfqpoint{0.000000in}{0.000000in}}{%
\pgfpathmoveto{\pgfqpoint{0.000000in}{0.000000in}}%
\pgfpathlineto{\pgfqpoint{-0.048611in}{0.000000in}}%
\pgfusepath{stroke,fill}%
}%
\begin{pgfscope}%
\pgfsys@transformshift{4.498277in}{2.530847in}%
\pgfsys@useobject{currentmarker}{}%
\end{pgfscope}%
\end{pgfscope}%
\begin{pgfscope}%
\definecolor{textcolor}{rgb}{0.000000,0.000000,0.000000}%
\pgfsetstrokecolor{textcolor}%
\pgfsetfillcolor{textcolor}%
\pgftext[x=3.975437in,y=2.478086in,left,base]{\color{textcolor}\sffamily\fontsize{10.000000}{12.000000}\selectfont −1.25}%
\end{pgfscope}%
\begin{pgfscope}%
\pgfpathrectangle{\pgfqpoint{4.498277in}{0.387222in}}{\pgfqpoint{3.206333in}{2.324444in}}%
\pgfusepath{clip}%
\pgfsetrectcap%
\pgfsetroundjoin%
\pgfsetlinewidth{1.505625pt}%
\definecolor{currentstroke}{rgb}{0.121569,0.466667,0.705882}%
\pgfsetstrokecolor{currentstroke}%
\pgfsetdash{}{0pt}%
\pgfpathmoveto{\pgfqpoint{4.652021in}{0.504428in}}%
\pgfpathlineto{\pgfqpoint{4.796963in}{0.594020in}}%
\pgfpathlineto{\pgfqpoint{4.941905in}{0.864234in}}%
\pgfpathlineto{\pgfqpoint{5.086848in}{0.757921in}}%
\pgfpathlineto{\pgfqpoint{5.231790in}{0.921991in}}%
\pgfpathlineto{\pgfqpoint{5.376732in}{1.009886in}}%
\pgfpathlineto{\pgfqpoint{5.521675in}{1.064148in}}%
\pgfpathlineto{\pgfqpoint{5.666617in}{1.174515in}}%
\pgfpathlineto{\pgfqpoint{5.811559in}{1.304804in}}%
\pgfpathlineto{\pgfqpoint{5.956502in}{1.408231in}}%
\pgfpathlineto{\pgfqpoint{6.101444in}{1.536473in}}%
\pgfpathlineto{\pgfqpoint{6.246386in}{1.628328in}}%
\pgfpathlineto{\pgfqpoint{6.391329in}{1.858099in}}%
\pgfpathlineto{\pgfqpoint{6.536271in}{1.947255in}}%
\pgfpathlineto{\pgfqpoint{6.681213in}{2.071533in}}%
\pgfpathlineto{\pgfqpoint{6.826156in}{2.179177in}}%
\pgfpathlineto{\pgfqpoint{6.971098in}{2.268252in}}%
\pgfpathlineto{\pgfqpoint{7.116040in}{2.350742in}}%
\pgfpathlineto{\pgfqpoint{7.260983in}{2.440188in}}%
\pgfpathlineto{\pgfqpoint{7.405925in}{2.529045in}}%
\pgfpathlineto{\pgfqpoint{7.550867in}{2.595982in}}%
\pgfusepath{stroke}%
\end{pgfscope}%
\begin{pgfscope}%
\pgfsetrectcap%
\pgfsetmiterjoin%
\pgfsetlinewidth{0.803000pt}%
\definecolor{currentstroke}{rgb}{0.000000,0.000000,0.000000}%
\pgfsetstrokecolor{currentstroke}%
\pgfsetdash{}{0pt}%
\pgfpathmoveto{\pgfqpoint{4.498277in}{0.387222in}}%
\pgfpathlineto{\pgfqpoint{4.498277in}{2.711667in}}%
\pgfusepath{stroke}%
\end{pgfscope}%
\begin{pgfscope}%
\pgfsetrectcap%
\pgfsetmiterjoin%
\pgfsetlinewidth{0.803000pt}%
\definecolor{currentstroke}{rgb}{0.000000,0.000000,0.000000}%
\pgfsetstrokecolor{currentstroke}%
\pgfsetdash{}{0pt}%
\pgfpathmoveto{\pgfqpoint{7.704610in}{0.387222in}}%
\pgfpathlineto{\pgfqpoint{7.704610in}{2.711667in}}%
\pgfusepath{stroke}%
\end{pgfscope}%
\begin{pgfscope}%
\pgfsetrectcap%
\pgfsetmiterjoin%
\pgfsetlinewidth{0.803000pt}%
\definecolor{currentstroke}{rgb}{0.000000,0.000000,0.000000}%
\pgfsetstrokecolor{currentstroke}%
\pgfsetdash{}{0pt}%
\pgfpathmoveto{\pgfqpoint{4.498277in}{0.387222in}}%
\pgfpathlineto{\pgfqpoint{7.704610in}{0.387222in}}%
\pgfusepath{stroke}%
\end{pgfscope}%
\begin{pgfscope}%
\pgfsetrectcap%
\pgfsetmiterjoin%
\pgfsetlinewidth{0.803000pt}%
\definecolor{currentstroke}{rgb}{0.000000,0.000000,0.000000}%
\pgfsetstrokecolor{currentstroke}%
\pgfsetdash{}{0pt}%
\pgfpathmoveto{\pgfqpoint{4.498277in}{2.711667in}}%
\pgfpathlineto{\pgfqpoint{7.704610in}{2.711667in}}%
\pgfusepath{stroke}%
\end{pgfscope}%
\begin{pgfscope}%
\definecolor{textcolor}{rgb}{0.000000,0.000000,0.000000}%
\pgfsetstrokecolor{textcolor}%
\pgfsetfillcolor{textcolor}%
\pgftext[x=6.101444in,y=2.795000in,,base]{\color{textcolor}\sffamily\fontsize{12.000000}{14.400000}\selectfont \(\displaystyle  N = 128 \)}%
\end{pgfscope}%
\end{pgfpicture}%
\makeatother%
\endgroup%
}
\caption{Internal energy $u$ for finer temperature $T$ using Metropolis--Hastings algorithm}
\label{Fig:HeatBig}
\end{figure}

\begin{figure}[htbp]
\centering
\scalebox{0.666}{%% Creator: Matplotlib, PGF backend
%%
%% To include the figure in your LaTeX document, write
%%   \input{<filename>.pgf}
%%
%% Make sure the required packages are loaded in your preamble
%%   \usepackage{pgf}
%%
%% Figures using additional raster images can only be included by \input if
%% they are in the same directory as the main LaTeX file. For loading figures
%% from other directories you can use the `import` package
%%   \usepackage{import}
%% and then include the figures with
%%   \import{<path to file>}{<filename>.pgf}
%%
%% Matplotlib used the following preamble
%%   \usepackage{fontspec}
%%   \setmainfont{DejaVuSerif.ttf}[Path=/home/lzh/anaconda3/envs/numana/lib/python3.7/site-packages/matplotlib/mpl-data/fonts/ttf/]
%%   \setsansfont{DejaVuSans.ttf}[Path=/home/lzh/anaconda3/envs/numana/lib/python3.7/site-packages/matplotlib/mpl-data/fonts/ttf/]
%%   \setmonofont{DejaVuSansMono.ttf}[Path=/home/lzh/anaconda3/envs/numana/lib/python3.7/site-packages/matplotlib/mpl-data/fonts/ttf/]
%%
\begingroup%
\makeatletter%
\begin{pgfpicture}%
\pgfpathrectangle{\pgfpointorigin}{\pgfqpoint{8.000000in}{6.000000in}}%
\pgfusepath{use as bounding box, clip}%
\begin{pgfscope}%
\pgfsetbuttcap%
\pgfsetmiterjoin%
\definecolor{currentfill}{rgb}{1.000000,1.000000,1.000000}%
\pgfsetfillcolor{currentfill}%
\pgfsetlinewidth{0.000000pt}%
\definecolor{currentstroke}{rgb}{1.000000,1.000000,1.000000}%
\pgfsetstrokecolor{currentstroke}%
\pgfsetdash{}{0pt}%
\pgfpathmoveto{\pgfqpoint{0.000000in}{0.000000in}}%
\pgfpathlineto{\pgfqpoint{8.000000in}{0.000000in}}%
\pgfpathlineto{\pgfqpoint{8.000000in}{6.000000in}}%
\pgfpathlineto{\pgfqpoint{0.000000in}{6.000000in}}%
\pgfpathclose%
\pgfusepath{fill}%
\end{pgfscope}%
\begin{pgfscope}%
\pgfsetbuttcap%
\pgfsetmiterjoin%
\definecolor{currentfill}{rgb}{1.000000,1.000000,1.000000}%
\pgfsetfillcolor{currentfill}%
\pgfsetlinewidth{0.000000pt}%
\definecolor{currentstroke}{rgb}{0.000000,0.000000,0.000000}%
\pgfsetstrokecolor{currentstroke}%
\pgfsetstrokeopacity{0.000000}%
\pgfsetdash{}{0pt}%
\pgfpathmoveto{\pgfqpoint{0.557986in}{3.311111in}}%
\pgfpathlineto{\pgfqpoint{3.873503in}{3.311111in}}%
\pgfpathlineto{\pgfqpoint{3.873503in}{5.627778in}}%
\pgfpathlineto{\pgfqpoint{0.557986in}{5.627778in}}%
\pgfpathclose%
\pgfusepath{fill}%
\end{pgfscope}%
\begin{pgfscope}%
\pgfpathrectangle{\pgfqpoint{0.557986in}{3.311111in}}{\pgfqpoint{3.315517in}{2.316667in}}%
\pgfusepath{clip}%
\pgfsetbuttcap%
\pgfsetroundjoin%
\definecolor{currentfill}{rgb}{0.121569,0.466667,0.705882}%
\pgfsetfillcolor{currentfill}%
\pgfsetfillopacity{0.300000}%
\pgfsetlinewidth{1.003750pt}%
\definecolor{currentstroke}{rgb}{0.121569,0.466667,0.705882}%
\pgfsetstrokecolor{currentstroke}%
\pgfsetstrokeopacity{0.300000}%
\pgfsetdash{}{0pt}%
\pgfpathmoveto{\pgfqpoint{0.720158in}{3.923836in}}%
\pgfpathlineto{\pgfqpoint{0.720158in}{3.528434in}}%
\pgfpathlineto{\pgfqpoint{0.869716in}{3.763009in}}%
\pgfpathlineto{\pgfqpoint{1.019275in}{4.025417in}}%
\pgfpathlineto{\pgfqpoint{1.168834in}{4.191447in}}%
\pgfpathlineto{\pgfqpoint{1.318392in}{4.719395in}}%
\pgfpathlineto{\pgfqpoint{1.467951in}{4.781316in}}%
\pgfpathlineto{\pgfqpoint{1.617510in}{4.899102in}}%
\pgfpathlineto{\pgfqpoint{1.767068in}{5.109541in}}%
\pgfpathlineto{\pgfqpoint{1.916627in}{5.303725in}}%
\pgfpathlineto{\pgfqpoint{2.066186in}{5.305942in}}%
\pgfpathlineto{\pgfqpoint{2.215744in}{5.399204in}}%
\pgfpathlineto{\pgfqpoint{2.365303in}{5.231101in}}%
\pgfpathlineto{\pgfqpoint{2.514862in}{5.178386in}}%
\pgfpathlineto{\pgfqpoint{2.664421in}{5.128049in}}%
\pgfpathlineto{\pgfqpoint{2.813979in}{4.954500in}}%
\pgfpathlineto{\pgfqpoint{2.963538in}{4.817328in}}%
\pgfpathlineto{\pgfqpoint{3.113097in}{4.528328in}}%
\pgfpathlineto{\pgfqpoint{3.262655in}{4.264225in}}%
\pgfpathlineto{\pgfqpoint{3.412214in}{3.969196in}}%
\pgfpathlineto{\pgfqpoint{3.561773in}{3.670060in}}%
\pgfpathlineto{\pgfqpoint{3.711331in}{3.416414in}}%
\pgfpathlineto{\pgfqpoint{3.711331in}{3.507837in}}%
\pgfpathlineto{\pgfqpoint{3.711331in}{3.507837in}}%
\pgfpathlineto{\pgfqpoint{3.561773in}{3.850288in}}%
\pgfpathlineto{\pgfqpoint{3.412214in}{4.030064in}}%
\pgfpathlineto{\pgfqpoint{3.262655in}{4.406553in}}%
\pgfpathlineto{\pgfqpoint{3.113097in}{4.669333in}}%
\pgfpathlineto{\pgfqpoint{2.963538in}{4.921249in}}%
\pgfpathlineto{\pgfqpoint{2.813979in}{5.025283in}}%
\pgfpathlineto{\pgfqpoint{2.664421in}{5.274325in}}%
\pgfpathlineto{\pgfqpoint{2.514862in}{5.381001in}}%
\pgfpathlineto{\pgfqpoint{2.365303in}{5.390379in}}%
\pgfpathlineto{\pgfqpoint{2.215744in}{5.522475in}}%
\pgfpathlineto{\pgfqpoint{2.066186in}{5.437792in}}%
\pgfpathlineto{\pgfqpoint{1.916627in}{5.489489in}}%
\pgfpathlineto{\pgfqpoint{1.767068in}{5.496883in}}%
\pgfpathlineto{\pgfqpoint{1.617510in}{5.135801in}}%
\pgfpathlineto{\pgfqpoint{1.467951in}{4.925214in}}%
\pgfpathlineto{\pgfqpoint{1.318392in}{4.870668in}}%
\pgfpathlineto{\pgfqpoint{1.168834in}{4.651703in}}%
\pgfpathlineto{\pgfqpoint{1.019275in}{4.340887in}}%
\pgfpathlineto{\pgfqpoint{0.869716in}{4.226989in}}%
\pgfpathlineto{\pgfqpoint{0.720158in}{3.923836in}}%
\pgfpathclose%
\pgfusepath{stroke,fill}%
\end{pgfscope}%
\begin{pgfscope}%
\pgfpathrectangle{\pgfqpoint{0.557986in}{3.311111in}}{\pgfqpoint{3.315517in}{2.316667in}}%
\pgfusepath{clip}%
\pgfsetbuttcap%
\pgfsetroundjoin%
\definecolor{currentfill}{rgb}{0.121569,0.466667,0.705882}%
\pgfsetfillcolor{currentfill}%
\pgfsetlinewidth{1.003750pt}%
\definecolor{currentstroke}{rgb}{0.121569,0.466667,0.705882}%
\pgfsetstrokecolor{currentstroke}%
\pgfsetdash{}{0pt}%
\pgfsys@defobject{currentmarker}{\pgfqpoint{-0.009821in}{-0.009821in}}{\pgfqpoint{0.009821in}{0.009821in}}{%
\pgfpathmoveto{\pgfqpoint{0.000000in}{-0.009821in}}%
\pgfpathcurveto{\pgfqpoint{0.002605in}{-0.009821in}}{\pgfqpoint{0.005103in}{-0.008786in}}{\pgfqpoint{0.006944in}{-0.006944in}}%
\pgfpathcurveto{\pgfqpoint{0.008786in}{-0.005103in}}{\pgfqpoint{0.009821in}{-0.002605in}}{\pgfqpoint{0.009821in}{0.000000in}}%
\pgfpathcurveto{\pgfqpoint{0.009821in}{0.002605in}}{\pgfqpoint{0.008786in}{0.005103in}}{\pgfqpoint{0.006944in}{0.006944in}}%
\pgfpathcurveto{\pgfqpoint{0.005103in}{0.008786in}}{\pgfqpoint{0.002605in}{0.009821in}}{\pgfqpoint{0.000000in}{0.009821in}}%
\pgfpathcurveto{\pgfqpoint{-0.002605in}{0.009821in}}{\pgfqpoint{-0.005103in}{0.008786in}}{\pgfqpoint{-0.006944in}{0.006944in}}%
\pgfpathcurveto{\pgfqpoint{-0.008786in}{0.005103in}}{\pgfqpoint{-0.009821in}{0.002605in}}{\pgfqpoint{-0.009821in}{0.000000in}}%
\pgfpathcurveto{\pgfqpoint{-0.009821in}{-0.002605in}}{\pgfqpoint{-0.008786in}{-0.005103in}}{\pgfqpoint{-0.006944in}{-0.006944in}}%
\pgfpathcurveto{\pgfqpoint{-0.005103in}{-0.008786in}}{\pgfqpoint{-0.002605in}{-0.009821in}}{\pgfqpoint{0.000000in}{-0.009821in}}%
\pgfpathclose%
\pgfusepath{stroke,fill}%
}%
\begin{pgfscope}%
\pgfsys@transformshift{0.720158in}{3.726135in}%
\pgfsys@useobject{currentmarker}{}%
\end{pgfscope}%
\begin{pgfscope}%
\pgfsys@transformshift{0.869716in}{3.994999in}%
\pgfsys@useobject{currentmarker}{}%
\end{pgfscope}%
\begin{pgfscope}%
\pgfsys@transformshift{1.019275in}{4.183152in}%
\pgfsys@useobject{currentmarker}{}%
\end{pgfscope}%
\begin{pgfscope}%
\pgfsys@transformshift{1.168834in}{4.421575in}%
\pgfsys@useobject{currentmarker}{}%
\end{pgfscope}%
\begin{pgfscope}%
\pgfsys@transformshift{1.318392in}{4.795032in}%
\pgfsys@useobject{currentmarker}{}%
\end{pgfscope}%
\begin{pgfscope}%
\pgfsys@transformshift{1.467951in}{4.853265in}%
\pgfsys@useobject{currentmarker}{}%
\end{pgfscope}%
\begin{pgfscope}%
\pgfsys@transformshift{1.617510in}{5.017452in}%
\pgfsys@useobject{currentmarker}{}%
\end{pgfscope}%
\begin{pgfscope}%
\pgfsys@transformshift{1.767068in}{5.303212in}%
\pgfsys@useobject{currentmarker}{}%
\end{pgfscope}%
\begin{pgfscope}%
\pgfsys@transformshift{1.916627in}{5.396607in}%
\pgfsys@useobject{currentmarker}{}%
\end{pgfscope}%
\begin{pgfscope}%
\pgfsys@transformshift{2.066186in}{5.371867in}%
\pgfsys@useobject{currentmarker}{}%
\end{pgfscope}%
\begin{pgfscope}%
\pgfsys@transformshift{2.215744in}{5.460839in}%
\pgfsys@useobject{currentmarker}{}%
\end{pgfscope}%
\begin{pgfscope}%
\pgfsys@transformshift{2.365303in}{5.310740in}%
\pgfsys@useobject{currentmarker}{}%
\end{pgfscope}%
\begin{pgfscope}%
\pgfsys@transformshift{2.514862in}{5.279693in}%
\pgfsys@useobject{currentmarker}{}%
\end{pgfscope}%
\begin{pgfscope}%
\pgfsys@transformshift{2.664421in}{5.201187in}%
\pgfsys@useobject{currentmarker}{}%
\end{pgfscope}%
\begin{pgfscope}%
\pgfsys@transformshift{2.813979in}{4.989892in}%
\pgfsys@useobject{currentmarker}{}%
\end{pgfscope}%
\begin{pgfscope}%
\pgfsys@transformshift{2.963538in}{4.869289in}%
\pgfsys@useobject{currentmarker}{}%
\end{pgfscope}%
\begin{pgfscope}%
\pgfsys@transformshift{3.113097in}{4.598831in}%
\pgfsys@useobject{currentmarker}{}%
\end{pgfscope}%
\begin{pgfscope}%
\pgfsys@transformshift{3.262655in}{4.335389in}%
\pgfsys@useobject{currentmarker}{}%
\end{pgfscope}%
\begin{pgfscope}%
\pgfsys@transformshift{3.412214in}{3.999630in}%
\pgfsys@useobject{currentmarker}{}%
\end{pgfscope}%
\begin{pgfscope}%
\pgfsys@transformshift{3.561773in}{3.760174in}%
\pgfsys@useobject{currentmarker}{}%
\end{pgfscope}%
\begin{pgfscope}%
\pgfsys@transformshift{3.711331in}{3.462126in}%
\pgfsys@useobject{currentmarker}{}%
\end{pgfscope}%
\end{pgfscope}%
\begin{pgfscope}%
\pgfsetbuttcap%
\pgfsetroundjoin%
\definecolor{currentfill}{rgb}{0.000000,0.000000,0.000000}%
\pgfsetfillcolor{currentfill}%
\pgfsetlinewidth{0.803000pt}%
\definecolor{currentstroke}{rgb}{0.000000,0.000000,0.000000}%
\pgfsetstrokecolor{currentstroke}%
\pgfsetdash{}{0pt}%
\pgfsys@defobject{currentmarker}{\pgfqpoint{0.000000in}{-0.048611in}}{\pgfqpoint{0.000000in}{0.000000in}}{%
\pgfpathmoveto{\pgfqpoint{0.000000in}{0.000000in}}%
\pgfpathlineto{\pgfqpoint{0.000000in}{-0.048611in}}%
\pgfusepath{stroke,fill}%
}%
\begin{pgfscope}%
\pgfsys@transformshift{1.168834in}{3.311111in}%
\pgfsys@useobject{currentmarker}{}%
\end{pgfscope}%
\end{pgfscope}%
\begin{pgfscope}%
\definecolor{textcolor}{rgb}{0.000000,0.000000,0.000000}%
\pgfsetstrokecolor{textcolor}%
\pgfsetfillcolor{textcolor}%
\pgftext[x=1.168834in,y=3.213889in,,top]{\color{textcolor}\sffamily\fontsize{10.000000}{12.000000}\selectfont 2.25}%
\end{pgfscope}%
\begin{pgfscope}%
\pgfsetbuttcap%
\pgfsetroundjoin%
\definecolor{currentfill}{rgb}{0.000000,0.000000,0.000000}%
\pgfsetfillcolor{currentfill}%
\pgfsetlinewidth{0.803000pt}%
\definecolor{currentstroke}{rgb}{0.000000,0.000000,0.000000}%
\pgfsetstrokecolor{currentstroke}%
\pgfsetdash{}{0pt}%
\pgfsys@defobject{currentmarker}{\pgfqpoint{0.000000in}{-0.048611in}}{\pgfqpoint{0.000000in}{0.000000in}}{%
\pgfpathmoveto{\pgfqpoint{0.000000in}{0.000000in}}%
\pgfpathlineto{\pgfqpoint{0.000000in}{-0.048611in}}%
\pgfusepath{stroke,fill}%
}%
\begin{pgfscope}%
\pgfsys@transformshift{1.916627in}{3.311111in}%
\pgfsys@useobject{currentmarker}{}%
\end{pgfscope}%
\end{pgfscope}%
\begin{pgfscope}%
\definecolor{textcolor}{rgb}{0.000000,0.000000,0.000000}%
\pgfsetstrokecolor{textcolor}%
\pgfsetfillcolor{textcolor}%
\pgftext[x=1.916627in,y=3.213889in,,top]{\color{textcolor}\sffamily\fontsize{10.000000}{12.000000}\selectfont 2.30}%
\end{pgfscope}%
\begin{pgfscope}%
\pgfsetbuttcap%
\pgfsetroundjoin%
\definecolor{currentfill}{rgb}{0.000000,0.000000,0.000000}%
\pgfsetfillcolor{currentfill}%
\pgfsetlinewidth{0.803000pt}%
\definecolor{currentstroke}{rgb}{0.000000,0.000000,0.000000}%
\pgfsetstrokecolor{currentstroke}%
\pgfsetdash{}{0pt}%
\pgfsys@defobject{currentmarker}{\pgfqpoint{0.000000in}{-0.048611in}}{\pgfqpoint{0.000000in}{0.000000in}}{%
\pgfpathmoveto{\pgfqpoint{0.000000in}{0.000000in}}%
\pgfpathlineto{\pgfqpoint{0.000000in}{-0.048611in}}%
\pgfusepath{stroke,fill}%
}%
\begin{pgfscope}%
\pgfsys@transformshift{2.664421in}{3.311111in}%
\pgfsys@useobject{currentmarker}{}%
\end{pgfscope}%
\end{pgfscope}%
\begin{pgfscope}%
\definecolor{textcolor}{rgb}{0.000000,0.000000,0.000000}%
\pgfsetstrokecolor{textcolor}%
\pgfsetfillcolor{textcolor}%
\pgftext[x=2.664421in,y=3.213889in,,top]{\color{textcolor}\sffamily\fontsize{10.000000}{12.000000}\selectfont 2.35}%
\end{pgfscope}%
\begin{pgfscope}%
\pgfsetbuttcap%
\pgfsetroundjoin%
\definecolor{currentfill}{rgb}{0.000000,0.000000,0.000000}%
\pgfsetfillcolor{currentfill}%
\pgfsetlinewidth{0.803000pt}%
\definecolor{currentstroke}{rgb}{0.000000,0.000000,0.000000}%
\pgfsetstrokecolor{currentstroke}%
\pgfsetdash{}{0pt}%
\pgfsys@defobject{currentmarker}{\pgfqpoint{0.000000in}{-0.048611in}}{\pgfqpoint{0.000000in}{0.000000in}}{%
\pgfpathmoveto{\pgfqpoint{0.000000in}{0.000000in}}%
\pgfpathlineto{\pgfqpoint{0.000000in}{-0.048611in}}%
\pgfusepath{stroke,fill}%
}%
\begin{pgfscope}%
\pgfsys@transformshift{3.412214in}{3.311111in}%
\pgfsys@useobject{currentmarker}{}%
\end{pgfscope}%
\end{pgfscope}%
\begin{pgfscope}%
\definecolor{textcolor}{rgb}{0.000000,0.000000,0.000000}%
\pgfsetstrokecolor{textcolor}%
\pgfsetfillcolor{textcolor}%
\pgftext[x=3.412214in,y=3.213889in,,top]{\color{textcolor}\sffamily\fontsize{10.000000}{12.000000}\selectfont 2.40}%
\end{pgfscope}%
\begin{pgfscope}%
\pgfsetbuttcap%
\pgfsetroundjoin%
\definecolor{currentfill}{rgb}{0.000000,0.000000,0.000000}%
\pgfsetfillcolor{currentfill}%
\pgfsetlinewidth{0.803000pt}%
\definecolor{currentstroke}{rgb}{0.000000,0.000000,0.000000}%
\pgfsetstrokecolor{currentstroke}%
\pgfsetdash{}{0pt}%
\pgfsys@defobject{currentmarker}{\pgfqpoint{-0.048611in}{0.000000in}}{\pgfqpoint{0.000000in}{0.000000in}}{%
\pgfpathmoveto{\pgfqpoint{0.000000in}{0.000000in}}%
\pgfpathlineto{\pgfqpoint{-0.048611in}{0.000000in}}%
\pgfusepath{stroke,fill}%
}%
\begin{pgfscope}%
\pgfsys@transformshift{0.557986in}{3.488128in}%
\pgfsys@useobject{currentmarker}{}%
\end{pgfscope}%
\end{pgfscope}%
\begin{pgfscope}%
\definecolor{textcolor}{rgb}{0.000000,0.000000,0.000000}%
\pgfsetstrokecolor{textcolor}%
\pgfsetfillcolor{textcolor}%
\pgftext[x=0.151519in,y=3.435367in,left,base]{\color{textcolor}\sffamily\fontsize{10.000000}{12.000000}\selectfont 1.35}%
\end{pgfscope}%
\begin{pgfscope}%
\pgfsetbuttcap%
\pgfsetroundjoin%
\definecolor{currentfill}{rgb}{0.000000,0.000000,0.000000}%
\pgfsetfillcolor{currentfill}%
\pgfsetlinewidth{0.803000pt}%
\definecolor{currentstroke}{rgb}{0.000000,0.000000,0.000000}%
\pgfsetstrokecolor{currentstroke}%
\pgfsetdash{}{0pt}%
\pgfsys@defobject{currentmarker}{\pgfqpoint{-0.048611in}{0.000000in}}{\pgfqpoint{0.000000in}{0.000000in}}{%
\pgfpathmoveto{\pgfqpoint{0.000000in}{0.000000in}}%
\pgfpathlineto{\pgfqpoint{-0.048611in}{0.000000in}}%
\pgfusepath{stroke,fill}%
}%
\begin{pgfscope}%
\pgfsys@transformshift{0.557986in}{3.967315in}%
\pgfsys@useobject{currentmarker}{}%
\end{pgfscope}%
\end{pgfscope}%
\begin{pgfscope}%
\definecolor{textcolor}{rgb}{0.000000,0.000000,0.000000}%
\pgfsetstrokecolor{textcolor}%
\pgfsetfillcolor{textcolor}%
\pgftext[x=0.151519in,y=3.914553in,left,base]{\color{textcolor}\sffamily\fontsize{10.000000}{12.000000}\selectfont 1.40}%
\end{pgfscope}%
\begin{pgfscope}%
\pgfsetbuttcap%
\pgfsetroundjoin%
\definecolor{currentfill}{rgb}{0.000000,0.000000,0.000000}%
\pgfsetfillcolor{currentfill}%
\pgfsetlinewidth{0.803000pt}%
\definecolor{currentstroke}{rgb}{0.000000,0.000000,0.000000}%
\pgfsetstrokecolor{currentstroke}%
\pgfsetdash{}{0pt}%
\pgfsys@defobject{currentmarker}{\pgfqpoint{-0.048611in}{0.000000in}}{\pgfqpoint{0.000000in}{0.000000in}}{%
\pgfpathmoveto{\pgfqpoint{0.000000in}{0.000000in}}%
\pgfpathlineto{\pgfqpoint{-0.048611in}{0.000000in}}%
\pgfusepath{stroke,fill}%
}%
\begin{pgfscope}%
\pgfsys@transformshift{0.557986in}{4.446501in}%
\pgfsys@useobject{currentmarker}{}%
\end{pgfscope}%
\end{pgfscope}%
\begin{pgfscope}%
\definecolor{textcolor}{rgb}{0.000000,0.000000,0.000000}%
\pgfsetstrokecolor{textcolor}%
\pgfsetfillcolor{textcolor}%
\pgftext[x=0.151519in,y=4.393740in,left,base]{\color{textcolor}\sffamily\fontsize{10.000000}{12.000000}\selectfont 1.45}%
\end{pgfscope}%
\begin{pgfscope}%
\pgfsetbuttcap%
\pgfsetroundjoin%
\definecolor{currentfill}{rgb}{0.000000,0.000000,0.000000}%
\pgfsetfillcolor{currentfill}%
\pgfsetlinewidth{0.803000pt}%
\definecolor{currentstroke}{rgb}{0.000000,0.000000,0.000000}%
\pgfsetstrokecolor{currentstroke}%
\pgfsetdash{}{0pt}%
\pgfsys@defobject{currentmarker}{\pgfqpoint{-0.048611in}{0.000000in}}{\pgfqpoint{0.000000in}{0.000000in}}{%
\pgfpathmoveto{\pgfqpoint{0.000000in}{0.000000in}}%
\pgfpathlineto{\pgfqpoint{-0.048611in}{0.000000in}}%
\pgfusepath{stroke,fill}%
}%
\begin{pgfscope}%
\pgfsys@transformshift{0.557986in}{4.925688in}%
\pgfsys@useobject{currentmarker}{}%
\end{pgfscope}%
\end{pgfscope}%
\begin{pgfscope}%
\definecolor{textcolor}{rgb}{0.000000,0.000000,0.000000}%
\pgfsetstrokecolor{textcolor}%
\pgfsetfillcolor{textcolor}%
\pgftext[x=0.151519in,y=4.872926in,left,base]{\color{textcolor}\sffamily\fontsize{10.000000}{12.000000}\selectfont 1.50}%
\end{pgfscope}%
\begin{pgfscope}%
\pgfsetbuttcap%
\pgfsetroundjoin%
\definecolor{currentfill}{rgb}{0.000000,0.000000,0.000000}%
\pgfsetfillcolor{currentfill}%
\pgfsetlinewidth{0.803000pt}%
\definecolor{currentstroke}{rgb}{0.000000,0.000000,0.000000}%
\pgfsetstrokecolor{currentstroke}%
\pgfsetdash{}{0pt}%
\pgfsys@defobject{currentmarker}{\pgfqpoint{-0.048611in}{0.000000in}}{\pgfqpoint{0.000000in}{0.000000in}}{%
\pgfpathmoveto{\pgfqpoint{0.000000in}{0.000000in}}%
\pgfpathlineto{\pgfqpoint{-0.048611in}{0.000000in}}%
\pgfusepath{stroke,fill}%
}%
\begin{pgfscope}%
\pgfsys@transformshift{0.557986in}{5.404874in}%
\pgfsys@useobject{currentmarker}{}%
\end{pgfscope}%
\end{pgfscope}%
\begin{pgfscope}%
\definecolor{textcolor}{rgb}{0.000000,0.000000,0.000000}%
\pgfsetstrokecolor{textcolor}%
\pgfsetfillcolor{textcolor}%
\pgftext[x=0.151519in,y=5.352113in,left,base]{\color{textcolor}\sffamily\fontsize{10.000000}{12.000000}\selectfont 1.55}%
\end{pgfscope}%
\begin{pgfscope}%
\pgfpathrectangle{\pgfqpoint{0.557986in}{3.311111in}}{\pgfqpoint{3.315517in}{2.316667in}}%
\pgfusepath{clip}%
\pgfsetrectcap%
\pgfsetroundjoin%
\pgfsetlinewidth{1.505625pt}%
\definecolor{currentstroke}{rgb}{0.121569,0.466667,0.705882}%
\pgfsetstrokecolor{currentstroke}%
\pgfsetdash{}{0pt}%
\pgfpathmoveto{\pgfqpoint{0.720158in}{3.726135in}}%
\pgfpathlineto{\pgfqpoint{0.869716in}{3.994999in}}%
\pgfpathlineto{\pgfqpoint{1.019275in}{4.183152in}}%
\pgfpathlineto{\pgfqpoint{1.168834in}{4.421575in}}%
\pgfpathlineto{\pgfqpoint{1.318392in}{4.795032in}}%
\pgfpathlineto{\pgfqpoint{1.467951in}{4.853265in}}%
\pgfpathlineto{\pgfqpoint{1.617510in}{5.017452in}}%
\pgfpathlineto{\pgfqpoint{1.767068in}{5.303212in}}%
\pgfpathlineto{\pgfqpoint{1.916627in}{5.396607in}}%
\pgfpathlineto{\pgfqpoint{2.066186in}{5.371867in}}%
\pgfpathlineto{\pgfqpoint{2.215744in}{5.460839in}}%
\pgfpathlineto{\pgfqpoint{2.365303in}{5.310740in}}%
\pgfpathlineto{\pgfqpoint{2.514862in}{5.279693in}}%
\pgfpathlineto{\pgfqpoint{2.664421in}{5.201187in}}%
\pgfpathlineto{\pgfqpoint{2.813979in}{4.989892in}}%
\pgfpathlineto{\pgfqpoint{2.963538in}{4.869289in}}%
\pgfpathlineto{\pgfqpoint{3.113097in}{4.598831in}}%
\pgfpathlineto{\pgfqpoint{3.262655in}{4.335389in}}%
\pgfpathlineto{\pgfqpoint{3.412214in}{3.999630in}}%
\pgfpathlineto{\pgfqpoint{3.561773in}{3.760174in}}%
\pgfpathlineto{\pgfqpoint{3.711331in}{3.462126in}}%
\pgfusepath{stroke}%
\end{pgfscope}%
\begin{pgfscope}%
\pgfsetrectcap%
\pgfsetmiterjoin%
\pgfsetlinewidth{0.803000pt}%
\definecolor{currentstroke}{rgb}{0.000000,0.000000,0.000000}%
\pgfsetstrokecolor{currentstroke}%
\pgfsetdash{}{0pt}%
\pgfpathmoveto{\pgfqpoint{0.557986in}{3.311111in}}%
\pgfpathlineto{\pgfqpoint{0.557986in}{5.627778in}}%
\pgfusepath{stroke}%
\end{pgfscope}%
\begin{pgfscope}%
\pgfsetrectcap%
\pgfsetmiterjoin%
\pgfsetlinewidth{0.803000pt}%
\definecolor{currentstroke}{rgb}{0.000000,0.000000,0.000000}%
\pgfsetstrokecolor{currentstroke}%
\pgfsetdash{}{0pt}%
\pgfpathmoveto{\pgfqpoint{3.873503in}{3.311111in}}%
\pgfpathlineto{\pgfqpoint{3.873503in}{5.627778in}}%
\pgfusepath{stroke}%
\end{pgfscope}%
\begin{pgfscope}%
\pgfsetrectcap%
\pgfsetmiterjoin%
\pgfsetlinewidth{0.803000pt}%
\definecolor{currentstroke}{rgb}{0.000000,0.000000,0.000000}%
\pgfsetstrokecolor{currentstroke}%
\pgfsetdash{}{0pt}%
\pgfpathmoveto{\pgfqpoint{0.557986in}{3.311111in}}%
\pgfpathlineto{\pgfqpoint{3.873503in}{3.311111in}}%
\pgfusepath{stroke}%
\end{pgfscope}%
\begin{pgfscope}%
\pgfsetrectcap%
\pgfsetmiterjoin%
\pgfsetlinewidth{0.803000pt}%
\definecolor{currentstroke}{rgb}{0.000000,0.000000,0.000000}%
\pgfsetstrokecolor{currentstroke}%
\pgfsetdash{}{0pt}%
\pgfpathmoveto{\pgfqpoint{0.557986in}{5.627778in}}%
\pgfpathlineto{\pgfqpoint{3.873503in}{5.627778in}}%
\pgfusepath{stroke}%
\end{pgfscope}%
\begin{pgfscope}%
\definecolor{textcolor}{rgb}{0.000000,0.000000,0.000000}%
\pgfsetstrokecolor{textcolor}%
\pgfsetfillcolor{textcolor}%
\pgftext[x=2.215744in,y=5.711111in,,base]{\color{textcolor}\sffamily\fontsize{12.000000}{14.400000}\selectfont \(\displaystyle  N = 16 \)}%
\end{pgfscope}%
\begin{pgfscope}%
\pgfsetbuttcap%
\pgfsetmiterjoin%
\definecolor{currentfill}{rgb}{1.000000,1.000000,1.000000}%
\pgfsetfillcolor{currentfill}%
\pgfsetlinewidth{0.000000pt}%
\definecolor{currentstroke}{rgb}{0.000000,0.000000,0.000000}%
\pgfsetstrokecolor{currentstroke}%
\pgfsetstrokeopacity{0.000000}%
\pgfsetdash{}{0pt}%
\pgfpathmoveto{\pgfqpoint{4.391558in}{3.311111in}}%
\pgfpathlineto{\pgfqpoint{7.707075in}{3.311111in}}%
\pgfpathlineto{\pgfqpoint{7.707075in}{5.627778in}}%
\pgfpathlineto{\pgfqpoint{4.391558in}{5.627778in}}%
\pgfpathclose%
\pgfusepath{fill}%
\end{pgfscope}%
\begin{pgfscope}%
\pgfpathrectangle{\pgfqpoint{4.391558in}{3.311111in}}{\pgfqpoint{3.315517in}{2.316667in}}%
\pgfusepath{clip}%
\pgfsetbuttcap%
\pgfsetroundjoin%
\definecolor{currentfill}{rgb}{0.121569,0.466667,0.705882}%
\pgfsetfillcolor{currentfill}%
\pgfsetfillopacity{0.300000}%
\pgfsetlinewidth{1.003750pt}%
\definecolor{currentstroke}{rgb}{0.121569,0.466667,0.705882}%
\pgfsetstrokecolor{currentstroke}%
\pgfsetstrokeopacity{0.300000}%
\pgfsetdash{}{0pt}%
\pgfpathmoveto{\pgfqpoint{4.553730in}{3.526942in}}%
\pgfpathlineto{\pgfqpoint{4.553730in}{3.416414in}}%
\pgfpathlineto{\pgfqpoint{4.703288in}{3.565769in}}%
\pgfpathlineto{\pgfqpoint{4.852847in}{3.676703in}}%
\pgfpathlineto{\pgfqpoint{5.002406in}{3.856527in}}%
\pgfpathlineto{\pgfqpoint{5.151965in}{3.984052in}}%
\pgfpathlineto{\pgfqpoint{5.301523in}{4.437031in}}%
\pgfpathlineto{\pgfqpoint{5.451082in}{4.574151in}}%
\pgfpathlineto{\pgfqpoint{5.600641in}{4.726179in}}%
\pgfpathlineto{\pgfqpoint{5.750199in}{4.778531in}}%
\pgfpathlineto{\pgfqpoint{5.899758in}{5.035196in}}%
\pgfpathlineto{\pgfqpoint{6.049317in}{5.084031in}}%
\pgfpathlineto{\pgfqpoint{6.198875in}{5.079842in}}%
\pgfpathlineto{\pgfqpoint{6.348434in}{5.081829in}}%
\pgfpathlineto{\pgfqpoint{6.497993in}{4.989675in}}%
\pgfpathlineto{\pgfqpoint{6.647551in}{4.797522in}}%
\pgfpathlineto{\pgfqpoint{6.797110in}{4.530904in}}%
\pgfpathlineto{\pgfqpoint{6.946669in}{4.305767in}}%
\pgfpathlineto{\pgfqpoint{7.096227in}{4.119391in}}%
\pgfpathlineto{\pgfqpoint{7.245786in}{3.796162in}}%
\pgfpathlineto{\pgfqpoint{7.395345in}{3.571411in}}%
\pgfpathlineto{\pgfqpoint{7.544904in}{3.443963in}}%
\pgfpathlineto{\pgfqpoint{7.544904in}{3.566688in}}%
\pgfpathlineto{\pgfqpoint{7.544904in}{3.566688in}}%
\pgfpathlineto{\pgfqpoint{7.395345in}{3.748915in}}%
\pgfpathlineto{\pgfqpoint{7.245786in}{4.075250in}}%
\pgfpathlineto{\pgfqpoint{7.096227in}{4.336937in}}%
\pgfpathlineto{\pgfqpoint{6.946669in}{4.551318in}}%
\pgfpathlineto{\pgfqpoint{6.797110in}{4.632544in}}%
\pgfpathlineto{\pgfqpoint{6.647551in}{4.909204in}}%
\pgfpathlineto{\pgfqpoint{6.497993in}{5.210984in}}%
\pgfpathlineto{\pgfqpoint{6.348434in}{5.329859in}}%
\pgfpathlineto{\pgfqpoint{6.198875in}{5.522475in}}%
\pgfpathlineto{\pgfqpoint{6.049317in}{5.456316in}}%
\pgfpathlineto{\pgfqpoint{5.899758in}{5.160074in}}%
\pgfpathlineto{\pgfqpoint{5.750199in}{5.352515in}}%
\pgfpathlineto{\pgfqpoint{5.600641in}{5.250990in}}%
\pgfpathlineto{\pgfqpoint{5.451082in}{4.666764in}}%
\pgfpathlineto{\pgfqpoint{5.301523in}{4.896477in}}%
\pgfpathlineto{\pgfqpoint{5.151965in}{4.388372in}}%
\pgfpathlineto{\pgfqpoint{5.002406in}{4.159151in}}%
\pgfpathlineto{\pgfqpoint{4.852847in}{3.898962in}}%
\pgfpathlineto{\pgfqpoint{4.703288in}{3.739961in}}%
\pgfpathlineto{\pgfqpoint{4.553730in}{3.526942in}}%
\pgfpathclose%
\pgfusepath{stroke,fill}%
\end{pgfscope}%
\begin{pgfscope}%
\pgfpathrectangle{\pgfqpoint{4.391558in}{3.311111in}}{\pgfqpoint{3.315517in}{2.316667in}}%
\pgfusepath{clip}%
\pgfsetbuttcap%
\pgfsetroundjoin%
\definecolor{currentfill}{rgb}{0.121569,0.466667,0.705882}%
\pgfsetfillcolor{currentfill}%
\pgfsetlinewidth{1.003750pt}%
\definecolor{currentstroke}{rgb}{0.121569,0.466667,0.705882}%
\pgfsetstrokecolor{currentstroke}%
\pgfsetdash{}{0pt}%
\pgfsys@defobject{currentmarker}{\pgfqpoint{-0.009821in}{-0.009821in}}{\pgfqpoint{0.009821in}{0.009821in}}{%
\pgfpathmoveto{\pgfqpoint{0.000000in}{-0.009821in}}%
\pgfpathcurveto{\pgfqpoint{0.002605in}{-0.009821in}}{\pgfqpoint{0.005103in}{-0.008786in}}{\pgfqpoint{0.006944in}{-0.006944in}}%
\pgfpathcurveto{\pgfqpoint{0.008786in}{-0.005103in}}{\pgfqpoint{0.009821in}{-0.002605in}}{\pgfqpoint{0.009821in}{0.000000in}}%
\pgfpathcurveto{\pgfqpoint{0.009821in}{0.002605in}}{\pgfqpoint{0.008786in}{0.005103in}}{\pgfqpoint{0.006944in}{0.006944in}}%
\pgfpathcurveto{\pgfqpoint{0.005103in}{0.008786in}}{\pgfqpoint{0.002605in}{0.009821in}}{\pgfqpoint{0.000000in}{0.009821in}}%
\pgfpathcurveto{\pgfqpoint{-0.002605in}{0.009821in}}{\pgfqpoint{-0.005103in}{0.008786in}}{\pgfqpoint{-0.006944in}{0.006944in}}%
\pgfpathcurveto{\pgfqpoint{-0.008786in}{0.005103in}}{\pgfqpoint{-0.009821in}{0.002605in}}{\pgfqpoint{-0.009821in}{0.000000in}}%
\pgfpathcurveto{\pgfqpoint{-0.009821in}{-0.002605in}}{\pgfqpoint{-0.008786in}{-0.005103in}}{\pgfqpoint{-0.006944in}{-0.006944in}}%
\pgfpathcurveto{\pgfqpoint{-0.005103in}{-0.008786in}}{\pgfqpoint{-0.002605in}{-0.009821in}}{\pgfqpoint{0.000000in}{-0.009821in}}%
\pgfpathclose%
\pgfusepath{stroke,fill}%
}%
\begin{pgfscope}%
\pgfsys@transformshift{4.553730in}{3.471678in}%
\pgfsys@useobject{currentmarker}{}%
\end{pgfscope}%
\begin{pgfscope}%
\pgfsys@transformshift{4.703288in}{3.652865in}%
\pgfsys@useobject{currentmarker}{}%
\end{pgfscope}%
\begin{pgfscope}%
\pgfsys@transformshift{4.852847in}{3.787832in}%
\pgfsys@useobject{currentmarker}{}%
\end{pgfscope}%
\begin{pgfscope}%
\pgfsys@transformshift{5.002406in}{4.007839in}%
\pgfsys@useobject{currentmarker}{}%
\end{pgfscope}%
\begin{pgfscope}%
\pgfsys@transformshift{5.151965in}{4.186212in}%
\pgfsys@useobject{currentmarker}{}%
\end{pgfscope}%
\begin{pgfscope}%
\pgfsys@transformshift{5.301523in}{4.666754in}%
\pgfsys@useobject{currentmarker}{}%
\end{pgfscope}%
\begin{pgfscope}%
\pgfsys@transformshift{5.451082in}{4.620457in}%
\pgfsys@useobject{currentmarker}{}%
\end{pgfscope}%
\begin{pgfscope}%
\pgfsys@transformshift{5.600641in}{4.988584in}%
\pgfsys@useobject{currentmarker}{}%
\end{pgfscope}%
\begin{pgfscope}%
\pgfsys@transformshift{5.750199in}{5.065523in}%
\pgfsys@useobject{currentmarker}{}%
\end{pgfscope}%
\begin{pgfscope}%
\pgfsys@transformshift{5.899758in}{5.097635in}%
\pgfsys@useobject{currentmarker}{}%
\end{pgfscope}%
\begin{pgfscope}%
\pgfsys@transformshift{6.049317in}{5.270174in}%
\pgfsys@useobject{currentmarker}{}%
\end{pgfscope}%
\begin{pgfscope}%
\pgfsys@transformshift{6.198875in}{5.301158in}%
\pgfsys@useobject{currentmarker}{}%
\end{pgfscope}%
\begin{pgfscope}%
\pgfsys@transformshift{6.348434in}{5.205844in}%
\pgfsys@useobject{currentmarker}{}%
\end{pgfscope}%
\begin{pgfscope}%
\pgfsys@transformshift{6.497993in}{5.100330in}%
\pgfsys@useobject{currentmarker}{}%
\end{pgfscope}%
\begin{pgfscope}%
\pgfsys@transformshift{6.647551in}{4.853363in}%
\pgfsys@useobject{currentmarker}{}%
\end{pgfscope}%
\begin{pgfscope}%
\pgfsys@transformshift{6.797110in}{4.581724in}%
\pgfsys@useobject{currentmarker}{}%
\end{pgfscope}%
\begin{pgfscope}%
\pgfsys@transformshift{6.946669in}{4.428542in}%
\pgfsys@useobject{currentmarker}{}%
\end{pgfscope}%
\begin{pgfscope}%
\pgfsys@transformshift{7.096227in}{4.228164in}%
\pgfsys@useobject{currentmarker}{}%
\end{pgfscope}%
\begin{pgfscope}%
\pgfsys@transformshift{7.245786in}{3.935706in}%
\pgfsys@useobject{currentmarker}{}%
\end{pgfscope}%
\begin{pgfscope}%
\pgfsys@transformshift{7.395345in}{3.660163in}%
\pgfsys@useobject{currentmarker}{}%
\end{pgfscope}%
\begin{pgfscope}%
\pgfsys@transformshift{7.544904in}{3.505326in}%
\pgfsys@useobject{currentmarker}{}%
\end{pgfscope}%
\end{pgfscope}%
\begin{pgfscope}%
\pgfsetbuttcap%
\pgfsetroundjoin%
\definecolor{currentfill}{rgb}{0.000000,0.000000,0.000000}%
\pgfsetfillcolor{currentfill}%
\pgfsetlinewidth{0.803000pt}%
\definecolor{currentstroke}{rgb}{0.000000,0.000000,0.000000}%
\pgfsetstrokecolor{currentstroke}%
\pgfsetdash{}{0pt}%
\pgfsys@defobject{currentmarker}{\pgfqpoint{0.000000in}{-0.048611in}}{\pgfqpoint{0.000000in}{0.000000in}}{%
\pgfpathmoveto{\pgfqpoint{0.000000in}{0.000000in}}%
\pgfpathlineto{\pgfqpoint{0.000000in}{-0.048611in}}%
\pgfusepath{stroke,fill}%
}%
\begin{pgfscope}%
\pgfsys@transformshift{4.703288in}{3.311111in}%
\pgfsys@useobject{currentmarker}{}%
\end{pgfscope}%
\end{pgfscope}%
\begin{pgfscope}%
\definecolor{textcolor}{rgb}{0.000000,0.000000,0.000000}%
\pgfsetstrokecolor{textcolor}%
\pgfsetfillcolor{textcolor}%
\pgftext[x=4.703288in,y=3.213889in,,top]{\color{textcolor}\sffamily\fontsize{10.000000}{12.000000}\selectfont 2.20}%
\end{pgfscope}%
\begin{pgfscope}%
\pgfsetbuttcap%
\pgfsetroundjoin%
\definecolor{currentfill}{rgb}{0.000000,0.000000,0.000000}%
\pgfsetfillcolor{currentfill}%
\pgfsetlinewidth{0.803000pt}%
\definecolor{currentstroke}{rgb}{0.000000,0.000000,0.000000}%
\pgfsetstrokecolor{currentstroke}%
\pgfsetdash{}{0pt}%
\pgfsys@defobject{currentmarker}{\pgfqpoint{0.000000in}{-0.048611in}}{\pgfqpoint{0.000000in}{0.000000in}}{%
\pgfpathmoveto{\pgfqpoint{0.000000in}{0.000000in}}%
\pgfpathlineto{\pgfqpoint{0.000000in}{-0.048611in}}%
\pgfusepath{stroke,fill}%
}%
\begin{pgfscope}%
\pgfsys@transformshift{5.451082in}{3.311111in}%
\pgfsys@useobject{currentmarker}{}%
\end{pgfscope}%
\end{pgfscope}%
\begin{pgfscope}%
\definecolor{textcolor}{rgb}{0.000000,0.000000,0.000000}%
\pgfsetstrokecolor{textcolor}%
\pgfsetfillcolor{textcolor}%
\pgftext[x=5.451082in,y=3.213889in,,top]{\color{textcolor}\sffamily\fontsize{10.000000}{12.000000}\selectfont 2.25}%
\end{pgfscope}%
\begin{pgfscope}%
\pgfsetbuttcap%
\pgfsetroundjoin%
\definecolor{currentfill}{rgb}{0.000000,0.000000,0.000000}%
\pgfsetfillcolor{currentfill}%
\pgfsetlinewidth{0.803000pt}%
\definecolor{currentstroke}{rgb}{0.000000,0.000000,0.000000}%
\pgfsetstrokecolor{currentstroke}%
\pgfsetdash{}{0pt}%
\pgfsys@defobject{currentmarker}{\pgfqpoint{0.000000in}{-0.048611in}}{\pgfqpoint{0.000000in}{0.000000in}}{%
\pgfpathmoveto{\pgfqpoint{0.000000in}{0.000000in}}%
\pgfpathlineto{\pgfqpoint{0.000000in}{-0.048611in}}%
\pgfusepath{stroke,fill}%
}%
\begin{pgfscope}%
\pgfsys@transformshift{6.198875in}{3.311111in}%
\pgfsys@useobject{currentmarker}{}%
\end{pgfscope}%
\end{pgfscope}%
\begin{pgfscope}%
\definecolor{textcolor}{rgb}{0.000000,0.000000,0.000000}%
\pgfsetstrokecolor{textcolor}%
\pgfsetfillcolor{textcolor}%
\pgftext[x=6.198875in,y=3.213889in,,top]{\color{textcolor}\sffamily\fontsize{10.000000}{12.000000}\selectfont 2.30}%
\end{pgfscope}%
\begin{pgfscope}%
\pgfsetbuttcap%
\pgfsetroundjoin%
\definecolor{currentfill}{rgb}{0.000000,0.000000,0.000000}%
\pgfsetfillcolor{currentfill}%
\pgfsetlinewidth{0.803000pt}%
\definecolor{currentstroke}{rgb}{0.000000,0.000000,0.000000}%
\pgfsetstrokecolor{currentstroke}%
\pgfsetdash{}{0pt}%
\pgfsys@defobject{currentmarker}{\pgfqpoint{0.000000in}{-0.048611in}}{\pgfqpoint{0.000000in}{0.000000in}}{%
\pgfpathmoveto{\pgfqpoint{0.000000in}{0.000000in}}%
\pgfpathlineto{\pgfqpoint{0.000000in}{-0.048611in}}%
\pgfusepath{stroke,fill}%
}%
\begin{pgfscope}%
\pgfsys@transformshift{6.946669in}{3.311111in}%
\pgfsys@useobject{currentmarker}{}%
\end{pgfscope}%
\end{pgfscope}%
\begin{pgfscope}%
\definecolor{textcolor}{rgb}{0.000000,0.000000,0.000000}%
\pgfsetstrokecolor{textcolor}%
\pgfsetfillcolor{textcolor}%
\pgftext[x=6.946669in,y=3.213889in,,top]{\color{textcolor}\sffamily\fontsize{10.000000}{12.000000}\selectfont 2.35}%
\end{pgfscope}%
\begin{pgfscope}%
\pgfsetbuttcap%
\pgfsetroundjoin%
\definecolor{currentfill}{rgb}{0.000000,0.000000,0.000000}%
\pgfsetfillcolor{currentfill}%
\pgfsetlinewidth{0.803000pt}%
\definecolor{currentstroke}{rgb}{0.000000,0.000000,0.000000}%
\pgfsetstrokecolor{currentstroke}%
\pgfsetdash{}{0pt}%
\pgfsys@defobject{currentmarker}{\pgfqpoint{0.000000in}{-0.048611in}}{\pgfqpoint{0.000000in}{0.000000in}}{%
\pgfpathmoveto{\pgfqpoint{0.000000in}{0.000000in}}%
\pgfpathlineto{\pgfqpoint{0.000000in}{-0.048611in}}%
\pgfusepath{stroke,fill}%
}%
\begin{pgfscope}%
\pgfsys@transformshift{7.694462in}{3.311111in}%
\pgfsys@useobject{currentmarker}{}%
\end{pgfscope}%
\end{pgfscope}%
\begin{pgfscope}%
\definecolor{textcolor}{rgb}{0.000000,0.000000,0.000000}%
\pgfsetstrokecolor{textcolor}%
\pgfsetfillcolor{textcolor}%
\pgftext[x=7.694462in,y=3.213889in,,top]{\color{textcolor}\sffamily\fontsize{10.000000}{12.000000}\selectfont 2.40}%
\end{pgfscope}%
\begin{pgfscope}%
\pgfsetbuttcap%
\pgfsetroundjoin%
\definecolor{currentfill}{rgb}{0.000000,0.000000,0.000000}%
\pgfsetfillcolor{currentfill}%
\pgfsetlinewidth{0.803000pt}%
\definecolor{currentstroke}{rgb}{0.000000,0.000000,0.000000}%
\pgfsetstrokecolor{currentstroke}%
\pgfsetdash{}{0pt}%
\pgfsys@defobject{currentmarker}{\pgfqpoint{-0.048611in}{0.000000in}}{\pgfqpoint{0.000000in}{0.000000in}}{%
\pgfpathmoveto{\pgfqpoint{0.000000in}{0.000000in}}%
\pgfpathlineto{\pgfqpoint{-0.048611in}{0.000000in}}%
\pgfusepath{stroke,fill}%
}%
\begin{pgfscope}%
\pgfsys@transformshift{4.391558in}{3.356105in}%
\pgfsys@useobject{currentmarker}{}%
\end{pgfscope}%
\end{pgfscope}%
\begin{pgfscope}%
\definecolor{textcolor}{rgb}{0.000000,0.000000,0.000000}%
\pgfsetstrokecolor{textcolor}%
\pgfsetfillcolor{textcolor}%
\pgftext[x=4.073457in,y=3.303343in,left,base]{\color{textcolor}\sffamily\fontsize{10.000000}{12.000000}\selectfont 1.3}%
\end{pgfscope}%
\begin{pgfscope}%
\pgfsetbuttcap%
\pgfsetroundjoin%
\definecolor{currentfill}{rgb}{0.000000,0.000000,0.000000}%
\pgfsetfillcolor{currentfill}%
\pgfsetlinewidth{0.803000pt}%
\definecolor{currentstroke}{rgb}{0.000000,0.000000,0.000000}%
\pgfsetstrokecolor{currentstroke}%
\pgfsetdash{}{0pt}%
\pgfsys@defobject{currentmarker}{\pgfqpoint{-0.048611in}{0.000000in}}{\pgfqpoint{0.000000in}{0.000000in}}{%
\pgfpathmoveto{\pgfqpoint{0.000000in}{0.000000in}}%
\pgfpathlineto{\pgfqpoint{-0.048611in}{0.000000in}}%
\pgfusepath{stroke,fill}%
}%
\begin{pgfscope}%
\pgfsys@transformshift{4.391558in}{3.665982in}%
\pgfsys@useobject{currentmarker}{}%
\end{pgfscope}%
\end{pgfscope}%
\begin{pgfscope}%
\definecolor{textcolor}{rgb}{0.000000,0.000000,0.000000}%
\pgfsetstrokecolor{textcolor}%
\pgfsetfillcolor{textcolor}%
\pgftext[x=4.073457in,y=3.613220in,left,base]{\color{textcolor}\sffamily\fontsize{10.000000}{12.000000}\selectfont 1.4}%
\end{pgfscope}%
\begin{pgfscope}%
\pgfsetbuttcap%
\pgfsetroundjoin%
\definecolor{currentfill}{rgb}{0.000000,0.000000,0.000000}%
\pgfsetfillcolor{currentfill}%
\pgfsetlinewidth{0.803000pt}%
\definecolor{currentstroke}{rgb}{0.000000,0.000000,0.000000}%
\pgfsetstrokecolor{currentstroke}%
\pgfsetdash{}{0pt}%
\pgfsys@defobject{currentmarker}{\pgfqpoint{-0.048611in}{0.000000in}}{\pgfqpoint{0.000000in}{0.000000in}}{%
\pgfpathmoveto{\pgfqpoint{0.000000in}{0.000000in}}%
\pgfpathlineto{\pgfqpoint{-0.048611in}{0.000000in}}%
\pgfusepath{stroke,fill}%
}%
\begin{pgfscope}%
\pgfsys@transformshift{4.391558in}{3.975859in}%
\pgfsys@useobject{currentmarker}{}%
\end{pgfscope}%
\end{pgfscope}%
\begin{pgfscope}%
\definecolor{textcolor}{rgb}{0.000000,0.000000,0.000000}%
\pgfsetstrokecolor{textcolor}%
\pgfsetfillcolor{textcolor}%
\pgftext[x=4.073457in,y=3.923098in,left,base]{\color{textcolor}\sffamily\fontsize{10.000000}{12.000000}\selectfont 1.5}%
\end{pgfscope}%
\begin{pgfscope}%
\pgfsetbuttcap%
\pgfsetroundjoin%
\definecolor{currentfill}{rgb}{0.000000,0.000000,0.000000}%
\pgfsetfillcolor{currentfill}%
\pgfsetlinewidth{0.803000pt}%
\definecolor{currentstroke}{rgb}{0.000000,0.000000,0.000000}%
\pgfsetstrokecolor{currentstroke}%
\pgfsetdash{}{0pt}%
\pgfsys@defobject{currentmarker}{\pgfqpoint{-0.048611in}{0.000000in}}{\pgfqpoint{0.000000in}{0.000000in}}{%
\pgfpathmoveto{\pgfqpoint{0.000000in}{0.000000in}}%
\pgfpathlineto{\pgfqpoint{-0.048611in}{0.000000in}}%
\pgfusepath{stroke,fill}%
}%
\begin{pgfscope}%
\pgfsys@transformshift{4.391558in}{4.285736in}%
\pgfsys@useobject{currentmarker}{}%
\end{pgfscope}%
\end{pgfscope}%
\begin{pgfscope}%
\definecolor{textcolor}{rgb}{0.000000,0.000000,0.000000}%
\pgfsetstrokecolor{textcolor}%
\pgfsetfillcolor{textcolor}%
\pgftext[x=4.073457in,y=4.232975in,left,base]{\color{textcolor}\sffamily\fontsize{10.000000}{12.000000}\selectfont 1.6}%
\end{pgfscope}%
\begin{pgfscope}%
\pgfsetbuttcap%
\pgfsetroundjoin%
\definecolor{currentfill}{rgb}{0.000000,0.000000,0.000000}%
\pgfsetfillcolor{currentfill}%
\pgfsetlinewidth{0.803000pt}%
\definecolor{currentstroke}{rgb}{0.000000,0.000000,0.000000}%
\pgfsetstrokecolor{currentstroke}%
\pgfsetdash{}{0pt}%
\pgfsys@defobject{currentmarker}{\pgfqpoint{-0.048611in}{0.000000in}}{\pgfqpoint{0.000000in}{0.000000in}}{%
\pgfpathmoveto{\pgfqpoint{0.000000in}{0.000000in}}%
\pgfpathlineto{\pgfqpoint{-0.048611in}{0.000000in}}%
\pgfusepath{stroke,fill}%
}%
\begin{pgfscope}%
\pgfsys@transformshift{4.391558in}{4.595614in}%
\pgfsys@useobject{currentmarker}{}%
\end{pgfscope}%
\end{pgfscope}%
\begin{pgfscope}%
\definecolor{textcolor}{rgb}{0.000000,0.000000,0.000000}%
\pgfsetstrokecolor{textcolor}%
\pgfsetfillcolor{textcolor}%
\pgftext[x=4.073457in,y=4.542852in,left,base]{\color{textcolor}\sffamily\fontsize{10.000000}{12.000000}\selectfont 1.7}%
\end{pgfscope}%
\begin{pgfscope}%
\pgfsetbuttcap%
\pgfsetroundjoin%
\definecolor{currentfill}{rgb}{0.000000,0.000000,0.000000}%
\pgfsetfillcolor{currentfill}%
\pgfsetlinewidth{0.803000pt}%
\definecolor{currentstroke}{rgb}{0.000000,0.000000,0.000000}%
\pgfsetstrokecolor{currentstroke}%
\pgfsetdash{}{0pt}%
\pgfsys@defobject{currentmarker}{\pgfqpoint{-0.048611in}{0.000000in}}{\pgfqpoint{0.000000in}{0.000000in}}{%
\pgfpathmoveto{\pgfqpoint{0.000000in}{0.000000in}}%
\pgfpathlineto{\pgfqpoint{-0.048611in}{0.000000in}}%
\pgfusepath{stroke,fill}%
}%
\begin{pgfscope}%
\pgfsys@transformshift{4.391558in}{4.905491in}%
\pgfsys@useobject{currentmarker}{}%
\end{pgfscope}%
\end{pgfscope}%
\begin{pgfscope}%
\definecolor{textcolor}{rgb}{0.000000,0.000000,0.000000}%
\pgfsetstrokecolor{textcolor}%
\pgfsetfillcolor{textcolor}%
\pgftext[x=4.073457in,y=4.852730in,left,base]{\color{textcolor}\sffamily\fontsize{10.000000}{12.000000}\selectfont 1.8}%
\end{pgfscope}%
\begin{pgfscope}%
\pgfsetbuttcap%
\pgfsetroundjoin%
\definecolor{currentfill}{rgb}{0.000000,0.000000,0.000000}%
\pgfsetfillcolor{currentfill}%
\pgfsetlinewidth{0.803000pt}%
\definecolor{currentstroke}{rgb}{0.000000,0.000000,0.000000}%
\pgfsetstrokecolor{currentstroke}%
\pgfsetdash{}{0pt}%
\pgfsys@defobject{currentmarker}{\pgfqpoint{-0.048611in}{0.000000in}}{\pgfqpoint{0.000000in}{0.000000in}}{%
\pgfpathmoveto{\pgfqpoint{0.000000in}{0.000000in}}%
\pgfpathlineto{\pgfqpoint{-0.048611in}{0.000000in}}%
\pgfusepath{stroke,fill}%
}%
\begin{pgfscope}%
\pgfsys@transformshift{4.391558in}{5.215368in}%
\pgfsys@useobject{currentmarker}{}%
\end{pgfscope}%
\end{pgfscope}%
\begin{pgfscope}%
\definecolor{textcolor}{rgb}{0.000000,0.000000,0.000000}%
\pgfsetstrokecolor{textcolor}%
\pgfsetfillcolor{textcolor}%
\pgftext[x=4.073457in,y=5.162607in,left,base]{\color{textcolor}\sffamily\fontsize{10.000000}{12.000000}\selectfont 1.9}%
\end{pgfscope}%
\begin{pgfscope}%
\pgfsetbuttcap%
\pgfsetroundjoin%
\definecolor{currentfill}{rgb}{0.000000,0.000000,0.000000}%
\pgfsetfillcolor{currentfill}%
\pgfsetlinewidth{0.803000pt}%
\definecolor{currentstroke}{rgb}{0.000000,0.000000,0.000000}%
\pgfsetstrokecolor{currentstroke}%
\pgfsetdash{}{0pt}%
\pgfsys@defobject{currentmarker}{\pgfqpoint{-0.048611in}{0.000000in}}{\pgfqpoint{0.000000in}{0.000000in}}{%
\pgfpathmoveto{\pgfqpoint{0.000000in}{0.000000in}}%
\pgfpathlineto{\pgfqpoint{-0.048611in}{0.000000in}}%
\pgfusepath{stroke,fill}%
}%
\begin{pgfscope}%
\pgfsys@transformshift{4.391558in}{5.525246in}%
\pgfsys@useobject{currentmarker}{}%
\end{pgfscope}%
\end{pgfscope}%
\begin{pgfscope}%
\definecolor{textcolor}{rgb}{0.000000,0.000000,0.000000}%
\pgfsetstrokecolor{textcolor}%
\pgfsetfillcolor{textcolor}%
\pgftext[x=4.073457in,y=5.472484in,left,base]{\color{textcolor}\sffamily\fontsize{10.000000}{12.000000}\selectfont 2.0}%
\end{pgfscope}%
\begin{pgfscope}%
\pgfpathrectangle{\pgfqpoint{4.391558in}{3.311111in}}{\pgfqpoint{3.315517in}{2.316667in}}%
\pgfusepath{clip}%
\pgfsetrectcap%
\pgfsetroundjoin%
\pgfsetlinewidth{1.505625pt}%
\definecolor{currentstroke}{rgb}{0.121569,0.466667,0.705882}%
\pgfsetstrokecolor{currentstroke}%
\pgfsetdash{}{0pt}%
\pgfpathmoveto{\pgfqpoint{4.553730in}{3.471678in}}%
\pgfpathlineto{\pgfqpoint{4.703288in}{3.652865in}}%
\pgfpathlineto{\pgfqpoint{4.852847in}{3.787832in}}%
\pgfpathlineto{\pgfqpoint{5.002406in}{4.007839in}}%
\pgfpathlineto{\pgfqpoint{5.151965in}{4.186212in}}%
\pgfpathlineto{\pgfqpoint{5.301523in}{4.666754in}}%
\pgfpathlineto{\pgfqpoint{5.451082in}{4.620457in}}%
\pgfpathlineto{\pgfqpoint{5.600641in}{4.988584in}}%
\pgfpathlineto{\pgfqpoint{5.750199in}{5.065523in}}%
\pgfpathlineto{\pgfqpoint{5.899758in}{5.097635in}}%
\pgfpathlineto{\pgfqpoint{6.049317in}{5.270174in}}%
\pgfpathlineto{\pgfqpoint{6.198875in}{5.301158in}}%
\pgfpathlineto{\pgfqpoint{6.348434in}{5.205844in}}%
\pgfpathlineto{\pgfqpoint{6.497993in}{5.100330in}}%
\pgfpathlineto{\pgfqpoint{6.647551in}{4.853363in}}%
\pgfpathlineto{\pgfqpoint{6.797110in}{4.581724in}}%
\pgfpathlineto{\pgfqpoint{6.946669in}{4.428542in}}%
\pgfpathlineto{\pgfqpoint{7.096227in}{4.228164in}}%
\pgfpathlineto{\pgfqpoint{7.245786in}{3.935706in}}%
\pgfpathlineto{\pgfqpoint{7.395345in}{3.660163in}}%
\pgfpathlineto{\pgfqpoint{7.544904in}{3.505326in}}%
\pgfusepath{stroke}%
\end{pgfscope}%
\begin{pgfscope}%
\pgfsetrectcap%
\pgfsetmiterjoin%
\pgfsetlinewidth{0.803000pt}%
\definecolor{currentstroke}{rgb}{0.000000,0.000000,0.000000}%
\pgfsetstrokecolor{currentstroke}%
\pgfsetdash{}{0pt}%
\pgfpathmoveto{\pgfqpoint{4.391558in}{3.311111in}}%
\pgfpathlineto{\pgfqpoint{4.391558in}{5.627778in}}%
\pgfusepath{stroke}%
\end{pgfscope}%
\begin{pgfscope}%
\pgfsetrectcap%
\pgfsetmiterjoin%
\pgfsetlinewidth{0.803000pt}%
\definecolor{currentstroke}{rgb}{0.000000,0.000000,0.000000}%
\pgfsetstrokecolor{currentstroke}%
\pgfsetdash{}{0pt}%
\pgfpathmoveto{\pgfqpoint{7.707075in}{3.311111in}}%
\pgfpathlineto{\pgfqpoint{7.707075in}{5.627778in}}%
\pgfusepath{stroke}%
\end{pgfscope}%
\begin{pgfscope}%
\pgfsetrectcap%
\pgfsetmiterjoin%
\pgfsetlinewidth{0.803000pt}%
\definecolor{currentstroke}{rgb}{0.000000,0.000000,0.000000}%
\pgfsetstrokecolor{currentstroke}%
\pgfsetdash{}{0pt}%
\pgfpathmoveto{\pgfqpoint{4.391558in}{3.311111in}}%
\pgfpathlineto{\pgfqpoint{7.707075in}{3.311111in}}%
\pgfusepath{stroke}%
\end{pgfscope}%
\begin{pgfscope}%
\pgfsetrectcap%
\pgfsetmiterjoin%
\pgfsetlinewidth{0.803000pt}%
\definecolor{currentstroke}{rgb}{0.000000,0.000000,0.000000}%
\pgfsetstrokecolor{currentstroke}%
\pgfsetdash{}{0pt}%
\pgfpathmoveto{\pgfqpoint{4.391558in}{5.627778in}}%
\pgfpathlineto{\pgfqpoint{7.707075in}{5.627778in}}%
\pgfusepath{stroke}%
\end{pgfscope}%
\begin{pgfscope}%
\definecolor{textcolor}{rgb}{0.000000,0.000000,0.000000}%
\pgfsetstrokecolor{textcolor}%
\pgfsetfillcolor{textcolor}%
\pgftext[x=6.049317in,y=5.711111in,,base]{\color{textcolor}\sffamily\fontsize{12.000000}{14.400000}\selectfont \(\displaystyle  N = 32 \)}%
\end{pgfscope}%
\begin{pgfscope}%
\pgfsetbuttcap%
\pgfsetmiterjoin%
\definecolor{currentfill}{rgb}{1.000000,1.000000,1.000000}%
\pgfsetfillcolor{currentfill}%
\pgfsetlinewidth{0.000000pt}%
\definecolor{currentstroke}{rgb}{0.000000,0.000000,0.000000}%
\pgfsetstrokecolor{currentstroke}%
\pgfsetstrokeopacity{0.000000}%
\pgfsetdash{}{0pt}%
\pgfpathmoveto{\pgfqpoint{0.557986in}{0.386111in}}%
\pgfpathlineto{\pgfqpoint{3.873503in}{0.386111in}}%
\pgfpathlineto{\pgfqpoint{3.873503in}{2.702778in}}%
\pgfpathlineto{\pgfqpoint{0.557986in}{2.702778in}}%
\pgfpathclose%
\pgfusepath{fill}%
\end{pgfscope}%
\begin{pgfscope}%
\pgfpathrectangle{\pgfqpoint{0.557986in}{0.386111in}}{\pgfqpoint{3.315517in}{2.316667in}}%
\pgfusepath{clip}%
\pgfsetbuttcap%
\pgfsetroundjoin%
\definecolor{currentfill}{rgb}{0.121569,0.466667,0.705882}%
\pgfsetfillcolor{currentfill}%
\pgfsetfillopacity{0.300000}%
\pgfsetlinewidth{1.003750pt}%
\definecolor{currentstroke}{rgb}{0.121569,0.466667,0.705882}%
\pgfsetstrokecolor{currentstroke}%
\pgfsetstrokeopacity{0.300000}%
\pgfsetdash{}{0pt}%
\pgfpathmoveto{\pgfqpoint{0.720158in}{0.841050in}}%
\pgfpathlineto{\pgfqpoint{0.720158in}{0.707552in}}%
\pgfpathlineto{\pgfqpoint{0.869716in}{0.714749in}}%
\pgfpathlineto{\pgfqpoint{1.019275in}{0.706080in}}%
\pgfpathlineto{\pgfqpoint{1.168834in}{0.815961in}}%
\pgfpathlineto{\pgfqpoint{1.318392in}{0.987676in}}%
\pgfpathlineto{\pgfqpoint{1.467951in}{0.983227in}}%
\pgfpathlineto{\pgfqpoint{1.617510in}{1.192678in}}%
\pgfpathlineto{\pgfqpoint{1.767068in}{1.415337in}}%
\pgfpathlineto{\pgfqpoint{1.916627in}{1.559459in}}%
\pgfpathlineto{\pgfqpoint{2.066186in}{1.415859in}}%
\pgfpathlineto{\pgfqpoint{2.215744in}{1.834713in}}%
\pgfpathlineto{\pgfqpoint{2.365303in}{1.734389in}}%
\pgfpathlineto{\pgfqpoint{2.514862in}{1.696870in}}%
\pgfpathlineto{\pgfqpoint{2.664421in}{1.412017in}}%
\pgfpathlineto{\pgfqpoint{2.813979in}{1.004753in}}%
\pgfpathlineto{\pgfqpoint{2.963538in}{1.023791in}}%
\pgfpathlineto{\pgfqpoint{3.113097in}{0.863502in}}%
\pgfpathlineto{\pgfqpoint{3.262655in}{0.786027in}}%
\pgfpathlineto{\pgfqpoint{3.412214in}{0.806228in}}%
\pgfpathlineto{\pgfqpoint{3.561773in}{0.617676in}}%
\pgfpathlineto{\pgfqpoint{3.711331in}{0.491414in}}%
\pgfpathlineto{\pgfqpoint{3.711331in}{0.698133in}}%
\pgfpathlineto{\pgfqpoint{3.711331in}{0.698133in}}%
\pgfpathlineto{\pgfqpoint{3.561773in}{0.838595in}}%
\pgfpathlineto{\pgfqpoint{3.412214in}{0.970632in}}%
\pgfpathlineto{\pgfqpoint{3.262655in}{1.014582in}}%
\pgfpathlineto{\pgfqpoint{3.113097in}{1.273521in}}%
\pgfpathlineto{\pgfqpoint{2.963538in}{1.485895in}}%
\pgfpathlineto{\pgfqpoint{2.813979in}{1.419598in}}%
\pgfpathlineto{\pgfqpoint{2.664421in}{1.831311in}}%
\pgfpathlineto{\pgfqpoint{2.514862in}{1.958165in}}%
\pgfpathlineto{\pgfqpoint{2.365303in}{2.234886in}}%
\pgfpathlineto{\pgfqpoint{2.215744in}{2.040163in}}%
\pgfpathlineto{\pgfqpoint{2.066186in}{2.597475in}}%
\pgfpathlineto{\pgfqpoint{1.916627in}{2.452519in}}%
\pgfpathlineto{\pgfqpoint{1.767068in}{1.730838in}}%
\pgfpathlineto{\pgfqpoint{1.617510in}{1.406216in}}%
\pgfpathlineto{\pgfqpoint{1.467951in}{1.332436in}}%
\pgfpathlineto{\pgfqpoint{1.318392in}{1.939155in}}%
\pgfpathlineto{\pgfqpoint{1.168834in}{1.044660in}}%
\pgfpathlineto{\pgfqpoint{1.019275in}{1.103751in}}%
\pgfpathlineto{\pgfqpoint{0.869716in}{0.859758in}}%
\pgfpathlineto{\pgfqpoint{0.720158in}{0.841050in}}%
\pgfpathclose%
\pgfusepath{stroke,fill}%
\end{pgfscope}%
\begin{pgfscope}%
\pgfpathrectangle{\pgfqpoint{0.557986in}{0.386111in}}{\pgfqpoint{3.315517in}{2.316667in}}%
\pgfusepath{clip}%
\pgfsetbuttcap%
\pgfsetroundjoin%
\definecolor{currentfill}{rgb}{0.121569,0.466667,0.705882}%
\pgfsetfillcolor{currentfill}%
\pgfsetlinewidth{1.003750pt}%
\definecolor{currentstroke}{rgb}{0.121569,0.466667,0.705882}%
\pgfsetstrokecolor{currentstroke}%
\pgfsetdash{}{0pt}%
\pgfsys@defobject{currentmarker}{\pgfqpoint{-0.009821in}{-0.009821in}}{\pgfqpoint{0.009821in}{0.009821in}}{%
\pgfpathmoveto{\pgfqpoint{0.000000in}{-0.009821in}}%
\pgfpathcurveto{\pgfqpoint{0.002605in}{-0.009821in}}{\pgfqpoint{0.005103in}{-0.008786in}}{\pgfqpoint{0.006944in}{-0.006944in}}%
\pgfpathcurveto{\pgfqpoint{0.008786in}{-0.005103in}}{\pgfqpoint{0.009821in}{-0.002605in}}{\pgfqpoint{0.009821in}{0.000000in}}%
\pgfpathcurveto{\pgfqpoint{0.009821in}{0.002605in}}{\pgfqpoint{0.008786in}{0.005103in}}{\pgfqpoint{0.006944in}{0.006944in}}%
\pgfpathcurveto{\pgfqpoint{0.005103in}{0.008786in}}{\pgfqpoint{0.002605in}{0.009821in}}{\pgfqpoint{0.000000in}{0.009821in}}%
\pgfpathcurveto{\pgfqpoint{-0.002605in}{0.009821in}}{\pgfqpoint{-0.005103in}{0.008786in}}{\pgfqpoint{-0.006944in}{0.006944in}}%
\pgfpathcurveto{\pgfqpoint{-0.008786in}{0.005103in}}{\pgfqpoint{-0.009821in}{0.002605in}}{\pgfqpoint{-0.009821in}{0.000000in}}%
\pgfpathcurveto{\pgfqpoint{-0.009821in}{-0.002605in}}{\pgfqpoint{-0.008786in}{-0.005103in}}{\pgfqpoint{-0.006944in}{-0.006944in}}%
\pgfpathcurveto{\pgfqpoint{-0.005103in}{-0.008786in}}{\pgfqpoint{-0.002605in}{-0.009821in}}{\pgfqpoint{0.000000in}{-0.009821in}}%
\pgfpathclose%
\pgfusepath{stroke,fill}%
}%
\begin{pgfscope}%
\pgfsys@transformshift{0.720158in}{0.774301in}%
\pgfsys@useobject{currentmarker}{}%
\end{pgfscope}%
\begin{pgfscope}%
\pgfsys@transformshift{0.869716in}{0.787253in}%
\pgfsys@useobject{currentmarker}{}%
\end{pgfscope}%
\begin{pgfscope}%
\pgfsys@transformshift{1.019275in}{0.904916in}%
\pgfsys@useobject{currentmarker}{}%
\end{pgfscope}%
\begin{pgfscope}%
\pgfsys@transformshift{1.168834in}{0.930311in}%
\pgfsys@useobject{currentmarker}{}%
\end{pgfscope}%
\begin{pgfscope}%
\pgfsys@transformshift{1.318392in}{1.463415in}%
\pgfsys@useobject{currentmarker}{}%
\end{pgfscope}%
\begin{pgfscope}%
\pgfsys@transformshift{1.467951in}{1.157831in}%
\pgfsys@useobject{currentmarker}{}%
\end{pgfscope}%
\begin{pgfscope}%
\pgfsys@transformshift{1.617510in}{1.299447in}%
\pgfsys@useobject{currentmarker}{}%
\end{pgfscope}%
\begin{pgfscope}%
\pgfsys@transformshift{1.767068in}{1.573087in}%
\pgfsys@useobject{currentmarker}{}%
\end{pgfscope}%
\begin{pgfscope}%
\pgfsys@transformshift{1.916627in}{2.005989in}%
\pgfsys@useobject{currentmarker}{}%
\end{pgfscope}%
\begin{pgfscope}%
\pgfsys@transformshift{2.066186in}{2.006667in}%
\pgfsys@useobject{currentmarker}{}%
\end{pgfscope}%
\begin{pgfscope}%
\pgfsys@transformshift{2.215744in}{1.937438in}%
\pgfsys@useobject{currentmarker}{}%
\end{pgfscope}%
\begin{pgfscope}%
\pgfsys@transformshift{2.365303in}{1.984638in}%
\pgfsys@useobject{currentmarker}{}%
\end{pgfscope}%
\begin{pgfscope}%
\pgfsys@transformshift{2.514862in}{1.827517in}%
\pgfsys@useobject{currentmarker}{}%
\end{pgfscope}%
\begin{pgfscope}%
\pgfsys@transformshift{2.664421in}{1.621664in}%
\pgfsys@useobject{currentmarker}{}%
\end{pgfscope}%
\begin{pgfscope}%
\pgfsys@transformshift{2.813979in}{1.212175in}%
\pgfsys@useobject{currentmarker}{}%
\end{pgfscope}%
\begin{pgfscope}%
\pgfsys@transformshift{2.963538in}{1.254843in}%
\pgfsys@useobject{currentmarker}{}%
\end{pgfscope}%
\begin{pgfscope}%
\pgfsys@transformshift{3.113097in}{1.068511in}%
\pgfsys@useobject{currentmarker}{}%
\end{pgfscope}%
\begin{pgfscope}%
\pgfsys@transformshift{3.262655in}{0.900304in}%
\pgfsys@useobject{currentmarker}{}%
\end{pgfscope}%
\begin{pgfscope}%
\pgfsys@transformshift{3.412214in}{0.888430in}%
\pgfsys@useobject{currentmarker}{}%
\end{pgfscope}%
\begin{pgfscope}%
\pgfsys@transformshift{3.561773in}{0.728135in}%
\pgfsys@useobject{currentmarker}{}%
\end{pgfscope}%
\begin{pgfscope}%
\pgfsys@transformshift{3.711331in}{0.594774in}%
\pgfsys@useobject{currentmarker}{}%
\end{pgfscope}%
\end{pgfscope}%
\begin{pgfscope}%
\pgfsetbuttcap%
\pgfsetroundjoin%
\definecolor{currentfill}{rgb}{0.000000,0.000000,0.000000}%
\pgfsetfillcolor{currentfill}%
\pgfsetlinewidth{0.803000pt}%
\definecolor{currentstroke}{rgb}{0.000000,0.000000,0.000000}%
\pgfsetstrokecolor{currentstroke}%
\pgfsetdash{}{0pt}%
\pgfsys@defobject{currentmarker}{\pgfqpoint{0.000000in}{-0.048611in}}{\pgfqpoint{0.000000in}{0.000000in}}{%
\pgfpathmoveto{\pgfqpoint{0.000000in}{0.000000in}}%
\pgfpathlineto{\pgfqpoint{0.000000in}{-0.048611in}}%
\pgfusepath{stroke,fill}%
}%
\begin{pgfscope}%
\pgfsys@transformshift{1.019275in}{0.386111in}%
\pgfsys@useobject{currentmarker}{}%
\end{pgfscope}%
\end{pgfscope}%
\begin{pgfscope}%
\definecolor{textcolor}{rgb}{0.000000,0.000000,0.000000}%
\pgfsetstrokecolor{textcolor}%
\pgfsetfillcolor{textcolor}%
\pgftext[x=1.019275in,y=0.288889in,,top]{\color{textcolor}\sffamily\fontsize{10.000000}{12.000000}\selectfont 2.20}%
\end{pgfscope}%
\begin{pgfscope}%
\pgfsetbuttcap%
\pgfsetroundjoin%
\definecolor{currentfill}{rgb}{0.000000,0.000000,0.000000}%
\pgfsetfillcolor{currentfill}%
\pgfsetlinewidth{0.803000pt}%
\definecolor{currentstroke}{rgb}{0.000000,0.000000,0.000000}%
\pgfsetstrokecolor{currentstroke}%
\pgfsetdash{}{0pt}%
\pgfsys@defobject{currentmarker}{\pgfqpoint{0.000000in}{-0.048611in}}{\pgfqpoint{0.000000in}{0.000000in}}{%
\pgfpathmoveto{\pgfqpoint{0.000000in}{0.000000in}}%
\pgfpathlineto{\pgfqpoint{0.000000in}{-0.048611in}}%
\pgfusepath{stroke,fill}%
}%
\begin{pgfscope}%
\pgfsys@transformshift{1.767068in}{0.386111in}%
\pgfsys@useobject{currentmarker}{}%
\end{pgfscope}%
\end{pgfscope}%
\begin{pgfscope}%
\definecolor{textcolor}{rgb}{0.000000,0.000000,0.000000}%
\pgfsetstrokecolor{textcolor}%
\pgfsetfillcolor{textcolor}%
\pgftext[x=1.767068in,y=0.288889in,,top]{\color{textcolor}\sffamily\fontsize{10.000000}{12.000000}\selectfont 2.25}%
\end{pgfscope}%
\begin{pgfscope}%
\pgfsetbuttcap%
\pgfsetroundjoin%
\definecolor{currentfill}{rgb}{0.000000,0.000000,0.000000}%
\pgfsetfillcolor{currentfill}%
\pgfsetlinewidth{0.803000pt}%
\definecolor{currentstroke}{rgb}{0.000000,0.000000,0.000000}%
\pgfsetstrokecolor{currentstroke}%
\pgfsetdash{}{0pt}%
\pgfsys@defobject{currentmarker}{\pgfqpoint{0.000000in}{-0.048611in}}{\pgfqpoint{0.000000in}{0.000000in}}{%
\pgfpathmoveto{\pgfqpoint{0.000000in}{0.000000in}}%
\pgfpathlineto{\pgfqpoint{0.000000in}{-0.048611in}}%
\pgfusepath{stroke,fill}%
}%
\begin{pgfscope}%
\pgfsys@transformshift{2.514862in}{0.386111in}%
\pgfsys@useobject{currentmarker}{}%
\end{pgfscope}%
\end{pgfscope}%
\begin{pgfscope}%
\definecolor{textcolor}{rgb}{0.000000,0.000000,0.000000}%
\pgfsetstrokecolor{textcolor}%
\pgfsetfillcolor{textcolor}%
\pgftext[x=2.514862in,y=0.288889in,,top]{\color{textcolor}\sffamily\fontsize{10.000000}{12.000000}\selectfont 2.30}%
\end{pgfscope}%
\begin{pgfscope}%
\pgfsetbuttcap%
\pgfsetroundjoin%
\definecolor{currentfill}{rgb}{0.000000,0.000000,0.000000}%
\pgfsetfillcolor{currentfill}%
\pgfsetlinewidth{0.803000pt}%
\definecolor{currentstroke}{rgb}{0.000000,0.000000,0.000000}%
\pgfsetstrokecolor{currentstroke}%
\pgfsetdash{}{0pt}%
\pgfsys@defobject{currentmarker}{\pgfqpoint{0.000000in}{-0.048611in}}{\pgfqpoint{0.000000in}{0.000000in}}{%
\pgfpathmoveto{\pgfqpoint{0.000000in}{0.000000in}}%
\pgfpathlineto{\pgfqpoint{0.000000in}{-0.048611in}}%
\pgfusepath{stroke,fill}%
}%
\begin{pgfscope}%
\pgfsys@transformshift{3.262655in}{0.386111in}%
\pgfsys@useobject{currentmarker}{}%
\end{pgfscope}%
\end{pgfscope}%
\begin{pgfscope}%
\definecolor{textcolor}{rgb}{0.000000,0.000000,0.000000}%
\pgfsetstrokecolor{textcolor}%
\pgfsetfillcolor{textcolor}%
\pgftext[x=3.262655in,y=0.288889in,,top]{\color{textcolor}\sffamily\fontsize{10.000000}{12.000000}\selectfont 2.35}%
\end{pgfscope}%
\begin{pgfscope}%
\pgfsetbuttcap%
\pgfsetroundjoin%
\definecolor{currentfill}{rgb}{0.000000,0.000000,0.000000}%
\pgfsetfillcolor{currentfill}%
\pgfsetlinewidth{0.803000pt}%
\definecolor{currentstroke}{rgb}{0.000000,0.000000,0.000000}%
\pgfsetstrokecolor{currentstroke}%
\pgfsetdash{}{0pt}%
\pgfsys@defobject{currentmarker}{\pgfqpoint{-0.048611in}{0.000000in}}{\pgfqpoint{0.000000in}{0.000000in}}{%
\pgfpathmoveto{\pgfqpoint{0.000000in}{0.000000in}}%
\pgfpathlineto{\pgfqpoint{-0.048611in}{0.000000in}}%
\pgfusepath{stroke,fill}%
}%
\begin{pgfscope}%
\pgfsys@transformshift{0.557986in}{0.667955in}%
\pgfsys@useobject{currentmarker}{}%
\end{pgfscope}%
\end{pgfscope}%
\begin{pgfscope}%
\definecolor{textcolor}{rgb}{0.000000,0.000000,0.000000}%
\pgfsetstrokecolor{textcolor}%
\pgfsetfillcolor{textcolor}%
\pgftext[x=0.151519in,y=0.615193in,left,base]{\color{textcolor}\sffamily\fontsize{10.000000}{12.000000}\selectfont 1.25}%
\end{pgfscope}%
\begin{pgfscope}%
\pgfsetbuttcap%
\pgfsetroundjoin%
\definecolor{currentfill}{rgb}{0.000000,0.000000,0.000000}%
\pgfsetfillcolor{currentfill}%
\pgfsetlinewidth{0.803000pt}%
\definecolor{currentstroke}{rgb}{0.000000,0.000000,0.000000}%
\pgfsetstrokecolor{currentstroke}%
\pgfsetdash{}{0pt}%
\pgfsys@defobject{currentmarker}{\pgfqpoint{-0.048611in}{0.000000in}}{\pgfqpoint{0.000000in}{0.000000in}}{%
\pgfpathmoveto{\pgfqpoint{0.000000in}{0.000000in}}%
\pgfpathlineto{\pgfqpoint{-0.048611in}{0.000000in}}%
\pgfusepath{stroke,fill}%
}%
\begin{pgfscope}%
\pgfsys@transformshift{0.557986in}{0.997289in}%
\pgfsys@useobject{currentmarker}{}%
\end{pgfscope}%
\end{pgfscope}%
\begin{pgfscope}%
\definecolor{textcolor}{rgb}{0.000000,0.000000,0.000000}%
\pgfsetstrokecolor{textcolor}%
\pgfsetfillcolor{textcolor}%
\pgftext[x=0.151519in,y=0.944528in,left,base]{\color{textcolor}\sffamily\fontsize{10.000000}{12.000000}\selectfont 1.50}%
\end{pgfscope}%
\begin{pgfscope}%
\pgfsetbuttcap%
\pgfsetroundjoin%
\definecolor{currentfill}{rgb}{0.000000,0.000000,0.000000}%
\pgfsetfillcolor{currentfill}%
\pgfsetlinewidth{0.803000pt}%
\definecolor{currentstroke}{rgb}{0.000000,0.000000,0.000000}%
\pgfsetstrokecolor{currentstroke}%
\pgfsetdash{}{0pt}%
\pgfsys@defobject{currentmarker}{\pgfqpoint{-0.048611in}{0.000000in}}{\pgfqpoint{0.000000in}{0.000000in}}{%
\pgfpathmoveto{\pgfqpoint{0.000000in}{0.000000in}}%
\pgfpathlineto{\pgfqpoint{-0.048611in}{0.000000in}}%
\pgfusepath{stroke,fill}%
}%
\begin{pgfscope}%
\pgfsys@transformshift{0.557986in}{1.326624in}%
\pgfsys@useobject{currentmarker}{}%
\end{pgfscope}%
\end{pgfscope}%
\begin{pgfscope}%
\definecolor{textcolor}{rgb}{0.000000,0.000000,0.000000}%
\pgfsetstrokecolor{textcolor}%
\pgfsetfillcolor{textcolor}%
\pgftext[x=0.151519in,y=1.273862in,left,base]{\color{textcolor}\sffamily\fontsize{10.000000}{12.000000}\selectfont 1.75}%
\end{pgfscope}%
\begin{pgfscope}%
\pgfsetbuttcap%
\pgfsetroundjoin%
\definecolor{currentfill}{rgb}{0.000000,0.000000,0.000000}%
\pgfsetfillcolor{currentfill}%
\pgfsetlinewidth{0.803000pt}%
\definecolor{currentstroke}{rgb}{0.000000,0.000000,0.000000}%
\pgfsetstrokecolor{currentstroke}%
\pgfsetdash{}{0pt}%
\pgfsys@defobject{currentmarker}{\pgfqpoint{-0.048611in}{0.000000in}}{\pgfqpoint{0.000000in}{0.000000in}}{%
\pgfpathmoveto{\pgfqpoint{0.000000in}{0.000000in}}%
\pgfpathlineto{\pgfqpoint{-0.048611in}{0.000000in}}%
\pgfusepath{stroke,fill}%
}%
\begin{pgfscope}%
\pgfsys@transformshift{0.557986in}{1.655958in}%
\pgfsys@useobject{currentmarker}{}%
\end{pgfscope}%
\end{pgfscope}%
\begin{pgfscope}%
\definecolor{textcolor}{rgb}{0.000000,0.000000,0.000000}%
\pgfsetstrokecolor{textcolor}%
\pgfsetfillcolor{textcolor}%
\pgftext[x=0.151519in,y=1.603196in,left,base]{\color{textcolor}\sffamily\fontsize{10.000000}{12.000000}\selectfont 2.00}%
\end{pgfscope}%
\begin{pgfscope}%
\pgfsetbuttcap%
\pgfsetroundjoin%
\definecolor{currentfill}{rgb}{0.000000,0.000000,0.000000}%
\pgfsetfillcolor{currentfill}%
\pgfsetlinewidth{0.803000pt}%
\definecolor{currentstroke}{rgb}{0.000000,0.000000,0.000000}%
\pgfsetstrokecolor{currentstroke}%
\pgfsetdash{}{0pt}%
\pgfsys@defobject{currentmarker}{\pgfqpoint{-0.048611in}{0.000000in}}{\pgfqpoint{0.000000in}{0.000000in}}{%
\pgfpathmoveto{\pgfqpoint{0.000000in}{0.000000in}}%
\pgfpathlineto{\pgfqpoint{-0.048611in}{0.000000in}}%
\pgfusepath{stroke,fill}%
}%
\begin{pgfscope}%
\pgfsys@transformshift{0.557986in}{1.985292in}%
\pgfsys@useobject{currentmarker}{}%
\end{pgfscope}%
\end{pgfscope}%
\begin{pgfscope}%
\definecolor{textcolor}{rgb}{0.000000,0.000000,0.000000}%
\pgfsetstrokecolor{textcolor}%
\pgfsetfillcolor{textcolor}%
\pgftext[x=0.151519in,y=1.932531in,left,base]{\color{textcolor}\sffamily\fontsize{10.000000}{12.000000}\selectfont 2.25}%
\end{pgfscope}%
\begin{pgfscope}%
\pgfsetbuttcap%
\pgfsetroundjoin%
\definecolor{currentfill}{rgb}{0.000000,0.000000,0.000000}%
\pgfsetfillcolor{currentfill}%
\pgfsetlinewidth{0.803000pt}%
\definecolor{currentstroke}{rgb}{0.000000,0.000000,0.000000}%
\pgfsetstrokecolor{currentstroke}%
\pgfsetdash{}{0pt}%
\pgfsys@defobject{currentmarker}{\pgfqpoint{-0.048611in}{0.000000in}}{\pgfqpoint{0.000000in}{0.000000in}}{%
\pgfpathmoveto{\pgfqpoint{0.000000in}{0.000000in}}%
\pgfpathlineto{\pgfqpoint{-0.048611in}{0.000000in}}%
\pgfusepath{stroke,fill}%
}%
\begin{pgfscope}%
\pgfsys@transformshift{0.557986in}{2.314627in}%
\pgfsys@useobject{currentmarker}{}%
\end{pgfscope}%
\end{pgfscope}%
\begin{pgfscope}%
\definecolor{textcolor}{rgb}{0.000000,0.000000,0.000000}%
\pgfsetstrokecolor{textcolor}%
\pgfsetfillcolor{textcolor}%
\pgftext[x=0.151519in,y=2.261865in,left,base]{\color{textcolor}\sffamily\fontsize{10.000000}{12.000000}\selectfont 2.50}%
\end{pgfscope}%
\begin{pgfscope}%
\pgfsetbuttcap%
\pgfsetroundjoin%
\definecolor{currentfill}{rgb}{0.000000,0.000000,0.000000}%
\pgfsetfillcolor{currentfill}%
\pgfsetlinewidth{0.803000pt}%
\definecolor{currentstroke}{rgb}{0.000000,0.000000,0.000000}%
\pgfsetstrokecolor{currentstroke}%
\pgfsetdash{}{0pt}%
\pgfsys@defobject{currentmarker}{\pgfqpoint{-0.048611in}{0.000000in}}{\pgfqpoint{0.000000in}{0.000000in}}{%
\pgfpathmoveto{\pgfqpoint{0.000000in}{0.000000in}}%
\pgfpathlineto{\pgfqpoint{-0.048611in}{0.000000in}}%
\pgfusepath{stroke,fill}%
}%
\begin{pgfscope}%
\pgfsys@transformshift{0.557986in}{2.643961in}%
\pgfsys@useobject{currentmarker}{}%
\end{pgfscope}%
\end{pgfscope}%
\begin{pgfscope}%
\definecolor{textcolor}{rgb}{0.000000,0.000000,0.000000}%
\pgfsetstrokecolor{textcolor}%
\pgfsetfillcolor{textcolor}%
\pgftext[x=0.151519in,y=2.591199in,left,base]{\color{textcolor}\sffamily\fontsize{10.000000}{12.000000}\selectfont 2.75}%
\end{pgfscope}%
\begin{pgfscope}%
\pgfpathrectangle{\pgfqpoint{0.557986in}{0.386111in}}{\pgfqpoint{3.315517in}{2.316667in}}%
\pgfusepath{clip}%
\pgfsetrectcap%
\pgfsetroundjoin%
\pgfsetlinewidth{1.505625pt}%
\definecolor{currentstroke}{rgb}{0.121569,0.466667,0.705882}%
\pgfsetstrokecolor{currentstroke}%
\pgfsetdash{}{0pt}%
\pgfpathmoveto{\pgfqpoint{0.720158in}{0.774301in}}%
\pgfpathlineto{\pgfqpoint{0.869716in}{0.787253in}}%
\pgfpathlineto{\pgfqpoint{1.019275in}{0.904916in}}%
\pgfpathlineto{\pgfqpoint{1.168834in}{0.930311in}}%
\pgfpathlineto{\pgfqpoint{1.318392in}{1.463415in}}%
\pgfpathlineto{\pgfqpoint{1.467951in}{1.157831in}}%
\pgfpathlineto{\pgfqpoint{1.617510in}{1.299447in}}%
\pgfpathlineto{\pgfqpoint{1.767068in}{1.573087in}}%
\pgfpathlineto{\pgfqpoint{1.916627in}{2.005989in}}%
\pgfpathlineto{\pgfqpoint{2.066186in}{2.006667in}}%
\pgfpathlineto{\pgfqpoint{2.215744in}{1.937438in}}%
\pgfpathlineto{\pgfqpoint{2.365303in}{1.984638in}}%
\pgfpathlineto{\pgfqpoint{2.514862in}{1.827517in}}%
\pgfpathlineto{\pgfqpoint{2.664421in}{1.621664in}}%
\pgfpathlineto{\pgfqpoint{2.813979in}{1.212175in}}%
\pgfpathlineto{\pgfqpoint{2.963538in}{1.254843in}}%
\pgfpathlineto{\pgfqpoint{3.113097in}{1.068511in}}%
\pgfpathlineto{\pgfqpoint{3.262655in}{0.900304in}}%
\pgfpathlineto{\pgfqpoint{3.412214in}{0.888430in}}%
\pgfpathlineto{\pgfqpoint{3.561773in}{0.728135in}}%
\pgfpathlineto{\pgfqpoint{3.711331in}{0.594774in}}%
\pgfusepath{stroke}%
\end{pgfscope}%
\begin{pgfscope}%
\pgfsetrectcap%
\pgfsetmiterjoin%
\pgfsetlinewidth{0.803000pt}%
\definecolor{currentstroke}{rgb}{0.000000,0.000000,0.000000}%
\pgfsetstrokecolor{currentstroke}%
\pgfsetdash{}{0pt}%
\pgfpathmoveto{\pgfqpoint{0.557986in}{0.386111in}}%
\pgfpathlineto{\pgfqpoint{0.557986in}{2.702778in}}%
\pgfusepath{stroke}%
\end{pgfscope}%
\begin{pgfscope}%
\pgfsetrectcap%
\pgfsetmiterjoin%
\pgfsetlinewidth{0.803000pt}%
\definecolor{currentstroke}{rgb}{0.000000,0.000000,0.000000}%
\pgfsetstrokecolor{currentstroke}%
\pgfsetdash{}{0pt}%
\pgfpathmoveto{\pgfqpoint{3.873503in}{0.386111in}}%
\pgfpathlineto{\pgfqpoint{3.873503in}{2.702778in}}%
\pgfusepath{stroke}%
\end{pgfscope}%
\begin{pgfscope}%
\pgfsetrectcap%
\pgfsetmiterjoin%
\pgfsetlinewidth{0.803000pt}%
\definecolor{currentstroke}{rgb}{0.000000,0.000000,0.000000}%
\pgfsetstrokecolor{currentstroke}%
\pgfsetdash{}{0pt}%
\pgfpathmoveto{\pgfqpoint{0.557986in}{0.386111in}}%
\pgfpathlineto{\pgfqpoint{3.873503in}{0.386111in}}%
\pgfusepath{stroke}%
\end{pgfscope}%
\begin{pgfscope}%
\pgfsetrectcap%
\pgfsetmiterjoin%
\pgfsetlinewidth{0.803000pt}%
\definecolor{currentstroke}{rgb}{0.000000,0.000000,0.000000}%
\pgfsetstrokecolor{currentstroke}%
\pgfsetdash{}{0pt}%
\pgfpathmoveto{\pgfqpoint{0.557986in}{2.702778in}}%
\pgfpathlineto{\pgfqpoint{3.873503in}{2.702778in}}%
\pgfusepath{stroke}%
\end{pgfscope}%
\begin{pgfscope}%
\definecolor{textcolor}{rgb}{0.000000,0.000000,0.000000}%
\pgfsetstrokecolor{textcolor}%
\pgfsetfillcolor{textcolor}%
\pgftext[x=2.215744in,y=2.786111in,,base]{\color{textcolor}\sffamily\fontsize{12.000000}{14.400000}\selectfont \(\displaystyle  N = 64 \)}%
\end{pgfscope}%
\begin{pgfscope}%
\pgfsetbuttcap%
\pgfsetmiterjoin%
\definecolor{currentfill}{rgb}{1.000000,1.000000,1.000000}%
\pgfsetfillcolor{currentfill}%
\pgfsetlinewidth{0.000000pt}%
\definecolor{currentstroke}{rgb}{0.000000,0.000000,0.000000}%
\pgfsetstrokecolor{currentstroke}%
\pgfsetstrokeopacity{0.000000}%
\pgfsetdash{}{0pt}%
\pgfpathmoveto{\pgfqpoint{4.391558in}{0.386111in}}%
\pgfpathlineto{\pgfqpoint{7.707075in}{0.386111in}}%
\pgfpathlineto{\pgfqpoint{7.707075in}{2.702778in}}%
\pgfpathlineto{\pgfqpoint{4.391558in}{2.702778in}}%
\pgfpathclose%
\pgfusepath{fill}%
\end{pgfscope}%
\begin{pgfscope}%
\pgfpathrectangle{\pgfqpoint{4.391558in}{0.386111in}}{\pgfqpoint{3.315517in}{2.316667in}}%
\pgfusepath{clip}%
\pgfsetbuttcap%
\pgfsetroundjoin%
\definecolor{currentfill}{rgb}{0.121569,0.466667,0.705882}%
\pgfsetfillcolor{currentfill}%
\pgfsetfillopacity{0.300000}%
\pgfsetlinewidth{1.003750pt}%
\definecolor{currentstroke}{rgb}{0.121569,0.466667,0.705882}%
\pgfsetstrokecolor{currentstroke}%
\pgfsetstrokeopacity{0.300000}%
\pgfsetdash{}{0pt}%
\pgfpathmoveto{\pgfqpoint{4.553730in}{0.796096in}}%
\pgfpathlineto{\pgfqpoint{4.553730in}{0.645591in}}%
\pgfpathlineto{\pgfqpoint{4.703288in}{0.602274in}}%
\pgfpathlineto{\pgfqpoint{4.852847in}{0.826245in}}%
\pgfpathlineto{\pgfqpoint{5.002406in}{0.736438in}}%
\pgfpathlineto{\pgfqpoint{5.151965in}{0.659594in}}%
\pgfpathlineto{\pgfqpoint{5.301523in}{0.491414in}}%
\pgfpathlineto{\pgfqpoint{5.451082in}{0.916871in}}%
\pgfpathlineto{\pgfqpoint{5.600641in}{1.012680in}}%
\pgfpathlineto{\pgfqpoint{5.750199in}{1.073688in}}%
\pgfpathlineto{\pgfqpoint{5.899758in}{0.707848in}}%
\pgfpathlineto{\pgfqpoint{6.049317in}{0.840126in}}%
\pgfpathlineto{\pgfqpoint{6.198875in}{0.987934in}}%
\pgfpathlineto{\pgfqpoint{6.348434in}{0.798822in}}%
\pgfpathlineto{\pgfqpoint{6.497993in}{0.838564in}}%
\pgfpathlineto{\pgfqpoint{6.647551in}{0.896111in}}%
\pgfpathlineto{\pgfqpoint{6.797110in}{0.783146in}}%
\pgfpathlineto{\pgfqpoint{6.946669in}{0.771321in}}%
\pgfpathlineto{\pgfqpoint{7.096227in}{0.740981in}}%
\pgfpathlineto{\pgfqpoint{7.245786in}{0.729563in}}%
\pgfpathlineto{\pgfqpoint{7.395345in}{0.698765in}}%
\pgfpathlineto{\pgfqpoint{7.544904in}{0.693373in}}%
\pgfpathlineto{\pgfqpoint{7.544904in}{0.770861in}}%
\pgfpathlineto{\pgfqpoint{7.544904in}{0.770861in}}%
\pgfpathlineto{\pgfqpoint{7.395345in}{0.813918in}}%
\pgfpathlineto{\pgfqpoint{7.245786in}{0.775496in}}%
\pgfpathlineto{\pgfqpoint{7.096227in}{0.933754in}}%
\pgfpathlineto{\pgfqpoint{6.946669in}{1.152611in}}%
\pgfpathlineto{\pgfqpoint{6.797110in}{0.944134in}}%
\pgfpathlineto{\pgfqpoint{6.647551in}{1.199194in}}%
\pgfpathlineto{\pgfqpoint{6.497993in}{1.714222in}}%
\pgfpathlineto{\pgfqpoint{6.348434in}{1.557452in}}%
\pgfpathlineto{\pgfqpoint{6.198875in}{1.625745in}}%
\pgfpathlineto{\pgfqpoint{6.049317in}{1.827223in}}%
\pgfpathlineto{\pgfqpoint{5.899758in}{1.929606in}}%
\pgfpathlineto{\pgfqpoint{5.750199in}{2.022955in}}%
\pgfpathlineto{\pgfqpoint{5.600641in}{1.335705in}}%
\pgfpathlineto{\pgfqpoint{5.451082in}{1.211150in}}%
\pgfpathlineto{\pgfqpoint{5.301523in}{2.597475in}}%
\pgfpathlineto{\pgfqpoint{5.151965in}{1.740285in}}%
\pgfpathlineto{\pgfqpoint{5.002406in}{1.246240in}}%
\pgfpathlineto{\pgfqpoint{4.852847in}{2.503689in}}%
\pgfpathlineto{\pgfqpoint{4.703288in}{1.139014in}}%
\pgfpathlineto{\pgfqpoint{4.553730in}{0.796096in}}%
\pgfpathclose%
\pgfusepath{stroke,fill}%
\end{pgfscope}%
\begin{pgfscope}%
\pgfpathrectangle{\pgfqpoint{4.391558in}{0.386111in}}{\pgfqpoint{3.315517in}{2.316667in}}%
\pgfusepath{clip}%
\pgfsetbuttcap%
\pgfsetroundjoin%
\definecolor{currentfill}{rgb}{0.121569,0.466667,0.705882}%
\pgfsetfillcolor{currentfill}%
\pgfsetlinewidth{1.003750pt}%
\definecolor{currentstroke}{rgb}{0.121569,0.466667,0.705882}%
\pgfsetstrokecolor{currentstroke}%
\pgfsetdash{}{0pt}%
\pgfsys@defobject{currentmarker}{\pgfqpoint{-0.009821in}{-0.009821in}}{\pgfqpoint{0.009821in}{0.009821in}}{%
\pgfpathmoveto{\pgfqpoint{0.000000in}{-0.009821in}}%
\pgfpathcurveto{\pgfqpoint{0.002605in}{-0.009821in}}{\pgfqpoint{0.005103in}{-0.008786in}}{\pgfqpoint{0.006944in}{-0.006944in}}%
\pgfpathcurveto{\pgfqpoint{0.008786in}{-0.005103in}}{\pgfqpoint{0.009821in}{-0.002605in}}{\pgfqpoint{0.009821in}{0.000000in}}%
\pgfpathcurveto{\pgfqpoint{0.009821in}{0.002605in}}{\pgfqpoint{0.008786in}{0.005103in}}{\pgfqpoint{0.006944in}{0.006944in}}%
\pgfpathcurveto{\pgfqpoint{0.005103in}{0.008786in}}{\pgfqpoint{0.002605in}{0.009821in}}{\pgfqpoint{0.000000in}{0.009821in}}%
\pgfpathcurveto{\pgfqpoint{-0.002605in}{0.009821in}}{\pgfqpoint{-0.005103in}{0.008786in}}{\pgfqpoint{-0.006944in}{0.006944in}}%
\pgfpathcurveto{\pgfqpoint{-0.008786in}{0.005103in}}{\pgfqpoint{-0.009821in}{0.002605in}}{\pgfqpoint{-0.009821in}{0.000000in}}%
\pgfpathcurveto{\pgfqpoint{-0.009821in}{-0.002605in}}{\pgfqpoint{-0.008786in}{-0.005103in}}{\pgfqpoint{-0.006944in}{-0.006944in}}%
\pgfpathcurveto{\pgfqpoint{-0.005103in}{-0.008786in}}{\pgfqpoint{-0.002605in}{-0.009821in}}{\pgfqpoint{0.000000in}{-0.009821in}}%
\pgfpathclose%
\pgfusepath{stroke,fill}%
}%
\begin{pgfscope}%
\pgfsys@transformshift{4.553730in}{0.720843in}%
\pgfsys@useobject{currentmarker}{}%
\end{pgfscope}%
\begin{pgfscope}%
\pgfsys@transformshift{4.703288in}{0.870644in}%
\pgfsys@useobject{currentmarker}{}%
\end{pgfscope}%
\begin{pgfscope}%
\pgfsys@transformshift{4.852847in}{1.664967in}%
\pgfsys@useobject{currentmarker}{}%
\end{pgfscope}%
\begin{pgfscope}%
\pgfsys@transformshift{5.002406in}{0.991339in}%
\pgfsys@useobject{currentmarker}{}%
\end{pgfscope}%
\begin{pgfscope}%
\pgfsys@transformshift{5.151965in}{1.199940in}%
\pgfsys@useobject{currentmarker}{}%
\end{pgfscope}%
\begin{pgfscope}%
\pgfsys@transformshift{5.301523in}{1.544444in}%
\pgfsys@useobject{currentmarker}{}%
\end{pgfscope}%
\begin{pgfscope}%
\pgfsys@transformshift{5.451082in}{1.064011in}%
\pgfsys@useobject{currentmarker}{}%
\end{pgfscope}%
\begin{pgfscope}%
\pgfsys@transformshift{5.600641in}{1.174193in}%
\pgfsys@useobject{currentmarker}{}%
\end{pgfscope}%
\begin{pgfscope}%
\pgfsys@transformshift{5.750199in}{1.548322in}%
\pgfsys@useobject{currentmarker}{}%
\end{pgfscope}%
\begin{pgfscope}%
\pgfsys@transformshift{5.899758in}{1.318727in}%
\pgfsys@useobject{currentmarker}{}%
\end{pgfscope}%
\begin{pgfscope}%
\pgfsys@transformshift{6.049317in}{1.333675in}%
\pgfsys@useobject{currentmarker}{}%
\end{pgfscope}%
\begin{pgfscope}%
\pgfsys@transformshift{6.198875in}{1.306840in}%
\pgfsys@useobject{currentmarker}{}%
\end{pgfscope}%
\begin{pgfscope}%
\pgfsys@transformshift{6.348434in}{1.178137in}%
\pgfsys@useobject{currentmarker}{}%
\end{pgfscope}%
\begin{pgfscope}%
\pgfsys@transformshift{6.497993in}{1.276393in}%
\pgfsys@useobject{currentmarker}{}%
\end{pgfscope}%
\begin{pgfscope}%
\pgfsys@transformshift{6.647551in}{1.047652in}%
\pgfsys@useobject{currentmarker}{}%
\end{pgfscope}%
\begin{pgfscope}%
\pgfsys@transformshift{6.797110in}{0.863640in}%
\pgfsys@useobject{currentmarker}{}%
\end{pgfscope}%
\begin{pgfscope}%
\pgfsys@transformshift{6.946669in}{0.961966in}%
\pgfsys@useobject{currentmarker}{}%
\end{pgfscope}%
\begin{pgfscope}%
\pgfsys@transformshift{7.096227in}{0.837368in}%
\pgfsys@useobject{currentmarker}{}%
\end{pgfscope}%
\begin{pgfscope}%
\pgfsys@transformshift{7.245786in}{0.752530in}%
\pgfsys@useobject{currentmarker}{}%
\end{pgfscope}%
\begin{pgfscope}%
\pgfsys@transformshift{7.395345in}{0.756342in}%
\pgfsys@useobject{currentmarker}{}%
\end{pgfscope}%
\begin{pgfscope}%
\pgfsys@transformshift{7.544904in}{0.732117in}%
\pgfsys@useobject{currentmarker}{}%
\end{pgfscope}%
\end{pgfscope}%
\begin{pgfscope}%
\pgfsetbuttcap%
\pgfsetroundjoin%
\definecolor{currentfill}{rgb}{0.000000,0.000000,0.000000}%
\pgfsetfillcolor{currentfill}%
\pgfsetlinewidth{0.803000pt}%
\definecolor{currentstroke}{rgb}{0.000000,0.000000,0.000000}%
\pgfsetstrokecolor{currentstroke}%
\pgfsetdash{}{0pt}%
\pgfsys@defobject{currentmarker}{\pgfqpoint{0.000000in}{-0.048611in}}{\pgfqpoint{0.000000in}{0.000000in}}{%
\pgfpathmoveto{\pgfqpoint{0.000000in}{0.000000in}}%
\pgfpathlineto{\pgfqpoint{0.000000in}{-0.048611in}}%
\pgfusepath{stroke,fill}%
}%
\begin{pgfscope}%
\pgfsys@transformshift{5.002406in}{0.386111in}%
\pgfsys@useobject{currentmarker}{}%
\end{pgfscope}%
\end{pgfscope}%
\begin{pgfscope}%
\definecolor{textcolor}{rgb}{0.000000,0.000000,0.000000}%
\pgfsetstrokecolor{textcolor}%
\pgfsetfillcolor{textcolor}%
\pgftext[x=5.002406in,y=0.288889in,,top]{\color{textcolor}\sffamily\fontsize{10.000000}{12.000000}\selectfont 2.20}%
\end{pgfscope}%
\begin{pgfscope}%
\pgfsetbuttcap%
\pgfsetroundjoin%
\definecolor{currentfill}{rgb}{0.000000,0.000000,0.000000}%
\pgfsetfillcolor{currentfill}%
\pgfsetlinewidth{0.803000pt}%
\definecolor{currentstroke}{rgb}{0.000000,0.000000,0.000000}%
\pgfsetstrokecolor{currentstroke}%
\pgfsetdash{}{0pt}%
\pgfsys@defobject{currentmarker}{\pgfqpoint{0.000000in}{-0.048611in}}{\pgfqpoint{0.000000in}{0.000000in}}{%
\pgfpathmoveto{\pgfqpoint{0.000000in}{0.000000in}}%
\pgfpathlineto{\pgfqpoint{0.000000in}{-0.048611in}}%
\pgfusepath{stroke,fill}%
}%
\begin{pgfscope}%
\pgfsys@transformshift{5.750199in}{0.386111in}%
\pgfsys@useobject{currentmarker}{}%
\end{pgfscope}%
\end{pgfscope}%
\begin{pgfscope}%
\definecolor{textcolor}{rgb}{0.000000,0.000000,0.000000}%
\pgfsetstrokecolor{textcolor}%
\pgfsetfillcolor{textcolor}%
\pgftext[x=5.750199in,y=0.288889in,,top]{\color{textcolor}\sffamily\fontsize{10.000000}{12.000000}\selectfont 2.25}%
\end{pgfscope}%
\begin{pgfscope}%
\pgfsetbuttcap%
\pgfsetroundjoin%
\definecolor{currentfill}{rgb}{0.000000,0.000000,0.000000}%
\pgfsetfillcolor{currentfill}%
\pgfsetlinewidth{0.803000pt}%
\definecolor{currentstroke}{rgb}{0.000000,0.000000,0.000000}%
\pgfsetstrokecolor{currentstroke}%
\pgfsetdash{}{0pt}%
\pgfsys@defobject{currentmarker}{\pgfqpoint{0.000000in}{-0.048611in}}{\pgfqpoint{0.000000in}{0.000000in}}{%
\pgfpathmoveto{\pgfqpoint{0.000000in}{0.000000in}}%
\pgfpathlineto{\pgfqpoint{0.000000in}{-0.048611in}}%
\pgfusepath{stroke,fill}%
}%
\begin{pgfscope}%
\pgfsys@transformshift{6.497993in}{0.386111in}%
\pgfsys@useobject{currentmarker}{}%
\end{pgfscope}%
\end{pgfscope}%
\begin{pgfscope}%
\definecolor{textcolor}{rgb}{0.000000,0.000000,0.000000}%
\pgfsetstrokecolor{textcolor}%
\pgfsetfillcolor{textcolor}%
\pgftext[x=6.497993in,y=0.288889in,,top]{\color{textcolor}\sffamily\fontsize{10.000000}{12.000000}\selectfont 2.30}%
\end{pgfscope}%
\begin{pgfscope}%
\pgfsetbuttcap%
\pgfsetroundjoin%
\definecolor{currentfill}{rgb}{0.000000,0.000000,0.000000}%
\pgfsetfillcolor{currentfill}%
\pgfsetlinewidth{0.803000pt}%
\definecolor{currentstroke}{rgb}{0.000000,0.000000,0.000000}%
\pgfsetstrokecolor{currentstroke}%
\pgfsetdash{}{0pt}%
\pgfsys@defobject{currentmarker}{\pgfqpoint{0.000000in}{-0.048611in}}{\pgfqpoint{0.000000in}{0.000000in}}{%
\pgfpathmoveto{\pgfqpoint{0.000000in}{0.000000in}}%
\pgfpathlineto{\pgfqpoint{0.000000in}{-0.048611in}}%
\pgfusepath{stroke,fill}%
}%
\begin{pgfscope}%
\pgfsys@transformshift{7.245786in}{0.386111in}%
\pgfsys@useobject{currentmarker}{}%
\end{pgfscope}%
\end{pgfscope}%
\begin{pgfscope}%
\definecolor{textcolor}{rgb}{0.000000,0.000000,0.000000}%
\pgfsetstrokecolor{textcolor}%
\pgfsetfillcolor{textcolor}%
\pgftext[x=7.245786in,y=0.288889in,,top]{\color{textcolor}\sffamily\fontsize{10.000000}{12.000000}\selectfont 2.35}%
\end{pgfscope}%
\begin{pgfscope}%
\pgfsetbuttcap%
\pgfsetroundjoin%
\definecolor{currentfill}{rgb}{0.000000,0.000000,0.000000}%
\pgfsetfillcolor{currentfill}%
\pgfsetlinewidth{0.803000pt}%
\definecolor{currentstroke}{rgb}{0.000000,0.000000,0.000000}%
\pgfsetstrokecolor{currentstroke}%
\pgfsetdash{}{0pt}%
\pgfsys@defobject{currentmarker}{\pgfqpoint{-0.048611in}{0.000000in}}{\pgfqpoint{0.000000in}{0.000000in}}{%
\pgfpathmoveto{\pgfqpoint{0.000000in}{0.000000in}}%
\pgfpathlineto{\pgfqpoint{-0.048611in}{0.000000in}}%
\pgfusepath{stroke,fill}%
}%
\begin{pgfscope}%
\pgfsys@transformshift{4.391558in}{0.579585in}%
\pgfsys@useobject{currentmarker}{}%
\end{pgfscope}%
\end{pgfscope}%
\begin{pgfscope}%
\definecolor{textcolor}{rgb}{0.000000,0.000000,0.000000}%
\pgfsetstrokecolor{textcolor}%
\pgfsetfillcolor{textcolor}%
\pgftext[x=4.073457in,y=0.526824in,left,base]{\color{textcolor}\sffamily\fontsize{10.000000}{12.000000}\selectfont 1.0}%
\end{pgfscope}%
\begin{pgfscope}%
\pgfsetbuttcap%
\pgfsetroundjoin%
\definecolor{currentfill}{rgb}{0.000000,0.000000,0.000000}%
\pgfsetfillcolor{currentfill}%
\pgfsetlinewidth{0.803000pt}%
\definecolor{currentstroke}{rgb}{0.000000,0.000000,0.000000}%
\pgfsetstrokecolor{currentstroke}%
\pgfsetdash{}{0pt}%
\pgfsys@defobject{currentmarker}{\pgfqpoint{-0.048611in}{0.000000in}}{\pgfqpoint{0.000000in}{0.000000in}}{%
\pgfpathmoveto{\pgfqpoint{0.000000in}{0.000000in}}%
\pgfpathlineto{\pgfqpoint{-0.048611in}{0.000000in}}%
\pgfusepath{stroke,fill}%
}%
\begin{pgfscope}%
\pgfsys@transformshift{4.391558in}{0.907075in}%
\pgfsys@useobject{currentmarker}{}%
\end{pgfscope}%
\end{pgfscope}%
\begin{pgfscope}%
\definecolor{textcolor}{rgb}{0.000000,0.000000,0.000000}%
\pgfsetstrokecolor{textcolor}%
\pgfsetfillcolor{textcolor}%
\pgftext[x=4.073457in,y=0.854314in,left,base]{\color{textcolor}\sffamily\fontsize{10.000000}{12.000000}\selectfont 1.5}%
\end{pgfscope}%
\begin{pgfscope}%
\pgfsetbuttcap%
\pgfsetroundjoin%
\definecolor{currentfill}{rgb}{0.000000,0.000000,0.000000}%
\pgfsetfillcolor{currentfill}%
\pgfsetlinewidth{0.803000pt}%
\definecolor{currentstroke}{rgb}{0.000000,0.000000,0.000000}%
\pgfsetstrokecolor{currentstroke}%
\pgfsetdash{}{0pt}%
\pgfsys@defobject{currentmarker}{\pgfqpoint{-0.048611in}{0.000000in}}{\pgfqpoint{0.000000in}{0.000000in}}{%
\pgfpathmoveto{\pgfqpoint{0.000000in}{0.000000in}}%
\pgfpathlineto{\pgfqpoint{-0.048611in}{0.000000in}}%
\pgfusepath{stroke,fill}%
}%
\begin{pgfscope}%
\pgfsys@transformshift{4.391558in}{1.234566in}%
\pgfsys@useobject{currentmarker}{}%
\end{pgfscope}%
\end{pgfscope}%
\begin{pgfscope}%
\definecolor{textcolor}{rgb}{0.000000,0.000000,0.000000}%
\pgfsetstrokecolor{textcolor}%
\pgfsetfillcolor{textcolor}%
\pgftext[x=4.073457in,y=1.181804in,left,base]{\color{textcolor}\sffamily\fontsize{10.000000}{12.000000}\selectfont 2.0}%
\end{pgfscope}%
\begin{pgfscope}%
\pgfsetbuttcap%
\pgfsetroundjoin%
\definecolor{currentfill}{rgb}{0.000000,0.000000,0.000000}%
\pgfsetfillcolor{currentfill}%
\pgfsetlinewidth{0.803000pt}%
\definecolor{currentstroke}{rgb}{0.000000,0.000000,0.000000}%
\pgfsetstrokecolor{currentstroke}%
\pgfsetdash{}{0pt}%
\pgfsys@defobject{currentmarker}{\pgfqpoint{-0.048611in}{0.000000in}}{\pgfqpoint{0.000000in}{0.000000in}}{%
\pgfpathmoveto{\pgfqpoint{0.000000in}{0.000000in}}%
\pgfpathlineto{\pgfqpoint{-0.048611in}{0.000000in}}%
\pgfusepath{stroke,fill}%
}%
\begin{pgfscope}%
\pgfsys@transformshift{4.391558in}{1.562056in}%
\pgfsys@useobject{currentmarker}{}%
\end{pgfscope}%
\end{pgfscope}%
\begin{pgfscope}%
\definecolor{textcolor}{rgb}{0.000000,0.000000,0.000000}%
\pgfsetstrokecolor{textcolor}%
\pgfsetfillcolor{textcolor}%
\pgftext[x=4.073457in,y=1.509294in,left,base]{\color{textcolor}\sffamily\fontsize{10.000000}{12.000000}\selectfont 2.5}%
\end{pgfscope}%
\begin{pgfscope}%
\pgfsetbuttcap%
\pgfsetroundjoin%
\definecolor{currentfill}{rgb}{0.000000,0.000000,0.000000}%
\pgfsetfillcolor{currentfill}%
\pgfsetlinewidth{0.803000pt}%
\definecolor{currentstroke}{rgb}{0.000000,0.000000,0.000000}%
\pgfsetstrokecolor{currentstroke}%
\pgfsetdash{}{0pt}%
\pgfsys@defobject{currentmarker}{\pgfqpoint{-0.048611in}{0.000000in}}{\pgfqpoint{0.000000in}{0.000000in}}{%
\pgfpathmoveto{\pgfqpoint{0.000000in}{0.000000in}}%
\pgfpathlineto{\pgfqpoint{-0.048611in}{0.000000in}}%
\pgfusepath{stroke,fill}%
}%
\begin{pgfscope}%
\pgfsys@transformshift{4.391558in}{1.889546in}%
\pgfsys@useobject{currentmarker}{}%
\end{pgfscope}%
\end{pgfscope}%
\begin{pgfscope}%
\definecolor{textcolor}{rgb}{0.000000,0.000000,0.000000}%
\pgfsetstrokecolor{textcolor}%
\pgfsetfillcolor{textcolor}%
\pgftext[x=4.073457in,y=1.836784in,left,base]{\color{textcolor}\sffamily\fontsize{10.000000}{12.000000}\selectfont 3.0}%
\end{pgfscope}%
\begin{pgfscope}%
\pgfsetbuttcap%
\pgfsetroundjoin%
\definecolor{currentfill}{rgb}{0.000000,0.000000,0.000000}%
\pgfsetfillcolor{currentfill}%
\pgfsetlinewidth{0.803000pt}%
\definecolor{currentstroke}{rgb}{0.000000,0.000000,0.000000}%
\pgfsetstrokecolor{currentstroke}%
\pgfsetdash{}{0pt}%
\pgfsys@defobject{currentmarker}{\pgfqpoint{-0.048611in}{0.000000in}}{\pgfqpoint{0.000000in}{0.000000in}}{%
\pgfpathmoveto{\pgfqpoint{0.000000in}{0.000000in}}%
\pgfpathlineto{\pgfqpoint{-0.048611in}{0.000000in}}%
\pgfusepath{stroke,fill}%
}%
\begin{pgfscope}%
\pgfsys@transformshift{4.391558in}{2.217036in}%
\pgfsys@useobject{currentmarker}{}%
\end{pgfscope}%
\end{pgfscope}%
\begin{pgfscope}%
\definecolor{textcolor}{rgb}{0.000000,0.000000,0.000000}%
\pgfsetstrokecolor{textcolor}%
\pgfsetfillcolor{textcolor}%
\pgftext[x=4.073457in,y=2.164274in,left,base]{\color{textcolor}\sffamily\fontsize{10.000000}{12.000000}\selectfont 3.5}%
\end{pgfscope}%
\begin{pgfscope}%
\pgfsetbuttcap%
\pgfsetroundjoin%
\definecolor{currentfill}{rgb}{0.000000,0.000000,0.000000}%
\pgfsetfillcolor{currentfill}%
\pgfsetlinewidth{0.803000pt}%
\definecolor{currentstroke}{rgb}{0.000000,0.000000,0.000000}%
\pgfsetstrokecolor{currentstroke}%
\pgfsetdash{}{0pt}%
\pgfsys@defobject{currentmarker}{\pgfqpoint{-0.048611in}{0.000000in}}{\pgfqpoint{0.000000in}{0.000000in}}{%
\pgfpathmoveto{\pgfqpoint{0.000000in}{0.000000in}}%
\pgfpathlineto{\pgfqpoint{-0.048611in}{0.000000in}}%
\pgfusepath{stroke,fill}%
}%
\begin{pgfscope}%
\pgfsys@transformshift{4.391558in}{2.544526in}%
\pgfsys@useobject{currentmarker}{}%
\end{pgfscope}%
\end{pgfscope}%
\begin{pgfscope}%
\definecolor{textcolor}{rgb}{0.000000,0.000000,0.000000}%
\pgfsetstrokecolor{textcolor}%
\pgfsetfillcolor{textcolor}%
\pgftext[x=4.073457in,y=2.491765in,left,base]{\color{textcolor}\sffamily\fontsize{10.000000}{12.000000}\selectfont 4.0}%
\end{pgfscope}%
\begin{pgfscope}%
\pgfpathrectangle{\pgfqpoint{4.391558in}{0.386111in}}{\pgfqpoint{3.315517in}{2.316667in}}%
\pgfusepath{clip}%
\pgfsetrectcap%
\pgfsetroundjoin%
\pgfsetlinewidth{1.505625pt}%
\definecolor{currentstroke}{rgb}{0.121569,0.466667,0.705882}%
\pgfsetstrokecolor{currentstroke}%
\pgfsetdash{}{0pt}%
\pgfpathmoveto{\pgfqpoint{4.553730in}{0.720843in}}%
\pgfpathlineto{\pgfqpoint{4.703288in}{0.870644in}}%
\pgfpathlineto{\pgfqpoint{4.852847in}{1.664967in}}%
\pgfpathlineto{\pgfqpoint{5.002406in}{0.991339in}}%
\pgfpathlineto{\pgfqpoint{5.151965in}{1.199940in}}%
\pgfpathlineto{\pgfqpoint{5.301523in}{1.544444in}}%
\pgfpathlineto{\pgfqpoint{5.451082in}{1.064011in}}%
\pgfpathlineto{\pgfqpoint{5.600641in}{1.174193in}}%
\pgfpathlineto{\pgfqpoint{5.750199in}{1.548322in}}%
\pgfpathlineto{\pgfqpoint{5.899758in}{1.318727in}}%
\pgfpathlineto{\pgfqpoint{6.049317in}{1.333675in}}%
\pgfpathlineto{\pgfqpoint{6.198875in}{1.306840in}}%
\pgfpathlineto{\pgfqpoint{6.348434in}{1.178137in}}%
\pgfpathlineto{\pgfqpoint{6.497993in}{1.276393in}}%
\pgfpathlineto{\pgfqpoint{6.647551in}{1.047652in}}%
\pgfpathlineto{\pgfqpoint{6.797110in}{0.863640in}}%
\pgfpathlineto{\pgfqpoint{6.946669in}{0.961966in}}%
\pgfpathlineto{\pgfqpoint{7.096227in}{0.837368in}}%
\pgfpathlineto{\pgfqpoint{7.245786in}{0.752530in}}%
\pgfpathlineto{\pgfqpoint{7.395345in}{0.756342in}}%
\pgfpathlineto{\pgfqpoint{7.544904in}{0.732117in}}%
\pgfusepath{stroke}%
\end{pgfscope}%
\begin{pgfscope}%
\pgfsetrectcap%
\pgfsetmiterjoin%
\pgfsetlinewidth{0.803000pt}%
\definecolor{currentstroke}{rgb}{0.000000,0.000000,0.000000}%
\pgfsetstrokecolor{currentstroke}%
\pgfsetdash{}{0pt}%
\pgfpathmoveto{\pgfqpoint{4.391558in}{0.386111in}}%
\pgfpathlineto{\pgfqpoint{4.391558in}{2.702778in}}%
\pgfusepath{stroke}%
\end{pgfscope}%
\begin{pgfscope}%
\pgfsetrectcap%
\pgfsetmiterjoin%
\pgfsetlinewidth{0.803000pt}%
\definecolor{currentstroke}{rgb}{0.000000,0.000000,0.000000}%
\pgfsetstrokecolor{currentstroke}%
\pgfsetdash{}{0pt}%
\pgfpathmoveto{\pgfqpoint{7.707075in}{0.386111in}}%
\pgfpathlineto{\pgfqpoint{7.707075in}{2.702778in}}%
\pgfusepath{stroke}%
\end{pgfscope}%
\begin{pgfscope}%
\pgfsetrectcap%
\pgfsetmiterjoin%
\pgfsetlinewidth{0.803000pt}%
\definecolor{currentstroke}{rgb}{0.000000,0.000000,0.000000}%
\pgfsetstrokecolor{currentstroke}%
\pgfsetdash{}{0pt}%
\pgfpathmoveto{\pgfqpoint{4.391558in}{0.386111in}}%
\pgfpathlineto{\pgfqpoint{7.707075in}{0.386111in}}%
\pgfusepath{stroke}%
\end{pgfscope}%
\begin{pgfscope}%
\pgfsetrectcap%
\pgfsetmiterjoin%
\pgfsetlinewidth{0.803000pt}%
\definecolor{currentstroke}{rgb}{0.000000,0.000000,0.000000}%
\pgfsetstrokecolor{currentstroke}%
\pgfsetdash{}{0pt}%
\pgfpathmoveto{\pgfqpoint{4.391558in}{2.702778in}}%
\pgfpathlineto{\pgfqpoint{7.707075in}{2.702778in}}%
\pgfusepath{stroke}%
\end{pgfscope}%
\begin{pgfscope}%
\definecolor{textcolor}{rgb}{0.000000,0.000000,0.000000}%
\pgfsetstrokecolor{textcolor}%
\pgfsetfillcolor{textcolor}%
\pgftext[x=6.049317in,y=2.786111in,,base]{\color{textcolor}\sffamily\fontsize{12.000000}{14.400000}\selectfont \(\displaystyle  N = 128 \)}%
\end{pgfscope}%
\end{pgfpicture}%
\makeatother%
\endgroup%
}
\caption{Specific heat $c$ for finer temperature $T$ using Metropolis--Hastings algorithm}
\label{Fig:CapBig}
\end{figure}

\begin{figure}[htbp]
\centering
\scalebox{0.666}{%% Creator: Matplotlib, PGF backend
%%
%% To include the figure in your LaTeX document, write
%%   \input{<filename>.pgf}
%%
%% Make sure the required packages are loaded in your preamble
%%   \usepackage{pgf}
%%
%% Figures using additional raster images can only be included by \input if
%% they are in the same directory as the main LaTeX file. For loading figures
%% from other directories you can use the `import` package
%%   \usepackage{import}
%% and then include the figures with
%%   \import{<path to file>}{<filename>.pgf}
%%
%% Matplotlib used the following preamble
%%   \usepackage{fontspec}
%%   \setmainfont{DejaVuSerif.ttf}[Path=/home/lzh/anaconda3/envs/numana/lib/python3.7/site-packages/matplotlib/mpl-data/fonts/ttf/]
%%   \setsansfont{DejaVuSans.ttf}[Path=/home/lzh/anaconda3/envs/numana/lib/python3.7/site-packages/matplotlib/mpl-data/fonts/ttf/]
%%   \setmonofont{DejaVuSansMono.ttf}[Path=/home/lzh/anaconda3/envs/numana/lib/python3.7/site-packages/matplotlib/mpl-data/fonts/ttf/]
%%
\begingroup%
\makeatletter%
\begin{pgfpicture}%
\pgfpathrectangle{\pgfpointorigin}{\pgfqpoint{8.000000in}{6.000000in}}%
\pgfusepath{use as bounding box, clip}%
\begin{pgfscope}%
\pgfsetbuttcap%
\pgfsetmiterjoin%
\definecolor{currentfill}{rgb}{1.000000,1.000000,1.000000}%
\pgfsetfillcolor{currentfill}%
\pgfsetlinewidth{0.000000pt}%
\definecolor{currentstroke}{rgb}{1.000000,1.000000,1.000000}%
\pgfsetstrokecolor{currentstroke}%
\pgfsetdash{}{0pt}%
\pgfpathmoveto{\pgfqpoint{0.000000in}{0.000000in}}%
\pgfpathlineto{\pgfqpoint{8.000000in}{0.000000in}}%
\pgfpathlineto{\pgfqpoint{8.000000in}{6.000000in}}%
\pgfpathlineto{\pgfqpoint{0.000000in}{6.000000in}}%
\pgfpathclose%
\pgfusepath{fill}%
\end{pgfscope}%
\begin{pgfscope}%
\pgfsetbuttcap%
\pgfsetmiterjoin%
\definecolor{currentfill}{rgb}{1.000000,1.000000,1.000000}%
\pgfsetfillcolor{currentfill}%
\pgfsetlinewidth{0.000000pt}%
\definecolor{currentstroke}{rgb}{0.000000,0.000000,0.000000}%
\pgfsetstrokecolor{currentstroke}%
\pgfsetstrokeopacity{0.000000}%
\pgfsetdash{}{0pt}%
\pgfpathmoveto{\pgfqpoint{0.555972in}{3.312222in}}%
\pgfpathlineto{\pgfqpoint{3.879180in}{3.312222in}}%
\pgfpathlineto{\pgfqpoint{3.879180in}{5.636667in}}%
\pgfpathlineto{\pgfqpoint{0.555972in}{5.636667in}}%
\pgfpathclose%
\pgfusepath{fill}%
\end{pgfscope}%
\begin{pgfscope}%
\pgfpathrectangle{\pgfqpoint{0.555972in}{3.312222in}}{\pgfqpoint{3.323208in}{2.324444in}}%
\pgfusepath{clip}%
\pgfsetbuttcap%
\pgfsetroundjoin%
\definecolor{currentfill}{rgb}{0.121569,0.466667,0.705882}%
\pgfsetfillcolor{currentfill}%
\pgfsetfillopacity{0.300000}%
\pgfsetlinewidth{1.003750pt}%
\definecolor{currentstroke}{rgb}{0.121569,0.466667,0.705882}%
\pgfsetstrokecolor{currentstroke}%
\pgfsetstrokeopacity{0.300000}%
\pgfsetdash{}{0pt}%
\pgfpathmoveto{\pgfqpoint{0.715320in}{5.531010in}}%
\pgfpathlineto{\pgfqpoint{0.715320in}{5.501127in}}%
\pgfpathlineto{\pgfqpoint{0.865545in}{5.411559in}}%
\pgfpathlineto{\pgfqpoint{1.015771in}{5.356677in}}%
\pgfpathlineto{\pgfqpoint{1.165997in}{5.260897in}}%
\pgfpathlineto{\pgfqpoint{1.316222in}{5.150764in}}%
\pgfpathlineto{\pgfqpoint{1.466448in}{5.066622in}}%
\pgfpathlineto{\pgfqpoint{1.616674in}{4.979344in}}%
\pgfpathlineto{\pgfqpoint{1.766899in}{4.834241in}}%
\pgfpathlineto{\pgfqpoint{1.917125in}{4.746254in}}%
\pgfpathlineto{\pgfqpoint{2.067351in}{4.642190in}}%
\pgfpathlineto{\pgfqpoint{2.217576in}{4.528598in}}%
\pgfpathlineto{\pgfqpoint{2.367802in}{4.413658in}}%
\pgfpathlineto{\pgfqpoint{2.518028in}{4.294949in}}%
\pgfpathlineto{\pgfqpoint{2.668253in}{4.205454in}}%
\pgfpathlineto{\pgfqpoint{2.818479in}{4.095772in}}%
\pgfpathlineto{\pgfqpoint{2.968705in}{3.961487in}}%
\pgfpathlineto{\pgfqpoint{3.118930in}{3.827975in}}%
\pgfpathlineto{\pgfqpoint{3.269156in}{3.754224in}}%
\pgfpathlineto{\pgfqpoint{3.419382in}{3.588225in}}%
\pgfpathlineto{\pgfqpoint{3.569607in}{3.501889in}}%
\pgfpathlineto{\pgfqpoint{3.719833in}{3.417879in}}%
\pgfpathlineto{\pgfqpoint{3.719833in}{3.451788in}}%
\pgfpathlineto{\pgfqpoint{3.719833in}{3.451788in}}%
\pgfpathlineto{\pgfqpoint{3.569607in}{3.548786in}}%
\pgfpathlineto{\pgfqpoint{3.419382in}{3.654412in}}%
\pgfpathlineto{\pgfqpoint{3.269156in}{3.768937in}}%
\pgfpathlineto{\pgfqpoint{3.118930in}{3.865250in}}%
\pgfpathlineto{\pgfqpoint{2.968705in}{3.983988in}}%
\pgfpathlineto{\pgfqpoint{2.818479in}{4.103472in}}%
\pgfpathlineto{\pgfqpoint{2.668253in}{4.224390in}}%
\pgfpathlineto{\pgfqpoint{2.518028in}{4.353249in}}%
\pgfpathlineto{\pgfqpoint{2.367802in}{4.473569in}}%
\pgfpathlineto{\pgfqpoint{2.217576in}{4.555917in}}%
\pgfpathlineto{\pgfqpoint{2.067351in}{4.684227in}}%
\pgfpathlineto{\pgfqpoint{1.917125in}{4.764230in}}%
\pgfpathlineto{\pgfqpoint{1.766899in}{4.900813in}}%
\pgfpathlineto{\pgfqpoint{1.616674in}{5.014967in}}%
\pgfpathlineto{\pgfqpoint{1.466448in}{5.108778in}}%
\pgfpathlineto{\pgfqpoint{1.316222in}{5.183815in}}%
\pgfpathlineto{\pgfqpoint{1.165997in}{5.298887in}}%
\pgfpathlineto{\pgfqpoint{1.015771in}{5.374574in}}%
\pgfpathlineto{\pgfqpoint{0.865545in}{5.467348in}}%
\pgfpathlineto{\pgfqpoint{0.715320in}{5.531010in}}%
\pgfpathclose%
\pgfusepath{stroke,fill}%
\end{pgfscope}%
\begin{pgfscope}%
\pgfpathrectangle{\pgfqpoint{0.555972in}{3.312222in}}{\pgfqpoint{3.323208in}{2.324444in}}%
\pgfusepath{clip}%
\pgfsetbuttcap%
\pgfsetroundjoin%
\definecolor{currentfill}{rgb}{0.121569,0.466667,0.705882}%
\pgfsetfillcolor{currentfill}%
\pgfsetlinewidth{1.003750pt}%
\definecolor{currentstroke}{rgb}{0.121569,0.466667,0.705882}%
\pgfsetstrokecolor{currentstroke}%
\pgfsetdash{}{0pt}%
\pgfsys@defobject{currentmarker}{\pgfqpoint{-0.009821in}{-0.009821in}}{\pgfqpoint{0.009821in}{0.009821in}}{%
\pgfpathmoveto{\pgfqpoint{0.000000in}{-0.009821in}}%
\pgfpathcurveto{\pgfqpoint{0.002605in}{-0.009821in}}{\pgfqpoint{0.005103in}{-0.008786in}}{\pgfqpoint{0.006944in}{-0.006944in}}%
\pgfpathcurveto{\pgfqpoint{0.008786in}{-0.005103in}}{\pgfqpoint{0.009821in}{-0.002605in}}{\pgfqpoint{0.009821in}{0.000000in}}%
\pgfpathcurveto{\pgfqpoint{0.009821in}{0.002605in}}{\pgfqpoint{0.008786in}{0.005103in}}{\pgfqpoint{0.006944in}{0.006944in}}%
\pgfpathcurveto{\pgfqpoint{0.005103in}{0.008786in}}{\pgfqpoint{0.002605in}{0.009821in}}{\pgfqpoint{0.000000in}{0.009821in}}%
\pgfpathcurveto{\pgfqpoint{-0.002605in}{0.009821in}}{\pgfqpoint{-0.005103in}{0.008786in}}{\pgfqpoint{-0.006944in}{0.006944in}}%
\pgfpathcurveto{\pgfqpoint{-0.008786in}{0.005103in}}{\pgfqpoint{-0.009821in}{0.002605in}}{\pgfqpoint{-0.009821in}{0.000000in}}%
\pgfpathcurveto{\pgfqpoint{-0.009821in}{-0.002605in}}{\pgfqpoint{-0.008786in}{-0.005103in}}{\pgfqpoint{-0.006944in}{-0.006944in}}%
\pgfpathcurveto{\pgfqpoint{-0.005103in}{-0.008786in}}{\pgfqpoint{-0.002605in}{-0.009821in}}{\pgfqpoint{0.000000in}{-0.009821in}}%
\pgfpathclose%
\pgfusepath{stroke,fill}%
}%
\begin{pgfscope}%
\pgfsys@transformshift{0.715320in}{5.516068in}%
\pgfsys@useobject{currentmarker}{}%
\end{pgfscope}%
\begin{pgfscope}%
\pgfsys@transformshift{0.865545in}{5.439453in}%
\pgfsys@useobject{currentmarker}{}%
\end{pgfscope}%
\begin{pgfscope}%
\pgfsys@transformshift{1.015771in}{5.365626in}%
\pgfsys@useobject{currentmarker}{}%
\end{pgfscope}%
\begin{pgfscope}%
\pgfsys@transformshift{1.165997in}{5.279892in}%
\pgfsys@useobject{currentmarker}{}%
\end{pgfscope}%
\begin{pgfscope}%
\pgfsys@transformshift{1.316222in}{5.167290in}%
\pgfsys@useobject{currentmarker}{}%
\end{pgfscope}%
\begin{pgfscope}%
\pgfsys@transformshift{1.466448in}{5.087700in}%
\pgfsys@useobject{currentmarker}{}%
\end{pgfscope}%
\begin{pgfscope}%
\pgfsys@transformshift{1.616674in}{4.997156in}%
\pgfsys@useobject{currentmarker}{}%
\end{pgfscope}%
\begin{pgfscope}%
\pgfsys@transformshift{1.766899in}{4.867527in}%
\pgfsys@useobject{currentmarker}{}%
\end{pgfscope}%
\begin{pgfscope}%
\pgfsys@transformshift{1.917125in}{4.755242in}%
\pgfsys@useobject{currentmarker}{}%
\end{pgfscope}%
\begin{pgfscope}%
\pgfsys@transformshift{2.067351in}{4.663209in}%
\pgfsys@useobject{currentmarker}{}%
\end{pgfscope}%
\begin{pgfscope}%
\pgfsys@transformshift{2.217576in}{4.542258in}%
\pgfsys@useobject{currentmarker}{}%
\end{pgfscope}%
\begin{pgfscope}%
\pgfsys@transformshift{2.367802in}{4.443613in}%
\pgfsys@useobject{currentmarker}{}%
\end{pgfscope}%
\begin{pgfscope}%
\pgfsys@transformshift{2.518028in}{4.324099in}%
\pgfsys@useobject{currentmarker}{}%
\end{pgfscope}%
\begin{pgfscope}%
\pgfsys@transformshift{2.668253in}{4.214922in}%
\pgfsys@useobject{currentmarker}{}%
\end{pgfscope}%
\begin{pgfscope}%
\pgfsys@transformshift{2.818479in}{4.099622in}%
\pgfsys@useobject{currentmarker}{}%
\end{pgfscope}%
\begin{pgfscope}%
\pgfsys@transformshift{2.968705in}{3.972738in}%
\pgfsys@useobject{currentmarker}{}%
\end{pgfscope}%
\begin{pgfscope}%
\pgfsys@transformshift{3.118930in}{3.846612in}%
\pgfsys@useobject{currentmarker}{}%
\end{pgfscope}%
\begin{pgfscope}%
\pgfsys@transformshift{3.269156in}{3.761581in}%
\pgfsys@useobject{currentmarker}{}%
\end{pgfscope}%
\begin{pgfscope}%
\pgfsys@transformshift{3.419382in}{3.621319in}%
\pgfsys@useobject{currentmarker}{}%
\end{pgfscope}%
\begin{pgfscope}%
\pgfsys@transformshift{3.569607in}{3.525338in}%
\pgfsys@useobject{currentmarker}{}%
\end{pgfscope}%
\begin{pgfscope}%
\pgfsys@transformshift{3.719833in}{3.434834in}%
\pgfsys@useobject{currentmarker}{}%
\end{pgfscope}%
\end{pgfscope}%
\begin{pgfscope}%
\pgfsetbuttcap%
\pgfsetroundjoin%
\definecolor{currentfill}{rgb}{0.000000,0.000000,0.000000}%
\pgfsetfillcolor{currentfill}%
\pgfsetlinewidth{0.803000pt}%
\definecolor{currentstroke}{rgb}{0.000000,0.000000,0.000000}%
\pgfsetstrokecolor{currentstroke}%
\pgfsetdash{}{0pt}%
\pgfsys@defobject{currentmarker}{\pgfqpoint{0.000000in}{-0.048611in}}{\pgfqpoint{0.000000in}{0.000000in}}{%
\pgfpathmoveto{\pgfqpoint{0.000000in}{0.000000in}}%
\pgfpathlineto{\pgfqpoint{0.000000in}{-0.048611in}}%
\pgfusepath{stroke,fill}%
}%
\begin{pgfscope}%
\pgfsys@transformshift{1.165997in}{3.312222in}%
\pgfsys@useobject{currentmarker}{}%
\end{pgfscope}%
\end{pgfscope}%
\begin{pgfscope}%
\definecolor{textcolor}{rgb}{0.000000,0.000000,0.000000}%
\pgfsetstrokecolor{textcolor}%
\pgfsetfillcolor{textcolor}%
\pgftext[x=1.165997in,y=3.215000in,,top]{\color{textcolor}\sffamily\fontsize{10.000000}{12.000000}\selectfont 2.25}%
\end{pgfscope}%
\begin{pgfscope}%
\pgfsetbuttcap%
\pgfsetroundjoin%
\definecolor{currentfill}{rgb}{0.000000,0.000000,0.000000}%
\pgfsetfillcolor{currentfill}%
\pgfsetlinewidth{0.803000pt}%
\definecolor{currentstroke}{rgb}{0.000000,0.000000,0.000000}%
\pgfsetstrokecolor{currentstroke}%
\pgfsetdash{}{0pt}%
\pgfsys@defobject{currentmarker}{\pgfqpoint{0.000000in}{-0.048611in}}{\pgfqpoint{0.000000in}{0.000000in}}{%
\pgfpathmoveto{\pgfqpoint{0.000000in}{0.000000in}}%
\pgfpathlineto{\pgfqpoint{0.000000in}{-0.048611in}}%
\pgfusepath{stroke,fill}%
}%
\begin{pgfscope}%
\pgfsys@transformshift{1.917125in}{3.312222in}%
\pgfsys@useobject{currentmarker}{}%
\end{pgfscope}%
\end{pgfscope}%
\begin{pgfscope}%
\definecolor{textcolor}{rgb}{0.000000,0.000000,0.000000}%
\pgfsetstrokecolor{textcolor}%
\pgfsetfillcolor{textcolor}%
\pgftext[x=1.917125in,y=3.215000in,,top]{\color{textcolor}\sffamily\fontsize{10.000000}{12.000000}\selectfont 2.30}%
\end{pgfscope}%
\begin{pgfscope}%
\pgfsetbuttcap%
\pgfsetroundjoin%
\definecolor{currentfill}{rgb}{0.000000,0.000000,0.000000}%
\pgfsetfillcolor{currentfill}%
\pgfsetlinewidth{0.803000pt}%
\definecolor{currentstroke}{rgb}{0.000000,0.000000,0.000000}%
\pgfsetstrokecolor{currentstroke}%
\pgfsetdash{}{0pt}%
\pgfsys@defobject{currentmarker}{\pgfqpoint{0.000000in}{-0.048611in}}{\pgfqpoint{0.000000in}{0.000000in}}{%
\pgfpathmoveto{\pgfqpoint{0.000000in}{0.000000in}}%
\pgfpathlineto{\pgfqpoint{0.000000in}{-0.048611in}}%
\pgfusepath{stroke,fill}%
}%
\begin{pgfscope}%
\pgfsys@transformshift{2.668253in}{3.312222in}%
\pgfsys@useobject{currentmarker}{}%
\end{pgfscope}%
\end{pgfscope}%
\begin{pgfscope}%
\definecolor{textcolor}{rgb}{0.000000,0.000000,0.000000}%
\pgfsetstrokecolor{textcolor}%
\pgfsetfillcolor{textcolor}%
\pgftext[x=2.668253in,y=3.215000in,,top]{\color{textcolor}\sffamily\fontsize{10.000000}{12.000000}\selectfont 2.35}%
\end{pgfscope}%
\begin{pgfscope}%
\pgfsetbuttcap%
\pgfsetroundjoin%
\definecolor{currentfill}{rgb}{0.000000,0.000000,0.000000}%
\pgfsetfillcolor{currentfill}%
\pgfsetlinewidth{0.803000pt}%
\definecolor{currentstroke}{rgb}{0.000000,0.000000,0.000000}%
\pgfsetstrokecolor{currentstroke}%
\pgfsetdash{}{0pt}%
\pgfsys@defobject{currentmarker}{\pgfqpoint{0.000000in}{-0.048611in}}{\pgfqpoint{0.000000in}{0.000000in}}{%
\pgfpathmoveto{\pgfqpoint{0.000000in}{0.000000in}}%
\pgfpathlineto{\pgfqpoint{0.000000in}{-0.048611in}}%
\pgfusepath{stroke,fill}%
}%
\begin{pgfscope}%
\pgfsys@transformshift{3.419382in}{3.312222in}%
\pgfsys@useobject{currentmarker}{}%
\end{pgfscope}%
\end{pgfscope}%
\begin{pgfscope}%
\definecolor{textcolor}{rgb}{0.000000,0.000000,0.000000}%
\pgfsetstrokecolor{textcolor}%
\pgfsetfillcolor{textcolor}%
\pgftext[x=3.419382in,y=3.215000in,,top]{\color{textcolor}\sffamily\fontsize{10.000000}{12.000000}\selectfont 2.40}%
\end{pgfscope}%
\begin{pgfscope}%
\pgfsetbuttcap%
\pgfsetroundjoin%
\definecolor{currentfill}{rgb}{0.000000,0.000000,0.000000}%
\pgfsetfillcolor{currentfill}%
\pgfsetlinewidth{0.803000pt}%
\definecolor{currentstroke}{rgb}{0.000000,0.000000,0.000000}%
\pgfsetstrokecolor{currentstroke}%
\pgfsetdash{}{0pt}%
\pgfsys@defobject{currentmarker}{\pgfqpoint{-0.048611in}{0.000000in}}{\pgfqpoint{0.000000in}{0.000000in}}{%
\pgfpathmoveto{\pgfqpoint{0.000000in}{0.000000in}}%
\pgfpathlineto{\pgfqpoint{-0.048611in}{0.000000in}}%
\pgfusepath{stroke,fill}%
}%
\begin{pgfscope}%
\pgfsys@transformshift{0.555972in}{3.514874in}%
\pgfsys@useobject{currentmarker}{}%
\end{pgfscope}%
\end{pgfscope}%
\begin{pgfscope}%
\definecolor{textcolor}{rgb}{0.000000,0.000000,0.000000}%
\pgfsetstrokecolor{textcolor}%
\pgfsetfillcolor{textcolor}%
\pgftext[x=0.149505in,y=3.462112in,left,base]{\color{textcolor}\sffamily\fontsize{10.000000}{12.000000}\selectfont 0.50}%
\end{pgfscope}%
\begin{pgfscope}%
\pgfsetbuttcap%
\pgfsetroundjoin%
\definecolor{currentfill}{rgb}{0.000000,0.000000,0.000000}%
\pgfsetfillcolor{currentfill}%
\pgfsetlinewidth{0.803000pt}%
\definecolor{currentstroke}{rgb}{0.000000,0.000000,0.000000}%
\pgfsetstrokecolor{currentstroke}%
\pgfsetdash{}{0pt}%
\pgfsys@defobject{currentmarker}{\pgfqpoint{-0.048611in}{0.000000in}}{\pgfqpoint{0.000000in}{0.000000in}}{%
\pgfpathmoveto{\pgfqpoint{0.000000in}{0.000000in}}%
\pgfpathlineto{\pgfqpoint{-0.048611in}{0.000000in}}%
\pgfusepath{stroke,fill}%
}%
\begin{pgfscope}%
\pgfsys@transformshift{0.555972in}{3.883355in}%
\pgfsys@useobject{currentmarker}{}%
\end{pgfscope}%
\end{pgfscope}%
\begin{pgfscope}%
\definecolor{textcolor}{rgb}{0.000000,0.000000,0.000000}%
\pgfsetstrokecolor{textcolor}%
\pgfsetfillcolor{textcolor}%
\pgftext[x=0.149505in,y=3.830594in,left,base]{\color{textcolor}\sffamily\fontsize{10.000000}{12.000000}\selectfont 0.55}%
\end{pgfscope}%
\begin{pgfscope}%
\pgfsetbuttcap%
\pgfsetroundjoin%
\definecolor{currentfill}{rgb}{0.000000,0.000000,0.000000}%
\pgfsetfillcolor{currentfill}%
\pgfsetlinewidth{0.803000pt}%
\definecolor{currentstroke}{rgb}{0.000000,0.000000,0.000000}%
\pgfsetstrokecolor{currentstroke}%
\pgfsetdash{}{0pt}%
\pgfsys@defobject{currentmarker}{\pgfqpoint{-0.048611in}{0.000000in}}{\pgfqpoint{0.000000in}{0.000000in}}{%
\pgfpathmoveto{\pgfqpoint{0.000000in}{0.000000in}}%
\pgfpathlineto{\pgfqpoint{-0.048611in}{0.000000in}}%
\pgfusepath{stroke,fill}%
}%
\begin{pgfscope}%
\pgfsys@transformshift{0.555972in}{4.251837in}%
\pgfsys@useobject{currentmarker}{}%
\end{pgfscope}%
\end{pgfscope}%
\begin{pgfscope}%
\definecolor{textcolor}{rgb}{0.000000,0.000000,0.000000}%
\pgfsetstrokecolor{textcolor}%
\pgfsetfillcolor{textcolor}%
\pgftext[x=0.149505in,y=4.199075in,left,base]{\color{textcolor}\sffamily\fontsize{10.000000}{12.000000}\selectfont 0.60}%
\end{pgfscope}%
\begin{pgfscope}%
\pgfsetbuttcap%
\pgfsetroundjoin%
\definecolor{currentfill}{rgb}{0.000000,0.000000,0.000000}%
\pgfsetfillcolor{currentfill}%
\pgfsetlinewidth{0.803000pt}%
\definecolor{currentstroke}{rgb}{0.000000,0.000000,0.000000}%
\pgfsetstrokecolor{currentstroke}%
\pgfsetdash{}{0pt}%
\pgfsys@defobject{currentmarker}{\pgfqpoint{-0.048611in}{0.000000in}}{\pgfqpoint{0.000000in}{0.000000in}}{%
\pgfpathmoveto{\pgfqpoint{0.000000in}{0.000000in}}%
\pgfpathlineto{\pgfqpoint{-0.048611in}{0.000000in}}%
\pgfusepath{stroke,fill}%
}%
\begin{pgfscope}%
\pgfsys@transformshift{0.555972in}{4.620318in}%
\pgfsys@useobject{currentmarker}{}%
\end{pgfscope}%
\end{pgfscope}%
\begin{pgfscope}%
\definecolor{textcolor}{rgb}{0.000000,0.000000,0.000000}%
\pgfsetstrokecolor{textcolor}%
\pgfsetfillcolor{textcolor}%
\pgftext[x=0.149505in,y=4.567557in,left,base]{\color{textcolor}\sffamily\fontsize{10.000000}{12.000000}\selectfont 0.65}%
\end{pgfscope}%
\begin{pgfscope}%
\pgfsetbuttcap%
\pgfsetroundjoin%
\definecolor{currentfill}{rgb}{0.000000,0.000000,0.000000}%
\pgfsetfillcolor{currentfill}%
\pgfsetlinewidth{0.803000pt}%
\definecolor{currentstroke}{rgb}{0.000000,0.000000,0.000000}%
\pgfsetstrokecolor{currentstroke}%
\pgfsetdash{}{0pt}%
\pgfsys@defobject{currentmarker}{\pgfqpoint{-0.048611in}{0.000000in}}{\pgfqpoint{0.000000in}{0.000000in}}{%
\pgfpathmoveto{\pgfqpoint{0.000000in}{0.000000in}}%
\pgfpathlineto{\pgfqpoint{-0.048611in}{0.000000in}}%
\pgfusepath{stroke,fill}%
}%
\begin{pgfscope}%
\pgfsys@transformshift{0.555972in}{4.988799in}%
\pgfsys@useobject{currentmarker}{}%
\end{pgfscope}%
\end{pgfscope}%
\begin{pgfscope}%
\definecolor{textcolor}{rgb}{0.000000,0.000000,0.000000}%
\pgfsetstrokecolor{textcolor}%
\pgfsetfillcolor{textcolor}%
\pgftext[x=0.149505in,y=4.936038in,left,base]{\color{textcolor}\sffamily\fontsize{10.000000}{12.000000}\selectfont 0.70}%
\end{pgfscope}%
\begin{pgfscope}%
\pgfsetbuttcap%
\pgfsetroundjoin%
\definecolor{currentfill}{rgb}{0.000000,0.000000,0.000000}%
\pgfsetfillcolor{currentfill}%
\pgfsetlinewidth{0.803000pt}%
\definecolor{currentstroke}{rgb}{0.000000,0.000000,0.000000}%
\pgfsetstrokecolor{currentstroke}%
\pgfsetdash{}{0pt}%
\pgfsys@defobject{currentmarker}{\pgfqpoint{-0.048611in}{0.000000in}}{\pgfqpoint{0.000000in}{0.000000in}}{%
\pgfpathmoveto{\pgfqpoint{0.000000in}{0.000000in}}%
\pgfpathlineto{\pgfqpoint{-0.048611in}{0.000000in}}%
\pgfusepath{stroke,fill}%
}%
\begin{pgfscope}%
\pgfsys@transformshift{0.555972in}{5.357281in}%
\pgfsys@useobject{currentmarker}{}%
\end{pgfscope}%
\end{pgfscope}%
\begin{pgfscope}%
\definecolor{textcolor}{rgb}{0.000000,0.000000,0.000000}%
\pgfsetstrokecolor{textcolor}%
\pgfsetfillcolor{textcolor}%
\pgftext[x=0.149505in,y=5.304519in,left,base]{\color{textcolor}\sffamily\fontsize{10.000000}{12.000000}\selectfont 0.75}%
\end{pgfscope}%
\begin{pgfscope}%
\pgfpathrectangle{\pgfqpoint{0.555972in}{3.312222in}}{\pgfqpoint{3.323208in}{2.324444in}}%
\pgfusepath{clip}%
\pgfsetrectcap%
\pgfsetroundjoin%
\pgfsetlinewidth{1.505625pt}%
\definecolor{currentstroke}{rgb}{0.121569,0.466667,0.705882}%
\pgfsetstrokecolor{currentstroke}%
\pgfsetdash{}{0pt}%
\pgfpathmoveto{\pgfqpoint{0.715320in}{5.516068in}}%
\pgfpathlineto{\pgfqpoint{0.865545in}{5.439453in}}%
\pgfpathlineto{\pgfqpoint{1.015771in}{5.365626in}}%
\pgfpathlineto{\pgfqpoint{1.165997in}{5.279892in}}%
\pgfpathlineto{\pgfqpoint{1.316222in}{5.167290in}}%
\pgfpathlineto{\pgfqpoint{1.466448in}{5.087700in}}%
\pgfpathlineto{\pgfqpoint{1.616674in}{4.997156in}}%
\pgfpathlineto{\pgfqpoint{1.766899in}{4.867527in}}%
\pgfpathlineto{\pgfqpoint{1.917125in}{4.755242in}}%
\pgfpathlineto{\pgfqpoint{2.067351in}{4.663209in}}%
\pgfpathlineto{\pgfqpoint{2.217576in}{4.542258in}}%
\pgfpathlineto{\pgfqpoint{2.367802in}{4.443613in}}%
\pgfpathlineto{\pgfqpoint{2.518028in}{4.324099in}}%
\pgfpathlineto{\pgfqpoint{2.668253in}{4.214922in}}%
\pgfpathlineto{\pgfqpoint{2.818479in}{4.099622in}}%
\pgfpathlineto{\pgfqpoint{2.968705in}{3.972738in}}%
\pgfpathlineto{\pgfqpoint{3.118930in}{3.846612in}}%
\pgfpathlineto{\pgfqpoint{3.269156in}{3.761581in}}%
\pgfpathlineto{\pgfqpoint{3.419382in}{3.621319in}}%
\pgfpathlineto{\pgfqpoint{3.569607in}{3.525338in}}%
\pgfpathlineto{\pgfqpoint{3.719833in}{3.434834in}}%
\pgfusepath{stroke}%
\end{pgfscope}%
\begin{pgfscope}%
\pgfsetrectcap%
\pgfsetmiterjoin%
\pgfsetlinewidth{0.803000pt}%
\definecolor{currentstroke}{rgb}{0.000000,0.000000,0.000000}%
\pgfsetstrokecolor{currentstroke}%
\pgfsetdash{}{0pt}%
\pgfpathmoveto{\pgfqpoint{0.555972in}{3.312222in}}%
\pgfpathlineto{\pgfqpoint{0.555972in}{5.636667in}}%
\pgfusepath{stroke}%
\end{pgfscope}%
\begin{pgfscope}%
\pgfsetrectcap%
\pgfsetmiterjoin%
\pgfsetlinewidth{0.803000pt}%
\definecolor{currentstroke}{rgb}{0.000000,0.000000,0.000000}%
\pgfsetstrokecolor{currentstroke}%
\pgfsetdash{}{0pt}%
\pgfpathmoveto{\pgfqpoint{3.879180in}{3.312222in}}%
\pgfpathlineto{\pgfqpoint{3.879180in}{5.636667in}}%
\pgfusepath{stroke}%
\end{pgfscope}%
\begin{pgfscope}%
\pgfsetrectcap%
\pgfsetmiterjoin%
\pgfsetlinewidth{0.803000pt}%
\definecolor{currentstroke}{rgb}{0.000000,0.000000,0.000000}%
\pgfsetstrokecolor{currentstroke}%
\pgfsetdash{}{0pt}%
\pgfpathmoveto{\pgfqpoint{0.555972in}{3.312222in}}%
\pgfpathlineto{\pgfqpoint{3.879180in}{3.312222in}}%
\pgfusepath{stroke}%
\end{pgfscope}%
\begin{pgfscope}%
\pgfsetrectcap%
\pgfsetmiterjoin%
\pgfsetlinewidth{0.803000pt}%
\definecolor{currentstroke}{rgb}{0.000000,0.000000,0.000000}%
\pgfsetstrokecolor{currentstroke}%
\pgfsetdash{}{0pt}%
\pgfpathmoveto{\pgfqpoint{0.555972in}{5.636667in}}%
\pgfpathlineto{\pgfqpoint{3.879180in}{5.636667in}}%
\pgfusepath{stroke}%
\end{pgfscope}%
\begin{pgfscope}%
\definecolor{textcolor}{rgb}{0.000000,0.000000,0.000000}%
\pgfsetstrokecolor{textcolor}%
\pgfsetfillcolor{textcolor}%
\pgftext[x=2.217576in,y=5.720000in,,base]{\color{textcolor}\sffamily\fontsize{12.000000}{14.400000}\selectfont \(\displaystyle  N = 16 \)}%
\end{pgfscope}%
\begin{pgfscope}%
\pgfsetbuttcap%
\pgfsetmiterjoin%
\definecolor{currentfill}{rgb}{1.000000,1.000000,1.000000}%
\pgfsetfillcolor{currentfill}%
\pgfsetlinewidth{0.000000pt}%
\definecolor{currentstroke}{rgb}{0.000000,0.000000,0.000000}%
\pgfsetstrokecolor{currentstroke}%
\pgfsetstrokeopacity{0.000000}%
\pgfsetdash{}{0pt}%
\pgfpathmoveto{\pgfqpoint{4.381402in}{3.312222in}}%
\pgfpathlineto{\pgfqpoint{7.704610in}{3.312222in}}%
\pgfpathlineto{\pgfqpoint{7.704610in}{5.636667in}}%
\pgfpathlineto{\pgfqpoint{4.381402in}{5.636667in}}%
\pgfpathclose%
\pgfusepath{fill}%
\end{pgfscope}%
\begin{pgfscope}%
\pgfpathrectangle{\pgfqpoint{4.381402in}{3.312222in}}{\pgfqpoint{3.323208in}{2.324444in}}%
\pgfusepath{clip}%
\pgfsetbuttcap%
\pgfsetroundjoin%
\definecolor{currentfill}{rgb}{0.121569,0.466667,0.705882}%
\pgfsetfillcolor{currentfill}%
\pgfsetfillopacity{0.300000}%
\pgfsetlinewidth{1.003750pt}%
\definecolor{currentstroke}{rgb}{0.121569,0.466667,0.705882}%
\pgfsetstrokecolor{currentstroke}%
\pgfsetstrokeopacity{0.300000}%
\pgfsetdash{}{0pt}%
\pgfpathmoveto{\pgfqpoint{4.540750in}{5.531010in}}%
\pgfpathlineto{\pgfqpoint{4.540750in}{5.498855in}}%
\pgfpathlineto{\pgfqpoint{4.690975in}{5.453107in}}%
\pgfpathlineto{\pgfqpoint{4.841201in}{5.397106in}}%
\pgfpathlineto{\pgfqpoint{4.991427in}{5.319805in}}%
\pgfpathlineto{\pgfqpoint{5.141652in}{5.256006in}}%
\pgfpathlineto{\pgfqpoint{5.291878in}{5.092500in}}%
\pgfpathlineto{\pgfqpoint{5.442104in}{5.093300in}}%
\pgfpathlineto{\pgfqpoint{5.592329in}{4.948498in}}%
\pgfpathlineto{\pgfqpoint{5.742555in}{4.814282in}}%
\pgfpathlineto{\pgfqpoint{5.892781in}{4.758519in}}%
\pgfpathlineto{\pgfqpoint{6.043006in}{4.564389in}}%
\pgfpathlineto{\pgfqpoint{6.193232in}{4.446842in}}%
\pgfpathlineto{\pgfqpoint{6.343458in}{4.236135in}}%
\pgfpathlineto{\pgfqpoint{6.493683in}{4.154953in}}%
\pgfpathlineto{\pgfqpoint{6.643909in}{4.043509in}}%
\pgfpathlineto{\pgfqpoint{6.794135in}{3.939077in}}%
\pgfpathlineto{\pgfqpoint{6.944360in}{3.859548in}}%
\pgfpathlineto{\pgfqpoint{7.094586in}{3.737370in}}%
\pgfpathlineto{\pgfqpoint{7.244812in}{3.573925in}}%
\pgfpathlineto{\pgfqpoint{7.395037in}{3.539556in}}%
\pgfpathlineto{\pgfqpoint{7.545263in}{3.417879in}}%
\pgfpathlineto{\pgfqpoint{7.545263in}{3.517583in}}%
\pgfpathlineto{\pgfqpoint{7.545263in}{3.517583in}}%
\pgfpathlineto{\pgfqpoint{7.395037in}{3.546929in}}%
\pgfpathlineto{\pgfqpoint{7.244812in}{3.648557in}}%
\pgfpathlineto{\pgfqpoint{7.094586in}{3.822486in}}%
\pgfpathlineto{\pgfqpoint{6.944360in}{3.899129in}}%
\pgfpathlineto{\pgfqpoint{6.794135in}{3.992502in}}%
\pgfpathlineto{\pgfqpoint{6.643909in}{4.124119in}}%
\pgfpathlineto{\pgfqpoint{6.493683in}{4.250005in}}%
\pgfpathlineto{\pgfqpoint{6.343458in}{4.403987in}}%
\pgfpathlineto{\pgfqpoint{6.193232in}{4.493075in}}%
\pgfpathlineto{\pgfqpoint{6.043006in}{4.670887in}}%
\pgfpathlineto{\pgfqpoint{5.892781in}{4.835560in}}%
\pgfpathlineto{\pgfqpoint{5.742555in}{4.939599in}}%
\pgfpathlineto{\pgfqpoint{5.592329in}{4.974793in}}%
\pgfpathlineto{\pgfqpoint{5.442104in}{5.111849in}}%
\pgfpathlineto{\pgfqpoint{5.291878in}{5.195928in}}%
\pgfpathlineto{\pgfqpoint{5.141652in}{5.303419in}}%
\pgfpathlineto{\pgfqpoint{4.991427in}{5.359899in}}%
\pgfpathlineto{\pgfqpoint{4.841201in}{5.431835in}}%
\pgfpathlineto{\pgfqpoint{4.690975in}{5.477230in}}%
\pgfpathlineto{\pgfqpoint{4.540750in}{5.531010in}}%
\pgfpathclose%
\pgfusepath{stroke,fill}%
\end{pgfscope}%
\begin{pgfscope}%
\pgfpathrectangle{\pgfqpoint{4.381402in}{3.312222in}}{\pgfqpoint{3.323208in}{2.324444in}}%
\pgfusepath{clip}%
\pgfsetbuttcap%
\pgfsetroundjoin%
\definecolor{currentfill}{rgb}{0.121569,0.466667,0.705882}%
\pgfsetfillcolor{currentfill}%
\pgfsetlinewidth{1.003750pt}%
\definecolor{currentstroke}{rgb}{0.121569,0.466667,0.705882}%
\pgfsetstrokecolor{currentstroke}%
\pgfsetdash{}{0pt}%
\pgfsys@defobject{currentmarker}{\pgfqpoint{-0.009821in}{-0.009821in}}{\pgfqpoint{0.009821in}{0.009821in}}{%
\pgfpathmoveto{\pgfqpoint{0.000000in}{-0.009821in}}%
\pgfpathcurveto{\pgfqpoint{0.002605in}{-0.009821in}}{\pgfqpoint{0.005103in}{-0.008786in}}{\pgfqpoint{0.006944in}{-0.006944in}}%
\pgfpathcurveto{\pgfqpoint{0.008786in}{-0.005103in}}{\pgfqpoint{0.009821in}{-0.002605in}}{\pgfqpoint{0.009821in}{0.000000in}}%
\pgfpathcurveto{\pgfqpoint{0.009821in}{0.002605in}}{\pgfqpoint{0.008786in}{0.005103in}}{\pgfqpoint{0.006944in}{0.006944in}}%
\pgfpathcurveto{\pgfqpoint{0.005103in}{0.008786in}}{\pgfqpoint{0.002605in}{0.009821in}}{\pgfqpoint{0.000000in}{0.009821in}}%
\pgfpathcurveto{\pgfqpoint{-0.002605in}{0.009821in}}{\pgfqpoint{-0.005103in}{0.008786in}}{\pgfqpoint{-0.006944in}{0.006944in}}%
\pgfpathcurveto{\pgfqpoint{-0.008786in}{0.005103in}}{\pgfqpoint{-0.009821in}{0.002605in}}{\pgfqpoint{-0.009821in}{0.000000in}}%
\pgfpathcurveto{\pgfqpoint{-0.009821in}{-0.002605in}}{\pgfqpoint{-0.008786in}{-0.005103in}}{\pgfqpoint{-0.006944in}{-0.006944in}}%
\pgfpathcurveto{\pgfqpoint{-0.005103in}{-0.008786in}}{\pgfqpoint{-0.002605in}{-0.009821in}}{\pgfqpoint{0.000000in}{-0.009821in}}%
\pgfpathclose%
\pgfusepath{stroke,fill}%
}%
\begin{pgfscope}%
\pgfsys@transformshift{4.540750in}{5.514933in}%
\pgfsys@useobject{currentmarker}{}%
\end{pgfscope}%
\begin{pgfscope}%
\pgfsys@transformshift{4.690975in}{5.465169in}%
\pgfsys@useobject{currentmarker}{}%
\end{pgfscope}%
\begin{pgfscope}%
\pgfsys@transformshift{4.841201in}{5.414470in}%
\pgfsys@useobject{currentmarker}{}%
\end{pgfscope}%
\begin{pgfscope}%
\pgfsys@transformshift{4.991427in}{5.339852in}%
\pgfsys@useobject{currentmarker}{}%
\end{pgfscope}%
\begin{pgfscope}%
\pgfsys@transformshift{5.141652in}{5.279712in}%
\pgfsys@useobject{currentmarker}{}%
\end{pgfscope}%
\begin{pgfscope}%
\pgfsys@transformshift{5.291878in}{5.144214in}%
\pgfsys@useobject{currentmarker}{}%
\end{pgfscope}%
\begin{pgfscope}%
\pgfsys@transformshift{5.442104in}{5.102574in}%
\pgfsys@useobject{currentmarker}{}%
\end{pgfscope}%
\begin{pgfscope}%
\pgfsys@transformshift{5.592329in}{4.961646in}%
\pgfsys@useobject{currentmarker}{}%
\end{pgfscope}%
\begin{pgfscope}%
\pgfsys@transformshift{5.742555in}{4.876941in}%
\pgfsys@useobject{currentmarker}{}%
\end{pgfscope}%
\begin{pgfscope}%
\pgfsys@transformshift{5.892781in}{4.797039in}%
\pgfsys@useobject{currentmarker}{}%
\end{pgfscope}%
\begin{pgfscope}%
\pgfsys@transformshift{6.043006in}{4.617638in}%
\pgfsys@useobject{currentmarker}{}%
\end{pgfscope}%
\begin{pgfscope}%
\pgfsys@transformshift{6.193232in}{4.469958in}%
\pgfsys@useobject{currentmarker}{}%
\end{pgfscope}%
\begin{pgfscope}%
\pgfsys@transformshift{6.343458in}{4.320061in}%
\pgfsys@useobject{currentmarker}{}%
\end{pgfscope}%
\begin{pgfscope}%
\pgfsys@transformshift{6.493683in}{4.202479in}%
\pgfsys@useobject{currentmarker}{}%
\end{pgfscope}%
\begin{pgfscope}%
\pgfsys@transformshift{6.643909in}{4.083814in}%
\pgfsys@useobject{currentmarker}{}%
\end{pgfscope}%
\begin{pgfscope}%
\pgfsys@transformshift{6.794135in}{3.965789in}%
\pgfsys@useobject{currentmarker}{}%
\end{pgfscope}%
\begin{pgfscope}%
\pgfsys@transformshift{6.944360in}{3.879338in}%
\pgfsys@useobject{currentmarker}{}%
\end{pgfscope}%
\begin{pgfscope}%
\pgfsys@transformshift{7.094586in}{3.779928in}%
\pgfsys@useobject{currentmarker}{}%
\end{pgfscope}%
\begin{pgfscope}%
\pgfsys@transformshift{7.244812in}{3.611241in}%
\pgfsys@useobject{currentmarker}{}%
\end{pgfscope}%
\begin{pgfscope}%
\pgfsys@transformshift{7.395037in}{3.543243in}%
\pgfsys@useobject{currentmarker}{}%
\end{pgfscope}%
\begin{pgfscope}%
\pgfsys@transformshift{7.545263in}{3.467731in}%
\pgfsys@useobject{currentmarker}{}%
\end{pgfscope}%
\end{pgfscope}%
\begin{pgfscope}%
\pgfsetbuttcap%
\pgfsetroundjoin%
\definecolor{currentfill}{rgb}{0.000000,0.000000,0.000000}%
\pgfsetfillcolor{currentfill}%
\pgfsetlinewidth{0.803000pt}%
\definecolor{currentstroke}{rgb}{0.000000,0.000000,0.000000}%
\pgfsetstrokecolor{currentstroke}%
\pgfsetdash{}{0pt}%
\pgfsys@defobject{currentmarker}{\pgfqpoint{0.000000in}{-0.048611in}}{\pgfqpoint{0.000000in}{0.000000in}}{%
\pgfpathmoveto{\pgfqpoint{0.000000in}{0.000000in}}%
\pgfpathlineto{\pgfqpoint{0.000000in}{-0.048611in}}%
\pgfusepath{stroke,fill}%
}%
\begin{pgfscope}%
\pgfsys@transformshift{4.690975in}{3.312222in}%
\pgfsys@useobject{currentmarker}{}%
\end{pgfscope}%
\end{pgfscope}%
\begin{pgfscope}%
\definecolor{textcolor}{rgb}{0.000000,0.000000,0.000000}%
\pgfsetstrokecolor{textcolor}%
\pgfsetfillcolor{textcolor}%
\pgftext[x=4.690975in,y=3.215000in,,top]{\color{textcolor}\sffamily\fontsize{10.000000}{12.000000}\selectfont 2.20}%
\end{pgfscope}%
\begin{pgfscope}%
\pgfsetbuttcap%
\pgfsetroundjoin%
\definecolor{currentfill}{rgb}{0.000000,0.000000,0.000000}%
\pgfsetfillcolor{currentfill}%
\pgfsetlinewidth{0.803000pt}%
\definecolor{currentstroke}{rgb}{0.000000,0.000000,0.000000}%
\pgfsetstrokecolor{currentstroke}%
\pgfsetdash{}{0pt}%
\pgfsys@defobject{currentmarker}{\pgfqpoint{0.000000in}{-0.048611in}}{\pgfqpoint{0.000000in}{0.000000in}}{%
\pgfpathmoveto{\pgfqpoint{0.000000in}{0.000000in}}%
\pgfpathlineto{\pgfqpoint{0.000000in}{-0.048611in}}%
\pgfusepath{stroke,fill}%
}%
\begin{pgfscope}%
\pgfsys@transformshift{5.442104in}{3.312222in}%
\pgfsys@useobject{currentmarker}{}%
\end{pgfscope}%
\end{pgfscope}%
\begin{pgfscope}%
\definecolor{textcolor}{rgb}{0.000000,0.000000,0.000000}%
\pgfsetstrokecolor{textcolor}%
\pgfsetfillcolor{textcolor}%
\pgftext[x=5.442104in,y=3.215000in,,top]{\color{textcolor}\sffamily\fontsize{10.000000}{12.000000}\selectfont 2.25}%
\end{pgfscope}%
\begin{pgfscope}%
\pgfsetbuttcap%
\pgfsetroundjoin%
\definecolor{currentfill}{rgb}{0.000000,0.000000,0.000000}%
\pgfsetfillcolor{currentfill}%
\pgfsetlinewidth{0.803000pt}%
\definecolor{currentstroke}{rgb}{0.000000,0.000000,0.000000}%
\pgfsetstrokecolor{currentstroke}%
\pgfsetdash{}{0pt}%
\pgfsys@defobject{currentmarker}{\pgfqpoint{0.000000in}{-0.048611in}}{\pgfqpoint{0.000000in}{0.000000in}}{%
\pgfpathmoveto{\pgfqpoint{0.000000in}{0.000000in}}%
\pgfpathlineto{\pgfqpoint{0.000000in}{-0.048611in}}%
\pgfusepath{stroke,fill}%
}%
\begin{pgfscope}%
\pgfsys@transformshift{6.193232in}{3.312222in}%
\pgfsys@useobject{currentmarker}{}%
\end{pgfscope}%
\end{pgfscope}%
\begin{pgfscope}%
\definecolor{textcolor}{rgb}{0.000000,0.000000,0.000000}%
\pgfsetstrokecolor{textcolor}%
\pgfsetfillcolor{textcolor}%
\pgftext[x=6.193232in,y=3.215000in,,top]{\color{textcolor}\sffamily\fontsize{10.000000}{12.000000}\selectfont 2.30}%
\end{pgfscope}%
\begin{pgfscope}%
\pgfsetbuttcap%
\pgfsetroundjoin%
\definecolor{currentfill}{rgb}{0.000000,0.000000,0.000000}%
\pgfsetfillcolor{currentfill}%
\pgfsetlinewidth{0.803000pt}%
\definecolor{currentstroke}{rgb}{0.000000,0.000000,0.000000}%
\pgfsetstrokecolor{currentstroke}%
\pgfsetdash{}{0pt}%
\pgfsys@defobject{currentmarker}{\pgfqpoint{0.000000in}{-0.048611in}}{\pgfqpoint{0.000000in}{0.000000in}}{%
\pgfpathmoveto{\pgfqpoint{0.000000in}{0.000000in}}%
\pgfpathlineto{\pgfqpoint{0.000000in}{-0.048611in}}%
\pgfusepath{stroke,fill}%
}%
\begin{pgfscope}%
\pgfsys@transformshift{6.944360in}{3.312222in}%
\pgfsys@useobject{currentmarker}{}%
\end{pgfscope}%
\end{pgfscope}%
\begin{pgfscope}%
\definecolor{textcolor}{rgb}{0.000000,0.000000,0.000000}%
\pgfsetstrokecolor{textcolor}%
\pgfsetfillcolor{textcolor}%
\pgftext[x=6.944360in,y=3.215000in,,top]{\color{textcolor}\sffamily\fontsize{10.000000}{12.000000}\selectfont 2.35}%
\end{pgfscope}%
\begin{pgfscope}%
\pgfsetbuttcap%
\pgfsetroundjoin%
\definecolor{currentfill}{rgb}{0.000000,0.000000,0.000000}%
\pgfsetfillcolor{currentfill}%
\pgfsetlinewidth{0.803000pt}%
\definecolor{currentstroke}{rgb}{0.000000,0.000000,0.000000}%
\pgfsetstrokecolor{currentstroke}%
\pgfsetdash{}{0pt}%
\pgfsys@defobject{currentmarker}{\pgfqpoint{0.000000in}{-0.048611in}}{\pgfqpoint{0.000000in}{0.000000in}}{%
\pgfpathmoveto{\pgfqpoint{0.000000in}{0.000000in}}%
\pgfpathlineto{\pgfqpoint{0.000000in}{-0.048611in}}%
\pgfusepath{stroke,fill}%
}%
\begin{pgfscope}%
\pgfsys@transformshift{7.695489in}{3.312222in}%
\pgfsys@useobject{currentmarker}{}%
\end{pgfscope}%
\end{pgfscope}%
\begin{pgfscope}%
\definecolor{textcolor}{rgb}{0.000000,0.000000,0.000000}%
\pgfsetstrokecolor{textcolor}%
\pgfsetfillcolor{textcolor}%
\pgftext[x=7.695489in,y=3.215000in,,top]{\color{textcolor}\sffamily\fontsize{10.000000}{12.000000}\selectfont 2.40}%
\end{pgfscope}%
\begin{pgfscope}%
\pgfsetbuttcap%
\pgfsetroundjoin%
\definecolor{currentfill}{rgb}{0.000000,0.000000,0.000000}%
\pgfsetfillcolor{currentfill}%
\pgfsetlinewidth{0.803000pt}%
\definecolor{currentstroke}{rgb}{0.000000,0.000000,0.000000}%
\pgfsetstrokecolor{currentstroke}%
\pgfsetdash{}{0pt}%
\pgfsys@defobject{currentmarker}{\pgfqpoint{-0.048611in}{0.000000in}}{\pgfqpoint{0.000000in}{0.000000in}}{%
\pgfpathmoveto{\pgfqpoint{0.000000in}{0.000000in}}%
\pgfpathlineto{\pgfqpoint{-0.048611in}{0.000000in}}%
\pgfusepath{stroke,fill}%
}%
\begin{pgfscope}%
\pgfsys@transformshift{4.381402in}{3.326651in}%
\pgfsys@useobject{currentmarker}{}%
\end{pgfscope}%
\end{pgfscope}%
\begin{pgfscope}%
\definecolor{textcolor}{rgb}{0.000000,0.000000,0.000000}%
\pgfsetstrokecolor{textcolor}%
\pgfsetfillcolor{textcolor}%
\pgftext[x=4.063301in,y=3.273889in,left,base]{\color{textcolor}\sffamily\fontsize{10.000000}{12.000000}\selectfont 0.3}%
\end{pgfscope}%
\begin{pgfscope}%
\pgfsetbuttcap%
\pgfsetroundjoin%
\definecolor{currentfill}{rgb}{0.000000,0.000000,0.000000}%
\pgfsetfillcolor{currentfill}%
\pgfsetlinewidth{0.803000pt}%
\definecolor{currentstroke}{rgb}{0.000000,0.000000,0.000000}%
\pgfsetstrokecolor{currentstroke}%
\pgfsetdash{}{0pt}%
\pgfsys@defobject{currentmarker}{\pgfqpoint{-0.048611in}{0.000000in}}{\pgfqpoint{0.000000in}{0.000000in}}{%
\pgfpathmoveto{\pgfqpoint{0.000000in}{0.000000in}}%
\pgfpathlineto{\pgfqpoint{-0.048611in}{0.000000in}}%
\pgfusepath{stroke,fill}%
}%
\begin{pgfscope}%
\pgfsys@transformshift{4.381402in}{3.766059in}%
\pgfsys@useobject{currentmarker}{}%
\end{pgfscope}%
\end{pgfscope}%
\begin{pgfscope}%
\definecolor{textcolor}{rgb}{0.000000,0.000000,0.000000}%
\pgfsetstrokecolor{textcolor}%
\pgfsetfillcolor{textcolor}%
\pgftext[x=4.063301in,y=3.713298in,left,base]{\color{textcolor}\sffamily\fontsize{10.000000}{12.000000}\selectfont 0.4}%
\end{pgfscope}%
\begin{pgfscope}%
\pgfsetbuttcap%
\pgfsetroundjoin%
\definecolor{currentfill}{rgb}{0.000000,0.000000,0.000000}%
\pgfsetfillcolor{currentfill}%
\pgfsetlinewidth{0.803000pt}%
\definecolor{currentstroke}{rgb}{0.000000,0.000000,0.000000}%
\pgfsetstrokecolor{currentstroke}%
\pgfsetdash{}{0pt}%
\pgfsys@defobject{currentmarker}{\pgfqpoint{-0.048611in}{0.000000in}}{\pgfqpoint{0.000000in}{0.000000in}}{%
\pgfpathmoveto{\pgfqpoint{0.000000in}{0.000000in}}%
\pgfpathlineto{\pgfqpoint{-0.048611in}{0.000000in}}%
\pgfusepath{stroke,fill}%
}%
\begin{pgfscope}%
\pgfsys@transformshift{4.381402in}{4.205468in}%
\pgfsys@useobject{currentmarker}{}%
\end{pgfscope}%
\end{pgfscope}%
\begin{pgfscope}%
\definecolor{textcolor}{rgb}{0.000000,0.000000,0.000000}%
\pgfsetstrokecolor{textcolor}%
\pgfsetfillcolor{textcolor}%
\pgftext[x=4.063301in,y=4.152706in,left,base]{\color{textcolor}\sffamily\fontsize{10.000000}{12.000000}\selectfont 0.5}%
\end{pgfscope}%
\begin{pgfscope}%
\pgfsetbuttcap%
\pgfsetroundjoin%
\definecolor{currentfill}{rgb}{0.000000,0.000000,0.000000}%
\pgfsetfillcolor{currentfill}%
\pgfsetlinewidth{0.803000pt}%
\definecolor{currentstroke}{rgb}{0.000000,0.000000,0.000000}%
\pgfsetstrokecolor{currentstroke}%
\pgfsetdash{}{0pt}%
\pgfsys@defobject{currentmarker}{\pgfqpoint{-0.048611in}{0.000000in}}{\pgfqpoint{0.000000in}{0.000000in}}{%
\pgfpathmoveto{\pgfqpoint{0.000000in}{0.000000in}}%
\pgfpathlineto{\pgfqpoint{-0.048611in}{0.000000in}}%
\pgfusepath{stroke,fill}%
}%
\begin{pgfscope}%
\pgfsys@transformshift{4.381402in}{4.644876in}%
\pgfsys@useobject{currentmarker}{}%
\end{pgfscope}%
\end{pgfscope}%
\begin{pgfscope}%
\definecolor{textcolor}{rgb}{0.000000,0.000000,0.000000}%
\pgfsetstrokecolor{textcolor}%
\pgfsetfillcolor{textcolor}%
\pgftext[x=4.063301in,y=4.592115in,left,base]{\color{textcolor}\sffamily\fontsize{10.000000}{12.000000}\selectfont 0.6}%
\end{pgfscope}%
\begin{pgfscope}%
\pgfsetbuttcap%
\pgfsetroundjoin%
\definecolor{currentfill}{rgb}{0.000000,0.000000,0.000000}%
\pgfsetfillcolor{currentfill}%
\pgfsetlinewidth{0.803000pt}%
\definecolor{currentstroke}{rgb}{0.000000,0.000000,0.000000}%
\pgfsetstrokecolor{currentstroke}%
\pgfsetdash{}{0pt}%
\pgfsys@defobject{currentmarker}{\pgfqpoint{-0.048611in}{0.000000in}}{\pgfqpoint{0.000000in}{0.000000in}}{%
\pgfpathmoveto{\pgfqpoint{0.000000in}{0.000000in}}%
\pgfpathlineto{\pgfqpoint{-0.048611in}{0.000000in}}%
\pgfusepath{stroke,fill}%
}%
\begin{pgfscope}%
\pgfsys@transformshift{4.381402in}{5.084285in}%
\pgfsys@useobject{currentmarker}{}%
\end{pgfscope}%
\end{pgfscope}%
\begin{pgfscope}%
\definecolor{textcolor}{rgb}{0.000000,0.000000,0.000000}%
\pgfsetstrokecolor{textcolor}%
\pgfsetfillcolor{textcolor}%
\pgftext[x=4.063301in,y=5.031523in,left,base]{\color{textcolor}\sffamily\fontsize{10.000000}{12.000000}\selectfont 0.7}%
\end{pgfscope}%
\begin{pgfscope}%
\pgfsetbuttcap%
\pgfsetroundjoin%
\definecolor{currentfill}{rgb}{0.000000,0.000000,0.000000}%
\pgfsetfillcolor{currentfill}%
\pgfsetlinewidth{0.803000pt}%
\definecolor{currentstroke}{rgb}{0.000000,0.000000,0.000000}%
\pgfsetstrokecolor{currentstroke}%
\pgfsetdash{}{0pt}%
\pgfsys@defobject{currentmarker}{\pgfqpoint{-0.048611in}{0.000000in}}{\pgfqpoint{0.000000in}{0.000000in}}{%
\pgfpathmoveto{\pgfqpoint{0.000000in}{0.000000in}}%
\pgfpathlineto{\pgfqpoint{-0.048611in}{0.000000in}}%
\pgfusepath{stroke,fill}%
}%
\begin{pgfscope}%
\pgfsys@transformshift{4.381402in}{5.523693in}%
\pgfsys@useobject{currentmarker}{}%
\end{pgfscope}%
\end{pgfscope}%
\begin{pgfscope}%
\definecolor{textcolor}{rgb}{0.000000,0.000000,0.000000}%
\pgfsetstrokecolor{textcolor}%
\pgfsetfillcolor{textcolor}%
\pgftext[x=4.063301in,y=5.470931in,left,base]{\color{textcolor}\sffamily\fontsize{10.000000}{12.000000}\selectfont 0.8}%
\end{pgfscope}%
\begin{pgfscope}%
\pgfpathrectangle{\pgfqpoint{4.381402in}{3.312222in}}{\pgfqpoint{3.323208in}{2.324444in}}%
\pgfusepath{clip}%
\pgfsetrectcap%
\pgfsetroundjoin%
\pgfsetlinewidth{1.505625pt}%
\definecolor{currentstroke}{rgb}{0.121569,0.466667,0.705882}%
\pgfsetstrokecolor{currentstroke}%
\pgfsetdash{}{0pt}%
\pgfpathmoveto{\pgfqpoint{4.540750in}{5.514933in}}%
\pgfpathlineto{\pgfqpoint{4.690975in}{5.465169in}}%
\pgfpathlineto{\pgfqpoint{4.841201in}{5.414470in}}%
\pgfpathlineto{\pgfqpoint{4.991427in}{5.339852in}}%
\pgfpathlineto{\pgfqpoint{5.141652in}{5.279712in}}%
\pgfpathlineto{\pgfqpoint{5.291878in}{5.144214in}}%
\pgfpathlineto{\pgfqpoint{5.442104in}{5.102574in}}%
\pgfpathlineto{\pgfqpoint{5.592329in}{4.961646in}}%
\pgfpathlineto{\pgfqpoint{5.742555in}{4.876941in}}%
\pgfpathlineto{\pgfqpoint{5.892781in}{4.797039in}}%
\pgfpathlineto{\pgfqpoint{6.043006in}{4.617638in}}%
\pgfpathlineto{\pgfqpoint{6.193232in}{4.469958in}}%
\pgfpathlineto{\pgfqpoint{6.343458in}{4.320061in}}%
\pgfpathlineto{\pgfqpoint{6.493683in}{4.202479in}}%
\pgfpathlineto{\pgfqpoint{6.643909in}{4.083814in}}%
\pgfpathlineto{\pgfqpoint{6.794135in}{3.965789in}}%
\pgfpathlineto{\pgfqpoint{6.944360in}{3.879338in}}%
\pgfpathlineto{\pgfqpoint{7.094586in}{3.779928in}}%
\pgfpathlineto{\pgfqpoint{7.244812in}{3.611241in}}%
\pgfpathlineto{\pgfqpoint{7.395037in}{3.543243in}}%
\pgfpathlineto{\pgfqpoint{7.545263in}{3.467731in}}%
\pgfusepath{stroke}%
\end{pgfscope}%
\begin{pgfscope}%
\pgfsetrectcap%
\pgfsetmiterjoin%
\pgfsetlinewidth{0.803000pt}%
\definecolor{currentstroke}{rgb}{0.000000,0.000000,0.000000}%
\pgfsetstrokecolor{currentstroke}%
\pgfsetdash{}{0pt}%
\pgfpathmoveto{\pgfqpoint{4.381402in}{3.312222in}}%
\pgfpathlineto{\pgfqpoint{4.381402in}{5.636667in}}%
\pgfusepath{stroke}%
\end{pgfscope}%
\begin{pgfscope}%
\pgfsetrectcap%
\pgfsetmiterjoin%
\pgfsetlinewidth{0.803000pt}%
\definecolor{currentstroke}{rgb}{0.000000,0.000000,0.000000}%
\pgfsetstrokecolor{currentstroke}%
\pgfsetdash{}{0pt}%
\pgfpathmoveto{\pgfqpoint{7.704610in}{3.312222in}}%
\pgfpathlineto{\pgfqpoint{7.704610in}{5.636667in}}%
\pgfusepath{stroke}%
\end{pgfscope}%
\begin{pgfscope}%
\pgfsetrectcap%
\pgfsetmiterjoin%
\pgfsetlinewidth{0.803000pt}%
\definecolor{currentstroke}{rgb}{0.000000,0.000000,0.000000}%
\pgfsetstrokecolor{currentstroke}%
\pgfsetdash{}{0pt}%
\pgfpathmoveto{\pgfqpoint{4.381402in}{3.312222in}}%
\pgfpathlineto{\pgfqpoint{7.704610in}{3.312222in}}%
\pgfusepath{stroke}%
\end{pgfscope}%
\begin{pgfscope}%
\pgfsetrectcap%
\pgfsetmiterjoin%
\pgfsetlinewidth{0.803000pt}%
\definecolor{currentstroke}{rgb}{0.000000,0.000000,0.000000}%
\pgfsetstrokecolor{currentstroke}%
\pgfsetdash{}{0pt}%
\pgfpathmoveto{\pgfqpoint{4.381402in}{5.636667in}}%
\pgfpathlineto{\pgfqpoint{7.704610in}{5.636667in}}%
\pgfusepath{stroke}%
\end{pgfscope}%
\begin{pgfscope}%
\definecolor{textcolor}{rgb}{0.000000,0.000000,0.000000}%
\pgfsetstrokecolor{textcolor}%
\pgfsetfillcolor{textcolor}%
\pgftext[x=6.043006in,y=5.720000in,,base]{\color{textcolor}\sffamily\fontsize{12.000000}{14.400000}\selectfont \(\displaystyle  N = 32 \)}%
\end{pgfscope}%
\begin{pgfscope}%
\pgfsetbuttcap%
\pgfsetmiterjoin%
\definecolor{currentfill}{rgb}{1.000000,1.000000,1.000000}%
\pgfsetfillcolor{currentfill}%
\pgfsetlinewidth{0.000000pt}%
\definecolor{currentstroke}{rgb}{0.000000,0.000000,0.000000}%
\pgfsetstrokecolor{currentstroke}%
\pgfsetstrokeopacity{0.000000}%
\pgfsetdash{}{0pt}%
\pgfpathmoveto{\pgfqpoint{0.555972in}{0.387222in}}%
\pgfpathlineto{\pgfqpoint{3.879180in}{0.387222in}}%
\pgfpathlineto{\pgfqpoint{3.879180in}{2.711667in}}%
\pgfpathlineto{\pgfqpoint{0.555972in}{2.711667in}}%
\pgfpathclose%
\pgfusepath{fill}%
\end{pgfscope}%
\begin{pgfscope}%
\pgfpathrectangle{\pgfqpoint{0.555972in}{0.387222in}}{\pgfqpoint{3.323208in}{2.324444in}}%
\pgfusepath{clip}%
\pgfsetbuttcap%
\pgfsetroundjoin%
\definecolor{currentfill}{rgb}{0.121569,0.466667,0.705882}%
\pgfsetfillcolor{currentfill}%
\pgfsetfillopacity{0.300000}%
\pgfsetlinewidth{1.003750pt}%
\definecolor{currentstroke}{rgb}{0.121569,0.466667,0.705882}%
\pgfsetstrokecolor{currentstroke}%
\pgfsetstrokeopacity{0.300000}%
\pgfsetdash{}{0pt}%
\pgfpathmoveto{\pgfqpoint{0.715320in}{2.601744in}}%
\pgfpathlineto{\pgfqpoint{0.715320in}{2.594655in}}%
\pgfpathlineto{\pgfqpoint{0.865545in}{2.555886in}}%
\pgfpathlineto{\pgfqpoint{1.015771in}{2.475485in}}%
\pgfpathlineto{\pgfqpoint{1.165997in}{2.476144in}}%
\pgfpathlineto{\pgfqpoint{1.316222in}{2.289226in}}%
\pgfpathlineto{\pgfqpoint{1.466448in}{2.321771in}}%
\pgfpathlineto{\pgfqpoint{1.616674in}{2.243444in}}%
\pgfpathlineto{\pgfqpoint{1.766899in}{2.154419in}}%
\pgfpathlineto{\pgfqpoint{1.917125in}{1.741441in}}%
\pgfpathlineto{\pgfqpoint{2.067351in}{1.575034in}}%
\pgfpathlineto{\pgfqpoint{2.217576in}{1.544858in}}%
\pgfpathlineto{\pgfqpoint{2.367802in}{1.484460in}}%
\pgfpathlineto{\pgfqpoint{2.518028in}{1.288210in}}%
\pgfpathlineto{\pgfqpoint{2.668253in}{1.004279in}}%
\pgfpathlineto{\pgfqpoint{2.818479in}{0.704043in}}%
\pgfpathlineto{\pgfqpoint{2.968705in}{0.798908in}}%
\pgfpathlineto{\pgfqpoint{3.118930in}{0.721832in}}%
\pgfpathlineto{\pgfqpoint{3.269156in}{0.616730in}}%
\pgfpathlineto{\pgfqpoint{3.419382in}{0.617453in}}%
\pgfpathlineto{\pgfqpoint{3.569607in}{0.511230in}}%
\pgfpathlineto{\pgfqpoint{3.719833in}{0.492879in}}%
\pgfpathlineto{\pgfqpoint{3.719833in}{0.578784in}}%
\pgfpathlineto{\pgfqpoint{3.719833in}{0.578784in}}%
\pgfpathlineto{\pgfqpoint{3.569607in}{0.677021in}}%
\pgfpathlineto{\pgfqpoint{3.419382in}{0.735327in}}%
\pgfpathlineto{\pgfqpoint{3.269156in}{0.740593in}}%
\pgfpathlineto{\pgfqpoint{3.118930in}{0.901507in}}%
\pgfpathlineto{\pgfqpoint{2.968705in}{1.026054in}}%
\pgfpathlineto{\pgfqpoint{2.818479in}{1.191263in}}%
\pgfpathlineto{\pgfqpoint{2.668253in}{1.465996in}}%
\pgfpathlineto{\pgfqpoint{2.518028in}{1.556658in}}%
\pgfpathlineto{\pgfqpoint{2.367802in}{1.634726in}}%
\pgfpathlineto{\pgfqpoint{2.217576in}{1.944683in}}%
\pgfpathlineto{\pgfqpoint{2.067351in}{2.097098in}}%
\pgfpathlineto{\pgfqpoint{1.917125in}{2.134783in}}%
\pgfpathlineto{\pgfqpoint{1.766899in}{2.206132in}}%
\pgfpathlineto{\pgfqpoint{1.616674in}{2.322755in}}%
\pgfpathlineto{\pgfqpoint{1.466448in}{2.403407in}}%
\pgfpathlineto{\pgfqpoint{1.316222in}{2.443053in}}%
\pgfpathlineto{\pgfqpoint{1.165997in}{2.509160in}}%
\pgfpathlineto{\pgfqpoint{1.015771in}{2.558906in}}%
\pgfpathlineto{\pgfqpoint{0.865545in}{2.575626in}}%
\pgfpathlineto{\pgfqpoint{0.715320in}{2.601744in}}%
\pgfpathclose%
\pgfusepath{stroke,fill}%
\end{pgfscope}%
\begin{pgfscope}%
\pgfpathrectangle{\pgfqpoint{0.555972in}{0.387222in}}{\pgfqpoint{3.323208in}{2.324444in}}%
\pgfusepath{clip}%
\pgfsetbuttcap%
\pgfsetroundjoin%
\definecolor{currentfill}{rgb}{0.121569,0.466667,0.705882}%
\pgfsetfillcolor{currentfill}%
\pgfsetlinewidth{1.003750pt}%
\definecolor{currentstroke}{rgb}{0.121569,0.466667,0.705882}%
\pgfsetstrokecolor{currentstroke}%
\pgfsetdash{}{0pt}%
\pgfsys@defobject{currentmarker}{\pgfqpoint{-0.009821in}{-0.009821in}}{\pgfqpoint{0.009821in}{0.009821in}}{%
\pgfpathmoveto{\pgfqpoint{0.000000in}{-0.009821in}}%
\pgfpathcurveto{\pgfqpoint{0.002605in}{-0.009821in}}{\pgfqpoint{0.005103in}{-0.008786in}}{\pgfqpoint{0.006944in}{-0.006944in}}%
\pgfpathcurveto{\pgfqpoint{0.008786in}{-0.005103in}}{\pgfqpoint{0.009821in}{-0.002605in}}{\pgfqpoint{0.009821in}{0.000000in}}%
\pgfpathcurveto{\pgfqpoint{0.009821in}{0.002605in}}{\pgfqpoint{0.008786in}{0.005103in}}{\pgfqpoint{0.006944in}{0.006944in}}%
\pgfpathcurveto{\pgfqpoint{0.005103in}{0.008786in}}{\pgfqpoint{0.002605in}{0.009821in}}{\pgfqpoint{0.000000in}{0.009821in}}%
\pgfpathcurveto{\pgfqpoint{-0.002605in}{0.009821in}}{\pgfqpoint{-0.005103in}{0.008786in}}{\pgfqpoint{-0.006944in}{0.006944in}}%
\pgfpathcurveto{\pgfqpoint{-0.008786in}{0.005103in}}{\pgfqpoint{-0.009821in}{0.002605in}}{\pgfqpoint{-0.009821in}{0.000000in}}%
\pgfpathcurveto{\pgfqpoint{-0.009821in}{-0.002605in}}{\pgfqpoint{-0.008786in}{-0.005103in}}{\pgfqpoint{-0.006944in}{-0.006944in}}%
\pgfpathcurveto{\pgfqpoint{-0.005103in}{-0.008786in}}{\pgfqpoint{-0.002605in}{-0.009821in}}{\pgfqpoint{0.000000in}{-0.009821in}}%
\pgfpathclose%
\pgfusepath{stroke,fill}%
}%
\begin{pgfscope}%
\pgfsys@transformshift{0.715320in}{2.598199in}%
\pgfsys@useobject{currentmarker}{}%
\end{pgfscope}%
\begin{pgfscope}%
\pgfsys@transformshift{0.865545in}{2.565756in}%
\pgfsys@useobject{currentmarker}{}%
\end{pgfscope}%
\begin{pgfscope}%
\pgfsys@transformshift{1.015771in}{2.517195in}%
\pgfsys@useobject{currentmarker}{}%
\end{pgfscope}%
\begin{pgfscope}%
\pgfsys@transformshift{1.165997in}{2.492652in}%
\pgfsys@useobject{currentmarker}{}%
\end{pgfscope}%
\begin{pgfscope}%
\pgfsys@transformshift{1.316222in}{2.366139in}%
\pgfsys@useobject{currentmarker}{}%
\end{pgfscope}%
\begin{pgfscope}%
\pgfsys@transformshift{1.466448in}{2.362589in}%
\pgfsys@useobject{currentmarker}{}%
\end{pgfscope}%
\begin{pgfscope}%
\pgfsys@transformshift{1.616674in}{2.283099in}%
\pgfsys@useobject{currentmarker}{}%
\end{pgfscope}%
\begin{pgfscope}%
\pgfsys@transformshift{1.766899in}{2.180275in}%
\pgfsys@useobject{currentmarker}{}%
\end{pgfscope}%
\begin{pgfscope}%
\pgfsys@transformshift{1.917125in}{1.938112in}%
\pgfsys@useobject{currentmarker}{}%
\end{pgfscope}%
\begin{pgfscope}%
\pgfsys@transformshift{2.067351in}{1.836066in}%
\pgfsys@useobject{currentmarker}{}%
\end{pgfscope}%
\begin{pgfscope}%
\pgfsys@transformshift{2.217576in}{1.744770in}%
\pgfsys@useobject{currentmarker}{}%
\end{pgfscope}%
\begin{pgfscope}%
\pgfsys@transformshift{2.367802in}{1.559593in}%
\pgfsys@useobject{currentmarker}{}%
\end{pgfscope}%
\begin{pgfscope}%
\pgfsys@transformshift{2.518028in}{1.422434in}%
\pgfsys@useobject{currentmarker}{}%
\end{pgfscope}%
\begin{pgfscope}%
\pgfsys@transformshift{2.668253in}{1.235138in}%
\pgfsys@useobject{currentmarker}{}%
\end{pgfscope}%
\begin{pgfscope}%
\pgfsys@transformshift{2.818479in}{0.947653in}%
\pgfsys@useobject{currentmarker}{}%
\end{pgfscope}%
\begin{pgfscope}%
\pgfsys@transformshift{2.968705in}{0.912481in}%
\pgfsys@useobject{currentmarker}{}%
\end{pgfscope}%
\begin{pgfscope}%
\pgfsys@transformshift{3.118930in}{0.811670in}%
\pgfsys@useobject{currentmarker}{}%
\end{pgfscope}%
\begin{pgfscope}%
\pgfsys@transformshift{3.269156in}{0.678662in}%
\pgfsys@useobject{currentmarker}{}%
\end{pgfscope}%
\begin{pgfscope}%
\pgfsys@transformshift{3.419382in}{0.676390in}%
\pgfsys@useobject{currentmarker}{}%
\end{pgfscope}%
\begin{pgfscope}%
\pgfsys@transformshift{3.569607in}{0.594126in}%
\pgfsys@useobject{currentmarker}{}%
\end{pgfscope}%
\begin{pgfscope}%
\pgfsys@transformshift{3.719833in}{0.535831in}%
\pgfsys@useobject{currentmarker}{}%
\end{pgfscope}%
\end{pgfscope}%
\begin{pgfscope}%
\pgfsetbuttcap%
\pgfsetroundjoin%
\definecolor{currentfill}{rgb}{0.000000,0.000000,0.000000}%
\pgfsetfillcolor{currentfill}%
\pgfsetlinewidth{0.803000pt}%
\definecolor{currentstroke}{rgb}{0.000000,0.000000,0.000000}%
\pgfsetstrokecolor{currentstroke}%
\pgfsetdash{}{0pt}%
\pgfsys@defobject{currentmarker}{\pgfqpoint{0.000000in}{-0.048611in}}{\pgfqpoint{0.000000in}{0.000000in}}{%
\pgfpathmoveto{\pgfqpoint{0.000000in}{0.000000in}}%
\pgfpathlineto{\pgfqpoint{0.000000in}{-0.048611in}}%
\pgfusepath{stroke,fill}%
}%
\begin{pgfscope}%
\pgfsys@transformshift{1.015771in}{0.387222in}%
\pgfsys@useobject{currentmarker}{}%
\end{pgfscope}%
\end{pgfscope}%
\begin{pgfscope}%
\definecolor{textcolor}{rgb}{0.000000,0.000000,0.000000}%
\pgfsetstrokecolor{textcolor}%
\pgfsetfillcolor{textcolor}%
\pgftext[x=1.015771in,y=0.290000in,,top]{\color{textcolor}\sffamily\fontsize{10.000000}{12.000000}\selectfont 2.20}%
\end{pgfscope}%
\begin{pgfscope}%
\pgfsetbuttcap%
\pgfsetroundjoin%
\definecolor{currentfill}{rgb}{0.000000,0.000000,0.000000}%
\pgfsetfillcolor{currentfill}%
\pgfsetlinewidth{0.803000pt}%
\definecolor{currentstroke}{rgb}{0.000000,0.000000,0.000000}%
\pgfsetstrokecolor{currentstroke}%
\pgfsetdash{}{0pt}%
\pgfsys@defobject{currentmarker}{\pgfqpoint{0.000000in}{-0.048611in}}{\pgfqpoint{0.000000in}{0.000000in}}{%
\pgfpathmoveto{\pgfqpoint{0.000000in}{0.000000in}}%
\pgfpathlineto{\pgfqpoint{0.000000in}{-0.048611in}}%
\pgfusepath{stroke,fill}%
}%
\begin{pgfscope}%
\pgfsys@transformshift{1.766899in}{0.387222in}%
\pgfsys@useobject{currentmarker}{}%
\end{pgfscope}%
\end{pgfscope}%
\begin{pgfscope}%
\definecolor{textcolor}{rgb}{0.000000,0.000000,0.000000}%
\pgfsetstrokecolor{textcolor}%
\pgfsetfillcolor{textcolor}%
\pgftext[x=1.766899in,y=0.290000in,,top]{\color{textcolor}\sffamily\fontsize{10.000000}{12.000000}\selectfont 2.25}%
\end{pgfscope}%
\begin{pgfscope}%
\pgfsetbuttcap%
\pgfsetroundjoin%
\definecolor{currentfill}{rgb}{0.000000,0.000000,0.000000}%
\pgfsetfillcolor{currentfill}%
\pgfsetlinewidth{0.803000pt}%
\definecolor{currentstroke}{rgb}{0.000000,0.000000,0.000000}%
\pgfsetstrokecolor{currentstroke}%
\pgfsetdash{}{0pt}%
\pgfsys@defobject{currentmarker}{\pgfqpoint{0.000000in}{-0.048611in}}{\pgfqpoint{0.000000in}{0.000000in}}{%
\pgfpathmoveto{\pgfqpoint{0.000000in}{0.000000in}}%
\pgfpathlineto{\pgfqpoint{0.000000in}{-0.048611in}}%
\pgfusepath{stroke,fill}%
}%
\begin{pgfscope}%
\pgfsys@transformshift{2.518028in}{0.387222in}%
\pgfsys@useobject{currentmarker}{}%
\end{pgfscope}%
\end{pgfscope}%
\begin{pgfscope}%
\definecolor{textcolor}{rgb}{0.000000,0.000000,0.000000}%
\pgfsetstrokecolor{textcolor}%
\pgfsetfillcolor{textcolor}%
\pgftext[x=2.518028in,y=0.290000in,,top]{\color{textcolor}\sffamily\fontsize{10.000000}{12.000000}\selectfont 2.30}%
\end{pgfscope}%
\begin{pgfscope}%
\pgfsetbuttcap%
\pgfsetroundjoin%
\definecolor{currentfill}{rgb}{0.000000,0.000000,0.000000}%
\pgfsetfillcolor{currentfill}%
\pgfsetlinewidth{0.803000pt}%
\definecolor{currentstroke}{rgb}{0.000000,0.000000,0.000000}%
\pgfsetstrokecolor{currentstroke}%
\pgfsetdash{}{0pt}%
\pgfsys@defobject{currentmarker}{\pgfqpoint{0.000000in}{-0.048611in}}{\pgfqpoint{0.000000in}{0.000000in}}{%
\pgfpathmoveto{\pgfqpoint{0.000000in}{0.000000in}}%
\pgfpathlineto{\pgfqpoint{0.000000in}{-0.048611in}}%
\pgfusepath{stroke,fill}%
}%
\begin{pgfscope}%
\pgfsys@transformshift{3.269156in}{0.387222in}%
\pgfsys@useobject{currentmarker}{}%
\end{pgfscope}%
\end{pgfscope}%
\begin{pgfscope}%
\definecolor{textcolor}{rgb}{0.000000,0.000000,0.000000}%
\pgfsetstrokecolor{textcolor}%
\pgfsetfillcolor{textcolor}%
\pgftext[x=3.269156in,y=0.290000in,,top]{\color{textcolor}\sffamily\fontsize{10.000000}{12.000000}\selectfont 2.35}%
\end{pgfscope}%
\begin{pgfscope}%
\pgfsetbuttcap%
\pgfsetroundjoin%
\definecolor{currentfill}{rgb}{0.000000,0.000000,0.000000}%
\pgfsetfillcolor{currentfill}%
\pgfsetlinewidth{0.803000pt}%
\definecolor{currentstroke}{rgb}{0.000000,0.000000,0.000000}%
\pgfsetstrokecolor{currentstroke}%
\pgfsetdash{}{0pt}%
\pgfsys@defobject{currentmarker}{\pgfqpoint{-0.048611in}{0.000000in}}{\pgfqpoint{0.000000in}{0.000000in}}{%
\pgfpathmoveto{\pgfqpoint{0.000000in}{0.000000in}}%
\pgfpathlineto{\pgfqpoint{-0.048611in}{0.000000in}}%
\pgfusepath{stroke,fill}%
}%
\begin{pgfscope}%
\pgfsys@transformshift{0.555972in}{0.576899in}%
\pgfsys@useobject{currentmarker}{}%
\end{pgfscope}%
\end{pgfscope}%
\begin{pgfscope}%
\definecolor{textcolor}{rgb}{0.000000,0.000000,0.000000}%
\pgfsetstrokecolor{textcolor}%
\pgfsetfillcolor{textcolor}%
\pgftext[x=0.237871in,y=0.524138in,left,base]{\color{textcolor}\sffamily\fontsize{10.000000}{12.000000}\selectfont 0.2}%
\end{pgfscope}%
\begin{pgfscope}%
\pgfsetbuttcap%
\pgfsetroundjoin%
\definecolor{currentfill}{rgb}{0.000000,0.000000,0.000000}%
\pgfsetfillcolor{currentfill}%
\pgfsetlinewidth{0.803000pt}%
\definecolor{currentstroke}{rgb}{0.000000,0.000000,0.000000}%
\pgfsetstrokecolor{currentstroke}%
\pgfsetdash{}{0pt}%
\pgfsys@defobject{currentmarker}{\pgfqpoint{-0.048611in}{0.000000in}}{\pgfqpoint{0.000000in}{0.000000in}}{%
\pgfpathmoveto{\pgfqpoint{0.000000in}{0.000000in}}%
\pgfpathlineto{\pgfqpoint{-0.048611in}{0.000000in}}%
\pgfusepath{stroke,fill}%
}%
\begin{pgfscope}%
\pgfsys@transformshift{0.555972in}{0.910469in}%
\pgfsys@useobject{currentmarker}{}%
\end{pgfscope}%
\end{pgfscope}%
\begin{pgfscope}%
\definecolor{textcolor}{rgb}{0.000000,0.000000,0.000000}%
\pgfsetstrokecolor{textcolor}%
\pgfsetfillcolor{textcolor}%
\pgftext[x=0.237871in,y=0.857707in,left,base]{\color{textcolor}\sffamily\fontsize{10.000000}{12.000000}\selectfont 0.3}%
\end{pgfscope}%
\begin{pgfscope}%
\pgfsetbuttcap%
\pgfsetroundjoin%
\definecolor{currentfill}{rgb}{0.000000,0.000000,0.000000}%
\pgfsetfillcolor{currentfill}%
\pgfsetlinewidth{0.803000pt}%
\definecolor{currentstroke}{rgb}{0.000000,0.000000,0.000000}%
\pgfsetstrokecolor{currentstroke}%
\pgfsetdash{}{0pt}%
\pgfsys@defobject{currentmarker}{\pgfqpoint{-0.048611in}{0.000000in}}{\pgfqpoint{0.000000in}{0.000000in}}{%
\pgfpathmoveto{\pgfqpoint{0.000000in}{0.000000in}}%
\pgfpathlineto{\pgfqpoint{-0.048611in}{0.000000in}}%
\pgfusepath{stroke,fill}%
}%
\begin{pgfscope}%
\pgfsys@transformshift{0.555972in}{1.244039in}%
\pgfsys@useobject{currentmarker}{}%
\end{pgfscope}%
\end{pgfscope}%
\begin{pgfscope}%
\definecolor{textcolor}{rgb}{0.000000,0.000000,0.000000}%
\pgfsetstrokecolor{textcolor}%
\pgfsetfillcolor{textcolor}%
\pgftext[x=0.237871in,y=1.191277in,left,base]{\color{textcolor}\sffamily\fontsize{10.000000}{12.000000}\selectfont 0.4}%
\end{pgfscope}%
\begin{pgfscope}%
\pgfsetbuttcap%
\pgfsetroundjoin%
\definecolor{currentfill}{rgb}{0.000000,0.000000,0.000000}%
\pgfsetfillcolor{currentfill}%
\pgfsetlinewidth{0.803000pt}%
\definecolor{currentstroke}{rgb}{0.000000,0.000000,0.000000}%
\pgfsetstrokecolor{currentstroke}%
\pgfsetdash{}{0pt}%
\pgfsys@defobject{currentmarker}{\pgfqpoint{-0.048611in}{0.000000in}}{\pgfqpoint{0.000000in}{0.000000in}}{%
\pgfpathmoveto{\pgfqpoint{0.000000in}{0.000000in}}%
\pgfpathlineto{\pgfqpoint{-0.048611in}{0.000000in}}%
\pgfusepath{stroke,fill}%
}%
\begin{pgfscope}%
\pgfsys@transformshift{0.555972in}{1.577609in}%
\pgfsys@useobject{currentmarker}{}%
\end{pgfscope}%
\end{pgfscope}%
\begin{pgfscope}%
\definecolor{textcolor}{rgb}{0.000000,0.000000,0.000000}%
\pgfsetstrokecolor{textcolor}%
\pgfsetfillcolor{textcolor}%
\pgftext[x=0.237871in,y=1.524847in,left,base]{\color{textcolor}\sffamily\fontsize{10.000000}{12.000000}\selectfont 0.5}%
\end{pgfscope}%
\begin{pgfscope}%
\pgfsetbuttcap%
\pgfsetroundjoin%
\definecolor{currentfill}{rgb}{0.000000,0.000000,0.000000}%
\pgfsetfillcolor{currentfill}%
\pgfsetlinewidth{0.803000pt}%
\definecolor{currentstroke}{rgb}{0.000000,0.000000,0.000000}%
\pgfsetstrokecolor{currentstroke}%
\pgfsetdash{}{0pt}%
\pgfsys@defobject{currentmarker}{\pgfqpoint{-0.048611in}{0.000000in}}{\pgfqpoint{0.000000in}{0.000000in}}{%
\pgfpathmoveto{\pgfqpoint{0.000000in}{0.000000in}}%
\pgfpathlineto{\pgfqpoint{-0.048611in}{0.000000in}}%
\pgfusepath{stroke,fill}%
}%
\begin{pgfscope}%
\pgfsys@transformshift{0.555972in}{1.911179in}%
\pgfsys@useobject{currentmarker}{}%
\end{pgfscope}%
\end{pgfscope}%
\begin{pgfscope}%
\definecolor{textcolor}{rgb}{0.000000,0.000000,0.000000}%
\pgfsetstrokecolor{textcolor}%
\pgfsetfillcolor{textcolor}%
\pgftext[x=0.237871in,y=1.858417in,left,base]{\color{textcolor}\sffamily\fontsize{10.000000}{12.000000}\selectfont 0.6}%
\end{pgfscope}%
\begin{pgfscope}%
\pgfsetbuttcap%
\pgfsetroundjoin%
\definecolor{currentfill}{rgb}{0.000000,0.000000,0.000000}%
\pgfsetfillcolor{currentfill}%
\pgfsetlinewidth{0.803000pt}%
\definecolor{currentstroke}{rgb}{0.000000,0.000000,0.000000}%
\pgfsetstrokecolor{currentstroke}%
\pgfsetdash{}{0pt}%
\pgfsys@defobject{currentmarker}{\pgfqpoint{-0.048611in}{0.000000in}}{\pgfqpoint{0.000000in}{0.000000in}}{%
\pgfpathmoveto{\pgfqpoint{0.000000in}{0.000000in}}%
\pgfpathlineto{\pgfqpoint{-0.048611in}{0.000000in}}%
\pgfusepath{stroke,fill}%
}%
\begin{pgfscope}%
\pgfsys@transformshift{0.555972in}{2.244749in}%
\pgfsys@useobject{currentmarker}{}%
\end{pgfscope}%
\end{pgfscope}%
\begin{pgfscope}%
\definecolor{textcolor}{rgb}{0.000000,0.000000,0.000000}%
\pgfsetstrokecolor{textcolor}%
\pgfsetfillcolor{textcolor}%
\pgftext[x=0.237871in,y=2.191987in,left,base]{\color{textcolor}\sffamily\fontsize{10.000000}{12.000000}\selectfont 0.7}%
\end{pgfscope}%
\begin{pgfscope}%
\pgfsetbuttcap%
\pgfsetroundjoin%
\definecolor{currentfill}{rgb}{0.000000,0.000000,0.000000}%
\pgfsetfillcolor{currentfill}%
\pgfsetlinewidth{0.803000pt}%
\definecolor{currentstroke}{rgb}{0.000000,0.000000,0.000000}%
\pgfsetstrokecolor{currentstroke}%
\pgfsetdash{}{0pt}%
\pgfsys@defobject{currentmarker}{\pgfqpoint{-0.048611in}{0.000000in}}{\pgfqpoint{0.000000in}{0.000000in}}{%
\pgfpathmoveto{\pgfqpoint{0.000000in}{0.000000in}}%
\pgfpathlineto{\pgfqpoint{-0.048611in}{0.000000in}}%
\pgfusepath{stroke,fill}%
}%
\begin{pgfscope}%
\pgfsys@transformshift{0.555972in}{2.578318in}%
\pgfsys@useobject{currentmarker}{}%
\end{pgfscope}%
\end{pgfscope}%
\begin{pgfscope}%
\definecolor{textcolor}{rgb}{0.000000,0.000000,0.000000}%
\pgfsetstrokecolor{textcolor}%
\pgfsetfillcolor{textcolor}%
\pgftext[x=0.237871in,y=2.525557in,left,base]{\color{textcolor}\sffamily\fontsize{10.000000}{12.000000}\selectfont 0.8}%
\end{pgfscope}%
\begin{pgfscope}%
\pgfpathrectangle{\pgfqpoint{0.555972in}{0.387222in}}{\pgfqpoint{3.323208in}{2.324444in}}%
\pgfusepath{clip}%
\pgfsetrectcap%
\pgfsetroundjoin%
\pgfsetlinewidth{1.505625pt}%
\definecolor{currentstroke}{rgb}{0.121569,0.466667,0.705882}%
\pgfsetstrokecolor{currentstroke}%
\pgfsetdash{}{0pt}%
\pgfpathmoveto{\pgfqpoint{0.715320in}{2.598199in}}%
\pgfpathlineto{\pgfqpoint{0.865545in}{2.565756in}}%
\pgfpathlineto{\pgfqpoint{1.015771in}{2.517195in}}%
\pgfpathlineto{\pgfqpoint{1.165997in}{2.492652in}}%
\pgfpathlineto{\pgfqpoint{1.316222in}{2.366139in}}%
\pgfpathlineto{\pgfqpoint{1.466448in}{2.362589in}}%
\pgfpathlineto{\pgfqpoint{1.616674in}{2.283099in}}%
\pgfpathlineto{\pgfqpoint{1.766899in}{2.180275in}}%
\pgfpathlineto{\pgfqpoint{1.917125in}{1.938112in}}%
\pgfpathlineto{\pgfqpoint{2.067351in}{1.836066in}}%
\pgfpathlineto{\pgfqpoint{2.217576in}{1.744770in}}%
\pgfpathlineto{\pgfqpoint{2.367802in}{1.559593in}}%
\pgfpathlineto{\pgfqpoint{2.518028in}{1.422434in}}%
\pgfpathlineto{\pgfqpoint{2.668253in}{1.235138in}}%
\pgfpathlineto{\pgfqpoint{2.818479in}{0.947653in}}%
\pgfpathlineto{\pgfqpoint{2.968705in}{0.912481in}}%
\pgfpathlineto{\pgfqpoint{3.118930in}{0.811670in}}%
\pgfpathlineto{\pgfqpoint{3.269156in}{0.678662in}}%
\pgfpathlineto{\pgfqpoint{3.419382in}{0.676390in}}%
\pgfpathlineto{\pgfqpoint{3.569607in}{0.594126in}}%
\pgfpathlineto{\pgfqpoint{3.719833in}{0.535831in}}%
\pgfusepath{stroke}%
\end{pgfscope}%
\begin{pgfscope}%
\pgfsetrectcap%
\pgfsetmiterjoin%
\pgfsetlinewidth{0.803000pt}%
\definecolor{currentstroke}{rgb}{0.000000,0.000000,0.000000}%
\pgfsetstrokecolor{currentstroke}%
\pgfsetdash{}{0pt}%
\pgfpathmoveto{\pgfqpoint{0.555972in}{0.387222in}}%
\pgfpathlineto{\pgfqpoint{0.555972in}{2.711667in}}%
\pgfusepath{stroke}%
\end{pgfscope}%
\begin{pgfscope}%
\pgfsetrectcap%
\pgfsetmiterjoin%
\pgfsetlinewidth{0.803000pt}%
\definecolor{currentstroke}{rgb}{0.000000,0.000000,0.000000}%
\pgfsetstrokecolor{currentstroke}%
\pgfsetdash{}{0pt}%
\pgfpathmoveto{\pgfqpoint{3.879180in}{0.387222in}}%
\pgfpathlineto{\pgfqpoint{3.879180in}{2.711667in}}%
\pgfusepath{stroke}%
\end{pgfscope}%
\begin{pgfscope}%
\pgfsetrectcap%
\pgfsetmiterjoin%
\pgfsetlinewidth{0.803000pt}%
\definecolor{currentstroke}{rgb}{0.000000,0.000000,0.000000}%
\pgfsetstrokecolor{currentstroke}%
\pgfsetdash{}{0pt}%
\pgfpathmoveto{\pgfqpoint{0.555972in}{0.387222in}}%
\pgfpathlineto{\pgfqpoint{3.879180in}{0.387222in}}%
\pgfusepath{stroke}%
\end{pgfscope}%
\begin{pgfscope}%
\pgfsetrectcap%
\pgfsetmiterjoin%
\pgfsetlinewidth{0.803000pt}%
\definecolor{currentstroke}{rgb}{0.000000,0.000000,0.000000}%
\pgfsetstrokecolor{currentstroke}%
\pgfsetdash{}{0pt}%
\pgfpathmoveto{\pgfqpoint{0.555972in}{2.711667in}}%
\pgfpathlineto{\pgfqpoint{3.879180in}{2.711667in}}%
\pgfusepath{stroke}%
\end{pgfscope}%
\begin{pgfscope}%
\definecolor{textcolor}{rgb}{0.000000,0.000000,0.000000}%
\pgfsetstrokecolor{textcolor}%
\pgfsetfillcolor{textcolor}%
\pgftext[x=2.217576in,y=2.795000in,,base]{\color{textcolor}\sffamily\fontsize{12.000000}{14.400000}\selectfont \(\displaystyle  N = 64 \)}%
\end{pgfscope}%
\begin{pgfscope}%
\pgfsetbuttcap%
\pgfsetmiterjoin%
\definecolor{currentfill}{rgb}{1.000000,1.000000,1.000000}%
\pgfsetfillcolor{currentfill}%
\pgfsetlinewidth{0.000000pt}%
\definecolor{currentstroke}{rgb}{0.000000,0.000000,0.000000}%
\pgfsetstrokecolor{currentstroke}%
\pgfsetstrokeopacity{0.000000}%
\pgfsetdash{}{0pt}%
\pgfpathmoveto{\pgfqpoint{4.381402in}{0.387222in}}%
\pgfpathlineto{\pgfqpoint{7.704610in}{0.387222in}}%
\pgfpathlineto{\pgfqpoint{7.704610in}{2.711667in}}%
\pgfpathlineto{\pgfqpoint{4.381402in}{2.711667in}}%
\pgfpathclose%
\pgfusepath{fill}%
\end{pgfscope}%
\begin{pgfscope}%
\pgfpathrectangle{\pgfqpoint{4.381402in}{0.387222in}}{\pgfqpoint{3.323208in}{2.324444in}}%
\pgfusepath{clip}%
\pgfsetbuttcap%
\pgfsetroundjoin%
\definecolor{currentfill}{rgb}{0.121569,0.466667,0.705882}%
\pgfsetfillcolor{currentfill}%
\pgfsetfillopacity{0.300000}%
\pgfsetlinewidth{1.003750pt}%
\definecolor{currentstroke}{rgb}{0.121569,0.466667,0.705882}%
\pgfsetstrokecolor{currentstroke}%
\pgfsetstrokeopacity{0.300000}%
\pgfsetdash{}{0pt}%
\pgfpathmoveto{\pgfqpoint{4.540750in}{2.289096in}}%
\pgfpathlineto{\pgfqpoint{4.540750in}{2.270852in}}%
\pgfpathlineto{\pgfqpoint{4.690975in}{2.168069in}}%
\pgfpathlineto{\pgfqpoint{4.841201in}{0.747170in}}%
\pgfpathlineto{\pgfqpoint{4.991427in}{2.131122in}}%
\pgfpathlineto{\pgfqpoint{5.141652in}{1.078008in}}%
\pgfpathlineto{\pgfqpoint{5.291878in}{1.169941in}}%
\pgfpathlineto{\pgfqpoint{5.442104in}{1.591444in}}%
\pgfpathlineto{\pgfqpoint{5.592329in}{1.853860in}}%
\pgfpathlineto{\pgfqpoint{5.742555in}{1.487579in}}%
\pgfpathlineto{\pgfqpoint{5.892781in}{0.913496in}}%
\pgfpathlineto{\pgfqpoint{6.043006in}{1.389091in}}%
\pgfpathlineto{\pgfqpoint{6.193232in}{1.519057in}}%
\pgfpathlineto{\pgfqpoint{6.343458in}{0.721929in}}%
\pgfpathlineto{\pgfqpoint{6.493683in}{0.656636in}}%
\pgfpathlineto{\pgfqpoint{6.643909in}{0.758534in}}%
\pgfpathlineto{\pgfqpoint{6.794135in}{0.647925in}}%
\pgfpathlineto{\pgfqpoint{6.944360in}{0.574055in}}%
\pgfpathlineto{\pgfqpoint{7.094586in}{0.578995in}}%
\pgfpathlineto{\pgfqpoint{7.244812in}{0.555824in}}%
\pgfpathlineto{\pgfqpoint{7.395037in}{0.492879in}}%
\pgfpathlineto{\pgfqpoint{7.545263in}{0.504635in}}%
\pgfpathlineto{\pgfqpoint{7.545263in}{0.660267in}}%
\pgfpathlineto{\pgfqpoint{7.545263in}{0.660267in}}%
\pgfpathlineto{\pgfqpoint{7.395037in}{0.635495in}}%
\pgfpathlineto{\pgfqpoint{7.244812in}{0.689163in}}%
\pgfpathlineto{\pgfqpoint{7.094586in}{0.746297in}}%
\pgfpathlineto{\pgfqpoint{6.944360in}{0.879472in}}%
\pgfpathlineto{\pgfqpoint{6.794135in}{0.720572in}}%
\pgfpathlineto{\pgfqpoint{6.643909in}{0.993466in}}%
\pgfpathlineto{\pgfqpoint{6.493683in}{1.301553in}}%
\pgfpathlineto{\pgfqpoint{6.343458in}{1.330451in}}%
\pgfpathlineto{\pgfqpoint{6.193232in}{1.736715in}}%
\pgfpathlineto{\pgfqpoint{6.043006in}{1.849386in}}%
\pgfpathlineto{\pgfqpoint{5.892781in}{2.262567in}}%
\pgfpathlineto{\pgfqpoint{5.742555in}{2.005338in}}%
\pgfpathlineto{\pgfqpoint{5.592329in}{2.019169in}}%
\pgfpathlineto{\pgfqpoint{5.442104in}{2.305613in}}%
\pgfpathlineto{\pgfqpoint{5.291878in}{2.407230in}}%
\pgfpathlineto{\pgfqpoint{5.141652in}{2.606010in}}%
\pgfpathlineto{\pgfqpoint{4.991427in}{2.216837in}}%
\pgfpathlineto{\pgfqpoint{4.841201in}{2.128280in}}%
\pgfpathlineto{\pgfqpoint{4.690975in}{2.290608in}}%
\pgfpathlineto{\pgfqpoint{4.540750in}{2.289096in}}%
\pgfpathclose%
\pgfusepath{stroke,fill}%
\end{pgfscope}%
\begin{pgfscope}%
\pgfpathrectangle{\pgfqpoint{4.381402in}{0.387222in}}{\pgfqpoint{3.323208in}{2.324444in}}%
\pgfusepath{clip}%
\pgfsetbuttcap%
\pgfsetroundjoin%
\definecolor{currentfill}{rgb}{0.121569,0.466667,0.705882}%
\pgfsetfillcolor{currentfill}%
\pgfsetlinewidth{1.003750pt}%
\definecolor{currentstroke}{rgb}{0.121569,0.466667,0.705882}%
\pgfsetstrokecolor{currentstroke}%
\pgfsetdash{}{0pt}%
\pgfsys@defobject{currentmarker}{\pgfqpoint{-0.009821in}{-0.009821in}}{\pgfqpoint{0.009821in}{0.009821in}}{%
\pgfpathmoveto{\pgfqpoint{0.000000in}{-0.009821in}}%
\pgfpathcurveto{\pgfqpoint{0.002605in}{-0.009821in}}{\pgfqpoint{0.005103in}{-0.008786in}}{\pgfqpoint{0.006944in}{-0.006944in}}%
\pgfpathcurveto{\pgfqpoint{0.008786in}{-0.005103in}}{\pgfqpoint{0.009821in}{-0.002605in}}{\pgfqpoint{0.009821in}{0.000000in}}%
\pgfpathcurveto{\pgfqpoint{0.009821in}{0.002605in}}{\pgfqpoint{0.008786in}{0.005103in}}{\pgfqpoint{0.006944in}{0.006944in}}%
\pgfpathcurveto{\pgfqpoint{0.005103in}{0.008786in}}{\pgfqpoint{0.002605in}{0.009821in}}{\pgfqpoint{0.000000in}{0.009821in}}%
\pgfpathcurveto{\pgfqpoint{-0.002605in}{0.009821in}}{\pgfqpoint{-0.005103in}{0.008786in}}{\pgfqpoint{-0.006944in}{0.006944in}}%
\pgfpathcurveto{\pgfqpoint{-0.008786in}{0.005103in}}{\pgfqpoint{-0.009821in}{0.002605in}}{\pgfqpoint{-0.009821in}{0.000000in}}%
\pgfpathcurveto{\pgfqpoint{-0.009821in}{-0.002605in}}{\pgfqpoint{-0.008786in}{-0.005103in}}{\pgfqpoint{-0.006944in}{-0.006944in}}%
\pgfpathcurveto{\pgfqpoint{-0.005103in}{-0.008786in}}{\pgfqpoint{-0.002605in}{-0.009821in}}{\pgfqpoint{0.000000in}{-0.009821in}}%
\pgfpathclose%
\pgfusepath{stroke,fill}%
}%
\begin{pgfscope}%
\pgfsys@transformshift{4.540750in}{2.279974in}%
\pgfsys@useobject{currentmarker}{}%
\end{pgfscope}%
\begin{pgfscope}%
\pgfsys@transformshift{4.690975in}{2.229338in}%
\pgfsys@useobject{currentmarker}{}%
\end{pgfscope}%
\begin{pgfscope}%
\pgfsys@transformshift{4.841201in}{1.437725in}%
\pgfsys@useobject{currentmarker}{}%
\end{pgfscope}%
\begin{pgfscope}%
\pgfsys@transformshift{4.991427in}{2.173980in}%
\pgfsys@useobject{currentmarker}{}%
\end{pgfscope}%
\begin{pgfscope}%
\pgfsys@transformshift{5.141652in}{1.842009in}%
\pgfsys@useobject{currentmarker}{}%
\end{pgfscope}%
\begin{pgfscope}%
\pgfsys@transformshift{5.291878in}{1.788585in}%
\pgfsys@useobject{currentmarker}{}%
\end{pgfscope}%
\begin{pgfscope}%
\pgfsys@transformshift{5.442104in}{1.948529in}%
\pgfsys@useobject{currentmarker}{}%
\end{pgfscope}%
\begin{pgfscope}%
\pgfsys@transformshift{5.592329in}{1.936514in}%
\pgfsys@useobject{currentmarker}{}%
\end{pgfscope}%
\begin{pgfscope}%
\pgfsys@transformshift{5.742555in}{1.746459in}%
\pgfsys@useobject{currentmarker}{}%
\end{pgfscope}%
\begin{pgfscope}%
\pgfsys@transformshift{5.892781in}{1.588031in}%
\pgfsys@useobject{currentmarker}{}%
\end{pgfscope}%
\begin{pgfscope}%
\pgfsys@transformshift{6.043006in}{1.619238in}%
\pgfsys@useobject{currentmarker}{}%
\end{pgfscope}%
\begin{pgfscope}%
\pgfsys@transformshift{6.193232in}{1.627886in}%
\pgfsys@useobject{currentmarker}{}%
\end{pgfscope}%
\begin{pgfscope}%
\pgfsys@transformshift{6.343458in}{1.026190in}%
\pgfsys@useobject{currentmarker}{}%
\end{pgfscope}%
\begin{pgfscope}%
\pgfsys@transformshift{6.493683in}{0.979094in}%
\pgfsys@useobject{currentmarker}{}%
\end{pgfscope}%
\begin{pgfscope}%
\pgfsys@transformshift{6.643909in}{0.876000in}%
\pgfsys@useobject{currentmarker}{}%
\end{pgfscope}%
\begin{pgfscope}%
\pgfsys@transformshift{6.794135in}{0.684249in}%
\pgfsys@useobject{currentmarker}{}%
\end{pgfscope}%
\begin{pgfscope}%
\pgfsys@transformshift{6.944360in}{0.726763in}%
\pgfsys@useobject{currentmarker}{}%
\end{pgfscope}%
\begin{pgfscope}%
\pgfsys@transformshift{7.094586in}{0.662646in}%
\pgfsys@useobject{currentmarker}{}%
\end{pgfscope}%
\begin{pgfscope}%
\pgfsys@transformshift{7.244812in}{0.622494in}%
\pgfsys@useobject{currentmarker}{}%
\end{pgfscope}%
\begin{pgfscope}%
\pgfsys@transformshift{7.395037in}{0.564187in}%
\pgfsys@useobject{currentmarker}{}%
\end{pgfscope}%
\begin{pgfscope}%
\pgfsys@transformshift{7.545263in}{0.582451in}%
\pgfsys@useobject{currentmarker}{}%
\end{pgfscope}%
\end{pgfscope}%
\begin{pgfscope}%
\pgfsetbuttcap%
\pgfsetroundjoin%
\definecolor{currentfill}{rgb}{0.000000,0.000000,0.000000}%
\pgfsetfillcolor{currentfill}%
\pgfsetlinewidth{0.803000pt}%
\definecolor{currentstroke}{rgb}{0.000000,0.000000,0.000000}%
\pgfsetstrokecolor{currentstroke}%
\pgfsetdash{}{0pt}%
\pgfsys@defobject{currentmarker}{\pgfqpoint{0.000000in}{-0.048611in}}{\pgfqpoint{0.000000in}{0.000000in}}{%
\pgfpathmoveto{\pgfqpoint{0.000000in}{0.000000in}}%
\pgfpathlineto{\pgfqpoint{0.000000in}{-0.048611in}}%
\pgfusepath{stroke,fill}%
}%
\begin{pgfscope}%
\pgfsys@transformshift{4.991427in}{0.387222in}%
\pgfsys@useobject{currentmarker}{}%
\end{pgfscope}%
\end{pgfscope}%
\begin{pgfscope}%
\definecolor{textcolor}{rgb}{0.000000,0.000000,0.000000}%
\pgfsetstrokecolor{textcolor}%
\pgfsetfillcolor{textcolor}%
\pgftext[x=4.991427in,y=0.290000in,,top]{\color{textcolor}\sffamily\fontsize{10.000000}{12.000000}\selectfont 2.20}%
\end{pgfscope}%
\begin{pgfscope}%
\pgfsetbuttcap%
\pgfsetroundjoin%
\definecolor{currentfill}{rgb}{0.000000,0.000000,0.000000}%
\pgfsetfillcolor{currentfill}%
\pgfsetlinewidth{0.803000pt}%
\definecolor{currentstroke}{rgb}{0.000000,0.000000,0.000000}%
\pgfsetstrokecolor{currentstroke}%
\pgfsetdash{}{0pt}%
\pgfsys@defobject{currentmarker}{\pgfqpoint{0.000000in}{-0.048611in}}{\pgfqpoint{0.000000in}{0.000000in}}{%
\pgfpathmoveto{\pgfqpoint{0.000000in}{0.000000in}}%
\pgfpathlineto{\pgfqpoint{0.000000in}{-0.048611in}}%
\pgfusepath{stroke,fill}%
}%
\begin{pgfscope}%
\pgfsys@transformshift{5.742555in}{0.387222in}%
\pgfsys@useobject{currentmarker}{}%
\end{pgfscope}%
\end{pgfscope}%
\begin{pgfscope}%
\definecolor{textcolor}{rgb}{0.000000,0.000000,0.000000}%
\pgfsetstrokecolor{textcolor}%
\pgfsetfillcolor{textcolor}%
\pgftext[x=5.742555in,y=0.290000in,,top]{\color{textcolor}\sffamily\fontsize{10.000000}{12.000000}\selectfont 2.25}%
\end{pgfscope}%
\begin{pgfscope}%
\pgfsetbuttcap%
\pgfsetroundjoin%
\definecolor{currentfill}{rgb}{0.000000,0.000000,0.000000}%
\pgfsetfillcolor{currentfill}%
\pgfsetlinewidth{0.803000pt}%
\definecolor{currentstroke}{rgb}{0.000000,0.000000,0.000000}%
\pgfsetstrokecolor{currentstroke}%
\pgfsetdash{}{0pt}%
\pgfsys@defobject{currentmarker}{\pgfqpoint{0.000000in}{-0.048611in}}{\pgfqpoint{0.000000in}{0.000000in}}{%
\pgfpathmoveto{\pgfqpoint{0.000000in}{0.000000in}}%
\pgfpathlineto{\pgfqpoint{0.000000in}{-0.048611in}}%
\pgfusepath{stroke,fill}%
}%
\begin{pgfscope}%
\pgfsys@transformshift{6.493683in}{0.387222in}%
\pgfsys@useobject{currentmarker}{}%
\end{pgfscope}%
\end{pgfscope}%
\begin{pgfscope}%
\definecolor{textcolor}{rgb}{0.000000,0.000000,0.000000}%
\pgfsetstrokecolor{textcolor}%
\pgfsetfillcolor{textcolor}%
\pgftext[x=6.493683in,y=0.290000in,,top]{\color{textcolor}\sffamily\fontsize{10.000000}{12.000000}\selectfont 2.30}%
\end{pgfscope}%
\begin{pgfscope}%
\pgfsetbuttcap%
\pgfsetroundjoin%
\definecolor{currentfill}{rgb}{0.000000,0.000000,0.000000}%
\pgfsetfillcolor{currentfill}%
\pgfsetlinewidth{0.803000pt}%
\definecolor{currentstroke}{rgb}{0.000000,0.000000,0.000000}%
\pgfsetstrokecolor{currentstroke}%
\pgfsetdash{}{0pt}%
\pgfsys@defobject{currentmarker}{\pgfqpoint{0.000000in}{-0.048611in}}{\pgfqpoint{0.000000in}{0.000000in}}{%
\pgfpathmoveto{\pgfqpoint{0.000000in}{0.000000in}}%
\pgfpathlineto{\pgfqpoint{0.000000in}{-0.048611in}}%
\pgfusepath{stroke,fill}%
}%
\begin{pgfscope}%
\pgfsys@transformshift{7.244812in}{0.387222in}%
\pgfsys@useobject{currentmarker}{}%
\end{pgfscope}%
\end{pgfscope}%
\begin{pgfscope}%
\definecolor{textcolor}{rgb}{0.000000,0.000000,0.000000}%
\pgfsetstrokecolor{textcolor}%
\pgfsetfillcolor{textcolor}%
\pgftext[x=7.244812in,y=0.290000in,,top]{\color{textcolor}\sffamily\fontsize{10.000000}{12.000000}\selectfont 2.35}%
\end{pgfscope}%
\begin{pgfscope}%
\pgfsetbuttcap%
\pgfsetroundjoin%
\definecolor{currentfill}{rgb}{0.000000,0.000000,0.000000}%
\pgfsetfillcolor{currentfill}%
\pgfsetlinewidth{0.803000pt}%
\definecolor{currentstroke}{rgb}{0.000000,0.000000,0.000000}%
\pgfsetstrokecolor{currentstroke}%
\pgfsetdash{}{0pt}%
\pgfsys@defobject{currentmarker}{\pgfqpoint{-0.048611in}{0.000000in}}{\pgfqpoint{0.000000in}{0.000000in}}{%
\pgfpathmoveto{\pgfqpoint{0.000000in}{0.000000in}}%
\pgfpathlineto{\pgfqpoint{-0.048611in}{0.000000in}}%
\pgfusepath{stroke,fill}%
}%
\begin{pgfscope}%
\pgfsys@transformshift{4.381402in}{0.804575in}%
\pgfsys@useobject{currentmarker}{}%
\end{pgfscope}%
\end{pgfscope}%
\begin{pgfscope}%
\definecolor{textcolor}{rgb}{0.000000,0.000000,0.000000}%
\pgfsetstrokecolor{textcolor}%
\pgfsetfillcolor{textcolor}%
\pgftext[x=4.063301in,y=0.751814in,left,base]{\color{textcolor}\sffamily\fontsize{10.000000}{12.000000}\selectfont 0.2}%
\end{pgfscope}%
\begin{pgfscope}%
\pgfsetbuttcap%
\pgfsetroundjoin%
\definecolor{currentfill}{rgb}{0.000000,0.000000,0.000000}%
\pgfsetfillcolor{currentfill}%
\pgfsetlinewidth{0.803000pt}%
\definecolor{currentstroke}{rgb}{0.000000,0.000000,0.000000}%
\pgfsetstrokecolor{currentstroke}%
\pgfsetdash{}{0pt}%
\pgfsys@defobject{currentmarker}{\pgfqpoint{-0.048611in}{0.000000in}}{\pgfqpoint{0.000000in}{0.000000in}}{%
\pgfpathmoveto{\pgfqpoint{0.000000in}{0.000000in}}%
\pgfpathlineto{\pgfqpoint{-0.048611in}{0.000000in}}%
\pgfusepath{stroke,fill}%
}%
\begin{pgfscope}%
\pgfsys@transformshift{4.381402in}{1.281169in}%
\pgfsys@useobject{currentmarker}{}%
\end{pgfscope}%
\end{pgfscope}%
\begin{pgfscope}%
\definecolor{textcolor}{rgb}{0.000000,0.000000,0.000000}%
\pgfsetstrokecolor{textcolor}%
\pgfsetfillcolor{textcolor}%
\pgftext[x=4.063301in,y=1.228408in,left,base]{\color{textcolor}\sffamily\fontsize{10.000000}{12.000000}\selectfont 0.4}%
\end{pgfscope}%
\begin{pgfscope}%
\pgfsetbuttcap%
\pgfsetroundjoin%
\definecolor{currentfill}{rgb}{0.000000,0.000000,0.000000}%
\pgfsetfillcolor{currentfill}%
\pgfsetlinewidth{0.803000pt}%
\definecolor{currentstroke}{rgb}{0.000000,0.000000,0.000000}%
\pgfsetstrokecolor{currentstroke}%
\pgfsetdash{}{0pt}%
\pgfsys@defobject{currentmarker}{\pgfqpoint{-0.048611in}{0.000000in}}{\pgfqpoint{0.000000in}{0.000000in}}{%
\pgfpathmoveto{\pgfqpoint{0.000000in}{0.000000in}}%
\pgfpathlineto{\pgfqpoint{-0.048611in}{0.000000in}}%
\pgfusepath{stroke,fill}%
}%
\begin{pgfscope}%
\pgfsys@transformshift{4.381402in}{1.757764in}%
\pgfsys@useobject{currentmarker}{}%
\end{pgfscope}%
\end{pgfscope}%
\begin{pgfscope}%
\definecolor{textcolor}{rgb}{0.000000,0.000000,0.000000}%
\pgfsetstrokecolor{textcolor}%
\pgfsetfillcolor{textcolor}%
\pgftext[x=4.063301in,y=1.705002in,left,base]{\color{textcolor}\sffamily\fontsize{10.000000}{12.000000}\selectfont 0.6}%
\end{pgfscope}%
\begin{pgfscope}%
\pgfsetbuttcap%
\pgfsetroundjoin%
\definecolor{currentfill}{rgb}{0.000000,0.000000,0.000000}%
\pgfsetfillcolor{currentfill}%
\pgfsetlinewidth{0.803000pt}%
\definecolor{currentstroke}{rgb}{0.000000,0.000000,0.000000}%
\pgfsetstrokecolor{currentstroke}%
\pgfsetdash{}{0pt}%
\pgfsys@defobject{currentmarker}{\pgfqpoint{-0.048611in}{0.000000in}}{\pgfqpoint{0.000000in}{0.000000in}}{%
\pgfpathmoveto{\pgfqpoint{0.000000in}{0.000000in}}%
\pgfpathlineto{\pgfqpoint{-0.048611in}{0.000000in}}%
\pgfusepath{stroke,fill}%
}%
\begin{pgfscope}%
\pgfsys@transformshift{4.381402in}{2.234358in}%
\pgfsys@useobject{currentmarker}{}%
\end{pgfscope}%
\end{pgfscope}%
\begin{pgfscope}%
\definecolor{textcolor}{rgb}{0.000000,0.000000,0.000000}%
\pgfsetstrokecolor{textcolor}%
\pgfsetfillcolor{textcolor}%
\pgftext[x=4.063301in,y=2.181596in,left,base]{\color{textcolor}\sffamily\fontsize{10.000000}{12.000000}\selectfont 0.8}%
\end{pgfscope}%
\begin{pgfscope}%
\pgfsetbuttcap%
\pgfsetroundjoin%
\definecolor{currentfill}{rgb}{0.000000,0.000000,0.000000}%
\pgfsetfillcolor{currentfill}%
\pgfsetlinewidth{0.803000pt}%
\definecolor{currentstroke}{rgb}{0.000000,0.000000,0.000000}%
\pgfsetstrokecolor{currentstroke}%
\pgfsetdash{}{0pt}%
\pgfsys@defobject{currentmarker}{\pgfqpoint{-0.048611in}{0.000000in}}{\pgfqpoint{0.000000in}{0.000000in}}{%
\pgfpathmoveto{\pgfqpoint{0.000000in}{0.000000in}}%
\pgfpathlineto{\pgfqpoint{-0.048611in}{0.000000in}}%
\pgfusepath{stroke,fill}%
}%
\begin{pgfscope}%
\pgfsys@transformshift{4.381402in}{2.710952in}%
\pgfsys@useobject{currentmarker}{}%
\end{pgfscope}%
\end{pgfscope}%
\begin{pgfscope}%
\definecolor{textcolor}{rgb}{0.000000,0.000000,0.000000}%
\pgfsetstrokecolor{textcolor}%
\pgfsetfillcolor{textcolor}%
\pgftext[x=4.063301in,y=2.658190in,left,base]{\color{textcolor}\sffamily\fontsize{10.000000}{12.000000}\selectfont 1.0}%
\end{pgfscope}%
\begin{pgfscope}%
\pgfpathrectangle{\pgfqpoint{4.381402in}{0.387222in}}{\pgfqpoint{3.323208in}{2.324444in}}%
\pgfusepath{clip}%
\pgfsetrectcap%
\pgfsetroundjoin%
\pgfsetlinewidth{1.505625pt}%
\definecolor{currentstroke}{rgb}{0.121569,0.466667,0.705882}%
\pgfsetstrokecolor{currentstroke}%
\pgfsetdash{}{0pt}%
\pgfpathmoveto{\pgfqpoint{4.540750in}{2.279974in}}%
\pgfpathlineto{\pgfqpoint{4.690975in}{2.229338in}}%
\pgfpathlineto{\pgfqpoint{4.841201in}{1.437725in}}%
\pgfpathlineto{\pgfqpoint{4.991427in}{2.173980in}}%
\pgfpathlineto{\pgfqpoint{5.141652in}{1.842009in}}%
\pgfpathlineto{\pgfqpoint{5.291878in}{1.788585in}}%
\pgfpathlineto{\pgfqpoint{5.442104in}{1.948529in}}%
\pgfpathlineto{\pgfqpoint{5.592329in}{1.936514in}}%
\pgfpathlineto{\pgfqpoint{5.742555in}{1.746459in}}%
\pgfpathlineto{\pgfqpoint{5.892781in}{1.588031in}}%
\pgfpathlineto{\pgfqpoint{6.043006in}{1.619238in}}%
\pgfpathlineto{\pgfqpoint{6.193232in}{1.627886in}}%
\pgfpathlineto{\pgfqpoint{6.343458in}{1.026190in}}%
\pgfpathlineto{\pgfqpoint{6.493683in}{0.979094in}}%
\pgfpathlineto{\pgfqpoint{6.643909in}{0.876000in}}%
\pgfpathlineto{\pgfqpoint{6.794135in}{0.684249in}}%
\pgfpathlineto{\pgfqpoint{6.944360in}{0.726763in}}%
\pgfpathlineto{\pgfqpoint{7.094586in}{0.662646in}}%
\pgfpathlineto{\pgfqpoint{7.244812in}{0.622494in}}%
\pgfpathlineto{\pgfqpoint{7.395037in}{0.564187in}}%
\pgfpathlineto{\pgfqpoint{7.545263in}{0.582451in}}%
\pgfusepath{stroke}%
\end{pgfscope}%
\begin{pgfscope}%
\pgfsetrectcap%
\pgfsetmiterjoin%
\pgfsetlinewidth{0.803000pt}%
\definecolor{currentstroke}{rgb}{0.000000,0.000000,0.000000}%
\pgfsetstrokecolor{currentstroke}%
\pgfsetdash{}{0pt}%
\pgfpathmoveto{\pgfqpoint{4.381402in}{0.387222in}}%
\pgfpathlineto{\pgfqpoint{4.381402in}{2.711667in}}%
\pgfusepath{stroke}%
\end{pgfscope}%
\begin{pgfscope}%
\pgfsetrectcap%
\pgfsetmiterjoin%
\pgfsetlinewidth{0.803000pt}%
\definecolor{currentstroke}{rgb}{0.000000,0.000000,0.000000}%
\pgfsetstrokecolor{currentstroke}%
\pgfsetdash{}{0pt}%
\pgfpathmoveto{\pgfqpoint{7.704610in}{0.387222in}}%
\pgfpathlineto{\pgfqpoint{7.704610in}{2.711667in}}%
\pgfusepath{stroke}%
\end{pgfscope}%
\begin{pgfscope}%
\pgfsetrectcap%
\pgfsetmiterjoin%
\pgfsetlinewidth{0.803000pt}%
\definecolor{currentstroke}{rgb}{0.000000,0.000000,0.000000}%
\pgfsetstrokecolor{currentstroke}%
\pgfsetdash{}{0pt}%
\pgfpathmoveto{\pgfqpoint{4.381402in}{0.387222in}}%
\pgfpathlineto{\pgfqpoint{7.704610in}{0.387222in}}%
\pgfusepath{stroke}%
\end{pgfscope}%
\begin{pgfscope}%
\pgfsetrectcap%
\pgfsetmiterjoin%
\pgfsetlinewidth{0.803000pt}%
\definecolor{currentstroke}{rgb}{0.000000,0.000000,0.000000}%
\pgfsetstrokecolor{currentstroke}%
\pgfsetdash{}{0pt}%
\pgfpathmoveto{\pgfqpoint{4.381402in}{2.711667in}}%
\pgfpathlineto{\pgfqpoint{7.704610in}{2.711667in}}%
\pgfusepath{stroke}%
\end{pgfscope}%
\begin{pgfscope}%
\definecolor{textcolor}{rgb}{0.000000,0.000000,0.000000}%
\pgfsetstrokecolor{textcolor}%
\pgfsetfillcolor{textcolor}%
\pgftext[x=6.043006in,y=2.795000in,,base]{\color{textcolor}\sffamily\fontsize{12.000000}{14.400000}\selectfont \(\displaystyle  N = 128 \)}%
\end{pgfscope}%
\end{pgfpicture}%
\makeatother%
\endgroup%
}
\caption{Magnetization $m$ for finer temperature $T$ using Metropolis--Hastings algorithm}
\label{Fig:MagBig}
\end{figure}

We may find that the curves suffer from big variance and oscillation below the critical temperature. This is because of the huge rejection rate of Metropolis--Hastings algorithm when the temperature is low: the energy of some configurations is so low that it will hardly transfer to another configuration. The critical temperature is about 2.25 from the figure.

\subsection{2-D case using kinetic Monte Carlo algorithm}

We then turn to the improved algorithm --- kinetic Monte Carlo algorithm. We use identical numerical settings and obtain the numerical results in Figure \ref{Fig:HeatSmallKMC}, \ref{Fig:CapSmallKMC}, \ref{Fig:MagSmallKMC}, \ref{Fig:HeatBigKMC}, \ref{Fig:CapBigKMC} and \ref{Fig:MagSmallKMC}.

\begin{figure}[htbp]
\centering
\scalebox{0.666}{%% Creator: Matplotlib, PGF backend
%%
%% To include the figure in your LaTeX document, write
%%   \input{<filename>.pgf}
%%
%% Make sure the required packages are loaded in your preamble
%%   \usepackage{pgf}
%%
%% Figures using additional raster images can only be included by \input if
%% they are in the same directory as the main LaTeX file. For loading figures
%% from other directories you can use the `import` package
%%   \usepackage{import}
%% and then include the figures with
%%   \import{<path to file>}{<filename>.pgf}
%%
%% Matplotlib used the following preamble
%%   \usepackage{fontspec}
%%   \setmainfont{DejaVuSerif.ttf}[Path=/home/lzh/anaconda3/envs/numana/lib/python3.7/site-packages/matplotlib/mpl-data/fonts/ttf/]
%%   \setsansfont{DejaVuSans.ttf}[Path=/home/lzh/anaconda3/envs/numana/lib/python3.7/site-packages/matplotlib/mpl-data/fonts/ttf/]
%%   \setmonofont{DejaVuSansMono.ttf}[Path=/home/lzh/anaconda3/envs/numana/lib/python3.7/site-packages/matplotlib/mpl-data/fonts/ttf/]
%%
\begingroup%
\makeatletter%
\begin{pgfpicture}%
\pgfpathrectangle{\pgfpointorigin}{\pgfqpoint{8.000000in}{6.000000in}}%
\pgfusepath{use as bounding box, clip}%
\begin{pgfscope}%
\pgfsetbuttcap%
\pgfsetmiterjoin%
\definecolor{currentfill}{rgb}{1.000000,1.000000,1.000000}%
\pgfsetfillcolor{currentfill}%
\pgfsetlinewidth{0.000000pt}%
\definecolor{currentstroke}{rgb}{1.000000,1.000000,1.000000}%
\pgfsetstrokecolor{currentstroke}%
\pgfsetdash{}{0pt}%
\pgfpathmoveto{\pgfqpoint{0.000000in}{0.000000in}}%
\pgfpathlineto{\pgfqpoint{8.000000in}{0.000000in}}%
\pgfpathlineto{\pgfqpoint{8.000000in}{6.000000in}}%
\pgfpathlineto{\pgfqpoint{0.000000in}{6.000000in}}%
\pgfpathclose%
\pgfusepath{fill}%
\end{pgfscope}%
\begin{pgfscope}%
\pgfsetbuttcap%
\pgfsetmiterjoin%
\definecolor{currentfill}{rgb}{1.000000,1.000000,1.000000}%
\pgfsetfillcolor{currentfill}%
\pgfsetlinewidth{0.000000pt}%
\definecolor{currentstroke}{rgb}{0.000000,0.000000,0.000000}%
\pgfsetstrokecolor{currentstroke}%
\pgfsetstrokeopacity{0.000000}%
\pgfsetdash{}{0pt}%
\pgfpathmoveto{\pgfqpoint{0.672222in}{3.312222in}}%
\pgfpathlineto{\pgfqpoint{3.890000in}{3.312222in}}%
\pgfpathlineto{\pgfqpoint{3.890000in}{5.636667in}}%
\pgfpathlineto{\pgfqpoint{0.672222in}{5.636667in}}%
\pgfpathclose%
\pgfusepath{fill}%
\end{pgfscope}%
\begin{pgfscope}%
\pgfpathrectangle{\pgfqpoint{0.672222in}{3.312222in}}{\pgfqpoint{3.217778in}{2.324444in}}%
\pgfusepath{clip}%
\pgfsetbuttcap%
\pgfsetroundjoin%
\definecolor{currentfill}{rgb}{0.121569,0.466667,0.705882}%
\pgfsetfillcolor{currentfill}%
\pgfsetfillopacity{0.300000}%
\pgfsetlinewidth{1.003750pt}%
\definecolor{currentstroke}{rgb}{0.121569,0.466667,0.705882}%
\pgfsetstrokecolor{currentstroke}%
\pgfsetstrokeopacity{0.300000}%
\pgfsetdash{}{0pt}%
\pgfpathmoveto{\pgfqpoint{0.826514in}{3.429185in}}%
\pgfpathlineto{\pgfqpoint{0.826514in}{3.429185in}}%
\pgfpathlineto{\pgfqpoint{0.899244in}{3.429168in}}%
\pgfpathlineto{\pgfqpoint{0.971974in}{3.429067in}}%
\pgfpathlineto{\pgfqpoint{1.044704in}{3.428711in}}%
\pgfpathlineto{\pgfqpoint{1.117434in}{3.427887in}}%
\pgfpathlineto{\pgfqpoint{1.190163in}{3.426622in}}%
\pgfpathlineto{\pgfqpoint{1.262893in}{3.425648in}}%
\pgfpathlineto{\pgfqpoint{1.335623in}{3.426637in}}%
\pgfpathlineto{\pgfqpoint{1.408353in}{3.431843in}}%
\pgfpathlineto{\pgfqpoint{1.481083in}{3.443439in}}%
\pgfpathlineto{\pgfqpoint{1.553813in}{3.463170in}}%
\pgfpathlineto{\pgfqpoint{1.626543in}{3.492468in}}%
\pgfpathlineto{\pgfqpoint{1.699272in}{3.532844in}}%
\pgfpathlineto{\pgfqpoint{1.772002in}{3.586343in}}%
\pgfpathlineto{\pgfqpoint{1.844732in}{3.655510in}}%
\pgfpathlineto{\pgfqpoint{1.917462in}{3.745211in}}%
\pgfpathlineto{\pgfqpoint{1.990192in}{3.862189in}}%
\pgfpathlineto{\pgfqpoint{2.062922in}{4.019319in}}%
\pgfpathlineto{\pgfqpoint{2.135651in}{4.222786in}}%
\pgfpathlineto{\pgfqpoint{2.208381in}{4.433787in}}%
\pgfpathlineto{\pgfqpoint{2.281111in}{4.608703in}}%
\pgfpathlineto{\pgfqpoint{2.353841in}{4.739901in}}%
\pgfpathlineto{\pgfqpoint{2.426571in}{4.839588in}}%
\pgfpathlineto{\pgfqpoint{2.499301in}{4.921776in}}%
\pgfpathlineto{\pgfqpoint{2.572030in}{4.991586in}}%
\pgfpathlineto{\pgfqpoint{2.644760in}{5.052078in}}%
\pgfpathlineto{\pgfqpoint{2.717490in}{5.105794in}}%
\pgfpathlineto{\pgfqpoint{2.790220in}{5.154227in}}%
\pgfpathlineto{\pgfqpoint{2.862950in}{5.197832in}}%
\pgfpathlineto{\pgfqpoint{2.935680in}{5.237746in}}%
\pgfpathlineto{\pgfqpoint{3.008410in}{5.274351in}}%
\pgfpathlineto{\pgfqpoint{3.081139in}{5.307932in}}%
\pgfpathlineto{\pgfqpoint{3.153869in}{5.338876in}}%
\pgfpathlineto{\pgfqpoint{3.226599in}{5.367622in}}%
\pgfpathlineto{\pgfqpoint{3.299329in}{5.394398in}}%
\pgfpathlineto{\pgfqpoint{3.372059in}{5.419410in}}%
\pgfpathlineto{\pgfqpoint{3.444789in}{5.442767in}}%
\pgfpathlineto{\pgfqpoint{3.517518in}{5.464517in}}%
\pgfpathlineto{\pgfqpoint{3.590248in}{5.485178in}}%
\pgfpathlineto{\pgfqpoint{3.662978in}{5.504722in}}%
\pgfpathlineto{\pgfqpoint{3.735708in}{5.523099in}}%
\pgfpathlineto{\pgfqpoint{3.735708in}{5.523382in}}%
\pgfpathlineto{\pgfqpoint{3.735708in}{5.523382in}}%
\pgfpathlineto{\pgfqpoint{3.662978in}{5.505097in}}%
\pgfpathlineto{\pgfqpoint{3.590248in}{5.485437in}}%
\pgfpathlineto{\pgfqpoint{3.517518in}{5.465015in}}%
\pgfpathlineto{\pgfqpoint{3.444789in}{5.442997in}}%
\pgfpathlineto{\pgfqpoint{3.372059in}{5.419551in}}%
\pgfpathlineto{\pgfqpoint{3.299329in}{5.394713in}}%
\pgfpathlineto{\pgfqpoint{3.226599in}{5.367936in}}%
\pgfpathlineto{\pgfqpoint{3.153869in}{5.339576in}}%
\pgfpathlineto{\pgfqpoint{3.081139in}{5.308277in}}%
\pgfpathlineto{\pgfqpoint{3.008410in}{5.274926in}}%
\pgfpathlineto{\pgfqpoint{2.935680in}{5.238147in}}%
\pgfpathlineto{\pgfqpoint{2.862950in}{5.198382in}}%
\pgfpathlineto{\pgfqpoint{2.790220in}{5.154614in}}%
\pgfpathlineto{\pgfqpoint{2.717490in}{5.106303in}}%
\pgfpathlineto{\pgfqpoint{2.644760in}{5.052651in}}%
\pgfpathlineto{\pgfqpoint{2.572030in}{4.991973in}}%
\pgfpathlineto{\pgfqpoint{2.499301in}{4.922669in}}%
\pgfpathlineto{\pgfqpoint{2.426571in}{4.840740in}}%
\pgfpathlineto{\pgfqpoint{2.353841in}{4.740859in}}%
\pgfpathlineto{\pgfqpoint{2.281111in}{4.611595in}}%
\pgfpathlineto{\pgfqpoint{2.208381in}{4.436672in}}%
\pgfpathlineto{\pgfqpoint{2.135651in}{4.223792in}}%
\pgfpathlineto{\pgfqpoint{2.062922in}{4.021501in}}%
\pgfpathlineto{\pgfqpoint{1.990192in}{3.863193in}}%
\pgfpathlineto{\pgfqpoint{1.917462in}{3.745266in}}%
\pgfpathlineto{\pgfqpoint{1.844732in}{3.655862in}}%
\pgfpathlineto{\pgfqpoint{1.772002in}{3.586540in}}%
\pgfpathlineto{\pgfqpoint{1.699272in}{3.532966in}}%
\pgfpathlineto{\pgfqpoint{1.626543in}{3.492503in}}%
\pgfpathlineto{\pgfqpoint{1.553813in}{3.463192in}}%
\pgfpathlineto{\pgfqpoint{1.481083in}{3.443464in}}%
\pgfpathlineto{\pgfqpoint{1.408353in}{3.431853in}}%
\pgfpathlineto{\pgfqpoint{1.335623in}{3.426639in}}%
\pgfpathlineto{\pgfqpoint{1.262893in}{3.425649in}}%
\pgfpathlineto{\pgfqpoint{1.190163in}{3.426622in}}%
\pgfpathlineto{\pgfqpoint{1.117434in}{3.427887in}}%
\pgfpathlineto{\pgfqpoint{1.044704in}{3.428712in}}%
\pgfpathlineto{\pgfqpoint{0.971974in}{3.429067in}}%
\pgfpathlineto{\pgfqpoint{0.899244in}{3.429168in}}%
\pgfpathlineto{\pgfqpoint{0.826514in}{3.429185in}}%
\pgfpathclose%
\pgfusepath{stroke,fill}%
\end{pgfscope}%
\begin{pgfscope}%
\pgfpathrectangle{\pgfqpoint{0.672222in}{3.312222in}}{\pgfqpoint{3.217778in}{2.324444in}}%
\pgfusepath{clip}%
\pgfsetbuttcap%
\pgfsetroundjoin%
\definecolor{currentfill}{rgb}{0.121569,0.466667,0.705882}%
\pgfsetfillcolor{currentfill}%
\pgfsetlinewidth{1.003750pt}%
\definecolor{currentstroke}{rgb}{0.121569,0.466667,0.705882}%
\pgfsetstrokecolor{currentstroke}%
\pgfsetdash{}{0pt}%
\pgfsys@defobject{currentmarker}{\pgfqpoint{-0.009821in}{-0.009821in}}{\pgfqpoint{0.009821in}{0.009821in}}{%
\pgfpathmoveto{\pgfqpoint{0.000000in}{-0.009821in}}%
\pgfpathcurveto{\pgfqpoint{0.002605in}{-0.009821in}}{\pgfqpoint{0.005103in}{-0.008786in}}{\pgfqpoint{0.006944in}{-0.006944in}}%
\pgfpathcurveto{\pgfqpoint{0.008786in}{-0.005103in}}{\pgfqpoint{0.009821in}{-0.002605in}}{\pgfqpoint{0.009821in}{0.000000in}}%
\pgfpathcurveto{\pgfqpoint{0.009821in}{0.002605in}}{\pgfqpoint{0.008786in}{0.005103in}}{\pgfqpoint{0.006944in}{0.006944in}}%
\pgfpathcurveto{\pgfqpoint{0.005103in}{0.008786in}}{\pgfqpoint{0.002605in}{0.009821in}}{\pgfqpoint{0.000000in}{0.009821in}}%
\pgfpathcurveto{\pgfqpoint{-0.002605in}{0.009821in}}{\pgfqpoint{-0.005103in}{0.008786in}}{\pgfqpoint{-0.006944in}{0.006944in}}%
\pgfpathcurveto{\pgfqpoint{-0.008786in}{0.005103in}}{\pgfqpoint{-0.009821in}{0.002605in}}{\pgfqpoint{-0.009821in}{0.000000in}}%
\pgfpathcurveto{\pgfqpoint{-0.009821in}{-0.002605in}}{\pgfqpoint{-0.008786in}{-0.005103in}}{\pgfqpoint{-0.006944in}{-0.006944in}}%
\pgfpathcurveto{\pgfqpoint{-0.005103in}{-0.008786in}}{\pgfqpoint{-0.002605in}{-0.009821in}}{\pgfqpoint{0.000000in}{-0.009821in}}%
\pgfpathclose%
\pgfusepath{stroke,fill}%
}%
\begin{pgfscope}%
\pgfsys@transformshift{0.826514in}{3.429185in}%
\pgfsys@useobject{currentmarker}{}%
\end{pgfscope}%
\begin{pgfscope}%
\pgfsys@transformshift{0.899244in}{3.429168in}%
\pgfsys@useobject{currentmarker}{}%
\end{pgfscope}%
\begin{pgfscope}%
\pgfsys@transformshift{0.971974in}{3.429067in}%
\pgfsys@useobject{currentmarker}{}%
\end{pgfscope}%
\begin{pgfscope}%
\pgfsys@transformshift{1.044704in}{3.428711in}%
\pgfsys@useobject{currentmarker}{}%
\end{pgfscope}%
\begin{pgfscope}%
\pgfsys@transformshift{1.117434in}{3.427887in}%
\pgfsys@useobject{currentmarker}{}%
\end{pgfscope}%
\begin{pgfscope}%
\pgfsys@transformshift{1.190163in}{3.426622in}%
\pgfsys@useobject{currentmarker}{}%
\end{pgfscope}%
\begin{pgfscope}%
\pgfsys@transformshift{1.262893in}{3.425649in}%
\pgfsys@useobject{currentmarker}{}%
\end{pgfscope}%
\begin{pgfscope}%
\pgfsys@transformshift{1.335623in}{3.426638in}%
\pgfsys@useobject{currentmarker}{}%
\end{pgfscope}%
\begin{pgfscope}%
\pgfsys@transformshift{1.408353in}{3.431848in}%
\pgfsys@useobject{currentmarker}{}%
\end{pgfscope}%
\begin{pgfscope}%
\pgfsys@transformshift{1.481083in}{3.443451in}%
\pgfsys@useobject{currentmarker}{}%
\end{pgfscope}%
\begin{pgfscope}%
\pgfsys@transformshift{1.553813in}{3.463181in}%
\pgfsys@useobject{currentmarker}{}%
\end{pgfscope}%
\begin{pgfscope}%
\pgfsys@transformshift{1.626543in}{3.492486in}%
\pgfsys@useobject{currentmarker}{}%
\end{pgfscope}%
\begin{pgfscope}%
\pgfsys@transformshift{1.699272in}{3.532905in}%
\pgfsys@useobject{currentmarker}{}%
\end{pgfscope}%
\begin{pgfscope}%
\pgfsys@transformshift{1.772002in}{3.586441in}%
\pgfsys@useobject{currentmarker}{}%
\end{pgfscope}%
\begin{pgfscope}%
\pgfsys@transformshift{1.844732in}{3.655686in}%
\pgfsys@useobject{currentmarker}{}%
\end{pgfscope}%
\begin{pgfscope}%
\pgfsys@transformshift{1.917462in}{3.745239in}%
\pgfsys@useobject{currentmarker}{}%
\end{pgfscope}%
\begin{pgfscope}%
\pgfsys@transformshift{1.990192in}{3.862691in}%
\pgfsys@useobject{currentmarker}{}%
\end{pgfscope}%
\begin{pgfscope}%
\pgfsys@transformshift{2.062922in}{4.020410in}%
\pgfsys@useobject{currentmarker}{}%
\end{pgfscope}%
\begin{pgfscope}%
\pgfsys@transformshift{2.135651in}{4.223289in}%
\pgfsys@useobject{currentmarker}{}%
\end{pgfscope}%
\begin{pgfscope}%
\pgfsys@transformshift{2.208381in}{4.435230in}%
\pgfsys@useobject{currentmarker}{}%
\end{pgfscope}%
\begin{pgfscope}%
\pgfsys@transformshift{2.281111in}{4.610149in}%
\pgfsys@useobject{currentmarker}{}%
\end{pgfscope}%
\begin{pgfscope}%
\pgfsys@transformshift{2.353841in}{4.740380in}%
\pgfsys@useobject{currentmarker}{}%
\end{pgfscope}%
\begin{pgfscope}%
\pgfsys@transformshift{2.426571in}{4.840164in}%
\pgfsys@useobject{currentmarker}{}%
\end{pgfscope}%
\begin{pgfscope}%
\pgfsys@transformshift{2.499301in}{4.922222in}%
\pgfsys@useobject{currentmarker}{}%
\end{pgfscope}%
\begin{pgfscope}%
\pgfsys@transformshift{2.572030in}{4.991779in}%
\pgfsys@useobject{currentmarker}{}%
\end{pgfscope}%
\begin{pgfscope}%
\pgfsys@transformshift{2.644760in}{5.052365in}%
\pgfsys@useobject{currentmarker}{}%
\end{pgfscope}%
\begin{pgfscope}%
\pgfsys@transformshift{2.717490in}{5.106049in}%
\pgfsys@useobject{currentmarker}{}%
\end{pgfscope}%
\begin{pgfscope}%
\pgfsys@transformshift{2.790220in}{5.154421in}%
\pgfsys@useobject{currentmarker}{}%
\end{pgfscope}%
\begin{pgfscope}%
\pgfsys@transformshift{2.862950in}{5.198107in}%
\pgfsys@useobject{currentmarker}{}%
\end{pgfscope}%
\begin{pgfscope}%
\pgfsys@transformshift{2.935680in}{5.237946in}%
\pgfsys@useobject{currentmarker}{}%
\end{pgfscope}%
\begin{pgfscope}%
\pgfsys@transformshift{3.008410in}{5.274638in}%
\pgfsys@useobject{currentmarker}{}%
\end{pgfscope}%
\begin{pgfscope}%
\pgfsys@transformshift{3.081139in}{5.308105in}%
\pgfsys@useobject{currentmarker}{}%
\end{pgfscope}%
\begin{pgfscope}%
\pgfsys@transformshift{3.153869in}{5.339226in}%
\pgfsys@useobject{currentmarker}{}%
\end{pgfscope}%
\begin{pgfscope}%
\pgfsys@transformshift{3.226599in}{5.367779in}%
\pgfsys@useobject{currentmarker}{}%
\end{pgfscope}%
\begin{pgfscope}%
\pgfsys@transformshift{3.299329in}{5.394556in}%
\pgfsys@useobject{currentmarker}{}%
\end{pgfscope}%
\begin{pgfscope}%
\pgfsys@transformshift{3.372059in}{5.419480in}%
\pgfsys@useobject{currentmarker}{}%
\end{pgfscope}%
\begin{pgfscope}%
\pgfsys@transformshift{3.444789in}{5.442882in}%
\pgfsys@useobject{currentmarker}{}%
\end{pgfscope}%
\begin{pgfscope}%
\pgfsys@transformshift{3.517518in}{5.464766in}%
\pgfsys@useobject{currentmarker}{}%
\end{pgfscope}%
\begin{pgfscope}%
\pgfsys@transformshift{3.590248in}{5.485308in}%
\pgfsys@useobject{currentmarker}{}%
\end{pgfscope}%
\begin{pgfscope}%
\pgfsys@transformshift{3.662978in}{5.504909in}%
\pgfsys@useobject{currentmarker}{}%
\end{pgfscope}%
\begin{pgfscope}%
\pgfsys@transformshift{3.735708in}{5.523240in}%
\pgfsys@useobject{currentmarker}{}%
\end{pgfscope}%
\end{pgfscope}%
\begin{pgfscope}%
\pgfsetbuttcap%
\pgfsetroundjoin%
\definecolor{currentfill}{rgb}{0.000000,0.000000,0.000000}%
\pgfsetfillcolor{currentfill}%
\pgfsetlinewidth{0.803000pt}%
\definecolor{currentstroke}{rgb}{0.000000,0.000000,0.000000}%
\pgfsetstrokecolor{currentstroke}%
\pgfsetdash{}{0pt}%
\pgfsys@defobject{currentmarker}{\pgfqpoint{0.000000in}{-0.048611in}}{\pgfqpoint{0.000000in}{0.000000in}}{%
\pgfpathmoveto{\pgfqpoint{0.000000in}{0.000000in}}%
\pgfpathlineto{\pgfqpoint{0.000000in}{-0.048611in}}%
\pgfusepath{stroke,fill}%
}%
\begin{pgfscope}%
\pgfsys@transformshift{1.190163in}{3.312222in}%
\pgfsys@useobject{currentmarker}{}%
\end{pgfscope}%
\end{pgfscope}%
\begin{pgfscope}%
\definecolor{textcolor}{rgb}{0.000000,0.000000,0.000000}%
\pgfsetstrokecolor{textcolor}%
\pgfsetfillcolor{textcolor}%
\pgftext[x=1.190163in,y=3.215000in,,top]{\color{textcolor}\sffamily\fontsize{10.000000}{12.000000}\selectfont 1}%
\end{pgfscope}%
\begin{pgfscope}%
\pgfsetbuttcap%
\pgfsetroundjoin%
\definecolor{currentfill}{rgb}{0.000000,0.000000,0.000000}%
\pgfsetfillcolor{currentfill}%
\pgfsetlinewidth{0.803000pt}%
\definecolor{currentstroke}{rgb}{0.000000,0.000000,0.000000}%
\pgfsetstrokecolor{currentstroke}%
\pgfsetdash{}{0pt}%
\pgfsys@defobject{currentmarker}{\pgfqpoint{0.000000in}{-0.048611in}}{\pgfqpoint{0.000000in}{0.000000in}}{%
\pgfpathmoveto{\pgfqpoint{0.000000in}{0.000000in}}%
\pgfpathlineto{\pgfqpoint{0.000000in}{-0.048611in}}%
\pgfusepath{stroke,fill}%
}%
\begin{pgfscope}%
\pgfsys@transformshift{1.917462in}{3.312222in}%
\pgfsys@useobject{currentmarker}{}%
\end{pgfscope}%
\end{pgfscope}%
\begin{pgfscope}%
\definecolor{textcolor}{rgb}{0.000000,0.000000,0.000000}%
\pgfsetstrokecolor{textcolor}%
\pgfsetfillcolor{textcolor}%
\pgftext[x=1.917462in,y=3.215000in,,top]{\color{textcolor}\sffamily\fontsize{10.000000}{12.000000}\selectfont 2}%
\end{pgfscope}%
\begin{pgfscope}%
\pgfsetbuttcap%
\pgfsetroundjoin%
\definecolor{currentfill}{rgb}{0.000000,0.000000,0.000000}%
\pgfsetfillcolor{currentfill}%
\pgfsetlinewidth{0.803000pt}%
\definecolor{currentstroke}{rgb}{0.000000,0.000000,0.000000}%
\pgfsetstrokecolor{currentstroke}%
\pgfsetdash{}{0pt}%
\pgfsys@defobject{currentmarker}{\pgfqpoint{0.000000in}{-0.048611in}}{\pgfqpoint{0.000000in}{0.000000in}}{%
\pgfpathmoveto{\pgfqpoint{0.000000in}{0.000000in}}%
\pgfpathlineto{\pgfqpoint{0.000000in}{-0.048611in}}%
\pgfusepath{stroke,fill}%
}%
\begin{pgfscope}%
\pgfsys@transformshift{2.644760in}{3.312222in}%
\pgfsys@useobject{currentmarker}{}%
\end{pgfscope}%
\end{pgfscope}%
\begin{pgfscope}%
\definecolor{textcolor}{rgb}{0.000000,0.000000,0.000000}%
\pgfsetstrokecolor{textcolor}%
\pgfsetfillcolor{textcolor}%
\pgftext[x=2.644760in,y=3.215000in,,top]{\color{textcolor}\sffamily\fontsize{10.000000}{12.000000}\selectfont 3}%
\end{pgfscope}%
\begin{pgfscope}%
\pgfsetbuttcap%
\pgfsetroundjoin%
\definecolor{currentfill}{rgb}{0.000000,0.000000,0.000000}%
\pgfsetfillcolor{currentfill}%
\pgfsetlinewidth{0.803000pt}%
\definecolor{currentstroke}{rgb}{0.000000,0.000000,0.000000}%
\pgfsetstrokecolor{currentstroke}%
\pgfsetdash{}{0pt}%
\pgfsys@defobject{currentmarker}{\pgfqpoint{0.000000in}{-0.048611in}}{\pgfqpoint{0.000000in}{0.000000in}}{%
\pgfpathmoveto{\pgfqpoint{0.000000in}{0.000000in}}%
\pgfpathlineto{\pgfqpoint{0.000000in}{-0.048611in}}%
\pgfusepath{stroke,fill}%
}%
\begin{pgfscope}%
\pgfsys@transformshift{3.372059in}{3.312222in}%
\pgfsys@useobject{currentmarker}{}%
\end{pgfscope}%
\end{pgfscope}%
\begin{pgfscope}%
\definecolor{textcolor}{rgb}{0.000000,0.000000,0.000000}%
\pgfsetstrokecolor{textcolor}%
\pgfsetfillcolor{textcolor}%
\pgftext[x=3.372059in,y=3.215000in,,top]{\color{textcolor}\sffamily\fontsize{10.000000}{12.000000}\selectfont 4}%
\end{pgfscope}%
\begin{pgfscope}%
\pgfsetbuttcap%
\pgfsetroundjoin%
\definecolor{currentfill}{rgb}{0.000000,0.000000,0.000000}%
\pgfsetfillcolor{currentfill}%
\pgfsetlinewidth{0.803000pt}%
\definecolor{currentstroke}{rgb}{0.000000,0.000000,0.000000}%
\pgfsetstrokecolor{currentstroke}%
\pgfsetdash{}{0pt}%
\pgfsys@defobject{currentmarker}{\pgfqpoint{-0.048611in}{0.000000in}}{\pgfqpoint{0.000000in}{0.000000in}}{%
\pgfpathmoveto{\pgfqpoint{0.000000in}{0.000000in}}%
\pgfpathlineto{\pgfqpoint{-0.048611in}{0.000000in}}%
\pgfusepath{stroke,fill}%
}%
\begin{pgfscope}%
\pgfsys@transformshift{0.672222in}{3.385122in}%
\pgfsys@useobject{currentmarker}{}%
\end{pgfscope}%
\end{pgfscope}%
\begin{pgfscope}%
\definecolor{textcolor}{rgb}{0.000000,0.000000,0.000000}%
\pgfsetstrokecolor{textcolor}%
\pgfsetfillcolor{textcolor}%
\pgftext[x=0.149382in,y=3.332360in,left,base]{\color{textcolor}\sffamily\fontsize{10.000000}{12.000000}\selectfont −2.00}%
\end{pgfscope}%
\begin{pgfscope}%
\pgfsetbuttcap%
\pgfsetroundjoin%
\definecolor{currentfill}{rgb}{0.000000,0.000000,0.000000}%
\pgfsetfillcolor{currentfill}%
\pgfsetlinewidth{0.803000pt}%
\definecolor{currentstroke}{rgb}{0.000000,0.000000,0.000000}%
\pgfsetstrokecolor{currentstroke}%
\pgfsetdash{}{0pt}%
\pgfsys@defobject{currentmarker}{\pgfqpoint{-0.048611in}{0.000000in}}{\pgfqpoint{0.000000in}{0.000000in}}{%
\pgfpathmoveto{\pgfqpoint{0.000000in}{0.000000in}}%
\pgfpathlineto{\pgfqpoint{-0.048611in}{0.000000in}}%
\pgfusepath{stroke,fill}%
}%
\begin{pgfscope}%
\pgfsys@transformshift{0.672222in}{3.737637in}%
\pgfsys@useobject{currentmarker}{}%
\end{pgfscope}%
\end{pgfscope}%
\begin{pgfscope}%
\definecolor{textcolor}{rgb}{0.000000,0.000000,0.000000}%
\pgfsetstrokecolor{textcolor}%
\pgfsetfillcolor{textcolor}%
\pgftext[x=0.149382in,y=3.684875in,left,base]{\color{textcolor}\sffamily\fontsize{10.000000}{12.000000}\selectfont −1.75}%
\end{pgfscope}%
\begin{pgfscope}%
\pgfsetbuttcap%
\pgfsetroundjoin%
\definecolor{currentfill}{rgb}{0.000000,0.000000,0.000000}%
\pgfsetfillcolor{currentfill}%
\pgfsetlinewidth{0.803000pt}%
\definecolor{currentstroke}{rgb}{0.000000,0.000000,0.000000}%
\pgfsetstrokecolor{currentstroke}%
\pgfsetdash{}{0pt}%
\pgfsys@defobject{currentmarker}{\pgfqpoint{-0.048611in}{0.000000in}}{\pgfqpoint{0.000000in}{0.000000in}}{%
\pgfpathmoveto{\pgfqpoint{0.000000in}{0.000000in}}%
\pgfpathlineto{\pgfqpoint{-0.048611in}{0.000000in}}%
\pgfusepath{stroke,fill}%
}%
\begin{pgfscope}%
\pgfsys@transformshift{0.672222in}{4.090151in}%
\pgfsys@useobject{currentmarker}{}%
\end{pgfscope}%
\end{pgfscope}%
\begin{pgfscope}%
\definecolor{textcolor}{rgb}{0.000000,0.000000,0.000000}%
\pgfsetstrokecolor{textcolor}%
\pgfsetfillcolor{textcolor}%
\pgftext[x=0.149382in,y=4.037390in,left,base]{\color{textcolor}\sffamily\fontsize{10.000000}{12.000000}\selectfont −1.50}%
\end{pgfscope}%
\begin{pgfscope}%
\pgfsetbuttcap%
\pgfsetroundjoin%
\definecolor{currentfill}{rgb}{0.000000,0.000000,0.000000}%
\pgfsetfillcolor{currentfill}%
\pgfsetlinewidth{0.803000pt}%
\definecolor{currentstroke}{rgb}{0.000000,0.000000,0.000000}%
\pgfsetstrokecolor{currentstroke}%
\pgfsetdash{}{0pt}%
\pgfsys@defobject{currentmarker}{\pgfqpoint{-0.048611in}{0.000000in}}{\pgfqpoint{0.000000in}{0.000000in}}{%
\pgfpathmoveto{\pgfqpoint{0.000000in}{0.000000in}}%
\pgfpathlineto{\pgfqpoint{-0.048611in}{0.000000in}}%
\pgfusepath{stroke,fill}%
}%
\begin{pgfscope}%
\pgfsys@transformshift{0.672222in}{4.442666in}%
\pgfsys@useobject{currentmarker}{}%
\end{pgfscope}%
\end{pgfscope}%
\begin{pgfscope}%
\definecolor{textcolor}{rgb}{0.000000,0.000000,0.000000}%
\pgfsetstrokecolor{textcolor}%
\pgfsetfillcolor{textcolor}%
\pgftext[x=0.149382in,y=4.389905in,left,base]{\color{textcolor}\sffamily\fontsize{10.000000}{12.000000}\selectfont −1.25}%
\end{pgfscope}%
\begin{pgfscope}%
\pgfsetbuttcap%
\pgfsetroundjoin%
\definecolor{currentfill}{rgb}{0.000000,0.000000,0.000000}%
\pgfsetfillcolor{currentfill}%
\pgfsetlinewidth{0.803000pt}%
\definecolor{currentstroke}{rgb}{0.000000,0.000000,0.000000}%
\pgfsetstrokecolor{currentstroke}%
\pgfsetdash{}{0pt}%
\pgfsys@defobject{currentmarker}{\pgfqpoint{-0.048611in}{0.000000in}}{\pgfqpoint{0.000000in}{0.000000in}}{%
\pgfpathmoveto{\pgfqpoint{0.000000in}{0.000000in}}%
\pgfpathlineto{\pgfqpoint{-0.048611in}{0.000000in}}%
\pgfusepath{stroke,fill}%
}%
\begin{pgfscope}%
\pgfsys@transformshift{0.672222in}{4.795181in}%
\pgfsys@useobject{currentmarker}{}%
\end{pgfscope}%
\end{pgfscope}%
\begin{pgfscope}%
\definecolor{textcolor}{rgb}{0.000000,0.000000,0.000000}%
\pgfsetstrokecolor{textcolor}%
\pgfsetfillcolor{textcolor}%
\pgftext[x=0.149382in,y=4.742419in,left,base]{\color{textcolor}\sffamily\fontsize{10.000000}{12.000000}\selectfont −1.00}%
\end{pgfscope}%
\begin{pgfscope}%
\pgfsetbuttcap%
\pgfsetroundjoin%
\definecolor{currentfill}{rgb}{0.000000,0.000000,0.000000}%
\pgfsetfillcolor{currentfill}%
\pgfsetlinewidth{0.803000pt}%
\definecolor{currentstroke}{rgb}{0.000000,0.000000,0.000000}%
\pgfsetstrokecolor{currentstroke}%
\pgfsetdash{}{0pt}%
\pgfsys@defobject{currentmarker}{\pgfqpoint{-0.048611in}{0.000000in}}{\pgfqpoint{0.000000in}{0.000000in}}{%
\pgfpathmoveto{\pgfqpoint{0.000000in}{0.000000in}}%
\pgfpathlineto{\pgfqpoint{-0.048611in}{0.000000in}}%
\pgfusepath{stroke,fill}%
}%
\begin{pgfscope}%
\pgfsys@transformshift{0.672222in}{5.147696in}%
\pgfsys@useobject{currentmarker}{}%
\end{pgfscope}%
\end{pgfscope}%
\begin{pgfscope}%
\definecolor{textcolor}{rgb}{0.000000,0.000000,0.000000}%
\pgfsetstrokecolor{textcolor}%
\pgfsetfillcolor{textcolor}%
\pgftext[x=0.149382in,y=5.094934in,left,base]{\color{textcolor}\sffamily\fontsize{10.000000}{12.000000}\selectfont −0.75}%
\end{pgfscope}%
\begin{pgfscope}%
\pgfsetbuttcap%
\pgfsetroundjoin%
\definecolor{currentfill}{rgb}{0.000000,0.000000,0.000000}%
\pgfsetfillcolor{currentfill}%
\pgfsetlinewidth{0.803000pt}%
\definecolor{currentstroke}{rgb}{0.000000,0.000000,0.000000}%
\pgfsetstrokecolor{currentstroke}%
\pgfsetdash{}{0pt}%
\pgfsys@defobject{currentmarker}{\pgfqpoint{-0.048611in}{0.000000in}}{\pgfqpoint{0.000000in}{0.000000in}}{%
\pgfpathmoveto{\pgfqpoint{0.000000in}{0.000000in}}%
\pgfpathlineto{\pgfqpoint{-0.048611in}{0.000000in}}%
\pgfusepath{stroke,fill}%
}%
\begin{pgfscope}%
\pgfsys@transformshift{0.672222in}{5.500210in}%
\pgfsys@useobject{currentmarker}{}%
\end{pgfscope}%
\end{pgfscope}%
\begin{pgfscope}%
\definecolor{textcolor}{rgb}{0.000000,0.000000,0.000000}%
\pgfsetstrokecolor{textcolor}%
\pgfsetfillcolor{textcolor}%
\pgftext[x=0.149382in,y=5.447449in,left,base]{\color{textcolor}\sffamily\fontsize{10.000000}{12.000000}\selectfont −0.50}%
\end{pgfscope}%
\begin{pgfscope}%
\pgfpathrectangle{\pgfqpoint{0.672222in}{3.312222in}}{\pgfqpoint{3.217778in}{2.324444in}}%
\pgfusepath{clip}%
\pgfsetrectcap%
\pgfsetroundjoin%
\pgfsetlinewidth{1.505625pt}%
\definecolor{currentstroke}{rgb}{0.121569,0.466667,0.705882}%
\pgfsetstrokecolor{currentstroke}%
\pgfsetdash{}{0pt}%
\pgfpathmoveto{\pgfqpoint{0.826514in}{3.429185in}}%
\pgfpathlineto{\pgfqpoint{0.899244in}{3.429168in}}%
\pgfpathlineto{\pgfqpoint{0.971974in}{3.429067in}}%
\pgfpathlineto{\pgfqpoint{1.044704in}{3.428711in}}%
\pgfpathlineto{\pgfqpoint{1.117434in}{3.427887in}}%
\pgfpathlineto{\pgfqpoint{1.190163in}{3.426622in}}%
\pgfpathlineto{\pgfqpoint{1.262893in}{3.425649in}}%
\pgfpathlineto{\pgfqpoint{1.335623in}{3.426638in}}%
\pgfpathlineto{\pgfqpoint{1.408353in}{3.431848in}}%
\pgfpathlineto{\pgfqpoint{1.481083in}{3.443451in}}%
\pgfpathlineto{\pgfqpoint{1.553813in}{3.463181in}}%
\pgfpathlineto{\pgfqpoint{1.626543in}{3.492486in}}%
\pgfpathlineto{\pgfqpoint{1.699272in}{3.532905in}}%
\pgfpathlineto{\pgfqpoint{1.772002in}{3.586441in}}%
\pgfpathlineto{\pgfqpoint{1.844732in}{3.655686in}}%
\pgfpathlineto{\pgfqpoint{1.917462in}{3.745239in}}%
\pgfpathlineto{\pgfqpoint{1.990192in}{3.862691in}}%
\pgfpathlineto{\pgfqpoint{2.062922in}{4.020410in}}%
\pgfpathlineto{\pgfqpoint{2.135651in}{4.223289in}}%
\pgfpathlineto{\pgfqpoint{2.208381in}{4.435230in}}%
\pgfpathlineto{\pgfqpoint{2.281111in}{4.610149in}}%
\pgfpathlineto{\pgfqpoint{2.353841in}{4.740380in}}%
\pgfpathlineto{\pgfqpoint{2.426571in}{4.840164in}}%
\pgfpathlineto{\pgfqpoint{2.499301in}{4.922222in}}%
\pgfpathlineto{\pgfqpoint{2.572030in}{4.991779in}}%
\pgfpathlineto{\pgfqpoint{2.644760in}{5.052365in}}%
\pgfpathlineto{\pgfqpoint{2.717490in}{5.106049in}}%
\pgfpathlineto{\pgfqpoint{2.790220in}{5.154421in}}%
\pgfpathlineto{\pgfqpoint{2.862950in}{5.198107in}}%
\pgfpathlineto{\pgfqpoint{2.935680in}{5.237946in}}%
\pgfpathlineto{\pgfqpoint{3.008410in}{5.274638in}}%
\pgfpathlineto{\pgfqpoint{3.081139in}{5.308105in}}%
\pgfpathlineto{\pgfqpoint{3.153869in}{5.339226in}}%
\pgfpathlineto{\pgfqpoint{3.226599in}{5.367779in}}%
\pgfpathlineto{\pgfqpoint{3.299329in}{5.394556in}}%
\pgfpathlineto{\pgfqpoint{3.372059in}{5.419480in}}%
\pgfpathlineto{\pgfqpoint{3.444789in}{5.442882in}}%
\pgfpathlineto{\pgfqpoint{3.517518in}{5.464766in}}%
\pgfpathlineto{\pgfqpoint{3.590248in}{5.485308in}}%
\pgfpathlineto{\pgfqpoint{3.662978in}{5.504909in}}%
\pgfpathlineto{\pgfqpoint{3.735708in}{5.523240in}}%
\pgfusepath{stroke}%
\end{pgfscope}%
\begin{pgfscope}%
\pgfsetrectcap%
\pgfsetmiterjoin%
\pgfsetlinewidth{0.803000pt}%
\definecolor{currentstroke}{rgb}{0.000000,0.000000,0.000000}%
\pgfsetstrokecolor{currentstroke}%
\pgfsetdash{}{0pt}%
\pgfpathmoveto{\pgfqpoint{0.672222in}{3.312222in}}%
\pgfpathlineto{\pgfqpoint{0.672222in}{5.636667in}}%
\pgfusepath{stroke}%
\end{pgfscope}%
\begin{pgfscope}%
\pgfsetrectcap%
\pgfsetmiterjoin%
\pgfsetlinewidth{0.803000pt}%
\definecolor{currentstroke}{rgb}{0.000000,0.000000,0.000000}%
\pgfsetstrokecolor{currentstroke}%
\pgfsetdash{}{0pt}%
\pgfpathmoveto{\pgfqpoint{3.890000in}{3.312222in}}%
\pgfpathlineto{\pgfqpoint{3.890000in}{5.636667in}}%
\pgfusepath{stroke}%
\end{pgfscope}%
\begin{pgfscope}%
\pgfsetrectcap%
\pgfsetmiterjoin%
\pgfsetlinewidth{0.803000pt}%
\definecolor{currentstroke}{rgb}{0.000000,0.000000,0.000000}%
\pgfsetstrokecolor{currentstroke}%
\pgfsetdash{}{0pt}%
\pgfpathmoveto{\pgfqpoint{0.672222in}{3.312222in}}%
\pgfpathlineto{\pgfqpoint{3.890000in}{3.312222in}}%
\pgfusepath{stroke}%
\end{pgfscope}%
\begin{pgfscope}%
\pgfsetrectcap%
\pgfsetmiterjoin%
\pgfsetlinewidth{0.803000pt}%
\definecolor{currentstroke}{rgb}{0.000000,0.000000,0.000000}%
\pgfsetstrokecolor{currentstroke}%
\pgfsetdash{}{0pt}%
\pgfpathmoveto{\pgfqpoint{0.672222in}{5.636667in}}%
\pgfpathlineto{\pgfqpoint{3.890000in}{5.636667in}}%
\pgfusepath{stroke}%
\end{pgfscope}%
\begin{pgfscope}%
\definecolor{textcolor}{rgb}{0.000000,0.000000,0.000000}%
\pgfsetstrokecolor{textcolor}%
\pgfsetfillcolor{textcolor}%
\pgftext[x=2.281111in,y=5.720000in,,base]{\color{textcolor}\sffamily\fontsize{12.000000}{14.400000}\selectfont \(\displaystyle  N = 16 \)}%
\end{pgfscope}%
\begin{pgfscope}%
\pgfsetbuttcap%
\pgfsetmiterjoin%
\definecolor{currentfill}{rgb}{1.000000,1.000000,1.000000}%
\pgfsetfillcolor{currentfill}%
\pgfsetlinewidth{0.000000pt}%
\definecolor{currentstroke}{rgb}{0.000000,0.000000,0.000000}%
\pgfsetstrokecolor{currentstroke}%
\pgfsetstrokeopacity{0.000000}%
\pgfsetdash{}{0pt}%
\pgfpathmoveto{\pgfqpoint{4.597222in}{3.312222in}}%
\pgfpathlineto{\pgfqpoint{7.815000in}{3.312222in}}%
\pgfpathlineto{\pgfqpoint{7.815000in}{5.636667in}}%
\pgfpathlineto{\pgfqpoint{4.597222in}{5.636667in}}%
\pgfpathclose%
\pgfusepath{fill}%
\end{pgfscope}%
\begin{pgfscope}%
\pgfpathrectangle{\pgfqpoint{4.597222in}{3.312222in}}{\pgfqpoint{3.217778in}{2.324444in}}%
\pgfusepath{clip}%
\pgfsetbuttcap%
\pgfsetroundjoin%
\definecolor{currentfill}{rgb}{0.121569,0.466667,0.705882}%
\pgfsetfillcolor{currentfill}%
\pgfsetfillopacity{0.300000}%
\pgfsetlinewidth{1.003750pt}%
\definecolor{currentstroke}{rgb}{0.121569,0.466667,0.705882}%
\pgfsetstrokecolor{currentstroke}%
\pgfsetstrokeopacity{0.300000}%
\pgfsetdash{}{0pt}%
\pgfpathmoveto{\pgfqpoint{4.751514in}{3.426548in}}%
\pgfpathlineto{\pgfqpoint{4.751514in}{3.426548in}}%
\pgfpathlineto{\pgfqpoint{4.751514in}{3.426548in}}%
\pgfpathlineto{\pgfqpoint{4.751514in}{3.426548in}}%
\pgfpathclose%
\pgfusepath{stroke,fill}%
\end{pgfscope}%
\begin{pgfscope}%
\pgfpathrectangle{\pgfqpoint{4.597222in}{3.312222in}}{\pgfqpoint{3.217778in}{2.324444in}}%
\pgfusepath{clip}%
\pgfsetbuttcap%
\pgfsetroundjoin%
\definecolor{currentfill}{rgb}{0.121569,0.466667,0.705882}%
\pgfsetfillcolor{currentfill}%
\pgfsetfillopacity{0.300000}%
\pgfsetlinewidth{1.003750pt}%
\definecolor{currentstroke}{rgb}{0.121569,0.466667,0.705882}%
\pgfsetstrokecolor{currentstroke}%
\pgfsetstrokeopacity{0.300000}%
\pgfsetdash{}{0pt}%
\pgfpathmoveto{\pgfqpoint{4.896974in}{3.426435in}}%
\pgfpathlineto{\pgfqpoint{4.896974in}{3.426435in}}%
\pgfpathlineto{\pgfqpoint{4.969704in}{3.426129in}}%
\pgfpathlineto{\pgfqpoint{5.042434in}{3.425648in}}%
\pgfpathlineto{\pgfqpoint{5.115163in}{3.425725in}}%
\pgfpathlineto{\pgfqpoint{5.187893in}{3.427905in}}%
\pgfpathlineto{\pgfqpoint{5.260623in}{3.433760in}}%
\pgfpathlineto{\pgfqpoint{5.333353in}{3.444402in}}%
\pgfpathlineto{\pgfqpoint{5.406083in}{3.460793in}}%
\pgfpathlineto{\pgfqpoint{5.478813in}{3.484103in}}%
\pgfpathlineto{\pgfqpoint{5.551543in}{3.515690in}}%
\pgfpathlineto{\pgfqpoint{5.624272in}{3.557389in}}%
\pgfpathlineto{\pgfqpoint{5.697002in}{3.611319in}}%
\pgfpathlineto{\pgfqpoint{5.769732in}{3.680568in}}%
\pgfpathlineto{\pgfqpoint{5.842462in}{3.769202in}}%
\pgfpathlineto{\pgfqpoint{5.915192in}{3.884830in}}%
\pgfpathlineto{\pgfqpoint{5.987922in}{4.045149in}}%
\pgfpathlineto{\pgfqpoint{6.060651in}{4.281892in}}%
\pgfpathlineto{\pgfqpoint{6.133381in}{4.507197in}}%
\pgfpathlineto{\pgfqpoint{6.206111in}{4.655512in}}%
\pgfpathlineto{\pgfqpoint{6.278841in}{4.765924in}}%
\pgfpathlineto{\pgfqpoint{6.351571in}{4.855746in}}%
\pgfpathlineto{\pgfqpoint{6.424301in}{4.932892in}}%
\pgfpathlineto{\pgfqpoint{6.497030in}{5.000064in}}%
\pgfpathlineto{\pgfqpoint{6.569760in}{5.059453in}}%
\pgfpathlineto{\pgfqpoint{6.642490in}{5.112387in}}%
\pgfpathlineto{\pgfqpoint{6.715220in}{5.159788in}}%
\pgfpathlineto{\pgfqpoint{6.787950in}{5.202307in}}%
\pgfpathlineto{\pgfqpoint{6.860680in}{5.242369in}}%
\pgfpathlineto{\pgfqpoint{6.933410in}{5.278177in}}%
\pgfpathlineto{\pgfqpoint{7.006139in}{5.311071in}}%
\pgfpathlineto{\pgfqpoint{7.078869in}{5.341579in}}%
\pgfpathlineto{\pgfqpoint{7.151599in}{5.369900in}}%
\pgfpathlineto{\pgfqpoint{7.224329in}{5.396366in}}%
\pgfpathlineto{\pgfqpoint{7.297059in}{5.420740in}}%
\pgfpathlineto{\pgfqpoint{7.369789in}{5.444023in}}%
\pgfpathlineto{\pgfqpoint{7.442518in}{5.465692in}}%
\pgfpathlineto{\pgfqpoint{7.515248in}{5.486060in}}%
\pgfpathlineto{\pgfqpoint{7.587978in}{5.505152in}}%
\pgfpathlineto{\pgfqpoint{7.660708in}{5.523105in}}%
\pgfpathlineto{\pgfqpoint{7.660708in}{5.523375in}}%
\pgfpathlineto{\pgfqpoint{7.660708in}{5.523375in}}%
\pgfpathlineto{\pgfqpoint{7.587978in}{5.505348in}}%
\pgfpathlineto{\pgfqpoint{7.515248in}{5.486378in}}%
\pgfpathlineto{\pgfqpoint{7.442518in}{5.466065in}}%
\pgfpathlineto{\pgfqpoint{7.369789in}{5.444420in}}%
\pgfpathlineto{\pgfqpoint{7.297059in}{5.421238in}}%
\pgfpathlineto{\pgfqpoint{7.224329in}{5.396815in}}%
\pgfpathlineto{\pgfqpoint{7.151599in}{5.370365in}}%
\pgfpathlineto{\pgfqpoint{7.078869in}{5.341849in}}%
\pgfpathlineto{\pgfqpoint{7.006139in}{5.311299in}}%
\pgfpathlineto{\pgfqpoint{6.933410in}{5.278523in}}%
\pgfpathlineto{\pgfqpoint{6.860680in}{5.242704in}}%
\pgfpathlineto{\pgfqpoint{6.787950in}{5.203466in}}%
\pgfpathlineto{\pgfqpoint{6.715220in}{5.160480in}}%
\pgfpathlineto{\pgfqpoint{6.642490in}{5.112543in}}%
\pgfpathlineto{\pgfqpoint{6.569760in}{5.059804in}}%
\pgfpathlineto{\pgfqpoint{6.497030in}{5.001324in}}%
\pgfpathlineto{\pgfqpoint{6.424301in}{4.933728in}}%
\pgfpathlineto{\pgfqpoint{6.351571in}{4.856684in}}%
\pgfpathlineto{\pgfqpoint{6.278841in}{4.766707in}}%
\pgfpathlineto{\pgfqpoint{6.206111in}{4.658451in}}%
\pgfpathlineto{\pgfqpoint{6.133381in}{4.512621in}}%
\pgfpathlineto{\pgfqpoint{6.060651in}{4.285134in}}%
\pgfpathlineto{\pgfqpoint{5.987922in}{4.046151in}}%
\pgfpathlineto{\pgfqpoint{5.915192in}{3.885875in}}%
\pgfpathlineto{\pgfqpoint{5.842462in}{3.769789in}}%
\pgfpathlineto{\pgfqpoint{5.769732in}{3.680743in}}%
\pgfpathlineto{\pgfqpoint{5.697002in}{3.611489in}}%
\pgfpathlineto{\pgfqpoint{5.624272in}{3.557500in}}%
\pgfpathlineto{\pgfqpoint{5.551543in}{3.515785in}}%
\pgfpathlineto{\pgfqpoint{5.478813in}{3.484144in}}%
\pgfpathlineto{\pgfqpoint{5.406083in}{3.460812in}}%
\pgfpathlineto{\pgfqpoint{5.333353in}{3.444420in}}%
\pgfpathlineto{\pgfqpoint{5.260623in}{3.433763in}}%
\pgfpathlineto{\pgfqpoint{5.187893in}{3.427907in}}%
\pgfpathlineto{\pgfqpoint{5.115163in}{3.425726in}}%
\pgfpathlineto{\pgfqpoint{5.042434in}{3.425649in}}%
\pgfpathlineto{\pgfqpoint{4.969704in}{3.426129in}}%
\pgfpathlineto{\pgfqpoint{4.896974in}{3.426435in}}%
\pgfpathclose%
\pgfusepath{stroke,fill}%
\end{pgfscope}%
\begin{pgfscope}%
\pgfpathrectangle{\pgfqpoint{4.597222in}{3.312222in}}{\pgfqpoint{3.217778in}{2.324444in}}%
\pgfusepath{clip}%
\pgfsetbuttcap%
\pgfsetroundjoin%
\definecolor{currentfill}{rgb}{0.121569,0.466667,0.705882}%
\pgfsetfillcolor{currentfill}%
\pgfsetlinewidth{1.003750pt}%
\definecolor{currentstroke}{rgb}{0.121569,0.466667,0.705882}%
\pgfsetstrokecolor{currentstroke}%
\pgfsetdash{}{0pt}%
\pgfsys@defobject{currentmarker}{\pgfqpoint{-0.009821in}{-0.009821in}}{\pgfqpoint{0.009821in}{0.009821in}}{%
\pgfpathmoveto{\pgfqpoint{0.000000in}{-0.009821in}}%
\pgfpathcurveto{\pgfqpoint{0.002605in}{-0.009821in}}{\pgfqpoint{0.005103in}{-0.008786in}}{\pgfqpoint{0.006944in}{-0.006944in}}%
\pgfpathcurveto{\pgfqpoint{0.008786in}{-0.005103in}}{\pgfqpoint{0.009821in}{-0.002605in}}{\pgfqpoint{0.009821in}{0.000000in}}%
\pgfpathcurveto{\pgfqpoint{0.009821in}{0.002605in}}{\pgfqpoint{0.008786in}{0.005103in}}{\pgfqpoint{0.006944in}{0.006944in}}%
\pgfpathcurveto{\pgfqpoint{0.005103in}{0.008786in}}{\pgfqpoint{0.002605in}{0.009821in}}{\pgfqpoint{0.000000in}{0.009821in}}%
\pgfpathcurveto{\pgfqpoint{-0.002605in}{0.009821in}}{\pgfqpoint{-0.005103in}{0.008786in}}{\pgfqpoint{-0.006944in}{0.006944in}}%
\pgfpathcurveto{\pgfqpoint{-0.008786in}{0.005103in}}{\pgfqpoint{-0.009821in}{0.002605in}}{\pgfqpoint{-0.009821in}{0.000000in}}%
\pgfpathcurveto{\pgfqpoint{-0.009821in}{-0.002605in}}{\pgfqpoint{-0.008786in}{-0.005103in}}{\pgfqpoint{-0.006944in}{-0.006944in}}%
\pgfpathcurveto{\pgfqpoint{-0.005103in}{-0.008786in}}{\pgfqpoint{-0.002605in}{-0.009821in}}{\pgfqpoint{0.000000in}{-0.009821in}}%
\pgfpathclose%
\pgfusepath{stroke,fill}%
}%
\begin{pgfscope}%
\pgfsys@transformshift{4.751514in}{3.426548in}%
\pgfsys@useobject{currentmarker}{}%
\end{pgfscope}%
\begin{pgfscope}%
\pgfsys@transformshift{4.824244in}{3.426532in}%
\pgfsys@useobject{currentmarker}{}%
\end{pgfscope}%
\begin{pgfscope}%
\pgfsys@transformshift{4.896974in}{3.426435in}%
\pgfsys@useobject{currentmarker}{}%
\end{pgfscope}%
\begin{pgfscope}%
\pgfsys@transformshift{4.969704in}{3.426129in}%
\pgfsys@useobject{currentmarker}{}%
\end{pgfscope}%
\begin{pgfscope}%
\pgfsys@transformshift{5.042434in}{3.425649in}%
\pgfsys@useobject{currentmarker}{}%
\end{pgfscope}%
\begin{pgfscope}%
\pgfsys@transformshift{5.115163in}{3.425725in}%
\pgfsys@useobject{currentmarker}{}%
\end{pgfscope}%
\begin{pgfscope}%
\pgfsys@transformshift{5.187893in}{3.427906in}%
\pgfsys@useobject{currentmarker}{}%
\end{pgfscope}%
\begin{pgfscope}%
\pgfsys@transformshift{5.260623in}{3.433762in}%
\pgfsys@useobject{currentmarker}{}%
\end{pgfscope}%
\begin{pgfscope}%
\pgfsys@transformshift{5.333353in}{3.444411in}%
\pgfsys@useobject{currentmarker}{}%
\end{pgfscope}%
\begin{pgfscope}%
\pgfsys@transformshift{5.406083in}{3.460803in}%
\pgfsys@useobject{currentmarker}{}%
\end{pgfscope}%
\begin{pgfscope}%
\pgfsys@transformshift{5.478813in}{3.484123in}%
\pgfsys@useobject{currentmarker}{}%
\end{pgfscope}%
\begin{pgfscope}%
\pgfsys@transformshift{5.551543in}{3.515737in}%
\pgfsys@useobject{currentmarker}{}%
\end{pgfscope}%
\begin{pgfscope}%
\pgfsys@transformshift{5.624272in}{3.557444in}%
\pgfsys@useobject{currentmarker}{}%
\end{pgfscope}%
\begin{pgfscope}%
\pgfsys@transformshift{5.697002in}{3.611404in}%
\pgfsys@useobject{currentmarker}{}%
\end{pgfscope}%
\begin{pgfscope}%
\pgfsys@transformshift{5.769732in}{3.680655in}%
\pgfsys@useobject{currentmarker}{}%
\end{pgfscope}%
\begin{pgfscope}%
\pgfsys@transformshift{5.842462in}{3.769495in}%
\pgfsys@useobject{currentmarker}{}%
\end{pgfscope}%
\begin{pgfscope}%
\pgfsys@transformshift{5.915192in}{3.885353in}%
\pgfsys@useobject{currentmarker}{}%
\end{pgfscope}%
\begin{pgfscope}%
\pgfsys@transformshift{5.987922in}{4.045650in}%
\pgfsys@useobject{currentmarker}{}%
\end{pgfscope}%
\begin{pgfscope}%
\pgfsys@transformshift{6.060651in}{4.283513in}%
\pgfsys@useobject{currentmarker}{}%
\end{pgfscope}%
\begin{pgfscope}%
\pgfsys@transformshift{6.133381in}{4.509909in}%
\pgfsys@useobject{currentmarker}{}%
\end{pgfscope}%
\begin{pgfscope}%
\pgfsys@transformshift{6.206111in}{4.656982in}%
\pgfsys@useobject{currentmarker}{}%
\end{pgfscope}%
\begin{pgfscope}%
\pgfsys@transformshift{6.278841in}{4.766316in}%
\pgfsys@useobject{currentmarker}{}%
\end{pgfscope}%
\begin{pgfscope}%
\pgfsys@transformshift{6.351571in}{4.856215in}%
\pgfsys@useobject{currentmarker}{}%
\end{pgfscope}%
\begin{pgfscope}%
\pgfsys@transformshift{6.424301in}{4.933310in}%
\pgfsys@useobject{currentmarker}{}%
\end{pgfscope}%
\begin{pgfscope}%
\pgfsys@transformshift{6.497030in}{5.000694in}%
\pgfsys@useobject{currentmarker}{}%
\end{pgfscope}%
\begin{pgfscope}%
\pgfsys@transformshift{6.569760in}{5.059628in}%
\pgfsys@useobject{currentmarker}{}%
\end{pgfscope}%
\begin{pgfscope}%
\pgfsys@transformshift{6.642490in}{5.112465in}%
\pgfsys@useobject{currentmarker}{}%
\end{pgfscope}%
\begin{pgfscope}%
\pgfsys@transformshift{6.715220in}{5.160134in}%
\pgfsys@useobject{currentmarker}{}%
\end{pgfscope}%
\begin{pgfscope}%
\pgfsys@transformshift{6.787950in}{5.202886in}%
\pgfsys@useobject{currentmarker}{}%
\end{pgfscope}%
\begin{pgfscope}%
\pgfsys@transformshift{6.860680in}{5.242536in}%
\pgfsys@useobject{currentmarker}{}%
\end{pgfscope}%
\begin{pgfscope}%
\pgfsys@transformshift{6.933410in}{5.278350in}%
\pgfsys@useobject{currentmarker}{}%
\end{pgfscope}%
\begin{pgfscope}%
\pgfsys@transformshift{7.006139in}{5.311185in}%
\pgfsys@useobject{currentmarker}{}%
\end{pgfscope}%
\begin{pgfscope}%
\pgfsys@transformshift{7.078869in}{5.341714in}%
\pgfsys@useobject{currentmarker}{}%
\end{pgfscope}%
\begin{pgfscope}%
\pgfsys@transformshift{7.151599in}{5.370132in}%
\pgfsys@useobject{currentmarker}{}%
\end{pgfscope}%
\begin{pgfscope}%
\pgfsys@transformshift{7.224329in}{5.396590in}%
\pgfsys@useobject{currentmarker}{}%
\end{pgfscope}%
\begin{pgfscope}%
\pgfsys@transformshift{7.297059in}{5.420989in}%
\pgfsys@useobject{currentmarker}{}%
\end{pgfscope}%
\begin{pgfscope}%
\pgfsys@transformshift{7.369789in}{5.444222in}%
\pgfsys@useobject{currentmarker}{}%
\end{pgfscope}%
\begin{pgfscope}%
\pgfsys@transformshift{7.442518in}{5.465878in}%
\pgfsys@useobject{currentmarker}{}%
\end{pgfscope}%
\begin{pgfscope}%
\pgfsys@transformshift{7.515248in}{5.486219in}%
\pgfsys@useobject{currentmarker}{}%
\end{pgfscope}%
\begin{pgfscope}%
\pgfsys@transformshift{7.587978in}{5.505250in}%
\pgfsys@useobject{currentmarker}{}%
\end{pgfscope}%
\begin{pgfscope}%
\pgfsys@transformshift{7.660708in}{5.523240in}%
\pgfsys@useobject{currentmarker}{}%
\end{pgfscope}%
\end{pgfscope}%
\begin{pgfscope}%
\pgfsetbuttcap%
\pgfsetroundjoin%
\definecolor{currentfill}{rgb}{0.000000,0.000000,0.000000}%
\pgfsetfillcolor{currentfill}%
\pgfsetlinewidth{0.803000pt}%
\definecolor{currentstroke}{rgb}{0.000000,0.000000,0.000000}%
\pgfsetstrokecolor{currentstroke}%
\pgfsetdash{}{0pt}%
\pgfsys@defobject{currentmarker}{\pgfqpoint{0.000000in}{-0.048611in}}{\pgfqpoint{0.000000in}{0.000000in}}{%
\pgfpathmoveto{\pgfqpoint{0.000000in}{0.000000in}}%
\pgfpathlineto{\pgfqpoint{0.000000in}{-0.048611in}}%
\pgfusepath{stroke,fill}%
}%
\begin{pgfscope}%
\pgfsys@transformshift{5.115163in}{3.312222in}%
\pgfsys@useobject{currentmarker}{}%
\end{pgfscope}%
\end{pgfscope}%
\begin{pgfscope}%
\definecolor{textcolor}{rgb}{0.000000,0.000000,0.000000}%
\pgfsetstrokecolor{textcolor}%
\pgfsetfillcolor{textcolor}%
\pgftext[x=5.115163in,y=3.215000in,,top]{\color{textcolor}\sffamily\fontsize{10.000000}{12.000000}\selectfont 1}%
\end{pgfscope}%
\begin{pgfscope}%
\pgfsetbuttcap%
\pgfsetroundjoin%
\definecolor{currentfill}{rgb}{0.000000,0.000000,0.000000}%
\pgfsetfillcolor{currentfill}%
\pgfsetlinewidth{0.803000pt}%
\definecolor{currentstroke}{rgb}{0.000000,0.000000,0.000000}%
\pgfsetstrokecolor{currentstroke}%
\pgfsetdash{}{0pt}%
\pgfsys@defobject{currentmarker}{\pgfqpoint{0.000000in}{-0.048611in}}{\pgfqpoint{0.000000in}{0.000000in}}{%
\pgfpathmoveto{\pgfqpoint{0.000000in}{0.000000in}}%
\pgfpathlineto{\pgfqpoint{0.000000in}{-0.048611in}}%
\pgfusepath{stroke,fill}%
}%
\begin{pgfscope}%
\pgfsys@transformshift{5.842462in}{3.312222in}%
\pgfsys@useobject{currentmarker}{}%
\end{pgfscope}%
\end{pgfscope}%
\begin{pgfscope}%
\definecolor{textcolor}{rgb}{0.000000,0.000000,0.000000}%
\pgfsetstrokecolor{textcolor}%
\pgfsetfillcolor{textcolor}%
\pgftext[x=5.842462in,y=3.215000in,,top]{\color{textcolor}\sffamily\fontsize{10.000000}{12.000000}\selectfont 2}%
\end{pgfscope}%
\begin{pgfscope}%
\pgfsetbuttcap%
\pgfsetroundjoin%
\definecolor{currentfill}{rgb}{0.000000,0.000000,0.000000}%
\pgfsetfillcolor{currentfill}%
\pgfsetlinewidth{0.803000pt}%
\definecolor{currentstroke}{rgb}{0.000000,0.000000,0.000000}%
\pgfsetstrokecolor{currentstroke}%
\pgfsetdash{}{0pt}%
\pgfsys@defobject{currentmarker}{\pgfqpoint{0.000000in}{-0.048611in}}{\pgfqpoint{0.000000in}{0.000000in}}{%
\pgfpathmoveto{\pgfqpoint{0.000000in}{0.000000in}}%
\pgfpathlineto{\pgfqpoint{0.000000in}{-0.048611in}}%
\pgfusepath{stroke,fill}%
}%
\begin{pgfscope}%
\pgfsys@transformshift{6.569760in}{3.312222in}%
\pgfsys@useobject{currentmarker}{}%
\end{pgfscope}%
\end{pgfscope}%
\begin{pgfscope}%
\definecolor{textcolor}{rgb}{0.000000,0.000000,0.000000}%
\pgfsetstrokecolor{textcolor}%
\pgfsetfillcolor{textcolor}%
\pgftext[x=6.569760in,y=3.215000in,,top]{\color{textcolor}\sffamily\fontsize{10.000000}{12.000000}\selectfont 3}%
\end{pgfscope}%
\begin{pgfscope}%
\pgfsetbuttcap%
\pgfsetroundjoin%
\definecolor{currentfill}{rgb}{0.000000,0.000000,0.000000}%
\pgfsetfillcolor{currentfill}%
\pgfsetlinewidth{0.803000pt}%
\definecolor{currentstroke}{rgb}{0.000000,0.000000,0.000000}%
\pgfsetstrokecolor{currentstroke}%
\pgfsetdash{}{0pt}%
\pgfsys@defobject{currentmarker}{\pgfqpoint{0.000000in}{-0.048611in}}{\pgfqpoint{0.000000in}{0.000000in}}{%
\pgfpathmoveto{\pgfqpoint{0.000000in}{0.000000in}}%
\pgfpathlineto{\pgfqpoint{0.000000in}{-0.048611in}}%
\pgfusepath{stroke,fill}%
}%
\begin{pgfscope}%
\pgfsys@transformshift{7.297059in}{3.312222in}%
\pgfsys@useobject{currentmarker}{}%
\end{pgfscope}%
\end{pgfscope}%
\begin{pgfscope}%
\definecolor{textcolor}{rgb}{0.000000,0.000000,0.000000}%
\pgfsetstrokecolor{textcolor}%
\pgfsetfillcolor{textcolor}%
\pgftext[x=7.297059in,y=3.215000in,,top]{\color{textcolor}\sffamily\fontsize{10.000000}{12.000000}\selectfont 4}%
\end{pgfscope}%
\begin{pgfscope}%
\pgfsetbuttcap%
\pgfsetroundjoin%
\definecolor{currentfill}{rgb}{0.000000,0.000000,0.000000}%
\pgfsetfillcolor{currentfill}%
\pgfsetlinewidth{0.803000pt}%
\definecolor{currentstroke}{rgb}{0.000000,0.000000,0.000000}%
\pgfsetstrokecolor{currentstroke}%
\pgfsetdash{}{0pt}%
\pgfsys@defobject{currentmarker}{\pgfqpoint{-0.048611in}{0.000000in}}{\pgfqpoint{0.000000in}{0.000000in}}{%
\pgfpathmoveto{\pgfqpoint{0.000000in}{0.000000in}}%
\pgfpathlineto{\pgfqpoint{-0.048611in}{0.000000in}}%
\pgfusepath{stroke,fill}%
}%
\begin{pgfscope}%
\pgfsys@transformshift{4.597222in}{3.415690in}%
\pgfsys@useobject{currentmarker}{}%
\end{pgfscope}%
\end{pgfscope}%
\begin{pgfscope}%
\definecolor{textcolor}{rgb}{0.000000,0.000000,0.000000}%
\pgfsetstrokecolor{textcolor}%
\pgfsetfillcolor{textcolor}%
\pgftext[x=4.074382in,y=3.362928in,left,base]{\color{textcolor}\sffamily\fontsize{10.000000}{12.000000}\selectfont −2.00}%
\end{pgfscope}%
\begin{pgfscope}%
\pgfsetbuttcap%
\pgfsetroundjoin%
\definecolor{currentfill}{rgb}{0.000000,0.000000,0.000000}%
\pgfsetfillcolor{currentfill}%
\pgfsetlinewidth{0.803000pt}%
\definecolor{currentstroke}{rgb}{0.000000,0.000000,0.000000}%
\pgfsetstrokecolor{currentstroke}%
\pgfsetdash{}{0pt}%
\pgfsys@defobject{currentmarker}{\pgfqpoint{-0.048611in}{0.000000in}}{\pgfqpoint{0.000000in}{0.000000in}}{%
\pgfpathmoveto{\pgfqpoint{0.000000in}{0.000000in}}%
\pgfpathlineto{\pgfqpoint{-0.048611in}{0.000000in}}%
\pgfusepath{stroke,fill}%
}%
\begin{pgfscope}%
\pgfsys@transformshift{4.597222in}{3.763194in}%
\pgfsys@useobject{currentmarker}{}%
\end{pgfscope}%
\end{pgfscope}%
\begin{pgfscope}%
\definecolor{textcolor}{rgb}{0.000000,0.000000,0.000000}%
\pgfsetstrokecolor{textcolor}%
\pgfsetfillcolor{textcolor}%
\pgftext[x=4.074382in,y=3.710432in,left,base]{\color{textcolor}\sffamily\fontsize{10.000000}{12.000000}\selectfont −1.75}%
\end{pgfscope}%
\begin{pgfscope}%
\pgfsetbuttcap%
\pgfsetroundjoin%
\definecolor{currentfill}{rgb}{0.000000,0.000000,0.000000}%
\pgfsetfillcolor{currentfill}%
\pgfsetlinewidth{0.803000pt}%
\definecolor{currentstroke}{rgb}{0.000000,0.000000,0.000000}%
\pgfsetstrokecolor{currentstroke}%
\pgfsetdash{}{0pt}%
\pgfsys@defobject{currentmarker}{\pgfqpoint{-0.048611in}{0.000000in}}{\pgfqpoint{0.000000in}{0.000000in}}{%
\pgfpathmoveto{\pgfqpoint{0.000000in}{0.000000in}}%
\pgfpathlineto{\pgfqpoint{-0.048611in}{0.000000in}}%
\pgfusepath{stroke,fill}%
}%
\begin{pgfscope}%
\pgfsys@transformshift{4.597222in}{4.110697in}%
\pgfsys@useobject{currentmarker}{}%
\end{pgfscope}%
\end{pgfscope}%
\begin{pgfscope}%
\definecolor{textcolor}{rgb}{0.000000,0.000000,0.000000}%
\pgfsetstrokecolor{textcolor}%
\pgfsetfillcolor{textcolor}%
\pgftext[x=4.074382in,y=4.057936in,left,base]{\color{textcolor}\sffamily\fontsize{10.000000}{12.000000}\selectfont −1.50}%
\end{pgfscope}%
\begin{pgfscope}%
\pgfsetbuttcap%
\pgfsetroundjoin%
\definecolor{currentfill}{rgb}{0.000000,0.000000,0.000000}%
\pgfsetfillcolor{currentfill}%
\pgfsetlinewidth{0.803000pt}%
\definecolor{currentstroke}{rgb}{0.000000,0.000000,0.000000}%
\pgfsetstrokecolor{currentstroke}%
\pgfsetdash{}{0pt}%
\pgfsys@defobject{currentmarker}{\pgfqpoint{-0.048611in}{0.000000in}}{\pgfqpoint{0.000000in}{0.000000in}}{%
\pgfpathmoveto{\pgfqpoint{0.000000in}{0.000000in}}%
\pgfpathlineto{\pgfqpoint{-0.048611in}{0.000000in}}%
\pgfusepath{stroke,fill}%
}%
\begin{pgfscope}%
\pgfsys@transformshift{4.597222in}{4.458201in}%
\pgfsys@useobject{currentmarker}{}%
\end{pgfscope}%
\end{pgfscope}%
\begin{pgfscope}%
\definecolor{textcolor}{rgb}{0.000000,0.000000,0.000000}%
\pgfsetstrokecolor{textcolor}%
\pgfsetfillcolor{textcolor}%
\pgftext[x=4.074382in,y=4.405439in,left,base]{\color{textcolor}\sffamily\fontsize{10.000000}{12.000000}\selectfont −1.25}%
\end{pgfscope}%
\begin{pgfscope}%
\pgfsetbuttcap%
\pgfsetroundjoin%
\definecolor{currentfill}{rgb}{0.000000,0.000000,0.000000}%
\pgfsetfillcolor{currentfill}%
\pgfsetlinewidth{0.803000pt}%
\definecolor{currentstroke}{rgb}{0.000000,0.000000,0.000000}%
\pgfsetstrokecolor{currentstroke}%
\pgfsetdash{}{0pt}%
\pgfsys@defobject{currentmarker}{\pgfqpoint{-0.048611in}{0.000000in}}{\pgfqpoint{0.000000in}{0.000000in}}{%
\pgfpathmoveto{\pgfqpoint{0.000000in}{0.000000in}}%
\pgfpathlineto{\pgfqpoint{-0.048611in}{0.000000in}}%
\pgfusepath{stroke,fill}%
}%
\begin{pgfscope}%
\pgfsys@transformshift{4.597222in}{4.805705in}%
\pgfsys@useobject{currentmarker}{}%
\end{pgfscope}%
\end{pgfscope}%
\begin{pgfscope}%
\definecolor{textcolor}{rgb}{0.000000,0.000000,0.000000}%
\pgfsetstrokecolor{textcolor}%
\pgfsetfillcolor{textcolor}%
\pgftext[x=4.074382in,y=4.752943in,left,base]{\color{textcolor}\sffamily\fontsize{10.000000}{12.000000}\selectfont −1.00}%
\end{pgfscope}%
\begin{pgfscope}%
\pgfsetbuttcap%
\pgfsetroundjoin%
\definecolor{currentfill}{rgb}{0.000000,0.000000,0.000000}%
\pgfsetfillcolor{currentfill}%
\pgfsetlinewidth{0.803000pt}%
\definecolor{currentstroke}{rgb}{0.000000,0.000000,0.000000}%
\pgfsetstrokecolor{currentstroke}%
\pgfsetdash{}{0pt}%
\pgfsys@defobject{currentmarker}{\pgfqpoint{-0.048611in}{0.000000in}}{\pgfqpoint{0.000000in}{0.000000in}}{%
\pgfpathmoveto{\pgfqpoint{0.000000in}{0.000000in}}%
\pgfpathlineto{\pgfqpoint{-0.048611in}{0.000000in}}%
\pgfusepath{stroke,fill}%
}%
\begin{pgfscope}%
\pgfsys@transformshift{4.597222in}{5.153208in}%
\pgfsys@useobject{currentmarker}{}%
\end{pgfscope}%
\end{pgfscope}%
\begin{pgfscope}%
\definecolor{textcolor}{rgb}{0.000000,0.000000,0.000000}%
\pgfsetstrokecolor{textcolor}%
\pgfsetfillcolor{textcolor}%
\pgftext[x=4.074382in,y=5.100447in,left,base]{\color{textcolor}\sffamily\fontsize{10.000000}{12.000000}\selectfont −0.75}%
\end{pgfscope}%
\begin{pgfscope}%
\pgfsetbuttcap%
\pgfsetroundjoin%
\definecolor{currentfill}{rgb}{0.000000,0.000000,0.000000}%
\pgfsetfillcolor{currentfill}%
\pgfsetlinewidth{0.803000pt}%
\definecolor{currentstroke}{rgb}{0.000000,0.000000,0.000000}%
\pgfsetstrokecolor{currentstroke}%
\pgfsetdash{}{0pt}%
\pgfsys@defobject{currentmarker}{\pgfqpoint{-0.048611in}{0.000000in}}{\pgfqpoint{0.000000in}{0.000000in}}{%
\pgfpathmoveto{\pgfqpoint{0.000000in}{0.000000in}}%
\pgfpathlineto{\pgfqpoint{-0.048611in}{0.000000in}}%
\pgfusepath{stroke,fill}%
}%
\begin{pgfscope}%
\pgfsys@transformshift{4.597222in}{5.500712in}%
\pgfsys@useobject{currentmarker}{}%
\end{pgfscope}%
\end{pgfscope}%
\begin{pgfscope}%
\definecolor{textcolor}{rgb}{0.000000,0.000000,0.000000}%
\pgfsetstrokecolor{textcolor}%
\pgfsetfillcolor{textcolor}%
\pgftext[x=4.074382in,y=5.447951in,left,base]{\color{textcolor}\sffamily\fontsize{10.000000}{12.000000}\selectfont −0.50}%
\end{pgfscope}%
\begin{pgfscope}%
\pgfpathrectangle{\pgfqpoint{4.597222in}{3.312222in}}{\pgfqpoint{3.217778in}{2.324444in}}%
\pgfusepath{clip}%
\pgfsetrectcap%
\pgfsetroundjoin%
\pgfsetlinewidth{1.505625pt}%
\definecolor{currentstroke}{rgb}{0.121569,0.466667,0.705882}%
\pgfsetstrokecolor{currentstroke}%
\pgfsetdash{}{0pt}%
\pgfpathmoveto{\pgfqpoint{4.751514in}{3.426548in}}%
\pgfpathlineto{\pgfqpoint{4.824244in}{3.426532in}}%
\pgfpathlineto{\pgfqpoint{4.896974in}{3.426435in}}%
\pgfpathlineto{\pgfqpoint{4.969704in}{3.426129in}}%
\pgfpathlineto{\pgfqpoint{5.042434in}{3.425649in}}%
\pgfpathlineto{\pgfqpoint{5.115163in}{3.425725in}}%
\pgfpathlineto{\pgfqpoint{5.187893in}{3.427906in}}%
\pgfpathlineto{\pgfqpoint{5.260623in}{3.433762in}}%
\pgfpathlineto{\pgfqpoint{5.333353in}{3.444411in}}%
\pgfpathlineto{\pgfqpoint{5.406083in}{3.460803in}}%
\pgfpathlineto{\pgfqpoint{5.478813in}{3.484123in}}%
\pgfpathlineto{\pgfqpoint{5.551543in}{3.515737in}}%
\pgfpathlineto{\pgfqpoint{5.624272in}{3.557444in}}%
\pgfpathlineto{\pgfqpoint{5.697002in}{3.611404in}}%
\pgfpathlineto{\pgfqpoint{5.769732in}{3.680655in}}%
\pgfpathlineto{\pgfqpoint{5.842462in}{3.769495in}}%
\pgfpathlineto{\pgfqpoint{5.915192in}{3.885353in}}%
\pgfpathlineto{\pgfqpoint{5.987922in}{4.045650in}}%
\pgfpathlineto{\pgfqpoint{6.060651in}{4.283513in}}%
\pgfpathlineto{\pgfqpoint{6.133381in}{4.509909in}}%
\pgfpathlineto{\pgfqpoint{6.206111in}{4.656982in}}%
\pgfpathlineto{\pgfqpoint{6.278841in}{4.766316in}}%
\pgfpathlineto{\pgfqpoint{6.351571in}{4.856215in}}%
\pgfpathlineto{\pgfqpoint{6.424301in}{4.933310in}}%
\pgfpathlineto{\pgfqpoint{6.497030in}{5.000694in}}%
\pgfpathlineto{\pgfqpoint{6.569760in}{5.059628in}}%
\pgfpathlineto{\pgfqpoint{6.642490in}{5.112465in}}%
\pgfpathlineto{\pgfqpoint{6.715220in}{5.160134in}}%
\pgfpathlineto{\pgfqpoint{6.787950in}{5.202886in}}%
\pgfpathlineto{\pgfqpoint{6.860680in}{5.242536in}}%
\pgfpathlineto{\pgfqpoint{6.933410in}{5.278350in}}%
\pgfpathlineto{\pgfqpoint{7.006139in}{5.311185in}}%
\pgfpathlineto{\pgfqpoint{7.078869in}{5.341714in}}%
\pgfpathlineto{\pgfqpoint{7.151599in}{5.370132in}}%
\pgfpathlineto{\pgfqpoint{7.224329in}{5.396590in}}%
\pgfpathlineto{\pgfqpoint{7.297059in}{5.420989in}}%
\pgfpathlineto{\pgfqpoint{7.369789in}{5.444222in}}%
\pgfpathlineto{\pgfqpoint{7.442518in}{5.465878in}}%
\pgfpathlineto{\pgfqpoint{7.515248in}{5.486219in}}%
\pgfpathlineto{\pgfqpoint{7.587978in}{5.505250in}}%
\pgfpathlineto{\pgfqpoint{7.660708in}{5.523240in}}%
\pgfusepath{stroke}%
\end{pgfscope}%
\begin{pgfscope}%
\pgfsetrectcap%
\pgfsetmiterjoin%
\pgfsetlinewidth{0.803000pt}%
\definecolor{currentstroke}{rgb}{0.000000,0.000000,0.000000}%
\pgfsetstrokecolor{currentstroke}%
\pgfsetdash{}{0pt}%
\pgfpathmoveto{\pgfqpoint{4.597222in}{3.312222in}}%
\pgfpathlineto{\pgfqpoint{4.597222in}{5.636667in}}%
\pgfusepath{stroke}%
\end{pgfscope}%
\begin{pgfscope}%
\pgfsetrectcap%
\pgfsetmiterjoin%
\pgfsetlinewidth{0.803000pt}%
\definecolor{currentstroke}{rgb}{0.000000,0.000000,0.000000}%
\pgfsetstrokecolor{currentstroke}%
\pgfsetdash{}{0pt}%
\pgfpathmoveto{\pgfqpoint{7.815000in}{3.312222in}}%
\pgfpathlineto{\pgfqpoint{7.815000in}{5.636667in}}%
\pgfusepath{stroke}%
\end{pgfscope}%
\begin{pgfscope}%
\pgfsetrectcap%
\pgfsetmiterjoin%
\pgfsetlinewidth{0.803000pt}%
\definecolor{currentstroke}{rgb}{0.000000,0.000000,0.000000}%
\pgfsetstrokecolor{currentstroke}%
\pgfsetdash{}{0pt}%
\pgfpathmoveto{\pgfqpoint{4.597222in}{3.312222in}}%
\pgfpathlineto{\pgfqpoint{7.815000in}{3.312222in}}%
\pgfusepath{stroke}%
\end{pgfscope}%
\begin{pgfscope}%
\pgfsetrectcap%
\pgfsetmiterjoin%
\pgfsetlinewidth{0.803000pt}%
\definecolor{currentstroke}{rgb}{0.000000,0.000000,0.000000}%
\pgfsetstrokecolor{currentstroke}%
\pgfsetdash{}{0pt}%
\pgfpathmoveto{\pgfqpoint{4.597222in}{5.636667in}}%
\pgfpathlineto{\pgfqpoint{7.815000in}{5.636667in}}%
\pgfusepath{stroke}%
\end{pgfscope}%
\begin{pgfscope}%
\definecolor{textcolor}{rgb}{0.000000,0.000000,0.000000}%
\pgfsetstrokecolor{textcolor}%
\pgfsetfillcolor{textcolor}%
\pgftext[x=6.206111in,y=5.720000in,,base]{\color{textcolor}\sffamily\fontsize{12.000000}{14.400000}\selectfont \(\displaystyle  N = 32 \)}%
\end{pgfscope}%
\begin{pgfscope}%
\pgfsetbuttcap%
\pgfsetmiterjoin%
\definecolor{currentfill}{rgb}{1.000000,1.000000,1.000000}%
\pgfsetfillcolor{currentfill}%
\pgfsetlinewidth{0.000000pt}%
\definecolor{currentstroke}{rgb}{0.000000,0.000000,0.000000}%
\pgfsetstrokecolor{currentstroke}%
\pgfsetstrokeopacity{0.000000}%
\pgfsetdash{}{0pt}%
\pgfpathmoveto{\pgfqpoint{0.672222in}{0.387222in}}%
\pgfpathlineto{\pgfqpoint{3.890000in}{0.387222in}}%
\pgfpathlineto{\pgfqpoint{3.890000in}{2.711667in}}%
\pgfpathlineto{\pgfqpoint{0.672222in}{2.711667in}}%
\pgfpathclose%
\pgfusepath{fill}%
\end{pgfscope}%
\begin{pgfscope}%
\pgfpathrectangle{\pgfqpoint{0.672222in}{0.387222in}}{\pgfqpoint{3.217778in}{2.324444in}}%
\pgfusepath{clip}%
\pgfsetbuttcap%
\pgfsetroundjoin%
\definecolor{currentfill}{rgb}{0.121569,0.466667,0.705882}%
\pgfsetfillcolor{currentfill}%
\pgfsetfillopacity{0.300000}%
\pgfsetlinewidth{1.003750pt}%
\definecolor{currentstroke}{rgb}{0.121569,0.466667,0.705882}%
\pgfsetstrokecolor{currentstroke}%
\pgfsetstrokeopacity{0.300000}%
\pgfsetdash{}{0pt}%
\pgfpathmoveto{\pgfqpoint{0.826514in}{0.500910in}}%
\pgfpathlineto{\pgfqpoint{0.826514in}{0.500910in}}%
\pgfpathlineto{\pgfqpoint{0.899244in}{0.500893in}}%
\pgfpathlineto{\pgfqpoint{0.971974in}{0.500808in}}%
\pgfpathlineto{\pgfqpoint{1.044704in}{0.500649in}}%
\pgfpathlineto{\pgfqpoint{1.117434in}{0.500917in}}%
\pgfpathlineto{\pgfqpoint{1.190163in}{0.502669in}}%
\pgfpathlineto{\pgfqpoint{1.262893in}{0.506804in}}%
\pgfpathlineto{\pgfqpoint{1.335623in}{0.514125in}}%
\pgfpathlineto{\pgfqpoint{1.408353in}{0.525604in}}%
\pgfpathlineto{\pgfqpoint{1.481083in}{0.542429in}}%
\pgfpathlineto{\pgfqpoint{1.553813in}{0.565934in}}%
\pgfpathlineto{\pgfqpoint{1.626543in}{0.597606in}}%
\pgfpathlineto{\pgfqpoint{1.699272in}{0.639177in}}%
\pgfpathlineto{\pgfqpoint{1.772002in}{0.692971in}}%
\pgfpathlineto{\pgfqpoint{1.844732in}{0.761901in}}%
\pgfpathlineto{\pgfqpoint{1.917462in}{0.850528in}}%
\pgfpathlineto{\pgfqpoint{1.990192in}{0.965725in}}%
\pgfpathlineto{\pgfqpoint{2.062922in}{1.125723in}}%
\pgfpathlineto{\pgfqpoint{2.135651in}{1.385001in}}%
\pgfpathlineto{\pgfqpoint{2.208381in}{1.599920in}}%
\pgfpathlineto{\pgfqpoint{2.281111in}{1.734797in}}%
\pgfpathlineto{\pgfqpoint{2.353841in}{1.843444in}}%
\pgfpathlineto{\pgfqpoint{2.426571in}{1.932966in}}%
\pgfpathlineto{\pgfqpoint{2.499301in}{2.009820in}}%
\pgfpathlineto{\pgfqpoint{2.572030in}{2.076934in}}%
\pgfpathlineto{\pgfqpoint{2.644760in}{2.136197in}}%
\pgfpathlineto{\pgfqpoint{2.717490in}{2.188735in}}%
\pgfpathlineto{\pgfqpoint{2.790220in}{2.235759in}}%
\pgfpathlineto{\pgfqpoint{2.862950in}{2.278976in}}%
\pgfpathlineto{\pgfqpoint{2.935680in}{2.317747in}}%
\pgfpathlineto{\pgfqpoint{3.008410in}{2.353915in}}%
\pgfpathlineto{\pgfqpoint{3.081139in}{2.386719in}}%
\pgfpathlineto{\pgfqpoint{3.153869in}{2.416958in}}%
\pgfpathlineto{\pgfqpoint{3.226599in}{2.445531in}}%
\pgfpathlineto{\pgfqpoint{3.299329in}{2.471612in}}%
\pgfpathlineto{\pgfqpoint{3.372059in}{2.496038in}}%
\pgfpathlineto{\pgfqpoint{3.444789in}{2.519158in}}%
\pgfpathlineto{\pgfqpoint{3.517518in}{2.540631in}}%
\pgfpathlineto{\pgfqpoint{3.590248in}{2.561005in}}%
\pgfpathlineto{\pgfqpoint{3.662978in}{2.579981in}}%
\pgfpathlineto{\pgfqpoint{3.735708in}{2.598052in}}%
\pgfpathlineto{\pgfqpoint{3.735708in}{2.598428in}}%
\pgfpathlineto{\pgfqpoint{3.735708in}{2.598428in}}%
\pgfpathlineto{\pgfqpoint{3.662978in}{2.580405in}}%
\pgfpathlineto{\pgfqpoint{3.590248in}{2.561500in}}%
\pgfpathlineto{\pgfqpoint{3.517518in}{2.541468in}}%
\pgfpathlineto{\pgfqpoint{3.444789in}{2.519602in}}%
\pgfpathlineto{\pgfqpoint{3.372059in}{2.496966in}}%
\pgfpathlineto{\pgfqpoint{3.299329in}{2.471859in}}%
\pgfpathlineto{\pgfqpoint{3.226599in}{2.446105in}}%
\pgfpathlineto{\pgfqpoint{3.153869in}{2.417719in}}%
\pgfpathlineto{\pgfqpoint{3.081139in}{2.386905in}}%
\pgfpathlineto{\pgfqpoint{3.008410in}{2.354115in}}%
\pgfpathlineto{\pgfqpoint{2.935680in}{2.318309in}}%
\pgfpathlineto{\pgfqpoint{2.862950in}{2.279623in}}%
\pgfpathlineto{\pgfqpoint{2.790220in}{2.236522in}}%
\pgfpathlineto{\pgfqpoint{2.717490in}{2.189284in}}%
\pgfpathlineto{\pgfqpoint{2.644760in}{2.136702in}}%
\pgfpathlineto{\pgfqpoint{2.572030in}{2.078201in}}%
\pgfpathlineto{\pgfqpoint{2.499301in}{2.010589in}}%
\pgfpathlineto{\pgfqpoint{2.426571in}{1.933999in}}%
\pgfpathlineto{\pgfqpoint{2.353841in}{1.844147in}}%
\pgfpathlineto{\pgfqpoint{2.281111in}{1.736634in}}%
\pgfpathlineto{\pgfqpoint{2.208381in}{1.602207in}}%
\pgfpathlineto{\pgfqpoint{2.135651in}{1.389765in}}%
\pgfpathlineto{\pgfqpoint{2.062922in}{1.128195in}}%
\pgfpathlineto{\pgfqpoint{1.990192in}{0.967016in}}%
\pgfpathlineto{\pgfqpoint{1.917462in}{0.850783in}}%
\pgfpathlineto{\pgfqpoint{1.844732in}{0.762158in}}%
\pgfpathlineto{\pgfqpoint{1.772002in}{0.693138in}}%
\pgfpathlineto{\pgfqpoint{1.699272in}{0.639318in}}%
\pgfpathlineto{\pgfqpoint{1.626543in}{0.597685in}}%
\pgfpathlineto{\pgfqpoint{1.553813in}{0.565962in}}%
\pgfpathlineto{\pgfqpoint{1.481083in}{0.542462in}}%
\pgfpathlineto{\pgfqpoint{1.408353in}{0.525627in}}%
\pgfpathlineto{\pgfqpoint{1.335623in}{0.514131in}}%
\pgfpathlineto{\pgfqpoint{1.262893in}{0.506809in}}%
\pgfpathlineto{\pgfqpoint{1.190163in}{0.502670in}}%
\pgfpathlineto{\pgfqpoint{1.117434in}{0.500917in}}%
\pgfpathlineto{\pgfqpoint{1.044704in}{0.500649in}}%
\pgfpathlineto{\pgfqpoint{0.971974in}{0.500808in}}%
\pgfpathlineto{\pgfqpoint{0.899244in}{0.500893in}}%
\pgfpathlineto{\pgfqpoint{0.826514in}{0.500910in}}%
\pgfpathclose%
\pgfusepath{stroke,fill}%
\end{pgfscope}%
\begin{pgfscope}%
\pgfpathrectangle{\pgfqpoint{0.672222in}{0.387222in}}{\pgfqpoint{3.217778in}{2.324444in}}%
\pgfusepath{clip}%
\pgfsetbuttcap%
\pgfsetroundjoin%
\definecolor{currentfill}{rgb}{0.121569,0.466667,0.705882}%
\pgfsetfillcolor{currentfill}%
\pgfsetlinewidth{1.003750pt}%
\definecolor{currentstroke}{rgb}{0.121569,0.466667,0.705882}%
\pgfsetstrokecolor{currentstroke}%
\pgfsetdash{}{0pt}%
\pgfsys@defobject{currentmarker}{\pgfqpoint{-0.009821in}{-0.009821in}}{\pgfqpoint{0.009821in}{0.009821in}}{%
\pgfpathmoveto{\pgfqpoint{0.000000in}{-0.009821in}}%
\pgfpathcurveto{\pgfqpoint{0.002605in}{-0.009821in}}{\pgfqpoint{0.005103in}{-0.008786in}}{\pgfqpoint{0.006944in}{-0.006944in}}%
\pgfpathcurveto{\pgfqpoint{0.008786in}{-0.005103in}}{\pgfqpoint{0.009821in}{-0.002605in}}{\pgfqpoint{0.009821in}{0.000000in}}%
\pgfpathcurveto{\pgfqpoint{0.009821in}{0.002605in}}{\pgfqpoint{0.008786in}{0.005103in}}{\pgfqpoint{0.006944in}{0.006944in}}%
\pgfpathcurveto{\pgfqpoint{0.005103in}{0.008786in}}{\pgfqpoint{0.002605in}{0.009821in}}{\pgfqpoint{0.000000in}{0.009821in}}%
\pgfpathcurveto{\pgfqpoint{-0.002605in}{0.009821in}}{\pgfqpoint{-0.005103in}{0.008786in}}{\pgfqpoint{-0.006944in}{0.006944in}}%
\pgfpathcurveto{\pgfqpoint{-0.008786in}{0.005103in}}{\pgfqpoint{-0.009821in}{0.002605in}}{\pgfqpoint{-0.009821in}{0.000000in}}%
\pgfpathcurveto{\pgfqpoint{-0.009821in}{-0.002605in}}{\pgfqpoint{-0.008786in}{-0.005103in}}{\pgfqpoint{-0.006944in}{-0.006944in}}%
\pgfpathcurveto{\pgfqpoint{-0.005103in}{-0.008786in}}{\pgfqpoint{-0.002605in}{-0.009821in}}{\pgfqpoint{0.000000in}{-0.009821in}}%
\pgfpathclose%
\pgfusepath{stroke,fill}%
}%
\begin{pgfscope}%
\pgfsys@transformshift{0.826514in}{0.500910in}%
\pgfsys@useobject{currentmarker}{}%
\end{pgfscope}%
\begin{pgfscope}%
\pgfsys@transformshift{0.899244in}{0.500893in}%
\pgfsys@useobject{currentmarker}{}%
\end{pgfscope}%
\begin{pgfscope}%
\pgfsys@transformshift{0.971974in}{0.500808in}%
\pgfsys@useobject{currentmarker}{}%
\end{pgfscope}%
\begin{pgfscope}%
\pgfsys@transformshift{1.044704in}{0.500649in}%
\pgfsys@useobject{currentmarker}{}%
\end{pgfscope}%
\begin{pgfscope}%
\pgfsys@transformshift{1.117434in}{0.500917in}%
\pgfsys@useobject{currentmarker}{}%
\end{pgfscope}%
\begin{pgfscope}%
\pgfsys@transformshift{1.190163in}{0.502670in}%
\pgfsys@useobject{currentmarker}{}%
\end{pgfscope}%
\begin{pgfscope}%
\pgfsys@transformshift{1.262893in}{0.506806in}%
\pgfsys@useobject{currentmarker}{}%
\end{pgfscope}%
\begin{pgfscope}%
\pgfsys@transformshift{1.335623in}{0.514128in}%
\pgfsys@useobject{currentmarker}{}%
\end{pgfscope}%
\begin{pgfscope}%
\pgfsys@transformshift{1.408353in}{0.525615in}%
\pgfsys@useobject{currentmarker}{}%
\end{pgfscope}%
\begin{pgfscope}%
\pgfsys@transformshift{1.481083in}{0.542446in}%
\pgfsys@useobject{currentmarker}{}%
\end{pgfscope}%
\begin{pgfscope}%
\pgfsys@transformshift{1.553813in}{0.565948in}%
\pgfsys@useobject{currentmarker}{}%
\end{pgfscope}%
\begin{pgfscope}%
\pgfsys@transformshift{1.626543in}{0.597645in}%
\pgfsys@useobject{currentmarker}{}%
\end{pgfscope}%
\begin{pgfscope}%
\pgfsys@transformshift{1.699272in}{0.639248in}%
\pgfsys@useobject{currentmarker}{}%
\end{pgfscope}%
\begin{pgfscope}%
\pgfsys@transformshift{1.772002in}{0.693054in}%
\pgfsys@useobject{currentmarker}{}%
\end{pgfscope}%
\begin{pgfscope}%
\pgfsys@transformshift{1.844732in}{0.762029in}%
\pgfsys@useobject{currentmarker}{}%
\end{pgfscope}%
\begin{pgfscope}%
\pgfsys@transformshift{1.917462in}{0.850655in}%
\pgfsys@useobject{currentmarker}{}%
\end{pgfscope}%
\begin{pgfscope}%
\pgfsys@transformshift{1.990192in}{0.966371in}%
\pgfsys@useobject{currentmarker}{}%
\end{pgfscope}%
\begin{pgfscope}%
\pgfsys@transformshift{2.062922in}{1.126959in}%
\pgfsys@useobject{currentmarker}{}%
\end{pgfscope}%
\begin{pgfscope}%
\pgfsys@transformshift{2.135651in}{1.387383in}%
\pgfsys@useobject{currentmarker}{}%
\end{pgfscope}%
\begin{pgfscope}%
\pgfsys@transformshift{2.208381in}{1.601064in}%
\pgfsys@useobject{currentmarker}{}%
\end{pgfscope}%
\begin{pgfscope}%
\pgfsys@transformshift{2.281111in}{1.735716in}%
\pgfsys@useobject{currentmarker}{}%
\end{pgfscope}%
\begin{pgfscope}%
\pgfsys@transformshift{2.353841in}{1.843795in}%
\pgfsys@useobject{currentmarker}{}%
\end{pgfscope}%
\begin{pgfscope}%
\pgfsys@transformshift{2.426571in}{1.933482in}%
\pgfsys@useobject{currentmarker}{}%
\end{pgfscope}%
\begin{pgfscope}%
\pgfsys@transformshift{2.499301in}{2.010205in}%
\pgfsys@useobject{currentmarker}{}%
\end{pgfscope}%
\begin{pgfscope}%
\pgfsys@transformshift{2.572030in}{2.077568in}%
\pgfsys@useobject{currentmarker}{}%
\end{pgfscope}%
\begin{pgfscope}%
\pgfsys@transformshift{2.644760in}{2.136450in}%
\pgfsys@useobject{currentmarker}{}%
\end{pgfscope}%
\begin{pgfscope}%
\pgfsys@transformshift{2.717490in}{2.189010in}%
\pgfsys@useobject{currentmarker}{}%
\end{pgfscope}%
\begin{pgfscope}%
\pgfsys@transformshift{2.790220in}{2.236140in}%
\pgfsys@useobject{currentmarker}{}%
\end{pgfscope}%
\begin{pgfscope}%
\pgfsys@transformshift{2.862950in}{2.279300in}%
\pgfsys@useobject{currentmarker}{}%
\end{pgfscope}%
\begin{pgfscope}%
\pgfsys@transformshift{2.935680in}{2.318028in}%
\pgfsys@useobject{currentmarker}{}%
\end{pgfscope}%
\begin{pgfscope}%
\pgfsys@transformshift{3.008410in}{2.354015in}%
\pgfsys@useobject{currentmarker}{}%
\end{pgfscope}%
\begin{pgfscope}%
\pgfsys@transformshift{3.081139in}{2.386812in}%
\pgfsys@useobject{currentmarker}{}%
\end{pgfscope}%
\begin{pgfscope}%
\pgfsys@transformshift{3.153869in}{2.417338in}%
\pgfsys@useobject{currentmarker}{}%
\end{pgfscope}%
\begin{pgfscope}%
\pgfsys@transformshift{3.226599in}{2.445818in}%
\pgfsys@useobject{currentmarker}{}%
\end{pgfscope}%
\begin{pgfscope}%
\pgfsys@transformshift{3.299329in}{2.471736in}%
\pgfsys@useobject{currentmarker}{}%
\end{pgfscope}%
\begin{pgfscope}%
\pgfsys@transformshift{3.372059in}{2.496502in}%
\pgfsys@useobject{currentmarker}{}%
\end{pgfscope}%
\begin{pgfscope}%
\pgfsys@transformshift{3.444789in}{2.519380in}%
\pgfsys@useobject{currentmarker}{}%
\end{pgfscope}%
\begin{pgfscope}%
\pgfsys@transformshift{3.517518in}{2.541050in}%
\pgfsys@useobject{currentmarker}{}%
\end{pgfscope}%
\begin{pgfscope}%
\pgfsys@transformshift{3.590248in}{2.561253in}%
\pgfsys@useobject{currentmarker}{}%
\end{pgfscope}%
\begin{pgfscope}%
\pgfsys@transformshift{3.662978in}{2.580193in}%
\pgfsys@useobject{currentmarker}{}%
\end{pgfscope}%
\begin{pgfscope}%
\pgfsys@transformshift{3.735708in}{2.598240in}%
\pgfsys@useobject{currentmarker}{}%
\end{pgfscope}%
\end{pgfscope}%
\begin{pgfscope}%
\pgfsetbuttcap%
\pgfsetroundjoin%
\definecolor{currentfill}{rgb}{0.000000,0.000000,0.000000}%
\pgfsetfillcolor{currentfill}%
\pgfsetlinewidth{0.803000pt}%
\definecolor{currentstroke}{rgb}{0.000000,0.000000,0.000000}%
\pgfsetstrokecolor{currentstroke}%
\pgfsetdash{}{0pt}%
\pgfsys@defobject{currentmarker}{\pgfqpoint{0.000000in}{-0.048611in}}{\pgfqpoint{0.000000in}{0.000000in}}{%
\pgfpathmoveto{\pgfqpoint{0.000000in}{0.000000in}}%
\pgfpathlineto{\pgfqpoint{0.000000in}{-0.048611in}}%
\pgfusepath{stroke,fill}%
}%
\begin{pgfscope}%
\pgfsys@transformshift{1.190163in}{0.387222in}%
\pgfsys@useobject{currentmarker}{}%
\end{pgfscope}%
\end{pgfscope}%
\begin{pgfscope}%
\definecolor{textcolor}{rgb}{0.000000,0.000000,0.000000}%
\pgfsetstrokecolor{textcolor}%
\pgfsetfillcolor{textcolor}%
\pgftext[x=1.190163in,y=0.290000in,,top]{\color{textcolor}\sffamily\fontsize{10.000000}{12.000000}\selectfont 1}%
\end{pgfscope}%
\begin{pgfscope}%
\pgfsetbuttcap%
\pgfsetroundjoin%
\definecolor{currentfill}{rgb}{0.000000,0.000000,0.000000}%
\pgfsetfillcolor{currentfill}%
\pgfsetlinewidth{0.803000pt}%
\definecolor{currentstroke}{rgb}{0.000000,0.000000,0.000000}%
\pgfsetstrokecolor{currentstroke}%
\pgfsetdash{}{0pt}%
\pgfsys@defobject{currentmarker}{\pgfqpoint{0.000000in}{-0.048611in}}{\pgfqpoint{0.000000in}{0.000000in}}{%
\pgfpathmoveto{\pgfqpoint{0.000000in}{0.000000in}}%
\pgfpathlineto{\pgfqpoint{0.000000in}{-0.048611in}}%
\pgfusepath{stroke,fill}%
}%
\begin{pgfscope}%
\pgfsys@transformshift{1.917462in}{0.387222in}%
\pgfsys@useobject{currentmarker}{}%
\end{pgfscope}%
\end{pgfscope}%
\begin{pgfscope}%
\definecolor{textcolor}{rgb}{0.000000,0.000000,0.000000}%
\pgfsetstrokecolor{textcolor}%
\pgfsetfillcolor{textcolor}%
\pgftext[x=1.917462in,y=0.290000in,,top]{\color{textcolor}\sffamily\fontsize{10.000000}{12.000000}\selectfont 2}%
\end{pgfscope}%
\begin{pgfscope}%
\pgfsetbuttcap%
\pgfsetroundjoin%
\definecolor{currentfill}{rgb}{0.000000,0.000000,0.000000}%
\pgfsetfillcolor{currentfill}%
\pgfsetlinewidth{0.803000pt}%
\definecolor{currentstroke}{rgb}{0.000000,0.000000,0.000000}%
\pgfsetstrokecolor{currentstroke}%
\pgfsetdash{}{0pt}%
\pgfsys@defobject{currentmarker}{\pgfqpoint{0.000000in}{-0.048611in}}{\pgfqpoint{0.000000in}{0.000000in}}{%
\pgfpathmoveto{\pgfqpoint{0.000000in}{0.000000in}}%
\pgfpathlineto{\pgfqpoint{0.000000in}{-0.048611in}}%
\pgfusepath{stroke,fill}%
}%
\begin{pgfscope}%
\pgfsys@transformshift{2.644760in}{0.387222in}%
\pgfsys@useobject{currentmarker}{}%
\end{pgfscope}%
\end{pgfscope}%
\begin{pgfscope}%
\definecolor{textcolor}{rgb}{0.000000,0.000000,0.000000}%
\pgfsetstrokecolor{textcolor}%
\pgfsetfillcolor{textcolor}%
\pgftext[x=2.644760in,y=0.290000in,,top]{\color{textcolor}\sffamily\fontsize{10.000000}{12.000000}\selectfont 3}%
\end{pgfscope}%
\begin{pgfscope}%
\pgfsetbuttcap%
\pgfsetroundjoin%
\definecolor{currentfill}{rgb}{0.000000,0.000000,0.000000}%
\pgfsetfillcolor{currentfill}%
\pgfsetlinewidth{0.803000pt}%
\definecolor{currentstroke}{rgb}{0.000000,0.000000,0.000000}%
\pgfsetstrokecolor{currentstroke}%
\pgfsetdash{}{0pt}%
\pgfsys@defobject{currentmarker}{\pgfqpoint{0.000000in}{-0.048611in}}{\pgfqpoint{0.000000in}{0.000000in}}{%
\pgfpathmoveto{\pgfqpoint{0.000000in}{0.000000in}}%
\pgfpathlineto{\pgfqpoint{0.000000in}{-0.048611in}}%
\pgfusepath{stroke,fill}%
}%
\begin{pgfscope}%
\pgfsys@transformshift{3.372059in}{0.387222in}%
\pgfsys@useobject{currentmarker}{}%
\end{pgfscope}%
\end{pgfscope}%
\begin{pgfscope}%
\definecolor{textcolor}{rgb}{0.000000,0.000000,0.000000}%
\pgfsetstrokecolor{textcolor}%
\pgfsetfillcolor{textcolor}%
\pgftext[x=3.372059in,y=0.290000in,,top]{\color{textcolor}\sffamily\fontsize{10.000000}{12.000000}\selectfont 4}%
\end{pgfscope}%
\begin{pgfscope}%
\pgfsetbuttcap%
\pgfsetroundjoin%
\definecolor{currentfill}{rgb}{0.000000,0.000000,0.000000}%
\pgfsetfillcolor{currentfill}%
\pgfsetlinewidth{0.803000pt}%
\definecolor{currentstroke}{rgb}{0.000000,0.000000,0.000000}%
\pgfsetstrokecolor{currentstroke}%
\pgfsetdash{}{0pt}%
\pgfsys@defobject{currentmarker}{\pgfqpoint{-0.048611in}{0.000000in}}{\pgfqpoint{0.000000in}{0.000000in}}{%
\pgfpathmoveto{\pgfqpoint{0.000000in}{0.000000in}}%
\pgfpathlineto{\pgfqpoint{-0.048611in}{0.000000in}}%
\pgfusepath{stroke,fill}%
}%
\begin{pgfscope}%
\pgfsys@transformshift{0.672222in}{0.498206in}%
\pgfsys@useobject{currentmarker}{}%
\end{pgfscope}%
\end{pgfscope}%
\begin{pgfscope}%
\definecolor{textcolor}{rgb}{0.000000,0.000000,0.000000}%
\pgfsetstrokecolor{textcolor}%
\pgfsetfillcolor{textcolor}%
\pgftext[x=0.149382in,y=0.445444in,left,base]{\color{textcolor}\sffamily\fontsize{10.000000}{12.000000}\selectfont −2.00}%
\end{pgfscope}%
\begin{pgfscope}%
\pgfsetbuttcap%
\pgfsetroundjoin%
\definecolor{currentfill}{rgb}{0.000000,0.000000,0.000000}%
\pgfsetfillcolor{currentfill}%
\pgfsetlinewidth{0.803000pt}%
\definecolor{currentstroke}{rgb}{0.000000,0.000000,0.000000}%
\pgfsetstrokecolor{currentstroke}%
\pgfsetdash{}{0pt}%
\pgfsys@defobject{currentmarker}{\pgfqpoint{-0.048611in}{0.000000in}}{\pgfqpoint{0.000000in}{0.000000in}}{%
\pgfpathmoveto{\pgfqpoint{0.000000in}{0.000000in}}%
\pgfpathlineto{\pgfqpoint{-0.048611in}{0.000000in}}%
\pgfusepath{stroke,fill}%
}%
\begin{pgfscope}%
\pgfsys@transformshift{0.672222in}{0.844470in}%
\pgfsys@useobject{currentmarker}{}%
\end{pgfscope}%
\end{pgfscope}%
\begin{pgfscope}%
\definecolor{textcolor}{rgb}{0.000000,0.000000,0.000000}%
\pgfsetstrokecolor{textcolor}%
\pgfsetfillcolor{textcolor}%
\pgftext[x=0.149382in,y=0.791709in,left,base]{\color{textcolor}\sffamily\fontsize{10.000000}{12.000000}\selectfont −1.75}%
\end{pgfscope}%
\begin{pgfscope}%
\pgfsetbuttcap%
\pgfsetroundjoin%
\definecolor{currentfill}{rgb}{0.000000,0.000000,0.000000}%
\pgfsetfillcolor{currentfill}%
\pgfsetlinewidth{0.803000pt}%
\definecolor{currentstroke}{rgb}{0.000000,0.000000,0.000000}%
\pgfsetstrokecolor{currentstroke}%
\pgfsetdash{}{0pt}%
\pgfsys@defobject{currentmarker}{\pgfqpoint{-0.048611in}{0.000000in}}{\pgfqpoint{0.000000in}{0.000000in}}{%
\pgfpathmoveto{\pgfqpoint{0.000000in}{0.000000in}}%
\pgfpathlineto{\pgfqpoint{-0.048611in}{0.000000in}}%
\pgfusepath{stroke,fill}%
}%
\begin{pgfscope}%
\pgfsys@transformshift{0.672222in}{1.190735in}%
\pgfsys@useobject{currentmarker}{}%
\end{pgfscope}%
\end{pgfscope}%
\begin{pgfscope}%
\definecolor{textcolor}{rgb}{0.000000,0.000000,0.000000}%
\pgfsetstrokecolor{textcolor}%
\pgfsetfillcolor{textcolor}%
\pgftext[x=0.149382in,y=1.137974in,left,base]{\color{textcolor}\sffamily\fontsize{10.000000}{12.000000}\selectfont −1.50}%
\end{pgfscope}%
\begin{pgfscope}%
\pgfsetbuttcap%
\pgfsetroundjoin%
\definecolor{currentfill}{rgb}{0.000000,0.000000,0.000000}%
\pgfsetfillcolor{currentfill}%
\pgfsetlinewidth{0.803000pt}%
\definecolor{currentstroke}{rgb}{0.000000,0.000000,0.000000}%
\pgfsetstrokecolor{currentstroke}%
\pgfsetdash{}{0pt}%
\pgfsys@defobject{currentmarker}{\pgfqpoint{-0.048611in}{0.000000in}}{\pgfqpoint{0.000000in}{0.000000in}}{%
\pgfpathmoveto{\pgfqpoint{0.000000in}{0.000000in}}%
\pgfpathlineto{\pgfqpoint{-0.048611in}{0.000000in}}%
\pgfusepath{stroke,fill}%
}%
\begin{pgfscope}%
\pgfsys@transformshift{0.672222in}{1.537000in}%
\pgfsys@useobject{currentmarker}{}%
\end{pgfscope}%
\end{pgfscope}%
\begin{pgfscope}%
\definecolor{textcolor}{rgb}{0.000000,0.000000,0.000000}%
\pgfsetstrokecolor{textcolor}%
\pgfsetfillcolor{textcolor}%
\pgftext[x=0.149382in,y=1.484239in,left,base]{\color{textcolor}\sffamily\fontsize{10.000000}{12.000000}\selectfont −1.25}%
\end{pgfscope}%
\begin{pgfscope}%
\pgfsetbuttcap%
\pgfsetroundjoin%
\definecolor{currentfill}{rgb}{0.000000,0.000000,0.000000}%
\pgfsetfillcolor{currentfill}%
\pgfsetlinewidth{0.803000pt}%
\definecolor{currentstroke}{rgb}{0.000000,0.000000,0.000000}%
\pgfsetstrokecolor{currentstroke}%
\pgfsetdash{}{0pt}%
\pgfsys@defobject{currentmarker}{\pgfqpoint{-0.048611in}{0.000000in}}{\pgfqpoint{0.000000in}{0.000000in}}{%
\pgfpathmoveto{\pgfqpoint{0.000000in}{0.000000in}}%
\pgfpathlineto{\pgfqpoint{-0.048611in}{0.000000in}}%
\pgfusepath{stroke,fill}%
}%
\begin{pgfscope}%
\pgfsys@transformshift{0.672222in}{1.883265in}%
\pgfsys@useobject{currentmarker}{}%
\end{pgfscope}%
\end{pgfscope}%
\begin{pgfscope}%
\definecolor{textcolor}{rgb}{0.000000,0.000000,0.000000}%
\pgfsetstrokecolor{textcolor}%
\pgfsetfillcolor{textcolor}%
\pgftext[x=0.149382in,y=1.830504in,left,base]{\color{textcolor}\sffamily\fontsize{10.000000}{12.000000}\selectfont −1.00}%
\end{pgfscope}%
\begin{pgfscope}%
\pgfsetbuttcap%
\pgfsetroundjoin%
\definecolor{currentfill}{rgb}{0.000000,0.000000,0.000000}%
\pgfsetfillcolor{currentfill}%
\pgfsetlinewidth{0.803000pt}%
\definecolor{currentstroke}{rgb}{0.000000,0.000000,0.000000}%
\pgfsetstrokecolor{currentstroke}%
\pgfsetdash{}{0pt}%
\pgfsys@defobject{currentmarker}{\pgfqpoint{-0.048611in}{0.000000in}}{\pgfqpoint{0.000000in}{0.000000in}}{%
\pgfpathmoveto{\pgfqpoint{0.000000in}{0.000000in}}%
\pgfpathlineto{\pgfqpoint{-0.048611in}{0.000000in}}%
\pgfusepath{stroke,fill}%
}%
\begin{pgfscope}%
\pgfsys@transformshift{0.672222in}{2.229530in}%
\pgfsys@useobject{currentmarker}{}%
\end{pgfscope}%
\end{pgfscope}%
\begin{pgfscope}%
\definecolor{textcolor}{rgb}{0.000000,0.000000,0.000000}%
\pgfsetstrokecolor{textcolor}%
\pgfsetfillcolor{textcolor}%
\pgftext[x=0.149382in,y=2.176768in,left,base]{\color{textcolor}\sffamily\fontsize{10.000000}{12.000000}\selectfont −0.75}%
\end{pgfscope}%
\begin{pgfscope}%
\pgfsetbuttcap%
\pgfsetroundjoin%
\definecolor{currentfill}{rgb}{0.000000,0.000000,0.000000}%
\pgfsetfillcolor{currentfill}%
\pgfsetlinewidth{0.803000pt}%
\definecolor{currentstroke}{rgb}{0.000000,0.000000,0.000000}%
\pgfsetstrokecolor{currentstroke}%
\pgfsetdash{}{0pt}%
\pgfsys@defobject{currentmarker}{\pgfqpoint{-0.048611in}{0.000000in}}{\pgfqpoint{0.000000in}{0.000000in}}{%
\pgfpathmoveto{\pgfqpoint{0.000000in}{0.000000in}}%
\pgfpathlineto{\pgfqpoint{-0.048611in}{0.000000in}}%
\pgfusepath{stroke,fill}%
}%
\begin{pgfscope}%
\pgfsys@transformshift{0.672222in}{2.575795in}%
\pgfsys@useobject{currentmarker}{}%
\end{pgfscope}%
\end{pgfscope}%
\begin{pgfscope}%
\definecolor{textcolor}{rgb}{0.000000,0.000000,0.000000}%
\pgfsetstrokecolor{textcolor}%
\pgfsetfillcolor{textcolor}%
\pgftext[x=0.149382in,y=2.523033in,left,base]{\color{textcolor}\sffamily\fontsize{10.000000}{12.000000}\selectfont −0.50}%
\end{pgfscope}%
\begin{pgfscope}%
\pgfpathrectangle{\pgfqpoint{0.672222in}{0.387222in}}{\pgfqpoint{3.217778in}{2.324444in}}%
\pgfusepath{clip}%
\pgfsetrectcap%
\pgfsetroundjoin%
\pgfsetlinewidth{1.505625pt}%
\definecolor{currentstroke}{rgb}{0.121569,0.466667,0.705882}%
\pgfsetstrokecolor{currentstroke}%
\pgfsetdash{}{0pt}%
\pgfpathmoveto{\pgfqpoint{0.826514in}{0.500910in}}%
\pgfpathlineto{\pgfqpoint{0.899244in}{0.500893in}}%
\pgfpathlineto{\pgfqpoint{0.971974in}{0.500808in}}%
\pgfpathlineto{\pgfqpoint{1.044704in}{0.500649in}}%
\pgfpathlineto{\pgfqpoint{1.117434in}{0.500917in}}%
\pgfpathlineto{\pgfqpoint{1.190163in}{0.502670in}}%
\pgfpathlineto{\pgfqpoint{1.262893in}{0.506806in}}%
\pgfpathlineto{\pgfqpoint{1.335623in}{0.514128in}}%
\pgfpathlineto{\pgfqpoint{1.408353in}{0.525615in}}%
\pgfpathlineto{\pgfqpoint{1.481083in}{0.542446in}}%
\pgfpathlineto{\pgfqpoint{1.553813in}{0.565948in}}%
\pgfpathlineto{\pgfqpoint{1.626543in}{0.597645in}}%
\pgfpathlineto{\pgfqpoint{1.699272in}{0.639248in}}%
\pgfpathlineto{\pgfqpoint{1.772002in}{0.693054in}}%
\pgfpathlineto{\pgfqpoint{1.844732in}{0.762029in}}%
\pgfpathlineto{\pgfqpoint{1.917462in}{0.850655in}}%
\pgfpathlineto{\pgfqpoint{1.990192in}{0.966371in}}%
\pgfpathlineto{\pgfqpoint{2.062922in}{1.126959in}}%
\pgfpathlineto{\pgfqpoint{2.135651in}{1.387383in}}%
\pgfpathlineto{\pgfqpoint{2.208381in}{1.601064in}}%
\pgfpathlineto{\pgfqpoint{2.281111in}{1.735716in}}%
\pgfpathlineto{\pgfqpoint{2.353841in}{1.843795in}}%
\pgfpathlineto{\pgfqpoint{2.426571in}{1.933482in}}%
\pgfpathlineto{\pgfqpoint{2.499301in}{2.010205in}}%
\pgfpathlineto{\pgfqpoint{2.572030in}{2.077568in}}%
\pgfpathlineto{\pgfqpoint{2.644760in}{2.136450in}}%
\pgfpathlineto{\pgfqpoint{2.717490in}{2.189010in}}%
\pgfpathlineto{\pgfqpoint{2.790220in}{2.236140in}}%
\pgfpathlineto{\pgfqpoint{2.862950in}{2.279300in}}%
\pgfpathlineto{\pgfqpoint{2.935680in}{2.318028in}}%
\pgfpathlineto{\pgfqpoint{3.008410in}{2.354015in}}%
\pgfpathlineto{\pgfqpoint{3.081139in}{2.386812in}}%
\pgfpathlineto{\pgfqpoint{3.153869in}{2.417338in}}%
\pgfpathlineto{\pgfqpoint{3.226599in}{2.445818in}}%
\pgfpathlineto{\pgfqpoint{3.299329in}{2.471736in}}%
\pgfpathlineto{\pgfqpoint{3.372059in}{2.496502in}}%
\pgfpathlineto{\pgfqpoint{3.444789in}{2.519380in}}%
\pgfpathlineto{\pgfqpoint{3.517518in}{2.541050in}}%
\pgfpathlineto{\pgfqpoint{3.590248in}{2.561253in}}%
\pgfpathlineto{\pgfqpoint{3.662978in}{2.580193in}}%
\pgfpathlineto{\pgfqpoint{3.735708in}{2.598240in}}%
\pgfusepath{stroke}%
\end{pgfscope}%
\begin{pgfscope}%
\pgfsetrectcap%
\pgfsetmiterjoin%
\pgfsetlinewidth{0.803000pt}%
\definecolor{currentstroke}{rgb}{0.000000,0.000000,0.000000}%
\pgfsetstrokecolor{currentstroke}%
\pgfsetdash{}{0pt}%
\pgfpathmoveto{\pgfqpoint{0.672222in}{0.387222in}}%
\pgfpathlineto{\pgfqpoint{0.672222in}{2.711667in}}%
\pgfusepath{stroke}%
\end{pgfscope}%
\begin{pgfscope}%
\pgfsetrectcap%
\pgfsetmiterjoin%
\pgfsetlinewidth{0.803000pt}%
\definecolor{currentstroke}{rgb}{0.000000,0.000000,0.000000}%
\pgfsetstrokecolor{currentstroke}%
\pgfsetdash{}{0pt}%
\pgfpathmoveto{\pgfqpoint{3.890000in}{0.387222in}}%
\pgfpathlineto{\pgfqpoint{3.890000in}{2.711667in}}%
\pgfusepath{stroke}%
\end{pgfscope}%
\begin{pgfscope}%
\pgfsetrectcap%
\pgfsetmiterjoin%
\pgfsetlinewidth{0.803000pt}%
\definecolor{currentstroke}{rgb}{0.000000,0.000000,0.000000}%
\pgfsetstrokecolor{currentstroke}%
\pgfsetdash{}{0pt}%
\pgfpathmoveto{\pgfqpoint{0.672222in}{0.387222in}}%
\pgfpathlineto{\pgfqpoint{3.890000in}{0.387222in}}%
\pgfusepath{stroke}%
\end{pgfscope}%
\begin{pgfscope}%
\pgfsetrectcap%
\pgfsetmiterjoin%
\pgfsetlinewidth{0.803000pt}%
\definecolor{currentstroke}{rgb}{0.000000,0.000000,0.000000}%
\pgfsetstrokecolor{currentstroke}%
\pgfsetdash{}{0pt}%
\pgfpathmoveto{\pgfqpoint{0.672222in}{2.711667in}}%
\pgfpathlineto{\pgfqpoint{3.890000in}{2.711667in}}%
\pgfusepath{stroke}%
\end{pgfscope}%
\begin{pgfscope}%
\definecolor{textcolor}{rgb}{0.000000,0.000000,0.000000}%
\pgfsetstrokecolor{textcolor}%
\pgfsetfillcolor{textcolor}%
\pgftext[x=2.281111in,y=2.795000in,,base]{\color{textcolor}\sffamily\fontsize{12.000000}{14.400000}\selectfont \(\displaystyle  N = 64 \)}%
\end{pgfscope}%
\begin{pgfscope}%
\pgfsetbuttcap%
\pgfsetmiterjoin%
\definecolor{currentfill}{rgb}{1.000000,1.000000,1.000000}%
\pgfsetfillcolor{currentfill}%
\pgfsetlinewidth{0.000000pt}%
\definecolor{currentstroke}{rgb}{0.000000,0.000000,0.000000}%
\pgfsetstrokecolor{currentstroke}%
\pgfsetstrokeopacity{0.000000}%
\pgfsetdash{}{0pt}%
\pgfpathmoveto{\pgfqpoint{4.597222in}{0.387222in}}%
\pgfpathlineto{\pgfqpoint{7.815000in}{0.387222in}}%
\pgfpathlineto{\pgfqpoint{7.815000in}{2.711667in}}%
\pgfpathlineto{\pgfqpoint{4.597222in}{2.711667in}}%
\pgfpathclose%
\pgfusepath{fill}%
\end{pgfscope}%
\begin{pgfscope}%
\pgfpathrectangle{\pgfqpoint{4.597222in}{0.387222in}}{\pgfqpoint{3.217778in}{2.324444in}}%
\pgfusepath{clip}%
\pgfsetbuttcap%
\pgfsetroundjoin%
\definecolor{currentfill}{rgb}{0.121569,0.466667,0.705882}%
\pgfsetfillcolor{currentfill}%
\pgfsetfillopacity{0.300000}%
\pgfsetlinewidth{1.003750pt}%
\definecolor{currentstroke}{rgb}{0.121569,0.466667,0.705882}%
\pgfsetstrokecolor{currentstroke}%
\pgfsetstrokeopacity{0.300000}%
\pgfsetdash{}{0pt}%
\pgfpathmoveto{\pgfqpoint{4.751514in}{0.500714in}}%
\pgfpathlineto{\pgfqpoint{4.751514in}{0.500714in}}%
\pgfpathlineto{\pgfqpoint{4.824244in}{0.500691in}}%
\pgfpathlineto{\pgfqpoint{4.896974in}{0.500649in}}%
\pgfpathlineto{\pgfqpoint{4.969704in}{0.500782in}}%
\pgfpathlineto{\pgfqpoint{5.042434in}{0.501702in}}%
\pgfpathlineto{\pgfqpoint{5.115163in}{0.504000in}}%
\pgfpathlineto{\pgfqpoint{5.187893in}{0.508398in}}%
\pgfpathlineto{\pgfqpoint{5.260623in}{0.515379in}}%
\pgfpathlineto{\pgfqpoint{5.333353in}{0.527350in}}%
\pgfpathlineto{\pgfqpoint{5.406083in}{0.544180in}}%
\pgfpathlineto{\pgfqpoint{5.478813in}{0.567654in}}%
\pgfpathlineto{\pgfqpoint{5.551543in}{0.571409in}}%
\pgfpathlineto{\pgfqpoint{5.624272in}{0.623198in}}%
\pgfpathlineto{\pgfqpoint{5.697002in}{0.694667in}}%
\pgfpathlineto{\pgfqpoint{5.769732in}{0.752069in}}%
\pgfpathlineto{\pgfqpoint{5.842462in}{0.851834in}}%
\pgfpathlineto{\pgfqpoint{5.915192in}{0.965631in}}%
\pgfpathlineto{\pgfqpoint{5.987922in}{1.128058in}}%
\pgfpathlineto{\pgfqpoint{6.060651in}{1.395793in}}%
\pgfpathlineto{\pgfqpoint{6.133381in}{1.601215in}}%
\pgfpathlineto{\pgfqpoint{6.206111in}{1.736345in}}%
\pgfpathlineto{\pgfqpoint{6.278841in}{1.843839in}}%
\pgfpathlineto{\pgfqpoint{6.351571in}{1.933876in}}%
\pgfpathlineto{\pgfqpoint{6.424301in}{2.010334in}}%
\pgfpathlineto{\pgfqpoint{6.497030in}{2.077507in}}%
\pgfpathlineto{\pgfqpoint{6.569760in}{2.136421in}}%
\pgfpathlineto{\pgfqpoint{6.642490in}{2.189165in}}%
\pgfpathlineto{\pgfqpoint{6.715220in}{2.236087in}}%
\pgfpathlineto{\pgfqpoint{6.787950in}{2.279022in}}%
\pgfpathlineto{\pgfqpoint{6.860680in}{2.318030in}}%
\pgfpathlineto{\pgfqpoint{6.933410in}{2.353987in}}%
\pgfpathlineto{\pgfqpoint{7.006139in}{2.386665in}}%
\pgfpathlineto{\pgfqpoint{7.078869in}{2.417294in}}%
\pgfpathlineto{\pgfqpoint{7.151599in}{2.445495in}}%
\pgfpathlineto{\pgfqpoint{7.224329in}{2.471520in}}%
\pgfpathlineto{\pgfqpoint{7.297059in}{2.496312in}}%
\pgfpathlineto{\pgfqpoint{7.369789in}{2.519097in}}%
\pgfpathlineto{\pgfqpoint{7.442518in}{2.540592in}}%
\pgfpathlineto{\pgfqpoint{7.515248in}{2.560834in}}%
\pgfpathlineto{\pgfqpoint{7.587978in}{2.580100in}}%
\pgfpathlineto{\pgfqpoint{7.660708in}{2.598043in}}%
\pgfpathlineto{\pgfqpoint{7.660708in}{2.598437in}}%
\pgfpathlineto{\pgfqpoint{7.660708in}{2.598437in}}%
\pgfpathlineto{\pgfqpoint{7.587978in}{2.580470in}}%
\pgfpathlineto{\pgfqpoint{7.515248in}{2.561462in}}%
\pgfpathlineto{\pgfqpoint{7.442518in}{2.541155in}}%
\pgfpathlineto{\pgfqpoint{7.369789in}{2.519632in}}%
\pgfpathlineto{\pgfqpoint{7.297059in}{2.496990in}}%
\pgfpathlineto{\pgfqpoint{7.224329in}{2.472038in}}%
\pgfpathlineto{\pgfqpoint{7.151599in}{2.445781in}}%
\pgfpathlineto{\pgfqpoint{7.078869in}{2.417885in}}%
\pgfpathlineto{\pgfqpoint{7.006139in}{2.387286in}}%
\pgfpathlineto{\pgfqpoint{6.933410in}{2.354307in}}%
\pgfpathlineto{\pgfqpoint{6.860680in}{2.318593in}}%
\pgfpathlineto{\pgfqpoint{6.787950in}{2.279340in}}%
\pgfpathlineto{\pgfqpoint{6.715220in}{2.236807in}}%
\pgfpathlineto{\pgfqpoint{6.642490in}{2.189283in}}%
\pgfpathlineto{\pgfqpoint{6.569760in}{2.136876in}}%
\pgfpathlineto{\pgfqpoint{6.497030in}{2.078163in}}%
\pgfpathlineto{\pgfqpoint{6.424301in}{2.011289in}}%
\pgfpathlineto{\pgfqpoint{6.351571in}{1.934491in}}%
\pgfpathlineto{\pgfqpoint{6.278841in}{1.846023in}}%
\pgfpathlineto{\pgfqpoint{6.206111in}{1.737963in}}%
\pgfpathlineto{\pgfqpoint{6.133381in}{1.602155in}}%
\pgfpathlineto{\pgfqpoint{6.060651in}{1.414317in}}%
\pgfpathlineto{\pgfqpoint{5.987922in}{1.128557in}}%
\pgfpathlineto{\pgfqpoint{5.915192in}{0.970800in}}%
\pgfpathlineto{\pgfqpoint{5.842462in}{0.852223in}}%
\pgfpathlineto{\pgfqpoint{5.769732in}{0.789723in}}%
\pgfpathlineto{\pgfqpoint{5.697002in}{0.694851in}}%
\pgfpathlineto{\pgfqpoint{5.624272in}{0.681051in}}%
\pgfpathlineto{\pgfqpoint{5.551543in}{0.662144in}}%
\pgfpathlineto{\pgfqpoint{5.478813in}{0.567694in}}%
\pgfpathlineto{\pgfqpoint{5.406083in}{0.544220in}}%
\pgfpathlineto{\pgfqpoint{5.333353in}{0.527363in}}%
\pgfpathlineto{\pgfqpoint{5.260623in}{0.516839in}}%
\pgfpathlineto{\pgfqpoint{5.187893in}{0.508403in}}%
\pgfpathlineto{\pgfqpoint{5.115163in}{0.504001in}}%
\pgfpathlineto{\pgfqpoint{5.042434in}{0.501702in}}%
\pgfpathlineto{\pgfqpoint{4.969704in}{0.500782in}}%
\pgfpathlineto{\pgfqpoint{4.896974in}{0.500649in}}%
\pgfpathlineto{\pgfqpoint{4.824244in}{0.500717in}}%
\pgfpathlineto{\pgfqpoint{4.751514in}{0.500714in}}%
\pgfpathclose%
\pgfusepath{stroke,fill}%
\end{pgfscope}%
\begin{pgfscope}%
\pgfpathrectangle{\pgfqpoint{4.597222in}{0.387222in}}{\pgfqpoint{3.217778in}{2.324444in}}%
\pgfusepath{clip}%
\pgfsetbuttcap%
\pgfsetroundjoin%
\definecolor{currentfill}{rgb}{0.121569,0.466667,0.705882}%
\pgfsetfillcolor{currentfill}%
\pgfsetlinewidth{1.003750pt}%
\definecolor{currentstroke}{rgb}{0.121569,0.466667,0.705882}%
\pgfsetstrokecolor{currentstroke}%
\pgfsetdash{}{0pt}%
\pgfsys@defobject{currentmarker}{\pgfqpoint{-0.009821in}{-0.009821in}}{\pgfqpoint{0.009821in}{0.009821in}}{%
\pgfpathmoveto{\pgfqpoint{0.000000in}{-0.009821in}}%
\pgfpathcurveto{\pgfqpoint{0.002605in}{-0.009821in}}{\pgfqpoint{0.005103in}{-0.008786in}}{\pgfqpoint{0.006944in}{-0.006944in}}%
\pgfpathcurveto{\pgfqpoint{0.008786in}{-0.005103in}}{\pgfqpoint{0.009821in}{-0.002605in}}{\pgfqpoint{0.009821in}{0.000000in}}%
\pgfpathcurveto{\pgfqpoint{0.009821in}{0.002605in}}{\pgfqpoint{0.008786in}{0.005103in}}{\pgfqpoint{0.006944in}{0.006944in}}%
\pgfpathcurveto{\pgfqpoint{0.005103in}{0.008786in}}{\pgfqpoint{0.002605in}{0.009821in}}{\pgfqpoint{0.000000in}{0.009821in}}%
\pgfpathcurveto{\pgfqpoint{-0.002605in}{0.009821in}}{\pgfqpoint{-0.005103in}{0.008786in}}{\pgfqpoint{-0.006944in}{0.006944in}}%
\pgfpathcurveto{\pgfqpoint{-0.008786in}{0.005103in}}{\pgfqpoint{-0.009821in}{0.002605in}}{\pgfqpoint{-0.009821in}{0.000000in}}%
\pgfpathcurveto{\pgfqpoint{-0.009821in}{-0.002605in}}{\pgfqpoint{-0.008786in}{-0.005103in}}{\pgfqpoint{-0.006944in}{-0.006944in}}%
\pgfpathcurveto{\pgfqpoint{-0.005103in}{-0.008786in}}{\pgfqpoint{-0.002605in}{-0.009821in}}{\pgfqpoint{0.000000in}{-0.009821in}}%
\pgfpathclose%
\pgfusepath{stroke,fill}%
}%
\begin{pgfscope}%
\pgfsys@transformshift{4.751514in}{0.500714in}%
\pgfsys@useobject{currentmarker}{}%
\end{pgfscope}%
\begin{pgfscope}%
\pgfsys@transformshift{4.824244in}{0.500704in}%
\pgfsys@useobject{currentmarker}{}%
\end{pgfscope}%
\begin{pgfscope}%
\pgfsys@transformshift{4.896974in}{0.500649in}%
\pgfsys@useobject{currentmarker}{}%
\end{pgfscope}%
\begin{pgfscope}%
\pgfsys@transformshift{4.969704in}{0.500782in}%
\pgfsys@useobject{currentmarker}{}%
\end{pgfscope}%
\begin{pgfscope}%
\pgfsys@transformshift{5.042434in}{0.501702in}%
\pgfsys@useobject{currentmarker}{}%
\end{pgfscope}%
\begin{pgfscope}%
\pgfsys@transformshift{5.115163in}{0.504001in}%
\pgfsys@useobject{currentmarker}{}%
\end{pgfscope}%
\begin{pgfscope}%
\pgfsys@transformshift{5.187893in}{0.508400in}%
\pgfsys@useobject{currentmarker}{}%
\end{pgfscope}%
\begin{pgfscope}%
\pgfsys@transformshift{5.260623in}{0.516109in}%
\pgfsys@useobject{currentmarker}{}%
\end{pgfscope}%
\begin{pgfscope}%
\pgfsys@transformshift{5.333353in}{0.527356in}%
\pgfsys@useobject{currentmarker}{}%
\end{pgfscope}%
\begin{pgfscope}%
\pgfsys@transformshift{5.406083in}{0.544200in}%
\pgfsys@useobject{currentmarker}{}%
\end{pgfscope}%
\begin{pgfscope}%
\pgfsys@transformshift{5.478813in}{0.567674in}%
\pgfsys@useobject{currentmarker}{}%
\end{pgfscope}%
\begin{pgfscope}%
\pgfsys@transformshift{5.551543in}{0.616776in}%
\pgfsys@useobject{currentmarker}{}%
\end{pgfscope}%
\begin{pgfscope}%
\pgfsys@transformshift{5.624272in}{0.652125in}%
\pgfsys@useobject{currentmarker}{}%
\end{pgfscope}%
\begin{pgfscope}%
\pgfsys@transformshift{5.697002in}{0.694759in}%
\pgfsys@useobject{currentmarker}{}%
\end{pgfscope}%
\begin{pgfscope}%
\pgfsys@transformshift{5.769732in}{0.770896in}%
\pgfsys@useobject{currentmarker}{}%
\end{pgfscope}%
\begin{pgfscope}%
\pgfsys@transformshift{5.842462in}{0.852028in}%
\pgfsys@useobject{currentmarker}{}%
\end{pgfscope}%
\begin{pgfscope}%
\pgfsys@transformshift{5.915192in}{0.968215in}%
\pgfsys@useobject{currentmarker}{}%
\end{pgfscope}%
\begin{pgfscope}%
\pgfsys@transformshift{5.987922in}{1.128307in}%
\pgfsys@useobject{currentmarker}{}%
\end{pgfscope}%
\begin{pgfscope}%
\pgfsys@transformshift{6.060651in}{1.405055in}%
\pgfsys@useobject{currentmarker}{}%
\end{pgfscope}%
\begin{pgfscope}%
\pgfsys@transformshift{6.133381in}{1.601685in}%
\pgfsys@useobject{currentmarker}{}%
\end{pgfscope}%
\begin{pgfscope}%
\pgfsys@transformshift{6.206111in}{1.737154in}%
\pgfsys@useobject{currentmarker}{}%
\end{pgfscope}%
\begin{pgfscope}%
\pgfsys@transformshift{6.278841in}{1.844931in}%
\pgfsys@useobject{currentmarker}{}%
\end{pgfscope}%
\begin{pgfscope}%
\pgfsys@transformshift{6.351571in}{1.934184in}%
\pgfsys@useobject{currentmarker}{}%
\end{pgfscope}%
\begin{pgfscope}%
\pgfsys@transformshift{6.424301in}{2.010812in}%
\pgfsys@useobject{currentmarker}{}%
\end{pgfscope}%
\begin{pgfscope}%
\pgfsys@transformshift{6.497030in}{2.077835in}%
\pgfsys@useobject{currentmarker}{}%
\end{pgfscope}%
\begin{pgfscope}%
\pgfsys@transformshift{6.569760in}{2.136648in}%
\pgfsys@useobject{currentmarker}{}%
\end{pgfscope}%
\begin{pgfscope}%
\pgfsys@transformshift{6.642490in}{2.189224in}%
\pgfsys@useobject{currentmarker}{}%
\end{pgfscope}%
\begin{pgfscope}%
\pgfsys@transformshift{6.715220in}{2.236447in}%
\pgfsys@useobject{currentmarker}{}%
\end{pgfscope}%
\begin{pgfscope}%
\pgfsys@transformshift{6.787950in}{2.279181in}%
\pgfsys@useobject{currentmarker}{}%
\end{pgfscope}%
\begin{pgfscope}%
\pgfsys@transformshift{6.860680in}{2.318312in}%
\pgfsys@useobject{currentmarker}{}%
\end{pgfscope}%
\begin{pgfscope}%
\pgfsys@transformshift{6.933410in}{2.354147in}%
\pgfsys@useobject{currentmarker}{}%
\end{pgfscope}%
\begin{pgfscope}%
\pgfsys@transformshift{7.006139in}{2.386976in}%
\pgfsys@useobject{currentmarker}{}%
\end{pgfscope}%
\begin{pgfscope}%
\pgfsys@transformshift{7.078869in}{2.417590in}%
\pgfsys@useobject{currentmarker}{}%
\end{pgfscope}%
\begin{pgfscope}%
\pgfsys@transformshift{7.151599in}{2.445638in}%
\pgfsys@useobject{currentmarker}{}%
\end{pgfscope}%
\begin{pgfscope}%
\pgfsys@transformshift{7.224329in}{2.471779in}%
\pgfsys@useobject{currentmarker}{}%
\end{pgfscope}%
\begin{pgfscope}%
\pgfsys@transformshift{7.297059in}{2.496651in}%
\pgfsys@useobject{currentmarker}{}%
\end{pgfscope}%
\begin{pgfscope}%
\pgfsys@transformshift{7.369789in}{2.519365in}%
\pgfsys@useobject{currentmarker}{}%
\end{pgfscope}%
\begin{pgfscope}%
\pgfsys@transformshift{7.442518in}{2.540874in}%
\pgfsys@useobject{currentmarker}{}%
\end{pgfscope}%
\begin{pgfscope}%
\pgfsys@transformshift{7.515248in}{2.561148in}%
\pgfsys@useobject{currentmarker}{}%
\end{pgfscope}%
\begin{pgfscope}%
\pgfsys@transformshift{7.587978in}{2.580285in}%
\pgfsys@useobject{currentmarker}{}%
\end{pgfscope}%
\begin{pgfscope}%
\pgfsys@transformshift{7.660708in}{2.598240in}%
\pgfsys@useobject{currentmarker}{}%
\end{pgfscope}%
\end{pgfscope}%
\begin{pgfscope}%
\pgfsetbuttcap%
\pgfsetroundjoin%
\definecolor{currentfill}{rgb}{0.000000,0.000000,0.000000}%
\pgfsetfillcolor{currentfill}%
\pgfsetlinewidth{0.803000pt}%
\definecolor{currentstroke}{rgb}{0.000000,0.000000,0.000000}%
\pgfsetstrokecolor{currentstroke}%
\pgfsetdash{}{0pt}%
\pgfsys@defobject{currentmarker}{\pgfqpoint{0.000000in}{-0.048611in}}{\pgfqpoint{0.000000in}{0.000000in}}{%
\pgfpathmoveto{\pgfqpoint{0.000000in}{0.000000in}}%
\pgfpathlineto{\pgfqpoint{0.000000in}{-0.048611in}}%
\pgfusepath{stroke,fill}%
}%
\begin{pgfscope}%
\pgfsys@transformshift{5.115163in}{0.387222in}%
\pgfsys@useobject{currentmarker}{}%
\end{pgfscope}%
\end{pgfscope}%
\begin{pgfscope}%
\definecolor{textcolor}{rgb}{0.000000,0.000000,0.000000}%
\pgfsetstrokecolor{textcolor}%
\pgfsetfillcolor{textcolor}%
\pgftext[x=5.115163in,y=0.290000in,,top]{\color{textcolor}\sffamily\fontsize{10.000000}{12.000000}\selectfont 1}%
\end{pgfscope}%
\begin{pgfscope}%
\pgfsetbuttcap%
\pgfsetroundjoin%
\definecolor{currentfill}{rgb}{0.000000,0.000000,0.000000}%
\pgfsetfillcolor{currentfill}%
\pgfsetlinewidth{0.803000pt}%
\definecolor{currentstroke}{rgb}{0.000000,0.000000,0.000000}%
\pgfsetstrokecolor{currentstroke}%
\pgfsetdash{}{0pt}%
\pgfsys@defobject{currentmarker}{\pgfqpoint{0.000000in}{-0.048611in}}{\pgfqpoint{0.000000in}{0.000000in}}{%
\pgfpathmoveto{\pgfqpoint{0.000000in}{0.000000in}}%
\pgfpathlineto{\pgfqpoint{0.000000in}{-0.048611in}}%
\pgfusepath{stroke,fill}%
}%
\begin{pgfscope}%
\pgfsys@transformshift{5.842462in}{0.387222in}%
\pgfsys@useobject{currentmarker}{}%
\end{pgfscope}%
\end{pgfscope}%
\begin{pgfscope}%
\definecolor{textcolor}{rgb}{0.000000,0.000000,0.000000}%
\pgfsetstrokecolor{textcolor}%
\pgfsetfillcolor{textcolor}%
\pgftext[x=5.842462in,y=0.290000in,,top]{\color{textcolor}\sffamily\fontsize{10.000000}{12.000000}\selectfont 2}%
\end{pgfscope}%
\begin{pgfscope}%
\pgfsetbuttcap%
\pgfsetroundjoin%
\definecolor{currentfill}{rgb}{0.000000,0.000000,0.000000}%
\pgfsetfillcolor{currentfill}%
\pgfsetlinewidth{0.803000pt}%
\definecolor{currentstroke}{rgb}{0.000000,0.000000,0.000000}%
\pgfsetstrokecolor{currentstroke}%
\pgfsetdash{}{0pt}%
\pgfsys@defobject{currentmarker}{\pgfqpoint{0.000000in}{-0.048611in}}{\pgfqpoint{0.000000in}{0.000000in}}{%
\pgfpathmoveto{\pgfqpoint{0.000000in}{0.000000in}}%
\pgfpathlineto{\pgfqpoint{0.000000in}{-0.048611in}}%
\pgfusepath{stroke,fill}%
}%
\begin{pgfscope}%
\pgfsys@transformshift{6.569760in}{0.387222in}%
\pgfsys@useobject{currentmarker}{}%
\end{pgfscope}%
\end{pgfscope}%
\begin{pgfscope}%
\definecolor{textcolor}{rgb}{0.000000,0.000000,0.000000}%
\pgfsetstrokecolor{textcolor}%
\pgfsetfillcolor{textcolor}%
\pgftext[x=6.569760in,y=0.290000in,,top]{\color{textcolor}\sffamily\fontsize{10.000000}{12.000000}\selectfont 3}%
\end{pgfscope}%
\begin{pgfscope}%
\pgfsetbuttcap%
\pgfsetroundjoin%
\definecolor{currentfill}{rgb}{0.000000,0.000000,0.000000}%
\pgfsetfillcolor{currentfill}%
\pgfsetlinewidth{0.803000pt}%
\definecolor{currentstroke}{rgb}{0.000000,0.000000,0.000000}%
\pgfsetstrokecolor{currentstroke}%
\pgfsetdash{}{0pt}%
\pgfsys@defobject{currentmarker}{\pgfqpoint{0.000000in}{-0.048611in}}{\pgfqpoint{0.000000in}{0.000000in}}{%
\pgfpathmoveto{\pgfqpoint{0.000000in}{0.000000in}}%
\pgfpathlineto{\pgfqpoint{0.000000in}{-0.048611in}}%
\pgfusepath{stroke,fill}%
}%
\begin{pgfscope}%
\pgfsys@transformshift{7.297059in}{0.387222in}%
\pgfsys@useobject{currentmarker}{}%
\end{pgfscope}%
\end{pgfscope}%
\begin{pgfscope}%
\definecolor{textcolor}{rgb}{0.000000,0.000000,0.000000}%
\pgfsetstrokecolor{textcolor}%
\pgfsetfillcolor{textcolor}%
\pgftext[x=7.297059in,y=0.290000in,,top]{\color{textcolor}\sffamily\fontsize{10.000000}{12.000000}\selectfont 4}%
\end{pgfscope}%
\begin{pgfscope}%
\pgfsetbuttcap%
\pgfsetroundjoin%
\definecolor{currentfill}{rgb}{0.000000,0.000000,0.000000}%
\pgfsetfillcolor{currentfill}%
\pgfsetlinewidth{0.803000pt}%
\definecolor{currentstroke}{rgb}{0.000000,0.000000,0.000000}%
\pgfsetstrokecolor{currentstroke}%
\pgfsetdash{}{0pt}%
\pgfsys@defobject{currentmarker}{\pgfqpoint{-0.048611in}{0.000000in}}{\pgfqpoint{0.000000in}{0.000000in}}{%
\pgfpathmoveto{\pgfqpoint{0.000000in}{0.000000in}}%
\pgfpathlineto{\pgfqpoint{-0.048611in}{0.000000in}}%
\pgfusepath{stroke,fill}%
}%
\begin{pgfscope}%
\pgfsys@transformshift{4.597222in}{0.500039in}%
\pgfsys@useobject{currentmarker}{}%
\end{pgfscope}%
\end{pgfscope}%
\begin{pgfscope}%
\definecolor{textcolor}{rgb}{0.000000,0.000000,0.000000}%
\pgfsetstrokecolor{textcolor}%
\pgfsetfillcolor{textcolor}%
\pgftext[x=4.074382in,y=0.447278in,left,base]{\color{textcolor}\sffamily\fontsize{10.000000}{12.000000}\selectfont −2.00}%
\end{pgfscope}%
\begin{pgfscope}%
\pgfsetbuttcap%
\pgfsetroundjoin%
\definecolor{currentfill}{rgb}{0.000000,0.000000,0.000000}%
\pgfsetfillcolor{currentfill}%
\pgfsetlinewidth{0.803000pt}%
\definecolor{currentstroke}{rgb}{0.000000,0.000000,0.000000}%
\pgfsetstrokecolor{currentstroke}%
\pgfsetdash{}{0pt}%
\pgfsys@defobject{currentmarker}{\pgfqpoint{-0.048611in}{0.000000in}}{\pgfqpoint{0.000000in}{0.000000in}}{%
\pgfpathmoveto{\pgfqpoint{0.000000in}{0.000000in}}%
\pgfpathlineto{\pgfqpoint{-0.048611in}{0.000000in}}%
\pgfusepath{stroke,fill}%
}%
\begin{pgfscope}%
\pgfsys@transformshift{4.597222in}{0.845988in}%
\pgfsys@useobject{currentmarker}{}%
\end{pgfscope}%
\end{pgfscope}%
\begin{pgfscope}%
\definecolor{textcolor}{rgb}{0.000000,0.000000,0.000000}%
\pgfsetstrokecolor{textcolor}%
\pgfsetfillcolor{textcolor}%
\pgftext[x=4.074382in,y=0.793227in,left,base]{\color{textcolor}\sffamily\fontsize{10.000000}{12.000000}\selectfont −1.75}%
\end{pgfscope}%
\begin{pgfscope}%
\pgfsetbuttcap%
\pgfsetroundjoin%
\definecolor{currentfill}{rgb}{0.000000,0.000000,0.000000}%
\pgfsetfillcolor{currentfill}%
\pgfsetlinewidth{0.803000pt}%
\definecolor{currentstroke}{rgb}{0.000000,0.000000,0.000000}%
\pgfsetstrokecolor{currentstroke}%
\pgfsetdash{}{0pt}%
\pgfsys@defobject{currentmarker}{\pgfqpoint{-0.048611in}{0.000000in}}{\pgfqpoint{0.000000in}{0.000000in}}{%
\pgfpathmoveto{\pgfqpoint{0.000000in}{0.000000in}}%
\pgfpathlineto{\pgfqpoint{-0.048611in}{0.000000in}}%
\pgfusepath{stroke,fill}%
}%
\begin{pgfscope}%
\pgfsys@transformshift{4.597222in}{1.191937in}%
\pgfsys@useobject{currentmarker}{}%
\end{pgfscope}%
\end{pgfscope}%
\begin{pgfscope}%
\definecolor{textcolor}{rgb}{0.000000,0.000000,0.000000}%
\pgfsetstrokecolor{textcolor}%
\pgfsetfillcolor{textcolor}%
\pgftext[x=4.074382in,y=1.139176in,left,base]{\color{textcolor}\sffamily\fontsize{10.000000}{12.000000}\selectfont −1.50}%
\end{pgfscope}%
\begin{pgfscope}%
\pgfsetbuttcap%
\pgfsetroundjoin%
\definecolor{currentfill}{rgb}{0.000000,0.000000,0.000000}%
\pgfsetfillcolor{currentfill}%
\pgfsetlinewidth{0.803000pt}%
\definecolor{currentstroke}{rgb}{0.000000,0.000000,0.000000}%
\pgfsetstrokecolor{currentstroke}%
\pgfsetdash{}{0pt}%
\pgfsys@defobject{currentmarker}{\pgfqpoint{-0.048611in}{0.000000in}}{\pgfqpoint{0.000000in}{0.000000in}}{%
\pgfpathmoveto{\pgfqpoint{0.000000in}{0.000000in}}%
\pgfpathlineto{\pgfqpoint{-0.048611in}{0.000000in}}%
\pgfusepath{stroke,fill}%
}%
\begin{pgfscope}%
\pgfsys@transformshift{4.597222in}{1.537886in}%
\pgfsys@useobject{currentmarker}{}%
\end{pgfscope}%
\end{pgfscope}%
\begin{pgfscope}%
\definecolor{textcolor}{rgb}{0.000000,0.000000,0.000000}%
\pgfsetstrokecolor{textcolor}%
\pgfsetfillcolor{textcolor}%
\pgftext[x=4.074382in,y=1.485124in,left,base]{\color{textcolor}\sffamily\fontsize{10.000000}{12.000000}\selectfont −1.25}%
\end{pgfscope}%
\begin{pgfscope}%
\pgfsetbuttcap%
\pgfsetroundjoin%
\definecolor{currentfill}{rgb}{0.000000,0.000000,0.000000}%
\pgfsetfillcolor{currentfill}%
\pgfsetlinewidth{0.803000pt}%
\definecolor{currentstroke}{rgb}{0.000000,0.000000,0.000000}%
\pgfsetstrokecolor{currentstroke}%
\pgfsetdash{}{0pt}%
\pgfsys@defobject{currentmarker}{\pgfqpoint{-0.048611in}{0.000000in}}{\pgfqpoint{0.000000in}{0.000000in}}{%
\pgfpathmoveto{\pgfqpoint{0.000000in}{0.000000in}}%
\pgfpathlineto{\pgfqpoint{-0.048611in}{0.000000in}}%
\pgfusepath{stroke,fill}%
}%
\begin{pgfscope}%
\pgfsys@transformshift{4.597222in}{1.883835in}%
\pgfsys@useobject{currentmarker}{}%
\end{pgfscope}%
\end{pgfscope}%
\begin{pgfscope}%
\definecolor{textcolor}{rgb}{0.000000,0.000000,0.000000}%
\pgfsetstrokecolor{textcolor}%
\pgfsetfillcolor{textcolor}%
\pgftext[x=4.074382in,y=1.831073in,left,base]{\color{textcolor}\sffamily\fontsize{10.000000}{12.000000}\selectfont −1.00}%
\end{pgfscope}%
\begin{pgfscope}%
\pgfsetbuttcap%
\pgfsetroundjoin%
\definecolor{currentfill}{rgb}{0.000000,0.000000,0.000000}%
\pgfsetfillcolor{currentfill}%
\pgfsetlinewidth{0.803000pt}%
\definecolor{currentstroke}{rgb}{0.000000,0.000000,0.000000}%
\pgfsetstrokecolor{currentstroke}%
\pgfsetdash{}{0pt}%
\pgfsys@defobject{currentmarker}{\pgfqpoint{-0.048611in}{0.000000in}}{\pgfqpoint{0.000000in}{0.000000in}}{%
\pgfpathmoveto{\pgfqpoint{0.000000in}{0.000000in}}%
\pgfpathlineto{\pgfqpoint{-0.048611in}{0.000000in}}%
\pgfusepath{stroke,fill}%
}%
\begin{pgfscope}%
\pgfsys@transformshift{4.597222in}{2.229783in}%
\pgfsys@useobject{currentmarker}{}%
\end{pgfscope}%
\end{pgfscope}%
\begin{pgfscope}%
\definecolor{textcolor}{rgb}{0.000000,0.000000,0.000000}%
\pgfsetstrokecolor{textcolor}%
\pgfsetfillcolor{textcolor}%
\pgftext[x=4.074382in,y=2.177022in,left,base]{\color{textcolor}\sffamily\fontsize{10.000000}{12.000000}\selectfont −0.75}%
\end{pgfscope}%
\begin{pgfscope}%
\pgfsetbuttcap%
\pgfsetroundjoin%
\definecolor{currentfill}{rgb}{0.000000,0.000000,0.000000}%
\pgfsetfillcolor{currentfill}%
\pgfsetlinewidth{0.803000pt}%
\definecolor{currentstroke}{rgb}{0.000000,0.000000,0.000000}%
\pgfsetstrokecolor{currentstroke}%
\pgfsetdash{}{0pt}%
\pgfsys@defobject{currentmarker}{\pgfqpoint{-0.048611in}{0.000000in}}{\pgfqpoint{0.000000in}{0.000000in}}{%
\pgfpathmoveto{\pgfqpoint{0.000000in}{0.000000in}}%
\pgfpathlineto{\pgfqpoint{-0.048611in}{0.000000in}}%
\pgfusepath{stroke,fill}%
}%
\begin{pgfscope}%
\pgfsys@transformshift{4.597222in}{2.575732in}%
\pgfsys@useobject{currentmarker}{}%
\end{pgfscope}%
\end{pgfscope}%
\begin{pgfscope}%
\definecolor{textcolor}{rgb}{0.000000,0.000000,0.000000}%
\pgfsetstrokecolor{textcolor}%
\pgfsetfillcolor{textcolor}%
\pgftext[x=4.074382in,y=2.522971in,left,base]{\color{textcolor}\sffamily\fontsize{10.000000}{12.000000}\selectfont −0.50}%
\end{pgfscope}%
\begin{pgfscope}%
\pgfpathrectangle{\pgfqpoint{4.597222in}{0.387222in}}{\pgfqpoint{3.217778in}{2.324444in}}%
\pgfusepath{clip}%
\pgfsetrectcap%
\pgfsetroundjoin%
\pgfsetlinewidth{1.505625pt}%
\definecolor{currentstroke}{rgb}{0.121569,0.466667,0.705882}%
\pgfsetstrokecolor{currentstroke}%
\pgfsetdash{}{0pt}%
\pgfpathmoveto{\pgfqpoint{4.751514in}{0.500714in}}%
\pgfpathlineto{\pgfqpoint{4.824244in}{0.500704in}}%
\pgfpathlineto{\pgfqpoint{4.896974in}{0.500649in}}%
\pgfpathlineto{\pgfqpoint{4.969704in}{0.500782in}}%
\pgfpathlineto{\pgfqpoint{5.042434in}{0.501702in}}%
\pgfpathlineto{\pgfqpoint{5.115163in}{0.504001in}}%
\pgfpathlineto{\pgfqpoint{5.187893in}{0.508400in}}%
\pgfpathlineto{\pgfqpoint{5.260623in}{0.516109in}}%
\pgfpathlineto{\pgfqpoint{5.333353in}{0.527356in}}%
\pgfpathlineto{\pgfqpoint{5.406083in}{0.544200in}}%
\pgfpathlineto{\pgfqpoint{5.478813in}{0.567674in}}%
\pgfpathlineto{\pgfqpoint{5.551543in}{0.616776in}}%
\pgfpathlineto{\pgfqpoint{5.624272in}{0.652125in}}%
\pgfpathlineto{\pgfqpoint{5.697002in}{0.694759in}}%
\pgfpathlineto{\pgfqpoint{5.769732in}{0.770896in}}%
\pgfpathlineto{\pgfqpoint{5.842462in}{0.852028in}}%
\pgfpathlineto{\pgfqpoint{5.915192in}{0.968215in}}%
\pgfpathlineto{\pgfqpoint{5.987922in}{1.128307in}}%
\pgfpathlineto{\pgfqpoint{6.060651in}{1.405055in}}%
\pgfpathlineto{\pgfqpoint{6.133381in}{1.601685in}}%
\pgfpathlineto{\pgfqpoint{6.206111in}{1.737154in}}%
\pgfpathlineto{\pgfqpoint{6.278841in}{1.844931in}}%
\pgfpathlineto{\pgfqpoint{6.351571in}{1.934184in}}%
\pgfpathlineto{\pgfqpoint{6.424301in}{2.010812in}}%
\pgfpathlineto{\pgfqpoint{6.497030in}{2.077835in}}%
\pgfpathlineto{\pgfqpoint{6.569760in}{2.136648in}}%
\pgfpathlineto{\pgfqpoint{6.642490in}{2.189224in}}%
\pgfpathlineto{\pgfqpoint{6.715220in}{2.236447in}}%
\pgfpathlineto{\pgfqpoint{6.787950in}{2.279181in}}%
\pgfpathlineto{\pgfqpoint{6.860680in}{2.318312in}}%
\pgfpathlineto{\pgfqpoint{6.933410in}{2.354147in}}%
\pgfpathlineto{\pgfqpoint{7.006139in}{2.386976in}}%
\pgfpathlineto{\pgfqpoint{7.078869in}{2.417590in}}%
\pgfpathlineto{\pgfqpoint{7.151599in}{2.445638in}}%
\pgfpathlineto{\pgfqpoint{7.224329in}{2.471779in}}%
\pgfpathlineto{\pgfqpoint{7.297059in}{2.496651in}}%
\pgfpathlineto{\pgfqpoint{7.369789in}{2.519365in}}%
\pgfpathlineto{\pgfqpoint{7.442518in}{2.540874in}}%
\pgfpathlineto{\pgfqpoint{7.515248in}{2.561148in}}%
\pgfpathlineto{\pgfqpoint{7.587978in}{2.580285in}}%
\pgfpathlineto{\pgfqpoint{7.660708in}{2.598240in}}%
\pgfusepath{stroke}%
\end{pgfscope}%
\begin{pgfscope}%
\pgfsetrectcap%
\pgfsetmiterjoin%
\pgfsetlinewidth{0.803000pt}%
\definecolor{currentstroke}{rgb}{0.000000,0.000000,0.000000}%
\pgfsetstrokecolor{currentstroke}%
\pgfsetdash{}{0pt}%
\pgfpathmoveto{\pgfqpoint{4.597222in}{0.387222in}}%
\pgfpathlineto{\pgfqpoint{4.597222in}{2.711667in}}%
\pgfusepath{stroke}%
\end{pgfscope}%
\begin{pgfscope}%
\pgfsetrectcap%
\pgfsetmiterjoin%
\pgfsetlinewidth{0.803000pt}%
\definecolor{currentstroke}{rgb}{0.000000,0.000000,0.000000}%
\pgfsetstrokecolor{currentstroke}%
\pgfsetdash{}{0pt}%
\pgfpathmoveto{\pgfqpoint{7.815000in}{0.387222in}}%
\pgfpathlineto{\pgfqpoint{7.815000in}{2.711667in}}%
\pgfusepath{stroke}%
\end{pgfscope}%
\begin{pgfscope}%
\pgfsetrectcap%
\pgfsetmiterjoin%
\pgfsetlinewidth{0.803000pt}%
\definecolor{currentstroke}{rgb}{0.000000,0.000000,0.000000}%
\pgfsetstrokecolor{currentstroke}%
\pgfsetdash{}{0pt}%
\pgfpathmoveto{\pgfqpoint{4.597222in}{0.387222in}}%
\pgfpathlineto{\pgfqpoint{7.815000in}{0.387222in}}%
\pgfusepath{stroke}%
\end{pgfscope}%
\begin{pgfscope}%
\pgfsetrectcap%
\pgfsetmiterjoin%
\pgfsetlinewidth{0.803000pt}%
\definecolor{currentstroke}{rgb}{0.000000,0.000000,0.000000}%
\pgfsetstrokecolor{currentstroke}%
\pgfsetdash{}{0pt}%
\pgfpathmoveto{\pgfqpoint{4.597222in}{2.711667in}}%
\pgfpathlineto{\pgfqpoint{7.815000in}{2.711667in}}%
\pgfusepath{stroke}%
\end{pgfscope}%
\begin{pgfscope}%
\definecolor{textcolor}{rgb}{0.000000,0.000000,0.000000}%
\pgfsetstrokecolor{textcolor}%
\pgfsetfillcolor{textcolor}%
\pgftext[x=6.206111in,y=2.795000in,,base]{\color{textcolor}\sffamily\fontsize{12.000000}{14.400000}\selectfont \(\displaystyle  N = 128 \)}%
\end{pgfscope}%
\end{pgfpicture}%
\makeatother%
\endgroup%
}
\caption{Internal energy $u$ for different temperatures $T$ using kinetic Monte Carlo algorithm}
\label{Fig:HeatSmallKMC}
\end{figure}

\begin{figure}[htbp]
\centering
\scalebox{0.666}{%% Creator: Matplotlib, PGF backend
%%
%% To include the figure in your LaTeX document, write
%%   \input{<filename>.pgf}
%%
%% Make sure the required packages are loaded in your preamble
%%   \usepackage{pgf}
%%
%% Figures using additional raster images can only be included by \input if
%% they are in the same directory as the main LaTeX file. For loading figures
%% from other directories you can use the `import` package
%%   \usepackage{import}
%% and then include the figures with
%%   \import{<path to file>}{<filename>.pgf}
%%
%% Matplotlib used the following preamble
%%   \usepackage{fontspec}
%%   \setmainfont{DejaVuSerif.ttf}[Path=/home/lzh/anaconda3/envs/numana/lib/python3.7/site-packages/matplotlib/mpl-data/fonts/ttf/]
%%   \setsansfont{DejaVuSans.ttf}[Path=/home/lzh/anaconda3/envs/numana/lib/python3.7/site-packages/matplotlib/mpl-data/fonts/ttf/]
%%   \setmonofont{DejaVuSansMono.ttf}[Path=/home/lzh/anaconda3/envs/numana/lib/python3.7/site-packages/matplotlib/mpl-data/fonts/ttf/]
%%
\begingroup%
\makeatletter%
\begin{pgfpicture}%
\pgfpathrectangle{\pgfpointorigin}{\pgfqpoint{4.000000in}{4.000000in}}%
\pgfusepath{use as bounding box, clip}%
\begin{pgfscope}%
\pgfsetbuttcap%
\pgfsetmiterjoin%
\definecolor{currentfill}{rgb}{1.000000,1.000000,1.000000}%
\pgfsetfillcolor{currentfill}%
\pgfsetlinewidth{0.000000pt}%
\definecolor{currentstroke}{rgb}{1.000000,1.000000,1.000000}%
\pgfsetstrokecolor{currentstroke}%
\pgfsetdash{}{0pt}%
\pgfpathmoveto{\pgfqpoint{0.000000in}{0.000000in}}%
\pgfpathlineto{\pgfqpoint{4.000000in}{0.000000in}}%
\pgfpathlineto{\pgfqpoint{4.000000in}{4.000000in}}%
\pgfpathlineto{\pgfqpoint{0.000000in}{4.000000in}}%
\pgfpathclose%
\pgfusepath{fill}%
\end{pgfscope}%
\begin{pgfscope}%
\pgfsetbuttcap%
\pgfsetmiterjoin%
\definecolor{currentfill}{rgb}{1.000000,1.000000,1.000000}%
\pgfsetfillcolor{currentfill}%
\pgfsetlinewidth{0.000000pt}%
\definecolor{currentstroke}{rgb}{0.000000,0.000000,0.000000}%
\pgfsetstrokecolor{currentstroke}%
\pgfsetstrokeopacity{0.000000}%
\pgfsetdash{}{0pt}%
\pgfpathmoveto{\pgfqpoint{0.510000in}{0.440000in}}%
\pgfpathlineto{\pgfqpoint{3.590000in}{0.440000in}}%
\pgfpathlineto{\pgfqpoint{3.590000in}{3.520000in}}%
\pgfpathlineto{\pgfqpoint{0.510000in}{3.520000in}}%
\pgfpathclose%
\pgfusepath{fill}%
\end{pgfscope}%
\begin{pgfscope}%
\pgfpathrectangle{\pgfqpoint{0.510000in}{0.440000in}}{\pgfqpoint{3.080000in}{3.080000in}}%
\pgfusepath{clip}%
\pgfsetbuttcap%
\pgfsetroundjoin%
\definecolor{currentfill}{rgb}{0.121569,0.466667,0.705882}%
\pgfsetfillcolor{currentfill}%
\pgfsetlinewidth{1.003750pt}%
\definecolor{currentstroke}{rgb}{0.121569,0.466667,0.705882}%
\pgfsetstrokecolor{currentstroke}%
\pgfsetdash{}{0pt}%
\pgfpathmoveto{\pgfqpoint{0.968786in}{2.091474in}}%
\pgfpathcurveto{\pgfqpoint{0.974610in}{2.091474in}}{\pgfqpoint{0.980196in}{2.093788in}}{\pgfqpoint{0.984314in}{2.097906in}}%
\pgfpathcurveto{\pgfqpoint{0.988432in}{2.102025in}}{\pgfqpoint{0.990746in}{2.107611in}}{\pgfqpoint{0.990746in}{2.113435in}}%
\pgfpathcurveto{\pgfqpoint{0.990746in}{2.119259in}}{\pgfqpoint{0.988432in}{2.124845in}}{\pgfqpoint{0.984314in}{2.128963in}}%
\pgfpathcurveto{\pgfqpoint{0.980196in}{2.133081in}}{\pgfqpoint{0.974610in}{2.135395in}}{\pgfqpoint{0.968786in}{2.135395in}}%
\pgfpathcurveto{\pgfqpoint{0.962962in}{2.135395in}}{\pgfqpoint{0.957376in}{2.133081in}}{\pgfqpoint{0.953258in}{2.128963in}}%
\pgfpathcurveto{\pgfqpoint{0.949140in}{2.124845in}}{\pgfqpoint{0.946826in}{2.119259in}}{\pgfqpoint{0.946826in}{2.113435in}}%
\pgfpathcurveto{\pgfqpoint{0.946826in}{2.107611in}}{\pgfqpoint{0.949140in}{2.102025in}}{\pgfqpoint{0.953258in}{2.097906in}}%
\pgfpathcurveto{\pgfqpoint{0.957376in}{2.093788in}}{\pgfqpoint{0.962962in}{2.091474in}}{\pgfqpoint{0.968786in}{2.091474in}}%
\pgfpathclose%
\pgfusepath{stroke,fill}%
\end{pgfscope}%
\begin{pgfscope}%
\pgfpathrectangle{\pgfqpoint{0.510000in}{0.440000in}}{\pgfqpoint{3.080000in}{3.080000in}}%
\pgfusepath{clip}%
\pgfsetbuttcap%
\pgfsetroundjoin%
\definecolor{currentfill}{rgb}{1.000000,0.498039,0.054902}%
\pgfsetfillcolor{currentfill}%
\pgfsetlinewidth{1.003750pt}%
\definecolor{currentstroke}{rgb}{1.000000,0.498039,0.054902}%
\pgfsetstrokecolor{currentstroke}%
\pgfsetdash{}{0pt}%
\pgfpathmoveto{\pgfqpoint{1.897784in}{0.876827in}}%
\pgfpathcurveto{\pgfqpoint{1.903608in}{0.876827in}}{\pgfqpoint{1.909194in}{0.879141in}}{\pgfqpoint{1.913312in}{0.883259in}}%
\pgfpathcurveto{\pgfqpoint{1.917430in}{0.887377in}}{\pgfqpoint{1.919744in}{0.892963in}}{\pgfqpoint{1.919744in}{0.898787in}}%
\pgfpathcurveto{\pgfqpoint{1.919744in}{0.904611in}}{\pgfqpoint{1.917430in}{0.910197in}}{\pgfqpoint{1.913312in}{0.914315in}}%
\pgfpathcurveto{\pgfqpoint{1.909194in}{0.918434in}}{\pgfqpoint{1.903608in}{0.920747in}}{\pgfqpoint{1.897784in}{0.920747in}}%
\pgfpathcurveto{\pgfqpoint{1.891960in}{0.920747in}}{\pgfqpoint{1.886374in}{0.918434in}}{\pgfqpoint{1.882255in}{0.914315in}}%
\pgfpathcurveto{\pgfqpoint{1.878137in}{0.910197in}}{\pgfqpoint{1.875823in}{0.904611in}}{\pgfqpoint{1.875823in}{0.898787in}}%
\pgfpathcurveto{\pgfqpoint{1.875823in}{0.892963in}}{\pgfqpoint{1.878137in}{0.887377in}}{\pgfqpoint{1.882255in}{0.883259in}}%
\pgfpathcurveto{\pgfqpoint{1.886374in}{0.879141in}}{\pgfqpoint{1.891960in}{0.876827in}}{\pgfqpoint{1.897784in}{0.876827in}}%
\pgfpathclose%
\pgfusepath{stroke,fill}%
\end{pgfscope}%
\begin{pgfscope}%
\pgfpathrectangle{\pgfqpoint{0.510000in}{0.440000in}}{\pgfqpoint{3.080000in}{3.080000in}}%
\pgfusepath{clip}%
\pgfsetbuttcap%
\pgfsetroundjoin%
\definecolor{currentfill}{rgb}{0.172549,0.627451,0.172549}%
\pgfsetfillcolor{currentfill}%
\pgfsetlinewidth{1.003750pt}%
\definecolor{currentstroke}{rgb}{0.172549,0.627451,0.172549}%
\pgfsetstrokecolor{currentstroke}%
\pgfsetdash{}{0pt}%
\pgfpathmoveto{\pgfqpoint{2.183431in}{1.805819in}}%
\pgfpathcurveto{\pgfqpoint{2.189255in}{1.805819in}}{\pgfqpoint{2.194841in}{1.808133in}}{\pgfqpoint{2.198959in}{1.812251in}}%
\pgfpathcurveto{\pgfqpoint{2.203077in}{1.816369in}}{\pgfqpoint{2.205391in}{1.821955in}}{\pgfqpoint{2.205391in}{1.827779in}}%
\pgfpathcurveto{\pgfqpoint{2.205391in}{1.833603in}}{\pgfqpoint{2.203077in}{1.839189in}}{\pgfqpoint{2.198959in}{1.843307in}}%
\pgfpathcurveto{\pgfqpoint{2.194841in}{1.847425in}}{\pgfqpoint{2.189255in}{1.849739in}}{\pgfqpoint{2.183431in}{1.849739in}}%
\pgfpathcurveto{\pgfqpoint{2.177607in}{1.849739in}}{\pgfqpoint{2.172021in}{1.847425in}}{\pgfqpoint{2.167903in}{1.843307in}}%
\pgfpathcurveto{\pgfqpoint{2.163785in}{1.839189in}}{\pgfqpoint{2.161471in}{1.833603in}}{\pgfqpoint{2.161471in}{1.827779in}}%
\pgfpathcurveto{\pgfqpoint{2.161471in}{1.821955in}}{\pgfqpoint{2.163785in}{1.816369in}}{\pgfqpoint{2.167903in}{1.812251in}}%
\pgfpathcurveto{\pgfqpoint{2.172021in}{1.808133in}}{\pgfqpoint{2.177607in}{1.805819in}}{\pgfqpoint{2.183431in}{1.805819in}}%
\pgfpathclose%
\pgfusepath{stroke,fill}%
\end{pgfscope}%
\begin{pgfscope}%
\pgfpathrectangle{\pgfqpoint{0.510000in}{0.440000in}}{\pgfqpoint{3.080000in}{3.080000in}}%
\pgfusepath{clip}%
\pgfsetbuttcap%
\pgfsetmiterjoin%
\definecolor{currentfill}{rgb}{0.100000,0.100000,0.100000}%
\pgfsetfillcolor{currentfill}%
\pgfsetfillopacity{0.100000}%
\pgfsetlinewidth{0.301125pt}%
\definecolor{currentstroke}{rgb}{0.000000,0.000000,0.000000}%
\pgfsetstrokecolor{currentstroke}%
\pgfsetdash{}{0pt}%
\pgfpathmoveto{\pgfqpoint{0.950000in}{0.880000in}}%
\pgfpathlineto{\pgfqpoint{0.950000in}{3.080000in}}%
\pgfpathlineto{\pgfqpoint{3.150000in}{0.880000in}}%
\pgfpathclose%
\pgfusepath{stroke,fill}%
\end{pgfscope}%
\begin{pgfscope}%
\pgfpathrectangle{\pgfqpoint{0.510000in}{0.440000in}}{\pgfqpoint{3.080000in}{3.080000in}}%
\pgfusepath{clip}%
\pgfsetrectcap%
\pgfsetroundjoin%
\pgfsetlinewidth{0.803000pt}%
\definecolor{currentstroke}{rgb}{0.690196,0.690196,0.690196}%
\pgfsetstrokecolor{currentstroke}%
\pgfsetdash{}{0pt}%
\pgfpathmoveto{\pgfqpoint{0.510000in}{0.440000in}}%
\pgfpathlineto{\pgfqpoint{0.510000in}{3.520000in}}%
\pgfusepath{stroke}%
\end{pgfscope}%
\begin{pgfscope}%
\pgfsetbuttcap%
\pgfsetroundjoin%
\definecolor{currentfill}{rgb}{0.000000,0.000000,0.000000}%
\pgfsetfillcolor{currentfill}%
\pgfsetlinewidth{0.803000pt}%
\definecolor{currentstroke}{rgb}{0.000000,0.000000,0.000000}%
\pgfsetstrokecolor{currentstroke}%
\pgfsetdash{}{0pt}%
\pgfsys@defobject{currentmarker}{\pgfqpoint{0.000000in}{-0.048611in}}{\pgfqpoint{0.000000in}{0.000000in}}{%
\pgfpathmoveto{\pgfqpoint{0.000000in}{0.000000in}}%
\pgfpathlineto{\pgfqpoint{0.000000in}{-0.048611in}}%
\pgfusepath{stroke,fill}%
}%
\begin{pgfscope}%
\pgfsys@transformshift{0.510000in}{0.440000in}%
\pgfsys@useobject{currentmarker}{}%
\end{pgfscope}%
\end{pgfscope}%
\begin{pgfscope}%
\definecolor{textcolor}{rgb}{0.000000,0.000000,0.000000}%
\pgfsetstrokecolor{textcolor}%
\pgfsetfillcolor{textcolor}%
\pgftext[x=0.510000in,y=0.342778in,,top]{\color{textcolor}\sffamily\fontsize{10.000000}{12.000000}\selectfont -0.2}%
\end{pgfscope}%
\begin{pgfscope}%
\pgfpathrectangle{\pgfqpoint{0.510000in}{0.440000in}}{\pgfqpoint{3.080000in}{3.080000in}}%
\pgfusepath{clip}%
\pgfsetrectcap%
\pgfsetroundjoin%
\pgfsetlinewidth{0.803000pt}%
\definecolor{currentstroke}{rgb}{0.690196,0.690196,0.690196}%
\pgfsetstrokecolor{currentstroke}%
\pgfsetdash{}{0pt}%
\pgfpathmoveto{\pgfqpoint{0.950000in}{0.440000in}}%
\pgfpathlineto{\pgfqpoint{0.950000in}{3.520000in}}%
\pgfusepath{stroke}%
\end{pgfscope}%
\begin{pgfscope}%
\pgfsetbuttcap%
\pgfsetroundjoin%
\definecolor{currentfill}{rgb}{0.000000,0.000000,0.000000}%
\pgfsetfillcolor{currentfill}%
\pgfsetlinewidth{0.803000pt}%
\definecolor{currentstroke}{rgb}{0.000000,0.000000,0.000000}%
\pgfsetstrokecolor{currentstroke}%
\pgfsetdash{}{0pt}%
\pgfsys@defobject{currentmarker}{\pgfqpoint{0.000000in}{-0.048611in}}{\pgfqpoint{0.000000in}{0.000000in}}{%
\pgfpathmoveto{\pgfqpoint{0.000000in}{0.000000in}}%
\pgfpathlineto{\pgfqpoint{0.000000in}{-0.048611in}}%
\pgfusepath{stroke,fill}%
}%
\begin{pgfscope}%
\pgfsys@transformshift{0.950000in}{0.440000in}%
\pgfsys@useobject{currentmarker}{}%
\end{pgfscope}%
\end{pgfscope}%
\begin{pgfscope}%
\definecolor{textcolor}{rgb}{0.000000,0.000000,0.000000}%
\pgfsetstrokecolor{textcolor}%
\pgfsetfillcolor{textcolor}%
\pgftext[x=0.950000in,y=0.342778in,,top]{\color{textcolor}\sffamily\fontsize{10.000000}{12.000000}\selectfont 0.0}%
\end{pgfscope}%
\begin{pgfscope}%
\pgfpathrectangle{\pgfqpoint{0.510000in}{0.440000in}}{\pgfqpoint{3.080000in}{3.080000in}}%
\pgfusepath{clip}%
\pgfsetrectcap%
\pgfsetroundjoin%
\pgfsetlinewidth{0.803000pt}%
\definecolor{currentstroke}{rgb}{0.690196,0.690196,0.690196}%
\pgfsetstrokecolor{currentstroke}%
\pgfsetdash{}{0pt}%
\pgfpathmoveto{\pgfqpoint{1.390000in}{0.440000in}}%
\pgfpathlineto{\pgfqpoint{1.390000in}{3.520000in}}%
\pgfusepath{stroke}%
\end{pgfscope}%
\begin{pgfscope}%
\pgfsetbuttcap%
\pgfsetroundjoin%
\definecolor{currentfill}{rgb}{0.000000,0.000000,0.000000}%
\pgfsetfillcolor{currentfill}%
\pgfsetlinewidth{0.803000pt}%
\definecolor{currentstroke}{rgb}{0.000000,0.000000,0.000000}%
\pgfsetstrokecolor{currentstroke}%
\pgfsetdash{}{0pt}%
\pgfsys@defobject{currentmarker}{\pgfqpoint{0.000000in}{-0.048611in}}{\pgfqpoint{0.000000in}{0.000000in}}{%
\pgfpathmoveto{\pgfqpoint{0.000000in}{0.000000in}}%
\pgfpathlineto{\pgfqpoint{0.000000in}{-0.048611in}}%
\pgfusepath{stroke,fill}%
}%
\begin{pgfscope}%
\pgfsys@transformshift{1.390000in}{0.440000in}%
\pgfsys@useobject{currentmarker}{}%
\end{pgfscope}%
\end{pgfscope}%
\begin{pgfscope}%
\definecolor{textcolor}{rgb}{0.000000,0.000000,0.000000}%
\pgfsetstrokecolor{textcolor}%
\pgfsetfillcolor{textcolor}%
\pgftext[x=1.390000in,y=0.342778in,,top]{\color{textcolor}\sffamily\fontsize{10.000000}{12.000000}\selectfont 0.2}%
\end{pgfscope}%
\begin{pgfscope}%
\pgfpathrectangle{\pgfqpoint{0.510000in}{0.440000in}}{\pgfqpoint{3.080000in}{3.080000in}}%
\pgfusepath{clip}%
\pgfsetrectcap%
\pgfsetroundjoin%
\pgfsetlinewidth{0.803000pt}%
\definecolor{currentstroke}{rgb}{0.690196,0.690196,0.690196}%
\pgfsetstrokecolor{currentstroke}%
\pgfsetdash{}{0pt}%
\pgfpathmoveto{\pgfqpoint{1.830000in}{0.440000in}}%
\pgfpathlineto{\pgfqpoint{1.830000in}{3.520000in}}%
\pgfusepath{stroke}%
\end{pgfscope}%
\begin{pgfscope}%
\pgfsetbuttcap%
\pgfsetroundjoin%
\definecolor{currentfill}{rgb}{0.000000,0.000000,0.000000}%
\pgfsetfillcolor{currentfill}%
\pgfsetlinewidth{0.803000pt}%
\definecolor{currentstroke}{rgb}{0.000000,0.000000,0.000000}%
\pgfsetstrokecolor{currentstroke}%
\pgfsetdash{}{0pt}%
\pgfsys@defobject{currentmarker}{\pgfqpoint{0.000000in}{-0.048611in}}{\pgfqpoint{0.000000in}{0.000000in}}{%
\pgfpathmoveto{\pgfqpoint{0.000000in}{0.000000in}}%
\pgfpathlineto{\pgfqpoint{0.000000in}{-0.048611in}}%
\pgfusepath{stroke,fill}%
}%
\begin{pgfscope}%
\pgfsys@transformshift{1.830000in}{0.440000in}%
\pgfsys@useobject{currentmarker}{}%
\end{pgfscope}%
\end{pgfscope}%
\begin{pgfscope}%
\definecolor{textcolor}{rgb}{0.000000,0.000000,0.000000}%
\pgfsetstrokecolor{textcolor}%
\pgfsetfillcolor{textcolor}%
\pgftext[x=1.830000in,y=0.342778in,,top]{\color{textcolor}\sffamily\fontsize{10.000000}{12.000000}\selectfont 0.4}%
\end{pgfscope}%
\begin{pgfscope}%
\pgfpathrectangle{\pgfqpoint{0.510000in}{0.440000in}}{\pgfqpoint{3.080000in}{3.080000in}}%
\pgfusepath{clip}%
\pgfsetrectcap%
\pgfsetroundjoin%
\pgfsetlinewidth{0.803000pt}%
\definecolor{currentstroke}{rgb}{0.690196,0.690196,0.690196}%
\pgfsetstrokecolor{currentstroke}%
\pgfsetdash{}{0pt}%
\pgfpathmoveto{\pgfqpoint{2.270000in}{0.440000in}}%
\pgfpathlineto{\pgfqpoint{2.270000in}{3.520000in}}%
\pgfusepath{stroke}%
\end{pgfscope}%
\begin{pgfscope}%
\pgfsetbuttcap%
\pgfsetroundjoin%
\definecolor{currentfill}{rgb}{0.000000,0.000000,0.000000}%
\pgfsetfillcolor{currentfill}%
\pgfsetlinewidth{0.803000pt}%
\definecolor{currentstroke}{rgb}{0.000000,0.000000,0.000000}%
\pgfsetstrokecolor{currentstroke}%
\pgfsetdash{}{0pt}%
\pgfsys@defobject{currentmarker}{\pgfqpoint{0.000000in}{-0.048611in}}{\pgfqpoint{0.000000in}{0.000000in}}{%
\pgfpathmoveto{\pgfqpoint{0.000000in}{0.000000in}}%
\pgfpathlineto{\pgfqpoint{0.000000in}{-0.048611in}}%
\pgfusepath{stroke,fill}%
}%
\begin{pgfscope}%
\pgfsys@transformshift{2.270000in}{0.440000in}%
\pgfsys@useobject{currentmarker}{}%
\end{pgfscope}%
\end{pgfscope}%
\begin{pgfscope}%
\definecolor{textcolor}{rgb}{0.000000,0.000000,0.000000}%
\pgfsetstrokecolor{textcolor}%
\pgfsetfillcolor{textcolor}%
\pgftext[x=2.270000in,y=0.342778in,,top]{\color{textcolor}\sffamily\fontsize{10.000000}{12.000000}\selectfont 0.6}%
\end{pgfscope}%
\begin{pgfscope}%
\pgfpathrectangle{\pgfqpoint{0.510000in}{0.440000in}}{\pgfqpoint{3.080000in}{3.080000in}}%
\pgfusepath{clip}%
\pgfsetrectcap%
\pgfsetroundjoin%
\pgfsetlinewidth{0.803000pt}%
\definecolor{currentstroke}{rgb}{0.690196,0.690196,0.690196}%
\pgfsetstrokecolor{currentstroke}%
\pgfsetdash{}{0pt}%
\pgfpathmoveto{\pgfqpoint{2.710000in}{0.440000in}}%
\pgfpathlineto{\pgfqpoint{2.710000in}{3.520000in}}%
\pgfusepath{stroke}%
\end{pgfscope}%
\begin{pgfscope}%
\pgfsetbuttcap%
\pgfsetroundjoin%
\definecolor{currentfill}{rgb}{0.000000,0.000000,0.000000}%
\pgfsetfillcolor{currentfill}%
\pgfsetlinewidth{0.803000pt}%
\definecolor{currentstroke}{rgb}{0.000000,0.000000,0.000000}%
\pgfsetstrokecolor{currentstroke}%
\pgfsetdash{}{0pt}%
\pgfsys@defobject{currentmarker}{\pgfqpoint{0.000000in}{-0.048611in}}{\pgfqpoint{0.000000in}{0.000000in}}{%
\pgfpathmoveto{\pgfqpoint{0.000000in}{0.000000in}}%
\pgfpathlineto{\pgfqpoint{0.000000in}{-0.048611in}}%
\pgfusepath{stroke,fill}%
}%
\begin{pgfscope}%
\pgfsys@transformshift{2.710000in}{0.440000in}%
\pgfsys@useobject{currentmarker}{}%
\end{pgfscope}%
\end{pgfscope}%
\begin{pgfscope}%
\definecolor{textcolor}{rgb}{0.000000,0.000000,0.000000}%
\pgfsetstrokecolor{textcolor}%
\pgfsetfillcolor{textcolor}%
\pgftext[x=2.710000in,y=0.342778in,,top]{\color{textcolor}\sffamily\fontsize{10.000000}{12.000000}\selectfont 0.8}%
\end{pgfscope}%
\begin{pgfscope}%
\pgfpathrectangle{\pgfqpoint{0.510000in}{0.440000in}}{\pgfqpoint{3.080000in}{3.080000in}}%
\pgfusepath{clip}%
\pgfsetrectcap%
\pgfsetroundjoin%
\pgfsetlinewidth{0.803000pt}%
\definecolor{currentstroke}{rgb}{0.690196,0.690196,0.690196}%
\pgfsetstrokecolor{currentstroke}%
\pgfsetdash{}{0pt}%
\pgfpathmoveto{\pgfqpoint{3.150000in}{0.440000in}}%
\pgfpathlineto{\pgfqpoint{3.150000in}{3.520000in}}%
\pgfusepath{stroke}%
\end{pgfscope}%
\begin{pgfscope}%
\pgfsetbuttcap%
\pgfsetroundjoin%
\definecolor{currentfill}{rgb}{0.000000,0.000000,0.000000}%
\pgfsetfillcolor{currentfill}%
\pgfsetlinewidth{0.803000pt}%
\definecolor{currentstroke}{rgb}{0.000000,0.000000,0.000000}%
\pgfsetstrokecolor{currentstroke}%
\pgfsetdash{}{0pt}%
\pgfsys@defobject{currentmarker}{\pgfqpoint{0.000000in}{-0.048611in}}{\pgfqpoint{0.000000in}{0.000000in}}{%
\pgfpathmoveto{\pgfqpoint{0.000000in}{0.000000in}}%
\pgfpathlineto{\pgfqpoint{0.000000in}{-0.048611in}}%
\pgfusepath{stroke,fill}%
}%
\begin{pgfscope}%
\pgfsys@transformshift{3.150000in}{0.440000in}%
\pgfsys@useobject{currentmarker}{}%
\end{pgfscope}%
\end{pgfscope}%
\begin{pgfscope}%
\definecolor{textcolor}{rgb}{0.000000,0.000000,0.000000}%
\pgfsetstrokecolor{textcolor}%
\pgfsetfillcolor{textcolor}%
\pgftext[x=3.150000in,y=0.342778in,,top]{\color{textcolor}\sffamily\fontsize{10.000000}{12.000000}\selectfont 1.0}%
\end{pgfscope}%
\begin{pgfscope}%
\pgfpathrectangle{\pgfqpoint{0.510000in}{0.440000in}}{\pgfqpoint{3.080000in}{3.080000in}}%
\pgfusepath{clip}%
\pgfsetrectcap%
\pgfsetroundjoin%
\pgfsetlinewidth{0.803000pt}%
\definecolor{currentstroke}{rgb}{0.690196,0.690196,0.690196}%
\pgfsetstrokecolor{currentstroke}%
\pgfsetdash{}{0pt}%
\pgfpathmoveto{\pgfqpoint{3.590000in}{0.440000in}}%
\pgfpathlineto{\pgfqpoint{3.590000in}{3.520000in}}%
\pgfusepath{stroke}%
\end{pgfscope}%
\begin{pgfscope}%
\pgfsetbuttcap%
\pgfsetroundjoin%
\definecolor{currentfill}{rgb}{0.000000,0.000000,0.000000}%
\pgfsetfillcolor{currentfill}%
\pgfsetlinewidth{0.803000pt}%
\definecolor{currentstroke}{rgb}{0.000000,0.000000,0.000000}%
\pgfsetstrokecolor{currentstroke}%
\pgfsetdash{}{0pt}%
\pgfsys@defobject{currentmarker}{\pgfqpoint{0.000000in}{-0.048611in}}{\pgfqpoint{0.000000in}{0.000000in}}{%
\pgfpathmoveto{\pgfqpoint{0.000000in}{0.000000in}}%
\pgfpathlineto{\pgfqpoint{0.000000in}{-0.048611in}}%
\pgfusepath{stroke,fill}%
}%
\begin{pgfscope}%
\pgfsys@transformshift{3.590000in}{0.440000in}%
\pgfsys@useobject{currentmarker}{}%
\end{pgfscope}%
\end{pgfscope}%
\begin{pgfscope}%
\definecolor{textcolor}{rgb}{0.000000,0.000000,0.000000}%
\pgfsetstrokecolor{textcolor}%
\pgfsetfillcolor{textcolor}%
\pgftext[x=3.590000in,y=0.342778in,,top]{\color{textcolor}\sffamily\fontsize{10.000000}{12.000000}\selectfont 1.2}%
\end{pgfscope}%
\begin{pgfscope}%
\pgfpathrectangle{\pgfqpoint{0.510000in}{0.440000in}}{\pgfqpoint{3.080000in}{3.080000in}}%
\pgfusepath{clip}%
\pgfsetrectcap%
\pgfsetroundjoin%
\pgfsetlinewidth{0.803000pt}%
\definecolor{currentstroke}{rgb}{0.690196,0.690196,0.690196}%
\pgfsetstrokecolor{currentstroke}%
\pgfsetdash{}{0pt}%
\pgfpathmoveto{\pgfqpoint{0.510000in}{0.440000in}}%
\pgfpathlineto{\pgfqpoint{3.590000in}{0.440000in}}%
\pgfusepath{stroke}%
\end{pgfscope}%
\begin{pgfscope}%
\pgfsetbuttcap%
\pgfsetroundjoin%
\definecolor{currentfill}{rgb}{0.000000,0.000000,0.000000}%
\pgfsetfillcolor{currentfill}%
\pgfsetlinewidth{0.803000pt}%
\definecolor{currentstroke}{rgb}{0.000000,0.000000,0.000000}%
\pgfsetstrokecolor{currentstroke}%
\pgfsetdash{}{0pt}%
\pgfsys@defobject{currentmarker}{\pgfqpoint{-0.048611in}{0.000000in}}{\pgfqpoint{0.000000in}{0.000000in}}{%
\pgfpathmoveto{\pgfqpoint{0.000000in}{0.000000in}}%
\pgfpathlineto{\pgfqpoint{-0.048611in}{0.000000in}}%
\pgfusepath{stroke,fill}%
}%
\begin{pgfscope}%
\pgfsys@transformshift{0.510000in}{0.440000in}%
\pgfsys@useobject{currentmarker}{}%
\end{pgfscope}%
\end{pgfscope}%
\begin{pgfscope}%
\definecolor{textcolor}{rgb}{0.000000,0.000000,0.000000}%
\pgfsetstrokecolor{textcolor}%
\pgfsetfillcolor{textcolor}%
\pgftext[x=0.075525in,y=0.387238in,left,base]{\color{textcolor}\sffamily\fontsize{10.000000}{12.000000}\selectfont -0.2}%
\end{pgfscope}%
\begin{pgfscope}%
\pgfpathrectangle{\pgfqpoint{0.510000in}{0.440000in}}{\pgfqpoint{3.080000in}{3.080000in}}%
\pgfusepath{clip}%
\pgfsetrectcap%
\pgfsetroundjoin%
\pgfsetlinewidth{0.803000pt}%
\definecolor{currentstroke}{rgb}{0.690196,0.690196,0.690196}%
\pgfsetstrokecolor{currentstroke}%
\pgfsetdash{}{0pt}%
\pgfpathmoveto{\pgfqpoint{0.510000in}{0.880000in}}%
\pgfpathlineto{\pgfqpoint{3.590000in}{0.880000in}}%
\pgfusepath{stroke}%
\end{pgfscope}%
\begin{pgfscope}%
\pgfsetbuttcap%
\pgfsetroundjoin%
\definecolor{currentfill}{rgb}{0.000000,0.000000,0.000000}%
\pgfsetfillcolor{currentfill}%
\pgfsetlinewidth{0.803000pt}%
\definecolor{currentstroke}{rgb}{0.000000,0.000000,0.000000}%
\pgfsetstrokecolor{currentstroke}%
\pgfsetdash{}{0pt}%
\pgfsys@defobject{currentmarker}{\pgfqpoint{-0.048611in}{0.000000in}}{\pgfqpoint{0.000000in}{0.000000in}}{%
\pgfpathmoveto{\pgfqpoint{0.000000in}{0.000000in}}%
\pgfpathlineto{\pgfqpoint{-0.048611in}{0.000000in}}%
\pgfusepath{stroke,fill}%
}%
\begin{pgfscope}%
\pgfsys@transformshift{0.510000in}{0.880000in}%
\pgfsys@useobject{currentmarker}{}%
\end{pgfscope}%
\end{pgfscope}%
\begin{pgfscope}%
\definecolor{textcolor}{rgb}{0.000000,0.000000,0.000000}%
\pgfsetstrokecolor{textcolor}%
\pgfsetfillcolor{textcolor}%
\pgftext[x=0.191898in,y=0.827238in,left,base]{\color{textcolor}\sffamily\fontsize{10.000000}{12.000000}\selectfont 0.0}%
\end{pgfscope}%
\begin{pgfscope}%
\pgfpathrectangle{\pgfqpoint{0.510000in}{0.440000in}}{\pgfqpoint{3.080000in}{3.080000in}}%
\pgfusepath{clip}%
\pgfsetrectcap%
\pgfsetroundjoin%
\pgfsetlinewidth{0.803000pt}%
\definecolor{currentstroke}{rgb}{0.690196,0.690196,0.690196}%
\pgfsetstrokecolor{currentstroke}%
\pgfsetdash{}{0pt}%
\pgfpathmoveto{\pgfqpoint{0.510000in}{1.320000in}}%
\pgfpathlineto{\pgfqpoint{3.590000in}{1.320000in}}%
\pgfusepath{stroke}%
\end{pgfscope}%
\begin{pgfscope}%
\pgfsetbuttcap%
\pgfsetroundjoin%
\definecolor{currentfill}{rgb}{0.000000,0.000000,0.000000}%
\pgfsetfillcolor{currentfill}%
\pgfsetlinewidth{0.803000pt}%
\definecolor{currentstroke}{rgb}{0.000000,0.000000,0.000000}%
\pgfsetstrokecolor{currentstroke}%
\pgfsetdash{}{0pt}%
\pgfsys@defobject{currentmarker}{\pgfqpoint{-0.048611in}{0.000000in}}{\pgfqpoint{0.000000in}{0.000000in}}{%
\pgfpathmoveto{\pgfqpoint{0.000000in}{0.000000in}}%
\pgfpathlineto{\pgfqpoint{-0.048611in}{0.000000in}}%
\pgfusepath{stroke,fill}%
}%
\begin{pgfscope}%
\pgfsys@transformshift{0.510000in}{1.320000in}%
\pgfsys@useobject{currentmarker}{}%
\end{pgfscope}%
\end{pgfscope}%
\begin{pgfscope}%
\definecolor{textcolor}{rgb}{0.000000,0.000000,0.000000}%
\pgfsetstrokecolor{textcolor}%
\pgfsetfillcolor{textcolor}%
\pgftext[x=0.191898in,y=1.267238in,left,base]{\color{textcolor}\sffamily\fontsize{10.000000}{12.000000}\selectfont 0.2}%
\end{pgfscope}%
\begin{pgfscope}%
\pgfpathrectangle{\pgfqpoint{0.510000in}{0.440000in}}{\pgfqpoint{3.080000in}{3.080000in}}%
\pgfusepath{clip}%
\pgfsetrectcap%
\pgfsetroundjoin%
\pgfsetlinewidth{0.803000pt}%
\definecolor{currentstroke}{rgb}{0.690196,0.690196,0.690196}%
\pgfsetstrokecolor{currentstroke}%
\pgfsetdash{}{0pt}%
\pgfpathmoveto{\pgfqpoint{0.510000in}{1.760000in}}%
\pgfpathlineto{\pgfqpoint{3.590000in}{1.760000in}}%
\pgfusepath{stroke}%
\end{pgfscope}%
\begin{pgfscope}%
\pgfsetbuttcap%
\pgfsetroundjoin%
\definecolor{currentfill}{rgb}{0.000000,0.000000,0.000000}%
\pgfsetfillcolor{currentfill}%
\pgfsetlinewidth{0.803000pt}%
\definecolor{currentstroke}{rgb}{0.000000,0.000000,0.000000}%
\pgfsetstrokecolor{currentstroke}%
\pgfsetdash{}{0pt}%
\pgfsys@defobject{currentmarker}{\pgfqpoint{-0.048611in}{0.000000in}}{\pgfqpoint{0.000000in}{0.000000in}}{%
\pgfpathmoveto{\pgfqpoint{0.000000in}{0.000000in}}%
\pgfpathlineto{\pgfqpoint{-0.048611in}{0.000000in}}%
\pgfusepath{stroke,fill}%
}%
\begin{pgfscope}%
\pgfsys@transformshift{0.510000in}{1.760000in}%
\pgfsys@useobject{currentmarker}{}%
\end{pgfscope}%
\end{pgfscope}%
\begin{pgfscope}%
\definecolor{textcolor}{rgb}{0.000000,0.000000,0.000000}%
\pgfsetstrokecolor{textcolor}%
\pgfsetfillcolor{textcolor}%
\pgftext[x=0.191898in,y=1.707238in,left,base]{\color{textcolor}\sffamily\fontsize{10.000000}{12.000000}\selectfont 0.4}%
\end{pgfscope}%
\begin{pgfscope}%
\pgfpathrectangle{\pgfqpoint{0.510000in}{0.440000in}}{\pgfqpoint{3.080000in}{3.080000in}}%
\pgfusepath{clip}%
\pgfsetrectcap%
\pgfsetroundjoin%
\pgfsetlinewidth{0.803000pt}%
\definecolor{currentstroke}{rgb}{0.690196,0.690196,0.690196}%
\pgfsetstrokecolor{currentstroke}%
\pgfsetdash{}{0pt}%
\pgfpathmoveto{\pgfqpoint{0.510000in}{2.200000in}}%
\pgfpathlineto{\pgfqpoint{3.590000in}{2.200000in}}%
\pgfusepath{stroke}%
\end{pgfscope}%
\begin{pgfscope}%
\pgfsetbuttcap%
\pgfsetroundjoin%
\definecolor{currentfill}{rgb}{0.000000,0.000000,0.000000}%
\pgfsetfillcolor{currentfill}%
\pgfsetlinewidth{0.803000pt}%
\definecolor{currentstroke}{rgb}{0.000000,0.000000,0.000000}%
\pgfsetstrokecolor{currentstroke}%
\pgfsetdash{}{0pt}%
\pgfsys@defobject{currentmarker}{\pgfqpoint{-0.048611in}{0.000000in}}{\pgfqpoint{0.000000in}{0.000000in}}{%
\pgfpathmoveto{\pgfqpoint{0.000000in}{0.000000in}}%
\pgfpathlineto{\pgfqpoint{-0.048611in}{0.000000in}}%
\pgfusepath{stroke,fill}%
}%
\begin{pgfscope}%
\pgfsys@transformshift{0.510000in}{2.200000in}%
\pgfsys@useobject{currentmarker}{}%
\end{pgfscope}%
\end{pgfscope}%
\begin{pgfscope}%
\definecolor{textcolor}{rgb}{0.000000,0.000000,0.000000}%
\pgfsetstrokecolor{textcolor}%
\pgfsetfillcolor{textcolor}%
\pgftext[x=0.191898in,y=2.147238in,left,base]{\color{textcolor}\sffamily\fontsize{10.000000}{12.000000}\selectfont 0.6}%
\end{pgfscope}%
\begin{pgfscope}%
\pgfpathrectangle{\pgfqpoint{0.510000in}{0.440000in}}{\pgfqpoint{3.080000in}{3.080000in}}%
\pgfusepath{clip}%
\pgfsetrectcap%
\pgfsetroundjoin%
\pgfsetlinewidth{0.803000pt}%
\definecolor{currentstroke}{rgb}{0.690196,0.690196,0.690196}%
\pgfsetstrokecolor{currentstroke}%
\pgfsetdash{}{0pt}%
\pgfpathmoveto{\pgfqpoint{0.510000in}{2.640000in}}%
\pgfpathlineto{\pgfqpoint{3.590000in}{2.640000in}}%
\pgfusepath{stroke}%
\end{pgfscope}%
\begin{pgfscope}%
\pgfsetbuttcap%
\pgfsetroundjoin%
\definecolor{currentfill}{rgb}{0.000000,0.000000,0.000000}%
\pgfsetfillcolor{currentfill}%
\pgfsetlinewidth{0.803000pt}%
\definecolor{currentstroke}{rgb}{0.000000,0.000000,0.000000}%
\pgfsetstrokecolor{currentstroke}%
\pgfsetdash{}{0pt}%
\pgfsys@defobject{currentmarker}{\pgfqpoint{-0.048611in}{0.000000in}}{\pgfqpoint{0.000000in}{0.000000in}}{%
\pgfpathmoveto{\pgfqpoint{0.000000in}{0.000000in}}%
\pgfpathlineto{\pgfqpoint{-0.048611in}{0.000000in}}%
\pgfusepath{stroke,fill}%
}%
\begin{pgfscope}%
\pgfsys@transformshift{0.510000in}{2.640000in}%
\pgfsys@useobject{currentmarker}{}%
\end{pgfscope}%
\end{pgfscope}%
\begin{pgfscope}%
\definecolor{textcolor}{rgb}{0.000000,0.000000,0.000000}%
\pgfsetstrokecolor{textcolor}%
\pgfsetfillcolor{textcolor}%
\pgftext[x=0.191898in,y=2.587238in,left,base]{\color{textcolor}\sffamily\fontsize{10.000000}{12.000000}\selectfont 0.8}%
\end{pgfscope}%
\begin{pgfscope}%
\pgfpathrectangle{\pgfqpoint{0.510000in}{0.440000in}}{\pgfqpoint{3.080000in}{3.080000in}}%
\pgfusepath{clip}%
\pgfsetrectcap%
\pgfsetroundjoin%
\pgfsetlinewidth{0.803000pt}%
\definecolor{currentstroke}{rgb}{0.690196,0.690196,0.690196}%
\pgfsetstrokecolor{currentstroke}%
\pgfsetdash{}{0pt}%
\pgfpathmoveto{\pgfqpoint{0.510000in}{3.080000in}}%
\pgfpathlineto{\pgfqpoint{3.590000in}{3.080000in}}%
\pgfusepath{stroke}%
\end{pgfscope}%
\begin{pgfscope}%
\pgfsetbuttcap%
\pgfsetroundjoin%
\definecolor{currentfill}{rgb}{0.000000,0.000000,0.000000}%
\pgfsetfillcolor{currentfill}%
\pgfsetlinewidth{0.803000pt}%
\definecolor{currentstroke}{rgb}{0.000000,0.000000,0.000000}%
\pgfsetstrokecolor{currentstroke}%
\pgfsetdash{}{0pt}%
\pgfsys@defobject{currentmarker}{\pgfqpoint{-0.048611in}{0.000000in}}{\pgfqpoint{0.000000in}{0.000000in}}{%
\pgfpathmoveto{\pgfqpoint{0.000000in}{0.000000in}}%
\pgfpathlineto{\pgfqpoint{-0.048611in}{0.000000in}}%
\pgfusepath{stroke,fill}%
}%
\begin{pgfscope}%
\pgfsys@transformshift{0.510000in}{3.080000in}%
\pgfsys@useobject{currentmarker}{}%
\end{pgfscope}%
\end{pgfscope}%
\begin{pgfscope}%
\definecolor{textcolor}{rgb}{0.000000,0.000000,0.000000}%
\pgfsetstrokecolor{textcolor}%
\pgfsetfillcolor{textcolor}%
\pgftext[x=0.191898in,y=3.027238in,left,base]{\color{textcolor}\sffamily\fontsize{10.000000}{12.000000}\selectfont 1.0}%
\end{pgfscope}%
\begin{pgfscope}%
\pgfpathrectangle{\pgfqpoint{0.510000in}{0.440000in}}{\pgfqpoint{3.080000in}{3.080000in}}%
\pgfusepath{clip}%
\pgfsetrectcap%
\pgfsetroundjoin%
\pgfsetlinewidth{0.803000pt}%
\definecolor{currentstroke}{rgb}{0.690196,0.690196,0.690196}%
\pgfsetstrokecolor{currentstroke}%
\pgfsetdash{}{0pt}%
\pgfpathmoveto{\pgfqpoint{0.510000in}{3.520000in}}%
\pgfpathlineto{\pgfqpoint{3.590000in}{3.520000in}}%
\pgfusepath{stroke}%
\end{pgfscope}%
\begin{pgfscope}%
\pgfsetbuttcap%
\pgfsetroundjoin%
\definecolor{currentfill}{rgb}{0.000000,0.000000,0.000000}%
\pgfsetfillcolor{currentfill}%
\pgfsetlinewidth{0.803000pt}%
\definecolor{currentstroke}{rgb}{0.000000,0.000000,0.000000}%
\pgfsetstrokecolor{currentstroke}%
\pgfsetdash{}{0pt}%
\pgfsys@defobject{currentmarker}{\pgfqpoint{-0.048611in}{0.000000in}}{\pgfqpoint{0.000000in}{0.000000in}}{%
\pgfpathmoveto{\pgfqpoint{0.000000in}{0.000000in}}%
\pgfpathlineto{\pgfqpoint{-0.048611in}{0.000000in}}%
\pgfusepath{stroke,fill}%
}%
\begin{pgfscope}%
\pgfsys@transformshift{0.510000in}{3.520000in}%
\pgfsys@useobject{currentmarker}{}%
\end{pgfscope}%
\end{pgfscope}%
\begin{pgfscope}%
\definecolor{textcolor}{rgb}{0.000000,0.000000,0.000000}%
\pgfsetstrokecolor{textcolor}%
\pgfsetfillcolor{textcolor}%
\pgftext[x=0.191898in,y=3.467238in,left,base]{\color{textcolor}\sffamily\fontsize{10.000000}{12.000000}\selectfont 1.2}%
\end{pgfscope}%
\begin{pgfscope}%
\pgfsetrectcap%
\pgfsetmiterjoin%
\pgfsetlinewidth{0.803000pt}%
\definecolor{currentstroke}{rgb}{0.000000,0.000000,0.000000}%
\pgfsetstrokecolor{currentstroke}%
\pgfsetdash{}{0pt}%
\pgfpathmoveto{\pgfqpoint{0.510000in}{0.440000in}}%
\pgfpathlineto{\pgfqpoint{0.510000in}{3.520000in}}%
\pgfusepath{stroke}%
\end{pgfscope}%
\begin{pgfscope}%
\pgfsetrectcap%
\pgfsetmiterjoin%
\pgfsetlinewidth{0.803000pt}%
\definecolor{currentstroke}{rgb}{0.000000,0.000000,0.000000}%
\pgfsetstrokecolor{currentstroke}%
\pgfsetdash{}{0pt}%
\pgfpathmoveto{\pgfqpoint{3.590000in}{0.440000in}}%
\pgfpathlineto{\pgfqpoint{3.590000in}{3.520000in}}%
\pgfusepath{stroke}%
\end{pgfscope}%
\begin{pgfscope}%
\pgfsetrectcap%
\pgfsetmiterjoin%
\pgfsetlinewidth{0.803000pt}%
\definecolor{currentstroke}{rgb}{0.000000,0.000000,0.000000}%
\pgfsetstrokecolor{currentstroke}%
\pgfsetdash{}{0pt}%
\pgfpathmoveto{\pgfqpoint{0.510000in}{0.440000in}}%
\pgfpathlineto{\pgfqpoint{3.590000in}{0.440000in}}%
\pgfusepath{stroke}%
\end{pgfscope}%
\begin{pgfscope}%
\pgfsetrectcap%
\pgfsetmiterjoin%
\pgfsetlinewidth{0.803000pt}%
\definecolor{currentstroke}{rgb}{0.000000,0.000000,0.000000}%
\pgfsetstrokecolor{currentstroke}%
\pgfsetdash{}{0pt}%
\pgfpathmoveto{\pgfqpoint{0.510000in}{3.520000in}}%
\pgfpathlineto{\pgfqpoint{3.590000in}{3.520000in}}%
\pgfusepath{stroke}%
\end{pgfscope}%
\begin{pgfscope}%
\definecolor{textcolor}{rgb}{0.000000,0.000000,0.000000}%
\pgfsetstrokecolor{textcolor}%
\pgfsetfillcolor{textcolor}%
\pgftext[x=0.968786in,y=2.113435in,left,base]{\color{textcolor}\sffamily\fontsize{10.000000}{12.000000}\selectfont 0.16667}%
\end{pgfscope}%
\begin{pgfscope}%
\definecolor{textcolor}{rgb}{0.000000,0.000000,0.000000}%
\pgfsetstrokecolor{textcolor}%
\pgfsetfillcolor{textcolor}%
\pgftext[x=1.897784in,y=0.898787in,left,base]{\color{textcolor}\sffamily\fontsize{10.000000}{12.000000}\selectfont 0.16667}%
\end{pgfscope}%
\begin{pgfscope}%
\definecolor{textcolor}{rgb}{0.000000,0.000000,0.000000}%
\pgfsetstrokecolor{textcolor}%
\pgfsetfillcolor{textcolor}%
\pgftext[x=2.183431in,y=1.827779in,left,base]{\color{textcolor}\sffamily\fontsize{10.000000}{12.000000}\selectfont 0.16667}%
\end{pgfscope}%
\end{pgfpicture}%
\makeatother%
\endgroup%
}
\caption{Specific heat $c$ for different temperatures $T$ using kinetic Monte Carlo algorithm}
\label{Fig:CapSmallKMC}
\end{figure}

\begin{figure}[htbp]
\centering
\scalebox{0.666}{%% Creator: Matplotlib, PGF backend
%%
%% To include the figure in your LaTeX document, write
%%   \input{<filename>.pgf}
%%
%% Make sure the required packages are loaded in your preamble
%%   \usepackage{pgf}
%%
%% Figures using additional raster images can only be included by \input if
%% they are in the same directory as the main LaTeX file. For loading figures
%% from other directories you can use the `import` package
%%   \usepackage{import}
%% and then include the figures with
%%   \import{<path to file>}{<filename>.pgf}
%%
%% Matplotlib used the following preamble
%%   \usepackage{fontspec}
%%   \setmainfont{DejaVuSerif.ttf}[Path=/home/lzh/anaconda3/envs/numana/lib/python3.7/site-packages/matplotlib/mpl-data/fonts/ttf/]
%%   \setsansfont{DejaVuSans.ttf}[Path=/home/lzh/anaconda3/envs/numana/lib/python3.7/site-packages/matplotlib/mpl-data/fonts/ttf/]
%%   \setmonofont{DejaVuSansMono.ttf}[Path=/home/lzh/anaconda3/envs/numana/lib/python3.7/site-packages/matplotlib/mpl-data/fonts/ttf/]
%%
\begingroup%
\makeatletter%
\begin{pgfpicture}%
\pgfpathrectangle{\pgfpointorigin}{\pgfqpoint{8.000000in}{6.000000in}}%
\pgfusepath{use as bounding box, clip}%
\begin{pgfscope}%
\pgfsetbuttcap%
\pgfsetmiterjoin%
\definecolor{currentfill}{rgb}{1.000000,1.000000,1.000000}%
\pgfsetfillcolor{currentfill}%
\pgfsetlinewidth{0.000000pt}%
\definecolor{currentstroke}{rgb}{1.000000,1.000000,1.000000}%
\pgfsetstrokecolor{currentstroke}%
\pgfsetdash{}{0pt}%
\pgfpathmoveto{\pgfqpoint{0.000000in}{0.000000in}}%
\pgfpathlineto{\pgfqpoint{8.000000in}{0.000000in}}%
\pgfpathlineto{\pgfqpoint{8.000000in}{6.000000in}}%
\pgfpathlineto{\pgfqpoint{0.000000in}{6.000000in}}%
\pgfpathclose%
\pgfusepath{fill}%
\end{pgfscope}%
\begin{pgfscope}%
\pgfsetbuttcap%
\pgfsetmiterjoin%
\definecolor{currentfill}{rgb}{1.000000,1.000000,1.000000}%
\pgfsetfillcolor{currentfill}%
\pgfsetlinewidth{0.000000pt}%
\definecolor{currentstroke}{rgb}{0.000000,0.000000,0.000000}%
\pgfsetstrokecolor{currentstroke}%
\pgfsetstrokeopacity{0.000000}%
\pgfsetdash{}{0pt}%
\pgfpathmoveto{\pgfqpoint{0.467222in}{3.312222in}}%
\pgfpathlineto{\pgfqpoint{3.890000in}{3.312222in}}%
\pgfpathlineto{\pgfqpoint{3.890000in}{5.636667in}}%
\pgfpathlineto{\pgfqpoint{0.467222in}{5.636667in}}%
\pgfpathclose%
\pgfusepath{fill}%
\end{pgfscope}%
\begin{pgfscope}%
\pgfpathrectangle{\pgfqpoint{0.467222in}{3.312222in}}{\pgfqpoint{3.422778in}{2.324444in}}%
\pgfusepath{clip}%
\pgfsetbuttcap%
\pgfsetroundjoin%
\definecolor{currentfill}{rgb}{0.121569,0.466667,0.705882}%
\pgfsetfillcolor{currentfill}%
\pgfsetfillopacity{0.300000}%
\pgfsetlinewidth{1.003750pt}%
\definecolor{currentstroke}{rgb}{0.121569,0.466667,0.705882}%
\pgfsetstrokecolor{currentstroke}%
\pgfsetstrokeopacity{0.300000}%
\pgfsetdash{}{0pt}%
\pgfpathmoveto{\pgfqpoint{0.631344in}{5.521941in}}%
\pgfpathlineto{\pgfqpoint{0.631344in}{5.521941in}}%
\pgfpathlineto{\pgfqpoint{0.708707in}{5.521948in}}%
\pgfpathlineto{\pgfqpoint{0.786071in}{5.521989in}}%
\pgfpathlineto{\pgfqpoint{0.863434in}{5.522134in}}%
\pgfpathlineto{\pgfqpoint{0.940797in}{5.522462in}}%
\pgfpathlineto{\pgfqpoint{1.018161in}{5.522943in}}%
\pgfpathlineto{\pgfqpoint{1.095524in}{5.523239in}}%
\pgfpathlineto{\pgfqpoint{1.172887in}{5.522601in}}%
\pgfpathlineto{\pgfqpoint{1.250251in}{5.519997in}}%
\pgfpathlineto{\pgfqpoint{1.327614in}{5.514364in}}%
\pgfpathlineto{\pgfqpoint{1.404978in}{5.504695in}}%
\pgfpathlineto{\pgfqpoint{1.482341in}{5.489891in}}%
\pgfpathlineto{\pgfqpoint{1.559704in}{5.468475in}}%
\pgfpathlineto{\pgfqpoint{1.637068in}{5.438169in}}%
\pgfpathlineto{\pgfqpoint{1.714431in}{5.395327in}}%
\pgfpathlineto{\pgfqpoint{1.791794in}{5.332638in}}%
\pgfpathlineto{\pgfqpoint{1.869158in}{5.232744in}}%
\pgfpathlineto{\pgfqpoint{1.946521in}{5.057498in}}%
\pgfpathlineto{\pgfqpoint{2.023884in}{4.768895in}}%
\pgfpathlineto{\pgfqpoint{2.101248in}{4.412220in}}%
\pgfpathlineto{\pgfqpoint{2.178611in}{4.113794in}}%
\pgfpathlineto{\pgfqpoint{2.255974in}{3.916993in}}%
\pgfpathlineto{\pgfqpoint{2.333338in}{3.793056in}}%
\pgfpathlineto{\pgfqpoint{2.410701in}{3.709897in}}%
\pgfpathlineto{\pgfqpoint{2.488065in}{3.651889in}}%
\pgfpathlineto{\pgfqpoint{2.565428in}{3.610422in}}%
\pgfpathlineto{\pgfqpoint{2.642791in}{3.577777in}}%
\pgfpathlineto{\pgfqpoint{2.720155in}{3.552773in}}%
\pgfpathlineto{\pgfqpoint{2.797518in}{3.531617in}}%
\pgfpathlineto{\pgfqpoint{2.874881in}{3.514319in}}%
\pgfpathlineto{\pgfqpoint{2.952245in}{3.499881in}}%
\pgfpathlineto{\pgfqpoint{3.029608in}{3.487076in}}%
\pgfpathlineto{\pgfqpoint{3.106971in}{3.476460in}}%
\pgfpathlineto{\pgfqpoint{3.184335in}{3.467786in}}%
\pgfpathlineto{\pgfqpoint{3.261698in}{3.459304in}}%
\pgfpathlineto{\pgfqpoint{3.339061in}{3.452307in}}%
\pgfpathlineto{\pgfqpoint{3.416425in}{3.445120in}}%
\pgfpathlineto{\pgfqpoint{3.493788in}{3.439743in}}%
\pgfpathlineto{\pgfqpoint{3.571152in}{3.434237in}}%
\pgfpathlineto{\pgfqpoint{3.648515in}{3.429783in}}%
\pgfpathlineto{\pgfqpoint{3.725878in}{3.425229in}}%
\pgfpathlineto{\pgfqpoint{3.725878in}{3.426070in}}%
\pgfpathlineto{\pgfqpoint{3.725878in}{3.426070in}}%
\pgfpathlineto{\pgfqpoint{3.648515in}{3.430561in}}%
\pgfpathlineto{\pgfqpoint{3.571152in}{3.434866in}}%
\pgfpathlineto{\pgfqpoint{3.493788in}{3.440656in}}%
\pgfpathlineto{\pgfqpoint{3.416425in}{3.445985in}}%
\pgfpathlineto{\pgfqpoint{3.339061in}{3.453199in}}%
\pgfpathlineto{\pgfqpoint{3.261698in}{3.460344in}}%
\pgfpathlineto{\pgfqpoint{3.184335in}{3.468422in}}%
\pgfpathlineto{\pgfqpoint{3.106971in}{3.477160in}}%
\pgfpathlineto{\pgfqpoint{3.029608in}{3.488851in}}%
\pgfpathlineto{\pgfqpoint{2.952245in}{3.500973in}}%
\pgfpathlineto{\pgfqpoint{2.874881in}{3.515159in}}%
\pgfpathlineto{\pgfqpoint{2.797518in}{3.532844in}}%
\pgfpathlineto{\pgfqpoint{2.720155in}{3.553482in}}%
\pgfpathlineto{\pgfqpoint{2.642791in}{3.579968in}}%
\pgfpathlineto{\pgfqpoint{2.565428in}{3.613044in}}%
\pgfpathlineto{\pgfqpoint{2.488065in}{3.654535in}}%
\pgfpathlineto{\pgfqpoint{2.410701in}{3.712490in}}%
\pgfpathlineto{\pgfqpoint{2.333338in}{3.798171in}}%
\pgfpathlineto{\pgfqpoint{2.255974in}{3.923111in}}%
\pgfpathlineto{\pgfqpoint{2.178611in}{4.121502in}}%
\pgfpathlineto{\pgfqpoint{2.101248in}{4.418961in}}%
\pgfpathlineto{\pgfqpoint{2.023884in}{4.771505in}}%
\pgfpathlineto{\pgfqpoint{1.946521in}{5.062678in}}%
\pgfpathlineto{\pgfqpoint{1.869158in}{5.234422in}}%
\pgfpathlineto{\pgfqpoint{1.791794in}{5.333055in}}%
\pgfpathlineto{\pgfqpoint{1.714431in}{5.395716in}}%
\pgfpathlineto{\pgfqpoint{1.637068in}{5.438345in}}%
\pgfpathlineto{\pgfqpoint{1.559704in}{5.468535in}}%
\pgfpathlineto{\pgfqpoint{1.482341in}{5.489920in}}%
\pgfpathlineto{\pgfqpoint{1.404978in}{5.504709in}}%
\pgfpathlineto{\pgfqpoint{1.327614in}{5.514375in}}%
\pgfpathlineto{\pgfqpoint{1.250251in}{5.520002in}}%
\pgfpathlineto{\pgfqpoint{1.172887in}{5.522602in}}%
\pgfpathlineto{\pgfqpoint{1.095524in}{5.523240in}}%
\pgfpathlineto{\pgfqpoint{1.018161in}{5.522943in}}%
\pgfpathlineto{\pgfqpoint{0.940797in}{5.522462in}}%
\pgfpathlineto{\pgfqpoint{0.863434in}{5.522134in}}%
\pgfpathlineto{\pgfqpoint{0.786071in}{5.521990in}}%
\pgfpathlineto{\pgfqpoint{0.708707in}{5.521948in}}%
\pgfpathlineto{\pgfqpoint{0.631344in}{5.521941in}}%
\pgfpathclose%
\pgfusepath{stroke,fill}%
\end{pgfscope}%
\begin{pgfscope}%
\pgfpathrectangle{\pgfqpoint{0.467222in}{3.312222in}}{\pgfqpoint{3.422778in}{2.324444in}}%
\pgfusepath{clip}%
\pgfsetbuttcap%
\pgfsetroundjoin%
\definecolor{currentfill}{rgb}{0.121569,0.466667,0.705882}%
\pgfsetfillcolor{currentfill}%
\pgfsetlinewidth{1.003750pt}%
\definecolor{currentstroke}{rgb}{0.121569,0.466667,0.705882}%
\pgfsetstrokecolor{currentstroke}%
\pgfsetdash{}{0pt}%
\pgfsys@defobject{currentmarker}{\pgfqpoint{-0.009821in}{-0.009821in}}{\pgfqpoint{0.009821in}{0.009821in}}{%
\pgfpathmoveto{\pgfqpoint{0.000000in}{-0.009821in}}%
\pgfpathcurveto{\pgfqpoint{0.002605in}{-0.009821in}}{\pgfqpoint{0.005103in}{-0.008786in}}{\pgfqpoint{0.006944in}{-0.006944in}}%
\pgfpathcurveto{\pgfqpoint{0.008786in}{-0.005103in}}{\pgfqpoint{0.009821in}{-0.002605in}}{\pgfqpoint{0.009821in}{0.000000in}}%
\pgfpathcurveto{\pgfqpoint{0.009821in}{0.002605in}}{\pgfqpoint{0.008786in}{0.005103in}}{\pgfqpoint{0.006944in}{0.006944in}}%
\pgfpathcurveto{\pgfqpoint{0.005103in}{0.008786in}}{\pgfqpoint{0.002605in}{0.009821in}}{\pgfqpoint{0.000000in}{0.009821in}}%
\pgfpathcurveto{\pgfqpoint{-0.002605in}{0.009821in}}{\pgfqpoint{-0.005103in}{0.008786in}}{\pgfqpoint{-0.006944in}{0.006944in}}%
\pgfpathcurveto{\pgfqpoint{-0.008786in}{0.005103in}}{\pgfqpoint{-0.009821in}{0.002605in}}{\pgfqpoint{-0.009821in}{0.000000in}}%
\pgfpathcurveto{\pgfqpoint{-0.009821in}{-0.002605in}}{\pgfqpoint{-0.008786in}{-0.005103in}}{\pgfqpoint{-0.006944in}{-0.006944in}}%
\pgfpathcurveto{\pgfqpoint{-0.005103in}{-0.008786in}}{\pgfqpoint{-0.002605in}{-0.009821in}}{\pgfqpoint{0.000000in}{-0.009821in}}%
\pgfpathclose%
\pgfusepath{stroke,fill}%
}%
\begin{pgfscope}%
\pgfsys@transformshift{0.631344in}{5.521941in}%
\pgfsys@useobject{currentmarker}{}%
\end{pgfscope}%
\begin{pgfscope}%
\pgfsys@transformshift{0.708707in}{5.521948in}%
\pgfsys@useobject{currentmarker}{}%
\end{pgfscope}%
\begin{pgfscope}%
\pgfsys@transformshift{0.786071in}{5.521990in}%
\pgfsys@useobject{currentmarker}{}%
\end{pgfscope}%
\begin{pgfscope}%
\pgfsys@transformshift{0.863434in}{5.522134in}%
\pgfsys@useobject{currentmarker}{}%
\end{pgfscope}%
\begin{pgfscope}%
\pgfsys@transformshift{0.940797in}{5.522462in}%
\pgfsys@useobject{currentmarker}{}%
\end{pgfscope}%
\begin{pgfscope}%
\pgfsys@transformshift{1.018161in}{5.522943in}%
\pgfsys@useobject{currentmarker}{}%
\end{pgfscope}%
\begin{pgfscope}%
\pgfsys@transformshift{1.095524in}{5.523239in}%
\pgfsys@useobject{currentmarker}{}%
\end{pgfscope}%
\begin{pgfscope}%
\pgfsys@transformshift{1.172887in}{5.522601in}%
\pgfsys@useobject{currentmarker}{}%
\end{pgfscope}%
\begin{pgfscope}%
\pgfsys@transformshift{1.250251in}{5.520000in}%
\pgfsys@useobject{currentmarker}{}%
\end{pgfscope}%
\begin{pgfscope}%
\pgfsys@transformshift{1.327614in}{5.514369in}%
\pgfsys@useobject{currentmarker}{}%
\end{pgfscope}%
\begin{pgfscope}%
\pgfsys@transformshift{1.404978in}{5.504702in}%
\pgfsys@useobject{currentmarker}{}%
\end{pgfscope}%
\begin{pgfscope}%
\pgfsys@transformshift{1.482341in}{5.489906in}%
\pgfsys@useobject{currentmarker}{}%
\end{pgfscope}%
\begin{pgfscope}%
\pgfsys@transformshift{1.559704in}{5.468505in}%
\pgfsys@useobject{currentmarker}{}%
\end{pgfscope}%
\begin{pgfscope}%
\pgfsys@transformshift{1.637068in}{5.438257in}%
\pgfsys@useobject{currentmarker}{}%
\end{pgfscope}%
\begin{pgfscope}%
\pgfsys@transformshift{1.714431in}{5.395521in}%
\pgfsys@useobject{currentmarker}{}%
\end{pgfscope}%
\begin{pgfscope}%
\pgfsys@transformshift{1.791794in}{5.332847in}%
\pgfsys@useobject{currentmarker}{}%
\end{pgfscope}%
\begin{pgfscope}%
\pgfsys@transformshift{1.869158in}{5.233583in}%
\pgfsys@useobject{currentmarker}{}%
\end{pgfscope}%
\begin{pgfscope}%
\pgfsys@transformshift{1.946521in}{5.060088in}%
\pgfsys@useobject{currentmarker}{}%
\end{pgfscope}%
\begin{pgfscope}%
\pgfsys@transformshift{2.023884in}{4.770200in}%
\pgfsys@useobject{currentmarker}{}%
\end{pgfscope}%
\begin{pgfscope}%
\pgfsys@transformshift{2.101248in}{4.415590in}%
\pgfsys@useobject{currentmarker}{}%
\end{pgfscope}%
\begin{pgfscope}%
\pgfsys@transformshift{2.178611in}{4.117648in}%
\pgfsys@useobject{currentmarker}{}%
\end{pgfscope}%
\begin{pgfscope}%
\pgfsys@transformshift{2.255974in}{3.920052in}%
\pgfsys@useobject{currentmarker}{}%
\end{pgfscope}%
\begin{pgfscope}%
\pgfsys@transformshift{2.333338in}{3.795613in}%
\pgfsys@useobject{currentmarker}{}%
\end{pgfscope}%
\begin{pgfscope}%
\pgfsys@transformshift{2.410701in}{3.711194in}%
\pgfsys@useobject{currentmarker}{}%
\end{pgfscope}%
\begin{pgfscope}%
\pgfsys@transformshift{2.488065in}{3.653212in}%
\pgfsys@useobject{currentmarker}{}%
\end{pgfscope}%
\begin{pgfscope}%
\pgfsys@transformshift{2.565428in}{3.611733in}%
\pgfsys@useobject{currentmarker}{}%
\end{pgfscope}%
\begin{pgfscope}%
\pgfsys@transformshift{2.642791in}{3.578873in}%
\pgfsys@useobject{currentmarker}{}%
\end{pgfscope}%
\begin{pgfscope}%
\pgfsys@transformshift{2.720155in}{3.553127in}%
\pgfsys@useobject{currentmarker}{}%
\end{pgfscope}%
\begin{pgfscope}%
\pgfsys@transformshift{2.797518in}{3.532230in}%
\pgfsys@useobject{currentmarker}{}%
\end{pgfscope}%
\begin{pgfscope}%
\pgfsys@transformshift{2.874881in}{3.514739in}%
\pgfsys@useobject{currentmarker}{}%
\end{pgfscope}%
\begin{pgfscope}%
\pgfsys@transformshift{2.952245in}{3.500427in}%
\pgfsys@useobject{currentmarker}{}%
\end{pgfscope}%
\begin{pgfscope}%
\pgfsys@transformshift{3.029608in}{3.487963in}%
\pgfsys@useobject{currentmarker}{}%
\end{pgfscope}%
\begin{pgfscope}%
\pgfsys@transformshift{3.106971in}{3.476810in}%
\pgfsys@useobject{currentmarker}{}%
\end{pgfscope}%
\begin{pgfscope}%
\pgfsys@transformshift{3.184335in}{3.468104in}%
\pgfsys@useobject{currentmarker}{}%
\end{pgfscope}%
\begin{pgfscope}%
\pgfsys@transformshift{3.261698in}{3.459824in}%
\pgfsys@useobject{currentmarker}{}%
\end{pgfscope}%
\begin{pgfscope}%
\pgfsys@transformshift{3.339061in}{3.452753in}%
\pgfsys@useobject{currentmarker}{}%
\end{pgfscope}%
\begin{pgfscope}%
\pgfsys@transformshift{3.416425in}{3.445552in}%
\pgfsys@useobject{currentmarker}{}%
\end{pgfscope}%
\begin{pgfscope}%
\pgfsys@transformshift{3.493788in}{3.440199in}%
\pgfsys@useobject{currentmarker}{}%
\end{pgfscope}%
\begin{pgfscope}%
\pgfsys@transformshift{3.571152in}{3.434552in}%
\pgfsys@useobject{currentmarker}{}%
\end{pgfscope}%
\begin{pgfscope}%
\pgfsys@transformshift{3.648515in}{3.430172in}%
\pgfsys@useobject{currentmarker}{}%
\end{pgfscope}%
\begin{pgfscope}%
\pgfsys@transformshift{3.725878in}{3.425650in}%
\pgfsys@useobject{currentmarker}{}%
\end{pgfscope}%
\end{pgfscope}%
\begin{pgfscope}%
\pgfsetbuttcap%
\pgfsetroundjoin%
\definecolor{currentfill}{rgb}{0.000000,0.000000,0.000000}%
\pgfsetfillcolor{currentfill}%
\pgfsetlinewidth{0.803000pt}%
\definecolor{currentstroke}{rgb}{0.000000,0.000000,0.000000}%
\pgfsetstrokecolor{currentstroke}%
\pgfsetdash{}{0pt}%
\pgfsys@defobject{currentmarker}{\pgfqpoint{0.000000in}{-0.048611in}}{\pgfqpoint{0.000000in}{0.000000in}}{%
\pgfpathmoveto{\pgfqpoint{0.000000in}{0.000000in}}%
\pgfpathlineto{\pgfqpoint{0.000000in}{-0.048611in}}%
\pgfusepath{stroke,fill}%
}%
\begin{pgfscope}%
\pgfsys@transformshift{1.018161in}{3.312222in}%
\pgfsys@useobject{currentmarker}{}%
\end{pgfscope}%
\end{pgfscope}%
\begin{pgfscope}%
\definecolor{textcolor}{rgb}{0.000000,0.000000,0.000000}%
\pgfsetstrokecolor{textcolor}%
\pgfsetfillcolor{textcolor}%
\pgftext[x=1.018161in,y=3.215000in,,top]{\color{textcolor}\sffamily\fontsize{10.000000}{12.000000}\selectfont 1}%
\end{pgfscope}%
\begin{pgfscope}%
\pgfsetbuttcap%
\pgfsetroundjoin%
\definecolor{currentfill}{rgb}{0.000000,0.000000,0.000000}%
\pgfsetfillcolor{currentfill}%
\pgfsetlinewidth{0.803000pt}%
\definecolor{currentstroke}{rgb}{0.000000,0.000000,0.000000}%
\pgfsetstrokecolor{currentstroke}%
\pgfsetdash{}{0pt}%
\pgfsys@defobject{currentmarker}{\pgfqpoint{0.000000in}{-0.048611in}}{\pgfqpoint{0.000000in}{0.000000in}}{%
\pgfpathmoveto{\pgfqpoint{0.000000in}{0.000000in}}%
\pgfpathlineto{\pgfqpoint{0.000000in}{-0.048611in}}%
\pgfusepath{stroke,fill}%
}%
\begin{pgfscope}%
\pgfsys@transformshift{1.791794in}{3.312222in}%
\pgfsys@useobject{currentmarker}{}%
\end{pgfscope}%
\end{pgfscope}%
\begin{pgfscope}%
\definecolor{textcolor}{rgb}{0.000000,0.000000,0.000000}%
\pgfsetstrokecolor{textcolor}%
\pgfsetfillcolor{textcolor}%
\pgftext[x=1.791794in,y=3.215000in,,top]{\color{textcolor}\sffamily\fontsize{10.000000}{12.000000}\selectfont 2}%
\end{pgfscope}%
\begin{pgfscope}%
\pgfsetbuttcap%
\pgfsetroundjoin%
\definecolor{currentfill}{rgb}{0.000000,0.000000,0.000000}%
\pgfsetfillcolor{currentfill}%
\pgfsetlinewidth{0.803000pt}%
\definecolor{currentstroke}{rgb}{0.000000,0.000000,0.000000}%
\pgfsetstrokecolor{currentstroke}%
\pgfsetdash{}{0pt}%
\pgfsys@defobject{currentmarker}{\pgfqpoint{0.000000in}{-0.048611in}}{\pgfqpoint{0.000000in}{0.000000in}}{%
\pgfpathmoveto{\pgfqpoint{0.000000in}{0.000000in}}%
\pgfpathlineto{\pgfqpoint{0.000000in}{-0.048611in}}%
\pgfusepath{stroke,fill}%
}%
\begin{pgfscope}%
\pgfsys@transformshift{2.565428in}{3.312222in}%
\pgfsys@useobject{currentmarker}{}%
\end{pgfscope}%
\end{pgfscope}%
\begin{pgfscope}%
\definecolor{textcolor}{rgb}{0.000000,0.000000,0.000000}%
\pgfsetstrokecolor{textcolor}%
\pgfsetfillcolor{textcolor}%
\pgftext[x=2.565428in,y=3.215000in,,top]{\color{textcolor}\sffamily\fontsize{10.000000}{12.000000}\selectfont 3}%
\end{pgfscope}%
\begin{pgfscope}%
\pgfsetbuttcap%
\pgfsetroundjoin%
\definecolor{currentfill}{rgb}{0.000000,0.000000,0.000000}%
\pgfsetfillcolor{currentfill}%
\pgfsetlinewidth{0.803000pt}%
\definecolor{currentstroke}{rgb}{0.000000,0.000000,0.000000}%
\pgfsetstrokecolor{currentstroke}%
\pgfsetdash{}{0pt}%
\pgfsys@defobject{currentmarker}{\pgfqpoint{0.000000in}{-0.048611in}}{\pgfqpoint{0.000000in}{0.000000in}}{%
\pgfpathmoveto{\pgfqpoint{0.000000in}{0.000000in}}%
\pgfpathlineto{\pgfqpoint{0.000000in}{-0.048611in}}%
\pgfusepath{stroke,fill}%
}%
\begin{pgfscope}%
\pgfsys@transformshift{3.339061in}{3.312222in}%
\pgfsys@useobject{currentmarker}{}%
\end{pgfscope}%
\end{pgfscope}%
\begin{pgfscope}%
\definecolor{textcolor}{rgb}{0.000000,0.000000,0.000000}%
\pgfsetstrokecolor{textcolor}%
\pgfsetfillcolor{textcolor}%
\pgftext[x=3.339061in,y=3.215000in,,top]{\color{textcolor}\sffamily\fontsize{10.000000}{12.000000}\selectfont 4}%
\end{pgfscope}%
\begin{pgfscope}%
\pgfsetbuttcap%
\pgfsetroundjoin%
\definecolor{currentfill}{rgb}{0.000000,0.000000,0.000000}%
\pgfsetfillcolor{currentfill}%
\pgfsetlinewidth{0.803000pt}%
\definecolor{currentstroke}{rgb}{0.000000,0.000000,0.000000}%
\pgfsetstrokecolor{currentstroke}%
\pgfsetdash{}{0pt}%
\pgfsys@defobject{currentmarker}{\pgfqpoint{-0.048611in}{0.000000in}}{\pgfqpoint{0.000000in}{0.000000in}}{%
\pgfpathmoveto{\pgfqpoint{0.000000in}{0.000000in}}%
\pgfpathlineto{\pgfqpoint{-0.048611in}{0.000000in}}%
\pgfusepath{stroke,fill}%
}%
\begin{pgfscope}%
\pgfsys@transformshift{0.467222in}{3.677332in}%
\pgfsys@useobject{currentmarker}{}%
\end{pgfscope}%
\end{pgfscope}%
\begin{pgfscope}%
\definecolor{textcolor}{rgb}{0.000000,0.000000,0.000000}%
\pgfsetstrokecolor{textcolor}%
\pgfsetfillcolor{textcolor}%
\pgftext[x=0.149121in,y=3.624570in,left,base]{\color{textcolor}\sffamily\fontsize{10.000000}{12.000000}\selectfont 0.2}%
\end{pgfscope}%
\begin{pgfscope}%
\pgfsetbuttcap%
\pgfsetroundjoin%
\definecolor{currentfill}{rgb}{0.000000,0.000000,0.000000}%
\pgfsetfillcolor{currentfill}%
\pgfsetlinewidth{0.803000pt}%
\definecolor{currentstroke}{rgb}{0.000000,0.000000,0.000000}%
\pgfsetstrokecolor{currentstroke}%
\pgfsetdash{}{0pt}%
\pgfsys@defobject{currentmarker}{\pgfqpoint{-0.048611in}{0.000000in}}{\pgfqpoint{0.000000in}{0.000000in}}{%
\pgfpathmoveto{\pgfqpoint{0.000000in}{0.000000in}}%
\pgfpathlineto{\pgfqpoint{-0.048611in}{0.000000in}}%
\pgfusepath{stroke,fill}%
}%
\begin{pgfscope}%
\pgfsys@transformshift{0.467222in}{4.143032in}%
\pgfsys@useobject{currentmarker}{}%
\end{pgfscope}%
\end{pgfscope}%
\begin{pgfscope}%
\definecolor{textcolor}{rgb}{0.000000,0.000000,0.000000}%
\pgfsetstrokecolor{textcolor}%
\pgfsetfillcolor{textcolor}%
\pgftext[x=0.149121in,y=4.090270in,left,base]{\color{textcolor}\sffamily\fontsize{10.000000}{12.000000}\selectfont 0.4}%
\end{pgfscope}%
\begin{pgfscope}%
\pgfsetbuttcap%
\pgfsetroundjoin%
\definecolor{currentfill}{rgb}{0.000000,0.000000,0.000000}%
\pgfsetfillcolor{currentfill}%
\pgfsetlinewidth{0.803000pt}%
\definecolor{currentstroke}{rgb}{0.000000,0.000000,0.000000}%
\pgfsetstrokecolor{currentstroke}%
\pgfsetdash{}{0pt}%
\pgfsys@defobject{currentmarker}{\pgfqpoint{-0.048611in}{0.000000in}}{\pgfqpoint{0.000000in}{0.000000in}}{%
\pgfpathmoveto{\pgfqpoint{0.000000in}{0.000000in}}%
\pgfpathlineto{\pgfqpoint{-0.048611in}{0.000000in}}%
\pgfusepath{stroke,fill}%
}%
\begin{pgfscope}%
\pgfsys@transformshift{0.467222in}{4.608732in}%
\pgfsys@useobject{currentmarker}{}%
\end{pgfscope}%
\end{pgfscope}%
\begin{pgfscope}%
\definecolor{textcolor}{rgb}{0.000000,0.000000,0.000000}%
\pgfsetstrokecolor{textcolor}%
\pgfsetfillcolor{textcolor}%
\pgftext[x=0.149121in,y=4.555970in,left,base]{\color{textcolor}\sffamily\fontsize{10.000000}{12.000000}\selectfont 0.6}%
\end{pgfscope}%
\begin{pgfscope}%
\pgfsetbuttcap%
\pgfsetroundjoin%
\definecolor{currentfill}{rgb}{0.000000,0.000000,0.000000}%
\pgfsetfillcolor{currentfill}%
\pgfsetlinewidth{0.803000pt}%
\definecolor{currentstroke}{rgb}{0.000000,0.000000,0.000000}%
\pgfsetstrokecolor{currentstroke}%
\pgfsetdash{}{0pt}%
\pgfsys@defobject{currentmarker}{\pgfqpoint{-0.048611in}{0.000000in}}{\pgfqpoint{0.000000in}{0.000000in}}{%
\pgfpathmoveto{\pgfqpoint{0.000000in}{0.000000in}}%
\pgfpathlineto{\pgfqpoint{-0.048611in}{0.000000in}}%
\pgfusepath{stroke,fill}%
}%
\begin{pgfscope}%
\pgfsys@transformshift{0.467222in}{5.074432in}%
\pgfsys@useobject{currentmarker}{}%
\end{pgfscope}%
\end{pgfscope}%
\begin{pgfscope}%
\definecolor{textcolor}{rgb}{0.000000,0.000000,0.000000}%
\pgfsetstrokecolor{textcolor}%
\pgfsetfillcolor{textcolor}%
\pgftext[x=0.149121in,y=5.021670in,left,base]{\color{textcolor}\sffamily\fontsize{10.000000}{12.000000}\selectfont 0.8}%
\end{pgfscope}%
\begin{pgfscope}%
\pgfsetbuttcap%
\pgfsetroundjoin%
\definecolor{currentfill}{rgb}{0.000000,0.000000,0.000000}%
\pgfsetfillcolor{currentfill}%
\pgfsetlinewidth{0.803000pt}%
\definecolor{currentstroke}{rgb}{0.000000,0.000000,0.000000}%
\pgfsetstrokecolor{currentstroke}%
\pgfsetdash{}{0pt}%
\pgfsys@defobject{currentmarker}{\pgfqpoint{-0.048611in}{0.000000in}}{\pgfqpoint{0.000000in}{0.000000in}}{%
\pgfpathmoveto{\pgfqpoint{0.000000in}{0.000000in}}%
\pgfpathlineto{\pgfqpoint{-0.048611in}{0.000000in}}%
\pgfusepath{stroke,fill}%
}%
\begin{pgfscope}%
\pgfsys@transformshift{0.467222in}{5.540132in}%
\pgfsys@useobject{currentmarker}{}%
\end{pgfscope}%
\end{pgfscope}%
\begin{pgfscope}%
\definecolor{textcolor}{rgb}{0.000000,0.000000,0.000000}%
\pgfsetstrokecolor{textcolor}%
\pgfsetfillcolor{textcolor}%
\pgftext[x=0.149121in,y=5.487370in,left,base]{\color{textcolor}\sffamily\fontsize{10.000000}{12.000000}\selectfont 1.0}%
\end{pgfscope}%
\begin{pgfscope}%
\pgfpathrectangle{\pgfqpoint{0.467222in}{3.312222in}}{\pgfqpoint{3.422778in}{2.324444in}}%
\pgfusepath{clip}%
\pgfsetrectcap%
\pgfsetroundjoin%
\pgfsetlinewidth{1.505625pt}%
\definecolor{currentstroke}{rgb}{0.121569,0.466667,0.705882}%
\pgfsetstrokecolor{currentstroke}%
\pgfsetdash{}{0pt}%
\pgfpathmoveto{\pgfqpoint{0.631344in}{5.521941in}}%
\pgfpathlineto{\pgfqpoint{0.708707in}{5.521948in}}%
\pgfpathlineto{\pgfqpoint{0.786071in}{5.521990in}}%
\pgfpathlineto{\pgfqpoint{0.863434in}{5.522134in}}%
\pgfpathlineto{\pgfqpoint{0.940797in}{5.522462in}}%
\pgfpathlineto{\pgfqpoint{1.018161in}{5.522943in}}%
\pgfpathlineto{\pgfqpoint{1.095524in}{5.523239in}}%
\pgfpathlineto{\pgfqpoint{1.172887in}{5.522601in}}%
\pgfpathlineto{\pgfqpoint{1.250251in}{5.520000in}}%
\pgfpathlineto{\pgfqpoint{1.327614in}{5.514369in}}%
\pgfpathlineto{\pgfqpoint{1.404978in}{5.504702in}}%
\pgfpathlineto{\pgfqpoint{1.482341in}{5.489906in}}%
\pgfpathlineto{\pgfqpoint{1.559704in}{5.468505in}}%
\pgfpathlineto{\pgfqpoint{1.637068in}{5.438257in}}%
\pgfpathlineto{\pgfqpoint{1.714431in}{5.395521in}}%
\pgfpathlineto{\pgfqpoint{1.791794in}{5.332847in}}%
\pgfpathlineto{\pgfqpoint{1.869158in}{5.233583in}}%
\pgfpathlineto{\pgfqpoint{1.946521in}{5.060088in}}%
\pgfpathlineto{\pgfqpoint{2.023884in}{4.770200in}}%
\pgfpathlineto{\pgfqpoint{2.101248in}{4.415590in}}%
\pgfpathlineto{\pgfqpoint{2.178611in}{4.117648in}}%
\pgfpathlineto{\pgfqpoint{2.255974in}{3.920052in}}%
\pgfpathlineto{\pgfqpoint{2.333338in}{3.795613in}}%
\pgfpathlineto{\pgfqpoint{2.410701in}{3.711194in}}%
\pgfpathlineto{\pgfqpoint{2.488065in}{3.653212in}}%
\pgfpathlineto{\pgfqpoint{2.565428in}{3.611733in}}%
\pgfpathlineto{\pgfqpoint{2.642791in}{3.578873in}}%
\pgfpathlineto{\pgfqpoint{2.720155in}{3.553127in}}%
\pgfpathlineto{\pgfqpoint{2.797518in}{3.532230in}}%
\pgfpathlineto{\pgfqpoint{2.874881in}{3.514739in}}%
\pgfpathlineto{\pgfqpoint{2.952245in}{3.500427in}}%
\pgfpathlineto{\pgfqpoint{3.029608in}{3.487963in}}%
\pgfpathlineto{\pgfqpoint{3.106971in}{3.476810in}}%
\pgfpathlineto{\pgfqpoint{3.184335in}{3.468104in}}%
\pgfpathlineto{\pgfqpoint{3.261698in}{3.459824in}}%
\pgfpathlineto{\pgfqpoint{3.339061in}{3.452753in}}%
\pgfpathlineto{\pgfqpoint{3.416425in}{3.445552in}}%
\pgfpathlineto{\pgfqpoint{3.493788in}{3.440199in}}%
\pgfpathlineto{\pgfqpoint{3.571152in}{3.434552in}}%
\pgfpathlineto{\pgfqpoint{3.648515in}{3.430172in}}%
\pgfpathlineto{\pgfqpoint{3.725878in}{3.425650in}}%
\pgfusepath{stroke}%
\end{pgfscope}%
\begin{pgfscope}%
\pgfsetrectcap%
\pgfsetmiterjoin%
\pgfsetlinewidth{0.803000pt}%
\definecolor{currentstroke}{rgb}{0.000000,0.000000,0.000000}%
\pgfsetstrokecolor{currentstroke}%
\pgfsetdash{}{0pt}%
\pgfpathmoveto{\pgfqpoint{0.467222in}{3.312222in}}%
\pgfpathlineto{\pgfqpoint{0.467222in}{5.636667in}}%
\pgfusepath{stroke}%
\end{pgfscope}%
\begin{pgfscope}%
\pgfsetrectcap%
\pgfsetmiterjoin%
\pgfsetlinewidth{0.803000pt}%
\definecolor{currentstroke}{rgb}{0.000000,0.000000,0.000000}%
\pgfsetstrokecolor{currentstroke}%
\pgfsetdash{}{0pt}%
\pgfpathmoveto{\pgfqpoint{3.890000in}{3.312222in}}%
\pgfpathlineto{\pgfqpoint{3.890000in}{5.636667in}}%
\pgfusepath{stroke}%
\end{pgfscope}%
\begin{pgfscope}%
\pgfsetrectcap%
\pgfsetmiterjoin%
\pgfsetlinewidth{0.803000pt}%
\definecolor{currentstroke}{rgb}{0.000000,0.000000,0.000000}%
\pgfsetstrokecolor{currentstroke}%
\pgfsetdash{}{0pt}%
\pgfpathmoveto{\pgfqpoint{0.467222in}{3.312222in}}%
\pgfpathlineto{\pgfqpoint{3.890000in}{3.312222in}}%
\pgfusepath{stroke}%
\end{pgfscope}%
\begin{pgfscope}%
\pgfsetrectcap%
\pgfsetmiterjoin%
\pgfsetlinewidth{0.803000pt}%
\definecolor{currentstroke}{rgb}{0.000000,0.000000,0.000000}%
\pgfsetstrokecolor{currentstroke}%
\pgfsetdash{}{0pt}%
\pgfpathmoveto{\pgfqpoint{0.467222in}{5.636667in}}%
\pgfpathlineto{\pgfqpoint{3.890000in}{5.636667in}}%
\pgfusepath{stroke}%
\end{pgfscope}%
\begin{pgfscope}%
\definecolor{textcolor}{rgb}{0.000000,0.000000,0.000000}%
\pgfsetstrokecolor{textcolor}%
\pgfsetfillcolor{textcolor}%
\pgftext[x=2.178611in,y=5.720000in,,base]{\color{textcolor}\sffamily\fontsize{12.000000}{14.400000}\selectfont \(\displaystyle  N = 16 \)}%
\end{pgfscope}%
\begin{pgfscope}%
\pgfsetbuttcap%
\pgfsetmiterjoin%
\definecolor{currentfill}{rgb}{1.000000,1.000000,1.000000}%
\pgfsetfillcolor{currentfill}%
\pgfsetlinewidth{0.000000pt}%
\definecolor{currentstroke}{rgb}{0.000000,0.000000,0.000000}%
\pgfsetstrokecolor{currentstroke}%
\pgfsetstrokeopacity{0.000000}%
\pgfsetdash{}{0pt}%
\pgfpathmoveto{\pgfqpoint{4.392222in}{3.312222in}}%
\pgfpathlineto{\pgfqpoint{7.815000in}{3.312222in}}%
\pgfpathlineto{\pgfqpoint{7.815000in}{5.636667in}}%
\pgfpathlineto{\pgfqpoint{4.392222in}{5.636667in}}%
\pgfpathclose%
\pgfusepath{fill}%
\end{pgfscope}%
\begin{pgfscope}%
\pgfpathrectangle{\pgfqpoint{4.392222in}{3.312222in}}{\pgfqpoint{3.422778in}{2.324444in}}%
\pgfusepath{clip}%
\pgfsetbuttcap%
\pgfsetroundjoin%
\definecolor{currentfill}{rgb}{0.121569,0.466667,0.705882}%
\pgfsetfillcolor{currentfill}%
\pgfsetfillopacity{0.300000}%
\pgfsetlinewidth{1.003750pt}%
\definecolor{currentstroke}{rgb}{0.121569,0.466667,0.705882}%
\pgfsetstrokecolor{currentstroke}%
\pgfsetstrokeopacity{0.300000}%
\pgfsetdash{}{0pt}%
\pgfpathmoveto{\pgfqpoint{4.556344in}{5.522897in}}%
\pgfpathlineto{\pgfqpoint{4.556344in}{5.522896in}}%
\pgfpathlineto{\pgfqpoint{4.633707in}{5.522903in}}%
\pgfpathlineto{\pgfqpoint{4.711071in}{5.522941in}}%
\pgfpathlineto{\pgfqpoint{4.788434in}{5.523060in}}%
\pgfpathlineto{\pgfqpoint{4.865797in}{5.523240in}}%
\pgfpathlineto{\pgfqpoint{4.943161in}{5.523177in}}%
\pgfpathlineto{\pgfqpoint{5.020524in}{5.522235in}}%
\pgfpathlineto{\pgfqpoint{5.097887in}{5.519738in}}%
\pgfpathlineto{\pgfqpoint{5.175251in}{5.515140in}}%
\pgfpathlineto{\pgfqpoint{5.252614in}{5.507897in}}%
\pgfpathlineto{\pgfqpoint{5.329978in}{5.497238in}}%
\pgfpathlineto{\pgfqpoint{5.407341in}{5.482154in}}%
\pgfpathlineto{\pgfqpoint{5.484704in}{5.461160in}}%
\pgfpathlineto{\pgfqpoint{5.562068in}{5.432029in}}%
\pgfpathlineto{\pgfqpoint{5.639431in}{5.391145in}}%
\pgfpathlineto{\pgfqpoint{5.716794in}{5.331446in}}%
\pgfpathlineto{\pgfqpoint{5.794158in}{5.237620in}}%
\pgfpathlineto{\pgfqpoint{5.871521in}{5.053818in}}%
\pgfpathlineto{\pgfqpoint{5.948884in}{4.566452in}}%
\pgfpathlineto{\pgfqpoint{6.026248in}{4.006557in}}%
\pgfpathlineto{\pgfqpoint{6.103611in}{3.762465in}}%
\pgfpathlineto{\pgfqpoint{6.180974in}{3.654250in}}%
\pgfpathlineto{\pgfqpoint{6.258338in}{3.593128in}}%
\pgfpathlineto{\pgfqpoint{6.335701in}{3.556936in}}%
\pgfpathlineto{\pgfqpoint{6.413065in}{3.530496in}}%
\pgfpathlineto{\pgfqpoint{6.490428in}{3.510111in}}%
\pgfpathlineto{\pgfqpoint{6.567791in}{3.494997in}}%
\pgfpathlineto{\pgfqpoint{6.645155in}{3.483920in}}%
\pgfpathlineto{\pgfqpoint{6.722518in}{3.474640in}}%
\pgfpathlineto{\pgfqpoint{6.799881in}{3.467347in}}%
\pgfpathlineto{\pgfqpoint{6.877245in}{3.460412in}}%
\pgfpathlineto{\pgfqpoint{6.954608in}{3.454123in}}%
\pgfpathlineto{\pgfqpoint{7.031971in}{3.449405in}}%
\pgfpathlineto{\pgfqpoint{7.109335in}{3.445179in}}%
\pgfpathlineto{\pgfqpoint{7.186698in}{3.440720in}}%
\pgfpathlineto{\pgfqpoint{7.264061in}{3.437799in}}%
\pgfpathlineto{\pgfqpoint{7.341425in}{3.435262in}}%
\pgfpathlineto{\pgfqpoint{7.418788in}{3.432000in}}%
\pgfpathlineto{\pgfqpoint{7.496152in}{3.429777in}}%
\pgfpathlineto{\pgfqpoint{7.573515in}{3.427554in}}%
\pgfpathlineto{\pgfqpoint{7.650878in}{3.425394in}}%
\pgfpathlineto{\pgfqpoint{7.650878in}{3.425904in}}%
\pgfpathlineto{\pgfqpoint{7.650878in}{3.425904in}}%
\pgfpathlineto{\pgfqpoint{7.573515in}{3.428386in}}%
\pgfpathlineto{\pgfqpoint{7.496152in}{3.430219in}}%
\pgfpathlineto{\pgfqpoint{7.418788in}{3.432830in}}%
\pgfpathlineto{\pgfqpoint{7.341425in}{3.435788in}}%
\pgfpathlineto{\pgfqpoint{7.264061in}{3.438221in}}%
\pgfpathlineto{\pgfqpoint{7.186698in}{3.442181in}}%
\pgfpathlineto{\pgfqpoint{7.109335in}{3.445777in}}%
\pgfpathlineto{\pgfqpoint{7.031971in}{3.450085in}}%
\pgfpathlineto{\pgfqpoint{6.954608in}{3.455554in}}%
\pgfpathlineto{\pgfqpoint{6.877245in}{3.460949in}}%
\pgfpathlineto{\pgfqpoint{6.799881in}{3.468432in}}%
\pgfpathlineto{\pgfqpoint{6.722518in}{3.475613in}}%
\pgfpathlineto{\pgfqpoint{6.645155in}{3.486441in}}%
\pgfpathlineto{\pgfqpoint{6.567791in}{3.498377in}}%
\pgfpathlineto{\pgfqpoint{6.490428in}{3.512495in}}%
\pgfpathlineto{\pgfqpoint{6.413065in}{3.532589in}}%
\pgfpathlineto{\pgfqpoint{6.335701in}{3.561255in}}%
\pgfpathlineto{\pgfqpoint{6.258338in}{3.600088in}}%
\pgfpathlineto{\pgfqpoint{6.180974in}{3.657367in}}%
\pgfpathlineto{\pgfqpoint{6.103611in}{3.772185in}}%
\pgfpathlineto{\pgfqpoint{6.026248in}{4.031754in}}%
\pgfpathlineto{\pgfqpoint{5.948884in}{4.584222in}}%
\pgfpathlineto{\pgfqpoint{5.871521in}{5.056877in}}%
\pgfpathlineto{\pgfqpoint{5.794158in}{5.239017in}}%
\pgfpathlineto{\pgfqpoint{5.716794in}{5.332106in}}%
\pgfpathlineto{\pgfqpoint{5.639431in}{5.391277in}}%
\pgfpathlineto{\pgfqpoint{5.562068in}{5.432158in}}%
\pgfpathlineto{\pgfqpoint{5.484704in}{5.461223in}}%
\pgfpathlineto{\pgfqpoint{5.407341in}{5.482200in}}%
\pgfpathlineto{\pgfqpoint{5.329978in}{5.497258in}}%
\pgfpathlineto{\pgfqpoint{5.252614in}{5.507906in}}%
\pgfpathlineto{\pgfqpoint{5.175251in}{5.515149in}}%
\pgfpathlineto{\pgfqpoint{5.097887in}{5.519740in}}%
\pgfpathlineto{\pgfqpoint{5.020524in}{5.522236in}}%
\pgfpathlineto{\pgfqpoint{4.943161in}{5.523178in}}%
\pgfpathlineto{\pgfqpoint{4.865797in}{5.523240in}}%
\pgfpathlineto{\pgfqpoint{4.788434in}{5.523060in}}%
\pgfpathlineto{\pgfqpoint{4.711071in}{5.522941in}}%
\pgfpathlineto{\pgfqpoint{4.633707in}{5.522903in}}%
\pgfpathlineto{\pgfqpoint{4.556344in}{5.522897in}}%
\pgfpathclose%
\pgfusepath{stroke,fill}%
\end{pgfscope}%
\begin{pgfscope}%
\pgfpathrectangle{\pgfqpoint{4.392222in}{3.312222in}}{\pgfqpoint{3.422778in}{2.324444in}}%
\pgfusepath{clip}%
\pgfsetbuttcap%
\pgfsetroundjoin%
\definecolor{currentfill}{rgb}{0.121569,0.466667,0.705882}%
\pgfsetfillcolor{currentfill}%
\pgfsetlinewidth{1.003750pt}%
\definecolor{currentstroke}{rgb}{0.121569,0.466667,0.705882}%
\pgfsetstrokecolor{currentstroke}%
\pgfsetdash{}{0pt}%
\pgfsys@defobject{currentmarker}{\pgfqpoint{-0.009821in}{-0.009821in}}{\pgfqpoint{0.009821in}{0.009821in}}{%
\pgfpathmoveto{\pgfqpoint{0.000000in}{-0.009821in}}%
\pgfpathcurveto{\pgfqpoint{0.002605in}{-0.009821in}}{\pgfqpoint{0.005103in}{-0.008786in}}{\pgfqpoint{0.006944in}{-0.006944in}}%
\pgfpathcurveto{\pgfqpoint{0.008786in}{-0.005103in}}{\pgfqpoint{0.009821in}{-0.002605in}}{\pgfqpoint{0.009821in}{0.000000in}}%
\pgfpathcurveto{\pgfqpoint{0.009821in}{0.002605in}}{\pgfqpoint{0.008786in}{0.005103in}}{\pgfqpoint{0.006944in}{0.006944in}}%
\pgfpathcurveto{\pgfqpoint{0.005103in}{0.008786in}}{\pgfqpoint{0.002605in}{0.009821in}}{\pgfqpoint{0.000000in}{0.009821in}}%
\pgfpathcurveto{\pgfqpoint{-0.002605in}{0.009821in}}{\pgfqpoint{-0.005103in}{0.008786in}}{\pgfqpoint{-0.006944in}{0.006944in}}%
\pgfpathcurveto{\pgfqpoint{-0.008786in}{0.005103in}}{\pgfqpoint{-0.009821in}{0.002605in}}{\pgfqpoint{-0.009821in}{0.000000in}}%
\pgfpathcurveto{\pgfqpoint{-0.009821in}{-0.002605in}}{\pgfqpoint{-0.008786in}{-0.005103in}}{\pgfqpoint{-0.006944in}{-0.006944in}}%
\pgfpathcurveto{\pgfqpoint{-0.005103in}{-0.008786in}}{\pgfqpoint{-0.002605in}{-0.009821in}}{\pgfqpoint{0.000000in}{-0.009821in}}%
\pgfpathclose%
\pgfusepath{stroke,fill}%
}%
\begin{pgfscope}%
\pgfsys@transformshift{4.556344in}{5.522897in}%
\pgfsys@useobject{currentmarker}{}%
\end{pgfscope}%
\begin{pgfscope}%
\pgfsys@transformshift{4.633707in}{5.522903in}%
\pgfsys@useobject{currentmarker}{}%
\end{pgfscope}%
\begin{pgfscope}%
\pgfsys@transformshift{4.711071in}{5.522941in}%
\pgfsys@useobject{currentmarker}{}%
\end{pgfscope}%
\begin{pgfscope}%
\pgfsys@transformshift{4.788434in}{5.523060in}%
\pgfsys@useobject{currentmarker}{}%
\end{pgfscope}%
\begin{pgfscope}%
\pgfsys@transformshift{4.865797in}{5.523240in}%
\pgfsys@useobject{currentmarker}{}%
\end{pgfscope}%
\begin{pgfscope}%
\pgfsys@transformshift{4.943161in}{5.523177in}%
\pgfsys@useobject{currentmarker}{}%
\end{pgfscope}%
\begin{pgfscope}%
\pgfsys@transformshift{5.020524in}{5.522235in}%
\pgfsys@useobject{currentmarker}{}%
\end{pgfscope}%
\begin{pgfscope}%
\pgfsys@transformshift{5.097887in}{5.519739in}%
\pgfsys@useobject{currentmarker}{}%
\end{pgfscope}%
\begin{pgfscope}%
\pgfsys@transformshift{5.175251in}{5.515145in}%
\pgfsys@useobject{currentmarker}{}%
\end{pgfscope}%
\begin{pgfscope}%
\pgfsys@transformshift{5.252614in}{5.507902in}%
\pgfsys@useobject{currentmarker}{}%
\end{pgfscope}%
\begin{pgfscope}%
\pgfsys@transformshift{5.329978in}{5.497248in}%
\pgfsys@useobject{currentmarker}{}%
\end{pgfscope}%
\begin{pgfscope}%
\pgfsys@transformshift{5.407341in}{5.482177in}%
\pgfsys@useobject{currentmarker}{}%
\end{pgfscope}%
\begin{pgfscope}%
\pgfsys@transformshift{5.484704in}{5.461191in}%
\pgfsys@useobject{currentmarker}{}%
\end{pgfscope}%
\begin{pgfscope}%
\pgfsys@transformshift{5.562068in}{5.432094in}%
\pgfsys@useobject{currentmarker}{}%
\end{pgfscope}%
\begin{pgfscope}%
\pgfsys@transformshift{5.639431in}{5.391211in}%
\pgfsys@useobject{currentmarker}{}%
\end{pgfscope}%
\begin{pgfscope}%
\pgfsys@transformshift{5.716794in}{5.331776in}%
\pgfsys@useobject{currentmarker}{}%
\end{pgfscope}%
\begin{pgfscope}%
\pgfsys@transformshift{5.794158in}{5.238318in}%
\pgfsys@useobject{currentmarker}{}%
\end{pgfscope}%
\begin{pgfscope}%
\pgfsys@transformshift{5.871521in}{5.055347in}%
\pgfsys@useobject{currentmarker}{}%
\end{pgfscope}%
\begin{pgfscope}%
\pgfsys@transformshift{5.948884in}{4.575337in}%
\pgfsys@useobject{currentmarker}{}%
\end{pgfscope}%
\begin{pgfscope}%
\pgfsys@transformshift{6.026248in}{4.019155in}%
\pgfsys@useobject{currentmarker}{}%
\end{pgfscope}%
\begin{pgfscope}%
\pgfsys@transformshift{6.103611in}{3.767325in}%
\pgfsys@useobject{currentmarker}{}%
\end{pgfscope}%
\begin{pgfscope}%
\pgfsys@transformshift{6.180974in}{3.655808in}%
\pgfsys@useobject{currentmarker}{}%
\end{pgfscope}%
\begin{pgfscope}%
\pgfsys@transformshift{6.258338in}{3.596608in}%
\pgfsys@useobject{currentmarker}{}%
\end{pgfscope}%
\begin{pgfscope}%
\pgfsys@transformshift{6.335701in}{3.559096in}%
\pgfsys@useobject{currentmarker}{}%
\end{pgfscope}%
\begin{pgfscope}%
\pgfsys@transformshift{6.413065in}{3.531543in}%
\pgfsys@useobject{currentmarker}{}%
\end{pgfscope}%
\begin{pgfscope}%
\pgfsys@transformshift{6.490428in}{3.511303in}%
\pgfsys@useobject{currentmarker}{}%
\end{pgfscope}%
\begin{pgfscope}%
\pgfsys@transformshift{6.567791in}{3.496687in}%
\pgfsys@useobject{currentmarker}{}%
\end{pgfscope}%
\begin{pgfscope}%
\pgfsys@transformshift{6.645155in}{3.485180in}%
\pgfsys@useobject{currentmarker}{}%
\end{pgfscope}%
\begin{pgfscope}%
\pgfsys@transformshift{6.722518in}{3.475127in}%
\pgfsys@useobject{currentmarker}{}%
\end{pgfscope}%
\begin{pgfscope}%
\pgfsys@transformshift{6.799881in}{3.467889in}%
\pgfsys@useobject{currentmarker}{}%
\end{pgfscope}%
\begin{pgfscope}%
\pgfsys@transformshift{6.877245in}{3.460681in}%
\pgfsys@useobject{currentmarker}{}%
\end{pgfscope}%
\begin{pgfscope}%
\pgfsys@transformshift{6.954608in}{3.454839in}%
\pgfsys@useobject{currentmarker}{}%
\end{pgfscope}%
\begin{pgfscope}%
\pgfsys@transformshift{7.031971in}{3.449745in}%
\pgfsys@useobject{currentmarker}{}%
\end{pgfscope}%
\begin{pgfscope}%
\pgfsys@transformshift{7.109335in}{3.445478in}%
\pgfsys@useobject{currentmarker}{}%
\end{pgfscope}%
\begin{pgfscope}%
\pgfsys@transformshift{7.186698in}{3.441451in}%
\pgfsys@useobject{currentmarker}{}%
\end{pgfscope}%
\begin{pgfscope}%
\pgfsys@transformshift{7.264061in}{3.438010in}%
\pgfsys@useobject{currentmarker}{}%
\end{pgfscope}%
\begin{pgfscope}%
\pgfsys@transformshift{7.341425in}{3.435525in}%
\pgfsys@useobject{currentmarker}{}%
\end{pgfscope}%
\begin{pgfscope}%
\pgfsys@transformshift{7.418788in}{3.432415in}%
\pgfsys@useobject{currentmarker}{}%
\end{pgfscope}%
\begin{pgfscope}%
\pgfsys@transformshift{7.496152in}{3.429998in}%
\pgfsys@useobject{currentmarker}{}%
\end{pgfscope}%
\begin{pgfscope}%
\pgfsys@transformshift{7.573515in}{3.427970in}%
\pgfsys@useobject{currentmarker}{}%
\end{pgfscope}%
\begin{pgfscope}%
\pgfsys@transformshift{7.650878in}{3.425649in}%
\pgfsys@useobject{currentmarker}{}%
\end{pgfscope}%
\end{pgfscope}%
\begin{pgfscope}%
\pgfsetbuttcap%
\pgfsetroundjoin%
\definecolor{currentfill}{rgb}{0.000000,0.000000,0.000000}%
\pgfsetfillcolor{currentfill}%
\pgfsetlinewidth{0.803000pt}%
\definecolor{currentstroke}{rgb}{0.000000,0.000000,0.000000}%
\pgfsetstrokecolor{currentstroke}%
\pgfsetdash{}{0pt}%
\pgfsys@defobject{currentmarker}{\pgfqpoint{0.000000in}{-0.048611in}}{\pgfqpoint{0.000000in}{0.000000in}}{%
\pgfpathmoveto{\pgfqpoint{0.000000in}{0.000000in}}%
\pgfpathlineto{\pgfqpoint{0.000000in}{-0.048611in}}%
\pgfusepath{stroke,fill}%
}%
\begin{pgfscope}%
\pgfsys@transformshift{4.943161in}{3.312222in}%
\pgfsys@useobject{currentmarker}{}%
\end{pgfscope}%
\end{pgfscope}%
\begin{pgfscope}%
\definecolor{textcolor}{rgb}{0.000000,0.000000,0.000000}%
\pgfsetstrokecolor{textcolor}%
\pgfsetfillcolor{textcolor}%
\pgftext[x=4.943161in,y=3.215000in,,top]{\color{textcolor}\sffamily\fontsize{10.000000}{12.000000}\selectfont 1}%
\end{pgfscope}%
\begin{pgfscope}%
\pgfsetbuttcap%
\pgfsetroundjoin%
\definecolor{currentfill}{rgb}{0.000000,0.000000,0.000000}%
\pgfsetfillcolor{currentfill}%
\pgfsetlinewidth{0.803000pt}%
\definecolor{currentstroke}{rgb}{0.000000,0.000000,0.000000}%
\pgfsetstrokecolor{currentstroke}%
\pgfsetdash{}{0pt}%
\pgfsys@defobject{currentmarker}{\pgfqpoint{0.000000in}{-0.048611in}}{\pgfqpoint{0.000000in}{0.000000in}}{%
\pgfpathmoveto{\pgfqpoint{0.000000in}{0.000000in}}%
\pgfpathlineto{\pgfqpoint{0.000000in}{-0.048611in}}%
\pgfusepath{stroke,fill}%
}%
\begin{pgfscope}%
\pgfsys@transformshift{5.716794in}{3.312222in}%
\pgfsys@useobject{currentmarker}{}%
\end{pgfscope}%
\end{pgfscope}%
\begin{pgfscope}%
\definecolor{textcolor}{rgb}{0.000000,0.000000,0.000000}%
\pgfsetstrokecolor{textcolor}%
\pgfsetfillcolor{textcolor}%
\pgftext[x=5.716794in,y=3.215000in,,top]{\color{textcolor}\sffamily\fontsize{10.000000}{12.000000}\selectfont 2}%
\end{pgfscope}%
\begin{pgfscope}%
\pgfsetbuttcap%
\pgfsetroundjoin%
\definecolor{currentfill}{rgb}{0.000000,0.000000,0.000000}%
\pgfsetfillcolor{currentfill}%
\pgfsetlinewidth{0.803000pt}%
\definecolor{currentstroke}{rgb}{0.000000,0.000000,0.000000}%
\pgfsetstrokecolor{currentstroke}%
\pgfsetdash{}{0pt}%
\pgfsys@defobject{currentmarker}{\pgfqpoint{0.000000in}{-0.048611in}}{\pgfqpoint{0.000000in}{0.000000in}}{%
\pgfpathmoveto{\pgfqpoint{0.000000in}{0.000000in}}%
\pgfpathlineto{\pgfqpoint{0.000000in}{-0.048611in}}%
\pgfusepath{stroke,fill}%
}%
\begin{pgfscope}%
\pgfsys@transformshift{6.490428in}{3.312222in}%
\pgfsys@useobject{currentmarker}{}%
\end{pgfscope}%
\end{pgfscope}%
\begin{pgfscope}%
\definecolor{textcolor}{rgb}{0.000000,0.000000,0.000000}%
\pgfsetstrokecolor{textcolor}%
\pgfsetfillcolor{textcolor}%
\pgftext[x=6.490428in,y=3.215000in,,top]{\color{textcolor}\sffamily\fontsize{10.000000}{12.000000}\selectfont 3}%
\end{pgfscope}%
\begin{pgfscope}%
\pgfsetbuttcap%
\pgfsetroundjoin%
\definecolor{currentfill}{rgb}{0.000000,0.000000,0.000000}%
\pgfsetfillcolor{currentfill}%
\pgfsetlinewidth{0.803000pt}%
\definecolor{currentstroke}{rgb}{0.000000,0.000000,0.000000}%
\pgfsetstrokecolor{currentstroke}%
\pgfsetdash{}{0pt}%
\pgfsys@defobject{currentmarker}{\pgfqpoint{0.000000in}{-0.048611in}}{\pgfqpoint{0.000000in}{0.000000in}}{%
\pgfpathmoveto{\pgfqpoint{0.000000in}{0.000000in}}%
\pgfpathlineto{\pgfqpoint{0.000000in}{-0.048611in}}%
\pgfusepath{stroke,fill}%
}%
\begin{pgfscope}%
\pgfsys@transformshift{7.264061in}{3.312222in}%
\pgfsys@useobject{currentmarker}{}%
\end{pgfscope}%
\end{pgfscope}%
\begin{pgfscope}%
\definecolor{textcolor}{rgb}{0.000000,0.000000,0.000000}%
\pgfsetstrokecolor{textcolor}%
\pgfsetfillcolor{textcolor}%
\pgftext[x=7.264061in,y=3.215000in,,top]{\color{textcolor}\sffamily\fontsize{10.000000}{12.000000}\selectfont 4}%
\end{pgfscope}%
\begin{pgfscope}%
\pgfsetbuttcap%
\pgfsetroundjoin%
\definecolor{currentfill}{rgb}{0.000000,0.000000,0.000000}%
\pgfsetfillcolor{currentfill}%
\pgfsetlinewidth{0.803000pt}%
\definecolor{currentstroke}{rgb}{0.000000,0.000000,0.000000}%
\pgfsetstrokecolor{currentstroke}%
\pgfsetdash{}{0pt}%
\pgfsys@defobject{currentmarker}{\pgfqpoint{-0.048611in}{0.000000in}}{\pgfqpoint{0.000000in}{0.000000in}}{%
\pgfpathmoveto{\pgfqpoint{0.000000in}{0.000000in}}%
\pgfpathlineto{\pgfqpoint{-0.048611in}{0.000000in}}%
\pgfusepath{stroke,fill}%
}%
\begin{pgfscope}%
\pgfsys@transformshift{4.392222in}{3.324667in}%
\pgfsys@useobject{currentmarker}{}%
\end{pgfscope}%
\end{pgfscope}%
\begin{pgfscope}%
\definecolor{textcolor}{rgb}{0.000000,0.000000,0.000000}%
\pgfsetstrokecolor{textcolor}%
\pgfsetfillcolor{textcolor}%
\pgftext[x=4.074121in,y=3.271906in,left,base]{\color{textcolor}\sffamily\fontsize{10.000000}{12.000000}\selectfont 0.0}%
\end{pgfscope}%
\begin{pgfscope}%
\pgfsetbuttcap%
\pgfsetroundjoin%
\definecolor{currentfill}{rgb}{0.000000,0.000000,0.000000}%
\pgfsetfillcolor{currentfill}%
\pgfsetlinewidth{0.803000pt}%
\definecolor{currentstroke}{rgb}{0.000000,0.000000,0.000000}%
\pgfsetstrokecolor{currentstroke}%
\pgfsetdash{}{0pt}%
\pgfsys@defobject{currentmarker}{\pgfqpoint{-0.048611in}{0.000000in}}{\pgfqpoint{0.000000in}{0.000000in}}{%
\pgfpathmoveto{\pgfqpoint{0.000000in}{0.000000in}}%
\pgfpathlineto{\pgfqpoint{-0.048611in}{0.000000in}}%
\pgfusepath{stroke,fill}%
}%
\begin{pgfscope}%
\pgfsys@transformshift{4.392222in}{3.765174in}%
\pgfsys@useobject{currentmarker}{}%
\end{pgfscope}%
\end{pgfscope}%
\begin{pgfscope}%
\definecolor{textcolor}{rgb}{0.000000,0.000000,0.000000}%
\pgfsetstrokecolor{textcolor}%
\pgfsetfillcolor{textcolor}%
\pgftext[x=4.074121in,y=3.712412in,left,base]{\color{textcolor}\sffamily\fontsize{10.000000}{12.000000}\selectfont 0.2}%
\end{pgfscope}%
\begin{pgfscope}%
\pgfsetbuttcap%
\pgfsetroundjoin%
\definecolor{currentfill}{rgb}{0.000000,0.000000,0.000000}%
\pgfsetfillcolor{currentfill}%
\pgfsetlinewidth{0.803000pt}%
\definecolor{currentstroke}{rgb}{0.000000,0.000000,0.000000}%
\pgfsetstrokecolor{currentstroke}%
\pgfsetdash{}{0pt}%
\pgfsys@defobject{currentmarker}{\pgfqpoint{-0.048611in}{0.000000in}}{\pgfqpoint{0.000000in}{0.000000in}}{%
\pgfpathmoveto{\pgfqpoint{0.000000in}{0.000000in}}%
\pgfpathlineto{\pgfqpoint{-0.048611in}{0.000000in}}%
\pgfusepath{stroke,fill}%
}%
\begin{pgfscope}%
\pgfsys@transformshift{4.392222in}{4.205680in}%
\pgfsys@useobject{currentmarker}{}%
\end{pgfscope}%
\end{pgfscope}%
\begin{pgfscope}%
\definecolor{textcolor}{rgb}{0.000000,0.000000,0.000000}%
\pgfsetstrokecolor{textcolor}%
\pgfsetfillcolor{textcolor}%
\pgftext[x=4.074121in,y=4.152918in,left,base]{\color{textcolor}\sffamily\fontsize{10.000000}{12.000000}\selectfont 0.4}%
\end{pgfscope}%
\begin{pgfscope}%
\pgfsetbuttcap%
\pgfsetroundjoin%
\definecolor{currentfill}{rgb}{0.000000,0.000000,0.000000}%
\pgfsetfillcolor{currentfill}%
\pgfsetlinewidth{0.803000pt}%
\definecolor{currentstroke}{rgb}{0.000000,0.000000,0.000000}%
\pgfsetstrokecolor{currentstroke}%
\pgfsetdash{}{0pt}%
\pgfsys@defobject{currentmarker}{\pgfqpoint{-0.048611in}{0.000000in}}{\pgfqpoint{0.000000in}{0.000000in}}{%
\pgfpathmoveto{\pgfqpoint{0.000000in}{0.000000in}}%
\pgfpathlineto{\pgfqpoint{-0.048611in}{0.000000in}}%
\pgfusepath{stroke,fill}%
}%
\begin{pgfscope}%
\pgfsys@transformshift{4.392222in}{4.646186in}%
\pgfsys@useobject{currentmarker}{}%
\end{pgfscope}%
\end{pgfscope}%
\begin{pgfscope}%
\definecolor{textcolor}{rgb}{0.000000,0.000000,0.000000}%
\pgfsetstrokecolor{textcolor}%
\pgfsetfillcolor{textcolor}%
\pgftext[x=4.074121in,y=4.593424in,left,base]{\color{textcolor}\sffamily\fontsize{10.000000}{12.000000}\selectfont 0.6}%
\end{pgfscope}%
\begin{pgfscope}%
\pgfsetbuttcap%
\pgfsetroundjoin%
\definecolor{currentfill}{rgb}{0.000000,0.000000,0.000000}%
\pgfsetfillcolor{currentfill}%
\pgfsetlinewidth{0.803000pt}%
\definecolor{currentstroke}{rgb}{0.000000,0.000000,0.000000}%
\pgfsetstrokecolor{currentstroke}%
\pgfsetdash{}{0pt}%
\pgfsys@defobject{currentmarker}{\pgfqpoint{-0.048611in}{0.000000in}}{\pgfqpoint{0.000000in}{0.000000in}}{%
\pgfpathmoveto{\pgfqpoint{0.000000in}{0.000000in}}%
\pgfpathlineto{\pgfqpoint{-0.048611in}{0.000000in}}%
\pgfusepath{stroke,fill}%
}%
\begin{pgfscope}%
\pgfsys@transformshift{4.392222in}{5.086692in}%
\pgfsys@useobject{currentmarker}{}%
\end{pgfscope}%
\end{pgfscope}%
\begin{pgfscope}%
\definecolor{textcolor}{rgb}{0.000000,0.000000,0.000000}%
\pgfsetstrokecolor{textcolor}%
\pgfsetfillcolor{textcolor}%
\pgftext[x=4.074121in,y=5.033930in,left,base]{\color{textcolor}\sffamily\fontsize{10.000000}{12.000000}\selectfont 0.8}%
\end{pgfscope}%
\begin{pgfscope}%
\pgfsetbuttcap%
\pgfsetroundjoin%
\definecolor{currentfill}{rgb}{0.000000,0.000000,0.000000}%
\pgfsetfillcolor{currentfill}%
\pgfsetlinewidth{0.803000pt}%
\definecolor{currentstroke}{rgb}{0.000000,0.000000,0.000000}%
\pgfsetstrokecolor{currentstroke}%
\pgfsetdash{}{0pt}%
\pgfsys@defobject{currentmarker}{\pgfqpoint{-0.048611in}{0.000000in}}{\pgfqpoint{0.000000in}{0.000000in}}{%
\pgfpathmoveto{\pgfqpoint{0.000000in}{0.000000in}}%
\pgfpathlineto{\pgfqpoint{-0.048611in}{0.000000in}}%
\pgfusepath{stroke,fill}%
}%
\begin{pgfscope}%
\pgfsys@transformshift{4.392222in}{5.527198in}%
\pgfsys@useobject{currentmarker}{}%
\end{pgfscope}%
\end{pgfscope}%
\begin{pgfscope}%
\definecolor{textcolor}{rgb}{0.000000,0.000000,0.000000}%
\pgfsetstrokecolor{textcolor}%
\pgfsetfillcolor{textcolor}%
\pgftext[x=4.074121in,y=5.474436in,left,base]{\color{textcolor}\sffamily\fontsize{10.000000}{12.000000}\selectfont 1.0}%
\end{pgfscope}%
\begin{pgfscope}%
\pgfpathrectangle{\pgfqpoint{4.392222in}{3.312222in}}{\pgfqpoint{3.422778in}{2.324444in}}%
\pgfusepath{clip}%
\pgfsetrectcap%
\pgfsetroundjoin%
\pgfsetlinewidth{1.505625pt}%
\definecolor{currentstroke}{rgb}{0.121569,0.466667,0.705882}%
\pgfsetstrokecolor{currentstroke}%
\pgfsetdash{}{0pt}%
\pgfpathmoveto{\pgfqpoint{4.556344in}{5.522897in}}%
\pgfpathlineto{\pgfqpoint{4.633707in}{5.522903in}}%
\pgfpathlineto{\pgfqpoint{4.711071in}{5.522941in}}%
\pgfpathlineto{\pgfqpoint{4.788434in}{5.523060in}}%
\pgfpathlineto{\pgfqpoint{4.865797in}{5.523240in}}%
\pgfpathlineto{\pgfqpoint{4.943161in}{5.523177in}}%
\pgfpathlineto{\pgfqpoint{5.020524in}{5.522235in}}%
\pgfpathlineto{\pgfqpoint{5.097887in}{5.519739in}}%
\pgfpathlineto{\pgfqpoint{5.175251in}{5.515145in}}%
\pgfpathlineto{\pgfqpoint{5.252614in}{5.507902in}}%
\pgfpathlineto{\pgfqpoint{5.329978in}{5.497248in}}%
\pgfpathlineto{\pgfqpoint{5.407341in}{5.482177in}}%
\pgfpathlineto{\pgfqpoint{5.484704in}{5.461191in}}%
\pgfpathlineto{\pgfqpoint{5.562068in}{5.432094in}}%
\pgfpathlineto{\pgfqpoint{5.639431in}{5.391211in}}%
\pgfpathlineto{\pgfqpoint{5.716794in}{5.331776in}}%
\pgfpathlineto{\pgfqpoint{5.794158in}{5.238318in}}%
\pgfpathlineto{\pgfqpoint{5.871521in}{5.055347in}}%
\pgfpathlineto{\pgfqpoint{5.948884in}{4.575337in}}%
\pgfpathlineto{\pgfqpoint{6.026248in}{4.019155in}}%
\pgfpathlineto{\pgfqpoint{6.103611in}{3.767325in}}%
\pgfpathlineto{\pgfqpoint{6.180974in}{3.655808in}}%
\pgfpathlineto{\pgfqpoint{6.258338in}{3.596608in}}%
\pgfpathlineto{\pgfqpoint{6.335701in}{3.559096in}}%
\pgfpathlineto{\pgfqpoint{6.413065in}{3.531543in}}%
\pgfpathlineto{\pgfqpoint{6.490428in}{3.511303in}}%
\pgfpathlineto{\pgfqpoint{6.567791in}{3.496687in}}%
\pgfpathlineto{\pgfqpoint{6.645155in}{3.485180in}}%
\pgfpathlineto{\pgfqpoint{6.722518in}{3.475127in}}%
\pgfpathlineto{\pgfqpoint{6.799881in}{3.467889in}}%
\pgfpathlineto{\pgfqpoint{6.877245in}{3.460681in}}%
\pgfpathlineto{\pgfqpoint{6.954608in}{3.454839in}}%
\pgfpathlineto{\pgfqpoint{7.031971in}{3.449745in}}%
\pgfpathlineto{\pgfqpoint{7.109335in}{3.445478in}}%
\pgfpathlineto{\pgfqpoint{7.186698in}{3.441451in}}%
\pgfpathlineto{\pgfqpoint{7.264061in}{3.438010in}}%
\pgfpathlineto{\pgfqpoint{7.341425in}{3.435525in}}%
\pgfpathlineto{\pgfqpoint{7.418788in}{3.432415in}}%
\pgfpathlineto{\pgfqpoint{7.496152in}{3.429998in}}%
\pgfpathlineto{\pgfqpoint{7.573515in}{3.427970in}}%
\pgfpathlineto{\pgfqpoint{7.650878in}{3.425649in}}%
\pgfusepath{stroke}%
\end{pgfscope}%
\begin{pgfscope}%
\pgfsetrectcap%
\pgfsetmiterjoin%
\pgfsetlinewidth{0.803000pt}%
\definecolor{currentstroke}{rgb}{0.000000,0.000000,0.000000}%
\pgfsetstrokecolor{currentstroke}%
\pgfsetdash{}{0pt}%
\pgfpathmoveto{\pgfqpoint{4.392222in}{3.312222in}}%
\pgfpathlineto{\pgfqpoint{4.392222in}{5.636667in}}%
\pgfusepath{stroke}%
\end{pgfscope}%
\begin{pgfscope}%
\pgfsetrectcap%
\pgfsetmiterjoin%
\pgfsetlinewidth{0.803000pt}%
\definecolor{currentstroke}{rgb}{0.000000,0.000000,0.000000}%
\pgfsetstrokecolor{currentstroke}%
\pgfsetdash{}{0pt}%
\pgfpathmoveto{\pgfqpoint{7.815000in}{3.312222in}}%
\pgfpathlineto{\pgfqpoint{7.815000in}{5.636667in}}%
\pgfusepath{stroke}%
\end{pgfscope}%
\begin{pgfscope}%
\pgfsetrectcap%
\pgfsetmiterjoin%
\pgfsetlinewidth{0.803000pt}%
\definecolor{currentstroke}{rgb}{0.000000,0.000000,0.000000}%
\pgfsetstrokecolor{currentstroke}%
\pgfsetdash{}{0pt}%
\pgfpathmoveto{\pgfqpoint{4.392222in}{3.312222in}}%
\pgfpathlineto{\pgfqpoint{7.815000in}{3.312222in}}%
\pgfusepath{stroke}%
\end{pgfscope}%
\begin{pgfscope}%
\pgfsetrectcap%
\pgfsetmiterjoin%
\pgfsetlinewidth{0.803000pt}%
\definecolor{currentstroke}{rgb}{0.000000,0.000000,0.000000}%
\pgfsetstrokecolor{currentstroke}%
\pgfsetdash{}{0pt}%
\pgfpathmoveto{\pgfqpoint{4.392222in}{5.636667in}}%
\pgfpathlineto{\pgfqpoint{7.815000in}{5.636667in}}%
\pgfusepath{stroke}%
\end{pgfscope}%
\begin{pgfscope}%
\definecolor{textcolor}{rgb}{0.000000,0.000000,0.000000}%
\pgfsetstrokecolor{textcolor}%
\pgfsetfillcolor{textcolor}%
\pgftext[x=6.103611in,y=5.720000in,,base]{\color{textcolor}\sffamily\fontsize{12.000000}{14.400000}\selectfont \(\displaystyle  N = 32 \)}%
\end{pgfscope}%
\begin{pgfscope}%
\pgfsetbuttcap%
\pgfsetmiterjoin%
\definecolor{currentfill}{rgb}{1.000000,1.000000,1.000000}%
\pgfsetfillcolor{currentfill}%
\pgfsetlinewidth{0.000000pt}%
\definecolor{currentstroke}{rgb}{0.000000,0.000000,0.000000}%
\pgfsetstrokecolor{currentstroke}%
\pgfsetstrokeopacity{0.000000}%
\pgfsetdash{}{0pt}%
\pgfpathmoveto{\pgfqpoint{0.467222in}{0.387222in}}%
\pgfpathlineto{\pgfqpoint{3.890000in}{0.387222in}}%
\pgfpathlineto{\pgfqpoint{3.890000in}{2.711667in}}%
\pgfpathlineto{\pgfqpoint{0.467222in}{2.711667in}}%
\pgfpathclose%
\pgfusepath{fill}%
\end{pgfscope}%
\begin{pgfscope}%
\pgfpathrectangle{\pgfqpoint{0.467222in}{0.387222in}}{\pgfqpoint{3.422778in}{2.324444in}}%
\pgfusepath{clip}%
\pgfsetbuttcap%
\pgfsetroundjoin%
\definecolor{currentfill}{rgb}{0.121569,0.466667,0.705882}%
\pgfsetfillcolor{currentfill}%
\pgfsetfillopacity{0.300000}%
\pgfsetlinewidth{1.003750pt}%
\definecolor{currentstroke}{rgb}{0.121569,0.466667,0.705882}%
\pgfsetstrokecolor{currentstroke}%
\pgfsetstrokeopacity{0.300000}%
\pgfsetdash{}{0pt}%
\pgfpathmoveto{\pgfqpoint{0.631344in}{2.598141in}}%
\pgfpathlineto{\pgfqpoint{0.631344in}{2.598141in}}%
\pgfpathlineto{\pgfqpoint{0.708707in}{2.598147in}}%
\pgfpathlineto{\pgfqpoint{0.786071in}{2.598180in}}%
\pgfpathlineto{\pgfqpoint{0.863434in}{2.598239in}}%
\pgfpathlineto{\pgfqpoint{0.940797in}{2.598127in}}%
\pgfpathlineto{\pgfqpoint{1.018161in}{2.597424in}}%
\pgfpathlineto{\pgfqpoint{1.095524in}{2.595752in}}%
\pgfpathlineto{\pgfqpoint{1.172887in}{2.592748in}}%
\pgfpathlineto{\pgfqpoint{1.250251in}{2.587926in}}%
\pgfpathlineto{\pgfqpoint{1.327614in}{2.580664in}}%
\pgfpathlineto{\pgfqpoint{1.404978in}{2.570171in}}%
\pgfpathlineto{\pgfqpoint{1.482341in}{2.555375in}}%
\pgfpathlineto{\pgfqpoint{1.559704in}{2.534876in}}%
\pgfpathlineto{\pgfqpoint{1.637068in}{2.506501in}}%
\pgfpathlineto{\pgfqpoint{1.714431in}{2.466574in}}%
\pgfpathlineto{\pgfqpoint{1.791794in}{2.408552in}}%
\pgfpathlineto{\pgfqpoint{1.869158in}{2.315967in}}%
\pgfpathlineto{\pgfqpoint{1.946521in}{2.132965in}}%
\pgfpathlineto{\pgfqpoint{2.023884in}{1.394770in}}%
\pgfpathlineto{\pgfqpoint{2.101248in}{0.790629in}}%
\pgfpathlineto{\pgfqpoint{2.178611in}{0.657569in}}%
\pgfpathlineto{\pgfqpoint{2.255974in}{0.606599in}}%
\pgfpathlineto{\pgfqpoint{2.333338in}{0.580328in}}%
\pgfpathlineto{\pgfqpoint{2.410701in}{0.563660in}}%
\pgfpathlineto{\pgfqpoint{2.488065in}{0.550339in}}%
\pgfpathlineto{\pgfqpoint{2.565428in}{0.541855in}}%
\pgfpathlineto{\pgfqpoint{2.642791in}{0.534562in}}%
\pgfpathlineto{\pgfqpoint{2.720155in}{0.529137in}}%
\pgfpathlineto{\pgfqpoint{2.797518in}{0.524151in}}%
\pgfpathlineto{\pgfqpoint{2.874881in}{0.520484in}}%
\pgfpathlineto{\pgfqpoint{2.952245in}{0.516629in}}%
\pgfpathlineto{\pgfqpoint{3.029608in}{0.514170in}}%
\pgfpathlineto{\pgfqpoint{3.106971in}{0.511846in}}%
\pgfpathlineto{\pgfqpoint{3.184335in}{0.509760in}}%
\pgfpathlineto{\pgfqpoint{3.261698in}{0.508280in}}%
\pgfpathlineto{\pgfqpoint{3.339061in}{0.506591in}}%
\pgfpathlineto{\pgfqpoint{3.416425in}{0.504629in}}%
\pgfpathlineto{\pgfqpoint{3.493788in}{0.503590in}}%
\pgfpathlineto{\pgfqpoint{3.571152in}{0.502104in}}%
\pgfpathlineto{\pgfqpoint{3.648515in}{0.501043in}}%
\pgfpathlineto{\pgfqpoint{3.725878in}{0.500287in}}%
\pgfpathlineto{\pgfqpoint{3.725878in}{0.501012in}}%
\pgfpathlineto{\pgfqpoint{3.725878in}{0.501012in}}%
\pgfpathlineto{\pgfqpoint{3.648515in}{0.502165in}}%
\pgfpathlineto{\pgfqpoint{3.571152in}{0.502906in}}%
\pgfpathlineto{\pgfqpoint{3.493788in}{0.504171in}}%
\pgfpathlineto{\pgfqpoint{3.416425in}{0.506135in}}%
\pgfpathlineto{\pgfqpoint{3.339061in}{0.507222in}}%
\pgfpathlineto{\pgfqpoint{3.261698in}{0.509024in}}%
\pgfpathlineto{\pgfqpoint{3.184335in}{0.510966in}}%
\pgfpathlineto{\pgfqpoint{3.106971in}{0.512510in}}%
\pgfpathlineto{\pgfqpoint{3.029608in}{0.515180in}}%
\pgfpathlineto{\pgfqpoint{2.952245in}{0.518672in}}%
\pgfpathlineto{\pgfqpoint{2.874881in}{0.521395in}}%
\pgfpathlineto{\pgfqpoint{2.797518in}{0.525315in}}%
\pgfpathlineto{\pgfqpoint{2.720155in}{0.529698in}}%
\pgfpathlineto{\pgfqpoint{2.642791in}{0.536446in}}%
\pgfpathlineto{\pgfqpoint{2.565428in}{0.543166in}}%
\pgfpathlineto{\pgfqpoint{2.488065in}{0.551611in}}%
\pgfpathlineto{\pgfqpoint{2.410701in}{0.566997in}}%
\pgfpathlineto{\pgfqpoint{2.333338in}{0.585415in}}%
\pgfpathlineto{\pgfqpoint{2.255974in}{0.615713in}}%
\pgfpathlineto{\pgfqpoint{2.178611in}{0.668191in}}%
\pgfpathlineto{\pgfqpoint{2.101248in}{0.794407in}}%
\pgfpathlineto{\pgfqpoint{2.023884in}{1.431610in}}%
\pgfpathlineto{\pgfqpoint{1.946521in}{2.137434in}}%
\pgfpathlineto{\pgfqpoint{1.869158in}{2.318345in}}%
\pgfpathlineto{\pgfqpoint{1.791794in}{2.408858in}}%
\pgfpathlineto{\pgfqpoint{1.714431in}{2.466801in}}%
\pgfpathlineto{\pgfqpoint{1.637068in}{2.506579in}}%
\pgfpathlineto{\pgfqpoint{1.559704in}{2.534962in}}%
\pgfpathlineto{\pgfqpoint{1.482341in}{2.555417in}}%
\pgfpathlineto{\pgfqpoint{1.404978in}{2.570184in}}%
\pgfpathlineto{\pgfqpoint{1.327614in}{2.580679in}}%
\pgfpathlineto{\pgfqpoint{1.250251in}{2.587935in}}%
\pgfpathlineto{\pgfqpoint{1.172887in}{2.592750in}}%
\pgfpathlineto{\pgfqpoint{1.095524in}{2.595754in}}%
\pgfpathlineto{\pgfqpoint{1.018161in}{2.597424in}}%
\pgfpathlineto{\pgfqpoint{0.940797in}{2.598128in}}%
\pgfpathlineto{\pgfqpoint{0.863434in}{2.598240in}}%
\pgfpathlineto{\pgfqpoint{0.786071in}{2.598180in}}%
\pgfpathlineto{\pgfqpoint{0.708707in}{2.598147in}}%
\pgfpathlineto{\pgfqpoint{0.631344in}{2.598141in}}%
\pgfpathclose%
\pgfusepath{stroke,fill}%
\end{pgfscope}%
\begin{pgfscope}%
\pgfpathrectangle{\pgfqpoint{0.467222in}{0.387222in}}{\pgfqpoint{3.422778in}{2.324444in}}%
\pgfusepath{clip}%
\pgfsetbuttcap%
\pgfsetroundjoin%
\definecolor{currentfill}{rgb}{0.121569,0.466667,0.705882}%
\pgfsetfillcolor{currentfill}%
\pgfsetlinewidth{1.003750pt}%
\definecolor{currentstroke}{rgb}{0.121569,0.466667,0.705882}%
\pgfsetstrokecolor{currentstroke}%
\pgfsetdash{}{0pt}%
\pgfsys@defobject{currentmarker}{\pgfqpoint{-0.009821in}{-0.009821in}}{\pgfqpoint{0.009821in}{0.009821in}}{%
\pgfpathmoveto{\pgfqpoint{0.000000in}{-0.009821in}}%
\pgfpathcurveto{\pgfqpoint{0.002605in}{-0.009821in}}{\pgfqpoint{0.005103in}{-0.008786in}}{\pgfqpoint{0.006944in}{-0.006944in}}%
\pgfpathcurveto{\pgfqpoint{0.008786in}{-0.005103in}}{\pgfqpoint{0.009821in}{-0.002605in}}{\pgfqpoint{0.009821in}{0.000000in}}%
\pgfpathcurveto{\pgfqpoint{0.009821in}{0.002605in}}{\pgfqpoint{0.008786in}{0.005103in}}{\pgfqpoint{0.006944in}{0.006944in}}%
\pgfpathcurveto{\pgfqpoint{0.005103in}{0.008786in}}{\pgfqpoint{0.002605in}{0.009821in}}{\pgfqpoint{0.000000in}{0.009821in}}%
\pgfpathcurveto{\pgfqpoint{-0.002605in}{0.009821in}}{\pgfqpoint{-0.005103in}{0.008786in}}{\pgfqpoint{-0.006944in}{0.006944in}}%
\pgfpathcurveto{\pgfqpoint{-0.008786in}{0.005103in}}{\pgfqpoint{-0.009821in}{0.002605in}}{\pgfqpoint{-0.009821in}{0.000000in}}%
\pgfpathcurveto{\pgfqpoint{-0.009821in}{-0.002605in}}{\pgfqpoint{-0.008786in}{-0.005103in}}{\pgfqpoint{-0.006944in}{-0.006944in}}%
\pgfpathcurveto{\pgfqpoint{-0.005103in}{-0.008786in}}{\pgfqpoint{-0.002605in}{-0.009821in}}{\pgfqpoint{0.000000in}{-0.009821in}}%
\pgfpathclose%
\pgfusepath{stroke,fill}%
}%
\begin{pgfscope}%
\pgfsys@transformshift{0.631344in}{2.598141in}%
\pgfsys@useobject{currentmarker}{}%
\end{pgfscope}%
\begin{pgfscope}%
\pgfsys@transformshift{0.708707in}{2.598147in}%
\pgfsys@useobject{currentmarker}{}%
\end{pgfscope}%
\begin{pgfscope}%
\pgfsys@transformshift{0.786071in}{2.598180in}%
\pgfsys@useobject{currentmarker}{}%
\end{pgfscope}%
\begin{pgfscope}%
\pgfsys@transformshift{0.863434in}{2.598240in}%
\pgfsys@useobject{currentmarker}{}%
\end{pgfscope}%
\begin{pgfscope}%
\pgfsys@transformshift{0.940797in}{2.598128in}%
\pgfsys@useobject{currentmarker}{}%
\end{pgfscope}%
\begin{pgfscope}%
\pgfsys@transformshift{1.018161in}{2.597424in}%
\pgfsys@useobject{currentmarker}{}%
\end{pgfscope}%
\begin{pgfscope}%
\pgfsys@transformshift{1.095524in}{2.595753in}%
\pgfsys@useobject{currentmarker}{}%
\end{pgfscope}%
\begin{pgfscope}%
\pgfsys@transformshift{1.172887in}{2.592749in}%
\pgfsys@useobject{currentmarker}{}%
\end{pgfscope}%
\begin{pgfscope}%
\pgfsys@transformshift{1.250251in}{2.587931in}%
\pgfsys@useobject{currentmarker}{}%
\end{pgfscope}%
\begin{pgfscope}%
\pgfsys@transformshift{1.327614in}{2.580671in}%
\pgfsys@useobject{currentmarker}{}%
\end{pgfscope}%
\begin{pgfscope}%
\pgfsys@transformshift{1.404978in}{2.570178in}%
\pgfsys@useobject{currentmarker}{}%
\end{pgfscope}%
\begin{pgfscope}%
\pgfsys@transformshift{1.482341in}{2.555396in}%
\pgfsys@useobject{currentmarker}{}%
\end{pgfscope}%
\begin{pgfscope}%
\pgfsys@transformshift{1.559704in}{2.534919in}%
\pgfsys@useobject{currentmarker}{}%
\end{pgfscope}%
\begin{pgfscope}%
\pgfsys@transformshift{1.637068in}{2.506540in}%
\pgfsys@useobject{currentmarker}{}%
\end{pgfscope}%
\begin{pgfscope}%
\pgfsys@transformshift{1.714431in}{2.466687in}%
\pgfsys@useobject{currentmarker}{}%
\end{pgfscope}%
\begin{pgfscope}%
\pgfsys@transformshift{1.791794in}{2.408705in}%
\pgfsys@useobject{currentmarker}{}%
\end{pgfscope}%
\begin{pgfscope}%
\pgfsys@transformshift{1.869158in}{2.317156in}%
\pgfsys@useobject{currentmarker}{}%
\end{pgfscope}%
\begin{pgfscope}%
\pgfsys@transformshift{1.946521in}{2.135200in}%
\pgfsys@useobject{currentmarker}{}%
\end{pgfscope}%
\begin{pgfscope}%
\pgfsys@transformshift{2.023884in}{1.413190in}%
\pgfsys@useobject{currentmarker}{}%
\end{pgfscope}%
\begin{pgfscope}%
\pgfsys@transformshift{2.101248in}{0.792518in}%
\pgfsys@useobject{currentmarker}{}%
\end{pgfscope}%
\begin{pgfscope}%
\pgfsys@transformshift{2.178611in}{0.662880in}%
\pgfsys@useobject{currentmarker}{}%
\end{pgfscope}%
\begin{pgfscope}%
\pgfsys@transformshift{2.255974in}{0.611156in}%
\pgfsys@useobject{currentmarker}{}%
\end{pgfscope}%
\begin{pgfscope}%
\pgfsys@transformshift{2.333338in}{0.582872in}%
\pgfsys@useobject{currentmarker}{}%
\end{pgfscope}%
\begin{pgfscope}%
\pgfsys@transformshift{2.410701in}{0.565328in}%
\pgfsys@useobject{currentmarker}{}%
\end{pgfscope}%
\begin{pgfscope}%
\pgfsys@transformshift{2.488065in}{0.550975in}%
\pgfsys@useobject{currentmarker}{}%
\end{pgfscope}%
\begin{pgfscope}%
\pgfsys@transformshift{2.565428in}{0.542511in}%
\pgfsys@useobject{currentmarker}{}%
\end{pgfscope}%
\begin{pgfscope}%
\pgfsys@transformshift{2.642791in}{0.535504in}%
\pgfsys@useobject{currentmarker}{}%
\end{pgfscope}%
\begin{pgfscope}%
\pgfsys@transformshift{2.720155in}{0.529417in}%
\pgfsys@useobject{currentmarker}{}%
\end{pgfscope}%
\begin{pgfscope}%
\pgfsys@transformshift{2.797518in}{0.524733in}%
\pgfsys@useobject{currentmarker}{}%
\end{pgfscope}%
\begin{pgfscope}%
\pgfsys@transformshift{2.874881in}{0.520940in}%
\pgfsys@useobject{currentmarker}{}%
\end{pgfscope}%
\begin{pgfscope}%
\pgfsys@transformshift{2.952245in}{0.517651in}%
\pgfsys@useobject{currentmarker}{}%
\end{pgfscope}%
\begin{pgfscope}%
\pgfsys@transformshift{3.029608in}{0.514675in}%
\pgfsys@useobject{currentmarker}{}%
\end{pgfscope}%
\begin{pgfscope}%
\pgfsys@transformshift{3.106971in}{0.512178in}%
\pgfsys@useobject{currentmarker}{}%
\end{pgfscope}%
\begin{pgfscope}%
\pgfsys@transformshift{3.184335in}{0.510363in}%
\pgfsys@useobject{currentmarker}{}%
\end{pgfscope}%
\begin{pgfscope}%
\pgfsys@transformshift{3.261698in}{0.508652in}%
\pgfsys@useobject{currentmarker}{}%
\end{pgfscope}%
\begin{pgfscope}%
\pgfsys@transformshift{3.339061in}{0.506906in}%
\pgfsys@useobject{currentmarker}{}%
\end{pgfscope}%
\begin{pgfscope}%
\pgfsys@transformshift{3.416425in}{0.505382in}%
\pgfsys@useobject{currentmarker}{}%
\end{pgfscope}%
\begin{pgfscope}%
\pgfsys@transformshift{3.493788in}{0.503881in}%
\pgfsys@useobject{currentmarker}{}%
\end{pgfscope}%
\begin{pgfscope}%
\pgfsys@transformshift{3.571152in}{0.502505in}%
\pgfsys@useobject{currentmarker}{}%
\end{pgfscope}%
\begin{pgfscope}%
\pgfsys@transformshift{3.648515in}{0.501604in}%
\pgfsys@useobject{currentmarker}{}%
\end{pgfscope}%
\begin{pgfscope}%
\pgfsys@transformshift{3.725878in}{0.500649in}%
\pgfsys@useobject{currentmarker}{}%
\end{pgfscope}%
\end{pgfscope}%
\begin{pgfscope}%
\pgfsetbuttcap%
\pgfsetroundjoin%
\definecolor{currentfill}{rgb}{0.000000,0.000000,0.000000}%
\pgfsetfillcolor{currentfill}%
\pgfsetlinewidth{0.803000pt}%
\definecolor{currentstroke}{rgb}{0.000000,0.000000,0.000000}%
\pgfsetstrokecolor{currentstroke}%
\pgfsetdash{}{0pt}%
\pgfsys@defobject{currentmarker}{\pgfqpoint{0.000000in}{-0.048611in}}{\pgfqpoint{0.000000in}{0.000000in}}{%
\pgfpathmoveto{\pgfqpoint{0.000000in}{0.000000in}}%
\pgfpathlineto{\pgfqpoint{0.000000in}{-0.048611in}}%
\pgfusepath{stroke,fill}%
}%
\begin{pgfscope}%
\pgfsys@transformshift{1.018161in}{0.387222in}%
\pgfsys@useobject{currentmarker}{}%
\end{pgfscope}%
\end{pgfscope}%
\begin{pgfscope}%
\definecolor{textcolor}{rgb}{0.000000,0.000000,0.000000}%
\pgfsetstrokecolor{textcolor}%
\pgfsetfillcolor{textcolor}%
\pgftext[x=1.018161in,y=0.290000in,,top]{\color{textcolor}\sffamily\fontsize{10.000000}{12.000000}\selectfont 1}%
\end{pgfscope}%
\begin{pgfscope}%
\pgfsetbuttcap%
\pgfsetroundjoin%
\definecolor{currentfill}{rgb}{0.000000,0.000000,0.000000}%
\pgfsetfillcolor{currentfill}%
\pgfsetlinewidth{0.803000pt}%
\definecolor{currentstroke}{rgb}{0.000000,0.000000,0.000000}%
\pgfsetstrokecolor{currentstroke}%
\pgfsetdash{}{0pt}%
\pgfsys@defobject{currentmarker}{\pgfqpoint{0.000000in}{-0.048611in}}{\pgfqpoint{0.000000in}{0.000000in}}{%
\pgfpathmoveto{\pgfqpoint{0.000000in}{0.000000in}}%
\pgfpathlineto{\pgfqpoint{0.000000in}{-0.048611in}}%
\pgfusepath{stroke,fill}%
}%
\begin{pgfscope}%
\pgfsys@transformshift{1.791794in}{0.387222in}%
\pgfsys@useobject{currentmarker}{}%
\end{pgfscope}%
\end{pgfscope}%
\begin{pgfscope}%
\definecolor{textcolor}{rgb}{0.000000,0.000000,0.000000}%
\pgfsetstrokecolor{textcolor}%
\pgfsetfillcolor{textcolor}%
\pgftext[x=1.791794in,y=0.290000in,,top]{\color{textcolor}\sffamily\fontsize{10.000000}{12.000000}\selectfont 2}%
\end{pgfscope}%
\begin{pgfscope}%
\pgfsetbuttcap%
\pgfsetroundjoin%
\definecolor{currentfill}{rgb}{0.000000,0.000000,0.000000}%
\pgfsetfillcolor{currentfill}%
\pgfsetlinewidth{0.803000pt}%
\definecolor{currentstroke}{rgb}{0.000000,0.000000,0.000000}%
\pgfsetstrokecolor{currentstroke}%
\pgfsetdash{}{0pt}%
\pgfsys@defobject{currentmarker}{\pgfqpoint{0.000000in}{-0.048611in}}{\pgfqpoint{0.000000in}{0.000000in}}{%
\pgfpathmoveto{\pgfqpoint{0.000000in}{0.000000in}}%
\pgfpathlineto{\pgfqpoint{0.000000in}{-0.048611in}}%
\pgfusepath{stroke,fill}%
}%
\begin{pgfscope}%
\pgfsys@transformshift{2.565428in}{0.387222in}%
\pgfsys@useobject{currentmarker}{}%
\end{pgfscope}%
\end{pgfscope}%
\begin{pgfscope}%
\definecolor{textcolor}{rgb}{0.000000,0.000000,0.000000}%
\pgfsetstrokecolor{textcolor}%
\pgfsetfillcolor{textcolor}%
\pgftext[x=2.565428in,y=0.290000in,,top]{\color{textcolor}\sffamily\fontsize{10.000000}{12.000000}\selectfont 3}%
\end{pgfscope}%
\begin{pgfscope}%
\pgfsetbuttcap%
\pgfsetroundjoin%
\definecolor{currentfill}{rgb}{0.000000,0.000000,0.000000}%
\pgfsetfillcolor{currentfill}%
\pgfsetlinewidth{0.803000pt}%
\definecolor{currentstroke}{rgb}{0.000000,0.000000,0.000000}%
\pgfsetstrokecolor{currentstroke}%
\pgfsetdash{}{0pt}%
\pgfsys@defobject{currentmarker}{\pgfqpoint{0.000000in}{-0.048611in}}{\pgfqpoint{0.000000in}{0.000000in}}{%
\pgfpathmoveto{\pgfqpoint{0.000000in}{0.000000in}}%
\pgfpathlineto{\pgfqpoint{0.000000in}{-0.048611in}}%
\pgfusepath{stroke,fill}%
}%
\begin{pgfscope}%
\pgfsys@transformshift{3.339061in}{0.387222in}%
\pgfsys@useobject{currentmarker}{}%
\end{pgfscope}%
\end{pgfscope}%
\begin{pgfscope}%
\definecolor{textcolor}{rgb}{0.000000,0.000000,0.000000}%
\pgfsetstrokecolor{textcolor}%
\pgfsetfillcolor{textcolor}%
\pgftext[x=3.339061in,y=0.290000in,,top]{\color{textcolor}\sffamily\fontsize{10.000000}{12.000000}\selectfont 4}%
\end{pgfscope}%
\begin{pgfscope}%
\pgfsetbuttcap%
\pgfsetroundjoin%
\definecolor{currentfill}{rgb}{0.000000,0.000000,0.000000}%
\pgfsetfillcolor{currentfill}%
\pgfsetlinewidth{0.803000pt}%
\definecolor{currentstroke}{rgb}{0.000000,0.000000,0.000000}%
\pgfsetstrokecolor{currentstroke}%
\pgfsetdash{}{0pt}%
\pgfsys@defobject{currentmarker}{\pgfqpoint{-0.048611in}{0.000000in}}{\pgfqpoint{0.000000in}{0.000000in}}{%
\pgfpathmoveto{\pgfqpoint{0.000000in}{0.000000in}}%
\pgfpathlineto{\pgfqpoint{-0.048611in}{0.000000in}}%
\pgfusepath{stroke,fill}%
}%
\begin{pgfscope}%
\pgfsys@transformshift{0.467222in}{0.451413in}%
\pgfsys@useobject{currentmarker}{}%
\end{pgfscope}%
\end{pgfscope}%
\begin{pgfscope}%
\definecolor{textcolor}{rgb}{0.000000,0.000000,0.000000}%
\pgfsetstrokecolor{textcolor}%
\pgfsetfillcolor{textcolor}%
\pgftext[x=0.149121in,y=0.398652in,left,base]{\color{textcolor}\sffamily\fontsize{10.000000}{12.000000}\selectfont 0.0}%
\end{pgfscope}%
\begin{pgfscope}%
\pgfsetbuttcap%
\pgfsetroundjoin%
\definecolor{currentfill}{rgb}{0.000000,0.000000,0.000000}%
\pgfsetfillcolor{currentfill}%
\pgfsetlinewidth{0.803000pt}%
\definecolor{currentstroke}{rgb}{0.000000,0.000000,0.000000}%
\pgfsetstrokecolor{currentstroke}%
\pgfsetdash{}{0pt}%
\pgfsys@defobject{currentmarker}{\pgfqpoint{-0.048611in}{0.000000in}}{\pgfqpoint{0.000000in}{0.000000in}}{%
\pgfpathmoveto{\pgfqpoint{0.000000in}{0.000000in}}%
\pgfpathlineto{\pgfqpoint{-0.048611in}{0.000000in}}%
\pgfusepath{stroke,fill}%
}%
\begin{pgfscope}%
\pgfsys@transformshift{0.467222in}{0.880969in}%
\pgfsys@useobject{currentmarker}{}%
\end{pgfscope}%
\end{pgfscope}%
\begin{pgfscope}%
\definecolor{textcolor}{rgb}{0.000000,0.000000,0.000000}%
\pgfsetstrokecolor{textcolor}%
\pgfsetfillcolor{textcolor}%
\pgftext[x=0.149121in,y=0.828207in,left,base]{\color{textcolor}\sffamily\fontsize{10.000000}{12.000000}\selectfont 0.2}%
\end{pgfscope}%
\begin{pgfscope}%
\pgfsetbuttcap%
\pgfsetroundjoin%
\definecolor{currentfill}{rgb}{0.000000,0.000000,0.000000}%
\pgfsetfillcolor{currentfill}%
\pgfsetlinewidth{0.803000pt}%
\definecolor{currentstroke}{rgb}{0.000000,0.000000,0.000000}%
\pgfsetstrokecolor{currentstroke}%
\pgfsetdash{}{0pt}%
\pgfsys@defobject{currentmarker}{\pgfqpoint{-0.048611in}{0.000000in}}{\pgfqpoint{0.000000in}{0.000000in}}{%
\pgfpathmoveto{\pgfqpoint{0.000000in}{0.000000in}}%
\pgfpathlineto{\pgfqpoint{-0.048611in}{0.000000in}}%
\pgfusepath{stroke,fill}%
}%
\begin{pgfscope}%
\pgfsys@transformshift{0.467222in}{1.310524in}%
\pgfsys@useobject{currentmarker}{}%
\end{pgfscope}%
\end{pgfscope}%
\begin{pgfscope}%
\definecolor{textcolor}{rgb}{0.000000,0.000000,0.000000}%
\pgfsetstrokecolor{textcolor}%
\pgfsetfillcolor{textcolor}%
\pgftext[x=0.149121in,y=1.257762in,left,base]{\color{textcolor}\sffamily\fontsize{10.000000}{12.000000}\selectfont 0.4}%
\end{pgfscope}%
\begin{pgfscope}%
\pgfsetbuttcap%
\pgfsetroundjoin%
\definecolor{currentfill}{rgb}{0.000000,0.000000,0.000000}%
\pgfsetfillcolor{currentfill}%
\pgfsetlinewidth{0.803000pt}%
\definecolor{currentstroke}{rgb}{0.000000,0.000000,0.000000}%
\pgfsetstrokecolor{currentstroke}%
\pgfsetdash{}{0pt}%
\pgfsys@defobject{currentmarker}{\pgfqpoint{-0.048611in}{0.000000in}}{\pgfqpoint{0.000000in}{0.000000in}}{%
\pgfpathmoveto{\pgfqpoint{0.000000in}{0.000000in}}%
\pgfpathlineto{\pgfqpoint{-0.048611in}{0.000000in}}%
\pgfusepath{stroke,fill}%
}%
\begin{pgfscope}%
\pgfsys@transformshift{0.467222in}{1.740079in}%
\pgfsys@useobject{currentmarker}{}%
\end{pgfscope}%
\end{pgfscope}%
\begin{pgfscope}%
\definecolor{textcolor}{rgb}{0.000000,0.000000,0.000000}%
\pgfsetstrokecolor{textcolor}%
\pgfsetfillcolor{textcolor}%
\pgftext[x=0.149121in,y=1.687317in,left,base]{\color{textcolor}\sffamily\fontsize{10.000000}{12.000000}\selectfont 0.6}%
\end{pgfscope}%
\begin{pgfscope}%
\pgfsetbuttcap%
\pgfsetroundjoin%
\definecolor{currentfill}{rgb}{0.000000,0.000000,0.000000}%
\pgfsetfillcolor{currentfill}%
\pgfsetlinewidth{0.803000pt}%
\definecolor{currentstroke}{rgb}{0.000000,0.000000,0.000000}%
\pgfsetstrokecolor{currentstroke}%
\pgfsetdash{}{0pt}%
\pgfsys@defobject{currentmarker}{\pgfqpoint{-0.048611in}{0.000000in}}{\pgfqpoint{0.000000in}{0.000000in}}{%
\pgfpathmoveto{\pgfqpoint{0.000000in}{0.000000in}}%
\pgfpathlineto{\pgfqpoint{-0.048611in}{0.000000in}}%
\pgfusepath{stroke,fill}%
}%
\begin{pgfscope}%
\pgfsys@transformshift{0.467222in}{2.169634in}%
\pgfsys@useobject{currentmarker}{}%
\end{pgfscope}%
\end{pgfscope}%
\begin{pgfscope}%
\definecolor{textcolor}{rgb}{0.000000,0.000000,0.000000}%
\pgfsetstrokecolor{textcolor}%
\pgfsetfillcolor{textcolor}%
\pgftext[x=0.149121in,y=2.116872in,left,base]{\color{textcolor}\sffamily\fontsize{10.000000}{12.000000}\selectfont 0.8}%
\end{pgfscope}%
\begin{pgfscope}%
\pgfsetbuttcap%
\pgfsetroundjoin%
\definecolor{currentfill}{rgb}{0.000000,0.000000,0.000000}%
\pgfsetfillcolor{currentfill}%
\pgfsetlinewidth{0.803000pt}%
\definecolor{currentstroke}{rgb}{0.000000,0.000000,0.000000}%
\pgfsetstrokecolor{currentstroke}%
\pgfsetdash{}{0pt}%
\pgfsys@defobject{currentmarker}{\pgfqpoint{-0.048611in}{0.000000in}}{\pgfqpoint{0.000000in}{0.000000in}}{%
\pgfpathmoveto{\pgfqpoint{0.000000in}{0.000000in}}%
\pgfpathlineto{\pgfqpoint{-0.048611in}{0.000000in}}%
\pgfusepath{stroke,fill}%
}%
\begin{pgfscope}%
\pgfsys@transformshift{0.467222in}{2.599189in}%
\pgfsys@useobject{currentmarker}{}%
\end{pgfscope}%
\end{pgfscope}%
\begin{pgfscope}%
\definecolor{textcolor}{rgb}{0.000000,0.000000,0.000000}%
\pgfsetstrokecolor{textcolor}%
\pgfsetfillcolor{textcolor}%
\pgftext[x=0.149121in,y=2.546427in,left,base]{\color{textcolor}\sffamily\fontsize{10.000000}{12.000000}\selectfont 1.0}%
\end{pgfscope}%
\begin{pgfscope}%
\pgfpathrectangle{\pgfqpoint{0.467222in}{0.387222in}}{\pgfqpoint{3.422778in}{2.324444in}}%
\pgfusepath{clip}%
\pgfsetrectcap%
\pgfsetroundjoin%
\pgfsetlinewidth{1.505625pt}%
\definecolor{currentstroke}{rgb}{0.121569,0.466667,0.705882}%
\pgfsetstrokecolor{currentstroke}%
\pgfsetdash{}{0pt}%
\pgfpathmoveto{\pgfqpoint{0.631344in}{2.598141in}}%
\pgfpathlineto{\pgfqpoint{0.708707in}{2.598147in}}%
\pgfpathlineto{\pgfqpoint{0.786071in}{2.598180in}}%
\pgfpathlineto{\pgfqpoint{0.863434in}{2.598240in}}%
\pgfpathlineto{\pgfqpoint{0.940797in}{2.598128in}}%
\pgfpathlineto{\pgfqpoint{1.018161in}{2.597424in}}%
\pgfpathlineto{\pgfqpoint{1.095524in}{2.595753in}}%
\pgfpathlineto{\pgfqpoint{1.172887in}{2.592749in}}%
\pgfpathlineto{\pgfqpoint{1.250251in}{2.587931in}}%
\pgfpathlineto{\pgfqpoint{1.327614in}{2.580671in}}%
\pgfpathlineto{\pgfqpoint{1.404978in}{2.570178in}}%
\pgfpathlineto{\pgfqpoint{1.482341in}{2.555396in}}%
\pgfpathlineto{\pgfqpoint{1.559704in}{2.534919in}}%
\pgfpathlineto{\pgfqpoint{1.637068in}{2.506540in}}%
\pgfpathlineto{\pgfqpoint{1.714431in}{2.466687in}}%
\pgfpathlineto{\pgfqpoint{1.791794in}{2.408705in}}%
\pgfpathlineto{\pgfqpoint{1.869158in}{2.317156in}}%
\pgfpathlineto{\pgfqpoint{1.946521in}{2.135200in}}%
\pgfpathlineto{\pgfqpoint{2.023884in}{1.413190in}}%
\pgfpathlineto{\pgfqpoint{2.101248in}{0.792518in}}%
\pgfpathlineto{\pgfqpoint{2.178611in}{0.662880in}}%
\pgfpathlineto{\pgfqpoint{2.255974in}{0.611156in}}%
\pgfpathlineto{\pgfqpoint{2.333338in}{0.582872in}}%
\pgfpathlineto{\pgfqpoint{2.410701in}{0.565328in}}%
\pgfpathlineto{\pgfqpoint{2.488065in}{0.550975in}}%
\pgfpathlineto{\pgfqpoint{2.565428in}{0.542511in}}%
\pgfpathlineto{\pgfqpoint{2.642791in}{0.535504in}}%
\pgfpathlineto{\pgfqpoint{2.720155in}{0.529417in}}%
\pgfpathlineto{\pgfqpoint{2.797518in}{0.524733in}}%
\pgfpathlineto{\pgfqpoint{2.874881in}{0.520940in}}%
\pgfpathlineto{\pgfqpoint{2.952245in}{0.517651in}}%
\pgfpathlineto{\pgfqpoint{3.029608in}{0.514675in}}%
\pgfpathlineto{\pgfqpoint{3.106971in}{0.512178in}}%
\pgfpathlineto{\pgfqpoint{3.184335in}{0.510363in}}%
\pgfpathlineto{\pgfqpoint{3.261698in}{0.508652in}}%
\pgfpathlineto{\pgfqpoint{3.339061in}{0.506906in}}%
\pgfpathlineto{\pgfqpoint{3.416425in}{0.505382in}}%
\pgfpathlineto{\pgfqpoint{3.493788in}{0.503881in}}%
\pgfpathlineto{\pgfqpoint{3.571152in}{0.502505in}}%
\pgfpathlineto{\pgfqpoint{3.648515in}{0.501604in}}%
\pgfpathlineto{\pgfqpoint{3.725878in}{0.500649in}}%
\pgfusepath{stroke}%
\end{pgfscope}%
\begin{pgfscope}%
\pgfsetrectcap%
\pgfsetmiterjoin%
\pgfsetlinewidth{0.803000pt}%
\definecolor{currentstroke}{rgb}{0.000000,0.000000,0.000000}%
\pgfsetstrokecolor{currentstroke}%
\pgfsetdash{}{0pt}%
\pgfpathmoveto{\pgfqpoint{0.467222in}{0.387222in}}%
\pgfpathlineto{\pgfqpoint{0.467222in}{2.711667in}}%
\pgfusepath{stroke}%
\end{pgfscope}%
\begin{pgfscope}%
\pgfsetrectcap%
\pgfsetmiterjoin%
\pgfsetlinewidth{0.803000pt}%
\definecolor{currentstroke}{rgb}{0.000000,0.000000,0.000000}%
\pgfsetstrokecolor{currentstroke}%
\pgfsetdash{}{0pt}%
\pgfpathmoveto{\pgfqpoint{3.890000in}{0.387222in}}%
\pgfpathlineto{\pgfqpoint{3.890000in}{2.711667in}}%
\pgfusepath{stroke}%
\end{pgfscope}%
\begin{pgfscope}%
\pgfsetrectcap%
\pgfsetmiterjoin%
\pgfsetlinewidth{0.803000pt}%
\definecolor{currentstroke}{rgb}{0.000000,0.000000,0.000000}%
\pgfsetstrokecolor{currentstroke}%
\pgfsetdash{}{0pt}%
\pgfpathmoveto{\pgfqpoint{0.467222in}{0.387222in}}%
\pgfpathlineto{\pgfqpoint{3.890000in}{0.387222in}}%
\pgfusepath{stroke}%
\end{pgfscope}%
\begin{pgfscope}%
\pgfsetrectcap%
\pgfsetmiterjoin%
\pgfsetlinewidth{0.803000pt}%
\definecolor{currentstroke}{rgb}{0.000000,0.000000,0.000000}%
\pgfsetstrokecolor{currentstroke}%
\pgfsetdash{}{0pt}%
\pgfpathmoveto{\pgfqpoint{0.467222in}{2.711667in}}%
\pgfpathlineto{\pgfqpoint{3.890000in}{2.711667in}}%
\pgfusepath{stroke}%
\end{pgfscope}%
\begin{pgfscope}%
\definecolor{textcolor}{rgb}{0.000000,0.000000,0.000000}%
\pgfsetstrokecolor{textcolor}%
\pgfsetfillcolor{textcolor}%
\pgftext[x=2.178611in,y=2.795000in,,base]{\color{textcolor}\sffamily\fontsize{12.000000}{14.400000}\selectfont \(\displaystyle  N = 64 \)}%
\end{pgfscope}%
\begin{pgfscope}%
\pgfsetbuttcap%
\pgfsetmiterjoin%
\definecolor{currentfill}{rgb}{1.000000,1.000000,1.000000}%
\pgfsetfillcolor{currentfill}%
\pgfsetlinewidth{0.000000pt}%
\definecolor{currentstroke}{rgb}{0.000000,0.000000,0.000000}%
\pgfsetstrokecolor{currentstroke}%
\pgfsetstrokeopacity{0.000000}%
\pgfsetdash{}{0pt}%
\pgfpathmoveto{\pgfqpoint{4.392222in}{0.387222in}}%
\pgfpathlineto{\pgfqpoint{7.815000in}{0.387222in}}%
\pgfpathlineto{\pgfqpoint{7.815000in}{2.711667in}}%
\pgfpathlineto{\pgfqpoint{4.392222in}{2.711667in}}%
\pgfpathclose%
\pgfusepath{fill}%
\end{pgfscope}%
\begin{pgfscope}%
\pgfpathrectangle{\pgfqpoint{4.392222in}{0.387222in}}{\pgfqpoint{3.422778in}{2.324444in}}%
\pgfusepath{clip}%
\pgfsetbuttcap%
\pgfsetroundjoin%
\definecolor{currentfill}{rgb}{0.121569,0.466667,0.705882}%
\pgfsetfillcolor{currentfill}%
\pgfsetfillopacity{0.300000}%
\pgfsetlinewidth{1.003750pt}%
\definecolor{currentstroke}{rgb}{0.121569,0.466667,0.705882}%
\pgfsetstrokecolor{currentstroke}%
\pgfsetstrokeopacity{0.300000}%
\pgfsetdash{}{0pt}%
\pgfpathmoveto{\pgfqpoint{4.556344in}{2.158485in}}%
\pgfpathlineto{\pgfqpoint{4.556344in}{2.158484in}}%
\pgfpathlineto{\pgfqpoint{4.633707in}{2.158095in}}%
\pgfpathlineto{\pgfqpoint{4.711071in}{2.158504in}}%
\pgfpathlineto{\pgfqpoint{4.788434in}{2.158462in}}%
\pgfpathlineto{\pgfqpoint{4.865797in}{2.158179in}}%
\pgfpathlineto{\pgfqpoint{4.943161in}{2.157464in}}%
\pgfpathlineto{\pgfqpoint{5.020524in}{2.156078in}}%
\pgfpathlineto{\pgfqpoint{5.097887in}{2.132909in}}%
\pgfpathlineto{\pgfqpoint{5.175251in}{2.149917in}}%
\pgfpathlineto{\pgfqpoint{5.252614in}{2.144235in}}%
\pgfpathlineto{\pgfqpoint{5.329978in}{2.136042in}}%
\pgfpathlineto{\pgfqpoint{5.407341in}{1.040447in}}%
\pgfpathlineto{\pgfqpoint{5.484704in}{1.400823in}}%
\pgfpathlineto{\pgfqpoint{5.562068in}{2.086268in}}%
\pgfpathlineto{\pgfqpoint{5.639431in}{1.601242in}}%
\pgfpathlineto{\pgfqpoint{5.716794in}{2.008901in}}%
\pgfpathlineto{\pgfqpoint{5.794158in}{1.910833in}}%
\pgfpathlineto{\pgfqpoint{5.871521in}{1.795456in}}%
\pgfpathlineto{\pgfqpoint{5.948884in}{0.790002in}}%
\pgfpathlineto{\pgfqpoint{6.026248in}{0.597442in}}%
\pgfpathlineto{\pgfqpoint{6.103611in}{0.557064in}}%
\pgfpathlineto{\pgfqpoint{6.180974in}{0.542153in}}%
\pgfpathlineto{\pgfqpoint{6.258338in}{0.531221in}}%
\pgfpathlineto{\pgfqpoint{6.335701in}{0.525430in}}%
\pgfpathlineto{\pgfqpoint{6.413065in}{0.518341in}}%
\pgfpathlineto{\pgfqpoint{6.490428in}{0.516931in}}%
\pgfpathlineto{\pgfqpoint{6.567791in}{0.514029in}}%
\pgfpathlineto{\pgfqpoint{6.645155in}{0.511209in}}%
\pgfpathlineto{\pgfqpoint{6.722518in}{0.510072in}}%
\pgfpathlineto{\pgfqpoint{6.799881in}{0.508348in}}%
\pgfpathlineto{\pgfqpoint{6.877245in}{0.506912in}}%
\pgfpathlineto{\pgfqpoint{6.954608in}{0.505975in}}%
\pgfpathlineto{\pgfqpoint{7.031971in}{0.504790in}}%
\pgfpathlineto{\pgfqpoint{7.109335in}{0.504565in}}%
\pgfpathlineto{\pgfqpoint{7.186698in}{0.503784in}}%
\pgfpathlineto{\pgfqpoint{7.264061in}{0.502920in}}%
\pgfpathlineto{\pgfqpoint{7.341425in}{0.502182in}}%
\pgfpathlineto{\pgfqpoint{7.418788in}{0.501554in}}%
\pgfpathlineto{\pgfqpoint{7.496152in}{0.501143in}}%
\pgfpathlineto{\pgfqpoint{7.573515in}{0.500984in}}%
\pgfpathlineto{\pgfqpoint{7.650878in}{0.500462in}}%
\pgfpathlineto{\pgfqpoint{7.650878in}{0.500893in}}%
\pgfpathlineto{\pgfqpoint{7.650878in}{0.500893in}}%
\pgfpathlineto{\pgfqpoint{7.573515in}{0.501581in}}%
\pgfpathlineto{\pgfqpoint{7.496152in}{0.501549in}}%
\pgfpathlineto{\pgfqpoint{7.418788in}{0.502404in}}%
\pgfpathlineto{\pgfqpoint{7.341425in}{0.502743in}}%
\pgfpathlineto{\pgfqpoint{7.264061in}{0.503287in}}%
\pgfpathlineto{\pgfqpoint{7.186698in}{0.503958in}}%
\pgfpathlineto{\pgfqpoint{7.109335in}{0.504965in}}%
\pgfpathlineto{\pgfqpoint{7.031971in}{0.506035in}}%
\pgfpathlineto{\pgfqpoint{6.954608in}{0.506659in}}%
\pgfpathlineto{\pgfqpoint{6.877245in}{0.508103in}}%
\pgfpathlineto{\pgfqpoint{6.799881in}{0.509425in}}%
\pgfpathlineto{\pgfqpoint{6.722518in}{0.510709in}}%
\pgfpathlineto{\pgfqpoint{6.645155in}{0.512929in}}%
\pgfpathlineto{\pgfqpoint{6.567791in}{0.514823in}}%
\pgfpathlineto{\pgfqpoint{6.490428in}{0.518214in}}%
\pgfpathlineto{\pgfqpoint{6.413065in}{0.521909in}}%
\pgfpathlineto{\pgfqpoint{6.335701in}{0.527327in}}%
\pgfpathlineto{\pgfqpoint{6.258338in}{0.534348in}}%
\pgfpathlineto{\pgfqpoint{6.180974in}{0.547435in}}%
\pgfpathlineto{\pgfqpoint{6.103611in}{0.571691in}}%
\pgfpathlineto{\pgfqpoint{6.026248in}{0.619889in}}%
\pgfpathlineto{\pgfqpoint{5.948884in}{1.100888in}}%
\pgfpathlineto{\pgfqpoint{5.871521in}{1.796678in}}%
\pgfpathlineto{\pgfqpoint{5.794158in}{1.951931in}}%
\pgfpathlineto{\pgfqpoint{5.716794in}{2.010565in}}%
\pgfpathlineto{\pgfqpoint{5.639431in}{2.256852in}}%
\pgfpathlineto{\pgfqpoint{5.562068in}{2.086364in}}%
\pgfpathlineto{\pgfqpoint{5.484704in}{2.422781in}}%
\pgfpathlineto{\pgfqpoint{5.407341in}{2.606010in}}%
\pgfpathlineto{\pgfqpoint{5.329978in}{2.136058in}}%
\pgfpathlineto{\pgfqpoint{5.252614in}{2.144249in}}%
\pgfpathlineto{\pgfqpoint{5.175251in}{2.149922in}}%
\pgfpathlineto{\pgfqpoint{5.097887in}{2.162931in}}%
\pgfpathlineto{\pgfqpoint{5.020524in}{2.156080in}}%
\pgfpathlineto{\pgfqpoint{4.943161in}{2.157464in}}%
\pgfpathlineto{\pgfqpoint{4.865797in}{2.158179in}}%
\pgfpathlineto{\pgfqpoint{4.788434in}{2.158463in}}%
\pgfpathlineto{\pgfqpoint{4.711071in}{2.158504in}}%
\pgfpathlineto{\pgfqpoint{4.633707in}{2.158664in}}%
\pgfpathlineto{\pgfqpoint{4.556344in}{2.158485in}}%
\pgfpathclose%
\pgfusepath{stroke,fill}%
\end{pgfscope}%
\begin{pgfscope}%
\pgfpathrectangle{\pgfqpoint{4.392222in}{0.387222in}}{\pgfqpoint{3.422778in}{2.324444in}}%
\pgfusepath{clip}%
\pgfsetbuttcap%
\pgfsetroundjoin%
\definecolor{currentfill}{rgb}{0.121569,0.466667,0.705882}%
\pgfsetfillcolor{currentfill}%
\pgfsetlinewidth{1.003750pt}%
\definecolor{currentstroke}{rgb}{0.121569,0.466667,0.705882}%
\pgfsetstrokecolor{currentstroke}%
\pgfsetdash{}{0pt}%
\pgfsys@defobject{currentmarker}{\pgfqpoint{-0.009821in}{-0.009821in}}{\pgfqpoint{0.009821in}{0.009821in}}{%
\pgfpathmoveto{\pgfqpoint{0.000000in}{-0.009821in}}%
\pgfpathcurveto{\pgfqpoint{0.002605in}{-0.009821in}}{\pgfqpoint{0.005103in}{-0.008786in}}{\pgfqpoint{0.006944in}{-0.006944in}}%
\pgfpathcurveto{\pgfqpoint{0.008786in}{-0.005103in}}{\pgfqpoint{0.009821in}{-0.002605in}}{\pgfqpoint{0.009821in}{0.000000in}}%
\pgfpathcurveto{\pgfqpoint{0.009821in}{0.002605in}}{\pgfqpoint{0.008786in}{0.005103in}}{\pgfqpoint{0.006944in}{0.006944in}}%
\pgfpathcurveto{\pgfqpoint{0.005103in}{0.008786in}}{\pgfqpoint{0.002605in}{0.009821in}}{\pgfqpoint{0.000000in}{0.009821in}}%
\pgfpathcurveto{\pgfqpoint{-0.002605in}{0.009821in}}{\pgfqpoint{-0.005103in}{0.008786in}}{\pgfqpoint{-0.006944in}{0.006944in}}%
\pgfpathcurveto{\pgfqpoint{-0.008786in}{0.005103in}}{\pgfqpoint{-0.009821in}{0.002605in}}{\pgfqpoint{-0.009821in}{0.000000in}}%
\pgfpathcurveto{\pgfqpoint{-0.009821in}{-0.002605in}}{\pgfqpoint{-0.008786in}{-0.005103in}}{\pgfqpoint{-0.006944in}{-0.006944in}}%
\pgfpathcurveto{\pgfqpoint{-0.005103in}{-0.008786in}}{\pgfqpoint{-0.002605in}{-0.009821in}}{\pgfqpoint{0.000000in}{-0.009821in}}%
\pgfpathclose%
\pgfusepath{stroke,fill}%
}%
\begin{pgfscope}%
\pgfsys@transformshift{4.556344in}{2.158484in}%
\pgfsys@useobject{currentmarker}{}%
\end{pgfscope}%
\begin{pgfscope}%
\pgfsys@transformshift{4.633707in}{2.158380in}%
\pgfsys@useobject{currentmarker}{}%
\end{pgfscope}%
\begin{pgfscope}%
\pgfsys@transformshift{4.711071in}{2.158504in}%
\pgfsys@useobject{currentmarker}{}%
\end{pgfscope}%
\begin{pgfscope}%
\pgfsys@transformshift{4.788434in}{2.158463in}%
\pgfsys@useobject{currentmarker}{}%
\end{pgfscope}%
\begin{pgfscope}%
\pgfsys@transformshift{4.865797in}{2.158179in}%
\pgfsys@useobject{currentmarker}{}%
\end{pgfscope}%
\begin{pgfscope}%
\pgfsys@transformshift{4.943161in}{2.157464in}%
\pgfsys@useobject{currentmarker}{}%
\end{pgfscope}%
\begin{pgfscope}%
\pgfsys@transformshift{5.020524in}{2.156079in}%
\pgfsys@useobject{currentmarker}{}%
\end{pgfscope}%
\begin{pgfscope}%
\pgfsys@transformshift{5.097887in}{2.147920in}%
\pgfsys@useobject{currentmarker}{}%
\end{pgfscope}%
\begin{pgfscope}%
\pgfsys@transformshift{5.175251in}{2.149919in}%
\pgfsys@useobject{currentmarker}{}%
\end{pgfscope}%
\begin{pgfscope}%
\pgfsys@transformshift{5.252614in}{2.144242in}%
\pgfsys@useobject{currentmarker}{}%
\end{pgfscope}%
\begin{pgfscope}%
\pgfsys@transformshift{5.329978in}{2.136050in}%
\pgfsys@useobject{currentmarker}{}%
\end{pgfscope}%
\begin{pgfscope}%
\pgfsys@transformshift{5.407341in}{1.823229in}%
\pgfsys@useobject{currentmarker}{}%
\end{pgfscope}%
\begin{pgfscope}%
\pgfsys@transformshift{5.484704in}{1.911802in}%
\pgfsys@useobject{currentmarker}{}%
\end{pgfscope}%
\begin{pgfscope}%
\pgfsys@transformshift{5.562068in}{2.086316in}%
\pgfsys@useobject{currentmarker}{}%
\end{pgfscope}%
\begin{pgfscope}%
\pgfsys@transformshift{5.639431in}{1.929047in}%
\pgfsys@useobject{currentmarker}{}%
\end{pgfscope}%
\begin{pgfscope}%
\pgfsys@transformshift{5.716794in}{2.009733in}%
\pgfsys@useobject{currentmarker}{}%
\end{pgfscope}%
\begin{pgfscope}%
\pgfsys@transformshift{5.794158in}{1.931382in}%
\pgfsys@useobject{currentmarker}{}%
\end{pgfscope}%
\begin{pgfscope}%
\pgfsys@transformshift{5.871521in}{1.796067in}%
\pgfsys@useobject{currentmarker}{}%
\end{pgfscope}%
\begin{pgfscope}%
\pgfsys@transformshift{5.948884in}{0.945445in}%
\pgfsys@useobject{currentmarker}{}%
\end{pgfscope}%
\begin{pgfscope}%
\pgfsys@transformshift{6.026248in}{0.608666in}%
\pgfsys@useobject{currentmarker}{}%
\end{pgfscope}%
\begin{pgfscope}%
\pgfsys@transformshift{6.103611in}{0.564377in}%
\pgfsys@useobject{currentmarker}{}%
\end{pgfscope}%
\begin{pgfscope}%
\pgfsys@transformshift{6.180974in}{0.544794in}%
\pgfsys@useobject{currentmarker}{}%
\end{pgfscope}%
\begin{pgfscope}%
\pgfsys@transformshift{6.258338in}{0.532785in}%
\pgfsys@useobject{currentmarker}{}%
\end{pgfscope}%
\begin{pgfscope}%
\pgfsys@transformshift{6.335701in}{0.526379in}%
\pgfsys@useobject{currentmarker}{}%
\end{pgfscope}%
\begin{pgfscope}%
\pgfsys@transformshift{6.413065in}{0.520125in}%
\pgfsys@useobject{currentmarker}{}%
\end{pgfscope}%
\begin{pgfscope}%
\pgfsys@transformshift{6.490428in}{0.517572in}%
\pgfsys@useobject{currentmarker}{}%
\end{pgfscope}%
\begin{pgfscope}%
\pgfsys@transformshift{6.567791in}{0.514426in}%
\pgfsys@useobject{currentmarker}{}%
\end{pgfscope}%
\begin{pgfscope}%
\pgfsys@transformshift{6.645155in}{0.512069in}%
\pgfsys@useobject{currentmarker}{}%
\end{pgfscope}%
\begin{pgfscope}%
\pgfsys@transformshift{6.722518in}{0.510391in}%
\pgfsys@useobject{currentmarker}{}%
\end{pgfscope}%
\begin{pgfscope}%
\pgfsys@transformshift{6.799881in}{0.508886in}%
\pgfsys@useobject{currentmarker}{}%
\end{pgfscope}%
\begin{pgfscope}%
\pgfsys@transformshift{6.877245in}{0.507507in}%
\pgfsys@useobject{currentmarker}{}%
\end{pgfscope}%
\begin{pgfscope}%
\pgfsys@transformshift{6.954608in}{0.506317in}%
\pgfsys@useobject{currentmarker}{}%
\end{pgfscope}%
\begin{pgfscope}%
\pgfsys@transformshift{7.031971in}{0.505412in}%
\pgfsys@useobject{currentmarker}{}%
\end{pgfscope}%
\begin{pgfscope}%
\pgfsys@transformshift{7.109335in}{0.504765in}%
\pgfsys@useobject{currentmarker}{}%
\end{pgfscope}%
\begin{pgfscope}%
\pgfsys@transformshift{7.186698in}{0.503871in}%
\pgfsys@useobject{currentmarker}{}%
\end{pgfscope}%
\begin{pgfscope}%
\pgfsys@transformshift{7.264061in}{0.503104in}%
\pgfsys@useobject{currentmarker}{}%
\end{pgfscope}%
\begin{pgfscope}%
\pgfsys@transformshift{7.341425in}{0.502462in}%
\pgfsys@useobject{currentmarker}{}%
\end{pgfscope}%
\begin{pgfscope}%
\pgfsys@transformshift{7.418788in}{0.501979in}%
\pgfsys@useobject{currentmarker}{}%
\end{pgfscope}%
\begin{pgfscope}%
\pgfsys@transformshift{7.496152in}{0.501346in}%
\pgfsys@useobject{currentmarker}{}%
\end{pgfscope}%
\begin{pgfscope}%
\pgfsys@transformshift{7.573515in}{0.501282in}%
\pgfsys@useobject{currentmarker}{}%
\end{pgfscope}%
\begin{pgfscope}%
\pgfsys@transformshift{7.650878in}{0.500678in}%
\pgfsys@useobject{currentmarker}{}%
\end{pgfscope}%
\end{pgfscope}%
\begin{pgfscope}%
\pgfsetbuttcap%
\pgfsetroundjoin%
\definecolor{currentfill}{rgb}{0.000000,0.000000,0.000000}%
\pgfsetfillcolor{currentfill}%
\pgfsetlinewidth{0.803000pt}%
\definecolor{currentstroke}{rgb}{0.000000,0.000000,0.000000}%
\pgfsetstrokecolor{currentstroke}%
\pgfsetdash{}{0pt}%
\pgfsys@defobject{currentmarker}{\pgfqpoint{0.000000in}{-0.048611in}}{\pgfqpoint{0.000000in}{0.000000in}}{%
\pgfpathmoveto{\pgfqpoint{0.000000in}{0.000000in}}%
\pgfpathlineto{\pgfqpoint{0.000000in}{-0.048611in}}%
\pgfusepath{stroke,fill}%
}%
\begin{pgfscope}%
\pgfsys@transformshift{4.943161in}{0.387222in}%
\pgfsys@useobject{currentmarker}{}%
\end{pgfscope}%
\end{pgfscope}%
\begin{pgfscope}%
\definecolor{textcolor}{rgb}{0.000000,0.000000,0.000000}%
\pgfsetstrokecolor{textcolor}%
\pgfsetfillcolor{textcolor}%
\pgftext[x=4.943161in,y=0.290000in,,top]{\color{textcolor}\sffamily\fontsize{10.000000}{12.000000}\selectfont 1}%
\end{pgfscope}%
\begin{pgfscope}%
\pgfsetbuttcap%
\pgfsetroundjoin%
\definecolor{currentfill}{rgb}{0.000000,0.000000,0.000000}%
\pgfsetfillcolor{currentfill}%
\pgfsetlinewidth{0.803000pt}%
\definecolor{currentstroke}{rgb}{0.000000,0.000000,0.000000}%
\pgfsetstrokecolor{currentstroke}%
\pgfsetdash{}{0pt}%
\pgfsys@defobject{currentmarker}{\pgfqpoint{0.000000in}{-0.048611in}}{\pgfqpoint{0.000000in}{0.000000in}}{%
\pgfpathmoveto{\pgfqpoint{0.000000in}{0.000000in}}%
\pgfpathlineto{\pgfqpoint{0.000000in}{-0.048611in}}%
\pgfusepath{stroke,fill}%
}%
\begin{pgfscope}%
\pgfsys@transformshift{5.716794in}{0.387222in}%
\pgfsys@useobject{currentmarker}{}%
\end{pgfscope}%
\end{pgfscope}%
\begin{pgfscope}%
\definecolor{textcolor}{rgb}{0.000000,0.000000,0.000000}%
\pgfsetstrokecolor{textcolor}%
\pgfsetfillcolor{textcolor}%
\pgftext[x=5.716794in,y=0.290000in,,top]{\color{textcolor}\sffamily\fontsize{10.000000}{12.000000}\selectfont 2}%
\end{pgfscope}%
\begin{pgfscope}%
\pgfsetbuttcap%
\pgfsetroundjoin%
\definecolor{currentfill}{rgb}{0.000000,0.000000,0.000000}%
\pgfsetfillcolor{currentfill}%
\pgfsetlinewidth{0.803000pt}%
\definecolor{currentstroke}{rgb}{0.000000,0.000000,0.000000}%
\pgfsetstrokecolor{currentstroke}%
\pgfsetdash{}{0pt}%
\pgfsys@defobject{currentmarker}{\pgfqpoint{0.000000in}{-0.048611in}}{\pgfqpoint{0.000000in}{0.000000in}}{%
\pgfpathmoveto{\pgfqpoint{0.000000in}{0.000000in}}%
\pgfpathlineto{\pgfqpoint{0.000000in}{-0.048611in}}%
\pgfusepath{stroke,fill}%
}%
\begin{pgfscope}%
\pgfsys@transformshift{6.490428in}{0.387222in}%
\pgfsys@useobject{currentmarker}{}%
\end{pgfscope}%
\end{pgfscope}%
\begin{pgfscope}%
\definecolor{textcolor}{rgb}{0.000000,0.000000,0.000000}%
\pgfsetstrokecolor{textcolor}%
\pgfsetfillcolor{textcolor}%
\pgftext[x=6.490428in,y=0.290000in,,top]{\color{textcolor}\sffamily\fontsize{10.000000}{12.000000}\selectfont 3}%
\end{pgfscope}%
\begin{pgfscope}%
\pgfsetbuttcap%
\pgfsetroundjoin%
\definecolor{currentfill}{rgb}{0.000000,0.000000,0.000000}%
\pgfsetfillcolor{currentfill}%
\pgfsetlinewidth{0.803000pt}%
\definecolor{currentstroke}{rgb}{0.000000,0.000000,0.000000}%
\pgfsetstrokecolor{currentstroke}%
\pgfsetdash{}{0pt}%
\pgfsys@defobject{currentmarker}{\pgfqpoint{0.000000in}{-0.048611in}}{\pgfqpoint{0.000000in}{0.000000in}}{%
\pgfpathmoveto{\pgfqpoint{0.000000in}{0.000000in}}%
\pgfpathlineto{\pgfqpoint{0.000000in}{-0.048611in}}%
\pgfusepath{stroke,fill}%
}%
\begin{pgfscope}%
\pgfsys@transformshift{7.264061in}{0.387222in}%
\pgfsys@useobject{currentmarker}{}%
\end{pgfscope}%
\end{pgfscope}%
\begin{pgfscope}%
\definecolor{textcolor}{rgb}{0.000000,0.000000,0.000000}%
\pgfsetstrokecolor{textcolor}%
\pgfsetfillcolor{textcolor}%
\pgftext[x=7.264061in,y=0.290000in,,top]{\color{textcolor}\sffamily\fontsize{10.000000}{12.000000}\selectfont 4}%
\end{pgfscope}%
\begin{pgfscope}%
\pgfsetbuttcap%
\pgfsetroundjoin%
\definecolor{currentfill}{rgb}{0.000000,0.000000,0.000000}%
\pgfsetfillcolor{currentfill}%
\pgfsetlinewidth{0.803000pt}%
\definecolor{currentstroke}{rgb}{0.000000,0.000000,0.000000}%
\pgfsetstrokecolor{currentstroke}%
\pgfsetdash{}{0pt}%
\pgfsys@defobject{currentmarker}{\pgfqpoint{-0.048611in}{0.000000in}}{\pgfqpoint{0.000000in}{0.000000in}}{%
\pgfpathmoveto{\pgfqpoint{0.000000in}{0.000000in}}%
\pgfpathlineto{\pgfqpoint{-0.048611in}{0.000000in}}%
\pgfusepath{stroke,fill}%
}%
\begin{pgfscope}%
\pgfsys@transformshift{4.392222in}{0.481509in}%
\pgfsys@useobject{currentmarker}{}%
\end{pgfscope}%
\end{pgfscope}%
\begin{pgfscope}%
\definecolor{textcolor}{rgb}{0.000000,0.000000,0.000000}%
\pgfsetstrokecolor{textcolor}%
\pgfsetfillcolor{textcolor}%
\pgftext[x=4.074121in,y=0.428748in,left,base]{\color{textcolor}\sffamily\fontsize{10.000000}{12.000000}\selectfont 0.0}%
\end{pgfscope}%
\begin{pgfscope}%
\pgfsetbuttcap%
\pgfsetroundjoin%
\definecolor{currentfill}{rgb}{0.000000,0.000000,0.000000}%
\pgfsetfillcolor{currentfill}%
\pgfsetlinewidth{0.803000pt}%
\definecolor{currentstroke}{rgb}{0.000000,0.000000,0.000000}%
\pgfsetstrokecolor{currentstroke}%
\pgfsetdash{}{0pt}%
\pgfsys@defobject{currentmarker}{\pgfqpoint{-0.048611in}{0.000000in}}{\pgfqpoint{0.000000in}{0.000000in}}{%
\pgfpathmoveto{\pgfqpoint{0.000000in}{0.000000in}}%
\pgfpathlineto{\pgfqpoint{-0.048611in}{0.000000in}}%
\pgfusepath{stroke,fill}%
}%
\begin{pgfscope}%
\pgfsys@transformshift{4.392222in}{0.816945in}%
\pgfsys@useobject{currentmarker}{}%
\end{pgfscope}%
\end{pgfscope}%
\begin{pgfscope}%
\definecolor{textcolor}{rgb}{0.000000,0.000000,0.000000}%
\pgfsetstrokecolor{textcolor}%
\pgfsetfillcolor{textcolor}%
\pgftext[x=4.074121in,y=0.764184in,left,base]{\color{textcolor}\sffamily\fontsize{10.000000}{12.000000}\selectfont 0.2}%
\end{pgfscope}%
\begin{pgfscope}%
\pgfsetbuttcap%
\pgfsetroundjoin%
\definecolor{currentfill}{rgb}{0.000000,0.000000,0.000000}%
\pgfsetfillcolor{currentfill}%
\pgfsetlinewidth{0.803000pt}%
\definecolor{currentstroke}{rgb}{0.000000,0.000000,0.000000}%
\pgfsetstrokecolor{currentstroke}%
\pgfsetdash{}{0pt}%
\pgfsys@defobject{currentmarker}{\pgfqpoint{-0.048611in}{0.000000in}}{\pgfqpoint{0.000000in}{0.000000in}}{%
\pgfpathmoveto{\pgfqpoint{0.000000in}{0.000000in}}%
\pgfpathlineto{\pgfqpoint{-0.048611in}{0.000000in}}%
\pgfusepath{stroke,fill}%
}%
\begin{pgfscope}%
\pgfsys@transformshift{4.392222in}{1.152381in}%
\pgfsys@useobject{currentmarker}{}%
\end{pgfscope}%
\end{pgfscope}%
\begin{pgfscope}%
\definecolor{textcolor}{rgb}{0.000000,0.000000,0.000000}%
\pgfsetstrokecolor{textcolor}%
\pgfsetfillcolor{textcolor}%
\pgftext[x=4.074121in,y=1.099620in,left,base]{\color{textcolor}\sffamily\fontsize{10.000000}{12.000000}\selectfont 0.4}%
\end{pgfscope}%
\begin{pgfscope}%
\pgfsetbuttcap%
\pgfsetroundjoin%
\definecolor{currentfill}{rgb}{0.000000,0.000000,0.000000}%
\pgfsetfillcolor{currentfill}%
\pgfsetlinewidth{0.803000pt}%
\definecolor{currentstroke}{rgb}{0.000000,0.000000,0.000000}%
\pgfsetstrokecolor{currentstroke}%
\pgfsetdash{}{0pt}%
\pgfsys@defobject{currentmarker}{\pgfqpoint{-0.048611in}{0.000000in}}{\pgfqpoint{0.000000in}{0.000000in}}{%
\pgfpathmoveto{\pgfqpoint{0.000000in}{0.000000in}}%
\pgfpathlineto{\pgfqpoint{-0.048611in}{0.000000in}}%
\pgfusepath{stroke,fill}%
}%
\begin{pgfscope}%
\pgfsys@transformshift{4.392222in}{1.487817in}%
\pgfsys@useobject{currentmarker}{}%
\end{pgfscope}%
\end{pgfscope}%
\begin{pgfscope}%
\definecolor{textcolor}{rgb}{0.000000,0.000000,0.000000}%
\pgfsetstrokecolor{textcolor}%
\pgfsetfillcolor{textcolor}%
\pgftext[x=4.074121in,y=1.435055in,left,base]{\color{textcolor}\sffamily\fontsize{10.000000}{12.000000}\selectfont 0.6}%
\end{pgfscope}%
\begin{pgfscope}%
\pgfsetbuttcap%
\pgfsetroundjoin%
\definecolor{currentfill}{rgb}{0.000000,0.000000,0.000000}%
\pgfsetfillcolor{currentfill}%
\pgfsetlinewidth{0.803000pt}%
\definecolor{currentstroke}{rgb}{0.000000,0.000000,0.000000}%
\pgfsetstrokecolor{currentstroke}%
\pgfsetdash{}{0pt}%
\pgfsys@defobject{currentmarker}{\pgfqpoint{-0.048611in}{0.000000in}}{\pgfqpoint{0.000000in}{0.000000in}}{%
\pgfpathmoveto{\pgfqpoint{0.000000in}{0.000000in}}%
\pgfpathlineto{\pgfqpoint{-0.048611in}{0.000000in}}%
\pgfusepath{stroke,fill}%
}%
\begin{pgfscope}%
\pgfsys@transformshift{4.392222in}{1.823253in}%
\pgfsys@useobject{currentmarker}{}%
\end{pgfscope}%
\end{pgfscope}%
\begin{pgfscope}%
\definecolor{textcolor}{rgb}{0.000000,0.000000,0.000000}%
\pgfsetstrokecolor{textcolor}%
\pgfsetfillcolor{textcolor}%
\pgftext[x=4.074121in,y=1.770491in,left,base]{\color{textcolor}\sffamily\fontsize{10.000000}{12.000000}\selectfont 0.8}%
\end{pgfscope}%
\begin{pgfscope}%
\pgfsetbuttcap%
\pgfsetroundjoin%
\definecolor{currentfill}{rgb}{0.000000,0.000000,0.000000}%
\pgfsetfillcolor{currentfill}%
\pgfsetlinewidth{0.803000pt}%
\definecolor{currentstroke}{rgb}{0.000000,0.000000,0.000000}%
\pgfsetstrokecolor{currentstroke}%
\pgfsetdash{}{0pt}%
\pgfsys@defobject{currentmarker}{\pgfqpoint{-0.048611in}{0.000000in}}{\pgfqpoint{0.000000in}{0.000000in}}{%
\pgfpathmoveto{\pgfqpoint{0.000000in}{0.000000in}}%
\pgfpathlineto{\pgfqpoint{-0.048611in}{0.000000in}}%
\pgfusepath{stroke,fill}%
}%
\begin{pgfscope}%
\pgfsys@transformshift{4.392222in}{2.158689in}%
\pgfsys@useobject{currentmarker}{}%
\end{pgfscope}%
\end{pgfscope}%
\begin{pgfscope}%
\definecolor{textcolor}{rgb}{0.000000,0.000000,0.000000}%
\pgfsetstrokecolor{textcolor}%
\pgfsetfillcolor{textcolor}%
\pgftext[x=4.074121in,y=2.105927in,left,base]{\color{textcolor}\sffamily\fontsize{10.000000}{12.000000}\selectfont 1.0}%
\end{pgfscope}%
\begin{pgfscope}%
\pgfsetbuttcap%
\pgfsetroundjoin%
\definecolor{currentfill}{rgb}{0.000000,0.000000,0.000000}%
\pgfsetfillcolor{currentfill}%
\pgfsetlinewidth{0.803000pt}%
\definecolor{currentstroke}{rgb}{0.000000,0.000000,0.000000}%
\pgfsetstrokecolor{currentstroke}%
\pgfsetdash{}{0pt}%
\pgfsys@defobject{currentmarker}{\pgfqpoint{-0.048611in}{0.000000in}}{\pgfqpoint{0.000000in}{0.000000in}}{%
\pgfpathmoveto{\pgfqpoint{0.000000in}{0.000000in}}%
\pgfpathlineto{\pgfqpoint{-0.048611in}{0.000000in}}%
\pgfusepath{stroke,fill}%
}%
\begin{pgfscope}%
\pgfsys@transformshift{4.392222in}{2.494125in}%
\pgfsys@useobject{currentmarker}{}%
\end{pgfscope}%
\end{pgfscope}%
\begin{pgfscope}%
\definecolor{textcolor}{rgb}{0.000000,0.000000,0.000000}%
\pgfsetstrokecolor{textcolor}%
\pgfsetfillcolor{textcolor}%
\pgftext[x=4.074121in,y=2.441363in,left,base]{\color{textcolor}\sffamily\fontsize{10.000000}{12.000000}\selectfont 1.2}%
\end{pgfscope}%
\begin{pgfscope}%
\pgfpathrectangle{\pgfqpoint{4.392222in}{0.387222in}}{\pgfqpoint{3.422778in}{2.324444in}}%
\pgfusepath{clip}%
\pgfsetrectcap%
\pgfsetroundjoin%
\pgfsetlinewidth{1.505625pt}%
\definecolor{currentstroke}{rgb}{0.121569,0.466667,0.705882}%
\pgfsetstrokecolor{currentstroke}%
\pgfsetdash{}{0pt}%
\pgfpathmoveto{\pgfqpoint{4.556344in}{2.158484in}}%
\pgfpathlineto{\pgfqpoint{4.633707in}{2.158380in}}%
\pgfpathlineto{\pgfqpoint{4.711071in}{2.158504in}}%
\pgfpathlineto{\pgfqpoint{4.788434in}{2.158463in}}%
\pgfpathlineto{\pgfqpoint{4.865797in}{2.158179in}}%
\pgfpathlineto{\pgfqpoint{4.943161in}{2.157464in}}%
\pgfpathlineto{\pgfqpoint{5.020524in}{2.156079in}}%
\pgfpathlineto{\pgfqpoint{5.097887in}{2.147920in}}%
\pgfpathlineto{\pgfqpoint{5.175251in}{2.149919in}}%
\pgfpathlineto{\pgfqpoint{5.252614in}{2.144242in}}%
\pgfpathlineto{\pgfqpoint{5.329978in}{2.136050in}}%
\pgfpathlineto{\pgfqpoint{5.407341in}{1.823229in}}%
\pgfpathlineto{\pgfqpoint{5.484704in}{1.911802in}}%
\pgfpathlineto{\pgfqpoint{5.562068in}{2.086316in}}%
\pgfpathlineto{\pgfqpoint{5.639431in}{1.929047in}}%
\pgfpathlineto{\pgfqpoint{5.716794in}{2.009733in}}%
\pgfpathlineto{\pgfqpoint{5.794158in}{1.931382in}}%
\pgfpathlineto{\pgfqpoint{5.871521in}{1.796067in}}%
\pgfpathlineto{\pgfqpoint{5.948884in}{0.945445in}}%
\pgfpathlineto{\pgfqpoint{6.026248in}{0.608666in}}%
\pgfpathlineto{\pgfqpoint{6.103611in}{0.564377in}}%
\pgfpathlineto{\pgfqpoint{6.180974in}{0.544794in}}%
\pgfpathlineto{\pgfqpoint{6.258338in}{0.532785in}}%
\pgfpathlineto{\pgfqpoint{6.335701in}{0.526379in}}%
\pgfpathlineto{\pgfqpoint{6.413065in}{0.520125in}}%
\pgfpathlineto{\pgfqpoint{6.490428in}{0.517572in}}%
\pgfpathlineto{\pgfqpoint{6.567791in}{0.514426in}}%
\pgfpathlineto{\pgfqpoint{6.645155in}{0.512069in}}%
\pgfpathlineto{\pgfqpoint{6.722518in}{0.510391in}}%
\pgfpathlineto{\pgfqpoint{6.799881in}{0.508886in}}%
\pgfpathlineto{\pgfqpoint{6.877245in}{0.507507in}}%
\pgfpathlineto{\pgfqpoint{6.954608in}{0.506317in}}%
\pgfpathlineto{\pgfqpoint{7.031971in}{0.505412in}}%
\pgfpathlineto{\pgfqpoint{7.109335in}{0.504765in}}%
\pgfpathlineto{\pgfqpoint{7.186698in}{0.503871in}}%
\pgfpathlineto{\pgfqpoint{7.264061in}{0.503104in}}%
\pgfpathlineto{\pgfqpoint{7.341425in}{0.502462in}}%
\pgfpathlineto{\pgfqpoint{7.418788in}{0.501979in}}%
\pgfpathlineto{\pgfqpoint{7.496152in}{0.501346in}}%
\pgfpathlineto{\pgfqpoint{7.573515in}{0.501282in}}%
\pgfpathlineto{\pgfqpoint{7.650878in}{0.500678in}}%
\pgfusepath{stroke}%
\end{pgfscope}%
\begin{pgfscope}%
\pgfsetrectcap%
\pgfsetmiterjoin%
\pgfsetlinewidth{0.803000pt}%
\definecolor{currentstroke}{rgb}{0.000000,0.000000,0.000000}%
\pgfsetstrokecolor{currentstroke}%
\pgfsetdash{}{0pt}%
\pgfpathmoveto{\pgfqpoint{4.392222in}{0.387222in}}%
\pgfpathlineto{\pgfqpoint{4.392222in}{2.711667in}}%
\pgfusepath{stroke}%
\end{pgfscope}%
\begin{pgfscope}%
\pgfsetrectcap%
\pgfsetmiterjoin%
\pgfsetlinewidth{0.803000pt}%
\definecolor{currentstroke}{rgb}{0.000000,0.000000,0.000000}%
\pgfsetstrokecolor{currentstroke}%
\pgfsetdash{}{0pt}%
\pgfpathmoveto{\pgfqpoint{7.815000in}{0.387222in}}%
\pgfpathlineto{\pgfqpoint{7.815000in}{2.711667in}}%
\pgfusepath{stroke}%
\end{pgfscope}%
\begin{pgfscope}%
\pgfsetrectcap%
\pgfsetmiterjoin%
\pgfsetlinewidth{0.803000pt}%
\definecolor{currentstroke}{rgb}{0.000000,0.000000,0.000000}%
\pgfsetstrokecolor{currentstroke}%
\pgfsetdash{}{0pt}%
\pgfpathmoveto{\pgfqpoint{4.392222in}{0.387222in}}%
\pgfpathlineto{\pgfqpoint{7.815000in}{0.387222in}}%
\pgfusepath{stroke}%
\end{pgfscope}%
\begin{pgfscope}%
\pgfsetrectcap%
\pgfsetmiterjoin%
\pgfsetlinewidth{0.803000pt}%
\definecolor{currentstroke}{rgb}{0.000000,0.000000,0.000000}%
\pgfsetstrokecolor{currentstroke}%
\pgfsetdash{}{0pt}%
\pgfpathmoveto{\pgfqpoint{4.392222in}{2.711667in}}%
\pgfpathlineto{\pgfqpoint{7.815000in}{2.711667in}}%
\pgfusepath{stroke}%
\end{pgfscope}%
\begin{pgfscope}%
\definecolor{textcolor}{rgb}{0.000000,0.000000,0.000000}%
\pgfsetstrokecolor{textcolor}%
\pgfsetfillcolor{textcolor}%
\pgftext[x=6.103611in,y=2.795000in,,base]{\color{textcolor}\sffamily\fontsize{12.000000}{14.400000}\selectfont \(\displaystyle  N = 128 \)}%
\end{pgfscope}%
\end{pgfpicture}%
\makeatother%
\endgroup%
}
\caption{Magnetization $m$ for different temperatures $T$ using kinetic Monte Carlo algorithm}
\label{Fig:MagSmallKMC}
\end{figure}

\begin{figure}[htbp]
\centering
\scalebox{0.666}{%% Creator: Matplotlib, PGF backend
%%
%% To include the figure in your LaTeX document, write
%%   \input{<filename>.pgf}
%%
%% Make sure the required packages are loaded in your preamble
%%   \usepackage{pgf}
%%
%% Figures using additional raster images can only be included by \input if
%% they are in the same directory as the main LaTeX file. For loading figures
%% from other directories you can use the `import` package
%%   \usepackage{import}
%% and then include the figures with
%%   \import{<path to file>}{<filename>.pgf}
%%
%% Matplotlib used the following preamble
%%   \usepackage{fontspec}
%%   \setmainfont{DejaVuSerif.ttf}[Path=/home/lzh/anaconda3/envs/numana/lib/python3.7/site-packages/matplotlib/mpl-data/fonts/ttf/]
%%   \setsansfont{DejaVuSans.ttf}[Path=/home/lzh/anaconda3/envs/numana/lib/python3.7/site-packages/matplotlib/mpl-data/fonts/ttf/]
%%   \setmonofont{DejaVuSansMono.ttf}[Path=/home/lzh/anaconda3/envs/numana/lib/python3.7/site-packages/matplotlib/mpl-data/fonts/ttf/]
%%
\begingroup%
\makeatletter%
\begin{pgfpicture}%
\pgfpathrectangle{\pgfpointorigin}{\pgfqpoint{8.000000in}{6.000000in}}%
\pgfusepath{use as bounding box, clip}%
\begin{pgfscope}%
\pgfsetbuttcap%
\pgfsetmiterjoin%
\definecolor{currentfill}{rgb}{1.000000,1.000000,1.000000}%
\pgfsetfillcolor{currentfill}%
\pgfsetlinewidth{0.000000pt}%
\definecolor{currentstroke}{rgb}{1.000000,1.000000,1.000000}%
\pgfsetstrokecolor{currentstroke}%
\pgfsetdash{}{0pt}%
\pgfpathmoveto{\pgfqpoint{0.000000in}{0.000000in}}%
\pgfpathlineto{\pgfqpoint{8.000000in}{0.000000in}}%
\pgfpathlineto{\pgfqpoint{8.000000in}{6.000000in}}%
\pgfpathlineto{\pgfqpoint{0.000000in}{6.000000in}}%
\pgfpathclose%
\pgfusepath{fill}%
\end{pgfscope}%
\begin{pgfscope}%
\pgfsetbuttcap%
\pgfsetmiterjoin%
\definecolor{currentfill}{rgb}{1.000000,1.000000,1.000000}%
\pgfsetfillcolor{currentfill}%
\pgfsetlinewidth{0.000000pt}%
\definecolor{currentstroke}{rgb}{0.000000,0.000000,0.000000}%
\pgfsetstrokecolor{currentstroke}%
\pgfsetstrokeopacity{0.000000}%
\pgfsetdash{}{0pt}%
\pgfpathmoveto{\pgfqpoint{0.672222in}{3.312222in}}%
\pgfpathlineto{\pgfqpoint{3.878555in}{3.312222in}}%
\pgfpathlineto{\pgfqpoint{3.878555in}{5.636667in}}%
\pgfpathlineto{\pgfqpoint{0.672222in}{5.636667in}}%
\pgfpathclose%
\pgfusepath{fill}%
\end{pgfscope}%
\begin{pgfscope}%
\pgfpathrectangle{\pgfqpoint{0.672222in}{3.312222in}}{\pgfqpoint{3.206333in}{2.324444in}}%
\pgfusepath{clip}%
\pgfsetbuttcap%
\pgfsetroundjoin%
\definecolor{currentfill}{rgb}{0.121569,0.466667,0.705882}%
\pgfsetfillcolor{currentfill}%
\pgfsetfillopacity{0.300000}%
\pgfsetlinewidth{1.003750pt}%
\definecolor{currentstroke}{rgb}{0.121569,0.466667,0.705882}%
\pgfsetstrokecolor{currentstroke}%
\pgfsetstrokeopacity{0.300000}%
\pgfsetdash{}{0pt}%
\pgfpathmoveto{\pgfqpoint{0.825965in}{3.428684in}}%
\pgfpathlineto{\pgfqpoint{0.825965in}{3.422674in}}%
\pgfpathlineto{\pgfqpoint{0.970908in}{3.521350in}}%
\pgfpathlineto{\pgfqpoint{1.115850in}{3.624664in}}%
\pgfpathlineto{\pgfqpoint{1.260792in}{3.716398in}}%
\pgfpathlineto{\pgfqpoint{1.405735in}{3.825205in}}%
\pgfpathlineto{\pgfqpoint{1.550677in}{3.931042in}}%
\pgfpathlineto{\pgfqpoint{1.695619in}{4.038821in}}%
\pgfpathlineto{\pgfqpoint{1.840562in}{4.143972in}}%
\pgfpathlineto{\pgfqpoint{1.985504in}{4.253628in}}%
\pgfpathlineto{\pgfqpoint{2.130446in}{4.362418in}}%
\pgfpathlineto{\pgfqpoint{2.275389in}{4.475496in}}%
\pgfpathlineto{\pgfqpoint{2.420331in}{4.580175in}}%
\pgfpathlineto{\pgfqpoint{2.565273in}{4.690314in}}%
\pgfpathlineto{\pgfqpoint{2.710216in}{4.796825in}}%
\pgfpathlineto{\pgfqpoint{2.855158in}{4.907901in}}%
\pgfpathlineto{\pgfqpoint{3.000100in}{5.014260in}}%
\pgfpathlineto{\pgfqpoint{3.145043in}{5.119603in}}%
\pgfpathlineto{\pgfqpoint{3.289985in}{5.221180in}}%
\pgfpathlineto{\pgfqpoint{3.434927in}{5.323343in}}%
\pgfpathlineto{\pgfqpoint{3.579870in}{5.426447in}}%
\pgfpathlineto{\pgfqpoint{3.724812in}{5.517775in}}%
\pgfpathlineto{\pgfqpoint{3.724812in}{5.528644in}}%
\pgfpathlineto{\pgfqpoint{3.724812in}{5.528644in}}%
\pgfpathlineto{\pgfqpoint{3.579870in}{5.433394in}}%
\pgfpathlineto{\pgfqpoint{3.434927in}{5.328971in}}%
\pgfpathlineto{\pgfqpoint{3.289985in}{5.226761in}}%
\pgfpathlineto{\pgfqpoint{3.145043in}{5.122440in}}%
\pgfpathlineto{\pgfqpoint{3.000100in}{5.026956in}}%
\pgfpathlineto{\pgfqpoint{2.855158in}{4.925648in}}%
\pgfpathlineto{\pgfqpoint{2.710216in}{4.813423in}}%
\pgfpathlineto{\pgfqpoint{2.565273in}{4.700544in}}%
\pgfpathlineto{\pgfqpoint{2.420331in}{4.592246in}}%
\pgfpathlineto{\pgfqpoint{2.275389in}{4.480286in}}%
\pgfpathlineto{\pgfqpoint{2.130446in}{4.367506in}}%
\pgfpathlineto{\pgfqpoint{1.985504in}{4.259123in}}%
\pgfpathlineto{\pgfqpoint{1.840562in}{4.150461in}}%
\pgfpathlineto{\pgfqpoint{1.695619in}{4.042888in}}%
\pgfpathlineto{\pgfqpoint{1.550677in}{3.937153in}}%
\pgfpathlineto{\pgfqpoint{1.405735in}{3.834836in}}%
\pgfpathlineto{\pgfqpoint{1.260792in}{3.730688in}}%
\pgfpathlineto{\pgfqpoint{1.115850in}{3.628198in}}%
\pgfpathlineto{\pgfqpoint{0.970908in}{3.529196in}}%
\pgfpathlineto{\pgfqpoint{0.825965in}{3.428684in}}%
\pgfpathclose%
\pgfusepath{stroke,fill}%
\end{pgfscope}%
\begin{pgfscope}%
\pgfpathrectangle{\pgfqpoint{0.672222in}{3.312222in}}{\pgfqpoint{3.206333in}{2.324444in}}%
\pgfusepath{clip}%
\pgfsetbuttcap%
\pgfsetroundjoin%
\definecolor{currentfill}{rgb}{0.121569,0.466667,0.705882}%
\pgfsetfillcolor{currentfill}%
\pgfsetlinewidth{1.003750pt}%
\definecolor{currentstroke}{rgb}{0.121569,0.466667,0.705882}%
\pgfsetstrokecolor{currentstroke}%
\pgfsetdash{}{0pt}%
\pgfsys@defobject{currentmarker}{\pgfqpoint{-0.009821in}{-0.009821in}}{\pgfqpoint{0.009821in}{0.009821in}}{%
\pgfpathmoveto{\pgfqpoint{0.000000in}{-0.009821in}}%
\pgfpathcurveto{\pgfqpoint{0.002605in}{-0.009821in}}{\pgfqpoint{0.005103in}{-0.008786in}}{\pgfqpoint{0.006944in}{-0.006944in}}%
\pgfpathcurveto{\pgfqpoint{0.008786in}{-0.005103in}}{\pgfqpoint{0.009821in}{-0.002605in}}{\pgfqpoint{0.009821in}{0.000000in}}%
\pgfpathcurveto{\pgfqpoint{0.009821in}{0.002605in}}{\pgfqpoint{0.008786in}{0.005103in}}{\pgfqpoint{0.006944in}{0.006944in}}%
\pgfpathcurveto{\pgfqpoint{0.005103in}{0.008786in}}{\pgfqpoint{0.002605in}{0.009821in}}{\pgfqpoint{0.000000in}{0.009821in}}%
\pgfpathcurveto{\pgfqpoint{-0.002605in}{0.009821in}}{\pgfqpoint{-0.005103in}{0.008786in}}{\pgfqpoint{-0.006944in}{0.006944in}}%
\pgfpathcurveto{\pgfqpoint{-0.008786in}{0.005103in}}{\pgfqpoint{-0.009821in}{0.002605in}}{\pgfqpoint{-0.009821in}{0.000000in}}%
\pgfpathcurveto{\pgfqpoint{-0.009821in}{-0.002605in}}{\pgfqpoint{-0.008786in}{-0.005103in}}{\pgfqpoint{-0.006944in}{-0.006944in}}%
\pgfpathcurveto{\pgfqpoint{-0.005103in}{-0.008786in}}{\pgfqpoint{-0.002605in}{-0.009821in}}{\pgfqpoint{0.000000in}{-0.009821in}}%
\pgfpathclose%
\pgfusepath{stroke,fill}%
}%
\begin{pgfscope}%
\pgfsys@transformshift{0.825965in}{3.425679in}%
\pgfsys@useobject{currentmarker}{}%
\end{pgfscope}%
\begin{pgfscope}%
\pgfsys@transformshift{0.970908in}{3.525273in}%
\pgfsys@useobject{currentmarker}{}%
\end{pgfscope}%
\begin{pgfscope}%
\pgfsys@transformshift{1.115850in}{3.626431in}%
\pgfsys@useobject{currentmarker}{}%
\end{pgfscope}%
\begin{pgfscope}%
\pgfsys@transformshift{1.260792in}{3.723543in}%
\pgfsys@useobject{currentmarker}{}%
\end{pgfscope}%
\begin{pgfscope}%
\pgfsys@transformshift{1.405735in}{3.830020in}%
\pgfsys@useobject{currentmarker}{}%
\end{pgfscope}%
\begin{pgfscope}%
\pgfsys@transformshift{1.550677in}{3.934098in}%
\pgfsys@useobject{currentmarker}{}%
\end{pgfscope}%
\begin{pgfscope}%
\pgfsys@transformshift{1.695619in}{4.040855in}%
\pgfsys@useobject{currentmarker}{}%
\end{pgfscope}%
\begin{pgfscope}%
\pgfsys@transformshift{1.840562in}{4.147217in}%
\pgfsys@useobject{currentmarker}{}%
\end{pgfscope}%
\begin{pgfscope}%
\pgfsys@transformshift{1.985504in}{4.256376in}%
\pgfsys@useobject{currentmarker}{}%
\end{pgfscope}%
\begin{pgfscope}%
\pgfsys@transformshift{2.130446in}{4.364962in}%
\pgfsys@useobject{currentmarker}{}%
\end{pgfscope}%
\begin{pgfscope}%
\pgfsys@transformshift{2.275389in}{4.477891in}%
\pgfsys@useobject{currentmarker}{}%
\end{pgfscope}%
\begin{pgfscope}%
\pgfsys@transformshift{2.420331in}{4.586210in}%
\pgfsys@useobject{currentmarker}{}%
\end{pgfscope}%
\begin{pgfscope}%
\pgfsys@transformshift{2.565273in}{4.695429in}%
\pgfsys@useobject{currentmarker}{}%
\end{pgfscope}%
\begin{pgfscope}%
\pgfsys@transformshift{2.710216in}{4.805124in}%
\pgfsys@useobject{currentmarker}{}%
\end{pgfscope}%
\begin{pgfscope}%
\pgfsys@transformshift{2.855158in}{4.916774in}%
\pgfsys@useobject{currentmarker}{}%
\end{pgfscope}%
\begin{pgfscope}%
\pgfsys@transformshift{3.000100in}{5.020608in}%
\pgfsys@useobject{currentmarker}{}%
\end{pgfscope}%
\begin{pgfscope}%
\pgfsys@transformshift{3.145043in}{5.121022in}%
\pgfsys@useobject{currentmarker}{}%
\end{pgfscope}%
\begin{pgfscope}%
\pgfsys@transformshift{3.289985in}{5.223971in}%
\pgfsys@useobject{currentmarker}{}%
\end{pgfscope}%
\begin{pgfscope}%
\pgfsys@transformshift{3.434927in}{5.326157in}%
\pgfsys@useobject{currentmarker}{}%
\end{pgfscope}%
\begin{pgfscope}%
\pgfsys@transformshift{3.579870in}{5.429920in}%
\pgfsys@useobject{currentmarker}{}%
\end{pgfscope}%
\begin{pgfscope}%
\pgfsys@transformshift{3.724812in}{5.523210in}%
\pgfsys@useobject{currentmarker}{}%
\end{pgfscope}%
\end{pgfscope}%
\begin{pgfscope}%
\pgfsetbuttcap%
\pgfsetroundjoin%
\definecolor{currentfill}{rgb}{0.000000,0.000000,0.000000}%
\pgfsetfillcolor{currentfill}%
\pgfsetlinewidth{0.803000pt}%
\definecolor{currentstroke}{rgb}{0.000000,0.000000,0.000000}%
\pgfsetstrokecolor{currentstroke}%
\pgfsetdash{}{0pt}%
\pgfsys@defobject{currentmarker}{\pgfqpoint{0.000000in}{-0.048611in}}{\pgfqpoint{0.000000in}{0.000000in}}{%
\pgfpathmoveto{\pgfqpoint{0.000000in}{0.000000in}}%
\pgfpathlineto{\pgfqpoint{0.000000in}{-0.048611in}}%
\pgfusepath{stroke,fill}%
}%
\begin{pgfscope}%
\pgfsys@transformshift{1.260792in}{3.312222in}%
\pgfsys@useobject{currentmarker}{}%
\end{pgfscope}%
\end{pgfscope}%
\begin{pgfscope}%
\definecolor{textcolor}{rgb}{0.000000,0.000000,0.000000}%
\pgfsetstrokecolor{textcolor}%
\pgfsetfillcolor{textcolor}%
\pgftext[x=1.260792in,y=3.215000in,,top]{\color{textcolor}\sffamily\fontsize{10.000000}{12.000000}\selectfont 2.25}%
\end{pgfscope}%
\begin{pgfscope}%
\pgfsetbuttcap%
\pgfsetroundjoin%
\definecolor{currentfill}{rgb}{0.000000,0.000000,0.000000}%
\pgfsetfillcolor{currentfill}%
\pgfsetlinewidth{0.803000pt}%
\definecolor{currentstroke}{rgb}{0.000000,0.000000,0.000000}%
\pgfsetstrokecolor{currentstroke}%
\pgfsetdash{}{0pt}%
\pgfsys@defobject{currentmarker}{\pgfqpoint{0.000000in}{-0.048611in}}{\pgfqpoint{0.000000in}{0.000000in}}{%
\pgfpathmoveto{\pgfqpoint{0.000000in}{0.000000in}}%
\pgfpathlineto{\pgfqpoint{0.000000in}{-0.048611in}}%
\pgfusepath{stroke,fill}%
}%
\begin{pgfscope}%
\pgfsys@transformshift{1.985504in}{3.312222in}%
\pgfsys@useobject{currentmarker}{}%
\end{pgfscope}%
\end{pgfscope}%
\begin{pgfscope}%
\definecolor{textcolor}{rgb}{0.000000,0.000000,0.000000}%
\pgfsetstrokecolor{textcolor}%
\pgfsetfillcolor{textcolor}%
\pgftext[x=1.985504in,y=3.215000in,,top]{\color{textcolor}\sffamily\fontsize{10.000000}{12.000000}\selectfont 2.30}%
\end{pgfscope}%
\begin{pgfscope}%
\pgfsetbuttcap%
\pgfsetroundjoin%
\definecolor{currentfill}{rgb}{0.000000,0.000000,0.000000}%
\pgfsetfillcolor{currentfill}%
\pgfsetlinewidth{0.803000pt}%
\definecolor{currentstroke}{rgb}{0.000000,0.000000,0.000000}%
\pgfsetstrokecolor{currentstroke}%
\pgfsetdash{}{0pt}%
\pgfsys@defobject{currentmarker}{\pgfqpoint{0.000000in}{-0.048611in}}{\pgfqpoint{0.000000in}{0.000000in}}{%
\pgfpathmoveto{\pgfqpoint{0.000000in}{0.000000in}}%
\pgfpathlineto{\pgfqpoint{0.000000in}{-0.048611in}}%
\pgfusepath{stroke,fill}%
}%
\begin{pgfscope}%
\pgfsys@transformshift{2.710216in}{3.312222in}%
\pgfsys@useobject{currentmarker}{}%
\end{pgfscope}%
\end{pgfscope}%
\begin{pgfscope}%
\definecolor{textcolor}{rgb}{0.000000,0.000000,0.000000}%
\pgfsetstrokecolor{textcolor}%
\pgfsetfillcolor{textcolor}%
\pgftext[x=2.710216in,y=3.215000in,,top]{\color{textcolor}\sffamily\fontsize{10.000000}{12.000000}\selectfont 2.35}%
\end{pgfscope}%
\begin{pgfscope}%
\pgfsetbuttcap%
\pgfsetroundjoin%
\definecolor{currentfill}{rgb}{0.000000,0.000000,0.000000}%
\pgfsetfillcolor{currentfill}%
\pgfsetlinewidth{0.803000pt}%
\definecolor{currentstroke}{rgb}{0.000000,0.000000,0.000000}%
\pgfsetstrokecolor{currentstroke}%
\pgfsetdash{}{0pt}%
\pgfsys@defobject{currentmarker}{\pgfqpoint{0.000000in}{-0.048611in}}{\pgfqpoint{0.000000in}{0.000000in}}{%
\pgfpathmoveto{\pgfqpoint{0.000000in}{0.000000in}}%
\pgfpathlineto{\pgfqpoint{0.000000in}{-0.048611in}}%
\pgfusepath{stroke,fill}%
}%
\begin{pgfscope}%
\pgfsys@transformshift{3.434927in}{3.312222in}%
\pgfsys@useobject{currentmarker}{}%
\end{pgfscope}%
\end{pgfscope}%
\begin{pgfscope}%
\definecolor{textcolor}{rgb}{0.000000,0.000000,0.000000}%
\pgfsetstrokecolor{textcolor}%
\pgfsetfillcolor{textcolor}%
\pgftext[x=3.434927in,y=3.215000in,,top]{\color{textcolor}\sffamily\fontsize{10.000000}{12.000000}\selectfont 2.40}%
\end{pgfscope}%
\begin{pgfscope}%
\pgfsetbuttcap%
\pgfsetroundjoin%
\definecolor{currentfill}{rgb}{0.000000,0.000000,0.000000}%
\pgfsetfillcolor{currentfill}%
\pgfsetlinewidth{0.803000pt}%
\definecolor{currentstroke}{rgb}{0.000000,0.000000,0.000000}%
\pgfsetstrokecolor{currentstroke}%
\pgfsetdash{}{0pt}%
\pgfsys@defobject{currentmarker}{\pgfqpoint{-0.048611in}{0.000000in}}{\pgfqpoint{0.000000in}{0.000000in}}{%
\pgfpathmoveto{\pgfqpoint{0.000000in}{0.000000in}}%
\pgfpathlineto{\pgfqpoint{-0.048611in}{0.000000in}}%
\pgfusepath{stroke,fill}%
}%
\begin{pgfscope}%
\pgfsys@transformshift{0.672222in}{3.591687in}%
\pgfsys@useobject{currentmarker}{}%
\end{pgfscope}%
\end{pgfscope}%
\begin{pgfscope}%
\definecolor{textcolor}{rgb}{0.000000,0.000000,0.000000}%
\pgfsetstrokecolor{textcolor}%
\pgfsetfillcolor{textcolor}%
\pgftext[x=0.149382in,y=3.538926in,left,base]{\color{textcolor}\sffamily\fontsize{10.000000}{12.000000}\selectfont −1.50}%
\end{pgfscope}%
\begin{pgfscope}%
\pgfsetbuttcap%
\pgfsetroundjoin%
\definecolor{currentfill}{rgb}{0.000000,0.000000,0.000000}%
\pgfsetfillcolor{currentfill}%
\pgfsetlinewidth{0.803000pt}%
\definecolor{currentstroke}{rgb}{0.000000,0.000000,0.000000}%
\pgfsetstrokecolor{currentstroke}%
\pgfsetdash{}{0pt}%
\pgfsys@defobject{currentmarker}{\pgfqpoint{-0.048611in}{0.000000in}}{\pgfqpoint{0.000000in}{0.000000in}}{%
\pgfpathmoveto{\pgfqpoint{0.000000in}{0.000000in}}%
\pgfpathlineto{\pgfqpoint{-0.048611in}{0.000000in}}%
\pgfusepath{stroke,fill}%
}%
\begin{pgfscope}%
\pgfsys@transformshift{0.672222in}{3.945482in}%
\pgfsys@useobject{currentmarker}{}%
\end{pgfscope}%
\end{pgfscope}%
\begin{pgfscope}%
\definecolor{textcolor}{rgb}{0.000000,0.000000,0.000000}%
\pgfsetstrokecolor{textcolor}%
\pgfsetfillcolor{textcolor}%
\pgftext[x=0.149382in,y=3.892721in,left,base]{\color{textcolor}\sffamily\fontsize{10.000000}{12.000000}\selectfont −1.45}%
\end{pgfscope}%
\begin{pgfscope}%
\pgfsetbuttcap%
\pgfsetroundjoin%
\definecolor{currentfill}{rgb}{0.000000,0.000000,0.000000}%
\pgfsetfillcolor{currentfill}%
\pgfsetlinewidth{0.803000pt}%
\definecolor{currentstroke}{rgb}{0.000000,0.000000,0.000000}%
\pgfsetstrokecolor{currentstroke}%
\pgfsetdash{}{0pt}%
\pgfsys@defobject{currentmarker}{\pgfqpoint{-0.048611in}{0.000000in}}{\pgfqpoint{0.000000in}{0.000000in}}{%
\pgfpathmoveto{\pgfqpoint{0.000000in}{0.000000in}}%
\pgfpathlineto{\pgfqpoint{-0.048611in}{0.000000in}}%
\pgfusepath{stroke,fill}%
}%
\begin{pgfscope}%
\pgfsys@transformshift{0.672222in}{4.299278in}%
\pgfsys@useobject{currentmarker}{}%
\end{pgfscope}%
\end{pgfscope}%
\begin{pgfscope}%
\definecolor{textcolor}{rgb}{0.000000,0.000000,0.000000}%
\pgfsetstrokecolor{textcolor}%
\pgfsetfillcolor{textcolor}%
\pgftext[x=0.149382in,y=4.246516in,left,base]{\color{textcolor}\sffamily\fontsize{10.000000}{12.000000}\selectfont −1.40}%
\end{pgfscope}%
\begin{pgfscope}%
\pgfsetbuttcap%
\pgfsetroundjoin%
\definecolor{currentfill}{rgb}{0.000000,0.000000,0.000000}%
\pgfsetfillcolor{currentfill}%
\pgfsetlinewidth{0.803000pt}%
\definecolor{currentstroke}{rgb}{0.000000,0.000000,0.000000}%
\pgfsetstrokecolor{currentstroke}%
\pgfsetdash{}{0pt}%
\pgfsys@defobject{currentmarker}{\pgfqpoint{-0.048611in}{0.000000in}}{\pgfqpoint{0.000000in}{0.000000in}}{%
\pgfpathmoveto{\pgfqpoint{0.000000in}{0.000000in}}%
\pgfpathlineto{\pgfqpoint{-0.048611in}{0.000000in}}%
\pgfusepath{stroke,fill}%
}%
\begin{pgfscope}%
\pgfsys@transformshift{0.672222in}{4.653073in}%
\pgfsys@useobject{currentmarker}{}%
\end{pgfscope}%
\end{pgfscope}%
\begin{pgfscope}%
\definecolor{textcolor}{rgb}{0.000000,0.000000,0.000000}%
\pgfsetstrokecolor{textcolor}%
\pgfsetfillcolor{textcolor}%
\pgftext[x=0.149382in,y=4.600312in,left,base]{\color{textcolor}\sffamily\fontsize{10.000000}{12.000000}\selectfont −1.35}%
\end{pgfscope}%
\begin{pgfscope}%
\pgfsetbuttcap%
\pgfsetroundjoin%
\definecolor{currentfill}{rgb}{0.000000,0.000000,0.000000}%
\pgfsetfillcolor{currentfill}%
\pgfsetlinewidth{0.803000pt}%
\definecolor{currentstroke}{rgb}{0.000000,0.000000,0.000000}%
\pgfsetstrokecolor{currentstroke}%
\pgfsetdash{}{0pt}%
\pgfsys@defobject{currentmarker}{\pgfqpoint{-0.048611in}{0.000000in}}{\pgfqpoint{0.000000in}{0.000000in}}{%
\pgfpathmoveto{\pgfqpoint{0.000000in}{0.000000in}}%
\pgfpathlineto{\pgfqpoint{-0.048611in}{0.000000in}}%
\pgfusepath{stroke,fill}%
}%
\begin{pgfscope}%
\pgfsys@transformshift{0.672222in}{5.006869in}%
\pgfsys@useobject{currentmarker}{}%
\end{pgfscope}%
\end{pgfscope}%
\begin{pgfscope}%
\definecolor{textcolor}{rgb}{0.000000,0.000000,0.000000}%
\pgfsetstrokecolor{textcolor}%
\pgfsetfillcolor{textcolor}%
\pgftext[x=0.149382in,y=4.954107in,left,base]{\color{textcolor}\sffamily\fontsize{10.000000}{12.000000}\selectfont −1.30}%
\end{pgfscope}%
\begin{pgfscope}%
\pgfsetbuttcap%
\pgfsetroundjoin%
\definecolor{currentfill}{rgb}{0.000000,0.000000,0.000000}%
\pgfsetfillcolor{currentfill}%
\pgfsetlinewidth{0.803000pt}%
\definecolor{currentstroke}{rgb}{0.000000,0.000000,0.000000}%
\pgfsetstrokecolor{currentstroke}%
\pgfsetdash{}{0pt}%
\pgfsys@defobject{currentmarker}{\pgfqpoint{-0.048611in}{0.000000in}}{\pgfqpoint{0.000000in}{0.000000in}}{%
\pgfpathmoveto{\pgfqpoint{0.000000in}{0.000000in}}%
\pgfpathlineto{\pgfqpoint{-0.048611in}{0.000000in}}%
\pgfusepath{stroke,fill}%
}%
\begin{pgfscope}%
\pgfsys@transformshift{0.672222in}{5.360664in}%
\pgfsys@useobject{currentmarker}{}%
\end{pgfscope}%
\end{pgfscope}%
\begin{pgfscope}%
\definecolor{textcolor}{rgb}{0.000000,0.000000,0.000000}%
\pgfsetstrokecolor{textcolor}%
\pgfsetfillcolor{textcolor}%
\pgftext[x=0.149382in,y=5.307902in,left,base]{\color{textcolor}\sffamily\fontsize{10.000000}{12.000000}\selectfont −1.25}%
\end{pgfscope}%
\begin{pgfscope}%
\pgfpathrectangle{\pgfqpoint{0.672222in}{3.312222in}}{\pgfqpoint{3.206333in}{2.324444in}}%
\pgfusepath{clip}%
\pgfsetrectcap%
\pgfsetroundjoin%
\pgfsetlinewidth{1.505625pt}%
\definecolor{currentstroke}{rgb}{0.121569,0.466667,0.705882}%
\pgfsetstrokecolor{currentstroke}%
\pgfsetdash{}{0pt}%
\pgfpathmoveto{\pgfqpoint{0.825965in}{3.425679in}}%
\pgfpathlineto{\pgfqpoint{0.970908in}{3.525273in}}%
\pgfpathlineto{\pgfqpoint{1.115850in}{3.626431in}}%
\pgfpathlineto{\pgfqpoint{1.260792in}{3.723543in}}%
\pgfpathlineto{\pgfqpoint{1.405735in}{3.830020in}}%
\pgfpathlineto{\pgfqpoint{1.550677in}{3.934098in}}%
\pgfpathlineto{\pgfqpoint{1.695619in}{4.040855in}}%
\pgfpathlineto{\pgfqpoint{1.840562in}{4.147217in}}%
\pgfpathlineto{\pgfqpoint{1.985504in}{4.256376in}}%
\pgfpathlineto{\pgfqpoint{2.130446in}{4.364962in}}%
\pgfpathlineto{\pgfqpoint{2.275389in}{4.477891in}}%
\pgfpathlineto{\pgfqpoint{2.420331in}{4.586210in}}%
\pgfpathlineto{\pgfqpoint{2.565273in}{4.695429in}}%
\pgfpathlineto{\pgfqpoint{2.710216in}{4.805124in}}%
\pgfpathlineto{\pgfqpoint{2.855158in}{4.916774in}}%
\pgfpathlineto{\pgfqpoint{3.000100in}{5.020608in}}%
\pgfpathlineto{\pgfqpoint{3.145043in}{5.121022in}}%
\pgfpathlineto{\pgfqpoint{3.289985in}{5.223971in}}%
\pgfpathlineto{\pgfqpoint{3.434927in}{5.326157in}}%
\pgfpathlineto{\pgfqpoint{3.579870in}{5.429920in}}%
\pgfpathlineto{\pgfqpoint{3.724812in}{5.523210in}}%
\pgfusepath{stroke}%
\end{pgfscope}%
\begin{pgfscope}%
\pgfsetrectcap%
\pgfsetmiterjoin%
\pgfsetlinewidth{0.803000pt}%
\definecolor{currentstroke}{rgb}{0.000000,0.000000,0.000000}%
\pgfsetstrokecolor{currentstroke}%
\pgfsetdash{}{0pt}%
\pgfpathmoveto{\pgfqpoint{0.672222in}{3.312222in}}%
\pgfpathlineto{\pgfqpoint{0.672222in}{5.636667in}}%
\pgfusepath{stroke}%
\end{pgfscope}%
\begin{pgfscope}%
\pgfsetrectcap%
\pgfsetmiterjoin%
\pgfsetlinewidth{0.803000pt}%
\definecolor{currentstroke}{rgb}{0.000000,0.000000,0.000000}%
\pgfsetstrokecolor{currentstroke}%
\pgfsetdash{}{0pt}%
\pgfpathmoveto{\pgfqpoint{3.878555in}{3.312222in}}%
\pgfpathlineto{\pgfqpoint{3.878555in}{5.636667in}}%
\pgfusepath{stroke}%
\end{pgfscope}%
\begin{pgfscope}%
\pgfsetrectcap%
\pgfsetmiterjoin%
\pgfsetlinewidth{0.803000pt}%
\definecolor{currentstroke}{rgb}{0.000000,0.000000,0.000000}%
\pgfsetstrokecolor{currentstroke}%
\pgfsetdash{}{0pt}%
\pgfpathmoveto{\pgfqpoint{0.672222in}{3.312222in}}%
\pgfpathlineto{\pgfqpoint{3.878555in}{3.312222in}}%
\pgfusepath{stroke}%
\end{pgfscope}%
\begin{pgfscope}%
\pgfsetrectcap%
\pgfsetmiterjoin%
\pgfsetlinewidth{0.803000pt}%
\definecolor{currentstroke}{rgb}{0.000000,0.000000,0.000000}%
\pgfsetstrokecolor{currentstroke}%
\pgfsetdash{}{0pt}%
\pgfpathmoveto{\pgfqpoint{0.672222in}{5.636667in}}%
\pgfpathlineto{\pgfqpoint{3.878555in}{5.636667in}}%
\pgfusepath{stroke}%
\end{pgfscope}%
\begin{pgfscope}%
\definecolor{textcolor}{rgb}{0.000000,0.000000,0.000000}%
\pgfsetstrokecolor{textcolor}%
\pgfsetfillcolor{textcolor}%
\pgftext[x=2.275389in,y=5.720000in,,base]{\color{textcolor}\sffamily\fontsize{12.000000}{14.400000}\selectfont \(\displaystyle  N = 16 \)}%
\end{pgfscope}%
\begin{pgfscope}%
\pgfsetbuttcap%
\pgfsetmiterjoin%
\definecolor{currentfill}{rgb}{1.000000,1.000000,1.000000}%
\pgfsetfillcolor{currentfill}%
\pgfsetlinewidth{0.000000pt}%
\definecolor{currentstroke}{rgb}{0.000000,0.000000,0.000000}%
\pgfsetstrokecolor{currentstroke}%
\pgfsetstrokeopacity{0.000000}%
\pgfsetdash{}{0pt}%
\pgfpathmoveto{\pgfqpoint{4.498277in}{3.312222in}}%
\pgfpathlineto{\pgfqpoint{7.704610in}{3.312222in}}%
\pgfpathlineto{\pgfqpoint{7.704610in}{5.636667in}}%
\pgfpathlineto{\pgfqpoint{4.498277in}{5.636667in}}%
\pgfpathclose%
\pgfusepath{fill}%
\end{pgfscope}%
\begin{pgfscope}%
\pgfpathrectangle{\pgfqpoint{4.498277in}{3.312222in}}{\pgfqpoint{3.206333in}{2.324444in}}%
\pgfusepath{clip}%
\pgfsetbuttcap%
\pgfsetroundjoin%
\definecolor{currentfill}{rgb}{0.121569,0.466667,0.705882}%
\pgfsetfillcolor{currentfill}%
\pgfsetfillopacity{0.300000}%
\pgfsetlinewidth{1.003750pt}%
\definecolor{currentstroke}{rgb}{0.121569,0.466667,0.705882}%
\pgfsetstrokecolor{currentstroke}%
\pgfsetstrokeopacity{0.300000}%
\pgfsetdash{}{0pt}%
\pgfpathmoveto{\pgfqpoint{4.652021in}{3.428967in}}%
\pgfpathlineto{\pgfqpoint{4.652021in}{3.422411in}}%
\pgfpathlineto{\pgfqpoint{4.796963in}{3.508390in}}%
\pgfpathlineto{\pgfqpoint{4.941905in}{3.591772in}}%
\pgfpathlineto{\pgfqpoint{5.086848in}{3.692650in}}%
\pgfpathlineto{\pgfqpoint{5.231790in}{3.781411in}}%
\pgfpathlineto{\pgfqpoint{5.376732in}{3.887659in}}%
\pgfpathlineto{\pgfqpoint{5.521675in}{3.981314in}}%
\pgfpathlineto{\pgfqpoint{5.666617in}{4.100098in}}%
\pgfpathlineto{\pgfqpoint{5.811559in}{4.216998in}}%
\pgfpathlineto{\pgfqpoint{5.956502in}{4.333315in}}%
\pgfpathlineto{\pgfqpoint{6.101444in}{4.456736in}}%
\pgfpathlineto{\pgfqpoint{6.246386in}{4.573169in}}%
\pgfpathlineto{\pgfqpoint{6.391329in}{4.681970in}}%
\pgfpathlineto{\pgfqpoint{6.536271in}{4.806768in}}%
\pgfpathlineto{\pgfqpoint{6.681213in}{4.919276in}}%
\pgfpathlineto{\pgfqpoint{6.826156in}{5.033130in}}%
\pgfpathlineto{\pgfqpoint{6.971098in}{5.136779in}}%
\pgfpathlineto{\pgfqpoint{7.116040in}{5.239402in}}%
\pgfpathlineto{\pgfqpoint{7.260983in}{5.331096in}}%
\pgfpathlineto{\pgfqpoint{7.405925in}{5.418278in}}%
\pgfpathlineto{\pgfqpoint{7.550867in}{5.509833in}}%
\pgfpathlineto{\pgfqpoint{7.550867in}{5.531010in}}%
\pgfpathlineto{\pgfqpoint{7.550867in}{5.531010in}}%
\pgfpathlineto{\pgfqpoint{7.405925in}{5.437855in}}%
\pgfpathlineto{\pgfqpoint{7.260983in}{5.341764in}}%
\pgfpathlineto{\pgfqpoint{7.116040in}{5.252471in}}%
\pgfpathlineto{\pgfqpoint{6.971098in}{5.151206in}}%
\pgfpathlineto{\pgfqpoint{6.826156in}{5.041437in}}%
\pgfpathlineto{\pgfqpoint{6.681213in}{4.936863in}}%
\pgfpathlineto{\pgfqpoint{6.536271in}{4.825202in}}%
\pgfpathlineto{\pgfqpoint{6.391329in}{4.710496in}}%
\pgfpathlineto{\pgfqpoint{6.246386in}{4.587040in}}%
\pgfpathlineto{\pgfqpoint{6.101444in}{4.460269in}}%
\pgfpathlineto{\pgfqpoint{5.956502in}{4.353380in}}%
\pgfpathlineto{\pgfqpoint{5.811559in}{4.239102in}}%
\pgfpathlineto{\pgfqpoint{5.666617in}{4.111007in}}%
\pgfpathlineto{\pgfqpoint{5.521675in}{4.007862in}}%
\pgfpathlineto{\pgfqpoint{5.376732in}{3.896711in}}%
\pgfpathlineto{\pgfqpoint{5.231790in}{3.797854in}}%
\pgfpathlineto{\pgfqpoint{5.086848in}{3.697774in}}%
\pgfpathlineto{\pgfqpoint{4.941905in}{3.610250in}}%
\pgfpathlineto{\pgfqpoint{4.796963in}{3.511780in}}%
\pgfpathlineto{\pgfqpoint{4.652021in}{3.428967in}}%
\pgfpathclose%
\pgfusepath{stroke,fill}%
\end{pgfscope}%
\begin{pgfscope}%
\pgfpathrectangle{\pgfqpoint{4.498277in}{3.312222in}}{\pgfqpoint{3.206333in}{2.324444in}}%
\pgfusepath{clip}%
\pgfsetbuttcap%
\pgfsetroundjoin%
\definecolor{currentfill}{rgb}{0.121569,0.466667,0.705882}%
\pgfsetfillcolor{currentfill}%
\pgfsetlinewidth{1.003750pt}%
\definecolor{currentstroke}{rgb}{0.121569,0.466667,0.705882}%
\pgfsetstrokecolor{currentstroke}%
\pgfsetdash{}{0pt}%
\pgfsys@defobject{currentmarker}{\pgfqpoint{-0.009821in}{-0.009821in}}{\pgfqpoint{0.009821in}{0.009821in}}{%
\pgfpathmoveto{\pgfqpoint{0.000000in}{-0.009821in}}%
\pgfpathcurveto{\pgfqpoint{0.002605in}{-0.009821in}}{\pgfqpoint{0.005103in}{-0.008786in}}{\pgfqpoint{0.006944in}{-0.006944in}}%
\pgfpathcurveto{\pgfqpoint{0.008786in}{-0.005103in}}{\pgfqpoint{0.009821in}{-0.002605in}}{\pgfqpoint{0.009821in}{0.000000in}}%
\pgfpathcurveto{\pgfqpoint{0.009821in}{0.002605in}}{\pgfqpoint{0.008786in}{0.005103in}}{\pgfqpoint{0.006944in}{0.006944in}}%
\pgfpathcurveto{\pgfqpoint{0.005103in}{0.008786in}}{\pgfqpoint{0.002605in}{0.009821in}}{\pgfqpoint{0.000000in}{0.009821in}}%
\pgfpathcurveto{\pgfqpoint{-0.002605in}{0.009821in}}{\pgfqpoint{-0.005103in}{0.008786in}}{\pgfqpoint{-0.006944in}{0.006944in}}%
\pgfpathcurveto{\pgfqpoint{-0.008786in}{0.005103in}}{\pgfqpoint{-0.009821in}{0.002605in}}{\pgfqpoint{-0.009821in}{0.000000in}}%
\pgfpathcurveto{\pgfqpoint{-0.009821in}{-0.002605in}}{\pgfqpoint{-0.008786in}{-0.005103in}}{\pgfqpoint{-0.006944in}{-0.006944in}}%
\pgfpathcurveto{\pgfqpoint{-0.005103in}{-0.008786in}}{\pgfqpoint{-0.002605in}{-0.009821in}}{\pgfqpoint{0.000000in}{-0.009821in}}%
\pgfpathclose%
\pgfusepath{stroke,fill}%
}%
\begin{pgfscope}%
\pgfsys@transformshift{4.652021in}{3.425689in}%
\pgfsys@useobject{currentmarker}{}%
\end{pgfscope}%
\begin{pgfscope}%
\pgfsys@transformshift{4.796963in}{3.510085in}%
\pgfsys@useobject{currentmarker}{}%
\end{pgfscope}%
\begin{pgfscope}%
\pgfsys@transformshift{4.941905in}{3.601011in}%
\pgfsys@useobject{currentmarker}{}%
\end{pgfscope}%
\begin{pgfscope}%
\pgfsys@transformshift{5.086848in}{3.695212in}%
\pgfsys@useobject{currentmarker}{}%
\end{pgfscope}%
\begin{pgfscope}%
\pgfsys@transformshift{5.231790in}{3.789633in}%
\pgfsys@useobject{currentmarker}{}%
\end{pgfscope}%
\begin{pgfscope}%
\pgfsys@transformshift{5.376732in}{3.892185in}%
\pgfsys@useobject{currentmarker}{}%
\end{pgfscope}%
\begin{pgfscope}%
\pgfsys@transformshift{5.521675in}{3.994588in}%
\pgfsys@useobject{currentmarker}{}%
\end{pgfscope}%
\begin{pgfscope}%
\pgfsys@transformshift{5.666617in}{4.105553in}%
\pgfsys@useobject{currentmarker}{}%
\end{pgfscope}%
\begin{pgfscope}%
\pgfsys@transformshift{5.811559in}{4.228050in}%
\pgfsys@useobject{currentmarker}{}%
\end{pgfscope}%
\begin{pgfscope}%
\pgfsys@transformshift{5.956502in}{4.343348in}%
\pgfsys@useobject{currentmarker}{}%
\end{pgfscope}%
\begin{pgfscope}%
\pgfsys@transformshift{6.101444in}{4.458502in}%
\pgfsys@useobject{currentmarker}{}%
\end{pgfscope}%
\begin{pgfscope}%
\pgfsys@transformshift{6.246386in}{4.580105in}%
\pgfsys@useobject{currentmarker}{}%
\end{pgfscope}%
\begin{pgfscope}%
\pgfsys@transformshift{6.391329in}{4.696233in}%
\pgfsys@useobject{currentmarker}{}%
\end{pgfscope}%
\begin{pgfscope}%
\pgfsys@transformshift{6.536271in}{4.815985in}%
\pgfsys@useobject{currentmarker}{}%
\end{pgfscope}%
\begin{pgfscope}%
\pgfsys@transformshift{6.681213in}{4.928069in}%
\pgfsys@useobject{currentmarker}{}%
\end{pgfscope}%
\begin{pgfscope}%
\pgfsys@transformshift{6.826156in}{5.037284in}%
\pgfsys@useobject{currentmarker}{}%
\end{pgfscope}%
\begin{pgfscope}%
\pgfsys@transformshift{6.971098in}{5.143992in}%
\pgfsys@useobject{currentmarker}{}%
\end{pgfscope}%
\begin{pgfscope}%
\pgfsys@transformshift{7.116040in}{5.245937in}%
\pgfsys@useobject{currentmarker}{}%
\end{pgfscope}%
\begin{pgfscope}%
\pgfsys@transformshift{7.260983in}{5.336430in}%
\pgfsys@useobject{currentmarker}{}%
\end{pgfscope}%
\begin{pgfscope}%
\pgfsys@transformshift{7.405925in}{5.428066in}%
\pgfsys@useobject{currentmarker}{}%
\end{pgfscope}%
\begin{pgfscope}%
\pgfsys@transformshift{7.550867in}{5.520421in}%
\pgfsys@useobject{currentmarker}{}%
\end{pgfscope}%
\end{pgfscope}%
\begin{pgfscope}%
\pgfsetbuttcap%
\pgfsetroundjoin%
\definecolor{currentfill}{rgb}{0.000000,0.000000,0.000000}%
\pgfsetfillcolor{currentfill}%
\pgfsetlinewidth{0.803000pt}%
\definecolor{currentstroke}{rgb}{0.000000,0.000000,0.000000}%
\pgfsetstrokecolor{currentstroke}%
\pgfsetdash{}{0pt}%
\pgfsys@defobject{currentmarker}{\pgfqpoint{0.000000in}{-0.048611in}}{\pgfqpoint{0.000000in}{0.000000in}}{%
\pgfpathmoveto{\pgfqpoint{0.000000in}{0.000000in}}%
\pgfpathlineto{\pgfqpoint{0.000000in}{-0.048611in}}%
\pgfusepath{stroke,fill}%
}%
\begin{pgfscope}%
\pgfsys@transformshift{4.796963in}{3.312222in}%
\pgfsys@useobject{currentmarker}{}%
\end{pgfscope}%
\end{pgfscope}%
\begin{pgfscope}%
\definecolor{textcolor}{rgb}{0.000000,0.000000,0.000000}%
\pgfsetstrokecolor{textcolor}%
\pgfsetfillcolor{textcolor}%
\pgftext[x=4.796963in,y=3.215000in,,top]{\color{textcolor}\sffamily\fontsize{10.000000}{12.000000}\selectfont 2.20}%
\end{pgfscope}%
\begin{pgfscope}%
\pgfsetbuttcap%
\pgfsetroundjoin%
\definecolor{currentfill}{rgb}{0.000000,0.000000,0.000000}%
\pgfsetfillcolor{currentfill}%
\pgfsetlinewidth{0.803000pt}%
\definecolor{currentstroke}{rgb}{0.000000,0.000000,0.000000}%
\pgfsetstrokecolor{currentstroke}%
\pgfsetdash{}{0pt}%
\pgfsys@defobject{currentmarker}{\pgfqpoint{0.000000in}{-0.048611in}}{\pgfqpoint{0.000000in}{0.000000in}}{%
\pgfpathmoveto{\pgfqpoint{0.000000in}{0.000000in}}%
\pgfpathlineto{\pgfqpoint{0.000000in}{-0.048611in}}%
\pgfusepath{stroke,fill}%
}%
\begin{pgfscope}%
\pgfsys@transformshift{5.521675in}{3.312222in}%
\pgfsys@useobject{currentmarker}{}%
\end{pgfscope}%
\end{pgfscope}%
\begin{pgfscope}%
\definecolor{textcolor}{rgb}{0.000000,0.000000,0.000000}%
\pgfsetstrokecolor{textcolor}%
\pgfsetfillcolor{textcolor}%
\pgftext[x=5.521675in,y=3.215000in,,top]{\color{textcolor}\sffamily\fontsize{10.000000}{12.000000}\selectfont 2.25}%
\end{pgfscope}%
\begin{pgfscope}%
\pgfsetbuttcap%
\pgfsetroundjoin%
\definecolor{currentfill}{rgb}{0.000000,0.000000,0.000000}%
\pgfsetfillcolor{currentfill}%
\pgfsetlinewidth{0.803000pt}%
\definecolor{currentstroke}{rgb}{0.000000,0.000000,0.000000}%
\pgfsetstrokecolor{currentstroke}%
\pgfsetdash{}{0pt}%
\pgfsys@defobject{currentmarker}{\pgfqpoint{0.000000in}{-0.048611in}}{\pgfqpoint{0.000000in}{0.000000in}}{%
\pgfpathmoveto{\pgfqpoint{0.000000in}{0.000000in}}%
\pgfpathlineto{\pgfqpoint{0.000000in}{-0.048611in}}%
\pgfusepath{stroke,fill}%
}%
\begin{pgfscope}%
\pgfsys@transformshift{6.246386in}{3.312222in}%
\pgfsys@useobject{currentmarker}{}%
\end{pgfscope}%
\end{pgfscope}%
\begin{pgfscope}%
\definecolor{textcolor}{rgb}{0.000000,0.000000,0.000000}%
\pgfsetstrokecolor{textcolor}%
\pgfsetfillcolor{textcolor}%
\pgftext[x=6.246386in,y=3.215000in,,top]{\color{textcolor}\sffamily\fontsize{10.000000}{12.000000}\selectfont 2.30}%
\end{pgfscope}%
\begin{pgfscope}%
\pgfsetbuttcap%
\pgfsetroundjoin%
\definecolor{currentfill}{rgb}{0.000000,0.000000,0.000000}%
\pgfsetfillcolor{currentfill}%
\pgfsetlinewidth{0.803000pt}%
\definecolor{currentstroke}{rgb}{0.000000,0.000000,0.000000}%
\pgfsetstrokecolor{currentstroke}%
\pgfsetdash{}{0pt}%
\pgfsys@defobject{currentmarker}{\pgfqpoint{0.000000in}{-0.048611in}}{\pgfqpoint{0.000000in}{0.000000in}}{%
\pgfpathmoveto{\pgfqpoint{0.000000in}{0.000000in}}%
\pgfpathlineto{\pgfqpoint{0.000000in}{-0.048611in}}%
\pgfusepath{stroke,fill}%
}%
\begin{pgfscope}%
\pgfsys@transformshift{6.971098in}{3.312222in}%
\pgfsys@useobject{currentmarker}{}%
\end{pgfscope}%
\end{pgfscope}%
\begin{pgfscope}%
\definecolor{textcolor}{rgb}{0.000000,0.000000,0.000000}%
\pgfsetstrokecolor{textcolor}%
\pgfsetfillcolor{textcolor}%
\pgftext[x=6.971098in,y=3.215000in,,top]{\color{textcolor}\sffamily\fontsize{10.000000}{12.000000}\selectfont 2.35}%
\end{pgfscope}%
\begin{pgfscope}%
\pgfsetbuttcap%
\pgfsetroundjoin%
\definecolor{currentfill}{rgb}{0.000000,0.000000,0.000000}%
\pgfsetfillcolor{currentfill}%
\pgfsetlinewidth{0.803000pt}%
\definecolor{currentstroke}{rgb}{0.000000,0.000000,0.000000}%
\pgfsetstrokecolor{currentstroke}%
\pgfsetdash{}{0pt}%
\pgfsys@defobject{currentmarker}{\pgfqpoint{0.000000in}{-0.048611in}}{\pgfqpoint{0.000000in}{0.000000in}}{%
\pgfpathmoveto{\pgfqpoint{0.000000in}{0.000000in}}%
\pgfpathlineto{\pgfqpoint{0.000000in}{-0.048611in}}%
\pgfusepath{stroke,fill}%
}%
\begin{pgfscope}%
\pgfsys@transformshift{7.695810in}{3.312222in}%
\pgfsys@useobject{currentmarker}{}%
\end{pgfscope}%
\end{pgfscope}%
\begin{pgfscope}%
\definecolor{textcolor}{rgb}{0.000000,0.000000,0.000000}%
\pgfsetstrokecolor{textcolor}%
\pgfsetfillcolor{textcolor}%
\pgftext[x=7.695810in,y=3.215000in,,top]{\color{textcolor}\sffamily\fontsize{10.000000}{12.000000}\selectfont 2.40}%
\end{pgfscope}%
\begin{pgfscope}%
\pgfsetbuttcap%
\pgfsetroundjoin%
\definecolor{currentfill}{rgb}{0.000000,0.000000,0.000000}%
\pgfsetfillcolor{currentfill}%
\pgfsetlinewidth{0.803000pt}%
\definecolor{currentstroke}{rgb}{0.000000,0.000000,0.000000}%
\pgfsetstrokecolor{currentstroke}%
\pgfsetdash{}{0pt}%
\pgfsys@defobject{currentmarker}{\pgfqpoint{-0.048611in}{0.000000in}}{\pgfqpoint{0.000000in}{0.000000in}}{%
\pgfpathmoveto{\pgfqpoint{0.000000in}{0.000000in}}%
\pgfpathlineto{\pgfqpoint{-0.048611in}{0.000000in}}%
\pgfusepath{stroke,fill}%
}%
\begin{pgfscope}%
\pgfsys@transformshift{4.498277in}{3.489226in}%
\pgfsys@useobject{currentmarker}{}%
\end{pgfscope}%
\end{pgfscope}%
\begin{pgfscope}%
\definecolor{textcolor}{rgb}{0.000000,0.000000,0.000000}%
\pgfsetstrokecolor{textcolor}%
\pgfsetfillcolor{textcolor}%
\pgftext[x=3.975437in,y=3.436464in,left,base]{\color{textcolor}\sffamily\fontsize{10.000000}{12.000000}\selectfont −1.55}%
\end{pgfscope}%
\begin{pgfscope}%
\pgfsetbuttcap%
\pgfsetroundjoin%
\definecolor{currentfill}{rgb}{0.000000,0.000000,0.000000}%
\pgfsetfillcolor{currentfill}%
\pgfsetlinewidth{0.803000pt}%
\definecolor{currentstroke}{rgb}{0.000000,0.000000,0.000000}%
\pgfsetstrokecolor{currentstroke}%
\pgfsetdash{}{0pt}%
\pgfsys@defobject{currentmarker}{\pgfqpoint{-0.048611in}{0.000000in}}{\pgfqpoint{0.000000in}{0.000000in}}{%
\pgfpathmoveto{\pgfqpoint{0.000000in}{0.000000in}}%
\pgfpathlineto{\pgfqpoint{-0.048611in}{0.000000in}}%
\pgfusepath{stroke,fill}%
}%
\begin{pgfscope}%
\pgfsys@transformshift{4.498277in}{3.802066in}%
\pgfsys@useobject{currentmarker}{}%
\end{pgfscope}%
\end{pgfscope}%
\begin{pgfscope}%
\definecolor{textcolor}{rgb}{0.000000,0.000000,0.000000}%
\pgfsetstrokecolor{textcolor}%
\pgfsetfillcolor{textcolor}%
\pgftext[x=3.975437in,y=3.749305in,left,base]{\color{textcolor}\sffamily\fontsize{10.000000}{12.000000}\selectfont −1.50}%
\end{pgfscope}%
\begin{pgfscope}%
\pgfsetbuttcap%
\pgfsetroundjoin%
\definecolor{currentfill}{rgb}{0.000000,0.000000,0.000000}%
\pgfsetfillcolor{currentfill}%
\pgfsetlinewidth{0.803000pt}%
\definecolor{currentstroke}{rgb}{0.000000,0.000000,0.000000}%
\pgfsetstrokecolor{currentstroke}%
\pgfsetdash{}{0pt}%
\pgfsys@defobject{currentmarker}{\pgfqpoint{-0.048611in}{0.000000in}}{\pgfqpoint{0.000000in}{0.000000in}}{%
\pgfpathmoveto{\pgfqpoint{0.000000in}{0.000000in}}%
\pgfpathlineto{\pgfqpoint{-0.048611in}{0.000000in}}%
\pgfusepath{stroke,fill}%
}%
\begin{pgfscope}%
\pgfsys@transformshift{4.498277in}{4.114907in}%
\pgfsys@useobject{currentmarker}{}%
\end{pgfscope}%
\end{pgfscope}%
\begin{pgfscope}%
\definecolor{textcolor}{rgb}{0.000000,0.000000,0.000000}%
\pgfsetstrokecolor{textcolor}%
\pgfsetfillcolor{textcolor}%
\pgftext[x=3.975437in,y=4.062145in,left,base]{\color{textcolor}\sffamily\fontsize{10.000000}{12.000000}\selectfont −1.45}%
\end{pgfscope}%
\begin{pgfscope}%
\pgfsetbuttcap%
\pgfsetroundjoin%
\definecolor{currentfill}{rgb}{0.000000,0.000000,0.000000}%
\pgfsetfillcolor{currentfill}%
\pgfsetlinewidth{0.803000pt}%
\definecolor{currentstroke}{rgb}{0.000000,0.000000,0.000000}%
\pgfsetstrokecolor{currentstroke}%
\pgfsetdash{}{0pt}%
\pgfsys@defobject{currentmarker}{\pgfqpoint{-0.048611in}{0.000000in}}{\pgfqpoint{0.000000in}{0.000000in}}{%
\pgfpathmoveto{\pgfqpoint{0.000000in}{0.000000in}}%
\pgfpathlineto{\pgfqpoint{-0.048611in}{0.000000in}}%
\pgfusepath{stroke,fill}%
}%
\begin{pgfscope}%
\pgfsys@transformshift{4.498277in}{4.427747in}%
\pgfsys@useobject{currentmarker}{}%
\end{pgfscope}%
\end{pgfscope}%
\begin{pgfscope}%
\definecolor{textcolor}{rgb}{0.000000,0.000000,0.000000}%
\pgfsetstrokecolor{textcolor}%
\pgfsetfillcolor{textcolor}%
\pgftext[x=3.975437in,y=4.374986in,left,base]{\color{textcolor}\sffamily\fontsize{10.000000}{12.000000}\selectfont −1.40}%
\end{pgfscope}%
\begin{pgfscope}%
\pgfsetbuttcap%
\pgfsetroundjoin%
\definecolor{currentfill}{rgb}{0.000000,0.000000,0.000000}%
\pgfsetfillcolor{currentfill}%
\pgfsetlinewidth{0.803000pt}%
\definecolor{currentstroke}{rgb}{0.000000,0.000000,0.000000}%
\pgfsetstrokecolor{currentstroke}%
\pgfsetdash{}{0pt}%
\pgfsys@defobject{currentmarker}{\pgfqpoint{-0.048611in}{0.000000in}}{\pgfqpoint{0.000000in}{0.000000in}}{%
\pgfpathmoveto{\pgfqpoint{0.000000in}{0.000000in}}%
\pgfpathlineto{\pgfqpoint{-0.048611in}{0.000000in}}%
\pgfusepath{stroke,fill}%
}%
\begin{pgfscope}%
\pgfsys@transformshift{4.498277in}{4.740588in}%
\pgfsys@useobject{currentmarker}{}%
\end{pgfscope}%
\end{pgfscope}%
\begin{pgfscope}%
\definecolor{textcolor}{rgb}{0.000000,0.000000,0.000000}%
\pgfsetstrokecolor{textcolor}%
\pgfsetfillcolor{textcolor}%
\pgftext[x=3.975437in,y=4.687826in,left,base]{\color{textcolor}\sffamily\fontsize{10.000000}{12.000000}\selectfont −1.35}%
\end{pgfscope}%
\begin{pgfscope}%
\pgfsetbuttcap%
\pgfsetroundjoin%
\definecolor{currentfill}{rgb}{0.000000,0.000000,0.000000}%
\pgfsetfillcolor{currentfill}%
\pgfsetlinewidth{0.803000pt}%
\definecolor{currentstroke}{rgb}{0.000000,0.000000,0.000000}%
\pgfsetstrokecolor{currentstroke}%
\pgfsetdash{}{0pt}%
\pgfsys@defobject{currentmarker}{\pgfqpoint{-0.048611in}{0.000000in}}{\pgfqpoint{0.000000in}{0.000000in}}{%
\pgfpathmoveto{\pgfqpoint{0.000000in}{0.000000in}}%
\pgfpathlineto{\pgfqpoint{-0.048611in}{0.000000in}}%
\pgfusepath{stroke,fill}%
}%
\begin{pgfscope}%
\pgfsys@transformshift{4.498277in}{5.053428in}%
\pgfsys@useobject{currentmarker}{}%
\end{pgfscope}%
\end{pgfscope}%
\begin{pgfscope}%
\definecolor{textcolor}{rgb}{0.000000,0.000000,0.000000}%
\pgfsetstrokecolor{textcolor}%
\pgfsetfillcolor{textcolor}%
\pgftext[x=3.975437in,y=5.000666in,left,base]{\color{textcolor}\sffamily\fontsize{10.000000}{12.000000}\selectfont −1.30}%
\end{pgfscope}%
\begin{pgfscope}%
\pgfsetbuttcap%
\pgfsetroundjoin%
\definecolor{currentfill}{rgb}{0.000000,0.000000,0.000000}%
\pgfsetfillcolor{currentfill}%
\pgfsetlinewidth{0.803000pt}%
\definecolor{currentstroke}{rgb}{0.000000,0.000000,0.000000}%
\pgfsetstrokecolor{currentstroke}%
\pgfsetdash{}{0pt}%
\pgfsys@defobject{currentmarker}{\pgfqpoint{-0.048611in}{0.000000in}}{\pgfqpoint{0.000000in}{0.000000in}}{%
\pgfpathmoveto{\pgfqpoint{0.000000in}{0.000000in}}%
\pgfpathlineto{\pgfqpoint{-0.048611in}{0.000000in}}%
\pgfusepath{stroke,fill}%
}%
\begin{pgfscope}%
\pgfsys@transformshift{4.498277in}{5.366268in}%
\pgfsys@useobject{currentmarker}{}%
\end{pgfscope}%
\end{pgfscope}%
\begin{pgfscope}%
\definecolor{textcolor}{rgb}{0.000000,0.000000,0.000000}%
\pgfsetstrokecolor{textcolor}%
\pgfsetfillcolor{textcolor}%
\pgftext[x=3.975437in,y=5.313507in,left,base]{\color{textcolor}\sffamily\fontsize{10.000000}{12.000000}\selectfont −1.25}%
\end{pgfscope}%
\begin{pgfscope}%
\pgfpathrectangle{\pgfqpoint{4.498277in}{3.312222in}}{\pgfqpoint{3.206333in}{2.324444in}}%
\pgfusepath{clip}%
\pgfsetrectcap%
\pgfsetroundjoin%
\pgfsetlinewidth{1.505625pt}%
\definecolor{currentstroke}{rgb}{0.121569,0.466667,0.705882}%
\pgfsetstrokecolor{currentstroke}%
\pgfsetdash{}{0pt}%
\pgfpathmoveto{\pgfqpoint{4.652021in}{3.425689in}}%
\pgfpathlineto{\pgfqpoint{4.796963in}{3.510085in}}%
\pgfpathlineto{\pgfqpoint{4.941905in}{3.601011in}}%
\pgfpathlineto{\pgfqpoint{5.086848in}{3.695212in}}%
\pgfpathlineto{\pgfqpoint{5.231790in}{3.789633in}}%
\pgfpathlineto{\pgfqpoint{5.376732in}{3.892185in}}%
\pgfpathlineto{\pgfqpoint{5.521675in}{3.994588in}}%
\pgfpathlineto{\pgfqpoint{5.666617in}{4.105553in}}%
\pgfpathlineto{\pgfqpoint{5.811559in}{4.228050in}}%
\pgfpathlineto{\pgfqpoint{5.956502in}{4.343348in}}%
\pgfpathlineto{\pgfqpoint{6.101444in}{4.458502in}}%
\pgfpathlineto{\pgfqpoint{6.246386in}{4.580105in}}%
\pgfpathlineto{\pgfqpoint{6.391329in}{4.696233in}}%
\pgfpathlineto{\pgfqpoint{6.536271in}{4.815985in}}%
\pgfpathlineto{\pgfqpoint{6.681213in}{4.928069in}}%
\pgfpathlineto{\pgfqpoint{6.826156in}{5.037284in}}%
\pgfpathlineto{\pgfqpoint{6.971098in}{5.143992in}}%
\pgfpathlineto{\pgfqpoint{7.116040in}{5.245937in}}%
\pgfpathlineto{\pgfqpoint{7.260983in}{5.336430in}}%
\pgfpathlineto{\pgfqpoint{7.405925in}{5.428066in}}%
\pgfpathlineto{\pgfqpoint{7.550867in}{5.520421in}}%
\pgfusepath{stroke}%
\end{pgfscope}%
\begin{pgfscope}%
\pgfsetrectcap%
\pgfsetmiterjoin%
\pgfsetlinewidth{0.803000pt}%
\definecolor{currentstroke}{rgb}{0.000000,0.000000,0.000000}%
\pgfsetstrokecolor{currentstroke}%
\pgfsetdash{}{0pt}%
\pgfpathmoveto{\pgfqpoint{4.498277in}{3.312222in}}%
\pgfpathlineto{\pgfqpoint{4.498277in}{5.636667in}}%
\pgfusepath{stroke}%
\end{pgfscope}%
\begin{pgfscope}%
\pgfsetrectcap%
\pgfsetmiterjoin%
\pgfsetlinewidth{0.803000pt}%
\definecolor{currentstroke}{rgb}{0.000000,0.000000,0.000000}%
\pgfsetstrokecolor{currentstroke}%
\pgfsetdash{}{0pt}%
\pgfpathmoveto{\pgfqpoint{7.704610in}{3.312222in}}%
\pgfpathlineto{\pgfqpoint{7.704610in}{5.636667in}}%
\pgfusepath{stroke}%
\end{pgfscope}%
\begin{pgfscope}%
\pgfsetrectcap%
\pgfsetmiterjoin%
\pgfsetlinewidth{0.803000pt}%
\definecolor{currentstroke}{rgb}{0.000000,0.000000,0.000000}%
\pgfsetstrokecolor{currentstroke}%
\pgfsetdash{}{0pt}%
\pgfpathmoveto{\pgfqpoint{4.498277in}{3.312222in}}%
\pgfpathlineto{\pgfqpoint{7.704610in}{3.312222in}}%
\pgfusepath{stroke}%
\end{pgfscope}%
\begin{pgfscope}%
\pgfsetrectcap%
\pgfsetmiterjoin%
\pgfsetlinewidth{0.803000pt}%
\definecolor{currentstroke}{rgb}{0.000000,0.000000,0.000000}%
\pgfsetstrokecolor{currentstroke}%
\pgfsetdash{}{0pt}%
\pgfpathmoveto{\pgfqpoint{4.498277in}{5.636667in}}%
\pgfpathlineto{\pgfqpoint{7.704610in}{5.636667in}}%
\pgfusepath{stroke}%
\end{pgfscope}%
\begin{pgfscope}%
\definecolor{textcolor}{rgb}{0.000000,0.000000,0.000000}%
\pgfsetstrokecolor{textcolor}%
\pgfsetfillcolor{textcolor}%
\pgftext[x=6.101444in,y=5.720000in,,base]{\color{textcolor}\sffamily\fontsize{12.000000}{14.400000}\selectfont \(\displaystyle  N = 32 \)}%
\end{pgfscope}%
\begin{pgfscope}%
\pgfsetbuttcap%
\pgfsetmiterjoin%
\definecolor{currentfill}{rgb}{1.000000,1.000000,1.000000}%
\pgfsetfillcolor{currentfill}%
\pgfsetlinewidth{0.000000pt}%
\definecolor{currentstroke}{rgb}{0.000000,0.000000,0.000000}%
\pgfsetstrokecolor{currentstroke}%
\pgfsetstrokeopacity{0.000000}%
\pgfsetdash{}{0pt}%
\pgfpathmoveto{\pgfqpoint{0.672222in}{0.387222in}}%
\pgfpathlineto{\pgfqpoint{3.878555in}{0.387222in}}%
\pgfpathlineto{\pgfqpoint{3.878555in}{2.711667in}}%
\pgfpathlineto{\pgfqpoint{0.672222in}{2.711667in}}%
\pgfpathclose%
\pgfusepath{fill}%
\end{pgfscope}%
\begin{pgfscope}%
\pgfpathrectangle{\pgfqpoint{0.672222in}{0.387222in}}{\pgfqpoint{3.206333in}{2.324444in}}%
\pgfusepath{clip}%
\pgfsetbuttcap%
\pgfsetroundjoin%
\definecolor{currentfill}{rgb}{0.121569,0.466667,0.705882}%
\pgfsetfillcolor{currentfill}%
\pgfsetfillopacity{0.300000}%
\pgfsetlinewidth{1.003750pt}%
\definecolor{currentstroke}{rgb}{0.121569,0.466667,0.705882}%
\pgfsetstrokecolor{currentstroke}%
\pgfsetstrokeopacity{0.300000}%
\pgfsetdash{}{0pt}%
\pgfpathmoveto{\pgfqpoint{0.825965in}{0.505476in}}%
\pgfpathlineto{\pgfqpoint{0.825965in}{0.495880in}}%
\pgfpathlineto{\pgfqpoint{0.970908in}{0.569556in}}%
\pgfpathlineto{\pgfqpoint{1.115850in}{0.660586in}}%
\pgfpathlineto{\pgfqpoint{1.260792in}{0.745010in}}%
\pgfpathlineto{\pgfqpoint{1.405735in}{0.842168in}}%
\pgfpathlineto{\pgfqpoint{1.550677in}{0.937918in}}%
\pgfpathlineto{\pgfqpoint{1.695619in}{1.050686in}}%
\pgfpathlineto{\pgfqpoint{1.840562in}{1.149393in}}%
\pgfpathlineto{\pgfqpoint{1.985504in}{1.267662in}}%
\pgfpathlineto{\pgfqpoint{2.130446in}{1.400210in}}%
\pgfpathlineto{\pgfqpoint{2.275389in}{1.549681in}}%
\pgfpathlineto{\pgfqpoint{2.420331in}{1.659394in}}%
\pgfpathlineto{\pgfqpoint{2.565273in}{1.796818in}}%
\pgfpathlineto{\pgfqpoint{2.710216in}{1.947079in}}%
\pgfpathlineto{\pgfqpoint{2.855158in}{2.054312in}}%
\pgfpathlineto{\pgfqpoint{3.000100in}{2.158329in}}%
\pgfpathlineto{\pgfqpoint{3.145043in}{2.261245in}}%
\pgfpathlineto{\pgfqpoint{3.289985in}{2.357611in}}%
\pgfpathlineto{\pgfqpoint{3.434927in}{2.421620in}}%
\pgfpathlineto{\pgfqpoint{3.579870in}{2.522481in}}%
\pgfpathlineto{\pgfqpoint{3.724812in}{2.594947in}}%
\pgfpathlineto{\pgfqpoint{3.724812in}{2.601475in}}%
\pgfpathlineto{\pgfqpoint{3.724812in}{2.601475in}}%
\pgfpathlineto{\pgfqpoint{3.579870in}{2.526823in}}%
\pgfpathlineto{\pgfqpoint{3.434927in}{2.454432in}}%
\pgfpathlineto{\pgfqpoint{3.289985in}{2.366621in}}%
\pgfpathlineto{\pgfqpoint{3.145043in}{2.275630in}}%
\pgfpathlineto{\pgfqpoint{3.000100in}{2.186515in}}%
\pgfpathlineto{\pgfqpoint{2.855158in}{2.090767in}}%
\pgfpathlineto{\pgfqpoint{2.710216in}{1.966458in}}%
\pgfpathlineto{\pgfqpoint{2.565273in}{1.840076in}}%
\pgfpathlineto{\pgfqpoint{2.420331in}{1.722730in}}%
\pgfpathlineto{\pgfqpoint{2.275389in}{1.572819in}}%
\pgfpathlineto{\pgfqpoint{2.130446in}{1.466716in}}%
\pgfpathlineto{\pgfqpoint{1.985504in}{1.296876in}}%
\pgfpathlineto{\pgfqpoint{1.840562in}{1.185250in}}%
\pgfpathlineto{\pgfqpoint{1.695619in}{1.057838in}}%
\pgfpathlineto{\pgfqpoint{1.550677in}{0.957983in}}%
\pgfpathlineto{\pgfqpoint{1.405735in}{0.848877in}}%
\pgfpathlineto{\pgfqpoint{1.260792in}{0.755233in}}%
\pgfpathlineto{\pgfqpoint{1.115850in}{0.672113in}}%
\pgfpathlineto{\pgfqpoint{0.970908in}{0.584979in}}%
\pgfpathlineto{\pgfqpoint{0.825965in}{0.505476in}}%
\pgfpathclose%
\pgfusepath{stroke,fill}%
\end{pgfscope}%
\begin{pgfscope}%
\pgfpathrectangle{\pgfqpoint{0.672222in}{0.387222in}}{\pgfqpoint{3.206333in}{2.324444in}}%
\pgfusepath{clip}%
\pgfsetbuttcap%
\pgfsetroundjoin%
\definecolor{currentfill}{rgb}{0.121569,0.466667,0.705882}%
\pgfsetfillcolor{currentfill}%
\pgfsetlinewidth{1.003750pt}%
\definecolor{currentstroke}{rgb}{0.121569,0.466667,0.705882}%
\pgfsetstrokecolor{currentstroke}%
\pgfsetdash{}{0pt}%
\pgfsys@defobject{currentmarker}{\pgfqpoint{-0.009821in}{-0.009821in}}{\pgfqpoint{0.009821in}{0.009821in}}{%
\pgfpathmoveto{\pgfqpoint{0.000000in}{-0.009821in}}%
\pgfpathcurveto{\pgfqpoint{0.002605in}{-0.009821in}}{\pgfqpoint{0.005103in}{-0.008786in}}{\pgfqpoint{0.006944in}{-0.006944in}}%
\pgfpathcurveto{\pgfqpoint{0.008786in}{-0.005103in}}{\pgfqpoint{0.009821in}{-0.002605in}}{\pgfqpoint{0.009821in}{0.000000in}}%
\pgfpathcurveto{\pgfqpoint{0.009821in}{0.002605in}}{\pgfqpoint{0.008786in}{0.005103in}}{\pgfqpoint{0.006944in}{0.006944in}}%
\pgfpathcurveto{\pgfqpoint{0.005103in}{0.008786in}}{\pgfqpoint{0.002605in}{0.009821in}}{\pgfqpoint{0.000000in}{0.009821in}}%
\pgfpathcurveto{\pgfqpoint{-0.002605in}{0.009821in}}{\pgfqpoint{-0.005103in}{0.008786in}}{\pgfqpoint{-0.006944in}{0.006944in}}%
\pgfpathcurveto{\pgfqpoint{-0.008786in}{0.005103in}}{\pgfqpoint{-0.009821in}{0.002605in}}{\pgfqpoint{-0.009821in}{0.000000in}}%
\pgfpathcurveto{\pgfqpoint{-0.009821in}{-0.002605in}}{\pgfqpoint{-0.008786in}{-0.005103in}}{\pgfqpoint{-0.006944in}{-0.006944in}}%
\pgfpathcurveto{\pgfqpoint{-0.005103in}{-0.008786in}}{\pgfqpoint{-0.002605in}{-0.009821in}}{\pgfqpoint{0.000000in}{-0.009821in}}%
\pgfpathclose%
\pgfusepath{stroke,fill}%
}%
\begin{pgfscope}%
\pgfsys@transformshift{0.825965in}{0.500678in}%
\pgfsys@useobject{currentmarker}{}%
\end{pgfscope}%
\begin{pgfscope}%
\pgfsys@transformshift{0.970908in}{0.577267in}%
\pgfsys@useobject{currentmarker}{}%
\end{pgfscope}%
\begin{pgfscope}%
\pgfsys@transformshift{1.115850in}{0.666349in}%
\pgfsys@useobject{currentmarker}{}%
\end{pgfscope}%
\begin{pgfscope}%
\pgfsys@transformshift{1.260792in}{0.750122in}%
\pgfsys@useobject{currentmarker}{}%
\end{pgfscope}%
\begin{pgfscope}%
\pgfsys@transformshift{1.405735in}{0.845522in}%
\pgfsys@useobject{currentmarker}{}%
\end{pgfscope}%
\begin{pgfscope}%
\pgfsys@transformshift{1.550677in}{0.947950in}%
\pgfsys@useobject{currentmarker}{}%
\end{pgfscope}%
\begin{pgfscope}%
\pgfsys@transformshift{1.695619in}{1.054262in}%
\pgfsys@useobject{currentmarker}{}%
\end{pgfscope}%
\begin{pgfscope}%
\pgfsys@transformshift{1.840562in}{1.167321in}%
\pgfsys@useobject{currentmarker}{}%
\end{pgfscope}%
\begin{pgfscope}%
\pgfsys@transformshift{1.985504in}{1.282269in}%
\pgfsys@useobject{currentmarker}{}%
\end{pgfscope}%
\begin{pgfscope}%
\pgfsys@transformshift{2.130446in}{1.433463in}%
\pgfsys@useobject{currentmarker}{}%
\end{pgfscope}%
\begin{pgfscope}%
\pgfsys@transformshift{2.275389in}{1.561250in}%
\pgfsys@useobject{currentmarker}{}%
\end{pgfscope}%
\begin{pgfscope}%
\pgfsys@transformshift{2.420331in}{1.691062in}%
\pgfsys@useobject{currentmarker}{}%
\end{pgfscope}%
\begin{pgfscope}%
\pgfsys@transformshift{2.565273in}{1.818447in}%
\pgfsys@useobject{currentmarker}{}%
\end{pgfscope}%
\begin{pgfscope}%
\pgfsys@transformshift{2.710216in}{1.956768in}%
\pgfsys@useobject{currentmarker}{}%
\end{pgfscope}%
\begin{pgfscope}%
\pgfsys@transformshift{2.855158in}{2.072540in}%
\pgfsys@useobject{currentmarker}{}%
\end{pgfscope}%
\begin{pgfscope}%
\pgfsys@transformshift{3.000100in}{2.172422in}%
\pgfsys@useobject{currentmarker}{}%
\end{pgfscope}%
\begin{pgfscope}%
\pgfsys@transformshift{3.145043in}{2.268437in}%
\pgfsys@useobject{currentmarker}{}%
\end{pgfscope}%
\begin{pgfscope}%
\pgfsys@transformshift{3.289985in}{2.362116in}%
\pgfsys@useobject{currentmarker}{}%
\end{pgfscope}%
\begin{pgfscope}%
\pgfsys@transformshift{3.434927in}{2.438026in}%
\pgfsys@useobject{currentmarker}{}%
\end{pgfscope}%
\begin{pgfscope}%
\pgfsys@transformshift{3.579870in}{2.524652in}%
\pgfsys@useobject{currentmarker}{}%
\end{pgfscope}%
\begin{pgfscope}%
\pgfsys@transformshift{3.724812in}{2.598211in}%
\pgfsys@useobject{currentmarker}{}%
\end{pgfscope}%
\end{pgfscope}%
\begin{pgfscope}%
\pgfsetbuttcap%
\pgfsetroundjoin%
\definecolor{currentfill}{rgb}{0.000000,0.000000,0.000000}%
\pgfsetfillcolor{currentfill}%
\pgfsetlinewidth{0.803000pt}%
\definecolor{currentstroke}{rgb}{0.000000,0.000000,0.000000}%
\pgfsetstrokecolor{currentstroke}%
\pgfsetdash{}{0pt}%
\pgfsys@defobject{currentmarker}{\pgfqpoint{0.000000in}{-0.048611in}}{\pgfqpoint{0.000000in}{0.000000in}}{%
\pgfpathmoveto{\pgfqpoint{0.000000in}{0.000000in}}%
\pgfpathlineto{\pgfqpoint{0.000000in}{-0.048611in}}%
\pgfusepath{stroke,fill}%
}%
\begin{pgfscope}%
\pgfsys@transformshift{1.115850in}{0.387222in}%
\pgfsys@useobject{currentmarker}{}%
\end{pgfscope}%
\end{pgfscope}%
\begin{pgfscope}%
\definecolor{textcolor}{rgb}{0.000000,0.000000,0.000000}%
\pgfsetstrokecolor{textcolor}%
\pgfsetfillcolor{textcolor}%
\pgftext[x=1.115850in,y=0.290000in,,top]{\color{textcolor}\sffamily\fontsize{10.000000}{12.000000}\selectfont 2.20}%
\end{pgfscope}%
\begin{pgfscope}%
\pgfsetbuttcap%
\pgfsetroundjoin%
\definecolor{currentfill}{rgb}{0.000000,0.000000,0.000000}%
\pgfsetfillcolor{currentfill}%
\pgfsetlinewidth{0.803000pt}%
\definecolor{currentstroke}{rgb}{0.000000,0.000000,0.000000}%
\pgfsetstrokecolor{currentstroke}%
\pgfsetdash{}{0pt}%
\pgfsys@defobject{currentmarker}{\pgfqpoint{0.000000in}{-0.048611in}}{\pgfqpoint{0.000000in}{0.000000in}}{%
\pgfpathmoveto{\pgfqpoint{0.000000in}{0.000000in}}%
\pgfpathlineto{\pgfqpoint{0.000000in}{-0.048611in}}%
\pgfusepath{stroke,fill}%
}%
\begin{pgfscope}%
\pgfsys@transformshift{1.840562in}{0.387222in}%
\pgfsys@useobject{currentmarker}{}%
\end{pgfscope}%
\end{pgfscope}%
\begin{pgfscope}%
\definecolor{textcolor}{rgb}{0.000000,0.000000,0.000000}%
\pgfsetstrokecolor{textcolor}%
\pgfsetfillcolor{textcolor}%
\pgftext[x=1.840562in,y=0.290000in,,top]{\color{textcolor}\sffamily\fontsize{10.000000}{12.000000}\selectfont 2.25}%
\end{pgfscope}%
\begin{pgfscope}%
\pgfsetbuttcap%
\pgfsetroundjoin%
\definecolor{currentfill}{rgb}{0.000000,0.000000,0.000000}%
\pgfsetfillcolor{currentfill}%
\pgfsetlinewidth{0.803000pt}%
\definecolor{currentstroke}{rgb}{0.000000,0.000000,0.000000}%
\pgfsetstrokecolor{currentstroke}%
\pgfsetdash{}{0pt}%
\pgfsys@defobject{currentmarker}{\pgfqpoint{0.000000in}{-0.048611in}}{\pgfqpoint{0.000000in}{0.000000in}}{%
\pgfpathmoveto{\pgfqpoint{0.000000in}{0.000000in}}%
\pgfpathlineto{\pgfqpoint{0.000000in}{-0.048611in}}%
\pgfusepath{stroke,fill}%
}%
\begin{pgfscope}%
\pgfsys@transformshift{2.565273in}{0.387222in}%
\pgfsys@useobject{currentmarker}{}%
\end{pgfscope}%
\end{pgfscope}%
\begin{pgfscope}%
\definecolor{textcolor}{rgb}{0.000000,0.000000,0.000000}%
\pgfsetstrokecolor{textcolor}%
\pgfsetfillcolor{textcolor}%
\pgftext[x=2.565273in,y=0.290000in,,top]{\color{textcolor}\sffamily\fontsize{10.000000}{12.000000}\selectfont 2.30}%
\end{pgfscope}%
\begin{pgfscope}%
\pgfsetbuttcap%
\pgfsetroundjoin%
\definecolor{currentfill}{rgb}{0.000000,0.000000,0.000000}%
\pgfsetfillcolor{currentfill}%
\pgfsetlinewidth{0.803000pt}%
\definecolor{currentstroke}{rgb}{0.000000,0.000000,0.000000}%
\pgfsetstrokecolor{currentstroke}%
\pgfsetdash{}{0pt}%
\pgfsys@defobject{currentmarker}{\pgfqpoint{0.000000in}{-0.048611in}}{\pgfqpoint{0.000000in}{0.000000in}}{%
\pgfpathmoveto{\pgfqpoint{0.000000in}{0.000000in}}%
\pgfpathlineto{\pgfqpoint{0.000000in}{-0.048611in}}%
\pgfusepath{stroke,fill}%
}%
\begin{pgfscope}%
\pgfsys@transformshift{3.289985in}{0.387222in}%
\pgfsys@useobject{currentmarker}{}%
\end{pgfscope}%
\end{pgfscope}%
\begin{pgfscope}%
\definecolor{textcolor}{rgb}{0.000000,0.000000,0.000000}%
\pgfsetstrokecolor{textcolor}%
\pgfsetfillcolor{textcolor}%
\pgftext[x=3.289985in,y=0.290000in,,top]{\color{textcolor}\sffamily\fontsize{10.000000}{12.000000}\selectfont 2.35}%
\end{pgfscope}%
\begin{pgfscope}%
\pgfsetbuttcap%
\pgfsetroundjoin%
\definecolor{currentfill}{rgb}{0.000000,0.000000,0.000000}%
\pgfsetfillcolor{currentfill}%
\pgfsetlinewidth{0.803000pt}%
\definecolor{currentstroke}{rgb}{0.000000,0.000000,0.000000}%
\pgfsetstrokecolor{currentstroke}%
\pgfsetdash{}{0pt}%
\pgfsys@defobject{currentmarker}{\pgfqpoint{-0.048611in}{0.000000in}}{\pgfqpoint{0.000000in}{0.000000in}}{%
\pgfpathmoveto{\pgfqpoint{0.000000in}{0.000000in}}%
\pgfpathlineto{\pgfqpoint{-0.048611in}{0.000000in}}%
\pgfusepath{stroke,fill}%
}%
\begin{pgfscope}%
\pgfsys@transformshift{0.672222in}{0.641275in}%
\pgfsys@useobject{currentmarker}{}%
\end{pgfscope}%
\end{pgfscope}%
\begin{pgfscope}%
\definecolor{textcolor}{rgb}{0.000000,0.000000,0.000000}%
\pgfsetstrokecolor{textcolor}%
\pgfsetfillcolor{textcolor}%
\pgftext[x=0.149382in,y=0.588513in,left,base]{\color{textcolor}\sffamily\fontsize{10.000000}{12.000000}\selectfont −1.55}%
\end{pgfscope}%
\begin{pgfscope}%
\pgfsetbuttcap%
\pgfsetroundjoin%
\definecolor{currentfill}{rgb}{0.000000,0.000000,0.000000}%
\pgfsetfillcolor{currentfill}%
\pgfsetlinewidth{0.803000pt}%
\definecolor{currentstroke}{rgb}{0.000000,0.000000,0.000000}%
\pgfsetstrokecolor{currentstroke}%
\pgfsetdash{}{0pt}%
\pgfsys@defobject{currentmarker}{\pgfqpoint{-0.048611in}{0.000000in}}{\pgfqpoint{0.000000in}{0.000000in}}{%
\pgfpathmoveto{\pgfqpoint{0.000000in}{0.000000in}}%
\pgfpathlineto{\pgfqpoint{-0.048611in}{0.000000in}}%
\pgfusepath{stroke,fill}%
}%
\begin{pgfscope}%
\pgfsys@transformshift{0.672222in}{0.945152in}%
\pgfsys@useobject{currentmarker}{}%
\end{pgfscope}%
\end{pgfscope}%
\begin{pgfscope}%
\definecolor{textcolor}{rgb}{0.000000,0.000000,0.000000}%
\pgfsetstrokecolor{textcolor}%
\pgfsetfillcolor{textcolor}%
\pgftext[x=0.149382in,y=0.892391in,left,base]{\color{textcolor}\sffamily\fontsize{10.000000}{12.000000}\selectfont −1.50}%
\end{pgfscope}%
\begin{pgfscope}%
\pgfsetbuttcap%
\pgfsetroundjoin%
\definecolor{currentfill}{rgb}{0.000000,0.000000,0.000000}%
\pgfsetfillcolor{currentfill}%
\pgfsetlinewidth{0.803000pt}%
\definecolor{currentstroke}{rgb}{0.000000,0.000000,0.000000}%
\pgfsetstrokecolor{currentstroke}%
\pgfsetdash{}{0pt}%
\pgfsys@defobject{currentmarker}{\pgfqpoint{-0.048611in}{0.000000in}}{\pgfqpoint{0.000000in}{0.000000in}}{%
\pgfpathmoveto{\pgfqpoint{0.000000in}{0.000000in}}%
\pgfpathlineto{\pgfqpoint{-0.048611in}{0.000000in}}%
\pgfusepath{stroke,fill}%
}%
\begin{pgfscope}%
\pgfsys@transformshift{0.672222in}{1.249030in}%
\pgfsys@useobject{currentmarker}{}%
\end{pgfscope}%
\end{pgfscope}%
\begin{pgfscope}%
\definecolor{textcolor}{rgb}{0.000000,0.000000,0.000000}%
\pgfsetstrokecolor{textcolor}%
\pgfsetfillcolor{textcolor}%
\pgftext[x=0.149382in,y=1.196269in,left,base]{\color{textcolor}\sffamily\fontsize{10.000000}{12.000000}\selectfont −1.45}%
\end{pgfscope}%
\begin{pgfscope}%
\pgfsetbuttcap%
\pgfsetroundjoin%
\definecolor{currentfill}{rgb}{0.000000,0.000000,0.000000}%
\pgfsetfillcolor{currentfill}%
\pgfsetlinewidth{0.803000pt}%
\definecolor{currentstroke}{rgb}{0.000000,0.000000,0.000000}%
\pgfsetstrokecolor{currentstroke}%
\pgfsetdash{}{0pt}%
\pgfsys@defobject{currentmarker}{\pgfqpoint{-0.048611in}{0.000000in}}{\pgfqpoint{0.000000in}{0.000000in}}{%
\pgfpathmoveto{\pgfqpoint{0.000000in}{0.000000in}}%
\pgfpathlineto{\pgfqpoint{-0.048611in}{0.000000in}}%
\pgfusepath{stroke,fill}%
}%
\begin{pgfscope}%
\pgfsys@transformshift{0.672222in}{1.552908in}%
\pgfsys@useobject{currentmarker}{}%
\end{pgfscope}%
\end{pgfscope}%
\begin{pgfscope}%
\definecolor{textcolor}{rgb}{0.000000,0.000000,0.000000}%
\pgfsetstrokecolor{textcolor}%
\pgfsetfillcolor{textcolor}%
\pgftext[x=0.149382in,y=1.500147in,left,base]{\color{textcolor}\sffamily\fontsize{10.000000}{12.000000}\selectfont −1.40}%
\end{pgfscope}%
\begin{pgfscope}%
\pgfsetbuttcap%
\pgfsetroundjoin%
\definecolor{currentfill}{rgb}{0.000000,0.000000,0.000000}%
\pgfsetfillcolor{currentfill}%
\pgfsetlinewidth{0.803000pt}%
\definecolor{currentstroke}{rgb}{0.000000,0.000000,0.000000}%
\pgfsetstrokecolor{currentstroke}%
\pgfsetdash{}{0pt}%
\pgfsys@defobject{currentmarker}{\pgfqpoint{-0.048611in}{0.000000in}}{\pgfqpoint{0.000000in}{0.000000in}}{%
\pgfpathmoveto{\pgfqpoint{0.000000in}{0.000000in}}%
\pgfpathlineto{\pgfqpoint{-0.048611in}{0.000000in}}%
\pgfusepath{stroke,fill}%
}%
\begin{pgfscope}%
\pgfsys@transformshift{0.672222in}{1.856786in}%
\pgfsys@useobject{currentmarker}{}%
\end{pgfscope}%
\end{pgfscope}%
\begin{pgfscope}%
\definecolor{textcolor}{rgb}{0.000000,0.000000,0.000000}%
\pgfsetstrokecolor{textcolor}%
\pgfsetfillcolor{textcolor}%
\pgftext[x=0.149382in,y=1.804025in,left,base]{\color{textcolor}\sffamily\fontsize{10.000000}{12.000000}\selectfont −1.35}%
\end{pgfscope}%
\begin{pgfscope}%
\pgfsetbuttcap%
\pgfsetroundjoin%
\definecolor{currentfill}{rgb}{0.000000,0.000000,0.000000}%
\pgfsetfillcolor{currentfill}%
\pgfsetlinewidth{0.803000pt}%
\definecolor{currentstroke}{rgb}{0.000000,0.000000,0.000000}%
\pgfsetstrokecolor{currentstroke}%
\pgfsetdash{}{0pt}%
\pgfsys@defobject{currentmarker}{\pgfqpoint{-0.048611in}{0.000000in}}{\pgfqpoint{0.000000in}{0.000000in}}{%
\pgfpathmoveto{\pgfqpoint{0.000000in}{0.000000in}}%
\pgfpathlineto{\pgfqpoint{-0.048611in}{0.000000in}}%
\pgfusepath{stroke,fill}%
}%
\begin{pgfscope}%
\pgfsys@transformshift{0.672222in}{2.160664in}%
\pgfsys@useobject{currentmarker}{}%
\end{pgfscope}%
\end{pgfscope}%
\begin{pgfscope}%
\definecolor{textcolor}{rgb}{0.000000,0.000000,0.000000}%
\pgfsetstrokecolor{textcolor}%
\pgfsetfillcolor{textcolor}%
\pgftext[x=0.149382in,y=2.107902in,left,base]{\color{textcolor}\sffamily\fontsize{10.000000}{12.000000}\selectfont −1.30}%
\end{pgfscope}%
\begin{pgfscope}%
\pgfsetbuttcap%
\pgfsetroundjoin%
\definecolor{currentfill}{rgb}{0.000000,0.000000,0.000000}%
\pgfsetfillcolor{currentfill}%
\pgfsetlinewidth{0.803000pt}%
\definecolor{currentstroke}{rgb}{0.000000,0.000000,0.000000}%
\pgfsetstrokecolor{currentstroke}%
\pgfsetdash{}{0pt}%
\pgfsys@defobject{currentmarker}{\pgfqpoint{-0.048611in}{0.000000in}}{\pgfqpoint{0.000000in}{0.000000in}}{%
\pgfpathmoveto{\pgfqpoint{0.000000in}{0.000000in}}%
\pgfpathlineto{\pgfqpoint{-0.048611in}{0.000000in}}%
\pgfusepath{stroke,fill}%
}%
\begin{pgfscope}%
\pgfsys@transformshift{0.672222in}{2.464542in}%
\pgfsys@useobject{currentmarker}{}%
\end{pgfscope}%
\end{pgfscope}%
\begin{pgfscope}%
\definecolor{textcolor}{rgb}{0.000000,0.000000,0.000000}%
\pgfsetstrokecolor{textcolor}%
\pgfsetfillcolor{textcolor}%
\pgftext[x=0.149382in,y=2.411780in,left,base]{\color{textcolor}\sffamily\fontsize{10.000000}{12.000000}\selectfont −1.25}%
\end{pgfscope}%
\begin{pgfscope}%
\pgfpathrectangle{\pgfqpoint{0.672222in}{0.387222in}}{\pgfqpoint{3.206333in}{2.324444in}}%
\pgfusepath{clip}%
\pgfsetrectcap%
\pgfsetroundjoin%
\pgfsetlinewidth{1.505625pt}%
\definecolor{currentstroke}{rgb}{0.121569,0.466667,0.705882}%
\pgfsetstrokecolor{currentstroke}%
\pgfsetdash{}{0pt}%
\pgfpathmoveto{\pgfqpoint{0.825965in}{0.500678in}}%
\pgfpathlineto{\pgfqpoint{0.970908in}{0.577267in}}%
\pgfpathlineto{\pgfqpoint{1.115850in}{0.666349in}}%
\pgfpathlineto{\pgfqpoint{1.260792in}{0.750122in}}%
\pgfpathlineto{\pgfqpoint{1.405735in}{0.845522in}}%
\pgfpathlineto{\pgfqpoint{1.550677in}{0.947950in}}%
\pgfpathlineto{\pgfqpoint{1.695619in}{1.054262in}}%
\pgfpathlineto{\pgfqpoint{1.840562in}{1.167321in}}%
\pgfpathlineto{\pgfqpoint{1.985504in}{1.282269in}}%
\pgfpathlineto{\pgfqpoint{2.130446in}{1.433463in}}%
\pgfpathlineto{\pgfqpoint{2.275389in}{1.561250in}}%
\pgfpathlineto{\pgfqpoint{2.420331in}{1.691062in}}%
\pgfpathlineto{\pgfqpoint{2.565273in}{1.818447in}}%
\pgfpathlineto{\pgfqpoint{2.710216in}{1.956768in}}%
\pgfpathlineto{\pgfqpoint{2.855158in}{2.072540in}}%
\pgfpathlineto{\pgfqpoint{3.000100in}{2.172422in}}%
\pgfpathlineto{\pgfqpoint{3.145043in}{2.268437in}}%
\pgfpathlineto{\pgfqpoint{3.289985in}{2.362116in}}%
\pgfpathlineto{\pgfqpoint{3.434927in}{2.438026in}}%
\pgfpathlineto{\pgfqpoint{3.579870in}{2.524652in}}%
\pgfpathlineto{\pgfqpoint{3.724812in}{2.598211in}}%
\pgfusepath{stroke}%
\end{pgfscope}%
\begin{pgfscope}%
\pgfsetrectcap%
\pgfsetmiterjoin%
\pgfsetlinewidth{0.803000pt}%
\definecolor{currentstroke}{rgb}{0.000000,0.000000,0.000000}%
\pgfsetstrokecolor{currentstroke}%
\pgfsetdash{}{0pt}%
\pgfpathmoveto{\pgfqpoint{0.672222in}{0.387222in}}%
\pgfpathlineto{\pgfqpoint{0.672222in}{2.711667in}}%
\pgfusepath{stroke}%
\end{pgfscope}%
\begin{pgfscope}%
\pgfsetrectcap%
\pgfsetmiterjoin%
\pgfsetlinewidth{0.803000pt}%
\definecolor{currentstroke}{rgb}{0.000000,0.000000,0.000000}%
\pgfsetstrokecolor{currentstroke}%
\pgfsetdash{}{0pt}%
\pgfpathmoveto{\pgfqpoint{3.878555in}{0.387222in}}%
\pgfpathlineto{\pgfqpoint{3.878555in}{2.711667in}}%
\pgfusepath{stroke}%
\end{pgfscope}%
\begin{pgfscope}%
\pgfsetrectcap%
\pgfsetmiterjoin%
\pgfsetlinewidth{0.803000pt}%
\definecolor{currentstroke}{rgb}{0.000000,0.000000,0.000000}%
\pgfsetstrokecolor{currentstroke}%
\pgfsetdash{}{0pt}%
\pgfpathmoveto{\pgfqpoint{0.672222in}{0.387222in}}%
\pgfpathlineto{\pgfqpoint{3.878555in}{0.387222in}}%
\pgfusepath{stroke}%
\end{pgfscope}%
\begin{pgfscope}%
\pgfsetrectcap%
\pgfsetmiterjoin%
\pgfsetlinewidth{0.803000pt}%
\definecolor{currentstroke}{rgb}{0.000000,0.000000,0.000000}%
\pgfsetstrokecolor{currentstroke}%
\pgfsetdash{}{0pt}%
\pgfpathmoveto{\pgfqpoint{0.672222in}{2.711667in}}%
\pgfpathlineto{\pgfqpoint{3.878555in}{2.711667in}}%
\pgfusepath{stroke}%
\end{pgfscope}%
\begin{pgfscope}%
\definecolor{textcolor}{rgb}{0.000000,0.000000,0.000000}%
\pgfsetstrokecolor{textcolor}%
\pgfsetfillcolor{textcolor}%
\pgftext[x=2.275389in,y=2.795000in,,base]{\color{textcolor}\sffamily\fontsize{12.000000}{14.400000}\selectfont \(\displaystyle  N = 64 \)}%
\end{pgfscope}%
\begin{pgfscope}%
\pgfsetbuttcap%
\pgfsetmiterjoin%
\definecolor{currentfill}{rgb}{1.000000,1.000000,1.000000}%
\pgfsetfillcolor{currentfill}%
\pgfsetlinewidth{0.000000pt}%
\definecolor{currentstroke}{rgb}{0.000000,0.000000,0.000000}%
\pgfsetstrokecolor{currentstroke}%
\pgfsetstrokeopacity{0.000000}%
\pgfsetdash{}{0pt}%
\pgfpathmoveto{\pgfqpoint{4.498277in}{0.387222in}}%
\pgfpathlineto{\pgfqpoint{7.704610in}{0.387222in}}%
\pgfpathlineto{\pgfqpoint{7.704610in}{2.711667in}}%
\pgfpathlineto{\pgfqpoint{4.498277in}{2.711667in}}%
\pgfpathclose%
\pgfusepath{fill}%
\end{pgfscope}%
\begin{pgfscope}%
\pgfpathrectangle{\pgfqpoint{4.498277in}{0.387222in}}{\pgfqpoint{3.206333in}{2.324444in}}%
\pgfusepath{clip}%
\pgfsetbuttcap%
\pgfsetroundjoin%
\definecolor{currentfill}{rgb}{0.121569,0.466667,0.705882}%
\pgfsetfillcolor{currentfill}%
\pgfsetfillopacity{0.300000}%
\pgfsetlinewidth{1.003750pt}%
\definecolor{currentstroke}{rgb}{0.121569,0.466667,0.705882}%
\pgfsetstrokecolor{currentstroke}%
\pgfsetstrokeopacity{0.300000}%
\pgfsetdash{}{0pt}%
\pgfpathmoveto{\pgfqpoint{4.652021in}{0.504827in}}%
\pgfpathlineto{\pgfqpoint{4.652021in}{0.496557in}}%
\pgfpathlineto{\pgfqpoint{4.796963in}{0.573957in}}%
\pgfpathlineto{\pgfqpoint{4.941905in}{0.653255in}}%
\pgfpathlineto{\pgfqpoint{5.086848in}{0.732156in}}%
\pgfpathlineto{\pgfqpoint{5.231790in}{0.819132in}}%
\pgfpathlineto{\pgfqpoint{5.376732in}{0.909849in}}%
\pgfpathlineto{\pgfqpoint{5.521675in}{1.019308in}}%
\pgfpathlineto{\pgfqpoint{5.666617in}{1.118651in}}%
\pgfpathlineto{\pgfqpoint{5.811559in}{1.212892in}}%
\pgfpathlineto{\pgfqpoint{5.956502in}{1.350923in}}%
\pgfpathlineto{\pgfqpoint{6.101444in}{1.474610in}}%
\pgfpathlineto{\pgfqpoint{6.246386in}{1.605896in}}%
\pgfpathlineto{\pgfqpoint{6.391329in}{1.796449in}}%
\pgfpathlineto{\pgfqpoint{6.536271in}{1.944295in}}%
\pgfpathlineto{\pgfqpoint{6.681213in}{2.043197in}}%
\pgfpathlineto{\pgfqpoint{6.826156in}{2.156810in}}%
\pgfpathlineto{\pgfqpoint{6.971098in}{2.262780in}}%
\pgfpathlineto{\pgfqpoint{7.116040in}{2.333231in}}%
\pgfpathlineto{\pgfqpoint{7.260983in}{2.421618in}}%
\pgfpathlineto{\pgfqpoint{7.405925in}{2.517339in}}%
\pgfpathlineto{\pgfqpoint{7.550867in}{2.586966in}}%
\pgfpathlineto{\pgfqpoint{7.550867in}{2.606010in}}%
\pgfpathlineto{\pgfqpoint{7.550867in}{2.606010in}}%
\pgfpathlineto{\pgfqpoint{7.405925in}{2.530916in}}%
\pgfpathlineto{\pgfqpoint{7.260983in}{2.452760in}}%
\pgfpathlineto{\pgfqpoint{7.116040in}{2.380771in}}%
\pgfpathlineto{\pgfqpoint{6.971098in}{2.274778in}}%
\pgfpathlineto{\pgfqpoint{6.826156in}{2.190431in}}%
\pgfpathlineto{\pgfqpoint{6.681213in}{2.073141in}}%
\pgfpathlineto{\pgfqpoint{6.536271in}{2.008778in}}%
\pgfpathlineto{\pgfqpoint{6.391329in}{1.822975in}}%
\pgfpathlineto{\pgfqpoint{6.246386in}{1.717336in}}%
\pgfpathlineto{\pgfqpoint{6.101444in}{1.556574in}}%
\pgfpathlineto{\pgfqpoint{5.956502in}{1.384781in}}%
\pgfpathlineto{\pgfqpoint{5.811559in}{1.253341in}}%
\pgfpathlineto{\pgfqpoint{5.666617in}{1.138590in}}%
\pgfpathlineto{\pgfqpoint{5.521675in}{1.045877in}}%
\pgfpathlineto{\pgfqpoint{5.376732in}{0.931955in}}%
\pgfpathlineto{\pgfqpoint{5.231790in}{0.836869in}}%
\pgfpathlineto{\pgfqpoint{5.086848in}{0.743181in}}%
\pgfpathlineto{\pgfqpoint{4.941905in}{0.664478in}}%
\pgfpathlineto{\pgfqpoint{4.796963in}{0.580432in}}%
\pgfpathlineto{\pgfqpoint{4.652021in}{0.504827in}}%
\pgfpathclose%
\pgfusepath{stroke,fill}%
\end{pgfscope}%
\begin{pgfscope}%
\pgfpathrectangle{\pgfqpoint{4.498277in}{0.387222in}}{\pgfqpoint{3.206333in}{2.324444in}}%
\pgfusepath{clip}%
\pgfsetbuttcap%
\pgfsetroundjoin%
\definecolor{currentfill}{rgb}{0.121569,0.466667,0.705882}%
\pgfsetfillcolor{currentfill}%
\pgfsetlinewidth{1.003750pt}%
\definecolor{currentstroke}{rgb}{0.121569,0.466667,0.705882}%
\pgfsetstrokecolor{currentstroke}%
\pgfsetdash{}{0pt}%
\pgfsys@defobject{currentmarker}{\pgfqpoint{-0.009821in}{-0.009821in}}{\pgfqpoint{0.009821in}{0.009821in}}{%
\pgfpathmoveto{\pgfqpoint{0.000000in}{-0.009821in}}%
\pgfpathcurveto{\pgfqpoint{0.002605in}{-0.009821in}}{\pgfqpoint{0.005103in}{-0.008786in}}{\pgfqpoint{0.006944in}{-0.006944in}}%
\pgfpathcurveto{\pgfqpoint{0.008786in}{-0.005103in}}{\pgfqpoint{0.009821in}{-0.002605in}}{\pgfqpoint{0.009821in}{0.000000in}}%
\pgfpathcurveto{\pgfqpoint{0.009821in}{0.002605in}}{\pgfqpoint{0.008786in}{0.005103in}}{\pgfqpoint{0.006944in}{0.006944in}}%
\pgfpathcurveto{\pgfqpoint{0.005103in}{0.008786in}}{\pgfqpoint{0.002605in}{0.009821in}}{\pgfqpoint{0.000000in}{0.009821in}}%
\pgfpathcurveto{\pgfqpoint{-0.002605in}{0.009821in}}{\pgfqpoint{-0.005103in}{0.008786in}}{\pgfqpoint{-0.006944in}{0.006944in}}%
\pgfpathcurveto{\pgfqpoint{-0.008786in}{0.005103in}}{\pgfqpoint{-0.009821in}{0.002605in}}{\pgfqpoint{-0.009821in}{0.000000in}}%
\pgfpathcurveto{\pgfqpoint{-0.009821in}{-0.002605in}}{\pgfqpoint{-0.008786in}{-0.005103in}}{\pgfqpoint{-0.006944in}{-0.006944in}}%
\pgfpathcurveto{\pgfqpoint{-0.005103in}{-0.008786in}}{\pgfqpoint{-0.002605in}{-0.009821in}}{\pgfqpoint{0.000000in}{-0.009821in}}%
\pgfpathclose%
\pgfusepath{stroke,fill}%
}%
\begin{pgfscope}%
\pgfsys@transformshift{4.652021in}{0.500692in}%
\pgfsys@useobject{currentmarker}{}%
\end{pgfscope}%
\begin{pgfscope}%
\pgfsys@transformshift{4.796963in}{0.577195in}%
\pgfsys@useobject{currentmarker}{}%
\end{pgfscope}%
\begin{pgfscope}%
\pgfsys@transformshift{4.941905in}{0.658866in}%
\pgfsys@useobject{currentmarker}{}%
\end{pgfscope}%
\begin{pgfscope}%
\pgfsys@transformshift{5.086848in}{0.737668in}%
\pgfsys@useobject{currentmarker}{}%
\end{pgfscope}%
\begin{pgfscope}%
\pgfsys@transformshift{5.231790in}{0.828001in}%
\pgfsys@useobject{currentmarker}{}%
\end{pgfscope}%
\begin{pgfscope}%
\pgfsys@transformshift{5.376732in}{0.920902in}%
\pgfsys@useobject{currentmarker}{}%
\end{pgfscope}%
\begin{pgfscope}%
\pgfsys@transformshift{5.521675in}{1.032593in}%
\pgfsys@useobject{currentmarker}{}%
\end{pgfscope}%
\begin{pgfscope}%
\pgfsys@transformshift{5.666617in}{1.128621in}%
\pgfsys@useobject{currentmarker}{}%
\end{pgfscope}%
\begin{pgfscope}%
\pgfsys@transformshift{5.811559in}{1.233117in}%
\pgfsys@useobject{currentmarker}{}%
\end{pgfscope}%
\begin{pgfscope}%
\pgfsys@transformshift{5.956502in}{1.367852in}%
\pgfsys@useobject{currentmarker}{}%
\end{pgfscope}%
\begin{pgfscope}%
\pgfsys@transformshift{6.101444in}{1.515592in}%
\pgfsys@useobject{currentmarker}{}%
\end{pgfscope}%
\begin{pgfscope}%
\pgfsys@transformshift{6.246386in}{1.661616in}%
\pgfsys@useobject{currentmarker}{}%
\end{pgfscope}%
\begin{pgfscope}%
\pgfsys@transformshift{6.391329in}{1.809712in}%
\pgfsys@useobject{currentmarker}{}%
\end{pgfscope}%
\begin{pgfscope}%
\pgfsys@transformshift{6.536271in}{1.976536in}%
\pgfsys@useobject{currentmarker}{}%
\end{pgfscope}%
\begin{pgfscope}%
\pgfsys@transformshift{6.681213in}{2.058169in}%
\pgfsys@useobject{currentmarker}{}%
\end{pgfscope}%
\begin{pgfscope}%
\pgfsys@transformshift{6.826156in}{2.173621in}%
\pgfsys@useobject{currentmarker}{}%
\end{pgfscope}%
\begin{pgfscope}%
\pgfsys@transformshift{6.971098in}{2.268779in}%
\pgfsys@useobject{currentmarker}{}%
\end{pgfscope}%
\begin{pgfscope}%
\pgfsys@transformshift{7.116040in}{2.357001in}%
\pgfsys@useobject{currentmarker}{}%
\end{pgfscope}%
\begin{pgfscope}%
\pgfsys@transformshift{7.260983in}{2.437189in}%
\pgfsys@useobject{currentmarker}{}%
\end{pgfscope}%
\begin{pgfscope}%
\pgfsys@transformshift{7.405925in}{2.524128in}%
\pgfsys@useobject{currentmarker}{}%
\end{pgfscope}%
\begin{pgfscope}%
\pgfsys@transformshift{7.550867in}{2.596488in}%
\pgfsys@useobject{currentmarker}{}%
\end{pgfscope}%
\end{pgfscope}%
\begin{pgfscope}%
\pgfsetbuttcap%
\pgfsetroundjoin%
\definecolor{currentfill}{rgb}{0.000000,0.000000,0.000000}%
\pgfsetfillcolor{currentfill}%
\pgfsetlinewidth{0.803000pt}%
\definecolor{currentstroke}{rgb}{0.000000,0.000000,0.000000}%
\pgfsetstrokecolor{currentstroke}%
\pgfsetdash{}{0pt}%
\pgfsys@defobject{currentmarker}{\pgfqpoint{0.000000in}{-0.048611in}}{\pgfqpoint{0.000000in}{0.000000in}}{%
\pgfpathmoveto{\pgfqpoint{0.000000in}{0.000000in}}%
\pgfpathlineto{\pgfqpoint{0.000000in}{-0.048611in}}%
\pgfusepath{stroke,fill}%
}%
\begin{pgfscope}%
\pgfsys@transformshift{5.086848in}{0.387222in}%
\pgfsys@useobject{currentmarker}{}%
\end{pgfscope}%
\end{pgfscope}%
\begin{pgfscope}%
\definecolor{textcolor}{rgb}{0.000000,0.000000,0.000000}%
\pgfsetstrokecolor{textcolor}%
\pgfsetfillcolor{textcolor}%
\pgftext[x=5.086848in,y=0.290000in,,top]{\color{textcolor}\sffamily\fontsize{10.000000}{12.000000}\selectfont 2.20}%
\end{pgfscope}%
\begin{pgfscope}%
\pgfsetbuttcap%
\pgfsetroundjoin%
\definecolor{currentfill}{rgb}{0.000000,0.000000,0.000000}%
\pgfsetfillcolor{currentfill}%
\pgfsetlinewidth{0.803000pt}%
\definecolor{currentstroke}{rgb}{0.000000,0.000000,0.000000}%
\pgfsetstrokecolor{currentstroke}%
\pgfsetdash{}{0pt}%
\pgfsys@defobject{currentmarker}{\pgfqpoint{0.000000in}{-0.048611in}}{\pgfqpoint{0.000000in}{0.000000in}}{%
\pgfpathmoveto{\pgfqpoint{0.000000in}{0.000000in}}%
\pgfpathlineto{\pgfqpoint{0.000000in}{-0.048611in}}%
\pgfusepath{stroke,fill}%
}%
\begin{pgfscope}%
\pgfsys@transformshift{5.811559in}{0.387222in}%
\pgfsys@useobject{currentmarker}{}%
\end{pgfscope}%
\end{pgfscope}%
\begin{pgfscope}%
\definecolor{textcolor}{rgb}{0.000000,0.000000,0.000000}%
\pgfsetstrokecolor{textcolor}%
\pgfsetfillcolor{textcolor}%
\pgftext[x=5.811559in,y=0.290000in,,top]{\color{textcolor}\sffamily\fontsize{10.000000}{12.000000}\selectfont 2.25}%
\end{pgfscope}%
\begin{pgfscope}%
\pgfsetbuttcap%
\pgfsetroundjoin%
\definecolor{currentfill}{rgb}{0.000000,0.000000,0.000000}%
\pgfsetfillcolor{currentfill}%
\pgfsetlinewidth{0.803000pt}%
\definecolor{currentstroke}{rgb}{0.000000,0.000000,0.000000}%
\pgfsetstrokecolor{currentstroke}%
\pgfsetdash{}{0pt}%
\pgfsys@defobject{currentmarker}{\pgfqpoint{0.000000in}{-0.048611in}}{\pgfqpoint{0.000000in}{0.000000in}}{%
\pgfpathmoveto{\pgfqpoint{0.000000in}{0.000000in}}%
\pgfpathlineto{\pgfqpoint{0.000000in}{-0.048611in}}%
\pgfusepath{stroke,fill}%
}%
\begin{pgfscope}%
\pgfsys@transformshift{6.536271in}{0.387222in}%
\pgfsys@useobject{currentmarker}{}%
\end{pgfscope}%
\end{pgfscope}%
\begin{pgfscope}%
\definecolor{textcolor}{rgb}{0.000000,0.000000,0.000000}%
\pgfsetstrokecolor{textcolor}%
\pgfsetfillcolor{textcolor}%
\pgftext[x=6.536271in,y=0.290000in,,top]{\color{textcolor}\sffamily\fontsize{10.000000}{12.000000}\selectfont 2.30}%
\end{pgfscope}%
\begin{pgfscope}%
\pgfsetbuttcap%
\pgfsetroundjoin%
\definecolor{currentfill}{rgb}{0.000000,0.000000,0.000000}%
\pgfsetfillcolor{currentfill}%
\pgfsetlinewidth{0.803000pt}%
\definecolor{currentstroke}{rgb}{0.000000,0.000000,0.000000}%
\pgfsetstrokecolor{currentstroke}%
\pgfsetdash{}{0pt}%
\pgfsys@defobject{currentmarker}{\pgfqpoint{0.000000in}{-0.048611in}}{\pgfqpoint{0.000000in}{0.000000in}}{%
\pgfpathmoveto{\pgfqpoint{0.000000in}{0.000000in}}%
\pgfpathlineto{\pgfqpoint{0.000000in}{-0.048611in}}%
\pgfusepath{stroke,fill}%
}%
\begin{pgfscope}%
\pgfsys@transformshift{7.260983in}{0.387222in}%
\pgfsys@useobject{currentmarker}{}%
\end{pgfscope}%
\end{pgfscope}%
\begin{pgfscope}%
\definecolor{textcolor}{rgb}{0.000000,0.000000,0.000000}%
\pgfsetstrokecolor{textcolor}%
\pgfsetfillcolor{textcolor}%
\pgftext[x=7.260983in,y=0.290000in,,top]{\color{textcolor}\sffamily\fontsize{10.000000}{12.000000}\selectfont 2.35}%
\end{pgfscope}%
\begin{pgfscope}%
\pgfsetbuttcap%
\pgfsetroundjoin%
\definecolor{currentfill}{rgb}{0.000000,0.000000,0.000000}%
\pgfsetfillcolor{currentfill}%
\pgfsetlinewidth{0.803000pt}%
\definecolor{currentstroke}{rgb}{0.000000,0.000000,0.000000}%
\pgfsetstrokecolor{currentstroke}%
\pgfsetdash{}{0pt}%
\pgfsys@defobject{currentmarker}{\pgfqpoint{-0.048611in}{0.000000in}}{\pgfqpoint{0.000000in}{0.000000in}}{%
\pgfpathmoveto{\pgfqpoint{0.000000in}{0.000000in}}%
\pgfpathlineto{\pgfqpoint{-0.048611in}{0.000000in}}%
\pgfusepath{stroke,fill}%
}%
\begin{pgfscope}%
\pgfsys@transformshift{4.498277in}{0.417428in}%
\pgfsys@useobject{currentmarker}{}%
\end{pgfscope}%
\end{pgfscope}%
\begin{pgfscope}%
\definecolor{textcolor}{rgb}{0.000000,0.000000,0.000000}%
\pgfsetstrokecolor{textcolor}%
\pgfsetfillcolor{textcolor}%
\pgftext[x=3.975437in,y=0.364667in,left,base]{\color{textcolor}\sffamily\fontsize{10.000000}{12.000000}\selectfont −1.60}%
\end{pgfscope}%
\begin{pgfscope}%
\pgfsetbuttcap%
\pgfsetroundjoin%
\definecolor{currentfill}{rgb}{0.000000,0.000000,0.000000}%
\pgfsetfillcolor{currentfill}%
\pgfsetlinewidth{0.803000pt}%
\definecolor{currentstroke}{rgb}{0.000000,0.000000,0.000000}%
\pgfsetstrokecolor{currentstroke}%
\pgfsetdash{}{0pt}%
\pgfsys@defobject{currentmarker}{\pgfqpoint{-0.048611in}{0.000000in}}{\pgfqpoint{0.000000in}{0.000000in}}{%
\pgfpathmoveto{\pgfqpoint{0.000000in}{0.000000in}}%
\pgfpathlineto{\pgfqpoint{-0.048611in}{0.000000in}}%
\pgfusepath{stroke,fill}%
}%
\begin{pgfscope}%
\pgfsys@transformshift{4.498277in}{0.719655in}%
\pgfsys@useobject{currentmarker}{}%
\end{pgfscope}%
\end{pgfscope}%
\begin{pgfscope}%
\definecolor{textcolor}{rgb}{0.000000,0.000000,0.000000}%
\pgfsetstrokecolor{textcolor}%
\pgfsetfillcolor{textcolor}%
\pgftext[x=3.975437in,y=0.666894in,left,base]{\color{textcolor}\sffamily\fontsize{10.000000}{12.000000}\selectfont −1.55}%
\end{pgfscope}%
\begin{pgfscope}%
\pgfsetbuttcap%
\pgfsetroundjoin%
\definecolor{currentfill}{rgb}{0.000000,0.000000,0.000000}%
\pgfsetfillcolor{currentfill}%
\pgfsetlinewidth{0.803000pt}%
\definecolor{currentstroke}{rgb}{0.000000,0.000000,0.000000}%
\pgfsetstrokecolor{currentstroke}%
\pgfsetdash{}{0pt}%
\pgfsys@defobject{currentmarker}{\pgfqpoint{-0.048611in}{0.000000in}}{\pgfqpoint{0.000000in}{0.000000in}}{%
\pgfpathmoveto{\pgfqpoint{0.000000in}{0.000000in}}%
\pgfpathlineto{\pgfqpoint{-0.048611in}{0.000000in}}%
\pgfusepath{stroke,fill}%
}%
\begin{pgfscope}%
\pgfsys@transformshift{4.498277in}{1.021882in}%
\pgfsys@useobject{currentmarker}{}%
\end{pgfscope}%
\end{pgfscope}%
\begin{pgfscope}%
\definecolor{textcolor}{rgb}{0.000000,0.000000,0.000000}%
\pgfsetstrokecolor{textcolor}%
\pgfsetfillcolor{textcolor}%
\pgftext[x=3.975437in,y=0.969121in,left,base]{\color{textcolor}\sffamily\fontsize{10.000000}{12.000000}\selectfont −1.50}%
\end{pgfscope}%
\begin{pgfscope}%
\pgfsetbuttcap%
\pgfsetroundjoin%
\definecolor{currentfill}{rgb}{0.000000,0.000000,0.000000}%
\pgfsetfillcolor{currentfill}%
\pgfsetlinewidth{0.803000pt}%
\definecolor{currentstroke}{rgb}{0.000000,0.000000,0.000000}%
\pgfsetstrokecolor{currentstroke}%
\pgfsetdash{}{0pt}%
\pgfsys@defobject{currentmarker}{\pgfqpoint{-0.048611in}{0.000000in}}{\pgfqpoint{0.000000in}{0.000000in}}{%
\pgfpathmoveto{\pgfqpoint{0.000000in}{0.000000in}}%
\pgfpathlineto{\pgfqpoint{-0.048611in}{0.000000in}}%
\pgfusepath{stroke,fill}%
}%
\begin{pgfscope}%
\pgfsys@transformshift{4.498277in}{1.324109in}%
\pgfsys@useobject{currentmarker}{}%
\end{pgfscope}%
\end{pgfscope}%
\begin{pgfscope}%
\definecolor{textcolor}{rgb}{0.000000,0.000000,0.000000}%
\pgfsetstrokecolor{textcolor}%
\pgfsetfillcolor{textcolor}%
\pgftext[x=3.975437in,y=1.271348in,left,base]{\color{textcolor}\sffamily\fontsize{10.000000}{12.000000}\selectfont −1.45}%
\end{pgfscope}%
\begin{pgfscope}%
\pgfsetbuttcap%
\pgfsetroundjoin%
\definecolor{currentfill}{rgb}{0.000000,0.000000,0.000000}%
\pgfsetfillcolor{currentfill}%
\pgfsetlinewidth{0.803000pt}%
\definecolor{currentstroke}{rgb}{0.000000,0.000000,0.000000}%
\pgfsetstrokecolor{currentstroke}%
\pgfsetdash{}{0pt}%
\pgfsys@defobject{currentmarker}{\pgfqpoint{-0.048611in}{0.000000in}}{\pgfqpoint{0.000000in}{0.000000in}}{%
\pgfpathmoveto{\pgfqpoint{0.000000in}{0.000000in}}%
\pgfpathlineto{\pgfqpoint{-0.048611in}{0.000000in}}%
\pgfusepath{stroke,fill}%
}%
\begin{pgfscope}%
\pgfsys@transformshift{4.498277in}{1.626336in}%
\pgfsys@useobject{currentmarker}{}%
\end{pgfscope}%
\end{pgfscope}%
\begin{pgfscope}%
\definecolor{textcolor}{rgb}{0.000000,0.000000,0.000000}%
\pgfsetstrokecolor{textcolor}%
\pgfsetfillcolor{textcolor}%
\pgftext[x=3.975437in,y=1.573575in,left,base]{\color{textcolor}\sffamily\fontsize{10.000000}{12.000000}\selectfont −1.40}%
\end{pgfscope}%
\begin{pgfscope}%
\pgfsetbuttcap%
\pgfsetroundjoin%
\definecolor{currentfill}{rgb}{0.000000,0.000000,0.000000}%
\pgfsetfillcolor{currentfill}%
\pgfsetlinewidth{0.803000pt}%
\definecolor{currentstroke}{rgb}{0.000000,0.000000,0.000000}%
\pgfsetstrokecolor{currentstroke}%
\pgfsetdash{}{0pt}%
\pgfsys@defobject{currentmarker}{\pgfqpoint{-0.048611in}{0.000000in}}{\pgfqpoint{0.000000in}{0.000000in}}{%
\pgfpathmoveto{\pgfqpoint{0.000000in}{0.000000in}}%
\pgfpathlineto{\pgfqpoint{-0.048611in}{0.000000in}}%
\pgfusepath{stroke,fill}%
}%
\begin{pgfscope}%
\pgfsys@transformshift{4.498277in}{1.928563in}%
\pgfsys@useobject{currentmarker}{}%
\end{pgfscope}%
\end{pgfscope}%
\begin{pgfscope}%
\definecolor{textcolor}{rgb}{0.000000,0.000000,0.000000}%
\pgfsetstrokecolor{textcolor}%
\pgfsetfillcolor{textcolor}%
\pgftext[x=3.975437in,y=1.875802in,left,base]{\color{textcolor}\sffamily\fontsize{10.000000}{12.000000}\selectfont −1.35}%
\end{pgfscope}%
\begin{pgfscope}%
\pgfsetbuttcap%
\pgfsetroundjoin%
\definecolor{currentfill}{rgb}{0.000000,0.000000,0.000000}%
\pgfsetfillcolor{currentfill}%
\pgfsetlinewidth{0.803000pt}%
\definecolor{currentstroke}{rgb}{0.000000,0.000000,0.000000}%
\pgfsetstrokecolor{currentstroke}%
\pgfsetdash{}{0pt}%
\pgfsys@defobject{currentmarker}{\pgfqpoint{-0.048611in}{0.000000in}}{\pgfqpoint{0.000000in}{0.000000in}}{%
\pgfpathmoveto{\pgfqpoint{0.000000in}{0.000000in}}%
\pgfpathlineto{\pgfqpoint{-0.048611in}{0.000000in}}%
\pgfusepath{stroke,fill}%
}%
\begin{pgfscope}%
\pgfsys@transformshift{4.498277in}{2.230790in}%
\pgfsys@useobject{currentmarker}{}%
\end{pgfscope}%
\end{pgfscope}%
\begin{pgfscope}%
\definecolor{textcolor}{rgb}{0.000000,0.000000,0.000000}%
\pgfsetstrokecolor{textcolor}%
\pgfsetfillcolor{textcolor}%
\pgftext[x=3.975437in,y=2.178029in,left,base]{\color{textcolor}\sffamily\fontsize{10.000000}{12.000000}\selectfont −1.30}%
\end{pgfscope}%
\begin{pgfscope}%
\pgfsetbuttcap%
\pgfsetroundjoin%
\definecolor{currentfill}{rgb}{0.000000,0.000000,0.000000}%
\pgfsetfillcolor{currentfill}%
\pgfsetlinewidth{0.803000pt}%
\definecolor{currentstroke}{rgb}{0.000000,0.000000,0.000000}%
\pgfsetstrokecolor{currentstroke}%
\pgfsetdash{}{0pt}%
\pgfsys@defobject{currentmarker}{\pgfqpoint{-0.048611in}{0.000000in}}{\pgfqpoint{0.000000in}{0.000000in}}{%
\pgfpathmoveto{\pgfqpoint{0.000000in}{0.000000in}}%
\pgfpathlineto{\pgfqpoint{-0.048611in}{0.000000in}}%
\pgfusepath{stroke,fill}%
}%
\begin{pgfscope}%
\pgfsys@transformshift{4.498277in}{2.533017in}%
\pgfsys@useobject{currentmarker}{}%
\end{pgfscope}%
\end{pgfscope}%
\begin{pgfscope}%
\definecolor{textcolor}{rgb}{0.000000,0.000000,0.000000}%
\pgfsetstrokecolor{textcolor}%
\pgfsetfillcolor{textcolor}%
\pgftext[x=3.975437in,y=2.480256in,left,base]{\color{textcolor}\sffamily\fontsize{10.000000}{12.000000}\selectfont −1.25}%
\end{pgfscope}%
\begin{pgfscope}%
\pgfpathrectangle{\pgfqpoint{4.498277in}{0.387222in}}{\pgfqpoint{3.206333in}{2.324444in}}%
\pgfusepath{clip}%
\pgfsetrectcap%
\pgfsetroundjoin%
\pgfsetlinewidth{1.505625pt}%
\definecolor{currentstroke}{rgb}{0.121569,0.466667,0.705882}%
\pgfsetstrokecolor{currentstroke}%
\pgfsetdash{}{0pt}%
\pgfpathmoveto{\pgfqpoint{4.652021in}{0.500692in}}%
\pgfpathlineto{\pgfqpoint{4.796963in}{0.577195in}}%
\pgfpathlineto{\pgfqpoint{4.941905in}{0.658866in}}%
\pgfpathlineto{\pgfqpoint{5.086848in}{0.737668in}}%
\pgfpathlineto{\pgfqpoint{5.231790in}{0.828001in}}%
\pgfpathlineto{\pgfqpoint{5.376732in}{0.920902in}}%
\pgfpathlineto{\pgfqpoint{5.521675in}{1.032593in}}%
\pgfpathlineto{\pgfqpoint{5.666617in}{1.128621in}}%
\pgfpathlineto{\pgfqpoint{5.811559in}{1.233117in}}%
\pgfpathlineto{\pgfqpoint{5.956502in}{1.367852in}}%
\pgfpathlineto{\pgfqpoint{6.101444in}{1.515592in}}%
\pgfpathlineto{\pgfqpoint{6.246386in}{1.661616in}}%
\pgfpathlineto{\pgfqpoint{6.391329in}{1.809712in}}%
\pgfpathlineto{\pgfqpoint{6.536271in}{1.976536in}}%
\pgfpathlineto{\pgfqpoint{6.681213in}{2.058169in}}%
\pgfpathlineto{\pgfqpoint{6.826156in}{2.173621in}}%
\pgfpathlineto{\pgfqpoint{6.971098in}{2.268779in}}%
\pgfpathlineto{\pgfqpoint{7.116040in}{2.357001in}}%
\pgfpathlineto{\pgfqpoint{7.260983in}{2.437189in}}%
\pgfpathlineto{\pgfqpoint{7.405925in}{2.524128in}}%
\pgfpathlineto{\pgfqpoint{7.550867in}{2.596488in}}%
\pgfusepath{stroke}%
\end{pgfscope}%
\begin{pgfscope}%
\pgfsetrectcap%
\pgfsetmiterjoin%
\pgfsetlinewidth{0.803000pt}%
\definecolor{currentstroke}{rgb}{0.000000,0.000000,0.000000}%
\pgfsetstrokecolor{currentstroke}%
\pgfsetdash{}{0pt}%
\pgfpathmoveto{\pgfqpoint{4.498277in}{0.387222in}}%
\pgfpathlineto{\pgfqpoint{4.498277in}{2.711667in}}%
\pgfusepath{stroke}%
\end{pgfscope}%
\begin{pgfscope}%
\pgfsetrectcap%
\pgfsetmiterjoin%
\pgfsetlinewidth{0.803000pt}%
\definecolor{currentstroke}{rgb}{0.000000,0.000000,0.000000}%
\pgfsetstrokecolor{currentstroke}%
\pgfsetdash{}{0pt}%
\pgfpathmoveto{\pgfqpoint{7.704610in}{0.387222in}}%
\pgfpathlineto{\pgfqpoint{7.704610in}{2.711667in}}%
\pgfusepath{stroke}%
\end{pgfscope}%
\begin{pgfscope}%
\pgfsetrectcap%
\pgfsetmiterjoin%
\pgfsetlinewidth{0.803000pt}%
\definecolor{currentstroke}{rgb}{0.000000,0.000000,0.000000}%
\pgfsetstrokecolor{currentstroke}%
\pgfsetdash{}{0pt}%
\pgfpathmoveto{\pgfqpoint{4.498277in}{0.387222in}}%
\pgfpathlineto{\pgfqpoint{7.704610in}{0.387222in}}%
\pgfusepath{stroke}%
\end{pgfscope}%
\begin{pgfscope}%
\pgfsetrectcap%
\pgfsetmiterjoin%
\pgfsetlinewidth{0.803000pt}%
\definecolor{currentstroke}{rgb}{0.000000,0.000000,0.000000}%
\pgfsetstrokecolor{currentstroke}%
\pgfsetdash{}{0pt}%
\pgfpathmoveto{\pgfqpoint{4.498277in}{2.711667in}}%
\pgfpathlineto{\pgfqpoint{7.704610in}{2.711667in}}%
\pgfusepath{stroke}%
\end{pgfscope}%
\begin{pgfscope}%
\definecolor{textcolor}{rgb}{0.000000,0.000000,0.000000}%
\pgfsetstrokecolor{textcolor}%
\pgfsetfillcolor{textcolor}%
\pgftext[x=6.101444in,y=2.795000in,,base]{\color{textcolor}\sffamily\fontsize{12.000000}{14.400000}\selectfont \(\displaystyle  N = 128 \)}%
\end{pgfscope}%
\end{pgfpicture}%
\makeatother%
\endgroup%
}
\caption{Internal energy $u$ for finer temperature $T$ using kinetic Monte Carlo algorithm}
\label{Fig:HeatBigKMC}
\end{figure}

\begin{figure}[htbp]
\centering
\scalebox{0.666}{%% Creator: Matplotlib, PGF backend
%%
%% To include the figure in your LaTeX document, write
%%   \input{<filename>.pgf}
%%
%% Make sure the required packages are loaded in your preamble
%%   \usepackage{pgf}
%%
%% Figures using additional raster images can only be included by \input if
%% they are in the same directory as the main LaTeX file. For loading figures
%% from other directories you can use the `import` package
%%   \usepackage{import}
%% and then include the figures with
%%   \import{<path to file>}{<filename>.pgf}
%%
%% Matplotlib used the following preamble
%%   \usepackage{fontspec}
%%   \setmainfont{DejaVuSerif.ttf}[Path=/home/lzh/anaconda3/envs/numana/lib/python3.7/site-packages/matplotlib/mpl-data/fonts/ttf/]
%%   \setsansfont{DejaVuSans.ttf}[Path=/home/lzh/anaconda3/envs/numana/lib/python3.7/site-packages/matplotlib/mpl-data/fonts/ttf/]
%%   \setmonofont{DejaVuSansMono.ttf}[Path=/home/lzh/anaconda3/envs/numana/lib/python3.7/site-packages/matplotlib/mpl-data/fonts/ttf/]
%%
\begingroup%
\makeatletter%
\begin{pgfpicture}%
\pgfpathrectangle{\pgfpointorigin}{\pgfqpoint{8.000000in}{6.000000in}}%
\pgfusepath{use as bounding box, clip}%
\begin{pgfscope}%
\pgfsetbuttcap%
\pgfsetmiterjoin%
\definecolor{currentfill}{rgb}{1.000000,1.000000,1.000000}%
\pgfsetfillcolor{currentfill}%
\pgfsetlinewidth{0.000000pt}%
\definecolor{currentstroke}{rgb}{1.000000,1.000000,1.000000}%
\pgfsetstrokecolor{currentstroke}%
\pgfsetdash{}{0pt}%
\pgfpathmoveto{\pgfqpoint{0.000000in}{0.000000in}}%
\pgfpathlineto{\pgfqpoint{8.000000in}{0.000000in}}%
\pgfpathlineto{\pgfqpoint{8.000000in}{6.000000in}}%
\pgfpathlineto{\pgfqpoint{0.000000in}{6.000000in}}%
\pgfpathclose%
\pgfusepath{fill}%
\end{pgfscope}%
\begin{pgfscope}%
\pgfsetbuttcap%
\pgfsetmiterjoin%
\definecolor{currentfill}{rgb}{1.000000,1.000000,1.000000}%
\pgfsetfillcolor{currentfill}%
\pgfsetlinewidth{0.000000pt}%
\definecolor{currentstroke}{rgb}{0.000000,0.000000,0.000000}%
\pgfsetstrokecolor{currentstroke}%
\pgfsetstrokeopacity{0.000000}%
\pgfsetdash{}{0pt}%
\pgfpathmoveto{\pgfqpoint{0.557222in}{3.312222in}}%
\pgfpathlineto{\pgfqpoint{3.879180in}{3.312222in}}%
\pgfpathlineto{\pgfqpoint{3.879180in}{5.636667in}}%
\pgfpathlineto{\pgfqpoint{0.557222in}{5.636667in}}%
\pgfpathclose%
\pgfusepath{fill}%
\end{pgfscope}%
\begin{pgfscope}%
\pgfpathrectangle{\pgfqpoint{0.557222in}{3.312222in}}{\pgfqpoint{3.321958in}{2.324444in}}%
\pgfusepath{clip}%
\pgfsetbuttcap%
\pgfsetroundjoin%
\definecolor{currentfill}{rgb}{0.121569,0.466667,0.705882}%
\pgfsetfillcolor{currentfill}%
\pgfsetfillopacity{0.300000}%
\pgfsetlinewidth{1.003750pt}%
\definecolor{currentstroke}{rgb}{0.121569,0.466667,0.705882}%
\pgfsetstrokecolor{currentstroke}%
\pgfsetstrokeopacity{0.300000}%
\pgfsetdash{}{0pt}%
\pgfpathmoveto{\pgfqpoint{0.716510in}{3.597893in}}%
\pgfpathlineto{\pgfqpoint{0.716510in}{3.483021in}}%
\pgfpathlineto{\pgfqpoint{0.866679in}{3.845383in}}%
\pgfpathlineto{\pgfqpoint{1.016848in}{4.212082in}}%
\pgfpathlineto{\pgfqpoint{1.167017in}{4.418591in}}%
\pgfpathlineto{\pgfqpoint{1.317186in}{4.751870in}}%
\pgfpathlineto{\pgfqpoint{1.467355in}{4.941003in}}%
\pgfpathlineto{\pgfqpoint{1.617525in}{5.159467in}}%
\pgfpathlineto{\pgfqpoint{1.767694in}{5.286981in}}%
\pgfpathlineto{\pgfqpoint{1.917863in}{5.420166in}}%
\pgfpathlineto{\pgfqpoint{2.068032in}{5.463586in}}%
\pgfpathlineto{\pgfqpoint{2.218201in}{5.490838in}}%
\pgfpathlineto{\pgfqpoint{2.368370in}{5.417321in}}%
\pgfpathlineto{\pgfqpoint{2.518540in}{5.395182in}}%
\pgfpathlineto{\pgfqpoint{2.668709in}{5.211758in}}%
\pgfpathlineto{\pgfqpoint{2.818878in}{5.073930in}}%
\pgfpathlineto{\pgfqpoint{2.969047in}{4.870366in}}%
\pgfpathlineto{\pgfqpoint{3.119216in}{4.636778in}}%
\pgfpathlineto{\pgfqpoint{3.269385in}{4.382760in}}%
\pgfpathlineto{\pgfqpoint{3.419554in}{4.048134in}}%
\pgfpathlineto{\pgfqpoint{3.569724in}{3.710611in}}%
\pgfpathlineto{\pgfqpoint{3.719893in}{3.417879in}}%
\pgfpathlineto{\pgfqpoint{3.719893in}{3.486926in}}%
\pgfpathlineto{\pgfqpoint{3.719893in}{3.486926in}}%
\pgfpathlineto{\pgfqpoint{3.569724in}{3.779689in}}%
\pgfpathlineto{\pgfqpoint{3.419554in}{4.148307in}}%
\pgfpathlineto{\pgfqpoint{3.269385in}{4.418430in}}%
\pgfpathlineto{\pgfqpoint{3.119216in}{4.679773in}}%
\pgfpathlineto{\pgfqpoint{2.969047in}{4.979879in}}%
\pgfpathlineto{\pgfqpoint{2.818878in}{5.154646in}}%
\pgfpathlineto{\pgfqpoint{2.668709in}{5.348161in}}%
\pgfpathlineto{\pgfqpoint{2.518540in}{5.468555in}}%
\pgfpathlineto{\pgfqpoint{2.368370in}{5.497930in}}%
\pgfpathlineto{\pgfqpoint{2.218201in}{5.531010in}}%
\pgfpathlineto{\pgfqpoint{2.068032in}{5.523318in}}%
\pgfpathlineto{\pgfqpoint{1.917863in}{5.511632in}}%
\pgfpathlineto{\pgfqpoint{1.767694in}{5.350341in}}%
\pgfpathlineto{\pgfqpoint{1.617525in}{5.269949in}}%
\pgfpathlineto{\pgfqpoint{1.467355in}{5.056700in}}%
\pgfpathlineto{\pgfqpoint{1.317186in}{4.855294in}}%
\pgfpathlineto{\pgfqpoint{1.167017in}{4.568400in}}%
\pgfpathlineto{\pgfqpoint{1.016848in}{4.292471in}}%
\pgfpathlineto{\pgfqpoint{0.866679in}{3.953788in}}%
\pgfpathlineto{\pgfqpoint{0.716510in}{3.597893in}}%
\pgfpathclose%
\pgfusepath{stroke,fill}%
\end{pgfscope}%
\begin{pgfscope}%
\pgfpathrectangle{\pgfqpoint{0.557222in}{3.312222in}}{\pgfqpoint{3.321958in}{2.324444in}}%
\pgfusepath{clip}%
\pgfsetbuttcap%
\pgfsetroundjoin%
\definecolor{currentfill}{rgb}{0.121569,0.466667,0.705882}%
\pgfsetfillcolor{currentfill}%
\pgfsetlinewidth{1.003750pt}%
\definecolor{currentstroke}{rgb}{0.121569,0.466667,0.705882}%
\pgfsetstrokecolor{currentstroke}%
\pgfsetdash{}{0pt}%
\pgfsys@defobject{currentmarker}{\pgfqpoint{-0.009821in}{-0.009821in}}{\pgfqpoint{0.009821in}{0.009821in}}{%
\pgfpathmoveto{\pgfqpoint{0.000000in}{-0.009821in}}%
\pgfpathcurveto{\pgfqpoint{0.002605in}{-0.009821in}}{\pgfqpoint{0.005103in}{-0.008786in}}{\pgfqpoint{0.006944in}{-0.006944in}}%
\pgfpathcurveto{\pgfqpoint{0.008786in}{-0.005103in}}{\pgfqpoint{0.009821in}{-0.002605in}}{\pgfqpoint{0.009821in}{0.000000in}}%
\pgfpathcurveto{\pgfqpoint{0.009821in}{0.002605in}}{\pgfqpoint{0.008786in}{0.005103in}}{\pgfqpoint{0.006944in}{0.006944in}}%
\pgfpathcurveto{\pgfqpoint{0.005103in}{0.008786in}}{\pgfqpoint{0.002605in}{0.009821in}}{\pgfqpoint{0.000000in}{0.009821in}}%
\pgfpathcurveto{\pgfqpoint{-0.002605in}{0.009821in}}{\pgfqpoint{-0.005103in}{0.008786in}}{\pgfqpoint{-0.006944in}{0.006944in}}%
\pgfpathcurveto{\pgfqpoint{-0.008786in}{0.005103in}}{\pgfqpoint{-0.009821in}{0.002605in}}{\pgfqpoint{-0.009821in}{0.000000in}}%
\pgfpathcurveto{\pgfqpoint{-0.009821in}{-0.002605in}}{\pgfqpoint{-0.008786in}{-0.005103in}}{\pgfqpoint{-0.006944in}{-0.006944in}}%
\pgfpathcurveto{\pgfqpoint{-0.005103in}{-0.008786in}}{\pgfqpoint{-0.002605in}{-0.009821in}}{\pgfqpoint{0.000000in}{-0.009821in}}%
\pgfpathclose%
\pgfusepath{stroke,fill}%
}%
\begin{pgfscope}%
\pgfsys@transformshift{0.716510in}{3.540457in}%
\pgfsys@useobject{currentmarker}{}%
\end{pgfscope}%
\begin{pgfscope}%
\pgfsys@transformshift{0.866679in}{3.899586in}%
\pgfsys@useobject{currentmarker}{}%
\end{pgfscope}%
\begin{pgfscope}%
\pgfsys@transformshift{1.016848in}{4.252277in}%
\pgfsys@useobject{currentmarker}{}%
\end{pgfscope}%
\begin{pgfscope}%
\pgfsys@transformshift{1.167017in}{4.493495in}%
\pgfsys@useobject{currentmarker}{}%
\end{pgfscope}%
\begin{pgfscope}%
\pgfsys@transformshift{1.317186in}{4.803582in}%
\pgfsys@useobject{currentmarker}{}%
\end{pgfscope}%
\begin{pgfscope}%
\pgfsys@transformshift{1.467355in}{4.998852in}%
\pgfsys@useobject{currentmarker}{}%
\end{pgfscope}%
\begin{pgfscope}%
\pgfsys@transformshift{1.617525in}{5.214708in}%
\pgfsys@useobject{currentmarker}{}%
\end{pgfscope}%
\begin{pgfscope}%
\pgfsys@transformshift{1.767694in}{5.318661in}%
\pgfsys@useobject{currentmarker}{}%
\end{pgfscope}%
\begin{pgfscope}%
\pgfsys@transformshift{1.917863in}{5.465899in}%
\pgfsys@useobject{currentmarker}{}%
\end{pgfscope}%
\begin{pgfscope}%
\pgfsys@transformshift{2.068032in}{5.493452in}%
\pgfsys@useobject{currentmarker}{}%
\end{pgfscope}%
\begin{pgfscope}%
\pgfsys@transformshift{2.218201in}{5.510924in}%
\pgfsys@useobject{currentmarker}{}%
\end{pgfscope}%
\begin{pgfscope}%
\pgfsys@transformshift{2.368370in}{5.457626in}%
\pgfsys@useobject{currentmarker}{}%
\end{pgfscope}%
\begin{pgfscope}%
\pgfsys@transformshift{2.518540in}{5.431869in}%
\pgfsys@useobject{currentmarker}{}%
\end{pgfscope}%
\begin{pgfscope}%
\pgfsys@transformshift{2.668709in}{5.279960in}%
\pgfsys@useobject{currentmarker}{}%
\end{pgfscope}%
\begin{pgfscope}%
\pgfsys@transformshift{2.818878in}{5.114288in}%
\pgfsys@useobject{currentmarker}{}%
\end{pgfscope}%
\begin{pgfscope}%
\pgfsys@transformshift{2.969047in}{4.925122in}%
\pgfsys@useobject{currentmarker}{}%
\end{pgfscope}%
\begin{pgfscope}%
\pgfsys@transformshift{3.119216in}{4.658275in}%
\pgfsys@useobject{currentmarker}{}%
\end{pgfscope}%
\begin{pgfscope}%
\pgfsys@transformshift{3.269385in}{4.400595in}%
\pgfsys@useobject{currentmarker}{}%
\end{pgfscope}%
\begin{pgfscope}%
\pgfsys@transformshift{3.419554in}{4.098221in}%
\pgfsys@useobject{currentmarker}{}%
\end{pgfscope}%
\begin{pgfscope}%
\pgfsys@transformshift{3.569724in}{3.745150in}%
\pgfsys@useobject{currentmarker}{}%
\end{pgfscope}%
\begin{pgfscope}%
\pgfsys@transformshift{3.719893in}{3.452403in}%
\pgfsys@useobject{currentmarker}{}%
\end{pgfscope}%
\end{pgfscope}%
\begin{pgfscope}%
\pgfsetbuttcap%
\pgfsetroundjoin%
\definecolor{currentfill}{rgb}{0.000000,0.000000,0.000000}%
\pgfsetfillcolor{currentfill}%
\pgfsetlinewidth{0.803000pt}%
\definecolor{currentstroke}{rgb}{0.000000,0.000000,0.000000}%
\pgfsetstrokecolor{currentstroke}%
\pgfsetdash{}{0pt}%
\pgfsys@defobject{currentmarker}{\pgfqpoint{0.000000in}{-0.048611in}}{\pgfqpoint{0.000000in}{0.000000in}}{%
\pgfpathmoveto{\pgfqpoint{0.000000in}{0.000000in}}%
\pgfpathlineto{\pgfqpoint{0.000000in}{-0.048611in}}%
\pgfusepath{stroke,fill}%
}%
\begin{pgfscope}%
\pgfsys@transformshift{1.167017in}{3.312222in}%
\pgfsys@useobject{currentmarker}{}%
\end{pgfscope}%
\end{pgfscope}%
\begin{pgfscope}%
\definecolor{textcolor}{rgb}{0.000000,0.000000,0.000000}%
\pgfsetstrokecolor{textcolor}%
\pgfsetfillcolor{textcolor}%
\pgftext[x=1.167017in,y=3.215000in,,top]{\color{textcolor}\sffamily\fontsize{10.000000}{12.000000}\selectfont 2.25}%
\end{pgfscope}%
\begin{pgfscope}%
\pgfsetbuttcap%
\pgfsetroundjoin%
\definecolor{currentfill}{rgb}{0.000000,0.000000,0.000000}%
\pgfsetfillcolor{currentfill}%
\pgfsetlinewidth{0.803000pt}%
\definecolor{currentstroke}{rgb}{0.000000,0.000000,0.000000}%
\pgfsetstrokecolor{currentstroke}%
\pgfsetdash{}{0pt}%
\pgfsys@defobject{currentmarker}{\pgfqpoint{0.000000in}{-0.048611in}}{\pgfqpoint{0.000000in}{0.000000in}}{%
\pgfpathmoveto{\pgfqpoint{0.000000in}{0.000000in}}%
\pgfpathlineto{\pgfqpoint{0.000000in}{-0.048611in}}%
\pgfusepath{stroke,fill}%
}%
\begin{pgfscope}%
\pgfsys@transformshift{1.917863in}{3.312222in}%
\pgfsys@useobject{currentmarker}{}%
\end{pgfscope}%
\end{pgfscope}%
\begin{pgfscope}%
\definecolor{textcolor}{rgb}{0.000000,0.000000,0.000000}%
\pgfsetstrokecolor{textcolor}%
\pgfsetfillcolor{textcolor}%
\pgftext[x=1.917863in,y=3.215000in,,top]{\color{textcolor}\sffamily\fontsize{10.000000}{12.000000}\selectfont 2.30}%
\end{pgfscope}%
\begin{pgfscope}%
\pgfsetbuttcap%
\pgfsetroundjoin%
\definecolor{currentfill}{rgb}{0.000000,0.000000,0.000000}%
\pgfsetfillcolor{currentfill}%
\pgfsetlinewidth{0.803000pt}%
\definecolor{currentstroke}{rgb}{0.000000,0.000000,0.000000}%
\pgfsetstrokecolor{currentstroke}%
\pgfsetdash{}{0pt}%
\pgfsys@defobject{currentmarker}{\pgfqpoint{0.000000in}{-0.048611in}}{\pgfqpoint{0.000000in}{0.000000in}}{%
\pgfpathmoveto{\pgfqpoint{0.000000in}{0.000000in}}%
\pgfpathlineto{\pgfqpoint{0.000000in}{-0.048611in}}%
\pgfusepath{stroke,fill}%
}%
\begin{pgfscope}%
\pgfsys@transformshift{2.668709in}{3.312222in}%
\pgfsys@useobject{currentmarker}{}%
\end{pgfscope}%
\end{pgfscope}%
\begin{pgfscope}%
\definecolor{textcolor}{rgb}{0.000000,0.000000,0.000000}%
\pgfsetstrokecolor{textcolor}%
\pgfsetfillcolor{textcolor}%
\pgftext[x=2.668709in,y=3.215000in,,top]{\color{textcolor}\sffamily\fontsize{10.000000}{12.000000}\selectfont 2.35}%
\end{pgfscope}%
\begin{pgfscope}%
\pgfsetbuttcap%
\pgfsetroundjoin%
\definecolor{currentfill}{rgb}{0.000000,0.000000,0.000000}%
\pgfsetfillcolor{currentfill}%
\pgfsetlinewidth{0.803000pt}%
\definecolor{currentstroke}{rgb}{0.000000,0.000000,0.000000}%
\pgfsetstrokecolor{currentstroke}%
\pgfsetdash{}{0pt}%
\pgfsys@defobject{currentmarker}{\pgfqpoint{0.000000in}{-0.048611in}}{\pgfqpoint{0.000000in}{0.000000in}}{%
\pgfpathmoveto{\pgfqpoint{0.000000in}{0.000000in}}%
\pgfpathlineto{\pgfqpoint{0.000000in}{-0.048611in}}%
\pgfusepath{stroke,fill}%
}%
\begin{pgfscope}%
\pgfsys@transformshift{3.419554in}{3.312222in}%
\pgfsys@useobject{currentmarker}{}%
\end{pgfscope}%
\end{pgfscope}%
\begin{pgfscope}%
\definecolor{textcolor}{rgb}{0.000000,0.000000,0.000000}%
\pgfsetstrokecolor{textcolor}%
\pgfsetfillcolor{textcolor}%
\pgftext[x=3.419554in,y=3.215000in,,top]{\color{textcolor}\sffamily\fontsize{10.000000}{12.000000}\selectfont 2.40}%
\end{pgfscope}%
\begin{pgfscope}%
\pgfsetbuttcap%
\pgfsetroundjoin%
\definecolor{currentfill}{rgb}{0.000000,0.000000,0.000000}%
\pgfsetfillcolor{currentfill}%
\pgfsetlinewidth{0.803000pt}%
\definecolor{currentstroke}{rgb}{0.000000,0.000000,0.000000}%
\pgfsetstrokecolor{currentstroke}%
\pgfsetdash{}{0pt}%
\pgfsys@defobject{currentmarker}{\pgfqpoint{-0.048611in}{0.000000in}}{\pgfqpoint{0.000000in}{0.000000in}}{%
\pgfpathmoveto{\pgfqpoint{0.000000in}{0.000000in}}%
\pgfpathlineto{\pgfqpoint{-0.048611in}{0.000000in}}%
\pgfusepath{stroke,fill}%
}%
\begin{pgfscope}%
\pgfsys@transformshift{0.557222in}{3.521300in}%
\pgfsys@useobject{currentmarker}{}%
\end{pgfscope}%
\end{pgfscope}%
\begin{pgfscope}%
\definecolor{textcolor}{rgb}{0.000000,0.000000,0.000000}%
\pgfsetstrokecolor{textcolor}%
\pgfsetfillcolor{textcolor}%
\pgftext[x=0.150755in,y=3.468538in,left,base]{\color{textcolor}\sffamily\fontsize{10.000000}{12.000000}\selectfont 1.35}%
\end{pgfscope}%
\begin{pgfscope}%
\pgfsetbuttcap%
\pgfsetroundjoin%
\definecolor{currentfill}{rgb}{0.000000,0.000000,0.000000}%
\pgfsetfillcolor{currentfill}%
\pgfsetlinewidth{0.803000pt}%
\definecolor{currentstroke}{rgb}{0.000000,0.000000,0.000000}%
\pgfsetstrokecolor{currentstroke}%
\pgfsetdash{}{0pt}%
\pgfsys@defobject{currentmarker}{\pgfqpoint{-0.048611in}{0.000000in}}{\pgfqpoint{0.000000in}{0.000000in}}{%
\pgfpathmoveto{\pgfqpoint{0.000000in}{0.000000in}}%
\pgfpathlineto{\pgfqpoint{-0.048611in}{0.000000in}}%
\pgfusepath{stroke,fill}%
}%
\begin{pgfscope}%
\pgfsys@transformshift{0.557222in}{4.025537in}%
\pgfsys@useobject{currentmarker}{}%
\end{pgfscope}%
\end{pgfscope}%
\begin{pgfscope}%
\definecolor{textcolor}{rgb}{0.000000,0.000000,0.000000}%
\pgfsetstrokecolor{textcolor}%
\pgfsetfillcolor{textcolor}%
\pgftext[x=0.150755in,y=3.972775in,left,base]{\color{textcolor}\sffamily\fontsize{10.000000}{12.000000}\selectfont 1.40}%
\end{pgfscope}%
\begin{pgfscope}%
\pgfsetbuttcap%
\pgfsetroundjoin%
\definecolor{currentfill}{rgb}{0.000000,0.000000,0.000000}%
\pgfsetfillcolor{currentfill}%
\pgfsetlinewidth{0.803000pt}%
\definecolor{currentstroke}{rgb}{0.000000,0.000000,0.000000}%
\pgfsetstrokecolor{currentstroke}%
\pgfsetdash{}{0pt}%
\pgfsys@defobject{currentmarker}{\pgfqpoint{-0.048611in}{0.000000in}}{\pgfqpoint{0.000000in}{0.000000in}}{%
\pgfpathmoveto{\pgfqpoint{0.000000in}{0.000000in}}%
\pgfpathlineto{\pgfqpoint{-0.048611in}{0.000000in}}%
\pgfusepath{stroke,fill}%
}%
\begin{pgfscope}%
\pgfsys@transformshift{0.557222in}{4.529774in}%
\pgfsys@useobject{currentmarker}{}%
\end{pgfscope}%
\end{pgfscope}%
\begin{pgfscope}%
\definecolor{textcolor}{rgb}{0.000000,0.000000,0.000000}%
\pgfsetstrokecolor{textcolor}%
\pgfsetfillcolor{textcolor}%
\pgftext[x=0.150755in,y=4.477013in,left,base]{\color{textcolor}\sffamily\fontsize{10.000000}{12.000000}\selectfont 1.45}%
\end{pgfscope}%
\begin{pgfscope}%
\pgfsetbuttcap%
\pgfsetroundjoin%
\definecolor{currentfill}{rgb}{0.000000,0.000000,0.000000}%
\pgfsetfillcolor{currentfill}%
\pgfsetlinewidth{0.803000pt}%
\definecolor{currentstroke}{rgb}{0.000000,0.000000,0.000000}%
\pgfsetstrokecolor{currentstroke}%
\pgfsetdash{}{0pt}%
\pgfsys@defobject{currentmarker}{\pgfqpoint{-0.048611in}{0.000000in}}{\pgfqpoint{0.000000in}{0.000000in}}{%
\pgfpathmoveto{\pgfqpoint{0.000000in}{0.000000in}}%
\pgfpathlineto{\pgfqpoint{-0.048611in}{0.000000in}}%
\pgfusepath{stroke,fill}%
}%
\begin{pgfscope}%
\pgfsys@transformshift{0.557222in}{5.034011in}%
\pgfsys@useobject{currentmarker}{}%
\end{pgfscope}%
\end{pgfscope}%
\begin{pgfscope}%
\definecolor{textcolor}{rgb}{0.000000,0.000000,0.000000}%
\pgfsetstrokecolor{textcolor}%
\pgfsetfillcolor{textcolor}%
\pgftext[x=0.150755in,y=4.981250in,left,base]{\color{textcolor}\sffamily\fontsize{10.000000}{12.000000}\selectfont 1.50}%
\end{pgfscope}%
\begin{pgfscope}%
\pgfsetbuttcap%
\pgfsetroundjoin%
\definecolor{currentfill}{rgb}{0.000000,0.000000,0.000000}%
\pgfsetfillcolor{currentfill}%
\pgfsetlinewidth{0.803000pt}%
\definecolor{currentstroke}{rgb}{0.000000,0.000000,0.000000}%
\pgfsetstrokecolor{currentstroke}%
\pgfsetdash{}{0pt}%
\pgfsys@defobject{currentmarker}{\pgfqpoint{-0.048611in}{0.000000in}}{\pgfqpoint{0.000000in}{0.000000in}}{%
\pgfpathmoveto{\pgfqpoint{0.000000in}{0.000000in}}%
\pgfpathlineto{\pgfqpoint{-0.048611in}{0.000000in}}%
\pgfusepath{stroke,fill}%
}%
\begin{pgfscope}%
\pgfsys@transformshift{0.557222in}{5.538248in}%
\pgfsys@useobject{currentmarker}{}%
\end{pgfscope}%
\end{pgfscope}%
\begin{pgfscope}%
\definecolor{textcolor}{rgb}{0.000000,0.000000,0.000000}%
\pgfsetstrokecolor{textcolor}%
\pgfsetfillcolor{textcolor}%
\pgftext[x=0.150755in,y=5.485487in,left,base]{\color{textcolor}\sffamily\fontsize{10.000000}{12.000000}\selectfont 1.55}%
\end{pgfscope}%
\begin{pgfscope}%
\pgfpathrectangle{\pgfqpoint{0.557222in}{3.312222in}}{\pgfqpoint{3.321958in}{2.324444in}}%
\pgfusepath{clip}%
\pgfsetrectcap%
\pgfsetroundjoin%
\pgfsetlinewidth{1.505625pt}%
\definecolor{currentstroke}{rgb}{0.121569,0.466667,0.705882}%
\pgfsetstrokecolor{currentstroke}%
\pgfsetdash{}{0pt}%
\pgfpathmoveto{\pgfqpoint{0.716510in}{3.540457in}}%
\pgfpathlineto{\pgfqpoint{0.866679in}{3.899586in}}%
\pgfpathlineto{\pgfqpoint{1.016848in}{4.252277in}}%
\pgfpathlineto{\pgfqpoint{1.167017in}{4.493495in}}%
\pgfpathlineto{\pgfqpoint{1.317186in}{4.803582in}}%
\pgfpathlineto{\pgfqpoint{1.467355in}{4.998852in}}%
\pgfpathlineto{\pgfqpoint{1.617525in}{5.214708in}}%
\pgfpathlineto{\pgfqpoint{1.767694in}{5.318661in}}%
\pgfpathlineto{\pgfqpoint{1.917863in}{5.465899in}}%
\pgfpathlineto{\pgfqpoint{2.068032in}{5.493452in}}%
\pgfpathlineto{\pgfqpoint{2.218201in}{5.510924in}}%
\pgfpathlineto{\pgfqpoint{2.368370in}{5.457626in}}%
\pgfpathlineto{\pgfqpoint{2.518540in}{5.431869in}}%
\pgfpathlineto{\pgfqpoint{2.668709in}{5.279960in}}%
\pgfpathlineto{\pgfqpoint{2.818878in}{5.114288in}}%
\pgfpathlineto{\pgfqpoint{2.969047in}{4.925122in}}%
\pgfpathlineto{\pgfqpoint{3.119216in}{4.658275in}}%
\pgfpathlineto{\pgfqpoint{3.269385in}{4.400595in}}%
\pgfpathlineto{\pgfqpoint{3.419554in}{4.098221in}}%
\pgfpathlineto{\pgfqpoint{3.569724in}{3.745150in}}%
\pgfpathlineto{\pgfqpoint{3.719893in}{3.452403in}}%
\pgfusepath{stroke}%
\end{pgfscope}%
\begin{pgfscope}%
\pgfsetrectcap%
\pgfsetmiterjoin%
\pgfsetlinewidth{0.803000pt}%
\definecolor{currentstroke}{rgb}{0.000000,0.000000,0.000000}%
\pgfsetstrokecolor{currentstroke}%
\pgfsetdash{}{0pt}%
\pgfpathmoveto{\pgfqpoint{0.557222in}{3.312222in}}%
\pgfpathlineto{\pgfqpoint{0.557222in}{5.636667in}}%
\pgfusepath{stroke}%
\end{pgfscope}%
\begin{pgfscope}%
\pgfsetrectcap%
\pgfsetmiterjoin%
\pgfsetlinewidth{0.803000pt}%
\definecolor{currentstroke}{rgb}{0.000000,0.000000,0.000000}%
\pgfsetstrokecolor{currentstroke}%
\pgfsetdash{}{0pt}%
\pgfpathmoveto{\pgfqpoint{3.879180in}{3.312222in}}%
\pgfpathlineto{\pgfqpoint{3.879180in}{5.636667in}}%
\pgfusepath{stroke}%
\end{pgfscope}%
\begin{pgfscope}%
\pgfsetrectcap%
\pgfsetmiterjoin%
\pgfsetlinewidth{0.803000pt}%
\definecolor{currentstroke}{rgb}{0.000000,0.000000,0.000000}%
\pgfsetstrokecolor{currentstroke}%
\pgfsetdash{}{0pt}%
\pgfpathmoveto{\pgfqpoint{0.557222in}{3.312222in}}%
\pgfpathlineto{\pgfqpoint{3.879180in}{3.312222in}}%
\pgfusepath{stroke}%
\end{pgfscope}%
\begin{pgfscope}%
\pgfsetrectcap%
\pgfsetmiterjoin%
\pgfsetlinewidth{0.803000pt}%
\definecolor{currentstroke}{rgb}{0.000000,0.000000,0.000000}%
\pgfsetstrokecolor{currentstroke}%
\pgfsetdash{}{0pt}%
\pgfpathmoveto{\pgfqpoint{0.557222in}{5.636667in}}%
\pgfpathlineto{\pgfqpoint{3.879180in}{5.636667in}}%
\pgfusepath{stroke}%
\end{pgfscope}%
\begin{pgfscope}%
\definecolor{textcolor}{rgb}{0.000000,0.000000,0.000000}%
\pgfsetstrokecolor{textcolor}%
\pgfsetfillcolor{textcolor}%
\pgftext[x=2.218201in,y=5.720000in,,base]{\color{textcolor}\sffamily\fontsize{12.000000}{14.400000}\selectfont \(\displaystyle  N = 16 \)}%
\end{pgfscope}%
\begin{pgfscope}%
\pgfsetbuttcap%
\pgfsetmiterjoin%
\definecolor{currentfill}{rgb}{1.000000,1.000000,1.000000}%
\pgfsetfillcolor{currentfill}%
\pgfsetlinewidth{0.000000pt}%
\definecolor{currentstroke}{rgb}{0.000000,0.000000,0.000000}%
\pgfsetstrokecolor{currentstroke}%
\pgfsetstrokeopacity{0.000000}%
\pgfsetdash{}{0pt}%
\pgfpathmoveto{\pgfqpoint{4.382652in}{3.312222in}}%
\pgfpathlineto{\pgfqpoint{7.704610in}{3.312222in}}%
\pgfpathlineto{\pgfqpoint{7.704610in}{5.636667in}}%
\pgfpathlineto{\pgfqpoint{4.382652in}{5.636667in}}%
\pgfpathclose%
\pgfusepath{fill}%
\end{pgfscope}%
\begin{pgfscope}%
\pgfpathrectangle{\pgfqpoint{4.382652in}{3.312222in}}{\pgfqpoint{3.321958in}{2.324444in}}%
\pgfusepath{clip}%
\pgfsetbuttcap%
\pgfsetroundjoin%
\definecolor{currentfill}{rgb}{0.121569,0.466667,0.705882}%
\pgfsetfillcolor{currentfill}%
\pgfsetfillopacity{0.300000}%
\pgfsetlinewidth{1.003750pt}%
\definecolor{currentstroke}{rgb}{0.121569,0.466667,0.705882}%
\pgfsetstrokecolor{currentstroke}%
\pgfsetstrokeopacity{0.300000}%
\pgfsetdash{}{0pt}%
\pgfpathmoveto{\pgfqpoint{4.541940in}{3.581583in}}%
\pgfpathlineto{\pgfqpoint{4.541940in}{3.417879in}}%
\pgfpathlineto{\pgfqpoint{4.692109in}{3.653684in}}%
\pgfpathlineto{\pgfqpoint{4.842278in}{3.775589in}}%
\pgfpathlineto{\pgfqpoint{4.992447in}{4.098907in}}%
\pgfpathlineto{\pgfqpoint{5.142617in}{4.238444in}}%
\pgfpathlineto{\pgfqpoint{5.292786in}{4.602993in}}%
\pgfpathlineto{\pgfqpoint{5.442955in}{4.709282in}}%
\pgfpathlineto{\pgfqpoint{5.593124in}{5.000884in}}%
\pgfpathlineto{\pgfqpoint{5.743293in}{5.181791in}}%
\pgfpathlineto{\pgfqpoint{5.893462in}{5.316691in}}%
\pgfpathlineto{\pgfqpoint{6.043631in}{5.424870in}}%
\pgfpathlineto{\pgfqpoint{6.193801in}{5.379751in}}%
\pgfpathlineto{\pgfqpoint{6.343970in}{5.247317in}}%
\pgfpathlineto{\pgfqpoint{6.494139in}{5.106998in}}%
\pgfpathlineto{\pgfqpoint{6.644308in}{4.970743in}}%
\pgfpathlineto{\pgfqpoint{6.794477in}{4.772387in}}%
\pgfpathlineto{\pgfqpoint{6.944646in}{4.483464in}}%
\pgfpathlineto{\pgfqpoint{7.094816in}{4.228662in}}%
\pgfpathlineto{\pgfqpoint{7.244985in}{3.941628in}}%
\pgfpathlineto{\pgfqpoint{7.395154in}{3.707982in}}%
\pgfpathlineto{\pgfqpoint{7.545323in}{3.459400in}}%
\pgfpathlineto{\pgfqpoint{7.545323in}{3.548470in}}%
\pgfpathlineto{\pgfqpoint{7.545323in}{3.548470in}}%
\pgfpathlineto{\pgfqpoint{7.395154in}{3.848444in}}%
\pgfpathlineto{\pgfqpoint{7.244985in}{4.054496in}}%
\pgfpathlineto{\pgfqpoint{7.094816in}{4.365142in}}%
\pgfpathlineto{\pgfqpoint{6.944646in}{4.657320in}}%
\pgfpathlineto{\pgfqpoint{6.794477in}{4.812351in}}%
\pgfpathlineto{\pgfqpoint{6.644308in}{5.049761in}}%
\pgfpathlineto{\pgfqpoint{6.494139in}{5.305736in}}%
\pgfpathlineto{\pgfqpoint{6.343970in}{5.390887in}}%
\pgfpathlineto{\pgfqpoint{6.193801in}{5.531010in}}%
\pgfpathlineto{\pgfqpoint{6.043631in}{5.473742in}}%
\pgfpathlineto{\pgfqpoint{5.893462in}{5.419225in}}%
\pgfpathlineto{\pgfqpoint{5.743293in}{5.340060in}}%
\pgfpathlineto{\pgfqpoint{5.593124in}{5.110651in}}%
\pgfpathlineto{\pgfqpoint{5.442955in}{4.956336in}}%
\pgfpathlineto{\pgfqpoint{5.292786in}{4.673809in}}%
\pgfpathlineto{\pgfqpoint{5.142617in}{4.507983in}}%
\pgfpathlineto{\pgfqpoint{4.992447in}{4.242332in}}%
\pgfpathlineto{\pgfqpoint{4.842278in}{4.119567in}}%
\pgfpathlineto{\pgfqpoint{4.692109in}{3.765147in}}%
\pgfpathlineto{\pgfqpoint{4.541940in}{3.581583in}}%
\pgfpathclose%
\pgfusepath{stroke,fill}%
\end{pgfscope}%
\begin{pgfscope}%
\pgfpathrectangle{\pgfqpoint{4.382652in}{3.312222in}}{\pgfqpoint{3.321958in}{2.324444in}}%
\pgfusepath{clip}%
\pgfsetbuttcap%
\pgfsetroundjoin%
\definecolor{currentfill}{rgb}{0.121569,0.466667,0.705882}%
\pgfsetfillcolor{currentfill}%
\pgfsetlinewidth{1.003750pt}%
\definecolor{currentstroke}{rgb}{0.121569,0.466667,0.705882}%
\pgfsetstrokecolor{currentstroke}%
\pgfsetdash{}{0pt}%
\pgfsys@defobject{currentmarker}{\pgfqpoint{-0.009821in}{-0.009821in}}{\pgfqpoint{0.009821in}{0.009821in}}{%
\pgfpathmoveto{\pgfqpoint{0.000000in}{-0.009821in}}%
\pgfpathcurveto{\pgfqpoint{0.002605in}{-0.009821in}}{\pgfqpoint{0.005103in}{-0.008786in}}{\pgfqpoint{0.006944in}{-0.006944in}}%
\pgfpathcurveto{\pgfqpoint{0.008786in}{-0.005103in}}{\pgfqpoint{0.009821in}{-0.002605in}}{\pgfqpoint{0.009821in}{0.000000in}}%
\pgfpathcurveto{\pgfqpoint{0.009821in}{0.002605in}}{\pgfqpoint{0.008786in}{0.005103in}}{\pgfqpoint{0.006944in}{0.006944in}}%
\pgfpathcurveto{\pgfqpoint{0.005103in}{0.008786in}}{\pgfqpoint{0.002605in}{0.009821in}}{\pgfqpoint{0.000000in}{0.009821in}}%
\pgfpathcurveto{\pgfqpoint{-0.002605in}{0.009821in}}{\pgfqpoint{-0.005103in}{0.008786in}}{\pgfqpoint{-0.006944in}{0.006944in}}%
\pgfpathcurveto{\pgfqpoint{-0.008786in}{0.005103in}}{\pgfqpoint{-0.009821in}{0.002605in}}{\pgfqpoint{-0.009821in}{0.000000in}}%
\pgfpathcurveto{\pgfqpoint{-0.009821in}{-0.002605in}}{\pgfqpoint{-0.008786in}{-0.005103in}}{\pgfqpoint{-0.006944in}{-0.006944in}}%
\pgfpathcurveto{\pgfqpoint{-0.005103in}{-0.008786in}}{\pgfqpoint{-0.002605in}{-0.009821in}}{\pgfqpoint{0.000000in}{-0.009821in}}%
\pgfpathclose%
\pgfusepath{stroke,fill}%
}%
\begin{pgfscope}%
\pgfsys@transformshift{4.541940in}{3.499731in}%
\pgfsys@useobject{currentmarker}{}%
\end{pgfscope}%
\begin{pgfscope}%
\pgfsys@transformshift{4.692109in}{3.709415in}%
\pgfsys@useobject{currentmarker}{}%
\end{pgfscope}%
\begin{pgfscope}%
\pgfsys@transformshift{4.842278in}{3.947578in}%
\pgfsys@useobject{currentmarker}{}%
\end{pgfscope}%
\begin{pgfscope}%
\pgfsys@transformshift{4.992447in}{4.170620in}%
\pgfsys@useobject{currentmarker}{}%
\end{pgfscope}%
\begin{pgfscope}%
\pgfsys@transformshift{5.142617in}{4.373214in}%
\pgfsys@useobject{currentmarker}{}%
\end{pgfscope}%
\begin{pgfscope}%
\pgfsys@transformshift{5.292786in}{4.638401in}%
\pgfsys@useobject{currentmarker}{}%
\end{pgfscope}%
\begin{pgfscope}%
\pgfsys@transformshift{5.442955in}{4.832809in}%
\pgfsys@useobject{currentmarker}{}%
\end{pgfscope}%
\begin{pgfscope}%
\pgfsys@transformshift{5.593124in}{5.055768in}%
\pgfsys@useobject{currentmarker}{}%
\end{pgfscope}%
\begin{pgfscope}%
\pgfsys@transformshift{5.743293in}{5.260925in}%
\pgfsys@useobject{currentmarker}{}%
\end{pgfscope}%
\begin{pgfscope}%
\pgfsys@transformshift{5.893462in}{5.367958in}%
\pgfsys@useobject{currentmarker}{}%
\end{pgfscope}%
\begin{pgfscope}%
\pgfsys@transformshift{6.043631in}{5.449306in}%
\pgfsys@useobject{currentmarker}{}%
\end{pgfscope}%
\begin{pgfscope}%
\pgfsys@transformshift{6.193801in}{5.455380in}%
\pgfsys@useobject{currentmarker}{}%
\end{pgfscope}%
\begin{pgfscope}%
\pgfsys@transformshift{6.343970in}{5.319102in}%
\pgfsys@useobject{currentmarker}{}%
\end{pgfscope}%
\begin{pgfscope}%
\pgfsys@transformshift{6.494139in}{5.206367in}%
\pgfsys@useobject{currentmarker}{}%
\end{pgfscope}%
\begin{pgfscope}%
\pgfsys@transformshift{6.644308in}{5.010252in}%
\pgfsys@useobject{currentmarker}{}%
\end{pgfscope}%
\begin{pgfscope}%
\pgfsys@transformshift{6.794477in}{4.792369in}%
\pgfsys@useobject{currentmarker}{}%
\end{pgfscope}%
\begin{pgfscope}%
\pgfsys@transformshift{6.944646in}{4.570392in}%
\pgfsys@useobject{currentmarker}{}%
\end{pgfscope}%
\begin{pgfscope}%
\pgfsys@transformshift{7.094816in}{4.296902in}%
\pgfsys@useobject{currentmarker}{}%
\end{pgfscope}%
\begin{pgfscope}%
\pgfsys@transformshift{7.244985in}{3.998062in}%
\pgfsys@useobject{currentmarker}{}%
\end{pgfscope}%
\begin{pgfscope}%
\pgfsys@transformshift{7.395154in}{3.778213in}%
\pgfsys@useobject{currentmarker}{}%
\end{pgfscope}%
\begin{pgfscope}%
\pgfsys@transformshift{7.545323in}{3.503935in}%
\pgfsys@useobject{currentmarker}{}%
\end{pgfscope}%
\end{pgfscope}%
\begin{pgfscope}%
\pgfsetbuttcap%
\pgfsetroundjoin%
\definecolor{currentfill}{rgb}{0.000000,0.000000,0.000000}%
\pgfsetfillcolor{currentfill}%
\pgfsetlinewidth{0.803000pt}%
\definecolor{currentstroke}{rgb}{0.000000,0.000000,0.000000}%
\pgfsetstrokecolor{currentstroke}%
\pgfsetdash{}{0pt}%
\pgfsys@defobject{currentmarker}{\pgfqpoint{0.000000in}{-0.048611in}}{\pgfqpoint{0.000000in}{0.000000in}}{%
\pgfpathmoveto{\pgfqpoint{0.000000in}{0.000000in}}%
\pgfpathlineto{\pgfqpoint{0.000000in}{-0.048611in}}%
\pgfusepath{stroke,fill}%
}%
\begin{pgfscope}%
\pgfsys@transformshift{4.692109in}{3.312222in}%
\pgfsys@useobject{currentmarker}{}%
\end{pgfscope}%
\end{pgfscope}%
\begin{pgfscope}%
\definecolor{textcolor}{rgb}{0.000000,0.000000,0.000000}%
\pgfsetstrokecolor{textcolor}%
\pgfsetfillcolor{textcolor}%
\pgftext[x=4.692109in,y=3.215000in,,top]{\color{textcolor}\sffamily\fontsize{10.000000}{12.000000}\selectfont 2.20}%
\end{pgfscope}%
\begin{pgfscope}%
\pgfsetbuttcap%
\pgfsetroundjoin%
\definecolor{currentfill}{rgb}{0.000000,0.000000,0.000000}%
\pgfsetfillcolor{currentfill}%
\pgfsetlinewidth{0.803000pt}%
\definecolor{currentstroke}{rgb}{0.000000,0.000000,0.000000}%
\pgfsetstrokecolor{currentstroke}%
\pgfsetdash{}{0pt}%
\pgfsys@defobject{currentmarker}{\pgfqpoint{0.000000in}{-0.048611in}}{\pgfqpoint{0.000000in}{0.000000in}}{%
\pgfpathmoveto{\pgfqpoint{0.000000in}{0.000000in}}%
\pgfpathlineto{\pgfqpoint{0.000000in}{-0.048611in}}%
\pgfusepath{stroke,fill}%
}%
\begin{pgfscope}%
\pgfsys@transformshift{5.442955in}{3.312222in}%
\pgfsys@useobject{currentmarker}{}%
\end{pgfscope}%
\end{pgfscope}%
\begin{pgfscope}%
\definecolor{textcolor}{rgb}{0.000000,0.000000,0.000000}%
\pgfsetstrokecolor{textcolor}%
\pgfsetfillcolor{textcolor}%
\pgftext[x=5.442955in,y=3.215000in,,top]{\color{textcolor}\sffamily\fontsize{10.000000}{12.000000}\selectfont 2.25}%
\end{pgfscope}%
\begin{pgfscope}%
\pgfsetbuttcap%
\pgfsetroundjoin%
\definecolor{currentfill}{rgb}{0.000000,0.000000,0.000000}%
\pgfsetfillcolor{currentfill}%
\pgfsetlinewidth{0.803000pt}%
\definecolor{currentstroke}{rgb}{0.000000,0.000000,0.000000}%
\pgfsetstrokecolor{currentstroke}%
\pgfsetdash{}{0pt}%
\pgfsys@defobject{currentmarker}{\pgfqpoint{0.000000in}{-0.048611in}}{\pgfqpoint{0.000000in}{0.000000in}}{%
\pgfpathmoveto{\pgfqpoint{0.000000in}{0.000000in}}%
\pgfpathlineto{\pgfqpoint{0.000000in}{-0.048611in}}%
\pgfusepath{stroke,fill}%
}%
\begin{pgfscope}%
\pgfsys@transformshift{6.193801in}{3.312222in}%
\pgfsys@useobject{currentmarker}{}%
\end{pgfscope}%
\end{pgfscope}%
\begin{pgfscope}%
\definecolor{textcolor}{rgb}{0.000000,0.000000,0.000000}%
\pgfsetstrokecolor{textcolor}%
\pgfsetfillcolor{textcolor}%
\pgftext[x=6.193801in,y=3.215000in,,top]{\color{textcolor}\sffamily\fontsize{10.000000}{12.000000}\selectfont 2.30}%
\end{pgfscope}%
\begin{pgfscope}%
\pgfsetbuttcap%
\pgfsetroundjoin%
\definecolor{currentfill}{rgb}{0.000000,0.000000,0.000000}%
\pgfsetfillcolor{currentfill}%
\pgfsetlinewidth{0.803000pt}%
\definecolor{currentstroke}{rgb}{0.000000,0.000000,0.000000}%
\pgfsetstrokecolor{currentstroke}%
\pgfsetdash{}{0pt}%
\pgfsys@defobject{currentmarker}{\pgfqpoint{0.000000in}{-0.048611in}}{\pgfqpoint{0.000000in}{0.000000in}}{%
\pgfpathmoveto{\pgfqpoint{0.000000in}{0.000000in}}%
\pgfpathlineto{\pgfqpoint{0.000000in}{-0.048611in}}%
\pgfusepath{stroke,fill}%
}%
\begin{pgfscope}%
\pgfsys@transformshift{6.944646in}{3.312222in}%
\pgfsys@useobject{currentmarker}{}%
\end{pgfscope}%
\end{pgfscope}%
\begin{pgfscope}%
\definecolor{textcolor}{rgb}{0.000000,0.000000,0.000000}%
\pgfsetstrokecolor{textcolor}%
\pgfsetfillcolor{textcolor}%
\pgftext[x=6.944646in,y=3.215000in,,top]{\color{textcolor}\sffamily\fontsize{10.000000}{12.000000}\selectfont 2.35}%
\end{pgfscope}%
\begin{pgfscope}%
\pgfsetbuttcap%
\pgfsetroundjoin%
\definecolor{currentfill}{rgb}{0.000000,0.000000,0.000000}%
\pgfsetfillcolor{currentfill}%
\pgfsetlinewidth{0.803000pt}%
\definecolor{currentstroke}{rgb}{0.000000,0.000000,0.000000}%
\pgfsetstrokecolor{currentstroke}%
\pgfsetdash{}{0pt}%
\pgfsys@defobject{currentmarker}{\pgfqpoint{0.000000in}{-0.048611in}}{\pgfqpoint{0.000000in}{0.000000in}}{%
\pgfpathmoveto{\pgfqpoint{0.000000in}{0.000000in}}%
\pgfpathlineto{\pgfqpoint{0.000000in}{-0.048611in}}%
\pgfusepath{stroke,fill}%
}%
\begin{pgfscope}%
\pgfsys@transformshift{7.695492in}{3.312222in}%
\pgfsys@useobject{currentmarker}{}%
\end{pgfscope}%
\end{pgfscope}%
\begin{pgfscope}%
\definecolor{textcolor}{rgb}{0.000000,0.000000,0.000000}%
\pgfsetstrokecolor{textcolor}%
\pgfsetfillcolor{textcolor}%
\pgftext[x=7.695492in,y=3.215000in,,top]{\color{textcolor}\sffamily\fontsize{10.000000}{12.000000}\selectfont 2.40}%
\end{pgfscope}%
\begin{pgfscope}%
\pgfsetbuttcap%
\pgfsetroundjoin%
\definecolor{currentfill}{rgb}{0.000000,0.000000,0.000000}%
\pgfsetfillcolor{currentfill}%
\pgfsetlinewidth{0.803000pt}%
\definecolor{currentstroke}{rgb}{0.000000,0.000000,0.000000}%
\pgfsetstrokecolor{currentstroke}%
\pgfsetdash{}{0pt}%
\pgfsys@defobject{currentmarker}{\pgfqpoint{-0.048611in}{0.000000in}}{\pgfqpoint{0.000000in}{0.000000in}}{%
\pgfpathmoveto{\pgfqpoint{0.000000in}{0.000000in}}%
\pgfpathlineto{\pgfqpoint{-0.048611in}{0.000000in}}%
\pgfusepath{stroke,fill}%
}%
\begin{pgfscope}%
\pgfsys@transformshift{4.382652in}{3.351489in}%
\pgfsys@useobject{currentmarker}{}%
\end{pgfscope}%
\end{pgfscope}%
\begin{pgfscope}%
\definecolor{textcolor}{rgb}{0.000000,0.000000,0.000000}%
\pgfsetstrokecolor{textcolor}%
\pgfsetfillcolor{textcolor}%
\pgftext[x=4.064551in,y=3.298728in,left,base]{\color{textcolor}\sffamily\fontsize{10.000000}{12.000000}\selectfont 1.3}%
\end{pgfscope}%
\begin{pgfscope}%
\pgfsetbuttcap%
\pgfsetroundjoin%
\definecolor{currentfill}{rgb}{0.000000,0.000000,0.000000}%
\pgfsetfillcolor{currentfill}%
\pgfsetlinewidth{0.803000pt}%
\definecolor{currentstroke}{rgb}{0.000000,0.000000,0.000000}%
\pgfsetstrokecolor{currentstroke}%
\pgfsetdash{}{0pt}%
\pgfsys@defobject{currentmarker}{\pgfqpoint{-0.048611in}{0.000000in}}{\pgfqpoint{0.000000in}{0.000000in}}{%
\pgfpathmoveto{\pgfqpoint{0.000000in}{0.000000in}}%
\pgfpathlineto{\pgfqpoint{-0.048611in}{0.000000in}}%
\pgfusepath{stroke,fill}%
}%
\begin{pgfscope}%
\pgfsys@transformshift{4.382652in}{3.698773in}%
\pgfsys@useobject{currentmarker}{}%
\end{pgfscope}%
\end{pgfscope}%
\begin{pgfscope}%
\definecolor{textcolor}{rgb}{0.000000,0.000000,0.000000}%
\pgfsetstrokecolor{textcolor}%
\pgfsetfillcolor{textcolor}%
\pgftext[x=4.064551in,y=3.646011in,left,base]{\color{textcolor}\sffamily\fontsize{10.000000}{12.000000}\selectfont 1.4}%
\end{pgfscope}%
\begin{pgfscope}%
\pgfsetbuttcap%
\pgfsetroundjoin%
\definecolor{currentfill}{rgb}{0.000000,0.000000,0.000000}%
\pgfsetfillcolor{currentfill}%
\pgfsetlinewidth{0.803000pt}%
\definecolor{currentstroke}{rgb}{0.000000,0.000000,0.000000}%
\pgfsetstrokecolor{currentstroke}%
\pgfsetdash{}{0pt}%
\pgfsys@defobject{currentmarker}{\pgfqpoint{-0.048611in}{0.000000in}}{\pgfqpoint{0.000000in}{0.000000in}}{%
\pgfpathmoveto{\pgfqpoint{0.000000in}{0.000000in}}%
\pgfpathlineto{\pgfqpoint{-0.048611in}{0.000000in}}%
\pgfusepath{stroke,fill}%
}%
\begin{pgfscope}%
\pgfsys@transformshift{4.382652in}{4.046056in}%
\pgfsys@useobject{currentmarker}{}%
\end{pgfscope}%
\end{pgfscope}%
\begin{pgfscope}%
\definecolor{textcolor}{rgb}{0.000000,0.000000,0.000000}%
\pgfsetstrokecolor{textcolor}%
\pgfsetfillcolor{textcolor}%
\pgftext[x=4.064551in,y=3.993294in,left,base]{\color{textcolor}\sffamily\fontsize{10.000000}{12.000000}\selectfont 1.5}%
\end{pgfscope}%
\begin{pgfscope}%
\pgfsetbuttcap%
\pgfsetroundjoin%
\definecolor{currentfill}{rgb}{0.000000,0.000000,0.000000}%
\pgfsetfillcolor{currentfill}%
\pgfsetlinewidth{0.803000pt}%
\definecolor{currentstroke}{rgb}{0.000000,0.000000,0.000000}%
\pgfsetstrokecolor{currentstroke}%
\pgfsetdash{}{0pt}%
\pgfsys@defobject{currentmarker}{\pgfqpoint{-0.048611in}{0.000000in}}{\pgfqpoint{0.000000in}{0.000000in}}{%
\pgfpathmoveto{\pgfqpoint{0.000000in}{0.000000in}}%
\pgfpathlineto{\pgfqpoint{-0.048611in}{0.000000in}}%
\pgfusepath{stroke,fill}%
}%
\begin{pgfscope}%
\pgfsys@transformshift{4.382652in}{4.393339in}%
\pgfsys@useobject{currentmarker}{}%
\end{pgfscope}%
\end{pgfscope}%
\begin{pgfscope}%
\definecolor{textcolor}{rgb}{0.000000,0.000000,0.000000}%
\pgfsetstrokecolor{textcolor}%
\pgfsetfillcolor{textcolor}%
\pgftext[x=4.064551in,y=4.340578in,left,base]{\color{textcolor}\sffamily\fontsize{10.000000}{12.000000}\selectfont 1.6}%
\end{pgfscope}%
\begin{pgfscope}%
\pgfsetbuttcap%
\pgfsetroundjoin%
\definecolor{currentfill}{rgb}{0.000000,0.000000,0.000000}%
\pgfsetfillcolor{currentfill}%
\pgfsetlinewidth{0.803000pt}%
\definecolor{currentstroke}{rgb}{0.000000,0.000000,0.000000}%
\pgfsetstrokecolor{currentstroke}%
\pgfsetdash{}{0pt}%
\pgfsys@defobject{currentmarker}{\pgfqpoint{-0.048611in}{0.000000in}}{\pgfqpoint{0.000000in}{0.000000in}}{%
\pgfpathmoveto{\pgfqpoint{0.000000in}{0.000000in}}%
\pgfpathlineto{\pgfqpoint{-0.048611in}{0.000000in}}%
\pgfusepath{stroke,fill}%
}%
\begin{pgfscope}%
\pgfsys@transformshift{4.382652in}{4.740623in}%
\pgfsys@useobject{currentmarker}{}%
\end{pgfscope}%
\end{pgfscope}%
\begin{pgfscope}%
\definecolor{textcolor}{rgb}{0.000000,0.000000,0.000000}%
\pgfsetstrokecolor{textcolor}%
\pgfsetfillcolor{textcolor}%
\pgftext[x=4.064551in,y=4.687861in,left,base]{\color{textcolor}\sffamily\fontsize{10.000000}{12.000000}\selectfont 1.7}%
\end{pgfscope}%
\begin{pgfscope}%
\pgfsetbuttcap%
\pgfsetroundjoin%
\definecolor{currentfill}{rgb}{0.000000,0.000000,0.000000}%
\pgfsetfillcolor{currentfill}%
\pgfsetlinewidth{0.803000pt}%
\definecolor{currentstroke}{rgb}{0.000000,0.000000,0.000000}%
\pgfsetstrokecolor{currentstroke}%
\pgfsetdash{}{0pt}%
\pgfsys@defobject{currentmarker}{\pgfqpoint{-0.048611in}{0.000000in}}{\pgfqpoint{0.000000in}{0.000000in}}{%
\pgfpathmoveto{\pgfqpoint{0.000000in}{0.000000in}}%
\pgfpathlineto{\pgfqpoint{-0.048611in}{0.000000in}}%
\pgfusepath{stroke,fill}%
}%
\begin{pgfscope}%
\pgfsys@transformshift{4.382652in}{5.087906in}%
\pgfsys@useobject{currentmarker}{}%
\end{pgfscope}%
\end{pgfscope}%
\begin{pgfscope}%
\definecolor{textcolor}{rgb}{0.000000,0.000000,0.000000}%
\pgfsetstrokecolor{textcolor}%
\pgfsetfillcolor{textcolor}%
\pgftext[x=4.064551in,y=5.035145in,left,base]{\color{textcolor}\sffamily\fontsize{10.000000}{12.000000}\selectfont 1.8}%
\end{pgfscope}%
\begin{pgfscope}%
\pgfsetbuttcap%
\pgfsetroundjoin%
\definecolor{currentfill}{rgb}{0.000000,0.000000,0.000000}%
\pgfsetfillcolor{currentfill}%
\pgfsetlinewidth{0.803000pt}%
\definecolor{currentstroke}{rgb}{0.000000,0.000000,0.000000}%
\pgfsetstrokecolor{currentstroke}%
\pgfsetdash{}{0pt}%
\pgfsys@defobject{currentmarker}{\pgfqpoint{-0.048611in}{0.000000in}}{\pgfqpoint{0.000000in}{0.000000in}}{%
\pgfpathmoveto{\pgfqpoint{0.000000in}{0.000000in}}%
\pgfpathlineto{\pgfqpoint{-0.048611in}{0.000000in}}%
\pgfusepath{stroke,fill}%
}%
\begin{pgfscope}%
\pgfsys@transformshift{4.382652in}{5.435190in}%
\pgfsys@useobject{currentmarker}{}%
\end{pgfscope}%
\end{pgfscope}%
\begin{pgfscope}%
\definecolor{textcolor}{rgb}{0.000000,0.000000,0.000000}%
\pgfsetstrokecolor{textcolor}%
\pgfsetfillcolor{textcolor}%
\pgftext[x=4.064551in,y=5.382428in,left,base]{\color{textcolor}\sffamily\fontsize{10.000000}{12.000000}\selectfont 1.9}%
\end{pgfscope}%
\begin{pgfscope}%
\pgfpathrectangle{\pgfqpoint{4.382652in}{3.312222in}}{\pgfqpoint{3.321958in}{2.324444in}}%
\pgfusepath{clip}%
\pgfsetrectcap%
\pgfsetroundjoin%
\pgfsetlinewidth{1.505625pt}%
\definecolor{currentstroke}{rgb}{0.121569,0.466667,0.705882}%
\pgfsetstrokecolor{currentstroke}%
\pgfsetdash{}{0pt}%
\pgfpathmoveto{\pgfqpoint{4.541940in}{3.499731in}}%
\pgfpathlineto{\pgfqpoint{4.692109in}{3.709415in}}%
\pgfpathlineto{\pgfqpoint{4.842278in}{3.947578in}}%
\pgfpathlineto{\pgfqpoint{4.992447in}{4.170620in}}%
\pgfpathlineto{\pgfqpoint{5.142617in}{4.373214in}}%
\pgfpathlineto{\pgfqpoint{5.292786in}{4.638401in}}%
\pgfpathlineto{\pgfqpoint{5.442955in}{4.832809in}}%
\pgfpathlineto{\pgfqpoint{5.593124in}{5.055768in}}%
\pgfpathlineto{\pgfqpoint{5.743293in}{5.260925in}}%
\pgfpathlineto{\pgfqpoint{5.893462in}{5.367958in}}%
\pgfpathlineto{\pgfqpoint{6.043631in}{5.449306in}}%
\pgfpathlineto{\pgfqpoint{6.193801in}{5.455380in}}%
\pgfpathlineto{\pgfqpoint{6.343970in}{5.319102in}}%
\pgfpathlineto{\pgfqpoint{6.494139in}{5.206367in}}%
\pgfpathlineto{\pgfqpoint{6.644308in}{5.010252in}}%
\pgfpathlineto{\pgfqpoint{6.794477in}{4.792369in}}%
\pgfpathlineto{\pgfqpoint{6.944646in}{4.570392in}}%
\pgfpathlineto{\pgfqpoint{7.094816in}{4.296902in}}%
\pgfpathlineto{\pgfqpoint{7.244985in}{3.998062in}}%
\pgfpathlineto{\pgfqpoint{7.395154in}{3.778213in}}%
\pgfpathlineto{\pgfqpoint{7.545323in}{3.503935in}}%
\pgfusepath{stroke}%
\end{pgfscope}%
\begin{pgfscope}%
\pgfsetrectcap%
\pgfsetmiterjoin%
\pgfsetlinewidth{0.803000pt}%
\definecolor{currentstroke}{rgb}{0.000000,0.000000,0.000000}%
\pgfsetstrokecolor{currentstroke}%
\pgfsetdash{}{0pt}%
\pgfpathmoveto{\pgfqpoint{4.382652in}{3.312222in}}%
\pgfpathlineto{\pgfqpoint{4.382652in}{5.636667in}}%
\pgfusepath{stroke}%
\end{pgfscope}%
\begin{pgfscope}%
\pgfsetrectcap%
\pgfsetmiterjoin%
\pgfsetlinewidth{0.803000pt}%
\definecolor{currentstroke}{rgb}{0.000000,0.000000,0.000000}%
\pgfsetstrokecolor{currentstroke}%
\pgfsetdash{}{0pt}%
\pgfpathmoveto{\pgfqpoint{7.704610in}{3.312222in}}%
\pgfpathlineto{\pgfqpoint{7.704610in}{5.636667in}}%
\pgfusepath{stroke}%
\end{pgfscope}%
\begin{pgfscope}%
\pgfsetrectcap%
\pgfsetmiterjoin%
\pgfsetlinewidth{0.803000pt}%
\definecolor{currentstroke}{rgb}{0.000000,0.000000,0.000000}%
\pgfsetstrokecolor{currentstroke}%
\pgfsetdash{}{0pt}%
\pgfpathmoveto{\pgfqpoint{4.382652in}{3.312222in}}%
\pgfpathlineto{\pgfqpoint{7.704610in}{3.312222in}}%
\pgfusepath{stroke}%
\end{pgfscope}%
\begin{pgfscope}%
\pgfsetrectcap%
\pgfsetmiterjoin%
\pgfsetlinewidth{0.803000pt}%
\definecolor{currentstroke}{rgb}{0.000000,0.000000,0.000000}%
\pgfsetstrokecolor{currentstroke}%
\pgfsetdash{}{0pt}%
\pgfpathmoveto{\pgfqpoint{4.382652in}{5.636667in}}%
\pgfpathlineto{\pgfqpoint{7.704610in}{5.636667in}}%
\pgfusepath{stroke}%
\end{pgfscope}%
\begin{pgfscope}%
\definecolor{textcolor}{rgb}{0.000000,0.000000,0.000000}%
\pgfsetstrokecolor{textcolor}%
\pgfsetfillcolor{textcolor}%
\pgftext[x=6.043631in,y=5.720000in,,base]{\color{textcolor}\sffamily\fontsize{12.000000}{14.400000}\selectfont \(\displaystyle  N = 32 \)}%
\end{pgfscope}%
\begin{pgfscope}%
\pgfsetbuttcap%
\pgfsetmiterjoin%
\definecolor{currentfill}{rgb}{1.000000,1.000000,1.000000}%
\pgfsetfillcolor{currentfill}%
\pgfsetlinewidth{0.000000pt}%
\definecolor{currentstroke}{rgb}{0.000000,0.000000,0.000000}%
\pgfsetstrokecolor{currentstroke}%
\pgfsetstrokeopacity{0.000000}%
\pgfsetdash{}{0pt}%
\pgfpathmoveto{\pgfqpoint{0.557222in}{0.387222in}}%
\pgfpathlineto{\pgfqpoint{3.879180in}{0.387222in}}%
\pgfpathlineto{\pgfqpoint{3.879180in}{2.711667in}}%
\pgfpathlineto{\pgfqpoint{0.557222in}{2.711667in}}%
\pgfpathclose%
\pgfusepath{fill}%
\end{pgfscope}%
\begin{pgfscope}%
\pgfpathrectangle{\pgfqpoint{0.557222in}{0.387222in}}{\pgfqpoint{3.321958in}{2.324444in}}%
\pgfusepath{clip}%
\pgfsetbuttcap%
\pgfsetroundjoin%
\definecolor{currentfill}{rgb}{0.121569,0.466667,0.705882}%
\pgfsetfillcolor{currentfill}%
\pgfsetfillopacity{0.300000}%
\pgfsetlinewidth{1.003750pt}%
\definecolor{currentstroke}{rgb}{0.121569,0.466667,0.705882}%
\pgfsetstrokecolor{currentstroke}%
\pgfsetstrokeopacity{0.300000}%
\pgfsetdash{}{0pt}%
\pgfpathmoveto{\pgfqpoint{0.716510in}{0.703317in}}%
\pgfpathlineto{\pgfqpoint{0.716510in}{0.654860in}}%
\pgfpathlineto{\pgfqpoint{0.866679in}{0.631153in}}%
\pgfpathlineto{\pgfqpoint{1.016848in}{0.849854in}}%
\pgfpathlineto{\pgfqpoint{1.167017in}{0.951825in}}%
\pgfpathlineto{\pgfqpoint{1.317186in}{1.119351in}}%
\pgfpathlineto{\pgfqpoint{1.467355in}{1.290740in}}%
\pgfpathlineto{\pgfqpoint{1.617525in}{1.579656in}}%
\pgfpathlineto{\pgfqpoint{1.767694in}{1.656213in}}%
\pgfpathlineto{\pgfqpoint{1.917863in}{1.840844in}}%
\pgfpathlineto{\pgfqpoint{2.068032in}{2.176651in}}%
\pgfpathlineto{\pgfqpoint{2.218201in}{2.263501in}}%
\pgfpathlineto{\pgfqpoint{2.368370in}{2.080174in}}%
\pgfpathlineto{\pgfqpoint{2.518540in}{1.974004in}}%
\pgfpathlineto{\pgfqpoint{2.668709in}{1.716387in}}%
\pgfpathlineto{\pgfqpoint{2.818878in}{1.478430in}}%
\pgfpathlineto{\pgfqpoint{2.969047in}{1.202092in}}%
\pgfpathlineto{\pgfqpoint{3.119216in}{0.996422in}}%
\pgfpathlineto{\pgfqpoint{3.269385in}{0.870649in}}%
\pgfpathlineto{\pgfqpoint{3.419554in}{0.750612in}}%
\pgfpathlineto{\pgfqpoint{3.569724in}{0.589942in}}%
\pgfpathlineto{\pgfqpoint{3.719893in}{0.492879in}}%
\pgfpathlineto{\pgfqpoint{3.719893in}{0.601276in}}%
\pgfpathlineto{\pgfqpoint{3.719893in}{0.601276in}}%
\pgfpathlineto{\pgfqpoint{3.569724in}{0.716850in}}%
\pgfpathlineto{\pgfqpoint{3.419554in}{0.854919in}}%
\pgfpathlineto{\pgfqpoint{3.269385in}{0.977448in}}%
\pgfpathlineto{\pgfqpoint{3.119216in}{1.186414in}}%
\pgfpathlineto{\pgfqpoint{2.969047in}{1.371309in}}%
\pgfpathlineto{\pgfqpoint{2.818878in}{1.697652in}}%
\pgfpathlineto{\pgfqpoint{2.668709in}{2.023741in}}%
\pgfpathlineto{\pgfqpoint{2.518540in}{2.289246in}}%
\pgfpathlineto{\pgfqpoint{2.368370in}{2.606010in}}%
\pgfpathlineto{\pgfqpoint{2.218201in}{2.572791in}}%
\pgfpathlineto{\pgfqpoint{2.068032in}{2.460314in}}%
\pgfpathlineto{\pgfqpoint{1.917863in}{2.202607in}}%
\pgfpathlineto{\pgfqpoint{1.767694in}{2.061706in}}%
\pgfpathlineto{\pgfqpoint{1.617525in}{1.808625in}}%
\pgfpathlineto{\pgfqpoint{1.467355in}{1.558203in}}%
\pgfpathlineto{\pgfqpoint{1.317186in}{1.330756in}}%
\pgfpathlineto{\pgfqpoint{1.167017in}{1.088928in}}%
\pgfpathlineto{\pgfqpoint{1.016848in}{0.977702in}}%
\pgfpathlineto{\pgfqpoint{0.866679in}{0.864300in}}%
\pgfpathlineto{\pgfqpoint{0.716510in}{0.703317in}}%
\pgfpathclose%
\pgfusepath{stroke,fill}%
\end{pgfscope}%
\begin{pgfscope}%
\pgfpathrectangle{\pgfqpoint{0.557222in}{0.387222in}}{\pgfqpoint{3.321958in}{2.324444in}}%
\pgfusepath{clip}%
\pgfsetbuttcap%
\pgfsetroundjoin%
\definecolor{currentfill}{rgb}{0.121569,0.466667,0.705882}%
\pgfsetfillcolor{currentfill}%
\pgfsetlinewidth{1.003750pt}%
\definecolor{currentstroke}{rgb}{0.121569,0.466667,0.705882}%
\pgfsetstrokecolor{currentstroke}%
\pgfsetdash{}{0pt}%
\pgfsys@defobject{currentmarker}{\pgfqpoint{-0.009821in}{-0.009821in}}{\pgfqpoint{0.009821in}{0.009821in}}{%
\pgfpathmoveto{\pgfqpoint{0.000000in}{-0.009821in}}%
\pgfpathcurveto{\pgfqpoint{0.002605in}{-0.009821in}}{\pgfqpoint{0.005103in}{-0.008786in}}{\pgfqpoint{0.006944in}{-0.006944in}}%
\pgfpathcurveto{\pgfqpoint{0.008786in}{-0.005103in}}{\pgfqpoint{0.009821in}{-0.002605in}}{\pgfqpoint{0.009821in}{0.000000in}}%
\pgfpathcurveto{\pgfqpoint{0.009821in}{0.002605in}}{\pgfqpoint{0.008786in}{0.005103in}}{\pgfqpoint{0.006944in}{0.006944in}}%
\pgfpathcurveto{\pgfqpoint{0.005103in}{0.008786in}}{\pgfqpoint{0.002605in}{0.009821in}}{\pgfqpoint{0.000000in}{0.009821in}}%
\pgfpathcurveto{\pgfqpoint{-0.002605in}{0.009821in}}{\pgfqpoint{-0.005103in}{0.008786in}}{\pgfqpoint{-0.006944in}{0.006944in}}%
\pgfpathcurveto{\pgfqpoint{-0.008786in}{0.005103in}}{\pgfqpoint{-0.009821in}{0.002605in}}{\pgfqpoint{-0.009821in}{0.000000in}}%
\pgfpathcurveto{\pgfqpoint{-0.009821in}{-0.002605in}}{\pgfqpoint{-0.008786in}{-0.005103in}}{\pgfqpoint{-0.006944in}{-0.006944in}}%
\pgfpathcurveto{\pgfqpoint{-0.005103in}{-0.008786in}}{\pgfqpoint{-0.002605in}{-0.009821in}}{\pgfqpoint{0.000000in}{-0.009821in}}%
\pgfpathclose%
\pgfusepath{stroke,fill}%
}%
\begin{pgfscope}%
\pgfsys@transformshift{0.716510in}{0.679089in}%
\pgfsys@useobject{currentmarker}{}%
\end{pgfscope}%
\begin{pgfscope}%
\pgfsys@transformshift{0.866679in}{0.747726in}%
\pgfsys@useobject{currentmarker}{}%
\end{pgfscope}%
\begin{pgfscope}%
\pgfsys@transformshift{1.016848in}{0.913778in}%
\pgfsys@useobject{currentmarker}{}%
\end{pgfscope}%
\begin{pgfscope}%
\pgfsys@transformshift{1.167017in}{1.020377in}%
\pgfsys@useobject{currentmarker}{}%
\end{pgfscope}%
\begin{pgfscope}%
\pgfsys@transformshift{1.317186in}{1.225053in}%
\pgfsys@useobject{currentmarker}{}%
\end{pgfscope}%
\begin{pgfscope}%
\pgfsys@transformshift{1.467355in}{1.424472in}%
\pgfsys@useobject{currentmarker}{}%
\end{pgfscope}%
\begin{pgfscope}%
\pgfsys@transformshift{1.617525in}{1.694141in}%
\pgfsys@useobject{currentmarker}{}%
\end{pgfscope}%
\begin{pgfscope}%
\pgfsys@transformshift{1.767694in}{1.858960in}%
\pgfsys@useobject{currentmarker}{}%
\end{pgfscope}%
\begin{pgfscope}%
\pgfsys@transformshift{1.917863in}{2.021725in}%
\pgfsys@useobject{currentmarker}{}%
\end{pgfscope}%
\begin{pgfscope}%
\pgfsys@transformshift{2.068032in}{2.318482in}%
\pgfsys@useobject{currentmarker}{}%
\end{pgfscope}%
\begin{pgfscope}%
\pgfsys@transformshift{2.218201in}{2.418146in}%
\pgfsys@useobject{currentmarker}{}%
\end{pgfscope}%
\begin{pgfscope}%
\pgfsys@transformshift{2.368370in}{2.343092in}%
\pgfsys@useobject{currentmarker}{}%
\end{pgfscope}%
\begin{pgfscope}%
\pgfsys@transformshift{2.518540in}{2.131625in}%
\pgfsys@useobject{currentmarker}{}%
\end{pgfscope}%
\begin{pgfscope}%
\pgfsys@transformshift{2.668709in}{1.870064in}%
\pgfsys@useobject{currentmarker}{}%
\end{pgfscope}%
\begin{pgfscope}%
\pgfsys@transformshift{2.818878in}{1.588041in}%
\pgfsys@useobject{currentmarker}{}%
\end{pgfscope}%
\begin{pgfscope}%
\pgfsys@transformshift{2.969047in}{1.286701in}%
\pgfsys@useobject{currentmarker}{}%
\end{pgfscope}%
\begin{pgfscope}%
\pgfsys@transformshift{3.119216in}{1.091418in}%
\pgfsys@useobject{currentmarker}{}%
\end{pgfscope}%
\begin{pgfscope}%
\pgfsys@transformshift{3.269385in}{0.924049in}%
\pgfsys@useobject{currentmarker}{}%
\end{pgfscope}%
\begin{pgfscope}%
\pgfsys@transformshift{3.419554in}{0.802765in}%
\pgfsys@useobject{currentmarker}{}%
\end{pgfscope}%
\begin{pgfscope}%
\pgfsys@transformshift{3.569724in}{0.653396in}%
\pgfsys@useobject{currentmarker}{}%
\end{pgfscope}%
\begin{pgfscope}%
\pgfsys@transformshift{3.719893in}{0.547077in}%
\pgfsys@useobject{currentmarker}{}%
\end{pgfscope}%
\end{pgfscope}%
\begin{pgfscope}%
\pgfsetbuttcap%
\pgfsetroundjoin%
\definecolor{currentfill}{rgb}{0.000000,0.000000,0.000000}%
\pgfsetfillcolor{currentfill}%
\pgfsetlinewidth{0.803000pt}%
\definecolor{currentstroke}{rgb}{0.000000,0.000000,0.000000}%
\pgfsetstrokecolor{currentstroke}%
\pgfsetdash{}{0pt}%
\pgfsys@defobject{currentmarker}{\pgfqpoint{0.000000in}{-0.048611in}}{\pgfqpoint{0.000000in}{0.000000in}}{%
\pgfpathmoveto{\pgfqpoint{0.000000in}{0.000000in}}%
\pgfpathlineto{\pgfqpoint{0.000000in}{-0.048611in}}%
\pgfusepath{stroke,fill}%
}%
\begin{pgfscope}%
\pgfsys@transformshift{1.016848in}{0.387222in}%
\pgfsys@useobject{currentmarker}{}%
\end{pgfscope}%
\end{pgfscope}%
\begin{pgfscope}%
\definecolor{textcolor}{rgb}{0.000000,0.000000,0.000000}%
\pgfsetstrokecolor{textcolor}%
\pgfsetfillcolor{textcolor}%
\pgftext[x=1.016848in,y=0.290000in,,top]{\color{textcolor}\sffamily\fontsize{10.000000}{12.000000}\selectfont 2.20}%
\end{pgfscope}%
\begin{pgfscope}%
\pgfsetbuttcap%
\pgfsetroundjoin%
\definecolor{currentfill}{rgb}{0.000000,0.000000,0.000000}%
\pgfsetfillcolor{currentfill}%
\pgfsetlinewidth{0.803000pt}%
\definecolor{currentstroke}{rgb}{0.000000,0.000000,0.000000}%
\pgfsetstrokecolor{currentstroke}%
\pgfsetdash{}{0pt}%
\pgfsys@defobject{currentmarker}{\pgfqpoint{0.000000in}{-0.048611in}}{\pgfqpoint{0.000000in}{0.000000in}}{%
\pgfpathmoveto{\pgfqpoint{0.000000in}{0.000000in}}%
\pgfpathlineto{\pgfqpoint{0.000000in}{-0.048611in}}%
\pgfusepath{stroke,fill}%
}%
\begin{pgfscope}%
\pgfsys@transformshift{1.767694in}{0.387222in}%
\pgfsys@useobject{currentmarker}{}%
\end{pgfscope}%
\end{pgfscope}%
\begin{pgfscope}%
\definecolor{textcolor}{rgb}{0.000000,0.000000,0.000000}%
\pgfsetstrokecolor{textcolor}%
\pgfsetfillcolor{textcolor}%
\pgftext[x=1.767694in,y=0.290000in,,top]{\color{textcolor}\sffamily\fontsize{10.000000}{12.000000}\selectfont 2.25}%
\end{pgfscope}%
\begin{pgfscope}%
\pgfsetbuttcap%
\pgfsetroundjoin%
\definecolor{currentfill}{rgb}{0.000000,0.000000,0.000000}%
\pgfsetfillcolor{currentfill}%
\pgfsetlinewidth{0.803000pt}%
\definecolor{currentstroke}{rgb}{0.000000,0.000000,0.000000}%
\pgfsetstrokecolor{currentstroke}%
\pgfsetdash{}{0pt}%
\pgfsys@defobject{currentmarker}{\pgfqpoint{0.000000in}{-0.048611in}}{\pgfqpoint{0.000000in}{0.000000in}}{%
\pgfpathmoveto{\pgfqpoint{0.000000in}{0.000000in}}%
\pgfpathlineto{\pgfqpoint{0.000000in}{-0.048611in}}%
\pgfusepath{stroke,fill}%
}%
\begin{pgfscope}%
\pgfsys@transformshift{2.518540in}{0.387222in}%
\pgfsys@useobject{currentmarker}{}%
\end{pgfscope}%
\end{pgfscope}%
\begin{pgfscope}%
\definecolor{textcolor}{rgb}{0.000000,0.000000,0.000000}%
\pgfsetstrokecolor{textcolor}%
\pgfsetfillcolor{textcolor}%
\pgftext[x=2.518540in,y=0.290000in,,top]{\color{textcolor}\sffamily\fontsize{10.000000}{12.000000}\selectfont 2.30}%
\end{pgfscope}%
\begin{pgfscope}%
\pgfsetbuttcap%
\pgfsetroundjoin%
\definecolor{currentfill}{rgb}{0.000000,0.000000,0.000000}%
\pgfsetfillcolor{currentfill}%
\pgfsetlinewidth{0.803000pt}%
\definecolor{currentstroke}{rgb}{0.000000,0.000000,0.000000}%
\pgfsetstrokecolor{currentstroke}%
\pgfsetdash{}{0pt}%
\pgfsys@defobject{currentmarker}{\pgfqpoint{0.000000in}{-0.048611in}}{\pgfqpoint{0.000000in}{0.000000in}}{%
\pgfpathmoveto{\pgfqpoint{0.000000in}{0.000000in}}%
\pgfpathlineto{\pgfqpoint{0.000000in}{-0.048611in}}%
\pgfusepath{stroke,fill}%
}%
\begin{pgfscope}%
\pgfsys@transformshift{3.269385in}{0.387222in}%
\pgfsys@useobject{currentmarker}{}%
\end{pgfscope}%
\end{pgfscope}%
\begin{pgfscope}%
\definecolor{textcolor}{rgb}{0.000000,0.000000,0.000000}%
\pgfsetstrokecolor{textcolor}%
\pgfsetfillcolor{textcolor}%
\pgftext[x=3.269385in,y=0.290000in,,top]{\color{textcolor}\sffamily\fontsize{10.000000}{12.000000}\selectfont 2.35}%
\end{pgfscope}%
\begin{pgfscope}%
\pgfsetbuttcap%
\pgfsetroundjoin%
\definecolor{currentfill}{rgb}{0.000000,0.000000,0.000000}%
\pgfsetfillcolor{currentfill}%
\pgfsetlinewidth{0.803000pt}%
\definecolor{currentstroke}{rgb}{0.000000,0.000000,0.000000}%
\pgfsetstrokecolor{currentstroke}%
\pgfsetdash{}{0pt}%
\pgfsys@defobject{currentmarker}{\pgfqpoint{-0.048611in}{0.000000in}}{\pgfqpoint{0.000000in}{0.000000in}}{%
\pgfpathmoveto{\pgfqpoint{0.000000in}{0.000000in}}%
\pgfpathlineto{\pgfqpoint{-0.048611in}{0.000000in}}%
\pgfusepath{stroke,fill}%
}%
\begin{pgfscope}%
\pgfsys@transformshift{0.557222in}{0.487270in}%
\pgfsys@useobject{currentmarker}{}%
\end{pgfscope}%
\end{pgfscope}%
\begin{pgfscope}%
\definecolor{textcolor}{rgb}{0.000000,0.000000,0.000000}%
\pgfsetstrokecolor{textcolor}%
\pgfsetfillcolor{textcolor}%
\pgftext[x=0.239121in,y=0.434508in,left,base]{\color{textcolor}\sffamily\fontsize{10.000000}{12.000000}\selectfont 1.2}%
\end{pgfscope}%
\begin{pgfscope}%
\pgfsetbuttcap%
\pgfsetroundjoin%
\definecolor{currentfill}{rgb}{0.000000,0.000000,0.000000}%
\pgfsetfillcolor{currentfill}%
\pgfsetlinewidth{0.803000pt}%
\definecolor{currentstroke}{rgb}{0.000000,0.000000,0.000000}%
\pgfsetstrokecolor{currentstroke}%
\pgfsetdash{}{0pt}%
\pgfsys@defobject{currentmarker}{\pgfqpoint{-0.048611in}{0.000000in}}{\pgfqpoint{0.000000in}{0.000000in}}{%
\pgfpathmoveto{\pgfqpoint{0.000000in}{0.000000in}}%
\pgfpathlineto{\pgfqpoint{-0.048611in}{0.000000in}}%
\pgfusepath{stroke,fill}%
}%
\begin{pgfscope}%
\pgfsys@transformshift{0.557222in}{0.847390in}%
\pgfsys@useobject{currentmarker}{}%
\end{pgfscope}%
\end{pgfscope}%
\begin{pgfscope}%
\definecolor{textcolor}{rgb}{0.000000,0.000000,0.000000}%
\pgfsetstrokecolor{textcolor}%
\pgfsetfillcolor{textcolor}%
\pgftext[x=0.239121in,y=0.794628in,left,base]{\color{textcolor}\sffamily\fontsize{10.000000}{12.000000}\selectfont 1.4}%
\end{pgfscope}%
\begin{pgfscope}%
\pgfsetbuttcap%
\pgfsetroundjoin%
\definecolor{currentfill}{rgb}{0.000000,0.000000,0.000000}%
\pgfsetfillcolor{currentfill}%
\pgfsetlinewidth{0.803000pt}%
\definecolor{currentstroke}{rgb}{0.000000,0.000000,0.000000}%
\pgfsetstrokecolor{currentstroke}%
\pgfsetdash{}{0pt}%
\pgfsys@defobject{currentmarker}{\pgfqpoint{-0.048611in}{0.000000in}}{\pgfqpoint{0.000000in}{0.000000in}}{%
\pgfpathmoveto{\pgfqpoint{0.000000in}{0.000000in}}%
\pgfpathlineto{\pgfqpoint{-0.048611in}{0.000000in}}%
\pgfusepath{stroke,fill}%
}%
\begin{pgfscope}%
\pgfsys@transformshift{0.557222in}{1.207510in}%
\pgfsys@useobject{currentmarker}{}%
\end{pgfscope}%
\end{pgfscope}%
\begin{pgfscope}%
\definecolor{textcolor}{rgb}{0.000000,0.000000,0.000000}%
\pgfsetstrokecolor{textcolor}%
\pgfsetfillcolor{textcolor}%
\pgftext[x=0.239121in,y=1.154749in,left,base]{\color{textcolor}\sffamily\fontsize{10.000000}{12.000000}\selectfont 1.6}%
\end{pgfscope}%
\begin{pgfscope}%
\pgfsetbuttcap%
\pgfsetroundjoin%
\definecolor{currentfill}{rgb}{0.000000,0.000000,0.000000}%
\pgfsetfillcolor{currentfill}%
\pgfsetlinewidth{0.803000pt}%
\definecolor{currentstroke}{rgb}{0.000000,0.000000,0.000000}%
\pgfsetstrokecolor{currentstroke}%
\pgfsetdash{}{0pt}%
\pgfsys@defobject{currentmarker}{\pgfqpoint{-0.048611in}{0.000000in}}{\pgfqpoint{0.000000in}{0.000000in}}{%
\pgfpathmoveto{\pgfqpoint{0.000000in}{0.000000in}}%
\pgfpathlineto{\pgfqpoint{-0.048611in}{0.000000in}}%
\pgfusepath{stroke,fill}%
}%
\begin{pgfscope}%
\pgfsys@transformshift{0.557222in}{1.567630in}%
\pgfsys@useobject{currentmarker}{}%
\end{pgfscope}%
\end{pgfscope}%
\begin{pgfscope}%
\definecolor{textcolor}{rgb}{0.000000,0.000000,0.000000}%
\pgfsetstrokecolor{textcolor}%
\pgfsetfillcolor{textcolor}%
\pgftext[x=0.239121in,y=1.514869in,left,base]{\color{textcolor}\sffamily\fontsize{10.000000}{12.000000}\selectfont 1.8}%
\end{pgfscope}%
\begin{pgfscope}%
\pgfsetbuttcap%
\pgfsetroundjoin%
\definecolor{currentfill}{rgb}{0.000000,0.000000,0.000000}%
\pgfsetfillcolor{currentfill}%
\pgfsetlinewidth{0.803000pt}%
\definecolor{currentstroke}{rgb}{0.000000,0.000000,0.000000}%
\pgfsetstrokecolor{currentstroke}%
\pgfsetdash{}{0pt}%
\pgfsys@defobject{currentmarker}{\pgfqpoint{-0.048611in}{0.000000in}}{\pgfqpoint{0.000000in}{0.000000in}}{%
\pgfpathmoveto{\pgfqpoint{0.000000in}{0.000000in}}%
\pgfpathlineto{\pgfqpoint{-0.048611in}{0.000000in}}%
\pgfusepath{stroke,fill}%
}%
\begin{pgfscope}%
\pgfsys@transformshift{0.557222in}{1.927750in}%
\pgfsys@useobject{currentmarker}{}%
\end{pgfscope}%
\end{pgfscope}%
\begin{pgfscope}%
\definecolor{textcolor}{rgb}{0.000000,0.000000,0.000000}%
\pgfsetstrokecolor{textcolor}%
\pgfsetfillcolor{textcolor}%
\pgftext[x=0.239121in,y=1.874989in,left,base]{\color{textcolor}\sffamily\fontsize{10.000000}{12.000000}\selectfont 2.0}%
\end{pgfscope}%
\begin{pgfscope}%
\pgfsetbuttcap%
\pgfsetroundjoin%
\definecolor{currentfill}{rgb}{0.000000,0.000000,0.000000}%
\pgfsetfillcolor{currentfill}%
\pgfsetlinewidth{0.803000pt}%
\definecolor{currentstroke}{rgb}{0.000000,0.000000,0.000000}%
\pgfsetstrokecolor{currentstroke}%
\pgfsetdash{}{0pt}%
\pgfsys@defobject{currentmarker}{\pgfqpoint{-0.048611in}{0.000000in}}{\pgfqpoint{0.000000in}{0.000000in}}{%
\pgfpathmoveto{\pgfqpoint{0.000000in}{0.000000in}}%
\pgfpathlineto{\pgfqpoint{-0.048611in}{0.000000in}}%
\pgfusepath{stroke,fill}%
}%
\begin{pgfscope}%
\pgfsys@transformshift{0.557222in}{2.287870in}%
\pgfsys@useobject{currentmarker}{}%
\end{pgfscope}%
\end{pgfscope}%
\begin{pgfscope}%
\definecolor{textcolor}{rgb}{0.000000,0.000000,0.000000}%
\pgfsetstrokecolor{textcolor}%
\pgfsetfillcolor{textcolor}%
\pgftext[x=0.239121in,y=2.235109in,left,base]{\color{textcolor}\sffamily\fontsize{10.000000}{12.000000}\selectfont 2.2}%
\end{pgfscope}%
\begin{pgfscope}%
\pgfsetbuttcap%
\pgfsetroundjoin%
\definecolor{currentfill}{rgb}{0.000000,0.000000,0.000000}%
\pgfsetfillcolor{currentfill}%
\pgfsetlinewidth{0.803000pt}%
\definecolor{currentstroke}{rgb}{0.000000,0.000000,0.000000}%
\pgfsetstrokecolor{currentstroke}%
\pgfsetdash{}{0pt}%
\pgfsys@defobject{currentmarker}{\pgfqpoint{-0.048611in}{0.000000in}}{\pgfqpoint{0.000000in}{0.000000in}}{%
\pgfpathmoveto{\pgfqpoint{0.000000in}{0.000000in}}%
\pgfpathlineto{\pgfqpoint{-0.048611in}{0.000000in}}%
\pgfusepath{stroke,fill}%
}%
\begin{pgfscope}%
\pgfsys@transformshift{0.557222in}{2.647990in}%
\pgfsys@useobject{currentmarker}{}%
\end{pgfscope}%
\end{pgfscope}%
\begin{pgfscope}%
\definecolor{textcolor}{rgb}{0.000000,0.000000,0.000000}%
\pgfsetstrokecolor{textcolor}%
\pgfsetfillcolor{textcolor}%
\pgftext[x=0.239121in,y=2.595229in,left,base]{\color{textcolor}\sffamily\fontsize{10.000000}{12.000000}\selectfont 2.4}%
\end{pgfscope}%
\begin{pgfscope}%
\pgfpathrectangle{\pgfqpoint{0.557222in}{0.387222in}}{\pgfqpoint{3.321958in}{2.324444in}}%
\pgfusepath{clip}%
\pgfsetrectcap%
\pgfsetroundjoin%
\pgfsetlinewidth{1.505625pt}%
\definecolor{currentstroke}{rgb}{0.121569,0.466667,0.705882}%
\pgfsetstrokecolor{currentstroke}%
\pgfsetdash{}{0pt}%
\pgfpathmoveto{\pgfqpoint{0.716510in}{0.679089in}}%
\pgfpathlineto{\pgfqpoint{0.866679in}{0.747726in}}%
\pgfpathlineto{\pgfqpoint{1.016848in}{0.913778in}}%
\pgfpathlineto{\pgfqpoint{1.167017in}{1.020377in}}%
\pgfpathlineto{\pgfqpoint{1.317186in}{1.225053in}}%
\pgfpathlineto{\pgfqpoint{1.467355in}{1.424472in}}%
\pgfpathlineto{\pgfqpoint{1.617525in}{1.694141in}}%
\pgfpathlineto{\pgfqpoint{1.767694in}{1.858960in}}%
\pgfpathlineto{\pgfqpoint{1.917863in}{2.021725in}}%
\pgfpathlineto{\pgfqpoint{2.068032in}{2.318482in}}%
\pgfpathlineto{\pgfqpoint{2.218201in}{2.418146in}}%
\pgfpathlineto{\pgfqpoint{2.368370in}{2.343092in}}%
\pgfpathlineto{\pgfqpoint{2.518540in}{2.131625in}}%
\pgfpathlineto{\pgfqpoint{2.668709in}{1.870064in}}%
\pgfpathlineto{\pgfqpoint{2.818878in}{1.588041in}}%
\pgfpathlineto{\pgfqpoint{2.969047in}{1.286701in}}%
\pgfpathlineto{\pgfqpoint{3.119216in}{1.091418in}}%
\pgfpathlineto{\pgfqpoint{3.269385in}{0.924049in}}%
\pgfpathlineto{\pgfqpoint{3.419554in}{0.802765in}}%
\pgfpathlineto{\pgfqpoint{3.569724in}{0.653396in}}%
\pgfpathlineto{\pgfqpoint{3.719893in}{0.547077in}}%
\pgfusepath{stroke}%
\end{pgfscope}%
\begin{pgfscope}%
\pgfsetrectcap%
\pgfsetmiterjoin%
\pgfsetlinewidth{0.803000pt}%
\definecolor{currentstroke}{rgb}{0.000000,0.000000,0.000000}%
\pgfsetstrokecolor{currentstroke}%
\pgfsetdash{}{0pt}%
\pgfpathmoveto{\pgfqpoint{0.557222in}{0.387222in}}%
\pgfpathlineto{\pgfqpoint{0.557222in}{2.711667in}}%
\pgfusepath{stroke}%
\end{pgfscope}%
\begin{pgfscope}%
\pgfsetrectcap%
\pgfsetmiterjoin%
\pgfsetlinewidth{0.803000pt}%
\definecolor{currentstroke}{rgb}{0.000000,0.000000,0.000000}%
\pgfsetstrokecolor{currentstroke}%
\pgfsetdash{}{0pt}%
\pgfpathmoveto{\pgfqpoint{3.879180in}{0.387222in}}%
\pgfpathlineto{\pgfqpoint{3.879180in}{2.711667in}}%
\pgfusepath{stroke}%
\end{pgfscope}%
\begin{pgfscope}%
\pgfsetrectcap%
\pgfsetmiterjoin%
\pgfsetlinewidth{0.803000pt}%
\definecolor{currentstroke}{rgb}{0.000000,0.000000,0.000000}%
\pgfsetstrokecolor{currentstroke}%
\pgfsetdash{}{0pt}%
\pgfpathmoveto{\pgfqpoint{0.557222in}{0.387222in}}%
\pgfpathlineto{\pgfqpoint{3.879180in}{0.387222in}}%
\pgfusepath{stroke}%
\end{pgfscope}%
\begin{pgfscope}%
\pgfsetrectcap%
\pgfsetmiterjoin%
\pgfsetlinewidth{0.803000pt}%
\definecolor{currentstroke}{rgb}{0.000000,0.000000,0.000000}%
\pgfsetstrokecolor{currentstroke}%
\pgfsetdash{}{0pt}%
\pgfpathmoveto{\pgfqpoint{0.557222in}{2.711667in}}%
\pgfpathlineto{\pgfqpoint{3.879180in}{2.711667in}}%
\pgfusepath{stroke}%
\end{pgfscope}%
\begin{pgfscope}%
\definecolor{textcolor}{rgb}{0.000000,0.000000,0.000000}%
\pgfsetstrokecolor{textcolor}%
\pgfsetfillcolor{textcolor}%
\pgftext[x=2.218201in,y=2.795000in,,base]{\color{textcolor}\sffamily\fontsize{12.000000}{14.400000}\selectfont \(\displaystyle  N = 64 \)}%
\end{pgfscope}%
\begin{pgfscope}%
\pgfsetbuttcap%
\pgfsetmiterjoin%
\definecolor{currentfill}{rgb}{1.000000,1.000000,1.000000}%
\pgfsetfillcolor{currentfill}%
\pgfsetlinewidth{0.000000pt}%
\definecolor{currentstroke}{rgb}{0.000000,0.000000,0.000000}%
\pgfsetstrokecolor{currentstroke}%
\pgfsetstrokeopacity{0.000000}%
\pgfsetdash{}{0pt}%
\pgfpathmoveto{\pgfqpoint{4.382652in}{0.387222in}}%
\pgfpathlineto{\pgfqpoint{7.704610in}{0.387222in}}%
\pgfpathlineto{\pgfqpoint{7.704610in}{2.711667in}}%
\pgfpathlineto{\pgfqpoint{4.382652in}{2.711667in}}%
\pgfpathclose%
\pgfusepath{fill}%
\end{pgfscope}%
\begin{pgfscope}%
\pgfpathrectangle{\pgfqpoint{4.382652in}{0.387222in}}{\pgfqpoint{3.321958in}{2.324444in}}%
\pgfusepath{clip}%
\pgfsetbuttcap%
\pgfsetroundjoin%
\definecolor{currentfill}{rgb}{0.121569,0.466667,0.705882}%
\pgfsetfillcolor{currentfill}%
\pgfsetfillopacity{0.300000}%
\pgfsetlinewidth{1.003750pt}%
\definecolor{currentstroke}{rgb}{0.121569,0.466667,0.705882}%
\pgfsetstrokecolor{currentstroke}%
\pgfsetstrokeopacity{0.300000}%
\pgfsetdash{}{0pt}%
\pgfpathmoveto{\pgfqpoint{4.541940in}{0.586233in}}%
\pgfpathlineto{\pgfqpoint{4.541940in}{0.538920in}}%
\pgfpathlineto{\pgfqpoint{4.692109in}{0.583784in}}%
\pgfpathlineto{\pgfqpoint{4.842278in}{0.618488in}}%
\pgfpathlineto{\pgfqpoint{4.992447in}{0.670060in}}%
\pgfpathlineto{\pgfqpoint{5.142617in}{0.808072in}}%
\pgfpathlineto{\pgfqpoint{5.292786in}{0.884748in}}%
\pgfpathlineto{\pgfqpoint{5.442955in}{1.006870in}}%
\pgfpathlineto{\pgfqpoint{5.593124in}{1.088062in}}%
\pgfpathlineto{\pgfqpoint{5.743293in}{1.080254in}}%
\pgfpathlineto{\pgfqpoint{5.893462in}{1.276048in}}%
\pgfpathlineto{\pgfqpoint{6.043631in}{1.284847in}}%
\pgfpathlineto{\pgfqpoint{6.193801in}{1.639469in}}%
\pgfpathlineto{\pgfqpoint{6.343970in}{1.768686in}}%
\pgfpathlineto{\pgfqpoint{6.494139in}{1.013458in}}%
\pgfpathlineto{\pgfqpoint{6.644308in}{1.214134in}}%
\pgfpathlineto{\pgfqpoint{6.794477in}{0.819631in}}%
\pgfpathlineto{\pgfqpoint{6.944646in}{0.803585in}}%
\pgfpathlineto{\pgfqpoint{7.094816in}{0.677854in}}%
\pgfpathlineto{\pgfqpoint{7.244985in}{0.675936in}}%
\pgfpathlineto{\pgfqpoint{7.395154in}{0.527061in}}%
\pgfpathlineto{\pgfqpoint{7.545323in}{0.492879in}}%
\pgfpathlineto{\pgfqpoint{7.545323in}{0.675995in}}%
\pgfpathlineto{\pgfqpoint{7.545323in}{0.675995in}}%
\pgfpathlineto{\pgfqpoint{7.395154in}{0.671210in}}%
\pgfpathlineto{\pgfqpoint{7.244985in}{0.862232in}}%
\pgfpathlineto{\pgfqpoint{7.094816in}{0.848425in}}%
\pgfpathlineto{\pgfqpoint{6.944646in}{0.928578in}}%
\pgfpathlineto{\pgfqpoint{6.794477in}{1.092525in}}%
\pgfpathlineto{\pgfqpoint{6.644308in}{1.320105in}}%
\pgfpathlineto{\pgfqpoint{6.494139in}{1.300777in}}%
\pgfpathlineto{\pgfqpoint{6.343970in}{2.000565in}}%
\pgfpathlineto{\pgfqpoint{6.193801in}{2.044018in}}%
\pgfpathlineto{\pgfqpoint{6.043631in}{2.606010in}}%
\pgfpathlineto{\pgfqpoint{5.893462in}{2.003349in}}%
\pgfpathlineto{\pgfqpoint{5.743293in}{1.580820in}}%
\pgfpathlineto{\pgfqpoint{5.593124in}{1.316842in}}%
\pgfpathlineto{\pgfqpoint{5.442955in}{1.243056in}}%
\pgfpathlineto{\pgfqpoint{5.292786in}{0.965562in}}%
\pgfpathlineto{\pgfqpoint{5.142617in}{0.904025in}}%
\pgfpathlineto{\pgfqpoint{4.992447in}{0.829153in}}%
\pgfpathlineto{\pgfqpoint{4.842278in}{0.806475in}}%
\pgfpathlineto{\pgfqpoint{4.692109in}{0.662067in}}%
\pgfpathlineto{\pgfqpoint{4.541940in}{0.586233in}}%
\pgfpathclose%
\pgfusepath{stroke,fill}%
\end{pgfscope}%
\begin{pgfscope}%
\pgfpathrectangle{\pgfqpoint{4.382652in}{0.387222in}}{\pgfqpoint{3.321958in}{2.324444in}}%
\pgfusepath{clip}%
\pgfsetbuttcap%
\pgfsetroundjoin%
\definecolor{currentfill}{rgb}{0.121569,0.466667,0.705882}%
\pgfsetfillcolor{currentfill}%
\pgfsetlinewidth{1.003750pt}%
\definecolor{currentstroke}{rgb}{0.121569,0.466667,0.705882}%
\pgfsetstrokecolor{currentstroke}%
\pgfsetdash{}{0pt}%
\pgfsys@defobject{currentmarker}{\pgfqpoint{-0.009821in}{-0.009821in}}{\pgfqpoint{0.009821in}{0.009821in}}{%
\pgfpathmoveto{\pgfqpoint{0.000000in}{-0.009821in}}%
\pgfpathcurveto{\pgfqpoint{0.002605in}{-0.009821in}}{\pgfqpoint{0.005103in}{-0.008786in}}{\pgfqpoint{0.006944in}{-0.006944in}}%
\pgfpathcurveto{\pgfqpoint{0.008786in}{-0.005103in}}{\pgfqpoint{0.009821in}{-0.002605in}}{\pgfqpoint{0.009821in}{0.000000in}}%
\pgfpathcurveto{\pgfqpoint{0.009821in}{0.002605in}}{\pgfqpoint{0.008786in}{0.005103in}}{\pgfqpoint{0.006944in}{0.006944in}}%
\pgfpathcurveto{\pgfqpoint{0.005103in}{0.008786in}}{\pgfqpoint{0.002605in}{0.009821in}}{\pgfqpoint{0.000000in}{0.009821in}}%
\pgfpathcurveto{\pgfqpoint{-0.002605in}{0.009821in}}{\pgfqpoint{-0.005103in}{0.008786in}}{\pgfqpoint{-0.006944in}{0.006944in}}%
\pgfpathcurveto{\pgfqpoint{-0.008786in}{0.005103in}}{\pgfqpoint{-0.009821in}{0.002605in}}{\pgfqpoint{-0.009821in}{0.000000in}}%
\pgfpathcurveto{\pgfqpoint{-0.009821in}{-0.002605in}}{\pgfqpoint{-0.008786in}{-0.005103in}}{\pgfqpoint{-0.006944in}{-0.006944in}}%
\pgfpathcurveto{\pgfqpoint{-0.005103in}{-0.008786in}}{\pgfqpoint{-0.002605in}{-0.009821in}}{\pgfqpoint{0.000000in}{-0.009821in}}%
\pgfpathclose%
\pgfusepath{stroke,fill}%
}%
\begin{pgfscope}%
\pgfsys@transformshift{4.541940in}{0.562576in}%
\pgfsys@useobject{currentmarker}{}%
\end{pgfscope}%
\begin{pgfscope}%
\pgfsys@transformshift{4.692109in}{0.622926in}%
\pgfsys@useobject{currentmarker}{}%
\end{pgfscope}%
\begin{pgfscope}%
\pgfsys@transformshift{4.842278in}{0.712481in}%
\pgfsys@useobject{currentmarker}{}%
\end{pgfscope}%
\begin{pgfscope}%
\pgfsys@transformshift{4.992447in}{0.749607in}%
\pgfsys@useobject{currentmarker}{}%
\end{pgfscope}%
\begin{pgfscope}%
\pgfsys@transformshift{5.142617in}{0.856048in}%
\pgfsys@useobject{currentmarker}{}%
\end{pgfscope}%
\begin{pgfscope}%
\pgfsys@transformshift{5.292786in}{0.925155in}%
\pgfsys@useobject{currentmarker}{}%
\end{pgfscope}%
\begin{pgfscope}%
\pgfsys@transformshift{5.442955in}{1.124963in}%
\pgfsys@useobject{currentmarker}{}%
\end{pgfscope}%
\begin{pgfscope}%
\pgfsys@transformshift{5.593124in}{1.202452in}%
\pgfsys@useobject{currentmarker}{}%
\end{pgfscope}%
\begin{pgfscope}%
\pgfsys@transformshift{5.743293in}{1.330537in}%
\pgfsys@useobject{currentmarker}{}%
\end{pgfscope}%
\begin{pgfscope}%
\pgfsys@transformshift{5.893462in}{1.639698in}%
\pgfsys@useobject{currentmarker}{}%
\end{pgfscope}%
\begin{pgfscope}%
\pgfsys@transformshift{6.043631in}{1.945429in}%
\pgfsys@useobject{currentmarker}{}%
\end{pgfscope}%
\begin{pgfscope}%
\pgfsys@transformshift{6.193801in}{1.841743in}%
\pgfsys@useobject{currentmarker}{}%
\end{pgfscope}%
\begin{pgfscope}%
\pgfsys@transformshift{6.343970in}{1.884626in}%
\pgfsys@useobject{currentmarker}{}%
\end{pgfscope}%
\begin{pgfscope}%
\pgfsys@transformshift{6.494139in}{1.157118in}%
\pgfsys@useobject{currentmarker}{}%
\end{pgfscope}%
\begin{pgfscope}%
\pgfsys@transformshift{6.644308in}{1.267119in}%
\pgfsys@useobject{currentmarker}{}%
\end{pgfscope}%
\begin{pgfscope}%
\pgfsys@transformshift{6.794477in}{0.956078in}%
\pgfsys@useobject{currentmarker}{}%
\end{pgfscope}%
\begin{pgfscope}%
\pgfsys@transformshift{6.944646in}{0.866081in}%
\pgfsys@useobject{currentmarker}{}%
\end{pgfscope}%
\begin{pgfscope}%
\pgfsys@transformshift{7.094816in}{0.763140in}%
\pgfsys@useobject{currentmarker}{}%
\end{pgfscope}%
\begin{pgfscope}%
\pgfsys@transformshift{7.244985in}{0.769084in}%
\pgfsys@useobject{currentmarker}{}%
\end{pgfscope}%
\begin{pgfscope}%
\pgfsys@transformshift{7.395154in}{0.599136in}%
\pgfsys@useobject{currentmarker}{}%
\end{pgfscope}%
\begin{pgfscope}%
\pgfsys@transformshift{7.545323in}{0.584437in}%
\pgfsys@useobject{currentmarker}{}%
\end{pgfscope}%
\end{pgfscope}%
\begin{pgfscope}%
\pgfsetbuttcap%
\pgfsetroundjoin%
\definecolor{currentfill}{rgb}{0.000000,0.000000,0.000000}%
\pgfsetfillcolor{currentfill}%
\pgfsetlinewidth{0.803000pt}%
\definecolor{currentstroke}{rgb}{0.000000,0.000000,0.000000}%
\pgfsetstrokecolor{currentstroke}%
\pgfsetdash{}{0pt}%
\pgfsys@defobject{currentmarker}{\pgfqpoint{0.000000in}{-0.048611in}}{\pgfqpoint{0.000000in}{0.000000in}}{%
\pgfpathmoveto{\pgfqpoint{0.000000in}{0.000000in}}%
\pgfpathlineto{\pgfqpoint{0.000000in}{-0.048611in}}%
\pgfusepath{stroke,fill}%
}%
\begin{pgfscope}%
\pgfsys@transformshift{4.992447in}{0.387222in}%
\pgfsys@useobject{currentmarker}{}%
\end{pgfscope}%
\end{pgfscope}%
\begin{pgfscope}%
\definecolor{textcolor}{rgb}{0.000000,0.000000,0.000000}%
\pgfsetstrokecolor{textcolor}%
\pgfsetfillcolor{textcolor}%
\pgftext[x=4.992447in,y=0.290000in,,top]{\color{textcolor}\sffamily\fontsize{10.000000}{12.000000}\selectfont 2.20}%
\end{pgfscope}%
\begin{pgfscope}%
\pgfsetbuttcap%
\pgfsetroundjoin%
\definecolor{currentfill}{rgb}{0.000000,0.000000,0.000000}%
\pgfsetfillcolor{currentfill}%
\pgfsetlinewidth{0.803000pt}%
\definecolor{currentstroke}{rgb}{0.000000,0.000000,0.000000}%
\pgfsetstrokecolor{currentstroke}%
\pgfsetdash{}{0pt}%
\pgfsys@defobject{currentmarker}{\pgfqpoint{0.000000in}{-0.048611in}}{\pgfqpoint{0.000000in}{0.000000in}}{%
\pgfpathmoveto{\pgfqpoint{0.000000in}{0.000000in}}%
\pgfpathlineto{\pgfqpoint{0.000000in}{-0.048611in}}%
\pgfusepath{stroke,fill}%
}%
\begin{pgfscope}%
\pgfsys@transformshift{5.743293in}{0.387222in}%
\pgfsys@useobject{currentmarker}{}%
\end{pgfscope}%
\end{pgfscope}%
\begin{pgfscope}%
\definecolor{textcolor}{rgb}{0.000000,0.000000,0.000000}%
\pgfsetstrokecolor{textcolor}%
\pgfsetfillcolor{textcolor}%
\pgftext[x=5.743293in,y=0.290000in,,top]{\color{textcolor}\sffamily\fontsize{10.000000}{12.000000}\selectfont 2.25}%
\end{pgfscope}%
\begin{pgfscope}%
\pgfsetbuttcap%
\pgfsetroundjoin%
\definecolor{currentfill}{rgb}{0.000000,0.000000,0.000000}%
\pgfsetfillcolor{currentfill}%
\pgfsetlinewidth{0.803000pt}%
\definecolor{currentstroke}{rgb}{0.000000,0.000000,0.000000}%
\pgfsetstrokecolor{currentstroke}%
\pgfsetdash{}{0pt}%
\pgfsys@defobject{currentmarker}{\pgfqpoint{0.000000in}{-0.048611in}}{\pgfqpoint{0.000000in}{0.000000in}}{%
\pgfpathmoveto{\pgfqpoint{0.000000in}{0.000000in}}%
\pgfpathlineto{\pgfqpoint{0.000000in}{-0.048611in}}%
\pgfusepath{stroke,fill}%
}%
\begin{pgfscope}%
\pgfsys@transformshift{6.494139in}{0.387222in}%
\pgfsys@useobject{currentmarker}{}%
\end{pgfscope}%
\end{pgfscope}%
\begin{pgfscope}%
\definecolor{textcolor}{rgb}{0.000000,0.000000,0.000000}%
\pgfsetstrokecolor{textcolor}%
\pgfsetfillcolor{textcolor}%
\pgftext[x=6.494139in,y=0.290000in,,top]{\color{textcolor}\sffamily\fontsize{10.000000}{12.000000}\selectfont 2.30}%
\end{pgfscope}%
\begin{pgfscope}%
\pgfsetbuttcap%
\pgfsetroundjoin%
\definecolor{currentfill}{rgb}{0.000000,0.000000,0.000000}%
\pgfsetfillcolor{currentfill}%
\pgfsetlinewidth{0.803000pt}%
\definecolor{currentstroke}{rgb}{0.000000,0.000000,0.000000}%
\pgfsetstrokecolor{currentstroke}%
\pgfsetdash{}{0pt}%
\pgfsys@defobject{currentmarker}{\pgfqpoint{0.000000in}{-0.048611in}}{\pgfqpoint{0.000000in}{0.000000in}}{%
\pgfpathmoveto{\pgfqpoint{0.000000in}{0.000000in}}%
\pgfpathlineto{\pgfqpoint{0.000000in}{-0.048611in}}%
\pgfusepath{stroke,fill}%
}%
\begin{pgfscope}%
\pgfsys@transformshift{7.244985in}{0.387222in}%
\pgfsys@useobject{currentmarker}{}%
\end{pgfscope}%
\end{pgfscope}%
\begin{pgfscope}%
\definecolor{textcolor}{rgb}{0.000000,0.000000,0.000000}%
\pgfsetstrokecolor{textcolor}%
\pgfsetfillcolor{textcolor}%
\pgftext[x=7.244985in,y=0.290000in,,top]{\color{textcolor}\sffamily\fontsize{10.000000}{12.000000}\selectfont 2.35}%
\end{pgfscope}%
\begin{pgfscope}%
\pgfsetbuttcap%
\pgfsetroundjoin%
\definecolor{currentfill}{rgb}{0.000000,0.000000,0.000000}%
\pgfsetfillcolor{currentfill}%
\pgfsetlinewidth{0.803000pt}%
\definecolor{currentstroke}{rgb}{0.000000,0.000000,0.000000}%
\pgfsetstrokecolor{currentstroke}%
\pgfsetdash{}{0pt}%
\pgfsys@defobject{currentmarker}{\pgfqpoint{-0.048611in}{0.000000in}}{\pgfqpoint{0.000000in}{0.000000in}}{%
\pgfpathmoveto{\pgfqpoint{0.000000in}{0.000000in}}%
\pgfpathlineto{\pgfqpoint{-0.048611in}{0.000000in}}%
\pgfusepath{stroke,fill}%
}%
\begin{pgfscope}%
\pgfsys@transformshift{4.382652in}{0.864551in}%
\pgfsys@useobject{currentmarker}{}%
\end{pgfscope}%
\end{pgfscope}%
\begin{pgfscope}%
\definecolor{textcolor}{rgb}{0.000000,0.000000,0.000000}%
\pgfsetstrokecolor{textcolor}%
\pgfsetfillcolor{textcolor}%
\pgftext[x=4.064551in,y=0.811790in,left,base]{\color{textcolor}\sffamily\fontsize{10.000000}{12.000000}\selectfont 1.5}%
\end{pgfscope}%
\begin{pgfscope}%
\pgfsetbuttcap%
\pgfsetroundjoin%
\definecolor{currentfill}{rgb}{0.000000,0.000000,0.000000}%
\pgfsetfillcolor{currentfill}%
\pgfsetlinewidth{0.803000pt}%
\definecolor{currentstroke}{rgb}{0.000000,0.000000,0.000000}%
\pgfsetstrokecolor{currentstroke}%
\pgfsetdash{}{0pt}%
\pgfsys@defobject{currentmarker}{\pgfqpoint{-0.048611in}{0.000000in}}{\pgfqpoint{0.000000in}{0.000000in}}{%
\pgfpathmoveto{\pgfqpoint{0.000000in}{0.000000in}}%
\pgfpathlineto{\pgfqpoint{-0.048611in}{0.000000in}}%
\pgfusepath{stroke,fill}%
}%
\begin{pgfscope}%
\pgfsys@transformshift{4.382652in}{1.407673in}%
\pgfsys@useobject{currentmarker}{}%
\end{pgfscope}%
\end{pgfscope}%
\begin{pgfscope}%
\definecolor{textcolor}{rgb}{0.000000,0.000000,0.000000}%
\pgfsetstrokecolor{textcolor}%
\pgfsetfillcolor{textcolor}%
\pgftext[x=4.064551in,y=1.354911in,left,base]{\color{textcolor}\sffamily\fontsize{10.000000}{12.000000}\selectfont 2.0}%
\end{pgfscope}%
\begin{pgfscope}%
\pgfsetbuttcap%
\pgfsetroundjoin%
\definecolor{currentfill}{rgb}{0.000000,0.000000,0.000000}%
\pgfsetfillcolor{currentfill}%
\pgfsetlinewidth{0.803000pt}%
\definecolor{currentstroke}{rgb}{0.000000,0.000000,0.000000}%
\pgfsetstrokecolor{currentstroke}%
\pgfsetdash{}{0pt}%
\pgfsys@defobject{currentmarker}{\pgfqpoint{-0.048611in}{0.000000in}}{\pgfqpoint{0.000000in}{0.000000in}}{%
\pgfpathmoveto{\pgfqpoint{0.000000in}{0.000000in}}%
\pgfpathlineto{\pgfqpoint{-0.048611in}{0.000000in}}%
\pgfusepath{stroke,fill}%
}%
\begin{pgfscope}%
\pgfsys@transformshift{4.382652in}{1.950795in}%
\pgfsys@useobject{currentmarker}{}%
\end{pgfscope}%
\end{pgfscope}%
\begin{pgfscope}%
\definecolor{textcolor}{rgb}{0.000000,0.000000,0.000000}%
\pgfsetstrokecolor{textcolor}%
\pgfsetfillcolor{textcolor}%
\pgftext[x=4.064551in,y=1.898033in,left,base]{\color{textcolor}\sffamily\fontsize{10.000000}{12.000000}\selectfont 2.5}%
\end{pgfscope}%
\begin{pgfscope}%
\pgfsetbuttcap%
\pgfsetroundjoin%
\definecolor{currentfill}{rgb}{0.000000,0.000000,0.000000}%
\pgfsetfillcolor{currentfill}%
\pgfsetlinewidth{0.803000pt}%
\definecolor{currentstroke}{rgb}{0.000000,0.000000,0.000000}%
\pgfsetstrokecolor{currentstroke}%
\pgfsetdash{}{0pt}%
\pgfsys@defobject{currentmarker}{\pgfqpoint{-0.048611in}{0.000000in}}{\pgfqpoint{0.000000in}{0.000000in}}{%
\pgfpathmoveto{\pgfqpoint{0.000000in}{0.000000in}}%
\pgfpathlineto{\pgfqpoint{-0.048611in}{0.000000in}}%
\pgfusepath{stroke,fill}%
}%
\begin{pgfscope}%
\pgfsys@transformshift{4.382652in}{2.493917in}%
\pgfsys@useobject{currentmarker}{}%
\end{pgfscope}%
\end{pgfscope}%
\begin{pgfscope}%
\definecolor{textcolor}{rgb}{0.000000,0.000000,0.000000}%
\pgfsetstrokecolor{textcolor}%
\pgfsetfillcolor{textcolor}%
\pgftext[x=4.064551in,y=2.441155in,left,base]{\color{textcolor}\sffamily\fontsize{10.000000}{12.000000}\selectfont 3.0}%
\end{pgfscope}%
\begin{pgfscope}%
\pgfpathrectangle{\pgfqpoint{4.382652in}{0.387222in}}{\pgfqpoint{3.321958in}{2.324444in}}%
\pgfusepath{clip}%
\pgfsetrectcap%
\pgfsetroundjoin%
\pgfsetlinewidth{1.505625pt}%
\definecolor{currentstroke}{rgb}{0.121569,0.466667,0.705882}%
\pgfsetstrokecolor{currentstroke}%
\pgfsetdash{}{0pt}%
\pgfpathmoveto{\pgfqpoint{4.541940in}{0.562576in}}%
\pgfpathlineto{\pgfqpoint{4.692109in}{0.622926in}}%
\pgfpathlineto{\pgfqpoint{4.842278in}{0.712481in}}%
\pgfpathlineto{\pgfqpoint{4.992447in}{0.749607in}}%
\pgfpathlineto{\pgfqpoint{5.142617in}{0.856048in}}%
\pgfpathlineto{\pgfqpoint{5.292786in}{0.925155in}}%
\pgfpathlineto{\pgfqpoint{5.442955in}{1.124963in}}%
\pgfpathlineto{\pgfqpoint{5.593124in}{1.202452in}}%
\pgfpathlineto{\pgfqpoint{5.743293in}{1.330537in}}%
\pgfpathlineto{\pgfqpoint{5.893462in}{1.639698in}}%
\pgfpathlineto{\pgfqpoint{6.043631in}{1.945429in}}%
\pgfpathlineto{\pgfqpoint{6.193801in}{1.841743in}}%
\pgfpathlineto{\pgfqpoint{6.343970in}{1.884626in}}%
\pgfpathlineto{\pgfqpoint{6.494139in}{1.157118in}}%
\pgfpathlineto{\pgfqpoint{6.644308in}{1.267119in}}%
\pgfpathlineto{\pgfqpoint{6.794477in}{0.956078in}}%
\pgfpathlineto{\pgfqpoint{6.944646in}{0.866081in}}%
\pgfpathlineto{\pgfqpoint{7.094816in}{0.763140in}}%
\pgfpathlineto{\pgfqpoint{7.244985in}{0.769084in}}%
\pgfpathlineto{\pgfqpoint{7.395154in}{0.599136in}}%
\pgfpathlineto{\pgfqpoint{7.545323in}{0.584437in}}%
\pgfusepath{stroke}%
\end{pgfscope}%
\begin{pgfscope}%
\pgfsetrectcap%
\pgfsetmiterjoin%
\pgfsetlinewidth{0.803000pt}%
\definecolor{currentstroke}{rgb}{0.000000,0.000000,0.000000}%
\pgfsetstrokecolor{currentstroke}%
\pgfsetdash{}{0pt}%
\pgfpathmoveto{\pgfqpoint{4.382652in}{0.387222in}}%
\pgfpathlineto{\pgfqpoint{4.382652in}{2.711667in}}%
\pgfusepath{stroke}%
\end{pgfscope}%
\begin{pgfscope}%
\pgfsetrectcap%
\pgfsetmiterjoin%
\pgfsetlinewidth{0.803000pt}%
\definecolor{currentstroke}{rgb}{0.000000,0.000000,0.000000}%
\pgfsetstrokecolor{currentstroke}%
\pgfsetdash{}{0pt}%
\pgfpathmoveto{\pgfqpoint{7.704610in}{0.387222in}}%
\pgfpathlineto{\pgfqpoint{7.704610in}{2.711667in}}%
\pgfusepath{stroke}%
\end{pgfscope}%
\begin{pgfscope}%
\pgfsetrectcap%
\pgfsetmiterjoin%
\pgfsetlinewidth{0.803000pt}%
\definecolor{currentstroke}{rgb}{0.000000,0.000000,0.000000}%
\pgfsetstrokecolor{currentstroke}%
\pgfsetdash{}{0pt}%
\pgfpathmoveto{\pgfqpoint{4.382652in}{0.387222in}}%
\pgfpathlineto{\pgfqpoint{7.704610in}{0.387222in}}%
\pgfusepath{stroke}%
\end{pgfscope}%
\begin{pgfscope}%
\pgfsetrectcap%
\pgfsetmiterjoin%
\pgfsetlinewidth{0.803000pt}%
\definecolor{currentstroke}{rgb}{0.000000,0.000000,0.000000}%
\pgfsetstrokecolor{currentstroke}%
\pgfsetdash{}{0pt}%
\pgfpathmoveto{\pgfqpoint{4.382652in}{2.711667in}}%
\pgfpathlineto{\pgfqpoint{7.704610in}{2.711667in}}%
\pgfusepath{stroke}%
\end{pgfscope}%
\begin{pgfscope}%
\definecolor{textcolor}{rgb}{0.000000,0.000000,0.000000}%
\pgfsetstrokecolor{textcolor}%
\pgfsetfillcolor{textcolor}%
\pgftext[x=6.043631in,y=2.795000in,,base]{\color{textcolor}\sffamily\fontsize{12.000000}{14.400000}\selectfont \(\displaystyle  N = 128 \)}%
\end{pgfscope}%
\end{pgfpicture}%
\makeatother%
\endgroup%
}
\caption{Specific heat $c$ for finer temperature $T$ using kinetic Monte Carlo algorithm}
\label{Fig:CapBigKMC}
\end{figure}

\begin{figure}[htbp]
\centering
\scalebox{0.666}{%% Creator: Matplotlib, PGF backend
%%
%% To include the figure in your LaTeX document, write
%%   \input{<filename>.pgf}
%%
%% Make sure the required packages are loaded in your preamble
%%   \usepackage{pgf}
%%
%% Figures using additional raster images can only be included by \input if
%% they are in the same directory as the main LaTeX file. For loading figures
%% from other directories you can use the `import` package
%%   \usepackage{import}
%% and then include the figures with
%%   \import{<path to file>}{<filename>.pgf}
%%
%% Matplotlib used the following preamble
%%   \usepackage{fontspec}
%%   \setmainfont{DejaVuSerif.ttf}[Path=/home/lzh/anaconda3/envs/numana/lib/python3.7/site-packages/matplotlib/mpl-data/fonts/ttf/]
%%   \setsansfont{DejaVuSans.ttf}[Path=/home/lzh/anaconda3/envs/numana/lib/python3.7/site-packages/matplotlib/mpl-data/fonts/ttf/]
%%   \setmonofont{DejaVuSansMono.ttf}[Path=/home/lzh/anaconda3/envs/numana/lib/python3.7/site-packages/matplotlib/mpl-data/fonts/ttf/]
%%
\begingroup%
\makeatletter%
\begin{pgfpicture}%
\pgfpathrectangle{\pgfpointorigin}{\pgfqpoint{8.000000in}{6.000000in}}%
\pgfusepath{use as bounding box, clip}%
\begin{pgfscope}%
\pgfsetbuttcap%
\pgfsetmiterjoin%
\definecolor{currentfill}{rgb}{1.000000,1.000000,1.000000}%
\pgfsetfillcolor{currentfill}%
\pgfsetlinewidth{0.000000pt}%
\definecolor{currentstroke}{rgb}{1.000000,1.000000,1.000000}%
\pgfsetstrokecolor{currentstroke}%
\pgfsetdash{}{0pt}%
\pgfpathmoveto{\pgfqpoint{0.000000in}{0.000000in}}%
\pgfpathlineto{\pgfqpoint{8.000000in}{0.000000in}}%
\pgfpathlineto{\pgfqpoint{8.000000in}{6.000000in}}%
\pgfpathlineto{\pgfqpoint{0.000000in}{6.000000in}}%
\pgfpathclose%
\pgfusepath{fill}%
\end{pgfscope}%
\begin{pgfscope}%
\pgfsetbuttcap%
\pgfsetmiterjoin%
\definecolor{currentfill}{rgb}{1.000000,1.000000,1.000000}%
\pgfsetfillcolor{currentfill}%
\pgfsetlinewidth{0.000000pt}%
\definecolor{currentstroke}{rgb}{0.000000,0.000000,0.000000}%
\pgfsetstrokecolor{currentstroke}%
\pgfsetstrokeopacity{0.000000}%
\pgfsetdash{}{0pt}%
\pgfpathmoveto{\pgfqpoint{0.556250in}{3.311111in}}%
\pgfpathlineto{\pgfqpoint{3.873503in}{3.311111in}}%
\pgfpathlineto{\pgfqpoint{3.873503in}{5.627778in}}%
\pgfpathlineto{\pgfqpoint{0.556250in}{5.627778in}}%
\pgfpathclose%
\pgfusepath{fill}%
\end{pgfscope}%
\begin{pgfscope}%
\pgfpathrectangle{\pgfqpoint{0.556250in}{3.311111in}}{\pgfqpoint{3.317253in}{2.316667in}}%
\pgfusepath{clip}%
\pgfsetbuttcap%
\pgfsetroundjoin%
\definecolor{currentfill}{rgb}{0.121569,0.466667,0.705882}%
\pgfsetfillcolor{currentfill}%
\pgfsetfillopacity{0.300000}%
\pgfsetlinewidth{1.003750pt}%
\definecolor{currentstroke}{rgb}{0.121569,0.466667,0.705882}%
\pgfsetstrokecolor{currentstroke}%
\pgfsetstrokeopacity{0.300000}%
\pgfsetdash{}{0pt}%
\pgfpathmoveto{\pgfqpoint{0.718506in}{5.516141in}}%
\pgfpathlineto{\pgfqpoint{0.718506in}{5.506776in}}%
\pgfpathlineto{\pgfqpoint{0.868143in}{5.422011in}}%
\pgfpathlineto{\pgfqpoint{1.017780in}{5.337509in}}%
\pgfpathlineto{\pgfqpoint{1.167417in}{5.244427in}}%
\pgfpathlineto{\pgfqpoint{1.317054in}{5.151916in}}%
\pgfpathlineto{\pgfqpoint{1.466691in}{5.059736in}}%
\pgfpathlineto{\pgfqpoint{1.616328in}{4.959467in}}%
\pgfpathlineto{\pgfqpoint{1.765965in}{4.858372in}}%
\pgfpathlineto{\pgfqpoint{1.915602in}{4.750058in}}%
\pgfpathlineto{\pgfqpoint{2.065239in}{4.645485in}}%
\pgfpathlineto{\pgfqpoint{2.214876in}{4.529543in}}%
\pgfpathlineto{\pgfqpoint{2.364513in}{4.414970in}}%
\pgfpathlineto{\pgfqpoint{2.514150in}{4.297653in}}%
\pgfpathlineto{\pgfqpoint{2.663787in}{4.182035in}}%
\pgfpathlineto{\pgfqpoint{2.813424in}{4.062610in}}%
\pgfpathlineto{\pgfqpoint{2.963061in}{3.956573in}}%
\pgfpathlineto{\pgfqpoint{3.112698in}{3.858900in}}%
\pgfpathlineto{\pgfqpoint{3.262335in}{3.747000in}}%
\pgfpathlineto{\pgfqpoint{3.411972in}{3.634568in}}%
\pgfpathlineto{\pgfqpoint{3.561609in}{3.518344in}}%
\pgfpathlineto{\pgfqpoint{3.711246in}{3.416414in}}%
\pgfpathlineto{\pgfqpoint{3.711246in}{3.440458in}}%
\pgfpathlineto{\pgfqpoint{3.711246in}{3.440458in}}%
\pgfpathlineto{\pgfqpoint{3.561609in}{3.530211in}}%
\pgfpathlineto{\pgfqpoint{3.411972in}{3.648574in}}%
\pgfpathlineto{\pgfqpoint{3.262335in}{3.759932in}}%
\pgfpathlineto{\pgfqpoint{3.112698in}{3.870030in}}%
\pgfpathlineto{\pgfqpoint{2.963061in}{3.979392in}}%
\pgfpathlineto{\pgfqpoint{2.813424in}{4.090565in}}%
\pgfpathlineto{\pgfqpoint{2.663787in}{4.213400in}}%
\pgfpathlineto{\pgfqpoint{2.514150in}{4.323853in}}%
\pgfpathlineto{\pgfqpoint{2.364513in}{4.434965in}}%
\pgfpathlineto{\pgfqpoint{2.214876in}{4.538512in}}%
\pgfpathlineto{\pgfqpoint{2.065239in}{4.655825in}}%
\pgfpathlineto{\pgfqpoint{1.915602in}{4.760183in}}%
\pgfpathlineto{\pgfqpoint{1.765965in}{4.868452in}}%
\pgfpathlineto{\pgfqpoint{1.616328in}{4.967724in}}%
\pgfpathlineto{\pgfqpoint{1.466691in}{5.067972in}}%
\pgfpathlineto{\pgfqpoint{1.317054in}{5.166598in}}%
\pgfpathlineto{\pgfqpoint{1.167417in}{5.267808in}}%
\pgfpathlineto{\pgfqpoint{1.017780in}{5.341087in}}%
\pgfpathlineto{\pgfqpoint{0.868143in}{5.434092in}}%
\pgfpathlineto{\pgfqpoint{0.718506in}{5.516141in}}%
\pgfpathclose%
\pgfusepath{stroke,fill}%
\end{pgfscope}%
\begin{pgfscope}%
\pgfpathrectangle{\pgfqpoint{0.556250in}{3.311111in}}{\pgfqpoint{3.317253in}{2.316667in}}%
\pgfusepath{clip}%
\pgfsetbuttcap%
\pgfsetroundjoin%
\definecolor{currentfill}{rgb}{0.121569,0.466667,0.705882}%
\pgfsetfillcolor{currentfill}%
\pgfsetlinewidth{1.003750pt}%
\definecolor{currentstroke}{rgb}{0.121569,0.466667,0.705882}%
\pgfsetstrokecolor{currentstroke}%
\pgfsetdash{}{0pt}%
\pgfsys@defobject{currentmarker}{\pgfqpoint{-0.009821in}{-0.009821in}}{\pgfqpoint{0.009821in}{0.009821in}}{%
\pgfpathmoveto{\pgfqpoint{0.000000in}{-0.009821in}}%
\pgfpathcurveto{\pgfqpoint{0.002605in}{-0.009821in}}{\pgfqpoint{0.005103in}{-0.008786in}}{\pgfqpoint{0.006944in}{-0.006944in}}%
\pgfpathcurveto{\pgfqpoint{0.008786in}{-0.005103in}}{\pgfqpoint{0.009821in}{-0.002605in}}{\pgfqpoint{0.009821in}{0.000000in}}%
\pgfpathcurveto{\pgfqpoint{0.009821in}{0.002605in}}{\pgfqpoint{0.008786in}{0.005103in}}{\pgfqpoint{0.006944in}{0.006944in}}%
\pgfpathcurveto{\pgfqpoint{0.005103in}{0.008786in}}{\pgfqpoint{0.002605in}{0.009821in}}{\pgfqpoint{0.000000in}{0.009821in}}%
\pgfpathcurveto{\pgfqpoint{-0.002605in}{0.009821in}}{\pgfqpoint{-0.005103in}{0.008786in}}{\pgfqpoint{-0.006944in}{0.006944in}}%
\pgfpathcurveto{\pgfqpoint{-0.008786in}{0.005103in}}{\pgfqpoint{-0.009821in}{0.002605in}}{\pgfqpoint{-0.009821in}{0.000000in}}%
\pgfpathcurveto{\pgfqpoint{-0.009821in}{-0.002605in}}{\pgfqpoint{-0.008786in}{-0.005103in}}{\pgfqpoint{-0.006944in}{-0.006944in}}%
\pgfpathcurveto{\pgfqpoint{-0.005103in}{-0.008786in}}{\pgfqpoint{-0.002605in}{-0.009821in}}{\pgfqpoint{0.000000in}{-0.009821in}}%
\pgfpathclose%
\pgfusepath{stroke,fill}%
}%
\begin{pgfscope}%
\pgfsys@transformshift{0.718506in}{5.511459in}%
\pgfsys@useobject{currentmarker}{}%
\end{pgfscope}%
\begin{pgfscope}%
\pgfsys@transformshift{0.868143in}{5.428052in}%
\pgfsys@useobject{currentmarker}{}%
\end{pgfscope}%
\begin{pgfscope}%
\pgfsys@transformshift{1.017780in}{5.339298in}%
\pgfsys@useobject{currentmarker}{}%
\end{pgfscope}%
\begin{pgfscope}%
\pgfsys@transformshift{1.167417in}{5.256118in}%
\pgfsys@useobject{currentmarker}{}%
\end{pgfscope}%
\begin{pgfscope}%
\pgfsys@transformshift{1.317054in}{5.159257in}%
\pgfsys@useobject{currentmarker}{}%
\end{pgfscope}%
\begin{pgfscope}%
\pgfsys@transformshift{1.466691in}{5.063854in}%
\pgfsys@useobject{currentmarker}{}%
\end{pgfscope}%
\begin{pgfscope}%
\pgfsys@transformshift{1.616328in}{4.963595in}%
\pgfsys@useobject{currentmarker}{}%
\end{pgfscope}%
\begin{pgfscope}%
\pgfsys@transformshift{1.765965in}{4.863412in}%
\pgfsys@useobject{currentmarker}{}%
\end{pgfscope}%
\begin{pgfscope}%
\pgfsys@transformshift{1.915602in}{4.755121in}%
\pgfsys@useobject{currentmarker}{}%
\end{pgfscope}%
\begin{pgfscope}%
\pgfsys@transformshift{2.065239in}{4.650655in}%
\pgfsys@useobject{currentmarker}{}%
\end{pgfscope}%
\begin{pgfscope}%
\pgfsys@transformshift{2.214876in}{4.534027in}%
\pgfsys@useobject{currentmarker}{}%
\end{pgfscope}%
\begin{pgfscope}%
\pgfsys@transformshift{2.364513in}{4.424967in}%
\pgfsys@useobject{currentmarker}{}%
\end{pgfscope}%
\begin{pgfscope}%
\pgfsys@transformshift{2.514150in}{4.310753in}%
\pgfsys@useobject{currentmarker}{}%
\end{pgfscope}%
\begin{pgfscope}%
\pgfsys@transformshift{2.663787in}{4.197718in}%
\pgfsys@useobject{currentmarker}{}%
\end{pgfscope}%
\begin{pgfscope}%
\pgfsys@transformshift{2.813424in}{4.076587in}%
\pgfsys@useobject{currentmarker}{}%
\end{pgfscope}%
\begin{pgfscope}%
\pgfsys@transformshift{2.963061in}{3.967983in}%
\pgfsys@useobject{currentmarker}{}%
\end{pgfscope}%
\begin{pgfscope}%
\pgfsys@transformshift{3.112698in}{3.864465in}%
\pgfsys@useobject{currentmarker}{}%
\end{pgfscope}%
\begin{pgfscope}%
\pgfsys@transformshift{3.262335in}{3.753466in}%
\pgfsys@useobject{currentmarker}{}%
\end{pgfscope}%
\begin{pgfscope}%
\pgfsys@transformshift{3.411972in}{3.641571in}%
\pgfsys@useobject{currentmarker}{}%
\end{pgfscope}%
\begin{pgfscope}%
\pgfsys@transformshift{3.561609in}{3.524277in}%
\pgfsys@useobject{currentmarker}{}%
\end{pgfscope}%
\begin{pgfscope}%
\pgfsys@transformshift{3.711246in}{3.428436in}%
\pgfsys@useobject{currentmarker}{}%
\end{pgfscope}%
\end{pgfscope}%
\begin{pgfscope}%
\pgfsetbuttcap%
\pgfsetroundjoin%
\definecolor{currentfill}{rgb}{0.000000,0.000000,0.000000}%
\pgfsetfillcolor{currentfill}%
\pgfsetlinewidth{0.803000pt}%
\definecolor{currentstroke}{rgb}{0.000000,0.000000,0.000000}%
\pgfsetstrokecolor{currentstroke}%
\pgfsetdash{}{0pt}%
\pgfsys@defobject{currentmarker}{\pgfqpoint{0.000000in}{-0.048611in}}{\pgfqpoint{0.000000in}{0.000000in}}{%
\pgfpathmoveto{\pgfqpoint{0.000000in}{0.000000in}}%
\pgfpathlineto{\pgfqpoint{0.000000in}{-0.048611in}}%
\pgfusepath{stroke,fill}%
}%
\begin{pgfscope}%
\pgfsys@transformshift{1.167417in}{3.311111in}%
\pgfsys@useobject{currentmarker}{}%
\end{pgfscope}%
\end{pgfscope}%
\begin{pgfscope}%
\definecolor{textcolor}{rgb}{0.000000,0.000000,0.000000}%
\pgfsetstrokecolor{textcolor}%
\pgfsetfillcolor{textcolor}%
\pgftext[x=1.167417in,y=3.213889in,,top]{\color{textcolor}\sffamily\fontsize{10.000000}{12.000000}\selectfont 2.25}%
\end{pgfscope}%
\begin{pgfscope}%
\pgfsetbuttcap%
\pgfsetroundjoin%
\definecolor{currentfill}{rgb}{0.000000,0.000000,0.000000}%
\pgfsetfillcolor{currentfill}%
\pgfsetlinewidth{0.803000pt}%
\definecolor{currentstroke}{rgb}{0.000000,0.000000,0.000000}%
\pgfsetstrokecolor{currentstroke}%
\pgfsetdash{}{0pt}%
\pgfsys@defobject{currentmarker}{\pgfqpoint{0.000000in}{-0.048611in}}{\pgfqpoint{0.000000in}{0.000000in}}{%
\pgfpathmoveto{\pgfqpoint{0.000000in}{0.000000in}}%
\pgfpathlineto{\pgfqpoint{0.000000in}{-0.048611in}}%
\pgfusepath{stroke,fill}%
}%
\begin{pgfscope}%
\pgfsys@transformshift{1.915602in}{3.311111in}%
\pgfsys@useobject{currentmarker}{}%
\end{pgfscope}%
\end{pgfscope}%
\begin{pgfscope}%
\definecolor{textcolor}{rgb}{0.000000,0.000000,0.000000}%
\pgfsetstrokecolor{textcolor}%
\pgfsetfillcolor{textcolor}%
\pgftext[x=1.915602in,y=3.213889in,,top]{\color{textcolor}\sffamily\fontsize{10.000000}{12.000000}\selectfont 2.30}%
\end{pgfscope}%
\begin{pgfscope}%
\pgfsetbuttcap%
\pgfsetroundjoin%
\definecolor{currentfill}{rgb}{0.000000,0.000000,0.000000}%
\pgfsetfillcolor{currentfill}%
\pgfsetlinewidth{0.803000pt}%
\definecolor{currentstroke}{rgb}{0.000000,0.000000,0.000000}%
\pgfsetstrokecolor{currentstroke}%
\pgfsetdash{}{0pt}%
\pgfsys@defobject{currentmarker}{\pgfqpoint{0.000000in}{-0.048611in}}{\pgfqpoint{0.000000in}{0.000000in}}{%
\pgfpathmoveto{\pgfqpoint{0.000000in}{0.000000in}}%
\pgfpathlineto{\pgfqpoint{0.000000in}{-0.048611in}}%
\pgfusepath{stroke,fill}%
}%
\begin{pgfscope}%
\pgfsys@transformshift{2.663787in}{3.311111in}%
\pgfsys@useobject{currentmarker}{}%
\end{pgfscope}%
\end{pgfscope}%
\begin{pgfscope}%
\definecolor{textcolor}{rgb}{0.000000,0.000000,0.000000}%
\pgfsetstrokecolor{textcolor}%
\pgfsetfillcolor{textcolor}%
\pgftext[x=2.663787in,y=3.213889in,,top]{\color{textcolor}\sffamily\fontsize{10.000000}{12.000000}\selectfont 2.35}%
\end{pgfscope}%
\begin{pgfscope}%
\pgfsetbuttcap%
\pgfsetroundjoin%
\definecolor{currentfill}{rgb}{0.000000,0.000000,0.000000}%
\pgfsetfillcolor{currentfill}%
\pgfsetlinewidth{0.803000pt}%
\definecolor{currentstroke}{rgb}{0.000000,0.000000,0.000000}%
\pgfsetstrokecolor{currentstroke}%
\pgfsetdash{}{0pt}%
\pgfsys@defobject{currentmarker}{\pgfqpoint{0.000000in}{-0.048611in}}{\pgfqpoint{0.000000in}{0.000000in}}{%
\pgfpathmoveto{\pgfqpoint{0.000000in}{0.000000in}}%
\pgfpathlineto{\pgfqpoint{0.000000in}{-0.048611in}}%
\pgfusepath{stroke,fill}%
}%
\begin{pgfscope}%
\pgfsys@transformshift{3.411972in}{3.311111in}%
\pgfsys@useobject{currentmarker}{}%
\end{pgfscope}%
\end{pgfscope}%
\begin{pgfscope}%
\definecolor{textcolor}{rgb}{0.000000,0.000000,0.000000}%
\pgfsetstrokecolor{textcolor}%
\pgfsetfillcolor{textcolor}%
\pgftext[x=3.411972in,y=3.213889in,,top]{\color{textcolor}\sffamily\fontsize{10.000000}{12.000000}\selectfont 2.40}%
\end{pgfscope}%
\begin{pgfscope}%
\pgfsetbuttcap%
\pgfsetroundjoin%
\definecolor{currentfill}{rgb}{0.000000,0.000000,0.000000}%
\pgfsetfillcolor{currentfill}%
\pgfsetlinewidth{0.803000pt}%
\definecolor{currentstroke}{rgb}{0.000000,0.000000,0.000000}%
\pgfsetstrokecolor{currentstroke}%
\pgfsetdash{}{0pt}%
\pgfsys@defobject{currentmarker}{\pgfqpoint{-0.048611in}{0.000000in}}{\pgfqpoint{0.000000in}{0.000000in}}{%
\pgfpathmoveto{\pgfqpoint{0.000000in}{0.000000in}}%
\pgfpathlineto{\pgfqpoint{-0.048611in}{0.000000in}}%
\pgfusepath{stroke,fill}%
}%
\begin{pgfscope}%
\pgfsys@transformshift{0.556250in}{3.520276in}%
\pgfsys@useobject{currentmarker}{}%
\end{pgfscope}%
\end{pgfscope}%
\begin{pgfscope}%
\definecolor{textcolor}{rgb}{0.000000,0.000000,0.000000}%
\pgfsetstrokecolor{textcolor}%
\pgfsetfillcolor{textcolor}%
\pgftext[x=0.149783in,y=3.467515in,left,base]{\color{textcolor}\sffamily\fontsize{10.000000}{12.000000}\selectfont 0.50}%
\end{pgfscope}%
\begin{pgfscope}%
\pgfsetbuttcap%
\pgfsetroundjoin%
\definecolor{currentfill}{rgb}{0.000000,0.000000,0.000000}%
\pgfsetfillcolor{currentfill}%
\pgfsetlinewidth{0.803000pt}%
\definecolor{currentstroke}{rgb}{0.000000,0.000000,0.000000}%
\pgfsetstrokecolor{currentstroke}%
\pgfsetdash{}{0pt}%
\pgfsys@defobject{currentmarker}{\pgfqpoint{-0.048611in}{0.000000in}}{\pgfqpoint{0.000000in}{0.000000in}}{%
\pgfpathmoveto{\pgfqpoint{0.000000in}{0.000000in}}%
\pgfpathlineto{\pgfqpoint{-0.048611in}{0.000000in}}%
\pgfusepath{stroke,fill}%
}%
\begin{pgfscope}%
\pgfsys@transformshift{0.556250in}{3.883757in}%
\pgfsys@useobject{currentmarker}{}%
\end{pgfscope}%
\end{pgfscope}%
\begin{pgfscope}%
\definecolor{textcolor}{rgb}{0.000000,0.000000,0.000000}%
\pgfsetstrokecolor{textcolor}%
\pgfsetfillcolor{textcolor}%
\pgftext[x=0.149783in,y=3.830996in,left,base]{\color{textcolor}\sffamily\fontsize{10.000000}{12.000000}\selectfont 0.55}%
\end{pgfscope}%
\begin{pgfscope}%
\pgfsetbuttcap%
\pgfsetroundjoin%
\definecolor{currentfill}{rgb}{0.000000,0.000000,0.000000}%
\pgfsetfillcolor{currentfill}%
\pgfsetlinewidth{0.803000pt}%
\definecolor{currentstroke}{rgb}{0.000000,0.000000,0.000000}%
\pgfsetstrokecolor{currentstroke}%
\pgfsetdash{}{0pt}%
\pgfsys@defobject{currentmarker}{\pgfqpoint{-0.048611in}{0.000000in}}{\pgfqpoint{0.000000in}{0.000000in}}{%
\pgfpathmoveto{\pgfqpoint{0.000000in}{0.000000in}}%
\pgfpathlineto{\pgfqpoint{-0.048611in}{0.000000in}}%
\pgfusepath{stroke,fill}%
}%
\begin{pgfscope}%
\pgfsys@transformshift{0.556250in}{4.247238in}%
\pgfsys@useobject{currentmarker}{}%
\end{pgfscope}%
\end{pgfscope}%
\begin{pgfscope}%
\definecolor{textcolor}{rgb}{0.000000,0.000000,0.000000}%
\pgfsetstrokecolor{textcolor}%
\pgfsetfillcolor{textcolor}%
\pgftext[x=0.149783in,y=4.194476in,left,base]{\color{textcolor}\sffamily\fontsize{10.000000}{12.000000}\selectfont 0.60}%
\end{pgfscope}%
\begin{pgfscope}%
\pgfsetbuttcap%
\pgfsetroundjoin%
\definecolor{currentfill}{rgb}{0.000000,0.000000,0.000000}%
\pgfsetfillcolor{currentfill}%
\pgfsetlinewidth{0.803000pt}%
\definecolor{currentstroke}{rgb}{0.000000,0.000000,0.000000}%
\pgfsetstrokecolor{currentstroke}%
\pgfsetdash{}{0pt}%
\pgfsys@defobject{currentmarker}{\pgfqpoint{-0.048611in}{0.000000in}}{\pgfqpoint{0.000000in}{0.000000in}}{%
\pgfpathmoveto{\pgfqpoint{0.000000in}{0.000000in}}%
\pgfpathlineto{\pgfqpoint{-0.048611in}{0.000000in}}%
\pgfusepath{stroke,fill}%
}%
\begin{pgfscope}%
\pgfsys@transformshift{0.556250in}{4.610718in}%
\pgfsys@useobject{currentmarker}{}%
\end{pgfscope}%
\end{pgfscope}%
\begin{pgfscope}%
\definecolor{textcolor}{rgb}{0.000000,0.000000,0.000000}%
\pgfsetstrokecolor{textcolor}%
\pgfsetfillcolor{textcolor}%
\pgftext[x=0.149783in,y=4.557957in,left,base]{\color{textcolor}\sffamily\fontsize{10.000000}{12.000000}\selectfont 0.65}%
\end{pgfscope}%
\begin{pgfscope}%
\pgfsetbuttcap%
\pgfsetroundjoin%
\definecolor{currentfill}{rgb}{0.000000,0.000000,0.000000}%
\pgfsetfillcolor{currentfill}%
\pgfsetlinewidth{0.803000pt}%
\definecolor{currentstroke}{rgb}{0.000000,0.000000,0.000000}%
\pgfsetstrokecolor{currentstroke}%
\pgfsetdash{}{0pt}%
\pgfsys@defobject{currentmarker}{\pgfqpoint{-0.048611in}{0.000000in}}{\pgfqpoint{0.000000in}{0.000000in}}{%
\pgfpathmoveto{\pgfqpoint{0.000000in}{0.000000in}}%
\pgfpathlineto{\pgfqpoint{-0.048611in}{0.000000in}}%
\pgfusepath{stroke,fill}%
}%
\begin{pgfscope}%
\pgfsys@transformshift{0.556250in}{4.974199in}%
\pgfsys@useobject{currentmarker}{}%
\end{pgfscope}%
\end{pgfscope}%
\begin{pgfscope}%
\definecolor{textcolor}{rgb}{0.000000,0.000000,0.000000}%
\pgfsetstrokecolor{textcolor}%
\pgfsetfillcolor{textcolor}%
\pgftext[x=0.149783in,y=4.921437in,left,base]{\color{textcolor}\sffamily\fontsize{10.000000}{12.000000}\selectfont 0.70}%
\end{pgfscope}%
\begin{pgfscope}%
\pgfsetbuttcap%
\pgfsetroundjoin%
\definecolor{currentfill}{rgb}{0.000000,0.000000,0.000000}%
\pgfsetfillcolor{currentfill}%
\pgfsetlinewidth{0.803000pt}%
\definecolor{currentstroke}{rgb}{0.000000,0.000000,0.000000}%
\pgfsetstrokecolor{currentstroke}%
\pgfsetdash{}{0pt}%
\pgfsys@defobject{currentmarker}{\pgfqpoint{-0.048611in}{0.000000in}}{\pgfqpoint{0.000000in}{0.000000in}}{%
\pgfpathmoveto{\pgfqpoint{0.000000in}{0.000000in}}%
\pgfpathlineto{\pgfqpoint{-0.048611in}{0.000000in}}%
\pgfusepath{stroke,fill}%
}%
\begin{pgfscope}%
\pgfsys@transformshift{0.556250in}{5.337680in}%
\pgfsys@useobject{currentmarker}{}%
\end{pgfscope}%
\end{pgfscope}%
\begin{pgfscope}%
\definecolor{textcolor}{rgb}{0.000000,0.000000,0.000000}%
\pgfsetstrokecolor{textcolor}%
\pgfsetfillcolor{textcolor}%
\pgftext[x=0.149783in,y=5.284918in,left,base]{\color{textcolor}\sffamily\fontsize{10.000000}{12.000000}\selectfont 0.75}%
\end{pgfscope}%
\begin{pgfscope}%
\pgfpathrectangle{\pgfqpoint{0.556250in}{3.311111in}}{\pgfqpoint{3.317253in}{2.316667in}}%
\pgfusepath{clip}%
\pgfsetrectcap%
\pgfsetroundjoin%
\pgfsetlinewidth{1.505625pt}%
\definecolor{currentstroke}{rgb}{0.121569,0.466667,0.705882}%
\pgfsetstrokecolor{currentstroke}%
\pgfsetdash{}{0pt}%
\pgfpathmoveto{\pgfqpoint{0.718506in}{5.511459in}}%
\pgfpathlineto{\pgfqpoint{0.868143in}{5.428052in}}%
\pgfpathlineto{\pgfqpoint{1.017780in}{5.339298in}}%
\pgfpathlineto{\pgfqpoint{1.167417in}{5.256118in}}%
\pgfpathlineto{\pgfqpoint{1.317054in}{5.159257in}}%
\pgfpathlineto{\pgfqpoint{1.466691in}{5.063854in}}%
\pgfpathlineto{\pgfqpoint{1.616328in}{4.963595in}}%
\pgfpathlineto{\pgfqpoint{1.765965in}{4.863412in}}%
\pgfpathlineto{\pgfqpoint{1.915602in}{4.755121in}}%
\pgfpathlineto{\pgfqpoint{2.065239in}{4.650655in}}%
\pgfpathlineto{\pgfqpoint{2.214876in}{4.534027in}}%
\pgfpathlineto{\pgfqpoint{2.364513in}{4.424967in}}%
\pgfpathlineto{\pgfqpoint{2.514150in}{4.310753in}}%
\pgfpathlineto{\pgfqpoint{2.663787in}{4.197718in}}%
\pgfpathlineto{\pgfqpoint{2.813424in}{4.076587in}}%
\pgfpathlineto{\pgfqpoint{2.963061in}{3.967983in}}%
\pgfpathlineto{\pgfqpoint{3.112698in}{3.864465in}}%
\pgfpathlineto{\pgfqpoint{3.262335in}{3.753466in}}%
\pgfpathlineto{\pgfqpoint{3.411972in}{3.641571in}}%
\pgfpathlineto{\pgfqpoint{3.561609in}{3.524277in}}%
\pgfpathlineto{\pgfqpoint{3.711246in}{3.428436in}}%
\pgfusepath{stroke}%
\end{pgfscope}%
\begin{pgfscope}%
\pgfsetrectcap%
\pgfsetmiterjoin%
\pgfsetlinewidth{0.803000pt}%
\definecolor{currentstroke}{rgb}{0.000000,0.000000,0.000000}%
\pgfsetstrokecolor{currentstroke}%
\pgfsetdash{}{0pt}%
\pgfpathmoveto{\pgfqpoint{0.556250in}{3.311111in}}%
\pgfpathlineto{\pgfqpoint{0.556250in}{5.627778in}}%
\pgfusepath{stroke}%
\end{pgfscope}%
\begin{pgfscope}%
\pgfsetrectcap%
\pgfsetmiterjoin%
\pgfsetlinewidth{0.803000pt}%
\definecolor{currentstroke}{rgb}{0.000000,0.000000,0.000000}%
\pgfsetstrokecolor{currentstroke}%
\pgfsetdash{}{0pt}%
\pgfpathmoveto{\pgfqpoint{3.873503in}{3.311111in}}%
\pgfpathlineto{\pgfqpoint{3.873503in}{5.627778in}}%
\pgfusepath{stroke}%
\end{pgfscope}%
\begin{pgfscope}%
\pgfsetrectcap%
\pgfsetmiterjoin%
\pgfsetlinewidth{0.803000pt}%
\definecolor{currentstroke}{rgb}{0.000000,0.000000,0.000000}%
\pgfsetstrokecolor{currentstroke}%
\pgfsetdash{}{0pt}%
\pgfpathmoveto{\pgfqpoint{0.556250in}{3.311111in}}%
\pgfpathlineto{\pgfqpoint{3.873503in}{3.311111in}}%
\pgfusepath{stroke}%
\end{pgfscope}%
\begin{pgfscope}%
\pgfsetrectcap%
\pgfsetmiterjoin%
\pgfsetlinewidth{0.803000pt}%
\definecolor{currentstroke}{rgb}{0.000000,0.000000,0.000000}%
\pgfsetstrokecolor{currentstroke}%
\pgfsetdash{}{0pt}%
\pgfpathmoveto{\pgfqpoint{0.556250in}{5.627778in}}%
\pgfpathlineto{\pgfqpoint{3.873503in}{5.627778in}}%
\pgfusepath{stroke}%
\end{pgfscope}%
\begin{pgfscope}%
\definecolor{textcolor}{rgb}{0.000000,0.000000,0.000000}%
\pgfsetstrokecolor{textcolor}%
\pgfsetfillcolor{textcolor}%
\pgftext[x=2.214876in,y=5.711111in,,base]{\color{textcolor}\sffamily\fontsize{12.000000}{14.400000}\selectfont \(\displaystyle  N = 16 \)}%
\end{pgfscope}%
\begin{pgfscope}%
\pgfsetbuttcap%
\pgfsetmiterjoin%
\definecolor{currentfill}{rgb}{1.000000,1.000000,1.000000}%
\pgfsetfillcolor{currentfill}%
\pgfsetlinewidth{0.000000pt}%
\definecolor{currentstroke}{rgb}{0.000000,0.000000,0.000000}%
\pgfsetstrokecolor{currentstroke}%
\pgfsetstrokeopacity{0.000000}%
\pgfsetdash{}{0pt}%
\pgfpathmoveto{\pgfqpoint{4.389822in}{3.311111in}}%
\pgfpathlineto{\pgfqpoint{7.707075in}{3.311111in}}%
\pgfpathlineto{\pgfqpoint{7.707075in}{5.627778in}}%
\pgfpathlineto{\pgfqpoint{4.389822in}{5.627778in}}%
\pgfpathclose%
\pgfusepath{fill}%
\end{pgfscope}%
\begin{pgfscope}%
\pgfpathrectangle{\pgfqpoint{4.389822in}{3.311111in}}{\pgfqpoint{3.317253in}{2.316667in}}%
\pgfusepath{clip}%
\pgfsetbuttcap%
\pgfsetroundjoin%
\definecolor{currentfill}{rgb}{0.121569,0.466667,0.705882}%
\pgfsetfillcolor{currentfill}%
\pgfsetfillopacity{0.300000}%
\pgfsetlinewidth{1.003750pt}%
\definecolor{currentstroke}{rgb}{0.121569,0.466667,0.705882}%
\pgfsetstrokecolor{currentstroke}%
\pgfsetstrokeopacity{0.300000}%
\pgfsetdash{}{0pt}%
\pgfpathmoveto{\pgfqpoint{4.552079in}{5.518420in}}%
\pgfpathlineto{\pgfqpoint{4.552079in}{5.504474in}}%
\pgfpathlineto{\pgfqpoint{4.701716in}{5.453717in}}%
\pgfpathlineto{\pgfqpoint{4.851353in}{5.378142in}}%
\pgfpathlineto{\pgfqpoint{5.000990in}{5.323827in}}%
\pgfpathlineto{\pgfqpoint{5.150627in}{5.245434in}}%
\pgfpathlineto{\pgfqpoint{5.300264in}{5.171452in}}%
\pgfpathlineto{\pgfqpoint{5.449901in}{5.064970in}}%
\pgfpathlineto{\pgfqpoint{5.599538in}{4.976763in}}%
\pgfpathlineto{\pgfqpoint{5.749175in}{4.839721in}}%
\pgfpathlineto{\pgfqpoint{5.898812in}{4.719730in}}%
\pgfpathlineto{\pgfqpoint{6.048449in}{4.626375in}}%
\pgfpathlineto{\pgfqpoint{6.198086in}{4.481822in}}%
\pgfpathlineto{\pgfqpoint{6.347723in}{4.339296in}}%
\pgfpathlineto{\pgfqpoint{6.497360in}{4.200963in}}%
\pgfpathlineto{\pgfqpoint{6.646997in}{4.079487in}}%
\pgfpathlineto{\pgfqpoint{6.796634in}{3.960768in}}%
\pgfpathlineto{\pgfqpoint{6.946271in}{3.831137in}}%
\pgfpathlineto{\pgfqpoint{7.095908in}{3.724766in}}%
\pgfpathlineto{\pgfqpoint{7.245545in}{3.625714in}}%
\pgfpathlineto{\pgfqpoint{7.395182in}{3.523426in}}%
\pgfpathlineto{\pgfqpoint{7.544819in}{3.416414in}}%
\pgfpathlineto{\pgfqpoint{7.544819in}{3.477506in}}%
\pgfpathlineto{\pgfqpoint{7.544819in}{3.477506in}}%
\pgfpathlineto{\pgfqpoint{7.395182in}{3.563517in}}%
\pgfpathlineto{\pgfqpoint{7.245545in}{3.660638in}}%
\pgfpathlineto{\pgfqpoint{7.095908in}{3.744152in}}%
\pgfpathlineto{\pgfqpoint{6.946271in}{3.869339in}}%
\pgfpathlineto{\pgfqpoint{6.796634in}{3.987096in}}%
\pgfpathlineto{\pgfqpoint{6.646997in}{4.124101in}}%
\pgfpathlineto{\pgfqpoint{6.497360in}{4.255864in}}%
\pgfpathlineto{\pgfqpoint{6.347723in}{4.400570in}}%
\pgfpathlineto{\pgfqpoint{6.198086in}{4.508807in}}%
\pgfpathlineto{\pgfqpoint{6.048449in}{4.635204in}}%
\pgfpathlineto{\pgfqpoint{5.898812in}{4.772925in}}%
\pgfpathlineto{\pgfqpoint{5.749175in}{4.885478in}}%
\pgfpathlineto{\pgfqpoint{5.599538in}{4.996422in}}%
\pgfpathlineto{\pgfqpoint{5.449901in}{5.119076in}}%
\pgfpathlineto{\pgfqpoint{5.300264in}{5.183241in}}%
\pgfpathlineto{\pgfqpoint{5.150627in}{5.278223in}}%
\pgfpathlineto{\pgfqpoint{5.000990in}{5.332660in}}%
\pgfpathlineto{\pgfqpoint{4.851353in}{5.413558in}}%
\pgfpathlineto{\pgfqpoint{4.701716in}{5.462101in}}%
\pgfpathlineto{\pgfqpoint{4.552079in}{5.518420in}}%
\pgfpathclose%
\pgfusepath{stroke,fill}%
\end{pgfscope}%
\begin{pgfscope}%
\pgfpathrectangle{\pgfqpoint{4.389822in}{3.311111in}}{\pgfqpoint{3.317253in}{2.316667in}}%
\pgfusepath{clip}%
\pgfsetbuttcap%
\pgfsetroundjoin%
\definecolor{currentfill}{rgb}{0.121569,0.466667,0.705882}%
\pgfsetfillcolor{currentfill}%
\pgfsetlinewidth{1.003750pt}%
\definecolor{currentstroke}{rgb}{0.121569,0.466667,0.705882}%
\pgfsetstrokecolor{currentstroke}%
\pgfsetdash{}{0pt}%
\pgfsys@defobject{currentmarker}{\pgfqpoint{-0.009821in}{-0.009821in}}{\pgfqpoint{0.009821in}{0.009821in}}{%
\pgfpathmoveto{\pgfqpoint{0.000000in}{-0.009821in}}%
\pgfpathcurveto{\pgfqpoint{0.002605in}{-0.009821in}}{\pgfqpoint{0.005103in}{-0.008786in}}{\pgfqpoint{0.006944in}{-0.006944in}}%
\pgfpathcurveto{\pgfqpoint{0.008786in}{-0.005103in}}{\pgfqpoint{0.009821in}{-0.002605in}}{\pgfqpoint{0.009821in}{0.000000in}}%
\pgfpathcurveto{\pgfqpoint{0.009821in}{0.002605in}}{\pgfqpoint{0.008786in}{0.005103in}}{\pgfqpoint{0.006944in}{0.006944in}}%
\pgfpathcurveto{\pgfqpoint{0.005103in}{0.008786in}}{\pgfqpoint{0.002605in}{0.009821in}}{\pgfqpoint{0.000000in}{0.009821in}}%
\pgfpathcurveto{\pgfqpoint{-0.002605in}{0.009821in}}{\pgfqpoint{-0.005103in}{0.008786in}}{\pgfqpoint{-0.006944in}{0.006944in}}%
\pgfpathcurveto{\pgfqpoint{-0.008786in}{0.005103in}}{\pgfqpoint{-0.009821in}{0.002605in}}{\pgfqpoint{-0.009821in}{0.000000in}}%
\pgfpathcurveto{\pgfqpoint{-0.009821in}{-0.002605in}}{\pgfqpoint{-0.008786in}{-0.005103in}}{\pgfqpoint{-0.006944in}{-0.006944in}}%
\pgfpathcurveto{\pgfqpoint{-0.005103in}{-0.008786in}}{\pgfqpoint{-0.002605in}{-0.009821in}}{\pgfqpoint{0.000000in}{-0.009821in}}%
\pgfpathclose%
\pgfusepath{stroke,fill}%
}%
\begin{pgfscope}%
\pgfsys@transformshift{4.552079in}{5.511447in}%
\pgfsys@useobject{currentmarker}{}%
\end{pgfscope}%
\begin{pgfscope}%
\pgfsys@transformshift{4.701716in}{5.457909in}%
\pgfsys@useobject{currentmarker}{}%
\end{pgfscope}%
\begin{pgfscope}%
\pgfsys@transformshift{4.851353in}{5.395850in}%
\pgfsys@useobject{currentmarker}{}%
\end{pgfscope}%
\begin{pgfscope}%
\pgfsys@transformshift{5.000990in}{5.328243in}%
\pgfsys@useobject{currentmarker}{}%
\end{pgfscope}%
\begin{pgfscope}%
\pgfsys@transformshift{5.150627in}{5.261828in}%
\pgfsys@useobject{currentmarker}{}%
\end{pgfscope}%
\begin{pgfscope}%
\pgfsys@transformshift{5.300264in}{5.177347in}%
\pgfsys@useobject{currentmarker}{}%
\end{pgfscope}%
\begin{pgfscope}%
\pgfsys@transformshift{5.449901in}{5.092023in}%
\pgfsys@useobject{currentmarker}{}%
\end{pgfscope}%
\begin{pgfscope}%
\pgfsys@transformshift{5.599538in}{4.986593in}%
\pgfsys@useobject{currentmarker}{}%
\end{pgfscope}%
\begin{pgfscope}%
\pgfsys@transformshift{5.749175in}{4.862600in}%
\pgfsys@useobject{currentmarker}{}%
\end{pgfscope}%
\begin{pgfscope}%
\pgfsys@transformshift{5.898812in}{4.746327in}%
\pgfsys@useobject{currentmarker}{}%
\end{pgfscope}%
\begin{pgfscope}%
\pgfsys@transformshift{6.048449in}{4.630789in}%
\pgfsys@useobject{currentmarker}{}%
\end{pgfscope}%
\begin{pgfscope}%
\pgfsys@transformshift{6.198086in}{4.495314in}%
\pgfsys@useobject{currentmarker}{}%
\end{pgfscope}%
\begin{pgfscope}%
\pgfsys@transformshift{6.347723in}{4.369933in}%
\pgfsys@useobject{currentmarker}{}%
\end{pgfscope}%
\begin{pgfscope}%
\pgfsys@transformshift{6.497360in}{4.228414in}%
\pgfsys@useobject{currentmarker}{}%
\end{pgfscope}%
\begin{pgfscope}%
\pgfsys@transformshift{6.646997in}{4.101794in}%
\pgfsys@useobject{currentmarker}{}%
\end{pgfscope}%
\begin{pgfscope}%
\pgfsys@transformshift{6.796634in}{3.973932in}%
\pgfsys@useobject{currentmarker}{}%
\end{pgfscope}%
\begin{pgfscope}%
\pgfsys@transformshift{6.946271in}{3.850238in}%
\pgfsys@useobject{currentmarker}{}%
\end{pgfscope}%
\begin{pgfscope}%
\pgfsys@transformshift{7.095908in}{3.734459in}%
\pgfsys@useobject{currentmarker}{}%
\end{pgfscope}%
\begin{pgfscope}%
\pgfsys@transformshift{7.245545in}{3.643176in}%
\pgfsys@useobject{currentmarker}{}%
\end{pgfscope}%
\begin{pgfscope}%
\pgfsys@transformshift{7.395182in}{3.543472in}%
\pgfsys@useobject{currentmarker}{}%
\end{pgfscope}%
\begin{pgfscope}%
\pgfsys@transformshift{7.544819in}{3.446960in}%
\pgfsys@useobject{currentmarker}{}%
\end{pgfscope}%
\end{pgfscope}%
\begin{pgfscope}%
\pgfsetbuttcap%
\pgfsetroundjoin%
\definecolor{currentfill}{rgb}{0.000000,0.000000,0.000000}%
\pgfsetfillcolor{currentfill}%
\pgfsetlinewidth{0.803000pt}%
\definecolor{currentstroke}{rgb}{0.000000,0.000000,0.000000}%
\pgfsetstrokecolor{currentstroke}%
\pgfsetdash{}{0pt}%
\pgfsys@defobject{currentmarker}{\pgfqpoint{0.000000in}{-0.048611in}}{\pgfqpoint{0.000000in}{0.000000in}}{%
\pgfpathmoveto{\pgfqpoint{0.000000in}{0.000000in}}%
\pgfpathlineto{\pgfqpoint{0.000000in}{-0.048611in}}%
\pgfusepath{stroke,fill}%
}%
\begin{pgfscope}%
\pgfsys@transformshift{4.701716in}{3.311111in}%
\pgfsys@useobject{currentmarker}{}%
\end{pgfscope}%
\end{pgfscope}%
\begin{pgfscope}%
\definecolor{textcolor}{rgb}{0.000000,0.000000,0.000000}%
\pgfsetstrokecolor{textcolor}%
\pgfsetfillcolor{textcolor}%
\pgftext[x=4.701716in,y=3.213889in,,top]{\color{textcolor}\sffamily\fontsize{10.000000}{12.000000}\selectfont 2.20}%
\end{pgfscope}%
\begin{pgfscope}%
\pgfsetbuttcap%
\pgfsetroundjoin%
\definecolor{currentfill}{rgb}{0.000000,0.000000,0.000000}%
\pgfsetfillcolor{currentfill}%
\pgfsetlinewidth{0.803000pt}%
\definecolor{currentstroke}{rgb}{0.000000,0.000000,0.000000}%
\pgfsetstrokecolor{currentstroke}%
\pgfsetdash{}{0pt}%
\pgfsys@defobject{currentmarker}{\pgfqpoint{0.000000in}{-0.048611in}}{\pgfqpoint{0.000000in}{0.000000in}}{%
\pgfpathmoveto{\pgfqpoint{0.000000in}{0.000000in}}%
\pgfpathlineto{\pgfqpoint{0.000000in}{-0.048611in}}%
\pgfusepath{stroke,fill}%
}%
\begin{pgfscope}%
\pgfsys@transformshift{5.449901in}{3.311111in}%
\pgfsys@useobject{currentmarker}{}%
\end{pgfscope}%
\end{pgfscope}%
\begin{pgfscope}%
\definecolor{textcolor}{rgb}{0.000000,0.000000,0.000000}%
\pgfsetstrokecolor{textcolor}%
\pgfsetfillcolor{textcolor}%
\pgftext[x=5.449901in,y=3.213889in,,top]{\color{textcolor}\sffamily\fontsize{10.000000}{12.000000}\selectfont 2.25}%
\end{pgfscope}%
\begin{pgfscope}%
\pgfsetbuttcap%
\pgfsetroundjoin%
\definecolor{currentfill}{rgb}{0.000000,0.000000,0.000000}%
\pgfsetfillcolor{currentfill}%
\pgfsetlinewidth{0.803000pt}%
\definecolor{currentstroke}{rgb}{0.000000,0.000000,0.000000}%
\pgfsetstrokecolor{currentstroke}%
\pgfsetdash{}{0pt}%
\pgfsys@defobject{currentmarker}{\pgfqpoint{0.000000in}{-0.048611in}}{\pgfqpoint{0.000000in}{0.000000in}}{%
\pgfpathmoveto{\pgfqpoint{0.000000in}{0.000000in}}%
\pgfpathlineto{\pgfqpoint{0.000000in}{-0.048611in}}%
\pgfusepath{stroke,fill}%
}%
\begin{pgfscope}%
\pgfsys@transformshift{6.198086in}{3.311111in}%
\pgfsys@useobject{currentmarker}{}%
\end{pgfscope}%
\end{pgfscope}%
\begin{pgfscope}%
\definecolor{textcolor}{rgb}{0.000000,0.000000,0.000000}%
\pgfsetstrokecolor{textcolor}%
\pgfsetfillcolor{textcolor}%
\pgftext[x=6.198086in,y=3.213889in,,top]{\color{textcolor}\sffamily\fontsize{10.000000}{12.000000}\selectfont 2.30}%
\end{pgfscope}%
\begin{pgfscope}%
\pgfsetbuttcap%
\pgfsetroundjoin%
\definecolor{currentfill}{rgb}{0.000000,0.000000,0.000000}%
\pgfsetfillcolor{currentfill}%
\pgfsetlinewidth{0.803000pt}%
\definecolor{currentstroke}{rgb}{0.000000,0.000000,0.000000}%
\pgfsetstrokecolor{currentstroke}%
\pgfsetdash{}{0pt}%
\pgfsys@defobject{currentmarker}{\pgfqpoint{0.000000in}{-0.048611in}}{\pgfqpoint{0.000000in}{0.000000in}}{%
\pgfpathmoveto{\pgfqpoint{0.000000in}{0.000000in}}%
\pgfpathlineto{\pgfqpoint{0.000000in}{-0.048611in}}%
\pgfusepath{stroke,fill}%
}%
\begin{pgfscope}%
\pgfsys@transformshift{6.946271in}{3.311111in}%
\pgfsys@useobject{currentmarker}{}%
\end{pgfscope}%
\end{pgfscope}%
\begin{pgfscope}%
\definecolor{textcolor}{rgb}{0.000000,0.000000,0.000000}%
\pgfsetstrokecolor{textcolor}%
\pgfsetfillcolor{textcolor}%
\pgftext[x=6.946271in,y=3.213889in,,top]{\color{textcolor}\sffamily\fontsize{10.000000}{12.000000}\selectfont 2.35}%
\end{pgfscope}%
\begin{pgfscope}%
\pgfsetbuttcap%
\pgfsetroundjoin%
\definecolor{currentfill}{rgb}{0.000000,0.000000,0.000000}%
\pgfsetfillcolor{currentfill}%
\pgfsetlinewidth{0.803000pt}%
\definecolor{currentstroke}{rgb}{0.000000,0.000000,0.000000}%
\pgfsetstrokecolor{currentstroke}%
\pgfsetdash{}{0pt}%
\pgfsys@defobject{currentmarker}{\pgfqpoint{0.000000in}{-0.048611in}}{\pgfqpoint{0.000000in}{0.000000in}}{%
\pgfpathmoveto{\pgfqpoint{0.000000in}{0.000000in}}%
\pgfpathlineto{\pgfqpoint{0.000000in}{-0.048611in}}%
\pgfusepath{stroke,fill}%
}%
\begin{pgfscope}%
\pgfsys@transformshift{7.694456in}{3.311111in}%
\pgfsys@useobject{currentmarker}{}%
\end{pgfscope}%
\end{pgfscope}%
\begin{pgfscope}%
\definecolor{textcolor}{rgb}{0.000000,0.000000,0.000000}%
\pgfsetstrokecolor{textcolor}%
\pgfsetfillcolor{textcolor}%
\pgftext[x=7.694456in,y=3.213889in,,top]{\color{textcolor}\sffamily\fontsize{10.000000}{12.000000}\selectfont 2.40}%
\end{pgfscope}%
\begin{pgfscope}%
\pgfsetbuttcap%
\pgfsetroundjoin%
\definecolor{currentfill}{rgb}{0.000000,0.000000,0.000000}%
\pgfsetfillcolor{currentfill}%
\pgfsetlinewidth{0.803000pt}%
\definecolor{currentstroke}{rgb}{0.000000,0.000000,0.000000}%
\pgfsetstrokecolor{currentstroke}%
\pgfsetdash{}{0pt}%
\pgfsys@defobject{currentmarker}{\pgfqpoint{-0.048611in}{0.000000in}}{\pgfqpoint{0.000000in}{0.000000in}}{%
\pgfpathmoveto{\pgfqpoint{0.000000in}{0.000000in}}%
\pgfpathlineto{\pgfqpoint{-0.048611in}{0.000000in}}%
\pgfusepath{stroke,fill}%
}%
\begin{pgfscope}%
\pgfsys@transformshift{4.389822in}{3.743567in}%
\pgfsys@useobject{currentmarker}{}%
\end{pgfscope}%
\end{pgfscope}%
\begin{pgfscope}%
\definecolor{textcolor}{rgb}{0.000000,0.000000,0.000000}%
\pgfsetstrokecolor{textcolor}%
\pgfsetfillcolor{textcolor}%
\pgftext[x=4.071721in,y=3.690805in,left,base]{\color{textcolor}\sffamily\fontsize{10.000000}{12.000000}\selectfont 0.4}%
\end{pgfscope}%
\begin{pgfscope}%
\pgfsetbuttcap%
\pgfsetroundjoin%
\definecolor{currentfill}{rgb}{0.000000,0.000000,0.000000}%
\pgfsetfillcolor{currentfill}%
\pgfsetlinewidth{0.803000pt}%
\definecolor{currentstroke}{rgb}{0.000000,0.000000,0.000000}%
\pgfsetstrokecolor{currentstroke}%
\pgfsetdash{}{0pt}%
\pgfsys@defobject{currentmarker}{\pgfqpoint{-0.048611in}{0.000000in}}{\pgfqpoint{0.000000in}{0.000000in}}{%
\pgfpathmoveto{\pgfqpoint{0.000000in}{0.000000in}}%
\pgfpathlineto{\pgfqpoint{-0.048611in}{0.000000in}}%
\pgfusepath{stroke,fill}%
}%
\begin{pgfscope}%
\pgfsys@transformshift{4.389822in}{4.188842in}%
\pgfsys@useobject{currentmarker}{}%
\end{pgfscope}%
\end{pgfscope}%
\begin{pgfscope}%
\definecolor{textcolor}{rgb}{0.000000,0.000000,0.000000}%
\pgfsetstrokecolor{textcolor}%
\pgfsetfillcolor{textcolor}%
\pgftext[x=4.071721in,y=4.136080in,left,base]{\color{textcolor}\sffamily\fontsize{10.000000}{12.000000}\selectfont 0.5}%
\end{pgfscope}%
\begin{pgfscope}%
\pgfsetbuttcap%
\pgfsetroundjoin%
\definecolor{currentfill}{rgb}{0.000000,0.000000,0.000000}%
\pgfsetfillcolor{currentfill}%
\pgfsetlinewidth{0.803000pt}%
\definecolor{currentstroke}{rgb}{0.000000,0.000000,0.000000}%
\pgfsetstrokecolor{currentstroke}%
\pgfsetdash{}{0pt}%
\pgfsys@defobject{currentmarker}{\pgfqpoint{-0.048611in}{0.000000in}}{\pgfqpoint{0.000000in}{0.000000in}}{%
\pgfpathmoveto{\pgfqpoint{0.000000in}{0.000000in}}%
\pgfpathlineto{\pgfqpoint{-0.048611in}{0.000000in}}%
\pgfusepath{stroke,fill}%
}%
\begin{pgfscope}%
\pgfsys@transformshift{4.389822in}{4.634117in}%
\pgfsys@useobject{currentmarker}{}%
\end{pgfscope}%
\end{pgfscope}%
\begin{pgfscope}%
\definecolor{textcolor}{rgb}{0.000000,0.000000,0.000000}%
\pgfsetstrokecolor{textcolor}%
\pgfsetfillcolor{textcolor}%
\pgftext[x=4.071721in,y=4.581355in,left,base]{\color{textcolor}\sffamily\fontsize{10.000000}{12.000000}\selectfont 0.6}%
\end{pgfscope}%
\begin{pgfscope}%
\pgfsetbuttcap%
\pgfsetroundjoin%
\definecolor{currentfill}{rgb}{0.000000,0.000000,0.000000}%
\pgfsetfillcolor{currentfill}%
\pgfsetlinewidth{0.803000pt}%
\definecolor{currentstroke}{rgb}{0.000000,0.000000,0.000000}%
\pgfsetstrokecolor{currentstroke}%
\pgfsetdash{}{0pt}%
\pgfsys@defobject{currentmarker}{\pgfqpoint{-0.048611in}{0.000000in}}{\pgfqpoint{0.000000in}{0.000000in}}{%
\pgfpathmoveto{\pgfqpoint{0.000000in}{0.000000in}}%
\pgfpathlineto{\pgfqpoint{-0.048611in}{0.000000in}}%
\pgfusepath{stroke,fill}%
}%
\begin{pgfscope}%
\pgfsys@transformshift{4.389822in}{5.079392in}%
\pgfsys@useobject{currentmarker}{}%
\end{pgfscope}%
\end{pgfscope}%
\begin{pgfscope}%
\definecolor{textcolor}{rgb}{0.000000,0.000000,0.000000}%
\pgfsetstrokecolor{textcolor}%
\pgfsetfillcolor{textcolor}%
\pgftext[x=4.071721in,y=5.026630in,left,base]{\color{textcolor}\sffamily\fontsize{10.000000}{12.000000}\selectfont 0.7}%
\end{pgfscope}%
\begin{pgfscope}%
\pgfsetbuttcap%
\pgfsetroundjoin%
\definecolor{currentfill}{rgb}{0.000000,0.000000,0.000000}%
\pgfsetfillcolor{currentfill}%
\pgfsetlinewidth{0.803000pt}%
\definecolor{currentstroke}{rgb}{0.000000,0.000000,0.000000}%
\pgfsetstrokecolor{currentstroke}%
\pgfsetdash{}{0pt}%
\pgfsys@defobject{currentmarker}{\pgfqpoint{-0.048611in}{0.000000in}}{\pgfqpoint{0.000000in}{0.000000in}}{%
\pgfpathmoveto{\pgfqpoint{0.000000in}{0.000000in}}%
\pgfpathlineto{\pgfqpoint{-0.048611in}{0.000000in}}%
\pgfusepath{stroke,fill}%
}%
\begin{pgfscope}%
\pgfsys@transformshift{4.389822in}{5.524667in}%
\pgfsys@useobject{currentmarker}{}%
\end{pgfscope}%
\end{pgfscope}%
\begin{pgfscope}%
\definecolor{textcolor}{rgb}{0.000000,0.000000,0.000000}%
\pgfsetstrokecolor{textcolor}%
\pgfsetfillcolor{textcolor}%
\pgftext[x=4.071721in,y=5.471905in,left,base]{\color{textcolor}\sffamily\fontsize{10.000000}{12.000000}\selectfont 0.8}%
\end{pgfscope}%
\begin{pgfscope}%
\pgfpathrectangle{\pgfqpoint{4.389822in}{3.311111in}}{\pgfqpoint{3.317253in}{2.316667in}}%
\pgfusepath{clip}%
\pgfsetrectcap%
\pgfsetroundjoin%
\pgfsetlinewidth{1.505625pt}%
\definecolor{currentstroke}{rgb}{0.121569,0.466667,0.705882}%
\pgfsetstrokecolor{currentstroke}%
\pgfsetdash{}{0pt}%
\pgfpathmoveto{\pgfqpoint{4.552079in}{5.511447in}}%
\pgfpathlineto{\pgfqpoint{4.701716in}{5.457909in}}%
\pgfpathlineto{\pgfqpoint{4.851353in}{5.395850in}}%
\pgfpathlineto{\pgfqpoint{5.000990in}{5.328243in}}%
\pgfpathlineto{\pgfqpoint{5.150627in}{5.261828in}}%
\pgfpathlineto{\pgfqpoint{5.300264in}{5.177347in}}%
\pgfpathlineto{\pgfqpoint{5.449901in}{5.092023in}}%
\pgfpathlineto{\pgfqpoint{5.599538in}{4.986593in}}%
\pgfpathlineto{\pgfqpoint{5.749175in}{4.862600in}}%
\pgfpathlineto{\pgfqpoint{5.898812in}{4.746327in}}%
\pgfpathlineto{\pgfqpoint{6.048449in}{4.630789in}}%
\pgfpathlineto{\pgfqpoint{6.198086in}{4.495314in}}%
\pgfpathlineto{\pgfqpoint{6.347723in}{4.369933in}}%
\pgfpathlineto{\pgfqpoint{6.497360in}{4.228414in}}%
\pgfpathlineto{\pgfqpoint{6.646997in}{4.101794in}}%
\pgfpathlineto{\pgfqpoint{6.796634in}{3.973932in}}%
\pgfpathlineto{\pgfqpoint{6.946271in}{3.850238in}}%
\pgfpathlineto{\pgfqpoint{7.095908in}{3.734459in}}%
\pgfpathlineto{\pgfqpoint{7.245545in}{3.643176in}}%
\pgfpathlineto{\pgfqpoint{7.395182in}{3.543472in}}%
\pgfpathlineto{\pgfqpoint{7.544819in}{3.446960in}}%
\pgfusepath{stroke}%
\end{pgfscope}%
\begin{pgfscope}%
\pgfsetrectcap%
\pgfsetmiterjoin%
\pgfsetlinewidth{0.803000pt}%
\definecolor{currentstroke}{rgb}{0.000000,0.000000,0.000000}%
\pgfsetstrokecolor{currentstroke}%
\pgfsetdash{}{0pt}%
\pgfpathmoveto{\pgfqpoint{4.389822in}{3.311111in}}%
\pgfpathlineto{\pgfqpoint{4.389822in}{5.627778in}}%
\pgfusepath{stroke}%
\end{pgfscope}%
\begin{pgfscope}%
\pgfsetrectcap%
\pgfsetmiterjoin%
\pgfsetlinewidth{0.803000pt}%
\definecolor{currentstroke}{rgb}{0.000000,0.000000,0.000000}%
\pgfsetstrokecolor{currentstroke}%
\pgfsetdash{}{0pt}%
\pgfpathmoveto{\pgfqpoint{7.707075in}{3.311111in}}%
\pgfpathlineto{\pgfqpoint{7.707075in}{5.627778in}}%
\pgfusepath{stroke}%
\end{pgfscope}%
\begin{pgfscope}%
\pgfsetrectcap%
\pgfsetmiterjoin%
\pgfsetlinewidth{0.803000pt}%
\definecolor{currentstroke}{rgb}{0.000000,0.000000,0.000000}%
\pgfsetstrokecolor{currentstroke}%
\pgfsetdash{}{0pt}%
\pgfpathmoveto{\pgfqpoint{4.389822in}{3.311111in}}%
\pgfpathlineto{\pgfqpoint{7.707075in}{3.311111in}}%
\pgfusepath{stroke}%
\end{pgfscope}%
\begin{pgfscope}%
\pgfsetrectcap%
\pgfsetmiterjoin%
\pgfsetlinewidth{0.803000pt}%
\definecolor{currentstroke}{rgb}{0.000000,0.000000,0.000000}%
\pgfsetstrokecolor{currentstroke}%
\pgfsetdash{}{0pt}%
\pgfpathmoveto{\pgfqpoint{4.389822in}{5.627778in}}%
\pgfpathlineto{\pgfqpoint{7.707075in}{5.627778in}}%
\pgfusepath{stroke}%
\end{pgfscope}%
\begin{pgfscope}%
\definecolor{textcolor}{rgb}{0.000000,0.000000,0.000000}%
\pgfsetstrokecolor{textcolor}%
\pgfsetfillcolor{textcolor}%
\pgftext[x=6.048449in,y=5.711111in,,base]{\color{textcolor}\sffamily\fontsize{12.000000}{14.400000}\selectfont \(\displaystyle  N = 32 \)}%
\end{pgfscope}%
\begin{pgfscope}%
\pgfsetbuttcap%
\pgfsetmiterjoin%
\definecolor{currentfill}{rgb}{1.000000,1.000000,1.000000}%
\pgfsetfillcolor{currentfill}%
\pgfsetlinewidth{0.000000pt}%
\definecolor{currentstroke}{rgb}{0.000000,0.000000,0.000000}%
\pgfsetstrokecolor{currentstroke}%
\pgfsetstrokeopacity{0.000000}%
\pgfsetdash{}{0pt}%
\pgfpathmoveto{\pgfqpoint{0.556250in}{0.386111in}}%
\pgfpathlineto{\pgfqpoint{3.873503in}{0.386111in}}%
\pgfpathlineto{\pgfqpoint{3.873503in}{2.702778in}}%
\pgfpathlineto{\pgfqpoint{0.556250in}{2.702778in}}%
\pgfpathclose%
\pgfusepath{fill}%
\end{pgfscope}%
\begin{pgfscope}%
\pgfpathrectangle{\pgfqpoint{0.556250in}{0.386111in}}{\pgfqpoint{3.317253in}{2.316667in}}%
\pgfusepath{clip}%
\pgfsetbuttcap%
\pgfsetroundjoin%
\definecolor{currentfill}{rgb}{0.121569,0.466667,0.705882}%
\pgfsetfillcolor{currentfill}%
\pgfsetfillopacity{0.300000}%
\pgfsetlinewidth{1.003750pt}%
\definecolor{currentstroke}{rgb}{0.121569,0.466667,0.705882}%
\pgfsetstrokecolor{currentstroke}%
\pgfsetstrokeopacity{0.300000}%
\pgfsetdash{}{0pt}%
\pgfpathmoveto{\pgfqpoint{0.718506in}{2.591228in}}%
\pgfpathlineto{\pgfqpoint{0.718506in}{2.581688in}}%
\pgfpathlineto{\pgfqpoint{0.868143in}{2.542488in}}%
\pgfpathlineto{\pgfqpoint{1.017780in}{2.497048in}}%
\pgfpathlineto{\pgfqpoint{1.167417in}{2.457121in}}%
\pgfpathlineto{\pgfqpoint{1.317054in}{2.395443in}}%
\pgfpathlineto{\pgfqpoint{1.466691in}{2.306337in}}%
\pgfpathlineto{\pgfqpoint{1.616328in}{2.233548in}}%
\pgfpathlineto{\pgfqpoint{1.765965in}{2.113340in}}%
\pgfpathlineto{\pgfqpoint{1.915602in}{2.024356in}}%
\pgfpathlineto{\pgfqpoint{2.065239in}{1.759660in}}%
\pgfpathlineto{\pgfqpoint{2.214876in}{1.646038in}}%
\pgfpathlineto{\pgfqpoint{2.364513in}{1.422530in}}%
\pgfpathlineto{\pgfqpoint{2.514150in}{1.271430in}}%
\pgfpathlineto{\pgfqpoint{2.663787in}{1.105631in}}%
\pgfpathlineto{\pgfqpoint{2.813424in}{0.929159in}}%
\pgfpathlineto{\pgfqpoint{2.963061in}{0.804878in}}%
\pgfpathlineto{\pgfqpoint{3.112698in}{0.765460in}}%
\pgfpathlineto{\pgfqpoint{3.262335in}{0.643558in}}%
\pgfpathlineto{\pgfqpoint{3.411972in}{0.590896in}}%
\pgfpathlineto{\pgfqpoint{3.561609in}{0.515140in}}%
\pgfpathlineto{\pgfqpoint{3.711246in}{0.491414in}}%
\pgfpathlineto{\pgfqpoint{3.711246in}{0.537443in}}%
\pgfpathlineto{\pgfqpoint{3.711246in}{0.537443in}}%
\pgfpathlineto{\pgfqpoint{3.561609in}{0.570700in}}%
\pgfpathlineto{\pgfqpoint{3.411972in}{0.679390in}}%
\pgfpathlineto{\pgfqpoint{3.262335in}{0.714508in}}%
\pgfpathlineto{\pgfqpoint{3.112698in}{0.812504in}}%
\pgfpathlineto{\pgfqpoint{2.963061in}{0.919227in}}%
\pgfpathlineto{\pgfqpoint{2.813424in}{1.036878in}}%
\pgfpathlineto{\pgfqpoint{2.663787in}{1.198061in}}%
\pgfpathlineto{\pgfqpoint{2.514150in}{1.437350in}}%
\pgfpathlineto{\pgfqpoint{2.364513in}{1.623348in}}%
\pgfpathlineto{\pgfqpoint{2.214876in}{1.729439in}}%
\pgfpathlineto{\pgfqpoint{2.065239in}{1.944025in}}%
\pgfpathlineto{\pgfqpoint{1.915602in}{2.080300in}}%
\pgfpathlineto{\pgfqpoint{1.765965in}{2.190832in}}%
\pgfpathlineto{\pgfqpoint{1.616328in}{2.257690in}}%
\pgfpathlineto{\pgfqpoint{1.466691in}{2.353108in}}%
\pgfpathlineto{\pgfqpoint{1.317054in}{2.415677in}}%
\pgfpathlineto{\pgfqpoint{1.167417in}{2.467775in}}%
\pgfpathlineto{\pgfqpoint{1.017780in}{2.512940in}}%
\pgfpathlineto{\pgfqpoint{0.868143in}{2.564836in}}%
\pgfpathlineto{\pgfqpoint{0.718506in}{2.591228in}}%
\pgfpathclose%
\pgfusepath{stroke,fill}%
\end{pgfscope}%
\begin{pgfscope}%
\pgfpathrectangle{\pgfqpoint{0.556250in}{0.386111in}}{\pgfqpoint{3.317253in}{2.316667in}}%
\pgfusepath{clip}%
\pgfsetbuttcap%
\pgfsetroundjoin%
\definecolor{currentfill}{rgb}{0.121569,0.466667,0.705882}%
\pgfsetfillcolor{currentfill}%
\pgfsetlinewidth{1.003750pt}%
\definecolor{currentstroke}{rgb}{0.121569,0.466667,0.705882}%
\pgfsetstrokecolor{currentstroke}%
\pgfsetdash{}{0pt}%
\pgfsys@defobject{currentmarker}{\pgfqpoint{-0.009821in}{-0.009821in}}{\pgfqpoint{0.009821in}{0.009821in}}{%
\pgfpathmoveto{\pgfqpoint{0.000000in}{-0.009821in}}%
\pgfpathcurveto{\pgfqpoint{0.002605in}{-0.009821in}}{\pgfqpoint{0.005103in}{-0.008786in}}{\pgfqpoint{0.006944in}{-0.006944in}}%
\pgfpathcurveto{\pgfqpoint{0.008786in}{-0.005103in}}{\pgfqpoint{0.009821in}{-0.002605in}}{\pgfqpoint{0.009821in}{0.000000in}}%
\pgfpathcurveto{\pgfqpoint{0.009821in}{0.002605in}}{\pgfqpoint{0.008786in}{0.005103in}}{\pgfqpoint{0.006944in}{0.006944in}}%
\pgfpathcurveto{\pgfqpoint{0.005103in}{0.008786in}}{\pgfqpoint{0.002605in}{0.009821in}}{\pgfqpoint{0.000000in}{0.009821in}}%
\pgfpathcurveto{\pgfqpoint{-0.002605in}{0.009821in}}{\pgfqpoint{-0.005103in}{0.008786in}}{\pgfqpoint{-0.006944in}{0.006944in}}%
\pgfpathcurveto{\pgfqpoint{-0.008786in}{0.005103in}}{\pgfqpoint{-0.009821in}{0.002605in}}{\pgfqpoint{-0.009821in}{0.000000in}}%
\pgfpathcurveto{\pgfqpoint{-0.009821in}{-0.002605in}}{\pgfqpoint{-0.008786in}{-0.005103in}}{\pgfqpoint{-0.006944in}{-0.006944in}}%
\pgfpathcurveto{\pgfqpoint{-0.005103in}{-0.008786in}}{\pgfqpoint{-0.002605in}{-0.009821in}}{\pgfqpoint{0.000000in}{-0.009821in}}%
\pgfpathclose%
\pgfusepath{stroke,fill}%
}%
\begin{pgfscope}%
\pgfsys@transformshift{0.718506in}{2.586458in}%
\pgfsys@useobject{currentmarker}{}%
\end{pgfscope}%
\begin{pgfscope}%
\pgfsys@transformshift{0.868143in}{2.553662in}%
\pgfsys@useobject{currentmarker}{}%
\end{pgfscope}%
\begin{pgfscope}%
\pgfsys@transformshift{1.017780in}{2.504994in}%
\pgfsys@useobject{currentmarker}{}%
\end{pgfscope}%
\begin{pgfscope}%
\pgfsys@transformshift{1.167417in}{2.462448in}%
\pgfsys@useobject{currentmarker}{}%
\end{pgfscope}%
\begin{pgfscope}%
\pgfsys@transformshift{1.317054in}{2.405560in}%
\pgfsys@useobject{currentmarker}{}%
\end{pgfscope}%
\begin{pgfscope}%
\pgfsys@transformshift{1.466691in}{2.329722in}%
\pgfsys@useobject{currentmarker}{}%
\end{pgfscope}%
\begin{pgfscope}%
\pgfsys@transformshift{1.616328in}{2.245619in}%
\pgfsys@useobject{currentmarker}{}%
\end{pgfscope}%
\begin{pgfscope}%
\pgfsys@transformshift{1.765965in}{2.152086in}%
\pgfsys@useobject{currentmarker}{}%
\end{pgfscope}%
\begin{pgfscope}%
\pgfsys@transformshift{1.915602in}{2.052328in}%
\pgfsys@useobject{currentmarker}{}%
\end{pgfscope}%
\begin{pgfscope}%
\pgfsys@transformshift{2.065239in}{1.851843in}%
\pgfsys@useobject{currentmarker}{}%
\end{pgfscope}%
\begin{pgfscope}%
\pgfsys@transformshift{2.214876in}{1.687739in}%
\pgfsys@useobject{currentmarker}{}%
\end{pgfscope}%
\begin{pgfscope}%
\pgfsys@transformshift{2.364513in}{1.522939in}%
\pgfsys@useobject{currentmarker}{}%
\end{pgfscope}%
\begin{pgfscope}%
\pgfsys@transformshift{2.514150in}{1.354390in}%
\pgfsys@useobject{currentmarker}{}%
\end{pgfscope}%
\begin{pgfscope}%
\pgfsys@transformshift{2.663787in}{1.151846in}%
\pgfsys@useobject{currentmarker}{}%
\end{pgfscope}%
\begin{pgfscope}%
\pgfsys@transformshift{2.813424in}{0.983019in}%
\pgfsys@useobject{currentmarker}{}%
\end{pgfscope}%
\begin{pgfscope}%
\pgfsys@transformshift{2.963061in}{0.862052in}%
\pgfsys@useobject{currentmarker}{}%
\end{pgfscope}%
\begin{pgfscope}%
\pgfsys@transformshift{3.112698in}{0.788982in}%
\pgfsys@useobject{currentmarker}{}%
\end{pgfscope}%
\begin{pgfscope}%
\pgfsys@transformshift{3.262335in}{0.679033in}%
\pgfsys@useobject{currentmarker}{}%
\end{pgfscope}%
\begin{pgfscope}%
\pgfsys@transformshift{3.411972in}{0.635143in}%
\pgfsys@useobject{currentmarker}{}%
\end{pgfscope}%
\begin{pgfscope}%
\pgfsys@transformshift{3.561609in}{0.542920in}%
\pgfsys@useobject{currentmarker}{}%
\end{pgfscope}%
\begin{pgfscope}%
\pgfsys@transformshift{3.711246in}{0.514429in}%
\pgfsys@useobject{currentmarker}{}%
\end{pgfscope}%
\end{pgfscope}%
\begin{pgfscope}%
\pgfsetbuttcap%
\pgfsetroundjoin%
\definecolor{currentfill}{rgb}{0.000000,0.000000,0.000000}%
\pgfsetfillcolor{currentfill}%
\pgfsetlinewidth{0.803000pt}%
\definecolor{currentstroke}{rgb}{0.000000,0.000000,0.000000}%
\pgfsetstrokecolor{currentstroke}%
\pgfsetdash{}{0pt}%
\pgfsys@defobject{currentmarker}{\pgfqpoint{0.000000in}{-0.048611in}}{\pgfqpoint{0.000000in}{0.000000in}}{%
\pgfpathmoveto{\pgfqpoint{0.000000in}{0.000000in}}%
\pgfpathlineto{\pgfqpoint{0.000000in}{-0.048611in}}%
\pgfusepath{stroke,fill}%
}%
\begin{pgfscope}%
\pgfsys@transformshift{1.017780in}{0.386111in}%
\pgfsys@useobject{currentmarker}{}%
\end{pgfscope}%
\end{pgfscope}%
\begin{pgfscope}%
\definecolor{textcolor}{rgb}{0.000000,0.000000,0.000000}%
\pgfsetstrokecolor{textcolor}%
\pgfsetfillcolor{textcolor}%
\pgftext[x=1.017780in,y=0.288889in,,top]{\color{textcolor}\sffamily\fontsize{10.000000}{12.000000}\selectfont 2.20}%
\end{pgfscope}%
\begin{pgfscope}%
\pgfsetbuttcap%
\pgfsetroundjoin%
\definecolor{currentfill}{rgb}{0.000000,0.000000,0.000000}%
\pgfsetfillcolor{currentfill}%
\pgfsetlinewidth{0.803000pt}%
\definecolor{currentstroke}{rgb}{0.000000,0.000000,0.000000}%
\pgfsetstrokecolor{currentstroke}%
\pgfsetdash{}{0pt}%
\pgfsys@defobject{currentmarker}{\pgfqpoint{0.000000in}{-0.048611in}}{\pgfqpoint{0.000000in}{0.000000in}}{%
\pgfpathmoveto{\pgfqpoint{0.000000in}{0.000000in}}%
\pgfpathlineto{\pgfqpoint{0.000000in}{-0.048611in}}%
\pgfusepath{stroke,fill}%
}%
\begin{pgfscope}%
\pgfsys@transformshift{1.765965in}{0.386111in}%
\pgfsys@useobject{currentmarker}{}%
\end{pgfscope}%
\end{pgfscope}%
\begin{pgfscope}%
\definecolor{textcolor}{rgb}{0.000000,0.000000,0.000000}%
\pgfsetstrokecolor{textcolor}%
\pgfsetfillcolor{textcolor}%
\pgftext[x=1.765965in,y=0.288889in,,top]{\color{textcolor}\sffamily\fontsize{10.000000}{12.000000}\selectfont 2.25}%
\end{pgfscope}%
\begin{pgfscope}%
\pgfsetbuttcap%
\pgfsetroundjoin%
\definecolor{currentfill}{rgb}{0.000000,0.000000,0.000000}%
\pgfsetfillcolor{currentfill}%
\pgfsetlinewidth{0.803000pt}%
\definecolor{currentstroke}{rgb}{0.000000,0.000000,0.000000}%
\pgfsetstrokecolor{currentstroke}%
\pgfsetdash{}{0pt}%
\pgfsys@defobject{currentmarker}{\pgfqpoint{0.000000in}{-0.048611in}}{\pgfqpoint{0.000000in}{0.000000in}}{%
\pgfpathmoveto{\pgfqpoint{0.000000in}{0.000000in}}%
\pgfpathlineto{\pgfqpoint{0.000000in}{-0.048611in}}%
\pgfusepath{stroke,fill}%
}%
\begin{pgfscope}%
\pgfsys@transformshift{2.514150in}{0.386111in}%
\pgfsys@useobject{currentmarker}{}%
\end{pgfscope}%
\end{pgfscope}%
\begin{pgfscope}%
\definecolor{textcolor}{rgb}{0.000000,0.000000,0.000000}%
\pgfsetstrokecolor{textcolor}%
\pgfsetfillcolor{textcolor}%
\pgftext[x=2.514150in,y=0.288889in,,top]{\color{textcolor}\sffamily\fontsize{10.000000}{12.000000}\selectfont 2.30}%
\end{pgfscope}%
\begin{pgfscope}%
\pgfsetbuttcap%
\pgfsetroundjoin%
\definecolor{currentfill}{rgb}{0.000000,0.000000,0.000000}%
\pgfsetfillcolor{currentfill}%
\pgfsetlinewidth{0.803000pt}%
\definecolor{currentstroke}{rgb}{0.000000,0.000000,0.000000}%
\pgfsetstrokecolor{currentstroke}%
\pgfsetdash{}{0pt}%
\pgfsys@defobject{currentmarker}{\pgfqpoint{0.000000in}{-0.048611in}}{\pgfqpoint{0.000000in}{0.000000in}}{%
\pgfpathmoveto{\pgfqpoint{0.000000in}{0.000000in}}%
\pgfpathlineto{\pgfqpoint{0.000000in}{-0.048611in}}%
\pgfusepath{stroke,fill}%
}%
\begin{pgfscope}%
\pgfsys@transformshift{3.262335in}{0.386111in}%
\pgfsys@useobject{currentmarker}{}%
\end{pgfscope}%
\end{pgfscope}%
\begin{pgfscope}%
\definecolor{textcolor}{rgb}{0.000000,0.000000,0.000000}%
\pgfsetstrokecolor{textcolor}%
\pgfsetfillcolor{textcolor}%
\pgftext[x=3.262335in,y=0.288889in,,top]{\color{textcolor}\sffamily\fontsize{10.000000}{12.000000}\selectfont 2.35}%
\end{pgfscope}%
\begin{pgfscope}%
\pgfsetbuttcap%
\pgfsetroundjoin%
\definecolor{currentfill}{rgb}{0.000000,0.000000,0.000000}%
\pgfsetfillcolor{currentfill}%
\pgfsetlinewidth{0.803000pt}%
\definecolor{currentstroke}{rgb}{0.000000,0.000000,0.000000}%
\pgfsetstrokecolor{currentstroke}%
\pgfsetdash{}{0pt}%
\pgfsys@defobject{currentmarker}{\pgfqpoint{-0.048611in}{0.000000in}}{\pgfqpoint{0.000000in}{0.000000in}}{%
\pgfpathmoveto{\pgfqpoint{0.000000in}{0.000000in}}%
\pgfpathlineto{\pgfqpoint{-0.048611in}{0.000000in}}%
\pgfusepath{stroke,fill}%
}%
\begin{pgfscope}%
\pgfsys@transformshift{0.556250in}{0.566898in}%
\pgfsys@useobject{currentmarker}{}%
\end{pgfscope}%
\end{pgfscope}%
\begin{pgfscope}%
\definecolor{textcolor}{rgb}{0.000000,0.000000,0.000000}%
\pgfsetstrokecolor{textcolor}%
\pgfsetfillcolor{textcolor}%
\pgftext[x=0.238148in,y=0.514136in,left,base]{\color{textcolor}\sffamily\fontsize{10.000000}{12.000000}\selectfont 0.2}%
\end{pgfscope}%
\begin{pgfscope}%
\pgfsetbuttcap%
\pgfsetroundjoin%
\definecolor{currentfill}{rgb}{0.000000,0.000000,0.000000}%
\pgfsetfillcolor{currentfill}%
\pgfsetlinewidth{0.803000pt}%
\definecolor{currentstroke}{rgb}{0.000000,0.000000,0.000000}%
\pgfsetstrokecolor{currentstroke}%
\pgfsetdash{}{0pt}%
\pgfsys@defobject{currentmarker}{\pgfqpoint{-0.048611in}{0.000000in}}{\pgfqpoint{0.000000in}{0.000000in}}{%
\pgfpathmoveto{\pgfqpoint{0.000000in}{0.000000in}}%
\pgfpathlineto{\pgfqpoint{-0.048611in}{0.000000in}}%
\pgfusepath{stroke,fill}%
}%
\begin{pgfscope}%
\pgfsys@transformshift{0.556250in}{0.899377in}%
\pgfsys@useobject{currentmarker}{}%
\end{pgfscope}%
\end{pgfscope}%
\begin{pgfscope}%
\definecolor{textcolor}{rgb}{0.000000,0.000000,0.000000}%
\pgfsetstrokecolor{textcolor}%
\pgfsetfillcolor{textcolor}%
\pgftext[x=0.238148in,y=0.846615in,left,base]{\color{textcolor}\sffamily\fontsize{10.000000}{12.000000}\selectfont 0.3}%
\end{pgfscope}%
\begin{pgfscope}%
\pgfsetbuttcap%
\pgfsetroundjoin%
\definecolor{currentfill}{rgb}{0.000000,0.000000,0.000000}%
\pgfsetfillcolor{currentfill}%
\pgfsetlinewidth{0.803000pt}%
\definecolor{currentstroke}{rgb}{0.000000,0.000000,0.000000}%
\pgfsetstrokecolor{currentstroke}%
\pgfsetdash{}{0pt}%
\pgfsys@defobject{currentmarker}{\pgfqpoint{-0.048611in}{0.000000in}}{\pgfqpoint{0.000000in}{0.000000in}}{%
\pgfpathmoveto{\pgfqpoint{0.000000in}{0.000000in}}%
\pgfpathlineto{\pgfqpoint{-0.048611in}{0.000000in}}%
\pgfusepath{stroke,fill}%
}%
\begin{pgfscope}%
\pgfsys@transformshift{0.556250in}{1.231855in}%
\pgfsys@useobject{currentmarker}{}%
\end{pgfscope}%
\end{pgfscope}%
\begin{pgfscope}%
\definecolor{textcolor}{rgb}{0.000000,0.000000,0.000000}%
\pgfsetstrokecolor{textcolor}%
\pgfsetfillcolor{textcolor}%
\pgftext[x=0.238148in,y=1.179094in,left,base]{\color{textcolor}\sffamily\fontsize{10.000000}{12.000000}\selectfont 0.4}%
\end{pgfscope}%
\begin{pgfscope}%
\pgfsetbuttcap%
\pgfsetroundjoin%
\definecolor{currentfill}{rgb}{0.000000,0.000000,0.000000}%
\pgfsetfillcolor{currentfill}%
\pgfsetlinewidth{0.803000pt}%
\definecolor{currentstroke}{rgb}{0.000000,0.000000,0.000000}%
\pgfsetstrokecolor{currentstroke}%
\pgfsetdash{}{0pt}%
\pgfsys@defobject{currentmarker}{\pgfqpoint{-0.048611in}{0.000000in}}{\pgfqpoint{0.000000in}{0.000000in}}{%
\pgfpathmoveto{\pgfqpoint{0.000000in}{0.000000in}}%
\pgfpathlineto{\pgfqpoint{-0.048611in}{0.000000in}}%
\pgfusepath{stroke,fill}%
}%
\begin{pgfscope}%
\pgfsys@transformshift{0.556250in}{1.564334in}%
\pgfsys@useobject{currentmarker}{}%
\end{pgfscope}%
\end{pgfscope}%
\begin{pgfscope}%
\definecolor{textcolor}{rgb}{0.000000,0.000000,0.000000}%
\pgfsetstrokecolor{textcolor}%
\pgfsetfillcolor{textcolor}%
\pgftext[x=0.238148in,y=1.511572in,left,base]{\color{textcolor}\sffamily\fontsize{10.000000}{12.000000}\selectfont 0.5}%
\end{pgfscope}%
\begin{pgfscope}%
\pgfsetbuttcap%
\pgfsetroundjoin%
\definecolor{currentfill}{rgb}{0.000000,0.000000,0.000000}%
\pgfsetfillcolor{currentfill}%
\pgfsetlinewidth{0.803000pt}%
\definecolor{currentstroke}{rgb}{0.000000,0.000000,0.000000}%
\pgfsetstrokecolor{currentstroke}%
\pgfsetdash{}{0pt}%
\pgfsys@defobject{currentmarker}{\pgfqpoint{-0.048611in}{0.000000in}}{\pgfqpoint{0.000000in}{0.000000in}}{%
\pgfpathmoveto{\pgfqpoint{0.000000in}{0.000000in}}%
\pgfpathlineto{\pgfqpoint{-0.048611in}{0.000000in}}%
\pgfusepath{stroke,fill}%
}%
\begin{pgfscope}%
\pgfsys@transformshift{0.556250in}{1.896812in}%
\pgfsys@useobject{currentmarker}{}%
\end{pgfscope}%
\end{pgfscope}%
\begin{pgfscope}%
\definecolor{textcolor}{rgb}{0.000000,0.000000,0.000000}%
\pgfsetstrokecolor{textcolor}%
\pgfsetfillcolor{textcolor}%
\pgftext[x=0.238148in,y=1.844051in,left,base]{\color{textcolor}\sffamily\fontsize{10.000000}{12.000000}\selectfont 0.6}%
\end{pgfscope}%
\begin{pgfscope}%
\pgfsetbuttcap%
\pgfsetroundjoin%
\definecolor{currentfill}{rgb}{0.000000,0.000000,0.000000}%
\pgfsetfillcolor{currentfill}%
\pgfsetlinewidth{0.803000pt}%
\definecolor{currentstroke}{rgb}{0.000000,0.000000,0.000000}%
\pgfsetstrokecolor{currentstroke}%
\pgfsetdash{}{0pt}%
\pgfsys@defobject{currentmarker}{\pgfqpoint{-0.048611in}{0.000000in}}{\pgfqpoint{0.000000in}{0.000000in}}{%
\pgfpathmoveto{\pgfqpoint{0.000000in}{0.000000in}}%
\pgfpathlineto{\pgfqpoint{-0.048611in}{0.000000in}}%
\pgfusepath{stroke,fill}%
}%
\begin{pgfscope}%
\pgfsys@transformshift{0.556250in}{2.229291in}%
\pgfsys@useobject{currentmarker}{}%
\end{pgfscope}%
\end{pgfscope}%
\begin{pgfscope}%
\definecolor{textcolor}{rgb}{0.000000,0.000000,0.000000}%
\pgfsetstrokecolor{textcolor}%
\pgfsetfillcolor{textcolor}%
\pgftext[x=0.238148in,y=2.176529in,left,base]{\color{textcolor}\sffamily\fontsize{10.000000}{12.000000}\selectfont 0.7}%
\end{pgfscope}%
\begin{pgfscope}%
\pgfsetbuttcap%
\pgfsetroundjoin%
\definecolor{currentfill}{rgb}{0.000000,0.000000,0.000000}%
\pgfsetfillcolor{currentfill}%
\pgfsetlinewidth{0.803000pt}%
\definecolor{currentstroke}{rgb}{0.000000,0.000000,0.000000}%
\pgfsetstrokecolor{currentstroke}%
\pgfsetdash{}{0pt}%
\pgfsys@defobject{currentmarker}{\pgfqpoint{-0.048611in}{0.000000in}}{\pgfqpoint{0.000000in}{0.000000in}}{%
\pgfpathmoveto{\pgfqpoint{0.000000in}{0.000000in}}%
\pgfpathlineto{\pgfqpoint{-0.048611in}{0.000000in}}%
\pgfusepath{stroke,fill}%
}%
\begin{pgfscope}%
\pgfsys@transformshift{0.556250in}{2.561769in}%
\pgfsys@useobject{currentmarker}{}%
\end{pgfscope}%
\end{pgfscope}%
\begin{pgfscope}%
\definecolor{textcolor}{rgb}{0.000000,0.000000,0.000000}%
\pgfsetstrokecolor{textcolor}%
\pgfsetfillcolor{textcolor}%
\pgftext[x=0.238148in,y=2.509008in,left,base]{\color{textcolor}\sffamily\fontsize{10.000000}{12.000000}\selectfont 0.8}%
\end{pgfscope}%
\begin{pgfscope}%
\pgfpathrectangle{\pgfqpoint{0.556250in}{0.386111in}}{\pgfqpoint{3.317253in}{2.316667in}}%
\pgfusepath{clip}%
\pgfsetrectcap%
\pgfsetroundjoin%
\pgfsetlinewidth{1.505625pt}%
\definecolor{currentstroke}{rgb}{0.121569,0.466667,0.705882}%
\pgfsetstrokecolor{currentstroke}%
\pgfsetdash{}{0pt}%
\pgfpathmoveto{\pgfqpoint{0.718506in}{2.586458in}}%
\pgfpathlineto{\pgfqpoint{0.868143in}{2.553662in}}%
\pgfpathlineto{\pgfqpoint{1.017780in}{2.504994in}}%
\pgfpathlineto{\pgfqpoint{1.167417in}{2.462448in}}%
\pgfpathlineto{\pgfqpoint{1.317054in}{2.405560in}}%
\pgfpathlineto{\pgfqpoint{1.466691in}{2.329722in}}%
\pgfpathlineto{\pgfqpoint{1.616328in}{2.245619in}}%
\pgfpathlineto{\pgfqpoint{1.765965in}{2.152086in}}%
\pgfpathlineto{\pgfqpoint{1.915602in}{2.052328in}}%
\pgfpathlineto{\pgfqpoint{2.065239in}{1.851843in}}%
\pgfpathlineto{\pgfqpoint{2.214876in}{1.687739in}}%
\pgfpathlineto{\pgfqpoint{2.364513in}{1.522939in}}%
\pgfpathlineto{\pgfqpoint{2.514150in}{1.354390in}}%
\pgfpathlineto{\pgfqpoint{2.663787in}{1.151846in}}%
\pgfpathlineto{\pgfqpoint{2.813424in}{0.983019in}}%
\pgfpathlineto{\pgfqpoint{2.963061in}{0.862052in}}%
\pgfpathlineto{\pgfqpoint{3.112698in}{0.788982in}}%
\pgfpathlineto{\pgfqpoint{3.262335in}{0.679033in}}%
\pgfpathlineto{\pgfqpoint{3.411972in}{0.635143in}}%
\pgfpathlineto{\pgfqpoint{3.561609in}{0.542920in}}%
\pgfpathlineto{\pgfqpoint{3.711246in}{0.514429in}}%
\pgfusepath{stroke}%
\end{pgfscope}%
\begin{pgfscope}%
\pgfsetrectcap%
\pgfsetmiterjoin%
\pgfsetlinewidth{0.803000pt}%
\definecolor{currentstroke}{rgb}{0.000000,0.000000,0.000000}%
\pgfsetstrokecolor{currentstroke}%
\pgfsetdash{}{0pt}%
\pgfpathmoveto{\pgfqpoint{0.556250in}{0.386111in}}%
\pgfpathlineto{\pgfqpoint{0.556250in}{2.702778in}}%
\pgfusepath{stroke}%
\end{pgfscope}%
\begin{pgfscope}%
\pgfsetrectcap%
\pgfsetmiterjoin%
\pgfsetlinewidth{0.803000pt}%
\definecolor{currentstroke}{rgb}{0.000000,0.000000,0.000000}%
\pgfsetstrokecolor{currentstroke}%
\pgfsetdash{}{0pt}%
\pgfpathmoveto{\pgfqpoint{3.873503in}{0.386111in}}%
\pgfpathlineto{\pgfqpoint{3.873503in}{2.702778in}}%
\pgfusepath{stroke}%
\end{pgfscope}%
\begin{pgfscope}%
\pgfsetrectcap%
\pgfsetmiterjoin%
\pgfsetlinewidth{0.803000pt}%
\definecolor{currentstroke}{rgb}{0.000000,0.000000,0.000000}%
\pgfsetstrokecolor{currentstroke}%
\pgfsetdash{}{0pt}%
\pgfpathmoveto{\pgfqpoint{0.556250in}{0.386111in}}%
\pgfpathlineto{\pgfqpoint{3.873503in}{0.386111in}}%
\pgfusepath{stroke}%
\end{pgfscope}%
\begin{pgfscope}%
\pgfsetrectcap%
\pgfsetmiterjoin%
\pgfsetlinewidth{0.803000pt}%
\definecolor{currentstroke}{rgb}{0.000000,0.000000,0.000000}%
\pgfsetstrokecolor{currentstroke}%
\pgfsetdash{}{0pt}%
\pgfpathmoveto{\pgfqpoint{0.556250in}{2.702778in}}%
\pgfpathlineto{\pgfqpoint{3.873503in}{2.702778in}}%
\pgfusepath{stroke}%
\end{pgfscope}%
\begin{pgfscope}%
\definecolor{textcolor}{rgb}{0.000000,0.000000,0.000000}%
\pgfsetstrokecolor{textcolor}%
\pgfsetfillcolor{textcolor}%
\pgftext[x=2.214876in,y=2.786111in,,base]{\color{textcolor}\sffamily\fontsize{12.000000}{14.400000}\selectfont \(\displaystyle  N = 64 \)}%
\end{pgfscope}%
\begin{pgfscope}%
\pgfsetbuttcap%
\pgfsetmiterjoin%
\definecolor{currentfill}{rgb}{1.000000,1.000000,1.000000}%
\pgfsetfillcolor{currentfill}%
\pgfsetlinewidth{0.000000pt}%
\definecolor{currentstroke}{rgb}{0.000000,0.000000,0.000000}%
\pgfsetstrokecolor{currentstroke}%
\pgfsetstrokeopacity{0.000000}%
\pgfsetdash{}{0pt}%
\pgfpathmoveto{\pgfqpoint{4.389822in}{0.386111in}}%
\pgfpathlineto{\pgfqpoint{7.707075in}{0.386111in}}%
\pgfpathlineto{\pgfqpoint{7.707075in}{2.702778in}}%
\pgfpathlineto{\pgfqpoint{4.389822in}{2.702778in}}%
\pgfpathclose%
\pgfusepath{fill}%
\end{pgfscope}%
\begin{pgfscope}%
\pgfpathrectangle{\pgfqpoint{4.389822in}{0.386111in}}{\pgfqpoint{3.317253in}{2.316667in}}%
\pgfusepath{clip}%
\pgfsetbuttcap%
\pgfsetroundjoin%
\definecolor{currentfill}{rgb}{0.121569,0.466667,0.705882}%
\pgfsetfillcolor{currentfill}%
\pgfsetfillopacity{0.300000}%
\pgfsetlinewidth{1.003750pt}%
\definecolor{currentstroke}{rgb}{0.121569,0.466667,0.705882}%
\pgfsetstrokecolor{currentstroke}%
\pgfsetstrokeopacity{0.300000}%
\pgfsetdash{}{0pt}%
\pgfpathmoveto{\pgfqpoint{4.552079in}{2.590618in}}%
\pgfpathlineto{\pgfqpoint{4.552079in}{2.582304in}}%
\pgfpathlineto{\pgfqpoint{4.701716in}{2.553263in}}%
\pgfpathlineto{\pgfqpoint{4.851353in}{2.518770in}}%
\pgfpathlineto{\pgfqpoint{5.000990in}{2.488342in}}%
\pgfpathlineto{\pgfqpoint{5.150627in}{2.441201in}}%
\pgfpathlineto{\pgfqpoint{5.300264in}{2.390306in}}%
\pgfpathlineto{\pgfqpoint{5.449901in}{2.300170in}}%
\pgfpathlineto{\pgfqpoint{5.599538in}{2.250232in}}%
\pgfpathlineto{\pgfqpoint{5.749175in}{2.153436in}}%
\pgfpathlineto{\pgfqpoint{5.898812in}{1.992442in}}%
\pgfpathlineto{\pgfqpoint{6.048449in}{1.637380in}}%
\pgfpathlineto{\pgfqpoint{6.198086in}{1.369932in}}%
\pgfpathlineto{\pgfqpoint{6.347723in}{1.302379in}}%
\pgfpathlineto{\pgfqpoint{6.497360in}{0.759500in}}%
\pgfpathlineto{\pgfqpoint{6.646997in}{0.875773in}}%
\pgfpathlineto{\pgfqpoint{6.796634in}{0.633727in}}%
\pgfpathlineto{\pgfqpoint{6.946271in}{0.682101in}}%
\pgfpathlineto{\pgfqpoint{7.095908in}{0.547004in}}%
\pgfpathlineto{\pgfqpoint{7.245545in}{0.525180in}}%
\pgfpathlineto{\pgfqpoint{7.395182in}{0.491414in}}%
\pgfpathlineto{\pgfqpoint{7.544819in}{0.538632in}}%
\pgfpathlineto{\pgfqpoint{7.544819in}{0.626643in}}%
\pgfpathlineto{\pgfqpoint{7.544819in}{0.626643in}}%
\pgfpathlineto{\pgfqpoint{7.395182in}{0.641019in}}%
\pgfpathlineto{\pgfqpoint{7.245545in}{0.688981in}}%
\pgfpathlineto{\pgfqpoint{7.095908in}{0.801942in}}%
\pgfpathlineto{\pgfqpoint{6.946271in}{0.747420in}}%
\pgfpathlineto{\pgfqpoint{6.796634in}{0.960790in}}%
\pgfpathlineto{\pgfqpoint{6.646997in}{1.079149in}}%
\pgfpathlineto{\pgfqpoint{6.497360in}{1.120722in}}%
\pgfpathlineto{\pgfqpoint{6.347723in}{1.455885in}}%
\pgfpathlineto{\pgfqpoint{6.198086in}{1.837746in}}%
\pgfpathlineto{\pgfqpoint{6.048449in}{2.028982in}}%
\pgfpathlineto{\pgfqpoint{5.898812in}{2.121900in}}%
\pgfpathlineto{\pgfqpoint{5.749175in}{2.256968in}}%
\pgfpathlineto{\pgfqpoint{5.599538in}{2.290253in}}%
\pgfpathlineto{\pgfqpoint{5.449901in}{2.352080in}}%
\pgfpathlineto{\pgfqpoint{5.300264in}{2.420994in}}%
\pgfpathlineto{\pgfqpoint{5.150627in}{2.460950in}}%
\pgfpathlineto{\pgfqpoint{5.000990in}{2.499675in}}%
\pgfpathlineto{\pgfqpoint{4.851353in}{2.530816in}}%
\pgfpathlineto{\pgfqpoint{4.701716in}{2.560284in}}%
\pgfpathlineto{\pgfqpoint{4.552079in}{2.590618in}}%
\pgfpathclose%
\pgfusepath{stroke,fill}%
\end{pgfscope}%
\begin{pgfscope}%
\pgfpathrectangle{\pgfqpoint{4.389822in}{0.386111in}}{\pgfqpoint{3.317253in}{2.316667in}}%
\pgfusepath{clip}%
\pgfsetbuttcap%
\pgfsetroundjoin%
\definecolor{currentfill}{rgb}{0.121569,0.466667,0.705882}%
\pgfsetfillcolor{currentfill}%
\pgfsetlinewidth{1.003750pt}%
\definecolor{currentstroke}{rgb}{0.121569,0.466667,0.705882}%
\pgfsetstrokecolor{currentstroke}%
\pgfsetdash{}{0pt}%
\pgfsys@defobject{currentmarker}{\pgfqpoint{-0.009821in}{-0.009821in}}{\pgfqpoint{0.009821in}{0.009821in}}{%
\pgfpathmoveto{\pgfqpoint{0.000000in}{-0.009821in}}%
\pgfpathcurveto{\pgfqpoint{0.002605in}{-0.009821in}}{\pgfqpoint{0.005103in}{-0.008786in}}{\pgfqpoint{0.006944in}{-0.006944in}}%
\pgfpathcurveto{\pgfqpoint{0.008786in}{-0.005103in}}{\pgfqpoint{0.009821in}{-0.002605in}}{\pgfqpoint{0.009821in}{0.000000in}}%
\pgfpathcurveto{\pgfqpoint{0.009821in}{0.002605in}}{\pgfqpoint{0.008786in}{0.005103in}}{\pgfqpoint{0.006944in}{0.006944in}}%
\pgfpathcurveto{\pgfqpoint{0.005103in}{0.008786in}}{\pgfqpoint{0.002605in}{0.009821in}}{\pgfqpoint{0.000000in}{0.009821in}}%
\pgfpathcurveto{\pgfqpoint{-0.002605in}{0.009821in}}{\pgfqpoint{-0.005103in}{0.008786in}}{\pgfqpoint{-0.006944in}{0.006944in}}%
\pgfpathcurveto{\pgfqpoint{-0.008786in}{0.005103in}}{\pgfqpoint{-0.009821in}{0.002605in}}{\pgfqpoint{-0.009821in}{0.000000in}}%
\pgfpathcurveto{\pgfqpoint{-0.009821in}{-0.002605in}}{\pgfqpoint{-0.008786in}{-0.005103in}}{\pgfqpoint{-0.006944in}{-0.006944in}}%
\pgfpathcurveto{\pgfqpoint{-0.005103in}{-0.008786in}}{\pgfqpoint{-0.002605in}{-0.009821in}}{\pgfqpoint{0.000000in}{-0.009821in}}%
\pgfpathclose%
\pgfusepath{stroke,fill}%
}%
\begin{pgfscope}%
\pgfsys@transformshift{4.552079in}{2.586461in}%
\pgfsys@useobject{currentmarker}{}%
\end{pgfscope}%
\begin{pgfscope}%
\pgfsys@transformshift{4.701716in}{2.556773in}%
\pgfsys@useobject{currentmarker}{}%
\end{pgfscope}%
\begin{pgfscope}%
\pgfsys@transformshift{4.851353in}{2.524793in}%
\pgfsys@useobject{currentmarker}{}%
\end{pgfscope}%
\begin{pgfscope}%
\pgfsys@transformshift{5.000990in}{2.494009in}%
\pgfsys@useobject{currentmarker}{}%
\end{pgfscope}%
\begin{pgfscope}%
\pgfsys@transformshift{5.150627in}{2.451075in}%
\pgfsys@useobject{currentmarker}{}%
\end{pgfscope}%
\begin{pgfscope}%
\pgfsys@transformshift{5.300264in}{2.405650in}%
\pgfsys@useobject{currentmarker}{}%
\end{pgfscope}%
\begin{pgfscope}%
\pgfsys@transformshift{5.449901in}{2.326125in}%
\pgfsys@useobject{currentmarker}{}%
\end{pgfscope}%
\begin{pgfscope}%
\pgfsys@transformshift{5.599538in}{2.270243in}%
\pgfsys@useobject{currentmarker}{}%
\end{pgfscope}%
\begin{pgfscope}%
\pgfsys@transformshift{5.749175in}{2.205202in}%
\pgfsys@useobject{currentmarker}{}%
\end{pgfscope}%
\begin{pgfscope}%
\pgfsys@transformshift{5.898812in}{2.057171in}%
\pgfsys@useobject{currentmarker}{}%
\end{pgfscope}%
\begin{pgfscope}%
\pgfsys@transformshift{6.048449in}{1.833181in}%
\pgfsys@useobject{currentmarker}{}%
\end{pgfscope}%
\begin{pgfscope}%
\pgfsys@transformshift{6.198086in}{1.603839in}%
\pgfsys@useobject{currentmarker}{}%
\end{pgfscope}%
\begin{pgfscope}%
\pgfsys@transformshift{6.347723in}{1.379132in}%
\pgfsys@useobject{currentmarker}{}%
\end{pgfscope}%
\begin{pgfscope}%
\pgfsys@transformshift{6.497360in}{0.940111in}%
\pgfsys@useobject{currentmarker}{}%
\end{pgfscope}%
\begin{pgfscope}%
\pgfsys@transformshift{6.646997in}{0.977461in}%
\pgfsys@useobject{currentmarker}{}%
\end{pgfscope}%
\begin{pgfscope}%
\pgfsys@transformshift{6.796634in}{0.797259in}%
\pgfsys@useobject{currentmarker}{}%
\end{pgfscope}%
\begin{pgfscope}%
\pgfsys@transformshift{6.946271in}{0.714760in}%
\pgfsys@useobject{currentmarker}{}%
\end{pgfscope}%
\begin{pgfscope}%
\pgfsys@transformshift{7.095908in}{0.674473in}%
\pgfsys@useobject{currentmarker}{}%
\end{pgfscope}%
\begin{pgfscope}%
\pgfsys@transformshift{7.245545in}{0.607080in}%
\pgfsys@useobject{currentmarker}{}%
\end{pgfscope}%
\begin{pgfscope}%
\pgfsys@transformshift{7.395182in}{0.566216in}%
\pgfsys@useobject{currentmarker}{}%
\end{pgfscope}%
\begin{pgfscope}%
\pgfsys@transformshift{7.544819in}{0.582637in}%
\pgfsys@useobject{currentmarker}{}%
\end{pgfscope}%
\end{pgfscope}%
\begin{pgfscope}%
\pgfsetbuttcap%
\pgfsetroundjoin%
\definecolor{currentfill}{rgb}{0.000000,0.000000,0.000000}%
\pgfsetfillcolor{currentfill}%
\pgfsetlinewidth{0.803000pt}%
\definecolor{currentstroke}{rgb}{0.000000,0.000000,0.000000}%
\pgfsetstrokecolor{currentstroke}%
\pgfsetdash{}{0pt}%
\pgfsys@defobject{currentmarker}{\pgfqpoint{0.000000in}{-0.048611in}}{\pgfqpoint{0.000000in}{0.000000in}}{%
\pgfpathmoveto{\pgfqpoint{0.000000in}{0.000000in}}%
\pgfpathlineto{\pgfqpoint{0.000000in}{-0.048611in}}%
\pgfusepath{stroke,fill}%
}%
\begin{pgfscope}%
\pgfsys@transformshift{5.000990in}{0.386111in}%
\pgfsys@useobject{currentmarker}{}%
\end{pgfscope}%
\end{pgfscope}%
\begin{pgfscope}%
\definecolor{textcolor}{rgb}{0.000000,0.000000,0.000000}%
\pgfsetstrokecolor{textcolor}%
\pgfsetfillcolor{textcolor}%
\pgftext[x=5.000990in,y=0.288889in,,top]{\color{textcolor}\sffamily\fontsize{10.000000}{12.000000}\selectfont 2.20}%
\end{pgfscope}%
\begin{pgfscope}%
\pgfsetbuttcap%
\pgfsetroundjoin%
\definecolor{currentfill}{rgb}{0.000000,0.000000,0.000000}%
\pgfsetfillcolor{currentfill}%
\pgfsetlinewidth{0.803000pt}%
\definecolor{currentstroke}{rgb}{0.000000,0.000000,0.000000}%
\pgfsetstrokecolor{currentstroke}%
\pgfsetdash{}{0pt}%
\pgfsys@defobject{currentmarker}{\pgfqpoint{0.000000in}{-0.048611in}}{\pgfqpoint{0.000000in}{0.000000in}}{%
\pgfpathmoveto{\pgfqpoint{0.000000in}{0.000000in}}%
\pgfpathlineto{\pgfqpoint{0.000000in}{-0.048611in}}%
\pgfusepath{stroke,fill}%
}%
\begin{pgfscope}%
\pgfsys@transformshift{5.749175in}{0.386111in}%
\pgfsys@useobject{currentmarker}{}%
\end{pgfscope}%
\end{pgfscope}%
\begin{pgfscope}%
\definecolor{textcolor}{rgb}{0.000000,0.000000,0.000000}%
\pgfsetstrokecolor{textcolor}%
\pgfsetfillcolor{textcolor}%
\pgftext[x=5.749175in,y=0.288889in,,top]{\color{textcolor}\sffamily\fontsize{10.000000}{12.000000}\selectfont 2.25}%
\end{pgfscope}%
\begin{pgfscope}%
\pgfsetbuttcap%
\pgfsetroundjoin%
\definecolor{currentfill}{rgb}{0.000000,0.000000,0.000000}%
\pgfsetfillcolor{currentfill}%
\pgfsetlinewidth{0.803000pt}%
\definecolor{currentstroke}{rgb}{0.000000,0.000000,0.000000}%
\pgfsetstrokecolor{currentstroke}%
\pgfsetdash{}{0pt}%
\pgfsys@defobject{currentmarker}{\pgfqpoint{0.000000in}{-0.048611in}}{\pgfqpoint{0.000000in}{0.000000in}}{%
\pgfpathmoveto{\pgfqpoint{0.000000in}{0.000000in}}%
\pgfpathlineto{\pgfqpoint{0.000000in}{-0.048611in}}%
\pgfusepath{stroke,fill}%
}%
\begin{pgfscope}%
\pgfsys@transformshift{6.497360in}{0.386111in}%
\pgfsys@useobject{currentmarker}{}%
\end{pgfscope}%
\end{pgfscope}%
\begin{pgfscope}%
\definecolor{textcolor}{rgb}{0.000000,0.000000,0.000000}%
\pgfsetstrokecolor{textcolor}%
\pgfsetfillcolor{textcolor}%
\pgftext[x=6.497360in,y=0.288889in,,top]{\color{textcolor}\sffamily\fontsize{10.000000}{12.000000}\selectfont 2.30}%
\end{pgfscope}%
\begin{pgfscope}%
\pgfsetbuttcap%
\pgfsetroundjoin%
\definecolor{currentfill}{rgb}{0.000000,0.000000,0.000000}%
\pgfsetfillcolor{currentfill}%
\pgfsetlinewidth{0.803000pt}%
\definecolor{currentstroke}{rgb}{0.000000,0.000000,0.000000}%
\pgfsetstrokecolor{currentstroke}%
\pgfsetdash{}{0pt}%
\pgfsys@defobject{currentmarker}{\pgfqpoint{0.000000in}{-0.048611in}}{\pgfqpoint{0.000000in}{0.000000in}}{%
\pgfpathmoveto{\pgfqpoint{0.000000in}{0.000000in}}%
\pgfpathlineto{\pgfqpoint{0.000000in}{-0.048611in}}%
\pgfusepath{stroke,fill}%
}%
\begin{pgfscope}%
\pgfsys@transformshift{7.245545in}{0.386111in}%
\pgfsys@useobject{currentmarker}{}%
\end{pgfscope}%
\end{pgfscope}%
\begin{pgfscope}%
\definecolor{textcolor}{rgb}{0.000000,0.000000,0.000000}%
\pgfsetstrokecolor{textcolor}%
\pgfsetfillcolor{textcolor}%
\pgftext[x=7.245545in,y=0.288889in,,top]{\color{textcolor}\sffamily\fontsize{10.000000}{12.000000}\selectfont 2.35}%
\end{pgfscope}%
\begin{pgfscope}%
\pgfsetbuttcap%
\pgfsetroundjoin%
\definecolor{currentfill}{rgb}{0.000000,0.000000,0.000000}%
\pgfsetfillcolor{currentfill}%
\pgfsetlinewidth{0.803000pt}%
\definecolor{currentstroke}{rgb}{0.000000,0.000000,0.000000}%
\pgfsetstrokecolor{currentstroke}%
\pgfsetdash{}{0pt}%
\pgfsys@defobject{currentmarker}{\pgfqpoint{-0.048611in}{0.000000in}}{\pgfqpoint{0.000000in}{0.000000in}}{%
\pgfpathmoveto{\pgfqpoint{0.000000in}{0.000000in}}%
\pgfpathlineto{\pgfqpoint{-0.048611in}{0.000000in}}%
\pgfusepath{stroke,fill}%
}%
\begin{pgfscope}%
\pgfsys@transformshift{4.389822in}{0.852519in}%
\pgfsys@useobject{currentmarker}{}%
\end{pgfscope}%
\end{pgfscope}%
\begin{pgfscope}%
\definecolor{textcolor}{rgb}{0.000000,0.000000,0.000000}%
\pgfsetstrokecolor{textcolor}%
\pgfsetfillcolor{textcolor}%
\pgftext[x=4.071721in,y=0.799758in,left,base]{\color{textcolor}\sffamily\fontsize{10.000000}{12.000000}\selectfont 0.2}%
\end{pgfscope}%
\begin{pgfscope}%
\pgfsetbuttcap%
\pgfsetroundjoin%
\definecolor{currentfill}{rgb}{0.000000,0.000000,0.000000}%
\pgfsetfillcolor{currentfill}%
\pgfsetlinewidth{0.803000pt}%
\definecolor{currentstroke}{rgb}{0.000000,0.000000,0.000000}%
\pgfsetstrokecolor{currentstroke}%
\pgfsetdash{}{0pt}%
\pgfsys@defobject{currentmarker}{\pgfqpoint{-0.048611in}{0.000000in}}{\pgfqpoint{0.000000in}{0.000000in}}{%
\pgfpathmoveto{\pgfqpoint{0.000000in}{0.000000in}}%
\pgfpathlineto{\pgfqpoint{-0.048611in}{0.000000in}}%
\pgfusepath{stroke,fill}%
}%
\begin{pgfscope}%
\pgfsys@transformshift{4.389822in}{1.412738in}%
\pgfsys@useobject{currentmarker}{}%
\end{pgfscope}%
\end{pgfscope}%
\begin{pgfscope}%
\definecolor{textcolor}{rgb}{0.000000,0.000000,0.000000}%
\pgfsetstrokecolor{textcolor}%
\pgfsetfillcolor{textcolor}%
\pgftext[x=4.071721in,y=1.359977in,left,base]{\color{textcolor}\sffamily\fontsize{10.000000}{12.000000}\selectfont 0.4}%
\end{pgfscope}%
\begin{pgfscope}%
\pgfsetbuttcap%
\pgfsetroundjoin%
\definecolor{currentfill}{rgb}{0.000000,0.000000,0.000000}%
\pgfsetfillcolor{currentfill}%
\pgfsetlinewidth{0.803000pt}%
\definecolor{currentstroke}{rgb}{0.000000,0.000000,0.000000}%
\pgfsetstrokecolor{currentstroke}%
\pgfsetdash{}{0pt}%
\pgfsys@defobject{currentmarker}{\pgfqpoint{-0.048611in}{0.000000in}}{\pgfqpoint{0.000000in}{0.000000in}}{%
\pgfpathmoveto{\pgfqpoint{0.000000in}{0.000000in}}%
\pgfpathlineto{\pgfqpoint{-0.048611in}{0.000000in}}%
\pgfusepath{stroke,fill}%
}%
\begin{pgfscope}%
\pgfsys@transformshift{4.389822in}{1.972957in}%
\pgfsys@useobject{currentmarker}{}%
\end{pgfscope}%
\end{pgfscope}%
\begin{pgfscope}%
\definecolor{textcolor}{rgb}{0.000000,0.000000,0.000000}%
\pgfsetstrokecolor{textcolor}%
\pgfsetfillcolor{textcolor}%
\pgftext[x=4.071721in,y=1.920196in,left,base]{\color{textcolor}\sffamily\fontsize{10.000000}{12.000000}\selectfont 0.6}%
\end{pgfscope}%
\begin{pgfscope}%
\pgfsetbuttcap%
\pgfsetroundjoin%
\definecolor{currentfill}{rgb}{0.000000,0.000000,0.000000}%
\pgfsetfillcolor{currentfill}%
\pgfsetlinewidth{0.803000pt}%
\definecolor{currentstroke}{rgb}{0.000000,0.000000,0.000000}%
\pgfsetstrokecolor{currentstroke}%
\pgfsetdash{}{0pt}%
\pgfsys@defobject{currentmarker}{\pgfqpoint{-0.048611in}{0.000000in}}{\pgfqpoint{0.000000in}{0.000000in}}{%
\pgfpathmoveto{\pgfqpoint{0.000000in}{0.000000in}}%
\pgfpathlineto{\pgfqpoint{-0.048611in}{0.000000in}}%
\pgfusepath{stroke,fill}%
}%
\begin{pgfscope}%
\pgfsys@transformshift{4.389822in}{2.533176in}%
\pgfsys@useobject{currentmarker}{}%
\end{pgfscope}%
\end{pgfscope}%
\begin{pgfscope}%
\definecolor{textcolor}{rgb}{0.000000,0.000000,0.000000}%
\pgfsetstrokecolor{textcolor}%
\pgfsetfillcolor{textcolor}%
\pgftext[x=4.071721in,y=2.480414in,left,base]{\color{textcolor}\sffamily\fontsize{10.000000}{12.000000}\selectfont 0.8}%
\end{pgfscope}%
\begin{pgfscope}%
\pgfpathrectangle{\pgfqpoint{4.389822in}{0.386111in}}{\pgfqpoint{3.317253in}{2.316667in}}%
\pgfusepath{clip}%
\pgfsetrectcap%
\pgfsetroundjoin%
\pgfsetlinewidth{1.505625pt}%
\definecolor{currentstroke}{rgb}{0.121569,0.466667,0.705882}%
\pgfsetstrokecolor{currentstroke}%
\pgfsetdash{}{0pt}%
\pgfpathmoveto{\pgfqpoint{4.552079in}{2.586461in}}%
\pgfpathlineto{\pgfqpoint{4.701716in}{2.556773in}}%
\pgfpathlineto{\pgfqpoint{4.851353in}{2.524793in}}%
\pgfpathlineto{\pgfqpoint{5.000990in}{2.494009in}}%
\pgfpathlineto{\pgfqpoint{5.150627in}{2.451075in}}%
\pgfpathlineto{\pgfqpoint{5.300264in}{2.405650in}}%
\pgfpathlineto{\pgfqpoint{5.449901in}{2.326125in}}%
\pgfpathlineto{\pgfqpoint{5.599538in}{2.270243in}}%
\pgfpathlineto{\pgfqpoint{5.749175in}{2.205202in}}%
\pgfpathlineto{\pgfqpoint{5.898812in}{2.057171in}}%
\pgfpathlineto{\pgfqpoint{6.048449in}{1.833181in}}%
\pgfpathlineto{\pgfqpoint{6.198086in}{1.603839in}}%
\pgfpathlineto{\pgfqpoint{6.347723in}{1.379132in}}%
\pgfpathlineto{\pgfqpoint{6.497360in}{0.940111in}}%
\pgfpathlineto{\pgfqpoint{6.646997in}{0.977461in}}%
\pgfpathlineto{\pgfqpoint{6.796634in}{0.797259in}}%
\pgfpathlineto{\pgfqpoint{6.946271in}{0.714760in}}%
\pgfpathlineto{\pgfqpoint{7.095908in}{0.674473in}}%
\pgfpathlineto{\pgfqpoint{7.245545in}{0.607080in}}%
\pgfpathlineto{\pgfqpoint{7.395182in}{0.566216in}}%
\pgfpathlineto{\pgfqpoint{7.544819in}{0.582637in}}%
\pgfusepath{stroke}%
\end{pgfscope}%
\begin{pgfscope}%
\pgfsetrectcap%
\pgfsetmiterjoin%
\pgfsetlinewidth{0.803000pt}%
\definecolor{currentstroke}{rgb}{0.000000,0.000000,0.000000}%
\pgfsetstrokecolor{currentstroke}%
\pgfsetdash{}{0pt}%
\pgfpathmoveto{\pgfqpoint{4.389822in}{0.386111in}}%
\pgfpathlineto{\pgfqpoint{4.389822in}{2.702778in}}%
\pgfusepath{stroke}%
\end{pgfscope}%
\begin{pgfscope}%
\pgfsetrectcap%
\pgfsetmiterjoin%
\pgfsetlinewidth{0.803000pt}%
\definecolor{currentstroke}{rgb}{0.000000,0.000000,0.000000}%
\pgfsetstrokecolor{currentstroke}%
\pgfsetdash{}{0pt}%
\pgfpathmoveto{\pgfqpoint{7.707075in}{0.386111in}}%
\pgfpathlineto{\pgfqpoint{7.707075in}{2.702778in}}%
\pgfusepath{stroke}%
\end{pgfscope}%
\begin{pgfscope}%
\pgfsetrectcap%
\pgfsetmiterjoin%
\pgfsetlinewidth{0.803000pt}%
\definecolor{currentstroke}{rgb}{0.000000,0.000000,0.000000}%
\pgfsetstrokecolor{currentstroke}%
\pgfsetdash{}{0pt}%
\pgfpathmoveto{\pgfqpoint{4.389822in}{0.386111in}}%
\pgfpathlineto{\pgfqpoint{7.707075in}{0.386111in}}%
\pgfusepath{stroke}%
\end{pgfscope}%
\begin{pgfscope}%
\pgfsetrectcap%
\pgfsetmiterjoin%
\pgfsetlinewidth{0.803000pt}%
\definecolor{currentstroke}{rgb}{0.000000,0.000000,0.000000}%
\pgfsetstrokecolor{currentstroke}%
\pgfsetdash{}{0pt}%
\pgfpathmoveto{\pgfqpoint{4.389822in}{2.702778in}}%
\pgfpathlineto{\pgfqpoint{7.707075in}{2.702778in}}%
\pgfusepath{stroke}%
\end{pgfscope}%
\begin{pgfscope}%
\definecolor{textcolor}{rgb}{0.000000,0.000000,0.000000}%
\pgfsetstrokecolor{textcolor}%
\pgfsetfillcolor{textcolor}%
\pgftext[x=6.048449in,y=2.786111in,,base]{\color{textcolor}\sffamily\fontsize{12.000000}{14.400000}\selectfont \(\displaystyle  N = 128 \)}%
\end{pgfscope}%
\end{pgfpicture}%
\makeatother%
\endgroup%
}
\caption{Magnetization $m$ for finer temperature $T$ using kinetic Monte Carlo algorithm}
\label{Fig:MagBigKMC}
\end{figure}

In this case the curve is much smoother than the Metropolis--Hastings algorithm. After zooming in, the curve of internal energy $u$ just growth like a straight line, but the heat capacity reaches its maximum in the interval. From the figures, we conclude that the critical temperature is at about 2.27. To be precise, the critical temperature of $ N = 16, 32, 64, 128 $ is about 2.31, 2.29, 2.28, 2.27 respectively. We find the analytical critical temperature is
\begin{equation}
T_{\text{c}} = \frac{2}{ \ln \rbr{ 1 + \sqrt{2} } } \approx 2.268185
\end{equation}
and this is close to the results yielded by numerical simulation.

\subsection{3-D case using kinetic Monte Carlo algorithm}

We now turn to consider the 3-D case. We use the identical number of trajectories and iterations. Kinetic Monte Carlo is deployed since the convergence for 3-D case is much harder to achieve. The course resolution figure is shown in Figure \ref{Fig:HeatSmall3D}, \ref{Fig:CapSmall3D} and \ref{Fig:MagSmall3D}. The zoomed figure is shown in Figure \ref{Fig:HeatBig3D}, \ref{Fig:CapBig3D} and \ref{Fig:MagBig3D}. Note that the zooming in windows has shifted for different $T$.

\begin{figure}[htbp]
\centering
\scalebox{0.666}{%% Creator: Matplotlib, PGF backend
%%
%% To include the figure in your LaTeX document, write
%%   \input{<filename>.pgf}
%%
%% Make sure the required packages are loaded in your preamble
%%   \usepackage{pgf}
%%
%% Figures using additional raster images can only be included by \input if
%% they are in the same directory as the main LaTeX file. For loading figures
%% from other directories you can use the `import` package
%%   \usepackage{import}
%% and then include the figures with
%%   \import{<path to file>}{<filename>.pgf}
%%
%% Matplotlib used the following preamble
%%   \usepackage{fontspec}
%%   \setmainfont{DejaVuSerif.ttf}[Path=/home/lzh/anaconda3/envs/numana/lib/python3.7/site-packages/matplotlib/mpl-data/fonts/ttf/]
%%   \setsansfont{DejaVuSans.ttf}[Path=/home/lzh/anaconda3/envs/numana/lib/python3.7/site-packages/matplotlib/mpl-data/fonts/ttf/]
%%   \setmonofont{DejaVuSansMono.ttf}[Path=/home/lzh/anaconda3/envs/numana/lib/python3.7/site-packages/matplotlib/mpl-data/fonts/ttf/]
%%
\begingroup%
\makeatletter%
\begin{pgfpicture}%
\pgfpathrectangle{\pgfpointorigin}{\pgfqpoint{8.000000in}{4.000000in}}%
\pgfusepath{use as bounding box, clip}%
\begin{pgfscope}%
\pgfsetbuttcap%
\pgfsetmiterjoin%
\definecolor{currentfill}{rgb}{1.000000,1.000000,1.000000}%
\pgfsetfillcolor{currentfill}%
\pgfsetlinewidth{0.000000pt}%
\definecolor{currentstroke}{rgb}{1.000000,1.000000,1.000000}%
\pgfsetstrokecolor{currentstroke}%
\pgfsetdash{}{0pt}%
\pgfpathmoveto{\pgfqpoint{0.000000in}{0.000000in}}%
\pgfpathlineto{\pgfqpoint{8.000000in}{0.000000in}}%
\pgfpathlineto{\pgfqpoint{8.000000in}{4.000000in}}%
\pgfpathlineto{\pgfqpoint{0.000000in}{4.000000in}}%
\pgfpathclose%
\pgfusepath{fill}%
\end{pgfscope}%
\begin{pgfscope}%
\pgfsetbuttcap%
\pgfsetmiterjoin%
\definecolor{currentfill}{rgb}{1.000000,1.000000,1.000000}%
\pgfsetfillcolor{currentfill}%
\pgfsetlinewidth{0.000000pt}%
\definecolor{currentstroke}{rgb}{0.000000,0.000000,0.000000}%
\pgfsetstrokecolor{currentstroke}%
\pgfsetstrokeopacity{0.000000}%
\pgfsetdash{}{0pt}%
\pgfpathmoveto{\pgfqpoint{1.000000in}{0.440000in}}%
\pgfpathlineto{\pgfqpoint{3.818182in}{0.440000in}}%
\pgfpathlineto{\pgfqpoint{3.818182in}{3.520000in}}%
\pgfpathlineto{\pgfqpoint{1.000000in}{3.520000in}}%
\pgfpathclose%
\pgfusepath{fill}%
\end{pgfscope}%
\begin{pgfscope}%
\pgfsetbuttcap%
\pgfsetmiterjoin%
\definecolor{currentfill}{rgb}{0.950000,0.950000,0.950000}%
\pgfsetfillcolor{currentfill}%
\pgfsetfillopacity{0.500000}%
\pgfsetlinewidth{1.003750pt}%
\definecolor{currentstroke}{rgb}{0.950000,0.950000,0.950000}%
\pgfsetstrokecolor{currentstroke}%
\pgfsetstrokeopacity{0.500000}%
\pgfsetdash{}{0pt}%
\pgfpathmoveto{\pgfqpoint{2.714177in}{1.884582in}}%
\pgfpathlineto{\pgfqpoint{3.520755in}{1.193628in}}%
\pgfpathlineto{\pgfqpoint{3.579156in}{2.584074in}}%
\pgfpathlineto{\pgfqpoint{2.727606in}{3.204883in}}%
\pgfusepath{stroke,fill}%
\end{pgfscope}%
\begin{pgfscope}%
\pgfsetbuttcap%
\pgfsetmiterjoin%
\definecolor{currentfill}{rgb}{0.900000,0.900000,0.900000}%
\pgfsetfillcolor{currentfill}%
\pgfsetfillopacity{0.500000}%
\pgfsetlinewidth{1.003750pt}%
\definecolor{currentstroke}{rgb}{0.900000,0.900000,0.900000}%
\pgfsetstrokecolor{currentstroke}%
\pgfsetstrokeopacity{0.500000}%
\pgfsetdash{}{0pt}%
\pgfpathmoveto{\pgfqpoint{2.714177in}{1.884582in}}%
\pgfpathlineto{\pgfqpoint{1.407596in}{1.498294in}}%
\pgfpathlineto{\pgfqpoint{1.352930in}{2.858408in}}%
\pgfpathlineto{\pgfqpoint{2.727606in}{3.204883in}}%
\pgfusepath{stroke,fill}%
\end{pgfscope}%
\begin{pgfscope}%
\pgfsetbuttcap%
\pgfsetmiterjoin%
\definecolor{currentfill}{rgb}{0.925000,0.925000,0.925000}%
\pgfsetfillcolor{currentfill}%
\pgfsetfillopacity{0.500000}%
\pgfsetlinewidth{1.003750pt}%
\definecolor{currentstroke}{rgb}{0.925000,0.925000,0.925000}%
\pgfsetstrokecolor{currentstroke}%
\pgfsetstrokeopacity{0.500000}%
\pgfsetdash{}{0pt}%
\pgfpathmoveto{\pgfqpoint{2.714177in}{1.884582in}}%
\pgfpathlineto{\pgfqpoint{1.407596in}{1.498294in}}%
\pgfpathlineto{\pgfqpoint{2.146055in}{0.743733in}}%
\pgfpathlineto{\pgfqpoint{3.520755in}{1.193628in}}%
\pgfusepath{stroke,fill}%
\end{pgfscope}%
\begin{pgfscope}%
\pgfsetrectcap%
\pgfsetroundjoin%
\pgfsetlinewidth{0.803000pt}%
\definecolor{currentstroke}{rgb}{0.000000,0.000000,0.000000}%
\pgfsetstrokecolor{currentstroke}%
\pgfsetdash{}{0pt}%
\pgfpathmoveto{\pgfqpoint{3.520755in}{1.193628in}}%
\pgfpathlineto{\pgfqpoint{2.146055in}{0.743733in}}%
\pgfusepath{stroke}%
\end{pgfscope}%
\begin{pgfscope}%
\pgfsetbuttcap%
\pgfsetroundjoin%
\pgfsetlinewidth{0.803000pt}%
\definecolor{currentstroke}{rgb}{0.690196,0.690196,0.690196}%
\pgfsetstrokecolor{currentstroke}%
\pgfsetdash{}{0pt}%
\pgfpathmoveto{\pgfqpoint{3.436945in}{1.166199in}}%
\pgfpathlineto{\pgfqpoint{2.634268in}{1.860957in}}%
\pgfpathlineto{\pgfqpoint{2.643704in}{3.183736in}}%
\pgfusepath{stroke}%
\end{pgfscope}%
\begin{pgfscope}%
\pgfsetbuttcap%
\pgfsetroundjoin%
\pgfsetlinewidth{0.803000pt}%
\definecolor{currentstroke}{rgb}{0.690196,0.690196,0.690196}%
\pgfsetstrokecolor{currentstroke}%
\pgfsetdash{}{0pt}%
\pgfpathmoveto{\pgfqpoint{3.204592in}{1.090158in}}%
\pgfpathlineto{\pgfqpoint{2.412903in}{1.795511in}}%
\pgfpathlineto{\pgfqpoint{2.411161in}{3.125126in}}%
\pgfusepath{stroke}%
\end{pgfscope}%
\begin{pgfscope}%
\pgfsetbuttcap%
\pgfsetroundjoin%
\pgfsetlinewidth{0.803000pt}%
\definecolor{currentstroke}{rgb}{0.690196,0.690196,0.690196}%
\pgfsetstrokecolor{currentstroke}%
\pgfsetdash{}{0pt}%
\pgfpathmoveto{\pgfqpoint{2.968546in}{1.012908in}}%
\pgfpathlineto{\pgfqpoint{2.188277in}{1.729101in}}%
\pgfpathlineto{\pgfqpoint{2.175018in}{3.065608in}}%
\pgfusepath{stroke}%
\end{pgfscope}%
\begin{pgfscope}%
\pgfsetbuttcap%
\pgfsetroundjoin%
\pgfsetlinewidth{0.803000pt}%
\definecolor{currentstroke}{rgb}{0.690196,0.690196,0.690196}%
\pgfsetstrokecolor{currentstroke}%
\pgfsetdash{}{0pt}%
\pgfpathmoveto{\pgfqpoint{2.728720in}{0.934420in}}%
\pgfpathlineto{\pgfqpoint{1.960319in}{1.661706in}}%
\pgfpathlineto{\pgfqpoint{1.935190in}{3.005161in}}%
\pgfusepath{stroke}%
\end{pgfscope}%
\begin{pgfscope}%
\pgfsetbuttcap%
\pgfsetroundjoin%
\pgfsetlinewidth{0.803000pt}%
\definecolor{currentstroke}{rgb}{0.690196,0.690196,0.690196}%
\pgfsetstrokecolor{currentstroke}%
\pgfsetdash{}{0pt}%
\pgfpathmoveto{\pgfqpoint{2.485022in}{0.854665in}}%
\pgfpathlineto{\pgfqpoint{1.728952in}{1.593303in}}%
\pgfpathlineto{\pgfqpoint{1.691590in}{2.943764in}}%
\pgfusepath{stroke}%
\end{pgfscope}%
\begin{pgfscope}%
\pgfsetbuttcap%
\pgfsetroundjoin%
\pgfsetlinewidth{0.803000pt}%
\definecolor{currentstroke}{rgb}{0.690196,0.690196,0.690196}%
\pgfsetstrokecolor{currentstroke}%
\pgfsetdash{}{0pt}%
\pgfpathmoveto{\pgfqpoint{2.237356in}{0.773612in}}%
\pgfpathlineto{\pgfqpoint{1.494101in}{1.523869in}}%
\pgfpathlineto{\pgfqpoint{1.444129in}{2.881394in}}%
\pgfusepath{stroke}%
\end{pgfscope}%
\begin{pgfscope}%
\pgfsetrectcap%
\pgfsetroundjoin%
\pgfsetlinewidth{0.803000pt}%
\definecolor{currentstroke}{rgb}{0.000000,0.000000,0.000000}%
\pgfsetstrokecolor{currentstroke}%
\pgfsetdash{}{0pt}%
\pgfpathmoveto{\pgfqpoint{3.430029in}{1.172185in}}%
\pgfpathlineto{\pgfqpoint{3.450801in}{1.154206in}}%
\pgfusepath{stroke}%
\end{pgfscope}%
\begin{pgfscope}%
\definecolor{textcolor}{rgb}{0.000000,0.000000,0.000000}%
\pgfsetstrokecolor{textcolor}%
\pgfsetfillcolor{textcolor}%
\pgftext[x=3.525746in,y=0.940342in,,top]{\color{textcolor}\sffamily\fontsize{10.000000}{12.000000}\selectfont 0.0}%
\end{pgfscope}%
\begin{pgfscope}%
\pgfsetrectcap%
\pgfsetroundjoin%
\pgfsetlinewidth{0.803000pt}%
\definecolor{currentstroke}{rgb}{0.000000,0.000000,0.000000}%
\pgfsetstrokecolor{currentstroke}%
\pgfsetdash{}{0pt}%
\pgfpathmoveto{\pgfqpoint{3.197767in}{1.096238in}}%
\pgfpathlineto{\pgfqpoint{3.218267in}{1.077974in}}%
\pgfusepath{stroke}%
\end{pgfscope}%
\begin{pgfscope}%
\definecolor{textcolor}{rgb}{0.000000,0.000000,0.000000}%
\pgfsetstrokecolor{textcolor}%
\pgfsetfillcolor{textcolor}%
\pgftext[x=3.293434in,y=0.862298in,,top]{\color{textcolor}\sffamily\fontsize{10.000000}{12.000000}\selectfont 0.2}%
\end{pgfscope}%
\begin{pgfscope}%
\pgfsetrectcap%
\pgfsetroundjoin%
\pgfsetlinewidth{0.803000pt}%
\definecolor{currentstroke}{rgb}{0.000000,0.000000,0.000000}%
\pgfsetstrokecolor{currentstroke}%
\pgfsetdash{}{0pt}%
\pgfpathmoveto{\pgfqpoint{2.961816in}{1.019085in}}%
\pgfpathlineto{\pgfqpoint{2.982032in}{1.000529in}}%
\pgfusepath{stroke}%
\end{pgfscope}%
\begin{pgfscope}%
\definecolor{textcolor}{rgb}{0.000000,0.000000,0.000000}%
\pgfsetstrokecolor{textcolor}%
\pgfsetfillcolor{textcolor}%
\pgftext[x=3.057419in,y=0.783011in,,top]{\color{textcolor}\sffamily\fontsize{10.000000}{12.000000}\selectfont 0.4}%
\end{pgfscope}%
\begin{pgfscope}%
\pgfsetrectcap%
\pgfsetroundjoin%
\pgfsetlinewidth{0.803000pt}%
\definecolor{currentstroke}{rgb}{0.000000,0.000000,0.000000}%
\pgfsetstrokecolor{currentstroke}%
\pgfsetdash{}{0pt}%
\pgfpathmoveto{\pgfqpoint{2.722089in}{0.940697in}}%
\pgfpathlineto{\pgfqpoint{2.742009in}{0.921843in}}%
\pgfusepath{stroke}%
\end{pgfscope}%
\begin{pgfscope}%
\definecolor{textcolor}{rgb}{0.000000,0.000000,0.000000}%
\pgfsetstrokecolor{textcolor}%
\pgfsetfillcolor{textcolor}%
\pgftext[x=2.817614in,y=0.702450in,,top]{\color{textcolor}\sffamily\fontsize{10.000000}{12.000000}\selectfont 0.6}%
\end{pgfscope}%
\begin{pgfscope}%
\pgfsetrectcap%
\pgfsetroundjoin%
\pgfsetlinewidth{0.803000pt}%
\definecolor{currentstroke}{rgb}{0.000000,0.000000,0.000000}%
\pgfsetstrokecolor{currentstroke}%
\pgfsetdash{}{0pt}%
\pgfpathmoveto{\pgfqpoint{2.478493in}{0.861044in}}%
\pgfpathlineto{\pgfqpoint{2.498105in}{0.841884in}}%
\pgfusepath{stroke}%
\end{pgfscope}%
\begin{pgfscope}%
\definecolor{textcolor}{rgb}{0.000000,0.000000,0.000000}%
\pgfsetstrokecolor{textcolor}%
\pgfsetfillcolor{textcolor}%
\pgftext[x=2.573925in,y=0.620584in,,top]{\color{textcolor}\sffamily\fontsize{10.000000}{12.000000}\selectfont 0.8}%
\end{pgfscope}%
\begin{pgfscope}%
\pgfsetrectcap%
\pgfsetroundjoin%
\pgfsetlinewidth{0.803000pt}%
\definecolor{currentstroke}{rgb}{0.000000,0.000000,0.000000}%
\pgfsetstrokecolor{currentstroke}%
\pgfsetdash{}{0pt}%
\pgfpathmoveto{\pgfqpoint{2.230934in}{0.780095in}}%
\pgfpathlineto{\pgfqpoint{2.250225in}{0.760622in}}%
\pgfusepath{stroke}%
\end{pgfscope}%
\begin{pgfscope}%
\definecolor{textcolor}{rgb}{0.000000,0.000000,0.000000}%
\pgfsetstrokecolor{textcolor}%
\pgfsetfillcolor{textcolor}%
\pgftext[x=2.326257in,y=0.537382in,,top]{\color{textcolor}\sffamily\fontsize{10.000000}{12.000000}\selectfont 1.0}%
\end{pgfscope}%
\begin{pgfscope}%
\pgfsetrectcap%
\pgfsetroundjoin%
\pgfsetlinewidth{0.803000pt}%
\definecolor{currentstroke}{rgb}{0.000000,0.000000,0.000000}%
\pgfsetstrokecolor{currentstroke}%
\pgfsetdash{}{0pt}%
\pgfpathmoveto{\pgfqpoint{1.407596in}{1.498294in}}%
\pgfpathlineto{\pgfqpoint{2.146055in}{0.743733in}}%
\pgfusepath{stroke}%
\end{pgfscope}%
\begin{pgfscope}%
\pgfsetbuttcap%
\pgfsetroundjoin%
\pgfsetlinewidth{0.803000pt}%
\definecolor{currentstroke}{rgb}{0.690196,0.690196,0.690196}%
\pgfsetstrokecolor{currentstroke}%
\pgfsetdash{}{0pt}%
\pgfpathmoveto{\pgfqpoint{2.777949in}{3.168181in}}%
\pgfpathlineto{\pgfqpoint{2.762037in}{1.843583in}}%
\pgfpathlineto{\pgfqpoint{1.451276in}{1.453662in}}%
\pgfusepath{stroke}%
\end{pgfscope}%
\begin{pgfscope}%
\pgfsetbuttcap%
\pgfsetroundjoin%
\pgfsetlinewidth{0.803000pt}%
\definecolor{currentstroke}{rgb}{0.690196,0.690196,0.690196}%
\pgfsetstrokecolor{currentstroke}%
\pgfsetdash{}{0pt}%
\pgfpathmoveto{\pgfqpoint{2.918559in}{3.065671in}}%
\pgfpathlineto{\pgfqpoint{2.895594in}{1.729172in}}%
\pgfpathlineto{\pgfqpoint{1.573260in}{1.329018in}}%
\pgfusepath{stroke}%
\end{pgfscope}%
\begin{pgfscope}%
\pgfsetbuttcap%
\pgfsetroundjoin%
\pgfsetlinewidth{0.803000pt}%
\definecolor{currentstroke}{rgb}{0.690196,0.690196,0.690196}%
\pgfsetstrokecolor{currentstroke}%
\pgfsetdash{}{0pt}%
\pgfpathmoveto{\pgfqpoint{3.062992in}{2.960375in}}%
\pgfpathlineto{\pgfqpoint{3.032602in}{1.611804in}}%
\pgfpathlineto{\pgfqpoint{1.698537in}{1.201009in}}%
\pgfusepath{stroke}%
\end{pgfscope}%
\begin{pgfscope}%
\pgfsetbuttcap%
\pgfsetroundjoin%
\pgfsetlinewidth{0.803000pt}%
\definecolor{currentstroke}{rgb}{0.690196,0.690196,0.690196}%
\pgfsetstrokecolor{currentstroke}%
\pgfsetdash{}{0pt}%
\pgfpathmoveto{\pgfqpoint{3.211405in}{2.852177in}}%
\pgfpathlineto{\pgfqpoint{3.173197in}{1.491363in}}%
\pgfpathlineto{\pgfqpoint{1.827243in}{1.069497in}}%
\pgfusepath{stroke}%
\end{pgfscope}%
\begin{pgfscope}%
\pgfsetbuttcap%
\pgfsetroundjoin%
\pgfsetlinewidth{0.803000pt}%
\definecolor{currentstroke}{rgb}{0.690196,0.690196,0.690196}%
\pgfsetstrokecolor{currentstroke}%
\pgfsetdash{}{0pt}%
\pgfpathmoveto{\pgfqpoint{3.363965in}{2.740956in}}%
\pgfpathlineto{\pgfqpoint{3.317522in}{1.367728in}}%
\pgfpathlineto{\pgfqpoint{1.959519in}{0.934337in}}%
\pgfusepath{stroke}%
\end{pgfscope}%
\begin{pgfscope}%
\pgfsetbuttcap%
\pgfsetroundjoin%
\pgfsetlinewidth{0.803000pt}%
\definecolor{currentstroke}{rgb}{0.690196,0.690196,0.690196}%
\pgfsetstrokecolor{currentstroke}%
\pgfsetdash{}{0pt}%
\pgfpathmoveto{\pgfqpoint{3.520848in}{2.626583in}}%
\pgfpathlineto{\pgfqpoint{3.465727in}{1.240768in}}%
\pgfpathlineto{\pgfqpoint{2.095516in}{0.795374in}}%
\pgfusepath{stroke}%
\end{pgfscope}%
\begin{pgfscope}%
\pgfsetrectcap%
\pgfsetroundjoin%
\pgfsetlinewidth{0.803000pt}%
\definecolor{currentstroke}{rgb}{0.000000,0.000000,0.000000}%
\pgfsetstrokecolor{currentstroke}%
\pgfsetdash{}{0pt}%
\pgfpathmoveto{\pgfqpoint{1.462214in}{1.456916in}}%
\pgfpathlineto{\pgfqpoint{1.429376in}{1.447148in}}%
\pgfusepath{stroke}%
\end{pgfscope}%
\begin{pgfscope}%
\definecolor{textcolor}{rgb}{0.000000,0.000000,0.000000}%
\pgfsetstrokecolor{textcolor}%
\pgfsetfillcolor{textcolor}%
\pgftext[x=1.307986in,y=1.268483in,,top]{\color{textcolor}\sffamily\fontsize{10.000000}{12.000000}\selectfont 0.0}%
\end{pgfscope}%
\begin{pgfscope}%
\pgfsetrectcap%
\pgfsetroundjoin%
\pgfsetlinewidth{0.803000pt}%
\definecolor{currentstroke}{rgb}{0.000000,0.000000,0.000000}%
\pgfsetstrokecolor{currentstroke}%
\pgfsetdash{}{0pt}%
\pgfpathmoveto{\pgfqpoint{1.584301in}{1.332359in}}%
\pgfpathlineto{\pgfqpoint{1.551155in}{1.322329in}}%
\pgfusepath{stroke}%
\end{pgfscope}%
\begin{pgfscope}%
\definecolor{textcolor}{rgb}{0.000000,0.000000,0.000000}%
\pgfsetstrokecolor{textcolor}%
\pgfsetfillcolor{textcolor}%
\pgftext[x=1.427974in,y=1.141467in,,top]{\color{textcolor}\sffamily\fontsize{10.000000}{12.000000}\selectfont 0.2}%
\end{pgfscope}%
\begin{pgfscope}%
\pgfsetrectcap%
\pgfsetroundjoin%
\pgfsetlinewidth{0.803000pt}%
\definecolor{currentstroke}{rgb}{0.000000,0.000000,0.000000}%
\pgfsetstrokecolor{currentstroke}%
\pgfsetdash{}{0pt}%
\pgfpathmoveto{\pgfqpoint{1.709682in}{1.204441in}}%
\pgfpathlineto{\pgfqpoint{1.676223in}{1.194138in}}%
\pgfusepath{stroke}%
\end{pgfscope}%
\begin{pgfscope}%
\definecolor{textcolor}{rgb}{0.000000,0.000000,0.000000}%
\pgfsetstrokecolor{textcolor}%
\pgfsetfillcolor{textcolor}%
\pgftext[x=1.551197in,y=1.011025in,,top]{\color{textcolor}\sffamily\fontsize{10.000000}{12.000000}\selectfont 0.4}%
\end{pgfscope}%
\begin{pgfscope}%
\pgfsetrectcap%
\pgfsetroundjoin%
\pgfsetlinewidth{0.803000pt}%
\definecolor{currentstroke}{rgb}{0.000000,0.000000,0.000000}%
\pgfsetstrokecolor{currentstroke}%
\pgfsetdash{}{0pt}%
\pgfpathmoveto{\pgfqpoint{1.838494in}{1.073023in}}%
\pgfpathlineto{\pgfqpoint{1.804716in}{1.062437in}}%
\pgfusepath{stroke}%
\end{pgfscope}%
\begin{pgfscope}%
\definecolor{textcolor}{rgb}{0.000000,0.000000,0.000000}%
\pgfsetstrokecolor{textcolor}%
\pgfsetfillcolor{textcolor}%
\pgftext[x=1.677790in,y=0.877016in,,top]{\color{textcolor}\sffamily\fontsize{10.000000}{12.000000}\selectfont 0.6}%
\end{pgfscope}%
\begin{pgfscope}%
\pgfsetrectcap%
\pgfsetroundjoin%
\pgfsetlinewidth{0.803000pt}%
\definecolor{currentstroke}{rgb}{0.000000,0.000000,0.000000}%
\pgfsetstrokecolor{currentstroke}%
\pgfsetdash{}{0pt}%
\pgfpathmoveto{\pgfqpoint{1.970877in}{0.937961in}}%
\pgfpathlineto{\pgfqpoint{1.936777in}{0.927079in}}%
\pgfusepath{stroke}%
\end{pgfscope}%
\begin{pgfscope}%
\definecolor{textcolor}{rgb}{0.000000,0.000000,0.000000}%
\pgfsetstrokecolor{textcolor}%
\pgfsetfillcolor{textcolor}%
\pgftext[x=1.807893in,y=0.739291in,,top]{\color{textcolor}\sffamily\fontsize{10.000000}{12.000000}\selectfont 0.8}%
\end{pgfscope}%
\begin{pgfscope}%
\pgfsetrectcap%
\pgfsetroundjoin%
\pgfsetlinewidth{0.803000pt}%
\definecolor{currentstroke}{rgb}{0.000000,0.000000,0.000000}%
\pgfsetstrokecolor{currentstroke}%
\pgfsetdash{}{0pt}%
\pgfpathmoveto{\pgfqpoint{2.106984in}{0.799101in}}%
\pgfpathlineto{\pgfqpoint{2.072556in}{0.787910in}}%
\pgfusepath{stroke}%
\end{pgfscope}%
\begin{pgfscope}%
\definecolor{textcolor}{rgb}{0.000000,0.000000,0.000000}%
\pgfsetstrokecolor{textcolor}%
\pgfsetfillcolor{textcolor}%
\pgftext[x=1.941653in,y=0.597696in,,top]{\color{textcolor}\sffamily\fontsize{10.000000}{12.000000}\selectfont 1.0}%
\end{pgfscope}%
\begin{pgfscope}%
\pgfsetrectcap%
\pgfsetroundjoin%
\pgfsetlinewidth{0.803000pt}%
\definecolor{currentstroke}{rgb}{0.000000,0.000000,0.000000}%
\pgfsetstrokecolor{currentstroke}%
\pgfsetdash{}{0pt}%
\pgfpathmoveto{\pgfqpoint{1.407596in}{1.498294in}}%
\pgfpathlineto{\pgfqpoint{1.352930in}{2.858408in}}%
\pgfusepath{stroke}%
\end{pgfscope}%
\begin{pgfscope}%
\pgfsetbuttcap%
\pgfsetroundjoin%
\pgfsetlinewidth{0.803000pt}%
\definecolor{currentstroke}{rgb}{0.690196,0.690196,0.690196}%
\pgfsetstrokecolor{currentstroke}%
\pgfsetdash{}{0pt}%
\pgfpathmoveto{\pgfqpoint{1.404280in}{1.580785in}}%
\pgfpathlineto{\pgfqpoint{2.714993in}{1.964823in}}%
\pgfpathlineto{\pgfqpoint{3.524292in}{1.277822in}}%
\pgfusepath{stroke}%
\end{pgfscope}%
\begin{pgfscope}%
\pgfsetbuttcap%
\pgfsetroundjoin%
\pgfsetlinewidth{0.803000pt}%
\definecolor{currentstroke}{rgb}{0.690196,0.690196,0.690196}%
\pgfsetstrokecolor{currentstroke}%
\pgfsetdash{}{0pt}%
\pgfpathmoveto{\pgfqpoint{1.395077in}{1.809771in}}%
\pgfpathlineto{\pgfqpoint{2.717257in}{2.187449in}}%
\pgfpathlineto{\pgfqpoint{3.534112in}{1.511629in}}%
\pgfusepath{stroke}%
\end{pgfscope}%
\begin{pgfscope}%
\pgfsetbuttcap%
\pgfsetroundjoin%
\pgfsetlinewidth{0.803000pt}%
\definecolor{currentstroke}{rgb}{0.690196,0.690196,0.690196}%
\pgfsetstrokecolor{currentstroke}%
\pgfsetdash{}{0pt}%
\pgfpathmoveto{\pgfqpoint{1.385710in}{2.042834in}}%
\pgfpathlineto{\pgfqpoint{2.719560in}{2.413873in}}%
\pgfpathlineto{\pgfqpoint{3.544113in}{1.749737in}}%
\pgfusepath{stroke}%
\end{pgfscope}%
\begin{pgfscope}%
\pgfsetbuttcap%
\pgfsetroundjoin%
\pgfsetlinewidth{0.803000pt}%
\definecolor{currentstroke}{rgb}{0.690196,0.690196,0.690196}%
\pgfsetstrokecolor{currentstroke}%
\pgfsetdash{}{0pt}%
\pgfpathmoveto{\pgfqpoint{1.376174in}{2.280084in}}%
\pgfpathlineto{\pgfqpoint{2.721903in}{2.644191in}}%
\pgfpathlineto{\pgfqpoint{3.554299in}{1.992268in}}%
\pgfusepath{stroke}%
\end{pgfscope}%
\begin{pgfscope}%
\pgfsetbuttcap%
\pgfsetroundjoin%
\pgfsetlinewidth{0.803000pt}%
\definecolor{currentstroke}{rgb}{0.690196,0.690196,0.690196}%
\pgfsetstrokecolor{currentstroke}%
\pgfsetdash{}{0pt}%
\pgfpathmoveto{\pgfqpoint{1.366466in}{2.521636in}}%
\pgfpathlineto{\pgfqpoint{2.724286in}{2.878505in}}%
\pgfpathlineto{\pgfqpoint{3.564677in}{2.239346in}}%
\pgfusepath{stroke}%
\end{pgfscope}%
\begin{pgfscope}%
\pgfsetbuttcap%
\pgfsetroundjoin%
\pgfsetlinewidth{0.803000pt}%
\definecolor{currentstroke}{rgb}{0.690196,0.690196,0.690196}%
\pgfsetstrokecolor{currentstroke}%
\pgfsetdash{}{0pt}%
\pgfpathmoveto{\pgfqpoint{1.356579in}{2.767607in}}%
\pgfpathlineto{\pgfqpoint{2.726711in}{3.116919in}}%
\pgfpathlineto{\pgfqpoint{3.575251in}{2.491100in}}%
\pgfusepath{stroke}%
\end{pgfscope}%
\begin{pgfscope}%
\pgfsetrectcap%
\pgfsetroundjoin%
\pgfsetlinewidth{0.803000pt}%
\definecolor{currentstroke}{rgb}{0.000000,0.000000,0.000000}%
\pgfsetstrokecolor{currentstroke}%
\pgfsetdash{}{0pt}%
\pgfpathmoveto{\pgfqpoint{1.415217in}{1.583990in}}%
\pgfpathlineto{\pgfqpoint{1.382384in}{1.574369in}}%
\pgfusepath{stroke}%
\end{pgfscope}%
\begin{pgfscope}%
\definecolor{textcolor}{rgb}{0.000000,0.000000,0.000000}%
\pgfsetstrokecolor{textcolor}%
\pgfsetfillcolor{textcolor}%
\pgftext[x=1.181686in,y=1.612595in,,top]{\color{textcolor}\sffamily\fontsize{10.000000}{12.000000}\selectfont 0.0}%
\end{pgfscope}%
\begin{pgfscope}%
\pgfsetrectcap%
\pgfsetroundjoin%
\pgfsetlinewidth{0.803000pt}%
\definecolor{currentstroke}{rgb}{0.000000,0.000000,0.000000}%
\pgfsetstrokecolor{currentstroke}%
\pgfsetdash{}{0pt}%
\pgfpathmoveto{\pgfqpoint{1.406114in}{1.812923in}}%
\pgfpathlineto{\pgfqpoint{1.372980in}{1.803459in}}%
\pgfusepath{stroke}%
\end{pgfscope}%
\begin{pgfscope}%
\definecolor{textcolor}{rgb}{0.000000,0.000000,0.000000}%
\pgfsetstrokecolor{textcolor}%
\pgfsetfillcolor{textcolor}%
\pgftext[x=1.170557in,y=1.841065in,,top]{\color{textcolor}\sffamily\fontsize{10.000000}{12.000000}\selectfont 0.2}%
\end{pgfscope}%
\begin{pgfscope}%
\pgfsetrectcap%
\pgfsetroundjoin%
\pgfsetlinewidth{0.803000pt}%
\definecolor{currentstroke}{rgb}{0.000000,0.000000,0.000000}%
\pgfsetstrokecolor{currentstroke}%
\pgfsetdash{}{0pt}%
\pgfpathmoveto{\pgfqpoint{1.396848in}{2.045932in}}%
\pgfpathlineto{\pgfqpoint{1.363409in}{2.036631in}}%
\pgfusepath{stroke}%
\end{pgfscope}%
\begin{pgfscope}%
\definecolor{textcolor}{rgb}{0.000000,0.000000,0.000000}%
\pgfsetstrokecolor{textcolor}%
\pgfsetfillcolor{textcolor}%
\pgftext[x=1.159231in,y=2.073588in,,top]{\color{textcolor}\sffamily\fontsize{10.000000}{12.000000}\selectfont 0.4}%
\end{pgfscope}%
\begin{pgfscope}%
\pgfsetrectcap%
\pgfsetroundjoin%
\pgfsetlinewidth{0.803000pt}%
\definecolor{currentstroke}{rgb}{0.000000,0.000000,0.000000}%
\pgfsetstrokecolor{currentstroke}%
\pgfsetdash{}{0pt}%
\pgfpathmoveto{\pgfqpoint{1.387416in}{2.283126in}}%
\pgfpathlineto{\pgfqpoint{1.353667in}{2.273995in}}%
\pgfusepath{stroke}%
\end{pgfscope}%
\begin{pgfscope}%
\definecolor{textcolor}{rgb}{0.000000,0.000000,0.000000}%
\pgfsetstrokecolor{textcolor}%
\pgfsetfillcolor{textcolor}%
\pgftext[x=1.147701in,y=2.310275in,,top]{\color{textcolor}\sffamily\fontsize{10.000000}{12.000000}\selectfont 0.6}%
\end{pgfscope}%
\begin{pgfscope}%
\pgfsetrectcap%
\pgfsetroundjoin%
\pgfsetlinewidth{0.803000pt}%
\definecolor{currentstroke}{rgb}{0.000000,0.000000,0.000000}%
\pgfsetstrokecolor{currentstroke}%
\pgfsetdash{}{0pt}%
\pgfpathmoveto{\pgfqpoint{1.377813in}{2.524619in}}%
\pgfpathlineto{\pgfqpoint{1.343747in}{2.515665in}}%
\pgfusepath{stroke}%
\end{pgfscope}%
\begin{pgfscope}%
\definecolor{textcolor}{rgb}{0.000000,0.000000,0.000000}%
\pgfsetstrokecolor{textcolor}%
\pgfsetfillcolor{textcolor}%
\pgftext[x=1.135964in,y=2.551237in,,top]{\color{textcolor}\sffamily\fontsize{10.000000}{12.000000}\selectfont 0.8}%
\end{pgfscope}%
\begin{pgfscope}%
\pgfsetrectcap%
\pgfsetroundjoin%
\pgfsetlinewidth{0.803000pt}%
\definecolor{currentstroke}{rgb}{0.000000,0.000000,0.000000}%
\pgfsetstrokecolor{currentstroke}%
\pgfsetdash{}{0pt}%
\pgfpathmoveto{\pgfqpoint{1.368034in}{2.770528in}}%
\pgfpathlineto{\pgfqpoint{1.333646in}{2.761761in}}%
\pgfusepath{stroke}%
\end{pgfscope}%
\begin{pgfscope}%
\definecolor{textcolor}{rgb}{0.000000,0.000000,0.000000}%
\pgfsetstrokecolor{textcolor}%
\pgfsetfillcolor{textcolor}%
\pgftext[x=1.124012in,y=2.796592in,,top]{\color{textcolor}\sffamily\fontsize{10.000000}{12.000000}\selectfont 1.0}%
\end{pgfscope}%
\begin{pgfscope}%
\pgfpathrectangle{\pgfqpoint{1.000000in}{0.570909in}}{\pgfqpoint{2.818182in}{2.818182in}}%
\pgfusepath{clip}%
\pgfsetbuttcap%
\pgfsetroundjoin%
\definecolor{currentfill}{rgb}{1.000000,0.498039,0.054902}%
\pgfsetfillcolor{currentfill}%
\pgfsetlinewidth{1.003750pt}%
\definecolor{currentstroke}{rgb}{1.000000,0.498039,0.054902}%
\pgfsetstrokecolor{currentstroke}%
\pgfsetdash{}{0pt}%
\pgfpathmoveto{\pgfqpoint{2.390761in}{1.768600in}}%
\pgfpathcurveto{\pgfqpoint{2.398998in}{1.768600in}}{\pgfqpoint{2.406898in}{1.771872in}}{\pgfqpoint{2.412722in}{1.777696in}}%
\pgfpathcurveto{\pgfqpoint{2.418545in}{1.783520in}}{\pgfqpoint{2.421818in}{1.791420in}}{\pgfqpoint{2.421818in}{1.799656in}}%
\pgfpathcurveto{\pgfqpoint{2.421818in}{1.807893in}}{\pgfqpoint{2.418545in}{1.815793in}}{\pgfqpoint{2.412722in}{1.821617in}}%
\pgfpathcurveto{\pgfqpoint{2.406898in}{1.827440in}}{\pgfqpoint{2.398998in}{1.830713in}}{\pgfqpoint{2.390761in}{1.830713in}}%
\pgfpathcurveto{\pgfqpoint{2.382525in}{1.830713in}}{\pgfqpoint{2.374625in}{1.827440in}}{\pgfqpoint{2.368801in}{1.821617in}}%
\pgfpathcurveto{\pgfqpoint{2.362977in}{1.815793in}}{\pgfqpoint{2.359705in}{1.807893in}}{\pgfqpoint{2.359705in}{1.799656in}}%
\pgfpathcurveto{\pgfqpoint{2.359705in}{1.791420in}}{\pgfqpoint{2.362977in}{1.783520in}}{\pgfqpoint{2.368801in}{1.777696in}}%
\pgfpathcurveto{\pgfqpoint{2.374625in}{1.771872in}}{\pgfqpoint{2.382525in}{1.768600in}}{\pgfqpoint{2.390761in}{1.768600in}}%
\pgfpathclose%
\pgfusepath{stroke,fill}%
\end{pgfscope}%
\begin{pgfscope}%
\definecolor{textcolor}{rgb}{0.000000,0.000000,0.000000}%
\pgfsetstrokecolor{textcolor}%
\pgfsetfillcolor{textcolor}%
\pgftext[x=2.716732in,y=2.440967in,left,base]{\color{textcolor}\sffamily\fontsize{10.000000}{12.000000}\selectfont 0.04166}%
\end{pgfscope}%
\begin{pgfscope}%
\definecolor{textcolor}{rgb}{0.000000,0.000000,0.000000}%
\pgfsetstrokecolor{textcolor}%
\pgfsetfillcolor{textcolor}%
\pgftext[x=2.390761in,y=1.799656in,left,base]{\color{textcolor}\sffamily\fontsize{10.000000}{12.000000}\selectfont 0.04166}%
\end{pgfscope}%
\begin{pgfscope}%
\definecolor{textcolor}{rgb}{0.000000,0.000000,0.000000}%
\pgfsetstrokecolor{textcolor}%
\pgfsetfillcolor{textcolor}%
\pgftext[x=2.218917in,y=1.901818in,left,base]{\color{textcolor}\sffamily\fontsize{10.000000}{12.000000}\selectfont 0.04166}%
\end{pgfscope}%
\begin{pgfscope}%
\definecolor{textcolor}{rgb}{0.000000,0.000000,0.000000}%
\pgfsetstrokecolor{textcolor}%
\pgfsetfillcolor{textcolor}%
\pgftext[x=2.957969in,y=1.682678in,left,base]{\color{textcolor}\sffamily\fontsize{10.000000}{12.000000}\selectfont 0.04166}%
\end{pgfscope}%
\begin{pgfscope}%
\pgfpathrectangle{\pgfqpoint{1.000000in}{0.570909in}}{\pgfqpoint{2.818182in}{2.818182in}}%
\pgfusepath{clip}%
\pgfsetbuttcap%
\pgfsetroundjoin%
\definecolor{currentfill}{rgb}{0.121569,0.466667,0.705882}%
\pgfsetfillcolor{currentfill}%
\pgfsetlinewidth{1.003750pt}%
\definecolor{currentstroke}{rgb}{0.121569,0.466667,0.705882}%
\pgfsetstrokecolor{currentstroke}%
\pgfsetdash{}{0pt}%
\pgfpathmoveto{\pgfqpoint{2.716732in}{2.409911in}}%
\pgfpathcurveto{\pgfqpoint{2.724968in}{2.409911in}}{\pgfqpoint{2.732868in}{2.413183in}}{\pgfqpoint{2.738692in}{2.419007in}}%
\pgfpathcurveto{\pgfqpoint{2.744516in}{2.424831in}}{\pgfqpoint{2.747788in}{2.432731in}}{\pgfqpoint{2.747788in}{2.440967in}}%
\pgfpathcurveto{\pgfqpoint{2.747788in}{2.449204in}}{\pgfqpoint{2.744516in}{2.457104in}}{\pgfqpoint{2.738692in}{2.462928in}}%
\pgfpathcurveto{\pgfqpoint{2.732868in}{2.468752in}}{\pgfqpoint{2.724968in}{2.472024in}}{\pgfqpoint{2.716732in}{2.472024in}}%
\pgfpathcurveto{\pgfqpoint{2.708495in}{2.472024in}}{\pgfqpoint{2.700595in}{2.468752in}}{\pgfqpoint{2.694771in}{2.462928in}}%
\pgfpathcurveto{\pgfqpoint{2.688947in}{2.457104in}}{\pgfqpoint{2.685675in}{2.449204in}}{\pgfqpoint{2.685675in}{2.440967in}}%
\pgfpathcurveto{\pgfqpoint{2.685675in}{2.432731in}}{\pgfqpoint{2.688947in}{2.424831in}}{\pgfqpoint{2.694771in}{2.419007in}}%
\pgfpathcurveto{\pgfqpoint{2.700595in}{2.413183in}}{\pgfqpoint{2.708495in}{2.409911in}}{\pgfqpoint{2.716732in}{2.409911in}}%
\pgfpathclose%
\pgfusepath{stroke,fill}%
\end{pgfscope}%
\begin{pgfscope}%
\pgfpathrectangle{\pgfqpoint{1.000000in}{0.570909in}}{\pgfqpoint{2.818182in}{2.818182in}}%
\pgfusepath{clip}%
\pgfsetbuttcap%
\pgfsetroundjoin%
\definecolor{currentfill}{rgb}{0.172549,0.627451,0.172549}%
\pgfsetfillcolor{currentfill}%
\pgfsetlinewidth{1.003750pt}%
\definecolor{currentstroke}{rgb}{0.172549,0.627451,0.172549}%
\pgfsetstrokecolor{currentstroke}%
\pgfsetdash{}{0pt}%
\pgfpathmoveto{\pgfqpoint{2.218917in}{1.870762in}}%
\pgfpathcurveto{\pgfqpoint{2.227153in}{1.870762in}}{\pgfqpoint{2.235053in}{1.874034in}}{\pgfqpoint{2.240877in}{1.879858in}}%
\pgfpathcurveto{\pgfqpoint{2.246701in}{1.885682in}}{\pgfqpoint{2.249973in}{1.893582in}}{\pgfqpoint{2.249973in}{1.901818in}}%
\pgfpathcurveto{\pgfqpoint{2.249973in}{1.910054in}}{\pgfqpoint{2.246701in}{1.917954in}}{\pgfqpoint{2.240877in}{1.923778in}}%
\pgfpathcurveto{\pgfqpoint{2.235053in}{1.929602in}}{\pgfqpoint{2.227153in}{1.932875in}}{\pgfqpoint{2.218917in}{1.932875in}}%
\pgfpathcurveto{\pgfqpoint{2.210680in}{1.932875in}}{\pgfqpoint{2.202780in}{1.929602in}}{\pgfqpoint{2.196956in}{1.923778in}}%
\pgfpathcurveto{\pgfqpoint{2.191133in}{1.917954in}}{\pgfqpoint{2.187860in}{1.910054in}}{\pgfqpoint{2.187860in}{1.901818in}}%
\pgfpathcurveto{\pgfqpoint{2.187860in}{1.893582in}}{\pgfqpoint{2.191133in}{1.885682in}}{\pgfqpoint{2.196956in}{1.879858in}}%
\pgfpathcurveto{\pgfqpoint{2.202780in}{1.874034in}}{\pgfqpoint{2.210680in}{1.870762in}}{\pgfqpoint{2.218917in}{1.870762in}}%
\pgfpathclose%
\pgfusepath{stroke,fill}%
\end{pgfscope}%
\begin{pgfscope}%
\pgfpathrectangle{\pgfqpoint{1.000000in}{0.570909in}}{\pgfqpoint{2.818182in}{2.818182in}}%
\pgfusepath{clip}%
\pgfsetbuttcap%
\pgfsetroundjoin%
\definecolor{currentfill}{rgb}{0.839216,0.152941,0.156863}%
\pgfsetfillcolor{currentfill}%
\pgfsetlinewidth{1.003750pt}%
\definecolor{currentstroke}{rgb}{0.839216,0.152941,0.156863}%
\pgfsetstrokecolor{currentstroke}%
\pgfsetdash{}{0pt}%
\pgfpathmoveto{\pgfqpoint{2.957969in}{1.651622in}}%
\pgfpathcurveto{\pgfqpoint{2.966205in}{1.651622in}}{\pgfqpoint{2.974105in}{1.654894in}}{\pgfqpoint{2.979929in}{1.660718in}}%
\pgfpathcurveto{\pgfqpoint{2.985753in}{1.666542in}}{\pgfqpoint{2.989025in}{1.674442in}}{\pgfqpoint{2.989025in}{1.682678in}}%
\pgfpathcurveto{\pgfqpoint{2.989025in}{1.690915in}}{\pgfqpoint{2.985753in}{1.698815in}}{\pgfqpoint{2.979929in}{1.704639in}}%
\pgfpathcurveto{\pgfqpoint{2.974105in}{1.710463in}}{\pgfqpoint{2.966205in}{1.713735in}}{\pgfqpoint{2.957969in}{1.713735in}}%
\pgfpathcurveto{\pgfqpoint{2.949732in}{1.713735in}}{\pgfqpoint{2.941832in}{1.710463in}}{\pgfqpoint{2.936008in}{1.704639in}}%
\pgfpathcurveto{\pgfqpoint{2.930184in}{1.698815in}}{\pgfqpoint{2.926912in}{1.690915in}}{\pgfqpoint{2.926912in}{1.682678in}}%
\pgfpathcurveto{\pgfqpoint{2.926912in}{1.674442in}}{\pgfqpoint{2.930184in}{1.666542in}}{\pgfqpoint{2.936008in}{1.660718in}}%
\pgfpathcurveto{\pgfqpoint{2.941832in}{1.654894in}}{\pgfqpoint{2.949732in}{1.651622in}}{\pgfqpoint{2.957969in}{1.651622in}}%
\pgfpathclose%
\pgfusepath{stroke,fill}%
\end{pgfscope}%
\begin{pgfscope}%
\pgfpathrectangle{\pgfqpoint{1.000000in}{0.570909in}}{\pgfqpoint{2.818182in}{2.818182in}}%
\pgfusepath{clip}%
\pgfsetbuttcap%
\pgfsetroundjoin%
\definecolor{currentfill}{rgb}{0.100000,0.100000,0.100000}%
\pgfsetfillcolor{currentfill}%
\pgfsetfillopacity{0.100000}%
\pgfsetlinewidth{0.301125pt}%
\definecolor{currentstroke}{rgb}{0.000000,0.000000,0.000000}%
\pgfsetstrokecolor{currentstroke}%
\pgfsetdash{}{0pt}%
\pgfpathmoveto{\pgfqpoint{2.682610in}{1.900366in}}%
\pgfpathlineto{\pgfqpoint{1.535168in}{1.562080in}}%
\pgfpathlineto{\pgfqpoint{2.692989in}{3.058384in}}%
\pgfpathclose%
\pgfusepath{stroke,fill}%
\end{pgfscope}%
\begin{pgfscope}%
\pgfpathrectangle{\pgfqpoint{1.000000in}{0.570909in}}{\pgfqpoint{2.818182in}{2.818182in}}%
\pgfusepath{clip}%
\pgfsetbuttcap%
\pgfsetroundjoin%
\definecolor{currentfill}{rgb}{0.100000,0.100000,0.100000}%
\pgfsetfillcolor{currentfill}%
\pgfsetfillopacity{0.100000}%
\pgfsetlinewidth{0.301125pt}%
\definecolor{currentstroke}{rgb}{0.000000,0.000000,0.000000}%
\pgfsetstrokecolor{currentstroke}%
\pgfsetdash{}{0pt}%
\pgfpathmoveto{\pgfqpoint{2.682610in}{1.900366in}}%
\pgfpathlineto{\pgfqpoint{3.385245in}{1.297693in}}%
\pgfpathlineto{\pgfqpoint{1.535168in}{1.562080in}}%
\pgfpathclose%
\pgfusepath{stroke,fill}%
\end{pgfscope}%
\begin{pgfscope}%
\pgfpathrectangle{\pgfqpoint{1.000000in}{0.570909in}}{\pgfqpoint{2.818182in}{2.818182in}}%
\pgfusepath{clip}%
\pgfsetbuttcap%
\pgfsetroundjoin%
\definecolor{currentfill}{rgb}{0.100000,0.100000,0.100000}%
\pgfsetfillcolor{currentfill}%
\pgfsetfillopacity{0.100000}%
\pgfsetlinewidth{0.301125pt}%
\definecolor{currentstroke}{rgb}{0.000000,0.000000,0.000000}%
\pgfsetstrokecolor{currentstroke}%
\pgfsetdash{}{0pt}%
\pgfpathmoveto{\pgfqpoint{2.682610in}{1.900366in}}%
\pgfpathlineto{\pgfqpoint{3.385245in}{1.297693in}}%
\pgfpathlineto{\pgfqpoint{2.692989in}{3.058384in}}%
\pgfpathclose%
\pgfusepath{stroke,fill}%
\end{pgfscope}%
\begin{pgfscope}%
\pgfpathrectangle{\pgfqpoint{1.000000in}{0.570909in}}{\pgfqpoint{2.818182in}{2.818182in}}%
\pgfusepath{clip}%
\pgfsetbuttcap%
\pgfsetroundjoin%
\definecolor{currentfill}{rgb}{0.100000,0.100000,0.100000}%
\pgfsetfillcolor{currentfill}%
\pgfsetfillopacity{0.100000}%
\pgfsetlinewidth{0.301125pt}%
\definecolor{currentstroke}{rgb}{0.000000,0.000000,0.000000}%
\pgfsetstrokecolor{currentstroke}%
\pgfsetdash{}{0pt}%
\pgfpathmoveto{\pgfqpoint{2.692989in}{3.058384in}}%
\pgfpathlineto{\pgfqpoint{3.385245in}{1.297693in}}%
\pgfpathlineto{\pgfqpoint{1.535168in}{1.562080in}}%
\pgfpathclose%
\pgfusepath{stroke,fill}%
\end{pgfscope}%
\begin{pgfscope}%
\pgfsetbuttcap%
\pgfsetmiterjoin%
\definecolor{currentfill}{rgb}{1.000000,1.000000,1.000000}%
\pgfsetfillcolor{currentfill}%
\pgfsetlinewidth{0.000000pt}%
\definecolor{currentstroke}{rgb}{0.000000,0.000000,0.000000}%
\pgfsetstrokecolor{currentstroke}%
\pgfsetstrokeopacity{0.000000}%
\pgfsetdash{}{0pt}%
\pgfpathmoveto{\pgfqpoint{4.381818in}{0.440000in}}%
\pgfpathlineto{\pgfqpoint{7.200000in}{0.440000in}}%
\pgfpathlineto{\pgfqpoint{7.200000in}{3.520000in}}%
\pgfpathlineto{\pgfqpoint{4.381818in}{3.520000in}}%
\pgfpathclose%
\pgfusepath{fill}%
\end{pgfscope}%
\begin{pgfscope}%
\pgfsetbuttcap%
\pgfsetmiterjoin%
\definecolor{currentfill}{rgb}{0.950000,0.950000,0.950000}%
\pgfsetfillcolor{currentfill}%
\pgfsetfillopacity{0.500000}%
\pgfsetlinewidth{1.003750pt}%
\definecolor{currentstroke}{rgb}{0.950000,0.950000,0.950000}%
\pgfsetstrokecolor{currentstroke}%
\pgfsetstrokeopacity{0.500000}%
\pgfsetdash{}{0pt}%
\pgfpathmoveto{\pgfqpoint{4.755412in}{1.193628in}}%
\pgfpathlineto{\pgfqpoint{5.561990in}{1.884582in}}%
\pgfpathlineto{\pgfqpoint{5.548561in}{3.204883in}}%
\pgfpathlineto{\pgfqpoint{4.697011in}{2.584074in}}%
\pgfusepath{stroke,fill}%
\end{pgfscope}%
\begin{pgfscope}%
\pgfsetbuttcap%
\pgfsetmiterjoin%
\definecolor{currentfill}{rgb}{0.900000,0.900000,0.900000}%
\pgfsetfillcolor{currentfill}%
\pgfsetfillopacity{0.500000}%
\pgfsetlinewidth{1.003750pt}%
\definecolor{currentstroke}{rgb}{0.900000,0.900000,0.900000}%
\pgfsetstrokecolor{currentstroke}%
\pgfsetstrokeopacity{0.500000}%
\pgfsetdash{}{0pt}%
\pgfpathmoveto{\pgfqpoint{5.561990in}{1.884582in}}%
\pgfpathlineto{\pgfqpoint{6.868571in}{1.498294in}}%
\pgfpathlineto{\pgfqpoint{6.923237in}{2.858408in}}%
\pgfpathlineto{\pgfqpoint{5.548561in}{3.204883in}}%
\pgfusepath{stroke,fill}%
\end{pgfscope}%
\begin{pgfscope}%
\pgfsetbuttcap%
\pgfsetmiterjoin%
\definecolor{currentfill}{rgb}{0.925000,0.925000,0.925000}%
\pgfsetfillcolor{currentfill}%
\pgfsetfillopacity{0.500000}%
\pgfsetlinewidth{1.003750pt}%
\definecolor{currentstroke}{rgb}{0.925000,0.925000,0.925000}%
\pgfsetstrokecolor{currentstroke}%
\pgfsetstrokeopacity{0.500000}%
\pgfsetdash{}{0pt}%
\pgfpathmoveto{\pgfqpoint{4.755412in}{1.193628in}}%
\pgfpathlineto{\pgfqpoint{6.130112in}{0.743733in}}%
\pgfpathlineto{\pgfqpoint{6.868571in}{1.498294in}}%
\pgfpathlineto{\pgfqpoint{5.561990in}{1.884582in}}%
\pgfusepath{stroke,fill}%
\end{pgfscope}%
\begin{pgfscope}%
\pgfsetrectcap%
\pgfsetroundjoin%
\pgfsetlinewidth{0.803000pt}%
\definecolor{currentstroke}{rgb}{0.000000,0.000000,0.000000}%
\pgfsetstrokecolor{currentstroke}%
\pgfsetdash{}{0pt}%
\pgfpathmoveto{\pgfqpoint{4.755412in}{1.193628in}}%
\pgfpathlineto{\pgfqpoint{6.130112in}{0.743733in}}%
\pgfusepath{stroke}%
\end{pgfscope}%
\begin{pgfscope}%
\pgfsetbuttcap%
\pgfsetroundjoin%
\pgfsetlinewidth{0.803000pt}%
\definecolor{currentstroke}{rgb}{0.690196,0.690196,0.690196}%
\pgfsetstrokecolor{currentstroke}%
\pgfsetdash{}{0pt}%
\pgfpathmoveto{\pgfqpoint{4.839222in}{1.166199in}}%
\pgfpathlineto{\pgfqpoint{5.641899in}{1.860957in}}%
\pgfpathlineto{\pgfqpoint{5.632463in}{3.183736in}}%
\pgfusepath{stroke}%
\end{pgfscope}%
\begin{pgfscope}%
\pgfsetbuttcap%
\pgfsetroundjoin%
\pgfsetlinewidth{0.803000pt}%
\definecolor{currentstroke}{rgb}{0.690196,0.690196,0.690196}%
\pgfsetstrokecolor{currentstroke}%
\pgfsetdash{}{0pt}%
\pgfpathmoveto{\pgfqpoint{5.071575in}{1.090158in}}%
\pgfpathlineto{\pgfqpoint{5.863264in}{1.795511in}}%
\pgfpathlineto{\pgfqpoint{5.865006in}{3.125126in}}%
\pgfusepath{stroke}%
\end{pgfscope}%
\begin{pgfscope}%
\pgfsetbuttcap%
\pgfsetroundjoin%
\pgfsetlinewidth{0.803000pt}%
\definecolor{currentstroke}{rgb}{0.690196,0.690196,0.690196}%
\pgfsetstrokecolor{currentstroke}%
\pgfsetdash{}{0pt}%
\pgfpathmoveto{\pgfqpoint{5.307621in}{1.012908in}}%
\pgfpathlineto{\pgfqpoint{6.087890in}{1.729101in}}%
\pgfpathlineto{\pgfqpoint{6.101149in}{3.065608in}}%
\pgfusepath{stroke}%
\end{pgfscope}%
\begin{pgfscope}%
\pgfsetbuttcap%
\pgfsetroundjoin%
\pgfsetlinewidth{0.803000pt}%
\definecolor{currentstroke}{rgb}{0.690196,0.690196,0.690196}%
\pgfsetstrokecolor{currentstroke}%
\pgfsetdash{}{0pt}%
\pgfpathmoveto{\pgfqpoint{5.547447in}{0.934420in}}%
\pgfpathlineto{\pgfqpoint{6.315848in}{1.661706in}}%
\pgfpathlineto{\pgfqpoint{6.340977in}{3.005161in}}%
\pgfusepath{stroke}%
\end{pgfscope}%
\begin{pgfscope}%
\pgfsetbuttcap%
\pgfsetroundjoin%
\pgfsetlinewidth{0.803000pt}%
\definecolor{currentstroke}{rgb}{0.690196,0.690196,0.690196}%
\pgfsetstrokecolor{currentstroke}%
\pgfsetdash{}{0pt}%
\pgfpathmoveto{\pgfqpoint{5.791145in}{0.854665in}}%
\pgfpathlineto{\pgfqpoint{6.547215in}{1.593303in}}%
\pgfpathlineto{\pgfqpoint{6.584577in}{2.943764in}}%
\pgfusepath{stroke}%
\end{pgfscope}%
\begin{pgfscope}%
\pgfsetbuttcap%
\pgfsetroundjoin%
\pgfsetlinewidth{0.803000pt}%
\definecolor{currentstroke}{rgb}{0.690196,0.690196,0.690196}%
\pgfsetstrokecolor{currentstroke}%
\pgfsetdash{}{0pt}%
\pgfpathmoveto{\pgfqpoint{6.038811in}{0.773612in}}%
\pgfpathlineto{\pgfqpoint{6.782066in}{1.523869in}}%
\pgfpathlineto{\pgfqpoint{6.832038in}{2.881394in}}%
\pgfusepath{stroke}%
\end{pgfscope}%
\begin{pgfscope}%
\pgfsetrectcap%
\pgfsetroundjoin%
\pgfsetlinewidth{0.803000pt}%
\definecolor{currentstroke}{rgb}{0.000000,0.000000,0.000000}%
\pgfsetstrokecolor{currentstroke}%
\pgfsetdash{}{0pt}%
\pgfpathmoveto{\pgfqpoint{4.846138in}{1.172185in}}%
\pgfpathlineto{\pgfqpoint{4.825366in}{1.154206in}}%
\pgfusepath{stroke}%
\end{pgfscope}%
\begin{pgfscope}%
\definecolor{textcolor}{rgb}{0.000000,0.000000,0.000000}%
\pgfsetstrokecolor{textcolor}%
\pgfsetfillcolor{textcolor}%
\pgftext[x=4.750421in,y=0.940342in,,top]{\color{textcolor}\sffamily\fontsize{10.000000}{12.000000}\selectfont 0.0}%
\end{pgfscope}%
\begin{pgfscope}%
\pgfsetrectcap%
\pgfsetroundjoin%
\pgfsetlinewidth{0.803000pt}%
\definecolor{currentstroke}{rgb}{0.000000,0.000000,0.000000}%
\pgfsetstrokecolor{currentstroke}%
\pgfsetdash{}{0pt}%
\pgfpathmoveto{\pgfqpoint{5.078400in}{1.096238in}}%
\pgfpathlineto{\pgfqpoint{5.057901in}{1.077974in}}%
\pgfusepath{stroke}%
\end{pgfscope}%
\begin{pgfscope}%
\definecolor{textcolor}{rgb}{0.000000,0.000000,0.000000}%
\pgfsetstrokecolor{textcolor}%
\pgfsetfillcolor{textcolor}%
\pgftext[x=4.982733in,y=0.862298in,,top]{\color{textcolor}\sffamily\fontsize{10.000000}{12.000000}\selectfont 0.2}%
\end{pgfscope}%
\begin{pgfscope}%
\pgfsetrectcap%
\pgfsetroundjoin%
\pgfsetlinewidth{0.803000pt}%
\definecolor{currentstroke}{rgb}{0.000000,0.000000,0.000000}%
\pgfsetstrokecolor{currentstroke}%
\pgfsetdash{}{0pt}%
\pgfpathmoveto{\pgfqpoint{5.314351in}{1.019085in}}%
\pgfpathlineto{\pgfqpoint{5.294135in}{1.000529in}}%
\pgfusepath{stroke}%
\end{pgfscope}%
\begin{pgfscope}%
\definecolor{textcolor}{rgb}{0.000000,0.000000,0.000000}%
\pgfsetstrokecolor{textcolor}%
\pgfsetfillcolor{textcolor}%
\pgftext[x=5.218748in,y=0.783011in,,top]{\color{textcolor}\sffamily\fontsize{10.000000}{12.000000}\selectfont 0.4}%
\end{pgfscope}%
\begin{pgfscope}%
\pgfsetrectcap%
\pgfsetroundjoin%
\pgfsetlinewidth{0.803000pt}%
\definecolor{currentstroke}{rgb}{0.000000,0.000000,0.000000}%
\pgfsetstrokecolor{currentstroke}%
\pgfsetdash{}{0pt}%
\pgfpathmoveto{\pgfqpoint{5.554078in}{0.940697in}}%
\pgfpathlineto{\pgfqpoint{5.534158in}{0.921843in}}%
\pgfusepath{stroke}%
\end{pgfscope}%
\begin{pgfscope}%
\definecolor{textcolor}{rgb}{0.000000,0.000000,0.000000}%
\pgfsetstrokecolor{textcolor}%
\pgfsetfillcolor{textcolor}%
\pgftext[x=5.458553in,y=0.702450in,,top]{\color{textcolor}\sffamily\fontsize{10.000000}{12.000000}\selectfont 0.6}%
\end{pgfscope}%
\begin{pgfscope}%
\pgfsetrectcap%
\pgfsetroundjoin%
\pgfsetlinewidth{0.803000pt}%
\definecolor{currentstroke}{rgb}{0.000000,0.000000,0.000000}%
\pgfsetstrokecolor{currentstroke}%
\pgfsetdash{}{0pt}%
\pgfpathmoveto{\pgfqpoint{5.797674in}{0.861044in}}%
\pgfpathlineto{\pgfqpoint{5.778062in}{0.841884in}}%
\pgfusepath{stroke}%
\end{pgfscope}%
\begin{pgfscope}%
\definecolor{textcolor}{rgb}{0.000000,0.000000,0.000000}%
\pgfsetstrokecolor{textcolor}%
\pgfsetfillcolor{textcolor}%
\pgftext[x=5.702242in,y=0.620584in,,top]{\color{textcolor}\sffamily\fontsize{10.000000}{12.000000}\selectfont 0.8}%
\end{pgfscope}%
\begin{pgfscope}%
\pgfsetrectcap%
\pgfsetroundjoin%
\pgfsetlinewidth{0.803000pt}%
\definecolor{currentstroke}{rgb}{0.000000,0.000000,0.000000}%
\pgfsetstrokecolor{currentstroke}%
\pgfsetdash{}{0pt}%
\pgfpathmoveto{\pgfqpoint{6.045233in}{0.780095in}}%
\pgfpathlineto{\pgfqpoint{6.025942in}{0.760622in}}%
\pgfusepath{stroke}%
\end{pgfscope}%
\begin{pgfscope}%
\definecolor{textcolor}{rgb}{0.000000,0.000000,0.000000}%
\pgfsetstrokecolor{textcolor}%
\pgfsetfillcolor{textcolor}%
\pgftext[x=5.949910in,y=0.537382in,,top]{\color{textcolor}\sffamily\fontsize{10.000000}{12.000000}\selectfont 1.0}%
\end{pgfscope}%
\begin{pgfscope}%
\pgfsetrectcap%
\pgfsetroundjoin%
\pgfsetlinewidth{0.803000pt}%
\definecolor{currentstroke}{rgb}{0.000000,0.000000,0.000000}%
\pgfsetstrokecolor{currentstroke}%
\pgfsetdash{}{0pt}%
\pgfpathmoveto{\pgfqpoint{6.868571in}{1.498294in}}%
\pgfpathlineto{\pgfqpoint{6.130112in}{0.743733in}}%
\pgfusepath{stroke}%
\end{pgfscope}%
\begin{pgfscope}%
\pgfsetbuttcap%
\pgfsetroundjoin%
\pgfsetlinewidth{0.803000pt}%
\definecolor{currentstroke}{rgb}{0.690196,0.690196,0.690196}%
\pgfsetstrokecolor{currentstroke}%
\pgfsetdash{}{0pt}%
\pgfpathmoveto{\pgfqpoint{4.755320in}{2.626583in}}%
\pgfpathlineto{\pgfqpoint{4.810441in}{1.240768in}}%
\pgfpathlineto{\pgfqpoint{6.180651in}{0.795374in}}%
\pgfusepath{stroke}%
\end{pgfscope}%
\begin{pgfscope}%
\pgfsetbuttcap%
\pgfsetroundjoin%
\pgfsetlinewidth{0.803000pt}%
\definecolor{currentstroke}{rgb}{0.690196,0.690196,0.690196}%
\pgfsetstrokecolor{currentstroke}%
\pgfsetdash{}{0pt}%
\pgfpathmoveto{\pgfqpoint{4.912202in}{2.740956in}}%
\pgfpathlineto{\pgfqpoint{4.958645in}{1.367728in}}%
\pgfpathlineto{\pgfqpoint{6.316648in}{0.934337in}}%
\pgfusepath{stroke}%
\end{pgfscope}%
\begin{pgfscope}%
\pgfsetbuttcap%
\pgfsetroundjoin%
\pgfsetlinewidth{0.803000pt}%
\definecolor{currentstroke}{rgb}{0.690196,0.690196,0.690196}%
\pgfsetstrokecolor{currentstroke}%
\pgfsetdash{}{0pt}%
\pgfpathmoveto{\pgfqpoint{5.064762in}{2.852177in}}%
\pgfpathlineto{\pgfqpoint{5.102970in}{1.491363in}}%
\pgfpathlineto{\pgfqpoint{6.448924in}{1.069497in}}%
\pgfusepath{stroke}%
\end{pgfscope}%
\begin{pgfscope}%
\pgfsetbuttcap%
\pgfsetroundjoin%
\pgfsetlinewidth{0.803000pt}%
\definecolor{currentstroke}{rgb}{0.690196,0.690196,0.690196}%
\pgfsetstrokecolor{currentstroke}%
\pgfsetdash{}{0pt}%
\pgfpathmoveto{\pgfqpoint{5.213175in}{2.960375in}}%
\pgfpathlineto{\pgfqpoint{5.243565in}{1.611804in}}%
\pgfpathlineto{\pgfqpoint{6.577630in}{1.201009in}}%
\pgfusepath{stroke}%
\end{pgfscope}%
\begin{pgfscope}%
\pgfsetbuttcap%
\pgfsetroundjoin%
\pgfsetlinewidth{0.803000pt}%
\definecolor{currentstroke}{rgb}{0.690196,0.690196,0.690196}%
\pgfsetstrokecolor{currentstroke}%
\pgfsetdash{}{0pt}%
\pgfpathmoveto{\pgfqpoint{5.357608in}{3.065671in}}%
\pgfpathlineto{\pgfqpoint{5.380573in}{1.729172in}}%
\pgfpathlineto{\pgfqpoint{6.702907in}{1.329018in}}%
\pgfusepath{stroke}%
\end{pgfscope}%
\begin{pgfscope}%
\pgfsetbuttcap%
\pgfsetroundjoin%
\pgfsetlinewidth{0.803000pt}%
\definecolor{currentstroke}{rgb}{0.690196,0.690196,0.690196}%
\pgfsetstrokecolor{currentstroke}%
\pgfsetdash{}{0pt}%
\pgfpathmoveto{\pgfqpoint{5.498218in}{3.168181in}}%
\pgfpathlineto{\pgfqpoint{5.514130in}{1.843583in}}%
\pgfpathlineto{\pgfqpoint{6.824892in}{1.453662in}}%
\pgfusepath{stroke}%
\end{pgfscope}%
\begin{pgfscope}%
\pgfsetrectcap%
\pgfsetroundjoin%
\pgfsetlinewidth{0.803000pt}%
\definecolor{currentstroke}{rgb}{0.000000,0.000000,0.000000}%
\pgfsetstrokecolor{currentstroke}%
\pgfsetdash{}{0pt}%
\pgfpathmoveto{\pgfqpoint{6.169183in}{0.799101in}}%
\pgfpathlineto{\pgfqpoint{6.203611in}{0.787910in}}%
\pgfusepath{stroke}%
\end{pgfscope}%
\begin{pgfscope}%
\definecolor{textcolor}{rgb}{0.000000,0.000000,0.000000}%
\pgfsetstrokecolor{textcolor}%
\pgfsetfillcolor{textcolor}%
\pgftext[x=6.334514in,y=0.597696in,,top]{\color{textcolor}\sffamily\fontsize{10.000000}{12.000000}\selectfont 0.0}%
\end{pgfscope}%
\begin{pgfscope}%
\pgfsetrectcap%
\pgfsetroundjoin%
\pgfsetlinewidth{0.803000pt}%
\definecolor{currentstroke}{rgb}{0.000000,0.000000,0.000000}%
\pgfsetstrokecolor{currentstroke}%
\pgfsetdash{}{0pt}%
\pgfpathmoveto{\pgfqpoint{6.305290in}{0.937961in}}%
\pgfpathlineto{\pgfqpoint{6.339390in}{0.927079in}}%
\pgfusepath{stroke}%
\end{pgfscope}%
\begin{pgfscope}%
\definecolor{textcolor}{rgb}{0.000000,0.000000,0.000000}%
\pgfsetstrokecolor{textcolor}%
\pgfsetfillcolor{textcolor}%
\pgftext[x=6.468274in,y=0.739291in,,top]{\color{textcolor}\sffamily\fontsize{10.000000}{12.000000}\selectfont 0.2}%
\end{pgfscope}%
\begin{pgfscope}%
\pgfsetrectcap%
\pgfsetroundjoin%
\pgfsetlinewidth{0.803000pt}%
\definecolor{currentstroke}{rgb}{0.000000,0.000000,0.000000}%
\pgfsetstrokecolor{currentstroke}%
\pgfsetdash{}{0pt}%
\pgfpathmoveto{\pgfqpoint{6.437674in}{1.073023in}}%
\pgfpathlineto{\pgfqpoint{6.471451in}{1.062437in}}%
\pgfusepath{stroke}%
\end{pgfscope}%
\begin{pgfscope}%
\definecolor{textcolor}{rgb}{0.000000,0.000000,0.000000}%
\pgfsetstrokecolor{textcolor}%
\pgfsetfillcolor{textcolor}%
\pgftext[x=6.598377in,y=0.877016in,,top]{\color{textcolor}\sffamily\fontsize{10.000000}{12.000000}\selectfont 0.4}%
\end{pgfscope}%
\begin{pgfscope}%
\pgfsetrectcap%
\pgfsetroundjoin%
\pgfsetlinewidth{0.803000pt}%
\definecolor{currentstroke}{rgb}{0.000000,0.000000,0.000000}%
\pgfsetstrokecolor{currentstroke}%
\pgfsetdash{}{0pt}%
\pgfpathmoveto{\pgfqpoint{6.566485in}{1.204441in}}%
\pgfpathlineto{\pgfqpoint{6.599944in}{1.194138in}}%
\pgfusepath{stroke}%
\end{pgfscope}%
\begin{pgfscope}%
\definecolor{textcolor}{rgb}{0.000000,0.000000,0.000000}%
\pgfsetstrokecolor{textcolor}%
\pgfsetfillcolor{textcolor}%
\pgftext[x=6.724970in,y=1.011025in,,top]{\color{textcolor}\sffamily\fontsize{10.000000}{12.000000}\selectfont 0.6}%
\end{pgfscope}%
\begin{pgfscope}%
\pgfsetrectcap%
\pgfsetroundjoin%
\pgfsetlinewidth{0.803000pt}%
\definecolor{currentstroke}{rgb}{0.000000,0.000000,0.000000}%
\pgfsetstrokecolor{currentstroke}%
\pgfsetdash{}{0pt}%
\pgfpathmoveto{\pgfqpoint{6.691866in}{1.332359in}}%
\pgfpathlineto{\pgfqpoint{6.725012in}{1.322329in}}%
\pgfusepath{stroke}%
\end{pgfscope}%
\begin{pgfscope}%
\definecolor{textcolor}{rgb}{0.000000,0.000000,0.000000}%
\pgfsetstrokecolor{textcolor}%
\pgfsetfillcolor{textcolor}%
\pgftext[x=6.848194in,y=1.141467in,,top]{\color{textcolor}\sffamily\fontsize{10.000000}{12.000000}\selectfont 0.8}%
\end{pgfscope}%
\begin{pgfscope}%
\pgfsetrectcap%
\pgfsetroundjoin%
\pgfsetlinewidth{0.803000pt}%
\definecolor{currentstroke}{rgb}{0.000000,0.000000,0.000000}%
\pgfsetstrokecolor{currentstroke}%
\pgfsetdash{}{0pt}%
\pgfpathmoveto{\pgfqpoint{6.813953in}{1.456916in}}%
\pgfpathlineto{\pgfqpoint{6.846791in}{1.447148in}}%
\pgfusepath{stroke}%
\end{pgfscope}%
\begin{pgfscope}%
\definecolor{textcolor}{rgb}{0.000000,0.000000,0.000000}%
\pgfsetstrokecolor{textcolor}%
\pgfsetfillcolor{textcolor}%
\pgftext[x=6.968181in,y=1.268483in,,top]{\color{textcolor}\sffamily\fontsize{10.000000}{12.000000}\selectfont 1.0}%
\end{pgfscope}%
\begin{pgfscope}%
\pgfsetrectcap%
\pgfsetroundjoin%
\pgfsetlinewidth{0.803000pt}%
\definecolor{currentstroke}{rgb}{0.000000,0.000000,0.000000}%
\pgfsetstrokecolor{currentstroke}%
\pgfsetdash{}{0pt}%
\pgfpathmoveto{\pgfqpoint{6.868571in}{1.498294in}}%
\pgfpathlineto{\pgfqpoint{6.923237in}{2.858408in}}%
\pgfusepath{stroke}%
\end{pgfscope}%
\begin{pgfscope}%
\pgfsetbuttcap%
\pgfsetroundjoin%
\pgfsetlinewidth{0.803000pt}%
\definecolor{currentstroke}{rgb}{0.690196,0.690196,0.690196}%
\pgfsetstrokecolor{currentstroke}%
\pgfsetdash{}{0pt}%
\pgfpathmoveto{\pgfqpoint{6.871887in}{1.580785in}}%
\pgfpathlineto{\pgfqpoint{5.561174in}{1.964823in}}%
\pgfpathlineto{\pgfqpoint{4.751875in}{1.277822in}}%
\pgfusepath{stroke}%
\end{pgfscope}%
\begin{pgfscope}%
\pgfsetbuttcap%
\pgfsetroundjoin%
\pgfsetlinewidth{0.803000pt}%
\definecolor{currentstroke}{rgb}{0.690196,0.690196,0.690196}%
\pgfsetstrokecolor{currentstroke}%
\pgfsetdash{}{0pt}%
\pgfpathmoveto{\pgfqpoint{6.881090in}{1.809771in}}%
\pgfpathlineto{\pgfqpoint{5.558910in}{2.187449in}}%
\pgfpathlineto{\pgfqpoint{4.742055in}{1.511629in}}%
\pgfusepath{stroke}%
\end{pgfscope}%
\begin{pgfscope}%
\pgfsetbuttcap%
\pgfsetroundjoin%
\pgfsetlinewidth{0.803000pt}%
\definecolor{currentstroke}{rgb}{0.690196,0.690196,0.690196}%
\pgfsetstrokecolor{currentstroke}%
\pgfsetdash{}{0pt}%
\pgfpathmoveto{\pgfqpoint{6.890457in}{2.042834in}}%
\pgfpathlineto{\pgfqpoint{5.556607in}{2.413873in}}%
\pgfpathlineto{\pgfqpoint{4.732054in}{1.749737in}}%
\pgfusepath{stroke}%
\end{pgfscope}%
\begin{pgfscope}%
\pgfsetbuttcap%
\pgfsetroundjoin%
\pgfsetlinewidth{0.803000pt}%
\definecolor{currentstroke}{rgb}{0.690196,0.690196,0.690196}%
\pgfsetstrokecolor{currentstroke}%
\pgfsetdash{}{0pt}%
\pgfpathmoveto{\pgfqpoint{6.899993in}{2.280084in}}%
\pgfpathlineto{\pgfqpoint{5.554264in}{2.644191in}}%
\pgfpathlineto{\pgfqpoint{4.721868in}{1.992268in}}%
\pgfusepath{stroke}%
\end{pgfscope}%
\begin{pgfscope}%
\pgfsetbuttcap%
\pgfsetroundjoin%
\pgfsetlinewidth{0.803000pt}%
\definecolor{currentstroke}{rgb}{0.690196,0.690196,0.690196}%
\pgfsetstrokecolor{currentstroke}%
\pgfsetdash{}{0pt}%
\pgfpathmoveto{\pgfqpoint{6.909702in}{2.521636in}}%
\pgfpathlineto{\pgfqpoint{5.551881in}{2.878505in}}%
\pgfpathlineto{\pgfqpoint{4.711490in}{2.239346in}}%
\pgfusepath{stroke}%
\end{pgfscope}%
\begin{pgfscope}%
\pgfsetbuttcap%
\pgfsetroundjoin%
\pgfsetlinewidth{0.803000pt}%
\definecolor{currentstroke}{rgb}{0.690196,0.690196,0.690196}%
\pgfsetstrokecolor{currentstroke}%
\pgfsetdash{}{0pt}%
\pgfpathmoveto{\pgfqpoint{6.919588in}{2.767607in}}%
\pgfpathlineto{\pgfqpoint{5.549456in}{3.116919in}}%
\pgfpathlineto{\pgfqpoint{4.700916in}{2.491100in}}%
\pgfusepath{stroke}%
\end{pgfscope}%
\begin{pgfscope}%
\pgfsetrectcap%
\pgfsetroundjoin%
\pgfsetlinewidth{0.803000pt}%
\definecolor{currentstroke}{rgb}{0.000000,0.000000,0.000000}%
\pgfsetstrokecolor{currentstroke}%
\pgfsetdash{}{0pt}%
\pgfpathmoveto{\pgfqpoint{6.860950in}{1.583990in}}%
\pgfpathlineto{\pgfqpoint{6.893784in}{1.574369in}}%
\pgfusepath{stroke}%
\end{pgfscope}%
\begin{pgfscope}%
\definecolor{textcolor}{rgb}{0.000000,0.000000,0.000000}%
\pgfsetstrokecolor{textcolor}%
\pgfsetfillcolor{textcolor}%
\pgftext[x=7.094481in,y=1.612595in,,top]{\color{textcolor}\sffamily\fontsize{10.000000}{12.000000}\selectfont 0.0}%
\end{pgfscope}%
\begin{pgfscope}%
\pgfsetrectcap%
\pgfsetroundjoin%
\pgfsetlinewidth{0.803000pt}%
\definecolor{currentstroke}{rgb}{0.000000,0.000000,0.000000}%
\pgfsetstrokecolor{currentstroke}%
\pgfsetdash{}{0pt}%
\pgfpathmoveto{\pgfqpoint{6.870053in}{1.812923in}}%
\pgfpathlineto{\pgfqpoint{6.903187in}{1.803459in}}%
\pgfusepath{stroke}%
\end{pgfscope}%
\begin{pgfscope}%
\definecolor{textcolor}{rgb}{0.000000,0.000000,0.000000}%
\pgfsetstrokecolor{textcolor}%
\pgfsetfillcolor{textcolor}%
\pgftext[x=7.105610in,y=1.841065in,,top]{\color{textcolor}\sffamily\fontsize{10.000000}{12.000000}\selectfont 0.2}%
\end{pgfscope}%
\begin{pgfscope}%
\pgfsetrectcap%
\pgfsetroundjoin%
\pgfsetlinewidth{0.803000pt}%
\definecolor{currentstroke}{rgb}{0.000000,0.000000,0.000000}%
\pgfsetstrokecolor{currentstroke}%
\pgfsetdash{}{0pt}%
\pgfpathmoveto{\pgfqpoint{6.879319in}{2.045932in}}%
\pgfpathlineto{\pgfqpoint{6.912758in}{2.036631in}}%
\pgfusepath{stroke}%
\end{pgfscope}%
\begin{pgfscope}%
\definecolor{textcolor}{rgb}{0.000000,0.000000,0.000000}%
\pgfsetstrokecolor{textcolor}%
\pgfsetfillcolor{textcolor}%
\pgftext[x=7.116936in,y=2.073588in,,top]{\color{textcolor}\sffamily\fontsize{10.000000}{12.000000}\selectfont 0.4}%
\end{pgfscope}%
\begin{pgfscope}%
\pgfsetrectcap%
\pgfsetroundjoin%
\pgfsetlinewidth{0.803000pt}%
\definecolor{currentstroke}{rgb}{0.000000,0.000000,0.000000}%
\pgfsetstrokecolor{currentstroke}%
\pgfsetdash{}{0pt}%
\pgfpathmoveto{\pgfqpoint{6.888751in}{2.283126in}}%
\pgfpathlineto{\pgfqpoint{6.922500in}{2.273995in}}%
\pgfusepath{stroke}%
\end{pgfscope}%
\begin{pgfscope}%
\definecolor{textcolor}{rgb}{0.000000,0.000000,0.000000}%
\pgfsetstrokecolor{textcolor}%
\pgfsetfillcolor{textcolor}%
\pgftext[x=7.128466in,y=2.310275in,,top]{\color{textcolor}\sffamily\fontsize{10.000000}{12.000000}\selectfont 0.6}%
\end{pgfscope}%
\begin{pgfscope}%
\pgfsetrectcap%
\pgfsetroundjoin%
\pgfsetlinewidth{0.803000pt}%
\definecolor{currentstroke}{rgb}{0.000000,0.000000,0.000000}%
\pgfsetstrokecolor{currentstroke}%
\pgfsetdash{}{0pt}%
\pgfpathmoveto{\pgfqpoint{6.898354in}{2.524619in}}%
\pgfpathlineto{\pgfqpoint{6.932420in}{2.515665in}}%
\pgfusepath{stroke}%
\end{pgfscope}%
\begin{pgfscope}%
\definecolor{textcolor}{rgb}{0.000000,0.000000,0.000000}%
\pgfsetstrokecolor{textcolor}%
\pgfsetfillcolor{textcolor}%
\pgftext[x=7.140203in,y=2.551237in,,top]{\color{textcolor}\sffamily\fontsize{10.000000}{12.000000}\selectfont 0.8}%
\end{pgfscope}%
\begin{pgfscope}%
\pgfsetrectcap%
\pgfsetroundjoin%
\pgfsetlinewidth{0.803000pt}%
\definecolor{currentstroke}{rgb}{0.000000,0.000000,0.000000}%
\pgfsetstrokecolor{currentstroke}%
\pgfsetdash{}{0pt}%
\pgfpathmoveto{\pgfqpoint{6.908133in}{2.770528in}}%
\pgfpathlineto{\pgfqpoint{6.942521in}{2.761761in}}%
\pgfusepath{stroke}%
\end{pgfscope}%
\begin{pgfscope}%
\definecolor{textcolor}{rgb}{0.000000,0.000000,0.000000}%
\pgfsetstrokecolor{textcolor}%
\pgfsetfillcolor{textcolor}%
\pgftext[x=7.152155in,y=2.796592in,,top]{\color{textcolor}\sffamily\fontsize{10.000000}{12.000000}\selectfont 1.0}%
\end{pgfscope}%
\begin{pgfscope}%
\pgfpathrectangle{\pgfqpoint{4.381818in}{0.570909in}}{\pgfqpoint{2.818182in}{2.818182in}}%
\pgfusepath{clip}%
\pgfsetbuttcap%
\pgfsetroundjoin%
\definecolor{currentfill}{rgb}{0.839216,0.152941,0.156863}%
\pgfsetfillcolor{currentfill}%
\pgfsetlinewidth{1.003750pt}%
\definecolor{currentstroke}{rgb}{0.839216,0.152941,0.156863}%
\pgfsetstrokecolor{currentstroke}%
\pgfsetdash{}{0pt}%
\pgfpathmoveto{\pgfqpoint{5.397979in}{1.719914in}}%
\pgfpathcurveto{\pgfqpoint{5.406215in}{1.719914in}}{\pgfqpoint{5.414115in}{1.723186in}}{\pgfqpoint{5.419939in}{1.729010in}}%
\pgfpathcurveto{\pgfqpoint{5.425763in}{1.734834in}}{\pgfqpoint{5.429036in}{1.742734in}}{\pgfqpoint{5.429036in}{1.750970in}}%
\pgfpathcurveto{\pgfqpoint{5.429036in}{1.759207in}}{\pgfqpoint{5.425763in}{1.767107in}}{\pgfqpoint{5.419939in}{1.772931in}}%
\pgfpathcurveto{\pgfqpoint{5.414115in}{1.778755in}}{\pgfqpoint{5.406215in}{1.782027in}}{\pgfqpoint{5.397979in}{1.782027in}}%
\pgfpathcurveto{\pgfqpoint{5.389743in}{1.782027in}}{\pgfqpoint{5.381843in}{1.778755in}}{\pgfqpoint{5.376019in}{1.772931in}}%
\pgfpathcurveto{\pgfqpoint{5.370195in}{1.767107in}}{\pgfqpoint{5.366923in}{1.759207in}}{\pgfqpoint{5.366923in}{1.750970in}}%
\pgfpathcurveto{\pgfqpoint{5.366923in}{1.742734in}}{\pgfqpoint{5.370195in}{1.734834in}}{\pgfqpoint{5.376019in}{1.729010in}}%
\pgfpathcurveto{\pgfqpoint{5.381843in}{1.723186in}}{\pgfqpoint{5.389743in}{1.719914in}}{\pgfqpoint{5.397979in}{1.719914in}}%
\pgfpathclose%
\pgfusepath{stroke,fill}%
\end{pgfscope}%
\begin{pgfscope}%
\definecolor{textcolor}{rgb}{0.000000,0.000000,0.000000}%
\pgfsetstrokecolor{textcolor}%
\pgfsetfillcolor{textcolor}%
\pgftext[x=5.028047in,y=2.014225in,left,base]{\color{textcolor}\sffamily\fontsize{10.000000}{12.000000}\selectfont 0.04166}%
\end{pgfscope}%
\begin{pgfscope}%
\definecolor{textcolor}{rgb}{0.000000,0.000000,0.000000}%
\pgfsetstrokecolor{textcolor}%
\pgfsetfillcolor{textcolor}%
\pgftext[x=5.352350in,y=1.323909in,left,base]{\color{textcolor}\sffamily\fontsize{10.000000}{12.000000}\selectfont 0.04166}%
\end{pgfscope}%
\begin{pgfscope}%
\definecolor{textcolor}{rgb}{0.000000,0.000000,0.000000}%
\pgfsetstrokecolor{textcolor}%
\pgfsetfillcolor{textcolor}%
\pgftext[x=5.657614in,y=1.543084in,left,base]{\color{textcolor}\sffamily\fontsize{10.000000}{12.000000}\selectfont 0.04166}%
\end{pgfscope}%
\begin{pgfscope}%
\definecolor{textcolor}{rgb}{0.000000,0.000000,0.000000}%
\pgfsetstrokecolor{textcolor}%
\pgfsetfillcolor{textcolor}%
\pgftext[x=5.397979in,y=1.750970in,left,base]{\color{textcolor}\sffamily\fontsize{10.000000}{12.000000}\selectfont 0.04166}%
\end{pgfscope}%
\begin{pgfscope}%
\pgfpathrectangle{\pgfqpoint{4.381818in}{0.570909in}}{\pgfqpoint{2.818182in}{2.818182in}}%
\pgfusepath{clip}%
\pgfsetbuttcap%
\pgfsetroundjoin%
\definecolor{currentfill}{rgb}{1.000000,0.498039,0.054902}%
\pgfsetfillcolor{currentfill}%
\pgfsetlinewidth{1.003750pt}%
\definecolor{currentstroke}{rgb}{1.000000,0.498039,0.054902}%
\pgfsetstrokecolor{currentstroke}%
\pgfsetdash{}{0pt}%
\pgfpathmoveto{\pgfqpoint{5.352350in}{1.292852in}}%
\pgfpathcurveto{\pgfqpoint{5.360586in}{1.292852in}}{\pgfqpoint{5.368486in}{1.296124in}}{\pgfqpoint{5.374310in}{1.301948in}}%
\pgfpathcurveto{\pgfqpoint{5.380134in}{1.307772in}}{\pgfqpoint{5.383406in}{1.315672in}}{\pgfqpoint{5.383406in}{1.323909in}}%
\pgfpathcurveto{\pgfqpoint{5.383406in}{1.332145in}}{\pgfqpoint{5.380134in}{1.340045in}}{\pgfqpoint{5.374310in}{1.345869in}}%
\pgfpathcurveto{\pgfqpoint{5.368486in}{1.351693in}}{\pgfqpoint{5.360586in}{1.354965in}}{\pgfqpoint{5.352350in}{1.354965in}}%
\pgfpathcurveto{\pgfqpoint{5.344114in}{1.354965in}}{\pgfqpoint{5.336213in}{1.351693in}}{\pgfqpoint{5.330390in}{1.345869in}}%
\pgfpathcurveto{\pgfqpoint{5.324566in}{1.340045in}}{\pgfqpoint{5.321293in}{1.332145in}}{\pgfqpoint{5.321293in}{1.323909in}}%
\pgfpathcurveto{\pgfqpoint{5.321293in}{1.315672in}}{\pgfqpoint{5.324566in}{1.307772in}}{\pgfqpoint{5.330390in}{1.301948in}}%
\pgfpathcurveto{\pgfqpoint{5.336213in}{1.296124in}}{\pgfqpoint{5.344114in}{1.292852in}}{\pgfqpoint{5.352350in}{1.292852in}}%
\pgfpathclose%
\pgfusepath{stroke,fill}%
\end{pgfscope}%
\begin{pgfscope}%
\pgfpathrectangle{\pgfqpoint{4.381818in}{0.570909in}}{\pgfqpoint{2.818182in}{2.818182in}}%
\pgfusepath{clip}%
\pgfsetbuttcap%
\pgfsetroundjoin%
\definecolor{currentfill}{rgb}{0.121569,0.466667,0.705882}%
\pgfsetfillcolor{currentfill}%
\pgfsetlinewidth{1.003750pt}%
\definecolor{currentstroke}{rgb}{0.121569,0.466667,0.705882}%
\pgfsetstrokecolor{currentstroke}%
\pgfsetdash{}{0pt}%
\pgfpathmoveto{\pgfqpoint{5.028047in}{1.983169in}}%
\pgfpathcurveto{\pgfqpoint{5.036283in}{1.983169in}}{\pgfqpoint{5.044183in}{1.986441in}}{\pgfqpoint{5.050007in}{1.992265in}}%
\pgfpathcurveto{\pgfqpoint{5.055831in}{1.998089in}}{\pgfqpoint{5.059103in}{2.005989in}}{\pgfqpoint{5.059103in}{2.014225in}}%
\pgfpathcurveto{\pgfqpoint{5.059103in}{2.022461in}}{\pgfqpoint{5.055831in}{2.030362in}}{\pgfqpoint{5.050007in}{2.036185in}}%
\pgfpathcurveto{\pgfqpoint{5.044183in}{2.042009in}}{\pgfqpoint{5.036283in}{2.045282in}}{\pgfqpoint{5.028047in}{2.045282in}}%
\pgfpathcurveto{\pgfqpoint{5.019810in}{2.045282in}}{\pgfqpoint{5.011910in}{2.042009in}}{\pgfqpoint{5.006086in}{2.036185in}}%
\pgfpathcurveto{\pgfqpoint{5.000262in}{2.030362in}}{\pgfqpoint{4.996990in}{2.022461in}}{\pgfqpoint{4.996990in}{2.014225in}}%
\pgfpathcurveto{\pgfqpoint{4.996990in}{2.005989in}}{\pgfqpoint{5.000262in}{1.998089in}}{\pgfqpoint{5.006086in}{1.992265in}}%
\pgfpathcurveto{\pgfqpoint{5.011910in}{1.986441in}}{\pgfqpoint{5.019810in}{1.983169in}}{\pgfqpoint{5.028047in}{1.983169in}}%
\pgfpathclose%
\pgfusepath{stroke,fill}%
\end{pgfscope}%
\begin{pgfscope}%
\pgfpathrectangle{\pgfqpoint{4.381818in}{0.570909in}}{\pgfqpoint{2.818182in}{2.818182in}}%
\pgfusepath{clip}%
\pgfsetbuttcap%
\pgfsetroundjoin%
\definecolor{currentfill}{rgb}{0.172549,0.627451,0.172549}%
\pgfsetfillcolor{currentfill}%
\pgfsetlinewidth{1.003750pt}%
\definecolor{currentstroke}{rgb}{0.172549,0.627451,0.172549}%
\pgfsetstrokecolor{currentstroke}%
\pgfsetdash{}{0pt}%
\pgfpathmoveto{\pgfqpoint{5.657614in}{1.512028in}}%
\pgfpathcurveto{\pgfqpoint{5.665850in}{1.512028in}}{\pgfqpoint{5.673750in}{1.515300in}}{\pgfqpoint{5.679574in}{1.521124in}}%
\pgfpathcurveto{\pgfqpoint{5.685398in}{1.526948in}}{\pgfqpoint{5.688670in}{1.534848in}}{\pgfqpoint{5.688670in}{1.543084in}}%
\pgfpathcurveto{\pgfqpoint{5.688670in}{1.551321in}}{\pgfqpoint{5.685398in}{1.559221in}}{\pgfqpoint{5.679574in}{1.565045in}}%
\pgfpathcurveto{\pgfqpoint{5.673750in}{1.570869in}}{\pgfqpoint{5.665850in}{1.574141in}}{\pgfqpoint{5.657614in}{1.574141in}}%
\pgfpathcurveto{\pgfqpoint{5.649377in}{1.574141in}}{\pgfqpoint{5.641477in}{1.570869in}}{\pgfqpoint{5.635653in}{1.565045in}}%
\pgfpathcurveto{\pgfqpoint{5.629829in}{1.559221in}}{\pgfqpoint{5.626557in}{1.551321in}}{\pgfqpoint{5.626557in}{1.543084in}}%
\pgfpathcurveto{\pgfqpoint{5.626557in}{1.534848in}}{\pgfqpoint{5.629829in}{1.526948in}}{\pgfqpoint{5.635653in}{1.521124in}}%
\pgfpathcurveto{\pgfqpoint{5.641477in}{1.515300in}}{\pgfqpoint{5.649377in}{1.512028in}}{\pgfqpoint{5.657614in}{1.512028in}}%
\pgfpathclose%
\pgfusepath{stroke,fill}%
\end{pgfscope}%
\begin{pgfscope}%
\pgfpathrectangle{\pgfqpoint{4.381818in}{0.570909in}}{\pgfqpoint{2.818182in}{2.818182in}}%
\pgfusepath{clip}%
\pgfsetbuttcap%
\pgfsetroundjoin%
\definecolor{currentfill}{rgb}{0.100000,0.100000,0.100000}%
\pgfsetfillcolor{currentfill}%
\pgfsetfillopacity{0.100000}%
\pgfsetlinewidth{0.301125pt}%
\definecolor{currentstroke}{rgb}{0.000000,0.000000,0.000000}%
\pgfsetstrokecolor{currentstroke}%
\pgfsetdash{}{0pt}%
\pgfpathmoveto{\pgfqpoint{4.890922in}{1.297693in}}%
\pgfpathlineto{\pgfqpoint{5.593557in}{1.900366in}}%
\pgfpathlineto{\pgfqpoint{6.090559in}{0.911119in}}%
\pgfpathclose%
\pgfusepath{stroke,fill}%
\end{pgfscope}%
\begin{pgfscope}%
\pgfpathrectangle{\pgfqpoint{4.381818in}{0.570909in}}{\pgfqpoint{2.818182in}{2.818182in}}%
\pgfusepath{clip}%
\pgfsetbuttcap%
\pgfsetroundjoin%
\definecolor{currentfill}{rgb}{0.100000,0.100000,0.100000}%
\pgfsetfillcolor{currentfill}%
\pgfsetfillopacity{0.100000}%
\pgfsetlinewidth{0.301125pt}%
\definecolor{currentstroke}{rgb}{0.000000,0.000000,0.000000}%
\pgfsetstrokecolor{currentstroke}%
\pgfsetdash{}{0pt}%
\pgfpathmoveto{\pgfqpoint{4.890922in}{1.297693in}}%
\pgfpathlineto{\pgfqpoint{5.593557in}{1.900366in}}%
\pgfpathlineto{\pgfqpoint{4.846638in}{2.509261in}}%
\pgfpathclose%
\pgfusepath{stroke,fill}%
\end{pgfscope}%
\begin{pgfscope}%
\pgfpathrectangle{\pgfqpoint{4.381818in}{0.570909in}}{\pgfqpoint{2.818182in}{2.818182in}}%
\pgfusepath{clip}%
\pgfsetbuttcap%
\pgfsetroundjoin%
\definecolor{currentfill}{rgb}{0.100000,0.100000,0.100000}%
\pgfsetfillcolor{currentfill}%
\pgfsetfillopacity{0.100000}%
\pgfsetlinewidth{0.301125pt}%
\definecolor{currentstroke}{rgb}{0.000000,0.000000,0.000000}%
\pgfsetstrokecolor{currentstroke}%
\pgfsetdash{}{0pt}%
\pgfpathmoveto{\pgfqpoint{4.846638in}{2.509261in}}%
\pgfpathlineto{\pgfqpoint{5.593557in}{1.900366in}}%
\pgfpathlineto{\pgfqpoint{6.090559in}{0.911119in}}%
\pgfpathclose%
\pgfusepath{stroke,fill}%
\end{pgfscope}%
\begin{pgfscope}%
\pgfpathrectangle{\pgfqpoint{4.381818in}{0.570909in}}{\pgfqpoint{2.818182in}{2.818182in}}%
\pgfusepath{clip}%
\pgfsetbuttcap%
\pgfsetroundjoin%
\definecolor{currentfill}{rgb}{0.100000,0.100000,0.100000}%
\pgfsetfillcolor{currentfill}%
\pgfsetfillopacity{0.100000}%
\pgfsetlinewidth{0.301125pt}%
\definecolor{currentstroke}{rgb}{0.000000,0.000000,0.000000}%
\pgfsetstrokecolor{currentstroke}%
\pgfsetdash{}{0pt}%
\pgfpathmoveto{\pgfqpoint{4.890922in}{1.297693in}}%
\pgfpathlineto{\pgfqpoint{6.090559in}{0.911119in}}%
\pgfpathlineto{\pgfqpoint{4.846638in}{2.509261in}}%
\pgfpathclose%
\pgfusepath{stroke,fill}%
\end{pgfscope}%
\end{pgfpicture}%
\makeatother%
\endgroup%
}
\caption{Internal energy $u$ for different temperatures $T$ for the 3-D problem using kinetic Monte Carlo algorithm}
\label{Fig:HeatSmall3D}
\end{figure}

\begin{figure}[htbp]
\centering
\scalebox{0.666}{%% Creator: Matplotlib, PGF backend
%%
%% To include the figure in your LaTeX document, write
%%   \input{<filename>.pgf}
%%
%% Make sure the required packages are loaded in your preamble
%%   \usepackage{pgf}
%%
%% Figures using additional raster images can only be included by \input if
%% they are in the same directory as the main LaTeX file. For loading figures
%% from other directories you can use the `import` package
%%   \usepackage{import}
%% and then include the figures with
%%   \import{<path to file>}{<filename>.pgf}
%%
%% Matplotlib used the following preamble
%%   \usepackage{fontspec}
%%   \setmainfont{DejaVuSerif.ttf}[Path=/home/lzh/anaconda3/envs/numana/lib/python3.7/site-packages/matplotlib/mpl-data/fonts/ttf/]
%%   \setsansfont{DejaVuSans.ttf}[Path=/home/lzh/anaconda3/envs/numana/lib/python3.7/site-packages/matplotlib/mpl-data/fonts/ttf/]
%%   \setmonofont{DejaVuSansMono.ttf}[Path=/home/lzh/anaconda3/envs/numana/lib/python3.7/site-packages/matplotlib/mpl-data/fonts/ttf/]
%%
\begingroup%
\makeatletter%
\begin{pgfpicture}%
\pgfpathrectangle{\pgfpointorigin}{\pgfqpoint{8.000000in}{6.000000in}}%
\pgfusepath{use as bounding box, clip}%
\begin{pgfscope}%
\pgfsetbuttcap%
\pgfsetmiterjoin%
\definecolor{currentfill}{rgb}{1.000000,1.000000,1.000000}%
\pgfsetfillcolor{currentfill}%
\pgfsetlinewidth{0.000000pt}%
\definecolor{currentstroke}{rgb}{1.000000,1.000000,1.000000}%
\pgfsetstrokecolor{currentstroke}%
\pgfsetdash{}{0pt}%
\pgfpathmoveto{\pgfqpoint{0.000000in}{0.000000in}}%
\pgfpathlineto{\pgfqpoint{8.000000in}{0.000000in}}%
\pgfpathlineto{\pgfqpoint{8.000000in}{6.000000in}}%
\pgfpathlineto{\pgfqpoint{0.000000in}{6.000000in}}%
\pgfpathclose%
\pgfusepath{fill}%
\end{pgfscope}%
\begin{pgfscope}%
\pgfsetbuttcap%
\pgfsetmiterjoin%
\definecolor{currentfill}{rgb}{1.000000,1.000000,1.000000}%
\pgfsetfillcolor{currentfill}%
\pgfsetlinewidth{0.000000pt}%
\definecolor{currentstroke}{rgb}{0.000000,0.000000,0.000000}%
\pgfsetstrokecolor{currentstroke}%
\pgfsetstrokeopacity{0.000000}%
\pgfsetdash{}{0pt}%
\pgfpathmoveto{\pgfqpoint{0.469444in}{3.311111in}}%
\pgfpathlineto{\pgfqpoint{3.876389in}{3.311111in}}%
\pgfpathlineto{\pgfqpoint{3.876389in}{5.627778in}}%
\pgfpathlineto{\pgfqpoint{0.469444in}{5.627778in}}%
\pgfpathclose%
\pgfusepath{fill}%
\end{pgfscope}%
\begin{pgfscope}%
\pgfpathrectangle{\pgfqpoint{0.469444in}{3.311111in}}{\pgfqpoint{3.406944in}{2.316667in}}%
\pgfusepath{clip}%
\pgfsetbuttcap%
\pgfsetroundjoin%
\definecolor{currentfill}{rgb}{0.121569,0.466667,0.705882}%
\pgfsetfillcolor{currentfill}%
\pgfsetfillopacity{0.300000}%
\pgfsetlinewidth{1.003750pt}%
\definecolor{currentstroke}{rgb}{0.121569,0.466667,0.705882}%
\pgfsetstrokecolor{currentstroke}%
\pgfsetstrokeopacity{0.300000}%
\pgfsetdash{}{0pt}%
\pgfpathmoveto{\pgfqpoint{0.636088in}{3.610156in}}%
\pgfpathlineto{\pgfqpoint{0.636088in}{3.609783in}}%
\pgfpathlineto{\pgfqpoint{0.712929in}{3.658550in}}%
\pgfpathlineto{\pgfqpoint{0.789771in}{3.710457in}}%
\pgfpathlineto{\pgfqpoint{0.866612in}{3.765426in}}%
\pgfpathlineto{\pgfqpoint{0.943454in}{3.825096in}}%
\pgfpathlineto{\pgfqpoint{1.020295in}{3.887055in}}%
\pgfpathlineto{\pgfqpoint{1.097137in}{3.953694in}}%
\pgfpathlineto{\pgfqpoint{1.173978in}{4.023667in}}%
\pgfpathlineto{\pgfqpoint{1.250819in}{4.099835in}}%
\pgfpathlineto{\pgfqpoint{1.327661in}{4.180591in}}%
\pgfpathlineto{\pgfqpoint{1.404502in}{4.267404in}}%
\pgfpathlineto{\pgfqpoint{1.481344in}{4.359962in}}%
\pgfpathlineto{\pgfqpoint{1.558185in}{4.462407in}}%
\pgfpathlineto{\pgfqpoint{1.635027in}{4.573904in}}%
\pgfpathlineto{\pgfqpoint{1.711868in}{4.702869in}}%
\pgfpathlineto{\pgfqpoint{1.788709in}{4.846102in}}%
\pgfpathlineto{\pgfqpoint{1.865551in}{5.016836in}}%
\pgfpathlineto{\pgfqpoint{1.942392in}{5.256064in}}%
\pgfpathlineto{\pgfqpoint{2.019234in}{5.508532in}}%
\pgfpathlineto{\pgfqpoint{2.096075in}{5.477856in}}%
\pgfpathlineto{\pgfqpoint{2.172917in}{5.096477in}}%
\pgfpathlineto{\pgfqpoint{2.249758in}{4.583898in}}%
\pgfpathlineto{\pgfqpoint{2.326600in}{4.193165in}}%
\pgfpathlineto{\pgfqpoint{2.403441in}{3.955857in}}%
\pgfpathlineto{\pgfqpoint{2.480282in}{3.805642in}}%
\pgfpathlineto{\pgfqpoint{2.557124in}{3.715692in}}%
\pgfpathlineto{\pgfqpoint{2.633965in}{3.653160in}}%
\pgfpathlineto{\pgfqpoint{2.710807in}{3.608795in}}%
\pgfpathlineto{\pgfqpoint{2.787648in}{3.575611in}}%
\pgfpathlineto{\pgfqpoint{2.864490in}{3.551778in}}%
\pgfpathlineto{\pgfqpoint{2.941331in}{3.530173in}}%
\pgfpathlineto{\pgfqpoint{3.018172in}{3.512767in}}%
\pgfpathlineto{\pgfqpoint{3.095014in}{3.498416in}}%
\pgfpathlineto{\pgfqpoint{3.171855in}{3.485508in}}%
\pgfpathlineto{\pgfqpoint{3.248697in}{3.474117in}}%
\pgfpathlineto{\pgfqpoint{3.325538in}{3.464049in}}%
\pgfpathlineto{\pgfqpoint{3.402380in}{3.455666in}}%
\pgfpathlineto{\pgfqpoint{3.479221in}{3.446981in}}%
\pgfpathlineto{\pgfqpoint{3.556063in}{3.439954in}}%
\pgfpathlineto{\pgfqpoint{3.632904in}{3.433387in}}%
\pgfpathlineto{\pgfqpoint{3.709745in}{3.427034in}}%
\pgfpathlineto{\pgfqpoint{3.709745in}{3.427698in}}%
\pgfpathlineto{\pgfqpoint{3.709745in}{3.427698in}}%
\pgfpathlineto{\pgfqpoint{3.632904in}{3.433872in}}%
\pgfpathlineto{\pgfqpoint{3.556063in}{3.440978in}}%
\pgfpathlineto{\pgfqpoint{3.479221in}{3.448583in}}%
\pgfpathlineto{\pgfqpoint{3.402380in}{3.456160in}}%
\pgfpathlineto{\pgfqpoint{3.325538in}{3.464967in}}%
\pgfpathlineto{\pgfqpoint{3.248697in}{3.475409in}}%
\pgfpathlineto{\pgfqpoint{3.171855in}{3.486629in}}%
\pgfpathlineto{\pgfqpoint{3.095014in}{3.499546in}}%
\pgfpathlineto{\pgfqpoint{3.018172in}{3.514373in}}%
\pgfpathlineto{\pgfqpoint{2.941331in}{3.531068in}}%
\pgfpathlineto{\pgfqpoint{2.864490in}{3.553012in}}%
\pgfpathlineto{\pgfqpoint{2.787648in}{3.578905in}}%
\pgfpathlineto{\pgfqpoint{2.710807in}{3.613638in}}%
\pgfpathlineto{\pgfqpoint{2.633965in}{3.656349in}}%
\pgfpathlineto{\pgfqpoint{2.557124in}{3.719629in}}%
\pgfpathlineto{\pgfqpoint{2.480282in}{3.811793in}}%
\pgfpathlineto{\pgfqpoint{2.403441in}{3.965829in}}%
\pgfpathlineto{\pgfqpoint{2.326600in}{4.210293in}}%
\pgfpathlineto{\pgfqpoint{2.249758in}{4.609020in}}%
\pgfpathlineto{\pgfqpoint{2.172917in}{5.111377in}}%
\pgfpathlineto{\pgfqpoint{2.096075in}{5.511263in}}%
\pgfpathlineto{\pgfqpoint{2.019234in}{5.514513in}}%
\pgfpathlineto{\pgfqpoint{1.942392in}{5.272316in}}%
\pgfpathlineto{\pgfqpoint{1.865551in}{5.042453in}}%
\pgfpathlineto{\pgfqpoint{1.788709in}{4.858141in}}%
\pgfpathlineto{\pgfqpoint{1.711868in}{4.705849in}}%
\pgfpathlineto{\pgfqpoint{1.635027in}{4.581248in}}%
\pgfpathlineto{\pgfqpoint{1.558185in}{4.467194in}}%
\pgfpathlineto{\pgfqpoint{1.481344in}{4.363086in}}%
\pgfpathlineto{\pgfqpoint{1.404502in}{4.269772in}}%
\pgfpathlineto{\pgfqpoint{1.327661in}{4.183624in}}%
\pgfpathlineto{\pgfqpoint{1.250819in}{4.102344in}}%
\pgfpathlineto{\pgfqpoint{1.173978in}{4.025920in}}%
\pgfpathlineto{\pgfqpoint{1.097137in}{3.955013in}}%
\pgfpathlineto{\pgfqpoint{1.020295in}{3.888294in}}%
\pgfpathlineto{\pgfqpoint{0.943454in}{3.825524in}}%
\pgfpathlineto{\pgfqpoint{0.866612in}{3.766340in}}%
\pgfpathlineto{\pgfqpoint{0.789771in}{3.711019in}}%
\pgfpathlineto{\pgfqpoint{0.712929in}{3.658956in}}%
\pgfpathlineto{\pgfqpoint{0.636088in}{3.610156in}}%
\pgfpathclose%
\pgfusepath{stroke,fill}%
\end{pgfscope}%
\begin{pgfscope}%
\pgfpathrectangle{\pgfqpoint{0.469444in}{3.311111in}}{\pgfqpoint{3.406944in}{2.316667in}}%
\pgfusepath{clip}%
\pgfsetbuttcap%
\pgfsetroundjoin%
\definecolor{currentfill}{rgb}{0.121569,0.466667,0.705882}%
\pgfsetfillcolor{currentfill}%
\pgfsetlinewidth{1.003750pt}%
\definecolor{currentstroke}{rgb}{0.121569,0.466667,0.705882}%
\pgfsetstrokecolor{currentstroke}%
\pgfsetdash{}{0pt}%
\pgfsys@defobject{currentmarker}{\pgfqpoint{-0.009821in}{-0.009821in}}{\pgfqpoint{0.009821in}{0.009821in}}{%
\pgfpathmoveto{\pgfqpoint{0.000000in}{-0.009821in}}%
\pgfpathcurveto{\pgfqpoint{0.002605in}{-0.009821in}}{\pgfqpoint{0.005103in}{-0.008786in}}{\pgfqpoint{0.006944in}{-0.006944in}}%
\pgfpathcurveto{\pgfqpoint{0.008786in}{-0.005103in}}{\pgfqpoint{0.009821in}{-0.002605in}}{\pgfqpoint{0.009821in}{0.000000in}}%
\pgfpathcurveto{\pgfqpoint{0.009821in}{0.002605in}}{\pgfqpoint{0.008786in}{0.005103in}}{\pgfqpoint{0.006944in}{0.006944in}}%
\pgfpathcurveto{\pgfqpoint{0.005103in}{0.008786in}}{\pgfqpoint{0.002605in}{0.009821in}}{\pgfqpoint{0.000000in}{0.009821in}}%
\pgfpathcurveto{\pgfqpoint{-0.002605in}{0.009821in}}{\pgfqpoint{-0.005103in}{0.008786in}}{\pgfqpoint{-0.006944in}{0.006944in}}%
\pgfpathcurveto{\pgfqpoint{-0.008786in}{0.005103in}}{\pgfqpoint{-0.009821in}{0.002605in}}{\pgfqpoint{-0.009821in}{0.000000in}}%
\pgfpathcurveto{\pgfqpoint{-0.009821in}{-0.002605in}}{\pgfqpoint{-0.008786in}{-0.005103in}}{\pgfqpoint{-0.006944in}{-0.006944in}}%
\pgfpathcurveto{\pgfqpoint{-0.005103in}{-0.008786in}}{\pgfqpoint{-0.002605in}{-0.009821in}}{\pgfqpoint{0.000000in}{-0.009821in}}%
\pgfpathclose%
\pgfusepath{stroke,fill}%
}%
\begin{pgfscope}%
\pgfsys@transformshift{0.636088in}{3.609969in}%
\pgfsys@useobject{currentmarker}{}%
\end{pgfscope}%
\begin{pgfscope}%
\pgfsys@transformshift{0.712929in}{3.658753in}%
\pgfsys@useobject{currentmarker}{}%
\end{pgfscope}%
\begin{pgfscope}%
\pgfsys@transformshift{0.789771in}{3.710738in}%
\pgfsys@useobject{currentmarker}{}%
\end{pgfscope}%
\begin{pgfscope}%
\pgfsys@transformshift{0.866612in}{3.765883in}%
\pgfsys@useobject{currentmarker}{}%
\end{pgfscope}%
\begin{pgfscope}%
\pgfsys@transformshift{0.943454in}{3.825310in}%
\pgfsys@useobject{currentmarker}{}%
\end{pgfscope}%
\begin{pgfscope}%
\pgfsys@transformshift{1.020295in}{3.887675in}%
\pgfsys@useobject{currentmarker}{}%
\end{pgfscope}%
\begin{pgfscope}%
\pgfsys@transformshift{1.097137in}{3.954354in}%
\pgfsys@useobject{currentmarker}{}%
\end{pgfscope}%
\begin{pgfscope}%
\pgfsys@transformshift{1.173978in}{4.024794in}%
\pgfsys@useobject{currentmarker}{}%
\end{pgfscope}%
\begin{pgfscope}%
\pgfsys@transformshift{1.250819in}{4.101090in}%
\pgfsys@useobject{currentmarker}{}%
\end{pgfscope}%
\begin{pgfscope}%
\pgfsys@transformshift{1.327661in}{4.182108in}%
\pgfsys@useobject{currentmarker}{}%
\end{pgfscope}%
\begin{pgfscope}%
\pgfsys@transformshift{1.404502in}{4.268588in}%
\pgfsys@useobject{currentmarker}{}%
\end{pgfscope}%
\begin{pgfscope}%
\pgfsys@transformshift{1.481344in}{4.361524in}%
\pgfsys@useobject{currentmarker}{}%
\end{pgfscope}%
\begin{pgfscope}%
\pgfsys@transformshift{1.558185in}{4.464801in}%
\pgfsys@useobject{currentmarker}{}%
\end{pgfscope}%
\begin{pgfscope}%
\pgfsys@transformshift{1.635027in}{4.577576in}%
\pgfsys@useobject{currentmarker}{}%
\end{pgfscope}%
\begin{pgfscope}%
\pgfsys@transformshift{1.711868in}{4.704359in}%
\pgfsys@useobject{currentmarker}{}%
\end{pgfscope}%
\begin{pgfscope}%
\pgfsys@transformshift{1.788709in}{4.852121in}%
\pgfsys@useobject{currentmarker}{}%
\end{pgfscope}%
\begin{pgfscope}%
\pgfsys@transformshift{1.865551in}{5.029645in}%
\pgfsys@useobject{currentmarker}{}%
\end{pgfscope}%
\begin{pgfscope}%
\pgfsys@transformshift{1.942392in}{5.264190in}%
\pgfsys@useobject{currentmarker}{}%
\end{pgfscope}%
\begin{pgfscope}%
\pgfsys@transformshift{2.019234in}{5.511523in}%
\pgfsys@useobject{currentmarker}{}%
\end{pgfscope}%
\begin{pgfscope}%
\pgfsys@transformshift{2.096075in}{5.494560in}%
\pgfsys@useobject{currentmarker}{}%
\end{pgfscope}%
\begin{pgfscope}%
\pgfsys@transformshift{2.172917in}{5.103927in}%
\pgfsys@useobject{currentmarker}{}%
\end{pgfscope}%
\begin{pgfscope}%
\pgfsys@transformshift{2.249758in}{4.596459in}%
\pgfsys@useobject{currentmarker}{}%
\end{pgfscope}%
\begin{pgfscope}%
\pgfsys@transformshift{2.326600in}{4.201729in}%
\pgfsys@useobject{currentmarker}{}%
\end{pgfscope}%
\begin{pgfscope}%
\pgfsys@transformshift{2.403441in}{3.960843in}%
\pgfsys@useobject{currentmarker}{}%
\end{pgfscope}%
\begin{pgfscope}%
\pgfsys@transformshift{2.480282in}{3.808717in}%
\pgfsys@useobject{currentmarker}{}%
\end{pgfscope}%
\begin{pgfscope}%
\pgfsys@transformshift{2.557124in}{3.717661in}%
\pgfsys@useobject{currentmarker}{}%
\end{pgfscope}%
\begin{pgfscope}%
\pgfsys@transformshift{2.633965in}{3.654754in}%
\pgfsys@useobject{currentmarker}{}%
\end{pgfscope}%
\begin{pgfscope}%
\pgfsys@transformshift{2.710807in}{3.611217in}%
\pgfsys@useobject{currentmarker}{}%
\end{pgfscope}%
\begin{pgfscope}%
\pgfsys@transformshift{2.787648in}{3.577258in}%
\pgfsys@useobject{currentmarker}{}%
\end{pgfscope}%
\begin{pgfscope}%
\pgfsys@transformshift{2.864490in}{3.552395in}%
\pgfsys@useobject{currentmarker}{}%
\end{pgfscope}%
\begin{pgfscope}%
\pgfsys@transformshift{2.941331in}{3.530621in}%
\pgfsys@useobject{currentmarker}{}%
\end{pgfscope}%
\begin{pgfscope}%
\pgfsys@transformshift{3.018172in}{3.513570in}%
\pgfsys@useobject{currentmarker}{}%
\end{pgfscope}%
\begin{pgfscope}%
\pgfsys@transformshift{3.095014in}{3.498981in}%
\pgfsys@useobject{currentmarker}{}%
\end{pgfscope}%
\begin{pgfscope}%
\pgfsys@transformshift{3.171855in}{3.486069in}%
\pgfsys@useobject{currentmarker}{}%
\end{pgfscope}%
\begin{pgfscope}%
\pgfsys@transformshift{3.248697in}{3.474763in}%
\pgfsys@useobject{currentmarker}{}%
\end{pgfscope}%
\begin{pgfscope}%
\pgfsys@transformshift{3.325538in}{3.464508in}%
\pgfsys@useobject{currentmarker}{}%
\end{pgfscope}%
\begin{pgfscope}%
\pgfsys@transformshift{3.402380in}{3.455913in}%
\pgfsys@useobject{currentmarker}{}%
\end{pgfscope}%
\begin{pgfscope}%
\pgfsys@transformshift{3.479221in}{3.447782in}%
\pgfsys@useobject{currentmarker}{}%
\end{pgfscope}%
\begin{pgfscope}%
\pgfsys@transformshift{3.556063in}{3.440466in}%
\pgfsys@useobject{currentmarker}{}%
\end{pgfscope}%
\begin{pgfscope}%
\pgfsys@transformshift{3.632904in}{3.433629in}%
\pgfsys@useobject{currentmarker}{}%
\end{pgfscope}%
\begin{pgfscope}%
\pgfsys@transformshift{3.709745in}{3.427366in}%
\pgfsys@useobject{currentmarker}{}%
\end{pgfscope}%
\end{pgfscope}%
\begin{pgfscope}%
\pgfsetbuttcap%
\pgfsetroundjoin%
\definecolor{currentfill}{rgb}{0.000000,0.000000,0.000000}%
\pgfsetfillcolor{currentfill}%
\pgfsetlinewidth{0.803000pt}%
\definecolor{currentstroke}{rgb}{0.000000,0.000000,0.000000}%
\pgfsetstrokecolor{currentstroke}%
\pgfsetdash{}{0pt}%
\pgfsys@defobject{currentmarker}{\pgfqpoint{0.000000in}{-0.048611in}}{\pgfqpoint{0.000000in}{0.000000in}}{%
\pgfpathmoveto{\pgfqpoint{0.000000in}{0.000000in}}%
\pgfpathlineto{\pgfqpoint{0.000000in}{-0.048611in}}%
\pgfusepath{stroke,fill}%
}%
\begin{pgfscope}%
\pgfsys@transformshift{1.020295in}{3.311111in}%
\pgfsys@useobject{currentmarker}{}%
\end{pgfscope}%
\end{pgfscope}%
\begin{pgfscope}%
\definecolor{textcolor}{rgb}{0.000000,0.000000,0.000000}%
\pgfsetstrokecolor{textcolor}%
\pgfsetfillcolor{textcolor}%
\pgftext[x=1.020295in,y=3.213889in,,top]{\color{textcolor}\sffamily\fontsize{10.000000}{12.000000}\selectfont 3}%
\end{pgfscope}%
\begin{pgfscope}%
\pgfsetbuttcap%
\pgfsetroundjoin%
\definecolor{currentfill}{rgb}{0.000000,0.000000,0.000000}%
\pgfsetfillcolor{currentfill}%
\pgfsetlinewidth{0.803000pt}%
\definecolor{currentstroke}{rgb}{0.000000,0.000000,0.000000}%
\pgfsetstrokecolor{currentstroke}%
\pgfsetdash{}{0pt}%
\pgfsys@defobject{currentmarker}{\pgfqpoint{0.000000in}{-0.048611in}}{\pgfqpoint{0.000000in}{0.000000in}}{%
\pgfpathmoveto{\pgfqpoint{0.000000in}{0.000000in}}%
\pgfpathlineto{\pgfqpoint{0.000000in}{-0.048611in}}%
\pgfusepath{stroke,fill}%
}%
\begin{pgfscope}%
\pgfsys@transformshift{1.788709in}{3.311111in}%
\pgfsys@useobject{currentmarker}{}%
\end{pgfscope}%
\end{pgfscope}%
\begin{pgfscope}%
\definecolor{textcolor}{rgb}{0.000000,0.000000,0.000000}%
\pgfsetstrokecolor{textcolor}%
\pgfsetfillcolor{textcolor}%
\pgftext[x=1.788709in,y=3.213889in,,top]{\color{textcolor}\sffamily\fontsize{10.000000}{12.000000}\selectfont 4}%
\end{pgfscope}%
\begin{pgfscope}%
\pgfsetbuttcap%
\pgfsetroundjoin%
\definecolor{currentfill}{rgb}{0.000000,0.000000,0.000000}%
\pgfsetfillcolor{currentfill}%
\pgfsetlinewidth{0.803000pt}%
\definecolor{currentstroke}{rgb}{0.000000,0.000000,0.000000}%
\pgfsetstrokecolor{currentstroke}%
\pgfsetdash{}{0pt}%
\pgfsys@defobject{currentmarker}{\pgfqpoint{0.000000in}{-0.048611in}}{\pgfqpoint{0.000000in}{0.000000in}}{%
\pgfpathmoveto{\pgfqpoint{0.000000in}{0.000000in}}%
\pgfpathlineto{\pgfqpoint{0.000000in}{-0.048611in}}%
\pgfusepath{stroke,fill}%
}%
\begin{pgfscope}%
\pgfsys@transformshift{2.557124in}{3.311111in}%
\pgfsys@useobject{currentmarker}{}%
\end{pgfscope}%
\end{pgfscope}%
\begin{pgfscope}%
\definecolor{textcolor}{rgb}{0.000000,0.000000,0.000000}%
\pgfsetstrokecolor{textcolor}%
\pgfsetfillcolor{textcolor}%
\pgftext[x=2.557124in,y=3.213889in,,top]{\color{textcolor}\sffamily\fontsize{10.000000}{12.000000}\selectfont 5}%
\end{pgfscope}%
\begin{pgfscope}%
\pgfsetbuttcap%
\pgfsetroundjoin%
\definecolor{currentfill}{rgb}{0.000000,0.000000,0.000000}%
\pgfsetfillcolor{currentfill}%
\pgfsetlinewidth{0.803000pt}%
\definecolor{currentstroke}{rgb}{0.000000,0.000000,0.000000}%
\pgfsetstrokecolor{currentstroke}%
\pgfsetdash{}{0pt}%
\pgfsys@defobject{currentmarker}{\pgfqpoint{0.000000in}{-0.048611in}}{\pgfqpoint{0.000000in}{0.000000in}}{%
\pgfpathmoveto{\pgfqpoint{0.000000in}{0.000000in}}%
\pgfpathlineto{\pgfqpoint{0.000000in}{-0.048611in}}%
\pgfusepath{stroke,fill}%
}%
\begin{pgfscope}%
\pgfsys@transformshift{3.325538in}{3.311111in}%
\pgfsys@useobject{currentmarker}{}%
\end{pgfscope}%
\end{pgfscope}%
\begin{pgfscope}%
\definecolor{textcolor}{rgb}{0.000000,0.000000,0.000000}%
\pgfsetstrokecolor{textcolor}%
\pgfsetfillcolor{textcolor}%
\pgftext[x=3.325538in,y=3.213889in,,top]{\color{textcolor}\sffamily\fontsize{10.000000}{12.000000}\selectfont 6}%
\end{pgfscope}%
\begin{pgfscope}%
\pgfsetbuttcap%
\pgfsetroundjoin%
\definecolor{currentfill}{rgb}{0.000000,0.000000,0.000000}%
\pgfsetfillcolor{currentfill}%
\pgfsetlinewidth{0.803000pt}%
\definecolor{currentstroke}{rgb}{0.000000,0.000000,0.000000}%
\pgfsetstrokecolor{currentstroke}%
\pgfsetdash{}{0pt}%
\pgfsys@defobject{currentmarker}{\pgfqpoint{-0.048611in}{0.000000in}}{\pgfqpoint{0.000000in}{0.000000in}}{%
\pgfpathmoveto{\pgfqpoint{0.000000in}{0.000000in}}%
\pgfpathlineto{\pgfqpoint{-0.048611in}{0.000000in}}%
\pgfusepath{stroke,fill}%
}%
\begin{pgfscope}%
\pgfsys@transformshift{0.469444in}{3.613960in}%
\pgfsys@useobject{currentmarker}{}%
\end{pgfscope}%
\end{pgfscope}%
\begin{pgfscope}%
\definecolor{textcolor}{rgb}{0.000000,0.000000,0.000000}%
\pgfsetstrokecolor{textcolor}%
\pgfsetfillcolor{textcolor}%
\pgftext[x=0.062977in,y=3.561199in,left,base]{\color{textcolor}\sffamily\fontsize{10.000000}{12.000000}\selectfont 0.25}%
\end{pgfscope}%
\begin{pgfscope}%
\pgfsetbuttcap%
\pgfsetroundjoin%
\definecolor{currentfill}{rgb}{0.000000,0.000000,0.000000}%
\pgfsetfillcolor{currentfill}%
\pgfsetlinewidth{0.803000pt}%
\definecolor{currentstroke}{rgb}{0.000000,0.000000,0.000000}%
\pgfsetstrokecolor{currentstroke}%
\pgfsetdash{}{0pt}%
\pgfsys@defobject{currentmarker}{\pgfqpoint{-0.048611in}{0.000000in}}{\pgfqpoint{0.000000in}{0.000000in}}{%
\pgfpathmoveto{\pgfqpoint{0.000000in}{0.000000in}}%
\pgfpathlineto{\pgfqpoint{-0.048611in}{0.000000in}}%
\pgfusepath{stroke,fill}%
}%
\begin{pgfscope}%
\pgfsys@transformshift{0.469444in}{3.927689in}%
\pgfsys@useobject{currentmarker}{}%
\end{pgfscope}%
\end{pgfscope}%
\begin{pgfscope}%
\definecolor{textcolor}{rgb}{0.000000,0.000000,0.000000}%
\pgfsetstrokecolor{textcolor}%
\pgfsetfillcolor{textcolor}%
\pgftext[x=0.062977in,y=3.874927in,left,base]{\color{textcolor}\sffamily\fontsize{10.000000}{12.000000}\selectfont 0.50}%
\end{pgfscope}%
\begin{pgfscope}%
\pgfsetbuttcap%
\pgfsetroundjoin%
\definecolor{currentfill}{rgb}{0.000000,0.000000,0.000000}%
\pgfsetfillcolor{currentfill}%
\pgfsetlinewidth{0.803000pt}%
\definecolor{currentstroke}{rgb}{0.000000,0.000000,0.000000}%
\pgfsetstrokecolor{currentstroke}%
\pgfsetdash{}{0pt}%
\pgfsys@defobject{currentmarker}{\pgfqpoint{-0.048611in}{0.000000in}}{\pgfqpoint{0.000000in}{0.000000in}}{%
\pgfpathmoveto{\pgfqpoint{0.000000in}{0.000000in}}%
\pgfpathlineto{\pgfqpoint{-0.048611in}{0.000000in}}%
\pgfusepath{stroke,fill}%
}%
\begin{pgfscope}%
\pgfsys@transformshift{0.469444in}{4.241417in}%
\pgfsys@useobject{currentmarker}{}%
\end{pgfscope}%
\end{pgfscope}%
\begin{pgfscope}%
\definecolor{textcolor}{rgb}{0.000000,0.000000,0.000000}%
\pgfsetstrokecolor{textcolor}%
\pgfsetfillcolor{textcolor}%
\pgftext[x=0.062977in,y=4.188655in,left,base]{\color{textcolor}\sffamily\fontsize{10.000000}{12.000000}\selectfont 0.75}%
\end{pgfscope}%
\begin{pgfscope}%
\pgfsetbuttcap%
\pgfsetroundjoin%
\definecolor{currentfill}{rgb}{0.000000,0.000000,0.000000}%
\pgfsetfillcolor{currentfill}%
\pgfsetlinewidth{0.803000pt}%
\definecolor{currentstroke}{rgb}{0.000000,0.000000,0.000000}%
\pgfsetstrokecolor{currentstroke}%
\pgfsetdash{}{0pt}%
\pgfsys@defobject{currentmarker}{\pgfqpoint{-0.048611in}{0.000000in}}{\pgfqpoint{0.000000in}{0.000000in}}{%
\pgfpathmoveto{\pgfqpoint{0.000000in}{0.000000in}}%
\pgfpathlineto{\pgfqpoint{-0.048611in}{0.000000in}}%
\pgfusepath{stroke,fill}%
}%
\begin{pgfscope}%
\pgfsys@transformshift{0.469444in}{4.555145in}%
\pgfsys@useobject{currentmarker}{}%
\end{pgfscope}%
\end{pgfscope}%
\begin{pgfscope}%
\definecolor{textcolor}{rgb}{0.000000,0.000000,0.000000}%
\pgfsetstrokecolor{textcolor}%
\pgfsetfillcolor{textcolor}%
\pgftext[x=0.062977in,y=4.502383in,left,base]{\color{textcolor}\sffamily\fontsize{10.000000}{12.000000}\selectfont 1.00}%
\end{pgfscope}%
\begin{pgfscope}%
\pgfsetbuttcap%
\pgfsetroundjoin%
\definecolor{currentfill}{rgb}{0.000000,0.000000,0.000000}%
\pgfsetfillcolor{currentfill}%
\pgfsetlinewidth{0.803000pt}%
\definecolor{currentstroke}{rgb}{0.000000,0.000000,0.000000}%
\pgfsetstrokecolor{currentstroke}%
\pgfsetdash{}{0pt}%
\pgfsys@defobject{currentmarker}{\pgfqpoint{-0.048611in}{0.000000in}}{\pgfqpoint{0.000000in}{0.000000in}}{%
\pgfpathmoveto{\pgfqpoint{0.000000in}{0.000000in}}%
\pgfpathlineto{\pgfqpoint{-0.048611in}{0.000000in}}%
\pgfusepath{stroke,fill}%
}%
\begin{pgfscope}%
\pgfsys@transformshift{0.469444in}{4.868873in}%
\pgfsys@useobject{currentmarker}{}%
\end{pgfscope}%
\end{pgfscope}%
\begin{pgfscope}%
\definecolor{textcolor}{rgb}{0.000000,0.000000,0.000000}%
\pgfsetstrokecolor{textcolor}%
\pgfsetfillcolor{textcolor}%
\pgftext[x=0.062977in,y=4.816112in,left,base]{\color{textcolor}\sffamily\fontsize{10.000000}{12.000000}\selectfont 1.25}%
\end{pgfscope}%
\begin{pgfscope}%
\pgfsetbuttcap%
\pgfsetroundjoin%
\definecolor{currentfill}{rgb}{0.000000,0.000000,0.000000}%
\pgfsetfillcolor{currentfill}%
\pgfsetlinewidth{0.803000pt}%
\definecolor{currentstroke}{rgb}{0.000000,0.000000,0.000000}%
\pgfsetstrokecolor{currentstroke}%
\pgfsetdash{}{0pt}%
\pgfsys@defobject{currentmarker}{\pgfqpoint{-0.048611in}{0.000000in}}{\pgfqpoint{0.000000in}{0.000000in}}{%
\pgfpathmoveto{\pgfqpoint{0.000000in}{0.000000in}}%
\pgfpathlineto{\pgfqpoint{-0.048611in}{0.000000in}}%
\pgfusepath{stroke,fill}%
}%
\begin{pgfscope}%
\pgfsys@transformshift{0.469444in}{5.182601in}%
\pgfsys@useobject{currentmarker}{}%
\end{pgfscope}%
\end{pgfscope}%
\begin{pgfscope}%
\definecolor{textcolor}{rgb}{0.000000,0.000000,0.000000}%
\pgfsetstrokecolor{textcolor}%
\pgfsetfillcolor{textcolor}%
\pgftext[x=0.062977in,y=5.129840in,left,base]{\color{textcolor}\sffamily\fontsize{10.000000}{12.000000}\selectfont 1.50}%
\end{pgfscope}%
\begin{pgfscope}%
\pgfsetbuttcap%
\pgfsetroundjoin%
\definecolor{currentfill}{rgb}{0.000000,0.000000,0.000000}%
\pgfsetfillcolor{currentfill}%
\pgfsetlinewidth{0.803000pt}%
\definecolor{currentstroke}{rgb}{0.000000,0.000000,0.000000}%
\pgfsetstrokecolor{currentstroke}%
\pgfsetdash{}{0pt}%
\pgfsys@defobject{currentmarker}{\pgfqpoint{-0.048611in}{0.000000in}}{\pgfqpoint{0.000000in}{0.000000in}}{%
\pgfpathmoveto{\pgfqpoint{0.000000in}{0.000000in}}%
\pgfpathlineto{\pgfqpoint{-0.048611in}{0.000000in}}%
\pgfusepath{stroke,fill}%
}%
\begin{pgfscope}%
\pgfsys@transformshift{0.469444in}{5.496329in}%
\pgfsys@useobject{currentmarker}{}%
\end{pgfscope}%
\end{pgfscope}%
\begin{pgfscope}%
\definecolor{textcolor}{rgb}{0.000000,0.000000,0.000000}%
\pgfsetstrokecolor{textcolor}%
\pgfsetfillcolor{textcolor}%
\pgftext[x=0.062977in,y=5.443568in,left,base]{\color{textcolor}\sffamily\fontsize{10.000000}{12.000000}\selectfont 1.75}%
\end{pgfscope}%
\begin{pgfscope}%
\pgfpathrectangle{\pgfqpoint{0.469444in}{3.311111in}}{\pgfqpoint{3.406944in}{2.316667in}}%
\pgfusepath{clip}%
\pgfsetrectcap%
\pgfsetroundjoin%
\pgfsetlinewidth{1.505625pt}%
\definecolor{currentstroke}{rgb}{0.121569,0.466667,0.705882}%
\pgfsetstrokecolor{currentstroke}%
\pgfsetdash{}{0pt}%
\pgfpathmoveto{\pgfqpoint{0.636088in}{3.609969in}}%
\pgfpathlineto{\pgfqpoint{0.712929in}{3.658753in}}%
\pgfpathlineto{\pgfqpoint{0.789771in}{3.710738in}}%
\pgfpathlineto{\pgfqpoint{0.866612in}{3.765883in}}%
\pgfpathlineto{\pgfqpoint{0.943454in}{3.825310in}}%
\pgfpathlineto{\pgfqpoint{1.020295in}{3.887675in}}%
\pgfpathlineto{\pgfqpoint{1.097137in}{3.954354in}}%
\pgfpathlineto{\pgfqpoint{1.173978in}{4.024794in}}%
\pgfpathlineto{\pgfqpoint{1.250819in}{4.101090in}}%
\pgfpathlineto{\pgfqpoint{1.327661in}{4.182108in}}%
\pgfpathlineto{\pgfqpoint{1.404502in}{4.268588in}}%
\pgfpathlineto{\pgfqpoint{1.481344in}{4.361524in}}%
\pgfpathlineto{\pgfqpoint{1.558185in}{4.464801in}}%
\pgfpathlineto{\pgfqpoint{1.635027in}{4.577576in}}%
\pgfpathlineto{\pgfqpoint{1.711868in}{4.704359in}}%
\pgfpathlineto{\pgfqpoint{1.788709in}{4.852121in}}%
\pgfpathlineto{\pgfqpoint{1.865551in}{5.029645in}}%
\pgfpathlineto{\pgfqpoint{1.942392in}{5.264190in}}%
\pgfpathlineto{\pgfqpoint{2.019234in}{5.511523in}}%
\pgfpathlineto{\pgfqpoint{2.096075in}{5.494560in}}%
\pgfpathlineto{\pgfqpoint{2.172917in}{5.103927in}}%
\pgfpathlineto{\pgfqpoint{2.249758in}{4.596459in}}%
\pgfpathlineto{\pgfqpoint{2.326600in}{4.201729in}}%
\pgfpathlineto{\pgfqpoint{2.403441in}{3.960843in}}%
\pgfpathlineto{\pgfqpoint{2.480282in}{3.808717in}}%
\pgfpathlineto{\pgfqpoint{2.557124in}{3.717661in}}%
\pgfpathlineto{\pgfqpoint{2.633965in}{3.654754in}}%
\pgfpathlineto{\pgfqpoint{2.710807in}{3.611217in}}%
\pgfpathlineto{\pgfqpoint{2.787648in}{3.577258in}}%
\pgfpathlineto{\pgfqpoint{2.864490in}{3.552395in}}%
\pgfpathlineto{\pgfqpoint{2.941331in}{3.530621in}}%
\pgfpathlineto{\pgfqpoint{3.018172in}{3.513570in}}%
\pgfpathlineto{\pgfqpoint{3.095014in}{3.498981in}}%
\pgfpathlineto{\pgfqpoint{3.171855in}{3.486069in}}%
\pgfpathlineto{\pgfqpoint{3.248697in}{3.474763in}}%
\pgfpathlineto{\pgfqpoint{3.325538in}{3.464508in}}%
\pgfpathlineto{\pgfqpoint{3.402380in}{3.455913in}}%
\pgfpathlineto{\pgfqpoint{3.479221in}{3.447782in}}%
\pgfpathlineto{\pgfqpoint{3.556063in}{3.440466in}}%
\pgfpathlineto{\pgfqpoint{3.632904in}{3.433629in}}%
\pgfpathlineto{\pgfqpoint{3.709745in}{3.427366in}}%
\pgfusepath{stroke}%
\end{pgfscope}%
\begin{pgfscope}%
\pgfsetrectcap%
\pgfsetmiterjoin%
\pgfsetlinewidth{0.803000pt}%
\definecolor{currentstroke}{rgb}{0.000000,0.000000,0.000000}%
\pgfsetstrokecolor{currentstroke}%
\pgfsetdash{}{0pt}%
\pgfpathmoveto{\pgfqpoint{0.469444in}{3.311111in}}%
\pgfpathlineto{\pgfqpoint{0.469444in}{5.627778in}}%
\pgfusepath{stroke}%
\end{pgfscope}%
\begin{pgfscope}%
\pgfsetrectcap%
\pgfsetmiterjoin%
\pgfsetlinewidth{0.803000pt}%
\definecolor{currentstroke}{rgb}{0.000000,0.000000,0.000000}%
\pgfsetstrokecolor{currentstroke}%
\pgfsetdash{}{0pt}%
\pgfpathmoveto{\pgfqpoint{3.876389in}{3.311111in}}%
\pgfpathlineto{\pgfqpoint{3.876389in}{5.627778in}}%
\pgfusepath{stroke}%
\end{pgfscope}%
\begin{pgfscope}%
\pgfsetrectcap%
\pgfsetmiterjoin%
\pgfsetlinewidth{0.803000pt}%
\definecolor{currentstroke}{rgb}{0.000000,0.000000,0.000000}%
\pgfsetstrokecolor{currentstroke}%
\pgfsetdash{}{0pt}%
\pgfpathmoveto{\pgfqpoint{0.469444in}{3.311111in}}%
\pgfpathlineto{\pgfqpoint{3.876389in}{3.311111in}}%
\pgfusepath{stroke}%
\end{pgfscope}%
\begin{pgfscope}%
\pgfsetrectcap%
\pgfsetmiterjoin%
\pgfsetlinewidth{0.803000pt}%
\definecolor{currentstroke}{rgb}{0.000000,0.000000,0.000000}%
\pgfsetstrokecolor{currentstroke}%
\pgfsetdash{}{0pt}%
\pgfpathmoveto{\pgfqpoint{0.469444in}{5.627778in}}%
\pgfpathlineto{\pgfqpoint{3.876389in}{5.627778in}}%
\pgfusepath{stroke}%
\end{pgfscope}%
\begin{pgfscope}%
\definecolor{textcolor}{rgb}{0.000000,0.000000,0.000000}%
\pgfsetstrokecolor{textcolor}%
\pgfsetfillcolor{textcolor}%
\pgftext[x=2.172917in,y=5.711111in,,base]{\color{textcolor}\sffamily\fontsize{12.000000}{14.400000}\selectfont \(\displaystyle  N = 8 \)}%
\end{pgfscope}%
\begin{pgfscope}%
\pgfsetbuttcap%
\pgfsetmiterjoin%
\definecolor{currentfill}{rgb}{1.000000,1.000000,1.000000}%
\pgfsetfillcolor{currentfill}%
\pgfsetlinewidth{0.000000pt}%
\definecolor{currentstroke}{rgb}{0.000000,0.000000,0.000000}%
\pgfsetstrokecolor{currentstroke}%
\pgfsetstrokeopacity{0.000000}%
\pgfsetdash{}{0pt}%
\pgfpathmoveto{\pgfqpoint{4.394444in}{3.311111in}}%
\pgfpathlineto{\pgfqpoint{7.801389in}{3.311111in}}%
\pgfpathlineto{\pgfqpoint{7.801389in}{5.627778in}}%
\pgfpathlineto{\pgfqpoint{4.394444in}{5.627778in}}%
\pgfpathclose%
\pgfusepath{fill}%
\end{pgfscope}%
\begin{pgfscope}%
\pgfpathrectangle{\pgfqpoint{4.394444in}{3.311111in}}{\pgfqpoint{3.406944in}{2.316667in}}%
\pgfusepath{clip}%
\pgfsetbuttcap%
\pgfsetroundjoin%
\definecolor{currentfill}{rgb}{0.121569,0.466667,0.705882}%
\pgfsetfillcolor{currentfill}%
\pgfsetfillopacity{0.300000}%
\pgfsetlinewidth{1.003750pt}%
\definecolor{currentstroke}{rgb}{0.121569,0.466667,0.705882}%
\pgfsetstrokecolor{currentstroke}%
\pgfsetstrokeopacity{0.300000}%
\pgfsetdash{}{0pt}%
\pgfpathmoveto{\pgfqpoint{4.561088in}{3.579531in}}%
\pgfpathlineto{\pgfqpoint{4.561088in}{3.579171in}}%
\pgfpathlineto{\pgfqpoint{4.637929in}{3.617915in}}%
\pgfpathlineto{\pgfqpoint{4.714771in}{3.659261in}}%
\pgfpathlineto{\pgfqpoint{4.791612in}{3.703327in}}%
\pgfpathlineto{\pgfqpoint{4.868454in}{3.750325in}}%
\pgfpathlineto{\pgfqpoint{4.945295in}{3.801436in}}%
\pgfpathlineto{\pgfqpoint{5.022137in}{3.855509in}}%
\pgfpathlineto{\pgfqpoint{5.098978in}{3.912083in}}%
\pgfpathlineto{\pgfqpoint{5.175819in}{3.972931in}}%
\pgfpathlineto{\pgfqpoint{5.252661in}{4.038319in}}%
\pgfpathlineto{\pgfqpoint{5.329502in}{4.107970in}}%
\pgfpathlineto{\pgfqpoint{5.406344in}{4.186771in}}%
\pgfpathlineto{\pgfqpoint{5.483185in}{4.261918in}}%
\pgfpathlineto{\pgfqpoint{5.560027in}{4.359559in}}%
\pgfpathlineto{\pgfqpoint{5.636868in}{4.453322in}}%
\pgfpathlineto{\pgfqpoint{5.713709in}{4.573762in}}%
\pgfpathlineto{\pgfqpoint{5.790551in}{4.694203in}}%
\pgfpathlineto{\pgfqpoint{5.867392in}{4.858976in}}%
\pgfpathlineto{\pgfqpoint{5.944234in}{5.079937in}}%
\pgfpathlineto{\pgfqpoint{6.021075in}{5.434359in}}%
\pgfpathlineto{\pgfqpoint{6.097917in}{5.048524in}}%
\pgfpathlineto{\pgfqpoint{6.174758in}{4.233057in}}%
\pgfpathlineto{\pgfqpoint{6.251600in}{3.890385in}}%
\pgfpathlineto{\pgfqpoint{6.328441in}{3.739865in}}%
\pgfpathlineto{\pgfqpoint{6.405282in}{3.661114in}}%
\pgfpathlineto{\pgfqpoint{6.482124in}{3.614190in}}%
\pgfpathlineto{\pgfqpoint{6.558965in}{3.581810in}}%
\pgfpathlineto{\pgfqpoint{6.635807in}{3.556533in}}%
\pgfpathlineto{\pgfqpoint{6.712648in}{3.535865in}}%
\pgfpathlineto{\pgfqpoint{6.789490in}{3.519538in}}%
\pgfpathlineto{\pgfqpoint{6.866331in}{3.505279in}}%
\pgfpathlineto{\pgfqpoint{6.943172in}{3.493588in}}%
\pgfpathlineto{\pgfqpoint{7.020014in}{3.481956in}}%
\pgfpathlineto{\pgfqpoint{7.096855in}{3.472991in}}%
\pgfpathlineto{\pgfqpoint{7.173697in}{3.464109in}}%
\pgfpathlineto{\pgfqpoint{7.250538in}{3.456059in}}%
\pgfpathlineto{\pgfqpoint{7.327380in}{3.449379in}}%
\pgfpathlineto{\pgfqpoint{7.404221in}{3.443197in}}%
\pgfpathlineto{\pgfqpoint{7.481063in}{3.437359in}}%
\pgfpathlineto{\pgfqpoint{7.557904in}{3.431961in}}%
\pgfpathlineto{\pgfqpoint{7.634745in}{3.427047in}}%
\pgfpathlineto{\pgfqpoint{7.634745in}{3.427769in}}%
\pgfpathlineto{\pgfqpoint{7.634745in}{3.427769in}}%
\pgfpathlineto{\pgfqpoint{7.557904in}{3.433121in}}%
\pgfpathlineto{\pgfqpoint{7.481063in}{3.437843in}}%
\pgfpathlineto{\pgfqpoint{7.404221in}{3.444076in}}%
\pgfpathlineto{\pgfqpoint{7.327380in}{3.450759in}}%
\pgfpathlineto{\pgfqpoint{7.250538in}{3.457172in}}%
\pgfpathlineto{\pgfqpoint{7.173697in}{3.464897in}}%
\pgfpathlineto{\pgfqpoint{7.096855in}{3.474062in}}%
\pgfpathlineto{\pgfqpoint{7.020014in}{3.483713in}}%
\pgfpathlineto{\pgfqpoint{6.943172in}{3.493988in}}%
\pgfpathlineto{\pgfqpoint{6.866331in}{3.507453in}}%
\pgfpathlineto{\pgfqpoint{6.789490in}{3.521533in}}%
\pgfpathlineto{\pgfqpoint{6.712648in}{3.537180in}}%
\pgfpathlineto{\pgfqpoint{6.635807in}{3.557670in}}%
\pgfpathlineto{\pgfqpoint{6.558965in}{3.583764in}}%
\pgfpathlineto{\pgfqpoint{6.482124in}{3.616403in}}%
\pgfpathlineto{\pgfqpoint{6.405282in}{3.665269in}}%
\pgfpathlineto{\pgfqpoint{6.328441in}{3.745149in}}%
\pgfpathlineto{\pgfqpoint{6.251600in}{3.898184in}}%
\pgfpathlineto{\pgfqpoint{6.174758in}{4.290766in}}%
\pgfpathlineto{\pgfqpoint{6.097917in}{5.098754in}}%
\pgfpathlineto{\pgfqpoint{6.021075in}{5.522475in}}%
\pgfpathlineto{\pgfqpoint{5.944234in}{5.150064in}}%
\pgfpathlineto{\pgfqpoint{5.867392in}{4.891632in}}%
\pgfpathlineto{\pgfqpoint{5.790551in}{4.719791in}}%
\pgfpathlineto{\pgfqpoint{5.713709in}{4.582989in}}%
\pgfpathlineto{\pgfqpoint{5.636868in}{4.470141in}}%
\pgfpathlineto{\pgfqpoint{5.560027in}{4.364462in}}%
\pgfpathlineto{\pgfqpoint{5.483185in}{4.273330in}}%
\pgfpathlineto{\pgfqpoint{5.406344in}{4.190383in}}%
\pgfpathlineto{\pgfqpoint{5.329502in}{4.112763in}}%
\pgfpathlineto{\pgfqpoint{5.252661in}{4.039937in}}%
\pgfpathlineto{\pgfqpoint{5.175819in}{3.974892in}}%
\pgfpathlineto{\pgfqpoint{5.098978in}{3.915361in}}%
\pgfpathlineto{\pgfqpoint{5.022137in}{3.855891in}}%
\pgfpathlineto{\pgfqpoint{4.945295in}{3.802471in}}%
\pgfpathlineto{\pgfqpoint{4.868454in}{3.752247in}}%
\pgfpathlineto{\pgfqpoint{4.791612in}{3.704608in}}%
\pgfpathlineto{\pgfqpoint{4.714771in}{3.659974in}}%
\pgfpathlineto{\pgfqpoint{4.637929in}{3.618232in}}%
\pgfpathlineto{\pgfqpoint{4.561088in}{3.579531in}}%
\pgfpathclose%
\pgfusepath{stroke,fill}%
\end{pgfscope}%
\begin{pgfscope}%
\pgfpathrectangle{\pgfqpoint{4.394444in}{3.311111in}}{\pgfqpoint{3.406944in}{2.316667in}}%
\pgfusepath{clip}%
\pgfsetbuttcap%
\pgfsetroundjoin%
\definecolor{currentfill}{rgb}{0.121569,0.466667,0.705882}%
\pgfsetfillcolor{currentfill}%
\pgfsetlinewidth{1.003750pt}%
\definecolor{currentstroke}{rgb}{0.121569,0.466667,0.705882}%
\pgfsetstrokecolor{currentstroke}%
\pgfsetdash{}{0pt}%
\pgfsys@defobject{currentmarker}{\pgfqpoint{-0.009821in}{-0.009821in}}{\pgfqpoint{0.009821in}{0.009821in}}{%
\pgfpathmoveto{\pgfqpoint{0.000000in}{-0.009821in}}%
\pgfpathcurveto{\pgfqpoint{0.002605in}{-0.009821in}}{\pgfqpoint{0.005103in}{-0.008786in}}{\pgfqpoint{0.006944in}{-0.006944in}}%
\pgfpathcurveto{\pgfqpoint{0.008786in}{-0.005103in}}{\pgfqpoint{0.009821in}{-0.002605in}}{\pgfqpoint{0.009821in}{0.000000in}}%
\pgfpathcurveto{\pgfqpoint{0.009821in}{0.002605in}}{\pgfqpoint{0.008786in}{0.005103in}}{\pgfqpoint{0.006944in}{0.006944in}}%
\pgfpathcurveto{\pgfqpoint{0.005103in}{0.008786in}}{\pgfqpoint{0.002605in}{0.009821in}}{\pgfqpoint{0.000000in}{0.009821in}}%
\pgfpathcurveto{\pgfqpoint{-0.002605in}{0.009821in}}{\pgfqpoint{-0.005103in}{0.008786in}}{\pgfqpoint{-0.006944in}{0.006944in}}%
\pgfpathcurveto{\pgfqpoint{-0.008786in}{0.005103in}}{\pgfqpoint{-0.009821in}{0.002605in}}{\pgfqpoint{-0.009821in}{0.000000in}}%
\pgfpathcurveto{\pgfqpoint{-0.009821in}{-0.002605in}}{\pgfqpoint{-0.008786in}{-0.005103in}}{\pgfqpoint{-0.006944in}{-0.006944in}}%
\pgfpathcurveto{\pgfqpoint{-0.005103in}{-0.008786in}}{\pgfqpoint{-0.002605in}{-0.009821in}}{\pgfqpoint{0.000000in}{-0.009821in}}%
\pgfpathclose%
\pgfusepath{stroke,fill}%
}%
\begin{pgfscope}%
\pgfsys@transformshift{4.561088in}{3.579351in}%
\pgfsys@useobject{currentmarker}{}%
\end{pgfscope}%
\begin{pgfscope}%
\pgfsys@transformshift{4.637929in}{3.618073in}%
\pgfsys@useobject{currentmarker}{}%
\end{pgfscope}%
\begin{pgfscope}%
\pgfsys@transformshift{4.714771in}{3.659617in}%
\pgfsys@useobject{currentmarker}{}%
\end{pgfscope}%
\begin{pgfscope}%
\pgfsys@transformshift{4.791612in}{3.703967in}%
\pgfsys@useobject{currentmarker}{}%
\end{pgfscope}%
\begin{pgfscope}%
\pgfsys@transformshift{4.868454in}{3.751286in}%
\pgfsys@useobject{currentmarker}{}%
\end{pgfscope}%
\begin{pgfscope}%
\pgfsys@transformshift{4.945295in}{3.801954in}%
\pgfsys@useobject{currentmarker}{}%
\end{pgfscope}%
\begin{pgfscope}%
\pgfsys@transformshift{5.022137in}{3.855700in}%
\pgfsys@useobject{currentmarker}{}%
\end{pgfscope}%
\begin{pgfscope}%
\pgfsys@transformshift{5.098978in}{3.913722in}%
\pgfsys@useobject{currentmarker}{}%
\end{pgfscope}%
\begin{pgfscope}%
\pgfsys@transformshift{5.175819in}{3.973911in}%
\pgfsys@useobject{currentmarker}{}%
\end{pgfscope}%
\begin{pgfscope}%
\pgfsys@transformshift{5.252661in}{4.039128in}%
\pgfsys@useobject{currentmarker}{}%
\end{pgfscope}%
\begin{pgfscope}%
\pgfsys@transformshift{5.329502in}{4.110366in}%
\pgfsys@useobject{currentmarker}{}%
\end{pgfscope}%
\begin{pgfscope}%
\pgfsys@transformshift{5.406344in}{4.188577in}%
\pgfsys@useobject{currentmarker}{}%
\end{pgfscope}%
\begin{pgfscope}%
\pgfsys@transformshift{5.483185in}{4.267624in}%
\pgfsys@useobject{currentmarker}{}%
\end{pgfscope}%
\begin{pgfscope}%
\pgfsys@transformshift{5.560027in}{4.362011in}%
\pgfsys@useobject{currentmarker}{}%
\end{pgfscope}%
\begin{pgfscope}%
\pgfsys@transformshift{5.636868in}{4.461731in}%
\pgfsys@useobject{currentmarker}{}%
\end{pgfscope}%
\begin{pgfscope}%
\pgfsys@transformshift{5.713709in}{4.578375in}%
\pgfsys@useobject{currentmarker}{}%
\end{pgfscope}%
\begin{pgfscope}%
\pgfsys@transformshift{5.790551in}{4.706997in}%
\pgfsys@useobject{currentmarker}{}%
\end{pgfscope}%
\begin{pgfscope}%
\pgfsys@transformshift{5.867392in}{4.875304in}%
\pgfsys@useobject{currentmarker}{}%
\end{pgfscope}%
\begin{pgfscope}%
\pgfsys@transformshift{5.944234in}{5.115001in}%
\pgfsys@useobject{currentmarker}{}%
\end{pgfscope}%
\begin{pgfscope}%
\pgfsys@transformshift{6.021075in}{5.478417in}%
\pgfsys@useobject{currentmarker}{}%
\end{pgfscope}%
\begin{pgfscope}%
\pgfsys@transformshift{6.097917in}{5.073639in}%
\pgfsys@useobject{currentmarker}{}%
\end{pgfscope}%
\begin{pgfscope}%
\pgfsys@transformshift{6.174758in}{4.261912in}%
\pgfsys@useobject{currentmarker}{}%
\end{pgfscope}%
\begin{pgfscope}%
\pgfsys@transformshift{6.251600in}{3.894284in}%
\pgfsys@useobject{currentmarker}{}%
\end{pgfscope}%
\begin{pgfscope}%
\pgfsys@transformshift{6.328441in}{3.742507in}%
\pgfsys@useobject{currentmarker}{}%
\end{pgfscope}%
\begin{pgfscope}%
\pgfsys@transformshift{6.405282in}{3.663192in}%
\pgfsys@useobject{currentmarker}{}%
\end{pgfscope}%
\begin{pgfscope}%
\pgfsys@transformshift{6.482124in}{3.615297in}%
\pgfsys@useobject{currentmarker}{}%
\end{pgfscope}%
\begin{pgfscope}%
\pgfsys@transformshift{6.558965in}{3.582787in}%
\pgfsys@useobject{currentmarker}{}%
\end{pgfscope}%
\begin{pgfscope}%
\pgfsys@transformshift{6.635807in}{3.557101in}%
\pgfsys@useobject{currentmarker}{}%
\end{pgfscope}%
\begin{pgfscope}%
\pgfsys@transformshift{6.712648in}{3.536523in}%
\pgfsys@useobject{currentmarker}{}%
\end{pgfscope}%
\begin{pgfscope}%
\pgfsys@transformshift{6.789490in}{3.520535in}%
\pgfsys@useobject{currentmarker}{}%
\end{pgfscope}%
\begin{pgfscope}%
\pgfsys@transformshift{6.866331in}{3.506366in}%
\pgfsys@useobject{currentmarker}{}%
\end{pgfscope}%
\begin{pgfscope}%
\pgfsys@transformshift{6.943172in}{3.493788in}%
\pgfsys@useobject{currentmarker}{}%
\end{pgfscope}%
\begin{pgfscope}%
\pgfsys@transformshift{7.020014in}{3.482835in}%
\pgfsys@useobject{currentmarker}{}%
\end{pgfscope}%
\begin{pgfscope}%
\pgfsys@transformshift{7.096855in}{3.473527in}%
\pgfsys@useobject{currentmarker}{}%
\end{pgfscope}%
\begin{pgfscope}%
\pgfsys@transformshift{7.173697in}{3.464503in}%
\pgfsys@useobject{currentmarker}{}%
\end{pgfscope}%
\begin{pgfscope}%
\pgfsys@transformshift{7.250538in}{3.456616in}%
\pgfsys@useobject{currentmarker}{}%
\end{pgfscope}%
\begin{pgfscope}%
\pgfsys@transformshift{7.327380in}{3.450069in}%
\pgfsys@useobject{currentmarker}{}%
\end{pgfscope}%
\begin{pgfscope}%
\pgfsys@transformshift{7.404221in}{3.443636in}%
\pgfsys@useobject{currentmarker}{}%
\end{pgfscope}%
\begin{pgfscope}%
\pgfsys@transformshift{7.481063in}{3.437601in}%
\pgfsys@useobject{currentmarker}{}%
\end{pgfscope}%
\begin{pgfscope}%
\pgfsys@transformshift{7.557904in}{3.432541in}%
\pgfsys@useobject{currentmarker}{}%
\end{pgfscope}%
\begin{pgfscope}%
\pgfsys@transformshift{7.634745in}{3.427408in}%
\pgfsys@useobject{currentmarker}{}%
\end{pgfscope}%
\end{pgfscope}%
\begin{pgfscope}%
\pgfsetbuttcap%
\pgfsetroundjoin%
\definecolor{currentfill}{rgb}{0.000000,0.000000,0.000000}%
\pgfsetfillcolor{currentfill}%
\pgfsetlinewidth{0.803000pt}%
\definecolor{currentstroke}{rgb}{0.000000,0.000000,0.000000}%
\pgfsetstrokecolor{currentstroke}%
\pgfsetdash{}{0pt}%
\pgfsys@defobject{currentmarker}{\pgfqpoint{0.000000in}{-0.048611in}}{\pgfqpoint{0.000000in}{0.000000in}}{%
\pgfpathmoveto{\pgfqpoint{0.000000in}{0.000000in}}%
\pgfpathlineto{\pgfqpoint{0.000000in}{-0.048611in}}%
\pgfusepath{stroke,fill}%
}%
\begin{pgfscope}%
\pgfsys@transformshift{4.945295in}{3.311111in}%
\pgfsys@useobject{currentmarker}{}%
\end{pgfscope}%
\end{pgfscope}%
\begin{pgfscope}%
\definecolor{textcolor}{rgb}{0.000000,0.000000,0.000000}%
\pgfsetstrokecolor{textcolor}%
\pgfsetfillcolor{textcolor}%
\pgftext[x=4.945295in,y=3.213889in,,top]{\color{textcolor}\sffamily\fontsize{10.000000}{12.000000}\selectfont 3}%
\end{pgfscope}%
\begin{pgfscope}%
\pgfsetbuttcap%
\pgfsetroundjoin%
\definecolor{currentfill}{rgb}{0.000000,0.000000,0.000000}%
\pgfsetfillcolor{currentfill}%
\pgfsetlinewidth{0.803000pt}%
\definecolor{currentstroke}{rgb}{0.000000,0.000000,0.000000}%
\pgfsetstrokecolor{currentstroke}%
\pgfsetdash{}{0pt}%
\pgfsys@defobject{currentmarker}{\pgfqpoint{0.000000in}{-0.048611in}}{\pgfqpoint{0.000000in}{0.000000in}}{%
\pgfpathmoveto{\pgfqpoint{0.000000in}{0.000000in}}%
\pgfpathlineto{\pgfqpoint{0.000000in}{-0.048611in}}%
\pgfusepath{stroke,fill}%
}%
\begin{pgfscope}%
\pgfsys@transformshift{5.713709in}{3.311111in}%
\pgfsys@useobject{currentmarker}{}%
\end{pgfscope}%
\end{pgfscope}%
\begin{pgfscope}%
\definecolor{textcolor}{rgb}{0.000000,0.000000,0.000000}%
\pgfsetstrokecolor{textcolor}%
\pgfsetfillcolor{textcolor}%
\pgftext[x=5.713709in,y=3.213889in,,top]{\color{textcolor}\sffamily\fontsize{10.000000}{12.000000}\selectfont 4}%
\end{pgfscope}%
\begin{pgfscope}%
\pgfsetbuttcap%
\pgfsetroundjoin%
\definecolor{currentfill}{rgb}{0.000000,0.000000,0.000000}%
\pgfsetfillcolor{currentfill}%
\pgfsetlinewidth{0.803000pt}%
\definecolor{currentstroke}{rgb}{0.000000,0.000000,0.000000}%
\pgfsetstrokecolor{currentstroke}%
\pgfsetdash{}{0pt}%
\pgfsys@defobject{currentmarker}{\pgfqpoint{0.000000in}{-0.048611in}}{\pgfqpoint{0.000000in}{0.000000in}}{%
\pgfpathmoveto{\pgfqpoint{0.000000in}{0.000000in}}%
\pgfpathlineto{\pgfqpoint{0.000000in}{-0.048611in}}%
\pgfusepath{stroke,fill}%
}%
\begin{pgfscope}%
\pgfsys@transformshift{6.482124in}{3.311111in}%
\pgfsys@useobject{currentmarker}{}%
\end{pgfscope}%
\end{pgfscope}%
\begin{pgfscope}%
\definecolor{textcolor}{rgb}{0.000000,0.000000,0.000000}%
\pgfsetstrokecolor{textcolor}%
\pgfsetfillcolor{textcolor}%
\pgftext[x=6.482124in,y=3.213889in,,top]{\color{textcolor}\sffamily\fontsize{10.000000}{12.000000}\selectfont 5}%
\end{pgfscope}%
\begin{pgfscope}%
\pgfsetbuttcap%
\pgfsetroundjoin%
\definecolor{currentfill}{rgb}{0.000000,0.000000,0.000000}%
\pgfsetfillcolor{currentfill}%
\pgfsetlinewidth{0.803000pt}%
\definecolor{currentstroke}{rgb}{0.000000,0.000000,0.000000}%
\pgfsetstrokecolor{currentstroke}%
\pgfsetdash{}{0pt}%
\pgfsys@defobject{currentmarker}{\pgfqpoint{0.000000in}{-0.048611in}}{\pgfqpoint{0.000000in}{0.000000in}}{%
\pgfpathmoveto{\pgfqpoint{0.000000in}{0.000000in}}%
\pgfpathlineto{\pgfqpoint{0.000000in}{-0.048611in}}%
\pgfusepath{stroke,fill}%
}%
\begin{pgfscope}%
\pgfsys@transformshift{7.250538in}{3.311111in}%
\pgfsys@useobject{currentmarker}{}%
\end{pgfscope}%
\end{pgfscope}%
\begin{pgfscope}%
\definecolor{textcolor}{rgb}{0.000000,0.000000,0.000000}%
\pgfsetstrokecolor{textcolor}%
\pgfsetfillcolor{textcolor}%
\pgftext[x=7.250538in,y=3.213889in,,top]{\color{textcolor}\sffamily\fontsize{10.000000}{12.000000}\selectfont 6}%
\end{pgfscope}%
\begin{pgfscope}%
\pgfsetbuttcap%
\pgfsetroundjoin%
\definecolor{currentfill}{rgb}{0.000000,0.000000,0.000000}%
\pgfsetfillcolor{currentfill}%
\pgfsetlinewidth{0.803000pt}%
\definecolor{currentstroke}{rgb}{0.000000,0.000000,0.000000}%
\pgfsetstrokecolor{currentstroke}%
\pgfsetdash{}{0pt}%
\pgfsys@defobject{currentmarker}{\pgfqpoint{-0.048611in}{0.000000in}}{\pgfqpoint{0.000000in}{0.000000in}}{%
\pgfpathmoveto{\pgfqpoint{0.000000in}{0.000000in}}%
\pgfpathlineto{\pgfqpoint{-0.048611in}{0.000000in}}%
\pgfusepath{stroke,fill}%
}%
\begin{pgfscope}%
\pgfsys@transformshift{4.394444in}{3.324910in}%
\pgfsys@useobject{currentmarker}{}%
\end{pgfscope}%
\end{pgfscope}%
\begin{pgfscope}%
\definecolor{textcolor}{rgb}{0.000000,0.000000,0.000000}%
\pgfsetstrokecolor{textcolor}%
\pgfsetfillcolor{textcolor}%
\pgftext[x=4.076343in,y=3.272148in,left,base]{\color{textcolor}\sffamily\fontsize{10.000000}{12.000000}\selectfont 0.0}%
\end{pgfscope}%
\begin{pgfscope}%
\pgfsetbuttcap%
\pgfsetroundjoin%
\definecolor{currentfill}{rgb}{0.000000,0.000000,0.000000}%
\pgfsetfillcolor{currentfill}%
\pgfsetlinewidth{0.803000pt}%
\definecolor{currentstroke}{rgb}{0.000000,0.000000,0.000000}%
\pgfsetstrokecolor{currentstroke}%
\pgfsetdash{}{0pt}%
\pgfsys@defobject{currentmarker}{\pgfqpoint{-0.048611in}{0.000000in}}{\pgfqpoint{0.000000in}{0.000000in}}{%
\pgfpathmoveto{\pgfqpoint{0.000000in}{0.000000in}}%
\pgfpathlineto{\pgfqpoint{-0.048611in}{0.000000in}}%
\pgfusepath{stroke,fill}%
}%
\begin{pgfscope}%
\pgfsys@transformshift{4.394444in}{3.833185in}%
\pgfsys@useobject{currentmarker}{}%
\end{pgfscope}%
\end{pgfscope}%
\begin{pgfscope}%
\definecolor{textcolor}{rgb}{0.000000,0.000000,0.000000}%
\pgfsetstrokecolor{textcolor}%
\pgfsetfillcolor{textcolor}%
\pgftext[x=4.076343in,y=3.780423in,left,base]{\color{textcolor}\sffamily\fontsize{10.000000}{12.000000}\selectfont 0.5}%
\end{pgfscope}%
\begin{pgfscope}%
\pgfsetbuttcap%
\pgfsetroundjoin%
\definecolor{currentfill}{rgb}{0.000000,0.000000,0.000000}%
\pgfsetfillcolor{currentfill}%
\pgfsetlinewidth{0.803000pt}%
\definecolor{currentstroke}{rgb}{0.000000,0.000000,0.000000}%
\pgfsetstrokecolor{currentstroke}%
\pgfsetdash{}{0pt}%
\pgfsys@defobject{currentmarker}{\pgfqpoint{-0.048611in}{0.000000in}}{\pgfqpoint{0.000000in}{0.000000in}}{%
\pgfpathmoveto{\pgfqpoint{0.000000in}{0.000000in}}%
\pgfpathlineto{\pgfqpoint{-0.048611in}{0.000000in}}%
\pgfusepath{stroke,fill}%
}%
\begin{pgfscope}%
\pgfsys@transformshift{4.394444in}{4.341460in}%
\pgfsys@useobject{currentmarker}{}%
\end{pgfscope}%
\end{pgfscope}%
\begin{pgfscope}%
\definecolor{textcolor}{rgb}{0.000000,0.000000,0.000000}%
\pgfsetstrokecolor{textcolor}%
\pgfsetfillcolor{textcolor}%
\pgftext[x=4.076343in,y=4.288698in,left,base]{\color{textcolor}\sffamily\fontsize{10.000000}{12.000000}\selectfont 1.0}%
\end{pgfscope}%
\begin{pgfscope}%
\pgfsetbuttcap%
\pgfsetroundjoin%
\definecolor{currentfill}{rgb}{0.000000,0.000000,0.000000}%
\pgfsetfillcolor{currentfill}%
\pgfsetlinewidth{0.803000pt}%
\definecolor{currentstroke}{rgb}{0.000000,0.000000,0.000000}%
\pgfsetstrokecolor{currentstroke}%
\pgfsetdash{}{0pt}%
\pgfsys@defobject{currentmarker}{\pgfqpoint{-0.048611in}{0.000000in}}{\pgfqpoint{0.000000in}{0.000000in}}{%
\pgfpathmoveto{\pgfqpoint{0.000000in}{0.000000in}}%
\pgfpathlineto{\pgfqpoint{-0.048611in}{0.000000in}}%
\pgfusepath{stroke,fill}%
}%
\begin{pgfscope}%
\pgfsys@transformshift{4.394444in}{4.849735in}%
\pgfsys@useobject{currentmarker}{}%
\end{pgfscope}%
\end{pgfscope}%
\begin{pgfscope}%
\definecolor{textcolor}{rgb}{0.000000,0.000000,0.000000}%
\pgfsetstrokecolor{textcolor}%
\pgfsetfillcolor{textcolor}%
\pgftext[x=4.076343in,y=4.796973in,left,base]{\color{textcolor}\sffamily\fontsize{10.000000}{12.000000}\selectfont 1.5}%
\end{pgfscope}%
\begin{pgfscope}%
\pgfsetbuttcap%
\pgfsetroundjoin%
\definecolor{currentfill}{rgb}{0.000000,0.000000,0.000000}%
\pgfsetfillcolor{currentfill}%
\pgfsetlinewidth{0.803000pt}%
\definecolor{currentstroke}{rgb}{0.000000,0.000000,0.000000}%
\pgfsetstrokecolor{currentstroke}%
\pgfsetdash{}{0pt}%
\pgfsys@defobject{currentmarker}{\pgfqpoint{-0.048611in}{0.000000in}}{\pgfqpoint{0.000000in}{0.000000in}}{%
\pgfpathmoveto{\pgfqpoint{0.000000in}{0.000000in}}%
\pgfpathlineto{\pgfqpoint{-0.048611in}{0.000000in}}%
\pgfusepath{stroke,fill}%
}%
\begin{pgfscope}%
\pgfsys@transformshift{4.394444in}{5.358010in}%
\pgfsys@useobject{currentmarker}{}%
\end{pgfscope}%
\end{pgfscope}%
\begin{pgfscope}%
\definecolor{textcolor}{rgb}{0.000000,0.000000,0.000000}%
\pgfsetstrokecolor{textcolor}%
\pgfsetfillcolor{textcolor}%
\pgftext[x=4.076343in,y=5.305248in,left,base]{\color{textcolor}\sffamily\fontsize{10.000000}{12.000000}\selectfont 2.0}%
\end{pgfscope}%
\begin{pgfscope}%
\pgfpathrectangle{\pgfqpoint{4.394444in}{3.311111in}}{\pgfqpoint{3.406944in}{2.316667in}}%
\pgfusepath{clip}%
\pgfsetrectcap%
\pgfsetroundjoin%
\pgfsetlinewidth{1.505625pt}%
\definecolor{currentstroke}{rgb}{0.121569,0.466667,0.705882}%
\pgfsetstrokecolor{currentstroke}%
\pgfsetdash{}{0pt}%
\pgfpathmoveto{\pgfqpoint{4.561088in}{3.579351in}}%
\pgfpathlineto{\pgfqpoint{4.637929in}{3.618073in}}%
\pgfpathlineto{\pgfqpoint{4.714771in}{3.659617in}}%
\pgfpathlineto{\pgfqpoint{4.791612in}{3.703967in}}%
\pgfpathlineto{\pgfqpoint{4.868454in}{3.751286in}}%
\pgfpathlineto{\pgfqpoint{4.945295in}{3.801954in}}%
\pgfpathlineto{\pgfqpoint{5.022137in}{3.855700in}}%
\pgfpathlineto{\pgfqpoint{5.098978in}{3.913722in}}%
\pgfpathlineto{\pgfqpoint{5.175819in}{3.973911in}}%
\pgfpathlineto{\pgfqpoint{5.252661in}{4.039128in}}%
\pgfpathlineto{\pgfqpoint{5.329502in}{4.110366in}}%
\pgfpathlineto{\pgfqpoint{5.406344in}{4.188577in}}%
\pgfpathlineto{\pgfqpoint{5.483185in}{4.267624in}}%
\pgfpathlineto{\pgfqpoint{5.560027in}{4.362011in}}%
\pgfpathlineto{\pgfqpoint{5.636868in}{4.461731in}}%
\pgfpathlineto{\pgfqpoint{5.713709in}{4.578375in}}%
\pgfpathlineto{\pgfqpoint{5.790551in}{4.706997in}}%
\pgfpathlineto{\pgfqpoint{5.867392in}{4.875304in}}%
\pgfpathlineto{\pgfqpoint{5.944234in}{5.115001in}}%
\pgfpathlineto{\pgfqpoint{6.021075in}{5.478417in}}%
\pgfpathlineto{\pgfqpoint{6.097917in}{5.073639in}}%
\pgfpathlineto{\pgfqpoint{6.174758in}{4.261912in}}%
\pgfpathlineto{\pgfqpoint{6.251600in}{3.894284in}}%
\pgfpathlineto{\pgfqpoint{6.328441in}{3.742507in}}%
\pgfpathlineto{\pgfqpoint{6.405282in}{3.663192in}}%
\pgfpathlineto{\pgfqpoint{6.482124in}{3.615297in}}%
\pgfpathlineto{\pgfqpoint{6.558965in}{3.582787in}}%
\pgfpathlineto{\pgfqpoint{6.635807in}{3.557101in}}%
\pgfpathlineto{\pgfqpoint{6.712648in}{3.536523in}}%
\pgfpathlineto{\pgfqpoint{6.789490in}{3.520535in}}%
\pgfpathlineto{\pgfqpoint{6.866331in}{3.506366in}}%
\pgfpathlineto{\pgfqpoint{6.943172in}{3.493788in}}%
\pgfpathlineto{\pgfqpoint{7.020014in}{3.482835in}}%
\pgfpathlineto{\pgfqpoint{7.096855in}{3.473527in}}%
\pgfpathlineto{\pgfqpoint{7.173697in}{3.464503in}}%
\pgfpathlineto{\pgfqpoint{7.250538in}{3.456616in}}%
\pgfpathlineto{\pgfqpoint{7.327380in}{3.450069in}}%
\pgfpathlineto{\pgfqpoint{7.404221in}{3.443636in}}%
\pgfpathlineto{\pgfqpoint{7.481063in}{3.437601in}}%
\pgfpathlineto{\pgfqpoint{7.557904in}{3.432541in}}%
\pgfpathlineto{\pgfqpoint{7.634745in}{3.427408in}}%
\pgfusepath{stroke}%
\end{pgfscope}%
\begin{pgfscope}%
\pgfsetrectcap%
\pgfsetmiterjoin%
\pgfsetlinewidth{0.803000pt}%
\definecolor{currentstroke}{rgb}{0.000000,0.000000,0.000000}%
\pgfsetstrokecolor{currentstroke}%
\pgfsetdash{}{0pt}%
\pgfpathmoveto{\pgfqpoint{4.394444in}{3.311111in}}%
\pgfpathlineto{\pgfqpoint{4.394444in}{5.627778in}}%
\pgfusepath{stroke}%
\end{pgfscope}%
\begin{pgfscope}%
\pgfsetrectcap%
\pgfsetmiterjoin%
\pgfsetlinewidth{0.803000pt}%
\definecolor{currentstroke}{rgb}{0.000000,0.000000,0.000000}%
\pgfsetstrokecolor{currentstroke}%
\pgfsetdash{}{0pt}%
\pgfpathmoveto{\pgfqpoint{7.801389in}{3.311111in}}%
\pgfpathlineto{\pgfqpoint{7.801389in}{5.627778in}}%
\pgfusepath{stroke}%
\end{pgfscope}%
\begin{pgfscope}%
\pgfsetrectcap%
\pgfsetmiterjoin%
\pgfsetlinewidth{0.803000pt}%
\definecolor{currentstroke}{rgb}{0.000000,0.000000,0.000000}%
\pgfsetstrokecolor{currentstroke}%
\pgfsetdash{}{0pt}%
\pgfpathmoveto{\pgfqpoint{4.394444in}{3.311111in}}%
\pgfpathlineto{\pgfqpoint{7.801389in}{3.311111in}}%
\pgfusepath{stroke}%
\end{pgfscope}%
\begin{pgfscope}%
\pgfsetrectcap%
\pgfsetmiterjoin%
\pgfsetlinewidth{0.803000pt}%
\definecolor{currentstroke}{rgb}{0.000000,0.000000,0.000000}%
\pgfsetstrokecolor{currentstroke}%
\pgfsetdash{}{0pt}%
\pgfpathmoveto{\pgfqpoint{4.394444in}{5.627778in}}%
\pgfpathlineto{\pgfqpoint{7.801389in}{5.627778in}}%
\pgfusepath{stroke}%
\end{pgfscope}%
\begin{pgfscope}%
\definecolor{textcolor}{rgb}{0.000000,0.000000,0.000000}%
\pgfsetstrokecolor{textcolor}%
\pgfsetfillcolor{textcolor}%
\pgftext[x=6.097917in,y=5.711111in,,base]{\color{textcolor}\sffamily\fontsize{12.000000}{14.400000}\selectfont \(\displaystyle  N = 12 \)}%
\end{pgfscope}%
\begin{pgfscope}%
\pgfsetbuttcap%
\pgfsetmiterjoin%
\definecolor{currentfill}{rgb}{1.000000,1.000000,1.000000}%
\pgfsetfillcolor{currentfill}%
\pgfsetlinewidth{0.000000pt}%
\definecolor{currentstroke}{rgb}{0.000000,0.000000,0.000000}%
\pgfsetstrokecolor{currentstroke}%
\pgfsetstrokeopacity{0.000000}%
\pgfsetdash{}{0pt}%
\pgfpathmoveto{\pgfqpoint{0.469444in}{0.386111in}}%
\pgfpathlineto{\pgfqpoint{3.876389in}{0.386111in}}%
\pgfpathlineto{\pgfqpoint{3.876389in}{2.702778in}}%
\pgfpathlineto{\pgfqpoint{0.469444in}{2.702778in}}%
\pgfpathclose%
\pgfusepath{fill}%
\end{pgfscope}%
\begin{pgfscope}%
\pgfpathrectangle{\pgfqpoint{0.469444in}{0.386111in}}{\pgfqpoint{3.406944in}{2.316667in}}%
\pgfusepath{clip}%
\pgfsetbuttcap%
\pgfsetroundjoin%
\definecolor{currentfill}{rgb}{0.121569,0.466667,0.705882}%
\pgfsetfillcolor{currentfill}%
\pgfsetfillopacity{0.300000}%
\pgfsetlinewidth{1.003750pt}%
\definecolor{currentstroke}{rgb}{0.121569,0.466667,0.705882}%
\pgfsetstrokecolor{currentstroke}%
\pgfsetstrokeopacity{0.300000}%
\pgfsetdash{}{0pt}%
\pgfpathmoveto{\pgfqpoint{0.636088in}{0.652818in}}%
\pgfpathlineto{\pgfqpoint{0.636088in}{0.651804in}}%
\pgfpathlineto{\pgfqpoint{0.712929in}{0.689875in}}%
\pgfpathlineto{\pgfqpoint{0.789771in}{0.730519in}}%
\pgfpathlineto{\pgfqpoint{0.866612in}{0.774091in}}%
\pgfpathlineto{\pgfqpoint{0.943454in}{0.820812in}}%
\pgfpathlineto{\pgfqpoint{1.020295in}{0.869710in}}%
\pgfpathlineto{\pgfqpoint{1.097137in}{0.922196in}}%
\pgfpathlineto{\pgfqpoint{1.173978in}{0.979022in}}%
\pgfpathlineto{\pgfqpoint{1.250819in}{1.039276in}}%
\pgfpathlineto{\pgfqpoint{1.327661in}{1.104463in}}%
\pgfpathlineto{\pgfqpoint{1.404502in}{1.174790in}}%
\pgfpathlineto{\pgfqpoint{1.481344in}{1.247654in}}%
\pgfpathlineto{\pgfqpoint{1.558185in}{1.329336in}}%
\pgfpathlineto{\pgfqpoint{1.635027in}{1.414523in}}%
\pgfpathlineto{\pgfqpoint{1.711868in}{1.506634in}}%
\pgfpathlineto{\pgfqpoint{1.788709in}{1.620823in}}%
\pgfpathlineto{\pgfqpoint{1.865551in}{1.744064in}}%
\pgfpathlineto{\pgfqpoint{1.942392in}{1.921312in}}%
\pgfpathlineto{\pgfqpoint{2.019234in}{2.120245in}}%
\pgfpathlineto{\pgfqpoint{2.096075in}{2.432787in}}%
\pgfpathlineto{\pgfqpoint{2.172917in}{2.379188in}}%
\pgfpathlineto{\pgfqpoint{2.249758in}{1.193585in}}%
\pgfpathlineto{\pgfqpoint{2.326600in}{0.868660in}}%
\pgfpathlineto{\pgfqpoint{2.403441in}{0.780692in}}%
\pgfpathlineto{\pgfqpoint{2.480282in}{0.720258in}}%
\pgfpathlineto{\pgfqpoint{2.557124in}{0.679531in}}%
\pgfpathlineto{\pgfqpoint{2.633965in}{0.652199in}}%
\pgfpathlineto{\pgfqpoint{2.710807in}{0.629569in}}%
\pgfpathlineto{\pgfqpoint{2.787648in}{0.609474in}}%
\pgfpathlineto{\pgfqpoint{2.864490in}{0.592993in}}%
\pgfpathlineto{\pgfqpoint{2.941331in}{0.579617in}}%
\pgfpathlineto{\pgfqpoint{3.018172in}{0.565933in}}%
\pgfpathlineto{\pgfqpoint{3.095014in}{0.554935in}}%
\pgfpathlineto{\pgfqpoint{3.171855in}{0.546983in}}%
\pgfpathlineto{\pgfqpoint{3.248697in}{0.538104in}}%
\pgfpathlineto{\pgfqpoint{3.325538in}{0.530460in}}%
\pgfpathlineto{\pgfqpoint{3.402380in}{0.523919in}}%
\pgfpathlineto{\pgfqpoint{3.479221in}{0.518140in}}%
\pgfpathlineto{\pgfqpoint{3.556063in}{0.512056in}}%
\pgfpathlineto{\pgfqpoint{3.632904in}{0.506877in}}%
\pgfpathlineto{\pgfqpoint{3.709745in}{0.502151in}}%
\pgfpathlineto{\pgfqpoint{3.709745in}{0.502664in}}%
\pgfpathlineto{\pgfqpoint{3.709745in}{0.502664in}}%
\pgfpathlineto{\pgfqpoint{3.632904in}{0.507393in}}%
\pgfpathlineto{\pgfqpoint{3.556063in}{0.513185in}}%
\pgfpathlineto{\pgfqpoint{3.479221in}{0.519248in}}%
\pgfpathlineto{\pgfqpoint{3.402380in}{0.526164in}}%
\pgfpathlineto{\pgfqpoint{3.325538in}{0.531661in}}%
\pgfpathlineto{\pgfqpoint{3.248697in}{0.538906in}}%
\pgfpathlineto{\pgfqpoint{3.171855in}{0.548312in}}%
\pgfpathlineto{\pgfqpoint{3.095014in}{0.557950in}}%
\pgfpathlineto{\pgfqpoint{3.018172in}{0.570064in}}%
\pgfpathlineto{\pgfqpoint{2.941331in}{0.580245in}}%
\pgfpathlineto{\pgfqpoint{2.864490in}{0.594640in}}%
\pgfpathlineto{\pgfqpoint{2.787648in}{0.610705in}}%
\pgfpathlineto{\pgfqpoint{2.710807in}{0.631353in}}%
\pgfpathlineto{\pgfqpoint{2.633965in}{0.656411in}}%
\pgfpathlineto{\pgfqpoint{2.557124in}{0.685704in}}%
\pgfpathlineto{\pgfqpoint{2.480282in}{0.725576in}}%
\pgfpathlineto{\pgfqpoint{2.403441in}{0.788436in}}%
\pgfpathlineto{\pgfqpoint{2.326600in}{0.897026in}}%
\pgfpathlineto{\pgfqpoint{2.249758in}{1.226604in}}%
\pgfpathlineto{\pgfqpoint{2.172917in}{2.434224in}}%
\pgfpathlineto{\pgfqpoint{2.096075in}{2.597475in}}%
\pgfpathlineto{\pgfqpoint{2.019234in}{2.141581in}}%
\pgfpathlineto{\pgfqpoint{1.942392in}{1.958602in}}%
\pgfpathlineto{\pgfqpoint{1.865551in}{1.775694in}}%
\pgfpathlineto{\pgfqpoint{1.788709in}{1.641750in}}%
\pgfpathlineto{\pgfqpoint{1.711868in}{1.525065in}}%
\pgfpathlineto{\pgfqpoint{1.635027in}{1.425411in}}%
\pgfpathlineto{\pgfqpoint{1.558185in}{1.333277in}}%
\pgfpathlineto{\pgfqpoint{1.481344in}{1.252804in}}%
\pgfpathlineto{\pgfqpoint{1.404502in}{1.179008in}}%
\pgfpathlineto{\pgfqpoint{1.327661in}{1.109163in}}%
\pgfpathlineto{\pgfqpoint{1.250819in}{1.041332in}}%
\pgfpathlineto{\pgfqpoint{1.173978in}{0.982745in}}%
\pgfpathlineto{\pgfqpoint{1.097137in}{0.923009in}}%
\pgfpathlineto{\pgfqpoint{1.020295in}{0.872192in}}%
\pgfpathlineto{\pgfqpoint{0.943454in}{0.822713in}}%
\pgfpathlineto{\pgfqpoint{0.866612in}{0.775600in}}%
\pgfpathlineto{\pgfqpoint{0.789771in}{0.731630in}}%
\pgfpathlineto{\pgfqpoint{0.712929in}{0.690447in}}%
\pgfpathlineto{\pgfqpoint{0.636088in}{0.652818in}}%
\pgfpathclose%
\pgfusepath{stroke,fill}%
\end{pgfscope}%
\begin{pgfscope}%
\pgfpathrectangle{\pgfqpoint{0.469444in}{0.386111in}}{\pgfqpoint{3.406944in}{2.316667in}}%
\pgfusepath{clip}%
\pgfsetbuttcap%
\pgfsetroundjoin%
\definecolor{currentfill}{rgb}{0.121569,0.466667,0.705882}%
\pgfsetfillcolor{currentfill}%
\pgfsetlinewidth{1.003750pt}%
\definecolor{currentstroke}{rgb}{0.121569,0.466667,0.705882}%
\pgfsetstrokecolor{currentstroke}%
\pgfsetdash{}{0pt}%
\pgfsys@defobject{currentmarker}{\pgfqpoint{-0.009821in}{-0.009821in}}{\pgfqpoint{0.009821in}{0.009821in}}{%
\pgfpathmoveto{\pgfqpoint{0.000000in}{-0.009821in}}%
\pgfpathcurveto{\pgfqpoint{0.002605in}{-0.009821in}}{\pgfqpoint{0.005103in}{-0.008786in}}{\pgfqpoint{0.006944in}{-0.006944in}}%
\pgfpathcurveto{\pgfqpoint{0.008786in}{-0.005103in}}{\pgfqpoint{0.009821in}{-0.002605in}}{\pgfqpoint{0.009821in}{0.000000in}}%
\pgfpathcurveto{\pgfqpoint{0.009821in}{0.002605in}}{\pgfqpoint{0.008786in}{0.005103in}}{\pgfqpoint{0.006944in}{0.006944in}}%
\pgfpathcurveto{\pgfqpoint{0.005103in}{0.008786in}}{\pgfqpoint{0.002605in}{0.009821in}}{\pgfqpoint{0.000000in}{0.009821in}}%
\pgfpathcurveto{\pgfqpoint{-0.002605in}{0.009821in}}{\pgfqpoint{-0.005103in}{0.008786in}}{\pgfqpoint{-0.006944in}{0.006944in}}%
\pgfpathcurveto{\pgfqpoint{-0.008786in}{0.005103in}}{\pgfqpoint{-0.009821in}{0.002605in}}{\pgfqpoint{-0.009821in}{0.000000in}}%
\pgfpathcurveto{\pgfqpoint{-0.009821in}{-0.002605in}}{\pgfqpoint{-0.008786in}{-0.005103in}}{\pgfqpoint{-0.006944in}{-0.006944in}}%
\pgfpathcurveto{\pgfqpoint{-0.005103in}{-0.008786in}}{\pgfqpoint{-0.002605in}{-0.009821in}}{\pgfqpoint{0.000000in}{-0.009821in}}%
\pgfpathclose%
\pgfusepath{stroke,fill}%
}%
\begin{pgfscope}%
\pgfsys@transformshift{0.636088in}{0.652311in}%
\pgfsys@useobject{currentmarker}{}%
\end{pgfscope}%
\begin{pgfscope}%
\pgfsys@transformshift{0.712929in}{0.690161in}%
\pgfsys@useobject{currentmarker}{}%
\end{pgfscope}%
\begin{pgfscope}%
\pgfsys@transformshift{0.789771in}{0.731075in}%
\pgfsys@useobject{currentmarker}{}%
\end{pgfscope}%
\begin{pgfscope}%
\pgfsys@transformshift{0.866612in}{0.774846in}%
\pgfsys@useobject{currentmarker}{}%
\end{pgfscope}%
\begin{pgfscope}%
\pgfsys@transformshift{0.943454in}{0.821763in}%
\pgfsys@useobject{currentmarker}{}%
\end{pgfscope}%
\begin{pgfscope}%
\pgfsys@transformshift{1.020295in}{0.870951in}%
\pgfsys@useobject{currentmarker}{}%
\end{pgfscope}%
\begin{pgfscope}%
\pgfsys@transformshift{1.097137in}{0.922603in}%
\pgfsys@useobject{currentmarker}{}%
\end{pgfscope}%
\begin{pgfscope}%
\pgfsys@transformshift{1.173978in}{0.980884in}%
\pgfsys@useobject{currentmarker}{}%
\end{pgfscope}%
\begin{pgfscope}%
\pgfsys@transformshift{1.250819in}{1.040304in}%
\pgfsys@useobject{currentmarker}{}%
\end{pgfscope}%
\begin{pgfscope}%
\pgfsys@transformshift{1.327661in}{1.106813in}%
\pgfsys@useobject{currentmarker}{}%
\end{pgfscope}%
\begin{pgfscope}%
\pgfsys@transformshift{1.404502in}{1.176899in}%
\pgfsys@useobject{currentmarker}{}%
\end{pgfscope}%
\begin{pgfscope}%
\pgfsys@transformshift{1.481344in}{1.250229in}%
\pgfsys@useobject{currentmarker}{}%
\end{pgfscope}%
\begin{pgfscope}%
\pgfsys@transformshift{1.558185in}{1.331307in}%
\pgfsys@useobject{currentmarker}{}%
\end{pgfscope}%
\begin{pgfscope}%
\pgfsys@transformshift{1.635027in}{1.419967in}%
\pgfsys@useobject{currentmarker}{}%
\end{pgfscope}%
\begin{pgfscope}%
\pgfsys@transformshift{1.711868in}{1.515850in}%
\pgfsys@useobject{currentmarker}{}%
\end{pgfscope}%
\begin{pgfscope}%
\pgfsys@transformshift{1.788709in}{1.631287in}%
\pgfsys@useobject{currentmarker}{}%
\end{pgfscope}%
\begin{pgfscope}%
\pgfsys@transformshift{1.865551in}{1.759879in}%
\pgfsys@useobject{currentmarker}{}%
\end{pgfscope}%
\begin{pgfscope}%
\pgfsys@transformshift{1.942392in}{1.939957in}%
\pgfsys@useobject{currentmarker}{}%
\end{pgfscope}%
\begin{pgfscope}%
\pgfsys@transformshift{2.019234in}{2.130913in}%
\pgfsys@useobject{currentmarker}{}%
\end{pgfscope}%
\begin{pgfscope}%
\pgfsys@transformshift{2.096075in}{2.515131in}%
\pgfsys@useobject{currentmarker}{}%
\end{pgfscope}%
\begin{pgfscope}%
\pgfsys@transformshift{2.172917in}{2.406706in}%
\pgfsys@useobject{currentmarker}{}%
\end{pgfscope}%
\begin{pgfscope}%
\pgfsys@transformshift{2.249758in}{1.210094in}%
\pgfsys@useobject{currentmarker}{}%
\end{pgfscope}%
\begin{pgfscope}%
\pgfsys@transformshift{2.326600in}{0.882843in}%
\pgfsys@useobject{currentmarker}{}%
\end{pgfscope}%
\begin{pgfscope}%
\pgfsys@transformshift{2.403441in}{0.784564in}%
\pgfsys@useobject{currentmarker}{}%
\end{pgfscope}%
\begin{pgfscope}%
\pgfsys@transformshift{2.480282in}{0.722917in}%
\pgfsys@useobject{currentmarker}{}%
\end{pgfscope}%
\begin{pgfscope}%
\pgfsys@transformshift{2.557124in}{0.682617in}%
\pgfsys@useobject{currentmarker}{}%
\end{pgfscope}%
\begin{pgfscope}%
\pgfsys@transformshift{2.633965in}{0.654305in}%
\pgfsys@useobject{currentmarker}{}%
\end{pgfscope}%
\begin{pgfscope}%
\pgfsys@transformshift{2.710807in}{0.630461in}%
\pgfsys@useobject{currentmarker}{}%
\end{pgfscope}%
\begin{pgfscope}%
\pgfsys@transformshift{2.787648in}{0.610090in}%
\pgfsys@useobject{currentmarker}{}%
\end{pgfscope}%
\begin{pgfscope}%
\pgfsys@transformshift{2.864490in}{0.593816in}%
\pgfsys@useobject{currentmarker}{}%
\end{pgfscope}%
\begin{pgfscope}%
\pgfsys@transformshift{2.941331in}{0.579931in}%
\pgfsys@useobject{currentmarker}{}%
\end{pgfscope}%
\begin{pgfscope}%
\pgfsys@transformshift{3.018172in}{0.567999in}%
\pgfsys@useobject{currentmarker}{}%
\end{pgfscope}%
\begin{pgfscope}%
\pgfsys@transformshift{3.095014in}{0.556442in}%
\pgfsys@useobject{currentmarker}{}%
\end{pgfscope}%
\begin{pgfscope}%
\pgfsys@transformshift{3.171855in}{0.547648in}%
\pgfsys@useobject{currentmarker}{}%
\end{pgfscope}%
\begin{pgfscope}%
\pgfsys@transformshift{3.248697in}{0.538505in}%
\pgfsys@useobject{currentmarker}{}%
\end{pgfscope}%
\begin{pgfscope}%
\pgfsys@transformshift{3.325538in}{0.531060in}%
\pgfsys@useobject{currentmarker}{}%
\end{pgfscope}%
\begin{pgfscope}%
\pgfsys@transformshift{3.402380in}{0.525042in}%
\pgfsys@useobject{currentmarker}{}%
\end{pgfscope}%
\begin{pgfscope}%
\pgfsys@transformshift{3.479221in}{0.518694in}%
\pgfsys@useobject{currentmarker}{}%
\end{pgfscope}%
\begin{pgfscope}%
\pgfsys@transformshift{3.556063in}{0.512620in}%
\pgfsys@useobject{currentmarker}{}%
\end{pgfscope}%
\begin{pgfscope}%
\pgfsys@transformshift{3.632904in}{0.507135in}%
\pgfsys@useobject{currentmarker}{}%
\end{pgfscope}%
\begin{pgfscope}%
\pgfsys@transformshift{3.709745in}{0.502407in}%
\pgfsys@useobject{currentmarker}{}%
\end{pgfscope}%
\end{pgfscope}%
\begin{pgfscope}%
\pgfsetbuttcap%
\pgfsetroundjoin%
\definecolor{currentfill}{rgb}{0.000000,0.000000,0.000000}%
\pgfsetfillcolor{currentfill}%
\pgfsetlinewidth{0.803000pt}%
\definecolor{currentstroke}{rgb}{0.000000,0.000000,0.000000}%
\pgfsetstrokecolor{currentstroke}%
\pgfsetdash{}{0pt}%
\pgfsys@defobject{currentmarker}{\pgfqpoint{0.000000in}{-0.048611in}}{\pgfqpoint{0.000000in}{0.000000in}}{%
\pgfpathmoveto{\pgfqpoint{0.000000in}{0.000000in}}%
\pgfpathlineto{\pgfqpoint{0.000000in}{-0.048611in}}%
\pgfusepath{stroke,fill}%
}%
\begin{pgfscope}%
\pgfsys@transformshift{1.020295in}{0.386111in}%
\pgfsys@useobject{currentmarker}{}%
\end{pgfscope}%
\end{pgfscope}%
\begin{pgfscope}%
\definecolor{textcolor}{rgb}{0.000000,0.000000,0.000000}%
\pgfsetstrokecolor{textcolor}%
\pgfsetfillcolor{textcolor}%
\pgftext[x=1.020295in,y=0.288889in,,top]{\color{textcolor}\sffamily\fontsize{10.000000}{12.000000}\selectfont 3}%
\end{pgfscope}%
\begin{pgfscope}%
\pgfsetbuttcap%
\pgfsetroundjoin%
\definecolor{currentfill}{rgb}{0.000000,0.000000,0.000000}%
\pgfsetfillcolor{currentfill}%
\pgfsetlinewidth{0.803000pt}%
\definecolor{currentstroke}{rgb}{0.000000,0.000000,0.000000}%
\pgfsetstrokecolor{currentstroke}%
\pgfsetdash{}{0pt}%
\pgfsys@defobject{currentmarker}{\pgfqpoint{0.000000in}{-0.048611in}}{\pgfqpoint{0.000000in}{0.000000in}}{%
\pgfpathmoveto{\pgfqpoint{0.000000in}{0.000000in}}%
\pgfpathlineto{\pgfqpoint{0.000000in}{-0.048611in}}%
\pgfusepath{stroke,fill}%
}%
\begin{pgfscope}%
\pgfsys@transformshift{1.788709in}{0.386111in}%
\pgfsys@useobject{currentmarker}{}%
\end{pgfscope}%
\end{pgfscope}%
\begin{pgfscope}%
\definecolor{textcolor}{rgb}{0.000000,0.000000,0.000000}%
\pgfsetstrokecolor{textcolor}%
\pgfsetfillcolor{textcolor}%
\pgftext[x=1.788709in,y=0.288889in,,top]{\color{textcolor}\sffamily\fontsize{10.000000}{12.000000}\selectfont 4}%
\end{pgfscope}%
\begin{pgfscope}%
\pgfsetbuttcap%
\pgfsetroundjoin%
\definecolor{currentfill}{rgb}{0.000000,0.000000,0.000000}%
\pgfsetfillcolor{currentfill}%
\pgfsetlinewidth{0.803000pt}%
\definecolor{currentstroke}{rgb}{0.000000,0.000000,0.000000}%
\pgfsetstrokecolor{currentstroke}%
\pgfsetdash{}{0pt}%
\pgfsys@defobject{currentmarker}{\pgfqpoint{0.000000in}{-0.048611in}}{\pgfqpoint{0.000000in}{0.000000in}}{%
\pgfpathmoveto{\pgfqpoint{0.000000in}{0.000000in}}%
\pgfpathlineto{\pgfqpoint{0.000000in}{-0.048611in}}%
\pgfusepath{stroke,fill}%
}%
\begin{pgfscope}%
\pgfsys@transformshift{2.557124in}{0.386111in}%
\pgfsys@useobject{currentmarker}{}%
\end{pgfscope}%
\end{pgfscope}%
\begin{pgfscope}%
\definecolor{textcolor}{rgb}{0.000000,0.000000,0.000000}%
\pgfsetstrokecolor{textcolor}%
\pgfsetfillcolor{textcolor}%
\pgftext[x=2.557124in,y=0.288889in,,top]{\color{textcolor}\sffamily\fontsize{10.000000}{12.000000}\selectfont 5}%
\end{pgfscope}%
\begin{pgfscope}%
\pgfsetbuttcap%
\pgfsetroundjoin%
\definecolor{currentfill}{rgb}{0.000000,0.000000,0.000000}%
\pgfsetfillcolor{currentfill}%
\pgfsetlinewidth{0.803000pt}%
\definecolor{currentstroke}{rgb}{0.000000,0.000000,0.000000}%
\pgfsetstrokecolor{currentstroke}%
\pgfsetdash{}{0pt}%
\pgfsys@defobject{currentmarker}{\pgfqpoint{0.000000in}{-0.048611in}}{\pgfqpoint{0.000000in}{0.000000in}}{%
\pgfpathmoveto{\pgfqpoint{0.000000in}{0.000000in}}%
\pgfpathlineto{\pgfqpoint{0.000000in}{-0.048611in}}%
\pgfusepath{stroke,fill}%
}%
\begin{pgfscope}%
\pgfsys@transformshift{3.325538in}{0.386111in}%
\pgfsys@useobject{currentmarker}{}%
\end{pgfscope}%
\end{pgfscope}%
\begin{pgfscope}%
\definecolor{textcolor}{rgb}{0.000000,0.000000,0.000000}%
\pgfsetstrokecolor{textcolor}%
\pgfsetfillcolor{textcolor}%
\pgftext[x=3.325538in,y=0.288889in,,top]{\color{textcolor}\sffamily\fontsize{10.000000}{12.000000}\selectfont 6}%
\end{pgfscope}%
\begin{pgfscope}%
\pgfsetbuttcap%
\pgfsetroundjoin%
\definecolor{currentfill}{rgb}{0.000000,0.000000,0.000000}%
\pgfsetfillcolor{currentfill}%
\pgfsetlinewidth{0.803000pt}%
\definecolor{currentstroke}{rgb}{0.000000,0.000000,0.000000}%
\pgfsetstrokecolor{currentstroke}%
\pgfsetdash{}{0pt}%
\pgfsys@defobject{currentmarker}{\pgfqpoint{-0.048611in}{0.000000in}}{\pgfqpoint{0.000000in}{0.000000in}}{%
\pgfpathmoveto{\pgfqpoint{0.000000in}{0.000000in}}%
\pgfpathlineto{\pgfqpoint{-0.048611in}{0.000000in}}%
\pgfusepath{stroke,fill}%
}%
\begin{pgfscope}%
\pgfsys@transformshift{0.469444in}{0.401589in}%
\pgfsys@useobject{currentmarker}{}%
\end{pgfscope}%
\end{pgfscope}%
\begin{pgfscope}%
\definecolor{textcolor}{rgb}{0.000000,0.000000,0.000000}%
\pgfsetstrokecolor{textcolor}%
\pgfsetfillcolor{textcolor}%
\pgftext[x=0.151343in,y=0.348828in,left,base]{\color{textcolor}\sffamily\fontsize{10.000000}{12.000000}\selectfont 0.0}%
\end{pgfscope}%
\begin{pgfscope}%
\pgfsetbuttcap%
\pgfsetroundjoin%
\definecolor{currentfill}{rgb}{0.000000,0.000000,0.000000}%
\pgfsetfillcolor{currentfill}%
\pgfsetlinewidth{0.803000pt}%
\definecolor{currentstroke}{rgb}{0.000000,0.000000,0.000000}%
\pgfsetstrokecolor{currentstroke}%
\pgfsetdash{}{0pt}%
\pgfsys@defobject{currentmarker}{\pgfqpoint{-0.048611in}{0.000000in}}{\pgfqpoint{0.000000in}{0.000000in}}{%
\pgfpathmoveto{\pgfqpoint{0.000000in}{0.000000in}}%
\pgfpathlineto{\pgfqpoint{-0.048611in}{0.000000in}}%
\pgfusepath{stroke,fill}%
}%
\begin{pgfscope}%
\pgfsys@transformshift{0.469444in}{0.902052in}%
\pgfsys@useobject{currentmarker}{}%
\end{pgfscope}%
\end{pgfscope}%
\begin{pgfscope}%
\definecolor{textcolor}{rgb}{0.000000,0.000000,0.000000}%
\pgfsetstrokecolor{textcolor}%
\pgfsetfillcolor{textcolor}%
\pgftext[x=0.151343in,y=0.849291in,left,base]{\color{textcolor}\sffamily\fontsize{10.000000}{12.000000}\selectfont 0.5}%
\end{pgfscope}%
\begin{pgfscope}%
\pgfsetbuttcap%
\pgfsetroundjoin%
\definecolor{currentfill}{rgb}{0.000000,0.000000,0.000000}%
\pgfsetfillcolor{currentfill}%
\pgfsetlinewidth{0.803000pt}%
\definecolor{currentstroke}{rgb}{0.000000,0.000000,0.000000}%
\pgfsetstrokecolor{currentstroke}%
\pgfsetdash{}{0pt}%
\pgfsys@defobject{currentmarker}{\pgfqpoint{-0.048611in}{0.000000in}}{\pgfqpoint{0.000000in}{0.000000in}}{%
\pgfpathmoveto{\pgfqpoint{0.000000in}{0.000000in}}%
\pgfpathlineto{\pgfqpoint{-0.048611in}{0.000000in}}%
\pgfusepath{stroke,fill}%
}%
\begin{pgfscope}%
\pgfsys@transformshift{0.469444in}{1.402515in}%
\pgfsys@useobject{currentmarker}{}%
\end{pgfscope}%
\end{pgfscope}%
\begin{pgfscope}%
\definecolor{textcolor}{rgb}{0.000000,0.000000,0.000000}%
\pgfsetstrokecolor{textcolor}%
\pgfsetfillcolor{textcolor}%
\pgftext[x=0.151343in,y=1.349753in,left,base]{\color{textcolor}\sffamily\fontsize{10.000000}{12.000000}\selectfont 1.0}%
\end{pgfscope}%
\begin{pgfscope}%
\pgfsetbuttcap%
\pgfsetroundjoin%
\definecolor{currentfill}{rgb}{0.000000,0.000000,0.000000}%
\pgfsetfillcolor{currentfill}%
\pgfsetlinewidth{0.803000pt}%
\definecolor{currentstroke}{rgb}{0.000000,0.000000,0.000000}%
\pgfsetstrokecolor{currentstroke}%
\pgfsetdash{}{0pt}%
\pgfsys@defobject{currentmarker}{\pgfqpoint{-0.048611in}{0.000000in}}{\pgfqpoint{0.000000in}{0.000000in}}{%
\pgfpathmoveto{\pgfqpoint{0.000000in}{0.000000in}}%
\pgfpathlineto{\pgfqpoint{-0.048611in}{0.000000in}}%
\pgfusepath{stroke,fill}%
}%
\begin{pgfscope}%
\pgfsys@transformshift{0.469444in}{1.902978in}%
\pgfsys@useobject{currentmarker}{}%
\end{pgfscope}%
\end{pgfscope}%
\begin{pgfscope}%
\definecolor{textcolor}{rgb}{0.000000,0.000000,0.000000}%
\pgfsetstrokecolor{textcolor}%
\pgfsetfillcolor{textcolor}%
\pgftext[x=0.151343in,y=1.850216in,left,base]{\color{textcolor}\sffamily\fontsize{10.000000}{12.000000}\selectfont 1.5}%
\end{pgfscope}%
\begin{pgfscope}%
\pgfsetbuttcap%
\pgfsetroundjoin%
\definecolor{currentfill}{rgb}{0.000000,0.000000,0.000000}%
\pgfsetfillcolor{currentfill}%
\pgfsetlinewidth{0.803000pt}%
\definecolor{currentstroke}{rgb}{0.000000,0.000000,0.000000}%
\pgfsetstrokecolor{currentstroke}%
\pgfsetdash{}{0pt}%
\pgfsys@defobject{currentmarker}{\pgfqpoint{-0.048611in}{0.000000in}}{\pgfqpoint{0.000000in}{0.000000in}}{%
\pgfpathmoveto{\pgfqpoint{0.000000in}{0.000000in}}%
\pgfpathlineto{\pgfqpoint{-0.048611in}{0.000000in}}%
\pgfusepath{stroke,fill}%
}%
\begin{pgfscope}%
\pgfsys@transformshift{0.469444in}{2.403440in}%
\pgfsys@useobject{currentmarker}{}%
\end{pgfscope}%
\end{pgfscope}%
\begin{pgfscope}%
\definecolor{textcolor}{rgb}{0.000000,0.000000,0.000000}%
\pgfsetstrokecolor{textcolor}%
\pgfsetfillcolor{textcolor}%
\pgftext[x=0.151343in,y=2.350679in,left,base]{\color{textcolor}\sffamily\fontsize{10.000000}{12.000000}\selectfont 2.0}%
\end{pgfscope}%
\begin{pgfscope}%
\pgfpathrectangle{\pgfqpoint{0.469444in}{0.386111in}}{\pgfqpoint{3.406944in}{2.316667in}}%
\pgfusepath{clip}%
\pgfsetrectcap%
\pgfsetroundjoin%
\pgfsetlinewidth{1.505625pt}%
\definecolor{currentstroke}{rgb}{0.121569,0.466667,0.705882}%
\pgfsetstrokecolor{currentstroke}%
\pgfsetdash{}{0pt}%
\pgfpathmoveto{\pgfqpoint{0.636088in}{0.652311in}}%
\pgfpathlineto{\pgfqpoint{0.712929in}{0.690161in}}%
\pgfpathlineto{\pgfqpoint{0.789771in}{0.731075in}}%
\pgfpathlineto{\pgfqpoint{0.866612in}{0.774846in}}%
\pgfpathlineto{\pgfqpoint{0.943454in}{0.821763in}}%
\pgfpathlineto{\pgfqpoint{1.020295in}{0.870951in}}%
\pgfpathlineto{\pgfqpoint{1.097137in}{0.922603in}}%
\pgfpathlineto{\pgfqpoint{1.173978in}{0.980884in}}%
\pgfpathlineto{\pgfqpoint{1.250819in}{1.040304in}}%
\pgfpathlineto{\pgfqpoint{1.327661in}{1.106813in}}%
\pgfpathlineto{\pgfqpoint{1.404502in}{1.176899in}}%
\pgfpathlineto{\pgfqpoint{1.481344in}{1.250229in}}%
\pgfpathlineto{\pgfqpoint{1.558185in}{1.331307in}}%
\pgfpathlineto{\pgfqpoint{1.635027in}{1.419967in}}%
\pgfpathlineto{\pgfqpoint{1.711868in}{1.515850in}}%
\pgfpathlineto{\pgfqpoint{1.788709in}{1.631287in}}%
\pgfpathlineto{\pgfqpoint{1.865551in}{1.759879in}}%
\pgfpathlineto{\pgfqpoint{1.942392in}{1.939957in}}%
\pgfpathlineto{\pgfqpoint{2.019234in}{2.130913in}}%
\pgfpathlineto{\pgfqpoint{2.096075in}{2.515131in}}%
\pgfpathlineto{\pgfqpoint{2.172917in}{2.406706in}}%
\pgfpathlineto{\pgfqpoint{2.249758in}{1.210094in}}%
\pgfpathlineto{\pgfqpoint{2.326600in}{0.882843in}}%
\pgfpathlineto{\pgfqpoint{2.403441in}{0.784564in}}%
\pgfpathlineto{\pgfqpoint{2.480282in}{0.722917in}}%
\pgfpathlineto{\pgfqpoint{2.557124in}{0.682617in}}%
\pgfpathlineto{\pgfqpoint{2.633965in}{0.654305in}}%
\pgfpathlineto{\pgfqpoint{2.710807in}{0.630461in}}%
\pgfpathlineto{\pgfqpoint{2.787648in}{0.610090in}}%
\pgfpathlineto{\pgfqpoint{2.864490in}{0.593816in}}%
\pgfpathlineto{\pgfqpoint{2.941331in}{0.579931in}}%
\pgfpathlineto{\pgfqpoint{3.018172in}{0.567999in}}%
\pgfpathlineto{\pgfqpoint{3.095014in}{0.556442in}}%
\pgfpathlineto{\pgfqpoint{3.171855in}{0.547648in}}%
\pgfpathlineto{\pgfqpoint{3.248697in}{0.538505in}}%
\pgfpathlineto{\pgfqpoint{3.325538in}{0.531060in}}%
\pgfpathlineto{\pgfqpoint{3.402380in}{0.525042in}}%
\pgfpathlineto{\pgfqpoint{3.479221in}{0.518694in}}%
\pgfpathlineto{\pgfqpoint{3.556063in}{0.512620in}}%
\pgfpathlineto{\pgfqpoint{3.632904in}{0.507135in}}%
\pgfpathlineto{\pgfqpoint{3.709745in}{0.502407in}}%
\pgfusepath{stroke}%
\end{pgfscope}%
\begin{pgfscope}%
\pgfsetrectcap%
\pgfsetmiterjoin%
\pgfsetlinewidth{0.803000pt}%
\definecolor{currentstroke}{rgb}{0.000000,0.000000,0.000000}%
\pgfsetstrokecolor{currentstroke}%
\pgfsetdash{}{0pt}%
\pgfpathmoveto{\pgfqpoint{0.469444in}{0.386111in}}%
\pgfpathlineto{\pgfqpoint{0.469444in}{2.702778in}}%
\pgfusepath{stroke}%
\end{pgfscope}%
\begin{pgfscope}%
\pgfsetrectcap%
\pgfsetmiterjoin%
\pgfsetlinewidth{0.803000pt}%
\definecolor{currentstroke}{rgb}{0.000000,0.000000,0.000000}%
\pgfsetstrokecolor{currentstroke}%
\pgfsetdash{}{0pt}%
\pgfpathmoveto{\pgfqpoint{3.876389in}{0.386111in}}%
\pgfpathlineto{\pgfqpoint{3.876389in}{2.702778in}}%
\pgfusepath{stroke}%
\end{pgfscope}%
\begin{pgfscope}%
\pgfsetrectcap%
\pgfsetmiterjoin%
\pgfsetlinewidth{0.803000pt}%
\definecolor{currentstroke}{rgb}{0.000000,0.000000,0.000000}%
\pgfsetstrokecolor{currentstroke}%
\pgfsetdash{}{0pt}%
\pgfpathmoveto{\pgfqpoint{0.469444in}{0.386111in}}%
\pgfpathlineto{\pgfqpoint{3.876389in}{0.386111in}}%
\pgfusepath{stroke}%
\end{pgfscope}%
\begin{pgfscope}%
\pgfsetrectcap%
\pgfsetmiterjoin%
\pgfsetlinewidth{0.803000pt}%
\definecolor{currentstroke}{rgb}{0.000000,0.000000,0.000000}%
\pgfsetstrokecolor{currentstroke}%
\pgfsetdash{}{0pt}%
\pgfpathmoveto{\pgfqpoint{0.469444in}{2.702778in}}%
\pgfpathlineto{\pgfqpoint{3.876389in}{2.702778in}}%
\pgfusepath{stroke}%
\end{pgfscope}%
\begin{pgfscope}%
\definecolor{textcolor}{rgb}{0.000000,0.000000,0.000000}%
\pgfsetstrokecolor{textcolor}%
\pgfsetfillcolor{textcolor}%
\pgftext[x=2.172917in,y=2.786111in,,base]{\color{textcolor}\sffamily\fontsize{12.000000}{14.400000}\selectfont \(\displaystyle  N = 16 \)}%
\end{pgfscope}%
\begin{pgfscope}%
\pgfsetbuttcap%
\pgfsetmiterjoin%
\definecolor{currentfill}{rgb}{1.000000,1.000000,1.000000}%
\pgfsetfillcolor{currentfill}%
\pgfsetlinewidth{0.000000pt}%
\definecolor{currentstroke}{rgb}{0.000000,0.000000,0.000000}%
\pgfsetstrokecolor{currentstroke}%
\pgfsetstrokeopacity{0.000000}%
\pgfsetdash{}{0pt}%
\pgfpathmoveto{\pgfqpoint{4.394444in}{0.386111in}}%
\pgfpathlineto{\pgfqpoint{7.801389in}{0.386111in}}%
\pgfpathlineto{\pgfqpoint{7.801389in}{2.702778in}}%
\pgfpathlineto{\pgfqpoint{4.394444in}{2.702778in}}%
\pgfpathclose%
\pgfusepath{fill}%
\end{pgfscope}%
\begin{pgfscope}%
\pgfpathrectangle{\pgfqpoint{4.394444in}{0.386111in}}{\pgfqpoint{3.406944in}{2.316667in}}%
\pgfusepath{clip}%
\pgfsetbuttcap%
\pgfsetroundjoin%
\definecolor{currentfill}{rgb}{0.121569,0.466667,0.705882}%
\pgfsetfillcolor{currentfill}%
\pgfsetfillopacity{0.300000}%
\pgfsetlinewidth{1.003750pt}%
\definecolor{currentstroke}{rgb}{0.121569,0.466667,0.705882}%
\pgfsetstrokecolor{currentstroke}%
\pgfsetstrokeopacity{0.300000}%
\pgfsetdash{}{0pt}%
\pgfpathmoveto{\pgfqpoint{4.561088in}{0.623398in}}%
\pgfpathlineto{\pgfqpoint{4.561088in}{0.622816in}}%
\pgfpathlineto{\pgfqpoint{4.637929in}{0.653528in}}%
\pgfpathlineto{\pgfqpoint{4.714771in}{0.685891in}}%
\pgfpathlineto{\pgfqpoint{4.791612in}{0.720988in}}%
\pgfpathlineto{\pgfqpoint{4.868454in}{0.758337in}}%
\pgfpathlineto{\pgfqpoint{4.945295in}{0.796636in}}%
\pgfpathlineto{\pgfqpoint{5.022137in}{0.840996in}}%
\pgfpathlineto{\pgfqpoint{5.098978in}{0.886247in}}%
\pgfpathlineto{\pgfqpoint{5.175819in}{0.932925in}}%
\pgfpathlineto{\pgfqpoint{5.252661in}{0.985758in}}%
\pgfpathlineto{\pgfqpoint{5.329502in}{1.037290in}}%
\pgfpathlineto{\pgfqpoint{5.406344in}{1.098406in}}%
\pgfpathlineto{\pgfqpoint{5.483185in}{1.163385in}}%
\pgfpathlineto{\pgfqpoint{5.560027in}{1.241841in}}%
\pgfpathlineto{\pgfqpoint{5.636868in}{1.302404in}}%
\pgfpathlineto{\pgfqpoint{5.713709in}{1.399911in}}%
\pgfpathlineto{\pgfqpoint{5.790551in}{1.508073in}}%
\pgfpathlineto{\pgfqpoint{5.867392in}{1.622251in}}%
\pgfpathlineto{\pgfqpoint{5.944234in}{1.777918in}}%
\pgfpathlineto{\pgfqpoint{6.021075in}{2.043728in}}%
\pgfpathlineto{\pgfqpoint{6.097917in}{2.161839in}}%
\pgfpathlineto{\pgfqpoint{6.174758in}{0.924549in}}%
\pgfpathlineto{\pgfqpoint{6.251600in}{0.792944in}}%
\pgfpathlineto{\pgfqpoint{6.328441in}{0.719356in}}%
\pgfpathlineto{\pgfqpoint{6.405282in}{0.675752in}}%
\pgfpathlineto{\pgfqpoint{6.482124in}{0.641404in}}%
\pgfpathlineto{\pgfqpoint{6.558965in}{0.618792in}}%
\pgfpathlineto{\pgfqpoint{6.635807in}{0.601596in}}%
\pgfpathlineto{\pgfqpoint{6.712648in}{0.585420in}}%
\pgfpathlineto{\pgfqpoint{6.789490in}{0.573660in}}%
\pgfpathlineto{\pgfqpoint{6.866331in}{0.561111in}}%
\pgfpathlineto{\pgfqpoint{6.943172in}{0.552349in}}%
\pgfpathlineto{\pgfqpoint{7.020014in}{0.543933in}}%
\pgfpathlineto{\pgfqpoint{7.096855in}{0.537931in}}%
\pgfpathlineto{\pgfqpoint{7.173697in}{0.529750in}}%
\pgfpathlineto{\pgfqpoint{7.250538in}{0.522988in}}%
\pgfpathlineto{\pgfqpoint{7.327380in}{0.518876in}}%
\pgfpathlineto{\pgfqpoint{7.404221in}{0.514225in}}%
\pgfpathlineto{\pgfqpoint{7.481063in}{0.509385in}}%
\pgfpathlineto{\pgfqpoint{7.557904in}{0.504553in}}%
\pgfpathlineto{\pgfqpoint{7.634745in}{0.501717in}}%
\pgfpathlineto{\pgfqpoint{7.634745in}{0.503102in}}%
\pgfpathlineto{\pgfqpoint{7.634745in}{0.503102in}}%
\pgfpathlineto{\pgfqpoint{7.557904in}{0.506899in}}%
\pgfpathlineto{\pgfqpoint{7.481063in}{0.510856in}}%
\pgfpathlineto{\pgfqpoint{7.404221in}{0.516372in}}%
\pgfpathlineto{\pgfqpoint{7.327380in}{0.520404in}}%
\pgfpathlineto{\pgfqpoint{7.250538in}{0.527836in}}%
\pgfpathlineto{\pgfqpoint{7.173697in}{0.531999in}}%
\pgfpathlineto{\pgfqpoint{7.096855in}{0.538449in}}%
\pgfpathlineto{\pgfqpoint{7.020014in}{0.548017in}}%
\pgfpathlineto{\pgfqpoint{6.943172in}{0.554162in}}%
\pgfpathlineto{\pgfqpoint{6.866331in}{0.563398in}}%
\pgfpathlineto{\pgfqpoint{6.789490in}{0.575459in}}%
\pgfpathlineto{\pgfqpoint{6.712648in}{0.587065in}}%
\pgfpathlineto{\pgfqpoint{6.635807in}{0.604641in}}%
\pgfpathlineto{\pgfqpoint{6.558965in}{0.626406in}}%
\pgfpathlineto{\pgfqpoint{6.482124in}{0.647170in}}%
\pgfpathlineto{\pgfqpoint{6.405282in}{0.681654in}}%
\pgfpathlineto{\pgfqpoint{6.328441in}{0.724448in}}%
\pgfpathlineto{\pgfqpoint{6.251600in}{0.799623in}}%
\pgfpathlineto{\pgfqpoint{6.174758in}{0.975178in}}%
\pgfpathlineto{\pgfqpoint{6.097917in}{2.597475in}}%
\pgfpathlineto{\pgfqpoint{6.021075in}{2.122287in}}%
\pgfpathlineto{\pgfqpoint{5.944234in}{1.875042in}}%
\pgfpathlineto{\pgfqpoint{5.867392in}{1.681565in}}%
\pgfpathlineto{\pgfqpoint{5.790551in}{1.531302in}}%
\pgfpathlineto{\pgfqpoint{5.713709in}{1.434003in}}%
\pgfpathlineto{\pgfqpoint{5.636868in}{1.357833in}}%
\pgfpathlineto{\pgfqpoint{5.560027in}{1.251598in}}%
\pgfpathlineto{\pgfqpoint{5.483185in}{1.173119in}}%
\pgfpathlineto{\pgfqpoint{5.406344in}{1.108701in}}%
\pgfpathlineto{\pgfqpoint{5.329502in}{1.049319in}}%
\pgfpathlineto{\pgfqpoint{5.252661in}{0.992009in}}%
\pgfpathlineto{\pgfqpoint{5.175819in}{0.943864in}}%
\pgfpathlineto{\pgfqpoint{5.098978in}{0.889907in}}%
\pgfpathlineto{\pgfqpoint{5.022137in}{0.843821in}}%
\pgfpathlineto{\pgfqpoint{4.945295in}{0.803649in}}%
\pgfpathlineto{\pgfqpoint{4.868454in}{0.761298in}}%
\pgfpathlineto{\pgfqpoint{4.791612in}{0.723454in}}%
\pgfpathlineto{\pgfqpoint{4.714771in}{0.687228in}}%
\pgfpathlineto{\pgfqpoint{4.637929in}{0.654446in}}%
\pgfpathlineto{\pgfqpoint{4.561088in}{0.623398in}}%
\pgfpathclose%
\pgfusepath{stroke,fill}%
\end{pgfscope}%
\begin{pgfscope}%
\pgfpathrectangle{\pgfqpoint{4.394444in}{0.386111in}}{\pgfqpoint{3.406944in}{2.316667in}}%
\pgfusepath{clip}%
\pgfsetbuttcap%
\pgfsetroundjoin%
\definecolor{currentfill}{rgb}{0.121569,0.466667,0.705882}%
\pgfsetfillcolor{currentfill}%
\pgfsetlinewidth{1.003750pt}%
\definecolor{currentstroke}{rgb}{0.121569,0.466667,0.705882}%
\pgfsetstrokecolor{currentstroke}%
\pgfsetdash{}{0pt}%
\pgfsys@defobject{currentmarker}{\pgfqpoint{-0.009821in}{-0.009821in}}{\pgfqpoint{0.009821in}{0.009821in}}{%
\pgfpathmoveto{\pgfqpoint{0.000000in}{-0.009821in}}%
\pgfpathcurveto{\pgfqpoint{0.002605in}{-0.009821in}}{\pgfqpoint{0.005103in}{-0.008786in}}{\pgfqpoint{0.006944in}{-0.006944in}}%
\pgfpathcurveto{\pgfqpoint{0.008786in}{-0.005103in}}{\pgfqpoint{0.009821in}{-0.002605in}}{\pgfqpoint{0.009821in}{0.000000in}}%
\pgfpathcurveto{\pgfqpoint{0.009821in}{0.002605in}}{\pgfqpoint{0.008786in}{0.005103in}}{\pgfqpoint{0.006944in}{0.006944in}}%
\pgfpathcurveto{\pgfqpoint{0.005103in}{0.008786in}}{\pgfqpoint{0.002605in}{0.009821in}}{\pgfqpoint{0.000000in}{0.009821in}}%
\pgfpathcurveto{\pgfqpoint{-0.002605in}{0.009821in}}{\pgfqpoint{-0.005103in}{0.008786in}}{\pgfqpoint{-0.006944in}{0.006944in}}%
\pgfpathcurveto{\pgfqpoint{-0.008786in}{0.005103in}}{\pgfqpoint{-0.009821in}{0.002605in}}{\pgfqpoint{-0.009821in}{0.000000in}}%
\pgfpathcurveto{\pgfqpoint{-0.009821in}{-0.002605in}}{\pgfqpoint{-0.008786in}{-0.005103in}}{\pgfqpoint{-0.006944in}{-0.006944in}}%
\pgfpathcurveto{\pgfqpoint{-0.005103in}{-0.008786in}}{\pgfqpoint{-0.002605in}{-0.009821in}}{\pgfqpoint{0.000000in}{-0.009821in}}%
\pgfpathclose%
\pgfusepath{stroke,fill}%
}%
\begin{pgfscope}%
\pgfsys@transformshift{4.561088in}{0.623107in}%
\pgfsys@useobject{currentmarker}{}%
\end{pgfscope}%
\begin{pgfscope}%
\pgfsys@transformshift{4.637929in}{0.653987in}%
\pgfsys@useobject{currentmarker}{}%
\end{pgfscope}%
\begin{pgfscope}%
\pgfsys@transformshift{4.714771in}{0.686560in}%
\pgfsys@useobject{currentmarker}{}%
\end{pgfscope}%
\begin{pgfscope}%
\pgfsys@transformshift{4.791612in}{0.722221in}%
\pgfsys@useobject{currentmarker}{}%
\end{pgfscope}%
\begin{pgfscope}%
\pgfsys@transformshift{4.868454in}{0.759817in}%
\pgfsys@useobject{currentmarker}{}%
\end{pgfscope}%
\begin{pgfscope}%
\pgfsys@transformshift{4.945295in}{0.800142in}%
\pgfsys@useobject{currentmarker}{}%
\end{pgfscope}%
\begin{pgfscope}%
\pgfsys@transformshift{5.022137in}{0.842408in}%
\pgfsys@useobject{currentmarker}{}%
\end{pgfscope}%
\begin{pgfscope}%
\pgfsys@transformshift{5.098978in}{0.888077in}%
\pgfsys@useobject{currentmarker}{}%
\end{pgfscope}%
\begin{pgfscope}%
\pgfsys@transformshift{5.175819in}{0.938394in}%
\pgfsys@useobject{currentmarker}{}%
\end{pgfscope}%
\begin{pgfscope}%
\pgfsys@transformshift{5.252661in}{0.988884in}%
\pgfsys@useobject{currentmarker}{}%
\end{pgfscope}%
\begin{pgfscope}%
\pgfsys@transformshift{5.329502in}{1.043305in}%
\pgfsys@useobject{currentmarker}{}%
\end{pgfscope}%
\begin{pgfscope}%
\pgfsys@transformshift{5.406344in}{1.103553in}%
\pgfsys@useobject{currentmarker}{}%
\end{pgfscope}%
\begin{pgfscope}%
\pgfsys@transformshift{5.483185in}{1.168252in}%
\pgfsys@useobject{currentmarker}{}%
\end{pgfscope}%
\begin{pgfscope}%
\pgfsys@transformshift{5.560027in}{1.246720in}%
\pgfsys@useobject{currentmarker}{}%
\end{pgfscope}%
\begin{pgfscope}%
\pgfsys@transformshift{5.636868in}{1.330119in}%
\pgfsys@useobject{currentmarker}{}%
\end{pgfscope}%
\begin{pgfscope}%
\pgfsys@transformshift{5.713709in}{1.416957in}%
\pgfsys@useobject{currentmarker}{}%
\end{pgfscope}%
\begin{pgfscope}%
\pgfsys@transformshift{5.790551in}{1.519687in}%
\pgfsys@useobject{currentmarker}{}%
\end{pgfscope}%
\begin{pgfscope}%
\pgfsys@transformshift{5.867392in}{1.651908in}%
\pgfsys@useobject{currentmarker}{}%
\end{pgfscope}%
\begin{pgfscope}%
\pgfsys@transformshift{5.944234in}{1.826480in}%
\pgfsys@useobject{currentmarker}{}%
\end{pgfscope}%
\begin{pgfscope}%
\pgfsys@transformshift{6.021075in}{2.083008in}%
\pgfsys@useobject{currentmarker}{}%
\end{pgfscope}%
\begin{pgfscope}%
\pgfsys@transformshift{6.097917in}{2.379657in}%
\pgfsys@useobject{currentmarker}{}%
\end{pgfscope}%
\begin{pgfscope}%
\pgfsys@transformshift{6.174758in}{0.949864in}%
\pgfsys@useobject{currentmarker}{}%
\end{pgfscope}%
\begin{pgfscope}%
\pgfsys@transformshift{6.251600in}{0.796284in}%
\pgfsys@useobject{currentmarker}{}%
\end{pgfscope}%
\begin{pgfscope}%
\pgfsys@transformshift{6.328441in}{0.721902in}%
\pgfsys@useobject{currentmarker}{}%
\end{pgfscope}%
\begin{pgfscope}%
\pgfsys@transformshift{6.405282in}{0.678703in}%
\pgfsys@useobject{currentmarker}{}%
\end{pgfscope}%
\begin{pgfscope}%
\pgfsys@transformshift{6.482124in}{0.644287in}%
\pgfsys@useobject{currentmarker}{}%
\end{pgfscope}%
\begin{pgfscope}%
\pgfsys@transformshift{6.558965in}{0.622599in}%
\pgfsys@useobject{currentmarker}{}%
\end{pgfscope}%
\begin{pgfscope}%
\pgfsys@transformshift{6.635807in}{0.603119in}%
\pgfsys@useobject{currentmarker}{}%
\end{pgfscope}%
\begin{pgfscope}%
\pgfsys@transformshift{6.712648in}{0.586242in}%
\pgfsys@useobject{currentmarker}{}%
\end{pgfscope}%
\begin{pgfscope}%
\pgfsys@transformshift{6.789490in}{0.574559in}%
\pgfsys@useobject{currentmarker}{}%
\end{pgfscope}%
\begin{pgfscope}%
\pgfsys@transformshift{6.866331in}{0.562255in}%
\pgfsys@useobject{currentmarker}{}%
\end{pgfscope}%
\begin{pgfscope}%
\pgfsys@transformshift{6.943172in}{0.553256in}%
\pgfsys@useobject{currentmarker}{}%
\end{pgfscope}%
\begin{pgfscope}%
\pgfsys@transformshift{7.020014in}{0.545975in}%
\pgfsys@useobject{currentmarker}{}%
\end{pgfscope}%
\begin{pgfscope}%
\pgfsys@transformshift{7.096855in}{0.538190in}%
\pgfsys@useobject{currentmarker}{}%
\end{pgfscope}%
\begin{pgfscope}%
\pgfsys@transformshift{7.173697in}{0.530875in}%
\pgfsys@useobject{currentmarker}{}%
\end{pgfscope}%
\begin{pgfscope}%
\pgfsys@transformshift{7.250538in}{0.525412in}%
\pgfsys@useobject{currentmarker}{}%
\end{pgfscope}%
\begin{pgfscope}%
\pgfsys@transformshift{7.327380in}{0.519640in}%
\pgfsys@useobject{currentmarker}{}%
\end{pgfscope}%
\begin{pgfscope}%
\pgfsys@transformshift{7.404221in}{0.515299in}%
\pgfsys@useobject{currentmarker}{}%
\end{pgfscope}%
\begin{pgfscope}%
\pgfsys@transformshift{7.481063in}{0.510120in}%
\pgfsys@useobject{currentmarker}{}%
\end{pgfscope}%
\begin{pgfscope}%
\pgfsys@transformshift{7.557904in}{0.505726in}%
\pgfsys@useobject{currentmarker}{}%
\end{pgfscope}%
\begin{pgfscope}%
\pgfsys@transformshift{7.634745in}{0.502410in}%
\pgfsys@useobject{currentmarker}{}%
\end{pgfscope}%
\end{pgfscope}%
\begin{pgfscope}%
\pgfsetbuttcap%
\pgfsetroundjoin%
\definecolor{currentfill}{rgb}{0.000000,0.000000,0.000000}%
\pgfsetfillcolor{currentfill}%
\pgfsetlinewidth{0.803000pt}%
\definecolor{currentstroke}{rgb}{0.000000,0.000000,0.000000}%
\pgfsetstrokecolor{currentstroke}%
\pgfsetdash{}{0pt}%
\pgfsys@defobject{currentmarker}{\pgfqpoint{0.000000in}{-0.048611in}}{\pgfqpoint{0.000000in}{0.000000in}}{%
\pgfpathmoveto{\pgfqpoint{0.000000in}{0.000000in}}%
\pgfpathlineto{\pgfqpoint{0.000000in}{-0.048611in}}%
\pgfusepath{stroke,fill}%
}%
\begin{pgfscope}%
\pgfsys@transformshift{4.945295in}{0.386111in}%
\pgfsys@useobject{currentmarker}{}%
\end{pgfscope}%
\end{pgfscope}%
\begin{pgfscope}%
\definecolor{textcolor}{rgb}{0.000000,0.000000,0.000000}%
\pgfsetstrokecolor{textcolor}%
\pgfsetfillcolor{textcolor}%
\pgftext[x=4.945295in,y=0.288889in,,top]{\color{textcolor}\sffamily\fontsize{10.000000}{12.000000}\selectfont 3}%
\end{pgfscope}%
\begin{pgfscope}%
\pgfsetbuttcap%
\pgfsetroundjoin%
\definecolor{currentfill}{rgb}{0.000000,0.000000,0.000000}%
\pgfsetfillcolor{currentfill}%
\pgfsetlinewidth{0.803000pt}%
\definecolor{currentstroke}{rgb}{0.000000,0.000000,0.000000}%
\pgfsetstrokecolor{currentstroke}%
\pgfsetdash{}{0pt}%
\pgfsys@defobject{currentmarker}{\pgfqpoint{0.000000in}{-0.048611in}}{\pgfqpoint{0.000000in}{0.000000in}}{%
\pgfpathmoveto{\pgfqpoint{0.000000in}{0.000000in}}%
\pgfpathlineto{\pgfqpoint{0.000000in}{-0.048611in}}%
\pgfusepath{stroke,fill}%
}%
\begin{pgfscope}%
\pgfsys@transformshift{5.713709in}{0.386111in}%
\pgfsys@useobject{currentmarker}{}%
\end{pgfscope}%
\end{pgfscope}%
\begin{pgfscope}%
\definecolor{textcolor}{rgb}{0.000000,0.000000,0.000000}%
\pgfsetstrokecolor{textcolor}%
\pgfsetfillcolor{textcolor}%
\pgftext[x=5.713709in,y=0.288889in,,top]{\color{textcolor}\sffamily\fontsize{10.000000}{12.000000}\selectfont 4}%
\end{pgfscope}%
\begin{pgfscope}%
\pgfsetbuttcap%
\pgfsetroundjoin%
\definecolor{currentfill}{rgb}{0.000000,0.000000,0.000000}%
\pgfsetfillcolor{currentfill}%
\pgfsetlinewidth{0.803000pt}%
\definecolor{currentstroke}{rgb}{0.000000,0.000000,0.000000}%
\pgfsetstrokecolor{currentstroke}%
\pgfsetdash{}{0pt}%
\pgfsys@defobject{currentmarker}{\pgfqpoint{0.000000in}{-0.048611in}}{\pgfqpoint{0.000000in}{0.000000in}}{%
\pgfpathmoveto{\pgfqpoint{0.000000in}{0.000000in}}%
\pgfpathlineto{\pgfqpoint{0.000000in}{-0.048611in}}%
\pgfusepath{stroke,fill}%
}%
\begin{pgfscope}%
\pgfsys@transformshift{6.482124in}{0.386111in}%
\pgfsys@useobject{currentmarker}{}%
\end{pgfscope}%
\end{pgfscope}%
\begin{pgfscope}%
\definecolor{textcolor}{rgb}{0.000000,0.000000,0.000000}%
\pgfsetstrokecolor{textcolor}%
\pgfsetfillcolor{textcolor}%
\pgftext[x=6.482124in,y=0.288889in,,top]{\color{textcolor}\sffamily\fontsize{10.000000}{12.000000}\selectfont 5}%
\end{pgfscope}%
\begin{pgfscope}%
\pgfsetbuttcap%
\pgfsetroundjoin%
\definecolor{currentfill}{rgb}{0.000000,0.000000,0.000000}%
\pgfsetfillcolor{currentfill}%
\pgfsetlinewidth{0.803000pt}%
\definecolor{currentstroke}{rgb}{0.000000,0.000000,0.000000}%
\pgfsetstrokecolor{currentstroke}%
\pgfsetdash{}{0pt}%
\pgfsys@defobject{currentmarker}{\pgfqpoint{0.000000in}{-0.048611in}}{\pgfqpoint{0.000000in}{0.000000in}}{%
\pgfpathmoveto{\pgfqpoint{0.000000in}{0.000000in}}%
\pgfpathlineto{\pgfqpoint{0.000000in}{-0.048611in}}%
\pgfusepath{stroke,fill}%
}%
\begin{pgfscope}%
\pgfsys@transformshift{7.250538in}{0.386111in}%
\pgfsys@useobject{currentmarker}{}%
\end{pgfscope}%
\end{pgfscope}%
\begin{pgfscope}%
\definecolor{textcolor}{rgb}{0.000000,0.000000,0.000000}%
\pgfsetstrokecolor{textcolor}%
\pgfsetfillcolor{textcolor}%
\pgftext[x=7.250538in,y=0.288889in,,top]{\color{textcolor}\sffamily\fontsize{10.000000}{12.000000}\selectfont 6}%
\end{pgfscope}%
\begin{pgfscope}%
\pgfsetbuttcap%
\pgfsetroundjoin%
\definecolor{currentfill}{rgb}{0.000000,0.000000,0.000000}%
\pgfsetfillcolor{currentfill}%
\pgfsetlinewidth{0.803000pt}%
\definecolor{currentstroke}{rgb}{0.000000,0.000000,0.000000}%
\pgfsetstrokecolor{currentstroke}%
\pgfsetdash{}{0pt}%
\pgfsys@defobject{currentmarker}{\pgfqpoint{-0.048611in}{0.000000in}}{\pgfqpoint{0.000000in}{0.000000in}}{%
\pgfpathmoveto{\pgfqpoint{0.000000in}{0.000000in}}%
\pgfpathlineto{\pgfqpoint{-0.048611in}{0.000000in}}%
\pgfusepath{stroke,fill}%
}%
\begin{pgfscope}%
\pgfsys@transformshift{4.394444in}{0.420327in}%
\pgfsys@useobject{currentmarker}{}%
\end{pgfscope}%
\end{pgfscope}%
\begin{pgfscope}%
\definecolor{textcolor}{rgb}{0.000000,0.000000,0.000000}%
\pgfsetstrokecolor{textcolor}%
\pgfsetfillcolor{textcolor}%
\pgftext[x=4.076343in,y=0.367565in,left,base]{\color{textcolor}\sffamily\fontsize{10.000000}{12.000000}\selectfont 0.0}%
\end{pgfscope}%
\begin{pgfscope}%
\pgfsetbuttcap%
\pgfsetroundjoin%
\definecolor{currentfill}{rgb}{0.000000,0.000000,0.000000}%
\pgfsetfillcolor{currentfill}%
\pgfsetlinewidth{0.803000pt}%
\definecolor{currentstroke}{rgb}{0.000000,0.000000,0.000000}%
\pgfsetstrokecolor{currentstroke}%
\pgfsetdash{}{0pt}%
\pgfsys@defobject{currentmarker}{\pgfqpoint{-0.048611in}{0.000000in}}{\pgfqpoint{0.000000in}{0.000000in}}{%
\pgfpathmoveto{\pgfqpoint{0.000000in}{0.000000in}}%
\pgfpathlineto{\pgfqpoint{-0.048611in}{0.000000in}}%
\pgfusepath{stroke,fill}%
}%
\begin{pgfscope}%
\pgfsys@transformshift{4.394444in}{0.824611in}%
\pgfsys@useobject{currentmarker}{}%
\end{pgfscope}%
\end{pgfscope}%
\begin{pgfscope}%
\definecolor{textcolor}{rgb}{0.000000,0.000000,0.000000}%
\pgfsetstrokecolor{textcolor}%
\pgfsetfillcolor{textcolor}%
\pgftext[x=4.076343in,y=0.771849in,left,base]{\color{textcolor}\sffamily\fontsize{10.000000}{12.000000}\selectfont 0.5}%
\end{pgfscope}%
\begin{pgfscope}%
\pgfsetbuttcap%
\pgfsetroundjoin%
\definecolor{currentfill}{rgb}{0.000000,0.000000,0.000000}%
\pgfsetfillcolor{currentfill}%
\pgfsetlinewidth{0.803000pt}%
\definecolor{currentstroke}{rgb}{0.000000,0.000000,0.000000}%
\pgfsetstrokecolor{currentstroke}%
\pgfsetdash{}{0pt}%
\pgfsys@defobject{currentmarker}{\pgfqpoint{-0.048611in}{0.000000in}}{\pgfqpoint{0.000000in}{0.000000in}}{%
\pgfpathmoveto{\pgfqpoint{0.000000in}{0.000000in}}%
\pgfpathlineto{\pgfqpoint{-0.048611in}{0.000000in}}%
\pgfusepath{stroke,fill}%
}%
\begin{pgfscope}%
\pgfsys@transformshift{4.394444in}{1.228894in}%
\pgfsys@useobject{currentmarker}{}%
\end{pgfscope}%
\end{pgfscope}%
\begin{pgfscope}%
\definecolor{textcolor}{rgb}{0.000000,0.000000,0.000000}%
\pgfsetstrokecolor{textcolor}%
\pgfsetfillcolor{textcolor}%
\pgftext[x=4.076343in,y=1.176133in,left,base]{\color{textcolor}\sffamily\fontsize{10.000000}{12.000000}\selectfont 1.0}%
\end{pgfscope}%
\begin{pgfscope}%
\pgfsetbuttcap%
\pgfsetroundjoin%
\definecolor{currentfill}{rgb}{0.000000,0.000000,0.000000}%
\pgfsetfillcolor{currentfill}%
\pgfsetlinewidth{0.803000pt}%
\definecolor{currentstroke}{rgb}{0.000000,0.000000,0.000000}%
\pgfsetstrokecolor{currentstroke}%
\pgfsetdash{}{0pt}%
\pgfsys@defobject{currentmarker}{\pgfqpoint{-0.048611in}{0.000000in}}{\pgfqpoint{0.000000in}{0.000000in}}{%
\pgfpathmoveto{\pgfqpoint{0.000000in}{0.000000in}}%
\pgfpathlineto{\pgfqpoint{-0.048611in}{0.000000in}}%
\pgfusepath{stroke,fill}%
}%
\begin{pgfscope}%
\pgfsys@transformshift{4.394444in}{1.633178in}%
\pgfsys@useobject{currentmarker}{}%
\end{pgfscope}%
\end{pgfscope}%
\begin{pgfscope}%
\definecolor{textcolor}{rgb}{0.000000,0.000000,0.000000}%
\pgfsetstrokecolor{textcolor}%
\pgfsetfillcolor{textcolor}%
\pgftext[x=4.076343in,y=1.580416in,left,base]{\color{textcolor}\sffamily\fontsize{10.000000}{12.000000}\selectfont 1.5}%
\end{pgfscope}%
\begin{pgfscope}%
\pgfsetbuttcap%
\pgfsetroundjoin%
\definecolor{currentfill}{rgb}{0.000000,0.000000,0.000000}%
\pgfsetfillcolor{currentfill}%
\pgfsetlinewidth{0.803000pt}%
\definecolor{currentstroke}{rgb}{0.000000,0.000000,0.000000}%
\pgfsetstrokecolor{currentstroke}%
\pgfsetdash{}{0pt}%
\pgfsys@defobject{currentmarker}{\pgfqpoint{-0.048611in}{0.000000in}}{\pgfqpoint{0.000000in}{0.000000in}}{%
\pgfpathmoveto{\pgfqpoint{0.000000in}{0.000000in}}%
\pgfpathlineto{\pgfqpoint{-0.048611in}{0.000000in}}%
\pgfusepath{stroke,fill}%
}%
\begin{pgfscope}%
\pgfsys@transformshift{4.394444in}{2.037461in}%
\pgfsys@useobject{currentmarker}{}%
\end{pgfscope}%
\end{pgfscope}%
\begin{pgfscope}%
\definecolor{textcolor}{rgb}{0.000000,0.000000,0.000000}%
\pgfsetstrokecolor{textcolor}%
\pgfsetfillcolor{textcolor}%
\pgftext[x=4.076343in,y=1.984700in,left,base]{\color{textcolor}\sffamily\fontsize{10.000000}{12.000000}\selectfont 2.0}%
\end{pgfscope}%
\begin{pgfscope}%
\pgfsetbuttcap%
\pgfsetroundjoin%
\definecolor{currentfill}{rgb}{0.000000,0.000000,0.000000}%
\pgfsetfillcolor{currentfill}%
\pgfsetlinewidth{0.803000pt}%
\definecolor{currentstroke}{rgb}{0.000000,0.000000,0.000000}%
\pgfsetstrokecolor{currentstroke}%
\pgfsetdash{}{0pt}%
\pgfsys@defobject{currentmarker}{\pgfqpoint{-0.048611in}{0.000000in}}{\pgfqpoint{0.000000in}{0.000000in}}{%
\pgfpathmoveto{\pgfqpoint{0.000000in}{0.000000in}}%
\pgfpathlineto{\pgfqpoint{-0.048611in}{0.000000in}}%
\pgfusepath{stroke,fill}%
}%
\begin{pgfscope}%
\pgfsys@transformshift{4.394444in}{2.441745in}%
\pgfsys@useobject{currentmarker}{}%
\end{pgfscope}%
\end{pgfscope}%
\begin{pgfscope}%
\definecolor{textcolor}{rgb}{0.000000,0.000000,0.000000}%
\pgfsetstrokecolor{textcolor}%
\pgfsetfillcolor{textcolor}%
\pgftext[x=4.076343in,y=2.388983in,left,base]{\color{textcolor}\sffamily\fontsize{10.000000}{12.000000}\selectfont 2.5}%
\end{pgfscope}%
\begin{pgfscope}%
\pgfpathrectangle{\pgfqpoint{4.394444in}{0.386111in}}{\pgfqpoint{3.406944in}{2.316667in}}%
\pgfusepath{clip}%
\pgfsetrectcap%
\pgfsetroundjoin%
\pgfsetlinewidth{1.505625pt}%
\definecolor{currentstroke}{rgb}{0.121569,0.466667,0.705882}%
\pgfsetstrokecolor{currentstroke}%
\pgfsetdash{}{0pt}%
\pgfpathmoveto{\pgfqpoint{4.561088in}{0.623107in}}%
\pgfpathlineto{\pgfqpoint{4.637929in}{0.653987in}}%
\pgfpathlineto{\pgfqpoint{4.714771in}{0.686560in}}%
\pgfpathlineto{\pgfqpoint{4.791612in}{0.722221in}}%
\pgfpathlineto{\pgfqpoint{4.868454in}{0.759817in}}%
\pgfpathlineto{\pgfqpoint{4.945295in}{0.800142in}}%
\pgfpathlineto{\pgfqpoint{5.022137in}{0.842408in}}%
\pgfpathlineto{\pgfqpoint{5.098978in}{0.888077in}}%
\pgfpathlineto{\pgfqpoint{5.175819in}{0.938394in}}%
\pgfpathlineto{\pgfqpoint{5.252661in}{0.988884in}}%
\pgfpathlineto{\pgfqpoint{5.329502in}{1.043305in}}%
\pgfpathlineto{\pgfqpoint{5.406344in}{1.103553in}}%
\pgfpathlineto{\pgfqpoint{5.483185in}{1.168252in}}%
\pgfpathlineto{\pgfqpoint{5.560027in}{1.246720in}}%
\pgfpathlineto{\pgfqpoint{5.636868in}{1.330119in}}%
\pgfpathlineto{\pgfqpoint{5.713709in}{1.416957in}}%
\pgfpathlineto{\pgfqpoint{5.790551in}{1.519687in}}%
\pgfpathlineto{\pgfqpoint{5.867392in}{1.651908in}}%
\pgfpathlineto{\pgfqpoint{5.944234in}{1.826480in}}%
\pgfpathlineto{\pgfqpoint{6.021075in}{2.083008in}}%
\pgfpathlineto{\pgfqpoint{6.097917in}{2.379657in}}%
\pgfpathlineto{\pgfqpoint{6.174758in}{0.949864in}}%
\pgfpathlineto{\pgfqpoint{6.251600in}{0.796284in}}%
\pgfpathlineto{\pgfqpoint{6.328441in}{0.721902in}}%
\pgfpathlineto{\pgfqpoint{6.405282in}{0.678703in}}%
\pgfpathlineto{\pgfqpoint{6.482124in}{0.644287in}}%
\pgfpathlineto{\pgfqpoint{6.558965in}{0.622599in}}%
\pgfpathlineto{\pgfqpoint{6.635807in}{0.603119in}}%
\pgfpathlineto{\pgfqpoint{6.712648in}{0.586242in}}%
\pgfpathlineto{\pgfqpoint{6.789490in}{0.574559in}}%
\pgfpathlineto{\pgfqpoint{6.866331in}{0.562255in}}%
\pgfpathlineto{\pgfqpoint{6.943172in}{0.553256in}}%
\pgfpathlineto{\pgfqpoint{7.020014in}{0.545975in}}%
\pgfpathlineto{\pgfqpoint{7.096855in}{0.538190in}}%
\pgfpathlineto{\pgfqpoint{7.173697in}{0.530875in}}%
\pgfpathlineto{\pgfqpoint{7.250538in}{0.525412in}}%
\pgfpathlineto{\pgfqpoint{7.327380in}{0.519640in}}%
\pgfpathlineto{\pgfqpoint{7.404221in}{0.515299in}}%
\pgfpathlineto{\pgfqpoint{7.481063in}{0.510120in}}%
\pgfpathlineto{\pgfqpoint{7.557904in}{0.505726in}}%
\pgfpathlineto{\pgfqpoint{7.634745in}{0.502410in}}%
\pgfusepath{stroke}%
\end{pgfscope}%
\begin{pgfscope}%
\pgfsetrectcap%
\pgfsetmiterjoin%
\pgfsetlinewidth{0.803000pt}%
\definecolor{currentstroke}{rgb}{0.000000,0.000000,0.000000}%
\pgfsetstrokecolor{currentstroke}%
\pgfsetdash{}{0pt}%
\pgfpathmoveto{\pgfqpoint{4.394444in}{0.386111in}}%
\pgfpathlineto{\pgfqpoint{4.394444in}{2.702778in}}%
\pgfusepath{stroke}%
\end{pgfscope}%
\begin{pgfscope}%
\pgfsetrectcap%
\pgfsetmiterjoin%
\pgfsetlinewidth{0.803000pt}%
\definecolor{currentstroke}{rgb}{0.000000,0.000000,0.000000}%
\pgfsetstrokecolor{currentstroke}%
\pgfsetdash{}{0pt}%
\pgfpathmoveto{\pgfqpoint{7.801389in}{0.386111in}}%
\pgfpathlineto{\pgfqpoint{7.801389in}{2.702778in}}%
\pgfusepath{stroke}%
\end{pgfscope}%
\begin{pgfscope}%
\pgfsetrectcap%
\pgfsetmiterjoin%
\pgfsetlinewidth{0.803000pt}%
\definecolor{currentstroke}{rgb}{0.000000,0.000000,0.000000}%
\pgfsetstrokecolor{currentstroke}%
\pgfsetdash{}{0pt}%
\pgfpathmoveto{\pgfqpoint{4.394444in}{0.386111in}}%
\pgfpathlineto{\pgfqpoint{7.801389in}{0.386111in}}%
\pgfusepath{stroke}%
\end{pgfscope}%
\begin{pgfscope}%
\pgfsetrectcap%
\pgfsetmiterjoin%
\pgfsetlinewidth{0.803000pt}%
\definecolor{currentstroke}{rgb}{0.000000,0.000000,0.000000}%
\pgfsetstrokecolor{currentstroke}%
\pgfsetdash{}{0pt}%
\pgfpathmoveto{\pgfqpoint{4.394444in}{2.702778in}}%
\pgfpathlineto{\pgfqpoint{7.801389in}{2.702778in}}%
\pgfusepath{stroke}%
\end{pgfscope}%
\begin{pgfscope}%
\definecolor{textcolor}{rgb}{0.000000,0.000000,0.000000}%
\pgfsetstrokecolor{textcolor}%
\pgfsetfillcolor{textcolor}%
\pgftext[x=6.097917in,y=2.786111in,,base]{\color{textcolor}\sffamily\fontsize{12.000000}{14.400000}\selectfont \(\displaystyle  N = 24 \)}%
\end{pgfscope}%
\end{pgfpicture}%
\makeatother%
\endgroup%
}
\caption{Specific heat $c$ for different temperatures $T$ for the 3-D problem using kinetic Monte Carlo algorithm}
\label{Fig:CapSmall3D}
\end{figure}

\begin{figure}[htbp]
\centering
\scalebox{0.666}{%% Creator: Matplotlib, PGF backend
%%
%% To include the figure in your LaTeX document, write
%%   \input{<filename>.pgf}
%%
%% Make sure the required packages are loaded in your preamble
%%   \usepackage{pgf}
%%
%% Figures using additional raster images can only be included by \input if
%% they are in the same directory as the main LaTeX file. For loading figures
%% from other directories you can use the `import` package
%%   \usepackage{import}
%% and then include the figures with
%%   \import{<path to file>}{<filename>.pgf}
%%
%% Matplotlib used the following preamble
%%   \usepackage{fontspec}
%%   \setmainfont{DejaVuSerif.ttf}[Path=/home/lzh/anaconda3/envs/numana/lib/python3.7/site-packages/matplotlib/mpl-data/fonts/ttf/]
%%   \setsansfont{DejaVuSans.ttf}[Path=/home/lzh/anaconda3/envs/numana/lib/python3.7/site-packages/matplotlib/mpl-data/fonts/ttf/]
%%   \setmonofont{DejaVuSansMono.ttf}[Path=/home/lzh/anaconda3/envs/numana/lib/python3.7/site-packages/matplotlib/mpl-data/fonts/ttf/]
%%
\begingroup%
\makeatletter%
\begin{pgfpicture}%
\pgfpathrectangle{\pgfpointorigin}{\pgfqpoint{8.000000in}{6.000000in}}%
\pgfusepath{use as bounding box, clip}%
\begin{pgfscope}%
\pgfsetbuttcap%
\pgfsetmiterjoin%
\definecolor{currentfill}{rgb}{1.000000,1.000000,1.000000}%
\pgfsetfillcolor{currentfill}%
\pgfsetlinewidth{0.000000pt}%
\definecolor{currentstroke}{rgb}{1.000000,1.000000,1.000000}%
\pgfsetstrokecolor{currentstroke}%
\pgfsetdash{}{0pt}%
\pgfpathmoveto{\pgfqpoint{0.000000in}{0.000000in}}%
\pgfpathlineto{\pgfqpoint{8.000000in}{0.000000in}}%
\pgfpathlineto{\pgfqpoint{8.000000in}{6.000000in}}%
\pgfpathlineto{\pgfqpoint{0.000000in}{6.000000in}}%
\pgfpathclose%
\pgfusepath{fill}%
\end{pgfscope}%
\begin{pgfscope}%
\pgfsetbuttcap%
\pgfsetmiterjoin%
\definecolor{currentfill}{rgb}{1.000000,1.000000,1.000000}%
\pgfsetfillcolor{currentfill}%
\pgfsetlinewidth{0.000000pt}%
\definecolor{currentstroke}{rgb}{0.000000,0.000000,0.000000}%
\pgfsetstrokecolor{currentstroke}%
\pgfsetstrokeopacity{0.000000}%
\pgfsetdash{}{0pt}%
\pgfpathmoveto{\pgfqpoint{0.467222in}{3.312222in}}%
\pgfpathlineto{\pgfqpoint{3.890000in}{3.312222in}}%
\pgfpathlineto{\pgfqpoint{3.890000in}{5.636667in}}%
\pgfpathlineto{\pgfqpoint{0.467222in}{5.636667in}}%
\pgfpathclose%
\pgfusepath{fill}%
\end{pgfscope}%
\begin{pgfscope}%
\pgfpathrectangle{\pgfqpoint{0.467222in}{3.312222in}}{\pgfqpoint{3.422778in}{2.324444in}}%
\pgfusepath{clip}%
\pgfsetbuttcap%
\pgfsetroundjoin%
\definecolor{currentfill}{rgb}{0.121569,0.466667,0.705882}%
\pgfsetfillcolor{currentfill}%
\pgfsetfillopacity{0.300000}%
\pgfsetlinewidth{1.003750pt}%
\definecolor{currentstroke}{rgb}{0.121569,0.466667,0.705882}%
\pgfsetstrokecolor{currentstroke}%
\pgfsetstrokeopacity{0.300000}%
\pgfsetdash{}{0pt}%
\pgfpathmoveto{\pgfqpoint{0.631344in}{5.523256in}}%
\pgfpathlineto{\pgfqpoint{0.631344in}{5.523224in}}%
\pgfpathlineto{\pgfqpoint{0.708707in}{5.511911in}}%
\pgfpathlineto{\pgfqpoint{0.786071in}{5.498698in}}%
\pgfpathlineto{\pgfqpoint{0.863434in}{5.483386in}}%
\pgfpathlineto{\pgfqpoint{0.940797in}{5.465788in}}%
\pgfpathlineto{\pgfqpoint{1.018161in}{5.445641in}}%
\pgfpathlineto{\pgfqpoint{1.095524in}{5.422685in}}%
\pgfpathlineto{\pgfqpoint{1.172887in}{5.396463in}}%
\pgfpathlineto{\pgfqpoint{1.250251in}{5.366657in}}%
\pgfpathlineto{\pgfqpoint{1.327614in}{5.332735in}}%
\pgfpathlineto{\pgfqpoint{1.404978in}{5.294211in}}%
\pgfpathlineto{\pgfqpoint{1.482341in}{5.250213in}}%
\pgfpathlineto{\pgfqpoint{1.559704in}{5.199614in}}%
\pgfpathlineto{\pgfqpoint{1.637068in}{5.141192in}}%
\pgfpathlineto{\pgfqpoint{1.714431in}{5.073182in}}%
\pgfpathlineto{\pgfqpoint{1.791794in}{4.991554in}}%
\pgfpathlineto{\pgfqpoint{1.869158in}{4.893183in}}%
\pgfpathlineto{\pgfqpoint{1.946521in}{4.767192in}}%
\pgfpathlineto{\pgfqpoint{2.023884in}{4.590977in}}%
\pgfpathlineto{\pgfqpoint{2.101248in}{4.373217in}}%
\pgfpathlineto{\pgfqpoint{2.178611in}{4.139956in}}%
\pgfpathlineto{\pgfqpoint{2.255974in}{3.948533in}}%
\pgfpathlineto{\pgfqpoint{2.333338in}{3.812011in}}%
\pgfpathlineto{\pgfqpoint{2.410701in}{3.721892in}}%
\pgfpathlineto{\pgfqpoint{2.488065in}{3.653988in}}%
\pgfpathlineto{\pgfqpoint{2.565428in}{3.608017in}}%
\pgfpathlineto{\pgfqpoint{2.642791in}{3.574012in}}%
\pgfpathlineto{\pgfqpoint{2.720155in}{3.548105in}}%
\pgfpathlineto{\pgfqpoint{2.797518in}{3.527259in}}%
\pgfpathlineto{\pgfqpoint{2.874881in}{3.509846in}}%
\pgfpathlineto{\pgfqpoint{2.952245in}{3.495521in}}%
\pgfpathlineto{\pgfqpoint{3.029608in}{3.483189in}}%
\pgfpathlineto{\pgfqpoint{3.106971in}{3.473709in}}%
\pgfpathlineto{\pgfqpoint{3.184335in}{3.465040in}}%
\pgfpathlineto{\pgfqpoint{3.261698in}{3.457193in}}%
\pgfpathlineto{\pgfqpoint{3.339061in}{3.449884in}}%
\pgfpathlineto{\pgfqpoint{3.416425in}{3.444406in}}%
\pgfpathlineto{\pgfqpoint{3.493788in}{3.438648in}}%
\pgfpathlineto{\pgfqpoint{3.571152in}{3.434271in}}%
\pgfpathlineto{\pgfqpoint{3.648515in}{3.429407in}}%
\pgfpathlineto{\pgfqpoint{3.725878in}{3.425515in}}%
\pgfpathlineto{\pgfqpoint{3.725878in}{3.425783in}}%
\pgfpathlineto{\pgfqpoint{3.725878in}{3.425783in}}%
\pgfpathlineto{\pgfqpoint{3.648515in}{3.429740in}}%
\pgfpathlineto{\pgfqpoint{3.571152in}{3.434977in}}%
\pgfpathlineto{\pgfqpoint{3.493788in}{3.439720in}}%
\pgfpathlineto{\pgfqpoint{3.416425in}{3.446007in}}%
\pgfpathlineto{\pgfqpoint{3.339061in}{3.451657in}}%
\pgfpathlineto{\pgfqpoint{3.261698in}{3.457979in}}%
\pgfpathlineto{\pgfqpoint{3.184335in}{3.466154in}}%
\pgfpathlineto{\pgfqpoint{3.106971in}{3.474982in}}%
\pgfpathlineto{\pgfqpoint{3.029608in}{3.485685in}}%
\pgfpathlineto{\pgfqpoint{2.952245in}{3.497517in}}%
\pgfpathlineto{\pgfqpoint{2.874881in}{3.512217in}}%
\pgfpathlineto{\pgfqpoint{2.797518in}{3.529761in}}%
\pgfpathlineto{\pgfqpoint{2.720155in}{3.550466in}}%
\pgfpathlineto{\pgfqpoint{2.642791in}{3.576034in}}%
\pgfpathlineto{\pgfqpoint{2.565428in}{3.610904in}}%
\pgfpathlineto{\pgfqpoint{2.488065in}{3.656310in}}%
\pgfpathlineto{\pgfqpoint{2.410701in}{3.723070in}}%
\pgfpathlineto{\pgfqpoint{2.333338in}{3.816197in}}%
\pgfpathlineto{\pgfqpoint{2.255974in}{3.951914in}}%
\pgfpathlineto{\pgfqpoint{2.178611in}{4.146083in}}%
\pgfpathlineto{\pgfqpoint{2.101248in}{4.378561in}}%
\pgfpathlineto{\pgfqpoint{2.023884in}{4.597040in}}%
\pgfpathlineto{\pgfqpoint{1.946521in}{4.767547in}}%
\pgfpathlineto{\pgfqpoint{1.869158in}{4.895061in}}%
\pgfpathlineto{\pgfqpoint{1.791794in}{4.993381in}}%
\pgfpathlineto{\pgfqpoint{1.714431in}{5.073921in}}%
\pgfpathlineto{\pgfqpoint{1.637068in}{5.141844in}}%
\pgfpathlineto{\pgfqpoint{1.559704in}{5.199980in}}%
\pgfpathlineto{\pgfqpoint{1.482341in}{5.250528in}}%
\pgfpathlineto{\pgfqpoint{1.404978in}{5.294565in}}%
\pgfpathlineto{\pgfqpoint{1.327614in}{5.332916in}}%
\pgfpathlineto{\pgfqpoint{1.250251in}{5.366794in}}%
\pgfpathlineto{\pgfqpoint{1.172887in}{5.396612in}}%
\pgfpathlineto{\pgfqpoint{1.095524in}{5.422752in}}%
\pgfpathlineto{\pgfqpoint{1.018161in}{5.445711in}}%
\pgfpathlineto{\pgfqpoint{0.940797in}{5.465884in}}%
\pgfpathlineto{\pgfqpoint{0.863434in}{5.483473in}}%
\pgfpathlineto{\pgfqpoint{0.786071in}{5.498752in}}%
\pgfpathlineto{\pgfqpoint{0.708707in}{5.511935in}}%
\pgfpathlineto{\pgfqpoint{0.631344in}{5.523256in}}%
\pgfpathclose%
\pgfusepath{stroke,fill}%
\end{pgfscope}%
\begin{pgfscope}%
\pgfpathrectangle{\pgfqpoint{0.467222in}{3.312222in}}{\pgfqpoint{3.422778in}{2.324444in}}%
\pgfusepath{clip}%
\pgfsetbuttcap%
\pgfsetroundjoin%
\definecolor{currentfill}{rgb}{0.121569,0.466667,0.705882}%
\pgfsetfillcolor{currentfill}%
\pgfsetlinewidth{1.003750pt}%
\definecolor{currentstroke}{rgb}{0.121569,0.466667,0.705882}%
\pgfsetstrokecolor{currentstroke}%
\pgfsetdash{}{0pt}%
\pgfsys@defobject{currentmarker}{\pgfqpoint{-0.009821in}{-0.009821in}}{\pgfqpoint{0.009821in}{0.009821in}}{%
\pgfpathmoveto{\pgfqpoint{0.000000in}{-0.009821in}}%
\pgfpathcurveto{\pgfqpoint{0.002605in}{-0.009821in}}{\pgfqpoint{0.005103in}{-0.008786in}}{\pgfqpoint{0.006944in}{-0.006944in}}%
\pgfpathcurveto{\pgfqpoint{0.008786in}{-0.005103in}}{\pgfqpoint{0.009821in}{-0.002605in}}{\pgfqpoint{0.009821in}{0.000000in}}%
\pgfpathcurveto{\pgfqpoint{0.009821in}{0.002605in}}{\pgfqpoint{0.008786in}{0.005103in}}{\pgfqpoint{0.006944in}{0.006944in}}%
\pgfpathcurveto{\pgfqpoint{0.005103in}{0.008786in}}{\pgfqpoint{0.002605in}{0.009821in}}{\pgfqpoint{0.000000in}{0.009821in}}%
\pgfpathcurveto{\pgfqpoint{-0.002605in}{0.009821in}}{\pgfqpoint{-0.005103in}{0.008786in}}{\pgfqpoint{-0.006944in}{0.006944in}}%
\pgfpathcurveto{\pgfqpoint{-0.008786in}{0.005103in}}{\pgfqpoint{-0.009821in}{0.002605in}}{\pgfqpoint{-0.009821in}{0.000000in}}%
\pgfpathcurveto{\pgfqpoint{-0.009821in}{-0.002605in}}{\pgfqpoint{-0.008786in}{-0.005103in}}{\pgfqpoint{-0.006944in}{-0.006944in}}%
\pgfpathcurveto{\pgfqpoint{-0.005103in}{-0.008786in}}{\pgfqpoint{-0.002605in}{-0.009821in}}{\pgfqpoint{0.000000in}{-0.009821in}}%
\pgfpathclose%
\pgfusepath{stroke,fill}%
}%
\begin{pgfscope}%
\pgfsys@transformshift{0.631344in}{5.523240in}%
\pgfsys@useobject{currentmarker}{}%
\end{pgfscope}%
\begin{pgfscope}%
\pgfsys@transformshift{0.708707in}{5.511923in}%
\pgfsys@useobject{currentmarker}{}%
\end{pgfscope}%
\begin{pgfscope}%
\pgfsys@transformshift{0.786071in}{5.498725in}%
\pgfsys@useobject{currentmarker}{}%
\end{pgfscope}%
\begin{pgfscope}%
\pgfsys@transformshift{0.863434in}{5.483430in}%
\pgfsys@useobject{currentmarker}{}%
\end{pgfscope}%
\begin{pgfscope}%
\pgfsys@transformshift{0.940797in}{5.465836in}%
\pgfsys@useobject{currentmarker}{}%
\end{pgfscope}%
\begin{pgfscope}%
\pgfsys@transformshift{1.018161in}{5.445676in}%
\pgfsys@useobject{currentmarker}{}%
\end{pgfscope}%
\begin{pgfscope}%
\pgfsys@transformshift{1.095524in}{5.422718in}%
\pgfsys@useobject{currentmarker}{}%
\end{pgfscope}%
\begin{pgfscope}%
\pgfsys@transformshift{1.172887in}{5.396537in}%
\pgfsys@useobject{currentmarker}{}%
\end{pgfscope}%
\begin{pgfscope}%
\pgfsys@transformshift{1.250251in}{5.366726in}%
\pgfsys@useobject{currentmarker}{}%
\end{pgfscope}%
\begin{pgfscope}%
\pgfsys@transformshift{1.327614in}{5.332825in}%
\pgfsys@useobject{currentmarker}{}%
\end{pgfscope}%
\begin{pgfscope}%
\pgfsys@transformshift{1.404978in}{5.294388in}%
\pgfsys@useobject{currentmarker}{}%
\end{pgfscope}%
\begin{pgfscope}%
\pgfsys@transformshift{1.482341in}{5.250370in}%
\pgfsys@useobject{currentmarker}{}%
\end{pgfscope}%
\begin{pgfscope}%
\pgfsys@transformshift{1.559704in}{5.199797in}%
\pgfsys@useobject{currentmarker}{}%
\end{pgfscope}%
\begin{pgfscope}%
\pgfsys@transformshift{1.637068in}{5.141518in}%
\pgfsys@useobject{currentmarker}{}%
\end{pgfscope}%
\begin{pgfscope}%
\pgfsys@transformshift{1.714431in}{5.073551in}%
\pgfsys@useobject{currentmarker}{}%
\end{pgfscope}%
\begin{pgfscope}%
\pgfsys@transformshift{1.791794in}{4.992468in}%
\pgfsys@useobject{currentmarker}{}%
\end{pgfscope}%
\begin{pgfscope}%
\pgfsys@transformshift{1.869158in}{4.894122in}%
\pgfsys@useobject{currentmarker}{}%
\end{pgfscope}%
\begin{pgfscope}%
\pgfsys@transformshift{1.946521in}{4.767370in}%
\pgfsys@useobject{currentmarker}{}%
\end{pgfscope}%
\begin{pgfscope}%
\pgfsys@transformshift{2.023884in}{4.594009in}%
\pgfsys@useobject{currentmarker}{}%
\end{pgfscope}%
\begin{pgfscope}%
\pgfsys@transformshift{2.101248in}{4.375889in}%
\pgfsys@useobject{currentmarker}{}%
\end{pgfscope}%
\begin{pgfscope}%
\pgfsys@transformshift{2.178611in}{4.143020in}%
\pgfsys@useobject{currentmarker}{}%
\end{pgfscope}%
\begin{pgfscope}%
\pgfsys@transformshift{2.255974in}{3.950223in}%
\pgfsys@useobject{currentmarker}{}%
\end{pgfscope}%
\begin{pgfscope}%
\pgfsys@transformshift{2.333338in}{3.814104in}%
\pgfsys@useobject{currentmarker}{}%
\end{pgfscope}%
\begin{pgfscope}%
\pgfsys@transformshift{2.410701in}{3.722481in}%
\pgfsys@useobject{currentmarker}{}%
\end{pgfscope}%
\begin{pgfscope}%
\pgfsys@transformshift{2.488065in}{3.655149in}%
\pgfsys@useobject{currentmarker}{}%
\end{pgfscope}%
\begin{pgfscope}%
\pgfsys@transformshift{2.565428in}{3.609461in}%
\pgfsys@useobject{currentmarker}{}%
\end{pgfscope}%
\begin{pgfscope}%
\pgfsys@transformshift{2.642791in}{3.575023in}%
\pgfsys@useobject{currentmarker}{}%
\end{pgfscope}%
\begin{pgfscope}%
\pgfsys@transformshift{2.720155in}{3.549286in}%
\pgfsys@useobject{currentmarker}{}%
\end{pgfscope}%
\begin{pgfscope}%
\pgfsys@transformshift{2.797518in}{3.528510in}%
\pgfsys@useobject{currentmarker}{}%
\end{pgfscope}%
\begin{pgfscope}%
\pgfsys@transformshift{2.874881in}{3.511032in}%
\pgfsys@useobject{currentmarker}{}%
\end{pgfscope}%
\begin{pgfscope}%
\pgfsys@transformshift{2.952245in}{3.496519in}%
\pgfsys@useobject{currentmarker}{}%
\end{pgfscope}%
\begin{pgfscope}%
\pgfsys@transformshift{3.029608in}{3.484437in}%
\pgfsys@useobject{currentmarker}{}%
\end{pgfscope}%
\begin{pgfscope}%
\pgfsys@transformshift{3.106971in}{3.474345in}%
\pgfsys@useobject{currentmarker}{}%
\end{pgfscope}%
\begin{pgfscope}%
\pgfsys@transformshift{3.184335in}{3.465597in}%
\pgfsys@useobject{currentmarker}{}%
\end{pgfscope}%
\begin{pgfscope}%
\pgfsys@transformshift{3.261698in}{3.457586in}%
\pgfsys@useobject{currentmarker}{}%
\end{pgfscope}%
\begin{pgfscope}%
\pgfsys@transformshift{3.339061in}{3.450770in}%
\pgfsys@useobject{currentmarker}{}%
\end{pgfscope}%
\begin{pgfscope}%
\pgfsys@transformshift{3.416425in}{3.445206in}%
\pgfsys@useobject{currentmarker}{}%
\end{pgfscope}%
\begin{pgfscope}%
\pgfsys@transformshift{3.493788in}{3.439184in}%
\pgfsys@useobject{currentmarker}{}%
\end{pgfscope}%
\begin{pgfscope}%
\pgfsys@transformshift{3.571152in}{3.434624in}%
\pgfsys@useobject{currentmarker}{}%
\end{pgfscope}%
\begin{pgfscope}%
\pgfsys@transformshift{3.648515in}{3.429573in}%
\pgfsys@useobject{currentmarker}{}%
\end{pgfscope}%
\begin{pgfscope}%
\pgfsys@transformshift{3.725878in}{3.425649in}%
\pgfsys@useobject{currentmarker}{}%
\end{pgfscope}%
\end{pgfscope}%
\begin{pgfscope}%
\pgfsetbuttcap%
\pgfsetroundjoin%
\definecolor{currentfill}{rgb}{0.000000,0.000000,0.000000}%
\pgfsetfillcolor{currentfill}%
\pgfsetlinewidth{0.803000pt}%
\definecolor{currentstroke}{rgb}{0.000000,0.000000,0.000000}%
\pgfsetstrokecolor{currentstroke}%
\pgfsetdash{}{0pt}%
\pgfsys@defobject{currentmarker}{\pgfqpoint{0.000000in}{-0.048611in}}{\pgfqpoint{0.000000in}{0.000000in}}{%
\pgfpathmoveto{\pgfqpoint{0.000000in}{0.000000in}}%
\pgfpathlineto{\pgfqpoint{0.000000in}{-0.048611in}}%
\pgfusepath{stroke,fill}%
}%
\begin{pgfscope}%
\pgfsys@transformshift{1.018161in}{3.312222in}%
\pgfsys@useobject{currentmarker}{}%
\end{pgfscope}%
\end{pgfscope}%
\begin{pgfscope}%
\definecolor{textcolor}{rgb}{0.000000,0.000000,0.000000}%
\pgfsetstrokecolor{textcolor}%
\pgfsetfillcolor{textcolor}%
\pgftext[x=1.018161in,y=3.215000in,,top]{\color{textcolor}\sffamily\fontsize{10.000000}{12.000000}\selectfont 3}%
\end{pgfscope}%
\begin{pgfscope}%
\pgfsetbuttcap%
\pgfsetroundjoin%
\definecolor{currentfill}{rgb}{0.000000,0.000000,0.000000}%
\pgfsetfillcolor{currentfill}%
\pgfsetlinewidth{0.803000pt}%
\definecolor{currentstroke}{rgb}{0.000000,0.000000,0.000000}%
\pgfsetstrokecolor{currentstroke}%
\pgfsetdash{}{0pt}%
\pgfsys@defobject{currentmarker}{\pgfqpoint{0.000000in}{-0.048611in}}{\pgfqpoint{0.000000in}{0.000000in}}{%
\pgfpathmoveto{\pgfqpoint{0.000000in}{0.000000in}}%
\pgfpathlineto{\pgfqpoint{0.000000in}{-0.048611in}}%
\pgfusepath{stroke,fill}%
}%
\begin{pgfscope}%
\pgfsys@transformshift{1.791794in}{3.312222in}%
\pgfsys@useobject{currentmarker}{}%
\end{pgfscope}%
\end{pgfscope}%
\begin{pgfscope}%
\definecolor{textcolor}{rgb}{0.000000,0.000000,0.000000}%
\pgfsetstrokecolor{textcolor}%
\pgfsetfillcolor{textcolor}%
\pgftext[x=1.791794in,y=3.215000in,,top]{\color{textcolor}\sffamily\fontsize{10.000000}{12.000000}\selectfont 4}%
\end{pgfscope}%
\begin{pgfscope}%
\pgfsetbuttcap%
\pgfsetroundjoin%
\definecolor{currentfill}{rgb}{0.000000,0.000000,0.000000}%
\pgfsetfillcolor{currentfill}%
\pgfsetlinewidth{0.803000pt}%
\definecolor{currentstroke}{rgb}{0.000000,0.000000,0.000000}%
\pgfsetstrokecolor{currentstroke}%
\pgfsetdash{}{0pt}%
\pgfsys@defobject{currentmarker}{\pgfqpoint{0.000000in}{-0.048611in}}{\pgfqpoint{0.000000in}{0.000000in}}{%
\pgfpathmoveto{\pgfqpoint{0.000000in}{0.000000in}}%
\pgfpathlineto{\pgfqpoint{0.000000in}{-0.048611in}}%
\pgfusepath{stroke,fill}%
}%
\begin{pgfscope}%
\pgfsys@transformshift{2.565428in}{3.312222in}%
\pgfsys@useobject{currentmarker}{}%
\end{pgfscope}%
\end{pgfscope}%
\begin{pgfscope}%
\definecolor{textcolor}{rgb}{0.000000,0.000000,0.000000}%
\pgfsetstrokecolor{textcolor}%
\pgfsetfillcolor{textcolor}%
\pgftext[x=2.565428in,y=3.215000in,,top]{\color{textcolor}\sffamily\fontsize{10.000000}{12.000000}\selectfont 5}%
\end{pgfscope}%
\begin{pgfscope}%
\pgfsetbuttcap%
\pgfsetroundjoin%
\definecolor{currentfill}{rgb}{0.000000,0.000000,0.000000}%
\pgfsetfillcolor{currentfill}%
\pgfsetlinewidth{0.803000pt}%
\definecolor{currentstroke}{rgb}{0.000000,0.000000,0.000000}%
\pgfsetstrokecolor{currentstroke}%
\pgfsetdash{}{0pt}%
\pgfsys@defobject{currentmarker}{\pgfqpoint{0.000000in}{-0.048611in}}{\pgfqpoint{0.000000in}{0.000000in}}{%
\pgfpathmoveto{\pgfqpoint{0.000000in}{0.000000in}}%
\pgfpathlineto{\pgfqpoint{0.000000in}{-0.048611in}}%
\pgfusepath{stroke,fill}%
}%
\begin{pgfscope}%
\pgfsys@transformshift{3.339061in}{3.312222in}%
\pgfsys@useobject{currentmarker}{}%
\end{pgfscope}%
\end{pgfscope}%
\begin{pgfscope}%
\definecolor{textcolor}{rgb}{0.000000,0.000000,0.000000}%
\pgfsetstrokecolor{textcolor}%
\pgfsetfillcolor{textcolor}%
\pgftext[x=3.339061in,y=3.215000in,,top]{\color{textcolor}\sffamily\fontsize{10.000000}{12.000000}\selectfont 6}%
\end{pgfscope}%
\begin{pgfscope}%
\pgfsetbuttcap%
\pgfsetroundjoin%
\definecolor{currentfill}{rgb}{0.000000,0.000000,0.000000}%
\pgfsetfillcolor{currentfill}%
\pgfsetlinewidth{0.803000pt}%
\definecolor{currentstroke}{rgb}{0.000000,0.000000,0.000000}%
\pgfsetstrokecolor{currentstroke}%
\pgfsetdash{}{0pt}%
\pgfsys@defobject{currentmarker}{\pgfqpoint{-0.048611in}{0.000000in}}{\pgfqpoint{0.000000in}{0.000000in}}{%
\pgfpathmoveto{\pgfqpoint{0.000000in}{0.000000in}}%
\pgfpathlineto{\pgfqpoint{-0.048611in}{0.000000in}}%
\pgfusepath{stroke,fill}%
}%
\begin{pgfscope}%
\pgfsys@transformshift{0.467222in}{3.714482in}%
\pgfsys@useobject{currentmarker}{}%
\end{pgfscope}%
\end{pgfscope}%
\begin{pgfscope}%
\definecolor{textcolor}{rgb}{0.000000,0.000000,0.000000}%
\pgfsetstrokecolor{textcolor}%
\pgfsetfillcolor{textcolor}%
\pgftext[x=0.149121in,y=3.661721in,left,base]{\color{textcolor}\sffamily\fontsize{10.000000}{12.000000}\selectfont 0.2}%
\end{pgfscope}%
\begin{pgfscope}%
\pgfsetbuttcap%
\pgfsetroundjoin%
\definecolor{currentfill}{rgb}{0.000000,0.000000,0.000000}%
\pgfsetfillcolor{currentfill}%
\pgfsetlinewidth{0.803000pt}%
\definecolor{currentstroke}{rgb}{0.000000,0.000000,0.000000}%
\pgfsetstrokecolor{currentstroke}%
\pgfsetdash{}{0pt}%
\pgfsys@defobject{currentmarker}{\pgfqpoint{-0.048611in}{0.000000in}}{\pgfqpoint{0.000000in}{0.000000in}}{%
\pgfpathmoveto{\pgfqpoint{0.000000in}{0.000000in}}%
\pgfpathlineto{\pgfqpoint{-0.048611in}{0.000000in}}%
\pgfusepath{stroke,fill}%
}%
\begin{pgfscope}%
\pgfsys@transformshift{0.467222in}{4.178740in}%
\pgfsys@useobject{currentmarker}{}%
\end{pgfscope}%
\end{pgfscope}%
\begin{pgfscope}%
\definecolor{textcolor}{rgb}{0.000000,0.000000,0.000000}%
\pgfsetstrokecolor{textcolor}%
\pgfsetfillcolor{textcolor}%
\pgftext[x=0.149121in,y=4.125978in,left,base]{\color{textcolor}\sffamily\fontsize{10.000000}{12.000000}\selectfont 0.4}%
\end{pgfscope}%
\begin{pgfscope}%
\pgfsetbuttcap%
\pgfsetroundjoin%
\definecolor{currentfill}{rgb}{0.000000,0.000000,0.000000}%
\pgfsetfillcolor{currentfill}%
\pgfsetlinewidth{0.803000pt}%
\definecolor{currentstroke}{rgb}{0.000000,0.000000,0.000000}%
\pgfsetstrokecolor{currentstroke}%
\pgfsetdash{}{0pt}%
\pgfsys@defobject{currentmarker}{\pgfqpoint{-0.048611in}{0.000000in}}{\pgfqpoint{0.000000in}{0.000000in}}{%
\pgfpathmoveto{\pgfqpoint{0.000000in}{0.000000in}}%
\pgfpathlineto{\pgfqpoint{-0.048611in}{0.000000in}}%
\pgfusepath{stroke,fill}%
}%
\begin{pgfscope}%
\pgfsys@transformshift{0.467222in}{4.642997in}%
\pgfsys@useobject{currentmarker}{}%
\end{pgfscope}%
\end{pgfscope}%
\begin{pgfscope}%
\definecolor{textcolor}{rgb}{0.000000,0.000000,0.000000}%
\pgfsetstrokecolor{textcolor}%
\pgfsetfillcolor{textcolor}%
\pgftext[x=0.149121in,y=4.590235in,left,base]{\color{textcolor}\sffamily\fontsize{10.000000}{12.000000}\selectfont 0.6}%
\end{pgfscope}%
\begin{pgfscope}%
\pgfsetbuttcap%
\pgfsetroundjoin%
\definecolor{currentfill}{rgb}{0.000000,0.000000,0.000000}%
\pgfsetfillcolor{currentfill}%
\pgfsetlinewidth{0.803000pt}%
\definecolor{currentstroke}{rgb}{0.000000,0.000000,0.000000}%
\pgfsetstrokecolor{currentstroke}%
\pgfsetdash{}{0pt}%
\pgfsys@defobject{currentmarker}{\pgfqpoint{-0.048611in}{0.000000in}}{\pgfqpoint{0.000000in}{0.000000in}}{%
\pgfpathmoveto{\pgfqpoint{0.000000in}{0.000000in}}%
\pgfpathlineto{\pgfqpoint{-0.048611in}{0.000000in}}%
\pgfusepath{stroke,fill}%
}%
\begin{pgfscope}%
\pgfsys@transformshift{0.467222in}{5.107254in}%
\pgfsys@useobject{currentmarker}{}%
\end{pgfscope}%
\end{pgfscope}%
\begin{pgfscope}%
\definecolor{textcolor}{rgb}{0.000000,0.000000,0.000000}%
\pgfsetstrokecolor{textcolor}%
\pgfsetfillcolor{textcolor}%
\pgftext[x=0.149121in,y=5.054492in,left,base]{\color{textcolor}\sffamily\fontsize{10.000000}{12.000000}\selectfont 0.8}%
\end{pgfscope}%
\begin{pgfscope}%
\pgfsetbuttcap%
\pgfsetroundjoin%
\definecolor{currentfill}{rgb}{0.000000,0.000000,0.000000}%
\pgfsetfillcolor{currentfill}%
\pgfsetlinewidth{0.803000pt}%
\definecolor{currentstroke}{rgb}{0.000000,0.000000,0.000000}%
\pgfsetstrokecolor{currentstroke}%
\pgfsetdash{}{0pt}%
\pgfsys@defobject{currentmarker}{\pgfqpoint{-0.048611in}{0.000000in}}{\pgfqpoint{0.000000in}{0.000000in}}{%
\pgfpathmoveto{\pgfqpoint{0.000000in}{0.000000in}}%
\pgfpathlineto{\pgfqpoint{-0.048611in}{0.000000in}}%
\pgfusepath{stroke,fill}%
}%
\begin{pgfscope}%
\pgfsys@transformshift{0.467222in}{5.571511in}%
\pgfsys@useobject{currentmarker}{}%
\end{pgfscope}%
\end{pgfscope}%
\begin{pgfscope}%
\definecolor{textcolor}{rgb}{0.000000,0.000000,0.000000}%
\pgfsetstrokecolor{textcolor}%
\pgfsetfillcolor{textcolor}%
\pgftext[x=0.149121in,y=5.518749in,left,base]{\color{textcolor}\sffamily\fontsize{10.000000}{12.000000}\selectfont 1.0}%
\end{pgfscope}%
\begin{pgfscope}%
\pgfpathrectangle{\pgfqpoint{0.467222in}{3.312222in}}{\pgfqpoint{3.422778in}{2.324444in}}%
\pgfusepath{clip}%
\pgfsetrectcap%
\pgfsetroundjoin%
\pgfsetlinewidth{1.505625pt}%
\definecolor{currentstroke}{rgb}{0.121569,0.466667,0.705882}%
\pgfsetstrokecolor{currentstroke}%
\pgfsetdash{}{0pt}%
\pgfpathmoveto{\pgfqpoint{0.631344in}{5.523240in}}%
\pgfpathlineto{\pgfqpoint{0.708707in}{5.511923in}}%
\pgfpathlineto{\pgfqpoint{0.786071in}{5.498725in}}%
\pgfpathlineto{\pgfqpoint{0.863434in}{5.483430in}}%
\pgfpathlineto{\pgfqpoint{0.940797in}{5.465836in}}%
\pgfpathlineto{\pgfqpoint{1.018161in}{5.445676in}}%
\pgfpathlineto{\pgfqpoint{1.095524in}{5.422718in}}%
\pgfpathlineto{\pgfqpoint{1.172887in}{5.396537in}}%
\pgfpathlineto{\pgfqpoint{1.250251in}{5.366726in}}%
\pgfpathlineto{\pgfqpoint{1.327614in}{5.332825in}}%
\pgfpathlineto{\pgfqpoint{1.404978in}{5.294388in}}%
\pgfpathlineto{\pgfqpoint{1.482341in}{5.250370in}}%
\pgfpathlineto{\pgfqpoint{1.559704in}{5.199797in}}%
\pgfpathlineto{\pgfqpoint{1.637068in}{5.141518in}}%
\pgfpathlineto{\pgfqpoint{1.714431in}{5.073551in}}%
\pgfpathlineto{\pgfqpoint{1.791794in}{4.992468in}}%
\pgfpathlineto{\pgfqpoint{1.869158in}{4.894122in}}%
\pgfpathlineto{\pgfqpoint{1.946521in}{4.767370in}}%
\pgfpathlineto{\pgfqpoint{2.023884in}{4.594009in}}%
\pgfpathlineto{\pgfqpoint{2.101248in}{4.375889in}}%
\pgfpathlineto{\pgfqpoint{2.178611in}{4.143020in}}%
\pgfpathlineto{\pgfqpoint{2.255974in}{3.950223in}}%
\pgfpathlineto{\pgfqpoint{2.333338in}{3.814104in}}%
\pgfpathlineto{\pgfqpoint{2.410701in}{3.722481in}}%
\pgfpathlineto{\pgfqpoint{2.488065in}{3.655149in}}%
\pgfpathlineto{\pgfqpoint{2.565428in}{3.609461in}}%
\pgfpathlineto{\pgfqpoint{2.642791in}{3.575023in}}%
\pgfpathlineto{\pgfqpoint{2.720155in}{3.549286in}}%
\pgfpathlineto{\pgfqpoint{2.797518in}{3.528510in}}%
\pgfpathlineto{\pgfqpoint{2.874881in}{3.511032in}}%
\pgfpathlineto{\pgfqpoint{2.952245in}{3.496519in}}%
\pgfpathlineto{\pgfqpoint{3.029608in}{3.484437in}}%
\pgfpathlineto{\pgfqpoint{3.106971in}{3.474345in}}%
\pgfpathlineto{\pgfqpoint{3.184335in}{3.465597in}}%
\pgfpathlineto{\pgfqpoint{3.261698in}{3.457586in}}%
\pgfpathlineto{\pgfqpoint{3.339061in}{3.450770in}}%
\pgfpathlineto{\pgfqpoint{3.416425in}{3.445206in}}%
\pgfpathlineto{\pgfqpoint{3.493788in}{3.439184in}}%
\pgfpathlineto{\pgfqpoint{3.571152in}{3.434624in}}%
\pgfpathlineto{\pgfqpoint{3.648515in}{3.429573in}}%
\pgfpathlineto{\pgfqpoint{3.725878in}{3.425649in}}%
\pgfusepath{stroke}%
\end{pgfscope}%
\begin{pgfscope}%
\pgfsetrectcap%
\pgfsetmiterjoin%
\pgfsetlinewidth{0.803000pt}%
\definecolor{currentstroke}{rgb}{0.000000,0.000000,0.000000}%
\pgfsetstrokecolor{currentstroke}%
\pgfsetdash{}{0pt}%
\pgfpathmoveto{\pgfqpoint{0.467222in}{3.312222in}}%
\pgfpathlineto{\pgfqpoint{0.467222in}{5.636667in}}%
\pgfusepath{stroke}%
\end{pgfscope}%
\begin{pgfscope}%
\pgfsetrectcap%
\pgfsetmiterjoin%
\pgfsetlinewidth{0.803000pt}%
\definecolor{currentstroke}{rgb}{0.000000,0.000000,0.000000}%
\pgfsetstrokecolor{currentstroke}%
\pgfsetdash{}{0pt}%
\pgfpathmoveto{\pgfqpoint{3.890000in}{3.312222in}}%
\pgfpathlineto{\pgfqpoint{3.890000in}{5.636667in}}%
\pgfusepath{stroke}%
\end{pgfscope}%
\begin{pgfscope}%
\pgfsetrectcap%
\pgfsetmiterjoin%
\pgfsetlinewidth{0.803000pt}%
\definecolor{currentstroke}{rgb}{0.000000,0.000000,0.000000}%
\pgfsetstrokecolor{currentstroke}%
\pgfsetdash{}{0pt}%
\pgfpathmoveto{\pgfqpoint{0.467222in}{3.312222in}}%
\pgfpathlineto{\pgfqpoint{3.890000in}{3.312222in}}%
\pgfusepath{stroke}%
\end{pgfscope}%
\begin{pgfscope}%
\pgfsetrectcap%
\pgfsetmiterjoin%
\pgfsetlinewidth{0.803000pt}%
\definecolor{currentstroke}{rgb}{0.000000,0.000000,0.000000}%
\pgfsetstrokecolor{currentstroke}%
\pgfsetdash{}{0pt}%
\pgfpathmoveto{\pgfqpoint{0.467222in}{5.636667in}}%
\pgfpathlineto{\pgfqpoint{3.890000in}{5.636667in}}%
\pgfusepath{stroke}%
\end{pgfscope}%
\begin{pgfscope}%
\definecolor{textcolor}{rgb}{0.000000,0.000000,0.000000}%
\pgfsetstrokecolor{textcolor}%
\pgfsetfillcolor{textcolor}%
\pgftext[x=2.178611in,y=5.720000in,,base]{\color{textcolor}\sffamily\fontsize{12.000000}{14.400000}\selectfont \(\displaystyle  N = 8 \)}%
\end{pgfscope}%
\begin{pgfscope}%
\pgfsetbuttcap%
\pgfsetmiterjoin%
\definecolor{currentfill}{rgb}{1.000000,1.000000,1.000000}%
\pgfsetfillcolor{currentfill}%
\pgfsetlinewidth{0.000000pt}%
\definecolor{currentstroke}{rgb}{0.000000,0.000000,0.000000}%
\pgfsetstrokecolor{currentstroke}%
\pgfsetstrokeopacity{0.000000}%
\pgfsetdash{}{0pt}%
\pgfpathmoveto{\pgfqpoint{4.392222in}{3.312222in}}%
\pgfpathlineto{\pgfqpoint{7.815000in}{3.312222in}}%
\pgfpathlineto{\pgfqpoint{7.815000in}{5.636667in}}%
\pgfpathlineto{\pgfqpoint{4.392222in}{5.636667in}}%
\pgfpathclose%
\pgfusepath{fill}%
\end{pgfscope}%
\begin{pgfscope}%
\pgfpathrectangle{\pgfqpoint{4.392222in}{3.312222in}}{\pgfqpoint{3.422778in}{2.324444in}}%
\pgfusepath{clip}%
\pgfsetbuttcap%
\pgfsetroundjoin%
\definecolor{currentfill}{rgb}{0.121569,0.466667,0.705882}%
\pgfsetfillcolor{currentfill}%
\pgfsetfillopacity{0.300000}%
\pgfsetlinewidth{1.003750pt}%
\definecolor{currentstroke}{rgb}{0.121569,0.466667,0.705882}%
\pgfsetstrokecolor{currentstroke}%
\pgfsetstrokeopacity{0.300000}%
\pgfsetdash{}{0pt}%
\pgfpathmoveto{\pgfqpoint{4.556344in}{5.523254in}}%
\pgfpathlineto{\pgfqpoint{4.556344in}{5.523224in}}%
\pgfpathlineto{\pgfqpoint{4.633707in}{5.512242in}}%
\pgfpathlineto{\pgfqpoint{4.711071in}{5.499443in}}%
\pgfpathlineto{\pgfqpoint{4.788434in}{5.484644in}}%
\pgfpathlineto{\pgfqpoint{4.865797in}{5.467665in}}%
\pgfpathlineto{\pgfqpoint{4.943161in}{5.448214in}}%
\pgfpathlineto{\pgfqpoint{5.020524in}{5.426033in}}%
\pgfpathlineto{\pgfqpoint{5.097887in}{5.400758in}}%
\pgfpathlineto{\pgfqpoint{5.175251in}{5.372107in}}%
\pgfpathlineto{\pgfqpoint{5.252614in}{5.339433in}}%
\pgfpathlineto{\pgfqpoint{5.329978in}{5.302125in}}%
\pgfpathlineto{\pgfqpoint{5.407341in}{5.259770in}}%
\pgfpathlineto{\pgfqpoint{5.484704in}{5.211068in}}%
\pgfpathlineto{\pgfqpoint{5.562068in}{5.155256in}}%
\pgfpathlineto{\pgfqpoint{5.639431in}{5.089334in}}%
\pgfpathlineto{\pgfqpoint{5.716794in}{5.011567in}}%
\pgfpathlineto{\pgfqpoint{5.794158in}{4.918312in}}%
\pgfpathlineto{\pgfqpoint{5.871521in}{4.800996in}}%
\pgfpathlineto{\pgfqpoint{5.948884in}{4.644596in}}%
\pgfpathlineto{\pgfqpoint{6.026248in}{4.394909in}}%
\pgfpathlineto{\pgfqpoint{6.103611in}{4.050161in}}%
\pgfpathlineto{\pgfqpoint{6.180974in}{3.777847in}}%
\pgfpathlineto{\pgfqpoint{6.258338in}{3.655056in}}%
\pgfpathlineto{\pgfqpoint{6.335701in}{3.587145in}}%
\pgfpathlineto{\pgfqpoint{6.413065in}{3.544432in}}%
\pgfpathlineto{\pgfqpoint{6.490428in}{3.519885in}}%
\pgfpathlineto{\pgfqpoint{6.567791in}{3.502271in}}%
\pgfpathlineto{\pgfqpoint{6.645155in}{3.488161in}}%
\pgfpathlineto{\pgfqpoint{6.722518in}{3.477215in}}%
\pgfpathlineto{\pgfqpoint{6.799881in}{3.467293in}}%
\pgfpathlineto{\pgfqpoint{6.877245in}{3.462158in}}%
\pgfpathlineto{\pgfqpoint{6.954608in}{3.454852in}}%
\pgfpathlineto{\pgfqpoint{7.031971in}{3.450696in}}%
\pgfpathlineto{\pgfqpoint{7.109335in}{3.445578in}}%
\pgfpathlineto{\pgfqpoint{7.186698in}{3.441607in}}%
\pgfpathlineto{\pgfqpoint{7.264061in}{3.437936in}}%
\pgfpathlineto{\pgfqpoint{7.341425in}{3.434765in}}%
\pgfpathlineto{\pgfqpoint{7.418788in}{3.431949in}}%
\pgfpathlineto{\pgfqpoint{7.496152in}{3.429516in}}%
\pgfpathlineto{\pgfqpoint{7.573515in}{3.427065in}}%
\pgfpathlineto{\pgfqpoint{7.650878in}{3.425161in}}%
\pgfpathlineto{\pgfqpoint{7.650878in}{3.426139in}}%
\pgfpathlineto{\pgfqpoint{7.650878in}{3.426139in}}%
\pgfpathlineto{\pgfqpoint{7.573515in}{3.427764in}}%
\pgfpathlineto{\pgfqpoint{7.496152in}{3.430190in}}%
\pgfpathlineto{\pgfqpoint{7.418788in}{3.432690in}}%
\pgfpathlineto{\pgfqpoint{7.341425in}{3.435893in}}%
\pgfpathlineto{\pgfqpoint{7.264061in}{3.439117in}}%
\pgfpathlineto{\pgfqpoint{7.186698in}{3.442624in}}%
\pgfpathlineto{\pgfqpoint{7.109335in}{3.446831in}}%
\pgfpathlineto{\pgfqpoint{7.031971in}{3.451222in}}%
\pgfpathlineto{\pgfqpoint{6.954608in}{3.457106in}}%
\pgfpathlineto{\pgfqpoint{6.877245in}{3.463498in}}%
\pgfpathlineto{\pgfqpoint{6.799881in}{3.471402in}}%
\pgfpathlineto{\pgfqpoint{6.722518in}{3.480278in}}%
\pgfpathlineto{\pgfqpoint{6.645155in}{3.490674in}}%
\pgfpathlineto{\pgfqpoint{6.567791in}{3.504746in}}%
\pgfpathlineto{\pgfqpoint{6.490428in}{3.523877in}}%
\pgfpathlineto{\pgfqpoint{6.413065in}{3.553223in}}%
\pgfpathlineto{\pgfqpoint{6.335701in}{3.594373in}}%
\pgfpathlineto{\pgfqpoint{6.258338in}{3.662677in}}%
\pgfpathlineto{\pgfqpoint{6.180974in}{3.797392in}}%
\pgfpathlineto{\pgfqpoint{6.103611in}{4.064051in}}%
\pgfpathlineto{\pgfqpoint{6.026248in}{4.407393in}}%
\pgfpathlineto{\pgfqpoint{5.948884in}{4.650667in}}%
\pgfpathlineto{\pgfqpoint{5.871521in}{4.804079in}}%
\pgfpathlineto{\pgfqpoint{5.794158in}{4.919765in}}%
\pgfpathlineto{\pgfqpoint{5.716794in}{5.012775in}}%
\pgfpathlineto{\pgfqpoint{5.639431in}{5.090168in}}%
\pgfpathlineto{\pgfqpoint{5.562068in}{5.155627in}}%
\pgfpathlineto{\pgfqpoint{5.484704in}{5.211483in}}%
\pgfpathlineto{\pgfqpoint{5.407341in}{5.260013in}}%
\pgfpathlineto{\pgfqpoint{5.329978in}{5.302518in}}%
\pgfpathlineto{\pgfqpoint{5.252614in}{5.339664in}}%
\pgfpathlineto{\pgfqpoint{5.175251in}{5.372250in}}%
\pgfpathlineto{\pgfqpoint{5.097887in}{5.400875in}}%
\pgfpathlineto{\pgfqpoint{5.020524in}{5.426130in}}%
\pgfpathlineto{\pgfqpoint{4.943161in}{5.448324in}}%
\pgfpathlineto{\pgfqpoint{4.865797in}{5.467735in}}%
\pgfpathlineto{\pgfqpoint{4.788434in}{5.484704in}}%
\pgfpathlineto{\pgfqpoint{4.711071in}{5.499506in}}%
\pgfpathlineto{\pgfqpoint{4.633707in}{5.512263in}}%
\pgfpathlineto{\pgfqpoint{4.556344in}{5.523254in}}%
\pgfpathclose%
\pgfusepath{stroke,fill}%
\end{pgfscope}%
\begin{pgfscope}%
\pgfpathrectangle{\pgfqpoint{4.392222in}{3.312222in}}{\pgfqpoint{3.422778in}{2.324444in}}%
\pgfusepath{clip}%
\pgfsetbuttcap%
\pgfsetroundjoin%
\definecolor{currentfill}{rgb}{0.121569,0.466667,0.705882}%
\pgfsetfillcolor{currentfill}%
\pgfsetlinewidth{1.003750pt}%
\definecolor{currentstroke}{rgb}{0.121569,0.466667,0.705882}%
\pgfsetstrokecolor{currentstroke}%
\pgfsetdash{}{0pt}%
\pgfsys@defobject{currentmarker}{\pgfqpoint{-0.009821in}{-0.009821in}}{\pgfqpoint{0.009821in}{0.009821in}}{%
\pgfpathmoveto{\pgfqpoint{0.000000in}{-0.009821in}}%
\pgfpathcurveto{\pgfqpoint{0.002605in}{-0.009821in}}{\pgfqpoint{0.005103in}{-0.008786in}}{\pgfqpoint{0.006944in}{-0.006944in}}%
\pgfpathcurveto{\pgfqpoint{0.008786in}{-0.005103in}}{\pgfqpoint{0.009821in}{-0.002605in}}{\pgfqpoint{0.009821in}{0.000000in}}%
\pgfpathcurveto{\pgfqpoint{0.009821in}{0.002605in}}{\pgfqpoint{0.008786in}{0.005103in}}{\pgfqpoint{0.006944in}{0.006944in}}%
\pgfpathcurveto{\pgfqpoint{0.005103in}{0.008786in}}{\pgfqpoint{0.002605in}{0.009821in}}{\pgfqpoint{0.000000in}{0.009821in}}%
\pgfpathcurveto{\pgfqpoint{-0.002605in}{0.009821in}}{\pgfqpoint{-0.005103in}{0.008786in}}{\pgfqpoint{-0.006944in}{0.006944in}}%
\pgfpathcurveto{\pgfqpoint{-0.008786in}{0.005103in}}{\pgfqpoint{-0.009821in}{0.002605in}}{\pgfqpoint{-0.009821in}{0.000000in}}%
\pgfpathcurveto{\pgfqpoint{-0.009821in}{-0.002605in}}{\pgfqpoint{-0.008786in}{-0.005103in}}{\pgfqpoint{-0.006944in}{-0.006944in}}%
\pgfpathcurveto{\pgfqpoint{-0.005103in}{-0.008786in}}{\pgfqpoint{-0.002605in}{-0.009821in}}{\pgfqpoint{0.000000in}{-0.009821in}}%
\pgfpathclose%
\pgfusepath{stroke,fill}%
}%
\begin{pgfscope}%
\pgfsys@transformshift{4.556344in}{5.523239in}%
\pgfsys@useobject{currentmarker}{}%
\end{pgfscope}%
\begin{pgfscope}%
\pgfsys@transformshift{4.633707in}{5.512253in}%
\pgfsys@useobject{currentmarker}{}%
\end{pgfscope}%
\begin{pgfscope}%
\pgfsys@transformshift{4.711071in}{5.499475in}%
\pgfsys@useobject{currentmarker}{}%
\end{pgfscope}%
\begin{pgfscope}%
\pgfsys@transformshift{4.788434in}{5.484674in}%
\pgfsys@useobject{currentmarker}{}%
\end{pgfscope}%
\begin{pgfscope}%
\pgfsys@transformshift{4.865797in}{5.467700in}%
\pgfsys@useobject{currentmarker}{}%
\end{pgfscope}%
\begin{pgfscope}%
\pgfsys@transformshift{4.943161in}{5.448269in}%
\pgfsys@useobject{currentmarker}{}%
\end{pgfscope}%
\begin{pgfscope}%
\pgfsys@transformshift{5.020524in}{5.426082in}%
\pgfsys@useobject{currentmarker}{}%
\end{pgfscope}%
\begin{pgfscope}%
\pgfsys@transformshift{5.097887in}{5.400816in}%
\pgfsys@useobject{currentmarker}{}%
\end{pgfscope}%
\begin{pgfscope}%
\pgfsys@transformshift{5.175251in}{5.372178in}%
\pgfsys@useobject{currentmarker}{}%
\end{pgfscope}%
\begin{pgfscope}%
\pgfsys@transformshift{5.252614in}{5.339549in}%
\pgfsys@useobject{currentmarker}{}%
\end{pgfscope}%
\begin{pgfscope}%
\pgfsys@transformshift{5.329978in}{5.302321in}%
\pgfsys@useobject{currentmarker}{}%
\end{pgfscope}%
\begin{pgfscope}%
\pgfsys@transformshift{5.407341in}{5.259891in}%
\pgfsys@useobject{currentmarker}{}%
\end{pgfscope}%
\begin{pgfscope}%
\pgfsys@transformshift{5.484704in}{5.211276in}%
\pgfsys@useobject{currentmarker}{}%
\end{pgfscope}%
\begin{pgfscope}%
\pgfsys@transformshift{5.562068in}{5.155442in}%
\pgfsys@useobject{currentmarker}{}%
\end{pgfscope}%
\begin{pgfscope}%
\pgfsys@transformshift{5.639431in}{5.089751in}%
\pgfsys@useobject{currentmarker}{}%
\end{pgfscope}%
\begin{pgfscope}%
\pgfsys@transformshift{5.716794in}{5.012171in}%
\pgfsys@useobject{currentmarker}{}%
\end{pgfscope}%
\begin{pgfscope}%
\pgfsys@transformshift{5.794158in}{4.919038in}%
\pgfsys@useobject{currentmarker}{}%
\end{pgfscope}%
\begin{pgfscope}%
\pgfsys@transformshift{5.871521in}{4.802538in}%
\pgfsys@useobject{currentmarker}{}%
\end{pgfscope}%
\begin{pgfscope}%
\pgfsys@transformshift{5.948884in}{4.647631in}%
\pgfsys@useobject{currentmarker}{}%
\end{pgfscope}%
\begin{pgfscope}%
\pgfsys@transformshift{6.026248in}{4.401151in}%
\pgfsys@useobject{currentmarker}{}%
\end{pgfscope}%
\begin{pgfscope}%
\pgfsys@transformshift{6.103611in}{4.057106in}%
\pgfsys@useobject{currentmarker}{}%
\end{pgfscope}%
\begin{pgfscope}%
\pgfsys@transformshift{6.180974in}{3.787619in}%
\pgfsys@useobject{currentmarker}{}%
\end{pgfscope}%
\begin{pgfscope}%
\pgfsys@transformshift{6.258338in}{3.658867in}%
\pgfsys@useobject{currentmarker}{}%
\end{pgfscope}%
\begin{pgfscope}%
\pgfsys@transformshift{6.335701in}{3.590759in}%
\pgfsys@useobject{currentmarker}{}%
\end{pgfscope}%
\begin{pgfscope}%
\pgfsys@transformshift{6.413065in}{3.548827in}%
\pgfsys@useobject{currentmarker}{}%
\end{pgfscope}%
\begin{pgfscope}%
\pgfsys@transformshift{6.490428in}{3.521881in}%
\pgfsys@useobject{currentmarker}{}%
\end{pgfscope}%
\begin{pgfscope}%
\pgfsys@transformshift{6.567791in}{3.503509in}%
\pgfsys@useobject{currentmarker}{}%
\end{pgfscope}%
\begin{pgfscope}%
\pgfsys@transformshift{6.645155in}{3.489418in}%
\pgfsys@useobject{currentmarker}{}%
\end{pgfscope}%
\begin{pgfscope}%
\pgfsys@transformshift{6.722518in}{3.478747in}%
\pgfsys@useobject{currentmarker}{}%
\end{pgfscope}%
\begin{pgfscope}%
\pgfsys@transformshift{6.799881in}{3.469347in}%
\pgfsys@useobject{currentmarker}{}%
\end{pgfscope}%
\begin{pgfscope}%
\pgfsys@transformshift{6.877245in}{3.462828in}%
\pgfsys@useobject{currentmarker}{}%
\end{pgfscope}%
\begin{pgfscope}%
\pgfsys@transformshift{6.954608in}{3.455979in}%
\pgfsys@useobject{currentmarker}{}%
\end{pgfscope}%
\begin{pgfscope}%
\pgfsys@transformshift{7.031971in}{3.450959in}%
\pgfsys@useobject{currentmarker}{}%
\end{pgfscope}%
\begin{pgfscope}%
\pgfsys@transformshift{7.109335in}{3.446204in}%
\pgfsys@useobject{currentmarker}{}%
\end{pgfscope}%
\begin{pgfscope}%
\pgfsys@transformshift{7.186698in}{3.442116in}%
\pgfsys@useobject{currentmarker}{}%
\end{pgfscope}%
\begin{pgfscope}%
\pgfsys@transformshift{7.264061in}{3.438526in}%
\pgfsys@useobject{currentmarker}{}%
\end{pgfscope}%
\begin{pgfscope}%
\pgfsys@transformshift{7.341425in}{3.435329in}%
\pgfsys@useobject{currentmarker}{}%
\end{pgfscope}%
\begin{pgfscope}%
\pgfsys@transformshift{7.418788in}{3.432319in}%
\pgfsys@useobject{currentmarker}{}%
\end{pgfscope}%
\begin{pgfscope}%
\pgfsys@transformshift{7.496152in}{3.429853in}%
\pgfsys@useobject{currentmarker}{}%
\end{pgfscope}%
\begin{pgfscope}%
\pgfsys@transformshift{7.573515in}{3.427415in}%
\pgfsys@useobject{currentmarker}{}%
\end{pgfscope}%
\begin{pgfscope}%
\pgfsys@transformshift{7.650878in}{3.425650in}%
\pgfsys@useobject{currentmarker}{}%
\end{pgfscope}%
\end{pgfscope}%
\begin{pgfscope}%
\pgfsetbuttcap%
\pgfsetroundjoin%
\definecolor{currentfill}{rgb}{0.000000,0.000000,0.000000}%
\pgfsetfillcolor{currentfill}%
\pgfsetlinewidth{0.803000pt}%
\definecolor{currentstroke}{rgb}{0.000000,0.000000,0.000000}%
\pgfsetstrokecolor{currentstroke}%
\pgfsetdash{}{0pt}%
\pgfsys@defobject{currentmarker}{\pgfqpoint{0.000000in}{-0.048611in}}{\pgfqpoint{0.000000in}{0.000000in}}{%
\pgfpathmoveto{\pgfqpoint{0.000000in}{0.000000in}}%
\pgfpathlineto{\pgfqpoint{0.000000in}{-0.048611in}}%
\pgfusepath{stroke,fill}%
}%
\begin{pgfscope}%
\pgfsys@transformshift{4.943161in}{3.312222in}%
\pgfsys@useobject{currentmarker}{}%
\end{pgfscope}%
\end{pgfscope}%
\begin{pgfscope}%
\definecolor{textcolor}{rgb}{0.000000,0.000000,0.000000}%
\pgfsetstrokecolor{textcolor}%
\pgfsetfillcolor{textcolor}%
\pgftext[x=4.943161in,y=3.215000in,,top]{\color{textcolor}\sffamily\fontsize{10.000000}{12.000000}\selectfont 3}%
\end{pgfscope}%
\begin{pgfscope}%
\pgfsetbuttcap%
\pgfsetroundjoin%
\definecolor{currentfill}{rgb}{0.000000,0.000000,0.000000}%
\pgfsetfillcolor{currentfill}%
\pgfsetlinewidth{0.803000pt}%
\definecolor{currentstroke}{rgb}{0.000000,0.000000,0.000000}%
\pgfsetstrokecolor{currentstroke}%
\pgfsetdash{}{0pt}%
\pgfsys@defobject{currentmarker}{\pgfqpoint{0.000000in}{-0.048611in}}{\pgfqpoint{0.000000in}{0.000000in}}{%
\pgfpathmoveto{\pgfqpoint{0.000000in}{0.000000in}}%
\pgfpathlineto{\pgfqpoint{0.000000in}{-0.048611in}}%
\pgfusepath{stroke,fill}%
}%
\begin{pgfscope}%
\pgfsys@transformshift{5.716794in}{3.312222in}%
\pgfsys@useobject{currentmarker}{}%
\end{pgfscope}%
\end{pgfscope}%
\begin{pgfscope}%
\definecolor{textcolor}{rgb}{0.000000,0.000000,0.000000}%
\pgfsetstrokecolor{textcolor}%
\pgfsetfillcolor{textcolor}%
\pgftext[x=5.716794in,y=3.215000in,,top]{\color{textcolor}\sffamily\fontsize{10.000000}{12.000000}\selectfont 4}%
\end{pgfscope}%
\begin{pgfscope}%
\pgfsetbuttcap%
\pgfsetroundjoin%
\definecolor{currentfill}{rgb}{0.000000,0.000000,0.000000}%
\pgfsetfillcolor{currentfill}%
\pgfsetlinewidth{0.803000pt}%
\definecolor{currentstroke}{rgb}{0.000000,0.000000,0.000000}%
\pgfsetstrokecolor{currentstroke}%
\pgfsetdash{}{0pt}%
\pgfsys@defobject{currentmarker}{\pgfqpoint{0.000000in}{-0.048611in}}{\pgfqpoint{0.000000in}{0.000000in}}{%
\pgfpathmoveto{\pgfqpoint{0.000000in}{0.000000in}}%
\pgfpathlineto{\pgfqpoint{0.000000in}{-0.048611in}}%
\pgfusepath{stroke,fill}%
}%
\begin{pgfscope}%
\pgfsys@transformshift{6.490428in}{3.312222in}%
\pgfsys@useobject{currentmarker}{}%
\end{pgfscope}%
\end{pgfscope}%
\begin{pgfscope}%
\definecolor{textcolor}{rgb}{0.000000,0.000000,0.000000}%
\pgfsetstrokecolor{textcolor}%
\pgfsetfillcolor{textcolor}%
\pgftext[x=6.490428in,y=3.215000in,,top]{\color{textcolor}\sffamily\fontsize{10.000000}{12.000000}\selectfont 5}%
\end{pgfscope}%
\begin{pgfscope}%
\pgfsetbuttcap%
\pgfsetroundjoin%
\definecolor{currentfill}{rgb}{0.000000,0.000000,0.000000}%
\pgfsetfillcolor{currentfill}%
\pgfsetlinewidth{0.803000pt}%
\definecolor{currentstroke}{rgb}{0.000000,0.000000,0.000000}%
\pgfsetstrokecolor{currentstroke}%
\pgfsetdash{}{0pt}%
\pgfsys@defobject{currentmarker}{\pgfqpoint{0.000000in}{-0.048611in}}{\pgfqpoint{0.000000in}{0.000000in}}{%
\pgfpathmoveto{\pgfqpoint{0.000000in}{0.000000in}}%
\pgfpathlineto{\pgfqpoint{0.000000in}{-0.048611in}}%
\pgfusepath{stroke,fill}%
}%
\begin{pgfscope}%
\pgfsys@transformshift{7.264061in}{3.312222in}%
\pgfsys@useobject{currentmarker}{}%
\end{pgfscope}%
\end{pgfscope}%
\begin{pgfscope}%
\definecolor{textcolor}{rgb}{0.000000,0.000000,0.000000}%
\pgfsetstrokecolor{textcolor}%
\pgfsetfillcolor{textcolor}%
\pgftext[x=7.264061in,y=3.215000in,,top]{\color{textcolor}\sffamily\fontsize{10.000000}{12.000000}\selectfont 6}%
\end{pgfscope}%
\begin{pgfscope}%
\pgfsetbuttcap%
\pgfsetroundjoin%
\definecolor{currentfill}{rgb}{0.000000,0.000000,0.000000}%
\pgfsetfillcolor{currentfill}%
\pgfsetlinewidth{0.803000pt}%
\definecolor{currentstroke}{rgb}{0.000000,0.000000,0.000000}%
\pgfsetstrokecolor{currentstroke}%
\pgfsetdash{}{0pt}%
\pgfsys@defobject{currentmarker}{\pgfqpoint{-0.048611in}{0.000000in}}{\pgfqpoint{0.000000in}{0.000000in}}{%
\pgfpathmoveto{\pgfqpoint{0.000000in}{0.000000in}}%
\pgfpathlineto{\pgfqpoint{-0.048611in}{0.000000in}}%
\pgfusepath{stroke,fill}%
}%
\begin{pgfscope}%
\pgfsys@transformshift{4.392222in}{3.333563in}%
\pgfsys@useobject{currentmarker}{}%
\end{pgfscope}%
\end{pgfscope}%
\begin{pgfscope}%
\definecolor{textcolor}{rgb}{0.000000,0.000000,0.000000}%
\pgfsetstrokecolor{textcolor}%
\pgfsetfillcolor{textcolor}%
\pgftext[x=4.074121in,y=3.280801in,left,base]{\color{textcolor}\sffamily\fontsize{10.000000}{12.000000}\selectfont 0.0}%
\end{pgfscope}%
\begin{pgfscope}%
\pgfsetbuttcap%
\pgfsetroundjoin%
\definecolor{currentfill}{rgb}{0.000000,0.000000,0.000000}%
\pgfsetfillcolor{currentfill}%
\pgfsetlinewidth{0.803000pt}%
\definecolor{currentstroke}{rgb}{0.000000,0.000000,0.000000}%
\pgfsetstrokecolor{currentstroke}%
\pgfsetdash{}{0pt}%
\pgfsys@defobject{currentmarker}{\pgfqpoint{-0.048611in}{0.000000in}}{\pgfqpoint{0.000000in}{0.000000in}}{%
\pgfpathmoveto{\pgfqpoint{0.000000in}{0.000000in}}%
\pgfpathlineto{\pgfqpoint{-0.048611in}{0.000000in}}%
\pgfusepath{stroke,fill}%
}%
\begin{pgfscope}%
\pgfsys@transformshift{4.392222in}{3.780698in}%
\pgfsys@useobject{currentmarker}{}%
\end{pgfscope}%
\end{pgfscope}%
\begin{pgfscope}%
\definecolor{textcolor}{rgb}{0.000000,0.000000,0.000000}%
\pgfsetstrokecolor{textcolor}%
\pgfsetfillcolor{textcolor}%
\pgftext[x=4.074121in,y=3.727936in,left,base]{\color{textcolor}\sffamily\fontsize{10.000000}{12.000000}\selectfont 0.2}%
\end{pgfscope}%
\begin{pgfscope}%
\pgfsetbuttcap%
\pgfsetroundjoin%
\definecolor{currentfill}{rgb}{0.000000,0.000000,0.000000}%
\pgfsetfillcolor{currentfill}%
\pgfsetlinewidth{0.803000pt}%
\definecolor{currentstroke}{rgb}{0.000000,0.000000,0.000000}%
\pgfsetstrokecolor{currentstroke}%
\pgfsetdash{}{0pt}%
\pgfsys@defobject{currentmarker}{\pgfqpoint{-0.048611in}{0.000000in}}{\pgfqpoint{0.000000in}{0.000000in}}{%
\pgfpathmoveto{\pgfqpoint{0.000000in}{0.000000in}}%
\pgfpathlineto{\pgfqpoint{-0.048611in}{0.000000in}}%
\pgfusepath{stroke,fill}%
}%
\begin{pgfscope}%
\pgfsys@transformshift{4.392222in}{4.227832in}%
\pgfsys@useobject{currentmarker}{}%
\end{pgfscope}%
\end{pgfscope}%
\begin{pgfscope}%
\definecolor{textcolor}{rgb}{0.000000,0.000000,0.000000}%
\pgfsetstrokecolor{textcolor}%
\pgfsetfillcolor{textcolor}%
\pgftext[x=4.074121in,y=4.175071in,left,base]{\color{textcolor}\sffamily\fontsize{10.000000}{12.000000}\selectfont 0.4}%
\end{pgfscope}%
\begin{pgfscope}%
\pgfsetbuttcap%
\pgfsetroundjoin%
\definecolor{currentfill}{rgb}{0.000000,0.000000,0.000000}%
\pgfsetfillcolor{currentfill}%
\pgfsetlinewidth{0.803000pt}%
\definecolor{currentstroke}{rgb}{0.000000,0.000000,0.000000}%
\pgfsetstrokecolor{currentstroke}%
\pgfsetdash{}{0pt}%
\pgfsys@defobject{currentmarker}{\pgfqpoint{-0.048611in}{0.000000in}}{\pgfqpoint{0.000000in}{0.000000in}}{%
\pgfpathmoveto{\pgfqpoint{0.000000in}{0.000000in}}%
\pgfpathlineto{\pgfqpoint{-0.048611in}{0.000000in}}%
\pgfusepath{stroke,fill}%
}%
\begin{pgfscope}%
\pgfsys@transformshift{4.392222in}{4.674967in}%
\pgfsys@useobject{currentmarker}{}%
\end{pgfscope}%
\end{pgfscope}%
\begin{pgfscope}%
\definecolor{textcolor}{rgb}{0.000000,0.000000,0.000000}%
\pgfsetstrokecolor{textcolor}%
\pgfsetfillcolor{textcolor}%
\pgftext[x=4.074121in,y=4.622206in,left,base]{\color{textcolor}\sffamily\fontsize{10.000000}{12.000000}\selectfont 0.6}%
\end{pgfscope}%
\begin{pgfscope}%
\pgfsetbuttcap%
\pgfsetroundjoin%
\definecolor{currentfill}{rgb}{0.000000,0.000000,0.000000}%
\pgfsetfillcolor{currentfill}%
\pgfsetlinewidth{0.803000pt}%
\definecolor{currentstroke}{rgb}{0.000000,0.000000,0.000000}%
\pgfsetstrokecolor{currentstroke}%
\pgfsetdash{}{0pt}%
\pgfsys@defobject{currentmarker}{\pgfqpoint{-0.048611in}{0.000000in}}{\pgfqpoint{0.000000in}{0.000000in}}{%
\pgfpathmoveto{\pgfqpoint{0.000000in}{0.000000in}}%
\pgfpathlineto{\pgfqpoint{-0.048611in}{0.000000in}}%
\pgfusepath{stroke,fill}%
}%
\begin{pgfscope}%
\pgfsys@transformshift{4.392222in}{5.122102in}%
\pgfsys@useobject{currentmarker}{}%
\end{pgfscope}%
\end{pgfscope}%
\begin{pgfscope}%
\definecolor{textcolor}{rgb}{0.000000,0.000000,0.000000}%
\pgfsetstrokecolor{textcolor}%
\pgfsetfillcolor{textcolor}%
\pgftext[x=4.074121in,y=5.069340in,left,base]{\color{textcolor}\sffamily\fontsize{10.000000}{12.000000}\selectfont 0.8}%
\end{pgfscope}%
\begin{pgfscope}%
\pgfsetbuttcap%
\pgfsetroundjoin%
\definecolor{currentfill}{rgb}{0.000000,0.000000,0.000000}%
\pgfsetfillcolor{currentfill}%
\pgfsetlinewidth{0.803000pt}%
\definecolor{currentstroke}{rgb}{0.000000,0.000000,0.000000}%
\pgfsetstrokecolor{currentstroke}%
\pgfsetdash{}{0pt}%
\pgfsys@defobject{currentmarker}{\pgfqpoint{-0.048611in}{0.000000in}}{\pgfqpoint{0.000000in}{0.000000in}}{%
\pgfpathmoveto{\pgfqpoint{0.000000in}{0.000000in}}%
\pgfpathlineto{\pgfqpoint{-0.048611in}{0.000000in}}%
\pgfusepath{stroke,fill}%
}%
\begin{pgfscope}%
\pgfsys@transformshift{4.392222in}{5.569237in}%
\pgfsys@useobject{currentmarker}{}%
\end{pgfscope}%
\end{pgfscope}%
\begin{pgfscope}%
\definecolor{textcolor}{rgb}{0.000000,0.000000,0.000000}%
\pgfsetstrokecolor{textcolor}%
\pgfsetfillcolor{textcolor}%
\pgftext[x=4.074121in,y=5.516475in,left,base]{\color{textcolor}\sffamily\fontsize{10.000000}{12.000000}\selectfont 1.0}%
\end{pgfscope}%
\begin{pgfscope}%
\pgfpathrectangle{\pgfqpoint{4.392222in}{3.312222in}}{\pgfqpoint{3.422778in}{2.324444in}}%
\pgfusepath{clip}%
\pgfsetrectcap%
\pgfsetroundjoin%
\pgfsetlinewidth{1.505625pt}%
\definecolor{currentstroke}{rgb}{0.121569,0.466667,0.705882}%
\pgfsetstrokecolor{currentstroke}%
\pgfsetdash{}{0pt}%
\pgfpathmoveto{\pgfqpoint{4.556344in}{5.523239in}}%
\pgfpathlineto{\pgfqpoint{4.633707in}{5.512253in}}%
\pgfpathlineto{\pgfqpoint{4.711071in}{5.499475in}}%
\pgfpathlineto{\pgfqpoint{4.788434in}{5.484674in}}%
\pgfpathlineto{\pgfqpoint{4.865797in}{5.467700in}}%
\pgfpathlineto{\pgfqpoint{4.943161in}{5.448269in}}%
\pgfpathlineto{\pgfqpoint{5.020524in}{5.426082in}}%
\pgfpathlineto{\pgfqpoint{5.097887in}{5.400816in}}%
\pgfpathlineto{\pgfqpoint{5.175251in}{5.372178in}}%
\pgfpathlineto{\pgfqpoint{5.252614in}{5.339549in}}%
\pgfpathlineto{\pgfqpoint{5.329978in}{5.302321in}}%
\pgfpathlineto{\pgfqpoint{5.407341in}{5.259891in}}%
\pgfpathlineto{\pgfqpoint{5.484704in}{5.211276in}}%
\pgfpathlineto{\pgfqpoint{5.562068in}{5.155442in}}%
\pgfpathlineto{\pgfqpoint{5.639431in}{5.089751in}}%
\pgfpathlineto{\pgfqpoint{5.716794in}{5.012171in}}%
\pgfpathlineto{\pgfqpoint{5.794158in}{4.919038in}}%
\pgfpathlineto{\pgfqpoint{5.871521in}{4.802538in}}%
\pgfpathlineto{\pgfqpoint{5.948884in}{4.647631in}}%
\pgfpathlineto{\pgfqpoint{6.026248in}{4.401151in}}%
\pgfpathlineto{\pgfqpoint{6.103611in}{4.057106in}}%
\pgfpathlineto{\pgfqpoint{6.180974in}{3.787619in}}%
\pgfpathlineto{\pgfqpoint{6.258338in}{3.658867in}}%
\pgfpathlineto{\pgfqpoint{6.335701in}{3.590759in}}%
\pgfpathlineto{\pgfqpoint{6.413065in}{3.548827in}}%
\pgfpathlineto{\pgfqpoint{6.490428in}{3.521881in}}%
\pgfpathlineto{\pgfqpoint{6.567791in}{3.503509in}}%
\pgfpathlineto{\pgfqpoint{6.645155in}{3.489418in}}%
\pgfpathlineto{\pgfqpoint{6.722518in}{3.478747in}}%
\pgfpathlineto{\pgfqpoint{6.799881in}{3.469347in}}%
\pgfpathlineto{\pgfqpoint{6.877245in}{3.462828in}}%
\pgfpathlineto{\pgfqpoint{6.954608in}{3.455979in}}%
\pgfpathlineto{\pgfqpoint{7.031971in}{3.450959in}}%
\pgfpathlineto{\pgfqpoint{7.109335in}{3.446204in}}%
\pgfpathlineto{\pgfqpoint{7.186698in}{3.442116in}}%
\pgfpathlineto{\pgfqpoint{7.264061in}{3.438526in}}%
\pgfpathlineto{\pgfqpoint{7.341425in}{3.435329in}}%
\pgfpathlineto{\pgfqpoint{7.418788in}{3.432319in}}%
\pgfpathlineto{\pgfqpoint{7.496152in}{3.429853in}}%
\pgfpathlineto{\pgfqpoint{7.573515in}{3.427415in}}%
\pgfpathlineto{\pgfqpoint{7.650878in}{3.425650in}}%
\pgfusepath{stroke}%
\end{pgfscope}%
\begin{pgfscope}%
\pgfsetrectcap%
\pgfsetmiterjoin%
\pgfsetlinewidth{0.803000pt}%
\definecolor{currentstroke}{rgb}{0.000000,0.000000,0.000000}%
\pgfsetstrokecolor{currentstroke}%
\pgfsetdash{}{0pt}%
\pgfpathmoveto{\pgfqpoint{4.392222in}{3.312222in}}%
\pgfpathlineto{\pgfqpoint{4.392222in}{5.636667in}}%
\pgfusepath{stroke}%
\end{pgfscope}%
\begin{pgfscope}%
\pgfsetrectcap%
\pgfsetmiterjoin%
\pgfsetlinewidth{0.803000pt}%
\definecolor{currentstroke}{rgb}{0.000000,0.000000,0.000000}%
\pgfsetstrokecolor{currentstroke}%
\pgfsetdash{}{0pt}%
\pgfpathmoveto{\pgfqpoint{7.815000in}{3.312222in}}%
\pgfpathlineto{\pgfqpoint{7.815000in}{5.636667in}}%
\pgfusepath{stroke}%
\end{pgfscope}%
\begin{pgfscope}%
\pgfsetrectcap%
\pgfsetmiterjoin%
\pgfsetlinewidth{0.803000pt}%
\definecolor{currentstroke}{rgb}{0.000000,0.000000,0.000000}%
\pgfsetstrokecolor{currentstroke}%
\pgfsetdash{}{0pt}%
\pgfpathmoveto{\pgfqpoint{4.392222in}{3.312222in}}%
\pgfpathlineto{\pgfqpoint{7.815000in}{3.312222in}}%
\pgfusepath{stroke}%
\end{pgfscope}%
\begin{pgfscope}%
\pgfsetrectcap%
\pgfsetmiterjoin%
\pgfsetlinewidth{0.803000pt}%
\definecolor{currentstroke}{rgb}{0.000000,0.000000,0.000000}%
\pgfsetstrokecolor{currentstroke}%
\pgfsetdash{}{0pt}%
\pgfpathmoveto{\pgfqpoint{4.392222in}{5.636667in}}%
\pgfpathlineto{\pgfqpoint{7.815000in}{5.636667in}}%
\pgfusepath{stroke}%
\end{pgfscope}%
\begin{pgfscope}%
\definecolor{textcolor}{rgb}{0.000000,0.000000,0.000000}%
\pgfsetstrokecolor{textcolor}%
\pgfsetfillcolor{textcolor}%
\pgftext[x=6.103611in,y=5.720000in,,base]{\color{textcolor}\sffamily\fontsize{12.000000}{14.400000}\selectfont \(\displaystyle  N = 12 \)}%
\end{pgfscope}%
\begin{pgfscope}%
\pgfsetbuttcap%
\pgfsetmiterjoin%
\definecolor{currentfill}{rgb}{1.000000,1.000000,1.000000}%
\pgfsetfillcolor{currentfill}%
\pgfsetlinewidth{0.000000pt}%
\definecolor{currentstroke}{rgb}{0.000000,0.000000,0.000000}%
\pgfsetstrokecolor{currentstroke}%
\pgfsetstrokeopacity{0.000000}%
\pgfsetdash{}{0pt}%
\pgfpathmoveto{\pgfqpoint{0.467222in}{0.387222in}}%
\pgfpathlineto{\pgfqpoint{3.890000in}{0.387222in}}%
\pgfpathlineto{\pgfqpoint{3.890000in}{2.711667in}}%
\pgfpathlineto{\pgfqpoint{0.467222in}{2.711667in}}%
\pgfpathclose%
\pgfusepath{fill}%
\end{pgfscope}%
\begin{pgfscope}%
\pgfpathrectangle{\pgfqpoint{0.467222in}{0.387222in}}{\pgfqpoint{3.422778in}{2.324444in}}%
\pgfusepath{clip}%
\pgfsetbuttcap%
\pgfsetroundjoin%
\definecolor{currentfill}{rgb}{0.121569,0.466667,0.705882}%
\pgfsetfillcolor{currentfill}%
\pgfsetfillopacity{0.300000}%
\pgfsetlinewidth{1.003750pt}%
\definecolor{currentstroke}{rgb}{0.121569,0.466667,0.705882}%
\pgfsetstrokecolor{currentstroke}%
\pgfsetstrokeopacity{0.300000}%
\pgfsetdash{}{0pt}%
\pgfpathmoveto{\pgfqpoint{0.631344in}{2.598259in}}%
\pgfpathlineto{\pgfqpoint{0.631344in}{2.598220in}}%
\pgfpathlineto{\pgfqpoint{0.708707in}{2.587412in}}%
\pgfpathlineto{\pgfqpoint{0.786071in}{2.574805in}}%
\pgfpathlineto{\pgfqpoint{0.863434in}{2.560229in}}%
\pgfpathlineto{\pgfqpoint{0.940797in}{2.543484in}}%
\pgfpathlineto{\pgfqpoint{1.018161in}{2.524312in}}%
\pgfpathlineto{\pgfqpoint{1.095524in}{2.502499in}}%
\pgfpathlineto{\pgfqpoint{1.172887in}{2.477581in}}%
\pgfpathlineto{\pgfqpoint{1.250251in}{2.449347in}}%
\pgfpathlineto{\pgfqpoint{1.327614in}{2.417211in}}%
\pgfpathlineto{\pgfqpoint{1.404978in}{2.380585in}}%
\pgfpathlineto{\pgfqpoint{1.482341in}{2.338662in}}%
\pgfpathlineto{\pgfqpoint{1.559704in}{2.290727in}}%
\pgfpathlineto{\pgfqpoint{1.637068in}{2.235483in}}%
\pgfpathlineto{\pgfqpoint{1.714431in}{2.171032in}}%
\pgfpathlineto{\pgfqpoint{1.791794in}{2.094677in}}%
\pgfpathlineto{\pgfqpoint{1.869158in}{2.002207in}}%
\pgfpathlineto{\pgfqpoint{1.946521in}{1.886516in}}%
\pgfpathlineto{\pgfqpoint{2.023884in}{1.735210in}}%
\pgfpathlineto{\pgfqpoint{2.101248in}{1.503813in}}%
\pgfpathlineto{\pgfqpoint{2.178611in}{1.071577in}}%
\pgfpathlineto{\pgfqpoint{2.255974in}{0.759768in}}%
\pgfpathlineto{\pgfqpoint{2.333338in}{0.646627in}}%
\pgfpathlineto{\pgfqpoint{2.410701in}{0.603443in}}%
\pgfpathlineto{\pgfqpoint{2.488065in}{0.577366in}}%
\pgfpathlineto{\pgfqpoint{2.565428in}{0.560695in}}%
\pgfpathlineto{\pgfqpoint{2.642791in}{0.548397in}}%
\pgfpathlineto{\pgfqpoint{2.720155in}{0.539915in}}%
\pgfpathlineto{\pgfqpoint{2.797518in}{0.534058in}}%
\pgfpathlineto{\pgfqpoint{2.874881in}{0.527185in}}%
\pgfpathlineto{\pgfqpoint{2.952245in}{0.523047in}}%
\pgfpathlineto{\pgfqpoint{3.029608in}{0.519493in}}%
\pgfpathlineto{\pgfqpoint{3.106971in}{0.515827in}}%
\pgfpathlineto{\pgfqpoint{3.184335in}{0.512945in}}%
\pgfpathlineto{\pgfqpoint{3.261698in}{0.511184in}}%
\pgfpathlineto{\pgfqpoint{3.339061in}{0.508533in}}%
\pgfpathlineto{\pgfqpoint{3.416425in}{0.506538in}}%
\pgfpathlineto{\pgfqpoint{3.493788in}{0.504681in}}%
\pgfpathlineto{\pgfqpoint{3.571152in}{0.502626in}}%
\pgfpathlineto{\pgfqpoint{3.648515in}{0.501508in}}%
\pgfpathlineto{\pgfqpoint{3.725878in}{0.500291in}}%
\pgfpathlineto{\pgfqpoint{3.725878in}{0.501008in}}%
\pgfpathlineto{\pgfqpoint{3.725878in}{0.501008in}}%
\pgfpathlineto{\pgfqpoint{3.648515in}{0.502598in}}%
\pgfpathlineto{\pgfqpoint{3.571152in}{0.503575in}}%
\pgfpathlineto{\pgfqpoint{3.493788in}{0.505564in}}%
\pgfpathlineto{\pgfqpoint{3.416425in}{0.507079in}}%
\pgfpathlineto{\pgfqpoint{3.339061in}{0.510049in}}%
\pgfpathlineto{\pgfqpoint{3.261698in}{0.511574in}}%
\pgfpathlineto{\pgfqpoint{3.184335in}{0.513629in}}%
\pgfpathlineto{\pgfqpoint{3.106971in}{0.517115in}}%
\pgfpathlineto{\pgfqpoint{3.029608in}{0.520943in}}%
\pgfpathlineto{\pgfqpoint{2.952245in}{0.524025in}}%
\pgfpathlineto{\pgfqpoint{2.874881in}{0.529537in}}%
\pgfpathlineto{\pgfqpoint{2.797518in}{0.536745in}}%
\pgfpathlineto{\pgfqpoint{2.720155in}{0.543050in}}%
\pgfpathlineto{\pgfqpoint{2.642791in}{0.552187in}}%
\pgfpathlineto{\pgfqpoint{2.565428in}{0.562560in}}%
\pgfpathlineto{\pgfqpoint{2.488065in}{0.581375in}}%
\pgfpathlineto{\pgfqpoint{2.410701in}{0.609564in}}%
\pgfpathlineto{\pgfqpoint{2.333338in}{0.654737in}}%
\pgfpathlineto{\pgfqpoint{2.255974in}{0.767756in}}%
\pgfpathlineto{\pgfqpoint{2.178611in}{1.080441in}}%
\pgfpathlineto{\pgfqpoint{2.101248in}{1.514235in}}%
\pgfpathlineto{\pgfqpoint{2.023884in}{1.739791in}}%
\pgfpathlineto{\pgfqpoint{1.946521in}{1.889397in}}%
\pgfpathlineto{\pgfqpoint{1.869158in}{2.004201in}}%
\pgfpathlineto{\pgfqpoint{1.791794in}{2.095215in}}%
\pgfpathlineto{\pgfqpoint{1.714431in}{2.171407in}}%
\pgfpathlineto{\pgfqpoint{1.637068in}{2.236225in}}%
\pgfpathlineto{\pgfqpoint{1.559704in}{2.291331in}}%
\pgfpathlineto{\pgfqpoint{1.482341in}{2.339012in}}%
\pgfpathlineto{\pgfqpoint{1.404978in}{2.380842in}}%
\pgfpathlineto{\pgfqpoint{1.327614in}{2.417449in}}%
\pgfpathlineto{\pgfqpoint{1.250251in}{2.449488in}}%
\pgfpathlineto{\pgfqpoint{1.172887in}{2.477764in}}%
\pgfpathlineto{\pgfqpoint{1.095524in}{2.502607in}}%
\pgfpathlineto{\pgfqpoint{1.018161in}{2.524409in}}%
\pgfpathlineto{\pgfqpoint{0.940797in}{2.543516in}}%
\pgfpathlineto{\pgfqpoint{0.863434in}{2.560264in}}%
\pgfpathlineto{\pgfqpoint{0.786071in}{2.574853in}}%
\pgfpathlineto{\pgfqpoint{0.708707in}{2.587432in}}%
\pgfpathlineto{\pgfqpoint{0.631344in}{2.598259in}}%
\pgfpathclose%
\pgfusepath{stroke,fill}%
\end{pgfscope}%
\begin{pgfscope}%
\pgfpathrectangle{\pgfqpoint{0.467222in}{0.387222in}}{\pgfqpoint{3.422778in}{2.324444in}}%
\pgfusepath{clip}%
\pgfsetbuttcap%
\pgfsetroundjoin%
\definecolor{currentfill}{rgb}{0.121569,0.466667,0.705882}%
\pgfsetfillcolor{currentfill}%
\pgfsetlinewidth{1.003750pt}%
\definecolor{currentstroke}{rgb}{0.121569,0.466667,0.705882}%
\pgfsetstrokecolor{currentstroke}%
\pgfsetdash{}{0pt}%
\pgfsys@defobject{currentmarker}{\pgfqpoint{-0.009821in}{-0.009821in}}{\pgfqpoint{0.009821in}{0.009821in}}{%
\pgfpathmoveto{\pgfqpoint{0.000000in}{-0.009821in}}%
\pgfpathcurveto{\pgfqpoint{0.002605in}{-0.009821in}}{\pgfqpoint{0.005103in}{-0.008786in}}{\pgfqpoint{0.006944in}{-0.006944in}}%
\pgfpathcurveto{\pgfqpoint{0.008786in}{-0.005103in}}{\pgfqpoint{0.009821in}{-0.002605in}}{\pgfqpoint{0.009821in}{0.000000in}}%
\pgfpathcurveto{\pgfqpoint{0.009821in}{0.002605in}}{\pgfqpoint{0.008786in}{0.005103in}}{\pgfqpoint{0.006944in}{0.006944in}}%
\pgfpathcurveto{\pgfqpoint{0.005103in}{0.008786in}}{\pgfqpoint{0.002605in}{0.009821in}}{\pgfqpoint{0.000000in}{0.009821in}}%
\pgfpathcurveto{\pgfqpoint{-0.002605in}{0.009821in}}{\pgfqpoint{-0.005103in}{0.008786in}}{\pgfqpoint{-0.006944in}{0.006944in}}%
\pgfpathcurveto{\pgfqpoint{-0.008786in}{0.005103in}}{\pgfqpoint{-0.009821in}{0.002605in}}{\pgfqpoint{-0.009821in}{0.000000in}}%
\pgfpathcurveto{\pgfqpoint{-0.009821in}{-0.002605in}}{\pgfqpoint{-0.008786in}{-0.005103in}}{\pgfqpoint{-0.006944in}{-0.006944in}}%
\pgfpathcurveto{\pgfqpoint{-0.005103in}{-0.008786in}}{\pgfqpoint{-0.002605in}{-0.009821in}}{\pgfqpoint{0.000000in}{-0.009821in}}%
\pgfpathclose%
\pgfusepath{stroke,fill}%
}%
\begin{pgfscope}%
\pgfsys@transformshift{0.631344in}{2.598239in}%
\pgfsys@useobject{currentmarker}{}%
\end{pgfscope}%
\begin{pgfscope}%
\pgfsys@transformshift{0.708707in}{2.587422in}%
\pgfsys@useobject{currentmarker}{}%
\end{pgfscope}%
\begin{pgfscope}%
\pgfsys@transformshift{0.786071in}{2.574829in}%
\pgfsys@useobject{currentmarker}{}%
\end{pgfscope}%
\begin{pgfscope}%
\pgfsys@transformshift{0.863434in}{2.560247in}%
\pgfsys@useobject{currentmarker}{}%
\end{pgfscope}%
\begin{pgfscope}%
\pgfsys@transformshift{0.940797in}{2.543500in}%
\pgfsys@useobject{currentmarker}{}%
\end{pgfscope}%
\begin{pgfscope}%
\pgfsys@transformshift{1.018161in}{2.524361in}%
\pgfsys@useobject{currentmarker}{}%
\end{pgfscope}%
\begin{pgfscope}%
\pgfsys@transformshift{1.095524in}{2.502553in}%
\pgfsys@useobject{currentmarker}{}%
\end{pgfscope}%
\begin{pgfscope}%
\pgfsys@transformshift{1.172887in}{2.477672in}%
\pgfsys@useobject{currentmarker}{}%
\end{pgfscope}%
\begin{pgfscope}%
\pgfsys@transformshift{1.250251in}{2.449418in}%
\pgfsys@useobject{currentmarker}{}%
\end{pgfscope}%
\begin{pgfscope}%
\pgfsys@transformshift{1.327614in}{2.417330in}%
\pgfsys@useobject{currentmarker}{}%
\end{pgfscope}%
\begin{pgfscope}%
\pgfsys@transformshift{1.404978in}{2.380713in}%
\pgfsys@useobject{currentmarker}{}%
\end{pgfscope}%
\begin{pgfscope}%
\pgfsys@transformshift{1.482341in}{2.338837in}%
\pgfsys@useobject{currentmarker}{}%
\end{pgfscope}%
\begin{pgfscope}%
\pgfsys@transformshift{1.559704in}{2.291029in}%
\pgfsys@useobject{currentmarker}{}%
\end{pgfscope}%
\begin{pgfscope}%
\pgfsys@transformshift{1.637068in}{2.235854in}%
\pgfsys@useobject{currentmarker}{}%
\end{pgfscope}%
\begin{pgfscope}%
\pgfsys@transformshift{1.714431in}{2.171220in}%
\pgfsys@useobject{currentmarker}{}%
\end{pgfscope}%
\begin{pgfscope}%
\pgfsys@transformshift{1.791794in}{2.094946in}%
\pgfsys@useobject{currentmarker}{}%
\end{pgfscope}%
\begin{pgfscope}%
\pgfsys@transformshift{1.869158in}{2.003204in}%
\pgfsys@useobject{currentmarker}{}%
\end{pgfscope}%
\begin{pgfscope}%
\pgfsys@transformshift{1.946521in}{1.887957in}%
\pgfsys@useobject{currentmarker}{}%
\end{pgfscope}%
\begin{pgfscope}%
\pgfsys@transformshift{2.023884in}{1.737501in}%
\pgfsys@useobject{currentmarker}{}%
\end{pgfscope}%
\begin{pgfscope}%
\pgfsys@transformshift{2.101248in}{1.509024in}%
\pgfsys@useobject{currentmarker}{}%
\end{pgfscope}%
\begin{pgfscope}%
\pgfsys@transformshift{2.178611in}{1.076009in}%
\pgfsys@useobject{currentmarker}{}%
\end{pgfscope}%
\begin{pgfscope}%
\pgfsys@transformshift{2.255974in}{0.763762in}%
\pgfsys@useobject{currentmarker}{}%
\end{pgfscope}%
\begin{pgfscope}%
\pgfsys@transformshift{2.333338in}{0.650682in}%
\pgfsys@useobject{currentmarker}{}%
\end{pgfscope}%
\begin{pgfscope}%
\pgfsys@transformshift{2.410701in}{0.606503in}%
\pgfsys@useobject{currentmarker}{}%
\end{pgfscope}%
\begin{pgfscope}%
\pgfsys@transformshift{2.488065in}{0.579370in}%
\pgfsys@useobject{currentmarker}{}%
\end{pgfscope}%
\begin{pgfscope}%
\pgfsys@transformshift{2.565428in}{0.561627in}%
\pgfsys@useobject{currentmarker}{}%
\end{pgfscope}%
\begin{pgfscope}%
\pgfsys@transformshift{2.642791in}{0.550292in}%
\pgfsys@useobject{currentmarker}{}%
\end{pgfscope}%
\begin{pgfscope}%
\pgfsys@transformshift{2.720155in}{0.541483in}%
\pgfsys@useobject{currentmarker}{}%
\end{pgfscope}%
\begin{pgfscope}%
\pgfsys@transformshift{2.797518in}{0.535402in}%
\pgfsys@useobject{currentmarker}{}%
\end{pgfscope}%
\begin{pgfscope}%
\pgfsys@transformshift{2.874881in}{0.528361in}%
\pgfsys@useobject{currentmarker}{}%
\end{pgfscope}%
\begin{pgfscope}%
\pgfsys@transformshift{2.952245in}{0.523536in}%
\pgfsys@useobject{currentmarker}{}%
\end{pgfscope}%
\begin{pgfscope}%
\pgfsys@transformshift{3.029608in}{0.520218in}%
\pgfsys@useobject{currentmarker}{}%
\end{pgfscope}%
\begin{pgfscope}%
\pgfsys@transformshift{3.106971in}{0.516471in}%
\pgfsys@useobject{currentmarker}{}%
\end{pgfscope}%
\begin{pgfscope}%
\pgfsys@transformshift{3.184335in}{0.513287in}%
\pgfsys@useobject{currentmarker}{}%
\end{pgfscope}%
\begin{pgfscope}%
\pgfsys@transformshift{3.261698in}{0.511379in}%
\pgfsys@useobject{currentmarker}{}%
\end{pgfscope}%
\begin{pgfscope}%
\pgfsys@transformshift{3.339061in}{0.509291in}%
\pgfsys@useobject{currentmarker}{}%
\end{pgfscope}%
\begin{pgfscope}%
\pgfsys@transformshift{3.416425in}{0.506808in}%
\pgfsys@useobject{currentmarker}{}%
\end{pgfscope}%
\begin{pgfscope}%
\pgfsys@transformshift{3.493788in}{0.505122in}%
\pgfsys@useobject{currentmarker}{}%
\end{pgfscope}%
\begin{pgfscope}%
\pgfsys@transformshift{3.571152in}{0.503100in}%
\pgfsys@useobject{currentmarker}{}%
\end{pgfscope}%
\begin{pgfscope}%
\pgfsys@transformshift{3.648515in}{0.502053in}%
\pgfsys@useobject{currentmarker}{}%
\end{pgfscope}%
\begin{pgfscope}%
\pgfsys@transformshift{3.725878in}{0.500649in}%
\pgfsys@useobject{currentmarker}{}%
\end{pgfscope}%
\end{pgfscope}%
\begin{pgfscope}%
\pgfsetbuttcap%
\pgfsetroundjoin%
\definecolor{currentfill}{rgb}{0.000000,0.000000,0.000000}%
\pgfsetfillcolor{currentfill}%
\pgfsetlinewidth{0.803000pt}%
\definecolor{currentstroke}{rgb}{0.000000,0.000000,0.000000}%
\pgfsetstrokecolor{currentstroke}%
\pgfsetdash{}{0pt}%
\pgfsys@defobject{currentmarker}{\pgfqpoint{0.000000in}{-0.048611in}}{\pgfqpoint{0.000000in}{0.000000in}}{%
\pgfpathmoveto{\pgfqpoint{0.000000in}{0.000000in}}%
\pgfpathlineto{\pgfqpoint{0.000000in}{-0.048611in}}%
\pgfusepath{stroke,fill}%
}%
\begin{pgfscope}%
\pgfsys@transformshift{1.018161in}{0.387222in}%
\pgfsys@useobject{currentmarker}{}%
\end{pgfscope}%
\end{pgfscope}%
\begin{pgfscope}%
\definecolor{textcolor}{rgb}{0.000000,0.000000,0.000000}%
\pgfsetstrokecolor{textcolor}%
\pgfsetfillcolor{textcolor}%
\pgftext[x=1.018161in,y=0.290000in,,top]{\color{textcolor}\sffamily\fontsize{10.000000}{12.000000}\selectfont 3}%
\end{pgfscope}%
\begin{pgfscope}%
\pgfsetbuttcap%
\pgfsetroundjoin%
\definecolor{currentfill}{rgb}{0.000000,0.000000,0.000000}%
\pgfsetfillcolor{currentfill}%
\pgfsetlinewidth{0.803000pt}%
\definecolor{currentstroke}{rgb}{0.000000,0.000000,0.000000}%
\pgfsetstrokecolor{currentstroke}%
\pgfsetdash{}{0pt}%
\pgfsys@defobject{currentmarker}{\pgfqpoint{0.000000in}{-0.048611in}}{\pgfqpoint{0.000000in}{0.000000in}}{%
\pgfpathmoveto{\pgfqpoint{0.000000in}{0.000000in}}%
\pgfpathlineto{\pgfqpoint{0.000000in}{-0.048611in}}%
\pgfusepath{stroke,fill}%
}%
\begin{pgfscope}%
\pgfsys@transformshift{1.791794in}{0.387222in}%
\pgfsys@useobject{currentmarker}{}%
\end{pgfscope}%
\end{pgfscope}%
\begin{pgfscope}%
\definecolor{textcolor}{rgb}{0.000000,0.000000,0.000000}%
\pgfsetstrokecolor{textcolor}%
\pgfsetfillcolor{textcolor}%
\pgftext[x=1.791794in,y=0.290000in,,top]{\color{textcolor}\sffamily\fontsize{10.000000}{12.000000}\selectfont 4}%
\end{pgfscope}%
\begin{pgfscope}%
\pgfsetbuttcap%
\pgfsetroundjoin%
\definecolor{currentfill}{rgb}{0.000000,0.000000,0.000000}%
\pgfsetfillcolor{currentfill}%
\pgfsetlinewidth{0.803000pt}%
\definecolor{currentstroke}{rgb}{0.000000,0.000000,0.000000}%
\pgfsetstrokecolor{currentstroke}%
\pgfsetdash{}{0pt}%
\pgfsys@defobject{currentmarker}{\pgfqpoint{0.000000in}{-0.048611in}}{\pgfqpoint{0.000000in}{0.000000in}}{%
\pgfpathmoveto{\pgfqpoint{0.000000in}{0.000000in}}%
\pgfpathlineto{\pgfqpoint{0.000000in}{-0.048611in}}%
\pgfusepath{stroke,fill}%
}%
\begin{pgfscope}%
\pgfsys@transformshift{2.565428in}{0.387222in}%
\pgfsys@useobject{currentmarker}{}%
\end{pgfscope}%
\end{pgfscope}%
\begin{pgfscope}%
\definecolor{textcolor}{rgb}{0.000000,0.000000,0.000000}%
\pgfsetstrokecolor{textcolor}%
\pgfsetfillcolor{textcolor}%
\pgftext[x=2.565428in,y=0.290000in,,top]{\color{textcolor}\sffamily\fontsize{10.000000}{12.000000}\selectfont 5}%
\end{pgfscope}%
\begin{pgfscope}%
\pgfsetbuttcap%
\pgfsetroundjoin%
\definecolor{currentfill}{rgb}{0.000000,0.000000,0.000000}%
\pgfsetfillcolor{currentfill}%
\pgfsetlinewidth{0.803000pt}%
\definecolor{currentstroke}{rgb}{0.000000,0.000000,0.000000}%
\pgfsetstrokecolor{currentstroke}%
\pgfsetdash{}{0pt}%
\pgfsys@defobject{currentmarker}{\pgfqpoint{0.000000in}{-0.048611in}}{\pgfqpoint{0.000000in}{0.000000in}}{%
\pgfpathmoveto{\pgfqpoint{0.000000in}{0.000000in}}%
\pgfpathlineto{\pgfqpoint{0.000000in}{-0.048611in}}%
\pgfusepath{stroke,fill}%
}%
\begin{pgfscope}%
\pgfsys@transformshift{3.339061in}{0.387222in}%
\pgfsys@useobject{currentmarker}{}%
\end{pgfscope}%
\end{pgfscope}%
\begin{pgfscope}%
\definecolor{textcolor}{rgb}{0.000000,0.000000,0.000000}%
\pgfsetstrokecolor{textcolor}%
\pgfsetfillcolor{textcolor}%
\pgftext[x=3.339061in,y=0.290000in,,top]{\color{textcolor}\sffamily\fontsize{10.000000}{12.000000}\selectfont 6}%
\end{pgfscope}%
\begin{pgfscope}%
\pgfsetbuttcap%
\pgfsetroundjoin%
\definecolor{currentfill}{rgb}{0.000000,0.000000,0.000000}%
\pgfsetfillcolor{currentfill}%
\pgfsetlinewidth{0.803000pt}%
\definecolor{currentstroke}{rgb}{0.000000,0.000000,0.000000}%
\pgfsetstrokecolor{currentstroke}%
\pgfsetdash{}{0pt}%
\pgfsys@defobject{currentmarker}{\pgfqpoint{-0.048611in}{0.000000in}}{\pgfqpoint{0.000000in}{0.000000in}}{%
\pgfpathmoveto{\pgfqpoint{0.000000in}{0.000000in}}%
\pgfpathlineto{\pgfqpoint{-0.048611in}{0.000000in}}%
\pgfusepath{stroke,fill}%
}%
\begin{pgfscope}%
\pgfsys@transformshift{0.467222in}{0.441788in}%
\pgfsys@useobject{currentmarker}{}%
\end{pgfscope}%
\end{pgfscope}%
\begin{pgfscope}%
\definecolor{textcolor}{rgb}{0.000000,0.000000,0.000000}%
\pgfsetstrokecolor{textcolor}%
\pgfsetfillcolor{textcolor}%
\pgftext[x=0.149121in,y=0.389026in,left,base]{\color{textcolor}\sffamily\fontsize{10.000000}{12.000000}\selectfont 0.0}%
\end{pgfscope}%
\begin{pgfscope}%
\pgfsetbuttcap%
\pgfsetroundjoin%
\definecolor{currentfill}{rgb}{0.000000,0.000000,0.000000}%
\pgfsetfillcolor{currentfill}%
\pgfsetlinewidth{0.803000pt}%
\definecolor{currentstroke}{rgb}{0.000000,0.000000,0.000000}%
\pgfsetstrokecolor{currentstroke}%
\pgfsetdash{}{0pt}%
\pgfsys@defobject{currentmarker}{\pgfqpoint{-0.048611in}{0.000000in}}{\pgfqpoint{0.000000in}{0.000000in}}{%
\pgfpathmoveto{\pgfqpoint{0.000000in}{0.000000in}}%
\pgfpathlineto{\pgfqpoint{-0.048611in}{0.000000in}}%
\pgfusepath{stroke,fill}%
}%
\begin{pgfscope}%
\pgfsys@transformshift{0.467222in}{0.882130in}%
\pgfsys@useobject{currentmarker}{}%
\end{pgfscope}%
\end{pgfscope}%
\begin{pgfscope}%
\definecolor{textcolor}{rgb}{0.000000,0.000000,0.000000}%
\pgfsetstrokecolor{textcolor}%
\pgfsetfillcolor{textcolor}%
\pgftext[x=0.149121in,y=0.829369in,left,base]{\color{textcolor}\sffamily\fontsize{10.000000}{12.000000}\selectfont 0.2}%
\end{pgfscope}%
\begin{pgfscope}%
\pgfsetbuttcap%
\pgfsetroundjoin%
\definecolor{currentfill}{rgb}{0.000000,0.000000,0.000000}%
\pgfsetfillcolor{currentfill}%
\pgfsetlinewidth{0.803000pt}%
\definecolor{currentstroke}{rgb}{0.000000,0.000000,0.000000}%
\pgfsetstrokecolor{currentstroke}%
\pgfsetdash{}{0pt}%
\pgfsys@defobject{currentmarker}{\pgfqpoint{-0.048611in}{0.000000in}}{\pgfqpoint{0.000000in}{0.000000in}}{%
\pgfpathmoveto{\pgfqpoint{0.000000in}{0.000000in}}%
\pgfpathlineto{\pgfqpoint{-0.048611in}{0.000000in}}%
\pgfusepath{stroke,fill}%
}%
\begin{pgfscope}%
\pgfsys@transformshift{0.467222in}{1.322473in}%
\pgfsys@useobject{currentmarker}{}%
\end{pgfscope}%
\end{pgfscope}%
\begin{pgfscope}%
\definecolor{textcolor}{rgb}{0.000000,0.000000,0.000000}%
\pgfsetstrokecolor{textcolor}%
\pgfsetfillcolor{textcolor}%
\pgftext[x=0.149121in,y=1.269711in,left,base]{\color{textcolor}\sffamily\fontsize{10.000000}{12.000000}\selectfont 0.4}%
\end{pgfscope}%
\begin{pgfscope}%
\pgfsetbuttcap%
\pgfsetroundjoin%
\definecolor{currentfill}{rgb}{0.000000,0.000000,0.000000}%
\pgfsetfillcolor{currentfill}%
\pgfsetlinewidth{0.803000pt}%
\definecolor{currentstroke}{rgb}{0.000000,0.000000,0.000000}%
\pgfsetstrokecolor{currentstroke}%
\pgfsetdash{}{0pt}%
\pgfsys@defobject{currentmarker}{\pgfqpoint{-0.048611in}{0.000000in}}{\pgfqpoint{0.000000in}{0.000000in}}{%
\pgfpathmoveto{\pgfqpoint{0.000000in}{0.000000in}}%
\pgfpathlineto{\pgfqpoint{-0.048611in}{0.000000in}}%
\pgfusepath{stroke,fill}%
}%
\begin{pgfscope}%
\pgfsys@transformshift{0.467222in}{1.762815in}%
\pgfsys@useobject{currentmarker}{}%
\end{pgfscope}%
\end{pgfscope}%
\begin{pgfscope}%
\definecolor{textcolor}{rgb}{0.000000,0.000000,0.000000}%
\pgfsetstrokecolor{textcolor}%
\pgfsetfillcolor{textcolor}%
\pgftext[x=0.149121in,y=1.710053in,left,base]{\color{textcolor}\sffamily\fontsize{10.000000}{12.000000}\selectfont 0.6}%
\end{pgfscope}%
\begin{pgfscope}%
\pgfsetbuttcap%
\pgfsetroundjoin%
\definecolor{currentfill}{rgb}{0.000000,0.000000,0.000000}%
\pgfsetfillcolor{currentfill}%
\pgfsetlinewidth{0.803000pt}%
\definecolor{currentstroke}{rgb}{0.000000,0.000000,0.000000}%
\pgfsetstrokecolor{currentstroke}%
\pgfsetdash{}{0pt}%
\pgfsys@defobject{currentmarker}{\pgfqpoint{-0.048611in}{0.000000in}}{\pgfqpoint{0.000000in}{0.000000in}}{%
\pgfpathmoveto{\pgfqpoint{0.000000in}{0.000000in}}%
\pgfpathlineto{\pgfqpoint{-0.048611in}{0.000000in}}%
\pgfusepath{stroke,fill}%
}%
\begin{pgfscope}%
\pgfsys@transformshift{0.467222in}{2.203157in}%
\pgfsys@useobject{currentmarker}{}%
\end{pgfscope}%
\end{pgfscope}%
\begin{pgfscope}%
\definecolor{textcolor}{rgb}{0.000000,0.000000,0.000000}%
\pgfsetstrokecolor{textcolor}%
\pgfsetfillcolor{textcolor}%
\pgftext[x=0.149121in,y=2.150396in,left,base]{\color{textcolor}\sffamily\fontsize{10.000000}{12.000000}\selectfont 0.8}%
\end{pgfscope}%
\begin{pgfscope}%
\pgfsetbuttcap%
\pgfsetroundjoin%
\definecolor{currentfill}{rgb}{0.000000,0.000000,0.000000}%
\pgfsetfillcolor{currentfill}%
\pgfsetlinewidth{0.803000pt}%
\definecolor{currentstroke}{rgb}{0.000000,0.000000,0.000000}%
\pgfsetstrokecolor{currentstroke}%
\pgfsetdash{}{0pt}%
\pgfsys@defobject{currentmarker}{\pgfqpoint{-0.048611in}{0.000000in}}{\pgfqpoint{0.000000in}{0.000000in}}{%
\pgfpathmoveto{\pgfqpoint{0.000000in}{0.000000in}}%
\pgfpathlineto{\pgfqpoint{-0.048611in}{0.000000in}}%
\pgfusepath{stroke,fill}%
}%
\begin{pgfscope}%
\pgfsys@transformshift{0.467222in}{2.643500in}%
\pgfsys@useobject{currentmarker}{}%
\end{pgfscope}%
\end{pgfscope}%
\begin{pgfscope}%
\definecolor{textcolor}{rgb}{0.000000,0.000000,0.000000}%
\pgfsetstrokecolor{textcolor}%
\pgfsetfillcolor{textcolor}%
\pgftext[x=0.149121in,y=2.590738in,left,base]{\color{textcolor}\sffamily\fontsize{10.000000}{12.000000}\selectfont 1.0}%
\end{pgfscope}%
\begin{pgfscope}%
\pgfpathrectangle{\pgfqpoint{0.467222in}{0.387222in}}{\pgfqpoint{3.422778in}{2.324444in}}%
\pgfusepath{clip}%
\pgfsetrectcap%
\pgfsetroundjoin%
\pgfsetlinewidth{1.505625pt}%
\definecolor{currentstroke}{rgb}{0.121569,0.466667,0.705882}%
\pgfsetstrokecolor{currentstroke}%
\pgfsetdash{}{0pt}%
\pgfpathmoveto{\pgfqpoint{0.631344in}{2.598239in}}%
\pgfpathlineto{\pgfqpoint{0.708707in}{2.587422in}}%
\pgfpathlineto{\pgfqpoint{0.786071in}{2.574829in}}%
\pgfpathlineto{\pgfqpoint{0.863434in}{2.560247in}}%
\pgfpathlineto{\pgfqpoint{0.940797in}{2.543500in}}%
\pgfpathlineto{\pgfqpoint{1.018161in}{2.524361in}}%
\pgfpathlineto{\pgfqpoint{1.095524in}{2.502553in}}%
\pgfpathlineto{\pgfqpoint{1.172887in}{2.477672in}}%
\pgfpathlineto{\pgfqpoint{1.250251in}{2.449418in}}%
\pgfpathlineto{\pgfqpoint{1.327614in}{2.417330in}}%
\pgfpathlineto{\pgfqpoint{1.404978in}{2.380713in}}%
\pgfpathlineto{\pgfqpoint{1.482341in}{2.338837in}}%
\pgfpathlineto{\pgfqpoint{1.559704in}{2.291029in}}%
\pgfpathlineto{\pgfqpoint{1.637068in}{2.235854in}}%
\pgfpathlineto{\pgfqpoint{1.714431in}{2.171220in}}%
\pgfpathlineto{\pgfqpoint{1.791794in}{2.094946in}}%
\pgfpathlineto{\pgfqpoint{1.869158in}{2.003204in}}%
\pgfpathlineto{\pgfqpoint{1.946521in}{1.887957in}}%
\pgfpathlineto{\pgfqpoint{2.023884in}{1.737501in}}%
\pgfpathlineto{\pgfqpoint{2.101248in}{1.509024in}}%
\pgfpathlineto{\pgfqpoint{2.178611in}{1.076009in}}%
\pgfpathlineto{\pgfqpoint{2.255974in}{0.763762in}}%
\pgfpathlineto{\pgfqpoint{2.333338in}{0.650682in}}%
\pgfpathlineto{\pgfqpoint{2.410701in}{0.606503in}}%
\pgfpathlineto{\pgfqpoint{2.488065in}{0.579370in}}%
\pgfpathlineto{\pgfqpoint{2.565428in}{0.561627in}}%
\pgfpathlineto{\pgfqpoint{2.642791in}{0.550292in}}%
\pgfpathlineto{\pgfqpoint{2.720155in}{0.541483in}}%
\pgfpathlineto{\pgfqpoint{2.797518in}{0.535402in}}%
\pgfpathlineto{\pgfqpoint{2.874881in}{0.528361in}}%
\pgfpathlineto{\pgfqpoint{2.952245in}{0.523536in}}%
\pgfpathlineto{\pgfqpoint{3.029608in}{0.520218in}}%
\pgfpathlineto{\pgfqpoint{3.106971in}{0.516471in}}%
\pgfpathlineto{\pgfqpoint{3.184335in}{0.513287in}}%
\pgfpathlineto{\pgfqpoint{3.261698in}{0.511379in}}%
\pgfpathlineto{\pgfqpoint{3.339061in}{0.509291in}}%
\pgfpathlineto{\pgfqpoint{3.416425in}{0.506808in}}%
\pgfpathlineto{\pgfqpoint{3.493788in}{0.505122in}}%
\pgfpathlineto{\pgfqpoint{3.571152in}{0.503100in}}%
\pgfpathlineto{\pgfqpoint{3.648515in}{0.502053in}}%
\pgfpathlineto{\pgfqpoint{3.725878in}{0.500649in}}%
\pgfusepath{stroke}%
\end{pgfscope}%
\begin{pgfscope}%
\pgfsetrectcap%
\pgfsetmiterjoin%
\pgfsetlinewidth{0.803000pt}%
\definecolor{currentstroke}{rgb}{0.000000,0.000000,0.000000}%
\pgfsetstrokecolor{currentstroke}%
\pgfsetdash{}{0pt}%
\pgfpathmoveto{\pgfqpoint{0.467222in}{0.387222in}}%
\pgfpathlineto{\pgfqpoint{0.467222in}{2.711667in}}%
\pgfusepath{stroke}%
\end{pgfscope}%
\begin{pgfscope}%
\pgfsetrectcap%
\pgfsetmiterjoin%
\pgfsetlinewidth{0.803000pt}%
\definecolor{currentstroke}{rgb}{0.000000,0.000000,0.000000}%
\pgfsetstrokecolor{currentstroke}%
\pgfsetdash{}{0pt}%
\pgfpathmoveto{\pgfqpoint{3.890000in}{0.387222in}}%
\pgfpathlineto{\pgfqpoint{3.890000in}{2.711667in}}%
\pgfusepath{stroke}%
\end{pgfscope}%
\begin{pgfscope}%
\pgfsetrectcap%
\pgfsetmiterjoin%
\pgfsetlinewidth{0.803000pt}%
\definecolor{currentstroke}{rgb}{0.000000,0.000000,0.000000}%
\pgfsetstrokecolor{currentstroke}%
\pgfsetdash{}{0pt}%
\pgfpathmoveto{\pgfqpoint{0.467222in}{0.387222in}}%
\pgfpathlineto{\pgfqpoint{3.890000in}{0.387222in}}%
\pgfusepath{stroke}%
\end{pgfscope}%
\begin{pgfscope}%
\pgfsetrectcap%
\pgfsetmiterjoin%
\pgfsetlinewidth{0.803000pt}%
\definecolor{currentstroke}{rgb}{0.000000,0.000000,0.000000}%
\pgfsetstrokecolor{currentstroke}%
\pgfsetdash{}{0pt}%
\pgfpathmoveto{\pgfqpoint{0.467222in}{2.711667in}}%
\pgfpathlineto{\pgfqpoint{3.890000in}{2.711667in}}%
\pgfusepath{stroke}%
\end{pgfscope}%
\begin{pgfscope}%
\definecolor{textcolor}{rgb}{0.000000,0.000000,0.000000}%
\pgfsetstrokecolor{textcolor}%
\pgfsetfillcolor{textcolor}%
\pgftext[x=2.178611in,y=2.795000in,,base]{\color{textcolor}\sffamily\fontsize{12.000000}{14.400000}\selectfont \(\displaystyle  N = 16 \)}%
\end{pgfscope}%
\begin{pgfscope}%
\pgfsetbuttcap%
\pgfsetmiterjoin%
\definecolor{currentfill}{rgb}{1.000000,1.000000,1.000000}%
\pgfsetfillcolor{currentfill}%
\pgfsetlinewidth{0.000000pt}%
\definecolor{currentstroke}{rgb}{0.000000,0.000000,0.000000}%
\pgfsetstrokecolor{currentstroke}%
\pgfsetstrokeopacity{0.000000}%
\pgfsetdash{}{0pt}%
\pgfpathmoveto{\pgfqpoint{4.392222in}{0.387222in}}%
\pgfpathlineto{\pgfqpoint{7.815000in}{0.387222in}}%
\pgfpathlineto{\pgfqpoint{7.815000in}{2.711667in}}%
\pgfpathlineto{\pgfqpoint{4.392222in}{2.711667in}}%
\pgfpathclose%
\pgfusepath{fill}%
\end{pgfscope}%
\begin{pgfscope}%
\pgfpathrectangle{\pgfqpoint{4.392222in}{0.387222in}}{\pgfqpoint{3.422778in}{2.324444in}}%
\pgfusepath{clip}%
\pgfsetbuttcap%
\pgfsetroundjoin%
\definecolor{currentfill}{rgb}{0.121569,0.466667,0.705882}%
\pgfsetfillcolor{currentfill}%
\pgfsetfillopacity{0.300000}%
\pgfsetlinewidth{1.003750pt}%
\definecolor{currentstroke}{rgb}{0.121569,0.466667,0.705882}%
\pgfsetstrokecolor{currentstroke}%
\pgfsetstrokeopacity{0.300000}%
\pgfsetdash{}{0pt}%
\pgfpathmoveto{\pgfqpoint{4.556344in}{2.598261in}}%
\pgfpathlineto{\pgfqpoint{4.556344in}{2.598218in}}%
\pgfpathlineto{\pgfqpoint{4.633707in}{2.587539in}}%
\pgfpathlineto{\pgfqpoint{4.711071in}{2.575078in}}%
\pgfpathlineto{\pgfqpoint{4.788434in}{2.560712in}}%
\pgfpathlineto{\pgfqpoint{4.865797in}{2.544166in}}%
\pgfpathlineto{\pgfqpoint{4.943161in}{2.525281in}}%
\pgfpathlineto{\pgfqpoint{5.020524in}{2.503642in}}%
\pgfpathlineto{\pgfqpoint{5.097887in}{2.479127in}}%
\pgfpathlineto{\pgfqpoint{5.175251in}{2.451198in}}%
\pgfpathlineto{\pgfqpoint{5.252614in}{2.419470in}}%
\pgfpathlineto{\pgfqpoint{5.329978in}{2.383485in}}%
\pgfpathlineto{\pgfqpoint{5.407341in}{2.342036in}}%
\pgfpathlineto{\pgfqpoint{5.484704in}{2.294596in}}%
\pgfpathlineto{\pgfqpoint{5.562068in}{2.240295in}}%
\pgfpathlineto{\pgfqpoint{5.639431in}{2.176388in}}%
\pgfpathlineto{\pgfqpoint{5.716794in}{2.100927in}}%
\pgfpathlineto{\pgfqpoint{5.794158in}{2.010147in}}%
\pgfpathlineto{\pgfqpoint{5.871521in}{1.896892in}}%
\pgfpathlineto{\pgfqpoint{5.948884in}{1.746729in}}%
\pgfpathlineto{\pgfqpoint{6.026248in}{1.518483in}}%
\pgfpathlineto{\pgfqpoint{6.103611in}{1.010030in}}%
\pgfpathlineto{\pgfqpoint{6.180974in}{0.628538in}}%
\pgfpathlineto{\pgfqpoint{6.258338in}{0.578312in}}%
\pgfpathlineto{\pgfqpoint{6.335701in}{0.553005in}}%
\pgfpathlineto{\pgfqpoint{6.413065in}{0.538932in}}%
\pgfpathlineto{\pgfqpoint{6.490428in}{0.530846in}}%
\pgfpathlineto{\pgfqpoint{6.567791in}{0.525630in}}%
\pgfpathlineto{\pgfqpoint{6.645155in}{0.520903in}}%
\pgfpathlineto{\pgfqpoint{6.722518in}{0.517154in}}%
\pgfpathlineto{\pgfqpoint{6.799881in}{0.514123in}}%
\pgfpathlineto{\pgfqpoint{6.877245in}{0.512461in}}%
\pgfpathlineto{\pgfqpoint{6.954608in}{0.510454in}}%
\pgfpathlineto{\pgfqpoint{7.031971in}{0.508771in}}%
\pgfpathlineto{\pgfqpoint{7.109335in}{0.507074in}}%
\pgfpathlineto{\pgfqpoint{7.186698in}{0.505271in}}%
\pgfpathlineto{\pgfqpoint{7.264061in}{0.504981in}}%
\pgfpathlineto{\pgfqpoint{7.341425in}{0.504049in}}%
\pgfpathlineto{\pgfqpoint{7.418788in}{0.502789in}}%
\pgfpathlineto{\pgfqpoint{7.496152in}{0.501752in}}%
\pgfpathlineto{\pgfqpoint{7.573515in}{0.501222in}}%
\pgfpathlineto{\pgfqpoint{7.650878in}{0.500415in}}%
\pgfpathlineto{\pgfqpoint{7.650878in}{0.500883in}}%
\pgfpathlineto{\pgfqpoint{7.650878in}{0.500883in}}%
\pgfpathlineto{\pgfqpoint{7.573515in}{0.501738in}}%
\pgfpathlineto{\pgfqpoint{7.496152in}{0.502676in}}%
\pgfpathlineto{\pgfqpoint{7.418788in}{0.503978in}}%
\pgfpathlineto{\pgfqpoint{7.341425in}{0.504437in}}%
\pgfpathlineto{\pgfqpoint{7.264061in}{0.505415in}}%
\pgfpathlineto{\pgfqpoint{7.186698in}{0.506249in}}%
\pgfpathlineto{\pgfqpoint{7.109335in}{0.508671in}}%
\pgfpathlineto{\pgfqpoint{7.031971in}{0.510000in}}%
\pgfpathlineto{\pgfqpoint{6.954608in}{0.512066in}}%
\pgfpathlineto{\pgfqpoint{6.877245in}{0.514439in}}%
\pgfpathlineto{\pgfqpoint{6.799881in}{0.516719in}}%
\pgfpathlineto{\pgfqpoint{6.722518in}{0.519676in}}%
\pgfpathlineto{\pgfqpoint{6.645155in}{0.523835in}}%
\pgfpathlineto{\pgfqpoint{6.567791in}{0.527809in}}%
\pgfpathlineto{\pgfqpoint{6.490428in}{0.535267in}}%
\pgfpathlineto{\pgfqpoint{6.413065in}{0.545553in}}%
\pgfpathlineto{\pgfqpoint{6.335701in}{0.560441in}}%
\pgfpathlineto{\pgfqpoint{6.258338in}{0.588109in}}%
\pgfpathlineto{\pgfqpoint{6.180974in}{0.665025in}}%
\pgfpathlineto{\pgfqpoint{6.103611in}{1.034771in}}%
\pgfpathlineto{\pgfqpoint{6.026248in}{1.528955in}}%
\pgfpathlineto{\pgfqpoint{5.948884in}{1.750162in}}%
\pgfpathlineto{\pgfqpoint{5.871521in}{1.898769in}}%
\pgfpathlineto{\pgfqpoint{5.794158in}{2.011288in}}%
\pgfpathlineto{\pgfqpoint{5.716794in}{2.101391in}}%
\pgfpathlineto{\pgfqpoint{5.639431in}{2.176673in}}%
\pgfpathlineto{\pgfqpoint{5.562068in}{2.240717in}}%
\pgfpathlineto{\pgfqpoint{5.484704in}{2.294953in}}%
\pgfpathlineto{\pgfqpoint{5.407341in}{2.342242in}}%
\pgfpathlineto{\pgfqpoint{5.329978in}{2.383624in}}%
\pgfpathlineto{\pgfqpoint{5.252614in}{2.419740in}}%
\pgfpathlineto{\pgfqpoint{5.175251in}{2.451382in}}%
\pgfpathlineto{\pgfqpoint{5.097887in}{2.479294in}}%
\pgfpathlineto{\pgfqpoint{5.020524in}{2.503800in}}%
\pgfpathlineto{\pgfqpoint{4.943161in}{2.525328in}}%
\pgfpathlineto{\pgfqpoint{4.865797in}{2.544267in}}%
\pgfpathlineto{\pgfqpoint{4.788434in}{2.560729in}}%
\pgfpathlineto{\pgfqpoint{4.711071in}{2.575147in}}%
\pgfpathlineto{\pgfqpoint{4.633707in}{2.587577in}}%
\pgfpathlineto{\pgfqpoint{4.556344in}{2.598261in}}%
\pgfpathclose%
\pgfusepath{stroke,fill}%
\end{pgfscope}%
\begin{pgfscope}%
\pgfpathrectangle{\pgfqpoint{4.392222in}{0.387222in}}{\pgfqpoint{3.422778in}{2.324444in}}%
\pgfusepath{clip}%
\pgfsetbuttcap%
\pgfsetroundjoin%
\definecolor{currentfill}{rgb}{0.121569,0.466667,0.705882}%
\pgfsetfillcolor{currentfill}%
\pgfsetlinewidth{1.003750pt}%
\definecolor{currentstroke}{rgb}{0.121569,0.466667,0.705882}%
\pgfsetstrokecolor{currentstroke}%
\pgfsetdash{}{0pt}%
\pgfsys@defobject{currentmarker}{\pgfqpoint{-0.009821in}{-0.009821in}}{\pgfqpoint{0.009821in}{0.009821in}}{%
\pgfpathmoveto{\pgfqpoint{0.000000in}{-0.009821in}}%
\pgfpathcurveto{\pgfqpoint{0.002605in}{-0.009821in}}{\pgfqpoint{0.005103in}{-0.008786in}}{\pgfqpoint{0.006944in}{-0.006944in}}%
\pgfpathcurveto{\pgfqpoint{0.008786in}{-0.005103in}}{\pgfqpoint{0.009821in}{-0.002605in}}{\pgfqpoint{0.009821in}{0.000000in}}%
\pgfpathcurveto{\pgfqpoint{0.009821in}{0.002605in}}{\pgfqpoint{0.008786in}{0.005103in}}{\pgfqpoint{0.006944in}{0.006944in}}%
\pgfpathcurveto{\pgfqpoint{0.005103in}{0.008786in}}{\pgfqpoint{0.002605in}{0.009821in}}{\pgfqpoint{0.000000in}{0.009821in}}%
\pgfpathcurveto{\pgfqpoint{-0.002605in}{0.009821in}}{\pgfqpoint{-0.005103in}{0.008786in}}{\pgfqpoint{-0.006944in}{0.006944in}}%
\pgfpathcurveto{\pgfqpoint{-0.008786in}{0.005103in}}{\pgfqpoint{-0.009821in}{0.002605in}}{\pgfqpoint{-0.009821in}{0.000000in}}%
\pgfpathcurveto{\pgfqpoint{-0.009821in}{-0.002605in}}{\pgfqpoint{-0.008786in}{-0.005103in}}{\pgfqpoint{-0.006944in}{-0.006944in}}%
\pgfpathcurveto{\pgfqpoint{-0.005103in}{-0.008786in}}{\pgfqpoint{-0.002605in}{-0.009821in}}{\pgfqpoint{0.000000in}{-0.009821in}}%
\pgfpathclose%
\pgfusepath{stroke,fill}%
}%
\begin{pgfscope}%
\pgfsys@transformshift{4.556344in}{2.598240in}%
\pgfsys@useobject{currentmarker}{}%
\end{pgfscope}%
\begin{pgfscope}%
\pgfsys@transformshift{4.633707in}{2.587558in}%
\pgfsys@useobject{currentmarker}{}%
\end{pgfscope}%
\begin{pgfscope}%
\pgfsys@transformshift{4.711071in}{2.575112in}%
\pgfsys@useobject{currentmarker}{}%
\end{pgfscope}%
\begin{pgfscope}%
\pgfsys@transformshift{4.788434in}{2.560720in}%
\pgfsys@useobject{currentmarker}{}%
\end{pgfscope}%
\begin{pgfscope}%
\pgfsys@transformshift{4.865797in}{2.544217in}%
\pgfsys@useobject{currentmarker}{}%
\end{pgfscope}%
\begin{pgfscope}%
\pgfsys@transformshift{4.943161in}{2.525304in}%
\pgfsys@useobject{currentmarker}{}%
\end{pgfscope}%
\begin{pgfscope}%
\pgfsys@transformshift{5.020524in}{2.503721in}%
\pgfsys@useobject{currentmarker}{}%
\end{pgfscope}%
\begin{pgfscope}%
\pgfsys@transformshift{5.097887in}{2.479211in}%
\pgfsys@useobject{currentmarker}{}%
\end{pgfscope}%
\begin{pgfscope}%
\pgfsys@transformshift{5.175251in}{2.451290in}%
\pgfsys@useobject{currentmarker}{}%
\end{pgfscope}%
\begin{pgfscope}%
\pgfsys@transformshift{5.252614in}{2.419605in}%
\pgfsys@useobject{currentmarker}{}%
\end{pgfscope}%
\begin{pgfscope}%
\pgfsys@transformshift{5.329978in}{2.383554in}%
\pgfsys@useobject{currentmarker}{}%
\end{pgfscope}%
\begin{pgfscope}%
\pgfsys@transformshift{5.407341in}{2.342139in}%
\pgfsys@useobject{currentmarker}{}%
\end{pgfscope}%
\begin{pgfscope}%
\pgfsys@transformshift{5.484704in}{2.294775in}%
\pgfsys@useobject{currentmarker}{}%
\end{pgfscope}%
\begin{pgfscope}%
\pgfsys@transformshift{5.562068in}{2.240506in}%
\pgfsys@useobject{currentmarker}{}%
\end{pgfscope}%
\begin{pgfscope}%
\pgfsys@transformshift{5.639431in}{2.176531in}%
\pgfsys@useobject{currentmarker}{}%
\end{pgfscope}%
\begin{pgfscope}%
\pgfsys@transformshift{5.716794in}{2.101159in}%
\pgfsys@useobject{currentmarker}{}%
\end{pgfscope}%
\begin{pgfscope}%
\pgfsys@transformshift{5.794158in}{2.010718in}%
\pgfsys@useobject{currentmarker}{}%
\end{pgfscope}%
\begin{pgfscope}%
\pgfsys@transformshift{5.871521in}{1.897830in}%
\pgfsys@useobject{currentmarker}{}%
\end{pgfscope}%
\begin{pgfscope}%
\pgfsys@transformshift{5.948884in}{1.748445in}%
\pgfsys@useobject{currentmarker}{}%
\end{pgfscope}%
\begin{pgfscope}%
\pgfsys@transformshift{6.026248in}{1.523719in}%
\pgfsys@useobject{currentmarker}{}%
\end{pgfscope}%
\begin{pgfscope}%
\pgfsys@transformshift{6.103611in}{1.022401in}%
\pgfsys@useobject{currentmarker}{}%
\end{pgfscope}%
\begin{pgfscope}%
\pgfsys@transformshift{6.180974in}{0.646782in}%
\pgfsys@useobject{currentmarker}{}%
\end{pgfscope}%
\begin{pgfscope}%
\pgfsys@transformshift{6.258338in}{0.583210in}%
\pgfsys@useobject{currentmarker}{}%
\end{pgfscope}%
\begin{pgfscope}%
\pgfsys@transformshift{6.335701in}{0.556723in}%
\pgfsys@useobject{currentmarker}{}%
\end{pgfscope}%
\begin{pgfscope}%
\pgfsys@transformshift{6.413065in}{0.542242in}%
\pgfsys@useobject{currentmarker}{}%
\end{pgfscope}%
\begin{pgfscope}%
\pgfsys@transformshift{6.490428in}{0.533056in}%
\pgfsys@useobject{currentmarker}{}%
\end{pgfscope}%
\begin{pgfscope}%
\pgfsys@transformshift{6.567791in}{0.526720in}%
\pgfsys@useobject{currentmarker}{}%
\end{pgfscope}%
\begin{pgfscope}%
\pgfsys@transformshift{6.645155in}{0.522369in}%
\pgfsys@useobject{currentmarker}{}%
\end{pgfscope}%
\begin{pgfscope}%
\pgfsys@transformshift{6.722518in}{0.518415in}%
\pgfsys@useobject{currentmarker}{}%
\end{pgfscope}%
\begin{pgfscope}%
\pgfsys@transformshift{6.799881in}{0.515421in}%
\pgfsys@useobject{currentmarker}{}%
\end{pgfscope}%
\begin{pgfscope}%
\pgfsys@transformshift{6.877245in}{0.513450in}%
\pgfsys@useobject{currentmarker}{}%
\end{pgfscope}%
\begin{pgfscope}%
\pgfsys@transformshift{6.954608in}{0.511260in}%
\pgfsys@useobject{currentmarker}{}%
\end{pgfscope}%
\begin{pgfscope}%
\pgfsys@transformshift{7.031971in}{0.509386in}%
\pgfsys@useobject{currentmarker}{}%
\end{pgfscope}%
\begin{pgfscope}%
\pgfsys@transformshift{7.109335in}{0.507873in}%
\pgfsys@useobject{currentmarker}{}%
\end{pgfscope}%
\begin{pgfscope}%
\pgfsys@transformshift{7.186698in}{0.505760in}%
\pgfsys@useobject{currentmarker}{}%
\end{pgfscope}%
\begin{pgfscope}%
\pgfsys@transformshift{7.264061in}{0.505198in}%
\pgfsys@useobject{currentmarker}{}%
\end{pgfscope}%
\begin{pgfscope}%
\pgfsys@transformshift{7.341425in}{0.504243in}%
\pgfsys@useobject{currentmarker}{}%
\end{pgfscope}%
\begin{pgfscope}%
\pgfsys@transformshift{7.418788in}{0.503384in}%
\pgfsys@useobject{currentmarker}{}%
\end{pgfscope}%
\begin{pgfscope}%
\pgfsys@transformshift{7.496152in}{0.502214in}%
\pgfsys@useobject{currentmarker}{}%
\end{pgfscope}%
\begin{pgfscope}%
\pgfsys@transformshift{7.573515in}{0.501480in}%
\pgfsys@useobject{currentmarker}{}%
\end{pgfscope}%
\begin{pgfscope}%
\pgfsys@transformshift{7.650878in}{0.500649in}%
\pgfsys@useobject{currentmarker}{}%
\end{pgfscope}%
\end{pgfscope}%
\begin{pgfscope}%
\pgfsetbuttcap%
\pgfsetroundjoin%
\definecolor{currentfill}{rgb}{0.000000,0.000000,0.000000}%
\pgfsetfillcolor{currentfill}%
\pgfsetlinewidth{0.803000pt}%
\definecolor{currentstroke}{rgb}{0.000000,0.000000,0.000000}%
\pgfsetstrokecolor{currentstroke}%
\pgfsetdash{}{0pt}%
\pgfsys@defobject{currentmarker}{\pgfqpoint{0.000000in}{-0.048611in}}{\pgfqpoint{0.000000in}{0.000000in}}{%
\pgfpathmoveto{\pgfqpoint{0.000000in}{0.000000in}}%
\pgfpathlineto{\pgfqpoint{0.000000in}{-0.048611in}}%
\pgfusepath{stroke,fill}%
}%
\begin{pgfscope}%
\pgfsys@transformshift{4.943161in}{0.387222in}%
\pgfsys@useobject{currentmarker}{}%
\end{pgfscope}%
\end{pgfscope}%
\begin{pgfscope}%
\definecolor{textcolor}{rgb}{0.000000,0.000000,0.000000}%
\pgfsetstrokecolor{textcolor}%
\pgfsetfillcolor{textcolor}%
\pgftext[x=4.943161in,y=0.290000in,,top]{\color{textcolor}\sffamily\fontsize{10.000000}{12.000000}\selectfont 3}%
\end{pgfscope}%
\begin{pgfscope}%
\pgfsetbuttcap%
\pgfsetroundjoin%
\definecolor{currentfill}{rgb}{0.000000,0.000000,0.000000}%
\pgfsetfillcolor{currentfill}%
\pgfsetlinewidth{0.803000pt}%
\definecolor{currentstroke}{rgb}{0.000000,0.000000,0.000000}%
\pgfsetstrokecolor{currentstroke}%
\pgfsetdash{}{0pt}%
\pgfsys@defobject{currentmarker}{\pgfqpoint{0.000000in}{-0.048611in}}{\pgfqpoint{0.000000in}{0.000000in}}{%
\pgfpathmoveto{\pgfqpoint{0.000000in}{0.000000in}}%
\pgfpathlineto{\pgfqpoint{0.000000in}{-0.048611in}}%
\pgfusepath{stroke,fill}%
}%
\begin{pgfscope}%
\pgfsys@transformshift{5.716794in}{0.387222in}%
\pgfsys@useobject{currentmarker}{}%
\end{pgfscope}%
\end{pgfscope}%
\begin{pgfscope}%
\definecolor{textcolor}{rgb}{0.000000,0.000000,0.000000}%
\pgfsetstrokecolor{textcolor}%
\pgfsetfillcolor{textcolor}%
\pgftext[x=5.716794in,y=0.290000in,,top]{\color{textcolor}\sffamily\fontsize{10.000000}{12.000000}\selectfont 4}%
\end{pgfscope}%
\begin{pgfscope}%
\pgfsetbuttcap%
\pgfsetroundjoin%
\definecolor{currentfill}{rgb}{0.000000,0.000000,0.000000}%
\pgfsetfillcolor{currentfill}%
\pgfsetlinewidth{0.803000pt}%
\definecolor{currentstroke}{rgb}{0.000000,0.000000,0.000000}%
\pgfsetstrokecolor{currentstroke}%
\pgfsetdash{}{0pt}%
\pgfsys@defobject{currentmarker}{\pgfqpoint{0.000000in}{-0.048611in}}{\pgfqpoint{0.000000in}{0.000000in}}{%
\pgfpathmoveto{\pgfqpoint{0.000000in}{0.000000in}}%
\pgfpathlineto{\pgfqpoint{0.000000in}{-0.048611in}}%
\pgfusepath{stroke,fill}%
}%
\begin{pgfscope}%
\pgfsys@transformshift{6.490428in}{0.387222in}%
\pgfsys@useobject{currentmarker}{}%
\end{pgfscope}%
\end{pgfscope}%
\begin{pgfscope}%
\definecolor{textcolor}{rgb}{0.000000,0.000000,0.000000}%
\pgfsetstrokecolor{textcolor}%
\pgfsetfillcolor{textcolor}%
\pgftext[x=6.490428in,y=0.290000in,,top]{\color{textcolor}\sffamily\fontsize{10.000000}{12.000000}\selectfont 5}%
\end{pgfscope}%
\begin{pgfscope}%
\pgfsetbuttcap%
\pgfsetroundjoin%
\definecolor{currentfill}{rgb}{0.000000,0.000000,0.000000}%
\pgfsetfillcolor{currentfill}%
\pgfsetlinewidth{0.803000pt}%
\definecolor{currentstroke}{rgb}{0.000000,0.000000,0.000000}%
\pgfsetstrokecolor{currentstroke}%
\pgfsetdash{}{0pt}%
\pgfsys@defobject{currentmarker}{\pgfqpoint{0.000000in}{-0.048611in}}{\pgfqpoint{0.000000in}{0.000000in}}{%
\pgfpathmoveto{\pgfqpoint{0.000000in}{0.000000in}}%
\pgfpathlineto{\pgfqpoint{0.000000in}{-0.048611in}}%
\pgfusepath{stroke,fill}%
}%
\begin{pgfscope}%
\pgfsys@transformshift{7.264061in}{0.387222in}%
\pgfsys@useobject{currentmarker}{}%
\end{pgfscope}%
\end{pgfscope}%
\begin{pgfscope}%
\definecolor{textcolor}{rgb}{0.000000,0.000000,0.000000}%
\pgfsetstrokecolor{textcolor}%
\pgfsetfillcolor{textcolor}%
\pgftext[x=7.264061in,y=0.290000in,,top]{\color{textcolor}\sffamily\fontsize{10.000000}{12.000000}\selectfont 6}%
\end{pgfscope}%
\begin{pgfscope}%
\pgfsetbuttcap%
\pgfsetroundjoin%
\definecolor{currentfill}{rgb}{0.000000,0.000000,0.000000}%
\pgfsetfillcolor{currentfill}%
\pgfsetlinewidth{0.803000pt}%
\definecolor{currentstroke}{rgb}{0.000000,0.000000,0.000000}%
\pgfsetstrokecolor{currentstroke}%
\pgfsetdash{}{0pt}%
\pgfsys@defobject{currentmarker}{\pgfqpoint{-0.048611in}{0.000000in}}{\pgfqpoint{0.000000in}{0.000000in}}{%
\pgfpathmoveto{\pgfqpoint{0.000000in}{0.000000in}}%
\pgfpathlineto{\pgfqpoint{-0.048611in}{0.000000in}}%
\pgfusepath{stroke,fill}%
}%
\begin{pgfscope}%
\pgfsys@transformshift{4.392222in}{0.469012in}%
\pgfsys@useobject{currentmarker}{}%
\end{pgfscope}%
\end{pgfscope}%
\begin{pgfscope}%
\definecolor{textcolor}{rgb}{0.000000,0.000000,0.000000}%
\pgfsetstrokecolor{textcolor}%
\pgfsetfillcolor{textcolor}%
\pgftext[x=4.074121in,y=0.416251in,left,base]{\color{textcolor}\sffamily\fontsize{10.000000}{12.000000}\selectfont 0.0}%
\end{pgfscope}%
\begin{pgfscope}%
\pgfsetbuttcap%
\pgfsetroundjoin%
\definecolor{currentfill}{rgb}{0.000000,0.000000,0.000000}%
\pgfsetfillcolor{currentfill}%
\pgfsetlinewidth{0.803000pt}%
\definecolor{currentstroke}{rgb}{0.000000,0.000000,0.000000}%
\pgfsetstrokecolor{currentstroke}%
\pgfsetdash{}{0pt}%
\pgfsys@defobject{currentmarker}{\pgfqpoint{-0.048611in}{0.000000in}}{\pgfqpoint{0.000000in}{0.000000in}}{%
\pgfpathmoveto{\pgfqpoint{0.000000in}{0.000000in}}%
\pgfpathlineto{\pgfqpoint{-0.048611in}{0.000000in}}%
\pgfusepath{stroke,fill}%
}%
\begin{pgfscope}%
\pgfsys@transformshift{4.392222in}{0.903793in}%
\pgfsys@useobject{currentmarker}{}%
\end{pgfscope}%
\end{pgfscope}%
\begin{pgfscope}%
\definecolor{textcolor}{rgb}{0.000000,0.000000,0.000000}%
\pgfsetstrokecolor{textcolor}%
\pgfsetfillcolor{textcolor}%
\pgftext[x=4.074121in,y=0.851032in,left,base]{\color{textcolor}\sffamily\fontsize{10.000000}{12.000000}\selectfont 0.2}%
\end{pgfscope}%
\begin{pgfscope}%
\pgfsetbuttcap%
\pgfsetroundjoin%
\definecolor{currentfill}{rgb}{0.000000,0.000000,0.000000}%
\pgfsetfillcolor{currentfill}%
\pgfsetlinewidth{0.803000pt}%
\definecolor{currentstroke}{rgb}{0.000000,0.000000,0.000000}%
\pgfsetstrokecolor{currentstroke}%
\pgfsetdash{}{0pt}%
\pgfsys@defobject{currentmarker}{\pgfqpoint{-0.048611in}{0.000000in}}{\pgfqpoint{0.000000in}{0.000000in}}{%
\pgfpathmoveto{\pgfqpoint{0.000000in}{0.000000in}}%
\pgfpathlineto{\pgfqpoint{-0.048611in}{0.000000in}}%
\pgfusepath{stroke,fill}%
}%
\begin{pgfscope}%
\pgfsys@transformshift{4.392222in}{1.338574in}%
\pgfsys@useobject{currentmarker}{}%
\end{pgfscope}%
\end{pgfscope}%
\begin{pgfscope}%
\definecolor{textcolor}{rgb}{0.000000,0.000000,0.000000}%
\pgfsetstrokecolor{textcolor}%
\pgfsetfillcolor{textcolor}%
\pgftext[x=4.074121in,y=1.285812in,left,base]{\color{textcolor}\sffamily\fontsize{10.000000}{12.000000}\selectfont 0.4}%
\end{pgfscope}%
\begin{pgfscope}%
\pgfsetbuttcap%
\pgfsetroundjoin%
\definecolor{currentfill}{rgb}{0.000000,0.000000,0.000000}%
\pgfsetfillcolor{currentfill}%
\pgfsetlinewidth{0.803000pt}%
\definecolor{currentstroke}{rgb}{0.000000,0.000000,0.000000}%
\pgfsetstrokecolor{currentstroke}%
\pgfsetdash{}{0pt}%
\pgfsys@defobject{currentmarker}{\pgfqpoint{-0.048611in}{0.000000in}}{\pgfqpoint{0.000000in}{0.000000in}}{%
\pgfpathmoveto{\pgfqpoint{0.000000in}{0.000000in}}%
\pgfpathlineto{\pgfqpoint{-0.048611in}{0.000000in}}%
\pgfusepath{stroke,fill}%
}%
\begin{pgfscope}%
\pgfsys@transformshift{4.392222in}{1.773355in}%
\pgfsys@useobject{currentmarker}{}%
\end{pgfscope}%
\end{pgfscope}%
\begin{pgfscope}%
\definecolor{textcolor}{rgb}{0.000000,0.000000,0.000000}%
\pgfsetstrokecolor{textcolor}%
\pgfsetfillcolor{textcolor}%
\pgftext[x=4.074121in,y=1.720593in,left,base]{\color{textcolor}\sffamily\fontsize{10.000000}{12.000000}\selectfont 0.6}%
\end{pgfscope}%
\begin{pgfscope}%
\pgfsetbuttcap%
\pgfsetroundjoin%
\definecolor{currentfill}{rgb}{0.000000,0.000000,0.000000}%
\pgfsetfillcolor{currentfill}%
\pgfsetlinewidth{0.803000pt}%
\definecolor{currentstroke}{rgb}{0.000000,0.000000,0.000000}%
\pgfsetstrokecolor{currentstroke}%
\pgfsetdash{}{0pt}%
\pgfsys@defobject{currentmarker}{\pgfqpoint{-0.048611in}{0.000000in}}{\pgfqpoint{0.000000in}{0.000000in}}{%
\pgfpathmoveto{\pgfqpoint{0.000000in}{0.000000in}}%
\pgfpathlineto{\pgfqpoint{-0.048611in}{0.000000in}}%
\pgfusepath{stroke,fill}%
}%
\begin{pgfscope}%
\pgfsys@transformshift{4.392222in}{2.208136in}%
\pgfsys@useobject{currentmarker}{}%
\end{pgfscope}%
\end{pgfscope}%
\begin{pgfscope}%
\definecolor{textcolor}{rgb}{0.000000,0.000000,0.000000}%
\pgfsetstrokecolor{textcolor}%
\pgfsetfillcolor{textcolor}%
\pgftext[x=4.074121in,y=2.155374in,left,base]{\color{textcolor}\sffamily\fontsize{10.000000}{12.000000}\selectfont 0.8}%
\end{pgfscope}%
\begin{pgfscope}%
\pgfsetbuttcap%
\pgfsetroundjoin%
\definecolor{currentfill}{rgb}{0.000000,0.000000,0.000000}%
\pgfsetfillcolor{currentfill}%
\pgfsetlinewidth{0.803000pt}%
\definecolor{currentstroke}{rgb}{0.000000,0.000000,0.000000}%
\pgfsetstrokecolor{currentstroke}%
\pgfsetdash{}{0pt}%
\pgfsys@defobject{currentmarker}{\pgfqpoint{-0.048611in}{0.000000in}}{\pgfqpoint{0.000000in}{0.000000in}}{%
\pgfpathmoveto{\pgfqpoint{0.000000in}{0.000000in}}%
\pgfpathlineto{\pgfqpoint{-0.048611in}{0.000000in}}%
\pgfusepath{stroke,fill}%
}%
\begin{pgfscope}%
\pgfsys@transformshift{4.392222in}{2.642916in}%
\pgfsys@useobject{currentmarker}{}%
\end{pgfscope}%
\end{pgfscope}%
\begin{pgfscope}%
\definecolor{textcolor}{rgb}{0.000000,0.000000,0.000000}%
\pgfsetstrokecolor{textcolor}%
\pgfsetfillcolor{textcolor}%
\pgftext[x=4.074121in,y=2.590155in,left,base]{\color{textcolor}\sffamily\fontsize{10.000000}{12.000000}\selectfont 1.0}%
\end{pgfscope}%
\begin{pgfscope}%
\pgfpathrectangle{\pgfqpoint{4.392222in}{0.387222in}}{\pgfqpoint{3.422778in}{2.324444in}}%
\pgfusepath{clip}%
\pgfsetrectcap%
\pgfsetroundjoin%
\pgfsetlinewidth{1.505625pt}%
\definecolor{currentstroke}{rgb}{0.121569,0.466667,0.705882}%
\pgfsetstrokecolor{currentstroke}%
\pgfsetdash{}{0pt}%
\pgfpathmoveto{\pgfqpoint{4.556344in}{2.598240in}}%
\pgfpathlineto{\pgfqpoint{4.633707in}{2.587558in}}%
\pgfpathlineto{\pgfqpoint{4.711071in}{2.575112in}}%
\pgfpathlineto{\pgfqpoint{4.788434in}{2.560720in}}%
\pgfpathlineto{\pgfqpoint{4.865797in}{2.544217in}}%
\pgfpathlineto{\pgfqpoint{4.943161in}{2.525304in}}%
\pgfpathlineto{\pgfqpoint{5.020524in}{2.503721in}}%
\pgfpathlineto{\pgfqpoint{5.097887in}{2.479211in}}%
\pgfpathlineto{\pgfqpoint{5.175251in}{2.451290in}}%
\pgfpathlineto{\pgfqpoint{5.252614in}{2.419605in}}%
\pgfpathlineto{\pgfqpoint{5.329978in}{2.383554in}}%
\pgfpathlineto{\pgfqpoint{5.407341in}{2.342139in}}%
\pgfpathlineto{\pgfqpoint{5.484704in}{2.294775in}}%
\pgfpathlineto{\pgfqpoint{5.562068in}{2.240506in}}%
\pgfpathlineto{\pgfqpoint{5.639431in}{2.176531in}}%
\pgfpathlineto{\pgfqpoint{5.716794in}{2.101159in}}%
\pgfpathlineto{\pgfqpoint{5.794158in}{2.010718in}}%
\pgfpathlineto{\pgfqpoint{5.871521in}{1.897830in}}%
\pgfpathlineto{\pgfqpoint{5.948884in}{1.748445in}}%
\pgfpathlineto{\pgfqpoint{6.026248in}{1.523719in}}%
\pgfpathlineto{\pgfqpoint{6.103611in}{1.022401in}}%
\pgfpathlineto{\pgfqpoint{6.180974in}{0.646782in}}%
\pgfpathlineto{\pgfqpoint{6.258338in}{0.583210in}}%
\pgfpathlineto{\pgfqpoint{6.335701in}{0.556723in}}%
\pgfpathlineto{\pgfqpoint{6.413065in}{0.542242in}}%
\pgfpathlineto{\pgfqpoint{6.490428in}{0.533056in}}%
\pgfpathlineto{\pgfqpoint{6.567791in}{0.526720in}}%
\pgfpathlineto{\pgfqpoint{6.645155in}{0.522369in}}%
\pgfpathlineto{\pgfqpoint{6.722518in}{0.518415in}}%
\pgfpathlineto{\pgfqpoint{6.799881in}{0.515421in}}%
\pgfpathlineto{\pgfqpoint{6.877245in}{0.513450in}}%
\pgfpathlineto{\pgfqpoint{6.954608in}{0.511260in}}%
\pgfpathlineto{\pgfqpoint{7.031971in}{0.509386in}}%
\pgfpathlineto{\pgfqpoint{7.109335in}{0.507873in}}%
\pgfpathlineto{\pgfqpoint{7.186698in}{0.505760in}}%
\pgfpathlineto{\pgfqpoint{7.264061in}{0.505198in}}%
\pgfpathlineto{\pgfqpoint{7.341425in}{0.504243in}}%
\pgfpathlineto{\pgfqpoint{7.418788in}{0.503384in}}%
\pgfpathlineto{\pgfqpoint{7.496152in}{0.502214in}}%
\pgfpathlineto{\pgfqpoint{7.573515in}{0.501480in}}%
\pgfpathlineto{\pgfqpoint{7.650878in}{0.500649in}}%
\pgfusepath{stroke}%
\end{pgfscope}%
\begin{pgfscope}%
\pgfsetrectcap%
\pgfsetmiterjoin%
\pgfsetlinewidth{0.803000pt}%
\definecolor{currentstroke}{rgb}{0.000000,0.000000,0.000000}%
\pgfsetstrokecolor{currentstroke}%
\pgfsetdash{}{0pt}%
\pgfpathmoveto{\pgfqpoint{4.392222in}{0.387222in}}%
\pgfpathlineto{\pgfqpoint{4.392222in}{2.711667in}}%
\pgfusepath{stroke}%
\end{pgfscope}%
\begin{pgfscope}%
\pgfsetrectcap%
\pgfsetmiterjoin%
\pgfsetlinewidth{0.803000pt}%
\definecolor{currentstroke}{rgb}{0.000000,0.000000,0.000000}%
\pgfsetstrokecolor{currentstroke}%
\pgfsetdash{}{0pt}%
\pgfpathmoveto{\pgfqpoint{7.815000in}{0.387222in}}%
\pgfpathlineto{\pgfqpoint{7.815000in}{2.711667in}}%
\pgfusepath{stroke}%
\end{pgfscope}%
\begin{pgfscope}%
\pgfsetrectcap%
\pgfsetmiterjoin%
\pgfsetlinewidth{0.803000pt}%
\definecolor{currentstroke}{rgb}{0.000000,0.000000,0.000000}%
\pgfsetstrokecolor{currentstroke}%
\pgfsetdash{}{0pt}%
\pgfpathmoveto{\pgfqpoint{4.392222in}{0.387222in}}%
\pgfpathlineto{\pgfqpoint{7.815000in}{0.387222in}}%
\pgfusepath{stroke}%
\end{pgfscope}%
\begin{pgfscope}%
\pgfsetrectcap%
\pgfsetmiterjoin%
\pgfsetlinewidth{0.803000pt}%
\definecolor{currentstroke}{rgb}{0.000000,0.000000,0.000000}%
\pgfsetstrokecolor{currentstroke}%
\pgfsetdash{}{0pt}%
\pgfpathmoveto{\pgfqpoint{4.392222in}{2.711667in}}%
\pgfpathlineto{\pgfqpoint{7.815000in}{2.711667in}}%
\pgfusepath{stroke}%
\end{pgfscope}%
\begin{pgfscope}%
\definecolor{textcolor}{rgb}{0.000000,0.000000,0.000000}%
\pgfsetstrokecolor{textcolor}%
\pgfsetfillcolor{textcolor}%
\pgftext[x=6.103611in,y=2.795000in,,base]{\color{textcolor}\sffamily\fontsize{12.000000}{14.400000}\selectfont \(\displaystyle  N = 24 \)}%
\end{pgfscope}%
\end{pgfpicture}%
\makeatother%
\endgroup%
}
\caption{Magnetization $m$ for different temperatures $T$ for the 3-D problem using kinetic Monte Carlo algorithm}
\label{Fig:MagSmall3D}
\end{figure}

\begin{figure}[htbp]
\centering
\scalebox{0.666}{%% Creator: Matplotlib, PGF backend
%%
%% To include the figure in your LaTeX document, write
%%   \input{<filename>.pgf}
%%
%% Make sure the required packages are loaded in your preamble
%%   \usepackage{pgf}
%%
%% Figures using additional raster images can only be included by \input if
%% they are in the same directory as the main LaTeX file. For loading figures
%% from other directories you can use the `import` package
%%   \usepackage{import}
%% and then include the figures with
%%   \import{<path to file>}{<filename>.pgf}
%%
%% Matplotlib used the following preamble
%%   \usepackage{fontspec}
%%   \setmainfont{DejaVuSerif.ttf}[Path=/home/lzh/anaconda3/envs/numana/lib/python3.7/site-packages/matplotlib/mpl-data/fonts/ttf/]
%%   \setsansfont{DejaVuSans.ttf}[Path=/home/lzh/anaconda3/envs/numana/lib/python3.7/site-packages/matplotlib/mpl-data/fonts/ttf/]
%%   \setmonofont{DejaVuSansMono.ttf}[Path=/home/lzh/anaconda3/envs/numana/lib/python3.7/site-packages/matplotlib/mpl-data/fonts/ttf/]
%%
\begingroup%
\makeatletter%
\begin{pgfpicture}%
\pgfpathrectangle{\pgfpointorigin}{\pgfqpoint{8.000000in}{6.000000in}}%
\pgfusepath{use as bounding box, clip}%
\begin{pgfscope}%
\pgfsetbuttcap%
\pgfsetmiterjoin%
\definecolor{currentfill}{rgb}{1.000000,1.000000,1.000000}%
\pgfsetfillcolor{currentfill}%
\pgfsetlinewidth{0.000000pt}%
\definecolor{currentstroke}{rgb}{1.000000,1.000000,1.000000}%
\pgfsetstrokecolor{currentstroke}%
\pgfsetdash{}{0pt}%
\pgfpathmoveto{\pgfqpoint{0.000000in}{0.000000in}}%
\pgfpathlineto{\pgfqpoint{8.000000in}{0.000000in}}%
\pgfpathlineto{\pgfqpoint{8.000000in}{6.000000in}}%
\pgfpathlineto{\pgfqpoint{0.000000in}{6.000000in}}%
\pgfpathclose%
\pgfusepath{fill}%
\end{pgfscope}%
\begin{pgfscope}%
\pgfsetbuttcap%
\pgfsetmiterjoin%
\definecolor{currentfill}{rgb}{1.000000,1.000000,1.000000}%
\pgfsetfillcolor{currentfill}%
\pgfsetlinewidth{0.000000pt}%
\definecolor{currentstroke}{rgb}{0.000000,0.000000,0.000000}%
\pgfsetstrokecolor{currentstroke}%
\pgfsetstrokeopacity{0.000000}%
\pgfsetdash{}{0pt}%
\pgfpathmoveto{\pgfqpoint{0.585764in}{3.311111in}}%
\pgfpathlineto{\pgfqpoint{3.828798in}{3.311111in}}%
\pgfpathlineto{\pgfqpoint{3.828798in}{5.627778in}}%
\pgfpathlineto{\pgfqpoint{0.585764in}{5.627778in}}%
\pgfpathclose%
\pgfusepath{fill}%
\end{pgfscope}%
\begin{pgfscope}%
\pgfpathrectangle{\pgfqpoint{0.585764in}{3.311111in}}{\pgfqpoint{3.243034in}{2.316667in}}%
\pgfusepath{clip}%
\pgfsetbuttcap%
\pgfsetroundjoin%
\definecolor{currentfill}{rgb}{0.121569,0.466667,0.705882}%
\pgfsetfillcolor{currentfill}%
\pgfsetfillopacity{0.300000}%
\pgfsetlinewidth{1.003750pt}%
\definecolor{currentstroke}{rgb}{0.121569,0.466667,0.705882}%
\pgfsetstrokecolor{currentstroke}%
\pgfsetstrokeopacity{0.300000}%
\pgfsetdash{}{0pt}%
\pgfpathmoveto{\pgfqpoint{0.744390in}{3.436431in}}%
\pgfpathlineto{\pgfqpoint{0.744390in}{3.418454in}}%
\pgfpathlineto{\pgfqpoint{0.890679in}{3.513574in}}%
\pgfpathlineto{\pgfqpoint{1.036968in}{3.609585in}}%
\pgfpathlineto{\pgfqpoint{1.183257in}{3.707513in}}%
\pgfpathlineto{\pgfqpoint{1.329546in}{3.810763in}}%
\pgfpathlineto{\pgfqpoint{1.475835in}{3.912795in}}%
\pgfpathlineto{\pgfqpoint{1.622125in}{4.016713in}}%
\pgfpathlineto{\pgfqpoint{1.768414in}{4.119317in}}%
\pgfpathlineto{\pgfqpoint{1.914703in}{4.224191in}}%
\pgfpathlineto{\pgfqpoint{2.060992in}{4.318801in}}%
\pgfpathlineto{\pgfqpoint{2.207281in}{4.432780in}}%
\pgfpathlineto{\pgfqpoint{2.353570in}{4.534088in}}%
\pgfpathlineto{\pgfqpoint{2.499859in}{4.643839in}}%
\pgfpathlineto{\pgfqpoint{2.646148in}{4.748403in}}%
\pgfpathlineto{\pgfqpoint{2.792437in}{4.861764in}}%
\pgfpathlineto{\pgfqpoint{2.938726in}{4.965614in}}%
\pgfpathlineto{\pgfqpoint{3.085015in}{5.083960in}}%
\pgfpathlineto{\pgfqpoint{3.231305in}{5.175747in}}%
\pgfpathlineto{\pgfqpoint{3.377594in}{5.297838in}}%
\pgfpathlineto{\pgfqpoint{3.523883in}{5.387394in}}%
\pgfpathlineto{\pgfqpoint{3.670172in}{5.502437in}}%
\pgfpathlineto{\pgfqpoint{3.670172in}{5.520456in}}%
\pgfpathlineto{\pgfqpoint{3.670172in}{5.520456in}}%
\pgfpathlineto{\pgfqpoint{3.523883in}{5.406436in}}%
\pgfpathlineto{\pgfqpoint{3.377594in}{5.305779in}}%
\pgfpathlineto{\pgfqpoint{3.231305in}{5.209678in}}%
\pgfpathlineto{\pgfqpoint{3.085015in}{5.094456in}}%
\pgfpathlineto{\pgfqpoint{2.938726in}{4.984231in}}%
\pgfpathlineto{\pgfqpoint{2.792437in}{4.876044in}}%
\pgfpathlineto{\pgfqpoint{2.646148in}{4.770180in}}%
\pgfpathlineto{\pgfqpoint{2.499859in}{4.652739in}}%
\pgfpathlineto{\pgfqpoint{2.353570in}{4.560695in}}%
\pgfpathlineto{\pgfqpoint{2.207281in}{4.445748in}}%
\pgfpathlineto{\pgfqpoint{2.060992in}{4.340250in}}%
\pgfpathlineto{\pgfqpoint{1.914703in}{4.238549in}}%
\pgfpathlineto{\pgfqpoint{1.768414in}{4.129407in}}%
\pgfpathlineto{\pgfqpoint{1.622125in}{4.022728in}}%
\pgfpathlineto{\pgfqpoint{1.475835in}{3.932915in}}%
\pgfpathlineto{\pgfqpoint{1.329546in}{3.817405in}}%
\pgfpathlineto{\pgfqpoint{1.183257in}{3.723862in}}%
\pgfpathlineto{\pgfqpoint{1.036968in}{3.629845in}}%
\pgfpathlineto{\pgfqpoint{0.890679in}{3.524254in}}%
\pgfpathlineto{\pgfqpoint{0.744390in}{3.436431in}}%
\pgfpathclose%
\pgfusepath{stroke,fill}%
\end{pgfscope}%
\begin{pgfscope}%
\pgfpathrectangle{\pgfqpoint{0.585764in}{3.311111in}}{\pgfqpoint{3.243034in}{2.316667in}}%
\pgfusepath{clip}%
\pgfsetbuttcap%
\pgfsetroundjoin%
\definecolor{currentfill}{rgb}{0.121569,0.466667,0.705882}%
\pgfsetfillcolor{currentfill}%
\pgfsetlinewidth{1.003750pt}%
\definecolor{currentstroke}{rgb}{0.121569,0.466667,0.705882}%
\pgfsetstrokecolor{currentstroke}%
\pgfsetdash{}{0pt}%
\pgfsys@defobject{currentmarker}{\pgfqpoint{-0.009821in}{-0.009821in}}{\pgfqpoint{0.009821in}{0.009821in}}{%
\pgfpathmoveto{\pgfqpoint{0.000000in}{-0.009821in}}%
\pgfpathcurveto{\pgfqpoint{0.002605in}{-0.009821in}}{\pgfqpoint{0.005103in}{-0.008786in}}{\pgfqpoint{0.006944in}{-0.006944in}}%
\pgfpathcurveto{\pgfqpoint{0.008786in}{-0.005103in}}{\pgfqpoint{0.009821in}{-0.002605in}}{\pgfqpoint{0.009821in}{0.000000in}}%
\pgfpathcurveto{\pgfqpoint{0.009821in}{0.002605in}}{\pgfqpoint{0.008786in}{0.005103in}}{\pgfqpoint{0.006944in}{0.006944in}}%
\pgfpathcurveto{\pgfqpoint{0.005103in}{0.008786in}}{\pgfqpoint{0.002605in}{0.009821in}}{\pgfqpoint{0.000000in}{0.009821in}}%
\pgfpathcurveto{\pgfqpoint{-0.002605in}{0.009821in}}{\pgfqpoint{-0.005103in}{0.008786in}}{\pgfqpoint{-0.006944in}{0.006944in}}%
\pgfpathcurveto{\pgfqpoint{-0.008786in}{0.005103in}}{\pgfqpoint{-0.009821in}{0.002605in}}{\pgfqpoint{-0.009821in}{0.000000in}}%
\pgfpathcurveto{\pgfqpoint{-0.009821in}{-0.002605in}}{\pgfqpoint{-0.008786in}{-0.005103in}}{\pgfqpoint{-0.006944in}{-0.006944in}}%
\pgfpathcurveto{\pgfqpoint{-0.005103in}{-0.008786in}}{\pgfqpoint{-0.002605in}{-0.009821in}}{\pgfqpoint{0.000000in}{-0.009821in}}%
\pgfpathclose%
\pgfusepath{stroke,fill}%
}%
\begin{pgfscope}%
\pgfsys@transformshift{0.744390in}{3.427442in}%
\pgfsys@useobject{currentmarker}{}%
\end{pgfscope}%
\begin{pgfscope}%
\pgfsys@transformshift{0.890679in}{3.518914in}%
\pgfsys@useobject{currentmarker}{}%
\end{pgfscope}%
\begin{pgfscope}%
\pgfsys@transformshift{1.036968in}{3.619715in}%
\pgfsys@useobject{currentmarker}{}%
\end{pgfscope}%
\begin{pgfscope}%
\pgfsys@transformshift{1.183257in}{3.715687in}%
\pgfsys@useobject{currentmarker}{}%
\end{pgfscope}%
\begin{pgfscope}%
\pgfsys@transformshift{1.329546in}{3.814084in}%
\pgfsys@useobject{currentmarker}{}%
\end{pgfscope}%
\begin{pgfscope}%
\pgfsys@transformshift{1.475835in}{3.922855in}%
\pgfsys@useobject{currentmarker}{}%
\end{pgfscope}%
\begin{pgfscope}%
\pgfsys@transformshift{1.622125in}{4.019721in}%
\pgfsys@useobject{currentmarker}{}%
\end{pgfscope}%
\begin{pgfscope}%
\pgfsys@transformshift{1.768414in}{4.124362in}%
\pgfsys@useobject{currentmarker}{}%
\end{pgfscope}%
\begin{pgfscope}%
\pgfsys@transformshift{1.914703in}{4.231370in}%
\pgfsys@useobject{currentmarker}{}%
\end{pgfscope}%
\begin{pgfscope}%
\pgfsys@transformshift{2.060992in}{4.329526in}%
\pgfsys@useobject{currentmarker}{}%
\end{pgfscope}%
\begin{pgfscope}%
\pgfsys@transformshift{2.207281in}{4.439264in}%
\pgfsys@useobject{currentmarker}{}%
\end{pgfscope}%
\begin{pgfscope}%
\pgfsys@transformshift{2.353570in}{4.547391in}%
\pgfsys@useobject{currentmarker}{}%
\end{pgfscope}%
\begin{pgfscope}%
\pgfsys@transformshift{2.499859in}{4.648289in}%
\pgfsys@useobject{currentmarker}{}%
\end{pgfscope}%
\begin{pgfscope}%
\pgfsys@transformshift{2.646148in}{4.759291in}%
\pgfsys@useobject{currentmarker}{}%
\end{pgfscope}%
\begin{pgfscope}%
\pgfsys@transformshift{2.792437in}{4.868904in}%
\pgfsys@useobject{currentmarker}{}%
\end{pgfscope}%
\begin{pgfscope}%
\pgfsys@transformshift{2.938726in}{4.974922in}%
\pgfsys@useobject{currentmarker}{}%
\end{pgfscope}%
\begin{pgfscope}%
\pgfsys@transformshift{3.085015in}{5.089208in}%
\pgfsys@useobject{currentmarker}{}%
\end{pgfscope}%
\begin{pgfscope}%
\pgfsys@transformshift{3.231305in}{5.192713in}%
\pgfsys@useobject{currentmarker}{}%
\end{pgfscope}%
\begin{pgfscope}%
\pgfsys@transformshift{3.377594in}{5.301808in}%
\pgfsys@useobject{currentmarker}{}%
\end{pgfscope}%
\begin{pgfscope}%
\pgfsys@transformshift{3.523883in}{5.396915in}%
\pgfsys@useobject{currentmarker}{}%
\end{pgfscope}%
\begin{pgfscope}%
\pgfsys@transformshift{3.670172in}{5.511447in}%
\pgfsys@useobject{currentmarker}{}%
\end{pgfscope}%
\end{pgfscope}%
\begin{pgfscope}%
\pgfsetbuttcap%
\pgfsetroundjoin%
\definecolor{currentfill}{rgb}{0.000000,0.000000,0.000000}%
\pgfsetfillcolor{currentfill}%
\pgfsetlinewidth{0.803000pt}%
\definecolor{currentstroke}{rgb}{0.000000,0.000000,0.000000}%
\pgfsetstrokecolor{currentstroke}%
\pgfsetdash{}{0pt}%
\pgfsys@defobject{currentmarker}{\pgfqpoint{0.000000in}{-0.048611in}}{\pgfqpoint{0.000000in}{0.000000in}}{%
\pgfpathmoveto{\pgfqpoint{0.000000in}{0.000000in}}%
\pgfpathlineto{\pgfqpoint{0.000000in}{-0.048611in}}%
\pgfusepath{stroke,fill}%
}%
\begin{pgfscope}%
\pgfsys@transformshift{0.598101in}{3.311111in}%
\pgfsys@useobject{currentmarker}{}%
\end{pgfscope}%
\end{pgfscope}%
\begin{pgfscope}%
\definecolor{textcolor}{rgb}{0.000000,0.000000,0.000000}%
\pgfsetstrokecolor{textcolor}%
\pgfsetfillcolor{textcolor}%
\pgftext[x=0.598101in,y=3.213889in,,top]{\color{textcolor}\sffamily\fontsize{10.000000}{12.000000}\selectfont 4.20}%
\end{pgfscope}%
\begin{pgfscope}%
\pgfsetbuttcap%
\pgfsetroundjoin%
\definecolor{currentfill}{rgb}{0.000000,0.000000,0.000000}%
\pgfsetfillcolor{currentfill}%
\pgfsetlinewidth{0.803000pt}%
\definecolor{currentstroke}{rgb}{0.000000,0.000000,0.000000}%
\pgfsetstrokecolor{currentstroke}%
\pgfsetdash{}{0pt}%
\pgfsys@defobject{currentmarker}{\pgfqpoint{0.000000in}{-0.048611in}}{\pgfqpoint{0.000000in}{0.000000in}}{%
\pgfpathmoveto{\pgfqpoint{0.000000in}{0.000000in}}%
\pgfpathlineto{\pgfqpoint{0.000000in}{-0.048611in}}%
\pgfusepath{stroke,fill}%
}%
\begin{pgfscope}%
\pgfsys@transformshift{1.329546in}{3.311111in}%
\pgfsys@useobject{currentmarker}{}%
\end{pgfscope}%
\end{pgfscope}%
\begin{pgfscope}%
\definecolor{textcolor}{rgb}{0.000000,0.000000,0.000000}%
\pgfsetstrokecolor{textcolor}%
\pgfsetfillcolor{textcolor}%
\pgftext[x=1.329546in,y=3.213889in,,top]{\color{textcolor}\sffamily\fontsize{10.000000}{12.000000}\selectfont 4.25}%
\end{pgfscope}%
\begin{pgfscope}%
\pgfsetbuttcap%
\pgfsetroundjoin%
\definecolor{currentfill}{rgb}{0.000000,0.000000,0.000000}%
\pgfsetfillcolor{currentfill}%
\pgfsetlinewidth{0.803000pt}%
\definecolor{currentstroke}{rgb}{0.000000,0.000000,0.000000}%
\pgfsetstrokecolor{currentstroke}%
\pgfsetdash{}{0pt}%
\pgfsys@defobject{currentmarker}{\pgfqpoint{0.000000in}{-0.048611in}}{\pgfqpoint{0.000000in}{0.000000in}}{%
\pgfpathmoveto{\pgfqpoint{0.000000in}{0.000000in}}%
\pgfpathlineto{\pgfqpoint{0.000000in}{-0.048611in}}%
\pgfusepath{stroke,fill}%
}%
\begin{pgfscope}%
\pgfsys@transformshift{2.060992in}{3.311111in}%
\pgfsys@useobject{currentmarker}{}%
\end{pgfscope}%
\end{pgfscope}%
\begin{pgfscope}%
\definecolor{textcolor}{rgb}{0.000000,0.000000,0.000000}%
\pgfsetstrokecolor{textcolor}%
\pgfsetfillcolor{textcolor}%
\pgftext[x=2.060992in,y=3.213889in,,top]{\color{textcolor}\sffamily\fontsize{10.000000}{12.000000}\selectfont 4.30}%
\end{pgfscope}%
\begin{pgfscope}%
\pgfsetbuttcap%
\pgfsetroundjoin%
\definecolor{currentfill}{rgb}{0.000000,0.000000,0.000000}%
\pgfsetfillcolor{currentfill}%
\pgfsetlinewidth{0.803000pt}%
\definecolor{currentstroke}{rgb}{0.000000,0.000000,0.000000}%
\pgfsetstrokecolor{currentstroke}%
\pgfsetdash{}{0pt}%
\pgfsys@defobject{currentmarker}{\pgfqpoint{0.000000in}{-0.048611in}}{\pgfqpoint{0.000000in}{0.000000in}}{%
\pgfpathmoveto{\pgfqpoint{0.000000in}{0.000000in}}%
\pgfpathlineto{\pgfqpoint{0.000000in}{-0.048611in}}%
\pgfusepath{stroke,fill}%
}%
\begin{pgfscope}%
\pgfsys@transformshift{2.792437in}{3.311111in}%
\pgfsys@useobject{currentmarker}{}%
\end{pgfscope}%
\end{pgfscope}%
\begin{pgfscope}%
\definecolor{textcolor}{rgb}{0.000000,0.000000,0.000000}%
\pgfsetstrokecolor{textcolor}%
\pgfsetfillcolor{textcolor}%
\pgftext[x=2.792437in,y=3.213889in,,top]{\color{textcolor}\sffamily\fontsize{10.000000}{12.000000}\selectfont 4.35}%
\end{pgfscope}%
\begin{pgfscope}%
\pgfsetbuttcap%
\pgfsetroundjoin%
\definecolor{currentfill}{rgb}{0.000000,0.000000,0.000000}%
\pgfsetfillcolor{currentfill}%
\pgfsetlinewidth{0.803000pt}%
\definecolor{currentstroke}{rgb}{0.000000,0.000000,0.000000}%
\pgfsetstrokecolor{currentstroke}%
\pgfsetdash{}{0pt}%
\pgfsys@defobject{currentmarker}{\pgfqpoint{0.000000in}{-0.048611in}}{\pgfqpoint{0.000000in}{0.000000in}}{%
\pgfpathmoveto{\pgfqpoint{0.000000in}{0.000000in}}%
\pgfpathlineto{\pgfqpoint{0.000000in}{-0.048611in}}%
\pgfusepath{stroke,fill}%
}%
\begin{pgfscope}%
\pgfsys@transformshift{3.523883in}{3.311111in}%
\pgfsys@useobject{currentmarker}{}%
\end{pgfscope}%
\end{pgfscope}%
\begin{pgfscope}%
\definecolor{textcolor}{rgb}{0.000000,0.000000,0.000000}%
\pgfsetstrokecolor{textcolor}%
\pgfsetfillcolor{textcolor}%
\pgftext[x=3.523883in,y=3.213889in,,top]{\color{textcolor}\sffamily\fontsize{10.000000}{12.000000}\selectfont 4.40}%
\end{pgfscope}%
\begin{pgfscope}%
\pgfsetbuttcap%
\pgfsetroundjoin%
\definecolor{currentfill}{rgb}{0.000000,0.000000,0.000000}%
\pgfsetfillcolor{currentfill}%
\pgfsetlinewidth{0.803000pt}%
\definecolor{currentstroke}{rgb}{0.000000,0.000000,0.000000}%
\pgfsetstrokecolor{currentstroke}%
\pgfsetdash{}{0pt}%
\pgfsys@defobject{currentmarker}{\pgfqpoint{-0.048611in}{0.000000in}}{\pgfqpoint{0.000000in}{0.000000in}}{%
\pgfpathmoveto{\pgfqpoint{0.000000in}{0.000000in}}%
\pgfpathlineto{\pgfqpoint{-0.048611in}{0.000000in}}%
\pgfusepath{stroke,fill}%
}%
\begin{pgfscope}%
\pgfsys@transformshift{0.585764in}{3.510120in}%
\pgfsys@useobject{currentmarker}{}%
\end{pgfscope}%
\end{pgfscope}%
\begin{pgfscope}%
\definecolor{textcolor}{rgb}{0.000000,0.000000,0.000000}%
\pgfsetstrokecolor{textcolor}%
\pgfsetfillcolor{textcolor}%
\pgftext[x=0.062923in,y=3.457359in,left,base]{\color{textcolor}\sffamily\fontsize{10.000000}{12.000000}\selectfont −1.60}%
\end{pgfscope}%
\begin{pgfscope}%
\pgfsetbuttcap%
\pgfsetroundjoin%
\definecolor{currentfill}{rgb}{0.000000,0.000000,0.000000}%
\pgfsetfillcolor{currentfill}%
\pgfsetlinewidth{0.803000pt}%
\definecolor{currentstroke}{rgb}{0.000000,0.000000,0.000000}%
\pgfsetstrokecolor{currentstroke}%
\pgfsetdash{}{0pt}%
\pgfsys@defobject{currentmarker}{\pgfqpoint{-0.048611in}{0.000000in}}{\pgfqpoint{0.000000in}{0.000000in}}{%
\pgfpathmoveto{\pgfqpoint{0.000000in}{0.000000in}}%
\pgfpathlineto{\pgfqpoint{-0.048611in}{0.000000in}}%
\pgfusepath{stroke,fill}%
}%
\begin{pgfscope}%
\pgfsys@transformshift{0.585764in}{3.812141in}%
\pgfsys@useobject{currentmarker}{}%
\end{pgfscope}%
\end{pgfscope}%
\begin{pgfscope}%
\definecolor{textcolor}{rgb}{0.000000,0.000000,0.000000}%
\pgfsetstrokecolor{textcolor}%
\pgfsetfillcolor{textcolor}%
\pgftext[x=0.062923in,y=3.759379in,left,base]{\color{textcolor}\sffamily\fontsize{10.000000}{12.000000}\selectfont −1.55}%
\end{pgfscope}%
\begin{pgfscope}%
\pgfsetbuttcap%
\pgfsetroundjoin%
\definecolor{currentfill}{rgb}{0.000000,0.000000,0.000000}%
\pgfsetfillcolor{currentfill}%
\pgfsetlinewidth{0.803000pt}%
\definecolor{currentstroke}{rgb}{0.000000,0.000000,0.000000}%
\pgfsetstrokecolor{currentstroke}%
\pgfsetdash{}{0pt}%
\pgfsys@defobject{currentmarker}{\pgfqpoint{-0.048611in}{0.000000in}}{\pgfqpoint{0.000000in}{0.000000in}}{%
\pgfpathmoveto{\pgfqpoint{0.000000in}{0.000000in}}%
\pgfpathlineto{\pgfqpoint{-0.048611in}{0.000000in}}%
\pgfusepath{stroke,fill}%
}%
\begin{pgfscope}%
\pgfsys@transformshift{0.585764in}{4.114161in}%
\pgfsys@useobject{currentmarker}{}%
\end{pgfscope}%
\end{pgfscope}%
\begin{pgfscope}%
\definecolor{textcolor}{rgb}{0.000000,0.000000,0.000000}%
\pgfsetstrokecolor{textcolor}%
\pgfsetfillcolor{textcolor}%
\pgftext[x=0.062923in,y=4.061400in,left,base]{\color{textcolor}\sffamily\fontsize{10.000000}{12.000000}\selectfont −1.50}%
\end{pgfscope}%
\begin{pgfscope}%
\pgfsetbuttcap%
\pgfsetroundjoin%
\definecolor{currentfill}{rgb}{0.000000,0.000000,0.000000}%
\pgfsetfillcolor{currentfill}%
\pgfsetlinewidth{0.803000pt}%
\definecolor{currentstroke}{rgb}{0.000000,0.000000,0.000000}%
\pgfsetstrokecolor{currentstroke}%
\pgfsetdash{}{0pt}%
\pgfsys@defobject{currentmarker}{\pgfqpoint{-0.048611in}{0.000000in}}{\pgfqpoint{0.000000in}{0.000000in}}{%
\pgfpathmoveto{\pgfqpoint{0.000000in}{0.000000in}}%
\pgfpathlineto{\pgfqpoint{-0.048611in}{0.000000in}}%
\pgfusepath{stroke,fill}%
}%
\begin{pgfscope}%
\pgfsys@transformshift{0.585764in}{4.416182in}%
\pgfsys@useobject{currentmarker}{}%
\end{pgfscope}%
\end{pgfscope}%
\begin{pgfscope}%
\definecolor{textcolor}{rgb}{0.000000,0.000000,0.000000}%
\pgfsetstrokecolor{textcolor}%
\pgfsetfillcolor{textcolor}%
\pgftext[x=0.062923in,y=4.363420in,left,base]{\color{textcolor}\sffamily\fontsize{10.000000}{12.000000}\selectfont −1.45}%
\end{pgfscope}%
\begin{pgfscope}%
\pgfsetbuttcap%
\pgfsetroundjoin%
\definecolor{currentfill}{rgb}{0.000000,0.000000,0.000000}%
\pgfsetfillcolor{currentfill}%
\pgfsetlinewidth{0.803000pt}%
\definecolor{currentstroke}{rgb}{0.000000,0.000000,0.000000}%
\pgfsetstrokecolor{currentstroke}%
\pgfsetdash{}{0pt}%
\pgfsys@defobject{currentmarker}{\pgfqpoint{-0.048611in}{0.000000in}}{\pgfqpoint{0.000000in}{0.000000in}}{%
\pgfpathmoveto{\pgfqpoint{0.000000in}{0.000000in}}%
\pgfpathlineto{\pgfqpoint{-0.048611in}{0.000000in}}%
\pgfusepath{stroke,fill}%
}%
\begin{pgfscope}%
\pgfsys@transformshift{0.585764in}{4.718202in}%
\pgfsys@useobject{currentmarker}{}%
\end{pgfscope}%
\end{pgfscope}%
\begin{pgfscope}%
\definecolor{textcolor}{rgb}{0.000000,0.000000,0.000000}%
\pgfsetstrokecolor{textcolor}%
\pgfsetfillcolor{textcolor}%
\pgftext[x=0.062923in,y=4.665441in,left,base]{\color{textcolor}\sffamily\fontsize{10.000000}{12.000000}\selectfont −1.40}%
\end{pgfscope}%
\begin{pgfscope}%
\pgfsetbuttcap%
\pgfsetroundjoin%
\definecolor{currentfill}{rgb}{0.000000,0.000000,0.000000}%
\pgfsetfillcolor{currentfill}%
\pgfsetlinewidth{0.803000pt}%
\definecolor{currentstroke}{rgb}{0.000000,0.000000,0.000000}%
\pgfsetstrokecolor{currentstroke}%
\pgfsetdash{}{0pt}%
\pgfsys@defobject{currentmarker}{\pgfqpoint{-0.048611in}{0.000000in}}{\pgfqpoint{0.000000in}{0.000000in}}{%
\pgfpathmoveto{\pgfqpoint{0.000000in}{0.000000in}}%
\pgfpathlineto{\pgfqpoint{-0.048611in}{0.000000in}}%
\pgfusepath{stroke,fill}%
}%
\begin{pgfscope}%
\pgfsys@transformshift{0.585764in}{5.020223in}%
\pgfsys@useobject{currentmarker}{}%
\end{pgfscope}%
\end{pgfscope}%
\begin{pgfscope}%
\definecolor{textcolor}{rgb}{0.000000,0.000000,0.000000}%
\pgfsetstrokecolor{textcolor}%
\pgfsetfillcolor{textcolor}%
\pgftext[x=0.062923in,y=4.967461in,left,base]{\color{textcolor}\sffamily\fontsize{10.000000}{12.000000}\selectfont −1.35}%
\end{pgfscope}%
\begin{pgfscope}%
\pgfsetbuttcap%
\pgfsetroundjoin%
\definecolor{currentfill}{rgb}{0.000000,0.000000,0.000000}%
\pgfsetfillcolor{currentfill}%
\pgfsetlinewidth{0.803000pt}%
\definecolor{currentstroke}{rgb}{0.000000,0.000000,0.000000}%
\pgfsetstrokecolor{currentstroke}%
\pgfsetdash{}{0pt}%
\pgfsys@defobject{currentmarker}{\pgfqpoint{-0.048611in}{0.000000in}}{\pgfqpoint{0.000000in}{0.000000in}}{%
\pgfpathmoveto{\pgfqpoint{0.000000in}{0.000000in}}%
\pgfpathlineto{\pgfqpoint{-0.048611in}{0.000000in}}%
\pgfusepath{stroke,fill}%
}%
\begin{pgfscope}%
\pgfsys@transformshift{0.585764in}{5.322243in}%
\pgfsys@useobject{currentmarker}{}%
\end{pgfscope}%
\end{pgfscope}%
\begin{pgfscope}%
\definecolor{textcolor}{rgb}{0.000000,0.000000,0.000000}%
\pgfsetstrokecolor{textcolor}%
\pgfsetfillcolor{textcolor}%
\pgftext[x=0.062923in,y=5.269482in,left,base]{\color{textcolor}\sffamily\fontsize{10.000000}{12.000000}\selectfont −1.30}%
\end{pgfscope}%
\begin{pgfscope}%
\pgfsetbuttcap%
\pgfsetroundjoin%
\definecolor{currentfill}{rgb}{0.000000,0.000000,0.000000}%
\pgfsetfillcolor{currentfill}%
\pgfsetlinewidth{0.803000pt}%
\definecolor{currentstroke}{rgb}{0.000000,0.000000,0.000000}%
\pgfsetstrokecolor{currentstroke}%
\pgfsetdash{}{0pt}%
\pgfsys@defobject{currentmarker}{\pgfqpoint{-0.048611in}{0.000000in}}{\pgfqpoint{0.000000in}{0.000000in}}{%
\pgfpathmoveto{\pgfqpoint{0.000000in}{0.000000in}}%
\pgfpathlineto{\pgfqpoint{-0.048611in}{0.000000in}}%
\pgfusepath{stroke,fill}%
}%
\begin{pgfscope}%
\pgfsys@transformshift{0.585764in}{5.624264in}%
\pgfsys@useobject{currentmarker}{}%
\end{pgfscope}%
\end{pgfscope}%
\begin{pgfscope}%
\definecolor{textcolor}{rgb}{0.000000,0.000000,0.000000}%
\pgfsetstrokecolor{textcolor}%
\pgfsetfillcolor{textcolor}%
\pgftext[x=0.062923in,y=5.571502in,left,base]{\color{textcolor}\sffamily\fontsize{10.000000}{12.000000}\selectfont −1.25}%
\end{pgfscope}%
\begin{pgfscope}%
\pgfpathrectangle{\pgfqpoint{0.585764in}{3.311111in}}{\pgfqpoint{3.243034in}{2.316667in}}%
\pgfusepath{clip}%
\pgfsetrectcap%
\pgfsetroundjoin%
\pgfsetlinewidth{1.505625pt}%
\definecolor{currentstroke}{rgb}{0.121569,0.466667,0.705882}%
\pgfsetstrokecolor{currentstroke}%
\pgfsetdash{}{0pt}%
\pgfpathmoveto{\pgfqpoint{0.744390in}{3.427442in}}%
\pgfpathlineto{\pgfqpoint{0.890679in}{3.518914in}}%
\pgfpathlineto{\pgfqpoint{1.036968in}{3.619715in}}%
\pgfpathlineto{\pgfqpoint{1.183257in}{3.715687in}}%
\pgfpathlineto{\pgfqpoint{1.329546in}{3.814084in}}%
\pgfpathlineto{\pgfqpoint{1.475835in}{3.922855in}}%
\pgfpathlineto{\pgfqpoint{1.622125in}{4.019721in}}%
\pgfpathlineto{\pgfqpoint{1.768414in}{4.124362in}}%
\pgfpathlineto{\pgfqpoint{1.914703in}{4.231370in}}%
\pgfpathlineto{\pgfqpoint{2.060992in}{4.329526in}}%
\pgfpathlineto{\pgfqpoint{2.207281in}{4.439264in}}%
\pgfpathlineto{\pgfqpoint{2.353570in}{4.547391in}}%
\pgfpathlineto{\pgfqpoint{2.499859in}{4.648289in}}%
\pgfpathlineto{\pgfqpoint{2.646148in}{4.759291in}}%
\pgfpathlineto{\pgfqpoint{2.792437in}{4.868904in}}%
\pgfpathlineto{\pgfqpoint{2.938726in}{4.974922in}}%
\pgfpathlineto{\pgfqpoint{3.085015in}{5.089208in}}%
\pgfpathlineto{\pgfqpoint{3.231305in}{5.192713in}}%
\pgfpathlineto{\pgfqpoint{3.377594in}{5.301808in}}%
\pgfpathlineto{\pgfqpoint{3.523883in}{5.396915in}}%
\pgfpathlineto{\pgfqpoint{3.670172in}{5.511447in}}%
\pgfusepath{stroke}%
\end{pgfscope}%
\begin{pgfscope}%
\pgfsetrectcap%
\pgfsetmiterjoin%
\pgfsetlinewidth{0.803000pt}%
\definecolor{currentstroke}{rgb}{0.000000,0.000000,0.000000}%
\pgfsetstrokecolor{currentstroke}%
\pgfsetdash{}{0pt}%
\pgfpathmoveto{\pgfqpoint{0.585764in}{3.311111in}}%
\pgfpathlineto{\pgfqpoint{0.585764in}{5.627778in}}%
\pgfusepath{stroke}%
\end{pgfscope}%
\begin{pgfscope}%
\pgfsetrectcap%
\pgfsetmiterjoin%
\pgfsetlinewidth{0.803000pt}%
\definecolor{currentstroke}{rgb}{0.000000,0.000000,0.000000}%
\pgfsetstrokecolor{currentstroke}%
\pgfsetdash{}{0pt}%
\pgfpathmoveto{\pgfqpoint{3.828798in}{3.311111in}}%
\pgfpathlineto{\pgfqpoint{3.828798in}{5.627778in}}%
\pgfusepath{stroke}%
\end{pgfscope}%
\begin{pgfscope}%
\pgfsetrectcap%
\pgfsetmiterjoin%
\pgfsetlinewidth{0.803000pt}%
\definecolor{currentstroke}{rgb}{0.000000,0.000000,0.000000}%
\pgfsetstrokecolor{currentstroke}%
\pgfsetdash{}{0pt}%
\pgfpathmoveto{\pgfqpoint{0.585764in}{3.311111in}}%
\pgfpathlineto{\pgfqpoint{3.828798in}{3.311111in}}%
\pgfusepath{stroke}%
\end{pgfscope}%
\begin{pgfscope}%
\pgfsetrectcap%
\pgfsetmiterjoin%
\pgfsetlinewidth{0.803000pt}%
\definecolor{currentstroke}{rgb}{0.000000,0.000000,0.000000}%
\pgfsetstrokecolor{currentstroke}%
\pgfsetdash{}{0pt}%
\pgfpathmoveto{\pgfqpoint{0.585764in}{5.627778in}}%
\pgfpathlineto{\pgfqpoint{3.828798in}{5.627778in}}%
\pgfusepath{stroke}%
\end{pgfscope}%
\begin{pgfscope}%
\definecolor{textcolor}{rgb}{0.000000,0.000000,0.000000}%
\pgfsetstrokecolor{textcolor}%
\pgfsetfillcolor{textcolor}%
\pgftext[x=2.207281in,y=5.711111in,,base]{\color{textcolor}\sffamily\fontsize{12.000000}{14.400000}\selectfont \(\displaystyle  N = 8 \)}%
\end{pgfscope}%
\begin{pgfscope}%
\pgfsetbuttcap%
\pgfsetmiterjoin%
\definecolor{currentfill}{rgb}{1.000000,1.000000,1.000000}%
\pgfsetfillcolor{currentfill}%
\pgfsetlinewidth{0.000000pt}%
\definecolor{currentstroke}{rgb}{0.000000,0.000000,0.000000}%
\pgfsetstrokecolor{currentstroke}%
\pgfsetstrokeopacity{0.000000}%
\pgfsetdash{}{0pt}%
\pgfpathmoveto{\pgfqpoint{4.558355in}{3.311111in}}%
\pgfpathlineto{\pgfqpoint{7.801389in}{3.311111in}}%
\pgfpathlineto{\pgfqpoint{7.801389in}{5.627778in}}%
\pgfpathlineto{\pgfqpoint{4.558355in}{5.627778in}}%
\pgfpathclose%
\pgfusepath{fill}%
\end{pgfscope}%
\begin{pgfscope}%
\pgfpathrectangle{\pgfqpoint{4.558355in}{3.311111in}}{\pgfqpoint{3.243034in}{2.316667in}}%
\pgfusepath{clip}%
\pgfsetbuttcap%
\pgfsetroundjoin%
\definecolor{currentfill}{rgb}{0.121569,0.466667,0.705882}%
\pgfsetfillcolor{currentfill}%
\pgfsetfillopacity{0.300000}%
\pgfsetlinewidth{1.003750pt}%
\definecolor{currentstroke}{rgb}{0.121569,0.466667,0.705882}%
\pgfsetstrokecolor{currentstroke}%
\pgfsetstrokeopacity{0.300000}%
\pgfsetdash{}{0pt}%
\pgfpathmoveto{\pgfqpoint{4.716981in}{3.437509in}}%
\pgfpathlineto{\pgfqpoint{4.716981in}{3.417408in}}%
\pgfpathlineto{\pgfqpoint{4.863270in}{3.526599in}}%
\pgfpathlineto{\pgfqpoint{5.009559in}{3.618580in}}%
\pgfpathlineto{\pgfqpoint{5.155848in}{3.717521in}}%
\pgfpathlineto{\pgfqpoint{5.302137in}{3.818657in}}%
\pgfpathlineto{\pgfqpoint{5.448426in}{3.916872in}}%
\pgfpathlineto{\pgfqpoint{5.594716in}{4.023687in}}%
\pgfpathlineto{\pgfqpoint{5.741005in}{4.137555in}}%
\pgfpathlineto{\pgfqpoint{5.887294in}{4.238638in}}%
\pgfpathlineto{\pgfqpoint{6.033583in}{4.340193in}}%
\pgfpathlineto{\pgfqpoint{6.179872in}{4.472279in}}%
\pgfpathlineto{\pgfqpoint{6.326161in}{4.580412in}}%
\pgfpathlineto{\pgfqpoint{6.472450in}{4.683937in}}%
\pgfpathlineto{\pgfqpoint{6.618739in}{4.788503in}}%
\pgfpathlineto{\pgfqpoint{6.765028in}{4.897868in}}%
\pgfpathlineto{\pgfqpoint{6.911317in}{4.990024in}}%
\pgfpathlineto{\pgfqpoint{7.057606in}{5.102347in}}%
\pgfpathlineto{\pgfqpoint{7.203895in}{5.219145in}}%
\pgfpathlineto{\pgfqpoint{7.350185in}{5.317908in}}%
\pgfpathlineto{\pgfqpoint{7.496474in}{5.399043in}}%
\pgfpathlineto{\pgfqpoint{7.642763in}{5.477705in}}%
\pgfpathlineto{\pgfqpoint{7.642763in}{5.522475in}}%
\pgfpathlineto{\pgfqpoint{7.642763in}{5.522475in}}%
\pgfpathlineto{\pgfqpoint{7.496474in}{5.428039in}}%
\pgfpathlineto{\pgfqpoint{7.350185in}{5.327717in}}%
\pgfpathlineto{\pgfqpoint{7.203895in}{5.257375in}}%
\pgfpathlineto{\pgfqpoint{7.057606in}{5.152000in}}%
\pgfpathlineto{\pgfqpoint{6.911317in}{5.026741in}}%
\pgfpathlineto{\pgfqpoint{6.765028in}{4.925264in}}%
\pgfpathlineto{\pgfqpoint{6.618739in}{4.806939in}}%
\pgfpathlineto{\pgfqpoint{6.472450in}{4.722758in}}%
\pgfpathlineto{\pgfqpoint{6.326161in}{4.590229in}}%
\pgfpathlineto{\pgfqpoint{6.179872in}{4.487174in}}%
\pgfpathlineto{\pgfqpoint{6.033583in}{4.369029in}}%
\pgfpathlineto{\pgfqpoint{5.887294in}{4.269611in}}%
\pgfpathlineto{\pgfqpoint{5.741005in}{4.158467in}}%
\pgfpathlineto{\pgfqpoint{5.594716in}{4.041871in}}%
\pgfpathlineto{\pgfqpoint{5.448426in}{3.955963in}}%
\pgfpathlineto{\pgfqpoint{5.302137in}{3.838158in}}%
\pgfpathlineto{\pgfqpoint{5.155848in}{3.742021in}}%
\pgfpathlineto{\pgfqpoint{5.009559in}{3.649636in}}%
\pgfpathlineto{\pgfqpoint{4.863270in}{3.542969in}}%
\pgfpathlineto{\pgfqpoint{4.716981in}{3.437509in}}%
\pgfpathclose%
\pgfusepath{stroke,fill}%
\end{pgfscope}%
\begin{pgfscope}%
\pgfpathrectangle{\pgfqpoint{4.558355in}{3.311111in}}{\pgfqpoint{3.243034in}{2.316667in}}%
\pgfusepath{clip}%
\pgfsetbuttcap%
\pgfsetroundjoin%
\definecolor{currentfill}{rgb}{0.121569,0.466667,0.705882}%
\pgfsetfillcolor{currentfill}%
\pgfsetlinewidth{1.003750pt}%
\definecolor{currentstroke}{rgb}{0.121569,0.466667,0.705882}%
\pgfsetstrokecolor{currentstroke}%
\pgfsetdash{}{0pt}%
\pgfsys@defobject{currentmarker}{\pgfqpoint{-0.009821in}{-0.009821in}}{\pgfqpoint{0.009821in}{0.009821in}}{%
\pgfpathmoveto{\pgfqpoint{0.000000in}{-0.009821in}}%
\pgfpathcurveto{\pgfqpoint{0.002605in}{-0.009821in}}{\pgfqpoint{0.005103in}{-0.008786in}}{\pgfqpoint{0.006944in}{-0.006944in}}%
\pgfpathcurveto{\pgfqpoint{0.008786in}{-0.005103in}}{\pgfqpoint{0.009821in}{-0.002605in}}{\pgfqpoint{0.009821in}{0.000000in}}%
\pgfpathcurveto{\pgfqpoint{0.009821in}{0.002605in}}{\pgfqpoint{0.008786in}{0.005103in}}{\pgfqpoint{0.006944in}{0.006944in}}%
\pgfpathcurveto{\pgfqpoint{0.005103in}{0.008786in}}{\pgfqpoint{0.002605in}{0.009821in}}{\pgfqpoint{0.000000in}{0.009821in}}%
\pgfpathcurveto{\pgfqpoint{-0.002605in}{0.009821in}}{\pgfqpoint{-0.005103in}{0.008786in}}{\pgfqpoint{-0.006944in}{0.006944in}}%
\pgfpathcurveto{\pgfqpoint{-0.008786in}{0.005103in}}{\pgfqpoint{-0.009821in}{0.002605in}}{\pgfqpoint{-0.009821in}{0.000000in}}%
\pgfpathcurveto{\pgfqpoint{-0.009821in}{-0.002605in}}{\pgfqpoint{-0.008786in}{-0.005103in}}{\pgfqpoint{-0.006944in}{-0.006944in}}%
\pgfpathcurveto{\pgfqpoint{-0.005103in}{-0.008786in}}{\pgfqpoint{-0.002605in}{-0.009821in}}{\pgfqpoint{0.000000in}{-0.009821in}}%
\pgfpathclose%
\pgfusepath{stroke,fill}%
}%
\begin{pgfscope}%
\pgfsys@transformshift{4.716981in}{3.427458in}%
\pgfsys@useobject{currentmarker}{}%
\end{pgfscope}%
\begin{pgfscope}%
\pgfsys@transformshift{4.863270in}{3.534784in}%
\pgfsys@useobject{currentmarker}{}%
\end{pgfscope}%
\begin{pgfscope}%
\pgfsys@transformshift{5.009559in}{3.634108in}%
\pgfsys@useobject{currentmarker}{}%
\end{pgfscope}%
\begin{pgfscope}%
\pgfsys@transformshift{5.155848in}{3.729771in}%
\pgfsys@useobject{currentmarker}{}%
\end{pgfscope}%
\begin{pgfscope}%
\pgfsys@transformshift{5.302137in}{3.828408in}%
\pgfsys@useobject{currentmarker}{}%
\end{pgfscope}%
\begin{pgfscope}%
\pgfsys@transformshift{5.448426in}{3.936417in}%
\pgfsys@useobject{currentmarker}{}%
\end{pgfscope}%
\begin{pgfscope}%
\pgfsys@transformshift{5.594716in}{4.032779in}%
\pgfsys@useobject{currentmarker}{}%
\end{pgfscope}%
\begin{pgfscope}%
\pgfsys@transformshift{5.741005in}{4.148011in}%
\pgfsys@useobject{currentmarker}{}%
\end{pgfscope}%
\begin{pgfscope}%
\pgfsys@transformshift{5.887294in}{4.254124in}%
\pgfsys@useobject{currentmarker}{}%
\end{pgfscope}%
\begin{pgfscope}%
\pgfsys@transformshift{6.033583in}{4.354611in}%
\pgfsys@useobject{currentmarker}{}%
\end{pgfscope}%
\begin{pgfscope}%
\pgfsys@transformshift{6.179872in}{4.479726in}%
\pgfsys@useobject{currentmarker}{}%
\end{pgfscope}%
\begin{pgfscope}%
\pgfsys@transformshift{6.326161in}{4.585320in}%
\pgfsys@useobject{currentmarker}{}%
\end{pgfscope}%
\begin{pgfscope}%
\pgfsys@transformshift{6.472450in}{4.703348in}%
\pgfsys@useobject{currentmarker}{}%
\end{pgfscope}%
\begin{pgfscope}%
\pgfsys@transformshift{6.618739in}{4.797721in}%
\pgfsys@useobject{currentmarker}{}%
\end{pgfscope}%
\begin{pgfscope}%
\pgfsys@transformshift{6.765028in}{4.911566in}%
\pgfsys@useobject{currentmarker}{}%
\end{pgfscope}%
\begin{pgfscope}%
\pgfsys@transformshift{6.911317in}{5.008383in}%
\pgfsys@useobject{currentmarker}{}%
\end{pgfscope}%
\begin{pgfscope}%
\pgfsys@transformshift{7.057606in}{5.127173in}%
\pgfsys@useobject{currentmarker}{}%
\end{pgfscope}%
\begin{pgfscope}%
\pgfsys@transformshift{7.203895in}{5.238260in}%
\pgfsys@useobject{currentmarker}{}%
\end{pgfscope}%
\begin{pgfscope}%
\pgfsys@transformshift{7.350185in}{5.322813in}%
\pgfsys@useobject{currentmarker}{}%
\end{pgfscope}%
\begin{pgfscope}%
\pgfsys@transformshift{7.496474in}{5.413541in}%
\pgfsys@useobject{currentmarker}{}%
\end{pgfscope}%
\begin{pgfscope}%
\pgfsys@transformshift{7.642763in}{5.500090in}%
\pgfsys@useobject{currentmarker}{}%
\end{pgfscope}%
\end{pgfscope}%
\begin{pgfscope}%
\pgfsetbuttcap%
\pgfsetroundjoin%
\definecolor{currentfill}{rgb}{0.000000,0.000000,0.000000}%
\pgfsetfillcolor{currentfill}%
\pgfsetlinewidth{0.803000pt}%
\definecolor{currentstroke}{rgb}{0.000000,0.000000,0.000000}%
\pgfsetstrokecolor{currentstroke}%
\pgfsetdash{}{0pt}%
\pgfsys@defobject{currentmarker}{\pgfqpoint{0.000000in}{-0.048611in}}{\pgfqpoint{0.000000in}{0.000000in}}{%
\pgfpathmoveto{\pgfqpoint{0.000000in}{0.000000in}}%
\pgfpathlineto{\pgfqpoint{0.000000in}{-0.048611in}}%
\pgfusepath{stroke,fill}%
}%
\begin{pgfscope}%
\pgfsys@transformshift{4.570692in}{3.311111in}%
\pgfsys@useobject{currentmarker}{}%
\end{pgfscope}%
\end{pgfscope}%
\begin{pgfscope}%
\definecolor{textcolor}{rgb}{0.000000,0.000000,0.000000}%
\pgfsetstrokecolor{textcolor}%
\pgfsetfillcolor{textcolor}%
\pgftext[x=4.570692in,y=3.213889in,,top]{\color{textcolor}\sffamily\fontsize{10.000000}{12.000000}\selectfont 4.30}%
\end{pgfscope}%
\begin{pgfscope}%
\pgfsetbuttcap%
\pgfsetroundjoin%
\definecolor{currentfill}{rgb}{0.000000,0.000000,0.000000}%
\pgfsetfillcolor{currentfill}%
\pgfsetlinewidth{0.803000pt}%
\definecolor{currentstroke}{rgb}{0.000000,0.000000,0.000000}%
\pgfsetstrokecolor{currentstroke}%
\pgfsetdash{}{0pt}%
\pgfsys@defobject{currentmarker}{\pgfqpoint{0.000000in}{-0.048611in}}{\pgfqpoint{0.000000in}{0.000000in}}{%
\pgfpathmoveto{\pgfqpoint{0.000000in}{0.000000in}}%
\pgfpathlineto{\pgfqpoint{0.000000in}{-0.048611in}}%
\pgfusepath{stroke,fill}%
}%
\begin{pgfscope}%
\pgfsys@transformshift{5.302137in}{3.311111in}%
\pgfsys@useobject{currentmarker}{}%
\end{pgfscope}%
\end{pgfscope}%
\begin{pgfscope}%
\definecolor{textcolor}{rgb}{0.000000,0.000000,0.000000}%
\pgfsetstrokecolor{textcolor}%
\pgfsetfillcolor{textcolor}%
\pgftext[x=5.302137in,y=3.213889in,,top]{\color{textcolor}\sffamily\fontsize{10.000000}{12.000000}\selectfont 4.35}%
\end{pgfscope}%
\begin{pgfscope}%
\pgfsetbuttcap%
\pgfsetroundjoin%
\definecolor{currentfill}{rgb}{0.000000,0.000000,0.000000}%
\pgfsetfillcolor{currentfill}%
\pgfsetlinewidth{0.803000pt}%
\definecolor{currentstroke}{rgb}{0.000000,0.000000,0.000000}%
\pgfsetstrokecolor{currentstroke}%
\pgfsetdash{}{0pt}%
\pgfsys@defobject{currentmarker}{\pgfqpoint{0.000000in}{-0.048611in}}{\pgfqpoint{0.000000in}{0.000000in}}{%
\pgfpathmoveto{\pgfqpoint{0.000000in}{0.000000in}}%
\pgfpathlineto{\pgfqpoint{0.000000in}{-0.048611in}}%
\pgfusepath{stroke,fill}%
}%
\begin{pgfscope}%
\pgfsys@transformshift{6.033583in}{3.311111in}%
\pgfsys@useobject{currentmarker}{}%
\end{pgfscope}%
\end{pgfscope}%
\begin{pgfscope}%
\definecolor{textcolor}{rgb}{0.000000,0.000000,0.000000}%
\pgfsetstrokecolor{textcolor}%
\pgfsetfillcolor{textcolor}%
\pgftext[x=6.033583in,y=3.213889in,,top]{\color{textcolor}\sffamily\fontsize{10.000000}{12.000000}\selectfont 4.40}%
\end{pgfscope}%
\begin{pgfscope}%
\pgfsetbuttcap%
\pgfsetroundjoin%
\definecolor{currentfill}{rgb}{0.000000,0.000000,0.000000}%
\pgfsetfillcolor{currentfill}%
\pgfsetlinewidth{0.803000pt}%
\definecolor{currentstroke}{rgb}{0.000000,0.000000,0.000000}%
\pgfsetstrokecolor{currentstroke}%
\pgfsetdash{}{0pt}%
\pgfsys@defobject{currentmarker}{\pgfqpoint{0.000000in}{-0.048611in}}{\pgfqpoint{0.000000in}{0.000000in}}{%
\pgfpathmoveto{\pgfqpoint{0.000000in}{0.000000in}}%
\pgfpathlineto{\pgfqpoint{0.000000in}{-0.048611in}}%
\pgfusepath{stroke,fill}%
}%
\begin{pgfscope}%
\pgfsys@transformshift{6.765028in}{3.311111in}%
\pgfsys@useobject{currentmarker}{}%
\end{pgfscope}%
\end{pgfscope}%
\begin{pgfscope}%
\definecolor{textcolor}{rgb}{0.000000,0.000000,0.000000}%
\pgfsetstrokecolor{textcolor}%
\pgfsetfillcolor{textcolor}%
\pgftext[x=6.765028in,y=3.213889in,,top]{\color{textcolor}\sffamily\fontsize{10.000000}{12.000000}\selectfont 4.45}%
\end{pgfscope}%
\begin{pgfscope}%
\pgfsetbuttcap%
\pgfsetroundjoin%
\definecolor{currentfill}{rgb}{0.000000,0.000000,0.000000}%
\pgfsetfillcolor{currentfill}%
\pgfsetlinewidth{0.803000pt}%
\definecolor{currentstroke}{rgb}{0.000000,0.000000,0.000000}%
\pgfsetstrokecolor{currentstroke}%
\pgfsetdash{}{0pt}%
\pgfsys@defobject{currentmarker}{\pgfqpoint{0.000000in}{-0.048611in}}{\pgfqpoint{0.000000in}{0.000000in}}{%
\pgfpathmoveto{\pgfqpoint{0.000000in}{0.000000in}}%
\pgfpathlineto{\pgfqpoint{0.000000in}{-0.048611in}}%
\pgfusepath{stroke,fill}%
}%
\begin{pgfscope}%
\pgfsys@transformshift{7.496474in}{3.311111in}%
\pgfsys@useobject{currentmarker}{}%
\end{pgfscope}%
\end{pgfscope}%
\begin{pgfscope}%
\definecolor{textcolor}{rgb}{0.000000,0.000000,0.000000}%
\pgfsetstrokecolor{textcolor}%
\pgfsetfillcolor{textcolor}%
\pgftext[x=7.496474in,y=3.213889in,,top]{\color{textcolor}\sffamily\fontsize{10.000000}{12.000000}\selectfont 4.50}%
\end{pgfscope}%
\begin{pgfscope}%
\pgfsetbuttcap%
\pgfsetroundjoin%
\definecolor{currentfill}{rgb}{0.000000,0.000000,0.000000}%
\pgfsetfillcolor{currentfill}%
\pgfsetlinewidth{0.803000pt}%
\definecolor{currentstroke}{rgb}{0.000000,0.000000,0.000000}%
\pgfsetstrokecolor{currentstroke}%
\pgfsetdash{}{0pt}%
\pgfsys@defobject{currentmarker}{\pgfqpoint{-0.048611in}{0.000000in}}{\pgfqpoint{0.000000in}{0.000000in}}{%
\pgfpathmoveto{\pgfqpoint{0.000000in}{0.000000in}}%
\pgfpathlineto{\pgfqpoint{-0.048611in}{0.000000in}}%
\pgfusepath{stroke,fill}%
}%
\begin{pgfscope}%
\pgfsys@transformshift{4.558355in}{3.715859in}%
\pgfsys@useobject{currentmarker}{}%
\end{pgfscope}%
\end{pgfscope}%
\begin{pgfscope}%
\definecolor{textcolor}{rgb}{0.000000,0.000000,0.000000}%
\pgfsetstrokecolor{textcolor}%
\pgfsetfillcolor{textcolor}%
\pgftext[x=4.123879in,y=3.663098in,left,base]{\color{textcolor}\sffamily\fontsize{10.000000}{12.000000}\selectfont −1.4}%
\end{pgfscope}%
\begin{pgfscope}%
\pgfsetbuttcap%
\pgfsetroundjoin%
\definecolor{currentfill}{rgb}{0.000000,0.000000,0.000000}%
\pgfsetfillcolor{currentfill}%
\pgfsetlinewidth{0.803000pt}%
\definecolor{currentstroke}{rgb}{0.000000,0.000000,0.000000}%
\pgfsetstrokecolor{currentstroke}%
\pgfsetdash{}{0pt}%
\pgfsys@defobject{currentmarker}{\pgfqpoint{-0.048611in}{0.000000in}}{\pgfqpoint{0.000000in}{0.000000in}}{%
\pgfpathmoveto{\pgfqpoint{0.000000in}{0.000000in}}%
\pgfpathlineto{\pgfqpoint{-0.048611in}{0.000000in}}%
\pgfusepath{stroke,fill}%
}%
\begin{pgfscope}%
\pgfsys@transformshift{4.558355in}{4.241017in}%
\pgfsys@useobject{currentmarker}{}%
\end{pgfscope}%
\end{pgfscope}%
\begin{pgfscope}%
\definecolor{textcolor}{rgb}{0.000000,0.000000,0.000000}%
\pgfsetstrokecolor{textcolor}%
\pgfsetfillcolor{textcolor}%
\pgftext[x=4.123879in,y=4.188255in,left,base]{\color{textcolor}\sffamily\fontsize{10.000000}{12.000000}\selectfont −1.3}%
\end{pgfscope}%
\begin{pgfscope}%
\pgfsetbuttcap%
\pgfsetroundjoin%
\definecolor{currentfill}{rgb}{0.000000,0.000000,0.000000}%
\pgfsetfillcolor{currentfill}%
\pgfsetlinewidth{0.803000pt}%
\definecolor{currentstroke}{rgb}{0.000000,0.000000,0.000000}%
\pgfsetstrokecolor{currentstroke}%
\pgfsetdash{}{0pt}%
\pgfsys@defobject{currentmarker}{\pgfqpoint{-0.048611in}{0.000000in}}{\pgfqpoint{0.000000in}{0.000000in}}{%
\pgfpathmoveto{\pgfqpoint{0.000000in}{0.000000in}}%
\pgfpathlineto{\pgfqpoint{-0.048611in}{0.000000in}}%
\pgfusepath{stroke,fill}%
}%
\begin{pgfscope}%
\pgfsys@transformshift{4.558355in}{4.766174in}%
\pgfsys@useobject{currentmarker}{}%
\end{pgfscope}%
\end{pgfscope}%
\begin{pgfscope}%
\definecolor{textcolor}{rgb}{0.000000,0.000000,0.000000}%
\pgfsetstrokecolor{textcolor}%
\pgfsetfillcolor{textcolor}%
\pgftext[x=4.123879in,y=4.713412in,left,base]{\color{textcolor}\sffamily\fontsize{10.000000}{12.000000}\selectfont −1.2}%
\end{pgfscope}%
\begin{pgfscope}%
\pgfsetbuttcap%
\pgfsetroundjoin%
\definecolor{currentfill}{rgb}{0.000000,0.000000,0.000000}%
\pgfsetfillcolor{currentfill}%
\pgfsetlinewidth{0.803000pt}%
\definecolor{currentstroke}{rgb}{0.000000,0.000000,0.000000}%
\pgfsetstrokecolor{currentstroke}%
\pgfsetdash{}{0pt}%
\pgfsys@defobject{currentmarker}{\pgfqpoint{-0.048611in}{0.000000in}}{\pgfqpoint{0.000000in}{0.000000in}}{%
\pgfpathmoveto{\pgfqpoint{0.000000in}{0.000000in}}%
\pgfpathlineto{\pgfqpoint{-0.048611in}{0.000000in}}%
\pgfusepath{stroke,fill}%
}%
\begin{pgfscope}%
\pgfsys@transformshift{4.558355in}{5.291331in}%
\pgfsys@useobject{currentmarker}{}%
\end{pgfscope}%
\end{pgfscope}%
\begin{pgfscope}%
\definecolor{textcolor}{rgb}{0.000000,0.000000,0.000000}%
\pgfsetstrokecolor{textcolor}%
\pgfsetfillcolor{textcolor}%
\pgftext[x=4.123879in,y=5.238570in,left,base]{\color{textcolor}\sffamily\fontsize{10.000000}{12.000000}\selectfont −1.1}%
\end{pgfscope}%
\begin{pgfscope}%
\pgfpathrectangle{\pgfqpoint{4.558355in}{3.311111in}}{\pgfqpoint{3.243034in}{2.316667in}}%
\pgfusepath{clip}%
\pgfsetrectcap%
\pgfsetroundjoin%
\pgfsetlinewidth{1.505625pt}%
\definecolor{currentstroke}{rgb}{0.121569,0.466667,0.705882}%
\pgfsetstrokecolor{currentstroke}%
\pgfsetdash{}{0pt}%
\pgfpathmoveto{\pgfqpoint{4.716981in}{3.427458in}}%
\pgfpathlineto{\pgfqpoint{4.863270in}{3.534784in}}%
\pgfpathlineto{\pgfqpoint{5.009559in}{3.634108in}}%
\pgfpathlineto{\pgfqpoint{5.155848in}{3.729771in}}%
\pgfpathlineto{\pgfqpoint{5.302137in}{3.828408in}}%
\pgfpathlineto{\pgfqpoint{5.448426in}{3.936417in}}%
\pgfpathlineto{\pgfqpoint{5.594716in}{4.032779in}}%
\pgfpathlineto{\pgfqpoint{5.741005in}{4.148011in}}%
\pgfpathlineto{\pgfqpoint{5.887294in}{4.254124in}}%
\pgfpathlineto{\pgfqpoint{6.033583in}{4.354611in}}%
\pgfpathlineto{\pgfqpoint{6.179872in}{4.479726in}}%
\pgfpathlineto{\pgfqpoint{6.326161in}{4.585320in}}%
\pgfpathlineto{\pgfqpoint{6.472450in}{4.703348in}}%
\pgfpathlineto{\pgfqpoint{6.618739in}{4.797721in}}%
\pgfpathlineto{\pgfqpoint{6.765028in}{4.911566in}}%
\pgfpathlineto{\pgfqpoint{6.911317in}{5.008383in}}%
\pgfpathlineto{\pgfqpoint{7.057606in}{5.127173in}}%
\pgfpathlineto{\pgfqpoint{7.203895in}{5.238260in}}%
\pgfpathlineto{\pgfqpoint{7.350185in}{5.322813in}}%
\pgfpathlineto{\pgfqpoint{7.496474in}{5.413541in}}%
\pgfpathlineto{\pgfqpoint{7.642763in}{5.500090in}}%
\pgfusepath{stroke}%
\end{pgfscope}%
\begin{pgfscope}%
\pgfsetrectcap%
\pgfsetmiterjoin%
\pgfsetlinewidth{0.803000pt}%
\definecolor{currentstroke}{rgb}{0.000000,0.000000,0.000000}%
\pgfsetstrokecolor{currentstroke}%
\pgfsetdash{}{0pt}%
\pgfpathmoveto{\pgfqpoint{4.558355in}{3.311111in}}%
\pgfpathlineto{\pgfqpoint{4.558355in}{5.627778in}}%
\pgfusepath{stroke}%
\end{pgfscope}%
\begin{pgfscope}%
\pgfsetrectcap%
\pgfsetmiterjoin%
\pgfsetlinewidth{0.803000pt}%
\definecolor{currentstroke}{rgb}{0.000000,0.000000,0.000000}%
\pgfsetstrokecolor{currentstroke}%
\pgfsetdash{}{0pt}%
\pgfpathmoveto{\pgfqpoint{7.801389in}{3.311111in}}%
\pgfpathlineto{\pgfqpoint{7.801389in}{5.627778in}}%
\pgfusepath{stroke}%
\end{pgfscope}%
\begin{pgfscope}%
\pgfsetrectcap%
\pgfsetmiterjoin%
\pgfsetlinewidth{0.803000pt}%
\definecolor{currentstroke}{rgb}{0.000000,0.000000,0.000000}%
\pgfsetstrokecolor{currentstroke}%
\pgfsetdash{}{0pt}%
\pgfpathmoveto{\pgfqpoint{4.558355in}{3.311111in}}%
\pgfpathlineto{\pgfqpoint{7.801389in}{3.311111in}}%
\pgfusepath{stroke}%
\end{pgfscope}%
\begin{pgfscope}%
\pgfsetrectcap%
\pgfsetmiterjoin%
\pgfsetlinewidth{0.803000pt}%
\definecolor{currentstroke}{rgb}{0.000000,0.000000,0.000000}%
\pgfsetstrokecolor{currentstroke}%
\pgfsetdash{}{0pt}%
\pgfpathmoveto{\pgfqpoint{4.558355in}{5.627778in}}%
\pgfpathlineto{\pgfqpoint{7.801389in}{5.627778in}}%
\pgfusepath{stroke}%
\end{pgfscope}%
\begin{pgfscope}%
\definecolor{textcolor}{rgb}{0.000000,0.000000,0.000000}%
\pgfsetstrokecolor{textcolor}%
\pgfsetfillcolor{textcolor}%
\pgftext[x=6.179872in,y=5.711111in,,base]{\color{textcolor}\sffamily\fontsize{12.000000}{14.400000}\selectfont \(\displaystyle  N = 12 \)}%
\end{pgfscope}%
\begin{pgfscope}%
\pgfsetbuttcap%
\pgfsetmiterjoin%
\definecolor{currentfill}{rgb}{1.000000,1.000000,1.000000}%
\pgfsetfillcolor{currentfill}%
\pgfsetlinewidth{0.000000pt}%
\definecolor{currentstroke}{rgb}{0.000000,0.000000,0.000000}%
\pgfsetstrokecolor{currentstroke}%
\pgfsetstrokeopacity{0.000000}%
\pgfsetdash{}{0pt}%
\pgfpathmoveto{\pgfqpoint{0.585764in}{0.386111in}}%
\pgfpathlineto{\pgfqpoint{3.828798in}{0.386111in}}%
\pgfpathlineto{\pgfqpoint{3.828798in}{2.702778in}}%
\pgfpathlineto{\pgfqpoint{0.585764in}{2.702778in}}%
\pgfpathclose%
\pgfusepath{fill}%
\end{pgfscope}%
\begin{pgfscope}%
\pgfpathrectangle{\pgfqpoint{0.585764in}{0.386111in}}{\pgfqpoint{3.243034in}{2.316667in}}%
\pgfusepath{clip}%
\pgfsetbuttcap%
\pgfsetroundjoin%
\definecolor{currentfill}{rgb}{0.121569,0.466667,0.705882}%
\pgfsetfillcolor{currentfill}%
\pgfsetfillopacity{0.300000}%
\pgfsetlinewidth{1.003750pt}%
\definecolor{currentstroke}{rgb}{0.121569,0.466667,0.705882}%
\pgfsetstrokecolor{currentstroke}%
\pgfsetstrokeopacity{0.300000}%
\pgfsetdash{}{0pt}%
\pgfpathmoveto{\pgfqpoint{0.744390in}{0.524537in}}%
\pgfpathlineto{\pgfqpoint{0.744390in}{0.491414in}}%
\pgfpathlineto{\pgfqpoint{0.890679in}{0.588478in}}%
\pgfpathlineto{\pgfqpoint{1.036968in}{0.687070in}}%
\pgfpathlineto{\pgfqpoint{1.183257in}{0.778820in}}%
\pgfpathlineto{\pgfqpoint{1.329546in}{0.884767in}}%
\pgfpathlineto{\pgfqpoint{1.475835in}{1.000529in}}%
\pgfpathlineto{\pgfqpoint{1.622125in}{1.109231in}}%
\pgfpathlineto{\pgfqpoint{1.768414in}{1.194609in}}%
\pgfpathlineto{\pgfqpoint{1.914703in}{1.317067in}}%
\pgfpathlineto{\pgfqpoint{2.060992in}{1.421644in}}%
\pgfpathlineto{\pgfqpoint{2.207281in}{1.527381in}}%
\pgfpathlineto{\pgfqpoint{2.353570in}{1.675945in}}%
\pgfpathlineto{\pgfqpoint{2.499859in}{1.760323in}}%
\pgfpathlineto{\pgfqpoint{2.646148in}{1.857712in}}%
\pgfpathlineto{\pgfqpoint{2.792437in}{2.017180in}}%
\pgfpathlineto{\pgfqpoint{2.938726in}{2.110732in}}%
\pgfpathlineto{\pgfqpoint{3.085015in}{2.207209in}}%
\pgfpathlineto{\pgfqpoint{3.231305in}{2.314395in}}%
\pgfpathlineto{\pgfqpoint{3.377594in}{2.394545in}}%
\pgfpathlineto{\pgfqpoint{3.523883in}{2.486994in}}%
\pgfpathlineto{\pgfqpoint{3.670172in}{2.567428in}}%
\pgfpathlineto{\pgfqpoint{3.670172in}{2.597475in}}%
\pgfpathlineto{\pgfqpoint{3.670172in}{2.597475in}}%
\pgfpathlineto{\pgfqpoint{3.523883in}{2.531443in}}%
\pgfpathlineto{\pgfqpoint{3.377594in}{2.440606in}}%
\pgfpathlineto{\pgfqpoint{3.231305in}{2.352575in}}%
\pgfpathlineto{\pgfqpoint{3.085015in}{2.259971in}}%
\pgfpathlineto{\pgfqpoint{2.938726in}{2.147968in}}%
\pgfpathlineto{\pgfqpoint{2.792437in}{2.032984in}}%
\pgfpathlineto{\pgfqpoint{2.646148in}{1.921550in}}%
\pgfpathlineto{\pgfqpoint{2.499859in}{1.840686in}}%
\pgfpathlineto{\pgfqpoint{2.353570in}{1.698391in}}%
\pgfpathlineto{\pgfqpoint{2.207281in}{1.591009in}}%
\pgfpathlineto{\pgfqpoint{2.060992in}{1.444451in}}%
\pgfpathlineto{\pgfqpoint{1.914703in}{1.332102in}}%
\pgfpathlineto{\pgfqpoint{1.768414in}{1.222837in}}%
\pgfpathlineto{\pgfqpoint{1.622125in}{1.131739in}}%
\pgfpathlineto{\pgfqpoint{1.475835in}{1.011074in}}%
\pgfpathlineto{\pgfqpoint{1.329546in}{0.902683in}}%
\pgfpathlineto{\pgfqpoint{1.183257in}{0.814031in}}%
\pgfpathlineto{\pgfqpoint{1.036968in}{0.710717in}}%
\pgfpathlineto{\pgfqpoint{0.890679in}{0.609885in}}%
\pgfpathlineto{\pgfqpoint{0.744390in}{0.524537in}}%
\pgfpathclose%
\pgfusepath{stroke,fill}%
\end{pgfscope}%
\begin{pgfscope}%
\pgfpathrectangle{\pgfqpoint{0.585764in}{0.386111in}}{\pgfqpoint{3.243034in}{2.316667in}}%
\pgfusepath{clip}%
\pgfsetbuttcap%
\pgfsetroundjoin%
\definecolor{currentfill}{rgb}{0.121569,0.466667,0.705882}%
\pgfsetfillcolor{currentfill}%
\pgfsetlinewidth{1.003750pt}%
\definecolor{currentstroke}{rgb}{0.121569,0.466667,0.705882}%
\pgfsetstrokecolor{currentstroke}%
\pgfsetdash{}{0pt}%
\pgfsys@defobject{currentmarker}{\pgfqpoint{-0.009821in}{-0.009821in}}{\pgfqpoint{0.009821in}{0.009821in}}{%
\pgfpathmoveto{\pgfqpoint{0.000000in}{-0.009821in}}%
\pgfpathcurveto{\pgfqpoint{0.002605in}{-0.009821in}}{\pgfqpoint{0.005103in}{-0.008786in}}{\pgfqpoint{0.006944in}{-0.006944in}}%
\pgfpathcurveto{\pgfqpoint{0.008786in}{-0.005103in}}{\pgfqpoint{0.009821in}{-0.002605in}}{\pgfqpoint{0.009821in}{0.000000in}}%
\pgfpathcurveto{\pgfqpoint{0.009821in}{0.002605in}}{\pgfqpoint{0.008786in}{0.005103in}}{\pgfqpoint{0.006944in}{0.006944in}}%
\pgfpathcurveto{\pgfqpoint{0.005103in}{0.008786in}}{\pgfqpoint{0.002605in}{0.009821in}}{\pgfqpoint{0.000000in}{0.009821in}}%
\pgfpathcurveto{\pgfqpoint{-0.002605in}{0.009821in}}{\pgfqpoint{-0.005103in}{0.008786in}}{\pgfqpoint{-0.006944in}{0.006944in}}%
\pgfpathcurveto{\pgfqpoint{-0.008786in}{0.005103in}}{\pgfqpoint{-0.009821in}{0.002605in}}{\pgfqpoint{-0.009821in}{0.000000in}}%
\pgfpathcurveto{\pgfqpoint{-0.009821in}{-0.002605in}}{\pgfqpoint{-0.008786in}{-0.005103in}}{\pgfqpoint{-0.006944in}{-0.006944in}}%
\pgfpathcurveto{\pgfqpoint{-0.005103in}{-0.008786in}}{\pgfqpoint{-0.002605in}{-0.009821in}}{\pgfqpoint{0.000000in}{-0.009821in}}%
\pgfpathclose%
\pgfusepath{stroke,fill}%
}%
\begin{pgfscope}%
\pgfsys@transformshift{0.744390in}{0.507976in}%
\pgfsys@useobject{currentmarker}{}%
\end{pgfscope}%
\begin{pgfscope}%
\pgfsys@transformshift{0.890679in}{0.599181in}%
\pgfsys@useobject{currentmarker}{}%
\end{pgfscope}%
\begin{pgfscope}%
\pgfsys@transformshift{1.036968in}{0.698893in}%
\pgfsys@useobject{currentmarker}{}%
\end{pgfscope}%
\begin{pgfscope}%
\pgfsys@transformshift{1.183257in}{0.796426in}%
\pgfsys@useobject{currentmarker}{}%
\end{pgfscope}%
\begin{pgfscope}%
\pgfsys@transformshift{1.329546in}{0.893725in}%
\pgfsys@useobject{currentmarker}{}%
\end{pgfscope}%
\begin{pgfscope}%
\pgfsys@transformshift{1.475835in}{1.005801in}%
\pgfsys@useobject{currentmarker}{}%
\end{pgfscope}%
\begin{pgfscope}%
\pgfsys@transformshift{1.622125in}{1.120485in}%
\pgfsys@useobject{currentmarker}{}%
\end{pgfscope}%
\begin{pgfscope}%
\pgfsys@transformshift{1.768414in}{1.208723in}%
\pgfsys@useobject{currentmarker}{}%
\end{pgfscope}%
\begin{pgfscope}%
\pgfsys@transformshift{1.914703in}{1.324585in}%
\pgfsys@useobject{currentmarker}{}%
\end{pgfscope}%
\begin{pgfscope}%
\pgfsys@transformshift{2.060992in}{1.433047in}%
\pgfsys@useobject{currentmarker}{}%
\end{pgfscope}%
\begin{pgfscope}%
\pgfsys@transformshift{2.207281in}{1.559195in}%
\pgfsys@useobject{currentmarker}{}%
\end{pgfscope}%
\begin{pgfscope}%
\pgfsys@transformshift{2.353570in}{1.687168in}%
\pgfsys@useobject{currentmarker}{}%
\end{pgfscope}%
\begin{pgfscope}%
\pgfsys@transformshift{2.499859in}{1.800505in}%
\pgfsys@useobject{currentmarker}{}%
\end{pgfscope}%
\begin{pgfscope}%
\pgfsys@transformshift{2.646148in}{1.889631in}%
\pgfsys@useobject{currentmarker}{}%
\end{pgfscope}%
\begin{pgfscope}%
\pgfsys@transformshift{2.792437in}{2.025082in}%
\pgfsys@useobject{currentmarker}{}%
\end{pgfscope}%
\begin{pgfscope}%
\pgfsys@transformshift{2.938726in}{2.129350in}%
\pgfsys@useobject{currentmarker}{}%
\end{pgfscope}%
\begin{pgfscope}%
\pgfsys@transformshift{3.085015in}{2.233590in}%
\pgfsys@useobject{currentmarker}{}%
\end{pgfscope}%
\begin{pgfscope}%
\pgfsys@transformshift{3.231305in}{2.333485in}%
\pgfsys@useobject{currentmarker}{}%
\end{pgfscope}%
\begin{pgfscope}%
\pgfsys@transformshift{3.377594in}{2.417575in}%
\pgfsys@useobject{currentmarker}{}%
\end{pgfscope}%
\begin{pgfscope}%
\pgfsys@transformshift{3.523883in}{2.509218in}%
\pgfsys@useobject{currentmarker}{}%
\end{pgfscope}%
\begin{pgfscope}%
\pgfsys@transformshift{3.670172in}{2.582452in}%
\pgfsys@useobject{currentmarker}{}%
\end{pgfscope}%
\end{pgfscope}%
\begin{pgfscope}%
\pgfsetbuttcap%
\pgfsetroundjoin%
\definecolor{currentfill}{rgb}{0.000000,0.000000,0.000000}%
\pgfsetfillcolor{currentfill}%
\pgfsetlinewidth{0.803000pt}%
\definecolor{currentstroke}{rgb}{0.000000,0.000000,0.000000}%
\pgfsetstrokecolor{currentstroke}%
\pgfsetdash{}{0pt}%
\pgfsys@defobject{currentmarker}{\pgfqpoint{0.000000in}{-0.048611in}}{\pgfqpoint{0.000000in}{0.000000in}}{%
\pgfpathmoveto{\pgfqpoint{0.000000in}{0.000000in}}%
\pgfpathlineto{\pgfqpoint{0.000000in}{-0.048611in}}%
\pgfusepath{stroke,fill}%
}%
\begin{pgfscope}%
\pgfsys@transformshift{0.890679in}{0.386111in}%
\pgfsys@useobject{currentmarker}{}%
\end{pgfscope}%
\end{pgfscope}%
\begin{pgfscope}%
\definecolor{textcolor}{rgb}{0.000000,0.000000,0.000000}%
\pgfsetstrokecolor{textcolor}%
\pgfsetfillcolor{textcolor}%
\pgftext[x=0.890679in,y=0.288889in,,top]{\color{textcolor}\sffamily\fontsize{10.000000}{12.000000}\selectfont 4.35}%
\end{pgfscope}%
\begin{pgfscope}%
\pgfsetbuttcap%
\pgfsetroundjoin%
\definecolor{currentfill}{rgb}{0.000000,0.000000,0.000000}%
\pgfsetfillcolor{currentfill}%
\pgfsetlinewidth{0.803000pt}%
\definecolor{currentstroke}{rgb}{0.000000,0.000000,0.000000}%
\pgfsetstrokecolor{currentstroke}%
\pgfsetdash{}{0pt}%
\pgfsys@defobject{currentmarker}{\pgfqpoint{0.000000in}{-0.048611in}}{\pgfqpoint{0.000000in}{0.000000in}}{%
\pgfpathmoveto{\pgfqpoint{0.000000in}{0.000000in}}%
\pgfpathlineto{\pgfqpoint{0.000000in}{-0.048611in}}%
\pgfusepath{stroke,fill}%
}%
\begin{pgfscope}%
\pgfsys@transformshift{1.622125in}{0.386111in}%
\pgfsys@useobject{currentmarker}{}%
\end{pgfscope}%
\end{pgfscope}%
\begin{pgfscope}%
\definecolor{textcolor}{rgb}{0.000000,0.000000,0.000000}%
\pgfsetstrokecolor{textcolor}%
\pgfsetfillcolor{textcolor}%
\pgftext[x=1.622125in,y=0.288889in,,top]{\color{textcolor}\sffamily\fontsize{10.000000}{12.000000}\selectfont 4.40}%
\end{pgfscope}%
\begin{pgfscope}%
\pgfsetbuttcap%
\pgfsetroundjoin%
\definecolor{currentfill}{rgb}{0.000000,0.000000,0.000000}%
\pgfsetfillcolor{currentfill}%
\pgfsetlinewidth{0.803000pt}%
\definecolor{currentstroke}{rgb}{0.000000,0.000000,0.000000}%
\pgfsetstrokecolor{currentstroke}%
\pgfsetdash{}{0pt}%
\pgfsys@defobject{currentmarker}{\pgfqpoint{0.000000in}{-0.048611in}}{\pgfqpoint{0.000000in}{0.000000in}}{%
\pgfpathmoveto{\pgfqpoint{0.000000in}{0.000000in}}%
\pgfpathlineto{\pgfqpoint{0.000000in}{-0.048611in}}%
\pgfusepath{stroke,fill}%
}%
\begin{pgfscope}%
\pgfsys@transformshift{2.353570in}{0.386111in}%
\pgfsys@useobject{currentmarker}{}%
\end{pgfscope}%
\end{pgfscope}%
\begin{pgfscope}%
\definecolor{textcolor}{rgb}{0.000000,0.000000,0.000000}%
\pgfsetstrokecolor{textcolor}%
\pgfsetfillcolor{textcolor}%
\pgftext[x=2.353570in,y=0.288889in,,top]{\color{textcolor}\sffamily\fontsize{10.000000}{12.000000}\selectfont 4.45}%
\end{pgfscope}%
\begin{pgfscope}%
\pgfsetbuttcap%
\pgfsetroundjoin%
\definecolor{currentfill}{rgb}{0.000000,0.000000,0.000000}%
\pgfsetfillcolor{currentfill}%
\pgfsetlinewidth{0.803000pt}%
\definecolor{currentstroke}{rgb}{0.000000,0.000000,0.000000}%
\pgfsetstrokecolor{currentstroke}%
\pgfsetdash{}{0pt}%
\pgfsys@defobject{currentmarker}{\pgfqpoint{0.000000in}{-0.048611in}}{\pgfqpoint{0.000000in}{0.000000in}}{%
\pgfpathmoveto{\pgfqpoint{0.000000in}{0.000000in}}%
\pgfpathlineto{\pgfqpoint{0.000000in}{-0.048611in}}%
\pgfusepath{stroke,fill}%
}%
\begin{pgfscope}%
\pgfsys@transformshift{3.085015in}{0.386111in}%
\pgfsys@useobject{currentmarker}{}%
\end{pgfscope}%
\end{pgfscope}%
\begin{pgfscope}%
\definecolor{textcolor}{rgb}{0.000000,0.000000,0.000000}%
\pgfsetstrokecolor{textcolor}%
\pgfsetfillcolor{textcolor}%
\pgftext[x=3.085015in,y=0.288889in,,top]{\color{textcolor}\sffamily\fontsize{10.000000}{12.000000}\selectfont 4.50}%
\end{pgfscope}%
\begin{pgfscope}%
\pgfsetbuttcap%
\pgfsetroundjoin%
\definecolor{currentfill}{rgb}{0.000000,0.000000,0.000000}%
\pgfsetfillcolor{currentfill}%
\pgfsetlinewidth{0.803000pt}%
\definecolor{currentstroke}{rgb}{0.000000,0.000000,0.000000}%
\pgfsetstrokecolor{currentstroke}%
\pgfsetdash{}{0pt}%
\pgfsys@defobject{currentmarker}{\pgfqpoint{0.000000in}{-0.048611in}}{\pgfqpoint{0.000000in}{0.000000in}}{%
\pgfpathmoveto{\pgfqpoint{0.000000in}{0.000000in}}%
\pgfpathlineto{\pgfqpoint{0.000000in}{-0.048611in}}%
\pgfusepath{stroke,fill}%
}%
\begin{pgfscope}%
\pgfsys@transformshift{3.816461in}{0.386111in}%
\pgfsys@useobject{currentmarker}{}%
\end{pgfscope}%
\end{pgfscope}%
\begin{pgfscope}%
\definecolor{textcolor}{rgb}{0.000000,0.000000,0.000000}%
\pgfsetstrokecolor{textcolor}%
\pgfsetfillcolor{textcolor}%
\pgftext[x=3.816461in,y=0.288889in,,top]{\color{textcolor}\sffamily\fontsize{10.000000}{12.000000}\selectfont 4.55}%
\end{pgfscope}%
\begin{pgfscope}%
\pgfsetbuttcap%
\pgfsetroundjoin%
\definecolor{currentfill}{rgb}{0.000000,0.000000,0.000000}%
\pgfsetfillcolor{currentfill}%
\pgfsetlinewidth{0.803000pt}%
\definecolor{currentstroke}{rgb}{0.000000,0.000000,0.000000}%
\pgfsetstrokecolor{currentstroke}%
\pgfsetdash{}{0pt}%
\pgfsys@defobject{currentmarker}{\pgfqpoint{-0.048611in}{0.000000in}}{\pgfqpoint{0.000000in}{0.000000in}}{%
\pgfpathmoveto{\pgfqpoint{0.000000in}{0.000000in}}%
\pgfpathlineto{\pgfqpoint{-0.048611in}{0.000000in}}%
\pgfusepath{stroke,fill}%
}%
\begin{pgfscope}%
\pgfsys@transformshift{0.585764in}{0.506212in}%
\pgfsys@useobject{currentmarker}{}%
\end{pgfscope}%
\end{pgfscope}%
\begin{pgfscope}%
\definecolor{textcolor}{rgb}{0.000000,0.000000,0.000000}%
\pgfsetstrokecolor{textcolor}%
\pgfsetfillcolor{textcolor}%
\pgftext[x=0.151288in,y=0.453450in,left,base]{\color{textcolor}\sffamily\fontsize{10.000000}{12.000000}\selectfont −1.4}%
\end{pgfscope}%
\begin{pgfscope}%
\pgfsetbuttcap%
\pgfsetroundjoin%
\definecolor{currentfill}{rgb}{0.000000,0.000000,0.000000}%
\pgfsetfillcolor{currentfill}%
\pgfsetlinewidth{0.803000pt}%
\definecolor{currentstroke}{rgb}{0.000000,0.000000,0.000000}%
\pgfsetstrokecolor{currentstroke}%
\pgfsetdash{}{0pt}%
\pgfsys@defobject{currentmarker}{\pgfqpoint{-0.048611in}{0.000000in}}{\pgfqpoint{0.000000in}{0.000000in}}{%
\pgfpathmoveto{\pgfqpoint{0.000000in}{0.000000in}}%
\pgfpathlineto{\pgfqpoint{-0.048611in}{0.000000in}}%
\pgfusepath{stroke,fill}%
}%
\begin{pgfscope}%
\pgfsys@transformshift{0.585764in}{1.010741in}%
\pgfsys@useobject{currentmarker}{}%
\end{pgfscope}%
\end{pgfscope}%
\begin{pgfscope}%
\definecolor{textcolor}{rgb}{0.000000,0.000000,0.000000}%
\pgfsetstrokecolor{textcolor}%
\pgfsetfillcolor{textcolor}%
\pgftext[x=0.151288in,y=0.957979in,left,base]{\color{textcolor}\sffamily\fontsize{10.000000}{12.000000}\selectfont −1.3}%
\end{pgfscope}%
\begin{pgfscope}%
\pgfsetbuttcap%
\pgfsetroundjoin%
\definecolor{currentfill}{rgb}{0.000000,0.000000,0.000000}%
\pgfsetfillcolor{currentfill}%
\pgfsetlinewidth{0.803000pt}%
\definecolor{currentstroke}{rgb}{0.000000,0.000000,0.000000}%
\pgfsetstrokecolor{currentstroke}%
\pgfsetdash{}{0pt}%
\pgfsys@defobject{currentmarker}{\pgfqpoint{-0.048611in}{0.000000in}}{\pgfqpoint{0.000000in}{0.000000in}}{%
\pgfpathmoveto{\pgfqpoint{0.000000in}{0.000000in}}%
\pgfpathlineto{\pgfqpoint{-0.048611in}{0.000000in}}%
\pgfusepath{stroke,fill}%
}%
\begin{pgfscope}%
\pgfsys@transformshift{0.585764in}{1.515269in}%
\pgfsys@useobject{currentmarker}{}%
\end{pgfscope}%
\end{pgfscope}%
\begin{pgfscope}%
\definecolor{textcolor}{rgb}{0.000000,0.000000,0.000000}%
\pgfsetstrokecolor{textcolor}%
\pgfsetfillcolor{textcolor}%
\pgftext[x=0.151288in,y=1.462508in,left,base]{\color{textcolor}\sffamily\fontsize{10.000000}{12.000000}\selectfont −1.2}%
\end{pgfscope}%
\begin{pgfscope}%
\pgfsetbuttcap%
\pgfsetroundjoin%
\definecolor{currentfill}{rgb}{0.000000,0.000000,0.000000}%
\pgfsetfillcolor{currentfill}%
\pgfsetlinewidth{0.803000pt}%
\definecolor{currentstroke}{rgb}{0.000000,0.000000,0.000000}%
\pgfsetstrokecolor{currentstroke}%
\pgfsetdash{}{0pt}%
\pgfsys@defobject{currentmarker}{\pgfqpoint{-0.048611in}{0.000000in}}{\pgfqpoint{0.000000in}{0.000000in}}{%
\pgfpathmoveto{\pgfqpoint{0.000000in}{0.000000in}}%
\pgfpathlineto{\pgfqpoint{-0.048611in}{0.000000in}}%
\pgfusepath{stroke,fill}%
}%
\begin{pgfscope}%
\pgfsys@transformshift{0.585764in}{2.019798in}%
\pgfsys@useobject{currentmarker}{}%
\end{pgfscope}%
\end{pgfscope}%
\begin{pgfscope}%
\definecolor{textcolor}{rgb}{0.000000,0.000000,0.000000}%
\pgfsetstrokecolor{textcolor}%
\pgfsetfillcolor{textcolor}%
\pgftext[x=0.151288in,y=1.967037in,left,base]{\color{textcolor}\sffamily\fontsize{10.000000}{12.000000}\selectfont −1.1}%
\end{pgfscope}%
\begin{pgfscope}%
\pgfsetbuttcap%
\pgfsetroundjoin%
\definecolor{currentfill}{rgb}{0.000000,0.000000,0.000000}%
\pgfsetfillcolor{currentfill}%
\pgfsetlinewidth{0.803000pt}%
\definecolor{currentstroke}{rgb}{0.000000,0.000000,0.000000}%
\pgfsetstrokecolor{currentstroke}%
\pgfsetdash{}{0pt}%
\pgfsys@defobject{currentmarker}{\pgfqpoint{-0.048611in}{0.000000in}}{\pgfqpoint{0.000000in}{0.000000in}}{%
\pgfpathmoveto{\pgfqpoint{0.000000in}{0.000000in}}%
\pgfpathlineto{\pgfqpoint{-0.048611in}{0.000000in}}%
\pgfusepath{stroke,fill}%
}%
\begin{pgfscope}%
\pgfsys@transformshift{0.585764in}{2.524327in}%
\pgfsys@useobject{currentmarker}{}%
\end{pgfscope}%
\end{pgfscope}%
\begin{pgfscope}%
\definecolor{textcolor}{rgb}{0.000000,0.000000,0.000000}%
\pgfsetstrokecolor{textcolor}%
\pgfsetfillcolor{textcolor}%
\pgftext[x=0.151288in,y=2.471566in,left,base]{\color{textcolor}\sffamily\fontsize{10.000000}{12.000000}\selectfont −1.0}%
\end{pgfscope}%
\begin{pgfscope}%
\pgfpathrectangle{\pgfqpoint{0.585764in}{0.386111in}}{\pgfqpoint{3.243034in}{2.316667in}}%
\pgfusepath{clip}%
\pgfsetrectcap%
\pgfsetroundjoin%
\pgfsetlinewidth{1.505625pt}%
\definecolor{currentstroke}{rgb}{0.121569,0.466667,0.705882}%
\pgfsetstrokecolor{currentstroke}%
\pgfsetdash{}{0pt}%
\pgfpathmoveto{\pgfqpoint{0.744390in}{0.507976in}}%
\pgfpathlineto{\pgfqpoint{0.890679in}{0.599181in}}%
\pgfpathlineto{\pgfqpoint{1.036968in}{0.698893in}}%
\pgfpathlineto{\pgfqpoint{1.183257in}{0.796426in}}%
\pgfpathlineto{\pgfqpoint{1.329546in}{0.893725in}}%
\pgfpathlineto{\pgfqpoint{1.475835in}{1.005801in}}%
\pgfpathlineto{\pgfqpoint{1.622125in}{1.120485in}}%
\pgfpathlineto{\pgfqpoint{1.768414in}{1.208723in}}%
\pgfpathlineto{\pgfqpoint{1.914703in}{1.324585in}}%
\pgfpathlineto{\pgfqpoint{2.060992in}{1.433047in}}%
\pgfpathlineto{\pgfqpoint{2.207281in}{1.559195in}}%
\pgfpathlineto{\pgfqpoint{2.353570in}{1.687168in}}%
\pgfpathlineto{\pgfqpoint{2.499859in}{1.800505in}}%
\pgfpathlineto{\pgfqpoint{2.646148in}{1.889631in}}%
\pgfpathlineto{\pgfqpoint{2.792437in}{2.025082in}}%
\pgfpathlineto{\pgfqpoint{2.938726in}{2.129350in}}%
\pgfpathlineto{\pgfqpoint{3.085015in}{2.233590in}}%
\pgfpathlineto{\pgfqpoint{3.231305in}{2.333485in}}%
\pgfpathlineto{\pgfqpoint{3.377594in}{2.417575in}}%
\pgfpathlineto{\pgfqpoint{3.523883in}{2.509218in}}%
\pgfpathlineto{\pgfqpoint{3.670172in}{2.582452in}}%
\pgfusepath{stroke}%
\end{pgfscope}%
\begin{pgfscope}%
\pgfsetrectcap%
\pgfsetmiterjoin%
\pgfsetlinewidth{0.803000pt}%
\definecolor{currentstroke}{rgb}{0.000000,0.000000,0.000000}%
\pgfsetstrokecolor{currentstroke}%
\pgfsetdash{}{0pt}%
\pgfpathmoveto{\pgfqpoint{0.585764in}{0.386111in}}%
\pgfpathlineto{\pgfqpoint{0.585764in}{2.702778in}}%
\pgfusepath{stroke}%
\end{pgfscope}%
\begin{pgfscope}%
\pgfsetrectcap%
\pgfsetmiterjoin%
\pgfsetlinewidth{0.803000pt}%
\definecolor{currentstroke}{rgb}{0.000000,0.000000,0.000000}%
\pgfsetstrokecolor{currentstroke}%
\pgfsetdash{}{0pt}%
\pgfpathmoveto{\pgfqpoint{3.828798in}{0.386111in}}%
\pgfpathlineto{\pgfqpoint{3.828798in}{2.702778in}}%
\pgfusepath{stroke}%
\end{pgfscope}%
\begin{pgfscope}%
\pgfsetrectcap%
\pgfsetmiterjoin%
\pgfsetlinewidth{0.803000pt}%
\definecolor{currentstroke}{rgb}{0.000000,0.000000,0.000000}%
\pgfsetstrokecolor{currentstroke}%
\pgfsetdash{}{0pt}%
\pgfpathmoveto{\pgfqpoint{0.585764in}{0.386111in}}%
\pgfpathlineto{\pgfqpoint{3.828798in}{0.386111in}}%
\pgfusepath{stroke}%
\end{pgfscope}%
\begin{pgfscope}%
\pgfsetrectcap%
\pgfsetmiterjoin%
\pgfsetlinewidth{0.803000pt}%
\definecolor{currentstroke}{rgb}{0.000000,0.000000,0.000000}%
\pgfsetstrokecolor{currentstroke}%
\pgfsetdash{}{0pt}%
\pgfpathmoveto{\pgfqpoint{0.585764in}{2.702778in}}%
\pgfpathlineto{\pgfqpoint{3.828798in}{2.702778in}}%
\pgfusepath{stroke}%
\end{pgfscope}%
\begin{pgfscope}%
\definecolor{textcolor}{rgb}{0.000000,0.000000,0.000000}%
\pgfsetstrokecolor{textcolor}%
\pgfsetfillcolor{textcolor}%
\pgftext[x=2.207281in,y=2.786111in,,base]{\color{textcolor}\sffamily\fontsize{12.000000}{14.400000}\selectfont \(\displaystyle  N = 16 \)}%
\end{pgfscope}%
\begin{pgfscope}%
\pgfsetbuttcap%
\pgfsetmiterjoin%
\definecolor{currentfill}{rgb}{1.000000,1.000000,1.000000}%
\pgfsetfillcolor{currentfill}%
\pgfsetlinewidth{0.000000pt}%
\definecolor{currentstroke}{rgb}{0.000000,0.000000,0.000000}%
\pgfsetstrokecolor{currentstroke}%
\pgfsetstrokeopacity{0.000000}%
\pgfsetdash{}{0pt}%
\pgfpathmoveto{\pgfqpoint{4.558355in}{0.386111in}}%
\pgfpathlineto{\pgfqpoint{7.801389in}{0.386111in}}%
\pgfpathlineto{\pgfqpoint{7.801389in}{2.702778in}}%
\pgfpathlineto{\pgfqpoint{4.558355in}{2.702778in}}%
\pgfpathclose%
\pgfusepath{fill}%
\end{pgfscope}%
\begin{pgfscope}%
\pgfpathrectangle{\pgfqpoint{4.558355in}{0.386111in}}{\pgfqpoint{3.243034in}{2.316667in}}%
\pgfusepath{clip}%
\pgfsetbuttcap%
\pgfsetroundjoin%
\definecolor{currentfill}{rgb}{0.121569,0.466667,0.705882}%
\pgfsetfillcolor{currentfill}%
\pgfsetfillopacity{0.300000}%
\pgfsetlinewidth{1.003750pt}%
\definecolor{currentstroke}{rgb}{0.121569,0.466667,0.705882}%
\pgfsetstrokecolor{currentstroke}%
\pgfsetstrokeopacity{0.300000}%
\pgfsetdash{}{0pt}%
\pgfpathmoveto{\pgfqpoint{4.716981in}{0.513168in}}%
\pgfpathlineto{\pgfqpoint{4.716981in}{0.491755in}}%
\pgfpathlineto{\pgfqpoint{4.863270in}{0.589940in}}%
\pgfpathlineto{\pgfqpoint{5.009559in}{0.689590in}}%
\pgfpathlineto{\pgfqpoint{5.155848in}{0.805722in}}%
\pgfpathlineto{\pgfqpoint{5.302137in}{0.910635in}}%
\pgfpathlineto{\pgfqpoint{5.448426in}{1.054438in}}%
\pgfpathlineto{\pgfqpoint{5.594716in}{1.137425in}}%
\pgfpathlineto{\pgfqpoint{5.741005in}{1.274516in}}%
\pgfpathlineto{\pgfqpoint{5.887294in}{1.428374in}}%
\pgfpathlineto{\pgfqpoint{6.033583in}{1.588194in}}%
\pgfpathlineto{\pgfqpoint{6.179872in}{1.669739in}}%
\pgfpathlineto{\pgfqpoint{6.326161in}{1.839181in}}%
\pgfpathlineto{\pgfqpoint{6.472450in}{2.005984in}}%
\pgfpathlineto{\pgfqpoint{6.618739in}{2.045232in}}%
\pgfpathlineto{\pgfqpoint{6.765028in}{2.162126in}}%
\pgfpathlineto{\pgfqpoint{6.911317in}{2.261859in}}%
\pgfpathlineto{\pgfqpoint{7.057606in}{2.355427in}}%
\pgfpathlineto{\pgfqpoint{7.203895in}{2.446467in}}%
\pgfpathlineto{\pgfqpoint{7.350185in}{2.472159in}}%
\pgfpathlineto{\pgfqpoint{7.496474in}{2.522668in}}%
\pgfpathlineto{\pgfqpoint{7.642763in}{2.561489in}}%
\pgfpathlineto{\pgfqpoint{7.642763in}{2.597475in}}%
\pgfpathlineto{\pgfqpoint{7.642763in}{2.597475in}}%
\pgfpathlineto{\pgfqpoint{7.496474in}{2.551772in}}%
\pgfpathlineto{\pgfqpoint{7.350185in}{2.515820in}}%
\pgfpathlineto{\pgfqpoint{7.203895in}{2.479383in}}%
\pgfpathlineto{\pgfqpoint{7.057606in}{2.406753in}}%
\pgfpathlineto{\pgfqpoint{6.911317in}{2.337386in}}%
\pgfpathlineto{\pgfqpoint{6.765028in}{2.298038in}}%
\pgfpathlineto{\pgfqpoint{6.618739in}{2.154267in}}%
\pgfpathlineto{\pgfqpoint{6.472450in}{2.039373in}}%
\pgfpathlineto{\pgfqpoint{6.326161in}{1.892378in}}%
\pgfpathlineto{\pgfqpoint{6.179872in}{1.787923in}}%
\pgfpathlineto{\pgfqpoint{6.033583in}{1.664442in}}%
\pgfpathlineto{\pgfqpoint{5.887294in}{1.482276in}}%
\pgfpathlineto{\pgfqpoint{5.741005in}{1.321878in}}%
\pgfpathlineto{\pgfqpoint{5.594716in}{1.194026in}}%
\pgfpathlineto{\pgfqpoint{5.448426in}{1.092185in}}%
\pgfpathlineto{\pgfqpoint{5.302137in}{0.967365in}}%
\pgfpathlineto{\pgfqpoint{5.155848in}{0.832637in}}%
\pgfpathlineto{\pgfqpoint{5.009559in}{0.726764in}}%
\pgfpathlineto{\pgfqpoint{4.863270in}{0.628603in}}%
\pgfpathlineto{\pgfqpoint{4.716981in}{0.513168in}}%
\pgfpathclose%
\pgfusepath{stroke,fill}%
\end{pgfscope}%
\begin{pgfscope}%
\pgfpathrectangle{\pgfqpoint{4.558355in}{0.386111in}}{\pgfqpoint{3.243034in}{2.316667in}}%
\pgfusepath{clip}%
\pgfsetbuttcap%
\pgfsetroundjoin%
\definecolor{currentfill}{rgb}{0.121569,0.466667,0.705882}%
\pgfsetfillcolor{currentfill}%
\pgfsetlinewidth{1.003750pt}%
\definecolor{currentstroke}{rgb}{0.121569,0.466667,0.705882}%
\pgfsetstrokecolor{currentstroke}%
\pgfsetdash{}{0pt}%
\pgfsys@defobject{currentmarker}{\pgfqpoint{-0.009821in}{-0.009821in}}{\pgfqpoint{0.009821in}{0.009821in}}{%
\pgfpathmoveto{\pgfqpoint{0.000000in}{-0.009821in}}%
\pgfpathcurveto{\pgfqpoint{0.002605in}{-0.009821in}}{\pgfqpoint{0.005103in}{-0.008786in}}{\pgfqpoint{0.006944in}{-0.006944in}}%
\pgfpathcurveto{\pgfqpoint{0.008786in}{-0.005103in}}{\pgfqpoint{0.009821in}{-0.002605in}}{\pgfqpoint{0.009821in}{0.000000in}}%
\pgfpathcurveto{\pgfqpoint{0.009821in}{0.002605in}}{\pgfqpoint{0.008786in}{0.005103in}}{\pgfqpoint{0.006944in}{0.006944in}}%
\pgfpathcurveto{\pgfqpoint{0.005103in}{0.008786in}}{\pgfqpoint{0.002605in}{0.009821in}}{\pgfqpoint{0.000000in}{0.009821in}}%
\pgfpathcurveto{\pgfqpoint{-0.002605in}{0.009821in}}{\pgfqpoint{-0.005103in}{0.008786in}}{\pgfqpoint{-0.006944in}{0.006944in}}%
\pgfpathcurveto{\pgfqpoint{-0.008786in}{0.005103in}}{\pgfqpoint{-0.009821in}{0.002605in}}{\pgfqpoint{-0.009821in}{0.000000in}}%
\pgfpathcurveto{\pgfqpoint{-0.009821in}{-0.002605in}}{\pgfqpoint{-0.008786in}{-0.005103in}}{\pgfqpoint{-0.006944in}{-0.006944in}}%
\pgfpathcurveto{\pgfqpoint{-0.005103in}{-0.008786in}}{\pgfqpoint{-0.002605in}{-0.009821in}}{\pgfqpoint{0.000000in}{-0.009821in}}%
\pgfpathclose%
\pgfusepath{stroke,fill}%
}%
\begin{pgfscope}%
\pgfsys@transformshift{4.716981in}{0.502462in}%
\pgfsys@useobject{currentmarker}{}%
\end{pgfscope}%
\begin{pgfscope}%
\pgfsys@transformshift{4.863270in}{0.609271in}%
\pgfsys@useobject{currentmarker}{}%
\end{pgfscope}%
\begin{pgfscope}%
\pgfsys@transformshift{5.009559in}{0.708177in}%
\pgfsys@useobject{currentmarker}{}%
\end{pgfscope}%
\begin{pgfscope}%
\pgfsys@transformshift{5.155848in}{0.819179in}%
\pgfsys@useobject{currentmarker}{}%
\end{pgfscope}%
\begin{pgfscope}%
\pgfsys@transformshift{5.302137in}{0.939000in}%
\pgfsys@useobject{currentmarker}{}%
\end{pgfscope}%
\begin{pgfscope}%
\pgfsys@transformshift{5.448426in}{1.073311in}%
\pgfsys@useobject{currentmarker}{}%
\end{pgfscope}%
\begin{pgfscope}%
\pgfsys@transformshift{5.594716in}{1.165726in}%
\pgfsys@useobject{currentmarker}{}%
\end{pgfscope}%
\begin{pgfscope}%
\pgfsys@transformshift{5.741005in}{1.298197in}%
\pgfsys@useobject{currentmarker}{}%
\end{pgfscope}%
\begin{pgfscope}%
\pgfsys@transformshift{5.887294in}{1.455325in}%
\pgfsys@useobject{currentmarker}{}%
\end{pgfscope}%
\begin{pgfscope}%
\pgfsys@transformshift{6.033583in}{1.626318in}%
\pgfsys@useobject{currentmarker}{}%
\end{pgfscope}%
\begin{pgfscope}%
\pgfsys@transformshift{6.179872in}{1.728831in}%
\pgfsys@useobject{currentmarker}{}%
\end{pgfscope}%
\begin{pgfscope}%
\pgfsys@transformshift{6.326161in}{1.865779in}%
\pgfsys@useobject{currentmarker}{}%
\end{pgfscope}%
\begin{pgfscope}%
\pgfsys@transformshift{6.472450in}{2.022678in}%
\pgfsys@useobject{currentmarker}{}%
\end{pgfscope}%
\begin{pgfscope}%
\pgfsys@transformshift{6.618739in}{2.099750in}%
\pgfsys@useobject{currentmarker}{}%
\end{pgfscope}%
\begin{pgfscope}%
\pgfsys@transformshift{6.765028in}{2.230082in}%
\pgfsys@useobject{currentmarker}{}%
\end{pgfscope}%
\begin{pgfscope}%
\pgfsys@transformshift{6.911317in}{2.299622in}%
\pgfsys@useobject{currentmarker}{}%
\end{pgfscope}%
\begin{pgfscope}%
\pgfsys@transformshift{7.057606in}{2.381090in}%
\pgfsys@useobject{currentmarker}{}%
\end{pgfscope}%
\begin{pgfscope}%
\pgfsys@transformshift{7.203895in}{2.462925in}%
\pgfsys@useobject{currentmarker}{}%
\end{pgfscope}%
\begin{pgfscope}%
\pgfsys@transformshift{7.350185in}{2.493989in}%
\pgfsys@useobject{currentmarker}{}%
\end{pgfscope}%
\begin{pgfscope}%
\pgfsys@transformshift{7.496474in}{2.537220in}%
\pgfsys@useobject{currentmarker}{}%
\end{pgfscope}%
\begin{pgfscope}%
\pgfsys@transformshift{7.642763in}{2.579482in}%
\pgfsys@useobject{currentmarker}{}%
\end{pgfscope}%
\end{pgfscope}%
\begin{pgfscope}%
\pgfsetbuttcap%
\pgfsetroundjoin%
\definecolor{currentfill}{rgb}{0.000000,0.000000,0.000000}%
\pgfsetfillcolor{currentfill}%
\pgfsetlinewidth{0.803000pt}%
\definecolor{currentstroke}{rgb}{0.000000,0.000000,0.000000}%
\pgfsetstrokecolor{currentstroke}%
\pgfsetdash{}{0pt}%
\pgfsys@defobject{currentmarker}{\pgfqpoint{0.000000in}{-0.048611in}}{\pgfqpoint{0.000000in}{0.000000in}}{%
\pgfpathmoveto{\pgfqpoint{0.000000in}{0.000000in}}%
\pgfpathlineto{\pgfqpoint{0.000000in}{-0.048611in}}%
\pgfusepath{stroke,fill}%
}%
\begin{pgfscope}%
\pgfsys@transformshift{5.009559in}{0.386111in}%
\pgfsys@useobject{currentmarker}{}%
\end{pgfscope}%
\end{pgfscope}%
\begin{pgfscope}%
\definecolor{textcolor}{rgb}{0.000000,0.000000,0.000000}%
\pgfsetstrokecolor{textcolor}%
\pgfsetfillcolor{textcolor}%
\pgftext[x=5.009559in,y=0.288889in,,top]{\color{textcolor}\sffamily\fontsize{10.000000}{12.000000}\selectfont 4.40}%
\end{pgfscope}%
\begin{pgfscope}%
\pgfsetbuttcap%
\pgfsetroundjoin%
\definecolor{currentfill}{rgb}{0.000000,0.000000,0.000000}%
\pgfsetfillcolor{currentfill}%
\pgfsetlinewidth{0.803000pt}%
\definecolor{currentstroke}{rgb}{0.000000,0.000000,0.000000}%
\pgfsetstrokecolor{currentstroke}%
\pgfsetdash{}{0pt}%
\pgfsys@defobject{currentmarker}{\pgfqpoint{0.000000in}{-0.048611in}}{\pgfqpoint{0.000000in}{0.000000in}}{%
\pgfpathmoveto{\pgfqpoint{0.000000in}{0.000000in}}%
\pgfpathlineto{\pgfqpoint{0.000000in}{-0.048611in}}%
\pgfusepath{stroke,fill}%
}%
\begin{pgfscope}%
\pgfsys@transformshift{5.741005in}{0.386111in}%
\pgfsys@useobject{currentmarker}{}%
\end{pgfscope}%
\end{pgfscope}%
\begin{pgfscope}%
\definecolor{textcolor}{rgb}{0.000000,0.000000,0.000000}%
\pgfsetstrokecolor{textcolor}%
\pgfsetfillcolor{textcolor}%
\pgftext[x=5.741005in,y=0.288889in,,top]{\color{textcolor}\sffamily\fontsize{10.000000}{12.000000}\selectfont 4.45}%
\end{pgfscope}%
\begin{pgfscope}%
\pgfsetbuttcap%
\pgfsetroundjoin%
\definecolor{currentfill}{rgb}{0.000000,0.000000,0.000000}%
\pgfsetfillcolor{currentfill}%
\pgfsetlinewidth{0.803000pt}%
\definecolor{currentstroke}{rgb}{0.000000,0.000000,0.000000}%
\pgfsetstrokecolor{currentstroke}%
\pgfsetdash{}{0pt}%
\pgfsys@defobject{currentmarker}{\pgfqpoint{0.000000in}{-0.048611in}}{\pgfqpoint{0.000000in}{0.000000in}}{%
\pgfpathmoveto{\pgfqpoint{0.000000in}{0.000000in}}%
\pgfpathlineto{\pgfqpoint{0.000000in}{-0.048611in}}%
\pgfusepath{stroke,fill}%
}%
\begin{pgfscope}%
\pgfsys@transformshift{6.472450in}{0.386111in}%
\pgfsys@useobject{currentmarker}{}%
\end{pgfscope}%
\end{pgfscope}%
\begin{pgfscope}%
\definecolor{textcolor}{rgb}{0.000000,0.000000,0.000000}%
\pgfsetstrokecolor{textcolor}%
\pgfsetfillcolor{textcolor}%
\pgftext[x=6.472450in,y=0.288889in,,top]{\color{textcolor}\sffamily\fontsize{10.000000}{12.000000}\selectfont 4.50}%
\end{pgfscope}%
\begin{pgfscope}%
\pgfsetbuttcap%
\pgfsetroundjoin%
\definecolor{currentfill}{rgb}{0.000000,0.000000,0.000000}%
\pgfsetfillcolor{currentfill}%
\pgfsetlinewidth{0.803000pt}%
\definecolor{currentstroke}{rgb}{0.000000,0.000000,0.000000}%
\pgfsetstrokecolor{currentstroke}%
\pgfsetdash{}{0pt}%
\pgfsys@defobject{currentmarker}{\pgfqpoint{0.000000in}{-0.048611in}}{\pgfqpoint{0.000000in}{0.000000in}}{%
\pgfpathmoveto{\pgfqpoint{0.000000in}{0.000000in}}%
\pgfpathlineto{\pgfqpoint{0.000000in}{-0.048611in}}%
\pgfusepath{stroke,fill}%
}%
\begin{pgfscope}%
\pgfsys@transformshift{7.203895in}{0.386111in}%
\pgfsys@useobject{currentmarker}{}%
\end{pgfscope}%
\end{pgfscope}%
\begin{pgfscope}%
\definecolor{textcolor}{rgb}{0.000000,0.000000,0.000000}%
\pgfsetstrokecolor{textcolor}%
\pgfsetfillcolor{textcolor}%
\pgftext[x=7.203895in,y=0.288889in,,top]{\color{textcolor}\sffamily\fontsize{10.000000}{12.000000}\selectfont 4.55}%
\end{pgfscope}%
\begin{pgfscope}%
\pgfsetbuttcap%
\pgfsetroundjoin%
\definecolor{currentfill}{rgb}{0.000000,0.000000,0.000000}%
\pgfsetfillcolor{currentfill}%
\pgfsetlinewidth{0.803000pt}%
\definecolor{currentstroke}{rgb}{0.000000,0.000000,0.000000}%
\pgfsetstrokecolor{currentstroke}%
\pgfsetdash{}{0pt}%
\pgfsys@defobject{currentmarker}{\pgfqpoint{-0.048611in}{0.000000in}}{\pgfqpoint{0.000000in}{0.000000in}}{%
\pgfpathmoveto{\pgfqpoint{0.000000in}{0.000000in}}%
\pgfpathlineto{\pgfqpoint{-0.048611in}{0.000000in}}%
\pgfusepath{stroke,fill}%
}%
\begin{pgfscope}%
\pgfsys@transformshift{4.558355in}{0.617934in}%
\pgfsys@useobject{currentmarker}{}%
\end{pgfscope}%
\end{pgfscope}%
\begin{pgfscope}%
\definecolor{textcolor}{rgb}{0.000000,0.000000,0.000000}%
\pgfsetstrokecolor{textcolor}%
\pgfsetfillcolor{textcolor}%
\pgftext[x=4.123879in,y=0.565173in,left,base]{\color{textcolor}\sffamily\fontsize{10.000000}{12.000000}\selectfont −1.3}%
\end{pgfscope}%
\begin{pgfscope}%
\pgfsetbuttcap%
\pgfsetroundjoin%
\definecolor{currentfill}{rgb}{0.000000,0.000000,0.000000}%
\pgfsetfillcolor{currentfill}%
\pgfsetlinewidth{0.803000pt}%
\definecolor{currentstroke}{rgb}{0.000000,0.000000,0.000000}%
\pgfsetstrokecolor{currentstroke}%
\pgfsetdash{}{0pt}%
\pgfsys@defobject{currentmarker}{\pgfqpoint{-0.048611in}{0.000000in}}{\pgfqpoint{0.000000in}{0.000000in}}{%
\pgfpathmoveto{\pgfqpoint{0.000000in}{0.000000in}}%
\pgfpathlineto{\pgfqpoint{-0.048611in}{0.000000in}}%
\pgfusepath{stroke,fill}%
}%
\begin{pgfscope}%
\pgfsys@transformshift{4.558355in}{1.149502in}%
\pgfsys@useobject{currentmarker}{}%
\end{pgfscope}%
\end{pgfscope}%
\begin{pgfscope}%
\definecolor{textcolor}{rgb}{0.000000,0.000000,0.000000}%
\pgfsetstrokecolor{textcolor}%
\pgfsetfillcolor{textcolor}%
\pgftext[x=4.123879in,y=1.096740in,left,base]{\color{textcolor}\sffamily\fontsize{10.000000}{12.000000}\selectfont −1.2}%
\end{pgfscope}%
\begin{pgfscope}%
\pgfsetbuttcap%
\pgfsetroundjoin%
\definecolor{currentfill}{rgb}{0.000000,0.000000,0.000000}%
\pgfsetfillcolor{currentfill}%
\pgfsetlinewidth{0.803000pt}%
\definecolor{currentstroke}{rgb}{0.000000,0.000000,0.000000}%
\pgfsetstrokecolor{currentstroke}%
\pgfsetdash{}{0pt}%
\pgfsys@defobject{currentmarker}{\pgfqpoint{-0.048611in}{0.000000in}}{\pgfqpoint{0.000000in}{0.000000in}}{%
\pgfpathmoveto{\pgfqpoint{0.000000in}{0.000000in}}%
\pgfpathlineto{\pgfqpoint{-0.048611in}{0.000000in}}%
\pgfusepath{stroke,fill}%
}%
\begin{pgfscope}%
\pgfsys@transformshift{4.558355in}{1.681069in}%
\pgfsys@useobject{currentmarker}{}%
\end{pgfscope}%
\end{pgfscope}%
\begin{pgfscope}%
\definecolor{textcolor}{rgb}{0.000000,0.000000,0.000000}%
\pgfsetstrokecolor{textcolor}%
\pgfsetfillcolor{textcolor}%
\pgftext[x=4.123879in,y=1.628307in,left,base]{\color{textcolor}\sffamily\fontsize{10.000000}{12.000000}\selectfont −1.1}%
\end{pgfscope}%
\begin{pgfscope}%
\pgfsetbuttcap%
\pgfsetroundjoin%
\definecolor{currentfill}{rgb}{0.000000,0.000000,0.000000}%
\pgfsetfillcolor{currentfill}%
\pgfsetlinewidth{0.803000pt}%
\definecolor{currentstroke}{rgb}{0.000000,0.000000,0.000000}%
\pgfsetstrokecolor{currentstroke}%
\pgfsetdash{}{0pt}%
\pgfsys@defobject{currentmarker}{\pgfqpoint{-0.048611in}{0.000000in}}{\pgfqpoint{0.000000in}{0.000000in}}{%
\pgfpathmoveto{\pgfqpoint{0.000000in}{0.000000in}}%
\pgfpathlineto{\pgfqpoint{-0.048611in}{0.000000in}}%
\pgfusepath{stroke,fill}%
}%
\begin{pgfscope}%
\pgfsys@transformshift{4.558355in}{2.212636in}%
\pgfsys@useobject{currentmarker}{}%
\end{pgfscope}%
\end{pgfscope}%
\begin{pgfscope}%
\definecolor{textcolor}{rgb}{0.000000,0.000000,0.000000}%
\pgfsetstrokecolor{textcolor}%
\pgfsetfillcolor{textcolor}%
\pgftext[x=4.123879in,y=2.159874in,left,base]{\color{textcolor}\sffamily\fontsize{10.000000}{12.000000}\selectfont −1.0}%
\end{pgfscope}%
\begin{pgfscope}%
\pgfpathrectangle{\pgfqpoint{4.558355in}{0.386111in}}{\pgfqpoint{3.243034in}{2.316667in}}%
\pgfusepath{clip}%
\pgfsetrectcap%
\pgfsetroundjoin%
\pgfsetlinewidth{1.505625pt}%
\definecolor{currentstroke}{rgb}{0.121569,0.466667,0.705882}%
\pgfsetstrokecolor{currentstroke}%
\pgfsetdash{}{0pt}%
\pgfpathmoveto{\pgfqpoint{4.716981in}{0.502462in}}%
\pgfpathlineto{\pgfqpoint{4.863270in}{0.609271in}}%
\pgfpathlineto{\pgfqpoint{5.009559in}{0.708177in}}%
\pgfpathlineto{\pgfqpoint{5.155848in}{0.819179in}}%
\pgfpathlineto{\pgfqpoint{5.302137in}{0.939000in}}%
\pgfpathlineto{\pgfqpoint{5.448426in}{1.073311in}}%
\pgfpathlineto{\pgfqpoint{5.594716in}{1.165726in}}%
\pgfpathlineto{\pgfqpoint{5.741005in}{1.298197in}}%
\pgfpathlineto{\pgfqpoint{5.887294in}{1.455325in}}%
\pgfpathlineto{\pgfqpoint{6.033583in}{1.626318in}}%
\pgfpathlineto{\pgfqpoint{6.179872in}{1.728831in}}%
\pgfpathlineto{\pgfqpoint{6.326161in}{1.865779in}}%
\pgfpathlineto{\pgfqpoint{6.472450in}{2.022678in}}%
\pgfpathlineto{\pgfqpoint{6.618739in}{2.099750in}}%
\pgfpathlineto{\pgfqpoint{6.765028in}{2.230082in}}%
\pgfpathlineto{\pgfqpoint{6.911317in}{2.299622in}}%
\pgfpathlineto{\pgfqpoint{7.057606in}{2.381090in}}%
\pgfpathlineto{\pgfqpoint{7.203895in}{2.462925in}}%
\pgfpathlineto{\pgfqpoint{7.350185in}{2.493989in}}%
\pgfpathlineto{\pgfqpoint{7.496474in}{2.537220in}}%
\pgfpathlineto{\pgfqpoint{7.642763in}{2.579482in}}%
\pgfusepath{stroke}%
\end{pgfscope}%
\begin{pgfscope}%
\pgfsetrectcap%
\pgfsetmiterjoin%
\pgfsetlinewidth{0.803000pt}%
\definecolor{currentstroke}{rgb}{0.000000,0.000000,0.000000}%
\pgfsetstrokecolor{currentstroke}%
\pgfsetdash{}{0pt}%
\pgfpathmoveto{\pgfqpoint{4.558355in}{0.386111in}}%
\pgfpathlineto{\pgfqpoint{4.558355in}{2.702778in}}%
\pgfusepath{stroke}%
\end{pgfscope}%
\begin{pgfscope}%
\pgfsetrectcap%
\pgfsetmiterjoin%
\pgfsetlinewidth{0.803000pt}%
\definecolor{currentstroke}{rgb}{0.000000,0.000000,0.000000}%
\pgfsetstrokecolor{currentstroke}%
\pgfsetdash{}{0pt}%
\pgfpathmoveto{\pgfqpoint{7.801389in}{0.386111in}}%
\pgfpathlineto{\pgfqpoint{7.801389in}{2.702778in}}%
\pgfusepath{stroke}%
\end{pgfscope}%
\begin{pgfscope}%
\pgfsetrectcap%
\pgfsetmiterjoin%
\pgfsetlinewidth{0.803000pt}%
\definecolor{currentstroke}{rgb}{0.000000,0.000000,0.000000}%
\pgfsetstrokecolor{currentstroke}%
\pgfsetdash{}{0pt}%
\pgfpathmoveto{\pgfqpoint{4.558355in}{0.386111in}}%
\pgfpathlineto{\pgfqpoint{7.801389in}{0.386111in}}%
\pgfusepath{stroke}%
\end{pgfscope}%
\begin{pgfscope}%
\pgfsetrectcap%
\pgfsetmiterjoin%
\pgfsetlinewidth{0.803000pt}%
\definecolor{currentstroke}{rgb}{0.000000,0.000000,0.000000}%
\pgfsetstrokecolor{currentstroke}%
\pgfsetdash{}{0pt}%
\pgfpathmoveto{\pgfqpoint{4.558355in}{2.702778in}}%
\pgfpathlineto{\pgfqpoint{7.801389in}{2.702778in}}%
\pgfusepath{stroke}%
\end{pgfscope}%
\begin{pgfscope}%
\definecolor{textcolor}{rgb}{0.000000,0.000000,0.000000}%
\pgfsetstrokecolor{textcolor}%
\pgfsetfillcolor{textcolor}%
\pgftext[x=6.179872in,y=2.786111in,,base]{\color{textcolor}\sffamily\fontsize{12.000000}{14.400000}\selectfont \(\displaystyle  N = 24 \)}%
\end{pgfscope}%
\end{pgfpicture}%
\makeatother%
\endgroup%
}
\caption{Internal energy $u$ for finer temperature $T$ for the 3-D problem using kinetic Monte Carlo algorithm}
\label{Fig:HeatBig3D}
\end{figure}

\begin{figure}[htbp]
\centering
\scalebox{0.666}{%% Creator: Matplotlib, PGF backend
%%
%% To include the figure in your LaTeX document, write
%%   \input{<filename>.pgf}
%%
%% Make sure the required packages are loaded in your preamble
%%   \usepackage{pgf}
%%
%% Figures using additional raster images can only be included by \input if
%% they are in the same directory as the main LaTeX file. For loading figures
%% from other directories you can use the `import` package
%%   \usepackage{import}
%% and then include the figures with
%%   \import{<path to file>}{<filename>.pgf}
%%
%% Matplotlib used the following preamble
%%   \usepackage{fontspec}
%%   \setmainfont{DejaVuSerif.ttf}[Path=/home/lzh/anaconda3/envs/numana/lib/python3.7/site-packages/matplotlib/mpl-data/fonts/ttf/]
%%   \setsansfont{DejaVuSans.ttf}[Path=/home/lzh/anaconda3/envs/numana/lib/python3.7/site-packages/matplotlib/mpl-data/fonts/ttf/]
%%   \setmonofont{DejaVuSansMono.ttf}[Path=/home/lzh/anaconda3/envs/numana/lib/python3.7/site-packages/matplotlib/mpl-data/fonts/ttf/]
%%
\begingroup%
\makeatletter%
\begin{pgfpicture}%
\pgfpathrectangle{\pgfpointorigin}{\pgfqpoint{8.000000in}{6.000000in}}%
\pgfusepath{use as bounding box, clip}%
\begin{pgfscope}%
\pgfsetbuttcap%
\pgfsetmiterjoin%
\definecolor{currentfill}{rgb}{1.000000,1.000000,1.000000}%
\pgfsetfillcolor{currentfill}%
\pgfsetlinewidth{0.000000pt}%
\definecolor{currentstroke}{rgb}{1.000000,1.000000,1.000000}%
\pgfsetstrokecolor{currentstroke}%
\pgfsetdash{}{0pt}%
\pgfpathmoveto{\pgfqpoint{0.000000in}{0.000000in}}%
\pgfpathlineto{\pgfqpoint{8.000000in}{0.000000in}}%
\pgfpathlineto{\pgfqpoint{8.000000in}{6.000000in}}%
\pgfpathlineto{\pgfqpoint{0.000000in}{6.000000in}}%
\pgfpathclose%
\pgfusepath{fill}%
\end{pgfscope}%
\begin{pgfscope}%
\pgfsetbuttcap%
\pgfsetmiterjoin%
\definecolor{currentfill}{rgb}{1.000000,1.000000,1.000000}%
\pgfsetfillcolor{currentfill}%
\pgfsetlinewidth{0.000000pt}%
\definecolor{currentstroke}{rgb}{0.000000,0.000000,0.000000}%
\pgfsetstrokecolor{currentstroke}%
\pgfsetstrokeopacity{0.000000}%
\pgfsetdash{}{0pt}%
\pgfpathmoveto{\pgfqpoint{0.557222in}{3.312222in}}%
\pgfpathlineto{\pgfqpoint{3.878555in}{3.312222in}}%
\pgfpathlineto{\pgfqpoint{3.878555in}{5.636667in}}%
\pgfpathlineto{\pgfqpoint{0.557222in}{5.636667in}}%
\pgfpathclose%
\pgfusepath{fill}%
\end{pgfscope}%
\begin{pgfscope}%
\pgfpathrectangle{\pgfqpoint{0.557222in}{3.312222in}}{\pgfqpoint{3.321333in}{2.324444in}}%
\pgfusepath{clip}%
\pgfsetbuttcap%
\pgfsetroundjoin%
\definecolor{currentfill}{rgb}{0.121569,0.466667,0.705882}%
\pgfsetfillcolor{currentfill}%
\pgfsetfillopacity{0.300000}%
\pgfsetlinewidth{1.003750pt}%
\definecolor{currentstroke}{rgb}{0.121569,0.466667,0.705882}%
\pgfsetstrokecolor{currentstroke}%
\pgfsetstrokeopacity{0.300000}%
\pgfsetdash{}{0pt}%
\pgfpathmoveto{\pgfqpoint{0.716480in}{3.672916in}}%
\pgfpathlineto{\pgfqpoint{0.716480in}{3.417879in}}%
\pgfpathlineto{\pgfqpoint{0.866621in}{3.626120in}}%
\pgfpathlineto{\pgfqpoint{1.016761in}{3.871857in}}%
\pgfpathlineto{\pgfqpoint{1.166902in}{3.987270in}}%
\pgfpathlineto{\pgfqpoint{1.317043in}{4.211289in}}%
\pgfpathlineto{\pgfqpoint{1.467184in}{4.414802in}}%
\pgfpathlineto{\pgfqpoint{1.617325in}{4.507409in}}%
\pgfpathlineto{\pgfqpoint{1.767466in}{4.662197in}}%
\pgfpathlineto{\pgfqpoint{1.917607in}{4.770355in}}%
\pgfpathlineto{\pgfqpoint{2.067748in}{4.908875in}}%
\pgfpathlineto{\pgfqpoint{2.217889in}{5.023077in}}%
\pgfpathlineto{\pgfqpoint{2.368030in}{5.106082in}}%
\pgfpathlineto{\pgfqpoint{2.518171in}{5.180012in}}%
\pgfpathlineto{\pgfqpoint{2.668311in}{5.263007in}}%
\pgfpathlineto{\pgfqpoint{2.818452in}{5.401152in}}%
\pgfpathlineto{\pgfqpoint{2.968593in}{5.305616in}}%
\pgfpathlineto{\pgfqpoint{3.118734in}{5.309602in}}%
\pgfpathlineto{\pgfqpoint{3.268875in}{5.081954in}}%
\pgfpathlineto{\pgfqpoint{3.419016in}{5.066298in}}%
\pgfpathlineto{\pgfqpoint{3.569157in}{4.915523in}}%
\pgfpathlineto{\pgfqpoint{3.719298in}{4.758516in}}%
\pgfpathlineto{\pgfqpoint{3.719298in}{4.877711in}}%
\pgfpathlineto{\pgfqpoint{3.719298in}{4.877711in}}%
\pgfpathlineto{\pgfqpoint{3.569157in}{5.029667in}}%
\pgfpathlineto{\pgfqpoint{3.419016in}{5.231818in}}%
\pgfpathlineto{\pgfqpoint{3.268875in}{5.385278in}}%
\pgfpathlineto{\pgfqpoint{3.118734in}{5.358833in}}%
\pgfpathlineto{\pgfqpoint{2.968593in}{5.428529in}}%
\pgfpathlineto{\pgfqpoint{2.818452in}{5.531010in}}%
\pgfpathlineto{\pgfqpoint{2.668311in}{5.491378in}}%
\pgfpathlineto{\pgfqpoint{2.518171in}{5.409003in}}%
\pgfpathlineto{\pgfqpoint{2.368030in}{5.378136in}}%
\pgfpathlineto{\pgfqpoint{2.217889in}{5.294052in}}%
\pgfpathlineto{\pgfqpoint{2.067748in}{5.153425in}}%
\pgfpathlineto{\pgfqpoint{1.917607in}{5.154178in}}%
\pgfpathlineto{\pgfqpoint{1.767466in}{4.925890in}}%
\pgfpathlineto{\pgfqpoint{1.617325in}{4.769365in}}%
\pgfpathlineto{\pgfqpoint{1.467184in}{4.534989in}}%
\pgfpathlineto{\pgfqpoint{1.317043in}{4.399321in}}%
\pgfpathlineto{\pgfqpoint{1.166902in}{4.234998in}}%
\pgfpathlineto{\pgfqpoint{1.016761in}{3.937211in}}%
\pgfpathlineto{\pgfqpoint{0.866621in}{3.801449in}}%
\pgfpathlineto{\pgfqpoint{0.716480in}{3.672916in}}%
\pgfpathclose%
\pgfusepath{stroke,fill}%
\end{pgfscope}%
\begin{pgfscope}%
\pgfpathrectangle{\pgfqpoint{0.557222in}{3.312222in}}{\pgfqpoint{3.321333in}{2.324444in}}%
\pgfusepath{clip}%
\pgfsetbuttcap%
\pgfsetroundjoin%
\definecolor{currentfill}{rgb}{0.121569,0.466667,0.705882}%
\pgfsetfillcolor{currentfill}%
\pgfsetlinewidth{1.003750pt}%
\definecolor{currentstroke}{rgb}{0.121569,0.466667,0.705882}%
\pgfsetstrokecolor{currentstroke}%
\pgfsetdash{}{0pt}%
\pgfsys@defobject{currentmarker}{\pgfqpoint{-0.009821in}{-0.009821in}}{\pgfqpoint{0.009821in}{0.009821in}}{%
\pgfpathmoveto{\pgfqpoint{0.000000in}{-0.009821in}}%
\pgfpathcurveto{\pgfqpoint{0.002605in}{-0.009821in}}{\pgfqpoint{0.005103in}{-0.008786in}}{\pgfqpoint{0.006944in}{-0.006944in}}%
\pgfpathcurveto{\pgfqpoint{0.008786in}{-0.005103in}}{\pgfqpoint{0.009821in}{-0.002605in}}{\pgfqpoint{0.009821in}{0.000000in}}%
\pgfpathcurveto{\pgfqpoint{0.009821in}{0.002605in}}{\pgfqpoint{0.008786in}{0.005103in}}{\pgfqpoint{0.006944in}{0.006944in}}%
\pgfpathcurveto{\pgfqpoint{0.005103in}{0.008786in}}{\pgfqpoint{0.002605in}{0.009821in}}{\pgfqpoint{0.000000in}{0.009821in}}%
\pgfpathcurveto{\pgfqpoint{-0.002605in}{0.009821in}}{\pgfqpoint{-0.005103in}{0.008786in}}{\pgfqpoint{-0.006944in}{0.006944in}}%
\pgfpathcurveto{\pgfqpoint{-0.008786in}{0.005103in}}{\pgfqpoint{-0.009821in}{0.002605in}}{\pgfqpoint{-0.009821in}{0.000000in}}%
\pgfpathcurveto{\pgfqpoint{-0.009821in}{-0.002605in}}{\pgfqpoint{-0.008786in}{-0.005103in}}{\pgfqpoint{-0.006944in}{-0.006944in}}%
\pgfpathcurveto{\pgfqpoint{-0.005103in}{-0.008786in}}{\pgfqpoint{-0.002605in}{-0.009821in}}{\pgfqpoint{0.000000in}{-0.009821in}}%
\pgfpathclose%
\pgfusepath{stroke,fill}%
}%
\begin{pgfscope}%
\pgfsys@transformshift{0.716480in}{3.545397in}%
\pgfsys@useobject{currentmarker}{}%
\end{pgfscope}%
\begin{pgfscope}%
\pgfsys@transformshift{0.866621in}{3.713785in}%
\pgfsys@useobject{currentmarker}{}%
\end{pgfscope}%
\begin{pgfscope}%
\pgfsys@transformshift{1.016761in}{3.904534in}%
\pgfsys@useobject{currentmarker}{}%
\end{pgfscope}%
\begin{pgfscope}%
\pgfsys@transformshift{1.166902in}{4.111134in}%
\pgfsys@useobject{currentmarker}{}%
\end{pgfscope}%
\begin{pgfscope}%
\pgfsys@transformshift{1.317043in}{4.305305in}%
\pgfsys@useobject{currentmarker}{}%
\end{pgfscope}%
\begin{pgfscope}%
\pgfsys@transformshift{1.467184in}{4.474895in}%
\pgfsys@useobject{currentmarker}{}%
\end{pgfscope}%
\begin{pgfscope}%
\pgfsys@transformshift{1.617325in}{4.638387in}%
\pgfsys@useobject{currentmarker}{}%
\end{pgfscope}%
\begin{pgfscope}%
\pgfsys@transformshift{1.767466in}{4.794043in}%
\pgfsys@useobject{currentmarker}{}%
\end{pgfscope}%
\begin{pgfscope}%
\pgfsys@transformshift{1.917607in}{4.962267in}%
\pgfsys@useobject{currentmarker}{}%
\end{pgfscope}%
\begin{pgfscope}%
\pgfsys@transformshift{2.067748in}{5.031150in}%
\pgfsys@useobject{currentmarker}{}%
\end{pgfscope}%
\begin{pgfscope}%
\pgfsys@transformshift{2.217889in}{5.158565in}%
\pgfsys@useobject{currentmarker}{}%
\end{pgfscope}%
\begin{pgfscope}%
\pgfsys@transformshift{2.368030in}{5.242109in}%
\pgfsys@useobject{currentmarker}{}%
\end{pgfscope}%
\begin{pgfscope}%
\pgfsys@transformshift{2.518171in}{5.294508in}%
\pgfsys@useobject{currentmarker}{}%
\end{pgfscope}%
\begin{pgfscope}%
\pgfsys@transformshift{2.668311in}{5.377193in}%
\pgfsys@useobject{currentmarker}{}%
\end{pgfscope}%
\begin{pgfscope}%
\pgfsys@transformshift{2.818452in}{5.466081in}%
\pgfsys@useobject{currentmarker}{}%
\end{pgfscope}%
\begin{pgfscope}%
\pgfsys@transformshift{2.968593in}{5.367072in}%
\pgfsys@useobject{currentmarker}{}%
\end{pgfscope}%
\begin{pgfscope}%
\pgfsys@transformshift{3.118734in}{5.334217in}%
\pgfsys@useobject{currentmarker}{}%
\end{pgfscope}%
\begin{pgfscope}%
\pgfsys@transformshift{3.268875in}{5.233616in}%
\pgfsys@useobject{currentmarker}{}%
\end{pgfscope}%
\begin{pgfscope}%
\pgfsys@transformshift{3.419016in}{5.149058in}%
\pgfsys@useobject{currentmarker}{}%
\end{pgfscope}%
\begin{pgfscope}%
\pgfsys@transformshift{3.569157in}{4.972595in}%
\pgfsys@useobject{currentmarker}{}%
\end{pgfscope}%
\begin{pgfscope}%
\pgfsys@transformshift{3.719298in}{4.818114in}%
\pgfsys@useobject{currentmarker}{}%
\end{pgfscope}%
\end{pgfscope}%
\begin{pgfscope}%
\pgfsetbuttcap%
\pgfsetroundjoin%
\definecolor{currentfill}{rgb}{0.000000,0.000000,0.000000}%
\pgfsetfillcolor{currentfill}%
\pgfsetlinewidth{0.803000pt}%
\definecolor{currentstroke}{rgb}{0.000000,0.000000,0.000000}%
\pgfsetstrokecolor{currentstroke}%
\pgfsetdash{}{0pt}%
\pgfsys@defobject{currentmarker}{\pgfqpoint{0.000000in}{-0.048611in}}{\pgfqpoint{0.000000in}{0.000000in}}{%
\pgfpathmoveto{\pgfqpoint{0.000000in}{0.000000in}}%
\pgfpathlineto{\pgfqpoint{0.000000in}{-0.048611in}}%
\pgfusepath{stroke,fill}%
}%
\begin{pgfscope}%
\pgfsys@transformshift{0.566339in}{3.312222in}%
\pgfsys@useobject{currentmarker}{}%
\end{pgfscope}%
\end{pgfscope}%
\begin{pgfscope}%
\definecolor{textcolor}{rgb}{0.000000,0.000000,0.000000}%
\pgfsetstrokecolor{textcolor}%
\pgfsetfillcolor{textcolor}%
\pgftext[x=0.566339in,y=3.215000in,,top]{\color{textcolor}\sffamily\fontsize{10.000000}{12.000000}\selectfont 4.20}%
\end{pgfscope}%
\begin{pgfscope}%
\pgfsetbuttcap%
\pgfsetroundjoin%
\definecolor{currentfill}{rgb}{0.000000,0.000000,0.000000}%
\pgfsetfillcolor{currentfill}%
\pgfsetlinewidth{0.803000pt}%
\definecolor{currentstroke}{rgb}{0.000000,0.000000,0.000000}%
\pgfsetstrokecolor{currentstroke}%
\pgfsetdash{}{0pt}%
\pgfsys@defobject{currentmarker}{\pgfqpoint{0.000000in}{-0.048611in}}{\pgfqpoint{0.000000in}{0.000000in}}{%
\pgfpathmoveto{\pgfqpoint{0.000000in}{0.000000in}}%
\pgfpathlineto{\pgfqpoint{0.000000in}{-0.048611in}}%
\pgfusepath{stroke,fill}%
}%
\begin{pgfscope}%
\pgfsys@transformshift{1.317043in}{3.312222in}%
\pgfsys@useobject{currentmarker}{}%
\end{pgfscope}%
\end{pgfscope}%
\begin{pgfscope}%
\definecolor{textcolor}{rgb}{0.000000,0.000000,0.000000}%
\pgfsetstrokecolor{textcolor}%
\pgfsetfillcolor{textcolor}%
\pgftext[x=1.317043in,y=3.215000in,,top]{\color{textcolor}\sffamily\fontsize{10.000000}{12.000000}\selectfont 4.25}%
\end{pgfscope}%
\begin{pgfscope}%
\pgfsetbuttcap%
\pgfsetroundjoin%
\definecolor{currentfill}{rgb}{0.000000,0.000000,0.000000}%
\pgfsetfillcolor{currentfill}%
\pgfsetlinewidth{0.803000pt}%
\definecolor{currentstroke}{rgb}{0.000000,0.000000,0.000000}%
\pgfsetstrokecolor{currentstroke}%
\pgfsetdash{}{0pt}%
\pgfsys@defobject{currentmarker}{\pgfqpoint{0.000000in}{-0.048611in}}{\pgfqpoint{0.000000in}{0.000000in}}{%
\pgfpathmoveto{\pgfqpoint{0.000000in}{0.000000in}}%
\pgfpathlineto{\pgfqpoint{0.000000in}{-0.048611in}}%
\pgfusepath{stroke,fill}%
}%
\begin{pgfscope}%
\pgfsys@transformshift{2.067748in}{3.312222in}%
\pgfsys@useobject{currentmarker}{}%
\end{pgfscope}%
\end{pgfscope}%
\begin{pgfscope}%
\definecolor{textcolor}{rgb}{0.000000,0.000000,0.000000}%
\pgfsetstrokecolor{textcolor}%
\pgfsetfillcolor{textcolor}%
\pgftext[x=2.067748in,y=3.215000in,,top]{\color{textcolor}\sffamily\fontsize{10.000000}{12.000000}\selectfont 4.30}%
\end{pgfscope}%
\begin{pgfscope}%
\pgfsetbuttcap%
\pgfsetroundjoin%
\definecolor{currentfill}{rgb}{0.000000,0.000000,0.000000}%
\pgfsetfillcolor{currentfill}%
\pgfsetlinewidth{0.803000pt}%
\definecolor{currentstroke}{rgb}{0.000000,0.000000,0.000000}%
\pgfsetstrokecolor{currentstroke}%
\pgfsetdash{}{0pt}%
\pgfsys@defobject{currentmarker}{\pgfqpoint{0.000000in}{-0.048611in}}{\pgfqpoint{0.000000in}{0.000000in}}{%
\pgfpathmoveto{\pgfqpoint{0.000000in}{0.000000in}}%
\pgfpathlineto{\pgfqpoint{0.000000in}{-0.048611in}}%
\pgfusepath{stroke,fill}%
}%
\begin{pgfscope}%
\pgfsys@transformshift{2.818452in}{3.312222in}%
\pgfsys@useobject{currentmarker}{}%
\end{pgfscope}%
\end{pgfscope}%
\begin{pgfscope}%
\definecolor{textcolor}{rgb}{0.000000,0.000000,0.000000}%
\pgfsetstrokecolor{textcolor}%
\pgfsetfillcolor{textcolor}%
\pgftext[x=2.818452in,y=3.215000in,,top]{\color{textcolor}\sffamily\fontsize{10.000000}{12.000000}\selectfont 4.35}%
\end{pgfscope}%
\begin{pgfscope}%
\pgfsetbuttcap%
\pgfsetroundjoin%
\definecolor{currentfill}{rgb}{0.000000,0.000000,0.000000}%
\pgfsetfillcolor{currentfill}%
\pgfsetlinewidth{0.803000pt}%
\definecolor{currentstroke}{rgb}{0.000000,0.000000,0.000000}%
\pgfsetstrokecolor{currentstroke}%
\pgfsetdash{}{0pt}%
\pgfsys@defobject{currentmarker}{\pgfqpoint{0.000000in}{-0.048611in}}{\pgfqpoint{0.000000in}{0.000000in}}{%
\pgfpathmoveto{\pgfqpoint{0.000000in}{0.000000in}}%
\pgfpathlineto{\pgfqpoint{0.000000in}{-0.048611in}}%
\pgfusepath{stroke,fill}%
}%
\begin{pgfscope}%
\pgfsys@transformshift{3.569157in}{3.312222in}%
\pgfsys@useobject{currentmarker}{}%
\end{pgfscope}%
\end{pgfscope}%
\begin{pgfscope}%
\definecolor{textcolor}{rgb}{0.000000,0.000000,0.000000}%
\pgfsetstrokecolor{textcolor}%
\pgfsetfillcolor{textcolor}%
\pgftext[x=3.569157in,y=3.215000in,,top]{\color{textcolor}\sffamily\fontsize{10.000000}{12.000000}\selectfont 4.40}%
\end{pgfscope}%
\begin{pgfscope}%
\pgfsetbuttcap%
\pgfsetroundjoin%
\definecolor{currentfill}{rgb}{0.000000,0.000000,0.000000}%
\pgfsetfillcolor{currentfill}%
\pgfsetlinewidth{0.803000pt}%
\definecolor{currentstroke}{rgb}{0.000000,0.000000,0.000000}%
\pgfsetstrokecolor{currentstroke}%
\pgfsetdash{}{0pt}%
\pgfsys@defobject{currentmarker}{\pgfqpoint{-0.048611in}{0.000000in}}{\pgfqpoint{0.000000in}{0.000000in}}{%
\pgfpathmoveto{\pgfqpoint{0.000000in}{0.000000in}}%
\pgfpathlineto{\pgfqpoint{-0.048611in}{0.000000in}}%
\pgfusepath{stroke,fill}%
}%
\begin{pgfscope}%
\pgfsys@transformshift{0.557222in}{3.612294in}%
\pgfsys@useobject{currentmarker}{}%
\end{pgfscope}%
\end{pgfscope}%
\begin{pgfscope}%
\definecolor{textcolor}{rgb}{0.000000,0.000000,0.000000}%
\pgfsetstrokecolor{textcolor}%
\pgfsetfillcolor{textcolor}%
\pgftext[x=0.150755in,y=3.559532in,left,base]{\color{textcolor}\sffamily\fontsize{10.000000}{12.000000}\selectfont 1.60}%
\end{pgfscope}%
\begin{pgfscope}%
\pgfsetbuttcap%
\pgfsetroundjoin%
\definecolor{currentfill}{rgb}{0.000000,0.000000,0.000000}%
\pgfsetfillcolor{currentfill}%
\pgfsetlinewidth{0.803000pt}%
\definecolor{currentstroke}{rgb}{0.000000,0.000000,0.000000}%
\pgfsetstrokecolor{currentstroke}%
\pgfsetdash{}{0pt}%
\pgfsys@defobject{currentmarker}{\pgfqpoint{-0.048611in}{0.000000in}}{\pgfqpoint{0.000000in}{0.000000in}}{%
\pgfpathmoveto{\pgfqpoint{0.000000in}{0.000000in}}%
\pgfpathlineto{\pgfqpoint{-0.048611in}{0.000000in}}%
\pgfusepath{stroke,fill}%
}%
\begin{pgfscope}%
\pgfsys@transformshift{0.557222in}{4.082278in}%
\pgfsys@useobject{currentmarker}{}%
\end{pgfscope}%
\end{pgfscope}%
\begin{pgfscope}%
\definecolor{textcolor}{rgb}{0.000000,0.000000,0.000000}%
\pgfsetstrokecolor{textcolor}%
\pgfsetfillcolor{textcolor}%
\pgftext[x=0.150755in,y=4.029517in,left,base]{\color{textcolor}\sffamily\fontsize{10.000000}{12.000000}\selectfont 1.65}%
\end{pgfscope}%
\begin{pgfscope}%
\pgfsetbuttcap%
\pgfsetroundjoin%
\definecolor{currentfill}{rgb}{0.000000,0.000000,0.000000}%
\pgfsetfillcolor{currentfill}%
\pgfsetlinewidth{0.803000pt}%
\definecolor{currentstroke}{rgb}{0.000000,0.000000,0.000000}%
\pgfsetstrokecolor{currentstroke}%
\pgfsetdash{}{0pt}%
\pgfsys@defobject{currentmarker}{\pgfqpoint{-0.048611in}{0.000000in}}{\pgfqpoint{0.000000in}{0.000000in}}{%
\pgfpathmoveto{\pgfqpoint{0.000000in}{0.000000in}}%
\pgfpathlineto{\pgfqpoint{-0.048611in}{0.000000in}}%
\pgfusepath{stroke,fill}%
}%
\begin{pgfscope}%
\pgfsys@transformshift{0.557222in}{4.552263in}%
\pgfsys@useobject{currentmarker}{}%
\end{pgfscope}%
\end{pgfscope}%
\begin{pgfscope}%
\definecolor{textcolor}{rgb}{0.000000,0.000000,0.000000}%
\pgfsetstrokecolor{textcolor}%
\pgfsetfillcolor{textcolor}%
\pgftext[x=0.150755in,y=4.499501in,left,base]{\color{textcolor}\sffamily\fontsize{10.000000}{12.000000}\selectfont 1.70}%
\end{pgfscope}%
\begin{pgfscope}%
\pgfsetbuttcap%
\pgfsetroundjoin%
\definecolor{currentfill}{rgb}{0.000000,0.000000,0.000000}%
\pgfsetfillcolor{currentfill}%
\pgfsetlinewidth{0.803000pt}%
\definecolor{currentstroke}{rgb}{0.000000,0.000000,0.000000}%
\pgfsetstrokecolor{currentstroke}%
\pgfsetdash{}{0pt}%
\pgfsys@defobject{currentmarker}{\pgfqpoint{-0.048611in}{0.000000in}}{\pgfqpoint{0.000000in}{0.000000in}}{%
\pgfpathmoveto{\pgfqpoint{0.000000in}{0.000000in}}%
\pgfpathlineto{\pgfqpoint{-0.048611in}{0.000000in}}%
\pgfusepath{stroke,fill}%
}%
\begin{pgfscope}%
\pgfsys@transformshift{0.557222in}{5.022247in}%
\pgfsys@useobject{currentmarker}{}%
\end{pgfscope}%
\end{pgfscope}%
\begin{pgfscope}%
\definecolor{textcolor}{rgb}{0.000000,0.000000,0.000000}%
\pgfsetstrokecolor{textcolor}%
\pgfsetfillcolor{textcolor}%
\pgftext[x=0.150755in,y=4.969485in,left,base]{\color{textcolor}\sffamily\fontsize{10.000000}{12.000000}\selectfont 1.75}%
\end{pgfscope}%
\begin{pgfscope}%
\pgfsetbuttcap%
\pgfsetroundjoin%
\definecolor{currentfill}{rgb}{0.000000,0.000000,0.000000}%
\pgfsetfillcolor{currentfill}%
\pgfsetlinewidth{0.803000pt}%
\definecolor{currentstroke}{rgb}{0.000000,0.000000,0.000000}%
\pgfsetstrokecolor{currentstroke}%
\pgfsetdash{}{0pt}%
\pgfsys@defobject{currentmarker}{\pgfqpoint{-0.048611in}{0.000000in}}{\pgfqpoint{0.000000in}{0.000000in}}{%
\pgfpathmoveto{\pgfqpoint{0.000000in}{0.000000in}}%
\pgfpathlineto{\pgfqpoint{-0.048611in}{0.000000in}}%
\pgfusepath{stroke,fill}%
}%
\begin{pgfscope}%
\pgfsys@transformshift{0.557222in}{5.492231in}%
\pgfsys@useobject{currentmarker}{}%
\end{pgfscope}%
\end{pgfscope}%
\begin{pgfscope}%
\definecolor{textcolor}{rgb}{0.000000,0.000000,0.000000}%
\pgfsetstrokecolor{textcolor}%
\pgfsetfillcolor{textcolor}%
\pgftext[x=0.150755in,y=5.439470in,left,base]{\color{textcolor}\sffamily\fontsize{10.000000}{12.000000}\selectfont 1.80}%
\end{pgfscope}%
\begin{pgfscope}%
\pgfpathrectangle{\pgfqpoint{0.557222in}{3.312222in}}{\pgfqpoint{3.321333in}{2.324444in}}%
\pgfusepath{clip}%
\pgfsetrectcap%
\pgfsetroundjoin%
\pgfsetlinewidth{1.505625pt}%
\definecolor{currentstroke}{rgb}{0.121569,0.466667,0.705882}%
\pgfsetstrokecolor{currentstroke}%
\pgfsetdash{}{0pt}%
\pgfpathmoveto{\pgfqpoint{0.716480in}{3.545397in}}%
\pgfpathlineto{\pgfqpoint{0.866621in}{3.713785in}}%
\pgfpathlineto{\pgfqpoint{1.016761in}{3.904534in}}%
\pgfpathlineto{\pgfqpoint{1.166902in}{4.111134in}}%
\pgfpathlineto{\pgfqpoint{1.317043in}{4.305305in}}%
\pgfpathlineto{\pgfqpoint{1.467184in}{4.474895in}}%
\pgfpathlineto{\pgfqpoint{1.617325in}{4.638387in}}%
\pgfpathlineto{\pgfqpoint{1.767466in}{4.794043in}}%
\pgfpathlineto{\pgfqpoint{1.917607in}{4.962267in}}%
\pgfpathlineto{\pgfqpoint{2.067748in}{5.031150in}}%
\pgfpathlineto{\pgfqpoint{2.217889in}{5.158565in}}%
\pgfpathlineto{\pgfqpoint{2.368030in}{5.242109in}}%
\pgfpathlineto{\pgfqpoint{2.518171in}{5.294508in}}%
\pgfpathlineto{\pgfqpoint{2.668311in}{5.377193in}}%
\pgfpathlineto{\pgfqpoint{2.818452in}{5.466081in}}%
\pgfpathlineto{\pgfqpoint{2.968593in}{5.367072in}}%
\pgfpathlineto{\pgfqpoint{3.118734in}{5.334217in}}%
\pgfpathlineto{\pgfqpoint{3.268875in}{5.233616in}}%
\pgfpathlineto{\pgfqpoint{3.419016in}{5.149058in}}%
\pgfpathlineto{\pgfqpoint{3.569157in}{4.972595in}}%
\pgfpathlineto{\pgfqpoint{3.719298in}{4.818114in}}%
\pgfusepath{stroke}%
\end{pgfscope}%
\begin{pgfscope}%
\pgfsetrectcap%
\pgfsetmiterjoin%
\pgfsetlinewidth{0.803000pt}%
\definecolor{currentstroke}{rgb}{0.000000,0.000000,0.000000}%
\pgfsetstrokecolor{currentstroke}%
\pgfsetdash{}{0pt}%
\pgfpathmoveto{\pgfqpoint{0.557222in}{3.312222in}}%
\pgfpathlineto{\pgfqpoint{0.557222in}{5.636667in}}%
\pgfusepath{stroke}%
\end{pgfscope}%
\begin{pgfscope}%
\pgfsetrectcap%
\pgfsetmiterjoin%
\pgfsetlinewidth{0.803000pt}%
\definecolor{currentstroke}{rgb}{0.000000,0.000000,0.000000}%
\pgfsetstrokecolor{currentstroke}%
\pgfsetdash{}{0pt}%
\pgfpathmoveto{\pgfqpoint{3.878555in}{3.312222in}}%
\pgfpathlineto{\pgfqpoint{3.878555in}{5.636667in}}%
\pgfusepath{stroke}%
\end{pgfscope}%
\begin{pgfscope}%
\pgfsetrectcap%
\pgfsetmiterjoin%
\pgfsetlinewidth{0.803000pt}%
\definecolor{currentstroke}{rgb}{0.000000,0.000000,0.000000}%
\pgfsetstrokecolor{currentstroke}%
\pgfsetdash{}{0pt}%
\pgfpathmoveto{\pgfqpoint{0.557222in}{3.312222in}}%
\pgfpathlineto{\pgfqpoint{3.878555in}{3.312222in}}%
\pgfusepath{stroke}%
\end{pgfscope}%
\begin{pgfscope}%
\pgfsetrectcap%
\pgfsetmiterjoin%
\pgfsetlinewidth{0.803000pt}%
\definecolor{currentstroke}{rgb}{0.000000,0.000000,0.000000}%
\pgfsetstrokecolor{currentstroke}%
\pgfsetdash{}{0pt}%
\pgfpathmoveto{\pgfqpoint{0.557222in}{5.636667in}}%
\pgfpathlineto{\pgfqpoint{3.878555in}{5.636667in}}%
\pgfusepath{stroke}%
\end{pgfscope}%
\begin{pgfscope}%
\definecolor{textcolor}{rgb}{0.000000,0.000000,0.000000}%
\pgfsetstrokecolor{textcolor}%
\pgfsetfillcolor{textcolor}%
\pgftext[x=2.217889in,y=5.720000in,,base]{\color{textcolor}\sffamily\fontsize{12.000000}{14.400000}\selectfont \(\displaystyle  N = 8 \)}%
\end{pgfscope}%
\begin{pgfscope}%
\pgfsetbuttcap%
\pgfsetmiterjoin%
\definecolor{currentfill}{rgb}{1.000000,1.000000,1.000000}%
\pgfsetfillcolor{currentfill}%
\pgfsetlinewidth{0.000000pt}%
\definecolor{currentstroke}{rgb}{0.000000,0.000000,0.000000}%
\pgfsetstrokecolor{currentstroke}%
\pgfsetstrokeopacity{0.000000}%
\pgfsetdash{}{0pt}%
\pgfpathmoveto{\pgfqpoint{4.493667in}{3.312222in}}%
\pgfpathlineto{\pgfqpoint{7.815000in}{3.312222in}}%
\pgfpathlineto{\pgfqpoint{7.815000in}{5.636667in}}%
\pgfpathlineto{\pgfqpoint{4.493667in}{5.636667in}}%
\pgfpathclose%
\pgfusepath{fill}%
\end{pgfscope}%
\begin{pgfscope}%
\pgfpathrectangle{\pgfqpoint{4.493667in}{3.312222in}}{\pgfqpoint{3.321333in}{2.324444in}}%
\pgfusepath{clip}%
\pgfsetbuttcap%
\pgfsetroundjoin%
\definecolor{currentfill}{rgb}{0.121569,0.466667,0.705882}%
\pgfsetfillcolor{currentfill}%
\pgfsetfillopacity{0.300000}%
\pgfsetlinewidth{1.003750pt}%
\definecolor{currentstroke}{rgb}{0.121569,0.466667,0.705882}%
\pgfsetstrokecolor{currentstroke}%
\pgfsetstrokeopacity{0.300000}%
\pgfsetdash{}{0pt}%
\pgfpathmoveto{\pgfqpoint{4.652924in}{4.130337in}}%
\pgfpathlineto{\pgfqpoint{4.652924in}{3.831880in}}%
\pgfpathlineto{\pgfqpoint{4.803065in}{4.081670in}}%
\pgfpathlineto{\pgfqpoint{4.953206in}{4.185650in}}%
\pgfpathlineto{\pgfqpoint{5.103347in}{4.326550in}}%
\pgfpathlineto{\pgfqpoint{5.253488in}{4.412782in}}%
\pgfpathlineto{\pgfqpoint{5.403629in}{4.694028in}}%
\pgfpathlineto{\pgfqpoint{5.553770in}{4.672016in}}%
\pgfpathlineto{\pgfqpoint{5.703911in}{4.732707in}}%
\pgfpathlineto{\pgfqpoint{5.854052in}{5.004509in}}%
\pgfpathlineto{\pgfqpoint{6.004193in}{4.813395in}}%
\pgfpathlineto{\pgfqpoint{6.154333in}{5.188264in}}%
\pgfpathlineto{\pgfqpoint{6.304474in}{5.162685in}}%
\pgfpathlineto{\pgfqpoint{6.454615in}{5.217063in}}%
\pgfpathlineto{\pgfqpoint{6.604756in}{5.048375in}}%
\pgfpathlineto{\pgfqpoint{6.754897in}{4.969868in}}%
\pgfpathlineto{\pgfqpoint{6.905038in}{4.741271in}}%
\pgfpathlineto{\pgfqpoint{7.055179in}{4.656925in}}%
\pgfpathlineto{\pgfqpoint{7.205320in}{4.323229in}}%
\pgfpathlineto{\pgfqpoint{7.355461in}{4.067751in}}%
\pgfpathlineto{\pgfqpoint{7.505602in}{3.697392in}}%
\pgfpathlineto{\pgfqpoint{7.655743in}{3.417879in}}%
\pgfpathlineto{\pgfqpoint{7.655743in}{3.554916in}}%
\pgfpathlineto{\pgfqpoint{7.655743in}{3.554916in}}%
\pgfpathlineto{\pgfqpoint{7.505602in}{3.873930in}}%
\pgfpathlineto{\pgfqpoint{7.355461in}{4.202160in}}%
\pgfpathlineto{\pgfqpoint{7.205320in}{4.592441in}}%
\pgfpathlineto{\pgfqpoint{7.055179in}{4.722552in}}%
\pgfpathlineto{\pgfqpoint{6.905038in}{5.031308in}}%
\pgfpathlineto{\pgfqpoint{6.754897in}{5.140423in}}%
\pgfpathlineto{\pgfqpoint{6.604756in}{5.219895in}}%
\pgfpathlineto{\pgfqpoint{6.454615in}{5.531010in}}%
\pgfpathlineto{\pgfqpoint{6.304474in}{5.370828in}}%
\pgfpathlineto{\pgfqpoint{6.154333in}{5.424746in}}%
\pgfpathlineto{\pgfqpoint{6.004193in}{5.293037in}}%
\pgfpathlineto{\pgfqpoint{5.854052in}{5.328443in}}%
\pgfpathlineto{\pgfqpoint{5.703911in}{5.153756in}}%
\pgfpathlineto{\pgfqpoint{5.553770in}{4.987146in}}%
\pgfpathlineto{\pgfqpoint{5.403629in}{4.923502in}}%
\pgfpathlineto{\pgfqpoint{5.253488in}{4.593892in}}%
\pgfpathlineto{\pgfqpoint{5.103347in}{4.445594in}}%
\pgfpathlineto{\pgfqpoint{4.953206in}{4.698401in}}%
\pgfpathlineto{\pgfqpoint{4.803065in}{4.282898in}}%
\pgfpathlineto{\pgfqpoint{4.652924in}{4.130337in}}%
\pgfpathclose%
\pgfusepath{stroke,fill}%
\end{pgfscope}%
\begin{pgfscope}%
\pgfpathrectangle{\pgfqpoint{4.493667in}{3.312222in}}{\pgfqpoint{3.321333in}{2.324444in}}%
\pgfusepath{clip}%
\pgfsetbuttcap%
\pgfsetroundjoin%
\definecolor{currentfill}{rgb}{0.121569,0.466667,0.705882}%
\pgfsetfillcolor{currentfill}%
\pgfsetlinewidth{1.003750pt}%
\definecolor{currentstroke}{rgb}{0.121569,0.466667,0.705882}%
\pgfsetstrokecolor{currentstroke}%
\pgfsetdash{}{0pt}%
\pgfsys@defobject{currentmarker}{\pgfqpoint{-0.009821in}{-0.009821in}}{\pgfqpoint{0.009821in}{0.009821in}}{%
\pgfpathmoveto{\pgfqpoint{0.000000in}{-0.009821in}}%
\pgfpathcurveto{\pgfqpoint{0.002605in}{-0.009821in}}{\pgfqpoint{0.005103in}{-0.008786in}}{\pgfqpoint{0.006944in}{-0.006944in}}%
\pgfpathcurveto{\pgfqpoint{0.008786in}{-0.005103in}}{\pgfqpoint{0.009821in}{-0.002605in}}{\pgfqpoint{0.009821in}{0.000000in}}%
\pgfpathcurveto{\pgfqpoint{0.009821in}{0.002605in}}{\pgfqpoint{0.008786in}{0.005103in}}{\pgfqpoint{0.006944in}{0.006944in}}%
\pgfpathcurveto{\pgfqpoint{0.005103in}{0.008786in}}{\pgfqpoint{0.002605in}{0.009821in}}{\pgfqpoint{0.000000in}{0.009821in}}%
\pgfpathcurveto{\pgfqpoint{-0.002605in}{0.009821in}}{\pgfqpoint{-0.005103in}{0.008786in}}{\pgfqpoint{-0.006944in}{0.006944in}}%
\pgfpathcurveto{\pgfqpoint{-0.008786in}{0.005103in}}{\pgfqpoint{-0.009821in}{0.002605in}}{\pgfqpoint{-0.009821in}{0.000000in}}%
\pgfpathcurveto{\pgfqpoint{-0.009821in}{-0.002605in}}{\pgfqpoint{-0.008786in}{-0.005103in}}{\pgfqpoint{-0.006944in}{-0.006944in}}%
\pgfpathcurveto{\pgfqpoint{-0.005103in}{-0.008786in}}{\pgfqpoint{-0.002605in}{-0.009821in}}{\pgfqpoint{0.000000in}{-0.009821in}}%
\pgfpathclose%
\pgfusepath{stroke,fill}%
}%
\begin{pgfscope}%
\pgfsys@transformshift{4.652924in}{3.981108in}%
\pgfsys@useobject{currentmarker}{}%
\end{pgfscope}%
\begin{pgfscope}%
\pgfsys@transformshift{4.803065in}{4.182284in}%
\pgfsys@useobject{currentmarker}{}%
\end{pgfscope}%
\begin{pgfscope}%
\pgfsys@transformshift{4.953206in}{4.442026in}%
\pgfsys@useobject{currentmarker}{}%
\end{pgfscope}%
\begin{pgfscope}%
\pgfsys@transformshift{5.103347in}{4.386072in}%
\pgfsys@useobject{currentmarker}{}%
\end{pgfscope}%
\begin{pgfscope}%
\pgfsys@transformshift{5.253488in}{4.503337in}%
\pgfsys@useobject{currentmarker}{}%
\end{pgfscope}%
\begin{pgfscope}%
\pgfsys@transformshift{5.403629in}{4.808765in}%
\pgfsys@useobject{currentmarker}{}%
\end{pgfscope}%
\begin{pgfscope}%
\pgfsys@transformshift{5.553770in}{4.829581in}%
\pgfsys@useobject{currentmarker}{}%
\end{pgfscope}%
\begin{pgfscope}%
\pgfsys@transformshift{5.703911in}{4.943231in}%
\pgfsys@useobject{currentmarker}{}%
\end{pgfscope}%
\begin{pgfscope}%
\pgfsys@transformshift{5.854052in}{5.166476in}%
\pgfsys@useobject{currentmarker}{}%
\end{pgfscope}%
\begin{pgfscope}%
\pgfsys@transformshift{6.004193in}{5.053216in}%
\pgfsys@useobject{currentmarker}{}%
\end{pgfscope}%
\begin{pgfscope}%
\pgfsys@transformshift{6.154333in}{5.306505in}%
\pgfsys@useobject{currentmarker}{}%
\end{pgfscope}%
\begin{pgfscope}%
\pgfsys@transformshift{6.304474in}{5.266757in}%
\pgfsys@useobject{currentmarker}{}%
\end{pgfscope}%
\begin{pgfscope}%
\pgfsys@transformshift{6.454615in}{5.374036in}%
\pgfsys@useobject{currentmarker}{}%
\end{pgfscope}%
\begin{pgfscope}%
\pgfsys@transformshift{6.604756in}{5.134135in}%
\pgfsys@useobject{currentmarker}{}%
\end{pgfscope}%
\begin{pgfscope}%
\pgfsys@transformshift{6.754897in}{5.055146in}%
\pgfsys@useobject{currentmarker}{}%
\end{pgfscope}%
\begin{pgfscope}%
\pgfsys@transformshift{6.905038in}{4.886289in}%
\pgfsys@useobject{currentmarker}{}%
\end{pgfscope}%
\begin{pgfscope}%
\pgfsys@transformshift{7.055179in}{4.689739in}%
\pgfsys@useobject{currentmarker}{}%
\end{pgfscope}%
\begin{pgfscope}%
\pgfsys@transformshift{7.205320in}{4.457835in}%
\pgfsys@useobject{currentmarker}{}%
\end{pgfscope}%
\begin{pgfscope}%
\pgfsys@transformshift{7.355461in}{4.134956in}%
\pgfsys@useobject{currentmarker}{}%
\end{pgfscope}%
\begin{pgfscope}%
\pgfsys@transformshift{7.505602in}{3.785661in}%
\pgfsys@useobject{currentmarker}{}%
\end{pgfscope}%
\begin{pgfscope}%
\pgfsys@transformshift{7.655743in}{3.486397in}%
\pgfsys@useobject{currentmarker}{}%
\end{pgfscope}%
\end{pgfscope}%
\begin{pgfscope}%
\pgfsetbuttcap%
\pgfsetroundjoin%
\definecolor{currentfill}{rgb}{0.000000,0.000000,0.000000}%
\pgfsetfillcolor{currentfill}%
\pgfsetlinewidth{0.803000pt}%
\definecolor{currentstroke}{rgb}{0.000000,0.000000,0.000000}%
\pgfsetstrokecolor{currentstroke}%
\pgfsetdash{}{0pt}%
\pgfsys@defobject{currentmarker}{\pgfqpoint{0.000000in}{-0.048611in}}{\pgfqpoint{0.000000in}{0.000000in}}{%
\pgfpathmoveto{\pgfqpoint{0.000000in}{0.000000in}}%
\pgfpathlineto{\pgfqpoint{0.000000in}{-0.048611in}}%
\pgfusepath{stroke,fill}%
}%
\begin{pgfscope}%
\pgfsys@transformshift{4.502784in}{3.312222in}%
\pgfsys@useobject{currentmarker}{}%
\end{pgfscope}%
\end{pgfscope}%
\begin{pgfscope}%
\definecolor{textcolor}{rgb}{0.000000,0.000000,0.000000}%
\pgfsetstrokecolor{textcolor}%
\pgfsetfillcolor{textcolor}%
\pgftext[x=4.502784in,y=3.215000in,,top]{\color{textcolor}\sffamily\fontsize{10.000000}{12.000000}\selectfont 4.30}%
\end{pgfscope}%
\begin{pgfscope}%
\pgfsetbuttcap%
\pgfsetroundjoin%
\definecolor{currentfill}{rgb}{0.000000,0.000000,0.000000}%
\pgfsetfillcolor{currentfill}%
\pgfsetlinewidth{0.803000pt}%
\definecolor{currentstroke}{rgb}{0.000000,0.000000,0.000000}%
\pgfsetstrokecolor{currentstroke}%
\pgfsetdash{}{0pt}%
\pgfsys@defobject{currentmarker}{\pgfqpoint{0.000000in}{-0.048611in}}{\pgfqpoint{0.000000in}{0.000000in}}{%
\pgfpathmoveto{\pgfqpoint{0.000000in}{0.000000in}}%
\pgfpathlineto{\pgfqpoint{0.000000in}{-0.048611in}}%
\pgfusepath{stroke,fill}%
}%
\begin{pgfscope}%
\pgfsys@transformshift{5.253488in}{3.312222in}%
\pgfsys@useobject{currentmarker}{}%
\end{pgfscope}%
\end{pgfscope}%
\begin{pgfscope}%
\definecolor{textcolor}{rgb}{0.000000,0.000000,0.000000}%
\pgfsetstrokecolor{textcolor}%
\pgfsetfillcolor{textcolor}%
\pgftext[x=5.253488in,y=3.215000in,,top]{\color{textcolor}\sffamily\fontsize{10.000000}{12.000000}\selectfont 4.35}%
\end{pgfscope}%
\begin{pgfscope}%
\pgfsetbuttcap%
\pgfsetroundjoin%
\definecolor{currentfill}{rgb}{0.000000,0.000000,0.000000}%
\pgfsetfillcolor{currentfill}%
\pgfsetlinewidth{0.803000pt}%
\definecolor{currentstroke}{rgb}{0.000000,0.000000,0.000000}%
\pgfsetstrokecolor{currentstroke}%
\pgfsetdash{}{0pt}%
\pgfsys@defobject{currentmarker}{\pgfqpoint{0.000000in}{-0.048611in}}{\pgfqpoint{0.000000in}{0.000000in}}{%
\pgfpathmoveto{\pgfqpoint{0.000000in}{0.000000in}}%
\pgfpathlineto{\pgfqpoint{0.000000in}{-0.048611in}}%
\pgfusepath{stroke,fill}%
}%
\begin{pgfscope}%
\pgfsys@transformshift{6.004193in}{3.312222in}%
\pgfsys@useobject{currentmarker}{}%
\end{pgfscope}%
\end{pgfscope}%
\begin{pgfscope}%
\definecolor{textcolor}{rgb}{0.000000,0.000000,0.000000}%
\pgfsetstrokecolor{textcolor}%
\pgfsetfillcolor{textcolor}%
\pgftext[x=6.004193in,y=3.215000in,,top]{\color{textcolor}\sffamily\fontsize{10.000000}{12.000000}\selectfont 4.40}%
\end{pgfscope}%
\begin{pgfscope}%
\pgfsetbuttcap%
\pgfsetroundjoin%
\definecolor{currentfill}{rgb}{0.000000,0.000000,0.000000}%
\pgfsetfillcolor{currentfill}%
\pgfsetlinewidth{0.803000pt}%
\definecolor{currentstroke}{rgb}{0.000000,0.000000,0.000000}%
\pgfsetstrokecolor{currentstroke}%
\pgfsetdash{}{0pt}%
\pgfsys@defobject{currentmarker}{\pgfqpoint{0.000000in}{-0.048611in}}{\pgfqpoint{0.000000in}{0.000000in}}{%
\pgfpathmoveto{\pgfqpoint{0.000000in}{0.000000in}}%
\pgfpathlineto{\pgfqpoint{0.000000in}{-0.048611in}}%
\pgfusepath{stroke,fill}%
}%
\begin{pgfscope}%
\pgfsys@transformshift{6.754897in}{3.312222in}%
\pgfsys@useobject{currentmarker}{}%
\end{pgfscope}%
\end{pgfscope}%
\begin{pgfscope}%
\definecolor{textcolor}{rgb}{0.000000,0.000000,0.000000}%
\pgfsetstrokecolor{textcolor}%
\pgfsetfillcolor{textcolor}%
\pgftext[x=6.754897in,y=3.215000in,,top]{\color{textcolor}\sffamily\fontsize{10.000000}{12.000000}\selectfont 4.45}%
\end{pgfscope}%
\begin{pgfscope}%
\pgfsetbuttcap%
\pgfsetroundjoin%
\definecolor{currentfill}{rgb}{0.000000,0.000000,0.000000}%
\pgfsetfillcolor{currentfill}%
\pgfsetlinewidth{0.803000pt}%
\definecolor{currentstroke}{rgb}{0.000000,0.000000,0.000000}%
\pgfsetstrokecolor{currentstroke}%
\pgfsetdash{}{0pt}%
\pgfsys@defobject{currentmarker}{\pgfqpoint{0.000000in}{-0.048611in}}{\pgfqpoint{0.000000in}{0.000000in}}{%
\pgfpathmoveto{\pgfqpoint{0.000000in}{0.000000in}}%
\pgfpathlineto{\pgfqpoint{0.000000in}{-0.048611in}}%
\pgfusepath{stroke,fill}%
}%
\begin{pgfscope}%
\pgfsys@transformshift{7.505602in}{3.312222in}%
\pgfsys@useobject{currentmarker}{}%
\end{pgfscope}%
\end{pgfscope}%
\begin{pgfscope}%
\definecolor{textcolor}{rgb}{0.000000,0.000000,0.000000}%
\pgfsetstrokecolor{textcolor}%
\pgfsetfillcolor{textcolor}%
\pgftext[x=7.505602in,y=3.215000in,,top]{\color{textcolor}\sffamily\fontsize{10.000000}{12.000000}\selectfont 4.50}%
\end{pgfscope}%
\begin{pgfscope}%
\pgfsetbuttcap%
\pgfsetroundjoin%
\definecolor{currentfill}{rgb}{0.000000,0.000000,0.000000}%
\pgfsetfillcolor{currentfill}%
\pgfsetlinewidth{0.803000pt}%
\definecolor{currentstroke}{rgb}{0.000000,0.000000,0.000000}%
\pgfsetstrokecolor{currentstroke}%
\pgfsetdash{}{0pt}%
\pgfsys@defobject{currentmarker}{\pgfqpoint{-0.048611in}{0.000000in}}{\pgfqpoint{0.000000in}{0.000000in}}{%
\pgfpathmoveto{\pgfqpoint{0.000000in}{0.000000in}}%
\pgfpathlineto{\pgfqpoint{-0.048611in}{0.000000in}}%
\pgfusepath{stroke,fill}%
}%
\begin{pgfscope}%
\pgfsys@transformshift{4.493667in}{3.314775in}%
\pgfsys@useobject{currentmarker}{}%
\end{pgfscope}%
\end{pgfscope}%
\begin{pgfscope}%
\definecolor{textcolor}{rgb}{0.000000,0.000000,0.000000}%
\pgfsetstrokecolor{textcolor}%
\pgfsetfillcolor{textcolor}%
\pgftext[x=4.175565in,y=3.262014in,left,base]{\color{textcolor}\sffamily\fontsize{10.000000}{12.000000}\selectfont 1.6}%
\end{pgfscope}%
\begin{pgfscope}%
\pgfsetbuttcap%
\pgfsetroundjoin%
\definecolor{currentfill}{rgb}{0.000000,0.000000,0.000000}%
\pgfsetfillcolor{currentfill}%
\pgfsetlinewidth{0.803000pt}%
\definecolor{currentstroke}{rgb}{0.000000,0.000000,0.000000}%
\pgfsetstrokecolor{currentstroke}%
\pgfsetdash{}{0pt}%
\pgfsys@defobject{currentmarker}{\pgfqpoint{-0.048611in}{0.000000in}}{\pgfqpoint{0.000000in}{0.000000in}}{%
\pgfpathmoveto{\pgfqpoint{0.000000in}{0.000000in}}%
\pgfpathlineto{\pgfqpoint{-0.048611in}{0.000000in}}%
\pgfusepath{stroke,fill}%
}%
\begin{pgfscope}%
\pgfsys@transformshift{4.493667in}{3.693134in}%
\pgfsys@useobject{currentmarker}{}%
\end{pgfscope}%
\end{pgfscope}%
\begin{pgfscope}%
\definecolor{textcolor}{rgb}{0.000000,0.000000,0.000000}%
\pgfsetstrokecolor{textcolor}%
\pgfsetfillcolor{textcolor}%
\pgftext[x=4.175565in,y=3.640373in,left,base]{\color{textcolor}\sffamily\fontsize{10.000000}{12.000000}\selectfont 1.7}%
\end{pgfscope}%
\begin{pgfscope}%
\pgfsetbuttcap%
\pgfsetroundjoin%
\definecolor{currentfill}{rgb}{0.000000,0.000000,0.000000}%
\pgfsetfillcolor{currentfill}%
\pgfsetlinewidth{0.803000pt}%
\definecolor{currentstroke}{rgb}{0.000000,0.000000,0.000000}%
\pgfsetstrokecolor{currentstroke}%
\pgfsetdash{}{0pt}%
\pgfsys@defobject{currentmarker}{\pgfqpoint{-0.048611in}{0.000000in}}{\pgfqpoint{0.000000in}{0.000000in}}{%
\pgfpathmoveto{\pgfqpoint{0.000000in}{0.000000in}}%
\pgfpathlineto{\pgfqpoint{-0.048611in}{0.000000in}}%
\pgfusepath{stroke,fill}%
}%
\begin{pgfscope}%
\pgfsys@transformshift{4.493667in}{4.071494in}%
\pgfsys@useobject{currentmarker}{}%
\end{pgfscope}%
\end{pgfscope}%
\begin{pgfscope}%
\definecolor{textcolor}{rgb}{0.000000,0.000000,0.000000}%
\pgfsetstrokecolor{textcolor}%
\pgfsetfillcolor{textcolor}%
\pgftext[x=4.175565in,y=4.018732in,left,base]{\color{textcolor}\sffamily\fontsize{10.000000}{12.000000}\selectfont 1.8}%
\end{pgfscope}%
\begin{pgfscope}%
\pgfsetbuttcap%
\pgfsetroundjoin%
\definecolor{currentfill}{rgb}{0.000000,0.000000,0.000000}%
\pgfsetfillcolor{currentfill}%
\pgfsetlinewidth{0.803000pt}%
\definecolor{currentstroke}{rgb}{0.000000,0.000000,0.000000}%
\pgfsetstrokecolor{currentstroke}%
\pgfsetdash{}{0pt}%
\pgfsys@defobject{currentmarker}{\pgfqpoint{-0.048611in}{0.000000in}}{\pgfqpoint{0.000000in}{0.000000in}}{%
\pgfpathmoveto{\pgfqpoint{0.000000in}{0.000000in}}%
\pgfpathlineto{\pgfqpoint{-0.048611in}{0.000000in}}%
\pgfusepath{stroke,fill}%
}%
\begin{pgfscope}%
\pgfsys@transformshift{4.493667in}{4.449853in}%
\pgfsys@useobject{currentmarker}{}%
\end{pgfscope}%
\end{pgfscope}%
\begin{pgfscope}%
\definecolor{textcolor}{rgb}{0.000000,0.000000,0.000000}%
\pgfsetstrokecolor{textcolor}%
\pgfsetfillcolor{textcolor}%
\pgftext[x=4.175565in,y=4.397091in,left,base]{\color{textcolor}\sffamily\fontsize{10.000000}{12.000000}\selectfont 1.9}%
\end{pgfscope}%
\begin{pgfscope}%
\pgfsetbuttcap%
\pgfsetroundjoin%
\definecolor{currentfill}{rgb}{0.000000,0.000000,0.000000}%
\pgfsetfillcolor{currentfill}%
\pgfsetlinewidth{0.803000pt}%
\definecolor{currentstroke}{rgb}{0.000000,0.000000,0.000000}%
\pgfsetstrokecolor{currentstroke}%
\pgfsetdash{}{0pt}%
\pgfsys@defobject{currentmarker}{\pgfqpoint{-0.048611in}{0.000000in}}{\pgfqpoint{0.000000in}{0.000000in}}{%
\pgfpathmoveto{\pgfqpoint{0.000000in}{0.000000in}}%
\pgfpathlineto{\pgfqpoint{-0.048611in}{0.000000in}}%
\pgfusepath{stroke,fill}%
}%
\begin{pgfscope}%
\pgfsys@transformshift{4.493667in}{4.828212in}%
\pgfsys@useobject{currentmarker}{}%
\end{pgfscope}%
\end{pgfscope}%
\begin{pgfscope}%
\definecolor{textcolor}{rgb}{0.000000,0.000000,0.000000}%
\pgfsetstrokecolor{textcolor}%
\pgfsetfillcolor{textcolor}%
\pgftext[x=4.175565in,y=4.775451in,left,base]{\color{textcolor}\sffamily\fontsize{10.000000}{12.000000}\selectfont 2.0}%
\end{pgfscope}%
\begin{pgfscope}%
\pgfsetbuttcap%
\pgfsetroundjoin%
\definecolor{currentfill}{rgb}{0.000000,0.000000,0.000000}%
\pgfsetfillcolor{currentfill}%
\pgfsetlinewidth{0.803000pt}%
\definecolor{currentstroke}{rgb}{0.000000,0.000000,0.000000}%
\pgfsetstrokecolor{currentstroke}%
\pgfsetdash{}{0pt}%
\pgfsys@defobject{currentmarker}{\pgfqpoint{-0.048611in}{0.000000in}}{\pgfqpoint{0.000000in}{0.000000in}}{%
\pgfpathmoveto{\pgfqpoint{0.000000in}{0.000000in}}%
\pgfpathlineto{\pgfqpoint{-0.048611in}{0.000000in}}%
\pgfusepath{stroke,fill}%
}%
\begin{pgfscope}%
\pgfsys@transformshift{4.493667in}{5.206571in}%
\pgfsys@useobject{currentmarker}{}%
\end{pgfscope}%
\end{pgfscope}%
\begin{pgfscope}%
\definecolor{textcolor}{rgb}{0.000000,0.000000,0.000000}%
\pgfsetstrokecolor{textcolor}%
\pgfsetfillcolor{textcolor}%
\pgftext[x=4.175565in,y=5.153810in,left,base]{\color{textcolor}\sffamily\fontsize{10.000000}{12.000000}\selectfont 2.1}%
\end{pgfscope}%
\begin{pgfscope}%
\pgfsetbuttcap%
\pgfsetroundjoin%
\definecolor{currentfill}{rgb}{0.000000,0.000000,0.000000}%
\pgfsetfillcolor{currentfill}%
\pgfsetlinewidth{0.803000pt}%
\definecolor{currentstroke}{rgb}{0.000000,0.000000,0.000000}%
\pgfsetstrokecolor{currentstroke}%
\pgfsetdash{}{0pt}%
\pgfsys@defobject{currentmarker}{\pgfqpoint{-0.048611in}{0.000000in}}{\pgfqpoint{0.000000in}{0.000000in}}{%
\pgfpathmoveto{\pgfqpoint{0.000000in}{0.000000in}}%
\pgfpathlineto{\pgfqpoint{-0.048611in}{0.000000in}}%
\pgfusepath{stroke,fill}%
}%
\begin{pgfscope}%
\pgfsys@transformshift{4.493667in}{5.584931in}%
\pgfsys@useobject{currentmarker}{}%
\end{pgfscope}%
\end{pgfscope}%
\begin{pgfscope}%
\definecolor{textcolor}{rgb}{0.000000,0.000000,0.000000}%
\pgfsetstrokecolor{textcolor}%
\pgfsetfillcolor{textcolor}%
\pgftext[x=4.175565in,y=5.532169in,left,base]{\color{textcolor}\sffamily\fontsize{10.000000}{12.000000}\selectfont 2.2}%
\end{pgfscope}%
\begin{pgfscope}%
\pgfpathrectangle{\pgfqpoint{4.493667in}{3.312222in}}{\pgfqpoint{3.321333in}{2.324444in}}%
\pgfusepath{clip}%
\pgfsetrectcap%
\pgfsetroundjoin%
\pgfsetlinewidth{1.505625pt}%
\definecolor{currentstroke}{rgb}{0.121569,0.466667,0.705882}%
\pgfsetstrokecolor{currentstroke}%
\pgfsetdash{}{0pt}%
\pgfpathmoveto{\pgfqpoint{4.652924in}{3.981108in}}%
\pgfpathlineto{\pgfqpoint{4.803065in}{4.182284in}}%
\pgfpathlineto{\pgfqpoint{4.953206in}{4.442026in}}%
\pgfpathlineto{\pgfqpoint{5.103347in}{4.386072in}}%
\pgfpathlineto{\pgfqpoint{5.253488in}{4.503337in}}%
\pgfpathlineto{\pgfqpoint{5.403629in}{4.808765in}}%
\pgfpathlineto{\pgfqpoint{5.553770in}{4.829581in}}%
\pgfpathlineto{\pgfqpoint{5.703911in}{4.943231in}}%
\pgfpathlineto{\pgfqpoint{5.854052in}{5.166476in}}%
\pgfpathlineto{\pgfqpoint{6.004193in}{5.053216in}}%
\pgfpathlineto{\pgfqpoint{6.154333in}{5.306505in}}%
\pgfpathlineto{\pgfqpoint{6.304474in}{5.266757in}}%
\pgfpathlineto{\pgfqpoint{6.454615in}{5.374036in}}%
\pgfpathlineto{\pgfqpoint{6.604756in}{5.134135in}}%
\pgfpathlineto{\pgfqpoint{6.754897in}{5.055146in}}%
\pgfpathlineto{\pgfqpoint{6.905038in}{4.886289in}}%
\pgfpathlineto{\pgfqpoint{7.055179in}{4.689739in}}%
\pgfpathlineto{\pgfqpoint{7.205320in}{4.457835in}}%
\pgfpathlineto{\pgfqpoint{7.355461in}{4.134956in}}%
\pgfpathlineto{\pgfqpoint{7.505602in}{3.785661in}}%
\pgfpathlineto{\pgfqpoint{7.655743in}{3.486397in}}%
\pgfusepath{stroke}%
\end{pgfscope}%
\begin{pgfscope}%
\pgfsetrectcap%
\pgfsetmiterjoin%
\pgfsetlinewidth{0.803000pt}%
\definecolor{currentstroke}{rgb}{0.000000,0.000000,0.000000}%
\pgfsetstrokecolor{currentstroke}%
\pgfsetdash{}{0pt}%
\pgfpathmoveto{\pgfqpoint{4.493667in}{3.312222in}}%
\pgfpathlineto{\pgfqpoint{4.493667in}{5.636667in}}%
\pgfusepath{stroke}%
\end{pgfscope}%
\begin{pgfscope}%
\pgfsetrectcap%
\pgfsetmiterjoin%
\pgfsetlinewidth{0.803000pt}%
\definecolor{currentstroke}{rgb}{0.000000,0.000000,0.000000}%
\pgfsetstrokecolor{currentstroke}%
\pgfsetdash{}{0pt}%
\pgfpathmoveto{\pgfqpoint{7.815000in}{3.312222in}}%
\pgfpathlineto{\pgfqpoint{7.815000in}{5.636667in}}%
\pgfusepath{stroke}%
\end{pgfscope}%
\begin{pgfscope}%
\pgfsetrectcap%
\pgfsetmiterjoin%
\pgfsetlinewidth{0.803000pt}%
\definecolor{currentstroke}{rgb}{0.000000,0.000000,0.000000}%
\pgfsetstrokecolor{currentstroke}%
\pgfsetdash{}{0pt}%
\pgfpathmoveto{\pgfqpoint{4.493667in}{3.312222in}}%
\pgfpathlineto{\pgfqpoint{7.815000in}{3.312222in}}%
\pgfusepath{stroke}%
\end{pgfscope}%
\begin{pgfscope}%
\pgfsetrectcap%
\pgfsetmiterjoin%
\pgfsetlinewidth{0.803000pt}%
\definecolor{currentstroke}{rgb}{0.000000,0.000000,0.000000}%
\pgfsetstrokecolor{currentstroke}%
\pgfsetdash{}{0pt}%
\pgfpathmoveto{\pgfqpoint{4.493667in}{5.636667in}}%
\pgfpathlineto{\pgfqpoint{7.815000in}{5.636667in}}%
\pgfusepath{stroke}%
\end{pgfscope}%
\begin{pgfscope}%
\definecolor{textcolor}{rgb}{0.000000,0.000000,0.000000}%
\pgfsetstrokecolor{textcolor}%
\pgfsetfillcolor{textcolor}%
\pgftext[x=6.154333in,y=5.720000in,,base]{\color{textcolor}\sffamily\fontsize{12.000000}{14.400000}\selectfont \(\displaystyle  N = 12 \)}%
\end{pgfscope}%
\begin{pgfscope}%
\pgfsetbuttcap%
\pgfsetmiterjoin%
\definecolor{currentfill}{rgb}{1.000000,1.000000,1.000000}%
\pgfsetfillcolor{currentfill}%
\pgfsetlinewidth{0.000000pt}%
\definecolor{currentstroke}{rgb}{0.000000,0.000000,0.000000}%
\pgfsetstrokecolor{currentstroke}%
\pgfsetstrokeopacity{0.000000}%
\pgfsetdash{}{0pt}%
\pgfpathmoveto{\pgfqpoint{0.557222in}{0.387222in}}%
\pgfpathlineto{\pgfqpoint{3.878555in}{0.387222in}}%
\pgfpathlineto{\pgfqpoint{3.878555in}{2.711667in}}%
\pgfpathlineto{\pgfqpoint{0.557222in}{2.711667in}}%
\pgfpathclose%
\pgfusepath{fill}%
\end{pgfscope}%
\begin{pgfscope}%
\pgfpathrectangle{\pgfqpoint{0.557222in}{0.387222in}}{\pgfqpoint{3.321333in}{2.324444in}}%
\pgfusepath{clip}%
\pgfsetbuttcap%
\pgfsetroundjoin%
\definecolor{currentfill}{rgb}{0.121569,0.466667,0.705882}%
\pgfsetfillcolor{currentfill}%
\pgfsetfillopacity{0.300000}%
\pgfsetlinewidth{1.003750pt}%
\definecolor{currentstroke}{rgb}{0.121569,0.466667,0.705882}%
\pgfsetstrokecolor{currentstroke}%
\pgfsetstrokeopacity{0.300000}%
\pgfsetdash{}{0pt}%
\pgfpathmoveto{\pgfqpoint{0.716480in}{1.646430in}}%
\pgfpathlineto{\pgfqpoint{0.716480in}{1.438942in}}%
\pgfpathlineto{\pgfqpoint{0.866621in}{1.501684in}}%
\pgfpathlineto{\pgfqpoint{1.016761in}{1.495716in}}%
\pgfpathlineto{\pgfqpoint{1.166902in}{1.652535in}}%
\pgfpathlineto{\pgfqpoint{1.317043in}{1.745961in}}%
\pgfpathlineto{\pgfqpoint{1.467184in}{1.917216in}}%
\pgfpathlineto{\pgfqpoint{1.617325in}{2.111955in}}%
\pgfpathlineto{\pgfqpoint{1.767466in}{1.903064in}}%
\pgfpathlineto{\pgfqpoint{1.917607in}{2.096764in}}%
\pgfpathlineto{\pgfqpoint{2.067748in}{2.093664in}}%
\pgfpathlineto{\pgfqpoint{2.217889in}{2.352322in}}%
\pgfpathlineto{\pgfqpoint{2.368030in}{2.325127in}}%
\pgfpathlineto{\pgfqpoint{2.518171in}{2.360365in}}%
\pgfpathlineto{\pgfqpoint{2.668311in}{2.245398in}}%
\pgfpathlineto{\pgfqpoint{2.818452in}{2.035883in}}%
\pgfpathlineto{\pgfqpoint{2.968593in}{1.897238in}}%
\pgfpathlineto{\pgfqpoint{3.118734in}{1.675899in}}%
\pgfpathlineto{\pgfqpoint{3.268875in}{1.409605in}}%
\pgfpathlineto{\pgfqpoint{3.419016in}{1.028249in}}%
\pgfpathlineto{\pgfqpoint{3.569157in}{0.644246in}}%
\pgfpathlineto{\pgfqpoint{3.719298in}{0.492879in}}%
\pgfpathlineto{\pgfqpoint{3.719298in}{0.718029in}}%
\pgfpathlineto{\pgfqpoint{3.719298in}{0.718029in}}%
\pgfpathlineto{\pgfqpoint{3.569157in}{1.002063in}}%
\pgfpathlineto{\pgfqpoint{3.419016in}{1.363585in}}%
\pgfpathlineto{\pgfqpoint{3.268875in}{1.553831in}}%
\pgfpathlineto{\pgfqpoint{3.118734in}{1.886544in}}%
\pgfpathlineto{\pgfqpoint{2.968593in}{2.126970in}}%
\pgfpathlineto{\pgfqpoint{2.818452in}{2.340972in}}%
\pgfpathlineto{\pgfqpoint{2.668311in}{2.506038in}}%
\pgfpathlineto{\pgfqpoint{2.518171in}{2.549044in}}%
\pgfpathlineto{\pgfqpoint{2.368030in}{2.606010in}}%
\pgfpathlineto{\pgfqpoint{2.217889in}{2.531533in}}%
\pgfpathlineto{\pgfqpoint{2.067748in}{2.505356in}}%
\pgfpathlineto{\pgfqpoint{1.917607in}{2.297037in}}%
\pgfpathlineto{\pgfqpoint{1.767466in}{2.184593in}}%
\pgfpathlineto{\pgfqpoint{1.617325in}{2.265799in}}%
\pgfpathlineto{\pgfqpoint{1.467184in}{2.093140in}}%
\pgfpathlineto{\pgfqpoint{1.317043in}{1.994826in}}%
\pgfpathlineto{\pgfqpoint{1.166902in}{1.767629in}}%
\pgfpathlineto{\pgfqpoint{1.016761in}{1.790397in}}%
\pgfpathlineto{\pgfqpoint{0.866621in}{1.773855in}}%
\pgfpathlineto{\pgfqpoint{0.716480in}{1.646430in}}%
\pgfpathclose%
\pgfusepath{stroke,fill}%
\end{pgfscope}%
\begin{pgfscope}%
\pgfpathrectangle{\pgfqpoint{0.557222in}{0.387222in}}{\pgfqpoint{3.321333in}{2.324444in}}%
\pgfusepath{clip}%
\pgfsetbuttcap%
\pgfsetroundjoin%
\definecolor{currentfill}{rgb}{0.121569,0.466667,0.705882}%
\pgfsetfillcolor{currentfill}%
\pgfsetlinewidth{1.003750pt}%
\definecolor{currentstroke}{rgb}{0.121569,0.466667,0.705882}%
\pgfsetstrokecolor{currentstroke}%
\pgfsetdash{}{0pt}%
\pgfsys@defobject{currentmarker}{\pgfqpoint{-0.009821in}{-0.009821in}}{\pgfqpoint{0.009821in}{0.009821in}}{%
\pgfpathmoveto{\pgfqpoint{0.000000in}{-0.009821in}}%
\pgfpathcurveto{\pgfqpoint{0.002605in}{-0.009821in}}{\pgfqpoint{0.005103in}{-0.008786in}}{\pgfqpoint{0.006944in}{-0.006944in}}%
\pgfpathcurveto{\pgfqpoint{0.008786in}{-0.005103in}}{\pgfqpoint{0.009821in}{-0.002605in}}{\pgfqpoint{0.009821in}{0.000000in}}%
\pgfpathcurveto{\pgfqpoint{0.009821in}{0.002605in}}{\pgfqpoint{0.008786in}{0.005103in}}{\pgfqpoint{0.006944in}{0.006944in}}%
\pgfpathcurveto{\pgfqpoint{0.005103in}{0.008786in}}{\pgfqpoint{0.002605in}{0.009821in}}{\pgfqpoint{0.000000in}{0.009821in}}%
\pgfpathcurveto{\pgfqpoint{-0.002605in}{0.009821in}}{\pgfqpoint{-0.005103in}{0.008786in}}{\pgfqpoint{-0.006944in}{0.006944in}}%
\pgfpathcurveto{\pgfqpoint{-0.008786in}{0.005103in}}{\pgfqpoint{-0.009821in}{0.002605in}}{\pgfqpoint{-0.009821in}{0.000000in}}%
\pgfpathcurveto{\pgfqpoint{-0.009821in}{-0.002605in}}{\pgfqpoint{-0.008786in}{-0.005103in}}{\pgfqpoint{-0.006944in}{-0.006944in}}%
\pgfpathcurveto{\pgfqpoint{-0.005103in}{-0.008786in}}{\pgfqpoint{-0.002605in}{-0.009821in}}{\pgfqpoint{0.000000in}{-0.009821in}}%
\pgfpathclose%
\pgfusepath{stroke,fill}%
}%
\begin{pgfscope}%
\pgfsys@transformshift{0.716480in}{1.542686in}%
\pgfsys@useobject{currentmarker}{}%
\end{pgfscope}%
\begin{pgfscope}%
\pgfsys@transformshift{0.866621in}{1.637770in}%
\pgfsys@useobject{currentmarker}{}%
\end{pgfscope}%
\begin{pgfscope}%
\pgfsys@transformshift{1.016761in}{1.643057in}%
\pgfsys@useobject{currentmarker}{}%
\end{pgfscope}%
\begin{pgfscope}%
\pgfsys@transformshift{1.166902in}{1.710082in}%
\pgfsys@useobject{currentmarker}{}%
\end{pgfscope}%
\begin{pgfscope}%
\pgfsys@transformshift{1.317043in}{1.870393in}%
\pgfsys@useobject{currentmarker}{}%
\end{pgfscope}%
\begin{pgfscope}%
\pgfsys@transformshift{1.467184in}{2.005178in}%
\pgfsys@useobject{currentmarker}{}%
\end{pgfscope}%
\begin{pgfscope}%
\pgfsys@transformshift{1.617325in}{2.188877in}%
\pgfsys@useobject{currentmarker}{}%
\end{pgfscope}%
\begin{pgfscope}%
\pgfsys@transformshift{1.767466in}{2.043828in}%
\pgfsys@useobject{currentmarker}{}%
\end{pgfscope}%
\begin{pgfscope}%
\pgfsys@transformshift{1.917607in}{2.196901in}%
\pgfsys@useobject{currentmarker}{}%
\end{pgfscope}%
\begin{pgfscope}%
\pgfsys@transformshift{2.067748in}{2.299510in}%
\pgfsys@useobject{currentmarker}{}%
\end{pgfscope}%
\begin{pgfscope}%
\pgfsys@transformshift{2.217889in}{2.441928in}%
\pgfsys@useobject{currentmarker}{}%
\end{pgfscope}%
\begin{pgfscope}%
\pgfsys@transformshift{2.368030in}{2.465569in}%
\pgfsys@useobject{currentmarker}{}%
\end{pgfscope}%
\begin{pgfscope}%
\pgfsys@transformshift{2.518171in}{2.454705in}%
\pgfsys@useobject{currentmarker}{}%
\end{pgfscope}%
\begin{pgfscope}%
\pgfsys@transformshift{2.668311in}{2.375718in}%
\pgfsys@useobject{currentmarker}{}%
\end{pgfscope}%
\begin{pgfscope}%
\pgfsys@transformshift{2.818452in}{2.188428in}%
\pgfsys@useobject{currentmarker}{}%
\end{pgfscope}%
\begin{pgfscope}%
\pgfsys@transformshift{2.968593in}{2.012104in}%
\pgfsys@useobject{currentmarker}{}%
\end{pgfscope}%
\begin{pgfscope}%
\pgfsys@transformshift{3.118734in}{1.781221in}%
\pgfsys@useobject{currentmarker}{}%
\end{pgfscope}%
\begin{pgfscope}%
\pgfsys@transformshift{3.268875in}{1.481718in}%
\pgfsys@useobject{currentmarker}{}%
\end{pgfscope}%
\begin{pgfscope}%
\pgfsys@transformshift{3.419016in}{1.195917in}%
\pgfsys@useobject{currentmarker}{}%
\end{pgfscope}%
\begin{pgfscope}%
\pgfsys@transformshift{3.569157in}{0.823155in}%
\pgfsys@useobject{currentmarker}{}%
\end{pgfscope}%
\begin{pgfscope}%
\pgfsys@transformshift{3.719298in}{0.605454in}%
\pgfsys@useobject{currentmarker}{}%
\end{pgfscope}%
\end{pgfscope}%
\begin{pgfscope}%
\pgfsetbuttcap%
\pgfsetroundjoin%
\definecolor{currentfill}{rgb}{0.000000,0.000000,0.000000}%
\pgfsetfillcolor{currentfill}%
\pgfsetlinewidth{0.803000pt}%
\definecolor{currentstroke}{rgb}{0.000000,0.000000,0.000000}%
\pgfsetstrokecolor{currentstroke}%
\pgfsetdash{}{0pt}%
\pgfsys@defobject{currentmarker}{\pgfqpoint{0.000000in}{-0.048611in}}{\pgfqpoint{0.000000in}{0.000000in}}{%
\pgfpathmoveto{\pgfqpoint{0.000000in}{0.000000in}}%
\pgfpathlineto{\pgfqpoint{0.000000in}{-0.048611in}}%
\pgfusepath{stroke,fill}%
}%
\begin{pgfscope}%
\pgfsys@transformshift{0.866621in}{0.387222in}%
\pgfsys@useobject{currentmarker}{}%
\end{pgfscope}%
\end{pgfscope}%
\begin{pgfscope}%
\definecolor{textcolor}{rgb}{0.000000,0.000000,0.000000}%
\pgfsetstrokecolor{textcolor}%
\pgfsetfillcolor{textcolor}%
\pgftext[x=0.866621in,y=0.290000in,,top]{\color{textcolor}\sffamily\fontsize{10.000000}{12.000000}\selectfont 4.35}%
\end{pgfscope}%
\begin{pgfscope}%
\pgfsetbuttcap%
\pgfsetroundjoin%
\definecolor{currentfill}{rgb}{0.000000,0.000000,0.000000}%
\pgfsetfillcolor{currentfill}%
\pgfsetlinewidth{0.803000pt}%
\definecolor{currentstroke}{rgb}{0.000000,0.000000,0.000000}%
\pgfsetstrokecolor{currentstroke}%
\pgfsetdash{}{0pt}%
\pgfsys@defobject{currentmarker}{\pgfqpoint{0.000000in}{-0.048611in}}{\pgfqpoint{0.000000in}{0.000000in}}{%
\pgfpathmoveto{\pgfqpoint{0.000000in}{0.000000in}}%
\pgfpathlineto{\pgfqpoint{0.000000in}{-0.048611in}}%
\pgfusepath{stroke,fill}%
}%
\begin{pgfscope}%
\pgfsys@transformshift{1.617325in}{0.387222in}%
\pgfsys@useobject{currentmarker}{}%
\end{pgfscope}%
\end{pgfscope}%
\begin{pgfscope}%
\definecolor{textcolor}{rgb}{0.000000,0.000000,0.000000}%
\pgfsetstrokecolor{textcolor}%
\pgfsetfillcolor{textcolor}%
\pgftext[x=1.617325in,y=0.290000in,,top]{\color{textcolor}\sffamily\fontsize{10.000000}{12.000000}\selectfont 4.40}%
\end{pgfscope}%
\begin{pgfscope}%
\pgfsetbuttcap%
\pgfsetroundjoin%
\definecolor{currentfill}{rgb}{0.000000,0.000000,0.000000}%
\pgfsetfillcolor{currentfill}%
\pgfsetlinewidth{0.803000pt}%
\definecolor{currentstroke}{rgb}{0.000000,0.000000,0.000000}%
\pgfsetstrokecolor{currentstroke}%
\pgfsetdash{}{0pt}%
\pgfsys@defobject{currentmarker}{\pgfqpoint{0.000000in}{-0.048611in}}{\pgfqpoint{0.000000in}{0.000000in}}{%
\pgfpathmoveto{\pgfqpoint{0.000000in}{0.000000in}}%
\pgfpathlineto{\pgfqpoint{0.000000in}{-0.048611in}}%
\pgfusepath{stroke,fill}%
}%
\begin{pgfscope}%
\pgfsys@transformshift{2.368030in}{0.387222in}%
\pgfsys@useobject{currentmarker}{}%
\end{pgfscope}%
\end{pgfscope}%
\begin{pgfscope}%
\definecolor{textcolor}{rgb}{0.000000,0.000000,0.000000}%
\pgfsetstrokecolor{textcolor}%
\pgfsetfillcolor{textcolor}%
\pgftext[x=2.368030in,y=0.290000in,,top]{\color{textcolor}\sffamily\fontsize{10.000000}{12.000000}\selectfont 4.45}%
\end{pgfscope}%
\begin{pgfscope}%
\pgfsetbuttcap%
\pgfsetroundjoin%
\definecolor{currentfill}{rgb}{0.000000,0.000000,0.000000}%
\pgfsetfillcolor{currentfill}%
\pgfsetlinewidth{0.803000pt}%
\definecolor{currentstroke}{rgb}{0.000000,0.000000,0.000000}%
\pgfsetstrokecolor{currentstroke}%
\pgfsetdash{}{0pt}%
\pgfsys@defobject{currentmarker}{\pgfqpoint{0.000000in}{-0.048611in}}{\pgfqpoint{0.000000in}{0.000000in}}{%
\pgfpathmoveto{\pgfqpoint{0.000000in}{0.000000in}}%
\pgfpathlineto{\pgfqpoint{0.000000in}{-0.048611in}}%
\pgfusepath{stroke,fill}%
}%
\begin{pgfscope}%
\pgfsys@transformshift{3.118734in}{0.387222in}%
\pgfsys@useobject{currentmarker}{}%
\end{pgfscope}%
\end{pgfscope}%
\begin{pgfscope}%
\definecolor{textcolor}{rgb}{0.000000,0.000000,0.000000}%
\pgfsetstrokecolor{textcolor}%
\pgfsetfillcolor{textcolor}%
\pgftext[x=3.118734in,y=0.290000in,,top]{\color{textcolor}\sffamily\fontsize{10.000000}{12.000000}\selectfont 4.50}%
\end{pgfscope}%
\begin{pgfscope}%
\pgfsetbuttcap%
\pgfsetroundjoin%
\definecolor{currentfill}{rgb}{0.000000,0.000000,0.000000}%
\pgfsetfillcolor{currentfill}%
\pgfsetlinewidth{0.803000pt}%
\definecolor{currentstroke}{rgb}{0.000000,0.000000,0.000000}%
\pgfsetstrokecolor{currentstroke}%
\pgfsetdash{}{0pt}%
\pgfsys@defobject{currentmarker}{\pgfqpoint{0.000000in}{-0.048611in}}{\pgfqpoint{0.000000in}{0.000000in}}{%
\pgfpathmoveto{\pgfqpoint{0.000000in}{0.000000in}}%
\pgfpathlineto{\pgfqpoint{0.000000in}{-0.048611in}}%
\pgfusepath{stroke,fill}%
}%
\begin{pgfscope}%
\pgfsys@transformshift{3.869439in}{0.387222in}%
\pgfsys@useobject{currentmarker}{}%
\end{pgfscope}%
\end{pgfscope}%
\begin{pgfscope}%
\definecolor{textcolor}{rgb}{0.000000,0.000000,0.000000}%
\pgfsetstrokecolor{textcolor}%
\pgfsetfillcolor{textcolor}%
\pgftext[x=3.869439in,y=0.290000in,,top]{\color{textcolor}\sffamily\fontsize{10.000000}{12.000000}\selectfont 4.55}%
\end{pgfscope}%
\begin{pgfscope}%
\pgfsetbuttcap%
\pgfsetroundjoin%
\definecolor{currentfill}{rgb}{0.000000,0.000000,0.000000}%
\pgfsetfillcolor{currentfill}%
\pgfsetlinewidth{0.803000pt}%
\definecolor{currentstroke}{rgb}{0.000000,0.000000,0.000000}%
\pgfsetstrokecolor{currentstroke}%
\pgfsetdash{}{0pt}%
\pgfsys@defobject{currentmarker}{\pgfqpoint{-0.048611in}{0.000000in}}{\pgfqpoint{0.000000in}{0.000000in}}{%
\pgfpathmoveto{\pgfqpoint{0.000000in}{0.000000in}}%
\pgfpathlineto{\pgfqpoint{-0.048611in}{0.000000in}}%
\pgfusepath{stroke,fill}%
}%
\begin{pgfscope}%
\pgfsys@transformshift{0.557222in}{0.627205in}%
\pgfsys@useobject{currentmarker}{}%
\end{pgfscope}%
\end{pgfscope}%
\begin{pgfscope}%
\definecolor{textcolor}{rgb}{0.000000,0.000000,0.000000}%
\pgfsetstrokecolor{textcolor}%
\pgfsetfillcolor{textcolor}%
\pgftext[x=0.239121in,y=0.574443in,left,base]{\color{textcolor}\sffamily\fontsize{10.000000}{12.000000}\selectfont 1.4}%
\end{pgfscope}%
\begin{pgfscope}%
\pgfsetbuttcap%
\pgfsetroundjoin%
\definecolor{currentfill}{rgb}{0.000000,0.000000,0.000000}%
\pgfsetfillcolor{currentfill}%
\pgfsetlinewidth{0.803000pt}%
\definecolor{currentstroke}{rgb}{0.000000,0.000000,0.000000}%
\pgfsetstrokecolor{currentstroke}%
\pgfsetdash{}{0pt}%
\pgfsys@defobject{currentmarker}{\pgfqpoint{-0.048611in}{0.000000in}}{\pgfqpoint{0.000000in}{0.000000in}}{%
\pgfpathmoveto{\pgfqpoint{0.000000in}{0.000000in}}%
\pgfpathlineto{\pgfqpoint{-0.048611in}{0.000000in}}%
\pgfusepath{stroke,fill}%
}%
\begin{pgfscope}%
\pgfsys@transformshift{0.557222in}{1.020238in}%
\pgfsys@useobject{currentmarker}{}%
\end{pgfscope}%
\end{pgfscope}%
\begin{pgfscope}%
\definecolor{textcolor}{rgb}{0.000000,0.000000,0.000000}%
\pgfsetstrokecolor{textcolor}%
\pgfsetfillcolor{textcolor}%
\pgftext[x=0.239121in,y=0.967476in,left,base]{\color{textcolor}\sffamily\fontsize{10.000000}{12.000000}\selectfont 1.6}%
\end{pgfscope}%
\begin{pgfscope}%
\pgfsetbuttcap%
\pgfsetroundjoin%
\definecolor{currentfill}{rgb}{0.000000,0.000000,0.000000}%
\pgfsetfillcolor{currentfill}%
\pgfsetlinewidth{0.803000pt}%
\definecolor{currentstroke}{rgb}{0.000000,0.000000,0.000000}%
\pgfsetstrokecolor{currentstroke}%
\pgfsetdash{}{0pt}%
\pgfsys@defobject{currentmarker}{\pgfqpoint{-0.048611in}{0.000000in}}{\pgfqpoint{0.000000in}{0.000000in}}{%
\pgfpathmoveto{\pgfqpoint{0.000000in}{0.000000in}}%
\pgfpathlineto{\pgfqpoint{-0.048611in}{0.000000in}}%
\pgfusepath{stroke,fill}%
}%
\begin{pgfscope}%
\pgfsys@transformshift{0.557222in}{1.413270in}%
\pgfsys@useobject{currentmarker}{}%
\end{pgfscope}%
\end{pgfscope}%
\begin{pgfscope}%
\definecolor{textcolor}{rgb}{0.000000,0.000000,0.000000}%
\pgfsetstrokecolor{textcolor}%
\pgfsetfillcolor{textcolor}%
\pgftext[x=0.239121in,y=1.360509in,left,base]{\color{textcolor}\sffamily\fontsize{10.000000}{12.000000}\selectfont 1.8}%
\end{pgfscope}%
\begin{pgfscope}%
\pgfsetbuttcap%
\pgfsetroundjoin%
\definecolor{currentfill}{rgb}{0.000000,0.000000,0.000000}%
\pgfsetfillcolor{currentfill}%
\pgfsetlinewidth{0.803000pt}%
\definecolor{currentstroke}{rgb}{0.000000,0.000000,0.000000}%
\pgfsetstrokecolor{currentstroke}%
\pgfsetdash{}{0pt}%
\pgfsys@defobject{currentmarker}{\pgfqpoint{-0.048611in}{0.000000in}}{\pgfqpoint{0.000000in}{0.000000in}}{%
\pgfpathmoveto{\pgfqpoint{0.000000in}{0.000000in}}%
\pgfpathlineto{\pgfqpoint{-0.048611in}{0.000000in}}%
\pgfusepath{stroke,fill}%
}%
\begin{pgfscope}%
\pgfsys@transformshift{0.557222in}{1.806303in}%
\pgfsys@useobject{currentmarker}{}%
\end{pgfscope}%
\end{pgfscope}%
\begin{pgfscope}%
\definecolor{textcolor}{rgb}{0.000000,0.000000,0.000000}%
\pgfsetstrokecolor{textcolor}%
\pgfsetfillcolor{textcolor}%
\pgftext[x=0.239121in,y=1.753541in,left,base]{\color{textcolor}\sffamily\fontsize{10.000000}{12.000000}\selectfont 2.0}%
\end{pgfscope}%
\begin{pgfscope}%
\pgfsetbuttcap%
\pgfsetroundjoin%
\definecolor{currentfill}{rgb}{0.000000,0.000000,0.000000}%
\pgfsetfillcolor{currentfill}%
\pgfsetlinewidth{0.803000pt}%
\definecolor{currentstroke}{rgb}{0.000000,0.000000,0.000000}%
\pgfsetstrokecolor{currentstroke}%
\pgfsetdash{}{0pt}%
\pgfsys@defobject{currentmarker}{\pgfqpoint{-0.048611in}{0.000000in}}{\pgfqpoint{0.000000in}{0.000000in}}{%
\pgfpathmoveto{\pgfqpoint{0.000000in}{0.000000in}}%
\pgfpathlineto{\pgfqpoint{-0.048611in}{0.000000in}}%
\pgfusepath{stroke,fill}%
}%
\begin{pgfscope}%
\pgfsys@transformshift{0.557222in}{2.199335in}%
\pgfsys@useobject{currentmarker}{}%
\end{pgfscope}%
\end{pgfscope}%
\begin{pgfscope}%
\definecolor{textcolor}{rgb}{0.000000,0.000000,0.000000}%
\pgfsetstrokecolor{textcolor}%
\pgfsetfillcolor{textcolor}%
\pgftext[x=0.239121in,y=2.146574in,left,base]{\color{textcolor}\sffamily\fontsize{10.000000}{12.000000}\selectfont 2.2}%
\end{pgfscope}%
\begin{pgfscope}%
\pgfsetbuttcap%
\pgfsetroundjoin%
\definecolor{currentfill}{rgb}{0.000000,0.000000,0.000000}%
\pgfsetfillcolor{currentfill}%
\pgfsetlinewidth{0.803000pt}%
\definecolor{currentstroke}{rgb}{0.000000,0.000000,0.000000}%
\pgfsetstrokecolor{currentstroke}%
\pgfsetdash{}{0pt}%
\pgfsys@defobject{currentmarker}{\pgfqpoint{-0.048611in}{0.000000in}}{\pgfqpoint{0.000000in}{0.000000in}}{%
\pgfpathmoveto{\pgfqpoint{0.000000in}{0.000000in}}%
\pgfpathlineto{\pgfqpoint{-0.048611in}{0.000000in}}%
\pgfusepath{stroke,fill}%
}%
\begin{pgfscope}%
\pgfsys@transformshift{0.557222in}{2.592368in}%
\pgfsys@useobject{currentmarker}{}%
\end{pgfscope}%
\end{pgfscope}%
\begin{pgfscope}%
\definecolor{textcolor}{rgb}{0.000000,0.000000,0.000000}%
\pgfsetstrokecolor{textcolor}%
\pgfsetfillcolor{textcolor}%
\pgftext[x=0.239121in,y=2.539607in,left,base]{\color{textcolor}\sffamily\fontsize{10.000000}{12.000000}\selectfont 2.4}%
\end{pgfscope}%
\begin{pgfscope}%
\pgfpathrectangle{\pgfqpoint{0.557222in}{0.387222in}}{\pgfqpoint{3.321333in}{2.324444in}}%
\pgfusepath{clip}%
\pgfsetrectcap%
\pgfsetroundjoin%
\pgfsetlinewidth{1.505625pt}%
\definecolor{currentstroke}{rgb}{0.121569,0.466667,0.705882}%
\pgfsetstrokecolor{currentstroke}%
\pgfsetdash{}{0pt}%
\pgfpathmoveto{\pgfqpoint{0.716480in}{1.542686in}}%
\pgfpathlineto{\pgfqpoint{0.866621in}{1.637770in}}%
\pgfpathlineto{\pgfqpoint{1.016761in}{1.643057in}}%
\pgfpathlineto{\pgfqpoint{1.166902in}{1.710082in}}%
\pgfpathlineto{\pgfqpoint{1.317043in}{1.870393in}}%
\pgfpathlineto{\pgfqpoint{1.467184in}{2.005178in}}%
\pgfpathlineto{\pgfqpoint{1.617325in}{2.188877in}}%
\pgfpathlineto{\pgfqpoint{1.767466in}{2.043828in}}%
\pgfpathlineto{\pgfqpoint{1.917607in}{2.196901in}}%
\pgfpathlineto{\pgfqpoint{2.067748in}{2.299510in}}%
\pgfpathlineto{\pgfqpoint{2.217889in}{2.441928in}}%
\pgfpathlineto{\pgfqpoint{2.368030in}{2.465569in}}%
\pgfpathlineto{\pgfqpoint{2.518171in}{2.454705in}}%
\pgfpathlineto{\pgfqpoint{2.668311in}{2.375718in}}%
\pgfpathlineto{\pgfqpoint{2.818452in}{2.188428in}}%
\pgfpathlineto{\pgfqpoint{2.968593in}{2.012104in}}%
\pgfpathlineto{\pgfqpoint{3.118734in}{1.781221in}}%
\pgfpathlineto{\pgfqpoint{3.268875in}{1.481718in}}%
\pgfpathlineto{\pgfqpoint{3.419016in}{1.195917in}}%
\pgfpathlineto{\pgfqpoint{3.569157in}{0.823155in}}%
\pgfpathlineto{\pgfqpoint{3.719298in}{0.605454in}}%
\pgfusepath{stroke}%
\end{pgfscope}%
\begin{pgfscope}%
\pgfsetrectcap%
\pgfsetmiterjoin%
\pgfsetlinewidth{0.803000pt}%
\definecolor{currentstroke}{rgb}{0.000000,0.000000,0.000000}%
\pgfsetstrokecolor{currentstroke}%
\pgfsetdash{}{0pt}%
\pgfpathmoveto{\pgfqpoint{0.557222in}{0.387222in}}%
\pgfpathlineto{\pgfqpoint{0.557222in}{2.711667in}}%
\pgfusepath{stroke}%
\end{pgfscope}%
\begin{pgfscope}%
\pgfsetrectcap%
\pgfsetmiterjoin%
\pgfsetlinewidth{0.803000pt}%
\definecolor{currentstroke}{rgb}{0.000000,0.000000,0.000000}%
\pgfsetstrokecolor{currentstroke}%
\pgfsetdash{}{0pt}%
\pgfpathmoveto{\pgfqpoint{3.878555in}{0.387222in}}%
\pgfpathlineto{\pgfqpoint{3.878555in}{2.711667in}}%
\pgfusepath{stroke}%
\end{pgfscope}%
\begin{pgfscope}%
\pgfsetrectcap%
\pgfsetmiterjoin%
\pgfsetlinewidth{0.803000pt}%
\definecolor{currentstroke}{rgb}{0.000000,0.000000,0.000000}%
\pgfsetstrokecolor{currentstroke}%
\pgfsetdash{}{0pt}%
\pgfpathmoveto{\pgfqpoint{0.557222in}{0.387222in}}%
\pgfpathlineto{\pgfqpoint{3.878555in}{0.387222in}}%
\pgfusepath{stroke}%
\end{pgfscope}%
\begin{pgfscope}%
\pgfsetrectcap%
\pgfsetmiterjoin%
\pgfsetlinewidth{0.803000pt}%
\definecolor{currentstroke}{rgb}{0.000000,0.000000,0.000000}%
\pgfsetstrokecolor{currentstroke}%
\pgfsetdash{}{0pt}%
\pgfpathmoveto{\pgfqpoint{0.557222in}{2.711667in}}%
\pgfpathlineto{\pgfqpoint{3.878555in}{2.711667in}}%
\pgfusepath{stroke}%
\end{pgfscope}%
\begin{pgfscope}%
\definecolor{textcolor}{rgb}{0.000000,0.000000,0.000000}%
\pgfsetstrokecolor{textcolor}%
\pgfsetfillcolor{textcolor}%
\pgftext[x=2.217889in,y=2.795000in,,base]{\color{textcolor}\sffamily\fontsize{12.000000}{14.400000}\selectfont \(\displaystyle  N = 16 \)}%
\end{pgfscope}%
\begin{pgfscope}%
\pgfsetbuttcap%
\pgfsetmiterjoin%
\definecolor{currentfill}{rgb}{1.000000,1.000000,1.000000}%
\pgfsetfillcolor{currentfill}%
\pgfsetlinewidth{0.000000pt}%
\definecolor{currentstroke}{rgb}{0.000000,0.000000,0.000000}%
\pgfsetstrokecolor{currentstroke}%
\pgfsetstrokeopacity{0.000000}%
\pgfsetdash{}{0pt}%
\pgfpathmoveto{\pgfqpoint{4.493667in}{0.387222in}}%
\pgfpathlineto{\pgfqpoint{7.815000in}{0.387222in}}%
\pgfpathlineto{\pgfqpoint{7.815000in}{2.711667in}}%
\pgfpathlineto{\pgfqpoint{4.493667in}{2.711667in}}%
\pgfpathclose%
\pgfusepath{fill}%
\end{pgfscope}%
\begin{pgfscope}%
\pgfpathrectangle{\pgfqpoint{4.493667in}{0.387222in}}{\pgfqpoint{3.321333in}{2.324444in}}%
\pgfusepath{clip}%
\pgfsetbuttcap%
\pgfsetroundjoin%
\definecolor{currentfill}{rgb}{0.121569,0.466667,0.705882}%
\pgfsetfillcolor{currentfill}%
\pgfsetfillopacity{0.300000}%
\pgfsetlinewidth{1.003750pt}%
\definecolor{currentstroke}{rgb}{0.121569,0.466667,0.705882}%
\pgfsetstrokecolor{currentstroke}%
\pgfsetstrokeopacity{0.300000}%
\pgfsetdash{}{0pt}%
\pgfpathmoveto{\pgfqpoint{4.652924in}{1.637828in}}%
\pgfpathlineto{\pgfqpoint{4.652924in}{1.474225in}}%
\pgfpathlineto{\pgfqpoint{4.803065in}{1.568553in}}%
\pgfpathlineto{\pgfqpoint{4.953206in}{1.621694in}}%
\pgfpathlineto{\pgfqpoint{5.103347in}{1.602059in}}%
\pgfpathlineto{\pgfqpoint{5.253488in}{1.556349in}}%
\pgfpathlineto{\pgfqpoint{5.403629in}{1.817882in}}%
\pgfpathlineto{\pgfqpoint{5.553770in}{1.759891in}}%
\pgfpathlineto{\pgfqpoint{5.703911in}{1.670415in}}%
\pgfpathlineto{\pgfqpoint{5.854052in}{1.913314in}}%
\pgfpathlineto{\pgfqpoint{6.004193in}{1.971687in}}%
\pgfpathlineto{\pgfqpoint{6.154333in}{1.877971in}}%
\pgfpathlineto{\pgfqpoint{6.304474in}{2.035620in}}%
\pgfpathlineto{\pgfqpoint{6.454615in}{1.749108in}}%
\pgfpathlineto{\pgfqpoint{6.604756in}{1.525652in}}%
\pgfpathlineto{\pgfqpoint{6.754897in}{1.224774in}}%
\pgfpathlineto{\pgfqpoint{6.905038in}{1.109751in}}%
\pgfpathlineto{\pgfqpoint{7.055179in}{0.861425in}}%
\pgfpathlineto{\pgfqpoint{7.205320in}{0.678074in}}%
\pgfpathlineto{\pgfqpoint{7.355461in}{0.624139in}}%
\pgfpathlineto{\pgfqpoint{7.505602in}{0.595311in}}%
\pgfpathlineto{\pgfqpoint{7.655743in}{0.492879in}}%
\pgfpathlineto{\pgfqpoint{7.655743in}{0.645701in}}%
\pgfpathlineto{\pgfqpoint{7.655743in}{0.645701in}}%
\pgfpathlineto{\pgfqpoint{7.505602in}{0.653598in}}%
\pgfpathlineto{\pgfqpoint{7.355461in}{0.756707in}}%
\pgfpathlineto{\pgfqpoint{7.205320in}{0.782929in}}%
\pgfpathlineto{\pgfqpoint{7.055179in}{0.995652in}}%
\pgfpathlineto{\pgfqpoint{6.905038in}{1.328929in}}%
\pgfpathlineto{\pgfqpoint{6.754897in}{1.533202in}}%
\pgfpathlineto{\pgfqpoint{6.604756in}{1.837038in}}%
\pgfpathlineto{\pgfqpoint{6.454615in}{2.144026in}}%
\pgfpathlineto{\pgfqpoint{6.304474in}{2.276200in}}%
\pgfpathlineto{\pgfqpoint{6.154333in}{2.524013in}}%
\pgfpathlineto{\pgfqpoint{6.004193in}{2.606010in}}%
\pgfpathlineto{\pgfqpoint{5.854052in}{2.303113in}}%
\pgfpathlineto{\pgfqpoint{5.703911in}{1.932022in}}%
\pgfpathlineto{\pgfqpoint{5.553770in}{2.039326in}}%
\pgfpathlineto{\pgfqpoint{5.403629in}{1.919021in}}%
\pgfpathlineto{\pgfqpoint{5.253488in}{1.846578in}}%
\pgfpathlineto{\pgfqpoint{5.103347in}{1.762105in}}%
\pgfpathlineto{\pgfqpoint{4.953206in}{1.725028in}}%
\pgfpathlineto{\pgfqpoint{4.803065in}{1.747686in}}%
\pgfpathlineto{\pgfqpoint{4.652924in}{1.637828in}}%
\pgfpathclose%
\pgfusepath{stroke,fill}%
\end{pgfscope}%
\begin{pgfscope}%
\pgfpathrectangle{\pgfqpoint{4.493667in}{0.387222in}}{\pgfqpoint{3.321333in}{2.324444in}}%
\pgfusepath{clip}%
\pgfsetbuttcap%
\pgfsetroundjoin%
\definecolor{currentfill}{rgb}{0.121569,0.466667,0.705882}%
\pgfsetfillcolor{currentfill}%
\pgfsetlinewidth{1.003750pt}%
\definecolor{currentstroke}{rgb}{0.121569,0.466667,0.705882}%
\pgfsetstrokecolor{currentstroke}%
\pgfsetdash{}{0pt}%
\pgfsys@defobject{currentmarker}{\pgfqpoint{-0.009821in}{-0.009821in}}{\pgfqpoint{0.009821in}{0.009821in}}{%
\pgfpathmoveto{\pgfqpoint{0.000000in}{-0.009821in}}%
\pgfpathcurveto{\pgfqpoint{0.002605in}{-0.009821in}}{\pgfqpoint{0.005103in}{-0.008786in}}{\pgfqpoint{0.006944in}{-0.006944in}}%
\pgfpathcurveto{\pgfqpoint{0.008786in}{-0.005103in}}{\pgfqpoint{0.009821in}{-0.002605in}}{\pgfqpoint{0.009821in}{0.000000in}}%
\pgfpathcurveto{\pgfqpoint{0.009821in}{0.002605in}}{\pgfqpoint{0.008786in}{0.005103in}}{\pgfqpoint{0.006944in}{0.006944in}}%
\pgfpathcurveto{\pgfqpoint{0.005103in}{0.008786in}}{\pgfqpoint{0.002605in}{0.009821in}}{\pgfqpoint{0.000000in}{0.009821in}}%
\pgfpathcurveto{\pgfqpoint{-0.002605in}{0.009821in}}{\pgfqpoint{-0.005103in}{0.008786in}}{\pgfqpoint{-0.006944in}{0.006944in}}%
\pgfpathcurveto{\pgfqpoint{-0.008786in}{0.005103in}}{\pgfqpoint{-0.009821in}{0.002605in}}{\pgfqpoint{-0.009821in}{0.000000in}}%
\pgfpathcurveto{\pgfqpoint{-0.009821in}{-0.002605in}}{\pgfqpoint{-0.008786in}{-0.005103in}}{\pgfqpoint{-0.006944in}{-0.006944in}}%
\pgfpathcurveto{\pgfqpoint{-0.005103in}{-0.008786in}}{\pgfqpoint{-0.002605in}{-0.009821in}}{\pgfqpoint{0.000000in}{-0.009821in}}%
\pgfpathclose%
\pgfusepath{stroke,fill}%
}%
\begin{pgfscope}%
\pgfsys@transformshift{4.652924in}{1.556027in}%
\pgfsys@useobject{currentmarker}{}%
\end{pgfscope}%
\begin{pgfscope}%
\pgfsys@transformshift{4.803065in}{1.658119in}%
\pgfsys@useobject{currentmarker}{}%
\end{pgfscope}%
\begin{pgfscope}%
\pgfsys@transformshift{4.953206in}{1.673361in}%
\pgfsys@useobject{currentmarker}{}%
\end{pgfscope}%
\begin{pgfscope}%
\pgfsys@transformshift{5.103347in}{1.682082in}%
\pgfsys@useobject{currentmarker}{}%
\end{pgfscope}%
\begin{pgfscope}%
\pgfsys@transformshift{5.253488in}{1.701464in}%
\pgfsys@useobject{currentmarker}{}%
\end{pgfscope}%
\begin{pgfscope}%
\pgfsys@transformshift{5.403629in}{1.868451in}%
\pgfsys@useobject{currentmarker}{}%
\end{pgfscope}%
\begin{pgfscope}%
\pgfsys@transformshift{5.553770in}{1.899609in}%
\pgfsys@useobject{currentmarker}{}%
\end{pgfscope}%
\begin{pgfscope}%
\pgfsys@transformshift{5.703911in}{1.801219in}%
\pgfsys@useobject{currentmarker}{}%
\end{pgfscope}%
\begin{pgfscope}%
\pgfsys@transformshift{5.854052in}{2.108213in}%
\pgfsys@useobject{currentmarker}{}%
\end{pgfscope}%
\begin{pgfscope}%
\pgfsys@transformshift{6.004193in}{2.288848in}%
\pgfsys@useobject{currentmarker}{}%
\end{pgfscope}%
\begin{pgfscope}%
\pgfsys@transformshift{6.154333in}{2.200992in}%
\pgfsys@useobject{currentmarker}{}%
\end{pgfscope}%
\begin{pgfscope}%
\pgfsys@transformshift{6.304474in}{2.155910in}%
\pgfsys@useobject{currentmarker}{}%
\end{pgfscope}%
\begin{pgfscope}%
\pgfsys@transformshift{6.454615in}{1.946567in}%
\pgfsys@useobject{currentmarker}{}%
\end{pgfscope}%
\begin{pgfscope}%
\pgfsys@transformshift{6.604756in}{1.681345in}%
\pgfsys@useobject{currentmarker}{}%
\end{pgfscope}%
\begin{pgfscope}%
\pgfsys@transformshift{6.754897in}{1.378988in}%
\pgfsys@useobject{currentmarker}{}%
\end{pgfscope}%
\begin{pgfscope}%
\pgfsys@transformshift{6.905038in}{1.219340in}%
\pgfsys@useobject{currentmarker}{}%
\end{pgfscope}%
\begin{pgfscope}%
\pgfsys@transformshift{7.055179in}{0.928539in}%
\pgfsys@useobject{currentmarker}{}%
\end{pgfscope}%
\begin{pgfscope}%
\pgfsys@transformshift{7.205320in}{0.730501in}%
\pgfsys@useobject{currentmarker}{}%
\end{pgfscope}%
\begin{pgfscope}%
\pgfsys@transformshift{7.355461in}{0.690423in}%
\pgfsys@useobject{currentmarker}{}%
\end{pgfscope}%
\begin{pgfscope}%
\pgfsys@transformshift{7.505602in}{0.624454in}%
\pgfsys@useobject{currentmarker}{}%
\end{pgfscope}%
\begin{pgfscope}%
\pgfsys@transformshift{7.655743in}{0.569290in}%
\pgfsys@useobject{currentmarker}{}%
\end{pgfscope}%
\end{pgfscope}%
\begin{pgfscope}%
\pgfsetbuttcap%
\pgfsetroundjoin%
\definecolor{currentfill}{rgb}{0.000000,0.000000,0.000000}%
\pgfsetfillcolor{currentfill}%
\pgfsetlinewidth{0.803000pt}%
\definecolor{currentstroke}{rgb}{0.000000,0.000000,0.000000}%
\pgfsetstrokecolor{currentstroke}%
\pgfsetdash{}{0pt}%
\pgfsys@defobject{currentmarker}{\pgfqpoint{0.000000in}{-0.048611in}}{\pgfqpoint{0.000000in}{0.000000in}}{%
\pgfpathmoveto{\pgfqpoint{0.000000in}{0.000000in}}%
\pgfpathlineto{\pgfqpoint{0.000000in}{-0.048611in}}%
\pgfusepath{stroke,fill}%
}%
\begin{pgfscope}%
\pgfsys@transformshift{4.953206in}{0.387222in}%
\pgfsys@useobject{currentmarker}{}%
\end{pgfscope}%
\end{pgfscope}%
\begin{pgfscope}%
\definecolor{textcolor}{rgb}{0.000000,0.000000,0.000000}%
\pgfsetstrokecolor{textcolor}%
\pgfsetfillcolor{textcolor}%
\pgftext[x=4.953206in,y=0.290000in,,top]{\color{textcolor}\sffamily\fontsize{10.000000}{12.000000}\selectfont 4.40}%
\end{pgfscope}%
\begin{pgfscope}%
\pgfsetbuttcap%
\pgfsetroundjoin%
\definecolor{currentfill}{rgb}{0.000000,0.000000,0.000000}%
\pgfsetfillcolor{currentfill}%
\pgfsetlinewidth{0.803000pt}%
\definecolor{currentstroke}{rgb}{0.000000,0.000000,0.000000}%
\pgfsetstrokecolor{currentstroke}%
\pgfsetdash{}{0pt}%
\pgfsys@defobject{currentmarker}{\pgfqpoint{0.000000in}{-0.048611in}}{\pgfqpoint{0.000000in}{0.000000in}}{%
\pgfpathmoveto{\pgfqpoint{0.000000in}{0.000000in}}%
\pgfpathlineto{\pgfqpoint{0.000000in}{-0.048611in}}%
\pgfusepath{stroke,fill}%
}%
\begin{pgfscope}%
\pgfsys@transformshift{5.703911in}{0.387222in}%
\pgfsys@useobject{currentmarker}{}%
\end{pgfscope}%
\end{pgfscope}%
\begin{pgfscope}%
\definecolor{textcolor}{rgb}{0.000000,0.000000,0.000000}%
\pgfsetstrokecolor{textcolor}%
\pgfsetfillcolor{textcolor}%
\pgftext[x=5.703911in,y=0.290000in,,top]{\color{textcolor}\sffamily\fontsize{10.000000}{12.000000}\selectfont 4.45}%
\end{pgfscope}%
\begin{pgfscope}%
\pgfsetbuttcap%
\pgfsetroundjoin%
\definecolor{currentfill}{rgb}{0.000000,0.000000,0.000000}%
\pgfsetfillcolor{currentfill}%
\pgfsetlinewidth{0.803000pt}%
\definecolor{currentstroke}{rgb}{0.000000,0.000000,0.000000}%
\pgfsetstrokecolor{currentstroke}%
\pgfsetdash{}{0pt}%
\pgfsys@defobject{currentmarker}{\pgfqpoint{0.000000in}{-0.048611in}}{\pgfqpoint{0.000000in}{0.000000in}}{%
\pgfpathmoveto{\pgfqpoint{0.000000in}{0.000000in}}%
\pgfpathlineto{\pgfqpoint{0.000000in}{-0.048611in}}%
\pgfusepath{stroke,fill}%
}%
\begin{pgfscope}%
\pgfsys@transformshift{6.454615in}{0.387222in}%
\pgfsys@useobject{currentmarker}{}%
\end{pgfscope}%
\end{pgfscope}%
\begin{pgfscope}%
\definecolor{textcolor}{rgb}{0.000000,0.000000,0.000000}%
\pgfsetstrokecolor{textcolor}%
\pgfsetfillcolor{textcolor}%
\pgftext[x=6.454615in,y=0.290000in,,top]{\color{textcolor}\sffamily\fontsize{10.000000}{12.000000}\selectfont 4.50}%
\end{pgfscope}%
\begin{pgfscope}%
\pgfsetbuttcap%
\pgfsetroundjoin%
\definecolor{currentfill}{rgb}{0.000000,0.000000,0.000000}%
\pgfsetfillcolor{currentfill}%
\pgfsetlinewidth{0.803000pt}%
\definecolor{currentstroke}{rgb}{0.000000,0.000000,0.000000}%
\pgfsetstrokecolor{currentstroke}%
\pgfsetdash{}{0pt}%
\pgfsys@defobject{currentmarker}{\pgfqpoint{0.000000in}{-0.048611in}}{\pgfqpoint{0.000000in}{0.000000in}}{%
\pgfpathmoveto{\pgfqpoint{0.000000in}{0.000000in}}%
\pgfpathlineto{\pgfqpoint{0.000000in}{-0.048611in}}%
\pgfusepath{stroke,fill}%
}%
\begin{pgfscope}%
\pgfsys@transformshift{7.205320in}{0.387222in}%
\pgfsys@useobject{currentmarker}{}%
\end{pgfscope}%
\end{pgfscope}%
\begin{pgfscope}%
\definecolor{textcolor}{rgb}{0.000000,0.000000,0.000000}%
\pgfsetstrokecolor{textcolor}%
\pgfsetfillcolor{textcolor}%
\pgftext[x=7.205320in,y=0.290000in,,top]{\color{textcolor}\sffamily\fontsize{10.000000}{12.000000}\selectfont 4.55}%
\end{pgfscope}%
\begin{pgfscope}%
\pgfsetbuttcap%
\pgfsetroundjoin%
\definecolor{currentfill}{rgb}{0.000000,0.000000,0.000000}%
\pgfsetfillcolor{currentfill}%
\pgfsetlinewidth{0.803000pt}%
\definecolor{currentstroke}{rgb}{0.000000,0.000000,0.000000}%
\pgfsetstrokecolor{currentstroke}%
\pgfsetdash{}{0pt}%
\pgfsys@defobject{currentmarker}{\pgfqpoint{-0.048611in}{0.000000in}}{\pgfqpoint{0.000000in}{0.000000in}}{%
\pgfpathmoveto{\pgfqpoint{0.000000in}{0.000000in}}%
\pgfpathlineto{\pgfqpoint{-0.048611in}{0.000000in}}%
\pgfusepath{stroke,fill}%
}%
\begin{pgfscope}%
\pgfsys@transformshift{4.493667in}{0.761223in}%
\pgfsys@useobject{currentmarker}{}%
\end{pgfscope}%
\end{pgfscope}%
\begin{pgfscope}%
\definecolor{textcolor}{rgb}{0.000000,0.000000,0.000000}%
\pgfsetstrokecolor{textcolor}%
\pgfsetfillcolor{textcolor}%
\pgftext[x=4.175565in,y=0.708461in,left,base]{\color{textcolor}\sffamily\fontsize{10.000000}{12.000000}\selectfont 1.0}%
\end{pgfscope}%
\begin{pgfscope}%
\pgfsetbuttcap%
\pgfsetroundjoin%
\definecolor{currentfill}{rgb}{0.000000,0.000000,0.000000}%
\pgfsetfillcolor{currentfill}%
\pgfsetlinewidth{0.803000pt}%
\definecolor{currentstroke}{rgb}{0.000000,0.000000,0.000000}%
\pgfsetstrokecolor{currentstroke}%
\pgfsetdash{}{0pt}%
\pgfsys@defobject{currentmarker}{\pgfqpoint{-0.048611in}{0.000000in}}{\pgfqpoint{0.000000in}{0.000000in}}{%
\pgfpathmoveto{\pgfqpoint{0.000000in}{0.000000in}}%
\pgfpathlineto{\pgfqpoint{-0.048611in}{0.000000in}}%
\pgfusepath{stroke,fill}%
}%
\begin{pgfscope}%
\pgfsys@transformshift{4.493667in}{1.167089in}%
\pgfsys@useobject{currentmarker}{}%
\end{pgfscope}%
\end{pgfscope}%
\begin{pgfscope}%
\definecolor{textcolor}{rgb}{0.000000,0.000000,0.000000}%
\pgfsetstrokecolor{textcolor}%
\pgfsetfillcolor{textcolor}%
\pgftext[x=4.175565in,y=1.114328in,left,base]{\color{textcolor}\sffamily\fontsize{10.000000}{12.000000}\selectfont 1.5}%
\end{pgfscope}%
\begin{pgfscope}%
\pgfsetbuttcap%
\pgfsetroundjoin%
\definecolor{currentfill}{rgb}{0.000000,0.000000,0.000000}%
\pgfsetfillcolor{currentfill}%
\pgfsetlinewidth{0.803000pt}%
\definecolor{currentstroke}{rgb}{0.000000,0.000000,0.000000}%
\pgfsetstrokecolor{currentstroke}%
\pgfsetdash{}{0pt}%
\pgfsys@defobject{currentmarker}{\pgfqpoint{-0.048611in}{0.000000in}}{\pgfqpoint{0.000000in}{0.000000in}}{%
\pgfpathmoveto{\pgfqpoint{0.000000in}{0.000000in}}%
\pgfpathlineto{\pgfqpoint{-0.048611in}{0.000000in}}%
\pgfusepath{stroke,fill}%
}%
\begin{pgfscope}%
\pgfsys@transformshift{4.493667in}{1.572955in}%
\pgfsys@useobject{currentmarker}{}%
\end{pgfscope}%
\end{pgfscope}%
\begin{pgfscope}%
\definecolor{textcolor}{rgb}{0.000000,0.000000,0.000000}%
\pgfsetstrokecolor{textcolor}%
\pgfsetfillcolor{textcolor}%
\pgftext[x=4.175565in,y=1.520194in,left,base]{\color{textcolor}\sffamily\fontsize{10.000000}{12.000000}\selectfont 2.0}%
\end{pgfscope}%
\begin{pgfscope}%
\pgfsetbuttcap%
\pgfsetroundjoin%
\definecolor{currentfill}{rgb}{0.000000,0.000000,0.000000}%
\pgfsetfillcolor{currentfill}%
\pgfsetlinewidth{0.803000pt}%
\definecolor{currentstroke}{rgb}{0.000000,0.000000,0.000000}%
\pgfsetstrokecolor{currentstroke}%
\pgfsetdash{}{0pt}%
\pgfsys@defobject{currentmarker}{\pgfqpoint{-0.048611in}{0.000000in}}{\pgfqpoint{0.000000in}{0.000000in}}{%
\pgfpathmoveto{\pgfqpoint{0.000000in}{0.000000in}}%
\pgfpathlineto{\pgfqpoint{-0.048611in}{0.000000in}}%
\pgfusepath{stroke,fill}%
}%
\begin{pgfscope}%
\pgfsys@transformshift{4.493667in}{1.978822in}%
\pgfsys@useobject{currentmarker}{}%
\end{pgfscope}%
\end{pgfscope}%
\begin{pgfscope}%
\definecolor{textcolor}{rgb}{0.000000,0.000000,0.000000}%
\pgfsetstrokecolor{textcolor}%
\pgfsetfillcolor{textcolor}%
\pgftext[x=4.175565in,y=1.926060in,left,base]{\color{textcolor}\sffamily\fontsize{10.000000}{12.000000}\selectfont 2.5}%
\end{pgfscope}%
\begin{pgfscope}%
\pgfsetbuttcap%
\pgfsetroundjoin%
\definecolor{currentfill}{rgb}{0.000000,0.000000,0.000000}%
\pgfsetfillcolor{currentfill}%
\pgfsetlinewidth{0.803000pt}%
\definecolor{currentstroke}{rgb}{0.000000,0.000000,0.000000}%
\pgfsetstrokecolor{currentstroke}%
\pgfsetdash{}{0pt}%
\pgfsys@defobject{currentmarker}{\pgfqpoint{-0.048611in}{0.000000in}}{\pgfqpoint{0.000000in}{0.000000in}}{%
\pgfpathmoveto{\pgfqpoint{0.000000in}{0.000000in}}%
\pgfpathlineto{\pgfqpoint{-0.048611in}{0.000000in}}%
\pgfusepath{stroke,fill}%
}%
\begin{pgfscope}%
\pgfsys@transformshift{4.493667in}{2.384688in}%
\pgfsys@useobject{currentmarker}{}%
\end{pgfscope}%
\end{pgfscope}%
\begin{pgfscope}%
\definecolor{textcolor}{rgb}{0.000000,0.000000,0.000000}%
\pgfsetstrokecolor{textcolor}%
\pgfsetfillcolor{textcolor}%
\pgftext[x=4.175565in,y=2.331926in,left,base]{\color{textcolor}\sffamily\fontsize{10.000000}{12.000000}\selectfont 3.0}%
\end{pgfscope}%
\begin{pgfscope}%
\pgfpathrectangle{\pgfqpoint{4.493667in}{0.387222in}}{\pgfqpoint{3.321333in}{2.324444in}}%
\pgfusepath{clip}%
\pgfsetrectcap%
\pgfsetroundjoin%
\pgfsetlinewidth{1.505625pt}%
\definecolor{currentstroke}{rgb}{0.121569,0.466667,0.705882}%
\pgfsetstrokecolor{currentstroke}%
\pgfsetdash{}{0pt}%
\pgfpathmoveto{\pgfqpoint{4.652924in}{1.556027in}}%
\pgfpathlineto{\pgfqpoint{4.803065in}{1.658119in}}%
\pgfpathlineto{\pgfqpoint{4.953206in}{1.673361in}}%
\pgfpathlineto{\pgfqpoint{5.103347in}{1.682082in}}%
\pgfpathlineto{\pgfqpoint{5.253488in}{1.701464in}}%
\pgfpathlineto{\pgfqpoint{5.403629in}{1.868451in}}%
\pgfpathlineto{\pgfqpoint{5.553770in}{1.899609in}}%
\pgfpathlineto{\pgfqpoint{5.703911in}{1.801219in}}%
\pgfpathlineto{\pgfqpoint{5.854052in}{2.108213in}}%
\pgfpathlineto{\pgfqpoint{6.004193in}{2.288848in}}%
\pgfpathlineto{\pgfqpoint{6.154333in}{2.200992in}}%
\pgfpathlineto{\pgfqpoint{6.304474in}{2.155910in}}%
\pgfpathlineto{\pgfqpoint{6.454615in}{1.946567in}}%
\pgfpathlineto{\pgfqpoint{6.604756in}{1.681345in}}%
\pgfpathlineto{\pgfqpoint{6.754897in}{1.378988in}}%
\pgfpathlineto{\pgfqpoint{6.905038in}{1.219340in}}%
\pgfpathlineto{\pgfqpoint{7.055179in}{0.928539in}}%
\pgfpathlineto{\pgfqpoint{7.205320in}{0.730501in}}%
\pgfpathlineto{\pgfqpoint{7.355461in}{0.690423in}}%
\pgfpathlineto{\pgfqpoint{7.505602in}{0.624454in}}%
\pgfpathlineto{\pgfqpoint{7.655743in}{0.569290in}}%
\pgfusepath{stroke}%
\end{pgfscope}%
\begin{pgfscope}%
\pgfsetrectcap%
\pgfsetmiterjoin%
\pgfsetlinewidth{0.803000pt}%
\definecolor{currentstroke}{rgb}{0.000000,0.000000,0.000000}%
\pgfsetstrokecolor{currentstroke}%
\pgfsetdash{}{0pt}%
\pgfpathmoveto{\pgfqpoint{4.493667in}{0.387222in}}%
\pgfpathlineto{\pgfqpoint{4.493667in}{2.711667in}}%
\pgfusepath{stroke}%
\end{pgfscope}%
\begin{pgfscope}%
\pgfsetrectcap%
\pgfsetmiterjoin%
\pgfsetlinewidth{0.803000pt}%
\definecolor{currentstroke}{rgb}{0.000000,0.000000,0.000000}%
\pgfsetstrokecolor{currentstroke}%
\pgfsetdash{}{0pt}%
\pgfpathmoveto{\pgfqpoint{7.815000in}{0.387222in}}%
\pgfpathlineto{\pgfqpoint{7.815000in}{2.711667in}}%
\pgfusepath{stroke}%
\end{pgfscope}%
\begin{pgfscope}%
\pgfsetrectcap%
\pgfsetmiterjoin%
\pgfsetlinewidth{0.803000pt}%
\definecolor{currentstroke}{rgb}{0.000000,0.000000,0.000000}%
\pgfsetstrokecolor{currentstroke}%
\pgfsetdash{}{0pt}%
\pgfpathmoveto{\pgfqpoint{4.493667in}{0.387222in}}%
\pgfpathlineto{\pgfqpoint{7.815000in}{0.387222in}}%
\pgfusepath{stroke}%
\end{pgfscope}%
\begin{pgfscope}%
\pgfsetrectcap%
\pgfsetmiterjoin%
\pgfsetlinewidth{0.803000pt}%
\definecolor{currentstroke}{rgb}{0.000000,0.000000,0.000000}%
\pgfsetstrokecolor{currentstroke}%
\pgfsetdash{}{0pt}%
\pgfpathmoveto{\pgfqpoint{4.493667in}{2.711667in}}%
\pgfpathlineto{\pgfqpoint{7.815000in}{2.711667in}}%
\pgfusepath{stroke}%
\end{pgfscope}%
\begin{pgfscope}%
\definecolor{textcolor}{rgb}{0.000000,0.000000,0.000000}%
\pgfsetstrokecolor{textcolor}%
\pgfsetfillcolor{textcolor}%
\pgftext[x=6.154333in,y=2.795000in,,base]{\color{textcolor}\sffamily\fontsize{12.000000}{14.400000}\selectfont \(\displaystyle  N = 24 \)}%
\end{pgfscope}%
\end{pgfpicture}%
\makeatother%
\endgroup%
}
\caption{Specific heat $c$ for finer temperature $T$ for the 3-D problem using kinetic Monte Carlo algorithm}
\label{Fig:CapBig3D}
\end{figure}

\begin{figure}[htbp]
\centering
\scalebox{0.666}{%% Creator: Matplotlib, PGF backend
%%
%% To include the figure in your LaTeX document, write
%%   \input{<filename>.pgf}
%%
%% Make sure the required packages are loaded in your preamble
%%   \usepackage{pgf}
%%
%% Figures using additional raster images can only be included by \input if
%% they are in the same directory as the main LaTeX file. For loading figures
%% from other directories you can use the `import` package
%%   \usepackage{import}
%% and then include the figures with
%%   \import{<path to file>}{<filename>.pgf}
%%
%% Matplotlib used the following preamble
%%   \usepackage{fontspec}
%%   \setmainfont{DejaVuSerif.ttf}[Path=/home/lzh/anaconda3/envs/numana/lib/python3.7/site-packages/matplotlib/mpl-data/fonts/ttf/]
%%   \setsansfont{DejaVuSans.ttf}[Path=/home/lzh/anaconda3/envs/numana/lib/python3.7/site-packages/matplotlib/mpl-data/fonts/ttf/]
%%   \setmonofont{DejaVuSansMono.ttf}[Path=/home/lzh/anaconda3/envs/numana/lib/python3.7/site-packages/matplotlib/mpl-data/fonts/ttf/]
%%
\begingroup%
\makeatletter%
\begin{pgfpicture}%
\pgfpathrectangle{\pgfpointorigin}{\pgfqpoint{8.000000in}{6.000000in}}%
\pgfusepath{use as bounding box, clip}%
\begin{pgfscope}%
\pgfsetbuttcap%
\pgfsetmiterjoin%
\definecolor{currentfill}{rgb}{1.000000,1.000000,1.000000}%
\pgfsetfillcolor{currentfill}%
\pgfsetlinewidth{0.000000pt}%
\definecolor{currentstroke}{rgb}{1.000000,1.000000,1.000000}%
\pgfsetstrokecolor{currentstroke}%
\pgfsetdash{}{0pt}%
\pgfpathmoveto{\pgfqpoint{0.000000in}{0.000000in}}%
\pgfpathlineto{\pgfqpoint{8.000000in}{0.000000in}}%
\pgfpathlineto{\pgfqpoint{8.000000in}{6.000000in}}%
\pgfpathlineto{\pgfqpoint{0.000000in}{6.000000in}}%
\pgfpathclose%
\pgfusepath{fill}%
\end{pgfscope}%
\begin{pgfscope}%
\pgfsetbuttcap%
\pgfsetmiterjoin%
\definecolor{currentfill}{rgb}{1.000000,1.000000,1.000000}%
\pgfsetfillcolor{currentfill}%
\pgfsetlinewidth{0.000000pt}%
\definecolor{currentstroke}{rgb}{0.000000,0.000000,0.000000}%
\pgfsetstrokecolor{currentstroke}%
\pgfsetstrokeopacity{0.000000}%
\pgfsetdash{}{0pt}%
\pgfpathmoveto{\pgfqpoint{0.555972in}{3.312222in}}%
\pgfpathlineto{\pgfqpoint{3.878555in}{3.312222in}}%
\pgfpathlineto{\pgfqpoint{3.878555in}{5.636667in}}%
\pgfpathlineto{\pgfqpoint{0.555972in}{5.636667in}}%
\pgfpathclose%
\pgfusepath{fill}%
\end{pgfscope}%
\begin{pgfscope}%
\pgfpathrectangle{\pgfqpoint{0.555972in}{3.312222in}}{\pgfqpoint{3.322583in}{2.324444in}}%
\pgfusepath{clip}%
\pgfsetbuttcap%
\pgfsetroundjoin%
\definecolor{currentfill}{rgb}{0.121569,0.466667,0.705882}%
\pgfsetfillcolor{currentfill}%
\pgfsetfillopacity{0.300000}%
\pgfsetlinewidth{1.003750pt}%
\definecolor{currentstroke}{rgb}{0.121569,0.466667,0.705882}%
\pgfsetstrokecolor{currentstroke}%
\pgfsetstrokeopacity{0.300000}%
\pgfsetdash{}{0pt}%
\pgfpathmoveto{\pgfqpoint{0.715290in}{5.531010in}}%
\pgfpathlineto{\pgfqpoint{0.715290in}{5.511605in}}%
\pgfpathlineto{\pgfqpoint{0.865487in}{5.440431in}}%
\pgfpathlineto{\pgfqpoint{1.015684in}{5.349840in}}%
\pgfpathlineto{\pgfqpoint{1.165882in}{5.268395in}}%
\pgfpathlineto{\pgfqpoint{1.316079in}{5.185211in}}%
\pgfpathlineto{\pgfqpoint{1.466277in}{5.083360in}}%
\pgfpathlineto{\pgfqpoint{1.616474in}{4.997688in}}%
\pgfpathlineto{\pgfqpoint{1.766671in}{4.898672in}}%
\pgfpathlineto{\pgfqpoint{1.916869in}{4.789061in}}%
\pgfpathlineto{\pgfqpoint{2.067066in}{4.698190in}}%
\pgfpathlineto{\pgfqpoint{2.217264in}{4.590857in}}%
\pgfpathlineto{\pgfqpoint{2.367461in}{4.474640in}}%
\pgfpathlineto{\pgfqpoint{2.517659in}{4.388162in}}%
\pgfpathlineto{\pgfqpoint{2.667856in}{4.259915in}}%
\pgfpathlineto{\pgfqpoint{2.818053in}{4.146677in}}%
\pgfpathlineto{\pgfqpoint{2.968251in}{4.030397in}}%
\pgfpathlineto{\pgfqpoint{3.118448in}{3.912067in}}%
\pgfpathlineto{\pgfqpoint{3.268646in}{3.777380in}}%
\pgfpathlineto{\pgfqpoint{3.418843in}{3.673431in}}%
\pgfpathlineto{\pgfqpoint{3.569040in}{3.558551in}}%
\pgfpathlineto{\pgfqpoint{3.719238in}{3.417879in}}%
\pgfpathlineto{\pgfqpoint{3.719238in}{3.450248in}}%
\pgfpathlineto{\pgfqpoint{3.719238in}{3.450248in}}%
\pgfpathlineto{\pgfqpoint{3.569040in}{3.581866in}}%
\pgfpathlineto{\pgfqpoint{3.418843in}{3.681569in}}%
\pgfpathlineto{\pgfqpoint{3.268646in}{3.828696in}}%
\pgfpathlineto{\pgfqpoint{3.118448in}{3.923040in}}%
\pgfpathlineto{\pgfqpoint{2.968251in}{4.055642in}}%
\pgfpathlineto{\pgfqpoint{2.818053in}{4.166993in}}%
\pgfpathlineto{\pgfqpoint{2.667856in}{4.288349in}}%
\pgfpathlineto{\pgfqpoint{2.517659in}{4.398945in}}%
\pgfpathlineto{\pgfqpoint{2.367461in}{4.512501in}}%
\pgfpathlineto{\pgfqpoint{2.217264in}{4.614087in}}%
\pgfpathlineto{\pgfqpoint{2.067066in}{4.726033in}}%
\pgfpathlineto{\pgfqpoint{1.916869in}{4.818645in}}%
\pgfpathlineto{\pgfqpoint{1.766671in}{4.911875in}}%
\pgfpathlineto{\pgfqpoint{1.616474in}{5.009523in}}%
\pgfpathlineto{\pgfqpoint{1.466277in}{5.104291in}}%
\pgfpathlineto{\pgfqpoint{1.316079in}{5.194185in}}%
\pgfpathlineto{\pgfqpoint{1.165882in}{5.287266in}}%
\pgfpathlineto{\pgfqpoint{1.015684in}{5.371328in}}%
\pgfpathlineto{\pgfqpoint{0.865487in}{5.450498in}}%
\pgfpathlineto{\pgfqpoint{0.715290in}{5.531010in}}%
\pgfpathclose%
\pgfusepath{stroke,fill}%
\end{pgfscope}%
\begin{pgfscope}%
\pgfpathrectangle{\pgfqpoint{0.555972in}{3.312222in}}{\pgfqpoint{3.322583in}{2.324444in}}%
\pgfusepath{clip}%
\pgfsetbuttcap%
\pgfsetroundjoin%
\definecolor{currentfill}{rgb}{0.121569,0.466667,0.705882}%
\pgfsetfillcolor{currentfill}%
\pgfsetlinewidth{1.003750pt}%
\definecolor{currentstroke}{rgb}{0.121569,0.466667,0.705882}%
\pgfsetstrokecolor{currentstroke}%
\pgfsetdash{}{0pt}%
\pgfsys@defobject{currentmarker}{\pgfqpoint{-0.009821in}{-0.009821in}}{\pgfqpoint{0.009821in}{0.009821in}}{%
\pgfpathmoveto{\pgfqpoint{0.000000in}{-0.009821in}}%
\pgfpathcurveto{\pgfqpoint{0.002605in}{-0.009821in}}{\pgfqpoint{0.005103in}{-0.008786in}}{\pgfqpoint{0.006944in}{-0.006944in}}%
\pgfpathcurveto{\pgfqpoint{0.008786in}{-0.005103in}}{\pgfqpoint{0.009821in}{-0.002605in}}{\pgfqpoint{0.009821in}{0.000000in}}%
\pgfpathcurveto{\pgfqpoint{0.009821in}{0.002605in}}{\pgfqpoint{0.008786in}{0.005103in}}{\pgfqpoint{0.006944in}{0.006944in}}%
\pgfpathcurveto{\pgfqpoint{0.005103in}{0.008786in}}{\pgfqpoint{0.002605in}{0.009821in}}{\pgfqpoint{0.000000in}{0.009821in}}%
\pgfpathcurveto{\pgfqpoint{-0.002605in}{0.009821in}}{\pgfqpoint{-0.005103in}{0.008786in}}{\pgfqpoint{-0.006944in}{0.006944in}}%
\pgfpathcurveto{\pgfqpoint{-0.008786in}{0.005103in}}{\pgfqpoint{-0.009821in}{0.002605in}}{\pgfqpoint{-0.009821in}{0.000000in}}%
\pgfpathcurveto{\pgfqpoint{-0.009821in}{-0.002605in}}{\pgfqpoint{-0.008786in}{-0.005103in}}{\pgfqpoint{-0.006944in}{-0.006944in}}%
\pgfpathcurveto{\pgfqpoint{-0.005103in}{-0.008786in}}{\pgfqpoint{-0.002605in}{-0.009821in}}{\pgfqpoint{0.000000in}{-0.009821in}}%
\pgfpathclose%
\pgfusepath{stroke,fill}%
}%
\begin{pgfscope}%
\pgfsys@transformshift{0.715290in}{5.521307in}%
\pgfsys@useobject{currentmarker}{}%
\end{pgfscope}%
\begin{pgfscope}%
\pgfsys@transformshift{0.865487in}{5.445464in}%
\pgfsys@useobject{currentmarker}{}%
\end{pgfscope}%
\begin{pgfscope}%
\pgfsys@transformshift{1.015684in}{5.360584in}%
\pgfsys@useobject{currentmarker}{}%
\end{pgfscope}%
\begin{pgfscope}%
\pgfsys@transformshift{1.165882in}{5.277830in}%
\pgfsys@useobject{currentmarker}{}%
\end{pgfscope}%
\begin{pgfscope}%
\pgfsys@transformshift{1.316079in}{5.189698in}%
\pgfsys@useobject{currentmarker}{}%
\end{pgfscope}%
\begin{pgfscope}%
\pgfsys@transformshift{1.466277in}{5.093825in}%
\pgfsys@useobject{currentmarker}{}%
\end{pgfscope}%
\begin{pgfscope}%
\pgfsys@transformshift{1.616474in}{5.003605in}%
\pgfsys@useobject{currentmarker}{}%
\end{pgfscope}%
\begin{pgfscope}%
\pgfsys@transformshift{1.766671in}{4.905274in}%
\pgfsys@useobject{currentmarker}{}%
\end{pgfscope}%
\begin{pgfscope}%
\pgfsys@transformshift{1.916869in}{4.803853in}%
\pgfsys@useobject{currentmarker}{}%
\end{pgfscope}%
\begin{pgfscope}%
\pgfsys@transformshift{2.067066in}{4.712111in}%
\pgfsys@useobject{currentmarker}{}%
\end{pgfscope}%
\begin{pgfscope}%
\pgfsys@transformshift{2.217264in}{4.602472in}%
\pgfsys@useobject{currentmarker}{}%
\end{pgfscope}%
\begin{pgfscope}%
\pgfsys@transformshift{2.367461in}{4.493570in}%
\pgfsys@useobject{currentmarker}{}%
\end{pgfscope}%
\begin{pgfscope}%
\pgfsys@transformshift{2.517659in}{4.393553in}%
\pgfsys@useobject{currentmarker}{}%
\end{pgfscope}%
\begin{pgfscope}%
\pgfsys@transformshift{2.667856in}{4.274132in}%
\pgfsys@useobject{currentmarker}{}%
\end{pgfscope}%
\begin{pgfscope}%
\pgfsys@transformshift{2.818053in}{4.156835in}%
\pgfsys@useobject{currentmarker}{}%
\end{pgfscope}%
\begin{pgfscope}%
\pgfsys@transformshift{2.968251in}{4.043020in}%
\pgfsys@useobject{currentmarker}{}%
\end{pgfscope}%
\begin{pgfscope}%
\pgfsys@transformshift{3.118448in}{3.917554in}%
\pgfsys@useobject{currentmarker}{}%
\end{pgfscope}%
\begin{pgfscope}%
\pgfsys@transformshift{3.268646in}{3.803038in}%
\pgfsys@useobject{currentmarker}{}%
\end{pgfscope}%
\begin{pgfscope}%
\pgfsys@transformshift{3.418843in}{3.677500in}%
\pgfsys@useobject{currentmarker}{}%
\end{pgfscope}%
\begin{pgfscope}%
\pgfsys@transformshift{3.569040in}{3.570209in}%
\pgfsys@useobject{currentmarker}{}%
\end{pgfscope}%
\begin{pgfscope}%
\pgfsys@transformshift{3.719238in}{3.434063in}%
\pgfsys@useobject{currentmarker}{}%
\end{pgfscope}%
\end{pgfscope}%
\begin{pgfscope}%
\pgfsetbuttcap%
\pgfsetroundjoin%
\definecolor{currentfill}{rgb}{0.000000,0.000000,0.000000}%
\pgfsetfillcolor{currentfill}%
\pgfsetlinewidth{0.803000pt}%
\definecolor{currentstroke}{rgb}{0.000000,0.000000,0.000000}%
\pgfsetstrokecolor{currentstroke}%
\pgfsetdash{}{0pt}%
\pgfsys@defobject{currentmarker}{\pgfqpoint{0.000000in}{-0.048611in}}{\pgfqpoint{0.000000in}{0.000000in}}{%
\pgfpathmoveto{\pgfqpoint{0.000000in}{0.000000in}}%
\pgfpathlineto{\pgfqpoint{0.000000in}{-0.048611in}}%
\pgfusepath{stroke,fill}%
}%
\begin{pgfscope}%
\pgfsys@transformshift{0.565092in}{3.312222in}%
\pgfsys@useobject{currentmarker}{}%
\end{pgfscope}%
\end{pgfscope}%
\begin{pgfscope}%
\definecolor{textcolor}{rgb}{0.000000,0.000000,0.000000}%
\pgfsetstrokecolor{textcolor}%
\pgfsetfillcolor{textcolor}%
\pgftext[x=0.565092in,y=3.215000in,,top]{\color{textcolor}\sffamily\fontsize{10.000000}{12.000000}\selectfont 4.20}%
\end{pgfscope}%
\begin{pgfscope}%
\pgfsetbuttcap%
\pgfsetroundjoin%
\definecolor{currentfill}{rgb}{0.000000,0.000000,0.000000}%
\pgfsetfillcolor{currentfill}%
\pgfsetlinewidth{0.803000pt}%
\definecolor{currentstroke}{rgb}{0.000000,0.000000,0.000000}%
\pgfsetstrokecolor{currentstroke}%
\pgfsetdash{}{0pt}%
\pgfsys@defobject{currentmarker}{\pgfqpoint{0.000000in}{-0.048611in}}{\pgfqpoint{0.000000in}{0.000000in}}{%
\pgfpathmoveto{\pgfqpoint{0.000000in}{0.000000in}}%
\pgfpathlineto{\pgfqpoint{0.000000in}{-0.048611in}}%
\pgfusepath{stroke,fill}%
}%
\begin{pgfscope}%
\pgfsys@transformshift{1.316079in}{3.312222in}%
\pgfsys@useobject{currentmarker}{}%
\end{pgfscope}%
\end{pgfscope}%
\begin{pgfscope}%
\definecolor{textcolor}{rgb}{0.000000,0.000000,0.000000}%
\pgfsetstrokecolor{textcolor}%
\pgfsetfillcolor{textcolor}%
\pgftext[x=1.316079in,y=3.215000in,,top]{\color{textcolor}\sffamily\fontsize{10.000000}{12.000000}\selectfont 4.25}%
\end{pgfscope}%
\begin{pgfscope}%
\pgfsetbuttcap%
\pgfsetroundjoin%
\definecolor{currentfill}{rgb}{0.000000,0.000000,0.000000}%
\pgfsetfillcolor{currentfill}%
\pgfsetlinewidth{0.803000pt}%
\definecolor{currentstroke}{rgb}{0.000000,0.000000,0.000000}%
\pgfsetstrokecolor{currentstroke}%
\pgfsetdash{}{0pt}%
\pgfsys@defobject{currentmarker}{\pgfqpoint{0.000000in}{-0.048611in}}{\pgfqpoint{0.000000in}{0.000000in}}{%
\pgfpathmoveto{\pgfqpoint{0.000000in}{0.000000in}}%
\pgfpathlineto{\pgfqpoint{0.000000in}{-0.048611in}}%
\pgfusepath{stroke,fill}%
}%
\begin{pgfscope}%
\pgfsys@transformshift{2.067066in}{3.312222in}%
\pgfsys@useobject{currentmarker}{}%
\end{pgfscope}%
\end{pgfscope}%
\begin{pgfscope}%
\definecolor{textcolor}{rgb}{0.000000,0.000000,0.000000}%
\pgfsetstrokecolor{textcolor}%
\pgfsetfillcolor{textcolor}%
\pgftext[x=2.067066in,y=3.215000in,,top]{\color{textcolor}\sffamily\fontsize{10.000000}{12.000000}\selectfont 4.30}%
\end{pgfscope}%
\begin{pgfscope}%
\pgfsetbuttcap%
\pgfsetroundjoin%
\definecolor{currentfill}{rgb}{0.000000,0.000000,0.000000}%
\pgfsetfillcolor{currentfill}%
\pgfsetlinewidth{0.803000pt}%
\definecolor{currentstroke}{rgb}{0.000000,0.000000,0.000000}%
\pgfsetstrokecolor{currentstroke}%
\pgfsetdash{}{0pt}%
\pgfsys@defobject{currentmarker}{\pgfqpoint{0.000000in}{-0.048611in}}{\pgfqpoint{0.000000in}{0.000000in}}{%
\pgfpathmoveto{\pgfqpoint{0.000000in}{0.000000in}}%
\pgfpathlineto{\pgfqpoint{0.000000in}{-0.048611in}}%
\pgfusepath{stroke,fill}%
}%
\begin{pgfscope}%
\pgfsys@transformshift{2.818053in}{3.312222in}%
\pgfsys@useobject{currentmarker}{}%
\end{pgfscope}%
\end{pgfscope}%
\begin{pgfscope}%
\definecolor{textcolor}{rgb}{0.000000,0.000000,0.000000}%
\pgfsetstrokecolor{textcolor}%
\pgfsetfillcolor{textcolor}%
\pgftext[x=2.818053in,y=3.215000in,,top]{\color{textcolor}\sffamily\fontsize{10.000000}{12.000000}\selectfont 4.35}%
\end{pgfscope}%
\begin{pgfscope}%
\pgfsetbuttcap%
\pgfsetroundjoin%
\definecolor{currentfill}{rgb}{0.000000,0.000000,0.000000}%
\pgfsetfillcolor{currentfill}%
\pgfsetlinewidth{0.803000pt}%
\definecolor{currentstroke}{rgb}{0.000000,0.000000,0.000000}%
\pgfsetstrokecolor{currentstroke}%
\pgfsetdash{}{0pt}%
\pgfsys@defobject{currentmarker}{\pgfqpoint{0.000000in}{-0.048611in}}{\pgfqpoint{0.000000in}{0.000000in}}{%
\pgfpathmoveto{\pgfqpoint{0.000000in}{0.000000in}}%
\pgfpathlineto{\pgfqpoint{0.000000in}{-0.048611in}}%
\pgfusepath{stroke,fill}%
}%
\begin{pgfscope}%
\pgfsys@transformshift{3.569040in}{3.312222in}%
\pgfsys@useobject{currentmarker}{}%
\end{pgfscope}%
\end{pgfscope}%
\begin{pgfscope}%
\definecolor{textcolor}{rgb}{0.000000,0.000000,0.000000}%
\pgfsetstrokecolor{textcolor}%
\pgfsetfillcolor{textcolor}%
\pgftext[x=3.569040in,y=3.215000in,,top]{\color{textcolor}\sffamily\fontsize{10.000000}{12.000000}\selectfont 4.40}%
\end{pgfscope}%
\begin{pgfscope}%
\pgfsetbuttcap%
\pgfsetroundjoin%
\definecolor{currentfill}{rgb}{0.000000,0.000000,0.000000}%
\pgfsetfillcolor{currentfill}%
\pgfsetlinewidth{0.803000pt}%
\definecolor{currentstroke}{rgb}{0.000000,0.000000,0.000000}%
\pgfsetstrokecolor{currentstroke}%
\pgfsetdash{}{0pt}%
\pgfsys@defobject{currentmarker}{\pgfqpoint{-0.048611in}{0.000000in}}{\pgfqpoint{0.000000in}{0.000000in}}{%
\pgfpathmoveto{\pgfqpoint{0.000000in}{0.000000in}}%
\pgfpathlineto{\pgfqpoint{-0.048611in}{0.000000in}}%
\pgfusepath{stroke,fill}%
}%
\begin{pgfscope}%
\pgfsys@transformshift{0.555972in}{3.430873in}%
\pgfsys@useobject{currentmarker}{}%
\end{pgfscope}%
\end{pgfscope}%
\begin{pgfscope}%
\definecolor{textcolor}{rgb}{0.000000,0.000000,0.000000}%
\pgfsetstrokecolor{textcolor}%
\pgfsetfillcolor{textcolor}%
\pgftext[x=0.061140in,y=3.378111in,left,base]{\color{textcolor}\sffamily\fontsize{10.000000}{12.000000}\selectfont 0.475}%
\end{pgfscope}%
\begin{pgfscope}%
\pgfsetbuttcap%
\pgfsetroundjoin%
\definecolor{currentfill}{rgb}{0.000000,0.000000,0.000000}%
\pgfsetfillcolor{currentfill}%
\pgfsetlinewidth{0.803000pt}%
\definecolor{currentstroke}{rgb}{0.000000,0.000000,0.000000}%
\pgfsetstrokecolor{currentstroke}%
\pgfsetdash{}{0pt}%
\pgfsys@defobject{currentmarker}{\pgfqpoint{-0.048611in}{0.000000in}}{\pgfqpoint{0.000000in}{0.000000in}}{%
\pgfpathmoveto{\pgfqpoint{0.000000in}{0.000000in}}%
\pgfpathlineto{\pgfqpoint{-0.048611in}{0.000000in}}%
\pgfusepath{stroke,fill}%
}%
\begin{pgfscope}%
\pgfsys@transformshift{0.555972in}{3.735526in}%
\pgfsys@useobject{currentmarker}{}%
\end{pgfscope}%
\end{pgfscope}%
\begin{pgfscope}%
\definecolor{textcolor}{rgb}{0.000000,0.000000,0.000000}%
\pgfsetstrokecolor{textcolor}%
\pgfsetfillcolor{textcolor}%
\pgftext[x=0.061140in,y=3.682764in,left,base]{\color{textcolor}\sffamily\fontsize{10.000000}{12.000000}\selectfont 0.500}%
\end{pgfscope}%
\begin{pgfscope}%
\pgfsetbuttcap%
\pgfsetroundjoin%
\definecolor{currentfill}{rgb}{0.000000,0.000000,0.000000}%
\pgfsetfillcolor{currentfill}%
\pgfsetlinewidth{0.803000pt}%
\definecolor{currentstroke}{rgb}{0.000000,0.000000,0.000000}%
\pgfsetstrokecolor{currentstroke}%
\pgfsetdash{}{0pt}%
\pgfsys@defobject{currentmarker}{\pgfqpoint{-0.048611in}{0.000000in}}{\pgfqpoint{0.000000in}{0.000000in}}{%
\pgfpathmoveto{\pgfqpoint{0.000000in}{0.000000in}}%
\pgfpathlineto{\pgfqpoint{-0.048611in}{0.000000in}}%
\pgfusepath{stroke,fill}%
}%
\begin{pgfscope}%
\pgfsys@transformshift{0.555972in}{4.040178in}%
\pgfsys@useobject{currentmarker}{}%
\end{pgfscope}%
\end{pgfscope}%
\begin{pgfscope}%
\definecolor{textcolor}{rgb}{0.000000,0.000000,0.000000}%
\pgfsetstrokecolor{textcolor}%
\pgfsetfillcolor{textcolor}%
\pgftext[x=0.061140in,y=3.987417in,left,base]{\color{textcolor}\sffamily\fontsize{10.000000}{12.000000}\selectfont 0.525}%
\end{pgfscope}%
\begin{pgfscope}%
\pgfsetbuttcap%
\pgfsetroundjoin%
\definecolor{currentfill}{rgb}{0.000000,0.000000,0.000000}%
\pgfsetfillcolor{currentfill}%
\pgfsetlinewidth{0.803000pt}%
\definecolor{currentstroke}{rgb}{0.000000,0.000000,0.000000}%
\pgfsetstrokecolor{currentstroke}%
\pgfsetdash{}{0pt}%
\pgfsys@defobject{currentmarker}{\pgfqpoint{-0.048611in}{0.000000in}}{\pgfqpoint{0.000000in}{0.000000in}}{%
\pgfpathmoveto{\pgfqpoint{0.000000in}{0.000000in}}%
\pgfpathlineto{\pgfqpoint{-0.048611in}{0.000000in}}%
\pgfusepath{stroke,fill}%
}%
\begin{pgfscope}%
\pgfsys@transformshift{0.555972in}{4.344831in}%
\pgfsys@useobject{currentmarker}{}%
\end{pgfscope}%
\end{pgfscope}%
\begin{pgfscope}%
\definecolor{textcolor}{rgb}{0.000000,0.000000,0.000000}%
\pgfsetstrokecolor{textcolor}%
\pgfsetfillcolor{textcolor}%
\pgftext[x=0.061140in,y=4.292069in,left,base]{\color{textcolor}\sffamily\fontsize{10.000000}{12.000000}\selectfont 0.550}%
\end{pgfscope}%
\begin{pgfscope}%
\pgfsetbuttcap%
\pgfsetroundjoin%
\definecolor{currentfill}{rgb}{0.000000,0.000000,0.000000}%
\pgfsetfillcolor{currentfill}%
\pgfsetlinewidth{0.803000pt}%
\definecolor{currentstroke}{rgb}{0.000000,0.000000,0.000000}%
\pgfsetstrokecolor{currentstroke}%
\pgfsetdash{}{0pt}%
\pgfsys@defobject{currentmarker}{\pgfqpoint{-0.048611in}{0.000000in}}{\pgfqpoint{0.000000in}{0.000000in}}{%
\pgfpathmoveto{\pgfqpoint{0.000000in}{0.000000in}}%
\pgfpathlineto{\pgfqpoint{-0.048611in}{0.000000in}}%
\pgfusepath{stroke,fill}%
}%
\begin{pgfscope}%
\pgfsys@transformshift{0.555972in}{4.649483in}%
\pgfsys@useobject{currentmarker}{}%
\end{pgfscope}%
\end{pgfscope}%
\begin{pgfscope}%
\definecolor{textcolor}{rgb}{0.000000,0.000000,0.000000}%
\pgfsetstrokecolor{textcolor}%
\pgfsetfillcolor{textcolor}%
\pgftext[x=0.061140in,y=4.596722in,left,base]{\color{textcolor}\sffamily\fontsize{10.000000}{12.000000}\selectfont 0.575}%
\end{pgfscope}%
\begin{pgfscope}%
\pgfsetbuttcap%
\pgfsetroundjoin%
\definecolor{currentfill}{rgb}{0.000000,0.000000,0.000000}%
\pgfsetfillcolor{currentfill}%
\pgfsetlinewidth{0.803000pt}%
\definecolor{currentstroke}{rgb}{0.000000,0.000000,0.000000}%
\pgfsetstrokecolor{currentstroke}%
\pgfsetdash{}{0pt}%
\pgfsys@defobject{currentmarker}{\pgfqpoint{-0.048611in}{0.000000in}}{\pgfqpoint{0.000000in}{0.000000in}}{%
\pgfpathmoveto{\pgfqpoint{0.000000in}{0.000000in}}%
\pgfpathlineto{\pgfqpoint{-0.048611in}{0.000000in}}%
\pgfusepath{stroke,fill}%
}%
\begin{pgfscope}%
\pgfsys@transformshift{0.555972in}{4.954136in}%
\pgfsys@useobject{currentmarker}{}%
\end{pgfscope}%
\end{pgfscope}%
\begin{pgfscope}%
\definecolor{textcolor}{rgb}{0.000000,0.000000,0.000000}%
\pgfsetstrokecolor{textcolor}%
\pgfsetfillcolor{textcolor}%
\pgftext[x=0.061140in,y=4.901374in,left,base]{\color{textcolor}\sffamily\fontsize{10.000000}{12.000000}\selectfont 0.600}%
\end{pgfscope}%
\begin{pgfscope}%
\pgfsetbuttcap%
\pgfsetroundjoin%
\definecolor{currentfill}{rgb}{0.000000,0.000000,0.000000}%
\pgfsetfillcolor{currentfill}%
\pgfsetlinewidth{0.803000pt}%
\definecolor{currentstroke}{rgb}{0.000000,0.000000,0.000000}%
\pgfsetstrokecolor{currentstroke}%
\pgfsetdash{}{0pt}%
\pgfsys@defobject{currentmarker}{\pgfqpoint{-0.048611in}{0.000000in}}{\pgfqpoint{0.000000in}{0.000000in}}{%
\pgfpathmoveto{\pgfqpoint{0.000000in}{0.000000in}}%
\pgfpathlineto{\pgfqpoint{-0.048611in}{0.000000in}}%
\pgfusepath{stroke,fill}%
}%
\begin{pgfscope}%
\pgfsys@transformshift{0.555972in}{5.258789in}%
\pgfsys@useobject{currentmarker}{}%
\end{pgfscope}%
\end{pgfscope}%
\begin{pgfscope}%
\definecolor{textcolor}{rgb}{0.000000,0.000000,0.000000}%
\pgfsetstrokecolor{textcolor}%
\pgfsetfillcolor{textcolor}%
\pgftext[x=0.061140in,y=5.206027in,left,base]{\color{textcolor}\sffamily\fontsize{10.000000}{12.000000}\selectfont 0.625}%
\end{pgfscope}%
\begin{pgfscope}%
\pgfsetbuttcap%
\pgfsetroundjoin%
\definecolor{currentfill}{rgb}{0.000000,0.000000,0.000000}%
\pgfsetfillcolor{currentfill}%
\pgfsetlinewidth{0.803000pt}%
\definecolor{currentstroke}{rgb}{0.000000,0.000000,0.000000}%
\pgfsetstrokecolor{currentstroke}%
\pgfsetdash{}{0pt}%
\pgfsys@defobject{currentmarker}{\pgfqpoint{-0.048611in}{0.000000in}}{\pgfqpoint{0.000000in}{0.000000in}}{%
\pgfpathmoveto{\pgfqpoint{0.000000in}{0.000000in}}%
\pgfpathlineto{\pgfqpoint{-0.048611in}{0.000000in}}%
\pgfusepath{stroke,fill}%
}%
\begin{pgfscope}%
\pgfsys@transformshift{0.555972in}{5.563441in}%
\pgfsys@useobject{currentmarker}{}%
\end{pgfscope}%
\end{pgfscope}%
\begin{pgfscope}%
\definecolor{textcolor}{rgb}{0.000000,0.000000,0.000000}%
\pgfsetstrokecolor{textcolor}%
\pgfsetfillcolor{textcolor}%
\pgftext[x=0.061140in,y=5.510680in,left,base]{\color{textcolor}\sffamily\fontsize{10.000000}{12.000000}\selectfont 0.650}%
\end{pgfscope}%
\begin{pgfscope}%
\pgfpathrectangle{\pgfqpoint{0.555972in}{3.312222in}}{\pgfqpoint{3.322583in}{2.324444in}}%
\pgfusepath{clip}%
\pgfsetrectcap%
\pgfsetroundjoin%
\pgfsetlinewidth{1.505625pt}%
\definecolor{currentstroke}{rgb}{0.121569,0.466667,0.705882}%
\pgfsetstrokecolor{currentstroke}%
\pgfsetdash{}{0pt}%
\pgfpathmoveto{\pgfqpoint{0.715290in}{5.521307in}}%
\pgfpathlineto{\pgfqpoint{0.865487in}{5.445464in}}%
\pgfpathlineto{\pgfqpoint{1.015684in}{5.360584in}}%
\pgfpathlineto{\pgfqpoint{1.165882in}{5.277830in}}%
\pgfpathlineto{\pgfqpoint{1.316079in}{5.189698in}}%
\pgfpathlineto{\pgfqpoint{1.466277in}{5.093825in}}%
\pgfpathlineto{\pgfqpoint{1.616474in}{5.003605in}}%
\pgfpathlineto{\pgfqpoint{1.766671in}{4.905274in}}%
\pgfpathlineto{\pgfqpoint{1.916869in}{4.803853in}}%
\pgfpathlineto{\pgfqpoint{2.067066in}{4.712111in}}%
\pgfpathlineto{\pgfqpoint{2.217264in}{4.602472in}}%
\pgfpathlineto{\pgfqpoint{2.367461in}{4.493570in}}%
\pgfpathlineto{\pgfqpoint{2.517659in}{4.393553in}}%
\pgfpathlineto{\pgfqpoint{2.667856in}{4.274132in}}%
\pgfpathlineto{\pgfqpoint{2.818053in}{4.156835in}}%
\pgfpathlineto{\pgfqpoint{2.968251in}{4.043020in}}%
\pgfpathlineto{\pgfqpoint{3.118448in}{3.917554in}}%
\pgfpathlineto{\pgfqpoint{3.268646in}{3.803038in}}%
\pgfpathlineto{\pgfqpoint{3.418843in}{3.677500in}}%
\pgfpathlineto{\pgfqpoint{3.569040in}{3.570209in}}%
\pgfpathlineto{\pgfqpoint{3.719238in}{3.434063in}}%
\pgfusepath{stroke}%
\end{pgfscope}%
\begin{pgfscope}%
\pgfsetrectcap%
\pgfsetmiterjoin%
\pgfsetlinewidth{0.803000pt}%
\definecolor{currentstroke}{rgb}{0.000000,0.000000,0.000000}%
\pgfsetstrokecolor{currentstroke}%
\pgfsetdash{}{0pt}%
\pgfpathmoveto{\pgfqpoint{0.555972in}{3.312222in}}%
\pgfpathlineto{\pgfqpoint{0.555972in}{5.636667in}}%
\pgfusepath{stroke}%
\end{pgfscope}%
\begin{pgfscope}%
\pgfsetrectcap%
\pgfsetmiterjoin%
\pgfsetlinewidth{0.803000pt}%
\definecolor{currentstroke}{rgb}{0.000000,0.000000,0.000000}%
\pgfsetstrokecolor{currentstroke}%
\pgfsetdash{}{0pt}%
\pgfpathmoveto{\pgfqpoint{3.878555in}{3.312222in}}%
\pgfpathlineto{\pgfqpoint{3.878555in}{5.636667in}}%
\pgfusepath{stroke}%
\end{pgfscope}%
\begin{pgfscope}%
\pgfsetrectcap%
\pgfsetmiterjoin%
\pgfsetlinewidth{0.803000pt}%
\definecolor{currentstroke}{rgb}{0.000000,0.000000,0.000000}%
\pgfsetstrokecolor{currentstroke}%
\pgfsetdash{}{0pt}%
\pgfpathmoveto{\pgfqpoint{0.555972in}{3.312222in}}%
\pgfpathlineto{\pgfqpoint{3.878555in}{3.312222in}}%
\pgfusepath{stroke}%
\end{pgfscope}%
\begin{pgfscope}%
\pgfsetrectcap%
\pgfsetmiterjoin%
\pgfsetlinewidth{0.803000pt}%
\definecolor{currentstroke}{rgb}{0.000000,0.000000,0.000000}%
\pgfsetstrokecolor{currentstroke}%
\pgfsetdash{}{0pt}%
\pgfpathmoveto{\pgfqpoint{0.555972in}{5.636667in}}%
\pgfpathlineto{\pgfqpoint{3.878555in}{5.636667in}}%
\pgfusepath{stroke}%
\end{pgfscope}%
\begin{pgfscope}%
\definecolor{textcolor}{rgb}{0.000000,0.000000,0.000000}%
\pgfsetstrokecolor{textcolor}%
\pgfsetfillcolor{textcolor}%
\pgftext[x=2.217264in,y=5.720000in,,base]{\color{textcolor}\sffamily\fontsize{12.000000}{14.400000}\selectfont \(\displaystyle  N = 8 \)}%
\end{pgfscope}%
\begin{pgfscope}%
\pgfsetbuttcap%
\pgfsetmiterjoin%
\definecolor{currentfill}{rgb}{1.000000,1.000000,1.000000}%
\pgfsetfillcolor{currentfill}%
\pgfsetlinewidth{0.000000pt}%
\definecolor{currentstroke}{rgb}{0.000000,0.000000,0.000000}%
\pgfsetstrokecolor{currentstroke}%
\pgfsetstrokeopacity{0.000000}%
\pgfsetdash{}{0pt}%
\pgfpathmoveto{\pgfqpoint{4.492417in}{3.312222in}}%
\pgfpathlineto{\pgfqpoint{7.815000in}{3.312222in}}%
\pgfpathlineto{\pgfqpoint{7.815000in}{5.636667in}}%
\pgfpathlineto{\pgfqpoint{4.492417in}{5.636667in}}%
\pgfpathclose%
\pgfusepath{fill}%
\end{pgfscope}%
\begin{pgfscope}%
\pgfpathrectangle{\pgfqpoint{4.492417in}{3.312222in}}{\pgfqpoint{3.322583in}{2.324444in}}%
\pgfusepath{clip}%
\pgfsetbuttcap%
\pgfsetroundjoin%
\definecolor{currentfill}{rgb}{0.121569,0.466667,0.705882}%
\pgfsetfillcolor{currentfill}%
\pgfsetfillopacity{0.300000}%
\pgfsetlinewidth{1.003750pt}%
\definecolor{currentstroke}{rgb}{0.121569,0.466667,0.705882}%
\pgfsetstrokecolor{currentstroke}%
\pgfsetstrokeopacity{0.300000}%
\pgfsetdash{}{0pt}%
\pgfpathmoveto{\pgfqpoint{4.651734in}{5.531010in}}%
\pgfpathlineto{\pgfqpoint{4.651734in}{5.510067in}}%
\pgfpathlineto{\pgfqpoint{4.801932in}{5.432364in}}%
\pgfpathlineto{\pgfqpoint{4.952129in}{5.345794in}}%
\pgfpathlineto{\pgfqpoint{5.102327in}{5.285522in}}%
\pgfpathlineto{\pgfqpoint{5.252524in}{5.209687in}}%
\pgfpathlineto{\pgfqpoint{5.402721in}{5.106223in}}%
\pgfpathlineto{\pgfqpoint{5.552919in}{5.038321in}}%
\pgfpathlineto{\pgfqpoint{5.703116in}{4.934893in}}%
\pgfpathlineto{\pgfqpoint{5.853314in}{4.828058in}}%
\pgfpathlineto{\pgfqpoint{6.003511in}{4.738035in}}%
\pgfpathlineto{\pgfqpoint{6.153708in}{4.621380in}}%
\pgfpathlineto{\pgfqpoint{6.303906in}{4.508049in}}%
\pgfpathlineto{\pgfqpoint{6.454103in}{4.359312in}}%
\pgfpathlineto{\pgfqpoint{6.604301in}{4.286396in}}%
\pgfpathlineto{\pgfqpoint{6.754498in}{4.143292in}}%
\pgfpathlineto{\pgfqpoint{6.904696in}{4.024149in}}%
\pgfpathlineto{\pgfqpoint{7.054893in}{3.879464in}}%
\pgfpathlineto{\pgfqpoint{7.205090in}{3.749283in}}%
\pgfpathlineto{\pgfqpoint{7.355288in}{3.670456in}}%
\pgfpathlineto{\pgfqpoint{7.505485in}{3.541117in}}%
\pgfpathlineto{\pgfqpoint{7.655683in}{3.417879in}}%
\pgfpathlineto{\pgfqpoint{7.655683in}{3.496912in}}%
\pgfpathlineto{\pgfqpoint{7.655683in}{3.496912in}}%
\pgfpathlineto{\pgfqpoint{7.505485in}{3.586917in}}%
\pgfpathlineto{\pgfqpoint{7.355288in}{3.684989in}}%
\pgfpathlineto{\pgfqpoint{7.205090in}{3.803047in}}%
\pgfpathlineto{\pgfqpoint{7.054893in}{3.957739in}}%
\pgfpathlineto{\pgfqpoint{6.904696in}{4.095644in}}%
\pgfpathlineto{\pgfqpoint{6.754498in}{4.196719in}}%
\pgfpathlineto{\pgfqpoint{6.604301in}{4.317837in}}%
\pgfpathlineto{\pgfqpoint{6.454103in}{4.423195in}}%
\pgfpathlineto{\pgfqpoint{6.303906in}{4.537561in}}%
\pgfpathlineto{\pgfqpoint{6.153708in}{4.635690in}}%
\pgfpathlineto{\pgfqpoint{6.003511in}{4.780569in}}%
\pgfpathlineto{\pgfqpoint{5.853314in}{4.866208in}}%
\pgfpathlineto{\pgfqpoint{5.703116in}{4.959398in}}%
\pgfpathlineto{\pgfqpoint{5.552919in}{5.059418in}}%
\pgfpathlineto{\pgfqpoint{5.402721in}{5.146321in}}%
\pgfpathlineto{\pgfqpoint{5.252524in}{5.228964in}}%
\pgfpathlineto{\pgfqpoint{5.102327in}{5.309877in}}%
\pgfpathlineto{\pgfqpoint{4.952129in}{5.381861in}}%
\pgfpathlineto{\pgfqpoint{4.801932in}{5.448673in}}%
\pgfpathlineto{\pgfqpoint{4.651734in}{5.531010in}}%
\pgfpathclose%
\pgfusepath{stroke,fill}%
\end{pgfscope}%
\begin{pgfscope}%
\pgfpathrectangle{\pgfqpoint{4.492417in}{3.312222in}}{\pgfqpoint{3.322583in}{2.324444in}}%
\pgfusepath{clip}%
\pgfsetbuttcap%
\pgfsetroundjoin%
\definecolor{currentfill}{rgb}{0.121569,0.466667,0.705882}%
\pgfsetfillcolor{currentfill}%
\pgfsetlinewidth{1.003750pt}%
\definecolor{currentstroke}{rgb}{0.121569,0.466667,0.705882}%
\pgfsetstrokecolor{currentstroke}%
\pgfsetdash{}{0pt}%
\pgfsys@defobject{currentmarker}{\pgfqpoint{-0.009821in}{-0.009821in}}{\pgfqpoint{0.009821in}{0.009821in}}{%
\pgfpathmoveto{\pgfqpoint{0.000000in}{-0.009821in}}%
\pgfpathcurveto{\pgfqpoint{0.002605in}{-0.009821in}}{\pgfqpoint{0.005103in}{-0.008786in}}{\pgfqpoint{0.006944in}{-0.006944in}}%
\pgfpathcurveto{\pgfqpoint{0.008786in}{-0.005103in}}{\pgfqpoint{0.009821in}{-0.002605in}}{\pgfqpoint{0.009821in}{0.000000in}}%
\pgfpathcurveto{\pgfqpoint{0.009821in}{0.002605in}}{\pgfqpoint{0.008786in}{0.005103in}}{\pgfqpoint{0.006944in}{0.006944in}}%
\pgfpathcurveto{\pgfqpoint{0.005103in}{0.008786in}}{\pgfqpoint{0.002605in}{0.009821in}}{\pgfqpoint{0.000000in}{0.009821in}}%
\pgfpathcurveto{\pgfqpoint{-0.002605in}{0.009821in}}{\pgfqpoint{-0.005103in}{0.008786in}}{\pgfqpoint{-0.006944in}{0.006944in}}%
\pgfpathcurveto{\pgfqpoint{-0.008786in}{0.005103in}}{\pgfqpoint{-0.009821in}{0.002605in}}{\pgfqpoint{-0.009821in}{0.000000in}}%
\pgfpathcurveto{\pgfqpoint{-0.009821in}{-0.002605in}}{\pgfqpoint{-0.008786in}{-0.005103in}}{\pgfqpoint{-0.006944in}{-0.006944in}}%
\pgfpathcurveto{\pgfqpoint{-0.005103in}{-0.008786in}}{\pgfqpoint{-0.002605in}{-0.009821in}}{\pgfqpoint{0.000000in}{-0.009821in}}%
\pgfpathclose%
\pgfusepath{stroke,fill}%
}%
\begin{pgfscope}%
\pgfsys@transformshift{4.651734in}{5.520538in}%
\pgfsys@useobject{currentmarker}{}%
\end{pgfscope}%
\begin{pgfscope}%
\pgfsys@transformshift{4.801932in}{5.440519in}%
\pgfsys@useobject{currentmarker}{}%
\end{pgfscope}%
\begin{pgfscope}%
\pgfsys@transformshift{4.952129in}{5.363827in}%
\pgfsys@useobject{currentmarker}{}%
\end{pgfscope}%
\begin{pgfscope}%
\pgfsys@transformshift{5.102327in}{5.297699in}%
\pgfsys@useobject{currentmarker}{}%
\end{pgfscope}%
\begin{pgfscope}%
\pgfsys@transformshift{5.252524in}{5.219326in}%
\pgfsys@useobject{currentmarker}{}%
\end{pgfscope}%
\begin{pgfscope}%
\pgfsys@transformshift{5.402721in}{5.126272in}%
\pgfsys@useobject{currentmarker}{}%
\end{pgfscope}%
\begin{pgfscope}%
\pgfsys@transformshift{5.552919in}{5.048869in}%
\pgfsys@useobject{currentmarker}{}%
\end{pgfscope}%
\begin{pgfscope}%
\pgfsys@transformshift{5.703116in}{4.947146in}%
\pgfsys@useobject{currentmarker}{}%
\end{pgfscope}%
\begin{pgfscope}%
\pgfsys@transformshift{5.853314in}{4.847133in}%
\pgfsys@useobject{currentmarker}{}%
\end{pgfscope}%
\begin{pgfscope}%
\pgfsys@transformshift{6.003511in}{4.759302in}%
\pgfsys@useobject{currentmarker}{}%
\end{pgfscope}%
\begin{pgfscope}%
\pgfsys@transformshift{6.153708in}{4.628535in}%
\pgfsys@useobject{currentmarker}{}%
\end{pgfscope}%
\begin{pgfscope}%
\pgfsys@transformshift{6.303906in}{4.522805in}%
\pgfsys@useobject{currentmarker}{}%
\end{pgfscope}%
\begin{pgfscope}%
\pgfsys@transformshift{6.454103in}{4.391254in}%
\pgfsys@useobject{currentmarker}{}%
\end{pgfscope}%
\begin{pgfscope}%
\pgfsys@transformshift{6.604301in}{4.302116in}%
\pgfsys@useobject{currentmarker}{}%
\end{pgfscope}%
\begin{pgfscope}%
\pgfsys@transformshift{6.754498in}{4.170006in}%
\pgfsys@useobject{currentmarker}{}%
\end{pgfscope}%
\begin{pgfscope}%
\pgfsys@transformshift{6.904696in}{4.059897in}%
\pgfsys@useobject{currentmarker}{}%
\end{pgfscope}%
\begin{pgfscope}%
\pgfsys@transformshift{7.054893in}{3.918601in}%
\pgfsys@useobject{currentmarker}{}%
\end{pgfscope}%
\begin{pgfscope}%
\pgfsys@transformshift{7.205090in}{3.776165in}%
\pgfsys@useobject{currentmarker}{}%
\end{pgfscope}%
\begin{pgfscope}%
\pgfsys@transformshift{7.355288in}{3.677723in}%
\pgfsys@useobject{currentmarker}{}%
\end{pgfscope}%
\begin{pgfscope}%
\pgfsys@transformshift{7.505485in}{3.564017in}%
\pgfsys@useobject{currentmarker}{}%
\end{pgfscope}%
\begin{pgfscope}%
\pgfsys@transformshift{7.655683in}{3.457395in}%
\pgfsys@useobject{currentmarker}{}%
\end{pgfscope}%
\end{pgfscope}%
\begin{pgfscope}%
\pgfsetbuttcap%
\pgfsetroundjoin%
\definecolor{currentfill}{rgb}{0.000000,0.000000,0.000000}%
\pgfsetfillcolor{currentfill}%
\pgfsetlinewidth{0.803000pt}%
\definecolor{currentstroke}{rgb}{0.000000,0.000000,0.000000}%
\pgfsetstrokecolor{currentstroke}%
\pgfsetdash{}{0pt}%
\pgfsys@defobject{currentmarker}{\pgfqpoint{0.000000in}{-0.048611in}}{\pgfqpoint{0.000000in}{0.000000in}}{%
\pgfpathmoveto{\pgfqpoint{0.000000in}{0.000000in}}%
\pgfpathlineto{\pgfqpoint{0.000000in}{-0.048611in}}%
\pgfusepath{stroke,fill}%
}%
\begin{pgfscope}%
\pgfsys@transformshift{4.501537in}{3.312222in}%
\pgfsys@useobject{currentmarker}{}%
\end{pgfscope}%
\end{pgfscope}%
\begin{pgfscope}%
\definecolor{textcolor}{rgb}{0.000000,0.000000,0.000000}%
\pgfsetstrokecolor{textcolor}%
\pgfsetfillcolor{textcolor}%
\pgftext[x=4.501537in,y=3.215000in,,top]{\color{textcolor}\sffamily\fontsize{10.000000}{12.000000}\selectfont 4.30}%
\end{pgfscope}%
\begin{pgfscope}%
\pgfsetbuttcap%
\pgfsetroundjoin%
\definecolor{currentfill}{rgb}{0.000000,0.000000,0.000000}%
\pgfsetfillcolor{currentfill}%
\pgfsetlinewidth{0.803000pt}%
\definecolor{currentstroke}{rgb}{0.000000,0.000000,0.000000}%
\pgfsetstrokecolor{currentstroke}%
\pgfsetdash{}{0pt}%
\pgfsys@defobject{currentmarker}{\pgfqpoint{0.000000in}{-0.048611in}}{\pgfqpoint{0.000000in}{0.000000in}}{%
\pgfpathmoveto{\pgfqpoint{0.000000in}{0.000000in}}%
\pgfpathlineto{\pgfqpoint{0.000000in}{-0.048611in}}%
\pgfusepath{stroke,fill}%
}%
\begin{pgfscope}%
\pgfsys@transformshift{5.252524in}{3.312222in}%
\pgfsys@useobject{currentmarker}{}%
\end{pgfscope}%
\end{pgfscope}%
\begin{pgfscope}%
\definecolor{textcolor}{rgb}{0.000000,0.000000,0.000000}%
\pgfsetstrokecolor{textcolor}%
\pgfsetfillcolor{textcolor}%
\pgftext[x=5.252524in,y=3.215000in,,top]{\color{textcolor}\sffamily\fontsize{10.000000}{12.000000}\selectfont 4.35}%
\end{pgfscope}%
\begin{pgfscope}%
\pgfsetbuttcap%
\pgfsetroundjoin%
\definecolor{currentfill}{rgb}{0.000000,0.000000,0.000000}%
\pgfsetfillcolor{currentfill}%
\pgfsetlinewidth{0.803000pt}%
\definecolor{currentstroke}{rgb}{0.000000,0.000000,0.000000}%
\pgfsetstrokecolor{currentstroke}%
\pgfsetdash{}{0pt}%
\pgfsys@defobject{currentmarker}{\pgfqpoint{0.000000in}{-0.048611in}}{\pgfqpoint{0.000000in}{0.000000in}}{%
\pgfpathmoveto{\pgfqpoint{0.000000in}{0.000000in}}%
\pgfpathlineto{\pgfqpoint{0.000000in}{-0.048611in}}%
\pgfusepath{stroke,fill}%
}%
\begin{pgfscope}%
\pgfsys@transformshift{6.003511in}{3.312222in}%
\pgfsys@useobject{currentmarker}{}%
\end{pgfscope}%
\end{pgfscope}%
\begin{pgfscope}%
\definecolor{textcolor}{rgb}{0.000000,0.000000,0.000000}%
\pgfsetstrokecolor{textcolor}%
\pgfsetfillcolor{textcolor}%
\pgftext[x=6.003511in,y=3.215000in,,top]{\color{textcolor}\sffamily\fontsize{10.000000}{12.000000}\selectfont 4.40}%
\end{pgfscope}%
\begin{pgfscope}%
\pgfsetbuttcap%
\pgfsetroundjoin%
\definecolor{currentfill}{rgb}{0.000000,0.000000,0.000000}%
\pgfsetfillcolor{currentfill}%
\pgfsetlinewidth{0.803000pt}%
\definecolor{currentstroke}{rgb}{0.000000,0.000000,0.000000}%
\pgfsetstrokecolor{currentstroke}%
\pgfsetdash{}{0pt}%
\pgfsys@defobject{currentmarker}{\pgfqpoint{0.000000in}{-0.048611in}}{\pgfqpoint{0.000000in}{0.000000in}}{%
\pgfpathmoveto{\pgfqpoint{0.000000in}{0.000000in}}%
\pgfpathlineto{\pgfqpoint{0.000000in}{-0.048611in}}%
\pgfusepath{stroke,fill}%
}%
\begin{pgfscope}%
\pgfsys@transformshift{6.754498in}{3.312222in}%
\pgfsys@useobject{currentmarker}{}%
\end{pgfscope}%
\end{pgfscope}%
\begin{pgfscope}%
\definecolor{textcolor}{rgb}{0.000000,0.000000,0.000000}%
\pgfsetstrokecolor{textcolor}%
\pgfsetfillcolor{textcolor}%
\pgftext[x=6.754498in,y=3.215000in,,top]{\color{textcolor}\sffamily\fontsize{10.000000}{12.000000}\selectfont 4.45}%
\end{pgfscope}%
\begin{pgfscope}%
\pgfsetbuttcap%
\pgfsetroundjoin%
\definecolor{currentfill}{rgb}{0.000000,0.000000,0.000000}%
\pgfsetfillcolor{currentfill}%
\pgfsetlinewidth{0.803000pt}%
\definecolor{currentstroke}{rgb}{0.000000,0.000000,0.000000}%
\pgfsetstrokecolor{currentstroke}%
\pgfsetdash{}{0pt}%
\pgfsys@defobject{currentmarker}{\pgfqpoint{0.000000in}{-0.048611in}}{\pgfqpoint{0.000000in}{0.000000in}}{%
\pgfpathmoveto{\pgfqpoint{0.000000in}{0.000000in}}%
\pgfpathlineto{\pgfqpoint{0.000000in}{-0.048611in}}%
\pgfusepath{stroke,fill}%
}%
\begin{pgfscope}%
\pgfsys@transformshift{7.505485in}{3.312222in}%
\pgfsys@useobject{currentmarker}{}%
\end{pgfscope}%
\end{pgfscope}%
\begin{pgfscope}%
\definecolor{textcolor}{rgb}{0.000000,0.000000,0.000000}%
\pgfsetstrokecolor{textcolor}%
\pgfsetfillcolor{textcolor}%
\pgftext[x=7.505485in,y=3.215000in,,top]{\color{textcolor}\sffamily\fontsize{10.000000}{12.000000}\selectfont 4.50}%
\end{pgfscope}%
\begin{pgfscope}%
\pgfsetbuttcap%
\pgfsetroundjoin%
\definecolor{currentfill}{rgb}{0.000000,0.000000,0.000000}%
\pgfsetfillcolor{currentfill}%
\pgfsetlinewidth{0.803000pt}%
\definecolor{currentstroke}{rgb}{0.000000,0.000000,0.000000}%
\pgfsetstrokecolor{currentstroke}%
\pgfsetdash{}{0pt}%
\pgfsys@defobject{currentmarker}{\pgfqpoint{-0.048611in}{0.000000in}}{\pgfqpoint{0.000000in}{0.000000in}}{%
\pgfpathmoveto{\pgfqpoint{0.000000in}{0.000000in}}%
\pgfpathlineto{\pgfqpoint{-0.048611in}{0.000000in}}%
\pgfusepath{stroke,fill}%
}%
\begin{pgfscope}%
\pgfsys@transformshift{4.492417in}{3.387190in}%
\pgfsys@useobject{currentmarker}{}%
\end{pgfscope}%
\end{pgfscope}%
\begin{pgfscope}%
\definecolor{textcolor}{rgb}{0.000000,0.000000,0.000000}%
\pgfsetstrokecolor{textcolor}%
\pgfsetfillcolor{textcolor}%
\pgftext[x=4.085950in,y=3.334429in,left,base]{\color{textcolor}\sffamily\fontsize{10.000000}{12.000000}\selectfont 0.30}%
\end{pgfscope}%
\begin{pgfscope}%
\pgfsetbuttcap%
\pgfsetroundjoin%
\definecolor{currentfill}{rgb}{0.000000,0.000000,0.000000}%
\pgfsetfillcolor{currentfill}%
\pgfsetlinewidth{0.803000pt}%
\definecolor{currentstroke}{rgb}{0.000000,0.000000,0.000000}%
\pgfsetstrokecolor{currentstroke}%
\pgfsetdash{}{0pt}%
\pgfsys@defobject{currentmarker}{\pgfqpoint{-0.048611in}{0.000000in}}{\pgfqpoint{0.000000in}{0.000000in}}{%
\pgfpathmoveto{\pgfqpoint{0.000000in}{0.000000in}}%
\pgfpathlineto{\pgfqpoint{-0.048611in}{0.000000in}}%
\pgfusepath{stroke,fill}%
}%
\begin{pgfscope}%
\pgfsys@transformshift{4.492417in}{3.768242in}%
\pgfsys@useobject{currentmarker}{}%
\end{pgfscope}%
\end{pgfscope}%
\begin{pgfscope}%
\definecolor{textcolor}{rgb}{0.000000,0.000000,0.000000}%
\pgfsetstrokecolor{textcolor}%
\pgfsetfillcolor{textcolor}%
\pgftext[x=4.085950in,y=3.715480in,left,base]{\color{textcolor}\sffamily\fontsize{10.000000}{12.000000}\selectfont 0.35}%
\end{pgfscope}%
\begin{pgfscope}%
\pgfsetbuttcap%
\pgfsetroundjoin%
\definecolor{currentfill}{rgb}{0.000000,0.000000,0.000000}%
\pgfsetfillcolor{currentfill}%
\pgfsetlinewidth{0.803000pt}%
\definecolor{currentstroke}{rgb}{0.000000,0.000000,0.000000}%
\pgfsetstrokecolor{currentstroke}%
\pgfsetdash{}{0pt}%
\pgfsys@defobject{currentmarker}{\pgfqpoint{-0.048611in}{0.000000in}}{\pgfqpoint{0.000000in}{0.000000in}}{%
\pgfpathmoveto{\pgfqpoint{0.000000in}{0.000000in}}%
\pgfpathlineto{\pgfqpoint{-0.048611in}{0.000000in}}%
\pgfusepath{stroke,fill}%
}%
\begin{pgfscope}%
\pgfsys@transformshift{4.492417in}{4.149293in}%
\pgfsys@useobject{currentmarker}{}%
\end{pgfscope}%
\end{pgfscope}%
\begin{pgfscope}%
\definecolor{textcolor}{rgb}{0.000000,0.000000,0.000000}%
\pgfsetstrokecolor{textcolor}%
\pgfsetfillcolor{textcolor}%
\pgftext[x=4.085950in,y=4.096531in,left,base]{\color{textcolor}\sffamily\fontsize{10.000000}{12.000000}\selectfont 0.40}%
\end{pgfscope}%
\begin{pgfscope}%
\pgfsetbuttcap%
\pgfsetroundjoin%
\definecolor{currentfill}{rgb}{0.000000,0.000000,0.000000}%
\pgfsetfillcolor{currentfill}%
\pgfsetlinewidth{0.803000pt}%
\definecolor{currentstroke}{rgb}{0.000000,0.000000,0.000000}%
\pgfsetstrokecolor{currentstroke}%
\pgfsetdash{}{0pt}%
\pgfsys@defobject{currentmarker}{\pgfqpoint{-0.048611in}{0.000000in}}{\pgfqpoint{0.000000in}{0.000000in}}{%
\pgfpathmoveto{\pgfqpoint{0.000000in}{0.000000in}}%
\pgfpathlineto{\pgfqpoint{-0.048611in}{0.000000in}}%
\pgfusepath{stroke,fill}%
}%
\begin{pgfscope}%
\pgfsys@transformshift{4.492417in}{4.530344in}%
\pgfsys@useobject{currentmarker}{}%
\end{pgfscope}%
\end{pgfscope}%
\begin{pgfscope}%
\definecolor{textcolor}{rgb}{0.000000,0.000000,0.000000}%
\pgfsetstrokecolor{textcolor}%
\pgfsetfillcolor{textcolor}%
\pgftext[x=4.085950in,y=4.477583in,left,base]{\color{textcolor}\sffamily\fontsize{10.000000}{12.000000}\selectfont 0.45}%
\end{pgfscope}%
\begin{pgfscope}%
\pgfsetbuttcap%
\pgfsetroundjoin%
\definecolor{currentfill}{rgb}{0.000000,0.000000,0.000000}%
\pgfsetfillcolor{currentfill}%
\pgfsetlinewidth{0.803000pt}%
\definecolor{currentstroke}{rgb}{0.000000,0.000000,0.000000}%
\pgfsetstrokecolor{currentstroke}%
\pgfsetdash{}{0pt}%
\pgfsys@defobject{currentmarker}{\pgfqpoint{-0.048611in}{0.000000in}}{\pgfqpoint{0.000000in}{0.000000in}}{%
\pgfpathmoveto{\pgfqpoint{0.000000in}{0.000000in}}%
\pgfpathlineto{\pgfqpoint{-0.048611in}{0.000000in}}%
\pgfusepath{stroke,fill}%
}%
\begin{pgfscope}%
\pgfsys@transformshift{4.492417in}{4.911395in}%
\pgfsys@useobject{currentmarker}{}%
\end{pgfscope}%
\end{pgfscope}%
\begin{pgfscope}%
\definecolor{textcolor}{rgb}{0.000000,0.000000,0.000000}%
\pgfsetstrokecolor{textcolor}%
\pgfsetfillcolor{textcolor}%
\pgftext[x=4.085950in,y=4.858634in,left,base]{\color{textcolor}\sffamily\fontsize{10.000000}{12.000000}\selectfont 0.50}%
\end{pgfscope}%
\begin{pgfscope}%
\pgfsetbuttcap%
\pgfsetroundjoin%
\definecolor{currentfill}{rgb}{0.000000,0.000000,0.000000}%
\pgfsetfillcolor{currentfill}%
\pgfsetlinewidth{0.803000pt}%
\definecolor{currentstroke}{rgb}{0.000000,0.000000,0.000000}%
\pgfsetstrokecolor{currentstroke}%
\pgfsetdash{}{0pt}%
\pgfsys@defobject{currentmarker}{\pgfqpoint{-0.048611in}{0.000000in}}{\pgfqpoint{0.000000in}{0.000000in}}{%
\pgfpathmoveto{\pgfqpoint{0.000000in}{0.000000in}}%
\pgfpathlineto{\pgfqpoint{-0.048611in}{0.000000in}}%
\pgfusepath{stroke,fill}%
}%
\begin{pgfscope}%
\pgfsys@transformshift{4.492417in}{5.292446in}%
\pgfsys@useobject{currentmarker}{}%
\end{pgfscope}%
\end{pgfscope}%
\begin{pgfscope}%
\definecolor{textcolor}{rgb}{0.000000,0.000000,0.000000}%
\pgfsetstrokecolor{textcolor}%
\pgfsetfillcolor{textcolor}%
\pgftext[x=4.085950in,y=5.239685in,left,base]{\color{textcolor}\sffamily\fontsize{10.000000}{12.000000}\selectfont 0.55}%
\end{pgfscope}%
\begin{pgfscope}%
\pgfpathrectangle{\pgfqpoint{4.492417in}{3.312222in}}{\pgfqpoint{3.322583in}{2.324444in}}%
\pgfusepath{clip}%
\pgfsetrectcap%
\pgfsetroundjoin%
\pgfsetlinewidth{1.505625pt}%
\definecolor{currentstroke}{rgb}{0.121569,0.466667,0.705882}%
\pgfsetstrokecolor{currentstroke}%
\pgfsetdash{}{0pt}%
\pgfpathmoveto{\pgfqpoint{4.651734in}{5.520538in}}%
\pgfpathlineto{\pgfqpoint{4.801932in}{5.440519in}}%
\pgfpathlineto{\pgfqpoint{4.952129in}{5.363827in}}%
\pgfpathlineto{\pgfqpoint{5.102327in}{5.297699in}}%
\pgfpathlineto{\pgfqpoint{5.252524in}{5.219326in}}%
\pgfpathlineto{\pgfqpoint{5.402721in}{5.126272in}}%
\pgfpathlineto{\pgfqpoint{5.552919in}{5.048869in}}%
\pgfpathlineto{\pgfqpoint{5.703116in}{4.947146in}}%
\pgfpathlineto{\pgfqpoint{5.853314in}{4.847133in}}%
\pgfpathlineto{\pgfqpoint{6.003511in}{4.759302in}}%
\pgfpathlineto{\pgfqpoint{6.153708in}{4.628535in}}%
\pgfpathlineto{\pgfqpoint{6.303906in}{4.522805in}}%
\pgfpathlineto{\pgfqpoint{6.454103in}{4.391254in}}%
\pgfpathlineto{\pgfqpoint{6.604301in}{4.302116in}}%
\pgfpathlineto{\pgfqpoint{6.754498in}{4.170006in}}%
\pgfpathlineto{\pgfqpoint{6.904696in}{4.059897in}}%
\pgfpathlineto{\pgfqpoint{7.054893in}{3.918601in}}%
\pgfpathlineto{\pgfqpoint{7.205090in}{3.776165in}}%
\pgfpathlineto{\pgfqpoint{7.355288in}{3.677723in}}%
\pgfpathlineto{\pgfqpoint{7.505485in}{3.564017in}}%
\pgfpathlineto{\pgfqpoint{7.655683in}{3.457395in}}%
\pgfusepath{stroke}%
\end{pgfscope}%
\begin{pgfscope}%
\pgfsetrectcap%
\pgfsetmiterjoin%
\pgfsetlinewidth{0.803000pt}%
\definecolor{currentstroke}{rgb}{0.000000,0.000000,0.000000}%
\pgfsetstrokecolor{currentstroke}%
\pgfsetdash{}{0pt}%
\pgfpathmoveto{\pgfqpoint{4.492417in}{3.312222in}}%
\pgfpathlineto{\pgfqpoint{4.492417in}{5.636667in}}%
\pgfusepath{stroke}%
\end{pgfscope}%
\begin{pgfscope}%
\pgfsetrectcap%
\pgfsetmiterjoin%
\pgfsetlinewidth{0.803000pt}%
\definecolor{currentstroke}{rgb}{0.000000,0.000000,0.000000}%
\pgfsetstrokecolor{currentstroke}%
\pgfsetdash{}{0pt}%
\pgfpathmoveto{\pgfqpoint{7.815000in}{3.312222in}}%
\pgfpathlineto{\pgfqpoint{7.815000in}{5.636667in}}%
\pgfusepath{stroke}%
\end{pgfscope}%
\begin{pgfscope}%
\pgfsetrectcap%
\pgfsetmiterjoin%
\pgfsetlinewidth{0.803000pt}%
\definecolor{currentstroke}{rgb}{0.000000,0.000000,0.000000}%
\pgfsetstrokecolor{currentstroke}%
\pgfsetdash{}{0pt}%
\pgfpathmoveto{\pgfqpoint{4.492417in}{3.312222in}}%
\pgfpathlineto{\pgfqpoint{7.815000in}{3.312222in}}%
\pgfusepath{stroke}%
\end{pgfscope}%
\begin{pgfscope}%
\pgfsetrectcap%
\pgfsetmiterjoin%
\pgfsetlinewidth{0.803000pt}%
\definecolor{currentstroke}{rgb}{0.000000,0.000000,0.000000}%
\pgfsetstrokecolor{currentstroke}%
\pgfsetdash{}{0pt}%
\pgfpathmoveto{\pgfqpoint{4.492417in}{5.636667in}}%
\pgfpathlineto{\pgfqpoint{7.815000in}{5.636667in}}%
\pgfusepath{stroke}%
\end{pgfscope}%
\begin{pgfscope}%
\definecolor{textcolor}{rgb}{0.000000,0.000000,0.000000}%
\pgfsetstrokecolor{textcolor}%
\pgfsetfillcolor{textcolor}%
\pgftext[x=6.153708in,y=5.720000in,,base]{\color{textcolor}\sffamily\fontsize{12.000000}{14.400000}\selectfont \(\displaystyle  N = 12 \)}%
\end{pgfscope}%
\begin{pgfscope}%
\pgfsetbuttcap%
\pgfsetmiterjoin%
\definecolor{currentfill}{rgb}{1.000000,1.000000,1.000000}%
\pgfsetfillcolor{currentfill}%
\pgfsetlinewidth{0.000000pt}%
\definecolor{currentstroke}{rgb}{0.000000,0.000000,0.000000}%
\pgfsetstrokecolor{currentstroke}%
\pgfsetstrokeopacity{0.000000}%
\pgfsetdash{}{0pt}%
\pgfpathmoveto{\pgfqpoint{0.555972in}{0.387222in}}%
\pgfpathlineto{\pgfqpoint{3.878555in}{0.387222in}}%
\pgfpathlineto{\pgfqpoint{3.878555in}{2.711667in}}%
\pgfpathlineto{\pgfqpoint{0.555972in}{2.711667in}}%
\pgfpathclose%
\pgfusepath{fill}%
\end{pgfscope}%
\begin{pgfscope}%
\pgfpathrectangle{\pgfqpoint{0.555972in}{0.387222in}}{\pgfqpoint{3.322583in}{2.324444in}}%
\pgfusepath{clip}%
\pgfsetbuttcap%
\pgfsetroundjoin%
\definecolor{currentfill}{rgb}{0.121569,0.466667,0.705882}%
\pgfsetfillcolor{currentfill}%
\pgfsetfillopacity{0.300000}%
\pgfsetlinewidth{1.003750pt}%
\definecolor{currentstroke}{rgb}{0.121569,0.466667,0.705882}%
\pgfsetstrokecolor{currentstroke}%
\pgfsetstrokeopacity{0.300000}%
\pgfsetdash{}{0pt}%
\pgfpathmoveto{\pgfqpoint{0.715290in}{2.606010in}}%
\pgfpathlineto{\pgfqpoint{0.715290in}{2.580402in}}%
\pgfpathlineto{\pgfqpoint{0.865487in}{2.527109in}}%
\pgfpathlineto{\pgfqpoint{1.015684in}{2.461032in}}%
\pgfpathlineto{\pgfqpoint{1.165882in}{2.393286in}}%
\pgfpathlineto{\pgfqpoint{1.316079in}{2.329547in}}%
\pgfpathlineto{\pgfqpoint{1.466277in}{2.248239in}}%
\pgfpathlineto{\pgfqpoint{1.616474in}{2.150988in}}%
\pgfpathlineto{\pgfqpoint{1.766671in}{2.084149in}}%
\pgfpathlineto{\pgfqpoint{1.916869in}{1.986896in}}%
\pgfpathlineto{\pgfqpoint{2.067066in}{1.878383in}}%
\pgfpathlineto{\pgfqpoint{2.217264in}{1.737534in}}%
\pgfpathlineto{\pgfqpoint{2.367461in}{1.629259in}}%
\pgfpathlineto{\pgfqpoint{2.517659in}{1.471218in}}%
\pgfpathlineto{\pgfqpoint{2.667856in}{1.382235in}}%
\pgfpathlineto{\pgfqpoint{2.818053in}{1.242006in}}%
\pgfpathlineto{\pgfqpoint{2.968251in}{1.104723in}}%
\pgfpathlineto{\pgfqpoint{3.118448in}{0.944310in}}%
\pgfpathlineto{\pgfqpoint{3.268646in}{0.827024in}}%
\pgfpathlineto{\pgfqpoint{3.418843in}{0.725387in}}%
\pgfpathlineto{\pgfqpoint{3.569040in}{0.596541in}}%
\pgfpathlineto{\pgfqpoint{3.719238in}{0.492879in}}%
\pgfpathlineto{\pgfqpoint{3.719238in}{0.553955in}}%
\pgfpathlineto{\pgfqpoint{3.719238in}{0.553955in}}%
\pgfpathlineto{\pgfqpoint{3.569040in}{0.658777in}}%
\pgfpathlineto{\pgfqpoint{3.418843in}{0.801373in}}%
\pgfpathlineto{\pgfqpoint{3.268646in}{0.915453in}}%
\pgfpathlineto{\pgfqpoint{3.118448in}{1.059672in}}%
\pgfpathlineto{\pgfqpoint{2.968251in}{1.183340in}}%
\pgfpathlineto{\pgfqpoint{2.818053in}{1.289740in}}%
\pgfpathlineto{\pgfqpoint{2.667856in}{1.476683in}}%
\pgfpathlineto{\pgfqpoint{2.517659in}{1.579797in}}%
\pgfpathlineto{\pgfqpoint{2.367461in}{1.657037in}}%
\pgfpathlineto{\pgfqpoint{2.217264in}{1.815880in}}%
\pgfpathlineto{\pgfqpoint{2.067066in}{1.923416in}}%
\pgfpathlineto{\pgfqpoint{1.916869in}{2.010197in}}%
\pgfpathlineto{\pgfqpoint{1.766671in}{2.117666in}}%
\pgfpathlineto{\pgfqpoint{1.616474in}{2.175486in}}%
\pgfpathlineto{\pgfqpoint{1.466277in}{2.258266in}}%
\pgfpathlineto{\pgfqpoint{1.316079in}{2.349182in}}%
\pgfpathlineto{\pgfqpoint{1.165882in}{2.421358in}}%
\pgfpathlineto{\pgfqpoint{1.015684in}{2.483160in}}%
\pgfpathlineto{\pgfqpoint{0.865487in}{2.545380in}}%
\pgfpathlineto{\pgfqpoint{0.715290in}{2.606010in}}%
\pgfpathclose%
\pgfusepath{stroke,fill}%
\end{pgfscope}%
\begin{pgfscope}%
\pgfpathrectangle{\pgfqpoint{0.555972in}{0.387222in}}{\pgfqpoint{3.322583in}{2.324444in}}%
\pgfusepath{clip}%
\pgfsetbuttcap%
\pgfsetroundjoin%
\definecolor{currentfill}{rgb}{0.121569,0.466667,0.705882}%
\pgfsetfillcolor{currentfill}%
\pgfsetlinewidth{1.003750pt}%
\definecolor{currentstroke}{rgb}{0.121569,0.466667,0.705882}%
\pgfsetstrokecolor{currentstroke}%
\pgfsetdash{}{0pt}%
\pgfsys@defobject{currentmarker}{\pgfqpoint{-0.009821in}{-0.009821in}}{\pgfqpoint{0.009821in}{0.009821in}}{%
\pgfpathmoveto{\pgfqpoint{0.000000in}{-0.009821in}}%
\pgfpathcurveto{\pgfqpoint{0.002605in}{-0.009821in}}{\pgfqpoint{0.005103in}{-0.008786in}}{\pgfqpoint{0.006944in}{-0.006944in}}%
\pgfpathcurveto{\pgfqpoint{0.008786in}{-0.005103in}}{\pgfqpoint{0.009821in}{-0.002605in}}{\pgfqpoint{0.009821in}{0.000000in}}%
\pgfpathcurveto{\pgfqpoint{0.009821in}{0.002605in}}{\pgfqpoint{0.008786in}{0.005103in}}{\pgfqpoint{0.006944in}{0.006944in}}%
\pgfpathcurveto{\pgfqpoint{0.005103in}{0.008786in}}{\pgfqpoint{0.002605in}{0.009821in}}{\pgfqpoint{0.000000in}{0.009821in}}%
\pgfpathcurveto{\pgfqpoint{-0.002605in}{0.009821in}}{\pgfqpoint{-0.005103in}{0.008786in}}{\pgfqpoint{-0.006944in}{0.006944in}}%
\pgfpathcurveto{\pgfqpoint{-0.008786in}{0.005103in}}{\pgfqpoint{-0.009821in}{0.002605in}}{\pgfqpoint{-0.009821in}{0.000000in}}%
\pgfpathcurveto{\pgfqpoint{-0.009821in}{-0.002605in}}{\pgfqpoint{-0.008786in}{-0.005103in}}{\pgfqpoint{-0.006944in}{-0.006944in}}%
\pgfpathcurveto{\pgfqpoint{-0.005103in}{-0.008786in}}{\pgfqpoint{-0.002605in}{-0.009821in}}{\pgfqpoint{0.000000in}{-0.009821in}}%
\pgfpathclose%
\pgfusepath{stroke,fill}%
}%
\begin{pgfscope}%
\pgfsys@transformshift{0.715290in}{2.593206in}%
\pgfsys@useobject{currentmarker}{}%
\end{pgfscope}%
\begin{pgfscope}%
\pgfsys@transformshift{0.865487in}{2.536245in}%
\pgfsys@useobject{currentmarker}{}%
\end{pgfscope}%
\begin{pgfscope}%
\pgfsys@transformshift{1.015684in}{2.472096in}%
\pgfsys@useobject{currentmarker}{}%
\end{pgfscope}%
\begin{pgfscope}%
\pgfsys@transformshift{1.165882in}{2.407322in}%
\pgfsys@useobject{currentmarker}{}%
\end{pgfscope}%
\begin{pgfscope}%
\pgfsys@transformshift{1.316079in}{2.339364in}%
\pgfsys@useobject{currentmarker}{}%
\end{pgfscope}%
\begin{pgfscope}%
\pgfsys@transformshift{1.466277in}{2.253253in}%
\pgfsys@useobject{currentmarker}{}%
\end{pgfscope}%
\begin{pgfscope}%
\pgfsys@transformshift{1.616474in}{2.163237in}%
\pgfsys@useobject{currentmarker}{}%
\end{pgfscope}%
\begin{pgfscope}%
\pgfsys@transformshift{1.766671in}{2.100907in}%
\pgfsys@useobject{currentmarker}{}%
\end{pgfscope}%
\begin{pgfscope}%
\pgfsys@transformshift{1.916869in}{1.998547in}%
\pgfsys@useobject{currentmarker}{}%
\end{pgfscope}%
\begin{pgfscope}%
\pgfsys@transformshift{2.067066in}{1.900900in}%
\pgfsys@useobject{currentmarker}{}%
\end{pgfscope}%
\begin{pgfscope}%
\pgfsys@transformshift{2.217264in}{1.776707in}%
\pgfsys@useobject{currentmarker}{}%
\end{pgfscope}%
\begin{pgfscope}%
\pgfsys@transformshift{2.367461in}{1.643148in}%
\pgfsys@useobject{currentmarker}{}%
\end{pgfscope}%
\begin{pgfscope}%
\pgfsys@transformshift{2.517659in}{1.525508in}%
\pgfsys@useobject{currentmarker}{}%
\end{pgfscope}%
\begin{pgfscope}%
\pgfsys@transformshift{2.667856in}{1.429459in}%
\pgfsys@useobject{currentmarker}{}%
\end{pgfscope}%
\begin{pgfscope}%
\pgfsys@transformshift{2.818053in}{1.265873in}%
\pgfsys@useobject{currentmarker}{}%
\end{pgfscope}%
\begin{pgfscope}%
\pgfsys@transformshift{2.968251in}{1.144032in}%
\pgfsys@useobject{currentmarker}{}%
\end{pgfscope}%
\begin{pgfscope}%
\pgfsys@transformshift{3.118448in}{1.001991in}%
\pgfsys@useobject{currentmarker}{}%
\end{pgfscope}%
\begin{pgfscope}%
\pgfsys@transformshift{3.268646in}{0.871239in}%
\pgfsys@useobject{currentmarker}{}%
\end{pgfscope}%
\begin{pgfscope}%
\pgfsys@transformshift{3.418843in}{0.763380in}%
\pgfsys@useobject{currentmarker}{}%
\end{pgfscope}%
\begin{pgfscope}%
\pgfsys@transformshift{3.569040in}{0.627659in}%
\pgfsys@useobject{currentmarker}{}%
\end{pgfscope}%
\begin{pgfscope}%
\pgfsys@transformshift{3.719238in}{0.523417in}%
\pgfsys@useobject{currentmarker}{}%
\end{pgfscope}%
\end{pgfscope}%
\begin{pgfscope}%
\pgfsetbuttcap%
\pgfsetroundjoin%
\definecolor{currentfill}{rgb}{0.000000,0.000000,0.000000}%
\pgfsetfillcolor{currentfill}%
\pgfsetlinewidth{0.803000pt}%
\definecolor{currentstroke}{rgb}{0.000000,0.000000,0.000000}%
\pgfsetstrokecolor{currentstroke}%
\pgfsetdash{}{0pt}%
\pgfsys@defobject{currentmarker}{\pgfqpoint{0.000000in}{-0.048611in}}{\pgfqpoint{0.000000in}{0.000000in}}{%
\pgfpathmoveto{\pgfqpoint{0.000000in}{0.000000in}}%
\pgfpathlineto{\pgfqpoint{0.000000in}{-0.048611in}}%
\pgfusepath{stroke,fill}%
}%
\begin{pgfscope}%
\pgfsys@transformshift{0.865487in}{0.387222in}%
\pgfsys@useobject{currentmarker}{}%
\end{pgfscope}%
\end{pgfscope}%
\begin{pgfscope}%
\definecolor{textcolor}{rgb}{0.000000,0.000000,0.000000}%
\pgfsetstrokecolor{textcolor}%
\pgfsetfillcolor{textcolor}%
\pgftext[x=0.865487in,y=0.290000in,,top]{\color{textcolor}\sffamily\fontsize{10.000000}{12.000000}\selectfont 4.35}%
\end{pgfscope}%
\begin{pgfscope}%
\pgfsetbuttcap%
\pgfsetroundjoin%
\definecolor{currentfill}{rgb}{0.000000,0.000000,0.000000}%
\pgfsetfillcolor{currentfill}%
\pgfsetlinewidth{0.803000pt}%
\definecolor{currentstroke}{rgb}{0.000000,0.000000,0.000000}%
\pgfsetstrokecolor{currentstroke}%
\pgfsetdash{}{0pt}%
\pgfsys@defobject{currentmarker}{\pgfqpoint{0.000000in}{-0.048611in}}{\pgfqpoint{0.000000in}{0.000000in}}{%
\pgfpathmoveto{\pgfqpoint{0.000000in}{0.000000in}}%
\pgfpathlineto{\pgfqpoint{0.000000in}{-0.048611in}}%
\pgfusepath{stroke,fill}%
}%
\begin{pgfscope}%
\pgfsys@transformshift{1.616474in}{0.387222in}%
\pgfsys@useobject{currentmarker}{}%
\end{pgfscope}%
\end{pgfscope}%
\begin{pgfscope}%
\definecolor{textcolor}{rgb}{0.000000,0.000000,0.000000}%
\pgfsetstrokecolor{textcolor}%
\pgfsetfillcolor{textcolor}%
\pgftext[x=1.616474in,y=0.290000in,,top]{\color{textcolor}\sffamily\fontsize{10.000000}{12.000000}\selectfont 4.40}%
\end{pgfscope}%
\begin{pgfscope}%
\pgfsetbuttcap%
\pgfsetroundjoin%
\definecolor{currentfill}{rgb}{0.000000,0.000000,0.000000}%
\pgfsetfillcolor{currentfill}%
\pgfsetlinewidth{0.803000pt}%
\definecolor{currentstroke}{rgb}{0.000000,0.000000,0.000000}%
\pgfsetstrokecolor{currentstroke}%
\pgfsetdash{}{0pt}%
\pgfsys@defobject{currentmarker}{\pgfqpoint{0.000000in}{-0.048611in}}{\pgfqpoint{0.000000in}{0.000000in}}{%
\pgfpathmoveto{\pgfqpoint{0.000000in}{0.000000in}}%
\pgfpathlineto{\pgfqpoint{0.000000in}{-0.048611in}}%
\pgfusepath{stroke,fill}%
}%
\begin{pgfscope}%
\pgfsys@transformshift{2.367461in}{0.387222in}%
\pgfsys@useobject{currentmarker}{}%
\end{pgfscope}%
\end{pgfscope}%
\begin{pgfscope}%
\definecolor{textcolor}{rgb}{0.000000,0.000000,0.000000}%
\pgfsetstrokecolor{textcolor}%
\pgfsetfillcolor{textcolor}%
\pgftext[x=2.367461in,y=0.290000in,,top]{\color{textcolor}\sffamily\fontsize{10.000000}{12.000000}\selectfont 4.45}%
\end{pgfscope}%
\begin{pgfscope}%
\pgfsetbuttcap%
\pgfsetroundjoin%
\definecolor{currentfill}{rgb}{0.000000,0.000000,0.000000}%
\pgfsetfillcolor{currentfill}%
\pgfsetlinewidth{0.803000pt}%
\definecolor{currentstroke}{rgb}{0.000000,0.000000,0.000000}%
\pgfsetstrokecolor{currentstroke}%
\pgfsetdash{}{0pt}%
\pgfsys@defobject{currentmarker}{\pgfqpoint{0.000000in}{-0.048611in}}{\pgfqpoint{0.000000in}{0.000000in}}{%
\pgfpathmoveto{\pgfqpoint{0.000000in}{0.000000in}}%
\pgfpathlineto{\pgfqpoint{0.000000in}{-0.048611in}}%
\pgfusepath{stroke,fill}%
}%
\begin{pgfscope}%
\pgfsys@transformshift{3.118448in}{0.387222in}%
\pgfsys@useobject{currentmarker}{}%
\end{pgfscope}%
\end{pgfscope}%
\begin{pgfscope}%
\definecolor{textcolor}{rgb}{0.000000,0.000000,0.000000}%
\pgfsetstrokecolor{textcolor}%
\pgfsetfillcolor{textcolor}%
\pgftext[x=3.118448in,y=0.290000in,,top]{\color{textcolor}\sffamily\fontsize{10.000000}{12.000000}\selectfont 4.50}%
\end{pgfscope}%
\begin{pgfscope}%
\pgfsetbuttcap%
\pgfsetroundjoin%
\definecolor{currentfill}{rgb}{0.000000,0.000000,0.000000}%
\pgfsetfillcolor{currentfill}%
\pgfsetlinewidth{0.803000pt}%
\definecolor{currentstroke}{rgb}{0.000000,0.000000,0.000000}%
\pgfsetstrokecolor{currentstroke}%
\pgfsetdash{}{0pt}%
\pgfsys@defobject{currentmarker}{\pgfqpoint{0.000000in}{-0.048611in}}{\pgfqpoint{0.000000in}{0.000000in}}{%
\pgfpathmoveto{\pgfqpoint{0.000000in}{0.000000in}}%
\pgfpathlineto{\pgfqpoint{0.000000in}{-0.048611in}}%
\pgfusepath{stroke,fill}%
}%
\begin{pgfscope}%
\pgfsys@transformshift{3.869435in}{0.387222in}%
\pgfsys@useobject{currentmarker}{}%
\end{pgfscope}%
\end{pgfscope}%
\begin{pgfscope}%
\definecolor{textcolor}{rgb}{0.000000,0.000000,0.000000}%
\pgfsetstrokecolor{textcolor}%
\pgfsetfillcolor{textcolor}%
\pgftext[x=3.869435in,y=0.290000in,,top]{\color{textcolor}\sffamily\fontsize{10.000000}{12.000000}\selectfont 4.55}%
\end{pgfscope}%
\begin{pgfscope}%
\pgfsetbuttcap%
\pgfsetroundjoin%
\definecolor{currentfill}{rgb}{0.000000,0.000000,0.000000}%
\pgfsetfillcolor{currentfill}%
\pgfsetlinewidth{0.803000pt}%
\definecolor{currentstroke}{rgb}{0.000000,0.000000,0.000000}%
\pgfsetstrokecolor{currentstroke}%
\pgfsetdash{}{0pt}%
\pgfsys@defobject{currentmarker}{\pgfqpoint{-0.048611in}{0.000000in}}{\pgfqpoint{0.000000in}{0.000000in}}{%
\pgfpathmoveto{\pgfqpoint{0.000000in}{0.000000in}}%
\pgfpathlineto{\pgfqpoint{-0.048611in}{0.000000in}}%
\pgfusepath{stroke,fill}%
}%
\begin{pgfscope}%
\pgfsys@transformshift{0.555972in}{0.465375in}%
\pgfsys@useobject{currentmarker}{}%
\end{pgfscope}%
\end{pgfscope}%
\begin{pgfscope}%
\definecolor{textcolor}{rgb}{0.000000,0.000000,0.000000}%
\pgfsetstrokecolor{textcolor}%
\pgfsetfillcolor{textcolor}%
\pgftext[x=0.149505in,y=0.412614in,left,base]{\color{textcolor}\sffamily\fontsize{10.000000}{12.000000}\selectfont 0.20}%
\end{pgfscope}%
\begin{pgfscope}%
\pgfsetbuttcap%
\pgfsetroundjoin%
\definecolor{currentfill}{rgb}{0.000000,0.000000,0.000000}%
\pgfsetfillcolor{currentfill}%
\pgfsetlinewidth{0.803000pt}%
\definecolor{currentstroke}{rgb}{0.000000,0.000000,0.000000}%
\pgfsetstrokecolor{currentstroke}%
\pgfsetdash{}{0pt}%
\pgfsys@defobject{currentmarker}{\pgfqpoint{-0.048611in}{0.000000in}}{\pgfqpoint{0.000000in}{0.000000in}}{%
\pgfpathmoveto{\pgfqpoint{0.000000in}{0.000000in}}%
\pgfpathlineto{\pgfqpoint{-0.048611in}{0.000000in}}%
\pgfusepath{stroke,fill}%
}%
\begin{pgfscope}%
\pgfsys@transformshift{0.555972in}{0.766906in}%
\pgfsys@useobject{currentmarker}{}%
\end{pgfscope}%
\end{pgfscope}%
\begin{pgfscope}%
\definecolor{textcolor}{rgb}{0.000000,0.000000,0.000000}%
\pgfsetstrokecolor{textcolor}%
\pgfsetfillcolor{textcolor}%
\pgftext[x=0.149505in,y=0.714144in,left,base]{\color{textcolor}\sffamily\fontsize{10.000000}{12.000000}\selectfont 0.25}%
\end{pgfscope}%
\begin{pgfscope}%
\pgfsetbuttcap%
\pgfsetroundjoin%
\definecolor{currentfill}{rgb}{0.000000,0.000000,0.000000}%
\pgfsetfillcolor{currentfill}%
\pgfsetlinewidth{0.803000pt}%
\definecolor{currentstroke}{rgb}{0.000000,0.000000,0.000000}%
\pgfsetstrokecolor{currentstroke}%
\pgfsetdash{}{0pt}%
\pgfsys@defobject{currentmarker}{\pgfqpoint{-0.048611in}{0.000000in}}{\pgfqpoint{0.000000in}{0.000000in}}{%
\pgfpathmoveto{\pgfqpoint{0.000000in}{0.000000in}}%
\pgfpathlineto{\pgfqpoint{-0.048611in}{0.000000in}}%
\pgfusepath{stroke,fill}%
}%
\begin{pgfscope}%
\pgfsys@transformshift{0.555972in}{1.068436in}%
\pgfsys@useobject{currentmarker}{}%
\end{pgfscope}%
\end{pgfscope}%
\begin{pgfscope}%
\definecolor{textcolor}{rgb}{0.000000,0.000000,0.000000}%
\pgfsetstrokecolor{textcolor}%
\pgfsetfillcolor{textcolor}%
\pgftext[x=0.149505in,y=1.015675in,left,base]{\color{textcolor}\sffamily\fontsize{10.000000}{12.000000}\selectfont 0.30}%
\end{pgfscope}%
\begin{pgfscope}%
\pgfsetbuttcap%
\pgfsetroundjoin%
\definecolor{currentfill}{rgb}{0.000000,0.000000,0.000000}%
\pgfsetfillcolor{currentfill}%
\pgfsetlinewidth{0.803000pt}%
\definecolor{currentstroke}{rgb}{0.000000,0.000000,0.000000}%
\pgfsetstrokecolor{currentstroke}%
\pgfsetdash{}{0pt}%
\pgfsys@defobject{currentmarker}{\pgfqpoint{-0.048611in}{0.000000in}}{\pgfqpoint{0.000000in}{0.000000in}}{%
\pgfpathmoveto{\pgfqpoint{0.000000in}{0.000000in}}%
\pgfpathlineto{\pgfqpoint{-0.048611in}{0.000000in}}%
\pgfusepath{stroke,fill}%
}%
\begin{pgfscope}%
\pgfsys@transformshift{0.555972in}{1.369967in}%
\pgfsys@useobject{currentmarker}{}%
\end{pgfscope}%
\end{pgfscope}%
\begin{pgfscope}%
\definecolor{textcolor}{rgb}{0.000000,0.000000,0.000000}%
\pgfsetstrokecolor{textcolor}%
\pgfsetfillcolor{textcolor}%
\pgftext[x=0.149505in,y=1.317206in,left,base]{\color{textcolor}\sffamily\fontsize{10.000000}{12.000000}\selectfont 0.35}%
\end{pgfscope}%
\begin{pgfscope}%
\pgfsetbuttcap%
\pgfsetroundjoin%
\definecolor{currentfill}{rgb}{0.000000,0.000000,0.000000}%
\pgfsetfillcolor{currentfill}%
\pgfsetlinewidth{0.803000pt}%
\definecolor{currentstroke}{rgb}{0.000000,0.000000,0.000000}%
\pgfsetstrokecolor{currentstroke}%
\pgfsetdash{}{0pt}%
\pgfsys@defobject{currentmarker}{\pgfqpoint{-0.048611in}{0.000000in}}{\pgfqpoint{0.000000in}{0.000000in}}{%
\pgfpathmoveto{\pgfqpoint{0.000000in}{0.000000in}}%
\pgfpathlineto{\pgfqpoint{-0.048611in}{0.000000in}}%
\pgfusepath{stroke,fill}%
}%
\begin{pgfscope}%
\pgfsys@transformshift{0.555972in}{1.671498in}%
\pgfsys@useobject{currentmarker}{}%
\end{pgfscope}%
\end{pgfscope}%
\begin{pgfscope}%
\definecolor{textcolor}{rgb}{0.000000,0.000000,0.000000}%
\pgfsetstrokecolor{textcolor}%
\pgfsetfillcolor{textcolor}%
\pgftext[x=0.149505in,y=1.618736in,left,base]{\color{textcolor}\sffamily\fontsize{10.000000}{12.000000}\selectfont 0.40}%
\end{pgfscope}%
\begin{pgfscope}%
\pgfsetbuttcap%
\pgfsetroundjoin%
\definecolor{currentfill}{rgb}{0.000000,0.000000,0.000000}%
\pgfsetfillcolor{currentfill}%
\pgfsetlinewidth{0.803000pt}%
\definecolor{currentstroke}{rgb}{0.000000,0.000000,0.000000}%
\pgfsetstrokecolor{currentstroke}%
\pgfsetdash{}{0pt}%
\pgfsys@defobject{currentmarker}{\pgfqpoint{-0.048611in}{0.000000in}}{\pgfqpoint{0.000000in}{0.000000in}}{%
\pgfpathmoveto{\pgfqpoint{0.000000in}{0.000000in}}%
\pgfpathlineto{\pgfqpoint{-0.048611in}{0.000000in}}%
\pgfusepath{stroke,fill}%
}%
\begin{pgfscope}%
\pgfsys@transformshift{0.555972in}{1.973028in}%
\pgfsys@useobject{currentmarker}{}%
\end{pgfscope}%
\end{pgfscope}%
\begin{pgfscope}%
\definecolor{textcolor}{rgb}{0.000000,0.000000,0.000000}%
\pgfsetstrokecolor{textcolor}%
\pgfsetfillcolor{textcolor}%
\pgftext[x=0.149505in,y=1.920267in,left,base]{\color{textcolor}\sffamily\fontsize{10.000000}{12.000000}\selectfont 0.45}%
\end{pgfscope}%
\begin{pgfscope}%
\pgfsetbuttcap%
\pgfsetroundjoin%
\definecolor{currentfill}{rgb}{0.000000,0.000000,0.000000}%
\pgfsetfillcolor{currentfill}%
\pgfsetlinewidth{0.803000pt}%
\definecolor{currentstroke}{rgb}{0.000000,0.000000,0.000000}%
\pgfsetstrokecolor{currentstroke}%
\pgfsetdash{}{0pt}%
\pgfsys@defobject{currentmarker}{\pgfqpoint{-0.048611in}{0.000000in}}{\pgfqpoint{0.000000in}{0.000000in}}{%
\pgfpathmoveto{\pgfqpoint{0.000000in}{0.000000in}}%
\pgfpathlineto{\pgfqpoint{-0.048611in}{0.000000in}}%
\pgfusepath{stroke,fill}%
}%
\begin{pgfscope}%
\pgfsys@transformshift{0.555972in}{2.274559in}%
\pgfsys@useobject{currentmarker}{}%
\end{pgfscope}%
\end{pgfscope}%
\begin{pgfscope}%
\definecolor{textcolor}{rgb}{0.000000,0.000000,0.000000}%
\pgfsetstrokecolor{textcolor}%
\pgfsetfillcolor{textcolor}%
\pgftext[x=0.149505in,y=2.221798in,left,base]{\color{textcolor}\sffamily\fontsize{10.000000}{12.000000}\selectfont 0.50}%
\end{pgfscope}%
\begin{pgfscope}%
\pgfsetbuttcap%
\pgfsetroundjoin%
\definecolor{currentfill}{rgb}{0.000000,0.000000,0.000000}%
\pgfsetfillcolor{currentfill}%
\pgfsetlinewidth{0.803000pt}%
\definecolor{currentstroke}{rgb}{0.000000,0.000000,0.000000}%
\pgfsetstrokecolor{currentstroke}%
\pgfsetdash{}{0pt}%
\pgfsys@defobject{currentmarker}{\pgfqpoint{-0.048611in}{0.000000in}}{\pgfqpoint{0.000000in}{0.000000in}}{%
\pgfpathmoveto{\pgfqpoint{0.000000in}{0.000000in}}%
\pgfpathlineto{\pgfqpoint{-0.048611in}{0.000000in}}%
\pgfusepath{stroke,fill}%
}%
\begin{pgfscope}%
\pgfsys@transformshift{0.555972in}{2.576090in}%
\pgfsys@useobject{currentmarker}{}%
\end{pgfscope}%
\end{pgfscope}%
\begin{pgfscope}%
\definecolor{textcolor}{rgb}{0.000000,0.000000,0.000000}%
\pgfsetstrokecolor{textcolor}%
\pgfsetfillcolor{textcolor}%
\pgftext[x=0.149505in,y=2.523328in,left,base]{\color{textcolor}\sffamily\fontsize{10.000000}{12.000000}\selectfont 0.55}%
\end{pgfscope}%
\begin{pgfscope}%
\pgfpathrectangle{\pgfqpoint{0.555972in}{0.387222in}}{\pgfqpoint{3.322583in}{2.324444in}}%
\pgfusepath{clip}%
\pgfsetrectcap%
\pgfsetroundjoin%
\pgfsetlinewidth{1.505625pt}%
\definecolor{currentstroke}{rgb}{0.121569,0.466667,0.705882}%
\pgfsetstrokecolor{currentstroke}%
\pgfsetdash{}{0pt}%
\pgfpathmoveto{\pgfqpoint{0.715290in}{2.593206in}}%
\pgfpathlineto{\pgfqpoint{0.865487in}{2.536245in}}%
\pgfpathlineto{\pgfqpoint{1.015684in}{2.472096in}}%
\pgfpathlineto{\pgfqpoint{1.165882in}{2.407322in}}%
\pgfpathlineto{\pgfqpoint{1.316079in}{2.339364in}}%
\pgfpathlineto{\pgfqpoint{1.466277in}{2.253253in}}%
\pgfpathlineto{\pgfqpoint{1.616474in}{2.163237in}}%
\pgfpathlineto{\pgfqpoint{1.766671in}{2.100907in}}%
\pgfpathlineto{\pgfqpoint{1.916869in}{1.998547in}}%
\pgfpathlineto{\pgfqpoint{2.067066in}{1.900900in}}%
\pgfpathlineto{\pgfqpoint{2.217264in}{1.776707in}}%
\pgfpathlineto{\pgfqpoint{2.367461in}{1.643148in}}%
\pgfpathlineto{\pgfqpoint{2.517659in}{1.525508in}}%
\pgfpathlineto{\pgfqpoint{2.667856in}{1.429459in}}%
\pgfpathlineto{\pgfqpoint{2.818053in}{1.265873in}}%
\pgfpathlineto{\pgfqpoint{2.968251in}{1.144032in}}%
\pgfpathlineto{\pgfqpoint{3.118448in}{1.001991in}}%
\pgfpathlineto{\pgfqpoint{3.268646in}{0.871239in}}%
\pgfpathlineto{\pgfqpoint{3.418843in}{0.763380in}}%
\pgfpathlineto{\pgfqpoint{3.569040in}{0.627659in}}%
\pgfpathlineto{\pgfqpoint{3.719238in}{0.523417in}}%
\pgfusepath{stroke}%
\end{pgfscope}%
\begin{pgfscope}%
\pgfsetrectcap%
\pgfsetmiterjoin%
\pgfsetlinewidth{0.803000pt}%
\definecolor{currentstroke}{rgb}{0.000000,0.000000,0.000000}%
\pgfsetstrokecolor{currentstroke}%
\pgfsetdash{}{0pt}%
\pgfpathmoveto{\pgfqpoint{0.555972in}{0.387222in}}%
\pgfpathlineto{\pgfqpoint{0.555972in}{2.711667in}}%
\pgfusepath{stroke}%
\end{pgfscope}%
\begin{pgfscope}%
\pgfsetrectcap%
\pgfsetmiterjoin%
\pgfsetlinewidth{0.803000pt}%
\definecolor{currentstroke}{rgb}{0.000000,0.000000,0.000000}%
\pgfsetstrokecolor{currentstroke}%
\pgfsetdash{}{0pt}%
\pgfpathmoveto{\pgfqpoint{3.878555in}{0.387222in}}%
\pgfpathlineto{\pgfqpoint{3.878555in}{2.711667in}}%
\pgfusepath{stroke}%
\end{pgfscope}%
\begin{pgfscope}%
\pgfsetrectcap%
\pgfsetmiterjoin%
\pgfsetlinewidth{0.803000pt}%
\definecolor{currentstroke}{rgb}{0.000000,0.000000,0.000000}%
\pgfsetstrokecolor{currentstroke}%
\pgfsetdash{}{0pt}%
\pgfpathmoveto{\pgfqpoint{0.555972in}{0.387222in}}%
\pgfpathlineto{\pgfqpoint{3.878555in}{0.387222in}}%
\pgfusepath{stroke}%
\end{pgfscope}%
\begin{pgfscope}%
\pgfsetrectcap%
\pgfsetmiterjoin%
\pgfsetlinewidth{0.803000pt}%
\definecolor{currentstroke}{rgb}{0.000000,0.000000,0.000000}%
\pgfsetstrokecolor{currentstroke}%
\pgfsetdash{}{0pt}%
\pgfpathmoveto{\pgfqpoint{0.555972in}{2.711667in}}%
\pgfpathlineto{\pgfqpoint{3.878555in}{2.711667in}}%
\pgfusepath{stroke}%
\end{pgfscope}%
\begin{pgfscope}%
\definecolor{textcolor}{rgb}{0.000000,0.000000,0.000000}%
\pgfsetstrokecolor{textcolor}%
\pgfsetfillcolor{textcolor}%
\pgftext[x=2.217264in,y=2.795000in,,base]{\color{textcolor}\sffamily\fontsize{12.000000}{14.400000}\selectfont \(\displaystyle  N = 16 \)}%
\end{pgfscope}%
\begin{pgfscope}%
\pgfsetbuttcap%
\pgfsetmiterjoin%
\definecolor{currentfill}{rgb}{1.000000,1.000000,1.000000}%
\pgfsetfillcolor{currentfill}%
\pgfsetlinewidth{0.000000pt}%
\definecolor{currentstroke}{rgb}{0.000000,0.000000,0.000000}%
\pgfsetstrokecolor{currentstroke}%
\pgfsetstrokeopacity{0.000000}%
\pgfsetdash{}{0pt}%
\pgfpathmoveto{\pgfqpoint{4.492417in}{0.387222in}}%
\pgfpathlineto{\pgfqpoint{7.815000in}{0.387222in}}%
\pgfpathlineto{\pgfqpoint{7.815000in}{2.711667in}}%
\pgfpathlineto{\pgfqpoint{4.492417in}{2.711667in}}%
\pgfpathclose%
\pgfusepath{fill}%
\end{pgfscope}%
\begin{pgfscope}%
\pgfpathrectangle{\pgfqpoint{4.492417in}{0.387222in}}{\pgfqpoint{3.322583in}{2.324444in}}%
\pgfusepath{clip}%
\pgfsetbuttcap%
\pgfsetroundjoin%
\definecolor{currentfill}{rgb}{0.121569,0.466667,0.705882}%
\pgfsetfillcolor{currentfill}%
\pgfsetfillopacity{0.300000}%
\pgfsetlinewidth{1.003750pt}%
\definecolor{currentstroke}{rgb}{0.121569,0.466667,0.705882}%
\pgfsetstrokecolor{currentstroke}%
\pgfsetstrokeopacity{0.300000}%
\pgfsetdash{}{0pt}%
\pgfpathmoveto{\pgfqpoint{4.651734in}{2.604432in}}%
\pgfpathlineto{\pgfqpoint{4.651734in}{2.591946in}}%
\pgfpathlineto{\pgfqpoint{4.801932in}{2.524048in}}%
\pgfpathlineto{\pgfqpoint{4.952129in}{2.468748in}}%
\pgfpathlineto{\pgfqpoint{5.102327in}{2.404510in}}%
\pgfpathlineto{\pgfqpoint{5.252524in}{2.311050in}}%
\pgfpathlineto{\pgfqpoint{5.402721in}{2.222722in}}%
\pgfpathlineto{\pgfqpoint{5.552919in}{2.145262in}}%
\pgfpathlineto{\pgfqpoint{5.703116in}{2.059815in}}%
\pgfpathlineto{\pgfqpoint{5.853314in}{1.882681in}}%
\pgfpathlineto{\pgfqpoint{6.003511in}{1.659306in}}%
\pgfpathlineto{\pgfqpoint{6.153708in}{1.532207in}}%
\pgfpathlineto{\pgfqpoint{6.303906in}{1.431786in}}%
\pgfpathlineto{\pgfqpoint{6.454103in}{1.224044in}}%
\pgfpathlineto{\pgfqpoint{6.604301in}{1.059552in}}%
\pgfpathlineto{\pgfqpoint{6.754498in}{0.807475in}}%
\pgfpathlineto{\pgfqpoint{6.904696in}{0.794919in}}%
\pgfpathlineto{\pgfqpoint{7.054893in}{0.712201in}}%
\pgfpathlineto{\pgfqpoint{7.205090in}{0.591630in}}%
\pgfpathlineto{\pgfqpoint{7.355288in}{0.569899in}}%
\pgfpathlineto{\pgfqpoint{7.505485in}{0.548511in}}%
\pgfpathlineto{\pgfqpoint{7.655683in}{0.492879in}}%
\pgfpathlineto{\pgfqpoint{7.655683in}{0.587738in}}%
\pgfpathlineto{\pgfqpoint{7.655683in}{0.587738in}}%
\pgfpathlineto{\pgfqpoint{7.505485in}{0.624698in}}%
\pgfpathlineto{\pgfqpoint{7.355288in}{0.704551in}}%
\pgfpathlineto{\pgfqpoint{7.205090in}{0.681423in}}%
\pgfpathlineto{\pgfqpoint{7.054893in}{0.822050in}}%
\pgfpathlineto{\pgfqpoint{6.904696in}{0.968468in}}%
\pgfpathlineto{\pgfqpoint{6.754498in}{1.138743in}}%
\pgfpathlineto{\pgfqpoint{6.604301in}{1.281809in}}%
\pgfpathlineto{\pgfqpoint{6.454103in}{1.287274in}}%
\pgfpathlineto{\pgfqpoint{6.303906in}{1.505983in}}%
\pgfpathlineto{\pgfqpoint{6.153708in}{1.754038in}}%
\pgfpathlineto{\pgfqpoint{6.003511in}{1.813558in}}%
\pgfpathlineto{\pgfqpoint{5.853314in}{1.971883in}}%
\pgfpathlineto{\pgfqpoint{5.703116in}{2.099858in}}%
\pgfpathlineto{\pgfqpoint{5.552919in}{2.212103in}}%
\pgfpathlineto{\pgfqpoint{5.402721in}{2.261699in}}%
\pgfpathlineto{\pgfqpoint{5.252524in}{2.364106in}}%
\pgfpathlineto{\pgfqpoint{5.102327in}{2.424678in}}%
\pgfpathlineto{\pgfqpoint{4.952129in}{2.495980in}}%
\pgfpathlineto{\pgfqpoint{4.801932in}{2.553097in}}%
\pgfpathlineto{\pgfqpoint{4.651734in}{2.604432in}}%
\pgfpathclose%
\pgfusepath{stroke,fill}%
\end{pgfscope}%
\begin{pgfscope}%
\pgfpathrectangle{\pgfqpoint{4.492417in}{0.387222in}}{\pgfqpoint{3.322583in}{2.324444in}}%
\pgfusepath{clip}%
\pgfsetbuttcap%
\pgfsetroundjoin%
\definecolor{currentfill}{rgb}{0.121569,0.466667,0.705882}%
\pgfsetfillcolor{currentfill}%
\pgfsetlinewidth{1.003750pt}%
\definecolor{currentstroke}{rgb}{0.121569,0.466667,0.705882}%
\pgfsetstrokecolor{currentstroke}%
\pgfsetdash{}{0pt}%
\pgfsys@defobject{currentmarker}{\pgfqpoint{-0.009821in}{-0.009821in}}{\pgfqpoint{0.009821in}{0.009821in}}{%
\pgfpathmoveto{\pgfqpoint{0.000000in}{-0.009821in}}%
\pgfpathcurveto{\pgfqpoint{0.002605in}{-0.009821in}}{\pgfqpoint{0.005103in}{-0.008786in}}{\pgfqpoint{0.006944in}{-0.006944in}}%
\pgfpathcurveto{\pgfqpoint{0.008786in}{-0.005103in}}{\pgfqpoint{0.009821in}{-0.002605in}}{\pgfqpoint{0.009821in}{0.000000in}}%
\pgfpathcurveto{\pgfqpoint{0.009821in}{0.002605in}}{\pgfqpoint{0.008786in}{0.005103in}}{\pgfqpoint{0.006944in}{0.006944in}}%
\pgfpathcurveto{\pgfqpoint{0.005103in}{0.008786in}}{\pgfqpoint{0.002605in}{0.009821in}}{\pgfqpoint{0.000000in}{0.009821in}}%
\pgfpathcurveto{\pgfqpoint{-0.002605in}{0.009821in}}{\pgfqpoint{-0.005103in}{0.008786in}}{\pgfqpoint{-0.006944in}{0.006944in}}%
\pgfpathcurveto{\pgfqpoint{-0.008786in}{0.005103in}}{\pgfqpoint{-0.009821in}{0.002605in}}{\pgfqpoint{-0.009821in}{0.000000in}}%
\pgfpathcurveto{\pgfqpoint{-0.009821in}{-0.002605in}}{\pgfqpoint{-0.008786in}{-0.005103in}}{\pgfqpoint{-0.006944in}{-0.006944in}}%
\pgfpathcurveto{\pgfqpoint{-0.005103in}{-0.008786in}}{\pgfqpoint{-0.002605in}{-0.009821in}}{\pgfqpoint{0.000000in}{-0.009821in}}%
\pgfpathclose%
\pgfusepath{stroke,fill}%
}%
\begin{pgfscope}%
\pgfsys@transformshift{4.651734in}{2.598189in}%
\pgfsys@useobject{currentmarker}{}%
\end{pgfscope}%
\begin{pgfscope}%
\pgfsys@transformshift{4.801932in}{2.538573in}%
\pgfsys@useobject{currentmarker}{}%
\end{pgfscope}%
\begin{pgfscope}%
\pgfsys@transformshift{4.952129in}{2.482364in}%
\pgfsys@useobject{currentmarker}{}%
\end{pgfscope}%
\begin{pgfscope}%
\pgfsys@transformshift{5.102327in}{2.414594in}%
\pgfsys@useobject{currentmarker}{}%
\end{pgfscope}%
\begin{pgfscope}%
\pgfsys@transformshift{5.252524in}{2.337578in}%
\pgfsys@useobject{currentmarker}{}%
\end{pgfscope}%
\begin{pgfscope}%
\pgfsys@transformshift{5.402721in}{2.242210in}%
\pgfsys@useobject{currentmarker}{}%
\end{pgfscope}%
\begin{pgfscope}%
\pgfsys@transformshift{5.552919in}{2.178683in}%
\pgfsys@useobject{currentmarker}{}%
\end{pgfscope}%
\begin{pgfscope}%
\pgfsys@transformshift{5.703116in}{2.079836in}%
\pgfsys@useobject{currentmarker}{}%
\end{pgfscope}%
\begin{pgfscope}%
\pgfsys@transformshift{5.853314in}{1.927282in}%
\pgfsys@useobject{currentmarker}{}%
\end{pgfscope}%
\begin{pgfscope}%
\pgfsys@transformshift{6.003511in}{1.736432in}%
\pgfsys@useobject{currentmarker}{}%
\end{pgfscope}%
\begin{pgfscope}%
\pgfsys@transformshift{6.153708in}{1.643123in}%
\pgfsys@useobject{currentmarker}{}%
\end{pgfscope}%
\begin{pgfscope}%
\pgfsys@transformshift{6.303906in}{1.468885in}%
\pgfsys@useobject{currentmarker}{}%
\end{pgfscope}%
\begin{pgfscope}%
\pgfsys@transformshift{6.454103in}{1.255659in}%
\pgfsys@useobject{currentmarker}{}%
\end{pgfscope}%
\begin{pgfscope}%
\pgfsys@transformshift{6.604301in}{1.170680in}%
\pgfsys@useobject{currentmarker}{}%
\end{pgfscope}%
\begin{pgfscope}%
\pgfsys@transformshift{6.754498in}{0.973109in}%
\pgfsys@useobject{currentmarker}{}%
\end{pgfscope}%
\begin{pgfscope}%
\pgfsys@transformshift{6.904696in}{0.881693in}%
\pgfsys@useobject{currentmarker}{}%
\end{pgfscope}%
\begin{pgfscope}%
\pgfsys@transformshift{7.054893in}{0.767126in}%
\pgfsys@useobject{currentmarker}{}%
\end{pgfscope}%
\begin{pgfscope}%
\pgfsys@transformshift{7.205090in}{0.636527in}%
\pgfsys@useobject{currentmarker}{}%
\end{pgfscope}%
\begin{pgfscope}%
\pgfsys@transformshift{7.355288in}{0.637225in}%
\pgfsys@useobject{currentmarker}{}%
\end{pgfscope}%
\begin{pgfscope}%
\pgfsys@transformshift{7.505485in}{0.586604in}%
\pgfsys@useobject{currentmarker}{}%
\end{pgfscope}%
\begin{pgfscope}%
\pgfsys@transformshift{7.655683in}{0.540309in}%
\pgfsys@useobject{currentmarker}{}%
\end{pgfscope}%
\end{pgfscope}%
\begin{pgfscope}%
\pgfsetbuttcap%
\pgfsetroundjoin%
\definecolor{currentfill}{rgb}{0.000000,0.000000,0.000000}%
\pgfsetfillcolor{currentfill}%
\pgfsetlinewidth{0.803000pt}%
\definecolor{currentstroke}{rgb}{0.000000,0.000000,0.000000}%
\pgfsetstrokecolor{currentstroke}%
\pgfsetdash{}{0pt}%
\pgfsys@defobject{currentmarker}{\pgfqpoint{0.000000in}{-0.048611in}}{\pgfqpoint{0.000000in}{0.000000in}}{%
\pgfpathmoveto{\pgfqpoint{0.000000in}{0.000000in}}%
\pgfpathlineto{\pgfqpoint{0.000000in}{-0.048611in}}%
\pgfusepath{stroke,fill}%
}%
\begin{pgfscope}%
\pgfsys@transformshift{4.952129in}{0.387222in}%
\pgfsys@useobject{currentmarker}{}%
\end{pgfscope}%
\end{pgfscope}%
\begin{pgfscope}%
\definecolor{textcolor}{rgb}{0.000000,0.000000,0.000000}%
\pgfsetstrokecolor{textcolor}%
\pgfsetfillcolor{textcolor}%
\pgftext[x=4.952129in,y=0.290000in,,top]{\color{textcolor}\sffamily\fontsize{10.000000}{12.000000}\selectfont 4.40}%
\end{pgfscope}%
\begin{pgfscope}%
\pgfsetbuttcap%
\pgfsetroundjoin%
\definecolor{currentfill}{rgb}{0.000000,0.000000,0.000000}%
\pgfsetfillcolor{currentfill}%
\pgfsetlinewidth{0.803000pt}%
\definecolor{currentstroke}{rgb}{0.000000,0.000000,0.000000}%
\pgfsetstrokecolor{currentstroke}%
\pgfsetdash{}{0pt}%
\pgfsys@defobject{currentmarker}{\pgfqpoint{0.000000in}{-0.048611in}}{\pgfqpoint{0.000000in}{0.000000in}}{%
\pgfpathmoveto{\pgfqpoint{0.000000in}{0.000000in}}%
\pgfpathlineto{\pgfqpoint{0.000000in}{-0.048611in}}%
\pgfusepath{stroke,fill}%
}%
\begin{pgfscope}%
\pgfsys@transformshift{5.703116in}{0.387222in}%
\pgfsys@useobject{currentmarker}{}%
\end{pgfscope}%
\end{pgfscope}%
\begin{pgfscope}%
\definecolor{textcolor}{rgb}{0.000000,0.000000,0.000000}%
\pgfsetstrokecolor{textcolor}%
\pgfsetfillcolor{textcolor}%
\pgftext[x=5.703116in,y=0.290000in,,top]{\color{textcolor}\sffamily\fontsize{10.000000}{12.000000}\selectfont 4.45}%
\end{pgfscope}%
\begin{pgfscope}%
\pgfsetbuttcap%
\pgfsetroundjoin%
\definecolor{currentfill}{rgb}{0.000000,0.000000,0.000000}%
\pgfsetfillcolor{currentfill}%
\pgfsetlinewidth{0.803000pt}%
\definecolor{currentstroke}{rgb}{0.000000,0.000000,0.000000}%
\pgfsetstrokecolor{currentstroke}%
\pgfsetdash{}{0pt}%
\pgfsys@defobject{currentmarker}{\pgfqpoint{0.000000in}{-0.048611in}}{\pgfqpoint{0.000000in}{0.000000in}}{%
\pgfpathmoveto{\pgfqpoint{0.000000in}{0.000000in}}%
\pgfpathlineto{\pgfqpoint{0.000000in}{-0.048611in}}%
\pgfusepath{stroke,fill}%
}%
\begin{pgfscope}%
\pgfsys@transformshift{6.454103in}{0.387222in}%
\pgfsys@useobject{currentmarker}{}%
\end{pgfscope}%
\end{pgfscope}%
\begin{pgfscope}%
\definecolor{textcolor}{rgb}{0.000000,0.000000,0.000000}%
\pgfsetstrokecolor{textcolor}%
\pgfsetfillcolor{textcolor}%
\pgftext[x=6.454103in,y=0.290000in,,top]{\color{textcolor}\sffamily\fontsize{10.000000}{12.000000}\selectfont 4.50}%
\end{pgfscope}%
\begin{pgfscope}%
\pgfsetbuttcap%
\pgfsetroundjoin%
\definecolor{currentfill}{rgb}{0.000000,0.000000,0.000000}%
\pgfsetfillcolor{currentfill}%
\pgfsetlinewidth{0.803000pt}%
\definecolor{currentstroke}{rgb}{0.000000,0.000000,0.000000}%
\pgfsetstrokecolor{currentstroke}%
\pgfsetdash{}{0pt}%
\pgfsys@defobject{currentmarker}{\pgfqpoint{0.000000in}{-0.048611in}}{\pgfqpoint{0.000000in}{0.000000in}}{%
\pgfpathmoveto{\pgfqpoint{0.000000in}{0.000000in}}%
\pgfpathlineto{\pgfqpoint{0.000000in}{-0.048611in}}%
\pgfusepath{stroke,fill}%
}%
\begin{pgfscope}%
\pgfsys@transformshift{7.205090in}{0.387222in}%
\pgfsys@useobject{currentmarker}{}%
\end{pgfscope}%
\end{pgfscope}%
\begin{pgfscope}%
\definecolor{textcolor}{rgb}{0.000000,0.000000,0.000000}%
\pgfsetstrokecolor{textcolor}%
\pgfsetfillcolor{textcolor}%
\pgftext[x=7.205090in,y=0.290000in,,top]{\color{textcolor}\sffamily\fontsize{10.000000}{12.000000}\selectfont 4.55}%
\end{pgfscope}%
\begin{pgfscope}%
\pgfsetbuttcap%
\pgfsetroundjoin%
\definecolor{currentfill}{rgb}{0.000000,0.000000,0.000000}%
\pgfsetfillcolor{currentfill}%
\pgfsetlinewidth{0.803000pt}%
\definecolor{currentstroke}{rgb}{0.000000,0.000000,0.000000}%
\pgfsetstrokecolor{currentstroke}%
\pgfsetdash{}{0pt}%
\pgfsys@defobject{currentmarker}{\pgfqpoint{-0.048611in}{0.000000in}}{\pgfqpoint{0.000000in}{0.000000in}}{%
\pgfpathmoveto{\pgfqpoint{0.000000in}{0.000000in}}%
\pgfpathlineto{\pgfqpoint{-0.048611in}{0.000000in}}%
\pgfusepath{stroke,fill}%
}%
\begin{pgfscope}%
\pgfsys@transformshift{4.492417in}{0.559943in}%
\pgfsys@useobject{currentmarker}{}%
\end{pgfscope}%
\end{pgfscope}%
\begin{pgfscope}%
\definecolor{textcolor}{rgb}{0.000000,0.000000,0.000000}%
\pgfsetstrokecolor{textcolor}%
\pgfsetfillcolor{textcolor}%
\pgftext[x=4.174315in,y=0.507182in,left,base]{\color{textcolor}\sffamily\fontsize{10.000000}{12.000000}\selectfont 0.1}%
\end{pgfscope}%
\begin{pgfscope}%
\pgfsetbuttcap%
\pgfsetroundjoin%
\definecolor{currentfill}{rgb}{0.000000,0.000000,0.000000}%
\pgfsetfillcolor{currentfill}%
\pgfsetlinewidth{0.803000pt}%
\definecolor{currentstroke}{rgb}{0.000000,0.000000,0.000000}%
\pgfsetstrokecolor{currentstroke}%
\pgfsetdash{}{0pt}%
\pgfsys@defobject{currentmarker}{\pgfqpoint{-0.048611in}{0.000000in}}{\pgfqpoint{0.000000in}{0.000000in}}{%
\pgfpathmoveto{\pgfqpoint{0.000000in}{0.000000in}}%
\pgfpathlineto{\pgfqpoint{-0.048611in}{0.000000in}}%
\pgfusepath{stroke,fill}%
}%
\begin{pgfscope}%
\pgfsys@transformshift{4.492417in}{1.057152in}%
\pgfsys@useobject{currentmarker}{}%
\end{pgfscope}%
\end{pgfscope}%
\begin{pgfscope}%
\definecolor{textcolor}{rgb}{0.000000,0.000000,0.000000}%
\pgfsetstrokecolor{textcolor}%
\pgfsetfillcolor{textcolor}%
\pgftext[x=4.174315in,y=1.004391in,left,base]{\color{textcolor}\sffamily\fontsize{10.000000}{12.000000}\selectfont 0.2}%
\end{pgfscope}%
\begin{pgfscope}%
\pgfsetbuttcap%
\pgfsetroundjoin%
\definecolor{currentfill}{rgb}{0.000000,0.000000,0.000000}%
\pgfsetfillcolor{currentfill}%
\pgfsetlinewidth{0.803000pt}%
\definecolor{currentstroke}{rgb}{0.000000,0.000000,0.000000}%
\pgfsetstrokecolor{currentstroke}%
\pgfsetdash{}{0pt}%
\pgfsys@defobject{currentmarker}{\pgfqpoint{-0.048611in}{0.000000in}}{\pgfqpoint{0.000000in}{0.000000in}}{%
\pgfpathmoveto{\pgfqpoint{0.000000in}{0.000000in}}%
\pgfpathlineto{\pgfqpoint{-0.048611in}{0.000000in}}%
\pgfusepath{stroke,fill}%
}%
\begin{pgfscope}%
\pgfsys@transformshift{4.492417in}{1.554361in}%
\pgfsys@useobject{currentmarker}{}%
\end{pgfscope}%
\end{pgfscope}%
\begin{pgfscope}%
\definecolor{textcolor}{rgb}{0.000000,0.000000,0.000000}%
\pgfsetstrokecolor{textcolor}%
\pgfsetfillcolor{textcolor}%
\pgftext[x=4.174315in,y=1.501600in,left,base]{\color{textcolor}\sffamily\fontsize{10.000000}{12.000000}\selectfont 0.3}%
\end{pgfscope}%
\begin{pgfscope}%
\pgfsetbuttcap%
\pgfsetroundjoin%
\definecolor{currentfill}{rgb}{0.000000,0.000000,0.000000}%
\pgfsetfillcolor{currentfill}%
\pgfsetlinewidth{0.803000pt}%
\definecolor{currentstroke}{rgb}{0.000000,0.000000,0.000000}%
\pgfsetstrokecolor{currentstroke}%
\pgfsetdash{}{0pt}%
\pgfsys@defobject{currentmarker}{\pgfqpoint{-0.048611in}{0.000000in}}{\pgfqpoint{0.000000in}{0.000000in}}{%
\pgfpathmoveto{\pgfqpoint{0.000000in}{0.000000in}}%
\pgfpathlineto{\pgfqpoint{-0.048611in}{0.000000in}}%
\pgfusepath{stroke,fill}%
}%
\begin{pgfscope}%
\pgfsys@transformshift{4.492417in}{2.051570in}%
\pgfsys@useobject{currentmarker}{}%
\end{pgfscope}%
\end{pgfscope}%
\begin{pgfscope}%
\definecolor{textcolor}{rgb}{0.000000,0.000000,0.000000}%
\pgfsetstrokecolor{textcolor}%
\pgfsetfillcolor{textcolor}%
\pgftext[x=4.174315in,y=1.998809in,left,base]{\color{textcolor}\sffamily\fontsize{10.000000}{12.000000}\selectfont 0.4}%
\end{pgfscope}%
\begin{pgfscope}%
\pgfsetbuttcap%
\pgfsetroundjoin%
\definecolor{currentfill}{rgb}{0.000000,0.000000,0.000000}%
\pgfsetfillcolor{currentfill}%
\pgfsetlinewidth{0.803000pt}%
\definecolor{currentstroke}{rgb}{0.000000,0.000000,0.000000}%
\pgfsetstrokecolor{currentstroke}%
\pgfsetdash{}{0pt}%
\pgfsys@defobject{currentmarker}{\pgfqpoint{-0.048611in}{0.000000in}}{\pgfqpoint{0.000000in}{0.000000in}}{%
\pgfpathmoveto{\pgfqpoint{0.000000in}{0.000000in}}%
\pgfpathlineto{\pgfqpoint{-0.048611in}{0.000000in}}%
\pgfusepath{stroke,fill}%
}%
\begin{pgfscope}%
\pgfsys@transformshift{4.492417in}{2.548779in}%
\pgfsys@useobject{currentmarker}{}%
\end{pgfscope}%
\end{pgfscope}%
\begin{pgfscope}%
\definecolor{textcolor}{rgb}{0.000000,0.000000,0.000000}%
\pgfsetstrokecolor{textcolor}%
\pgfsetfillcolor{textcolor}%
\pgftext[x=4.174315in,y=2.496018in,left,base]{\color{textcolor}\sffamily\fontsize{10.000000}{12.000000}\selectfont 0.5}%
\end{pgfscope}%
\begin{pgfscope}%
\pgfpathrectangle{\pgfqpoint{4.492417in}{0.387222in}}{\pgfqpoint{3.322583in}{2.324444in}}%
\pgfusepath{clip}%
\pgfsetrectcap%
\pgfsetroundjoin%
\pgfsetlinewidth{1.505625pt}%
\definecolor{currentstroke}{rgb}{0.121569,0.466667,0.705882}%
\pgfsetstrokecolor{currentstroke}%
\pgfsetdash{}{0pt}%
\pgfpathmoveto{\pgfqpoint{4.651734in}{2.598189in}}%
\pgfpathlineto{\pgfqpoint{4.801932in}{2.538573in}}%
\pgfpathlineto{\pgfqpoint{4.952129in}{2.482364in}}%
\pgfpathlineto{\pgfqpoint{5.102327in}{2.414594in}}%
\pgfpathlineto{\pgfqpoint{5.252524in}{2.337578in}}%
\pgfpathlineto{\pgfqpoint{5.402721in}{2.242210in}}%
\pgfpathlineto{\pgfqpoint{5.552919in}{2.178683in}}%
\pgfpathlineto{\pgfqpoint{5.703116in}{2.079836in}}%
\pgfpathlineto{\pgfqpoint{5.853314in}{1.927282in}}%
\pgfpathlineto{\pgfqpoint{6.003511in}{1.736432in}}%
\pgfpathlineto{\pgfqpoint{6.153708in}{1.643123in}}%
\pgfpathlineto{\pgfqpoint{6.303906in}{1.468885in}}%
\pgfpathlineto{\pgfqpoint{6.454103in}{1.255659in}}%
\pgfpathlineto{\pgfqpoint{6.604301in}{1.170680in}}%
\pgfpathlineto{\pgfqpoint{6.754498in}{0.973109in}}%
\pgfpathlineto{\pgfqpoint{6.904696in}{0.881693in}}%
\pgfpathlineto{\pgfqpoint{7.054893in}{0.767126in}}%
\pgfpathlineto{\pgfqpoint{7.205090in}{0.636527in}}%
\pgfpathlineto{\pgfqpoint{7.355288in}{0.637225in}}%
\pgfpathlineto{\pgfqpoint{7.505485in}{0.586604in}}%
\pgfpathlineto{\pgfqpoint{7.655683in}{0.540309in}}%
\pgfusepath{stroke}%
\end{pgfscope}%
\begin{pgfscope}%
\pgfsetrectcap%
\pgfsetmiterjoin%
\pgfsetlinewidth{0.803000pt}%
\definecolor{currentstroke}{rgb}{0.000000,0.000000,0.000000}%
\pgfsetstrokecolor{currentstroke}%
\pgfsetdash{}{0pt}%
\pgfpathmoveto{\pgfqpoint{4.492417in}{0.387222in}}%
\pgfpathlineto{\pgfqpoint{4.492417in}{2.711667in}}%
\pgfusepath{stroke}%
\end{pgfscope}%
\begin{pgfscope}%
\pgfsetrectcap%
\pgfsetmiterjoin%
\pgfsetlinewidth{0.803000pt}%
\definecolor{currentstroke}{rgb}{0.000000,0.000000,0.000000}%
\pgfsetstrokecolor{currentstroke}%
\pgfsetdash{}{0pt}%
\pgfpathmoveto{\pgfqpoint{7.815000in}{0.387222in}}%
\pgfpathlineto{\pgfqpoint{7.815000in}{2.711667in}}%
\pgfusepath{stroke}%
\end{pgfscope}%
\begin{pgfscope}%
\pgfsetrectcap%
\pgfsetmiterjoin%
\pgfsetlinewidth{0.803000pt}%
\definecolor{currentstroke}{rgb}{0.000000,0.000000,0.000000}%
\pgfsetstrokecolor{currentstroke}%
\pgfsetdash{}{0pt}%
\pgfpathmoveto{\pgfqpoint{4.492417in}{0.387222in}}%
\pgfpathlineto{\pgfqpoint{7.815000in}{0.387222in}}%
\pgfusepath{stroke}%
\end{pgfscope}%
\begin{pgfscope}%
\pgfsetrectcap%
\pgfsetmiterjoin%
\pgfsetlinewidth{0.803000pt}%
\definecolor{currentstroke}{rgb}{0.000000,0.000000,0.000000}%
\pgfsetstrokecolor{currentstroke}%
\pgfsetdash{}{0pt}%
\pgfpathmoveto{\pgfqpoint{4.492417in}{2.711667in}}%
\pgfpathlineto{\pgfqpoint{7.815000in}{2.711667in}}%
\pgfusepath{stroke}%
\end{pgfscope}%
\begin{pgfscope}%
\definecolor{textcolor}{rgb}{0.000000,0.000000,0.000000}%
\pgfsetstrokecolor{textcolor}%
\pgfsetfillcolor{textcolor}%
\pgftext[x=6.153708in,y=2.795000in,,base]{\color{textcolor}\sffamily\fontsize{12.000000}{14.400000}\selectfont \(\displaystyle  N = 24 \)}%
\end{pgfscope}%
\end{pgfpicture}%
\makeatother%
\endgroup%
}
\caption{Magnetization $m$ for finer temperature $T$ for the 3-D problem using kinetic Monte Carlo algorithm}
\label{Fig:MagBig3D}
\end{figure}

The behavior of the 3-D case is rather similar to the 2-D one: near the critical temperature, the internal energy suddenly jumps up and there is a sharp maximum of heat capacity near the critical temperature, symboling the phase transition. From Figure \ref{Fig:CapBig3D}, we conclude that the critical temperatures are about 4.35, 4.43, 4.45, 4.47 for $ N = 8, 12, 16, 24 $. The phenomenon of varying temperature differs from the 2-D case. The limiting critical temperature is about $ T_{\text{c}} \approx 4.511536 $, which is again close to our numerical simulation. The finite number of $N$ may account for the difference.

\section{Conclusion}

We have verified the phase transition in the 2-D and 3-D Ising model, by investigating the internal energy $u$, specific heat $c$, magnetization $m$. The estimated critical temperature of the 2-D case is about 2.27 and the analytical solution is $ T_{\text{c}} \approx 2.268185 $. The estimated critical temperature of the 3-D case is about 4.47 and the more precise solution is $ T_{\text{c}} \approx 4.511536 $. Our numerical results fit quite well.

The main difficulty lies in the computational side. The Markov Chain Monte Carlo algorithm it self requires a huge amount of computation to both convergence and reduction in variance. As a result, we adopt the kinetic Monte Carlo algorithm, implement the program in C, uses OpenMP to perform parallel sampling, and finally apply Intel MKL RNG routines to enhance the numerical results. We also apply some data structure techniques in kinetic Monte Carlo to keeps the transition step with time complexity $ O \rbr{1} $. However, the whole experiments takes a long time of 10 hours on a personal 4-core machine.

\end{document}
