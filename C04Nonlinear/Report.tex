%! TeX encoding = UTF-8
%! TeX program = LuaLaTeX

\documentclass[english, nochinese]{pnote}
\usepackage[paper, cgu]{pdef}
\usepackage{pgf}
\usepackage{caption}

\DeclareMathOperator{\ope}{\mathrm{E}}

\title{Report of Project of Chapter 4}
\author{Zhihan Li, 1600010653}
\date{April 18, 2019}

\begin{document}

\maketitle

\textbf{Problem (Page 133 Coding Exercise 6).} We consider the Gaussian cubature on simplexes, and use non-linear solvers and optimizers to tackle the constraints given by exactness of polynomials. We use the Newton's method to solve the linear system directly if the degree of freedom and the number of constraints match, or otherwise cast the problem into a sum of squares problems and apply Nesterov accelerated projection gradient method. We calculate the cubatures numerically, and then analytically consider some cubatures inspired by symmetry and the numerical results.

\section{Problem formulation}

\subsection{Gaussian cubature}

Given a $M$-simplex $S$ in $\mathbb{R}^M$ with vertices $\mathbf{v}_t$ for $ t = 0, 1, 2, \cdots, M $, we consider the Gaussian cubature on $S$. Generally speaking, a cubature consists of $n$ points $\mathbf{x}_t$ with weights $w_t$ for $ t = 1, 2, \cdots, N $ and calculate
\begin{equation}
I \rbr{f} = \sum_{ t = 1 }^N w_t f \rbr{\mathbf{x}_t}
\end{equation}
to approximate
\begin{equation}
\int_S f \rbr{\mathbf{x}} \sd \mathbf{x}.
\end{equation}
A Gaussian cubature of algebraic degree $D$ is a cubature exact on $ P_D \rbr{S} $, or say
\begin{equation} \label{Eq:Cons}
I \rbr{ \mathbf{x} \mapsto \prod_{ t = 1 }^M x_t^{i_t} } = \int_S \prod_{ t = 1 }^M x_t^{i_t} \sd \mathbf{x}
\end{equation}
for $ i_t \ge 0 $, $ \sum_{ t = 1 }^M i_t = D $. Here we denote $ \mathbf{x} = \rbr{ x_1, x_2, \cdots, x_M } $. One prominent problem is to determine the minimal $N$.

\subsection{Reference simplex}

We consider the problem on the reference $m$-simplex $\hat{S}$, which has vertices $ \hat{\mathbf{v}}_0 = 0 $, $ \hat{\mathbf{v}}_t = \mathbf{e}_t $ for $ t = 1, 2, \cdots, M $. Here $\mathbf{e}_t$ is the standard basis of $\mathbb{R}^M$. By matching the vertices, we have the affine map
\begin{equation}
T : \hat{S} \rightarrow S, \hat{\mathbf{x}} \mapsto \mathbf{v}_0 + \sum_{ t = 1 }^M x_t \rbr{ \mathbf{v}_t - \mathbf{v}_0 }.
\end{equation}
Hence, if we have already obtain a cubature on $\hat{S}$, say $\hat{I}$, we may calculate
\begin{equation}
I \rbr{f} = \rbr{ \nabla T } \hat{I} \rbr{\hat{f}}.
\end{equation}
Where $ \hat{f} = T^{\text{t}} \rbr{f} = f \circ T $. Note that $T^{\text{t}}$ is a bijection between $ P_D \rbr{S} $ and $ P_D (\hat{S}) $. This implies that if $\hat{I}$ is a Gaussian cubature of algebraic degree $D$ on $ P_D (\hat{S}) $, we immediately obtain $I$ is again a Gaussian cubature of algebraic degree $D$. This process is totally reversible, and therefore we only need to tackle the Gaussian cubature on the reference simplex $\hat{S}$.

On the reference simplex, it is much easier to check the constraints. We introduce 
Since
\begin{equation}
c_{ i_1, i_2, \cdots, i_M } = \int_{\hat{S}} \prod_{ r = 1 }^M x_t^{i_t} \sd \mathbf{x} = \bfrac{ \prod_{ t = 1 }^M \Gamma \rbr{ i_t + 1 } }{ \Gamma \rbr{ \sum_{ t = 1 }^M i_t + M + 1 } },
\end{equation}
we denote $ \pbr{ i_1, i_2, \cdots, i_M } $ to be the constraint \eqref{Eq:Cons}
\begin{equation}
f_{ i_1, i_2, \cdots, i_M } \rbr{ \mathbf{x}_{\cdot}, w_{\cdot} } = c_{ i_1, i_2, \cdots, i_M } - \sum_{ t = 1 }^N w_t \prod_{ r = 1 }^M x_t^{i_t} = 0.
\end{equation}
Moreover, we group the weights to decrease the degree of freedom. Arrange the $N$ nodes into $C$ groups by indices $ c_t \in \cbr{ 1, 2, \cdots, C } $ for $ t = 1, 2, \cdots, N $. The groups can be abbreviated as $ \sbr{ c_1, c_2, \cdots, c_N } $. We then enforce
\begin{equation}
w_t = u_{c_t}.
\end{equation}
By this mean, we only need to solve $\mathbf{x}_{\cdot}$ and $u_{\cdot}$ out of the constraints
\begin{equation}
f_{i_{\cdot}} \rbr{ \mathbf{x}_{\cdot}, u_{\cdot} } = f_{i_{\cdot}} \rbr{ \mathbf{x}_{\cdot}, u_{c_{\cdot}} } = 0.
\end{equation}

Denote we enforce $L$ constraints $ i^1_{\cdot}, i^2_{\cdot}, \cdots, i^L_{\cdot} $. Since the basis of $ P_D (\hat{K}) $ is $ \mathcal{I}_D = \cbr{ i_{\cdot} : i_t \ge 0, \sum_{ t = 1 }^M i_t \le D } $, to impose the exactness over $ P_D (\hat{K}) $, we have at least
\begin{equation}
\cbr{i^{\cdot}_{\cdot}} \subseteq \mathcal{I}_D.
\end{equation}
This implies the number of constraints
\begin{equation}
L \ge \binom{ D + M }{M}.
\end{equation}
We have
\begin{equation}
1 \le C \le N,
\end{equation}
and therefore the total degree of freedom is
\begin{equation}
N M + 1 \le N M + C \le \rbr{ N + 1 } M.
\end{equation}

\section{Solvers}

\subsection{Newton's method}

When the number of constraints and the total degree of freedom match, say $ L = N M + C $ we may directly use the Newtons method to solve this problem. We iterate $\mathbf{x}_{\cdot}$ and $u_{\cdot}$ under the constraints $f_{i_{\cdot}^{\cdot}}$. The iterations of Newton's method is given by
\begin{equation} 
\msbr{ \mathbf{x}_{\cdot}^{\rbr{ s + 1 }} \\ u_{\cdot}^{\rbr{ s + 1 }} } = \msbr{ \mathbf{x}_{\cdot}^{\rbr{s}} \\ u_{\cdot}^{\rbr{s}} }  - \rbr{ \nabla f_{i_{\cdot}^{\cdot}} \rbr{ \mathbf{x}_{\cdot}^{\rbr{s}}, u_{\cdot}^{\rbr{s}} } }^{-1} f_{i_{\cdot}^{\cdot}} \rbr{ \mathbf{x}_{\cdot}^{\rbr{s}}, u_{\cdot}^{\rbr{s}} }.
\end{equation}
Here the matrix inverse involves only a matrix of size $ L \times \rbr{ N M + C } $ which is rather small. Hence, we does not bother to consider Broyden's method.

We abbreviate the process with $K$ steps of iterations to be $ \text{N} \rbr{K} $.

\subsection{Nesterov accelerated projection gradient method}

When $ L \ge N M + C $, the system is underdetermined and we may find many solutions. If we want find only one point in the solution set, Newton's method is not much applicable. Hence, we consider the sum of squares problem
\begin{equation}
F \rbr{ \mathbf{x}_{\cdot}, u_{\cdot} } = \frac{1}{2} \sum_{ t = 1 }^L f_{i_{\cdot}^t}^2 \rbr{ \mathbf{x}_{\cdot}, u_{\cdot} }
\end{equation}
and perform optimization techniques on $F$. We have tested Newton's method on this problem, but it always gets stuck at local maximizers because of the complex landscape of $F$. Hence, we head to use first order methods. One popular choice is Nesterov accelerated gradient method. The iteration process is described as follows. Given $\mathbf{x}_{\cdot}^{\rbr{0}} $ and $u_{\cdot}^{\rbr{0}}$ as initial values, and we set $ \mathbf{x}_{\cdot}^{\rbr{-1}} = \mathbf{x}_{\cdot}^{\rbr{0}} $, $ u_{\cdot}^{\rbr{-1}} = u_{\cdot}^{\rbr{0}} $. The iterations with step size $\eta$ is given by
\begin{gather}
\msbr{ \mathbf{x}_{\cdot}^{\rbr{ s + 1 / 2 }} \\ u_{\cdot}^{\rbr{ s + 1 / 2 }} } = \msbr{ \mathbf{x}_{\cdot}^{\rbr{s}} \\ u_{\cdot}^{\rbr{s}} }  + \frac{ s - 1 }{ s + 2 } \rbr{ \msbr{ \mathbf{x}_{\cdot}^{\rbr{s}} \\ u_{\cdot}^{\rbr{s}} } - \msbr{ \mathbf{x}_{\cdot}^{\rbr{ s - 1 }} \\ u_{\cdot}^{\rbr{ s - 1 }} } }, \\
\msbr{ \mathbf{x}_{\cdot}^{\rbr{ s + 1 / 2 }} \\ u_{\cdot}^{\rbr{ s + 1 / 2 }} } = P \rbr{ \msbr{ \mathbf{x}_{\cdot}^{\rbr{ s + 1 / 2 }} \\ u_{\cdot}^{\rbr{ s + 1 / 2 }} } - \eta \nabla F \rbr{\msbr{ \mathbf{x}_{\cdot}^{\rbr{ s + 1 / 2 }} \\ u_{\cdot}^{\rbr{ s + 1 / 2 }} }} }.
\end{gather}
Here $P$ is a projection operator on the admissible set, which we choose to be $ S^N \times \rbr{\mathbb{R}^{+0}}^C $ here. We constrain $ \mathbf{x}_{\cdot} \in S $ to avoid extrapolation, and $ u_{\cdot} \in \mathbb{R}^{+0} $ to keep numerical stability. We use \emph{orthogonal} projection here.

We abbreviate the process with $K$ steps of iterations to be $ \text{FPG} \rbr{ K, \eta } $.

In many cases, the Newton's method has only local convergence. This means for randomized initial values, the Newton's method can diverge in most cases. We solve the problem by applying the accelerated gradient method and the Newton's method sequentially, denoted by $ \text{FPG} \rbr{ K_1, \eta } + \text{N} \rbr{K_2} $.

\section{Numerical results}

In the following numerical experiments, we always initialize by $ \mathbf{x}_{\cdot} \sim \mathcal{N} \rbr{ \mathbf{1} / \rbr{ M + 1 }, 10^{-4} \mathbf{I} } $ and $ u_{\cdot} \sim U \rbr{ 0, 2 / M N } $. Note that $ \mathbf{1} / \rbr{ M + 1 } $ is the barycenter of $\hat{S}$, and $ \ope \sum_{ t = 1 }^N u_{c_t} = 1 / M ! $ corresponds to volume of $\hat{S}$.

\subsection{The two-dimensional case}

Here $ M = 2 $. We take the notation $ \mathbf{x} = \rbr{ x, y } $.

\subsubsection{The case $ D = 1 $}

The case $ D = 1 $ can be analytically computed. By using $\mathcal{I}_1$ as constraints, we have $ L = 3 $ and we take $ N = 1 $. The only choice for groups is $\sbr{1}$ with $ C = 1 $. The equations turns out to be
\begin{gather}
w_1 = 1 / 2, \\
w_1 x_1 = 1 / 6, \\
w_1 y_1 = 1 / 6.
\end{gather}
The solution is $ \rbr{ x_1, y_1, w_1 } = \rbr{ 1 / 3, 1 / 3, 1 / 6 } $. Hence, we have
\begin{equation}
\hat{I} (\hat{f}) = \frac{1}{2} \hat{f} \rbr{ \frac{1}{3}, \frac{1}{3} },
\end{equation}
and
\begin{equation}
I \rbr{f} = \abs{S} \hat{f} \rbr{ \frac{1}{3} \rbr{ \mathbf{v}_0 + \mathbf{v}_1 + \mathbf{v}_2 } }.
\end{equation}
As for the case from the textbook, where $S$ is an equilateral triangle, we have
\begin{equation}
I \rbr{f} = \frac{\sqrt{3}}{4} f \rbr{ \frac{1}{2}, \frac{\sqrt{3}}{6} }.
\end{equation}

We apply $ \text{N} \rbr{30} $ to solve the problem directly. The numerical algorithm succeeds to yield the correct solution. The figure is given in Figure \ref{Fig:M2D1}.

\begin{figure}[htbp]
\centering
\scalebox{0.75}{%% Creator: Matplotlib, PGF backend
%%
%% To include the figure in your LaTeX document, write
%%   \input{<filename>.pgf}
%%
%% Make sure the required packages are loaded in your preamble
%%   \usepackage{pgf}
%%
%% Figures using additional raster images can only be included by \input if
%% they are in the same directory as the main LaTeX file. For loading figures
%% from other directories you can use the `import` package
%%   \usepackage{import}
%% and then include the figures with
%%   \import{<path to file>}{<filename>.pgf}
%%
%% Matplotlib used the following preamble
%%   \usepackage{fontspec}
%%   \setmainfont{DejaVuSerif.ttf}[Path=/home/lzh/anaconda3/envs/numana/lib/python3.7/site-packages/matplotlib/mpl-data/fonts/ttf/]
%%   \setsansfont{DejaVuSans.ttf}[Path=/home/lzh/anaconda3/envs/numana/lib/python3.7/site-packages/matplotlib/mpl-data/fonts/ttf/]
%%   \setmonofont{DejaVuSansMono.ttf}[Path=/home/lzh/anaconda3/envs/numana/lib/python3.7/site-packages/matplotlib/mpl-data/fonts/ttf/]
%%
\begingroup%
\makeatletter%
\begin{pgfpicture}%
\pgfpathrectangle{\pgfpointorigin}{\pgfqpoint{12.000000in}{4.000000in}}%
\pgfusepath{use as bounding box, clip}%
\begin{pgfscope}%
\pgfsetbuttcap%
\pgfsetmiterjoin%
\definecolor{currentfill}{rgb}{1.000000,1.000000,1.000000}%
\pgfsetfillcolor{currentfill}%
\pgfsetlinewidth{0.000000pt}%
\definecolor{currentstroke}{rgb}{1.000000,1.000000,1.000000}%
\pgfsetstrokecolor{currentstroke}%
\pgfsetdash{}{0pt}%
\pgfpathmoveto{\pgfqpoint{0.000000in}{0.000000in}}%
\pgfpathlineto{\pgfqpoint{12.000000in}{0.000000in}}%
\pgfpathlineto{\pgfqpoint{12.000000in}{4.000000in}}%
\pgfpathlineto{\pgfqpoint{0.000000in}{4.000000in}}%
\pgfpathclose%
\pgfusepath{fill}%
\end{pgfscope}%
\begin{pgfscope}%
\pgfsys@transformshift{0.120000in}{0.090000in}%
\pgftext[left,bottom]{\pgfimage[interpolate=true,width=3.830000in,height=3.760000in]{Figure01-img0.png}}%
\end{pgfscope}%
\begin{pgfscope}%
\pgfsys@transformshift{4.100000in}{0.070000in}%
\pgftext[left,bottom]{\pgfimage[interpolate=true,width=3.930000in,height=3.780000in]{Figure01-img1.png}}%
\end{pgfscope}%
\begin{pgfscope}%
\pgfsys@transformshift{8.050000in}{0.150000in}%
\pgftext[left,bottom]{\pgfimage[interpolate=true,width=3.950000in,height=3.780000in]{Figure01-img2.png}}%
\end{pgfscope}%
\end{pgfpicture}%
\makeatother%
\endgroup%
}
\caption{Figure of the two-dimensional cubature of $ D = 1 $}
\label{Fig:M2D1}
\end{figure}

In this case, the smallest $N$ is $1$.

\subsubsection{The case $ D = 2 $}

The case $ D = 2 $ is more subtle.

Since $ \abs{\mathcal{I}_2} = 6 $, we may consider using $ L = 6 $ constraints from $\mathcal{I}_2$ and $ N = 2 $ nodes. The only choice for groups is $ \sbr{ 1, 2, 3 } $ with $ C = 3 $. However, using Newton's method the numerical diverge rapidly. using $ \text{FPG} \rbr{ 10^6, 10^{-2} }$ the final value of sum of squares is $ F ( \mathbf{x}_{\cdot}^{\rbr{K}}, u_{\cdot}^{\rbr{K}} ) = \text{\input{Text1.txt}} $. This means the non-existence of solutions.

We turn to consider increasing $N$. We set $ N = 3 $. With groups $ \sbr{ 1, 1, 1 } $ and $ C = 1 $, the degree of freedom is $ 2 N + C = 7 $.

By using constraints $\mathcal{I}_2$ only, we have $ L = 6 $. We apply $ \text{FPG} \rbr{ 10^5, 10^{-2} } $ to get the following results in Figure \ref{Fig:M2D2Ran} for different initial values.

\begin{figure}[htbp]
\centering
\scalebox{0.75}{%% Creator: Matplotlib, PGF backend
%%
%% To include the figure in your LaTeX document, write
%%   \input{<filename>.pgf}
%%
%% Make sure the required packages are loaded in your preamble
%%   \usepackage{pgf}
%%
%% Figures using additional raster images can only be included by \input if
%% they are in the same directory as the main LaTeX file. For loading figures
%% from other directories you can use the `import` package
%%   \usepackage{import}
%% and then include the figures with
%%   \import{<path to file>}{<filename>.pgf}
%%
%% Matplotlib used the following preamble
%%   \usepackage{fontspec}
%%   \setmainfont{DejaVuSerif.ttf}[Path=/home/lzh/anaconda3/envs/numana/lib/python3.7/site-packages/matplotlib/mpl-data/fonts/ttf/]
%%   \setsansfont{DejaVuSans.ttf}[Path=/home/lzh/anaconda3/envs/numana/lib/python3.7/site-packages/matplotlib/mpl-data/fonts/ttf/]
%%   \setmonofont{DejaVuSansMono.ttf}[Path=/home/lzh/anaconda3/envs/numana/lib/python3.7/site-packages/matplotlib/mpl-data/fonts/ttf/]
%%
\begingroup%
\makeatletter%
\begin{pgfpicture}%
\pgfpathrectangle{\pgfpointorigin}{\pgfqpoint{4.000000in}{4.000000in}}%
\pgfusepath{use as bounding box, clip}%
\begin{pgfscope}%
\pgfsetbuttcap%
\pgfsetmiterjoin%
\definecolor{currentfill}{rgb}{1.000000,1.000000,1.000000}%
\pgfsetfillcolor{currentfill}%
\pgfsetlinewidth{0.000000pt}%
\definecolor{currentstroke}{rgb}{1.000000,1.000000,1.000000}%
\pgfsetstrokecolor{currentstroke}%
\pgfsetdash{}{0pt}%
\pgfpathmoveto{\pgfqpoint{0.000000in}{0.000000in}}%
\pgfpathlineto{\pgfqpoint{4.000000in}{0.000000in}}%
\pgfpathlineto{\pgfqpoint{4.000000in}{4.000000in}}%
\pgfpathlineto{\pgfqpoint{0.000000in}{4.000000in}}%
\pgfpathclose%
\pgfusepath{fill}%
\end{pgfscope}%
\begin{pgfscope}%
\pgfsetbuttcap%
\pgfsetmiterjoin%
\definecolor{currentfill}{rgb}{1.000000,1.000000,1.000000}%
\pgfsetfillcolor{currentfill}%
\pgfsetlinewidth{0.000000pt}%
\definecolor{currentstroke}{rgb}{0.000000,0.000000,0.000000}%
\pgfsetstrokecolor{currentstroke}%
\pgfsetstrokeopacity{0.000000}%
\pgfsetdash{}{0pt}%
\pgfpathmoveto{\pgfqpoint{0.510000in}{0.440000in}}%
\pgfpathlineto{\pgfqpoint{3.590000in}{0.440000in}}%
\pgfpathlineto{\pgfqpoint{3.590000in}{3.520000in}}%
\pgfpathlineto{\pgfqpoint{0.510000in}{3.520000in}}%
\pgfpathclose%
\pgfusepath{fill}%
\end{pgfscope}%
\begin{pgfscope}%
\pgfpathrectangle{\pgfqpoint{0.510000in}{0.440000in}}{\pgfqpoint{3.080000in}{3.080000in}}%
\pgfusepath{clip}%
\pgfsetbuttcap%
\pgfsetroundjoin%
\definecolor{currentfill}{rgb}{0.121569,0.466667,0.705882}%
\pgfsetfillcolor{currentfill}%
\pgfsetlinewidth{1.003750pt}%
\definecolor{currentstroke}{rgb}{0.121569,0.466667,0.705882}%
\pgfsetstrokecolor{currentstroke}%
\pgfsetdash{}{0pt}%
\pgfpathmoveto{\pgfqpoint{1.501005in}{1.067479in}}%
\pgfpathcurveto{\pgfqpoint{1.506828in}{1.067479in}}{\pgfqpoint{1.512415in}{1.069793in}}{\pgfqpoint{1.516533in}{1.073911in}}%
\pgfpathcurveto{\pgfqpoint{1.520651in}{1.078030in}}{\pgfqpoint{1.522965in}{1.083616in}}{\pgfqpoint{1.522965in}{1.089440in}}%
\pgfpathcurveto{\pgfqpoint{1.522965in}{1.095264in}}{\pgfqpoint{1.520651in}{1.100850in}}{\pgfqpoint{1.516533in}{1.104968in}}%
\pgfpathcurveto{\pgfqpoint{1.512415in}{1.109086in}}{\pgfqpoint{1.506828in}{1.111400in}}{\pgfqpoint{1.501005in}{1.111400in}}%
\pgfpathcurveto{\pgfqpoint{1.495181in}{1.111400in}}{\pgfqpoint{1.489594in}{1.109086in}}{\pgfqpoint{1.485476in}{1.104968in}}%
\pgfpathcurveto{\pgfqpoint{1.481358in}{1.100850in}}{\pgfqpoint{1.479044in}{1.095264in}}{\pgfqpoint{1.479044in}{1.089440in}}%
\pgfpathcurveto{\pgfqpoint{1.479044in}{1.083616in}}{\pgfqpoint{1.481358in}{1.078030in}}{\pgfqpoint{1.485476in}{1.073911in}}%
\pgfpathcurveto{\pgfqpoint{1.489594in}{1.069793in}}{\pgfqpoint{1.495181in}{1.067479in}}{\pgfqpoint{1.501005in}{1.067479in}}%
\pgfpathclose%
\pgfusepath{stroke,fill}%
\end{pgfscope}%
\begin{pgfscope}%
\pgfpathrectangle{\pgfqpoint{0.510000in}{0.440000in}}{\pgfqpoint{3.080000in}{3.080000in}}%
\pgfusepath{clip}%
\pgfsetbuttcap%
\pgfsetroundjoin%
\definecolor{currentfill}{rgb}{1.000000,0.498039,0.054902}%
\pgfsetfillcolor{currentfill}%
\pgfsetlinewidth{1.003750pt}%
\definecolor{currentstroke}{rgb}{1.000000,0.498039,0.054902}%
\pgfsetstrokecolor{currentstroke}%
\pgfsetdash{}{0pt}%
\pgfpathmoveto{\pgfqpoint{2.389631in}{1.408960in}}%
\pgfpathcurveto{\pgfqpoint{2.395455in}{1.408960in}}{\pgfqpoint{2.401041in}{1.411274in}}{\pgfqpoint{2.405159in}{1.415392in}}%
\pgfpathcurveto{\pgfqpoint{2.409277in}{1.419510in}}{\pgfqpoint{2.411591in}{1.425096in}}{\pgfqpoint{2.411591in}{1.430920in}}%
\pgfpathcurveto{\pgfqpoint{2.411591in}{1.436744in}}{\pgfqpoint{2.409277in}{1.442330in}}{\pgfqpoint{2.405159in}{1.446448in}}%
\pgfpathcurveto{\pgfqpoint{2.401041in}{1.450566in}}{\pgfqpoint{2.395455in}{1.452880in}}{\pgfqpoint{2.389631in}{1.452880in}}%
\pgfpathcurveto{\pgfqpoint{2.383807in}{1.452880in}}{\pgfqpoint{2.378221in}{1.450566in}}{\pgfqpoint{2.374103in}{1.446448in}}%
\pgfpathcurveto{\pgfqpoint{2.369985in}{1.442330in}}{\pgfqpoint{2.367671in}{1.436744in}}{\pgfqpoint{2.367671in}{1.430920in}}%
\pgfpathcurveto{\pgfqpoint{2.367671in}{1.425096in}}{\pgfqpoint{2.369985in}{1.419510in}}{\pgfqpoint{2.374103in}{1.415392in}}%
\pgfpathcurveto{\pgfqpoint{2.378221in}{1.411274in}}{\pgfqpoint{2.383807in}{1.408960in}}{\pgfqpoint{2.389631in}{1.408960in}}%
\pgfpathclose%
\pgfusepath{stroke,fill}%
\end{pgfscope}%
\begin{pgfscope}%
\pgfpathrectangle{\pgfqpoint{0.510000in}{0.440000in}}{\pgfqpoint{3.080000in}{3.080000in}}%
\pgfusepath{clip}%
\pgfsetbuttcap%
\pgfsetroundjoin%
\definecolor{currentfill}{rgb}{0.172549,0.627451,0.172549}%
\pgfsetfillcolor{currentfill}%
\pgfsetlinewidth{1.003750pt}%
\definecolor{currentstroke}{rgb}{0.172549,0.627451,0.172549}%
\pgfsetstrokecolor{currentstroke}%
\pgfsetdash{}{0pt}%
\pgfpathmoveto{\pgfqpoint{1.159379in}{2.297694in}}%
\pgfpathcurveto{\pgfqpoint{1.165203in}{2.297694in}}{\pgfqpoint{1.170789in}{2.300008in}}{\pgfqpoint{1.174907in}{2.304126in}}%
\pgfpathcurveto{\pgfqpoint{1.179025in}{2.308244in}}{\pgfqpoint{1.181339in}{2.313831in}}{\pgfqpoint{1.181339in}{2.319655in}}%
\pgfpathcurveto{\pgfqpoint{1.181339in}{2.325479in}}{\pgfqpoint{1.179025in}{2.331065in}}{\pgfqpoint{1.174907in}{2.335183in}}%
\pgfpathcurveto{\pgfqpoint{1.170789in}{2.339301in}}{\pgfqpoint{1.165203in}{2.341615in}}{\pgfqpoint{1.159379in}{2.341615in}}%
\pgfpathcurveto{\pgfqpoint{1.153555in}{2.341615in}}{\pgfqpoint{1.147969in}{2.339301in}}{\pgfqpoint{1.143851in}{2.335183in}}%
\pgfpathcurveto{\pgfqpoint{1.139732in}{2.331065in}}{\pgfqpoint{1.137419in}{2.325479in}}{\pgfqpoint{1.137419in}{2.319655in}}%
\pgfpathcurveto{\pgfqpoint{1.137419in}{2.313831in}}{\pgfqpoint{1.139732in}{2.308244in}}{\pgfqpoint{1.143851in}{2.304126in}}%
\pgfpathcurveto{\pgfqpoint{1.147969in}{2.300008in}}{\pgfqpoint{1.153555in}{2.297694in}}{\pgfqpoint{1.159379in}{2.297694in}}%
\pgfpathclose%
\pgfusepath{stroke,fill}%
\end{pgfscope}%
\begin{pgfscope}%
\pgfpathrectangle{\pgfqpoint{0.510000in}{0.440000in}}{\pgfqpoint{3.080000in}{3.080000in}}%
\pgfusepath{clip}%
\pgfsetbuttcap%
\pgfsetmiterjoin%
\definecolor{currentfill}{rgb}{0.100000,0.100000,0.100000}%
\pgfsetfillcolor{currentfill}%
\pgfsetfillopacity{0.100000}%
\pgfsetlinewidth{0.301125pt}%
\definecolor{currentstroke}{rgb}{0.000000,0.000000,0.000000}%
\pgfsetstrokecolor{currentstroke}%
\pgfsetdash{}{0pt}%
\pgfpathmoveto{\pgfqpoint{0.950000in}{0.880000in}}%
\pgfpathlineto{\pgfqpoint{0.950000in}{3.080000in}}%
\pgfpathlineto{\pgfqpoint{3.150000in}{0.880000in}}%
\pgfpathclose%
\pgfusepath{stroke,fill}%
\end{pgfscope}%
\begin{pgfscope}%
\pgfpathrectangle{\pgfqpoint{0.510000in}{0.440000in}}{\pgfqpoint{3.080000in}{3.080000in}}%
\pgfusepath{clip}%
\pgfsetrectcap%
\pgfsetroundjoin%
\pgfsetlinewidth{0.803000pt}%
\definecolor{currentstroke}{rgb}{0.690196,0.690196,0.690196}%
\pgfsetstrokecolor{currentstroke}%
\pgfsetdash{}{0pt}%
\pgfpathmoveto{\pgfqpoint{0.510000in}{0.440000in}}%
\pgfpathlineto{\pgfqpoint{0.510000in}{3.520000in}}%
\pgfusepath{stroke}%
\end{pgfscope}%
\begin{pgfscope}%
\pgfsetbuttcap%
\pgfsetroundjoin%
\definecolor{currentfill}{rgb}{0.000000,0.000000,0.000000}%
\pgfsetfillcolor{currentfill}%
\pgfsetlinewidth{0.803000pt}%
\definecolor{currentstroke}{rgb}{0.000000,0.000000,0.000000}%
\pgfsetstrokecolor{currentstroke}%
\pgfsetdash{}{0pt}%
\pgfsys@defobject{currentmarker}{\pgfqpoint{0.000000in}{-0.048611in}}{\pgfqpoint{0.000000in}{0.000000in}}{%
\pgfpathmoveto{\pgfqpoint{0.000000in}{0.000000in}}%
\pgfpathlineto{\pgfqpoint{0.000000in}{-0.048611in}}%
\pgfusepath{stroke,fill}%
}%
\begin{pgfscope}%
\pgfsys@transformshift{0.510000in}{0.440000in}%
\pgfsys@useobject{currentmarker}{}%
\end{pgfscope}%
\end{pgfscope}%
\begin{pgfscope}%
\definecolor{textcolor}{rgb}{0.000000,0.000000,0.000000}%
\pgfsetstrokecolor{textcolor}%
\pgfsetfillcolor{textcolor}%
\pgftext[x=0.510000in,y=0.342778in,,top]{\color{textcolor}\sffamily\fontsize{10.000000}{12.000000}\selectfont -0.2}%
\end{pgfscope}%
\begin{pgfscope}%
\pgfpathrectangle{\pgfqpoint{0.510000in}{0.440000in}}{\pgfqpoint{3.080000in}{3.080000in}}%
\pgfusepath{clip}%
\pgfsetrectcap%
\pgfsetroundjoin%
\pgfsetlinewidth{0.803000pt}%
\definecolor{currentstroke}{rgb}{0.690196,0.690196,0.690196}%
\pgfsetstrokecolor{currentstroke}%
\pgfsetdash{}{0pt}%
\pgfpathmoveto{\pgfqpoint{0.950000in}{0.440000in}}%
\pgfpathlineto{\pgfqpoint{0.950000in}{3.520000in}}%
\pgfusepath{stroke}%
\end{pgfscope}%
\begin{pgfscope}%
\pgfsetbuttcap%
\pgfsetroundjoin%
\definecolor{currentfill}{rgb}{0.000000,0.000000,0.000000}%
\pgfsetfillcolor{currentfill}%
\pgfsetlinewidth{0.803000pt}%
\definecolor{currentstroke}{rgb}{0.000000,0.000000,0.000000}%
\pgfsetstrokecolor{currentstroke}%
\pgfsetdash{}{0pt}%
\pgfsys@defobject{currentmarker}{\pgfqpoint{0.000000in}{-0.048611in}}{\pgfqpoint{0.000000in}{0.000000in}}{%
\pgfpathmoveto{\pgfqpoint{0.000000in}{0.000000in}}%
\pgfpathlineto{\pgfqpoint{0.000000in}{-0.048611in}}%
\pgfusepath{stroke,fill}%
}%
\begin{pgfscope}%
\pgfsys@transformshift{0.950000in}{0.440000in}%
\pgfsys@useobject{currentmarker}{}%
\end{pgfscope}%
\end{pgfscope}%
\begin{pgfscope}%
\definecolor{textcolor}{rgb}{0.000000,0.000000,0.000000}%
\pgfsetstrokecolor{textcolor}%
\pgfsetfillcolor{textcolor}%
\pgftext[x=0.950000in,y=0.342778in,,top]{\color{textcolor}\sffamily\fontsize{10.000000}{12.000000}\selectfont 0.0}%
\end{pgfscope}%
\begin{pgfscope}%
\pgfpathrectangle{\pgfqpoint{0.510000in}{0.440000in}}{\pgfqpoint{3.080000in}{3.080000in}}%
\pgfusepath{clip}%
\pgfsetrectcap%
\pgfsetroundjoin%
\pgfsetlinewidth{0.803000pt}%
\definecolor{currentstroke}{rgb}{0.690196,0.690196,0.690196}%
\pgfsetstrokecolor{currentstroke}%
\pgfsetdash{}{0pt}%
\pgfpathmoveto{\pgfqpoint{1.390000in}{0.440000in}}%
\pgfpathlineto{\pgfqpoint{1.390000in}{3.520000in}}%
\pgfusepath{stroke}%
\end{pgfscope}%
\begin{pgfscope}%
\pgfsetbuttcap%
\pgfsetroundjoin%
\definecolor{currentfill}{rgb}{0.000000,0.000000,0.000000}%
\pgfsetfillcolor{currentfill}%
\pgfsetlinewidth{0.803000pt}%
\definecolor{currentstroke}{rgb}{0.000000,0.000000,0.000000}%
\pgfsetstrokecolor{currentstroke}%
\pgfsetdash{}{0pt}%
\pgfsys@defobject{currentmarker}{\pgfqpoint{0.000000in}{-0.048611in}}{\pgfqpoint{0.000000in}{0.000000in}}{%
\pgfpathmoveto{\pgfqpoint{0.000000in}{0.000000in}}%
\pgfpathlineto{\pgfqpoint{0.000000in}{-0.048611in}}%
\pgfusepath{stroke,fill}%
}%
\begin{pgfscope}%
\pgfsys@transformshift{1.390000in}{0.440000in}%
\pgfsys@useobject{currentmarker}{}%
\end{pgfscope}%
\end{pgfscope}%
\begin{pgfscope}%
\definecolor{textcolor}{rgb}{0.000000,0.000000,0.000000}%
\pgfsetstrokecolor{textcolor}%
\pgfsetfillcolor{textcolor}%
\pgftext[x=1.390000in,y=0.342778in,,top]{\color{textcolor}\sffamily\fontsize{10.000000}{12.000000}\selectfont 0.2}%
\end{pgfscope}%
\begin{pgfscope}%
\pgfpathrectangle{\pgfqpoint{0.510000in}{0.440000in}}{\pgfqpoint{3.080000in}{3.080000in}}%
\pgfusepath{clip}%
\pgfsetrectcap%
\pgfsetroundjoin%
\pgfsetlinewidth{0.803000pt}%
\definecolor{currentstroke}{rgb}{0.690196,0.690196,0.690196}%
\pgfsetstrokecolor{currentstroke}%
\pgfsetdash{}{0pt}%
\pgfpathmoveto{\pgfqpoint{1.830000in}{0.440000in}}%
\pgfpathlineto{\pgfqpoint{1.830000in}{3.520000in}}%
\pgfusepath{stroke}%
\end{pgfscope}%
\begin{pgfscope}%
\pgfsetbuttcap%
\pgfsetroundjoin%
\definecolor{currentfill}{rgb}{0.000000,0.000000,0.000000}%
\pgfsetfillcolor{currentfill}%
\pgfsetlinewidth{0.803000pt}%
\definecolor{currentstroke}{rgb}{0.000000,0.000000,0.000000}%
\pgfsetstrokecolor{currentstroke}%
\pgfsetdash{}{0pt}%
\pgfsys@defobject{currentmarker}{\pgfqpoint{0.000000in}{-0.048611in}}{\pgfqpoint{0.000000in}{0.000000in}}{%
\pgfpathmoveto{\pgfqpoint{0.000000in}{0.000000in}}%
\pgfpathlineto{\pgfqpoint{0.000000in}{-0.048611in}}%
\pgfusepath{stroke,fill}%
}%
\begin{pgfscope}%
\pgfsys@transformshift{1.830000in}{0.440000in}%
\pgfsys@useobject{currentmarker}{}%
\end{pgfscope}%
\end{pgfscope}%
\begin{pgfscope}%
\definecolor{textcolor}{rgb}{0.000000,0.000000,0.000000}%
\pgfsetstrokecolor{textcolor}%
\pgfsetfillcolor{textcolor}%
\pgftext[x=1.830000in,y=0.342778in,,top]{\color{textcolor}\sffamily\fontsize{10.000000}{12.000000}\selectfont 0.4}%
\end{pgfscope}%
\begin{pgfscope}%
\pgfpathrectangle{\pgfqpoint{0.510000in}{0.440000in}}{\pgfqpoint{3.080000in}{3.080000in}}%
\pgfusepath{clip}%
\pgfsetrectcap%
\pgfsetroundjoin%
\pgfsetlinewidth{0.803000pt}%
\definecolor{currentstroke}{rgb}{0.690196,0.690196,0.690196}%
\pgfsetstrokecolor{currentstroke}%
\pgfsetdash{}{0pt}%
\pgfpathmoveto{\pgfqpoint{2.270000in}{0.440000in}}%
\pgfpathlineto{\pgfqpoint{2.270000in}{3.520000in}}%
\pgfusepath{stroke}%
\end{pgfscope}%
\begin{pgfscope}%
\pgfsetbuttcap%
\pgfsetroundjoin%
\definecolor{currentfill}{rgb}{0.000000,0.000000,0.000000}%
\pgfsetfillcolor{currentfill}%
\pgfsetlinewidth{0.803000pt}%
\definecolor{currentstroke}{rgb}{0.000000,0.000000,0.000000}%
\pgfsetstrokecolor{currentstroke}%
\pgfsetdash{}{0pt}%
\pgfsys@defobject{currentmarker}{\pgfqpoint{0.000000in}{-0.048611in}}{\pgfqpoint{0.000000in}{0.000000in}}{%
\pgfpathmoveto{\pgfqpoint{0.000000in}{0.000000in}}%
\pgfpathlineto{\pgfqpoint{0.000000in}{-0.048611in}}%
\pgfusepath{stroke,fill}%
}%
\begin{pgfscope}%
\pgfsys@transformshift{2.270000in}{0.440000in}%
\pgfsys@useobject{currentmarker}{}%
\end{pgfscope}%
\end{pgfscope}%
\begin{pgfscope}%
\definecolor{textcolor}{rgb}{0.000000,0.000000,0.000000}%
\pgfsetstrokecolor{textcolor}%
\pgfsetfillcolor{textcolor}%
\pgftext[x=2.270000in,y=0.342778in,,top]{\color{textcolor}\sffamily\fontsize{10.000000}{12.000000}\selectfont 0.6}%
\end{pgfscope}%
\begin{pgfscope}%
\pgfpathrectangle{\pgfqpoint{0.510000in}{0.440000in}}{\pgfqpoint{3.080000in}{3.080000in}}%
\pgfusepath{clip}%
\pgfsetrectcap%
\pgfsetroundjoin%
\pgfsetlinewidth{0.803000pt}%
\definecolor{currentstroke}{rgb}{0.690196,0.690196,0.690196}%
\pgfsetstrokecolor{currentstroke}%
\pgfsetdash{}{0pt}%
\pgfpathmoveto{\pgfqpoint{2.710000in}{0.440000in}}%
\pgfpathlineto{\pgfqpoint{2.710000in}{3.520000in}}%
\pgfusepath{stroke}%
\end{pgfscope}%
\begin{pgfscope}%
\pgfsetbuttcap%
\pgfsetroundjoin%
\definecolor{currentfill}{rgb}{0.000000,0.000000,0.000000}%
\pgfsetfillcolor{currentfill}%
\pgfsetlinewidth{0.803000pt}%
\definecolor{currentstroke}{rgb}{0.000000,0.000000,0.000000}%
\pgfsetstrokecolor{currentstroke}%
\pgfsetdash{}{0pt}%
\pgfsys@defobject{currentmarker}{\pgfqpoint{0.000000in}{-0.048611in}}{\pgfqpoint{0.000000in}{0.000000in}}{%
\pgfpathmoveto{\pgfqpoint{0.000000in}{0.000000in}}%
\pgfpathlineto{\pgfqpoint{0.000000in}{-0.048611in}}%
\pgfusepath{stroke,fill}%
}%
\begin{pgfscope}%
\pgfsys@transformshift{2.710000in}{0.440000in}%
\pgfsys@useobject{currentmarker}{}%
\end{pgfscope}%
\end{pgfscope}%
\begin{pgfscope}%
\definecolor{textcolor}{rgb}{0.000000,0.000000,0.000000}%
\pgfsetstrokecolor{textcolor}%
\pgfsetfillcolor{textcolor}%
\pgftext[x=2.710000in,y=0.342778in,,top]{\color{textcolor}\sffamily\fontsize{10.000000}{12.000000}\selectfont 0.8}%
\end{pgfscope}%
\begin{pgfscope}%
\pgfpathrectangle{\pgfqpoint{0.510000in}{0.440000in}}{\pgfqpoint{3.080000in}{3.080000in}}%
\pgfusepath{clip}%
\pgfsetrectcap%
\pgfsetroundjoin%
\pgfsetlinewidth{0.803000pt}%
\definecolor{currentstroke}{rgb}{0.690196,0.690196,0.690196}%
\pgfsetstrokecolor{currentstroke}%
\pgfsetdash{}{0pt}%
\pgfpathmoveto{\pgfqpoint{3.150000in}{0.440000in}}%
\pgfpathlineto{\pgfqpoint{3.150000in}{3.520000in}}%
\pgfusepath{stroke}%
\end{pgfscope}%
\begin{pgfscope}%
\pgfsetbuttcap%
\pgfsetroundjoin%
\definecolor{currentfill}{rgb}{0.000000,0.000000,0.000000}%
\pgfsetfillcolor{currentfill}%
\pgfsetlinewidth{0.803000pt}%
\definecolor{currentstroke}{rgb}{0.000000,0.000000,0.000000}%
\pgfsetstrokecolor{currentstroke}%
\pgfsetdash{}{0pt}%
\pgfsys@defobject{currentmarker}{\pgfqpoint{0.000000in}{-0.048611in}}{\pgfqpoint{0.000000in}{0.000000in}}{%
\pgfpathmoveto{\pgfqpoint{0.000000in}{0.000000in}}%
\pgfpathlineto{\pgfqpoint{0.000000in}{-0.048611in}}%
\pgfusepath{stroke,fill}%
}%
\begin{pgfscope}%
\pgfsys@transformshift{3.150000in}{0.440000in}%
\pgfsys@useobject{currentmarker}{}%
\end{pgfscope}%
\end{pgfscope}%
\begin{pgfscope}%
\definecolor{textcolor}{rgb}{0.000000,0.000000,0.000000}%
\pgfsetstrokecolor{textcolor}%
\pgfsetfillcolor{textcolor}%
\pgftext[x=3.150000in,y=0.342778in,,top]{\color{textcolor}\sffamily\fontsize{10.000000}{12.000000}\selectfont 1.0}%
\end{pgfscope}%
\begin{pgfscope}%
\pgfpathrectangle{\pgfqpoint{0.510000in}{0.440000in}}{\pgfqpoint{3.080000in}{3.080000in}}%
\pgfusepath{clip}%
\pgfsetrectcap%
\pgfsetroundjoin%
\pgfsetlinewidth{0.803000pt}%
\definecolor{currentstroke}{rgb}{0.690196,0.690196,0.690196}%
\pgfsetstrokecolor{currentstroke}%
\pgfsetdash{}{0pt}%
\pgfpathmoveto{\pgfqpoint{3.590000in}{0.440000in}}%
\pgfpathlineto{\pgfqpoint{3.590000in}{3.520000in}}%
\pgfusepath{stroke}%
\end{pgfscope}%
\begin{pgfscope}%
\pgfsetbuttcap%
\pgfsetroundjoin%
\definecolor{currentfill}{rgb}{0.000000,0.000000,0.000000}%
\pgfsetfillcolor{currentfill}%
\pgfsetlinewidth{0.803000pt}%
\definecolor{currentstroke}{rgb}{0.000000,0.000000,0.000000}%
\pgfsetstrokecolor{currentstroke}%
\pgfsetdash{}{0pt}%
\pgfsys@defobject{currentmarker}{\pgfqpoint{0.000000in}{-0.048611in}}{\pgfqpoint{0.000000in}{0.000000in}}{%
\pgfpathmoveto{\pgfqpoint{0.000000in}{0.000000in}}%
\pgfpathlineto{\pgfqpoint{0.000000in}{-0.048611in}}%
\pgfusepath{stroke,fill}%
}%
\begin{pgfscope}%
\pgfsys@transformshift{3.590000in}{0.440000in}%
\pgfsys@useobject{currentmarker}{}%
\end{pgfscope}%
\end{pgfscope}%
\begin{pgfscope}%
\definecolor{textcolor}{rgb}{0.000000,0.000000,0.000000}%
\pgfsetstrokecolor{textcolor}%
\pgfsetfillcolor{textcolor}%
\pgftext[x=3.590000in,y=0.342778in,,top]{\color{textcolor}\sffamily\fontsize{10.000000}{12.000000}\selectfont 1.2}%
\end{pgfscope}%
\begin{pgfscope}%
\pgfpathrectangle{\pgfqpoint{0.510000in}{0.440000in}}{\pgfqpoint{3.080000in}{3.080000in}}%
\pgfusepath{clip}%
\pgfsetrectcap%
\pgfsetroundjoin%
\pgfsetlinewidth{0.803000pt}%
\definecolor{currentstroke}{rgb}{0.690196,0.690196,0.690196}%
\pgfsetstrokecolor{currentstroke}%
\pgfsetdash{}{0pt}%
\pgfpathmoveto{\pgfqpoint{0.510000in}{0.440000in}}%
\pgfpathlineto{\pgfqpoint{3.590000in}{0.440000in}}%
\pgfusepath{stroke}%
\end{pgfscope}%
\begin{pgfscope}%
\pgfsetbuttcap%
\pgfsetroundjoin%
\definecolor{currentfill}{rgb}{0.000000,0.000000,0.000000}%
\pgfsetfillcolor{currentfill}%
\pgfsetlinewidth{0.803000pt}%
\definecolor{currentstroke}{rgb}{0.000000,0.000000,0.000000}%
\pgfsetstrokecolor{currentstroke}%
\pgfsetdash{}{0pt}%
\pgfsys@defobject{currentmarker}{\pgfqpoint{-0.048611in}{0.000000in}}{\pgfqpoint{0.000000in}{0.000000in}}{%
\pgfpathmoveto{\pgfqpoint{0.000000in}{0.000000in}}%
\pgfpathlineto{\pgfqpoint{-0.048611in}{0.000000in}}%
\pgfusepath{stroke,fill}%
}%
\begin{pgfscope}%
\pgfsys@transformshift{0.510000in}{0.440000in}%
\pgfsys@useobject{currentmarker}{}%
\end{pgfscope}%
\end{pgfscope}%
\begin{pgfscope}%
\definecolor{textcolor}{rgb}{0.000000,0.000000,0.000000}%
\pgfsetstrokecolor{textcolor}%
\pgfsetfillcolor{textcolor}%
\pgftext[x=0.075525in,y=0.387238in,left,base]{\color{textcolor}\sffamily\fontsize{10.000000}{12.000000}\selectfont -0.2}%
\end{pgfscope}%
\begin{pgfscope}%
\pgfpathrectangle{\pgfqpoint{0.510000in}{0.440000in}}{\pgfqpoint{3.080000in}{3.080000in}}%
\pgfusepath{clip}%
\pgfsetrectcap%
\pgfsetroundjoin%
\pgfsetlinewidth{0.803000pt}%
\definecolor{currentstroke}{rgb}{0.690196,0.690196,0.690196}%
\pgfsetstrokecolor{currentstroke}%
\pgfsetdash{}{0pt}%
\pgfpathmoveto{\pgfqpoint{0.510000in}{0.880000in}}%
\pgfpathlineto{\pgfqpoint{3.590000in}{0.880000in}}%
\pgfusepath{stroke}%
\end{pgfscope}%
\begin{pgfscope}%
\pgfsetbuttcap%
\pgfsetroundjoin%
\definecolor{currentfill}{rgb}{0.000000,0.000000,0.000000}%
\pgfsetfillcolor{currentfill}%
\pgfsetlinewidth{0.803000pt}%
\definecolor{currentstroke}{rgb}{0.000000,0.000000,0.000000}%
\pgfsetstrokecolor{currentstroke}%
\pgfsetdash{}{0pt}%
\pgfsys@defobject{currentmarker}{\pgfqpoint{-0.048611in}{0.000000in}}{\pgfqpoint{0.000000in}{0.000000in}}{%
\pgfpathmoveto{\pgfqpoint{0.000000in}{0.000000in}}%
\pgfpathlineto{\pgfqpoint{-0.048611in}{0.000000in}}%
\pgfusepath{stroke,fill}%
}%
\begin{pgfscope}%
\pgfsys@transformshift{0.510000in}{0.880000in}%
\pgfsys@useobject{currentmarker}{}%
\end{pgfscope}%
\end{pgfscope}%
\begin{pgfscope}%
\definecolor{textcolor}{rgb}{0.000000,0.000000,0.000000}%
\pgfsetstrokecolor{textcolor}%
\pgfsetfillcolor{textcolor}%
\pgftext[x=0.191898in,y=0.827238in,left,base]{\color{textcolor}\sffamily\fontsize{10.000000}{12.000000}\selectfont 0.0}%
\end{pgfscope}%
\begin{pgfscope}%
\pgfpathrectangle{\pgfqpoint{0.510000in}{0.440000in}}{\pgfqpoint{3.080000in}{3.080000in}}%
\pgfusepath{clip}%
\pgfsetrectcap%
\pgfsetroundjoin%
\pgfsetlinewidth{0.803000pt}%
\definecolor{currentstroke}{rgb}{0.690196,0.690196,0.690196}%
\pgfsetstrokecolor{currentstroke}%
\pgfsetdash{}{0pt}%
\pgfpathmoveto{\pgfqpoint{0.510000in}{1.320000in}}%
\pgfpathlineto{\pgfqpoint{3.590000in}{1.320000in}}%
\pgfusepath{stroke}%
\end{pgfscope}%
\begin{pgfscope}%
\pgfsetbuttcap%
\pgfsetroundjoin%
\definecolor{currentfill}{rgb}{0.000000,0.000000,0.000000}%
\pgfsetfillcolor{currentfill}%
\pgfsetlinewidth{0.803000pt}%
\definecolor{currentstroke}{rgb}{0.000000,0.000000,0.000000}%
\pgfsetstrokecolor{currentstroke}%
\pgfsetdash{}{0pt}%
\pgfsys@defobject{currentmarker}{\pgfqpoint{-0.048611in}{0.000000in}}{\pgfqpoint{0.000000in}{0.000000in}}{%
\pgfpathmoveto{\pgfqpoint{0.000000in}{0.000000in}}%
\pgfpathlineto{\pgfqpoint{-0.048611in}{0.000000in}}%
\pgfusepath{stroke,fill}%
}%
\begin{pgfscope}%
\pgfsys@transformshift{0.510000in}{1.320000in}%
\pgfsys@useobject{currentmarker}{}%
\end{pgfscope}%
\end{pgfscope}%
\begin{pgfscope}%
\definecolor{textcolor}{rgb}{0.000000,0.000000,0.000000}%
\pgfsetstrokecolor{textcolor}%
\pgfsetfillcolor{textcolor}%
\pgftext[x=0.191898in,y=1.267238in,left,base]{\color{textcolor}\sffamily\fontsize{10.000000}{12.000000}\selectfont 0.2}%
\end{pgfscope}%
\begin{pgfscope}%
\pgfpathrectangle{\pgfqpoint{0.510000in}{0.440000in}}{\pgfqpoint{3.080000in}{3.080000in}}%
\pgfusepath{clip}%
\pgfsetrectcap%
\pgfsetroundjoin%
\pgfsetlinewidth{0.803000pt}%
\definecolor{currentstroke}{rgb}{0.690196,0.690196,0.690196}%
\pgfsetstrokecolor{currentstroke}%
\pgfsetdash{}{0pt}%
\pgfpathmoveto{\pgfqpoint{0.510000in}{1.760000in}}%
\pgfpathlineto{\pgfqpoint{3.590000in}{1.760000in}}%
\pgfusepath{stroke}%
\end{pgfscope}%
\begin{pgfscope}%
\pgfsetbuttcap%
\pgfsetroundjoin%
\definecolor{currentfill}{rgb}{0.000000,0.000000,0.000000}%
\pgfsetfillcolor{currentfill}%
\pgfsetlinewidth{0.803000pt}%
\definecolor{currentstroke}{rgb}{0.000000,0.000000,0.000000}%
\pgfsetstrokecolor{currentstroke}%
\pgfsetdash{}{0pt}%
\pgfsys@defobject{currentmarker}{\pgfqpoint{-0.048611in}{0.000000in}}{\pgfqpoint{0.000000in}{0.000000in}}{%
\pgfpathmoveto{\pgfqpoint{0.000000in}{0.000000in}}%
\pgfpathlineto{\pgfqpoint{-0.048611in}{0.000000in}}%
\pgfusepath{stroke,fill}%
}%
\begin{pgfscope}%
\pgfsys@transformshift{0.510000in}{1.760000in}%
\pgfsys@useobject{currentmarker}{}%
\end{pgfscope}%
\end{pgfscope}%
\begin{pgfscope}%
\definecolor{textcolor}{rgb}{0.000000,0.000000,0.000000}%
\pgfsetstrokecolor{textcolor}%
\pgfsetfillcolor{textcolor}%
\pgftext[x=0.191898in,y=1.707238in,left,base]{\color{textcolor}\sffamily\fontsize{10.000000}{12.000000}\selectfont 0.4}%
\end{pgfscope}%
\begin{pgfscope}%
\pgfpathrectangle{\pgfqpoint{0.510000in}{0.440000in}}{\pgfqpoint{3.080000in}{3.080000in}}%
\pgfusepath{clip}%
\pgfsetrectcap%
\pgfsetroundjoin%
\pgfsetlinewidth{0.803000pt}%
\definecolor{currentstroke}{rgb}{0.690196,0.690196,0.690196}%
\pgfsetstrokecolor{currentstroke}%
\pgfsetdash{}{0pt}%
\pgfpathmoveto{\pgfqpoint{0.510000in}{2.200000in}}%
\pgfpathlineto{\pgfqpoint{3.590000in}{2.200000in}}%
\pgfusepath{stroke}%
\end{pgfscope}%
\begin{pgfscope}%
\pgfsetbuttcap%
\pgfsetroundjoin%
\definecolor{currentfill}{rgb}{0.000000,0.000000,0.000000}%
\pgfsetfillcolor{currentfill}%
\pgfsetlinewidth{0.803000pt}%
\definecolor{currentstroke}{rgb}{0.000000,0.000000,0.000000}%
\pgfsetstrokecolor{currentstroke}%
\pgfsetdash{}{0pt}%
\pgfsys@defobject{currentmarker}{\pgfqpoint{-0.048611in}{0.000000in}}{\pgfqpoint{0.000000in}{0.000000in}}{%
\pgfpathmoveto{\pgfqpoint{0.000000in}{0.000000in}}%
\pgfpathlineto{\pgfqpoint{-0.048611in}{0.000000in}}%
\pgfusepath{stroke,fill}%
}%
\begin{pgfscope}%
\pgfsys@transformshift{0.510000in}{2.200000in}%
\pgfsys@useobject{currentmarker}{}%
\end{pgfscope}%
\end{pgfscope}%
\begin{pgfscope}%
\definecolor{textcolor}{rgb}{0.000000,0.000000,0.000000}%
\pgfsetstrokecolor{textcolor}%
\pgfsetfillcolor{textcolor}%
\pgftext[x=0.191898in,y=2.147238in,left,base]{\color{textcolor}\sffamily\fontsize{10.000000}{12.000000}\selectfont 0.6}%
\end{pgfscope}%
\begin{pgfscope}%
\pgfpathrectangle{\pgfqpoint{0.510000in}{0.440000in}}{\pgfqpoint{3.080000in}{3.080000in}}%
\pgfusepath{clip}%
\pgfsetrectcap%
\pgfsetroundjoin%
\pgfsetlinewidth{0.803000pt}%
\definecolor{currentstroke}{rgb}{0.690196,0.690196,0.690196}%
\pgfsetstrokecolor{currentstroke}%
\pgfsetdash{}{0pt}%
\pgfpathmoveto{\pgfqpoint{0.510000in}{2.640000in}}%
\pgfpathlineto{\pgfqpoint{3.590000in}{2.640000in}}%
\pgfusepath{stroke}%
\end{pgfscope}%
\begin{pgfscope}%
\pgfsetbuttcap%
\pgfsetroundjoin%
\definecolor{currentfill}{rgb}{0.000000,0.000000,0.000000}%
\pgfsetfillcolor{currentfill}%
\pgfsetlinewidth{0.803000pt}%
\definecolor{currentstroke}{rgb}{0.000000,0.000000,0.000000}%
\pgfsetstrokecolor{currentstroke}%
\pgfsetdash{}{0pt}%
\pgfsys@defobject{currentmarker}{\pgfqpoint{-0.048611in}{0.000000in}}{\pgfqpoint{0.000000in}{0.000000in}}{%
\pgfpathmoveto{\pgfqpoint{0.000000in}{0.000000in}}%
\pgfpathlineto{\pgfqpoint{-0.048611in}{0.000000in}}%
\pgfusepath{stroke,fill}%
}%
\begin{pgfscope}%
\pgfsys@transformshift{0.510000in}{2.640000in}%
\pgfsys@useobject{currentmarker}{}%
\end{pgfscope}%
\end{pgfscope}%
\begin{pgfscope}%
\definecolor{textcolor}{rgb}{0.000000,0.000000,0.000000}%
\pgfsetstrokecolor{textcolor}%
\pgfsetfillcolor{textcolor}%
\pgftext[x=0.191898in,y=2.587238in,left,base]{\color{textcolor}\sffamily\fontsize{10.000000}{12.000000}\selectfont 0.8}%
\end{pgfscope}%
\begin{pgfscope}%
\pgfpathrectangle{\pgfqpoint{0.510000in}{0.440000in}}{\pgfqpoint{3.080000in}{3.080000in}}%
\pgfusepath{clip}%
\pgfsetrectcap%
\pgfsetroundjoin%
\pgfsetlinewidth{0.803000pt}%
\definecolor{currentstroke}{rgb}{0.690196,0.690196,0.690196}%
\pgfsetstrokecolor{currentstroke}%
\pgfsetdash{}{0pt}%
\pgfpathmoveto{\pgfqpoint{0.510000in}{3.080000in}}%
\pgfpathlineto{\pgfqpoint{3.590000in}{3.080000in}}%
\pgfusepath{stroke}%
\end{pgfscope}%
\begin{pgfscope}%
\pgfsetbuttcap%
\pgfsetroundjoin%
\definecolor{currentfill}{rgb}{0.000000,0.000000,0.000000}%
\pgfsetfillcolor{currentfill}%
\pgfsetlinewidth{0.803000pt}%
\definecolor{currentstroke}{rgb}{0.000000,0.000000,0.000000}%
\pgfsetstrokecolor{currentstroke}%
\pgfsetdash{}{0pt}%
\pgfsys@defobject{currentmarker}{\pgfqpoint{-0.048611in}{0.000000in}}{\pgfqpoint{0.000000in}{0.000000in}}{%
\pgfpathmoveto{\pgfqpoint{0.000000in}{0.000000in}}%
\pgfpathlineto{\pgfqpoint{-0.048611in}{0.000000in}}%
\pgfusepath{stroke,fill}%
}%
\begin{pgfscope}%
\pgfsys@transformshift{0.510000in}{3.080000in}%
\pgfsys@useobject{currentmarker}{}%
\end{pgfscope}%
\end{pgfscope}%
\begin{pgfscope}%
\definecolor{textcolor}{rgb}{0.000000,0.000000,0.000000}%
\pgfsetstrokecolor{textcolor}%
\pgfsetfillcolor{textcolor}%
\pgftext[x=0.191898in,y=3.027238in,left,base]{\color{textcolor}\sffamily\fontsize{10.000000}{12.000000}\selectfont 1.0}%
\end{pgfscope}%
\begin{pgfscope}%
\pgfpathrectangle{\pgfqpoint{0.510000in}{0.440000in}}{\pgfqpoint{3.080000in}{3.080000in}}%
\pgfusepath{clip}%
\pgfsetrectcap%
\pgfsetroundjoin%
\pgfsetlinewidth{0.803000pt}%
\definecolor{currentstroke}{rgb}{0.690196,0.690196,0.690196}%
\pgfsetstrokecolor{currentstroke}%
\pgfsetdash{}{0pt}%
\pgfpathmoveto{\pgfqpoint{0.510000in}{3.520000in}}%
\pgfpathlineto{\pgfqpoint{3.590000in}{3.520000in}}%
\pgfusepath{stroke}%
\end{pgfscope}%
\begin{pgfscope}%
\pgfsetbuttcap%
\pgfsetroundjoin%
\definecolor{currentfill}{rgb}{0.000000,0.000000,0.000000}%
\pgfsetfillcolor{currentfill}%
\pgfsetlinewidth{0.803000pt}%
\definecolor{currentstroke}{rgb}{0.000000,0.000000,0.000000}%
\pgfsetstrokecolor{currentstroke}%
\pgfsetdash{}{0pt}%
\pgfsys@defobject{currentmarker}{\pgfqpoint{-0.048611in}{0.000000in}}{\pgfqpoint{0.000000in}{0.000000in}}{%
\pgfpathmoveto{\pgfqpoint{0.000000in}{0.000000in}}%
\pgfpathlineto{\pgfqpoint{-0.048611in}{0.000000in}}%
\pgfusepath{stroke,fill}%
}%
\begin{pgfscope}%
\pgfsys@transformshift{0.510000in}{3.520000in}%
\pgfsys@useobject{currentmarker}{}%
\end{pgfscope}%
\end{pgfscope}%
\begin{pgfscope}%
\definecolor{textcolor}{rgb}{0.000000,0.000000,0.000000}%
\pgfsetstrokecolor{textcolor}%
\pgfsetfillcolor{textcolor}%
\pgftext[x=0.191898in,y=3.467238in,left,base]{\color{textcolor}\sffamily\fontsize{10.000000}{12.000000}\selectfont 1.2}%
\end{pgfscope}%
\begin{pgfscope}%
\pgfsetrectcap%
\pgfsetmiterjoin%
\pgfsetlinewidth{0.803000pt}%
\definecolor{currentstroke}{rgb}{0.000000,0.000000,0.000000}%
\pgfsetstrokecolor{currentstroke}%
\pgfsetdash{}{0pt}%
\pgfpathmoveto{\pgfqpoint{0.510000in}{0.440000in}}%
\pgfpathlineto{\pgfqpoint{0.510000in}{3.520000in}}%
\pgfusepath{stroke}%
\end{pgfscope}%
\begin{pgfscope}%
\pgfsetrectcap%
\pgfsetmiterjoin%
\pgfsetlinewidth{0.803000pt}%
\definecolor{currentstroke}{rgb}{0.000000,0.000000,0.000000}%
\pgfsetstrokecolor{currentstroke}%
\pgfsetdash{}{0pt}%
\pgfpathmoveto{\pgfqpoint{3.590000in}{0.440000in}}%
\pgfpathlineto{\pgfqpoint{3.590000in}{3.520000in}}%
\pgfusepath{stroke}%
\end{pgfscope}%
\begin{pgfscope}%
\pgfsetrectcap%
\pgfsetmiterjoin%
\pgfsetlinewidth{0.803000pt}%
\definecolor{currentstroke}{rgb}{0.000000,0.000000,0.000000}%
\pgfsetstrokecolor{currentstroke}%
\pgfsetdash{}{0pt}%
\pgfpathmoveto{\pgfqpoint{0.510000in}{0.440000in}}%
\pgfpathlineto{\pgfqpoint{3.590000in}{0.440000in}}%
\pgfusepath{stroke}%
\end{pgfscope}%
\begin{pgfscope}%
\pgfsetrectcap%
\pgfsetmiterjoin%
\pgfsetlinewidth{0.803000pt}%
\definecolor{currentstroke}{rgb}{0.000000,0.000000,0.000000}%
\pgfsetstrokecolor{currentstroke}%
\pgfsetdash{}{0pt}%
\pgfpathmoveto{\pgfqpoint{0.510000in}{3.520000in}}%
\pgfpathlineto{\pgfqpoint{3.590000in}{3.520000in}}%
\pgfusepath{stroke}%
\end{pgfscope}%
\begin{pgfscope}%
\definecolor{textcolor}{rgb}{0.000000,0.000000,0.000000}%
\pgfsetstrokecolor{textcolor}%
\pgfsetfillcolor{textcolor}%
\pgftext[x=1.501005in,y=1.089440in,left,base]{\color{textcolor}\sffamily\fontsize{10.000000}{12.000000}\selectfont 0.16667}%
\end{pgfscope}%
\begin{pgfscope}%
\definecolor{textcolor}{rgb}{0.000000,0.000000,0.000000}%
\pgfsetstrokecolor{textcolor}%
\pgfsetfillcolor{textcolor}%
\pgftext[x=2.389631in,y=1.430920in,left,base]{\color{textcolor}\sffamily\fontsize{10.000000}{12.000000}\selectfont 0.16667}%
\end{pgfscope}%
\begin{pgfscope}%
\definecolor{textcolor}{rgb}{0.000000,0.000000,0.000000}%
\pgfsetstrokecolor{textcolor}%
\pgfsetfillcolor{textcolor}%
\pgftext[x=1.159379in,y=2.319655in,left,base]{\color{textcolor}\sffamily\fontsize{10.000000}{12.000000}\selectfont 0.16667}%
\end{pgfscope}%
\end{pgfpicture}%
\makeatother%
\endgroup%
}~
\scalebox{0.75}{%% Creator: Matplotlib, PGF backend
%%
%% To include the figure in your LaTeX document, write
%%   \input{<filename>.pgf}
%%
%% Make sure the required packages are loaded in your preamble
%%   \usepackage{pgf}
%%
%% Figures using additional raster images can only be included by \input if
%% they are in the same directory as the main LaTeX file. For loading figures
%% from other directories you can use the `import` package
%%   \usepackage{import}
%% and then include the figures with
%%   \import{<path to file>}{<filename>.pgf}
%%
%% Matplotlib used the following preamble
%%   \usepackage{fontspec}
%%   \setmainfont{DejaVuSerif.ttf}[Path=/home/lzh/anaconda3/envs/numana/lib/python3.7/site-packages/matplotlib/mpl-data/fonts/ttf/]
%%   \setsansfont{DejaVuSans.ttf}[Path=/home/lzh/anaconda3/envs/numana/lib/python3.7/site-packages/matplotlib/mpl-data/fonts/ttf/]
%%   \setmonofont{DejaVuSansMono.ttf}[Path=/home/lzh/anaconda3/envs/numana/lib/python3.7/site-packages/matplotlib/mpl-data/fonts/ttf/]
%%
\begingroup%
\makeatletter%
\begin{pgfpicture}%
\pgfpathrectangle{\pgfpointorigin}{\pgfqpoint{8.000000in}{6.000000in}}%
\pgfusepath{use as bounding box, clip}%
\begin{pgfscope}%
\pgfsetbuttcap%
\pgfsetmiterjoin%
\definecolor{currentfill}{rgb}{1.000000,1.000000,1.000000}%
\pgfsetfillcolor{currentfill}%
\pgfsetlinewidth{0.000000pt}%
\definecolor{currentstroke}{rgb}{1.000000,1.000000,1.000000}%
\pgfsetstrokecolor{currentstroke}%
\pgfsetdash{}{0pt}%
\pgfpathmoveto{\pgfqpoint{0.000000in}{0.000000in}}%
\pgfpathlineto{\pgfqpoint{8.000000in}{0.000000in}}%
\pgfpathlineto{\pgfqpoint{8.000000in}{6.000000in}}%
\pgfpathlineto{\pgfqpoint{0.000000in}{6.000000in}}%
\pgfpathclose%
\pgfusepath{fill}%
\end{pgfscope}%
\begin{pgfscope}%
\pgfsetbuttcap%
\pgfsetmiterjoin%
\definecolor{currentfill}{rgb}{1.000000,1.000000,1.000000}%
\pgfsetfillcolor{currentfill}%
\pgfsetlinewidth{0.000000pt}%
\definecolor{currentstroke}{rgb}{0.000000,0.000000,0.000000}%
\pgfsetstrokecolor{currentstroke}%
\pgfsetstrokeopacity{0.000000}%
\pgfsetdash{}{0pt}%
\pgfpathmoveto{\pgfqpoint{0.557986in}{3.363889in}}%
\pgfpathlineto{\pgfqpoint{3.876389in}{3.363889in}}%
\pgfpathlineto{\pgfqpoint{3.876389in}{5.627778in}}%
\pgfpathlineto{\pgfqpoint{0.557986in}{5.627778in}}%
\pgfpathclose%
\pgfusepath{fill}%
\end{pgfscope}%
\begin{pgfscope}%
\pgfpathrectangle{\pgfqpoint{0.557986in}{3.363889in}}{\pgfqpoint{3.318403in}{2.263889in}}%
\pgfusepath{clip}%
\pgfsetbuttcap%
\pgfsetroundjoin%
\definecolor{currentfill}{rgb}{0.121569,0.466667,0.705882}%
\pgfsetfillcolor{currentfill}%
\pgfsetlinewidth{1.003750pt}%
\definecolor{currentstroke}{rgb}{0.121569,0.466667,0.705882}%
\pgfsetstrokecolor{currentstroke}%
\pgfsetdash{}{0pt}%
\pgfsys@defobject{currentmarker}{\pgfqpoint{-0.009821in}{-0.009821in}}{\pgfqpoint{0.009821in}{0.009821in}}{%
\pgfpathmoveto{\pgfqpoint{0.000000in}{-0.009821in}}%
\pgfpathcurveto{\pgfqpoint{0.002605in}{-0.009821in}}{\pgfqpoint{0.005103in}{-0.008786in}}{\pgfqpoint{0.006944in}{-0.006944in}}%
\pgfpathcurveto{\pgfqpoint{0.008786in}{-0.005103in}}{\pgfqpoint{0.009821in}{-0.002605in}}{\pgfqpoint{0.009821in}{0.000000in}}%
\pgfpathcurveto{\pgfqpoint{0.009821in}{0.002605in}}{\pgfqpoint{0.008786in}{0.005103in}}{\pgfqpoint{0.006944in}{0.006944in}}%
\pgfpathcurveto{\pgfqpoint{0.005103in}{0.008786in}}{\pgfqpoint{0.002605in}{0.009821in}}{\pgfqpoint{0.000000in}{0.009821in}}%
\pgfpathcurveto{\pgfqpoint{-0.002605in}{0.009821in}}{\pgfqpoint{-0.005103in}{0.008786in}}{\pgfqpoint{-0.006944in}{0.006944in}}%
\pgfpathcurveto{\pgfqpoint{-0.008786in}{0.005103in}}{\pgfqpoint{-0.009821in}{0.002605in}}{\pgfqpoint{-0.009821in}{0.000000in}}%
\pgfpathcurveto{\pgfqpoint{-0.009821in}{-0.002605in}}{\pgfqpoint{-0.008786in}{-0.005103in}}{\pgfqpoint{-0.006944in}{-0.006944in}}%
\pgfpathcurveto{\pgfqpoint{-0.005103in}{-0.008786in}}{\pgfqpoint{-0.002605in}{-0.009821in}}{\pgfqpoint{0.000000in}{-0.009821in}}%
\pgfpathclose%
\pgfusepath{stroke,fill}%
}%
\begin{pgfscope}%
\pgfsys@transformshift{0.708823in}{3.839424in}%
\pgfsys@useobject{currentmarker}{}%
\end{pgfscope}%
\begin{pgfscope}%
\pgfsys@transformshift{0.859609in}{3.895280in}%
\pgfsys@useobject{currentmarker}{}%
\end{pgfscope}%
\begin{pgfscope}%
\pgfsys@transformshift{1.010395in}{4.628915in}%
\pgfsys@useobject{currentmarker}{}%
\end{pgfscope}%
\begin{pgfscope}%
\pgfsys@transformshift{1.161181in}{4.801410in}%
\pgfsys@useobject{currentmarker}{}%
\end{pgfscope}%
\begin{pgfscope}%
\pgfsys@transformshift{1.311967in}{4.655646in}%
\pgfsys@useobject{currentmarker}{}%
\end{pgfscope}%
\begin{pgfscope}%
\pgfsys@transformshift{1.462753in}{4.916088in}%
\pgfsys@useobject{currentmarker}{}%
\end{pgfscope}%
\begin{pgfscope}%
\pgfsys@transformshift{1.613539in}{4.879613in}%
\pgfsys@useobject{currentmarker}{}%
\end{pgfscope}%
\begin{pgfscope}%
\pgfsys@transformshift{1.764325in}{4.690595in}%
\pgfsys@useobject{currentmarker}{}%
\end{pgfscope}%
\begin{pgfscope}%
\pgfsys@transformshift{1.915111in}{4.999268in}%
\pgfsys@useobject{currentmarker}{}%
\end{pgfscope}%
\begin{pgfscope}%
\pgfsys@transformshift{2.065897in}{5.011569in}%
\pgfsys@useobject{currentmarker}{}%
\end{pgfscope}%
\begin{pgfscope}%
\pgfsys@transformshift{2.216683in}{5.011487in}%
\pgfsys@useobject{currentmarker}{}%
\end{pgfscope}%
\begin{pgfscope}%
\pgfsys@transformshift{2.367469in}{5.011527in}%
\pgfsys@useobject{currentmarker}{}%
\end{pgfscope}%
\begin{pgfscope}%
\pgfsys@transformshift{2.518255in}{5.011511in}%
\pgfsys@useobject{currentmarker}{}%
\end{pgfscope}%
\begin{pgfscope}%
\pgfsys@transformshift{2.669041in}{5.011527in}%
\pgfsys@useobject{currentmarker}{}%
\end{pgfscope}%
\begin{pgfscope}%
\pgfsys@transformshift{2.819827in}{5.011524in}%
\pgfsys@useobject{currentmarker}{}%
\end{pgfscope}%
\begin{pgfscope}%
\pgfsys@transformshift{2.970613in}{5.011547in}%
\pgfsys@useobject{currentmarker}{}%
\end{pgfscope}%
\begin{pgfscope}%
\pgfsys@transformshift{3.121399in}{5.011541in}%
\pgfsys@useobject{currentmarker}{}%
\end{pgfscope}%
\begin{pgfscope}%
\pgfsys@transformshift{3.272185in}{5.011555in}%
\pgfsys@useobject{currentmarker}{}%
\end{pgfscope}%
\begin{pgfscope}%
\pgfsys@transformshift{3.422971in}{5.011543in}%
\pgfsys@useobject{currentmarker}{}%
\end{pgfscope}%
\begin{pgfscope}%
\pgfsys@transformshift{3.573756in}{5.011551in}%
\pgfsys@useobject{currentmarker}{}%
\end{pgfscope}%
\begin{pgfscope}%
\pgfsys@transformshift{3.724542in}{5.011551in}%
\pgfsys@useobject{currentmarker}{}%
\end{pgfscope}%
\end{pgfscope}%
\begin{pgfscope}%
\pgfpathrectangle{\pgfqpoint{0.557986in}{3.363889in}}{\pgfqpoint{3.318403in}{2.263889in}}%
\pgfusepath{clip}%
\pgfsetbuttcap%
\pgfsetroundjoin%
\definecolor{currentfill}{rgb}{0.121569,0.466667,0.705882}%
\pgfsetfillcolor{currentfill}%
\pgfsetfillopacity{0.300000}%
\pgfsetlinewidth{1.003750pt}%
\definecolor{currentstroke}{rgb}{0.121569,0.466667,0.705882}%
\pgfsetstrokecolor{currentstroke}%
\pgfsetstrokeopacity{0.300000}%
\pgfsetdash{}{0pt}%
\pgfpathmoveto{\pgfqpoint{0.708823in}{4.129527in}}%
\pgfpathlineto{\pgfqpoint{0.708823in}{3.549321in}}%
\pgfpathlineto{\pgfqpoint{0.859609in}{3.594130in}}%
\pgfpathlineto{\pgfqpoint{1.010395in}{4.125324in}}%
\pgfpathlineto{\pgfqpoint{1.161181in}{4.410400in}}%
\pgfpathlineto{\pgfqpoint{1.311967in}{3.787348in}}%
\pgfpathlineto{\pgfqpoint{1.462753in}{4.668186in}}%
\pgfpathlineto{\pgfqpoint{1.613539in}{4.536131in}}%
\pgfpathlineto{\pgfqpoint{1.764325in}{3.856317in}}%
\pgfpathlineto{\pgfqpoint{1.915111in}{4.966961in}}%
\pgfpathlineto{\pgfqpoint{2.065897in}{5.011400in}}%
\pgfpathlineto{\pgfqpoint{2.216683in}{5.011340in}}%
\pgfpathlineto{\pgfqpoint{2.367469in}{5.011483in}}%
\pgfpathlineto{\pgfqpoint{2.518255in}{5.011446in}}%
\pgfpathlineto{\pgfqpoint{2.669041in}{5.011474in}}%
\pgfpathlineto{\pgfqpoint{2.819827in}{5.011462in}}%
\pgfpathlineto{\pgfqpoint{2.970613in}{5.011511in}}%
\pgfpathlineto{\pgfqpoint{3.121399in}{5.011523in}}%
\pgfpathlineto{\pgfqpoint{3.272185in}{5.011542in}}%
\pgfpathlineto{\pgfqpoint{3.422971in}{5.011517in}}%
\pgfpathlineto{\pgfqpoint{3.573756in}{5.011537in}}%
\pgfpathlineto{\pgfqpoint{3.724542in}{5.011546in}}%
\pgfpathlineto{\pgfqpoint{3.724542in}{5.011557in}}%
\pgfpathlineto{\pgfqpoint{3.724542in}{5.011557in}}%
\pgfpathlineto{\pgfqpoint{3.573756in}{5.011564in}}%
\pgfpathlineto{\pgfqpoint{3.422971in}{5.011569in}}%
\pgfpathlineto{\pgfqpoint{3.272185in}{5.011569in}}%
\pgfpathlineto{\pgfqpoint{3.121399in}{5.011559in}}%
\pgfpathlineto{\pgfqpoint{2.970613in}{5.011582in}}%
\pgfpathlineto{\pgfqpoint{2.819827in}{5.011587in}}%
\pgfpathlineto{\pgfqpoint{2.669041in}{5.011580in}}%
\pgfpathlineto{\pgfqpoint{2.518255in}{5.011575in}}%
\pgfpathlineto{\pgfqpoint{2.367469in}{5.011572in}}%
\pgfpathlineto{\pgfqpoint{2.216683in}{5.011634in}}%
\pgfpathlineto{\pgfqpoint{2.065897in}{5.011739in}}%
\pgfpathlineto{\pgfqpoint{1.915111in}{5.031575in}}%
\pgfpathlineto{\pgfqpoint{1.764325in}{5.524874in}}%
\pgfpathlineto{\pgfqpoint{1.613539in}{5.223096in}}%
\pgfpathlineto{\pgfqpoint{1.462753in}{5.163991in}}%
\pgfpathlineto{\pgfqpoint{1.311967in}{5.523945in}}%
\pgfpathlineto{\pgfqpoint{1.161181in}{5.192420in}}%
\pgfpathlineto{\pgfqpoint{1.010395in}{5.132505in}}%
\pgfpathlineto{\pgfqpoint{0.859609in}{4.196430in}}%
\pgfpathlineto{\pgfqpoint{0.708823in}{4.129527in}}%
\pgfpathclose%
\pgfusepath{stroke,fill}%
\end{pgfscope}%
\begin{pgfscope}%
\pgfpathrectangle{\pgfqpoint{0.557986in}{3.363889in}}{\pgfqpoint{3.318403in}{2.263889in}}%
\pgfusepath{clip}%
\pgfsetbuttcap%
\pgfsetroundjoin%
\definecolor{currentfill}{rgb}{1.000000,0.498039,0.054902}%
\pgfsetfillcolor{currentfill}%
\pgfsetlinewidth{1.003750pt}%
\definecolor{currentstroke}{rgb}{1.000000,0.498039,0.054902}%
\pgfsetstrokecolor{currentstroke}%
\pgfsetdash{}{0pt}%
\pgfsys@defobject{currentmarker}{\pgfqpoint{-0.009821in}{-0.009821in}}{\pgfqpoint{0.009821in}{0.009821in}}{%
\pgfpathmoveto{\pgfqpoint{0.000000in}{-0.009821in}}%
\pgfpathcurveto{\pgfqpoint{0.002605in}{-0.009821in}}{\pgfqpoint{0.005103in}{-0.008786in}}{\pgfqpoint{0.006944in}{-0.006944in}}%
\pgfpathcurveto{\pgfqpoint{0.008786in}{-0.005103in}}{\pgfqpoint{0.009821in}{-0.002605in}}{\pgfqpoint{0.009821in}{0.000000in}}%
\pgfpathcurveto{\pgfqpoint{0.009821in}{0.002605in}}{\pgfqpoint{0.008786in}{0.005103in}}{\pgfqpoint{0.006944in}{0.006944in}}%
\pgfpathcurveto{\pgfqpoint{0.005103in}{0.008786in}}{\pgfqpoint{0.002605in}{0.009821in}}{\pgfqpoint{0.000000in}{0.009821in}}%
\pgfpathcurveto{\pgfqpoint{-0.002605in}{0.009821in}}{\pgfqpoint{-0.005103in}{0.008786in}}{\pgfqpoint{-0.006944in}{0.006944in}}%
\pgfpathcurveto{\pgfqpoint{-0.008786in}{0.005103in}}{\pgfqpoint{-0.009821in}{0.002605in}}{\pgfqpoint{-0.009821in}{0.000000in}}%
\pgfpathcurveto{\pgfqpoint{-0.009821in}{-0.002605in}}{\pgfqpoint{-0.008786in}{-0.005103in}}{\pgfqpoint{-0.006944in}{-0.006944in}}%
\pgfpathcurveto{\pgfqpoint{-0.005103in}{-0.008786in}}{\pgfqpoint{-0.002605in}{-0.009821in}}{\pgfqpoint{0.000000in}{-0.009821in}}%
\pgfpathclose%
\pgfusepath{stroke,fill}%
}%
\begin{pgfscope}%
\pgfsys@transformshift{0.708823in}{3.900071in}%
\pgfsys@useobject{currentmarker}{}%
\end{pgfscope}%
\begin{pgfscope}%
\pgfsys@transformshift{0.859609in}{4.010608in}%
\pgfsys@useobject{currentmarker}{}%
\end{pgfscope}%
\begin{pgfscope}%
\pgfsys@transformshift{1.010395in}{4.198230in}%
\pgfsys@useobject{currentmarker}{}%
\end{pgfscope}%
\begin{pgfscope}%
\pgfsys@transformshift{1.161181in}{4.861184in}%
\pgfsys@useobject{currentmarker}{}%
\end{pgfscope}%
\begin{pgfscope}%
\pgfsys@transformshift{1.311967in}{4.504774in}%
\pgfsys@useobject{currentmarker}{}%
\end{pgfscope}%
\begin{pgfscope}%
\pgfsys@transformshift{1.462753in}{4.662495in}%
\pgfsys@useobject{currentmarker}{}%
\end{pgfscope}%
\begin{pgfscope}%
\pgfsys@transformshift{1.613539in}{4.686331in}%
\pgfsys@useobject{currentmarker}{}%
\end{pgfscope}%
\begin{pgfscope}%
\pgfsys@transformshift{1.764325in}{4.993549in}%
\pgfsys@useobject{currentmarker}{}%
\end{pgfscope}%
\begin{pgfscope}%
\pgfsys@transformshift{1.915111in}{4.992256in}%
\pgfsys@useobject{currentmarker}{}%
\end{pgfscope}%
\begin{pgfscope}%
\pgfsys@transformshift{2.065897in}{4.992991in}%
\pgfsys@useobject{currentmarker}{}%
\end{pgfscope}%
\begin{pgfscope}%
\pgfsys@transformshift{2.216683in}{4.993445in}%
\pgfsys@useobject{currentmarker}{}%
\end{pgfscope}%
\begin{pgfscope}%
\pgfsys@transformshift{2.367469in}{4.992858in}%
\pgfsys@useobject{currentmarker}{}%
\end{pgfscope}%
\begin{pgfscope}%
\pgfsys@transformshift{2.518255in}{4.992895in}%
\pgfsys@useobject{currentmarker}{}%
\end{pgfscope}%
\begin{pgfscope}%
\pgfsys@transformshift{2.669041in}{4.992941in}%
\pgfsys@useobject{currentmarker}{}%
\end{pgfscope}%
\begin{pgfscope}%
\pgfsys@transformshift{2.819827in}{4.992881in}%
\pgfsys@useobject{currentmarker}{}%
\end{pgfscope}%
\begin{pgfscope}%
\pgfsys@transformshift{2.970613in}{4.992896in}%
\pgfsys@useobject{currentmarker}{}%
\end{pgfscope}%
\begin{pgfscope}%
\pgfsys@transformshift{3.121399in}{4.992917in}%
\pgfsys@useobject{currentmarker}{}%
\end{pgfscope}%
\begin{pgfscope}%
\pgfsys@transformshift{3.272185in}{4.992870in}%
\pgfsys@useobject{currentmarker}{}%
\end{pgfscope}%
\begin{pgfscope}%
\pgfsys@transformshift{3.422971in}{4.992896in}%
\pgfsys@useobject{currentmarker}{}%
\end{pgfscope}%
\begin{pgfscope}%
\pgfsys@transformshift{3.573756in}{4.992926in}%
\pgfsys@useobject{currentmarker}{}%
\end{pgfscope}%
\begin{pgfscope}%
\pgfsys@transformshift{3.724542in}{4.992870in}%
\pgfsys@useobject{currentmarker}{}%
\end{pgfscope}%
\end{pgfscope}%
\begin{pgfscope}%
\pgfpathrectangle{\pgfqpoint{0.557986in}{3.363889in}}{\pgfqpoint{3.318403in}{2.263889in}}%
\pgfusepath{clip}%
\pgfsetbuttcap%
\pgfsetroundjoin%
\definecolor{currentfill}{rgb}{1.000000,0.498039,0.054902}%
\pgfsetfillcolor{currentfill}%
\pgfsetfillopacity{0.300000}%
\pgfsetlinewidth{1.003750pt}%
\definecolor{currentstroke}{rgb}{1.000000,0.498039,0.054902}%
\pgfsetstrokecolor{currentstroke}%
\pgfsetstrokeopacity{0.300000}%
\pgfsetdash{}{0pt}%
\pgfpathmoveto{\pgfqpoint{0.708823in}{4.290679in}}%
\pgfpathlineto{\pgfqpoint{0.708823in}{3.509462in}}%
\pgfpathlineto{\pgfqpoint{0.859609in}{3.704856in}}%
\pgfpathlineto{\pgfqpoint{1.010395in}{3.653418in}}%
\pgfpathlineto{\pgfqpoint{1.161181in}{4.670864in}}%
\pgfpathlineto{\pgfqpoint{1.311967in}{3.778879in}}%
\pgfpathlineto{\pgfqpoint{1.462753in}{3.843370in}}%
\pgfpathlineto{\pgfqpoint{1.613539in}{3.889822in}}%
\pgfpathlineto{\pgfqpoint{1.764325in}{4.991041in}}%
\pgfpathlineto{\pgfqpoint{1.915111in}{4.991633in}}%
\pgfpathlineto{\pgfqpoint{2.065897in}{4.992001in}}%
\pgfpathlineto{\pgfqpoint{2.216683in}{4.992974in}}%
\pgfpathlineto{\pgfqpoint{2.367469in}{4.992336in}}%
\pgfpathlineto{\pgfqpoint{2.518255in}{4.992180in}}%
\pgfpathlineto{\pgfqpoint{2.669041in}{4.992686in}}%
\pgfpathlineto{\pgfqpoint{2.819827in}{4.992500in}}%
\pgfpathlineto{\pgfqpoint{2.970613in}{4.992681in}}%
\pgfpathlineto{\pgfqpoint{3.121399in}{4.992764in}}%
\pgfpathlineto{\pgfqpoint{3.272185in}{4.992798in}}%
\pgfpathlineto{\pgfqpoint{3.422971in}{4.992797in}}%
\pgfpathlineto{\pgfqpoint{3.573756in}{4.992876in}}%
\pgfpathlineto{\pgfqpoint{3.724542in}{4.992847in}}%
\pgfpathlineto{\pgfqpoint{3.724542in}{4.992894in}}%
\pgfpathlineto{\pgfqpoint{3.724542in}{4.992894in}}%
\pgfpathlineto{\pgfqpoint{3.573756in}{4.992976in}}%
\pgfpathlineto{\pgfqpoint{3.422971in}{4.992995in}}%
\pgfpathlineto{\pgfqpoint{3.272185in}{4.992942in}}%
\pgfpathlineto{\pgfqpoint{3.121399in}{4.993070in}}%
\pgfpathlineto{\pgfqpoint{2.970613in}{4.993111in}}%
\pgfpathlineto{\pgfqpoint{2.819827in}{4.993263in}}%
\pgfpathlineto{\pgfqpoint{2.669041in}{4.993195in}}%
\pgfpathlineto{\pgfqpoint{2.518255in}{4.993611in}}%
\pgfpathlineto{\pgfqpoint{2.367469in}{4.993380in}}%
\pgfpathlineto{\pgfqpoint{2.216683in}{4.993916in}}%
\pgfpathlineto{\pgfqpoint{2.065897in}{4.993981in}}%
\pgfpathlineto{\pgfqpoint{1.915111in}{4.992879in}}%
\pgfpathlineto{\pgfqpoint{1.764325in}{4.996057in}}%
\pgfpathlineto{\pgfqpoint{1.613539in}{5.482841in}}%
\pgfpathlineto{\pgfqpoint{1.462753in}{5.481620in}}%
\pgfpathlineto{\pgfqpoint{1.311967in}{5.230670in}}%
\pgfpathlineto{\pgfqpoint{1.161181in}{5.051504in}}%
\pgfpathlineto{\pgfqpoint{1.010395in}{4.743042in}}%
\pgfpathlineto{\pgfqpoint{0.859609in}{4.316360in}}%
\pgfpathlineto{\pgfqpoint{0.708823in}{4.290679in}}%
\pgfpathclose%
\pgfusepath{stroke,fill}%
\end{pgfscope}%
\begin{pgfscope}%
\pgfpathrectangle{\pgfqpoint{0.557986in}{3.363889in}}{\pgfqpoint{3.318403in}{2.263889in}}%
\pgfusepath{clip}%
\pgfsetbuttcap%
\pgfsetroundjoin%
\definecolor{currentfill}{rgb}{0.172549,0.627451,0.172549}%
\pgfsetfillcolor{currentfill}%
\pgfsetlinewidth{1.003750pt}%
\definecolor{currentstroke}{rgb}{0.172549,0.627451,0.172549}%
\pgfsetstrokecolor{currentstroke}%
\pgfsetdash{}{0pt}%
\pgfsys@defobject{currentmarker}{\pgfqpoint{-0.009821in}{-0.009821in}}{\pgfqpoint{0.009821in}{0.009821in}}{%
\pgfpathmoveto{\pgfqpoint{0.000000in}{-0.009821in}}%
\pgfpathcurveto{\pgfqpoint{0.002605in}{-0.009821in}}{\pgfqpoint{0.005103in}{-0.008786in}}{\pgfqpoint{0.006944in}{-0.006944in}}%
\pgfpathcurveto{\pgfqpoint{0.008786in}{-0.005103in}}{\pgfqpoint{0.009821in}{-0.002605in}}{\pgfqpoint{0.009821in}{0.000000in}}%
\pgfpathcurveto{\pgfqpoint{0.009821in}{0.002605in}}{\pgfqpoint{0.008786in}{0.005103in}}{\pgfqpoint{0.006944in}{0.006944in}}%
\pgfpathcurveto{\pgfqpoint{0.005103in}{0.008786in}}{\pgfqpoint{0.002605in}{0.009821in}}{\pgfqpoint{0.000000in}{0.009821in}}%
\pgfpathcurveto{\pgfqpoint{-0.002605in}{0.009821in}}{\pgfqpoint{-0.005103in}{0.008786in}}{\pgfqpoint{-0.006944in}{0.006944in}}%
\pgfpathcurveto{\pgfqpoint{-0.008786in}{0.005103in}}{\pgfqpoint{-0.009821in}{0.002605in}}{\pgfqpoint{-0.009821in}{0.000000in}}%
\pgfpathcurveto{\pgfqpoint{-0.009821in}{-0.002605in}}{\pgfqpoint{-0.008786in}{-0.005103in}}{\pgfqpoint{-0.006944in}{-0.006944in}}%
\pgfpathcurveto{\pgfqpoint{-0.005103in}{-0.008786in}}{\pgfqpoint{-0.002605in}{-0.009821in}}{\pgfqpoint{0.000000in}{-0.009821in}}%
\pgfpathclose%
\pgfusepath{stroke,fill}%
}%
\begin{pgfscope}%
\pgfsys@transformshift{0.708823in}{3.885386in}%
\pgfsys@useobject{currentmarker}{}%
\end{pgfscope}%
\begin{pgfscope}%
\pgfsys@transformshift{0.859609in}{3.928319in}%
\pgfsys@useobject{currentmarker}{}%
\end{pgfscope}%
\begin{pgfscope}%
\pgfsys@transformshift{1.010395in}{3.944716in}%
\pgfsys@useobject{currentmarker}{}%
\end{pgfscope}%
\begin{pgfscope}%
\pgfsys@transformshift{1.161181in}{4.209986in}%
\pgfsys@useobject{currentmarker}{}%
\end{pgfscope}%
\begin{pgfscope}%
\pgfsys@transformshift{1.311967in}{4.561953in}%
\pgfsys@useobject{currentmarker}{}%
\end{pgfscope}%
\begin{pgfscope}%
\pgfsys@transformshift{1.462753in}{4.621749in}%
\pgfsys@useobject{currentmarker}{}%
\end{pgfscope}%
\begin{pgfscope}%
\pgfsys@transformshift{1.613539in}{4.886221in}%
\pgfsys@useobject{currentmarker}{}%
\end{pgfscope}%
\begin{pgfscope}%
\pgfsys@transformshift{1.764325in}{4.889795in}%
\pgfsys@useobject{currentmarker}{}%
\end{pgfscope}%
\begin{pgfscope}%
\pgfsys@transformshift{1.915111in}{4.881060in}%
\pgfsys@useobject{currentmarker}{}%
\end{pgfscope}%
\begin{pgfscope}%
\pgfsys@transformshift{2.065897in}{4.886999in}%
\pgfsys@useobject{currentmarker}{}%
\end{pgfscope}%
\begin{pgfscope}%
\pgfsys@transformshift{2.216683in}{4.887890in}%
\pgfsys@useobject{currentmarker}{}%
\end{pgfscope}%
\begin{pgfscope}%
\pgfsys@transformshift{2.367469in}{4.879207in}%
\pgfsys@useobject{currentmarker}{}%
\end{pgfscope}%
\begin{pgfscope}%
\pgfsys@transformshift{2.518255in}{4.883891in}%
\pgfsys@useobject{currentmarker}{}%
\end{pgfscope}%
\begin{pgfscope}%
\pgfsys@transformshift{2.669041in}{4.884801in}%
\pgfsys@useobject{currentmarker}{}%
\end{pgfscope}%
\begin{pgfscope}%
\pgfsys@transformshift{2.819827in}{4.882169in}%
\pgfsys@useobject{currentmarker}{}%
\end{pgfscope}%
\begin{pgfscope}%
\pgfsys@transformshift{2.970613in}{4.882493in}%
\pgfsys@useobject{currentmarker}{}%
\end{pgfscope}%
\begin{pgfscope}%
\pgfsys@transformshift{3.121399in}{4.882825in}%
\pgfsys@useobject{currentmarker}{}%
\end{pgfscope}%
\begin{pgfscope}%
\pgfsys@transformshift{3.272185in}{4.882584in}%
\pgfsys@useobject{currentmarker}{}%
\end{pgfscope}%
\begin{pgfscope}%
\pgfsys@transformshift{3.422971in}{4.883307in}%
\pgfsys@useobject{currentmarker}{}%
\end{pgfscope}%
\begin{pgfscope}%
\pgfsys@transformshift{3.573756in}{4.883094in}%
\pgfsys@useobject{currentmarker}{}%
\end{pgfscope}%
\begin{pgfscope}%
\pgfsys@transformshift{3.724542in}{4.883098in}%
\pgfsys@useobject{currentmarker}{}%
\end{pgfscope}%
\end{pgfscope}%
\begin{pgfscope}%
\pgfpathrectangle{\pgfqpoint{0.557986in}{3.363889in}}{\pgfqpoint{3.318403in}{2.263889in}}%
\pgfusepath{clip}%
\pgfsetbuttcap%
\pgfsetroundjoin%
\definecolor{currentfill}{rgb}{0.172549,0.627451,0.172549}%
\pgfsetfillcolor{currentfill}%
\pgfsetfillopacity{0.300000}%
\pgfsetlinewidth{1.003750pt}%
\definecolor{currentstroke}{rgb}{0.172549,0.627451,0.172549}%
\pgfsetstrokecolor{currentstroke}%
\pgfsetstrokeopacity{0.300000}%
\pgfsetdash{}{0pt}%
\pgfpathmoveto{\pgfqpoint{0.708823in}{4.303980in}}%
\pgfpathlineto{\pgfqpoint{0.708823in}{3.466793in}}%
\pgfpathlineto{\pgfqpoint{0.859609in}{3.562580in}}%
\pgfpathlineto{\pgfqpoint{1.010395in}{3.474480in}}%
\pgfpathlineto{\pgfqpoint{1.161181in}{3.629183in}}%
\pgfpathlineto{\pgfqpoint{1.311967in}{3.962912in}}%
\pgfpathlineto{\pgfqpoint{1.462753in}{4.098776in}}%
\pgfpathlineto{\pgfqpoint{1.613539in}{4.867434in}}%
\pgfpathlineto{\pgfqpoint{1.764325in}{4.880684in}}%
\pgfpathlineto{\pgfqpoint{1.915111in}{4.857279in}}%
\pgfpathlineto{\pgfqpoint{2.065897in}{4.881748in}}%
\pgfpathlineto{\pgfqpoint{2.216683in}{4.879704in}}%
\pgfpathlineto{\pgfqpoint{2.367469in}{4.871829in}}%
\pgfpathlineto{\pgfqpoint{2.518255in}{4.879202in}}%
\pgfpathlineto{\pgfqpoint{2.669041in}{4.883792in}}%
\pgfpathlineto{\pgfqpoint{2.819827in}{4.879931in}}%
\pgfpathlineto{\pgfqpoint{2.970613in}{4.880667in}}%
\pgfpathlineto{\pgfqpoint{3.121399in}{4.880628in}}%
\pgfpathlineto{\pgfqpoint{3.272185in}{4.881724in}}%
\pgfpathlineto{\pgfqpoint{3.422971in}{4.882241in}}%
\pgfpathlineto{\pgfqpoint{3.573756in}{4.882439in}}%
\pgfpathlineto{\pgfqpoint{3.724542in}{4.882618in}}%
\pgfpathlineto{\pgfqpoint{3.724542in}{4.883579in}}%
\pgfpathlineto{\pgfqpoint{3.724542in}{4.883579in}}%
\pgfpathlineto{\pgfqpoint{3.573756in}{4.883749in}}%
\pgfpathlineto{\pgfqpoint{3.422971in}{4.884373in}}%
\pgfpathlineto{\pgfqpoint{3.272185in}{4.883443in}}%
\pgfpathlineto{\pgfqpoint{3.121399in}{4.885022in}}%
\pgfpathlineto{\pgfqpoint{2.970613in}{4.884320in}}%
\pgfpathlineto{\pgfqpoint{2.819827in}{4.884407in}}%
\pgfpathlineto{\pgfqpoint{2.669041in}{4.885809in}}%
\pgfpathlineto{\pgfqpoint{2.518255in}{4.888581in}}%
\pgfpathlineto{\pgfqpoint{2.367469in}{4.886584in}}%
\pgfpathlineto{\pgfqpoint{2.216683in}{4.896076in}}%
\pgfpathlineto{\pgfqpoint{2.065897in}{4.892251in}}%
\pgfpathlineto{\pgfqpoint{1.915111in}{4.904842in}}%
\pgfpathlineto{\pgfqpoint{1.764325in}{4.898906in}}%
\pgfpathlineto{\pgfqpoint{1.613539in}{4.905008in}}%
\pgfpathlineto{\pgfqpoint{1.462753in}{5.144722in}}%
\pgfpathlineto{\pgfqpoint{1.311967in}{5.160994in}}%
\pgfpathlineto{\pgfqpoint{1.161181in}{4.790788in}}%
\pgfpathlineto{\pgfqpoint{1.010395in}{4.414952in}}%
\pgfpathlineto{\pgfqpoint{0.859609in}{4.294057in}}%
\pgfpathlineto{\pgfqpoint{0.708823in}{4.303980in}}%
\pgfpathclose%
\pgfusepath{stroke,fill}%
\end{pgfscope}%
\begin{pgfscope}%
\pgfpathrectangle{\pgfqpoint{0.557986in}{3.363889in}}{\pgfqpoint{3.318403in}{2.263889in}}%
\pgfusepath{clip}%
\pgfsetbuttcap%
\pgfsetroundjoin%
\definecolor{currentfill}{rgb}{0.839216,0.152941,0.156863}%
\pgfsetfillcolor{currentfill}%
\pgfsetlinewidth{1.003750pt}%
\definecolor{currentstroke}{rgb}{0.839216,0.152941,0.156863}%
\pgfsetstrokecolor{currentstroke}%
\pgfsetdash{}{0pt}%
\pgfsys@defobject{currentmarker}{\pgfqpoint{-0.009821in}{-0.009821in}}{\pgfqpoint{0.009821in}{0.009821in}}{%
\pgfpathmoveto{\pgfqpoint{0.000000in}{-0.009821in}}%
\pgfpathcurveto{\pgfqpoint{0.002605in}{-0.009821in}}{\pgfqpoint{0.005103in}{-0.008786in}}{\pgfqpoint{0.006944in}{-0.006944in}}%
\pgfpathcurveto{\pgfqpoint{0.008786in}{-0.005103in}}{\pgfqpoint{0.009821in}{-0.002605in}}{\pgfqpoint{0.009821in}{0.000000in}}%
\pgfpathcurveto{\pgfqpoint{0.009821in}{0.002605in}}{\pgfqpoint{0.008786in}{0.005103in}}{\pgfqpoint{0.006944in}{0.006944in}}%
\pgfpathcurveto{\pgfqpoint{0.005103in}{0.008786in}}{\pgfqpoint{0.002605in}{0.009821in}}{\pgfqpoint{0.000000in}{0.009821in}}%
\pgfpathcurveto{\pgfqpoint{-0.002605in}{0.009821in}}{\pgfqpoint{-0.005103in}{0.008786in}}{\pgfqpoint{-0.006944in}{0.006944in}}%
\pgfpathcurveto{\pgfqpoint{-0.008786in}{0.005103in}}{\pgfqpoint{-0.009821in}{0.002605in}}{\pgfqpoint{-0.009821in}{0.000000in}}%
\pgfpathcurveto{\pgfqpoint{-0.009821in}{-0.002605in}}{\pgfqpoint{-0.008786in}{-0.005103in}}{\pgfqpoint{-0.006944in}{-0.006944in}}%
\pgfpathcurveto{\pgfqpoint{-0.005103in}{-0.008786in}}{\pgfqpoint{-0.002605in}{-0.009821in}}{\pgfqpoint{0.000000in}{-0.009821in}}%
\pgfpathclose%
\pgfusepath{stroke,fill}%
}%
\begin{pgfscope}%
\pgfsys@transformshift{0.708823in}{3.769642in}%
\pgfsys@useobject{currentmarker}{}%
\end{pgfscope}%
\begin{pgfscope}%
\pgfsys@transformshift{0.859609in}{4.130453in}%
\pgfsys@useobject{currentmarker}{}%
\end{pgfscope}%
\begin{pgfscope}%
\pgfsys@transformshift{1.010395in}{3.875313in}%
\pgfsys@useobject{currentmarker}{}%
\end{pgfscope}%
\begin{pgfscope}%
\pgfsys@transformshift{1.161181in}{3.988322in}%
\pgfsys@useobject{currentmarker}{}%
\end{pgfscope}%
\begin{pgfscope}%
\pgfsys@transformshift{1.311967in}{4.316108in}%
\pgfsys@useobject{currentmarker}{}%
\end{pgfscope}%
\begin{pgfscope}%
\pgfsys@transformshift{1.462753in}{4.225938in}%
\pgfsys@useobject{currentmarker}{}%
\end{pgfscope}%
\begin{pgfscope}%
\pgfsys@transformshift{1.613539in}{4.033362in}%
\pgfsys@useobject{currentmarker}{}%
\end{pgfscope}%
\begin{pgfscope}%
\pgfsys@transformshift{1.764325in}{4.175058in}%
\pgfsys@useobject{currentmarker}{}%
\end{pgfscope}%
\begin{pgfscope}%
\pgfsys@transformshift{1.915111in}{4.130992in}%
\pgfsys@useobject{currentmarker}{}%
\end{pgfscope}%
\begin{pgfscope}%
\pgfsys@transformshift{2.065897in}{4.113252in}%
\pgfsys@useobject{currentmarker}{}%
\end{pgfscope}%
\begin{pgfscope}%
\pgfsys@transformshift{2.216683in}{4.094158in}%
\pgfsys@useobject{currentmarker}{}%
\end{pgfscope}%
\begin{pgfscope}%
\pgfsys@transformshift{2.367469in}{4.153160in}%
\pgfsys@useobject{currentmarker}{}%
\end{pgfscope}%
\begin{pgfscope}%
\pgfsys@transformshift{2.518255in}{4.115981in}%
\pgfsys@useobject{currentmarker}{}%
\end{pgfscope}%
\begin{pgfscope}%
\pgfsys@transformshift{2.669041in}{4.117308in}%
\pgfsys@useobject{currentmarker}{}%
\end{pgfscope}%
\begin{pgfscope}%
\pgfsys@transformshift{2.819827in}{4.124745in}%
\pgfsys@useobject{currentmarker}{}%
\end{pgfscope}%
\begin{pgfscope}%
\pgfsys@transformshift{2.970613in}{4.113389in}%
\pgfsys@useobject{currentmarker}{}%
\end{pgfscope}%
\begin{pgfscope}%
\pgfsys@transformshift{3.121399in}{4.135437in}%
\pgfsys@useobject{currentmarker}{}%
\end{pgfscope}%
\begin{pgfscope}%
\pgfsys@transformshift{3.272185in}{4.113997in}%
\pgfsys@useobject{currentmarker}{}%
\end{pgfscope}%
\begin{pgfscope}%
\pgfsys@transformshift{3.422971in}{4.120470in}%
\pgfsys@useobject{currentmarker}{}%
\end{pgfscope}%
\begin{pgfscope}%
\pgfsys@transformshift{3.573756in}{4.118492in}%
\pgfsys@useobject{currentmarker}{}%
\end{pgfscope}%
\begin{pgfscope}%
\pgfsys@transformshift{3.724542in}{4.119344in}%
\pgfsys@useobject{currentmarker}{}%
\end{pgfscope}%
\end{pgfscope}%
\begin{pgfscope}%
\pgfpathrectangle{\pgfqpoint{0.557986in}{3.363889in}}{\pgfqpoint{3.318403in}{2.263889in}}%
\pgfusepath{clip}%
\pgfsetbuttcap%
\pgfsetroundjoin%
\definecolor{currentfill}{rgb}{0.839216,0.152941,0.156863}%
\pgfsetfillcolor{currentfill}%
\pgfsetfillopacity{0.300000}%
\pgfsetlinewidth{1.003750pt}%
\definecolor{currentstroke}{rgb}{0.839216,0.152941,0.156863}%
\pgfsetstrokecolor{currentstroke}%
\pgfsetstrokeopacity{0.300000}%
\pgfsetdash{}{0pt}%
\pgfpathmoveto{\pgfqpoint{0.708823in}{3.992717in}}%
\pgfpathlineto{\pgfqpoint{0.708823in}{3.546568in}}%
\pgfpathlineto{\pgfqpoint{0.859609in}{3.784783in}}%
\pgfpathlineto{\pgfqpoint{1.010395in}{3.648220in}}%
\pgfpathlineto{\pgfqpoint{1.161181in}{3.794873in}}%
\pgfpathlineto{\pgfqpoint{1.311967in}{4.079985in}}%
\pgfpathlineto{\pgfqpoint{1.462753in}{3.891717in}}%
\pgfpathlineto{\pgfqpoint{1.613539in}{3.852596in}}%
\pgfpathlineto{\pgfqpoint{1.764325in}{4.082865in}}%
\pgfpathlineto{\pgfqpoint{1.915111in}{4.050095in}}%
\pgfpathlineto{\pgfqpoint{2.065897in}{4.039060in}}%
\pgfpathlineto{\pgfqpoint{2.216683in}{4.048817in}}%
\pgfpathlineto{\pgfqpoint{2.367469in}{4.122324in}}%
\pgfpathlineto{\pgfqpoint{2.518255in}{4.055218in}}%
\pgfpathlineto{\pgfqpoint{2.669041in}{4.089211in}}%
\pgfpathlineto{\pgfqpoint{2.819827in}{4.107676in}}%
\pgfpathlineto{\pgfqpoint{2.970613in}{4.095718in}}%
\pgfpathlineto{\pgfqpoint{3.121399in}{4.118947in}}%
\pgfpathlineto{\pgfqpoint{3.272185in}{4.103953in}}%
\pgfpathlineto{\pgfqpoint{3.422971in}{4.114093in}}%
\pgfpathlineto{\pgfqpoint{3.573756in}{4.114423in}}%
\pgfpathlineto{\pgfqpoint{3.724542in}{4.116187in}}%
\pgfpathlineto{\pgfqpoint{3.724542in}{4.122501in}}%
\pgfpathlineto{\pgfqpoint{3.724542in}{4.122501in}}%
\pgfpathlineto{\pgfqpoint{3.573756in}{4.122561in}}%
\pgfpathlineto{\pgfqpoint{3.422971in}{4.126846in}}%
\pgfpathlineto{\pgfqpoint{3.272185in}{4.124041in}}%
\pgfpathlineto{\pgfqpoint{3.121399in}{4.151927in}}%
\pgfpathlineto{\pgfqpoint{2.970613in}{4.131061in}}%
\pgfpathlineto{\pgfqpoint{2.819827in}{4.141815in}}%
\pgfpathlineto{\pgfqpoint{2.669041in}{4.145404in}}%
\pgfpathlineto{\pgfqpoint{2.518255in}{4.176744in}}%
\pgfpathlineto{\pgfqpoint{2.367469in}{4.183996in}}%
\pgfpathlineto{\pgfqpoint{2.216683in}{4.139498in}}%
\pgfpathlineto{\pgfqpoint{2.065897in}{4.187443in}}%
\pgfpathlineto{\pgfqpoint{1.915111in}{4.211888in}}%
\pgfpathlineto{\pgfqpoint{1.764325in}{4.267250in}}%
\pgfpathlineto{\pgfqpoint{1.613539in}{4.214129in}}%
\pgfpathlineto{\pgfqpoint{1.462753in}{4.560158in}}%
\pgfpathlineto{\pgfqpoint{1.311967in}{4.552230in}}%
\pgfpathlineto{\pgfqpoint{1.161181in}{4.181772in}}%
\pgfpathlineto{\pgfqpoint{1.010395in}{4.102406in}}%
\pgfpathlineto{\pgfqpoint{0.859609in}{4.476123in}}%
\pgfpathlineto{\pgfqpoint{0.708823in}{3.992717in}}%
\pgfpathclose%
\pgfusepath{stroke,fill}%
\end{pgfscope}%
\begin{pgfscope}%
\pgfpathrectangle{\pgfqpoint{0.557986in}{3.363889in}}{\pgfqpoint{3.318403in}{2.263889in}}%
\pgfusepath{clip}%
\pgfsetbuttcap%
\pgfsetroundjoin%
\definecolor{currentfill}{rgb}{0.580392,0.403922,0.741176}%
\pgfsetfillcolor{currentfill}%
\pgfsetlinewidth{1.003750pt}%
\definecolor{currentstroke}{rgb}{0.580392,0.403922,0.741176}%
\pgfsetstrokecolor{currentstroke}%
\pgfsetdash{}{0pt}%
\pgfsys@defobject{currentmarker}{\pgfqpoint{-0.009821in}{-0.009821in}}{\pgfqpoint{0.009821in}{0.009821in}}{%
\pgfpathmoveto{\pgfqpoint{0.000000in}{-0.009821in}}%
\pgfpathcurveto{\pgfqpoint{0.002605in}{-0.009821in}}{\pgfqpoint{0.005103in}{-0.008786in}}{\pgfqpoint{0.006944in}{-0.006944in}}%
\pgfpathcurveto{\pgfqpoint{0.008786in}{-0.005103in}}{\pgfqpoint{0.009821in}{-0.002605in}}{\pgfqpoint{0.009821in}{0.000000in}}%
\pgfpathcurveto{\pgfqpoint{0.009821in}{0.002605in}}{\pgfqpoint{0.008786in}{0.005103in}}{\pgfqpoint{0.006944in}{0.006944in}}%
\pgfpathcurveto{\pgfqpoint{0.005103in}{0.008786in}}{\pgfqpoint{0.002605in}{0.009821in}}{\pgfqpoint{0.000000in}{0.009821in}}%
\pgfpathcurveto{\pgfqpoint{-0.002605in}{0.009821in}}{\pgfqpoint{-0.005103in}{0.008786in}}{\pgfqpoint{-0.006944in}{0.006944in}}%
\pgfpathcurveto{\pgfqpoint{-0.008786in}{0.005103in}}{\pgfqpoint{-0.009821in}{0.002605in}}{\pgfqpoint{-0.009821in}{0.000000in}}%
\pgfpathcurveto{\pgfqpoint{-0.009821in}{-0.002605in}}{\pgfqpoint{-0.008786in}{-0.005103in}}{\pgfqpoint{-0.006944in}{-0.006944in}}%
\pgfpathcurveto{\pgfqpoint{-0.005103in}{-0.008786in}}{\pgfqpoint{-0.002605in}{-0.009821in}}{\pgfqpoint{0.000000in}{-0.009821in}}%
\pgfpathclose%
\pgfusepath{stroke,fill}%
}%
\begin{pgfscope}%
\pgfsys@transformshift{0.708823in}{3.710452in}%
\pgfsys@useobject{currentmarker}{}%
\end{pgfscope}%
\begin{pgfscope}%
\pgfsys@transformshift{0.859609in}{3.891680in}%
\pgfsys@useobject{currentmarker}{}%
\end{pgfscope}%
\begin{pgfscope}%
\pgfsys@transformshift{1.010395in}{3.809931in}%
\pgfsys@useobject{currentmarker}{}%
\end{pgfscope}%
\begin{pgfscope}%
\pgfsys@transformshift{1.161181in}{3.822166in}%
\pgfsys@useobject{currentmarker}{}%
\end{pgfscope}%
\begin{pgfscope}%
\pgfsys@transformshift{1.311967in}{3.828266in}%
\pgfsys@useobject{currentmarker}{}%
\end{pgfscope}%
\begin{pgfscope}%
\pgfsys@transformshift{1.462753in}{3.781314in}%
\pgfsys@useobject{currentmarker}{}%
\end{pgfscope}%
\begin{pgfscope}%
\pgfsys@transformshift{1.613539in}{3.804810in}%
\pgfsys@useobject{currentmarker}{}%
\end{pgfscope}%
\begin{pgfscope}%
\pgfsys@transformshift{1.764325in}{3.803899in}%
\pgfsys@useobject{currentmarker}{}%
\end{pgfscope}%
\begin{pgfscope}%
\pgfsys@transformshift{1.915111in}{3.808713in}%
\pgfsys@useobject{currentmarker}{}%
\end{pgfscope}%
\begin{pgfscope}%
\pgfsys@transformshift{2.065897in}{3.810884in}%
\pgfsys@useobject{currentmarker}{}%
\end{pgfscope}%
\begin{pgfscope}%
\pgfsys@transformshift{2.216683in}{3.798051in}%
\pgfsys@useobject{currentmarker}{}%
\end{pgfscope}%
\begin{pgfscope}%
\pgfsys@transformshift{2.367469in}{3.794654in}%
\pgfsys@useobject{currentmarker}{}%
\end{pgfscope}%
\begin{pgfscope}%
\pgfsys@transformshift{2.518255in}{3.806475in}%
\pgfsys@useobject{currentmarker}{}%
\end{pgfscope}%
\begin{pgfscope}%
\pgfsys@transformshift{2.669041in}{3.797010in}%
\pgfsys@useobject{currentmarker}{}%
\end{pgfscope}%
\begin{pgfscope}%
\pgfsys@transformshift{2.819827in}{3.802214in}%
\pgfsys@useobject{currentmarker}{}%
\end{pgfscope}%
\begin{pgfscope}%
\pgfsys@transformshift{2.970613in}{3.805823in}%
\pgfsys@useobject{currentmarker}{}%
\end{pgfscope}%
\begin{pgfscope}%
\pgfsys@transformshift{3.121399in}{3.799601in}%
\pgfsys@useobject{currentmarker}{}%
\end{pgfscope}%
\begin{pgfscope}%
\pgfsys@transformshift{3.272185in}{3.800664in}%
\pgfsys@useobject{currentmarker}{}%
\end{pgfscope}%
\begin{pgfscope}%
\pgfsys@transformshift{3.422971in}{3.800410in}%
\pgfsys@useobject{currentmarker}{}%
\end{pgfscope}%
\begin{pgfscope}%
\pgfsys@transformshift{3.573756in}{3.801209in}%
\pgfsys@useobject{currentmarker}{}%
\end{pgfscope}%
\begin{pgfscope}%
\pgfsys@transformshift{3.724542in}{3.801497in}%
\pgfsys@useobject{currentmarker}{}%
\end{pgfscope}%
\end{pgfscope}%
\begin{pgfscope}%
\pgfpathrectangle{\pgfqpoint{0.557986in}{3.363889in}}{\pgfqpoint{3.318403in}{2.263889in}}%
\pgfusepath{clip}%
\pgfsetbuttcap%
\pgfsetroundjoin%
\definecolor{currentfill}{rgb}{0.580392,0.403922,0.741176}%
\pgfsetfillcolor{currentfill}%
\pgfsetfillopacity{0.300000}%
\pgfsetlinewidth{1.003750pt}%
\definecolor{currentstroke}{rgb}{0.580392,0.403922,0.741176}%
\pgfsetstrokecolor{currentstroke}%
\pgfsetstrokeopacity{0.300000}%
\pgfsetdash{}{0pt}%
\pgfpathmoveto{\pgfqpoint{0.708823in}{3.900924in}}%
\pgfpathlineto{\pgfqpoint{0.708823in}{3.519981in}}%
\pgfpathlineto{\pgfqpoint{0.859609in}{3.591683in}}%
\pgfpathlineto{\pgfqpoint{1.010395in}{3.667293in}}%
\pgfpathlineto{\pgfqpoint{1.161181in}{3.704313in}}%
\pgfpathlineto{\pgfqpoint{1.311967in}{3.780013in}}%
\pgfpathlineto{\pgfqpoint{1.462753in}{3.680921in}}%
\pgfpathlineto{\pgfqpoint{1.613539in}{3.771122in}}%
\pgfpathlineto{\pgfqpoint{1.764325in}{3.748663in}}%
\pgfpathlineto{\pgfqpoint{1.915111in}{3.776732in}}%
\pgfpathlineto{\pgfqpoint{2.065897in}{3.781462in}}%
\pgfpathlineto{\pgfqpoint{2.216683in}{3.778335in}}%
\pgfpathlineto{\pgfqpoint{2.367469in}{3.780351in}}%
\pgfpathlineto{\pgfqpoint{2.518255in}{3.800792in}}%
\pgfpathlineto{\pgfqpoint{2.669041in}{3.790704in}}%
\pgfpathlineto{\pgfqpoint{2.819827in}{3.799408in}}%
\pgfpathlineto{\pgfqpoint{2.970613in}{3.802528in}}%
\pgfpathlineto{\pgfqpoint{3.121399in}{3.796527in}}%
\pgfpathlineto{\pgfqpoint{3.272185in}{3.798255in}}%
\pgfpathlineto{\pgfqpoint{3.422971in}{3.798351in}}%
\pgfpathlineto{\pgfqpoint{3.573756in}{3.800696in}}%
\pgfpathlineto{\pgfqpoint{3.724542in}{3.800734in}}%
\pgfpathlineto{\pgfqpoint{3.724542in}{3.802259in}}%
\pgfpathlineto{\pgfqpoint{3.724542in}{3.802259in}}%
\pgfpathlineto{\pgfqpoint{3.573756in}{3.801722in}}%
\pgfpathlineto{\pgfqpoint{3.422971in}{3.802468in}}%
\pgfpathlineto{\pgfqpoint{3.272185in}{3.803073in}}%
\pgfpathlineto{\pgfqpoint{3.121399in}{3.802675in}}%
\pgfpathlineto{\pgfqpoint{2.970613in}{3.809117in}}%
\pgfpathlineto{\pgfqpoint{2.819827in}{3.805020in}}%
\pgfpathlineto{\pgfqpoint{2.669041in}{3.803316in}}%
\pgfpathlineto{\pgfqpoint{2.518255in}{3.812158in}}%
\pgfpathlineto{\pgfqpoint{2.367469in}{3.808957in}}%
\pgfpathlineto{\pgfqpoint{2.216683in}{3.817766in}}%
\pgfpathlineto{\pgfqpoint{2.065897in}{3.840305in}}%
\pgfpathlineto{\pgfqpoint{1.915111in}{3.840694in}}%
\pgfpathlineto{\pgfqpoint{1.764325in}{3.859135in}}%
\pgfpathlineto{\pgfqpoint{1.613539in}{3.838499in}}%
\pgfpathlineto{\pgfqpoint{1.462753in}{3.881706in}}%
\pgfpathlineto{\pgfqpoint{1.311967in}{3.876519in}}%
\pgfpathlineto{\pgfqpoint{1.161181in}{3.940019in}}%
\pgfpathlineto{\pgfqpoint{1.010395in}{3.952569in}}%
\pgfpathlineto{\pgfqpoint{0.859609in}{4.191677in}}%
\pgfpathlineto{\pgfqpoint{0.708823in}{3.900924in}}%
\pgfpathclose%
\pgfusepath{stroke,fill}%
\end{pgfscope}%
\begin{pgfscope}%
\pgfsetbuttcap%
\pgfsetroundjoin%
\definecolor{currentfill}{rgb}{0.000000,0.000000,0.000000}%
\pgfsetfillcolor{currentfill}%
\pgfsetlinewidth{0.803000pt}%
\definecolor{currentstroke}{rgb}{0.000000,0.000000,0.000000}%
\pgfsetstrokecolor{currentstroke}%
\pgfsetdash{}{0pt}%
\pgfsys@defobject{currentmarker}{\pgfqpoint{0.000000in}{-0.048611in}}{\pgfqpoint{0.000000in}{0.000000in}}{%
\pgfpathmoveto{\pgfqpoint{0.000000in}{0.000000in}}%
\pgfpathlineto{\pgfqpoint{0.000000in}{-0.048611in}}%
\pgfusepath{stroke,fill}%
}%
\begin{pgfscope}%
\pgfsys@transformshift{1.071951in}{3.363889in}%
\pgfsys@useobject{currentmarker}{}%
\end{pgfscope}%
\end{pgfscope}%
\begin{pgfscope}%
\definecolor{textcolor}{rgb}{0.000000,0.000000,0.000000}%
\pgfsetstrokecolor{textcolor}%
\pgfsetfillcolor{textcolor}%
\pgftext[x=1.071951in,y=3.266667in,,top]{\color{textcolor}\sffamily\fontsize{10.000000}{12.000000}\selectfont \(\displaystyle {10^{4}}\)}%
\end{pgfscope}%
\begin{pgfscope}%
\pgfsetbuttcap%
\pgfsetroundjoin%
\definecolor{currentfill}{rgb}{0.000000,0.000000,0.000000}%
\pgfsetfillcolor{currentfill}%
\pgfsetlinewidth{0.803000pt}%
\definecolor{currentstroke}{rgb}{0.000000,0.000000,0.000000}%
\pgfsetstrokecolor{currentstroke}%
\pgfsetdash{}{0pt}%
\pgfsys@defobject{currentmarker}{\pgfqpoint{0.000000in}{-0.048611in}}{\pgfqpoint{0.000000in}{0.000000in}}{%
\pgfpathmoveto{\pgfqpoint{0.000000in}{0.000000in}}%
\pgfpathlineto{\pgfqpoint{0.000000in}{-0.048611in}}%
\pgfusepath{stroke,fill}%
}%
\begin{pgfscope}%
\pgfsys@transformshift{1.675095in}{3.363889in}%
\pgfsys@useobject{currentmarker}{}%
\end{pgfscope}%
\end{pgfscope}%
\begin{pgfscope}%
\definecolor{textcolor}{rgb}{0.000000,0.000000,0.000000}%
\pgfsetstrokecolor{textcolor}%
\pgfsetfillcolor{textcolor}%
\pgftext[x=1.675095in,y=3.266667in,,top]{\color{textcolor}\sffamily\fontsize{10.000000}{12.000000}\selectfont \(\displaystyle {10^{5}}\)}%
\end{pgfscope}%
\begin{pgfscope}%
\pgfsetbuttcap%
\pgfsetroundjoin%
\definecolor{currentfill}{rgb}{0.000000,0.000000,0.000000}%
\pgfsetfillcolor{currentfill}%
\pgfsetlinewidth{0.803000pt}%
\definecolor{currentstroke}{rgb}{0.000000,0.000000,0.000000}%
\pgfsetstrokecolor{currentstroke}%
\pgfsetdash{}{0pt}%
\pgfsys@defobject{currentmarker}{\pgfqpoint{0.000000in}{-0.048611in}}{\pgfqpoint{0.000000in}{0.000000in}}{%
\pgfpathmoveto{\pgfqpoint{0.000000in}{0.000000in}}%
\pgfpathlineto{\pgfqpoint{0.000000in}{-0.048611in}}%
\pgfusepath{stroke,fill}%
}%
\begin{pgfscope}%
\pgfsys@transformshift{2.278239in}{3.363889in}%
\pgfsys@useobject{currentmarker}{}%
\end{pgfscope}%
\end{pgfscope}%
\begin{pgfscope}%
\definecolor{textcolor}{rgb}{0.000000,0.000000,0.000000}%
\pgfsetstrokecolor{textcolor}%
\pgfsetfillcolor{textcolor}%
\pgftext[x=2.278239in,y=3.266667in,,top]{\color{textcolor}\sffamily\fontsize{10.000000}{12.000000}\selectfont \(\displaystyle {10^{6}}\)}%
\end{pgfscope}%
\begin{pgfscope}%
\pgfsetbuttcap%
\pgfsetroundjoin%
\definecolor{currentfill}{rgb}{0.000000,0.000000,0.000000}%
\pgfsetfillcolor{currentfill}%
\pgfsetlinewidth{0.803000pt}%
\definecolor{currentstroke}{rgb}{0.000000,0.000000,0.000000}%
\pgfsetstrokecolor{currentstroke}%
\pgfsetdash{}{0pt}%
\pgfsys@defobject{currentmarker}{\pgfqpoint{0.000000in}{-0.048611in}}{\pgfqpoint{0.000000in}{0.000000in}}{%
\pgfpathmoveto{\pgfqpoint{0.000000in}{0.000000in}}%
\pgfpathlineto{\pgfqpoint{0.000000in}{-0.048611in}}%
\pgfusepath{stroke,fill}%
}%
\begin{pgfscope}%
\pgfsys@transformshift{2.881383in}{3.363889in}%
\pgfsys@useobject{currentmarker}{}%
\end{pgfscope}%
\end{pgfscope}%
\begin{pgfscope}%
\definecolor{textcolor}{rgb}{0.000000,0.000000,0.000000}%
\pgfsetstrokecolor{textcolor}%
\pgfsetfillcolor{textcolor}%
\pgftext[x=2.881383in,y=3.266667in,,top]{\color{textcolor}\sffamily\fontsize{10.000000}{12.000000}\selectfont \(\displaystyle {10^{7}}\)}%
\end{pgfscope}%
\begin{pgfscope}%
\pgfsetbuttcap%
\pgfsetroundjoin%
\definecolor{currentfill}{rgb}{0.000000,0.000000,0.000000}%
\pgfsetfillcolor{currentfill}%
\pgfsetlinewidth{0.803000pt}%
\definecolor{currentstroke}{rgb}{0.000000,0.000000,0.000000}%
\pgfsetstrokecolor{currentstroke}%
\pgfsetdash{}{0pt}%
\pgfsys@defobject{currentmarker}{\pgfqpoint{0.000000in}{-0.048611in}}{\pgfqpoint{0.000000in}{0.000000in}}{%
\pgfpathmoveto{\pgfqpoint{0.000000in}{0.000000in}}%
\pgfpathlineto{\pgfqpoint{0.000000in}{-0.048611in}}%
\pgfusepath{stroke,fill}%
}%
\begin{pgfscope}%
\pgfsys@transformshift{3.484527in}{3.363889in}%
\pgfsys@useobject{currentmarker}{}%
\end{pgfscope}%
\end{pgfscope}%
\begin{pgfscope}%
\definecolor{textcolor}{rgb}{0.000000,0.000000,0.000000}%
\pgfsetstrokecolor{textcolor}%
\pgfsetfillcolor{textcolor}%
\pgftext[x=3.484527in,y=3.266667in,,top]{\color{textcolor}\sffamily\fontsize{10.000000}{12.000000}\selectfont \(\displaystyle {10^{8}}\)}%
\end{pgfscope}%
\begin{pgfscope}%
\pgfsetbuttcap%
\pgfsetroundjoin%
\definecolor{currentfill}{rgb}{0.000000,0.000000,0.000000}%
\pgfsetfillcolor{currentfill}%
\pgfsetlinewidth{0.602250pt}%
\definecolor{currentstroke}{rgb}{0.000000,0.000000,0.000000}%
\pgfsetstrokecolor{currentstroke}%
\pgfsetdash{}{0pt}%
\pgfsys@defobject{currentmarker}{\pgfqpoint{0.000000in}{-0.027778in}}{\pgfqpoint{0.000000in}{0.000000in}}{%
\pgfpathmoveto{\pgfqpoint{0.000000in}{0.000000in}}%
\pgfpathlineto{\pgfqpoint{0.000000in}{-0.027778in}}%
\pgfusepath{stroke,fill}%
}%
\begin{pgfscope}%
\pgfsys@transformshift{0.650372in}{3.363889in}%
\pgfsys@useobject{currentmarker}{}%
\end{pgfscope}%
\end{pgfscope}%
\begin{pgfscope}%
\pgfsetbuttcap%
\pgfsetroundjoin%
\definecolor{currentfill}{rgb}{0.000000,0.000000,0.000000}%
\pgfsetfillcolor{currentfill}%
\pgfsetlinewidth{0.602250pt}%
\definecolor{currentstroke}{rgb}{0.000000,0.000000,0.000000}%
\pgfsetstrokecolor{currentstroke}%
\pgfsetdash{}{0pt}%
\pgfsys@defobject{currentmarker}{\pgfqpoint{0.000000in}{-0.027778in}}{\pgfqpoint{0.000000in}{0.000000in}}{%
\pgfpathmoveto{\pgfqpoint{0.000000in}{0.000000in}}%
\pgfpathlineto{\pgfqpoint{0.000000in}{-0.027778in}}%
\pgfusepath{stroke,fill}%
}%
\begin{pgfscope}%
\pgfsys@transformshift{0.756580in}{3.363889in}%
\pgfsys@useobject{currentmarker}{}%
\end{pgfscope}%
\end{pgfscope}%
\begin{pgfscope}%
\pgfsetbuttcap%
\pgfsetroundjoin%
\definecolor{currentfill}{rgb}{0.000000,0.000000,0.000000}%
\pgfsetfillcolor{currentfill}%
\pgfsetlinewidth{0.602250pt}%
\definecolor{currentstroke}{rgb}{0.000000,0.000000,0.000000}%
\pgfsetstrokecolor{currentstroke}%
\pgfsetdash{}{0pt}%
\pgfsys@defobject{currentmarker}{\pgfqpoint{0.000000in}{-0.027778in}}{\pgfqpoint{0.000000in}{0.000000in}}{%
\pgfpathmoveto{\pgfqpoint{0.000000in}{0.000000in}}%
\pgfpathlineto{\pgfqpoint{0.000000in}{-0.027778in}}%
\pgfusepath{stroke,fill}%
}%
\begin{pgfscope}%
\pgfsys@transformshift{0.831936in}{3.363889in}%
\pgfsys@useobject{currentmarker}{}%
\end{pgfscope}%
\end{pgfscope}%
\begin{pgfscope}%
\pgfsetbuttcap%
\pgfsetroundjoin%
\definecolor{currentfill}{rgb}{0.000000,0.000000,0.000000}%
\pgfsetfillcolor{currentfill}%
\pgfsetlinewidth{0.602250pt}%
\definecolor{currentstroke}{rgb}{0.000000,0.000000,0.000000}%
\pgfsetstrokecolor{currentstroke}%
\pgfsetdash{}{0pt}%
\pgfsys@defobject{currentmarker}{\pgfqpoint{0.000000in}{-0.027778in}}{\pgfqpoint{0.000000in}{0.000000in}}{%
\pgfpathmoveto{\pgfqpoint{0.000000in}{0.000000in}}%
\pgfpathlineto{\pgfqpoint{0.000000in}{-0.027778in}}%
\pgfusepath{stroke,fill}%
}%
\begin{pgfscope}%
\pgfsys@transformshift{0.890387in}{3.363889in}%
\pgfsys@useobject{currentmarker}{}%
\end{pgfscope}%
\end{pgfscope}%
\begin{pgfscope}%
\pgfsetbuttcap%
\pgfsetroundjoin%
\definecolor{currentfill}{rgb}{0.000000,0.000000,0.000000}%
\pgfsetfillcolor{currentfill}%
\pgfsetlinewidth{0.602250pt}%
\definecolor{currentstroke}{rgb}{0.000000,0.000000,0.000000}%
\pgfsetstrokecolor{currentstroke}%
\pgfsetdash{}{0pt}%
\pgfsys@defobject{currentmarker}{\pgfqpoint{0.000000in}{-0.027778in}}{\pgfqpoint{0.000000in}{0.000000in}}{%
\pgfpathmoveto{\pgfqpoint{0.000000in}{0.000000in}}%
\pgfpathlineto{\pgfqpoint{0.000000in}{-0.027778in}}%
\pgfusepath{stroke,fill}%
}%
\begin{pgfscope}%
\pgfsys@transformshift{0.938145in}{3.363889in}%
\pgfsys@useobject{currentmarker}{}%
\end{pgfscope}%
\end{pgfscope}%
\begin{pgfscope}%
\pgfsetbuttcap%
\pgfsetroundjoin%
\definecolor{currentfill}{rgb}{0.000000,0.000000,0.000000}%
\pgfsetfillcolor{currentfill}%
\pgfsetlinewidth{0.602250pt}%
\definecolor{currentstroke}{rgb}{0.000000,0.000000,0.000000}%
\pgfsetstrokecolor{currentstroke}%
\pgfsetdash{}{0pt}%
\pgfsys@defobject{currentmarker}{\pgfqpoint{0.000000in}{-0.027778in}}{\pgfqpoint{0.000000in}{0.000000in}}{%
\pgfpathmoveto{\pgfqpoint{0.000000in}{0.000000in}}%
\pgfpathlineto{\pgfqpoint{0.000000in}{-0.027778in}}%
\pgfusepath{stroke,fill}%
}%
\begin{pgfscope}%
\pgfsys@transformshift{0.978523in}{3.363889in}%
\pgfsys@useobject{currentmarker}{}%
\end{pgfscope}%
\end{pgfscope}%
\begin{pgfscope}%
\pgfsetbuttcap%
\pgfsetroundjoin%
\definecolor{currentfill}{rgb}{0.000000,0.000000,0.000000}%
\pgfsetfillcolor{currentfill}%
\pgfsetlinewidth{0.602250pt}%
\definecolor{currentstroke}{rgb}{0.000000,0.000000,0.000000}%
\pgfsetstrokecolor{currentstroke}%
\pgfsetdash{}{0pt}%
\pgfsys@defobject{currentmarker}{\pgfqpoint{0.000000in}{-0.027778in}}{\pgfqpoint{0.000000in}{0.000000in}}{%
\pgfpathmoveto{\pgfqpoint{0.000000in}{0.000000in}}%
\pgfpathlineto{\pgfqpoint{0.000000in}{-0.027778in}}%
\pgfusepath{stroke,fill}%
}%
\begin{pgfscope}%
\pgfsys@transformshift{1.013501in}{3.363889in}%
\pgfsys@useobject{currentmarker}{}%
\end{pgfscope}%
\end{pgfscope}%
\begin{pgfscope}%
\pgfsetbuttcap%
\pgfsetroundjoin%
\definecolor{currentfill}{rgb}{0.000000,0.000000,0.000000}%
\pgfsetfillcolor{currentfill}%
\pgfsetlinewidth{0.602250pt}%
\definecolor{currentstroke}{rgb}{0.000000,0.000000,0.000000}%
\pgfsetstrokecolor{currentstroke}%
\pgfsetdash{}{0pt}%
\pgfsys@defobject{currentmarker}{\pgfqpoint{0.000000in}{-0.027778in}}{\pgfqpoint{0.000000in}{0.000000in}}{%
\pgfpathmoveto{\pgfqpoint{0.000000in}{0.000000in}}%
\pgfpathlineto{\pgfqpoint{0.000000in}{-0.027778in}}%
\pgfusepath{stroke,fill}%
}%
\begin{pgfscope}%
\pgfsys@transformshift{1.044353in}{3.363889in}%
\pgfsys@useobject{currentmarker}{}%
\end{pgfscope}%
\end{pgfscope}%
\begin{pgfscope}%
\pgfsetbuttcap%
\pgfsetroundjoin%
\definecolor{currentfill}{rgb}{0.000000,0.000000,0.000000}%
\pgfsetfillcolor{currentfill}%
\pgfsetlinewidth{0.602250pt}%
\definecolor{currentstroke}{rgb}{0.000000,0.000000,0.000000}%
\pgfsetstrokecolor{currentstroke}%
\pgfsetdash{}{0pt}%
\pgfsys@defobject{currentmarker}{\pgfqpoint{0.000000in}{-0.027778in}}{\pgfqpoint{0.000000in}{0.000000in}}{%
\pgfpathmoveto{\pgfqpoint{0.000000in}{0.000000in}}%
\pgfpathlineto{\pgfqpoint{0.000000in}{-0.027778in}}%
\pgfusepath{stroke,fill}%
}%
\begin{pgfscope}%
\pgfsys@transformshift{1.253516in}{3.363889in}%
\pgfsys@useobject{currentmarker}{}%
\end{pgfscope}%
\end{pgfscope}%
\begin{pgfscope}%
\pgfsetbuttcap%
\pgfsetroundjoin%
\definecolor{currentfill}{rgb}{0.000000,0.000000,0.000000}%
\pgfsetfillcolor{currentfill}%
\pgfsetlinewidth{0.602250pt}%
\definecolor{currentstroke}{rgb}{0.000000,0.000000,0.000000}%
\pgfsetstrokecolor{currentstroke}%
\pgfsetdash{}{0pt}%
\pgfsys@defobject{currentmarker}{\pgfqpoint{0.000000in}{-0.027778in}}{\pgfqpoint{0.000000in}{0.000000in}}{%
\pgfpathmoveto{\pgfqpoint{0.000000in}{0.000000in}}%
\pgfpathlineto{\pgfqpoint{0.000000in}{-0.027778in}}%
\pgfusepath{stroke,fill}%
}%
\begin{pgfscope}%
\pgfsys@transformshift{1.359724in}{3.363889in}%
\pgfsys@useobject{currentmarker}{}%
\end{pgfscope}%
\end{pgfscope}%
\begin{pgfscope}%
\pgfsetbuttcap%
\pgfsetroundjoin%
\definecolor{currentfill}{rgb}{0.000000,0.000000,0.000000}%
\pgfsetfillcolor{currentfill}%
\pgfsetlinewidth{0.602250pt}%
\definecolor{currentstroke}{rgb}{0.000000,0.000000,0.000000}%
\pgfsetstrokecolor{currentstroke}%
\pgfsetdash{}{0pt}%
\pgfsys@defobject{currentmarker}{\pgfqpoint{0.000000in}{-0.027778in}}{\pgfqpoint{0.000000in}{0.000000in}}{%
\pgfpathmoveto{\pgfqpoint{0.000000in}{0.000000in}}%
\pgfpathlineto{\pgfqpoint{0.000000in}{-0.027778in}}%
\pgfusepath{stroke,fill}%
}%
\begin{pgfscope}%
\pgfsys@transformshift{1.435080in}{3.363889in}%
\pgfsys@useobject{currentmarker}{}%
\end{pgfscope}%
\end{pgfscope}%
\begin{pgfscope}%
\pgfsetbuttcap%
\pgfsetroundjoin%
\definecolor{currentfill}{rgb}{0.000000,0.000000,0.000000}%
\pgfsetfillcolor{currentfill}%
\pgfsetlinewidth{0.602250pt}%
\definecolor{currentstroke}{rgb}{0.000000,0.000000,0.000000}%
\pgfsetstrokecolor{currentstroke}%
\pgfsetdash{}{0pt}%
\pgfsys@defobject{currentmarker}{\pgfqpoint{0.000000in}{-0.027778in}}{\pgfqpoint{0.000000in}{0.000000in}}{%
\pgfpathmoveto{\pgfqpoint{0.000000in}{0.000000in}}%
\pgfpathlineto{\pgfqpoint{0.000000in}{-0.027778in}}%
\pgfusepath{stroke,fill}%
}%
\begin{pgfscope}%
\pgfsys@transformshift{1.493531in}{3.363889in}%
\pgfsys@useobject{currentmarker}{}%
\end{pgfscope}%
\end{pgfscope}%
\begin{pgfscope}%
\pgfsetbuttcap%
\pgfsetroundjoin%
\definecolor{currentfill}{rgb}{0.000000,0.000000,0.000000}%
\pgfsetfillcolor{currentfill}%
\pgfsetlinewidth{0.602250pt}%
\definecolor{currentstroke}{rgb}{0.000000,0.000000,0.000000}%
\pgfsetstrokecolor{currentstroke}%
\pgfsetdash{}{0pt}%
\pgfsys@defobject{currentmarker}{\pgfqpoint{0.000000in}{-0.027778in}}{\pgfqpoint{0.000000in}{0.000000in}}{%
\pgfpathmoveto{\pgfqpoint{0.000000in}{0.000000in}}%
\pgfpathlineto{\pgfqpoint{0.000000in}{-0.027778in}}%
\pgfusepath{stroke,fill}%
}%
\begin{pgfscope}%
\pgfsys@transformshift{1.541289in}{3.363889in}%
\pgfsys@useobject{currentmarker}{}%
\end{pgfscope}%
\end{pgfscope}%
\begin{pgfscope}%
\pgfsetbuttcap%
\pgfsetroundjoin%
\definecolor{currentfill}{rgb}{0.000000,0.000000,0.000000}%
\pgfsetfillcolor{currentfill}%
\pgfsetlinewidth{0.602250pt}%
\definecolor{currentstroke}{rgb}{0.000000,0.000000,0.000000}%
\pgfsetstrokecolor{currentstroke}%
\pgfsetdash{}{0pt}%
\pgfsys@defobject{currentmarker}{\pgfqpoint{0.000000in}{-0.027778in}}{\pgfqpoint{0.000000in}{0.000000in}}{%
\pgfpathmoveto{\pgfqpoint{0.000000in}{0.000000in}}%
\pgfpathlineto{\pgfqpoint{0.000000in}{-0.027778in}}%
\pgfusepath{stroke,fill}%
}%
\begin{pgfscope}%
\pgfsys@transformshift{1.581667in}{3.363889in}%
\pgfsys@useobject{currentmarker}{}%
\end{pgfscope}%
\end{pgfscope}%
\begin{pgfscope}%
\pgfsetbuttcap%
\pgfsetroundjoin%
\definecolor{currentfill}{rgb}{0.000000,0.000000,0.000000}%
\pgfsetfillcolor{currentfill}%
\pgfsetlinewidth{0.602250pt}%
\definecolor{currentstroke}{rgb}{0.000000,0.000000,0.000000}%
\pgfsetstrokecolor{currentstroke}%
\pgfsetdash{}{0pt}%
\pgfsys@defobject{currentmarker}{\pgfqpoint{0.000000in}{-0.027778in}}{\pgfqpoint{0.000000in}{0.000000in}}{%
\pgfpathmoveto{\pgfqpoint{0.000000in}{0.000000in}}%
\pgfpathlineto{\pgfqpoint{0.000000in}{-0.027778in}}%
\pgfusepath{stroke,fill}%
}%
\begin{pgfscope}%
\pgfsys@transformshift{1.616645in}{3.363889in}%
\pgfsys@useobject{currentmarker}{}%
\end{pgfscope}%
\end{pgfscope}%
\begin{pgfscope}%
\pgfsetbuttcap%
\pgfsetroundjoin%
\definecolor{currentfill}{rgb}{0.000000,0.000000,0.000000}%
\pgfsetfillcolor{currentfill}%
\pgfsetlinewidth{0.602250pt}%
\definecolor{currentstroke}{rgb}{0.000000,0.000000,0.000000}%
\pgfsetstrokecolor{currentstroke}%
\pgfsetdash{}{0pt}%
\pgfsys@defobject{currentmarker}{\pgfqpoint{0.000000in}{-0.027778in}}{\pgfqpoint{0.000000in}{0.000000in}}{%
\pgfpathmoveto{\pgfqpoint{0.000000in}{0.000000in}}%
\pgfpathlineto{\pgfqpoint{0.000000in}{-0.027778in}}%
\pgfusepath{stroke,fill}%
}%
\begin{pgfscope}%
\pgfsys@transformshift{1.647497in}{3.363889in}%
\pgfsys@useobject{currentmarker}{}%
\end{pgfscope}%
\end{pgfscope}%
\begin{pgfscope}%
\pgfsetbuttcap%
\pgfsetroundjoin%
\definecolor{currentfill}{rgb}{0.000000,0.000000,0.000000}%
\pgfsetfillcolor{currentfill}%
\pgfsetlinewidth{0.602250pt}%
\definecolor{currentstroke}{rgb}{0.000000,0.000000,0.000000}%
\pgfsetstrokecolor{currentstroke}%
\pgfsetdash{}{0pt}%
\pgfsys@defobject{currentmarker}{\pgfqpoint{0.000000in}{-0.027778in}}{\pgfqpoint{0.000000in}{0.000000in}}{%
\pgfpathmoveto{\pgfqpoint{0.000000in}{0.000000in}}%
\pgfpathlineto{\pgfqpoint{0.000000in}{-0.027778in}}%
\pgfusepath{stroke,fill}%
}%
\begin{pgfscope}%
\pgfsys@transformshift{1.856660in}{3.363889in}%
\pgfsys@useobject{currentmarker}{}%
\end{pgfscope}%
\end{pgfscope}%
\begin{pgfscope}%
\pgfsetbuttcap%
\pgfsetroundjoin%
\definecolor{currentfill}{rgb}{0.000000,0.000000,0.000000}%
\pgfsetfillcolor{currentfill}%
\pgfsetlinewidth{0.602250pt}%
\definecolor{currentstroke}{rgb}{0.000000,0.000000,0.000000}%
\pgfsetstrokecolor{currentstroke}%
\pgfsetdash{}{0pt}%
\pgfsys@defobject{currentmarker}{\pgfqpoint{0.000000in}{-0.027778in}}{\pgfqpoint{0.000000in}{0.000000in}}{%
\pgfpathmoveto{\pgfqpoint{0.000000in}{0.000000in}}%
\pgfpathlineto{\pgfqpoint{0.000000in}{-0.027778in}}%
\pgfusepath{stroke,fill}%
}%
\begin{pgfscope}%
\pgfsys@transformshift{1.962868in}{3.363889in}%
\pgfsys@useobject{currentmarker}{}%
\end{pgfscope}%
\end{pgfscope}%
\begin{pgfscope}%
\pgfsetbuttcap%
\pgfsetroundjoin%
\definecolor{currentfill}{rgb}{0.000000,0.000000,0.000000}%
\pgfsetfillcolor{currentfill}%
\pgfsetlinewidth{0.602250pt}%
\definecolor{currentstroke}{rgb}{0.000000,0.000000,0.000000}%
\pgfsetstrokecolor{currentstroke}%
\pgfsetdash{}{0pt}%
\pgfsys@defobject{currentmarker}{\pgfqpoint{0.000000in}{-0.027778in}}{\pgfqpoint{0.000000in}{0.000000in}}{%
\pgfpathmoveto{\pgfqpoint{0.000000in}{0.000000in}}%
\pgfpathlineto{\pgfqpoint{0.000000in}{-0.027778in}}%
\pgfusepath{stroke,fill}%
}%
\begin{pgfscope}%
\pgfsys@transformshift{2.038224in}{3.363889in}%
\pgfsys@useobject{currentmarker}{}%
\end{pgfscope}%
\end{pgfscope}%
\begin{pgfscope}%
\pgfsetbuttcap%
\pgfsetroundjoin%
\definecolor{currentfill}{rgb}{0.000000,0.000000,0.000000}%
\pgfsetfillcolor{currentfill}%
\pgfsetlinewidth{0.602250pt}%
\definecolor{currentstroke}{rgb}{0.000000,0.000000,0.000000}%
\pgfsetstrokecolor{currentstroke}%
\pgfsetdash{}{0pt}%
\pgfsys@defobject{currentmarker}{\pgfqpoint{0.000000in}{-0.027778in}}{\pgfqpoint{0.000000in}{0.000000in}}{%
\pgfpathmoveto{\pgfqpoint{0.000000in}{0.000000in}}%
\pgfpathlineto{\pgfqpoint{0.000000in}{-0.027778in}}%
\pgfusepath{stroke,fill}%
}%
\begin{pgfscope}%
\pgfsys@transformshift{2.096675in}{3.363889in}%
\pgfsys@useobject{currentmarker}{}%
\end{pgfscope}%
\end{pgfscope}%
\begin{pgfscope}%
\pgfsetbuttcap%
\pgfsetroundjoin%
\definecolor{currentfill}{rgb}{0.000000,0.000000,0.000000}%
\pgfsetfillcolor{currentfill}%
\pgfsetlinewidth{0.602250pt}%
\definecolor{currentstroke}{rgb}{0.000000,0.000000,0.000000}%
\pgfsetstrokecolor{currentstroke}%
\pgfsetdash{}{0pt}%
\pgfsys@defobject{currentmarker}{\pgfqpoint{0.000000in}{-0.027778in}}{\pgfqpoint{0.000000in}{0.000000in}}{%
\pgfpathmoveto{\pgfqpoint{0.000000in}{0.000000in}}%
\pgfpathlineto{\pgfqpoint{0.000000in}{-0.027778in}}%
\pgfusepath{stroke,fill}%
}%
\begin{pgfscope}%
\pgfsys@transformshift{2.144433in}{3.363889in}%
\pgfsys@useobject{currentmarker}{}%
\end{pgfscope}%
\end{pgfscope}%
\begin{pgfscope}%
\pgfsetbuttcap%
\pgfsetroundjoin%
\definecolor{currentfill}{rgb}{0.000000,0.000000,0.000000}%
\pgfsetfillcolor{currentfill}%
\pgfsetlinewidth{0.602250pt}%
\definecolor{currentstroke}{rgb}{0.000000,0.000000,0.000000}%
\pgfsetstrokecolor{currentstroke}%
\pgfsetdash{}{0pt}%
\pgfsys@defobject{currentmarker}{\pgfqpoint{0.000000in}{-0.027778in}}{\pgfqpoint{0.000000in}{0.000000in}}{%
\pgfpathmoveto{\pgfqpoint{0.000000in}{0.000000in}}%
\pgfpathlineto{\pgfqpoint{0.000000in}{-0.027778in}}%
\pgfusepath{stroke,fill}%
}%
\begin{pgfscope}%
\pgfsys@transformshift{2.184811in}{3.363889in}%
\pgfsys@useobject{currentmarker}{}%
\end{pgfscope}%
\end{pgfscope}%
\begin{pgfscope}%
\pgfsetbuttcap%
\pgfsetroundjoin%
\definecolor{currentfill}{rgb}{0.000000,0.000000,0.000000}%
\pgfsetfillcolor{currentfill}%
\pgfsetlinewidth{0.602250pt}%
\definecolor{currentstroke}{rgb}{0.000000,0.000000,0.000000}%
\pgfsetstrokecolor{currentstroke}%
\pgfsetdash{}{0pt}%
\pgfsys@defobject{currentmarker}{\pgfqpoint{0.000000in}{-0.027778in}}{\pgfqpoint{0.000000in}{0.000000in}}{%
\pgfpathmoveto{\pgfqpoint{0.000000in}{0.000000in}}%
\pgfpathlineto{\pgfqpoint{0.000000in}{-0.027778in}}%
\pgfusepath{stroke,fill}%
}%
\begin{pgfscope}%
\pgfsys@transformshift{2.219789in}{3.363889in}%
\pgfsys@useobject{currentmarker}{}%
\end{pgfscope}%
\end{pgfscope}%
\begin{pgfscope}%
\pgfsetbuttcap%
\pgfsetroundjoin%
\definecolor{currentfill}{rgb}{0.000000,0.000000,0.000000}%
\pgfsetfillcolor{currentfill}%
\pgfsetlinewidth{0.602250pt}%
\definecolor{currentstroke}{rgb}{0.000000,0.000000,0.000000}%
\pgfsetstrokecolor{currentstroke}%
\pgfsetdash{}{0pt}%
\pgfsys@defobject{currentmarker}{\pgfqpoint{0.000000in}{-0.027778in}}{\pgfqpoint{0.000000in}{0.000000in}}{%
\pgfpathmoveto{\pgfqpoint{0.000000in}{0.000000in}}%
\pgfpathlineto{\pgfqpoint{0.000000in}{-0.027778in}}%
\pgfusepath{stroke,fill}%
}%
\begin{pgfscope}%
\pgfsys@transformshift{2.250641in}{3.363889in}%
\pgfsys@useobject{currentmarker}{}%
\end{pgfscope}%
\end{pgfscope}%
\begin{pgfscope}%
\pgfsetbuttcap%
\pgfsetroundjoin%
\definecolor{currentfill}{rgb}{0.000000,0.000000,0.000000}%
\pgfsetfillcolor{currentfill}%
\pgfsetlinewidth{0.602250pt}%
\definecolor{currentstroke}{rgb}{0.000000,0.000000,0.000000}%
\pgfsetstrokecolor{currentstroke}%
\pgfsetdash{}{0pt}%
\pgfsys@defobject{currentmarker}{\pgfqpoint{0.000000in}{-0.027778in}}{\pgfqpoint{0.000000in}{0.000000in}}{%
\pgfpathmoveto{\pgfqpoint{0.000000in}{0.000000in}}%
\pgfpathlineto{\pgfqpoint{0.000000in}{-0.027778in}}%
\pgfusepath{stroke,fill}%
}%
\begin{pgfscope}%
\pgfsys@transformshift{2.459804in}{3.363889in}%
\pgfsys@useobject{currentmarker}{}%
\end{pgfscope}%
\end{pgfscope}%
\begin{pgfscope}%
\pgfsetbuttcap%
\pgfsetroundjoin%
\definecolor{currentfill}{rgb}{0.000000,0.000000,0.000000}%
\pgfsetfillcolor{currentfill}%
\pgfsetlinewidth{0.602250pt}%
\definecolor{currentstroke}{rgb}{0.000000,0.000000,0.000000}%
\pgfsetstrokecolor{currentstroke}%
\pgfsetdash{}{0pt}%
\pgfsys@defobject{currentmarker}{\pgfqpoint{0.000000in}{-0.027778in}}{\pgfqpoint{0.000000in}{0.000000in}}{%
\pgfpathmoveto{\pgfqpoint{0.000000in}{0.000000in}}%
\pgfpathlineto{\pgfqpoint{0.000000in}{-0.027778in}}%
\pgfusepath{stroke,fill}%
}%
\begin{pgfscope}%
\pgfsys@transformshift{2.566012in}{3.363889in}%
\pgfsys@useobject{currentmarker}{}%
\end{pgfscope}%
\end{pgfscope}%
\begin{pgfscope}%
\pgfsetbuttcap%
\pgfsetroundjoin%
\definecolor{currentfill}{rgb}{0.000000,0.000000,0.000000}%
\pgfsetfillcolor{currentfill}%
\pgfsetlinewidth{0.602250pt}%
\definecolor{currentstroke}{rgb}{0.000000,0.000000,0.000000}%
\pgfsetstrokecolor{currentstroke}%
\pgfsetdash{}{0pt}%
\pgfsys@defobject{currentmarker}{\pgfqpoint{0.000000in}{-0.027778in}}{\pgfqpoint{0.000000in}{0.000000in}}{%
\pgfpathmoveto{\pgfqpoint{0.000000in}{0.000000in}}%
\pgfpathlineto{\pgfqpoint{0.000000in}{-0.027778in}}%
\pgfusepath{stroke,fill}%
}%
\begin{pgfscope}%
\pgfsys@transformshift{2.641368in}{3.363889in}%
\pgfsys@useobject{currentmarker}{}%
\end{pgfscope}%
\end{pgfscope}%
\begin{pgfscope}%
\pgfsetbuttcap%
\pgfsetroundjoin%
\definecolor{currentfill}{rgb}{0.000000,0.000000,0.000000}%
\pgfsetfillcolor{currentfill}%
\pgfsetlinewidth{0.602250pt}%
\definecolor{currentstroke}{rgb}{0.000000,0.000000,0.000000}%
\pgfsetstrokecolor{currentstroke}%
\pgfsetdash{}{0pt}%
\pgfsys@defobject{currentmarker}{\pgfqpoint{0.000000in}{-0.027778in}}{\pgfqpoint{0.000000in}{0.000000in}}{%
\pgfpathmoveto{\pgfqpoint{0.000000in}{0.000000in}}%
\pgfpathlineto{\pgfqpoint{0.000000in}{-0.027778in}}%
\pgfusepath{stroke,fill}%
}%
\begin{pgfscope}%
\pgfsys@transformshift{2.699819in}{3.363889in}%
\pgfsys@useobject{currentmarker}{}%
\end{pgfscope}%
\end{pgfscope}%
\begin{pgfscope}%
\pgfsetbuttcap%
\pgfsetroundjoin%
\definecolor{currentfill}{rgb}{0.000000,0.000000,0.000000}%
\pgfsetfillcolor{currentfill}%
\pgfsetlinewidth{0.602250pt}%
\definecolor{currentstroke}{rgb}{0.000000,0.000000,0.000000}%
\pgfsetstrokecolor{currentstroke}%
\pgfsetdash{}{0pt}%
\pgfsys@defobject{currentmarker}{\pgfqpoint{0.000000in}{-0.027778in}}{\pgfqpoint{0.000000in}{0.000000in}}{%
\pgfpathmoveto{\pgfqpoint{0.000000in}{0.000000in}}%
\pgfpathlineto{\pgfqpoint{0.000000in}{-0.027778in}}%
\pgfusepath{stroke,fill}%
}%
\begin{pgfscope}%
\pgfsys@transformshift{2.747577in}{3.363889in}%
\pgfsys@useobject{currentmarker}{}%
\end{pgfscope}%
\end{pgfscope}%
\begin{pgfscope}%
\pgfsetbuttcap%
\pgfsetroundjoin%
\definecolor{currentfill}{rgb}{0.000000,0.000000,0.000000}%
\pgfsetfillcolor{currentfill}%
\pgfsetlinewidth{0.602250pt}%
\definecolor{currentstroke}{rgb}{0.000000,0.000000,0.000000}%
\pgfsetstrokecolor{currentstroke}%
\pgfsetdash{}{0pt}%
\pgfsys@defobject{currentmarker}{\pgfqpoint{0.000000in}{-0.027778in}}{\pgfqpoint{0.000000in}{0.000000in}}{%
\pgfpathmoveto{\pgfqpoint{0.000000in}{0.000000in}}%
\pgfpathlineto{\pgfqpoint{0.000000in}{-0.027778in}}%
\pgfusepath{stroke,fill}%
}%
\begin{pgfscope}%
\pgfsys@transformshift{2.787955in}{3.363889in}%
\pgfsys@useobject{currentmarker}{}%
\end{pgfscope}%
\end{pgfscope}%
\begin{pgfscope}%
\pgfsetbuttcap%
\pgfsetroundjoin%
\definecolor{currentfill}{rgb}{0.000000,0.000000,0.000000}%
\pgfsetfillcolor{currentfill}%
\pgfsetlinewidth{0.602250pt}%
\definecolor{currentstroke}{rgb}{0.000000,0.000000,0.000000}%
\pgfsetstrokecolor{currentstroke}%
\pgfsetdash{}{0pt}%
\pgfsys@defobject{currentmarker}{\pgfqpoint{0.000000in}{-0.027778in}}{\pgfqpoint{0.000000in}{0.000000in}}{%
\pgfpathmoveto{\pgfqpoint{0.000000in}{0.000000in}}%
\pgfpathlineto{\pgfqpoint{0.000000in}{-0.027778in}}%
\pgfusepath{stroke,fill}%
}%
\begin{pgfscope}%
\pgfsys@transformshift{2.822933in}{3.363889in}%
\pgfsys@useobject{currentmarker}{}%
\end{pgfscope}%
\end{pgfscope}%
\begin{pgfscope}%
\pgfsetbuttcap%
\pgfsetroundjoin%
\definecolor{currentfill}{rgb}{0.000000,0.000000,0.000000}%
\pgfsetfillcolor{currentfill}%
\pgfsetlinewidth{0.602250pt}%
\definecolor{currentstroke}{rgb}{0.000000,0.000000,0.000000}%
\pgfsetstrokecolor{currentstroke}%
\pgfsetdash{}{0pt}%
\pgfsys@defobject{currentmarker}{\pgfqpoint{0.000000in}{-0.027778in}}{\pgfqpoint{0.000000in}{0.000000in}}{%
\pgfpathmoveto{\pgfqpoint{0.000000in}{0.000000in}}%
\pgfpathlineto{\pgfqpoint{0.000000in}{-0.027778in}}%
\pgfusepath{stroke,fill}%
}%
\begin{pgfscope}%
\pgfsys@transformshift{2.853785in}{3.363889in}%
\pgfsys@useobject{currentmarker}{}%
\end{pgfscope}%
\end{pgfscope}%
\begin{pgfscope}%
\pgfsetbuttcap%
\pgfsetroundjoin%
\definecolor{currentfill}{rgb}{0.000000,0.000000,0.000000}%
\pgfsetfillcolor{currentfill}%
\pgfsetlinewidth{0.602250pt}%
\definecolor{currentstroke}{rgb}{0.000000,0.000000,0.000000}%
\pgfsetstrokecolor{currentstroke}%
\pgfsetdash{}{0pt}%
\pgfsys@defobject{currentmarker}{\pgfqpoint{0.000000in}{-0.027778in}}{\pgfqpoint{0.000000in}{0.000000in}}{%
\pgfpathmoveto{\pgfqpoint{0.000000in}{0.000000in}}%
\pgfpathlineto{\pgfqpoint{0.000000in}{-0.027778in}}%
\pgfusepath{stroke,fill}%
}%
\begin{pgfscope}%
\pgfsys@transformshift{3.062948in}{3.363889in}%
\pgfsys@useobject{currentmarker}{}%
\end{pgfscope}%
\end{pgfscope}%
\begin{pgfscope}%
\pgfsetbuttcap%
\pgfsetroundjoin%
\definecolor{currentfill}{rgb}{0.000000,0.000000,0.000000}%
\pgfsetfillcolor{currentfill}%
\pgfsetlinewidth{0.602250pt}%
\definecolor{currentstroke}{rgb}{0.000000,0.000000,0.000000}%
\pgfsetstrokecolor{currentstroke}%
\pgfsetdash{}{0pt}%
\pgfsys@defobject{currentmarker}{\pgfqpoint{0.000000in}{-0.027778in}}{\pgfqpoint{0.000000in}{0.000000in}}{%
\pgfpathmoveto{\pgfqpoint{0.000000in}{0.000000in}}%
\pgfpathlineto{\pgfqpoint{0.000000in}{-0.027778in}}%
\pgfusepath{stroke,fill}%
}%
\begin{pgfscope}%
\pgfsys@transformshift{3.169156in}{3.363889in}%
\pgfsys@useobject{currentmarker}{}%
\end{pgfscope}%
\end{pgfscope}%
\begin{pgfscope}%
\pgfsetbuttcap%
\pgfsetroundjoin%
\definecolor{currentfill}{rgb}{0.000000,0.000000,0.000000}%
\pgfsetfillcolor{currentfill}%
\pgfsetlinewidth{0.602250pt}%
\definecolor{currentstroke}{rgb}{0.000000,0.000000,0.000000}%
\pgfsetstrokecolor{currentstroke}%
\pgfsetdash{}{0pt}%
\pgfsys@defobject{currentmarker}{\pgfqpoint{0.000000in}{-0.027778in}}{\pgfqpoint{0.000000in}{0.000000in}}{%
\pgfpathmoveto{\pgfqpoint{0.000000in}{0.000000in}}%
\pgfpathlineto{\pgfqpoint{0.000000in}{-0.027778in}}%
\pgfusepath{stroke,fill}%
}%
\begin{pgfscope}%
\pgfsys@transformshift{3.244512in}{3.363889in}%
\pgfsys@useobject{currentmarker}{}%
\end{pgfscope}%
\end{pgfscope}%
\begin{pgfscope}%
\pgfsetbuttcap%
\pgfsetroundjoin%
\definecolor{currentfill}{rgb}{0.000000,0.000000,0.000000}%
\pgfsetfillcolor{currentfill}%
\pgfsetlinewidth{0.602250pt}%
\definecolor{currentstroke}{rgb}{0.000000,0.000000,0.000000}%
\pgfsetstrokecolor{currentstroke}%
\pgfsetdash{}{0pt}%
\pgfsys@defobject{currentmarker}{\pgfqpoint{0.000000in}{-0.027778in}}{\pgfqpoint{0.000000in}{0.000000in}}{%
\pgfpathmoveto{\pgfqpoint{0.000000in}{0.000000in}}%
\pgfpathlineto{\pgfqpoint{0.000000in}{-0.027778in}}%
\pgfusepath{stroke,fill}%
}%
\begin{pgfscope}%
\pgfsys@transformshift{3.302963in}{3.363889in}%
\pgfsys@useobject{currentmarker}{}%
\end{pgfscope}%
\end{pgfscope}%
\begin{pgfscope}%
\pgfsetbuttcap%
\pgfsetroundjoin%
\definecolor{currentfill}{rgb}{0.000000,0.000000,0.000000}%
\pgfsetfillcolor{currentfill}%
\pgfsetlinewidth{0.602250pt}%
\definecolor{currentstroke}{rgb}{0.000000,0.000000,0.000000}%
\pgfsetstrokecolor{currentstroke}%
\pgfsetdash{}{0pt}%
\pgfsys@defobject{currentmarker}{\pgfqpoint{0.000000in}{-0.027778in}}{\pgfqpoint{0.000000in}{0.000000in}}{%
\pgfpathmoveto{\pgfqpoint{0.000000in}{0.000000in}}%
\pgfpathlineto{\pgfqpoint{0.000000in}{-0.027778in}}%
\pgfusepath{stroke,fill}%
}%
\begin{pgfscope}%
\pgfsys@transformshift{3.350721in}{3.363889in}%
\pgfsys@useobject{currentmarker}{}%
\end{pgfscope}%
\end{pgfscope}%
\begin{pgfscope}%
\pgfsetbuttcap%
\pgfsetroundjoin%
\definecolor{currentfill}{rgb}{0.000000,0.000000,0.000000}%
\pgfsetfillcolor{currentfill}%
\pgfsetlinewidth{0.602250pt}%
\definecolor{currentstroke}{rgb}{0.000000,0.000000,0.000000}%
\pgfsetstrokecolor{currentstroke}%
\pgfsetdash{}{0pt}%
\pgfsys@defobject{currentmarker}{\pgfqpoint{0.000000in}{-0.027778in}}{\pgfqpoint{0.000000in}{0.000000in}}{%
\pgfpathmoveto{\pgfqpoint{0.000000in}{0.000000in}}%
\pgfpathlineto{\pgfqpoint{0.000000in}{-0.027778in}}%
\pgfusepath{stroke,fill}%
}%
\begin{pgfscope}%
\pgfsys@transformshift{3.391099in}{3.363889in}%
\pgfsys@useobject{currentmarker}{}%
\end{pgfscope}%
\end{pgfscope}%
\begin{pgfscope}%
\pgfsetbuttcap%
\pgfsetroundjoin%
\definecolor{currentfill}{rgb}{0.000000,0.000000,0.000000}%
\pgfsetfillcolor{currentfill}%
\pgfsetlinewidth{0.602250pt}%
\definecolor{currentstroke}{rgb}{0.000000,0.000000,0.000000}%
\pgfsetstrokecolor{currentstroke}%
\pgfsetdash{}{0pt}%
\pgfsys@defobject{currentmarker}{\pgfqpoint{0.000000in}{-0.027778in}}{\pgfqpoint{0.000000in}{0.000000in}}{%
\pgfpathmoveto{\pgfqpoint{0.000000in}{0.000000in}}%
\pgfpathlineto{\pgfqpoint{0.000000in}{-0.027778in}}%
\pgfusepath{stroke,fill}%
}%
\begin{pgfscope}%
\pgfsys@transformshift{3.426077in}{3.363889in}%
\pgfsys@useobject{currentmarker}{}%
\end{pgfscope}%
\end{pgfscope}%
\begin{pgfscope}%
\pgfsetbuttcap%
\pgfsetroundjoin%
\definecolor{currentfill}{rgb}{0.000000,0.000000,0.000000}%
\pgfsetfillcolor{currentfill}%
\pgfsetlinewidth{0.602250pt}%
\definecolor{currentstroke}{rgb}{0.000000,0.000000,0.000000}%
\pgfsetstrokecolor{currentstroke}%
\pgfsetdash{}{0pt}%
\pgfsys@defobject{currentmarker}{\pgfqpoint{0.000000in}{-0.027778in}}{\pgfqpoint{0.000000in}{0.000000in}}{%
\pgfpathmoveto{\pgfqpoint{0.000000in}{0.000000in}}%
\pgfpathlineto{\pgfqpoint{0.000000in}{-0.027778in}}%
\pgfusepath{stroke,fill}%
}%
\begin{pgfscope}%
\pgfsys@transformshift{3.456929in}{3.363889in}%
\pgfsys@useobject{currentmarker}{}%
\end{pgfscope}%
\end{pgfscope}%
\begin{pgfscope}%
\pgfsetbuttcap%
\pgfsetroundjoin%
\definecolor{currentfill}{rgb}{0.000000,0.000000,0.000000}%
\pgfsetfillcolor{currentfill}%
\pgfsetlinewidth{0.602250pt}%
\definecolor{currentstroke}{rgb}{0.000000,0.000000,0.000000}%
\pgfsetstrokecolor{currentstroke}%
\pgfsetdash{}{0pt}%
\pgfsys@defobject{currentmarker}{\pgfqpoint{0.000000in}{-0.027778in}}{\pgfqpoint{0.000000in}{0.000000in}}{%
\pgfpathmoveto{\pgfqpoint{0.000000in}{0.000000in}}%
\pgfpathlineto{\pgfqpoint{0.000000in}{-0.027778in}}%
\pgfusepath{stroke,fill}%
}%
\begin{pgfscope}%
\pgfsys@transformshift{3.666092in}{3.363889in}%
\pgfsys@useobject{currentmarker}{}%
\end{pgfscope}%
\end{pgfscope}%
\begin{pgfscope}%
\pgfsetbuttcap%
\pgfsetroundjoin%
\definecolor{currentfill}{rgb}{0.000000,0.000000,0.000000}%
\pgfsetfillcolor{currentfill}%
\pgfsetlinewidth{0.602250pt}%
\definecolor{currentstroke}{rgb}{0.000000,0.000000,0.000000}%
\pgfsetstrokecolor{currentstroke}%
\pgfsetdash{}{0pt}%
\pgfsys@defobject{currentmarker}{\pgfqpoint{0.000000in}{-0.027778in}}{\pgfqpoint{0.000000in}{0.000000in}}{%
\pgfpathmoveto{\pgfqpoint{0.000000in}{0.000000in}}%
\pgfpathlineto{\pgfqpoint{0.000000in}{-0.027778in}}%
\pgfusepath{stroke,fill}%
}%
\begin{pgfscope}%
\pgfsys@transformshift{3.772300in}{3.363889in}%
\pgfsys@useobject{currentmarker}{}%
\end{pgfscope}%
\end{pgfscope}%
\begin{pgfscope}%
\pgfsetbuttcap%
\pgfsetroundjoin%
\definecolor{currentfill}{rgb}{0.000000,0.000000,0.000000}%
\pgfsetfillcolor{currentfill}%
\pgfsetlinewidth{0.602250pt}%
\definecolor{currentstroke}{rgb}{0.000000,0.000000,0.000000}%
\pgfsetstrokecolor{currentstroke}%
\pgfsetdash{}{0pt}%
\pgfsys@defobject{currentmarker}{\pgfqpoint{0.000000in}{-0.027778in}}{\pgfqpoint{0.000000in}{0.000000in}}{%
\pgfpathmoveto{\pgfqpoint{0.000000in}{0.000000in}}%
\pgfpathlineto{\pgfqpoint{0.000000in}{-0.027778in}}%
\pgfusepath{stroke,fill}%
}%
\begin{pgfscope}%
\pgfsys@transformshift{3.847656in}{3.363889in}%
\pgfsys@useobject{currentmarker}{}%
\end{pgfscope}%
\end{pgfscope}%
\begin{pgfscope}%
\pgfsetbuttcap%
\pgfsetroundjoin%
\definecolor{currentfill}{rgb}{0.000000,0.000000,0.000000}%
\pgfsetfillcolor{currentfill}%
\pgfsetlinewidth{0.803000pt}%
\definecolor{currentstroke}{rgb}{0.000000,0.000000,0.000000}%
\pgfsetstrokecolor{currentstroke}%
\pgfsetdash{}{0pt}%
\pgfsys@defobject{currentmarker}{\pgfqpoint{-0.048611in}{0.000000in}}{\pgfqpoint{0.000000in}{0.000000in}}{%
\pgfpathmoveto{\pgfqpoint{0.000000in}{0.000000in}}%
\pgfpathlineto{\pgfqpoint{-0.048611in}{0.000000in}}%
\pgfusepath{stroke,fill}%
}%
\begin{pgfscope}%
\pgfsys@transformshift{0.557986in}{3.549819in}%
\pgfsys@useobject{currentmarker}{}%
\end{pgfscope}%
\end{pgfscope}%
\begin{pgfscope}%
\definecolor{textcolor}{rgb}{0.000000,0.000000,0.000000}%
\pgfsetstrokecolor{textcolor}%
\pgfsetfillcolor{textcolor}%
\pgftext[x=0.239884in,y=3.497057in,left,base]{\color{textcolor}\sffamily\fontsize{10.000000}{12.000000}\selectfont 0.0}%
\end{pgfscope}%
\begin{pgfscope}%
\pgfsetbuttcap%
\pgfsetroundjoin%
\definecolor{currentfill}{rgb}{0.000000,0.000000,0.000000}%
\pgfsetfillcolor{currentfill}%
\pgfsetlinewidth{0.803000pt}%
\definecolor{currentstroke}{rgb}{0.000000,0.000000,0.000000}%
\pgfsetstrokecolor{currentstroke}%
\pgfsetdash{}{0pt}%
\pgfsys@defobject{currentmarker}{\pgfqpoint{-0.048611in}{0.000000in}}{\pgfqpoint{0.000000in}{0.000000in}}{%
\pgfpathmoveto{\pgfqpoint{0.000000in}{0.000000in}}%
\pgfpathlineto{\pgfqpoint{-0.048611in}{0.000000in}}%
\pgfusepath{stroke,fill}%
}%
\begin{pgfscope}%
\pgfsys@transformshift{0.557986in}{3.842377in}%
\pgfsys@useobject{currentmarker}{}%
\end{pgfscope}%
\end{pgfscope}%
\begin{pgfscope}%
\definecolor{textcolor}{rgb}{0.000000,0.000000,0.000000}%
\pgfsetstrokecolor{textcolor}%
\pgfsetfillcolor{textcolor}%
\pgftext[x=0.239884in,y=3.789615in,left,base]{\color{textcolor}\sffamily\fontsize{10.000000}{12.000000}\selectfont 0.2}%
\end{pgfscope}%
\begin{pgfscope}%
\pgfsetbuttcap%
\pgfsetroundjoin%
\definecolor{currentfill}{rgb}{0.000000,0.000000,0.000000}%
\pgfsetfillcolor{currentfill}%
\pgfsetlinewidth{0.803000pt}%
\definecolor{currentstroke}{rgb}{0.000000,0.000000,0.000000}%
\pgfsetstrokecolor{currentstroke}%
\pgfsetdash{}{0pt}%
\pgfsys@defobject{currentmarker}{\pgfqpoint{-0.048611in}{0.000000in}}{\pgfqpoint{0.000000in}{0.000000in}}{%
\pgfpathmoveto{\pgfqpoint{0.000000in}{0.000000in}}%
\pgfpathlineto{\pgfqpoint{-0.048611in}{0.000000in}}%
\pgfusepath{stroke,fill}%
}%
\begin{pgfscope}%
\pgfsys@transformshift{0.557986in}{4.134935in}%
\pgfsys@useobject{currentmarker}{}%
\end{pgfscope}%
\end{pgfscope}%
\begin{pgfscope}%
\definecolor{textcolor}{rgb}{0.000000,0.000000,0.000000}%
\pgfsetstrokecolor{textcolor}%
\pgfsetfillcolor{textcolor}%
\pgftext[x=0.239884in,y=4.082173in,left,base]{\color{textcolor}\sffamily\fontsize{10.000000}{12.000000}\selectfont 0.4}%
\end{pgfscope}%
\begin{pgfscope}%
\pgfsetbuttcap%
\pgfsetroundjoin%
\definecolor{currentfill}{rgb}{0.000000,0.000000,0.000000}%
\pgfsetfillcolor{currentfill}%
\pgfsetlinewidth{0.803000pt}%
\definecolor{currentstroke}{rgb}{0.000000,0.000000,0.000000}%
\pgfsetstrokecolor{currentstroke}%
\pgfsetdash{}{0pt}%
\pgfsys@defobject{currentmarker}{\pgfqpoint{-0.048611in}{0.000000in}}{\pgfqpoint{0.000000in}{0.000000in}}{%
\pgfpathmoveto{\pgfqpoint{0.000000in}{0.000000in}}%
\pgfpathlineto{\pgfqpoint{-0.048611in}{0.000000in}}%
\pgfusepath{stroke,fill}%
}%
\begin{pgfscope}%
\pgfsys@transformshift{0.557986in}{4.427493in}%
\pgfsys@useobject{currentmarker}{}%
\end{pgfscope}%
\end{pgfscope}%
\begin{pgfscope}%
\definecolor{textcolor}{rgb}{0.000000,0.000000,0.000000}%
\pgfsetstrokecolor{textcolor}%
\pgfsetfillcolor{textcolor}%
\pgftext[x=0.239884in,y=4.374731in,left,base]{\color{textcolor}\sffamily\fontsize{10.000000}{12.000000}\selectfont 0.6}%
\end{pgfscope}%
\begin{pgfscope}%
\pgfsetbuttcap%
\pgfsetroundjoin%
\definecolor{currentfill}{rgb}{0.000000,0.000000,0.000000}%
\pgfsetfillcolor{currentfill}%
\pgfsetlinewidth{0.803000pt}%
\definecolor{currentstroke}{rgb}{0.000000,0.000000,0.000000}%
\pgfsetstrokecolor{currentstroke}%
\pgfsetdash{}{0pt}%
\pgfsys@defobject{currentmarker}{\pgfqpoint{-0.048611in}{0.000000in}}{\pgfqpoint{0.000000in}{0.000000in}}{%
\pgfpathmoveto{\pgfqpoint{0.000000in}{0.000000in}}%
\pgfpathlineto{\pgfqpoint{-0.048611in}{0.000000in}}%
\pgfusepath{stroke,fill}%
}%
\begin{pgfscope}%
\pgfsys@transformshift{0.557986in}{4.720051in}%
\pgfsys@useobject{currentmarker}{}%
\end{pgfscope}%
\end{pgfscope}%
\begin{pgfscope}%
\definecolor{textcolor}{rgb}{0.000000,0.000000,0.000000}%
\pgfsetstrokecolor{textcolor}%
\pgfsetfillcolor{textcolor}%
\pgftext[x=0.239884in,y=4.667289in,left,base]{\color{textcolor}\sffamily\fontsize{10.000000}{12.000000}\selectfont 0.8}%
\end{pgfscope}%
\begin{pgfscope}%
\pgfsetbuttcap%
\pgfsetroundjoin%
\definecolor{currentfill}{rgb}{0.000000,0.000000,0.000000}%
\pgfsetfillcolor{currentfill}%
\pgfsetlinewidth{0.803000pt}%
\definecolor{currentstroke}{rgb}{0.000000,0.000000,0.000000}%
\pgfsetstrokecolor{currentstroke}%
\pgfsetdash{}{0pt}%
\pgfsys@defobject{currentmarker}{\pgfqpoint{-0.048611in}{0.000000in}}{\pgfqpoint{0.000000in}{0.000000in}}{%
\pgfpathmoveto{\pgfqpoint{0.000000in}{0.000000in}}%
\pgfpathlineto{\pgfqpoint{-0.048611in}{0.000000in}}%
\pgfusepath{stroke,fill}%
}%
\begin{pgfscope}%
\pgfsys@transformshift{0.557986in}{5.012609in}%
\pgfsys@useobject{currentmarker}{}%
\end{pgfscope}%
\end{pgfscope}%
\begin{pgfscope}%
\definecolor{textcolor}{rgb}{0.000000,0.000000,0.000000}%
\pgfsetstrokecolor{textcolor}%
\pgfsetfillcolor{textcolor}%
\pgftext[x=0.239884in,y=4.959847in,left,base]{\color{textcolor}\sffamily\fontsize{10.000000}{12.000000}\selectfont 1.0}%
\end{pgfscope}%
\begin{pgfscope}%
\pgfsetbuttcap%
\pgfsetroundjoin%
\definecolor{currentfill}{rgb}{0.000000,0.000000,0.000000}%
\pgfsetfillcolor{currentfill}%
\pgfsetlinewidth{0.803000pt}%
\definecolor{currentstroke}{rgb}{0.000000,0.000000,0.000000}%
\pgfsetstrokecolor{currentstroke}%
\pgfsetdash{}{0pt}%
\pgfsys@defobject{currentmarker}{\pgfqpoint{-0.048611in}{0.000000in}}{\pgfqpoint{0.000000in}{0.000000in}}{%
\pgfpathmoveto{\pgfqpoint{0.000000in}{0.000000in}}%
\pgfpathlineto{\pgfqpoint{-0.048611in}{0.000000in}}%
\pgfusepath{stroke,fill}%
}%
\begin{pgfscope}%
\pgfsys@transformshift{0.557986in}{5.305167in}%
\pgfsys@useobject{currentmarker}{}%
\end{pgfscope}%
\end{pgfscope}%
\begin{pgfscope}%
\definecolor{textcolor}{rgb}{0.000000,0.000000,0.000000}%
\pgfsetstrokecolor{textcolor}%
\pgfsetfillcolor{textcolor}%
\pgftext[x=0.239884in,y=5.252405in,left,base]{\color{textcolor}\sffamily\fontsize{10.000000}{12.000000}\selectfont 1.2}%
\end{pgfscope}%
\begin{pgfscope}%
\pgfsetbuttcap%
\pgfsetroundjoin%
\definecolor{currentfill}{rgb}{0.000000,0.000000,0.000000}%
\pgfsetfillcolor{currentfill}%
\pgfsetlinewidth{0.803000pt}%
\definecolor{currentstroke}{rgb}{0.000000,0.000000,0.000000}%
\pgfsetstrokecolor{currentstroke}%
\pgfsetdash{}{0pt}%
\pgfsys@defobject{currentmarker}{\pgfqpoint{-0.048611in}{0.000000in}}{\pgfqpoint{0.000000in}{0.000000in}}{%
\pgfpathmoveto{\pgfqpoint{0.000000in}{0.000000in}}%
\pgfpathlineto{\pgfqpoint{-0.048611in}{0.000000in}}%
\pgfusepath{stroke,fill}%
}%
\begin{pgfscope}%
\pgfsys@transformshift{0.557986in}{5.597725in}%
\pgfsys@useobject{currentmarker}{}%
\end{pgfscope}%
\end{pgfscope}%
\begin{pgfscope}%
\definecolor{textcolor}{rgb}{0.000000,0.000000,0.000000}%
\pgfsetstrokecolor{textcolor}%
\pgfsetfillcolor{textcolor}%
\pgftext[x=0.239884in,y=5.544963in,left,base]{\color{textcolor}\sffamily\fontsize{10.000000}{12.000000}\selectfont 1.4}%
\end{pgfscope}%
\begin{pgfscope}%
\pgfpathrectangle{\pgfqpoint{0.557986in}{3.363889in}}{\pgfqpoint{3.318403in}{2.263889in}}%
\pgfusepath{clip}%
\pgfsetrectcap%
\pgfsetroundjoin%
\pgfsetlinewidth{1.505625pt}%
\definecolor{currentstroke}{rgb}{0.121569,0.466667,0.705882}%
\pgfsetstrokecolor{currentstroke}%
\pgfsetdash{}{0pt}%
\pgfpathmoveto{\pgfqpoint{0.708823in}{3.839424in}}%
\pgfpathlineto{\pgfqpoint{0.859609in}{3.895280in}}%
\pgfpathlineto{\pgfqpoint{1.010395in}{4.628915in}}%
\pgfpathlineto{\pgfqpoint{1.161181in}{4.801410in}}%
\pgfpathlineto{\pgfqpoint{1.311967in}{4.655646in}}%
\pgfpathlineto{\pgfqpoint{1.462753in}{4.916088in}}%
\pgfpathlineto{\pgfqpoint{1.613539in}{4.879613in}}%
\pgfpathlineto{\pgfqpoint{1.764325in}{4.690595in}}%
\pgfpathlineto{\pgfqpoint{1.915111in}{4.999268in}}%
\pgfpathlineto{\pgfqpoint{2.065897in}{5.011569in}}%
\pgfpathlineto{\pgfqpoint{2.216683in}{5.011487in}}%
\pgfpathlineto{\pgfqpoint{2.367469in}{5.011527in}}%
\pgfpathlineto{\pgfqpoint{2.518255in}{5.011511in}}%
\pgfpathlineto{\pgfqpoint{2.669041in}{5.011527in}}%
\pgfpathlineto{\pgfqpoint{2.819827in}{5.011524in}}%
\pgfpathlineto{\pgfqpoint{2.970613in}{5.011547in}}%
\pgfpathlineto{\pgfqpoint{3.121399in}{5.011541in}}%
\pgfpathlineto{\pgfqpoint{3.272185in}{5.011555in}}%
\pgfpathlineto{\pgfqpoint{3.422971in}{5.011543in}}%
\pgfpathlineto{\pgfqpoint{3.573756in}{5.011551in}}%
\pgfpathlineto{\pgfqpoint{3.724542in}{5.011551in}}%
\pgfusepath{stroke}%
\end{pgfscope}%
\begin{pgfscope}%
\pgfpathrectangle{\pgfqpoint{0.557986in}{3.363889in}}{\pgfqpoint{3.318403in}{2.263889in}}%
\pgfusepath{clip}%
\pgfsetrectcap%
\pgfsetroundjoin%
\pgfsetlinewidth{1.505625pt}%
\definecolor{currentstroke}{rgb}{1.000000,0.498039,0.054902}%
\pgfsetstrokecolor{currentstroke}%
\pgfsetdash{}{0pt}%
\pgfpathmoveto{\pgfqpoint{0.708823in}{3.900071in}}%
\pgfpathlineto{\pgfqpoint{0.859609in}{4.010608in}}%
\pgfpathlineto{\pgfqpoint{1.010395in}{4.198230in}}%
\pgfpathlineto{\pgfqpoint{1.161181in}{4.861184in}}%
\pgfpathlineto{\pgfqpoint{1.311967in}{4.504774in}}%
\pgfpathlineto{\pgfqpoint{1.462753in}{4.662495in}}%
\pgfpathlineto{\pgfqpoint{1.613539in}{4.686331in}}%
\pgfpathlineto{\pgfqpoint{1.764325in}{4.993549in}}%
\pgfpathlineto{\pgfqpoint{1.915111in}{4.992256in}}%
\pgfpathlineto{\pgfqpoint{2.065897in}{4.992991in}}%
\pgfpathlineto{\pgfqpoint{2.216683in}{4.993445in}}%
\pgfpathlineto{\pgfqpoint{2.367469in}{4.992858in}}%
\pgfpathlineto{\pgfqpoint{2.518255in}{4.992895in}}%
\pgfpathlineto{\pgfqpoint{2.669041in}{4.992941in}}%
\pgfpathlineto{\pgfqpoint{2.819827in}{4.992881in}}%
\pgfpathlineto{\pgfqpoint{2.970613in}{4.992896in}}%
\pgfpathlineto{\pgfqpoint{3.121399in}{4.992917in}}%
\pgfpathlineto{\pgfqpoint{3.272185in}{4.992870in}}%
\pgfpathlineto{\pgfqpoint{3.422971in}{4.992896in}}%
\pgfpathlineto{\pgfqpoint{3.573756in}{4.992926in}}%
\pgfpathlineto{\pgfqpoint{3.724542in}{4.992870in}}%
\pgfusepath{stroke}%
\end{pgfscope}%
\begin{pgfscope}%
\pgfpathrectangle{\pgfqpoint{0.557986in}{3.363889in}}{\pgfqpoint{3.318403in}{2.263889in}}%
\pgfusepath{clip}%
\pgfsetrectcap%
\pgfsetroundjoin%
\pgfsetlinewidth{1.505625pt}%
\definecolor{currentstroke}{rgb}{0.172549,0.627451,0.172549}%
\pgfsetstrokecolor{currentstroke}%
\pgfsetdash{}{0pt}%
\pgfpathmoveto{\pgfqpoint{0.708823in}{3.885386in}}%
\pgfpathlineto{\pgfqpoint{0.859609in}{3.928319in}}%
\pgfpathlineto{\pgfqpoint{1.010395in}{3.944716in}}%
\pgfpathlineto{\pgfqpoint{1.161181in}{4.209986in}}%
\pgfpathlineto{\pgfqpoint{1.311967in}{4.561953in}}%
\pgfpathlineto{\pgfqpoint{1.462753in}{4.621749in}}%
\pgfpathlineto{\pgfqpoint{1.613539in}{4.886221in}}%
\pgfpathlineto{\pgfqpoint{1.764325in}{4.889795in}}%
\pgfpathlineto{\pgfqpoint{1.915111in}{4.881060in}}%
\pgfpathlineto{\pgfqpoint{2.065897in}{4.886999in}}%
\pgfpathlineto{\pgfqpoint{2.216683in}{4.887890in}}%
\pgfpathlineto{\pgfqpoint{2.367469in}{4.879207in}}%
\pgfpathlineto{\pgfqpoint{2.518255in}{4.883891in}}%
\pgfpathlineto{\pgfqpoint{2.669041in}{4.884801in}}%
\pgfpathlineto{\pgfqpoint{2.819827in}{4.882169in}}%
\pgfpathlineto{\pgfqpoint{2.970613in}{4.882493in}}%
\pgfpathlineto{\pgfqpoint{3.121399in}{4.882825in}}%
\pgfpathlineto{\pgfqpoint{3.272185in}{4.882584in}}%
\pgfpathlineto{\pgfqpoint{3.422971in}{4.883307in}}%
\pgfpathlineto{\pgfqpoint{3.573756in}{4.883094in}}%
\pgfpathlineto{\pgfqpoint{3.724542in}{4.883098in}}%
\pgfusepath{stroke}%
\end{pgfscope}%
\begin{pgfscope}%
\pgfpathrectangle{\pgfqpoint{0.557986in}{3.363889in}}{\pgfqpoint{3.318403in}{2.263889in}}%
\pgfusepath{clip}%
\pgfsetrectcap%
\pgfsetroundjoin%
\pgfsetlinewidth{1.505625pt}%
\definecolor{currentstroke}{rgb}{0.839216,0.152941,0.156863}%
\pgfsetstrokecolor{currentstroke}%
\pgfsetdash{}{0pt}%
\pgfpathmoveto{\pgfqpoint{0.708823in}{3.769642in}}%
\pgfpathlineto{\pgfqpoint{0.859609in}{4.130453in}}%
\pgfpathlineto{\pgfqpoint{1.010395in}{3.875313in}}%
\pgfpathlineto{\pgfqpoint{1.161181in}{3.988322in}}%
\pgfpathlineto{\pgfqpoint{1.311967in}{4.316108in}}%
\pgfpathlineto{\pgfqpoint{1.462753in}{4.225938in}}%
\pgfpathlineto{\pgfqpoint{1.613539in}{4.033362in}}%
\pgfpathlineto{\pgfqpoint{1.764325in}{4.175058in}}%
\pgfpathlineto{\pgfqpoint{1.915111in}{4.130992in}}%
\pgfpathlineto{\pgfqpoint{2.065897in}{4.113252in}}%
\pgfpathlineto{\pgfqpoint{2.216683in}{4.094158in}}%
\pgfpathlineto{\pgfqpoint{2.367469in}{4.153160in}}%
\pgfpathlineto{\pgfqpoint{2.518255in}{4.115981in}}%
\pgfpathlineto{\pgfqpoint{2.669041in}{4.117308in}}%
\pgfpathlineto{\pgfqpoint{2.819827in}{4.124745in}}%
\pgfpathlineto{\pgfqpoint{2.970613in}{4.113389in}}%
\pgfpathlineto{\pgfqpoint{3.121399in}{4.135437in}}%
\pgfpathlineto{\pgfqpoint{3.272185in}{4.113997in}}%
\pgfpathlineto{\pgfqpoint{3.422971in}{4.120470in}}%
\pgfpathlineto{\pgfqpoint{3.573756in}{4.118492in}}%
\pgfpathlineto{\pgfqpoint{3.724542in}{4.119344in}}%
\pgfusepath{stroke}%
\end{pgfscope}%
\begin{pgfscope}%
\pgfpathrectangle{\pgfqpoint{0.557986in}{3.363889in}}{\pgfqpoint{3.318403in}{2.263889in}}%
\pgfusepath{clip}%
\pgfsetrectcap%
\pgfsetroundjoin%
\pgfsetlinewidth{1.505625pt}%
\definecolor{currentstroke}{rgb}{0.580392,0.403922,0.741176}%
\pgfsetstrokecolor{currentstroke}%
\pgfsetdash{}{0pt}%
\pgfpathmoveto{\pgfqpoint{0.708823in}{3.710452in}}%
\pgfpathlineto{\pgfqpoint{0.859609in}{3.891680in}}%
\pgfpathlineto{\pgfqpoint{1.010395in}{3.809931in}}%
\pgfpathlineto{\pgfqpoint{1.161181in}{3.822166in}}%
\pgfpathlineto{\pgfqpoint{1.311967in}{3.828266in}}%
\pgfpathlineto{\pgfqpoint{1.462753in}{3.781314in}}%
\pgfpathlineto{\pgfqpoint{1.613539in}{3.804810in}}%
\pgfpathlineto{\pgfqpoint{1.764325in}{3.803899in}}%
\pgfpathlineto{\pgfqpoint{1.915111in}{3.808713in}}%
\pgfpathlineto{\pgfqpoint{2.065897in}{3.810884in}}%
\pgfpathlineto{\pgfqpoint{2.216683in}{3.798051in}}%
\pgfpathlineto{\pgfqpoint{2.367469in}{3.794654in}}%
\pgfpathlineto{\pgfqpoint{2.518255in}{3.806475in}}%
\pgfpathlineto{\pgfqpoint{2.669041in}{3.797010in}}%
\pgfpathlineto{\pgfqpoint{2.819827in}{3.802214in}}%
\pgfpathlineto{\pgfqpoint{2.970613in}{3.805823in}}%
\pgfpathlineto{\pgfqpoint{3.121399in}{3.799601in}}%
\pgfpathlineto{\pgfqpoint{3.272185in}{3.800664in}}%
\pgfpathlineto{\pgfqpoint{3.422971in}{3.800410in}}%
\pgfpathlineto{\pgfqpoint{3.573756in}{3.801209in}}%
\pgfpathlineto{\pgfqpoint{3.724542in}{3.801497in}}%
\pgfusepath{stroke}%
\end{pgfscope}%
\begin{pgfscope}%
\pgfsetrectcap%
\pgfsetmiterjoin%
\pgfsetlinewidth{0.803000pt}%
\definecolor{currentstroke}{rgb}{0.000000,0.000000,0.000000}%
\pgfsetstrokecolor{currentstroke}%
\pgfsetdash{}{0pt}%
\pgfpathmoveto{\pgfqpoint{0.557986in}{3.363889in}}%
\pgfpathlineto{\pgfqpoint{0.557986in}{5.627778in}}%
\pgfusepath{stroke}%
\end{pgfscope}%
\begin{pgfscope}%
\pgfsetrectcap%
\pgfsetmiterjoin%
\pgfsetlinewidth{0.803000pt}%
\definecolor{currentstroke}{rgb}{0.000000,0.000000,0.000000}%
\pgfsetstrokecolor{currentstroke}%
\pgfsetdash{}{0pt}%
\pgfpathmoveto{\pgfqpoint{3.876389in}{3.363889in}}%
\pgfpathlineto{\pgfqpoint{3.876389in}{5.627778in}}%
\pgfusepath{stroke}%
\end{pgfscope}%
\begin{pgfscope}%
\pgfsetrectcap%
\pgfsetmiterjoin%
\pgfsetlinewidth{0.803000pt}%
\definecolor{currentstroke}{rgb}{0.000000,0.000000,0.000000}%
\pgfsetstrokecolor{currentstroke}%
\pgfsetdash{}{0pt}%
\pgfpathmoveto{\pgfqpoint{0.557986in}{3.363889in}}%
\pgfpathlineto{\pgfqpoint{3.876389in}{3.363889in}}%
\pgfusepath{stroke}%
\end{pgfscope}%
\begin{pgfscope}%
\pgfsetrectcap%
\pgfsetmiterjoin%
\pgfsetlinewidth{0.803000pt}%
\definecolor{currentstroke}{rgb}{0.000000,0.000000,0.000000}%
\pgfsetstrokecolor{currentstroke}%
\pgfsetdash{}{0pt}%
\pgfpathmoveto{\pgfqpoint{0.557986in}{5.627778in}}%
\pgfpathlineto{\pgfqpoint{3.876389in}{5.627778in}}%
\pgfusepath{stroke}%
\end{pgfscope}%
\begin{pgfscope}%
\definecolor{textcolor}{rgb}{0.000000,0.000000,0.000000}%
\pgfsetstrokecolor{textcolor}%
\pgfsetfillcolor{textcolor}%
\pgftext[x=2.217187in,y=5.711111in,,base]{\color{textcolor}\sffamily\fontsize{12.000000}{14.400000}\selectfont \(\displaystyle  N = 16 \)}%
\end{pgfscope}%
\begin{pgfscope}%
\pgfsetbuttcap%
\pgfsetmiterjoin%
\definecolor{currentfill}{rgb}{1.000000,1.000000,1.000000}%
\pgfsetfillcolor{currentfill}%
\pgfsetlinewidth{0.000000pt}%
\definecolor{currentstroke}{rgb}{0.000000,0.000000,0.000000}%
\pgfsetstrokecolor{currentstroke}%
\pgfsetstrokeopacity{0.000000}%
\pgfsetdash{}{0pt}%
\pgfpathmoveto{\pgfqpoint{4.482986in}{3.363889in}}%
\pgfpathlineto{\pgfqpoint{7.801389in}{3.363889in}}%
\pgfpathlineto{\pgfqpoint{7.801389in}{5.627778in}}%
\pgfpathlineto{\pgfqpoint{4.482986in}{5.627778in}}%
\pgfpathclose%
\pgfusepath{fill}%
\end{pgfscope}%
\begin{pgfscope}%
\pgfpathrectangle{\pgfqpoint{4.482986in}{3.363889in}}{\pgfqpoint{3.318403in}{2.263889in}}%
\pgfusepath{clip}%
\pgfsetbuttcap%
\pgfsetroundjoin%
\definecolor{currentfill}{rgb}{0.121569,0.466667,0.705882}%
\pgfsetfillcolor{currentfill}%
\pgfsetlinewidth{1.003750pt}%
\definecolor{currentstroke}{rgb}{0.121569,0.466667,0.705882}%
\pgfsetstrokecolor{currentstroke}%
\pgfsetdash{}{0pt}%
\pgfsys@defobject{currentmarker}{\pgfqpoint{-0.009821in}{-0.009821in}}{\pgfqpoint{0.009821in}{0.009821in}}{%
\pgfpathmoveto{\pgfqpoint{0.000000in}{-0.009821in}}%
\pgfpathcurveto{\pgfqpoint{0.002605in}{-0.009821in}}{\pgfqpoint{0.005103in}{-0.008786in}}{\pgfqpoint{0.006944in}{-0.006944in}}%
\pgfpathcurveto{\pgfqpoint{0.008786in}{-0.005103in}}{\pgfqpoint{0.009821in}{-0.002605in}}{\pgfqpoint{0.009821in}{0.000000in}}%
\pgfpathcurveto{\pgfqpoint{0.009821in}{0.002605in}}{\pgfqpoint{0.008786in}{0.005103in}}{\pgfqpoint{0.006944in}{0.006944in}}%
\pgfpathcurveto{\pgfqpoint{0.005103in}{0.008786in}}{\pgfqpoint{0.002605in}{0.009821in}}{\pgfqpoint{0.000000in}{0.009821in}}%
\pgfpathcurveto{\pgfqpoint{-0.002605in}{0.009821in}}{\pgfqpoint{-0.005103in}{0.008786in}}{\pgfqpoint{-0.006944in}{0.006944in}}%
\pgfpathcurveto{\pgfqpoint{-0.008786in}{0.005103in}}{\pgfqpoint{-0.009821in}{0.002605in}}{\pgfqpoint{-0.009821in}{0.000000in}}%
\pgfpathcurveto{\pgfqpoint{-0.009821in}{-0.002605in}}{\pgfqpoint{-0.008786in}{-0.005103in}}{\pgfqpoint{-0.006944in}{-0.006944in}}%
\pgfpathcurveto{\pgfqpoint{-0.005103in}{-0.008786in}}{\pgfqpoint{-0.002605in}{-0.009821in}}{\pgfqpoint{0.000000in}{-0.009821in}}%
\pgfpathclose%
\pgfusepath{stroke,fill}%
}%
\begin{pgfscope}%
\pgfsys@transformshift{4.633823in}{3.764997in}%
\pgfsys@useobject{currentmarker}{}%
\end{pgfscope}%
\begin{pgfscope}%
\pgfsys@transformshift{4.784609in}{3.749698in}%
\pgfsys@useobject{currentmarker}{}%
\end{pgfscope}%
\begin{pgfscope}%
\pgfsys@transformshift{4.935395in}{3.790628in}%
\pgfsys@useobject{currentmarker}{}%
\end{pgfscope}%
\begin{pgfscope}%
\pgfsys@transformshift{5.086181in}{3.894543in}%
\pgfsys@useobject{currentmarker}{}%
\end{pgfscope}%
\begin{pgfscope}%
\pgfsys@transformshift{5.236967in}{3.926825in}%
\pgfsys@useobject{currentmarker}{}%
\end{pgfscope}%
\begin{pgfscope}%
\pgfsys@transformshift{5.387753in}{4.151888in}%
\pgfsys@useobject{currentmarker}{}%
\end{pgfscope}%
\begin{pgfscope}%
\pgfsys@transformshift{5.538539in}{4.192528in}%
\pgfsys@useobject{currentmarker}{}%
\end{pgfscope}%
\begin{pgfscope}%
\pgfsys@transformshift{5.689325in}{4.633324in}%
\pgfsys@useobject{currentmarker}{}%
\end{pgfscope}%
\begin{pgfscope}%
\pgfsys@transformshift{5.840111in}{4.399581in}%
\pgfsys@useobject{currentmarker}{}%
\end{pgfscope}%
\begin{pgfscope}%
\pgfsys@transformshift{5.990897in}{4.773934in}%
\pgfsys@useobject{currentmarker}{}%
\end{pgfscope}%
\begin{pgfscope}%
\pgfsys@transformshift{6.141683in}{4.433524in}%
\pgfsys@useobject{currentmarker}{}%
\end{pgfscope}%
\begin{pgfscope}%
\pgfsys@transformshift{6.292469in}{4.762273in}%
\pgfsys@useobject{currentmarker}{}%
\end{pgfscope}%
\begin{pgfscope}%
\pgfsys@transformshift{6.443255in}{5.055804in}%
\pgfsys@useobject{currentmarker}{}%
\end{pgfscope}%
\begin{pgfscope}%
\pgfsys@transformshift{6.594041in}{4.567356in}%
\pgfsys@useobject{currentmarker}{}%
\end{pgfscope}%
\begin{pgfscope}%
\pgfsys@transformshift{6.744827in}{5.055779in}%
\pgfsys@useobject{currentmarker}{}%
\end{pgfscope}%
\begin{pgfscope}%
\pgfsys@transformshift{6.895613in}{5.055784in}%
\pgfsys@useobject{currentmarker}{}%
\end{pgfscope}%
\begin{pgfscope}%
\pgfsys@transformshift{7.046399in}{5.040685in}%
\pgfsys@useobject{currentmarker}{}%
\end{pgfscope}%
\begin{pgfscope}%
\pgfsys@transformshift{7.197185in}{5.055773in}%
\pgfsys@useobject{currentmarker}{}%
\end{pgfscope}%
\begin{pgfscope}%
\pgfsys@transformshift{7.347971in}{5.055789in}%
\pgfsys@useobject{currentmarker}{}%
\end{pgfscope}%
\begin{pgfscope}%
\pgfsys@transformshift{7.498756in}{5.055795in}%
\pgfsys@useobject{currentmarker}{}%
\end{pgfscope}%
\begin{pgfscope}%
\pgfsys@transformshift{7.649542in}{5.055788in}%
\pgfsys@useobject{currentmarker}{}%
\end{pgfscope}%
\end{pgfscope}%
\begin{pgfscope}%
\pgfpathrectangle{\pgfqpoint{4.482986in}{3.363889in}}{\pgfqpoint{3.318403in}{2.263889in}}%
\pgfusepath{clip}%
\pgfsetbuttcap%
\pgfsetroundjoin%
\definecolor{currentfill}{rgb}{0.121569,0.466667,0.705882}%
\pgfsetfillcolor{currentfill}%
\pgfsetfillopacity{0.300000}%
\pgfsetlinewidth{1.003750pt}%
\definecolor{currentstroke}{rgb}{0.121569,0.466667,0.705882}%
\pgfsetstrokecolor{currentstroke}%
\pgfsetstrokeopacity{0.300000}%
\pgfsetdash{}{0pt}%
\pgfpathmoveto{\pgfqpoint{4.633823in}{3.863167in}}%
\pgfpathlineto{\pgfqpoint{4.633823in}{3.666828in}}%
\pgfpathlineto{\pgfqpoint{4.784609in}{3.687143in}}%
\pgfpathlineto{\pgfqpoint{4.935395in}{3.632686in}}%
\pgfpathlineto{\pgfqpoint{5.086181in}{3.525142in}}%
\pgfpathlineto{\pgfqpoint{5.236967in}{3.714503in}}%
\pgfpathlineto{\pgfqpoint{5.387753in}{3.716250in}}%
\pgfpathlineto{\pgfqpoint{5.538539in}{3.466793in}}%
\pgfpathlineto{\pgfqpoint{5.689325in}{3.873495in}}%
\pgfpathlineto{\pgfqpoint{5.840111in}{3.582275in}}%
\pgfpathlineto{\pgfqpoint{5.990897in}{4.042022in}}%
\pgfpathlineto{\pgfqpoint{6.141683in}{3.497167in}}%
\pgfpathlineto{\pgfqpoint{6.292469in}{3.999672in}}%
\pgfpathlineto{\pgfqpoint{6.443255in}{5.055725in}}%
\pgfpathlineto{\pgfqpoint{6.594041in}{3.832256in}}%
\pgfpathlineto{\pgfqpoint{6.744827in}{5.055744in}}%
\pgfpathlineto{\pgfqpoint{6.895613in}{5.055756in}}%
\pgfpathlineto{\pgfqpoint{7.046399in}{5.001424in}}%
\pgfpathlineto{\pgfqpoint{7.197185in}{5.055757in}}%
\pgfpathlineto{\pgfqpoint{7.347971in}{5.055758in}}%
\pgfpathlineto{\pgfqpoint{7.498756in}{5.055791in}}%
\pgfpathlineto{\pgfqpoint{7.649542in}{5.055781in}}%
\pgfpathlineto{\pgfqpoint{7.649542in}{5.055796in}}%
\pgfpathlineto{\pgfqpoint{7.649542in}{5.055796in}}%
\pgfpathlineto{\pgfqpoint{7.498756in}{5.055800in}}%
\pgfpathlineto{\pgfqpoint{7.347971in}{5.055820in}}%
\pgfpathlineto{\pgfqpoint{7.197185in}{5.055789in}}%
\pgfpathlineto{\pgfqpoint{7.046399in}{5.079947in}}%
\pgfpathlineto{\pgfqpoint{6.895613in}{5.055812in}}%
\pgfpathlineto{\pgfqpoint{6.744827in}{5.055814in}}%
\pgfpathlineto{\pgfqpoint{6.594041in}{5.302457in}}%
\pgfpathlineto{\pgfqpoint{6.443255in}{5.055883in}}%
\pgfpathlineto{\pgfqpoint{6.292469in}{5.524874in}}%
\pgfpathlineto{\pgfqpoint{6.141683in}{5.369881in}}%
\pgfpathlineto{\pgfqpoint{5.990897in}{5.505845in}}%
\pgfpathlineto{\pgfqpoint{5.840111in}{5.216888in}}%
\pgfpathlineto{\pgfqpoint{5.689325in}{5.393152in}}%
\pgfpathlineto{\pgfqpoint{5.538539in}{4.918263in}}%
\pgfpathlineto{\pgfqpoint{5.387753in}{4.587526in}}%
\pgfpathlineto{\pgfqpoint{5.236967in}{4.139147in}}%
\pgfpathlineto{\pgfqpoint{5.086181in}{4.263944in}}%
\pgfpathlineto{\pgfqpoint{4.935395in}{3.948570in}}%
\pgfpathlineto{\pgfqpoint{4.784609in}{3.812253in}}%
\pgfpathlineto{\pgfqpoint{4.633823in}{3.863167in}}%
\pgfpathclose%
\pgfusepath{stroke,fill}%
\end{pgfscope}%
\begin{pgfscope}%
\pgfpathrectangle{\pgfqpoint{4.482986in}{3.363889in}}{\pgfqpoint{3.318403in}{2.263889in}}%
\pgfusepath{clip}%
\pgfsetbuttcap%
\pgfsetroundjoin%
\definecolor{currentfill}{rgb}{1.000000,0.498039,0.054902}%
\pgfsetfillcolor{currentfill}%
\pgfsetlinewidth{1.003750pt}%
\definecolor{currentstroke}{rgb}{1.000000,0.498039,0.054902}%
\pgfsetstrokecolor{currentstroke}%
\pgfsetdash{}{0pt}%
\pgfsys@defobject{currentmarker}{\pgfqpoint{-0.009821in}{-0.009821in}}{\pgfqpoint{0.009821in}{0.009821in}}{%
\pgfpathmoveto{\pgfqpoint{0.000000in}{-0.009821in}}%
\pgfpathcurveto{\pgfqpoint{0.002605in}{-0.009821in}}{\pgfqpoint{0.005103in}{-0.008786in}}{\pgfqpoint{0.006944in}{-0.006944in}}%
\pgfpathcurveto{\pgfqpoint{0.008786in}{-0.005103in}}{\pgfqpoint{0.009821in}{-0.002605in}}{\pgfqpoint{0.009821in}{0.000000in}}%
\pgfpathcurveto{\pgfqpoint{0.009821in}{0.002605in}}{\pgfqpoint{0.008786in}{0.005103in}}{\pgfqpoint{0.006944in}{0.006944in}}%
\pgfpathcurveto{\pgfqpoint{0.005103in}{0.008786in}}{\pgfqpoint{0.002605in}{0.009821in}}{\pgfqpoint{0.000000in}{0.009821in}}%
\pgfpathcurveto{\pgfqpoint{-0.002605in}{0.009821in}}{\pgfqpoint{-0.005103in}{0.008786in}}{\pgfqpoint{-0.006944in}{0.006944in}}%
\pgfpathcurveto{\pgfqpoint{-0.008786in}{0.005103in}}{\pgfqpoint{-0.009821in}{0.002605in}}{\pgfqpoint{-0.009821in}{0.000000in}}%
\pgfpathcurveto{\pgfqpoint{-0.009821in}{-0.002605in}}{\pgfqpoint{-0.008786in}{-0.005103in}}{\pgfqpoint{-0.006944in}{-0.006944in}}%
\pgfpathcurveto{\pgfqpoint{-0.005103in}{-0.008786in}}{\pgfqpoint{-0.002605in}{-0.009821in}}{\pgfqpoint{0.000000in}{-0.009821in}}%
\pgfpathclose%
\pgfusepath{stroke,fill}%
}%
\begin{pgfscope}%
\pgfsys@transformshift{4.633823in}{3.791985in}%
\pgfsys@useobject{currentmarker}{}%
\end{pgfscope}%
\begin{pgfscope}%
\pgfsys@transformshift{4.784609in}{3.801622in}%
\pgfsys@useobject{currentmarker}{}%
\end{pgfscope}%
\begin{pgfscope}%
\pgfsys@transformshift{4.935395in}{3.812519in}%
\pgfsys@useobject{currentmarker}{}%
\end{pgfscope}%
\begin{pgfscope}%
\pgfsys@transformshift{5.086181in}{3.806716in}%
\pgfsys@useobject{currentmarker}{}%
\end{pgfscope}%
\begin{pgfscope}%
\pgfsys@transformshift{5.236967in}{3.948761in}%
\pgfsys@useobject{currentmarker}{}%
\end{pgfscope}%
\begin{pgfscope}%
\pgfsys@transformshift{5.387753in}{3.951914in}%
\pgfsys@useobject{currentmarker}{}%
\end{pgfscope}%
\begin{pgfscope}%
\pgfsys@transformshift{5.538539in}{3.974789in}%
\pgfsys@useobject{currentmarker}{}%
\end{pgfscope}%
\begin{pgfscope}%
\pgfsys@transformshift{5.689325in}{4.399587in}%
\pgfsys@useobject{currentmarker}{}%
\end{pgfscope}%
\begin{pgfscope}%
\pgfsys@transformshift{5.840111in}{4.390414in}%
\pgfsys@useobject{currentmarker}{}%
\end{pgfscope}%
\begin{pgfscope}%
\pgfsys@transformshift{5.990897in}{4.709515in}%
\pgfsys@useobject{currentmarker}{}%
\end{pgfscope}%
\begin{pgfscope}%
\pgfsys@transformshift{6.141683in}{4.737247in}%
\pgfsys@useobject{currentmarker}{}%
\end{pgfscope}%
\begin{pgfscope}%
\pgfsys@transformshift{6.292469in}{5.036418in}%
\pgfsys@useobject{currentmarker}{}%
\end{pgfscope}%
\begin{pgfscope}%
\pgfsys@transformshift{6.443255in}{4.984157in}%
\pgfsys@useobject{currentmarker}{}%
\end{pgfscope}%
\begin{pgfscope}%
\pgfsys@transformshift{6.594041in}{4.887584in}%
\pgfsys@useobject{currentmarker}{}%
\end{pgfscope}%
\begin{pgfscope}%
\pgfsys@transformshift{6.744827in}{5.038014in}%
\pgfsys@useobject{currentmarker}{}%
\end{pgfscope}%
\begin{pgfscope}%
\pgfsys@transformshift{6.895613in}{5.037999in}%
\pgfsys@useobject{currentmarker}{}%
\end{pgfscope}%
\begin{pgfscope}%
\pgfsys@transformshift{7.046399in}{5.037887in}%
\pgfsys@useobject{currentmarker}{}%
\end{pgfscope}%
\begin{pgfscope}%
\pgfsys@transformshift{7.197185in}{5.037968in}%
\pgfsys@useobject{currentmarker}{}%
\end{pgfscope}%
\begin{pgfscope}%
\pgfsys@transformshift{7.347971in}{5.037860in}%
\pgfsys@useobject{currentmarker}{}%
\end{pgfscope}%
\begin{pgfscope}%
\pgfsys@transformshift{7.498756in}{5.037874in}%
\pgfsys@useobject{currentmarker}{}%
\end{pgfscope}%
\begin{pgfscope}%
\pgfsys@transformshift{7.649542in}{5.037922in}%
\pgfsys@useobject{currentmarker}{}%
\end{pgfscope}%
\end{pgfscope}%
\begin{pgfscope}%
\pgfpathrectangle{\pgfqpoint{4.482986in}{3.363889in}}{\pgfqpoint{3.318403in}{2.263889in}}%
\pgfusepath{clip}%
\pgfsetbuttcap%
\pgfsetroundjoin%
\definecolor{currentfill}{rgb}{1.000000,0.498039,0.054902}%
\pgfsetfillcolor{currentfill}%
\pgfsetfillopacity{0.300000}%
\pgfsetlinewidth{1.003750pt}%
\definecolor{currentstroke}{rgb}{1.000000,0.498039,0.054902}%
\pgfsetstrokecolor{currentstroke}%
\pgfsetstrokeopacity{0.300000}%
\pgfsetdash{}{0pt}%
\pgfpathmoveto{\pgfqpoint{4.633823in}{3.971940in}}%
\pgfpathlineto{\pgfqpoint{4.633823in}{3.612030in}}%
\pgfpathlineto{\pgfqpoint{4.784609in}{3.570440in}}%
\pgfpathlineto{\pgfqpoint{4.935395in}{3.714190in}}%
\pgfpathlineto{\pgfqpoint{5.086181in}{3.615441in}}%
\pgfpathlineto{\pgfqpoint{5.236967in}{3.772842in}}%
\pgfpathlineto{\pgfqpoint{5.387753in}{3.862363in}}%
\pgfpathlineto{\pgfqpoint{5.538539in}{3.598416in}}%
\pgfpathlineto{\pgfqpoint{5.689325in}{3.770700in}}%
\pgfpathlineto{\pgfqpoint{5.840111in}{3.826495in}}%
\pgfpathlineto{\pgfqpoint{5.990897in}{3.946586in}}%
\pgfpathlineto{\pgfqpoint{6.141683in}{4.042612in}}%
\pgfpathlineto{\pgfqpoint{6.292469in}{5.031927in}}%
\pgfpathlineto{\pgfqpoint{6.443255in}{4.845125in}}%
\pgfpathlineto{\pgfqpoint{6.594041in}{4.499202in}}%
\pgfpathlineto{\pgfqpoint{6.744827in}{5.037960in}}%
\pgfpathlineto{\pgfqpoint{6.895613in}{5.037894in}}%
\pgfpathlineto{\pgfqpoint{7.046399in}{5.037701in}}%
\pgfpathlineto{\pgfqpoint{7.197185in}{5.037856in}}%
\pgfpathlineto{\pgfqpoint{7.347971in}{5.037828in}}%
\pgfpathlineto{\pgfqpoint{7.498756in}{5.037814in}}%
\pgfpathlineto{\pgfqpoint{7.649542in}{5.037879in}}%
\pgfpathlineto{\pgfqpoint{7.649542in}{5.037966in}}%
\pgfpathlineto{\pgfqpoint{7.649542in}{5.037966in}}%
\pgfpathlineto{\pgfqpoint{7.498756in}{5.037933in}}%
\pgfpathlineto{\pgfqpoint{7.347971in}{5.037893in}}%
\pgfpathlineto{\pgfqpoint{7.197185in}{5.038080in}}%
\pgfpathlineto{\pgfqpoint{7.046399in}{5.038073in}}%
\pgfpathlineto{\pgfqpoint{6.895613in}{5.038104in}}%
\pgfpathlineto{\pgfqpoint{6.744827in}{5.038068in}}%
\pgfpathlineto{\pgfqpoint{6.594041in}{5.275966in}}%
\pgfpathlineto{\pgfqpoint{6.443255in}{5.123188in}}%
\pgfpathlineto{\pgfqpoint{6.292469in}{5.040910in}}%
\pgfpathlineto{\pgfqpoint{6.141683in}{5.431882in}}%
\pgfpathlineto{\pgfqpoint{5.990897in}{5.472445in}}%
\pgfpathlineto{\pgfqpoint{5.840111in}{4.954332in}}%
\pgfpathlineto{\pgfqpoint{5.689325in}{5.028473in}}%
\pgfpathlineto{\pgfqpoint{5.538539in}{4.351162in}}%
\pgfpathlineto{\pgfqpoint{5.387753in}{4.041464in}}%
\pgfpathlineto{\pgfqpoint{5.236967in}{4.124681in}}%
\pgfpathlineto{\pgfqpoint{5.086181in}{3.997990in}}%
\pgfpathlineto{\pgfqpoint{4.935395in}{3.910848in}}%
\pgfpathlineto{\pgfqpoint{4.784609in}{4.032804in}}%
\pgfpathlineto{\pgfqpoint{4.633823in}{3.971940in}}%
\pgfpathclose%
\pgfusepath{stroke,fill}%
\end{pgfscope}%
\begin{pgfscope}%
\pgfpathrectangle{\pgfqpoint{4.482986in}{3.363889in}}{\pgfqpoint{3.318403in}{2.263889in}}%
\pgfusepath{clip}%
\pgfsetbuttcap%
\pgfsetroundjoin%
\definecolor{currentfill}{rgb}{0.172549,0.627451,0.172549}%
\pgfsetfillcolor{currentfill}%
\pgfsetlinewidth{1.003750pt}%
\definecolor{currentstroke}{rgb}{0.172549,0.627451,0.172549}%
\pgfsetstrokecolor{currentstroke}%
\pgfsetdash{}{0pt}%
\pgfsys@defobject{currentmarker}{\pgfqpoint{-0.009821in}{-0.009821in}}{\pgfqpoint{0.009821in}{0.009821in}}{%
\pgfpathmoveto{\pgfqpoint{0.000000in}{-0.009821in}}%
\pgfpathcurveto{\pgfqpoint{0.002605in}{-0.009821in}}{\pgfqpoint{0.005103in}{-0.008786in}}{\pgfqpoint{0.006944in}{-0.006944in}}%
\pgfpathcurveto{\pgfqpoint{0.008786in}{-0.005103in}}{\pgfqpoint{0.009821in}{-0.002605in}}{\pgfqpoint{0.009821in}{0.000000in}}%
\pgfpathcurveto{\pgfqpoint{0.009821in}{0.002605in}}{\pgfqpoint{0.008786in}{0.005103in}}{\pgfqpoint{0.006944in}{0.006944in}}%
\pgfpathcurveto{\pgfqpoint{0.005103in}{0.008786in}}{\pgfqpoint{0.002605in}{0.009821in}}{\pgfqpoint{0.000000in}{0.009821in}}%
\pgfpathcurveto{\pgfqpoint{-0.002605in}{0.009821in}}{\pgfqpoint{-0.005103in}{0.008786in}}{\pgfqpoint{-0.006944in}{0.006944in}}%
\pgfpathcurveto{\pgfqpoint{-0.008786in}{0.005103in}}{\pgfqpoint{-0.009821in}{0.002605in}}{\pgfqpoint{-0.009821in}{0.000000in}}%
\pgfpathcurveto{\pgfqpoint{-0.009821in}{-0.002605in}}{\pgfqpoint{-0.008786in}{-0.005103in}}{\pgfqpoint{-0.006944in}{-0.006944in}}%
\pgfpathcurveto{\pgfqpoint{-0.005103in}{-0.008786in}}{\pgfqpoint{-0.002605in}{-0.009821in}}{\pgfqpoint{0.000000in}{-0.009821in}}%
\pgfpathclose%
\pgfusepath{stroke,fill}%
}%
\begin{pgfscope}%
\pgfsys@transformshift{4.633823in}{3.763868in}%
\pgfsys@useobject{currentmarker}{}%
\end{pgfscope}%
\begin{pgfscope}%
\pgfsys@transformshift{4.784609in}{3.749270in}%
\pgfsys@useobject{currentmarker}{}%
\end{pgfscope}%
\begin{pgfscope}%
\pgfsys@transformshift{4.935395in}{3.732977in}%
\pgfsys@useobject{currentmarker}{}%
\end{pgfscope}%
\begin{pgfscope}%
\pgfsys@transformshift{5.086181in}{3.873622in}%
\pgfsys@useobject{currentmarker}{}%
\end{pgfscope}%
\begin{pgfscope}%
\pgfsys@transformshift{5.236967in}{3.788222in}%
\pgfsys@useobject{currentmarker}{}%
\end{pgfscope}%
\begin{pgfscope}%
\pgfsys@transformshift{5.387753in}{4.168671in}%
\pgfsys@useobject{currentmarker}{}%
\end{pgfscope}%
\begin{pgfscope}%
\pgfsys@transformshift{5.538539in}{4.120823in}%
\pgfsys@useobject{currentmarker}{}%
\end{pgfscope}%
\begin{pgfscope}%
\pgfsys@transformshift{5.689325in}{3.919974in}%
\pgfsys@useobject{currentmarker}{}%
\end{pgfscope}%
\begin{pgfscope}%
\pgfsys@transformshift{5.840111in}{4.305093in}%
\pgfsys@useobject{currentmarker}{}%
\end{pgfscope}%
\begin{pgfscope}%
\pgfsys@transformshift{5.990897in}{4.227048in}%
\pgfsys@useobject{currentmarker}{}%
\end{pgfscope}%
\begin{pgfscope}%
\pgfsys@transformshift{6.141683in}{4.686787in}%
\pgfsys@useobject{currentmarker}{}%
\end{pgfscope}%
\begin{pgfscope}%
\pgfsys@transformshift{6.292469in}{4.932556in}%
\pgfsys@useobject{currentmarker}{}%
\end{pgfscope}%
\begin{pgfscope}%
\pgfsys@transformshift{6.443255in}{4.935088in}%
\pgfsys@useobject{currentmarker}{}%
\end{pgfscope}%
\begin{pgfscope}%
\pgfsys@transformshift{6.594041in}{4.935533in}%
\pgfsys@useobject{currentmarker}{}%
\end{pgfscope}%
\begin{pgfscope}%
\pgfsys@transformshift{6.744827in}{4.932031in}%
\pgfsys@useobject{currentmarker}{}%
\end{pgfscope}%
\begin{pgfscope}%
\pgfsys@transformshift{6.895613in}{4.934299in}%
\pgfsys@useobject{currentmarker}{}%
\end{pgfscope}%
\begin{pgfscope}%
\pgfsys@transformshift{7.046399in}{4.931718in}%
\pgfsys@useobject{currentmarker}{}%
\end{pgfscope}%
\begin{pgfscope}%
\pgfsys@transformshift{7.197185in}{4.932086in}%
\pgfsys@useobject{currentmarker}{}%
\end{pgfscope}%
\begin{pgfscope}%
\pgfsys@transformshift{7.347971in}{4.932742in}%
\pgfsys@useobject{currentmarker}{}%
\end{pgfscope}%
\begin{pgfscope}%
\pgfsys@transformshift{7.498756in}{4.932514in}%
\pgfsys@useobject{currentmarker}{}%
\end{pgfscope}%
\begin{pgfscope}%
\pgfsys@transformshift{7.649542in}{4.932821in}%
\pgfsys@useobject{currentmarker}{}%
\end{pgfscope}%
\end{pgfscope}%
\begin{pgfscope}%
\pgfpathrectangle{\pgfqpoint{4.482986in}{3.363889in}}{\pgfqpoint{3.318403in}{2.263889in}}%
\pgfusepath{clip}%
\pgfsetbuttcap%
\pgfsetroundjoin%
\definecolor{currentfill}{rgb}{0.172549,0.627451,0.172549}%
\pgfsetfillcolor{currentfill}%
\pgfsetfillopacity{0.300000}%
\pgfsetlinewidth{1.003750pt}%
\definecolor{currentstroke}{rgb}{0.172549,0.627451,0.172549}%
\pgfsetstrokecolor{currentstroke}%
\pgfsetstrokeopacity{0.300000}%
\pgfsetdash{}{0pt}%
\pgfpathmoveto{\pgfqpoint{4.633823in}{3.820673in}}%
\pgfpathlineto{\pgfqpoint{4.633823in}{3.707064in}}%
\pgfpathlineto{\pgfqpoint{4.784609in}{3.705669in}}%
\pgfpathlineto{\pgfqpoint{4.935395in}{3.680250in}}%
\pgfpathlineto{\pgfqpoint{5.086181in}{3.722318in}}%
\pgfpathlineto{\pgfqpoint{5.236967in}{3.616272in}}%
\pgfpathlineto{\pgfqpoint{5.387753in}{3.590006in}}%
\pgfpathlineto{\pgfqpoint{5.538539in}{4.006717in}}%
\pgfpathlineto{\pgfqpoint{5.689325in}{3.884588in}}%
\pgfpathlineto{\pgfqpoint{5.840111in}{3.673993in}}%
\pgfpathlineto{\pgfqpoint{5.990897in}{3.662523in}}%
\pgfpathlineto{\pgfqpoint{6.141683in}{4.291937in}}%
\pgfpathlineto{\pgfqpoint{6.292469in}{4.925906in}}%
\pgfpathlineto{\pgfqpoint{6.443255in}{4.929544in}}%
\pgfpathlineto{\pgfqpoint{6.594041in}{4.932143in}}%
\pgfpathlineto{\pgfqpoint{6.744827in}{4.927476in}}%
\pgfpathlineto{\pgfqpoint{6.895613in}{4.930481in}}%
\pgfpathlineto{\pgfqpoint{7.046399in}{4.929544in}}%
\pgfpathlineto{\pgfqpoint{7.197185in}{4.930247in}}%
\pgfpathlineto{\pgfqpoint{7.347971in}{4.932033in}}%
\pgfpathlineto{\pgfqpoint{7.498756in}{4.931961in}}%
\pgfpathlineto{\pgfqpoint{7.649542in}{4.932421in}}%
\pgfpathlineto{\pgfqpoint{7.649542in}{4.933221in}}%
\pgfpathlineto{\pgfqpoint{7.649542in}{4.933221in}}%
\pgfpathlineto{\pgfqpoint{7.498756in}{4.933068in}}%
\pgfpathlineto{\pgfqpoint{7.347971in}{4.933451in}}%
\pgfpathlineto{\pgfqpoint{7.197185in}{4.933925in}}%
\pgfpathlineto{\pgfqpoint{7.046399in}{4.933893in}}%
\pgfpathlineto{\pgfqpoint{6.895613in}{4.938118in}}%
\pgfpathlineto{\pgfqpoint{6.744827in}{4.936587in}}%
\pgfpathlineto{\pgfqpoint{6.594041in}{4.938924in}}%
\pgfpathlineto{\pgfqpoint{6.443255in}{4.940633in}}%
\pgfpathlineto{\pgfqpoint{6.292469in}{4.939207in}}%
\pgfpathlineto{\pgfqpoint{6.141683in}{5.081638in}}%
\pgfpathlineto{\pgfqpoint{5.990897in}{4.791573in}}%
\pgfpathlineto{\pgfqpoint{5.840111in}{4.936192in}}%
\pgfpathlineto{\pgfqpoint{5.689325in}{3.955359in}}%
\pgfpathlineto{\pgfqpoint{5.538539in}{4.234929in}}%
\pgfpathlineto{\pgfqpoint{5.387753in}{4.747335in}}%
\pgfpathlineto{\pgfqpoint{5.236967in}{3.960172in}}%
\pgfpathlineto{\pgfqpoint{5.086181in}{4.024926in}}%
\pgfpathlineto{\pgfqpoint{4.935395in}{3.785704in}}%
\pgfpathlineto{\pgfqpoint{4.784609in}{3.792871in}}%
\pgfpathlineto{\pgfqpoint{4.633823in}{3.820673in}}%
\pgfpathclose%
\pgfusepath{stroke,fill}%
\end{pgfscope}%
\begin{pgfscope}%
\pgfpathrectangle{\pgfqpoint{4.482986in}{3.363889in}}{\pgfqpoint{3.318403in}{2.263889in}}%
\pgfusepath{clip}%
\pgfsetbuttcap%
\pgfsetroundjoin%
\definecolor{currentfill}{rgb}{0.839216,0.152941,0.156863}%
\pgfsetfillcolor{currentfill}%
\pgfsetlinewidth{1.003750pt}%
\definecolor{currentstroke}{rgb}{0.839216,0.152941,0.156863}%
\pgfsetstrokecolor{currentstroke}%
\pgfsetdash{}{0pt}%
\pgfsys@defobject{currentmarker}{\pgfqpoint{-0.009821in}{-0.009821in}}{\pgfqpoint{0.009821in}{0.009821in}}{%
\pgfpathmoveto{\pgfqpoint{0.000000in}{-0.009821in}}%
\pgfpathcurveto{\pgfqpoint{0.002605in}{-0.009821in}}{\pgfqpoint{0.005103in}{-0.008786in}}{\pgfqpoint{0.006944in}{-0.006944in}}%
\pgfpathcurveto{\pgfqpoint{0.008786in}{-0.005103in}}{\pgfqpoint{0.009821in}{-0.002605in}}{\pgfqpoint{0.009821in}{0.000000in}}%
\pgfpathcurveto{\pgfqpoint{0.009821in}{0.002605in}}{\pgfqpoint{0.008786in}{0.005103in}}{\pgfqpoint{0.006944in}{0.006944in}}%
\pgfpathcurveto{\pgfqpoint{0.005103in}{0.008786in}}{\pgfqpoint{0.002605in}{0.009821in}}{\pgfqpoint{0.000000in}{0.009821in}}%
\pgfpathcurveto{\pgfqpoint{-0.002605in}{0.009821in}}{\pgfqpoint{-0.005103in}{0.008786in}}{\pgfqpoint{-0.006944in}{0.006944in}}%
\pgfpathcurveto{\pgfqpoint{-0.008786in}{0.005103in}}{\pgfqpoint{-0.009821in}{0.002605in}}{\pgfqpoint{-0.009821in}{0.000000in}}%
\pgfpathcurveto{\pgfqpoint{-0.009821in}{-0.002605in}}{\pgfqpoint{-0.008786in}{-0.005103in}}{\pgfqpoint{-0.006944in}{-0.006944in}}%
\pgfpathcurveto{\pgfqpoint{-0.005103in}{-0.008786in}}{\pgfqpoint{-0.002605in}{-0.009821in}}{\pgfqpoint{0.000000in}{-0.009821in}}%
\pgfpathclose%
\pgfusepath{stroke,fill}%
}%
\begin{pgfscope}%
\pgfsys@transformshift{4.633823in}{3.719885in}%
\pgfsys@useobject{currentmarker}{}%
\end{pgfscope}%
\begin{pgfscope}%
\pgfsys@transformshift{4.784609in}{3.766371in}%
\pgfsys@useobject{currentmarker}{}%
\end{pgfscope}%
\begin{pgfscope}%
\pgfsys@transformshift{4.935395in}{3.715219in}%
\pgfsys@useobject{currentmarker}{}%
\end{pgfscope}%
\begin{pgfscope}%
\pgfsys@transformshift{5.086181in}{3.801762in}%
\pgfsys@useobject{currentmarker}{}%
\end{pgfscope}%
\begin{pgfscope}%
\pgfsys@transformshift{5.236967in}{3.821271in}%
\pgfsys@useobject{currentmarker}{}%
\end{pgfscope}%
\begin{pgfscope}%
\pgfsys@transformshift{5.387753in}{3.886073in}%
\pgfsys@useobject{currentmarker}{}%
\end{pgfscope}%
\begin{pgfscope}%
\pgfsys@transformshift{5.538539in}{3.798609in}%
\pgfsys@useobject{currentmarker}{}%
\end{pgfscope}%
\begin{pgfscope}%
\pgfsys@transformshift{5.689325in}{3.894287in}%
\pgfsys@useobject{currentmarker}{}%
\end{pgfscope}%
\begin{pgfscope}%
\pgfsys@transformshift{5.840111in}{3.929800in}%
\pgfsys@useobject{currentmarker}{}%
\end{pgfscope}%
\begin{pgfscope}%
\pgfsys@transformshift{5.990897in}{4.006194in}%
\pgfsys@useobject{currentmarker}{}%
\end{pgfscope}%
\begin{pgfscope}%
\pgfsys@transformshift{6.141683in}{3.951557in}%
\pgfsys@useobject{currentmarker}{}%
\end{pgfscope}%
\begin{pgfscope}%
\pgfsys@transformshift{6.292469in}{3.943651in}%
\pgfsys@useobject{currentmarker}{}%
\end{pgfscope}%
\begin{pgfscope}%
\pgfsys@transformshift{6.443255in}{3.952953in}%
\pgfsys@useobject{currentmarker}{}%
\end{pgfscope}%
\begin{pgfscope}%
\pgfsys@transformshift{6.594041in}{3.926751in}%
\pgfsys@useobject{currentmarker}{}%
\end{pgfscope}%
\begin{pgfscope}%
\pgfsys@transformshift{6.744827in}{3.925015in}%
\pgfsys@useobject{currentmarker}{}%
\end{pgfscope}%
\begin{pgfscope}%
\pgfsys@transformshift{6.895613in}{3.935325in}%
\pgfsys@useobject{currentmarker}{}%
\end{pgfscope}%
\begin{pgfscope}%
\pgfsys@transformshift{7.046399in}{3.945034in}%
\pgfsys@useobject{currentmarker}{}%
\end{pgfscope}%
\begin{pgfscope}%
\pgfsys@transformshift{7.197185in}{3.938416in}%
\pgfsys@useobject{currentmarker}{}%
\end{pgfscope}%
\begin{pgfscope}%
\pgfsys@transformshift{7.347971in}{3.944363in}%
\pgfsys@useobject{currentmarker}{}%
\end{pgfscope}%
\begin{pgfscope}%
\pgfsys@transformshift{7.498756in}{3.941043in}%
\pgfsys@useobject{currentmarker}{}%
\end{pgfscope}%
\begin{pgfscope}%
\pgfsys@transformshift{7.649542in}{3.939452in}%
\pgfsys@useobject{currentmarker}{}%
\end{pgfscope}%
\end{pgfscope}%
\begin{pgfscope}%
\pgfpathrectangle{\pgfqpoint{4.482986in}{3.363889in}}{\pgfqpoint{3.318403in}{2.263889in}}%
\pgfusepath{clip}%
\pgfsetbuttcap%
\pgfsetroundjoin%
\definecolor{currentfill}{rgb}{0.839216,0.152941,0.156863}%
\pgfsetfillcolor{currentfill}%
\pgfsetfillopacity{0.300000}%
\pgfsetlinewidth{1.003750pt}%
\definecolor{currentstroke}{rgb}{0.839216,0.152941,0.156863}%
\pgfsetstrokecolor{currentstroke}%
\pgfsetstrokeopacity{0.300000}%
\pgfsetdash{}{0pt}%
\pgfpathmoveto{\pgfqpoint{4.633823in}{3.789954in}}%
\pgfpathlineto{\pgfqpoint{4.633823in}{3.649816in}}%
\pgfpathlineto{\pgfqpoint{4.784609in}{3.694353in}}%
\pgfpathlineto{\pgfqpoint{4.935395in}{3.695138in}}%
\pgfpathlineto{\pgfqpoint{5.086181in}{3.735947in}}%
\pgfpathlineto{\pgfqpoint{5.236967in}{3.691952in}}%
\pgfpathlineto{\pgfqpoint{5.387753in}{3.711730in}}%
\pgfpathlineto{\pgfqpoint{5.538539in}{3.679754in}}%
\pgfpathlineto{\pgfqpoint{5.689325in}{3.802766in}}%
\pgfpathlineto{\pgfqpoint{5.840111in}{3.765984in}}%
\pgfpathlineto{\pgfqpoint{5.990897in}{3.799425in}}%
\pgfpathlineto{\pgfqpoint{6.141683in}{3.885856in}}%
\pgfpathlineto{\pgfqpoint{6.292469in}{3.854209in}}%
\pgfpathlineto{\pgfqpoint{6.443255in}{3.832996in}}%
\pgfpathlineto{\pgfqpoint{6.594041in}{3.886566in}}%
\pgfpathlineto{\pgfqpoint{6.744827in}{3.881565in}}%
\pgfpathlineto{\pgfqpoint{6.895613in}{3.930521in}}%
\pgfpathlineto{\pgfqpoint{7.046399in}{3.934403in}}%
\pgfpathlineto{\pgfqpoint{7.197185in}{3.920798in}}%
\pgfpathlineto{\pgfqpoint{7.347971in}{3.936026in}}%
\pgfpathlineto{\pgfqpoint{7.498756in}{3.935992in}}%
\pgfpathlineto{\pgfqpoint{7.649542in}{3.936473in}}%
\pgfpathlineto{\pgfqpoint{7.649542in}{3.942430in}}%
\pgfpathlineto{\pgfqpoint{7.649542in}{3.942430in}}%
\pgfpathlineto{\pgfqpoint{7.498756in}{3.946094in}}%
\pgfpathlineto{\pgfqpoint{7.347971in}{3.952701in}}%
\pgfpathlineto{\pgfqpoint{7.197185in}{3.956034in}}%
\pgfpathlineto{\pgfqpoint{7.046399in}{3.955665in}}%
\pgfpathlineto{\pgfqpoint{6.895613in}{3.940128in}}%
\pgfpathlineto{\pgfqpoint{6.744827in}{3.968466in}}%
\pgfpathlineto{\pgfqpoint{6.594041in}{3.966936in}}%
\pgfpathlineto{\pgfqpoint{6.443255in}{4.072909in}}%
\pgfpathlineto{\pgfqpoint{6.292469in}{4.033094in}}%
\pgfpathlineto{\pgfqpoint{6.141683in}{4.017257in}}%
\pgfpathlineto{\pgfqpoint{5.990897in}{4.212963in}}%
\pgfpathlineto{\pgfqpoint{5.840111in}{4.093616in}}%
\pgfpathlineto{\pgfqpoint{5.689325in}{3.985809in}}%
\pgfpathlineto{\pgfqpoint{5.538539in}{3.917465in}}%
\pgfpathlineto{\pgfqpoint{5.387753in}{4.060416in}}%
\pgfpathlineto{\pgfqpoint{5.236967in}{3.950589in}}%
\pgfpathlineto{\pgfqpoint{5.086181in}{3.867578in}}%
\pgfpathlineto{\pgfqpoint{4.935395in}{3.735299in}}%
\pgfpathlineto{\pgfqpoint{4.784609in}{3.838390in}}%
\pgfpathlineto{\pgfqpoint{4.633823in}{3.789954in}}%
\pgfpathclose%
\pgfusepath{stroke,fill}%
\end{pgfscope}%
\begin{pgfscope}%
\pgfpathrectangle{\pgfqpoint{4.482986in}{3.363889in}}{\pgfqpoint{3.318403in}{2.263889in}}%
\pgfusepath{clip}%
\pgfsetbuttcap%
\pgfsetroundjoin%
\definecolor{currentfill}{rgb}{0.580392,0.403922,0.741176}%
\pgfsetfillcolor{currentfill}%
\pgfsetlinewidth{1.003750pt}%
\definecolor{currentstroke}{rgb}{0.580392,0.403922,0.741176}%
\pgfsetstrokecolor{currentstroke}%
\pgfsetdash{}{0pt}%
\pgfsys@defobject{currentmarker}{\pgfqpoint{-0.009821in}{-0.009821in}}{\pgfqpoint{0.009821in}{0.009821in}}{%
\pgfpathmoveto{\pgfqpoint{0.000000in}{-0.009821in}}%
\pgfpathcurveto{\pgfqpoint{0.002605in}{-0.009821in}}{\pgfqpoint{0.005103in}{-0.008786in}}{\pgfqpoint{0.006944in}{-0.006944in}}%
\pgfpathcurveto{\pgfqpoint{0.008786in}{-0.005103in}}{\pgfqpoint{0.009821in}{-0.002605in}}{\pgfqpoint{0.009821in}{0.000000in}}%
\pgfpathcurveto{\pgfqpoint{0.009821in}{0.002605in}}{\pgfqpoint{0.008786in}{0.005103in}}{\pgfqpoint{0.006944in}{0.006944in}}%
\pgfpathcurveto{\pgfqpoint{0.005103in}{0.008786in}}{\pgfqpoint{0.002605in}{0.009821in}}{\pgfqpoint{0.000000in}{0.009821in}}%
\pgfpathcurveto{\pgfqpoint{-0.002605in}{0.009821in}}{\pgfqpoint{-0.005103in}{0.008786in}}{\pgfqpoint{-0.006944in}{0.006944in}}%
\pgfpathcurveto{\pgfqpoint{-0.008786in}{0.005103in}}{\pgfqpoint{-0.009821in}{0.002605in}}{\pgfqpoint{-0.009821in}{0.000000in}}%
\pgfpathcurveto{\pgfqpoint{-0.009821in}{-0.002605in}}{\pgfqpoint{-0.008786in}{-0.005103in}}{\pgfqpoint{-0.006944in}{-0.006944in}}%
\pgfpathcurveto{\pgfqpoint{-0.005103in}{-0.008786in}}{\pgfqpoint{-0.002605in}{-0.009821in}}{\pgfqpoint{0.000000in}{-0.009821in}}%
\pgfpathclose%
\pgfusepath{stroke,fill}%
}%
\begin{pgfscope}%
\pgfsys@transformshift{4.633823in}{3.721978in}%
\pgfsys@useobject{currentmarker}{}%
\end{pgfscope}%
\begin{pgfscope}%
\pgfsys@transformshift{4.784609in}{3.741195in}%
\pgfsys@useobject{currentmarker}{}%
\end{pgfscope}%
\begin{pgfscope}%
\pgfsys@transformshift{4.935395in}{3.806489in}%
\pgfsys@useobject{currentmarker}{}%
\end{pgfscope}%
\begin{pgfscope}%
\pgfsys@transformshift{5.086181in}{3.837762in}%
\pgfsys@useobject{currentmarker}{}%
\end{pgfscope}%
\begin{pgfscope}%
\pgfsys@transformshift{5.236967in}{3.749054in}%
\pgfsys@useobject{currentmarker}{}%
\end{pgfscope}%
\begin{pgfscope}%
\pgfsys@transformshift{5.387753in}{3.771638in}%
\pgfsys@useobject{currentmarker}{}%
\end{pgfscope}%
\begin{pgfscope}%
\pgfsys@transformshift{5.538539in}{3.793508in}%
\pgfsys@useobject{currentmarker}{}%
\end{pgfscope}%
\begin{pgfscope}%
\pgfsys@transformshift{5.689325in}{3.782050in}%
\pgfsys@useobject{currentmarker}{}%
\end{pgfscope}%
\begin{pgfscope}%
\pgfsys@transformshift{5.840111in}{3.770886in}%
\pgfsys@useobject{currentmarker}{}%
\end{pgfscope}%
\begin{pgfscope}%
\pgfsys@transformshift{5.990897in}{3.767747in}%
\pgfsys@useobject{currentmarker}{}%
\end{pgfscope}%
\begin{pgfscope}%
\pgfsys@transformshift{6.141683in}{3.771821in}%
\pgfsys@useobject{currentmarker}{}%
\end{pgfscope}%
\begin{pgfscope}%
\pgfsys@transformshift{6.292469in}{3.776206in}%
\pgfsys@useobject{currentmarker}{}%
\end{pgfscope}%
\begin{pgfscope}%
\pgfsys@transformshift{6.443255in}{3.768919in}%
\pgfsys@useobject{currentmarker}{}%
\end{pgfscope}%
\begin{pgfscope}%
\pgfsys@transformshift{6.594041in}{3.772142in}%
\pgfsys@useobject{currentmarker}{}%
\end{pgfscope}%
\begin{pgfscope}%
\pgfsys@transformshift{6.744827in}{3.777330in}%
\pgfsys@useobject{currentmarker}{}%
\end{pgfscope}%
\begin{pgfscope}%
\pgfsys@transformshift{6.895613in}{3.775099in}%
\pgfsys@useobject{currentmarker}{}%
\end{pgfscope}%
\begin{pgfscope}%
\pgfsys@transformshift{7.046399in}{3.777212in}%
\pgfsys@useobject{currentmarker}{}%
\end{pgfscope}%
\begin{pgfscope}%
\pgfsys@transformshift{7.197185in}{3.776424in}%
\pgfsys@useobject{currentmarker}{}%
\end{pgfscope}%
\begin{pgfscope}%
\pgfsys@transformshift{7.347971in}{3.775934in}%
\pgfsys@useobject{currentmarker}{}%
\end{pgfscope}%
\begin{pgfscope}%
\pgfsys@transformshift{7.498756in}{3.776603in}%
\pgfsys@useobject{currentmarker}{}%
\end{pgfscope}%
\begin{pgfscope}%
\pgfsys@transformshift{7.649542in}{3.775983in}%
\pgfsys@useobject{currentmarker}{}%
\end{pgfscope}%
\end{pgfscope}%
\begin{pgfscope}%
\pgfpathrectangle{\pgfqpoint{4.482986in}{3.363889in}}{\pgfqpoint{3.318403in}{2.263889in}}%
\pgfusepath{clip}%
\pgfsetbuttcap%
\pgfsetroundjoin%
\definecolor{currentfill}{rgb}{0.580392,0.403922,0.741176}%
\pgfsetfillcolor{currentfill}%
\pgfsetfillopacity{0.300000}%
\pgfsetlinewidth{1.003750pt}%
\definecolor{currentstroke}{rgb}{0.580392,0.403922,0.741176}%
\pgfsetstrokecolor{currentstroke}%
\pgfsetstrokeopacity{0.300000}%
\pgfsetdash{}{0pt}%
\pgfpathmoveto{\pgfqpoint{4.633823in}{3.756832in}}%
\pgfpathlineto{\pgfqpoint{4.633823in}{3.687125in}}%
\pgfpathlineto{\pgfqpoint{4.784609in}{3.691179in}}%
\pgfpathlineto{\pgfqpoint{4.935395in}{3.710677in}}%
\pgfpathlineto{\pgfqpoint{5.086181in}{3.730258in}}%
\pgfpathlineto{\pgfqpoint{5.236967in}{3.677485in}}%
\pgfpathlineto{\pgfqpoint{5.387753in}{3.670862in}}%
\pgfpathlineto{\pgfqpoint{5.538539in}{3.717786in}}%
\pgfpathlineto{\pgfqpoint{5.689325in}{3.770289in}}%
\pgfpathlineto{\pgfqpoint{5.840111in}{3.730749in}}%
\pgfpathlineto{\pgfqpoint{5.990897in}{3.739977in}}%
\pgfpathlineto{\pgfqpoint{6.141683in}{3.757447in}}%
\pgfpathlineto{\pgfqpoint{6.292469in}{3.764237in}}%
\pgfpathlineto{\pgfqpoint{6.443255in}{3.758442in}}%
\pgfpathlineto{\pgfqpoint{6.594041in}{3.767200in}}%
\pgfpathlineto{\pgfqpoint{6.744827in}{3.771578in}}%
\pgfpathlineto{\pgfqpoint{6.895613in}{3.772848in}}%
\pgfpathlineto{\pgfqpoint{7.046399in}{3.773886in}}%
\pgfpathlineto{\pgfqpoint{7.197185in}{3.775288in}}%
\pgfpathlineto{\pgfqpoint{7.347971in}{3.775180in}}%
\pgfpathlineto{\pgfqpoint{7.498756in}{3.775434in}}%
\pgfpathlineto{\pgfqpoint{7.649542in}{3.774874in}}%
\pgfpathlineto{\pgfqpoint{7.649542in}{3.777091in}}%
\pgfpathlineto{\pgfqpoint{7.649542in}{3.777091in}}%
\pgfpathlineto{\pgfqpoint{7.498756in}{3.777773in}}%
\pgfpathlineto{\pgfqpoint{7.347971in}{3.776689in}}%
\pgfpathlineto{\pgfqpoint{7.197185in}{3.777559in}}%
\pgfpathlineto{\pgfqpoint{7.046399in}{3.780537in}}%
\pgfpathlineto{\pgfqpoint{6.895613in}{3.777350in}}%
\pgfpathlineto{\pgfqpoint{6.744827in}{3.783082in}}%
\pgfpathlineto{\pgfqpoint{6.594041in}{3.777085in}}%
\pgfpathlineto{\pgfqpoint{6.443255in}{3.779397in}}%
\pgfpathlineto{\pgfqpoint{6.292469in}{3.788175in}}%
\pgfpathlineto{\pgfqpoint{6.141683in}{3.786195in}}%
\pgfpathlineto{\pgfqpoint{5.990897in}{3.795517in}}%
\pgfpathlineto{\pgfqpoint{5.840111in}{3.811023in}}%
\pgfpathlineto{\pgfqpoint{5.689325in}{3.793810in}}%
\pgfpathlineto{\pgfqpoint{5.538539in}{3.869230in}}%
\pgfpathlineto{\pgfqpoint{5.387753in}{3.872415in}}%
\pgfpathlineto{\pgfqpoint{5.236967in}{3.820622in}}%
\pgfpathlineto{\pgfqpoint{5.086181in}{3.945266in}}%
\pgfpathlineto{\pgfqpoint{4.935395in}{3.902302in}}%
\pgfpathlineto{\pgfqpoint{4.784609in}{3.791210in}}%
\pgfpathlineto{\pgfqpoint{4.633823in}{3.756832in}}%
\pgfpathclose%
\pgfusepath{stroke,fill}%
\end{pgfscope}%
\begin{pgfscope}%
\pgfsetbuttcap%
\pgfsetroundjoin%
\definecolor{currentfill}{rgb}{0.000000,0.000000,0.000000}%
\pgfsetfillcolor{currentfill}%
\pgfsetlinewidth{0.803000pt}%
\definecolor{currentstroke}{rgb}{0.000000,0.000000,0.000000}%
\pgfsetstrokecolor{currentstroke}%
\pgfsetdash{}{0pt}%
\pgfsys@defobject{currentmarker}{\pgfqpoint{0.000000in}{-0.048611in}}{\pgfqpoint{0.000000in}{0.000000in}}{%
\pgfpathmoveto{\pgfqpoint{0.000000in}{0.000000in}}%
\pgfpathlineto{\pgfqpoint{0.000000in}{-0.048611in}}%
\pgfusepath{stroke,fill}%
}%
\begin{pgfscope}%
\pgfsys@transformshift{4.996951in}{3.363889in}%
\pgfsys@useobject{currentmarker}{}%
\end{pgfscope}%
\end{pgfscope}%
\begin{pgfscope}%
\definecolor{textcolor}{rgb}{0.000000,0.000000,0.000000}%
\pgfsetstrokecolor{textcolor}%
\pgfsetfillcolor{textcolor}%
\pgftext[x=4.996951in,y=3.266667in,,top]{\color{textcolor}\sffamily\fontsize{10.000000}{12.000000}\selectfont \(\displaystyle {10^{4}}\)}%
\end{pgfscope}%
\begin{pgfscope}%
\pgfsetbuttcap%
\pgfsetroundjoin%
\definecolor{currentfill}{rgb}{0.000000,0.000000,0.000000}%
\pgfsetfillcolor{currentfill}%
\pgfsetlinewidth{0.803000pt}%
\definecolor{currentstroke}{rgb}{0.000000,0.000000,0.000000}%
\pgfsetstrokecolor{currentstroke}%
\pgfsetdash{}{0pt}%
\pgfsys@defobject{currentmarker}{\pgfqpoint{0.000000in}{-0.048611in}}{\pgfqpoint{0.000000in}{0.000000in}}{%
\pgfpathmoveto{\pgfqpoint{0.000000in}{0.000000in}}%
\pgfpathlineto{\pgfqpoint{0.000000in}{-0.048611in}}%
\pgfusepath{stroke,fill}%
}%
\begin{pgfscope}%
\pgfsys@transformshift{5.600095in}{3.363889in}%
\pgfsys@useobject{currentmarker}{}%
\end{pgfscope}%
\end{pgfscope}%
\begin{pgfscope}%
\definecolor{textcolor}{rgb}{0.000000,0.000000,0.000000}%
\pgfsetstrokecolor{textcolor}%
\pgfsetfillcolor{textcolor}%
\pgftext[x=5.600095in,y=3.266667in,,top]{\color{textcolor}\sffamily\fontsize{10.000000}{12.000000}\selectfont \(\displaystyle {10^{5}}\)}%
\end{pgfscope}%
\begin{pgfscope}%
\pgfsetbuttcap%
\pgfsetroundjoin%
\definecolor{currentfill}{rgb}{0.000000,0.000000,0.000000}%
\pgfsetfillcolor{currentfill}%
\pgfsetlinewidth{0.803000pt}%
\definecolor{currentstroke}{rgb}{0.000000,0.000000,0.000000}%
\pgfsetstrokecolor{currentstroke}%
\pgfsetdash{}{0pt}%
\pgfsys@defobject{currentmarker}{\pgfqpoint{0.000000in}{-0.048611in}}{\pgfqpoint{0.000000in}{0.000000in}}{%
\pgfpathmoveto{\pgfqpoint{0.000000in}{0.000000in}}%
\pgfpathlineto{\pgfqpoint{0.000000in}{-0.048611in}}%
\pgfusepath{stroke,fill}%
}%
\begin{pgfscope}%
\pgfsys@transformshift{6.203239in}{3.363889in}%
\pgfsys@useobject{currentmarker}{}%
\end{pgfscope}%
\end{pgfscope}%
\begin{pgfscope}%
\definecolor{textcolor}{rgb}{0.000000,0.000000,0.000000}%
\pgfsetstrokecolor{textcolor}%
\pgfsetfillcolor{textcolor}%
\pgftext[x=6.203239in,y=3.266667in,,top]{\color{textcolor}\sffamily\fontsize{10.000000}{12.000000}\selectfont \(\displaystyle {10^{6}}\)}%
\end{pgfscope}%
\begin{pgfscope}%
\pgfsetbuttcap%
\pgfsetroundjoin%
\definecolor{currentfill}{rgb}{0.000000,0.000000,0.000000}%
\pgfsetfillcolor{currentfill}%
\pgfsetlinewidth{0.803000pt}%
\definecolor{currentstroke}{rgb}{0.000000,0.000000,0.000000}%
\pgfsetstrokecolor{currentstroke}%
\pgfsetdash{}{0pt}%
\pgfsys@defobject{currentmarker}{\pgfqpoint{0.000000in}{-0.048611in}}{\pgfqpoint{0.000000in}{0.000000in}}{%
\pgfpathmoveto{\pgfqpoint{0.000000in}{0.000000in}}%
\pgfpathlineto{\pgfqpoint{0.000000in}{-0.048611in}}%
\pgfusepath{stroke,fill}%
}%
\begin{pgfscope}%
\pgfsys@transformshift{6.806383in}{3.363889in}%
\pgfsys@useobject{currentmarker}{}%
\end{pgfscope}%
\end{pgfscope}%
\begin{pgfscope}%
\definecolor{textcolor}{rgb}{0.000000,0.000000,0.000000}%
\pgfsetstrokecolor{textcolor}%
\pgfsetfillcolor{textcolor}%
\pgftext[x=6.806383in,y=3.266667in,,top]{\color{textcolor}\sffamily\fontsize{10.000000}{12.000000}\selectfont \(\displaystyle {10^{7}}\)}%
\end{pgfscope}%
\begin{pgfscope}%
\pgfsetbuttcap%
\pgfsetroundjoin%
\definecolor{currentfill}{rgb}{0.000000,0.000000,0.000000}%
\pgfsetfillcolor{currentfill}%
\pgfsetlinewidth{0.803000pt}%
\definecolor{currentstroke}{rgb}{0.000000,0.000000,0.000000}%
\pgfsetstrokecolor{currentstroke}%
\pgfsetdash{}{0pt}%
\pgfsys@defobject{currentmarker}{\pgfqpoint{0.000000in}{-0.048611in}}{\pgfqpoint{0.000000in}{0.000000in}}{%
\pgfpathmoveto{\pgfqpoint{0.000000in}{0.000000in}}%
\pgfpathlineto{\pgfqpoint{0.000000in}{-0.048611in}}%
\pgfusepath{stroke,fill}%
}%
\begin{pgfscope}%
\pgfsys@transformshift{7.409527in}{3.363889in}%
\pgfsys@useobject{currentmarker}{}%
\end{pgfscope}%
\end{pgfscope}%
\begin{pgfscope}%
\definecolor{textcolor}{rgb}{0.000000,0.000000,0.000000}%
\pgfsetstrokecolor{textcolor}%
\pgfsetfillcolor{textcolor}%
\pgftext[x=7.409527in,y=3.266667in,,top]{\color{textcolor}\sffamily\fontsize{10.000000}{12.000000}\selectfont \(\displaystyle {10^{8}}\)}%
\end{pgfscope}%
\begin{pgfscope}%
\pgfsetbuttcap%
\pgfsetroundjoin%
\definecolor{currentfill}{rgb}{0.000000,0.000000,0.000000}%
\pgfsetfillcolor{currentfill}%
\pgfsetlinewidth{0.602250pt}%
\definecolor{currentstroke}{rgb}{0.000000,0.000000,0.000000}%
\pgfsetstrokecolor{currentstroke}%
\pgfsetdash{}{0pt}%
\pgfsys@defobject{currentmarker}{\pgfqpoint{0.000000in}{-0.027778in}}{\pgfqpoint{0.000000in}{0.000000in}}{%
\pgfpathmoveto{\pgfqpoint{0.000000in}{0.000000in}}%
\pgfpathlineto{\pgfqpoint{0.000000in}{-0.027778in}}%
\pgfusepath{stroke,fill}%
}%
\begin{pgfscope}%
\pgfsys@transformshift{4.575372in}{3.363889in}%
\pgfsys@useobject{currentmarker}{}%
\end{pgfscope}%
\end{pgfscope}%
\begin{pgfscope}%
\pgfsetbuttcap%
\pgfsetroundjoin%
\definecolor{currentfill}{rgb}{0.000000,0.000000,0.000000}%
\pgfsetfillcolor{currentfill}%
\pgfsetlinewidth{0.602250pt}%
\definecolor{currentstroke}{rgb}{0.000000,0.000000,0.000000}%
\pgfsetstrokecolor{currentstroke}%
\pgfsetdash{}{0pt}%
\pgfsys@defobject{currentmarker}{\pgfqpoint{0.000000in}{-0.027778in}}{\pgfqpoint{0.000000in}{0.000000in}}{%
\pgfpathmoveto{\pgfqpoint{0.000000in}{0.000000in}}%
\pgfpathlineto{\pgfqpoint{0.000000in}{-0.027778in}}%
\pgfusepath{stroke,fill}%
}%
\begin{pgfscope}%
\pgfsys@transformshift{4.681580in}{3.363889in}%
\pgfsys@useobject{currentmarker}{}%
\end{pgfscope}%
\end{pgfscope}%
\begin{pgfscope}%
\pgfsetbuttcap%
\pgfsetroundjoin%
\definecolor{currentfill}{rgb}{0.000000,0.000000,0.000000}%
\pgfsetfillcolor{currentfill}%
\pgfsetlinewidth{0.602250pt}%
\definecolor{currentstroke}{rgb}{0.000000,0.000000,0.000000}%
\pgfsetstrokecolor{currentstroke}%
\pgfsetdash{}{0pt}%
\pgfsys@defobject{currentmarker}{\pgfqpoint{0.000000in}{-0.027778in}}{\pgfqpoint{0.000000in}{0.000000in}}{%
\pgfpathmoveto{\pgfqpoint{0.000000in}{0.000000in}}%
\pgfpathlineto{\pgfqpoint{0.000000in}{-0.027778in}}%
\pgfusepath{stroke,fill}%
}%
\begin{pgfscope}%
\pgfsys@transformshift{4.756936in}{3.363889in}%
\pgfsys@useobject{currentmarker}{}%
\end{pgfscope}%
\end{pgfscope}%
\begin{pgfscope}%
\pgfsetbuttcap%
\pgfsetroundjoin%
\definecolor{currentfill}{rgb}{0.000000,0.000000,0.000000}%
\pgfsetfillcolor{currentfill}%
\pgfsetlinewidth{0.602250pt}%
\definecolor{currentstroke}{rgb}{0.000000,0.000000,0.000000}%
\pgfsetstrokecolor{currentstroke}%
\pgfsetdash{}{0pt}%
\pgfsys@defobject{currentmarker}{\pgfqpoint{0.000000in}{-0.027778in}}{\pgfqpoint{0.000000in}{0.000000in}}{%
\pgfpathmoveto{\pgfqpoint{0.000000in}{0.000000in}}%
\pgfpathlineto{\pgfqpoint{0.000000in}{-0.027778in}}%
\pgfusepath{stroke,fill}%
}%
\begin{pgfscope}%
\pgfsys@transformshift{4.815387in}{3.363889in}%
\pgfsys@useobject{currentmarker}{}%
\end{pgfscope}%
\end{pgfscope}%
\begin{pgfscope}%
\pgfsetbuttcap%
\pgfsetroundjoin%
\definecolor{currentfill}{rgb}{0.000000,0.000000,0.000000}%
\pgfsetfillcolor{currentfill}%
\pgfsetlinewidth{0.602250pt}%
\definecolor{currentstroke}{rgb}{0.000000,0.000000,0.000000}%
\pgfsetstrokecolor{currentstroke}%
\pgfsetdash{}{0pt}%
\pgfsys@defobject{currentmarker}{\pgfqpoint{0.000000in}{-0.027778in}}{\pgfqpoint{0.000000in}{0.000000in}}{%
\pgfpathmoveto{\pgfqpoint{0.000000in}{0.000000in}}%
\pgfpathlineto{\pgfqpoint{0.000000in}{-0.027778in}}%
\pgfusepath{stroke,fill}%
}%
\begin{pgfscope}%
\pgfsys@transformshift{4.863145in}{3.363889in}%
\pgfsys@useobject{currentmarker}{}%
\end{pgfscope}%
\end{pgfscope}%
\begin{pgfscope}%
\pgfsetbuttcap%
\pgfsetroundjoin%
\definecolor{currentfill}{rgb}{0.000000,0.000000,0.000000}%
\pgfsetfillcolor{currentfill}%
\pgfsetlinewidth{0.602250pt}%
\definecolor{currentstroke}{rgb}{0.000000,0.000000,0.000000}%
\pgfsetstrokecolor{currentstroke}%
\pgfsetdash{}{0pt}%
\pgfsys@defobject{currentmarker}{\pgfqpoint{0.000000in}{-0.027778in}}{\pgfqpoint{0.000000in}{0.000000in}}{%
\pgfpathmoveto{\pgfqpoint{0.000000in}{0.000000in}}%
\pgfpathlineto{\pgfqpoint{0.000000in}{-0.027778in}}%
\pgfusepath{stroke,fill}%
}%
\begin{pgfscope}%
\pgfsys@transformshift{4.903523in}{3.363889in}%
\pgfsys@useobject{currentmarker}{}%
\end{pgfscope}%
\end{pgfscope}%
\begin{pgfscope}%
\pgfsetbuttcap%
\pgfsetroundjoin%
\definecolor{currentfill}{rgb}{0.000000,0.000000,0.000000}%
\pgfsetfillcolor{currentfill}%
\pgfsetlinewidth{0.602250pt}%
\definecolor{currentstroke}{rgb}{0.000000,0.000000,0.000000}%
\pgfsetstrokecolor{currentstroke}%
\pgfsetdash{}{0pt}%
\pgfsys@defobject{currentmarker}{\pgfqpoint{0.000000in}{-0.027778in}}{\pgfqpoint{0.000000in}{0.000000in}}{%
\pgfpathmoveto{\pgfqpoint{0.000000in}{0.000000in}}%
\pgfpathlineto{\pgfqpoint{0.000000in}{-0.027778in}}%
\pgfusepath{stroke,fill}%
}%
\begin{pgfscope}%
\pgfsys@transformshift{4.938501in}{3.363889in}%
\pgfsys@useobject{currentmarker}{}%
\end{pgfscope}%
\end{pgfscope}%
\begin{pgfscope}%
\pgfsetbuttcap%
\pgfsetroundjoin%
\definecolor{currentfill}{rgb}{0.000000,0.000000,0.000000}%
\pgfsetfillcolor{currentfill}%
\pgfsetlinewidth{0.602250pt}%
\definecolor{currentstroke}{rgb}{0.000000,0.000000,0.000000}%
\pgfsetstrokecolor{currentstroke}%
\pgfsetdash{}{0pt}%
\pgfsys@defobject{currentmarker}{\pgfqpoint{0.000000in}{-0.027778in}}{\pgfqpoint{0.000000in}{0.000000in}}{%
\pgfpathmoveto{\pgfqpoint{0.000000in}{0.000000in}}%
\pgfpathlineto{\pgfqpoint{0.000000in}{-0.027778in}}%
\pgfusepath{stroke,fill}%
}%
\begin{pgfscope}%
\pgfsys@transformshift{4.969353in}{3.363889in}%
\pgfsys@useobject{currentmarker}{}%
\end{pgfscope}%
\end{pgfscope}%
\begin{pgfscope}%
\pgfsetbuttcap%
\pgfsetroundjoin%
\definecolor{currentfill}{rgb}{0.000000,0.000000,0.000000}%
\pgfsetfillcolor{currentfill}%
\pgfsetlinewidth{0.602250pt}%
\definecolor{currentstroke}{rgb}{0.000000,0.000000,0.000000}%
\pgfsetstrokecolor{currentstroke}%
\pgfsetdash{}{0pt}%
\pgfsys@defobject{currentmarker}{\pgfqpoint{0.000000in}{-0.027778in}}{\pgfqpoint{0.000000in}{0.000000in}}{%
\pgfpathmoveto{\pgfqpoint{0.000000in}{0.000000in}}%
\pgfpathlineto{\pgfqpoint{0.000000in}{-0.027778in}}%
\pgfusepath{stroke,fill}%
}%
\begin{pgfscope}%
\pgfsys@transformshift{5.178516in}{3.363889in}%
\pgfsys@useobject{currentmarker}{}%
\end{pgfscope}%
\end{pgfscope}%
\begin{pgfscope}%
\pgfsetbuttcap%
\pgfsetroundjoin%
\definecolor{currentfill}{rgb}{0.000000,0.000000,0.000000}%
\pgfsetfillcolor{currentfill}%
\pgfsetlinewidth{0.602250pt}%
\definecolor{currentstroke}{rgb}{0.000000,0.000000,0.000000}%
\pgfsetstrokecolor{currentstroke}%
\pgfsetdash{}{0pt}%
\pgfsys@defobject{currentmarker}{\pgfqpoint{0.000000in}{-0.027778in}}{\pgfqpoint{0.000000in}{0.000000in}}{%
\pgfpathmoveto{\pgfqpoint{0.000000in}{0.000000in}}%
\pgfpathlineto{\pgfqpoint{0.000000in}{-0.027778in}}%
\pgfusepath{stroke,fill}%
}%
\begin{pgfscope}%
\pgfsys@transformshift{5.284724in}{3.363889in}%
\pgfsys@useobject{currentmarker}{}%
\end{pgfscope}%
\end{pgfscope}%
\begin{pgfscope}%
\pgfsetbuttcap%
\pgfsetroundjoin%
\definecolor{currentfill}{rgb}{0.000000,0.000000,0.000000}%
\pgfsetfillcolor{currentfill}%
\pgfsetlinewidth{0.602250pt}%
\definecolor{currentstroke}{rgb}{0.000000,0.000000,0.000000}%
\pgfsetstrokecolor{currentstroke}%
\pgfsetdash{}{0pt}%
\pgfsys@defobject{currentmarker}{\pgfqpoint{0.000000in}{-0.027778in}}{\pgfqpoint{0.000000in}{0.000000in}}{%
\pgfpathmoveto{\pgfqpoint{0.000000in}{0.000000in}}%
\pgfpathlineto{\pgfqpoint{0.000000in}{-0.027778in}}%
\pgfusepath{stroke,fill}%
}%
\begin{pgfscope}%
\pgfsys@transformshift{5.360080in}{3.363889in}%
\pgfsys@useobject{currentmarker}{}%
\end{pgfscope}%
\end{pgfscope}%
\begin{pgfscope}%
\pgfsetbuttcap%
\pgfsetroundjoin%
\definecolor{currentfill}{rgb}{0.000000,0.000000,0.000000}%
\pgfsetfillcolor{currentfill}%
\pgfsetlinewidth{0.602250pt}%
\definecolor{currentstroke}{rgb}{0.000000,0.000000,0.000000}%
\pgfsetstrokecolor{currentstroke}%
\pgfsetdash{}{0pt}%
\pgfsys@defobject{currentmarker}{\pgfqpoint{0.000000in}{-0.027778in}}{\pgfqpoint{0.000000in}{0.000000in}}{%
\pgfpathmoveto{\pgfqpoint{0.000000in}{0.000000in}}%
\pgfpathlineto{\pgfqpoint{0.000000in}{-0.027778in}}%
\pgfusepath{stroke,fill}%
}%
\begin{pgfscope}%
\pgfsys@transformshift{5.418531in}{3.363889in}%
\pgfsys@useobject{currentmarker}{}%
\end{pgfscope}%
\end{pgfscope}%
\begin{pgfscope}%
\pgfsetbuttcap%
\pgfsetroundjoin%
\definecolor{currentfill}{rgb}{0.000000,0.000000,0.000000}%
\pgfsetfillcolor{currentfill}%
\pgfsetlinewidth{0.602250pt}%
\definecolor{currentstroke}{rgb}{0.000000,0.000000,0.000000}%
\pgfsetstrokecolor{currentstroke}%
\pgfsetdash{}{0pt}%
\pgfsys@defobject{currentmarker}{\pgfqpoint{0.000000in}{-0.027778in}}{\pgfqpoint{0.000000in}{0.000000in}}{%
\pgfpathmoveto{\pgfqpoint{0.000000in}{0.000000in}}%
\pgfpathlineto{\pgfqpoint{0.000000in}{-0.027778in}}%
\pgfusepath{stroke,fill}%
}%
\begin{pgfscope}%
\pgfsys@transformshift{5.466289in}{3.363889in}%
\pgfsys@useobject{currentmarker}{}%
\end{pgfscope}%
\end{pgfscope}%
\begin{pgfscope}%
\pgfsetbuttcap%
\pgfsetroundjoin%
\definecolor{currentfill}{rgb}{0.000000,0.000000,0.000000}%
\pgfsetfillcolor{currentfill}%
\pgfsetlinewidth{0.602250pt}%
\definecolor{currentstroke}{rgb}{0.000000,0.000000,0.000000}%
\pgfsetstrokecolor{currentstroke}%
\pgfsetdash{}{0pt}%
\pgfsys@defobject{currentmarker}{\pgfqpoint{0.000000in}{-0.027778in}}{\pgfqpoint{0.000000in}{0.000000in}}{%
\pgfpathmoveto{\pgfqpoint{0.000000in}{0.000000in}}%
\pgfpathlineto{\pgfqpoint{0.000000in}{-0.027778in}}%
\pgfusepath{stroke,fill}%
}%
\begin{pgfscope}%
\pgfsys@transformshift{5.506667in}{3.363889in}%
\pgfsys@useobject{currentmarker}{}%
\end{pgfscope}%
\end{pgfscope}%
\begin{pgfscope}%
\pgfsetbuttcap%
\pgfsetroundjoin%
\definecolor{currentfill}{rgb}{0.000000,0.000000,0.000000}%
\pgfsetfillcolor{currentfill}%
\pgfsetlinewidth{0.602250pt}%
\definecolor{currentstroke}{rgb}{0.000000,0.000000,0.000000}%
\pgfsetstrokecolor{currentstroke}%
\pgfsetdash{}{0pt}%
\pgfsys@defobject{currentmarker}{\pgfqpoint{0.000000in}{-0.027778in}}{\pgfqpoint{0.000000in}{0.000000in}}{%
\pgfpathmoveto{\pgfqpoint{0.000000in}{0.000000in}}%
\pgfpathlineto{\pgfqpoint{0.000000in}{-0.027778in}}%
\pgfusepath{stroke,fill}%
}%
\begin{pgfscope}%
\pgfsys@transformshift{5.541645in}{3.363889in}%
\pgfsys@useobject{currentmarker}{}%
\end{pgfscope}%
\end{pgfscope}%
\begin{pgfscope}%
\pgfsetbuttcap%
\pgfsetroundjoin%
\definecolor{currentfill}{rgb}{0.000000,0.000000,0.000000}%
\pgfsetfillcolor{currentfill}%
\pgfsetlinewidth{0.602250pt}%
\definecolor{currentstroke}{rgb}{0.000000,0.000000,0.000000}%
\pgfsetstrokecolor{currentstroke}%
\pgfsetdash{}{0pt}%
\pgfsys@defobject{currentmarker}{\pgfqpoint{0.000000in}{-0.027778in}}{\pgfqpoint{0.000000in}{0.000000in}}{%
\pgfpathmoveto{\pgfqpoint{0.000000in}{0.000000in}}%
\pgfpathlineto{\pgfqpoint{0.000000in}{-0.027778in}}%
\pgfusepath{stroke,fill}%
}%
\begin{pgfscope}%
\pgfsys@transformshift{5.572497in}{3.363889in}%
\pgfsys@useobject{currentmarker}{}%
\end{pgfscope}%
\end{pgfscope}%
\begin{pgfscope}%
\pgfsetbuttcap%
\pgfsetroundjoin%
\definecolor{currentfill}{rgb}{0.000000,0.000000,0.000000}%
\pgfsetfillcolor{currentfill}%
\pgfsetlinewidth{0.602250pt}%
\definecolor{currentstroke}{rgb}{0.000000,0.000000,0.000000}%
\pgfsetstrokecolor{currentstroke}%
\pgfsetdash{}{0pt}%
\pgfsys@defobject{currentmarker}{\pgfqpoint{0.000000in}{-0.027778in}}{\pgfqpoint{0.000000in}{0.000000in}}{%
\pgfpathmoveto{\pgfqpoint{0.000000in}{0.000000in}}%
\pgfpathlineto{\pgfqpoint{0.000000in}{-0.027778in}}%
\pgfusepath{stroke,fill}%
}%
\begin{pgfscope}%
\pgfsys@transformshift{5.781660in}{3.363889in}%
\pgfsys@useobject{currentmarker}{}%
\end{pgfscope}%
\end{pgfscope}%
\begin{pgfscope}%
\pgfsetbuttcap%
\pgfsetroundjoin%
\definecolor{currentfill}{rgb}{0.000000,0.000000,0.000000}%
\pgfsetfillcolor{currentfill}%
\pgfsetlinewidth{0.602250pt}%
\definecolor{currentstroke}{rgb}{0.000000,0.000000,0.000000}%
\pgfsetstrokecolor{currentstroke}%
\pgfsetdash{}{0pt}%
\pgfsys@defobject{currentmarker}{\pgfqpoint{0.000000in}{-0.027778in}}{\pgfqpoint{0.000000in}{0.000000in}}{%
\pgfpathmoveto{\pgfqpoint{0.000000in}{0.000000in}}%
\pgfpathlineto{\pgfqpoint{0.000000in}{-0.027778in}}%
\pgfusepath{stroke,fill}%
}%
\begin{pgfscope}%
\pgfsys@transformshift{5.887868in}{3.363889in}%
\pgfsys@useobject{currentmarker}{}%
\end{pgfscope}%
\end{pgfscope}%
\begin{pgfscope}%
\pgfsetbuttcap%
\pgfsetroundjoin%
\definecolor{currentfill}{rgb}{0.000000,0.000000,0.000000}%
\pgfsetfillcolor{currentfill}%
\pgfsetlinewidth{0.602250pt}%
\definecolor{currentstroke}{rgb}{0.000000,0.000000,0.000000}%
\pgfsetstrokecolor{currentstroke}%
\pgfsetdash{}{0pt}%
\pgfsys@defobject{currentmarker}{\pgfqpoint{0.000000in}{-0.027778in}}{\pgfqpoint{0.000000in}{0.000000in}}{%
\pgfpathmoveto{\pgfqpoint{0.000000in}{0.000000in}}%
\pgfpathlineto{\pgfqpoint{0.000000in}{-0.027778in}}%
\pgfusepath{stroke,fill}%
}%
\begin{pgfscope}%
\pgfsys@transformshift{5.963224in}{3.363889in}%
\pgfsys@useobject{currentmarker}{}%
\end{pgfscope}%
\end{pgfscope}%
\begin{pgfscope}%
\pgfsetbuttcap%
\pgfsetroundjoin%
\definecolor{currentfill}{rgb}{0.000000,0.000000,0.000000}%
\pgfsetfillcolor{currentfill}%
\pgfsetlinewidth{0.602250pt}%
\definecolor{currentstroke}{rgb}{0.000000,0.000000,0.000000}%
\pgfsetstrokecolor{currentstroke}%
\pgfsetdash{}{0pt}%
\pgfsys@defobject{currentmarker}{\pgfqpoint{0.000000in}{-0.027778in}}{\pgfqpoint{0.000000in}{0.000000in}}{%
\pgfpathmoveto{\pgfqpoint{0.000000in}{0.000000in}}%
\pgfpathlineto{\pgfqpoint{0.000000in}{-0.027778in}}%
\pgfusepath{stroke,fill}%
}%
\begin{pgfscope}%
\pgfsys@transformshift{6.021675in}{3.363889in}%
\pgfsys@useobject{currentmarker}{}%
\end{pgfscope}%
\end{pgfscope}%
\begin{pgfscope}%
\pgfsetbuttcap%
\pgfsetroundjoin%
\definecolor{currentfill}{rgb}{0.000000,0.000000,0.000000}%
\pgfsetfillcolor{currentfill}%
\pgfsetlinewidth{0.602250pt}%
\definecolor{currentstroke}{rgb}{0.000000,0.000000,0.000000}%
\pgfsetstrokecolor{currentstroke}%
\pgfsetdash{}{0pt}%
\pgfsys@defobject{currentmarker}{\pgfqpoint{0.000000in}{-0.027778in}}{\pgfqpoint{0.000000in}{0.000000in}}{%
\pgfpathmoveto{\pgfqpoint{0.000000in}{0.000000in}}%
\pgfpathlineto{\pgfqpoint{0.000000in}{-0.027778in}}%
\pgfusepath{stroke,fill}%
}%
\begin{pgfscope}%
\pgfsys@transformshift{6.069433in}{3.363889in}%
\pgfsys@useobject{currentmarker}{}%
\end{pgfscope}%
\end{pgfscope}%
\begin{pgfscope}%
\pgfsetbuttcap%
\pgfsetroundjoin%
\definecolor{currentfill}{rgb}{0.000000,0.000000,0.000000}%
\pgfsetfillcolor{currentfill}%
\pgfsetlinewidth{0.602250pt}%
\definecolor{currentstroke}{rgb}{0.000000,0.000000,0.000000}%
\pgfsetstrokecolor{currentstroke}%
\pgfsetdash{}{0pt}%
\pgfsys@defobject{currentmarker}{\pgfqpoint{0.000000in}{-0.027778in}}{\pgfqpoint{0.000000in}{0.000000in}}{%
\pgfpathmoveto{\pgfqpoint{0.000000in}{0.000000in}}%
\pgfpathlineto{\pgfqpoint{0.000000in}{-0.027778in}}%
\pgfusepath{stroke,fill}%
}%
\begin{pgfscope}%
\pgfsys@transformshift{6.109811in}{3.363889in}%
\pgfsys@useobject{currentmarker}{}%
\end{pgfscope}%
\end{pgfscope}%
\begin{pgfscope}%
\pgfsetbuttcap%
\pgfsetroundjoin%
\definecolor{currentfill}{rgb}{0.000000,0.000000,0.000000}%
\pgfsetfillcolor{currentfill}%
\pgfsetlinewidth{0.602250pt}%
\definecolor{currentstroke}{rgb}{0.000000,0.000000,0.000000}%
\pgfsetstrokecolor{currentstroke}%
\pgfsetdash{}{0pt}%
\pgfsys@defobject{currentmarker}{\pgfqpoint{0.000000in}{-0.027778in}}{\pgfqpoint{0.000000in}{0.000000in}}{%
\pgfpathmoveto{\pgfqpoint{0.000000in}{0.000000in}}%
\pgfpathlineto{\pgfqpoint{0.000000in}{-0.027778in}}%
\pgfusepath{stroke,fill}%
}%
\begin{pgfscope}%
\pgfsys@transformshift{6.144789in}{3.363889in}%
\pgfsys@useobject{currentmarker}{}%
\end{pgfscope}%
\end{pgfscope}%
\begin{pgfscope}%
\pgfsetbuttcap%
\pgfsetroundjoin%
\definecolor{currentfill}{rgb}{0.000000,0.000000,0.000000}%
\pgfsetfillcolor{currentfill}%
\pgfsetlinewidth{0.602250pt}%
\definecolor{currentstroke}{rgb}{0.000000,0.000000,0.000000}%
\pgfsetstrokecolor{currentstroke}%
\pgfsetdash{}{0pt}%
\pgfsys@defobject{currentmarker}{\pgfqpoint{0.000000in}{-0.027778in}}{\pgfqpoint{0.000000in}{0.000000in}}{%
\pgfpathmoveto{\pgfqpoint{0.000000in}{0.000000in}}%
\pgfpathlineto{\pgfqpoint{0.000000in}{-0.027778in}}%
\pgfusepath{stroke,fill}%
}%
\begin{pgfscope}%
\pgfsys@transformshift{6.175641in}{3.363889in}%
\pgfsys@useobject{currentmarker}{}%
\end{pgfscope}%
\end{pgfscope}%
\begin{pgfscope}%
\pgfsetbuttcap%
\pgfsetroundjoin%
\definecolor{currentfill}{rgb}{0.000000,0.000000,0.000000}%
\pgfsetfillcolor{currentfill}%
\pgfsetlinewidth{0.602250pt}%
\definecolor{currentstroke}{rgb}{0.000000,0.000000,0.000000}%
\pgfsetstrokecolor{currentstroke}%
\pgfsetdash{}{0pt}%
\pgfsys@defobject{currentmarker}{\pgfqpoint{0.000000in}{-0.027778in}}{\pgfqpoint{0.000000in}{0.000000in}}{%
\pgfpathmoveto{\pgfqpoint{0.000000in}{0.000000in}}%
\pgfpathlineto{\pgfqpoint{0.000000in}{-0.027778in}}%
\pgfusepath{stroke,fill}%
}%
\begin{pgfscope}%
\pgfsys@transformshift{6.384804in}{3.363889in}%
\pgfsys@useobject{currentmarker}{}%
\end{pgfscope}%
\end{pgfscope}%
\begin{pgfscope}%
\pgfsetbuttcap%
\pgfsetroundjoin%
\definecolor{currentfill}{rgb}{0.000000,0.000000,0.000000}%
\pgfsetfillcolor{currentfill}%
\pgfsetlinewidth{0.602250pt}%
\definecolor{currentstroke}{rgb}{0.000000,0.000000,0.000000}%
\pgfsetstrokecolor{currentstroke}%
\pgfsetdash{}{0pt}%
\pgfsys@defobject{currentmarker}{\pgfqpoint{0.000000in}{-0.027778in}}{\pgfqpoint{0.000000in}{0.000000in}}{%
\pgfpathmoveto{\pgfqpoint{0.000000in}{0.000000in}}%
\pgfpathlineto{\pgfqpoint{0.000000in}{-0.027778in}}%
\pgfusepath{stroke,fill}%
}%
\begin{pgfscope}%
\pgfsys@transformshift{6.491012in}{3.363889in}%
\pgfsys@useobject{currentmarker}{}%
\end{pgfscope}%
\end{pgfscope}%
\begin{pgfscope}%
\pgfsetbuttcap%
\pgfsetroundjoin%
\definecolor{currentfill}{rgb}{0.000000,0.000000,0.000000}%
\pgfsetfillcolor{currentfill}%
\pgfsetlinewidth{0.602250pt}%
\definecolor{currentstroke}{rgb}{0.000000,0.000000,0.000000}%
\pgfsetstrokecolor{currentstroke}%
\pgfsetdash{}{0pt}%
\pgfsys@defobject{currentmarker}{\pgfqpoint{0.000000in}{-0.027778in}}{\pgfqpoint{0.000000in}{0.000000in}}{%
\pgfpathmoveto{\pgfqpoint{0.000000in}{0.000000in}}%
\pgfpathlineto{\pgfqpoint{0.000000in}{-0.027778in}}%
\pgfusepath{stroke,fill}%
}%
\begin{pgfscope}%
\pgfsys@transformshift{6.566368in}{3.363889in}%
\pgfsys@useobject{currentmarker}{}%
\end{pgfscope}%
\end{pgfscope}%
\begin{pgfscope}%
\pgfsetbuttcap%
\pgfsetroundjoin%
\definecolor{currentfill}{rgb}{0.000000,0.000000,0.000000}%
\pgfsetfillcolor{currentfill}%
\pgfsetlinewidth{0.602250pt}%
\definecolor{currentstroke}{rgb}{0.000000,0.000000,0.000000}%
\pgfsetstrokecolor{currentstroke}%
\pgfsetdash{}{0pt}%
\pgfsys@defobject{currentmarker}{\pgfqpoint{0.000000in}{-0.027778in}}{\pgfqpoint{0.000000in}{0.000000in}}{%
\pgfpathmoveto{\pgfqpoint{0.000000in}{0.000000in}}%
\pgfpathlineto{\pgfqpoint{0.000000in}{-0.027778in}}%
\pgfusepath{stroke,fill}%
}%
\begin{pgfscope}%
\pgfsys@transformshift{6.624819in}{3.363889in}%
\pgfsys@useobject{currentmarker}{}%
\end{pgfscope}%
\end{pgfscope}%
\begin{pgfscope}%
\pgfsetbuttcap%
\pgfsetroundjoin%
\definecolor{currentfill}{rgb}{0.000000,0.000000,0.000000}%
\pgfsetfillcolor{currentfill}%
\pgfsetlinewidth{0.602250pt}%
\definecolor{currentstroke}{rgb}{0.000000,0.000000,0.000000}%
\pgfsetstrokecolor{currentstroke}%
\pgfsetdash{}{0pt}%
\pgfsys@defobject{currentmarker}{\pgfqpoint{0.000000in}{-0.027778in}}{\pgfqpoint{0.000000in}{0.000000in}}{%
\pgfpathmoveto{\pgfqpoint{0.000000in}{0.000000in}}%
\pgfpathlineto{\pgfqpoint{0.000000in}{-0.027778in}}%
\pgfusepath{stroke,fill}%
}%
\begin{pgfscope}%
\pgfsys@transformshift{6.672577in}{3.363889in}%
\pgfsys@useobject{currentmarker}{}%
\end{pgfscope}%
\end{pgfscope}%
\begin{pgfscope}%
\pgfsetbuttcap%
\pgfsetroundjoin%
\definecolor{currentfill}{rgb}{0.000000,0.000000,0.000000}%
\pgfsetfillcolor{currentfill}%
\pgfsetlinewidth{0.602250pt}%
\definecolor{currentstroke}{rgb}{0.000000,0.000000,0.000000}%
\pgfsetstrokecolor{currentstroke}%
\pgfsetdash{}{0pt}%
\pgfsys@defobject{currentmarker}{\pgfqpoint{0.000000in}{-0.027778in}}{\pgfqpoint{0.000000in}{0.000000in}}{%
\pgfpathmoveto{\pgfqpoint{0.000000in}{0.000000in}}%
\pgfpathlineto{\pgfqpoint{0.000000in}{-0.027778in}}%
\pgfusepath{stroke,fill}%
}%
\begin{pgfscope}%
\pgfsys@transformshift{6.712955in}{3.363889in}%
\pgfsys@useobject{currentmarker}{}%
\end{pgfscope}%
\end{pgfscope}%
\begin{pgfscope}%
\pgfsetbuttcap%
\pgfsetroundjoin%
\definecolor{currentfill}{rgb}{0.000000,0.000000,0.000000}%
\pgfsetfillcolor{currentfill}%
\pgfsetlinewidth{0.602250pt}%
\definecolor{currentstroke}{rgb}{0.000000,0.000000,0.000000}%
\pgfsetstrokecolor{currentstroke}%
\pgfsetdash{}{0pt}%
\pgfsys@defobject{currentmarker}{\pgfqpoint{0.000000in}{-0.027778in}}{\pgfqpoint{0.000000in}{0.000000in}}{%
\pgfpathmoveto{\pgfqpoint{0.000000in}{0.000000in}}%
\pgfpathlineto{\pgfqpoint{0.000000in}{-0.027778in}}%
\pgfusepath{stroke,fill}%
}%
\begin{pgfscope}%
\pgfsys@transformshift{6.747933in}{3.363889in}%
\pgfsys@useobject{currentmarker}{}%
\end{pgfscope}%
\end{pgfscope}%
\begin{pgfscope}%
\pgfsetbuttcap%
\pgfsetroundjoin%
\definecolor{currentfill}{rgb}{0.000000,0.000000,0.000000}%
\pgfsetfillcolor{currentfill}%
\pgfsetlinewidth{0.602250pt}%
\definecolor{currentstroke}{rgb}{0.000000,0.000000,0.000000}%
\pgfsetstrokecolor{currentstroke}%
\pgfsetdash{}{0pt}%
\pgfsys@defobject{currentmarker}{\pgfqpoint{0.000000in}{-0.027778in}}{\pgfqpoint{0.000000in}{0.000000in}}{%
\pgfpathmoveto{\pgfqpoint{0.000000in}{0.000000in}}%
\pgfpathlineto{\pgfqpoint{0.000000in}{-0.027778in}}%
\pgfusepath{stroke,fill}%
}%
\begin{pgfscope}%
\pgfsys@transformshift{6.778785in}{3.363889in}%
\pgfsys@useobject{currentmarker}{}%
\end{pgfscope}%
\end{pgfscope}%
\begin{pgfscope}%
\pgfsetbuttcap%
\pgfsetroundjoin%
\definecolor{currentfill}{rgb}{0.000000,0.000000,0.000000}%
\pgfsetfillcolor{currentfill}%
\pgfsetlinewidth{0.602250pt}%
\definecolor{currentstroke}{rgb}{0.000000,0.000000,0.000000}%
\pgfsetstrokecolor{currentstroke}%
\pgfsetdash{}{0pt}%
\pgfsys@defobject{currentmarker}{\pgfqpoint{0.000000in}{-0.027778in}}{\pgfqpoint{0.000000in}{0.000000in}}{%
\pgfpathmoveto{\pgfqpoint{0.000000in}{0.000000in}}%
\pgfpathlineto{\pgfqpoint{0.000000in}{-0.027778in}}%
\pgfusepath{stroke,fill}%
}%
\begin{pgfscope}%
\pgfsys@transformshift{6.987948in}{3.363889in}%
\pgfsys@useobject{currentmarker}{}%
\end{pgfscope}%
\end{pgfscope}%
\begin{pgfscope}%
\pgfsetbuttcap%
\pgfsetroundjoin%
\definecolor{currentfill}{rgb}{0.000000,0.000000,0.000000}%
\pgfsetfillcolor{currentfill}%
\pgfsetlinewidth{0.602250pt}%
\definecolor{currentstroke}{rgb}{0.000000,0.000000,0.000000}%
\pgfsetstrokecolor{currentstroke}%
\pgfsetdash{}{0pt}%
\pgfsys@defobject{currentmarker}{\pgfqpoint{0.000000in}{-0.027778in}}{\pgfqpoint{0.000000in}{0.000000in}}{%
\pgfpathmoveto{\pgfqpoint{0.000000in}{0.000000in}}%
\pgfpathlineto{\pgfqpoint{0.000000in}{-0.027778in}}%
\pgfusepath{stroke,fill}%
}%
\begin{pgfscope}%
\pgfsys@transformshift{7.094156in}{3.363889in}%
\pgfsys@useobject{currentmarker}{}%
\end{pgfscope}%
\end{pgfscope}%
\begin{pgfscope}%
\pgfsetbuttcap%
\pgfsetroundjoin%
\definecolor{currentfill}{rgb}{0.000000,0.000000,0.000000}%
\pgfsetfillcolor{currentfill}%
\pgfsetlinewidth{0.602250pt}%
\definecolor{currentstroke}{rgb}{0.000000,0.000000,0.000000}%
\pgfsetstrokecolor{currentstroke}%
\pgfsetdash{}{0pt}%
\pgfsys@defobject{currentmarker}{\pgfqpoint{0.000000in}{-0.027778in}}{\pgfqpoint{0.000000in}{0.000000in}}{%
\pgfpathmoveto{\pgfqpoint{0.000000in}{0.000000in}}%
\pgfpathlineto{\pgfqpoint{0.000000in}{-0.027778in}}%
\pgfusepath{stroke,fill}%
}%
\begin{pgfscope}%
\pgfsys@transformshift{7.169512in}{3.363889in}%
\pgfsys@useobject{currentmarker}{}%
\end{pgfscope}%
\end{pgfscope}%
\begin{pgfscope}%
\pgfsetbuttcap%
\pgfsetroundjoin%
\definecolor{currentfill}{rgb}{0.000000,0.000000,0.000000}%
\pgfsetfillcolor{currentfill}%
\pgfsetlinewidth{0.602250pt}%
\definecolor{currentstroke}{rgb}{0.000000,0.000000,0.000000}%
\pgfsetstrokecolor{currentstroke}%
\pgfsetdash{}{0pt}%
\pgfsys@defobject{currentmarker}{\pgfqpoint{0.000000in}{-0.027778in}}{\pgfqpoint{0.000000in}{0.000000in}}{%
\pgfpathmoveto{\pgfqpoint{0.000000in}{0.000000in}}%
\pgfpathlineto{\pgfqpoint{0.000000in}{-0.027778in}}%
\pgfusepath{stroke,fill}%
}%
\begin{pgfscope}%
\pgfsys@transformshift{7.227963in}{3.363889in}%
\pgfsys@useobject{currentmarker}{}%
\end{pgfscope}%
\end{pgfscope}%
\begin{pgfscope}%
\pgfsetbuttcap%
\pgfsetroundjoin%
\definecolor{currentfill}{rgb}{0.000000,0.000000,0.000000}%
\pgfsetfillcolor{currentfill}%
\pgfsetlinewidth{0.602250pt}%
\definecolor{currentstroke}{rgb}{0.000000,0.000000,0.000000}%
\pgfsetstrokecolor{currentstroke}%
\pgfsetdash{}{0pt}%
\pgfsys@defobject{currentmarker}{\pgfqpoint{0.000000in}{-0.027778in}}{\pgfqpoint{0.000000in}{0.000000in}}{%
\pgfpathmoveto{\pgfqpoint{0.000000in}{0.000000in}}%
\pgfpathlineto{\pgfqpoint{0.000000in}{-0.027778in}}%
\pgfusepath{stroke,fill}%
}%
\begin{pgfscope}%
\pgfsys@transformshift{7.275721in}{3.363889in}%
\pgfsys@useobject{currentmarker}{}%
\end{pgfscope}%
\end{pgfscope}%
\begin{pgfscope}%
\pgfsetbuttcap%
\pgfsetroundjoin%
\definecolor{currentfill}{rgb}{0.000000,0.000000,0.000000}%
\pgfsetfillcolor{currentfill}%
\pgfsetlinewidth{0.602250pt}%
\definecolor{currentstroke}{rgb}{0.000000,0.000000,0.000000}%
\pgfsetstrokecolor{currentstroke}%
\pgfsetdash{}{0pt}%
\pgfsys@defobject{currentmarker}{\pgfqpoint{0.000000in}{-0.027778in}}{\pgfqpoint{0.000000in}{0.000000in}}{%
\pgfpathmoveto{\pgfqpoint{0.000000in}{0.000000in}}%
\pgfpathlineto{\pgfqpoint{0.000000in}{-0.027778in}}%
\pgfusepath{stroke,fill}%
}%
\begin{pgfscope}%
\pgfsys@transformshift{7.316099in}{3.363889in}%
\pgfsys@useobject{currentmarker}{}%
\end{pgfscope}%
\end{pgfscope}%
\begin{pgfscope}%
\pgfsetbuttcap%
\pgfsetroundjoin%
\definecolor{currentfill}{rgb}{0.000000,0.000000,0.000000}%
\pgfsetfillcolor{currentfill}%
\pgfsetlinewidth{0.602250pt}%
\definecolor{currentstroke}{rgb}{0.000000,0.000000,0.000000}%
\pgfsetstrokecolor{currentstroke}%
\pgfsetdash{}{0pt}%
\pgfsys@defobject{currentmarker}{\pgfqpoint{0.000000in}{-0.027778in}}{\pgfqpoint{0.000000in}{0.000000in}}{%
\pgfpathmoveto{\pgfqpoint{0.000000in}{0.000000in}}%
\pgfpathlineto{\pgfqpoint{0.000000in}{-0.027778in}}%
\pgfusepath{stroke,fill}%
}%
\begin{pgfscope}%
\pgfsys@transformshift{7.351077in}{3.363889in}%
\pgfsys@useobject{currentmarker}{}%
\end{pgfscope}%
\end{pgfscope}%
\begin{pgfscope}%
\pgfsetbuttcap%
\pgfsetroundjoin%
\definecolor{currentfill}{rgb}{0.000000,0.000000,0.000000}%
\pgfsetfillcolor{currentfill}%
\pgfsetlinewidth{0.602250pt}%
\definecolor{currentstroke}{rgb}{0.000000,0.000000,0.000000}%
\pgfsetstrokecolor{currentstroke}%
\pgfsetdash{}{0pt}%
\pgfsys@defobject{currentmarker}{\pgfqpoint{0.000000in}{-0.027778in}}{\pgfqpoint{0.000000in}{0.000000in}}{%
\pgfpathmoveto{\pgfqpoint{0.000000in}{0.000000in}}%
\pgfpathlineto{\pgfqpoint{0.000000in}{-0.027778in}}%
\pgfusepath{stroke,fill}%
}%
\begin{pgfscope}%
\pgfsys@transformshift{7.381929in}{3.363889in}%
\pgfsys@useobject{currentmarker}{}%
\end{pgfscope}%
\end{pgfscope}%
\begin{pgfscope}%
\pgfsetbuttcap%
\pgfsetroundjoin%
\definecolor{currentfill}{rgb}{0.000000,0.000000,0.000000}%
\pgfsetfillcolor{currentfill}%
\pgfsetlinewidth{0.602250pt}%
\definecolor{currentstroke}{rgb}{0.000000,0.000000,0.000000}%
\pgfsetstrokecolor{currentstroke}%
\pgfsetdash{}{0pt}%
\pgfsys@defobject{currentmarker}{\pgfqpoint{0.000000in}{-0.027778in}}{\pgfqpoint{0.000000in}{0.000000in}}{%
\pgfpathmoveto{\pgfqpoint{0.000000in}{0.000000in}}%
\pgfpathlineto{\pgfqpoint{0.000000in}{-0.027778in}}%
\pgfusepath{stroke,fill}%
}%
\begin{pgfscope}%
\pgfsys@transformshift{7.591092in}{3.363889in}%
\pgfsys@useobject{currentmarker}{}%
\end{pgfscope}%
\end{pgfscope}%
\begin{pgfscope}%
\pgfsetbuttcap%
\pgfsetroundjoin%
\definecolor{currentfill}{rgb}{0.000000,0.000000,0.000000}%
\pgfsetfillcolor{currentfill}%
\pgfsetlinewidth{0.602250pt}%
\definecolor{currentstroke}{rgb}{0.000000,0.000000,0.000000}%
\pgfsetstrokecolor{currentstroke}%
\pgfsetdash{}{0pt}%
\pgfsys@defobject{currentmarker}{\pgfqpoint{0.000000in}{-0.027778in}}{\pgfqpoint{0.000000in}{0.000000in}}{%
\pgfpathmoveto{\pgfqpoint{0.000000in}{0.000000in}}%
\pgfpathlineto{\pgfqpoint{0.000000in}{-0.027778in}}%
\pgfusepath{stroke,fill}%
}%
\begin{pgfscope}%
\pgfsys@transformshift{7.697300in}{3.363889in}%
\pgfsys@useobject{currentmarker}{}%
\end{pgfscope}%
\end{pgfscope}%
\begin{pgfscope}%
\pgfsetbuttcap%
\pgfsetroundjoin%
\definecolor{currentfill}{rgb}{0.000000,0.000000,0.000000}%
\pgfsetfillcolor{currentfill}%
\pgfsetlinewidth{0.602250pt}%
\definecolor{currentstroke}{rgb}{0.000000,0.000000,0.000000}%
\pgfsetstrokecolor{currentstroke}%
\pgfsetdash{}{0pt}%
\pgfsys@defobject{currentmarker}{\pgfqpoint{0.000000in}{-0.027778in}}{\pgfqpoint{0.000000in}{0.000000in}}{%
\pgfpathmoveto{\pgfqpoint{0.000000in}{0.000000in}}%
\pgfpathlineto{\pgfqpoint{0.000000in}{-0.027778in}}%
\pgfusepath{stroke,fill}%
}%
\begin{pgfscope}%
\pgfsys@transformshift{7.772656in}{3.363889in}%
\pgfsys@useobject{currentmarker}{}%
\end{pgfscope}%
\end{pgfscope}%
\begin{pgfscope}%
\pgfsetbuttcap%
\pgfsetroundjoin%
\definecolor{currentfill}{rgb}{0.000000,0.000000,0.000000}%
\pgfsetfillcolor{currentfill}%
\pgfsetlinewidth{0.803000pt}%
\definecolor{currentstroke}{rgb}{0.000000,0.000000,0.000000}%
\pgfsetstrokecolor{currentstroke}%
\pgfsetdash{}{0pt}%
\pgfsys@defobject{currentmarker}{\pgfqpoint{-0.048611in}{0.000000in}}{\pgfqpoint{0.000000in}{0.000000in}}{%
\pgfpathmoveto{\pgfqpoint{0.000000in}{0.000000in}}%
\pgfpathlineto{\pgfqpoint{-0.048611in}{0.000000in}}%
\pgfusepath{stroke,fill}%
}%
\begin{pgfscope}%
\pgfsys@transformshift{4.482986in}{3.657265in}%
\pgfsys@useobject{currentmarker}{}%
\end{pgfscope}%
\end{pgfscope}%
\begin{pgfscope}%
\definecolor{textcolor}{rgb}{0.000000,0.000000,0.000000}%
\pgfsetstrokecolor{textcolor}%
\pgfsetfillcolor{textcolor}%
\pgftext[x=4.076519in,y=3.604504in,left,base]{\color{textcolor}\sffamily\fontsize{10.000000}{12.000000}\selectfont 0.00}%
\end{pgfscope}%
\begin{pgfscope}%
\pgfsetbuttcap%
\pgfsetroundjoin%
\definecolor{currentfill}{rgb}{0.000000,0.000000,0.000000}%
\pgfsetfillcolor{currentfill}%
\pgfsetlinewidth{0.803000pt}%
\definecolor{currentstroke}{rgb}{0.000000,0.000000,0.000000}%
\pgfsetstrokecolor{currentstroke}%
\pgfsetdash{}{0pt}%
\pgfsys@defobject{currentmarker}{\pgfqpoint{-0.048611in}{0.000000in}}{\pgfqpoint{0.000000in}{0.000000in}}{%
\pgfpathmoveto{\pgfqpoint{0.000000in}{0.000000in}}%
\pgfpathlineto{\pgfqpoint{-0.048611in}{0.000000in}}%
\pgfusepath{stroke,fill}%
}%
\begin{pgfscope}%
\pgfsys@transformshift{4.482986in}{4.007150in}%
\pgfsys@useobject{currentmarker}{}%
\end{pgfscope}%
\end{pgfscope}%
\begin{pgfscope}%
\definecolor{textcolor}{rgb}{0.000000,0.000000,0.000000}%
\pgfsetstrokecolor{textcolor}%
\pgfsetfillcolor{textcolor}%
\pgftext[x=4.076519in,y=3.954388in,left,base]{\color{textcolor}\sffamily\fontsize{10.000000}{12.000000}\selectfont 0.25}%
\end{pgfscope}%
\begin{pgfscope}%
\pgfsetbuttcap%
\pgfsetroundjoin%
\definecolor{currentfill}{rgb}{0.000000,0.000000,0.000000}%
\pgfsetfillcolor{currentfill}%
\pgfsetlinewidth{0.803000pt}%
\definecolor{currentstroke}{rgb}{0.000000,0.000000,0.000000}%
\pgfsetstrokecolor{currentstroke}%
\pgfsetdash{}{0pt}%
\pgfsys@defobject{currentmarker}{\pgfqpoint{-0.048611in}{0.000000in}}{\pgfqpoint{0.000000in}{0.000000in}}{%
\pgfpathmoveto{\pgfqpoint{0.000000in}{0.000000in}}%
\pgfpathlineto{\pgfqpoint{-0.048611in}{0.000000in}}%
\pgfusepath{stroke,fill}%
}%
\begin{pgfscope}%
\pgfsys@transformshift{4.482986in}{4.357034in}%
\pgfsys@useobject{currentmarker}{}%
\end{pgfscope}%
\end{pgfscope}%
\begin{pgfscope}%
\definecolor{textcolor}{rgb}{0.000000,0.000000,0.000000}%
\pgfsetstrokecolor{textcolor}%
\pgfsetfillcolor{textcolor}%
\pgftext[x=4.076519in,y=4.304273in,left,base]{\color{textcolor}\sffamily\fontsize{10.000000}{12.000000}\selectfont 0.50}%
\end{pgfscope}%
\begin{pgfscope}%
\pgfsetbuttcap%
\pgfsetroundjoin%
\definecolor{currentfill}{rgb}{0.000000,0.000000,0.000000}%
\pgfsetfillcolor{currentfill}%
\pgfsetlinewidth{0.803000pt}%
\definecolor{currentstroke}{rgb}{0.000000,0.000000,0.000000}%
\pgfsetstrokecolor{currentstroke}%
\pgfsetdash{}{0pt}%
\pgfsys@defobject{currentmarker}{\pgfqpoint{-0.048611in}{0.000000in}}{\pgfqpoint{0.000000in}{0.000000in}}{%
\pgfpathmoveto{\pgfqpoint{0.000000in}{0.000000in}}%
\pgfpathlineto{\pgfqpoint{-0.048611in}{0.000000in}}%
\pgfusepath{stroke,fill}%
}%
\begin{pgfscope}%
\pgfsys@transformshift{4.482986in}{4.706919in}%
\pgfsys@useobject{currentmarker}{}%
\end{pgfscope}%
\end{pgfscope}%
\begin{pgfscope}%
\definecolor{textcolor}{rgb}{0.000000,0.000000,0.000000}%
\pgfsetstrokecolor{textcolor}%
\pgfsetfillcolor{textcolor}%
\pgftext[x=4.076519in,y=4.654157in,left,base]{\color{textcolor}\sffamily\fontsize{10.000000}{12.000000}\selectfont 0.75}%
\end{pgfscope}%
\begin{pgfscope}%
\pgfsetbuttcap%
\pgfsetroundjoin%
\definecolor{currentfill}{rgb}{0.000000,0.000000,0.000000}%
\pgfsetfillcolor{currentfill}%
\pgfsetlinewidth{0.803000pt}%
\definecolor{currentstroke}{rgb}{0.000000,0.000000,0.000000}%
\pgfsetstrokecolor{currentstroke}%
\pgfsetdash{}{0pt}%
\pgfsys@defobject{currentmarker}{\pgfqpoint{-0.048611in}{0.000000in}}{\pgfqpoint{0.000000in}{0.000000in}}{%
\pgfpathmoveto{\pgfqpoint{0.000000in}{0.000000in}}%
\pgfpathlineto{\pgfqpoint{-0.048611in}{0.000000in}}%
\pgfusepath{stroke,fill}%
}%
\begin{pgfscope}%
\pgfsys@transformshift{4.482986in}{5.056803in}%
\pgfsys@useobject{currentmarker}{}%
\end{pgfscope}%
\end{pgfscope}%
\begin{pgfscope}%
\definecolor{textcolor}{rgb}{0.000000,0.000000,0.000000}%
\pgfsetstrokecolor{textcolor}%
\pgfsetfillcolor{textcolor}%
\pgftext[x=4.076519in,y=5.004041in,left,base]{\color{textcolor}\sffamily\fontsize{10.000000}{12.000000}\selectfont 1.00}%
\end{pgfscope}%
\begin{pgfscope}%
\pgfsetbuttcap%
\pgfsetroundjoin%
\definecolor{currentfill}{rgb}{0.000000,0.000000,0.000000}%
\pgfsetfillcolor{currentfill}%
\pgfsetlinewidth{0.803000pt}%
\definecolor{currentstroke}{rgb}{0.000000,0.000000,0.000000}%
\pgfsetstrokecolor{currentstroke}%
\pgfsetdash{}{0pt}%
\pgfsys@defobject{currentmarker}{\pgfqpoint{-0.048611in}{0.000000in}}{\pgfqpoint{0.000000in}{0.000000in}}{%
\pgfpathmoveto{\pgfqpoint{0.000000in}{0.000000in}}%
\pgfpathlineto{\pgfqpoint{-0.048611in}{0.000000in}}%
\pgfusepath{stroke,fill}%
}%
\begin{pgfscope}%
\pgfsys@transformshift{4.482986in}{5.406687in}%
\pgfsys@useobject{currentmarker}{}%
\end{pgfscope}%
\end{pgfscope}%
\begin{pgfscope}%
\definecolor{textcolor}{rgb}{0.000000,0.000000,0.000000}%
\pgfsetstrokecolor{textcolor}%
\pgfsetfillcolor{textcolor}%
\pgftext[x=4.076519in,y=5.353926in,left,base]{\color{textcolor}\sffamily\fontsize{10.000000}{12.000000}\selectfont 1.25}%
\end{pgfscope}%
\begin{pgfscope}%
\pgfpathrectangle{\pgfqpoint{4.482986in}{3.363889in}}{\pgfqpoint{3.318403in}{2.263889in}}%
\pgfusepath{clip}%
\pgfsetrectcap%
\pgfsetroundjoin%
\pgfsetlinewidth{1.505625pt}%
\definecolor{currentstroke}{rgb}{0.121569,0.466667,0.705882}%
\pgfsetstrokecolor{currentstroke}%
\pgfsetdash{}{0pt}%
\pgfpathmoveto{\pgfqpoint{4.633823in}{3.764997in}}%
\pgfpathlineto{\pgfqpoint{4.784609in}{3.749698in}}%
\pgfpathlineto{\pgfqpoint{4.935395in}{3.790628in}}%
\pgfpathlineto{\pgfqpoint{5.086181in}{3.894543in}}%
\pgfpathlineto{\pgfqpoint{5.236967in}{3.926825in}}%
\pgfpathlineto{\pgfqpoint{5.387753in}{4.151888in}}%
\pgfpathlineto{\pgfqpoint{5.538539in}{4.192528in}}%
\pgfpathlineto{\pgfqpoint{5.689325in}{4.633324in}}%
\pgfpathlineto{\pgfqpoint{5.840111in}{4.399581in}}%
\pgfpathlineto{\pgfqpoint{5.990897in}{4.773934in}}%
\pgfpathlineto{\pgfqpoint{6.141683in}{4.433524in}}%
\pgfpathlineto{\pgfqpoint{6.292469in}{4.762273in}}%
\pgfpathlineto{\pgfqpoint{6.443255in}{5.055804in}}%
\pgfpathlineto{\pgfqpoint{6.594041in}{4.567356in}}%
\pgfpathlineto{\pgfqpoint{6.744827in}{5.055779in}}%
\pgfpathlineto{\pgfqpoint{6.895613in}{5.055784in}}%
\pgfpathlineto{\pgfqpoint{7.046399in}{5.040685in}}%
\pgfpathlineto{\pgfqpoint{7.197185in}{5.055773in}}%
\pgfpathlineto{\pgfqpoint{7.347971in}{5.055789in}}%
\pgfpathlineto{\pgfqpoint{7.498756in}{5.055795in}}%
\pgfpathlineto{\pgfqpoint{7.649542in}{5.055788in}}%
\pgfusepath{stroke}%
\end{pgfscope}%
\begin{pgfscope}%
\pgfpathrectangle{\pgfqpoint{4.482986in}{3.363889in}}{\pgfqpoint{3.318403in}{2.263889in}}%
\pgfusepath{clip}%
\pgfsetrectcap%
\pgfsetroundjoin%
\pgfsetlinewidth{1.505625pt}%
\definecolor{currentstroke}{rgb}{1.000000,0.498039,0.054902}%
\pgfsetstrokecolor{currentstroke}%
\pgfsetdash{}{0pt}%
\pgfpathmoveto{\pgfqpoint{4.633823in}{3.791985in}}%
\pgfpathlineto{\pgfqpoint{4.784609in}{3.801622in}}%
\pgfpathlineto{\pgfqpoint{4.935395in}{3.812519in}}%
\pgfpathlineto{\pgfqpoint{5.086181in}{3.806716in}}%
\pgfpathlineto{\pgfqpoint{5.236967in}{3.948761in}}%
\pgfpathlineto{\pgfqpoint{5.387753in}{3.951914in}}%
\pgfpathlineto{\pgfqpoint{5.538539in}{3.974789in}}%
\pgfpathlineto{\pgfqpoint{5.689325in}{4.399587in}}%
\pgfpathlineto{\pgfqpoint{5.840111in}{4.390414in}}%
\pgfpathlineto{\pgfqpoint{5.990897in}{4.709515in}}%
\pgfpathlineto{\pgfqpoint{6.141683in}{4.737247in}}%
\pgfpathlineto{\pgfqpoint{6.292469in}{5.036418in}}%
\pgfpathlineto{\pgfqpoint{6.443255in}{4.984157in}}%
\pgfpathlineto{\pgfqpoint{6.594041in}{4.887584in}}%
\pgfpathlineto{\pgfqpoint{6.744827in}{5.038014in}}%
\pgfpathlineto{\pgfqpoint{6.895613in}{5.037999in}}%
\pgfpathlineto{\pgfqpoint{7.046399in}{5.037887in}}%
\pgfpathlineto{\pgfqpoint{7.197185in}{5.037968in}}%
\pgfpathlineto{\pgfqpoint{7.347971in}{5.037860in}}%
\pgfpathlineto{\pgfqpoint{7.498756in}{5.037874in}}%
\pgfpathlineto{\pgfqpoint{7.649542in}{5.037922in}}%
\pgfusepath{stroke}%
\end{pgfscope}%
\begin{pgfscope}%
\pgfpathrectangle{\pgfqpoint{4.482986in}{3.363889in}}{\pgfqpoint{3.318403in}{2.263889in}}%
\pgfusepath{clip}%
\pgfsetrectcap%
\pgfsetroundjoin%
\pgfsetlinewidth{1.505625pt}%
\definecolor{currentstroke}{rgb}{0.172549,0.627451,0.172549}%
\pgfsetstrokecolor{currentstroke}%
\pgfsetdash{}{0pt}%
\pgfpathmoveto{\pgfqpoint{4.633823in}{3.763868in}}%
\pgfpathlineto{\pgfqpoint{4.784609in}{3.749270in}}%
\pgfpathlineto{\pgfqpoint{4.935395in}{3.732977in}}%
\pgfpathlineto{\pgfqpoint{5.086181in}{3.873622in}}%
\pgfpathlineto{\pgfqpoint{5.236967in}{3.788222in}}%
\pgfpathlineto{\pgfqpoint{5.387753in}{4.168671in}}%
\pgfpathlineto{\pgfqpoint{5.538539in}{4.120823in}}%
\pgfpathlineto{\pgfqpoint{5.689325in}{3.919974in}}%
\pgfpathlineto{\pgfqpoint{5.840111in}{4.305093in}}%
\pgfpathlineto{\pgfqpoint{5.990897in}{4.227048in}}%
\pgfpathlineto{\pgfqpoint{6.141683in}{4.686787in}}%
\pgfpathlineto{\pgfqpoint{6.292469in}{4.932556in}}%
\pgfpathlineto{\pgfqpoint{6.443255in}{4.935088in}}%
\pgfpathlineto{\pgfqpoint{6.594041in}{4.935533in}}%
\pgfpathlineto{\pgfqpoint{6.744827in}{4.932031in}}%
\pgfpathlineto{\pgfqpoint{6.895613in}{4.934299in}}%
\pgfpathlineto{\pgfqpoint{7.046399in}{4.931718in}}%
\pgfpathlineto{\pgfqpoint{7.197185in}{4.932086in}}%
\pgfpathlineto{\pgfqpoint{7.347971in}{4.932742in}}%
\pgfpathlineto{\pgfqpoint{7.498756in}{4.932514in}}%
\pgfpathlineto{\pgfqpoint{7.649542in}{4.932821in}}%
\pgfusepath{stroke}%
\end{pgfscope}%
\begin{pgfscope}%
\pgfpathrectangle{\pgfqpoint{4.482986in}{3.363889in}}{\pgfqpoint{3.318403in}{2.263889in}}%
\pgfusepath{clip}%
\pgfsetrectcap%
\pgfsetroundjoin%
\pgfsetlinewidth{1.505625pt}%
\definecolor{currentstroke}{rgb}{0.839216,0.152941,0.156863}%
\pgfsetstrokecolor{currentstroke}%
\pgfsetdash{}{0pt}%
\pgfpathmoveto{\pgfqpoint{4.633823in}{3.719885in}}%
\pgfpathlineto{\pgfqpoint{4.784609in}{3.766371in}}%
\pgfpathlineto{\pgfqpoint{4.935395in}{3.715219in}}%
\pgfpathlineto{\pgfqpoint{5.086181in}{3.801762in}}%
\pgfpathlineto{\pgfqpoint{5.236967in}{3.821271in}}%
\pgfpathlineto{\pgfqpoint{5.387753in}{3.886073in}}%
\pgfpathlineto{\pgfqpoint{5.538539in}{3.798609in}}%
\pgfpathlineto{\pgfqpoint{5.689325in}{3.894287in}}%
\pgfpathlineto{\pgfqpoint{5.840111in}{3.929800in}}%
\pgfpathlineto{\pgfqpoint{5.990897in}{4.006194in}}%
\pgfpathlineto{\pgfqpoint{6.141683in}{3.951557in}}%
\pgfpathlineto{\pgfqpoint{6.292469in}{3.943651in}}%
\pgfpathlineto{\pgfqpoint{6.443255in}{3.952953in}}%
\pgfpathlineto{\pgfqpoint{6.594041in}{3.926751in}}%
\pgfpathlineto{\pgfqpoint{6.744827in}{3.925015in}}%
\pgfpathlineto{\pgfqpoint{6.895613in}{3.935325in}}%
\pgfpathlineto{\pgfqpoint{7.046399in}{3.945034in}}%
\pgfpathlineto{\pgfqpoint{7.197185in}{3.938416in}}%
\pgfpathlineto{\pgfqpoint{7.347971in}{3.944363in}}%
\pgfpathlineto{\pgfqpoint{7.498756in}{3.941043in}}%
\pgfpathlineto{\pgfqpoint{7.649542in}{3.939452in}}%
\pgfusepath{stroke}%
\end{pgfscope}%
\begin{pgfscope}%
\pgfpathrectangle{\pgfqpoint{4.482986in}{3.363889in}}{\pgfqpoint{3.318403in}{2.263889in}}%
\pgfusepath{clip}%
\pgfsetrectcap%
\pgfsetroundjoin%
\pgfsetlinewidth{1.505625pt}%
\definecolor{currentstroke}{rgb}{0.580392,0.403922,0.741176}%
\pgfsetstrokecolor{currentstroke}%
\pgfsetdash{}{0pt}%
\pgfpathmoveto{\pgfqpoint{4.633823in}{3.721978in}}%
\pgfpathlineto{\pgfqpoint{4.784609in}{3.741195in}}%
\pgfpathlineto{\pgfqpoint{4.935395in}{3.806489in}}%
\pgfpathlineto{\pgfqpoint{5.086181in}{3.837762in}}%
\pgfpathlineto{\pgfqpoint{5.236967in}{3.749054in}}%
\pgfpathlineto{\pgfqpoint{5.387753in}{3.771638in}}%
\pgfpathlineto{\pgfqpoint{5.538539in}{3.793508in}}%
\pgfpathlineto{\pgfqpoint{5.689325in}{3.782050in}}%
\pgfpathlineto{\pgfqpoint{5.840111in}{3.770886in}}%
\pgfpathlineto{\pgfqpoint{5.990897in}{3.767747in}}%
\pgfpathlineto{\pgfqpoint{6.141683in}{3.771821in}}%
\pgfpathlineto{\pgfqpoint{6.292469in}{3.776206in}}%
\pgfpathlineto{\pgfqpoint{6.443255in}{3.768919in}}%
\pgfpathlineto{\pgfqpoint{6.594041in}{3.772142in}}%
\pgfpathlineto{\pgfqpoint{6.744827in}{3.777330in}}%
\pgfpathlineto{\pgfqpoint{6.895613in}{3.775099in}}%
\pgfpathlineto{\pgfqpoint{7.046399in}{3.777212in}}%
\pgfpathlineto{\pgfqpoint{7.197185in}{3.776424in}}%
\pgfpathlineto{\pgfqpoint{7.347971in}{3.775934in}}%
\pgfpathlineto{\pgfqpoint{7.498756in}{3.776603in}}%
\pgfpathlineto{\pgfqpoint{7.649542in}{3.775983in}}%
\pgfusepath{stroke}%
\end{pgfscope}%
\begin{pgfscope}%
\pgfsetrectcap%
\pgfsetmiterjoin%
\pgfsetlinewidth{0.803000pt}%
\definecolor{currentstroke}{rgb}{0.000000,0.000000,0.000000}%
\pgfsetstrokecolor{currentstroke}%
\pgfsetdash{}{0pt}%
\pgfpathmoveto{\pgfqpoint{4.482986in}{3.363889in}}%
\pgfpathlineto{\pgfqpoint{4.482986in}{5.627778in}}%
\pgfusepath{stroke}%
\end{pgfscope}%
\begin{pgfscope}%
\pgfsetrectcap%
\pgfsetmiterjoin%
\pgfsetlinewidth{0.803000pt}%
\definecolor{currentstroke}{rgb}{0.000000,0.000000,0.000000}%
\pgfsetstrokecolor{currentstroke}%
\pgfsetdash{}{0pt}%
\pgfpathmoveto{\pgfqpoint{7.801389in}{3.363889in}}%
\pgfpathlineto{\pgfqpoint{7.801389in}{5.627778in}}%
\pgfusepath{stroke}%
\end{pgfscope}%
\begin{pgfscope}%
\pgfsetrectcap%
\pgfsetmiterjoin%
\pgfsetlinewidth{0.803000pt}%
\definecolor{currentstroke}{rgb}{0.000000,0.000000,0.000000}%
\pgfsetstrokecolor{currentstroke}%
\pgfsetdash{}{0pt}%
\pgfpathmoveto{\pgfqpoint{4.482986in}{3.363889in}}%
\pgfpathlineto{\pgfqpoint{7.801389in}{3.363889in}}%
\pgfusepath{stroke}%
\end{pgfscope}%
\begin{pgfscope}%
\pgfsetrectcap%
\pgfsetmiterjoin%
\pgfsetlinewidth{0.803000pt}%
\definecolor{currentstroke}{rgb}{0.000000,0.000000,0.000000}%
\pgfsetstrokecolor{currentstroke}%
\pgfsetdash{}{0pt}%
\pgfpathmoveto{\pgfqpoint{4.482986in}{5.627778in}}%
\pgfpathlineto{\pgfqpoint{7.801389in}{5.627778in}}%
\pgfusepath{stroke}%
\end{pgfscope}%
\begin{pgfscope}%
\definecolor{textcolor}{rgb}{0.000000,0.000000,0.000000}%
\pgfsetstrokecolor{textcolor}%
\pgfsetfillcolor{textcolor}%
\pgftext[x=6.142187in,y=5.711111in,,base]{\color{textcolor}\sffamily\fontsize{12.000000}{14.400000}\selectfont \(\displaystyle  N = 32 \)}%
\end{pgfscope}%
\begin{pgfscope}%
\pgfsetbuttcap%
\pgfsetmiterjoin%
\definecolor{currentfill}{rgb}{1.000000,1.000000,1.000000}%
\pgfsetfillcolor{currentfill}%
\pgfsetlinewidth{0.000000pt}%
\definecolor{currentstroke}{rgb}{0.000000,0.000000,0.000000}%
\pgfsetstrokecolor{currentstroke}%
\pgfsetstrokeopacity{0.000000}%
\pgfsetdash{}{0pt}%
\pgfpathmoveto{\pgfqpoint{0.557986in}{0.438889in}}%
\pgfpathlineto{\pgfqpoint{3.876389in}{0.438889in}}%
\pgfpathlineto{\pgfqpoint{3.876389in}{2.702778in}}%
\pgfpathlineto{\pgfqpoint{0.557986in}{2.702778in}}%
\pgfpathclose%
\pgfusepath{fill}%
\end{pgfscope}%
\begin{pgfscope}%
\pgfpathrectangle{\pgfqpoint{0.557986in}{0.438889in}}{\pgfqpoint{3.318403in}{2.263889in}}%
\pgfusepath{clip}%
\pgfsetbuttcap%
\pgfsetroundjoin%
\definecolor{currentfill}{rgb}{0.121569,0.466667,0.705882}%
\pgfsetfillcolor{currentfill}%
\pgfsetlinewidth{1.003750pt}%
\definecolor{currentstroke}{rgb}{0.121569,0.466667,0.705882}%
\pgfsetstrokecolor{currentstroke}%
\pgfsetdash{}{0pt}%
\pgfsys@defobject{currentmarker}{\pgfqpoint{-0.009821in}{-0.009821in}}{\pgfqpoint{0.009821in}{0.009821in}}{%
\pgfpathmoveto{\pgfqpoint{0.000000in}{-0.009821in}}%
\pgfpathcurveto{\pgfqpoint{0.002605in}{-0.009821in}}{\pgfqpoint{0.005103in}{-0.008786in}}{\pgfqpoint{0.006944in}{-0.006944in}}%
\pgfpathcurveto{\pgfqpoint{0.008786in}{-0.005103in}}{\pgfqpoint{0.009821in}{-0.002605in}}{\pgfqpoint{0.009821in}{0.000000in}}%
\pgfpathcurveto{\pgfqpoint{0.009821in}{0.002605in}}{\pgfqpoint{0.008786in}{0.005103in}}{\pgfqpoint{0.006944in}{0.006944in}}%
\pgfpathcurveto{\pgfqpoint{0.005103in}{0.008786in}}{\pgfqpoint{0.002605in}{0.009821in}}{\pgfqpoint{0.000000in}{0.009821in}}%
\pgfpathcurveto{\pgfqpoint{-0.002605in}{0.009821in}}{\pgfqpoint{-0.005103in}{0.008786in}}{\pgfqpoint{-0.006944in}{0.006944in}}%
\pgfpathcurveto{\pgfqpoint{-0.008786in}{0.005103in}}{\pgfqpoint{-0.009821in}{0.002605in}}{\pgfqpoint{-0.009821in}{0.000000in}}%
\pgfpathcurveto{\pgfqpoint{-0.009821in}{-0.002605in}}{\pgfqpoint{-0.008786in}{-0.005103in}}{\pgfqpoint{-0.006944in}{-0.006944in}}%
\pgfpathcurveto{\pgfqpoint{-0.005103in}{-0.008786in}}{\pgfqpoint{-0.002605in}{-0.009821in}}{\pgfqpoint{0.000000in}{-0.009821in}}%
\pgfpathclose%
\pgfusepath{stroke,fill}%
}%
\begin{pgfscope}%
\pgfsys@transformshift{0.708823in}{0.677637in}%
\pgfsys@useobject{currentmarker}{}%
\end{pgfscope}%
\begin{pgfscope}%
\pgfsys@transformshift{0.859609in}{0.665474in}%
\pgfsys@useobject{currentmarker}{}%
\end{pgfscope}%
\begin{pgfscope}%
\pgfsys@transformshift{1.010395in}{0.704744in}%
\pgfsys@useobject{currentmarker}{}%
\end{pgfscope}%
\begin{pgfscope}%
\pgfsys@transformshift{1.161181in}{0.696222in}%
\pgfsys@useobject{currentmarker}{}%
\end{pgfscope}%
\begin{pgfscope}%
\pgfsys@transformshift{1.311967in}{0.741800in}%
\pgfsys@useobject{currentmarker}{}%
\end{pgfscope}%
\begin{pgfscope}%
\pgfsys@transformshift{1.462753in}{0.754834in}%
\pgfsys@useobject{currentmarker}{}%
\end{pgfscope}%
\begin{pgfscope}%
\pgfsys@transformshift{1.613539in}{0.828864in}%
\pgfsys@useobject{currentmarker}{}%
\end{pgfscope}%
\begin{pgfscope}%
\pgfsys@transformshift{1.764325in}{0.936655in}%
\pgfsys@useobject{currentmarker}{}%
\end{pgfscope}%
\begin{pgfscope}%
\pgfsys@transformshift{1.915111in}{1.059509in}%
\pgfsys@useobject{currentmarker}{}%
\end{pgfscope}%
\begin{pgfscope}%
\pgfsys@transformshift{2.065897in}{0.890936in}%
\pgfsys@useobject{currentmarker}{}%
\end{pgfscope}%
\begin{pgfscope}%
\pgfsys@transformshift{2.216683in}{0.983823in}%
\pgfsys@useobject{currentmarker}{}%
\end{pgfscope}%
\begin{pgfscope}%
\pgfsys@transformshift{2.367469in}{1.293789in}%
\pgfsys@useobject{currentmarker}{}%
\end{pgfscope}%
\begin{pgfscope}%
\pgfsys@transformshift{2.518255in}{1.459498in}%
\pgfsys@useobject{currentmarker}{}%
\end{pgfscope}%
\begin{pgfscope}%
\pgfsys@transformshift{2.669041in}{2.050375in}%
\pgfsys@useobject{currentmarker}{}%
\end{pgfscope}%
\begin{pgfscope}%
\pgfsys@transformshift{2.819827in}{1.935330in}%
\pgfsys@useobject{currentmarker}{}%
\end{pgfscope}%
\begin{pgfscope}%
\pgfsys@transformshift{2.970613in}{1.835961in}%
\pgfsys@useobject{currentmarker}{}%
\end{pgfscope}%
\begin{pgfscope}%
\pgfsys@transformshift{3.121399in}{2.105455in}%
\pgfsys@useobject{currentmarker}{}%
\end{pgfscope}%
\begin{pgfscope}%
\pgfsys@transformshift{3.272185in}{1.841539in}%
\pgfsys@useobject{currentmarker}{}%
\end{pgfscope}%
\begin{pgfscope}%
\pgfsys@transformshift{3.422971in}{1.892115in}%
\pgfsys@useobject{currentmarker}{}%
\end{pgfscope}%
\begin{pgfscope}%
\pgfsys@transformshift{3.573756in}{2.130019in}%
\pgfsys@useobject{currentmarker}{}%
\end{pgfscope}%
\begin{pgfscope}%
\pgfsys@transformshift{3.724542in}{2.010187in}%
\pgfsys@useobject{currentmarker}{}%
\end{pgfscope}%
\end{pgfscope}%
\begin{pgfscope}%
\pgfpathrectangle{\pgfqpoint{0.557986in}{0.438889in}}{\pgfqpoint{3.318403in}{2.263889in}}%
\pgfusepath{clip}%
\pgfsetbuttcap%
\pgfsetroundjoin%
\definecolor{currentfill}{rgb}{0.121569,0.466667,0.705882}%
\pgfsetfillcolor{currentfill}%
\pgfsetfillopacity{0.300000}%
\pgfsetlinewidth{1.003750pt}%
\definecolor{currentstroke}{rgb}{0.121569,0.466667,0.705882}%
\pgfsetstrokecolor{currentstroke}%
\pgfsetstrokeopacity{0.300000}%
\pgfsetdash{}{0pt}%
\pgfpathmoveto{\pgfqpoint{0.708823in}{0.718164in}}%
\pgfpathlineto{\pgfqpoint{0.708823in}{0.637110in}}%
\pgfpathlineto{\pgfqpoint{0.859609in}{0.654559in}}%
\pgfpathlineto{\pgfqpoint{1.010395in}{0.629657in}}%
\pgfpathlineto{\pgfqpoint{1.161181in}{0.665361in}}%
\pgfpathlineto{\pgfqpoint{1.311967in}{0.707876in}}%
\pgfpathlineto{\pgfqpoint{1.462753in}{0.650313in}}%
\pgfpathlineto{\pgfqpoint{1.613539in}{0.710611in}}%
\pgfpathlineto{\pgfqpoint{1.764325in}{0.541793in}}%
\pgfpathlineto{\pgfqpoint{1.915111in}{1.043917in}}%
\pgfpathlineto{\pgfqpoint{2.065897in}{0.711899in}}%
\pgfpathlineto{\pgfqpoint{2.216683in}{0.715802in}}%
\pgfpathlineto{\pgfqpoint{2.367469in}{1.085089in}}%
\pgfpathlineto{\pgfqpoint{2.518255in}{0.549824in}}%
\pgfpathlineto{\pgfqpoint{2.669041in}{1.947977in}}%
\pgfpathlineto{\pgfqpoint{2.819827in}{1.430001in}}%
\pgfpathlineto{\pgfqpoint{2.970613in}{1.072048in}}%
\pgfpathlineto{\pgfqpoint{3.121399in}{2.041641in}}%
\pgfpathlineto{\pgfqpoint{3.272185in}{1.293992in}}%
\pgfpathlineto{\pgfqpoint{3.422971in}{1.274038in}}%
\pgfpathlineto{\pgfqpoint{3.573756in}{2.130002in}}%
\pgfpathlineto{\pgfqpoint{3.724542in}{1.698864in}}%
\pgfpathlineto{\pgfqpoint{3.724542in}{2.321509in}}%
\pgfpathlineto{\pgfqpoint{3.724542in}{2.321509in}}%
\pgfpathlineto{\pgfqpoint{3.573756in}{2.130036in}}%
\pgfpathlineto{\pgfqpoint{3.422971in}{2.510193in}}%
\pgfpathlineto{\pgfqpoint{3.272185in}{2.389086in}}%
\pgfpathlineto{\pgfqpoint{3.121399in}{2.169270in}}%
\pgfpathlineto{\pgfqpoint{2.970613in}{2.599874in}}%
\pgfpathlineto{\pgfqpoint{2.819827in}{2.440658in}}%
\pgfpathlineto{\pgfqpoint{2.669041in}{2.152773in}}%
\pgfpathlineto{\pgfqpoint{2.518255in}{2.369172in}}%
\pgfpathlineto{\pgfqpoint{2.367469in}{1.502488in}}%
\pgfpathlineto{\pgfqpoint{2.216683in}{1.251844in}}%
\pgfpathlineto{\pgfqpoint{2.065897in}{1.069972in}}%
\pgfpathlineto{\pgfqpoint{1.915111in}{1.075100in}}%
\pgfpathlineto{\pgfqpoint{1.764325in}{1.331518in}}%
\pgfpathlineto{\pgfqpoint{1.613539in}{0.947117in}}%
\pgfpathlineto{\pgfqpoint{1.462753in}{0.859356in}}%
\pgfpathlineto{\pgfqpoint{1.311967in}{0.775723in}}%
\pgfpathlineto{\pgfqpoint{1.161181in}{0.727084in}}%
\pgfpathlineto{\pgfqpoint{1.010395in}{0.779831in}}%
\pgfpathlineto{\pgfqpoint{0.859609in}{0.676388in}}%
\pgfpathlineto{\pgfqpoint{0.708823in}{0.718164in}}%
\pgfpathclose%
\pgfusepath{stroke,fill}%
\end{pgfscope}%
\begin{pgfscope}%
\pgfpathrectangle{\pgfqpoint{0.557986in}{0.438889in}}{\pgfqpoint{3.318403in}{2.263889in}}%
\pgfusepath{clip}%
\pgfsetbuttcap%
\pgfsetroundjoin%
\definecolor{currentfill}{rgb}{1.000000,0.498039,0.054902}%
\pgfsetfillcolor{currentfill}%
\pgfsetlinewidth{1.003750pt}%
\definecolor{currentstroke}{rgb}{1.000000,0.498039,0.054902}%
\pgfsetstrokecolor{currentstroke}%
\pgfsetdash{}{0pt}%
\pgfsys@defobject{currentmarker}{\pgfqpoint{-0.009821in}{-0.009821in}}{\pgfqpoint{0.009821in}{0.009821in}}{%
\pgfpathmoveto{\pgfqpoint{0.000000in}{-0.009821in}}%
\pgfpathcurveto{\pgfqpoint{0.002605in}{-0.009821in}}{\pgfqpoint{0.005103in}{-0.008786in}}{\pgfqpoint{0.006944in}{-0.006944in}}%
\pgfpathcurveto{\pgfqpoint{0.008786in}{-0.005103in}}{\pgfqpoint{0.009821in}{-0.002605in}}{\pgfqpoint{0.009821in}{0.000000in}}%
\pgfpathcurveto{\pgfqpoint{0.009821in}{0.002605in}}{\pgfqpoint{0.008786in}{0.005103in}}{\pgfqpoint{0.006944in}{0.006944in}}%
\pgfpathcurveto{\pgfqpoint{0.005103in}{0.008786in}}{\pgfqpoint{0.002605in}{0.009821in}}{\pgfqpoint{0.000000in}{0.009821in}}%
\pgfpathcurveto{\pgfqpoint{-0.002605in}{0.009821in}}{\pgfqpoint{-0.005103in}{0.008786in}}{\pgfqpoint{-0.006944in}{0.006944in}}%
\pgfpathcurveto{\pgfqpoint{-0.008786in}{0.005103in}}{\pgfqpoint{-0.009821in}{0.002605in}}{\pgfqpoint{-0.009821in}{0.000000in}}%
\pgfpathcurveto{\pgfqpoint{-0.009821in}{-0.002605in}}{\pgfqpoint{-0.008786in}{-0.005103in}}{\pgfqpoint{-0.006944in}{-0.006944in}}%
\pgfpathcurveto{\pgfqpoint{-0.005103in}{-0.008786in}}{\pgfqpoint{-0.002605in}{-0.009821in}}{\pgfqpoint{0.000000in}{-0.009821in}}%
\pgfpathclose%
\pgfusepath{stroke,fill}%
}%
\begin{pgfscope}%
\pgfsys@transformshift{0.708823in}{0.671527in}%
\pgfsys@useobject{currentmarker}{}%
\end{pgfscope}%
\begin{pgfscope}%
\pgfsys@transformshift{0.859609in}{0.678928in}%
\pgfsys@useobject{currentmarker}{}%
\end{pgfscope}%
\begin{pgfscope}%
\pgfsys@transformshift{1.010395in}{0.677951in}%
\pgfsys@useobject{currentmarker}{}%
\end{pgfscope}%
\begin{pgfscope}%
\pgfsys@transformshift{1.161181in}{0.717109in}%
\pgfsys@useobject{currentmarker}{}%
\end{pgfscope}%
\begin{pgfscope}%
\pgfsys@transformshift{1.311967in}{0.690322in}%
\pgfsys@useobject{currentmarker}{}%
\end{pgfscope}%
\begin{pgfscope}%
\pgfsys@transformshift{1.462753in}{0.709863in}%
\pgfsys@useobject{currentmarker}{}%
\end{pgfscope}%
\begin{pgfscope}%
\pgfsys@transformshift{1.613539in}{0.751513in}%
\pgfsys@useobject{currentmarker}{}%
\end{pgfscope}%
\begin{pgfscope}%
\pgfsys@transformshift{1.764325in}{0.820832in}%
\pgfsys@useobject{currentmarker}{}%
\end{pgfscope}%
\begin{pgfscope}%
\pgfsys@transformshift{1.915111in}{0.878569in}%
\pgfsys@useobject{currentmarker}{}%
\end{pgfscope}%
\begin{pgfscope}%
\pgfsys@transformshift{2.065897in}{0.914162in}%
\pgfsys@useobject{currentmarker}{}%
\end{pgfscope}%
\begin{pgfscope}%
\pgfsys@transformshift{2.216683in}{1.137951in}%
\pgfsys@useobject{currentmarker}{}%
\end{pgfscope}%
\begin{pgfscope}%
\pgfsys@transformshift{2.367469in}{1.175775in}%
\pgfsys@useobject{currentmarker}{}%
\end{pgfscope}%
\begin{pgfscope}%
\pgfsys@transformshift{2.518255in}{1.410782in}%
\pgfsys@useobject{currentmarker}{}%
\end{pgfscope}%
\begin{pgfscope}%
\pgfsys@transformshift{2.669041in}{1.700231in}%
\pgfsys@useobject{currentmarker}{}%
\end{pgfscope}%
\begin{pgfscope}%
\pgfsys@transformshift{2.819827in}{2.063352in}%
\pgfsys@useobject{currentmarker}{}%
\end{pgfscope}%
\begin{pgfscope}%
\pgfsys@transformshift{2.970613in}{2.111064in}%
\pgfsys@useobject{currentmarker}{}%
\end{pgfscope}%
\begin{pgfscope}%
\pgfsys@transformshift{3.121399in}{1.880269in}%
\pgfsys@useobject{currentmarker}{}%
\end{pgfscope}%
\begin{pgfscope}%
\pgfsys@transformshift{3.272185in}{1.837341in}%
\pgfsys@useobject{currentmarker}{}%
\end{pgfscope}%
\begin{pgfscope}%
\pgfsys@transformshift{3.422971in}{2.111034in}%
\pgfsys@useobject{currentmarker}{}%
\end{pgfscope}%
\begin{pgfscope}%
\pgfsys@transformshift{3.573756in}{2.111044in}%
\pgfsys@useobject{currentmarker}{}%
\end{pgfscope}%
\begin{pgfscope}%
\pgfsys@transformshift{3.724542in}{2.042036in}%
\pgfsys@useobject{currentmarker}{}%
\end{pgfscope}%
\end{pgfscope}%
\begin{pgfscope}%
\pgfpathrectangle{\pgfqpoint{0.557986in}{0.438889in}}{\pgfqpoint{3.318403in}{2.263889in}}%
\pgfusepath{clip}%
\pgfsetbuttcap%
\pgfsetroundjoin%
\definecolor{currentfill}{rgb}{1.000000,0.498039,0.054902}%
\pgfsetfillcolor{currentfill}%
\pgfsetfillopacity{0.300000}%
\pgfsetlinewidth{1.003750pt}%
\definecolor{currentstroke}{rgb}{1.000000,0.498039,0.054902}%
\pgfsetstrokecolor{currentstroke}%
\pgfsetstrokeopacity{0.300000}%
\pgfsetdash{}{0pt}%
\pgfpathmoveto{\pgfqpoint{0.708823in}{0.693524in}}%
\pgfpathlineto{\pgfqpoint{0.708823in}{0.649529in}}%
\pgfpathlineto{\pgfqpoint{0.859609in}{0.660819in}}%
\pgfpathlineto{\pgfqpoint{1.010395in}{0.667162in}}%
\pgfpathlineto{\pgfqpoint{1.161181in}{0.693979in}}%
\pgfpathlineto{\pgfqpoint{1.311967in}{0.623772in}}%
\pgfpathlineto{\pgfqpoint{1.462753in}{0.630075in}}%
\pgfpathlineto{\pgfqpoint{1.613539in}{0.671609in}}%
\pgfpathlineto{\pgfqpoint{1.764325in}{0.773559in}}%
\pgfpathlineto{\pgfqpoint{1.915111in}{0.641307in}}%
\pgfpathlineto{\pgfqpoint{2.065897in}{0.763457in}}%
\pgfpathlineto{\pgfqpoint{2.216683in}{0.876327in}}%
\pgfpathlineto{\pgfqpoint{2.367469in}{0.719466in}}%
\pgfpathlineto{\pgfqpoint{2.518255in}{0.615356in}}%
\pgfpathlineto{\pgfqpoint{2.669041in}{0.906592in}}%
\pgfpathlineto{\pgfqpoint{2.819827in}{1.987516in}}%
\pgfpathlineto{\pgfqpoint{2.970613in}{2.110907in}}%
\pgfpathlineto{\pgfqpoint{3.121399in}{1.280697in}}%
\pgfpathlineto{\pgfqpoint{3.272185in}{1.126399in}}%
\pgfpathlineto{\pgfqpoint{3.422971in}{2.110935in}}%
\pgfpathlineto{\pgfqpoint{3.573756in}{2.110952in}}%
\pgfpathlineto{\pgfqpoint{3.724542in}{1.862729in}}%
\pgfpathlineto{\pgfqpoint{3.724542in}{2.221343in}}%
\pgfpathlineto{\pgfqpoint{3.724542in}{2.221343in}}%
\pgfpathlineto{\pgfqpoint{3.573756in}{2.111137in}}%
\pgfpathlineto{\pgfqpoint{3.422971in}{2.111134in}}%
\pgfpathlineto{\pgfqpoint{3.272185in}{2.548283in}}%
\pgfpathlineto{\pgfqpoint{3.121399in}{2.479841in}}%
\pgfpathlineto{\pgfqpoint{2.970613in}{2.111221in}}%
\pgfpathlineto{\pgfqpoint{2.819827in}{2.139188in}}%
\pgfpathlineto{\pgfqpoint{2.669041in}{2.493871in}}%
\pgfpathlineto{\pgfqpoint{2.518255in}{2.206209in}}%
\pgfpathlineto{\pgfqpoint{2.367469in}{1.632083in}}%
\pgfpathlineto{\pgfqpoint{2.216683in}{1.399575in}}%
\pgfpathlineto{\pgfqpoint{2.065897in}{1.064867in}}%
\pgfpathlineto{\pgfqpoint{1.915111in}{1.115832in}}%
\pgfpathlineto{\pgfqpoint{1.764325in}{0.868105in}}%
\pgfpathlineto{\pgfqpoint{1.613539in}{0.831417in}}%
\pgfpathlineto{\pgfqpoint{1.462753in}{0.789651in}}%
\pgfpathlineto{\pgfqpoint{1.311967in}{0.756871in}}%
\pgfpathlineto{\pgfqpoint{1.161181in}{0.740239in}}%
\pgfpathlineto{\pgfqpoint{1.010395in}{0.688740in}}%
\pgfpathlineto{\pgfqpoint{0.859609in}{0.697037in}}%
\pgfpathlineto{\pgfqpoint{0.708823in}{0.693524in}}%
\pgfpathclose%
\pgfusepath{stroke,fill}%
\end{pgfscope}%
\begin{pgfscope}%
\pgfpathrectangle{\pgfqpoint{0.557986in}{0.438889in}}{\pgfqpoint{3.318403in}{2.263889in}}%
\pgfusepath{clip}%
\pgfsetbuttcap%
\pgfsetroundjoin%
\definecolor{currentfill}{rgb}{0.172549,0.627451,0.172549}%
\pgfsetfillcolor{currentfill}%
\pgfsetlinewidth{1.003750pt}%
\definecolor{currentstroke}{rgb}{0.172549,0.627451,0.172549}%
\pgfsetstrokecolor{currentstroke}%
\pgfsetdash{}{0pt}%
\pgfsys@defobject{currentmarker}{\pgfqpoint{-0.009821in}{-0.009821in}}{\pgfqpoint{0.009821in}{0.009821in}}{%
\pgfpathmoveto{\pgfqpoint{0.000000in}{-0.009821in}}%
\pgfpathcurveto{\pgfqpoint{0.002605in}{-0.009821in}}{\pgfqpoint{0.005103in}{-0.008786in}}{\pgfqpoint{0.006944in}{-0.006944in}}%
\pgfpathcurveto{\pgfqpoint{0.008786in}{-0.005103in}}{\pgfqpoint{0.009821in}{-0.002605in}}{\pgfqpoint{0.009821in}{0.000000in}}%
\pgfpathcurveto{\pgfqpoint{0.009821in}{0.002605in}}{\pgfqpoint{0.008786in}{0.005103in}}{\pgfqpoint{0.006944in}{0.006944in}}%
\pgfpathcurveto{\pgfqpoint{0.005103in}{0.008786in}}{\pgfqpoint{0.002605in}{0.009821in}}{\pgfqpoint{0.000000in}{0.009821in}}%
\pgfpathcurveto{\pgfqpoint{-0.002605in}{0.009821in}}{\pgfqpoint{-0.005103in}{0.008786in}}{\pgfqpoint{-0.006944in}{0.006944in}}%
\pgfpathcurveto{\pgfqpoint{-0.008786in}{0.005103in}}{\pgfqpoint{-0.009821in}{0.002605in}}{\pgfqpoint{-0.009821in}{0.000000in}}%
\pgfpathcurveto{\pgfqpoint{-0.009821in}{-0.002605in}}{\pgfqpoint{-0.008786in}{-0.005103in}}{\pgfqpoint{-0.006944in}{-0.006944in}}%
\pgfpathcurveto{\pgfqpoint{-0.005103in}{-0.008786in}}{\pgfqpoint{-0.002605in}{-0.009821in}}{\pgfqpoint{0.000000in}{-0.009821in}}%
\pgfpathclose%
\pgfusepath{stroke,fill}%
}%
\begin{pgfscope}%
\pgfsys@transformshift{0.708823in}{0.680541in}%
\pgfsys@useobject{currentmarker}{}%
\end{pgfscope}%
\begin{pgfscope}%
\pgfsys@transformshift{0.859609in}{0.679997in}%
\pgfsys@useobject{currentmarker}{}%
\end{pgfscope}%
\begin{pgfscope}%
\pgfsys@transformshift{1.010395in}{0.708723in}%
\pgfsys@useobject{currentmarker}{}%
\end{pgfscope}%
\begin{pgfscope}%
\pgfsys@transformshift{1.161181in}{0.715009in}%
\pgfsys@useobject{currentmarker}{}%
\end{pgfscope}%
\begin{pgfscope}%
\pgfsys@transformshift{1.311967in}{0.742244in}%
\pgfsys@useobject{currentmarker}{}%
\end{pgfscope}%
\begin{pgfscope}%
\pgfsys@transformshift{1.462753in}{0.765060in}%
\pgfsys@useobject{currentmarker}{}%
\end{pgfscope}%
\begin{pgfscope}%
\pgfsys@transformshift{1.613539in}{0.762023in}%
\pgfsys@useobject{currentmarker}{}%
\end{pgfscope}%
\begin{pgfscope}%
\pgfsys@transformshift{1.764325in}{0.857474in}%
\pgfsys@useobject{currentmarker}{}%
\end{pgfscope}%
\begin{pgfscope}%
\pgfsys@transformshift{1.915111in}{0.805723in}%
\pgfsys@useobject{currentmarker}{}%
\end{pgfscope}%
\begin{pgfscope}%
\pgfsys@transformshift{2.065897in}{0.949944in}%
\pgfsys@useobject{currentmarker}{}%
\end{pgfscope}%
\begin{pgfscope}%
\pgfsys@transformshift{2.216683in}{1.102171in}%
\pgfsys@useobject{currentmarker}{}%
\end{pgfscope}%
\begin{pgfscope}%
\pgfsys@transformshift{2.367469in}{0.947582in}%
\pgfsys@useobject{currentmarker}{}%
\end{pgfscope}%
\begin{pgfscope}%
\pgfsys@transformshift{2.518255in}{1.030691in}%
\pgfsys@useobject{currentmarker}{}%
\end{pgfscope}%
\begin{pgfscope}%
\pgfsys@transformshift{2.669041in}{1.292833in}%
\pgfsys@useobject{currentmarker}{}%
\end{pgfscope}%
\begin{pgfscope}%
\pgfsys@transformshift{2.819827in}{1.981429in}%
\pgfsys@useobject{currentmarker}{}%
\end{pgfscope}%
\begin{pgfscope}%
\pgfsys@transformshift{2.970613in}{1.986670in}%
\pgfsys@useobject{currentmarker}{}%
\end{pgfscope}%
\begin{pgfscope}%
\pgfsys@transformshift{3.121399in}{1.777189in}%
\pgfsys@useobject{currentmarker}{}%
\end{pgfscope}%
\begin{pgfscope}%
\pgfsys@transformshift{3.272185in}{1.930304in}%
\pgfsys@useobject{currentmarker}{}%
\end{pgfscope}%
\begin{pgfscope}%
\pgfsys@transformshift{3.422971in}{1.999769in}%
\pgfsys@useobject{currentmarker}{}%
\end{pgfscope}%
\begin{pgfscope}%
\pgfsys@transformshift{3.573756in}{1.999389in}%
\pgfsys@useobject{currentmarker}{}%
\end{pgfscope}%
\begin{pgfscope}%
\pgfsys@transformshift{3.724542in}{1.999199in}%
\pgfsys@useobject{currentmarker}{}%
\end{pgfscope}%
\end{pgfscope}%
\begin{pgfscope}%
\pgfpathrectangle{\pgfqpoint{0.557986in}{0.438889in}}{\pgfqpoint{3.318403in}{2.263889in}}%
\pgfusepath{clip}%
\pgfsetbuttcap%
\pgfsetroundjoin%
\definecolor{currentfill}{rgb}{0.172549,0.627451,0.172549}%
\pgfsetfillcolor{currentfill}%
\pgfsetfillopacity{0.300000}%
\pgfsetlinewidth{1.003750pt}%
\definecolor{currentstroke}{rgb}{0.172549,0.627451,0.172549}%
\pgfsetstrokecolor{currentstroke}%
\pgfsetstrokeopacity{0.300000}%
\pgfsetdash{}{0pt}%
\pgfpathmoveto{\pgfqpoint{0.708823in}{0.722498in}}%
\pgfpathlineto{\pgfqpoint{0.708823in}{0.638585in}}%
\pgfpathlineto{\pgfqpoint{0.859609in}{0.653403in}}%
\pgfpathlineto{\pgfqpoint{1.010395in}{0.660361in}}%
\pgfpathlineto{\pgfqpoint{1.161181in}{0.632324in}}%
\pgfpathlineto{\pgfqpoint{1.311967in}{0.689922in}}%
\pgfpathlineto{\pgfqpoint{1.462753in}{0.739996in}}%
\pgfpathlineto{\pgfqpoint{1.613539in}{0.721636in}}%
\pgfpathlineto{\pgfqpoint{1.764325in}{0.546675in}}%
\pgfpathlineto{\pgfqpoint{1.915111in}{0.641239in}}%
\pgfpathlineto{\pgfqpoint{2.065897in}{0.637163in}}%
\pgfpathlineto{\pgfqpoint{2.216683in}{0.759014in}}%
\pgfpathlineto{\pgfqpoint{2.367469in}{0.627167in}}%
\pgfpathlineto{\pgfqpoint{2.518255in}{0.848108in}}%
\pgfpathlineto{\pgfqpoint{2.669041in}{0.631905in}}%
\pgfpathlineto{\pgfqpoint{2.819827in}{1.945514in}}%
\pgfpathlineto{\pgfqpoint{2.970613in}{1.953815in}}%
\pgfpathlineto{\pgfqpoint{3.121399in}{1.201441in}}%
\pgfpathlineto{\pgfqpoint{3.272185in}{1.750920in}}%
\pgfpathlineto{\pgfqpoint{3.422971in}{1.998616in}}%
\pgfpathlineto{\pgfqpoint{3.573756in}{1.998343in}}%
\pgfpathlineto{\pgfqpoint{3.724542in}{1.998258in}}%
\pgfpathlineto{\pgfqpoint{3.724542in}{2.000139in}}%
\pgfpathlineto{\pgfqpoint{3.724542in}{2.000139in}}%
\pgfpathlineto{\pgfqpoint{3.573756in}{2.000435in}}%
\pgfpathlineto{\pgfqpoint{3.422971in}{2.000922in}}%
\pgfpathlineto{\pgfqpoint{3.272185in}{2.109688in}}%
\pgfpathlineto{\pgfqpoint{3.121399in}{2.352938in}}%
\pgfpathlineto{\pgfqpoint{2.970613in}{2.019525in}}%
\pgfpathlineto{\pgfqpoint{2.819827in}{2.017344in}}%
\pgfpathlineto{\pgfqpoint{2.669041in}{1.953761in}}%
\pgfpathlineto{\pgfqpoint{2.518255in}{1.213275in}}%
\pgfpathlineto{\pgfqpoint{2.367469in}{1.267997in}}%
\pgfpathlineto{\pgfqpoint{2.216683in}{1.445327in}}%
\pgfpathlineto{\pgfqpoint{2.065897in}{1.262724in}}%
\pgfpathlineto{\pgfqpoint{1.915111in}{0.970207in}}%
\pgfpathlineto{\pgfqpoint{1.764325in}{1.168273in}}%
\pgfpathlineto{\pgfqpoint{1.613539in}{0.802409in}}%
\pgfpathlineto{\pgfqpoint{1.462753in}{0.790125in}}%
\pgfpathlineto{\pgfqpoint{1.311967in}{0.794566in}}%
\pgfpathlineto{\pgfqpoint{1.161181in}{0.797694in}}%
\pgfpathlineto{\pgfqpoint{1.010395in}{0.757084in}}%
\pgfpathlineto{\pgfqpoint{0.859609in}{0.706590in}}%
\pgfpathlineto{\pgfqpoint{0.708823in}{0.722498in}}%
\pgfpathclose%
\pgfusepath{stroke,fill}%
\end{pgfscope}%
\begin{pgfscope}%
\pgfpathrectangle{\pgfqpoint{0.557986in}{0.438889in}}{\pgfqpoint{3.318403in}{2.263889in}}%
\pgfusepath{clip}%
\pgfsetbuttcap%
\pgfsetroundjoin%
\definecolor{currentfill}{rgb}{0.839216,0.152941,0.156863}%
\pgfsetfillcolor{currentfill}%
\pgfsetlinewidth{1.003750pt}%
\definecolor{currentstroke}{rgb}{0.839216,0.152941,0.156863}%
\pgfsetstrokecolor{currentstroke}%
\pgfsetdash{}{0pt}%
\pgfsys@defobject{currentmarker}{\pgfqpoint{-0.009821in}{-0.009821in}}{\pgfqpoint{0.009821in}{0.009821in}}{%
\pgfpathmoveto{\pgfqpoint{0.000000in}{-0.009821in}}%
\pgfpathcurveto{\pgfqpoint{0.002605in}{-0.009821in}}{\pgfqpoint{0.005103in}{-0.008786in}}{\pgfqpoint{0.006944in}{-0.006944in}}%
\pgfpathcurveto{\pgfqpoint{0.008786in}{-0.005103in}}{\pgfqpoint{0.009821in}{-0.002605in}}{\pgfqpoint{0.009821in}{0.000000in}}%
\pgfpathcurveto{\pgfqpoint{0.009821in}{0.002605in}}{\pgfqpoint{0.008786in}{0.005103in}}{\pgfqpoint{0.006944in}{0.006944in}}%
\pgfpathcurveto{\pgfqpoint{0.005103in}{0.008786in}}{\pgfqpoint{0.002605in}{0.009821in}}{\pgfqpoint{0.000000in}{0.009821in}}%
\pgfpathcurveto{\pgfqpoint{-0.002605in}{0.009821in}}{\pgfqpoint{-0.005103in}{0.008786in}}{\pgfqpoint{-0.006944in}{0.006944in}}%
\pgfpathcurveto{\pgfqpoint{-0.008786in}{0.005103in}}{\pgfqpoint{-0.009821in}{0.002605in}}{\pgfqpoint{-0.009821in}{0.000000in}}%
\pgfpathcurveto{\pgfqpoint{-0.009821in}{-0.002605in}}{\pgfqpoint{-0.008786in}{-0.005103in}}{\pgfqpoint{-0.006944in}{-0.006944in}}%
\pgfpathcurveto{\pgfqpoint{-0.005103in}{-0.008786in}}{\pgfqpoint{-0.002605in}{-0.009821in}}{\pgfqpoint{0.000000in}{-0.009821in}}%
\pgfpathclose%
\pgfusepath{stroke,fill}%
}%
\begin{pgfscope}%
\pgfsys@transformshift{0.708823in}{0.666813in}%
\pgfsys@useobject{currentmarker}{}%
\end{pgfscope}%
\begin{pgfscope}%
\pgfsys@transformshift{0.859609in}{0.674807in}%
\pgfsys@useobject{currentmarker}{}%
\end{pgfscope}%
\begin{pgfscope}%
\pgfsys@transformshift{1.010395in}{0.685725in}%
\pgfsys@useobject{currentmarker}{}%
\end{pgfscope}%
\begin{pgfscope}%
\pgfsys@transformshift{1.161181in}{0.688401in}%
\pgfsys@useobject{currentmarker}{}%
\end{pgfscope}%
\begin{pgfscope}%
\pgfsys@transformshift{1.311967in}{0.700754in}%
\pgfsys@useobject{currentmarker}{}%
\end{pgfscope}%
\begin{pgfscope}%
\pgfsys@transformshift{1.462753in}{0.708376in}%
\pgfsys@useobject{currentmarker}{}%
\end{pgfscope}%
\begin{pgfscope}%
\pgfsys@transformshift{1.613539in}{0.704879in}%
\pgfsys@useobject{currentmarker}{}%
\end{pgfscope}%
\begin{pgfscope}%
\pgfsys@transformshift{1.764325in}{0.748116in}%
\pgfsys@useobject{currentmarker}{}%
\end{pgfscope}%
\begin{pgfscope}%
\pgfsys@transformshift{1.915111in}{0.837081in}%
\pgfsys@useobject{currentmarker}{}%
\end{pgfscope}%
\begin{pgfscope}%
\pgfsys@transformshift{2.065897in}{0.831179in}%
\pgfsys@useobject{currentmarker}{}%
\end{pgfscope}%
\begin{pgfscope}%
\pgfsys@transformshift{2.216683in}{0.819256in}%
\pgfsys@useobject{currentmarker}{}%
\end{pgfscope}%
\begin{pgfscope}%
\pgfsys@transformshift{2.367469in}{0.771371in}%
\pgfsys@useobject{currentmarker}{}%
\end{pgfscope}%
\begin{pgfscope}%
\pgfsys@transformshift{2.518255in}{0.747130in}%
\pgfsys@useobject{currentmarker}{}%
\end{pgfscope}%
\begin{pgfscope}%
\pgfsys@transformshift{2.669041in}{0.775828in}%
\pgfsys@useobject{currentmarker}{}%
\end{pgfscope}%
\begin{pgfscope}%
\pgfsys@transformshift{2.819827in}{0.820274in}%
\pgfsys@useobject{currentmarker}{}%
\end{pgfscope}%
\begin{pgfscope}%
\pgfsys@transformshift{2.970613in}{0.805079in}%
\pgfsys@useobject{currentmarker}{}%
\end{pgfscope}%
\begin{pgfscope}%
\pgfsys@transformshift{3.121399in}{0.804892in}%
\pgfsys@useobject{currentmarker}{}%
\end{pgfscope}%
\begin{pgfscope}%
\pgfsys@transformshift{3.272185in}{0.798488in}%
\pgfsys@useobject{currentmarker}{}%
\end{pgfscope}%
\begin{pgfscope}%
\pgfsys@transformshift{3.422971in}{0.787199in}%
\pgfsys@useobject{currentmarker}{}%
\end{pgfscope}%
\begin{pgfscope}%
\pgfsys@transformshift{3.573756in}{0.794933in}%
\pgfsys@useobject{currentmarker}{}%
\end{pgfscope}%
\begin{pgfscope}%
\pgfsys@transformshift{3.724542in}{0.795994in}%
\pgfsys@useobject{currentmarker}{}%
\end{pgfscope}%
\end{pgfscope}%
\begin{pgfscope}%
\pgfpathrectangle{\pgfqpoint{0.557986in}{0.438889in}}{\pgfqpoint{3.318403in}{2.263889in}}%
\pgfusepath{clip}%
\pgfsetbuttcap%
\pgfsetroundjoin%
\definecolor{currentfill}{rgb}{0.839216,0.152941,0.156863}%
\pgfsetfillcolor{currentfill}%
\pgfsetfillopacity{0.300000}%
\pgfsetlinewidth{1.003750pt}%
\definecolor{currentstroke}{rgb}{0.839216,0.152941,0.156863}%
\pgfsetstrokecolor{currentstroke}%
\pgfsetstrokeopacity{0.300000}%
\pgfsetdash{}{0pt}%
\pgfpathmoveto{\pgfqpoint{0.708823in}{0.676935in}}%
\pgfpathlineto{\pgfqpoint{0.708823in}{0.656690in}}%
\pgfpathlineto{\pgfqpoint{0.859609in}{0.656720in}}%
\pgfpathlineto{\pgfqpoint{1.010395in}{0.646293in}}%
\pgfpathlineto{\pgfqpoint{1.161181in}{0.653408in}}%
\pgfpathlineto{\pgfqpoint{1.311967in}{0.657112in}}%
\pgfpathlineto{\pgfqpoint{1.462753in}{0.662949in}}%
\pgfpathlineto{\pgfqpoint{1.613539in}{0.657412in}}%
\pgfpathlineto{\pgfqpoint{1.764325in}{0.648065in}}%
\pgfpathlineto{\pgfqpoint{1.915111in}{0.783094in}}%
\pgfpathlineto{\pgfqpoint{2.065897in}{0.733145in}}%
\pgfpathlineto{\pgfqpoint{2.216683in}{0.687744in}}%
\pgfpathlineto{\pgfqpoint{2.367469in}{0.679564in}}%
\pgfpathlineto{\pgfqpoint{2.518255in}{0.707235in}}%
\pgfpathlineto{\pgfqpoint{2.669041in}{0.717788in}}%
\pgfpathlineto{\pgfqpoint{2.819827in}{0.797403in}}%
\pgfpathlineto{\pgfqpoint{2.970613in}{0.759904in}}%
\pgfpathlineto{\pgfqpoint{3.121399in}{0.783153in}}%
\pgfpathlineto{\pgfqpoint{3.272185in}{0.777641in}}%
\pgfpathlineto{\pgfqpoint{3.422971in}{0.782524in}}%
\pgfpathlineto{\pgfqpoint{3.573756in}{0.787959in}}%
\pgfpathlineto{\pgfqpoint{3.724542in}{0.790182in}}%
\pgfpathlineto{\pgfqpoint{3.724542in}{0.801806in}}%
\pgfpathlineto{\pgfqpoint{3.724542in}{0.801806in}}%
\pgfpathlineto{\pgfqpoint{3.573756in}{0.801907in}}%
\pgfpathlineto{\pgfqpoint{3.422971in}{0.791874in}}%
\pgfpathlineto{\pgfqpoint{3.272185in}{0.819336in}}%
\pgfpathlineto{\pgfqpoint{3.121399in}{0.826630in}}%
\pgfpathlineto{\pgfqpoint{2.970613in}{0.850254in}}%
\pgfpathlineto{\pgfqpoint{2.819827in}{0.843144in}}%
\pgfpathlineto{\pgfqpoint{2.669041in}{0.833867in}}%
\pgfpathlineto{\pgfqpoint{2.518255in}{0.787025in}}%
\pgfpathlineto{\pgfqpoint{2.367469in}{0.863178in}}%
\pgfpathlineto{\pgfqpoint{2.216683in}{0.950768in}}%
\pgfpathlineto{\pgfqpoint{2.065897in}{0.929213in}}%
\pgfpathlineto{\pgfqpoint{1.915111in}{0.891069in}}%
\pgfpathlineto{\pgfqpoint{1.764325in}{0.848167in}}%
\pgfpathlineto{\pgfqpoint{1.613539in}{0.752346in}}%
\pgfpathlineto{\pgfqpoint{1.462753in}{0.753802in}}%
\pgfpathlineto{\pgfqpoint{1.311967in}{0.744396in}}%
\pgfpathlineto{\pgfqpoint{1.161181in}{0.723394in}}%
\pgfpathlineto{\pgfqpoint{1.010395in}{0.725157in}}%
\pgfpathlineto{\pgfqpoint{0.859609in}{0.692895in}}%
\pgfpathlineto{\pgfqpoint{0.708823in}{0.676935in}}%
\pgfpathclose%
\pgfusepath{stroke,fill}%
\end{pgfscope}%
\begin{pgfscope}%
\pgfpathrectangle{\pgfqpoint{0.557986in}{0.438889in}}{\pgfqpoint{3.318403in}{2.263889in}}%
\pgfusepath{clip}%
\pgfsetbuttcap%
\pgfsetroundjoin%
\definecolor{currentfill}{rgb}{0.580392,0.403922,0.741176}%
\pgfsetfillcolor{currentfill}%
\pgfsetlinewidth{1.003750pt}%
\definecolor{currentstroke}{rgb}{0.580392,0.403922,0.741176}%
\pgfsetstrokecolor{currentstroke}%
\pgfsetdash{}{0pt}%
\pgfsys@defobject{currentmarker}{\pgfqpoint{-0.009821in}{-0.009821in}}{\pgfqpoint{0.009821in}{0.009821in}}{%
\pgfpathmoveto{\pgfqpoint{0.000000in}{-0.009821in}}%
\pgfpathcurveto{\pgfqpoint{0.002605in}{-0.009821in}}{\pgfqpoint{0.005103in}{-0.008786in}}{\pgfqpoint{0.006944in}{-0.006944in}}%
\pgfpathcurveto{\pgfqpoint{0.008786in}{-0.005103in}}{\pgfqpoint{0.009821in}{-0.002605in}}{\pgfqpoint{0.009821in}{0.000000in}}%
\pgfpathcurveto{\pgfqpoint{0.009821in}{0.002605in}}{\pgfqpoint{0.008786in}{0.005103in}}{\pgfqpoint{0.006944in}{0.006944in}}%
\pgfpathcurveto{\pgfqpoint{0.005103in}{0.008786in}}{\pgfqpoint{0.002605in}{0.009821in}}{\pgfqpoint{0.000000in}{0.009821in}}%
\pgfpathcurveto{\pgfqpoint{-0.002605in}{0.009821in}}{\pgfqpoint{-0.005103in}{0.008786in}}{\pgfqpoint{-0.006944in}{0.006944in}}%
\pgfpathcurveto{\pgfqpoint{-0.008786in}{0.005103in}}{\pgfqpoint{-0.009821in}{0.002605in}}{\pgfqpoint{-0.009821in}{0.000000in}}%
\pgfpathcurveto{\pgfqpoint{-0.009821in}{-0.002605in}}{\pgfqpoint{-0.008786in}{-0.005103in}}{\pgfqpoint{-0.006944in}{-0.006944in}}%
\pgfpathcurveto{\pgfqpoint{-0.005103in}{-0.008786in}}{\pgfqpoint{-0.002605in}{-0.009821in}}{\pgfqpoint{0.000000in}{-0.009821in}}%
\pgfpathclose%
\pgfusepath{stroke,fill}%
}%
\begin{pgfscope}%
\pgfsys@transformshift{0.708823in}{0.671586in}%
\pgfsys@useobject{currentmarker}{}%
\end{pgfscope}%
\begin{pgfscope}%
\pgfsys@transformshift{0.859609in}{0.678810in}%
\pgfsys@useobject{currentmarker}{}%
\end{pgfscope}%
\begin{pgfscope}%
\pgfsys@transformshift{1.010395in}{0.676971in}%
\pgfsys@useobject{currentmarker}{}%
\end{pgfscope}%
\begin{pgfscope}%
\pgfsys@transformshift{1.161181in}{0.665443in}%
\pgfsys@useobject{currentmarker}{}%
\end{pgfscope}%
\begin{pgfscope}%
\pgfsys@transformshift{1.311967in}{0.716756in}%
\pgfsys@useobject{currentmarker}{}%
\end{pgfscope}%
\begin{pgfscope}%
\pgfsys@transformshift{1.462753in}{0.711667in}%
\pgfsys@useobject{currentmarker}{}%
\end{pgfscope}%
\begin{pgfscope}%
\pgfsys@transformshift{1.613539in}{0.707580in}%
\pgfsys@useobject{currentmarker}{}%
\end{pgfscope}%
\begin{pgfscope}%
\pgfsys@transformshift{1.764325in}{0.747807in}%
\pgfsys@useobject{currentmarker}{}%
\end{pgfscope}%
\begin{pgfscope}%
\pgfsys@transformshift{1.915111in}{0.723403in}%
\pgfsys@useobject{currentmarker}{}%
\end{pgfscope}%
\begin{pgfscope}%
\pgfsys@transformshift{2.065897in}{0.703011in}%
\pgfsys@useobject{currentmarker}{}%
\end{pgfscope}%
\begin{pgfscope}%
\pgfsys@transformshift{2.216683in}{0.713746in}%
\pgfsys@useobject{currentmarker}{}%
\end{pgfscope}%
\begin{pgfscope}%
\pgfsys@transformshift{2.367469in}{0.702142in}%
\pgfsys@useobject{currentmarker}{}%
\end{pgfscope}%
\begin{pgfscope}%
\pgfsys@transformshift{2.518255in}{0.708099in}%
\pgfsys@useobject{currentmarker}{}%
\end{pgfscope}%
\begin{pgfscope}%
\pgfsys@transformshift{2.669041in}{0.714705in}%
\pgfsys@useobject{currentmarker}{}%
\end{pgfscope}%
\begin{pgfscope}%
\pgfsys@transformshift{2.819827in}{0.707268in}%
\pgfsys@useobject{currentmarker}{}%
\end{pgfscope}%
\begin{pgfscope}%
\pgfsys@transformshift{2.970613in}{0.707771in}%
\pgfsys@useobject{currentmarker}{}%
\end{pgfscope}%
\begin{pgfscope}%
\pgfsys@transformshift{3.121399in}{0.709327in}%
\pgfsys@useobject{currentmarker}{}%
\end{pgfscope}%
\begin{pgfscope}%
\pgfsys@transformshift{3.272185in}{0.709310in}%
\pgfsys@useobject{currentmarker}{}%
\end{pgfscope}%
\begin{pgfscope}%
\pgfsys@transformshift{3.422971in}{0.709991in}%
\pgfsys@useobject{currentmarker}{}%
\end{pgfscope}%
\begin{pgfscope}%
\pgfsys@transformshift{3.573756in}{0.709863in}%
\pgfsys@useobject{currentmarker}{}%
\end{pgfscope}%
\begin{pgfscope}%
\pgfsys@transformshift{3.724542in}{0.710772in}%
\pgfsys@useobject{currentmarker}{}%
\end{pgfscope}%
\end{pgfscope}%
\begin{pgfscope}%
\pgfpathrectangle{\pgfqpoint{0.557986in}{0.438889in}}{\pgfqpoint{3.318403in}{2.263889in}}%
\pgfusepath{clip}%
\pgfsetbuttcap%
\pgfsetroundjoin%
\definecolor{currentfill}{rgb}{0.580392,0.403922,0.741176}%
\pgfsetfillcolor{currentfill}%
\pgfsetfillopacity{0.300000}%
\pgfsetlinewidth{1.003750pt}%
\definecolor{currentstroke}{rgb}{0.580392,0.403922,0.741176}%
\pgfsetstrokecolor{currentstroke}%
\pgfsetstrokeopacity{0.300000}%
\pgfsetdash{}{0pt}%
\pgfpathmoveto{\pgfqpoint{0.708823in}{0.687294in}}%
\pgfpathlineto{\pgfqpoint{0.708823in}{0.655879in}}%
\pgfpathlineto{\pgfqpoint{0.859609in}{0.648051in}}%
\pgfpathlineto{\pgfqpoint{1.010395in}{0.655895in}}%
\pgfpathlineto{\pgfqpoint{1.161181in}{0.656645in}}%
\pgfpathlineto{\pgfqpoint{1.311967in}{0.660543in}}%
\pgfpathlineto{\pgfqpoint{1.462753in}{0.647994in}}%
\pgfpathlineto{\pgfqpoint{1.613539in}{0.666743in}}%
\pgfpathlineto{\pgfqpoint{1.764325in}{0.677808in}}%
\pgfpathlineto{\pgfqpoint{1.915111in}{0.658999in}}%
\pgfpathlineto{\pgfqpoint{2.065897in}{0.689818in}}%
\pgfpathlineto{\pgfqpoint{2.216683in}{0.699137in}}%
\pgfpathlineto{\pgfqpoint{2.367469in}{0.682851in}}%
\pgfpathlineto{\pgfqpoint{2.518255in}{0.697725in}}%
\pgfpathlineto{\pgfqpoint{2.669041in}{0.705201in}}%
\pgfpathlineto{\pgfqpoint{2.819827in}{0.702000in}}%
\pgfpathlineto{\pgfqpoint{2.970613in}{0.701992in}}%
\pgfpathlineto{\pgfqpoint{3.121399in}{0.707062in}}%
\pgfpathlineto{\pgfqpoint{3.272185in}{0.706681in}}%
\pgfpathlineto{\pgfqpoint{3.422971in}{0.708752in}}%
\pgfpathlineto{\pgfqpoint{3.573756in}{0.708809in}}%
\pgfpathlineto{\pgfqpoint{3.724542in}{0.710089in}}%
\pgfpathlineto{\pgfqpoint{3.724542in}{0.711455in}}%
\pgfpathlineto{\pgfqpoint{3.724542in}{0.711455in}}%
\pgfpathlineto{\pgfqpoint{3.573756in}{0.710918in}}%
\pgfpathlineto{\pgfqpoint{3.422971in}{0.711230in}}%
\pgfpathlineto{\pgfqpoint{3.272185in}{0.711940in}}%
\pgfpathlineto{\pgfqpoint{3.121399in}{0.711593in}}%
\pgfpathlineto{\pgfqpoint{2.970613in}{0.713550in}}%
\pgfpathlineto{\pgfqpoint{2.819827in}{0.712536in}}%
\pgfpathlineto{\pgfqpoint{2.669041in}{0.724210in}}%
\pgfpathlineto{\pgfqpoint{2.518255in}{0.718473in}}%
\pgfpathlineto{\pgfqpoint{2.367469in}{0.721432in}}%
\pgfpathlineto{\pgfqpoint{2.216683in}{0.728356in}}%
\pgfpathlineto{\pgfqpoint{2.065897in}{0.716204in}}%
\pgfpathlineto{\pgfqpoint{1.915111in}{0.787808in}}%
\pgfpathlineto{\pgfqpoint{1.764325in}{0.817805in}}%
\pgfpathlineto{\pgfqpoint{1.613539in}{0.748418in}}%
\pgfpathlineto{\pgfqpoint{1.462753in}{0.775339in}}%
\pgfpathlineto{\pgfqpoint{1.311967in}{0.772968in}}%
\pgfpathlineto{\pgfqpoint{1.161181in}{0.674241in}}%
\pgfpathlineto{\pgfqpoint{1.010395in}{0.698046in}}%
\pgfpathlineto{\pgfqpoint{0.859609in}{0.709569in}}%
\pgfpathlineto{\pgfqpoint{0.708823in}{0.687294in}}%
\pgfpathclose%
\pgfusepath{stroke,fill}%
\end{pgfscope}%
\begin{pgfscope}%
\pgfsetbuttcap%
\pgfsetroundjoin%
\definecolor{currentfill}{rgb}{0.000000,0.000000,0.000000}%
\pgfsetfillcolor{currentfill}%
\pgfsetlinewidth{0.803000pt}%
\definecolor{currentstroke}{rgb}{0.000000,0.000000,0.000000}%
\pgfsetstrokecolor{currentstroke}%
\pgfsetdash{}{0pt}%
\pgfsys@defobject{currentmarker}{\pgfqpoint{0.000000in}{-0.048611in}}{\pgfqpoint{0.000000in}{0.000000in}}{%
\pgfpathmoveto{\pgfqpoint{0.000000in}{0.000000in}}%
\pgfpathlineto{\pgfqpoint{0.000000in}{-0.048611in}}%
\pgfusepath{stroke,fill}%
}%
\begin{pgfscope}%
\pgfsys@transformshift{1.071951in}{0.438889in}%
\pgfsys@useobject{currentmarker}{}%
\end{pgfscope}%
\end{pgfscope}%
\begin{pgfscope}%
\definecolor{textcolor}{rgb}{0.000000,0.000000,0.000000}%
\pgfsetstrokecolor{textcolor}%
\pgfsetfillcolor{textcolor}%
\pgftext[x=1.071951in,y=0.341667in,,top]{\color{textcolor}\sffamily\fontsize{10.000000}{12.000000}\selectfont \(\displaystyle {10^{4}}\)}%
\end{pgfscope}%
\begin{pgfscope}%
\pgfsetbuttcap%
\pgfsetroundjoin%
\definecolor{currentfill}{rgb}{0.000000,0.000000,0.000000}%
\pgfsetfillcolor{currentfill}%
\pgfsetlinewidth{0.803000pt}%
\definecolor{currentstroke}{rgb}{0.000000,0.000000,0.000000}%
\pgfsetstrokecolor{currentstroke}%
\pgfsetdash{}{0pt}%
\pgfsys@defobject{currentmarker}{\pgfqpoint{0.000000in}{-0.048611in}}{\pgfqpoint{0.000000in}{0.000000in}}{%
\pgfpathmoveto{\pgfqpoint{0.000000in}{0.000000in}}%
\pgfpathlineto{\pgfqpoint{0.000000in}{-0.048611in}}%
\pgfusepath{stroke,fill}%
}%
\begin{pgfscope}%
\pgfsys@transformshift{1.675095in}{0.438889in}%
\pgfsys@useobject{currentmarker}{}%
\end{pgfscope}%
\end{pgfscope}%
\begin{pgfscope}%
\definecolor{textcolor}{rgb}{0.000000,0.000000,0.000000}%
\pgfsetstrokecolor{textcolor}%
\pgfsetfillcolor{textcolor}%
\pgftext[x=1.675095in,y=0.341667in,,top]{\color{textcolor}\sffamily\fontsize{10.000000}{12.000000}\selectfont \(\displaystyle {10^{5}}\)}%
\end{pgfscope}%
\begin{pgfscope}%
\pgfsetbuttcap%
\pgfsetroundjoin%
\definecolor{currentfill}{rgb}{0.000000,0.000000,0.000000}%
\pgfsetfillcolor{currentfill}%
\pgfsetlinewidth{0.803000pt}%
\definecolor{currentstroke}{rgb}{0.000000,0.000000,0.000000}%
\pgfsetstrokecolor{currentstroke}%
\pgfsetdash{}{0pt}%
\pgfsys@defobject{currentmarker}{\pgfqpoint{0.000000in}{-0.048611in}}{\pgfqpoint{0.000000in}{0.000000in}}{%
\pgfpathmoveto{\pgfqpoint{0.000000in}{0.000000in}}%
\pgfpathlineto{\pgfqpoint{0.000000in}{-0.048611in}}%
\pgfusepath{stroke,fill}%
}%
\begin{pgfscope}%
\pgfsys@transformshift{2.278239in}{0.438889in}%
\pgfsys@useobject{currentmarker}{}%
\end{pgfscope}%
\end{pgfscope}%
\begin{pgfscope}%
\definecolor{textcolor}{rgb}{0.000000,0.000000,0.000000}%
\pgfsetstrokecolor{textcolor}%
\pgfsetfillcolor{textcolor}%
\pgftext[x=2.278239in,y=0.341667in,,top]{\color{textcolor}\sffamily\fontsize{10.000000}{12.000000}\selectfont \(\displaystyle {10^{6}}\)}%
\end{pgfscope}%
\begin{pgfscope}%
\pgfsetbuttcap%
\pgfsetroundjoin%
\definecolor{currentfill}{rgb}{0.000000,0.000000,0.000000}%
\pgfsetfillcolor{currentfill}%
\pgfsetlinewidth{0.803000pt}%
\definecolor{currentstroke}{rgb}{0.000000,0.000000,0.000000}%
\pgfsetstrokecolor{currentstroke}%
\pgfsetdash{}{0pt}%
\pgfsys@defobject{currentmarker}{\pgfqpoint{0.000000in}{-0.048611in}}{\pgfqpoint{0.000000in}{0.000000in}}{%
\pgfpathmoveto{\pgfqpoint{0.000000in}{0.000000in}}%
\pgfpathlineto{\pgfqpoint{0.000000in}{-0.048611in}}%
\pgfusepath{stroke,fill}%
}%
\begin{pgfscope}%
\pgfsys@transformshift{2.881383in}{0.438889in}%
\pgfsys@useobject{currentmarker}{}%
\end{pgfscope}%
\end{pgfscope}%
\begin{pgfscope}%
\definecolor{textcolor}{rgb}{0.000000,0.000000,0.000000}%
\pgfsetstrokecolor{textcolor}%
\pgfsetfillcolor{textcolor}%
\pgftext[x=2.881383in,y=0.341667in,,top]{\color{textcolor}\sffamily\fontsize{10.000000}{12.000000}\selectfont \(\displaystyle {10^{7}}\)}%
\end{pgfscope}%
\begin{pgfscope}%
\pgfsetbuttcap%
\pgfsetroundjoin%
\definecolor{currentfill}{rgb}{0.000000,0.000000,0.000000}%
\pgfsetfillcolor{currentfill}%
\pgfsetlinewidth{0.803000pt}%
\definecolor{currentstroke}{rgb}{0.000000,0.000000,0.000000}%
\pgfsetstrokecolor{currentstroke}%
\pgfsetdash{}{0pt}%
\pgfsys@defobject{currentmarker}{\pgfqpoint{0.000000in}{-0.048611in}}{\pgfqpoint{0.000000in}{0.000000in}}{%
\pgfpathmoveto{\pgfqpoint{0.000000in}{0.000000in}}%
\pgfpathlineto{\pgfqpoint{0.000000in}{-0.048611in}}%
\pgfusepath{stroke,fill}%
}%
\begin{pgfscope}%
\pgfsys@transformshift{3.484527in}{0.438889in}%
\pgfsys@useobject{currentmarker}{}%
\end{pgfscope}%
\end{pgfscope}%
\begin{pgfscope}%
\definecolor{textcolor}{rgb}{0.000000,0.000000,0.000000}%
\pgfsetstrokecolor{textcolor}%
\pgfsetfillcolor{textcolor}%
\pgftext[x=3.484527in,y=0.341667in,,top]{\color{textcolor}\sffamily\fontsize{10.000000}{12.000000}\selectfont \(\displaystyle {10^{8}}\)}%
\end{pgfscope}%
\begin{pgfscope}%
\pgfsetbuttcap%
\pgfsetroundjoin%
\definecolor{currentfill}{rgb}{0.000000,0.000000,0.000000}%
\pgfsetfillcolor{currentfill}%
\pgfsetlinewidth{0.602250pt}%
\definecolor{currentstroke}{rgb}{0.000000,0.000000,0.000000}%
\pgfsetstrokecolor{currentstroke}%
\pgfsetdash{}{0pt}%
\pgfsys@defobject{currentmarker}{\pgfqpoint{0.000000in}{-0.027778in}}{\pgfqpoint{0.000000in}{0.000000in}}{%
\pgfpathmoveto{\pgfqpoint{0.000000in}{0.000000in}}%
\pgfpathlineto{\pgfqpoint{0.000000in}{-0.027778in}}%
\pgfusepath{stroke,fill}%
}%
\begin{pgfscope}%
\pgfsys@transformshift{0.650372in}{0.438889in}%
\pgfsys@useobject{currentmarker}{}%
\end{pgfscope}%
\end{pgfscope}%
\begin{pgfscope}%
\pgfsetbuttcap%
\pgfsetroundjoin%
\definecolor{currentfill}{rgb}{0.000000,0.000000,0.000000}%
\pgfsetfillcolor{currentfill}%
\pgfsetlinewidth{0.602250pt}%
\definecolor{currentstroke}{rgb}{0.000000,0.000000,0.000000}%
\pgfsetstrokecolor{currentstroke}%
\pgfsetdash{}{0pt}%
\pgfsys@defobject{currentmarker}{\pgfqpoint{0.000000in}{-0.027778in}}{\pgfqpoint{0.000000in}{0.000000in}}{%
\pgfpathmoveto{\pgfqpoint{0.000000in}{0.000000in}}%
\pgfpathlineto{\pgfqpoint{0.000000in}{-0.027778in}}%
\pgfusepath{stroke,fill}%
}%
\begin{pgfscope}%
\pgfsys@transformshift{0.756580in}{0.438889in}%
\pgfsys@useobject{currentmarker}{}%
\end{pgfscope}%
\end{pgfscope}%
\begin{pgfscope}%
\pgfsetbuttcap%
\pgfsetroundjoin%
\definecolor{currentfill}{rgb}{0.000000,0.000000,0.000000}%
\pgfsetfillcolor{currentfill}%
\pgfsetlinewidth{0.602250pt}%
\definecolor{currentstroke}{rgb}{0.000000,0.000000,0.000000}%
\pgfsetstrokecolor{currentstroke}%
\pgfsetdash{}{0pt}%
\pgfsys@defobject{currentmarker}{\pgfqpoint{0.000000in}{-0.027778in}}{\pgfqpoint{0.000000in}{0.000000in}}{%
\pgfpathmoveto{\pgfqpoint{0.000000in}{0.000000in}}%
\pgfpathlineto{\pgfqpoint{0.000000in}{-0.027778in}}%
\pgfusepath{stroke,fill}%
}%
\begin{pgfscope}%
\pgfsys@transformshift{0.831936in}{0.438889in}%
\pgfsys@useobject{currentmarker}{}%
\end{pgfscope}%
\end{pgfscope}%
\begin{pgfscope}%
\pgfsetbuttcap%
\pgfsetroundjoin%
\definecolor{currentfill}{rgb}{0.000000,0.000000,0.000000}%
\pgfsetfillcolor{currentfill}%
\pgfsetlinewidth{0.602250pt}%
\definecolor{currentstroke}{rgb}{0.000000,0.000000,0.000000}%
\pgfsetstrokecolor{currentstroke}%
\pgfsetdash{}{0pt}%
\pgfsys@defobject{currentmarker}{\pgfqpoint{0.000000in}{-0.027778in}}{\pgfqpoint{0.000000in}{0.000000in}}{%
\pgfpathmoveto{\pgfqpoint{0.000000in}{0.000000in}}%
\pgfpathlineto{\pgfqpoint{0.000000in}{-0.027778in}}%
\pgfusepath{stroke,fill}%
}%
\begin{pgfscope}%
\pgfsys@transformshift{0.890387in}{0.438889in}%
\pgfsys@useobject{currentmarker}{}%
\end{pgfscope}%
\end{pgfscope}%
\begin{pgfscope}%
\pgfsetbuttcap%
\pgfsetroundjoin%
\definecolor{currentfill}{rgb}{0.000000,0.000000,0.000000}%
\pgfsetfillcolor{currentfill}%
\pgfsetlinewidth{0.602250pt}%
\definecolor{currentstroke}{rgb}{0.000000,0.000000,0.000000}%
\pgfsetstrokecolor{currentstroke}%
\pgfsetdash{}{0pt}%
\pgfsys@defobject{currentmarker}{\pgfqpoint{0.000000in}{-0.027778in}}{\pgfqpoint{0.000000in}{0.000000in}}{%
\pgfpathmoveto{\pgfqpoint{0.000000in}{0.000000in}}%
\pgfpathlineto{\pgfqpoint{0.000000in}{-0.027778in}}%
\pgfusepath{stroke,fill}%
}%
\begin{pgfscope}%
\pgfsys@transformshift{0.938145in}{0.438889in}%
\pgfsys@useobject{currentmarker}{}%
\end{pgfscope}%
\end{pgfscope}%
\begin{pgfscope}%
\pgfsetbuttcap%
\pgfsetroundjoin%
\definecolor{currentfill}{rgb}{0.000000,0.000000,0.000000}%
\pgfsetfillcolor{currentfill}%
\pgfsetlinewidth{0.602250pt}%
\definecolor{currentstroke}{rgb}{0.000000,0.000000,0.000000}%
\pgfsetstrokecolor{currentstroke}%
\pgfsetdash{}{0pt}%
\pgfsys@defobject{currentmarker}{\pgfqpoint{0.000000in}{-0.027778in}}{\pgfqpoint{0.000000in}{0.000000in}}{%
\pgfpathmoveto{\pgfqpoint{0.000000in}{0.000000in}}%
\pgfpathlineto{\pgfqpoint{0.000000in}{-0.027778in}}%
\pgfusepath{stroke,fill}%
}%
\begin{pgfscope}%
\pgfsys@transformshift{0.978523in}{0.438889in}%
\pgfsys@useobject{currentmarker}{}%
\end{pgfscope}%
\end{pgfscope}%
\begin{pgfscope}%
\pgfsetbuttcap%
\pgfsetroundjoin%
\definecolor{currentfill}{rgb}{0.000000,0.000000,0.000000}%
\pgfsetfillcolor{currentfill}%
\pgfsetlinewidth{0.602250pt}%
\definecolor{currentstroke}{rgb}{0.000000,0.000000,0.000000}%
\pgfsetstrokecolor{currentstroke}%
\pgfsetdash{}{0pt}%
\pgfsys@defobject{currentmarker}{\pgfqpoint{0.000000in}{-0.027778in}}{\pgfqpoint{0.000000in}{0.000000in}}{%
\pgfpathmoveto{\pgfqpoint{0.000000in}{0.000000in}}%
\pgfpathlineto{\pgfqpoint{0.000000in}{-0.027778in}}%
\pgfusepath{stroke,fill}%
}%
\begin{pgfscope}%
\pgfsys@transformshift{1.013501in}{0.438889in}%
\pgfsys@useobject{currentmarker}{}%
\end{pgfscope}%
\end{pgfscope}%
\begin{pgfscope}%
\pgfsetbuttcap%
\pgfsetroundjoin%
\definecolor{currentfill}{rgb}{0.000000,0.000000,0.000000}%
\pgfsetfillcolor{currentfill}%
\pgfsetlinewidth{0.602250pt}%
\definecolor{currentstroke}{rgb}{0.000000,0.000000,0.000000}%
\pgfsetstrokecolor{currentstroke}%
\pgfsetdash{}{0pt}%
\pgfsys@defobject{currentmarker}{\pgfqpoint{0.000000in}{-0.027778in}}{\pgfqpoint{0.000000in}{0.000000in}}{%
\pgfpathmoveto{\pgfqpoint{0.000000in}{0.000000in}}%
\pgfpathlineto{\pgfqpoint{0.000000in}{-0.027778in}}%
\pgfusepath{stroke,fill}%
}%
\begin{pgfscope}%
\pgfsys@transformshift{1.044353in}{0.438889in}%
\pgfsys@useobject{currentmarker}{}%
\end{pgfscope}%
\end{pgfscope}%
\begin{pgfscope}%
\pgfsetbuttcap%
\pgfsetroundjoin%
\definecolor{currentfill}{rgb}{0.000000,0.000000,0.000000}%
\pgfsetfillcolor{currentfill}%
\pgfsetlinewidth{0.602250pt}%
\definecolor{currentstroke}{rgb}{0.000000,0.000000,0.000000}%
\pgfsetstrokecolor{currentstroke}%
\pgfsetdash{}{0pt}%
\pgfsys@defobject{currentmarker}{\pgfqpoint{0.000000in}{-0.027778in}}{\pgfqpoint{0.000000in}{0.000000in}}{%
\pgfpathmoveto{\pgfqpoint{0.000000in}{0.000000in}}%
\pgfpathlineto{\pgfqpoint{0.000000in}{-0.027778in}}%
\pgfusepath{stroke,fill}%
}%
\begin{pgfscope}%
\pgfsys@transformshift{1.253516in}{0.438889in}%
\pgfsys@useobject{currentmarker}{}%
\end{pgfscope}%
\end{pgfscope}%
\begin{pgfscope}%
\pgfsetbuttcap%
\pgfsetroundjoin%
\definecolor{currentfill}{rgb}{0.000000,0.000000,0.000000}%
\pgfsetfillcolor{currentfill}%
\pgfsetlinewidth{0.602250pt}%
\definecolor{currentstroke}{rgb}{0.000000,0.000000,0.000000}%
\pgfsetstrokecolor{currentstroke}%
\pgfsetdash{}{0pt}%
\pgfsys@defobject{currentmarker}{\pgfqpoint{0.000000in}{-0.027778in}}{\pgfqpoint{0.000000in}{0.000000in}}{%
\pgfpathmoveto{\pgfqpoint{0.000000in}{0.000000in}}%
\pgfpathlineto{\pgfqpoint{0.000000in}{-0.027778in}}%
\pgfusepath{stroke,fill}%
}%
\begin{pgfscope}%
\pgfsys@transformshift{1.359724in}{0.438889in}%
\pgfsys@useobject{currentmarker}{}%
\end{pgfscope}%
\end{pgfscope}%
\begin{pgfscope}%
\pgfsetbuttcap%
\pgfsetroundjoin%
\definecolor{currentfill}{rgb}{0.000000,0.000000,0.000000}%
\pgfsetfillcolor{currentfill}%
\pgfsetlinewidth{0.602250pt}%
\definecolor{currentstroke}{rgb}{0.000000,0.000000,0.000000}%
\pgfsetstrokecolor{currentstroke}%
\pgfsetdash{}{0pt}%
\pgfsys@defobject{currentmarker}{\pgfqpoint{0.000000in}{-0.027778in}}{\pgfqpoint{0.000000in}{0.000000in}}{%
\pgfpathmoveto{\pgfqpoint{0.000000in}{0.000000in}}%
\pgfpathlineto{\pgfqpoint{0.000000in}{-0.027778in}}%
\pgfusepath{stroke,fill}%
}%
\begin{pgfscope}%
\pgfsys@transformshift{1.435080in}{0.438889in}%
\pgfsys@useobject{currentmarker}{}%
\end{pgfscope}%
\end{pgfscope}%
\begin{pgfscope}%
\pgfsetbuttcap%
\pgfsetroundjoin%
\definecolor{currentfill}{rgb}{0.000000,0.000000,0.000000}%
\pgfsetfillcolor{currentfill}%
\pgfsetlinewidth{0.602250pt}%
\definecolor{currentstroke}{rgb}{0.000000,0.000000,0.000000}%
\pgfsetstrokecolor{currentstroke}%
\pgfsetdash{}{0pt}%
\pgfsys@defobject{currentmarker}{\pgfqpoint{0.000000in}{-0.027778in}}{\pgfqpoint{0.000000in}{0.000000in}}{%
\pgfpathmoveto{\pgfqpoint{0.000000in}{0.000000in}}%
\pgfpathlineto{\pgfqpoint{0.000000in}{-0.027778in}}%
\pgfusepath{stroke,fill}%
}%
\begin{pgfscope}%
\pgfsys@transformshift{1.493531in}{0.438889in}%
\pgfsys@useobject{currentmarker}{}%
\end{pgfscope}%
\end{pgfscope}%
\begin{pgfscope}%
\pgfsetbuttcap%
\pgfsetroundjoin%
\definecolor{currentfill}{rgb}{0.000000,0.000000,0.000000}%
\pgfsetfillcolor{currentfill}%
\pgfsetlinewidth{0.602250pt}%
\definecolor{currentstroke}{rgb}{0.000000,0.000000,0.000000}%
\pgfsetstrokecolor{currentstroke}%
\pgfsetdash{}{0pt}%
\pgfsys@defobject{currentmarker}{\pgfqpoint{0.000000in}{-0.027778in}}{\pgfqpoint{0.000000in}{0.000000in}}{%
\pgfpathmoveto{\pgfqpoint{0.000000in}{0.000000in}}%
\pgfpathlineto{\pgfqpoint{0.000000in}{-0.027778in}}%
\pgfusepath{stroke,fill}%
}%
\begin{pgfscope}%
\pgfsys@transformshift{1.541289in}{0.438889in}%
\pgfsys@useobject{currentmarker}{}%
\end{pgfscope}%
\end{pgfscope}%
\begin{pgfscope}%
\pgfsetbuttcap%
\pgfsetroundjoin%
\definecolor{currentfill}{rgb}{0.000000,0.000000,0.000000}%
\pgfsetfillcolor{currentfill}%
\pgfsetlinewidth{0.602250pt}%
\definecolor{currentstroke}{rgb}{0.000000,0.000000,0.000000}%
\pgfsetstrokecolor{currentstroke}%
\pgfsetdash{}{0pt}%
\pgfsys@defobject{currentmarker}{\pgfqpoint{0.000000in}{-0.027778in}}{\pgfqpoint{0.000000in}{0.000000in}}{%
\pgfpathmoveto{\pgfqpoint{0.000000in}{0.000000in}}%
\pgfpathlineto{\pgfqpoint{0.000000in}{-0.027778in}}%
\pgfusepath{stroke,fill}%
}%
\begin{pgfscope}%
\pgfsys@transformshift{1.581667in}{0.438889in}%
\pgfsys@useobject{currentmarker}{}%
\end{pgfscope}%
\end{pgfscope}%
\begin{pgfscope}%
\pgfsetbuttcap%
\pgfsetroundjoin%
\definecolor{currentfill}{rgb}{0.000000,0.000000,0.000000}%
\pgfsetfillcolor{currentfill}%
\pgfsetlinewidth{0.602250pt}%
\definecolor{currentstroke}{rgb}{0.000000,0.000000,0.000000}%
\pgfsetstrokecolor{currentstroke}%
\pgfsetdash{}{0pt}%
\pgfsys@defobject{currentmarker}{\pgfqpoint{0.000000in}{-0.027778in}}{\pgfqpoint{0.000000in}{0.000000in}}{%
\pgfpathmoveto{\pgfqpoint{0.000000in}{0.000000in}}%
\pgfpathlineto{\pgfqpoint{0.000000in}{-0.027778in}}%
\pgfusepath{stroke,fill}%
}%
\begin{pgfscope}%
\pgfsys@transformshift{1.616645in}{0.438889in}%
\pgfsys@useobject{currentmarker}{}%
\end{pgfscope}%
\end{pgfscope}%
\begin{pgfscope}%
\pgfsetbuttcap%
\pgfsetroundjoin%
\definecolor{currentfill}{rgb}{0.000000,0.000000,0.000000}%
\pgfsetfillcolor{currentfill}%
\pgfsetlinewidth{0.602250pt}%
\definecolor{currentstroke}{rgb}{0.000000,0.000000,0.000000}%
\pgfsetstrokecolor{currentstroke}%
\pgfsetdash{}{0pt}%
\pgfsys@defobject{currentmarker}{\pgfqpoint{0.000000in}{-0.027778in}}{\pgfqpoint{0.000000in}{0.000000in}}{%
\pgfpathmoveto{\pgfqpoint{0.000000in}{0.000000in}}%
\pgfpathlineto{\pgfqpoint{0.000000in}{-0.027778in}}%
\pgfusepath{stroke,fill}%
}%
\begin{pgfscope}%
\pgfsys@transformshift{1.647497in}{0.438889in}%
\pgfsys@useobject{currentmarker}{}%
\end{pgfscope}%
\end{pgfscope}%
\begin{pgfscope}%
\pgfsetbuttcap%
\pgfsetroundjoin%
\definecolor{currentfill}{rgb}{0.000000,0.000000,0.000000}%
\pgfsetfillcolor{currentfill}%
\pgfsetlinewidth{0.602250pt}%
\definecolor{currentstroke}{rgb}{0.000000,0.000000,0.000000}%
\pgfsetstrokecolor{currentstroke}%
\pgfsetdash{}{0pt}%
\pgfsys@defobject{currentmarker}{\pgfqpoint{0.000000in}{-0.027778in}}{\pgfqpoint{0.000000in}{0.000000in}}{%
\pgfpathmoveto{\pgfqpoint{0.000000in}{0.000000in}}%
\pgfpathlineto{\pgfqpoint{0.000000in}{-0.027778in}}%
\pgfusepath{stroke,fill}%
}%
\begin{pgfscope}%
\pgfsys@transformshift{1.856660in}{0.438889in}%
\pgfsys@useobject{currentmarker}{}%
\end{pgfscope}%
\end{pgfscope}%
\begin{pgfscope}%
\pgfsetbuttcap%
\pgfsetroundjoin%
\definecolor{currentfill}{rgb}{0.000000,0.000000,0.000000}%
\pgfsetfillcolor{currentfill}%
\pgfsetlinewidth{0.602250pt}%
\definecolor{currentstroke}{rgb}{0.000000,0.000000,0.000000}%
\pgfsetstrokecolor{currentstroke}%
\pgfsetdash{}{0pt}%
\pgfsys@defobject{currentmarker}{\pgfqpoint{0.000000in}{-0.027778in}}{\pgfqpoint{0.000000in}{0.000000in}}{%
\pgfpathmoveto{\pgfqpoint{0.000000in}{0.000000in}}%
\pgfpathlineto{\pgfqpoint{0.000000in}{-0.027778in}}%
\pgfusepath{stroke,fill}%
}%
\begin{pgfscope}%
\pgfsys@transformshift{1.962868in}{0.438889in}%
\pgfsys@useobject{currentmarker}{}%
\end{pgfscope}%
\end{pgfscope}%
\begin{pgfscope}%
\pgfsetbuttcap%
\pgfsetroundjoin%
\definecolor{currentfill}{rgb}{0.000000,0.000000,0.000000}%
\pgfsetfillcolor{currentfill}%
\pgfsetlinewidth{0.602250pt}%
\definecolor{currentstroke}{rgb}{0.000000,0.000000,0.000000}%
\pgfsetstrokecolor{currentstroke}%
\pgfsetdash{}{0pt}%
\pgfsys@defobject{currentmarker}{\pgfqpoint{0.000000in}{-0.027778in}}{\pgfqpoint{0.000000in}{0.000000in}}{%
\pgfpathmoveto{\pgfqpoint{0.000000in}{0.000000in}}%
\pgfpathlineto{\pgfqpoint{0.000000in}{-0.027778in}}%
\pgfusepath{stroke,fill}%
}%
\begin{pgfscope}%
\pgfsys@transformshift{2.038224in}{0.438889in}%
\pgfsys@useobject{currentmarker}{}%
\end{pgfscope}%
\end{pgfscope}%
\begin{pgfscope}%
\pgfsetbuttcap%
\pgfsetroundjoin%
\definecolor{currentfill}{rgb}{0.000000,0.000000,0.000000}%
\pgfsetfillcolor{currentfill}%
\pgfsetlinewidth{0.602250pt}%
\definecolor{currentstroke}{rgb}{0.000000,0.000000,0.000000}%
\pgfsetstrokecolor{currentstroke}%
\pgfsetdash{}{0pt}%
\pgfsys@defobject{currentmarker}{\pgfqpoint{0.000000in}{-0.027778in}}{\pgfqpoint{0.000000in}{0.000000in}}{%
\pgfpathmoveto{\pgfqpoint{0.000000in}{0.000000in}}%
\pgfpathlineto{\pgfqpoint{0.000000in}{-0.027778in}}%
\pgfusepath{stroke,fill}%
}%
\begin{pgfscope}%
\pgfsys@transformshift{2.096675in}{0.438889in}%
\pgfsys@useobject{currentmarker}{}%
\end{pgfscope}%
\end{pgfscope}%
\begin{pgfscope}%
\pgfsetbuttcap%
\pgfsetroundjoin%
\definecolor{currentfill}{rgb}{0.000000,0.000000,0.000000}%
\pgfsetfillcolor{currentfill}%
\pgfsetlinewidth{0.602250pt}%
\definecolor{currentstroke}{rgb}{0.000000,0.000000,0.000000}%
\pgfsetstrokecolor{currentstroke}%
\pgfsetdash{}{0pt}%
\pgfsys@defobject{currentmarker}{\pgfqpoint{0.000000in}{-0.027778in}}{\pgfqpoint{0.000000in}{0.000000in}}{%
\pgfpathmoveto{\pgfqpoint{0.000000in}{0.000000in}}%
\pgfpathlineto{\pgfqpoint{0.000000in}{-0.027778in}}%
\pgfusepath{stroke,fill}%
}%
\begin{pgfscope}%
\pgfsys@transformshift{2.144433in}{0.438889in}%
\pgfsys@useobject{currentmarker}{}%
\end{pgfscope}%
\end{pgfscope}%
\begin{pgfscope}%
\pgfsetbuttcap%
\pgfsetroundjoin%
\definecolor{currentfill}{rgb}{0.000000,0.000000,0.000000}%
\pgfsetfillcolor{currentfill}%
\pgfsetlinewidth{0.602250pt}%
\definecolor{currentstroke}{rgb}{0.000000,0.000000,0.000000}%
\pgfsetstrokecolor{currentstroke}%
\pgfsetdash{}{0pt}%
\pgfsys@defobject{currentmarker}{\pgfqpoint{0.000000in}{-0.027778in}}{\pgfqpoint{0.000000in}{0.000000in}}{%
\pgfpathmoveto{\pgfqpoint{0.000000in}{0.000000in}}%
\pgfpathlineto{\pgfqpoint{0.000000in}{-0.027778in}}%
\pgfusepath{stroke,fill}%
}%
\begin{pgfscope}%
\pgfsys@transformshift{2.184811in}{0.438889in}%
\pgfsys@useobject{currentmarker}{}%
\end{pgfscope}%
\end{pgfscope}%
\begin{pgfscope}%
\pgfsetbuttcap%
\pgfsetroundjoin%
\definecolor{currentfill}{rgb}{0.000000,0.000000,0.000000}%
\pgfsetfillcolor{currentfill}%
\pgfsetlinewidth{0.602250pt}%
\definecolor{currentstroke}{rgb}{0.000000,0.000000,0.000000}%
\pgfsetstrokecolor{currentstroke}%
\pgfsetdash{}{0pt}%
\pgfsys@defobject{currentmarker}{\pgfqpoint{0.000000in}{-0.027778in}}{\pgfqpoint{0.000000in}{0.000000in}}{%
\pgfpathmoveto{\pgfqpoint{0.000000in}{0.000000in}}%
\pgfpathlineto{\pgfqpoint{0.000000in}{-0.027778in}}%
\pgfusepath{stroke,fill}%
}%
\begin{pgfscope}%
\pgfsys@transformshift{2.219789in}{0.438889in}%
\pgfsys@useobject{currentmarker}{}%
\end{pgfscope}%
\end{pgfscope}%
\begin{pgfscope}%
\pgfsetbuttcap%
\pgfsetroundjoin%
\definecolor{currentfill}{rgb}{0.000000,0.000000,0.000000}%
\pgfsetfillcolor{currentfill}%
\pgfsetlinewidth{0.602250pt}%
\definecolor{currentstroke}{rgb}{0.000000,0.000000,0.000000}%
\pgfsetstrokecolor{currentstroke}%
\pgfsetdash{}{0pt}%
\pgfsys@defobject{currentmarker}{\pgfqpoint{0.000000in}{-0.027778in}}{\pgfqpoint{0.000000in}{0.000000in}}{%
\pgfpathmoveto{\pgfqpoint{0.000000in}{0.000000in}}%
\pgfpathlineto{\pgfqpoint{0.000000in}{-0.027778in}}%
\pgfusepath{stroke,fill}%
}%
\begin{pgfscope}%
\pgfsys@transformshift{2.250641in}{0.438889in}%
\pgfsys@useobject{currentmarker}{}%
\end{pgfscope}%
\end{pgfscope}%
\begin{pgfscope}%
\pgfsetbuttcap%
\pgfsetroundjoin%
\definecolor{currentfill}{rgb}{0.000000,0.000000,0.000000}%
\pgfsetfillcolor{currentfill}%
\pgfsetlinewidth{0.602250pt}%
\definecolor{currentstroke}{rgb}{0.000000,0.000000,0.000000}%
\pgfsetstrokecolor{currentstroke}%
\pgfsetdash{}{0pt}%
\pgfsys@defobject{currentmarker}{\pgfqpoint{0.000000in}{-0.027778in}}{\pgfqpoint{0.000000in}{0.000000in}}{%
\pgfpathmoveto{\pgfqpoint{0.000000in}{0.000000in}}%
\pgfpathlineto{\pgfqpoint{0.000000in}{-0.027778in}}%
\pgfusepath{stroke,fill}%
}%
\begin{pgfscope}%
\pgfsys@transformshift{2.459804in}{0.438889in}%
\pgfsys@useobject{currentmarker}{}%
\end{pgfscope}%
\end{pgfscope}%
\begin{pgfscope}%
\pgfsetbuttcap%
\pgfsetroundjoin%
\definecolor{currentfill}{rgb}{0.000000,0.000000,0.000000}%
\pgfsetfillcolor{currentfill}%
\pgfsetlinewidth{0.602250pt}%
\definecolor{currentstroke}{rgb}{0.000000,0.000000,0.000000}%
\pgfsetstrokecolor{currentstroke}%
\pgfsetdash{}{0pt}%
\pgfsys@defobject{currentmarker}{\pgfqpoint{0.000000in}{-0.027778in}}{\pgfqpoint{0.000000in}{0.000000in}}{%
\pgfpathmoveto{\pgfqpoint{0.000000in}{0.000000in}}%
\pgfpathlineto{\pgfqpoint{0.000000in}{-0.027778in}}%
\pgfusepath{stroke,fill}%
}%
\begin{pgfscope}%
\pgfsys@transformshift{2.566012in}{0.438889in}%
\pgfsys@useobject{currentmarker}{}%
\end{pgfscope}%
\end{pgfscope}%
\begin{pgfscope}%
\pgfsetbuttcap%
\pgfsetroundjoin%
\definecolor{currentfill}{rgb}{0.000000,0.000000,0.000000}%
\pgfsetfillcolor{currentfill}%
\pgfsetlinewidth{0.602250pt}%
\definecolor{currentstroke}{rgb}{0.000000,0.000000,0.000000}%
\pgfsetstrokecolor{currentstroke}%
\pgfsetdash{}{0pt}%
\pgfsys@defobject{currentmarker}{\pgfqpoint{0.000000in}{-0.027778in}}{\pgfqpoint{0.000000in}{0.000000in}}{%
\pgfpathmoveto{\pgfqpoint{0.000000in}{0.000000in}}%
\pgfpathlineto{\pgfqpoint{0.000000in}{-0.027778in}}%
\pgfusepath{stroke,fill}%
}%
\begin{pgfscope}%
\pgfsys@transformshift{2.641368in}{0.438889in}%
\pgfsys@useobject{currentmarker}{}%
\end{pgfscope}%
\end{pgfscope}%
\begin{pgfscope}%
\pgfsetbuttcap%
\pgfsetroundjoin%
\definecolor{currentfill}{rgb}{0.000000,0.000000,0.000000}%
\pgfsetfillcolor{currentfill}%
\pgfsetlinewidth{0.602250pt}%
\definecolor{currentstroke}{rgb}{0.000000,0.000000,0.000000}%
\pgfsetstrokecolor{currentstroke}%
\pgfsetdash{}{0pt}%
\pgfsys@defobject{currentmarker}{\pgfqpoint{0.000000in}{-0.027778in}}{\pgfqpoint{0.000000in}{0.000000in}}{%
\pgfpathmoveto{\pgfqpoint{0.000000in}{0.000000in}}%
\pgfpathlineto{\pgfqpoint{0.000000in}{-0.027778in}}%
\pgfusepath{stroke,fill}%
}%
\begin{pgfscope}%
\pgfsys@transformshift{2.699819in}{0.438889in}%
\pgfsys@useobject{currentmarker}{}%
\end{pgfscope}%
\end{pgfscope}%
\begin{pgfscope}%
\pgfsetbuttcap%
\pgfsetroundjoin%
\definecolor{currentfill}{rgb}{0.000000,0.000000,0.000000}%
\pgfsetfillcolor{currentfill}%
\pgfsetlinewidth{0.602250pt}%
\definecolor{currentstroke}{rgb}{0.000000,0.000000,0.000000}%
\pgfsetstrokecolor{currentstroke}%
\pgfsetdash{}{0pt}%
\pgfsys@defobject{currentmarker}{\pgfqpoint{0.000000in}{-0.027778in}}{\pgfqpoint{0.000000in}{0.000000in}}{%
\pgfpathmoveto{\pgfqpoint{0.000000in}{0.000000in}}%
\pgfpathlineto{\pgfqpoint{0.000000in}{-0.027778in}}%
\pgfusepath{stroke,fill}%
}%
\begin{pgfscope}%
\pgfsys@transformshift{2.747577in}{0.438889in}%
\pgfsys@useobject{currentmarker}{}%
\end{pgfscope}%
\end{pgfscope}%
\begin{pgfscope}%
\pgfsetbuttcap%
\pgfsetroundjoin%
\definecolor{currentfill}{rgb}{0.000000,0.000000,0.000000}%
\pgfsetfillcolor{currentfill}%
\pgfsetlinewidth{0.602250pt}%
\definecolor{currentstroke}{rgb}{0.000000,0.000000,0.000000}%
\pgfsetstrokecolor{currentstroke}%
\pgfsetdash{}{0pt}%
\pgfsys@defobject{currentmarker}{\pgfqpoint{0.000000in}{-0.027778in}}{\pgfqpoint{0.000000in}{0.000000in}}{%
\pgfpathmoveto{\pgfqpoint{0.000000in}{0.000000in}}%
\pgfpathlineto{\pgfqpoint{0.000000in}{-0.027778in}}%
\pgfusepath{stroke,fill}%
}%
\begin{pgfscope}%
\pgfsys@transformshift{2.787955in}{0.438889in}%
\pgfsys@useobject{currentmarker}{}%
\end{pgfscope}%
\end{pgfscope}%
\begin{pgfscope}%
\pgfsetbuttcap%
\pgfsetroundjoin%
\definecolor{currentfill}{rgb}{0.000000,0.000000,0.000000}%
\pgfsetfillcolor{currentfill}%
\pgfsetlinewidth{0.602250pt}%
\definecolor{currentstroke}{rgb}{0.000000,0.000000,0.000000}%
\pgfsetstrokecolor{currentstroke}%
\pgfsetdash{}{0pt}%
\pgfsys@defobject{currentmarker}{\pgfqpoint{0.000000in}{-0.027778in}}{\pgfqpoint{0.000000in}{0.000000in}}{%
\pgfpathmoveto{\pgfqpoint{0.000000in}{0.000000in}}%
\pgfpathlineto{\pgfqpoint{0.000000in}{-0.027778in}}%
\pgfusepath{stroke,fill}%
}%
\begin{pgfscope}%
\pgfsys@transformshift{2.822933in}{0.438889in}%
\pgfsys@useobject{currentmarker}{}%
\end{pgfscope}%
\end{pgfscope}%
\begin{pgfscope}%
\pgfsetbuttcap%
\pgfsetroundjoin%
\definecolor{currentfill}{rgb}{0.000000,0.000000,0.000000}%
\pgfsetfillcolor{currentfill}%
\pgfsetlinewidth{0.602250pt}%
\definecolor{currentstroke}{rgb}{0.000000,0.000000,0.000000}%
\pgfsetstrokecolor{currentstroke}%
\pgfsetdash{}{0pt}%
\pgfsys@defobject{currentmarker}{\pgfqpoint{0.000000in}{-0.027778in}}{\pgfqpoint{0.000000in}{0.000000in}}{%
\pgfpathmoveto{\pgfqpoint{0.000000in}{0.000000in}}%
\pgfpathlineto{\pgfqpoint{0.000000in}{-0.027778in}}%
\pgfusepath{stroke,fill}%
}%
\begin{pgfscope}%
\pgfsys@transformshift{2.853785in}{0.438889in}%
\pgfsys@useobject{currentmarker}{}%
\end{pgfscope}%
\end{pgfscope}%
\begin{pgfscope}%
\pgfsetbuttcap%
\pgfsetroundjoin%
\definecolor{currentfill}{rgb}{0.000000,0.000000,0.000000}%
\pgfsetfillcolor{currentfill}%
\pgfsetlinewidth{0.602250pt}%
\definecolor{currentstroke}{rgb}{0.000000,0.000000,0.000000}%
\pgfsetstrokecolor{currentstroke}%
\pgfsetdash{}{0pt}%
\pgfsys@defobject{currentmarker}{\pgfqpoint{0.000000in}{-0.027778in}}{\pgfqpoint{0.000000in}{0.000000in}}{%
\pgfpathmoveto{\pgfqpoint{0.000000in}{0.000000in}}%
\pgfpathlineto{\pgfqpoint{0.000000in}{-0.027778in}}%
\pgfusepath{stroke,fill}%
}%
\begin{pgfscope}%
\pgfsys@transformshift{3.062948in}{0.438889in}%
\pgfsys@useobject{currentmarker}{}%
\end{pgfscope}%
\end{pgfscope}%
\begin{pgfscope}%
\pgfsetbuttcap%
\pgfsetroundjoin%
\definecolor{currentfill}{rgb}{0.000000,0.000000,0.000000}%
\pgfsetfillcolor{currentfill}%
\pgfsetlinewidth{0.602250pt}%
\definecolor{currentstroke}{rgb}{0.000000,0.000000,0.000000}%
\pgfsetstrokecolor{currentstroke}%
\pgfsetdash{}{0pt}%
\pgfsys@defobject{currentmarker}{\pgfqpoint{0.000000in}{-0.027778in}}{\pgfqpoint{0.000000in}{0.000000in}}{%
\pgfpathmoveto{\pgfqpoint{0.000000in}{0.000000in}}%
\pgfpathlineto{\pgfqpoint{0.000000in}{-0.027778in}}%
\pgfusepath{stroke,fill}%
}%
\begin{pgfscope}%
\pgfsys@transformshift{3.169156in}{0.438889in}%
\pgfsys@useobject{currentmarker}{}%
\end{pgfscope}%
\end{pgfscope}%
\begin{pgfscope}%
\pgfsetbuttcap%
\pgfsetroundjoin%
\definecolor{currentfill}{rgb}{0.000000,0.000000,0.000000}%
\pgfsetfillcolor{currentfill}%
\pgfsetlinewidth{0.602250pt}%
\definecolor{currentstroke}{rgb}{0.000000,0.000000,0.000000}%
\pgfsetstrokecolor{currentstroke}%
\pgfsetdash{}{0pt}%
\pgfsys@defobject{currentmarker}{\pgfqpoint{0.000000in}{-0.027778in}}{\pgfqpoint{0.000000in}{0.000000in}}{%
\pgfpathmoveto{\pgfqpoint{0.000000in}{0.000000in}}%
\pgfpathlineto{\pgfqpoint{0.000000in}{-0.027778in}}%
\pgfusepath{stroke,fill}%
}%
\begin{pgfscope}%
\pgfsys@transformshift{3.244512in}{0.438889in}%
\pgfsys@useobject{currentmarker}{}%
\end{pgfscope}%
\end{pgfscope}%
\begin{pgfscope}%
\pgfsetbuttcap%
\pgfsetroundjoin%
\definecolor{currentfill}{rgb}{0.000000,0.000000,0.000000}%
\pgfsetfillcolor{currentfill}%
\pgfsetlinewidth{0.602250pt}%
\definecolor{currentstroke}{rgb}{0.000000,0.000000,0.000000}%
\pgfsetstrokecolor{currentstroke}%
\pgfsetdash{}{0pt}%
\pgfsys@defobject{currentmarker}{\pgfqpoint{0.000000in}{-0.027778in}}{\pgfqpoint{0.000000in}{0.000000in}}{%
\pgfpathmoveto{\pgfqpoint{0.000000in}{0.000000in}}%
\pgfpathlineto{\pgfqpoint{0.000000in}{-0.027778in}}%
\pgfusepath{stroke,fill}%
}%
\begin{pgfscope}%
\pgfsys@transformshift{3.302963in}{0.438889in}%
\pgfsys@useobject{currentmarker}{}%
\end{pgfscope}%
\end{pgfscope}%
\begin{pgfscope}%
\pgfsetbuttcap%
\pgfsetroundjoin%
\definecolor{currentfill}{rgb}{0.000000,0.000000,0.000000}%
\pgfsetfillcolor{currentfill}%
\pgfsetlinewidth{0.602250pt}%
\definecolor{currentstroke}{rgb}{0.000000,0.000000,0.000000}%
\pgfsetstrokecolor{currentstroke}%
\pgfsetdash{}{0pt}%
\pgfsys@defobject{currentmarker}{\pgfqpoint{0.000000in}{-0.027778in}}{\pgfqpoint{0.000000in}{0.000000in}}{%
\pgfpathmoveto{\pgfqpoint{0.000000in}{0.000000in}}%
\pgfpathlineto{\pgfqpoint{0.000000in}{-0.027778in}}%
\pgfusepath{stroke,fill}%
}%
\begin{pgfscope}%
\pgfsys@transformshift{3.350721in}{0.438889in}%
\pgfsys@useobject{currentmarker}{}%
\end{pgfscope}%
\end{pgfscope}%
\begin{pgfscope}%
\pgfsetbuttcap%
\pgfsetroundjoin%
\definecolor{currentfill}{rgb}{0.000000,0.000000,0.000000}%
\pgfsetfillcolor{currentfill}%
\pgfsetlinewidth{0.602250pt}%
\definecolor{currentstroke}{rgb}{0.000000,0.000000,0.000000}%
\pgfsetstrokecolor{currentstroke}%
\pgfsetdash{}{0pt}%
\pgfsys@defobject{currentmarker}{\pgfqpoint{0.000000in}{-0.027778in}}{\pgfqpoint{0.000000in}{0.000000in}}{%
\pgfpathmoveto{\pgfqpoint{0.000000in}{0.000000in}}%
\pgfpathlineto{\pgfqpoint{0.000000in}{-0.027778in}}%
\pgfusepath{stroke,fill}%
}%
\begin{pgfscope}%
\pgfsys@transformshift{3.391099in}{0.438889in}%
\pgfsys@useobject{currentmarker}{}%
\end{pgfscope}%
\end{pgfscope}%
\begin{pgfscope}%
\pgfsetbuttcap%
\pgfsetroundjoin%
\definecolor{currentfill}{rgb}{0.000000,0.000000,0.000000}%
\pgfsetfillcolor{currentfill}%
\pgfsetlinewidth{0.602250pt}%
\definecolor{currentstroke}{rgb}{0.000000,0.000000,0.000000}%
\pgfsetstrokecolor{currentstroke}%
\pgfsetdash{}{0pt}%
\pgfsys@defobject{currentmarker}{\pgfqpoint{0.000000in}{-0.027778in}}{\pgfqpoint{0.000000in}{0.000000in}}{%
\pgfpathmoveto{\pgfqpoint{0.000000in}{0.000000in}}%
\pgfpathlineto{\pgfqpoint{0.000000in}{-0.027778in}}%
\pgfusepath{stroke,fill}%
}%
\begin{pgfscope}%
\pgfsys@transformshift{3.426077in}{0.438889in}%
\pgfsys@useobject{currentmarker}{}%
\end{pgfscope}%
\end{pgfscope}%
\begin{pgfscope}%
\pgfsetbuttcap%
\pgfsetroundjoin%
\definecolor{currentfill}{rgb}{0.000000,0.000000,0.000000}%
\pgfsetfillcolor{currentfill}%
\pgfsetlinewidth{0.602250pt}%
\definecolor{currentstroke}{rgb}{0.000000,0.000000,0.000000}%
\pgfsetstrokecolor{currentstroke}%
\pgfsetdash{}{0pt}%
\pgfsys@defobject{currentmarker}{\pgfqpoint{0.000000in}{-0.027778in}}{\pgfqpoint{0.000000in}{0.000000in}}{%
\pgfpathmoveto{\pgfqpoint{0.000000in}{0.000000in}}%
\pgfpathlineto{\pgfqpoint{0.000000in}{-0.027778in}}%
\pgfusepath{stroke,fill}%
}%
\begin{pgfscope}%
\pgfsys@transformshift{3.456929in}{0.438889in}%
\pgfsys@useobject{currentmarker}{}%
\end{pgfscope}%
\end{pgfscope}%
\begin{pgfscope}%
\pgfsetbuttcap%
\pgfsetroundjoin%
\definecolor{currentfill}{rgb}{0.000000,0.000000,0.000000}%
\pgfsetfillcolor{currentfill}%
\pgfsetlinewidth{0.602250pt}%
\definecolor{currentstroke}{rgb}{0.000000,0.000000,0.000000}%
\pgfsetstrokecolor{currentstroke}%
\pgfsetdash{}{0pt}%
\pgfsys@defobject{currentmarker}{\pgfqpoint{0.000000in}{-0.027778in}}{\pgfqpoint{0.000000in}{0.000000in}}{%
\pgfpathmoveto{\pgfqpoint{0.000000in}{0.000000in}}%
\pgfpathlineto{\pgfqpoint{0.000000in}{-0.027778in}}%
\pgfusepath{stroke,fill}%
}%
\begin{pgfscope}%
\pgfsys@transformshift{3.666092in}{0.438889in}%
\pgfsys@useobject{currentmarker}{}%
\end{pgfscope}%
\end{pgfscope}%
\begin{pgfscope}%
\pgfsetbuttcap%
\pgfsetroundjoin%
\definecolor{currentfill}{rgb}{0.000000,0.000000,0.000000}%
\pgfsetfillcolor{currentfill}%
\pgfsetlinewidth{0.602250pt}%
\definecolor{currentstroke}{rgb}{0.000000,0.000000,0.000000}%
\pgfsetstrokecolor{currentstroke}%
\pgfsetdash{}{0pt}%
\pgfsys@defobject{currentmarker}{\pgfqpoint{0.000000in}{-0.027778in}}{\pgfqpoint{0.000000in}{0.000000in}}{%
\pgfpathmoveto{\pgfqpoint{0.000000in}{0.000000in}}%
\pgfpathlineto{\pgfqpoint{0.000000in}{-0.027778in}}%
\pgfusepath{stroke,fill}%
}%
\begin{pgfscope}%
\pgfsys@transformshift{3.772300in}{0.438889in}%
\pgfsys@useobject{currentmarker}{}%
\end{pgfscope}%
\end{pgfscope}%
\begin{pgfscope}%
\pgfsetbuttcap%
\pgfsetroundjoin%
\definecolor{currentfill}{rgb}{0.000000,0.000000,0.000000}%
\pgfsetfillcolor{currentfill}%
\pgfsetlinewidth{0.602250pt}%
\definecolor{currentstroke}{rgb}{0.000000,0.000000,0.000000}%
\pgfsetstrokecolor{currentstroke}%
\pgfsetdash{}{0pt}%
\pgfsys@defobject{currentmarker}{\pgfqpoint{0.000000in}{-0.027778in}}{\pgfqpoint{0.000000in}{0.000000in}}{%
\pgfpathmoveto{\pgfqpoint{0.000000in}{0.000000in}}%
\pgfpathlineto{\pgfqpoint{0.000000in}{-0.027778in}}%
\pgfusepath{stroke,fill}%
}%
\begin{pgfscope}%
\pgfsys@transformshift{3.847656in}{0.438889in}%
\pgfsys@useobject{currentmarker}{}%
\end{pgfscope}%
\end{pgfscope}%
\begin{pgfscope}%
\pgfsetbuttcap%
\pgfsetroundjoin%
\definecolor{currentfill}{rgb}{0.000000,0.000000,0.000000}%
\pgfsetfillcolor{currentfill}%
\pgfsetlinewidth{0.803000pt}%
\definecolor{currentstroke}{rgb}{0.000000,0.000000,0.000000}%
\pgfsetstrokecolor{currentstroke}%
\pgfsetdash{}{0pt}%
\pgfsys@defobject{currentmarker}{\pgfqpoint{-0.048611in}{0.000000in}}{\pgfqpoint{0.000000in}{0.000000in}}{%
\pgfpathmoveto{\pgfqpoint{0.000000in}{0.000000in}}%
\pgfpathlineto{\pgfqpoint{-0.048611in}{0.000000in}}%
\pgfusepath{stroke,fill}%
}%
\begin{pgfscope}%
\pgfsys@transformshift{0.557986in}{0.647407in}%
\pgfsys@useobject{currentmarker}{}%
\end{pgfscope}%
\end{pgfscope}%
\begin{pgfscope}%
\definecolor{textcolor}{rgb}{0.000000,0.000000,0.000000}%
\pgfsetstrokecolor{textcolor}%
\pgfsetfillcolor{textcolor}%
\pgftext[x=0.239884in,y=0.594645in,left,base]{\color{textcolor}\sffamily\fontsize{10.000000}{12.000000}\selectfont 0.0}%
\end{pgfscope}%
\begin{pgfscope}%
\pgfsetbuttcap%
\pgfsetroundjoin%
\definecolor{currentfill}{rgb}{0.000000,0.000000,0.000000}%
\pgfsetfillcolor{currentfill}%
\pgfsetlinewidth{0.803000pt}%
\definecolor{currentstroke}{rgb}{0.000000,0.000000,0.000000}%
\pgfsetstrokecolor{currentstroke}%
\pgfsetdash{}{0pt}%
\pgfsys@defobject{currentmarker}{\pgfqpoint{-0.048611in}{0.000000in}}{\pgfqpoint{0.000000in}{0.000000in}}{%
\pgfpathmoveto{\pgfqpoint{0.000000in}{0.000000in}}%
\pgfpathlineto{\pgfqpoint{-0.048611in}{0.000000in}}%
\pgfusepath{stroke,fill}%
}%
\begin{pgfscope}%
\pgfsys@transformshift{0.557986in}{0.944143in}%
\pgfsys@useobject{currentmarker}{}%
\end{pgfscope}%
\end{pgfscope}%
\begin{pgfscope}%
\definecolor{textcolor}{rgb}{0.000000,0.000000,0.000000}%
\pgfsetstrokecolor{textcolor}%
\pgfsetfillcolor{textcolor}%
\pgftext[x=0.239884in,y=0.891382in,left,base]{\color{textcolor}\sffamily\fontsize{10.000000}{12.000000}\selectfont 0.2}%
\end{pgfscope}%
\begin{pgfscope}%
\pgfsetbuttcap%
\pgfsetroundjoin%
\definecolor{currentfill}{rgb}{0.000000,0.000000,0.000000}%
\pgfsetfillcolor{currentfill}%
\pgfsetlinewidth{0.803000pt}%
\definecolor{currentstroke}{rgb}{0.000000,0.000000,0.000000}%
\pgfsetstrokecolor{currentstroke}%
\pgfsetdash{}{0pt}%
\pgfsys@defobject{currentmarker}{\pgfqpoint{-0.048611in}{0.000000in}}{\pgfqpoint{0.000000in}{0.000000in}}{%
\pgfpathmoveto{\pgfqpoint{0.000000in}{0.000000in}}%
\pgfpathlineto{\pgfqpoint{-0.048611in}{0.000000in}}%
\pgfusepath{stroke,fill}%
}%
\begin{pgfscope}%
\pgfsys@transformshift{0.557986in}{1.240879in}%
\pgfsys@useobject{currentmarker}{}%
\end{pgfscope}%
\end{pgfscope}%
\begin{pgfscope}%
\definecolor{textcolor}{rgb}{0.000000,0.000000,0.000000}%
\pgfsetstrokecolor{textcolor}%
\pgfsetfillcolor{textcolor}%
\pgftext[x=0.239884in,y=1.188118in,left,base]{\color{textcolor}\sffamily\fontsize{10.000000}{12.000000}\selectfont 0.4}%
\end{pgfscope}%
\begin{pgfscope}%
\pgfsetbuttcap%
\pgfsetroundjoin%
\definecolor{currentfill}{rgb}{0.000000,0.000000,0.000000}%
\pgfsetfillcolor{currentfill}%
\pgfsetlinewidth{0.803000pt}%
\definecolor{currentstroke}{rgb}{0.000000,0.000000,0.000000}%
\pgfsetstrokecolor{currentstroke}%
\pgfsetdash{}{0pt}%
\pgfsys@defobject{currentmarker}{\pgfqpoint{-0.048611in}{0.000000in}}{\pgfqpoint{0.000000in}{0.000000in}}{%
\pgfpathmoveto{\pgfqpoint{0.000000in}{0.000000in}}%
\pgfpathlineto{\pgfqpoint{-0.048611in}{0.000000in}}%
\pgfusepath{stroke,fill}%
}%
\begin{pgfscope}%
\pgfsys@transformshift{0.557986in}{1.537616in}%
\pgfsys@useobject{currentmarker}{}%
\end{pgfscope}%
\end{pgfscope}%
\begin{pgfscope}%
\definecolor{textcolor}{rgb}{0.000000,0.000000,0.000000}%
\pgfsetstrokecolor{textcolor}%
\pgfsetfillcolor{textcolor}%
\pgftext[x=0.239884in,y=1.484854in,left,base]{\color{textcolor}\sffamily\fontsize{10.000000}{12.000000}\selectfont 0.6}%
\end{pgfscope}%
\begin{pgfscope}%
\pgfsetbuttcap%
\pgfsetroundjoin%
\definecolor{currentfill}{rgb}{0.000000,0.000000,0.000000}%
\pgfsetfillcolor{currentfill}%
\pgfsetlinewidth{0.803000pt}%
\definecolor{currentstroke}{rgb}{0.000000,0.000000,0.000000}%
\pgfsetstrokecolor{currentstroke}%
\pgfsetdash{}{0pt}%
\pgfsys@defobject{currentmarker}{\pgfqpoint{-0.048611in}{0.000000in}}{\pgfqpoint{0.000000in}{0.000000in}}{%
\pgfpathmoveto{\pgfqpoint{0.000000in}{0.000000in}}%
\pgfpathlineto{\pgfqpoint{-0.048611in}{0.000000in}}%
\pgfusepath{stroke,fill}%
}%
\begin{pgfscope}%
\pgfsys@transformshift{0.557986in}{1.834352in}%
\pgfsys@useobject{currentmarker}{}%
\end{pgfscope}%
\end{pgfscope}%
\begin{pgfscope}%
\definecolor{textcolor}{rgb}{0.000000,0.000000,0.000000}%
\pgfsetstrokecolor{textcolor}%
\pgfsetfillcolor{textcolor}%
\pgftext[x=0.239884in,y=1.781591in,left,base]{\color{textcolor}\sffamily\fontsize{10.000000}{12.000000}\selectfont 0.8}%
\end{pgfscope}%
\begin{pgfscope}%
\pgfsetbuttcap%
\pgfsetroundjoin%
\definecolor{currentfill}{rgb}{0.000000,0.000000,0.000000}%
\pgfsetfillcolor{currentfill}%
\pgfsetlinewidth{0.803000pt}%
\definecolor{currentstroke}{rgb}{0.000000,0.000000,0.000000}%
\pgfsetstrokecolor{currentstroke}%
\pgfsetdash{}{0pt}%
\pgfsys@defobject{currentmarker}{\pgfqpoint{-0.048611in}{0.000000in}}{\pgfqpoint{0.000000in}{0.000000in}}{%
\pgfpathmoveto{\pgfqpoint{0.000000in}{0.000000in}}%
\pgfpathlineto{\pgfqpoint{-0.048611in}{0.000000in}}%
\pgfusepath{stroke,fill}%
}%
\begin{pgfscope}%
\pgfsys@transformshift{0.557986in}{2.131088in}%
\pgfsys@useobject{currentmarker}{}%
\end{pgfscope}%
\end{pgfscope}%
\begin{pgfscope}%
\definecolor{textcolor}{rgb}{0.000000,0.000000,0.000000}%
\pgfsetstrokecolor{textcolor}%
\pgfsetfillcolor{textcolor}%
\pgftext[x=0.239884in,y=2.078327in,left,base]{\color{textcolor}\sffamily\fontsize{10.000000}{12.000000}\selectfont 1.0}%
\end{pgfscope}%
\begin{pgfscope}%
\pgfsetbuttcap%
\pgfsetroundjoin%
\definecolor{currentfill}{rgb}{0.000000,0.000000,0.000000}%
\pgfsetfillcolor{currentfill}%
\pgfsetlinewidth{0.803000pt}%
\definecolor{currentstroke}{rgb}{0.000000,0.000000,0.000000}%
\pgfsetstrokecolor{currentstroke}%
\pgfsetdash{}{0pt}%
\pgfsys@defobject{currentmarker}{\pgfqpoint{-0.048611in}{0.000000in}}{\pgfqpoint{0.000000in}{0.000000in}}{%
\pgfpathmoveto{\pgfqpoint{0.000000in}{0.000000in}}%
\pgfpathlineto{\pgfqpoint{-0.048611in}{0.000000in}}%
\pgfusepath{stroke,fill}%
}%
\begin{pgfscope}%
\pgfsys@transformshift{0.557986in}{2.427825in}%
\pgfsys@useobject{currentmarker}{}%
\end{pgfscope}%
\end{pgfscope}%
\begin{pgfscope}%
\definecolor{textcolor}{rgb}{0.000000,0.000000,0.000000}%
\pgfsetstrokecolor{textcolor}%
\pgfsetfillcolor{textcolor}%
\pgftext[x=0.239884in,y=2.375063in,left,base]{\color{textcolor}\sffamily\fontsize{10.000000}{12.000000}\selectfont 1.2}%
\end{pgfscope}%
\begin{pgfscope}%
\pgfpathrectangle{\pgfqpoint{0.557986in}{0.438889in}}{\pgfqpoint{3.318403in}{2.263889in}}%
\pgfusepath{clip}%
\pgfsetrectcap%
\pgfsetroundjoin%
\pgfsetlinewidth{1.505625pt}%
\definecolor{currentstroke}{rgb}{0.121569,0.466667,0.705882}%
\pgfsetstrokecolor{currentstroke}%
\pgfsetdash{}{0pt}%
\pgfpathmoveto{\pgfqpoint{0.708823in}{0.677637in}}%
\pgfpathlineto{\pgfqpoint{0.859609in}{0.665474in}}%
\pgfpathlineto{\pgfqpoint{1.010395in}{0.704744in}}%
\pgfpathlineto{\pgfqpoint{1.161181in}{0.696222in}}%
\pgfpathlineto{\pgfqpoint{1.311967in}{0.741800in}}%
\pgfpathlineto{\pgfqpoint{1.462753in}{0.754834in}}%
\pgfpathlineto{\pgfqpoint{1.613539in}{0.828864in}}%
\pgfpathlineto{\pgfqpoint{1.764325in}{0.936655in}}%
\pgfpathlineto{\pgfqpoint{1.915111in}{1.059509in}}%
\pgfpathlineto{\pgfqpoint{2.065897in}{0.890936in}}%
\pgfpathlineto{\pgfqpoint{2.216683in}{0.983823in}}%
\pgfpathlineto{\pgfqpoint{2.367469in}{1.293789in}}%
\pgfpathlineto{\pgfqpoint{2.518255in}{1.459498in}}%
\pgfpathlineto{\pgfqpoint{2.669041in}{2.050375in}}%
\pgfpathlineto{\pgfqpoint{2.819827in}{1.935330in}}%
\pgfpathlineto{\pgfqpoint{2.970613in}{1.835961in}}%
\pgfpathlineto{\pgfqpoint{3.121399in}{2.105455in}}%
\pgfpathlineto{\pgfqpoint{3.272185in}{1.841539in}}%
\pgfpathlineto{\pgfqpoint{3.422971in}{1.892115in}}%
\pgfpathlineto{\pgfqpoint{3.573756in}{2.130019in}}%
\pgfpathlineto{\pgfqpoint{3.724542in}{2.010187in}}%
\pgfusepath{stroke}%
\end{pgfscope}%
\begin{pgfscope}%
\pgfpathrectangle{\pgfqpoint{0.557986in}{0.438889in}}{\pgfqpoint{3.318403in}{2.263889in}}%
\pgfusepath{clip}%
\pgfsetrectcap%
\pgfsetroundjoin%
\pgfsetlinewidth{1.505625pt}%
\definecolor{currentstroke}{rgb}{1.000000,0.498039,0.054902}%
\pgfsetstrokecolor{currentstroke}%
\pgfsetdash{}{0pt}%
\pgfpathmoveto{\pgfqpoint{0.708823in}{0.671527in}}%
\pgfpathlineto{\pgfqpoint{0.859609in}{0.678928in}}%
\pgfpathlineto{\pgfqpoint{1.010395in}{0.677951in}}%
\pgfpathlineto{\pgfqpoint{1.161181in}{0.717109in}}%
\pgfpathlineto{\pgfqpoint{1.311967in}{0.690322in}}%
\pgfpathlineto{\pgfqpoint{1.462753in}{0.709863in}}%
\pgfpathlineto{\pgfqpoint{1.613539in}{0.751513in}}%
\pgfpathlineto{\pgfqpoint{1.764325in}{0.820832in}}%
\pgfpathlineto{\pgfqpoint{1.915111in}{0.878569in}}%
\pgfpathlineto{\pgfqpoint{2.065897in}{0.914162in}}%
\pgfpathlineto{\pgfqpoint{2.216683in}{1.137951in}}%
\pgfpathlineto{\pgfqpoint{2.367469in}{1.175775in}}%
\pgfpathlineto{\pgfqpoint{2.518255in}{1.410782in}}%
\pgfpathlineto{\pgfqpoint{2.669041in}{1.700231in}}%
\pgfpathlineto{\pgfqpoint{2.819827in}{2.063352in}}%
\pgfpathlineto{\pgfqpoint{2.970613in}{2.111064in}}%
\pgfpathlineto{\pgfqpoint{3.121399in}{1.880269in}}%
\pgfpathlineto{\pgfqpoint{3.272185in}{1.837341in}}%
\pgfpathlineto{\pgfqpoint{3.422971in}{2.111034in}}%
\pgfpathlineto{\pgfqpoint{3.573756in}{2.111044in}}%
\pgfpathlineto{\pgfqpoint{3.724542in}{2.042036in}}%
\pgfusepath{stroke}%
\end{pgfscope}%
\begin{pgfscope}%
\pgfpathrectangle{\pgfqpoint{0.557986in}{0.438889in}}{\pgfqpoint{3.318403in}{2.263889in}}%
\pgfusepath{clip}%
\pgfsetrectcap%
\pgfsetroundjoin%
\pgfsetlinewidth{1.505625pt}%
\definecolor{currentstroke}{rgb}{0.172549,0.627451,0.172549}%
\pgfsetstrokecolor{currentstroke}%
\pgfsetdash{}{0pt}%
\pgfpathmoveto{\pgfqpoint{0.708823in}{0.680541in}}%
\pgfpathlineto{\pgfqpoint{0.859609in}{0.679997in}}%
\pgfpathlineto{\pgfqpoint{1.010395in}{0.708723in}}%
\pgfpathlineto{\pgfqpoint{1.161181in}{0.715009in}}%
\pgfpathlineto{\pgfqpoint{1.311967in}{0.742244in}}%
\pgfpathlineto{\pgfqpoint{1.462753in}{0.765060in}}%
\pgfpathlineto{\pgfqpoint{1.613539in}{0.762023in}}%
\pgfpathlineto{\pgfqpoint{1.764325in}{0.857474in}}%
\pgfpathlineto{\pgfqpoint{1.915111in}{0.805723in}}%
\pgfpathlineto{\pgfqpoint{2.065897in}{0.949944in}}%
\pgfpathlineto{\pgfqpoint{2.216683in}{1.102171in}}%
\pgfpathlineto{\pgfqpoint{2.367469in}{0.947582in}}%
\pgfpathlineto{\pgfqpoint{2.518255in}{1.030691in}}%
\pgfpathlineto{\pgfqpoint{2.669041in}{1.292833in}}%
\pgfpathlineto{\pgfqpoint{2.819827in}{1.981429in}}%
\pgfpathlineto{\pgfqpoint{2.970613in}{1.986670in}}%
\pgfpathlineto{\pgfqpoint{3.121399in}{1.777189in}}%
\pgfpathlineto{\pgfqpoint{3.272185in}{1.930304in}}%
\pgfpathlineto{\pgfqpoint{3.422971in}{1.999769in}}%
\pgfpathlineto{\pgfqpoint{3.573756in}{1.999389in}}%
\pgfpathlineto{\pgfqpoint{3.724542in}{1.999199in}}%
\pgfusepath{stroke}%
\end{pgfscope}%
\begin{pgfscope}%
\pgfpathrectangle{\pgfqpoint{0.557986in}{0.438889in}}{\pgfqpoint{3.318403in}{2.263889in}}%
\pgfusepath{clip}%
\pgfsetrectcap%
\pgfsetroundjoin%
\pgfsetlinewidth{1.505625pt}%
\definecolor{currentstroke}{rgb}{0.839216,0.152941,0.156863}%
\pgfsetstrokecolor{currentstroke}%
\pgfsetdash{}{0pt}%
\pgfpathmoveto{\pgfqpoint{0.708823in}{0.666813in}}%
\pgfpathlineto{\pgfqpoint{0.859609in}{0.674807in}}%
\pgfpathlineto{\pgfqpoint{1.010395in}{0.685725in}}%
\pgfpathlineto{\pgfqpoint{1.161181in}{0.688401in}}%
\pgfpathlineto{\pgfqpoint{1.311967in}{0.700754in}}%
\pgfpathlineto{\pgfqpoint{1.462753in}{0.708376in}}%
\pgfpathlineto{\pgfqpoint{1.613539in}{0.704879in}}%
\pgfpathlineto{\pgfqpoint{1.764325in}{0.748116in}}%
\pgfpathlineto{\pgfqpoint{1.915111in}{0.837081in}}%
\pgfpathlineto{\pgfqpoint{2.065897in}{0.831179in}}%
\pgfpathlineto{\pgfqpoint{2.216683in}{0.819256in}}%
\pgfpathlineto{\pgfqpoint{2.367469in}{0.771371in}}%
\pgfpathlineto{\pgfqpoint{2.518255in}{0.747130in}}%
\pgfpathlineto{\pgfqpoint{2.669041in}{0.775828in}}%
\pgfpathlineto{\pgfqpoint{2.819827in}{0.820274in}}%
\pgfpathlineto{\pgfqpoint{2.970613in}{0.805079in}}%
\pgfpathlineto{\pgfqpoint{3.121399in}{0.804892in}}%
\pgfpathlineto{\pgfqpoint{3.272185in}{0.798488in}}%
\pgfpathlineto{\pgfqpoint{3.422971in}{0.787199in}}%
\pgfpathlineto{\pgfqpoint{3.573756in}{0.794933in}}%
\pgfpathlineto{\pgfqpoint{3.724542in}{0.795994in}}%
\pgfusepath{stroke}%
\end{pgfscope}%
\begin{pgfscope}%
\pgfpathrectangle{\pgfqpoint{0.557986in}{0.438889in}}{\pgfqpoint{3.318403in}{2.263889in}}%
\pgfusepath{clip}%
\pgfsetrectcap%
\pgfsetroundjoin%
\pgfsetlinewidth{1.505625pt}%
\definecolor{currentstroke}{rgb}{0.580392,0.403922,0.741176}%
\pgfsetstrokecolor{currentstroke}%
\pgfsetdash{}{0pt}%
\pgfpathmoveto{\pgfqpoint{0.708823in}{0.671586in}}%
\pgfpathlineto{\pgfqpoint{0.859609in}{0.678810in}}%
\pgfpathlineto{\pgfqpoint{1.010395in}{0.676971in}}%
\pgfpathlineto{\pgfqpoint{1.161181in}{0.665443in}}%
\pgfpathlineto{\pgfqpoint{1.311967in}{0.716756in}}%
\pgfpathlineto{\pgfqpoint{1.462753in}{0.711667in}}%
\pgfpathlineto{\pgfqpoint{1.613539in}{0.707580in}}%
\pgfpathlineto{\pgfqpoint{1.764325in}{0.747807in}}%
\pgfpathlineto{\pgfqpoint{1.915111in}{0.723403in}}%
\pgfpathlineto{\pgfqpoint{2.065897in}{0.703011in}}%
\pgfpathlineto{\pgfqpoint{2.216683in}{0.713746in}}%
\pgfpathlineto{\pgfqpoint{2.367469in}{0.702142in}}%
\pgfpathlineto{\pgfqpoint{2.518255in}{0.708099in}}%
\pgfpathlineto{\pgfqpoint{2.669041in}{0.714705in}}%
\pgfpathlineto{\pgfqpoint{2.819827in}{0.707268in}}%
\pgfpathlineto{\pgfqpoint{2.970613in}{0.707771in}}%
\pgfpathlineto{\pgfqpoint{3.121399in}{0.709327in}}%
\pgfpathlineto{\pgfqpoint{3.272185in}{0.709310in}}%
\pgfpathlineto{\pgfqpoint{3.422971in}{0.709991in}}%
\pgfpathlineto{\pgfqpoint{3.573756in}{0.709863in}}%
\pgfpathlineto{\pgfqpoint{3.724542in}{0.710772in}}%
\pgfusepath{stroke}%
\end{pgfscope}%
\begin{pgfscope}%
\pgfsetrectcap%
\pgfsetmiterjoin%
\pgfsetlinewidth{0.803000pt}%
\definecolor{currentstroke}{rgb}{0.000000,0.000000,0.000000}%
\pgfsetstrokecolor{currentstroke}%
\pgfsetdash{}{0pt}%
\pgfpathmoveto{\pgfqpoint{0.557986in}{0.438889in}}%
\pgfpathlineto{\pgfqpoint{0.557986in}{2.702778in}}%
\pgfusepath{stroke}%
\end{pgfscope}%
\begin{pgfscope}%
\pgfsetrectcap%
\pgfsetmiterjoin%
\pgfsetlinewidth{0.803000pt}%
\definecolor{currentstroke}{rgb}{0.000000,0.000000,0.000000}%
\pgfsetstrokecolor{currentstroke}%
\pgfsetdash{}{0pt}%
\pgfpathmoveto{\pgfqpoint{3.876389in}{0.438889in}}%
\pgfpathlineto{\pgfqpoint{3.876389in}{2.702778in}}%
\pgfusepath{stroke}%
\end{pgfscope}%
\begin{pgfscope}%
\pgfsetrectcap%
\pgfsetmiterjoin%
\pgfsetlinewidth{0.803000pt}%
\definecolor{currentstroke}{rgb}{0.000000,0.000000,0.000000}%
\pgfsetstrokecolor{currentstroke}%
\pgfsetdash{}{0pt}%
\pgfpathmoveto{\pgfqpoint{0.557986in}{0.438889in}}%
\pgfpathlineto{\pgfqpoint{3.876389in}{0.438889in}}%
\pgfusepath{stroke}%
\end{pgfscope}%
\begin{pgfscope}%
\pgfsetrectcap%
\pgfsetmiterjoin%
\pgfsetlinewidth{0.803000pt}%
\definecolor{currentstroke}{rgb}{0.000000,0.000000,0.000000}%
\pgfsetstrokecolor{currentstroke}%
\pgfsetdash{}{0pt}%
\pgfpathmoveto{\pgfqpoint{0.557986in}{2.702778in}}%
\pgfpathlineto{\pgfqpoint{3.876389in}{2.702778in}}%
\pgfusepath{stroke}%
\end{pgfscope}%
\begin{pgfscope}%
\definecolor{textcolor}{rgb}{0.000000,0.000000,0.000000}%
\pgfsetstrokecolor{textcolor}%
\pgfsetfillcolor{textcolor}%
\pgftext[x=2.217187in,y=2.786111in,,base]{\color{textcolor}\sffamily\fontsize{12.000000}{14.400000}\selectfont \(\displaystyle  N = 64 \)}%
\end{pgfscope}%
\begin{pgfscope}%
\pgfsetbuttcap%
\pgfsetmiterjoin%
\definecolor{currentfill}{rgb}{1.000000,1.000000,1.000000}%
\pgfsetfillcolor{currentfill}%
\pgfsetlinewidth{0.000000pt}%
\definecolor{currentstroke}{rgb}{0.000000,0.000000,0.000000}%
\pgfsetstrokecolor{currentstroke}%
\pgfsetstrokeopacity{0.000000}%
\pgfsetdash{}{0pt}%
\pgfpathmoveto{\pgfqpoint{4.482986in}{0.438889in}}%
\pgfpathlineto{\pgfqpoint{7.801389in}{0.438889in}}%
\pgfpathlineto{\pgfqpoint{7.801389in}{2.702778in}}%
\pgfpathlineto{\pgfqpoint{4.482986in}{2.702778in}}%
\pgfpathclose%
\pgfusepath{fill}%
\end{pgfscope}%
\begin{pgfscope}%
\pgfpathrectangle{\pgfqpoint{4.482986in}{0.438889in}}{\pgfqpoint{3.318403in}{2.263889in}}%
\pgfusepath{clip}%
\pgfsetbuttcap%
\pgfsetroundjoin%
\definecolor{currentfill}{rgb}{0.121569,0.466667,0.705882}%
\pgfsetfillcolor{currentfill}%
\pgfsetlinewidth{1.003750pt}%
\definecolor{currentstroke}{rgb}{0.121569,0.466667,0.705882}%
\pgfsetstrokecolor{currentstroke}%
\pgfsetdash{}{0pt}%
\pgfsys@defobject{currentmarker}{\pgfqpoint{-0.009821in}{-0.009821in}}{\pgfqpoint{0.009821in}{0.009821in}}{%
\pgfpathmoveto{\pgfqpoint{0.000000in}{-0.009821in}}%
\pgfpathcurveto{\pgfqpoint{0.002605in}{-0.009821in}}{\pgfqpoint{0.005103in}{-0.008786in}}{\pgfqpoint{0.006944in}{-0.006944in}}%
\pgfpathcurveto{\pgfqpoint{0.008786in}{-0.005103in}}{\pgfqpoint{0.009821in}{-0.002605in}}{\pgfqpoint{0.009821in}{0.000000in}}%
\pgfpathcurveto{\pgfqpoint{0.009821in}{0.002605in}}{\pgfqpoint{0.008786in}{0.005103in}}{\pgfqpoint{0.006944in}{0.006944in}}%
\pgfpathcurveto{\pgfqpoint{0.005103in}{0.008786in}}{\pgfqpoint{0.002605in}{0.009821in}}{\pgfqpoint{0.000000in}{0.009821in}}%
\pgfpathcurveto{\pgfqpoint{-0.002605in}{0.009821in}}{\pgfqpoint{-0.005103in}{0.008786in}}{\pgfqpoint{-0.006944in}{0.006944in}}%
\pgfpathcurveto{\pgfqpoint{-0.008786in}{0.005103in}}{\pgfqpoint{-0.009821in}{0.002605in}}{\pgfqpoint{-0.009821in}{0.000000in}}%
\pgfpathcurveto{\pgfqpoint{-0.009821in}{-0.002605in}}{\pgfqpoint{-0.008786in}{-0.005103in}}{\pgfqpoint{-0.006944in}{-0.006944in}}%
\pgfpathcurveto{\pgfqpoint{-0.005103in}{-0.008786in}}{\pgfqpoint{-0.002605in}{-0.009821in}}{\pgfqpoint{0.000000in}{-0.009821in}}%
\pgfpathclose%
\pgfusepath{stroke,fill}%
}%
\begin{pgfscope}%
\pgfsys@transformshift{4.633823in}{0.764749in}%
\pgfsys@useobject{currentmarker}{}%
\end{pgfscope}%
\begin{pgfscope}%
\pgfsys@transformshift{4.784609in}{0.767561in}%
\pgfsys@useobject{currentmarker}{}%
\end{pgfscope}%
\begin{pgfscope}%
\pgfsys@transformshift{4.935395in}{0.756919in}%
\pgfsys@useobject{currentmarker}{}%
\end{pgfscope}%
\begin{pgfscope}%
\pgfsys@transformshift{5.086181in}{0.759788in}%
\pgfsys@useobject{currentmarker}{}%
\end{pgfscope}%
\begin{pgfscope}%
\pgfsys@transformshift{5.236967in}{0.768130in}%
\pgfsys@useobject{currentmarker}{}%
\end{pgfscope}%
\begin{pgfscope}%
\pgfsys@transformshift{5.387753in}{0.786866in}%
\pgfsys@useobject{currentmarker}{}%
\end{pgfscope}%
\begin{pgfscope}%
\pgfsys@transformshift{5.538539in}{0.792462in}%
\pgfsys@useobject{currentmarker}{}%
\end{pgfscope}%
\begin{pgfscope}%
\pgfsys@transformshift{5.689325in}{0.792702in}%
\pgfsys@useobject{currentmarker}{}%
\end{pgfscope}%
\begin{pgfscope}%
\pgfsys@transformshift{5.840111in}{0.835332in}%
\pgfsys@useobject{currentmarker}{}%
\end{pgfscope}%
\begin{pgfscope}%
\pgfsys@transformshift{5.990897in}{0.786811in}%
\pgfsys@useobject{currentmarker}{}%
\end{pgfscope}%
\begin{pgfscope}%
\pgfsys@transformshift{6.141683in}{0.804543in}%
\pgfsys@useobject{currentmarker}{}%
\end{pgfscope}%
\begin{pgfscope}%
\pgfsys@transformshift{6.292469in}{0.828774in}%
\pgfsys@useobject{currentmarker}{}%
\end{pgfscope}%
\begin{pgfscope}%
\pgfsys@transformshift{6.443255in}{0.864559in}%
\pgfsys@useobject{currentmarker}{}%
\end{pgfscope}%
\begin{pgfscope}%
\pgfsys@transformshift{6.594041in}{0.851787in}%
\pgfsys@useobject{currentmarker}{}%
\end{pgfscope}%
\begin{pgfscope}%
\pgfsys@transformshift{6.744827in}{1.088990in}%
\pgfsys@useobject{currentmarker}{}%
\end{pgfscope}%
\begin{pgfscope}%
\pgfsys@transformshift{6.895613in}{1.055189in}%
\pgfsys@useobject{currentmarker}{}%
\end{pgfscope}%
\begin{pgfscope}%
\pgfsys@transformshift{7.046399in}{1.487584in}%
\pgfsys@useobject{currentmarker}{}%
\end{pgfscope}%
\begin{pgfscope}%
\pgfsys@transformshift{7.197185in}{1.634492in}%
\pgfsys@useobject{currentmarker}{}%
\end{pgfscope}%
\begin{pgfscope}%
\pgfsys@transformshift{7.347971in}{0.933548in}%
\pgfsys@useobject{currentmarker}{}%
\end{pgfscope}%
\begin{pgfscope}%
\pgfsys@transformshift{7.498756in}{1.544487in}%
\pgfsys@useobject{currentmarker}{}%
\end{pgfscope}%
\begin{pgfscope}%
\pgfsys@transformshift{7.649542in}{1.363508in}%
\pgfsys@useobject{currentmarker}{}%
\end{pgfscope}%
\end{pgfscope}%
\begin{pgfscope}%
\pgfpathrectangle{\pgfqpoint{4.482986in}{0.438889in}}{\pgfqpoint{3.318403in}{2.263889in}}%
\pgfusepath{clip}%
\pgfsetbuttcap%
\pgfsetroundjoin%
\definecolor{currentfill}{rgb}{0.121569,0.466667,0.705882}%
\pgfsetfillcolor{currentfill}%
\pgfsetfillopacity{0.300000}%
\pgfsetlinewidth{1.003750pt}%
\definecolor{currentstroke}{rgb}{0.121569,0.466667,0.705882}%
\pgfsetstrokecolor{currentstroke}%
\pgfsetstrokeopacity{0.300000}%
\pgfsetdash{}{0pt}%
\pgfpathmoveto{\pgfqpoint{4.633823in}{0.774420in}}%
\pgfpathlineto{\pgfqpoint{4.633823in}{0.755078in}}%
\pgfpathlineto{\pgfqpoint{4.784609in}{0.759195in}}%
\pgfpathlineto{\pgfqpoint{4.935395in}{0.750956in}}%
\pgfpathlineto{\pgfqpoint{5.086181in}{0.748972in}}%
\pgfpathlineto{\pgfqpoint{5.236967in}{0.754222in}}%
\pgfpathlineto{\pgfqpoint{5.387753in}{0.757155in}}%
\pgfpathlineto{\pgfqpoint{5.538539in}{0.764500in}}%
\pgfpathlineto{\pgfqpoint{5.689325in}{0.756154in}}%
\pgfpathlineto{\pgfqpoint{5.840111in}{0.811698in}}%
\pgfpathlineto{\pgfqpoint{5.990897in}{0.739408in}}%
\pgfpathlineto{\pgfqpoint{6.141683in}{0.739665in}}%
\pgfpathlineto{\pgfqpoint{6.292469in}{0.772666in}}%
\pgfpathlineto{\pgfqpoint{6.443255in}{0.740392in}}%
\pgfpathlineto{\pgfqpoint{6.594041in}{0.773335in}}%
\pgfpathlineto{\pgfqpoint{6.744827in}{0.752282in}}%
\pgfpathlineto{\pgfqpoint{6.895613in}{0.791346in}}%
\pgfpathlineto{\pgfqpoint{7.046399in}{0.702963in}}%
\pgfpathlineto{\pgfqpoint{7.197185in}{0.983051in}}%
\pgfpathlineto{\pgfqpoint{7.347971in}{0.764210in}}%
\pgfpathlineto{\pgfqpoint{7.498756in}{0.900461in}}%
\pgfpathlineto{\pgfqpoint{7.649542in}{0.569181in}}%
\pgfpathlineto{\pgfqpoint{7.649542in}{2.157835in}}%
\pgfpathlineto{\pgfqpoint{7.649542in}{2.157835in}}%
\pgfpathlineto{\pgfqpoint{7.498756in}{2.188512in}}%
\pgfpathlineto{\pgfqpoint{7.347971in}{1.102886in}}%
\pgfpathlineto{\pgfqpoint{7.197185in}{2.285934in}}%
\pgfpathlineto{\pgfqpoint{7.046399in}{2.272205in}}%
\pgfpathlineto{\pgfqpoint{6.895613in}{1.319032in}}%
\pgfpathlineto{\pgfqpoint{6.744827in}{1.425698in}}%
\pgfpathlineto{\pgfqpoint{6.594041in}{0.930239in}}%
\pgfpathlineto{\pgfqpoint{6.443255in}{0.988726in}}%
\pgfpathlineto{\pgfqpoint{6.292469in}{0.884882in}}%
\pgfpathlineto{\pgfqpoint{6.141683in}{0.869421in}}%
\pgfpathlineto{\pgfqpoint{5.990897in}{0.834214in}}%
\pgfpathlineto{\pgfqpoint{5.840111in}{0.858966in}}%
\pgfpathlineto{\pgfqpoint{5.689325in}{0.829250in}}%
\pgfpathlineto{\pgfqpoint{5.538539in}{0.820425in}}%
\pgfpathlineto{\pgfqpoint{5.387753in}{0.816576in}}%
\pgfpathlineto{\pgfqpoint{5.236967in}{0.782038in}}%
\pgfpathlineto{\pgfqpoint{5.086181in}{0.770604in}}%
\pgfpathlineto{\pgfqpoint{4.935395in}{0.762881in}}%
\pgfpathlineto{\pgfqpoint{4.784609in}{0.775926in}}%
\pgfpathlineto{\pgfqpoint{4.633823in}{0.774420in}}%
\pgfpathclose%
\pgfusepath{stroke,fill}%
\end{pgfscope}%
\begin{pgfscope}%
\pgfpathrectangle{\pgfqpoint{4.482986in}{0.438889in}}{\pgfqpoint{3.318403in}{2.263889in}}%
\pgfusepath{clip}%
\pgfsetbuttcap%
\pgfsetroundjoin%
\definecolor{currentfill}{rgb}{1.000000,0.498039,0.054902}%
\pgfsetfillcolor{currentfill}%
\pgfsetlinewidth{1.003750pt}%
\definecolor{currentstroke}{rgb}{1.000000,0.498039,0.054902}%
\pgfsetstrokecolor{currentstroke}%
\pgfsetdash{}{0pt}%
\pgfsys@defobject{currentmarker}{\pgfqpoint{-0.009821in}{-0.009821in}}{\pgfqpoint{0.009821in}{0.009821in}}{%
\pgfpathmoveto{\pgfqpoint{0.000000in}{-0.009821in}}%
\pgfpathcurveto{\pgfqpoint{0.002605in}{-0.009821in}}{\pgfqpoint{0.005103in}{-0.008786in}}{\pgfqpoint{0.006944in}{-0.006944in}}%
\pgfpathcurveto{\pgfqpoint{0.008786in}{-0.005103in}}{\pgfqpoint{0.009821in}{-0.002605in}}{\pgfqpoint{0.009821in}{0.000000in}}%
\pgfpathcurveto{\pgfqpoint{0.009821in}{0.002605in}}{\pgfqpoint{0.008786in}{0.005103in}}{\pgfqpoint{0.006944in}{0.006944in}}%
\pgfpathcurveto{\pgfqpoint{0.005103in}{0.008786in}}{\pgfqpoint{0.002605in}{0.009821in}}{\pgfqpoint{0.000000in}{0.009821in}}%
\pgfpathcurveto{\pgfqpoint{-0.002605in}{0.009821in}}{\pgfqpoint{-0.005103in}{0.008786in}}{\pgfqpoint{-0.006944in}{0.006944in}}%
\pgfpathcurveto{\pgfqpoint{-0.008786in}{0.005103in}}{\pgfqpoint{-0.009821in}{0.002605in}}{\pgfqpoint{-0.009821in}{0.000000in}}%
\pgfpathcurveto{\pgfqpoint{-0.009821in}{-0.002605in}}{\pgfqpoint{-0.008786in}{-0.005103in}}{\pgfqpoint{-0.006944in}{-0.006944in}}%
\pgfpathcurveto{\pgfqpoint{-0.005103in}{-0.008786in}}{\pgfqpoint{-0.002605in}{-0.009821in}}{\pgfqpoint{0.000000in}{-0.009821in}}%
\pgfpathclose%
\pgfusepath{stroke,fill}%
}%
\begin{pgfscope}%
\pgfsys@transformshift{4.633823in}{0.761548in}%
\pgfsys@useobject{currentmarker}{}%
\end{pgfscope}%
\begin{pgfscope}%
\pgfsys@transformshift{4.784609in}{0.765046in}%
\pgfsys@useobject{currentmarker}{}%
\end{pgfscope}%
\begin{pgfscope}%
\pgfsys@transformshift{4.935395in}{0.759543in}%
\pgfsys@useobject{currentmarker}{}%
\end{pgfscope}%
\begin{pgfscope}%
\pgfsys@transformshift{5.086181in}{0.755540in}%
\pgfsys@useobject{currentmarker}{}%
\end{pgfscope}%
\begin{pgfscope}%
\pgfsys@transformshift{5.236967in}{0.763161in}%
\pgfsys@useobject{currentmarker}{}%
\end{pgfscope}%
\begin{pgfscope}%
\pgfsys@transformshift{5.387753in}{0.783863in}%
\pgfsys@useobject{currentmarker}{}%
\end{pgfscope}%
\begin{pgfscope}%
\pgfsys@transformshift{5.538539in}{0.763983in}%
\pgfsys@useobject{currentmarker}{}%
\end{pgfscope}%
\begin{pgfscope}%
\pgfsys@transformshift{5.689325in}{0.776281in}%
\pgfsys@useobject{currentmarker}{}%
\end{pgfscope}%
\begin{pgfscope}%
\pgfsys@transformshift{5.840111in}{0.786858in}%
\pgfsys@useobject{currentmarker}{}%
\end{pgfscope}%
\begin{pgfscope}%
\pgfsys@transformshift{5.990897in}{0.820554in}%
\pgfsys@useobject{currentmarker}{}%
\end{pgfscope}%
\begin{pgfscope}%
\pgfsys@transformshift{6.141683in}{0.771508in}%
\pgfsys@useobject{currentmarker}{}%
\end{pgfscope}%
\begin{pgfscope}%
\pgfsys@transformshift{6.292469in}{0.912148in}%
\pgfsys@useobject{currentmarker}{}%
\end{pgfscope}%
\begin{pgfscope}%
\pgfsys@transformshift{6.443255in}{0.856542in}%
\pgfsys@useobject{currentmarker}{}%
\end{pgfscope}%
\begin{pgfscope}%
\pgfsys@transformshift{6.594041in}{0.916649in}%
\pgfsys@useobject{currentmarker}{}%
\end{pgfscope}%
\begin{pgfscope}%
\pgfsys@transformshift{6.744827in}{1.040257in}%
\pgfsys@useobject{currentmarker}{}%
\end{pgfscope}%
\begin{pgfscope}%
\pgfsys@transformshift{6.895613in}{1.227234in}%
\pgfsys@useobject{currentmarker}{}%
\end{pgfscope}%
\begin{pgfscope}%
\pgfsys@transformshift{7.046399in}{1.061679in}%
\pgfsys@useobject{currentmarker}{}%
\end{pgfscope}%
\begin{pgfscope}%
\pgfsys@transformshift{7.197185in}{1.622452in}%
\pgfsys@useobject{currentmarker}{}%
\end{pgfscope}%
\begin{pgfscope}%
\pgfsys@transformshift{7.347971in}{1.756146in}%
\pgfsys@useobject{currentmarker}{}%
\end{pgfscope}%
\begin{pgfscope}%
\pgfsys@transformshift{7.498756in}{1.565727in}%
\pgfsys@useobject{currentmarker}{}%
\end{pgfscope}%
\begin{pgfscope}%
\pgfsys@transformshift{7.649542in}{1.672714in}%
\pgfsys@useobject{currentmarker}{}%
\end{pgfscope}%
\end{pgfscope}%
\begin{pgfscope}%
\pgfpathrectangle{\pgfqpoint{4.482986in}{0.438889in}}{\pgfqpoint{3.318403in}{2.263889in}}%
\pgfusepath{clip}%
\pgfsetbuttcap%
\pgfsetroundjoin%
\definecolor{currentfill}{rgb}{1.000000,0.498039,0.054902}%
\pgfsetfillcolor{currentfill}%
\pgfsetfillopacity{0.300000}%
\pgfsetlinewidth{1.003750pt}%
\definecolor{currentstroke}{rgb}{1.000000,0.498039,0.054902}%
\pgfsetstrokecolor{currentstroke}%
\pgfsetstrokeopacity{0.300000}%
\pgfsetdash{}{0pt}%
\pgfpathmoveto{\pgfqpoint{4.633823in}{0.772767in}}%
\pgfpathlineto{\pgfqpoint{4.633823in}{0.750329in}}%
\pgfpathlineto{\pgfqpoint{4.784609in}{0.755327in}}%
\pgfpathlineto{\pgfqpoint{4.935395in}{0.745210in}}%
\pgfpathlineto{\pgfqpoint{5.086181in}{0.747656in}}%
\pgfpathlineto{\pgfqpoint{5.236967in}{0.758111in}}%
\pgfpathlineto{\pgfqpoint{5.387753in}{0.769071in}}%
\pgfpathlineto{\pgfqpoint{5.538539in}{0.754717in}}%
\pgfpathlineto{\pgfqpoint{5.689325in}{0.750667in}}%
\pgfpathlineto{\pgfqpoint{5.840111in}{0.756280in}}%
\pgfpathlineto{\pgfqpoint{5.990897in}{0.762543in}}%
\pgfpathlineto{\pgfqpoint{6.141683in}{0.754769in}}%
\pgfpathlineto{\pgfqpoint{6.292469in}{0.779004in}}%
\pgfpathlineto{\pgfqpoint{6.443255in}{0.758541in}}%
\pgfpathlineto{\pgfqpoint{6.594041in}{0.705387in}}%
\pgfpathlineto{\pgfqpoint{6.744827in}{0.744497in}}%
\pgfpathlineto{\pgfqpoint{6.895613in}{1.011439in}}%
\pgfpathlineto{\pgfqpoint{7.046399in}{0.607297in}}%
\pgfpathlineto{\pgfqpoint{7.197185in}{0.792538in}}%
\pgfpathlineto{\pgfqpoint{7.347971in}{0.912418in}}%
\pgfpathlineto{\pgfqpoint{7.498756in}{0.541793in}}%
\pgfpathlineto{\pgfqpoint{7.649542in}{0.783542in}}%
\pgfpathlineto{\pgfqpoint{7.649542in}{2.561887in}}%
\pgfpathlineto{\pgfqpoint{7.649542in}{2.561887in}}%
\pgfpathlineto{\pgfqpoint{7.498756in}{2.589661in}}%
\pgfpathlineto{\pgfqpoint{7.347971in}{2.599874in}}%
\pgfpathlineto{\pgfqpoint{7.197185in}{2.452366in}}%
\pgfpathlineto{\pgfqpoint{7.046399in}{1.516062in}}%
\pgfpathlineto{\pgfqpoint{6.895613in}{1.443029in}}%
\pgfpathlineto{\pgfqpoint{6.744827in}{1.336016in}}%
\pgfpathlineto{\pgfqpoint{6.594041in}{1.127911in}}%
\pgfpathlineto{\pgfqpoint{6.443255in}{0.954544in}}%
\pgfpathlineto{\pgfqpoint{6.292469in}{1.045293in}}%
\pgfpathlineto{\pgfqpoint{6.141683in}{0.788246in}}%
\pgfpathlineto{\pgfqpoint{5.990897in}{0.878566in}}%
\pgfpathlineto{\pgfqpoint{5.840111in}{0.817436in}}%
\pgfpathlineto{\pgfqpoint{5.689325in}{0.801895in}}%
\pgfpathlineto{\pgfqpoint{5.538539in}{0.773248in}}%
\pgfpathlineto{\pgfqpoint{5.387753in}{0.798656in}}%
\pgfpathlineto{\pgfqpoint{5.236967in}{0.768212in}}%
\pgfpathlineto{\pgfqpoint{5.086181in}{0.763424in}}%
\pgfpathlineto{\pgfqpoint{4.935395in}{0.773875in}}%
\pgfpathlineto{\pgfqpoint{4.784609in}{0.774764in}}%
\pgfpathlineto{\pgfqpoint{4.633823in}{0.772767in}}%
\pgfpathclose%
\pgfusepath{stroke,fill}%
\end{pgfscope}%
\begin{pgfscope}%
\pgfpathrectangle{\pgfqpoint{4.482986in}{0.438889in}}{\pgfqpoint{3.318403in}{2.263889in}}%
\pgfusepath{clip}%
\pgfsetbuttcap%
\pgfsetroundjoin%
\definecolor{currentfill}{rgb}{0.172549,0.627451,0.172549}%
\pgfsetfillcolor{currentfill}%
\pgfsetlinewidth{1.003750pt}%
\definecolor{currentstroke}{rgb}{0.172549,0.627451,0.172549}%
\pgfsetstrokecolor{currentstroke}%
\pgfsetdash{}{0pt}%
\pgfsys@defobject{currentmarker}{\pgfqpoint{-0.009821in}{-0.009821in}}{\pgfqpoint{0.009821in}{0.009821in}}{%
\pgfpathmoveto{\pgfqpoint{0.000000in}{-0.009821in}}%
\pgfpathcurveto{\pgfqpoint{0.002605in}{-0.009821in}}{\pgfqpoint{0.005103in}{-0.008786in}}{\pgfqpoint{0.006944in}{-0.006944in}}%
\pgfpathcurveto{\pgfqpoint{0.008786in}{-0.005103in}}{\pgfqpoint{0.009821in}{-0.002605in}}{\pgfqpoint{0.009821in}{0.000000in}}%
\pgfpathcurveto{\pgfqpoint{0.009821in}{0.002605in}}{\pgfqpoint{0.008786in}{0.005103in}}{\pgfqpoint{0.006944in}{0.006944in}}%
\pgfpathcurveto{\pgfqpoint{0.005103in}{0.008786in}}{\pgfqpoint{0.002605in}{0.009821in}}{\pgfqpoint{0.000000in}{0.009821in}}%
\pgfpathcurveto{\pgfqpoint{-0.002605in}{0.009821in}}{\pgfqpoint{-0.005103in}{0.008786in}}{\pgfqpoint{-0.006944in}{0.006944in}}%
\pgfpathcurveto{\pgfqpoint{-0.008786in}{0.005103in}}{\pgfqpoint{-0.009821in}{0.002605in}}{\pgfqpoint{-0.009821in}{0.000000in}}%
\pgfpathcurveto{\pgfqpoint{-0.009821in}{-0.002605in}}{\pgfqpoint{-0.008786in}{-0.005103in}}{\pgfqpoint{-0.006944in}{-0.006944in}}%
\pgfpathcurveto{\pgfqpoint{-0.005103in}{-0.008786in}}{\pgfqpoint{-0.002605in}{-0.009821in}}{\pgfqpoint{0.000000in}{-0.009821in}}%
\pgfpathclose%
\pgfusepath{stroke,fill}%
}%
\begin{pgfscope}%
\pgfsys@transformshift{4.633823in}{0.757559in}%
\pgfsys@useobject{currentmarker}{}%
\end{pgfscope}%
\begin{pgfscope}%
\pgfsys@transformshift{4.784609in}{0.762641in}%
\pgfsys@useobject{currentmarker}{}%
\end{pgfscope}%
\begin{pgfscope}%
\pgfsys@transformshift{4.935395in}{0.754460in}%
\pgfsys@useobject{currentmarker}{}%
\end{pgfscope}%
\begin{pgfscope}%
\pgfsys@transformshift{5.086181in}{0.770158in}%
\pgfsys@useobject{currentmarker}{}%
\end{pgfscope}%
\begin{pgfscope}%
\pgfsys@transformshift{5.236967in}{0.769649in}%
\pgfsys@useobject{currentmarker}{}%
\end{pgfscope}%
\begin{pgfscope}%
\pgfsys@transformshift{5.387753in}{0.765221in}%
\pgfsys@useobject{currentmarker}{}%
\end{pgfscope}%
\begin{pgfscope}%
\pgfsys@transformshift{5.538539in}{0.786852in}%
\pgfsys@useobject{currentmarker}{}%
\end{pgfscope}%
\begin{pgfscope}%
\pgfsys@transformshift{5.689325in}{0.773442in}%
\pgfsys@useobject{currentmarker}{}%
\end{pgfscope}%
\begin{pgfscope}%
\pgfsys@transformshift{5.840111in}{0.857947in}%
\pgfsys@useobject{currentmarker}{}%
\end{pgfscope}%
\begin{pgfscope}%
\pgfsys@transformshift{5.990897in}{0.789164in}%
\pgfsys@useobject{currentmarker}{}%
\end{pgfscope}%
\begin{pgfscope}%
\pgfsys@transformshift{6.141683in}{0.799222in}%
\pgfsys@useobject{currentmarker}{}%
\end{pgfscope}%
\begin{pgfscope}%
\pgfsys@transformshift{6.292469in}{0.911399in}%
\pgfsys@useobject{currentmarker}{}%
\end{pgfscope}%
\begin{pgfscope}%
\pgfsys@transformshift{6.443255in}{1.002522in}%
\pgfsys@useobject{currentmarker}{}%
\end{pgfscope}%
\begin{pgfscope}%
\pgfsys@transformshift{6.594041in}{1.152909in}%
\pgfsys@useobject{currentmarker}{}%
\end{pgfscope}%
\begin{pgfscope}%
\pgfsys@transformshift{6.744827in}{1.142144in}%
\pgfsys@useobject{currentmarker}{}%
\end{pgfscope}%
\begin{pgfscope}%
\pgfsys@transformshift{6.895613in}{0.841196in}%
\pgfsys@useobject{currentmarker}{}%
\end{pgfscope}%
\begin{pgfscope}%
\pgfsys@transformshift{7.046399in}{1.382422in}%
\pgfsys@useobject{currentmarker}{}%
\end{pgfscope}%
\begin{pgfscope}%
\pgfsys@transformshift{7.197185in}{1.793381in}%
\pgfsys@useobject{currentmarker}{}%
\end{pgfscope}%
\begin{pgfscope}%
\pgfsys@transformshift{7.347971in}{1.474619in}%
\pgfsys@useobject{currentmarker}{}%
\end{pgfscope}%
\begin{pgfscope}%
\pgfsys@transformshift{7.498756in}{2.110144in}%
\pgfsys@useobject{currentmarker}{}%
\end{pgfscope}%
\begin{pgfscope}%
\pgfsys@transformshift{7.649542in}{1.870352in}%
\pgfsys@useobject{currentmarker}{}%
\end{pgfscope}%
\end{pgfscope}%
\begin{pgfscope}%
\pgfpathrectangle{\pgfqpoint{4.482986in}{0.438889in}}{\pgfqpoint{3.318403in}{2.263889in}}%
\pgfusepath{clip}%
\pgfsetbuttcap%
\pgfsetroundjoin%
\definecolor{currentfill}{rgb}{0.172549,0.627451,0.172549}%
\pgfsetfillcolor{currentfill}%
\pgfsetfillopacity{0.300000}%
\pgfsetlinewidth{1.003750pt}%
\definecolor{currentstroke}{rgb}{0.172549,0.627451,0.172549}%
\pgfsetstrokecolor{currentstroke}%
\pgfsetstrokeopacity{0.300000}%
\pgfsetdash{}{0pt}%
\pgfpathmoveto{\pgfqpoint{4.633823in}{0.773351in}}%
\pgfpathlineto{\pgfqpoint{4.633823in}{0.741767in}}%
\pgfpathlineto{\pgfqpoint{4.784609in}{0.747902in}}%
\pgfpathlineto{\pgfqpoint{4.935395in}{0.751095in}}%
\pgfpathlineto{\pgfqpoint{5.086181in}{0.752537in}}%
\pgfpathlineto{\pgfqpoint{5.236967in}{0.745467in}}%
\pgfpathlineto{\pgfqpoint{5.387753in}{0.754114in}}%
\pgfpathlineto{\pgfqpoint{5.538539in}{0.750579in}}%
\pgfpathlineto{\pgfqpoint{5.689325in}{0.758323in}}%
\pgfpathlineto{\pgfqpoint{5.840111in}{0.788052in}}%
\pgfpathlineto{\pgfqpoint{5.990897in}{0.738447in}}%
\pgfpathlineto{\pgfqpoint{6.141683in}{0.715985in}}%
\pgfpathlineto{\pgfqpoint{6.292469in}{0.870381in}}%
\pgfpathlineto{\pgfqpoint{6.443255in}{0.752686in}}%
\pgfpathlineto{\pgfqpoint{6.594041in}{0.874066in}}%
\pgfpathlineto{\pgfqpoint{6.744827in}{0.673743in}}%
\pgfpathlineto{\pgfqpoint{6.895613in}{0.769416in}}%
\pgfpathlineto{\pgfqpoint{7.046399in}{0.654641in}}%
\pgfpathlineto{\pgfqpoint{7.197185in}{1.101894in}}%
\pgfpathlineto{\pgfqpoint{7.347971in}{0.835487in}}%
\pgfpathlineto{\pgfqpoint{7.498756in}{2.064803in}}%
\pgfpathlineto{\pgfqpoint{7.649542in}{1.155846in}}%
\pgfpathlineto{\pgfqpoint{7.649542in}{2.584858in}}%
\pgfpathlineto{\pgfqpoint{7.649542in}{2.584858in}}%
\pgfpathlineto{\pgfqpoint{7.498756in}{2.155484in}}%
\pgfpathlineto{\pgfqpoint{7.347971in}{2.113751in}}%
\pgfpathlineto{\pgfqpoint{7.197185in}{2.484868in}}%
\pgfpathlineto{\pgfqpoint{7.046399in}{2.110203in}}%
\pgfpathlineto{\pgfqpoint{6.895613in}{0.912975in}}%
\pgfpathlineto{\pgfqpoint{6.744827in}{1.610545in}}%
\pgfpathlineto{\pgfqpoint{6.594041in}{1.431753in}}%
\pgfpathlineto{\pgfqpoint{6.443255in}{1.252358in}}%
\pgfpathlineto{\pgfqpoint{6.292469in}{0.952418in}}%
\pgfpathlineto{\pgfqpoint{6.141683in}{0.882460in}}%
\pgfpathlineto{\pgfqpoint{5.990897in}{0.839881in}}%
\pgfpathlineto{\pgfqpoint{5.840111in}{0.927843in}}%
\pgfpathlineto{\pgfqpoint{5.689325in}{0.788562in}}%
\pgfpathlineto{\pgfqpoint{5.538539in}{0.823126in}}%
\pgfpathlineto{\pgfqpoint{5.387753in}{0.776328in}}%
\pgfpathlineto{\pgfqpoint{5.236967in}{0.793832in}}%
\pgfpathlineto{\pgfqpoint{5.086181in}{0.787779in}}%
\pgfpathlineto{\pgfqpoint{4.935395in}{0.757826in}}%
\pgfpathlineto{\pgfqpoint{4.784609in}{0.777380in}}%
\pgfpathlineto{\pgfqpoint{4.633823in}{0.773351in}}%
\pgfpathclose%
\pgfusepath{stroke,fill}%
\end{pgfscope}%
\begin{pgfscope}%
\pgfpathrectangle{\pgfqpoint{4.482986in}{0.438889in}}{\pgfqpoint{3.318403in}{2.263889in}}%
\pgfusepath{clip}%
\pgfsetbuttcap%
\pgfsetroundjoin%
\definecolor{currentfill}{rgb}{0.839216,0.152941,0.156863}%
\pgfsetfillcolor{currentfill}%
\pgfsetlinewidth{1.003750pt}%
\definecolor{currentstroke}{rgb}{0.839216,0.152941,0.156863}%
\pgfsetstrokecolor{currentstroke}%
\pgfsetdash{}{0pt}%
\pgfsys@defobject{currentmarker}{\pgfqpoint{-0.009821in}{-0.009821in}}{\pgfqpoint{0.009821in}{0.009821in}}{%
\pgfpathmoveto{\pgfqpoint{0.000000in}{-0.009821in}}%
\pgfpathcurveto{\pgfqpoint{0.002605in}{-0.009821in}}{\pgfqpoint{0.005103in}{-0.008786in}}{\pgfqpoint{0.006944in}{-0.006944in}}%
\pgfpathcurveto{\pgfqpoint{0.008786in}{-0.005103in}}{\pgfqpoint{0.009821in}{-0.002605in}}{\pgfqpoint{0.009821in}{0.000000in}}%
\pgfpathcurveto{\pgfqpoint{0.009821in}{0.002605in}}{\pgfqpoint{0.008786in}{0.005103in}}{\pgfqpoint{0.006944in}{0.006944in}}%
\pgfpathcurveto{\pgfqpoint{0.005103in}{0.008786in}}{\pgfqpoint{0.002605in}{0.009821in}}{\pgfqpoint{0.000000in}{0.009821in}}%
\pgfpathcurveto{\pgfqpoint{-0.002605in}{0.009821in}}{\pgfqpoint{-0.005103in}{0.008786in}}{\pgfqpoint{-0.006944in}{0.006944in}}%
\pgfpathcurveto{\pgfqpoint{-0.008786in}{0.005103in}}{\pgfqpoint{-0.009821in}{0.002605in}}{\pgfqpoint{-0.009821in}{0.000000in}}%
\pgfpathcurveto{\pgfqpoint{-0.009821in}{-0.002605in}}{\pgfqpoint{-0.008786in}{-0.005103in}}{\pgfqpoint{-0.006944in}{-0.006944in}}%
\pgfpathcurveto{\pgfqpoint{-0.005103in}{-0.008786in}}{\pgfqpoint{-0.002605in}{-0.009821in}}{\pgfqpoint{0.000000in}{-0.009821in}}%
\pgfpathclose%
\pgfusepath{stroke,fill}%
}%
\begin{pgfscope}%
\pgfsys@transformshift{4.633823in}{0.757423in}%
\pgfsys@useobject{currentmarker}{}%
\end{pgfscope}%
\begin{pgfscope}%
\pgfsys@transformshift{4.784609in}{0.758159in}%
\pgfsys@useobject{currentmarker}{}%
\end{pgfscope}%
\begin{pgfscope}%
\pgfsys@transformshift{4.935395in}{0.759199in}%
\pgfsys@useobject{currentmarker}{}%
\end{pgfscope}%
\begin{pgfscope}%
\pgfsys@transformshift{5.086181in}{0.766876in}%
\pgfsys@useobject{currentmarker}{}%
\end{pgfscope}%
\begin{pgfscope}%
\pgfsys@transformshift{5.236967in}{0.756045in}%
\pgfsys@useobject{currentmarker}{}%
\end{pgfscope}%
\begin{pgfscope}%
\pgfsys@transformshift{5.387753in}{0.768701in}%
\pgfsys@useobject{currentmarker}{}%
\end{pgfscope}%
\begin{pgfscope}%
\pgfsys@transformshift{5.538539in}{0.763696in}%
\pgfsys@useobject{currentmarker}{}%
\end{pgfscope}%
\begin{pgfscope}%
\pgfsys@transformshift{5.689325in}{0.773673in}%
\pgfsys@useobject{currentmarker}{}%
\end{pgfscope}%
\begin{pgfscope}%
\pgfsys@transformshift{5.840111in}{0.802583in}%
\pgfsys@useobject{currentmarker}{}%
\end{pgfscope}%
\begin{pgfscope}%
\pgfsys@transformshift{5.990897in}{0.803897in}%
\pgfsys@useobject{currentmarker}{}%
\end{pgfscope}%
\begin{pgfscope}%
\pgfsys@transformshift{6.141683in}{0.800905in}%
\pgfsys@useobject{currentmarker}{}%
\end{pgfscope}%
\begin{pgfscope}%
\pgfsys@transformshift{6.292469in}{0.784364in}%
\pgfsys@useobject{currentmarker}{}%
\end{pgfscope}%
\begin{pgfscope}%
\pgfsys@transformshift{6.443255in}{0.802315in}%
\pgfsys@useobject{currentmarker}{}%
\end{pgfscope}%
\begin{pgfscope}%
\pgfsys@transformshift{6.594041in}{0.814135in}%
\pgfsys@useobject{currentmarker}{}%
\end{pgfscope}%
\begin{pgfscope}%
\pgfsys@transformshift{6.744827in}{0.808637in}%
\pgfsys@useobject{currentmarker}{}%
\end{pgfscope}%
\begin{pgfscope}%
\pgfsys@transformshift{6.895613in}{0.841750in}%
\pgfsys@useobject{currentmarker}{}%
\end{pgfscope}%
\begin{pgfscope}%
\pgfsys@transformshift{7.046399in}{0.819765in}%
\pgfsys@useobject{currentmarker}{}%
\end{pgfscope}%
\begin{pgfscope}%
\pgfsys@transformshift{7.197185in}{0.831620in}%
\pgfsys@useobject{currentmarker}{}%
\end{pgfscope}%
\begin{pgfscope}%
\pgfsys@transformshift{7.347971in}{0.823799in}%
\pgfsys@useobject{currentmarker}{}%
\end{pgfscope}%
\begin{pgfscope}%
\pgfsys@transformshift{7.498756in}{0.825166in}%
\pgfsys@useobject{currentmarker}{}%
\end{pgfscope}%
\begin{pgfscope}%
\pgfsys@transformshift{7.649542in}{0.820728in}%
\pgfsys@useobject{currentmarker}{}%
\end{pgfscope}%
\end{pgfscope}%
\begin{pgfscope}%
\pgfpathrectangle{\pgfqpoint{4.482986in}{0.438889in}}{\pgfqpoint{3.318403in}{2.263889in}}%
\pgfusepath{clip}%
\pgfsetbuttcap%
\pgfsetroundjoin%
\definecolor{currentfill}{rgb}{0.839216,0.152941,0.156863}%
\pgfsetfillcolor{currentfill}%
\pgfsetfillopacity{0.300000}%
\pgfsetlinewidth{1.003750pt}%
\definecolor{currentstroke}{rgb}{0.839216,0.152941,0.156863}%
\pgfsetstrokecolor{currentstroke}%
\pgfsetstrokeopacity{0.300000}%
\pgfsetdash{}{0pt}%
\pgfpathmoveto{\pgfqpoint{4.633823in}{0.764049in}}%
\pgfpathlineto{\pgfqpoint{4.633823in}{0.750798in}}%
\pgfpathlineto{\pgfqpoint{4.784609in}{0.747831in}}%
\pgfpathlineto{\pgfqpoint{4.935395in}{0.755863in}}%
\pgfpathlineto{\pgfqpoint{5.086181in}{0.751642in}}%
\pgfpathlineto{\pgfqpoint{5.236967in}{0.750076in}}%
\pgfpathlineto{\pgfqpoint{5.387753in}{0.750186in}}%
\pgfpathlineto{\pgfqpoint{5.538539in}{0.751421in}}%
\pgfpathlineto{\pgfqpoint{5.689325in}{0.758961in}}%
\pgfpathlineto{\pgfqpoint{5.840111in}{0.751167in}}%
\pgfpathlineto{\pgfqpoint{5.990897in}{0.760037in}}%
\pgfpathlineto{\pgfqpoint{6.141683in}{0.746462in}}%
\pgfpathlineto{\pgfqpoint{6.292469in}{0.759034in}}%
\pgfpathlineto{\pgfqpoint{6.443255in}{0.766670in}}%
\pgfpathlineto{\pgfqpoint{6.594041in}{0.781331in}}%
\pgfpathlineto{\pgfqpoint{6.744827in}{0.789138in}}%
\pgfpathlineto{\pgfqpoint{6.895613in}{0.798782in}}%
\pgfpathlineto{\pgfqpoint{7.046399in}{0.800903in}}%
\pgfpathlineto{\pgfqpoint{7.197185in}{0.812031in}}%
\pgfpathlineto{\pgfqpoint{7.347971in}{0.820750in}}%
\pgfpathlineto{\pgfqpoint{7.498756in}{0.813710in}}%
\pgfpathlineto{\pgfqpoint{7.649542in}{0.818159in}}%
\pgfpathlineto{\pgfqpoint{7.649542in}{0.823297in}}%
\pgfpathlineto{\pgfqpoint{7.649542in}{0.823297in}}%
\pgfpathlineto{\pgfqpoint{7.498756in}{0.836623in}}%
\pgfpathlineto{\pgfqpoint{7.347971in}{0.826849in}}%
\pgfpathlineto{\pgfqpoint{7.197185in}{0.851210in}}%
\pgfpathlineto{\pgfqpoint{7.046399in}{0.838628in}}%
\pgfpathlineto{\pgfqpoint{6.895613in}{0.884718in}}%
\pgfpathlineto{\pgfqpoint{6.744827in}{0.828136in}}%
\pgfpathlineto{\pgfqpoint{6.594041in}{0.846939in}}%
\pgfpathlineto{\pgfqpoint{6.443255in}{0.837960in}}%
\pgfpathlineto{\pgfqpoint{6.292469in}{0.809693in}}%
\pgfpathlineto{\pgfqpoint{6.141683in}{0.855349in}}%
\pgfpathlineto{\pgfqpoint{5.990897in}{0.847757in}}%
\pgfpathlineto{\pgfqpoint{5.840111in}{0.853999in}}%
\pgfpathlineto{\pgfqpoint{5.689325in}{0.788384in}}%
\pgfpathlineto{\pgfqpoint{5.538539in}{0.775971in}}%
\pgfpathlineto{\pgfqpoint{5.387753in}{0.787216in}}%
\pgfpathlineto{\pgfqpoint{5.236967in}{0.762015in}}%
\pgfpathlineto{\pgfqpoint{5.086181in}{0.782110in}}%
\pgfpathlineto{\pgfqpoint{4.935395in}{0.762535in}}%
\pgfpathlineto{\pgfqpoint{4.784609in}{0.768487in}}%
\pgfpathlineto{\pgfqpoint{4.633823in}{0.764049in}}%
\pgfpathclose%
\pgfusepath{stroke,fill}%
\end{pgfscope}%
\begin{pgfscope}%
\pgfpathrectangle{\pgfqpoint{4.482986in}{0.438889in}}{\pgfqpoint{3.318403in}{2.263889in}}%
\pgfusepath{clip}%
\pgfsetbuttcap%
\pgfsetroundjoin%
\definecolor{currentfill}{rgb}{0.580392,0.403922,0.741176}%
\pgfsetfillcolor{currentfill}%
\pgfsetlinewidth{1.003750pt}%
\definecolor{currentstroke}{rgb}{0.580392,0.403922,0.741176}%
\pgfsetstrokecolor{currentstroke}%
\pgfsetdash{}{0pt}%
\pgfsys@defobject{currentmarker}{\pgfqpoint{-0.009821in}{-0.009821in}}{\pgfqpoint{0.009821in}{0.009821in}}{%
\pgfpathmoveto{\pgfqpoint{0.000000in}{-0.009821in}}%
\pgfpathcurveto{\pgfqpoint{0.002605in}{-0.009821in}}{\pgfqpoint{0.005103in}{-0.008786in}}{\pgfqpoint{0.006944in}{-0.006944in}}%
\pgfpathcurveto{\pgfqpoint{0.008786in}{-0.005103in}}{\pgfqpoint{0.009821in}{-0.002605in}}{\pgfqpoint{0.009821in}{0.000000in}}%
\pgfpathcurveto{\pgfqpoint{0.009821in}{0.002605in}}{\pgfqpoint{0.008786in}{0.005103in}}{\pgfqpoint{0.006944in}{0.006944in}}%
\pgfpathcurveto{\pgfqpoint{0.005103in}{0.008786in}}{\pgfqpoint{0.002605in}{0.009821in}}{\pgfqpoint{0.000000in}{0.009821in}}%
\pgfpathcurveto{\pgfqpoint{-0.002605in}{0.009821in}}{\pgfqpoint{-0.005103in}{0.008786in}}{\pgfqpoint{-0.006944in}{0.006944in}}%
\pgfpathcurveto{\pgfqpoint{-0.008786in}{0.005103in}}{\pgfqpoint{-0.009821in}{0.002605in}}{\pgfqpoint{-0.009821in}{0.000000in}}%
\pgfpathcurveto{\pgfqpoint{-0.009821in}{-0.002605in}}{\pgfqpoint{-0.008786in}{-0.005103in}}{\pgfqpoint{-0.006944in}{-0.006944in}}%
\pgfpathcurveto{\pgfqpoint{-0.005103in}{-0.008786in}}{\pgfqpoint{-0.002605in}{-0.009821in}}{\pgfqpoint{0.000000in}{-0.009821in}}%
\pgfpathclose%
\pgfusepath{stroke,fill}%
}%
\begin{pgfscope}%
\pgfsys@transformshift{4.633823in}{0.762455in}%
\pgfsys@useobject{currentmarker}{}%
\end{pgfscope}%
\begin{pgfscope}%
\pgfsys@transformshift{4.784609in}{0.759105in}%
\pgfsys@useobject{currentmarker}{}%
\end{pgfscope}%
\begin{pgfscope}%
\pgfsys@transformshift{4.935395in}{0.770185in}%
\pgfsys@useobject{currentmarker}{}%
\end{pgfscope}%
\begin{pgfscope}%
\pgfsys@transformshift{5.086181in}{0.759677in}%
\pgfsys@useobject{currentmarker}{}%
\end{pgfscope}%
\begin{pgfscope}%
\pgfsys@transformshift{5.236967in}{0.771885in}%
\pgfsys@useobject{currentmarker}{}%
\end{pgfscope}%
\begin{pgfscope}%
\pgfsys@transformshift{5.387753in}{0.768289in}%
\pgfsys@useobject{currentmarker}{}%
\end{pgfscope}%
\begin{pgfscope}%
\pgfsys@transformshift{5.538539in}{0.780615in}%
\pgfsys@useobject{currentmarker}{}%
\end{pgfscope}%
\begin{pgfscope}%
\pgfsys@transformshift{5.689325in}{0.786678in}%
\pgfsys@useobject{currentmarker}{}%
\end{pgfscope}%
\begin{pgfscope}%
\pgfsys@transformshift{5.840111in}{0.792690in}%
\pgfsys@useobject{currentmarker}{}%
\end{pgfscope}%
\begin{pgfscope}%
\pgfsys@transformshift{5.990897in}{0.770196in}%
\pgfsys@useobject{currentmarker}{}%
\end{pgfscope}%
\begin{pgfscope}%
\pgfsys@transformshift{6.141683in}{0.777113in}%
\pgfsys@useobject{currentmarker}{}%
\end{pgfscope}%
\begin{pgfscope}%
\pgfsys@transformshift{6.292469in}{0.774979in}%
\pgfsys@useobject{currentmarker}{}%
\end{pgfscope}%
\begin{pgfscope}%
\pgfsys@transformshift{6.443255in}{0.779010in}%
\pgfsys@useobject{currentmarker}{}%
\end{pgfscope}%
\begin{pgfscope}%
\pgfsys@transformshift{6.594041in}{0.781131in}%
\pgfsys@useobject{currentmarker}{}%
\end{pgfscope}%
\begin{pgfscope}%
\pgfsys@transformshift{6.744827in}{0.778297in}%
\pgfsys@useobject{currentmarker}{}%
\end{pgfscope}%
\begin{pgfscope}%
\pgfsys@transformshift{6.895613in}{0.777872in}%
\pgfsys@useobject{currentmarker}{}%
\end{pgfscope}%
\begin{pgfscope}%
\pgfsys@transformshift{7.046399in}{0.781578in}%
\pgfsys@useobject{currentmarker}{}%
\end{pgfscope}%
\begin{pgfscope}%
\pgfsys@transformshift{7.197185in}{0.780584in}%
\pgfsys@useobject{currentmarker}{}%
\end{pgfscope}%
\begin{pgfscope}%
\pgfsys@transformshift{7.347971in}{0.780324in}%
\pgfsys@useobject{currentmarker}{}%
\end{pgfscope}%
\begin{pgfscope}%
\pgfsys@transformshift{7.498756in}{0.780137in}%
\pgfsys@useobject{currentmarker}{}%
\end{pgfscope}%
\begin{pgfscope}%
\pgfsys@transformshift{7.649542in}{0.779765in}%
\pgfsys@useobject{currentmarker}{}%
\end{pgfscope}%
\end{pgfscope}%
\begin{pgfscope}%
\pgfpathrectangle{\pgfqpoint{4.482986in}{0.438889in}}{\pgfqpoint{3.318403in}{2.263889in}}%
\pgfusepath{clip}%
\pgfsetbuttcap%
\pgfsetroundjoin%
\definecolor{currentfill}{rgb}{0.580392,0.403922,0.741176}%
\pgfsetfillcolor{currentfill}%
\pgfsetfillopacity{0.300000}%
\pgfsetlinewidth{1.003750pt}%
\definecolor{currentstroke}{rgb}{0.580392,0.403922,0.741176}%
\pgfsetstrokecolor{currentstroke}%
\pgfsetstrokeopacity{0.300000}%
\pgfsetdash{}{0pt}%
\pgfpathmoveto{\pgfqpoint{4.633823in}{0.764615in}}%
\pgfpathlineto{\pgfqpoint{4.633823in}{0.760294in}}%
\pgfpathlineto{\pgfqpoint{4.784609in}{0.750271in}}%
\pgfpathlineto{\pgfqpoint{4.935395in}{0.746419in}}%
\pgfpathlineto{\pgfqpoint{5.086181in}{0.747172in}}%
\pgfpathlineto{\pgfqpoint{5.236967in}{0.756207in}}%
\pgfpathlineto{\pgfqpoint{5.387753in}{0.752301in}}%
\pgfpathlineto{\pgfqpoint{5.538539in}{0.746237in}}%
\pgfpathlineto{\pgfqpoint{5.689325in}{0.764218in}}%
\pgfpathlineto{\pgfqpoint{5.840111in}{0.757663in}}%
\pgfpathlineto{\pgfqpoint{5.990897in}{0.763530in}}%
\pgfpathlineto{\pgfqpoint{6.141683in}{0.758254in}}%
\pgfpathlineto{\pgfqpoint{6.292469in}{0.764674in}}%
\pgfpathlineto{\pgfqpoint{6.443255in}{0.767998in}}%
\pgfpathlineto{\pgfqpoint{6.594041in}{0.768310in}}%
\pgfpathlineto{\pgfqpoint{6.744827in}{0.770953in}}%
\pgfpathlineto{\pgfqpoint{6.895613in}{0.773385in}}%
\pgfpathlineto{\pgfqpoint{7.046399in}{0.777636in}}%
\pgfpathlineto{\pgfqpoint{7.197185in}{0.776734in}}%
\pgfpathlineto{\pgfqpoint{7.347971in}{0.779063in}}%
\pgfpathlineto{\pgfqpoint{7.498756in}{0.778250in}}%
\pgfpathlineto{\pgfqpoint{7.649542in}{0.778948in}}%
\pgfpathlineto{\pgfqpoint{7.649542in}{0.780582in}}%
\pgfpathlineto{\pgfqpoint{7.649542in}{0.780582in}}%
\pgfpathlineto{\pgfqpoint{7.498756in}{0.782024in}}%
\pgfpathlineto{\pgfqpoint{7.347971in}{0.781585in}}%
\pgfpathlineto{\pgfqpoint{7.197185in}{0.784434in}}%
\pgfpathlineto{\pgfqpoint{7.046399in}{0.785521in}}%
\pgfpathlineto{\pgfqpoint{6.895613in}{0.782359in}}%
\pgfpathlineto{\pgfqpoint{6.744827in}{0.785641in}}%
\pgfpathlineto{\pgfqpoint{6.594041in}{0.793952in}}%
\pgfpathlineto{\pgfqpoint{6.443255in}{0.790022in}}%
\pgfpathlineto{\pgfqpoint{6.292469in}{0.785285in}}%
\pgfpathlineto{\pgfqpoint{6.141683in}{0.795971in}}%
\pgfpathlineto{\pgfqpoint{5.990897in}{0.776863in}}%
\pgfpathlineto{\pgfqpoint{5.840111in}{0.827718in}}%
\pgfpathlineto{\pgfqpoint{5.689325in}{0.809138in}}%
\pgfpathlineto{\pgfqpoint{5.538539in}{0.814994in}}%
\pgfpathlineto{\pgfqpoint{5.387753in}{0.784276in}}%
\pgfpathlineto{\pgfqpoint{5.236967in}{0.787564in}}%
\pgfpathlineto{\pgfqpoint{5.086181in}{0.772181in}}%
\pgfpathlineto{\pgfqpoint{4.935395in}{0.793952in}}%
\pgfpathlineto{\pgfqpoint{4.784609in}{0.767938in}}%
\pgfpathlineto{\pgfqpoint{4.633823in}{0.764615in}}%
\pgfpathclose%
\pgfusepath{stroke,fill}%
\end{pgfscope}%
\begin{pgfscope}%
\pgfsetbuttcap%
\pgfsetroundjoin%
\definecolor{currentfill}{rgb}{0.000000,0.000000,0.000000}%
\pgfsetfillcolor{currentfill}%
\pgfsetlinewidth{0.803000pt}%
\definecolor{currentstroke}{rgb}{0.000000,0.000000,0.000000}%
\pgfsetstrokecolor{currentstroke}%
\pgfsetdash{}{0pt}%
\pgfsys@defobject{currentmarker}{\pgfqpoint{0.000000in}{-0.048611in}}{\pgfqpoint{0.000000in}{0.000000in}}{%
\pgfpathmoveto{\pgfqpoint{0.000000in}{0.000000in}}%
\pgfpathlineto{\pgfqpoint{0.000000in}{-0.048611in}}%
\pgfusepath{stroke,fill}%
}%
\begin{pgfscope}%
\pgfsys@transformshift{4.996951in}{0.438889in}%
\pgfsys@useobject{currentmarker}{}%
\end{pgfscope}%
\end{pgfscope}%
\begin{pgfscope}%
\definecolor{textcolor}{rgb}{0.000000,0.000000,0.000000}%
\pgfsetstrokecolor{textcolor}%
\pgfsetfillcolor{textcolor}%
\pgftext[x=4.996951in,y=0.341667in,,top]{\color{textcolor}\sffamily\fontsize{10.000000}{12.000000}\selectfont \(\displaystyle {10^{4}}\)}%
\end{pgfscope}%
\begin{pgfscope}%
\pgfsetbuttcap%
\pgfsetroundjoin%
\definecolor{currentfill}{rgb}{0.000000,0.000000,0.000000}%
\pgfsetfillcolor{currentfill}%
\pgfsetlinewidth{0.803000pt}%
\definecolor{currentstroke}{rgb}{0.000000,0.000000,0.000000}%
\pgfsetstrokecolor{currentstroke}%
\pgfsetdash{}{0pt}%
\pgfsys@defobject{currentmarker}{\pgfqpoint{0.000000in}{-0.048611in}}{\pgfqpoint{0.000000in}{0.000000in}}{%
\pgfpathmoveto{\pgfqpoint{0.000000in}{0.000000in}}%
\pgfpathlineto{\pgfqpoint{0.000000in}{-0.048611in}}%
\pgfusepath{stroke,fill}%
}%
\begin{pgfscope}%
\pgfsys@transformshift{5.600095in}{0.438889in}%
\pgfsys@useobject{currentmarker}{}%
\end{pgfscope}%
\end{pgfscope}%
\begin{pgfscope}%
\definecolor{textcolor}{rgb}{0.000000,0.000000,0.000000}%
\pgfsetstrokecolor{textcolor}%
\pgfsetfillcolor{textcolor}%
\pgftext[x=5.600095in,y=0.341667in,,top]{\color{textcolor}\sffamily\fontsize{10.000000}{12.000000}\selectfont \(\displaystyle {10^{5}}\)}%
\end{pgfscope}%
\begin{pgfscope}%
\pgfsetbuttcap%
\pgfsetroundjoin%
\definecolor{currentfill}{rgb}{0.000000,0.000000,0.000000}%
\pgfsetfillcolor{currentfill}%
\pgfsetlinewidth{0.803000pt}%
\definecolor{currentstroke}{rgb}{0.000000,0.000000,0.000000}%
\pgfsetstrokecolor{currentstroke}%
\pgfsetdash{}{0pt}%
\pgfsys@defobject{currentmarker}{\pgfqpoint{0.000000in}{-0.048611in}}{\pgfqpoint{0.000000in}{0.000000in}}{%
\pgfpathmoveto{\pgfqpoint{0.000000in}{0.000000in}}%
\pgfpathlineto{\pgfqpoint{0.000000in}{-0.048611in}}%
\pgfusepath{stroke,fill}%
}%
\begin{pgfscope}%
\pgfsys@transformshift{6.203239in}{0.438889in}%
\pgfsys@useobject{currentmarker}{}%
\end{pgfscope}%
\end{pgfscope}%
\begin{pgfscope}%
\definecolor{textcolor}{rgb}{0.000000,0.000000,0.000000}%
\pgfsetstrokecolor{textcolor}%
\pgfsetfillcolor{textcolor}%
\pgftext[x=6.203239in,y=0.341667in,,top]{\color{textcolor}\sffamily\fontsize{10.000000}{12.000000}\selectfont \(\displaystyle {10^{6}}\)}%
\end{pgfscope}%
\begin{pgfscope}%
\pgfsetbuttcap%
\pgfsetroundjoin%
\definecolor{currentfill}{rgb}{0.000000,0.000000,0.000000}%
\pgfsetfillcolor{currentfill}%
\pgfsetlinewidth{0.803000pt}%
\definecolor{currentstroke}{rgb}{0.000000,0.000000,0.000000}%
\pgfsetstrokecolor{currentstroke}%
\pgfsetdash{}{0pt}%
\pgfsys@defobject{currentmarker}{\pgfqpoint{0.000000in}{-0.048611in}}{\pgfqpoint{0.000000in}{0.000000in}}{%
\pgfpathmoveto{\pgfqpoint{0.000000in}{0.000000in}}%
\pgfpathlineto{\pgfqpoint{0.000000in}{-0.048611in}}%
\pgfusepath{stroke,fill}%
}%
\begin{pgfscope}%
\pgfsys@transformshift{6.806383in}{0.438889in}%
\pgfsys@useobject{currentmarker}{}%
\end{pgfscope}%
\end{pgfscope}%
\begin{pgfscope}%
\definecolor{textcolor}{rgb}{0.000000,0.000000,0.000000}%
\pgfsetstrokecolor{textcolor}%
\pgfsetfillcolor{textcolor}%
\pgftext[x=6.806383in,y=0.341667in,,top]{\color{textcolor}\sffamily\fontsize{10.000000}{12.000000}\selectfont \(\displaystyle {10^{7}}\)}%
\end{pgfscope}%
\begin{pgfscope}%
\pgfsetbuttcap%
\pgfsetroundjoin%
\definecolor{currentfill}{rgb}{0.000000,0.000000,0.000000}%
\pgfsetfillcolor{currentfill}%
\pgfsetlinewidth{0.803000pt}%
\definecolor{currentstroke}{rgb}{0.000000,0.000000,0.000000}%
\pgfsetstrokecolor{currentstroke}%
\pgfsetdash{}{0pt}%
\pgfsys@defobject{currentmarker}{\pgfqpoint{0.000000in}{-0.048611in}}{\pgfqpoint{0.000000in}{0.000000in}}{%
\pgfpathmoveto{\pgfqpoint{0.000000in}{0.000000in}}%
\pgfpathlineto{\pgfqpoint{0.000000in}{-0.048611in}}%
\pgfusepath{stroke,fill}%
}%
\begin{pgfscope}%
\pgfsys@transformshift{7.409527in}{0.438889in}%
\pgfsys@useobject{currentmarker}{}%
\end{pgfscope}%
\end{pgfscope}%
\begin{pgfscope}%
\definecolor{textcolor}{rgb}{0.000000,0.000000,0.000000}%
\pgfsetstrokecolor{textcolor}%
\pgfsetfillcolor{textcolor}%
\pgftext[x=7.409527in,y=0.341667in,,top]{\color{textcolor}\sffamily\fontsize{10.000000}{12.000000}\selectfont \(\displaystyle {10^{8}}\)}%
\end{pgfscope}%
\begin{pgfscope}%
\pgfsetbuttcap%
\pgfsetroundjoin%
\definecolor{currentfill}{rgb}{0.000000,0.000000,0.000000}%
\pgfsetfillcolor{currentfill}%
\pgfsetlinewidth{0.602250pt}%
\definecolor{currentstroke}{rgb}{0.000000,0.000000,0.000000}%
\pgfsetstrokecolor{currentstroke}%
\pgfsetdash{}{0pt}%
\pgfsys@defobject{currentmarker}{\pgfqpoint{0.000000in}{-0.027778in}}{\pgfqpoint{0.000000in}{0.000000in}}{%
\pgfpathmoveto{\pgfqpoint{0.000000in}{0.000000in}}%
\pgfpathlineto{\pgfqpoint{0.000000in}{-0.027778in}}%
\pgfusepath{stroke,fill}%
}%
\begin{pgfscope}%
\pgfsys@transformshift{4.575372in}{0.438889in}%
\pgfsys@useobject{currentmarker}{}%
\end{pgfscope}%
\end{pgfscope}%
\begin{pgfscope}%
\pgfsetbuttcap%
\pgfsetroundjoin%
\definecolor{currentfill}{rgb}{0.000000,0.000000,0.000000}%
\pgfsetfillcolor{currentfill}%
\pgfsetlinewidth{0.602250pt}%
\definecolor{currentstroke}{rgb}{0.000000,0.000000,0.000000}%
\pgfsetstrokecolor{currentstroke}%
\pgfsetdash{}{0pt}%
\pgfsys@defobject{currentmarker}{\pgfqpoint{0.000000in}{-0.027778in}}{\pgfqpoint{0.000000in}{0.000000in}}{%
\pgfpathmoveto{\pgfqpoint{0.000000in}{0.000000in}}%
\pgfpathlineto{\pgfqpoint{0.000000in}{-0.027778in}}%
\pgfusepath{stroke,fill}%
}%
\begin{pgfscope}%
\pgfsys@transformshift{4.681580in}{0.438889in}%
\pgfsys@useobject{currentmarker}{}%
\end{pgfscope}%
\end{pgfscope}%
\begin{pgfscope}%
\pgfsetbuttcap%
\pgfsetroundjoin%
\definecolor{currentfill}{rgb}{0.000000,0.000000,0.000000}%
\pgfsetfillcolor{currentfill}%
\pgfsetlinewidth{0.602250pt}%
\definecolor{currentstroke}{rgb}{0.000000,0.000000,0.000000}%
\pgfsetstrokecolor{currentstroke}%
\pgfsetdash{}{0pt}%
\pgfsys@defobject{currentmarker}{\pgfqpoint{0.000000in}{-0.027778in}}{\pgfqpoint{0.000000in}{0.000000in}}{%
\pgfpathmoveto{\pgfqpoint{0.000000in}{0.000000in}}%
\pgfpathlineto{\pgfqpoint{0.000000in}{-0.027778in}}%
\pgfusepath{stroke,fill}%
}%
\begin{pgfscope}%
\pgfsys@transformshift{4.756936in}{0.438889in}%
\pgfsys@useobject{currentmarker}{}%
\end{pgfscope}%
\end{pgfscope}%
\begin{pgfscope}%
\pgfsetbuttcap%
\pgfsetroundjoin%
\definecolor{currentfill}{rgb}{0.000000,0.000000,0.000000}%
\pgfsetfillcolor{currentfill}%
\pgfsetlinewidth{0.602250pt}%
\definecolor{currentstroke}{rgb}{0.000000,0.000000,0.000000}%
\pgfsetstrokecolor{currentstroke}%
\pgfsetdash{}{0pt}%
\pgfsys@defobject{currentmarker}{\pgfqpoint{0.000000in}{-0.027778in}}{\pgfqpoint{0.000000in}{0.000000in}}{%
\pgfpathmoveto{\pgfqpoint{0.000000in}{0.000000in}}%
\pgfpathlineto{\pgfqpoint{0.000000in}{-0.027778in}}%
\pgfusepath{stroke,fill}%
}%
\begin{pgfscope}%
\pgfsys@transformshift{4.815387in}{0.438889in}%
\pgfsys@useobject{currentmarker}{}%
\end{pgfscope}%
\end{pgfscope}%
\begin{pgfscope}%
\pgfsetbuttcap%
\pgfsetroundjoin%
\definecolor{currentfill}{rgb}{0.000000,0.000000,0.000000}%
\pgfsetfillcolor{currentfill}%
\pgfsetlinewidth{0.602250pt}%
\definecolor{currentstroke}{rgb}{0.000000,0.000000,0.000000}%
\pgfsetstrokecolor{currentstroke}%
\pgfsetdash{}{0pt}%
\pgfsys@defobject{currentmarker}{\pgfqpoint{0.000000in}{-0.027778in}}{\pgfqpoint{0.000000in}{0.000000in}}{%
\pgfpathmoveto{\pgfqpoint{0.000000in}{0.000000in}}%
\pgfpathlineto{\pgfqpoint{0.000000in}{-0.027778in}}%
\pgfusepath{stroke,fill}%
}%
\begin{pgfscope}%
\pgfsys@transformshift{4.863145in}{0.438889in}%
\pgfsys@useobject{currentmarker}{}%
\end{pgfscope}%
\end{pgfscope}%
\begin{pgfscope}%
\pgfsetbuttcap%
\pgfsetroundjoin%
\definecolor{currentfill}{rgb}{0.000000,0.000000,0.000000}%
\pgfsetfillcolor{currentfill}%
\pgfsetlinewidth{0.602250pt}%
\definecolor{currentstroke}{rgb}{0.000000,0.000000,0.000000}%
\pgfsetstrokecolor{currentstroke}%
\pgfsetdash{}{0pt}%
\pgfsys@defobject{currentmarker}{\pgfqpoint{0.000000in}{-0.027778in}}{\pgfqpoint{0.000000in}{0.000000in}}{%
\pgfpathmoveto{\pgfqpoint{0.000000in}{0.000000in}}%
\pgfpathlineto{\pgfqpoint{0.000000in}{-0.027778in}}%
\pgfusepath{stroke,fill}%
}%
\begin{pgfscope}%
\pgfsys@transformshift{4.903523in}{0.438889in}%
\pgfsys@useobject{currentmarker}{}%
\end{pgfscope}%
\end{pgfscope}%
\begin{pgfscope}%
\pgfsetbuttcap%
\pgfsetroundjoin%
\definecolor{currentfill}{rgb}{0.000000,0.000000,0.000000}%
\pgfsetfillcolor{currentfill}%
\pgfsetlinewidth{0.602250pt}%
\definecolor{currentstroke}{rgb}{0.000000,0.000000,0.000000}%
\pgfsetstrokecolor{currentstroke}%
\pgfsetdash{}{0pt}%
\pgfsys@defobject{currentmarker}{\pgfqpoint{0.000000in}{-0.027778in}}{\pgfqpoint{0.000000in}{0.000000in}}{%
\pgfpathmoveto{\pgfqpoint{0.000000in}{0.000000in}}%
\pgfpathlineto{\pgfqpoint{0.000000in}{-0.027778in}}%
\pgfusepath{stroke,fill}%
}%
\begin{pgfscope}%
\pgfsys@transformshift{4.938501in}{0.438889in}%
\pgfsys@useobject{currentmarker}{}%
\end{pgfscope}%
\end{pgfscope}%
\begin{pgfscope}%
\pgfsetbuttcap%
\pgfsetroundjoin%
\definecolor{currentfill}{rgb}{0.000000,0.000000,0.000000}%
\pgfsetfillcolor{currentfill}%
\pgfsetlinewidth{0.602250pt}%
\definecolor{currentstroke}{rgb}{0.000000,0.000000,0.000000}%
\pgfsetstrokecolor{currentstroke}%
\pgfsetdash{}{0pt}%
\pgfsys@defobject{currentmarker}{\pgfqpoint{0.000000in}{-0.027778in}}{\pgfqpoint{0.000000in}{0.000000in}}{%
\pgfpathmoveto{\pgfqpoint{0.000000in}{0.000000in}}%
\pgfpathlineto{\pgfqpoint{0.000000in}{-0.027778in}}%
\pgfusepath{stroke,fill}%
}%
\begin{pgfscope}%
\pgfsys@transformshift{4.969353in}{0.438889in}%
\pgfsys@useobject{currentmarker}{}%
\end{pgfscope}%
\end{pgfscope}%
\begin{pgfscope}%
\pgfsetbuttcap%
\pgfsetroundjoin%
\definecolor{currentfill}{rgb}{0.000000,0.000000,0.000000}%
\pgfsetfillcolor{currentfill}%
\pgfsetlinewidth{0.602250pt}%
\definecolor{currentstroke}{rgb}{0.000000,0.000000,0.000000}%
\pgfsetstrokecolor{currentstroke}%
\pgfsetdash{}{0pt}%
\pgfsys@defobject{currentmarker}{\pgfqpoint{0.000000in}{-0.027778in}}{\pgfqpoint{0.000000in}{0.000000in}}{%
\pgfpathmoveto{\pgfqpoint{0.000000in}{0.000000in}}%
\pgfpathlineto{\pgfqpoint{0.000000in}{-0.027778in}}%
\pgfusepath{stroke,fill}%
}%
\begin{pgfscope}%
\pgfsys@transformshift{5.178516in}{0.438889in}%
\pgfsys@useobject{currentmarker}{}%
\end{pgfscope}%
\end{pgfscope}%
\begin{pgfscope}%
\pgfsetbuttcap%
\pgfsetroundjoin%
\definecolor{currentfill}{rgb}{0.000000,0.000000,0.000000}%
\pgfsetfillcolor{currentfill}%
\pgfsetlinewidth{0.602250pt}%
\definecolor{currentstroke}{rgb}{0.000000,0.000000,0.000000}%
\pgfsetstrokecolor{currentstroke}%
\pgfsetdash{}{0pt}%
\pgfsys@defobject{currentmarker}{\pgfqpoint{0.000000in}{-0.027778in}}{\pgfqpoint{0.000000in}{0.000000in}}{%
\pgfpathmoveto{\pgfqpoint{0.000000in}{0.000000in}}%
\pgfpathlineto{\pgfqpoint{0.000000in}{-0.027778in}}%
\pgfusepath{stroke,fill}%
}%
\begin{pgfscope}%
\pgfsys@transformshift{5.284724in}{0.438889in}%
\pgfsys@useobject{currentmarker}{}%
\end{pgfscope}%
\end{pgfscope}%
\begin{pgfscope}%
\pgfsetbuttcap%
\pgfsetroundjoin%
\definecolor{currentfill}{rgb}{0.000000,0.000000,0.000000}%
\pgfsetfillcolor{currentfill}%
\pgfsetlinewidth{0.602250pt}%
\definecolor{currentstroke}{rgb}{0.000000,0.000000,0.000000}%
\pgfsetstrokecolor{currentstroke}%
\pgfsetdash{}{0pt}%
\pgfsys@defobject{currentmarker}{\pgfqpoint{0.000000in}{-0.027778in}}{\pgfqpoint{0.000000in}{0.000000in}}{%
\pgfpathmoveto{\pgfqpoint{0.000000in}{0.000000in}}%
\pgfpathlineto{\pgfqpoint{0.000000in}{-0.027778in}}%
\pgfusepath{stroke,fill}%
}%
\begin{pgfscope}%
\pgfsys@transformshift{5.360080in}{0.438889in}%
\pgfsys@useobject{currentmarker}{}%
\end{pgfscope}%
\end{pgfscope}%
\begin{pgfscope}%
\pgfsetbuttcap%
\pgfsetroundjoin%
\definecolor{currentfill}{rgb}{0.000000,0.000000,0.000000}%
\pgfsetfillcolor{currentfill}%
\pgfsetlinewidth{0.602250pt}%
\definecolor{currentstroke}{rgb}{0.000000,0.000000,0.000000}%
\pgfsetstrokecolor{currentstroke}%
\pgfsetdash{}{0pt}%
\pgfsys@defobject{currentmarker}{\pgfqpoint{0.000000in}{-0.027778in}}{\pgfqpoint{0.000000in}{0.000000in}}{%
\pgfpathmoveto{\pgfqpoint{0.000000in}{0.000000in}}%
\pgfpathlineto{\pgfqpoint{0.000000in}{-0.027778in}}%
\pgfusepath{stroke,fill}%
}%
\begin{pgfscope}%
\pgfsys@transformshift{5.418531in}{0.438889in}%
\pgfsys@useobject{currentmarker}{}%
\end{pgfscope}%
\end{pgfscope}%
\begin{pgfscope}%
\pgfsetbuttcap%
\pgfsetroundjoin%
\definecolor{currentfill}{rgb}{0.000000,0.000000,0.000000}%
\pgfsetfillcolor{currentfill}%
\pgfsetlinewidth{0.602250pt}%
\definecolor{currentstroke}{rgb}{0.000000,0.000000,0.000000}%
\pgfsetstrokecolor{currentstroke}%
\pgfsetdash{}{0pt}%
\pgfsys@defobject{currentmarker}{\pgfqpoint{0.000000in}{-0.027778in}}{\pgfqpoint{0.000000in}{0.000000in}}{%
\pgfpathmoveto{\pgfqpoint{0.000000in}{0.000000in}}%
\pgfpathlineto{\pgfqpoint{0.000000in}{-0.027778in}}%
\pgfusepath{stroke,fill}%
}%
\begin{pgfscope}%
\pgfsys@transformshift{5.466289in}{0.438889in}%
\pgfsys@useobject{currentmarker}{}%
\end{pgfscope}%
\end{pgfscope}%
\begin{pgfscope}%
\pgfsetbuttcap%
\pgfsetroundjoin%
\definecolor{currentfill}{rgb}{0.000000,0.000000,0.000000}%
\pgfsetfillcolor{currentfill}%
\pgfsetlinewidth{0.602250pt}%
\definecolor{currentstroke}{rgb}{0.000000,0.000000,0.000000}%
\pgfsetstrokecolor{currentstroke}%
\pgfsetdash{}{0pt}%
\pgfsys@defobject{currentmarker}{\pgfqpoint{0.000000in}{-0.027778in}}{\pgfqpoint{0.000000in}{0.000000in}}{%
\pgfpathmoveto{\pgfqpoint{0.000000in}{0.000000in}}%
\pgfpathlineto{\pgfqpoint{0.000000in}{-0.027778in}}%
\pgfusepath{stroke,fill}%
}%
\begin{pgfscope}%
\pgfsys@transformshift{5.506667in}{0.438889in}%
\pgfsys@useobject{currentmarker}{}%
\end{pgfscope}%
\end{pgfscope}%
\begin{pgfscope}%
\pgfsetbuttcap%
\pgfsetroundjoin%
\definecolor{currentfill}{rgb}{0.000000,0.000000,0.000000}%
\pgfsetfillcolor{currentfill}%
\pgfsetlinewidth{0.602250pt}%
\definecolor{currentstroke}{rgb}{0.000000,0.000000,0.000000}%
\pgfsetstrokecolor{currentstroke}%
\pgfsetdash{}{0pt}%
\pgfsys@defobject{currentmarker}{\pgfqpoint{0.000000in}{-0.027778in}}{\pgfqpoint{0.000000in}{0.000000in}}{%
\pgfpathmoveto{\pgfqpoint{0.000000in}{0.000000in}}%
\pgfpathlineto{\pgfqpoint{0.000000in}{-0.027778in}}%
\pgfusepath{stroke,fill}%
}%
\begin{pgfscope}%
\pgfsys@transformshift{5.541645in}{0.438889in}%
\pgfsys@useobject{currentmarker}{}%
\end{pgfscope}%
\end{pgfscope}%
\begin{pgfscope}%
\pgfsetbuttcap%
\pgfsetroundjoin%
\definecolor{currentfill}{rgb}{0.000000,0.000000,0.000000}%
\pgfsetfillcolor{currentfill}%
\pgfsetlinewidth{0.602250pt}%
\definecolor{currentstroke}{rgb}{0.000000,0.000000,0.000000}%
\pgfsetstrokecolor{currentstroke}%
\pgfsetdash{}{0pt}%
\pgfsys@defobject{currentmarker}{\pgfqpoint{0.000000in}{-0.027778in}}{\pgfqpoint{0.000000in}{0.000000in}}{%
\pgfpathmoveto{\pgfqpoint{0.000000in}{0.000000in}}%
\pgfpathlineto{\pgfqpoint{0.000000in}{-0.027778in}}%
\pgfusepath{stroke,fill}%
}%
\begin{pgfscope}%
\pgfsys@transformshift{5.572497in}{0.438889in}%
\pgfsys@useobject{currentmarker}{}%
\end{pgfscope}%
\end{pgfscope}%
\begin{pgfscope}%
\pgfsetbuttcap%
\pgfsetroundjoin%
\definecolor{currentfill}{rgb}{0.000000,0.000000,0.000000}%
\pgfsetfillcolor{currentfill}%
\pgfsetlinewidth{0.602250pt}%
\definecolor{currentstroke}{rgb}{0.000000,0.000000,0.000000}%
\pgfsetstrokecolor{currentstroke}%
\pgfsetdash{}{0pt}%
\pgfsys@defobject{currentmarker}{\pgfqpoint{0.000000in}{-0.027778in}}{\pgfqpoint{0.000000in}{0.000000in}}{%
\pgfpathmoveto{\pgfqpoint{0.000000in}{0.000000in}}%
\pgfpathlineto{\pgfqpoint{0.000000in}{-0.027778in}}%
\pgfusepath{stroke,fill}%
}%
\begin{pgfscope}%
\pgfsys@transformshift{5.781660in}{0.438889in}%
\pgfsys@useobject{currentmarker}{}%
\end{pgfscope}%
\end{pgfscope}%
\begin{pgfscope}%
\pgfsetbuttcap%
\pgfsetroundjoin%
\definecolor{currentfill}{rgb}{0.000000,0.000000,0.000000}%
\pgfsetfillcolor{currentfill}%
\pgfsetlinewidth{0.602250pt}%
\definecolor{currentstroke}{rgb}{0.000000,0.000000,0.000000}%
\pgfsetstrokecolor{currentstroke}%
\pgfsetdash{}{0pt}%
\pgfsys@defobject{currentmarker}{\pgfqpoint{0.000000in}{-0.027778in}}{\pgfqpoint{0.000000in}{0.000000in}}{%
\pgfpathmoveto{\pgfqpoint{0.000000in}{0.000000in}}%
\pgfpathlineto{\pgfqpoint{0.000000in}{-0.027778in}}%
\pgfusepath{stroke,fill}%
}%
\begin{pgfscope}%
\pgfsys@transformshift{5.887868in}{0.438889in}%
\pgfsys@useobject{currentmarker}{}%
\end{pgfscope}%
\end{pgfscope}%
\begin{pgfscope}%
\pgfsetbuttcap%
\pgfsetroundjoin%
\definecolor{currentfill}{rgb}{0.000000,0.000000,0.000000}%
\pgfsetfillcolor{currentfill}%
\pgfsetlinewidth{0.602250pt}%
\definecolor{currentstroke}{rgb}{0.000000,0.000000,0.000000}%
\pgfsetstrokecolor{currentstroke}%
\pgfsetdash{}{0pt}%
\pgfsys@defobject{currentmarker}{\pgfqpoint{0.000000in}{-0.027778in}}{\pgfqpoint{0.000000in}{0.000000in}}{%
\pgfpathmoveto{\pgfqpoint{0.000000in}{0.000000in}}%
\pgfpathlineto{\pgfqpoint{0.000000in}{-0.027778in}}%
\pgfusepath{stroke,fill}%
}%
\begin{pgfscope}%
\pgfsys@transformshift{5.963224in}{0.438889in}%
\pgfsys@useobject{currentmarker}{}%
\end{pgfscope}%
\end{pgfscope}%
\begin{pgfscope}%
\pgfsetbuttcap%
\pgfsetroundjoin%
\definecolor{currentfill}{rgb}{0.000000,0.000000,0.000000}%
\pgfsetfillcolor{currentfill}%
\pgfsetlinewidth{0.602250pt}%
\definecolor{currentstroke}{rgb}{0.000000,0.000000,0.000000}%
\pgfsetstrokecolor{currentstroke}%
\pgfsetdash{}{0pt}%
\pgfsys@defobject{currentmarker}{\pgfqpoint{0.000000in}{-0.027778in}}{\pgfqpoint{0.000000in}{0.000000in}}{%
\pgfpathmoveto{\pgfqpoint{0.000000in}{0.000000in}}%
\pgfpathlineto{\pgfqpoint{0.000000in}{-0.027778in}}%
\pgfusepath{stroke,fill}%
}%
\begin{pgfscope}%
\pgfsys@transformshift{6.021675in}{0.438889in}%
\pgfsys@useobject{currentmarker}{}%
\end{pgfscope}%
\end{pgfscope}%
\begin{pgfscope}%
\pgfsetbuttcap%
\pgfsetroundjoin%
\definecolor{currentfill}{rgb}{0.000000,0.000000,0.000000}%
\pgfsetfillcolor{currentfill}%
\pgfsetlinewidth{0.602250pt}%
\definecolor{currentstroke}{rgb}{0.000000,0.000000,0.000000}%
\pgfsetstrokecolor{currentstroke}%
\pgfsetdash{}{0pt}%
\pgfsys@defobject{currentmarker}{\pgfqpoint{0.000000in}{-0.027778in}}{\pgfqpoint{0.000000in}{0.000000in}}{%
\pgfpathmoveto{\pgfqpoint{0.000000in}{0.000000in}}%
\pgfpathlineto{\pgfqpoint{0.000000in}{-0.027778in}}%
\pgfusepath{stroke,fill}%
}%
\begin{pgfscope}%
\pgfsys@transformshift{6.069433in}{0.438889in}%
\pgfsys@useobject{currentmarker}{}%
\end{pgfscope}%
\end{pgfscope}%
\begin{pgfscope}%
\pgfsetbuttcap%
\pgfsetroundjoin%
\definecolor{currentfill}{rgb}{0.000000,0.000000,0.000000}%
\pgfsetfillcolor{currentfill}%
\pgfsetlinewidth{0.602250pt}%
\definecolor{currentstroke}{rgb}{0.000000,0.000000,0.000000}%
\pgfsetstrokecolor{currentstroke}%
\pgfsetdash{}{0pt}%
\pgfsys@defobject{currentmarker}{\pgfqpoint{0.000000in}{-0.027778in}}{\pgfqpoint{0.000000in}{0.000000in}}{%
\pgfpathmoveto{\pgfqpoint{0.000000in}{0.000000in}}%
\pgfpathlineto{\pgfqpoint{0.000000in}{-0.027778in}}%
\pgfusepath{stroke,fill}%
}%
\begin{pgfscope}%
\pgfsys@transformshift{6.109811in}{0.438889in}%
\pgfsys@useobject{currentmarker}{}%
\end{pgfscope}%
\end{pgfscope}%
\begin{pgfscope}%
\pgfsetbuttcap%
\pgfsetroundjoin%
\definecolor{currentfill}{rgb}{0.000000,0.000000,0.000000}%
\pgfsetfillcolor{currentfill}%
\pgfsetlinewidth{0.602250pt}%
\definecolor{currentstroke}{rgb}{0.000000,0.000000,0.000000}%
\pgfsetstrokecolor{currentstroke}%
\pgfsetdash{}{0pt}%
\pgfsys@defobject{currentmarker}{\pgfqpoint{0.000000in}{-0.027778in}}{\pgfqpoint{0.000000in}{0.000000in}}{%
\pgfpathmoveto{\pgfqpoint{0.000000in}{0.000000in}}%
\pgfpathlineto{\pgfqpoint{0.000000in}{-0.027778in}}%
\pgfusepath{stroke,fill}%
}%
\begin{pgfscope}%
\pgfsys@transformshift{6.144789in}{0.438889in}%
\pgfsys@useobject{currentmarker}{}%
\end{pgfscope}%
\end{pgfscope}%
\begin{pgfscope}%
\pgfsetbuttcap%
\pgfsetroundjoin%
\definecolor{currentfill}{rgb}{0.000000,0.000000,0.000000}%
\pgfsetfillcolor{currentfill}%
\pgfsetlinewidth{0.602250pt}%
\definecolor{currentstroke}{rgb}{0.000000,0.000000,0.000000}%
\pgfsetstrokecolor{currentstroke}%
\pgfsetdash{}{0pt}%
\pgfsys@defobject{currentmarker}{\pgfqpoint{0.000000in}{-0.027778in}}{\pgfqpoint{0.000000in}{0.000000in}}{%
\pgfpathmoveto{\pgfqpoint{0.000000in}{0.000000in}}%
\pgfpathlineto{\pgfqpoint{0.000000in}{-0.027778in}}%
\pgfusepath{stroke,fill}%
}%
\begin{pgfscope}%
\pgfsys@transformshift{6.175641in}{0.438889in}%
\pgfsys@useobject{currentmarker}{}%
\end{pgfscope}%
\end{pgfscope}%
\begin{pgfscope}%
\pgfsetbuttcap%
\pgfsetroundjoin%
\definecolor{currentfill}{rgb}{0.000000,0.000000,0.000000}%
\pgfsetfillcolor{currentfill}%
\pgfsetlinewidth{0.602250pt}%
\definecolor{currentstroke}{rgb}{0.000000,0.000000,0.000000}%
\pgfsetstrokecolor{currentstroke}%
\pgfsetdash{}{0pt}%
\pgfsys@defobject{currentmarker}{\pgfqpoint{0.000000in}{-0.027778in}}{\pgfqpoint{0.000000in}{0.000000in}}{%
\pgfpathmoveto{\pgfqpoint{0.000000in}{0.000000in}}%
\pgfpathlineto{\pgfqpoint{0.000000in}{-0.027778in}}%
\pgfusepath{stroke,fill}%
}%
\begin{pgfscope}%
\pgfsys@transformshift{6.384804in}{0.438889in}%
\pgfsys@useobject{currentmarker}{}%
\end{pgfscope}%
\end{pgfscope}%
\begin{pgfscope}%
\pgfsetbuttcap%
\pgfsetroundjoin%
\definecolor{currentfill}{rgb}{0.000000,0.000000,0.000000}%
\pgfsetfillcolor{currentfill}%
\pgfsetlinewidth{0.602250pt}%
\definecolor{currentstroke}{rgb}{0.000000,0.000000,0.000000}%
\pgfsetstrokecolor{currentstroke}%
\pgfsetdash{}{0pt}%
\pgfsys@defobject{currentmarker}{\pgfqpoint{0.000000in}{-0.027778in}}{\pgfqpoint{0.000000in}{0.000000in}}{%
\pgfpathmoveto{\pgfqpoint{0.000000in}{0.000000in}}%
\pgfpathlineto{\pgfqpoint{0.000000in}{-0.027778in}}%
\pgfusepath{stroke,fill}%
}%
\begin{pgfscope}%
\pgfsys@transformshift{6.491012in}{0.438889in}%
\pgfsys@useobject{currentmarker}{}%
\end{pgfscope}%
\end{pgfscope}%
\begin{pgfscope}%
\pgfsetbuttcap%
\pgfsetroundjoin%
\definecolor{currentfill}{rgb}{0.000000,0.000000,0.000000}%
\pgfsetfillcolor{currentfill}%
\pgfsetlinewidth{0.602250pt}%
\definecolor{currentstroke}{rgb}{0.000000,0.000000,0.000000}%
\pgfsetstrokecolor{currentstroke}%
\pgfsetdash{}{0pt}%
\pgfsys@defobject{currentmarker}{\pgfqpoint{0.000000in}{-0.027778in}}{\pgfqpoint{0.000000in}{0.000000in}}{%
\pgfpathmoveto{\pgfqpoint{0.000000in}{0.000000in}}%
\pgfpathlineto{\pgfqpoint{0.000000in}{-0.027778in}}%
\pgfusepath{stroke,fill}%
}%
\begin{pgfscope}%
\pgfsys@transformshift{6.566368in}{0.438889in}%
\pgfsys@useobject{currentmarker}{}%
\end{pgfscope}%
\end{pgfscope}%
\begin{pgfscope}%
\pgfsetbuttcap%
\pgfsetroundjoin%
\definecolor{currentfill}{rgb}{0.000000,0.000000,0.000000}%
\pgfsetfillcolor{currentfill}%
\pgfsetlinewidth{0.602250pt}%
\definecolor{currentstroke}{rgb}{0.000000,0.000000,0.000000}%
\pgfsetstrokecolor{currentstroke}%
\pgfsetdash{}{0pt}%
\pgfsys@defobject{currentmarker}{\pgfqpoint{0.000000in}{-0.027778in}}{\pgfqpoint{0.000000in}{0.000000in}}{%
\pgfpathmoveto{\pgfqpoint{0.000000in}{0.000000in}}%
\pgfpathlineto{\pgfqpoint{0.000000in}{-0.027778in}}%
\pgfusepath{stroke,fill}%
}%
\begin{pgfscope}%
\pgfsys@transformshift{6.624819in}{0.438889in}%
\pgfsys@useobject{currentmarker}{}%
\end{pgfscope}%
\end{pgfscope}%
\begin{pgfscope}%
\pgfsetbuttcap%
\pgfsetroundjoin%
\definecolor{currentfill}{rgb}{0.000000,0.000000,0.000000}%
\pgfsetfillcolor{currentfill}%
\pgfsetlinewidth{0.602250pt}%
\definecolor{currentstroke}{rgb}{0.000000,0.000000,0.000000}%
\pgfsetstrokecolor{currentstroke}%
\pgfsetdash{}{0pt}%
\pgfsys@defobject{currentmarker}{\pgfqpoint{0.000000in}{-0.027778in}}{\pgfqpoint{0.000000in}{0.000000in}}{%
\pgfpathmoveto{\pgfqpoint{0.000000in}{0.000000in}}%
\pgfpathlineto{\pgfqpoint{0.000000in}{-0.027778in}}%
\pgfusepath{stroke,fill}%
}%
\begin{pgfscope}%
\pgfsys@transformshift{6.672577in}{0.438889in}%
\pgfsys@useobject{currentmarker}{}%
\end{pgfscope}%
\end{pgfscope}%
\begin{pgfscope}%
\pgfsetbuttcap%
\pgfsetroundjoin%
\definecolor{currentfill}{rgb}{0.000000,0.000000,0.000000}%
\pgfsetfillcolor{currentfill}%
\pgfsetlinewidth{0.602250pt}%
\definecolor{currentstroke}{rgb}{0.000000,0.000000,0.000000}%
\pgfsetstrokecolor{currentstroke}%
\pgfsetdash{}{0pt}%
\pgfsys@defobject{currentmarker}{\pgfqpoint{0.000000in}{-0.027778in}}{\pgfqpoint{0.000000in}{0.000000in}}{%
\pgfpathmoveto{\pgfqpoint{0.000000in}{0.000000in}}%
\pgfpathlineto{\pgfqpoint{0.000000in}{-0.027778in}}%
\pgfusepath{stroke,fill}%
}%
\begin{pgfscope}%
\pgfsys@transformshift{6.712955in}{0.438889in}%
\pgfsys@useobject{currentmarker}{}%
\end{pgfscope}%
\end{pgfscope}%
\begin{pgfscope}%
\pgfsetbuttcap%
\pgfsetroundjoin%
\definecolor{currentfill}{rgb}{0.000000,0.000000,0.000000}%
\pgfsetfillcolor{currentfill}%
\pgfsetlinewidth{0.602250pt}%
\definecolor{currentstroke}{rgb}{0.000000,0.000000,0.000000}%
\pgfsetstrokecolor{currentstroke}%
\pgfsetdash{}{0pt}%
\pgfsys@defobject{currentmarker}{\pgfqpoint{0.000000in}{-0.027778in}}{\pgfqpoint{0.000000in}{0.000000in}}{%
\pgfpathmoveto{\pgfqpoint{0.000000in}{0.000000in}}%
\pgfpathlineto{\pgfqpoint{0.000000in}{-0.027778in}}%
\pgfusepath{stroke,fill}%
}%
\begin{pgfscope}%
\pgfsys@transformshift{6.747933in}{0.438889in}%
\pgfsys@useobject{currentmarker}{}%
\end{pgfscope}%
\end{pgfscope}%
\begin{pgfscope}%
\pgfsetbuttcap%
\pgfsetroundjoin%
\definecolor{currentfill}{rgb}{0.000000,0.000000,0.000000}%
\pgfsetfillcolor{currentfill}%
\pgfsetlinewidth{0.602250pt}%
\definecolor{currentstroke}{rgb}{0.000000,0.000000,0.000000}%
\pgfsetstrokecolor{currentstroke}%
\pgfsetdash{}{0pt}%
\pgfsys@defobject{currentmarker}{\pgfqpoint{0.000000in}{-0.027778in}}{\pgfqpoint{0.000000in}{0.000000in}}{%
\pgfpathmoveto{\pgfqpoint{0.000000in}{0.000000in}}%
\pgfpathlineto{\pgfqpoint{0.000000in}{-0.027778in}}%
\pgfusepath{stroke,fill}%
}%
\begin{pgfscope}%
\pgfsys@transformshift{6.778785in}{0.438889in}%
\pgfsys@useobject{currentmarker}{}%
\end{pgfscope}%
\end{pgfscope}%
\begin{pgfscope}%
\pgfsetbuttcap%
\pgfsetroundjoin%
\definecolor{currentfill}{rgb}{0.000000,0.000000,0.000000}%
\pgfsetfillcolor{currentfill}%
\pgfsetlinewidth{0.602250pt}%
\definecolor{currentstroke}{rgb}{0.000000,0.000000,0.000000}%
\pgfsetstrokecolor{currentstroke}%
\pgfsetdash{}{0pt}%
\pgfsys@defobject{currentmarker}{\pgfqpoint{0.000000in}{-0.027778in}}{\pgfqpoint{0.000000in}{0.000000in}}{%
\pgfpathmoveto{\pgfqpoint{0.000000in}{0.000000in}}%
\pgfpathlineto{\pgfqpoint{0.000000in}{-0.027778in}}%
\pgfusepath{stroke,fill}%
}%
\begin{pgfscope}%
\pgfsys@transformshift{6.987948in}{0.438889in}%
\pgfsys@useobject{currentmarker}{}%
\end{pgfscope}%
\end{pgfscope}%
\begin{pgfscope}%
\pgfsetbuttcap%
\pgfsetroundjoin%
\definecolor{currentfill}{rgb}{0.000000,0.000000,0.000000}%
\pgfsetfillcolor{currentfill}%
\pgfsetlinewidth{0.602250pt}%
\definecolor{currentstroke}{rgb}{0.000000,0.000000,0.000000}%
\pgfsetstrokecolor{currentstroke}%
\pgfsetdash{}{0pt}%
\pgfsys@defobject{currentmarker}{\pgfqpoint{0.000000in}{-0.027778in}}{\pgfqpoint{0.000000in}{0.000000in}}{%
\pgfpathmoveto{\pgfqpoint{0.000000in}{0.000000in}}%
\pgfpathlineto{\pgfqpoint{0.000000in}{-0.027778in}}%
\pgfusepath{stroke,fill}%
}%
\begin{pgfscope}%
\pgfsys@transformshift{7.094156in}{0.438889in}%
\pgfsys@useobject{currentmarker}{}%
\end{pgfscope}%
\end{pgfscope}%
\begin{pgfscope}%
\pgfsetbuttcap%
\pgfsetroundjoin%
\definecolor{currentfill}{rgb}{0.000000,0.000000,0.000000}%
\pgfsetfillcolor{currentfill}%
\pgfsetlinewidth{0.602250pt}%
\definecolor{currentstroke}{rgb}{0.000000,0.000000,0.000000}%
\pgfsetstrokecolor{currentstroke}%
\pgfsetdash{}{0pt}%
\pgfsys@defobject{currentmarker}{\pgfqpoint{0.000000in}{-0.027778in}}{\pgfqpoint{0.000000in}{0.000000in}}{%
\pgfpathmoveto{\pgfqpoint{0.000000in}{0.000000in}}%
\pgfpathlineto{\pgfqpoint{0.000000in}{-0.027778in}}%
\pgfusepath{stroke,fill}%
}%
\begin{pgfscope}%
\pgfsys@transformshift{7.169512in}{0.438889in}%
\pgfsys@useobject{currentmarker}{}%
\end{pgfscope}%
\end{pgfscope}%
\begin{pgfscope}%
\pgfsetbuttcap%
\pgfsetroundjoin%
\definecolor{currentfill}{rgb}{0.000000,0.000000,0.000000}%
\pgfsetfillcolor{currentfill}%
\pgfsetlinewidth{0.602250pt}%
\definecolor{currentstroke}{rgb}{0.000000,0.000000,0.000000}%
\pgfsetstrokecolor{currentstroke}%
\pgfsetdash{}{0pt}%
\pgfsys@defobject{currentmarker}{\pgfqpoint{0.000000in}{-0.027778in}}{\pgfqpoint{0.000000in}{0.000000in}}{%
\pgfpathmoveto{\pgfqpoint{0.000000in}{0.000000in}}%
\pgfpathlineto{\pgfqpoint{0.000000in}{-0.027778in}}%
\pgfusepath{stroke,fill}%
}%
\begin{pgfscope}%
\pgfsys@transformshift{7.227963in}{0.438889in}%
\pgfsys@useobject{currentmarker}{}%
\end{pgfscope}%
\end{pgfscope}%
\begin{pgfscope}%
\pgfsetbuttcap%
\pgfsetroundjoin%
\definecolor{currentfill}{rgb}{0.000000,0.000000,0.000000}%
\pgfsetfillcolor{currentfill}%
\pgfsetlinewidth{0.602250pt}%
\definecolor{currentstroke}{rgb}{0.000000,0.000000,0.000000}%
\pgfsetstrokecolor{currentstroke}%
\pgfsetdash{}{0pt}%
\pgfsys@defobject{currentmarker}{\pgfqpoint{0.000000in}{-0.027778in}}{\pgfqpoint{0.000000in}{0.000000in}}{%
\pgfpathmoveto{\pgfqpoint{0.000000in}{0.000000in}}%
\pgfpathlineto{\pgfqpoint{0.000000in}{-0.027778in}}%
\pgfusepath{stroke,fill}%
}%
\begin{pgfscope}%
\pgfsys@transformshift{7.275721in}{0.438889in}%
\pgfsys@useobject{currentmarker}{}%
\end{pgfscope}%
\end{pgfscope}%
\begin{pgfscope}%
\pgfsetbuttcap%
\pgfsetroundjoin%
\definecolor{currentfill}{rgb}{0.000000,0.000000,0.000000}%
\pgfsetfillcolor{currentfill}%
\pgfsetlinewidth{0.602250pt}%
\definecolor{currentstroke}{rgb}{0.000000,0.000000,0.000000}%
\pgfsetstrokecolor{currentstroke}%
\pgfsetdash{}{0pt}%
\pgfsys@defobject{currentmarker}{\pgfqpoint{0.000000in}{-0.027778in}}{\pgfqpoint{0.000000in}{0.000000in}}{%
\pgfpathmoveto{\pgfqpoint{0.000000in}{0.000000in}}%
\pgfpathlineto{\pgfqpoint{0.000000in}{-0.027778in}}%
\pgfusepath{stroke,fill}%
}%
\begin{pgfscope}%
\pgfsys@transformshift{7.316099in}{0.438889in}%
\pgfsys@useobject{currentmarker}{}%
\end{pgfscope}%
\end{pgfscope}%
\begin{pgfscope}%
\pgfsetbuttcap%
\pgfsetroundjoin%
\definecolor{currentfill}{rgb}{0.000000,0.000000,0.000000}%
\pgfsetfillcolor{currentfill}%
\pgfsetlinewidth{0.602250pt}%
\definecolor{currentstroke}{rgb}{0.000000,0.000000,0.000000}%
\pgfsetstrokecolor{currentstroke}%
\pgfsetdash{}{0pt}%
\pgfsys@defobject{currentmarker}{\pgfqpoint{0.000000in}{-0.027778in}}{\pgfqpoint{0.000000in}{0.000000in}}{%
\pgfpathmoveto{\pgfqpoint{0.000000in}{0.000000in}}%
\pgfpathlineto{\pgfqpoint{0.000000in}{-0.027778in}}%
\pgfusepath{stroke,fill}%
}%
\begin{pgfscope}%
\pgfsys@transformshift{7.351077in}{0.438889in}%
\pgfsys@useobject{currentmarker}{}%
\end{pgfscope}%
\end{pgfscope}%
\begin{pgfscope}%
\pgfsetbuttcap%
\pgfsetroundjoin%
\definecolor{currentfill}{rgb}{0.000000,0.000000,0.000000}%
\pgfsetfillcolor{currentfill}%
\pgfsetlinewidth{0.602250pt}%
\definecolor{currentstroke}{rgb}{0.000000,0.000000,0.000000}%
\pgfsetstrokecolor{currentstroke}%
\pgfsetdash{}{0pt}%
\pgfsys@defobject{currentmarker}{\pgfqpoint{0.000000in}{-0.027778in}}{\pgfqpoint{0.000000in}{0.000000in}}{%
\pgfpathmoveto{\pgfqpoint{0.000000in}{0.000000in}}%
\pgfpathlineto{\pgfqpoint{0.000000in}{-0.027778in}}%
\pgfusepath{stroke,fill}%
}%
\begin{pgfscope}%
\pgfsys@transformshift{7.381929in}{0.438889in}%
\pgfsys@useobject{currentmarker}{}%
\end{pgfscope}%
\end{pgfscope}%
\begin{pgfscope}%
\pgfsetbuttcap%
\pgfsetroundjoin%
\definecolor{currentfill}{rgb}{0.000000,0.000000,0.000000}%
\pgfsetfillcolor{currentfill}%
\pgfsetlinewidth{0.602250pt}%
\definecolor{currentstroke}{rgb}{0.000000,0.000000,0.000000}%
\pgfsetstrokecolor{currentstroke}%
\pgfsetdash{}{0pt}%
\pgfsys@defobject{currentmarker}{\pgfqpoint{0.000000in}{-0.027778in}}{\pgfqpoint{0.000000in}{0.000000in}}{%
\pgfpathmoveto{\pgfqpoint{0.000000in}{0.000000in}}%
\pgfpathlineto{\pgfqpoint{0.000000in}{-0.027778in}}%
\pgfusepath{stroke,fill}%
}%
\begin{pgfscope}%
\pgfsys@transformshift{7.591092in}{0.438889in}%
\pgfsys@useobject{currentmarker}{}%
\end{pgfscope}%
\end{pgfscope}%
\begin{pgfscope}%
\pgfsetbuttcap%
\pgfsetroundjoin%
\definecolor{currentfill}{rgb}{0.000000,0.000000,0.000000}%
\pgfsetfillcolor{currentfill}%
\pgfsetlinewidth{0.602250pt}%
\definecolor{currentstroke}{rgb}{0.000000,0.000000,0.000000}%
\pgfsetstrokecolor{currentstroke}%
\pgfsetdash{}{0pt}%
\pgfsys@defobject{currentmarker}{\pgfqpoint{0.000000in}{-0.027778in}}{\pgfqpoint{0.000000in}{0.000000in}}{%
\pgfpathmoveto{\pgfqpoint{0.000000in}{0.000000in}}%
\pgfpathlineto{\pgfqpoint{0.000000in}{-0.027778in}}%
\pgfusepath{stroke,fill}%
}%
\begin{pgfscope}%
\pgfsys@transformshift{7.697300in}{0.438889in}%
\pgfsys@useobject{currentmarker}{}%
\end{pgfscope}%
\end{pgfscope}%
\begin{pgfscope}%
\pgfsetbuttcap%
\pgfsetroundjoin%
\definecolor{currentfill}{rgb}{0.000000,0.000000,0.000000}%
\pgfsetfillcolor{currentfill}%
\pgfsetlinewidth{0.602250pt}%
\definecolor{currentstroke}{rgb}{0.000000,0.000000,0.000000}%
\pgfsetstrokecolor{currentstroke}%
\pgfsetdash{}{0pt}%
\pgfsys@defobject{currentmarker}{\pgfqpoint{0.000000in}{-0.027778in}}{\pgfqpoint{0.000000in}{0.000000in}}{%
\pgfpathmoveto{\pgfqpoint{0.000000in}{0.000000in}}%
\pgfpathlineto{\pgfqpoint{0.000000in}{-0.027778in}}%
\pgfusepath{stroke,fill}%
}%
\begin{pgfscope}%
\pgfsys@transformshift{7.772656in}{0.438889in}%
\pgfsys@useobject{currentmarker}{}%
\end{pgfscope}%
\end{pgfscope}%
\begin{pgfscope}%
\pgfsetbuttcap%
\pgfsetroundjoin%
\definecolor{currentfill}{rgb}{0.000000,0.000000,0.000000}%
\pgfsetfillcolor{currentfill}%
\pgfsetlinewidth{0.803000pt}%
\definecolor{currentstroke}{rgb}{0.000000,0.000000,0.000000}%
\pgfsetstrokecolor{currentstroke}%
\pgfsetdash{}{0pt}%
\pgfsys@defobject{currentmarker}{\pgfqpoint{-0.048611in}{0.000000in}}{\pgfqpoint{0.000000in}{0.000000in}}{%
\pgfpathmoveto{\pgfqpoint{0.000000in}{0.000000in}}%
\pgfpathlineto{\pgfqpoint{-0.048611in}{0.000000in}}%
\pgfusepath{stroke,fill}%
}%
\begin{pgfscope}%
\pgfsys@transformshift{4.482986in}{0.440982in}%
\pgfsys@useobject{currentmarker}{}%
\end{pgfscope}%
\end{pgfscope}%
\begin{pgfscope}%
\definecolor{textcolor}{rgb}{0.000000,0.000000,0.000000}%
\pgfsetstrokecolor{textcolor}%
\pgfsetfillcolor{textcolor}%
\pgftext[x=4.048511in,y=0.388220in,left,base]{\color{textcolor}\sffamily\fontsize{10.000000}{12.000000}\selectfont −0.2}%
\end{pgfscope}%
\begin{pgfscope}%
\pgfsetbuttcap%
\pgfsetroundjoin%
\definecolor{currentfill}{rgb}{0.000000,0.000000,0.000000}%
\pgfsetfillcolor{currentfill}%
\pgfsetlinewidth{0.803000pt}%
\definecolor{currentstroke}{rgb}{0.000000,0.000000,0.000000}%
\pgfsetstrokecolor{currentstroke}%
\pgfsetdash{}{0pt}%
\pgfsys@defobject{currentmarker}{\pgfqpoint{-0.048611in}{0.000000in}}{\pgfqpoint{0.000000in}{0.000000in}}{%
\pgfpathmoveto{\pgfqpoint{0.000000in}{0.000000in}}%
\pgfpathlineto{\pgfqpoint{-0.048611in}{0.000000in}}%
\pgfusepath{stroke,fill}%
}%
\begin{pgfscope}%
\pgfsys@transformshift{4.482986in}{0.747697in}%
\pgfsys@useobject{currentmarker}{}%
\end{pgfscope}%
\end{pgfscope}%
\begin{pgfscope}%
\definecolor{textcolor}{rgb}{0.000000,0.000000,0.000000}%
\pgfsetstrokecolor{textcolor}%
\pgfsetfillcolor{textcolor}%
\pgftext[x=4.164884in,y=0.694935in,left,base]{\color{textcolor}\sffamily\fontsize{10.000000}{12.000000}\selectfont 0.0}%
\end{pgfscope}%
\begin{pgfscope}%
\pgfsetbuttcap%
\pgfsetroundjoin%
\definecolor{currentfill}{rgb}{0.000000,0.000000,0.000000}%
\pgfsetfillcolor{currentfill}%
\pgfsetlinewidth{0.803000pt}%
\definecolor{currentstroke}{rgb}{0.000000,0.000000,0.000000}%
\pgfsetstrokecolor{currentstroke}%
\pgfsetdash{}{0pt}%
\pgfsys@defobject{currentmarker}{\pgfqpoint{-0.048611in}{0.000000in}}{\pgfqpoint{0.000000in}{0.000000in}}{%
\pgfpathmoveto{\pgfqpoint{0.000000in}{0.000000in}}%
\pgfpathlineto{\pgfqpoint{-0.048611in}{0.000000in}}%
\pgfusepath{stroke,fill}%
}%
\begin{pgfscope}%
\pgfsys@transformshift{4.482986in}{1.054412in}%
\pgfsys@useobject{currentmarker}{}%
\end{pgfscope}%
\end{pgfscope}%
\begin{pgfscope}%
\definecolor{textcolor}{rgb}{0.000000,0.000000,0.000000}%
\pgfsetstrokecolor{textcolor}%
\pgfsetfillcolor{textcolor}%
\pgftext[x=4.164884in,y=1.001650in,left,base]{\color{textcolor}\sffamily\fontsize{10.000000}{12.000000}\selectfont 0.2}%
\end{pgfscope}%
\begin{pgfscope}%
\pgfsetbuttcap%
\pgfsetroundjoin%
\definecolor{currentfill}{rgb}{0.000000,0.000000,0.000000}%
\pgfsetfillcolor{currentfill}%
\pgfsetlinewidth{0.803000pt}%
\definecolor{currentstroke}{rgb}{0.000000,0.000000,0.000000}%
\pgfsetstrokecolor{currentstroke}%
\pgfsetdash{}{0pt}%
\pgfsys@defobject{currentmarker}{\pgfqpoint{-0.048611in}{0.000000in}}{\pgfqpoint{0.000000in}{0.000000in}}{%
\pgfpathmoveto{\pgfqpoint{0.000000in}{0.000000in}}%
\pgfpathlineto{\pgfqpoint{-0.048611in}{0.000000in}}%
\pgfusepath{stroke,fill}%
}%
\begin{pgfscope}%
\pgfsys@transformshift{4.482986in}{1.361127in}%
\pgfsys@useobject{currentmarker}{}%
\end{pgfscope}%
\end{pgfscope}%
\begin{pgfscope}%
\definecolor{textcolor}{rgb}{0.000000,0.000000,0.000000}%
\pgfsetstrokecolor{textcolor}%
\pgfsetfillcolor{textcolor}%
\pgftext[x=4.164884in,y=1.308365in,left,base]{\color{textcolor}\sffamily\fontsize{10.000000}{12.000000}\selectfont 0.4}%
\end{pgfscope}%
\begin{pgfscope}%
\pgfsetbuttcap%
\pgfsetroundjoin%
\definecolor{currentfill}{rgb}{0.000000,0.000000,0.000000}%
\pgfsetfillcolor{currentfill}%
\pgfsetlinewidth{0.803000pt}%
\definecolor{currentstroke}{rgb}{0.000000,0.000000,0.000000}%
\pgfsetstrokecolor{currentstroke}%
\pgfsetdash{}{0pt}%
\pgfsys@defobject{currentmarker}{\pgfqpoint{-0.048611in}{0.000000in}}{\pgfqpoint{0.000000in}{0.000000in}}{%
\pgfpathmoveto{\pgfqpoint{0.000000in}{0.000000in}}%
\pgfpathlineto{\pgfqpoint{-0.048611in}{0.000000in}}%
\pgfusepath{stroke,fill}%
}%
\begin{pgfscope}%
\pgfsys@transformshift{4.482986in}{1.667842in}%
\pgfsys@useobject{currentmarker}{}%
\end{pgfscope}%
\end{pgfscope}%
\begin{pgfscope}%
\definecolor{textcolor}{rgb}{0.000000,0.000000,0.000000}%
\pgfsetstrokecolor{textcolor}%
\pgfsetfillcolor{textcolor}%
\pgftext[x=4.164884in,y=1.615080in,left,base]{\color{textcolor}\sffamily\fontsize{10.000000}{12.000000}\selectfont 0.6}%
\end{pgfscope}%
\begin{pgfscope}%
\pgfsetbuttcap%
\pgfsetroundjoin%
\definecolor{currentfill}{rgb}{0.000000,0.000000,0.000000}%
\pgfsetfillcolor{currentfill}%
\pgfsetlinewidth{0.803000pt}%
\definecolor{currentstroke}{rgb}{0.000000,0.000000,0.000000}%
\pgfsetstrokecolor{currentstroke}%
\pgfsetdash{}{0pt}%
\pgfsys@defobject{currentmarker}{\pgfqpoint{-0.048611in}{0.000000in}}{\pgfqpoint{0.000000in}{0.000000in}}{%
\pgfpathmoveto{\pgfqpoint{0.000000in}{0.000000in}}%
\pgfpathlineto{\pgfqpoint{-0.048611in}{0.000000in}}%
\pgfusepath{stroke,fill}%
}%
\begin{pgfscope}%
\pgfsys@transformshift{4.482986in}{1.974557in}%
\pgfsys@useobject{currentmarker}{}%
\end{pgfscope}%
\end{pgfscope}%
\begin{pgfscope}%
\definecolor{textcolor}{rgb}{0.000000,0.000000,0.000000}%
\pgfsetstrokecolor{textcolor}%
\pgfsetfillcolor{textcolor}%
\pgftext[x=4.164884in,y=1.921795in,left,base]{\color{textcolor}\sffamily\fontsize{10.000000}{12.000000}\selectfont 0.8}%
\end{pgfscope}%
\begin{pgfscope}%
\pgfsetbuttcap%
\pgfsetroundjoin%
\definecolor{currentfill}{rgb}{0.000000,0.000000,0.000000}%
\pgfsetfillcolor{currentfill}%
\pgfsetlinewidth{0.803000pt}%
\definecolor{currentstroke}{rgb}{0.000000,0.000000,0.000000}%
\pgfsetstrokecolor{currentstroke}%
\pgfsetdash{}{0pt}%
\pgfsys@defobject{currentmarker}{\pgfqpoint{-0.048611in}{0.000000in}}{\pgfqpoint{0.000000in}{0.000000in}}{%
\pgfpathmoveto{\pgfqpoint{0.000000in}{0.000000in}}%
\pgfpathlineto{\pgfqpoint{-0.048611in}{0.000000in}}%
\pgfusepath{stroke,fill}%
}%
\begin{pgfscope}%
\pgfsys@transformshift{4.482986in}{2.281272in}%
\pgfsys@useobject{currentmarker}{}%
\end{pgfscope}%
\end{pgfscope}%
\begin{pgfscope}%
\definecolor{textcolor}{rgb}{0.000000,0.000000,0.000000}%
\pgfsetstrokecolor{textcolor}%
\pgfsetfillcolor{textcolor}%
\pgftext[x=4.164884in,y=2.228510in,left,base]{\color{textcolor}\sffamily\fontsize{10.000000}{12.000000}\selectfont 1.0}%
\end{pgfscope}%
\begin{pgfscope}%
\pgfsetbuttcap%
\pgfsetroundjoin%
\definecolor{currentfill}{rgb}{0.000000,0.000000,0.000000}%
\pgfsetfillcolor{currentfill}%
\pgfsetlinewidth{0.803000pt}%
\definecolor{currentstroke}{rgb}{0.000000,0.000000,0.000000}%
\pgfsetstrokecolor{currentstroke}%
\pgfsetdash{}{0pt}%
\pgfsys@defobject{currentmarker}{\pgfqpoint{-0.048611in}{0.000000in}}{\pgfqpoint{0.000000in}{0.000000in}}{%
\pgfpathmoveto{\pgfqpoint{0.000000in}{0.000000in}}%
\pgfpathlineto{\pgfqpoint{-0.048611in}{0.000000in}}%
\pgfusepath{stroke,fill}%
}%
\begin{pgfscope}%
\pgfsys@transformshift{4.482986in}{2.587987in}%
\pgfsys@useobject{currentmarker}{}%
\end{pgfscope}%
\end{pgfscope}%
\begin{pgfscope}%
\definecolor{textcolor}{rgb}{0.000000,0.000000,0.000000}%
\pgfsetstrokecolor{textcolor}%
\pgfsetfillcolor{textcolor}%
\pgftext[x=4.164884in,y=2.535225in,left,base]{\color{textcolor}\sffamily\fontsize{10.000000}{12.000000}\selectfont 1.2}%
\end{pgfscope}%
\begin{pgfscope}%
\pgfpathrectangle{\pgfqpoint{4.482986in}{0.438889in}}{\pgfqpoint{3.318403in}{2.263889in}}%
\pgfusepath{clip}%
\pgfsetrectcap%
\pgfsetroundjoin%
\pgfsetlinewidth{1.505625pt}%
\definecolor{currentstroke}{rgb}{0.121569,0.466667,0.705882}%
\pgfsetstrokecolor{currentstroke}%
\pgfsetdash{}{0pt}%
\pgfpathmoveto{\pgfqpoint{4.633823in}{0.764749in}}%
\pgfpathlineto{\pgfqpoint{4.784609in}{0.767561in}}%
\pgfpathlineto{\pgfqpoint{4.935395in}{0.756919in}}%
\pgfpathlineto{\pgfqpoint{5.086181in}{0.759788in}}%
\pgfpathlineto{\pgfqpoint{5.236967in}{0.768130in}}%
\pgfpathlineto{\pgfqpoint{5.387753in}{0.786866in}}%
\pgfpathlineto{\pgfqpoint{5.538539in}{0.792462in}}%
\pgfpathlineto{\pgfqpoint{5.689325in}{0.792702in}}%
\pgfpathlineto{\pgfqpoint{5.840111in}{0.835332in}}%
\pgfpathlineto{\pgfqpoint{5.990897in}{0.786811in}}%
\pgfpathlineto{\pgfqpoint{6.141683in}{0.804543in}}%
\pgfpathlineto{\pgfqpoint{6.292469in}{0.828774in}}%
\pgfpathlineto{\pgfqpoint{6.443255in}{0.864559in}}%
\pgfpathlineto{\pgfqpoint{6.594041in}{0.851787in}}%
\pgfpathlineto{\pgfqpoint{6.744827in}{1.088990in}}%
\pgfpathlineto{\pgfqpoint{6.895613in}{1.055189in}}%
\pgfpathlineto{\pgfqpoint{7.046399in}{1.487584in}}%
\pgfpathlineto{\pgfqpoint{7.197185in}{1.634492in}}%
\pgfpathlineto{\pgfqpoint{7.347971in}{0.933548in}}%
\pgfpathlineto{\pgfqpoint{7.498756in}{1.544487in}}%
\pgfpathlineto{\pgfqpoint{7.649542in}{1.363508in}}%
\pgfusepath{stroke}%
\end{pgfscope}%
\begin{pgfscope}%
\pgfpathrectangle{\pgfqpoint{4.482986in}{0.438889in}}{\pgfqpoint{3.318403in}{2.263889in}}%
\pgfusepath{clip}%
\pgfsetrectcap%
\pgfsetroundjoin%
\pgfsetlinewidth{1.505625pt}%
\definecolor{currentstroke}{rgb}{1.000000,0.498039,0.054902}%
\pgfsetstrokecolor{currentstroke}%
\pgfsetdash{}{0pt}%
\pgfpathmoveto{\pgfqpoint{4.633823in}{0.761548in}}%
\pgfpathlineto{\pgfqpoint{4.784609in}{0.765046in}}%
\pgfpathlineto{\pgfqpoint{4.935395in}{0.759543in}}%
\pgfpathlineto{\pgfqpoint{5.086181in}{0.755540in}}%
\pgfpathlineto{\pgfqpoint{5.236967in}{0.763161in}}%
\pgfpathlineto{\pgfqpoint{5.387753in}{0.783863in}}%
\pgfpathlineto{\pgfqpoint{5.538539in}{0.763983in}}%
\pgfpathlineto{\pgfqpoint{5.689325in}{0.776281in}}%
\pgfpathlineto{\pgfqpoint{5.840111in}{0.786858in}}%
\pgfpathlineto{\pgfqpoint{5.990897in}{0.820554in}}%
\pgfpathlineto{\pgfqpoint{6.141683in}{0.771508in}}%
\pgfpathlineto{\pgfqpoint{6.292469in}{0.912148in}}%
\pgfpathlineto{\pgfqpoint{6.443255in}{0.856542in}}%
\pgfpathlineto{\pgfqpoint{6.594041in}{0.916649in}}%
\pgfpathlineto{\pgfqpoint{6.744827in}{1.040257in}}%
\pgfpathlineto{\pgfqpoint{6.895613in}{1.227234in}}%
\pgfpathlineto{\pgfqpoint{7.046399in}{1.061679in}}%
\pgfpathlineto{\pgfqpoint{7.197185in}{1.622452in}}%
\pgfpathlineto{\pgfqpoint{7.347971in}{1.756146in}}%
\pgfpathlineto{\pgfqpoint{7.498756in}{1.565727in}}%
\pgfpathlineto{\pgfqpoint{7.649542in}{1.672714in}}%
\pgfusepath{stroke}%
\end{pgfscope}%
\begin{pgfscope}%
\pgfpathrectangle{\pgfqpoint{4.482986in}{0.438889in}}{\pgfqpoint{3.318403in}{2.263889in}}%
\pgfusepath{clip}%
\pgfsetrectcap%
\pgfsetroundjoin%
\pgfsetlinewidth{1.505625pt}%
\definecolor{currentstroke}{rgb}{0.172549,0.627451,0.172549}%
\pgfsetstrokecolor{currentstroke}%
\pgfsetdash{}{0pt}%
\pgfpathmoveto{\pgfqpoint{4.633823in}{0.757559in}}%
\pgfpathlineto{\pgfqpoint{4.784609in}{0.762641in}}%
\pgfpathlineto{\pgfqpoint{4.935395in}{0.754460in}}%
\pgfpathlineto{\pgfqpoint{5.086181in}{0.770158in}}%
\pgfpathlineto{\pgfqpoint{5.236967in}{0.769649in}}%
\pgfpathlineto{\pgfqpoint{5.387753in}{0.765221in}}%
\pgfpathlineto{\pgfqpoint{5.538539in}{0.786852in}}%
\pgfpathlineto{\pgfqpoint{5.689325in}{0.773442in}}%
\pgfpathlineto{\pgfqpoint{5.840111in}{0.857947in}}%
\pgfpathlineto{\pgfqpoint{5.990897in}{0.789164in}}%
\pgfpathlineto{\pgfqpoint{6.141683in}{0.799222in}}%
\pgfpathlineto{\pgfqpoint{6.292469in}{0.911399in}}%
\pgfpathlineto{\pgfqpoint{6.443255in}{1.002522in}}%
\pgfpathlineto{\pgfqpoint{6.594041in}{1.152909in}}%
\pgfpathlineto{\pgfqpoint{6.744827in}{1.142144in}}%
\pgfpathlineto{\pgfqpoint{6.895613in}{0.841196in}}%
\pgfpathlineto{\pgfqpoint{7.046399in}{1.382422in}}%
\pgfpathlineto{\pgfqpoint{7.197185in}{1.793381in}}%
\pgfpathlineto{\pgfqpoint{7.347971in}{1.474619in}}%
\pgfpathlineto{\pgfqpoint{7.498756in}{2.110144in}}%
\pgfpathlineto{\pgfqpoint{7.649542in}{1.870352in}}%
\pgfusepath{stroke}%
\end{pgfscope}%
\begin{pgfscope}%
\pgfpathrectangle{\pgfqpoint{4.482986in}{0.438889in}}{\pgfqpoint{3.318403in}{2.263889in}}%
\pgfusepath{clip}%
\pgfsetrectcap%
\pgfsetroundjoin%
\pgfsetlinewidth{1.505625pt}%
\definecolor{currentstroke}{rgb}{0.839216,0.152941,0.156863}%
\pgfsetstrokecolor{currentstroke}%
\pgfsetdash{}{0pt}%
\pgfpathmoveto{\pgfqpoint{4.633823in}{0.757423in}}%
\pgfpathlineto{\pgfqpoint{4.784609in}{0.758159in}}%
\pgfpathlineto{\pgfqpoint{4.935395in}{0.759199in}}%
\pgfpathlineto{\pgfqpoint{5.086181in}{0.766876in}}%
\pgfpathlineto{\pgfqpoint{5.236967in}{0.756045in}}%
\pgfpathlineto{\pgfqpoint{5.387753in}{0.768701in}}%
\pgfpathlineto{\pgfqpoint{5.538539in}{0.763696in}}%
\pgfpathlineto{\pgfqpoint{5.689325in}{0.773673in}}%
\pgfpathlineto{\pgfqpoint{5.840111in}{0.802583in}}%
\pgfpathlineto{\pgfqpoint{5.990897in}{0.803897in}}%
\pgfpathlineto{\pgfqpoint{6.141683in}{0.800905in}}%
\pgfpathlineto{\pgfqpoint{6.292469in}{0.784364in}}%
\pgfpathlineto{\pgfqpoint{6.443255in}{0.802315in}}%
\pgfpathlineto{\pgfqpoint{6.594041in}{0.814135in}}%
\pgfpathlineto{\pgfqpoint{6.744827in}{0.808637in}}%
\pgfpathlineto{\pgfqpoint{6.895613in}{0.841750in}}%
\pgfpathlineto{\pgfqpoint{7.046399in}{0.819765in}}%
\pgfpathlineto{\pgfqpoint{7.197185in}{0.831620in}}%
\pgfpathlineto{\pgfqpoint{7.347971in}{0.823799in}}%
\pgfpathlineto{\pgfqpoint{7.498756in}{0.825166in}}%
\pgfpathlineto{\pgfqpoint{7.649542in}{0.820728in}}%
\pgfusepath{stroke}%
\end{pgfscope}%
\begin{pgfscope}%
\pgfpathrectangle{\pgfqpoint{4.482986in}{0.438889in}}{\pgfqpoint{3.318403in}{2.263889in}}%
\pgfusepath{clip}%
\pgfsetrectcap%
\pgfsetroundjoin%
\pgfsetlinewidth{1.505625pt}%
\definecolor{currentstroke}{rgb}{0.580392,0.403922,0.741176}%
\pgfsetstrokecolor{currentstroke}%
\pgfsetdash{}{0pt}%
\pgfpathmoveto{\pgfqpoint{4.633823in}{0.762455in}}%
\pgfpathlineto{\pgfqpoint{4.784609in}{0.759105in}}%
\pgfpathlineto{\pgfqpoint{4.935395in}{0.770185in}}%
\pgfpathlineto{\pgfqpoint{5.086181in}{0.759677in}}%
\pgfpathlineto{\pgfqpoint{5.236967in}{0.771885in}}%
\pgfpathlineto{\pgfqpoint{5.387753in}{0.768289in}}%
\pgfpathlineto{\pgfqpoint{5.538539in}{0.780615in}}%
\pgfpathlineto{\pgfqpoint{5.689325in}{0.786678in}}%
\pgfpathlineto{\pgfqpoint{5.840111in}{0.792690in}}%
\pgfpathlineto{\pgfqpoint{5.990897in}{0.770196in}}%
\pgfpathlineto{\pgfqpoint{6.141683in}{0.777113in}}%
\pgfpathlineto{\pgfqpoint{6.292469in}{0.774979in}}%
\pgfpathlineto{\pgfqpoint{6.443255in}{0.779010in}}%
\pgfpathlineto{\pgfqpoint{6.594041in}{0.781131in}}%
\pgfpathlineto{\pgfqpoint{6.744827in}{0.778297in}}%
\pgfpathlineto{\pgfqpoint{6.895613in}{0.777872in}}%
\pgfpathlineto{\pgfqpoint{7.046399in}{0.781578in}}%
\pgfpathlineto{\pgfqpoint{7.197185in}{0.780584in}}%
\pgfpathlineto{\pgfqpoint{7.347971in}{0.780324in}}%
\pgfpathlineto{\pgfqpoint{7.498756in}{0.780137in}}%
\pgfpathlineto{\pgfqpoint{7.649542in}{0.779765in}}%
\pgfusepath{stroke}%
\end{pgfscope}%
\begin{pgfscope}%
\pgfsetrectcap%
\pgfsetmiterjoin%
\pgfsetlinewidth{0.803000pt}%
\definecolor{currentstroke}{rgb}{0.000000,0.000000,0.000000}%
\pgfsetstrokecolor{currentstroke}%
\pgfsetdash{}{0pt}%
\pgfpathmoveto{\pgfqpoint{4.482986in}{0.438889in}}%
\pgfpathlineto{\pgfqpoint{4.482986in}{2.702778in}}%
\pgfusepath{stroke}%
\end{pgfscope}%
\begin{pgfscope}%
\pgfsetrectcap%
\pgfsetmiterjoin%
\pgfsetlinewidth{0.803000pt}%
\definecolor{currentstroke}{rgb}{0.000000,0.000000,0.000000}%
\pgfsetstrokecolor{currentstroke}%
\pgfsetdash{}{0pt}%
\pgfpathmoveto{\pgfqpoint{7.801389in}{0.438889in}}%
\pgfpathlineto{\pgfqpoint{7.801389in}{2.702778in}}%
\pgfusepath{stroke}%
\end{pgfscope}%
\begin{pgfscope}%
\pgfsetrectcap%
\pgfsetmiterjoin%
\pgfsetlinewidth{0.803000pt}%
\definecolor{currentstroke}{rgb}{0.000000,0.000000,0.000000}%
\pgfsetstrokecolor{currentstroke}%
\pgfsetdash{}{0pt}%
\pgfpathmoveto{\pgfqpoint{4.482986in}{0.438889in}}%
\pgfpathlineto{\pgfqpoint{7.801389in}{0.438889in}}%
\pgfusepath{stroke}%
\end{pgfscope}%
\begin{pgfscope}%
\pgfsetrectcap%
\pgfsetmiterjoin%
\pgfsetlinewidth{0.803000pt}%
\definecolor{currentstroke}{rgb}{0.000000,0.000000,0.000000}%
\pgfsetstrokecolor{currentstroke}%
\pgfsetdash{}{0pt}%
\pgfpathmoveto{\pgfqpoint{4.482986in}{2.702778in}}%
\pgfpathlineto{\pgfqpoint{7.801389in}{2.702778in}}%
\pgfusepath{stroke}%
\end{pgfscope}%
\begin{pgfscope}%
\definecolor{textcolor}{rgb}{0.000000,0.000000,0.000000}%
\pgfsetstrokecolor{textcolor}%
\pgfsetfillcolor{textcolor}%
\pgftext[x=6.142187in,y=2.786111in,,base]{\color{textcolor}\sffamily\fontsize{12.000000}{14.400000}\selectfont \(\displaystyle  N = 128 \)}%
\end{pgfscope}%
\begin{pgfscope}%
\pgfsetbuttcap%
\pgfsetmiterjoin%
\definecolor{currentfill}{rgb}{1.000000,1.000000,1.000000}%
\pgfsetfillcolor{currentfill}%
\pgfsetfillopacity{0.800000}%
\pgfsetlinewidth{1.003750pt}%
\definecolor{currentstroke}{rgb}{0.800000,0.800000,0.800000}%
\pgfsetstrokecolor{currentstroke}%
\pgfsetstrokeopacity{0.800000}%
\pgfsetdash{}{0pt}%
\pgfpathmoveto{\pgfqpoint{4.580208in}{1.572381in}}%
\pgfpathlineto{\pgfqpoint{5.487760in}{1.572381in}}%
\pgfpathquadraticcurveto{\pgfqpoint{5.515537in}{1.572381in}}{\pgfqpoint{5.515537in}{1.600158in}}%
\pgfpathlineto{\pgfqpoint{5.515537in}{2.605556in}}%
\pgfpathquadraticcurveto{\pgfqpoint{5.515537in}{2.633333in}}{\pgfqpoint{5.487760in}{2.633333in}}%
\pgfpathlineto{\pgfqpoint{4.580208in}{2.633333in}}%
\pgfpathquadraticcurveto{\pgfqpoint{4.552431in}{2.633333in}}{\pgfqpoint{4.552431in}{2.605556in}}%
\pgfpathlineto{\pgfqpoint{4.552431in}{1.600158in}}%
\pgfpathquadraticcurveto{\pgfqpoint{4.552431in}{1.572381in}}{\pgfqpoint{4.580208in}{1.572381in}}%
\pgfpathclose%
\pgfusepath{stroke,fill}%
\end{pgfscope}%
\begin{pgfscope}%
\pgfsetrectcap%
\pgfsetroundjoin%
\pgfsetlinewidth{1.505625pt}%
\definecolor{currentstroke}{rgb}{0.121569,0.466667,0.705882}%
\pgfsetstrokecolor{currentstroke}%
\pgfsetdash{}{0pt}%
\pgfpathmoveto{\pgfqpoint{4.607986in}{2.520866in}}%
\pgfpathlineto{\pgfqpoint{4.885764in}{2.520866in}}%
\pgfusepath{stroke}%
\end{pgfscope}%
\begin{pgfscope}%
\definecolor{textcolor}{rgb}{0.000000,0.000000,0.000000}%
\pgfsetstrokecolor{textcolor}%
\pgfsetfillcolor{textcolor}%
\pgftext[x=4.996875in,y=2.472255in,left,base]{\color{textcolor}\sffamily\fontsize{10.000000}{12.000000}\selectfont \(\displaystyle  T = 1.0 \)}%
\end{pgfscope}%
\begin{pgfscope}%
\pgfsetrectcap%
\pgfsetroundjoin%
\pgfsetlinewidth{1.505625pt}%
\definecolor{currentstroke}{rgb}{1.000000,0.498039,0.054902}%
\pgfsetstrokecolor{currentstroke}%
\pgfsetdash{}{0pt}%
\pgfpathmoveto{\pgfqpoint{4.607986in}{2.317009in}}%
\pgfpathlineto{\pgfqpoint{4.885764in}{2.317009in}}%
\pgfusepath{stroke}%
\end{pgfscope}%
\begin{pgfscope}%
\definecolor{textcolor}{rgb}{0.000000,0.000000,0.000000}%
\pgfsetstrokecolor{textcolor}%
\pgfsetfillcolor{textcolor}%
\pgftext[x=4.996875in,y=2.268397in,left,base]{\color{textcolor}\sffamily\fontsize{10.000000}{12.000000}\selectfont \(\displaystyle  T = 1.5 \)}%
\end{pgfscope}%
\begin{pgfscope}%
\pgfsetrectcap%
\pgfsetroundjoin%
\pgfsetlinewidth{1.505625pt}%
\definecolor{currentstroke}{rgb}{0.172549,0.627451,0.172549}%
\pgfsetstrokecolor{currentstroke}%
\pgfsetdash{}{0pt}%
\pgfpathmoveto{\pgfqpoint{4.607986in}{2.113151in}}%
\pgfpathlineto{\pgfqpoint{4.885764in}{2.113151in}}%
\pgfusepath{stroke}%
\end{pgfscope}%
\begin{pgfscope}%
\definecolor{textcolor}{rgb}{0.000000,0.000000,0.000000}%
\pgfsetstrokecolor{textcolor}%
\pgfsetfillcolor{textcolor}%
\pgftext[x=4.996875in,y=2.064540in,left,base]{\color{textcolor}\sffamily\fontsize{10.000000}{12.000000}\selectfont \(\displaystyle  T = 2.0 \)}%
\end{pgfscope}%
\begin{pgfscope}%
\pgfsetrectcap%
\pgfsetroundjoin%
\pgfsetlinewidth{1.505625pt}%
\definecolor{currentstroke}{rgb}{0.839216,0.152941,0.156863}%
\pgfsetstrokecolor{currentstroke}%
\pgfsetdash{}{0pt}%
\pgfpathmoveto{\pgfqpoint{4.607986in}{1.909294in}}%
\pgfpathlineto{\pgfqpoint{4.885764in}{1.909294in}}%
\pgfusepath{stroke}%
\end{pgfscope}%
\begin{pgfscope}%
\definecolor{textcolor}{rgb}{0.000000,0.000000,0.000000}%
\pgfsetstrokecolor{textcolor}%
\pgfsetfillcolor{textcolor}%
\pgftext[x=4.996875in,y=1.860683in,left,base]{\color{textcolor}\sffamily\fontsize{10.000000}{12.000000}\selectfont \(\displaystyle  T = 2.5 \)}%
\end{pgfscope}%
\begin{pgfscope}%
\pgfsetrectcap%
\pgfsetroundjoin%
\pgfsetlinewidth{1.505625pt}%
\definecolor{currentstroke}{rgb}{0.580392,0.403922,0.741176}%
\pgfsetstrokecolor{currentstroke}%
\pgfsetdash{}{0pt}%
\pgfpathmoveto{\pgfqpoint{4.607986in}{1.705437in}}%
\pgfpathlineto{\pgfqpoint{4.885764in}{1.705437in}}%
\pgfusepath{stroke}%
\end{pgfscope}%
\begin{pgfscope}%
\definecolor{textcolor}{rgb}{0.000000,0.000000,0.000000}%
\pgfsetstrokecolor{textcolor}%
\pgfsetfillcolor{textcolor}%
\pgftext[x=4.996875in,y=1.656826in,left,base]{\color{textcolor}\sffamily\fontsize{10.000000}{12.000000}\selectfont \(\displaystyle  T = 3.0 \)}%
\end{pgfscope}%
\end{pgfpicture}%
\makeatother%
\endgroup%
}
\scalebox{0.75}{%% Creator: Matplotlib, PGF backend
%%
%% To include the figure in your LaTeX document, write
%%   \input{<filename>.pgf}
%%
%% Make sure the required packages are loaded in your preamble
%%   \usepackage{pgf}
%%
%% Figures using additional raster images can only be included by \input if
%% they are in the same directory as the main LaTeX file. For loading figures
%% from other directories you can use the `import` package
%%   \usepackage{import}
%% and then include the figures with
%%   \import{<path to file>}{<filename>.pgf}
%%
%% Matplotlib used the following preamble
%%   \usepackage{fontspec}
%%   \setmainfont{DejaVuSerif.ttf}[Path=/home/lzh/anaconda3/envs/numana/lib/python3.7/site-packages/matplotlib/mpl-data/fonts/ttf/]
%%   \setsansfont{DejaVuSans.ttf}[Path=/home/lzh/anaconda3/envs/numana/lib/python3.7/site-packages/matplotlib/mpl-data/fonts/ttf/]
%%   \setmonofont{DejaVuSansMono.ttf}[Path=/home/lzh/anaconda3/envs/numana/lib/python3.7/site-packages/matplotlib/mpl-data/fonts/ttf/]
%%
\begingroup%
\makeatletter%
\begin{pgfpicture}%
\pgfpathrectangle{\pgfpointorigin}{\pgfqpoint{12.000000in}{4.000000in}}%
\pgfusepath{use as bounding box, clip}%
\begin{pgfscope}%
\pgfsetbuttcap%
\pgfsetmiterjoin%
\definecolor{currentfill}{rgb}{1.000000,1.000000,1.000000}%
\pgfsetfillcolor{currentfill}%
\pgfsetlinewidth{0.000000pt}%
\definecolor{currentstroke}{rgb}{1.000000,1.000000,1.000000}%
\pgfsetstrokecolor{currentstroke}%
\pgfsetdash{}{0pt}%
\pgfpathmoveto{\pgfqpoint{0.000000in}{0.000000in}}%
\pgfpathlineto{\pgfqpoint{12.000000in}{0.000000in}}%
\pgfpathlineto{\pgfqpoint{12.000000in}{4.000000in}}%
\pgfpathlineto{\pgfqpoint{0.000000in}{4.000000in}}%
\pgfpathclose%
\pgfusepath{fill}%
\end{pgfscope}%
\begin{pgfscope}%
\pgfsys@transformshift{0.120000in}{0.085000in}%
\pgftext[left,bottom]{\pgfimage[interpolate=true,width=3.830000in,height=3.765000in]{Figure04-img0.png}}%
\end{pgfscope}%
\begin{pgfscope}%
\pgfsys@transformshift{4.100000in}{0.120000in}%
\pgftext[left,bottom]{\pgfimage[interpolate=true,width=3.930000in,height=3.730000in]{Figure04-img1.png}}%
\end{pgfscope}%
\begin{pgfscope}%
\pgfsys@transformshift{8.050000in}{0.150000in}%
\pgftext[left,bottom]{\pgfimage[interpolate=true,width=3.950000in,height=3.705000in]{Figure04-img2.png}}%
\end{pgfscope}%
\end{pgfpicture}%
\makeatother%
\endgroup%
}~
\scalebox{0.75}{%% Creator: Matplotlib, PGF backend
%%
%% To include the figure in your LaTeX document, write
%%   \input{<filename>.pgf}
%%
%% Make sure the required packages are loaded in your preamble
%%   \usepackage{pgf}
%%
%% Figures using additional raster images can only be included by \input if
%% they are in the same directory as the main LaTeX file. For loading figures
%% from other directories you can use the `import` package
%%   \usepackage{import}
%% and then include the figures with
%%   \import{<path to file>}{<filename>.pgf}
%%
%% Matplotlib used the following preamble
%%   \usepackage{fontspec}
%%   \setmainfont{DejaVuSerif.ttf}[Path=/home/lzh/anaconda3/envs/numana/lib/python3.7/site-packages/matplotlib/mpl-data/fonts/ttf/]
%%   \setsansfont{DejaVuSans.ttf}[Path=/home/lzh/anaconda3/envs/numana/lib/python3.7/site-packages/matplotlib/mpl-data/fonts/ttf/]
%%   \setmonofont{DejaVuSansMono.ttf}[Path=/home/lzh/anaconda3/envs/numana/lib/python3.7/site-packages/matplotlib/mpl-data/fonts/ttf/]
%%
\begingroup%
\makeatletter%
\begin{pgfpicture}%
\pgfpathrectangle{\pgfpointorigin}{\pgfqpoint{12.000000in}{4.000000in}}%
\pgfusepath{use as bounding box, clip}%
\begin{pgfscope}%
\pgfsetbuttcap%
\pgfsetmiterjoin%
\definecolor{currentfill}{rgb}{1.000000,1.000000,1.000000}%
\pgfsetfillcolor{currentfill}%
\pgfsetlinewidth{0.000000pt}%
\definecolor{currentstroke}{rgb}{1.000000,1.000000,1.000000}%
\pgfsetstrokecolor{currentstroke}%
\pgfsetdash{}{0pt}%
\pgfpathmoveto{\pgfqpoint{0.000000in}{0.000000in}}%
\pgfpathlineto{\pgfqpoint{12.000000in}{0.000000in}}%
\pgfpathlineto{\pgfqpoint{12.000000in}{4.000000in}}%
\pgfpathlineto{\pgfqpoint{0.000000in}{4.000000in}}%
\pgfpathclose%
\pgfusepath{fill}%
\end{pgfscope}%
\begin{pgfscope}%
\pgfsys@transformshift{0.120000in}{0.105000in}%
\pgftext[left,bottom]{\pgfimage[interpolate=true,width=3.830000in,height=3.745000in]{Figure05-img0.png}}%
\end{pgfscope}%
\begin{pgfscope}%
\pgfsys@transformshift{4.100000in}{0.070000in}%
\pgftext[left,bottom]{\pgfimage[interpolate=true,width=3.930000in,height=3.780000in]{Figure05-img1.png}}%
\end{pgfscope}%
\begin{pgfscope}%
\pgfsys@transformshift{8.050000in}{0.150000in}%
\pgftext[left,bottom]{\pgfimage[interpolate=true,width=3.950000in,height=3.780000in]{Figure05-img2.png}}%
\end{pgfscope}%
\end{pgfpicture}%
\makeatother%
\endgroup%
}
\scalebox{0.75}{%% Creator: Matplotlib, PGF backend
%%
%% To include the figure in your LaTeX document, write
%%   \input{<filename>.pgf}
%%
%% Make sure the required packages are loaded in your preamble
%%   \usepackage{pgf}
%%
%% Figures using additional raster images can only be included by \input if
%% they are in the same directory as the main LaTeX file. For loading figures
%% from other directories you can use the `import` package
%%   \usepackage{import}
%% and then include the figures with
%%   \import{<path to file>}{<filename>.pgf}
%%
%% Matplotlib used the following preamble
%%   \usepackage{fontspec}
%%   \setmainfont{DejaVuSerif.ttf}[Path=/home/lzh/anaconda3/envs/numana/lib/python3.7/site-packages/matplotlib/mpl-data/fonts/ttf/]
%%   \setsansfont{DejaVuSans.ttf}[Path=/home/lzh/anaconda3/envs/numana/lib/python3.7/site-packages/matplotlib/mpl-data/fonts/ttf/]
%%   \setmonofont{DejaVuSansMono.ttf}[Path=/home/lzh/anaconda3/envs/numana/lib/python3.7/site-packages/matplotlib/mpl-data/fonts/ttf/]
%%
\begingroup%
\makeatletter%
\begin{pgfpicture}%
\pgfpathrectangle{\pgfpointorigin}{\pgfqpoint{8.000000in}{6.000000in}}%
\pgfusepath{use as bounding box, clip}%
\begin{pgfscope}%
\pgfsetbuttcap%
\pgfsetmiterjoin%
\definecolor{currentfill}{rgb}{1.000000,1.000000,1.000000}%
\pgfsetfillcolor{currentfill}%
\pgfsetlinewidth{0.000000pt}%
\definecolor{currentstroke}{rgb}{1.000000,1.000000,1.000000}%
\pgfsetstrokecolor{currentstroke}%
\pgfsetdash{}{0pt}%
\pgfpathmoveto{\pgfqpoint{0.000000in}{0.000000in}}%
\pgfpathlineto{\pgfqpoint{8.000000in}{0.000000in}}%
\pgfpathlineto{\pgfqpoint{8.000000in}{6.000000in}}%
\pgfpathlineto{\pgfqpoint{0.000000in}{6.000000in}}%
\pgfpathclose%
\pgfusepath{fill}%
\end{pgfscope}%
\begin{pgfscope}%
\pgfsetbuttcap%
\pgfsetmiterjoin%
\definecolor{currentfill}{rgb}{1.000000,1.000000,1.000000}%
\pgfsetfillcolor{currentfill}%
\pgfsetlinewidth{0.000000pt}%
\definecolor{currentstroke}{rgb}{0.000000,0.000000,0.000000}%
\pgfsetstrokecolor{currentstroke}%
\pgfsetstrokeopacity{0.000000}%
\pgfsetdash{}{0pt}%
\pgfpathmoveto{\pgfqpoint{0.467708in}{3.311111in}}%
\pgfpathlineto{\pgfqpoint{3.875521in}{3.311111in}}%
\pgfpathlineto{\pgfqpoint{3.875521in}{5.627778in}}%
\pgfpathlineto{\pgfqpoint{0.467708in}{5.627778in}}%
\pgfpathclose%
\pgfusepath{fill}%
\end{pgfscope}%
\begin{pgfscope}%
\pgfpathrectangle{\pgfqpoint{0.467708in}{3.311111in}}{\pgfqpoint{3.407812in}{2.316667in}}%
\pgfusepath{clip}%
\pgfsetbuttcap%
\pgfsetroundjoin%
\definecolor{currentfill}{rgb}{0.121569,0.466667,0.705882}%
\pgfsetfillcolor{currentfill}%
\pgfsetfillopacity{0.300000}%
\pgfsetlinewidth{1.003750pt}%
\definecolor{currentstroke}{rgb}{0.121569,0.466667,0.705882}%
\pgfsetstrokecolor{currentstroke}%
\pgfsetstrokeopacity{0.300000}%
\pgfsetdash{}{0pt}%
\pgfpathmoveto{\pgfqpoint{0.634394in}{5.510248in}}%
\pgfpathlineto{\pgfqpoint{0.634394in}{5.510248in}}%
\pgfpathlineto{\pgfqpoint{0.711255in}{5.510255in}}%
\pgfpathlineto{\pgfqpoint{0.788116in}{5.510296in}}%
\pgfpathlineto{\pgfqpoint{0.864977in}{5.510439in}}%
\pgfpathlineto{\pgfqpoint{0.941838in}{5.510765in}}%
\pgfpathlineto{\pgfqpoint{1.018699in}{5.511244in}}%
\pgfpathlineto{\pgfqpoint{1.095560in}{5.511538in}}%
\pgfpathlineto{\pgfqpoint{1.172421in}{5.510903in}}%
\pgfpathlineto{\pgfqpoint{1.249282in}{5.508317in}}%
\pgfpathlineto{\pgfqpoint{1.326143in}{5.502719in}}%
\pgfpathlineto{\pgfqpoint{1.403004in}{5.493112in}}%
\pgfpathlineto{\pgfqpoint{1.479865in}{5.478403in}}%
\pgfpathlineto{\pgfqpoint{1.556726in}{5.457123in}}%
\pgfpathlineto{\pgfqpoint{1.633587in}{5.427011in}}%
\pgfpathlineto{\pgfqpoint{1.710448in}{5.384442in}}%
\pgfpathlineto{\pgfqpoint{1.787309in}{5.322154in}}%
\pgfpathlineto{\pgfqpoint{1.864171in}{5.222898in}}%
\pgfpathlineto{\pgfqpoint{1.941032in}{5.048773in}}%
\pgfpathlineto{\pgfqpoint{2.017893in}{4.762014in}}%
\pgfpathlineto{\pgfqpoint{2.094754in}{4.407617in}}%
\pgfpathlineto{\pgfqpoint{2.171615in}{4.111098in}}%
\pgfpathlineto{\pgfqpoint{2.248476in}{3.915555in}}%
\pgfpathlineto{\pgfqpoint{2.325337in}{3.792409in}}%
\pgfpathlineto{\pgfqpoint{2.402198in}{3.709782in}}%
\pgfpathlineto{\pgfqpoint{2.479059in}{3.652145in}}%
\pgfpathlineto{\pgfqpoint{2.555920in}{3.610943in}}%
\pgfpathlineto{\pgfqpoint{2.632781in}{3.578507in}}%
\pgfpathlineto{\pgfqpoint{2.709642in}{3.553662in}}%
\pgfpathlineto{\pgfqpoint{2.786503in}{3.532641in}}%
\pgfpathlineto{\pgfqpoint{2.863364in}{3.515453in}}%
\pgfpathlineto{\pgfqpoint{2.940225in}{3.501109in}}%
\pgfpathlineto{\pgfqpoint{3.017086in}{3.488385in}}%
\pgfpathlineto{\pgfqpoint{3.093947in}{3.477837in}}%
\pgfpathlineto{\pgfqpoint{3.170808in}{3.469218in}}%
\pgfpathlineto{\pgfqpoint{3.247669in}{3.460790in}}%
\pgfpathlineto{\pgfqpoint{3.324530in}{3.453838in}}%
\pgfpathlineto{\pgfqpoint{3.401391in}{3.446697in}}%
\pgfpathlineto{\pgfqpoint{3.478252in}{3.441354in}}%
\pgfpathlineto{\pgfqpoint{3.555113in}{3.435884in}}%
\pgfpathlineto{\pgfqpoint{3.631974in}{3.431458in}}%
\pgfpathlineto{\pgfqpoint{3.708835in}{3.426933in}}%
\pgfpathlineto{\pgfqpoint{3.708835in}{3.427769in}}%
\pgfpathlineto{\pgfqpoint{3.708835in}{3.427769in}}%
\pgfpathlineto{\pgfqpoint{3.631974in}{3.432231in}}%
\pgfpathlineto{\pgfqpoint{3.555113in}{3.436509in}}%
\pgfpathlineto{\pgfqpoint{3.478252in}{3.442261in}}%
\pgfpathlineto{\pgfqpoint{3.401391in}{3.447556in}}%
\pgfpathlineto{\pgfqpoint{3.324530in}{3.454725in}}%
\pgfpathlineto{\pgfqpoint{3.247669in}{3.461823in}}%
\pgfpathlineto{\pgfqpoint{3.170808in}{3.469850in}}%
\pgfpathlineto{\pgfqpoint{3.093947in}{3.478533in}}%
\pgfpathlineto{\pgfqpoint{3.017086in}{3.490149in}}%
\pgfpathlineto{\pgfqpoint{2.940225in}{3.502193in}}%
\pgfpathlineto{\pgfqpoint{2.863364in}{3.516289in}}%
\pgfpathlineto{\pgfqpoint{2.786503in}{3.533860in}}%
\pgfpathlineto{\pgfqpoint{2.709642in}{3.554366in}}%
\pgfpathlineto{\pgfqpoint{2.632781in}{3.580684in}}%
\pgfpathlineto{\pgfqpoint{2.555920in}{3.613548in}}%
\pgfpathlineto{\pgfqpoint{2.479059in}{3.654774in}}%
\pgfpathlineto{\pgfqpoint{2.402198in}{3.712358in}}%
\pgfpathlineto{\pgfqpoint{2.325337in}{3.797492in}}%
\pgfpathlineto{\pgfqpoint{2.248476in}{3.921634in}}%
\pgfpathlineto{\pgfqpoint{2.171615in}{4.118757in}}%
\pgfpathlineto{\pgfqpoint{2.094754in}{4.414316in}}%
\pgfpathlineto{\pgfqpoint{2.017893in}{4.764607in}}%
\pgfpathlineto{\pgfqpoint{1.941032in}{5.053919in}}%
\pgfpathlineto{\pgfqpoint{1.864171in}{5.224566in}}%
\pgfpathlineto{\pgfqpoint{1.787309in}{5.322569in}}%
\pgfpathlineto{\pgfqpoint{1.710448in}{5.384829in}}%
\pgfpathlineto{\pgfqpoint{1.633587in}{5.427186in}}%
\pgfpathlineto{\pgfqpoint{1.556726in}{5.457183in}}%
\pgfpathlineto{\pgfqpoint{1.479865in}{5.478431in}}%
\pgfpathlineto{\pgfqpoint{1.403004in}{5.493126in}}%
\pgfpathlineto{\pgfqpoint{1.326143in}{5.502730in}}%
\pgfpathlineto{\pgfqpoint{1.249282in}{5.508321in}}%
\pgfpathlineto{\pgfqpoint{1.172421in}{5.510904in}}%
\pgfpathlineto{\pgfqpoint{1.095560in}{5.511538in}}%
\pgfpathlineto{\pgfqpoint{1.018699in}{5.511244in}}%
\pgfpathlineto{\pgfqpoint{0.941838in}{5.510765in}}%
\pgfpathlineto{\pgfqpoint{0.864977in}{5.510439in}}%
\pgfpathlineto{\pgfqpoint{0.788116in}{5.510296in}}%
\pgfpathlineto{\pgfqpoint{0.711255in}{5.510255in}}%
\pgfpathlineto{\pgfqpoint{0.634394in}{5.510248in}}%
\pgfpathclose%
\pgfusepath{stroke,fill}%
\end{pgfscope}%
\begin{pgfscope}%
\pgfpathrectangle{\pgfqpoint{0.467708in}{3.311111in}}{\pgfqpoint{3.407812in}{2.316667in}}%
\pgfusepath{clip}%
\pgfsetbuttcap%
\pgfsetroundjoin%
\definecolor{currentfill}{rgb}{0.121569,0.466667,0.705882}%
\pgfsetfillcolor{currentfill}%
\pgfsetlinewidth{1.003750pt}%
\definecolor{currentstroke}{rgb}{0.121569,0.466667,0.705882}%
\pgfsetstrokecolor{currentstroke}%
\pgfsetdash{}{0pt}%
\pgfsys@defobject{currentmarker}{\pgfqpoint{-0.009821in}{-0.009821in}}{\pgfqpoint{0.009821in}{0.009821in}}{%
\pgfpathmoveto{\pgfqpoint{0.000000in}{-0.009821in}}%
\pgfpathcurveto{\pgfqpoint{0.002605in}{-0.009821in}}{\pgfqpoint{0.005103in}{-0.008786in}}{\pgfqpoint{0.006944in}{-0.006944in}}%
\pgfpathcurveto{\pgfqpoint{0.008786in}{-0.005103in}}{\pgfqpoint{0.009821in}{-0.002605in}}{\pgfqpoint{0.009821in}{0.000000in}}%
\pgfpathcurveto{\pgfqpoint{0.009821in}{0.002605in}}{\pgfqpoint{0.008786in}{0.005103in}}{\pgfqpoint{0.006944in}{0.006944in}}%
\pgfpathcurveto{\pgfqpoint{0.005103in}{0.008786in}}{\pgfqpoint{0.002605in}{0.009821in}}{\pgfqpoint{0.000000in}{0.009821in}}%
\pgfpathcurveto{\pgfqpoint{-0.002605in}{0.009821in}}{\pgfqpoint{-0.005103in}{0.008786in}}{\pgfqpoint{-0.006944in}{0.006944in}}%
\pgfpathcurveto{\pgfqpoint{-0.008786in}{0.005103in}}{\pgfqpoint{-0.009821in}{0.002605in}}{\pgfqpoint{-0.009821in}{0.000000in}}%
\pgfpathcurveto{\pgfqpoint{-0.009821in}{-0.002605in}}{\pgfqpoint{-0.008786in}{-0.005103in}}{\pgfqpoint{-0.006944in}{-0.006944in}}%
\pgfpathcurveto{\pgfqpoint{-0.005103in}{-0.008786in}}{\pgfqpoint{-0.002605in}{-0.009821in}}{\pgfqpoint{0.000000in}{-0.009821in}}%
\pgfpathclose%
\pgfusepath{stroke,fill}%
}%
\begin{pgfscope}%
\pgfsys@transformshift{0.634394in}{5.510248in}%
\pgfsys@useobject{currentmarker}{}%
\end{pgfscope}%
\begin{pgfscope}%
\pgfsys@transformshift{0.711255in}{5.510255in}%
\pgfsys@useobject{currentmarker}{}%
\end{pgfscope}%
\begin{pgfscope}%
\pgfsys@transformshift{0.788116in}{5.510296in}%
\pgfsys@useobject{currentmarker}{}%
\end{pgfscope}%
\begin{pgfscope}%
\pgfsys@transformshift{0.864977in}{5.510439in}%
\pgfsys@useobject{currentmarker}{}%
\end{pgfscope}%
\begin{pgfscope}%
\pgfsys@transformshift{0.941838in}{5.510765in}%
\pgfsys@useobject{currentmarker}{}%
\end{pgfscope}%
\begin{pgfscope}%
\pgfsys@transformshift{1.018699in}{5.511244in}%
\pgfsys@useobject{currentmarker}{}%
\end{pgfscope}%
\begin{pgfscope}%
\pgfsys@transformshift{1.095560in}{5.511538in}%
\pgfsys@useobject{currentmarker}{}%
\end{pgfscope}%
\begin{pgfscope}%
\pgfsys@transformshift{1.172421in}{5.510904in}%
\pgfsys@useobject{currentmarker}{}%
\end{pgfscope}%
\begin{pgfscope}%
\pgfsys@transformshift{1.249282in}{5.508319in}%
\pgfsys@useobject{currentmarker}{}%
\end{pgfscope}%
\begin{pgfscope}%
\pgfsys@transformshift{1.326143in}{5.502724in}%
\pgfsys@useobject{currentmarker}{}%
\end{pgfscope}%
\begin{pgfscope}%
\pgfsys@transformshift{1.403004in}{5.493119in}%
\pgfsys@useobject{currentmarker}{}%
\end{pgfscope}%
\begin{pgfscope}%
\pgfsys@transformshift{1.479865in}{5.478417in}%
\pgfsys@useobject{currentmarker}{}%
\end{pgfscope}%
\begin{pgfscope}%
\pgfsys@transformshift{1.556726in}{5.457153in}%
\pgfsys@useobject{currentmarker}{}%
\end{pgfscope}%
\begin{pgfscope}%
\pgfsys@transformshift{1.633587in}{5.427098in}%
\pgfsys@useobject{currentmarker}{}%
\end{pgfscope}%
\begin{pgfscope}%
\pgfsys@transformshift{1.710448in}{5.384636in}%
\pgfsys@useobject{currentmarker}{}%
\end{pgfscope}%
\begin{pgfscope}%
\pgfsys@transformshift{1.787309in}{5.322362in}%
\pgfsys@useobject{currentmarker}{}%
\end{pgfscope}%
\begin{pgfscope}%
\pgfsys@transformshift{1.864171in}{5.223732in}%
\pgfsys@useobject{currentmarker}{}%
\end{pgfscope}%
\begin{pgfscope}%
\pgfsys@transformshift{1.941032in}{5.051346in}%
\pgfsys@useobject{currentmarker}{}%
\end{pgfscope}%
\begin{pgfscope}%
\pgfsys@transformshift{2.017893in}{4.763310in}%
\pgfsys@useobject{currentmarker}{}%
\end{pgfscope}%
\begin{pgfscope}%
\pgfsys@transformshift{2.094754in}{4.410966in}%
\pgfsys@useobject{currentmarker}{}%
\end{pgfscope}%
\begin{pgfscope}%
\pgfsys@transformshift{2.171615in}{4.114928in}%
\pgfsys@useobject{currentmarker}{}%
\end{pgfscope}%
\begin{pgfscope}%
\pgfsys@transformshift{2.248476in}{3.918595in}%
\pgfsys@useobject{currentmarker}{}%
\end{pgfscope}%
\begin{pgfscope}%
\pgfsys@transformshift{2.325337in}{3.794951in}%
\pgfsys@useobject{currentmarker}{}%
\end{pgfscope}%
\begin{pgfscope}%
\pgfsys@transformshift{2.402198in}{3.711070in}%
\pgfsys@useobject{currentmarker}{}%
\end{pgfscope}%
\begin{pgfscope}%
\pgfsys@transformshift{2.479059in}{3.653459in}%
\pgfsys@useobject{currentmarker}{}%
\end{pgfscope}%
\begin{pgfscope}%
\pgfsys@transformshift{2.555920in}{3.612245in}%
\pgfsys@useobject{currentmarker}{}%
\end{pgfscope}%
\begin{pgfscope}%
\pgfsys@transformshift{2.632781in}{3.579595in}%
\pgfsys@useobject{currentmarker}{}%
\end{pgfscope}%
\begin{pgfscope}%
\pgfsys@transformshift{2.709642in}{3.554014in}%
\pgfsys@useobject{currentmarker}{}%
\end{pgfscope}%
\begin{pgfscope}%
\pgfsys@transformshift{2.786503in}{3.533251in}%
\pgfsys@useobject{currentmarker}{}%
\end{pgfscope}%
\begin{pgfscope}%
\pgfsys@transformshift{2.863364in}{3.515871in}%
\pgfsys@useobject{currentmarker}{}%
\end{pgfscope}%
\begin{pgfscope}%
\pgfsys@transformshift{2.940225in}{3.501651in}%
\pgfsys@useobject{currentmarker}{}%
\end{pgfscope}%
\begin{pgfscope}%
\pgfsys@transformshift{3.017086in}{3.489267in}%
\pgfsys@useobject{currentmarker}{}%
\end{pgfscope}%
\begin{pgfscope}%
\pgfsys@transformshift{3.093947in}{3.478185in}%
\pgfsys@useobject{currentmarker}{}%
\end{pgfscope}%
\begin{pgfscope}%
\pgfsys@transformshift{3.170808in}{3.469534in}%
\pgfsys@useobject{currentmarker}{}%
\end{pgfscope}%
\begin{pgfscope}%
\pgfsys@transformshift{3.247669in}{3.461307in}%
\pgfsys@useobject{currentmarker}{}%
\end{pgfscope}%
\begin{pgfscope}%
\pgfsys@transformshift{3.324530in}{3.454282in}%
\pgfsys@useobject{currentmarker}{}%
\end{pgfscope}%
\begin{pgfscope}%
\pgfsys@transformshift{3.401391in}{3.447126in}%
\pgfsys@useobject{currentmarker}{}%
\end{pgfscope}%
\begin{pgfscope}%
\pgfsys@transformshift{3.478252in}{3.441808in}%
\pgfsys@useobject{currentmarker}{}%
\end{pgfscope}%
\begin{pgfscope}%
\pgfsys@transformshift{3.555113in}{3.436196in}%
\pgfsys@useobject{currentmarker}{}%
\end{pgfscope}%
\begin{pgfscope}%
\pgfsys@transformshift{3.631974in}{3.431844in}%
\pgfsys@useobject{currentmarker}{}%
\end{pgfscope}%
\begin{pgfscope}%
\pgfsys@transformshift{3.708835in}{3.427351in}%
\pgfsys@useobject{currentmarker}{}%
\end{pgfscope}%
\end{pgfscope}%
\begin{pgfscope}%
\pgfsetbuttcap%
\pgfsetroundjoin%
\definecolor{currentfill}{rgb}{0.000000,0.000000,0.000000}%
\pgfsetfillcolor{currentfill}%
\pgfsetlinewidth{0.803000pt}%
\definecolor{currentstroke}{rgb}{0.000000,0.000000,0.000000}%
\pgfsetstrokecolor{currentstroke}%
\pgfsetdash{}{0pt}%
\pgfsys@defobject{currentmarker}{\pgfqpoint{0.000000in}{-0.048611in}}{\pgfqpoint{0.000000in}{0.000000in}}{%
\pgfpathmoveto{\pgfqpoint{0.000000in}{0.000000in}}%
\pgfpathlineto{\pgfqpoint{0.000000in}{-0.048611in}}%
\pgfusepath{stroke,fill}%
}%
\begin{pgfscope}%
\pgfsys@transformshift{1.018699in}{3.311111in}%
\pgfsys@useobject{currentmarker}{}%
\end{pgfscope}%
\end{pgfscope}%
\begin{pgfscope}%
\definecolor{textcolor}{rgb}{0.000000,0.000000,0.000000}%
\pgfsetstrokecolor{textcolor}%
\pgfsetfillcolor{textcolor}%
\pgftext[x=1.018699in,y=3.213889in,,top]{\color{textcolor}\sffamily\fontsize{10.000000}{12.000000}\selectfont 1}%
\end{pgfscope}%
\begin{pgfscope}%
\pgfsetbuttcap%
\pgfsetroundjoin%
\definecolor{currentfill}{rgb}{0.000000,0.000000,0.000000}%
\pgfsetfillcolor{currentfill}%
\pgfsetlinewidth{0.803000pt}%
\definecolor{currentstroke}{rgb}{0.000000,0.000000,0.000000}%
\pgfsetstrokecolor{currentstroke}%
\pgfsetdash{}{0pt}%
\pgfsys@defobject{currentmarker}{\pgfqpoint{0.000000in}{-0.048611in}}{\pgfqpoint{0.000000in}{0.000000in}}{%
\pgfpathmoveto{\pgfqpoint{0.000000in}{0.000000in}}%
\pgfpathlineto{\pgfqpoint{0.000000in}{-0.048611in}}%
\pgfusepath{stroke,fill}%
}%
\begin{pgfscope}%
\pgfsys@transformshift{1.787309in}{3.311111in}%
\pgfsys@useobject{currentmarker}{}%
\end{pgfscope}%
\end{pgfscope}%
\begin{pgfscope}%
\definecolor{textcolor}{rgb}{0.000000,0.000000,0.000000}%
\pgfsetstrokecolor{textcolor}%
\pgfsetfillcolor{textcolor}%
\pgftext[x=1.787309in,y=3.213889in,,top]{\color{textcolor}\sffamily\fontsize{10.000000}{12.000000}\selectfont 2}%
\end{pgfscope}%
\begin{pgfscope}%
\pgfsetbuttcap%
\pgfsetroundjoin%
\definecolor{currentfill}{rgb}{0.000000,0.000000,0.000000}%
\pgfsetfillcolor{currentfill}%
\pgfsetlinewidth{0.803000pt}%
\definecolor{currentstroke}{rgb}{0.000000,0.000000,0.000000}%
\pgfsetstrokecolor{currentstroke}%
\pgfsetdash{}{0pt}%
\pgfsys@defobject{currentmarker}{\pgfqpoint{0.000000in}{-0.048611in}}{\pgfqpoint{0.000000in}{0.000000in}}{%
\pgfpathmoveto{\pgfqpoint{0.000000in}{0.000000in}}%
\pgfpathlineto{\pgfqpoint{0.000000in}{-0.048611in}}%
\pgfusepath{stroke,fill}%
}%
\begin{pgfscope}%
\pgfsys@transformshift{2.555920in}{3.311111in}%
\pgfsys@useobject{currentmarker}{}%
\end{pgfscope}%
\end{pgfscope}%
\begin{pgfscope}%
\definecolor{textcolor}{rgb}{0.000000,0.000000,0.000000}%
\pgfsetstrokecolor{textcolor}%
\pgfsetfillcolor{textcolor}%
\pgftext[x=2.555920in,y=3.213889in,,top]{\color{textcolor}\sffamily\fontsize{10.000000}{12.000000}\selectfont 3}%
\end{pgfscope}%
\begin{pgfscope}%
\pgfsetbuttcap%
\pgfsetroundjoin%
\definecolor{currentfill}{rgb}{0.000000,0.000000,0.000000}%
\pgfsetfillcolor{currentfill}%
\pgfsetlinewidth{0.803000pt}%
\definecolor{currentstroke}{rgb}{0.000000,0.000000,0.000000}%
\pgfsetstrokecolor{currentstroke}%
\pgfsetdash{}{0pt}%
\pgfsys@defobject{currentmarker}{\pgfqpoint{0.000000in}{-0.048611in}}{\pgfqpoint{0.000000in}{0.000000in}}{%
\pgfpathmoveto{\pgfqpoint{0.000000in}{0.000000in}}%
\pgfpathlineto{\pgfqpoint{0.000000in}{-0.048611in}}%
\pgfusepath{stroke,fill}%
}%
\begin{pgfscope}%
\pgfsys@transformshift{3.324530in}{3.311111in}%
\pgfsys@useobject{currentmarker}{}%
\end{pgfscope}%
\end{pgfscope}%
\begin{pgfscope}%
\definecolor{textcolor}{rgb}{0.000000,0.000000,0.000000}%
\pgfsetstrokecolor{textcolor}%
\pgfsetfillcolor{textcolor}%
\pgftext[x=3.324530in,y=3.213889in,,top]{\color{textcolor}\sffamily\fontsize{10.000000}{12.000000}\selectfont 4}%
\end{pgfscope}%
\begin{pgfscope}%
\pgfsetbuttcap%
\pgfsetroundjoin%
\definecolor{currentfill}{rgb}{0.000000,0.000000,0.000000}%
\pgfsetfillcolor{currentfill}%
\pgfsetlinewidth{0.803000pt}%
\definecolor{currentstroke}{rgb}{0.000000,0.000000,0.000000}%
\pgfsetstrokecolor{currentstroke}%
\pgfsetdash{}{0pt}%
\pgfsys@defobject{currentmarker}{\pgfqpoint{-0.048611in}{0.000000in}}{\pgfqpoint{0.000000in}{0.000000in}}{%
\pgfpathmoveto{\pgfqpoint{0.000000in}{0.000000in}}%
\pgfpathlineto{\pgfqpoint{-0.048611in}{0.000000in}}%
\pgfusepath{stroke,fill}%
}%
\begin{pgfscope}%
\pgfsys@transformshift{0.467708in}{3.677425in}%
\pgfsys@useobject{currentmarker}{}%
\end{pgfscope}%
\end{pgfscope}%
\begin{pgfscope}%
\definecolor{textcolor}{rgb}{0.000000,0.000000,0.000000}%
\pgfsetstrokecolor{textcolor}%
\pgfsetfillcolor{textcolor}%
\pgftext[x=0.149607in,y=3.624664in,left,base]{\color{textcolor}\sffamily\fontsize{10.000000}{12.000000}\selectfont 0.2}%
\end{pgfscope}%
\begin{pgfscope}%
\pgfsetbuttcap%
\pgfsetroundjoin%
\definecolor{currentfill}{rgb}{0.000000,0.000000,0.000000}%
\pgfsetfillcolor{currentfill}%
\pgfsetlinewidth{0.803000pt}%
\definecolor{currentstroke}{rgb}{0.000000,0.000000,0.000000}%
\pgfsetstrokecolor{currentstroke}%
\pgfsetdash{}{0pt}%
\pgfsys@defobject{currentmarker}{\pgfqpoint{-0.048611in}{0.000000in}}{\pgfqpoint{0.000000in}{0.000000in}}{%
\pgfpathmoveto{\pgfqpoint{0.000000in}{0.000000in}}%
\pgfpathlineto{\pgfqpoint{-0.048611in}{0.000000in}}%
\pgfusepath{stroke,fill}%
}%
\begin{pgfscope}%
\pgfsys@transformshift{0.467708in}{4.140149in}%
\pgfsys@useobject{currentmarker}{}%
\end{pgfscope}%
\end{pgfscope}%
\begin{pgfscope}%
\definecolor{textcolor}{rgb}{0.000000,0.000000,0.000000}%
\pgfsetstrokecolor{textcolor}%
\pgfsetfillcolor{textcolor}%
\pgftext[x=0.149607in,y=4.087388in,left,base]{\color{textcolor}\sffamily\fontsize{10.000000}{12.000000}\selectfont 0.4}%
\end{pgfscope}%
\begin{pgfscope}%
\pgfsetbuttcap%
\pgfsetroundjoin%
\definecolor{currentfill}{rgb}{0.000000,0.000000,0.000000}%
\pgfsetfillcolor{currentfill}%
\pgfsetlinewidth{0.803000pt}%
\definecolor{currentstroke}{rgb}{0.000000,0.000000,0.000000}%
\pgfsetstrokecolor{currentstroke}%
\pgfsetdash{}{0pt}%
\pgfsys@defobject{currentmarker}{\pgfqpoint{-0.048611in}{0.000000in}}{\pgfqpoint{0.000000in}{0.000000in}}{%
\pgfpathmoveto{\pgfqpoint{0.000000in}{0.000000in}}%
\pgfpathlineto{\pgfqpoint{-0.048611in}{0.000000in}}%
\pgfusepath{stroke,fill}%
}%
\begin{pgfscope}%
\pgfsys@transformshift{0.467708in}{4.602874in}%
\pgfsys@useobject{currentmarker}{}%
\end{pgfscope}%
\end{pgfscope}%
\begin{pgfscope}%
\definecolor{textcolor}{rgb}{0.000000,0.000000,0.000000}%
\pgfsetstrokecolor{textcolor}%
\pgfsetfillcolor{textcolor}%
\pgftext[x=0.149607in,y=4.550112in,left,base]{\color{textcolor}\sffamily\fontsize{10.000000}{12.000000}\selectfont 0.6}%
\end{pgfscope}%
\begin{pgfscope}%
\pgfsetbuttcap%
\pgfsetroundjoin%
\definecolor{currentfill}{rgb}{0.000000,0.000000,0.000000}%
\pgfsetfillcolor{currentfill}%
\pgfsetlinewidth{0.803000pt}%
\definecolor{currentstroke}{rgb}{0.000000,0.000000,0.000000}%
\pgfsetstrokecolor{currentstroke}%
\pgfsetdash{}{0pt}%
\pgfsys@defobject{currentmarker}{\pgfqpoint{-0.048611in}{0.000000in}}{\pgfqpoint{0.000000in}{0.000000in}}{%
\pgfpathmoveto{\pgfqpoint{0.000000in}{0.000000in}}%
\pgfpathlineto{\pgfqpoint{-0.048611in}{0.000000in}}%
\pgfusepath{stroke,fill}%
}%
\begin{pgfscope}%
\pgfsys@transformshift{0.467708in}{5.065598in}%
\pgfsys@useobject{currentmarker}{}%
\end{pgfscope}%
\end{pgfscope}%
\begin{pgfscope}%
\definecolor{textcolor}{rgb}{0.000000,0.000000,0.000000}%
\pgfsetstrokecolor{textcolor}%
\pgfsetfillcolor{textcolor}%
\pgftext[x=0.149607in,y=5.012837in,left,base]{\color{textcolor}\sffamily\fontsize{10.000000}{12.000000}\selectfont 0.8}%
\end{pgfscope}%
\begin{pgfscope}%
\pgfsetbuttcap%
\pgfsetroundjoin%
\definecolor{currentfill}{rgb}{0.000000,0.000000,0.000000}%
\pgfsetfillcolor{currentfill}%
\pgfsetlinewidth{0.803000pt}%
\definecolor{currentstroke}{rgb}{0.000000,0.000000,0.000000}%
\pgfsetstrokecolor{currentstroke}%
\pgfsetdash{}{0pt}%
\pgfsys@defobject{currentmarker}{\pgfqpoint{-0.048611in}{0.000000in}}{\pgfqpoint{0.000000in}{0.000000in}}{%
\pgfpathmoveto{\pgfqpoint{0.000000in}{0.000000in}}%
\pgfpathlineto{\pgfqpoint{-0.048611in}{0.000000in}}%
\pgfusepath{stroke,fill}%
}%
\begin{pgfscope}%
\pgfsys@transformshift{0.467708in}{5.528323in}%
\pgfsys@useobject{currentmarker}{}%
\end{pgfscope}%
\end{pgfscope}%
\begin{pgfscope}%
\definecolor{textcolor}{rgb}{0.000000,0.000000,0.000000}%
\pgfsetstrokecolor{textcolor}%
\pgfsetfillcolor{textcolor}%
\pgftext[x=0.149607in,y=5.475561in,left,base]{\color{textcolor}\sffamily\fontsize{10.000000}{12.000000}\selectfont 1.0}%
\end{pgfscope}%
\begin{pgfscope}%
\pgfpathrectangle{\pgfqpoint{0.467708in}{3.311111in}}{\pgfqpoint{3.407812in}{2.316667in}}%
\pgfusepath{clip}%
\pgfsetrectcap%
\pgfsetroundjoin%
\pgfsetlinewidth{1.505625pt}%
\definecolor{currentstroke}{rgb}{0.121569,0.466667,0.705882}%
\pgfsetstrokecolor{currentstroke}%
\pgfsetdash{}{0pt}%
\pgfpathmoveto{\pgfqpoint{0.634394in}{5.510248in}}%
\pgfpathlineto{\pgfqpoint{0.711255in}{5.510255in}}%
\pgfpathlineto{\pgfqpoint{0.788116in}{5.510296in}}%
\pgfpathlineto{\pgfqpoint{0.864977in}{5.510439in}}%
\pgfpathlineto{\pgfqpoint{0.941838in}{5.510765in}}%
\pgfpathlineto{\pgfqpoint{1.018699in}{5.511244in}}%
\pgfpathlineto{\pgfqpoint{1.095560in}{5.511538in}}%
\pgfpathlineto{\pgfqpoint{1.172421in}{5.510904in}}%
\pgfpathlineto{\pgfqpoint{1.249282in}{5.508319in}}%
\pgfpathlineto{\pgfqpoint{1.326143in}{5.502724in}}%
\pgfpathlineto{\pgfqpoint{1.403004in}{5.493119in}}%
\pgfpathlineto{\pgfqpoint{1.479865in}{5.478417in}}%
\pgfpathlineto{\pgfqpoint{1.556726in}{5.457153in}}%
\pgfpathlineto{\pgfqpoint{1.633587in}{5.427098in}}%
\pgfpathlineto{\pgfqpoint{1.710448in}{5.384636in}}%
\pgfpathlineto{\pgfqpoint{1.787309in}{5.322362in}}%
\pgfpathlineto{\pgfqpoint{1.864171in}{5.223732in}}%
\pgfpathlineto{\pgfqpoint{1.941032in}{5.051346in}}%
\pgfpathlineto{\pgfqpoint{2.017893in}{4.763310in}}%
\pgfpathlineto{\pgfqpoint{2.094754in}{4.410966in}}%
\pgfpathlineto{\pgfqpoint{2.171615in}{4.114928in}}%
\pgfpathlineto{\pgfqpoint{2.248476in}{3.918595in}}%
\pgfpathlineto{\pgfqpoint{2.325337in}{3.794951in}}%
\pgfpathlineto{\pgfqpoint{2.402198in}{3.711070in}}%
\pgfpathlineto{\pgfqpoint{2.479059in}{3.653459in}}%
\pgfpathlineto{\pgfqpoint{2.555920in}{3.612245in}}%
\pgfpathlineto{\pgfqpoint{2.632781in}{3.579595in}}%
\pgfpathlineto{\pgfqpoint{2.709642in}{3.554014in}}%
\pgfpathlineto{\pgfqpoint{2.786503in}{3.533251in}}%
\pgfpathlineto{\pgfqpoint{2.863364in}{3.515871in}}%
\pgfpathlineto{\pgfqpoint{2.940225in}{3.501651in}}%
\pgfpathlineto{\pgfqpoint{3.017086in}{3.489267in}}%
\pgfpathlineto{\pgfqpoint{3.093947in}{3.478185in}}%
\pgfpathlineto{\pgfqpoint{3.170808in}{3.469534in}}%
\pgfpathlineto{\pgfqpoint{3.247669in}{3.461307in}}%
\pgfpathlineto{\pgfqpoint{3.324530in}{3.454282in}}%
\pgfpathlineto{\pgfqpoint{3.401391in}{3.447126in}}%
\pgfpathlineto{\pgfqpoint{3.478252in}{3.441808in}}%
\pgfpathlineto{\pgfqpoint{3.555113in}{3.436196in}}%
\pgfpathlineto{\pgfqpoint{3.631974in}{3.431844in}}%
\pgfpathlineto{\pgfqpoint{3.708835in}{3.427351in}}%
\pgfusepath{stroke}%
\end{pgfscope}%
\begin{pgfscope}%
\pgfsetrectcap%
\pgfsetmiterjoin%
\pgfsetlinewidth{0.803000pt}%
\definecolor{currentstroke}{rgb}{0.000000,0.000000,0.000000}%
\pgfsetstrokecolor{currentstroke}%
\pgfsetdash{}{0pt}%
\pgfpathmoveto{\pgfqpoint{0.467708in}{3.311111in}}%
\pgfpathlineto{\pgfqpoint{0.467708in}{5.627778in}}%
\pgfusepath{stroke}%
\end{pgfscope}%
\begin{pgfscope}%
\pgfsetrectcap%
\pgfsetmiterjoin%
\pgfsetlinewidth{0.803000pt}%
\definecolor{currentstroke}{rgb}{0.000000,0.000000,0.000000}%
\pgfsetstrokecolor{currentstroke}%
\pgfsetdash{}{0pt}%
\pgfpathmoveto{\pgfqpoint{3.875521in}{3.311111in}}%
\pgfpathlineto{\pgfqpoint{3.875521in}{5.627778in}}%
\pgfusepath{stroke}%
\end{pgfscope}%
\begin{pgfscope}%
\pgfsetrectcap%
\pgfsetmiterjoin%
\pgfsetlinewidth{0.803000pt}%
\definecolor{currentstroke}{rgb}{0.000000,0.000000,0.000000}%
\pgfsetstrokecolor{currentstroke}%
\pgfsetdash{}{0pt}%
\pgfpathmoveto{\pgfqpoint{0.467708in}{3.311111in}}%
\pgfpathlineto{\pgfqpoint{3.875521in}{3.311111in}}%
\pgfusepath{stroke}%
\end{pgfscope}%
\begin{pgfscope}%
\pgfsetrectcap%
\pgfsetmiterjoin%
\pgfsetlinewidth{0.803000pt}%
\definecolor{currentstroke}{rgb}{0.000000,0.000000,0.000000}%
\pgfsetstrokecolor{currentstroke}%
\pgfsetdash{}{0pt}%
\pgfpathmoveto{\pgfqpoint{0.467708in}{5.627778in}}%
\pgfpathlineto{\pgfqpoint{3.875521in}{5.627778in}}%
\pgfusepath{stroke}%
\end{pgfscope}%
\begin{pgfscope}%
\definecolor{textcolor}{rgb}{0.000000,0.000000,0.000000}%
\pgfsetstrokecolor{textcolor}%
\pgfsetfillcolor{textcolor}%
\pgftext[x=2.171615in,y=5.711111in,,base]{\color{textcolor}\sffamily\fontsize{12.000000}{14.400000}\selectfont \(\displaystyle  N = 16 \)}%
\end{pgfscope}%
\begin{pgfscope}%
\pgfsetbuttcap%
\pgfsetmiterjoin%
\definecolor{currentfill}{rgb}{1.000000,1.000000,1.000000}%
\pgfsetfillcolor{currentfill}%
\pgfsetlinewidth{0.000000pt}%
\definecolor{currentstroke}{rgb}{0.000000,0.000000,0.000000}%
\pgfsetstrokecolor{currentstroke}%
\pgfsetstrokeopacity{0.000000}%
\pgfsetdash{}{0pt}%
\pgfpathmoveto{\pgfqpoint{4.393576in}{3.311111in}}%
\pgfpathlineto{\pgfqpoint{7.801389in}{3.311111in}}%
\pgfpathlineto{\pgfqpoint{7.801389in}{5.627778in}}%
\pgfpathlineto{\pgfqpoint{4.393576in}{5.627778in}}%
\pgfpathclose%
\pgfusepath{fill}%
\end{pgfscope}%
\begin{pgfscope}%
\pgfpathrectangle{\pgfqpoint{4.393576in}{3.311111in}}{\pgfqpoint{3.407812in}{2.316667in}}%
\pgfusepath{clip}%
\pgfsetbuttcap%
\pgfsetroundjoin%
\definecolor{currentfill}{rgb}{0.121569,0.466667,0.705882}%
\pgfsetfillcolor{currentfill}%
\pgfsetfillopacity{0.300000}%
\pgfsetlinewidth{1.003750pt}%
\definecolor{currentstroke}{rgb}{0.121569,0.466667,0.705882}%
\pgfsetstrokecolor{currentstroke}%
\pgfsetstrokeopacity{0.300000}%
\pgfsetdash{}{0pt}%
\pgfpathmoveto{\pgfqpoint{4.560262in}{5.511198in}}%
\pgfpathlineto{\pgfqpoint{4.560262in}{5.511197in}}%
\pgfpathlineto{\pgfqpoint{4.637123in}{5.511204in}}%
\pgfpathlineto{\pgfqpoint{4.713984in}{5.511242in}}%
\pgfpathlineto{\pgfqpoint{4.790845in}{5.511360in}}%
\pgfpathlineto{\pgfqpoint{4.867706in}{5.511539in}}%
\pgfpathlineto{\pgfqpoint{4.944567in}{5.511477in}}%
\pgfpathlineto{\pgfqpoint{5.021428in}{5.510540in}}%
\pgfpathlineto{\pgfqpoint{5.098289in}{5.508060in}}%
\pgfpathlineto{\pgfqpoint{5.175150in}{5.503491in}}%
\pgfpathlineto{\pgfqpoint{5.252011in}{5.496294in}}%
\pgfpathlineto{\pgfqpoint{5.328872in}{5.485703in}}%
\pgfpathlineto{\pgfqpoint{5.405733in}{5.470715in}}%
\pgfpathlineto{\pgfqpoint{5.482595in}{5.449856in}}%
\pgfpathlineto{\pgfqpoint{5.559456in}{5.420911in}}%
\pgfpathlineto{\pgfqpoint{5.636317in}{5.380288in}}%
\pgfpathlineto{\pgfqpoint{5.713178in}{5.320970in}}%
\pgfpathlineto{\pgfqpoint{5.790039in}{5.227743in}}%
\pgfpathlineto{\pgfqpoint{5.866900in}{5.045116in}}%
\pgfpathlineto{\pgfqpoint{5.943761in}{4.560864in}}%
\pgfpathlineto{\pgfqpoint{6.020622in}{4.004546in}}%
\pgfpathlineto{\pgfqpoint{6.097483in}{3.762015in}}%
\pgfpathlineto{\pgfqpoint{6.174344in}{3.654490in}}%
\pgfpathlineto{\pgfqpoint{6.251205in}{3.593759in}}%
\pgfpathlineto{\pgfqpoint{6.328066in}{3.557798in}}%
\pgfpathlineto{\pgfqpoint{6.404927in}{3.531527in}}%
\pgfpathlineto{\pgfqpoint{6.481788in}{3.511273in}}%
\pgfpathlineto{\pgfqpoint{6.558649in}{3.496256in}}%
\pgfpathlineto{\pgfqpoint{6.635510in}{3.485249in}}%
\pgfpathlineto{\pgfqpoint{6.712371in}{3.476029in}}%
\pgfpathlineto{\pgfqpoint{6.789232in}{3.468782in}}%
\pgfpathlineto{\pgfqpoint{6.866093in}{3.461891in}}%
\pgfpathlineto{\pgfqpoint{6.942954in}{3.455643in}}%
\pgfpathlineto{\pgfqpoint{7.019815in}{3.450954in}}%
\pgfpathlineto{\pgfqpoint{7.096676in}{3.446755in}}%
\pgfpathlineto{\pgfqpoint{7.173537in}{3.442325in}}%
\pgfpathlineto{\pgfqpoint{7.250398in}{3.439423in}}%
\pgfpathlineto{\pgfqpoint{7.327259in}{3.436901in}}%
\pgfpathlineto{\pgfqpoint{7.404120in}{3.433660in}}%
\pgfpathlineto{\pgfqpoint{7.480981in}{3.431452in}}%
\pgfpathlineto{\pgfqpoint{7.557842in}{3.429243in}}%
\pgfpathlineto{\pgfqpoint{7.634703in}{3.427097in}}%
\pgfpathlineto{\pgfqpoint{7.634703in}{3.427604in}}%
\pgfpathlineto{\pgfqpoint{7.634703in}{3.427604in}}%
\pgfpathlineto{\pgfqpoint{7.557842in}{3.430069in}}%
\pgfpathlineto{\pgfqpoint{7.480981in}{3.431891in}}%
\pgfpathlineto{\pgfqpoint{7.404120in}{3.434486in}}%
\pgfpathlineto{\pgfqpoint{7.327259in}{3.437424in}}%
\pgfpathlineto{\pgfqpoint{7.250398in}{3.439842in}}%
\pgfpathlineto{\pgfqpoint{7.173537in}{3.443777in}}%
\pgfpathlineto{\pgfqpoint{7.096676in}{3.447350in}}%
\pgfpathlineto{\pgfqpoint{7.019815in}{3.451630in}}%
\pgfpathlineto{\pgfqpoint{6.942954in}{3.457064in}}%
\pgfpathlineto{\pgfqpoint{6.866093in}{3.462425in}}%
\pgfpathlineto{\pgfqpoint{6.789232in}{3.469860in}}%
\pgfpathlineto{\pgfqpoint{6.712371in}{3.476995in}}%
\pgfpathlineto{\pgfqpoint{6.635510in}{3.487753in}}%
\pgfpathlineto{\pgfqpoint{6.558649in}{3.499613in}}%
\pgfpathlineto{\pgfqpoint{6.481788in}{3.513641in}}%
\pgfpathlineto{\pgfqpoint{6.404927in}{3.533607in}}%
\pgfpathlineto{\pgfqpoint{6.328066in}{3.562090in}}%
\pgfpathlineto{\pgfqpoint{6.251205in}{3.600675in}}%
\pgfpathlineto{\pgfqpoint{6.174344in}{3.657588in}}%
\pgfpathlineto{\pgfqpoint{6.097483in}{3.771672in}}%
\pgfpathlineto{\pgfqpoint{6.020622in}{4.029582in}}%
\pgfpathlineto{\pgfqpoint{5.943761in}{4.578521in}}%
\pgfpathlineto{\pgfqpoint{5.866900in}{5.048156in}}%
\pgfpathlineto{\pgfqpoint{5.790039in}{5.229132in}}%
\pgfpathlineto{\pgfqpoint{5.713178in}{5.321626in}}%
\pgfpathlineto{\pgfqpoint{5.636317in}{5.380419in}}%
\pgfpathlineto{\pgfqpoint{5.559456in}{5.421039in}}%
\pgfpathlineto{\pgfqpoint{5.482595in}{5.449918in}}%
\pgfpathlineto{\pgfqpoint{5.405733in}{5.470761in}}%
\pgfpathlineto{\pgfqpoint{5.328872in}{5.485723in}}%
\pgfpathlineto{\pgfqpoint{5.252011in}{5.496303in}}%
\pgfpathlineto{\pgfqpoint{5.175150in}{5.503500in}}%
\pgfpathlineto{\pgfqpoint{5.098289in}{5.508061in}}%
\pgfpathlineto{\pgfqpoint{5.021428in}{5.510541in}}%
\pgfpathlineto{\pgfqpoint{4.944567in}{5.511477in}}%
\pgfpathlineto{\pgfqpoint{4.867706in}{5.511539in}}%
\pgfpathlineto{\pgfqpoint{4.790845in}{5.511360in}}%
\pgfpathlineto{\pgfqpoint{4.713984in}{5.511242in}}%
\pgfpathlineto{\pgfqpoint{4.637123in}{5.511204in}}%
\pgfpathlineto{\pgfqpoint{4.560262in}{5.511198in}}%
\pgfpathclose%
\pgfusepath{stroke,fill}%
\end{pgfscope}%
\begin{pgfscope}%
\pgfpathrectangle{\pgfqpoint{4.393576in}{3.311111in}}{\pgfqpoint{3.407812in}{2.316667in}}%
\pgfusepath{clip}%
\pgfsetbuttcap%
\pgfsetroundjoin%
\definecolor{currentfill}{rgb}{0.121569,0.466667,0.705882}%
\pgfsetfillcolor{currentfill}%
\pgfsetlinewidth{1.003750pt}%
\definecolor{currentstroke}{rgb}{0.121569,0.466667,0.705882}%
\pgfsetstrokecolor{currentstroke}%
\pgfsetdash{}{0pt}%
\pgfsys@defobject{currentmarker}{\pgfqpoint{-0.009821in}{-0.009821in}}{\pgfqpoint{0.009821in}{0.009821in}}{%
\pgfpathmoveto{\pgfqpoint{0.000000in}{-0.009821in}}%
\pgfpathcurveto{\pgfqpoint{0.002605in}{-0.009821in}}{\pgfqpoint{0.005103in}{-0.008786in}}{\pgfqpoint{0.006944in}{-0.006944in}}%
\pgfpathcurveto{\pgfqpoint{0.008786in}{-0.005103in}}{\pgfqpoint{0.009821in}{-0.002605in}}{\pgfqpoint{0.009821in}{0.000000in}}%
\pgfpathcurveto{\pgfqpoint{0.009821in}{0.002605in}}{\pgfqpoint{0.008786in}{0.005103in}}{\pgfqpoint{0.006944in}{0.006944in}}%
\pgfpathcurveto{\pgfqpoint{0.005103in}{0.008786in}}{\pgfqpoint{0.002605in}{0.009821in}}{\pgfqpoint{0.000000in}{0.009821in}}%
\pgfpathcurveto{\pgfqpoint{-0.002605in}{0.009821in}}{\pgfqpoint{-0.005103in}{0.008786in}}{\pgfqpoint{-0.006944in}{0.006944in}}%
\pgfpathcurveto{\pgfqpoint{-0.008786in}{0.005103in}}{\pgfqpoint{-0.009821in}{0.002605in}}{\pgfqpoint{-0.009821in}{0.000000in}}%
\pgfpathcurveto{\pgfqpoint{-0.009821in}{-0.002605in}}{\pgfqpoint{-0.008786in}{-0.005103in}}{\pgfqpoint{-0.006944in}{-0.006944in}}%
\pgfpathcurveto{\pgfqpoint{-0.005103in}{-0.008786in}}{\pgfqpoint{-0.002605in}{-0.009821in}}{\pgfqpoint{0.000000in}{-0.009821in}}%
\pgfpathclose%
\pgfusepath{stroke,fill}%
}%
\begin{pgfscope}%
\pgfsys@transformshift{4.560262in}{5.511198in}%
\pgfsys@useobject{currentmarker}{}%
\end{pgfscope}%
\begin{pgfscope}%
\pgfsys@transformshift{4.637123in}{5.511204in}%
\pgfsys@useobject{currentmarker}{}%
\end{pgfscope}%
\begin{pgfscope}%
\pgfsys@transformshift{4.713984in}{5.511242in}%
\pgfsys@useobject{currentmarker}{}%
\end{pgfscope}%
\begin{pgfscope}%
\pgfsys@transformshift{4.790845in}{5.511360in}%
\pgfsys@useobject{currentmarker}{}%
\end{pgfscope}%
\begin{pgfscope}%
\pgfsys@transformshift{4.867706in}{5.511539in}%
\pgfsys@useobject{currentmarker}{}%
\end{pgfscope}%
\begin{pgfscope}%
\pgfsys@transformshift{4.944567in}{5.511477in}%
\pgfsys@useobject{currentmarker}{}%
\end{pgfscope}%
\begin{pgfscope}%
\pgfsys@transformshift{5.021428in}{5.510541in}%
\pgfsys@useobject{currentmarker}{}%
\end{pgfscope}%
\begin{pgfscope}%
\pgfsys@transformshift{5.098289in}{5.508060in}%
\pgfsys@useobject{currentmarker}{}%
\end{pgfscope}%
\begin{pgfscope}%
\pgfsys@transformshift{5.175150in}{5.503495in}%
\pgfsys@useobject{currentmarker}{}%
\end{pgfscope}%
\begin{pgfscope}%
\pgfsys@transformshift{5.252011in}{5.496298in}%
\pgfsys@useobject{currentmarker}{}%
\end{pgfscope}%
\begin{pgfscope}%
\pgfsys@transformshift{5.328872in}{5.485713in}%
\pgfsys@useobject{currentmarker}{}%
\end{pgfscope}%
\begin{pgfscope}%
\pgfsys@transformshift{5.405733in}{5.470738in}%
\pgfsys@useobject{currentmarker}{}%
\end{pgfscope}%
\begin{pgfscope}%
\pgfsys@transformshift{5.482595in}{5.449887in}%
\pgfsys@useobject{currentmarker}{}%
\end{pgfscope}%
\begin{pgfscope}%
\pgfsys@transformshift{5.559456in}{5.420975in}%
\pgfsys@useobject{currentmarker}{}%
\end{pgfscope}%
\begin{pgfscope}%
\pgfsys@transformshift{5.636317in}{5.380353in}%
\pgfsys@useobject{currentmarker}{}%
\end{pgfscope}%
\begin{pgfscope}%
\pgfsys@transformshift{5.713178in}{5.321298in}%
\pgfsys@useobject{currentmarker}{}%
\end{pgfscope}%
\begin{pgfscope}%
\pgfsys@transformshift{5.790039in}{5.228438in}%
\pgfsys@useobject{currentmarker}{}%
\end{pgfscope}%
\begin{pgfscope}%
\pgfsys@transformshift{5.866900in}{5.046636in}%
\pgfsys@useobject{currentmarker}{}%
\end{pgfscope}%
\begin{pgfscope}%
\pgfsys@transformshift{5.943761in}{4.569692in}%
\pgfsys@useobject{currentmarker}{}%
\end{pgfscope}%
\begin{pgfscope}%
\pgfsys@transformshift{6.020622in}{4.017064in}%
\pgfsys@useobject{currentmarker}{}%
\end{pgfscope}%
\begin{pgfscope}%
\pgfsys@transformshift{6.097483in}{3.766843in}%
\pgfsys@useobject{currentmarker}{}%
\end{pgfscope}%
\begin{pgfscope}%
\pgfsys@transformshift{6.174344in}{3.656039in}%
\pgfsys@useobject{currentmarker}{}%
\end{pgfscope}%
\begin{pgfscope}%
\pgfsys@transformshift{6.251205in}{3.597217in}%
\pgfsys@useobject{currentmarker}{}%
\end{pgfscope}%
\begin{pgfscope}%
\pgfsys@transformshift{6.328066in}{3.559944in}%
\pgfsys@useobject{currentmarker}{}%
\end{pgfscope}%
\begin{pgfscope}%
\pgfsys@transformshift{6.404927in}{3.532567in}%
\pgfsys@useobject{currentmarker}{}%
\end{pgfscope}%
\begin{pgfscope}%
\pgfsys@transformshift{6.481788in}{3.512457in}%
\pgfsys@useobject{currentmarker}{}%
\end{pgfscope}%
\begin{pgfscope}%
\pgfsys@transformshift{6.558649in}{3.497934in}%
\pgfsys@useobject{currentmarker}{}%
\end{pgfscope}%
\begin{pgfscope}%
\pgfsys@transformshift{6.635510in}{3.486501in}%
\pgfsys@useobject{currentmarker}{}%
\end{pgfscope}%
\begin{pgfscope}%
\pgfsys@transformshift{6.712371in}{3.476512in}%
\pgfsys@useobject{currentmarker}{}%
\end{pgfscope}%
\begin{pgfscope}%
\pgfsys@transformshift{6.789232in}{3.469321in}%
\pgfsys@useobject{currentmarker}{}%
\end{pgfscope}%
\begin{pgfscope}%
\pgfsys@transformshift{6.866093in}{3.462158in}%
\pgfsys@useobject{currentmarker}{}%
\end{pgfscope}%
\begin{pgfscope}%
\pgfsys@transformshift{6.942954in}{3.456354in}%
\pgfsys@useobject{currentmarker}{}%
\end{pgfscope}%
\begin{pgfscope}%
\pgfsys@transformshift{7.019815in}{3.451292in}%
\pgfsys@useobject{currentmarker}{}%
\end{pgfscope}%
\begin{pgfscope}%
\pgfsys@transformshift{7.096676in}{3.447052in}%
\pgfsys@useobject{currentmarker}{}%
\end{pgfscope}%
\begin{pgfscope}%
\pgfsys@transformshift{7.173537in}{3.443051in}%
\pgfsys@useobject{currentmarker}{}%
\end{pgfscope}%
\begin{pgfscope}%
\pgfsys@transformshift{7.250398in}{3.439632in}%
\pgfsys@useobject{currentmarker}{}%
\end{pgfscope}%
\begin{pgfscope}%
\pgfsys@transformshift{7.327259in}{3.437163in}%
\pgfsys@useobject{currentmarker}{}%
\end{pgfscope}%
\begin{pgfscope}%
\pgfsys@transformshift{7.404120in}{3.434073in}%
\pgfsys@useobject{currentmarker}{}%
\end{pgfscope}%
\begin{pgfscope}%
\pgfsys@transformshift{7.480981in}{3.431671in}%
\pgfsys@useobject{currentmarker}{}%
\end{pgfscope}%
\begin{pgfscope}%
\pgfsys@transformshift{7.557842in}{3.429656in}%
\pgfsys@useobject{currentmarker}{}%
\end{pgfscope}%
\begin{pgfscope}%
\pgfsys@transformshift{7.634703in}{3.427350in}%
\pgfsys@useobject{currentmarker}{}%
\end{pgfscope}%
\end{pgfscope}%
\begin{pgfscope}%
\pgfsetbuttcap%
\pgfsetroundjoin%
\definecolor{currentfill}{rgb}{0.000000,0.000000,0.000000}%
\pgfsetfillcolor{currentfill}%
\pgfsetlinewidth{0.803000pt}%
\definecolor{currentstroke}{rgb}{0.000000,0.000000,0.000000}%
\pgfsetstrokecolor{currentstroke}%
\pgfsetdash{}{0pt}%
\pgfsys@defobject{currentmarker}{\pgfqpoint{0.000000in}{-0.048611in}}{\pgfqpoint{0.000000in}{0.000000in}}{%
\pgfpathmoveto{\pgfqpoint{0.000000in}{0.000000in}}%
\pgfpathlineto{\pgfqpoint{0.000000in}{-0.048611in}}%
\pgfusepath{stroke,fill}%
}%
\begin{pgfscope}%
\pgfsys@transformshift{4.944567in}{3.311111in}%
\pgfsys@useobject{currentmarker}{}%
\end{pgfscope}%
\end{pgfscope}%
\begin{pgfscope}%
\definecolor{textcolor}{rgb}{0.000000,0.000000,0.000000}%
\pgfsetstrokecolor{textcolor}%
\pgfsetfillcolor{textcolor}%
\pgftext[x=4.944567in,y=3.213889in,,top]{\color{textcolor}\sffamily\fontsize{10.000000}{12.000000}\selectfont 1}%
\end{pgfscope}%
\begin{pgfscope}%
\pgfsetbuttcap%
\pgfsetroundjoin%
\definecolor{currentfill}{rgb}{0.000000,0.000000,0.000000}%
\pgfsetfillcolor{currentfill}%
\pgfsetlinewidth{0.803000pt}%
\definecolor{currentstroke}{rgb}{0.000000,0.000000,0.000000}%
\pgfsetstrokecolor{currentstroke}%
\pgfsetdash{}{0pt}%
\pgfsys@defobject{currentmarker}{\pgfqpoint{0.000000in}{-0.048611in}}{\pgfqpoint{0.000000in}{0.000000in}}{%
\pgfpathmoveto{\pgfqpoint{0.000000in}{0.000000in}}%
\pgfpathlineto{\pgfqpoint{0.000000in}{-0.048611in}}%
\pgfusepath{stroke,fill}%
}%
\begin{pgfscope}%
\pgfsys@transformshift{5.713178in}{3.311111in}%
\pgfsys@useobject{currentmarker}{}%
\end{pgfscope}%
\end{pgfscope}%
\begin{pgfscope}%
\definecolor{textcolor}{rgb}{0.000000,0.000000,0.000000}%
\pgfsetstrokecolor{textcolor}%
\pgfsetfillcolor{textcolor}%
\pgftext[x=5.713178in,y=3.213889in,,top]{\color{textcolor}\sffamily\fontsize{10.000000}{12.000000}\selectfont 2}%
\end{pgfscope}%
\begin{pgfscope}%
\pgfsetbuttcap%
\pgfsetroundjoin%
\definecolor{currentfill}{rgb}{0.000000,0.000000,0.000000}%
\pgfsetfillcolor{currentfill}%
\pgfsetlinewidth{0.803000pt}%
\definecolor{currentstroke}{rgb}{0.000000,0.000000,0.000000}%
\pgfsetstrokecolor{currentstroke}%
\pgfsetdash{}{0pt}%
\pgfsys@defobject{currentmarker}{\pgfqpoint{0.000000in}{-0.048611in}}{\pgfqpoint{0.000000in}{0.000000in}}{%
\pgfpathmoveto{\pgfqpoint{0.000000in}{0.000000in}}%
\pgfpathlineto{\pgfqpoint{0.000000in}{-0.048611in}}%
\pgfusepath{stroke,fill}%
}%
\begin{pgfscope}%
\pgfsys@transformshift{6.481788in}{3.311111in}%
\pgfsys@useobject{currentmarker}{}%
\end{pgfscope}%
\end{pgfscope}%
\begin{pgfscope}%
\definecolor{textcolor}{rgb}{0.000000,0.000000,0.000000}%
\pgfsetstrokecolor{textcolor}%
\pgfsetfillcolor{textcolor}%
\pgftext[x=6.481788in,y=3.213889in,,top]{\color{textcolor}\sffamily\fontsize{10.000000}{12.000000}\selectfont 3}%
\end{pgfscope}%
\begin{pgfscope}%
\pgfsetbuttcap%
\pgfsetroundjoin%
\definecolor{currentfill}{rgb}{0.000000,0.000000,0.000000}%
\pgfsetfillcolor{currentfill}%
\pgfsetlinewidth{0.803000pt}%
\definecolor{currentstroke}{rgb}{0.000000,0.000000,0.000000}%
\pgfsetstrokecolor{currentstroke}%
\pgfsetdash{}{0pt}%
\pgfsys@defobject{currentmarker}{\pgfqpoint{0.000000in}{-0.048611in}}{\pgfqpoint{0.000000in}{0.000000in}}{%
\pgfpathmoveto{\pgfqpoint{0.000000in}{0.000000in}}%
\pgfpathlineto{\pgfqpoint{0.000000in}{-0.048611in}}%
\pgfusepath{stroke,fill}%
}%
\begin{pgfscope}%
\pgfsys@transformshift{7.250398in}{3.311111in}%
\pgfsys@useobject{currentmarker}{}%
\end{pgfscope}%
\end{pgfscope}%
\begin{pgfscope}%
\definecolor{textcolor}{rgb}{0.000000,0.000000,0.000000}%
\pgfsetstrokecolor{textcolor}%
\pgfsetfillcolor{textcolor}%
\pgftext[x=7.250398in,y=3.213889in,,top]{\color{textcolor}\sffamily\fontsize{10.000000}{12.000000}\selectfont 4}%
\end{pgfscope}%
\begin{pgfscope}%
\pgfsetbuttcap%
\pgfsetroundjoin%
\definecolor{currentfill}{rgb}{0.000000,0.000000,0.000000}%
\pgfsetfillcolor{currentfill}%
\pgfsetlinewidth{0.803000pt}%
\definecolor{currentstroke}{rgb}{0.000000,0.000000,0.000000}%
\pgfsetstrokecolor{currentstroke}%
\pgfsetdash{}{0pt}%
\pgfsys@defobject{currentmarker}{\pgfqpoint{-0.048611in}{0.000000in}}{\pgfqpoint{0.000000in}{0.000000in}}{%
\pgfpathmoveto{\pgfqpoint{0.000000in}{0.000000in}}%
\pgfpathlineto{\pgfqpoint{-0.048611in}{0.000000in}}%
\pgfusepath{stroke,fill}%
}%
\begin{pgfscope}%
\pgfsys@transformshift{4.393576in}{3.327014in}%
\pgfsys@useobject{currentmarker}{}%
\end{pgfscope}%
\end{pgfscope}%
\begin{pgfscope}%
\definecolor{textcolor}{rgb}{0.000000,0.000000,0.000000}%
\pgfsetstrokecolor{textcolor}%
\pgfsetfillcolor{textcolor}%
\pgftext[x=4.075475in,y=3.274252in,left,base]{\color{textcolor}\sffamily\fontsize{10.000000}{12.000000}\selectfont 0.0}%
\end{pgfscope}%
\begin{pgfscope}%
\pgfsetbuttcap%
\pgfsetroundjoin%
\definecolor{currentfill}{rgb}{0.000000,0.000000,0.000000}%
\pgfsetfillcolor{currentfill}%
\pgfsetlinewidth{0.803000pt}%
\definecolor{currentstroke}{rgb}{0.000000,0.000000,0.000000}%
\pgfsetstrokecolor{currentstroke}%
\pgfsetdash{}{0pt}%
\pgfsys@defobject{currentmarker}{\pgfqpoint{-0.048611in}{0.000000in}}{\pgfqpoint{0.000000in}{0.000000in}}{%
\pgfpathmoveto{\pgfqpoint{0.000000in}{0.000000in}}%
\pgfpathlineto{\pgfqpoint{-0.048611in}{0.000000in}}%
\pgfusepath{stroke,fill}%
}%
\begin{pgfscope}%
\pgfsys@transformshift{4.393576in}{3.764705in}%
\pgfsys@useobject{currentmarker}{}%
\end{pgfscope}%
\end{pgfscope}%
\begin{pgfscope}%
\definecolor{textcolor}{rgb}{0.000000,0.000000,0.000000}%
\pgfsetstrokecolor{textcolor}%
\pgfsetfillcolor{textcolor}%
\pgftext[x=4.075475in,y=3.711944in,left,base]{\color{textcolor}\sffamily\fontsize{10.000000}{12.000000}\selectfont 0.2}%
\end{pgfscope}%
\begin{pgfscope}%
\pgfsetbuttcap%
\pgfsetroundjoin%
\definecolor{currentfill}{rgb}{0.000000,0.000000,0.000000}%
\pgfsetfillcolor{currentfill}%
\pgfsetlinewidth{0.803000pt}%
\definecolor{currentstroke}{rgb}{0.000000,0.000000,0.000000}%
\pgfsetstrokecolor{currentstroke}%
\pgfsetdash{}{0pt}%
\pgfsys@defobject{currentmarker}{\pgfqpoint{-0.048611in}{0.000000in}}{\pgfqpoint{0.000000in}{0.000000in}}{%
\pgfpathmoveto{\pgfqpoint{0.000000in}{0.000000in}}%
\pgfpathlineto{\pgfqpoint{-0.048611in}{0.000000in}}%
\pgfusepath{stroke,fill}%
}%
\begin{pgfscope}%
\pgfsys@transformshift{4.393576in}{4.202397in}%
\pgfsys@useobject{currentmarker}{}%
\end{pgfscope}%
\end{pgfscope}%
\begin{pgfscope}%
\definecolor{textcolor}{rgb}{0.000000,0.000000,0.000000}%
\pgfsetstrokecolor{textcolor}%
\pgfsetfillcolor{textcolor}%
\pgftext[x=4.075475in,y=4.149635in,left,base]{\color{textcolor}\sffamily\fontsize{10.000000}{12.000000}\selectfont 0.4}%
\end{pgfscope}%
\begin{pgfscope}%
\pgfsetbuttcap%
\pgfsetroundjoin%
\definecolor{currentfill}{rgb}{0.000000,0.000000,0.000000}%
\pgfsetfillcolor{currentfill}%
\pgfsetlinewidth{0.803000pt}%
\definecolor{currentstroke}{rgb}{0.000000,0.000000,0.000000}%
\pgfsetstrokecolor{currentstroke}%
\pgfsetdash{}{0pt}%
\pgfsys@defobject{currentmarker}{\pgfqpoint{-0.048611in}{0.000000in}}{\pgfqpoint{0.000000in}{0.000000in}}{%
\pgfpathmoveto{\pgfqpoint{0.000000in}{0.000000in}}%
\pgfpathlineto{\pgfqpoint{-0.048611in}{0.000000in}}%
\pgfusepath{stroke,fill}%
}%
\begin{pgfscope}%
\pgfsys@transformshift{4.393576in}{4.640088in}%
\pgfsys@useobject{currentmarker}{}%
\end{pgfscope}%
\end{pgfscope}%
\begin{pgfscope}%
\definecolor{textcolor}{rgb}{0.000000,0.000000,0.000000}%
\pgfsetstrokecolor{textcolor}%
\pgfsetfillcolor{textcolor}%
\pgftext[x=4.075475in,y=4.587327in,left,base]{\color{textcolor}\sffamily\fontsize{10.000000}{12.000000}\selectfont 0.6}%
\end{pgfscope}%
\begin{pgfscope}%
\pgfsetbuttcap%
\pgfsetroundjoin%
\definecolor{currentfill}{rgb}{0.000000,0.000000,0.000000}%
\pgfsetfillcolor{currentfill}%
\pgfsetlinewidth{0.803000pt}%
\definecolor{currentstroke}{rgb}{0.000000,0.000000,0.000000}%
\pgfsetstrokecolor{currentstroke}%
\pgfsetdash{}{0pt}%
\pgfsys@defobject{currentmarker}{\pgfqpoint{-0.048611in}{0.000000in}}{\pgfqpoint{0.000000in}{0.000000in}}{%
\pgfpathmoveto{\pgfqpoint{0.000000in}{0.000000in}}%
\pgfpathlineto{\pgfqpoint{-0.048611in}{0.000000in}}%
\pgfusepath{stroke,fill}%
}%
\begin{pgfscope}%
\pgfsys@transformshift{4.393576in}{5.077780in}%
\pgfsys@useobject{currentmarker}{}%
\end{pgfscope}%
\end{pgfscope}%
\begin{pgfscope}%
\definecolor{textcolor}{rgb}{0.000000,0.000000,0.000000}%
\pgfsetstrokecolor{textcolor}%
\pgfsetfillcolor{textcolor}%
\pgftext[x=4.075475in,y=5.025018in,left,base]{\color{textcolor}\sffamily\fontsize{10.000000}{12.000000}\selectfont 0.8}%
\end{pgfscope}%
\begin{pgfscope}%
\pgfsetbuttcap%
\pgfsetroundjoin%
\definecolor{currentfill}{rgb}{0.000000,0.000000,0.000000}%
\pgfsetfillcolor{currentfill}%
\pgfsetlinewidth{0.803000pt}%
\definecolor{currentstroke}{rgb}{0.000000,0.000000,0.000000}%
\pgfsetstrokecolor{currentstroke}%
\pgfsetdash{}{0pt}%
\pgfsys@defobject{currentmarker}{\pgfqpoint{-0.048611in}{0.000000in}}{\pgfqpoint{0.000000in}{0.000000in}}{%
\pgfpathmoveto{\pgfqpoint{0.000000in}{0.000000in}}%
\pgfpathlineto{\pgfqpoint{-0.048611in}{0.000000in}}%
\pgfusepath{stroke,fill}%
}%
\begin{pgfscope}%
\pgfsys@transformshift{4.393576in}{5.515471in}%
\pgfsys@useobject{currentmarker}{}%
\end{pgfscope}%
\end{pgfscope}%
\begin{pgfscope}%
\definecolor{textcolor}{rgb}{0.000000,0.000000,0.000000}%
\pgfsetstrokecolor{textcolor}%
\pgfsetfillcolor{textcolor}%
\pgftext[x=4.075475in,y=5.462710in,left,base]{\color{textcolor}\sffamily\fontsize{10.000000}{12.000000}\selectfont 1.0}%
\end{pgfscope}%
\begin{pgfscope}%
\pgfpathrectangle{\pgfqpoint{4.393576in}{3.311111in}}{\pgfqpoint{3.407812in}{2.316667in}}%
\pgfusepath{clip}%
\pgfsetrectcap%
\pgfsetroundjoin%
\pgfsetlinewidth{1.505625pt}%
\definecolor{currentstroke}{rgb}{0.121569,0.466667,0.705882}%
\pgfsetstrokecolor{currentstroke}%
\pgfsetdash{}{0pt}%
\pgfpathmoveto{\pgfqpoint{4.560262in}{5.511198in}}%
\pgfpathlineto{\pgfqpoint{4.637123in}{5.511204in}}%
\pgfpathlineto{\pgfqpoint{4.713984in}{5.511242in}}%
\pgfpathlineto{\pgfqpoint{4.790845in}{5.511360in}}%
\pgfpathlineto{\pgfqpoint{4.867706in}{5.511539in}}%
\pgfpathlineto{\pgfqpoint{4.944567in}{5.511477in}}%
\pgfpathlineto{\pgfqpoint{5.021428in}{5.510541in}}%
\pgfpathlineto{\pgfqpoint{5.098289in}{5.508060in}}%
\pgfpathlineto{\pgfqpoint{5.175150in}{5.503495in}}%
\pgfpathlineto{\pgfqpoint{5.252011in}{5.496298in}}%
\pgfpathlineto{\pgfqpoint{5.328872in}{5.485713in}}%
\pgfpathlineto{\pgfqpoint{5.405733in}{5.470738in}}%
\pgfpathlineto{\pgfqpoint{5.482595in}{5.449887in}}%
\pgfpathlineto{\pgfqpoint{5.559456in}{5.420975in}}%
\pgfpathlineto{\pgfqpoint{5.636317in}{5.380353in}}%
\pgfpathlineto{\pgfqpoint{5.713178in}{5.321298in}}%
\pgfpathlineto{\pgfqpoint{5.790039in}{5.228438in}}%
\pgfpathlineto{\pgfqpoint{5.866900in}{5.046636in}}%
\pgfpathlineto{\pgfqpoint{5.943761in}{4.569692in}}%
\pgfpathlineto{\pgfqpoint{6.020622in}{4.017064in}}%
\pgfpathlineto{\pgfqpoint{6.097483in}{3.766843in}}%
\pgfpathlineto{\pgfqpoint{6.174344in}{3.656039in}}%
\pgfpathlineto{\pgfqpoint{6.251205in}{3.597217in}}%
\pgfpathlineto{\pgfqpoint{6.328066in}{3.559944in}}%
\pgfpathlineto{\pgfqpoint{6.404927in}{3.532567in}}%
\pgfpathlineto{\pgfqpoint{6.481788in}{3.512457in}}%
\pgfpathlineto{\pgfqpoint{6.558649in}{3.497934in}}%
\pgfpathlineto{\pgfqpoint{6.635510in}{3.486501in}}%
\pgfpathlineto{\pgfqpoint{6.712371in}{3.476512in}}%
\pgfpathlineto{\pgfqpoint{6.789232in}{3.469321in}}%
\pgfpathlineto{\pgfqpoint{6.866093in}{3.462158in}}%
\pgfpathlineto{\pgfqpoint{6.942954in}{3.456354in}}%
\pgfpathlineto{\pgfqpoint{7.019815in}{3.451292in}}%
\pgfpathlineto{\pgfqpoint{7.096676in}{3.447052in}}%
\pgfpathlineto{\pgfqpoint{7.173537in}{3.443051in}}%
\pgfpathlineto{\pgfqpoint{7.250398in}{3.439632in}}%
\pgfpathlineto{\pgfqpoint{7.327259in}{3.437163in}}%
\pgfpathlineto{\pgfqpoint{7.404120in}{3.434073in}}%
\pgfpathlineto{\pgfqpoint{7.480981in}{3.431671in}}%
\pgfpathlineto{\pgfqpoint{7.557842in}{3.429656in}}%
\pgfpathlineto{\pgfqpoint{7.634703in}{3.427350in}}%
\pgfusepath{stroke}%
\end{pgfscope}%
\begin{pgfscope}%
\pgfsetrectcap%
\pgfsetmiterjoin%
\pgfsetlinewidth{0.803000pt}%
\definecolor{currentstroke}{rgb}{0.000000,0.000000,0.000000}%
\pgfsetstrokecolor{currentstroke}%
\pgfsetdash{}{0pt}%
\pgfpathmoveto{\pgfqpoint{4.393576in}{3.311111in}}%
\pgfpathlineto{\pgfqpoint{4.393576in}{5.627778in}}%
\pgfusepath{stroke}%
\end{pgfscope}%
\begin{pgfscope}%
\pgfsetrectcap%
\pgfsetmiterjoin%
\pgfsetlinewidth{0.803000pt}%
\definecolor{currentstroke}{rgb}{0.000000,0.000000,0.000000}%
\pgfsetstrokecolor{currentstroke}%
\pgfsetdash{}{0pt}%
\pgfpathmoveto{\pgfqpoint{7.801389in}{3.311111in}}%
\pgfpathlineto{\pgfqpoint{7.801389in}{5.627778in}}%
\pgfusepath{stroke}%
\end{pgfscope}%
\begin{pgfscope}%
\pgfsetrectcap%
\pgfsetmiterjoin%
\pgfsetlinewidth{0.803000pt}%
\definecolor{currentstroke}{rgb}{0.000000,0.000000,0.000000}%
\pgfsetstrokecolor{currentstroke}%
\pgfsetdash{}{0pt}%
\pgfpathmoveto{\pgfqpoint{4.393576in}{3.311111in}}%
\pgfpathlineto{\pgfqpoint{7.801389in}{3.311111in}}%
\pgfusepath{stroke}%
\end{pgfscope}%
\begin{pgfscope}%
\pgfsetrectcap%
\pgfsetmiterjoin%
\pgfsetlinewidth{0.803000pt}%
\definecolor{currentstroke}{rgb}{0.000000,0.000000,0.000000}%
\pgfsetstrokecolor{currentstroke}%
\pgfsetdash{}{0pt}%
\pgfpathmoveto{\pgfqpoint{4.393576in}{5.627778in}}%
\pgfpathlineto{\pgfqpoint{7.801389in}{5.627778in}}%
\pgfusepath{stroke}%
\end{pgfscope}%
\begin{pgfscope}%
\definecolor{textcolor}{rgb}{0.000000,0.000000,0.000000}%
\pgfsetstrokecolor{textcolor}%
\pgfsetfillcolor{textcolor}%
\pgftext[x=6.097483in,y=5.711111in,,base]{\color{textcolor}\sffamily\fontsize{12.000000}{14.400000}\selectfont \(\displaystyle  N = 32 \)}%
\end{pgfscope}%
\begin{pgfscope}%
\pgfsetbuttcap%
\pgfsetmiterjoin%
\definecolor{currentfill}{rgb}{1.000000,1.000000,1.000000}%
\pgfsetfillcolor{currentfill}%
\pgfsetlinewidth{0.000000pt}%
\definecolor{currentstroke}{rgb}{0.000000,0.000000,0.000000}%
\pgfsetstrokecolor{currentstroke}%
\pgfsetstrokeopacity{0.000000}%
\pgfsetdash{}{0pt}%
\pgfpathmoveto{\pgfqpoint{0.467708in}{0.386111in}}%
\pgfpathlineto{\pgfqpoint{3.875521in}{0.386111in}}%
\pgfpathlineto{\pgfqpoint{3.875521in}{2.702778in}}%
\pgfpathlineto{\pgfqpoint{0.467708in}{2.702778in}}%
\pgfpathclose%
\pgfusepath{fill}%
\end{pgfscope}%
\begin{pgfscope}%
\pgfpathrectangle{\pgfqpoint{0.467708in}{0.386111in}}{\pgfqpoint{3.407812in}{2.316667in}}%
\pgfusepath{clip}%
\pgfsetbuttcap%
\pgfsetroundjoin%
\definecolor{currentfill}{rgb}{0.121569,0.466667,0.705882}%
\pgfsetfillcolor{currentfill}%
\pgfsetfillopacity{0.300000}%
\pgfsetlinewidth{1.003750pt}%
\definecolor{currentstroke}{rgb}{0.121569,0.466667,0.705882}%
\pgfsetstrokecolor{currentstroke}%
\pgfsetstrokeopacity{0.300000}%
\pgfsetdash{}{0pt}%
\pgfpathmoveto{\pgfqpoint{0.634394in}{2.586440in}}%
\pgfpathlineto{\pgfqpoint{0.634394in}{2.586440in}}%
\pgfpathlineto{\pgfqpoint{0.711255in}{2.586446in}}%
\pgfpathlineto{\pgfqpoint{0.788116in}{2.586479in}}%
\pgfpathlineto{\pgfqpoint{0.864977in}{2.586538in}}%
\pgfpathlineto{\pgfqpoint{0.941838in}{2.586427in}}%
\pgfpathlineto{\pgfqpoint{1.018699in}{2.585728in}}%
\pgfpathlineto{\pgfqpoint{1.095560in}{2.584067in}}%
\pgfpathlineto{\pgfqpoint{1.172421in}{2.581081in}}%
\pgfpathlineto{\pgfqpoint{1.249282in}{2.576290in}}%
\pgfpathlineto{\pgfqpoint{1.326143in}{2.569075in}}%
\pgfpathlineto{\pgfqpoint{1.403004in}{2.558649in}}%
\pgfpathlineto{\pgfqpoint{1.479865in}{2.543947in}}%
\pgfpathlineto{\pgfqpoint{1.556726in}{2.523579in}}%
\pgfpathlineto{\pgfqpoint{1.633587in}{2.495385in}}%
\pgfpathlineto{\pgfqpoint{1.710448in}{2.455714in}}%
\pgfpathlineto{\pgfqpoint{1.787309in}{2.398063in}}%
\pgfpathlineto{\pgfqpoint{1.864171in}{2.306070in}}%
\pgfpathlineto{\pgfqpoint{1.941032in}{2.124236in}}%
\pgfpathlineto{\pgfqpoint{2.017893in}{1.390759in}}%
\pgfpathlineto{\pgfqpoint{2.094754in}{0.790477in}}%
\pgfpathlineto{\pgfqpoint{2.171615in}{0.658267in}}%
\pgfpathlineto{\pgfqpoint{2.248476in}{0.607624in}}%
\pgfpathlineto{\pgfqpoint{2.325337in}{0.581520in}}%
\pgfpathlineto{\pgfqpoint{2.402198in}{0.564958in}}%
\pgfpathlineto{\pgfqpoint{2.479059in}{0.551723in}}%
\pgfpathlineto{\pgfqpoint{2.555920in}{0.543293in}}%
\pgfpathlineto{\pgfqpoint{2.632781in}{0.536047in}}%
\pgfpathlineto{\pgfqpoint{2.709642in}{0.530656in}}%
\pgfpathlineto{\pgfqpoint{2.786503in}{0.525702in}}%
\pgfpathlineto{\pgfqpoint{2.863364in}{0.522058in}}%
\pgfpathlineto{\pgfqpoint{2.940225in}{0.518228in}}%
\pgfpathlineto{\pgfqpoint{3.017086in}{0.515785in}}%
\pgfpathlineto{\pgfqpoint{3.093947in}{0.513476in}}%
\pgfpathlineto{\pgfqpoint{3.170808in}{0.511403in}}%
\pgfpathlineto{\pgfqpoint{3.247669in}{0.509932in}}%
\pgfpathlineto{\pgfqpoint{3.324530in}{0.508254in}}%
\pgfpathlineto{\pgfqpoint{3.401391in}{0.506305in}}%
\pgfpathlineto{\pgfqpoint{3.478252in}{0.505273in}}%
\pgfpathlineto{\pgfqpoint{3.555113in}{0.503796in}}%
\pgfpathlineto{\pgfqpoint{3.631974in}{0.502742in}}%
\pgfpathlineto{\pgfqpoint{3.708835in}{0.501990in}}%
\pgfpathlineto{\pgfqpoint{3.708835in}{0.502711in}}%
\pgfpathlineto{\pgfqpoint{3.708835in}{0.502711in}}%
\pgfpathlineto{\pgfqpoint{3.631974in}{0.503857in}}%
\pgfpathlineto{\pgfqpoint{3.555113in}{0.504593in}}%
\pgfpathlineto{\pgfqpoint{3.478252in}{0.505850in}}%
\pgfpathlineto{\pgfqpoint{3.401391in}{0.507801in}}%
\pgfpathlineto{\pgfqpoint{3.324530in}{0.508882in}}%
\pgfpathlineto{\pgfqpoint{3.247669in}{0.510671in}}%
\pgfpathlineto{\pgfqpoint{3.170808in}{0.512602in}}%
\pgfpathlineto{\pgfqpoint{3.093947in}{0.514136in}}%
\pgfpathlineto{\pgfqpoint{3.017086in}{0.516788in}}%
\pgfpathlineto{\pgfqpoint{2.940225in}{0.520259in}}%
\pgfpathlineto{\pgfqpoint{2.863364in}{0.522964in}}%
\pgfpathlineto{\pgfqpoint{2.786503in}{0.526859in}}%
\pgfpathlineto{\pgfqpoint{2.709642in}{0.531214in}}%
\pgfpathlineto{\pgfqpoint{2.632781in}{0.537919in}}%
\pgfpathlineto{\pgfqpoint{2.555920in}{0.544596in}}%
\pgfpathlineto{\pgfqpoint{2.479059in}{0.552987in}}%
\pgfpathlineto{\pgfqpoint{2.402198in}{0.568274in}}%
\pgfpathlineto{\pgfqpoint{2.325337in}{0.586575in}}%
\pgfpathlineto{\pgfqpoint{2.248476in}{0.616680in}}%
\pgfpathlineto{\pgfqpoint{2.171615in}{0.668821in}}%
\pgfpathlineto{\pgfqpoint{2.094754in}{0.794231in}}%
\pgfpathlineto{\pgfqpoint{2.017893in}{1.427363in}}%
\pgfpathlineto{\pgfqpoint{1.941032in}{2.128677in}}%
\pgfpathlineto{\pgfqpoint{1.864171in}{2.308432in}}%
\pgfpathlineto{\pgfqpoint{1.787309in}{2.398366in}}%
\pgfpathlineto{\pgfqpoint{1.710448in}{2.455939in}}%
\pgfpathlineto{\pgfqpoint{1.633587in}{2.495463in}}%
\pgfpathlineto{\pgfqpoint{1.556726in}{2.523665in}}%
\pgfpathlineto{\pgfqpoint{1.479865in}{2.543990in}}%
\pgfpathlineto{\pgfqpoint{1.403004in}{2.558662in}}%
\pgfpathlineto{\pgfqpoint{1.326143in}{2.569090in}}%
\pgfpathlineto{\pgfqpoint{1.249282in}{2.576300in}}%
\pgfpathlineto{\pgfqpoint{1.172421in}{2.581083in}}%
\pgfpathlineto{\pgfqpoint{1.095560in}{2.584069in}}%
\pgfpathlineto{\pgfqpoint{1.018699in}{2.585728in}}%
\pgfpathlineto{\pgfqpoint{0.941838in}{2.586427in}}%
\pgfpathlineto{\pgfqpoint{0.864977in}{2.586538in}}%
\pgfpathlineto{\pgfqpoint{0.788116in}{2.586479in}}%
\pgfpathlineto{\pgfqpoint{0.711255in}{2.586446in}}%
\pgfpathlineto{\pgfqpoint{0.634394in}{2.586440in}}%
\pgfpathclose%
\pgfusepath{stroke,fill}%
\end{pgfscope}%
\begin{pgfscope}%
\pgfpathrectangle{\pgfqpoint{0.467708in}{0.386111in}}{\pgfqpoint{3.407812in}{2.316667in}}%
\pgfusepath{clip}%
\pgfsetbuttcap%
\pgfsetroundjoin%
\definecolor{currentfill}{rgb}{0.121569,0.466667,0.705882}%
\pgfsetfillcolor{currentfill}%
\pgfsetlinewidth{1.003750pt}%
\definecolor{currentstroke}{rgb}{0.121569,0.466667,0.705882}%
\pgfsetstrokecolor{currentstroke}%
\pgfsetdash{}{0pt}%
\pgfsys@defobject{currentmarker}{\pgfqpoint{-0.009821in}{-0.009821in}}{\pgfqpoint{0.009821in}{0.009821in}}{%
\pgfpathmoveto{\pgfqpoint{0.000000in}{-0.009821in}}%
\pgfpathcurveto{\pgfqpoint{0.002605in}{-0.009821in}}{\pgfqpoint{0.005103in}{-0.008786in}}{\pgfqpoint{0.006944in}{-0.006944in}}%
\pgfpathcurveto{\pgfqpoint{0.008786in}{-0.005103in}}{\pgfqpoint{0.009821in}{-0.002605in}}{\pgfqpoint{0.009821in}{0.000000in}}%
\pgfpathcurveto{\pgfqpoint{0.009821in}{0.002605in}}{\pgfqpoint{0.008786in}{0.005103in}}{\pgfqpoint{0.006944in}{0.006944in}}%
\pgfpathcurveto{\pgfqpoint{0.005103in}{0.008786in}}{\pgfqpoint{0.002605in}{0.009821in}}{\pgfqpoint{0.000000in}{0.009821in}}%
\pgfpathcurveto{\pgfqpoint{-0.002605in}{0.009821in}}{\pgfqpoint{-0.005103in}{0.008786in}}{\pgfqpoint{-0.006944in}{0.006944in}}%
\pgfpathcurveto{\pgfqpoint{-0.008786in}{0.005103in}}{\pgfqpoint{-0.009821in}{0.002605in}}{\pgfqpoint{-0.009821in}{0.000000in}}%
\pgfpathcurveto{\pgfqpoint{-0.009821in}{-0.002605in}}{\pgfqpoint{-0.008786in}{-0.005103in}}{\pgfqpoint{-0.006944in}{-0.006944in}}%
\pgfpathcurveto{\pgfqpoint{-0.005103in}{-0.008786in}}{\pgfqpoint{-0.002605in}{-0.009821in}}{\pgfqpoint{0.000000in}{-0.009821in}}%
\pgfpathclose%
\pgfusepath{stroke,fill}%
}%
\begin{pgfscope}%
\pgfsys@transformshift{0.634394in}{2.586440in}%
\pgfsys@useobject{currentmarker}{}%
\end{pgfscope}%
\begin{pgfscope}%
\pgfsys@transformshift{0.711255in}{2.586446in}%
\pgfsys@useobject{currentmarker}{}%
\end{pgfscope}%
\begin{pgfscope}%
\pgfsys@transformshift{0.788116in}{2.586479in}%
\pgfsys@useobject{currentmarker}{}%
\end{pgfscope}%
\begin{pgfscope}%
\pgfsys@transformshift{0.864977in}{2.586538in}%
\pgfsys@useobject{currentmarker}{}%
\end{pgfscope}%
\begin{pgfscope}%
\pgfsys@transformshift{0.941838in}{2.586427in}%
\pgfsys@useobject{currentmarker}{}%
\end{pgfscope}%
\begin{pgfscope}%
\pgfsys@transformshift{1.018699in}{2.585728in}%
\pgfsys@useobject{currentmarker}{}%
\end{pgfscope}%
\begin{pgfscope}%
\pgfsys@transformshift{1.095560in}{2.584068in}%
\pgfsys@useobject{currentmarker}{}%
\end{pgfscope}%
\begin{pgfscope}%
\pgfsys@transformshift{1.172421in}{2.581082in}%
\pgfsys@useobject{currentmarker}{}%
\end{pgfscope}%
\begin{pgfscope}%
\pgfsys@transformshift{1.249282in}{2.576295in}%
\pgfsys@useobject{currentmarker}{}%
\end{pgfscope}%
\begin{pgfscope}%
\pgfsys@transformshift{1.326143in}{2.569082in}%
\pgfsys@useobject{currentmarker}{}%
\end{pgfscope}%
\begin{pgfscope}%
\pgfsys@transformshift{1.403004in}{2.558656in}%
\pgfsys@useobject{currentmarker}{}%
\end{pgfscope}%
\begin{pgfscope}%
\pgfsys@transformshift{1.479865in}{2.543968in}%
\pgfsys@useobject{currentmarker}{}%
\end{pgfscope}%
\begin{pgfscope}%
\pgfsys@transformshift{1.556726in}{2.523622in}%
\pgfsys@useobject{currentmarker}{}%
\end{pgfscope}%
\begin{pgfscope}%
\pgfsys@transformshift{1.633587in}{2.495424in}%
\pgfsys@useobject{currentmarker}{}%
\end{pgfscope}%
\begin{pgfscope}%
\pgfsys@transformshift{1.710448in}{2.455826in}%
\pgfsys@useobject{currentmarker}{}%
\end{pgfscope}%
\begin{pgfscope}%
\pgfsys@transformshift{1.787309in}{2.398215in}%
\pgfsys@useobject{currentmarker}{}%
\end{pgfscope}%
\begin{pgfscope}%
\pgfsys@transformshift{1.864171in}{2.307251in}%
\pgfsys@useobject{currentmarker}{}%
\end{pgfscope}%
\begin{pgfscope}%
\pgfsys@transformshift{1.941032in}{2.126457in}%
\pgfsys@useobject{currentmarker}{}%
\end{pgfscope}%
\begin{pgfscope}%
\pgfsys@transformshift{2.017893in}{1.409061in}%
\pgfsys@useobject{currentmarker}{}%
\end{pgfscope}%
\begin{pgfscope}%
\pgfsys@transformshift{2.094754in}{0.792354in}%
\pgfsys@useobject{currentmarker}{}%
\end{pgfscope}%
\begin{pgfscope}%
\pgfsys@transformshift{2.171615in}{0.663544in}%
\pgfsys@useobject{currentmarker}{}%
\end{pgfscope}%
\begin{pgfscope}%
\pgfsys@transformshift{2.248476in}{0.612152in}%
\pgfsys@useobject{currentmarker}{}%
\end{pgfscope}%
\begin{pgfscope}%
\pgfsys@transformshift{2.325337in}{0.584048in}%
\pgfsys@useobject{currentmarker}{}%
\end{pgfscope}%
\begin{pgfscope}%
\pgfsys@transformshift{2.402198in}{0.566616in}%
\pgfsys@useobject{currentmarker}{}%
\end{pgfscope}%
\begin{pgfscope}%
\pgfsys@transformshift{2.479059in}{0.552355in}%
\pgfsys@useobject{currentmarker}{}%
\end{pgfscope}%
\begin{pgfscope}%
\pgfsys@transformshift{2.555920in}{0.543945in}%
\pgfsys@useobject{currentmarker}{}%
\end{pgfscope}%
\begin{pgfscope}%
\pgfsys@transformshift{2.632781in}{0.536983in}%
\pgfsys@useobject{currentmarker}{}%
\end{pgfscope}%
\begin{pgfscope}%
\pgfsys@transformshift{2.709642in}{0.530935in}%
\pgfsys@useobject{currentmarker}{}%
\end{pgfscope}%
\begin{pgfscope}%
\pgfsys@transformshift{2.786503in}{0.526280in}%
\pgfsys@useobject{currentmarker}{}%
\end{pgfscope}%
\begin{pgfscope}%
\pgfsys@transformshift{2.863364in}{0.522511in}%
\pgfsys@useobject{currentmarker}{}%
\end{pgfscope}%
\begin{pgfscope}%
\pgfsys@transformshift{2.940225in}{0.519243in}%
\pgfsys@useobject{currentmarker}{}%
\end{pgfscope}%
\begin{pgfscope}%
\pgfsys@transformshift{3.017086in}{0.516286in}%
\pgfsys@useobject{currentmarker}{}%
\end{pgfscope}%
\begin{pgfscope}%
\pgfsys@transformshift{3.093947in}{0.513806in}%
\pgfsys@useobject{currentmarker}{}%
\end{pgfscope}%
\begin{pgfscope}%
\pgfsys@transformshift{3.170808in}{0.512002in}%
\pgfsys@useobject{currentmarker}{}%
\end{pgfscope}%
\begin{pgfscope}%
\pgfsys@transformshift{3.247669in}{0.510302in}%
\pgfsys@useobject{currentmarker}{}%
\end{pgfscope}%
\begin{pgfscope}%
\pgfsys@transformshift{3.324530in}{0.508568in}%
\pgfsys@useobject{currentmarker}{}%
\end{pgfscope}%
\begin{pgfscope}%
\pgfsys@transformshift{3.401391in}{0.507053in}%
\pgfsys@useobject{currentmarker}{}%
\end{pgfscope}%
\begin{pgfscope}%
\pgfsys@transformshift{3.478252in}{0.505562in}%
\pgfsys@useobject{currentmarker}{}%
\end{pgfscope}%
\begin{pgfscope}%
\pgfsys@transformshift{3.555113in}{0.504194in}%
\pgfsys@useobject{currentmarker}{}%
\end{pgfscope}%
\begin{pgfscope}%
\pgfsys@transformshift{3.631974in}{0.503300in}%
\pgfsys@useobject{currentmarker}{}%
\end{pgfscope}%
\begin{pgfscope}%
\pgfsys@transformshift{3.708835in}{0.502351in}%
\pgfsys@useobject{currentmarker}{}%
\end{pgfscope}%
\end{pgfscope}%
\begin{pgfscope}%
\pgfsetbuttcap%
\pgfsetroundjoin%
\definecolor{currentfill}{rgb}{0.000000,0.000000,0.000000}%
\pgfsetfillcolor{currentfill}%
\pgfsetlinewidth{0.803000pt}%
\definecolor{currentstroke}{rgb}{0.000000,0.000000,0.000000}%
\pgfsetstrokecolor{currentstroke}%
\pgfsetdash{}{0pt}%
\pgfsys@defobject{currentmarker}{\pgfqpoint{0.000000in}{-0.048611in}}{\pgfqpoint{0.000000in}{0.000000in}}{%
\pgfpathmoveto{\pgfqpoint{0.000000in}{0.000000in}}%
\pgfpathlineto{\pgfqpoint{0.000000in}{-0.048611in}}%
\pgfusepath{stroke,fill}%
}%
\begin{pgfscope}%
\pgfsys@transformshift{1.018699in}{0.386111in}%
\pgfsys@useobject{currentmarker}{}%
\end{pgfscope}%
\end{pgfscope}%
\begin{pgfscope}%
\definecolor{textcolor}{rgb}{0.000000,0.000000,0.000000}%
\pgfsetstrokecolor{textcolor}%
\pgfsetfillcolor{textcolor}%
\pgftext[x=1.018699in,y=0.288889in,,top]{\color{textcolor}\sffamily\fontsize{10.000000}{12.000000}\selectfont 1}%
\end{pgfscope}%
\begin{pgfscope}%
\pgfsetbuttcap%
\pgfsetroundjoin%
\definecolor{currentfill}{rgb}{0.000000,0.000000,0.000000}%
\pgfsetfillcolor{currentfill}%
\pgfsetlinewidth{0.803000pt}%
\definecolor{currentstroke}{rgb}{0.000000,0.000000,0.000000}%
\pgfsetstrokecolor{currentstroke}%
\pgfsetdash{}{0pt}%
\pgfsys@defobject{currentmarker}{\pgfqpoint{0.000000in}{-0.048611in}}{\pgfqpoint{0.000000in}{0.000000in}}{%
\pgfpathmoveto{\pgfqpoint{0.000000in}{0.000000in}}%
\pgfpathlineto{\pgfqpoint{0.000000in}{-0.048611in}}%
\pgfusepath{stroke,fill}%
}%
\begin{pgfscope}%
\pgfsys@transformshift{1.787309in}{0.386111in}%
\pgfsys@useobject{currentmarker}{}%
\end{pgfscope}%
\end{pgfscope}%
\begin{pgfscope}%
\definecolor{textcolor}{rgb}{0.000000,0.000000,0.000000}%
\pgfsetstrokecolor{textcolor}%
\pgfsetfillcolor{textcolor}%
\pgftext[x=1.787309in,y=0.288889in,,top]{\color{textcolor}\sffamily\fontsize{10.000000}{12.000000}\selectfont 2}%
\end{pgfscope}%
\begin{pgfscope}%
\pgfsetbuttcap%
\pgfsetroundjoin%
\definecolor{currentfill}{rgb}{0.000000,0.000000,0.000000}%
\pgfsetfillcolor{currentfill}%
\pgfsetlinewidth{0.803000pt}%
\definecolor{currentstroke}{rgb}{0.000000,0.000000,0.000000}%
\pgfsetstrokecolor{currentstroke}%
\pgfsetdash{}{0pt}%
\pgfsys@defobject{currentmarker}{\pgfqpoint{0.000000in}{-0.048611in}}{\pgfqpoint{0.000000in}{0.000000in}}{%
\pgfpathmoveto{\pgfqpoint{0.000000in}{0.000000in}}%
\pgfpathlineto{\pgfqpoint{0.000000in}{-0.048611in}}%
\pgfusepath{stroke,fill}%
}%
\begin{pgfscope}%
\pgfsys@transformshift{2.555920in}{0.386111in}%
\pgfsys@useobject{currentmarker}{}%
\end{pgfscope}%
\end{pgfscope}%
\begin{pgfscope}%
\definecolor{textcolor}{rgb}{0.000000,0.000000,0.000000}%
\pgfsetstrokecolor{textcolor}%
\pgfsetfillcolor{textcolor}%
\pgftext[x=2.555920in,y=0.288889in,,top]{\color{textcolor}\sffamily\fontsize{10.000000}{12.000000}\selectfont 3}%
\end{pgfscope}%
\begin{pgfscope}%
\pgfsetbuttcap%
\pgfsetroundjoin%
\definecolor{currentfill}{rgb}{0.000000,0.000000,0.000000}%
\pgfsetfillcolor{currentfill}%
\pgfsetlinewidth{0.803000pt}%
\definecolor{currentstroke}{rgb}{0.000000,0.000000,0.000000}%
\pgfsetstrokecolor{currentstroke}%
\pgfsetdash{}{0pt}%
\pgfsys@defobject{currentmarker}{\pgfqpoint{0.000000in}{-0.048611in}}{\pgfqpoint{0.000000in}{0.000000in}}{%
\pgfpathmoveto{\pgfqpoint{0.000000in}{0.000000in}}%
\pgfpathlineto{\pgfqpoint{0.000000in}{-0.048611in}}%
\pgfusepath{stroke,fill}%
}%
\begin{pgfscope}%
\pgfsys@transformshift{3.324530in}{0.386111in}%
\pgfsys@useobject{currentmarker}{}%
\end{pgfscope}%
\end{pgfscope}%
\begin{pgfscope}%
\definecolor{textcolor}{rgb}{0.000000,0.000000,0.000000}%
\pgfsetstrokecolor{textcolor}%
\pgfsetfillcolor{textcolor}%
\pgftext[x=3.324530in,y=0.288889in,,top]{\color{textcolor}\sffamily\fontsize{10.000000}{12.000000}\selectfont 4}%
\end{pgfscope}%
\begin{pgfscope}%
\pgfsetbuttcap%
\pgfsetroundjoin%
\definecolor{currentfill}{rgb}{0.000000,0.000000,0.000000}%
\pgfsetfillcolor{currentfill}%
\pgfsetlinewidth{0.803000pt}%
\definecolor{currentstroke}{rgb}{0.000000,0.000000,0.000000}%
\pgfsetstrokecolor{currentstroke}%
\pgfsetdash{}{0pt}%
\pgfsys@defobject{currentmarker}{\pgfqpoint{-0.048611in}{0.000000in}}{\pgfqpoint{0.000000in}{0.000000in}}{%
\pgfpathmoveto{\pgfqpoint{0.000000in}{0.000000in}}%
\pgfpathlineto{\pgfqpoint{-0.048611in}{0.000000in}}%
\pgfusepath{stroke,fill}%
}%
\begin{pgfscope}%
\pgfsys@transformshift{0.467708in}{0.453429in}%
\pgfsys@useobject{currentmarker}{}%
\end{pgfscope}%
\end{pgfscope}%
\begin{pgfscope}%
\definecolor{textcolor}{rgb}{0.000000,0.000000,0.000000}%
\pgfsetstrokecolor{textcolor}%
\pgfsetfillcolor{textcolor}%
\pgftext[x=0.149607in,y=0.400668in,left,base]{\color{textcolor}\sffamily\fontsize{10.000000}{12.000000}\selectfont 0.0}%
\end{pgfscope}%
\begin{pgfscope}%
\pgfsetbuttcap%
\pgfsetroundjoin%
\definecolor{currentfill}{rgb}{0.000000,0.000000,0.000000}%
\pgfsetfillcolor{currentfill}%
\pgfsetlinewidth{0.803000pt}%
\definecolor{currentstroke}{rgb}{0.000000,0.000000,0.000000}%
\pgfsetstrokecolor{currentstroke}%
\pgfsetdash{}{0pt}%
\pgfsys@defobject{currentmarker}{\pgfqpoint{-0.048611in}{0.000000in}}{\pgfqpoint{0.000000in}{0.000000in}}{%
\pgfpathmoveto{\pgfqpoint{0.000000in}{0.000000in}}%
\pgfpathlineto{\pgfqpoint{-0.048611in}{0.000000in}}%
\pgfusepath{stroke,fill}%
}%
\begin{pgfscope}%
\pgfsys@transformshift{0.467708in}{0.880240in}%
\pgfsys@useobject{currentmarker}{}%
\end{pgfscope}%
\end{pgfscope}%
\begin{pgfscope}%
\definecolor{textcolor}{rgb}{0.000000,0.000000,0.000000}%
\pgfsetstrokecolor{textcolor}%
\pgfsetfillcolor{textcolor}%
\pgftext[x=0.149607in,y=0.827478in,left,base]{\color{textcolor}\sffamily\fontsize{10.000000}{12.000000}\selectfont 0.2}%
\end{pgfscope}%
\begin{pgfscope}%
\pgfsetbuttcap%
\pgfsetroundjoin%
\definecolor{currentfill}{rgb}{0.000000,0.000000,0.000000}%
\pgfsetfillcolor{currentfill}%
\pgfsetlinewidth{0.803000pt}%
\definecolor{currentstroke}{rgb}{0.000000,0.000000,0.000000}%
\pgfsetstrokecolor{currentstroke}%
\pgfsetdash{}{0pt}%
\pgfsys@defobject{currentmarker}{\pgfqpoint{-0.048611in}{0.000000in}}{\pgfqpoint{0.000000in}{0.000000in}}{%
\pgfpathmoveto{\pgfqpoint{0.000000in}{0.000000in}}%
\pgfpathlineto{\pgfqpoint{-0.048611in}{0.000000in}}%
\pgfusepath{stroke,fill}%
}%
\begin{pgfscope}%
\pgfsys@transformshift{0.467708in}{1.307050in}%
\pgfsys@useobject{currentmarker}{}%
\end{pgfscope}%
\end{pgfscope}%
\begin{pgfscope}%
\definecolor{textcolor}{rgb}{0.000000,0.000000,0.000000}%
\pgfsetstrokecolor{textcolor}%
\pgfsetfillcolor{textcolor}%
\pgftext[x=0.149607in,y=1.254289in,left,base]{\color{textcolor}\sffamily\fontsize{10.000000}{12.000000}\selectfont 0.4}%
\end{pgfscope}%
\begin{pgfscope}%
\pgfsetbuttcap%
\pgfsetroundjoin%
\definecolor{currentfill}{rgb}{0.000000,0.000000,0.000000}%
\pgfsetfillcolor{currentfill}%
\pgfsetlinewidth{0.803000pt}%
\definecolor{currentstroke}{rgb}{0.000000,0.000000,0.000000}%
\pgfsetstrokecolor{currentstroke}%
\pgfsetdash{}{0pt}%
\pgfsys@defobject{currentmarker}{\pgfqpoint{-0.048611in}{0.000000in}}{\pgfqpoint{0.000000in}{0.000000in}}{%
\pgfpathmoveto{\pgfqpoint{0.000000in}{0.000000in}}%
\pgfpathlineto{\pgfqpoint{-0.048611in}{0.000000in}}%
\pgfusepath{stroke,fill}%
}%
\begin{pgfscope}%
\pgfsys@transformshift{0.467708in}{1.733861in}%
\pgfsys@useobject{currentmarker}{}%
\end{pgfscope}%
\end{pgfscope}%
\begin{pgfscope}%
\definecolor{textcolor}{rgb}{0.000000,0.000000,0.000000}%
\pgfsetstrokecolor{textcolor}%
\pgfsetfillcolor{textcolor}%
\pgftext[x=0.149607in,y=1.681099in,left,base]{\color{textcolor}\sffamily\fontsize{10.000000}{12.000000}\selectfont 0.6}%
\end{pgfscope}%
\begin{pgfscope}%
\pgfsetbuttcap%
\pgfsetroundjoin%
\definecolor{currentfill}{rgb}{0.000000,0.000000,0.000000}%
\pgfsetfillcolor{currentfill}%
\pgfsetlinewidth{0.803000pt}%
\definecolor{currentstroke}{rgb}{0.000000,0.000000,0.000000}%
\pgfsetstrokecolor{currentstroke}%
\pgfsetdash{}{0pt}%
\pgfsys@defobject{currentmarker}{\pgfqpoint{-0.048611in}{0.000000in}}{\pgfqpoint{0.000000in}{0.000000in}}{%
\pgfpathmoveto{\pgfqpoint{0.000000in}{0.000000in}}%
\pgfpathlineto{\pgfqpoint{-0.048611in}{0.000000in}}%
\pgfusepath{stroke,fill}%
}%
\begin{pgfscope}%
\pgfsys@transformshift{0.467708in}{2.160671in}%
\pgfsys@useobject{currentmarker}{}%
\end{pgfscope}%
\end{pgfscope}%
\begin{pgfscope}%
\definecolor{textcolor}{rgb}{0.000000,0.000000,0.000000}%
\pgfsetstrokecolor{textcolor}%
\pgfsetfillcolor{textcolor}%
\pgftext[x=0.149607in,y=2.107910in,left,base]{\color{textcolor}\sffamily\fontsize{10.000000}{12.000000}\selectfont 0.8}%
\end{pgfscope}%
\begin{pgfscope}%
\pgfsetbuttcap%
\pgfsetroundjoin%
\definecolor{currentfill}{rgb}{0.000000,0.000000,0.000000}%
\pgfsetfillcolor{currentfill}%
\pgfsetlinewidth{0.803000pt}%
\definecolor{currentstroke}{rgb}{0.000000,0.000000,0.000000}%
\pgfsetstrokecolor{currentstroke}%
\pgfsetdash{}{0pt}%
\pgfsys@defobject{currentmarker}{\pgfqpoint{-0.048611in}{0.000000in}}{\pgfqpoint{0.000000in}{0.000000in}}{%
\pgfpathmoveto{\pgfqpoint{0.000000in}{0.000000in}}%
\pgfpathlineto{\pgfqpoint{-0.048611in}{0.000000in}}%
\pgfusepath{stroke,fill}%
}%
\begin{pgfscope}%
\pgfsys@transformshift{0.467708in}{2.587481in}%
\pgfsys@useobject{currentmarker}{}%
\end{pgfscope}%
\end{pgfscope}%
\begin{pgfscope}%
\definecolor{textcolor}{rgb}{0.000000,0.000000,0.000000}%
\pgfsetstrokecolor{textcolor}%
\pgfsetfillcolor{textcolor}%
\pgftext[x=0.149607in,y=2.534720in,left,base]{\color{textcolor}\sffamily\fontsize{10.000000}{12.000000}\selectfont 1.0}%
\end{pgfscope}%
\begin{pgfscope}%
\pgfpathrectangle{\pgfqpoint{0.467708in}{0.386111in}}{\pgfqpoint{3.407812in}{2.316667in}}%
\pgfusepath{clip}%
\pgfsetrectcap%
\pgfsetroundjoin%
\pgfsetlinewidth{1.505625pt}%
\definecolor{currentstroke}{rgb}{0.121569,0.466667,0.705882}%
\pgfsetstrokecolor{currentstroke}%
\pgfsetdash{}{0pt}%
\pgfpathmoveto{\pgfqpoint{0.634394in}{2.586440in}}%
\pgfpathlineto{\pgfqpoint{0.711255in}{2.586446in}}%
\pgfpathlineto{\pgfqpoint{0.788116in}{2.586479in}}%
\pgfpathlineto{\pgfqpoint{0.864977in}{2.586538in}}%
\pgfpathlineto{\pgfqpoint{0.941838in}{2.586427in}}%
\pgfpathlineto{\pgfqpoint{1.018699in}{2.585728in}}%
\pgfpathlineto{\pgfqpoint{1.095560in}{2.584068in}}%
\pgfpathlineto{\pgfqpoint{1.172421in}{2.581082in}}%
\pgfpathlineto{\pgfqpoint{1.249282in}{2.576295in}}%
\pgfpathlineto{\pgfqpoint{1.326143in}{2.569082in}}%
\pgfpathlineto{\pgfqpoint{1.403004in}{2.558656in}}%
\pgfpathlineto{\pgfqpoint{1.479865in}{2.543968in}}%
\pgfpathlineto{\pgfqpoint{1.556726in}{2.523622in}}%
\pgfpathlineto{\pgfqpoint{1.633587in}{2.495424in}}%
\pgfpathlineto{\pgfqpoint{1.710448in}{2.455826in}}%
\pgfpathlineto{\pgfqpoint{1.787309in}{2.398215in}}%
\pgfpathlineto{\pgfqpoint{1.864171in}{2.307251in}}%
\pgfpathlineto{\pgfqpoint{1.941032in}{2.126457in}}%
\pgfpathlineto{\pgfqpoint{2.017893in}{1.409061in}}%
\pgfpathlineto{\pgfqpoint{2.094754in}{0.792354in}}%
\pgfpathlineto{\pgfqpoint{2.171615in}{0.663544in}}%
\pgfpathlineto{\pgfqpoint{2.248476in}{0.612152in}}%
\pgfpathlineto{\pgfqpoint{2.325337in}{0.584048in}}%
\pgfpathlineto{\pgfqpoint{2.402198in}{0.566616in}}%
\pgfpathlineto{\pgfqpoint{2.479059in}{0.552355in}}%
\pgfpathlineto{\pgfqpoint{2.555920in}{0.543945in}}%
\pgfpathlineto{\pgfqpoint{2.632781in}{0.536983in}}%
\pgfpathlineto{\pgfqpoint{2.709642in}{0.530935in}}%
\pgfpathlineto{\pgfqpoint{2.786503in}{0.526280in}}%
\pgfpathlineto{\pgfqpoint{2.863364in}{0.522511in}}%
\pgfpathlineto{\pgfqpoint{2.940225in}{0.519243in}}%
\pgfpathlineto{\pgfqpoint{3.017086in}{0.516286in}}%
\pgfpathlineto{\pgfqpoint{3.093947in}{0.513806in}}%
\pgfpathlineto{\pgfqpoint{3.170808in}{0.512002in}}%
\pgfpathlineto{\pgfqpoint{3.247669in}{0.510302in}}%
\pgfpathlineto{\pgfqpoint{3.324530in}{0.508568in}}%
\pgfpathlineto{\pgfqpoint{3.401391in}{0.507053in}}%
\pgfpathlineto{\pgfqpoint{3.478252in}{0.505562in}}%
\pgfpathlineto{\pgfqpoint{3.555113in}{0.504194in}}%
\pgfpathlineto{\pgfqpoint{3.631974in}{0.503300in}}%
\pgfpathlineto{\pgfqpoint{3.708835in}{0.502351in}}%
\pgfusepath{stroke}%
\end{pgfscope}%
\begin{pgfscope}%
\pgfsetrectcap%
\pgfsetmiterjoin%
\pgfsetlinewidth{0.803000pt}%
\definecolor{currentstroke}{rgb}{0.000000,0.000000,0.000000}%
\pgfsetstrokecolor{currentstroke}%
\pgfsetdash{}{0pt}%
\pgfpathmoveto{\pgfqpoint{0.467708in}{0.386111in}}%
\pgfpathlineto{\pgfqpoint{0.467708in}{2.702778in}}%
\pgfusepath{stroke}%
\end{pgfscope}%
\begin{pgfscope}%
\pgfsetrectcap%
\pgfsetmiterjoin%
\pgfsetlinewidth{0.803000pt}%
\definecolor{currentstroke}{rgb}{0.000000,0.000000,0.000000}%
\pgfsetstrokecolor{currentstroke}%
\pgfsetdash{}{0pt}%
\pgfpathmoveto{\pgfqpoint{3.875521in}{0.386111in}}%
\pgfpathlineto{\pgfqpoint{3.875521in}{2.702778in}}%
\pgfusepath{stroke}%
\end{pgfscope}%
\begin{pgfscope}%
\pgfsetrectcap%
\pgfsetmiterjoin%
\pgfsetlinewidth{0.803000pt}%
\definecolor{currentstroke}{rgb}{0.000000,0.000000,0.000000}%
\pgfsetstrokecolor{currentstroke}%
\pgfsetdash{}{0pt}%
\pgfpathmoveto{\pgfqpoint{0.467708in}{0.386111in}}%
\pgfpathlineto{\pgfqpoint{3.875521in}{0.386111in}}%
\pgfusepath{stroke}%
\end{pgfscope}%
\begin{pgfscope}%
\pgfsetrectcap%
\pgfsetmiterjoin%
\pgfsetlinewidth{0.803000pt}%
\definecolor{currentstroke}{rgb}{0.000000,0.000000,0.000000}%
\pgfsetstrokecolor{currentstroke}%
\pgfsetdash{}{0pt}%
\pgfpathmoveto{\pgfqpoint{0.467708in}{2.702778in}}%
\pgfpathlineto{\pgfqpoint{3.875521in}{2.702778in}}%
\pgfusepath{stroke}%
\end{pgfscope}%
\begin{pgfscope}%
\definecolor{textcolor}{rgb}{0.000000,0.000000,0.000000}%
\pgfsetstrokecolor{textcolor}%
\pgfsetfillcolor{textcolor}%
\pgftext[x=2.171615in,y=2.786111in,,base]{\color{textcolor}\sffamily\fontsize{12.000000}{14.400000}\selectfont \(\displaystyle  N = 64 \)}%
\end{pgfscope}%
\begin{pgfscope}%
\pgfsetbuttcap%
\pgfsetmiterjoin%
\definecolor{currentfill}{rgb}{1.000000,1.000000,1.000000}%
\pgfsetfillcolor{currentfill}%
\pgfsetlinewidth{0.000000pt}%
\definecolor{currentstroke}{rgb}{0.000000,0.000000,0.000000}%
\pgfsetstrokecolor{currentstroke}%
\pgfsetstrokeopacity{0.000000}%
\pgfsetdash{}{0pt}%
\pgfpathmoveto{\pgfqpoint{4.393576in}{0.386111in}}%
\pgfpathlineto{\pgfqpoint{7.801389in}{0.386111in}}%
\pgfpathlineto{\pgfqpoint{7.801389in}{2.702778in}}%
\pgfpathlineto{\pgfqpoint{4.393576in}{2.702778in}}%
\pgfpathclose%
\pgfusepath{fill}%
\end{pgfscope}%
\begin{pgfscope}%
\pgfpathrectangle{\pgfqpoint{4.393576in}{0.386111in}}{\pgfqpoint{3.407812in}{2.316667in}}%
\pgfusepath{clip}%
\pgfsetbuttcap%
\pgfsetroundjoin%
\definecolor{currentfill}{rgb}{0.121569,0.466667,0.705882}%
\pgfsetfillcolor{currentfill}%
\pgfsetfillopacity{0.300000}%
\pgfsetlinewidth{1.003750pt}%
\definecolor{currentstroke}{rgb}{0.121569,0.466667,0.705882}%
\pgfsetstrokecolor{currentstroke}%
\pgfsetstrokeopacity{0.300000}%
\pgfsetdash{}{0pt}%
\pgfpathmoveto{\pgfqpoint{4.560262in}{2.152131in}}%
\pgfpathlineto{\pgfqpoint{4.560262in}{2.152131in}}%
\pgfpathlineto{\pgfqpoint{4.637123in}{2.151744in}}%
\pgfpathlineto{\pgfqpoint{4.713984in}{2.152150in}}%
\pgfpathlineto{\pgfqpoint{4.790845in}{2.152109in}}%
\pgfpathlineto{\pgfqpoint{4.867706in}{2.151827in}}%
\pgfpathlineto{\pgfqpoint{4.944567in}{2.151116in}}%
\pgfpathlineto{\pgfqpoint{5.021428in}{2.149736in}}%
\pgfpathlineto{\pgfqpoint{5.098289in}{2.126681in}}%
\pgfpathlineto{\pgfqpoint{5.175150in}{2.143605in}}%
\pgfpathlineto{\pgfqpoint{5.252011in}{2.137951in}}%
\pgfpathlineto{\pgfqpoint{5.328872in}{2.129798in}}%
\pgfpathlineto{\pgfqpoint{5.405733in}{1.039545in}}%
\pgfpathlineto{\pgfqpoint{5.482595in}{1.398164in}}%
\pgfpathlineto{\pgfqpoint{5.559456in}{2.080266in}}%
\pgfpathlineto{\pgfqpoint{5.636317in}{1.597606in}}%
\pgfpathlineto{\pgfqpoint{5.713178in}{2.003277in}}%
\pgfpathlineto{\pgfqpoint{5.790039in}{1.905687in}}%
\pgfpathlineto{\pgfqpoint{5.866900in}{1.790873in}}%
\pgfpathlineto{\pgfqpoint{5.943761in}{0.790321in}}%
\pgfpathlineto{\pgfqpoint{6.020622in}{0.598700in}}%
\pgfpathlineto{\pgfqpoint{6.097483in}{0.558519in}}%
\pgfpathlineto{\pgfqpoint{6.174344in}{0.543681in}}%
\pgfpathlineto{\pgfqpoint{6.251205in}{0.532802in}}%
\pgfpathlineto{\pgfqpoint{6.328066in}{0.527039in}}%
\pgfpathlineto{\pgfqpoint{6.404927in}{0.519984in}}%
\pgfpathlineto{\pgfqpoint{6.481788in}{0.518581in}}%
\pgfpathlineto{\pgfqpoint{6.558649in}{0.515693in}}%
\pgfpathlineto{\pgfqpoint{6.635510in}{0.512888in}}%
\pgfpathlineto{\pgfqpoint{6.712371in}{0.511756in}}%
\pgfpathlineto{\pgfqpoint{6.789232in}{0.510040in}}%
\pgfpathlineto{\pgfqpoint{6.866093in}{0.508611in}}%
\pgfpathlineto{\pgfqpoint{6.942954in}{0.507679in}}%
\pgfpathlineto{\pgfqpoint{7.019815in}{0.506499in}}%
\pgfpathlineto{\pgfqpoint{7.096676in}{0.506276in}}%
\pgfpathlineto{\pgfqpoint{7.173537in}{0.505499in}}%
\pgfpathlineto{\pgfqpoint{7.250398in}{0.504639in}}%
\pgfpathlineto{\pgfqpoint{7.327259in}{0.503904in}}%
\pgfpathlineto{\pgfqpoint{7.404120in}{0.503280in}}%
\pgfpathlineto{\pgfqpoint{7.480981in}{0.502870in}}%
\pgfpathlineto{\pgfqpoint{7.557842in}{0.502712in}}%
\pgfpathlineto{\pgfqpoint{7.634703in}{0.502193in}}%
\pgfpathlineto{\pgfqpoint{7.634703in}{0.502621in}}%
\pgfpathlineto{\pgfqpoint{7.634703in}{0.502621in}}%
\pgfpathlineto{\pgfqpoint{7.557842in}{0.503306in}}%
\pgfpathlineto{\pgfqpoint{7.480981in}{0.503274in}}%
\pgfpathlineto{\pgfqpoint{7.404120in}{0.504126in}}%
\pgfpathlineto{\pgfqpoint{7.327259in}{0.504462in}}%
\pgfpathlineto{\pgfqpoint{7.250398in}{0.505004in}}%
\pgfpathlineto{\pgfqpoint{7.173537in}{0.505671in}}%
\pgfpathlineto{\pgfqpoint{7.096676in}{0.506673in}}%
\pgfpathlineto{\pgfqpoint{7.019815in}{0.507738in}}%
\pgfpathlineto{\pgfqpoint{6.942954in}{0.508359in}}%
\pgfpathlineto{\pgfqpoint{6.866093in}{0.509796in}}%
\pgfpathlineto{\pgfqpoint{6.789232in}{0.511112in}}%
\pgfpathlineto{\pgfqpoint{6.712371in}{0.512389in}}%
\pgfpathlineto{\pgfqpoint{6.635510in}{0.514599in}}%
\pgfpathlineto{\pgfqpoint{6.558649in}{0.516483in}}%
\pgfpathlineto{\pgfqpoint{6.481788in}{0.519858in}}%
\pgfpathlineto{\pgfqpoint{6.404927in}{0.523535in}}%
\pgfpathlineto{\pgfqpoint{6.328066in}{0.528927in}}%
\pgfpathlineto{\pgfqpoint{6.251205in}{0.535914in}}%
\pgfpathlineto{\pgfqpoint{6.174344in}{0.548937in}}%
\pgfpathlineto{\pgfqpoint{6.097483in}{0.573074in}}%
\pgfpathlineto{\pgfqpoint{6.020622in}{0.621038in}}%
\pgfpathlineto{\pgfqpoint{5.943761in}{1.099691in}}%
\pgfpathlineto{\pgfqpoint{5.866900in}{1.792088in}}%
\pgfpathlineto{\pgfqpoint{5.790039in}{1.946584in}}%
\pgfpathlineto{\pgfqpoint{5.713178in}{2.004933in}}%
\pgfpathlineto{\pgfqpoint{5.636317in}{2.250019in}}%
\pgfpathlineto{\pgfqpoint{5.559456in}{2.080362in}}%
\pgfpathlineto{\pgfqpoint{5.482595in}{2.415139in}}%
\pgfpathlineto{\pgfqpoint{5.405733in}{2.597475in}}%
\pgfpathlineto{\pgfqpoint{5.328872in}{2.129814in}}%
\pgfpathlineto{\pgfqpoint{5.252011in}{2.137965in}}%
\pgfpathlineto{\pgfqpoint{5.175150in}{2.143610in}}%
\pgfpathlineto{\pgfqpoint{5.098289in}{2.156556in}}%
\pgfpathlineto{\pgfqpoint{5.021428in}{2.149738in}}%
\pgfpathlineto{\pgfqpoint{4.944567in}{2.151116in}}%
\pgfpathlineto{\pgfqpoint{4.867706in}{2.151827in}}%
\pgfpathlineto{\pgfqpoint{4.790845in}{2.152109in}}%
\pgfpathlineto{\pgfqpoint{4.713984in}{2.152151in}}%
\pgfpathlineto{\pgfqpoint{4.637123in}{2.152310in}}%
\pgfpathlineto{\pgfqpoint{4.560262in}{2.152131in}}%
\pgfpathclose%
\pgfusepath{stroke,fill}%
\end{pgfscope}%
\begin{pgfscope}%
\pgfpathrectangle{\pgfqpoint{4.393576in}{0.386111in}}{\pgfqpoint{3.407812in}{2.316667in}}%
\pgfusepath{clip}%
\pgfsetbuttcap%
\pgfsetroundjoin%
\definecolor{currentfill}{rgb}{0.121569,0.466667,0.705882}%
\pgfsetfillcolor{currentfill}%
\pgfsetlinewidth{1.003750pt}%
\definecolor{currentstroke}{rgb}{0.121569,0.466667,0.705882}%
\pgfsetstrokecolor{currentstroke}%
\pgfsetdash{}{0pt}%
\pgfsys@defobject{currentmarker}{\pgfqpoint{-0.009821in}{-0.009821in}}{\pgfqpoint{0.009821in}{0.009821in}}{%
\pgfpathmoveto{\pgfqpoint{0.000000in}{-0.009821in}}%
\pgfpathcurveto{\pgfqpoint{0.002605in}{-0.009821in}}{\pgfqpoint{0.005103in}{-0.008786in}}{\pgfqpoint{0.006944in}{-0.006944in}}%
\pgfpathcurveto{\pgfqpoint{0.008786in}{-0.005103in}}{\pgfqpoint{0.009821in}{-0.002605in}}{\pgfqpoint{0.009821in}{0.000000in}}%
\pgfpathcurveto{\pgfqpoint{0.009821in}{0.002605in}}{\pgfqpoint{0.008786in}{0.005103in}}{\pgfqpoint{0.006944in}{0.006944in}}%
\pgfpathcurveto{\pgfqpoint{0.005103in}{0.008786in}}{\pgfqpoint{0.002605in}{0.009821in}}{\pgfqpoint{0.000000in}{0.009821in}}%
\pgfpathcurveto{\pgfqpoint{-0.002605in}{0.009821in}}{\pgfqpoint{-0.005103in}{0.008786in}}{\pgfqpoint{-0.006944in}{0.006944in}}%
\pgfpathcurveto{\pgfqpoint{-0.008786in}{0.005103in}}{\pgfqpoint{-0.009821in}{0.002605in}}{\pgfqpoint{-0.009821in}{0.000000in}}%
\pgfpathcurveto{\pgfqpoint{-0.009821in}{-0.002605in}}{\pgfqpoint{-0.008786in}{-0.005103in}}{\pgfqpoint{-0.006944in}{-0.006944in}}%
\pgfpathcurveto{\pgfqpoint{-0.005103in}{-0.008786in}}{\pgfqpoint{-0.002605in}{-0.009821in}}{\pgfqpoint{0.000000in}{-0.009821in}}%
\pgfpathclose%
\pgfusepath{stroke,fill}%
}%
\begin{pgfscope}%
\pgfsys@transformshift{4.560262in}{2.152131in}%
\pgfsys@useobject{currentmarker}{}%
\end{pgfscope}%
\begin{pgfscope}%
\pgfsys@transformshift{4.637123in}{2.152027in}%
\pgfsys@useobject{currentmarker}{}%
\end{pgfscope}%
\begin{pgfscope}%
\pgfsys@transformshift{4.713984in}{2.152151in}%
\pgfsys@useobject{currentmarker}{}%
\end{pgfscope}%
\begin{pgfscope}%
\pgfsys@transformshift{4.790845in}{2.152109in}%
\pgfsys@useobject{currentmarker}{}%
\end{pgfscope}%
\begin{pgfscope}%
\pgfsys@transformshift{4.867706in}{2.151827in}%
\pgfsys@useobject{currentmarker}{}%
\end{pgfscope}%
\begin{pgfscope}%
\pgfsys@transformshift{4.944567in}{2.151116in}%
\pgfsys@useobject{currentmarker}{}%
\end{pgfscope}%
\begin{pgfscope}%
\pgfsys@transformshift{5.021428in}{2.149737in}%
\pgfsys@useobject{currentmarker}{}%
\end{pgfscope}%
\begin{pgfscope}%
\pgfsys@transformshift{5.098289in}{2.141618in}%
\pgfsys@useobject{currentmarker}{}%
\end{pgfscope}%
\begin{pgfscope}%
\pgfsys@transformshift{5.175150in}{2.143608in}%
\pgfsys@useobject{currentmarker}{}%
\end{pgfscope}%
\begin{pgfscope}%
\pgfsys@transformshift{5.252011in}{2.137958in}%
\pgfsys@useobject{currentmarker}{}%
\end{pgfscope}%
\begin{pgfscope}%
\pgfsys@transformshift{5.328872in}{2.129806in}%
\pgfsys@useobject{currentmarker}{}%
\end{pgfscope}%
\begin{pgfscope}%
\pgfsys@transformshift{5.405733in}{1.818510in}%
\pgfsys@useobject{currentmarker}{}%
\end{pgfscope}%
\begin{pgfscope}%
\pgfsys@transformshift{5.482595in}{1.906651in}%
\pgfsys@useobject{currentmarker}{}%
\end{pgfscope}%
\begin{pgfscope}%
\pgfsys@transformshift{5.559456in}{2.080314in}%
\pgfsys@useobject{currentmarker}{}%
\end{pgfscope}%
\begin{pgfscope}%
\pgfsys@transformshift{5.636317in}{1.923812in}%
\pgfsys@useobject{currentmarker}{}%
\end{pgfscope}%
\begin{pgfscope}%
\pgfsys@transformshift{5.713178in}{2.004105in}%
\pgfsys@useobject{currentmarker}{}%
\end{pgfscope}%
\begin{pgfscope}%
\pgfsys@transformshift{5.790039in}{1.926136in}%
\pgfsys@useobject{currentmarker}{}%
\end{pgfscope}%
\begin{pgfscope}%
\pgfsys@transformshift{5.866900in}{1.791481in}%
\pgfsys@useobject{currentmarker}{}%
\end{pgfscope}%
\begin{pgfscope}%
\pgfsys@transformshift{5.943761in}{0.945006in}%
\pgfsys@useobject{currentmarker}{}%
\end{pgfscope}%
\begin{pgfscope}%
\pgfsys@transformshift{6.020622in}{0.609869in}%
\pgfsys@useobject{currentmarker}{}%
\end{pgfscope}%
\begin{pgfscope}%
\pgfsys@transformshift{6.097483in}{0.565796in}%
\pgfsys@useobject{currentmarker}{}%
\end{pgfscope}%
\begin{pgfscope}%
\pgfsys@transformshift{6.174344in}{0.546309in}%
\pgfsys@useobject{currentmarker}{}%
\end{pgfscope}%
\begin{pgfscope}%
\pgfsys@transformshift{6.251205in}{0.534358in}%
\pgfsys@useobject{currentmarker}{}%
\end{pgfscope}%
\begin{pgfscope}%
\pgfsys@transformshift{6.328066in}{0.527983in}%
\pgfsys@useobject{currentmarker}{}%
\end{pgfscope}%
\begin{pgfscope}%
\pgfsys@transformshift{6.404927in}{0.521760in}%
\pgfsys@useobject{currentmarker}{}%
\end{pgfscope}%
\begin{pgfscope}%
\pgfsys@transformshift{6.481788in}{0.519219in}%
\pgfsys@useobject{currentmarker}{}%
\end{pgfscope}%
\begin{pgfscope}%
\pgfsys@transformshift{6.558649in}{0.516088in}%
\pgfsys@useobject{currentmarker}{}%
\end{pgfscope}%
\begin{pgfscope}%
\pgfsys@transformshift{6.635510in}{0.513743in}%
\pgfsys@useobject{currentmarker}{}%
\end{pgfscope}%
\begin{pgfscope}%
\pgfsys@transformshift{6.712371in}{0.512073in}%
\pgfsys@useobject{currentmarker}{}%
\end{pgfscope}%
\begin{pgfscope}%
\pgfsys@transformshift{6.789232in}{0.510576in}%
\pgfsys@useobject{currentmarker}{}%
\end{pgfscope}%
\begin{pgfscope}%
\pgfsys@transformshift{6.866093in}{0.509204in}%
\pgfsys@useobject{currentmarker}{}%
\end{pgfscope}%
\begin{pgfscope}%
\pgfsys@transformshift{6.942954in}{0.508019in}%
\pgfsys@useobject{currentmarker}{}%
\end{pgfscope}%
\begin{pgfscope}%
\pgfsys@transformshift{7.019815in}{0.507119in}%
\pgfsys@useobject{currentmarker}{}%
\end{pgfscope}%
\begin{pgfscope}%
\pgfsys@transformshift{7.096676in}{0.506475in}%
\pgfsys@useobject{currentmarker}{}%
\end{pgfscope}%
\begin{pgfscope}%
\pgfsys@transformshift{7.173537in}{0.505585in}%
\pgfsys@useobject{currentmarker}{}%
\end{pgfscope}%
\begin{pgfscope}%
\pgfsys@transformshift{7.250398in}{0.504821in}%
\pgfsys@useobject{currentmarker}{}%
\end{pgfscope}%
\begin{pgfscope}%
\pgfsys@transformshift{7.327259in}{0.504183in}%
\pgfsys@useobject{currentmarker}{}%
\end{pgfscope}%
\begin{pgfscope}%
\pgfsys@transformshift{7.404120in}{0.503703in}%
\pgfsys@useobject{currentmarker}{}%
\end{pgfscope}%
\begin{pgfscope}%
\pgfsys@transformshift{7.480981in}{0.503072in}%
\pgfsys@useobject{currentmarker}{}%
\end{pgfscope}%
\begin{pgfscope}%
\pgfsys@transformshift{7.557842in}{0.503009in}%
\pgfsys@useobject{currentmarker}{}%
\end{pgfscope}%
\begin{pgfscope}%
\pgfsys@transformshift{7.634703in}{0.502407in}%
\pgfsys@useobject{currentmarker}{}%
\end{pgfscope}%
\end{pgfscope}%
\begin{pgfscope}%
\pgfsetbuttcap%
\pgfsetroundjoin%
\definecolor{currentfill}{rgb}{0.000000,0.000000,0.000000}%
\pgfsetfillcolor{currentfill}%
\pgfsetlinewidth{0.803000pt}%
\definecolor{currentstroke}{rgb}{0.000000,0.000000,0.000000}%
\pgfsetstrokecolor{currentstroke}%
\pgfsetdash{}{0pt}%
\pgfsys@defobject{currentmarker}{\pgfqpoint{0.000000in}{-0.048611in}}{\pgfqpoint{0.000000in}{0.000000in}}{%
\pgfpathmoveto{\pgfqpoint{0.000000in}{0.000000in}}%
\pgfpathlineto{\pgfqpoint{0.000000in}{-0.048611in}}%
\pgfusepath{stroke,fill}%
}%
\begin{pgfscope}%
\pgfsys@transformshift{4.944567in}{0.386111in}%
\pgfsys@useobject{currentmarker}{}%
\end{pgfscope}%
\end{pgfscope}%
\begin{pgfscope}%
\definecolor{textcolor}{rgb}{0.000000,0.000000,0.000000}%
\pgfsetstrokecolor{textcolor}%
\pgfsetfillcolor{textcolor}%
\pgftext[x=4.944567in,y=0.288889in,,top]{\color{textcolor}\sffamily\fontsize{10.000000}{12.000000}\selectfont 1}%
\end{pgfscope}%
\begin{pgfscope}%
\pgfsetbuttcap%
\pgfsetroundjoin%
\definecolor{currentfill}{rgb}{0.000000,0.000000,0.000000}%
\pgfsetfillcolor{currentfill}%
\pgfsetlinewidth{0.803000pt}%
\definecolor{currentstroke}{rgb}{0.000000,0.000000,0.000000}%
\pgfsetstrokecolor{currentstroke}%
\pgfsetdash{}{0pt}%
\pgfsys@defobject{currentmarker}{\pgfqpoint{0.000000in}{-0.048611in}}{\pgfqpoint{0.000000in}{0.000000in}}{%
\pgfpathmoveto{\pgfqpoint{0.000000in}{0.000000in}}%
\pgfpathlineto{\pgfqpoint{0.000000in}{-0.048611in}}%
\pgfusepath{stroke,fill}%
}%
\begin{pgfscope}%
\pgfsys@transformshift{5.713178in}{0.386111in}%
\pgfsys@useobject{currentmarker}{}%
\end{pgfscope}%
\end{pgfscope}%
\begin{pgfscope}%
\definecolor{textcolor}{rgb}{0.000000,0.000000,0.000000}%
\pgfsetstrokecolor{textcolor}%
\pgfsetfillcolor{textcolor}%
\pgftext[x=5.713178in,y=0.288889in,,top]{\color{textcolor}\sffamily\fontsize{10.000000}{12.000000}\selectfont 2}%
\end{pgfscope}%
\begin{pgfscope}%
\pgfsetbuttcap%
\pgfsetroundjoin%
\definecolor{currentfill}{rgb}{0.000000,0.000000,0.000000}%
\pgfsetfillcolor{currentfill}%
\pgfsetlinewidth{0.803000pt}%
\definecolor{currentstroke}{rgb}{0.000000,0.000000,0.000000}%
\pgfsetstrokecolor{currentstroke}%
\pgfsetdash{}{0pt}%
\pgfsys@defobject{currentmarker}{\pgfqpoint{0.000000in}{-0.048611in}}{\pgfqpoint{0.000000in}{0.000000in}}{%
\pgfpathmoveto{\pgfqpoint{0.000000in}{0.000000in}}%
\pgfpathlineto{\pgfqpoint{0.000000in}{-0.048611in}}%
\pgfusepath{stroke,fill}%
}%
\begin{pgfscope}%
\pgfsys@transformshift{6.481788in}{0.386111in}%
\pgfsys@useobject{currentmarker}{}%
\end{pgfscope}%
\end{pgfscope}%
\begin{pgfscope}%
\definecolor{textcolor}{rgb}{0.000000,0.000000,0.000000}%
\pgfsetstrokecolor{textcolor}%
\pgfsetfillcolor{textcolor}%
\pgftext[x=6.481788in,y=0.288889in,,top]{\color{textcolor}\sffamily\fontsize{10.000000}{12.000000}\selectfont 3}%
\end{pgfscope}%
\begin{pgfscope}%
\pgfsetbuttcap%
\pgfsetroundjoin%
\definecolor{currentfill}{rgb}{0.000000,0.000000,0.000000}%
\pgfsetfillcolor{currentfill}%
\pgfsetlinewidth{0.803000pt}%
\definecolor{currentstroke}{rgb}{0.000000,0.000000,0.000000}%
\pgfsetstrokecolor{currentstroke}%
\pgfsetdash{}{0pt}%
\pgfsys@defobject{currentmarker}{\pgfqpoint{0.000000in}{-0.048611in}}{\pgfqpoint{0.000000in}{0.000000in}}{%
\pgfpathmoveto{\pgfqpoint{0.000000in}{0.000000in}}%
\pgfpathlineto{\pgfqpoint{0.000000in}{-0.048611in}}%
\pgfusepath{stroke,fill}%
}%
\begin{pgfscope}%
\pgfsys@transformshift{7.250398in}{0.386111in}%
\pgfsys@useobject{currentmarker}{}%
\end{pgfscope}%
\end{pgfscope}%
\begin{pgfscope}%
\definecolor{textcolor}{rgb}{0.000000,0.000000,0.000000}%
\pgfsetstrokecolor{textcolor}%
\pgfsetfillcolor{textcolor}%
\pgftext[x=7.250398in,y=0.288889in,,top]{\color{textcolor}\sffamily\fontsize{10.000000}{12.000000}\selectfont 4}%
\end{pgfscope}%
\begin{pgfscope}%
\pgfsetbuttcap%
\pgfsetroundjoin%
\definecolor{currentfill}{rgb}{0.000000,0.000000,0.000000}%
\pgfsetfillcolor{currentfill}%
\pgfsetlinewidth{0.803000pt}%
\definecolor{currentstroke}{rgb}{0.000000,0.000000,0.000000}%
\pgfsetstrokecolor{currentstroke}%
\pgfsetdash{}{0pt}%
\pgfsys@defobject{currentmarker}{\pgfqpoint{-0.048611in}{0.000000in}}{\pgfqpoint{0.000000in}{0.000000in}}{%
\pgfpathmoveto{\pgfqpoint{0.000000in}{0.000000in}}%
\pgfpathlineto{\pgfqpoint{-0.048611in}{0.000000in}}%
\pgfusepath{stroke,fill}%
}%
\begin{pgfscope}%
\pgfsys@transformshift{4.393576in}{0.483332in}%
\pgfsys@useobject{currentmarker}{}%
\end{pgfscope}%
\end{pgfscope}%
\begin{pgfscope}%
\definecolor{textcolor}{rgb}{0.000000,0.000000,0.000000}%
\pgfsetstrokecolor{textcolor}%
\pgfsetfillcolor{textcolor}%
\pgftext[x=4.075475in,y=0.430571in,left,base]{\color{textcolor}\sffamily\fontsize{10.000000}{12.000000}\selectfont 0.0}%
\end{pgfscope}%
\begin{pgfscope}%
\pgfsetbuttcap%
\pgfsetroundjoin%
\definecolor{currentfill}{rgb}{0.000000,0.000000,0.000000}%
\pgfsetfillcolor{currentfill}%
\pgfsetlinewidth{0.803000pt}%
\definecolor{currentstroke}{rgb}{0.000000,0.000000,0.000000}%
\pgfsetstrokecolor{currentstroke}%
\pgfsetdash{}{0pt}%
\pgfsys@defobject{currentmarker}{\pgfqpoint{-0.048611in}{0.000000in}}{\pgfqpoint{0.000000in}{0.000000in}}{%
\pgfpathmoveto{\pgfqpoint{0.000000in}{0.000000in}}%
\pgfpathlineto{\pgfqpoint{-0.048611in}{0.000000in}}%
\pgfusepath{stroke,fill}%
}%
\begin{pgfscope}%
\pgfsys@transformshift{4.393576in}{0.817133in}%
\pgfsys@useobject{currentmarker}{}%
\end{pgfscope}%
\end{pgfscope}%
\begin{pgfscope}%
\definecolor{textcolor}{rgb}{0.000000,0.000000,0.000000}%
\pgfsetstrokecolor{textcolor}%
\pgfsetfillcolor{textcolor}%
\pgftext[x=4.075475in,y=0.764371in,left,base]{\color{textcolor}\sffamily\fontsize{10.000000}{12.000000}\selectfont 0.2}%
\end{pgfscope}%
\begin{pgfscope}%
\pgfsetbuttcap%
\pgfsetroundjoin%
\definecolor{currentfill}{rgb}{0.000000,0.000000,0.000000}%
\pgfsetfillcolor{currentfill}%
\pgfsetlinewidth{0.803000pt}%
\definecolor{currentstroke}{rgb}{0.000000,0.000000,0.000000}%
\pgfsetstrokecolor{currentstroke}%
\pgfsetdash{}{0pt}%
\pgfsys@defobject{currentmarker}{\pgfqpoint{-0.048611in}{0.000000in}}{\pgfqpoint{0.000000in}{0.000000in}}{%
\pgfpathmoveto{\pgfqpoint{0.000000in}{0.000000in}}%
\pgfpathlineto{\pgfqpoint{-0.048611in}{0.000000in}}%
\pgfusepath{stroke,fill}%
}%
\begin{pgfscope}%
\pgfsys@transformshift{4.393576in}{1.150933in}%
\pgfsys@useobject{currentmarker}{}%
\end{pgfscope}%
\end{pgfscope}%
\begin{pgfscope}%
\definecolor{textcolor}{rgb}{0.000000,0.000000,0.000000}%
\pgfsetstrokecolor{textcolor}%
\pgfsetfillcolor{textcolor}%
\pgftext[x=4.075475in,y=1.098172in,left,base]{\color{textcolor}\sffamily\fontsize{10.000000}{12.000000}\selectfont 0.4}%
\end{pgfscope}%
\begin{pgfscope}%
\pgfsetbuttcap%
\pgfsetroundjoin%
\definecolor{currentfill}{rgb}{0.000000,0.000000,0.000000}%
\pgfsetfillcolor{currentfill}%
\pgfsetlinewidth{0.803000pt}%
\definecolor{currentstroke}{rgb}{0.000000,0.000000,0.000000}%
\pgfsetstrokecolor{currentstroke}%
\pgfsetdash{}{0pt}%
\pgfsys@defobject{currentmarker}{\pgfqpoint{-0.048611in}{0.000000in}}{\pgfqpoint{0.000000in}{0.000000in}}{%
\pgfpathmoveto{\pgfqpoint{0.000000in}{0.000000in}}%
\pgfpathlineto{\pgfqpoint{-0.048611in}{0.000000in}}%
\pgfusepath{stroke,fill}%
}%
\begin{pgfscope}%
\pgfsys@transformshift{4.393576in}{1.484734in}%
\pgfsys@useobject{currentmarker}{}%
\end{pgfscope}%
\end{pgfscope}%
\begin{pgfscope}%
\definecolor{textcolor}{rgb}{0.000000,0.000000,0.000000}%
\pgfsetstrokecolor{textcolor}%
\pgfsetfillcolor{textcolor}%
\pgftext[x=4.075475in,y=1.431972in,left,base]{\color{textcolor}\sffamily\fontsize{10.000000}{12.000000}\selectfont 0.6}%
\end{pgfscope}%
\begin{pgfscope}%
\pgfsetbuttcap%
\pgfsetroundjoin%
\definecolor{currentfill}{rgb}{0.000000,0.000000,0.000000}%
\pgfsetfillcolor{currentfill}%
\pgfsetlinewidth{0.803000pt}%
\definecolor{currentstroke}{rgb}{0.000000,0.000000,0.000000}%
\pgfsetstrokecolor{currentstroke}%
\pgfsetdash{}{0pt}%
\pgfsys@defobject{currentmarker}{\pgfqpoint{-0.048611in}{0.000000in}}{\pgfqpoint{0.000000in}{0.000000in}}{%
\pgfpathmoveto{\pgfqpoint{0.000000in}{0.000000in}}%
\pgfpathlineto{\pgfqpoint{-0.048611in}{0.000000in}}%
\pgfusepath{stroke,fill}%
}%
\begin{pgfscope}%
\pgfsys@transformshift{4.393576in}{1.818534in}%
\pgfsys@useobject{currentmarker}{}%
\end{pgfscope}%
\end{pgfscope}%
\begin{pgfscope}%
\definecolor{textcolor}{rgb}{0.000000,0.000000,0.000000}%
\pgfsetstrokecolor{textcolor}%
\pgfsetfillcolor{textcolor}%
\pgftext[x=4.075475in,y=1.765772in,left,base]{\color{textcolor}\sffamily\fontsize{10.000000}{12.000000}\selectfont 0.8}%
\end{pgfscope}%
\begin{pgfscope}%
\pgfsetbuttcap%
\pgfsetroundjoin%
\definecolor{currentfill}{rgb}{0.000000,0.000000,0.000000}%
\pgfsetfillcolor{currentfill}%
\pgfsetlinewidth{0.803000pt}%
\definecolor{currentstroke}{rgb}{0.000000,0.000000,0.000000}%
\pgfsetstrokecolor{currentstroke}%
\pgfsetdash{}{0pt}%
\pgfsys@defobject{currentmarker}{\pgfqpoint{-0.048611in}{0.000000in}}{\pgfqpoint{0.000000in}{0.000000in}}{%
\pgfpathmoveto{\pgfqpoint{0.000000in}{0.000000in}}%
\pgfpathlineto{\pgfqpoint{-0.048611in}{0.000000in}}%
\pgfusepath{stroke,fill}%
}%
\begin{pgfscope}%
\pgfsys@transformshift{4.393576in}{2.152334in}%
\pgfsys@useobject{currentmarker}{}%
\end{pgfscope}%
\end{pgfscope}%
\begin{pgfscope}%
\definecolor{textcolor}{rgb}{0.000000,0.000000,0.000000}%
\pgfsetstrokecolor{textcolor}%
\pgfsetfillcolor{textcolor}%
\pgftext[x=4.075475in,y=2.099573in,left,base]{\color{textcolor}\sffamily\fontsize{10.000000}{12.000000}\selectfont 1.0}%
\end{pgfscope}%
\begin{pgfscope}%
\pgfsetbuttcap%
\pgfsetroundjoin%
\definecolor{currentfill}{rgb}{0.000000,0.000000,0.000000}%
\pgfsetfillcolor{currentfill}%
\pgfsetlinewidth{0.803000pt}%
\definecolor{currentstroke}{rgb}{0.000000,0.000000,0.000000}%
\pgfsetstrokecolor{currentstroke}%
\pgfsetdash{}{0pt}%
\pgfsys@defobject{currentmarker}{\pgfqpoint{-0.048611in}{0.000000in}}{\pgfqpoint{0.000000in}{0.000000in}}{%
\pgfpathmoveto{\pgfqpoint{0.000000in}{0.000000in}}%
\pgfpathlineto{\pgfqpoint{-0.048611in}{0.000000in}}%
\pgfusepath{stroke,fill}%
}%
\begin{pgfscope}%
\pgfsys@transformshift{4.393576in}{2.486135in}%
\pgfsys@useobject{currentmarker}{}%
\end{pgfscope}%
\end{pgfscope}%
\begin{pgfscope}%
\definecolor{textcolor}{rgb}{0.000000,0.000000,0.000000}%
\pgfsetstrokecolor{textcolor}%
\pgfsetfillcolor{textcolor}%
\pgftext[x=4.075475in,y=2.433373in,left,base]{\color{textcolor}\sffamily\fontsize{10.000000}{12.000000}\selectfont 1.2}%
\end{pgfscope}%
\begin{pgfscope}%
\pgfpathrectangle{\pgfqpoint{4.393576in}{0.386111in}}{\pgfqpoint{3.407812in}{2.316667in}}%
\pgfusepath{clip}%
\pgfsetrectcap%
\pgfsetroundjoin%
\pgfsetlinewidth{1.505625pt}%
\definecolor{currentstroke}{rgb}{0.121569,0.466667,0.705882}%
\pgfsetstrokecolor{currentstroke}%
\pgfsetdash{}{0pt}%
\pgfpathmoveto{\pgfqpoint{4.560262in}{2.152131in}}%
\pgfpathlineto{\pgfqpoint{4.637123in}{2.152027in}}%
\pgfpathlineto{\pgfqpoint{4.713984in}{2.152151in}}%
\pgfpathlineto{\pgfqpoint{4.790845in}{2.152109in}}%
\pgfpathlineto{\pgfqpoint{4.867706in}{2.151827in}}%
\pgfpathlineto{\pgfqpoint{4.944567in}{2.151116in}}%
\pgfpathlineto{\pgfqpoint{5.021428in}{2.149737in}}%
\pgfpathlineto{\pgfqpoint{5.098289in}{2.141618in}}%
\pgfpathlineto{\pgfqpoint{5.175150in}{2.143608in}}%
\pgfpathlineto{\pgfqpoint{5.252011in}{2.137958in}}%
\pgfpathlineto{\pgfqpoint{5.328872in}{2.129806in}}%
\pgfpathlineto{\pgfqpoint{5.405733in}{1.818510in}}%
\pgfpathlineto{\pgfqpoint{5.482595in}{1.906651in}}%
\pgfpathlineto{\pgfqpoint{5.559456in}{2.080314in}}%
\pgfpathlineto{\pgfqpoint{5.636317in}{1.923812in}}%
\pgfpathlineto{\pgfqpoint{5.713178in}{2.004105in}}%
\pgfpathlineto{\pgfqpoint{5.790039in}{1.926136in}}%
\pgfpathlineto{\pgfqpoint{5.866900in}{1.791481in}}%
\pgfpathlineto{\pgfqpoint{5.943761in}{0.945006in}}%
\pgfpathlineto{\pgfqpoint{6.020622in}{0.609869in}}%
\pgfpathlineto{\pgfqpoint{6.097483in}{0.565796in}}%
\pgfpathlineto{\pgfqpoint{6.174344in}{0.546309in}}%
\pgfpathlineto{\pgfqpoint{6.251205in}{0.534358in}}%
\pgfpathlineto{\pgfqpoint{6.328066in}{0.527983in}}%
\pgfpathlineto{\pgfqpoint{6.404927in}{0.521760in}}%
\pgfpathlineto{\pgfqpoint{6.481788in}{0.519219in}}%
\pgfpathlineto{\pgfqpoint{6.558649in}{0.516088in}}%
\pgfpathlineto{\pgfqpoint{6.635510in}{0.513743in}}%
\pgfpathlineto{\pgfqpoint{6.712371in}{0.512073in}}%
\pgfpathlineto{\pgfqpoint{6.789232in}{0.510576in}}%
\pgfpathlineto{\pgfqpoint{6.866093in}{0.509204in}}%
\pgfpathlineto{\pgfqpoint{6.942954in}{0.508019in}}%
\pgfpathlineto{\pgfqpoint{7.019815in}{0.507119in}}%
\pgfpathlineto{\pgfqpoint{7.096676in}{0.506475in}}%
\pgfpathlineto{\pgfqpoint{7.173537in}{0.505585in}}%
\pgfpathlineto{\pgfqpoint{7.250398in}{0.504821in}}%
\pgfpathlineto{\pgfqpoint{7.327259in}{0.504183in}}%
\pgfpathlineto{\pgfqpoint{7.404120in}{0.503703in}}%
\pgfpathlineto{\pgfqpoint{7.480981in}{0.503072in}}%
\pgfpathlineto{\pgfqpoint{7.557842in}{0.503009in}}%
\pgfpathlineto{\pgfqpoint{7.634703in}{0.502407in}}%
\pgfusepath{stroke}%
\end{pgfscope}%
\begin{pgfscope}%
\pgfsetrectcap%
\pgfsetmiterjoin%
\pgfsetlinewidth{0.803000pt}%
\definecolor{currentstroke}{rgb}{0.000000,0.000000,0.000000}%
\pgfsetstrokecolor{currentstroke}%
\pgfsetdash{}{0pt}%
\pgfpathmoveto{\pgfqpoint{4.393576in}{0.386111in}}%
\pgfpathlineto{\pgfqpoint{4.393576in}{2.702778in}}%
\pgfusepath{stroke}%
\end{pgfscope}%
\begin{pgfscope}%
\pgfsetrectcap%
\pgfsetmiterjoin%
\pgfsetlinewidth{0.803000pt}%
\definecolor{currentstroke}{rgb}{0.000000,0.000000,0.000000}%
\pgfsetstrokecolor{currentstroke}%
\pgfsetdash{}{0pt}%
\pgfpathmoveto{\pgfqpoint{7.801389in}{0.386111in}}%
\pgfpathlineto{\pgfqpoint{7.801389in}{2.702778in}}%
\pgfusepath{stroke}%
\end{pgfscope}%
\begin{pgfscope}%
\pgfsetrectcap%
\pgfsetmiterjoin%
\pgfsetlinewidth{0.803000pt}%
\definecolor{currentstroke}{rgb}{0.000000,0.000000,0.000000}%
\pgfsetstrokecolor{currentstroke}%
\pgfsetdash{}{0pt}%
\pgfpathmoveto{\pgfqpoint{4.393576in}{0.386111in}}%
\pgfpathlineto{\pgfqpoint{7.801389in}{0.386111in}}%
\pgfusepath{stroke}%
\end{pgfscope}%
\begin{pgfscope}%
\pgfsetrectcap%
\pgfsetmiterjoin%
\pgfsetlinewidth{0.803000pt}%
\definecolor{currentstroke}{rgb}{0.000000,0.000000,0.000000}%
\pgfsetstrokecolor{currentstroke}%
\pgfsetdash{}{0pt}%
\pgfpathmoveto{\pgfqpoint{4.393576in}{2.702778in}}%
\pgfpathlineto{\pgfqpoint{7.801389in}{2.702778in}}%
\pgfusepath{stroke}%
\end{pgfscope}%
\begin{pgfscope}%
\definecolor{textcolor}{rgb}{0.000000,0.000000,0.000000}%
\pgfsetstrokecolor{textcolor}%
\pgfsetfillcolor{textcolor}%
\pgftext[x=6.097483in,y=2.786111in,,base]{\color{textcolor}\sffamily\fontsize{12.000000}{14.400000}\selectfont \(\displaystyle  N = 128 \)}%
\end{pgfscope}%
\end{pgfpicture}%
\makeatother%
\endgroup%
}~
\scalebox{0.75}{%% Creator: Matplotlib, PGF backend
%%
%% To include the figure in your LaTeX document, write
%%   \input{<filename>.pgf}
%%
%% Make sure the required packages are loaded in your preamble
%%   \usepackage{pgf}
%%
%% Figures using additional raster images can only be included by \input if
%% they are in the same directory as the main LaTeX file. For loading figures
%% from other directories you can use the `import` package
%%   \usepackage{import}
%% and then include the figures with
%%   \import{<path to file>}{<filename>.pgf}
%%
%% Matplotlib used the following preamble
%%   \usepackage{fontspec}
%%   \setmainfont{DejaVuSerif.ttf}[Path=/home/lzh/anaconda3/envs/numana/lib/python3.7/site-packages/matplotlib/mpl-data/fonts/ttf/]
%%   \setsansfont{DejaVuSans.ttf}[Path=/home/lzh/anaconda3/envs/numana/lib/python3.7/site-packages/matplotlib/mpl-data/fonts/ttf/]
%%   \setmonofont{DejaVuSansMono.ttf}[Path=/home/lzh/anaconda3/envs/numana/lib/python3.7/site-packages/matplotlib/mpl-data/fonts/ttf/]
%%
\begingroup%
\makeatletter%
\begin{pgfpicture}%
\pgfpathrectangle{\pgfpointorigin}{\pgfqpoint{4.000000in}{4.000000in}}%
\pgfusepath{use as bounding box, clip}%
\begin{pgfscope}%
\pgfsetbuttcap%
\pgfsetmiterjoin%
\definecolor{currentfill}{rgb}{1.000000,1.000000,1.000000}%
\pgfsetfillcolor{currentfill}%
\pgfsetlinewidth{0.000000pt}%
\definecolor{currentstroke}{rgb}{1.000000,1.000000,1.000000}%
\pgfsetstrokecolor{currentstroke}%
\pgfsetdash{}{0pt}%
\pgfpathmoveto{\pgfqpoint{0.000000in}{0.000000in}}%
\pgfpathlineto{\pgfqpoint{4.000000in}{0.000000in}}%
\pgfpathlineto{\pgfqpoint{4.000000in}{4.000000in}}%
\pgfpathlineto{\pgfqpoint{0.000000in}{4.000000in}}%
\pgfpathclose%
\pgfusepath{fill}%
\end{pgfscope}%
\begin{pgfscope}%
\pgfsetbuttcap%
\pgfsetmiterjoin%
\definecolor{currentfill}{rgb}{1.000000,1.000000,1.000000}%
\pgfsetfillcolor{currentfill}%
\pgfsetlinewidth{0.000000pt}%
\definecolor{currentstroke}{rgb}{0.000000,0.000000,0.000000}%
\pgfsetstrokecolor{currentstroke}%
\pgfsetstrokeopacity{0.000000}%
\pgfsetdash{}{0pt}%
\pgfpathmoveto{\pgfqpoint{0.510000in}{0.440000in}}%
\pgfpathlineto{\pgfqpoint{3.590000in}{0.440000in}}%
\pgfpathlineto{\pgfqpoint{3.590000in}{3.520000in}}%
\pgfpathlineto{\pgfqpoint{0.510000in}{3.520000in}}%
\pgfpathclose%
\pgfusepath{fill}%
\end{pgfscope}%
\begin{pgfscope}%
\pgfpathrectangle{\pgfqpoint{0.510000in}{0.440000in}}{\pgfqpoint{3.080000in}{3.080000in}}%
\pgfusepath{clip}%
\pgfsetbuttcap%
\pgfsetroundjoin%
\definecolor{currentfill}{rgb}{0.121569,0.466667,0.705882}%
\pgfsetfillcolor{currentfill}%
\pgfsetlinewidth{1.003750pt}%
\definecolor{currentstroke}{rgb}{0.121569,0.466667,0.705882}%
\pgfsetstrokecolor{currentstroke}%
\pgfsetdash{}{0pt}%
\pgfpathmoveto{\pgfqpoint{2.362160in}{1.011691in}}%
\pgfpathcurveto{\pgfqpoint{2.367984in}{1.011691in}}{\pgfqpoint{2.373570in}{1.014005in}}{\pgfqpoint{2.377688in}{1.018123in}}%
\pgfpathcurveto{\pgfqpoint{2.381807in}{1.022241in}}{\pgfqpoint{2.384120in}{1.027827in}}{\pgfqpoint{2.384120in}{1.033651in}}%
\pgfpathcurveto{\pgfqpoint{2.384120in}{1.039475in}}{\pgfqpoint{2.381807in}{1.045061in}}{\pgfqpoint{2.377688in}{1.049179in}}%
\pgfpathcurveto{\pgfqpoint{2.373570in}{1.053297in}}{\pgfqpoint{2.367984in}{1.055611in}}{\pgfqpoint{2.362160in}{1.055611in}}%
\pgfpathcurveto{\pgfqpoint{2.356336in}{1.055611in}}{\pgfqpoint{2.350750in}{1.053297in}}{\pgfqpoint{2.346632in}{1.049179in}}%
\pgfpathcurveto{\pgfqpoint{2.342514in}{1.045061in}}{\pgfqpoint{2.340200in}{1.039475in}}{\pgfqpoint{2.340200in}{1.033651in}}%
\pgfpathcurveto{\pgfqpoint{2.340200in}{1.027827in}}{\pgfqpoint{2.342514in}{1.022241in}}{\pgfqpoint{2.346632in}{1.018123in}}%
\pgfpathcurveto{\pgfqpoint{2.350750in}{1.014005in}}{\pgfqpoint{2.356336in}{1.011691in}}{\pgfqpoint{2.362160in}{1.011691in}}%
\pgfpathclose%
\pgfusepath{stroke,fill}%
\end{pgfscope}%
\begin{pgfscope}%
\pgfpathrectangle{\pgfqpoint{0.510000in}{0.440000in}}{\pgfqpoint{3.080000in}{3.080000in}}%
\pgfusepath{clip}%
\pgfsetbuttcap%
\pgfsetroundjoin%
\definecolor{currentfill}{rgb}{1.000000,0.498039,0.054902}%
\pgfsetfillcolor{currentfill}%
\pgfsetlinewidth{1.003750pt}%
\definecolor{currentstroke}{rgb}{1.000000,0.498039,0.054902}%
\pgfsetstrokecolor{currentstroke}%
\pgfsetdash{}{0pt}%
\pgfpathmoveto{\pgfqpoint{1.584188in}{2.270199in}}%
\pgfpathcurveto{\pgfqpoint{1.590012in}{2.270199in}}{\pgfqpoint{1.595598in}{2.272513in}}{\pgfqpoint{1.599716in}{2.276631in}}%
\pgfpathcurveto{\pgfqpoint{1.603834in}{2.280749in}}{\pgfqpoint{1.606148in}{2.286335in}}{\pgfqpoint{1.606148in}{2.292159in}}%
\pgfpathcurveto{\pgfqpoint{1.606148in}{2.297983in}}{\pgfqpoint{1.603834in}{2.303569in}}{\pgfqpoint{1.599716in}{2.307688in}}%
\pgfpathcurveto{\pgfqpoint{1.595598in}{2.311806in}}{\pgfqpoint{1.590012in}{2.314120in}}{\pgfqpoint{1.584188in}{2.314120in}}%
\pgfpathcurveto{\pgfqpoint{1.578364in}{2.314120in}}{\pgfqpoint{1.572778in}{2.311806in}}{\pgfqpoint{1.568660in}{2.307688in}}%
\pgfpathcurveto{\pgfqpoint{1.564542in}{2.303569in}}{\pgfqpoint{1.562228in}{2.297983in}}{\pgfqpoint{1.562228in}{2.292159in}}%
\pgfpathcurveto{\pgfqpoint{1.562228in}{2.286335in}}{\pgfqpoint{1.564542in}{2.280749in}}{\pgfqpoint{1.568660in}{2.276631in}}%
\pgfpathcurveto{\pgfqpoint{1.572778in}{2.272513in}}{\pgfqpoint{1.578364in}{2.270199in}}{\pgfqpoint{1.584188in}{2.270199in}}%
\pgfpathclose%
\pgfusepath{stroke,fill}%
\end{pgfscope}%
\begin{pgfscope}%
\pgfpathrectangle{\pgfqpoint{0.510000in}{0.440000in}}{\pgfqpoint{3.080000in}{3.080000in}}%
\pgfusepath{clip}%
\pgfsetbuttcap%
\pgfsetroundjoin%
\definecolor{currentfill}{rgb}{0.172549,0.627451,0.172549}%
\pgfsetfillcolor{currentfill}%
\pgfsetlinewidth{1.003750pt}%
\definecolor{currentstroke}{rgb}{0.172549,0.627451,0.172549}%
\pgfsetstrokecolor{currentstroke}%
\pgfsetdash{}{0pt}%
\pgfpathmoveto{\pgfqpoint{1.103652in}{1.492230in}}%
\pgfpathcurveto{\pgfqpoint{1.109476in}{1.492230in}}{\pgfqpoint{1.115062in}{1.494544in}}{\pgfqpoint{1.119181in}{1.498662in}}%
\pgfpathcurveto{\pgfqpoint{1.123299in}{1.502780in}}{\pgfqpoint{1.125613in}{1.508366in}}{\pgfqpoint{1.125613in}{1.514190in}}%
\pgfpathcurveto{\pgfqpoint{1.125613in}{1.520014in}}{\pgfqpoint{1.123299in}{1.525600in}}{\pgfqpoint{1.119181in}{1.529718in}}%
\pgfpathcurveto{\pgfqpoint{1.115062in}{1.533837in}}{\pgfqpoint{1.109476in}{1.536150in}}{\pgfqpoint{1.103652in}{1.536150in}}%
\pgfpathcurveto{\pgfqpoint{1.097828in}{1.536150in}}{\pgfqpoint{1.092242in}{1.533837in}}{\pgfqpoint{1.088124in}{1.529718in}}%
\pgfpathcurveto{\pgfqpoint{1.084006in}{1.525600in}}{\pgfqpoint{1.081692in}{1.520014in}}{\pgfqpoint{1.081692in}{1.514190in}}%
\pgfpathcurveto{\pgfqpoint{1.081692in}{1.508366in}}{\pgfqpoint{1.084006in}{1.502780in}}{\pgfqpoint{1.088124in}{1.498662in}}%
\pgfpathcurveto{\pgfqpoint{1.092242in}{1.494544in}}{\pgfqpoint{1.097828in}{1.492230in}}{\pgfqpoint{1.103652in}{1.492230in}}%
\pgfpathclose%
\pgfusepath{stroke,fill}%
\end{pgfscope}%
\begin{pgfscope}%
\pgfpathrectangle{\pgfqpoint{0.510000in}{0.440000in}}{\pgfqpoint{3.080000in}{3.080000in}}%
\pgfusepath{clip}%
\pgfsetbuttcap%
\pgfsetmiterjoin%
\definecolor{currentfill}{rgb}{0.100000,0.100000,0.100000}%
\pgfsetfillcolor{currentfill}%
\pgfsetfillopacity{0.100000}%
\pgfsetlinewidth{0.301125pt}%
\definecolor{currentstroke}{rgb}{0.000000,0.000000,0.000000}%
\pgfsetstrokecolor{currentstroke}%
\pgfsetdash{}{0pt}%
\pgfpathmoveto{\pgfqpoint{0.950000in}{0.880000in}}%
\pgfpathlineto{\pgfqpoint{0.950000in}{3.080000in}}%
\pgfpathlineto{\pgfqpoint{3.150000in}{0.880000in}}%
\pgfpathclose%
\pgfusepath{stroke,fill}%
\end{pgfscope}%
\begin{pgfscope}%
\pgfpathrectangle{\pgfqpoint{0.510000in}{0.440000in}}{\pgfqpoint{3.080000in}{3.080000in}}%
\pgfusepath{clip}%
\pgfsetrectcap%
\pgfsetroundjoin%
\pgfsetlinewidth{0.803000pt}%
\definecolor{currentstroke}{rgb}{0.690196,0.690196,0.690196}%
\pgfsetstrokecolor{currentstroke}%
\pgfsetdash{}{0pt}%
\pgfpathmoveto{\pgfqpoint{0.510000in}{0.440000in}}%
\pgfpathlineto{\pgfqpoint{0.510000in}{3.520000in}}%
\pgfusepath{stroke}%
\end{pgfscope}%
\begin{pgfscope}%
\pgfsetbuttcap%
\pgfsetroundjoin%
\definecolor{currentfill}{rgb}{0.000000,0.000000,0.000000}%
\pgfsetfillcolor{currentfill}%
\pgfsetlinewidth{0.803000pt}%
\definecolor{currentstroke}{rgb}{0.000000,0.000000,0.000000}%
\pgfsetstrokecolor{currentstroke}%
\pgfsetdash{}{0pt}%
\pgfsys@defobject{currentmarker}{\pgfqpoint{0.000000in}{-0.048611in}}{\pgfqpoint{0.000000in}{0.000000in}}{%
\pgfpathmoveto{\pgfqpoint{0.000000in}{0.000000in}}%
\pgfpathlineto{\pgfqpoint{0.000000in}{-0.048611in}}%
\pgfusepath{stroke,fill}%
}%
\begin{pgfscope}%
\pgfsys@transformshift{0.510000in}{0.440000in}%
\pgfsys@useobject{currentmarker}{}%
\end{pgfscope}%
\end{pgfscope}%
\begin{pgfscope}%
\definecolor{textcolor}{rgb}{0.000000,0.000000,0.000000}%
\pgfsetstrokecolor{textcolor}%
\pgfsetfillcolor{textcolor}%
\pgftext[x=0.510000in,y=0.342778in,,top]{\color{textcolor}\sffamily\fontsize{10.000000}{12.000000}\selectfont -0.2}%
\end{pgfscope}%
\begin{pgfscope}%
\pgfpathrectangle{\pgfqpoint{0.510000in}{0.440000in}}{\pgfqpoint{3.080000in}{3.080000in}}%
\pgfusepath{clip}%
\pgfsetrectcap%
\pgfsetroundjoin%
\pgfsetlinewidth{0.803000pt}%
\definecolor{currentstroke}{rgb}{0.690196,0.690196,0.690196}%
\pgfsetstrokecolor{currentstroke}%
\pgfsetdash{}{0pt}%
\pgfpathmoveto{\pgfqpoint{0.950000in}{0.440000in}}%
\pgfpathlineto{\pgfqpoint{0.950000in}{3.520000in}}%
\pgfusepath{stroke}%
\end{pgfscope}%
\begin{pgfscope}%
\pgfsetbuttcap%
\pgfsetroundjoin%
\definecolor{currentfill}{rgb}{0.000000,0.000000,0.000000}%
\pgfsetfillcolor{currentfill}%
\pgfsetlinewidth{0.803000pt}%
\definecolor{currentstroke}{rgb}{0.000000,0.000000,0.000000}%
\pgfsetstrokecolor{currentstroke}%
\pgfsetdash{}{0pt}%
\pgfsys@defobject{currentmarker}{\pgfqpoint{0.000000in}{-0.048611in}}{\pgfqpoint{0.000000in}{0.000000in}}{%
\pgfpathmoveto{\pgfqpoint{0.000000in}{0.000000in}}%
\pgfpathlineto{\pgfqpoint{0.000000in}{-0.048611in}}%
\pgfusepath{stroke,fill}%
}%
\begin{pgfscope}%
\pgfsys@transformshift{0.950000in}{0.440000in}%
\pgfsys@useobject{currentmarker}{}%
\end{pgfscope}%
\end{pgfscope}%
\begin{pgfscope}%
\definecolor{textcolor}{rgb}{0.000000,0.000000,0.000000}%
\pgfsetstrokecolor{textcolor}%
\pgfsetfillcolor{textcolor}%
\pgftext[x=0.950000in,y=0.342778in,,top]{\color{textcolor}\sffamily\fontsize{10.000000}{12.000000}\selectfont 0.0}%
\end{pgfscope}%
\begin{pgfscope}%
\pgfpathrectangle{\pgfqpoint{0.510000in}{0.440000in}}{\pgfqpoint{3.080000in}{3.080000in}}%
\pgfusepath{clip}%
\pgfsetrectcap%
\pgfsetroundjoin%
\pgfsetlinewidth{0.803000pt}%
\definecolor{currentstroke}{rgb}{0.690196,0.690196,0.690196}%
\pgfsetstrokecolor{currentstroke}%
\pgfsetdash{}{0pt}%
\pgfpathmoveto{\pgfqpoint{1.390000in}{0.440000in}}%
\pgfpathlineto{\pgfqpoint{1.390000in}{3.520000in}}%
\pgfusepath{stroke}%
\end{pgfscope}%
\begin{pgfscope}%
\pgfsetbuttcap%
\pgfsetroundjoin%
\definecolor{currentfill}{rgb}{0.000000,0.000000,0.000000}%
\pgfsetfillcolor{currentfill}%
\pgfsetlinewidth{0.803000pt}%
\definecolor{currentstroke}{rgb}{0.000000,0.000000,0.000000}%
\pgfsetstrokecolor{currentstroke}%
\pgfsetdash{}{0pt}%
\pgfsys@defobject{currentmarker}{\pgfqpoint{0.000000in}{-0.048611in}}{\pgfqpoint{0.000000in}{0.000000in}}{%
\pgfpathmoveto{\pgfqpoint{0.000000in}{0.000000in}}%
\pgfpathlineto{\pgfqpoint{0.000000in}{-0.048611in}}%
\pgfusepath{stroke,fill}%
}%
\begin{pgfscope}%
\pgfsys@transformshift{1.390000in}{0.440000in}%
\pgfsys@useobject{currentmarker}{}%
\end{pgfscope}%
\end{pgfscope}%
\begin{pgfscope}%
\definecolor{textcolor}{rgb}{0.000000,0.000000,0.000000}%
\pgfsetstrokecolor{textcolor}%
\pgfsetfillcolor{textcolor}%
\pgftext[x=1.390000in,y=0.342778in,,top]{\color{textcolor}\sffamily\fontsize{10.000000}{12.000000}\selectfont 0.2}%
\end{pgfscope}%
\begin{pgfscope}%
\pgfpathrectangle{\pgfqpoint{0.510000in}{0.440000in}}{\pgfqpoint{3.080000in}{3.080000in}}%
\pgfusepath{clip}%
\pgfsetrectcap%
\pgfsetroundjoin%
\pgfsetlinewidth{0.803000pt}%
\definecolor{currentstroke}{rgb}{0.690196,0.690196,0.690196}%
\pgfsetstrokecolor{currentstroke}%
\pgfsetdash{}{0pt}%
\pgfpathmoveto{\pgfqpoint{1.830000in}{0.440000in}}%
\pgfpathlineto{\pgfqpoint{1.830000in}{3.520000in}}%
\pgfusepath{stroke}%
\end{pgfscope}%
\begin{pgfscope}%
\pgfsetbuttcap%
\pgfsetroundjoin%
\definecolor{currentfill}{rgb}{0.000000,0.000000,0.000000}%
\pgfsetfillcolor{currentfill}%
\pgfsetlinewidth{0.803000pt}%
\definecolor{currentstroke}{rgb}{0.000000,0.000000,0.000000}%
\pgfsetstrokecolor{currentstroke}%
\pgfsetdash{}{0pt}%
\pgfsys@defobject{currentmarker}{\pgfqpoint{0.000000in}{-0.048611in}}{\pgfqpoint{0.000000in}{0.000000in}}{%
\pgfpathmoveto{\pgfqpoint{0.000000in}{0.000000in}}%
\pgfpathlineto{\pgfqpoint{0.000000in}{-0.048611in}}%
\pgfusepath{stroke,fill}%
}%
\begin{pgfscope}%
\pgfsys@transformshift{1.830000in}{0.440000in}%
\pgfsys@useobject{currentmarker}{}%
\end{pgfscope}%
\end{pgfscope}%
\begin{pgfscope}%
\definecolor{textcolor}{rgb}{0.000000,0.000000,0.000000}%
\pgfsetstrokecolor{textcolor}%
\pgfsetfillcolor{textcolor}%
\pgftext[x=1.830000in,y=0.342778in,,top]{\color{textcolor}\sffamily\fontsize{10.000000}{12.000000}\selectfont 0.4}%
\end{pgfscope}%
\begin{pgfscope}%
\pgfpathrectangle{\pgfqpoint{0.510000in}{0.440000in}}{\pgfqpoint{3.080000in}{3.080000in}}%
\pgfusepath{clip}%
\pgfsetrectcap%
\pgfsetroundjoin%
\pgfsetlinewidth{0.803000pt}%
\definecolor{currentstroke}{rgb}{0.690196,0.690196,0.690196}%
\pgfsetstrokecolor{currentstroke}%
\pgfsetdash{}{0pt}%
\pgfpathmoveto{\pgfqpoint{2.270000in}{0.440000in}}%
\pgfpathlineto{\pgfqpoint{2.270000in}{3.520000in}}%
\pgfusepath{stroke}%
\end{pgfscope}%
\begin{pgfscope}%
\pgfsetbuttcap%
\pgfsetroundjoin%
\definecolor{currentfill}{rgb}{0.000000,0.000000,0.000000}%
\pgfsetfillcolor{currentfill}%
\pgfsetlinewidth{0.803000pt}%
\definecolor{currentstroke}{rgb}{0.000000,0.000000,0.000000}%
\pgfsetstrokecolor{currentstroke}%
\pgfsetdash{}{0pt}%
\pgfsys@defobject{currentmarker}{\pgfqpoint{0.000000in}{-0.048611in}}{\pgfqpoint{0.000000in}{0.000000in}}{%
\pgfpathmoveto{\pgfqpoint{0.000000in}{0.000000in}}%
\pgfpathlineto{\pgfqpoint{0.000000in}{-0.048611in}}%
\pgfusepath{stroke,fill}%
}%
\begin{pgfscope}%
\pgfsys@transformshift{2.270000in}{0.440000in}%
\pgfsys@useobject{currentmarker}{}%
\end{pgfscope}%
\end{pgfscope}%
\begin{pgfscope}%
\definecolor{textcolor}{rgb}{0.000000,0.000000,0.000000}%
\pgfsetstrokecolor{textcolor}%
\pgfsetfillcolor{textcolor}%
\pgftext[x=2.270000in,y=0.342778in,,top]{\color{textcolor}\sffamily\fontsize{10.000000}{12.000000}\selectfont 0.6}%
\end{pgfscope}%
\begin{pgfscope}%
\pgfpathrectangle{\pgfqpoint{0.510000in}{0.440000in}}{\pgfqpoint{3.080000in}{3.080000in}}%
\pgfusepath{clip}%
\pgfsetrectcap%
\pgfsetroundjoin%
\pgfsetlinewidth{0.803000pt}%
\definecolor{currentstroke}{rgb}{0.690196,0.690196,0.690196}%
\pgfsetstrokecolor{currentstroke}%
\pgfsetdash{}{0pt}%
\pgfpathmoveto{\pgfqpoint{2.710000in}{0.440000in}}%
\pgfpathlineto{\pgfqpoint{2.710000in}{3.520000in}}%
\pgfusepath{stroke}%
\end{pgfscope}%
\begin{pgfscope}%
\pgfsetbuttcap%
\pgfsetroundjoin%
\definecolor{currentfill}{rgb}{0.000000,0.000000,0.000000}%
\pgfsetfillcolor{currentfill}%
\pgfsetlinewidth{0.803000pt}%
\definecolor{currentstroke}{rgb}{0.000000,0.000000,0.000000}%
\pgfsetstrokecolor{currentstroke}%
\pgfsetdash{}{0pt}%
\pgfsys@defobject{currentmarker}{\pgfqpoint{0.000000in}{-0.048611in}}{\pgfqpoint{0.000000in}{0.000000in}}{%
\pgfpathmoveto{\pgfqpoint{0.000000in}{0.000000in}}%
\pgfpathlineto{\pgfqpoint{0.000000in}{-0.048611in}}%
\pgfusepath{stroke,fill}%
}%
\begin{pgfscope}%
\pgfsys@transformshift{2.710000in}{0.440000in}%
\pgfsys@useobject{currentmarker}{}%
\end{pgfscope}%
\end{pgfscope}%
\begin{pgfscope}%
\definecolor{textcolor}{rgb}{0.000000,0.000000,0.000000}%
\pgfsetstrokecolor{textcolor}%
\pgfsetfillcolor{textcolor}%
\pgftext[x=2.710000in,y=0.342778in,,top]{\color{textcolor}\sffamily\fontsize{10.000000}{12.000000}\selectfont 0.8}%
\end{pgfscope}%
\begin{pgfscope}%
\pgfpathrectangle{\pgfqpoint{0.510000in}{0.440000in}}{\pgfqpoint{3.080000in}{3.080000in}}%
\pgfusepath{clip}%
\pgfsetrectcap%
\pgfsetroundjoin%
\pgfsetlinewidth{0.803000pt}%
\definecolor{currentstroke}{rgb}{0.690196,0.690196,0.690196}%
\pgfsetstrokecolor{currentstroke}%
\pgfsetdash{}{0pt}%
\pgfpathmoveto{\pgfqpoint{3.150000in}{0.440000in}}%
\pgfpathlineto{\pgfqpoint{3.150000in}{3.520000in}}%
\pgfusepath{stroke}%
\end{pgfscope}%
\begin{pgfscope}%
\pgfsetbuttcap%
\pgfsetroundjoin%
\definecolor{currentfill}{rgb}{0.000000,0.000000,0.000000}%
\pgfsetfillcolor{currentfill}%
\pgfsetlinewidth{0.803000pt}%
\definecolor{currentstroke}{rgb}{0.000000,0.000000,0.000000}%
\pgfsetstrokecolor{currentstroke}%
\pgfsetdash{}{0pt}%
\pgfsys@defobject{currentmarker}{\pgfqpoint{0.000000in}{-0.048611in}}{\pgfqpoint{0.000000in}{0.000000in}}{%
\pgfpathmoveto{\pgfqpoint{0.000000in}{0.000000in}}%
\pgfpathlineto{\pgfqpoint{0.000000in}{-0.048611in}}%
\pgfusepath{stroke,fill}%
}%
\begin{pgfscope}%
\pgfsys@transformshift{3.150000in}{0.440000in}%
\pgfsys@useobject{currentmarker}{}%
\end{pgfscope}%
\end{pgfscope}%
\begin{pgfscope}%
\definecolor{textcolor}{rgb}{0.000000,0.000000,0.000000}%
\pgfsetstrokecolor{textcolor}%
\pgfsetfillcolor{textcolor}%
\pgftext[x=3.150000in,y=0.342778in,,top]{\color{textcolor}\sffamily\fontsize{10.000000}{12.000000}\selectfont 1.0}%
\end{pgfscope}%
\begin{pgfscope}%
\pgfpathrectangle{\pgfqpoint{0.510000in}{0.440000in}}{\pgfqpoint{3.080000in}{3.080000in}}%
\pgfusepath{clip}%
\pgfsetrectcap%
\pgfsetroundjoin%
\pgfsetlinewidth{0.803000pt}%
\definecolor{currentstroke}{rgb}{0.690196,0.690196,0.690196}%
\pgfsetstrokecolor{currentstroke}%
\pgfsetdash{}{0pt}%
\pgfpathmoveto{\pgfqpoint{3.590000in}{0.440000in}}%
\pgfpathlineto{\pgfqpoint{3.590000in}{3.520000in}}%
\pgfusepath{stroke}%
\end{pgfscope}%
\begin{pgfscope}%
\pgfsetbuttcap%
\pgfsetroundjoin%
\definecolor{currentfill}{rgb}{0.000000,0.000000,0.000000}%
\pgfsetfillcolor{currentfill}%
\pgfsetlinewidth{0.803000pt}%
\definecolor{currentstroke}{rgb}{0.000000,0.000000,0.000000}%
\pgfsetstrokecolor{currentstroke}%
\pgfsetdash{}{0pt}%
\pgfsys@defobject{currentmarker}{\pgfqpoint{0.000000in}{-0.048611in}}{\pgfqpoint{0.000000in}{0.000000in}}{%
\pgfpathmoveto{\pgfqpoint{0.000000in}{0.000000in}}%
\pgfpathlineto{\pgfqpoint{0.000000in}{-0.048611in}}%
\pgfusepath{stroke,fill}%
}%
\begin{pgfscope}%
\pgfsys@transformshift{3.590000in}{0.440000in}%
\pgfsys@useobject{currentmarker}{}%
\end{pgfscope}%
\end{pgfscope}%
\begin{pgfscope}%
\definecolor{textcolor}{rgb}{0.000000,0.000000,0.000000}%
\pgfsetstrokecolor{textcolor}%
\pgfsetfillcolor{textcolor}%
\pgftext[x=3.590000in,y=0.342778in,,top]{\color{textcolor}\sffamily\fontsize{10.000000}{12.000000}\selectfont 1.2}%
\end{pgfscope}%
\begin{pgfscope}%
\pgfpathrectangle{\pgfqpoint{0.510000in}{0.440000in}}{\pgfqpoint{3.080000in}{3.080000in}}%
\pgfusepath{clip}%
\pgfsetrectcap%
\pgfsetroundjoin%
\pgfsetlinewidth{0.803000pt}%
\definecolor{currentstroke}{rgb}{0.690196,0.690196,0.690196}%
\pgfsetstrokecolor{currentstroke}%
\pgfsetdash{}{0pt}%
\pgfpathmoveto{\pgfqpoint{0.510000in}{0.440000in}}%
\pgfpathlineto{\pgfqpoint{3.590000in}{0.440000in}}%
\pgfusepath{stroke}%
\end{pgfscope}%
\begin{pgfscope}%
\pgfsetbuttcap%
\pgfsetroundjoin%
\definecolor{currentfill}{rgb}{0.000000,0.000000,0.000000}%
\pgfsetfillcolor{currentfill}%
\pgfsetlinewidth{0.803000pt}%
\definecolor{currentstroke}{rgb}{0.000000,0.000000,0.000000}%
\pgfsetstrokecolor{currentstroke}%
\pgfsetdash{}{0pt}%
\pgfsys@defobject{currentmarker}{\pgfqpoint{-0.048611in}{0.000000in}}{\pgfqpoint{0.000000in}{0.000000in}}{%
\pgfpathmoveto{\pgfqpoint{0.000000in}{0.000000in}}%
\pgfpathlineto{\pgfqpoint{-0.048611in}{0.000000in}}%
\pgfusepath{stroke,fill}%
}%
\begin{pgfscope}%
\pgfsys@transformshift{0.510000in}{0.440000in}%
\pgfsys@useobject{currentmarker}{}%
\end{pgfscope}%
\end{pgfscope}%
\begin{pgfscope}%
\definecolor{textcolor}{rgb}{0.000000,0.000000,0.000000}%
\pgfsetstrokecolor{textcolor}%
\pgfsetfillcolor{textcolor}%
\pgftext[x=0.075525in,y=0.387238in,left,base]{\color{textcolor}\sffamily\fontsize{10.000000}{12.000000}\selectfont -0.2}%
\end{pgfscope}%
\begin{pgfscope}%
\pgfpathrectangle{\pgfqpoint{0.510000in}{0.440000in}}{\pgfqpoint{3.080000in}{3.080000in}}%
\pgfusepath{clip}%
\pgfsetrectcap%
\pgfsetroundjoin%
\pgfsetlinewidth{0.803000pt}%
\definecolor{currentstroke}{rgb}{0.690196,0.690196,0.690196}%
\pgfsetstrokecolor{currentstroke}%
\pgfsetdash{}{0pt}%
\pgfpathmoveto{\pgfqpoint{0.510000in}{0.880000in}}%
\pgfpathlineto{\pgfqpoint{3.590000in}{0.880000in}}%
\pgfusepath{stroke}%
\end{pgfscope}%
\begin{pgfscope}%
\pgfsetbuttcap%
\pgfsetroundjoin%
\definecolor{currentfill}{rgb}{0.000000,0.000000,0.000000}%
\pgfsetfillcolor{currentfill}%
\pgfsetlinewidth{0.803000pt}%
\definecolor{currentstroke}{rgb}{0.000000,0.000000,0.000000}%
\pgfsetstrokecolor{currentstroke}%
\pgfsetdash{}{0pt}%
\pgfsys@defobject{currentmarker}{\pgfqpoint{-0.048611in}{0.000000in}}{\pgfqpoint{0.000000in}{0.000000in}}{%
\pgfpathmoveto{\pgfqpoint{0.000000in}{0.000000in}}%
\pgfpathlineto{\pgfqpoint{-0.048611in}{0.000000in}}%
\pgfusepath{stroke,fill}%
}%
\begin{pgfscope}%
\pgfsys@transformshift{0.510000in}{0.880000in}%
\pgfsys@useobject{currentmarker}{}%
\end{pgfscope}%
\end{pgfscope}%
\begin{pgfscope}%
\definecolor{textcolor}{rgb}{0.000000,0.000000,0.000000}%
\pgfsetstrokecolor{textcolor}%
\pgfsetfillcolor{textcolor}%
\pgftext[x=0.191898in,y=0.827238in,left,base]{\color{textcolor}\sffamily\fontsize{10.000000}{12.000000}\selectfont 0.0}%
\end{pgfscope}%
\begin{pgfscope}%
\pgfpathrectangle{\pgfqpoint{0.510000in}{0.440000in}}{\pgfqpoint{3.080000in}{3.080000in}}%
\pgfusepath{clip}%
\pgfsetrectcap%
\pgfsetroundjoin%
\pgfsetlinewidth{0.803000pt}%
\definecolor{currentstroke}{rgb}{0.690196,0.690196,0.690196}%
\pgfsetstrokecolor{currentstroke}%
\pgfsetdash{}{0pt}%
\pgfpathmoveto{\pgfqpoint{0.510000in}{1.320000in}}%
\pgfpathlineto{\pgfqpoint{3.590000in}{1.320000in}}%
\pgfusepath{stroke}%
\end{pgfscope}%
\begin{pgfscope}%
\pgfsetbuttcap%
\pgfsetroundjoin%
\definecolor{currentfill}{rgb}{0.000000,0.000000,0.000000}%
\pgfsetfillcolor{currentfill}%
\pgfsetlinewidth{0.803000pt}%
\definecolor{currentstroke}{rgb}{0.000000,0.000000,0.000000}%
\pgfsetstrokecolor{currentstroke}%
\pgfsetdash{}{0pt}%
\pgfsys@defobject{currentmarker}{\pgfqpoint{-0.048611in}{0.000000in}}{\pgfqpoint{0.000000in}{0.000000in}}{%
\pgfpathmoveto{\pgfqpoint{0.000000in}{0.000000in}}%
\pgfpathlineto{\pgfqpoint{-0.048611in}{0.000000in}}%
\pgfusepath{stroke,fill}%
}%
\begin{pgfscope}%
\pgfsys@transformshift{0.510000in}{1.320000in}%
\pgfsys@useobject{currentmarker}{}%
\end{pgfscope}%
\end{pgfscope}%
\begin{pgfscope}%
\definecolor{textcolor}{rgb}{0.000000,0.000000,0.000000}%
\pgfsetstrokecolor{textcolor}%
\pgfsetfillcolor{textcolor}%
\pgftext[x=0.191898in,y=1.267238in,left,base]{\color{textcolor}\sffamily\fontsize{10.000000}{12.000000}\selectfont 0.2}%
\end{pgfscope}%
\begin{pgfscope}%
\pgfpathrectangle{\pgfqpoint{0.510000in}{0.440000in}}{\pgfqpoint{3.080000in}{3.080000in}}%
\pgfusepath{clip}%
\pgfsetrectcap%
\pgfsetroundjoin%
\pgfsetlinewidth{0.803000pt}%
\definecolor{currentstroke}{rgb}{0.690196,0.690196,0.690196}%
\pgfsetstrokecolor{currentstroke}%
\pgfsetdash{}{0pt}%
\pgfpathmoveto{\pgfqpoint{0.510000in}{1.760000in}}%
\pgfpathlineto{\pgfqpoint{3.590000in}{1.760000in}}%
\pgfusepath{stroke}%
\end{pgfscope}%
\begin{pgfscope}%
\pgfsetbuttcap%
\pgfsetroundjoin%
\definecolor{currentfill}{rgb}{0.000000,0.000000,0.000000}%
\pgfsetfillcolor{currentfill}%
\pgfsetlinewidth{0.803000pt}%
\definecolor{currentstroke}{rgb}{0.000000,0.000000,0.000000}%
\pgfsetstrokecolor{currentstroke}%
\pgfsetdash{}{0pt}%
\pgfsys@defobject{currentmarker}{\pgfqpoint{-0.048611in}{0.000000in}}{\pgfqpoint{0.000000in}{0.000000in}}{%
\pgfpathmoveto{\pgfqpoint{0.000000in}{0.000000in}}%
\pgfpathlineto{\pgfqpoint{-0.048611in}{0.000000in}}%
\pgfusepath{stroke,fill}%
}%
\begin{pgfscope}%
\pgfsys@transformshift{0.510000in}{1.760000in}%
\pgfsys@useobject{currentmarker}{}%
\end{pgfscope}%
\end{pgfscope}%
\begin{pgfscope}%
\definecolor{textcolor}{rgb}{0.000000,0.000000,0.000000}%
\pgfsetstrokecolor{textcolor}%
\pgfsetfillcolor{textcolor}%
\pgftext[x=0.191898in,y=1.707238in,left,base]{\color{textcolor}\sffamily\fontsize{10.000000}{12.000000}\selectfont 0.4}%
\end{pgfscope}%
\begin{pgfscope}%
\pgfpathrectangle{\pgfqpoint{0.510000in}{0.440000in}}{\pgfqpoint{3.080000in}{3.080000in}}%
\pgfusepath{clip}%
\pgfsetrectcap%
\pgfsetroundjoin%
\pgfsetlinewidth{0.803000pt}%
\definecolor{currentstroke}{rgb}{0.690196,0.690196,0.690196}%
\pgfsetstrokecolor{currentstroke}%
\pgfsetdash{}{0pt}%
\pgfpathmoveto{\pgfqpoint{0.510000in}{2.200000in}}%
\pgfpathlineto{\pgfqpoint{3.590000in}{2.200000in}}%
\pgfusepath{stroke}%
\end{pgfscope}%
\begin{pgfscope}%
\pgfsetbuttcap%
\pgfsetroundjoin%
\definecolor{currentfill}{rgb}{0.000000,0.000000,0.000000}%
\pgfsetfillcolor{currentfill}%
\pgfsetlinewidth{0.803000pt}%
\definecolor{currentstroke}{rgb}{0.000000,0.000000,0.000000}%
\pgfsetstrokecolor{currentstroke}%
\pgfsetdash{}{0pt}%
\pgfsys@defobject{currentmarker}{\pgfqpoint{-0.048611in}{0.000000in}}{\pgfqpoint{0.000000in}{0.000000in}}{%
\pgfpathmoveto{\pgfqpoint{0.000000in}{0.000000in}}%
\pgfpathlineto{\pgfqpoint{-0.048611in}{0.000000in}}%
\pgfusepath{stroke,fill}%
}%
\begin{pgfscope}%
\pgfsys@transformshift{0.510000in}{2.200000in}%
\pgfsys@useobject{currentmarker}{}%
\end{pgfscope}%
\end{pgfscope}%
\begin{pgfscope}%
\definecolor{textcolor}{rgb}{0.000000,0.000000,0.000000}%
\pgfsetstrokecolor{textcolor}%
\pgfsetfillcolor{textcolor}%
\pgftext[x=0.191898in,y=2.147238in,left,base]{\color{textcolor}\sffamily\fontsize{10.000000}{12.000000}\selectfont 0.6}%
\end{pgfscope}%
\begin{pgfscope}%
\pgfpathrectangle{\pgfqpoint{0.510000in}{0.440000in}}{\pgfqpoint{3.080000in}{3.080000in}}%
\pgfusepath{clip}%
\pgfsetrectcap%
\pgfsetroundjoin%
\pgfsetlinewidth{0.803000pt}%
\definecolor{currentstroke}{rgb}{0.690196,0.690196,0.690196}%
\pgfsetstrokecolor{currentstroke}%
\pgfsetdash{}{0pt}%
\pgfpathmoveto{\pgfqpoint{0.510000in}{2.640000in}}%
\pgfpathlineto{\pgfqpoint{3.590000in}{2.640000in}}%
\pgfusepath{stroke}%
\end{pgfscope}%
\begin{pgfscope}%
\pgfsetbuttcap%
\pgfsetroundjoin%
\definecolor{currentfill}{rgb}{0.000000,0.000000,0.000000}%
\pgfsetfillcolor{currentfill}%
\pgfsetlinewidth{0.803000pt}%
\definecolor{currentstroke}{rgb}{0.000000,0.000000,0.000000}%
\pgfsetstrokecolor{currentstroke}%
\pgfsetdash{}{0pt}%
\pgfsys@defobject{currentmarker}{\pgfqpoint{-0.048611in}{0.000000in}}{\pgfqpoint{0.000000in}{0.000000in}}{%
\pgfpathmoveto{\pgfqpoint{0.000000in}{0.000000in}}%
\pgfpathlineto{\pgfqpoint{-0.048611in}{0.000000in}}%
\pgfusepath{stroke,fill}%
}%
\begin{pgfscope}%
\pgfsys@transformshift{0.510000in}{2.640000in}%
\pgfsys@useobject{currentmarker}{}%
\end{pgfscope}%
\end{pgfscope}%
\begin{pgfscope}%
\definecolor{textcolor}{rgb}{0.000000,0.000000,0.000000}%
\pgfsetstrokecolor{textcolor}%
\pgfsetfillcolor{textcolor}%
\pgftext[x=0.191898in,y=2.587238in,left,base]{\color{textcolor}\sffamily\fontsize{10.000000}{12.000000}\selectfont 0.8}%
\end{pgfscope}%
\begin{pgfscope}%
\pgfpathrectangle{\pgfqpoint{0.510000in}{0.440000in}}{\pgfqpoint{3.080000in}{3.080000in}}%
\pgfusepath{clip}%
\pgfsetrectcap%
\pgfsetroundjoin%
\pgfsetlinewidth{0.803000pt}%
\definecolor{currentstroke}{rgb}{0.690196,0.690196,0.690196}%
\pgfsetstrokecolor{currentstroke}%
\pgfsetdash{}{0pt}%
\pgfpathmoveto{\pgfqpoint{0.510000in}{3.080000in}}%
\pgfpathlineto{\pgfqpoint{3.590000in}{3.080000in}}%
\pgfusepath{stroke}%
\end{pgfscope}%
\begin{pgfscope}%
\pgfsetbuttcap%
\pgfsetroundjoin%
\definecolor{currentfill}{rgb}{0.000000,0.000000,0.000000}%
\pgfsetfillcolor{currentfill}%
\pgfsetlinewidth{0.803000pt}%
\definecolor{currentstroke}{rgb}{0.000000,0.000000,0.000000}%
\pgfsetstrokecolor{currentstroke}%
\pgfsetdash{}{0pt}%
\pgfsys@defobject{currentmarker}{\pgfqpoint{-0.048611in}{0.000000in}}{\pgfqpoint{0.000000in}{0.000000in}}{%
\pgfpathmoveto{\pgfqpoint{0.000000in}{0.000000in}}%
\pgfpathlineto{\pgfqpoint{-0.048611in}{0.000000in}}%
\pgfusepath{stroke,fill}%
}%
\begin{pgfscope}%
\pgfsys@transformshift{0.510000in}{3.080000in}%
\pgfsys@useobject{currentmarker}{}%
\end{pgfscope}%
\end{pgfscope}%
\begin{pgfscope}%
\definecolor{textcolor}{rgb}{0.000000,0.000000,0.000000}%
\pgfsetstrokecolor{textcolor}%
\pgfsetfillcolor{textcolor}%
\pgftext[x=0.191898in,y=3.027238in,left,base]{\color{textcolor}\sffamily\fontsize{10.000000}{12.000000}\selectfont 1.0}%
\end{pgfscope}%
\begin{pgfscope}%
\pgfpathrectangle{\pgfqpoint{0.510000in}{0.440000in}}{\pgfqpoint{3.080000in}{3.080000in}}%
\pgfusepath{clip}%
\pgfsetrectcap%
\pgfsetroundjoin%
\pgfsetlinewidth{0.803000pt}%
\definecolor{currentstroke}{rgb}{0.690196,0.690196,0.690196}%
\pgfsetstrokecolor{currentstroke}%
\pgfsetdash{}{0pt}%
\pgfpathmoveto{\pgfqpoint{0.510000in}{3.520000in}}%
\pgfpathlineto{\pgfqpoint{3.590000in}{3.520000in}}%
\pgfusepath{stroke}%
\end{pgfscope}%
\begin{pgfscope}%
\pgfsetbuttcap%
\pgfsetroundjoin%
\definecolor{currentfill}{rgb}{0.000000,0.000000,0.000000}%
\pgfsetfillcolor{currentfill}%
\pgfsetlinewidth{0.803000pt}%
\definecolor{currentstroke}{rgb}{0.000000,0.000000,0.000000}%
\pgfsetstrokecolor{currentstroke}%
\pgfsetdash{}{0pt}%
\pgfsys@defobject{currentmarker}{\pgfqpoint{-0.048611in}{0.000000in}}{\pgfqpoint{0.000000in}{0.000000in}}{%
\pgfpathmoveto{\pgfqpoint{0.000000in}{0.000000in}}%
\pgfpathlineto{\pgfqpoint{-0.048611in}{0.000000in}}%
\pgfusepath{stroke,fill}%
}%
\begin{pgfscope}%
\pgfsys@transformshift{0.510000in}{3.520000in}%
\pgfsys@useobject{currentmarker}{}%
\end{pgfscope}%
\end{pgfscope}%
\begin{pgfscope}%
\definecolor{textcolor}{rgb}{0.000000,0.000000,0.000000}%
\pgfsetstrokecolor{textcolor}%
\pgfsetfillcolor{textcolor}%
\pgftext[x=0.191898in,y=3.467238in,left,base]{\color{textcolor}\sffamily\fontsize{10.000000}{12.000000}\selectfont 1.2}%
\end{pgfscope}%
\begin{pgfscope}%
\pgfsetrectcap%
\pgfsetmiterjoin%
\pgfsetlinewidth{0.803000pt}%
\definecolor{currentstroke}{rgb}{0.000000,0.000000,0.000000}%
\pgfsetstrokecolor{currentstroke}%
\pgfsetdash{}{0pt}%
\pgfpathmoveto{\pgfqpoint{0.510000in}{0.440000in}}%
\pgfpathlineto{\pgfqpoint{0.510000in}{3.520000in}}%
\pgfusepath{stroke}%
\end{pgfscope}%
\begin{pgfscope}%
\pgfsetrectcap%
\pgfsetmiterjoin%
\pgfsetlinewidth{0.803000pt}%
\definecolor{currentstroke}{rgb}{0.000000,0.000000,0.000000}%
\pgfsetstrokecolor{currentstroke}%
\pgfsetdash{}{0pt}%
\pgfpathmoveto{\pgfqpoint{3.590000in}{0.440000in}}%
\pgfpathlineto{\pgfqpoint{3.590000in}{3.520000in}}%
\pgfusepath{stroke}%
\end{pgfscope}%
\begin{pgfscope}%
\pgfsetrectcap%
\pgfsetmiterjoin%
\pgfsetlinewidth{0.803000pt}%
\definecolor{currentstroke}{rgb}{0.000000,0.000000,0.000000}%
\pgfsetstrokecolor{currentstroke}%
\pgfsetdash{}{0pt}%
\pgfpathmoveto{\pgfqpoint{0.510000in}{0.440000in}}%
\pgfpathlineto{\pgfqpoint{3.590000in}{0.440000in}}%
\pgfusepath{stroke}%
\end{pgfscope}%
\begin{pgfscope}%
\pgfsetrectcap%
\pgfsetmiterjoin%
\pgfsetlinewidth{0.803000pt}%
\definecolor{currentstroke}{rgb}{0.000000,0.000000,0.000000}%
\pgfsetstrokecolor{currentstroke}%
\pgfsetdash{}{0pt}%
\pgfpathmoveto{\pgfqpoint{0.510000in}{3.520000in}}%
\pgfpathlineto{\pgfqpoint{3.590000in}{3.520000in}}%
\pgfusepath{stroke}%
\end{pgfscope}%
\begin{pgfscope}%
\definecolor{textcolor}{rgb}{0.000000,0.000000,0.000000}%
\pgfsetstrokecolor{textcolor}%
\pgfsetfillcolor{textcolor}%
\pgftext[x=2.362160in,y=1.033651in,left,base]{\color{textcolor}\sffamily\fontsize{10.000000}{12.000000}\selectfont 0.16667}%
\end{pgfscope}%
\begin{pgfscope}%
\definecolor{textcolor}{rgb}{0.000000,0.000000,0.000000}%
\pgfsetstrokecolor{textcolor}%
\pgfsetfillcolor{textcolor}%
\pgftext[x=1.584188in,y=2.292159in,left,base]{\color{textcolor}\sffamily\fontsize{10.000000}{12.000000}\selectfont 0.16667}%
\end{pgfscope}%
\begin{pgfscope}%
\definecolor{textcolor}{rgb}{0.000000,0.000000,0.000000}%
\pgfsetstrokecolor{textcolor}%
\pgfsetfillcolor{textcolor}%
\pgftext[x=1.103652in,y=1.514190in,left,base]{\color{textcolor}\sffamily\fontsize{10.000000}{12.000000}\selectfont 0.16667}%
\end{pgfscope}%
\end{pgfpicture}%
\makeatother%
\endgroup%
}
\caption{Figures of the two-dimensional cubatures of $ D = 2 $}
\label{Fig:M2D2Ran}
\end{figure}

It can be seen that the weights $w_{\cdot}$ always converge to $ 1 / 6 $, and the figure itself enjoys symmetry.

We may also add a new constraint to increase the number of constraints. We use $ \text{FPG} \rbr{ 10^4, 10^{-2} } + \text{N} \rbr{30} $ to optimize. We constrain $ \mathcal{I}_2 \cup \cbr{ \pbr{ 0, 3 } } $ and get the results in Figure \ref{Fig:M2D203}.

\begin{figure}[htbp]
\centering
\scalebox{0.75}{%% Creator: Matplotlib, PGF backend
%%
%% To include the figure in your LaTeX document, write
%%   \input{<filename>.pgf}
%%
%% Make sure the required packages are loaded in your preamble
%%   \usepackage{pgf}
%%
%% Figures using additional raster images can only be included by \input if
%% they are in the same directory as the main LaTeX file. For loading figures
%% from other directories you can use the `import` package
%%   \usepackage{import}
%% and then include the figures with
%%   \import{<path to file>}{<filename>.pgf}
%%
%% Matplotlib used the following preamble
%%   \usepackage{fontspec}
%%   \setmainfont{DejaVuSerif.ttf}[Path=/home/lzh/anaconda3/envs/numana/lib/python3.7/site-packages/matplotlib/mpl-data/fonts/ttf/]
%%   \setsansfont{DejaVuSans.ttf}[Path=/home/lzh/anaconda3/envs/numana/lib/python3.7/site-packages/matplotlib/mpl-data/fonts/ttf/]
%%   \setmonofont{DejaVuSansMono.ttf}[Path=/home/lzh/anaconda3/envs/numana/lib/python3.7/site-packages/matplotlib/mpl-data/fonts/ttf/]
%%
\begingroup%
\makeatletter%
\begin{pgfpicture}%
\pgfpathrectangle{\pgfpointorigin}{\pgfqpoint{8.000000in}{6.000000in}}%
\pgfusepath{use as bounding box, clip}%
\begin{pgfscope}%
\pgfsetbuttcap%
\pgfsetmiterjoin%
\definecolor{currentfill}{rgb}{1.000000,1.000000,1.000000}%
\pgfsetfillcolor{currentfill}%
\pgfsetlinewidth{0.000000pt}%
\definecolor{currentstroke}{rgb}{1.000000,1.000000,1.000000}%
\pgfsetstrokecolor{currentstroke}%
\pgfsetdash{}{0pt}%
\pgfpathmoveto{\pgfqpoint{0.000000in}{0.000000in}}%
\pgfpathlineto{\pgfqpoint{8.000000in}{0.000000in}}%
\pgfpathlineto{\pgfqpoint{8.000000in}{6.000000in}}%
\pgfpathlineto{\pgfqpoint{0.000000in}{6.000000in}}%
\pgfpathclose%
\pgfusepath{fill}%
\end{pgfscope}%
\begin{pgfscope}%
\pgfsetbuttcap%
\pgfsetmiterjoin%
\definecolor{currentfill}{rgb}{1.000000,1.000000,1.000000}%
\pgfsetfillcolor{currentfill}%
\pgfsetlinewidth{0.000000pt}%
\definecolor{currentstroke}{rgb}{0.000000,0.000000,0.000000}%
\pgfsetstrokecolor{currentstroke}%
\pgfsetstrokeopacity{0.000000}%
\pgfsetdash{}{0pt}%
\pgfpathmoveto{\pgfqpoint{0.557222in}{3.312222in}}%
\pgfpathlineto{\pgfqpoint{3.879180in}{3.312222in}}%
\pgfpathlineto{\pgfqpoint{3.879180in}{5.636667in}}%
\pgfpathlineto{\pgfqpoint{0.557222in}{5.636667in}}%
\pgfpathclose%
\pgfusepath{fill}%
\end{pgfscope}%
\begin{pgfscope}%
\pgfpathrectangle{\pgfqpoint{0.557222in}{3.312222in}}{\pgfqpoint{3.321958in}{2.324444in}}%
\pgfusepath{clip}%
\pgfsetbuttcap%
\pgfsetroundjoin%
\definecolor{currentfill}{rgb}{0.121569,0.466667,0.705882}%
\pgfsetfillcolor{currentfill}%
\pgfsetfillopacity{0.300000}%
\pgfsetlinewidth{1.003750pt}%
\definecolor{currentstroke}{rgb}{0.121569,0.466667,0.705882}%
\pgfsetstrokecolor{currentstroke}%
\pgfsetstrokeopacity{0.300000}%
\pgfsetdash{}{0pt}%
\pgfpathmoveto{\pgfqpoint{0.716510in}{3.597893in}}%
\pgfpathlineto{\pgfqpoint{0.716510in}{3.483021in}}%
\pgfpathlineto{\pgfqpoint{0.866679in}{3.845383in}}%
\pgfpathlineto{\pgfqpoint{1.016848in}{4.212082in}}%
\pgfpathlineto{\pgfqpoint{1.167017in}{4.418591in}}%
\pgfpathlineto{\pgfqpoint{1.317186in}{4.751870in}}%
\pgfpathlineto{\pgfqpoint{1.467355in}{4.941003in}}%
\pgfpathlineto{\pgfqpoint{1.617525in}{5.159467in}}%
\pgfpathlineto{\pgfqpoint{1.767694in}{5.286981in}}%
\pgfpathlineto{\pgfqpoint{1.917863in}{5.420166in}}%
\pgfpathlineto{\pgfqpoint{2.068032in}{5.463586in}}%
\pgfpathlineto{\pgfqpoint{2.218201in}{5.490838in}}%
\pgfpathlineto{\pgfqpoint{2.368370in}{5.417321in}}%
\pgfpathlineto{\pgfqpoint{2.518540in}{5.395182in}}%
\pgfpathlineto{\pgfqpoint{2.668709in}{5.211758in}}%
\pgfpathlineto{\pgfqpoint{2.818878in}{5.073930in}}%
\pgfpathlineto{\pgfqpoint{2.969047in}{4.870366in}}%
\pgfpathlineto{\pgfqpoint{3.119216in}{4.636778in}}%
\pgfpathlineto{\pgfqpoint{3.269385in}{4.382760in}}%
\pgfpathlineto{\pgfqpoint{3.419554in}{4.048134in}}%
\pgfpathlineto{\pgfqpoint{3.569724in}{3.710611in}}%
\pgfpathlineto{\pgfqpoint{3.719893in}{3.417879in}}%
\pgfpathlineto{\pgfqpoint{3.719893in}{3.486926in}}%
\pgfpathlineto{\pgfqpoint{3.719893in}{3.486926in}}%
\pgfpathlineto{\pgfqpoint{3.569724in}{3.779689in}}%
\pgfpathlineto{\pgfqpoint{3.419554in}{4.148307in}}%
\pgfpathlineto{\pgfqpoint{3.269385in}{4.418430in}}%
\pgfpathlineto{\pgfqpoint{3.119216in}{4.679773in}}%
\pgfpathlineto{\pgfqpoint{2.969047in}{4.979879in}}%
\pgfpathlineto{\pgfqpoint{2.818878in}{5.154646in}}%
\pgfpathlineto{\pgfqpoint{2.668709in}{5.348161in}}%
\pgfpathlineto{\pgfqpoint{2.518540in}{5.468555in}}%
\pgfpathlineto{\pgfqpoint{2.368370in}{5.497930in}}%
\pgfpathlineto{\pgfqpoint{2.218201in}{5.531010in}}%
\pgfpathlineto{\pgfqpoint{2.068032in}{5.523318in}}%
\pgfpathlineto{\pgfqpoint{1.917863in}{5.511632in}}%
\pgfpathlineto{\pgfqpoint{1.767694in}{5.350341in}}%
\pgfpathlineto{\pgfqpoint{1.617525in}{5.269949in}}%
\pgfpathlineto{\pgfqpoint{1.467355in}{5.056700in}}%
\pgfpathlineto{\pgfqpoint{1.317186in}{4.855294in}}%
\pgfpathlineto{\pgfqpoint{1.167017in}{4.568400in}}%
\pgfpathlineto{\pgfqpoint{1.016848in}{4.292471in}}%
\pgfpathlineto{\pgfqpoint{0.866679in}{3.953788in}}%
\pgfpathlineto{\pgfqpoint{0.716510in}{3.597893in}}%
\pgfpathclose%
\pgfusepath{stroke,fill}%
\end{pgfscope}%
\begin{pgfscope}%
\pgfpathrectangle{\pgfqpoint{0.557222in}{3.312222in}}{\pgfqpoint{3.321958in}{2.324444in}}%
\pgfusepath{clip}%
\pgfsetbuttcap%
\pgfsetroundjoin%
\definecolor{currentfill}{rgb}{0.121569,0.466667,0.705882}%
\pgfsetfillcolor{currentfill}%
\pgfsetlinewidth{1.003750pt}%
\definecolor{currentstroke}{rgb}{0.121569,0.466667,0.705882}%
\pgfsetstrokecolor{currentstroke}%
\pgfsetdash{}{0pt}%
\pgfsys@defobject{currentmarker}{\pgfqpoint{-0.009821in}{-0.009821in}}{\pgfqpoint{0.009821in}{0.009821in}}{%
\pgfpathmoveto{\pgfqpoint{0.000000in}{-0.009821in}}%
\pgfpathcurveto{\pgfqpoint{0.002605in}{-0.009821in}}{\pgfqpoint{0.005103in}{-0.008786in}}{\pgfqpoint{0.006944in}{-0.006944in}}%
\pgfpathcurveto{\pgfqpoint{0.008786in}{-0.005103in}}{\pgfqpoint{0.009821in}{-0.002605in}}{\pgfqpoint{0.009821in}{0.000000in}}%
\pgfpathcurveto{\pgfqpoint{0.009821in}{0.002605in}}{\pgfqpoint{0.008786in}{0.005103in}}{\pgfqpoint{0.006944in}{0.006944in}}%
\pgfpathcurveto{\pgfqpoint{0.005103in}{0.008786in}}{\pgfqpoint{0.002605in}{0.009821in}}{\pgfqpoint{0.000000in}{0.009821in}}%
\pgfpathcurveto{\pgfqpoint{-0.002605in}{0.009821in}}{\pgfqpoint{-0.005103in}{0.008786in}}{\pgfqpoint{-0.006944in}{0.006944in}}%
\pgfpathcurveto{\pgfqpoint{-0.008786in}{0.005103in}}{\pgfqpoint{-0.009821in}{0.002605in}}{\pgfqpoint{-0.009821in}{0.000000in}}%
\pgfpathcurveto{\pgfqpoint{-0.009821in}{-0.002605in}}{\pgfqpoint{-0.008786in}{-0.005103in}}{\pgfqpoint{-0.006944in}{-0.006944in}}%
\pgfpathcurveto{\pgfqpoint{-0.005103in}{-0.008786in}}{\pgfqpoint{-0.002605in}{-0.009821in}}{\pgfqpoint{0.000000in}{-0.009821in}}%
\pgfpathclose%
\pgfusepath{stroke,fill}%
}%
\begin{pgfscope}%
\pgfsys@transformshift{0.716510in}{3.540457in}%
\pgfsys@useobject{currentmarker}{}%
\end{pgfscope}%
\begin{pgfscope}%
\pgfsys@transformshift{0.866679in}{3.899586in}%
\pgfsys@useobject{currentmarker}{}%
\end{pgfscope}%
\begin{pgfscope}%
\pgfsys@transformshift{1.016848in}{4.252277in}%
\pgfsys@useobject{currentmarker}{}%
\end{pgfscope}%
\begin{pgfscope}%
\pgfsys@transformshift{1.167017in}{4.493495in}%
\pgfsys@useobject{currentmarker}{}%
\end{pgfscope}%
\begin{pgfscope}%
\pgfsys@transformshift{1.317186in}{4.803582in}%
\pgfsys@useobject{currentmarker}{}%
\end{pgfscope}%
\begin{pgfscope}%
\pgfsys@transformshift{1.467355in}{4.998852in}%
\pgfsys@useobject{currentmarker}{}%
\end{pgfscope}%
\begin{pgfscope}%
\pgfsys@transformshift{1.617525in}{5.214708in}%
\pgfsys@useobject{currentmarker}{}%
\end{pgfscope}%
\begin{pgfscope}%
\pgfsys@transformshift{1.767694in}{5.318661in}%
\pgfsys@useobject{currentmarker}{}%
\end{pgfscope}%
\begin{pgfscope}%
\pgfsys@transformshift{1.917863in}{5.465899in}%
\pgfsys@useobject{currentmarker}{}%
\end{pgfscope}%
\begin{pgfscope}%
\pgfsys@transformshift{2.068032in}{5.493452in}%
\pgfsys@useobject{currentmarker}{}%
\end{pgfscope}%
\begin{pgfscope}%
\pgfsys@transformshift{2.218201in}{5.510924in}%
\pgfsys@useobject{currentmarker}{}%
\end{pgfscope}%
\begin{pgfscope}%
\pgfsys@transformshift{2.368370in}{5.457626in}%
\pgfsys@useobject{currentmarker}{}%
\end{pgfscope}%
\begin{pgfscope}%
\pgfsys@transformshift{2.518540in}{5.431869in}%
\pgfsys@useobject{currentmarker}{}%
\end{pgfscope}%
\begin{pgfscope}%
\pgfsys@transformshift{2.668709in}{5.279960in}%
\pgfsys@useobject{currentmarker}{}%
\end{pgfscope}%
\begin{pgfscope}%
\pgfsys@transformshift{2.818878in}{5.114288in}%
\pgfsys@useobject{currentmarker}{}%
\end{pgfscope}%
\begin{pgfscope}%
\pgfsys@transformshift{2.969047in}{4.925122in}%
\pgfsys@useobject{currentmarker}{}%
\end{pgfscope}%
\begin{pgfscope}%
\pgfsys@transformshift{3.119216in}{4.658275in}%
\pgfsys@useobject{currentmarker}{}%
\end{pgfscope}%
\begin{pgfscope}%
\pgfsys@transformshift{3.269385in}{4.400595in}%
\pgfsys@useobject{currentmarker}{}%
\end{pgfscope}%
\begin{pgfscope}%
\pgfsys@transformshift{3.419554in}{4.098221in}%
\pgfsys@useobject{currentmarker}{}%
\end{pgfscope}%
\begin{pgfscope}%
\pgfsys@transformshift{3.569724in}{3.745150in}%
\pgfsys@useobject{currentmarker}{}%
\end{pgfscope}%
\begin{pgfscope}%
\pgfsys@transformshift{3.719893in}{3.452403in}%
\pgfsys@useobject{currentmarker}{}%
\end{pgfscope}%
\end{pgfscope}%
\begin{pgfscope}%
\pgfsetbuttcap%
\pgfsetroundjoin%
\definecolor{currentfill}{rgb}{0.000000,0.000000,0.000000}%
\pgfsetfillcolor{currentfill}%
\pgfsetlinewidth{0.803000pt}%
\definecolor{currentstroke}{rgb}{0.000000,0.000000,0.000000}%
\pgfsetstrokecolor{currentstroke}%
\pgfsetdash{}{0pt}%
\pgfsys@defobject{currentmarker}{\pgfqpoint{0.000000in}{-0.048611in}}{\pgfqpoint{0.000000in}{0.000000in}}{%
\pgfpathmoveto{\pgfqpoint{0.000000in}{0.000000in}}%
\pgfpathlineto{\pgfqpoint{0.000000in}{-0.048611in}}%
\pgfusepath{stroke,fill}%
}%
\begin{pgfscope}%
\pgfsys@transformshift{1.167017in}{3.312222in}%
\pgfsys@useobject{currentmarker}{}%
\end{pgfscope}%
\end{pgfscope}%
\begin{pgfscope}%
\definecolor{textcolor}{rgb}{0.000000,0.000000,0.000000}%
\pgfsetstrokecolor{textcolor}%
\pgfsetfillcolor{textcolor}%
\pgftext[x=1.167017in,y=3.215000in,,top]{\color{textcolor}\sffamily\fontsize{10.000000}{12.000000}\selectfont 2.25}%
\end{pgfscope}%
\begin{pgfscope}%
\pgfsetbuttcap%
\pgfsetroundjoin%
\definecolor{currentfill}{rgb}{0.000000,0.000000,0.000000}%
\pgfsetfillcolor{currentfill}%
\pgfsetlinewidth{0.803000pt}%
\definecolor{currentstroke}{rgb}{0.000000,0.000000,0.000000}%
\pgfsetstrokecolor{currentstroke}%
\pgfsetdash{}{0pt}%
\pgfsys@defobject{currentmarker}{\pgfqpoint{0.000000in}{-0.048611in}}{\pgfqpoint{0.000000in}{0.000000in}}{%
\pgfpathmoveto{\pgfqpoint{0.000000in}{0.000000in}}%
\pgfpathlineto{\pgfqpoint{0.000000in}{-0.048611in}}%
\pgfusepath{stroke,fill}%
}%
\begin{pgfscope}%
\pgfsys@transformshift{1.917863in}{3.312222in}%
\pgfsys@useobject{currentmarker}{}%
\end{pgfscope}%
\end{pgfscope}%
\begin{pgfscope}%
\definecolor{textcolor}{rgb}{0.000000,0.000000,0.000000}%
\pgfsetstrokecolor{textcolor}%
\pgfsetfillcolor{textcolor}%
\pgftext[x=1.917863in,y=3.215000in,,top]{\color{textcolor}\sffamily\fontsize{10.000000}{12.000000}\selectfont 2.30}%
\end{pgfscope}%
\begin{pgfscope}%
\pgfsetbuttcap%
\pgfsetroundjoin%
\definecolor{currentfill}{rgb}{0.000000,0.000000,0.000000}%
\pgfsetfillcolor{currentfill}%
\pgfsetlinewidth{0.803000pt}%
\definecolor{currentstroke}{rgb}{0.000000,0.000000,0.000000}%
\pgfsetstrokecolor{currentstroke}%
\pgfsetdash{}{0pt}%
\pgfsys@defobject{currentmarker}{\pgfqpoint{0.000000in}{-0.048611in}}{\pgfqpoint{0.000000in}{0.000000in}}{%
\pgfpathmoveto{\pgfqpoint{0.000000in}{0.000000in}}%
\pgfpathlineto{\pgfqpoint{0.000000in}{-0.048611in}}%
\pgfusepath{stroke,fill}%
}%
\begin{pgfscope}%
\pgfsys@transformshift{2.668709in}{3.312222in}%
\pgfsys@useobject{currentmarker}{}%
\end{pgfscope}%
\end{pgfscope}%
\begin{pgfscope}%
\definecolor{textcolor}{rgb}{0.000000,0.000000,0.000000}%
\pgfsetstrokecolor{textcolor}%
\pgfsetfillcolor{textcolor}%
\pgftext[x=2.668709in,y=3.215000in,,top]{\color{textcolor}\sffamily\fontsize{10.000000}{12.000000}\selectfont 2.35}%
\end{pgfscope}%
\begin{pgfscope}%
\pgfsetbuttcap%
\pgfsetroundjoin%
\definecolor{currentfill}{rgb}{0.000000,0.000000,0.000000}%
\pgfsetfillcolor{currentfill}%
\pgfsetlinewidth{0.803000pt}%
\definecolor{currentstroke}{rgb}{0.000000,0.000000,0.000000}%
\pgfsetstrokecolor{currentstroke}%
\pgfsetdash{}{0pt}%
\pgfsys@defobject{currentmarker}{\pgfqpoint{0.000000in}{-0.048611in}}{\pgfqpoint{0.000000in}{0.000000in}}{%
\pgfpathmoveto{\pgfqpoint{0.000000in}{0.000000in}}%
\pgfpathlineto{\pgfqpoint{0.000000in}{-0.048611in}}%
\pgfusepath{stroke,fill}%
}%
\begin{pgfscope}%
\pgfsys@transformshift{3.419554in}{3.312222in}%
\pgfsys@useobject{currentmarker}{}%
\end{pgfscope}%
\end{pgfscope}%
\begin{pgfscope}%
\definecolor{textcolor}{rgb}{0.000000,0.000000,0.000000}%
\pgfsetstrokecolor{textcolor}%
\pgfsetfillcolor{textcolor}%
\pgftext[x=3.419554in,y=3.215000in,,top]{\color{textcolor}\sffamily\fontsize{10.000000}{12.000000}\selectfont 2.40}%
\end{pgfscope}%
\begin{pgfscope}%
\pgfsetbuttcap%
\pgfsetroundjoin%
\definecolor{currentfill}{rgb}{0.000000,0.000000,0.000000}%
\pgfsetfillcolor{currentfill}%
\pgfsetlinewidth{0.803000pt}%
\definecolor{currentstroke}{rgb}{0.000000,0.000000,0.000000}%
\pgfsetstrokecolor{currentstroke}%
\pgfsetdash{}{0pt}%
\pgfsys@defobject{currentmarker}{\pgfqpoint{-0.048611in}{0.000000in}}{\pgfqpoint{0.000000in}{0.000000in}}{%
\pgfpathmoveto{\pgfqpoint{0.000000in}{0.000000in}}%
\pgfpathlineto{\pgfqpoint{-0.048611in}{0.000000in}}%
\pgfusepath{stroke,fill}%
}%
\begin{pgfscope}%
\pgfsys@transformshift{0.557222in}{3.521300in}%
\pgfsys@useobject{currentmarker}{}%
\end{pgfscope}%
\end{pgfscope}%
\begin{pgfscope}%
\definecolor{textcolor}{rgb}{0.000000,0.000000,0.000000}%
\pgfsetstrokecolor{textcolor}%
\pgfsetfillcolor{textcolor}%
\pgftext[x=0.150755in,y=3.468538in,left,base]{\color{textcolor}\sffamily\fontsize{10.000000}{12.000000}\selectfont 1.35}%
\end{pgfscope}%
\begin{pgfscope}%
\pgfsetbuttcap%
\pgfsetroundjoin%
\definecolor{currentfill}{rgb}{0.000000,0.000000,0.000000}%
\pgfsetfillcolor{currentfill}%
\pgfsetlinewidth{0.803000pt}%
\definecolor{currentstroke}{rgb}{0.000000,0.000000,0.000000}%
\pgfsetstrokecolor{currentstroke}%
\pgfsetdash{}{0pt}%
\pgfsys@defobject{currentmarker}{\pgfqpoint{-0.048611in}{0.000000in}}{\pgfqpoint{0.000000in}{0.000000in}}{%
\pgfpathmoveto{\pgfqpoint{0.000000in}{0.000000in}}%
\pgfpathlineto{\pgfqpoint{-0.048611in}{0.000000in}}%
\pgfusepath{stroke,fill}%
}%
\begin{pgfscope}%
\pgfsys@transformshift{0.557222in}{4.025537in}%
\pgfsys@useobject{currentmarker}{}%
\end{pgfscope}%
\end{pgfscope}%
\begin{pgfscope}%
\definecolor{textcolor}{rgb}{0.000000,0.000000,0.000000}%
\pgfsetstrokecolor{textcolor}%
\pgfsetfillcolor{textcolor}%
\pgftext[x=0.150755in,y=3.972775in,left,base]{\color{textcolor}\sffamily\fontsize{10.000000}{12.000000}\selectfont 1.40}%
\end{pgfscope}%
\begin{pgfscope}%
\pgfsetbuttcap%
\pgfsetroundjoin%
\definecolor{currentfill}{rgb}{0.000000,0.000000,0.000000}%
\pgfsetfillcolor{currentfill}%
\pgfsetlinewidth{0.803000pt}%
\definecolor{currentstroke}{rgb}{0.000000,0.000000,0.000000}%
\pgfsetstrokecolor{currentstroke}%
\pgfsetdash{}{0pt}%
\pgfsys@defobject{currentmarker}{\pgfqpoint{-0.048611in}{0.000000in}}{\pgfqpoint{0.000000in}{0.000000in}}{%
\pgfpathmoveto{\pgfqpoint{0.000000in}{0.000000in}}%
\pgfpathlineto{\pgfqpoint{-0.048611in}{0.000000in}}%
\pgfusepath{stroke,fill}%
}%
\begin{pgfscope}%
\pgfsys@transformshift{0.557222in}{4.529774in}%
\pgfsys@useobject{currentmarker}{}%
\end{pgfscope}%
\end{pgfscope}%
\begin{pgfscope}%
\definecolor{textcolor}{rgb}{0.000000,0.000000,0.000000}%
\pgfsetstrokecolor{textcolor}%
\pgfsetfillcolor{textcolor}%
\pgftext[x=0.150755in,y=4.477013in,left,base]{\color{textcolor}\sffamily\fontsize{10.000000}{12.000000}\selectfont 1.45}%
\end{pgfscope}%
\begin{pgfscope}%
\pgfsetbuttcap%
\pgfsetroundjoin%
\definecolor{currentfill}{rgb}{0.000000,0.000000,0.000000}%
\pgfsetfillcolor{currentfill}%
\pgfsetlinewidth{0.803000pt}%
\definecolor{currentstroke}{rgb}{0.000000,0.000000,0.000000}%
\pgfsetstrokecolor{currentstroke}%
\pgfsetdash{}{0pt}%
\pgfsys@defobject{currentmarker}{\pgfqpoint{-0.048611in}{0.000000in}}{\pgfqpoint{0.000000in}{0.000000in}}{%
\pgfpathmoveto{\pgfqpoint{0.000000in}{0.000000in}}%
\pgfpathlineto{\pgfqpoint{-0.048611in}{0.000000in}}%
\pgfusepath{stroke,fill}%
}%
\begin{pgfscope}%
\pgfsys@transformshift{0.557222in}{5.034011in}%
\pgfsys@useobject{currentmarker}{}%
\end{pgfscope}%
\end{pgfscope}%
\begin{pgfscope}%
\definecolor{textcolor}{rgb}{0.000000,0.000000,0.000000}%
\pgfsetstrokecolor{textcolor}%
\pgfsetfillcolor{textcolor}%
\pgftext[x=0.150755in,y=4.981250in,left,base]{\color{textcolor}\sffamily\fontsize{10.000000}{12.000000}\selectfont 1.50}%
\end{pgfscope}%
\begin{pgfscope}%
\pgfsetbuttcap%
\pgfsetroundjoin%
\definecolor{currentfill}{rgb}{0.000000,0.000000,0.000000}%
\pgfsetfillcolor{currentfill}%
\pgfsetlinewidth{0.803000pt}%
\definecolor{currentstroke}{rgb}{0.000000,0.000000,0.000000}%
\pgfsetstrokecolor{currentstroke}%
\pgfsetdash{}{0pt}%
\pgfsys@defobject{currentmarker}{\pgfqpoint{-0.048611in}{0.000000in}}{\pgfqpoint{0.000000in}{0.000000in}}{%
\pgfpathmoveto{\pgfqpoint{0.000000in}{0.000000in}}%
\pgfpathlineto{\pgfqpoint{-0.048611in}{0.000000in}}%
\pgfusepath{stroke,fill}%
}%
\begin{pgfscope}%
\pgfsys@transformshift{0.557222in}{5.538248in}%
\pgfsys@useobject{currentmarker}{}%
\end{pgfscope}%
\end{pgfscope}%
\begin{pgfscope}%
\definecolor{textcolor}{rgb}{0.000000,0.000000,0.000000}%
\pgfsetstrokecolor{textcolor}%
\pgfsetfillcolor{textcolor}%
\pgftext[x=0.150755in,y=5.485487in,left,base]{\color{textcolor}\sffamily\fontsize{10.000000}{12.000000}\selectfont 1.55}%
\end{pgfscope}%
\begin{pgfscope}%
\pgfpathrectangle{\pgfqpoint{0.557222in}{3.312222in}}{\pgfqpoint{3.321958in}{2.324444in}}%
\pgfusepath{clip}%
\pgfsetrectcap%
\pgfsetroundjoin%
\pgfsetlinewidth{1.505625pt}%
\definecolor{currentstroke}{rgb}{0.121569,0.466667,0.705882}%
\pgfsetstrokecolor{currentstroke}%
\pgfsetdash{}{0pt}%
\pgfpathmoveto{\pgfqpoint{0.716510in}{3.540457in}}%
\pgfpathlineto{\pgfqpoint{0.866679in}{3.899586in}}%
\pgfpathlineto{\pgfqpoint{1.016848in}{4.252277in}}%
\pgfpathlineto{\pgfqpoint{1.167017in}{4.493495in}}%
\pgfpathlineto{\pgfqpoint{1.317186in}{4.803582in}}%
\pgfpathlineto{\pgfqpoint{1.467355in}{4.998852in}}%
\pgfpathlineto{\pgfqpoint{1.617525in}{5.214708in}}%
\pgfpathlineto{\pgfqpoint{1.767694in}{5.318661in}}%
\pgfpathlineto{\pgfqpoint{1.917863in}{5.465899in}}%
\pgfpathlineto{\pgfqpoint{2.068032in}{5.493452in}}%
\pgfpathlineto{\pgfqpoint{2.218201in}{5.510924in}}%
\pgfpathlineto{\pgfqpoint{2.368370in}{5.457626in}}%
\pgfpathlineto{\pgfqpoint{2.518540in}{5.431869in}}%
\pgfpathlineto{\pgfqpoint{2.668709in}{5.279960in}}%
\pgfpathlineto{\pgfqpoint{2.818878in}{5.114288in}}%
\pgfpathlineto{\pgfqpoint{2.969047in}{4.925122in}}%
\pgfpathlineto{\pgfqpoint{3.119216in}{4.658275in}}%
\pgfpathlineto{\pgfqpoint{3.269385in}{4.400595in}}%
\pgfpathlineto{\pgfqpoint{3.419554in}{4.098221in}}%
\pgfpathlineto{\pgfqpoint{3.569724in}{3.745150in}}%
\pgfpathlineto{\pgfqpoint{3.719893in}{3.452403in}}%
\pgfusepath{stroke}%
\end{pgfscope}%
\begin{pgfscope}%
\pgfsetrectcap%
\pgfsetmiterjoin%
\pgfsetlinewidth{0.803000pt}%
\definecolor{currentstroke}{rgb}{0.000000,0.000000,0.000000}%
\pgfsetstrokecolor{currentstroke}%
\pgfsetdash{}{0pt}%
\pgfpathmoveto{\pgfqpoint{0.557222in}{3.312222in}}%
\pgfpathlineto{\pgfqpoint{0.557222in}{5.636667in}}%
\pgfusepath{stroke}%
\end{pgfscope}%
\begin{pgfscope}%
\pgfsetrectcap%
\pgfsetmiterjoin%
\pgfsetlinewidth{0.803000pt}%
\definecolor{currentstroke}{rgb}{0.000000,0.000000,0.000000}%
\pgfsetstrokecolor{currentstroke}%
\pgfsetdash{}{0pt}%
\pgfpathmoveto{\pgfqpoint{3.879180in}{3.312222in}}%
\pgfpathlineto{\pgfqpoint{3.879180in}{5.636667in}}%
\pgfusepath{stroke}%
\end{pgfscope}%
\begin{pgfscope}%
\pgfsetrectcap%
\pgfsetmiterjoin%
\pgfsetlinewidth{0.803000pt}%
\definecolor{currentstroke}{rgb}{0.000000,0.000000,0.000000}%
\pgfsetstrokecolor{currentstroke}%
\pgfsetdash{}{0pt}%
\pgfpathmoveto{\pgfqpoint{0.557222in}{3.312222in}}%
\pgfpathlineto{\pgfqpoint{3.879180in}{3.312222in}}%
\pgfusepath{stroke}%
\end{pgfscope}%
\begin{pgfscope}%
\pgfsetrectcap%
\pgfsetmiterjoin%
\pgfsetlinewidth{0.803000pt}%
\definecolor{currentstroke}{rgb}{0.000000,0.000000,0.000000}%
\pgfsetstrokecolor{currentstroke}%
\pgfsetdash{}{0pt}%
\pgfpathmoveto{\pgfqpoint{0.557222in}{5.636667in}}%
\pgfpathlineto{\pgfqpoint{3.879180in}{5.636667in}}%
\pgfusepath{stroke}%
\end{pgfscope}%
\begin{pgfscope}%
\definecolor{textcolor}{rgb}{0.000000,0.000000,0.000000}%
\pgfsetstrokecolor{textcolor}%
\pgfsetfillcolor{textcolor}%
\pgftext[x=2.218201in,y=5.720000in,,base]{\color{textcolor}\sffamily\fontsize{12.000000}{14.400000}\selectfont \(\displaystyle  N = 16 \)}%
\end{pgfscope}%
\begin{pgfscope}%
\pgfsetbuttcap%
\pgfsetmiterjoin%
\definecolor{currentfill}{rgb}{1.000000,1.000000,1.000000}%
\pgfsetfillcolor{currentfill}%
\pgfsetlinewidth{0.000000pt}%
\definecolor{currentstroke}{rgb}{0.000000,0.000000,0.000000}%
\pgfsetstrokecolor{currentstroke}%
\pgfsetstrokeopacity{0.000000}%
\pgfsetdash{}{0pt}%
\pgfpathmoveto{\pgfqpoint{4.382652in}{3.312222in}}%
\pgfpathlineto{\pgfqpoint{7.704610in}{3.312222in}}%
\pgfpathlineto{\pgfqpoint{7.704610in}{5.636667in}}%
\pgfpathlineto{\pgfqpoint{4.382652in}{5.636667in}}%
\pgfpathclose%
\pgfusepath{fill}%
\end{pgfscope}%
\begin{pgfscope}%
\pgfpathrectangle{\pgfqpoint{4.382652in}{3.312222in}}{\pgfqpoint{3.321958in}{2.324444in}}%
\pgfusepath{clip}%
\pgfsetbuttcap%
\pgfsetroundjoin%
\definecolor{currentfill}{rgb}{0.121569,0.466667,0.705882}%
\pgfsetfillcolor{currentfill}%
\pgfsetfillopacity{0.300000}%
\pgfsetlinewidth{1.003750pt}%
\definecolor{currentstroke}{rgb}{0.121569,0.466667,0.705882}%
\pgfsetstrokecolor{currentstroke}%
\pgfsetstrokeopacity{0.300000}%
\pgfsetdash{}{0pt}%
\pgfpathmoveto{\pgfqpoint{4.541940in}{3.581583in}}%
\pgfpathlineto{\pgfqpoint{4.541940in}{3.417879in}}%
\pgfpathlineto{\pgfqpoint{4.692109in}{3.653684in}}%
\pgfpathlineto{\pgfqpoint{4.842278in}{3.775589in}}%
\pgfpathlineto{\pgfqpoint{4.992447in}{4.098907in}}%
\pgfpathlineto{\pgfqpoint{5.142617in}{4.238444in}}%
\pgfpathlineto{\pgfqpoint{5.292786in}{4.602993in}}%
\pgfpathlineto{\pgfqpoint{5.442955in}{4.709282in}}%
\pgfpathlineto{\pgfqpoint{5.593124in}{5.000884in}}%
\pgfpathlineto{\pgfqpoint{5.743293in}{5.181791in}}%
\pgfpathlineto{\pgfqpoint{5.893462in}{5.316691in}}%
\pgfpathlineto{\pgfqpoint{6.043631in}{5.424870in}}%
\pgfpathlineto{\pgfqpoint{6.193801in}{5.379751in}}%
\pgfpathlineto{\pgfqpoint{6.343970in}{5.247317in}}%
\pgfpathlineto{\pgfqpoint{6.494139in}{5.106998in}}%
\pgfpathlineto{\pgfqpoint{6.644308in}{4.970743in}}%
\pgfpathlineto{\pgfqpoint{6.794477in}{4.772387in}}%
\pgfpathlineto{\pgfqpoint{6.944646in}{4.483464in}}%
\pgfpathlineto{\pgfqpoint{7.094816in}{4.228662in}}%
\pgfpathlineto{\pgfqpoint{7.244985in}{3.941628in}}%
\pgfpathlineto{\pgfqpoint{7.395154in}{3.707982in}}%
\pgfpathlineto{\pgfqpoint{7.545323in}{3.459400in}}%
\pgfpathlineto{\pgfqpoint{7.545323in}{3.548470in}}%
\pgfpathlineto{\pgfqpoint{7.545323in}{3.548470in}}%
\pgfpathlineto{\pgfqpoint{7.395154in}{3.848444in}}%
\pgfpathlineto{\pgfqpoint{7.244985in}{4.054496in}}%
\pgfpathlineto{\pgfqpoint{7.094816in}{4.365142in}}%
\pgfpathlineto{\pgfqpoint{6.944646in}{4.657320in}}%
\pgfpathlineto{\pgfqpoint{6.794477in}{4.812351in}}%
\pgfpathlineto{\pgfqpoint{6.644308in}{5.049761in}}%
\pgfpathlineto{\pgfqpoint{6.494139in}{5.305736in}}%
\pgfpathlineto{\pgfqpoint{6.343970in}{5.390887in}}%
\pgfpathlineto{\pgfqpoint{6.193801in}{5.531010in}}%
\pgfpathlineto{\pgfqpoint{6.043631in}{5.473742in}}%
\pgfpathlineto{\pgfqpoint{5.893462in}{5.419225in}}%
\pgfpathlineto{\pgfqpoint{5.743293in}{5.340060in}}%
\pgfpathlineto{\pgfqpoint{5.593124in}{5.110651in}}%
\pgfpathlineto{\pgfqpoint{5.442955in}{4.956336in}}%
\pgfpathlineto{\pgfqpoint{5.292786in}{4.673809in}}%
\pgfpathlineto{\pgfqpoint{5.142617in}{4.507983in}}%
\pgfpathlineto{\pgfqpoint{4.992447in}{4.242332in}}%
\pgfpathlineto{\pgfqpoint{4.842278in}{4.119567in}}%
\pgfpathlineto{\pgfqpoint{4.692109in}{3.765147in}}%
\pgfpathlineto{\pgfqpoint{4.541940in}{3.581583in}}%
\pgfpathclose%
\pgfusepath{stroke,fill}%
\end{pgfscope}%
\begin{pgfscope}%
\pgfpathrectangle{\pgfqpoint{4.382652in}{3.312222in}}{\pgfqpoint{3.321958in}{2.324444in}}%
\pgfusepath{clip}%
\pgfsetbuttcap%
\pgfsetroundjoin%
\definecolor{currentfill}{rgb}{0.121569,0.466667,0.705882}%
\pgfsetfillcolor{currentfill}%
\pgfsetlinewidth{1.003750pt}%
\definecolor{currentstroke}{rgb}{0.121569,0.466667,0.705882}%
\pgfsetstrokecolor{currentstroke}%
\pgfsetdash{}{0pt}%
\pgfsys@defobject{currentmarker}{\pgfqpoint{-0.009821in}{-0.009821in}}{\pgfqpoint{0.009821in}{0.009821in}}{%
\pgfpathmoveto{\pgfqpoint{0.000000in}{-0.009821in}}%
\pgfpathcurveto{\pgfqpoint{0.002605in}{-0.009821in}}{\pgfqpoint{0.005103in}{-0.008786in}}{\pgfqpoint{0.006944in}{-0.006944in}}%
\pgfpathcurveto{\pgfqpoint{0.008786in}{-0.005103in}}{\pgfqpoint{0.009821in}{-0.002605in}}{\pgfqpoint{0.009821in}{0.000000in}}%
\pgfpathcurveto{\pgfqpoint{0.009821in}{0.002605in}}{\pgfqpoint{0.008786in}{0.005103in}}{\pgfqpoint{0.006944in}{0.006944in}}%
\pgfpathcurveto{\pgfqpoint{0.005103in}{0.008786in}}{\pgfqpoint{0.002605in}{0.009821in}}{\pgfqpoint{0.000000in}{0.009821in}}%
\pgfpathcurveto{\pgfqpoint{-0.002605in}{0.009821in}}{\pgfqpoint{-0.005103in}{0.008786in}}{\pgfqpoint{-0.006944in}{0.006944in}}%
\pgfpathcurveto{\pgfqpoint{-0.008786in}{0.005103in}}{\pgfqpoint{-0.009821in}{0.002605in}}{\pgfqpoint{-0.009821in}{0.000000in}}%
\pgfpathcurveto{\pgfqpoint{-0.009821in}{-0.002605in}}{\pgfqpoint{-0.008786in}{-0.005103in}}{\pgfqpoint{-0.006944in}{-0.006944in}}%
\pgfpathcurveto{\pgfqpoint{-0.005103in}{-0.008786in}}{\pgfqpoint{-0.002605in}{-0.009821in}}{\pgfqpoint{0.000000in}{-0.009821in}}%
\pgfpathclose%
\pgfusepath{stroke,fill}%
}%
\begin{pgfscope}%
\pgfsys@transformshift{4.541940in}{3.499731in}%
\pgfsys@useobject{currentmarker}{}%
\end{pgfscope}%
\begin{pgfscope}%
\pgfsys@transformshift{4.692109in}{3.709415in}%
\pgfsys@useobject{currentmarker}{}%
\end{pgfscope}%
\begin{pgfscope}%
\pgfsys@transformshift{4.842278in}{3.947578in}%
\pgfsys@useobject{currentmarker}{}%
\end{pgfscope}%
\begin{pgfscope}%
\pgfsys@transformshift{4.992447in}{4.170620in}%
\pgfsys@useobject{currentmarker}{}%
\end{pgfscope}%
\begin{pgfscope}%
\pgfsys@transformshift{5.142617in}{4.373214in}%
\pgfsys@useobject{currentmarker}{}%
\end{pgfscope}%
\begin{pgfscope}%
\pgfsys@transformshift{5.292786in}{4.638401in}%
\pgfsys@useobject{currentmarker}{}%
\end{pgfscope}%
\begin{pgfscope}%
\pgfsys@transformshift{5.442955in}{4.832809in}%
\pgfsys@useobject{currentmarker}{}%
\end{pgfscope}%
\begin{pgfscope}%
\pgfsys@transformshift{5.593124in}{5.055768in}%
\pgfsys@useobject{currentmarker}{}%
\end{pgfscope}%
\begin{pgfscope}%
\pgfsys@transformshift{5.743293in}{5.260925in}%
\pgfsys@useobject{currentmarker}{}%
\end{pgfscope}%
\begin{pgfscope}%
\pgfsys@transformshift{5.893462in}{5.367958in}%
\pgfsys@useobject{currentmarker}{}%
\end{pgfscope}%
\begin{pgfscope}%
\pgfsys@transformshift{6.043631in}{5.449306in}%
\pgfsys@useobject{currentmarker}{}%
\end{pgfscope}%
\begin{pgfscope}%
\pgfsys@transformshift{6.193801in}{5.455380in}%
\pgfsys@useobject{currentmarker}{}%
\end{pgfscope}%
\begin{pgfscope}%
\pgfsys@transformshift{6.343970in}{5.319102in}%
\pgfsys@useobject{currentmarker}{}%
\end{pgfscope}%
\begin{pgfscope}%
\pgfsys@transformshift{6.494139in}{5.206367in}%
\pgfsys@useobject{currentmarker}{}%
\end{pgfscope}%
\begin{pgfscope}%
\pgfsys@transformshift{6.644308in}{5.010252in}%
\pgfsys@useobject{currentmarker}{}%
\end{pgfscope}%
\begin{pgfscope}%
\pgfsys@transformshift{6.794477in}{4.792369in}%
\pgfsys@useobject{currentmarker}{}%
\end{pgfscope}%
\begin{pgfscope}%
\pgfsys@transformshift{6.944646in}{4.570392in}%
\pgfsys@useobject{currentmarker}{}%
\end{pgfscope}%
\begin{pgfscope}%
\pgfsys@transformshift{7.094816in}{4.296902in}%
\pgfsys@useobject{currentmarker}{}%
\end{pgfscope}%
\begin{pgfscope}%
\pgfsys@transformshift{7.244985in}{3.998062in}%
\pgfsys@useobject{currentmarker}{}%
\end{pgfscope}%
\begin{pgfscope}%
\pgfsys@transformshift{7.395154in}{3.778213in}%
\pgfsys@useobject{currentmarker}{}%
\end{pgfscope}%
\begin{pgfscope}%
\pgfsys@transformshift{7.545323in}{3.503935in}%
\pgfsys@useobject{currentmarker}{}%
\end{pgfscope}%
\end{pgfscope}%
\begin{pgfscope}%
\pgfsetbuttcap%
\pgfsetroundjoin%
\definecolor{currentfill}{rgb}{0.000000,0.000000,0.000000}%
\pgfsetfillcolor{currentfill}%
\pgfsetlinewidth{0.803000pt}%
\definecolor{currentstroke}{rgb}{0.000000,0.000000,0.000000}%
\pgfsetstrokecolor{currentstroke}%
\pgfsetdash{}{0pt}%
\pgfsys@defobject{currentmarker}{\pgfqpoint{0.000000in}{-0.048611in}}{\pgfqpoint{0.000000in}{0.000000in}}{%
\pgfpathmoveto{\pgfqpoint{0.000000in}{0.000000in}}%
\pgfpathlineto{\pgfqpoint{0.000000in}{-0.048611in}}%
\pgfusepath{stroke,fill}%
}%
\begin{pgfscope}%
\pgfsys@transformshift{4.692109in}{3.312222in}%
\pgfsys@useobject{currentmarker}{}%
\end{pgfscope}%
\end{pgfscope}%
\begin{pgfscope}%
\definecolor{textcolor}{rgb}{0.000000,0.000000,0.000000}%
\pgfsetstrokecolor{textcolor}%
\pgfsetfillcolor{textcolor}%
\pgftext[x=4.692109in,y=3.215000in,,top]{\color{textcolor}\sffamily\fontsize{10.000000}{12.000000}\selectfont 2.20}%
\end{pgfscope}%
\begin{pgfscope}%
\pgfsetbuttcap%
\pgfsetroundjoin%
\definecolor{currentfill}{rgb}{0.000000,0.000000,0.000000}%
\pgfsetfillcolor{currentfill}%
\pgfsetlinewidth{0.803000pt}%
\definecolor{currentstroke}{rgb}{0.000000,0.000000,0.000000}%
\pgfsetstrokecolor{currentstroke}%
\pgfsetdash{}{0pt}%
\pgfsys@defobject{currentmarker}{\pgfqpoint{0.000000in}{-0.048611in}}{\pgfqpoint{0.000000in}{0.000000in}}{%
\pgfpathmoveto{\pgfqpoint{0.000000in}{0.000000in}}%
\pgfpathlineto{\pgfqpoint{0.000000in}{-0.048611in}}%
\pgfusepath{stroke,fill}%
}%
\begin{pgfscope}%
\pgfsys@transformshift{5.442955in}{3.312222in}%
\pgfsys@useobject{currentmarker}{}%
\end{pgfscope}%
\end{pgfscope}%
\begin{pgfscope}%
\definecolor{textcolor}{rgb}{0.000000,0.000000,0.000000}%
\pgfsetstrokecolor{textcolor}%
\pgfsetfillcolor{textcolor}%
\pgftext[x=5.442955in,y=3.215000in,,top]{\color{textcolor}\sffamily\fontsize{10.000000}{12.000000}\selectfont 2.25}%
\end{pgfscope}%
\begin{pgfscope}%
\pgfsetbuttcap%
\pgfsetroundjoin%
\definecolor{currentfill}{rgb}{0.000000,0.000000,0.000000}%
\pgfsetfillcolor{currentfill}%
\pgfsetlinewidth{0.803000pt}%
\definecolor{currentstroke}{rgb}{0.000000,0.000000,0.000000}%
\pgfsetstrokecolor{currentstroke}%
\pgfsetdash{}{0pt}%
\pgfsys@defobject{currentmarker}{\pgfqpoint{0.000000in}{-0.048611in}}{\pgfqpoint{0.000000in}{0.000000in}}{%
\pgfpathmoveto{\pgfqpoint{0.000000in}{0.000000in}}%
\pgfpathlineto{\pgfqpoint{0.000000in}{-0.048611in}}%
\pgfusepath{stroke,fill}%
}%
\begin{pgfscope}%
\pgfsys@transformshift{6.193801in}{3.312222in}%
\pgfsys@useobject{currentmarker}{}%
\end{pgfscope}%
\end{pgfscope}%
\begin{pgfscope}%
\definecolor{textcolor}{rgb}{0.000000,0.000000,0.000000}%
\pgfsetstrokecolor{textcolor}%
\pgfsetfillcolor{textcolor}%
\pgftext[x=6.193801in,y=3.215000in,,top]{\color{textcolor}\sffamily\fontsize{10.000000}{12.000000}\selectfont 2.30}%
\end{pgfscope}%
\begin{pgfscope}%
\pgfsetbuttcap%
\pgfsetroundjoin%
\definecolor{currentfill}{rgb}{0.000000,0.000000,0.000000}%
\pgfsetfillcolor{currentfill}%
\pgfsetlinewidth{0.803000pt}%
\definecolor{currentstroke}{rgb}{0.000000,0.000000,0.000000}%
\pgfsetstrokecolor{currentstroke}%
\pgfsetdash{}{0pt}%
\pgfsys@defobject{currentmarker}{\pgfqpoint{0.000000in}{-0.048611in}}{\pgfqpoint{0.000000in}{0.000000in}}{%
\pgfpathmoveto{\pgfqpoint{0.000000in}{0.000000in}}%
\pgfpathlineto{\pgfqpoint{0.000000in}{-0.048611in}}%
\pgfusepath{stroke,fill}%
}%
\begin{pgfscope}%
\pgfsys@transformshift{6.944646in}{3.312222in}%
\pgfsys@useobject{currentmarker}{}%
\end{pgfscope}%
\end{pgfscope}%
\begin{pgfscope}%
\definecolor{textcolor}{rgb}{0.000000,0.000000,0.000000}%
\pgfsetstrokecolor{textcolor}%
\pgfsetfillcolor{textcolor}%
\pgftext[x=6.944646in,y=3.215000in,,top]{\color{textcolor}\sffamily\fontsize{10.000000}{12.000000}\selectfont 2.35}%
\end{pgfscope}%
\begin{pgfscope}%
\pgfsetbuttcap%
\pgfsetroundjoin%
\definecolor{currentfill}{rgb}{0.000000,0.000000,0.000000}%
\pgfsetfillcolor{currentfill}%
\pgfsetlinewidth{0.803000pt}%
\definecolor{currentstroke}{rgb}{0.000000,0.000000,0.000000}%
\pgfsetstrokecolor{currentstroke}%
\pgfsetdash{}{0pt}%
\pgfsys@defobject{currentmarker}{\pgfqpoint{0.000000in}{-0.048611in}}{\pgfqpoint{0.000000in}{0.000000in}}{%
\pgfpathmoveto{\pgfqpoint{0.000000in}{0.000000in}}%
\pgfpathlineto{\pgfqpoint{0.000000in}{-0.048611in}}%
\pgfusepath{stroke,fill}%
}%
\begin{pgfscope}%
\pgfsys@transformshift{7.695492in}{3.312222in}%
\pgfsys@useobject{currentmarker}{}%
\end{pgfscope}%
\end{pgfscope}%
\begin{pgfscope}%
\definecolor{textcolor}{rgb}{0.000000,0.000000,0.000000}%
\pgfsetstrokecolor{textcolor}%
\pgfsetfillcolor{textcolor}%
\pgftext[x=7.695492in,y=3.215000in,,top]{\color{textcolor}\sffamily\fontsize{10.000000}{12.000000}\selectfont 2.40}%
\end{pgfscope}%
\begin{pgfscope}%
\pgfsetbuttcap%
\pgfsetroundjoin%
\definecolor{currentfill}{rgb}{0.000000,0.000000,0.000000}%
\pgfsetfillcolor{currentfill}%
\pgfsetlinewidth{0.803000pt}%
\definecolor{currentstroke}{rgb}{0.000000,0.000000,0.000000}%
\pgfsetstrokecolor{currentstroke}%
\pgfsetdash{}{0pt}%
\pgfsys@defobject{currentmarker}{\pgfqpoint{-0.048611in}{0.000000in}}{\pgfqpoint{0.000000in}{0.000000in}}{%
\pgfpathmoveto{\pgfqpoint{0.000000in}{0.000000in}}%
\pgfpathlineto{\pgfqpoint{-0.048611in}{0.000000in}}%
\pgfusepath{stroke,fill}%
}%
\begin{pgfscope}%
\pgfsys@transformshift{4.382652in}{3.351489in}%
\pgfsys@useobject{currentmarker}{}%
\end{pgfscope}%
\end{pgfscope}%
\begin{pgfscope}%
\definecolor{textcolor}{rgb}{0.000000,0.000000,0.000000}%
\pgfsetstrokecolor{textcolor}%
\pgfsetfillcolor{textcolor}%
\pgftext[x=4.064551in,y=3.298728in,left,base]{\color{textcolor}\sffamily\fontsize{10.000000}{12.000000}\selectfont 1.3}%
\end{pgfscope}%
\begin{pgfscope}%
\pgfsetbuttcap%
\pgfsetroundjoin%
\definecolor{currentfill}{rgb}{0.000000,0.000000,0.000000}%
\pgfsetfillcolor{currentfill}%
\pgfsetlinewidth{0.803000pt}%
\definecolor{currentstroke}{rgb}{0.000000,0.000000,0.000000}%
\pgfsetstrokecolor{currentstroke}%
\pgfsetdash{}{0pt}%
\pgfsys@defobject{currentmarker}{\pgfqpoint{-0.048611in}{0.000000in}}{\pgfqpoint{0.000000in}{0.000000in}}{%
\pgfpathmoveto{\pgfqpoint{0.000000in}{0.000000in}}%
\pgfpathlineto{\pgfqpoint{-0.048611in}{0.000000in}}%
\pgfusepath{stroke,fill}%
}%
\begin{pgfscope}%
\pgfsys@transformshift{4.382652in}{3.698773in}%
\pgfsys@useobject{currentmarker}{}%
\end{pgfscope}%
\end{pgfscope}%
\begin{pgfscope}%
\definecolor{textcolor}{rgb}{0.000000,0.000000,0.000000}%
\pgfsetstrokecolor{textcolor}%
\pgfsetfillcolor{textcolor}%
\pgftext[x=4.064551in,y=3.646011in,left,base]{\color{textcolor}\sffamily\fontsize{10.000000}{12.000000}\selectfont 1.4}%
\end{pgfscope}%
\begin{pgfscope}%
\pgfsetbuttcap%
\pgfsetroundjoin%
\definecolor{currentfill}{rgb}{0.000000,0.000000,0.000000}%
\pgfsetfillcolor{currentfill}%
\pgfsetlinewidth{0.803000pt}%
\definecolor{currentstroke}{rgb}{0.000000,0.000000,0.000000}%
\pgfsetstrokecolor{currentstroke}%
\pgfsetdash{}{0pt}%
\pgfsys@defobject{currentmarker}{\pgfqpoint{-0.048611in}{0.000000in}}{\pgfqpoint{0.000000in}{0.000000in}}{%
\pgfpathmoveto{\pgfqpoint{0.000000in}{0.000000in}}%
\pgfpathlineto{\pgfqpoint{-0.048611in}{0.000000in}}%
\pgfusepath{stroke,fill}%
}%
\begin{pgfscope}%
\pgfsys@transformshift{4.382652in}{4.046056in}%
\pgfsys@useobject{currentmarker}{}%
\end{pgfscope}%
\end{pgfscope}%
\begin{pgfscope}%
\definecolor{textcolor}{rgb}{0.000000,0.000000,0.000000}%
\pgfsetstrokecolor{textcolor}%
\pgfsetfillcolor{textcolor}%
\pgftext[x=4.064551in,y=3.993294in,left,base]{\color{textcolor}\sffamily\fontsize{10.000000}{12.000000}\selectfont 1.5}%
\end{pgfscope}%
\begin{pgfscope}%
\pgfsetbuttcap%
\pgfsetroundjoin%
\definecolor{currentfill}{rgb}{0.000000,0.000000,0.000000}%
\pgfsetfillcolor{currentfill}%
\pgfsetlinewidth{0.803000pt}%
\definecolor{currentstroke}{rgb}{0.000000,0.000000,0.000000}%
\pgfsetstrokecolor{currentstroke}%
\pgfsetdash{}{0pt}%
\pgfsys@defobject{currentmarker}{\pgfqpoint{-0.048611in}{0.000000in}}{\pgfqpoint{0.000000in}{0.000000in}}{%
\pgfpathmoveto{\pgfqpoint{0.000000in}{0.000000in}}%
\pgfpathlineto{\pgfqpoint{-0.048611in}{0.000000in}}%
\pgfusepath{stroke,fill}%
}%
\begin{pgfscope}%
\pgfsys@transformshift{4.382652in}{4.393339in}%
\pgfsys@useobject{currentmarker}{}%
\end{pgfscope}%
\end{pgfscope}%
\begin{pgfscope}%
\definecolor{textcolor}{rgb}{0.000000,0.000000,0.000000}%
\pgfsetstrokecolor{textcolor}%
\pgfsetfillcolor{textcolor}%
\pgftext[x=4.064551in,y=4.340578in,left,base]{\color{textcolor}\sffamily\fontsize{10.000000}{12.000000}\selectfont 1.6}%
\end{pgfscope}%
\begin{pgfscope}%
\pgfsetbuttcap%
\pgfsetroundjoin%
\definecolor{currentfill}{rgb}{0.000000,0.000000,0.000000}%
\pgfsetfillcolor{currentfill}%
\pgfsetlinewidth{0.803000pt}%
\definecolor{currentstroke}{rgb}{0.000000,0.000000,0.000000}%
\pgfsetstrokecolor{currentstroke}%
\pgfsetdash{}{0pt}%
\pgfsys@defobject{currentmarker}{\pgfqpoint{-0.048611in}{0.000000in}}{\pgfqpoint{0.000000in}{0.000000in}}{%
\pgfpathmoveto{\pgfqpoint{0.000000in}{0.000000in}}%
\pgfpathlineto{\pgfqpoint{-0.048611in}{0.000000in}}%
\pgfusepath{stroke,fill}%
}%
\begin{pgfscope}%
\pgfsys@transformshift{4.382652in}{4.740623in}%
\pgfsys@useobject{currentmarker}{}%
\end{pgfscope}%
\end{pgfscope}%
\begin{pgfscope}%
\definecolor{textcolor}{rgb}{0.000000,0.000000,0.000000}%
\pgfsetstrokecolor{textcolor}%
\pgfsetfillcolor{textcolor}%
\pgftext[x=4.064551in,y=4.687861in,left,base]{\color{textcolor}\sffamily\fontsize{10.000000}{12.000000}\selectfont 1.7}%
\end{pgfscope}%
\begin{pgfscope}%
\pgfsetbuttcap%
\pgfsetroundjoin%
\definecolor{currentfill}{rgb}{0.000000,0.000000,0.000000}%
\pgfsetfillcolor{currentfill}%
\pgfsetlinewidth{0.803000pt}%
\definecolor{currentstroke}{rgb}{0.000000,0.000000,0.000000}%
\pgfsetstrokecolor{currentstroke}%
\pgfsetdash{}{0pt}%
\pgfsys@defobject{currentmarker}{\pgfqpoint{-0.048611in}{0.000000in}}{\pgfqpoint{0.000000in}{0.000000in}}{%
\pgfpathmoveto{\pgfqpoint{0.000000in}{0.000000in}}%
\pgfpathlineto{\pgfqpoint{-0.048611in}{0.000000in}}%
\pgfusepath{stroke,fill}%
}%
\begin{pgfscope}%
\pgfsys@transformshift{4.382652in}{5.087906in}%
\pgfsys@useobject{currentmarker}{}%
\end{pgfscope}%
\end{pgfscope}%
\begin{pgfscope}%
\definecolor{textcolor}{rgb}{0.000000,0.000000,0.000000}%
\pgfsetstrokecolor{textcolor}%
\pgfsetfillcolor{textcolor}%
\pgftext[x=4.064551in,y=5.035145in,left,base]{\color{textcolor}\sffamily\fontsize{10.000000}{12.000000}\selectfont 1.8}%
\end{pgfscope}%
\begin{pgfscope}%
\pgfsetbuttcap%
\pgfsetroundjoin%
\definecolor{currentfill}{rgb}{0.000000,0.000000,0.000000}%
\pgfsetfillcolor{currentfill}%
\pgfsetlinewidth{0.803000pt}%
\definecolor{currentstroke}{rgb}{0.000000,0.000000,0.000000}%
\pgfsetstrokecolor{currentstroke}%
\pgfsetdash{}{0pt}%
\pgfsys@defobject{currentmarker}{\pgfqpoint{-0.048611in}{0.000000in}}{\pgfqpoint{0.000000in}{0.000000in}}{%
\pgfpathmoveto{\pgfqpoint{0.000000in}{0.000000in}}%
\pgfpathlineto{\pgfqpoint{-0.048611in}{0.000000in}}%
\pgfusepath{stroke,fill}%
}%
\begin{pgfscope}%
\pgfsys@transformshift{4.382652in}{5.435190in}%
\pgfsys@useobject{currentmarker}{}%
\end{pgfscope}%
\end{pgfscope}%
\begin{pgfscope}%
\definecolor{textcolor}{rgb}{0.000000,0.000000,0.000000}%
\pgfsetstrokecolor{textcolor}%
\pgfsetfillcolor{textcolor}%
\pgftext[x=4.064551in,y=5.382428in,left,base]{\color{textcolor}\sffamily\fontsize{10.000000}{12.000000}\selectfont 1.9}%
\end{pgfscope}%
\begin{pgfscope}%
\pgfpathrectangle{\pgfqpoint{4.382652in}{3.312222in}}{\pgfqpoint{3.321958in}{2.324444in}}%
\pgfusepath{clip}%
\pgfsetrectcap%
\pgfsetroundjoin%
\pgfsetlinewidth{1.505625pt}%
\definecolor{currentstroke}{rgb}{0.121569,0.466667,0.705882}%
\pgfsetstrokecolor{currentstroke}%
\pgfsetdash{}{0pt}%
\pgfpathmoveto{\pgfqpoint{4.541940in}{3.499731in}}%
\pgfpathlineto{\pgfqpoint{4.692109in}{3.709415in}}%
\pgfpathlineto{\pgfqpoint{4.842278in}{3.947578in}}%
\pgfpathlineto{\pgfqpoint{4.992447in}{4.170620in}}%
\pgfpathlineto{\pgfqpoint{5.142617in}{4.373214in}}%
\pgfpathlineto{\pgfqpoint{5.292786in}{4.638401in}}%
\pgfpathlineto{\pgfqpoint{5.442955in}{4.832809in}}%
\pgfpathlineto{\pgfqpoint{5.593124in}{5.055768in}}%
\pgfpathlineto{\pgfqpoint{5.743293in}{5.260925in}}%
\pgfpathlineto{\pgfqpoint{5.893462in}{5.367958in}}%
\pgfpathlineto{\pgfqpoint{6.043631in}{5.449306in}}%
\pgfpathlineto{\pgfqpoint{6.193801in}{5.455380in}}%
\pgfpathlineto{\pgfqpoint{6.343970in}{5.319102in}}%
\pgfpathlineto{\pgfqpoint{6.494139in}{5.206367in}}%
\pgfpathlineto{\pgfqpoint{6.644308in}{5.010252in}}%
\pgfpathlineto{\pgfqpoint{6.794477in}{4.792369in}}%
\pgfpathlineto{\pgfqpoint{6.944646in}{4.570392in}}%
\pgfpathlineto{\pgfqpoint{7.094816in}{4.296902in}}%
\pgfpathlineto{\pgfqpoint{7.244985in}{3.998062in}}%
\pgfpathlineto{\pgfqpoint{7.395154in}{3.778213in}}%
\pgfpathlineto{\pgfqpoint{7.545323in}{3.503935in}}%
\pgfusepath{stroke}%
\end{pgfscope}%
\begin{pgfscope}%
\pgfsetrectcap%
\pgfsetmiterjoin%
\pgfsetlinewidth{0.803000pt}%
\definecolor{currentstroke}{rgb}{0.000000,0.000000,0.000000}%
\pgfsetstrokecolor{currentstroke}%
\pgfsetdash{}{0pt}%
\pgfpathmoveto{\pgfqpoint{4.382652in}{3.312222in}}%
\pgfpathlineto{\pgfqpoint{4.382652in}{5.636667in}}%
\pgfusepath{stroke}%
\end{pgfscope}%
\begin{pgfscope}%
\pgfsetrectcap%
\pgfsetmiterjoin%
\pgfsetlinewidth{0.803000pt}%
\definecolor{currentstroke}{rgb}{0.000000,0.000000,0.000000}%
\pgfsetstrokecolor{currentstroke}%
\pgfsetdash{}{0pt}%
\pgfpathmoveto{\pgfqpoint{7.704610in}{3.312222in}}%
\pgfpathlineto{\pgfqpoint{7.704610in}{5.636667in}}%
\pgfusepath{stroke}%
\end{pgfscope}%
\begin{pgfscope}%
\pgfsetrectcap%
\pgfsetmiterjoin%
\pgfsetlinewidth{0.803000pt}%
\definecolor{currentstroke}{rgb}{0.000000,0.000000,0.000000}%
\pgfsetstrokecolor{currentstroke}%
\pgfsetdash{}{0pt}%
\pgfpathmoveto{\pgfqpoint{4.382652in}{3.312222in}}%
\pgfpathlineto{\pgfqpoint{7.704610in}{3.312222in}}%
\pgfusepath{stroke}%
\end{pgfscope}%
\begin{pgfscope}%
\pgfsetrectcap%
\pgfsetmiterjoin%
\pgfsetlinewidth{0.803000pt}%
\definecolor{currentstroke}{rgb}{0.000000,0.000000,0.000000}%
\pgfsetstrokecolor{currentstroke}%
\pgfsetdash{}{0pt}%
\pgfpathmoveto{\pgfqpoint{4.382652in}{5.636667in}}%
\pgfpathlineto{\pgfqpoint{7.704610in}{5.636667in}}%
\pgfusepath{stroke}%
\end{pgfscope}%
\begin{pgfscope}%
\definecolor{textcolor}{rgb}{0.000000,0.000000,0.000000}%
\pgfsetstrokecolor{textcolor}%
\pgfsetfillcolor{textcolor}%
\pgftext[x=6.043631in,y=5.720000in,,base]{\color{textcolor}\sffamily\fontsize{12.000000}{14.400000}\selectfont \(\displaystyle  N = 32 \)}%
\end{pgfscope}%
\begin{pgfscope}%
\pgfsetbuttcap%
\pgfsetmiterjoin%
\definecolor{currentfill}{rgb}{1.000000,1.000000,1.000000}%
\pgfsetfillcolor{currentfill}%
\pgfsetlinewidth{0.000000pt}%
\definecolor{currentstroke}{rgb}{0.000000,0.000000,0.000000}%
\pgfsetstrokecolor{currentstroke}%
\pgfsetstrokeopacity{0.000000}%
\pgfsetdash{}{0pt}%
\pgfpathmoveto{\pgfqpoint{0.557222in}{0.387222in}}%
\pgfpathlineto{\pgfqpoint{3.879180in}{0.387222in}}%
\pgfpathlineto{\pgfqpoint{3.879180in}{2.711667in}}%
\pgfpathlineto{\pgfqpoint{0.557222in}{2.711667in}}%
\pgfpathclose%
\pgfusepath{fill}%
\end{pgfscope}%
\begin{pgfscope}%
\pgfpathrectangle{\pgfqpoint{0.557222in}{0.387222in}}{\pgfqpoint{3.321958in}{2.324444in}}%
\pgfusepath{clip}%
\pgfsetbuttcap%
\pgfsetroundjoin%
\definecolor{currentfill}{rgb}{0.121569,0.466667,0.705882}%
\pgfsetfillcolor{currentfill}%
\pgfsetfillopacity{0.300000}%
\pgfsetlinewidth{1.003750pt}%
\definecolor{currentstroke}{rgb}{0.121569,0.466667,0.705882}%
\pgfsetstrokecolor{currentstroke}%
\pgfsetstrokeopacity{0.300000}%
\pgfsetdash{}{0pt}%
\pgfpathmoveto{\pgfqpoint{0.716510in}{0.703317in}}%
\pgfpathlineto{\pgfqpoint{0.716510in}{0.654860in}}%
\pgfpathlineto{\pgfqpoint{0.866679in}{0.631153in}}%
\pgfpathlineto{\pgfqpoint{1.016848in}{0.849854in}}%
\pgfpathlineto{\pgfqpoint{1.167017in}{0.951825in}}%
\pgfpathlineto{\pgfqpoint{1.317186in}{1.119351in}}%
\pgfpathlineto{\pgfqpoint{1.467355in}{1.290740in}}%
\pgfpathlineto{\pgfqpoint{1.617525in}{1.579656in}}%
\pgfpathlineto{\pgfqpoint{1.767694in}{1.656213in}}%
\pgfpathlineto{\pgfqpoint{1.917863in}{1.840844in}}%
\pgfpathlineto{\pgfqpoint{2.068032in}{2.176651in}}%
\pgfpathlineto{\pgfqpoint{2.218201in}{2.263501in}}%
\pgfpathlineto{\pgfqpoint{2.368370in}{2.080174in}}%
\pgfpathlineto{\pgfqpoint{2.518540in}{1.974004in}}%
\pgfpathlineto{\pgfqpoint{2.668709in}{1.716387in}}%
\pgfpathlineto{\pgfqpoint{2.818878in}{1.478430in}}%
\pgfpathlineto{\pgfqpoint{2.969047in}{1.202092in}}%
\pgfpathlineto{\pgfqpoint{3.119216in}{0.996422in}}%
\pgfpathlineto{\pgfqpoint{3.269385in}{0.870649in}}%
\pgfpathlineto{\pgfqpoint{3.419554in}{0.750612in}}%
\pgfpathlineto{\pgfqpoint{3.569724in}{0.589942in}}%
\pgfpathlineto{\pgfqpoint{3.719893in}{0.492879in}}%
\pgfpathlineto{\pgfqpoint{3.719893in}{0.601276in}}%
\pgfpathlineto{\pgfqpoint{3.719893in}{0.601276in}}%
\pgfpathlineto{\pgfqpoint{3.569724in}{0.716850in}}%
\pgfpathlineto{\pgfqpoint{3.419554in}{0.854919in}}%
\pgfpathlineto{\pgfqpoint{3.269385in}{0.977448in}}%
\pgfpathlineto{\pgfqpoint{3.119216in}{1.186414in}}%
\pgfpathlineto{\pgfqpoint{2.969047in}{1.371309in}}%
\pgfpathlineto{\pgfqpoint{2.818878in}{1.697652in}}%
\pgfpathlineto{\pgfqpoint{2.668709in}{2.023741in}}%
\pgfpathlineto{\pgfqpoint{2.518540in}{2.289246in}}%
\pgfpathlineto{\pgfqpoint{2.368370in}{2.606010in}}%
\pgfpathlineto{\pgfqpoint{2.218201in}{2.572791in}}%
\pgfpathlineto{\pgfqpoint{2.068032in}{2.460314in}}%
\pgfpathlineto{\pgfqpoint{1.917863in}{2.202607in}}%
\pgfpathlineto{\pgfqpoint{1.767694in}{2.061706in}}%
\pgfpathlineto{\pgfqpoint{1.617525in}{1.808625in}}%
\pgfpathlineto{\pgfqpoint{1.467355in}{1.558203in}}%
\pgfpathlineto{\pgfqpoint{1.317186in}{1.330756in}}%
\pgfpathlineto{\pgfqpoint{1.167017in}{1.088928in}}%
\pgfpathlineto{\pgfqpoint{1.016848in}{0.977702in}}%
\pgfpathlineto{\pgfqpoint{0.866679in}{0.864300in}}%
\pgfpathlineto{\pgfqpoint{0.716510in}{0.703317in}}%
\pgfpathclose%
\pgfusepath{stroke,fill}%
\end{pgfscope}%
\begin{pgfscope}%
\pgfpathrectangle{\pgfqpoint{0.557222in}{0.387222in}}{\pgfqpoint{3.321958in}{2.324444in}}%
\pgfusepath{clip}%
\pgfsetbuttcap%
\pgfsetroundjoin%
\definecolor{currentfill}{rgb}{0.121569,0.466667,0.705882}%
\pgfsetfillcolor{currentfill}%
\pgfsetlinewidth{1.003750pt}%
\definecolor{currentstroke}{rgb}{0.121569,0.466667,0.705882}%
\pgfsetstrokecolor{currentstroke}%
\pgfsetdash{}{0pt}%
\pgfsys@defobject{currentmarker}{\pgfqpoint{-0.009821in}{-0.009821in}}{\pgfqpoint{0.009821in}{0.009821in}}{%
\pgfpathmoveto{\pgfqpoint{0.000000in}{-0.009821in}}%
\pgfpathcurveto{\pgfqpoint{0.002605in}{-0.009821in}}{\pgfqpoint{0.005103in}{-0.008786in}}{\pgfqpoint{0.006944in}{-0.006944in}}%
\pgfpathcurveto{\pgfqpoint{0.008786in}{-0.005103in}}{\pgfqpoint{0.009821in}{-0.002605in}}{\pgfqpoint{0.009821in}{0.000000in}}%
\pgfpathcurveto{\pgfqpoint{0.009821in}{0.002605in}}{\pgfqpoint{0.008786in}{0.005103in}}{\pgfqpoint{0.006944in}{0.006944in}}%
\pgfpathcurveto{\pgfqpoint{0.005103in}{0.008786in}}{\pgfqpoint{0.002605in}{0.009821in}}{\pgfqpoint{0.000000in}{0.009821in}}%
\pgfpathcurveto{\pgfqpoint{-0.002605in}{0.009821in}}{\pgfqpoint{-0.005103in}{0.008786in}}{\pgfqpoint{-0.006944in}{0.006944in}}%
\pgfpathcurveto{\pgfqpoint{-0.008786in}{0.005103in}}{\pgfqpoint{-0.009821in}{0.002605in}}{\pgfqpoint{-0.009821in}{0.000000in}}%
\pgfpathcurveto{\pgfqpoint{-0.009821in}{-0.002605in}}{\pgfqpoint{-0.008786in}{-0.005103in}}{\pgfqpoint{-0.006944in}{-0.006944in}}%
\pgfpathcurveto{\pgfqpoint{-0.005103in}{-0.008786in}}{\pgfqpoint{-0.002605in}{-0.009821in}}{\pgfqpoint{0.000000in}{-0.009821in}}%
\pgfpathclose%
\pgfusepath{stroke,fill}%
}%
\begin{pgfscope}%
\pgfsys@transformshift{0.716510in}{0.679089in}%
\pgfsys@useobject{currentmarker}{}%
\end{pgfscope}%
\begin{pgfscope}%
\pgfsys@transformshift{0.866679in}{0.747726in}%
\pgfsys@useobject{currentmarker}{}%
\end{pgfscope}%
\begin{pgfscope}%
\pgfsys@transformshift{1.016848in}{0.913778in}%
\pgfsys@useobject{currentmarker}{}%
\end{pgfscope}%
\begin{pgfscope}%
\pgfsys@transformshift{1.167017in}{1.020377in}%
\pgfsys@useobject{currentmarker}{}%
\end{pgfscope}%
\begin{pgfscope}%
\pgfsys@transformshift{1.317186in}{1.225053in}%
\pgfsys@useobject{currentmarker}{}%
\end{pgfscope}%
\begin{pgfscope}%
\pgfsys@transformshift{1.467355in}{1.424472in}%
\pgfsys@useobject{currentmarker}{}%
\end{pgfscope}%
\begin{pgfscope}%
\pgfsys@transformshift{1.617525in}{1.694141in}%
\pgfsys@useobject{currentmarker}{}%
\end{pgfscope}%
\begin{pgfscope}%
\pgfsys@transformshift{1.767694in}{1.858960in}%
\pgfsys@useobject{currentmarker}{}%
\end{pgfscope}%
\begin{pgfscope}%
\pgfsys@transformshift{1.917863in}{2.021725in}%
\pgfsys@useobject{currentmarker}{}%
\end{pgfscope}%
\begin{pgfscope}%
\pgfsys@transformshift{2.068032in}{2.318482in}%
\pgfsys@useobject{currentmarker}{}%
\end{pgfscope}%
\begin{pgfscope}%
\pgfsys@transformshift{2.218201in}{2.418146in}%
\pgfsys@useobject{currentmarker}{}%
\end{pgfscope}%
\begin{pgfscope}%
\pgfsys@transformshift{2.368370in}{2.343092in}%
\pgfsys@useobject{currentmarker}{}%
\end{pgfscope}%
\begin{pgfscope}%
\pgfsys@transformshift{2.518540in}{2.131625in}%
\pgfsys@useobject{currentmarker}{}%
\end{pgfscope}%
\begin{pgfscope}%
\pgfsys@transformshift{2.668709in}{1.870064in}%
\pgfsys@useobject{currentmarker}{}%
\end{pgfscope}%
\begin{pgfscope}%
\pgfsys@transformshift{2.818878in}{1.588041in}%
\pgfsys@useobject{currentmarker}{}%
\end{pgfscope}%
\begin{pgfscope}%
\pgfsys@transformshift{2.969047in}{1.286701in}%
\pgfsys@useobject{currentmarker}{}%
\end{pgfscope}%
\begin{pgfscope}%
\pgfsys@transformshift{3.119216in}{1.091418in}%
\pgfsys@useobject{currentmarker}{}%
\end{pgfscope}%
\begin{pgfscope}%
\pgfsys@transformshift{3.269385in}{0.924049in}%
\pgfsys@useobject{currentmarker}{}%
\end{pgfscope}%
\begin{pgfscope}%
\pgfsys@transformshift{3.419554in}{0.802765in}%
\pgfsys@useobject{currentmarker}{}%
\end{pgfscope}%
\begin{pgfscope}%
\pgfsys@transformshift{3.569724in}{0.653396in}%
\pgfsys@useobject{currentmarker}{}%
\end{pgfscope}%
\begin{pgfscope}%
\pgfsys@transformshift{3.719893in}{0.547077in}%
\pgfsys@useobject{currentmarker}{}%
\end{pgfscope}%
\end{pgfscope}%
\begin{pgfscope}%
\pgfsetbuttcap%
\pgfsetroundjoin%
\definecolor{currentfill}{rgb}{0.000000,0.000000,0.000000}%
\pgfsetfillcolor{currentfill}%
\pgfsetlinewidth{0.803000pt}%
\definecolor{currentstroke}{rgb}{0.000000,0.000000,0.000000}%
\pgfsetstrokecolor{currentstroke}%
\pgfsetdash{}{0pt}%
\pgfsys@defobject{currentmarker}{\pgfqpoint{0.000000in}{-0.048611in}}{\pgfqpoint{0.000000in}{0.000000in}}{%
\pgfpathmoveto{\pgfqpoint{0.000000in}{0.000000in}}%
\pgfpathlineto{\pgfqpoint{0.000000in}{-0.048611in}}%
\pgfusepath{stroke,fill}%
}%
\begin{pgfscope}%
\pgfsys@transformshift{1.016848in}{0.387222in}%
\pgfsys@useobject{currentmarker}{}%
\end{pgfscope}%
\end{pgfscope}%
\begin{pgfscope}%
\definecolor{textcolor}{rgb}{0.000000,0.000000,0.000000}%
\pgfsetstrokecolor{textcolor}%
\pgfsetfillcolor{textcolor}%
\pgftext[x=1.016848in,y=0.290000in,,top]{\color{textcolor}\sffamily\fontsize{10.000000}{12.000000}\selectfont 2.20}%
\end{pgfscope}%
\begin{pgfscope}%
\pgfsetbuttcap%
\pgfsetroundjoin%
\definecolor{currentfill}{rgb}{0.000000,0.000000,0.000000}%
\pgfsetfillcolor{currentfill}%
\pgfsetlinewidth{0.803000pt}%
\definecolor{currentstroke}{rgb}{0.000000,0.000000,0.000000}%
\pgfsetstrokecolor{currentstroke}%
\pgfsetdash{}{0pt}%
\pgfsys@defobject{currentmarker}{\pgfqpoint{0.000000in}{-0.048611in}}{\pgfqpoint{0.000000in}{0.000000in}}{%
\pgfpathmoveto{\pgfqpoint{0.000000in}{0.000000in}}%
\pgfpathlineto{\pgfqpoint{0.000000in}{-0.048611in}}%
\pgfusepath{stroke,fill}%
}%
\begin{pgfscope}%
\pgfsys@transformshift{1.767694in}{0.387222in}%
\pgfsys@useobject{currentmarker}{}%
\end{pgfscope}%
\end{pgfscope}%
\begin{pgfscope}%
\definecolor{textcolor}{rgb}{0.000000,0.000000,0.000000}%
\pgfsetstrokecolor{textcolor}%
\pgfsetfillcolor{textcolor}%
\pgftext[x=1.767694in,y=0.290000in,,top]{\color{textcolor}\sffamily\fontsize{10.000000}{12.000000}\selectfont 2.25}%
\end{pgfscope}%
\begin{pgfscope}%
\pgfsetbuttcap%
\pgfsetroundjoin%
\definecolor{currentfill}{rgb}{0.000000,0.000000,0.000000}%
\pgfsetfillcolor{currentfill}%
\pgfsetlinewidth{0.803000pt}%
\definecolor{currentstroke}{rgb}{0.000000,0.000000,0.000000}%
\pgfsetstrokecolor{currentstroke}%
\pgfsetdash{}{0pt}%
\pgfsys@defobject{currentmarker}{\pgfqpoint{0.000000in}{-0.048611in}}{\pgfqpoint{0.000000in}{0.000000in}}{%
\pgfpathmoveto{\pgfqpoint{0.000000in}{0.000000in}}%
\pgfpathlineto{\pgfqpoint{0.000000in}{-0.048611in}}%
\pgfusepath{stroke,fill}%
}%
\begin{pgfscope}%
\pgfsys@transformshift{2.518540in}{0.387222in}%
\pgfsys@useobject{currentmarker}{}%
\end{pgfscope}%
\end{pgfscope}%
\begin{pgfscope}%
\definecolor{textcolor}{rgb}{0.000000,0.000000,0.000000}%
\pgfsetstrokecolor{textcolor}%
\pgfsetfillcolor{textcolor}%
\pgftext[x=2.518540in,y=0.290000in,,top]{\color{textcolor}\sffamily\fontsize{10.000000}{12.000000}\selectfont 2.30}%
\end{pgfscope}%
\begin{pgfscope}%
\pgfsetbuttcap%
\pgfsetroundjoin%
\definecolor{currentfill}{rgb}{0.000000,0.000000,0.000000}%
\pgfsetfillcolor{currentfill}%
\pgfsetlinewidth{0.803000pt}%
\definecolor{currentstroke}{rgb}{0.000000,0.000000,0.000000}%
\pgfsetstrokecolor{currentstroke}%
\pgfsetdash{}{0pt}%
\pgfsys@defobject{currentmarker}{\pgfqpoint{0.000000in}{-0.048611in}}{\pgfqpoint{0.000000in}{0.000000in}}{%
\pgfpathmoveto{\pgfqpoint{0.000000in}{0.000000in}}%
\pgfpathlineto{\pgfqpoint{0.000000in}{-0.048611in}}%
\pgfusepath{stroke,fill}%
}%
\begin{pgfscope}%
\pgfsys@transformshift{3.269385in}{0.387222in}%
\pgfsys@useobject{currentmarker}{}%
\end{pgfscope}%
\end{pgfscope}%
\begin{pgfscope}%
\definecolor{textcolor}{rgb}{0.000000,0.000000,0.000000}%
\pgfsetstrokecolor{textcolor}%
\pgfsetfillcolor{textcolor}%
\pgftext[x=3.269385in,y=0.290000in,,top]{\color{textcolor}\sffamily\fontsize{10.000000}{12.000000}\selectfont 2.35}%
\end{pgfscope}%
\begin{pgfscope}%
\pgfsetbuttcap%
\pgfsetroundjoin%
\definecolor{currentfill}{rgb}{0.000000,0.000000,0.000000}%
\pgfsetfillcolor{currentfill}%
\pgfsetlinewidth{0.803000pt}%
\definecolor{currentstroke}{rgb}{0.000000,0.000000,0.000000}%
\pgfsetstrokecolor{currentstroke}%
\pgfsetdash{}{0pt}%
\pgfsys@defobject{currentmarker}{\pgfqpoint{-0.048611in}{0.000000in}}{\pgfqpoint{0.000000in}{0.000000in}}{%
\pgfpathmoveto{\pgfqpoint{0.000000in}{0.000000in}}%
\pgfpathlineto{\pgfqpoint{-0.048611in}{0.000000in}}%
\pgfusepath{stroke,fill}%
}%
\begin{pgfscope}%
\pgfsys@transformshift{0.557222in}{0.487270in}%
\pgfsys@useobject{currentmarker}{}%
\end{pgfscope}%
\end{pgfscope}%
\begin{pgfscope}%
\definecolor{textcolor}{rgb}{0.000000,0.000000,0.000000}%
\pgfsetstrokecolor{textcolor}%
\pgfsetfillcolor{textcolor}%
\pgftext[x=0.239121in,y=0.434508in,left,base]{\color{textcolor}\sffamily\fontsize{10.000000}{12.000000}\selectfont 1.2}%
\end{pgfscope}%
\begin{pgfscope}%
\pgfsetbuttcap%
\pgfsetroundjoin%
\definecolor{currentfill}{rgb}{0.000000,0.000000,0.000000}%
\pgfsetfillcolor{currentfill}%
\pgfsetlinewidth{0.803000pt}%
\definecolor{currentstroke}{rgb}{0.000000,0.000000,0.000000}%
\pgfsetstrokecolor{currentstroke}%
\pgfsetdash{}{0pt}%
\pgfsys@defobject{currentmarker}{\pgfqpoint{-0.048611in}{0.000000in}}{\pgfqpoint{0.000000in}{0.000000in}}{%
\pgfpathmoveto{\pgfqpoint{0.000000in}{0.000000in}}%
\pgfpathlineto{\pgfqpoint{-0.048611in}{0.000000in}}%
\pgfusepath{stroke,fill}%
}%
\begin{pgfscope}%
\pgfsys@transformshift{0.557222in}{0.847390in}%
\pgfsys@useobject{currentmarker}{}%
\end{pgfscope}%
\end{pgfscope}%
\begin{pgfscope}%
\definecolor{textcolor}{rgb}{0.000000,0.000000,0.000000}%
\pgfsetstrokecolor{textcolor}%
\pgfsetfillcolor{textcolor}%
\pgftext[x=0.239121in,y=0.794628in,left,base]{\color{textcolor}\sffamily\fontsize{10.000000}{12.000000}\selectfont 1.4}%
\end{pgfscope}%
\begin{pgfscope}%
\pgfsetbuttcap%
\pgfsetroundjoin%
\definecolor{currentfill}{rgb}{0.000000,0.000000,0.000000}%
\pgfsetfillcolor{currentfill}%
\pgfsetlinewidth{0.803000pt}%
\definecolor{currentstroke}{rgb}{0.000000,0.000000,0.000000}%
\pgfsetstrokecolor{currentstroke}%
\pgfsetdash{}{0pt}%
\pgfsys@defobject{currentmarker}{\pgfqpoint{-0.048611in}{0.000000in}}{\pgfqpoint{0.000000in}{0.000000in}}{%
\pgfpathmoveto{\pgfqpoint{0.000000in}{0.000000in}}%
\pgfpathlineto{\pgfqpoint{-0.048611in}{0.000000in}}%
\pgfusepath{stroke,fill}%
}%
\begin{pgfscope}%
\pgfsys@transformshift{0.557222in}{1.207510in}%
\pgfsys@useobject{currentmarker}{}%
\end{pgfscope}%
\end{pgfscope}%
\begin{pgfscope}%
\definecolor{textcolor}{rgb}{0.000000,0.000000,0.000000}%
\pgfsetstrokecolor{textcolor}%
\pgfsetfillcolor{textcolor}%
\pgftext[x=0.239121in,y=1.154749in,left,base]{\color{textcolor}\sffamily\fontsize{10.000000}{12.000000}\selectfont 1.6}%
\end{pgfscope}%
\begin{pgfscope}%
\pgfsetbuttcap%
\pgfsetroundjoin%
\definecolor{currentfill}{rgb}{0.000000,0.000000,0.000000}%
\pgfsetfillcolor{currentfill}%
\pgfsetlinewidth{0.803000pt}%
\definecolor{currentstroke}{rgb}{0.000000,0.000000,0.000000}%
\pgfsetstrokecolor{currentstroke}%
\pgfsetdash{}{0pt}%
\pgfsys@defobject{currentmarker}{\pgfqpoint{-0.048611in}{0.000000in}}{\pgfqpoint{0.000000in}{0.000000in}}{%
\pgfpathmoveto{\pgfqpoint{0.000000in}{0.000000in}}%
\pgfpathlineto{\pgfqpoint{-0.048611in}{0.000000in}}%
\pgfusepath{stroke,fill}%
}%
\begin{pgfscope}%
\pgfsys@transformshift{0.557222in}{1.567630in}%
\pgfsys@useobject{currentmarker}{}%
\end{pgfscope}%
\end{pgfscope}%
\begin{pgfscope}%
\definecolor{textcolor}{rgb}{0.000000,0.000000,0.000000}%
\pgfsetstrokecolor{textcolor}%
\pgfsetfillcolor{textcolor}%
\pgftext[x=0.239121in,y=1.514869in,left,base]{\color{textcolor}\sffamily\fontsize{10.000000}{12.000000}\selectfont 1.8}%
\end{pgfscope}%
\begin{pgfscope}%
\pgfsetbuttcap%
\pgfsetroundjoin%
\definecolor{currentfill}{rgb}{0.000000,0.000000,0.000000}%
\pgfsetfillcolor{currentfill}%
\pgfsetlinewidth{0.803000pt}%
\definecolor{currentstroke}{rgb}{0.000000,0.000000,0.000000}%
\pgfsetstrokecolor{currentstroke}%
\pgfsetdash{}{0pt}%
\pgfsys@defobject{currentmarker}{\pgfqpoint{-0.048611in}{0.000000in}}{\pgfqpoint{0.000000in}{0.000000in}}{%
\pgfpathmoveto{\pgfqpoint{0.000000in}{0.000000in}}%
\pgfpathlineto{\pgfqpoint{-0.048611in}{0.000000in}}%
\pgfusepath{stroke,fill}%
}%
\begin{pgfscope}%
\pgfsys@transformshift{0.557222in}{1.927750in}%
\pgfsys@useobject{currentmarker}{}%
\end{pgfscope}%
\end{pgfscope}%
\begin{pgfscope}%
\definecolor{textcolor}{rgb}{0.000000,0.000000,0.000000}%
\pgfsetstrokecolor{textcolor}%
\pgfsetfillcolor{textcolor}%
\pgftext[x=0.239121in,y=1.874989in,left,base]{\color{textcolor}\sffamily\fontsize{10.000000}{12.000000}\selectfont 2.0}%
\end{pgfscope}%
\begin{pgfscope}%
\pgfsetbuttcap%
\pgfsetroundjoin%
\definecolor{currentfill}{rgb}{0.000000,0.000000,0.000000}%
\pgfsetfillcolor{currentfill}%
\pgfsetlinewidth{0.803000pt}%
\definecolor{currentstroke}{rgb}{0.000000,0.000000,0.000000}%
\pgfsetstrokecolor{currentstroke}%
\pgfsetdash{}{0pt}%
\pgfsys@defobject{currentmarker}{\pgfqpoint{-0.048611in}{0.000000in}}{\pgfqpoint{0.000000in}{0.000000in}}{%
\pgfpathmoveto{\pgfqpoint{0.000000in}{0.000000in}}%
\pgfpathlineto{\pgfqpoint{-0.048611in}{0.000000in}}%
\pgfusepath{stroke,fill}%
}%
\begin{pgfscope}%
\pgfsys@transformshift{0.557222in}{2.287870in}%
\pgfsys@useobject{currentmarker}{}%
\end{pgfscope}%
\end{pgfscope}%
\begin{pgfscope}%
\definecolor{textcolor}{rgb}{0.000000,0.000000,0.000000}%
\pgfsetstrokecolor{textcolor}%
\pgfsetfillcolor{textcolor}%
\pgftext[x=0.239121in,y=2.235109in,left,base]{\color{textcolor}\sffamily\fontsize{10.000000}{12.000000}\selectfont 2.2}%
\end{pgfscope}%
\begin{pgfscope}%
\pgfsetbuttcap%
\pgfsetroundjoin%
\definecolor{currentfill}{rgb}{0.000000,0.000000,0.000000}%
\pgfsetfillcolor{currentfill}%
\pgfsetlinewidth{0.803000pt}%
\definecolor{currentstroke}{rgb}{0.000000,0.000000,0.000000}%
\pgfsetstrokecolor{currentstroke}%
\pgfsetdash{}{0pt}%
\pgfsys@defobject{currentmarker}{\pgfqpoint{-0.048611in}{0.000000in}}{\pgfqpoint{0.000000in}{0.000000in}}{%
\pgfpathmoveto{\pgfqpoint{0.000000in}{0.000000in}}%
\pgfpathlineto{\pgfqpoint{-0.048611in}{0.000000in}}%
\pgfusepath{stroke,fill}%
}%
\begin{pgfscope}%
\pgfsys@transformshift{0.557222in}{2.647990in}%
\pgfsys@useobject{currentmarker}{}%
\end{pgfscope}%
\end{pgfscope}%
\begin{pgfscope}%
\definecolor{textcolor}{rgb}{0.000000,0.000000,0.000000}%
\pgfsetstrokecolor{textcolor}%
\pgfsetfillcolor{textcolor}%
\pgftext[x=0.239121in,y=2.595229in,left,base]{\color{textcolor}\sffamily\fontsize{10.000000}{12.000000}\selectfont 2.4}%
\end{pgfscope}%
\begin{pgfscope}%
\pgfpathrectangle{\pgfqpoint{0.557222in}{0.387222in}}{\pgfqpoint{3.321958in}{2.324444in}}%
\pgfusepath{clip}%
\pgfsetrectcap%
\pgfsetroundjoin%
\pgfsetlinewidth{1.505625pt}%
\definecolor{currentstroke}{rgb}{0.121569,0.466667,0.705882}%
\pgfsetstrokecolor{currentstroke}%
\pgfsetdash{}{0pt}%
\pgfpathmoveto{\pgfqpoint{0.716510in}{0.679089in}}%
\pgfpathlineto{\pgfqpoint{0.866679in}{0.747726in}}%
\pgfpathlineto{\pgfqpoint{1.016848in}{0.913778in}}%
\pgfpathlineto{\pgfqpoint{1.167017in}{1.020377in}}%
\pgfpathlineto{\pgfqpoint{1.317186in}{1.225053in}}%
\pgfpathlineto{\pgfqpoint{1.467355in}{1.424472in}}%
\pgfpathlineto{\pgfqpoint{1.617525in}{1.694141in}}%
\pgfpathlineto{\pgfqpoint{1.767694in}{1.858960in}}%
\pgfpathlineto{\pgfqpoint{1.917863in}{2.021725in}}%
\pgfpathlineto{\pgfqpoint{2.068032in}{2.318482in}}%
\pgfpathlineto{\pgfqpoint{2.218201in}{2.418146in}}%
\pgfpathlineto{\pgfqpoint{2.368370in}{2.343092in}}%
\pgfpathlineto{\pgfqpoint{2.518540in}{2.131625in}}%
\pgfpathlineto{\pgfqpoint{2.668709in}{1.870064in}}%
\pgfpathlineto{\pgfqpoint{2.818878in}{1.588041in}}%
\pgfpathlineto{\pgfqpoint{2.969047in}{1.286701in}}%
\pgfpathlineto{\pgfqpoint{3.119216in}{1.091418in}}%
\pgfpathlineto{\pgfqpoint{3.269385in}{0.924049in}}%
\pgfpathlineto{\pgfqpoint{3.419554in}{0.802765in}}%
\pgfpathlineto{\pgfqpoint{3.569724in}{0.653396in}}%
\pgfpathlineto{\pgfqpoint{3.719893in}{0.547077in}}%
\pgfusepath{stroke}%
\end{pgfscope}%
\begin{pgfscope}%
\pgfsetrectcap%
\pgfsetmiterjoin%
\pgfsetlinewidth{0.803000pt}%
\definecolor{currentstroke}{rgb}{0.000000,0.000000,0.000000}%
\pgfsetstrokecolor{currentstroke}%
\pgfsetdash{}{0pt}%
\pgfpathmoveto{\pgfqpoint{0.557222in}{0.387222in}}%
\pgfpathlineto{\pgfqpoint{0.557222in}{2.711667in}}%
\pgfusepath{stroke}%
\end{pgfscope}%
\begin{pgfscope}%
\pgfsetrectcap%
\pgfsetmiterjoin%
\pgfsetlinewidth{0.803000pt}%
\definecolor{currentstroke}{rgb}{0.000000,0.000000,0.000000}%
\pgfsetstrokecolor{currentstroke}%
\pgfsetdash{}{0pt}%
\pgfpathmoveto{\pgfqpoint{3.879180in}{0.387222in}}%
\pgfpathlineto{\pgfqpoint{3.879180in}{2.711667in}}%
\pgfusepath{stroke}%
\end{pgfscope}%
\begin{pgfscope}%
\pgfsetrectcap%
\pgfsetmiterjoin%
\pgfsetlinewidth{0.803000pt}%
\definecolor{currentstroke}{rgb}{0.000000,0.000000,0.000000}%
\pgfsetstrokecolor{currentstroke}%
\pgfsetdash{}{0pt}%
\pgfpathmoveto{\pgfqpoint{0.557222in}{0.387222in}}%
\pgfpathlineto{\pgfqpoint{3.879180in}{0.387222in}}%
\pgfusepath{stroke}%
\end{pgfscope}%
\begin{pgfscope}%
\pgfsetrectcap%
\pgfsetmiterjoin%
\pgfsetlinewidth{0.803000pt}%
\definecolor{currentstroke}{rgb}{0.000000,0.000000,0.000000}%
\pgfsetstrokecolor{currentstroke}%
\pgfsetdash{}{0pt}%
\pgfpathmoveto{\pgfqpoint{0.557222in}{2.711667in}}%
\pgfpathlineto{\pgfqpoint{3.879180in}{2.711667in}}%
\pgfusepath{stroke}%
\end{pgfscope}%
\begin{pgfscope}%
\definecolor{textcolor}{rgb}{0.000000,0.000000,0.000000}%
\pgfsetstrokecolor{textcolor}%
\pgfsetfillcolor{textcolor}%
\pgftext[x=2.218201in,y=2.795000in,,base]{\color{textcolor}\sffamily\fontsize{12.000000}{14.400000}\selectfont \(\displaystyle  N = 64 \)}%
\end{pgfscope}%
\begin{pgfscope}%
\pgfsetbuttcap%
\pgfsetmiterjoin%
\definecolor{currentfill}{rgb}{1.000000,1.000000,1.000000}%
\pgfsetfillcolor{currentfill}%
\pgfsetlinewidth{0.000000pt}%
\definecolor{currentstroke}{rgb}{0.000000,0.000000,0.000000}%
\pgfsetstrokecolor{currentstroke}%
\pgfsetstrokeopacity{0.000000}%
\pgfsetdash{}{0pt}%
\pgfpathmoveto{\pgfqpoint{4.382652in}{0.387222in}}%
\pgfpathlineto{\pgfqpoint{7.704610in}{0.387222in}}%
\pgfpathlineto{\pgfqpoint{7.704610in}{2.711667in}}%
\pgfpathlineto{\pgfqpoint{4.382652in}{2.711667in}}%
\pgfpathclose%
\pgfusepath{fill}%
\end{pgfscope}%
\begin{pgfscope}%
\pgfpathrectangle{\pgfqpoint{4.382652in}{0.387222in}}{\pgfqpoint{3.321958in}{2.324444in}}%
\pgfusepath{clip}%
\pgfsetbuttcap%
\pgfsetroundjoin%
\definecolor{currentfill}{rgb}{0.121569,0.466667,0.705882}%
\pgfsetfillcolor{currentfill}%
\pgfsetfillopacity{0.300000}%
\pgfsetlinewidth{1.003750pt}%
\definecolor{currentstroke}{rgb}{0.121569,0.466667,0.705882}%
\pgfsetstrokecolor{currentstroke}%
\pgfsetstrokeopacity{0.300000}%
\pgfsetdash{}{0pt}%
\pgfpathmoveto{\pgfqpoint{4.541940in}{0.586233in}}%
\pgfpathlineto{\pgfqpoint{4.541940in}{0.538920in}}%
\pgfpathlineto{\pgfqpoint{4.692109in}{0.583784in}}%
\pgfpathlineto{\pgfqpoint{4.842278in}{0.618488in}}%
\pgfpathlineto{\pgfqpoint{4.992447in}{0.670060in}}%
\pgfpathlineto{\pgfqpoint{5.142617in}{0.808072in}}%
\pgfpathlineto{\pgfqpoint{5.292786in}{0.884748in}}%
\pgfpathlineto{\pgfqpoint{5.442955in}{1.006870in}}%
\pgfpathlineto{\pgfqpoint{5.593124in}{1.088062in}}%
\pgfpathlineto{\pgfqpoint{5.743293in}{1.080254in}}%
\pgfpathlineto{\pgfqpoint{5.893462in}{1.276048in}}%
\pgfpathlineto{\pgfqpoint{6.043631in}{1.284847in}}%
\pgfpathlineto{\pgfqpoint{6.193801in}{1.639469in}}%
\pgfpathlineto{\pgfqpoint{6.343970in}{1.768686in}}%
\pgfpathlineto{\pgfqpoint{6.494139in}{1.013458in}}%
\pgfpathlineto{\pgfqpoint{6.644308in}{1.214134in}}%
\pgfpathlineto{\pgfqpoint{6.794477in}{0.819631in}}%
\pgfpathlineto{\pgfqpoint{6.944646in}{0.803585in}}%
\pgfpathlineto{\pgfqpoint{7.094816in}{0.677854in}}%
\pgfpathlineto{\pgfqpoint{7.244985in}{0.675936in}}%
\pgfpathlineto{\pgfqpoint{7.395154in}{0.527061in}}%
\pgfpathlineto{\pgfqpoint{7.545323in}{0.492879in}}%
\pgfpathlineto{\pgfqpoint{7.545323in}{0.675995in}}%
\pgfpathlineto{\pgfqpoint{7.545323in}{0.675995in}}%
\pgfpathlineto{\pgfqpoint{7.395154in}{0.671210in}}%
\pgfpathlineto{\pgfqpoint{7.244985in}{0.862232in}}%
\pgfpathlineto{\pgfqpoint{7.094816in}{0.848425in}}%
\pgfpathlineto{\pgfqpoint{6.944646in}{0.928578in}}%
\pgfpathlineto{\pgfqpoint{6.794477in}{1.092525in}}%
\pgfpathlineto{\pgfqpoint{6.644308in}{1.320105in}}%
\pgfpathlineto{\pgfqpoint{6.494139in}{1.300777in}}%
\pgfpathlineto{\pgfqpoint{6.343970in}{2.000565in}}%
\pgfpathlineto{\pgfqpoint{6.193801in}{2.044018in}}%
\pgfpathlineto{\pgfqpoint{6.043631in}{2.606010in}}%
\pgfpathlineto{\pgfqpoint{5.893462in}{2.003349in}}%
\pgfpathlineto{\pgfqpoint{5.743293in}{1.580820in}}%
\pgfpathlineto{\pgfqpoint{5.593124in}{1.316842in}}%
\pgfpathlineto{\pgfqpoint{5.442955in}{1.243056in}}%
\pgfpathlineto{\pgfqpoint{5.292786in}{0.965562in}}%
\pgfpathlineto{\pgfqpoint{5.142617in}{0.904025in}}%
\pgfpathlineto{\pgfqpoint{4.992447in}{0.829153in}}%
\pgfpathlineto{\pgfqpoint{4.842278in}{0.806475in}}%
\pgfpathlineto{\pgfqpoint{4.692109in}{0.662067in}}%
\pgfpathlineto{\pgfqpoint{4.541940in}{0.586233in}}%
\pgfpathclose%
\pgfusepath{stroke,fill}%
\end{pgfscope}%
\begin{pgfscope}%
\pgfpathrectangle{\pgfqpoint{4.382652in}{0.387222in}}{\pgfqpoint{3.321958in}{2.324444in}}%
\pgfusepath{clip}%
\pgfsetbuttcap%
\pgfsetroundjoin%
\definecolor{currentfill}{rgb}{0.121569,0.466667,0.705882}%
\pgfsetfillcolor{currentfill}%
\pgfsetlinewidth{1.003750pt}%
\definecolor{currentstroke}{rgb}{0.121569,0.466667,0.705882}%
\pgfsetstrokecolor{currentstroke}%
\pgfsetdash{}{0pt}%
\pgfsys@defobject{currentmarker}{\pgfqpoint{-0.009821in}{-0.009821in}}{\pgfqpoint{0.009821in}{0.009821in}}{%
\pgfpathmoveto{\pgfqpoint{0.000000in}{-0.009821in}}%
\pgfpathcurveto{\pgfqpoint{0.002605in}{-0.009821in}}{\pgfqpoint{0.005103in}{-0.008786in}}{\pgfqpoint{0.006944in}{-0.006944in}}%
\pgfpathcurveto{\pgfqpoint{0.008786in}{-0.005103in}}{\pgfqpoint{0.009821in}{-0.002605in}}{\pgfqpoint{0.009821in}{0.000000in}}%
\pgfpathcurveto{\pgfqpoint{0.009821in}{0.002605in}}{\pgfqpoint{0.008786in}{0.005103in}}{\pgfqpoint{0.006944in}{0.006944in}}%
\pgfpathcurveto{\pgfqpoint{0.005103in}{0.008786in}}{\pgfqpoint{0.002605in}{0.009821in}}{\pgfqpoint{0.000000in}{0.009821in}}%
\pgfpathcurveto{\pgfqpoint{-0.002605in}{0.009821in}}{\pgfqpoint{-0.005103in}{0.008786in}}{\pgfqpoint{-0.006944in}{0.006944in}}%
\pgfpathcurveto{\pgfqpoint{-0.008786in}{0.005103in}}{\pgfqpoint{-0.009821in}{0.002605in}}{\pgfqpoint{-0.009821in}{0.000000in}}%
\pgfpathcurveto{\pgfqpoint{-0.009821in}{-0.002605in}}{\pgfqpoint{-0.008786in}{-0.005103in}}{\pgfqpoint{-0.006944in}{-0.006944in}}%
\pgfpathcurveto{\pgfqpoint{-0.005103in}{-0.008786in}}{\pgfqpoint{-0.002605in}{-0.009821in}}{\pgfqpoint{0.000000in}{-0.009821in}}%
\pgfpathclose%
\pgfusepath{stroke,fill}%
}%
\begin{pgfscope}%
\pgfsys@transformshift{4.541940in}{0.562576in}%
\pgfsys@useobject{currentmarker}{}%
\end{pgfscope}%
\begin{pgfscope}%
\pgfsys@transformshift{4.692109in}{0.622926in}%
\pgfsys@useobject{currentmarker}{}%
\end{pgfscope}%
\begin{pgfscope}%
\pgfsys@transformshift{4.842278in}{0.712481in}%
\pgfsys@useobject{currentmarker}{}%
\end{pgfscope}%
\begin{pgfscope}%
\pgfsys@transformshift{4.992447in}{0.749607in}%
\pgfsys@useobject{currentmarker}{}%
\end{pgfscope}%
\begin{pgfscope}%
\pgfsys@transformshift{5.142617in}{0.856048in}%
\pgfsys@useobject{currentmarker}{}%
\end{pgfscope}%
\begin{pgfscope}%
\pgfsys@transformshift{5.292786in}{0.925155in}%
\pgfsys@useobject{currentmarker}{}%
\end{pgfscope}%
\begin{pgfscope}%
\pgfsys@transformshift{5.442955in}{1.124963in}%
\pgfsys@useobject{currentmarker}{}%
\end{pgfscope}%
\begin{pgfscope}%
\pgfsys@transformshift{5.593124in}{1.202452in}%
\pgfsys@useobject{currentmarker}{}%
\end{pgfscope}%
\begin{pgfscope}%
\pgfsys@transformshift{5.743293in}{1.330537in}%
\pgfsys@useobject{currentmarker}{}%
\end{pgfscope}%
\begin{pgfscope}%
\pgfsys@transformshift{5.893462in}{1.639698in}%
\pgfsys@useobject{currentmarker}{}%
\end{pgfscope}%
\begin{pgfscope}%
\pgfsys@transformshift{6.043631in}{1.945429in}%
\pgfsys@useobject{currentmarker}{}%
\end{pgfscope}%
\begin{pgfscope}%
\pgfsys@transformshift{6.193801in}{1.841743in}%
\pgfsys@useobject{currentmarker}{}%
\end{pgfscope}%
\begin{pgfscope}%
\pgfsys@transformshift{6.343970in}{1.884626in}%
\pgfsys@useobject{currentmarker}{}%
\end{pgfscope}%
\begin{pgfscope}%
\pgfsys@transformshift{6.494139in}{1.157118in}%
\pgfsys@useobject{currentmarker}{}%
\end{pgfscope}%
\begin{pgfscope}%
\pgfsys@transformshift{6.644308in}{1.267119in}%
\pgfsys@useobject{currentmarker}{}%
\end{pgfscope}%
\begin{pgfscope}%
\pgfsys@transformshift{6.794477in}{0.956078in}%
\pgfsys@useobject{currentmarker}{}%
\end{pgfscope}%
\begin{pgfscope}%
\pgfsys@transformshift{6.944646in}{0.866081in}%
\pgfsys@useobject{currentmarker}{}%
\end{pgfscope}%
\begin{pgfscope}%
\pgfsys@transformshift{7.094816in}{0.763140in}%
\pgfsys@useobject{currentmarker}{}%
\end{pgfscope}%
\begin{pgfscope}%
\pgfsys@transformshift{7.244985in}{0.769084in}%
\pgfsys@useobject{currentmarker}{}%
\end{pgfscope}%
\begin{pgfscope}%
\pgfsys@transformshift{7.395154in}{0.599136in}%
\pgfsys@useobject{currentmarker}{}%
\end{pgfscope}%
\begin{pgfscope}%
\pgfsys@transformshift{7.545323in}{0.584437in}%
\pgfsys@useobject{currentmarker}{}%
\end{pgfscope}%
\end{pgfscope}%
\begin{pgfscope}%
\pgfsetbuttcap%
\pgfsetroundjoin%
\definecolor{currentfill}{rgb}{0.000000,0.000000,0.000000}%
\pgfsetfillcolor{currentfill}%
\pgfsetlinewidth{0.803000pt}%
\definecolor{currentstroke}{rgb}{0.000000,0.000000,0.000000}%
\pgfsetstrokecolor{currentstroke}%
\pgfsetdash{}{0pt}%
\pgfsys@defobject{currentmarker}{\pgfqpoint{0.000000in}{-0.048611in}}{\pgfqpoint{0.000000in}{0.000000in}}{%
\pgfpathmoveto{\pgfqpoint{0.000000in}{0.000000in}}%
\pgfpathlineto{\pgfqpoint{0.000000in}{-0.048611in}}%
\pgfusepath{stroke,fill}%
}%
\begin{pgfscope}%
\pgfsys@transformshift{4.992447in}{0.387222in}%
\pgfsys@useobject{currentmarker}{}%
\end{pgfscope}%
\end{pgfscope}%
\begin{pgfscope}%
\definecolor{textcolor}{rgb}{0.000000,0.000000,0.000000}%
\pgfsetstrokecolor{textcolor}%
\pgfsetfillcolor{textcolor}%
\pgftext[x=4.992447in,y=0.290000in,,top]{\color{textcolor}\sffamily\fontsize{10.000000}{12.000000}\selectfont 2.20}%
\end{pgfscope}%
\begin{pgfscope}%
\pgfsetbuttcap%
\pgfsetroundjoin%
\definecolor{currentfill}{rgb}{0.000000,0.000000,0.000000}%
\pgfsetfillcolor{currentfill}%
\pgfsetlinewidth{0.803000pt}%
\definecolor{currentstroke}{rgb}{0.000000,0.000000,0.000000}%
\pgfsetstrokecolor{currentstroke}%
\pgfsetdash{}{0pt}%
\pgfsys@defobject{currentmarker}{\pgfqpoint{0.000000in}{-0.048611in}}{\pgfqpoint{0.000000in}{0.000000in}}{%
\pgfpathmoveto{\pgfqpoint{0.000000in}{0.000000in}}%
\pgfpathlineto{\pgfqpoint{0.000000in}{-0.048611in}}%
\pgfusepath{stroke,fill}%
}%
\begin{pgfscope}%
\pgfsys@transformshift{5.743293in}{0.387222in}%
\pgfsys@useobject{currentmarker}{}%
\end{pgfscope}%
\end{pgfscope}%
\begin{pgfscope}%
\definecolor{textcolor}{rgb}{0.000000,0.000000,0.000000}%
\pgfsetstrokecolor{textcolor}%
\pgfsetfillcolor{textcolor}%
\pgftext[x=5.743293in,y=0.290000in,,top]{\color{textcolor}\sffamily\fontsize{10.000000}{12.000000}\selectfont 2.25}%
\end{pgfscope}%
\begin{pgfscope}%
\pgfsetbuttcap%
\pgfsetroundjoin%
\definecolor{currentfill}{rgb}{0.000000,0.000000,0.000000}%
\pgfsetfillcolor{currentfill}%
\pgfsetlinewidth{0.803000pt}%
\definecolor{currentstroke}{rgb}{0.000000,0.000000,0.000000}%
\pgfsetstrokecolor{currentstroke}%
\pgfsetdash{}{0pt}%
\pgfsys@defobject{currentmarker}{\pgfqpoint{0.000000in}{-0.048611in}}{\pgfqpoint{0.000000in}{0.000000in}}{%
\pgfpathmoveto{\pgfqpoint{0.000000in}{0.000000in}}%
\pgfpathlineto{\pgfqpoint{0.000000in}{-0.048611in}}%
\pgfusepath{stroke,fill}%
}%
\begin{pgfscope}%
\pgfsys@transformshift{6.494139in}{0.387222in}%
\pgfsys@useobject{currentmarker}{}%
\end{pgfscope}%
\end{pgfscope}%
\begin{pgfscope}%
\definecolor{textcolor}{rgb}{0.000000,0.000000,0.000000}%
\pgfsetstrokecolor{textcolor}%
\pgfsetfillcolor{textcolor}%
\pgftext[x=6.494139in,y=0.290000in,,top]{\color{textcolor}\sffamily\fontsize{10.000000}{12.000000}\selectfont 2.30}%
\end{pgfscope}%
\begin{pgfscope}%
\pgfsetbuttcap%
\pgfsetroundjoin%
\definecolor{currentfill}{rgb}{0.000000,0.000000,0.000000}%
\pgfsetfillcolor{currentfill}%
\pgfsetlinewidth{0.803000pt}%
\definecolor{currentstroke}{rgb}{0.000000,0.000000,0.000000}%
\pgfsetstrokecolor{currentstroke}%
\pgfsetdash{}{0pt}%
\pgfsys@defobject{currentmarker}{\pgfqpoint{0.000000in}{-0.048611in}}{\pgfqpoint{0.000000in}{0.000000in}}{%
\pgfpathmoveto{\pgfqpoint{0.000000in}{0.000000in}}%
\pgfpathlineto{\pgfqpoint{0.000000in}{-0.048611in}}%
\pgfusepath{stroke,fill}%
}%
\begin{pgfscope}%
\pgfsys@transformshift{7.244985in}{0.387222in}%
\pgfsys@useobject{currentmarker}{}%
\end{pgfscope}%
\end{pgfscope}%
\begin{pgfscope}%
\definecolor{textcolor}{rgb}{0.000000,0.000000,0.000000}%
\pgfsetstrokecolor{textcolor}%
\pgfsetfillcolor{textcolor}%
\pgftext[x=7.244985in,y=0.290000in,,top]{\color{textcolor}\sffamily\fontsize{10.000000}{12.000000}\selectfont 2.35}%
\end{pgfscope}%
\begin{pgfscope}%
\pgfsetbuttcap%
\pgfsetroundjoin%
\definecolor{currentfill}{rgb}{0.000000,0.000000,0.000000}%
\pgfsetfillcolor{currentfill}%
\pgfsetlinewidth{0.803000pt}%
\definecolor{currentstroke}{rgb}{0.000000,0.000000,0.000000}%
\pgfsetstrokecolor{currentstroke}%
\pgfsetdash{}{0pt}%
\pgfsys@defobject{currentmarker}{\pgfqpoint{-0.048611in}{0.000000in}}{\pgfqpoint{0.000000in}{0.000000in}}{%
\pgfpathmoveto{\pgfqpoint{0.000000in}{0.000000in}}%
\pgfpathlineto{\pgfqpoint{-0.048611in}{0.000000in}}%
\pgfusepath{stroke,fill}%
}%
\begin{pgfscope}%
\pgfsys@transformshift{4.382652in}{0.864551in}%
\pgfsys@useobject{currentmarker}{}%
\end{pgfscope}%
\end{pgfscope}%
\begin{pgfscope}%
\definecolor{textcolor}{rgb}{0.000000,0.000000,0.000000}%
\pgfsetstrokecolor{textcolor}%
\pgfsetfillcolor{textcolor}%
\pgftext[x=4.064551in,y=0.811790in,left,base]{\color{textcolor}\sffamily\fontsize{10.000000}{12.000000}\selectfont 1.5}%
\end{pgfscope}%
\begin{pgfscope}%
\pgfsetbuttcap%
\pgfsetroundjoin%
\definecolor{currentfill}{rgb}{0.000000,0.000000,0.000000}%
\pgfsetfillcolor{currentfill}%
\pgfsetlinewidth{0.803000pt}%
\definecolor{currentstroke}{rgb}{0.000000,0.000000,0.000000}%
\pgfsetstrokecolor{currentstroke}%
\pgfsetdash{}{0pt}%
\pgfsys@defobject{currentmarker}{\pgfqpoint{-0.048611in}{0.000000in}}{\pgfqpoint{0.000000in}{0.000000in}}{%
\pgfpathmoveto{\pgfqpoint{0.000000in}{0.000000in}}%
\pgfpathlineto{\pgfqpoint{-0.048611in}{0.000000in}}%
\pgfusepath{stroke,fill}%
}%
\begin{pgfscope}%
\pgfsys@transformshift{4.382652in}{1.407673in}%
\pgfsys@useobject{currentmarker}{}%
\end{pgfscope}%
\end{pgfscope}%
\begin{pgfscope}%
\definecolor{textcolor}{rgb}{0.000000,0.000000,0.000000}%
\pgfsetstrokecolor{textcolor}%
\pgfsetfillcolor{textcolor}%
\pgftext[x=4.064551in,y=1.354911in,left,base]{\color{textcolor}\sffamily\fontsize{10.000000}{12.000000}\selectfont 2.0}%
\end{pgfscope}%
\begin{pgfscope}%
\pgfsetbuttcap%
\pgfsetroundjoin%
\definecolor{currentfill}{rgb}{0.000000,0.000000,0.000000}%
\pgfsetfillcolor{currentfill}%
\pgfsetlinewidth{0.803000pt}%
\definecolor{currentstroke}{rgb}{0.000000,0.000000,0.000000}%
\pgfsetstrokecolor{currentstroke}%
\pgfsetdash{}{0pt}%
\pgfsys@defobject{currentmarker}{\pgfqpoint{-0.048611in}{0.000000in}}{\pgfqpoint{0.000000in}{0.000000in}}{%
\pgfpathmoveto{\pgfqpoint{0.000000in}{0.000000in}}%
\pgfpathlineto{\pgfqpoint{-0.048611in}{0.000000in}}%
\pgfusepath{stroke,fill}%
}%
\begin{pgfscope}%
\pgfsys@transformshift{4.382652in}{1.950795in}%
\pgfsys@useobject{currentmarker}{}%
\end{pgfscope}%
\end{pgfscope}%
\begin{pgfscope}%
\definecolor{textcolor}{rgb}{0.000000,0.000000,0.000000}%
\pgfsetstrokecolor{textcolor}%
\pgfsetfillcolor{textcolor}%
\pgftext[x=4.064551in,y=1.898033in,left,base]{\color{textcolor}\sffamily\fontsize{10.000000}{12.000000}\selectfont 2.5}%
\end{pgfscope}%
\begin{pgfscope}%
\pgfsetbuttcap%
\pgfsetroundjoin%
\definecolor{currentfill}{rgb}{0.000000,0.000000,0.000000}%
\pgfsetfillcolor{currentfill}%
\pgfsetlinewidth{0.803000pt}%
\definecolor{currentstroke}{rgb}{0.000000,0.000000,0.000000}%
\pgfsetstrokecolor{currentstroke}%
\pgfsetdash{}{0pt}%
\pgfsys@defobject{currentmarker}{\pgfqpoint{-0.048611in}{0.000000in}}{\pgfqpoint{0.000000in}{0.000000in}}{%
\pgfpathmoveto{\pgfqpoint{0.000000in}{0.000000in}}%
\pgfpathlineto{\pgfqpoint{-0.048611in}{0.000000in}}%
\pgfusepath{stroke,fill}%
}%
\begin{pgfscope}%
\pgfsys@transformshift{4.382652in}{2.493917in}%
\pgfsys@useobject{currentmarker}{}%
\end{pgfscope}%
\end{pgfscope}%
\begin{pgfscope}%
\definecolor{textcolor}{rgb}{0.000000,0.000000,0.000000}%
\pgfsetstrokecolor{textcolor}%
\pgfsetfillcolor{textcolor}%
\pgftext[x=4.064551in,y=2.441155in,left,base]{\color{textcolor}\sffamily\fontsize{10.000000}{12.000000}\selectfont 3.0}%
\end{pgfscope}%
\begin{pgfscope}%
\pgfpathrectangle{\pgfqpoint{4.382652in}{0.387222in}}{\pgfqpoint{3.321958in}{2.324444in}}%
\pgfusepath{clip}%
\pgfsetrectcap%
\pgfsetroundjoin%
\pgfsetlinewidth{1.505625pt}%
\definecolor{currentstroke}{rgb}{0.121569,0.466667,0.705882}%
\pgfsetstrokecolor{currentstroke}%
\pgfsetdash{}{0pt}%
\pgfpathmoveto{\pgfqpoint{4.541940in}{0.562576in}}%
\pgfpathlineto{\pgfqpoint{4.692109in}{0.622926in}}%
\pgfpathlineto{\pgfqpoint{4.842278in}{0.712481in}}%
\pgfpathlineto{\pgfqpoint{4.992447in}{0.749607in}}%
\pgfpathlineto{\pgfqpoint{5.142617in}{0.856048in}}%
\pgfpathlineto{\pgfqpoint{5.292786in}{0.925155in}}%
\pgfpathlineto{\pgfqpoint{5.442955in}{1.124963in}}%
\pgfpathlineto{\pgfqpoint{5.593124in}{1.202452in}}%
\pgfpathlineto{\pgfqpoint{5.743293in}{1.330537in}}%
\pgfpathlineto{\pgfqpoint{5.893462in}{1.639698in}}%
\pgfpathlineto{\pgfqpoint{6.043631in}{1.945429in}}%
\pgfpathlineto{\pgfqpoint{6.193801in}{1.841743in}}%
\pgfpathlineto{\pgfqpoint{6.343970in}{1.884626in}}%
\pgfpathlineto{\pgfqpoint{6.494139in}{1.157118in}}%
\pgfpathlineto{\pgfqpoint{6.644308in}{1.267119in}}%
\pgfpathlineto{\pgfqpoint{6.794477in}{0.956078in}}%
\pgfpathlineto{\pgfqpoint{6.944646in}{0.866081in}}%
\pgfpathlineto{\pgfqpoint{7.094816in}{0.763140in}}%
\pgfpathlineto{\pgfqpoint{7.244985in}{0.769084in}}%
\pgfpathlineto{\pgfqpoint{7.395154in}{0.599136in}}%
\pgfpathlineto{\pgfqpoint{7.545323in}{0.584437in}}%
\pgfusepath{stroke}%
\end{pgfscope}%
\begin{pgfscope}%
\pgfsetrectcap%
\pgfsetmiterjoin%
\pgfsetlinewidth{0.803000pt}%
\definecolor{currentstroke}{rgb}{0.000000,0.000000,0.000000}%
\pgfsetstrokecolor{currentstroke}%
\pgfsetdash{}{0pt}%
\pgfpathmoveto{\pgfqpoint{4.382652in}{0.387222in}}%
\pgfpathlineto{\pgfqpoint{4.382652in}{2.711667in}}%
\pgfusepath{stroke}%
\end{pgfscope}%
\begin{pgfscope}%
\pgfsetrectcap%
\pgfsetmiterjoin%
\pgfsetlinewidth{0.803000pt}%
\definecolor{currentstroke}{rgb}{0.000000,0.000000,0.000000}%
\pgfsetstrokecolor{currentstroke}%
\pgfsetdash{}{0pt}%
\pgfpathmoveto{\pgfqpoint{7.704610in}{0.387222in}}%
\pgfpathlineto{\pgfqpoint{7.704610in}{2.711667in}}%
\pgfusepath{stroke}%
\end{pgfscope}%
\begin{pgfscope}%
\pgfsetrectcap%
\pgfsetmiterjoin%
\pgfsetlinewidth{0.803000pt}%
\definecolor{currentstroke}{rgb}{0.000000,0.000000,0.000000}%
\pgfsetstrokecolor{currentstroke}%
\pgfsetdash{}{0pt}%
\pgfpathmoveto{\pgfqpoint{4.382652in}{0.387222in}}%
\pgfpathlineto{\pgfqpoint{7.704610in}{0.387222in}}%
\pgfusepath{stroke}%
\end{pgfscope}%
\begin{pgfscope}%
\pgfsetrectcap%
\pgfsetmiterjoin%
\pgfsetlinewidth{0.803000pt}%
\definecolor{currentstroke}{rgb}{0.000000,0.000000,0.000000}%
\pgfsetstrokecolor{currentstroke}%
\pgfsetdash{}{0pt}%
\pgfpathmoveto{\pgfqpoint{4.382652in}{2.711667in}}%
\pgfpathlineto{\pgfqpoint{7.704610in}{2.711667in}}%
\pgfusepath{stroke}%
\end{pgfscope}%
\begin{pgfscope}%
\definecolor{textcolor}{rgb}{0.000000,0.000000,0.000000}%
\pgfsetstrokecolor{textcolor}%
\pgfsetfillcolor{textcolor}%
\pgftext[x=6.043631in,y=2.795000in,,base]{\color{textcolor}\sffamily\fontsize{12.000000}{14.400000}\selectfont \(\displaystyle  N = 128 \)}%
\end{pgfscope}%
\end{pgfpicture}%
\makeatother%
\endgroup%
}~
\scalebox{0.75}{%% Creator: Matplotlib, PGF backend
%%
%% To include the figure in your LaTeX document, write
%%   \input{<filename>.pgf}
%%
%% Make sure the required packages are loaded in your preamble
%%   \usepackage{pgf}
%%
%% Figures using additional raster images can only be included by \input if
%% they are in the same directory as the main LaTeX file. For loading figures
%% from other directories you can use the `import` package
%%   \usepackage{import}
%% and then include the figures with
%%   \import{<path to file>}{<filename>.pgf}
%%
%% Matplotlib used the following preamble
%%   \usepackage{fontspec}
%%   \setmainfont{DejaVuSerif.ttf}[Path=/home/lzh/anaconda3/envs/numana/lib/python3.7/site-packages/matplotlib/mpl-data/fonts/ttf/]
%%   \setsansfont{DejaVuSans.ttf}[Path=/home/lzh/anaconda3/envs/numana/lib/python3.7/site-packages/matplotlib/mpl-data/fonts/ttf/]
%%   \setmonofont{DejaVuSansMono.ttf}[Path=/home/lzh/anaconda3/envs/numana/lib/python3.7/site-packages/matplotlib/mpl-data/fonts/ttf/]
%%
\begingroup%
\makeatletter%
\begin{pgfpicture}%
\pgfpathrectangle{\pgfpointorigin}{\pgfqpoint{4.000000in}{4.000000in}}%
\pgfusepath{use as bounding box, clip}%
\begin{pgfscope}%
\pgfsetbuttcap%
\pgfsetmiterjoin%
\definecolor{currentfill}{rgb}{1.000000,1.000000,1.000000}%
\pgfsetfillcolor{currentfill}%
\pgfsetlinewidth{0.000000pt}%
\definecolor{currentstroke}{rgb}{1.000000,1.000000,1.000000}%
\pgfsetstrokecolor{currentstroke}%
\pgfsetdash{}{0pt}%
\pgfpathmoveto{\pgfqpoint{0.000000in}{0.000000in}}%
\pgfpathlineto{\pgfqpoint{4.000000in}{0.000000in}}%
\pgfpathlineto{\pgfqpoint{4.000000in}{4.000000in}}%
\pgfpathlineto{\pgfqpoint{0.000000in}{4.000000in}}%
\pgfpathclose%
\pgfusepath{fill}%
\end{pgfscope}%
\begin{pgfscope}%
\pgfsetbuttcap%
\pgfsetmiterjoin%
\definecolor{currentfill}{rgb}{1.000000,1.000000,1.000000}%
\pgfsetfillcolor{currentfill}%
\pgfsetlinewidth{0.000000pt}%
\definecolor{currentstroke}{rgb}{0.000000,0.000000,0.000000}%
\pgfsetstrokecolor{currentstroke}%
\pgfsetstrokeopacity{0.000000}%
\pgfsetdash{}{0pt}%
\pgfpathmoveto{\pgfqpoint{0.510000in}{0.440000in}}%
\pgfpathlineto{\pgfqpoint{3.590000in}{0.440000in}}%
\pgfpathlineto{\pgfqpoint{3.590000in}{3.520000in}}%
\pgfpathlineto{\pgfqpoint{0.510000in}{3.520000in}}%
\pgfpathclose%
\pgfusepath{fill}%
\end{pgfscope}%
\begin{pgfscope}%
\pgfpathrectangle{\pgfqpoint{0.510000in}{0.440000in}}{\pgfqpoint{3.080000in}{3.080000in}}%
\pgfusepath{clip}%
\pgfsetbuttcap%
\pgfsetroundjoin%
\definecolor{currentfill}{rgb}{0.121569,0.466667,0.705882}%
\pgfsetfillcolor{currentfill}%
\pgfsetlinewidth{1.003750pt}%
\definecolor{currentstroke}{rgb}{0.121569,0.466667,0.705882}%
\pgfsetstrokecolor{currentstroke}%
\pgfsetdash{}{0pt}%
\pgfpathmoveto{\pgfqpoint{2.399861in}{1.097926in}}%
\pgfpathcurveto{\pgfqpoint{2.405685in}{1.097926in}}{\pgfqpoint{2.411271in}{1.100239in}}{\pgfqpoint{2.415389in}{1.104358in}}%
\pgfpathcurveto{\pgfqpoint{2.419507in}{1.108476in}}{\pgfqpoint{2.421821in}{1.114062in}}{\pgfqpoint{2.421821in}{1.119886in}}%
\pgfpathcurveto{\pgfqpoint{2.421821in}{1.125710in}}{\pgfqpoint{2.419507in}{1.131296in}}{\pgfqpoint{2.415389in}{1.135414in}}%
\pgfpathcurveto{\pgfqpoint{2.411271in}{1.139532in}}{\pgfqpoint{2.405685in}{1.141846in}}{\pgfqpoint{2.399861in}{1.141846in}}%
\pgfpathcurveto{\pgfqpoint{2.394037in}{1.141846in}}{\pgfqpoint{2.388451in}{1.139532in}}{\pgfqpoint{2.384333in}{1.135414in}}%
\pgfpathcurveto{\pgfqpoint{2.380214in}{1.131296in}}{\pgfqpoint{2.377901in}{1.125710in}}{\pgfqpoint{2.377901in}{1.119886in}}%
\pgfpathcurveto{\pgfqpoint{2.377901in}{1.114062in}}{\pgfqpoint{2.380214in}{1.108476in}}{\pgfqpoint{2.384333in}{1.104358in}}%
\pgfpathcurveto{\pgfqpoint{2.388451in}{1.100239in}}{\pgfqpoint{2.394037in}{1.097926in}}{\pgfqpoint{2.399861in}{1.097926in}}%
\pgfpathclose%
\pgfusepath{stroke,fill}%
\end{pgfscope}%
\begin{pgfscope}%
\pgfpathrectangle{\pgfqpoint{0.510000in}{0.440000in}}{\pgfqpoint{3.080000in}{3.080000in}}%
\pgfusepath{clip}%
\pgfsetbuttcap%
\pgfsetroundjoin%
\definecolor{currentfill}{rgb}{1.000000,0.498039,0.054902}%
\pgfsetfillcolor{currentfill}%
\pgfsetlinewidth{1.003750pt}%
\definecolor{currentstroke}{rgb}{1.000000,0.498039,0.054902}%
\pgfsetstrokecolor{currentstroke}%
\pgfsetdash{}{0pt}%
\pgfpathmoveto{\pgfqpoint{1.460253in}{2.307901in}}%
\pgfpathcurveto{\pgfqpoint{1.466077in}{2.307901in}}{\pgfqpoint{1.471664in}{2.310214in}}{\pgfqpoint{1.475782in}{2.314333in}}%
\pgfpathcurveto{\pgfqpoint{1.479900in}{2.318451in}}{\pgfqpoint{1.482214in}{2.324037in}}{\pgfqpoint{1.482214in}{2.329861in}}%
\pgfpathcurveto{\pgfqpoint{1.482214in}{2.335685in}}{\pgfqpoint{1.479900in}{2.341271in}}{\pgfqpoint{1.475782in}{2.345389in}}%
\pgfpathcurveto{\pgfqpoint{1.471664in}{2.349507in}}{\pgfqpoint{1.466077in}{2.351821in}}{\pgfqpoint{1.460253in}{2.351821in}}%
\pgfpathcurveto{\pgfqpoint{1.454429in}{2.351821in}}{\pgfqpoint{1.448843in}{2.349507in}}{\pgfqpoint{1.444725in}{2.345389in}}%
\pgfpathcurveto{\pgfqpoint{1.440607in}{2.341271in}}{\pgfqpoint{1.438293in}{2.335685in}}{\pgfqpoint{1.438293in}{2.329861in}}%
\pgfpathcurveto{\pgfqpoint{1.438293in}{2.324037in}}{\pgfqpoint{1.440607in}{2.318451in}}{\pgfqpoint{1.444725in}{2.314333in}}%
\pgfpathcurveto{\pgfqpoint{1.448843in}{2.310214in}}{\pgfqpoint{1.454429in}{2.307901in}}{\pgfqpoint{1.460253in}{2.307901in}}%
\pgfpathclose%
\pgfusepath{stroke,fill}%
\end{pgfscope}%
\begin{pgfscope}%
\pgfpathrectangle{\pgfqpoint{0.510000in}{0.440000in}}{\pgfqpoint{3.080000in}{3.080000in}}%
\pgfusepath{clip}%
\pgfsetbuttcap%
\pgfsetroundjoin%
\definecolor{currentfill}{rgb}{0.172549,0.627451,0.172549}%
\pgfsetfillcolor{currentfill}%
\pgfsetlinewidth{1.003750pt}%
\definecolor{currentstroke}{rgb}{0.172549,0.627451,0.172549}%
\pgfsetstrokecolor{currentstroke}%
\pgfsetdash{}{0pt}%
\pgfpathmoveto{\pgfqpoint{1.189886in}{1.368293in}}%
\pgfpathcurveto{\pgfqpoint{1.195710in}{1.368293in}}{\pgfqpoint{1.201296in}{1.370607in}}{\pgfqpoint{1.205414in}{1.374725in}}%
\pgfpathcurveto{\pgfqpoint{1.209532in}{1.378843in}}{\pgfqpoint{1.211846in}{1.384429in}}{\pgfqpoint{1.211846in}{1.390253in}}%
\pgfpathcurveto{\pgfqpoint{1.211846in}{1.396077in}}{\pgfqpoint{1.209532in}{1.401664in}}{\pgfqpoint{1.205414in}{1.405782in}}%
\pgfpathcurveto{\pgfqpoint{1.201296in}{1.409900in}}{\pgfqpoint{1.195710in}{1.412214in}}{\pgfqpoint{1.189886in}{1.412214in}}%
\pgfpathcurveto{\pgfqpoint{1.184062in}{1.412214in}}{\pgfqpoint{1.178476in}{1.409900in}}{\pgfqpoint{1.174358in}{1.405782in}}%
\pgfpathcurveto{\pgfqpoint{1.170239in}{1.401664in}}{\pgfqpoint{1.167926in}{1.396077in}}{\pgfqpoint{1.167926in}{1.390253in}}%
\pgfpathcurveto{\pgfqpoint{1.167926in}{1.384429in}}{\pgfqpoint{1.170239in}{1.378843in}}{\pgfqpoint{1.174358in}{1.374725in}}%
\pgfpathcurveto{\pgfqpoint{1.178476in}{1.370607in}}{\pgfqpoint{1.184062in}{1.368293in}}{\pgfqpoint{1.189886in}{1.368293in}}%
\pgfpathclose%
\pgfusepath{stroke,fill}%
\end{pgfscope}%
\begin{pgfscope}%
\pgfpathrectangle{\pgfqpoint{0.510000in}{0.440000in}}{\pgfqpoint{3.080000in}{3.080000in}}%
\pgfusepath{clip}%
\pgfsetbuttcap%
\pgfsetmiterjoin%
\definecolor{currentfill}{rgb}{0.100000,0.100000,0.100000}%
\pgfsetfillcolor{currentfill}%
\pgfsetfillopacity{0.100000}%
\pgfsetlinewidth{0.301125pt}%
\definecolor{currentstroke}{rgb}{0.000000,0.000000,0.000000}%
\pgfsetstrokecolor{currentstroke}%
\pgfsetdash{}{0pt}%
\pgfpathmoveto{\pgfqpoint{0.950000in}{0.880000in}}%
\pgfpathlineto{\pgfqpoint{0.950000in}{3.080000in}}%
\pgfpathlineto{\pgfqpoint{3.150000in}{0.880000in}}%
\pgfpathclose%
\pgfusepath{stroke,fill}%
\end{pgfscope}%
\begin{pgfscope}%
\pgfpathrectangle{\pgfqpoint{0.510000in}{0.440000in}}{\pgfqpoint{3.080000in}{3.080000in}}%
\pgfusepath{clip}%
\pgfsetrectcap%
\pgfsetroundjoin%
\pgfsetlinewidth{0.803000pt}%
\definecolor{currentstroke}{rgb}{0.690196,0.690196,0.690196}%
\pgfsetstrokecolor{currentstroke}%
\pgfsetdash{}{0pt}%
\pgfpathmoveto{\pgfqpoint{0.510000in}{0.440000in}}%
\pgfpathlineto{\pgfqpoint{0.510000in}{3.520000in}}%
\pgfusepath{stroke}%
\end{pgfscope}%
\begin{pgfscope}%
\pgfsetbuttcap%
\pgfsetroundjoin%
\definecolor{currentfill}{rgb}{0.000000,0.000000,0.000000}%
\pgfsetfillcolor{currentfill}%
\pgfsetlinewidth{0.803000pt}%
\definecolor{currentstroke}{rgb}{0.000000,0.000000,0.000000}%
\pgfsetstrokecolor{currentstroke}%
\pgfsetdash{}{0pt}%
\pgfsys@defobject{currentmarker}{\pgfqpoint{0.000000in}{-0.048611in}}{\pgfqpoint{0.000000in}{0.000000in}}{%
\pgfpathmoveto{\pgfqpoint{0.000000in}{0.000000in}}%
\pgfpathlineto{\pgfqpoint{0.000000in}{-0.048611in}}%
\pgfusepath{stroke,fill}%
}%
\begin{pgfscope}%
\pgfsys@transformshift{0.510000in}{0.440000in}%
\pgfsys@useobject{currentmarker}{}%
\end{pgfscope}%
\end{pgfscope}%
\begin{pgfscope}%
\definecolor{textcolor}{rgb}{0.000000,0.000000,0.000000}%
\pgfsetstrokecolor{textcolor}%
\pgfsetfillcolor{textcolor}%
\pgftext[x=0.510000in,y=0.342778in,,top]{\color{textcolor}\sffamily\fontsize{10.000000}{12.000000}\selectfont -0.2}%
\end{pgfscope}%
\begin{pgfscope}%
\pgfpathrectangle{\pgfqpoint{0.510000in}{0.440000in}}{\pgfqpoint{3.080000in}{3.080000in}}%
\pgfusepath{clip}%
\pgfsetrectcap%
\pgfsetroundjoin%
\pgfsetlinewidth{0.803000pt}%
\definecolor{currentstroke}{rgb}{0.690196,0.690196,0.690196}%
\pgfsetstrokecolor{currentstroke}%
\pgfsetdash{}{0pt}%
\pgfpathmoveto{\pgfqpoint{0.950000in}{0.440000in}}%
\pgfpathlineto{\pgfqpoint{0.950000in}{3.520000in}}%
\pgfusepath{stroke}%
\end{pgfscope}%
\begin{pgfscope}%
\pgfsetbuttcap%
\pgfsetroundjoin%
\definecolor{currentfill}{rgb}{0.000000,0.000000,0.000000}%
\pgfsetfillcolor{currentfill}%
\pgfsetlinewidth{0.803000pt}%
\definecolor{currentstroke}{rgb}{0.000000,0.000000,0.000000}%
\pgfsetstrokecolor{currentstroke}%
\pgfsetdash{}{0pt}%
\pgfsys@defobject{currentmarker}{\pgfqpoint{0.000000in}{-0.048611in}}{\pgfqpoint{0.000000in}{0.000000in}}{%
\pgfpathmoveto{\pgfqpoint{0.000000in}{0.000000in}}%
\pgfpathlineto{\pgfqpoint{0.000000in}{-0.048611in}}%
\pgfusepath{stroke,fill}%
}%
\begin{pgfscope}%
\pgfsys@transformshift{0.950000in}{0.440000in}%
\pgfsys@useobject{currentmarker}{}%
\end{pgfscope}%
\end{pgfscope}%
\begin{pgfscope}%
\definecolor{textcolor}{rgb}{0.000000,0.000000,0.000000}%
\pgfsetstrokecolor{textcolor}%
\pgfsetfillcolor{textcolor}%
\pgftext[x=0.950000in,y=0.342778in,,top]{\color{textcolor}\sffamily\fontsize{10.000000}{12.000000}\selectfont 0.0}%
\end{pgfscope}%
\begin{pgfscope}%
\pgfpathrectangle{\pgfqpoint{0.510000in}{0.440000in}}{\pgfqpoint{3.080000in}{3.080000in}}%
\pgfusepath{clip}%
\pgfsetrectcap%
\pgfsetroundjoin%
\pgfsetlinewidth{0.803000pt}%
\definecolor{currentstroke}{rgb}{0.690196,0.690196,0.690196}%
\pgfsetstrokecolor{currentstroke}%
\pgfsetdash{}{0pt}%
\pgfpathmoveto{\pgfqpoint{1.390000in}{0.440000in}}%
\pgfpathlineto{\pgfqpoint{1.390000in}{3.520000in}}%
\pgfusepath{stroke}%
\end{pgfscope}%
\begin{pgfscope}%
\pgfsetbuttcap%
\pgfsetroundjoin%
\definecolor{currentfill}{rgb}{0.000000,0.000000,0.000000}%
\pgfsetfillcolor{currentfill}%
\pgfsetlinewidth{0.803000pt}%
\definecolor{currentstroke}{rgb}{0.000000,0.000000,0.000000}%
\pgfsetstrokecolor{currentstroke}%
\pgfsetdash{}{0pt}%
\pgfsys@defobject{currentmarker}{\pgfqpoint{0.000000in}{-0.048611in}}{\pgfqpoint{0.000000in}{0.000000in}}{%
\pgfpathmoveto{\pgfqpoint{0.000000in}{0.000000in}}%
\pgfpathlineto{\pgfqpoint{0.000000in}{-0.048611in}}%
\pgfusepath{stroke,fill}%
}%
\begin{pgfscope}%
\pgfsys@transformshift{1.390000in}{0.440000in}%
\pgfsys@useobject{currentmarker}{}%
\end{pgfscope}%
\end{pgfscope}%
\begin{pgfscope}%
\definecolor{textcolor}{rgb}{0.000000,0.000000,0.000000}%
\pgfsetstrokecolor{textcolor}%
\pgfsetfillcolor{textcolor}%
\pgftext[x=1.390000in,y=0.342778in,,top]{\color{textcolor}\sffamily\fontsize{10.000000}{12.000000}\selectfont 0.2}%
\end{pgfscope}%
\begin{pgfscope}%
\pgfpathrectangle{\pgfqpoint{0.510000in}{0.440000in}}{\pgfqpoint{3.080000in}{3.080000in}}%
\pgfusepath{clip}%
\pgfsetrectcap%
\pgfsetroundjoin%
\pgfsetlinewidth{0.803000pt}%
\definecolor{currentstroke}{rgb}{0.690196,0.690196,0.690196}%
\pgfsetstrokecolor{currentstroke}%
\pgfsetdash{}{0pt}%
\pgfpathmoveto{\pgfqpoint{1.830000in}{0.440000in}}%
\pgfpathlineto{\pgfqpoint{1.830000in}{3.520000in}}%
\pgfusepath{stroke}%
\end{pgfscope}%
\begin{pgfscope}%
\pgfsetbuttcap%
\pgfsetroundjoin%
\definecolor{currentfill}{rgb}{0.000000,0.000000,0.000000}%
\pgfsetfillcolor{currentfill}%
\pgfsetlinewidth{0.803000pt}%
\definecolor{currentstroke}{rgb}{0.000000,0.000000,0.000000}%
\pgfsetstrokecolor{currentstroke}%
\pgfsetdash{}{0pt}%
\pgfsys@defobject{currentmarker}{\pgfqpoint{0.000000in}{-0.048611in}}{\pgfqpoint{0.000000in}{0.000000in}}{%
\pgfpathmoveto{\pgfqpoint{0.000000in}{0.000000in}}%
\pgfpathlineto{\pgfqpoint{0.000000in}{-0.048611in}}%
\pgfusepath{stroke,fill}%
}%
\begin{pgfscope}%
\pgfsys@transformshift{1.830000in}{0.440000in}%
\pgfsys@useobject{currentmarker}{}%
\end{pgfscope}%
\end{pgfscope}%
\begin{pgfscope}%
\definecolor{textcolor}{rgb}{0.000000,0.000000,0.000000}%
\pgfsetstrokecolor{textcolor}%
\pgfsetfillcolor{textcolor}%
\pgftext[x=1.830000in,y=0.342778in,,top]{\color{textcolor}\sffamily\fontsize{10.000000}{12.000000}\selectfont 0.4}%
\end{pgfscope}%
\begin{pgfscope}%
\pgfpathrectangle{\pgfqpoint{0.510000in}{0.440000in}}{\pgfqpoint{3.080000in}{3.080000in}}%
\pgfusepath{clip}%
\pgfsetrectcap%
\pgfsetroundjoin%
\pgfsetlinewidth{0.803000pt}%
\definecolor{currentstroke}{rgb}{0.690196,0.690196,0.690196}%
\pgfsetstrokecolor{currentstroke}%
\pgfsetdash{}{0pt}%
\pgfpathmoveto{\pgfqpoint{2.270000in}{0.440000in}}%
\pgfpathlineto{\pgfqpoint{2.270000in}{3.520000in}}%
\pgfusepath{stroke}%
\end{pgfscope}%
\begin{pgfscope}%
\pgfsetbuttcap%
\pgfsetroundjoin%
\definecolor{currentfill}{rgb}{0.000000,0.000000,0.000000}%
\pgfsetfillcolor{currentfill}%
\pgfsetlinewidth{0.803000pt}%
\definecolor{currentstroke}{rgb}{0.000000,0.000000,0.000000}%
\pgfsetstrokecolor{currentstroke}%
\pgfsetdash{}{0pt}%
\pgfsys@defobject{currentmarker}{\pgfqpoint{0.000000in}{-0.048611in}}{\pgfqpoint{0.000000in}{0.000000in}}{%
\pgfpathmoveto{\pgfqpoint{0.000000in}{0.000000in}}%
\pgfpathlineto{\pgfqpoint{0.000000in}{-0.048611in}}%
\pgfusepath{stroke,fill}%
}%
\begin{pgfscope}%
\pgfsys@transformshift{2.270000in}{0.440000in}%
\pgfsys@useobject{currentmarker}{}%
\end{pgfscope}%
\end{pgfscope}%
\begin{pgfscope}%
\definecolor{textcolor}{rgb}{0.000000,0.000000,0.000000}%
\pgfsetstrokecolor{textcolor}%
\pgfsetfillcolor{textcolor}%
\pgftext[x=2.270000in,y=0.342778in,,top]{\color{textcolor}\sffamily\fontsize{10.000000}{12.000000}\selectfont 0.6}%
\end{pgfscope}%
\begin{pgfscope}%
\pgfpathrectangle{\pgfqpoint{0.510000in}{0.440000in}}{\pgfqpoint{3.080000in}{3.080000in}}%
\pgfusepath{clip}%
\pgfsetrectcap%
\pgfsetroundjoin%
\pgfsetlinewidth{0.803000pt}%
\definecolor{currentstroke}{rgb}{0.690196,0.690196,0.690196}%
\pgfsetstrokecolor{currentstroke}%
\pgfsetdash{}{0pt}%
\pgfpathmoveto{\pgfqpoint{2.710000in}{0.440000in}}%
\pgfpathlineto{\pgfqpoint{2.710000in}{3.520000in}}%
\pgfusepath{stroke}%
\end{pgfscope}%
\begin{pgfscope}%
\pgfsetbuttcap%
\pgfsetroundjoin%
\definecolor{currentfill}{rgb}{0.000000,0.000000,0.000000}%
\pgfsetfillcolor{currentfill}%
\pgfsetlinewidth{0.803000pt}%
\definecolor{currentstroke}{rgb}{0.000000,0.000000,0.000000}%
\pgfsetstrokecolor{currentstroke}%
\pgfsetdash{}{0pt}%
\pgfsys@defobject{currentmarker}{\pgfqpoint{0.000000in}{-0.048611in}}{\pgfqpoint{0.000000in}{0.000000in}}{%
\pgfpathmoveto{\pgfqpoint{0.000000in}{0.000000in}}%
\pgfpathlineto{\pgfqpoint{0.000000in}{-0.048611in}}%
\pgfusepath{stroke,fill}%
}%
\begin{pgfscope}%
\pgfsys@transformshift{2.710000in}{0.440000in}%
\pgfsys@useobject{currentmarker}{}%
\end{pgfscope}%
\end{pgfscope}%
\begin{pgfscope}%
\definecolor{textcolor}{rgb}{0.000000,0.000000,0.000000}%
\pgfsetstrokecolor{textcolor}%
\pgfsetfillcolor{textcolor}%
\pgftext[x=2.710000in,y=0.342778in,,top]{\color{textcolor}\sffamily\fontsize{10.000000}{12.000000}\selectfont 0.8}%
\end{pgfscope}%
\begin{pgfscope}%
\pgfpathrectangle{\pgfqpoint{0.510000in}{0.440000in}}{\pgfqpoint{3.080000in}{3.080000in}}%
\pgfusepath{clip}%
\pgfsetrectcap%
\pgfsetroundjoin%
\pgfsetlinewidth{0.803000pt}%
\definecolor{currentstroke}{rgb}{0.690196,0.690196,0.690196}%
\pgfsetstrokecolor{currentstroke}%
\pgfsetdash{}{0pt}%
\pgfpathmoveto{\pgfqpoint{3.150000in}{0.440000in}}%
\pgfpathlineto{\pgfqpoint{3.150000in}{3.520000in}}%
\pgfusepath{stroke}%
\end{pgfscope}%
\begin{pgfscope}%
\pgfsetbuttcap%
\pgfsetroundjoin%
\definecolor{currentfill}{rgb}{0.000000,0.000000,0.000000}%
\pgfsetfillcolor{currentfill}%
\pgfsetlinewidth{0.803000pt}%
\definecolor{currentstroke}{rgb}{0.000000,0.000000,0.000000}%
\pgfsetstrokecolor{currentstroke}%
\pgfsetdash{}{0pt}%
\pgfsys@defobject{currentmarker}{\pgfqpoint{0.000000in}{-0.048611in}}{\pgfqpoint{0.000000in}{0.000000in}}{%
\pgfpathmoveto{\pgfqpoint{0.000000in}{0.000000in}}%
\pgfpathlineto{\pgfqpoint{0.000000in}{-0.048611in}}%
\pgfusepath{stroke,fill}%
}%
\begin{pgfscope}%
\pgfsys@transformshift{3.150000in}{0.440000in}%
\pgfsys@useobject{currentmarker}{}%
\end{pgfscope}%
\end{pgfscope}%
\begin{pgfscope}%
\definecolor{textcolor}{rgb}{0.000000,0.000000,0.000000}%
\pgfsetstrokecolor{textcolor}%
\pgfsetfillcolor{textcolor}%
\pgftext[x=3.150000in,y=0.342778in,,top]{\color{textcolor}\sffamily\fontsize{10.000000}{12.000000}\selectfont 1.0}%
\end{pgfscope}%
\begin{pgfscope}%
\pgfpathrectangle{\pgfqpoint{0.510000in}{0.440000in}}{\pgfqpoint{3.080000in}{3.080000in}}%
\pgfusepath{clip}%
\pgfsetrectcap%
\pgfsetroundjoin%
\pgfsetlinewidth{0.803000pt}%
\definecolor{currentstroke}{rgb}{0.690196,0.690196,0.690196}%
\pgfsetstrokecolor{currentstroke}%
\pgfsetdash{}{0pt}%
\pgfpathmoveto{\pgfqpoint{3.590000in}{0.440000in}}%
\pgfpathlineto{\pgfqpoint{3.590000in}{3.520000in}}%
\pgfusepath{stroke}%
\end{pgfscope}%
\begin{pgfscope}%
\pgfsetbuttcap%
\pgfsetroundjoin%
\definecolor{currentfill}{rgb}{0.000000,0.000000,0.000000}%
\pgfsetfillcolor{currentfill}%
\pgfsetlinewidth{0.803000pt}%
\definecolor{currentstroke}{rgb}{0.000000,0.000000,0.000000}%
\pgfsetstrokecolor{currentstroke}%
\pgfsetdash{}{0pt}%
\pgfsys@defobject{currentmarker}{\pgfqpoint{0.000000in}{-0.048611in}}{\pgfqpoint{0.000000in}{0.000000in}}{%
\pgfpathmoveto{\pgfqpoint{0.000000in}{0.000000in}}%
\pgfpathlineto{\pgfqpoint{0.000000in}{-0.048611in}}%
\pgfusepath{stroke,fill}%
}%
\begin{pgfscope}%
\pgfsys@transformshift{3.590000in}{0.440000in}%
\pgfsys@useobject{currentmarker}{}%
\end{pgfscope}%
\end{pgfscope}%
\begin{pgfscope}%
\definecolor{textcolor}{rgb}{0.000000,0.000000,0.000000}%
\pgfsetstrokecolor{textcolor}%
\pgfsetfillcolor{textcolor}%
\pgftext[x=3.590000in,y=0.342778in,,top]{\color{textcolor}\sffamily\fontsize{10.000000}{12.000000}\selectfont 1.2}%
\end{pgfscope}%
\begin{pgfscope}%
\pgfpathrectangle{\pgfqpoint{0.510000in}{0.440000in}}{\pgfqpoint{3.080000in}{3.080000in}}%
\pgfusepath{clip}%
\pgfsetrectcap%
\pgfsetroundjoin%
\pgfsetlinewidth{0.803000pt}%
\definecolor{currentstroke}{rgb}{0.690196,0.690196,0.690196}%
\pgfsetstrokecolor{currentstroke}%
\pgfsetdash{}{0pt}%
\pgfpathmoveto{\pgfqpoint{0.510000in}{0.440000in}}%
\pgfpathlineto{\pgfqpoint{3.590000in}{0.440000in}}%
\pgfusepath{stroke}%
\end{pgfscope}%
\begin{pgfscope}%
\pgfsetbuttcap%
\pgfsetroundjoin%
\definecolor{currentfill}{rgb}{0.000000,0.000000,0.000000}%
\pgfsetfillcolor{currentfill}%
\pgfsetlinewidth{0.803000pt}%
\definecolor{currentstroke}{rgb}{0.000000,0.000000,0.000000}%
\pgfsetstrokecolor{currentstroke}%
\pgfsetdash{}{0pt}%
\pgfsys@defobject{currentmarker}{\pgfqpoint{-0.048611in}{0.000000in}}{\pgfqpoint{0.000000in}{0.000000in}}{%
\pgfpathmoveto{\pgfqpoint{0.000000in}{0.000000in}}%
\pgfpathlineto{\pgfqpoint{-0.048611in}{0.000000in}}%
\pgfusepath{stroke,fill}%
}%
\begin{pgfscope}%
\pgfsys@transformshift{0.510000in}{0.440000in}%
\pgfsys@useobject{currentmarker}{}%
\end{pgfscope}%
\end{pgfscope}%
\begin{pgfscope}%
\definecolor{textcolor}{rgb}{0.000000,0.000000,0.000000}%
\pgfsetstrokecolor{textcolor}%
\pgfsetfillcolor{textcolor}%
\pgftext[x=0.075525in,y=0.387238in,left,base]{\color{textcolor}\sffamily\fontsize{10.000000}{12.000000}\selectfont -0.2}%
\end{pgfscope}%
\begin{pgfscope}%
\pgfpathrectangle{\pgfqpoint{0.510000in}{0.440000in}}{\pgfqpoint{3.080000in}{3.080000in}}%
\pgfusepath{clip}%
\pgfsetrectcap%
\pgfsetroundjoin%
\pgfsetlinewidth{0.803000pt}%
\definecolor{currentstroke}{rgb}{0.690196,0.690196,0.690196}%
\pgfsetstrokecolor{currentstroke}%
\pgfsetdash{}{0pt}%
\pgfpathmoveto{\pgfqpoint{0.510000in}{0.880000in}}%
\pgfpathlineto{\pgfqpoint{3.590000in}{0.880000in}}%
\pgfusepath{stroke}%
\end{pgfscope}%
\begin{pgfscope}%
\pgfsetbuttcap%
\pgfsetroundjoin%
\definecolor{currentfill}{rgb}{0.000000,0.000000,0.000000}%
\pgfsetfillcolor{currentfill}%
\pgfsetlinewidth{0.803000pt}%
\definecolor{currentstroke}{rgb}{0.000000,0.000000,0.000000}%
\pgfsetstrokecolor{currentstroke}%
\pgfsetdash{}{0pt}%
\pgfsys@defobject{currentmarker}{\pgfqpoint{-0.048611in}{0.000000in}}{\pgfqpoint{0.000000in}{0.000000in}}{%
\pgfpathmoveto{\pgfqpoint{0.000000in}{0.000000in}}%
\pgfpathlineto{\pgfqpoint{-0.048611in}{0.000000in}}%
\pgfusepath{stroke,fill}%
}%
\begin{pgfscope}%
\pgfsys@transformshift{0.510000in}{0.880000in}%
\pgfsys@useobject{currentmarker}{}%
\end{pgfscope}%
\end{pgfscope}%
\begin{pgfscope}%
\definecolor{textcolor}{rgb}{0.000000,0.000000,0.000000}%
\pgfsetstrokecolor{textcolor}%
\pgfsetfillcolor{textcolor}%
\pgftext[x=0.191898in,y=0.827238in,left,base]{\color{textcolor}\sffamily\fontsize{10.000000}{12.000000}\selectfont 0.0}%
\end{pgfscope}%
\begin{pgfscope}%
\pgfpathrectangle{\pgfqpoint{0.510000in}{0.440000in}}{\pgfqpoint{3.080000in}{3.080000in}}%
\pgfusepath{clip}%
\pgfsetrectcap%
\pgfsetroundjoin%
\pgfsetlinewidth{0.803000pt}%
\definecolor{currentstroke}{rgb}{0.690196,0.690196,0.690196}%
\pgfsetstrokecolor{currentstroke}%
\pgfsetdash{}{0pt}%
\pgfpathmoveto{\pgfqpoint{0.510000in}{1.320000in}}%
\pgfpathlineto{\pgfqpoint{3.590000in}{1.320000in}}%
\pgfusepath{stroke}%
\end{pgfscope}%
\begin{pgfscope}%
\pgfsetbuttcap%
\pgfsetroundjoin%
\definecolor{currentfill}{rgb}{0.000000,0.000000,0.000000}%
\pgfsetfillcolor{currentfill}%
\pgfsetlinewidth{0.803000pt}%
\definecolor{currentstroke}{rgb}{0.000000,0.000000,0.000000}%
\pgfsetstrokecolor{currentstroke}%
\pgfsetdash{}{0pt}%
\pgfsys@defobject{currentmarker}{\pgfqpoint{-0.048611in}{0.000000in}}{\pgfqpoint{0.000000in}{0.000000in}}{%
\pgfpathmoveto{\pgfqpoint{0.000000in}{0.000000in}}%
\pgfpathlineto{\pgfqpoint{-0.048611in}{0.000000in}}%
\pgfusepath{stroke,fill}%
}%
\begin{pgfscope}%
\pgfsys@transformshift{0.510000in}{1.320000in}%
\pgfsys@useobject{currentmarker}{}%
\end{pgfscope}%
\end{pgfscope}%
\begin{pgfscope}%
\definecolor{textcolor}{rgb}{0.000000,0.000000,0.000000}%
\pgfsetstrokecolor{textcolor}%
\pgfsetfillcolor{textcolor}%
\pgftext[x=0.191898in,y=1.267238in,left,base]{\color{textcolor}\sffamily\fontsize{10.000000}{12.000000}\selectfont 0.2}%
\end{pgfscope}%
\begin{pgfscope}%
\pgfpathrectangle{\pgfqpoint{0.510000in}{0.440000in}}{\pgfqpoint{3.080000in}{3.080000in}}%
\pgfusepath{clip}%
\pgfsetrectcap%
\pgfsetroundjoin%
\pgfsetlinewidth{0.803000pt}%
\definecolor{currentstroke}{rgb}{0.690196,0.690196,0.690196}%
\pgfsetstrokecolor{currentstroke}%
\pgfsetdash{}{0pt}%
\pgfpathmoveto{\pgfqpoint{0.510000in}{1.760000in}}%
\pgfpathlineto{\pgfqpoint{3.590000in}{1.760000in}}%
\pgfusepath{stroke}%
\end{pgfscope}%
\begin{pgfscope}%
\pgfsetbuttcap%
\pgfsetroundjoin%
\definecolor{currentfill}{rgb}{0.000000,0.000000,0.000000}%
\pgfsetfillcolor{currentfill}%
\pgfsetlinewidth{0.803000pt}%
\definecolor{currentstroke}{rgb}{0.000000,0.000000,0.000000}%
\pgfsetstrokecolor{currentstroke}%
\pgfsetdash{}{0pt}%
\pgfsys@defobject{currentmarker}{\pgfqpoint{-0.048611in}{0.000000in}}{\pgfqpoint{0.000000in}{0.000000in}}{%
\pgfpathmoveto{\pgfqpoint{0.000000in}{0.000000in}}%
\pgfpathlineto{\pgfqpoint{-0.048611in}{0.000000in}}%
\pgfusepath{stroke,fill}%
}%
\begin{pgfscope}%
\pgfsys@transformshift{0.510000in}{1.760000in}%
\pgfsys@useobject{currentmarker}{}%
\end{pgfscope}%
\end{pgfscope}%
\begin{pgfscope}%
\definecolor{textcolor}{rgb}{0.000000,0.000000,0.000000}%
\pgfsetstrokecolor{textcolor}%
\pgfsetfillcolor{textcolor}%
\pgftext[x=0.191898in,y=1.707238in,left,base]{\color{textcolor}\sffamily\fontsize{10.000000}{12.000000}\selectfont 0.4}%
\end{pgfscope}%
\begin{pgfscope}%
\pgfpathrectangle{\pgfqpoint{0.510000in}{0.440000in}}{\pgfqpoint{3.080000in}{3.080000in}}%
\pgfusepath{clip}%
\pgfsetrectcap%
\pgfsetroundjoin%
\pgfsetlinewidth{0.803000pt}%
\definecolor{currentstroke}{rgb}{0.690196,0.690196,0.690196}%
\pgfsetstrokecolor{currentstroke}%
\pgfsetdash{}{0pt}%
\pgfpathmoveto{\pgfqpoint{0.510000in}{2.200000in}}%
\pgfpathlineto{\pgfqpoint{3.590000in}{2.200000in}}%
\pgfusepath{stroke}%
\end{pgfscope}%
\begin{pgfscope}%
\pgfsetbuttcap%
\pgfsetroundjoin%
\definecolor{currentfill}{rgb}{0.000000,0.000000,0.000000}%
\pgfsetfillcolor{currentfill}%
\pgfsetlinewidth{0.803000pt}%
\definecolor{currentstroke}{rgb}{0.000000,0.000000,0.000000}%
\pgfsetstrokecolor{currentstroke}%
\pgfsetdash{}{0pt}%
\pgfsys@defobject{currentmarker}{\pgfqpoint{-0.048611in}{0.000000in}}{\pgfqpoint{0.000000in}{0.000000in}}{%
\pgfpathmoveto{\pgfqpoint{0.000000in}{0.000000in}}%
\pgfpathlineto{\pgfqpoint{-0.048611in}{0.000000in}}%
\pgfusepath{stroke,fill}%
}%
\begin{pgfscope}%
\pgfsys@transformshift{0.510000in}{2.200000in}%
\pgfsys@useobject{currentmarker}{}%
\end{pgfscope}%
\end{pgfscope}%
\begin{pgfscope}%
\definecolor{textcolor}{rgb}{0.000000,0.000000,0.000000}%
\pgfsetstrokecolor{textcolor}%
\pgfsetfillcolor{textcolor}%
\pgftext[x=0.191898in,y=2.147238in,left,base]{\color{textcolor}\sffamily\fontsize{10.000000}{12.000000}\selectfont 0.6}%
\end{pgfscope}%
\begin{pgfscope}%
\pgfpathrectangle{\pgfqpoint{0.510000in}{0.440000in}}{\pgfqpoint{3.080000in}{3.080000in}}%
\pgfusepath{clip}%
\pgfsetrectcap%
\pgfsetroundjoin%
\pgfsetlinewidth{0.803000pt}%
\definecolor{currentstroke}{rgb}{0.690196,0.690196,0.690196}%
\pgfsetstrokecolor{currentstroke}%
\pgfsetdash{}{0pt}%
\pgfpathmoveto{\pgfqpoint{0.510000in}{2.640000in}}%
\pgfpathlineto{\pgfqpoint{3.590000in}{2.640000in}}%
\pgfusepath{stroke}%
\end{pgfscope}%
\begin{pgfscope}%
\pgfsetbuttcap%
\pgfsetroundjoin%
\definecolor{currentfill}{rgb}{0.000000,0.000000,0.000000}%
\pgfsetfillcolor{currentfill}%
\pgfsetlinewidth{0.803000pt}%
\definecolor{currentstroke}{rgb}{0.000000,0.000000,0.000000}%
\pgfsetstrokecolor{currentstroke}%
\pgfsetdash{}{0pt}%
\pgfsys@defobject{currentmarker}{\pgfqpoint{-0.048611in}{0.000000in}}{\pgfqpoint{0.000000in}{0.000000in}}{%
\pgfpathmoveto{\pgfqpoint{0.000000in}{0.000000in}}%
\pgfpathlineto{\pgfqpoint{-0.048611in}{0.000000in}}%
\pgfusepath{stroke,fill}%
}%
\begin{pgfscope}%
\pgfsys@transformshift{0.510000in}{2.640000in}%
\pgfsys@useobject{currentmarker}{}%
\end{pgfscope}%
\end{pgfscope}%
\begin{pgfscope}%
\definecolor{textcolor}{rgb}{0.000000,0.000000,0.000000}%
\pgfsetstrokecolor{textcolor}%
\pgfsetfillcolor{textcolor}%
\pgftext[x=0.191898in,y=2.587238in,left,base]{\color{textcolor}\sffamily\fontsize{10.000000}{12.000000}\selectfont 0.8}%
\end{pgfscope}%
\begin{pgfscope}%
\pgfpathrectangle{\pgfqpoint{0.510000in}{0.440000in}}{\pgfqpoint{3.080000in}{3.080000in}}%
\pgfusepath{clip}%
\pgfsetrectcap%
\pgfsetroundjoin%
\pgfsetlinewidth{0.803000pt}%
\definecolor{currentstroke}{rgb}{0.690196,0.690196,0.690196}%
\pgfsetstrokecolor{currentstroke}%
\pgfsetdash{}{0pt}%
\pgfpathmoveto{\pgfqpoint{0.510000in}{3.080000in}}%
\pgfpathlineto{\pgfqpoint{3.590000in}{3.080000in}}%
\pgfusepath{stroke}%
\end{pgfscope}%
\begin{pgfscope}%
\pgfsetbuttcap%
\pgfsetroundjoin%
\definecolor{currentfill}{rgb}{0.000000,0.000000,0.000000}%
\pgfsetfillcolor{currentfill}%
\pgfsetlinewidth{0.803000pt}%
\definecolor{currentstroke}{rgb}{0.000000,0.000000,0.000000}%
\pgfsetstrokecolor{currentstroke}%
\pgfsetdash{}{0pt}%
\pgfsys@defobject{currentmarker}{\pgfqpoint{-0.048611in}{0.000000in}}{\pgfqpoint{0.000000in}{0.000000in}}{%
\pgfpathmoveto{\pgfqpoint{0.000000in}{0.000000in}}%
\pgfpathlineto{\pgfqpoint{-0.048611in}{0.000000in}}%
\pgfusepath{stroke,fill}%
}%
\begin{pgfscope}%
\pgfsys@transformshift{0.510000in}{3.080000in}%
\pgfsys@useobject{currentmarker}{}%
\end{pgfscope}%
\end{pgfscope}%
\begin{pgfscope}%
\definecolor{textcolor}{rgb}{0.000000,0.000000,0.000000}%
\pgfsetstrokecolor{textcolor}%
\pgfsetfillcolor{textcolor}%
\pgftext[x=0.191898in,y=3.027238in,left,base]{\color{textcolor}\sffamily\fontsize{10.000000}{12.000000}\selectfont 1.0}%
\end{pgfscope}%
\begin{pgfscope}%
\pgfpathrectangle{\pgfqpoint{0.510000in}{0.440000in}}{\pgfqpoint{3.080000in}{3.080000in}}%
\pgfusepath{clip}%
\pgfsetrectcap%
\pgfsetroundjoin%
\pgfsetlinewidth{0.803000pt}%
\definecolor{currentstroke}{rgb}{0.690196,0.690196,0.690196}%
\pgfsetstrokecolor{currentstroke}%
\pgfsetdash{}{0pt}%
\pgfpathmoveto{\pgfqpoint{0.510000in}{3.520000in}}%
\pgfpathlineto{\pgfqpoint{3.590000in}{3.520000in}}%
\pgfusepath{stroke}%
\end{pgfscope}%
\begin{pgfscope}%
\pgfsetbuttcap%
\pgfsetroundjoin%
\definecolor{currentfill}{rgb}{0.000000,0.000000,0.000000}%
\pgfsetfillcolor{currentfill}%
\pgfsetlinewidth{0.803000pt}%
\definecolor{currentstroke}{rgb}{0.000000,0.000000,0.000000}%
\pgfsetstrokecolor{currentstroke}%
\pgfsetdash{}{0pt}%
\pgfsys@defobject{currentmarker}{\pgfqpoint{-0.048611in}{0.000000in}}{\pgfqpoint{0.000000in}{0.000000in}}{%
\pgfpathmoveto{\pgfqpoint{0.000000in}{0.000000in}}%
\pgfpathlineto{\pgfqpoint{-0.048611in}{0.000000in}}%
\pgfusepath{stroke,fill}%
}%
\begin{pgfscope}%
\pgfsys@transformshift{0.510000in}{3.520000in}%
\pgfsys@useobject{currentmarker}{}%
\end{pgfscope}%
\end{pgfscope}%
\begin{pgfscope}%
\definecolor{textcolor}{rgb}{0.000000,0.000000,0.000000}%
\pgfsetstrokecolor{textcolor}%
\pgfsetfillcolor{textcolor}%
\pgftext[x=0.191898in,y=3.467238in,left,base]{\color{textcolor}\sffamily\fontsize{10.000000}{12.000000}\selectfont 1.2}%
\end{pgfscope}%
\begin{pgfscope}%
\pgfsetrectcap%
\pgfsetmiterjoin%
\pgfsetlinewidth{0.803000pt}%
\definecolor{currentstroke}{rgb}{0.000000,0.000000,0.000000}%
\pgfsetstrokecolor{currentstroke}%
\pgfsetdash{}{0pt}%
\pgfpathmoveto{\pgfqpoint{0.510000in}{0.440000in}}%
\pgfpathlineto{\pgfqpoint{0.510000in}{3.520000in}}%
\pgfusepath{stroke}%
\end{pgfscope}%
\begin{pgfscope}%
\pgfsetrectcap%
\pgfsetmiterjoin%
\pgfsetlinewidth{0.803000pt}%
\definecolor{currentstroke}{rgb}{0.000000,0.000000,0.000000}%
\pgfsetstrokecolor{currentstroke}%
\pgfsetdash{}{0pt}%
\pgfpathmoveto{\pgfqpoint{3.590000in}{0.440000in}}%
\pgfpathlineto{\pgfqpoint{3.590000in}{3.520000in}}%
\pgfusepath{stroke}%
\end{pgfscope}%
\begin{pgfscope}%
\pgfsetrectcap%
\pgfsetmiterjoin%
\pgfsetlinewidth{0.803000pt}%
\definecolor{currentstroke}{rgb}{0.000000,0.000000,0.000000}%
\pgfsetstrokecolor{currentstroke}%
\pgfsetdash{}{0pt}%
\pgfpathmoveto{\pgfqpoint{0.510000in}{0.440000in}}%
\pgfpathlineto{\pgfqpoint{3.590000in}{0.440000in}}%
\pgfusepath{stroke}%
\end{pgfscope}%
\begin{pgfscope}%
\pgfsetrectcap%
\pgfsetmiterjoin%
\pgfsetlinewidth{0.803000pt}%
\definecolor{currentstroke}{rgb}{0.000000,0.000000,0.000000}%
\pgfsetstrokecolor{currentstroke}%
\pgfsetdash{}{0pt}%
\pgfpathmoveto{\pgfqpoint{0.510000in}{3.520000in}}%
\pgfpathlineto{\pgfqpoint{3.590000in}{3.520000in}}%
\pgfusepath{stroke}%
\end{pgfscope}%
\begin{pgfscope}%
\definecolor{textcolor}{rgb}{0.000000,0.000000,0.000000}%
\pgfsetstrokecolor{textcolor}%
\pgfsetfillcolor{textcolor}%
\pgftext[x=2.399861in,y=1.119886in,left,base]{\color{textcolor}\sffamily\fontsize{10.000000}{12.000000}\selectfont 0.16667}%
\end{pgfscope}%
\begin{pgfscope}%
\definecolor{textcolor}{rgb}{0.000000,0.000000,0.000000}%
\pgfsetstrokecolor{textcolor}%
\pgfsetfillcolor{textcolor}%
\pgftext[x=1.460253in,y=2.329861in,left,base]{\color{textcolor}\sffamily\fontsize{10.000000}{12.000000}\selectfont 0.16667}%
\end{pgfscope}%
\begin{pgfscope}%
\definecolor{textcolor}{rgb}{0.000000,0.000000,0.000000}%
\pgfsetstrokecolor{textcolor}%
\pgfsetfillcolor{textcolor}%
\pgftext[x=1.189886in,y=1.390253in,left,base]{\color{textcolor}\sffamily\fontsize{10.000000}{12.000000}\selectfont 0.16667}%
\end{pgfscope}%
\end{pgfpicture}%
\makeatother%
\endgroup%
}
\caption{Figures of the two-dimensional cubatures of $ D = 2 $ with extra constraint $ \pbr{ 0, 3 } $}
\label{Fig:M2D203}
\end{figure}

If we constrain $ \mathcal{I}_2 \cup \cbr{ \pbr{ 1, 2 } } $, the results are shown in Figure \ref{Fig:M2D212}.

\begin{figure}[htbp]
\centering
\scalebox{0.75}{%% Creator: Matplotlib, PGF backend
%%
%% To include the figure in your LaTeX document, write
%%   \input{<filename>.pgf}
%%
%% Make sure the required packages are loaded in your preamble
%%   \usepackage{pgf}
%%
%% Figures using additional raster images can only be included by \input if
%% they are in the same directory as the main LaTeX file. For loading figures
%% from other directories you can use the `import` package
%%   \usepackage{import}
%% and then include the figures with
%%   \import{<path to file>}{<filename>.pgf}
%%
%% Matplotlib used the following preamble
%%   \usepackage{fontspec}
%%   \setmainfont{DejaVuSerif.ttf}[Path=/home/lzh/anaconda3/envs/numana/lib/python3.7/site-packages/matplotlib/mpl-data/fonts/ttf/]
%%   \setsansfont{DejaVuSans.ttf}[Path=/home/lzh/anaconda3/envs/numana/lib/python3.7/site-packages/matplotlib/mpl-data/fonts/ttf/]
%%   \setmonofont{DejaVuSansMono.ttf}[Path=/home/lzh/anaconda3/envs/numana/lib/python3.7/site-packages/matplotlib/mpl-data/fonts/ttf/]
%%
\begingroup%
\makeatletter%
\begin{pgfpicture}%
\pgfpathrectangle{\pgfpointorigin}{\pgfqpoint{8.000000in}{6.000000in}}%
\pgfusepath{use as bounding box, clip}%
\begin{pgfscope}%
\pgfsetbuttcap%
\pgfsetmiterjoin%
\definecolor{currentfill}{rgb}{1.000000,1.000000,1.000000}%
\pgfsetfillcolor{currentfill}%
\pgfsetlinewidth{0.000000pt}%
\definecolor{currentstroke}{rgb}{1.000000,1.000000,1.000000}%
\pgfsetstrokecolor{currentstroke}%
\pgfsetdash{}{0pt}%
\pgfpathmoveto{\pgfqpoint{0.000000in}{0.000000in}}%
\pgfpathlineto{\pgfqpoint{8.000000in}{0.000000in}}%
\pgfpathlineto{\pgfqpoint{8.000000in}{6.000000in}}%
\pgfpathlineto{\pgfqpoint{0.000000in}{6.000000in}}%
\pgfpathclose%
\pgfusepath{fill}%
\end{pgfscope}%
\begin{pgfscope}%
\pgfsetbuttcap%
\pgfsetmiterjoin%
\definecolor{currentfill}{rgb}{1.000000,1.000000,1.000000}%
\pgfsetfillcolor{currentfill}%
\pgfsetlinewidth{0.000000pt}%
\definecolor{currentstroke}{rgb}{0.000000,0.000000,0.000000}%
\pgfsetstrokecolor{currentstroke}%
\pgfsetstrokeopacity{0.000000}%
\pgfsetdash{}{0pt}%
\pgfpathmoveto{\pgfqpoint{0.672222in}{3.312222in}}%
\pgfpathlineto{\pgfqpoint{3.890000in}{3.312222in}}%
\pgfpathlineto{\pgfqpoint{3.890000in}{5.636667in}}%
\pgfpathlineto{\pgfqpoint{0.672222in}{5.636667in}}%
\pgfpathclose%
\pgfusepath{fill}%
\end{pgfscope}%
\begin{pgfscope}%
\pgfpathrectangle{\pgfqpoint{0.672222in}{3.312222in}}{\pgfqpoint{3.217778in}{2.324444in}}%
\pgfusepath{clip}%
\pgfsetbuttcap%
\pgfsetroundjoin%
\definecolor{currentfill}{rgb}{0.121569,0.466667,0.705882}%
\pgfsetfillcolor{currentfill}%
\pgfsetfillopacity{0.300000}%
\pgfsetlinewidth{1.003750pt}%
\definecolor{currentstroke}{rgb}{0.121569,0.466667,0.705882}%
\pgfsetstrokecolor{currentstroke}%
\pgfsetstrokeopacity{0.300000}%
\pgfsetdash{}{0pt}%
\pgfpathmoveto{\pgfqpoint{0.826514in}{3.425650in}}%
\pgfpathlineto{\pgfqpoint{0.826514in}{3.425648in}}%
\pgfpathlineto{\pgfqpoint{0.899244in}{3.425665in}}%
\pgfpathlineto{\pgfqpoint{0.971974in}{3.425767in}}%
\pgfpathlineto{\pgfqpoint{1.044704in}{3.426155in}}%
\pgfpathlineto{\pgfqpoint{1.117434in}{3.427212in}}%
\pgfpathlineto{\pgfqpoint{1.190163in}{3.429563in}}%
\pgfpathlineto{\pgfqpoint{1.262893in}{3.433928in}}%
\pgfpathlineto{\pgfqpoint{1.335623in}{3.441303in}}%
\pgfpathlineto{\pgfqpoint{1.408353in}{3.452807in}}%
\pgfpathlineto{\pgfqpoint{1.481083in}{3.469795in}}%
\pgfpathlineto{\pgfqpoint{1.553813in}{3.492940in}}%
\pgfpathlineto{\pgfqpoint{1.626543in}{3.524630in}}%
\pgfpathlineto{\pgfqpoint{1.699272in}{3.566402in}}%
\pgfpathlineto{\pgfqpoint{1.772002in}{3.620217in}}%
\pgfpathlineto{\pgfqpoint{1.844732in}{3.688966in}}%
\pgfpathlineto{\pgfqpoint{1.917462in}{3.777105in}}%
\pgfpathlineto{\pgfqpoint{1.990192in}{3.891902in}}%
\pgfpathlineto{\pgfqpoint{2.062922in}{4.044190in}}%
\pgfpathlineto{\pgfqpoint{2.135651in}{4.241990in}}%
\pgfpathlineto{\pgfqpoint{2.208381in}{4.455357in}}%
\pgfpathlineto{\pgfqpoint{2.281111in}{4.624489in}}%
\pgfpathlineto{\pgfqpoint{2.353841in}{4.753316in}}%
\pgfpathlineto{\pgfqpoint{2.426571in}{4.851801in}}%
\pgfpathlineto{\pgfqpoint{2.499301in}{4.933408in}}%
\pgfpathlineto{\pgfqpoint{2.572030in}{5.001048in}}%
\pgfpathlineto{\pgfqpoint{2.644760in}{5.061396in}}%
\pgfpathlineto{\pgfqpoint{2.717490in}{5.113311in}}%
\pgfpathlineto{\pgfqpoint{2.790220in}{5.161246in}}%
\pgfpathlineto{\pgfqpoint{2.862950in}{5.204103in}}%
\pgfpathlineto{\pgfqpoint{2.935680in}{5.242741in}}%
\pgfpathlineto{\pgfqpoint{3.008410in}{5.279032in}}%
\pgfpathlineto{\pgfqpoint{3.081139in}{5.311407in}}%
\pgfpathlineto{\pgfqpoint{3.153869in}{5.342045in}}%
\pgfpathlineto{\pgfqpoint{3.226599in}{5.370615in}}%
\pgfpathlineto{\pgfqpoint{3.299329in}{5.396394in}}%
\pgfpathlineto{\pgfqpoint{3.372059in}{5.421360in}}%
\pgfpathlineto{\pgfqpoint{3.444789in}{5.444375in}}%
\pgfpathlineto{\pgfqpoint{3.517518in}{5.465453in}}%
\pgfpathlineto{\pgfqpoint{3.590248in}{5.485698in}}%
\pgfpathlineto{\pgfqpoint{3.662978in}{5.504586in}}%
\pgfpathlineto{\pgfqpoint{3.735708in}{5.522952in}}%
\pgfpathlineto{\pgfqpoint{3.735708in}{5.523528in}}%
\pgfpathlineto{\pgfqpoint{3.735708in}{5.523528in}}%
\pgfpathlineto{\pgfqpoint{3.662978in}{5.505592in}}%
\pgfpathlineto{\pgfqpoint{3.590248in}{5.486570in}}%
\pgfpathlineto{\pgfqpoint{3.517518in}{5.466219in}}%
\pgfpathlineto{\pgfqpoint{3.444789in}{5.444646in}}%
\pgfpathlineto{\pgfqpoint{3.372059in}{5.421778in}}%
\pgfpathlineto{\pgfqpoint{3.299329in}{5.397511in}}%
\pgfpathlineto{\pgfqpoint{3.226599in}{5.370763in}}%
\pgfpathlineto{\pgfqpoint{3.153869in}{5.342771in}}%
\pgfpathlineto{\pgfqpoint{3.081139in}{5.312337in}}%
\pgfpathlineto{\pgfqpoint{3.008410in}{5.279350in}}%
\pgfpathlineto{\pgfqpoint{2.935680in}{5.243987in}}%
\pgfpathlineto{\pgfqpoint{2.862950in}{5.204829in}}%
\pgfpathlineto{\pgfqpoint{2.790220in}{5.161721in}}%
\pgfpathlineto{\pgfqpoint{2.717490in}{5.115062in}}%
\pgfpathlineto{\pgfqpoint{2.644760in}{5.061928in}}%
\pgfpathlineto{\pgfqpoint{2.572030in}{5.002476in}}%
\pgfpathlineto{\pgfqpoint{2.499301in}{4.933888in}}%
\pgfpathlineto{\pgfqpoint{2.426571in}{4.854401in}}%
\pgfpathlineto{\pgfqpoint{2.353841in}{4.755491in}}%
\pgfpathlineto{\pgfqpoint{2.281111in}{4.630368in}}%
\pgfpathlineto{\pgfqpoint{2.208381in}{4.457565in}}%
\pgfpathlineto{\pgfqpoint{2.135651in}{4.245544in}}%
\pgfpathlineto{\pgfqpoint{2.062922in}{4.049394in}}%
\pgfpathlineto{\pgfqpoint{1.990192in}{3.894235in}}%
\pgfpathlineto{\pgfqpoint{1.917462in}{3.777793in}}%
\pgfpathlineto{\pgfqpoint{1.844732in}{3.689594in}}%
\pgfpathlineto{\pgfqpoint{1.772002in}{3.620519in}}%
\pgfpathlineto{\pgfqpoint{1.699272in}{3.566529in}}%
\pgfpathlineto{\pgfqpoint{1.626543in}{3.524749in}}%
\pgfpathlineto{\pgfqpoint{1.553813in}{3.493316in}}%
\pgfpathlineto{\pgfqpoint{1.481083in}{3.469834in}}%
\pgfpathlineto{\pgfqpoint{1.408353in}{3.453025in}}%
\pgfpathlineto{\pgfqpoint{1.335623in}{3.441483in}}%
\pgfpathlineto{\pgfqpoint{1.262893in}{3.434064in}}%
\pgfpathlineto{\pgfqpoint{1.190163in}{3.429604in}}%
\pgfpathlineto{\pgfqpoint{1.117434in}{3.427243in}}%
\pgfpathlineto{\pgfqpoint{1.044704in}{3.426162in}}%
\pgfpathlineto{\pgfqpoint{0.971974in}{3.425772in}}%
\pgfpathlineto{\pgfqpoint{0.899244in}{3.425668in}}%
\pgfpathlineto{\pgfqpoint{0.826514in}{3.425650in}}%
\pgfpathclose%
\pgfusepath{stroke,fill}%
\end{pgfscope}%
\begin{pgfscope}%
\pgfpathrectangle{\pgfqpoint{0.672222in}{3.312222in}}{\pgfqpoint{3.217778in}{2.324444in}}%
\pgfusepath{clip}%
\pgfsetbuttcap%
\pgfsetroundjoin%
\definecolor{currentfill}{rgb}{0.121569,0.466667,0.705882}%
\pgfsetfillcolor{currentfill}%
\pgfsetlinewidth{1.003750pt}%
\definecolor{currentstroke}{rgb}{0.121569,0.466667,0.705882}%
\pgfsetstrokecolor{currentstroke}%
\pgfsetdash{}{0pt}%
\pgfsys@defobject{currentmarker}{\pgfqpoint{-0.009821in}{-0.009821in}}{\pgfqpoint{0.009821in}{0.009821in}}{%
\pgfpathmoveto{\pgfqpoint{0.000000in}{-0.009821in}}%
\pgfpathcurveto{\pgfqpoint{0.002605in}{-0.009821in}}{\pgfqpoint{0.005103in}{-0.008786in}}{\pgfqpoint{0.006944in}{-0.006944in}}%
\pgfpathcurveto{\pgfqpoint{0.008786in}{-0.005103in}}{\pgfqpoint{0.009821in}{-0.002605in}}{\pgfqpoint{0.009821in}{0.000000in}}%
\pgfpathcurveto{\pgfqpoint{0.009821in}{0.002605in}}{\pgfqpoint{0.008786in}{0.005103in}}{\pgfqpoint{0.006944in}{0.006944in}}%
\pgfpathcurveto{\pgfqpoint{0.005103in}{0.008786in}}{\pgfqpoint{0.002605in}{0.009821in}}{\pgfqpoint{0.000000in}{0.009821in}}%
\pgfpathcurveto{\pgfqpoint{-0.002605in}{0.009821in}}{\pgfqpoint{-0.005103in}{0.008786in}}{\pgfqpoint{-0.006944in}{0.006944in}}%
\pgfpathcurveto{\pgfqpoint{-0.008786in}{0.005103in}}{\pgfqpoint{-0.009821in}{0.002605in}}{\pgfqpoint{-0.009821in}{0.000000in}}%
\pgfpathcurveto{\pgfqpoint{-0.009821in}{-0.002605in}}{\pgfqpoint{-0.008786in}{-0.005103in}}{\pgfqpoint{-0.006944in}{-0.006944in}}%
\pgfpathcurveto{\pgfqpoint{-0.005103in}{-0.008786in}}{\pgfqpoint{-0.002605in}{-0.009821in}}{\pgfqpoint{0.000000in}{-0.009821in}}%
\pgfpathclose%
\pgfusepath{stroke,fill}%
}%
\begin{pgfscope}%
\pgfsys@transformshift{0.826514in}{3.425649in}%
\pgfsys@useobject{currentmarker}{}%
\end{pgfscope}%
\begin{pgfscope}%
\pgfsys@transformshift{0.899244in}{3.425667in}%
\pgfsys@useobject{currentmarker}{}%
\end{pgfscope}%
\begin{pgfscope}%
\pgfsys@transformshift{0.971974in}{3.425770in}%
\pgfsys@useobject{currentmarker}{}%
\end{pgfscope}%
\begin{pgfscope}%
\pgfsys@transformshift{1.044704in}{3.426158in}%
\pgfsys@useobject{currentmarker}{}%
\end{pgfscope}%
\begin{pgfscope}%
\pgfsys@transformshift{1.117434in}{3.427227in}%
\pgfsys@useobject{currentmarker}{}%
\end{pgfscope}%
\begin{pgfscope}%
\pgfsys@transformshift{1.190163in}{3.429583in}%
\pgfsys@useobject{currentmarker}{}%
\end{pgfscope}%
\begin{pgfscope}%
\pgfsys@transformshift{1.262893in}{3.433996in}%
\pgfsys@useobject{currentmarker}{}%
\end{pgfscope}%
\begin{pgfscope}%
\pgfsys@transformshift{1.335623in}{3.441393in}%
\pgfsys@useobject{currentmarker}{}%
\end{pgfscope}%
\begin{pgfscope}%
\pgfsys@transformshift{1.408353in}{3.452916in}%
\pgfsys@useobject{currentmarker}{}%
\end{pgfscope}%
\begin{pgfscope}%
\pgfsys@transformshift{1.481083in}{3.469815in}%
\pgfsys@useobject{currentmarker}{}%
\end{pgfscope}%
\begin{pgfscope}%
\pgfsys@transformshift{1.553813in}{3.493128in}%
\pgfsys@useobject{currentmarker}{}%
\end{pgfscope}%
\begin{pgfscope}%
\pgfsys@transformshift{1.626543in}{3.524689in}%
\pgfsys@useobject{currentmarker}{}%
\end{pgfscope}%
\begin{pgfscope}%
\pgfsys@transformshift{1.699272in}{3.566466in}%
\pgfsys@useobject{currentmarker}{}%
\end{pgfscope}%
\begin{pgfscope}%
\pgfsys@transformshift{1.772002in}{3.620368in}%
\pgfsys@useobject{currentmarker}{}%
\end{pgfscope}%
\begin{pgfscope}%
\pgfsys@transformshift{1.844732in}{3.689280in}%
\pgfsys@useobject{currentmarker}{}%
\end{pgfscope}%
\begin{pgfscope}%
\pgfsys@transformshift{1.917462in}{3.777449in}%
\pgfsys@useobject{currentmarker}{}%
\end{pgfscope}%
\begin{pgfscope}%
\pgfsys@transformshift{1.990192in}{3.893068in}%
\pgfsys@useobject{currentmarker}{}%
\end{pgfscope}%
\begin{pgfscope}%
\pgfsys@transformshift{2.062922in}{4.046792in}%
\pgfsys@useobject{currentmarker}{}%
\end{pgfscope}%
\begin{pgfscope}%
\pgfsys@transformshift{2.135651in}{4.243767in}%
\pgfsys@useobject{currentmarker}{}%
\end{pgfscope}%
\begin{pgfscope}%
\pgfsys@transformshift{2.208381in}{4.456461in}%
\pgfsys@useobject{currentmarker}{}%
\end{pgfscope}%
\begin{pgfscope}%
\pgfsys@transformshift{2.281111in}{4.627428in}%
\pgfsys@useobject{currentmarker}{}%
\end{pgfscope}%
\begin{pgfscope}%
\pgfsys@transformshift{2.353841in}{4.754403in}%
\pgfsys@useobject{currentmarker}{}%
\end{pgfscope}%
\begin{pgfscope}%
\pgfsys@transformshift{2.426571in}{4.853101in}%
\pgfsys@useobject{currentmarker}{}%
\end{pgfscope}%
\begin{pgfscope}%
\pgfsys@transformshift{2.499301in}{4.933648in}%
\pgfsys@useobject{currentmarker}{}%
\end{pgfscope}%
\begin{pgfscope}%
\pgfsys@transformshift{2.572030in}{5.001762in}%
\pgfsys@useobject{currentmarker}{}%
\end{pgfscope}%
\begin{pgfscope}%
\pgfsys@transformshift{2.644760in}{5.061662in}%
\pgfsys@useobject{currentmarker}{}%
\end{pgfscope}%
\begin{pgfscope}%
\pgfsys@transformshift{2.717490in}{5.114187in}%
\pgfsys@useobject{currentmarker}{}%
\end{pgfscope}%
\begin{pgfscope}%
\pgfsys@transformshift{2.790220in}{5.161483in}%
\pgfsys@useobject{currentmarker}{}%
\end{pgfscope}%
\begin{pgfscope}%
\pgfsys@transformshift{2.862950in}{5.204466in}%
\pgfsys@useobject{currentmarker}{}%
\end{pgfscope}%
\begin{pgfscope}%
\pgfsys@transformshift{2.935680in}{5.243364in}%
\pgfsys@useobject{currentmarker}{}%
\end{pgfscope}%
\begin{pgfscope}%
\pgfsys@transformshift{3.008410in}{5.279191in}%
\pgfsys@useobject{currentmarker}{}%
\end{pgfscope}%
\begin{pgfscope}%
\pgfsys@transformshift{3.081139in}{5.311872in}%
\pgfsys@useobject{currentmarker}{}%
\end{pgfscope}%
\begin{pgfscope}%
\pgfsys@transformshift{3.153869in}{5.342408in}%
\pgfsys@useobject{currentmarker}{}%
\end{pgfscope}%
\begin{pgfscope}%
\pgfsys@transformshift{3.226599in}{5.370689in}%
\pgfsys@useobject{currentmarker}{}%
\end{pgfscope}%
\begin{pgfscope}%
\pgfsys@transformshift{3.299329in}{5.396952in}%
\pgfsys@useobject{currentmarker}{}%
\end{pgfscope}%
\begin{pgfscope}%
\pgfsys@transformshift{3.372059in}{5.421569in}%
\pgfsys@useobject{currentmarker}{}%
\end{pgfscope}%
\begin{pgfscope}%
\pgfsys@transformshift{3.444789in}{5.444510in}%
\pgfsys@useobject{currentmarker}{}%
\end{pgfscope}%
\begin{pgfscope}%
\pgfsys@transformshift{3.517518in}{5.465836in}%
\pgfsys@useobject{currentmarker}{}%
\end{pgfscope}%
\begin{pgfscope}%
\pgfsys@transformshift{3.590248in}{5.486134in}%
\pgfsys@useobject{currentmarker}{}%
\end{pgfscope}%
\begin{pgfscope}%
\pgfsys@transformshift{3.662978in}{5.505089in}%
\pgfsys@useobject{currentmarker}{}%
\end{pgfscope}%
\begin{pgfscope}%
\pgfsys@transformshift{3.735708in}{5.523240in}%
\pgfsys@useobject{currentmarker}{}%
\end{pgfscope}%
\end{pgfscope}%
\begin{pgfscope}%
\pgfsetbuttcap%
\pgfsetroundjoin%
\definecolor{currentfill}{rgb}{0.000000,0.000000,0.000000}%
\pgfsetfillcolor{currentfill}%
\pgfsetlinewidth{0.803000pt}%
\definecolor{currentstroke}{rgb}{0.000000,0.000000,0.000000}%
\pgfsetstrokecolor{currentstroke}%
\pgfsetdash{}{0pt}%
\pgfsys@defobject{currentmarker}{\pgfqpoint{0.000000in}{-0.048611in}}{\pgfqpoint{0.000000in}{0.000000in}}{%
\pgfpathmoveto{\pgfqpoint{0.000000in}{0.000000in}}%
\pgfpathlineto{\pgfqpoint{0.000000in}{-0.048611in}}%
\pgfusepath{stroke,fill}%
}%
\begin{pgfscope}%
\pgfsys@transformshift{1.190163in}{3.312222in}%
\pgfsys@useobject{currentmarker}{}%
\end{pgfscope}%
\end{pgfscope}%
\begin{pgfscope}%
\definecolor{textcolor}{rgb}{0.000000,0.000000,0.000000}%
\pgfsetstrokecolor{textcolor}%
\pgfsetfillcolor{textcolor}%
\pgftext[x=1.190163in,y=3.215000in,,top]{\color{textcolor}\sffamily\fontsize{10.000000}{12.000000}\selectfont 1}%
\end{pgfscope}%
\begin{pgfscope}%
\pgfsetbuttcap%
\pgfsetroundjoin%
\definecolor{currentfill}{rgb}{0.000000,0.000000,0.000000}%
\pgfsetfillcolor{currentfill}%
\pgfsetlinewidth{0.803000pt}%
\definecolor{currentstroke}{rgb}{0.000000,0.000000,0.000000}%
\pgfsetstrokecolor{currentstroke}%
\pgfsetdash{}{0pt}%
\pgfsys@defobject{currentmarker}{\pgfqpoint{0.000000in}{-0.048611in}}{\pgfqpoint{0.000000in}{0.000000in}}{%
\pgfpathmoveto{\pgfqpoint{0.000000in}{0.000000in}}%
\pgfpathlineto{\pgfqpoint{0.000000in}{-0.048611in}}%
\pgfusepath{stroke,fill}%
}%
\begin{pgfscope}%
\pgfsys@transformshift{1.917462in}{3.312222in}%
\pgfsys@useobject{currentmarker}{}%
\end{pgfscope}%
\end{pgfscope}%
\begin{pgfscope}%
\definecolor{textcolor}{rgb}{0.000000,0.000000,0.000000}%
\pgfsetstrokecolor{textcolor}%
\pgfsetfillcolor{textcolor}%
\pgftext[x=1.917462in,y=3.215000in,,top]{\color{textcolor}\sffamily\fontsize{10.000000}{12.000000}\selectfont 2}%
\end{pgfscope}%
\begin{pgfscope}%
\pgfsetbuttcap%
\pgfsetroundjoin%
\definecolor{currentfill}{rgb}{0.000000,0.000000,0.000000}%
\pgfsetfillcolor{currentfill}%
\pgfsetlinewidth{0.803000pt}%
\definecolor{currentstroke}{rgb}{0.000000,0.000000,0.000000}%
\pgfsetstrokecolor{currentstroke}%
\pgfsetdash{}{0pt}%
\pgfsys@defobject{currentmarker}{\pgfqpoint{0.000000in}{-0.048611in}}{\pgfqpoint{0.000000in}{0.000000in}}{%
\pgfpathmoveto{\pgfqpoint{0.000000in}{0.000000in}}%
\pgfpathlineto{\pgfqpoint{0.000000in}{-0.048611in}}%
\pgfusepath{stroke,fill}%
}%
\begin{pgfscope}%
\pgfsys@transformshift{2.644760in}{3.312222in}%
\pgfsys@useobject{currentmarker}{}%
\end{pgfscope}%
\end{pgfscope}%
\begin{pgfscope}%
\definecolor{textcolor}{rgb}{0.000000,0.000000,0.000000}%
\pgfsetstrokecolor{textcolor}%
\pgfsetfillcolor{textcolor}%
\pgftext[x=2.644760in,y=3.215000in,,top]{\color{textcolor}\sffamily\fontsize{10.000000}{12.000000}\selectfont 3}%
\end{pgfscope}%
\begin{pgfscope}%
\pgfsetbuttcap%
\pgfsetroundjoin%
\definecolor{currentfill}{rgb}{0.000000,0.000000,0.000000}%
\pgfsetfillcolor{currentfill}%
\pgfsetlinewidth{0.803000pt}%
\definecolor{currentstroke}{rgb}{0.000000,0.000000,0.000000}%
\pgfsetstrokecolor{currentstroke}%
\pgfsetdash{}{0pt}%
\pgfsys@defobject{currentmarker}{\pgfqpoint{0.000000in}{-0.048611in}}{\pgfqpoint{0.000000in}{0.000000in}}{%
\pgfpathmoveto{\pgfqpoint{0.000000in}{0.000000in}}%
\pgfpathlineto{\pgfqpoint{0.000000in}{-0.048611in}}%
\pgfusepath{stroke,fill}%
}%
\begin{pgfscope}%
\pgfsys@transformshift{3.372059in}{3.312222in}%
\pgfsys@useobject{currentmarker}{}%
\end{pgfscope}%
\end{pgfscope}%
\begin{pgfscope}%
\definecolor{textcolor}{rgb}{0.000000,0.000000,0.000000}%
\pgfsetstrokecolor{textcolor}%
\pgfsetfillcolor{textcolor}%
\pgftext[x=3.372059in,y=3.215000in,,top]{\color{textcolor}\sffamily\fontsize{10.000000}{12.000000}\selectfont 4}%
\end{pgfscope}%
\begin{pgfscope}%
\pgfsetbuttcap%
\pgfsetroundjoin%
\definecolor{currentfill}{rgb}{0.000000,0.000000,0.000000}%
\pgfsetfillcolor{currentfill}%
\pgfsetlinewidth{0.803000pt}%
\definecolor{currentstroke}{rgb}{0.000000,0.000000,0.000000}%
\pgfsetstrokecolor{currentstroke}%
\pgfsetdash{}{0pt}%
\pgfsys@defobject{currentmarker}{\pgfqpoint{-0.048611in}{0.000000in}}{\pgfqpoint{0.000000in}{0.000000in}}{%
\pgfpathmoveto{\pgfqpoint{0.000000in}{0.000000in}}%
\pgfpathlineto{\pgfqpoint{-0.048611in}{0.000000in}}%
\pgfusepath{stroke,fill}%
}%
\begin{pgfscope}%
\pgfsys@transformshift{0.672222in}{3.425648in}%
\pgfsys@useobject{currentmarker}{}%
\end{pgfscope}%
\end{pgfscope}%
\begin{pgfscope}%
\definecolor{textcolor}{rgb}{0.000000,0.000000,0.000000}%
\pgfsetstrokecolor{textcolor}%
\pgfsetfillcolor{textcolor}%
\pgftext[x=0.149382in,y=3.372886in,left,base]{\color{textcolor}\sffamily\fontsize{10.000000}{12.000000}\selectfont −2.00}%
\end{pgfscope}%
\begin{pgfscope}%
\pgfsetbuttcap%
\pgfsetroundjoin%
\definecolor{currentfill}{rgb}{0.000000,0.000000,0.000000}%
\pgfsetfillcolor{currentfill}%
\pgfsetlinewidth{0.803000pt}%
\definecolor{currentstroke}{rgb}{0.000000,0.000000,0.000000}%
\pgfsetstrokecolor{currentstroke}%
\pgfsetdash{}{0pt}%
\pgfsys@defobject{currentmarker}{\pgfqpoint{-0.048611in}{0.000000in}}{\pgfqpoint{0.000000in}{0.000000in}}{%
\pgfpathmoveto{\pgfqpoint{0.000000in}{0.000000in}}%
\pgfpathlineto{\pgfqpoint{-0.048611in}{0.000000in}}%
\pgfusepath{stroke,fill}%
}%
\begin{pgfscope}%
\pgfsys@transformshift{0.672222in}{3.771493in}%
\pgfsys@useobject{currentmarker}{}%
\end{pgfscope}%
\end{pgfscope}%
\begin{pgfscope}%
\definecolor{textcolor}{rgb}{0.000000,0.000000,0.000000}%
\pgfsetstrokecolor{textcolor}%
\pgfsetfillcolor{textcolor}%
\pgftext[x=0.149382in,y=3.718732in,left,base]{\color{textcolor}\sffamily\fontsize{10.000000}{12.000000}\selectfont −1.75}%
\end{pgfscope}%
\begin{pgfscope}%
\pgfsetbuttcap%
\pgfsetroundjoin%
\definecolor{currentfill}{rgb}{0.000000,0.000000,0.000000}%
\pgfsetfillcolor{currentfill}%
\pgfsetlinewidth{0.803000pt}%
\definecolor{currentstroke}{rgb}{0.000000,0.000000,0.000000}%
\pgfsetstrokecolor{currentstroke}%
\pgfsetdash{}{0pt}%
\pgfsys@defobject{currentmarker}{\pgfqpoint{-0.048611in}{0.000000in}}{\pgfqpoint{0.000000in}{0.000000in}}{%
\pgfpathmoveto{\pgfqpoint{0.000000in}{0.000000in}}%
\pgfpathlineto{\pgfqpoint{-0.048611in}{0.000000in}}%
\pgfusepath{stroke,fill}%
}%
\begin{pgfscope}%
\pgfsys@transformshift{0.672222in}{4.117339in}%
\pgfsys@useobject{currentmarker}{}%
\end{pgfscope}%
\end{pgfscope}%
\begin{pgfscope}%
\definecolor{textcolor}{rgb}{0.000000,0.000000,0.000000}%
\pgfsetstrokecolor{textcolor}%
\pgfsetfillcolor{textcolor}%
\pgftext[x=0.149382in,y=4.064577in,left,base]{\color{textcolor}\sffamily\fontsize{10.000000}{12.000000}\selectfont −1.50}%
\end{pgfscope}%
\begin{pgfscope}%
\pgfsetbuttcap%
\pgfsetroundjoin%
\definecolor{currentfill}{rgb}{0.000000,0.000000,0.000000}%
\pgfsetfillcolor{currentfill}%
\pgfsetlinewidth{0.803000pt}%
\definecolor{currentstroke}{rgb}{0.000000,0.000000,0.000000}%
\pgfsetstrokecolor{currentstroke}%
\pgfsetdash{}{0pt}%
\pgfsys@defobject{currentmarker}{\pgfqpoint{-0.048611in}{0.000000in}}{\pgfqpoint{0.000000in}{0.000000in}}{%
\pgfpathmoveto{\pgfqpoint{0.000000in}{0.000000in}}%
\pgfpathlineto{\pgfqpoint{-0.048611in}{0.000000in}}%
\pgfusepath{stroke,fill}%
}%
\begin{pgfscope}%
\pgfsys@transformshift{0.672222in}{4.463184in}%
\pgfsys@useobject{currentmarker}{}%
\end{pgfscope}%
\end{pgfscope}%
\begin{pgfscope}%
\definecolor{textcolor}{rgb}{0.000000,0.000000,0.000000}%
\pgfsetstrokecolor{textcolor}%
\pgfsetfillcolor{textcolor}%
\pgftext[x=0.149382in,y=4.410422in,left,base]{\color{textcolor}\sffamily\fontsize{10.000000}{12.000000}\selectfont −1.25}%
\end{pgfscope}%
\begin{pgfscope}%
\pgfsetbuttcap%
\pgfsetroundjoin%
\definecolor{currentfill}{rgb}{0.000000,0.000000,0.000000}%
\pgfsetfillcolor{currentfill}%
\pgfsetlinewidth{0.803000pt}%
\definecolor{currentstroke}{rgb}{0.000000,0.000000,0.000000}%
\pgfsetstrokecolor{currentstroke}%
\pgfsetdash{}{0pt}%
\pgfsys@defobject{currentmarker}{\pgfqpoint{-0.048611in}{0.000000in}}{\pgfqpoint{0.000000in}{0.000000in}}{%
\pgfpathmoveto{\pgfqpoint{0.000000in}{0.000000in}}%
\pgfpathlineto{\pgfqpoint{-0.048611in}{0.000000in}}%
\pgfusepath{stroke,fill}%
}%
\begin{pgfscope}%
\pgfsys@transformshift{0.672222in}{4.809029in}%
\pgfsys@useobject{currentmarker}{}%
\end{pgfscope}%
\end{pgfscope}%
\begin{pgfscope}%
\definecolor{textcolor}{rgb}{0.000000,0.000000,0.000000}%
\pgfsetstrokecolor{textcolor}%
\pgfsetfillcolor{textcolor}%
\pgftext[x=0.149382in,y=4.756268in,left,base]{\color{textcolor}\sffamily\fontsize{10.000000}{12.000000}\selectfont −1.00}%
\end{pgfscope}%
\begin{pgfscope}%
\pgfsetbuttcap%
\pgfsetroundjoin%
\definecolor{currentfill}{rgb}{0.000000,0.000000,0.000000}%
\pgfsetfillcolor{currentfill}%
\pgfsetlinewidth{0.803000pt}%
\definecolor{currentstroke}{rgb}{0.000000,0.000000,0.000000}%
\pgfsetstrokecolor{currentstroke}%
\pgfsetdash{}{0pt}%
\pgfsys@defobject{currentmarker}{\pgfqpoint{-0.048611in}{0.000000in}}{\pgfqpoint{0.000000in}{0.000000in}}{%
\pgfpathmoveto{\pgfqpoint{0.000000in}{0.000000in}}%
\pgfpathlineto{\pgfqpoint{-0.048611in}{0.000000in}}%
\pgfusepath{stroke,fill}%
}%
\begin{pgfscope}%
\pgfsys@transformshift{0.672222in}{5.154875in}%
\pgfsys@useobject{currentmarker}{}%
\end{pgfscope}%
\end{pgfscope}%
\begin{pgfscope}%
\definecolor{textcolor}{rgb}{0.000000,0.000000,0.000000}%
\pgfsetstrokecolor{textcolor}%
\pgfsetfillcolor{textcolor}%
\pgftext[x=0.149382in,y=5.102113in,left,base]{\color{textcolor}\sffamily\fontsize{10.000000}{12.000000}\selectfont −0.75}%
\end{pgfscope}%
\begin{pgfscope}%
\pgfsetbuttcap%
\pgfsetroundjoin%
\definecolor{currentfill}{rgb}{0.000000,0.000000,0.000000}%
\pgfsetfillcolor{currentfill}%
\pgfsetlinewidth{0.803000pt}%
\definecolor{currentstroke}{rgb}{0.000000,0.000000,0.000000}%
\pgfsetstrokecolor{currentstroke}%
\pgfsetdash{}{0pt}%
\pgfsys@defobject{currentmarker}{\pgfqpoint{-0.048611in}{0.000000in}}{\pgfqpoint{0.000000in}{0.000000in}}{%
\pgfpathmoveto{\pgfqpoint{0.000000in}{0.000000in}}%
\pgfpathlineto{\pgfqpoint{-0.048611in}{0.000000in}}%
\pgfusepath{stroke,fill}%
}%
\begin{pgfscope}%
\pgfsys@transformshift{0.672222in}{5.500720in}%
\pgfsys@useobject{currentmarker}{}%
\end{pgfscope}%
\end{pgfscope}%
\begin{pgfscope}%
\definecolor{textcolor}{rgb}{0.000000,0.000000,0.000000}%
\pgfsetstrokecolor{textcolor}%
\pgfsetfillcolor{textcolor}%
\pgftext[x=0.149382in,y=5.447958in,left,base]{\color{textcolor}\sffamily\fontsize{10.000000}{12.000000}\selectfont −0.50}%
\end{pgfscope}%
\begin{pgfscope}%
\pgfpathrectangle{\pgfqpoint{0.672222in}{3.312222in}}{\pgfqpoint{3.217778in}{2.324444in}}%
\pgfusepath{clip}%
\pgfsetrectcap%
\pgfsetroundjoin%
\pgfsetlinewidth{1.505625pt}%
\definecolor{currentstroke}{rgb}{0.121569,0.466667,0.705882}%
\pgfsetstrokecolor{currentstroke}%
\pgfsetdash{}{0pt}%
\pgfpathmoveto{\pgfqpoint{0.826514in}{3.425649in}}%
\pgfpathlineto{\pgfqpoint{0.899244in}{3.425667in}}%
\pgfpathlineto{\pgfqpoint{0.971974in}{3.425770in}}%
\pgfpathlineto{\pgfqpoint{1.044704in}{3.426158in}}%
\pgfpathlineto{\pgfqpoint{1.117434in}{3.427227in}}%
\pgfpathlineto{\pgfqpoint{1.190163in}{3.429583in}}%
\pgfpathlineto{\pgfqpoint{1.262893in}{3.433996in}}%
\pgfpathlineto{\pgfqpoint{1.335623in}{3.441393in}}%
\pgfpathlineto{\pgfqpoint{1.408353in}{3.452916in}}%
\pgfpathlineto{\pgfqpoint{1.481083in}{3.469815in}}%
\pgfpathlineto{\pgfqpoint{1.553813in}{3.493128in}}%
\pgfpathlineto{\pgfqpoint{1.626543in}{3.524689in}}%
\pgfpathlineto{\pgfqpoint{1.699272in}{3.566466in}}%
\pgfpathlineto{\pgfqpoint{1.772002in}{3.620368in}}%
\pgfpathlineto{\pgfqpoint{1.844732in}{3.689280in}}%
\pgfpathlineto{\pgfqpoint{1.917462in}{3.777449in}}%
\pgfpathlineto{\pgfqpoint{1.990192in}{3.893068in}}%
\pgfpathlineto{\pgfqpoint{2.062922in}{4.046792in}}%
\pgfpathlineto{\pgfqpoint{2.135651in}{4.243767in}}%
\pgfpathlineto{\pgfqpoint{2.208381in}{4.456461in}}%
\pgfpathlineto{\pgfqpoint{2.281111in}{4.627428in}}%
\pgfpathlineto{\pgfqpoint{2.353841in}{4.754403in}}%
\pgfpathlineto{\pgfqpoint{2.426571in}{4.853101in}}%
\pgfpathlineto{\pgfqpoint{2.499301in}{4.933648in}}%
\pgfpathlineto{\pgfqpoint{2.572030in}{5.001762in}}%
\pgfpathlineto{\pgfqpoint{2.644760in}{5.061662in}}%
\pgfpathlineto{\pgfqpoint{2.717490in}{5.114187in}}%
\pgfpathlineto{\pgfqpoint{2.790220in}{5.161483in}}%
\pgfpathlineto{\pgfqpoint{2.862950in}{5.204466in}}%
\pgfpathlineto{\pgfqpoint{2.935680in}{5.243364in}}%
\pgfpathlineto{\pgfqpoint{3.008410in}{5.279191in}}%
\pgfpathlineto{\pgfqpoint{3.081139in}{5.311872in}}%
\pgfpathlineto{\pgfqpoint{3.153869in}{5.342408in}}%
\pgfpathlineto{\pgfqpoint{3.226599in}{5.370689in}}%
\pgfpathlineto{\pgfqpoint{3.299329in}{5.396952in}}%
\pgfpathlineto{\pgfqpoint{3.372059in}{5.421569in}}%
\pgfpathlineto{\pgfqpoint{3.444789in}{5.444510in}}%
\pgfpathlineto{\pgfqpoint{3.517518in}{5.465836in}}%
\pgfpathlineto{\pgfqpoint{3.590248in}{5.486134in}}%
\pgfpathlineto{\pgfqpoint{3.662978in}{5.505089in}}%
\pgfpathlineto{\pgfqpoint{3.735708in}{5.523240in}}%
\pgfusepath{stroke}%
\end{pgfscope}%
\begin{pgfscope}%
\pgfsetrectcap%
\pgfsetmiterjoin%
\pgfsetlinewidth{0.803000pt}%
\definecolor{currentstroke}{rgb}{0.000000,0.000000,0.000000}%
\pgfsetstrokecolor{currentstroke}%
\pgfsetdash{}{0pt}%
\pgfpathmoveto{\pgfqpoint{0.672222in}{3.312222in}}%
\pgfpathlineto{\pgfqpoint{0.672222in}{5.636667in}}%
\pgfusepath{stroke}%
\end{pgfscope}%
\begin{pgfscope}%
\pgfsetrectcap%
\pgfsetmiterjoin%
\pgfsetlinewidth{0.803000pt}%
\definecolor{currentstroke}{rgb}{0.000000,0.000000,0.000000}%
\pgfsetstrokecolor{currentstroke}%
\pgfsetdash{}{0pt}%
\pgfpathmoveto{\pgfqpoint{3.890000in}{3.312222in}}%
\pgfpathlineto{\pgfqpoint{3.890000in}{5.636667in}}%
\pgfusepath{stroke}%
\end{pgfscope}%
\begin{pgfscope}%
\pgfsetrectcap%
\pgfsetmiterjoin%
\pgfsetlinewidth{0.803000pt}%
\definecolor{currentstroke}{rgb}{0.000000,0.000000,0.000000}%
\pgfsetstrokecolor{currentstroke}%
\pgfsetdash{}{0pt}%
\pgfpathmoveto{\pgfqpoint{0.672222in}{3.312222in}}%
\pgfpathlineto{\pgfqpoint{3.890000in}{3.312222in}}%
\pgfusepath{stroke}%
\end{pgfscope}%
\begin{pgfscope}%
\pgfsetrectcap%
\pgfsetmiterjoin%
\pgfsetlinewidth{0.803000pt}%
\definecolor{currentstroke}{rgb}{0.000000,0.000000,0.000000}%
\pgfsetstrokecolor{currentstroke}%
\pgfsetdash{}{0pt}%
\pgfpathmoveto{\pgfqpoint{0.672222in}{5.636667in}}%
\pgfpathlineto{\pgfqpoint{3.890000in}{5.636667in}}%
\pgfusepath{stroke}%
\end{pgfscope}%
\begin{pgfscope}%
\definecolor{textcolor}{rgb}{0.000000,0.000000,0.000000}%
\pgfsetstrokecolor{textcolor}%
\pgfsetfillcolor{textcolor}%
\pgftext[x=2.281111in,y=5.720000in,,base]{\color{textcolor}\sffamily\fontsize{12.000000}{14.400000}\selectfont \(\displaystyle  N = 16 \)}%
\end{pgfscope}%
\begin{pgfscope}%
\pgfsetbuttcap%
\pgfsetmiterjoin%
\definecolor{currentfill}{rgb}{1.000000,1.000000,1.000000}%
\pgfsetfillcolor{currentfill}%
\pgfsetlinewidth{0.000000pt}%
\definecolor{currentstroke}{rgb}{0.000000,0.000000,0.000000}%
\pgfsetstrokecolor{currentstroke}%
\pgfsetstrokeopacity{0.000000}%
\pgfsetdash{}{0pt}%
\pgfpathmoveto{\pgfqpoint{4.597222in}{3.312222in}}%
\pgfpathlineto{\pgfqpoint{7.815000in}{3.312222in}}%
\pgfpathlineto{\pgfqpoint{7.815000in}{5.636667in}}%
\pgfpathlineto{\pgfqpoint{4.597222in}{5.636667in}}%
\pgfpathclose%
\pgfusepath{fill}%
\end{pgfscope}%
\begin{pgfscope}%
\pgfpathrectangle{\pgfqpoint{4.597222in}{3.312222in}}{\pgfqpoint{3.217778in}{2.324444in}}%
\pgfusepath{clip}%
\pgfsetbuttcap%
\pgfsetroundjoin%
\definecolor{currentfill}{rgb}{0.121569,0.466667,0.705882}%
\pgfsetfillcolor{currentfill}%
\pgfsetfillopacity{0.300000}%
\pgfsetlinewidth{1.003750pt}%
\definecolor{currentstroke}{rgb}{0.121569,0.466667,0.705882}%
\pgfsetstrokecolor{currentstroke}%
\pgfsetstrokeopacity{0.300000}%
\pgfsetdash{}{0pt}%
\pgfpathmoveto{\pgfqpoint{4.751514in}{3.425649in}}%
\pgfpathlineto{\pgfqpoint{4.751514in}{3.425648in}}%
\pgfpathlineto{\pgfqpoint{4.824244in}{3.425665in}}%
\pgfpathlineto{\pgfqpoint{4.896974in}{3.425769in}}%
\pgfpathlineto{\pgfqpoint{4.969704in}{3.426151in}}%
\pgfpathlineto{\pgfqpoint{5.042434in}{3.427217in}}%
\pgfpathlineto{\pgfqpoint{5.115163in}{3.429558in}}%
\pgfpathlineto{\pgfqpoint{5.187893in}{3.433948in}}%
\pgfpathlineto{\pgfqpoint{5.260623in}{3.441346in}}%
\pgfpathlineto{\pgfqpoint{5.333353in}{3.452897in}}%
\pgfpathlineto{\pgfqpoint{5.406083in}{3.469727in}}%
\pgfpathlineto{\pgfqpoint{5.478813in}{3.493085in}}%
\pgfpathlineto{\pgfqpoint{5.551543in}{3.524752in}}%
\pgfpathlineto{\pgfqpoint{5.624272in}{3.566496in}}%
\pgfpathlineto{\pgfqpoint{5.697002in}{3.619966in}}%
\pgfpathlineto{\pgfqpoint{5.769732in}{3.688511in}}%
\pgfpathlineto{\pgfqpoint{5.842462in}{3.776418in}}%
\pgfpathlineto{\pgfqpoint{5.915192in}{3.892586in}}%
\pgfpathlineto{\pgfqpoint{5.987922in}{4.050893in}}%
\pgfpathlineto{\pgfqpoint{6.060651in}{4.290070in}}%
\pgfpathlineto{\pgfqpoint{6.133381in}{4.512903in}}%
\pgfpathlineto{\pgfqpoint{6.206111in}{4.659429in}}%
\pgfpathlineto{\pgfqpoint{6.278841in}{4.768301in}}%
\pgfpathlineto{\pgfqpoint{6.351571in}{4.858303in}}%
\pgfpathlineto{\pgfqpoint{6.424301in}{4.935321in}}%
\pgfpathlineto{\pgfqpoint{6.497030in}{5.001731in}}%
\pgfpathlineto{\pgfqpoint{6.569760in}{5.061553in}}%
\pgfpathlineto{\pgfqpoint{6.642490in}{5.113906in}}%
\pgfpathlineto{\pgfqpoint{6.715220in}{5.161236in}}%
\pgfpathlineto{\pgfqpoint{6.787950in}{5.203985in}}%
\pgfpathlineto{\pgfqpoint{6.860680in}{5.243139in}}%
\pgfpathlineto{\pgfqpoint{6.933410in}{5.278768in}}%
\pgfpathlineto{\pgfqpoint{7.006139in}{5.311299in}}%
\pgfpathlineto{\pgfqpoint{7.078869in}{5.342199in}}%
\pgfpathlineto{\pgfqpoint{7.151599in}{5.369773in}}%
\pgfpathlineto{\pgfqpoint{7.224329in}{5.396641in}}%
\pgfpathlineto{\pgfqpoint{7.297059in}{5.421429in}}%
\pgfpathlineto{\pgfqpoint{7.369789in}{5.443953in}}%
\pgfpathlineto{\pgfqpoint{7.442518in}{5.465446in}}%
\pgfpathlineto{\pgfqpoint{7.515248in}{5.485806in}}%
\pgfpathlineto{\pgfqpoint{7.587978in}{5.504805in}}%
\pgfpathlineto{\pgfqpoint{7.660708in}{5.523074in}}%
\pgfpathlineto{\pgfqpoint{7.660708in}{5.523407in}}%
\pgfpathlineto{\pgfqpoint{7.660708in}{5.523407in}}%
\pgfpathlineto{\pgfqpoint{7.587978in}{5.505406in}}%
\pgfpathlineto{\pgfqpoint{7.515248in}{5.486163in}}%
\pgfpathlineto{\pgfqpoint{7.442518in}{5.466256in}}%
\pgfpathlineto{\pgfqpoint{7.369789in}{5.444627in}}%
\pgfpathlineto{\pgfqpoint{7.297059in}{5.421496in}}%
\pgfpathlineto{\pgfqpoint{7.224329in}{5.397124in}}%
\pgfpathlineto{\pgfqpoint{7.151599in}{5.371075in}}%
\pgfpathlineto{\pgfqpoint{7.078869in}{5.342519in}}%
\pgfpathlineto{\pgfqpoint{7.006139in}{5.312257in}}%
\pgfpathlineto{\pgfqpoint{6.933410in}{5.279367in}}%
\pgfpathlineto{\pgfqpoint{6.860680in}{5.243559in}}%
\pgfpathlineto{\pgfqpoint{6.787950in}{5.204463in}}%
\pgfpathlineto{\pgfqpoint{6.715220in}{5.161645in}}%
\pgfpathlineto{\pgfqpoint{6.642490in}{5.114036in}}%
\pgfpathlineto{\pgfqpoint{6.569760in}{5.062196in}}%
\pgfpathlineto{\pgfqpoint{6.497030in}{5.002908in}}%
\pgfpathlineto{\pgfqpoint{6.424301in}{4.936821in}}%
\pgfpathlineto{\pgfqpoint{6.351571in}{4.859450in}}%
\pgfpathlineto{\pgfqpoint{6.278841in}{4.770504in}}%
\pgfpathlineto{\pgfqpoint{6.206111in}{4.660035in}}%
\pgfpathlineto{\pgfqpoint{6.133381in}{4.516381in}}%
\pgfpathlineto{\pgfqpoint{6.060651in}{4.294203in}}%
\pgfpathlineto{\pgfqpoint{5.987922in}{4.054839in}}%
\pgfpathlineto{\pgfqpoint{5.915192in}{3.894390in}}%
\pgfpathlineto{\pgfqpoint{5.842462in}{3.778375in}}%
\pgfpathlineto{\pgfqpoint{5.769732in}{3.689381in}}%
\pgfpathlineto{\pgfqpoint{5.697002in}{3.620885in}}%
\pgfpathlineto{\pgfqpoint{5.624272in}{3.566803in}}%
\pgfpathlineto{\pgfqpoint{5.551543in}{3.525033in}}%
\pgfpathlineto{\pgfqpoint{5.478813in}{3.493401in}}%
\pgfpathlineto{\pgfqpoint{5.406083in}{3.469933in}}%
\pgfpathlineto{\pgfqpoint{5.333353in}{3.453007in}}%
\pgfpathlineto{\pgfqpoint{5.260623in}{3.441453in}}%
\pgfpathlineto{\pgfqpoint{5.187893in}{3.434029in}}%
\pgfpathlineto{\pgfqpoint{5.115163in}{3.429614in}}%
\pgfpathlineto{\pgfqpoint{5.042434in}{3.427258in}}%
\pgfpathlineto{\pgfqpoint{4.969704in}{3.426167in}}%
\pgfpathlineto{\pgfqpoint{4.896974in}{3.425774in}}%
\pgfpathlineto{\pgfqpoint{4.824244in}{3.425666in}}%
\pgfpathlineto{\pgfqpoint{4.751514in}{3.425649in}}%
\pgfpathclose%
\pgfusepath{stroke,fill}%
\end{pgfscope}%
\begin{pgfscope}%
\pgfpathrectangle{\pgfqpoint{4.597222in}{3.312222in}}{\pgfqpoint{3.217778in}{2.324444in}}%
\pgfusepath{clip}%
\pgfsetbuttcap%
\pgfsetroundjoin%
\definecolor{currentfill}{rgb}{0.121569,0.466667,0.705882}%
\pgfsetfillcolor{currentfill}%
\pgfsetlinewidth{1.003750pt}%
\definecolor{currentstroke}{rgb}{0.121569,0.466667,0.705882}%
\pgfsetstrokecolor{currentstroke}%
\pgfsetdash{}{0pt}%
\pgfsys@defobject{currentmarker}{\pgfqpoint{-0.009821in}{-0.009821in}}{\pgfqpoint{0.009821in}{0.009821in}}{%
\pgfpathmoveto{\pgfqpoint{0.000000in}{-0.009821in}}%
\pgfpathcurveto{\pgfqpoint{0.002605in}{-0.009821in}}{\pgfqpoint{0.005103in}{-0.008786in}}{\pgfqpoint{0.006944in}{-0.006944in}}%
\pgfpathcurveto{\pgfqpoint{0.008786in}{-0.005103in}}{\pgfqpoint{0.009821in}{-0.002605in}}{\pgfqpoint{0.009821in}{0.000000in}}%
\pgfpathcurveto{\pgfqpoint{0.009821in}{0.002605in}}{\pgfqpoint{0.008786in}{0.005103in}}{\pgfqpoint{0.006944in}{0.006944in}}%
\pgfpathcurveto{\pgfqpoint{0.005103in}{0.008786in}}{\pgfqpoint{0.002605in}{0.009821in}}{\pgfqpoint{0.000000in}{0.009821in}}%
\pgfpathcurveto{\pgfqpoint{-0.002605in}{0.009821in}}{\pgfqpoint{-0.005103in}{0.008786in}}{\pgfqpoint{-0.006944in}{0.006944in}}%
\pgfpathcurveto{\pgfqpoint{-0.008786in}{0.005103in}}{\pgfqpoint{-0.009821in}{0.002605in}}{\pgfqpoint{-0.009821in}{0.000000in}}%
\pgfpathcurveto{\pgfqpoint{-0.009821in}{-0.002605in}}{\pgfqpoint{-0.008786in}{-0.005103in}}{\pgfqpoint{-0.006944in}{-0.006944in}}%
\pgfpathcurveto{\pgfqpoint{-0.005103in}{-0.008786in}}{\pgfqpoint{-0.002605in}{-0.009821in}}{\pgfqpoint{0.000000in}{-0.009821in}}%
\pgfpathclose%
\pgfusepath{stroke,fill}%
}%
\begin{pgfscope}%
\pgfsys@transformshift{4.751514in}{3.425649in}%
\pgfsys@useobject{currentmarker}{}%
\end{pgfscope}%
\begin{pgfscope}%
\pgfsys@transformshift{4.824244in}{3.425666in}%
\pgfsys@useobject{currentmarker}{}%
\end{pgfscope}%
\begin{pgfscope}%
\pgfsys@transformshift{4.896974in}{3.425771in}%
\pgfsys@useobject{currentmarker}{}%
\end{pgfscope}%
\begin{pgfscope}%
\pgfsys@transformshift{4.969704in}{3.426159in}%
\pgfsys@useobject{currentmarker}{}%
\end{pgfscope}%
\begin{pgfscope}%
\pgfsys@transformshift{5.042434in}{3.427237in}%
\pgfsys@useobject{currentmarker}{}%
\end{pgfscope}%
\begin{pgfscope}%
\pgfsys@transformshift{5.115163in}{3.429586in}%
\pgfsys@useobject{currentmarker}{}%
\end{pgfscope}%
\begin{pgfscope}%
\pgfsys@transformshift{5.187893in}{3.433988in}%
\pgfsys@useobject{currentmarker}{}%
\end{pgfscope}%
\begin{pgfscope}%
\pgfsys@transformshift{5.260623in}{3.441399in}%
\pgfsys@useobject{currentmarker}{}%
\end{pgfscope}%
\begin{pgfscope}%
\pgfsys@transformshift{5.333353in}{3.452952in}%
\pgfsys@useobject{currentmarker}{}%
\end{pgfscope}%
\begin{pgfscope}%
\pgfsys@transformshift{5.406083in}{3.469830in}%
\pgfsys@useobject{currentmarker}{}%
\end{pgfscope}%
\begin{pgfscope}%
\pgfsys@transformshift{5.478813in}{3.493243in}%
\pgfsys@useobject{currentmarker}{}%
\end{pgfscope}%
\begin{pgfscope}%
\pgfsys@transformshift{5.551543in}{3.524892in}%
\pgfsys@useobject{currentmarker}{}%
\end{pgfscope}%
\begin{pgfscope}%
\pgfsys@transformshift{5.624272in}{3.566649in}%
\pgfsys@useobject{currentmarker}{}%
\end{pgfscope}%
\begin{pgfscope}%
\pgfsys@transformshift{5.697002in}{3.620426in}%
\pgfsys@useobject{currentmarker}{}%
\end{pgfscope}%
\begin{pgfscope}%
\pgfsys@transformshift{5.769732in}{3.688946in}%
\pgfsys@useobject{currentmarker}{}%
\end{pgfscope}%
\begin{pgfscope}%
\pgfsys@transformshift{5.842462in}{3.777397in}%
\pgfsys@useobject{currentmarker}{}%
\end{pgfscope}%
\begin{pgfscope}%
\pgfsys@transformshift{5.915192in}{3.893488in}%
\pgfsys@useobject{currentmarker}{}%
\end{pgfscope}%
\begin{pgfscope}%
\pgfsys@transformshift{5.987922in}{4.052866in}%
\pgfsys@useobject{currentmarker}{}%
\end{pgfscope}%
\begin{pgfscope}%
\pgfsys@transformshift{6.060651in}{4.292136in}%
\pgfsys@useobject{currentmarker}{}%
\end{pgfscope}%
\begin{pgfscope}%
\pgfsys@transformshift{6.133381in}{4.514642in}%
\pgfsys@useobject{currentmarker}{}%
\end{pgfscope}%
\begin{pgfscope}%
\pgfsys@transformshift{6.206111in}{4.659732in}%
\pgfsys@useobject{currentmarker}{}%
\end{pgfscope}%
\begin{pgfscope}%
\pgfsys@transformshift{6.278841in}{4.769403in}%
\pgfsys@useobject{currentmarker}{}%
\end{pgfscope}%
\begin{pgfscope}%
\pgfsys@transformshift{6.351571in}{4.858877in}%
\pgfsys@useobject{currentmarker}{}%
\end{pgfscope}%
\begin{pgfscope}%
\pgfsys@transformshift{6.424301in}{4.936071in}%
\pgfsys@useobject{currentmarker}{}%
\end{pgfscope}%
\begin{pgfscope}%
\pgfsys@transformshift{6.497030in}{5.002320in}%
\pgfsys@useobject{currentmarker}{}%
\end{pgfscope}%
\begin{pgfscope}%
\pgfsys@transformshift{6.569760in}{5.061874in}%
\pgfsys@useobject{currentmarker}{}%
\end{pgfscope}%
\begin{pgfscope}%
\pgfsys@transformshift{6.642490in}{5.113971in}%
\pgfsys@useobject{currentmarker}{}%
\end{pgfscope}%
\begin{pgfscope}%
\pgfsys@transformshift{6.715220in}{5.161441in}%
\pgfsys@useobject{currentmarker}{}%
\end{pgfscope}%
\begin{pgfscope}%
\pgfsys@transformshift{6.787950in}{5.204224in}%
\pgfsys@useobject{currentmarker}{}%
\end{pgfscope}%
\begin{pgfscope}%
\pgfsys@transformshift{6.860680in}{5.243349in}%
\pgfsys@useobject{currentmarker}{}%
\end{pgfscope}%
\begin{pgfscope}%
\pgfsys@transformshift{6.933410in}{5.279068in}%
\pgfsys@useobject{currentmarker}{}%
\end{pgfscope}%
\begin{pgfscope}%
\pgfsys@transformshift{7.006139in}{5.311778in}%
\pgfsys@useobject{currentmarker}{}%
\end{pgfscope}%
\begin{pgfscope}%
\pgfsys@transformshift{7.078869in}{5.342359in}%
\pgfsys@useobject{currentmarker}{}%
\end{pgfscope}%
\begin{pgfscope}%
\pgfsys@transformshift{7.151599in}{5.370424in}%
\pgfsys@useobject{currentmarker}{}%
\end{pgfscope}%
\begin{pgfscope}%
\pgfsys@transformshift{7.224329in}{5.396883in}%
\pgfsys@useobject{currentmarker}{}%
\end{pgfscope}%
\begin{pgfscope}%
\pgfsys@transformshift{7.297059in}{5.421463in}%
\pgfsys@useobject{currentmarker}{}%
\end{pgfscope}%
\begin{pgfscope}%
\pgfsys@transformshift{7.369789in}{5.444290in}%
\pgfsys@useobject{currentmarker}{}%
\end{pgfscope}%
\begin{pgfscope}%
\pgfsys@transformshift{7.442518in}{5.465851in}%
\pgfsys@useobject{currentmarker}{}%
\end{pgfscope}%
\begin{pgfscope}%
\pgfsys@transformshift{7.515248in}{5.485984in}%
\pgfsys@useobject{currentmarker}{}%
\end{pgfscope}%
\begin{pgfscope}%
\pgfsys@transformshift{7.587978in}{5.505105in}%
\pgfsys@useobject{currentmarker}{}%
\end{pgfscope}%
\begin{pgfscope}%
\pgfsys@transformshift{7.660708in}{5.523240in}%
\pgfsys@useobject{currentmarker}{}%
\end{pgfscope}%
\end{pgfscope}%
\begin{pgfscope}%
\pgfsetbuttcap%
\pgfsetroundjoin%
\definecolor{currentfill}{rgb}{0.000000,0.000000,0.000000}%
\pgfsetfillcolor{currentfill}%
\pgfsetlinewidth{0.803000pt}%
\definecolor{currentstroke}{rgb}{0.000000,0.000000,0.000000}%
\pgfsetstrokecolor{currentstroke}%
\pgfsetdash{}{0pt}%
\pgfsys@defobject{currentmarker}{\pgfqpoint{0.000000in}{-0.048611in}}{\pgfqpoint{0.000000in}{0.000000in}}{%
\pgfpathmoveto{\pgfqpoint{0.000000in}{0.000000in}}%
\pgfpathlineto{\pgfqpoint{0.000000in}{-0.048611in}}%
\pgfusepath{stroke,fill}%
}%
\begin{pgfscope}%
\pgfsys@transformshift{5.115163in}{3.312222in}%
\pgfsys@useobject{currentmarker}{}%
\end{pgfscope}%
\end{pgfscope}%
\begin{pgfscope}%
\definecolor{textcolor}{rgb}{0.000000,0.000000,0.000000}%
\pgfsetstrokecolor{textcolor}%
\pgfsetfillcolor{textcolor}%
\pgftext[x=5.115163in,y=3.215000in,,top]{\color{textcolor}\sffamily\fontsize{10.000000}{12.000000}\selectfont 1}%
\end{pgfscope}%
\begin{pgfscope}%
\pgfsetbuttcap%
\pgfsetroundjoin%
\definecolor{currentfill}{rgb}{0.000000,0.000000,0.000000}%
\pgfsetfillcolor{currentfill}%
\pgfsetlinewidth{0.803000pt}%
\definecolor{currentstroke}{rgb}{0.000000,0.000000,0.000000}%
\pgfsetstrokecolor{currentstroke}%
\pgfsetdash{}{0pt}%
\pgfsys@defobject{currentmarker}{\pgfqpoint{0.000000in}{-0.048611in}}{\pgfqpoint{0.000000in}{0.000000in}}{%
\pgfpathmoveto{\pgfqpoint{0.000000in}{0.000000in}}%
\pgfpathlineto{\pgfqpoint{0.000000in}{-0.048611in}}%
\pgfusepath{stroke,fill}%
}%
\begin{pgfscope}%
\pgfsys@transformshift{5.842462in}{3.312222in}%
\pgfsys@useobject{currentmarker}{}%
\end{pgfscope}%
\end{pgfscope}%
\begin{pgfscope}%
\definecolor{textcolor}{rgb}{0.000000,0.000000,0.000000}%
\pgfsetstrokecolor{textcolor}%
\pgfsetfillcolor{textcolor}%
\pgftext[x=5.842462in,y=3.215000in,,top]{\color{textcolor}\sffamily\fontsize{10.000000}{12.000000}\selectfont 2}%
\end{pgfscope}%
\begin{pgfscope}%
\pgfsetbuttcap%
\pgfsetroundjoin%
\definecolor{currentfill}{rgb}{0.000000,0.000000,0.000000}%
\pgfsetfillcolor{currentfill}%
\pgfsetlinewidth{0.803000pt}%
\definecolor{currentstroke}{rgb}{0.000000,0.000000,0.000000}%
\pgfsetstrokecolor{currentstroke}%
\pgfsetdash{}{0pt}%
\pgfsys@defobject{currentmarker}{\pgfqpoint{0.000000in}{-0.048611in}}{\pgfqpoint{0.000000in}{0.000000in}}{%
\pgfpathmoveto{\pgfqpoint{0.000000in}{0.000000in}}%
\pgfpathlineto{\pgfqpoint{0.000000in}{-0.048611in}}%
\pgfusepath{stroke,fill}%
}%
\begin{pgfscope}%
\pgfsys@transformshift{6.569760in}{3.312222in}%
\pgfsys@useobject{currentmarker}{}%
\end{pgfscope}%
\end{pgfscope}%
\begin{pgfscope}%
\definecolor{textcolor}{rgb}{0.000000,0.000000,0.000000}%
\pgfsetstrokecolor{textcolor}%
\pgfsetfillcolor{textcolor}%
\pgftext[x=6.569760in,y=3.215000in,,top]{\color{textcolor}\sffamily\fontsize{10.000000}{12.000000}\selectfont 3}%
\end{pgfscope}%
\begin{pgfscope}%
\pgfsetbuttcap%
\pgfsetroundjoin%
\definecolor{currentfill}{rgb}{0.000000,0.000000,0.000000}%
\pgfsetfillcolor{currentfill}%
\pgfsetlinewidth{0.803000pt}%
\definecolor{currentstroke}{rgb}{0.000000,0.000000,0.000000}%
\pgfsetstrokecolor{currentstroke}%
\pgfsetdash{}{0pt}%
\pgfsys@defobject{currentmarker}{\pgfqpoint{0.000000in}{-0.048611in}}{\pgfqpoint{0.000000in}{0.000000in}}{%
\pgfpathmoveto{\pgfqpoint{0.000000in}{0.000000in}}%
\pgfpathlineto{\pgfqpoint{0.000000in}{-0.048611in}}%
\pgfusepath{stroke,fill}%
}%
\begin{pgfscope}%
\pgfsys@transformshift{7.297059in}{3.312222in}%
\pgfsys@useobject{currentmarker}{}%
\end{pgfscope}%
\end{pgfscope}%
\begin{pgfscope}%
\definecolor{textcolor}{rgb}{0.000000,0.000000,0.000000}%
\pgfsetstrokecolor{textcolor}%
\pgfsetfillcolor{textcolor}%
\pgftext[x=7.297059in,y=3.215000in,,top]{\color{textcolor}\sffamily\fontsize{10.000000}{12.000000}\selectfont 4}%
\end{pgfscope}%
\begin{pgfscope}%
\pgfsetbuttcap%
\pgfsetroundjoin%
\definecolor{currentfill}{rgb}{0.000000,0.000000,0.000000}%
\pgfsetfillcolor{currentfill}%
\pgfsetlinewidth{0.803000pt}%
\definecolor{currentstroke}{rgb}{0.000000,0.000000,0.000000}%
\pgfsetstrokecolor{currentstroke}%
\pgfsetdash{}{0pt}%
\pgfsys@defobject{currentmarker}{\pgfqpoint{-0.048611in}{0.000000in}}{\pgfqpoint{0.000000in}{0.000000in}}{%
\pgfpathmoveto{\pgfqpoint{0.000000in}{0.000000in}}%
\pgfpathlineto{\pgfqpoint{-0.048611in}{0.000000in}}%
\pgfusepath{stroke,fill}%
}%
\begin{pgfscope}%
\pgfsys@transformshift{4.597222in}{3.425648in}%
\pgfsys@useobject{currentmarker}{}%
\end{pgfscope}%
\end{pgfscope}%
\begin{pgfscope}%
\definecolor{textcolor}{rgb}{0.000000,0.000000,0.000000}%
\pgfsetstrokecolor{textcolor}%
\pgfsetfillcolor{textcolor}%
\pgftext[x=4.074382in,y=3.372886in,left,base]{\color{textcolor}\sffamily\fontsize{10.000000}{12.000000}\selectfont −2.00}%
\end{pgfscope}%
\begin{pgfscope}%
\pgfsetbuttcap%
\pgfsetroundjoin%
\definecolor{currentfill}{rgb}{0.000000,0.000000,0.000000}%
\pgfsetfillcolor{currentfill}%
\pgfsetlinewidth{0.803000pt}%
\definecolor{currentstroke}{rgb}{0.000000,0.000000,0.000000}%
\pgfsetstrokecolor{currentstroke}%
\pgfsetdash{}{0pt}%
\pgfsys@defobject{currentmarker}{\pgfqpoint{-0.048611in}{0.000000in}}{\pgfqpoint{0.000000in}{0.000000in}}{%
\pgfpathmoveto{\pgfqpoint{0.000000in}{0.000000in}}%
\pgfpathlineto{\pgfqpoint{-0.048611in}{0.000000in}}%
\pgfusepath{stroke,fill}%
}%
\begin{pgfscope}%
\pgfsys@transformshift{4.597222in}{3.771469in}%
\pgfsys@useobject{currentmarker}{}%
\end{pgfscope}%
\end{pgfscope}%
\begin{pgfscope}%
\definecolor{textcolor}{rgb}{0.000000,0.000000,0.000000}%
\pgfsetstrokecolor{textcolor}%
\pgfsetfillcolor{textcolor}%
\pgftext[x=4.074382in,y=3.718708in,left,base]{\color{textcolor}\sffamily\fontsize{10.000000}{12.000000}\selectfont −1.75}%
\end{pgfscope}%
\begin{pgfscope}%
\pgfsetbuttcap%
\pgfsetroundjoin%
\definecolor{currentfill}{rgb}{0.000000,0.000000,0.000000}%
\pgfsetfillcolor{currentfill}%
\pgfsetlinewidth{0.803000pt}%
\definecolor{currentstroke}{rgb}{0.000000,0.000000,0.000000}%
\pgfsetstrokecolor{currentstroke}%
\pgfsetdash{}{0pt}%
\pgfsys@defobject{currentmarker}{\pgfqpoint{-0.048611in}{0.000000in}}{\pgfqpoint{0.000000in}{0.000000in}}{%
\pgfpathmoveto{\pgfqpoint{0.000000in}{0.000000in}}%
\pgfpathlineto{\pgfqpoint{-0.048611in}{0.000000in}}%
\pgfusepath{stroke,fill}%
}%
\begin{pgfscope}%
\pgfsys@transformshift{4.597222in}{4.117291in}%
\pgfsys@useobject{currentmarker}{}%
\end{pgfscope}%
\end{pgfscope}%
\begin{pgfscope}%
\definecolor{textcolor}{rgb}{0.000000,0.000000,0.000000}%
\pgfsetstrokecolor{textcolor}%
\pgfsetfillcolor{textcolor}%
\pgftext[x=4.074382in,y=4.064530in,left,base]{\color{textcolor}\sffamily\fontsize{10.000000}{12.000000}\selectfont −1.50}%
\end{pgfscope}%
\begin{pgfscope}%
\pgfsetbuttcap%
\pgfsetroundjoin%
\definecolor{currentfill}{rgb}{0.000000,0.000000,0.000000}%
\pgfsetfillcolor{currentfill}%
\pgfsetlinewidth{0.803000pt}%
\definecolor{currentstroke}{rgb}{0.000000,0.000000,0.000000}%
\pgfsetstrokecolor{currentstroke}%
\pgfsetdash{}{0pt}%
\pgfsys@defobject{currentmarker}{\pgfqpoint{-0.048611in}{0.000000in}}{\pgfqpoint{0.000000in}{0.000000in}}{%
\pgfpathmoveto{\pgfqpoint{0.000000in}{0.000000in}}%
\pgfpathlineto{\pgfqpoint{-0.048611in}{0.000000in}}%
\pgfusepath{stroke,fill}%
}%
\begin{pgfscope}%
\pgfsys@transformshift{4.597222in}{4.463113in}%
\pgfsys@useobject{currentmarker}{}%
\end{pgfscope}%
\end{pgfscope}%
\begin{pgfscope}%
\definecolor{textcolor}{rgb}{0.000000,0.000000,0.000000}%
\pgfsetstrokecolor{textcolor}%
\pgfsetfillcolor{textcolor}%
\pgftext[x=4.074382in,y=4.410352in,left,base]{\color{textcolor}\sffamily\fontsize{10.000000}{12.000000}\selectfont −1.25}%
\end{pgfscope}%
\begin{pgfscope}%
\pgfsetbuttcap%
\pgfsetroundjoin%
\definecolor{currentfill}{rgb}{0.000000,0.000000,0.000000}%
\pgfsetfillcolor{currentfill}%
\pgfsetlinewidth{0.803000pt}%
\definecolor{currentstroke}{rgb}{0.000000,0.000000,0.000000}%
\pgfsetstrokecolor{currentstroke}%
\pgfsetdash{}{0pt}%
\pgfsys@defobject{currentmarker}{\pgfqpoint{-0.048611in}{0.000000in}}{\pgfqpoint{0.000000in}{0.000000in}}{%
\pgfpathmoveto{\pgfqpoint{0.000000in}{0.000000in}}%
\pgfpathlineto{\pgfqpoint{-0.048611in}{0.000000in}}%
\pgfusepath{stroke,fill}%
}%
\begin{pgfscope}%
\pgfsys@transformshift{4.597222in}{4.808935in}%
\pgfsys@useobject{currentmarker}{}%
\end{pgfscope}%
\end{pgfscope}%
\begin{pgfscope}%
\definecolor{textcolor}{rgb}{0.000000,0.000000,0.000000}%
\pgfsetstrokecolor{textcolor}%
\pgfsetfillcolor{textcolor}%
\pgftext[x=4.074382in,y=4.756174in,left,base]{\color{textcolor}\sffamily\fontsize{10.000000}{12.000000}\selectfont −1.00}%
\end{pgfscope}%
\begin{pgfscope}%
\pgfsetbuttcap%
\pgfsetroundjoin%
\definecolor{currentfill}{rgb}{0.000000,0.000000,0.000000}%
\pgfsetfillcolor{currentfill}%
\pgfsetlinewidth{0.803000pt}%
\definecolor{currentstroke}{rgb}{0.000000,0.000000,0.000000}%
\pgfsetstrokecolor{currentstroke}%
\pgfsetdash{}{0pt}%
\pgfsys@defobject{currentmarker}{\pgfqpoint{-0.048611in}{0.000000in}}{\pgfqpoint{0.000000in}{0.000000in}}{%
\pgfpathmoveto{\pgfqpoint{0.000000in}{0.000000in}}%
\pgfpathlineto{\pgfqpoint{-0.048611in}{0.000000in}}%
\pgfusepath{stroke,fill}%
}%
\begin{pgfscope}%
\pgfsys@transformshift{4.597222in}{5.154757in}%
\pgfsys@useobject{currentmarker}{}%
\end{pgfscope}%
\end{pgfscope}%
\begin{pgfscope}%
\definecolor{textcolor}{rgb}{0.000000,0.000000,0.000000}%
\pgfsetstrokecolor{textcolor}%
\pgfsetfillcolor{textcolor}%
\pgftext[x=4.074382in,y=5.101996in,left,base]{\color{textcolor}\sffamily\fontsize{10.000000}{12.000000}\selectfont −0.75}%
\end{pgfscope}%
\begin{pgfscope}%
\pgfsetbuttcap%
\pgfsetroundjoin%
\definecolor{currentfill}{rgb}{0.000000,0.000000,0.000000}%
\pgfsetfillcolor{currentfill}%
\pgfsetlinewidth{0.803000pt}%
\definecolor{currentstroke}{rgb}{0.000000,0.000000,0.000000}%
\pgfsetstrokecolor{currentstroke}%
\pgfsetdash{}{0pt}%
\pgfsys@defobject{currentmarker}{\pgfqpoint{-0.048611in}{0.000000in}}{\pgfqpoint{0.000000in}{0.000000in}}{%
\pgfpathmoveto{\pgfqpoint{0.000000in}{0.000000in}}%
\pgfpathlineto{\pgfqpoint{-0.048611in}{0.000000in}}%
\pgfusepath{stroke,fill}%
}%
\begin{pgfscope}%
\pgfsys@transformshift{4.597222in}{5.500579in}%
\pgfsys@useobject{currentmarker}{}%
\end{pgfscope}%
\end{pgfscope}%
\begin{pgfscope}%
\definecolor{textcolor}{rgb}{0.000000,0.000000,0.000000}%
\pgfsetstrokecolor{textcolor}%
\pgfsetfillcolor{textcolor}%
\pgftext[x=4.074382in,y=5.447818in,left,base]{\color{textcolor}\sffamily\fontsize{10.000000}{12.000000}\selectfont −0.50}%
\end{pgfscope}%
\begin{pgfscope}%
\pgfpathrectangle{\pgfqpoint{4.597222in}{3.312222in}}{\pgfqpoint{3.217778in}{2.324444in}}%
\pgfusepath{clip}%
\pgfsetrectcap%
\pgfsetroundjoin%
\pgfsetlinewidth{1.505625pt}%
\definecolor{currentstroke}{rgb}{0.121569,0.466667,0.705882}%
\pgfsetstrokecolor{currentstroke}%
\pgfsetdash{}{0pt}%
\pgfpathmoveto{\pgfqpoint{4.751514in}{3.425649in}}%
\pgfpathlineto{\pgfqpoint{4.824244in}{3.425666in}}%
\pgfpathlineto{\pgfqpoint{4.896974in}{3.425771in}}%
\pgfpathlineto{\pgfqpoint{4.969704in}{3.426159in}}%
\pgfpathlineto{\pgfqpoint{5.042434in}{3.427237in}}%
\pgfpathlineto{\pgfqpoint{5.115163in}{3.429586in}}%
\pgfpathlineto{\pgfqpoint{5.187893in}{3.433988in}}%
\pgfpathlineto{\pgfqpoint{5.260623in}{3.441399in}}%
\pgfpathlineto{\pgfqpoint{5.333353in}{3.452952in}}%
\pgfpathlineto{\pgfqpoint{5.406083in}{3.469830in}}%
\pgfpathlineto{\pgfqpoint{5.478813in}{3.493243in}}%
\pgfpathlineto{\pgfqpoint{5.551543in}{3.524892in}}%
\pgfpathlineto{\pgfqpoint{5.624272in}{3.566649in}}%
\pgfpathlineto{\pgfqpoint{5.697002in}{3.620426in}}%
\pgfpathlineto{\pgfqpoint{5.769732in}{3.688946in}}%
\pgfpathlineto{\pgfqpoint{5.842462in}{3.777397in}}%
\pgfpathlineto{\pgfqpoint{5.915192in}{3.893488in}}%
\pgfpathlineto{\pgfqpoint{5.987922in}{4.052866in}}%
\pgfpathlineto{\pgfqpoint{6.060651in}{4.292136in}}%
\pgfpathlineto{\pgfqpoint{6.133381in}{4.514642in}}%
\pgfpathlineto{\pgfqpoint{6.206111in}{4.659732in}}%
\pgfpathlineto{\pgfqpoint{6.278841in}{4.769403in}}%
\pgfpathlineto{\pgfqpoint{6.351571in}{4.858877in}}%
\pgfpathlineto{\pgfqpoint{6.424301in}{4.936071in}}%
\pgfpathlineto{\pgfqpoint{6.497030in}{5.002320in}}%
\pgfpathlineto{\pgfqpoint{6.569760in}{5.061874in}}%
\pgfpathlineto{\pgfqpoint{6.642490in}{5.113971in}}%
\pgfpathlineto{\pgfqpoint{6.715220in}{5.161441in}}%
\pgfpathlineto{\pgfqpoint{6.787950in}{5.204224in}}%
\pgfpathlineto{\pgfqpoint{6.860680in}{5.243349in}}%
\pgfpathlineto{\pgfqpoint{6.933410in}{5.279068in}}%
\pgfpathlineto{\pgfqpoint{7.006139in}{5.311778in}}%
\pgfpathlineto{\pgfqpoint{7.078869in}{5.342359in}}%
\pgfpathlineto{\pgfqpoint{7.151599in}{5.370424in}}%
\pgfpathlineto{\pgfqpoint{7.224329in}{5.396883in}}%
\pgfpathlineto{\pgfqpoint{7.297059in}{5.421463in}}%
\pgfpathlineto{\pgfqpoint{7.369789in}{5.444290in}}%
\pgfpathlineto{\pgfqpoint{7.442518in}{5.465851in}}%
\pgfpathlineto{\pgfqpoint{7.515248in}{5.485984in}}%
\pgfpathlineto{\pgfqpoint{7.587978in}{5.505105in}}%
\pgfpathlineto{\pgfqpoint{7.660708in}{5.523240in}}%
\pgfusepath{stroke}%
\end{pgfscope}%
\begin{pgfscope}%
\pgfsetrectcap%
\pgfsetmiterjoin%
\pgfsetlinewidth{0.803000pt}%
\definecolor{currentstroke}{rgb}{0.000000,0.000000,0.000000}%
\pgfsetstrokecolor{currentstroke}%
\pgfsetdash{}{0pt}%
\pgfpathmoveto{\pgfqpoint{4.597222in}{3.312222in}}%
\pgfpathlineto{\pgfqpoint{4.597222in}{5.636667in}}%
\pgfusepath{stroke}%
\end{pgfscope}%
\begin{pgfscope}%
\pgfsetrectcap%
\pgfsetmiterjoin%
\pgfsetlinewidth{0.803000pt}%
\definecolor{currentstroke}{rgb}{0.000000,0.000000,0.000000}%
\pgfsetstrokecolor{currentstroke}%
\pgfsetdash{}{0pt}%
\pgfpathmoveto{\pgfqpoint{7.815000in}{3.312222in}}%
\pgfpathlineto{\pgfqpoint{7.815000in}{5.636667in}}%
\pgfusepath{stroke}%
\end{pgfscope}%
\begin{pgfscope}%
\pgfsetrectcap%
\pgfsetmiterjoin%
\pgfsetlinewidth{0.803000pt}%
\definecolor{currentstroke}{rgb}{0.000000,0.000000,0.000000}%
\pgfsetstrokecolor{currentstroke}%
\pgfsetdash{}{0pt}%
\pgfpathmoveto{\pgfqpoint{4.597222in}{3.312222in}}%
\pgfpathlineto{\pgfqpoint{7.815000in}{3.312222in}}%
\pgfusepath{stroke}%
\end{pgfscope}%
\begin{pgfscope}%
\pgfsetrectcap%
\pgfsetmiterjoin%
\pgfsetlinewidth{0.803000pt}%
\definecolor{currentstroke}{rgb}{0.000000,0.000000,0.000000}%
\pgfsetstrokecolor{currentstroke}%
\pgfsetdash{}{0pt}%
\pgfpathmoveto{\pgfqpoint{4.597222in}{5.636667in}}%
\pgfpathlineto{\pgfqpoint{7.815000in}{5.636667in}}%
\pgfusepath{stroke}%
\end{pgfscope}%
\begin{pgfscope}%
\definecolor{textcolor}{rgb}{0.000000,0.000000,0.000000}%
\pgfsetstrokecolor{textcolor}%
\pgfsetfillcolor{textcolor}%
\pgftext[x=6.206111in,y=5.720000in,,base]{\color{textcolor}\sffamily\fontsize{12.000000}{14.400000}\selectfont \(\displaystyle  N = 32 \)}%
\end{pgfscope}%
\begin{pgfscope}%
\pgfsetbuttcap%
\pgfsetmiterjoin%
\definecolor{currentfill}{rgb}{1.000000,1.000000,1.000000}%
\pgfsetfillcolor{currentfill}%
\pgfsetlinewidth{0.000000pt}%
\definecolor{currentstroke}{rgb}{0.000000,0.000000,0.000000}%
\pgfsetstrokecolor{currentstroke}%
\pgfsetstrokeopacity{0.000000}%
\pgfsetdash{}{0pt}%
\pgfpathmoveto{\pgfqpoint{0.672222in}{0.387222in}}%
\pgfpathlineto{\pgfqpoint{3.890000in}{0.387222in}}%
\pgfpathlineto{\pgfqpoint{3.890000in}{2.711667in}}%
\pgfpathlineto{\pgfqpoint{0.672222in}{2.711667in}}%
\pgfpathclose%
\pgfusepath{fill}%
\end{pgfscope}%
\begin{pgfscope}%
\pgfpathrectangle{\pgfqpoint{0.672222in}{0.387222in}}{\pgfqpoint{3.217778in}{2.324444in}}%
\pgfusepath{clip}%
\pgfsetbuttcap%
\pgfsetroundjoin%
\definecolor{currentfill}{rgb}{0.121569,0.466667,0.705882}%
\pgfsetfillcolor{currentfill}%
\pgfsetfillopacity{0.300000}%
\pgfsetlinewidth{1.003750pt}%
\definecolor{currentstroke}{rgb}{0.121569,0.466667,0.705882}%
\pgfsetstrokecolor{currentstroke}%
\pgfsetstrokeopacity{0.300000}%
\pgfsetdash{}{0pt}%
\pgfpathmoveto{\pgfqpoint{0.826514in}{0.610585in}}%
\pgfpathlineto{\pgfqpoint{0.826514in}{0.496662in}}%
\pgfpathlineto{\pgfqpoint{0.899244in}{0.512532in}}%
\pgfpathlineto{\pgfqpoint{0.971974in}{0.494120in}}%
\pgfpathlineto{\pgfqpoint{1.044704in}{0.492879in}}%
\pgfpathlineto{\pgfqpoint{1.117434in}{0.525851in}}%
\pgfpathlineto{\pgfqpoint{1.190163in}{0.522667in}}%
\pgfpathlineto{\pgfqpoint{1.262893in}{0.539916in}}%
\pgfpathlineto{\pgfqpoint{1.335623in}{0.547189in}}%
\pgfpathlineto{\pgfqpoint{1.408353in}{0.558568in}}%
\pgfpathlineto{\pgfqpoint{1.481083in}{0.575112in}}%
\pgfpathlineto{\pgfqpoint{1.553813in}{0.598109in}}%
\pgfpathlineto{\pgfqpoint{1.626543in}{0.629647in}}%
\pgfpathlineto{\pgfqpoint{1.699272in}{0.670171in}}%
\pgfpathlineto{\pgfqpoint{1.772002in}{0.723252in}}%
\pgfpathlineto{\pgfqpoint{1.844732in}{0.791256in}}%
\pgfpathlineto{\pgfqpoint{1.917462in}{0.878679in}}%
\pgfpathlineto{\pgfqpoint{1.990192in}{0.991415in}}%
\pgfpathlineto{\pgfqpoint{2.062922in}{1.145539in}}%
\pgfpathlineto{\pgfqpoint{2.135651in}{1.401799in}}%
\pgfpathlineto{\pgfqpoint{2.208381in}{1.615695in}}%
\pgfpathlineto{\pgfqpoint{2.281111in}{1.748279in}}%
\pgfpathlineto{\pgfqpoint{2.353841in}{1.854942in}}%
\pgfpathlineto{\pgfqpoint{2.426571in}{1.943677in}}%
\pgfpathlineto{\pgfqpoint{2.499301in}{2.018610in}}%
\pgfpathlineto{\pgfqpoint{2.572030in}{2.084632in}}%
\pgfpathlineto{\pgfqpoint{2.644760in}{2.142923in}}%
\pgfpathlineto{\pgfqpoint{2.717490in}{2.194872in}}%
\pgfpathlineto{\pgfqpoint{2.790220in}{2.241304in}}%
\pgfpathlineto{\pgfqpoint{2.862950in}{2.283667in}}%
\pgfpathlineto{\pgfqpoint{2.935680in}{2.322090in}}%
\pgfpathlineto{\pgfqpoint{3.008410in}{2.357872in}}%
\pgfpathlineto{\pgfqpoint{3.081139in}{2.389911in}}%
\pgfpathlineto{\pgfqpoint{3.153869in}{2.419734in}}%
\pgfpathlineto{\pgfqpoint{3.226599in}{2.447592in}}%
\pgfpathlineto{\pgfqpoint{3.299329in}{2.473699in}}%
\pgfpathlineto{\pgfqpoint{3.372059in}{2.498011in}}%
\pgfpathlineto{\pgfqpoint{3.444789in}{2.520391in}}%
\pgfpathlineto{\pgfqpoint{3.517518in}{2.541663in}}%
\pgfpathlineto{\pgfqpoint{3.590248in}{2.561525in}}%
\pgfpathlineto{\pgfqpoint{3.662978in}{2.580353in}}%
\pgfpathlineto{\pgfqpoint{3.735708in}{2.598101in}}%
\pgfpathlineto{\pgfqpoint{3.735708in}{2.598322in}}%
\pgfpathlineto{\pgfqpoint{3.735708in}{2.598322in}}%
\pgfpathlineto{\pgfqpoint{3.662978in}{2.580782in}}%
\pgfpathlineto{\pgfqpoint{3.590248in}{2.561636in}}%
\pgfpathlineto{\pgfqpoint{3.517518in}{2.542517in}}%
\pgfpathlineto{\pgfqpoint{3.444789in}{2.521019in}}%
\pgfpathlineto{\pgfqpoint{3.372059in}{2.498618in}}%
\pgfpathlineto{\pgfqpoint{3.299329in}{2.474159in}}%
\pgfpathlineto{\pgfqpoint{3.226599in}{2.448713in}}%
\pgfpathlineto{\pgfqpoint{3.153869in}{2.420560in}}%
\pgfpathlineto{\pgfqpoint{3.081139in}{2.390612in}}%
\pgfpathlineto{\pgfqpoint{3.008410in}{2.358229in}}%
\pgfpathlineto{\pgfqpoint{2.935680in}{2.322799in}}%
\pgfpathlineto{\pgfqpoint{2.862950in}{2.284430in}}%
\pgfpathlineto{\pgfqpoint{2.790220in}{2.242299in}}%
\pgfpathlineto{\pgfqpoint{2.717490in}{2.195801in}}%
\pgfpathlineto{\pgfqpoint{2.644760in}{2.144531in}}%
\pgfpathlineto{\pgfqpoint{2.572030in}{2.085973in}}%
\pgfpathlineto{\pgfqpoint{2.499301in}{2.020013in}}%
\pgfpathlineto{\pgfqpoint{2.426571in}{1.945029in}}%
\pgfpathlineto{\pgfqpoint{2.353841in}{1.857467in}}%
\pgfpathlineto{\pgfqpoint{2.281111in}{1.752182in}}%
\pgfpathlineto{\pgfqpoint{2.208381in}{1.617134in}}%
\pgfpathlineto{\pgfqpoint{2.135651in}{1.421473in}}%
\pgfpathlineto{\pgfqpoint{2.062922in}{1.149284in}}%
\pgfpathlineto{\pgfqpoint{1.990192in}{0.993133in}}%
\pgfpathlineto{\pgfqpoint{1.917462in}{0.879457in}}%
\pgfpathlineto{\pgfqpoint{1.844732in}{0.791930in}}%
\pgfpathlineto{\pgfqpoint{1.772002in}{0.723951in}}%
\pgfpathlineto{\pgfqpoint{1.699272in}{0.670827in}}%
\pgfpathlineto{\pgfqpoint{1.626543in}{0.629818in}}%
\pgfpathlineto{\pgfqpoint{1.553813in}{0.598507in}}%
\pgfpathlineto{\pgfqpoint{1.481083in}{0.575336in}}%
\pgfpathlineto{\pgfqpoint{1.408353in}{0.558652in}}%
\pgfpathlineto{\pgfqpoint{1.335623in}{0.547361in}}%
\pgfpathlineto{\pgfqpoint{1.262893in}{0.539944in}}%
\pgfpathlineto{\pgfqpoint{1.190163in}{0.564609in}}%
\pgfpathlineto{\pgfqpoint{1.117434in}{0.660205in}}%
\pgfpathlineto{\pgfqpoint{1.044704in}{0.620756in}}%
\pgfpathlineto{\pgfqpoint{0.971974in}{0.616703in}}%
\pgfpathlineto{\pgfqpoint{0.899244in}{0.574909in}}%
\pgfpathlineto{\pgfqpoint{0.826514in}{0.610585in}}%
\pgfpathclose%
\pgfusepath{stroke,fill}%
\end{pgfscope}%
\begin{pgfscope}%
\pgfpathrectangle{\pgfqpoint{0.672222in}{0.387222in}}{\pgfqpoint{3.217778in}{2.324444in}}%
\pgfusepath{clip}%
\pgfsetbuttcap%
\pgfsetroundjoin%
\definecolor{currentfill}{rgb}{0.121569,0.466667,0.705882}%
\pgfsetfillcolor{currentfill}%
\pgfsetlinewidth{1.003750pt}%
\definecolor{currentstroke}{rgb}{0.121569,0.466667,0.705882}%
\pgfsetstrokecolor{currentstroke}%
\pgfsetdash{}{0pt}%
\pgfsys@defobject{currentmarker}{\pgfqpoint{-0.009821in}{-0.009821in}}{\pgfqpoint{0.009821in}{0.009821in}}{%
\pgfpathmoveto{\pgfqpoint{0.000000in}{-0.009821in}}%
\pgfpathcurveto{\pgfqpoint{0.002605in}{-0.009821in}}{\pgfqpoint{0.005103in}{-0.008786in}}{\pgfqpoint{0.006944in}{-0.006944in}}%
\pgfpathcurveto{\pgfqpoint{0.008786in}{-0.005103in}}{\pgfqpoint{0.009821in}{-0.002605in}}{\pgfqpoint{0.009821in}{0.000000in}}%
\pgfpathcurveto{\pgfqpoint{0.009821in}{0.002605in}}{\pgfqpoint{0.008786in}{0.005103in}}{\pgfqpoint{0.006944in}{0.006944in}}%
\pgfpathcurveto{\pgfqpoint{0.005103in}{0.008786in}}{\pgfqpoint{0.002605in}{0.009821in}}{\pgfqpoint{0.000000in}{0.009821in}}%
\pgfpathcurveto{\pgfqpoint{-0.002605in}{0.009821in}}{\pgfqpoint{-0.005103in}{0.008786in}}{\pgfqpoint{-0.006944in}{0.006944in}}%
\pgfpathcurveto{\pgfqpoint{-0.008786in}{0.005103in}}{\pgfqpoint{-0.009821in}{0.002605in}}{\pgfqpoint{-0.009821in}{0.000000in}}%
\pgfpathcurveto{\pgfqpoint{-0.009821in}{-0.002605in}}{\pgfqpoint{-0.008786in}{-0.005103in}}{\pgfqpoint{-0.006944in}{-0.006944in}}%
\pgfpathcurveto{\pgfqpoint{-0.005103in}{-0.008786in}}{\pgfqpoint{-0.002605in}{-0.009821in}}{\pgfqpoint{0.000000in}{-0.009821in}}%
\pgfpathclose%
\pgfusepath{stroke,fill}%
}%
\begin{pgfscope}%
\pgfsys@transformshift{0.826514in}{0.553623in}%
\pgfsys@useobject{currentmarker}{}%
\end{pgfscope}%
\begin{pgfscope}%
\pgfsys@transformshift{0.899244in}{0.543721in}%
\pgfsys@useobject{currentmarker}{}%
\end{pgfscope}%
\begin{pgfscope}%
\pgfsys@transformshift{0.971974in}{0.555412in}%
\pgfsys@useobject{currentmarker}{}%
\end{pgfscope}%
\begin{pgfscope}%
\pgfsys@transformshift{1.044704in}{0.556817in}%
\pgfsys@useobject{currentmarker}{}%
\end{pgfscope}%
\begin{pgfscope}%
\pgfsys@transformshift{1.117434in}{0.593028in}%
\pgfsys@useobject{currentmarker}{}%
\end{pgfscope}%
\begin{pgfscope}%
\pgfsys@transformshift{1.190163in}{0.543638in}%
\pgfsys@useobject{currentmarker}{}%
\end{pgfscope}%
\begin{pgfscope}%
\pgfsys@transformshift{1.262893in}{0.539930in}%
\pgfsys@useobject{currentmarker}{}%
\end{pgfscope}%
\begin{pgfscope}%
\pgfsys@transformshift{1.335623in}{0.547275in}%
\pgfsys@useobject{currentmarker}{}%
\end{pgfscope}%
\begin{pgfscope}%
\pgfsys@transformshift{1.408353in}{0.558610in}%
\pgfsys@useobject{currentmarker}{}%
\end{pgfscope}%
\begin{pgfscope}%
\pgfsys@transformshift{1.481083in}{0.575224in}%
\pgfsys@useobject{currentmarker}{}%
\end{pgfscope}%
\begin{pgfscope}%
\pgfsys@transformshift{1.553813in}{0.598308in}%
\pgfsys@useobject{currentmarker}{}%
\end{pgfscope}%
\begin{pgfscope}%
\pgfsys@transformshift{1.626543in}{0.629733in}%
\pgfsys@useobject{currentmarker}{}%
\end{pgfscope}%
\begin{pgfscope}%
\pgfsys@transformshift{1.699272in}{0.670499in}%
\pgfsys@useobject{currentmarker}{}%
\end{pgfscope}%
\begin{pgfscope}%
\pgfsys@transformshift{1.772002in}{0.723601in}%
\pgfsys@useobject{currentmarker}{}%
\end{pgfscope}%
\begin{pgfscope}%
\pgfsys@transformshift{1.844732in}{0.791593in}%
\pgfsys@useobject{currentmarker}{}%
\end{pgfscope}%
\begin{pgfscope}%
\pgfsys@transformshift{1.917462in}{0.879068in}%
\pgfsys@useobject{currentmarker}{}%
\end{pgfscope}%
\begin{pgfscope}%
\pgfsys@transformshift{1.990192in}{0.992274in}%
\pgfsys@useobject{currentmarker}{}%
\end{pgfscope}%
\begin{pgfscope}%
\pgfsys@transformshift{2.062922in}{1.147411in}%
\pgfsys@useobject{currentmarker}{}%
\end{pgfscope}%
\begin{pgfscope}%
\pgfsys@transformshift{2.135651in}{1.411636in}%
\pgfsys@useobject{currentmarker}{}%
\end{pgfscope}%
\begin{pgfscope}%
\pgfsys@transformshift{2.208381in}{1.616415in}%
\pgfsys@useobject{currentmarker}{}%
\end{pgfscope}%
\begin{pgfscope}%
\pgfsys@transformshift{2.281111in}{1.750230in}%
\pgfsys@useobject{currentmarker}{}%
\end{pgfscope}%
\begin{pgfscope}%
\pgfsys@transformshift{2.353841in}{1.856205in}%
\pgfsys@useobject{currentmarker}{}%
\end{pgfscope}%
\begin{pgfscope}%
\pgfsys@transformshift{2.426571in}{1.944353in}%
\pgfsys@useobject{currentmarker}{}%
\end{pgfscope}%
\begin{pgfscope}%
\pgfsys@transformshift{2.499301in}{2.019311in}%
\pgfsys@useobject{currentmarker}{}%
\end{pgfscope}%
\begin{pgfscope}%
\pgfsys@transformshift{2.572030in}{2.085302in}%
\pgfsys@useobject{currentmarker}{}%
\end{pgfscope}%
\begin{pgfscope}%
\pgfsys@transformshift{2.644760in}{2.143727in}%
\pgfsys@useobject{currentmarker}{}%
\end{pgfscope}%
\begin{pgfscope}%
\pgfsys@transformshift{2.717490in}{2.195336in}%
\pgfsys@useobject{currentmarker}{}%
\end{pgfscope}%
\begin{pgfscope}%
\pgfsys@transformshift{2.790220in}{2.241802in}%
\pgfsys@useobject{currentmarker}{}%
\end{pgfscope}%
\begin{pgfscope}%
\pgfsys@transformshift{2.862950in}{2.284049in}%
\pgfsys@useobject{currentmarker}{}%
\end{pgfscope}%
\begin{pgfscope}%
\pgfsys@transformshift{2.935680in}{2.322444in}%
\pgfsys@useobject{currentmarker}{}%
\end{pgfscope}%
\begin{pgfscope}%
\pgfsys@transformshift{3.008410in}{2.358051in}%
\pgfsys@useobject{currentmarker}{}%
\end{pgfscope}%
\begin{pgfscope}%
\pgfsys@transformshift{3.081139in}{2.390261in}%
\pgfsys@useobject{currentmarker}{}%
\end{pgfscope}%
\begin{pgfscope}%
\pgfsys@transformshift{3.153869in}{2.420147in}%
\pgfsys@useobject{currentmarker}{}%
\end{pgfscope}%
\begin{pgfscope}%
\pgfsys@transformshift{3.226599in}{2.448153in}%
\pgfsys@useobject{currentmarker}{}%
\end{pgfscope}%
\begin{pgfscope}%
\pgfsys@transformshift{3.299329in}{2.473929in}%
\pgfsys@useobject{currentmarker}{}%
\end{pgfscope}%
\begin{pgfscope}%
\pgfsys@transformshift{3.372059in}{2.498315in}%
\pgfsys@useobject{currentmarker}{}%
\end{pgfscope}%
\begin{pgfscope}%
\pgfsys@transformshift{3.444789in}{2.520705in}%
\pgfsys@useobject{currentmarker}{}%
\end{pgfscope}%
\begin{pgfscope}%
\pgfsys@transformshift{3.517518in}{2.542090in}%
\pgfsys@useobject{currentmarker}{}%
\end{pgfscope}%
\begin{pgfscope}%
\pgfsys@transformshift{3.590248in}{2.561581in}%
\pgfsys@useobject{currentmarker}{}%
\end{pgfscope}%
\begin{pgfscope}%
\pgfsys@transformshift{3.662978in}{2.580567in}%
\pgfsys@useobject{currentmarker}{}%
\end{pgfscope}%
\begin{pgfscope}%
\pgfsys@transformshift{3.735708in}{2.598212in}%
\pgfsys@useobject{currentmarker}{}%
\end{pgfscope}%
\end{pgfscope}%
\begin{pgfscope}%
\pgfsetbuttcap%
\pgfsetroundjoin%
\definecolor{currentfill}{rgb}{0.000000,0.000000,0.000000}%
\pgfsetfillcolor{currentfill}%
\pgfsetlinewidth{0.803000pt}%
\definecolor{currentstroke}{rgb}{0.000000,0.000000,0.000000}%
\pgfsetstrokecolor{currentstroke}%
\pgfsetdash{}{0pt}%
\pgfsys@defobject{currentmarker}{\pgfqpoint{0.000000in}{-0.048611in}}{\pgfqpoint{0.000000in}{0.000000in}}{%
\pgfpathmoveto{\pgfqpoint{0.000000in}{0.000000in}}%
\pgfpathlineto{\pgfqpoint{0.000000in}{-0.048611in}}%
\pgfusepath{stroke,fill}%
}%
\begin{pgfscope}%
\pgfsys@transformshift{1.190163in}{0.387222in}%
\pgfsys@useobject{currentmarker}{}%
\end{pgfscope}%
\end{pgfscope}%
\begin{pgfscope}%
\definecolor{textcolor}{rgb}{0.000000,0.000000,0.000000}%
\pgfsetstrokecolor{textcolor}%
\pgfsetfillcolor{textcolor}%
\pgftext[x=1.190163in,y=0.290000in,,top]{\color{textcolor}\sffamily\fontsize{10.000000}{12.000000}\selectfont 1}%
\end{pgfscope}%
\begin{pgfscope}%
\pgfsetbuttcap%
\pgfsetroundjoin%
\definecolor{currentfill}{rgb}{0.000000,0.000000,0.000000}%
\pgfsetfillcolor{currentfill}%
\pgfsetlinewidth{0.803000pt}%
\definecolor{currentstroke}{rgb}{0.000000,0.000000,0.000000}%
\pgfsetstrokecolor{currentstroke}%
\pgfsetdash{}{0pt}%
\pgfsys@defobject{currentmarker}{\pgfqpoint{0.000000in}{-0.048611in}}{\pgfqpoint{0.000000in}{0.000000in}}{%
\pgfpathmoveto{\pgfqpoint{0.000000in}{0.000000in}}%
\pgfpathlineto{\pgfqpoint{0.000000in}{-0.048611in}}%
\pgfusepath{stroke,fill}%
}%
\begin{pgfscope}%
\pgfsys@transformshift{1.917462in}{0.387222in}%
\pgfsys@useobject{currentmarker}{}%
\end{pgfscope}%
\end{pgfscope}%
\begin{pgfscope}%
\definecolor{textcolor}{rgb}{0.000000,0.000000,0.000000}%
\pgfsetstrokecolor{textcolor}%
\pgfsetfillcolor{textcolor}%
\pgftext[x=1.917462in,y=0.290000in,,top]{\color{textcolor}\sffamily\fontsize{10.000000}{12.000000}\selectfont 2}%
\end{pgfscope}%
\begin{pgfscope}%
\pgfsetbuttcap%
\pgfsetroundjoin%
\definecolor{currentfill}{rgb}{0.000000,0.000000,0.000000}%
\pgfsetfillcolor{currentfill}%
\pgfsetlinewidth{0.803000pt}%
\definecolor{currentstroke}{rgb}{0.000000,0.000000,0.000000}%
\pgfsetstrokecolor{currentstroke}%
\pgfsetdash{}{0pt}%
\pgfsys@defobject{currentmarker}{\pgfqpoint{0.000000in}{-0.048611in}}{\pgfqpoint{0.000000in}{0.000000in}}{%
\pgfpathmoveto{\pgfqpoint{0.000000in}{0.000000in}}%
\pgfpathlineto{\pgfqpoint{0.000000in}{-0.048611in}}%
\pgfusepath{stroke,fill}%
}%
\begin{pgfscope}%
\pgfsys@transformshift{2.644760in}{0.387222in}%
\pgfsys@useobject{currentmarker}{}%
\end{pgfscope}%
\end{pgfscope}%
\begin{pgfscope}%
\definecolor{textcolor}{rgb}{0.000000,0.000000,0.000000}%
\pgfsetstrokecolor{textcolor}%
\pgfsetfillcolor{textcolor}%
\pgftext[x=2.644760in,y=0.290000in,,top]{\color{textcolor}\sffamily\fontsize{10.000000}{12.000000}\selectfont 3}%
\end{pgfscope}%
\begin{pgfscope}%
\pgfsetbuttcap%
\pgfsetroundjoin%
\definecolor{currentfill}{rgb}{0.000000,0.000000,0.000000}%
\pgfsetfillcolor{currentfill}%
\pgfsetlinewidth{0.803000pt}%
\definecolor{currentstroke}{rgb}{0.000000,0.000000,0.000000}%
\pgfsetstrokecolor{currentstroke}%
\pgfsetdash{}{0pt}%
\pgfsys@defobject{currentmarker}{\pgfqpoint{0.000000in}{-0.048611in}}{\pgfqpoint{0.000000in}{0.000000in}}{%
\pgfpathmoveto{\pgfqpoint{0.000000in}{0.000000in}}%
\pgfpathlineto{\pgfqpoint{0.000000in}{-0.048611in}}%
\pgfusepath{stroke,fill}%
}%
\begin{pgfscope}%
\pgfsys@transformshift{3.372059in}{0.387222in}%
\pgfsys@useobject{currentmarker}{}%
\end{pgfscope}%
\end{pgfscope}%
\begin{pgfscope}%
\definecolor{textcolor}{rgb}{0.000000,0.000000,0.000000}%
\pgfsetstrokecolor{textcolor}%
\pgfsetfillcolor{textcolor}%
\pgftext[x=3.372059in,y=0.290000in,,top]{\color{textcolor}\sffamily\fontsize{10.000000}{12.000000}\selectfont 4}%
\end{pgfscope}%
\begin{pgfscope}%
\pgfsetbuttcap%
\pgfsetroundjoin%
\definecolor{currentfill}{rgb}{0.000000,0.000000,0.000000}%
\pgfsetfillcolor{currentfill}%
\pgfsetlinewidth{0.803000pt}%
\definecolor{currentstroke}{rgb}{0.000000,0.000000,0.000000}%
\pgfsetstrokecolor{currentstroke}%
\pgfsetdash{}{0pt}%
\pgfsys@defobject{currentmarker}{\pgfqpoint{-0.048611in}{0.000000in}}{\pgfqpoint{0.000000in}{0.000000in}}{%
\pgfpathmoveto{\pgfqpoint{0.000000in}{0.000000in}}%
\pgfpathlineto{\pgfqpoint{-0.048611in}{0.000000in}}%
\pgfusepath{stroke,fill}%
}%
\begin{pgfscope}%
\pgfsys@transformshift{0.672222in}{0.531698in}%
\pgfsys@useobject{currentmarker}{}%
\end{pgfscope}%
\end{pgfscope}%
\begin{pgfscope}%
\definecolor{textcolor}{rgb}{0.000000,0.000000,0.000000}%
\pgfsetstrokecolor{textcolor}%
\pgfsetfillcolor{textcolor}%
\pgftext[x=0.149382in,y=0.478936in,left,base]{\color{textcolor}\sffamily\fontsize{10.000000}{12.000000}\selectfont −2.00}%
\end{pgfscope}%
\begin{pgfscope}%
\pgfsetbuttcap%
\pgfsetroundjoin%
\definecolor{currentfill}{rgb}{0.000000,0.000000,0.000000}%
\pgfsetfillcolor{currentfill}%
\pgfsetlinewidth{0.803000pt}%
\definecolor{currentstroke}{rgb}{0.000000,0.000000,0.000000}%
\pgfsetstrokecolor{currentstroke}%
\pgfsetdash{}{0pt}%
\pgfsys@defobject{currentmarker}{\pgfqpoint{-0.048611in}{0.000000in}}{\pgfqpoint{0.000000in}{0.000000in}}{%
\pgfpathmoveto{\pgfqpoint{0.000000in}{0.000000in}}%
\pgfpathlineto{\pgfqpoint{-0.048611in}{0.000000in}}%
\pgfusepath{stroke,fill}%
}%
\begin{pgfscope}%
\pgfsys@transformshift{0.672222in}{0.872436in}%
\pgfsys@useobject{currentmarker}{}%
\end{pgfscope}%
\end{pgfscope}%
\begin{pgfscope}%
\definecolor{textcolor}{rgb}{0.000000,0.000000,0.000000}%
\pgfsetstrokecolor{textcolor}%
\pgfsetfillcolor{textcolor}%
\pgftext[x=0.149382in,y=0.819675in,left,base]{\color{textcolor}\sffamily\fontsize{10.000000}{12.000000}\selectfont −1.75}%
\end{pgfscope}%
\begin{pgfscope}%
\pgfsetbuttcap%
\pgfsetroundjoin%
\definecolor{currentfill}{rgb}{0.000000,0.000000,0.000000}%
\pgfsetfillcolor{currentfill}%
\pgfsetlinewidth{0.803000pt}%
\definecolor{currentstroke}{rgb}{0.000000,0.000000,0.000000}%
\pgfsetstrokecolor{currentstroke}%
\pgfsetdash{}{0pt}%
\pgfsys@defobject{currentmarker}{\pgfqpoint{-0.048611in}{0.000000in}}{\pgfqpoint{0.000000in}{0.000000in}}{%
\pgfpathmoveto{\pgfqpoint{0.000000in}{0.000000in}}%
\pgfpathlineto{\pgfqpoint{-0.048611in}{0.000000in}}%
\pgfusepath{stroke,fill}%
}%
\begin{pgfscope}%
\pgfsys@transformshift{0.672222in}{1.213175in}%
\pgfsys@useobject{currentmarker}{}%
\end{pgfscope}%
\end{pgfscope}%
\begin{pgfscope}%
\definecolor{textcolor}{rgb}{0.000000,0.000000,0.000000}%
\pgfsetstrokecolor{textcolor}%
\pgfsetfillcolor{textcolor}%
\pgftext[x=0.149382in,y=1.160414in,left,base]{\color{textcolor}\sffamily\fontsize{10.000000}{12.000000}\selectfont −1.50}%
\end{pgfscope}%
\begin{pgfscope}%
\pgfsetbuttcap%
\pgfsetroundjoin%
\definecolor{currentfill}{rgb}{0.000000,0.000000,0.000000}%
\pgfsetfillcolor{currentfill}%
\pgfsetlinewidth{0.803000pt}%
\definecolor{currentstroke}{rgb}{0.000000,0.000000,0.000000}%
\pgfsetstrokecolor{currentstroke}%
\pgfsetdash{}{0pt}%
\pgfsys@defobject{currentmarker}{\pgfqpoint{-0.048611in}{0.000000in}}{\pgfqpoint{0.000000in}{0.000000in}}{%
\pgfpathmoveto{\pgfqpoint{0.000000in}{0.000000in}}%
\pgfpathlineto{\pgfqpoint{-0.048611in}{0.000000in}}%
\pgfusepath{stroke,fill}%
}%
\begin{pgfscope}%
\pgfsys@transformshift{0.672222in}{1.553914in}%
\pgfsys@useobject{currentmarker}{}%
\end{pgfscope}%
\end{pgfscope}%
\begin{pgfscope}%
\definecolor{textcolor}{rgb}{0.000000,0.000000,0.000000}%
\pgfsetstrokecolor{textcolor}%
\pgfsetfillcolor{textcolor}%
\pgftext[x=0.149382in,y=1.501152in,left,base]{\color{textcolor}\sffamily\fontsize{10.000000}{12.000000}\selectfont −1.25}%
\end{pgfscope}%
\begin{pgfscope}%
\pgfsetbuttcap%
\pgfsetroundjoin%
\definecolor{currentfill}{rgb}{0.000000,0.000000,0.000000}%
\pgfsetfillcolor{currentfill}%
\pgfsetlinewidth{0.803000pt}%
\definecolor{currentstroke}{rgb}{0.000000,0.000000,0.000000}%
\pgfsetstrokecolor{currentstroke}%
\pgfsetdash{}{0pt}%
\pgfsys@defobject{currentmarker}{\pgfqpoint{-0.048611in}{0.000000in}}{\pgfqpoint{0.000000in}{0.000000in}}{%
\pgfpathmoveto{\pgfqpoint{0.000000in}{0.000000in}}%
\pgfpathlineto{\pgfqpoint{-0.048611in}{0.000000in}}%
\pgfusepath{stroke,fill}%
}%
\begin{pgfscope}%
\pgfsys@transformshift{0.672222in}{1.894653in}%
\pgfsys@useobject{currentmarker}{}%
\end{pgfscope}%
\end{pgfscope}%
\begin{pgfscope}%
\definecolor{textcolor}{rgb}{0.000000,0.000000,0.000000}%
\pgfsetstrokecolor{textcolor}%
\pgfsetfillcolor{textcolor}%
\pgftext[x=0.149382in,y=1.841891in,left,base]{\color{textcolor}\sffamily\fontsize{10.000000}{12.000000}\selectfont −1.00}%
\end{pgfscope}%
\begin{pgfscope}%
\pgfsetbuttcap%
\pgfsetroundjoin%
\definecolor{currentfill}{rgb}{0.000000,0.000000,0.000000}%
\pgfsetfillcolor{currentfill}%
\pgfsetlinewidth{0.803000pt}%
\definecolor{currentstroke}{rgb}{0.000000,0.000000,0.000000}%
\pgfsetstrokecolor{currentstroke}%
\pgfsetdash{}{0pt}%
\pgfsys@defobject{currentmarker}{\pgfqpoint{-0.048611in}{0.000000in}}{\pgfqpoint{0.000000in}{0.000000in}}{%
\pgfpathmoveto{\pgfqpoint{0.000000in}{0.000000in}}%
\pgfpathlineto{\pgfqpoint{-0.048611in}{0.000000in}}%
\pgfusepath{stroke,fill}%
}%
\begin{pgfscope}%
\pgfsys@transformshift{0.672222in}{2.235391in}%
\pgfsys@useobject{currentmarker}{}%
\end{pgfscope}%
\end{pgfscope}%
\begin{pgfscope}%
\definecolor{textcolor}{rgb}{0.000000,0.000000,0.000000}%
\pgfsetstrokecolor{textcolor}%
\pgfsetfillcolor{textcolor}%
\pgftext[x=0.149382in,y=2.182630in,left,base]{\color{textcolor}\sffamily\fontsize{10.000000}{12.000000}\selectfont −0.75}%
\end{pgfscope}%
\begin{pgfscope}%
\pgfsetbuttcap%
\pgfsetroundjoin%
\definecolor{currentfill}{rgb}{0.000000,0.000000,0.000000}%
\pgfsetfillcolor{currentfill}%
\pgfsetlinewidth{0.803000pt}%
\definecolor{currentstroke}{rgb}{0.000000,0.000000,0.000000}%
\pgfsetstrokecolor{currentstroke}%
\pgfsetdash{}{0pt}%
\pgfsys@defobject{currentmarker}{\pgfqpoint{-0.048611in}{0.000000in}}{\pgfqpoint{0.000000in}{0.000000in}}{%
\pgfpathmoveto{\pgfqpoint{0.000000in}{0.000000in}}%
\pgfpathlineto{\pgfqpoint{-0.048611in}{0.000000in}}%
\pgfusepath{stroke,fill}%
}%
\begin{pgfscope}%
\pgfsys@transformshift{0.672222in}{2.576130in}%
\pgfsys@useobject{currentmarker}{}%
\end{pgfscope}%
\end{pgfscope}%
\begin{pgfscope}%
\definecolor{textcolor}{rgb}{0.000000,0.000000,0.000000}%
\pgfsetstrokecolor{textcolor}%
\pgfsetfillcolor{textcolor}%
\pgftext[x=0.149382in,y=2.523369in,left,base]{\color{textcolor}\sffamily\fontsize{10.000000}{12.000000}\selectfont −0.50}%
\end{pgfscope}%
\begin{pgfscope}%
\pgfpathrectangle{\pgfqpoint{0.672222in}{0.387222in}}{\pgfqpoint{3.217778in}{2.324444in}}%
\pgfusepath{clip}%
\pgfsetrectcap%
\pgfsetroundjoin%
\pgfsetlinewidth{1.505625pt}%
\definecolor{currentstroke}{rgb}{0.121569,0.466667,0.705882}%
\pgfsetstrokecolor{currentstroke}%
\pgfsetdash{}{0pt}%
\pgfpathmoveto{\pgfqpoint{0.826514in}{0.553623in}}%
\pgfpathlineto{\pgfqpoint{0.899244in}{0.543721in}}%
\pgfpathlineto{\pgfqpoint{0.971974in}{0.555412in}}%
\pgfpathlineto{\pgfqpoint{1.044704in}{0.556817in}}%
\pgfpathlineto{\pgfqpoint{1.117434in}{0.593028in}}%
\pgfpathlineto{\pgfqpoint{1.190163in}{0.543638in}}%
\pgfpathlineto{\pgfqpoint{1.262893in}{0.539930in}}%
\pgfpathlineto{\pgfqpoint{1.335623in}{0.547275in}}%
\pgfpathlineto{\pgfqpoint{1.408353in}{0.558610in}}%
\pgfpathlineto{\pgfqpoint{1.481083in}{0.575224in}}%
\pgfpathlineto{\pgfqpoint{1.553813in}{0.598308in}}%
\pgfpathlineto{\pgfqpoint{1.626543in}{0.629733in}}%
\pgfpathlineto{\pgfqpoint{1.699272in}{0.670499in}}%
\pgfpathlineto{\pgfqpoint{1.772002in}{0.723601in}}%
\pgfpathlineto{\pgfqpoint{1.844732in}{0.791593in}}%
\pgfpathlineto{\pgfqpoint{1.917462in}{0.879068in}}%
\pgfpathlineto{\pgfqpoint{1.990192in}{0.992274in}}%
\pgfpathlineto{\pgfqpoint{2.062922in}{1.147411in}}%
\pgfpathlineto{\pgfqpoint{2.135651in}{1.411636in}}%
\pgfpathlineto{\pgfqpoint{2.208381in}{1.616415in}}%
\pgfpathlineto{\pgfqpoint{2.281111in}{1.750230in}}%
\pgfpathlineto{\pgfqpoint{2.353841in}{1.856205in}}%
\pgfpathlineto{\pgfqpoint{2.426571in}{1.944353in}}%
\pgfpathlineto{\pgfqpoint{2.499301in}{2.019311in}}%
\pgfpathlineto{\pgfqpoint{2.572030in}{2.085302in}}%
\pgfpathlineto{\pgfqpoint{2.644760in}{2.143727in}}%
\pgfpathlineto{\pgfqpoint{2.717490in}{2.195336in}}%
\pgfpathlineto{\pgfqpoint{2.790220in}{2.241802in}}%
\pgfpathlineto{\pgfqpoint{2.862950in}{2.284049in}}%
\pgfpathlineto{\pgfqpoint{2.935680in}{2.322444in}}%
\pgfpathlineto{\pgfqpoint{3.008410in}{2.358051in}}%
\pgfpathlineto{\pgfqpoint{3.081139in}{2.390261in}}%
\pgfpathlineto{\pgfqpoint{3.153869in}{2.420147in}}%
\pgfpathlineto{\pgfqpoint{3.226599in}{2.448153in}}%
\pgfpathlineto{\pgfqpoint{3.299329in}{2.473929in}}%
\pgfpathlineto{\pgfqpoint{3.372059in}{2.498315in}}%
\pgfpathlineto{\pgfqpoint{3.444789in}{2.520705in}}%
\pgfpathlineto{\pgfqpoint{3.517518in}{2.542090in}}%
\pgfpathlineto{\pgfqpoint{3.590248in}{2.561581in}}%
\pgfpathlineto{\pgfqpoint{3.662978in}{2.580567in}}%
\pgfpathlineto{\pgfqpoint{3.735708in}{2.598212in}}%
\pgfusepath{stroke}%
\end{pgfscope}%
\begin{pgfscope}%
\pgfsetrectcap%
\pgfsetmiterjoin%
\pgfsetlinewidth{0.803000pt}%
\definecolor{currentstroke}{rgb}{0.000000,0.000000,0.000000}%
\pgfsetstrokecolor{currentstroke}%
\pgfsetdash{}{0pt}%
\pgfpathmoveto{\pgfqpoint{0.672222in}{0.387222in}}%
\pgfpathlineto{\pgfqpoint{0.672222in}{2.711667in}}%
\pgfusepath{stroke}%
\end{pgfscope}%
\begin{pgfscope}%
\pgfsetrectcap%
\pgfsetmiterjoin%
\pgfsetlinewidth{0.803000pt}%
\definecolor{currentstroke}{rgb}{0.000000,0.000000,0.000000}%
\pgfsetstrokecolor{currentstroke}%
\pgfsetdash{}{0pt}%
\pgfpathmoveto{\pgfqpoint{3.890000in}{0.387222in}}%
\pgfpathlineto{\pgfqpoint{3.890000in}{2.711667in}}%
\pgfusepath{stroke}%
\end{pgfscope}%
\begin{pgfscope}%
\pgfsetrectcap%
\pgfsetmiterjoin%
\pgfsetlinewidth{0.803000pt}%
\definecolor{currentstroke}{rgb}{0.000000,0.000000,0.000000}%
\pgfsetstrokecolor{currentstroke}%
\pgfsetdash{}{0pt}%
\pgfpathmoveto{\pgfqpoint{0.672222in}{0.387222in}}%
\pgfpathlineto{\pgfqpoint{3.890000in}{0.387222in}}%
\pgfusepath{stroke}%
\end{pgfscope}%
\begin{pgfscope}%
\pgfsetrectcap%
\pgfsetmiterjoin%
\pgfsetlinewidth{0.803000pt}%
\definecolor{currentstroke}{rgb}{0.000000,0.000000,0.000000}%
\pgfsetstrokecolor{currentstroke}%
\pgfsetdash{}{0pt}%
\pgfpathmoveto{\pgfqpoint{0.672222in}{2.711667in}}%
\pgfpathlineto{\pgfqpoint{3.890000in}{2.711667in}}%
\pgfusepath{stroke}%
\end{pgfscope}%
\begin{pgfscope}%
\definecolor{textcolor}{rgb}{0.000000,0.000000,0.000000}%
\pgfsetstrokecolor{textcolor}%
\pgfsetfillcolor{textcolor}%
\pgftext[x=2.281111in,y=2.795000in,,base]{\color{textcolor}\sffamily\fontsize{12.000000}{14.400000}\selectfont \(\displaystyle  N = 64 \)}%
\end{pgfscope}%
\begin{pgfscope}%
\pgfsetbuttcap%
\pgfsetmiterjoin%
\definecolor{currentfill}{rgb}{1.000000,1.000000,1.000000}%
\pgfsetfillcolor{currentfill}%
\pgfsetlinewidth{0.000000pt}%
\definecolor{currentstroke}{rgb}{0.000000,0.000000,0.000000}%
\pgfsetstrokecolor{currentstroke}%
\pgfsetstrokeopacity{0.000000}%
\pgfsetdash{}{0pt}%
\pgfpathmoveto{\pgfqpoint{4.597222in}{0.387222in}}%
\pgfpathlineto{\pgfqpoint{7.815000in}{0.387222in}}%
\pgfpathlineto{\pgfqpoint{7.815000in}{2.711667in}}%
\pgfpathlineto{\pgfqpoint{4.597222in}{2.711667in}}%
\pgfpathclose%
\pgfusepath{fill}%
\end{pgfscope}%
\begin{pgfscope}%
\pgfpathrectangle{\pgfqpoint{4.597222in}{0.387222in}}{\pgfqpoint{3.217778in}{2.324444in}}%
\pgfusepath{clip}%
\pgfsetbuttcap%
\pgfsetroundjoin%
\definecolor{currentfill}{rgb}{0.121569,0.466667,0.705882}%
\pgfsetfillcolor{currentfill}%
\pgfsetfillopacity{0.300000}%
\pgfsetlinewidth{1.003750pt}%
\definecolor{currentstroke}{rgb}{0.121569,0.466667,0.705882}%
\pgfsetstrokecolor{currentstroke}%
\pgfsetstrokeopacity{0.300000}%
\pgfsetdash{}{0pt}%
\pgfpathmoveto{\pgfqpoint{4.751514in}{0.513169in}}%
\pgfpathlineto{\pgfqpoint{4.751514in}{0.513168in}}%
\pgfpathlineto{\pgfqpoint{4.824244in}{0.517799in}}%
\pgfpathlineto{\pgfqpoint{4.896974in}{0.494180in}}%
\pgfpathlineto{\pgfqpoint{4.969704in}{0.492879in}}%
\pgfpathlineto{\pgfqpoint{5.042434in}{0.501709in}}%
\pgfpathlineto{\pgfqpoint{5.115163in}{0.522551in}}%
\pgfpathlineto{\pgfqpoint{5.187893in}{0.498657in}}%
\pgfpathlineto{\pgfqpoint{5.260623in}{0.507293in}}%
\pgfpathlineto{\pgfqpoint{5.333353in}{0.515511in}}%
\pgfpathlineto{\pgfqpoint{5.406083in}{0.530973in}}%
\pgfpathlineto{\pgfqpoint{5.478813in}{0.587246in}}%
\pgfpathlineto{\pgfqpoint{5.551543in}{0.571431in}}%
\pgfpathlineto{\pgfqpoint{5.624272in}{0.635022in}}%
\pgfpathlineto{\pgfqpoint{5.697002in}{0.676800in}}%
\pgfpathlineto{\pgfqpoint{5.769732in}{0.744038in}}%
\pgfpathlineto{\pgfqpoint{5.842462in}{0.831712in}}%
\pgfpathlineto{\pgfqpoint{5.915192in}{0.953530in}}%
\pgfpathlineto{\pgfqpoint{5.987922in}{1.130814in}}%
\pgfpathlineto{\pgfqpoint{6.060651in}{1.400111in}}%
\pgfpathlineto{\pgfqpoint{6.133381in}{1.604100in}}%
\pgfpathlineto{\pgfqpoint{6.206111in}{1.741329in}}%
\pgfpathlineto{\pgfqpoint{6.278841in}{1.848322in}}%
\pgfpathlineto{\pgfqpoint{6.351571in}{1.937725in}}%
\pgfpathlineto{\pgfqpoint{6.424301in}{2.013943in}}%
\pgfpathlineto{\pgfqpoint{6.497030in}{2.080533in}}%
\pgfpathlineto{\pgfqpoint{6.569760in}{2.139275in}}%
\pgfpathlineto{\pgfqpoint{6.642490in}{2.191264in}}%
\pgfpathlineto{\pgfqpoint{6.715220in}{2.238431in}}%
\pgfpathlineto{\pgfqpoint{6.787950in}{2.280932in}}%
\pgfpathlineto{\pgfqpoint{6.860680in}{2.319977in}}%
\pgfpathlineto{\pgfqpoint{6.933410in}{2.355456in}}%
\pgfpathlineto{\pgfqpoint{7.006139in}{2.388030in}}%
\pgfpathlineto{\pgfqpoint{7.078869in}{2.418554in}}%
\pgfpathlineto{\pgfqpoint{7.151599in}{2.446382in}}%
\pgfpathlineto{\pgfqpoint{7.224329in}{2.472718in}}%
\pgfpathlineto{\pgfqpoint{7.297059in}{2.497005in}}%
\pgfpathlineto{\pgfqpoint{7.369789in}{2.519627in}}%
\pgfpathlineto{\pgfqpoint{7.442518in}{2.541152in}}%
\pgfpathlineto{\pgfqpoint{7.515248in}{2.561288in}}%
\pgfpathlineto{\pgfqpoint{7.587978in}{2.580146in}}%
\pgfpathlineto{\pgfqpoint{7.660708in}{2.597949in}}%
\pgfpathlineto{\pgfqpoint{7.660708in}{2.598473in}}%
\pgfpathlineto{\pgfqpoint{7.660708in}{2.598473in}}%
\pgfpathlineto{\pgfqpoint{7.587978in}{2.580662in}}%
\pgfpathlineto{\pgfqpoint{7.515248in}{2.561555in}}%
\pgfpathlineto{\pgfqpoint{7.442518in}{2.541736in}}%
\pgfpathlineto{\pgfqpoint{7.369789in}{2.520334in}}%
\pgfpathlineto{\pgfqpoint{7.297059in}{2.497754in}}%
\pgfpathlineto{\pgfqpoint{7.224329in}{2.472968in}}%
\pgfpathlineto{\pgfqpoint{7.151599in}{2.446877in}}%
\pgfpathlineto{\pgfqpoint{7.078869in}{2.419179in}}%
\pgfpathlineto{\pgfqpoint{7.006139in}{2.388984in}}%
\pgfpathlineto{\pgfqpoint{6.933410in}{2.355968in}}%
\pgfpathlineto{\pgfqpoint{6.860680in}{2.320575in}}%
\pgfpathlineto{\pgfqpoint{6.787950in}{2.281450in}}%
\pgfpathlineto{\pgfqpoint{6.715220in}{2.238942in}}%
\pgfpathlineto{\pgfqpoint{6.642490in}{2.191970in}}%
\pgfpathlineto{\pgfqpoint{6.569760in}{2.140119in}}%
\pgfpathlineto{\pgfqpoint{6.497030in}{2.081627in}}%
\pgfpathlineto{\pgfqpoint{6.424301in}{2.014805in}}%
\pgfpathlineto{\pgfqpoint{6.351571in}{1.939426in}}%
\pgfpathlineto{\pgfqpoint{6.278841in}{1.850874in}}%
\pgfpathlineto{\pgfqpoint{6.206111in}{1.742742in}}%
\pgfpathlineto{\pgfqpoint{6.133381in}{1.609612in}}%
\pgfpathlineto{\pgfqpoint{6.060651in}{1.421011in}}%
\pgfpathlineto{\pgfqpoint{5.987922in}{1.168360in}}%
\pgfpathlineto{\pgfqpoint{5.915192in}{1.049029in}}%
\pgfpathlineto{\pgfqpoint{5.842462in}{0.933075in}}%
\pgfpathlineto{\pgfqpoint{5.769732in}{0.844824in}}%
\pgfpathlineto{\pgfqpoint{5.697002in}{0.773268in}}%
\pgfpathlineto{\pgfqpoint{5.624272in}{0.742774in}}%
\pgfpathlineto{\pgfqpoint{5.551543in}{0.702688in}}%
\pgfpathlineto{\pgfqpoint{5.478813in}{0.673973in}}%
\pgfpathlineto{\pgfqpoint{5.406083in}{0.615394in}}%
\pgfpathlineto{\pgfqpoint{5.333353in}{0.596115in}}%
\pgfpathlineto{\pgfqpoint{5.260623in}{0.626105in}}%
\pgfpathlineto{\pgfqpoint{5.187893in}{0.572755in}}%
\pgfpathlineto{\pgfqpoint{5.115163in}{0.593611in}}%
\pgfpathlineto{\pgfqpoint{5.042434in}{0.579876in}}%
\pgfpathlineto{\pgfqpoint{4.969704in}{0.602917in}}%
\pgfpathlineto{\pgfqpoint{4.896974in}{0.556315in}}%
\pgfpathlineto{\pgfqpoint{4.824244in}{0.577421in}}%
\pgfpathlineto{\pgfqpoint{4.751514in}{0.513169in}}%
\pgfpathclose%
\pgfusepath{stroke,fill}%
\end{pgfscope}%
\begin{pgfscope}%
\pgfpathrectangle{\pgfqpoint{4.597222in}{0.387222in}}{\pgfqpoint{3.217778in}{2.324444in}}%
\pgfusepath{clip}%
\pgfsetbuttcap%
\pgfsetroundjoin%
\definecolor{currentfill}{rgb}{0.121569,0.466667,0.705882}%
\pgfsetfillcolor{currentfill}%
\pgfsetlinewidth{1.003750pt}%
\definecolor{currentstroke}{rgb}{0.121569,0.466667,0.705882}%
\pgfsetstrokecolor{currentstroke}%
\pgfsetdash{}{0pt}%
\pgfsys@defobject{currentmarker}{\pgfqpoint{-0.009821in}{-0.009821in}}{\pgfqpoint{0.009821in}{0.009821in}}{%
\pgfpathmoveto{\pgfqpoint{0.000000in}{-0.009821in}}%
\pgfpathcurveto{\pgfqpoint{0.002605in}{-0.009821in}}{\pgfqpoint{0.005103in}{-0.008786in}}{\pgfqpoint{0.006944in}{-0.006944in}}%
\pgfpathcurveto{\pgfqpoint{0.008786in}{-0.005103in}}{\pgfqpoint{0.009821in}{-0.002605in}}{\pgfqpoint{0.009821in}{0.000000in}}%
\pgfpathcurveto{\pgfqpoint{0.009821in}{0.002605in}}{\pgfqpoint{0.008786in}{0.005103in}}{\pgfqpoint{0.006944in}{0.006944in}}%
\pgfpathcurveto{\pgfqpoint{0.005103in}{0.008786in}}{\pgfqpoint{0.002605in}{0.009821in}}{\pgfqpoint{0.000000in}{0.009821in}}%
\pgfpathcurveto{\pgfqpoint{-0.002605in}{0.009821in}}{\pgfqpoint{-0.005103in}{0.008786in}}{\pgfqpoint{-0.006944in}{0.006944in}}%
\pgfpathcurveto{\pgfqpoint{-0.008786in}{0.005103in}}{\pgfqpoint{-0.009821in}{0.002605in}}{\pgfqpoint{-0.009821in}{0.000000in}}%
\pgfpathcurveto{\pgfqpoint{-0.009821in}{-0.002605in}}{\pgfqpoint{-0.008786in}{-0.005103in}}{\pgfqpoint{-0.006944in}{-0.006944in}}%
\pgfpathcurveto{\pgfqpoint{-0.005103in}{-0.008786in}}{\pgfqpoint{-0.002605in}{-0.009821in}}{\pgfqpoint{0.000000in}{-0.009821in}}%
\pgfpathclose%
\pgfusepath{stroke,fill}%
}%
\begin{pgfscope}%
\pgfsys@transformshift{4.751514in}{0.513168in}%
\pgfsys@useobject{currentmarker}{}%
\end{pgfscope}%
\begin{pgfscope}%
\pgfsys@transformshift{4.824244in}{0.547610in}%
\pgfsys@useobject{currentmarker}{}%
\end{pgfscope}%
\begin{pgfscope}%
\pgfsys@transformshift{4.896974in}{0.525248in}%
\pgfsys@useobject{currentmarker}{}%
\end{pgfscope}%
\begin{pgfscope}%
\pgfsys@transformshift{4.969704in}{0.547898in}%
\pgfsys@useobject{currentmarker}{}%
\end{pgfscope}%
\begin{pgfscope}%
\pgfsys@transformshift{5.042434in}{0.540792in}%
\pgfsys@useobject{currentmarker}{}%
\end{pgfscope}%
\begin{pgfscope}%
\pgfsys@transformshift{5.115163in}{0.558081in}%
\pgfsys@useobject{currentmarker}{}%
\end{pgfscope}%
\begin{pgfscope}%
\pgfsys@transformshift{5.187893in}{0.535706in}%
\pgfsys@useobject{currentmarker}{}%
\end{pgfscope}%
\begin{pgfscope}%
\pgfsys@transformshift{5.260623in}{0.566699in}%
\pgfsys@useobject{currentmarker}{}%
\end{pgfscope}%
\begin{pgfscope}%
\pgfsys@transformshift{5.333353in}{0.555813in}%
\pgfsys@useobject{currentmarker}{}%
\end{pgfscope}%
\begin{pgfscope}%
\pgfsys@transformshift{5.406083in}{0.573184in}%
\pgfsys@useobject{currentmarker}{}%
\end{pgfscope}%
\begin{pgfscope}%
\pgfsys@transformshift{5.478813in}{0.630610in}%
\pgfsys@useobject{currentmarker}{}%
\end{pgfscope}%
\begin{pgfscope}%
\pgfsys@transformshift{5.551543in}{0.637059in}%
\pgfsys@useobject{currentmarker}{}%
\end{pgfscope}%
\begin{pgfscope}%
\pgfsys@transformshift{5.624272in}{0.688898in}%
\pgfsys@useobject{currentmarker}{}%
\end{pgfscope}%
\begin{pgfscope}%
\pgfsys@transformshift{5.697002in}{0.725034in}%
\pgfsys@useobject{currentmarker}{}%
\end{pgfscope}%
\begin{pgfscope}%
\pgfsys@transformshift{5.769732in}{0.794431in}%
\pgfsys@useobject{currentmarker}{}%
\end{pgfscope}%
\begin{pgfscope}%
\pgfsys@transformshift{5.842462in}{0.882394in}%
\pgfsys@useobject{currentmarker}{}%
\end{pgfscope}%
\begin{pgfscope}%
\pgfsys@transformshift{5.915192in}{1.001279in}%
\pgfsys@useobject{currentmarker}{}%
\end{pgfscope}%
\begin{pgfscope}%
\pgfsys@transformshift{5.987922in}{1.149587in}%
\pgfsys@useobject{currentmarker}{}%
\end{pgfscope}%
\begin{pgfscope}%
\pgfsys@transformshift{6.060651in}{1.410561in}%
\pgfsys@useobject{currentmarker}{}%
\end{pgfscope}%
\begin{pgfscope}%
\pgfsys@transformshift{6.133381in}{1.606856in}%
\pgfsys@useobject{currentmarker}{}%
\end{pgfscope}%
\begin{pgfscope}%
\pgfsys@transformshift{6.206111in}{1.742035in}%
\pgfsys@useobject{currentmarker}{}%
\end{pgfscope}%
\begin{pgfscope}%
\pgfsys@transformshift{6.278841in}{1.849598in}%
\pgfsys@useobject{currentmarker}{}%
\end{pgfscope}%
\begin{pgfscope}%
\pgfsys@transformshift{6.351571in}{1.938576in}%
\pgfsys@useobject{currentmarker}{}%
\end{pgfscope}%
\begin{pgfscope}%
\pgfsys@transformshift{6.424301in}{2.014374in}%
\pgfsys@useobject{currentmarker}{}%
\end{pgfscope}%
\begin{pgfscope}%
\pgfsys@transformshift{6.497030in}{2.081080in}%
\pgfsys@useobject{currentmarker}{}%
\end{pgfscope}%
\begin{pgfscope}%
\pgfsys@transformshift{6.569760in}{2.139697in}%
\pgfsys@useobject{currentmarker}{}%
\end{pgfscope}%
\begin{pgfscope}%
\pgfsys@transformshift{6.642490in}{2.191617in}%
\pgfsys@useobject{currentmarker}{}%
\end{pgfscope}%
\begin{pgfscope}%
\pgfsys@transformshift{6.715220in}{2.238687in}%
\pgfsys@useobject{currentmarker}{}%
\end{pgfscope}%
\begin{pgfscope}%
\pgfsys@transformshift{6.787950in}{2.281191in}%
\pgfsys@useobject{currentmarker}{}%
\end{pgfscope}%
\begin{pgfscope}%
\pgfsys@transformshift{6.860680in}{2.320276in}%
\pgfsys@useobject{currentmarker}{}%
\end{pgfscope}%
\begin{pgfscope}%
\pgfsys@transformshift{6.933410in}{2.355712in}%
\pgfsys@useobject{currentmarker}{}%
\end{pgfscope}%
\begin{pgfscope}%
\pgfsys@transformshift{7.006139in}{2.388507in}%
\pgfsys@useobject{currentmarker}{}%
\end{pgfscope}%
\begin{pgfscope}%
\pgfsys@transformshift{7.078869in}{2.418867in}%
\pgfsys@useobject{currentmarker}{}%
\end{pgfscope}%
\begin{pgfscope}%
\pgfsys@transformshift{7.151599in}{2.446630in}%
\pgfsys@useobject{currentmarker}{}%
\end{pgfscope}%
\begin{pgfscope}%
\pgfsys@transformshift{7.224329in}{2.472843in}%
\pgfsys@useobject{currentmarker}{}%
\end{pgfscope}%
\begin{pgfscope}%
\pgfsys@transformshift{7.297059in}{2.497379in}%
\pgfsys@useobject{currentmarker}{}%
\end{pgfscope}%
\begin{pgfscope}%
\pgfsys@transformshift{7.369789in}{2.519981in}%
\pgfsys@useobject{currentmarker}{}%
\end{pgfscope}%
\begin{pgfscope}%
\pgfsys@transformshift{7.442518in}{2.541444in}%
\pgfsys@useobject{currentmarker}{}%
\end{pgfscope}%
\begin{pgfscope}%
\pgfsys@transformshift{7.515248in}{2.561422in}%
\pgfsys@useobject{currentmarker}{}%
\end{pgfscope}%
\begin{pgfscope}%
\pgfsys@transformshift{7.587978in}{2.580404in}%
\pgfsys@useobject{currentmarker}{}%
\end{pgfscope}%
\begin{pgfscope}%
\pgfsys@transformshift{7.660708in}{2.598211in}%
\pgfsys@useobject{currentmarker}{}%
\end{pgfscope}%
\end{pgfscope}%
\begin{pgfscope}%
\pgfsetbuttcap%
\pgfsetroundjoin%
\definecolor{currentfill}{rgb}{0.000000,0.000000,0.000000}%
\pgfsetfillcolor{currentfill}%
\pgfsetlinewidth{0.803000pt}%
\definecolor{currentstroke}{rgb}{0.000000,0.000000,0.000000}%
\pgfsetstrokecolor{currentstroke}%
\pgfsetdash{}{0pt}%
\pgfsys@defobject{currentmarker}{\pgfqpoint{0.000000in}{-0.048611in}}{\pgfqpoint{0.000000in}{0.000000in}}{%
\pgfpathmoveto{\pgfqpoint{0.000000in}{0.000000in}}%
\pgfpathlineto{\pgfqpoint{0.000000in}{-0.048611in}}%
\pgfusepath{stroke,fill}%
}%
\begin{pgfscope}%
\pgfsys@transformshift{5.115163in}{0.387222in}%
\pgfsys@useobject{currentmarker}{}%
\end{pgfscope}%
\end{pgfscope}%
\begin{pgfscope}%
\definecolor{textcolor}{rgb}{0.000000,0.000000,0.000000}%
\pgfsetstrokecolor{textcolor}%
\pgfsetfillcolor{textcolor}%
\pgftext[x=5.115163in,y=0.290000in,,top]{\color{textcolor}\sffamily\fontsize{10.000000}{12.000000}\selectfont 1}%
\end{pgfscope}%
\begin{pgfscope}%
\pgfsetbuttcap%
\pgfsetroundjoin%
\definecolor{currentfill}{rgb}{0.000000,0.000000,0.000000}%
\pgfsetfillcolor{currentfill}%
\pgfsetlinewidth{0.803000pt}%
\definecolor{currentstroke}{rgb}{0.000000,0.000000,0.000000}%
\pgfsetstrokecolor{currentstroke}%
\pgfsetdash{}{0pt}%
\pgfsys@defobject{currentmarker}{\pgfqpoint{0.000000in}{-0.048611in}}{\pgfqpoint{0.000000in}{0.000000in}}{%
\pgfpathmoveto{\pgfqpoint{0.000000in}{0.000000in}}%
\pgfpathlineto{\pgfqpoint{0.000000in}{-0.048611in}}%
\pgfusepath{stroke,fill}%
}%
\begin{pgfscope}%
\pgfsys@transformshift{5.842462in}{0.387222in}%
\pgfsys@useobject{currentmarker}{}%
\end{pgfscope}%
\end{pgfscope}%
\begin{pgfscope}%
\definecolor{textcolor}{rgb}{0.000000,0.000000,0.000000}%
\pgfsetstrokecolor{textcolor}%
\pgfsetfillcolor{textcolor}%
\pgftext[x=5.842462in,y=0.290000in,,top]{\color{textcolor}\sffamily\fontsize{10.000000}{12.000000}\selectfont 2}%
\end{pgfscope}%
\begin{pgfscope}%
\pgfsetbuttcap%
\pgfsetroundjoin%
\definecolor{currentfill}{rgb}{0.000000,0.000000,0.000000}%
\pgfsetfillcolor{currentfill}%
\pgfsetlinewidth{0.803000pt}%
\definecolor{currentstroke}{rgb}{0.000000,0.000000,0.000000}%
\pgfsetstrokecolor{currentstroke}%
\pgfsetdash{}{0pt}%
\pgfsys@defobject{currentmarker}{\pgfqpoint{0.000000in}{-0.048611in}}{\pgfqpoint{0.000000in}{0.000000in}}{%
\pgfpathmoveto{\pgfqpoint{0.000000in}{0.000000in}}%
\pgfpathlineto{\pgfqpoint{0.000000in}{-0.048611in}}%
\pgfusepath{stroke,fill}%
}%
\begin{pgfscope}%
\pgfsys@transformshift{6.569760in}{0.387222in}%
\pgfsys@useobject{currentmarker}{}%
\end{pgfscope}%
\end{pgfscope}%
\begin{pgfscope}%
\definecolor{textcolor}{rgb}{0.000000,0.000000,0.000000}%
\pgfsetstrokecolor{textcolor}%
\pgfsetfillcolor{textcolor}%
\pgftext[x=6.569760in,y=0.290000in,,top]{\color{textcolor}\sffamily\fontsize{10.000000}{12.000000}\selectfont 3}%
\end{pgfscope}%
\begin{pgfscope}%
\pgfsetbuttcap%
\pgfsetroundjoin%
\definecolor{currentfill}{rgb}{0.000000,0.000000,0.000000}%
\pgfsetfillcolor{currentfill}%
\pgfsetlinewidth{0.803000pt}%
\definecolor{currentstroke}{rgb}{0.000000,0.000000,0.000000}%
\pgfsetstrokecolor{currentstroke}%
\pgfsetdash{}{0pt}%
\pgfsys@defobject{currentmarker}{\pgfqpoint{0.000000in}{-0.048611in}}{\pgfqpoint{0.000000in}{0.000000in}}{%
\pgfpathmoveto{\pgfqpoint{0.000000in}{0.000000in}}%
\pgfpathlineto{\pgfqpoint{0.000000in}{-0.048611in}}%
\pgfusepath{stroke,fill}%
}%
\begin{pgfscope}%
\pgfsys@transformshift{7.297059in}{0.387222in}%
\pgfsys@useobject{currentmarker}{}%
\end{pgfscope}%
\end{pgfscope}%
\begin{pgfscope}%
\definecolor{textcolor}{rgb}{0.000000,0.000000,0.000000}%
\pgfsetstrokecolor{textcolor}%
\pgfsetfillcolor{textcolor}%
\pgftext[x=7.297059in,y=0.290000in,,top]{\color{textcolor}\sffamily\fontsize{10.000000}{12.000000}\selectfont 4}%
\end{pgfscope}%
\begin{pgfscope}%
\pgfsetbuttcap%
\pgfsetroundjoin%
\definecolor{currentfill}{rgb}{0.000000,0.000000,0.000000}%
\pgfsetfillcolor{currentfill}%
\pgfsetlinewidth{0.803000pt}%
\definecolor{currentstroke}{rgb}{0.000000,0.000000,0.000000}%
\pgfsetstrokecolor{currentstroke}%
\pgfsetdash{}{0pt}%
\pgfsys@defobject{currentmarker}{\pgfqpoint{-0.048611in}{0.000000in}}{\pgfqpoint{0.000000in}{0.000000in}}{%
\pgfpathmoveto{\pgfqpoint{0.000000in}{0.000000in}}%
\pgfpathlineto{\pgfqpoint{-0.048611in}{0.000000in}}%
\pgfusepath{stroke,fill}%
}%
\begin{pgfscope}%
\pgfsys@transformshift{4.597222in}{0.513167in}%
\pgfsys@useobject{currentmarker}{}%
\end{pgfscope}%
\end{pgfscope}%
\begin{pgfscope}%
\definecolor{textcolor}{rgb}{0.000000,0.000000,0.000000}%
\pgfsetstrokecolor{textcolor}%
\pgfsetfillcolor{textcolor}%
\pgftext[x=4.074382in,y=0.460405in,left,base]{\color{textcolor}\sffamily\fontsize{10.000000}{12.000000}\selectfont −2.00}%
\end{pgfscope}%
\begin{pgfscope}%
\pgfsetbuttcap%
\pgfsetroundjoin%
\definecolor{currentfill}{rgb}{0.000000,0.000000,0.000000}%
\pgfsetfillcolor{currentfill}%
\pgfsetlinewidth{0.803000pt}%
\definecolor{currentstroke}{rgb}{0.000000,0.000000,0.000000}%
\pgfsetstrokecolor{currentstroke}%
\pgfsetdash{}{0pt}%
\pgfsys@defobject{currentmarker}{\pgfqpoint{-0.048611in}{0.000000in}}{\pgfqpoint{0.000000in}{0.000000in}}{%
\pgfpathmoveto{\pgfqpoint{0.000000in}{0.000000in}}%
\pgfpathlineto{\pgfqpoint{-0.048611in}{0.000000in}}%
\pgfusepath{stroke,fill}%
}%
\begin{pgfscope}%
\pgfsys@transformshift{4.597222in}{0.856969in}%
\pgfsys@useobject{currentmarker}{}%
\end{pgfscope}%
\end{pgfscope}%
\begin{pgfscope}%
\definecolor{textcolor}{rgb}{0.000000,0.000000,0.000000}%
\pgfsetstrokecolor{textcolor}%
\pgfsetfillcolor{textcolor}%
\pgftext[x=4.074382in,y=0.804208in,left,base]{\color{textcolor}\sffamily\fontsize{10.000000}{12.000000}\selectfont −1.75}%
\end{pgfscope}%
\begin{pgfscope}%
\pgfsetbuttcap%
\pgfsetroundjoin%
\definecolor{currentfill}{rgb}{0.000000,0.000000,0.000000}%
\pgfsetfillcolor{currentfill}%
\pgfsetlinewidth{0.803000pt}%
\definecolor{currentstroke}{rgb}{0.000000,0.000000,0.000000}%
\pgfsetstrokecolor{currentstroke}%
\pgfsetdash{}{0pt}%
\pgfsys@defobject{currentmarker}{\pgfqpoint{-0.048611in}{0.000000in}}{\pgfqpoint{0.000000in}{0.000000in}}{%
\pgfpathmoveto{\pgfqpoint{0.000000in}{0.000000in}}%
\pgfpathlineto{\pgfqpoint{-0.048611in}{0.000000in}}%
\pgfusepath{stroke,fill}%
}%
\begin{pgfscope}%
\pgfsys@transformshift{4.597222in}{1.200772in}%
\pgfsys@useobject{currentmarker}{}%
\end{pgfscope}%
\end{pgfscope}%
\begin{pgfscope}%
\definecolor{textcolor}{rgb}{0.000000,0.000000,0.000000}%
\pgfsetstrokecolor{textcolor}%
\pgfsetfillcolor{textcolor}%
\pgftext[x=4.074382in,y=1.148010in,left,base]{\color{textcolor}\sffamily\fontsize{10.000000}{12.000000}\selectfont −1.50}%
\end{pgfscope}%
\begin{pgfscope}%
\pgfsetbuttcap%
\pgfsetroundjoin%
\definecolor{currentfill}{rgb}{0.000000,0.000000,0.000000}%
\pgfsetfillcolor{currentfill}%
\pgfsetlinewidth{0.803000pt}%
\definecolor{currentstroke}{rgb}{0.000000,0.000000,0.000000}%
\pgfsetstrokecolor{currentstroke}%
\pgfsetdash{}{0pt}%
\pgfsys@defobject{currentmarker}{\pgfqpoint{-0.048611in}{0.000000in}}{\pgfqpoint{0.000000in}{0.000000in}}{%
\pgfpathmoveto{\pgfqpoint{0.000000in}{0.000000in}}%
\pgfpathlineto{\pgfqpoint{-0.048611in}{0.000000in}}%
\pgfusepath{stroke,fill}%
}%
\begin{pgfscope}%
\pgfsys@transformshift{4.597222in}{1.544574in}%
\pgfsys@useobject{currentmarker}{}%
\end{pgfscope}%
\end{pgfscope}%
\begin{pgfscope}%
\definecolor{textcolor}{rgb}{0.000000,0.000000,0.000000}%
\pgfsetstrokecolor{textcolor}%
\pgfsetfillcolor{textcolor}%
\pgftext[x=4.074382in,y=1.491812in,left,base]{\color{textcolor}\sffamily\fontsize{10.000000}{12.000000}\selectfont −1.25}%
\end{pgfscope}%
\begin{pgfscope}%
\pgfsetbuttcap%
\pgfsetroundjoin%
\definecolor{currentfill}{rgb}{0.000000,0.000000,0.000000}%
\pgfsetfillcolor{currentfill}%
\pgfsetlinewidth{0.803000pt}%
\definecolor{currentstroke}{rgb}{0.000000,0.000000,0.000000}%
\pgfsetstrokecolor{currentstroke}%
\pgfsetdash{}{0pt}%
\pgfsys@defobject{currentmarker}{\pgfqpoint{-0.048611in}{0.000000in}}{\pgfqpoint{0.000000in}{0.000000in}}{%
\pgfpathmoveto{\pgfqpoint{0.000000in}{0.000000in}}%
\pgfpathlineto{\pgfqpoint{-0.048611in}{0.000000in}}%
\pgfusepath{stroke,fill}%
}%
\begin{pgfscope}%
\pgfsys@transformshift{4.597222in}{1.888376in}%
\pgfsys@useobject{currentmarker}{}%
\end{pgfscope}%
\end{pgfscope}%
\begin{pgfscope}%
\definecolor{textcolor}{rgb}{0.000000,0.000000,0.000000}%
\pgfsetstrokecolor{textcolor}%
\pgfsetfillcolor{textcolor}%
\pgftext[x=4.074382in,y=1.835615in,left,base]{\color{textcolor}\sffamily\fontsize{10.000000}{12.000000}\selectfont −1.00}%
\end{pgfscope}%
\begin{pgfscope}%
\pgfsetbuttcap%
\pgfsetroundjoin%
\definecolor{currentfill}{rgb}{0.000000,0.000000,0.000000}%
\pgfsetfillcolor{currentfill}%
\pgfsetlinewidth{0.803000pt}%
\definecolor{currentstroke}{rgb}{0.000000,0.000000,0.000000}%
\pgfsetstrokecolor{currentstroke}%
\pgfsetdash{}{0pt}%
\pgfsys@defobject{currentmarker}{\pgfqpoint{-0.048611in}{0.000000in}}{\pgfqpoint{0.000000in}{0.000000in}}{%
\pgfpathmoveto{\pgfqpoint{0.000000in}{0.000000in}}%
\pgfpathlineto{\pgfqpoint{-0.048611in}{0.000000in}}%
\pgfusepath{stroke,fill}%
}%
\begin{pgfscope}%
\pgfsys@transformshift{4.597222in}{2.232178in}%
\pgfsys@useobject{currentmarker}{}%
\end{pgfscope}%
\end{pgfscope}%
\begin{pgfscope}%
\definecolor{textcolor}{rgb}{0.000000,0.000000,0.000000}%
\pgfsetstrokecolor{textcolor}%
\pgfsetfillcolor{textcolor}%
\pgftext[x=4.074382in,y=2.179417in,left,base]{\color{textcolor}\sffamily\fontsize{10.000000}{12.000000}\selectfont −0.75}%
\end{pgfscope}%
\begin{pgfscope}%
\pgfsetbuttcap%
\pgfsetroundjoin%
\definecolor{currentfill}{rgb}{0.000000,0.000000,0.000000}%
\pgfsetfillcolor{currentfill}%
\pgfsetlinewidth{0.803000pt}%
\definecolor{currentstroke}{rgb}{0.000000,0.000000,0.000000}%
\pgfsetstrokecolor{currentstroke}%
\pgfsetdash{}{0pt}%
\pgfsys@defobject{currentmarker}{\pgfqpoint{-0.048611in}{0.000000in}}{\pgfqpoint{0.000000in}{0.000000in}}{%
\pgfpathmoveto{\pgfqpoint{0.000000in}{0.000000in}}%
\pgfpathlineto{\pgfqpoint{-0.048611in}{0.000000in}}%
\pgfusepath{stroke,fill}%
}%
\begin{pgfscope}%
\pgfsys@transformshift{4.597222in}{2.575981in}%
\pgfsys@useobject{currentmarker}{}%
\end{pgfscope}%
\end{pgfscope}%
\begin{pgfscope}%
\definecolor{textcolor}{rgb}{0.000000,0.000000,0.000000}%
\pgfsetstrokecolor{textcolor}%
\pgfsetfillcolor{textcolor}%
\pgftext[x=4.074382in,y=2.523219in,left,base]{\color{textcolor}\sffamily\fontsize{10.000000}{12.000000}\selectfont −0.50}%
\end{pgfscope}%
\begin{pgfscope}%
\pgfpathrectangle{\pgfqpoint{4.597222in}{0.387222in}}{\pgfqpoint{3.217778in}{2.324444in}}%
\pgfusepath{clip}%
\pgfsetrectcap%
\pgfsetroundjoin%
\pgfsetlinewidth{1.505625pt}%
\definecolor{currentstroke}{rgb}{0.121569,0.466667,0.705882}%
\pgfsetstrokecolor{currentstroke}%
\pgfsetdash{}{0pt}%
\pgfpathmoveto{\pgfqpoint{4.751514in}{0.513168in}}%
\pgfpathlineto{\pgfqpoint{4.824244in}{0.547610in}}%
\pgfpathlineto{\pgfqpoint{4.896974in}{0.525248in}}%
\pgfpathlineto{\pgfqpoint{4.969704in}{0.547898in}}%
\pgfpathlineto{\pgfqpoint{5.042434in}{0.540792in}}%
\pgfpathlineto{\pgfqpoint{5.115163in}{0.558081in}}%
\pgfpathlineto{\pgfqpoint{5.187893in}{0.535706in}}%
\pgfpathlineto{\pgfqpoint{5.260623in}{0.566699in}}%
\pgfpathlineto{\pgfqpoint{5.333353in}{0.555813in}}%
\pgfpathlineto{\pgfqpoint{5.406083in}{0.573184in}}%
\pgfpathlineto{\pgfqpoint{5.478813in}{0.630610in}}%
\pgfpathlineto{\pgfqpoint{5.551543in}{0.637059in}}%
\pgfpathlineto{\pgfqpoint{5.624272in}{0.688898in}}%
\pgfpathlineto{\pgfqpoint{5.697002in}{0.725034in}}%
\pgfpathlineto{\pgfqpoint{5.769732in}{0.794431in}}%
\pgfpathlineto{\pgfqpoint{5.842462in}{0.882394in}}%
\pgfpathlineto{\pgfqpoint{5.915192in}{1.001279in}}%
\pgfpathlineto{\pgfqpoint{5.987922in}{1.149587in}}%
\pgfpathlineto{\pgfqpoint{6.060651in}{1.410561in}}%
\pgfpathlineto{\pgfqpoint{6.133381in}{1.606856in}}%
\pgfpathlineto{\pgfqpoint{6.206111in}{1.742035in}}%
\pgfpathlineto{\pgfqpoint{6.278841in}{1.849598in}}%
\pgfpathlineto{\pgfqpoint{6.351571in}{1.938576in}}%
\pgfpathlineto{\pgfqpoint{6.424301in}{2.014374in}}%
\pgfpathlineto{\pgfqpoint{6.497030in}{2.081080in}}%
\pgfpathlineto{\pgfqpoint{6.569760in}{2.139697in}}%
\pgfpathlineto{\pgfqpoint{6.642490in}{2.191617in}}%
\pgfpathlineto{\pgfqpoint{6.715220in}{2.238687in}}%
\pgfpathlineto{\pgfqpoint{6.787950in}{2.281191in}}%
\pgfpathlineto{\pgfqpoint{6.860680in}{2.320276in}}%
\pgfpathlineto{\pgfqpoint{6.933410in}{2.355712in}}%
\pgfpathlineto{\pgfqpoint{7.006139in}{2.388507in}}%
\pgfpathlineto{\pgfqpoint{7.078869in}{2.418867in}}%
\pgfpathlineto{\pgfqpoint{7.151599in}{2.446630in}}%
\pgfpathlineto{\pgfqpoint{7.224329in}{2.472843in}}%
\pgfpathlineto{\pgfqpoint{7.297059in}{2.497379in}}%
\pgfpathlineto{\pgfqpoint{7.369789in}{2.519981in}}%
\pgfpathlineto{\pgfqpoint{7.442518in}{2.541444in}}%
\pgfpathlineto{\pgfqpoint{7.515248in}{2.561422in}}%
\pgfpathlineto{\pgfqpoint{7.587978in}{2.580404in}}%
\pgfpathlineto{\pgfqpoint{7.660708in}{2.598211in}}%
\pgfusepath{stroke}%
\end{pgfscope}%
\begin{pgfscope}%
\pgfsetrectcap%
\pgfsetmiterjoin%
\pgfsetlinewidth{0.803000pt}%
\definecolor{currentstroke}{rgb}{0.000000,0.000000,0.000000}%
\pgfsetstrokecolor{currentstroke}%
\pgfsetdash{}{0pt}%
\pgfpathmoveto{\pgfqpoint{4.597222in}{0.387222in}}%
\pgfpathlineto{\pgfqpoint{4.597222in}{2.711667in}}%
\pgfusepath{stroke}%
\end{pgfscope}%
\begin{pgfscope}%
\pgfsetrectcap%
\pgfsetmiterjoin%
\pgfsetlinewidth{0.803000pt}%
\definecolor{currentstroke}{rgb}{0.000000,0.000000,0.000000}%
\pgfsetstrokecolor{currentstroke}%
\pgfsetdash{}{0pt}%
\pgfpathmoveto{\pgfqpoint{7.815000in}{0.387222in}}%
\pgfpathlineto{\pgfqpoint{7.815000in}{2.711667in}}%
\pgfusepath{stroke}%
\end{pgfscope}%
\begin{pgfscope}%
\pgfsetrectcap%
\pgfsetmiterjoin%
\pgfsetlinewidth{0.803000pt}%
\definecolor{currentstroke}{rgb}{0.000000,0.000000,0.000000}%
\pgfsetstrokecolor{currentstroke}%
\pgfsetdash{}{0pt}%
\pgfpathmoveto{\pgfqpoint{4.597222in}{0.387222in}}%
\pgfpathlineto{\pgfqpoint{7.815000in}{0.387222in}}%
\pgfusepath{stroke}%
\end{pgfscope}%
\begin{pgfscope}%
\pgfsetrectcap%
\pgfsetmiterjoin%
\pgfsetlinewidth{0.803000pt}%
\definecolor{currentstroke}{rgb}{0.000000,0.000000,0.000000}%
\pgfsetstrokecolor{currentstroke}%
\pgfsetdash{}{0pt}%
\pgfpathmoveto{\pgfqpoint{4.597222in}{2.711667in}}%
\pgfpathlineto{\pgfqpoint{7.815000in}{2.711667in}}%
\pgfusepath{stroke}%
\end{pgfscope}%
\begin{pgfscope}%
\definecolor{textcolor}{rgb}{0.000000,0.000000,0.000000}%
\pgfsetstrokecolor{textcolor}%
\pgfsetfillcolor{textcolor}%
\pgftext[x=6.206111in,y=2.795000in,,base]{\color{textcolor}\sffamily\fontsize{12.000000}{14.400000}\selectfont \(\displaystyle  N = 128 \)}%
\end{pgfscope}%
\end{pgfpicture}%
\makeatother%
\endgroup%
}~
\scalebox{0.75}{%% Creator: Matplotlib, PGF backend
%%
%% To include the figure in your LaTeX document, write
%%   \input{<filename>.pgf}
%%
%% Make sure the required packages are loaded in your preamble
%%   \usepackage{pgf}
%%
%% Figures using additional raster images can only be included by \input if
%% they are in the same directory as the main LaTeX file. For loading figures
%% from other directories you can use the `import` package
%%   \usepackage{import}
%% and then include the figures with
%%   \import{<path to file>}{<filename>.pgf}
%%
%% Matplotlib used the following preamble
%%   \usepackage{fontspec}
%%   \setmainfont{DejaVuSerif.ttf}[Path=/home/lzh/anaconda3/envs/numana/lib/python3.7/site-packages/matplotlib/mpl-data/fonts/ttf/]
%%   \setsansfont{DejaVuSans.ttf}[Path=/home/lzh/anaconda3/envs/numana/lib/python3.7/site-packages/matplotlib/mpl-data/fonts/ttf/]
%%   \setmonofont{DejaVuSansMono.ttf}[Path=/home/lzh/anaconda3/envs/numana/lib/python3.7/site-packages/matplotlib/mpl-data/fonts/ttf/]
%%
\begingroup%
\makeatletter%
\begin{pgfpicture}%
\pgfpathrectangle{\pgfpointorigin}{\pgfqpoint{4.000000in}{4.000000in}}%
\pgfusepath{use as bounding box, clip}%
\begin{pgfscope}%
\pgfsetbuttcap%
\pgfsetmiterjoin%
\definecolor{currentfill}{rgb}{1.000000,1.000000,1.000000}%
\pgfsetfillcolor{currentfill}%
\pgfsetlinewidth{0.000000pt}%
\definecolor{currentstroke}{rgb}{1.000000,1.000000,1.000000}%
\pgfsetstrokecolor{currentstroke}%
\pgfsetdash{}{0pt}%
\pgfpathmoveto{\pgfqpoint{0.000000in}{0.000000in}}%
\pgfpathlineto{\pgfqpoint{4.000000in}{0.000000in}}%
\pgfpathlineto{\pgfqpoint{4.000000in}{4.000000in}}%
\pgfpathlineto{\pgfqpoint{0.000000in}{4.000000in}}%
\pgfpathclose%
\pgfusepath{fill}%
\end{pgfscope}%
\begin{pgfscope}%
\pgfsetbuttcap%
\pgfsetmiterjoin%
\definecolor{currentfill}{rgb}{1.000000,1.000000,1.000000}%
\pgfsetfillcolor{currentfill}%
\pgfsetlinewidth{0.000000pt}%
\definecolor{currentstroke}{rgb}{0.000000,0.000000,0.000000}%
\pgfsetstrokecolor{currentstroke}%
\pgfsetstrokeopacity{0.000000}%
\pgfsetdash{}{0pt}%
\pgfpathmoveto{\pgfqpoint{0.510000in}{0.440000in}}%
\pgfpathlineto{\pgfqpoint{3.590000in}{0.440000in}}%
\pgfpathlineto{\pgfqpoint{3.590000in}{3.520000in}}%
\pgfpathlineto{\pgfqpoint{0.510000in}{3.520000in}}%
\pgfpathclose%
\pgfusepath{fill}%
\end{pgfscope}%
\begin{pgfscope}%
\pgfpathrectangle{\pgfqpoint{0.510000in}{0.440000in}}{\pgfqpoint{3.080000in}{3.080000in}}%
\pgfusepath{clip}%
\pgfsetbuttcap%
\pgfsetroundjoin%
\definecolor{currentfill}{rgb}{0.121569,0.466667,0.705882}%
\pgfsetfillcolor{currentfill}%
\pgfsetlinewidth{1.003750pt}%
\definecolor{currentstroke}{rgb}{0.121569,0.466667,0.705882}%
\pgfsetstrokecolor{currentstroke}%
\pgfsetdash{}{0pt}%
\pgfpathmoveto{\pgfqpoint{0.985338in}{2.135140in}}%
\pgfpathcurveto{\pgfqpoint{0.991162in}{2.135140in}}{\pgfqpoint{0.996748in}{2.137454in}}{\pgfqpoint{1.000866in}{2.141572in}}%
\pgfpathcurveto{\pgfqpoint{1.004984in}{2.145690in}}{\pgfqpoint{1.007298in}{2.151276in}}{\pgfqpoint{1.007298in}{2.157100in}}%
\pgfpathcurveto{\pgfqpoint{1.007298in}{2.162924in}}{\pgfqpoint{1.004984in}{2.168510in}}{\pgfqpoint{1.000866in}{2.172629in}}%
\pgfpathcurveto{\pgfqpoint{0.996748in}{2.176747in}}{\pgfqpoint{0.991162in}{2.179061in}}{\pgfqpoint{0.985338in}{2.179061in}}%
\pgfpathcurveto{\pgfqpoint{0.979514in}{2.179061in}}{\pgfqpoint{0.973928in}{2.176747in}}{\pgfqpoint{0.969810in}{2.172629in}}%
\pgfpathcurveto{\pgfqpoint{0.965691in}{2.168510in}}{\pgfqpoint{0.963378in}{2.162924in}}{\pgfqpoint{0.963378in}{2.157100in}}%
\pgfpathcurveto{\pgfqpoint{0.963378in}{2.151276in}}{\pgfqpoint{0.965691in}{2.145690in}}{\pgfqpoint{0.969810in}{2.141572in}}%
\pgfpathcurveto{\pgfqpoint{0.973928in}{2.137454in}}{\pgfqpoint{0.979514in}{2.135140in}}{\pgfqpoint{0.985338in}{2.135140in}}%
\pgfpathclose%
\pgfusepath{stroke,fill}%
\end{pgfscope}%
\begin{pgfscope}%
\pgfpathrectangle{\pgfqpoint{0.510000in}{0.440000in}}{\pgfqpoint{3.080000in}{3.080000in}}%
\pgfusepath{clip}%
\pgfsetbuttcap%
\pgfsetroundjoin%
\definecolor{currentfill}{rgb}{1.000000,0.498039,0.054902}%
\pgfsetfillcolor{currentfill}%
\pgfsetlinewidth{1.003750pt}%
\definecolor{currentstroke}{rgb}{1.000000,0.498039,0.054902}%
\pgfsetstrokecolor{currentstroke}%
\pgfsetdash{}{0pt}%
\pgfpathmoveto{\pgfqpoint{1.837562in}{0.893378in}}%
\pgfpathcurveto{\pgfqpoint{1.843386in}{0.893378in}}{\pgfqpoint{1.848972in}{0.895691in}}{\pgfqpoint{1.853090in}{0.899810in}}%
\pgfpathcurveto{\pgfqpoint{1.857208in}{0.903928in}}{\pgfqpoint{1.859522in}{0.909514in}}{\pgfqpoint{1.859522in}{0.915338in}}%
\pgfpathcurveto{\pgfqpoint{1.859522in}{0.921162in}}{\pgfqpoint{1.857208in}{0.926748in}}{\pgfqpoint{1.853090in}{0.930866in}}%
\pgfpathcurveto{\pgfqpoint{1.848972in}{0.934984in}}{\pgfqpoint{1.843386in}{0.937298in}}{\pgfqpoint{1.837562in}{0.937298in}}%
\pgfpathcurveto{\pgfqpoint{1.831738in}{0.937298in}}{\pgfqpoint{1.826152in}{0.934984in}}{\pgfqpoint{1.822034in}{0.930866in}}%
\pgfpathcurveto{\pgfqpoint{1.817915in}{0.926748in}}{\pgfqpoint{1.815602in}{0.921162in}}{\pgfqpoint{1.815602in}{0.915338in}}%
\pgfpathcurveto{\pgfqpoint{1.815602in}{0.909514in}}{\pgfqpoint{1.817915in}{0.903928in}}{\pgfqpoint{1.822034in}{0.899810in}}%
\pgfpathcurveto{\pgfqpoint{1.826152in}{0.895691in}}{\pgfqpoint{1.831738in}{0.893378in}}{\pgfqpoint{1.837562in}{0.893378in}}%
\pgfpathclose%
\pgfusepath{stroke,fill}%
\end{pgfscope}%
\begin{pgfscope}%
\pgfpathrectangle{\pgfqpoint{0.510000in}{0.440000in}}{\pgfqpoint{3.080000in}{3.080000in}}%
\pgfusepath{clip}%
\pgfsetbuttcap%
\pgfsetroundjoin%
\definecolor{currentfill}{rgb}{0.172549,0.627451,0.172549}%
\pgfsetfillcolor{currentfill}%
\pgfsetlinewidth{1.003750pt}%
\definecolor{currentstroke}{rgb}{0.172549,0.627451,0.172549}%
\pgfsetstrokecolor{currentstroke}%
\pgfsetdash{}{0pt}%
\pgfpathmoveto{\pgfqpoint{2.227100in}{1.745602in}}%
\pgfpathcurveto{\pgfqpoint{2.232924in}{1.745602in}}{\pgfqpoint{2.238510in}{1.747915in}}{\pgfqpoint{2.242629in}{1.752034in}}%
\pgfpathcurveto{\pgfqpoint{2.246747in}{1.756152in}}{\pgfqpoint{2.249061in}{1.761738in}}{\pgfqpoint{2.249061in}{1.767562in}}%
\pgfpathcurveto{\pgfqpoint{2.249061in}{1.773386in}}{\pgfqpoint{2.246747in}{1.778972in}}{\pgfqpoint{2.242629in}{1.783090in}}%
\pgfpathcurveto{\pgfqpoint{2.238510in}{1.787208in}}{\pgfqpoint{2.232924in}{1.789522in}}{\pgfqpoint{2.227100in}{1.789522in}}%
\pgfpathcurveto{\pgfqpoint{2.221276in}{1.789522in}}{\pgfqpoint{2.215690in}{1.787208in}}{\pgfqpoint{2.211572in}{1.783090in}}%
\pgfpathcurveto{\pgfqpoint{2.207454in}{1.778972in}}{\pgfqpoint{2.205140in}{1.773386in}}{\pgfqpoint{2.205140in}{1.767562in}}%
\pgfpathcurveto{\pgfqpoint{2.205140in}{1.761738in}}{\pgfqpoint{2.207454in}{1.756152in}}{\pgfqpoint{2.211572in}{1.752034in}}%
\pgfpathcurveto{\pgfqpoint{2.215690in}{1.747915in}}{\pgfqpoint{2.221276in}{1.745602in}}{\pgfqpoint{2.227100in}{1.745602in}}%
\pgfpathclose%
\pgfusepath{stroke,fill}%
\end{pgfscope}%
\begin{pgfscope}%
\pgfpathrectangle{\pgfqpoint{0.510000in}{0.440000in}}{\pgfqpoint{3.080000in}{3.080000in}}%
\pgfusepath{clip}%
\pgfsetbuttcap%
\pgfsetmiterjoin%
\definecolor{currentfill}{rgb}{0.100000,0.100000,0.100000}%
\pgfsetfillcolor{currentfill}%
\pgfsetfillopacity{0.100000}%
\pgfsetlinewidth{0.301125pt}%
\definecolor{currentstroke}{rgb}{0.000000,0.000000,0.000000}%
\pgfsetstrokecolor{currentstroke}%
\pgfsetdash{}{0pt}%
\pgfpathmoveto{\pgfqpoint{0.950000in}{0.880000in}}%
\pgfpathlineto{\pgfqpoint{0.950000in}{3.080000in}}%
\pgfpathlineto{\pgfqpoint{3.150000in}{0.880000in}}%
\pgfpathclose%
\pgfusepath{stroke,fill}%
\end{pgfscope}%
\begin{pgfscope}%
\pgfpathrectangle{\pgfqpoint{0.510000in}{0.440000in}}{\pgfqpoint{3.080000in}{3.080000in}}%
\pgfusepath{clip}%
\pgfsetrectcap%
\pgfsetroundjoin%
\pgfsetlinewidth{0.803000pt}%
\definecolor{currentstroke}{rgb}{0.690196,0.690196,0.690196}%
\pgfsetstrokecolor{currentstroke}%
\pgfsetdash{}{0pt}%
\pgfpathmoveto{\pgfqpoint{0.510000in}{0.440000in}}%
\pgfpathlineto{\pgfqpoint{0.510000in}{3.520000in}}%
\pgfusepath{stroke}%
\end{pgfscope}%
\begin{pgfscope}%
\pgfsetbuttcap%
\pgfsetroundjoin%
\definecolor{currentfill}{rgb}{0.000000,0.000000,0.000000}%
\pgfsetfillcolor{currentfill}%
\pgfsetlinewidth{0.803000pt}%
\definecolor{currentstroke}{rgb}{0.000000,0.000000,0.000000}%
\pgfsetstrokecolor{currentstroke}%
\pgfsetdash{}{0pt}%
\pgfsys@defobject{currentmarker}{\pgfqpoint{0.000000in}{-0.048611in}}{\pgfqpoint{0.000000in}{0.000000in}}{%
\pgfpathmoveto{\pgfqpoint{0.000000in}{0.000000in}}%
\pgfpathlineto{\pgfqpoint{0.000000in}{-0.048611in}}%
\pgfusepath{stroke,fill}%
}%
\begin{pgfscope}%
\pgfsys@transformshift{0.510000in}{0.440000in}%
\pgfsys@useobject{currentmarker}{}%
\end{pgfscope}%
\end{pgfscope}%
\begin{pgfscope}%
\definecolor{textcolor}{rgb}{0.000000,0.000000,0.000000}%
\pgfsetstrokecolor{textcolor}%
\pgfsetfillcolor{textcolor}%
\pgftext[x=0.510000in,y=0.342778in,,top]{\color{textcolor}\sffamily\fontsize{10.000000}{12.000000}\selectfont -0.2}%
\end{pgfscope}%
\begin{pgfscope}%
\pgfpathrectangle{\pgfqpoint{0.510000in}{0.440000in}}{\pgfqpoint{3.080000in}{3.080000in}}%
\pgfusepath{clip}%
\pgfsetrectcap%
\pgfsetroundjoin%
\pgfsetlinewidth{0.803000pt}%
\definecolor{currentstroke}{rgb}{0.690196,0.690196,0.690196}%
\pgfsetstrokecolor{currentstroke}%
\pgfsetdash{}{0pt}%
\pgfpathmoveto{\pgfqpoint{0.950000in}{0.440000in}}%
\pgfpathlineto{\pgfqpoint{0.950000in}{3.520000in}}%
\pgfusepath{stroke}%
\end{pgfscope}%
\begin{pgfscope}%
\pgfsetbuttcap%
\pgfsetroundjoin%
\definecolor{currentfill}{rgb}{0.000000,0.000000,0.000000}%
\pgfsetfillcolor{currentfill}%
\pgfsetlinewidth{0.803000pt}%
\definecolor{currentstroke}{rgb}{0.000000,0.000000,0.000000}%
\pgfsetstrokecolor{currentstroke}%
\pgfsetdash{}{0pt}%
\pgfsys@defobject{currentmarker}{\pgfqpoint{0.000000in}{-0.048611in}}{\pgfqpoint{0.000000in}{0.000000in}}{%
\pgfpathmoveto{\pgfqpoint{0.000000in}{0.000000in}}%
\pgfpathlineto{\pgfqpoint{0.000000in}{-0.048611in}}%
\pgfusepath{stroke,fill}%
}%
\begin{pgfscope}%
\pgfsys@transformshift{0.950000in}{0.440000in}%
\pgfsys@useobject{currentmarker}{}%
\end{pgfscope}%
\end{pgfscope}%
\begin{pgfscope}%
\definecolor{textcolor}{rgb}{0.000000,0.000000,0.000000}%
\pgfsetstrokecolor{textcolor}%
\pgfsetfillcolor{textcolor}%
\pgftext[x=0.950000in,y=0.342778in,,top]{\color{textcolor}\sffamily\fontsize{10.000000}{12.000000}\selectfont 0.0}%
\end{pgfscope}%
\begin{pgfscope}%
\pgfpathrectangle{\pgfqpoint{0.510000in}{0.440000in}}{\pgfqpoint{3.080000in}{3.080000in}}%
\pgfusepath{clip}%
\pgfsetrectcap%
\pgfsetroundjoin%
\pgfsetlinewidth{0.803000pt}%
\definecolor{currentstroke}{rgb}{0.690196,0.690196,0.690196}%
\pgfsetstrokecolor{currentstroke}%
\pgfsetdash{}{0pt}%
\pgfpathmoveto{\pgfqpoint{1.390000in}{0.440000in}}%
\pgfpathlineto{\pgfqpoint{1.390000in}{3.520000in}}%
\pgfusepath{stroke}%
\end{pgfscope}%
\begin{pgfscope}%
\pgfsetbuttcap%
\pgfsetroundjoin%
\definecolor{currentfill}{rgb}{0.000000,0.000000,0.000000}%
\pgfsetfillcolor{currentfill}%
\pgfsetlinewidth{0.803000pt}%
\definecolor{currentstroke}{rgb}{0.000000,0.000000,0.000000}%
\pgfsetstrokecolor{currentstroke}%
\pgfsetdash{}{0pt}%
\pgfsys@defobject{currentmarker}{\pgfqpoint{0.000000in}{-0.048611in}}{\pgfqpoint{0.000000in}{0.000000in}}{%
\pgfpathmoveto{\pgfqpoint{0.000000in}{0.000000in}}%
\pgfpathlineto{\pgfqpoint{0.000000in}{-0.048611in}}%
\pgfusepath{stroke,fill}%
}%
\begin{pgfscope}%
\pgfsys@transformshift{1.390000in}{0.440000in}%
\pgfsys@useobject{currentmarker}{}%
\end{pgfscope}%
\end{pgfscope}%
\begin{pgfscope}%
\definecolor{textcolor}{rgb}{0.000000,0.000000,0.000000}%
\pgfsetstrokecolor{textcolor}%
\pgfsetfillcolor{textcolor}%
\pgftext[x=1.390000in,y=0.342778in,,top]{\color{textcolor}\sffamily\fontsize{10.000000}{12.000000}\selectfont 0.2}%
\end{pgfscope}%
\begin{pgfscope}%
\pgfpathrectangle{\pgfqpoint{0.510000in}{0.440000in}}{\pgfqpoint{3.080000in}{3.080000in}}%
\pgfusepath{clip}%
\pgfsetrectcap%
\pgfsetroundjoin%
\pgfsetlinewidth{0.803000pt}%
\definecolor{currentstroke}{rgb}{0.690196,0.690196,0.690196}%
\pgfsetstrokecolor{currentstroke}%
\pgfsetdash{}{0pt}%
\pgfpathmoveto{\pgfqpoint{1.830000in}{0.440000in}}%
\pgfpathlineto{\pgfqpoint{1.830000in}{3.520000in}}%
\pgfusepath{stroke}%
\end{pgfscope}%
\begin{pgfscope}%
\pgfsetbuttcap%
\pgfsetroundjoin%
\definecolor{currentfill}{rgb}{0.000000,0.000000,0.000000}%
\pgfsetfillcolor{currentfill}%
\pgfsetlinewidth{0.803000pt}%
\definecolor{currentstroke}{rgb}{0.000000,0.000000,0.000000}%
\pgfsetstrokecolor{currentstroke}%
\pgfsetdash{}{0pt}%
\pgfsys@defobject{currentmarker}{\pgfqpoint{0.000000in}{-0.048611in}}{\pgfqpoint{0.000000in}{0.000000in}}{%
\pgfpathmoveto{\pgfqpoint{0.000000in}{0.000000in}}%
\pgfpathlineto{\pgfqpoint{0.000000in}{-0.048611in}}%
\pgfusepath{stroke,fill}%
}%
\begin{pgfscope}%
\pgfsys@transformshift{1.830000in}{0.440000in}%
\pgfsys@useobject{currentmarker}{}%
\end{pgfscope}%
\end{pgfscope}%
\begin{pgfscope}%
\definecolor{textcolor}{rgb}{0.000000,0.000000,0.000000}%
\pgfsetstrokecolor{textcolor}%
\pgfsetfillcolor{textcolor}%
\pgftext[x=1.830000in,y=0.342778in,,top]{\color{textcolor}\sffamily\fontsize{10.000000}{12.000000}\selectfont 0.4}%
\end{pgfscope}%
\begin{pgfscope}%
\pgfpathrectangle{\pgfqpoint{0.510000in}{0.440000in}}{\pgfqpoint{3.080000in}{3.080000in}}%
\pgfusepath{clip}%
\pgfsetrectcap%
\pgfsetroundjoin%
\pgfsetlinewidth{0.803000pt}%
\definecolor{currentstroke}{rgb}{0.690196,0.690196,0.690196}%
\pgfsetstrokecolor{currentstroke}%
\pgfsetdash{}{0pt}%
\pgfpathmoveto{\pgfqpoint{2.270000in}{0.440000in}}%
\pgfpathlineto{\pgfqpoint{2.270000in}{3.520000in}}%
\pgfusepath{stroke}%
\end{pgfscope}%
\begin{pgfscope}%
\pgfsetbuttcap%
\pgfsetroundjoin%
\definecolor{currentfill}{rgb}{0.000000,0.000000,0.000000}%
\pgfsetfillcolor{currentfill}%
\pgfsetlinewidth{0.803000pt}%
\definecolor{currentstroke}{rgb}{0.000000,0.000000,0.000000}%
\pgfsetstrokecolor{currentstroke}%
\pgfsetdash{}{0pt}%
\pgfsys@defobject{currentmarker}{\pgfqpoint{0.000000in}{-0.048611in}}{\pgfqpoint{0.000000in}{0.000000in}}{%
\pgfpathmoveto{\pgfqpoint{0.000000in}{0.000000in}}%
\pgfpathlineto{\pgfqpoint{0.000000in}{-0.048611in}}%
\pgfusepath{stroke,fill}%
}%
\begin{pgfscope}%
\pgfsys@transformshift{2.270000in}{0.440000in}%
\pgfsys@useobject{currentmarker}{}%
\end{pgfscope}%
\end{pgfscope}%
\begin{pgfscope}%
\definecolor{textcolor}{rgb}{0.000000,0.000000,0.000000}%
\pgfsetstrokecolor{textcolor}%
\pgfsetfillcolor{textcolor}%
\pgftext[x=2.270000in,y=0.342778in,,top]{\color{textcolor}\sffamily\fontsize{10.000000}{12.000000}\selectfont 0.6}%
\end{pgfscope}%
\begin{pgfscope}%
\pgfpathrectangle{\pgfqpoint{0.510000in}{0.440000in}}{\pgfqpoint{3.080000in}{3.080000in}}%
\pgfusepath{clip}%
\pgfsetrectcap%
\pgfsetroundjoin%
\pgfsetlinewidth{0.803000pt}%
\definecolor{currentstroke}{rgb}{0.690196,0.690196,0.690196}%
\pgfsetstrokecolor{currentstroke}%
\pgfsetdash{}{0pt}%
\pgfpathmoveto{\pgfqpoint{2.710000in}{0.440000in}}%
\pgfpathlineto{\pgfqpoint{2.710000in}{3.520000in}}%
\pgfusepath{stroke}%
\end{pgfscope}%
\begin{pgfscope}%
\pgfsetbuttcap%
\pgfsetroundjoin%
\definecolor{currentfill}{rgb}{0.000000,0.000000,0.000000}%
\pgfsetfillcolor{currentfill}%
\pgfsetlinewidth{0.803000pt}%
\definecolor{currentstroke}{rgb}{0.000000,0.000000,0.000000}%
\pgfsetstrokecolor{currentstroke}%
\pgfsetdash{}{0pt}%
\pgfsys@defobject{currentmarker}{\pgfqpoint{0.000000in}{-0.048611in}}{\pgfqpoint{0.000000in}{0.000000in}}{%
\pgfpathmoveto{\pgfqpoint{0.000000in}{0.000000in}}%
\pgfpathlineto{\pgfqpoint{0.000000in}{-0.048611in}}%
\pgfusepath{stroke,fill}%
}%
\begin{pgfscope}%
\pgfsys@transformshift{2.710000in}{0.440000in}%
\pgfsys@useobject{currentmarker}{}%
\end{pgfscope}%
\end{pgfscope}%
\begin{pgfscope}%
\definecolor{textcolor}{rgb}{0.000000,0.000000,0.000000}%
\pgfsetstrokecolor{textcolor}%
\pgfsetfillcolor{textcolor}%
\pgftext[x=2.710000in,y=0.342778in,,top]{\color{textcolor}\sffamily\fontsize{10.000000}{12.000000}\selectfont 0.8}%
\end{pgfscope}%
\begin{pgfscope}%
\pgfpathrectangle{\pgfqpoint{0.510000in}{0.440000in}}{\pgfqpoint{3.080000in}{3.080000in}}%
\pgfusepath{clip}%
\pgfsetrectcap%
\pgfsetroundjoin%
\pgfsetlinewidth{0.803000pt}%
\definecolor{currentstroke}{rgb}{0.690196,0.690196,0.690196}%
\pgfsetstrokecolor{currentstroke}%
\pgfsetdash{}{0pt}%
\pgfpathmoveto{\pgfqpoint{3.150000in}{0.440000in}}%
\pgfpathlineto{\pgfqpoint{3.150000in}{3.520000in}}%
\pgfusepath{stroke}%
\end{pgfscope}%
\begin{pgfscope}%
\pgfsetbuttcap%
\pgfsetroundjoin%
\definecolor{currentfill}{rgb}{0.000000,0.000000,0.000000}%
\pgfsetfillcolor{currentfill}%
\pgfsetlinewidth{0.803000pt}%
\definecolor{currentstroke}{rgb}{0.000000,0.000000,0.000000}%
\pgfsetstrokecolor{currentstroke}%
\pgfsetdash{}{0pt}%
\pgfsys@defobject{currentmarker}{\pgfqpoint{0.000000in}{-0.048611in}}{\pgfqpoint{0.000000in}{0.000000in}}{%
\pgfpathmoveto{\pgfqpoint{0.000000in}{0.000000in}}%
\pgfpathlineto{\pgfqpoint{0.000000in}{-0.048611in}}%
\pgfusepath{stroke,fill}%
}%
\begin{pgfscope}%
\pgfsys@transformshift{3.150000in}{0.440000in}%
\pgfsys@useobject{currentmarker}{}%
\end{pgfscope}%
\end{pgfscope}%
\begin{pgfscope}%
\definecolor{textcolor}{rgb}{0.000000,0.000000,0.000000}%
\pgfsetstrokecolor{textcolor}%
\pgfsetfillcolor{textcolor}%
\pgftext[x=3.150000in,y=0.342778in,,top]{\color{textcolor}\sffamily\fontsize{10.000000}{12.000000}\selectfont 1.0}%
\end{pgfscope}%
\begin{pgfscope}%
\pgfpathrectangle{\pgfqpoint{0.510000in}{0.440000in}}{\pgfqpoint{3.080000in}{3.080000in}}%
\pgfusepath{clip}%
\pgfsetrectcap%
\pgfsetroundjoin%
\pgfsetlinewidth{0.803000pt}%
\definecolor{currentstroke}{rgb}{0.690196,0.690196,0.690196}%
\pgfsetstrokecolor{currentstroke}%
\pgfsetdash{}{0pt}%
\pgfpathmoveto{\pgfqpoint{3.590000in}{0.440000in}}%
\pgfpathlineto{\pgfqpoint{3.590000in}{3.520000in}}%
\pgfusepath{stroke}%
\end{pgfscope}%
\begin{pgfscope}%
\pgfsetbuttcap%
\pgfsetroundjoin%
\definecolor{currentfill}{rgb}{0.000000,0.000000,0.000000}%
\pgfsetfillcolor{currentfill}%
\pgfsetlinewidth{0.803000pt}%
\definecolor{currentstroke}{rgb}{0.000000,0.000000,0.000000}%
\pgfsetstrokecolor{currentstroke}%
\pgfsetdash{}{0pt}%
\pgfsys@defobject{currentmarker}{\pgfqpoint{0.000000in}{-0.048611in}}{\pgfqpoint{0.000000in}{0.000000in}}{%
\pgfpathmoveto{\pgfqpoint{0.000000in}{0.000000in}}%
\pgfpathlineto{\pgfqpoint{0.000000in}{-0.048611in}}%
\pgfusepath{stroke,fill}%
}%
\begin{pgfscope}%
\pgfsys@transformshift{3.590000in}{0.440000in}%
\pgfsys@useobject{currentmarker}{}%
\end{pgfscope}%
\end{pgfscope}%
\begin{pgfscope}%
\definecolor{textcolor}{rgb}{0.000000,0.000000,0.000000}%
\pgfsetstrokecolor{textcolor}%
\pgfsetfillcolor{textcolor}%
\pgftext[x=3.590000in,y=0.342778in,,top]{\color{textcolor}\sffamily\fontsize{10.000000}{12.000000}\selectfont 1.2}%
\end{pgfscope}%
\begin{pgfscope}%
\pgfpathrectangle{\pgfqpoint{0.510000in}{0.440000in}}{\pgfqpoint{3.080000in}{3.080000in}}%
\pgfusepath{clip}%
\pgfsetrectcap%
\pgfsetroundjoin%
\pgfsetlinewidth{0.803000pt}%
\definecolor{currentstroke}{rgb}{0.690196,0.690196,0.690196}%
\pgfsetstrokecolor{currentstroke}%
\pgfsetdash{}{0pt}%
\pgfpathmoveto{\pgfqpoint{0.510000in}{0.440000in}}%
\pgfpathlineto{\pgfqpoint{3.590000in}{0.440000in}}%
\pgfusepath{stroke}%
\end{pgfscope}%
\begin{pgfscope}%
\pgfsetbuttcap%
\pgfsetroundjoin%
\definecolor{currentfill}{rgb}{0.000000,0.000000,0.000000}%
\pgfsetfillcolor{currentfill}%
\pgfsetlinewidth{0.803000pt}%
\definecolor{currentstroke}{rgb}{0.000000,0.000000,0.000000}%
\pgfsetstrokecolor{currentstroke}%
\pgfsetdash{}{0pt}%
\pgfsys@defobject{currentmarker}{\pgfqpoint{-0.048611in}{0.000000in}}{\pgfqpoint{0.000000in}{0.000000in}}{%
\pgfpathmoveto{\pgfqpoint{0.000000in}{0.000000in}}%
\pgfpathlineto{\pgfqpoint{-0.048611in}{0.000000in}}%
\pgfusepath{stroke,fill}%
}%
\begin{pgfscope}%
\pgfsys@transformshift{0.510000in}{0.440000in}%
\pgfsys@useobject{currentmarker}{}%
\end{pgfscope}%
\end{pgfscope}%
\begin{pgfscope}%
\definecolor{textcolor}{rgb}{0.000000,0.000000,0.000000}%
\pgfsetstrokecolor{textcolor}%
\pgfsetfillcolor{textcolor}%
\pgftext[x=0.075525in,y=0.387238in,left,base]{\color{textcolor}\sffamily\fontsize{10.000000}{12.000000}\selectfont -0.2}%
\end{pgfscope}%
\begin{pgfscope}%
\pgfpathrectangle{\pgfqpoint{0.510000in}{0.440000in}}{\pgfqpoint{3.080000in}{3.080000in}}%
\pgfusepath{clip}%
\pgfsetrectcap%
\pgfsetroundjoin%
\pgfsetlinewidth{0.803000pt}%
\definecolor{currentstroke}{rgb}{0.690196,0.690196,0.690196}%
\pgfsetstrokecolor{currentstroke}%
\pgfsetdash{}{0pt}%
\pgfpathmoveto{\pgfqpoint{0.510000in}{0.880000in}}%
\pgfpathlineto{\pgfqpoint{3.590000in}{0.880000in}}%
\pgfusepath{stroke}%
\end{pgfscope}%
\begin{pgfscope}%
\pgfsetbuttcap%
\pgfsetroundjoin%
\definecolor{currentfill}{rgb}{0.000000,0.000000,0.000000}%
\pgfsetfillcolor{currentfill}%
\pgfsetlinewidth{0.803000pt}%
\definecolor{currentstroke}{rgb}{0.000000,0.000000,0.000000}%
\pgfsetstrokecolor{currentstroke}%
\pgfsetdash{}{0pt}%
\pgfsys@defobject{currentmarker}{\pgfqpoint{-0.048611in}{0.000000in}}{\pgfqpoint{0.000000in}{0.000000in}}{%
\pgfpathmoveto{\pgfqpoint{0.000000in}{0.000000in}}%
\pgfpathlineto{\pgfqpoint{-0.048611in}{0.000000in}}%
\pgfusepath{stroke,fill}%
}%
\begin{pgfscope}%
\pgfsys@transformshift{0.510000in}{0.880000in}%
\pgfsys@useobject{currentmarker}{}%
\end{pgfscope}%
\end{pgfscope}%
\begin{pgfscope}%
\definecolor{textcolor}{rgb}{0.000000,0.000000,0.000000}%
\pgfsetstrokecolor{textcolor}%
\pgfsetfillcolor{textcolor}%
\pgftext[x=0.191898in,y=0.827238in,left,base]{\color{textcolor}\sffamily\fontsize{10.000000}{12.000000}\selectfont 0.0}%
\end{pgfscope}%
\begin{pgfscope}%
\pgfpathrectangle{\pgfqpoint{0.510000in}{0.440000in}}{\pgfqpoint{3.080000in}{3.080000in}}%
\pgfusepath{clip}%
\pgfsetrectcap%
\pgfsetroundjoin%
\pgfsetlinewidth{0.803000pt}%
\definecolor{currentstroke}{rgb}{0.690196,0.690196,0.690196}%
\pgfsetstrokecolor{currentstroke}%
\pgfsetdash{}{0pt}%
\pgfpathmoveto{\pgfqpoint{0.510000in}{1.320000in}}%
\pgfpathlineto{\pgfqpoint{3.590000in}{1.320000in}}%
\pgfusepath{stroke}%
\end{pgfscope}%
\begin{pgfscope}%
\pgfsetbuttcap%
\pgfsetroundjoin%
\definecolor{currentfill}{rgb}{0.000000,0.000000,0.000000}%
\pgfsetfillcolor{currentfill}%
\pgfsetlinewidth{0.803000pt}%
\definecolor{currentstroke}{rgb}{0.000000,0.000000,0.000000}%
\pgfsetstrokecolor{currentstroke}%
\pgfsetdash{}{0pt}%
\pgfsys@defobject{currentmarker}{\pgfqpoint{-0.048611in}{0.000000in}}{\pgfqpoint{0.000000in}{0.000000in}}{%
\pgfpathmoveto{\pgfqpoint{0.000000in}{0.000000in}}%
\pgfpathlineto{\pgfqpoint{-0.048611in}{0.000000in}}%
\pgfusepath{stroke,fill}%
}%
\begin{pgfscope}%
\pgfsys@transformshift{0.510000in}{1.320000in}%
\pgfsys@useobject{currentmarker}{}%
\end{pgfscope}%
\end{pgfscope}%
\begin{pgfscope}%
\definecolor{textcolor}{rgb}{0.000000,0.000000,0.000000}%
\pgfsetstrokecolor{textcolor}%
\pgfsetfillcolor{textcolor}%
\pgftext[x=0.191898in,y=1.267238in,left,base]{\color{textcolor}\sffamily\fontsize{10.000000}{12.000000}\selectfont 0.2}%
\end{pgfscope}%
\begin{pgfscope}%
\pgfpathrectangle{\pgfqpoint{0.510000in}{0.440000in}}{\pgfqpoint{3.080000in}{3.080000in}}%
\pgfusepath{clip}%
\pgfsetrectcap%
\pgfsetroundjoin%
\pgfsetlinewidth{0.803000pt}%
\definecolor{currentstroke}{rgb}{0.690196,0.690196,0.690196}%
\pgfsetstrokecolor{currentstroke}%
\pgfsetdash{}{0pt}%
\pgfpathmoveto{\pgfqpoint{0.510000in}{1.760000in}}%
\pgfpathlineto{\pgfqpoint{3.590000in}{1.760000in}}%
\pgfusepath{stroke}%
\end{pgfscope}%
\begin{pgfscope}%
\pgfsetbuttcap%
\pgfsetroundjoin%
\definecolor{currentfill}{rgb}{0.000000,0.000000,0.000000}%
\pgfsetfillcolor{currentfill}%
\pgfsetlinewidth{0.803000pt}%
\definecolor{currentstroke}{rgb}{0.000000,0.000000,0.000000}%
\pgfsetstrokecolor{currentstroke}%
\pgfsetdash{}{0pt}%
\pgfsys@defobject{currentmarker}{\pgfqpoint{-0.048611in}{0.000000in}}{\pgfqpoint{0.000000in}{0.000000in}}{%
\pgfpathmoveto{\pgfqpoint{0.000000in}{0.000000in}}%
\pgfpathlineto{\pgfqpoint{-0.048611in}{0.000000in}}%
\pgfusepath{stroke,fill}%
}%
\begin{pgfscope}%
\pgfsys@transformshift{0.510000in}{1.760000in}%
\pgfsys@useobject{currentmarker}{}%
\end{pgfscope}%
\end{pgfscope}%
\begin{pgfscope}%
\definecolor{textcolor}{rgb}{0.000000,0.000000,0.000000}%
\pgfsetstrokecolor{textcolor}%
\pgfsetfillcolor{textcolor}%
\pgftext[x=0.191898in,y=1.707238in,left,base]{\color{textcolor}\sffamily\fontsize{10.000000}{12.000000}\selectfont 0.4}%
\end{pgfscope}%
\begin{pgfscope}%
\pgfpathrectangle{\pgfqpoint{0.510000in}{0.440000in}}{\pgfqpoint{3.080000in}{3.080000in}}%
\pgfusepath{clip}%
\pgfsetrectcap%
\pgfsetroundjoin%
\pgfsetlinewidth{0.803000pt}%
\definecolor{currentstroke}{rgb}{0.690196,0.690196,0.690196}%
\pgfsetstrokecolor{currentstroke}%
\pgfsetdash{}{0pt}%
\pgfpathmoveto{\pgfqpoint{0.510000in}{2.200000in}}%
\pgfpathlineto{\pgfqpoint{3.590000in}{2.200000in}}%
\pgfusepath{stroke}%
\end{pgfscope}%
\begin{pgfscope}%
\pgfsetbuttcap%
\pgfsetroundjoin%
\definecolor{currentfill}{rgb}{0.000000,0.000000,0.000000}%
\pgfsetfillcolor{currentfill}%
\pgfsetlinewidth{0.803000pt}%
\definecolor{currentstroke}{rgb}{0.000000,0.000000,0.000000}%
\pgfsetstrokecolor{currentstroke}%
\pgfsetdash{}{0pt}%
\pgfsys@defobject{currentmarker}{\pgfqpoint{-0.048611in}{0.000000in}}{\pgfqpoint{0.000000in}{0.000000in}}{%
\pgfpathmoveto{\pgfqpoint{0.000000in}{0.000000in}}%
\pgfpathlineto{\pgfqpoint{-0.048611in}{0.000000in}}%
\pgfusepath{stroke,fill}%
}%
\begin{pgfscope}%
\pgfsys@transformshift{0.510000in}{2.200000in}%
\pgfsys@useobject{currentmarker}{}%
\end{pgfscope}%
\end{pgfscope}%
\begin{pgfscope}%
\definecolor{textcolor}{rgb}{0.000000,0.000000,0.000000}%
\pgfsetstrokecolor{textcolor}%
\pgfsetfillcolor{textcolor}%
\pgftext[x=0.191898in,y=2.147238in,left,base]{\color{textcolor}\sffamily\fontsize{10.000000}{12.000000}\selectfont 0.6}%
\end{pgfscope}%
\begin{pgfscope}%
\pgfpathrectangle{\pgfqpoint{0.510000in}{0.440000in}}{\pgfqpoint{3.080000in}{3.080000in}}%
\pgfusepath{clip}%
\pgfsetrectcap%
\pgfsetroundjoin%
\pgfsetlinewidth{0.803000pt}%
\definecolor{currentstroke}{rgb}{0.690196,0.690196,0.690196}%
\pgfsetstrokecolor{currentstroke}%
\pgfsetdash{}{0pt}%
\pgfpathmoveto{\pgfqpoint{0.510000in}{2.640000in}}%
\pgfpathlineto{\pgfqpoint{3.590000in}{2.640000in}}%
\pgfusepath{stroke}%
\end{pgfscope}%
\begin{pgfscope}%
\pgfsetbuttcap%
\pgfsetroundjoin%
\definecolor{currentfill}{rgb}{0.000000,0.000000,0.000000}%
\pgfsetfillcolor{currentfill}%
\pgfsetlinewidth{0.803000pt}%
\definecolor{currentstroke}{rgb}{0.000000,0.000000,0.000000}%
\pgfsetstrokecolor{currentstroke}%
\pgfsetdash{}{0pt}%
\pgfsys@defobject{currentmarker}{\pgfqpoint{-0.048611in}{0.000000in}}{\pgfqpoint{0.000000in}{0.000000in}}{%
\pgfpathmoveto{\pgfqpoint{0.000000in}{0.000000in}}%
\pgfpathlineto{\pgfqpoint{-0.048611in}{0.000000in}}%
\pgfusepath{stroke,fill}%
}%
\begin{pgfscope}%
\pgfsys@transformshift{0.510000in}{2.640000in}%
\pgfsys@useobject{currentmarker}{}%
\end{pgfscope}%
\end{pgfscope}%
\begin{pgfscope}%
\definecolor{textcolor}{rgb}{0.000000,0.000000,0.000000}%
\pgfsetstrokecolor{textcolor}%
\pgfsetfillcolor{textcolor}%
\pgftext[x=0.191898in,y=2.587238in,left,base]{\color{textcolor}\sffamily\fontsize{10.000000}{12.000000}\selectfont 0.8}%
\end{pgfscope}%
\begin{pgfscope}%
\pgfpathrectangle{\pgfqpoint{0.510000in}{0.440000in}}{\pgfqpoint{3.080000in}{3.080000in}}%
\pgfusepath{clip}%
\pgfsetrectcap%
\pgfsetroundjoin%
\pgfsetlinewidth{0.803000pt}%
\definecolor{currentstroke}{rgb}{0.690196,0.690196,0.690196}%
\pgfsetstrokecolor{currentstroke}%
\pgfsetdash{}{0pt}%
\pgfpathmoveto{\pgfqpoint{0.510000in}{3.080000in}}%
\pgfpathlineto{\pgfqpoint{3.590000in}{3.080000in}}%
\pgfusepath{stroke}%
\end{pgfscope}%
\begin{pgfscope}%
\pgfsetbuttcap%
\pgfsetroundjoin%
\definecolor{currentfill}{rgb}{0.000000,0.000000,0.000000}%
\pgfsetfillcolor{currentfill}%
\pgfsetlinewidth{0.803000pt}%
\definecolor{currentstroke}{rgb}{0.000000,0.000000,0.000000}%
\pgfsetstrokecolor{currentstroke}%
\pgfsetdash{}{0pt}%
\pgfsys@defobject{currentmarker}{\pgfqpoint{-0.048611in}{0.000000in}}{\pgfqpoint{0.000000in}{0.000000in}}{%
\pgfpathmoveto{\pgfqpoint{0.000000in}{0.000000in}}%
\pgfpathlineto{\pgfqpoint{-0.048611in}{0.000000in}}%
\pgfusepath{stroke,fill}%
}%
\begin{pgfscope}%
\pgfsys@transformshift{0.510000in}{3.080000in}%
\pgfsys@useobject{currentmarker}{}%
\end{pgfscope}%
\end{pgfscope}%
\begin{pgfscope}%
\definecolor{textcolor}{rgb}{0.000000,0.000000,0.000000}%
\pgfsetstrokecolor{textcolor}%
\pgfsetfillcolor{textcolor}%
\pgftext[x=0.191898in,y=3.027238in,left,base]{\color{textcolor}\sffamily\fontsize{10.000000}{12.000000}\selectfont 1.0}%
\end{pgfscope}%
\begin{pgfscope}%
\pgfpathrectangle{\pgfqpoint{0.510000in}{0.440000in}}{\pgfqpoint{3.080000in}{3.080000in}}%
\pgfusepath{clip}%
\pgfsetrectcap%
\pgfsetroundjoin%
\pgfsetlinewidth{0.803000pt}%
\definecolor{currentstroke}{rgb}{0.690196,0.690196,0.690196}%
\pgfsetstrokecolor{currentstroke}%
\pgfsetdash{}{0pt}%
\pgfpathmoveto{\pgfqpoint{0.510000in}{3.520000in}}%
\pgfpathlineto{\pgfqpoint{3.590000in}{3.520000in}}%
\pgfusepath{stroke}%
\end{pgfscope}%
\begin{pgfscope}%
\pgfsetbuttcap%
\pgfsetroundjoin%
\definecolor{currentfill}{rgb}{0.000000,0.000000,0.000000}%
\pgfsetfillcolor{currentfill}%
\pgfsetlinewidth{0.803000pt}%
\definecolor{currentstroke}{rgb}{0.000000,0.000000,0.000000}%
\pgfsetstrokecolor{currentstroke}%
\pgfsetdash{}{0pt}%
\pgfsys@defobject{currentmarker}{\pgfqpoint{-0.048611in}{0.000000in}}{\pgfqpoint{0.000000in}{0.000000in}}{%
\pgfpathmoveto{\pgfqpoint{0.000000in}{0.000000in}}%
\pgfpathlineto{\pgfqpoint{-0.048611in}{0.000000in}}%
\pgfusepath{stroke,fill}%
}%
\begin{pgfscope}%
\pgfsys@transformshift{0.510000in}{3.520000in}%
\pgfsys@useobject{currentmarker}{}%
\end{pgfscope}%
\end{pgfscope}%
\begin{pgfscope}%
\definecolor{textcolor}{rgb}{0.000000,0.000000,0.000000}%
\pgfsetstrokecolor{textcolor}%
\pgfsetfillcolor{textcolor}%
\pgftext[x=0.191898in,y=3.467238in,left,base]{\color{textcolor}\sffamily\fontsize{10.000000}{12.000000}\selectfont 1.2}%
\end{pgfscope}%
\begin{pgfscope}%
\pgfsetrectcap%
\pgfsetmiterjoin%
\pgfsetlinewidth{0.803000pt}%
\definecolor{currentstroke}{rgb}{0.000000,0.000000,0.000000}%
\pgfsetstrokecolor{currentstroke}%
\pgfsetdash{}{0pt}%
\pgfpathmoveto{\pgfqpoint{0.510000in}{0.440000in}}%
\pgfpathlineto{\pgfqpoint{0.510000in}{3.520000in}}%
\pgfusepath{stroke}%
\end{pgfscope}%
\begin{pgfscope}%
\pgfsetrectcap%
\pgfsetmiterjoin%
\pgfsetlinewidth{0.803000pt}%
\definecolor{currentstroke}{rgb}{0.000000,0.000000,0.000000}%
\pgfsetstrokecolor{currentstroke}%
\pgfsetdash{}{0pt}%
\pgfpathmoveto{\pgfqpoint{3.590000in}{0.440000in}}%
\pgfpathlineto{\pgfqpoint{3.590000in}{3.520000in}}%
\pgfusepath{stroke}%
\end{pgfscope}%
\begin{pgfscope}%
\pgfsetrectcap%
\pgfsetmiterjoin%
\pgfsetlinewidth{0.803000pt}%
\definecolor{currentstroke}{rgb}{0.000000,0.000000,0.000000}%
\pgfsetstrokecolor{currentstroke}%
\pgfsetdash{}{0pt}%
\pgfpathmoveto{\pgfqpoint{0.510000in}{0.440000in}}%
\pgfpathlineto{\pgfqpoint{3.590000in}{0.440000in}}%
\pgfusepath{stroke}%
\end{pgfscope}%
\begin{pgfscope}%
\pgfsetrectcap%
\pgfsetmiterjoin%
\pgfsetlinewidth{0.803000pt}%
\definecolor{currentstroke}{rgb}{0.000000,0.000000,0.000000}%
\pgfsetstrokecolor{currentstroke}%
\pgfsetdash{}{0pt}%
\pgfpathmoveto{\pgfqpoint{0.510000in}{3.520000in}}%
\pgfpathlineto{\pgfqpoint{3.590000in}{3.520000in}}%
\pgfusepath{stroke}%
\end{pgfscope}%
\begin{pgfscope}%
\definecolor{textcolor}{rgb}{0.000000,0.000000,0.000000}%
\pgfsetstrokecolor{textcolor}%
\pgfsetfillcolor{textcolor}%
\pgftext[x=0.985338in,y=2.157100in,left,base]{\color{textcolor}\sffamily\fontsize{10.000000}{12.000000}\selectfont 0.16667}%
\end{pgfscope}%
\begin{pgfscope}%
\definecolor{textcolor}{rgb}{0.000000,0.000000,0.000000}%
\pgfsetstrokecolor{textcolor}%
\pgfsetfillcolor{textcolor}%
\pgftext[x=1.837562in,y=0.915338in,left,base]{\color{textcolor}\sffamily\fontsize{10.000000}{12.000000}\selectfont 0.16667}%
\end{pgfscope}%
\begin{pgfscope}%
\definecolor{textcolor}{rgb}{0.000000,0.000000,0.000000}%
\pgfsetstrokecolor{textcolor}%
\pgfsetfillcolor{textcolor}%
\pgftext[x=2.227100in,y=1.767562in,left,base]{\color{textcolor}\sffamily\fontsize{10.000000}{12.000000}\selectfont 0.16667}%
\end{pgfscope}%
\end{pgfpicture}%
\makeatother%
\endgroup%
}
\caption{Figures of the two-dimensional cubatures of $ D = 2 $ with extra constraint $ \pbr{ 1, 2 } $}
\label{Fig:M2D212}
\end{figure}

In conclusion, in this case the minimal $N$ is $3$.

\subsubsection{The case $ D = 3 $}

We optimize for $ D = 3 $, $ N = 3 $ with $ \text{FPG} \rbr{ 10^6, 10^{-2} } $ and get the residual $ F ( \mathbf{x}_{\cdot}^{\rbr{K}}, u_{\cdot}^{\rbr{K}} ) = \text{\input{Text2.txt}} $. This implies $ N \le 3 $ is impossible.

We turn to try $ N = 4 $. In this case, since $ \abs{\mathcal{I}_3} = 10 $, we may group with $ C = 2 $ and thus try $ \sbr{ 1, 1, 2, 2 } $. Using $\mathcal{I}_3$ for constraints, we have $ L = 2 N + C = 10 $. We use $ \text{FPG} \rbr{ 10^4, 10^{-2} } + \text{N} \rbr{30} $ to optimize and get the results in Figure \ref{Fig:M2D3}.

\begin{figure}[htbp]
\centering
\scalebox{0.75}{%% Creator: Matplotlib, PGF backend
%%
%% To include the figure in your LaTeX document, write
%%   \input{<filename>.pgf}
%%
%% Make sure the required packages are loaded in your preamble
%%   \usepackage{pgf}
%%
%% Figures using additional raster images can only be included by \input if
%% they are in the same directory as the main LaTeX file. For loading figures
%% from other directories you can use the `import` package
%%   \usepackage{import}
%% and then include the figures with
%%   \import{<path to file>}{<filename>.pgf}
%%
%% Matplotlib used the following preamble
%%   \usepackage{fontspec}
%%   \setmainfont{DejaVuSerif.ttf}[Path=/home/lzh/anaconda3/envs/numana/lib/python3.7/site-packages/matplotlib/mpl-data/fonts/ttf/]
%%   \setsansfont{DejaVuSans.ttf}[Path=/home/lzh/anaconda3/envs/numana/lib/python3.7/site-packages/matplotlib/mpl-data/fonts/ttf/]
%%   \setmonofont{DejaVuSansMono.ttf}[Path=/home/lzh/anaconda3/envs/numana/lib/python3.7/site-packages/matplotlib/mpl-data/fonts/ttf/]
%%
\begingroup%
\makeatletter%
\begin{pgfpicture}%
\pgfpathrectangle{\pgfpointorigin}{\pgfqpoint{12.000000in}{4.000000in}}%
\pgfusepath{use as bounding box, clip}%
\begin{pgfscope}%
\pgfsetbuttcap%
\pgfsetmiterjoin%
\definecolor{currentfill}{rgb}{1.000000,1.000000,1.000000}%
\pgfsetfillcolor{currentfill}%
\pgfsetlinewidth{0.000000pt}%
\definecolor{currentstroke}{rgb}{1.000000,1.000000,1.000000}%
\pgfsetstrokecolor{currentstroke}%
\pgfsetdash{}{0pt}%
\pgfpathmoveto{\pgfqpoint{0.000000in}{0.000000in}}%
\pgfpathlineto{\pgfqpoint{12.000000in}{0.000000in}}%
\pgfpathlineto{\pgfqpoint{12.000000in}{4.000000in}}%
\pgfpathlineto{\pgfqpoint{0.000000in}{4.000000in}}%
\pgfpathclose%
\pgfusepath{fill}%
\end{pgfscope}%
\begin{pgfscope}%
\pgfsys@transformshift{0.120000in}{0.065000in}%
\pgftext[left,bottom]{\pgfimage[interpolate=true,width=3.830000in,height=3.785000in]{Figure12-img0.png}}%
\end{pgfscope}%
\begin{pgfscope}%
\pgfsys@transformshift{4.100000in}{0.075000in}%
\pgftext[left,bottom]{\pgfimage[interpolate=true,width=3.930000in,height=3.775000in]{Figure12-img1.png}}%
\end{pgfscope}%
\begin{pgfscope}%
\pgfsys@transformshift{8.050000in}{0.150000in}%
\pgftext[left,bottom]{\pgfimage[interpolate=true,width=3.950000in,height=3.775000in]{Figure12-img2.png}}%
\end{pgfscope}%
\end{pgfpicture}%
\makeatother%
\endgroup%
}~
\scalebox{0.75}{%% Creator: Matplotlib, PGF backend
%%
%% To include the figure in your LaTeX document, write
%%   \input{<filename>.pgf}
%%
%% Make sure the required packages are loaded in your preamble
%%   \usepackage{pgf}
%%
%% Figures using additional raster images can only be included by \input if
%% they are in the same directory as the main LaTeX file. For loading figures
%% from other directories you can use the `import` package
%%   \usepackage{import}
%% and then include the figures with
%%   \import{<path to file>}{<filename>.pgf}
%%
%% Matplotlib used the following preamble
%%   \usepackage{fontspec}
%%   \setmainfont{DejaVuSerif.ttf}[Path=/home/lzh/anaconda3/envs/numana/lib/python3.7/site-packages/matplotlib/mpl-data/fonts/ttf/]
%%   \setsansfont{DejaVuSans.ttf}[Path=/home/lzh/anaconda3/envs/numana/lib/python3.7/site-packages/matplotlib/mpl-data/fonts/ttf/]
%%   \setmonofont{DejaVuSansMono.ttf}[Path=/home/lzh/anaconda3/envs/numana/lib/python3.7/site-packages/matplotlib/mpl-data/fonts/ttf/]
%%
\begingroup%
\makeatletter%
\begin{pgfpicture}%
\pgfpathrectangle{\pgfpointorigin}{\pgfqpoint{4.000000in}{4.000000in}}%
\pgfusepath{use as bounding box, clip}%
\begin{pgfscope}%
\pgfsetbuttcap%
\pgfsetmiterjoin%
\definecolor{currentfill}{rgb}{1.000000,1.000000,1.000000}%
\pgfsetfillcolor{currentfill}%
\pgfsetlinewidth{0.000000pt}%
\definecolor{currentstroke}{rgb}{1.000000,1.000000,1.000000}%
\pgfsetstrokecolor{currentstroke}%
\pgfsetdash{}{0pt}%
\pgfpathmoveto{\pgfqpoint{0.000000in}{0.000000in}}%
\pgfpathlineto{\pgfqpoint{4.000000in}{0.000000in}}%
\pgfpathlineto{\pgfqpoint{4.000000in}{4.000000in}}%
\pgfpathlineto{\pgfqpoint{0.000000in}{4.000000in}}%
\pgfpathclose%
\pgfusepath{fill}%
\end{pgfscope}%
\begin{pgfscope}%
\pgfsetbuttcap%
\pgfsetmiterjoin%
\definecolor{currentfill}{rgb}{1.000000,1.000000,1.000000}%
\pgfsetfillcolor{currentfill}%
\pgfsetlinewidth{0.000000pt}%
\definecolor{currentstroke}{rgb}{0.000000,0.000000,0.000000}%
\pgfsetstrokecolor{currentstroke}%
\pgfsetstrokeopacity{0.000000}%
\pgfsetdash{}{0pt}%
\pgfpathmoveto{\pgfqpoint{0.510000in}{0.440000in}}%
\pgfpathlineto{\pgfqpoint{3.590000in}{0.440000in}}%
\pgfpathlineto{\pgfqpoint{3.590000in}{3.520000in}}%
\pgfpathlineto{\pgfqpoint{0.510000in}{3.520000in}}%
\pgfpathclose%
\pgfusepath{fill}%
\end{pgfscope}%
\begin{pgfscope}%
\pgfpathrectangle{\pgfqpoint{0.510000in}{0.440000in}}{\pgfqpoint{3.080000in}{3.080000in}}%
\pgfusepath{clip}%
\pgfsetbuttcap%
\pgfsetroundjoin%
\definecolor{currentfill}{rgb}{0.121569,0.466667,0.705882}%
\pgfsetfillcolor{currentfill}%
\pgfsetlinewidth{1.003750pt}%
\definecolor{currentstroke}{rgb}{0.121569,0.466667,0.705882}%
\pgfsetstrokecolor{currentstroke}%
\pgfsetdash{}{0pt}%
\pgfpathmoveto{\pgfqpoint{1.230473in}{1.817803in}}%
\pgfpathcurveto{\pgfqpoint{1.236297in}{1.817803in}}{\pgfqpoint{1.241883in}{1.820117in}}{\pgfqpoint{1.246001in}{1.824235in}}%
\pgfpathcurveto{\pgfqpoint{1.250120in}{1.828353in}}{\pgfqpoint{1.252433in}{1.833939in}}{\pgfqpoint{1.252433in}{1.839763in}}%
\pgfpathcurveto{\pgfqpoint{1.252433in}{1.845587in}}{\pgfqpoint{1.250120in}{1.851174in}}{\pgfqpoint{1.246001in}{1.855292in}}%
\pgfpathcurveto{\pgfqpoint{1.241883in}{1.859410in}}{\pgfqpoint{1.236297in}{1.861724in}}{\pgfqpoint{1.230473in}{1.861724in}}%
\pgfpathcurveto{\pgfqpoint{1.224649in}{1.861724in}}{\pgfqpoint{1.219063in}{1.859410in}}{\pgfqpoint{1.214945in}{1.855292in}}%
\pgfpathcurveto{\pgfqpoint{1.210827in}{1.851174in}}{\pgfqpoint{1.208513in}{1.845587in}}{\pgfqpoint{1.208513in}{1.839763in}}%
\pgfpathcurveto{\pgfqpoint{1.208513in}{1.833939in}}{\pgfqpoint{1.210827in}{1.828353in}}{\pgfqpoint{1.214945in}{1.824235in}}%
\pgfpathcurveto{\pgfqpoint{1.219063in}{1.820117in}}{\pgfqpoint{1.224649in}{1.817803in}}{\pgfqpoint{1.230473in}{1.817803in}}%
\pgfpathclose%
\pgfusepath{stroke,fill}%
\end{pgfscope}%
\begin{pgfscope}%
\pgfpathrectangle{\pgfqpoint{0.510000in}{0.440000in}}{\pgfqpoint{3.080000in}{3.080000in}}%
\pgfusepath{clip}%
\pgfsetbuttcap%
\pgfsetroundjoin%
\definecolor{currentfill}{rgb}{1.000000,0.498039,0.054902}%
\pgfsetfillcolor{currentfill}%
\pgfsetlinewidth{1.003750pt}%
\definecolor{currentstroke}{rgb}{1.000000,0.498039,0.054902}%
\pgfsetstrokecolor{currentstroke}%
\pgfsetdash{}{0pt}%
\pgfpathmoveto{\pgfqpoint{2.379121in}{1.243479in}}%
\pgfpathcurveto{\pgfqpoint{2.384945in}{1.243479in}}{\pgfqpoint{2.390531in}{1.245793in}}{\pgfqpoint{2.394649in}{1.249911in}}%
\pgfpathcurveto{\pgfqpoint{2.398767in}{1.254029in}}{\pgfqpoint{2.401081in}{1.259616in}}{\pgfqpoint{2.401081in}{1.265440in}}%
\pgfpathcurveto{\pgfqpoint{2.401081in}{1.271264in}}{\pgfqpoint{2.398767in}{1.276850in}}{\pgfqpoint{2.394649in}{1.280968in}}%
\pgfpathcurveto{\pgfqpoint{2.390531in}{1.285086in}}{\pgfqpoint{2.384945in}{1.287400in}}{\pgfqpoint{2.379121in}{1.287400in}}%
\pgfpathcurveto{\pgfqpoint{2.373297in}{1.287400in}}{\pgfqpoint{2.367711in}{1.285086in}}{\pgfqpoint{2.363593in}{1.280968in}}%
\pgfpathcurveto{\pgfqpoint{2.359474in}{1.276850in}}{\pgfqpoint{2.357161in}{1.271264in}}{\pgfqpoint{2.357161in}{1.265440in}}%
\pgfpathcurveto{\pgfqpoint{2.357161in}{1.259616in}}{\pgfqpoint{2.359474in}{1.254029in}}{\pgfqpoint{2.363593in}{1.249911in}}%
\pgfpathcurveto{\pgfqpoint{2.367711in}{1.245793in}}{\pgfqpoint{2.373297in}{1.243479in}}{\pgfqpoint{2.379121in}{1.243479in}}%
\pgfpathclose%
\pgfusepath{stroke,fill}%
\end{pgfscope}%
\begin{pgfscope}%
\pgfpathrectangle{\pgfqpoint{0.510000in}{0.440000in}}{\pgfqpoint{3.080000in}{3.080000in}}%
\pgfusepath{clip}%
\pgfsetbuttcap%
\pgfsetroundjoin%
\definecolor{currentfill}{rgb}{0.172549,0.627451,0.172549}%
\pgfsetfillcolor{currentfill}%
\pgfsetlinewidth{1.003750pt}%
\definecolor{currentstroke}{rgb}{0.172549,0.627451,0.172549}%
\pgfsetstrokecolor{currentstroke}%
\pgfsetdash{}{0pt}%
\pgfpathmoveto{\pgfqpoint{1.390000in}{0.908167in}}%
\pgfpathcurveto{\pgfqpoint{1.395824in}{0.908167in}}{\pgfqpoint{1.401410in}{0.910481in}}{\pgfqpoint{1.405528in}{0.914599in}}%
\pgfpathcurveto{\pgfqpoint{1.409646in}{0.918717in}}{\pgfqpoint{1.411960in}{0.924303in}}{\pgfqpoint{1.411960in}{0.930127in}}%
\pgfpathcurveto{\pgfqpoint{1.411960in}{0.935951in}}{\pgfqpoint{1.409646in}{0.941537in}}{\pgfqpoint{1.405528in}{0.945655in}}%
\pgfpathcurveto{\pgfqpoint{1.401410in}{0.949773in}}{\pgfqpoint{1.395824in}{0.952087in}}{\pgfqpoint{1.390000in}{0.952087in}}%
\pgfpathcurveto{\pgfqpoint{1.384176in}{0.952087in}}{\pgfqpoint{1.378590in}{0.949773in}}{\pgfqpoint{1.374472in}{0.945655in}}%
\pgfpathcurveto{\pgfqpoint{1.370354in}{0.941537in}}{\pgfqpoint{1.368040in}{0.935951in}}{\pgfqpoint{1.368040in}{0.930127in}}%
\pgfpathcurveto{\pgfqpoint{1.368040in}{0.924303in}}{\pgfqpoint{1.370354in}{0.918717in}}{\pgfqpoint{1.374472in}{0.914599in}}%
\pgfpathcurveto{\pgfqpoint{1.378590in}{0.910481in}}{\pgfqpoint{1.384176in}{0.908167in}}{\pgfqpoint{1.390000in}{0.908167in}}%
\pgfpathclose%
\pgfusepath{stroke,fill}%
\end{pgfscope}%
\begin{pgfscope}%
\pgfpathrectangle{\pgfqpoint{0.510000in}{0.440000in}}{\pgfqpoint{3.080000in}{3.080000in}}%
\pgfusepath{clip}%
\pgfsetbuttcap%
\pgfsetroundjoin%
\definecolor{currentfill}{rgb}{0.839216,0.152941,0.156863}%
\pgfsetfillcolor{currentfill}%
\pgfsetlinewidth{1.003750pt}%
\definecolor{currentstroke}{rgb}{0.839216,0.152941,0.156863}%
\pgfsetstrokecolor{currentstroke}%
\pgfsetdash{}{0pt}%
\pgfpathmoveto{\pgfqpoint{1.390000in}{2.567913in}}%
\pgfpathcurveto{\pgfqpoint{1.395824in}{2.567913in}}{\pgfqpoint{1.401410in}{2.570227in}}{\pgfqpoint{1.405528in}{2.574345in}}%
\pgfpathcurveto{\pgfqpoint{1.409646in}{2.578463in}}{\pgfqpoint{1.411960in}{2.584049in}}{\pgfqpoint{1.411960in}{2.589873in}}%
\pgfpathcurveto{\pgfqpoint{1.411960in}{2.595697in}}{\pgfqpoint{1.409646in}{2.601283in}}{\pgfqpoint{1.405528in}{2.605401in}}%
\pgfpathcurveto{\pgfqpoint{1.401410in}{2.609519in}}{\pgfqpoint{1.395824in}{2.611833in}}{\pgfqpoint{1.390000in}{2.611833in}}%
\pgfpathcurveto{\pgfqpoint{1.384176in}{2.611833in}}{\pgfqpoint{1.378590in}{2.609519in}}{\pgfqpoint{1.374472in}{2.605401in}}%
\pgfpathcurveto{\pgfqpoint{1.370354in}{2.601283in}}{\pgfqpoint{1.368040in}{2.595697in}}{\pgfqpoint{1.368040in}{2.589873in}}%
\pgfpathcurveto{\pgfqpoint{1.368040in}{2.584049in}}{\pgfqpoint{1.370354in}{2.578463in}}{\pgfqpoint{1.374472in}{2.574345in}}%
\pgfpathcurveto{\pgfqpoint{1.378590in}{2.570227in}}{\pgfqpoint{1.384176in}{2.567913in}}{\pgfqpoint{1.390000in}{2.567913in}}%
\pgfpathclose%
\pgfusepath{stroke,fill}%
\end{pgfscope}%
\begin{pgfscope}%
\pgfpathrectangle{\pgfqpoint{0.510000in}{0.440000in}}{\pgfqpoint{3.080000in}{3.080000in}}%
\pgfusepath{clip}%
\pgfsetbuttcap%
\pgfsetmiterjoin%
\definecolor{currentfill}{rgb}{0.100000,0.100000,0.100000}%
\pgfsetfillcolor{currentfill}%
\pgfsetfillopacity{0.100000}%
\pgfsetlinewidth{0.301125pt}%
\definecolor{currentstroke}{rgb}{0.000000,0.000000,0.000000}%
\pgfsetstrokecolor{currentstroke}%
\pgfsetdash{}{0pt}%
\pgfpathmoveto{\pgfqpoint{0.950000in}{0.880000in}}%
\pgfpathlineto{\pgfqpoint{0.950000in}{3.080000in}}%
\pgfpathlineto{\pgfqpoint{3.150000in}{0.880000in}}%
\pgfpathclose%
\pgfusepath{stroke,fill}%
\end{pgfscope}%
\begin{pgfscope}%
\pgfpathrectangle{\pgfqpoint{0.510000in}{0.440000in}}{\pgfqpoint{3.080000in}{3.080000in}}%
\pgfusepath{clip}%
\pgfsetrectcap%
\pgfsetroundjoin%
\pgfsetlinewidth{0.803000pt}%
\definecolor{currentstroke}{rgb}{0.690196,0.690196,0.690196}%
\pgfsetstrokecolor{currentstroke}%
\pgfsetdash{}{0pt}%
\pgfpathmoveto{\pgfqpoint{0.510000in}{0.440000in}}%
\pgfpathlineto{\pgfqpoint{0.510000in}{3.520000in}}%
\pgfusepath{stroke}%
\end{pgfscope}%
\begin{pgfscope}%
\pgfsetbuttcap%
\pgfsetroundjoin%
\definecolor{currentfill}{rgb}{0.000000,0.000000,0.000000}%
\pgfsetfillcolor{currentfill}%
\pgfsetlinewidth{0.803000pt}%
\definecolor{currentstroke}{rgb}{0.000000,0.000000,0.000000}%
\pgfsetstrokecolor{currentstroke}%
\pgfsetdash{}{0pt}%
\pgfsys@defobject{currentmarker}{\pgfqpoint{0.000000in}{-0.048611in}}{\pgfqpoint{0.000000in}{0.000000in}}{%
\pgfpathmoveto{\pgfqpoint{0.000000in}{0.000000in}}%
\pgfpathlineto{\pgfqpoint{0.000000in}{-0.048611in}}%
\pgfusepath{stroke,fill}%
}%
\begin{pgfscope}%
\pgfsys@transformshift{0.510000in}{0.440000in}%
\pgfsys@useobject{currentmarker}{}%
\end{pgfscope}%
\end{pgfscope}%
\begin{pgfscope}%
\definecolor{textcolor}{rgb}{0.000000,0.000000,0.000000}%
\pgfsetstrokecolor{textcolor}%
\pgfsetfillcolor{textcolor}%
\pgftext[x=0.510000in,y=0.342778in,,top]{\color{textcolor}\sffamily\fontsize{10.000000}{12.000000}\selectfont -0.2}%
\end{pgfscope}%
\begin{pgfscope}%
\pgfpathrectangle{\pgfqpoint{0.510000in}{0.440000in}}{\pgfqpoint{3.080000in}{3.080000in}}%
\pgfusepath{clip}%
\pgfsetrectcap%
\pgfsetroundjoin%
\pgfsetlinewidth{0.803000pt}%
\definecolor{currentstroke}{rgb}{0.690196,0.690196,0.690196}%
\pgfsetstrokecolor{currentstroke}%
\pgfsetdash{}{0pt}%
\pgfpathmoveto{\pgfqpoint{0.950000in}{0.440000in}}%
\pgfpathlineto{\pgfqpoint{0.950000in}{3.520000in}}%
\pgfusepath{stroke}%
\end{pgfscope}%
\begin{pgfscope}%
\pgfsetbuttcap%
\pgfsetroundjoin%
\definecolor{currentfill}{rgb}{0.000000,0.000000,0.000000}%
\pgfsetfillcolor{currentfill}%
\pgfsetlinewidth{0.803000pt}%
\definecolor{currentstroke}{rgb}{0.000000,0.000000,0.000000}%
\pgfsetstrokecolor{currentstroke}%
\pgfsetdash{}{0pt}%
\pgfsys@defobject{currentmarker}{\pgfqpoint{0.000000in}{-0.048611in}}{\pgfqpoint{0.000000in}{0.000000in}}{%
\pgfpathmoveto{\pgfqpoint{0.000000in}{0.000000in}}%
\pgfpathlineto{\pgfqpoint{0.000000in}{-0.048611in}}%
\pgfusepath{stroke,fill}%
}%
\begin{pgfscope}%
\pgfsys@transformshift{0.950000in}{0.440000in}%
\pgfsys@useobject{currentmarker}{}%
\end{pgfscope}%
\end{pgfscope}%
\begin{pgfscope}%
\definecolor{textcolor}{rgb}{0.000000,0.000000,0.000000}%
\pgfsetstrokecolor{textcolor}%
\pgfsetfillcolor{textcolor}%
\pgftext[x=0.950000in,y=0.342778in,,top]{\color{textcolor}\sffamily\fontsize{10.000000}{12.000000}\selectfont 0.0}%
\end{pgfscope}%
\begin{pgfscope}%
\pgfpathrectangle{\pgfqpoint{0.510000in}{0.440000in}}{\pgfqpoint{3.080000in}{3.080000in}}%
\pgfusepath{clip}%
\pgfsetrectcap%
\pgfsetroundjoin%
\pgfsetlinewidth{0.803000pt}%
\definecolor{currentstroke}{rgb}{0.690196,0.690196,0.690196}%
\pgfsetstrokecolor{currentstroke}%
\pgfsetdash{}{0pt}%
\pgfpathmoveto{\pgfqpoint{1.390000in}{0.440000in}}%
\pgfpathlineto{\pgfqpoint{1.390000in}{3.520000in}}%
\pgfusepath{stroke}%
\end{pgfscope}%
\begin{pgfscope}%
\pgfsetbuttcap%
\pgfsetroundjoin%
\definecolor{currentfill}{rgb}{0.000000,0.000000,0.000000}%
\pgfsetfillcolor{currentfill}%
\pgfsetlinewidth{0.803000pt}%
\definecolor{currentstroke}{rgb}{0.000000,0.000000,0.000000}%
\pgfsetstrokecolor{currentstroke}%
\pgfsetdash{}{0pt}%
\pgfsys@defobject{currentmarker}{\pgfqpoint{0.000000in}{-0.048611in}}{\pgfqpoint{0.000000in}{0.000000in}}{%
\pgfpathmoveto{\pgfqpoint{0.000000in}{0.000000in}}%
\pgfpathlineto{\pgfqpoint{0.000000in}{-0.048611in}}%
\pgfusepath{stroke,fill}%
}%
\begin{pgfscope}%
\pgfsys@transformshift{1.390000in}{0.440000in}%
\pgfsys@useobject{currentmarker}{}%
\end{pgfscope}%
\end{pgfscope}%
\begin{pgfscope}%
\definecolor{textcolor}{rgb}{0.000000,0.000000,0.000000}%
\pgfsetstrokecolor{textcolor}%
\pgfsetfillcolor{textcolor}%
\pgftext[x=1.390000in,y=0.342778in,,top]{\color{textcolor}\sffamily\fontsize{10.000000}{12.000000}\selectfont 0.2}%
\end{pgfscope}%
\begin{pgfscope}%
\pgfpathrectangle{\pgfqpoint{0.510000in}{0.440000in}}{\pgfqpoint{3.080000in}{3.080000in}}%
\pgfusepath{clip}%
\pgfsetrectcap%
\pgfsetroundjoin%
\pgfsetlinewidth{0.803000pt}%
\definecolor{currentstroke}{rgb}{0.690196,0.690196,0.690196}%
\pgfsetstrokecolor{currentstroke}%
\pgfsetdash{}{0pt}%
\pgfpathmoveto{\pgfqpoint{1.830000in}{0.440000in}}%
\pgfpathlineto{\pgfqpoint{1.830000in}{3.520000in}}%
\pgfusepath{stroke}%
\end{pgfscope}%
\begin{pgfscope}%
\pgfsetbuttcap%
\pgfsetroundjoin%
\definecolor{currentfill}{rgb}{0.000000,0.000000,0.000000}%
\pgfsetfillcolor{currentfill}%
\pgfsetlinewidth{0.803000pt}%
\definecolor{currentstroke}{rgb}{0.000000,0.000000,0.000000}%
\pgfsetstrokecolor{currentstroke}%
\pgfsetdash{}{0pt}%
\pgfsys@defobject{currentmarker}{\pgfqpoint{0.000000in}{-0.048611in}}{\pgfqpoint{0.000000in}{0.000000in}}{%
\pgfpathmoveto{\pgfqpoint{0.000000in}{0.000000in}}%
\pgfpathlineto{\pgfqpoint{0.000000in}{-0.048611in}}%
\pgfusepath{stroke,fill}%
}%
\begin{pgfscope}%
\pgfsys@transformshift{1.830000in}{0.440000in}%
\pgfsys@useobject{currentmarker}{}%
\end{pgfscope}%
\end{pgfscope}%
\begin{pgfscope}%
\definecolor{textcolor}{rgb}{0.000000,0.000000,0.000000}%
\pgfsetstrokecolor{textcolor}%
\pgfsetfillcolor{textcolor}%
\pgftext[x=1.830000in,y=0.342778in,,top]{\color{textcolor}\sffamily\fontsize{10.000000}{12.000000}\selectfont 0.4}%
\end{pgfscope}%
\begin{pgfscope}%
\pgfpathrectangle{\pgfqpoint{0.510000in}{0.440000in}}{\pgfqpoint{3.080000in}{3.080000in}}%
\pgfusepath{clip}%
\pgfsetrectcap%
\pgfsetroundjoin%
\pgfsetlinewidth{0.803000pt}%
\definecolor{currentstroke}{rgb}{0.690196,0.690196,0.690196}%
\pgfsetstrokecolor{currentstroke}%
\pgfsetdash{}{0pt}%
\pgfpathmoveto{\pgfqpoint{2.270000in}{0.440000in}}%
\pgfpathlineto{\pgfqpoint{2.270000in}{3.520000in}}%
\pgfusepath{stroke}%
\end{pgfscope}%
\begin{pgfscope}%
\pgfsetbuttcap%
\pgfsetroundjoin%
\definecolor{currentfill}{rgb}{0.000000,0.000000,0.000000}%
\pgfsetfillcolor{currentfill}%
\pgfsetlinewidth{0.803000pt}%
\definecolor{currentstroke}{rgb}{0.000000,0.000000,0.000000}%
\pgfsetstrokecolor{currentstroke}%
\pgfsetdash{}{0pt}%
\pgfsys@defobject{currentmarker}{\pgfqpoint{0.000000in}{-0.048611in}}{\pgfqpoint{0.000000in}{0.000000in}}{%
\pgfpathmoveto{\pgfqpoint{0.000000in}{0.000000in}}%
\pgfpathlineto{\pgfqpoint{0.000000in}{-0.048611in}}%
\pgfusepath{stroke,fill}%
}%
\begin{pgfscope}%
\pgfsys@transformshift{2.270000in}{0.440000in}%
\pgfsys@useobject{currentmarker}{}%
\end{pgfscope}%
\end{pgfscope}%
\begin{pgfscope}%
\definecolor{textcolor}{rgb}{0.000000,0.000000,0.000000}%
\pgfsetstrokecolor{textcolor}%
\pgfsetfillcolor{textcolor}%
\pgftext[x=2.270000in,y=0.342778in,,top]{\color{textcolor}\sffamily\fontsize{10.000000}{12.000000}\selectfont 0.6}%
\end{pgfscope}%
\begin{pgfscope}%
\pgfpathrectangle{\pgfqpoint{0.510000in}{0.440000in}}{\pgfqpoint{3.080000in}{3.080000in}}%
\pgfusepath{clip}%
\pgfsetrectcap%
\pgfsetroundjoin%
\pgfsetlinewidth{0.803000pt}%
\definecolor{currentstroke}{rgb}{0.690196,0.690196,0.690196}%
\pgfsetstrokecolor{currentstroke}%
\pgfsetdash{}{0pt}%
\pgfpathmoveto{\pgfqpoint{2.710000in}{0.440000in}}%
\pgfpathlineto{\pgfqpoint{2.710000in}{3.520000in}}%
\pgfusepath{stroke}%
\end{pgfscope}%
\begin{pgfscope}%
\pgfsetbuttcap%
\pgfsetroundjoin%
\definecolor{currentfill}{rgb}{0.000000,0.000000,0.000000}%
\pgfsetfillcolor{currentfill}%
\pgfsetlinewidth{0.803000pt}%
\definecolor{currentstroke}{rgb}{0.000000,0.000000,0.000000}%
\pgfsetstrokecolor{currentstroke}%
\pgfsetdash{}{0pt}%
\pgfsys@defobject{currentmarker}{\pgfqpoint{0.000000in}{-0.048611in}}{\pgfqpoint{0.000000in}{0.000000in}}{%
\pgfpathmoveto{\pgfqpoint{0.000000in}{0.000000in}}%
\pgfpathlineto{\pgfqpoint{0.000000in}{-0.048611in}}%
\pgfusepath{stroke,fill}%
}%
\begin{pgfscope}%
\pgfsys@transformshift{2.710000in}{0.440000in}%
\pgfsys@useobject{currentmarker}{}%
\end{pgfscope}%
\end{pgfscope}%
\begin{pgfscope}%
\definecolor{textcolor}{rgb}{0.000000,0.000000,0.000000}%
\pgfsetstrokecolor{textcolor}%
\pgfsetfillcolor{textcolor}%
\pgftext[x=2.710000in,y=0.342778in,,top]{\color{textcolor}\sffamily\fontsize{10.000000}{12.000000}\selectfont 0.8}%
\end{pgfscope}%
\begin{pgfscope}%
\pgfpathrectangle{\pgfqpoint{0.510000in}{0.440000in}}{\pgfqpoint{3.080000in}{3.080000in}}%
\pgfusepath{clip}%
\pgfsetrectcap%
\pgfsetroundjoin%
\pgfsetlinewidth{0.803000pt}%
\definecolor{currentstroke}{rgb}{0.690196,0.690196,0.690196}%
\pgfsetstrokecolor{currentstroke}%
\pgfsetdash{}{0pt}%
\pgfpathmoveto{\pgfqpoint{3.150000in}{0.440000in}}%
\pgfpathlineto{\pgfqpoint{3.150000in}{3.520000in}}%
\pgfusepath{stroke}%
\end{pgfscope}%
\begin{pgfscope}%
\pgfsetbuttcap%
\pgfsetroundjoin%
\definecolor{currentfill}{rgb}{0.000000,0.000000,0.000000}%
\pgfsetfillcolor{currentfill}%
\pgfsetlinewidth{0.803000pt}%
\definecolor{currentstroke}{rgb}{0.000000,0.000000,0.000000}%
\pgfsetstrokecolor{currentstroke}%
\pgfsetdash{}{0pt}%
\pgfsys@defobject{currentmarker}{\pgfqpoint{0.000000in}{-0.048611in}}{\pgfqpoint{0.000000in}{0.000000in}}{%
\pgfpathmoveto{\pgfqpoint{0.000000in}{0.000000in}}%
\pgfpathlineto{\pgfqpoint{0.000000in}{-0.048611in}}%
\pgfusepath{stroke,fill}%
}%
\begin{pgfscope}%
\pgfsys@transformshift{3.150000in}{0.440000in}%
\pgfsys@useobject{currentmarker}{}%
\end{pgfscope}%
\end{pgfscope}%
\begin{pgfscope}%
\definecolor{textcolor}{rgb}{0.000000,0.000000,0.000000}%
\pgfsetstrokecolor{textcolor}%
\pgfsetfillcolor{textcolor}%
\pgftext[x=3.150000in,y=0.342778in,,top]{\color{textcolor}\sffamily\fontsize{10.000000}{12.000000}\selectfont 1.0}%
\end{pgfscope}%
\begin{pgfscope}%
\pgfpathrectangle{\pgfqpoint{0.510000in}{0.440000in}}{\pgfqpoint{3.080000in}{3.080000in}}%
\pgfusepath{clip}%
\pgfsetrectcap%
\pgfsetroundjoin%
\pgfsetlinewidth{0.803000pt}%
\definecolor{currentstroke}{rgb}{0.690196,0.690196,0.690196}%
\pgfsetstrokecolor{currentstroke}%
\pgfsetdash{}{0pt}%
\pgfpathmoveto{\pgfqpoint{3.590000in}{0.440000in}}%
\pgfpathlineto{\pgfqpoint{3.590000in}{3.520000in}}%
\pgfusepath{stroke}%
\end{pgfscope}%
\begin{pgfscope}%
\pgfsetbuttcap%
\pgfsetroundjoin%
\definecolor{currentfill}{rgb}{0.000000,0.000000,0.000000}%
\pgfsetfillcolor{currentfill}%
\pgfsetlinewidth{0.803000pt}%
\definecolor{currentstroke}{rgb}{0.000000,0.000000,0.000000}%
\pgfsetstrokecolor{currentstroke}%
\pgfsetdash{}{0pt}%
\pgfsys@defobject{currentmarker}{\pgfqpoint{0.000000in}{-0.048611in}}{\pgfqpoint{0.000000in}{0.000000in}}{%
\pgfpathmoveto{\pgfqpoint{0.000000in}{0.000000in}}%
\pgfpathlineto{\pgfqpoint{0.000000in}{-0.048611in}}%
\pgfusepath{stroke,fill}%
}%
\begin{pgfscope}%
\pgfsys@transformshift{3.590000in}{0.440000in}%
\pgfsys@useobject{currentmarker}{}%
\end{pgfscope}%
\end{pgfscope}%
\begin{pgfscope}%
\definecolor{textcolor}{rgb}{0.000000,0.000000,0.000000}%
\pgfsetstrokecolor{textcolor}%
\pgfsetfillcolor{textcolor}%
\pgftext[x=3.590000in,y=0.342778in,,top]{\color{textcolor}\sffamily\fontsize{10.000000}{12.000000}\selectfont 1.2}%
\end{pgfscope}%
\begin{pgfscope}%
\pgfpathrectangle{\pgfqpoint{0.510000in}{0.440000in}}{\pgfqpoint{3.080000in}{3.080000in}}%
\pgfusepath{clip}%
\pgfsetrectcap%
\pgfsetroundjoin%
\pgfsetlinewidth{0.803000pt}%
\definecolor{currentstroke}{rgb}{0.690196,0.690196,0.690196}%
\pgfsetstrokecolor{currentstroke}%
\pgfsetdash{}{0pt}%
\pgfpathmoveto{\pgfqpoint{0.510000in}{0.440000in}}%
\pgfpathlineto{\pgfqpoint{3.590000in}{0.440000in}}%
\pgfusepath{stroke}%
\end{pgfscope}%
\begin{pgfscope}%
\pgfsetbuttcap%
\pgfsetroundjoin%
\definecolor{currentfill}{rgb}{0.000000,0.000000,0.000000}%
\pgfsetfillcolor{currentfill}%
\pgfsetlinewidth{0.803000pt}%
\definecolor{currentstroke}{rgb}{0.000000,0.000000,0.000000}%
\pgfsetstrokecolor{currentstroke}%
\pgfsetdash{}{0pt}%
\pgfsys@defobject{currentmarker}{\pgfqpoint{-0.048611in}{0.000000in}}{\pgfqpoint{0.000000in}{0.000000in}}{%
\pgfpathmoveto{\pgfqpoint{0.000000in}{0.000000in}}%
\pgfpathlineto{\pgfqpoint{-0.048611in}{0.000000in}}%
\pgfusepath{stroke,fill}%
}%
\begin{pgfscope}%
\pgfsys@transformshift{0.510000in}{0.440000in}%
\pgfsys@useobject{currentmarker}{}%
\end{pgfscope}%
\end{pgfscope}%
\begin{pgfscope}%
\definecolor{textcolor}{rgb}{0.000000,0.000000,0.000000}%
\pgfsetstrokecolor{textcolor}%
\pgfsetfillcolor{textcolor}%
\pgftext[x=0.075525in,y=0.387238in,left,base]{\color{textcolor}\sffamily\fontsize{10.000000}{12.000000}\selectfont -0.2}%
\end{pgfscope}%
\begin{pgfscope}%
\pgfpathrectangle{\pgfqpoint{0.510000in}{0.440000in}}{\pgfqpoint{3.080000in}{3.080000in}}%
\pgfusepath{clip}%
\pgfsetrectcap%
\pgfsetroundjoin%
\pgfsetlinewidth{0.803000pt}%
\definecolor{currentstroke}{rgb}{0.690196,0.690196,0.690196}%
\pgfsetstrokecolor{currentstroke}%
\pgfsetdash{}{0pt}%
\pgfpathmoveto{\pgfqpoint{0.510000in}{0.880000in}}%
\pgfpathlineto{\pgfqpoint{3.590000in}{0.880000in}}%
\pgfusepath{stroke}%
\end{pgfscope}%
\begin{pgfscope}%
\pgfsetbuttcap%
\pgfsetroundjoin%
\definecolor{currentfill}{rgb}{0.000000,0.000000,0.000000}%
\pgfsetfillcolor{currentfill}%
\pgfsetlinewidth{0.803000pt}%
\definecolor{currentstroke}{rgb}{0.000000,0.000000,0.000000}%
\pgfsetstrokecolor{currentstroke}%
\pgfsetdash{}{0pt}%
\pgfsys@defobject{currentmarker}{\pgfqpoint{-0.048611in}{0.000000in}}{\pgfqpoint{0.000000in}{0.000000in}}{%
\pgfpathmoveto{\pgfqpoint{0.000000in}{0.000000in}}%
\pgfpathlineto{\pgfqpoint{-0.048611in}{0.000000in}}%
\pgfusepath{stroke,fill}%
}%
\begin{pgfscope}%
\pgfsys@transformshift{0.510000in}{0.880000in}%
\pgfsys@useobject{currentmarker}{}%
\end{pgfscope}%
\end{pgfscope}%
\begin{pgfscope}%
\definecolor{textcolor}{rgb}{0.000000,0.000000,0.000000}%
\pgfsetstrokecolor{textcolor}%
\pgfsetfillcolor{textcolor}%
\pgftext[x=0.191898in,y=0.827238in,left,base]{\color{textcolor}\sffamily\fontsize{10.000000}{12.000000}\selectfont 0.0}%
\end{pgfscope}%
\begin{pgfscope}%
\pgfpathrectangle{\pgfqpoint{0.510000in}{0.440000in}}{\pgfqpoint{3.080000in}{3.080000in}}%
\pgfusepath{clip}%
\pgfsetrectcap%
\pgfsetroundjoin%
\pgfsetlinewidth{0.803000pt}%
\definecolor{currentstroke}{rgb}{0.690196,0.690196,0.690196}%
\pgfsetstrokecolor{currentstroke}%
\pgfsetdash{}{0pt}%
\pgfpathmoveto{\pgfqpoint{0.510000in}{1.320000in}}%
\pgfpathlineto{\pgfqpoint{3.590000in}{1.320000in}}%
\pgfusepath{stroke}%
\end{pgfscope}%
\begin{pgfscope}%
\pgfsetbuttcap%
\pgfsetroundjoin%
\definecolor{currentfill}{rgb}{0.000000,0.000000,0.000000}%
\pgfsetfillcolor{currentfill}%
\pgfsetlinewidth{0.803000pt}%
\definecolor{currentstroke}{rgb}{0.000000,0.000000,0.000000}%
\pgfsetstrokecolor{currentstroke}%
\pgfsetdash{}{0pt}%
\pgfsys@defobject{currentmarker}{\pgfqpoint{-0.048611in}{0.000000in}}{\pgfqpoint{0.000000in}{0.000000in}}{%
\pgfpathmoveto{\pgfqpoint{0.000000in}{0.000000in}}%
\pgfpathlineto{\pgfqpoint{-0.048611in}{0.000000in}}%
\pgfusepath{stroke,fill}%
}%
\begin{pgfscope}%
\pgfsys@transformshift{0.510000in}{1.320000in}%
\pgfsys@useobject{currentmarker}{}%
\end{pgfscope}%
\end{pgfscope}%
\begin{pgfscope}%
\definecolor{textcolor}{rgb}{0.000000,0.000000,0.000000}%
\pgfsetstrokecolor{textcolor}%
\pgfsetfillcolor{textcolor}%
\pgftext[x=0.191898in,y=1.267238in,left,base]{\color{textcolor}\sffamily\fontsize{10.000000}{12.000000}\selectfont 0.2}%
\end{pgfscope}%
\begin{pgfscope}%
\pgfpathrectangle{\pgfqpoint{0.510000in}{0.440000in}}{\pgfqpoint{3.080000in}{3.080000in}}%
\pgfusepath{clip}%
\pgfsetrectcap%
\pgfsetroundjoin%
\pgfsetlinewidth{0.803000pt}%
\definecolor{currentstroke}{rgb}{0.690196,0.690196,0.690196}%
\pgfsetstrokecolor{currentstroke}%
\pgfsetdash{}{0pt}%
\pgfpathmoveto{\pgfqpoint{0.510000in}{1.760000in}}%
\pgfpathlineto{\pgfqpoint{3.590000in}{1.760000in}}%
\pgfusepath{stroke}%
\end{pgfscope}%
\begin{pgfscope}%
\pgfsetbuttcap%
\pgfsetroundjoin%
\definecolor{currentfill}{rgb}{0.000000,0.000000,0.000000}%
\pgfsetfillcolor{currentfill}%
\pgfsetlinewidth{0.803000pt}%
\definecolor{currentstroke}{rgb}{0.000000,0.000000,0.000000}%
\pgfsetstrokecolor{currentstroke}%
\pgfsetdash{}{0pt}%
\pgfsys@defobject{currentmarker}{\pgfqpoint{-0.048611in}{0.000000in}}{\pgfqpoint{0.000000in}{0.000000in}}{%
\pgfpathmoveto{\pgfqpoint{0.000000in}{0.000000in}}%
\pgfpathlineto{\pgfqpoint{-0.048611in}{0.000000in}}%
\pgfusepath{stroke,fill}%
}%
\begin{pgfscope}%
\pgfsys@transformshift{0.510000in}{1.760000in}%
\pgfsys@useobject{currentmarker}{}%
\end{pgfscope}%
\end{pgfscope}%
\begin{pgfscope}%
\definecolor{textcolor}{rgb}{0.000000,0.000000,0.000000}%
\pgfsetstrokecolor{textcolor}%
\pgfsetfillcolor{textcolor}%
\pgftext[x=0.191898in,y=1.707238in,left,base]{\color{textcolor}\sffamily\fontsize{10.000000}{12.000000}\selectfont 0.4}%
\end{pgfscope}%
\begin{pgfscope}%
\pgfpathrectangle{\pgfqpoint{0.510000in}{0.440000in}}{\pgfqpoint{3.080000in}{3.080000in}}%
\pgfusepath{clip}%
\pgfsetrectcap%
\pgfsetroundjoin%
\pgfsetlinewidth{0.803000pt}%
\definecolor{currentstroke}{rgb}{0.690196,0.690196,0.690196}%
\pgfsetstrokecolor{currentstroke}%
\pgfsetdash{}{0pt}%
\pgfpathmoveto{\pgfqpoint{0.510000in}{2.200000in}}%
\pgfpathlineto{\pgfqpoint{3.590000in}{2.200000in}}%
\pgfusepath{stroke}%
\end{pgfscope}%
\begin{pgfscope}%
\pgfsetbuttcap%
\pgfsetroundjoin%
\definecolor{currentfill}{rgb}{0.000000,0.000000,0.000000}%
\pgfsetfillcolor{currentfill}%
\pgfsetlinewidth{0.803000pt}%
\definecolor{currentstroke}{rgb}{0.000000,0.000000,0.000000}%
\pgfsetstrokecolor{currentstroke}%
\pgfsetdash{}{0pt}%
\pgfsys@defobject{currentmarker}{\pgfqpoint{-0.048611in}{0.000000in}}{\pgfqpoint{0.000000in}{0.000000in}}{%
\pgfpathmoveto{\pgfqpoint{0.000000in}{0.000000in}}%
\pgfpathlineto{\pgfqpoint{-0.048611in}{0.000000in}}%
\pgfusepath{stroke,fill}%
}%
\begin{pgfscope}%
\pgfsys@transformshift{0.510000in}{2.200000in}%
\pgfsys@useobject{currentmarker}{}%
\end{pgfscope}%
\end{pgfscope}%
\begin{pgfscope}%
\definecolor{textcolor}{rgb}{0.000000,0.000000,0.000000}%
\pgfsetstrokecolor{textcolor}%
\pgfsetfillcolor{textcolor}%
\pgftext[x=0.191898in,y=2.147238in,left,base]{\color{textcolor}\sffamily\fontsize{10.000000}{12.000000}\selectfont 0.6}%
\end{pgfscope}%
\begin{pgfscope}%
\pgfpathrectangle{\pgfqpoint{0.510000in}{0.440000in}}{\pgfqpoint{3.080000in}{3.080000in}}%
\pgfusepath{clip}%
\pgfsetrectcap%
\pgfsetroundjoin%
\pgfsetlinewidth{0.803000pt}%
\definecolor{currentstroke}{rgb}{0.690196,0.690196,0.690196}%
\pgfsetstrokecolor{currentstroke}%
\pgfsetdash{}{0pt}%
\pgfpathmoveto{\pgfqpoint{0.510000in}{2.640000in}}%
\pgfpathlineto{\pgfqpoint{3.590000in}{2.640000in}}%
\pgfusepath{stroke}%
\end{pgfscope}%
\begin{pgfscope}%
\pgfsetbuttcap%
\pgfsetroundjoin%
\definecolor{currentfill}{rgb}{0.000000,0.000000,0.000000}%
\pgfsetfillcolor{currentfill}%
\pgfsetlinewidth{0.803000pt}%
\definecolor{currentstroke}{rgb}{0.000000,0.000000,0.000000}%
\pgfsetstrokecolor{currentstroke}%
\pgfsetdash{}{0pt}%
\pgfsys@defobject{currentmarker}{\pgfqpoint{-0.048611in}{0.000000in}}{\pgfqpoint{0.000000in}{0.000000in}}{%
\pgfpathmoveto{\pgfqpoint{0.000000in}{0.000000in}}%
\pgfpathlineto{\pgfqpoint{-0.048611in}{0.000000in}}%
\pgfusepath{stroke,fill}%
}%
\begin{pgfscope}%
\pgfsys@transformshift{0.510000in}{2.640000in}%
\pgfsys@useobject{currentmarker}{}%
\end{pgfscope}%
\end{pgfscope}%
\begin{pgfscope}%
\definecolor{textcolor}{rgb}{0.000000,0.000000,0.000000}%
\pgfsetstrokecolor{textcolor}%
\pgfsetfillcolor{textcolor}%
\pgftext[x=0.191898in,y=2.587238in,left,base]{\color{textcolor}\sffamily\fontsize{10.000000}{12.000000}\selectfont 0.8}%
\end{pgfscope}%
\begin{pgfscope}%
\pgfpathrectangle{\pgfqpoint{0.510000in}{0.440000in}}{\pgfqpoint{3.080000in}{3.080000in}}%
\pgfusepath{clip}%
\pgfsetrectcap%
\pgfsetroundjoin%
\pgfsetlinewidth{0.803000pt}%
\definecolor{currentstroke}{rgb}{0.690196,0.690196,0.690196}%
\pgfsetstrokecolor{currentstroke}%
\pgfsetdash{}{0pt}%
\pgfpathmoveto{\pgfqpoint{0.510000in}{3.080000in}}%
\pgfpathlineto{\pgfqpoint{3.590000in}{3.080000in}}%
\pgfusepath{stroke}%
\end{pgfscope}%
\begin{pgfscope}%
\pgfsetbuttcap%
\pgfsetroundjoin%
\definecolor{currentfill}{rgb}{0.000000,0.000000,0.000000}%
\pgfsetfillcolor{currentfill}%
\pgfsetlinewidth{0.803000pt}%
\definecolor{currentstroke}{rgb}{0.000000,0.000000,0.000000}%
\pgfsetstrokecolor{currentstroke}%
\pgfsetdash{}{0pt}%
\pgfsys@defobject{currentmarker}{\pgfqpoint{-0.048611in}{0.000000in}}{\pgfqpoint{0.000000in}{0.000000in}}{%
\pgfpathmoveto{\pgfqpoint{0.000000in}{0.000000in}}%
\pgfpathlineto{\pgfqpoint{-0.048611in}{0.000000in}}%
\pgfusepath{stroke,fill}%
}%
\begin{pgfscope}%
\pgfsys@transformshift{0.510000in}{3.080000in}%
\pgfsys@useobject{currentmarker}{}%
\end{pgfscope}%
\end{pgfscope}%
\begin{pgfscope}%
\definecolor{textcolor}{rgb}{0.000000,0.000000,0.000000}%
\pgfsetstrokecolor{textcolor}%
\pgfsetfillcolor{textcolor}%
\pgftext[x=0.191898in,y=3.027238in,left,base]{\color{textcolor}\sffamily\fontsize{10.000000}{12.000000}\selectfont 1.0}%
\end{pgfscope}%
\begin{pgfscope}%
\pgfpathrectangle{\pgfqpoint{0.510000in}{0.440000in}}{\pgfqpoint{3.080000in}{3.080000in}}%
\pgfusepath{clip}%
\pgfsetrectcap%
\pgfsetroundjoin%
\pgfsetlinewidth{0.803000pt}%
\definecolor{currentstroke}{rgb}{0.690196,0.690196,0.690196}%
\pgfsetstrokecolor{currentstroke}%
\pgfsetdash{}{0pt}%
\pgfpathmoveto{\pgfqpoint{0.510000in}{3.520000in}}%
\pgfpathlineto{\pgfqpoint{3.590000in}{3.520000in}}%
\pgfusepath{stroke}%
\end{pgfscope}%
\begin{pgfscope}%
\pgfsetbuttcap%
\pgfsetroundjoin%
\definecolor{currentfill}{rgb}{0.000000,0.000000,0.000000}%
\pgfsetfillcolor{currentfill}%
\pgfsetlinewidth{0.803000pt}%
\definecolor{currentstroke}{rgb}{0.000000,0.000000,0.000000}%
\pgfsetstrokecolor{currentstroke}%
\pgfsetdash{}{0pt}%
\pgfsys@defobject{currentmarker}{\pgfqpoint{-0.048611in}{0.000000in}}{\pgfqpoint{0.000000in}{0.000000in}}{%
\pgfpathmoveto{\pgfqpoint{0.000000in}{0.000000in}}%
\pgfpathlineto{\pgfqpoint{-0.048611in}{0.000000in}}%
\pgfusepath{stroke,fill}%
}%
\begin{pgfscope}%
\pgfsys@transformshift{0.510000in}{3.520000in}%
\pgfsys@useobject{currentmarker}{}%
\end{pgfscope}%
\end{pgfscope}%
\begin{pgfscope}%
\definecolor{textcolor}{rgb}{0.000000,0.000000,0.000000}%
\pgfsetstrokecolor{textcolor}%
\pgfsetfillcolor{textcolor}%
\pgftext[x=0.191898in,y=3.467238in,left,base]{\color{textcolor}\sffamily\fontsize{10.000000}{12.000000}\selectfont 1.2}%
\end{pgfscope}%
\begin{pgfscope}%
\pgfsetrectcap%
\pgfsetmiterjoin%
\pgfsetlinewidth{0.803000pt}%
\definecolor{currentstroke}{rgb}{0.000000,0.000000,0.000000}%
\pgfsetstrokecolor{currentstroke}%
\pgfsetdash{}{0pt}%
\pgfpathmoveto{\pgfqpoint{0.510000in}{0.440000in}}%
\pgfpathlineto{\pgfqpoint{0.510000in}{3.520000in}}%
\pgfusepath{stroke}%
\end{pgfscope}%
\begin{pgfscope}%
\pgfsetrectcap%
\pgfsetmiterjoin%
\pgfsetlinewidth{0.803000pt}%
\definecolor{currentstroke}{rgb}{0.000000,0.000000,0.000000}%
\pgfsetstrokecolor{currentstroke}%
\pgfsetdash{}{0pt}%
\pgfpathmoveto{\pgfqpoint{3.590000in}{0.440000in}}%
\pgfpathlineto{\pgfqpoint{3.590000in}{3.520000in}}%
\pgfusepath{stroke}%
\end{pgfscope}%
\begin{pgfscope}%
\pgfsetrectcap%
\pgfsetmiterjoin%
\pgfsetlinewidth{0.803000pt}%
\definecolor{currentstroke}{rgb}{0.000000,0.000000,0.000000}%
\pgfsetstrokecolor{currentstroke}%
\pgfsetdash{}{0pt}%
\pgfpathmoveto{\pgfqpoint{0.510000in}{0.440000in}}%
\pgfpathlineto{\pgfqpoint{3.590000in}{0.440000in}}%
\pgfusepath{stroke}%
\end{pgfscope}%
\begin{pgfscope}%
\pgfsetrectcap%
\pgfsetmiterjoin%
\pgfsetlinewidth{0.803000pt}%
\definecolor{currentstroke}{rgb}{0.000000,0.000000,0.000000}%
\pgfsetstrokecolor{currentstroke}%
\pgfsetdash{}{0pt}%
\pgfpathmoveto{\pgfqpoint{0.510000in}{3.520000in}}%
\pgfpathlineto{\pgfqpoint{3.590000in}{3.520000in}}%
\pgfusepath{stroke}%
\end{pgfscope}%
\begin{pgfscope}%
\definecolor{textcolor}{rgb}{0.000000,0.000000,0.000000}%
\pgfsetstrokecolor{textcolor}%
\pgfsetfillcolor{textcolor}%
\pgftext[x=1.230473in,y=1.839763in,left,base]{\color{textcolor}\sffamily\fontsize{10.000000}{12.000000}\selectfont 0.17679}%
\end{pgfscope}%
\begin{pgfscope}%
\definecolor{textcolor}{rgb}{0.000000,0.000000,0.000000}%
\pgfsetstrokecolor{textcolor}%
\pgfsetfillcolor{textcolor}%
\pgftext[x=2.379121in,y=1.265440in,left,base]{\color{textcolor}\sffamily\fontsize{10.000000}{12.000000}\selectfont 0.17679}%
\end{pgfscope}%
\begin{pgfscope}%
\definecolor{textcolor}{rgb}{0.000000,0.000000,0.000000}%
\pgfsetstrokecolor{textcolor}%
\pgfsetfillcolor{textcolor}%
\pgftext[x=1.390000in,y=0.930127in,left,base]{\color{textcolor}\sffamily\fontsize{10.000000}{12.000000}\selectfont 0.07321}%
\end{pgfscope}%
\begin{pgfscope}%
\definecolor{textcolor}{rgb}{0.000000,0.000000,0.000000}%
\pgfsetstrokecolor{textcolor}%
\pgfsetfillcolor{textcolor}%
\pgftext[x=1.390000in,y=2.589873in,left,base]{\color{textcolor}\sffamily\fontsize{10.000000}{12.000000}\selectfont 0.07321}%
\end{pgfscope}%
\end{pgfpicture}%
\makeatother%
\endgroup%
}
\caption{Figures of the two-dimensional cubatures of $ D = 3 $}
\label{Fig:M2D3}
\end{figure}

In this case, the smallest $N$ is $4$.

\subsection{The three-dimensional case}

Here $ M = 3 $. We take the notation $ \mathbf{x} = \rbr{ x, y, z } $.

\subsubsection{The case $ D = 1 $}

The case $ D = 1 $ can be analytically computed. By using $\mathcal{I}_1$ as constraints, we have $ L = 3 $ and we take $ N = 1 $. The only choice for groups is $\sbr{1}$ with $ C = 1 $. The equations turns out to be
\begin{gather}
w_1 = 1 / 6, \\
w_1 x_1 = 1 / 24, \\
w_1 y_1 = 1 / 24, \\
w_1 z_1 = 1 / 24
\end{gather}
The solution is $ \rbr{ x_1, y_1, z_1, w_1 } = \rbr{ 1 / 4, 1 / 4, 1 / 4, 1 / 6 } $. Hence, we have
\begin{equation}
\hat{I} (\hat{f}) = \frac{1}{6} \hat{f} \rbr{ \frac{1}{4}, \frac{1}{4}, \frac{1}{4} },
\end{equation}
and
\begin{equation}
I \rbr{f} = \abs{S} \hat{f} \rbr{ \frac{1}{4} \rbr{ \mathbf{v}_0 + \mathbf{v}_1 + \mathbf{v}_2 } }.
\end{equation}

We apply $ \text{N} \rbr{30} $ to solve the problem directly. The numerical algorithm succeeds to yield the correct solution. The figure is given in Figure \ref{Fig:M3D1}.

\begin{figure}[htbp]
\centering
\scalebox{0.75}{%% Creator: Matplotlib, PGF backend
%%
%% To include the figure in your LaTeX document, write
%%   \input{<filename>.pgf}
%%
%% Make sure the required packages are loaded in your preamble
%%   \usepackage{pgf}
%%
%% Figures using additional raster images can only be included by \input if
%% they are in the same directory as the main LaTeX file. For loading figures
%% from other directories you can use the `import` package
%%   \usepackage{import}
%% and then include the figures with
%%   \import{<path to file>}{<filename>.pgf}
%%
%% Matplotlib used the following preamble
%%   \usepackage{fontspec}
%%   \setmainfont{DejaVuSerif.ttf}[Path=/home/lzh/anaconda3/envs/numana/lib/python3.7/site-packages/matplotlib/mpl-data/fonts/ttf/]
%%   \setsansfont{DejaVuSans.ttf}[Path=/home/lzh/anaconda3/envs/numana/lib/python3.7/site-packages/matplotlib/mpl-data/fonts/ttf/]
%%   \setmonofont{DejaVuSansMono.ttf}[Path=/home/lzh/anaconda3/envs/numana/lib/python3.7/site-packages/matplotlib/mpl-data/fonts/ttf/]
%%
\begingroup%
\makeatletter%
\begin{pgfpicture}%
\pgfpathrectangle{\pgfpointorigin}{\pgfqpoint{8.000000in}{6.000000in}}%
\pgfusepath{use as bounding box, clip}%
\begin{pgfscope}%
\pgfsetbuttcap%
\pgfsetmiterjoin%
\definecolor{currentfill}{rgb}{1.000000,1.000000,1.000000}%
\pgfsetfillcolor{currentfill}%
\pgfsetlinewidth{0.000000pt}%
\definecolor{currentstroke}{rgb}{1.000000,1.000000,1.000000}%
\pgfsetstrokecolor{currentstroke}%
\pgfsetdash{}{0pt}%
\pgfpathmoveto{\pgfqpoint{0.000000in}{0.000000in}}%
\pgfpathlineto{\pgfqpoint{8.000000in}{0.000000in}}%
\pgfpathlineto{\pgfqpoint{8.000000in}{6.000000in}}%
\pgfpathlineto{\pgfqpoint{0.000000in}{6.000000in}}%
\pgfpathclose%
\pgfusepath{fill}%
\end{pgfscope}%
\begin{pgfscope}%
\pgfsetbuttcap%
\pgfsetmiterjoin%
\definecolor{currentfill}{rgb}{1.000000,1.000000,1.000000}%
\pgfsetfillcolor{currentfill}%
\pgfsetlinewidth{0.000000pt}%
\definecolor{currentstroke}{rgb}{0.000000,0.000000,0.000000}%
\pgfsetstrokecolor{currentstroke}%
\pgfsetstrokeopacity{0.000000}%
\pgfsetdash{}{0pt}%
\pgfpathmoveto{\pgfqpoint{0.557222in}{3.312222in}}%
\pgfpathlineto{\pgfqpoint{3.879180in}{3.312222in}}%
\pgfpathlineto{\pgfqpoint{3.879180in}{5.636667in}}%
\pgfpathlineto{\pgfqpoint{0.557222in}{5.636667in}}%
\pgfpathclose%
\pgfusepath{fill}%
\end{pgfscope}%
\begin{pgfscope}%
\pgfpathrectangle{\pgfqpoint{0.557222in}{3.312222in}}{\pgfqpoint{3.321958in}{2.324444in}}%
\pgfusepath{clip}%
\pgfsetbuttcap%
\pgfsetroundjoin%
\definecolor{currentfill}{rgb}{0.121569,0.466667,0.705882}%
\pgfsetfillcolor{currentfill}%
\pgfsetfillopacity{0.300000}%
\pgfsetlinewidth{1.003750pt}%
\definecolor{currentstroke}{rgb}{0.121569,0.466667,0.705882}%
\pgfsetstrokecolor{currentstroke}%
\pgfsetstrokeopacity{0.300000}%
\pgfsetdash{}{0pt}%
\pgfpathmoveto{\pgfqpoint{0.716510in}{3.927004in}}%
\pgfpathlineto{\pgfqpoint{0.716510in}{3.530275in}}%
\pgfpathlineto{\pgfqpoint{0.866679in}{3.765637in}}%
\pgfpathlineto{\pgfqpoint{1.016848in}{4.028926in}}%
\pgfpathlineto{\pgfqpoint{1.167017in}{4.195513in}}%
\pgfpathlineto{\pgfqpoint{1.317186in}{4.725234in}}%
\pgfpathlineto{\pgfqpoint{1.467355in}{4.787363in}}%
\pgfpathlineto{\pgfqpoint{1.617525in}{4.905545in}}%
\pgfpathlineto{\pgfqpoint{1.767694in}{5.116690in}}%
\pgfpathlineto{\pgfqpoint{1.917863in}{5.311526in}}%
\pgfpathlineto{\pgfqpoint{2.068032in}{5.313750in}}%
\pgfpathlineto{\pgfqpoint{2.218201in}{5.407325in}}%
\pgfpathlineto{\pgfqpoint{2.368370in}{5.238658in}}%
\pgfpathlineto{\pgfqpoint{2.518540in}{5.185766in}}%
\pgfpathlineto{\pgfqpoint{2.668709in}{5.135260in}}%
\pgfpathlineto{\pgfqpoint{2.818878in}{4.961129in}}%
\pgfpathlineto{\pgfqpoint{2.969047in}{4.823496in}}%
\pgfpathlineto{\pgfqpoint{3.119216in}{4.533526in}}%
\pgfpathlineto{\pgfqpoint{3.269385in}{4.268536in}}%
\pgfpathlineto{\pgfqpoint{3.419554in}{3.972517in}}%
\pgfpathlineto{\pgfqpoint{3.569724in}{3.672376in}}%
\pgfpathlineto{\pgfqpoint{3.719893in}{3.417879in}}%
\pgfpathlineto{\pgfqpoint{3.719893in}{3.509609in}}%
\pgfpathlineto{\pgfqpoint{3.719893in}{3.509609in}}%
\pgfpathlineto{\pgfqpoint{3.569724in}{3.853209in}}%
\pgfpathlineto{\pgfqpoint{3.419554in}{4.033589in}}%
\pgfpathlineto{\pgfqpoint{3.269385in}{4.411342in}}%
\pgfpathlineto{\pgfqpoint{3.119216in}{4.675004in}}%
\pgfpathlineto{\pgfqpoint{2.969047in}{4.927766in}}%
\pgfpathlineto{\pgfqpoint{2.818878in}{5.032149in}}%
\pgfpathlineto{\pgfqpoint{2.668709in}{5.282028in}}%
\pgfpathlineto{\pgfqpoint{2.518540in}{5.389062in}}%
\pgfpathlineto{\pgfqpoint{2.368370in}{5.398471in}}%
\pgfpathlineto{\pgfqpoint{2.218201in}{5.531010in}}%
\pgfpathlineto{\pgfqpoint{2.068032in}{5.446044in}}%
\pgfpathlineto{\pgfqpoint{1.917863in}{5.497914in}}%
\pgfpathlineto{\pgfqpoint{1.767694in}{5.505333in}}%
\pgfpathlineto{\pgfqpoint{1.617525in}{5.143038in}}%
\pgfpathlineto{\pgfqpoint{1.467355in}{4.931744in}}%
\pgfpathlineto{\pgfqpoint{1.317186in}{4.877015in}}%
\pgfpathlineto{\pgfqpoint{1.167017in}{4.657315in}}%
\pgfpathlineto{\pgfqpoint{1.016848in}{4.345455in}}%
\pgfpathlineto{\pgfqpoint{0.866679in}{4.231175in}}%
\pgfpathlineto{\pgfqpoint{0.716510in}{3.927004in}}%
\pgfpathclose%
\pgfusepath{stroke,fill}%
\end{pgfscope}%
\begin{pgfscope}%
\pgfpathrectangle{\pgfqpoint{0.557222in}{3.312222in}}{\pgfqpoint{3.321958in}{2.324444in}}%
\pgfusepath{clip}%
\pgfsetbuttcap%
\pgfsetroundjoin%
\definecolor{currentfill}{rgb}{0.121569,0.466667,0.705882}%
\pgfsetfillcolor{currentfill}%
\pgfsetlinewidth{1.003750pt}%
\definecolor{currentstroke}{rgb}{0.121569,0.466667,0.705882}%
\pgfsetstrokecolor{currentstroke}%
\pgfsetdash{}{0pt}%
\pgfsys@defobject{currentmarker}{\pgfqpoint{-0.009821in}{-0.009821in}}{\pgfqpoint{0.009821in}{0.009821in}}{%
\pgfpathmoveto{\pgfqpoint{0.000000in}{-0.009821in}}%
\pgfpathcurveto{\pgfqpoint{0.002605in}{-0.009821in}}{\pgfqpoint{0.005103in}{-0.008786in}}{\pgfqpoint{0.006944in}{-0.006944in}}%
\pgfpathcurveto{\pgfqpoint{0.008786in}{-0.005103in}}{\pgfqpoint{0.009821in}{-0.002605in}}{\pgfqpoint{0.009821in}{0.000000in}}%
\pgfpathcurveto{\pgfqpoint{0.009821in}{0.002605in}}{\pgfqpoint{0.008786in}{0.005103in}}{\pgfqpoint{0.006944in}{0.006944in}}%
\pgfpathcurveto{\pgfqpoint{0.005103in}{0.008786in}}{\pgfqpoint{0.002605in}{0.009821in}}{\pgfqpoint{0.000000in}{0.009821in}}%
\pgfpathcurveto{\pgfqpoint{-0.002605in}{0.009821in}}{\pgfqpoint{-0.005103in}{0.008786in}}{\pgfqpoint{-0.006944in}{0.006944in}}%
\pgfpathcurveto{\pgfqpoint{-0.008786in}{0.005103in}}{\pgfqpoint{-0.009821in}{0.002605in}}{\pgfqpoint{-0.009821in}{0.000000in}}%
\pgfpathcurveto{\pgfqpoint{-0.009821in}{-0.002605in}}{\pgfqpoint{-0.008786in}{-0.005103in}}{\pgfqpoint{-0.006944in}{-0.006944in}}%
\pgfpathcurveto{\pgfqpoint{-0.005103in}{-0.008786in}}{\pgfqpoint{-0.002605in}{-0.009821in}}{\pgfqpoint{0.000000in}{-0.009821in}}%
\pgfpathclose%
\pgfusepath{stroke,fill}%
}%
\begin{pgfscope}%
\pgfsys@transformshift{0.716510in}{3.728639in}%
\pgfsys@useobject{currentmarker}{}%
\end{pgfscope}%
\begin{pgfscope}%
\pgfsys@transformshift{0.866679in}{3.998406in}%
\pgfsys@useobject{currentmarker}{}%
\end{pgfscope}%
\begin{pgfscope}%
\pgfsys@transformshift{1.016848in}{4.187191in}%
\pgfsys@useobject{currentmarker}{}%
\end{pgfscope}%
\begin{pgfscope}%
\pgfsys@transformshift{1.167017in}{4.426414in}%
\pgfsys@useobject{currentmarker}{}%
\end{pgfscope}%
\begin{pgfscope}%
\pgfsys@transformshift{1.317186in}{4.801125in}%
\pgfsys@useobject{currentmarker}{}%
\end{pgfscope}%
\begin{pgfscope}%
\pgfsys@transformshift{1.467355in}{4.859554in}%
\pgfsys@useobject{currentmarker}{}%
\end{pgfscope}%
\begin{pgfscope}%
\pgfsys@transformshift{1.617525in}{5.024291in}%
\pgfsys@useobject{currentmarker}{}%
\end{pgfscope}%
\begin{pgfscope}%
\pgfsys@transformshift{1.767694in}{5.311012in}%
\pgfsys@useobject{currentmarker}{}%
\end{pgfscope}%
\begin{pgfscope}%
\pgfsys@transformshift{1.917863in}{5.404720in}%
\pgfsys@useobject{currentmarker}{}%
\end{pgfscope}%
\begin{pgfscope}%
\pgfsys@transformshift{2.068032in}{5.379897in}%
\pgfsys@useobject{currentmarker}{}%
\end{pgfscope}%
\begin{pgfscope}%
\pgfsys@transformshift{2.218201in}{5.469168in}%
\pgfsys@useobject{currentmarker}{}%
\end{pgfscope}%
\begin{pgfscope}%
\pgfsys@transformshift{2.368370in}{5.318564in}%
\pgfsys@useobject{currentmarker}{}%
\end{pgfscope}%
\begin{pgfscope}%
\pgfsys@transformshift{2.518540in}{5.287414in}%
\pgfsys@useobject{currentmarker}{}%
\end{pgfscope}%
\begin{pgfscope}%
\pgfsys@transformshift{2.668709in}{5.208644in}%
\pgfsys@useobject{currentmarker}{}%
\end{pgfscope}%
\begin{pgfscope}%
\pgfsys@transformshift{2.818878in}{4.996639in}%
\pgfsys@useobject{currentmarker}{}%
\end{pgfscope}%
\begin{pgfscope}%
\pgfsys@transformshift{2.969047in}{4.875631in}%
\pgfsys@useobject{currentmarker}{}%
\end{pgfscope}%
\begin{pgfscope}%
\pgfsys@transformshift{3.119216in}{4.604265in}%
\pgfsys@useobject{currentmarker}{}%
\end{pgfscope}%
\begin{pgfscope}%
\pgfsys@transformshift{3.269385in}{4.339939in}%
\pgfsys@useobject{currentmarker}{}%
\end{pgfscope}%
\begin{pgfscope}%
\pgfsys@transformshift{3.419554in}{4.003053in}%
\pgfsys@useobject{currentmarker}{}%
\end{pgfscope}%
\begin{pgfscope}%
\pgfsys@transformshift{3.569724in}{3.762792in}%
\pgfsys@useobject{currentmarker}{}%
\end{pgfscope}%
\begin{pgfscope}%
\pgfsys@transformshift{3.719893in}{3.463744in}%
\pgfsys@useobject{currentmarker}{}%
\end{pgfscope}%
\end{pgfscope}%
\begin{pgfscope}%
\pgfsetbuttcap%
\pgfsetroundjoin%
\definecolor{currentfill}{rgb}{0.000000,0.000000,0.000000}%
\pgfsetfillcolor{currentfill}%
\pgfsetlinewidth{0.803000pt}%
\definecolor{currentstroke}{rgb}{0.000000,0.000000,0.000000}%
\pgfsetstrokecolor{currentstroke}%
\pgfsetdash{}{0pt}%
\pgfsys@defobject{currentmarker}{\pgfqpoint{0.000000in}{-0.048611in}}{\pgfqpoint{0.000000in}{0.000000in}}{%
\pgfpathmoveto{\pgfqpoint{0.000000in}{0.000000in}}%
\pgfpathlineto{\pgfqpoint{0.000000in}{-0.048611in}}%
\pgfusepath{stroke,fill}%
}%
\begin{pgfscope}%
\pgfsys@transformshift{1.167017in}{3.312222in}%
\pgfsys@useobject{currentmarker}{}%
\end{pgfscope}%
\end{pgfscope}%
\begin{pgfscope}%
\definecolor{textcolor}{rgb}{0.000000,0.000000,0.000000}%
\pgfsetstrokecolor{textcolor}%
\pgfsetfillcolor{textcolor}%
\pgftext[x=1.167017in,y=3.215000in,,top]{\color{textcolor}\sffamily\fontsize{10.000000}{12.000000}\selectfont 2.25}%
\end{pgfscope}%
\begin{pgfscope}%
\pgfsetbuttcap%
\pgfsetroundjoin%
\definecolor{currentfill}{rgb}{0.000000,0.000000,0.000000}%
\pgfsetfillcolor{currentfill}%
\pgfsetlinewidth{0.803000pt}%
\definecolor{currentstroke}{rgb}{0.000000,0.000000,0.000000}%
\pgfsetstrokecolor{currentstroke}%
\pgfsetdash{}{0pt}%
\pgfsys@defobject{currentmarker}{\pgfqpoint{0.000000in}{-0.048611in}}{\pgfqpoint{0.000000in}{0.000000in}}{%
\pgfpathmoveto{\pgfqpoint{0.000000in}{0.000000in}}%
\pgfpathlineto{\pgfqpoint{0.000000in}{-0.048611in}}%
\pgfusepath{stroke,fill}%
}%
\begin{pgfscope}%
\pgfsys@transformshift{1.917863in}{3.312222in}%
\pgfsys@useobject{currentmarker}{}%
\end{pgfscope}%
\end{pgfscope}%
\begin{pgfscope}%
\definecolor{textcolor}{rgb}{0.000000,0.000000,0.000000}%
\pgfsetstrokecolor{textcolor}%
\pgfsetfillcolor{textcolor}%
\pgftext[x=1.917863in,y=3.215000in,,top]{\color{textcolor}\sffamily\fontsize{10.000000}{12.000000}\selectfont 2.30}%
\end{pgfscope}%
\begin{pgfscope}%
\pgfsetbuttcap%
\pgfsetroundjoin%
\definecolor{currentfill}{rgb}{0.000000,0.000000,0.000000}%
\pgfsetfillcolor{currentfill}%
\pgfsetlinewidth{0.803000pt}%
\definecolor{currentstroke}{rgb}{0.000000,0.000000,0.000000}%
\pgfsetstrokecolor{currentstroke}%
\pgfsetdash{}{0pt}%
\pgfsys@defobject{currentmarker}{\pgfqpoint{0.000000in}{-0.048611in}}{\pgfqpoint{0.000000in}{0.000000in}}{%
\pgfpathmoveto{\pgfqpoint{0.000000in}{0.000000in}}%
\pgfpathlineto{\pgfqpoint{0.000000in}{-0.048611in}}%
\pgfusepath{stroke,fill}%
}%
\begin{pgfscope}%
\pgfsys@transformshift{2.668709in}{3.312222in}%
\pgfsys@useobject{currentmarker}{}%
\end{pgfscope}%
\end{pgfscope}%
\begin{pgfscope}%
\definecolor{textcolor}{rgb}{0.000000,0.000000,0.000000}%
\pgfsetstrokecolor{textcolor}%
\pgfsetfillcolor{textcolor}%
\pgftext[x=2.668709in,y=3.215000in,,top]{\color{textcolor}\sffamily\fontsize{10.000000}{12.000000}\selectfont 2.35}%
\end{pgfscope}%
\begin{pgfscope}%
\pgfsetbuttcap%
\pgfsetroundjoin%
\definecolor{currentfill}{rgb}{0.000000,0.000000,0.000000}%
\pgfsetfillcolor{currentfill}%
\pgfsetlinewidth{0.803000pt}%
\definecolor{currentstroke}{rgb}{0.000000,0.000000,0.000000}%
\pgfsetstrokecolor{currentstroke}%
\pgfsetdash{}{0pt}%
\pgfsys@defobject{currentmarker}{\pgfqpoint{0.000000in}{-0.048611in}}{\pgfqpoint{0.000000in}{0.000000in}}{%
\pgfpathmoveto{\pgfqpoint{0.000000in}{0.000000in}}%
\pgfpathlineto{\pgfqpoint{0.000000in}{-0.048611in}}%
\pgfusepath{stroke,fill}%
}%
\begin{pgfscope}%
\pgfsys@transformshift{3.419554in}{3.312222in}%
\pgfsys@useobject{currentmarker}{}%
\end{pgfscope}%
\end{pgfscope}%
\begin{pgfscope}%
\definecolor{textcolor}{rgb}{0.000000,0.000000,0.000000}%
\pgfsetstrokecolor{textcolor}%
\pgfsetfillcolor{textcolor}%
\pgftext[x=3.419554in,y=3.215000in,,top]{\color{textcolor}\sffamily\fontsize{10.000000}{12.000000}\selectfont 2.40}%
\end{pgfscope}%
\begin{pgfscope}%
\pgfsetbuttcap%
\pgfsetroundjoin%
\definecolor{currentfill}{rgb}{0.000000,0.000000,0.000000}%
\pgfsetfillcolor{currentfill}%
\pgfsetlinewidth{0.803000pt}%
\definecolor{currentstroke}{rgb}{0.000000,0.000000,0.000000}%
\pgfsetstrokecolor{currentstroke}%
\pgfsetdash{}{0pt}%
\pgfsys@defobject{currentmarker}{\pgfqpoint{-0.048611in}{0.000000in}}{\pgfqpoint{0.000000in}{0.000000in}}{%
\pgfpathmoveto{\pgfqpoint{0.000000in}{0.000000in}}%
\pgfpathlineto{\pgfqpoint{-0.048611in}{0.000000in}}%
\pgfusepath{stroke,fill}%
}%
\begin{pgfscope}%
\pgfsys@transformshift{0.557222in}{3.489834in}%
\pgfsys@useobject{currentmarker}{}%
\end{pgfscope}%
\end{pgfscope}%
\begin{pgfscope}%
\definecolor{textcolor}{rgb}{0.000000,0.000000,0.000000}%
\pgfsetstrokecolor{textcolor}%
\pgfsetfillcolor{textcolor}%
\pgftext[x=0.150755in,y=3.437072in,left,base]{\color{textcolor}\sffamily\fontsize{10.000000}{12.000000}\selectfont 1.35}%
\end{pgfscope}%
\begin{pgfscope}%
\pgfsetbuttcap%
\pgfsetroundjoin%
\definecolor{currentfill}{rgb}{0.000000,0.000000,0.000000}%
\pgfsetfillcolor{currentfill}%
\pgfsetlinewidth{0.803000pt}%
\definecolor{currentstroke}{rgb}{0.000000,0.000000,0.000000}%
\pgfsetstrokecolor{currentstroke}%
\pgfsetdash{}{0pt}%
\pgfsys@defobject{currentmarker}{\pgfqpoint{-0.048611in}{0.000000in}}{\pgfqpoint{0.000000in}{0.000000in}}{%
\pgfpathmoveto{\pgfqpoint{0.000000in}{0.000000in}}%
\pgfpathlineto{\pgfqpoint{-0.048611in}{0.000000in}}%
\pgfusepath{stroke,fill}%
}%
\begin{pgfscope}%
\pgfsys@transformshift{0.557222in}{3.970629in}%
\pgfsys@useobject{currentmarker}{}%
\end{pgfscope}%
\end{pgfscope}%
\begin{pgfscope}%
\definecolor{textcolor}{rgb}{0.000000,0.000000,0.000000}%
\pgfsetstrokecolor{textcolor}%
\pgfsetfillcolor{textcolor}%
\pgftext[x=0.150755in,y=3.917867in,left,base]{\color{textcolor}\sffamily\fontsize{10.000000}{12.000000}\selectfont 1.40}%
\end{pgfscope}%
\begin{pgfscope}%
\pgfsetbuttcap%
\pgfsetroundjoin%
\definecolor{currentfill}{rgb}{0.000000,0.000000,0.000000}%
\pgfsetfillcolor{currentfill}%
\pgfsetlinewidth{0.803000pt}%
\definecolor{currentstroke}{rgb}{0.000000,0.000000,0.000000}%
\pgfsetstrokecolor{currentstroke}%
\pgfsetdash{}{0pt}%
\pgfsys@defobject{currentmarker}{\pgfqpoint{-0.048611in}{0.000000in}}{\pgfqpoint{0.000000in}{0.000000in}}{%
\pgfpathmoveto{\pgfqpoint{0.000000in}{0.000000in}}%
\pgfpathlineto{\pgfqpoint{-0.048611in}{0.000000in}}%
\pgfusepath{stroke,fill}%
}%
\begin{pgfscope}%
\pgfsys@transformshift{0.557222in}{4.451424in}%
\pgfsys@useobject{currentmarker}{}%
\end{pgfscope}%
\end{pgfscope}%
\begin{pgfscope}%
\definecolor{textcolor}{rgb}{0.000000,0.000000,0.000000}%
\pgfsetstrokecolor{textcolor}%
\pgfsetfillcolor{textcolor}%
\pgftext[x=0.150755in,y=4.398663in,left,base]{\color{textcolor}\sffamily\fontsize{10.000000}{12.000000}\selectfont 1.45}%
\end{pgfscope}%
\begin{pgfscope}%
\pgfsetbuttcap%
\pgfsetroundjoin%
\definecolor{currentfill}{rgb}{0.000000,0.000000,0.000000}%
\pgfsetfillcolor{currentfill}%
\pgfsetlinewidth{0.803000pt}%
\definecolor{currentstroke}{rgb}{0.000000,0.000000,0.000000}%
\pgfsetstrokecolor{currentstroke}%
\pgfsetdash{}{0pt}%
\pgfsys@defobject{currentmarker}{\pgfqpoint{-0.048611in}{0.000000in}}{\pgfqpoint{0.000000in}{0.000000in}}{%
\pgfpathmoveto{\pgfqpoint{0.000000in}{0.000000in}}%
\pgfpathlineto{\pgfqpoint{-0.048611in}{0.000000in}}%
\pgfusepath{stroke,fill}%
}%
\begin{pgfscope}%
\pgfsys@transformshift{0.557222in}{4.932219in}%
\pgfsys@useobject{currentmarker}{}%
\end{pgfscope}%
\end{pgfscope}%
\begin{pgfscope}%
\definecolor{textcolor}{rgb}{0.000000,0.000000,0.000000}%
\pgfsetstrokecolor{textcolor}%
\pgfsetfillcolor{textcolor}%
\pgftext[x=0.150755in,y=4.879458in,left,base]{\color{textcolor}\sffamily\fontsize{10.000000}{12.000000}\selectfont 1.50}%
\end{pgfscope}%
\begin{pgfscope}%
\pgfsetbuttcap%
\pgfsetroundjoin%
\definecolor{currentfill}{rgb}{0.000000,0.000000,0.000000}%
\pgfsetfillcolor{currentfill}%
\pgfsetlinewidth{0.803000pt}%
\definecolor{currentstroke}{rgb}{0.000000,0.000000,0.000000}%
\pgfsetstrokecolor{currentstroke}%
\pgfsetdash{}{0pt}%
\pgfsys@defobject{currentmarker}{\pgfqpoint{-0.048611in}{0.000000in}}{\pgfqpoint{0.000000in}{0.000000in}}{%
\pgfpathmoveto{\pgfqpoint{0.000000in}{0.000000in}}%
\pgfpathlineto{\pgfqpoint{-0.048611in}{0.000000in}}%
\pgfusepath{stroke,fill}%
}%
\begin{pgfscope}%
\pgfsys@transformshift{0.557222in}{5.413015in}%
\pgfsys@useobject{currentmarker}{}%
\end{pgfscope}%
\end{pgfscope}%
\begin{pgfscope}%
\definecolor{textcolor}{rgb}{0.000000,0.000000,0.000000}%
\pgfsetstrokecolor{textcolor}%
\pgfsetfillcolor{textcolor}%
\pgftext[x=0.150755in,y=5.360253in,left,base]{\color{textcolor}\sffamily\fontsize{10.000000}{12.000000}\selectfont 1.55}%
\end{pgfscope}%
\begin{pgfscope}%
\pgfpathrectangle{\pgfqpoint{0.557222in}{3.312222in}}{\pgfqpoint{3.321958in}{2.324444in}}%
\pgfusepath{clip}%
\pgfsetrectcap%
\pgfsetroundjoin%
\pgfsetlinewidth{1.505625pt}%
\definecolor{currentstroke}{rgb}{0.121569,0.466667,0.705882}%
\pgfsetstrokecolor{currentstroke}%
\pgfsetdash{}{0pt}%
\pgfpathmoveto{\pgfqpoint{0.716510in}{3.728639in}}%
\pgfpathlineto{\pgfqpoint{0.866679in}{3.998406in}}%
\pgfpathlineto{\pgfqpoint{1.016848in}{4.187191in}}%
\pgfpathlineto{\pgfqpoint{1.167017in}{4.426414in}}%
\pgfpathlineto{\pgfqpoint{1.317186in}{4.801125in}}%
\pgfpathlineto{\pgfqpoint{1.467355in}{4.859554in}}%
\pgfpathlineto{\pgfqpoint{1.617525in}{5.024291in}}%
\pgfpathlineto{\pgfqpoint{1.767694in}{5.311012in}}%
\pgfpathlineto{\pgfqpoint{1.917863in}{5.404720in}}%
\pgfpathlineto{\pgfqpoint{2.068032in}{5.379897in}}%
\pgfpathlineto{\pgfqpoint{2.218201in}{5.469168in}}%
\pgfpathlineto{\pgfqpoint{2.368370in}{5.318564in}}%
\pgfpathlineto{\pgfqpoint{2.518540in}{5.287414in}}%
\pgfpathlineto{\pgfqpoint{2.668709in}{5.208644in}}%
\pgfpathlineto{\pgfqpoint{2.818878in}{4.996639in}}%
\pgfpathlineto{\pgfqpoint{2.969047in}{4.875631in}}%
\pgfpathlineto{\pgfqpoint{3.119216in}{4.604265in}}%
\pgfpathlineto{\pgfqpoint{3.269385in}{4.339939in}}%
\pgfpathlineto{\pgfqpoint{3.419554in}{4.003053in}}%
\pgfpathlineto{\pgfqpoint{3.569724in}{3.762792in}}%
\pgfpathlineto{\pgfqpoint{3.719893in}{3.463744in}}%
\pgfusepath{stroke}%
\end{pgfscope}%
\begin{pgfscope}%
\pgfsetrectcap%
\pgfsetmiterjoin%
\pgfsetlinewidth{0.803000pt}%
\definecolor{currentstroke}{rgb}{0.000000,0.000000,0.000000}%
\pgfsetstrokecolor{currentstroke}%
\pgfsetdash{}{0pt}%
\pgfpathmoveto{\pgfqpoint{0.557222in}{3.312222in}}%
\pgfpathlineto{\pgfqpoint{0.557222in}{5.636667in}}%
\pgfusepath{stroke}%
\end{pgfscope}%
\begin{pgfscope}%
\pgfsetrectcap%
\pgfsetmiterjoin%
\pgfsetlinewidth{0.803000pt}%
\definecolor{currentstroke}{rgb}{0.000000,0.000000,0.000000}%
\pgfsetstrokecolor{currentstroke}%
\pgfsetdash{}{0pt}%
\pgfpathmoveto{\pgfqpoint{3.879180in}{3.312222in}}%
\pgfpathlineto{\pgfqpoint{3.879180in}{5.636667in}}%
\pgfusepath{stroke}%
\end{pgfscope}%
\begin{pgfscope}%
\pgfsetrectcap%
\pgfsetmiterjoin%
\pgfsetlinewidth{0.803000pt}%
\definecolor{currentstroke}{rgb}{0.000000,0.000000,0.000000}%
\pgfsetstrokecolor{currentstroke}%
\pgfsetdash{}{0pt}%
\pgfpathmoveto{\pgfqpoint{0.557222in}{3.312222in}}%
\pgfpathlineto{\pgfqpoint{3.879180in}{3.312222in}}%
\pgfusepath{stroke}%
\end{pgfscope}%
\begin{pgfscope}%
\pgfsetrectcap%
\pgfsetmiterjoin%
\pgfsetlinewidth{0.803000pt}%
\definecolor{currentstroke}{rgb}{0.000000,0.000000,0.000000}%
\pgfsetstrokecolor{currentstroke}%
\pgfsetdash{}{0pt}%
\pgfpathmoveto{\pgfqpoint{0.557222in}{5.636667in}}%
\pgfpathlineto{\pgfqpoint{3.879180in}{5.636667in}}%
\pgfusepath{stroke}%
\end{pgfscope}%
\begin{pgfscope}%
\definecolor{textcolor}{rgb}{0.000000,0.000000,0.000000}%
\pgfsetstrokecolor{textcolor}%
\pgfsetfillcolor{textcolor}%
\pgftext[x=2.218201in,y=5.720000in,,base]{\color{textcolor}\sffamily\fontsize{12.000000}{14.400000}\selectfont \(\displaystyle  N = 16 \)}%
\end{pgfscope}%
\begin{pgfscope}%
\pgfsetbuttcap%
\pgfsetmiterjoin%
\definecolor{currentfill}{rgb}{1.000000,1.000000,1.000000}%
\pgfsetfillcolor{currentfill}%
\pgfsetlinewidth{0.000000pt}%
\definecolor{currentstroke}{rgb}{0.000000,0.000000,0.000000}%
\pgfsetstrokecolor{currentstroke}%
\pgfsetstrokeopacity{0.000000}%
\pgfsetdash{}{0pt}%
\pgfpathmoveto{\pgfqpoint{4.382652in}{3.312222in}}%
\pgfpathlineto{\pgfqpoint{7.704610in}{3.312222in}}%
\pgfpathlineto{\pgfqpoint{7.704610in}{5.636667in}}%
\pgfpathlineto{\pgfqpoint{4.382652in}{5.636667in}}%
\pgfpathclose%
\pgfusepath{fill}%
\end{pgfscope}%
\begin{pgfscope}%
\pgfpathrectangle{\pgfqpoint{4.382652in}{3.312222in}}{\pgfqpoint{3.321958in}{2.324444in}}%
\pgfusepath{clip}%
\pgfsetbuttcap%
\pgfsetroundjoin%
\definecolor{currentfill}{rgb}{0.121569,0.466667,0.705882}%
\pgfsetfillcolor{currentfill}%
\pgfsetfillopacity{0.300000}%
\pgfsetlinewidth{1.003750pt}%
\definecolor{currentstroke}{rgb}{0.121569,0.466667,0.705882}%
\pgfsetstrokecolor{currentstroke}%
\pgfsetstrokeopacity{0.300000}%
\pgfsetdash{}{0pt}%
\pgfpathmoveto{\pgfqpoint{4.541940in}{3.528777in}}%
\pgfpathlineto{\pgfqpoint{4.541940in}{3.417879in}}%
\pgfpathlineto{\pgfqpoint{4.692109in}{3.567735in}}%
\pgfpathlineto{\pgfqpoint{4.842278in}{3.679041in}}%
\pgfpathlineto{\pgfqpoint{4.992447in}{3.859469in}}%
\pgfpathlineto{\pgfqpoint{5.142617in}{3.987423in}}%
\pgfpathlineto{\pgfqpoint{5.292786in}{4.441922in}}%
\pgfpathlineto{\pgfqpoint{5.442955in}{4.579502in}}%
\pgfpathlineto{\pgfqpoint{5.593124in}{4.732041in}}%
\pgfpathlineto{\pgfqpoint{5.743293in}{4.784569in}}%
\pgfpathlineto{\pgfqpoint{5.893462in}{5.042095in}}%
\pgfpathlineto{\pgfqpoint{6.043631in}{5.091094in}}%
\pgfpathlineto{\pgfqpoint{6.193801in}{5.086891in}}%
\pgfpathlineto{\pgfqpoint{6.343970in}{5.088885in}}%
\pgfpathlineto{\pgfqpoint{6.494139in}{4.996422in}}%
\pgfpathlineto{\pgfqpoint{6.644308in}{4.803624in}}%
\pgfpathlineto{\pgfqpoint{6.794477in}{4.536111in}}%
\pgfpathlineto{\pgfqpoint{6.944646in}{4.310218in}}%
\pgfpathlineto{\pgfqpoint{7.094816in}{4.123216in}}%
\pgfpathlineto{\pgfqpoint{7.244985in}{3.798902in}}%
\pgfpathlineto{\pgfqpoint{7.395154in}{3.573396in}}%
\pgfpathlineto{\pgfqpoint{7.545323in}{3.445521in}}%
\pgfpathlineto{\pgfqpoint{7.545323in}{3.568657in}}%
\pgfpathlineto{\pgfqpoint{7.545323in}{3.568657in}}%
\pgfpathlineto{\pgfqpoint{7.395154in}{3.751496in}}%
\pgfpathlineto{\pgfqpoint{7.244985in}{4.078926in}}%
\pgfpathlineto{\pgfqpoint{7.094816in}{4.341492in}}%
\pgfpathlineto{\pgfqpoint{6.944646in}{4.556593in}}%
\pgfpathlineto{\pgfqpoint{6.794477in}{4.638091in}}%
\pgfpathlineto{\pgfqpoint{6.644308in}{4.915681in}}%
\pgfpathlineto{\pgfqpoint{6.494139in}{5.218474in}}%
\pgfpathlineto{\pgfqpoint{6.343970in}{5.337747in}}%
\pgfpathlineto{\pgfqpoint{6.193801in}{5.531010in}}%
\pgfpathlineto{\pgfqpoint{6.043631in}{5.464630in}}%
\pgfpathlineto{\pgfqpoint{5.893462in}{5.167392in}}%
\pgfpathlineto{\pgfqpoint{5.743293in}{5.360479in}}%
\pgfpathlineto{\pgfqpoint{5.593124in}{5.258614in}}%
\pgfpathlineto{\pgfqpoint{5.442955in}{4.672426in}}%
\pgfpathlineto{\pgfqpoint{5.292786in}{4.902910in}}%
\pgfpathlineto{\pgfqpoint{5.142617in}{4.393099in}}%
\pgfpathlineto{\pgfqpoint{4.992447in}{4.163109in}}%
\pgfpathlineto{\pgfqpoint{4.842278in}{3.902046in}}%
\pgfpathlineto{\pgfqpoint{4.692109in}{3.742512in}}%
\pgfpathlineto{\pgfqpoint{4.541940in}{3.528777in}}%
\pgfpathclose%
\pgfusepath{stroke,fill}%
\end{pgfscope}%
\begin{pgfscope}%
\pgfpathrectangle{\pgfqpoint{4.382652in}{3.312222in}}{\pgfqpoint{3.321958in}{2.324444in}}%
\pgfusepath{clip}%
\pgfsetbuttcap%
\pgfsetroundjoin%
\definecolor{currentfill}{rgb}{0.121569,0.466667,0.705882}%
\pgfsetfillcolor{currentfill}%
\pgfsetlinewidth{1.003750pt}%
\definecolor{currentstroke}{rgb}{0.121569,0.466667,0.705882}%
\pgfsetstrokecolor{currentstroke}%
\pgfsetdash{}{0pt}%
\pgfsys@defobject{currentmarker}{\pgfqpoint{-0.009821in}{-0.009821in}}{\pgfqpoint{0.009821in}{0.009821in}}{%
\pgfpathmoveto{\pgfqpoint{0.000000in}{-0.009821in}}%
\pgfpathcurveto{\pgfqpoint{0.002605in}{-0.009821in}}{\pgfqpoint{0.005103in}{-0.008786in}}{\pgfqpoint{0.006944in}{-0.006944in}}%
\pgfpathcurveto{\pgfqpoint{0.008786in}{-0.005103in}}{\pgfqpoint{0.009821in}{-0.002605in}}{\pgfqpoint{0.009821in}{0.000000in}}%
\pgfpathcurveto{\pgfqpoint{0.009821in}{0.002605in}}{\pgfqpoint{0.008786in}{0.005103in}}{\pgfqpoint{0.006944in}{0.006944in}}%
\pgfpathcurveto{\pgfqpoint{0.005103in}{0.008786in}}{\pgfqpoint{0.002605in}{0.009821in}}{\pgfqpoint{0.000000in}{0.009821in}}%
\pgfpathcurveto{\pgfqpoint{-0.002605in}{0.009821in}}{\pgfqpoint{-0.005103in}{0.008786in}}{\pgfqpoint{-0.006944in}{0.006944in}}%
\pgfpathcurveto{\pgfqpoint{-0.008786in}{0.005103in}}{\pgfqpoint{-0.009821in}{0.002605in}}{\pgfqpoint{-0.009821in}{0.000000in}}%
\pgfpathcurveto{\pgfqpoint{-0.009821in}{-0.002605in}}{\pgfqpoint{-0.008786in}{-0.005103in}}{\pgfqpoint{-0.006944in}{-0.006944in}}%
\pgfpathcurveto{\pgfqpoint{-0.005103in}{-0.008786in}}{\pgfqpoint{-0.002605in}{-0.009821in}}{\pgfqpoint{0.000000in}{-0.009821in}}%
\pgfpathclose%
\pgfusepath{stroke,fill}%
}%
\begin{pgfscope}%
\pgfsys@transformshift{4.541940in}{3.473328in}%
\pgfsys@useobject{currentmarker}{}%
\end{pgfscope}%
\begin{pgfscope}%
\pgfsys@transformshift{4.692109in}{3.655124in}%
\pgfsys@useobject{currentmarker}{}%
\end{pgfscope}%
\begin{pgfscope}%
\pgfsys@transformshift{4.842278in}{3.790544in}%
\pgfsys@useobject{currentmarker}{}%
\end{pgfscope}%
\begin{pgfscope}%
\pgfsys@transformshift{4.992447in}{4.011289in}%
\pgfsys@useobject{currentmarker}{}%
\end{pgfscope}%
\begin{pgfscope}%
\pgfsys@transformshift{5.142617in}{4.190261in}%
\pgfsys@useobject{currentmarker}{}%
\end{pgfscope}%
\begin{pgfscope}%
\pgfsys@transformshift{5.292786in}{4.672416in}%
\pgfsys@useobject{currentmarker}{}%
\end{pgfscope}%
\begin{pgfscope}%
\pgfsys@transformshift{5.442955in}{4.625964in}%
\pgfsys@useobject{currentmarker}{}%
\end{pgfscope}%
\begin{pgfscope}%
\pgfsys@transformshift{5.593124in}{4.995327in}%
\pgfsys@useobject{currentmarker}{}%
\end{pgfscope}%
\begin{pgfscope}%
\pgfsys@transformshift{5.743293in}{5.072524in}%
\pgfsys@useobject{currentmarker}{}%
\end{pgfscope}%
\begin{pgfscope}%
\pgfsys@transformshift{5.893462in}{5.104744in}%
\pgfsys@useobject{currentmarker}{}%
\end{pgfscope}%
\begin{pgfscope}%
\pgfsys@transformshift{6.043631in}{5.277862in}%
\pgfsys@useobject{currentmarker}{}%
\end{pgfscope}%
\begin{pgfscope}%
\pgfsys@transformshift{6.193801in}{5.308950in}%
\pgfsys@useobject{currentmarker}{}%
\end{pgfscope}%
\begin{pgfscope}%
\pgfsys@transformshift{6.343970in}{5.213316in}%
\pgfsys@useobject{currentmarker}{}%
\end{pgfscope}%
\begin{pgfscope}%
\pgfsys@transformshift{6.494139in}{5.107448in}%
\pgfsys@useobject{currentmarker}{}%
\end{pgfscope}%
\begin{pgfscope}%
\pgfsys@transformshift{6.644308in}{4.859652in}%
\pgfsys@useobject{currentmarker}{}%
\end{pgfscope}%
\begin{pgfscope}%
\pgfsys@transformshift{6.794477in}{4.587101in}%
\pgfsys@useobject{currentmarker}{}%
\end{pgfscope}%
\begin{pgfscope}%
\pgfsys@transformshift{6.944646in}{4.433405in}%
\pgfsys@useobject{currentmarker}{}%
\end{pgfscope}%
\begin{pgfscope}%
\pgfsys@transformshift{7.094816in}{4.232354in}%
\pgfsys@useobject{currentmarker}{}%
\end{pgfscope}%
\begin{pgfscope}%
\pgfsys@transformshift{7.244985in}{3.938914in}%
\pgfsys@useobject{currentmarker}{}%
\end{pgfscope}%
\begin{pgfscope}%
\pgfsys@transformshift{7.395154in}{3.662446in}%
\pgfsys@useobject{currentmarker}{}%
\end{pgfscope}%
\begin{pgfscope}%
\pgfsys@transformshift{7.545323in}{3.507089in}%
\pgfsys@useobject{currentmarker}{}%
\end{pgfscope}%
\end{pgfscope}%
\begin{pgfscope}%
\pgfsetbuttcap%
\pgfsetroundjoin%
\definecolor{currentfill}{rgb}{0.000000,0.000000,0.000000}%
\pgfsetfillcolor{currentfill}%
\pgfsetlinewidth{0.803000pt}%
\definecolor{currentstroke}{rgb}{0.000000,0.000000,0.000000}%
\pgfsetstrokecolor{currentstroke}%
\pgfsetdash{}{0pt}%
\pgfsys@defobject{currentmarker}{\pgfqpoint{0.000000in}{-0.048611in}}{\pgfqpoint{0.000000in}{0.000000in}}{%
\pgfpathmoveto{\pgfqpoint{0.000000in}{0.000000in}}%
\pgfpathlineto{\pgfqpoint{0.000000in}{-0.048611in}}%
\pgfusepath{stroke,fill}%
}%
\begin{pgfscope}%
\pgfsys@transformshift{4.692109in}{3.312222in}%
\pgfsys@useobject{currentmarker}{}%
\end{pgfscope}%
\end{pgfscope}%
\begin{pgfscope}%
\definecolor{textcolor}{rgb}{0.000000,0.000000,0.000000}%
\pgfsetstrokecolor{textcolor}%
\pgfsetfillcolor{textcolor}%
\pgftext[x=4.692109in,y=3.215000in,,top]{\color{textcolor}\sffamily\fontsize{10.000000}{12.000000}\selectfont 2.20}%
\end{pgfscope}%
\begin{pgfscope}%
\pgfsetbuttcap%
\pgfsetroundjoin%
\definecolor{currentfill}{rgb}{0.000000,0.000000,0.000000}%
\pgfsetfillcolor{currentfill}%
\pgfsetlinewidth{0.803000pt}%
\definecolor{currentstroke}{rgb}{0.000000,0.000000,0.000000}%
\pgfsetstrokecolor{currentstroke}%
\pgfsetdash{}{0pt}%
\pgfsys@defobject{currentmarker}{\pgfqpoint{0.000000in}{-0.048611in}}{\pgfqpoint{0.000000in}{0.000000in}}{%
\pgfpathmoveto{\pgfqpoint{0.000000in}{0.000000in}}%
\pgfpathlineto{\pgfqpoint{0.000000in}{-0.048611in}}%
\pgfusepath{stroke,fill}%
}%
\begin{pgfscope}%
\pgfsys@transformshift{5.442955in}{3.312222in}%
\pgfsys@useobject{currentmarker}{}%
\end{pgfscope}%
\end{pgfscope}%
\begin{pgfscope}%
\definecolor{textcolor}{rgb}{0.000000,0.000000,0.000000}%
\pgfsetstrokecolor{textcolor}%
\pgfsetfillcolor{textcolor}%
\pgftext[x=5.442955in,y=3.215000in,,top]{\color{textcolor}\sffamily\fontsize{10.000000}{12.000000}\selectfont 2.25}%
\end{pgfscope}%
\begin{pgfscope}%
\pgfsetbuttcap%
\pgfsetroundjoin%
\definecolor{currentfill}{rgb}{0.000000,0.000000,0.000000}%
\pgfsetfillcolor{currentfill}%
\pgfsetlinewidth{0.803000pt}%
\definecolor{currentstroke}{rgb}{0.000000,0.000000,0.000000}%
\pgfsetstrokecolor{currentstroke}%
\pgfsetdash{}{0pt}%
\pgfsys@defobject{currentmarker}{\pgfqpoint{0.000000in}{-0.048611in}}{\pgfqpoint{0.000000in}{0.000000in}}{%
\pgfpathmoveto{\pgfqpoint{0.000000in}{0.000000in}}%
\pgfpathlineto{\pgfqpoint{0.000000in}{-0.048611in}}%
\pgfusepath{stroke,fill}%
}%
\begin{pgfscope}%
\pgfsys@transformshift{6.193801in}{3.312222in}%
\pgfsys@useobject{currentmarker}{}%
\end{pgfscope}%
\end{pgfscope}%
\begin{pgfscope}%
\definecolor{textcolor}{rgb}{0.000000,0.000000,0.000000}%
\pgfsetstrokecolor{textcolor}%
\pgfsetfillcolor{textcolor}%
\pgftext[x=6.193801in,y=3.215000in,,top]{\color{textcolor}\sffamily\fontsize{10.000000}{12.000000}\selectfont 2.30}%
\end{pgfscope}%
\begin{pgfscope}%
\pgfsetbuttcap%
\pgfsetroundjoin%
\definecolor{currentfill}{rgb}{0.000000,0.000000,0.000000}%
\pgfsetfillcolor{currentfill}%
\pgfsetlinewidth{0.803000pt}%
\definecolor{currentstroke}{rgb}{0.000000,0.000000,0.000000}%
\pgfsetstrokecolor{currentstroke}%
\pgfsetdash{}{0pt}%
\pgfsys@defobject{currentmarker}{\pgfqpoint{0.000000in}{-0.048611in}}{\pgfqpoint{0.000000in}{0.000000in}}{%
\pgfpathmoveto{\pgfqpoint{0.000000in}{0.000000in}}%
\pgfpathlineto{\pgfqpoint{0.000000in}{-0.048611in}}%
\pgfusepath{stroke,fill}%
}%
\begin{pgfscope}%
\pgfsys@transformshift{6.944646in}{3.312222in}%
\pgfsys@useobject{currentmarker}{}%
\end{pgfscope}%
\end{pgfscope}%
\begin{pgfscope}%
\definecolor{textcolor}{rgb}{0.000000,0.000000,0.000000}%
\pgfsetstrokecolor{textcolor}%
\pgfsetfillcolor{textcolor}%
\pgftext[x=6.944646in,y=3.215000in,,top]{\color{textcolor}\sffamily\fontsize{10.000000}{12.000000}\selectfont 2.35}%
\end{pgfscope}%
\begin{pgfscope}%
\pgfsetbuttcap%
\pgfsetroundjoin%
\definecolor{currentfill}{rgb}{0.000000,0.000000,0.000000}%
\pgfsetfillcolor{currentfill}%
\pgfsetlinewidth{0.803000pt}%
\definecolor{currentstroke}{rgb}{0.000000,0.000000,0.000000}%
\pgfsetstrokecolor{currentstroke}%
\pgfsetdash{}{0pt}%
\pgfsys@defobject{currentmarker}{\pgfqpoint{0.000000in}{-0.048611in}}{\pgfqpoint{0.000000in}{0.000000in}}{%
\pgfpathmoveto{\pgfqpoint{0.000000in}{0.000000in}}%
\pgfpathlineto{\pgfqpoint{0.000000in}{-0.048611in}}%
\pgfusepath{stroke,fill}%
}%
\begin{pgfscope}%
\pgfsys@transformshift{7.695492in}{3.312222in}%
\pgfsys@useobject{currentmarker}{}%
\end{pgfscope}%
\end{pgfscope}%
\begin{pgfscope}%
\definecolor{textcolor}{rgb}{0.000000,0.000000,0.000000}%
\pgfsetstrokecolor{textcolor}%
\pgfsetfillcolor{textcolor}%
\pgftext[x=7.695492in,y=3.215000in,,top]{\color{textcolor}\sffamily\fontsize{10.000000}{12.000000}\selectfont 2.40}%
\end{pgfscope}%
\begin{pgfscope}%
\pgfsetbuttcap%
\pgfsetroundjoin%
\definecolor{currentfill}{rgb}{0.000000,0.000000,0.000000}%
\pgfsetfillcolor{currentfill}%
\pgfsetlinewidth{0.803000pt}%
\definecolor{currentstroke}{rgb}{0.000000,0.000000,0.000000}%
\pgfsetstrokecolor{currentstroke}%
\pgfsetdash{}{0pt}%
\pgfsys@defobject{currentmarker}{\pgfqpoint{-0.048611in}{0.000000in}}{\pgfqpoint{0.000000in}{0.000000in}}{%
\pgfpathmoveto{\pgfqpoint{0.000000in}{0.000000in}}%
\pgfpathlineto{\pgfqpoint{-0.048611in}{0.000000in}}%
\pgfusepath{stroke,fill}%
}%
\begin{pgfscope}%
\pgfsys@transformshift{4.382652in}{3.357367in}%
\pgfsys@useobject{currentmarker}{}%
\end{pgfscope}%
\end{pgfscope}%
\begin{pgfscope}%
\definecolor{textcolor}{rgb}{0.000000,0.000000,0.000000}%
\pgfsetstrokecolor{textcolor}%
\pgfsetfillcolor{textcolor}%
\pgftext[x=4.064551in,y=3.304605in,left,base]{\color{textcolor}\sffamily\fontsize{10.000000}{12.000000}\selectfont 1.3}%
\end{pgfscope}%
\begin{pgfscope}%
\pgfsetbuttcap%
\pgfsetroundjoin%
\definecolor{currentfill}{rgb}{0.000000,0.000000,0.000000}%
\pgfsetfillcolor{currentfill}%
\pgfsetlinewidth{0.803000pt}%
\definecolor{currentstroke}{rgb}{0.000000,0.000000,0.000000}%
\pgfsetstrokecolor{currentstroke}%
\pgfsetdash{}{0pt}%
\pgfsys@defobject{currentmarker}{\pgfqpoint{-0.048611in}{0.000000in}}{\pgfqpoint{0.000000in}{0.000000in}}{%
\pgfpathmoveto{\pgfqpoint{0.000000in}{0.000000in}}%
\pgfpathlineto{\pgfqpoint{-0.048611in}{0.000000in}}%
\pgfusepath{stroke,fill}%
}%
\begin{pgfscope}%
\pgfsys@transformshift{4.382652in}{3.668284in}%
\pgfsys@useobject{currentmarker}{}%
\end{pgfscope}%
\end{pgfscope}%
\begin{pgfscope}%
\definecolor{textcolor}{rgb}{0.000000,0.000000,0.000000}%
\pgfsetstrokecolor{textcolor}%
\pgfsetfillcolor{textcolor}%
\pgftext[x=4.064551in,y=3.615523in,left,base]{\color{textcolor}\sffamily\fontsize{10.000000}{12.000000}\selectfont 1.4}%
\end{pgfscope}%
\begin{pgfscope}%
\pgfsetbuttcap%
\pgfsetroundjoin%
\definecolor{currentfill}{rgb}{0.000000,0.000000,0.000000}%
\pgfsetfillcolor{currentfill}%
\pgfsetlinewidth{0.803000pt}%
\definecolor{currentstroke}{rgb}{0.000000,0.000000,0.000000}%
\pgfsetstrokecolor{currentstroke}%
\pgfsetdash{}{0pt}%
\pgfsys@defobject{currentmarker}{\pgfqpoint{-0.048611in}{0.000000in}}{\pgfqpoint{0.000000in}{0.000000in}}{%
\pgfpathmoveto{\pgfqpoint{0.000000in}{0.000000in}}%
\pgfpathlineto{\pgfqpoint{-0.048611in}{0.000000in}}%
\pgfusepath{stroke,fill}%
}%
\begin{pgfscope}%
\pgfsys@transformshift{4.382652in}{3.979202in}%
\pgfsys@useobject{currentmarker}{}%
\end{pgfscope}%
\end{pgfscope}%
\begin{pgfscope}%
\definecolor{textcolor}{rgb}{0.000000,0.000000,0.000000}%
\pgfsetstrokecolor{textcolor}%
\pgfsetfillcolor{textcolor}%
\pgftext[x=4.064551in,y=3.926440in,left,base]{\color{textcolor}\sffamily\fontsize{10.000000}{12.000000}\selectfont 1.5}%
\end{pgfscope}%
\begin{pgfscope}%
\pgfsetbuttcap%
\pgfsetroundjoin%
\definecolor{currentfill}{rgb}{0.000000,0.000000,0.000000}%
\pgfsetfillcolor{currentfill}%
\pgfsetlinewidth{0.803000pt}%
\definecolor{currentstroke}{rgb}{0.000000,0.000000,0.000000}%
\pgfsetstrokecolor{currentstroke}%
\pgfsetdash{}{0pt}%
\pgfsys@defobject{currentmarker}{\pgfqpoint{-0.048611in}{0.000000in}}{\pgfqpoint{0.000000in}{0.000000in}}{%
\pgfpathmoveto{\pgfqpoint{0.000000in}{0.000000in}}%
\pgfpathlineto{\pgfqpoint{-0.048611in}{0.000000in}}%
\pgfusepath{stroke,fill}%
}%
\begin{pgfscope}%
\pgfsys@transformshift{4.382652in}{4.290120in}%
\pgfsys@useobject{currentmarker}{}%
\end{pgfscope}%
\end{pgfscope}%
\begin{pgfscope}%
\definecolor{textcolor}{rgb}{0.000000,0.000000,0.000000}%
\pgfsetstrokecolor{textcolor}%
\pgfsetfillcolor{textcolor}%
\pgftext[x=4.064551in,y=4.237358in,left,base]{\color{textcolor}\sffamily\fontsize{10.000000}{12.000000}\selectfont 1.6}%
\end{pgfscope}%
\begin{pgfscope}%
\pgfsetbuttcap%
\pgfsetroundjoin%
\definecolor{currentfill}{rgb}{0.000000,0.000000,0.000000}%
\pgfsetfillcolor{currentfill}%
\pgfsetlinewidth{0.803000pt}%
\definecolor{currentstroke}{rgb}{0.000000,0.000000,0.000000}%
\pgfsetstrokecolor{currentstroke}%
\pgfsetdash{}{0pt}%
\pgfsys@defobject{currentmarker}{\pgfqpoint{-0.048611in}{0.000000in}}{\pgfqpoint{0.000000in}{0.000000in}}{%
\pgfpathmoveto{\pgfqpoint{0.000000in}{0.000000in}}%
\pgfpathlineto{\pgfqpoint{-0.048611in}{0.000000in}}%
\pgfusepath{stroke,fill}%
}%
\begin{pgfscope}%
\pgfsys@transformshift{4.382652in}{4.601037in}%
\pgfsys@useobject{currentmarker}{}%
\end{pgfscope}%
\end{pgfscope}%
\begin{pgfscope}%
\definecolor{textcolor}{rgb}{0.000000,0.000000,0.000000}%
\pgfsetstrokecolor{textcolor}%
\pgfsetfillcolor{textcolor}%
\pgftext[x=4.064551in,y=4.548276in,left,base]{\color{textcolor}\sffamily\fontsize{10.000000}{12.000000}\selectfont 1.7}%
\end{pgfscope}%
\begin{pgfscope}%
\pgfsetbuttcap%
\pgfsetroundjoin%
\definecolor{currentfill}{rgb}{0.000000,0.000000,0.000000}%
\pgfsetfillcolor{currentfill}%
\pgfsetlinewidth{0.803000pt}%
\definecolor{currentstroke}{rgb}{0.000000,0.000000,0.000000}%
\pgfsetstrokecolor{currentstroke}%
\pgfsetdash{}{0pt}%
\pgfsys@defobject{currentmarker}{\pgfqpoint{-0.048611in}{0.000000in}}{\pgfqpoint{0.000000in}{0.000000in}}{%
\pgfpathmoveto{\pgfqpoint{0.000000in}{0.000000in}}%
\pgfpathlineto{\pgfqpoint{-0.048611in}{0.000000in}}%
\pgfusepath{stroke,fill}%
}%
\begin{pgfscope}%
\pgfsys@transformshift{4.382652in}{4.911955in}%
\pgfsys@useobject{currentmarker}{}%
\end{pgfscope}%
\end{pgfscope}%
\begin{pgfscope}%
\definecolor{textcolor}{rgb}{0.000000,0.000000,0.000000}%
\pgfsetstrokecolor{textcolor}%
\pgfsetfillcolor{textcolor}%
\pgftext[x=4.064551in,y=4.859193in,left,base]{\color{textcolor}\sffamily\fontsize{10.000000}{12.000000}\selectfont 1.8}%
\end{pgfscope}%
\begin{pgfscope}%
\pgfsetbuttcap%
\pgfsetroundjoin%
\definecolor{currentfill}{rgb}{0.000000,0.000000,0.000000}%
\pgfsetfillcolor{currentfill}%
\pgfsetlinewidth{0.803000pt}%
\definecolor{currentstroke}{rgb}{0.000000,0.000000,0.000000}%
\pgfsetstrokecolor{currentstroke}%
\pgfsetdash{}{0pt}%
\pgfsys@defobject{currentmarker}{\pgfqpoint{-0.048611in}{0.000000in}}{\pgfqpoint{0.000000in}{0.000000in}}{%
\pgfpathmoveto{\pgfqpoint{0.000000in}{0.000000in}}%
\pgfpathlineto{\pgfqpoint{-0.048611in}{0.000000in}}%
\pgfusepath{stroke,fill}%
}%
\begin{pgfscope}%
\pgfsys@transformshift{4.382652in}{5.222873in}%
\pgfsys@useobject{currentmarker}{}%
\end{pgfscope}%
\end{pgfscope}%
\begin{pgfscope}%
\definecolor{textcolor}{rgb}{0.000000,0.000000,0.000000}%
\pgfsetstrokecolor{textcolor}%
\pgfsetfillcolor{textcolor}%
\pgftext[x=4.064551in,y=5.170111in,left,base]{\color{textcolor}\sffamily\fontsize{10.000000}{12.000000}\selectfont 1.9}%
\end{pgfscope}%
\begin{pgfscope}%
\pgfsetbuttcap%
\pgfsetroundjoin%
\definecolor{currentfill}{rgb}{0.000000,0.000000,0.000000}%
\pgfsetfillcolor{currentfill}%
\pgfsetlinewidth{0.803000pt}%
\definecolor{currentstroke}{rgb}{0.000000,0.000000,0.000000}%
\pgfsetstrokecolor{currentstroke}%
\pgfsetdash{}{0pt}%
\pgfsys@defobject{currentmarker}{\pgfqpoint{-0.048611in}{0.000000in}}{\pgfqpoint{0.000000in}{0.000000in}}{%
\pgfpathmoveto{\pgfqpoint{0.000000in}{0.000000in}}%
\pgfpathlineto{\pgfqpoint{-0.048611in}{0.000000in}}%
\pgfusepath{stroke,fill}%
}%
\begin{pgfscope}%
\pgfsys@transformshift{4.382652in}{5.533790in}%
\pgfsys@useobject{currentmarker}{}%
\end{pgfscope}%
\end{pgfscope}%
\begin{pgfscope}%
\definecolor{textcolor}{rgb}{0.000000,0.000000,0.000000}%
\pgfsetstrokecolor{textcolor}%
\pgfsetfillcolor{textcolor}%
\pgftext[x=4.064551in,y=5.481029in,left,base]{\color{textcolor}\sffamily\fontsize{10.000000}{12.000000}\selectfont 2.0}%
\end{pgfscope}%
\begin{pgfscope}%
\pgfpathrectangle{\pgfqpoint{4.382652in}{3.312222in}}{\pgfqpoint{3.321958in}{2.324444in}}%
\pgfusepath{clip}%
\pgfsetrectcap%
\pgfsetroundjoin%
\pgfsetlinewidth{1.505625pt}%
\definecolor{currentstroke}{rgb}{0.121569,0.466667,0.705882}%
\pgfsetstrokecolor{currentstroke}%
\pgfsetdash{}{0pt}%
\pgfpathmoveto{\pgfqpoint{4.541940in}{3.473328in}}%
\pgfpathlineto{\pgfqpoint{4.692109in}{3.655124in}}%
\pgfpathlineto{\pgfqpoint{4.842278in}{3.790544in}}%
\pgfpathlineto{\pgfqpoint{4.992447in}{4.011289in}}%
\pgfpathlineto{\pgfqpoint{5.142617in}{4.190261in}}%
\pgfpathlineto{\pgfqpoint{5.292786in}{4.672416in}}%
\pgfpathlineto{\pgfqpoint{5.442955in}{4.625964in}}%
\pgfpathlineto{\pgfqpoint{5.593124in}{4.995327in}}%
\pgfpathlineto{\pgfqpoint{5.743293in}{5.072524in}}%
\pgfpathlineto{\pgfqpoint{5.893462in}{5.104744in}}%
\pgfpathlineto{\pgfqpoint{6.043631in}{5.277862in}}%
\pgfpathlineto{\pgfqpoint{6.193801in}{5.308950in}}%
\pgfpathlineto{\pgfqpoint{6.343970in}{5.213316in}}%
\pgfpathlineto{\pgfqpoint{6.494139in}{5.107448in}}%
\pgfpathlineto{\pgfqpoint{6.644308in}{4.859652in}}%
\pgfpathlineto{\pgfqpoint{6.794477in}{4.587101in}}%
\pgfpathlineto{\pgfqpoint{6.944646in}{4.433405in}}%
\pgfpathlineto{\pgfqpoint{7.094816in}{4.232354in}}%
\pgfpathlineto{\pgfqpoint{7.244985in}{3.938914in}}%
\pgfpathlineto{\pgfqpoint{7.395154in}{3.662446in}}%
\pgfpathlineto{\pgfqpoint{7.545323in}{3.507089in}}%
\pgfusepath{stroke}%
\end{pgfscope}%
\begin{pgfscope}%
\pgfsetrectcap%
\pgfsetmiterjoin%
\pgfsetlinewidth{0.803000pt}%
\definecolor{currentstroke}{rgb}{0.000000,0.000000,0.000000}%
\pgfsetstrokecolor{currentstroke}%
\pgfsetdash{}{0pt}%
\pgfpathmoveto{\pgfqpoint{4.382652in}{3.312222in}}%
\pgfpathlineto{\pgfqpoint{4.382652in}{5.636667in}}%
\pgfusepath{stroke}%
\end{pgfscope}%
\begin{pgfscope}%
\pgfsetrectcap%
\pgfsetmiterjoin%
\pgfsetlinewidth{0.803000pt}%
\definecolor{currentstroke}{rgb}{0.000000,0.000000,0.000000}%
\pgfsetstrokecolor{currentstroke}%
\pgfsetdash{}{0pt}%
\pgfpathmoveto{\pgfqpoint{7.704610in}{3.312222in}}%
\pgfpathlineto{\pgfqpoint{7.704610in}{5.636667in}}%
\pgfusepath{stroke}%
\end{pgfscope}%
\begin{pgfscope}%
\pgfsetrectcap%
\pgfsetmiterjoin%
\pgfsetlinewidth{0.803000pt}%
\definecolor{currentstroke}{rgb}{0.000000,0.000000,0.000000}%
\pgfsetstrokecolor{currentstroke}%
\pgfsetdash{}{0pt}%
\pgfpathmoveto{\pgfqpoint{4.382652in}{3.312222in}}%
\pgfpathlineto{\pgfqpoint{7.704610in}{3.312222in}}%
\pgfusepath{stroke}%
\end{pgfscope}%
\begin{pgfscope}%
\pgfsetrectcap%
\pgfsetmiterjoin%
\pgfsetlinewidth{0.803000pt}%
\definecolor{currentstroke}{rgb}{0.000000,0.000000,0.000000}%
\pgfsetstrokecolor{currentstroke}%
\pgfsetdash{}{0pt}%
\pgfpathmoveto{\pgfqpoint{4.382652in}{5.636667in}}%
\pgfpathlineto{\pgfqpoint{7.704610in}{5.636667in}}%
\pgfusepath{stroke}%
\end{pgfscope}%
\begin{pgfscope}%
\definecolor{textcolor}{rgb}{0.000000,0.000000,0.000000}%
\pgfsetstrokecolor{textcolor}%
\pgfsetfillcolor{textcolor}%
\pgftext[x=6.043631in,y=5.720000in,,base]{\color{textcolor}\sffamily\fontsize{12.000000}{14.400000}\selectfont \(\displaystyle  N = 32 \)}%
\end{pgfscope}%
\begin{pgfscope}%
\pgfsetbuttcap%
\pgfsetmiterjoin%
\definecolor{currentfill}{rgb}{1.000000,1.000000,1.000000}%
\pgfsetfillcolor{currentfill}%
\pgfsetlinewidth{0.000000pt}%
\definecolor{currentstroke}{rgb}{0.000000,0.000000,0.000000}%
\pgfsetstrokecolor{currentstroke}%
\pgfsetstrokeopacity{0.000000}%
\pgfsetdash{}{0pt}%
\pgfpathmoveto{\pgfqpoint{0.557222in}{0.387222in}}%
\pgfpathlineto{\pgfqpoint{3.879180in}{0.387222in}}%
\pgfpathlineto{\pgfqpoint{3.879180in}{2.711667in}}%
\pgfpathlineto{\pgfqpoint{0.557222in}{2.711667in}}%
\pgfpathclose%
\pgfusepath{fill}%
\end{pgfscope}%
\begin{pgfscope}%
\pgfpathrectangle{\pgfqpoint{0.557222in}{0.387222in}}{\pgfqpoint{3.321958in}{2.324444in}}%
\pgfusepath{clip}%
\pgfsetbuttcap%
\pgfsetroundjoin%
\definecolor{currentfill}{rgb}{0.121569,0.466667,0.705882}%
\pgfsetfillcolor{currentfill}%
\pgfsetfillopacity{0.300000}%
\pgfsetlinewidth{1.003750pt}%
\definecolor{currentstroke}{rgb}{0.121569,0.466667,0.705882}%
\pgfsetstrokecolor{currentstroke}%
\pgfsetstrokeopacity{0.300000}%
\pgfsetdash{}{0pt}%
\pgfpathmoveto{\pgfqpoint{0.716510in}{0.843689in}}%
\pgfpathlineto{\pgfqpoint{0.716510in}{0.709742in}}%
\pgfpathlineto{\pgfqpoint{0.866679in}{0.716963in}}%
\pgfpathlineto{\pgfqpoint{1.016848in}{0.708265in}}%
\pgfpathlineto{\pgfqpoint{1.167017in}{0.818516in}}%
\pgfpathlineto{\pgfqpoint{1.317186in}{0.990807in}}%
\pgfpathlineto{\pgfqpoint{1.467355in}{0.986343in}}%
\pgfpathlineto{\pgfqpoint{1.617525in}{1.196497in}}%
\pgfpathlineto{\pgfqpoint{1.767694in}{1.419904in}}%
\pgfpathlineto{\pgfqpoint{1.917863in}{1.564509in}}%
\pgfpathlineto{\pgfqpoint{2.068032in}{1.420427in}}%
\pgfpathlineto{\pgfqpoint{2.218201in}{1.840687in}}%
\pgfpathlineto{\pgfqpoint{2.368370in}{1.740027in}}%
\pgfpathlineto{\pgfqpoint{2.518540in}{1.702381in}}%
\pgfpathlineto{\pgfqpoint{2.668709in}{1.416572in}}%
\pgfpathlineto{\pgfqpoint{2.818878in}{1.007941in}}%
\pgfpathlineto{\pgfqpoint{2.969047in}{1.027043in}}%
\pgfpathlineto{\pgfqpoint{3.119216in}{0.866216in}}%
\pgfpathlineto{\pgfqpoint{3.269385in}{0.788481in}}%
\pgfpathlineto{\pgfqpoint{3.419554in}{0.808750in}}%
\pgfpathlineto{\pgfqpoint{3.569724in}{0.619564in}}%
\pgfpathlineto{\pgfqpoint{3.719893in}{0.492879in}}%
\pgfpathlineto{\pgfqpoint{3.719893in}{0.700292in}}%
\pgfpathlineto{\pgfqpoint{3.719893in}{0.700292in}}%
\pgfpathlineto{\pgfqpoint{3.569724in}{0.841225in}}%
\pgfpathlineto{\pgfqpoint{3.419554in}{0.973706in}}%
\pgfpathlineto{\pgfqpoint{3.269385in}{1.017803in}}%
\pgfpathlineto{\pgfqpoint{3.119216in}{1.277611in}}%
\pgfpathlineto{\pgfqpoint{2.969047in}{1.490699in}}%
\pgfpathlineto{\pgfqpoint{2.818878in}{1.424179in}}%
\pgfpathlineto{\pgfqpoint{2.668709in}{1.837274in}}%
\pgfpathlineto{\pgfqpoint{2.518540in}{1.964554in}}%
\pgfpathlineto{\pgfqpoint{2.368370in}{2.242204in}}%
\pgfpathlineto{\pgfqpoint{2.218201in}{2.046827in}}%
\pgfpathlineto{\pgfqpoint{2.068032in}{2.606010in}}%
\pgfpathlineto{\pgfqpoint{1.917863in}{2.460568in}}%
\pgfpathlineto{\pgfqpoint{1.767694in}{1.736463in}}%
\pgfpathlineto{\pgfqpoint{1.617525in}{1.410752in}}%
\pgfpathlineto{\pgfqpoint{1.467355in}{1.336724in}}%
\pgfpathlineto{\pgfqpoint{1.317186in}{1.945480in}}%
\pgfpathlineto{\pgfqpoint{1.167017in}{1.047982in}}%
\pgfpathlineto{\pgfqpoint{1.016848in}{1.107272in}}%
\pgfpathlineto{\pgfqpoint{0.866679in}{0.862459in}}%
\pgfpathlineto{\pgfqpoint{0.716510in}{0.843689in}}%
\pgfpathclose%
\pgfusepath{stroke,fill}%
\end{pgfscope}%
\begin{pgfscope}%
\pgfpathrectangle{\pgfqpoint{0.557222in}{0.387222in}}{\pgfqpoint{3.321958in}{2.324444in}}%
\pgfusepath{clip}%
\pgfsetbuttcap%
\pgfsetroundjoin%
\definecolor{currentfill}{rgb}{0.121569,0.466667,0.705882}%
\pgfsetfillcolor{currentfill}%
\pgfsetlinewidth{1.003750pt}%
\definecolor{currentstroke}{rgb}{0.121569,0.466667,0.705882}%
\pgfsetstrokecolor{currentstroke}%
\pgfsetdash{}{0pt}%
\pgfsys@defobject{currentmarker}{\pgfqpoint{-0.009821in}{-0.009821in}}{\pgfqpoint{0.009821in}{0.009821in}}{%
\pgfpathmoveto{\pgfqpoint{0.000000in}{-0.009821in}}%
\pgfpathcurveto{\pgfqpoint{0.002605in}{-0.009821in}}{\pgfqpoint{0.005103in}{-0.008786in}}{\pgfqpoint{0.006944in}{-0.006944in}}%
\pgfpathcurveto{\pgfqpoint{0.008786in}{-0.005103in}}{\pgfqpoint{0.009821in}{-0.002605in}}{\pgfqpoint{0.009821in}{0.000000in}}%
\pgfpathcurveto{\pgfqpoint{0.009821in}{0.002605in}}{\pgfqpoint{0.008786in}{0.005103in}}{\pgfqpoint{0.006944in}{0.006944in}}%
\pgfpathcurveto{\pgfqpoint{0.005103in}{0.008786in}}{\pgfqpoint{0.002605in}{0.009821in}}{\pgfqpoint{0.000000in}{0.009821in}}%
\pgfpathcurveto{\pgfqpoint{-0.002605in}{0.009821in}}{\pgfqpoint{-0.005103in}{0.008786in}}{\pgfqpoint{-0.006944in}{0.006944in}}%
\pgfpathcurveto{\pgfqpoint{-0.008786in}{0.005103in}}{\pgfqpoint{-0.009821in}{0.002605in}}{\pgfqpoint{-0.009821in}{0.000000in}}%
\pgfpathcurveto{\pgfqpoint{-0.009821in}{-0.002605in}}{\pgfqpoint{-0.008786in}{-0.005103in}}{\pgfqpoint{-0.006944in}{-0.006944in}}%
\pgfpathcurveto{\pgfqpoint{-0.005103in}{-0.008786in}}{\pgfqpoint{-0.002605in}{-0.009821in}}{\pgfqpoint{0.000000in}{-0.009821in}}%
\pgfpathclose%
\pgfusepath{stroke,fill}%
}%
\begin{pgfscope}%
\pgfsys@transformshift{0.716510in}{0.776716in}%
\pgfsys@useobject{currentmarker}{}%
\end{pgfscope}%
\begin{pgfscope}%
\pgfsys@transformshift{0.866679in}{0.789711in}%
\pgfsys@useobject{currentmarker}{}%
\end{pgfscope}%
\begin{pgfscope}%
\pgfsys@transformshift{1.016848in}{0.907769in}%
\pgfsys@useobject{currentmarker}{}%
\end{pgfscope}%
\begin{pgfscope}%
\pgfsys@transformshift{1.167017in}{0.933249in}%
\pgfsys@useobject{currentmarker}{}%
\end{pgfscope}%
\begin{pgfscope}%
\pgfsys@transformshift{1.317186in}{1.468143in}%
\pgfsys@useobject{currentmarker}{}%
\end{pgfscope}%
\begin{pgfscope}%
\pgfsys@transformshift{1.467355in}{1.161533in}%
\pgfsys@useobject{currentmarker}{}%
\end{pgfscope}%
\begin{pgfscope}%
\pgfsys@transformshift{1.617525in}{1.303624in}%
\pgfsys@useobject{currentmarker}{}%
\end{pgfscope}%
\begin{pgfscope}%
\pgfsys@transformshift{1.767694in}{1.578184in}%
\pgfsys@useobject{currentmarker}{}%
\end{pgfscope}%
\begin{pgfscope}%
\pgfsys@transformshift{1.917863in}{2.012539in}%
\pgfsys@useobject{currentmarker}{}%
\end{pgfscope}%
\begin{pgfscope}%
\pgfsys@transformshift{2.068032in}{2.013219in}%
\pgfsys@useobject{currentmarker}{}%
\end{pgfscope}%
\begin{pgfscope}%
\pgfsys@transformshift{2.218201in}{1.943757in}%
\pgfsys@useobject{currentmarker}{}%
\end{pgfscope}%
\begin{pgfscope}%
\pgfsys@transformshift{2.368370in}{1.991116in}%
\pgfsys@useobject{currentmarker}{}%
\end{pgfscope}%
\begin{pgfscope}%
\pgfsys@transformshift{2.518540in}{1.833468in}%
\pgfsys@useobject{currentmarker}{}%
\end{pgfscope}%
\begin{pgfscope}%
\pgfsys@transformshift{2.668709in}{1.626923in}%
\pgfsys@useobject{currentmarker}{}%
\end{pgfscope}%
\begin{pgfscope}%
\pgfsys@transformshift{2.818878in}{1.216060in}%
\pgfsys@useobject{currentmarker}{}%
\end{pgfscope}%
\begin{pgfscope}%
\pgfsys@transformshift{2.969047in}{1.258871in}%
\pgfsys@useobject{currentmarker}{}%
\end{pgfscope}%
\begin{pgfscope}%
\pgfsys@transformshift{3.119216in}{1.071914in}%
\pgfsys@useobject{currentmarker}{}%
\end{pgfscope}%
\begin{pgfscope}%
\pgfsys@transformshift{3.269385in}{0.903142in}%
\pgfsys@useobject{currentmarker}{}%
\end{pgfscope}%
\begin{pgfscope}%
\pgfsys@transformshift{3.419554in}{0.891228in}%
\pgfsys@useobject{currentmarker}{}%
\end{pgfscope}%
\begin{pgfscope}%
\pgfsys@transformshift{3.569724in}{0.730395in}%
\pgfsys@useobject{currentmarker}{}%
\end{pgfscope}%
\begin{pgfscope}%
\pgfsys@transformshift{3.719893in}{0.596585in}%
\pgfsys@useobject{currentmarker}{}%
\end{pgfscope}%
\end{pgfscope}%
\begin{pgfscope}%
\pgfsetbuttcap%
\pgfsetroundjoin%
\definecolor{currentfill}{rgb}{0.000000,0.000000,0.000000}%
\pgfsetfillcolor{currentfill}%
\pgfsetlinewidth{0.803000pt}%
\definecolor{currentstroke}{rgb}{0.000000,0.000000,0.000000}%
\pgfsetstrokecolor{currentstroke}%
\pgfsetdash{}{0pt}%
\pgfsys@defobject{currentmarker}{\pgfqpoint{0.000000in}{-0.048611in}}{\pgfqpoint{0.000000in}{0.000000in}}{%
\pgfpathmoveto{\pgfqpoint{0.000000in}{0.000000in}}%
\pgfpathlineto{\pgfqpoint{0.000000in}{-0.048611in}}%
\pgfusepath{stroke,fill}%
}%
\begin{pgfscope}%
\pgfsys@transformshift{1.016848in}{0.387222in}%
\pgfsys@useobject{currentmarker}{}%
\end{pgfscope}%
\end{pgfscope}%
\begin{pgfscope}%
\definecolor{textcolor}{rgb}{0.000000,0.000000,0.000000}%
\pgfsetstrokecolor{textcolor}%
\pgfsetfillcolor{textcolor}%
\pgftext[x=1.016848in,y=0.290000in,,top]{\color{textcolor}\sffamily\fontsize{10.000000}{12.000000}\selectfont 2.20}%
\end{pgfscope}%
\begin{pgfscope}%
\pgfsetbuttcap%
\pgfsetroundjoin%
\definecolor{currentfill}{rgb}{0.000000,0.000000,0.000000}%
\pgfsetfillcolor{currentfill}%
\pgfsetlinewidth{0.803000pt}%
\definecolor{currentstroke}{rgb}{0.000000,0.000000,0.000000}%
\pgfsetstrokecolor{currentstroke}%
\pgfsetdash{}{0pt}%
\pgfsys@defobject{currentmarker}{\pgfqpoint{0.000000in}{-0.048611in}}{\pgfqpoint{0.000000in}{0.000000in}}{%
\pgfpathmoveto{\pgfqpoint{0.000000in}{0.000000in}}%
\pgfpathlineto{\pgfqpoint{0.000000in}{-0.048611in}}%
\pgfusepath{stroke,fill}%
}%
\begin{pgfscope}%
\pgfsys@transformshift{1.767694in}{0.387222in}%
\pgfsys@useobject{currentmarker}{}%
\end{pgfscope}%
\end{pgfscope}%
\begin{pgfscope}%
\definecolor{textcolor}{rgb}{0.000000,0.000000,0.000000}%
\pgfsetstrokecolor{textcolor}%
\pgfsetfillcolor{textcolor}%
\pgftext[x=1.767694in,y=0.290000in,,top]{\color{textcolor}\sffamily\fontsize{10.000000}{12.000000}\selectfont 2.25}%
\end{pgfscope}%
\begin{pgfscope}%
\pgfsetbuttcap%
\pgfsetroundjoin%
\definecolor{currentfill}{rgb}{0.000000,0.000000,0.000000}%
\pgfsetfillcolor{currentfill}%
\pgfsetlinewidth{0.803000pt}%
\definecolor{currentstroke}{rgb}{0.000000,0.000000,0.000000}%
\pgfsetstrokecolor{currentstroke}%
\pgfsetdash{}{0pt}%
\pgfsys@defobject{currentmarker}{\pgfqpoint{0.000000in}{-0.048611in}}{\pgfqpoint{0.000000in}{0.000000in}}{%
\pgfpathmoveto{\pgfqpoint{0.000000in}{0.000000in}}%
\pgfpathlineto{\pgfqpoint{0.000000in}{-0.048611in}}%
\pgfusepath{stroke,fill}%
}%
\begin{pgfscope}%
\pgfsys@transformshift{2.518540in}{0.387222in}%
\pgfsys@useobject{currentmarker}{}%
\end{pgfscope}%
\end{pgfscope}%
\begin{pgfscope}%
\definecolor{textcolor}{rgb}{0.000000,0.000000,0.000000}%
\pgfsetstrokecolor{textcolor}%
\pgfsetfillcolor{textcolor}%
\pgftext[x=2.518540in,y=0.290000in,,top]{\color{textcolor}\sffamily\fontsize{10.000000}{12.000000}\selectfont 2.30}%
\end{pgfscope}%
\begin{pgfscope}%
\pgfsetbuttcap%
\pgfsetroundjoin%
\definecolor{currentfill}{rgb}{0.000000,0.000000,0.000000}%
\pgfsetfillcolor{currentfill}%
\pgfsetlinewidth{0.803000pt}%
\definecolor{currentstroke}{rgb}{0.000000,0.000000,0.000000}%
\pgfsetstrokecolor{currentstroke}%
\pgfsetdash{}{0pt}%
\pgfsys@defobject{currentmarker}{\pgfqpoint{0.000000in}{-0.048611in}}{\pgfqpoint{0.000000in}{0.000000in}}{%
\pgfpathmoveto{\pgfqpoint{0.000000in}{0.000000in}}%
\pgfpathlineto{\pgfqpoint{0.000000in}{-0.048611in}}%
\pgfusepath{stroke,fill}%
}%
\begin{pgfscope}%
\pgfsys@transformshift{3.269385in}{0.387222in}%
\pgfsys@useobject{currentmarker}{}%
\end{pgfscope}%
\end{pgfscope}%
\begin{pgfscope}%
\definecolor{textcolor}{rgb}{0.000000,0.000000,0.000000}%
\pgfsetstrokecolor{textcolor}%
\pgfsetfillcolor{textcolor}%
\pgftext[x=3.269385in,y=0.290000in,,top]{\color{textcolor}\sffamily\fontsize{10.000000}{12.000000}\selectfont 2.35}%
\end{pgfscope}%
\begin{pgfscope}%
\pgfsetbuttcap%
\pgfsetroundjoin%
\definecolor{currentfill}{rgb}{0.000000,0.000000,0.000000}%
\pgfsetfillcolor{currentfill}%
\pgfsetlinewidth{0.803000pt}%
\definecolor{currentstroke}{rgb}{0.000000,0.000000,0.000000}%
\pgfsetstrokecolor{currentstroke}%
\pgfsetdash{}{0pt}%
\pgfsys@defobject{currentmarker}{\pgfqpoint{-0.048611in}{0.000000in}}{\pgfqpoint{0.000000in}{0.000000in}}{%
\pgfpathmoveto{\pgfqpoint{0.000000in}{0.000000in}}%
\pgfpathlineto{\pgfqpoint{-0.048611in}{0.000000in}}%
\pgfusepath{stroke,fill}%
}%
\begin{pgfscope}%
\pgfsys@transformshift{0.557222in}{0.670012in}%
\pgfsys@useobject{currentmarker}{}%
\end{pgfscope}%
\end{pgfscope}%
\begin{pgfscope}%
\definecolor{textcolor}{rgb}{0.000000,0.000000,0.000000}%
\pgfsetstrokecolor{textcolor}%
\pgfsetfillcolor{textcolor}%
\pgftext[x=0.150755in,y=0.617251in,left,base]{\color{textcolor}\sffamily\fontsize{10.000000}{12.000000}\selectfont 1.25}%
\end{pgfscope}%
\begin{pgfscope}%
\pgfsetbuttcap%
\pgfsetroundjoin%
\definecolor{currentfill}{rgb}{0.000000,0.000000,0.000000}%
\pgfsetfillcolor{currentfill}%
\pgfsetlinewidth{0.803000pt}%
\definecolor{currentstroke}{rgb}{0.000000,0.000000,0.000000}%
\pgfsetstrokecolor{currentstroke}%
\pgfsetdash{}{0pt}%
\pgfsys@defobject{currentmarker}{\pgfqpoint{-0.048611in}{0.000000in}}{\pgfqpoint{0.000000in}{0.000000in}}{%
\pgfpathmoveto{\pgfqpoint{0.000000in}{0.000000in}}%
\pgfpathlineto{\pgfqpoint{-0.048611in}{0.000000in}}%
\pgfusepath{stroke,fill}%
}%
\begin{pgfscope}%
\pgfsys@transformshift{0.557222in}{1.000452in}%
\pgfsys@useobject{currentmarker}{}%
\end{pgfscope}%
\end{pgfscope}%
\begin{pgfscope}%
\definecolor{textcolor}{rgb}{0.000000,0.000000,0.000000}%
\pgfsetstrokecolor{textcolor}%
\pgfsetfillcolor{textcolor}%
\pgftext[x=0.150755in,y=0.947691in,left,base]{\color{textcolor}\sffamily\fontsize{10.000000}{12.000000}\selectfont 1.50}%
\end{pgfscope}%
\begin{pgfscope}%
\pgfsetbuttcap%
\pgfsetroundjoin%
\definecolor{currentfill}{rgb}{0.000000,0.000000,0.000000}%
\pgfsetfillcolor{currentfill}%
\pgfsetlinewidth{0.803000pt}%
\definecolor{currentstroke}{rgb}{0.000000,0.000000,0.000000}%
\pgfsetstrokecolor{currentstroke}%
\pgfsetdash{}{0pt}%
\pgfsys@defobject{currentmarker}{\pgfqpoint{-0.048611in}{0.000000in}}{\pgfqpoint{0.000000in}{0.000000in}}{%
\pgfpathmoveto{\pgfqpoint{0.000000in}{0.000000in}}%
\pgfpathlineto{\pgfqpoint{-0.048611in}{0.000000in}}%
\pgfusepath{stroke,fill}%
}%
\begin{pgfscope}%
\pgfsys@transformshift{0.557222in}{1.330892in}%
\pgfsys@useobject{currentmarker}{}%
\end{pgfscope}%
\end{pgfscope}%
\begin{pgfscope}%
\definecolor{textcolor}{rgb}{0.000000,0.000000,0.000000}%
\pgfsetstrokecolor{textcolor}%
\pgfsetfillcolor{textcolor}%
\pgftext[x=0.150755in,y=1.278131in,left,base]{\color{textcolor}\sffamily\fontsize{10.000000}{12.000000}\selectfont 1.75}%
\end{pgfscope}%
\begin{pgfscope}%
\pgfsetbuttcap%
\pgfsetroundjoin%
\definecolor{currentfill}{rgb}{0.000000,0.000000,0.000000}%
\pgfsetfillcolor{currentfill}%
\pgfsetlinewidth{0.803000pt}%
\definecolor{currentstroke}{rgb}{0.000000,0.000000,0.000000}%
\pgfsetstrokecolor{currentstroke}%
\pgfsetdash{}{0pt}%
\pgfsys@defobject{currentmarker}{\pgfqpoint{-0.048611in}{0.000000in}}{\pgfqpoint{0.000000in}{0.000000in}}{%
\pgfpathmoveto{\pgfqpoint{0.000000in}{0.000000in}}%
\pgfpathlineto{\pgfqpoint{-0.048611in}{0.000000in}}%
\pgfusepath{stroke,fill}%
}%
\begin{pgfscope}%
\pgfsys@transformshift{0.557222in}{1.661332in}%
\pgfsys@useobject{currentmarker}{}%
\end{pgfscope}%
\end{pgfscope}%
\begin{pgfscope}%
\definecolor{textcolor}{rgb}{0.000000,0.000000,0.000000}%
\pgfsetstrokecolor{textcolor}%
\pgfsetfillcolor{textcolor}%
\pgftext[x=0.150755in,y=1.608571in,left,base]{\color{textcolor}\sffamily\fontsize{10.000000}{12.000000}\selectfont 2.00}%
\end{pgfscope}%
\begin{pgfscope}%
\pgfsetbuttcap%
\pgfsetroundjoin%
\definecolor{currentfill}{rgb}{0.000000,0.000000,0.000000}%
\pgfsetfillcolor{currentfill}%
\pgfsetlinewidth{0.803000pt}%
\definecolor{currentstroke}{rgb}{0.000000,0.000000,0.000000}%
\pgfsetstrokecolor{currentstroke}%
\pgfsetdash{}{0pt}%
\pgfsys@defobject{currentmarker}{\pgfqpoint{-0.048611in}{0.000000in}}{\pgfqpoint{0.000000in}{0.000000in}}{%
\pgfpathmoveto{\pgfqpoint{0.000000in}{0.000000in}}%
\pgfpathlineto{\pgfqpoint{-0.048611in}{0.000000in}}%
\pgfusepath{stroke,fill}%
}%
\begin{pgfscope}%
\pgfsys@transformshift{0.557222in}{1.991772in}%
\pgfsys@useobject{currentmarker}{}%
\end{pgfscope}%
\end{pgfscope}%
\begin{pgfscope}%
\definecolor{textcolor}{rgb}{0.000000,0.000000,0.000000}%
\pgfsetstrokecolor{textcolor}%
\pgfsetfillcolor{textcolor}%
\pgftext[x=0.150755in,y=1.939011in,left,base]{\color{textcolor}\sffamily\fontsize{10.000000}{12.000000}\selectfont 2.25}%
\end{pgfscope}%
\begin{pgfscope}%
\pgfsetbuttcap%
\pgfsetroundjoin%
\definecolor{currentfill}{rgb}{0.000000,0.000000,0.000000}%
\pgfsetfillcolor{currentfill}%
\pgfsetlinewidth{0.803000pt}%
\definecolor{currentstroke}{rgb}{0.000000,0.000000,0.000000}%
\pgfsetstrokecolor{currentstroke}%
\pgfsetdash{}{0pt}%
\pgfsys@defobject{currentmarker}{\pgfqpoint{-0.048611in}{0.000000in}}{\pgfqpoint{0.000000in}{0.000000in}}{%
\pgfpathmoveto{\pgfqpoint{0.000000in}{0.000000in}}%
\pgfpathlineto{\pgfqpoint{-0.048611in}{0.000000in}}%
\pgfusepath{stroke,fill}%
}%
\begin{pgfscope}%
\pgfsys@transformshift{0.557222in}{2.322212in}%
\pgfsys@useobject{currentmarker}{}%
\end{pgfscope}%
\end{pgfscope}%
\begin{pgfscope}%
\definecolor{textcolor}{rgb}{0.000000,0.000000,0.000000}%
\pgfsetstrokecolor{textcolor}%
\pgfsetfillcolor{textcolor}%
\pgftext[x=0.150755in,y=2.269451in,left,base]{\color{textcolor}\sffamily\fontsize{10.000000}{12.000000}\selectfont 2.50}%
\end{pgfscope}%
\begin{pgfscope}%
\pgfsetbuttcap%
\pgfsetroundjoin%
\definecolor{currentfill}{rgb}{0.000000,0.000000,0.000000}%
\pgfsetfillcolor{currentfill}%
\pgfsetlinewidth{0.803000pt}%
\definecolor{currentstroke}{rgb}{0.000000,0.000000,0.000000}%
\pgfsetstrokecolor{currentstroke}%
\pgfsetdash{}{0pt}%
\pgfsys@defobject{currentmarker}{\pgfqpoint{-0.048611in}{0.000000in}}{\pgfqpoint{0.000000in}{0.000000in}}{%
\pgfpathmoveto{\pgfqpoint{0.000000in}{0.000000in}}%
\pgfpathlineto{\pgfqpoint{-0.048611in}{0.000000in}}%
\pgfusepath{stroke,fill}%
}%
\begin{pgfscope}%
\pgfsys@transformshift{0.557222in}{2.652652in}%
\pgfsys@useobject{currentmarker}{}%
\end{pgfscope}%
\end{pgfscope}%
\begin{pgfscope}%
\definecolor{textcolor}{rgb}{0.000000,0.000000,0.000000}%
\pgfsetstrokecolor{textcolor}%
\pgfsetfillcolor{textcolor}%
\pgftext[x=0.150755in,y=2.599891in,left,base]{\color{textcolor}\sffamily\fontsize{10.000000}{12.000000}\selectfont 2.75}%
\end{pgfscope}%
\begin{pgfscope}%
\pgfpathrectangle{\pgfqpoint{0.557222in}{0.387222in}}{\pgfqpoint{3.321958in}{2.324444in}}%
\pgfusepath{clip}%
\pgfsetrectcap%
\pgfsetroundjoin%
\pgfsetlinewidth{1.505625pt}%
\definecolor{currentstroke}{rgb}{0.121569,0.466667,0.705882}%
\pgfsetstrokecolor{currentstroke}%
\pgfsetdash{}{0pt}%
\pgfpathmoveto{\pgfqpoint{0.716510in}{0.776716in}}%
\pgfpathlineto{\pgfqpoint{0.866679in}{0.789711in}}%
\pgfpathlineto{\pgfqpoint{1.016848in}{0.907769in}}%
\pgfpathlineto{\pgfqpoint{1.167017in}{0.933249in}}%
\pgfpathlineto{\pgfqpoint{1.317186in}{1.468143in}}%
\pgfpathlineto{\pgfqpoint{1.467355in}{1.161533in}}%
\pgfpathlineto{\pgfqpoint{1.617525in}{1.303624in}}%
\pgfpathlineto{\pgfqpoint{1.767694in}{1.578184in}}%
\pgfpathlineto{\pgfqpoint{1.917863in}{2.012539in}}%
\pgfpathlineto{\pgfqpoint{2.068032in}{2.013219in}}%
\pgfpathlineto{\pgfqpoint{2.218201in}{1.943757in}}%
\pgfpathlineto{\pgfqpoint{2.368370in}{1.991116in}}%
\pgfpathlineto{\pgfqpoint{2.518540in}{1.833468in}}%
\pgfpathlineto{\pgfqpoint{2.668709in}{1.626923in}}%
\pgfpathlineto{\pgfqpoint{2.818878in}{1.216060in}}%
\pgfpathlineto{\pgfqpoint{2.969047in}{1.258871in}}%
\pgfpathlineto{\pgfqpoint{3.119216in}{1.071914in}}%
\pgfpathlineto{\pgfqpoint{3.269385in}{0.903142in}}%
\pgfpathlineto{\pgfqpoint{3.419554in}{0.891228in}}%
\pgfpathlineto{\pgfqpoint{3.569724in}{0.730395in}}%
\pgfpathlineto{\pgfqpoint{3.719893in}{0.596585in}}%
\pgfusepath{stroke}%
\end{pgfscope}%
\begin{pgfscope}%
\pgfsetrectcap%
\pgfsetmiterjoin%
\pgfsetlinewidth{0.803000pt}%
\definecolor{currentstroke}{rgb}{0.000000,0.000000,0.000000}%
\pgfsetstrokecolor{currentstroke}%
\pgfsetdash{}{0pt}%
\pgfpathmoveto{\pgfqpoint{0.557222in}{0.387222in}}%
\pgfpathlineto{\pgfqpoint{0.557222in}{2.711667in}}%
\pgfusepath{stroke}%
\end{pgfscope}%
\begin{pgfscope}%
\pgfsetrectcap%
\pgfsetmiterjoin%
\pgfsetlinewidth{0.803000pt}%
\definecolor{currentstroke}{rgb}{0.000000,0.000000,0.000000}%
\pgfsetstrokecolor{currentstroke}%
\pgfsetdash{}{0pt}%
\pgfpathmoveto{\pgfqpoint{3.879180in}{0.387222in}}%
\pgfpathlineto{\pgfqpoint{3.879180in}{2.711667in}}%
\pgfusepath{stroke}%
\end{pgfscope}%
\begin{pgfscope}%
\pgfsetrectcap%
\pgfsetmiterjoin%
\pgfsetlinewidth{0.803000pt}%
\definecolor{currentstroke}{rgb}{0.000000,0.000000,0.000000}%
\pgfsetstrokecolor{currentstroke}%
\pgfsetdash{}{0pt}%
\pgfpathmoveto{\pgfqpoint{0.557222in}{0.387222in}}%
\pgfpathlineto{\pgfqpoint{3.879180in}{0.387222in}}%
\pgfusepath{stroke}%
\end{pgfscope}%
\begin{pgfscope}%
\pgfsetrectcap%
\pgfsetmiterjoin%
\pgfsetlinewidth{0.803000pt}%
\definecolor{currentstroke}{rgb}{0.000000,0.000000,0.000000}%
\pgfsetstrokecolor{currentstroke}%
\pgfsetdash{}{0pt}%
\pgfpathmoveto{\pgfqpoint{0.557222in}{2.711667in}}%
\pgfpathlineto{\pgfqpoint{3.879180in}{2.711667in}}%
\pgfusepath{stroke}%
\end{pgfscope}%
\begin{pgfscope}%
\definecolor{textcolor}{rgb}{0.000000,0.000000,0.000000}%
\pgfsetstrokecolor{textcolor}%
\pgfsetfillcolor{textcolor}%
\pgftext[x=2.218201in,y=2.795000in,,base]{\color{textcolor}\sffamily\fontsize{12.000000}{14.400000}\selectfont \(\displaystyle  N = 64 \)}%
\end{pgfscope}%
\begin{pgfscope}%
\pgfsetbuttcap%
\pgfsetmiterjoin%
\definecolor{currentfill}{rgb}{1.000000,1.000000,1.000000}%
\pgfsetfillcolor{currentfill}%
\pgfsetlinewidth{0.000000pt}%
\definecolor{currentstroke}{rgb}{0.000000,0.000000,0.000000}%
\pgfsetstrokecolor{currentstroke}%
\pgfsetstrokeopacity{0.000000}%
\pgfsetdash{}{0pt}%
\pgfpathmoveto{\pgfqpoint{4.382652in}{0.387222in}}%
\pgfpathlineto{\pgfqpoint{7.704610in}{0.387222in}}%
\pgfpathlineto{\pgfqpoint{7.704610in}{2.711667in}}%
\pgfpathlineto{\pgfqpoint{4.382652in}{2.711667in}}%
\pgfpathclose%
\pgfusepath{fill}%
\end{pgfscope}%
\begin{pgfscope}%
\pgfpathrectangle{\pgfqpoint{4.382652in}{0.387222in}}{\pgfqpoint{3.321958in}{2.324444in}}%
\pgfusepath{clip}%
\pgfsetbuttcap%
\pgfsetroundjoin%
\definecolor{currentfill}{rgb}{0.121569,0.466667,0.705882}%
\pgfsetfillcolor{currentfill}%
\pgfsetfillopacity{0.300000}%
\pgfsetlinewidth{1.003750pt}%
\definecolor{currentstroke}{rgb}{0.121569,0.466667,0.705882}%
\pgfsetstrokecolor{currentstroke}%
\pgfsetstrokeopacity{0.300000}%
\pgfsetdash{}{0pt}%
\pgfpathmoveto{\pgfqpoint{4.541940in}{0.798583in}}%
\pgfpathlineto{\pgfqpoint{4.541940in}{0.647573in}}%
\pgfpathlineto{\pgfqpoint{4.692109in}{0.604111in}}%
\pgfpathlineto{\pgfqpoint{4.842278in}{0.828834in}}%
\pgfpathlineto{\pgfqpoint{4.992447in}{0.738725in}}%
\pgfpathlineto{\pgfqpoint{5.142617in}{0.661624in}}%
\pgfpathlineto{\pgfqpoint{5.292786in}{0.492879in}}%
\pgfpathlineto{\pgfqpoint{5.442955in}{0.919764in}}%
\pgfpathlineto{\pgfqpoint{5.593124in}{1.015895in}}%
\pgfpathlineto{\pgfqpoint{5.743293in}{1.077108in}}%
\pgfpathlineto{\pgfqpoint{5.893462in}{0.710039in}}%
\pgfpathlineto{\pgfqpoint{6.043631in}{0.842761in}}%
\pgfpathlineto{\pgfqpoint{6.193801in}{0.991066in}}%
\pgfpathlineto{\pgfqpoint{6.343970in}{0.801318in}}%
\pgfpathlineto{\pgfqpoint{6.494139in}{0.841194in}}%
\pgfpathlineto{\pgfqpoint{6.644308in}{0.898934in}}%
\pgfpathlineto{\pgfqpoint{6.794477in}{0.785590in}}%
\pgfpathlineto{\pgfqpoint{6.944646in}{0.773726in}}%
\pgfpathlineto{\pgfqpoint{7.094816in}{0.743284in}}%
\pgfpathlineto{\pgfqpoint{7.244985in}{0.731827in}}%
\pgfpathlineto{\pgfqpoint{7.395154in}{0.700926in}}%
\pgfpathlineto{\pgfqpoint{7.545323in}{0.695516in}}%
\pgfpathlineto{\pgfqpoint{7.545323in}{0.773264in}}%
\pgfpathlineto{\pgfqpoint{7.545323in}{0.773264in}}%
\pgfpathlineto{\pgfqpoint{7.395154in}{0.816466in}}%
\pgfpathlineto{\pgfqpoint{7.244985in}{0.777915in}}%
\pgfpathlineto{\pgfqpoint{7.094816in}{0.936704in}}%
\pgfpathlineto{\pgfqpoint{6.944646in}{1.156296in}}%
\pgfpathlineto{\pgfqpoint{6.794477in}{0.947118in}}%
\pgfpathlineto{\pgfqpoint{6.644308in}{1.203035in}}%
\pgfpathlineto{\pgfqpoint{6.494139in}{1.719792in}}%
\pgfpathlineto{\pgfqpoint{6.343970in}{1.562496in}}%
\pgfpathlineto{\pgfqpoint{6.193801in}{1.631018in}}%
\pgfpathlineto{\pgfqpoint{6.043631in}{1.833173in}}%
\pgfpathlineto{\pgfqpoint{5.893462in}{1.935899in}}%
\pgfpathlineto{\pgfqpoint{5.743293in}{2.029562in}}%
\pgfpathlineto{\pgfqpoint{5.593124in}{1.340005in}}%
\pgfpathlineto{\pgfqpoint{5.442955in}{1.215031in}}%
\pgfpathlineto{\pgfqpoint{5.292786in}{2.606010in}}%
\pgfpathlineto{\pgfqpoint{5.142617in}{1.745942in}}%
\pgfpathlineto{\pgfqpoint{4.992447in}{1.250239in}}%
\pgfpathlineto{\pgfqpoint{4.842278in}{2.511909in}}%
\pgfpathlineto{\pgfqpoint{4.692109in}{1.142653in}}%
\pgfpathlineto{\pgfqpoint{4.541940in}{0.798583in}}%
\pgfpathclose%
\pgfusepath{stroke,fill}%
\end{pgfscope}%
\begin{pgfscope}%
\pgfpathrectangle{\pgfqpoint{4.382652in}{0.387222in}}{\pgfqpoint{3.321958in}{2.324444in}}%
\pgfusepath{clip}%
\pgfsetbuttcap%
\pgfsetroundjoin%
\definecolor{currentfill}{rgb}{0.121569,0.466667,0.705882}%
\pgfsetfillcolor{currentfill}%
\pgfsetlinewidth{1.003750pt}%
\definecolor{currentstroke}{rgb}{0.121569,0.466667,0.705882}%
\pgfsetstrokecolor{currentstroke}%
\pgfsetdash{}{0pt}%
\pgfsys@defobject{currentmarker}{\pgfqpoint{-0.009821in}{-0.009821in}}{\pgfqpoint{0.009821in}{0.009821in}}{%
\pgfpathmoveto{\pgfqpoint{0.000000in}{-0.009821in}}%
\pgfpathcurveto{\pgfqpoint{0.002605in}{-0.009821in}}{\pgfqpoint{0.005103in}{-0.008786in}}{\pgfqpoint{0.006944in}{-0.006944in}}%
\pgfpathcurveto{\pgfqpoint{0.008786in}{-0.005103in}}{\pgfqpoint{0.009821in}{-0.002605in}}{\pgfqpoint{0.009821in}{0.000000in}}%
\pgfpathcurveto{\pgfqpoint{0.009821in}{0.002605in}}{\pgfqpoint{0.008786in}{0.005103in}}{\pgfqpoint{0.006944in}{0.006944in}}%
\pgfpathcurveto{\pgfqpoint{0.005103in}{0.008786in}}{\pgfqpoint{0.002605in}{0.009821in}}{\pgfqpoint{0.000000in}{0.009821in}}%
\pgfpathcurveto{\pgfqpoint{-0.002605in}{0.009821in}}{\pgfqpoint{-0.005103in}{0.008786in}}{\pgfqpoint{-0.006944in}{0.006944in}}%
\pgfpathcurveto{\pgfqpoint{-0.008786in}{0.005103in}}{\pgfqpoint{-0.009821in}{0.002605in}}{\pgfqpoint{-0.009821in}{0.000000in}}%
\pgfpathcurveto{\pgfqpoint{-0.009821in}{-0.002605in}}{\pgfqpoint{-0.008786in}{-0.005103in}}{\pgfqpoint{-0.006944in}{-0.006944in}}%
\pgfpathcurveto{\pgfqpoint{-0.005103in}{-0.008786in}}{\pgfqpoint{-0.002605in}{-0.009821in}}{\pgfqpoint{0.000000in}{-0.009821in}}%
\pgfpathclose%
\pgfusepath{stroke,fill}%
}%
\begin{pgfscope}%
\pgfsys@transformshift{4.541940in}{0.723078in}%
\pgfsys@useobject{currentmarker}{}%
\end{pgfscope}%
\begin{pgfscope}%
\pgfsys@transformshift{4.692109in}{0.873382in}%
\pgfsys@useobject{currentmarker}{}%
\end{pgfscope}%
\begin{pgfscope}%
\pgfsys@transformshift{4.842278in}{1.670372in}%
\pgfsys@useobject{currentmarker}{}%
\end{pgfscope}%
\begin{pgfscope}%
\pgfsys@transformshift{4.992447in}{0.994482in}%
\pgfsys@useobject{currentmarker}{}%
\end{pgfscope}%
\begin{pgfscope}%
\pgfsys@transformshift{5.142617in}{1.203783in}%
\pgfsys@useobject{currentmarker}{}%
\end{pgfscope}%
\begin{pgfscope}%
\pgfsys@transformshift{5.292786in}{1.549444in}%
\pgfsys@useobject{currentmarker}{}%
\end{pgfscope}%
\begin{pgfscope}%
\pgfsys@transformshift{5.442955in}{1.067398in}%
\pgfsys@useobject{currentmarker}{}%
\end{pgfscope}%
\begin{pgfscope}%
\pgfsys@transformshift{5.593124in}{1.177950in}%
\pgfsys@useobject{currentmarker}{}%
\end{pgfscope}%
\begin{pgfscope}%
\pgfsys@transformshift{5.743293in}{1.553335in}%
\pgfsys@useobject{currentmarker}{}%
\end{pgfscope}%
\begin{pgfscope}%
\pgfsys@transformshift{5.893462in}{1.322969in}%
\pgfsys@useobject{currentmarker}{}%
\end{pgfscope}%
\begin{pgfscope}%
\pgfsys@transformshift{6.043631in}{1.337967in}%
\pgfsys@useobject{currentmarker}{}%
\end{pgfscope}%
\begin{pgfscope}%
\pgfsys@transformshift{6.193801in}{1.311042in}%
\pgfsys@useobject{currentmarker}{}%
\end{pgfscope}%
\begin{pgfscope}%
\pgfsys@transformshift{6.343970in}{1.181907in}%
\pgfsys@useobject{currentmarker}{}%
\end{pgfscope}%
\begin{pgfscope}%
\pgfsys@transformshift{6.494139in}{1.280493in}%
\pgfsys@useobject{currentmarker}{}%
\end{pgfscope}%
\begin{pgfscope}%
\pgfsys@transformshift{6.644308in}{1.050984in}%
\pgfsys@useobject{currentmarker}{}%
\end{pgfscope}%
\begin{pgfscope}%
\pgfsys@transformshift{6.794477in}{0.866354in}%
\pgfsys@useobject{currentmarker}{}%
\end{pgfscope}%
\begin{pgfscope}%
\pgfsys@transformshift{6.944646in}{0.965011in}%
\pgfsys@useobject{currentmarker}{}%
\end{pgfscope}%
\begin{pgfscope}%
\pgfsys@transformshift{7.094816in}{0.839994in}%
\pgfsys@useobject{currentmarker}{}%
\end{pgfscope}%
\begin{pgfscope}%
\pgfsys@transformshift{7.244985in}{0.754871in}%
\pgfsys@useobject{currentmarker}{}%
\end{pgfscope}%
\begin{pgfscope}%
\pgfsys@transformshift{7.395154in}{0.758696in}%
\pgfsys@useobject{currentmarker}{}%
\end{pgfscope}%
\begin{pgfscope}%
\pgfsys@transformshift{7.545323in}{0.734390in}%
\pgfsys@useobject{currentmarker}{}%
\end{pgfscope}%
\end{pgfscope}%
\begin{pgfscope}%
\pgfsetbuttcap%
\pgfsetroundjoin%
\definecolor{currentfill}{rgb}{0.000000,0.000000,0.000000}%
\pgfsetfillcolor{currentfill}%
\pgfsetlinewidth{0.803000pt}%
\definecolor{currentstroke}{rgb}{0.000000,0.000000,0.000000}%
\pgfsetstrokecolor{currentstroke}%
\pgfsetdash{}{0pt}%
\pgfsys@defobject{currentmarker}{\pgfqpoint{0.000000in}{-0.048611in}}{\pgfqpoint{0.000000in}{0.000000in}}{%
\pgfpathmoveto{\pgfqpoint{0.000000in}{0.000000in}}%
\pgfpathlineto{\pgfqpoint{0.000000in}{-0.048611in}}%
\pgfusepath{stroke,fill}%
}%
\begin{pgfscope}%
\pgfsys@transformshift{4.992447in}{0.387222in}%
\pgfsys@useobject{currentmarker}{}%
\end{pgfscope}%
\end{pgfscope}%
\begin{pgfscope}%
\definecolor{textcolor}{rgb}{0.000000,0.000000,0.000000}%
\pgfsetstrokecolor{textcolor}%
\pgfsetfillcolor{textcolor}%
\pgftext[x=4.992447in,y=0.290000in,,top]{\color{textcolor}\sffamily\fontsize{10.000000}{12.000000}\selectfont 2.20}%
\end{pgfscope}%
\begin{pgfscope}%
\pgfsetbuttcap%
\pgfsetroundjoin%
\definecolor{currentfill}{rgb}{0.000000,0.000000,0.000000}%
\pgfsetfillcolor{currentfill}%
\pgfsetlinewidth{0.803000pt}%
\definecolor{currentstroke}{rgb}{0.000000,0.000000,0.000000}%
\pgfsetstrokecolor{currentstroke}%
\pgfsetdash{}{0pt}%
\pgfsys@defobject{currentmarker}{\pgfqpoint{0.000000in}{-0.048611in}}{\pgfqpoint{0.000000in}{0.000000in}}{%
\pgfpathmoveto{\pgfqpoint{0.000000in}{0.000000in}}%
\pgfpathlineto{\pgfqpoint{0.000000in}{-0.048611in}}%
\pgfusepath{stroke,fill}%
}%
\begin{pgfscope}%
\pgfsys@transformshift{5.743293in}{0.387222in}%
\pgfsys@useobject{currentmarker}{}%
\end{pgfscope}%
\end{pgfscope}%
\begin{pgfscope}%
\definecolor{textcolor}{rgb}{0.000000,0.000000,0.000000}%
\pgfsetstrokecolor{textcolor}%
\pgfsetfillcolor{textcolor}%
\pgftext[x=5.743293in,y=0.290000in,,top]{\color{textcolor}\sffamily\fontsize{10.000000}{12.000000}\selectfont 2.25}%
\end{pgfscope}%
\begin{pgfscope}%
\pgfsetbuttcap%
\pgfsetroundjoin%
\definecolor{currentfill}{rgb}{0.000000,0.000000,0.000000}%
\pgfsetfillcolor{currentfill}%
\pgfsetlinewidth{0.803000pt}%
\definecolor{currentstroke}{rgb}{0.000000,0.000000,0.000000}%
\pgfsetstrokecolor{currentstroke}%
\pgfsetdash{}{0pt}%
\pgfsys@defobject{currentmarker}{\pgfqpoint{0.000000in}{-0.048611in}}{\pgfqpoint{0.000000in}{0.000000in}}{%
\pgfpathmoveto{\pgfqpoint{0.000000in}{0.000000in}}%
\pgfpathlineto{\pgfqpoint{0.000000in}{-0.048611in}}%
\pgfusepath{stroke,fill}%
}%
\begin{pgfscope}%
\pgfsys@transformshift{6.494139in}{0.387222in}%
\pgfsys@useobject{currentmarker}{}%
\end{pgfscope}%
\end{pgfscope}%
\begin{pgfscope}%
\definecolor{textcolor}{rgb}{0.000000,0.000000,0.000000}%
\pgfsetstrokecolor{textcolor}%
\pgfsetfillcolor{textcolor}%
\pgftext[x=6.494139in,y=0.290000in,,top]{\color{textcolor}\sffamily\fontsize{10.000000}{12.000000}\selectfont 2.30}%
\end{pgfscope}%
\begin{pgfscope}%
\pgfsetbuttcap%
\pgfsetroundjoin%
\definecolor{currentfill}{rgb}{0.000000,0.000000,0.000000}%
\pgfsetfillcolor{currentfill}%
\pgfsetlinewidth{0.803000pt}%
\definecolor{currentstroke}{rgb}{0.000000,0.000000,0.000000}%
\pgfsetstrokecolor{currentstroke}%
\pgfsetdash{}{0pt}%
\pgfsys@defobject{currentmarker}{\pgfqpoint{0.000000in}{-0.048611in}}{\pgfqpoint{0.000000in}{0.000000in}}{%
\pgfpathmoveto{\pgfqpoint{0.000000in}{0.000000in}}%
\pgfpathlineto{\pgfqpoint{0.000000in}{-0.048611in}}%
\pgfusepath{stroke,fill}%
}%
\begin{pgfscope}%
\pgfsys@transformshift{7.244985in}{0.387222in}%
\pgfsys@useobject{currentmarker}{}%
\end{pgfscope}%
\end{pgfscope}%
\begin{pgfscope}%
\definecolor{textcolor}{rgb}{0.000000,0.000000,0.000000}%
\pgfsetstrokecolor{textcolor}%
\pgfsetfillcolor{textcolor}%
\pgftext[x=7.244985in,y=0.290000in,,top]{\color{textcolor}\sffamily\fontsize{10.000000}{12.000000}\selectfont 2.35}%
\end{pgfscope}%
\begin{pgfscope}%
\pgfsetbuttcap%
\pgfsetroundjoin%
\definecolor{currentfill}{rgb}{0.000000,0.000000,0.000000}%
\pgfsetfillcolor{currentfill}%
\pgfsetlinewidth{0.803000pt}%
\definecolor{currentstroke}{rgb}{0.000000,0.000000,0.000000}%
\pgfsetstrokecolor{currentstroke}%
\pgfsetdash{}{0pt}%
\pgfsys@defobject{currentmarker}{\pgfqpoint{-0.048611in}{0.000000in}}{\pgfqpoint{0.000000in}{0.000000in}}{%
\pgfpathmoveto{\pgfqpoint{0.000000in}{0.000000in}}%
\pgfpathlineto{\pgfqpoint{-0.048611in}{0.000000in}}%
\pgfusepath{stroke,fill}%
}%
\begin{pgfscope}%
\pgfsys@transformshift{4.382652in}{0.581346in}%
\pgfsys@useobject{currentmarker}{}%
\end{pgfscope}%
\end{pgfscope}%
\begin{pgfscope}%
\definecolor{textcolor}{rgb}{0.000000,0.000000,0.000000}%
\pgfsetstrokecolor{textcolor}%
\pgfsetfillcolor{textcolor}%
\pgftext[x=4.064551in,y=0.528584in,left,base]{\color{textcolor}\sffamily\fontsize{10.000000}{12.000000}\selectfont 1.0}%
\end{pgfscope}%
\begin{pgfscope}%
\pgfsetbuttcap%
\pgfsetroundjoin%
\definecolor{currentfill}{rgb}{0.000000,0.000000,0.000000}%
\pgfsetfillcolor{currentfill}%
\pgfsetlinewidth{0.803000pt}%
\definecolor{currentstroke}{rgb}{0.000000,0.000000,0.000000}%
\pgfsetstrokecolor{currentstroke}%
\pgfsetdash{}{0pt}%
\pgfsys@defobject{currentmarker}{\pgfqpoint{-0.048611in}{0.000000in}}{\pgfqpoint{0.000000in}{0.000000in}}{%
\pgfpathmoveto{\pgfqpoint{0.000000in}{0.000000in}}%
\pgfpathlineto{\pgfqpoint{-0.048611in}{0.000000in}}%
\pgfusepath{stroke,fill}%
}%
\begin{pgfscope}%
\pgfsys@transformshift{4.382652in}{0.909936in}%
\pgfsys@useobject{currentmarker}{}%
\end{pgfscope}%
\end{pgfscope}%
\begin{pgfscope}%
\definecolor{textcolor}{rgb}{0.000000,0.000000,0.000000}%
\pgfsetstrokecolor{textcolor}%
\pgfsetfillcolor{textcolor}%
\pgftext[x=4.064551in,y=0.857174in,left,base]{\color{textcolor}\sffamily\fontsize{10.000000}{12.000000}\selectfont 1.5}%
\end{pgfscope}%
\begin{pgfscope}%
\pgfsetbuttcap%
\pgfsetroundjoin%
\definecolor{currentfill}{rgb}{0.000000,0.000000,0.000000}%
\pgfsetfillcolor{currentfill}%
\pgfsetlinewidth{0.803000pt}%
\definecolor{currentstroke}{rgb}{0.000000,0.000000,0.000000}%
\pgfsetstrokecolor{currentstroke}%
\pgfsetdash{}{0pt}%
\pgfsys@defobject{currentmarker}{\pgfqpoint{-0.048611in}{0.000000in}}{\pgfqpoint{0.000000in}{0.000000in}}{%
\pgfpathmoveto{\pgfqpoint{0.000000in}{0.000000in}}%
\pgfpathlineto{\pgfqpoint{-0.048611in}{0.000000in}}%
\pgfusepath{stroke,fill}%
}%
\begin{pgfscope}%
\pgfsys@transformshift{4.382652in}{1.238525in}%
\pgfsys@useobject{currentmarker}{}%
\end{pgfscope}%
\end{pgfscope}%
\begin{pgfscope}%
\definecolor{textcolor}{rgb}{0.000000,0.000000,0.000000}%
\pgfsetstrokecolor{textcolor}%
\pgfsetfillcolor{textcolor}%
\pgftext[x=4.064551in,y=1.185764in,left,base]{\color{textcolor}\sffamily\fontsize{10.000000}{12.000000}\selectfont 2.0}%
\end{pgfscope}%
\begin{pgfscope}%
\pgfsetbuttcap%
\pgfsetroundjoin%
\definecolor{currentfill}{rgb}{0.000000,0.000000,0.000000}%
\pgfsetfillcolor{currentfill}%
\pgfsetlinewidth{0.803000pt}%
\definecolor{currentstroke}{rgb}{0.000000,0.000000,0.000000}%
\pgfsetstrokecolor{currentstroke}%
\pgfsetdash{}{0pt}%
\pgfsys@defobject{currentmarker}{\pgfqpoint{-0.048611in}{0.000000in}}{\pgfqpoint{0.000000in}{0.000000in}}{%
\pgfpathmoveto{\pgfqpoint{0.000000in}{0.000000in}}%
\pgfpathlineto{\pgfqpoint{-0.048611in}{0.000000in}}%
\pgfusepath{stroke,fill}%
}%
\begin{pgfscope}%
\pgfsys@transformshift{4.382652in}{1.567115in}%
\pgfsys@useobject{currentmarker}{}%
\end{pgfscope}%
\end{pgfscope}%
\begin{pgfscope}%
\definecolor{textcolor}{rgb}{0.000000,0.000000,0.000000}%
\pgfsetstrokecolor{textcolor}%
\pgfsetfillcolor{textcolor}%
\pgftext[x=4.064551in,y=1.514353in,left,base]{\color{textcolor}\sffamily\fontsize{10.000000}{12.000000}\selectfont 2.5}%
\end{pgfscope}%
\begin{pgfscope}%
\pgfsetbuttcap%
\pgfsetroundjoin%
\definecolor{currentfill}{rgb}{0.000000,0.000000,0.000000}%
\pgfsetfillcolor{currentfill}%
\pgfsetlinewidth{0.803000pt}%
\definecolor{currentstroke}{rgb}{0.000000,0.000000,0.000000}%
\pgfsetstrokecolor{currentstroke}%
\pgfsetdash{}{0pt}%
\pgfsys@defobject{currentmarker}{\pgfqpoint{-0.048611in}{0.000000in}}{\pgfqpoint{0.000000in}{0.000000in}}{%
\pgfpathmoveto{\pgfqpoint{0.000000in}{0.000000in}}%
\pgfpathlineto{\pgfqpoint{-0.048611in}{0.000000in}}%
\pgfusepath{stroke,fill}%
}%
\begin{pgfscope}%
\pgfsys@transformshift{4.382652in}{1.895704in}%
\pgfsys@useobject{currentmarker}{}%
\end{pgfscope}%
\end{pgfscope}%
\begin{pgfscope}%
\definecolor{textcolor}{rgb}{0.000000,0.000000,0.000000}%
\pgfsetstrokecolor{textcolor}%
\pgfsetfillcolor{textcolor}%
\pgftext[x=4.064551in,y=1.842943in,left,base]{\color{textcolor}\sffamily\fontsize{10.000000}{12.000000}\selectfont 3.0}%
\end{pgfscope}%
\begin{pgfscope}%
\pgfsetbuttcap%
\pgfsetroundjoin%
\definecolor{currentfill}{rgb}{0.000000,0.000000,0.000000}%
\pgfsetfillcolor{currentfill}%
\pgfsetlinewidth{0.803000pt}%
\definecolor{currentstroke}{rgb}{0.000000,0.000000,0.000000}%
\pgfsetstrokecolor{currentstroke}%
\pgfsetdash{}{0pt}%
\pgfsys@defobject{currentmarker}{\pgfqpoint{-0.048611in}{0.000000in}}{\pgfqpoint{0.000000in}{0.000000in}}{%
\pgfpathmoveto{\pgfqpoint{0.000000in}{0.000000in}}%
\pgfpathlineto{\pgfqpoint{-0.048611in}{0.000000in}}%
\pgfusepath{stroke,fill}%
}%
\begin{pgfscope}%
\pgfsys@transformshift{4.382652in}{2.224294in}%
\pgfsys@useobject{currentmarker}{}%
\end{pgfscope}%
\end{pgfscope}%
\begin{pgfscope}%
\definecolor{textcolor}{rgb}{0.000000,0.000000,0.000000}%
\pgfsetstrokecolor{textcolor}%
\pgfsetfillcolor{textcolor}%
\pgftext[x=4.064551in,y=2.171533in,left,base]{\color{textcolor}\sffamily\fontsize{10.000000}{12.000000}\selectfont 3.5}%
\end{pgfscope}%
\begin{pgfscope}%
\pgfsetbuttcap%
\pgfsetroundjoin%
\definecolor{currentfill}{rgb}{0.000000,0.000000,0.000000}%
\pgfsetfillcolor{currentfill}%
\pgfsetlinewidth{0.803000pt}%
\definecolor{currentstroke}{rgb}{0.000000,0.000000,0.000000}%
\pgfsetstrokecolor{currentstroke}%
\pgfsetdash{}{0pt}%
\pgfsys@defobject{currentmarker}{\pgfqpoint{-0.048611in}{0.000000in}}{\pgfqpoint{0.000000in}{0.000000in}}{%
\pgfpathmoveto{\pgfqpoint{0.000000in}{0.000000in}}%
\pgfpathlineto{\pgfqpoint{-0.048611in}{0.000000in}}%
\pgfusepath{stroke,fill}%
}%
\begin{pgfscope}%
\pgfsys@transformshift{4.382652in}{2.552884in}%
\pgfsys@useobject{currentmarker}{}%
\end{pgfscope}%
\end{pgfscope}%
\begin{pgfscope}%
\definecolor{textcolor}{rgb}{0.000000,0.000000,0.000000}%
\pgfsetstrokecolor{textcolor}%
\pgfsetfillcolor{textcolor}%
\pgftext[x=4.064551in,y=2.500122in,left,base]{\color{textcolor}\sffamily\fontsize{10.000000}{12.000000}\selectfont 4.0}%
\end{pgfscope}%
\begin{pgfscope}%
\pgfpathrectangle{\pgfqpoint{4.382652in}{0.387222in}}{\pgfqpoint{3.321958in}{2.324444in}}%
\pgfusepath{clip}%
\pgfsetrectcap%
\pgfsetroundjoin%
\pgfsetlinewidth{1.505625pt}%
\definecolor{currentstroke}{rgb}{0.121569,0.466667,0.705882}%
\pgfsetstrokecolor{currentstroke}%
\pgfsetdash{}{0pt}%
\pgfpathmoveto{\pgfqpoint{4.541940in}{0.723078in}}%
\pgfpathlineto{\pgfqpoint{4.692109in}{0.873382in}}%
\pgfpathlineto{\pgfqpoint{4.842278in}{1.670372in}}%
\pgfpathlineto{\pgfqpoint{4.992447in}{0.994482in}}%
\pgfpathlineto{\pgfqpoint{5.142617in}{1.203783in}}%
\pgfpathlineto{\pgfqpoint{5.292786in}{1.549444in}}%
\pgfpathlineto{\pgfqpoint{5.442955in}{1.067398in}}%
\pgfpathlineto{\pgfqpoint{5.593124in}{1.177950in}}%
\pgfpathlineto{\pgfqpoint{5.743293in}{1.553335in}}%
\pgfpathlineto{\pgfqpoint{5.893462in}{1.322969in}}%
\pgfpathlineto{\pgfqpoint{6.043631in}{1.337967in}}%
\pgfpathlineto{\pgfqpoint{6.193801in}{1.311042in}}%
\pgfpathlineto{\pgfqpoint{6.343970in}{1.181907in}}%
\pgfpathlineto{\pgfqpoint{6.494139in}{1.280493in}}%
\pgfpathlineto{\pgfqpoint{6.644308in}{1.050984in}}%
\pgfpathlineto{\pgfqpoint{6.794477in}{0.866354in}}%
\pgfpathlineto{\pgfqpoint{6.944646in}{0.965011in}}%
\pgfpathlineto{\pgfqpoint{7.094816in}{0.839994in}}%
\pgfpathlineto{\pgfqpoint{7.244985in}{0.754871in}}%
\pgfpathlineto{\pgfqpoint{7.395154in}{0.758696in}}%
\pgfpathlineto{\pgfqpoint{7.545323in}{0.734390in}}%
\pgfusepath{stroke}%
\end{pgfscope}%
\begin{pgfscope}%
\pgfsetrectcap%
\pgfsetmiterjoin%
\pgfsetlinewidth{0.803000pt}%
\definecolor{currentstroke}{rgb}{0.000000,0.000000,0.000000}%
\pgfsetstrokecolor{currentstroke}%
\pgfsetdash{}{0pt}%
\pgfpathmoveto{\pgfqpoint{4.382652in}{0.387222in}}%
\pgfpathlineto{\pgfqpoint{4.382652in}{2.711667in}}%
\pgfusepath{stroke}%
\end{pgfscope}%
\begin{pgfscope}%
\pgfsetrectcap%
\pgfsetmiterjoin%
\pgfsetlinewidth{0.803000pt}%
\definecolor{currentstroke}{rgb}{0.000000,0.000000,0.000000}%
\pgfsetstrokecolor{currentstroke}%
\pgfsetdash{}{0pt}%
\pgfpathmoveto{\pgfqpoint{7.704610in}{0.387222in}}%
\pgfpathlineto{\pgfqpoint{7.704610in}{2.711667in}}%
\pgfusepath{stroke}%
\end{pgfscope}%
\begin{pgfscope}%
\pgfsetrectcap%
\pgfsetmiterjoin%
\pgfsetlinewidth{0.803000pt}%
\definecolor{currentstroke}{rgb}{0.000000,0.000000,0.000000}%
\pgfsetstrokecolor{currentstroke}%
\pgfsetdash{}{0pt}%
\pgfpathmoveto{\pgfqpoint{4.382652in}{0.387222in}}%
\pgfpathlineto{\pgfqpoint{7.704610in}{0.387222in}}%
\pgfusepath{stroke}%
\end{pgfscope}%
\begin{pgfscope}%
\pgfsetrectcap%
\pgfsetmiterjoin%
\pgfsetlinewidth{0.803000pt}%
\definecolor{currentstroke}{rgb}{0.000000,0.000000,0.000000}%
\pgfsetstrokecolor{currentstroke}%
\pgfsetdash{}{0pt}%
\pgfpathmoveto{\pgfqpoint{4.382652in}{2.711667in}}%
\pgfpathlineto{\pgfqpoint{7.704610in}{2.711667in}}%
\pgfusepath{stroke}%
\end{pgfscope}%
\begin{pgfscope}%
\definecolor{textcolor}{rgb}{0.000000,0.000000,0.000000}%
\pgfsetstrokecolor{textcolor}%
\pgfsetfillcolor{textcolor}%
\pgftext[x=6.043631in,y=2.795000in,,base]{\color{textcolor}\sffamily\fontsize{12.000000}{14.400000}\selectfont \(\displaystyle  N = 128 \)}%
\end{pgfscope}%
\end{pgfpicture}%
\makeatother%
\endgroup%
}
\caption{Figure of the three-dimensional cubature of $ D = 1 $}
\label{Fig:M3D1}
\end{figure}

In this case, the smallest $N$ is $1$.

\subsubsection{The case $ D = 2 $}

We optimize for $ D = 2 $, $ N = 3 $ with $ \text{FPG} \rbr{ 10^6, 10^{-2} } $ and get the residual $ F ( \mathbf{x}_{\cdot}^{\rbr{K}}, u_{\cdot}^{\rbr{K}} ) = \text{\input{Text3.txt}} $. This implies $ N \le 3 $ is impossible.

We turn to try $ N = 4 $. In this case, since $ \abs{\mathcal{I}_2} = 10 $, we may group with $ C = 1 $ and $ \sbr{ 1, 1, 1, 1 } $. We have $ 3 N + C = 13 $. With some degree of freedom not fixed, We use $ \text{FPG} \rbr{ 10^5, 10^{-2} } $ and get the following results in Figure \ref{Fig:M3D2Ran}.

\begin{figure}[htbp]
\centering
\scalebox{0.75}{%% Creator: Matplotlib, PGF backend
%%
%% To include the figure in your LaTeX document, write
%%   \input{<filename>.pgf}
%%
%% Make sure the required packages are loaded in your preamble
%%   \usepackage{pgf}
%%
%% Figures using additional raster images can only be included by \input if
%% they are in the same directory as the main LaTeX file. For loading figures
%% from other directories you can use the `import` package
%%   \usepackage{import}
%% and then include the figures with
%%   \import{<path to file>}{<filename>.pgf}
%%
%% Matplotlib used the following preamble
%%   \usepackage{fontspec}
%%   \setmainfont{DejaVuSerif.ttf}[Path=/home/lzh/anaconda3/envs/numana/lib/python3.7/site-packages/matplotlib/mpl-data/fonts/ttf/]
%%   \setsansfont{DejaVuSans.ttf}[Path=/home/lzh/anaconda3/envs/numana/lib/python3.7/site-packages/matplotlib/mpl-data/fonts/ttf/]
%%   \setmonofont{DejaVuSansMono.ttf}[Path=/home/lzh/anaconda3/envs/numana/lib/python3.7/site-packages/matplotlib/mpl-data/fonts/ttf/]
%%
\begingroup%
\makeatletter%
\begin{pgfpicture}%
\pgfpathrectangle{\pgfpointorigin}{\pgfqpoint{8.000000in}{6.000000in}}%
\pgfusepath{use as bounding box, clip}%
\begin{pgfscope}%
\pgfsetbuttcap%
\pgfsetmiterjoin%
\definecolor{currentfill}{rgb}{1.000000,1.000000,1.000000}%
\pgfsetfillcolor{currentfill}%
\pgfsetlinewidth{0.000000pt}%
\definecolor{currentstroke}{rgb}{1.000000,1.000000,1.000000}%
\pgfsetstrokecolor{currentstroke}%
\pgfsetdash{}{0pt}%
\pgfpathmoveto{\pgfqpoint{0.000000in}{0.000000in}}%
\pgfpathlineto{\pgfqpoint{8.000000in}{0.000000in}}%
\pgfpathlineto{\pgfqpoint{8.000000in}{6.000000in}}%
\pgfpathlineto{\pgfqpoint{0.000000in}{6.000000in}}%
\pgfpathclose%
\pgfusepath{fill}%
\end{pgfscope}%
\begin{pgfscope}%
\pgfsetbuttcap%
\pgfsetmiterjoin%
\definecolor{currentfill}{rgb}{1.000000,1.000000,1.000000}%
\pgfsetfillcolor{currentfill}%
\pgfsetlinewidth{0.000000pt}%
\definecolor{currentstroke}{rgb}{0.000000,0.000000,0.000000}%
\pgfsetstrokecolor{currentstroke}%
\pgfsetstrokeopacity{0.000000}%
\pgfsetdash{}{0pt}%
\pgfpathmoveto{\pgfqpoint{0.556250in}{3.311111in}}%
\pgfpathlineto{\pgfqpoint{3.873503in}{3.311111in}}%
\pgfpathlineto{\pgfqpoint{3.873503in}{5.627778in}}%
\pgfpathlineto{\pgfqpoint{0.556250in}{5.627778in}}%
\pgfpathclose%
\pgfusepath{fill}%
\end{pgfscope}%
\begin{pgfscope}%
\pgfpathrectangle{\pgfqpoint{0.556250in}{3.311111in}}{\pgfqpoint{3.317253in}{2.316667in}}%
\pgfusepath{clip}%
\pgfsetbuttcap%
\pgfsetroundjoin%
\definecolor{currentfill}{rgb}{0.121569,0.466667,0.705882}%
\pgfsetfillcolor{currentfill}%
\pgfsetfillopacity{0.300000}%
\pgfsetlinewidth{1.003750pt}%
\definecolor{currentstroke}{rgb}{0.121569,0.466667,0.705882}%
\pgfsetstrokecolor{currentstroke}%
\pgfsetstrokeopacity{0.300000}%
\pgfsetdash{}{0pt}%
\pgfpathmoveto{\pgfqpoint{0.718506in}{5.522475in}}%
\pgfpathlineto{\pgfqpoint{0.718506in}{5.492691in}}%
\pgfpathlineto{\pgfqpoint{0.868143in}{5.403423in}}%
\pgfpathlineto{\pgfqpoint{1.017780in}{5.348725in}}%
\pgfpathlineto{\pgfqpoint{1.167417in}{5.253266in}}%
\pgfpathlineto{\pgfqpoint{1.317054in}{5.143501in}}%
\pgfpathlineto{\pgfqpoint{1.466691in}{5.059640in}}%
\pgfpathlineto{\pgfqpoint{1.616328in}{4.972655in}}%
\pgfpathlineto{\pgfqpoint{1.765965in}{4.828037in}}%
\pgfpathlineto{\pgfqpoint{1.915602in}{4.740345in}}%
\pgfpathlineto{\pgfqpoint{2.065239in}{4.636629in}}%
\pgfpathlineto{\pgfqpoint{2.214876in}{4.523417in}}%
\pgfpathlineto{\pgfqpoint{2.364513in}{4.408861in}}%
\pgfpathlineto{\pgfqpoint{2.514150in}{4.290550in}}%
\pgfpathlineto{\pgfqpoint{2.663787in}{4.201354in}}%
\pgfpathlineto{\pgfqpoint{2.813424in}{4.092039in}}%
\pgfpathlineto{\pgfqpoint{2.963061in}{3.958204in}}%
\pgfpathlineto{\pgfqpoint{3.112698in}{3.825138in}}%
\pgfpathlineto{\pgfqpoint{3.262335in}{3.751634in}}%
\pgfpathlineto{\pgfqpoint{3.411972in}{3.586191in}}%
\pgfpathlineto{\pgfqpoint{3.561609in}{3.500143in}}%
\pgfpathlineto{\pgfqpoint{3.711246in}{3.416414in}}%
\pgfpathlineto{\pgfqpoint{3.711246in}{3.450210in}}%
\pgfpathlineto{\pgfqpoint{3.711246in}{3.450210in}}%
\pgfpathlineto{\pgfqpoint{3.561609in}{3.546884in}}%
\pgfpathlineto{\pgfqpoint{3.411972in}{3.652156in}}%
\pgfpathlineto{\pgfqpoint{3.262335in}{3.766298in}}%
\pgfpathlineto{\pgfqpoint{3.112698in}{3.862288in}}%
\pgfpathlineto{\pgfqpoint{2.963061in}{3.980629in}}%
\pgfpathlineto{\pgfqpoint{2.813424in}{4.099713in}}%
\pgfpathlineto{\pgfqpoint{2.663787in}{4.220227in}}%
\pgfpathlineto{\pgfqpoint{2.514150in}{4.348655in}}%
\pgfpathlineto{\pgfqpoint{2.364513in}{4.468572in}}%
\pgfpathlineto{\pgfqpoint{2.214876in}{4.550644in}}%
\pgfpathlineto{\pgfqpoint{2.065239in}{4.678525in}}%
\pgfpathlineto{\pgfqpoint{1.915602in}{4.758260in}}%
\pgfpathlineto{\pgfqpoint{1.765965in}{4.894386in}}%
\pgfpathlineto{\pgfqpoint{1.616328in}{5.008158in}}%
\pgfpathlineto{\pgfqpoint{1.466691in}{5.101656in}}%
\pgfpathlineto{\pgfqpoint{1.317054in}{5.176442in}}%
\pgfpathlineto{\pgfqpoint{1.167417in}{5.291129in}}%
\pgfpathlineto{\pgfqpoint{1.017780in}{5.366562in}}%
\pgfpathlineto{\pgfqpoint{0.868143in}{5.459026in}}%
\pgfpathlineto{\pgfqpoint{0.718506in}{5.522475in}}%
\pgfpathclose%
\pgfusepath{stroke,fill}%
\end{pgfscope}%
\begin{pgfscope}%
\pgfpathrectangle{\pgfqpoint{0.556250in}{3.311111in}}{\pgfqpoint{3.317253in}{2.316667in}}%
\pgfusepath{clip}%
\pgfsetbuttcap%
\pgfsetroundjoin%
\definecolor{currentfill}{rgb}{0.121569,0.466667,0.705882}%
\pgfsetfillcolor{currentfill}%
\pgfsetlinewidth{1.003750pt}%
\definecolor{currentstroke}{rgb}{0.121569,0.466667,0.705882}%
\pgfsetstrokecolor{currentstroke}%
\pgfsetdash{}{0pt}%
\pgfsys@defobject{currentmarker}{\pgfqpoint{-0.009821in}{-0.009821in}}{\pgfqpoint{0.009821in}{0.009821in}}{%
\pgfpathmoveto{\pgfqpoint{0.000000in}{-0.009821in}}%
\pgfpathcurveto{\pgfqpoint{0.002605in}{-0.009821in}}{\pgfqpoint{0.005103in}{-0.008786in}}{\pgfqpoint{0.006944in}{-0.006944in}}%
\pgfpathcurveto{\pgfqpoint{0.008786in}{-0.005103in}}{\pgfqpoint{0.009821in}{-0.002605in}}{\pgfqpoint{0.009821in}{0.000000in}}%
\pgfpathcurveto{\pgfqpoint{0.009821in}{0.002605in}}{\pgfqpoint{0.008786in}{0.005103in}}{\pgfqpoint{0.006944in}{0.006944in}}%
\pgfpathcurveto{\pgfqpoint{0.005103in}{0.008786in}}{\pgfqpoint{0.002605in}{0.009821in}}{\pgfqpoint{0.000000in}{0.009821in}}%
\pgfpathcurveto{\pgfqpoint{-0.002605in}{0.009821in}}{\pgfqpoint{-0.005103in}{0.008786in}}{\pgfqpoint{-0.006944in}{0.006944in}}%
\pgfpathcurveto{\pgfqpoint{-0.008786in}{0.005103in}}{\pgfqpoint{-0.009821in}{0.002605in}}{\pgfqpoint{-0.009821in}{0.000000in}}%
\pgfpathcurveto{\pgfqpoint{-0.009821in}{-0.002605in}}{\pgfqpoint{-0.008786in}{-0.005103in}}{\pgfqpoint{-0.006944in}{-0.006944in}}%
\pgfpathcurveto{\pgfqpoint{-0.005103in}{-0.008786in}}{\pgfqpoint{-0.002605in}{-0.009821in}}{\pgfqpoint{0.000000in}{-0.009821in}}%
\pgfpathclose%
\pgfusepath{stroke,fill}%
}%
\begin{pgfscope}%
\pgfsys@transformshift{0.718506in}{5.507583in}%
\pgfsys@useobject{currentmarker}{}%
\end{pgfscope}%
\begin{pgfscope}%
\pgfsys@transformshift{0.868143in}{5.431224in}%
\pgfsys@useobject{currentmarker}{}%
\end{pgfscope}%
\begin{pgfscope}%
\pgfsys@transformshift{1.017780in}{5.357644in}%
\pgfsys@useobject{currentmarker}{}%
\end{pgfscope}%
\begin{pgfscope}%
\pgfsys@transformshift{1.167417in}{5.272197in}%
\pgfsys@useobject{currentmarker}{}%
\end{pgfscope}%
\begin{pgfscope}%
\pgfsys@transformshift{1.317054in}{5.159971in}%
\pgfsys@useobject{currentmarker}{}%
\end{pgfscope}%
\begin{pgfscope}%
\pgfsys@transformshift{1.466691in}{5.080648in}%
\pgfsys@useobject{currentmarker}{}%
\end{pgfscope}%
\begin{pgfscope}%
\pgfsys@transformshift{1.616328in}{4.990407in}%
\pgfsys@useobject{currentmarker}{}%
\end{pgfscope}%
\begin{pgfscope}%
\pgfsys@transformshift{1.765965in}{4.861212in}%
\pgfsys@useobject{currentmarker}{}%
\end{pgfscope}%
\begin{pgfscope}%
\pgfsys@transformshift{1.915602in}{4.749302in}%
\pgfsys@useobject{currentmarker}{}%
\end{pgfscope}%
\begin{pgfscope}%
\pgfsys@transformshift{2.065239in}{4.657577in}%
\pgfsys@useobject{currentmarker}{}%
\end{pgfscope}%
\begin{pgfscope}%
\pgfsys@transformshift{2.214876in}{4.537031in}%
\pgfsys@useobject{currentmarker}{}%
\end{pgfscope}%
\begin{pgfscope}%
\pgfsys@transformshift{2.364513in}{4.438716in}%
\pgfsys@useobject{currentmarker}{}%
\end{pgfscope}%
\begin{pgfscope}%
\pgfsys@transformshift{2.514150in}{4.319602in}%
\pgfsys@useobject{currentmarker}{}%
\end{pgfscope}%
\begin{pgfscope}%
\pgfsys@transformshift{2.663787in}{4.210791in}%
\pgfsys@useobject{currentmarker}{}%
\end{pgfscope}%
\begin{pgfscope}%
\pgfsys@transformshift{2.813424in}{4.095876in}%
\pgfsys@useobject{currentmarker}{}%
\end{pgfscope}%
\begin{pgfscope}%
\pgfsys@transformshift{2.963061in}{3.969416in}%
\pgfsys@useobject{currentmarker}{}%
\end{pgfscope}%
\begin{pgfscope}%
\pgfsys@transformshift{3.112698in}{3.843713in}%
\pgfsys@useobject{currentmarker}{}%
\end{pgfscope}%
\begin{pgfscope}%
\pgfsys@transformshift{3.262335in}{3.758966in}%
\pgfsys@useobject{currentmarker}{}%
\end{pgfscope}%
\begin{pgfscope}%
\pgfsys@transformshift{3.411972in}{3.619174in}%
\pgfsys@useobject{currentmarker}{}%
\end{pgfscope}%
\begin{pgfscope}%
\pgfsys@transformshift{3.561609in}{3.523514in}%
\pgfsys@useobject{currentmarker}{}%
\end{pgfscope}%
\begin{pgfscope}%
\pgfsys@transformshift{3.711246in}{3.433312in}%
\pgfsys@useobject{currentmarker}{}%
\end{pgfscope}%
\end{pgfscope}%
\begin{pgfscope}%
\pgfsetbuttcap%
\pgfsetroundjoin%
\definecolor{currentfill}{rgb}{0.000000,0.000000,0.000000}%
\pgfsetfillcolor{currentfill}%
\pgfsetlinewidth{0.803000pt}%
\definecolor{currentstroke}{rgb}{0.000000,0.000000,0.000000}%
\pgfsetstrokecolor{currentstroke}%
\pgfsetdash{}{0pt}%
\pgfsys@defobject{currentmarker}{\pgfqpoint{0.000000in}{-0.048611in}}{\pgfqpoint{0.000000in}{0.000000in}}{%
\pgfpathmoveto{\pgfqpoint{0.000000in}{0.000000in}}%
\pgfpathlineto{\pgfqpoint{0.000000in}{-0.048611in}}%
\pgfusepath{stroke,fill}%
}%
\begin{pgfscope}%
\pgfsys@transformshift{1.167417in}{3.311111in}%
\pgfsys@useobject{currentmarker}{}%
\end{pgfscope}%
\end{pgfscope}%
\begin{pgfscope}%
\definecolor{textcolor}{rgb}{0.000000,0.000000,0.000000}%
\pgfsetstrokecolor{textcolor}%
\pgfsetfillcolor{textcolor}%
\pgftext[x=1.167417in,y=3.213889in,,top]{\color{textcolor}\sffamily\fontsize{10.000000}{12.000000}\selectfont 2.25}%
\end{pgfscope}%
\begin{pgfscope}%
\pgfsetbuttcap%
\pgfsetroundjoin%
\definecolor{currentfill}{rgb}{0.000000,0.000000,0.000000}%
\pgfsetfillcolor{currentfill}%
\pgfsetlinewidth{0.803000pt}%
\definecolor{currentstroke}{rgb}{0.000000,0.000000,0.000000}%
\pgfsetstrokecolor{currentstroke}%
\pgfsetdash{}{0pt}%
\pgfsys@defobject{currentmarker}{\pgfqpoint{0.000000in}{-0.048611in}}{\pgfqpoint{0.000000in}{0.000000in}}{%
\pgfpathmoveto{\pgfqpoint{0.000000in}{0.000000in}}%
\pgfpathlineto{\pgfqpoint{0.000000in}{-0.048611in}}%
\pgfusepath{stroke,fill}%
}%
\begin{pgfscope}%
\pgfsys@transformshift{1.915602in}{3.311111in}%
\pgfsys@useobject{currentmarker}{}%
\end{pgfscope}%
\end{pgfscope}%
\begin{pgfscope}%
\definecolor{textcolor}{rgb}{0.000000,0.000000,0.000000}%
\pgfsetstrokecolor{textcolor}%
\pgfsetfillcolor{textcolor}%
\pgftext[x=1.915602in,y=3.213889in,,top]{\color{textcolor}\sffamily\fontsize{10.000000}{12.000000}\selectfont 2.30}%
\end{pgfscope}%
\begin{pgfscope}%
\pgfsetbuttcap%
\pgfsetroundjoin%
\definecolor{currentfill}{rgb}{0.000000,0.000000,0.000000}%
\pgfsetfillcolor{currentfill}%
\pgfsetlinewidth{0.803000pt}%
\definecolor{currentstroke}{rgb}{0.000000,0.000000,0.000000}%
\pgfsetstrokecolor{currentstroke}%
\pgfsetdash{}{0pt}%
\pgfsys@defobject{currentmarker}{\pgfqpoint{0.000000in}{-0.048611in}}{\pgfqpoint{0.000000in}{0.000000in}}{%
\pgfpathmoveto{\pgfqpoint{0.000000in}{0.000000in}}%
\pgfpathlineto{\pgfqpoint{0.000000in}{-0.048611in}}%
\pgfusepath{stroke,fill}%
}%
\begin{pgfscope}%
\pgfsys@transformshift{2.663787in}{3.311111in}%
\pgfsys@useobject{currentmarker}{}%
\end{pgfscope}%
\end{pgfscope}%
\begin{pgfscope}%
\definecolor{textcolor}{rgb}{0.000000,0.000000,0.000000}%
\pgfsetstrokecolor{textcolor}%
\pgfsetfillcolor{textcolor}%
\pgftext[x=2.663787in,y=3.213889in,,top]{\color{textcolor}\sffamily\fontsize{10.000000}{12.000000}\selectfont 2.35}%
\end{pgfscope}%
\begin{pgfscope}%
\pgfsetbuttcap%
\pgfsetroundjoin%
\definecolor{currentfill}{rgb}{0.000000,0.000000,0.000000}%
\pgfsetfillcolor{currentfill}%
\pgfsetlinewidth{0.803000pt}%
\definecolor{currentstroke}{rgb}{0.000000,0.000000,0.000000}%
\pgfsetstrokecolor{currentstroke}%
\pgfsetdash{}{0pt}%
\pgfsys@defobject{currentmarker}{\pgfqpoint{0.000000in}{-0.048611in}}{\pgfqpoint{0.000000in}{0.000000in}}{%
\pgfpathmoveto{\pgfqpoint{0.000000in}{0.000000in}}%
\pgfpathlineto{\pgfqpoint{0.000000in}{-0.048611in}}%
\pgfusepath{stroke,fill}%
}%
\begin{pgfscope}%
\pgfsys@transformshift{3.411972in}{3.311111in}%
\pgfsys@useobject{currentmarker}{}%
\end{pgfscope}%
\end{pgfscope}%
\begin{pgfscope}%
\definecolor{textcolor}{rgb}{0.000000,0.000000,0.000000}%
\pgfsetstrokecolor{textcolor}%
\pgfsetfillcolor{textcolor}%
\pgftext[x=3.411972in,y=3.213889in,,top]{\color{textcolor}\sffamily\fontsize{10.000000}{12.000000}\selectfont 2.40}%
\end{pgfscope}%
\begin{pgfscope}%
\pgfsetbuttcap%
\pgfsetroundjoin%
\definecolor{currentfill}{rgb}{0.000000,0.000000,0.000000}%
\pgfsetfillcolor{currentfill}%
\pgfsetlinewidth{0.803000pt}%
\definecolor{currentstroke}{rgb}{0.000000,0.000000,0.000000}%
\pgfsetstrokecolor{currentstroke}%
\pgfsetdash{}{0pt}%
\pgfsys@defobject{currentmarker}{\pgfqpoint{-0.048611in}{0.000000in}}{\pgfqpoint{0.000000in}{0.000000in}}{%
\pgfpathmoveto{\pgfqpoint{0.000000in}{0.000000in}}%
\pgfpathlineto{\pgfqpoint{-0.048611in}{0.000000in}}%
\pgfusepath{stroke,fill}%
}%
\begin{pgfscope}%
\pgfsys@transformshift{0.556250in}{3.513085in}%
\pgfsys@useobject{currentmarker}{}%
\end{pgfscope}%
\end{pgfscope}%
\begin{pgfscope}%
\definecolor{textcolor}{rgb}{0.000000,0.000000,0.000000}%
\pgfsetstrokecolor{textcolor}%
\pgfsetfillcolor{textcolor}%
\pgftext[x=0.149783in,y=3.460323in,left,base]{\color{textcolor}\sffamily\fontsize{10.000000}{12.000000}\selectfont 0.50}%
\end{pgfscope}%
\begin{pgfscope}%
\pgfsetbuttcap%
\pgfsetroundjoin%
\definecolor{currentfill}{rgb}{0.000000,0.000000,0.000000}%
\pgfsetfillcolor{currentfill}%
\pgfsetlinewidth{0.803000pt}%
\definecolor{currentstroke}{rgb}{0.000000,0.000000,0.000000}%
\pgfsetstrokecolor{currentstroke}%
\pgfsetdash{}{0pt}%
\pgfsys@defobject{currentmarker}{\pgfqpoint{-0.048611in}{0.000000in}}{\pgfqpoint{0.000000in}{0.000000in}}{%
\pgfpathmoveto{\pgfqpoint{0.000000in}{0.000000in}}%
\pgfpathlineto{\pgfqpoint{-0.048611in}{0.000000in}}%
\pgfusepath{stroke,fill}%
}%
\begin{pgfscope}%
\pgfsys@transformshift{0.556250in}{3.880333in}%
\pgfsys@useobject{currentmarker}{}%
\end{pgfscope}%
\end{pgfscope}%
\begin{pgfscope}%
\definecolor{textcolor}{rgb}{0.000000,0.000000,0.000000}%
\pgfsetstrokecolor{textcolor}%
\pgfsetfillcolor{textcolor}%
\pgftext[x=0.149783in,y=3.827572in,left,base]{\color{textcolor}\sffamily\fontsize{10.000000}{12.000000}\selectfont 0.55}%
\end{pgfscope}%
\begin{pgfscope}%
\pgfsetbuttcap%
\pgfsetroundjoin%
\definecolor{currentfill}{rgb}{0.000000,0.000000,0.000000}%
\pgfsetfillcolor{currentfill}%
\pgfsetlinewidth{0.803000pt}%
\definecolor{currentstroke}{rgb}{0.000000,0.000000,0.000000}%
\pgfsetstrokecolor{currentstroke}%
\pgfsetdash{}{0pt}%
\pgfsys@defobject{currentmarker}{\pgfqpoint{-0.048611in}{0.000000in}}{\pgfqpoint{0.000000in}{0.000000in}}{%
\pgfpathmoveto{\pgfqpoint{0.000000in}{0.000000in}}%
\pgfpathlineto{\pgfqpoint{-0.048611in}{0.000000in}}%
\pgfusepath{stroke,fill}%
}%
\begin{pgfscope}%
\pgfsys@transformshift{0.556250in}{4.247582in}%
\pgfsys@useobject{currentmarker}{}%
\end{pgfscope}%
\end{pgfscope}%
\begin{pgfscope}%
\definecolor{textcolor}{rgb}{0.000000,0.000000,0.000000}%
\pgfsetstrokecolor{textcolor}%
\pgfsetfillcolor{textcolor}%
\pgftext[x=0.149783in,y=4.194820in,left,base]{\color{textcolor}\sffamily\fontsize{10.000000}{12.000000}\selectfont 0.60}%
\end{pgfscope}%
\begin{pgfscope}%
\pgfsetbuttcap%
\pgfsetroundjoin%
\definecolor{currentfill}{rgb}{0.000000,0.000000,0.000000}%
\pgfsetfillcolor{currentfill}%
\pgfsetlinewidth{0.803000pt}%
\definecolor{currentstroke}{rgb}{0.000000,0.000000,0.000000}%
\pgfsetstrokecolor{currentstroke}%
\pgfsetdash{}{0pt}%
\pgfsys@defobject{currentmarker}{\pgfqpoint{-0.048611in}{0.000000in}}{\pgfqpoint{0.000000in}{0.000000in}}{%
\pgfpathmoveto{\pgfqpoint{0.000000in}{0.000000in}}%
\pgfpathlineto{\pgfqpoint{-0.048611in}{0.000000in}}%
\pgfusepath{stroke,fill}%
}%
\begin{pgfscope}%
\pgfsys@transformshift{0.556250in}{4.614830in}%
\pgfsys@useobject{currentmarker}{}%
\end{pgfscope}%
\end{pgfscope}%
\begin{pgfscope}%
\definecolor{textcolor}{rgb}{0.000000,0.000000,0.000000}%
\pgfsetstrokecolor{textcolor}%
\pgfsetfillcolor{textcolor}%
\pgftext[x=0.149783in,y=4.562068in,left,base]{\color{textcolor}\sffamily\fontsize{10.000000}{12.000000}\selectfont 0.65}%
\end{pgfscope}%
\begin{pgfscope}%
\pgfsetbuttcap%
\pgfsetroundjoin%
\definecolor{currentfill}{rgb}{0.000000,0.000000,0.000000}%
\pgfsetfillcolor{currentfill}%
\pgfsetlinewidth{0.803000pt}%
\definecolor{currentstroke}{rgb}{0.000000,0.000000,0.000000}%
\pgfsetstrokecolor{currentstroke}%
\pgfsetdash{}{0pt}%
\pgfsys@defobject{currentmarker}{\pgfqpoint{-0.048611in}{0.000000in}}{\pgfqpoint{0.000000in}{0.000000in}}{%
\pgfpathmoveto{\pgfqpoint{0.000000in}{0.000000in}}%
\pgfpathlineto{\pgfqpoint{-0.048611in}{0.000000in}}%
\pgfusepath{stroke,fill}%
}%
\begin{pgfscope}%
\pgfsys@transformshift{0.556250in}{4.982078in}%
\pgfsys@useobject{currentmarker}{}%
\end{pgfscope}%
\end{pgfscope}%
\begin{pgfscope}%
\definecolor{textcolor}{rgb}{0.000000,0.000000,0.000000}%
\pgfsetstrokecolor{textcolor}%
\pgfsetfillcolor{textcolor}%
\pgftext[x=0.149783in,y=4.929317in,left,base]{\color{textcolor}\sffamily\fontsize{10.000000}{12.000000}\selectfont 0.70}%
\end{pgfscope}%
\begin{pgfscope}%
\pgfsetbuttcap%
\pgfsetroundjoin%
\definecolor{currentfill}{rgb}{0.000000,0.000000,0.000000}%
\pgfsetfillcolor{currentfill}%
\pgfsetlinewidth{0.803000pt}%
\definecolor{currentstroke}{rgb}{0.000000,0.000000,0.000000}%
\pgfsetstrokecolor{currentstroke}%
\pgfsetdash{}{0pt}%
\pgfsys@defobject{currentmarker}{\pgfqpoint{-0.048611in}{0.000000in}}{\pgfqpoint{0.000000in}{0.000000in}}{%
\pgfpathmoveto{\pgfqpoint{0.000000in}{0.000000in}}%
\pgfpathlineto{\pgfqpoint{-0.048611in}{0.000000in}}%
\pgfusepath{stroke,fill}%
}%
\begin{pgfscope}%
\pgfsys@transformshift{0.556250in}{5.349327in}%
\pgfsys@useobject{currentmarker}{}%
\end{pgfscope}%
\end{pgfscope}%
\begin{pgfscope}%
\definecolor{textcolor}{rgb}{0.000000,0.000000,0.000000}%
\pgfsetstrokecolor{textcolor}%
\pgfsetfillcolor{textcolor}%
\pgftext[x=0.149783in,y=5.296565in,left,base]{\color{textcolor}\sffamily\fontsize{10.000000}{12.000000}\selectfont 0.75}%
\end{pgfscope}%
\begin{pgfscope}%
\pgfpathrectangle{\pgfqpoint{0.556250in}{3.311111in}}{\pgfqpoint{3.317253in}{2.316667in}}%
\pgfusepath{clip}%
\pgfsetrectcap%
\pgfsetroundjoin%
\pgfsetlinewidth{1.505625pt}%
\definecolor{currentstroke}{rgb}{0.121569,0.466667,0.705882}%
\pgfsetstrokecolor{currentstroke}%
\pgfsetdash{}{0pt}%
\pgfpathmoveto{\pgfqpoint{0.718506in}{5.507583in}}%
\pgfpathlineto{\pgfqpoint{0.868143in}{5.431224in}}%
\pgfpathlineto{\pgfqpoint{1.017780in}{5.357644in}}%
\pgfpathlineto{\pgfqpoint{1.167417in}{5.272197in}}%
\pgfpathlineto{\pgfqpoint{1.317054in}{5.159971in}}%
\pgfpathlineto{\pgfqpoint{1.466691in}{5.080648in}}%
\pgfpathlineto{\pgfqpoint{1.616328in}{4.990407in}}%
\pgfpathlineto{\pgfqpoint{1.765965in}{4.861212in}}%
\pgfpathlineto{\pgfqpoint{1.915602in}{4.749302in}}%
\pgfpathlineto{\pgfqpoint{2.065239in}{4.657577in}}%
\pgfpathlineto{\pgfqpoint{2.214876in}{4.537031in}}%
\pgfpathlineto{\pgfqpoint{2.364513in}{4.438716in}}%
\pgfpathlineto{\pgfqpoint{2.514150in}{4.319602in}}%
\pgfpathlineto{\pgfqpoint{2.663787in}{4.210791in}}%
\pgfpathlineto{\pgfqpoint{2.813424in}{4.095876in}}%
\pgfpathlineto{\pgfqpoint{2.963061in}{3.969416in}}%
\pgfpathlineto{\pgfqpoint{3.112698in}{3.843713in}}%
\pgfpathlineto{\pgfqpoint{3.262335in}{3.758966in}}%
\pgfpathlineto{\pgfqpoint{3.411972in}{3.619174in}}%
\pgfpathlineto{\pgfqpoint{3.561609in}{3.523514in}}%
\pgfpathlineto{\pgfqpoint{3.711246in}{3.433312in}}%
\pgfusepath{stroke}%
\end{pgfscope}%
\begin{pgfscope}%
\pgfsetrectcap%
\pgfsetmiterjoin%
\pgfsetlinewidth{0.803000pt}%
\definecolor{currentstroke}{rgb}{0.000000,0.000000,0.000000}%
\pgfsetstrokecolor{currentstroke}%
\pgfsetdash{}{0pt}%
\pgfpathmoveto{\pgfqpoint{0.556250in}{3.311111in}}%
\pgfpathlineto{\pgfqpoint{0.556250in}{5.627778in}}%
\pgfusepath{stroke}%
\end{pgfscope}%
\begin{pgfscope}%
\pgfsetrectcap%
\pgfsetmiterjoin%
\pgfsetlinewidth{0.803000pt}%
\definecolor{currentstroke}{rgb}{0.000000,0.000000,0.000000}%
\pgfsetstrokecolor{currentstroke}%
\pgfsetdash{}{0pt}%
\pgfpathmoveto{\pgfqpoint{3.873503in}{3.311111in}}%
\pgfpathlineto{\pgfqpoint{3.873503in}{5.627778in}}%
\pgfusepath{stroke}%
\end{pgfscope}%
\begin{pgfscope}%
\pgfsetrectcap%
\pgfsetmiterjoin%
\pgfsetlinewidth{0.803000pt}%
\definecolor{currentstroke}{rgb}{0.000000,0.000000,0.000000}%
\pgfsetstrokecolor{currentstroke}%
\pgfsetdash{}{0pt}%
\pgfpathmoveto{\pgfqpoint{0.556250in}{3.311111in}}%
\pgfpathlineto{\pgfqpoint{3.873503in}{3.311111in}}%
\pgfusepath{stroke}%
\end{pgfscope}%
\begin{pgfscope}%
\pgfsetrectcap%
\pgfsetmiterjoin%
\pgfsetlinewidth{0.803000pt}%
\definecolor{currentstroke}{rgb}{0.000000,0.000000,0.000000}%
\pgfsetstrokecolor{currentstroke}%
\pgfsetdash{}{0pt}%
\pgfpathmoveto{\pgfqpoint{0.556250in}{5.627778in}}%
\pgfpathlineto{\pgfqpoint{3.873503in}{5.627778in}}%
\pgfusepath{stroke}%
\end{pgfscope}%
\begin{pgfscope}%
\definecolor{textcolor}{rgb}{0.000000,0.000000,0.000000}%
\pgfsetstrokecolor{textcolor}%
\pgfsetfillcolor{textcolor}%
\pgftext[x=2.214876in,y=5.711111in,,base]{\color{textcolor}\sffamily\fontsize{12.000000}{14.400000}\selectfont \(\displaystyle  N = 16 \)}%
\end{pgfscope}%
\begin{pgfscope}%
\pgfsetbuttcap%
\pgfsetmiterjoin%
\definecolor{currentfill}{rgb}{1.000000,1.000000,1.000000}%
\pgfsetfillcolor{currentfill}%
\pgfsetlinewidth{0.000000pt}%
\definecolor{currentstroke}{rgb}{0.000000,0.000000,0.000000}%
\pgfsetstrokecolor{currentstroke}%
\pgfsetstrokeopacity{0.000000}%
\pgfsetdash{}{0pt}%
\pgfpathmoveto{\pgfqpoint{4.389822in}{3.311111in}}%
\pgfpathlineto{\pgfqpoint{7.707075in}{3.311111in}}%
\pgfpathlineto{\pgfqpoint{7.707075in}{5.627778in}}%
\pgfpathlineto{\pgfqpoint{4.389822in}{5.627778in}}%
\pgfpathclose%
\pgfusepath{fill}%
\end{pgfscope}%
\begin{pgfscope}%
\pgfpathrectangle{\pgfqpoint{4.389822in}{3.311111in}}{\pgfqpoint{3.317253in}{2.316667in}}%
\pgfusepath{clip}%
\pgfsetbuttcap%
\pgfsetroundjoin%
\definecolor{currentfill}{rgb}{0.121569,0.466667,0.705882}%
\pgfsetfillcolor{currentfill}%
\pgfsetfillopacity{0.300000}%
\pgfsetlinewidth{1.003750pt}%
\definecolor{currentstroke}{rgb}{0.121569,0.466667,0.705882}%
\pgfsetstrokecolor{currentstroke}%
\pgfsetstrokeopacity{0.300000}%
\pgfsetdash{}{0pt}%
\pgfpathmoveto{\pgfqpoint{4.552079in}{5.522475in}}%
\pgfpathlineto{\pgfqpoint{4.552079in}{5.490427in}}%
\pgfpathlineto{\pgfqpoint{4.701716in}{5.444833in}}%
\pgfpathlineto{\pgfqpoint{4.851353in}{5.389018in}}%
\pgfpathlineto{\pgfqpoint{5.000990in}{5.311977in}}%
\pgfpathlineto{\pgfqpoint{5.150627in}{5.248390in}}%
\pgfpathlineto{\pgfqpoint{5.300264in}{5.085432in}}%
\pgfpathlineto{\pgfqpoint{5.449901in}{5.086229in}}%
\pgfpathlineto{\pgfqpoint{5.599538in}{4.941912in}}%
\pgfpathlineto{\pgfqpoint{5.749175in}{4.808145in}}%
\pgfpathlineto{\pgfqpoint{5.898812in}{4.752568in}}%
\pgfpathlineto{\pgfqpoint{6.048449in}{4.559088in}}%
\pgfpathlineto{\pgfqpoint{6.198086in}{4.441934in}}%
\pgfpathlineto{\pgfqpoint{6.347723in}{4.231932in}}%
\pgfpathlineto{\pgfqpoint{6.497360in}{4.151023in}}%
\pgfpathlineto{\pgfqpoint{6.646997in}{4.039951in}}%
\pgfpathlineto{\pgfqpoint{6.796634in}{3.935868in}}%
\pgfpathlineto{\pgfqpoint{6.946271in}{3.856605in}}%
\pgfpathlineto{\pgfqpoint{7.095908in}{3.734837in}}%
\pgfpathlineto{\pgfqpoint{7.245545in}{3.571938in}}%
\pgfpathlineto{\pgfqpoint{7.395182in}{3.537684in}}%
\pgfpathlineto{\pgfqpoint{7.544819in}{3.416414in}}%
\pgfpathlineto{\pgfqpoint{7.544819in}{3.515785in}}%
\pgfpathlineto{\pgfqpoint{7.544819in}{3.515785in}}%
\pgfpathlineto{\pgfqpoint{7.395182in}{3.545033in}}%
\pgfpathlineto{\pgfqpoint{7.245545in}{3.646321in}}%
\pgfpathlineto{\pgfqpoint{7.095908in}{3.819667in}}%
\pgfpathlineto{\pgfqpoint{6.946271in}{3.896054in}}%
\pgfpathlineto{\pgfqpoint{6.796634in}{3.989115in}}%
\pgfpathlineto{\pgfqpoint{6.646997in}{4.120291in}}%
\pgfpathlineto{\pgfqpoint{6.497360in}{4.245756in}}%
\pgfpathlineto{\pgfqpoint{6.347723in}{4.399223in}}%
\pgfpathlineto{\pgfqpoint{6.198086in}{4.488013in}}%
\pgfpathlineto{\pgfqpoint{6.048449in}{4.665230in}}%
\pgfpathlineto{\pgfqpoint{5.898812in}{4.829352in}}%
\pgfpathlineto{\pgfqpoint{5.749175in}{4.933043in}}%
\pgfpathlineto{\pgfqpoint{5.599538in}{4.968119in}}%
\pgfpathlineto{\pgfqpoint{5.449901in}{5.104716in}}%
\pgfpathlineto{\pgfqpoint{5.300264in}{5.188514in}}%
\pgfpathlineto{\pgfqpoint{5.150627in}{5.295645in}}%
\pgfpathlineto{\pgfqpoint{5.000990in}{5.351936in}}%
\pgfpathlineto{\pgfqpoint{4.851353in}{5.423631in}}%
\pgfpathlineto{\pgfqpoint{4.701716in}{5.468875in}}%
\pgfpathlineto{\pgfqpoint{4.552079in}{5.522475in}}%
\pgfpathclose%
\pgfusepath{stroke,fill}%
\end{pgfscope}%
\begin{pgfscope}%
\pgfpathrectangle{\pgfqpoint{4.389822in}{3.311111in}}{\pgfqpoint{3.317253in}{2.316667in}}%
\pgfusepath{clip}%
\pgfsetbuttcap%
\pgfsetroundjoin%
\definecolor{currentfill}{rgb}{0.121569,0.466667,0.705882}%
\pgfsetfillcolor{currentfill}%
\pgfsetlinewidth{1.003750pt}%
\definecolor{currentstroke}{rgb}{0.121569,0.466667,0.705882}%
\pgfsetstrokecolor{currentstroke}%
\pgfsetdash{}{0pt}%
\pgfsys@defobject{currentmarker}{\pgfqpoint{-0.009821in}{-0.009821in}}{\pgfqpoint{0.009821in}{0.009821in}}{%
\pgfpathmoveto{\pgfqpoint{0.000000in}{-0.009821in}}%
\pgfpathcurveto{\pgfqpoint{0.002605in}{-0.009821in}}{\pgfqpoint{0.005103in}{-0.008786in}}{\pgfqpoint{0.006944in}{-0.006944in}}%
\pgfpathcurveto{\pgfqpoint{0.008786in}{-0.005103in}}{\pgfqpoint{0.009821in}{-0.002605in}}{\pgfqpoint{0.009821in}{0.000000in}}%
\pgfpathcurveto{\pgfqpoint{0.009821in}{0.002605in}}{\pgfqpoint{0.008786in}{0.005103in}}{\pgfqpoint{0.006944in}{0.006944in}}%
\pgfpathcurveto{\pgfqpoint{0.005103in}{0.008786in}}{\pgfqpoint{0.002605in}{0.009821in}}{\pgfqpoint{0.000000in}{0.009821in}}%
\pgfpathcurveto{\pgfqpoint{-0.002605in}{0.009821in}}{\pgfqpoint{-0.005103in}{0.008786in}}{\pgfqpoint{-0.006944in}{0.006944in}}%
\pgfpathcurveto{\pgfqpoint{-0.008786in}{0.005103in}}{\pgfqpoint{-0.009821in}{0.002605in}}{\pgfqpoint{-0.009821in}{0.000000in}}%
\pgfpathcurveto{\pgfqpoint{-0.009821in}{-0.002605in}}{\pgfqpoint{-0.008786in}{-0.005103in}}{\pgfqpoint{-0.006944in}{-0.006944in}}%
\pgfpathcurveto{\pgfqpoint{-0.005103in}{-0.008786in}}{\pgfqpoint{-0.002605in}{-0.009821in}}{\pgfqpoint{0.000000in}{-0.009821in}}%
\pgfpathclose%
\pgfusepath{stroke,fill}%
}%
\begin{pgfscope}%
\pgfsys@transformshift{4.552079in}{5.506451in}%
\pgfsys@useobject{currentmarker}{}%
\end{pgfscope}%
\begin{pgfscope}%
\pgfsys@transformshift{4.701716in}{5.456854in}%
\pgfsys@useobject{currentmarker}{}%
\end{pgfscope}%
\begin{pgfscope}%
\pgfsys@transformshift{4.851353in}{5.406325in}%
\pgfsys@useobject{currentmarker}{}%
\end{pgfscope}%
\begin{pgfscope}%
\pgfsys@transformshift{5.000990in}{5.331956in}%
\pgfsys@useobject{currentmarker}{}%
\end{pgfscope}%
\begin{pgfscope}%
\pgfsys@transformshift{5.150627in}{5.272018in}%
\pgfsys@useobject{currentmarker}{}%
\end{pgfscope}%
\begin{pgfscope}%
\pgfsys@transformshift{5.300264in}{5.136973in}%
\pgfsys@useobject{currentmarker}{}%
\end{pgfscope}%
\begin{pgfscope}%
\pgfsys@transformshift{5.449901in}{5.095473in}%
\pgfsys@useobject{currentmarker}{}%
\end{pgfscope}%
\begin{pgfscope}%
\pgfsys@transformshift{5.599538in}{4.955015in}%
\pgfsys@useobject{currentmarker}{}%
\end{pgfscope}%
\begin{pgfscope}%
\pgfsys@transformshift{5.749175in}{4.870594in}%
\pgfsys@useobject{currentmarker}{}%
\end{pgfscope}%
\begin{pgfscope}%
\pgfsys@transformshift{5.898812in}{4.790960in}%
\pgfsys@useobject{currentmarker}{}%
\end{pgfscope}%
\begin{pgfscope}%
\pgfsys@transformshift{6.048449in}{4.612159in}%
\pgfsys@useobject{currentmarker}{}%
\end{pgfscope}%
\begin{pgfscope}%
\pgfsys@transformshift{6.198086in}{4.464973in}%
\pgfsys@useobject{currentmarker}{}%
\end{pgfscope}%
\begin{pgfscope}%
\pgfsys@transformshift{6.347723in}{4.315578in}%
\pgfsys@useobject{currentmarker}{}%
\end{pgfscope}%
\begin{pgfscope}%
\pgfsys@transformshift{6.497360in}{4.198389in}%
\pgfsys@useobject{currentmarker}{}%
\end{pgfscope}%
\begin{pgfscope}%
\pgfsys@transformshift{6.646997in}{4.080121in}%
\pgfsys@useobject{currentmarker}{}%
\end{pgfscope}%
\begin{pgfscope}%
\pgfsys@transformshift{6.796634in}{3.962491in}%
\pgfsys@useobject{currentmarker}{}%
\end{pgfscope}%
\begin{pgfscope}%
\pgfsys@transformshift{6.946271in}{3.876330in}%
\pgfsys@useobject{currentmarker}{}%
\end{pgfscope}%
\begin{pgfscope}%
\pgfsys@transformshift{7.095908in}{3.777252in}%
\pgfsys@useobject{currentmarker}{}%
\end{pgfscope}%
\begin{pgfscope}%
\pgfsys@transformshift{7.245545in}{3.609129in}%
\pgfsys@useobject{currentmarker}{}%
\end{pgfscope}%
\begin{pgfscope}%
\pgfsys@transformshift{7.395182in}{3.541359in}%
\pgfsys@useobject{currentmarker}{}%
\end{pgfscope}%
\begin{pgfscope}%
\pgfsys@transformshift{7.544819in}{3.466100in}%
\pgfsys@useobject{currentmarker}{}%
\end{pgfscope}%
\end{pgfscope}%
\begin{pgfscope}%
\pgfsetbuttcap%
\pgfsetroundjoin%
\definecolor{currentfill}{rgb}{0.000000,0.000000,0.000000}%
\pgfsetfillcolor{currentfill}%
\pgfsetlinewidth{0.803000pt}%
\definecolor{currentstroke}{rgb}{0.000000,0.000000,0.000000}%
\pgfsetstrokecolor{currentstroke}%
\pgfsetdash{}{0pt}%
\pgfsys@defobject{currentmarker}{\pgfqpoint{0.000000in}{-0.048611in}}{\pgfqpoint{0.000000in}{0.000000in}}{%
\pgfpathmoveto{\pgfqpoint{0.000000in}{0.000000in}}%
\pgfpathlineto{\pgfqpoint{0.000000in}{-0.048611in}}%
\pgfusepath{stroke,fill}%
}%
\begin{pgfscope}%
\pgfsys@transformshift{4.701716in}{3.311111in}%
\pgfsys@useobject{currentmarker}{}%
\end{pgfscope}%
\end{pgfscope}%
\begin{pgfscope}%
\definecolor{textcolor}{rgb}{0.000000,0.000000,0.000000}%
\pgfsetstrokecolor{textcolor}%
\pgfsetfillcolor{textcolor}%
\pgftext[x=4.701716in,y=3.213889in,,top]{\color{textcolor}\sffamily\fontsize{10.000000}{12.000000}\selectfont 2.20}%
\end{pgfscope}%
\begin{pgfscope}%
\pgfsetbuttcap%
\pgfsetroundjoin%
\definecolor{currentfill}{rgb}{0.000000,0.000000,0.000000}%
\pgfsetfillcolor{currentfill}%
\pgfsetlinewidth{0.803000pt}%
\definecolor{currentstroke}{rgb}{0.000000,0.000000,0.000000}%
\pgfsetstrokecolor{currentstroke}%
\pgfsetdash{}{0pt}%
\pgfsys@defobject{currentmarker}{\pgfqpoint{0.000000in}{-0.048611in}}{\pgfqpoint{0.000000in}{0.000000in}}{%
\pgfpathmoveto{\pgfqpoint{0.000000in}{0.000000in}}%
\pgfpathlineto{\pgfqpoint{0.000000in}{-0.048611in}}%
\pgfusepath{stroke,fill}%
}%
\begin{pgfscope}%
\pgfsys@transformshift{5.449901in}{3.311111in}%
\pgfsys@useobject{currentmarker}{}%
\end{pgfscope}%
\end{pgfscope}%
\begin{pgfscope}%
\definecolor{textcolor}{rgb}{0.000000,0.000000,0.000000}%
\pgfsetstrokecolor{textcolor}%
\pgfsetfillcolor{textcolor}%
\pgftext[x=5.449901in,y=3.213889in,,top]{\color{textcolor}\sffamily\fontsize{10.000000}{12.000000}\selectfont 2.25}%
\end{pgfscope}%
\begin{pgfscope}%
\pgfsetbuttcap%
\pgfsetroundjoin%
\definecolor{currentfill}{rgb}{0.000000,0.000000,0.000000}%
\pgfsetfillcolor{currentfill}%
\pgfsetlinewidth{0.803000pt}%
\definecolor{currentstroke}{rgb}{0.000000,0.000000,0.000000}%
\pgfsetstrokecolor{currentstroke}%
\pgfsetdash{}{0pt}%
\pgfsys@defobject{currentmarker}{\pgfqpoint{0.000000in}{-0.048611in}}{\pgfqpoint{0.000000in}{0.000000in}}{%
\pgfpathmoveto{\pgfqpoint{0.000000in}{0.000000in}}%
\pgfpathlineto{\pgfqpoint{0.000000in}{-0.048611in}}%
\pgfusepath{stroke,fill}%
}%
\begin{pgfscope}%
\pgfsys@transformshift{6.198086in}{3.311111in}%
\pgfsys@useobject{currentmarker}{}%
\end{pgfscope}%
\end{pgfscope}%
\begin{pgfscope}%
\definecolor{textcolor}{rgb}{0.000000,0.000000,0.000000}%
\pgfsetstrokecolor{textcolor}%
\pgfsetfillcolor{textcolor}%
\pgftext[x=6.198086in,y=3.213889in,,top]{\color{textcolor}\sffamily\fontsize{10.000000}{12.000000}\selectfont 2.30}%
\end{pgfscope}%
\begin{pgfscope}%
\pgfsetbuttcap%
\pgfsetroundjoin%
\definecolor{currentfill}{rgb}{0.000000,0.000000,0.000000}%
\pgfsetfillcolor{currentfill}%
\pgfsetlinewidth{0.803000pt}%
\definecolor{currentstroke}{rgb}{0.000000,0.000000,0.000000}%
\pgfsetstrokecolor{currentstroke}%
\pgfsetdash{}{0pt}%
\pgfsys@defobject{currentmarker}{\pgfqpoint{0.000000in}{-0.048611in}}{\pgfqpoint{0.000000in}{0.000000in}}{%
\pgfpathmoveto{\pgfqpoint{0.000000in}{0.000000in}}%
\pgfpathlineto{\pgfqpoint{0.000000in}{-0.048611in}}%
\pgfusepath{stroke,fill}%
}%
\begin{pgfscope}%
\pgfsys@transformshift{6.946271in}{3.311111in}%
\pgfsys@useobject{currentmarker}{}%
\end{pgfscope}%
\end{pgfscope}%
\begin{pgfscope}%
\definecolor{textcolor}{rgb}{0.000000,0.000000,0.000000}%
\pgfsetstrokecolor{textcolor}%
\pgfsetfillcolor{textcolor}%
\pgftext[x=6.946271in,y=3.213889in,,top]{\color{textcolor}\sffamily\fontsize{10.000000}{12.000000}\selectfont 2.35}%
\end{pgfscope}%
\begin{pgfscope}%
\pgfsetbuttcap%
\pgfsetroundjoin%
\definecolor{currentfill}{rgb}{0.000000,0.000000,0.000000}%
\pgfsetfillcolor{currentfill}%
\pgfsetlinewidth{0.803000pt}%
\definecolor{currentstroke}{rgb}{0.000000,0.000000,0.000000}%
\pgfsetstrokecolor{currentstroke}%
\pgfsetdash{}{0pt}%
\pgfsys@defobject{currentmarker}{\pgfqpoint{0.000000in}{-0.048611in}}{\pgfqpoint{0.000000in}{0.000000in}}{%
\pgfpathmoveto{\pgfqpoint{0.000000in}{0.000000in}}%
\pgfpathlineto{\pgfqpoint{0.000000in}{-0.048611in}}%
\pgfusepath{stroke,fill}%
}%
\begin{pgfscope}%
\pgfsys@transformshift{7.694456in}{3.311111in}%
\pgfsys@useobject{currentmarker}{}%
\end{pgfscope}%
\end{pgfscope}%
\begin{pgfscope}%
\definecolor{textcolor}{rgb}{0.000000,0.000000,0.000000}%
\pgfsetstrokecolor{textcolor}%
\pgfsetfillcolor{textcolor}%
\pgftext[x=7.694456in,y=3.213889in,,top]{\color{textcolor}\sffamily\fontsize{10.000000}{12.000000}\selectfont 2.40}%
\end{pgfscope}%
\begin{pgfscope}%
\pgfsetbuttcap%
\pgfsetroundjoin%
\definecolor{currentfill}{rgb}{0.000000,0.000000,0.000000}%
\pgfsetfillcolor{currentfill}%
\pgfsetlinewidth{0.803000pt}%
\definecolor{currentstroke}{rgb}{0.000000,0.000000,0.000000}%
\pgfsetstrokecolor{currentstroke}%
\pgfsetdash{}{0pt}%
\pgfsys@defobject{currentmarker}{\pgfqpoint{-0.048611in}{0.000000in}}{\pgfqpoint{0.000000in}{0.000000in}}{%
\pgfpathmoveto{\pgfqpoint{0.000000in}{0.000000in}}%
\pgfpathlineto{\pgfqpoint{-0.048611in}{0.000000in}}%
\pgfusepath{stroke,fill}%
}%
\begin{pgfscope}%
\pgfsys@transformshift{4.389822in}{3.325492in}%
\pgfsys@useobject{currentmarker}{}%
\end{pgfscope}%
\end{pgfscope}%
\begin{pgfscope}%
\definecolor{textcolor}{rgb}{0.000000,0.000000,0.000000}%
\pgfsetstrokecolor{textcolor}%
\pgfsetfillcolor{textcolor}%
\pgftext[x=4.071721in,y=3.272730in,left,base]{\color{textcolor}\sffamily\fontsize{10.000000}{12.000000}\selectfont 0.3}%
\end{pgfscope}%
\begin{pgfscope}%
\pgfsetbuttcap%
\pgfsetroundjoin%
\definecolor{currentfill}{rgb}{0.000000,0.000000,0.000000}%
\pgfsetfillcolor{currentfill}%
\pgfsetlinewidth{0.803000pt}%
\definecolor{currentstroke}{rgb}{0.000000,0.000000,0.000000}%
\pgfsetstrokecolor{currentstroke}%
\pgfsetdash{}{0pt}%
\pgfsys@defobject{currentmarker}{\pgfqpoint{-0.048611in}{0.000000in}}{\pgfqpoint{0.000000in}{0.000000in}}{%
\pgfpathmoveto{\pgfqpoint{0.000000in}{0.000000in}}%
\pgfpathlineto{\pgfqpoint{-0.048611in}{0.000000in}}%
\pgfusepath{stroke,fill}%
}%
\begin{pgfscope}%
\pgfsys@transformshift{4.389822in}{3.763430in}%
\pgfsys@useobject{currentmarker}{}%
\end{pgfscope}%
\end{pgfscope}%
\begin{pgfscope}%
\definecolor{textcolor}{rgb}{0.000000,0.000000,0.000000}%
\pgfsetstrokecolor{textcolor}%
\pgfsetfillcolor{textcolor}%
\pgftext[x=4.071721in,y=3.710668in,left,base]{\color{textcolor}\sffamily\fontsize{10.000000}{12.000000}\selectfont 0.4}%
\end{pgfscope}%
\begin{pgfscope}%
\pgfsetbuttcap%
\pgfsetroundjoin%
\definecolor{currentfill}{rgb}{0.000000,0.000000,0.000000}%
\pgfsetfillcolor{currentfill}%
\pgfsetlinewidth{0.803000pt}%
\definecolor{currentstroke}{rgb}{0.000000,0.000000,0.000000}%
\pgfsetstrokecolor{currentstroke}%
\pgfsetdash{}{0pt}%
\pgfsys@defobject{currentmarker}{\pgfqpoint{-0.048611in}{0.000000in}}{\pgfqpoint{0.000000in}{0.000000in}}{%
\pgfpathmoveto{\pgfqpoint{0.000000in}{0.000000in}}%
\pgfpathlineto{\pgfqpoint{-0.048611in}{0.000000in}}%
\pgfusepath{stroke,fill}%
}%
\begin{pgfscope}%
\pgfsys@transformshift{4.389822in}{4.201368in}%
\pgfsys@useobject{currentmarker}{}%
\end{pgfscope}%
\end{pgfscope}%
\begin{pgfscope}%
\definecolor{textcolor}{rgb}{0.000000,0.000000,0.000000}%
\pgfsetstrokecolor{textcolor}%
\pgfsetfillcolor{textcolor}%
\pgftext[x=4.071721in,y=4.148606in,left,base]{\color{textcolor}\sffamily\fontsize{10.000000}{12.000000}\selectfont 0.5}%
\end{pgfscope}%
\begin{pgfscope}%
\pgfsetbuttcap%
\pgfsetroundjoin%
\definecolor{currentfill}{rgb}{0.000000,0.000000,0.000000}%
\pgfsetfillcolor{currentfill}%
\pgfsetlinewidth{0.803000pt}%
\definecolor{currentstroke}{rgb}{0.000000,0.000000,0.000000}%
\pgfsetstrokecolor{currentstroke}%
\pgfsetdash{}{0pt}%
\pgfsys@defobject{currentmarker}{\pgfqpoint{-0.048611in}{0.000000in}}{\pgfqpoint{0.000000in}{0.000000in}}{%
\pgfpathmoveto{\pgfqpoint{0.000000in}{0.000000in}}%
\pgfpathlineto{\pgfqpoint{-0.048611in}{0.000000in}}%
\pgfusepath{stroke,fill}%
}%
\begin{pgfscope}%
\pgfsys@transformshift{4.389822in}{4.639306in}%
\pgfsys@useobject{currentmarker}{}%
\end{pgfscope}%
\end{pgfscope}%
\begin{pgfscope}%
\definecolor{textcolor}{rgb}{0.000000,0.000000,0.000000}%
\pgfsetstrokecolor{textcolor}%
\pgfsetfillcolor{textcolor}%
\pgftext[x=4.071721in,y=4.586544in,left,base]{\color{textcolor}\sffamily\fontsize{10.000000}{12.000000}\selectfont 0.6}%
\end{pgfscope}%
\begin{pgfscope}%
\pgfsetbuttcap%
\pgfsetroundjoin%
\definecolor{currentfill}{rgb}{0.000000,0.000000,0.000000}%
\pgfsetfillcolor{currentfill}%
\pgfsetlinewidth{0.803000pt}%
\definecolor{currentstroke}{rgb}{0.000000,0.000000,0.000000}%
\pgfsetstrokecolor{currentstroke}%
\pgfsetdash{}{0pt}%
\pgfsys@defobject{currentmarker}{\pgfqpoint{-0.048611in}{0.000000in}}{\pgfqpoint{0.000000in}{0.000000in}}{%
\pgfpathmoveto{\pgfqpoint{0.000000in}{0.000000in}}%
\pgfpathlineto{\pgfqpoint{-0.048611in}{0.000000in}}%
\pgfusepath{stroke,fill}%
}%
\begin{pgfscope}%
\pgfsys@transformshift{4.389822in}{5.077244in}%
\pgfsys@useobject{currentmarker}{}%
\end{pgfscope}%
\end{pgfscope}%
\begin{pgfscope}%
\definecolor{textcolor}{rgb}{0.000000,0.000000,0.000000}%
\pgfsetstrokecolor{textcolor}%
\pgfsetfillcolor{textcolor}%
\pgftext[x=4.071721in,y=5.024482in,left,base]{\color{textcolor}\sffamily\fontsize{10.000000}{12.000000}\selectfont 0.7}%
\end{pgfscope}%
\begin{pgfscope}%
\pgfsetbuttcap%
\pgfsetroundjoin%
\definecolor{currentfill}{rgb}{0.000000,0.000000,0.000000}%
\pgfsetfillcolor{currentfill}%
\pgfsetlinewidth{0.803000pt}%
\definecolor{currentstroke}{rgb}{0.000000,0.000000,0.000000}%
\pgfsetstrokecolor{currentstroke}%
\pgfsetdash{}{0pt}%
\pgfsys@defobject{currentmarker}{\pgfqpoint{-0.048611in}{0.000000in}}{\pgfqpoint{0.000000in}{0.000000in}}{%
\pgfpathmoveto{\pgfqpoint{0.000000in}{0.000000in}}%
\pgfpathlineto{\pgfqpoint{-0.048611in}{0.000000in}}%
\pgfusepath{stroke,fill}%
}%
\begin{pgfscope}%
\pgfsys@transformshift{4.389822in}{5.515182in}%
\pgfsys@useobject{currentmarker}{}%
\end{pgfscope}%
\end{pgfscope}%
\begin{pgfscope}%
\definecolor{textcolor}{rgb}{0.000000,0.000000,0.000000}%
\pgfsetstrokecolor{textcolor}%
\pgfsetfillcolor{textcolor}%
\pgftext[x=4.071721in,y=5.462420in,left,base]{\color{textcolor}\sffamily\fontsize{10.000000}{12.000000}\selectfont 0.8}%
\end{pgfscope}%
\begin{pgfscope}%
\pgfpathrectangle{\pgfqpoint{4.389822in}{3.311111in}}{\pgfqpoint{3.317253in}{2.316667in}}%
\pgfusepath{clip}%
\pgfsetrectcap%
\pgfsetroundjoin%
\pgfsetlinewidth{1.505625pt}%
\definecolor{currentstroke}{rgb}{0.121569,0.466667,0.705882}%
\pgfsetstrokecolor{currentstroke}%
\pgfsetdash{}{0pt}%
\pgfpathmoveto{\pgfqpoint{4.552079in}{5.506451in}}%
\pgfpathlineto{\pgfqpoint{4.701716in}{5.456854in}}%
\pgfpathlineto{\pgfqpoint{4.851353in}{5.406325in}}%
\pgfpathlineto{\pgfqpoint{5.000990in}{5.331956in}}%
\pgfpathlineto{\pgfqpoint{5.150627in}{5.272018in}}%
\pgfpathlineto{\pgfqpoint{5.300264in}{5.136973in}}%
\pgfpathlineto{\pgfqpoint{5.449901in}{5.095473in}}%
\pgfpathlineto{\pgfqpoint{5.599538in}{4.955015in}}%
\pgfpathlineto{\pgfqpoint{5.749175in}{4.870594in}}%
\pgfpathlineto{\pgfqpoint{5.898812in}{4.790960in}}%
\pgfpathlineto{\pgfqpoint{6.048449in}{4.612159in}}%
\pgfpathlineto{\pgfqpoint{6.198086in}{4.464973in}}%
\pgfpathlineto{\pgfqpoint{6.347723in}{4.315578in}}%
\pgfpathlineto{\pgfqpoint{6.497360in}{4.198389in}}%
\pgfpathlineto{\pgfqpoint{6.646997in}{4.080121in}}%
\pgfpathlineto{\pgfqpoint{6.796634in}{3.962491in}}%
\pgfpathlineto{\pgfqpoint{6.946271in}{3.876330in}}%
\pgfpathlineto{\pgfqpoint{7.095908in}{3.777252in}}%
\pgfpathlineto{\pgfqpoint{7.245545in}{3.609129in}}%
\pgfpathlineto{\pgfqpoint{7.395182in}{3.541359in}}%
\pgfpathlineto{\pgfqpoint{7.544819in}{3.466100in}}%
\pgfusepath{stroke}%
\end{pgfscope}%
\begin{pgfscope}%
\pgfsetrectcap%
\pgfsetmiterjoin%
\pgfsetlinewidth{0.803000pt}%
\definecolor{currentstroke}{rgb}{0.000000,0.000000,0.000000}%
\pgfsetstrokecolor{currentstroke}%
\pgfsetdash{}{0pt}%
\pgfpathmoveto{\pgfqpoint{4.389822in}{3.311111in}}%
\pgfpathlineto{\pgfqpoint{4.389822in}{5.627778in}}%
\pgfusepath{stroke}%
\end{pgfscope}%
\begin{pgfscope}%
\pgfsetrectcap%
\pgfsetmiterjoin%
\pgfsetlinewidth{0.803000pt}%
\definecolor{currentstroke}{rgb}{0.000000,0.000000,0.000000}%
\pgfsetstrokecolor{currentstroke}%
\pgfsetdash{}{0pt}%
\pgfpathmoveto{\pgfqpoint{7.707075in}{3.311111in}}%
\pgfpathlineto{\pgfqpoint{7.707075in}{5.627778in}}%
\pgfusepath{stroke}%
\end{pgfscope}%
\begin{pgfscope}%
\pgfsetrectcap%
\pgfsetmiterjoin%
\pgfsetlinewidth{0.803000pt}%
\definecolor{currentstroke}{rgb}{0.000000,0.000000,0.000000}%
\pgfsetstrokecolor{currentstroke}%
\pgfsetdash{}{0pt}%
\pgfpathmoveto{\pgfqpoint{4.389822in}{3.311111in}}%
\pgfpathlineto{\pgfqpoint{7.707075in}{3.311111in}}%
\pgfusepath{stroke}%
\end{pgfscope}%
\begin{pgfscope}%
\pgfsetrectcap%
\pgfsetmiterjoin%
\pgfsetlinewidth{0.803000pt}%
\definecolor{currentstroke}{rgb}{0.000000,0.000000,0.000000}%
\pgfsetstrokecolor{currentstroke}%
\pgfsetdash{}{0pt}%
\pgfpathmoveto{\pgfqpoint{4.389822in}{5.627778in}}%
\pgfpathlineto{\pgfqpoint{7.707075in}{5.627778in}}%
\pgfusepath{stroke}%
\end{pgfscope}%
\begin{pgfscope}%
\definecolor{textcolor}{rgb}{0.000000,0.000000,0.000000}%
\pgfsetstrokecolor{textcolor}%
\pgfsetfillcolor{textcolor}%
\pgftext[x=6.048449in,y=5.711111in,,base]{\color{textcolor}\sffamily\fontsize{12.000000}{14.400000}\selectfont \(\displaystyle  N = 32 \)}%
\end{pgfscope}%
\begin{pgfscope}%
\pgfsetbuttcap%
\pgfsetmiterjoin%
\definecolor{currentfill}{rgb}{1.000000,1.000000,1.000000}%
\pgfsetfillcolor{currentfill}%
\pgfsetlinewidth{0.000000pt}%
\definecolor{currentstroke}{rgb}{0.000000,0.000000,0.000000}%
\pgfsetstrokecolor{currentstroke}%
\pgfsetstrokeopacity{0.000000}%
\pgfsetdash{}{0pt}%
\pgfpathmoveto{\pgfqpoint{0.556250in}{0.386111in}}%
\pgfpathlineto{\pgfqpoint{3.873503in}{0.386111in}}%
\pgfpathlineto{\pgfqpoint{3.873503in}{2.702778in}}%
\pgfpathlineto{\pgfqpoint{0.556250in}{2.702778in}}%
\pgfpathclose%
\pgfusepath{fill}%
\end{pgfscope}%
\begin{pgfscope}%
\pgfpathrectangle{\pgfqpoint{0.556250in}{0.386111in}}{\pgfqpoint{3.317253in}{2.316667in}}%
\pgfusepath{clip}%
\pgfsetbuttcap%
\pgfsetroundjoin%
\definecolor{currentfill}{rgb}{0.121569,0.466667,0.705882}%
\pgfsetfillcolor{currentfill}%
\pgfsetfillopacity{0.300000}%
\pgfsetlinewidth{1.003750pt}%
\definecolor{currentstroke}{rgb}{0.121569,0.466667,0.705882}%
\pgfsetstrokecolor{currentstroke}%
\pgfsetstrokeopacity{0.300000}%
\pgfsetdash{}{0pt}%
\pgfpathmoveto{\pgfqpoint{0.718506in}{2.589992in}}%
\pgfpathlineto{\pgfqpoint{0.718506in}{2.582937in}}%
\pgfpathlineto{\pgfqpoint{0.868143in}{2.544358in}}%
\pgfpathlineto{\pgfqpoint{1.017780in}{2.464349in}}%
\pgfpathlineto{\pgfqpoint{1.167417in}{2.465005in}}%
\pgfpathlineto{\pgfqpoint{1.317054in}{2.278999in}}%
\pgfpathlineto{\pgfqpoint{1.466691in}{2.311385in}}%
\pgfpathlineto{\pgfqpoint{1.616328in}{2.233440in}}%
\pgfpathlineto{\pgfqpoint{1.765965in}{2.144849in}}%
\pgfpathlineto{\pgfqpoint{1.915602in}{1.733886in}}%
\pgfpathlineto{\pgfqpoint{2.065239in}{1.568290in}}%
\pgfpathlineto{\pgfqpoint{2.214876in}{1.538261in}}%
\pgfpathlineto{\pgfqpoint{2.364513in}{1.478158in}}%
\pgfpathlineto{\pgfqpoint{2.514150in}{1.282866in}}%
\pgfpathlineto{\pgfqpoint{2.663787in}{1.000320in}}%
\pgfpathlineto{\pgfqpoint{2.813424in}{0.701548in}}%
\pgfpathlineto{\pgfqpoint{2.963061in}{0.795951in}}%
\pgfpathlineto{\pgfqpoint{3.112698in}{0.719251in}}%
\pgfpathlineto{\pgfqpoint{3.262335in}{0.614662in}}%
\pgfpathlineto{\pgfqpoint{3.411972in}{0.615381in}}%
\pgfpathlineto{\pgfqpoint{3.561609in}{0.509676in}}%
\pgfpathlineto{\pgfqpoint{3.711246in}{0.491414in}}%
\pgfpathlineto{\pgfqpoint{3.711246in}{0.576900in}}%
\pgfpathlineto{\pgfqpoint{3.711246in}{0.576900in}}%
\pgfpathlineto{\pgfqpoint{3.561609in}{0.674658in}}%
\pgfpathlineto{\pgfqpoint{3.411972in}{0.732680in}}%
\pgfpathlineto{\pgfqpoint{3.262335in}{0.737920in}}%
\pgfpathlineto{\pgfqpoint{3.112698in}{0.898049in}}%
\pgfpathlineto{\pgfqpoint{2.963061in}{1.021988in}}%
\pgfpathlineto{\pgfqpoint{2.813424in}{1.186392in}}%
\pgfpathlineto{\pgfqpoint{2.663787in}{1.459785in}}%
\pgfpathlineto{\pgfqpoint{2.514150in}{1.550005in}}%
\pgfpathlineto{\pgfqpoint{2.364513in}{1.627692in}}%
\pgfpathlineto{\pgfqpoint{2.214876in}{1.936137in}}%
\pgfpathlineto{\pgfqpoint{2.065239in}{2.087808in}}%
\pgfpathlineto{\pgfqpoint{1.915602in}{2.125309in}}%
\pgfpathlineto{\pgfqpoint{1.765965in}{2.196310in}}%
\pgfpathlineto{\pgfqpoint{1.616328in}{2.312364in}}%
\pgfpathlineto{\pgfqpoint{1.466691in}{2.392623in}}%
\pgfpathlineto{\pgfqpoint{1.317054in}{2.432075in}}%
\pgfpathlineto{\pgfqpoint{1.167417in}{2.497859in}}%
\pgfpathlineto{\pgfqpoint{1.017780in}{2.547363in}}%
\pgfpathlineto{\pgfqpoint{0.868143in}{2.564001in}}%
\pgfpathlineto{\pgfqpoint{0.718506in}{2.589992in}}%
\pgfpathclose%
\pgfusepath{stroke,fill}%
\end{pgfscope}%
\begin{pgfscope}%
\pgfpathrectangle{\pgfqpoint{0.556250in}{0.386111in}}{\pgfqpoint{3.317253in}{2.316667in}}%
\pgfusepath{clip}%
\pgfsetbuttcap%
\pgfsetroundjoin%
\definecolor{currentfill}{rgb}{0.121569,0.466667,0.705882}%
\pgfsetfillcolor{currentfill}%
\pgfsetlinewidth{1.003750pt}%
\definecolor{currentstroke}{rgb}{0.121569,0.466667,0.705882}%
\pgfsetstrokecolor{currentstroke}%
\pgfsetdash{}{0pt}%
\pgfsys@defobject{currentmarker}{\pgfqpoint{-0.009821in}{-0.009821in}}{\pgfqpoint{0.009821in}{0.009821in}}{%
\pgfpathmoveto{\pgfqpoint{0.000000in}{-0.009821in}}%
\pgfpathcurveto{\pgfqpoint{0.002605in}{-0.009821in}}{\pgfqpoint{0.005103in}{-0.008786in}}{\pgfqpoint{0.006944in}{-0.006944in}}%
\pgfpathcurveto{\pgfqpoint{0.008786in}{-0.005103in}}{\pgfqpoint{0.009821in}{-0.002605in}}{\pgfqpoint{0.009821in}{0.000000in}}%
\pgfpathcurveto{\pgfqpoint{0.009821in}{0.002605in}}{\pgfqpoint{0.008786in}{0.005103in}}{\pgfqpoint{0.006944in}{0.006944in}}%
\pgfpathcurveto{\pgfqpoint{0.005103in}{0.008786in}}{\pgfqpoint{0.002605in}{0.009821in}}{\pgfqpoint{0.000000in}{0.009821in}}%
\pgfpathcurveto{\pgfqpoint{-0.002605in}{0.009821in}}{\pgfqpoint{-0.005103in}{0.008786in}}{\pgfqpoint{-0.006944in}{0.006944in}}%
\pgfpathcurveto{\pgfqpoint{-0.008786in}{0.005103in}}{\pgfqpoint{-0.009821in}{0.002605in}}{\pgfqpoint{-0.009821in}{0.000000in}}%
\pgfpathcurveto{\pgfqpoint{-0.009821in}{-0.002605in}}{\pgfqpoint{-0.008786in}{-0.005103in}}{\pgfqpoint{-0.006944in}{-0.006944in}}%
\pgfpathcurveto{\pgfqpoint{-0.005103in}{-0.008786in}}{\pgfqpoint{-0.002605in}{-0.009821in}}{\pgfqpoint{0.000000in}{-0.009821in}}%
\pgfpathclose%
\pgfusepath{stroke,fill}%
}%
\begin{pgfscope}%
\pgfsys@transformshift{0.718506in}{2.586465in}%
\pgfsys@useobject{currentmarker}{}%
\end{pgfscope}%
\begin{pgfscope}%
\pgfsys@transformshift{0.868143in}{2.554179in}%
\pgfsys@useobject{currentmarker}{}%
\end{pgfscope}%
\begin{pgfscope}%
\pgfsys@transformshift{1.017780in}{2.505856in}%
\pgfsys@useobject{currentmarker}{}%
\end{pgfscope}%
\begin{pgfscope}%
\pgfsys@transformshift{1.167417in}{2.481432in}%
\pgfsys@useobject{currentmarker}{}%
\end{pgfscope}%
\begin{pgfscope}%
\pgfsys@transformshift{1.317054in}{2.355537in}%
\pgfsys@useobject{currentmarker}{}%
\end{pgfscope}%
\begin{pgfscope}%
\pgfsys@transformshift{1.466691in}{2.352004in}%
\pgfsys@useobject{currentmarker}{}%
\end{pgfscope}%
\begin{pgfscope}%
\pgfsys@transformshift{1.616328in}{2.272902in}%
\pgfsys@useobject{currentmarker}{}%
\end{pgfscope}%
\begin{pgfscope}%
\pgfsys@transformshift{1.765965in}{2.170579in}%
\pgfsys@useobject{currentmarker}{}%
\end{pgfscope}%
\begin{pgfscope}%
\pgfsys@transformshift{1.915602in}{1.929597in}%
\pgfsys@useobject{currentmarker}{}%
\end{pgfscope}%
\begin{pgfscope}%
\pgfsys@transformshift{2.065239in}{1.828049in}%
\pgfsys@useobject{currentmarker}{}%
\end{pgfscope}%
\begin{pgfscope}%
\pgfsys@transformshift{2.214876in}{1.737199in}%
\pgfsys@useobject{currentmarker}{}%
\end{pgfscope}%
\begin{pgfscope}%
\pgfsys@transformshift{2.364513in}{1.552925in}%
\pgfsys@useobject{currentmarker}{}%
\end{pgfscope}%
\begin{pgfscope}%
\pgfsys@transformshift{2.514150in}{1.416435in}%
\pgfsys@useobject{currentmarker}{}%
\end{pgfscope}%
\begin{pgfscope}%
\pgfsys@transformshift{2.663787in}{1.230052in}%
\pgfsys@useobject{currentmarker}{}%
\end{pgfscope}%
\begin{pgfscope}%
\pgfsys@transformshift{2.813424in}{0.943970in}%
\pgfsys@useobject{currentmarker}{}%
\end{pgfscope}%
\begin{pgfscope}%
\pgfsys@transformshift{2.963061in}{0.908969in}%
\pgfsys@useobject{currentmarker}{}%
\end{pgfscope}%
\begin{pgfscope}%
\pgfsys@transformshift{3.112698in}{0.808650in}%
\pgfsys@useobject{currentmarker}{}%
\end{pgfscope}%
\begin{pgfscope}%
\pgfsys@transformshift{3.262335in}{0.676291in}%
\pgfsys@useobject{currentmarker}{}%
\end{pgfscope}%
\begin{pgfscope}%
\pgfsys@transformshift{3.411972in}{0.674030in}%
\pgfsys@useobject{currentmarker}{}%
\end{pgfscope}%
\begin{pgfscope}%
\pgfsys@transformshift{3.561609in}{0.592167in}%
\pgfsys@useobject{currentmarker}{}%
\end{pgfscope}%
\begin{pgfscope}%
\pgfsys@transformshift{3.711246in}{0.534157in}%
\pgfsys@useobject{currentmarker}{}%
\end{pgfscope}%
\end{pgfscope}%
\begin{pgfscope}%
\pgfsetbuttcap%
\pgfsetroundjoin%
\definecolor{currentfill}{rgb}{0.000000,0.000000,0.000000}%
\pgfsetfillcolor{currentfill}%
\pgfsetlinewidth{0.803000pt}%
\definecolor{currentstroke}{rgb}{0.000000,0.000000,0.000000}%
\pgfsetstrokecolor{currentstroke}%
\pgfsetdash{}{0pt}%
\pgfsys@defobject{currentmarker}{\pgfqpoint{0.000000in}{-0.048611in}}{\pgfqpoint{0.000000in}{0.000000in}}{%
\pgfpathmoveto{\pgfqpoint{0.000000in}{0.000000in}}%
\pgfpathlineto{\pgfqpoint{0.000000in}{-0.048611in}}%
\pgfusepath{stroke,fill}%
}%
\begin{pgfscope}%
\pgfsys@transformshift{1.017780in}{0.386111in}%
\pgfsys@useobject{currentmarker}{}%
\end{pgfscope}%
\end{pgfscope}%
\begin{pgfscope}%
\definecolor{textcolor}{rgb}{0.000000,0.000000,0.000000}%
\pgfsetstrokecolor{textcolor}%
\pgfsetfillcolor{textcolor}%
\pgftext[x=1.017780in,y=0.288889in,,top]{\color{textcolor}\sffamily\fontsize{10.000000}{12.000000}\selectfont 2.20}%
\end{pgfscope}%
\begin{pgfscope}%
\pgfsetbuttcap%
\pgfsetroundjoin%
\definecolor{currentfill}{rgb}{0.000000,0.000000,0.000000}%
\pgfsetfillcolor{currentfill}%
\pgfsetlinewidth{0.803000pt}%
\definecolor{currentstroke}{rgb}{0.000000,0.000000,0.000000}%
\pgfsetstrokecolor{currentstroke}%
\pgfsetdash{}{0pt}%
\pgfsys@defobject{currentmarker}{\pgfqpoint{0.000000in}{-0.048611in}}{\pgfqpoint{0.000000in}{0.000000in}}{%
\pgfpathmoveto{\pgfqpoint{0.000000in}{0.000000in}}%
\pgfpathlineto{\pgfqpoint{0.000000in}{-0.048611in}}%
\pgfusepath{stroke,fill}%
}%
\begin{pgfscope}%
\pgfsys@transformshift{1.765965in}{0.386111in}%
\pgfsys@useobject{currentmarker}{}%
\end{pgfscope}%
\end{pgfscope}%
\begin{pgfscope}%
\definecolor{textcolor}{rgb}{0.000000,0.000000,0.000000}%
\pgfsetstrokecolor{textcolor}%
\pgfsetfillcolor{textcolor}%
\pgftext[x=1.765965in,y=0.288889in,,top]{\color{textcolor}\sffamily\fontsize{10.000000}{12.000000}\selectfont 2.25}%
\end{pgfscope}%
\begin{pgfscope}%
\pgfsetbuttcap%
\pgfsetroundjoin%
\definecolor{currentfill}{rgb}{0.000000,0.000000,0.000000}%
\pgfsetfillcolor{currentfill}%
\pgfsetlinewidth{0.803000pt}%
\definecolor{currentstroke}{rgb}{0.000000,0.000000,0.000000}%
\pgfsetstrokecolor{currentstroke}%
\pgfsetdash{}{0pt}%
\pgfsys@defobject{currentmarker}{\pgfqpoint{0.000000in}{-0.048611in}}{\pgfqpoint{0.000000in}{0.000000in}}{%
\pgfpathmoveto{\pgfqpoint{0.000000in}{0.000000in}}%
\pgfpathlineto{\pgfqpoint{0.000000in}{-0.048611in}}%
\pgfusepath{stroke,fill}%
}%
\begin{pgfscope}%
\pgfsys@transformshift{2.514150in}{0.386111in}%
\pgfsys@useobject{currentmarker}{}%
\end{pgfscope}%
\end{pgfscope}%
\begin{pgfscope}%
\definecolor{textcolor}{rgb}{0.000000,0.000000,0.000000}%
\pgfsetstrokecolor{textcolor}%
\pgfsetfillcolor{textcolor}%
\pgftext[x=2.514150in,y=0.288889in,,top]{\color{textcolor}\sffamily\fontsize{10.000000}{12.000000}\selectfont 2.30}%
\end{pgfscope}%
\begin{pgfscope}%
\pgfsetbuttcap%
\pgfsetroundjoin%
\definecolor{currentfill}{rgb}{0.000000,0.000000,0.000000}%
\pgfsetfillcolor{currentfill}%
\pgfsetlinewidth{0.803000pt}%
\definecolor{currentstroke}{rgb}{0.000000,0.000000,0.000000}%
\pgfsetstrokecolor{currentstroke}%
\pgfsetdash{}{0pt}%
\pgfsys@defobject{currentmarker}{\pgfqpoint{0.000000in}{-0.048611in}}{\pgfqpoint{0.000000in}{0.000000in}}{%
\pgfpathmoveto{\pgfqpoint{0.000000in}{0.000000in}}%
\pgfpathlineto{\pgfqpoint{0.000000in}{-0.048611in}}%
\pgfusepath{stroke,fill}%
}%
\begin{pgfscope}%
\pgfsys@transformshift{3.262335in}{0.386111in}%
\pgfsys@useobject{currentmarker}{}%
\end{pgfscope}%
\end{pgfscope}%
\begin{pgfscope}%
\definecolor{textcolor}{rgb}{0.000000,0.000000,0.000000}%
\pgfsetstrokecolor{textcolor}%
\pgfsetfillcolor{textcolor}%
\pgftext[x=3.262335in,y=0.288889in,,top]{\color{textcolor}\sffamily\fontsize{10.000000}{12.000000}\selectfont 2.35}%
\end{pgfscope}%
\begin{pgfscope}%
\pgfsetbuttcap%
\pgfsetroundjoin%
\definecolor{currentfill}{rgb}{0.000000,0.000000,0.000000}%
\pgfsetfillcolor{currentfill}%
\pgfsetlinewidth{0.803000pt}%
\definecolor{currentstroke}{rgb}{0.000000,0.000000,0.000000}%
\pgfsetstrokecolor{currentstroke}%
\pgfsetdash{}{0pt}%
\pgfsys@defobject{currentmarker}{\pgfqpoint{-0.048611in}{0.000000in}}{\pgfqpoint{0.000000in}{0.000000in}}{%
\pgfpathmoveto{\pgfqpoint{0.000000in}{0.000000in}}%
\pgfpathlineto{\pgfqpoint{-0.048611in}{0.000000in}}%
\pgfusepath{stroke,fill}%
}%
\begin{pgfscope}%
\pgfsys@transformshift{0.556250in}{0.575025in}%
\pgfsys@useobject{currentmarker}{}%
\end{pgfscope}%
\end{pgfscope}%
\begin{pgfscope}%
\definecolor{textcolor}{rgb}{0.000000,0.000000,0.000000}%
\pgfsetstrokecolor{textcolor}%
\pgfsetfillcolor{textcolor}%
\pgftext[x=0.238148in,y=0.522263in,left,base]{\color{textcolor}\sffamily\fontsize{10.000000}{12.000000}\selectfont 0.2}%
\end{pgfscope}%
\begin{pgfscope}%
\pgfsetbuttcap%
\pgfsetroundjoin%
\definecolor{currentfill}{rgb}{0.000000,0.000000,0.000000}%
\pgfsetfillcolor{currentfill}%
\pgfsetlinewidth{0.803000pt}%
\definecolor{currentstroke}{rgb}{0.000000,0.000000,0.000000}%
\pgfsetstrokecolor{currentstroke}%
\pgfsetdash{}{0pt}%
\pgfsys@defobject{currentmarker}{\pgfqpoint{-0.048611in}{0.000000in}}{\pgfqpoint{0.000000in}{0.000000in}}{%
\pgfpathmoveto{\pgfqpoint{0.000000in}{0.000000in}}%
\pgfpathlineto{\pgfqpoint{-0.048611in}{0.000000in}}%
\pgfusepath{stroke,fill}%
}%
\begin{pgfscope}%
\pgfsys@transformshift{0.556250in}{0.906967in}%
\pgfsys@useobject{currentmarker}{}%
\end{pgfscope}%
\end{pgfscope}%
\begin{pgfscope}%
\definecolor{textcolor}{rgb}{0.000000,0.000000,0.000000}%
\pgfsetstrokecolor{textcolor}%
\pgfsetfillcolor{textcolor}%
\pgftext[x=0.238148in,y=0.854206in,left,base]{\color{textcolor}\sffamily\fontsize{10.000000}{12.000000}\selectfont 0.3}%
\end{pgfscope}%
\begin{pgfscope}%
\pgfsetbuttcap%
\pgfsetroundjoin%
\definecolor{currentfill}{rgb}{0.000000,0.000000,0.000000}%
\pgfsetfillcolor{currentfill}%
\pgfsetlinewidth{0.803000pt}%
\definecolor{currentstroke}{rgb}{0.000000,0.000000,0.000000}%
\pgfsetstrokecolor{currentstroke}%
\pgfsetdash{}{0pt}%
\pgfsys@defobject{currentmarker}{\pgfqpoint{-0.048611in}{0.000000in}}{\pgfqpoint{0.000000in}{0.000000in}}{%
\pgfpathmoveto{\pgfqpoint{0.000000in}{0.000000in}}%
\pgfpathlineto{\pgfqpoint{-0.048611in}{0.000000in}}%
\pgfusepath{stroke,fill}%
}%
\begin{pgfscope}%
\pgfsys@transformshift{0.556250in}{1.238910in}%
\pgfsys@useobject{currentmarker}{}%
\end{pgfscope}%
\end{pgfscope}%
\begin{pgfscope}%
\definecolor{textcolor}{rgb}{0.000000,0.000000,0.000000}%
\pgfsetstrokecolor{textcolor}%
\pgfsetfillcolor{textcolor}%
\pgftext[x=0.238148in,y=1.186149in,left,base]{\color{textcolor}\sffamily\fontsize{10.000000}{12.000000}\selectfont 0.4}%
\end{pgfscope}%
\begin{pgfscope}%
\pgfsetbuttcap%
\pgfsetroundjoin%
\definecolor{currentfill}{rgb}{0.000000,0.000000,0.000000}%
\pgfsetfillcolor{currentfill}%
\pgfsetlinewidth{0.803000pt}%
\definecolor{currentstroke}{rgb}{0.000000,0.000000,0.000000}%
\pgfsetstrokecolor{currentstroke}%
\pgfsetdash{}{0pt}%
\pgfsys@defobject{currentmarker}{\pgfqpoint{-0.048611in}{0.000000in}}{\pgfqpoint{0.000000in}{0.000000in}}{%
\pgfpathmoveto{\pgfqpoint{0.000000in}{0.000000in}}%
\pgfpathlineto{\pgfqpoint{-0.048611in}{0.000000in}}%
\pgfusepath{stroke,fill}%
}%
\begin{pgfscope}%
\pgfsys@transformshift{0.556250in}{1.570853in}%
\pgfsys@useobject{currentmarker}{}%
\end{pgfscope}%
\end{pgfscope}%
\begin{pgfscope}%
\definecolor{textcolor}{rgb}{0.000000,0.000000,0.000000}%
\pgfsetstrokecolor{textcolor}%
\pgfsetfillcolor{textcolor}%
\pgftext[x=0.238148in,y=1.518091in,left,base]{\color{textcolor}\sffamily\fontsize{10.000000}{12.000000}\selectfont 0.5}%
\end{pgfscope}%
\begin{pgfscope}%
\pgfsetbuttcap%
\pgfsetroundjoin%
\definecolor{currentfill}{rgb}{0.000000,0.000000,0.000000}%
\pgfsetfillcolor{currentfill}%
\pgfsetlinewidth{0.803000pt}%
\definecolor{currentstroke}{rgb}{0.000000,0.000000,0.000000}%
\pgfsetstrokecolor{currentstroke}%
\pgfsetdash{}{0pt}%
\pgfsys@defobject{currentmarker}{\pgfqpoint{-0.048611in}{0.000000in}}{\pgfqpoint{0.000000in}{0.000000in}}{%
\pgfpathmoveto{\pgfqpoint{0.000000in}{0.000000in}}%
\pgfpathlineto{\pgfqpoint{-0.048611in}{0.000000in}}%
\pgfusepath{stroke,fill}%
}%
\begin{pgfscope}%
\pgfsys@transformshift{0.556250in}{1.902795in}%
\pgfsys@useobject{currentmarker}{}%
\end{pgfscope}%
\end{pgfscope}%
\begin{pgfscope}%
\definecolor{textcolor}{rgb}{0.000000,0.000000,0.000000}%
\pgfsetstrokecolor{textcolor}%
\pgfsetfillcolor{textcolor}%
\pgftext[x=0.238148in,y=1.850034in,left,base]{\color{textcolor}\sffamily\fontsize{10.000000}{12.000000}\selectfont 0.6}%
\end{pgfscope}%
\begin{pgfscope}%
\pgfsetbuttcap%
\pgfsetroundjoin%
\definecolor{currentfill}{rgb}{0.000000,0.000000,0.000000}%
\pgfsetfillcolor{currentfill}%
\pgfsetlinewidth{0.803000pt}%
\definecolor{currentstroke}{rgb}{0.000000,0.000000,0.000000}%
\pgfsetstrokecolor{currentstroke}%
\pgfsetdash{}{0pt}%
\pgfsys@defobject{currentmarker}{\pgfqpoint{-0.048611in}{0.000000in}}{\pgfqpoint{0.000000in}{0.000000in}}{%
\pgfpathmoveto{\pgfqpoint{0.000000in}{0.000000in}}%
\pgfpathlineto{\pgfqpoint{-0.048611in}{0.000000in}}%
\pgfusepath{stroke,fill}%
}%
\begin{pgfscope}%
\pgfsys@transformshift{0.556250in}{2.234738in}%
\pgfsys@useobject{currentmarker}{}%
\end{pgfscope}%
\end{pgfscope}%
\begin{pgfscope}%
\definecolor{textcolor}{rgb}{0.000000,0.000000,0.000000}%
\pgfsetstrokecolor{textcolor}%
\pgfsetfillcolor{textcolor}%
\pgftext[x=0.238148in,y=2.181977in,left,base]{\color{textcolor}\sffamily\fontsize{10.000000}{12.000000}\selectfont 0.7}%
\end{pgfscope}%
\begin{pgfscope}%
\pgfsetbuttcap%
\pgfsetroundjoin%
\definecolor{currentfill}{rgb}{0.000000,0.000000,0.000000}%
\pgfsetfillcolor{currentfill}%
\pgfsetlinewidth{0.803000pt}%
\definecolor{currentstroke}{rgb}{0.000000,0.000000,0.000000}%
\pgfsetstrokecolor{currentstroke}%
\pgfsetdash{}{0pt}%
\pgfsys@defobject{currentmarker}{\pgfqpoint{-0.048611in}{0.000000in}}{\pgfqpoint{0.000000in}{0.000000in}}{%
\pgfpathmoveto{\pgfqpoint{0.000000in}{0.000000in}}%
\pgfpathlineto{\pgfqpoint{-0.048611in}{0.000000in}}%
\pgfusepath{stroke,fill}%
}%
\begin{pgfscope}%
\pgfsys@transformshift{0.556250in}{2.566681in}%
\pgfsys@useobject{currentmarker}{}%
\end{pgfscope}%
\end{pgfscope}%
\begin{pgfscope}%
\definecolor{textcolor}{rgb}{0.000000,0.000000,0.000000}%
\pgfsetstrokecolor{textcolor}%
\pgfsetfillcolor{textcolor}%
\pgftext[x=0.238148in,y=2.513919in,left,base]{\color{textcolor}\sffamily\fontsize{10.000000}{12.000000}\selectfont 0.8}%
\end{pgfscope}%
\begin{pgfscope}%
\pgfpathrectangle{\pgfqpoint{0.556250in}{0.386111in}}{\pgfqpoint{3.317253in}{2.316667in}}%
\pgfusepath{clip}%
\pgfsetrectcap%
\pgfsetroundjoin%
\pgfsetlinewidth{1.505625pt}%
\definecolor{currentstroke}{rgb}{0.121569,0.466667,0.705882}%
\pgfsetstrokecolor{currentstroke}%
\pgfsetdash{}{0pt}%
\pgfpathmoveto{\pgfqpoint{0.718506in}{2.586465in}}%
\pgfpathlineto{\pgfqpoint{0.868143in}{2.554179in}}%
\pgfpathlineto{\pgfqpoint{1.017780in}{2.505856in}}%
\pgfpathlineto{\pgfqpoint{1.167417in}{2.481432in}}%
\pgfpathlineto{\pgfqpoint{1.317054in}{2.355537in}}%
\pgfpathlineto{\pgfqpoint{1.466691in}{2.352004in}}%
\pgfpathlineto{\pgfqpoint{1.616328in}{2.272902in}}%
\pgfpathlineto{\pgfqpoint{1.765965in}{2.170579in}}%
\pgfpathlineto{\pgfqpoint{1.915602in}{1.929597in}}%
\pgfpathlineto{\pgfqpoint{2.065239in}{1.828049in}}%
\pgfpathlineto{\pgfqpoint{2.214876in}{1.737199in}}%
\pgfpathlineto{\pgfqpoint{2.364513in}{1.552925in}}%
\pgfpathlineto{\pgfqpoint{2.514150in}{1.416435in}}%
\pgfpathlineto{\pgfqpoint{2.663787in}{1.230052in}}%
\pgfpathlineto{\pgfqpoint{2.813424in}{0.943970in}}%
\pgfpathlineto{\pgfqpoint{2.963061in}{0.908969in}}%
\pgfpathlineto{\pgfqpoint{3.112698in}{0.808650in}}%
\pgfpathlineto{\pgfqpoint{3.262335in}{0.676291in}}%
\pgfpathlineto{\pgfqpoint{3.411972in}{0.674030in}}%
\pgfpathlineto{\pgfqpoint{3.561609in}{0.592167in}}%
\pgfpathlineto{\pgfqpoint{3.711246in}{0.534157in}}%
\pgfusepath{stroke}%
\end{pgfscope}%
\begin{pgfscope}%
\pgfsetrectcap%
\pgfsetmiterjoin%
\pgfsetlinewidth{0.803000pt}%
\definecolor{currentstroke}{rgb}{0.000000,0.000000,0.000000}%
\pgfsetstrokecolor{currentstroke}%
\pgfsetdash{}{0pt}%
\pgfpathmoveto{\pgfqpoint{0.556250in}{0.386111in}}%
\pgfpathlineto{\pgfqpoint{0.556250in}{2.702778in}}%
\pgfusepath{stroke}%
\end{pgfscope}%
\begin{pgfscope}%
\pgfsetrectcap%
\pgfsetmiterjoin%
\pgfsetlinewidth{0.803000pt}%
\definecolor{currentstroke}{rgb}{0.000000,0.000000,0.000000}%
\pgfsetstrokecolor{currentstroke}%
\pgfsetdash{}{0pt}%
\pgfpathmoveto{\pgfqpoint{3.873503in}{0.386111in}}%
\pgfpathlineto{\pgfqpoint{3.873503in}{2.702778in}}%
\pgfusepath{stroke}%
\end{pgfscope}%
\begin{pgfscope}%
\pgfsetrectcap%
\pgfsetmiterjoin%
\pgfsetlinewidth{0.803000pt}%
\definecolor{currentstroke}{rgb}{0.000000,0.000000,0.000000}%
\pgfsetstrokecolor{currentstroke}%
\pgfsetdash{}{0pt}%
\pgfpathmoveto{\pgfqpoint{0.556250in}{0.386111in}}%
\pgfpathlineto{\pgfqpoint{3.873503in}{0.386111in}}%
\pgfusepath{stroke}%
\end{pgfscope}%
\begin{pgfscope}%
\pgfsetrectcap%
\pgfsetmiterjoin%
\pgfsetlinewidth{0.803000pt}%
\definecolor{currentstroke}{rgb}{0.000000,0.000000,0.000000}%
\pgfsetstrokecolor{currentstroke}%
\pgfsetdash{}{0pt}%
\pgfpathmoveto{\pgfqpoint{0.556250in}{2.702778in}}%
\pgfpathlineto{\pgfqpoint{3.873503in}{2.702778in}}%
\pgfusepath{stroke}%
\end{pgfscope}%
\begin{pgfscope}%
\definecolor{textcolor}{rgb}{0.000000,0.000000,0.000000}%
\pgfsetstrokecolor{textcolor}%
\pgfsetfillcolor{textcolor}%
\pgftext[x=2.214876in,y=2.786111in,,base]{\color{textcolor}\sffamily\fontsize{12.000000}{14.400000}\selectfont \(\displaystyle  N = 64 \)}%
\end{pgfscope}%
\begin{pgfscope}%
\pgfsetbuttcap%
\pgfsetmiterjoin%
\definecolor{currentfill}{rgb}{1.000000,1.000000,1.000000}%
\pgfsetfillcolor{currentfill}%
\pgfsetlinewidth{0.000000pt}%
\definecolor{currentstroke}{rgb}{0.000000,0.000000,0.000000}%
\pgfsetstrokecolor{currentstroke}%
\pgfsetstrokeopacity{0.000000}%
\pgfsetdash{}{0pt}%
\pgfpathmoveto{\pgfqpoint{4.389822in}{0.386111in}}%
\pgfpathlineto{\pgfqpoint{7.707075in}{0.386111in}}%
\pgfpathlineto{\pgfqpoint{7.707075in}{2.702778in}}%
\pgfpathlineto{\pgfqpoint{4.389822in}{2.702778in}}%
\pgfpathclose%
\pgfusepath{fill}%
\end{pgfscope}%
\begin{pgfscope}%
\pgfpathrectangle{\pgfqpoint{4.389822in}{0.386111in}}{\pgfqpoint{3.317253in}{2.316667in}}%
\pgfusepath{clip}%
\pgfsetbuttcap%
\pgfsetroundjoin%
\definecolor{currentfill}{rgb}{0.121569,0.466667,0.705882}%
\pgfsetfillcolor{currentfill}%
\pgfsetfillopacity{0.300000}%
\pgfsetlinewidth{1.003750pt}%
\definecolor{currentstroke}{rgb}{0.121569,0.466667,0.705882}%
\pgfsetstrokecolor{currentstroke}%
\pgfsetstrokeopacity{0.300000}%
\pgfsetdash{}{0pt}%
\pgfpathmoveto{\pgfqpoint{4.552079in}{2.281621in}}%
\pgfpathlineto{\pgfqpoint{4.552079in}{2.263438in}}%
\pgfpathlineto{\pgfqpoint{4.701716in}{2.160999in}}%
\pgfpathlineto{\pgfqpoint{4.851353in}{0.744855in}}%
\pgfpathlineto{\pgfqpoint{5.000990in}{2.124176in}}%
\pgfpathlineto{\pgfqpoint{5.150627in}{1.074585in}}%
\pgfpathlineto{\pgfqpoint{5.300264in}{1.166211in}}%
\pgfpathlineto{\pgfqpoint{5.449901in}{1.586304in}}%
\pgfpathlineto{\pgfqpoint{5.599538in}{1.847841in}}%
\pgfpathlineto{\pgfqpoint{5.749175in}{1.482786in}}%
\pgfpathlineto{\pgfqpoint{5.898812in}{0.910624in}}%
\pgfpathlineto{\pgfqpoint{6.048449in}{1.384628in}}%
\pgfpathlineto{\pgfqpoint{6.198086in}{1.514158in}}%
\pgfpathlineto{\pgfqpoint{6.347723in}{0.719698in}}%
\pgfpathlineto{\pgfqpoint{6.497360in}{0.654623in}}%
\pgfpathlineto{\pgfqpoint{6.646997in}{0.756180in}}%
\pgfpathlineto{\pgfqpoint{6.796634in}{0.645942in}}%
\pgfpathlineto{\pgfqpoint{6.946271in}{0.572318in}}%
\pgfpathlineto{\pgfqpoint{7.095908in}{0.577242in}}%
\pgfpathlineto{\pgfqpoint{7.245545in}{0.554149in}}%
\pgfpathlineto{\pgfqpoint{7.395182in}{0.491414in}}%
\pgfpathlineto{\pgfqpoint{7.544819in}{0.503131in}}%
\pgfpathlineto{\pgfqpoint{7.544819in}{0.658242in}}%
\pgfpathlineto{\pgfqpoint{7.544819in}{0.658242in}}%
\pgfpathlineto{\pgfqpoint{7.395182in}{0.633553in}}%
\pgfpathlineto{\pgfqpoint{7.245545in}{0.687042in}}%
\pgfpathlineto{\pgfqpoint{7.095908in}{0.743984in}}%
\pgfpathlineto{\pgfqpoint{6.946271in}{0.876713in}}%
\pgfpathlineto{\pgfqpoint{6.796634in}{0.718345in}}%
\pgfpathlineto{\pgfqpoint{6.646997in}{0.990326in}}%
\pgfpathlineto{\pgfqpoint{6.497360in}{1.297382in}}%
\pgfpathlineto{\pgfqpoint{6.347723in}{1.326184in}}%
\pgfpathlineto{\pgfqpoint{6.198086in}{1.731089in}}%
\pgfpathlineto{\pgfqpoint{6.048449in}{1.843382in}}%
\pgfpathlineto{\pgfqpoint{5.898812in}{2.255180in}}%
\pgfpathlineto{\pgfqpoint{5.749175in}{1.998813in}}%
\pgfpathlineto{\pgfqpoint{5.599538in}{2.012597in}}%
\pgfpathlineto{\pgfqpoint{5.449901in}{2.298083in}}%
\pgfpathlineto{\pgfqpoint{5.300264in}{2.399360in}}%
\pgfpathlineto{\pgfqpoint{5.150627in}{2.597475in}}%
\pgfpathlineto{\pgfqpoint{5.000990in}{2.209604in}}%
\pgfpathlineto{\pgfqpoint{4.851353in}{2.121344in}}%
\pgfpathlineto{\pgfqpoint{4.701716in}{2.283128in}}%
\pgfpathlineto{\pgfqpoint{4.552079in}{2.281621in}}%
\pgfpathclose%
\pgfusepath{stroke,fill}%
\end{pgfscope}%
\begin{pgfscope}%
\pgfpathrectangle{\pgfqpoint{4.389822in}{0.386111in}}{\pgfqpoint{3.317253in}{2.316667in}}%
\pgfusepath{clip}%
\pgfsetbuttcap%
\pgfsetroundjoin%
\definecolor{currentfill}{rgb}{0.121569,0.466667,0.705882}%
\pgfsetfillcolor{currentfill}%
\pgfsetlinewidth{1.003750pt}%
\definecolor{currentstroke}{rgb}{0.121569,0.466667,0.705882}%
\pgfsetstrokecolor{currentstroke}%
\pgfsetdash{}{0pt}%
\pgfsys@defobject{currentmarker}{\pgfqpoint{-0.009821in}{-0.009821in}}{\pgfqpoint{0.009821in}{0.009821in}}{%
\pgfpathmoveto{\pgfqpoint{0.000000in}{-0.009821in}}%
\pgfpathcurveto{\pgfqpoint{0.002605in}{-0.009821in}}{\pgfqpoint{0.005103in}{-0.008786in}}{\pgfqpoint{0.006944in}{-0.006944in}}%
\pgfpathcurveto{\pgfqpoint{0.008786in}{-0.005103in}}{\pgfqpoint{0.009821in}{-0.002605in}}{\pgfqpoint{0.009821in}{0.000000in}}%
\pgfpathcurveto{\pgfqpoint{0.009821in}{0.002605in}}{\pgfqpoint{0.008786in}{0.005103in}}{\pgfqpoint{0.006944in}{0.006944in}}%
\pgfpathcurveto{\pgfqpoint{0.005103in}{0.008786in}}{\pgfqpoint{0.002605in}{0.009821in}}{\pgfqpoint{0.000000in}{0.009821in}}%
\pgfpathcurveto{\pgfqpoint{-0.002605in}{0.009821in}}{\pgfqpoint{-0.005103in}{0.008786in}}{\pgfqpoint{-0.006944in}{0.006944in}}%
\pgfpathcurveto{\pgfqpoint{-0.008786in}{0.005103in}}{\pgfqpoint{-0.009821in}{0.002605in}}{\pgfqpoint{-0.009821in}{0.000000in}}%
\pgfpathcurveto{\pgfqpoint{-0.009821in}{-0.002605in}}{\pgfqpoint{-0.008786in}{-0.005103in}}{\pgfqpoint{-0.006944in}{-0.006944in}}%
\pgfpathcurveto{\pgfqpoint{-0.005103in}{-0.008786in}}{\pgfqpoint{-0.002605in}{-0.009821in}}{\pgfqpoint{0.000000in}{-0.009821in}}%
\pgfpathclose%
\pgfusepath{stroke,fill}%
}%
\begin{pgfscope}%
\pgfsys@transformshift{4.552079in}{2.272530in}%
\pgfsys@useobject{currentmarker}{}%
\end{pgfscope}%
\begin{pgfscope}%
\pgfsys@transformshift{4.701716in}{2.222063in}%
\pgfsys@useobject{currentmarker}{}%
\end{pgfscope}%
\begin{pgfscope}%
\pgfsys@transformshift{4.851353in}{1.433099in}%
\pgfsys@useobject{currentmarker}{}%
\end{pgfscope}%
\begin{pgfscope}%
\pgfsys@transformshift{5.000990in}{2.166890in}%
\pgfsys@useobject{currentmarker}{}%
\end{pgfscope}%
\begin{pgfscope}%
\pgfsys@transformshift{5.150627in}{1.836030in}%
\pgfsys@useobject{currentmarker}{}%
\end{pgfscope}%
\begin{pgfscope}%
\pgfsys@transformshift{5.300264in}{1.782785in}%
\pgfsys@useobject{currentmarker}{}%
\end{pgfscope}%
\begin{pgfscope}%
\pgfsys@transformshift{5.449901in}{1.942193in}%
\pgfsys@useobject{currentmarker}{}%
\end{pgfscope}%
\begin{pgfscope}%
\pgfsys@transformshift{5.599538in}{1.930219in}%
\pgfsys@useobject{currentmarker}{}%
\end{pgfscope}%
\begin{pgfscope}%
\pgfsys@transformshift{5.749175in}{1.740800in}%
\pgfsys@useobject{currentmarker}{}%
\end{pgfscope}%
\begin{pgfscope}%
\pgfsys@transformshift{5.898812in}{1.582902in}%
\pgfsys@useobject{currentmarker}{}%
\end{pgfscope}%
\begin{pgfscope}%
\pgfsys@transformshift{6.048449in}{1.614005in}%
\pgfsys@useobject{currentmarker}{}%
\end{pgfscope}%
\begin{pgfscope}%
\pgfsys@transformshift{6.198086in}{1.622623in}%
\pgfsys@useobject{currentmarker}{}%
\end{pgfscope}%
\begin{pgfscope}%
\pgfsys@transformshift{6.347723in}{1.022941in}%
\pgfsys@useobject{currentmarker}{}%
\end{pgfscope}%
\begin{pgfscope}%
\pgfsys@transformshift{6.497360in}{0.976003in}%
\pgfsys@useobject{currentmarker}{}%
\end{pgfscope}%
\begin{pgfscope}%
\pgfsys@transformshift{6.646997in}{0.873253in}%
\pgfsys@useobject{currentmarker}{}%
\end{pgfscope}%
\begin{pgfscope}%
\pgfsys@transformshift{6.796634in}{0.682144in}%
\pgfsys@useobject{currentmarker}{}%
\end{pgfscope}%
\begin{pgfscope}%
\pgfsys@transformshift{6.946271in}{0.724516in}%
\pgfsys@useobject{currentmarker}{}%
\end{pgfscope}%
\begin{pgfscope}%
\pgfsys@transformshift{7.095908in}{0.660613in}%
\pgfsys@useobject{currentmarker}{}%
\end{pgfscope}%
\begin{pgfscope}%
\pgfsys@transformshift{7.245545in}{0.620595in}%
\pgfsys@useobject{currentmarker}{}%
\end{pgfscope}%
\begin{pgfscope}%
\pgfsys@transformshift{7.395182in}{0.562484in}%
\pgfsys@useobject{currentmarker}{}%
\end{pgfscope}%
\begin{pgfscope}%
\pgfsys@transformshift{7.544819in}{0.580687in}%
\pgfsys@useobject{currentmarker}{}%
\end{pgfscope}%
\end{pgfscope}%
\begin{pgfscope}%
\pgfsetbuttcap%
\pgfsetroundjoin%
\definecolor{currentfill}{rgb}{0.000000,0.000000,0.000000}%
\pgfsetfillcolor{currentfill}%
\pgfsetlinewidth{0.803000pt}%
\definecolor{currentstroke}{rgb}{0.000000,0.000000,0.000000}%
\pgfsetstrokecolor{currentstroke}%
\pgfsetdash{}{0pt}%
\pgfsys@defobject{currentmarker}{\pgfqpoint{0.000000in}{-0.048611in}}{\pgfqpoint{0.000000in}{0.000000in}}{%
\pgfpathmoveto{\pgfqpoint{0.000000in}{0.000000in}}%
\pgfpathlineto{\pgfqpoint{0.000000in}{-0.048611in}}%
\pgfusepath{stroke,fill}%
}%
\begin{pgfscope}%
\pgfsys@transformshift{5.000990in}{0.386111in}%
\pgfsys@useobject{currentmarker}{}%
\end{pgfscope}%
\end{pgfscope}%
\begin{pgfscope}%
\definecolor{textcolor}{rgb}{0.000000,0.000000,0.000000}%
\pgfsetstrokecolor{textcolor}%
\pgfsetfillcolor{textcolor}%
\pgftext[x=5.000990in,y=0.288889in,,top]{\color{textcolor}\sffamily\fontsize{10.000000}{12.000000}\selectfont 2.20}%
\end{pgfscope}%
\begin{pgfscope}%
\pgfsetbuttcap%
\pgfsetroundjoin%
\definecolor{currentfill}{rgb}{0.000000,0.000000,0.000000}%
\pgfsetfillcolor{currentfill}%
\pgfsetlinewidth{0.803000pt}%
\definecolor{currentstroke}{rgb}{0.000000,0.000000,0.000000}%
\pgfsetstrokecolor{currentstroke}%
\pgfsetdash{}{0pt}%
\pgfsys@defobject{currentmarker}{\pgfqpoint{0.000000in}{-0.048611in}}{\pgfqpoint{0.000000in}{0.000000in}}{%
\pgfpathmoveto{\pgfqpoint{0.000000in}{0.000000in}}%
\pgfpathlineto{\pgfqpoint{0.000000in}{-0.048611in}}%
\pgfusepath{stroke,fill}%
}%
\begin{pgfscope}%
\pgfsys@transformshift{5.749175in}{0.386111in}%
\pgfsys@useobject{currentmarker}{}%
\end{pgfscope}%
\end{pgfscope}%
\begin{pgfscope}%
\definecolor{textcolor}{rgb}{0.000000,0.000000,0.000000}%
\pgfsetstrokecolor{textcolor}%
\pgfsetfillcolor{textcolor}%
\pgftext[x=5.749175in,y=0.288889in,,top]{\color{textcolor}\sffamily\fontsize{10.000000}{12.000000}\selectfont 2.25}%
\end{pgfscope}%
\begin{pgfscope}%
\pgfsetbuttcap%
\pgfsetroundjoin%
\definecolor{currentfill}{rgb}{0.000000,0.000000,0.000000}%
\pgfsetfillcolor{currentfill}%
\pgfsetlinewidth{0.803000pt}%
\definecolor{currentstroke}{rgb}{0.000000,0.000000,0.000000}%
\pgfsetstrokecolor{currentstroke}%
\pgfsetdash{}{0pt}%
\pgfsys@defobject{currentmarker}{\pgfqpoint{0.000000in}{-0.048611in}}{\pgfqpoint{0.000000in}{0.000000in}}{%
\pgfpathmoveto{\pgfqpoint{0.000000in}{0.000000in}}%
\pgfpathlineto{\pgfqpoint{0.000000in}{-0.048611in}}%
\pgfusepath{stroke,fill}%
}%
\begin{pgfscope}%
\pgfsys@transformshift{6.497360in}{0.386111in}%
\pgfsys@useobject{currentmarker}{}%
\end{pgfscope}%
\end{pgfscope}%
\begin{pgfscope}%
\definecolor{textcolor}{rgb}{0.000000,0.000000,0.000000}%
\pgfsetstrokecolor{textcolor}%
\pgfsetfillcolor{textcolor}%
\pgftext[x=6.497360in,y=0.288889in,,top]{\color{textcolor}\sffamily\fontsize{10.000000}{12.000000}\selectfont 2.30}%
\end{pgfscope}%
\begin{pgfscope}%
\pgfsetbuttcap%
\pgfsetroundjoin%
\definecolor{currentfill}{rgb}{0.000000,0.000000,0.000000}%
\pgfsetfillcolor{currentfill}%
\pgfsetlinewidth{0.803000pt}%
\definecolor{currentstroke}{rgb}{0.000000,0.000000,0.000000}%
\pgfsetstrokecolor{currentstroke}%
\pgfsetdash{}{0pt}%
\pgfsys@defobject{currentmarker}{\pgfqpoint{0.000000in}{-0.048611in}}{\pgfqpoint{0.000000in}{0.000000in}}{%
\pgfpathmoveto{\pgfqpoint{0.000000in}{0.000000in}}%
\pgfpathlineto{\pgfqpoint{0.000000in}{-0.048611in}}%
\pgfusepath{stroke,fill}%
}%
\begin{pgfscope}%
\pgfsys@transformshift{7.245545in}{0.386111in}%
\pgfsys@useobject{currentmarker}{}%
\end{pgfscope}%
\end{pgfscope}%
\begin{pgfscope}%
\definecolor{textcolor}{rgb}{0.000000,0.000000,0.000000}%
\pgfsetstrokecolor{textcolor}%
\pgfsetfillcolor{textcolor}%
\pgftext[x=7.245545in,y=0.288889in,,top]{\color{textcolor}\sffamily\fontsize{10.000000}{12.000000}\selectfont 2.35}%
\end{pgfscope}%
\begin{pgfscope}%
\pgfsetbuttcap%
\pgfsetroundjoin%
\definecolor{currentfill}{rgb}{0.000000,0.000000,0.000000}%
\pgfsetfillcolor{currentfill}%
\pgfsetlinewidth{0.803000pt}%
\definecolor{currentstroke}{rgb}{0.000000,0.000000,0.000000}%
\pgfsetstrokecolor{currentstroke}%
\pgfsetdash{}{0pt}%
\pgfsys@defobject{currentmarker}{\pgfqpoint{-0.048611in}{0.000000in}}{\pgfqpoint{0.000000in}{0.000000in}}{%
\pgfpathmoveto{\pgfqpoint{0.000000in}{0.000000in}}%
\pgfpathlineto{\pgfqpoint{-0.048611in}{0.000000in}}%
\pgfusepath{stroke,fill}%
}%
\begin{pgfscope}%
\pgfsys@transformshift{4.389822in}{0.802068in}%
\pgfsys@useobject{currentmarker}{}%
\end{pgfscope}%
\end{pgfscope}%
\begin{pgfscope}%
\definecolor{textcolor}{rgb}{0.000000,0.000000,0.000000}%
\pgfsetstrokecolor{textcolor}%
\pgfsetfillcolor{textcolor}%
\pgftext[x=4.071721in,y=0.749306in,left,base]{\color{textcolor}\sffamily\fontsize{10.000000}{12.000000}\selectfont 0.2}%
\end{pgfscope}%
\begin{pgfscope}%
\pgfsetbuttcap%
\pgfsetroundjoin%
\definecolor{currentfill}{rgb}{0.000000,0.000000,0.000000}%
\pgfsetfillcolor{currentfill}%
\pgfsetlinewidth{0.803000pt}%
\definecolor{currentstroke}{rgb}{0.000000,0.000000,0.000000}%
\pgfsetstrokecolor{currentstroke}%
\pgfsetdash{}{0pt}%
\pgfsys@defobject{currentmarker}{\pgfqpoint{-0.048611in}{0.000000in}}{\pgfqpoint{0.000000in}{0.000000in}}{%
\pgfpathmoveto{\pgfqpoint{0.000000in}{0.000000in}}%
\pgfpathlineto{\pgfqpoint{-0.048611in}{0.000000in}}%
\pgfusepath{stroke,fill}%
}%
\begin{pgfscope}%
\pgfsys@transformshift{4.389822in}{1.277067in}%
\pgfsys@useobject{currentmarker}{}%
\end{pgfscope}%
\end{pgfscope}%
\begin{pgfscope}%
\definecolor{textcolor}{rgb}{0.000000,0.000000,0.000000}%
\pgfsetstrokecolor{textcolor}%
\pgfsetfillcolor{textcolor}%
\pgftext[x=4.071721in,y=1.224305in,left,base]{\color{textcolor}\sffamily\fontsize{10.000000}{12.000000}\selectfont 0.4}%
\end{pgfscope}%
\begin{pgfscope}%
\pgfsetbuttcap%
\pgfsetroundjoin%
\definecolor{currentfill}{rgb}{0.000000,0.000000,0.000000}%
\pgfsetfillcolor{currentfill}%
\pgfsetlinewidth{0.803000pt}%
\definecolor{currentstroke}{rgb}{0.000000,0.000000,0.000000}%
\pgfsetstrokecolor{currentstroke}%
\pgfsetdash{}{0pt}%
\pgfsys@defobject{currentmarker}{\pgfqpoint{-0.048611in}{0.000000in}}{\pgfqpoint{0.000000in}{0.000000in}}{%
\pgfpathmoveto{\pgfqpoint{0.000000in}{0.000000in}}%
\pgfpathlineto{\pgfqpoint{-0.048611in}{0.000000in}}%
\pgfusepath{stroke,fill}%
}%
\begin{pgfscope}%
\pgfsys@transformshift{4.389822in}{1.752066in}%
\pgfsys@useobject{currentmarker}{}%
\end{pgfscope}%
\end{pgfscope}%
\begin{pgfscope}%
\definecolor{textcolor}{rgb}{0.000000,0.000000,0.000000}%
\pgfsetstrokecolor{textcolor}%
\pgfsetfillcolor{textcolor}%
\pgftext[x=4.071721in,y=1.699305in,left,base]{\color{textcolor}\sffamily\fontsize{10.000000}{12.000000}\selectfont 0.6}%
\end{pgfscope}%
\begin{pgfscope}%
\pgfsetbuttcap%
\pgfsetroundjoin%
\definecolor{currentfill}{rgb}{0.000000,0.000000,0.000000}%
\pgfsetfillcolor{currentfill}%
\pgfsetlinewidth{0.803000pt}%
\definecolor{currentstroke}{rgb}{0.000000,0.000000,0.000000}%
\pgfsetstrokecolor{currentstroke}%
\pgfsetdash{}{0pt}%
\pgfsys@defobject{currentmarker}{\pgfqpoint{-0.048611in}{0.000000in}}{\pgfqpoint{0.000000in}{0.000000in}}{%
\pgfpathmoveto{\pgfqpoint{0.000000in}{0.000000in}}%
\pgfpathlineto{\pgfqpoint{-0.048611in}{0.000000in}}%
\pgfusepath{stroke,fill}%
}%
\begin{pgfscope}%
\pgfsys@transformshift{4.389822in}{2.227066in}%
\pgfsys@useobject{currentmarker}{}%
\end{pgfscope}%
\end{pgfscope}%
\begin{pgfscope}%
\definecolor{textcolor}{rgb}{0.000000,0.000000,0.000000}%
\pgfsetstrokecolor{textcolor}%
\pgfsetfillcolor{textcolor}%
\pgftext[x=4.071721in,y=2.174304in,left,base]{\color{textcolor}\sffamily\fontsize{10.000000}{12.000000}\selectfont 0.8}%
\end{pgfscope}%
\begin{pgfscope}%
\pgfsetbuttcap%
\pgfsetroundjoin%
\definecolor{currentfill}{rgb}{0.000000,0.000000,0.000000}%
\pgfsetfillcolor{currentfill}%
\pgfsetlinewidth{0.803000pt}%
\definecolor{currentstroke}{rgb}{0.000000,0.000000,0.000000}%
\pgfsetstrokecolor{currentstroke}%
\pgfsetdash{}{0pt}%
\pgfsys@defobject{currentmarker}{\pgfqpoint{-0.048611in}{0.000000in}}{\pgfqpoint{0.000000in}{0.000000in}}{%
\pgfpathmoveto{\pgfqpoint{0.000000in}{0.000000in}}%
\pgfpathlineto{\pgfqpoint{-0.048611in}{0.000000in}}%
\pgfusepath{stroke,fill}%
}%
\begin{pgfscope}%
\pgfsys@transformshift{4.389822in}{2.702065in}%
\pgfsys@useobject{currentmarker}{}%
\end{pgfscope}%
\end{pgfscope}%
\begin{pgfscope}%
\definecolor{textcolor}{rgb}{0.000000,0.000000,0.000000}%
\pgfsetstrokecolor{textcolor}%
\pgfsetfillcolor{textcolor}%
\pgftext[x=4.071721in,y=2.649304in,left,base]{\color{textcolor}\sffamily\fontsize{10.000000}{12.000000}\selectfont 1.0}%
\end{pgfscope}%
\begin{pgfscope}%
\pgfpathrectangle{\pgfqpoint{4.389822in}{0.386111in}}{\pgfqpoint{3.317253in}{2.316667in}}%
\pgfusepath{clip}%
\pgfsetrectcap%
\pgfsetroundjoin%
\pgfsetlinewidth{1.505625pt}%
\definecolor{currentstroke}{rgb}{0.121569,0.466667,0.705882}%
\pgfsetstrokecolor{currentstroke}%
\pgfsetdash{}{0pt}%
\pgfpathmoveto{\pgfqpoint{4.552079in}{2.272530in}}%
\pgfpathlineto{\pgfqpoint{4.701716in}{2.222063in}}%
\pgfpathlineto{\pgfqpoint{4.851353in}{1.433099in}}%
\pgfpathlineto{\pgfqpoint{5.000990in}{2.166890in}}%
\pgfpathlineto{\pgfqpoint{5.150627in}{1.836030in}}%
\pgfpathlineto{\pgfqpoint{5.300264in}{1.782785in}}%
\pgfpathlineto{\pgfqpoint{5.449901in}{1.942193in}}%
\pgfpathlineto{\pgfqpoint{5.599538in}{1.930219in}}%
\pgfpathlineto{\pgfqpoint{5.749175in}{1.740800in}}%
\pgfpathlineto{\pgfqpoint{5.898812in}{1.582902in}}%
\pgfpathlineto{\pgfqpoint{6.048449in}{1.614005in}}%
\pgfpathlineto{\pgfqpoint{6.198086in}{1.622623in}}%
\pgfpathlineto{\pgfqpoint{6.347723in}{1.022941in}}%
\pgfpathlineto{\pgfqpoint{6.497360in}{0.976003in}}%
\pgfpathlineto{\pgfqpoint{6.646997in}{0.873253in}}%
\pgfpathlineto{\pgfqpoint{6.796634in}{0.682144in}}%
\pgfpathlineto{\pgfqpoint{6.946271in}{0.724516in}}%
\pgfpathlineto{\pgfqpoint{7.095908in}{0.660613in}}%
\pgfpathlineto{\pgfqpoint{7.245545in}{0.620595in}}%
\pgfpathlineto{\pgfqpoint{7.395182in}{0.562484in}}%
\pgfpathlineto{\pgfqpoint{7.544819in}{0.580687in}}%
\pgfusepath{stroke}%
\end{pgfscope}%
\begin{pgfscope}%
\pgfsetrectcap%
\pgfsetmiterjoin%
\pgfsetlinewidth{0.803000pt}%
\definecolor{currentstroke}{rgb}{0.000000,0.000000,0.000000}%
\pgfsetstrokecolor{currentstroke}%
\pgfsetdash{}{0pt}%
\pgfpathmoveto{\pgfqpoint{4.389822in}{0.386111in}}%
\pgfpathlineto{\pgfqpoint{4.389822in}{2.702778in}}%
\pgfusepath{stroke}%
\end{pgfscope}%
\begin{pgfscope}%
\pgfsetrectcap%
\pgfsetmiterjoin%
\pgfsetlinewidth{0.803000pt}%
\definecolor{currentstroke}{rgb}{0.000000,0.000000,0.000000}%
\pgfsetstrokecolor{currentstroke}%
\pgfsetdash{}{0pt}%
\pgfpathmoveto{\pgfqpoint{7.707075in}{0.386111in}}%
\pgfpathlineto{\pgfqpoint{7.707075in}{2.702778in}}%
\pgfusepath{stroke}%
\end{pgfscope}%
\begin{pgfscope}%
\pgfsetrectcap%
\pgfsetmiterjoin%
\pgfsetlinewidth{0.803000pt}%
\definecolor{currentstroke}{rgb}{0.000000,0.000000,0.000000}%
\pgfsetstrokecolor{currentstroke}%
\pgfsetdash{}{0pt}%
\pgfpathmoveto{\pgfqpoint{4.389822in}{0.386111in}}%
\pgfpathlineto{\pgfqpoint{7.707075in}{0.386111in}}%
\pgfusepath{stroke}%
\end{pgfscope}%
\begin{pgfscope}%
\pgfsetrectcap%
\pgfsetmiterjoin%
\pgfsetlinewidth{0.803000pt}%
\definecolor{currentstroke}{rgb}{0.000000,0.000000,0.000000}%
\pgfsetstrokecolor{currentstroke}%
\pgfsetdash{}{0pt}%
\pgfpathmoveto{\pgfqpoint{4.389822in}{2.702778in}}%
\pgfpathlineto{\pgfqpoint{7.707075in}{2.702778in}}%
\pgfusepath{stroke}%
\end{pgfscope}%
\begin{pgfscope}%
\definecolor{textcolor}{rgb}{0.000000,0.000000,0.000000}%
\pgfsetstrokecolor{textcolor}%
\pgfsetfillcolor{textcolor}%
\pgftext[x=6.048449in,y=2.786111in,,base]{\color{textcolor}\sffamily\fontsize{12.000000}{14.400000}\selectfont \(\displaystyle  N = 128 \)}%
\end{pgfscope}%
\end{pgfpicture}%
\makeatother%
\endgroup%
}
\scalebox{0.75}{%% Creator: Matplotlib, PGF backend
%%
%% To include the figure in your LaTeX document, write
%%   \input{<filename>.pgf}
%%
%% Make sure the required packages are loaded in your preamble
%%   \usepackage{pgf}
%%
%% Figures using additional raster images can only be included by \input if
%% they are in the same directory as the main LaTeX file. For loading figures
%% from other directories you can use the `import` package
%%   \usepackage{import}
%% and then include the figures with
%%   \import{<path to file>}{<filename>.pgf}
%%
%% Matplotlib used the following preamble
%%   \usepackage{fontspec}
%%   \setmainfont{DejaVuSerif.ttf}[Path=/home/lzh/anaconda3/envs/numana/lib/python3.7/site-packages/matplotlib/mpl-data/fonts/ttf/]
%%   \setsansfont{DejaVuSans.ttf}[Path=/home/lzh/anaconda3/envs/numana/lib/python3.7/site-packages/matplotlib/mpl-data/fonts/ttf/]
%%   \setmonofont{DejaVuSansMono.ttf}[Path=/home/lzh/anaconda3/envs/numana/lib/python3.7/site-packages/matplotlib/mpl-data/fonts/ttf/]
%%
\begingroup%
\makeatletter%
\begin{pgfpicture}%
\pgfpathrectangle{\pgfpointorigin}{\pgfqpoint{8.000000in}{6.000000in}}%
\pgfusepath{use as bounding box, clip}%
\begin{pgfscope}%
\pgfsetbuttcap%
\pgfsetmiterjoin%
\definecolor{currentfill}{rgb}{1.000000,1.000000,1.000000}%
\pgfsetfillcolor{currentfill}%
\pgfsetlinewidth{0.000000pt}%
\definecolor{currentstroke}{rgb}{1.000000,1.000000,1.000000}%
\pgfsetstrokecolor{currentstroke}%
\pgfsetdash{}{0pt}%
\pgfpathmoveto{\pgfqpoint{0.000000in}{0.000000in}}%
\pgfpathlineto{\pgfqpoint{8.000000in}{0.000000in}}%
\pgfpathlineto{\pgfqpoint{8.000000in}{6.000000in}}%
\pgfpathlineto{\pgfqpoint{0.000000in}{6.000000in}}%
\pgfpathclose%
\pgfusepath{fill}%
\end{pgfscope}%
\begin{pgfscope}%
\pgfsetbuttcap%
\pgfsetmiterjoin%
\definecolor{currentfill}{rgb}{1.000000,1.000000,1.000000}%
\pgfsetfillcolor{currentfill}%
\pgfsetlinewidth{0.000000pt}%
\definecolor{currentstroke}{rgb}{0.000000,0.000000,0.000000}%
\pgfsetstrokecolor{currentstroke}%
\pgfsetstrokeopacity{0.000000}%
\pgfsetdash{}{0pt}%
\pgfpathmoveto{\pgfqpoint{0.584722in}{3.312222in}}%
\pgfpathlineto{\pgfqpoint{3.890000in}{3.312222in}}%
\pgfpathlineto{\pgfqpoint{3.890000in}{5.636667in}}%
\pgfpathlineto{\pgfqpoint{0.584722in}{5.636667in}}%
\pgfpathclose%
\pgfusepath{fill}%
\end{pgfscope}%
\begin{pgfscope}%
\pgfpathrectangle{\pgfqpoint{0.584722in}{3.312222in}}{\pgfqpoint{3.305278in}{2.324444in}}%
\pgfusepath{clip}%
\pgfsetbuttcap%
\pgfsetroundjoin%
\definecolor{currentfill}{rgb}{0.121569,0.466667,0.705882}%
\pgfsetfillcolor{currentfill}%
\pgfsetfillopacity{0.300000}%
\pgfsetlinewidth{1.003750pt}%
\definecolor{currentstroke}{rgb}{0.121569,0.466667,0.705882}%
\pgfsetstrokecolor{currentstroke}%
\pgfsetstrokeopacity{0.300000}%
\pgfsetdash{}{0pt}%
\pgfpathmoveto{\pgfqpoint{0.743210in}{3.425679in}}%
\pgfpathlineto{\pgfqpoint{0.743210in}{3.425618in}}%
\pgfpathlineto{\pgfqpoint{0.817917in}{3.449271in}}%
\pgfpathlineto{\pgfqpoint{0.892625in}{3.476415in}}%
\pgfpathlineto{\pgfqpoint{0.967333in}{3.507371in}}%
\pgfpathlineto{\pgfqpoint{1.042040in}{3.542360in}}%
\pgfpathlineto{\pgfqpoint{1.116748in}{3.581708in}}%
\pgfpathlineto{\pgfqpoint{1.191455in}{3.625484in}}%
\pgfpathlineto{\pgfqpoint{1.266163in}{3.674123in}}%
\pgfpathlineto{\pgfqpoint{1.340870in}{3.728095in}}%
\pgfpathlineto{\pgfqpoint{1.415578in}{3.787534in}}%
\pgfpathlineto{\pgfqpoint{1.490285in}{3.852550in}}%
\pgfpathlineto{\pgfqpoint{1.564993in}{3.924281in}}%
\pgfpathlineto{\pgfqpoint{1.639701in}{4.003021in}}%
\pgfpathlineto{\pgfqpoint{1.714408in}{4.088996in}}%
\pgfpathlineto{\pgfqpoint{1.789116in}{4.183479in}}%
\pgfpathlineto{\pgfqpoint{1.863823in}{4.287630in}}%
\pgfpathlineto{\pgfqpoint{1.938531in}{4.403036in}}%
\pgfpathlineto{\pgfqpoint{2.013238in}{4.533961in}}%
\pgfpathlineto{\pgfqpoint{2.087946in}{4.681055in}}%
\pgfpathlineto{\pgfqpoint{2.162654in}{4.838938in}}%
\pgfpathlineto{\pgfqpoint{2.237361in}{4.982961in}}%
\pgfpathlineto{\pgfqpoint{2.312069in}{5.092421in}}%
\pgfpathlineto{\pgfqpoint{2.386776in}{5.168819in}}%
\pgfpathlineto{\pgfqpoint{2.461484in}{5.222722in}}%
\pgfpathlineto{\pgfqpoint{2.536191in}{5.264460in}}%
\pgfpathlineto{\pgfqpoint{2.610899in}{5.296627in}}%
\pgfpathlineto{\pgfqpoint{2.685606in}{5.323828in}}%
\pgfpathlineto{\pgfqpoint{2.760314in}{5.346990in}}%
\pgfpathlineto{\pgfqpoint{2.835022in}{5.367894in}}%
\pgfpathlineto{\pgfqpoint{2.909729in}{5.386338in}}%
\pgfpathlineto{\pgfqpoint{2.984437in}{5.403744in}}%
\pgfpathlineto{\pgfqpoint{3.059144in}{5.419176in}}%
\pgfpathlineto{\pgfqpoint{3.133852in}{5.433826in}}%
\pgfpathlineto{\pgfqpoint{3.208559in}{5.447229in}}%
\pgfpathlineto{\pgfqpoint{3.283267in}{5.460138in}}%
\pgfpathlineto{\pgfqpoint{3.357975in}{5.472081in}}%
\pgfpathlineto{\pgfqpoint{3.432682in}{5.483118in}}%
\pgfpathlineto{\pgfqpoint{3.507390in}{5.494160in}}%
\pgfpathlineto{\pgfqpoint{3.582097in}{5.503993in}}%
\pgfpathlineto{\pgfqpoint{3.656805in}{5.513865in}}%
\pgfpathlineto{\pgfqpoint{3.731512in}{5.523061in}}%
\pgfpathlineto{\pgfqpoint{3.731512in}{5.523419in}}%
\pgfpathlineto{\pgfqpoint{3.731512in}{5.523419in}}%
\pgfpathlineto{\pgfqpoint{3.656805in}{5.514210in}}%
\pgfpathlineto{\pgfqpoint{3.582097in}{5.504568in}}%
\pgfpathlineto{\pgfqpoint{3.507390in}{5.494331in}}%
\pgfpathlineto{\pgfqpoint{3.432682in}{5.483649in}}%
\pgfpathlineto{\pgfqpoint{3.357975in}{5.472267in}}%
\pgfpathlineto{\pgfqpoint{3.283267in}{5.460320in}}%
\pgfpathlineto{\pgfqpoint{3.208559in}{5.447868in}}%
\pgfpathlineto{\pgfqpoint{3.133852in}{5.434474in}}%
\pgfpathlineto{\pgfqpoint{3.059144in}{5.419814in}}%
\pgfpathlineto{\pgfqpoint{2.984437in}{5.404138in}}%
\pgfpathlineto{\pgfqpoint{2.909729in}{5.386861in}}%
\pgfpathlineto{\pgfqpoint{2.835022in}{5.368321in}}%
\pgfpathlineto{\pgfqpoint{2.760314in}{5.347415in}}%
\pgfpathlineto{\pgfqpoint{2.685606in}{5.324722in}}%
\pgfpathlineto{\pgfqpoint{2.610899in}{5.297491in}}%
\pgfpathlineto{\pgfqpoint{2.536191in}{5.265331in}}%
\pgfpathlineto{\pgfqpoint{2.461484in}{5.223605in}}%
\pgfpathlineto{\pgfqpoint{2.386776in}{5.170419in}}%
\pgfpathlineto{\pgfqpoint{2.312069in}{5.094697in}}%
\pgfpathlineto{\pgfqpoint{2.237361in}{4.985651in}}%
\pgfpathlineto{\pgfqpoint{2.162654in}{4.840818in}}%
\pgfpathlineto{\pgfqpoint{2.087946in}{4.684919in}}%
\pgfpathlineto{\pgfqpoint{2.013238in}{4.534313in}}%
\pgfpathlineto{\pgfqpoint{1.938531in}{4.404778in}}%
\pgfpathlineto{\pgfqpoint{1.863823in}{4.289431in}}%
\pgfpathlineto{\pgfqpoint{1.789116in}{4.184464in}}%
\pgfpathlineto{\pgfqpoint{1.714408in}{4.089863in}}%
\pgfpathlineto{\pgfqpoint{1.639701in}{4.003566in}}%
\pgfpathlineto{\pgfqpoint{1.564993in}{3.924835in}}%
\pgfpathlineto{\pgfqpoint{1.490285in}{3.853111in}}%
\pgfpathlineto{\pgfqpoint{1.415578in}{3.787838in}}%
\pgfpathlineto{\pgfqpoint{1.340870in}{3.728321in}}%
\pgfpathlineto{\pgfqpoint{1.266163in}{3.674400in}}%
\pgfpathlineto{\pgfqpoint{1.191455in}{3.625605in}}%
\pgfpathlineto{\pgfqpoint{1.116748in}{3.581836in}}%
\pgfpathlineto{\pgfqpoint{1.042040in}{3.542548in}}%
\pgfpathlineto{\pgfqpoint{0.967333in}{3.507538in}}%
\pgfpathlineto{\pgfqpoint{0.892625in}{3.476524in}}%
\pgfpathlineto{\pgfqpoint{0.817917in}{3.449319in}}%
\pgfpathlineto{\pgfqpoint{0.743210in}{3.425679in}}%
\pgfpathclose%
\pgfusepath{stroke,fill}%
\end{pgfscope}%
\begin{pgfscope}%
\pgfpathrectangle{\pgfqpoint{0.584722in}{3.312222in}}{\pgfqpoint{3.305278in}{2.324444in}}%
\pgfusepath{clip}%
\pgfsetbuttcap%
\pgfsetroundjoin%
\definecolor{currentfill}{rgb}{0.121569,0.466667,0.705882}%
\pgfsetfillcolor{currentfill}%
\pgfsetlinewidth{1.003750pt}%
\definecolor{currentstroke}{rgb}{0.121569,0.466667,0.705882}%
\pgfsetstrokecolor{currentstroke}%
\pgfsetdash{}{0pt}%
\pgfsys@defobject{currentmarker}{\pgfqpoint{-0.009821in}{-0.009821in}}{\pgfqpoint{0.009821in}{0.009821in}}{%
\pgfpathmoveto{\pgfqpoint{0.000000in}{-0.009821in}}%
\pgfpathcurveto{\pgfqpoint{0.002605in}{-0.009821in}}{\pgfqpoint{0.005103in}{-0.008786in}}{\pgfqpoint{0.006944in}{-0.006944in}}%
\pgfpathcurveto{\pgfqpoint{0.008786in}{-0.005103in}}{\pgfqpoint{0.009821in}{-0.002605in}}{\pgfqpoint{0.009821in}{0.000000in}}%
\pgfpathcurveto{\pgfqpoint{0.009821in}{0.002605in}}{\pgfqpoint{0.008786in}{0.005103in}}{\pgfqpoint{0.006944in}{0.006944in}}%
\pgfpathcurveto{\pgfqpoint{0.005103in}{0.008786in}}{\pgfqpoint{0.002605in}{0.009821in}}{\pgfqpoint{0.000000in}{0.009821in}}%
\pgfpathcurveto{\pgfqpoint{-0.002605in}{0.009821in}}{\pgfqpoint{-0.005103in}{0.008786in}}{\pgfqpoint{-0.006944in}{0.006944in}}%
\pgfpathcurveto{\pgfqpoint{-0.008786in}{0.005103in}}{\pgfqpoint{-0.009821in}{0.002605in}}{\pgfqpoint{-0.009821in}{0.000000in}}%
\pgfpathcurveto{\pgfqpoint{-0.009821in}{-0.002605in}}{\pgfqpoint{-0.008786in}{-0.005103in}}{\pgfqpoint{-0.006944in}{-0.006944in}}%
\pgfpathcurveto{\pgfqpoint{-0.005103in}{-0.008786in}}{\pgfqpoint{-0.002605in}{-0.009821in}}{\pgfqpoint{0.000000in}{-0.009821in}}%
\pgfpathclose%
\pgfusepath{stroke,fill}%
}%
\begin{pgfscope}%
\pgfsys@transformshift{0.743210in}{3.425649in}%
\pgfsys@useobject{currentmarker}{}%
\end{pgfscope}%
\begin{pgfscope}%
\pgfsys@transformshift{0.817917in}{3.449295in}%
\pgfsys@useobject{currentmarker}{}%
\end{pgfscope}%
\begin{pgfscope}%
\pgfsys@transformshift{0.892625in}{3.476470in}%
\pgfsys@useobject{currentmarker}{}%
\end{pgfscope}%
\begin{pgfscope}%
\pgfsys@transformshift{0.967333in}{3.507455in}%
\pgfsys@useobject{currentmarker}{}%
\end{pgfscope}%
\begin{pgfscope}%
\pgfsys@transformshift{1.042040in}{3.542454in}%
\pgfsys@useobject{currentmarker}{}%
\end{pgfscope}%
\begin{pgfscope}%
\pgfsys@transformshift{1.116748in}{3.581772in}%
\pgfsys@useobject{currentmarker}{}%
\end{pgfscope}%
\begin{pgfscope}%
\pgfsys@transformshift{1.191455in}{3.625544in}%
\pgfsys@useobject{currentmarker}{}%
\end{pgfscope}%
\begin{pgfscope}%
\pgfsys@transformshift{1.266163in}{3.674261in}%
\pgfsys@useobject{currentmarker}{}%
\end{pgfscope}%
\begin{pgfscope}%
\pgfsys@transformshift{1.340870in}{3.728208in}%
\pgfsys@useobject{currentmarker}{}%
\end{pgfscope}%
\begin{pgfscope}%
\pgfsys@transformshift{1.415578in}{3.787686in}%
\pgfsys@useobject{currentmarker}{}%
\end{pgfscope}%
\begin{pgfscope}%
\pgfsys@transformshift{1.490285in}{3.852831in}%
\pgfsys@useobject{currentmarker}{}%
\end{pgfscope}%
\begin{pgfscope}%
\pgfsys@transformshift{1.564993in}{3.924558in}%
\pgfsys@useobject{currentmarker}{}%
\end{pgfscope}%
\begin{pgfscope}%
\pgfsys@transformshift{1.639701in}{4.003293in}%
\pgfsys@useobject{currentmarker}{}%
\end{pgfscope}%
\begin{pgfscope}%
\pgfsys@transformshift{1.714408in}{4.089430in}%
\pgfsys@useobject{currentmarker}{}%
\end{pgfscope}%
\begin{pgfscope}%
\pgfsys@transformshift{1.789116in}{4.183972in}%
\pgfsys@useobject{currentmarker}{}%
\end{pgfscope}%
\begin{pgfscope}%
\pgfsys@transformshift{1.863823in}{4.288530in}%
\pgfsys@useobject{currentmarker}{}%
\end{pgfscope}%
\begin{pgfscope}%
\pgfsys@transformshift{1.938531in}{4.403907in}%
\pgfsys@useobject{currentmarker}{}%
\end{pgfscope}%
\begin{pgfscope}%
\pgfsys@transformshift{2.013238in}{4.534137in}%
\pgfsys@useobject{currentmarker}{}%
\end{pgfscope}%
\begin{pgfscope}%
\pgfsys@transformshift{2.087946in}{4.682987in}%
\pgfsys@useobject{currentmarker}{}%
\end{pgfscope}%
\begin{pgfscope}%
\pgfsys@transformshift{2.162654in}{4.839878in}%
\pgfsys@useobject{currentmarker}{}%
\end{pgfscope}%
\begin{pgfscope}%
\pgfsys@transformshift{2.237361in}{4.984306in}%
\pgfsys@useobject{currentmarker}{}%
\end{pgfscope}%
\begin{pgfscope}%
\pgfsys@transformshift{2.312069in}{5.093559in}%
\pgfsys@useobject{currentmarker}{}%
\end{pgfscope}%
\begin{pgfscope}%
\pgfsys@transformshift{2.386776in}{5.169619in}%
\pgfsys@useobject{currentmarker}{}%
\end{pgfscope}%
\begin{pgfscope}%
\pgfsys@transformshift{2.461484in}{5.223164in}%
\pgfsys@useobject{currentmarker}{}%
\end{pgfscope}%
\begin{pgfscope}%
\pgfsys@transformshift{2.536191in}{5.264895in}%
\pgfsys@useobject{currentmarker}{}%
\end{pgfscope}%
\begin{pgfscope}%
\pgfsys@transformshift{2.610899in}{5.297059in}%
\pgfsys@useobject{currentmarker}{}%
\end{pgfscope}%
\begin{pgfscope}%
\pgfsys@transformshift{2.685606in}{5.324275in}%
\pgfsys@useobject{currentmarker}{}%
\end{pgfscope}%
\begin{pgfscope}%
\pgfsys@transformshift{2.760314in}{5.347202in}%
\pgfsys@useobject{currentmarker}{}%
\end{pgfscope}%
\begin{pgfscope}%
\pgfsys@transformshift{2.835022in}{5.368107in}%
\pgfsys@useobject{currentmarker}{}%
\end{pgfscope}%
\begin{pgfscope}%
\pgfsys@transformshift{2.909729in}{5.386600in}%
\pgfsys@useobject{currentmarker}{}%
\end{pgfscope}%
\begin{pgfscope}%
\pgfsys@transformshift{2.984437in}{5.403941in}%
\pgfsys@useobject{currentmarker}{}%
\end{pgfscope}%
\begin{pgfscope}%
\pgfsys@transformshift{3.059144in}{5.419495in}%
\pgfsys@useobject{currentmarker}{}%
\end{pgfscope}%
\begin{pgfscope}%
\pgfsys@transformshift{3.133852in}{5.434150in}%
\pgfsys@useobject{currentmarker}{}%
\end{pgfscope}%
\begin{pgfscope}%
\pgfsys@transformshift{3.208559in}{5.447548in}%
\pgfsys@useobject{currentmarker}{}%
\end{pgfscope}%
\begin{pgfscope}%
\pgfsys@transformshift{3.283267in}{5.460229in}%
\pgfsys@useobject{currentmarker}{}%
\end{pgfscope}%
\begin{pgfscope}%
\pgfsys@transformshift{3.357975in}{5.472174in}%
\pgfsys@useobject{currentmarker}{}%
\end{pgfscope}%
\begin{pgfscope}%
\pgfsys@transformshift{3.432682in}{5.483384in}%
\pgfsys@useobject{currentmarker}{}%
\end{pgfscope}%
\begin{pgfscope}%
\pgfsys@transformshift{3.507390in}{5.494246in}%
\pgfsys@useobject{currentmarker}{}%
\end{pgfscope}%
\begin{pgfscope}%
\pgfsys@transformshift{3.582097in}{5.504280in}%
\pgfsys@useobject{currentmarker}{}%
\end{pgfscope}%
\begin{pgfscope}%
\pgfsys@transformshift{3.656805in}{5.514038in}%
\pgfsys@useobject{currentmarker}{}%
\end{pgfscope}%
\begin{pgfscope}%
\pgfsys@transformshift{3.731512in}{5.523240in}%
\pgfsys@useobject{currentmarker}{}%
\end{pgfscope}%
\end{pgfscope}%
\begin{pgfscope}%
\pgfsetbuttcap%
\pgfsetroundjoin%
\definecolor{currentfill}{rgb}{0.000000,0.000000,0.000000}%
\pgfsetfillcolor{currentfill}%
\pgfsetlinewidth{0.803000pt}%
\definecolor{currentstroke}{rgb}{0.000000,0.000000,0.000000}%
\pgfsetstrokecolor{currentstroke}%
\pgfsetdash{}{0pt}%
\pgfsys@defobject{currentmarker}{\pgfqpoint{0.000000in}{-0.048611in}}{\pgfqpoint{0.000000in}{0.000000in}}{%
\pgfpathmoveto{\pgfqpoint{0.000000in}{0.000000in}}%
\pgfpathlineto{\pgfqpoint{0.000000in}{-0.048611in}}%
\pgfusepath{stroke,fill}%
}%
\begin{pgfscope}%
\pgfsys@transformshift{1.116748in}{3.312222in}%
\pgfsys@useobject{currentmarker}{}%
\end{pgfscope}%
\end{pgfscope}%
\begin{pgfscope}%
\definecolor{textcolor}{rgb}{0.000000,0.000000,0.000000}%
\pgfsetstrokecolor{textcolor}%
\pgfsetfillcolor{textcolor}%
\pgftext[x=1.116748in,y=3.215000in,,top]{\color{textcolor}\sffamily\fontsize{10.000000}{12.000000}\selectfont 3}%
\end{pgfscope}%
\begin{pgfscope}%
\pgfsetbuttcap%
\pgfsetroundjoin%
\definecolor{currentfill}{rgb}{0.000000,0.000000,0.000000}%
\pgfsetfillcolor{currentfill}%
\pgfsetlinewidth{0.803000pt}%
\definecolor{currentstroke}{rgb}{0.000000,0.000000,0.000000}%
\pgfsetstrokecolor{currentstroke}%
\pgfsetdash{}{0pt}%
\pgfsys@defobject{currentmarker}{\pgfqpoint{0.000000in}{-0.048611in}}{\pgfqpoint{0.000000in}{0.000000in}}{%
\pgfpathmoveto{\pgfqpoint{0.000000in}{0.000000in}}%
\pgfpathlineto{\pgfqpoint{0.000000in}{-0.048611in}}%
\pgfusepath{stroke,fill}%
}%
\begin{pgfscope}%
\pgfsys@transformshift{1.863823in}{3.312222in}%
\pgfsys@useobject{currentmarker}{}%
\end{pgfscope}%
\end{pgfscope}%
\begin{pgfscope}%
\definecolor{textcolor}{rgb}{0.000000,0.000000,0.000000}%
\pgfsetstrokecolor{textcolor}%
\pgfsetfillcolor{textcolor}%
\pgftext[x=1.863823in,y=3.215000in,,top]{\color{textcolor}\sffamily\fontsize{10.000000}{12.000000}\selectfont 4}%
\end{pgfscope}%
\begin{pgfscope}%
\pgfsetbuttcap%
\pgfsetroundjoin%
\definecolor{currentfill}{rgb}{0.000000,0.000000,0.000000}%
\pgfsetfillcolor{currentfill}%
\pgfsetlinewidth{0.803000pt}%
\definecolor{currentstroke}{rgb}{0.000000,0.000000,0.000000}%
\pgfsetstrokecolor{currentstroke}%
\pgfsetdash{}{0pt}%
\pgfsys@defobject{currentmarker}{\pgfqpoint{0.000000in}{-0.048611in}}{\pgfqpoint{0.000000in}{0.000000in}}{%
\pgfpathmoveto{\pgfqpoint{0.000000in}{0.000000in}}%
\pgfpathlineto{\pgfqpoint{0.000000in}{-0.048611in}}%
\pgfusepath{stroke,fill}%
}%
\begin{pgfscope}%
\pgfsys@transformshift{2.610899in}{3.312222in}%
\pgfsys@useobject{currentmarker}{}%
\end{pgfscope}%
\end{pgfscope}%
\begin{pgfscope}%
\definecolor{textcolor}{rgb}{0.000000,0.000000,0.000000}%
\pgfsetstrokecolor{textcolor}%
\pgfsetfillcolor{textcolor}%
\pgftext[x=2.610899in,y=3.215000in,,top]{\color{textcolor}\sffamily\fontsize{10.000000}{12.000000}\selectfont 5}%
\end{pgfscope}%
\begin{pgfscope}%
\pgfsetbuttcap%
\pgfsetroundjoin%
\definecolor{currentfill}{rgb}{0.000000,0.000000,0.000000}%
\pgfsetfillcolor{currentfill}%
\pgfsetlinewidth{0.803000pt}%
\definecolor{currentstroke}{rgb}{0.000000,0.000000,0.000000}%
\pgfsetstrokecolor{currentstroke}%
\pgfsetdash{}{0pt}%
\pgfsys@defobject{currentmarker}{\pgfqpoint{0.000000in}{-0.048611in}}{\pgfqpoint{0.000000in}{0.000000in}}{%
\pgfpathmoveto{\pgfqpoint{0.000000in}{0.000000in}}%
\pgfpathlineto{\pgfqpoint{0.000000in}{-0.048611in}}%
\pgfusepath{stroke,fill}%
}%
\begin{pgfscope}%
\pgfsys@transformshift{3.357975in}{3.312222in}%
\pgfsys@useobject{currentmarker}{}%
\end{pgfscope}%
\end{pgfscope}%
\begin{pgfscope}%
\definecolor{textcolor}{rgb}{0.000000,0.000000,0.000000}%
\pgfsetstrokecolor{textcolor}%
\pgfsetfillcolor{textcolor}%
\pgftext[x=3.357975in,y=3.215000in,,top]{\color{textcolor}\sffamily\fontsize{10.000000}{12.000000}\selectfont 6}%
\end{pgfscope}%
\begin{pgfscope}%
\pgfsetbuttcap%
\pgfsetroundjoin%
\definecolor{currentfill}{rgb}{0.000000,0.000000,0.000000}%
\pgfsetfillcolor{currentfill}%
\pgfsetlinewidth{0.803000pt}%
\definecolor{currentstroke}{rgb}{0.000000,0.000000,0.000000}%
\pgfsetstrokecolor{currentstroke}%
\pgfsetdash{}{0pt}%
\pgfsys@defobject{currentmarker}{\pgfqpoint{-0.048611in}{0.000000in}}{\pgfqpoint{0.000000in}{0.000000in}}{%
\pgfpathmoveto{\pgfqpoint{0.000000in}{0.000000in}}%
\pgfpathlineto{\pgfqpoint{-0.048611in}{0.000000in}}%
\pgfusepath{stroke,fill}%
}%
\begin{pgfscope}%
\pgfsys@transformshift{0.584722in}{3.320411in}%
\pgfsys@useobject{currentmarker}{}%
\end{pgfscope}%
\end{pgfscope}%
\begin{pgfscope}%
\definecolor{textcolor}{rgb}{0.000000,0.000000,0.000000}%
\pgfsetstrokecolor{textcolor}%
\pgfsetfillcolor{textcolor}%
\pgftext[x=0.150247in,y=3.267650in,left,base]{\color{textcolor}\sffamily\fontsize{10.000000}{12.000000}\selectfont −3.0}%
\end{pgfscope}%
\begin{pgfscope}%
\pgfsetbuttcap%
\pgfsetroundjoin%
\definecolor{currentfill}{rgb}{0.000000,0.000000,0.000000}%
\pgfsetfillcolor{currentfill}%
\pgfsetlinewidth{0.803000pt}%
\definecolor{currentstroke}{rgb}{0.000000,0.000000,0.000000}%
\pgfsetstrokecolor{currentstroke}%
\pgfsetdash{}{0pt}%
\pgfsys@defobject{currentmarker}{\pgfqpoint{-0.048611in}{0.000000in}}{\pgfqpoint{0.000000in}{0.000000in}}{%
\pgfpathmoveto{\pgfqpoint{0.000000in}{0.000000in}}%
\pgfpathlineto{\pgfqpoint{-0.048611in}{0.000000in}}%
\pgfusepath{stroke,fill}%
}%
\begin{pgfscope}%
\pgfsys@transformshift{0.584722in}{3.763551in}%
\pgfsys@useobject{currentmarker}{}%
\end{pgfscope}%
\end{pgfscope}%
\begin{pgfscope}%
\definecolor{textcolor}{rgb}{0.000000,0.000000,0.000000}%
\pgfsetstrokecolor{textcolor}%
\pgfsetfillcolor{textcolor}%
\pgftext[x=0.150247in,y=3.710790in,left,base]{\color{textcolor}\sffamily\fontsize{10.000000}{12.000000}\selectfont −2.5}%
\end{pgfscope}%
\begin{pgfscope}%
\pgfsetbuttcap%
\pgfsetroundjoin%
\definecolor{currentfill}{rgb}{0.000000,0.000000,0.000000}%
\pgfsetfillcolor{currentfill}%
\pgfsetlinewidth{0.803000pt}%
\definecolor{currentstroke}{rgb}{0.000000,0.000000,0.000000}%
\pgfsetstrokecolor{currentstroke}%
\pgfsetdash{}{0pt}%
\pgfsys@defobject{currentmarker}{\pgfqpoint{-0.048611in}{0.000000in}}{\pgfqpoint{0.000000in}{0.000000in}}{%
\pgfpathmoveto{\pgfqpoint{0.000000in}{0.000000in}}%
\pgfpathlineto{\pgfqpoint{-0.048611in}{0.000000in}}%
\pgfusepath{stroke,fill}%
}%
\begin{pgfscope}%
\pgfsys@transformshift{0.584722in}{4.206692in}%
\pgfsys@useobject{currentmarker}{}%
\end{pgfscope}%
\end{pgfscope}%
\begin{pgfscope}%
\definecolor{textcolor}{rgb}{0.000000,0.000000,0.000000}%
\pgfsetstrokecolor{textcolor}%
\pgfsetfillcolor{textcolor}%
\pgftext[x=0.150247in,y=4.153930in,left,base]{\color{textcolor}\sffamily\fontsize{10.000000}{12.000000}\selectfont −2.0}%
\end{pgfscope}%
\begin{pgfscope}%
\pgfsetbuttcap%
\pgfsetroundjoin%
\definecolor{currentfill}{rgb}{0.000000,0.000000,0.000000}%
\pgfsetfillcolor{currentfill}%
\pgfsetlinewidth{0.803000pt}%
\definecolor{currentstroke}{rgb}{0.000000,0.000000,0.000000}%
\pgfsetstrokecolor{currentstroke}%
\pgfsetdash{}{0pt}%
\pgfsys@defobject{currentmarker}{\pgfqpoint{-0.048611in}{0.000000in}}{\pgfqpoint{0.000000in}{0.000000in}}{%
\pgfpathmoveto{\pgfqpoint{0.000000in}{0.000000in}}%
\pgfpathlineto{\pgfqpoint{-0.048611in}{0.000000in}}%
\pgfusepath{stroke,fill}%
}%
\begin{pgfscope}%
\pgfsys@transformshift{0.584722in}{4.649832in}%
\pgfsys@useobject{currentmarker}{}%
\end{pgfscope}%
\end{pgfscope}%
\begin{pgfscope}%
\definecolor{textcolor}{rgb}{0.000000,0.000000,0.000000}%
\pgfsetstrokecolor{textcolor}%
\pgfsetfillcolor{textcolor}%
\pgftext[x=0.150247in,y=4.597071in,left,base]{\color{textcolor}\sffamily\fontsize{10.000000}{12.000000}\selectfont −1.5}%
\end{pgfscope}%
\begin{pgfscope}%
\pgfsetbuttcap%
\pgfsetroundjoin%
\definecolor{currentfill}{rgb}{0.000000,0.000000,0.000000}%
\pgfsetfillcolor{currentfill}%
\pgfsetlinewidth{0.803000pt}%
\definecolor{currentstroke}{rgb}{0.000000,0.000000,0.000000}%
\pgfsetstrokecolor{currentstroke}%
\pgfsetdash{}{0pt}%
\pgfsys@defobject{currentmarker}{\pgfqpoint{-0.048611in}{0.000000in}}{\pgfqpoint{0.000000in}{0.000000in}}{%
\pgfpathmoveto{\pgfqpoint{0.000000in}{0.000000in}}%
\pgfpathlineto{\pgfqpoint{-0.048611in}{0.000000in}}%
\pgfusepath{stroke,fill}%
}%
\begin{pgfscope}%
\pgfsys@transformshift{0.584722in}{5.092972in}%
\pgfsys@useobject{currentmarker}{}%
\end{pgfscope}%
\end{pgfscope}%
\begin{pgfscope}%
\definecolor{textcolor}{rgb}{0.000000,0.000000,0.000000}%
\pgfsetstrokecolor{textcolor}%
\pgfsetfillcolor{textcolor}%
\pgftext[x=0.150247in,y=5.040211in,left,base]{\color{textcolor}\sffamily\fontsize{10.000000}{12.000000}\selectfont −1.0}%
\end{pgfscope}%
\begin{pgfscope}%
\pgfsetbuttcap%
\pgfsetroundjoin%
\definecolor{currentfill}{rgb}{0.000000,0.000000,0.000000}%
\pgfsetfillcolor{currentfill}%
\pgfsetlinewidth{0.803000pt}%
\definecolor{currentstroke}{rgb}{0.000000,0.000000,0.000000}%
\pgfsetstrokecolor{currentstroke}%
\pgfsetdash{}{0pt}%
\pgfsys@defobject{currentmarker}{\pgfqpoint{-0.048611in}{0.000000in}}{\pgfqpoint{0.000000in}{0.000000in}}{%
\pgfpathmoveto{\pgfqpoint{0.000000in}{0.000000in}}%
\pgfpathlineto{\pgfqpoint{-0.048611in}{0.000000in}}%
\pgfusepath{stroke,fill}%
}%
\begin{pgfscope}%
\pgfsys@transformshift{0.584722in}{5.536113in}%
\pgfsys@useobject{currentmarker}{}%
\end{pgfscope}%
\end{pgfscope}%
\begin{pgfscope}%
\definecolor{textcolor}{rgb}{0.000000,0.000000,0.000000}%
\pgfsetstrokecolor{textcolor}%
\pgfsetfillcolor{textcolor}%
\pgftext[x=0.150247in,y=5.483351in,left,base]{\color{textcolor}\sffamily\fontsize{10.000000}{12.000000}\selectfont −0.5}%
\end{pgfscope}%
\begin{pgfscope}%
\pgfpathrectangle{\pgfqpoint{0.584722in}{3.312222in}}{\pgfqpoint{3.305278in}{2.324444in}}%
\pgfusepath{clip}%
\pgfsetrectcap%
\pgfsetroundjoin%
\pgfsetlinewidth{1.505625pt}%
\definecolor{currentstroke}{rgb}{0.121569,0.466667,0.705882}%
\pgfsetstrokecolor{currentstroke}%
\pgfsetdash{}{0pt}%
\pgfpathmoveto{\pgfqpoint{0.743210in}{3.425649in}}%
\pgfpathlineto{\pgfqpoint{0.817917in}{3.449295in}}%
\pgfpathlineto{\pgfqpoint{0.892625in}{3.476470in}}%
\pgfpathlineto{\pgfqpoint{0.967333in}{3.507455in}}%
\pgfpathlineto{\pgfqpoint{1.042040in}{3.542454in}}%
\pgfpathlineto{\pgfqpoint{1.116748in}{3.581772in}}%
\pgfpathlineto{\pgfqpoint{1.191455in}{3.625544in}}%
\pgfpathlineto{\pgfqpoint{1.266163in}{3.674261in}}%
\pgfpathlineto{\pgfqpoint{1.340870in}{3.728208in}}%
\pgfpathlineto{\pgfqpoint{1.415578in}{3.787686in}}%
\pgfpathlineto{\pgfqpoint{1.490285in}{3.852831in}}%
\pgfpathlineto{\pgfqpoint{1.564993in}{3.924558in}}%
\pgfpathlineto{\pgfqpoint{1.639701in}{4.003293in}}%
\pgfpathlineto{\pgfqpoint{1.714408in}{4.089430in}}%
\pgfpathlineto{\pgfqpoint{1.789116in}{4.183972in}}%
\pgfpathlineto{\pgfqpoint{1.863823in}{4.288530in}}%
\pgfpathlineto{\pgfqpoint{1.938531in}{4.403907in}}%
\pgfpathlineto{\pgfqpoint{2.013238in}{4.534137in}}%
\pgfpathlineto{\pgfqpoint{2.087946in}{4.682987in}}%
\pgfpathlineto{\pgfqpoint{2.162654in}{4.839878in}}%
\pgfpathlineto{\pgfqpoint{2.237361in}{4.984306in}}%
\pgfpathlineto{\pgfqpoint{2.312069in}{5.093559in}}%
\pgfpathlineto{\pgfqpoint{2.386776in}{5.169619in}}%
\pgfpathlineto{\pgfqpoint{2.461484in}{5.223164in}}%
\pgfpathlineto{\pgfqpoint{2.536191in}{5.264895in}}%
\pgfpathlineto{\pgfqpoint{2.610899in}{5.297059in}}%
\pgfpathlineto{\pgfqpoint{2.685606in}{5.324275in}}%
\pgfpathlineto{\pgfqpoint{2.760314in}{5.347202in}}%
\pgfpathlineto{\pgfqpoint{2.835022in}{5.368107in}}%
\pgfpathlineto{\pgfqpoint{2.909729in}{5.386600in}}%
\pgfpathlineto{\pgfqpoint{2.984437in}{5.403941in}}%
\pgfpathlineto{\pgfqpoint{3.059144in}{5.419495in}}%
\pgfpathlineto{\pgfqpoint{3.133852in}{5.434150in}}%
\pgfpathlineto{\pgfqpoint{3.208559in}{5.447548in}}%
\pgfpathlineto{\pgfqpoint{3.283267in}{5.460229in}}%
\pgfpathlineto{\pgfqpoint{3.357975in}{5.472174in}}%
\pgfpathlineto{\pgfqpoint{3.432682in}{5.483384in}}%
\pgfpathlineto{\pgfqpoint{3.507390in}{5.494246in}}%
\pgfpathlineto{\pgfqpoint{3.582097in}{5.504280in}}%
\pgfpathlineto{\pgfqpoint{3.656805in}{5.514038in}}%
\pgfpathlineto{\pgfqpoint{3.731512in}{5.523240in}}%
\pgfusepath{stroke}%
\end{pgfscope}%
\begin{pgfscope}%
\pgfsetrectcap%
\pgfsetmiterjoin%
\pgfsetlinewidth{0.803000pt}%
\definecolor{currentstroke}{rgb}{0.000000,0.000000,0.000000}%
\pgfsetstrokecolor{currentstroke}%
\pgfsetdash{}{0pt}%
\pgfpathmoveto{\pgfqpoint{0.584722in}{3.312222in}}%
\pgfpathlineto{\pgfqpoint{0.584722in}{5.636667in}}%
\pgfusepath{stroke}%
\end{pgfscope}%
\begin{pgfscope}%
\pgfsetrectcap%
\pgfsetmiterjoin%
\pgfsetlinewidth{0.803000pt}%
\definecolor{currentstroke}{rgb}{0.000000,0.000000,0.000000}%
\pgfsetstrokecolor{currentstroke}%
\pgfsetdash{}{0pt}%
\pgfpathmoveto{\pgfqpoint{3.890000in}{3.312222in}}%
\pgfpathlineto{\pgfqpoint{3.890000in}{5.636667in}}%
\pgfusepath{stroke}%
\end{pgfscope}%
\begin{pgfscope}%
\pgfsetrectcap%
\pgfsetmiterjoin%
\pgfsetlinewidth{0.803000pt}%
\definecolor{currentstroke}{rgb}{0.000000,0.000000,0.000000}%
\pgfsetstrokecolor{currentstroke}%
\pgfsetdash{}{0pt}%
\pgfpathmoveto{\pgfqpoint{0.584722in}{3.312222in}}%
\pgfpathlineto{\pgfqpoint{3.890000in}{3.312222in}}%
\pgfusepath{stroke}%
\end{pgfscope}%
\begin{pgfscope}%
\pgfsetrectcap%
\pgfsetmiterjoin%
\pgfsetlinewidth{0.803000pt}%
\definecolor{currentstroke}{rgb}{0.000000,0.000000,0.000000}%
\pgfsetstrokecolor{currentstroke}%
\pgfsetdash{}{0pt}%
\pgfpathmoveto{\pgfqpoint{0.584722in}{5.636667in}}%
\pgfpathlineto{\pgfqpoint{3.890000in}{5.636667in}}%
\pgfusepath{stroke}%
\end{pgfscope}%
\begin{pgfscope}%
\definecolor{textcolor}{rgb}{0.000000,0.000000,0.000000}%
\pgfsetstrokecolor{textcolor}%
\pgfsetfillcolor{textcolor}%
\pgftext[x=2.237361in,y=5.720000in,,base]{\color{textcolor}\sffamily\fontsize{12.000000}{14.400000}\selectfont \(\displaystyle  N = 8 \)}%
\end{pgfscope}%
\begin{pgfscope}%
\pgfsetbuttcap%
\pgfsetmiterjoin%
\definecolor{currentfill}{rgb}{1.000000,1.000000,1.000000}%
\pgfsetfillcolor{currentfill}%
\pgfsetlinewidth{0.000000pt}%
\definecolor{currentstroke}{rgb}{0.000000,0.000000,0.000000}%
\pgfsetstrokecolor{currentstroke}%
\pgfsetstrokeopacity{0.000000}%
\pgfsetdash{}{0pt}%
\pgfpathmoveto{\pgfqpoint{4.509722in}{3.312222in}}%
\pgfpathlineto{\pgfqpoint{7.815000in}{3.312222in}}%
\pgfpathlineto{\pgfqpoint{7.815000in}{5.636667in}}%
\pgfpathlineto{\pgfqpoint{4.509722in}{5.636667in}}%
\pgfpathclose%
\pgfusepath{fill}%
\end{pgfscope}%
\begin{pgfscope}%
\pgfpathrectangle{\pgfqpoint{4.509722in}{3.312222in}}{\pgfqpoint{3.305278in}{2.324444in}}%
\pgfusepath{clip}%
\pgfsetbuttcap%
\pgfsetroundjoin%
\definecolor{currentfill}{rgb}{0.121569,0.466667,0.705882}%
\pgfsetfillcolor{currentfill}%
\pgfsetfillopacity{0.300000}%
\pgfsetlinewidth{1.003750pt}%
\definecolor{currentstroke}{rgb}{0.121569,0.466667,0.705882}%
\pgfsetstrokecolor{currentstroke}%
\pgfsetstrokeopacity{0.300000}%
\pgfsetdash{}{0pt}%
\pgfpathmoveto{\pgfqpoint{4.668210in}{3.425681in}}%
\pgfpathlineto{\pgfqpoint{4.668210in}{3.425616in}}%
\pgfpathlineto{\pgfqpoint{4.742917in}{3.449466in}}%
\pgfpathlineto{\pgfqpoint{4.817625in}{3.476736in}}%
\pgfpathlineto{\pgfqpoint{4.892333in}{3.507872in}}%
\pgfpathlineto{\pgfqpoint{4.967040in}{3.542923in}}%
\pgfpathlineto{\pgfqpoint{5.041748in}{3.582215in}}%
\pgfpathlineto{\pgfqpoint{5.116455in}{3.626136in}}%
\pgfpathlineto{\pgfqpoint{5.191163in}{3.674906in}}%
\pgfpathlineto{\pgfqpoint{5.265870in}{3.728686in}}%
\pgfpathlineto{\pgfqpoint{5.340578in}{3.788036in}}%
\pgfpathlineto{\pgfqpoint{5.415285in}{3.853388in}}%
\pgfpathlineto{\pgfqpoint{5.489993in}{3.925212in}}%
\pgfpathlineto{\pgfqpoint{5.564701in}{4.003710in}}%
\pgfpathlineto{\pgfqpoint{5.639408in}{4.089552in}}%
\pgfpathlineto{\pgfqpoint{5.714116in}{4.184058in}}%
\pgfpathlineto{\pgfqpoint{5.788823in}{4.288092in}}%
\pgfpathlineto{\pgfqpoint{5.863531in}{4.402517in}}%
\pgfpathlineto{\pgfqpoint{5.938238in}{4.529630in}}%
\pgfpathlineto{\pgfqpoint{6.012946in}{4.673447in}}%
\pgfpathlineto{\pgfqpoint{6.087654in}{4.845990in}}%
\pgfpathlineto{\pgfqpoint{6.162361in}{5.022508in}}%
\pgfpathlineto{\pgfqpoint{6.237069in}{5.139607in}}%
\pgfpathlineto{\pgfqpoint{6.311776in}{5.203460in}}%
\pgfpathlineto{\pgfqpoint{6.386484in}{5.245901in}}%
\pgfpathlineto{\pgfqpoint{6.461191in}{5.278587in}}%
\pgfpathlineto{\pgfqpoint{6.535899in}{5.306527in}}%
\pgfpathlineto{\pgfqpoint{6.610606in}{5.330408in}}%
\pgfpathlineto{\pgfqpoint{6.685314in}{5.351859in}}%
\pgfpathlineto{\pgfqpoint{6.760022in}{5.370907in}}%
\pgfpathlineto{\pgfqpoint{6.834729in}{5.388967in}}%
\pgfpathlineto{\pgfqpoint{6.909437in}{5.404914in}}%
\pgfpathlineto{\pgfqpoint{6.984144in}{5.420536in}}%
\pgfpathlineto{\pgfqpoint{7.058852in}{5.434714in}}%
\pgfpathlineto{\pgfqpoint{7.133559in}{5.448011in}}%
\pgfpathlineto{\pgfqpoint{7.208267in}{5.460549in}}%
\pgfpathlineto{\pgfqpoint{7.282975in}{5.472346in}}%
\pgfpathlineto{\pgfqpoint{7.357682in}{5.483616in}}%
\pgfpathlineto{\pgfqpoint{7.432390in}{5.494231in}}%
\pgfpathlineto{\pgfqpoint{7.507097in}{5.504324in}}%
\pgfpathlineto{\pgfqpoint{7.581805in}{5.513884in}}%
\pgfpathlineto{\pgfqpoint{7.656512in}{5.523139in}}%
\pgfpathlineto{\pgfqpoint{7.656512in}{5.523342in}}%
\pgfpathlineto{\pgfqpoint{7.656512in}{5.523342in}}%
\pgfpathlineto{\pgfqpoint{7.581805in}{5.514408in}}%
\pgfpathlineto{\pgfqpoint{7.507097in}{5.504710in}}%
\pgfpathlineto{\pgfqpoint{7.432390in}{5.494738in}}%
\pgfpathlineto{\pgfqpoint{7.357682in}{5.483914in}}%
\pgfpathlineto{\pgfqpoint{7.282975in}{5.472510in}}%
\pgfpathlineto{\pgfqpoint{7.208267in}{5.460871in}}%
\pgfpathlineto{\pgfqpoint{7.133559in}{5.448166in}}%
\pgfpathlineto{\pgfqpoint{7.058852in}{5.434856in}}%
\pgfpathlineto{\pgfqpoint{6.984144in}{5.421149in}}%
\pgfpathlineto{\pgfqpoint{6.909437in}{5.405875in}}%
\pgfpathlineto{\pgfqpoint{6.834729in}{5.389099in}}%
\pgfpathlineto{\pgfqpoint{6.760022in}{5.371802in}}%
\pgfpathlineto{\pgfqpoint{6.685314in}{5.352457in}}%
\pgfpathlineto{\pgfqpoint{6.610606in}{5.331209in}}%
\pgfpathlineto{\pgfqpoint{6.535899in}{5.307450in}}%
\pgfpathlineto{\pgfqpoint{6.461191in}{5.280035in}}%
\pgfpathlineto{\pgfqpoint{6.386484in}{5.247122in}}%
\pgfpathlineto{\pgfqpoint{6.311776in}{5.205977in}}%
\pgfpathlineto{\pgfqpoint{6.237069in}{5.145022in}}%
\pgfpathlineto{\pgfqpoint{6.162361in}{5.028016in}}%
\pgfpathlineto{\pgfqpoint{6.087654in}{4.851778in}}%
\pgfpathlineto{\pgfqpoint{6.012946in}{4.677154in}}%
\pgfpathlineto{\pgfqpoint{5.938238in}{4.532192in}}%
\pgfpathlineto{\pgfqpoint{5.863531in}{4.403877in}}%
\pgfpathlineto{\pgfqpoint{5.788823in}{4.289404in}}%
\pgfpathlineto{\pgfqpoint{5.714116in}{4.185142in}}%
\pgfpathlineto{\pgfqpoint{5.639408in}{4.089992in}}%
\pgfpathlineto{\pgfqpoint{5.564701in}{4.004300in}}%
\pgfpathlineto{\pgfqpoint{5.489993in}{3.925640in}}%
\pgfpathlineto{\pgfqpoint{5.415285in}{3.854028in}}%
\pgfpathlineto{\pgfqpoint{5.340578in}{3.788419in}}%
\pgfpathlineto{\pgfqpoint{5.265870in}{3.728942in}}%
\pgfpathlineto{\pgfqpoint{5.191163in}{3.675138in}}%
\pgfpathlineto{\pgfqpoint{5.116455in}{3.626329in}}%
\pgfpathlineto{\pgfqpoint{5.041748in}{3.582433in}}%
\pgfpathlineto{\pgfqpoint{4.967040in}{3.543060in}}%
\pgfpathlineto{\pgfqpoint{4.892333in}{3.507989in}}%
\pgfpathlineto{\pgfqpoint{4.817625in}{3.476869in}}%
\pgfpathlineto{\pgfqpoint{4.742917in}{3.449507in}}%
\pgfpathlineto{\pgfqpoint{4.668210in}{3.425681in}}%
\pgfpathclose%
\pgfusepath{stroke,fill}%
\end{pgfscope}%
\begin{pgfscope}%
\pgfpathrectangle{\pgfqpoint{4.509722in}{3.312222in}}{\pgfqpoint{3.305278in}{2.324444in}}%
\pgfusepath{clip}%
\pgfsetbuttcap%
\pgfsetroundjoin%
\definecolor{currentfill}{rgb}{0.121569,0.466667,0.705882}%
\pgfsetfillcolor{currentfill}%
\pgfsetlinewidth{1.003750pt}%
\definecolor{currentstroke}{rgb}{0.121569,0.466667,0.705882}%
\pgfsetstrokecolor{currentstroke}%
\pgfsetdash{}{0pt}%
\pgfsys@defobject{currentmarker}{\pgfqpoint{-0.009821in}{-0.009821in}}{\pgfqpoint{0.009821in}{0.009821in}}{%
\pgfpathmoveto{\pgfqpoint{0.000000in}{-0.009821in}}%
\pgfpathcurveto{\pgfqpoint{0.002605in}{-0.009821in}}{\pgfqpoint{0.005103in}{-0.008786in}}{\pgfqpoint{0.006944in}{-0.006944in}}%
\pgfpathcurveto{\pgfqpoint{0.008786in}{-0.005103in}}{\pgfqpoint{0.009821in}{-0.002605in}}{\pgfqpoint{0.009821in}{0.000000in}}%
\pgfpathcurveto{\pgfqpoint{0.009821in}{0.002605in}}{\pgfqpoint{0.008786in}{0.005103in}}{\pgfqpoint{0.006944in}{0.006944in}}%
\pgfpathcurveto{\pgfqpoint{0.005103in}{0.008786in}}{\pgfqpoint{0.002605in}{0.009821in}}{\pgfqpoint{0.000000in}{0.009821in}}%
\pgfpathcurveto{\pgfqpoint{-0.002605in}{0.009821in}}{\pgfqpoint{-0.005103in}{0.008786in}}{\pgfqpoint{-0.006944in}{0.006944in}}%
\pgfpathcurveto{\pgfqpoint{-0.008786in}{0.005103in}}{\pgfqpoint{-0.009821in}{0.002605in}}{\pgfqpoint{-0.009821in}{0.000000in}}%
\pgfpathcurveto{\pgfqpoint{-0.009821in}{-0.002605in}}{\pgfqpoint{-0.008786in}{-0.005103in}}{\pgfqpoint{-0.006944in}{-0.006944in}}%
\pgfpathcurveto{\pgfqpoint{-0.005103in}{-0.008786in}}{\pgfqpoint{-0.002605in}{-0.009821in}}{\pgfqpoint{0.000000in}{-0.009821in}}%
\pgfpathclose%
\pgfusepath{stroke,fill}%
}%
\begin{pgfscope}%
\pgfsys@transformshift{4.668210in}{3.425649in}%
\pgfsys@useobject{currentmarker}{}%
\end{pgfscope}%
\begin{pgfscope}%
\pgfsys@transformshift{4.742917in}{3.449487in}%
\pgfsys@useobject{currentmarker}{}%
\end{pgfscope}%
\begin{pgfscope}%
\pgfsys@transformshift{4.817625in}{3.476803in}%
\pgfsys@useobject{currentmarker}{}%
\end{pgfscope}%
\begin{pgfscope}%
\pgfsys@transformshift{4.892333in}{3.507930in}%
\pgfsys@useobject{currentmarker}{}%
\end{pgfscope}%
\begin{pgfscope}%
\pgfsys@transformshift{4.967040in}{3.542991in}%
\pgfsys@useobject{currentmarker}{}%
\end{pgfscope}%
\begin{pgfscope}%
\pgfsys@transformshift{5.041748in}{3.582324in}%
\pgfsys@useobject{currentmarker}{}%
\end{pgfscope}%
\begin{pgfscope}%
\pgfsys@transformshift{5.116455in}{3.626233in}%
\pgfsys@useobject{currentmarker}{}%
\end{pgfscope}%
\begin{pgfscope}%
\pgfsys@transformshift{5.191163in}{3.675022in}%
\pgfsys@useobject{currentmarker}{}%
\end{pgfscope}%
\begin{pgfscope}%
\pgfsys@transformshift{5.265870in}{3.728814in}%
\pgfsys@useobject{currentmarker}{}%
\end{pgfscope}%
\begin{pgfscope}%
\pgfsys@transformshift{5.340578in}{3.788227in}%
\pgfsys@useobject{currentmarker}{}%
\end{pgfscope}%
\begin{pgfscope}%
\pgfsys@transformshift{5.415285in}{3.853708in}%
\pgfsys@useobject{currentmarker}{}%
\end{pgfscope}%
\begin{pgfscope}%
\pgfsys@transformshift{5.489993in}{3.925426in}%
\pgfsys@useobject{currentmarker}{}%
\end{pgfscope}%
\begin{pgfscope}%
\pgfsys@transformshift{5.564701in}{4.004005in}%
\pgfsys@useobject{currentmarker}{}%
\end{pgfscope}%
\begin{pgfscope}%
\pgfsys@transformshift{5.639408in}{4.089772in}%
\pgfsys@useobject{currentmarker}{}%
\end{pgfscope}%
\begin{pgfscope}%
\pgfsys@transformshift{5.714116in}{4.184600in}%
\pgfsys@useobject{currentmarker}{}%
\end{pgfscope}%
\begin{pgfscope}%
\pgfsys@transformshift{5.788823in}{4.288748in}%
\pgfsys@useobject{currentmarker}{}%
\end{pgfscope}%
\begin{pgfscope}%
\pgfsys@transformshift{5.863531in}{4.403197in}%
\pgfsys@useobject{currentmarker}{}%
\end{pgfscope}%
\begin{pgfscope}%
\pgfsys@transformshift{5.938238in}{4.530911in}%
\pgfsys@useobject{currentmarker}{}%
\end{pgfscope}%
\begin{pgfscope}%
\pgfsys@transformshift{6.012946in}{4.675301in}%
\pgfsys@useobject{currentmarker}{}%
\end{pgfscope}%
\begin{pgfscope}%
\pgfsys@transformshift{6.087654in}{4.848884in}%
\pgfsys@useobject{currentmarker}{}%
\end{pgfscope}%
\begin{pgfscope}%
\pgfsys@transformshift{6.162361in}{5.025262in}%
\pgfsys@useobject{currentmarker}{}%
\end{pgfscope}%
\begin{pgfscope}%
\pgfsys@transformshift{6.237069in}{5.142314in}%
\pgfsys@useobject{currentmarker}{}%
\end{pgfscope}%
\begin{pgfscope}%
\pgfsys@transformshift{6.311776in}{5.204718in}%
\pgfsys@useobject{currentmarker}{}%
\end{pgfscope}%
\begin{pgfscope}%
\pgfsys@transformshift{6.386484in}{5.246511in}%
\pgfsys@useobject{currentmarker}{}%
\end{pgfscope}%
\begin{pgfscope}%
\pgfsys@transformshift{6.461191in}{5.279311in}%
\pgfsys@useobject{currentmarker}{}%
\end{pgfscope}%
\begin{pgfscope}%
\pgfsys@transformshift{6.535899in}{5.306988in}%
\pgfsys@useobject{currentmarker}{}%
\end{pgfscope}%
\begin{pgfscope}%
\pgfsys@transformshift{6.610606in}{5.330808in}%
\pgfsys@useobject{currentmarker}{}%
\end{pgfscope}%
\begin{pgfscope}%
\pgfsys@transformshift{6.685314in}{5.352158in}%
\pgfsys@useobject{currentmarker}{}%
\end{pgfscope}%
\begin{pgfscope}%
\pgfsys@transformshift{6.760022in}{5.371355in}%
\pgfsys@useobject{currentmarker}{}%
\end{pgfscope}%
\begin{pgfscope}%
\pgfsys@transformshift{6.834729in}{5.389033in}%
\pgfsys@useobject{currentmarker}{}%
\end{pgfscope}%
\begin{pgfscope}%
\pgfsys@transformshift{6.909437in}{5.405395in}%
\pgfsys@useobject{currentmarker}{}%
\end{pgfscope}%
\begin{pgfscope}%
\pgfsys@transformshift{6.984144in}{5.420842in}%
\pgfsys@useobject{currentmarker}{}%
\end{pgfscope}%
\begin{pgfscope}%
\pgfsys@transformshift{7.058852in}{5.434785in}%
\pgfsys@useobject{currentmarker}{}%
\end{pgfscope}%
\begin{pgfscope}%
\pgfsys@transformshift{7.133559in}{5.448089in}%
\pgfsys@useobject{currentmarker}{}%
\end{pgfscope}%
\begin{pgfscope}%
\pgfsys@transformshift{7.208267in}{5.460710in}%
\pgfsys@useobject{currentmarker}{}%
\end{pgfscope}%
\begin{pgfscope}%
\pgfsys@transformshift{7.282975in}{5.472428in}%
\pgfsys@useobject{currentmarker}{}%
\end{pgfscope}%
\begin{pgfscope}%
\pgfsys@transformshift{7.357682in}{5.483765in}%
\pgfsys@useobject{currentmarker}{}%
\end{pgfscope}%
\begin{pgfscope}%
\pgfsys@transformshift{7.432390in}{5.494485in}%
\pgfsys@useobject{currentmarker}{}%
\end{pgfscope}%
\begin{pgfscope}%
\pgfsys@transformshift{7.507097in}{5.504517in}%
\pgfsys@useobject{currentmarker}{}%
\end{pgfscope}%
\begin{pgfscope}%
\pgfsys@transformshift{7.581805in}{5.514146in}%
\pgfsys@useobject{currentmarker}{}%
\end{pgfscope}%
\begin{pgfscope}%
\pgfsys@transformshift{7.656512in}{5.523240in}%
\pgfsys@useobject{currentmarker}{}%
\end{pgfscope}%
\end{pgfscope}%
\begin{pgfscope}%
\pgfsetbuttcap%
\pgfsetroundjoin%
\definecolor{currentfill}{rgb}{0.000000,0.000000,0.000000}%
\pgfsetfillcolor{currentfill}%
\pgfsetlinewidth{0.803000pt}%
\definecolor{currentstroke}{rgb}{0.000000,0.000000,0.000000}%
\pgfsetstrokecolor{currentstroke}%
\pgfsetdash{}{0pt}%
\pgfsys@defobject{currentmarker}{\pgfqpoint{0.000000in}{-0.048611in}}{\pgfqpoint{0.000000in}{0.000000in}}{%
\pgfpathmoveto{\pgfqpoint{0.000000in}{0.000000in}}%
\pgfpathlineto{\pgfqpoint{0.000000in}{-0.048611in}}%
\pgfusepath{stroke,fill}%
}%
\begin{pgfscope}%
\pgfsys@transformshift{5.041748in}{3.312222in}%
\pgfsys@useobject{currentmarker}{}%
\end{pgfscope}%
\end{pgfscope}%
\begin{pgfscope}%
\definecolor{textcolor}{rgb}{0.000000,0.000000,0.000000}%
\pgfsetstrokecolor{textcolor}%
\pgfsetfillcolor{textcolor}%
\pgftext[x=5.041748in,y=3.215000in,,top]{\color{textcolor}\sffamily\fontsize{10.000000}{12.000000}\selectfont 3}%
\end{pgfscope}%
\begin{pgfscope}%
\pgfsetbuttcap%
\pgfsetroundjoin%
\definecolor{currentfill}{rgb}{0.000000,0.000000,0.000000}%
\pgfsetfillcolor{currentfill}%
\pgfsetlinewidth{0.803000pt}%
\definecolor{currentstroke}{rgb}{0.000000,0.000000,0.000000}%
\pgfsetstrokecolor{currentstroke}%
\pgfsetdash{}{0pt}%
\pgfsys@defobject{currentmarker}{\pgfqpoint{0.000000in}{-0.048611in}}{\pgfqpoint{0.000000in}{0.000000in}}{%
\pgfpathmoveto{\pgfqpoint{0.000000in}{0.000000in}}%
\pgfpathlineto{\pgfqpoint{0.000000in}{-0.048611in}}%
\pgfusepath{stroke,fill}%
}%
\begin{pgfscope}%
\pgfsys@transformshift{5.788823in}{3.312222in}%
\pgfsys@useobject{currentmarker}{}%
\end{pgfscope}%
\end{pgfscope}%
\begin{pgfscope}%
\definecolor{textcolor}{rgb}{0.000000,0.000000,0.000000}%
\pgfsetstrokecolor{textcolor}%
\pgfsetfillcolor{textcolor}%
\pgftext[x=5.788823in,y=3.215000in,,top]{\color{textcolor}\sffamily\fontsize{10.000000}{12.000000}\selectfont 4}%
\end{pgfscope}%
\begin{pgfscope}%
\pgfsetbuttcap%
\pgfsetroundjoin%
\definecolor{currentfill}{rgb}{0.000000,0.000000,0.000000}%
\pgfsetfillcolor{currentfill}%
\pgfsetlinewidth{0.803000pt}%
\definecolor{currentstroke}{rgb}{0.000000,0.000000,0.000000}%
\pgfsetstrokecolor{currentstroke}%
\pgfsetdash{}{0pt}%
\pgfsys@defobject{currentmarker}{\pgfqpoint{0.000000in}{-0.048611in}}{\pgfqpoint{0.000000in}{0.000000in}}{%
\pgfpathmoveto{\pgfqpoint{0.000000in}{0.000000in}}%
\pgfpathlineto{\pgfqpoint{0.000000in}{-0.048611in}}%
\pgfusepath{stroke,fill}%
}%
\begin{pgfscope}%
\pgfsys@transformshift{6.535899in}{3.312222in}%
\pgfsys@useobject{currentmarker}{}%
\end{pgfscope}%
\end{pgfscope}%
\begin{pgfscope}%
\definecolor{textcolor}{rgb}{0.000000,0.000000,0.000000}%
\pgfsetstrokecolor{textcolor}%
\pgfsetfillcolor{textcolor}%
\pgftext[x=6.535899in,y=3.215000in,,top]{\color{textcolor}\sffamily\fontsize{10.000000}{12.000000}\selectfont 5}%
\end{pgfscope}%
\begin{pgfscope}%
\pgfsetbuttcap%
\pgfsetroundjoin%
\definecolor{currentfill}{rgb}{0.000000,0.000000,0.000000}%
\pgfsetfillcolor{currentfill}%
\pgfsetlinewidth{0.803000pt}%
\definecolor{currentstroke}{rgb}{0.000000,0.000000,0.000000}%
\pgfsetstrokecolor{currentstroke}%
\pgfsetdash{}{0pt}%
\pgfsys@defobject{currentmarker}{\pgfqpoint{0.000000in}{-0.048611in}}{\pgfqpoint{0.000000in}{0.000000in}}{%
\pgfpathmoveto{\pgfqpoint{0.000000in}{0.000000in}}%
\pgfpathlineto{\pgfqpoint{0.000000in}{-0.048611in}}%
\pgfusepath{stroke,fill}%
}%
\begin{pgfscope}%
\pgfsys@transformshift{7.282975in}{3.312222in}%
\pgfsys@useobject{currentmarker}{}%
\end{pgfscope}%
\end{pgfscope}%
\begin{pgfscope}%
\definecolor{textcolor}{rgb}{0.000000,0.000000,0.000000}%
\pgfsetstrokecolor{textcolor}%
\pgfsetfillcolor{textcolor}%
\pgftext[x=7.282975in,y=3.215000in,,top]{\color{textcolor}\sffamily\fontsize{10.000000}{12.000000}\selectfont 6}%
\end{pgfscope}%
\begin{pgfscope}%
\pgfsetbuttcap%
\pgfsetroundjoin%
\definecolor{currentfill}{rgb}{0.000000,0.000000,0.000000}%
\pgfsetfillcolor{currentfill}%
\pgfsetlinewidth{0.803000pt}%
\definecolor{currentstroke}{rgb}{0.000000,0.000000,0.000000}%
\pgfsetstrokecolor{currentstroke}%
\pgfsetdash{}{0pt}%
\pgfsys@defobject{currentmarker}{\pgfqpoint{-0.048611in}{0.000000in}}{\pgfqpoint{0.000000in}{0.000000in}}{%
\pgfpathmoveto{\pgfqpoint{0.000000in}{0.000000in}}%
\pgfpathlineto{\pgfqpoint{-0.048611in}{0.000000in}}%
\pgfusepath{stroke,fill}%
}%
\begin{pgfscope}%
\pgfsys@transformshift{4.509722in}{3.321564in}%
\pgfsys@useobject{currentmarker}{}%
\end{pgfscope}%
\end{pgfscope}%
\begin{pgfscope}%
\definecolor{textcolor}{rgb}{0.000000,0.000000,0.000000}%
\pgfsetstrokecolor{textcolor}%
\pgfsetfillcolor{textcolor}%
\pgftext[x=4.075247in,y=3.268802in,left,base]{\color{textcolor}\sffamily\fontsize{10.000000}{12.000000}\selectfont −3.0}%
\end{pgfscope}%
\begin{pgfscope}%
\pgfsetbuttcap%
\pgfsetroundjoin%
\definecolor{currentfill}{rgb}{0.000000,0.000000,0.000000}%
\pgfsetfillcolor{currentfill}%
\pgfsetlinewidth{0.803000pt}%
\definecolor{currentstroke}{rgb}{0.000000,0.000000,0.000000}%
\pgfsetstrokecolor{currentstroke}%
\pgfsetdash{}{0pt}%
\pgfsys@defobject{currentmarker}{\pgfqpoint{-0.048611in}{0.000000in}}{\pgfqpoint{0.000000in}{0.000000in}}{%
\pgfpathmoveto{\pgfqpoint{0.000000in}{0.000000in}}%
\pgfpathlineto{\pgfqpoint{-0.048611in}{0.000000in}}%
\pgfusepath{stroke,fill}%
}%
\begin{pgfscope}%
\pgfsys@transformshift{4.509722in}{3.764465in}%
\pgfsys@useobject{currentmarker}{}%
\end{pgfscope}%
\end{pgfscope}%
\begin{pgfscope}%
\definecolor{textcolor}{rgb}{0.000000,0.000000,0.000000}%
\pgfsetstrokecolor{textcolor}%
\pgfsetfillcolor{textcolor}%
\pgftext[x=4.075247in,y=3.711703in,left,base]{\color{textcolor}\sffamily\fontsize{10.000000}{12.000000}\selectfont −2.5}%
\end{pgfscope}%
\begin{pgfscope}%
\pgfsetbuttcap%
\pgfsetroundjoin%
\definecolor{currentfill}{rgb}{0.000000,0.000000,0.000000}%
\pgfsetfillcolor{currentfill}%
\pgfsetlinewidth{0.803000pt}%
\definecolor{currentstroke}{rgb}{0.000000,0.000000,0.000000}%
\pgfsetstrokecolor{currentstroke}%
\pgfsetdash{}{0pt}%
\pgfsys@defobject{currentmarker}{\pgfqpoint{-0.048611in}{0.000000in}}{\pgfqpoint{0.000000in}{0.000000in}}{%
\pgfpathmoveto{\pgfqpoint{0.000000in}{0.000000in}}%
\pgfpathlineto{\pgfqpoint{-0.048611in}{0.000000in}}%
\pgfusepath{stroke,fill}%
}%
\begin{pgfscope}%
\pgfsys@transformshift{4.509722in}{4.207366in}%
\pgfsys@useobject{currentmarker}{}%
\end{pgfscope}%
\end{pgfscope}%
\begin{pgfscope}%
\definecolor{textcolor}{rgb}{0.000000,0.000000,0.000000}%
\pgfsetstrokecolor{textcolor}%
\pgfsetfillcolor{textcolor}%
\pgftext[x=4.075247in,y=4.154604in,left,base]{\color{textcolor}\sffamily\fontsize{10.000000}{12.000000}\selectfont −2.0}%
\end{pgfscope}%
\begin{pgfscope}%
\pgfsetbuttcap%
\pgfsetroundjoin%
\definecolor{currentfill}{rgb}{0.000000,0.000000,0.000000}%
\pgfsetfillcolor{currentfill}%
\pgfsetlinewidth{0.803000pt}%
\definecolor{currentstroke}{rgb}{0.000000,0.000000,0.000000}%
\pgfsetstrokecolor{currentstroke}%
\pgfsetdash{}{0pt}%
\pgfsys@defobject{currentmarker}{\pgfqpoint{-0.048611in}{0.000000in}}{\pgfqpoint{0.000000in}{0.000000in}}{%
\pgfpathmoveto{\pgfqpoint{0.000000in}{0.000000in}}%
\pgfpathlineto{\pgfqpoint{-0.048611in}{0.000000in}}%
\pgfusepath{stroke,fill}%
}%
\begin{pgfscope}%
\pgfsys@transformshift{4.509722in}{4.650267in}%
\pgfsys@useobject{currentmarker}{}%
\end{pgfscope}%
\end{pgfscope}%
\begin{pgfscope}%
\definecolor{textcolor}{rgb}{0.000000,0.000000,0.000000}%
\pgfsetstrokecolor{textcolor}%
\pgfsetfillcolor{textcolor}%
\pgftext[x=4.075247in,y=4.597505in,left,base]{\color{textcolor}\sffamily\fontsize{10.000000}{12.000000}\selectfont −1.5}%
\end{pgfscope}%
\begin{pgfscope}%
\pgfsetbuttcap%
\pgfsetroundjoin%
\definecolor{currentfill}{rgb}{0.000000,0.000000,0.000000}%
\pgfsetfillcolor{currentfill}%
\pgfsetlinewidth{0.803000pt}%
\definecolor{currentstroke}{rgb}{0.000000,0.000000,0.000000}%
\pgfsetstrokecolor{currentstroke}%
\pgfsetdash{}{0pt}%
\pgfsys@defobject{currentmarker}{\pgfqpoint{-0.048611in}{0.000000in}}{\pgfqpoint{0.000000in}{0.000000in}}{%
\pgfpathmoveto{\pgfqpoint{0.000000in}{0.000000in}}%
\pgfpathlineto{\pgfqpoint{-0.048611in}{0.000000in}}%
\pgfusepath{stroke,fill}%
}%
\begin{pgfscope}%
\pgfsys@transformshift{4.509722in}{5.093168in}%
\pgfsys@useobject{currentmarker}{}%
\end{pgfscope}%
\end{pgfscope}%
\begin{pgfscope}%
\definecolor{textcolor}{rgb}{0.000000,0.000000,0.000000}%
\pgfsetstrokecolor{textcolor}%
\pgfsetfillcolor{textcolor}%
\pgftext[x=4.075247in,y=5.040406in,left,base]{\color{textcolor}\sffamily\fontsize{10.000000}{12.000000}\selectfont −1.0}%
\end{pgfscope}%
\begin{pgfscope}%
\pgfsetbuttcap%
\pgfsetroundjoin%
\definecolor{currentfill}{rgb}{0.000000,0.000000,0.000000}%
\pgfsetfillcolor{currentfill}%
\pgfsetlinewidth{0.803000pt}%
\definecolor{currentstroke}{rgb}{0.000000,0.000000,0.000000}%
\pgfsetstrokecolor{currentstroke}%
\pgfsetdash{}{0pt}%
\pgfsys@defobject{currentmarker}{\pgfqpoint{-0.048611in}{0.000000in}}{\pgfqpoint{0.000000in}{0.000000in}}{%
\pgfpathmoveto{\pgfqpoint{0.000000in}{0.000000in}}%
\pgfpathlineto{\pgfqpoint{-0.048611in}{0.000000in}}%
\pgfusepath{stroke,fill}%
}%
\begin{pgfscope}%
\pgfsys@transformshift{4.509722in}{5.536069in}%
\pgfsys@useobject{currentmarker}{}%
\end{pgfscope}%
\end{pgfscope}%
\begin{pgfscope}%
\definecolor{textcolor}{rgb}{0.000000,0.000000,0.000000}%
\pgfsetstrokecolor{textcolor}%
\pgfsetfillcolor{textcolor}%
\pgftext[x=4.075247in,y=5.483307in,left,base]{\color{textcolor}\sffamily\fontsize{10.000000}{12.000000}\selectfont −0.5}%
\end{pgfscope}%
\begin{pgfscope}%
\pgfpathrectangle{\pgfqpoint{4.509722in}{3.312222in}}{\pgfqpoint{3.305278in}{2.324444in}}%
\pgfusepath{clip}%
\pgfsetrectcap%
\pgfsetroundjoin%
\pgfsetlinewidth{1.505625pt}%
\definecolor{currentstroke}{rgb}{0.121569,0.466667,0.705882}%
\pgfsetstrokecolor{currentstroke}%
\pgfsetdash{}{0pt}%
\pgfpathmoveto{\pgfqpoint{4.668210in}{3.425649in}}%
\pgfpathlineto{\pgfqpoint{4.742917in}{3.449487in}}%
\pgfpathlineto{\pgfqpoint{4.817625in}{3.476803in}}%
\pgfpathlineto{\pgfqpoint{4.892333in}{3.507930in}}%
\pgfpathlineto{\pgfqpoint{4.967040in}{3.542991in}}%
\pgfpathlineto{\pgfqpoint{5.041748in}{3.582324in}}%
\pgfpathlineto{\pgfqpoint{5.116455in}{3.626233in}}%
\pgfpathlineto{\pgfqpoint{5.191163in}{3.675022in}}%
\pgfpathlineto{\pgfqpoint{5.265870in}{3.728814in}}%
\pgfpathlineto{\pgfqpoint{5.340578in}{3.788227in}}%
\pgfpathlineto{\pgfqpoint{5.415285in}{3.853708in}}%
\pgfpathlineto{\pgfqpoint{5.489993in}{3.925426in}}%
\pgfpathlineto{\pgfqpoint{5.564701in}{4.004005in}}%
\pgfpathlineto{\pgfqpoint{5.639408in}{4.089772in}}%
\pgfpathlineto{\pgfqpoint{5.714116in}{4.184600in}}%
\pgfpathlineto{\pgfqpoint{5.788823in}{4.288748in}}%
\pgfpathlineto{\pgfqpoint{5.863531in}{4.403197in}}%
\pgfpathlineto{\pgfqpoint{5.938238in}{4.530911in}}%
\pgfpathlineto{\pgfqpoint{6.012946in}{4.675301in}}%
\pgfpathlineto{\pgfqpoint{6.087654in}{4.848884in}}%
\pgfpathlineto{\pgfqpoint{6.162361in}{5.025262in}}%
\pgfpathlineto{\pgfqpoint{6.237069in}{5.142314in}}%
\pgfpathlineto{\pgfqpoint{6.311776in}{5.204718in}}%
\pgfpathlineto{\pgfqpoint{6.386484in}{5.246511in}}%
\pgfpathlineto{\pgfqpoint{6.461191in}{5.279311in}}%
\pgfpathlineto{\pgfqpoint{6.535899in}{5.306988in}}%
\pgfpathlineto{\pgfqpoint{6.610606in}{5.330808in}}%
\pgfpathlineto{\pgfqpoint{6.685314in}{5.352158in}}%
\pgfpathlineto{\pgfqpoint{6.760022in}{5.371355in}}%
\pgfpathlineto{\pgfqpoint{6.834729in}{5.389033in}}%
\pgfpathlineto{\pgfqpoint{6.909437in}{5.405395in}}%
\pgfpathlineto{\pgfqpoint{6.984144in}{5.420842in}}%
\pgfpathlineto{\pgfqpoint{7.058852in}{5.434785in}}%
\pgfpathlineto{\pgfqpoint{7.133559in}{5.448089in}}%
\pgfpathlineto{\pgfqpoint{7.208267in}{5.460710in}}%
\pgfpathlineto{\pgfqpoint{7.282975in}{5.472428in}}%
\pgfpathlineto{\pgfqpoint{7.357682in}{5.483765in}}%
\pgfpathlineto{\pgfqpoint{7.432390in}{5.494485in}}%
\pgfpathlineto{\pgfqpoint{7.507097in}{5.504517in}}%
\pgfpathlineto{\pgfqpoint{7.581805in}{5.514146in}}%
\pgfpathlineto{\pgfqpoint{7.656512in}{5.523240in}}%
\pgfusepath{stroke}%
\end{pgfscope}%
\begin{pgfscope}%
\pgfsetrectcap%
\pgfsetmiterjoin%
\pgfsetlinewidth{0.803000pt}%
\definecolor{currentstroke}{rgb}{0.000000,0.000000,0.000000}%
\pgfsetstrokecolor{currentstroke}%
\pgfsetdash{}{0pt}%
\pgfpathmoveto{\pgfqpoint{4.509722in}{3.312222in}}%
\pgfpathlineto{\pgfqpoint{4.509722in}{5.636667in}}%
\pgfusepath{stroke}%
\end{pgfscope}%
\begin{pgfscope}%
\pgfsetrectcap%
\pgfsetmiterjoin%
\pgfsetlinewidth{0.803000pt}%
\definecolor{currentstroke}{rgb}{0.000000,0.000000,0.000000}%
\pgfsetstrokecolor{currentstroke}%
\pgfsetdash{}{0pt}%
\pgfpathmoveto{\pgfqpoint{7.815000in}{3.312222in}}%
\pgfpathlineto{\pgfqpoint{7.815000in}{5.636667in}}%
\pgfusepath{stroke}%
\end{pgfscope}%
\begin{pgfscope}%
\pgfsetrectcap%
\pgfsetmiterjoin%
\pgfsetlinewidth{0.803000pt}%
\definecolor{currentstroke}{rgb}{0.000000,0.000000,0.000000}%
\pgfsetstrokecolor{currentstroke}%
\pgfsetdash{}{0pt}%
\pgfpathmoveto{\pgfqpoint{4.509722in}{3.312222in}}%
\pgfpathlineto{\pgfqpoint{7.815000in}{3.312222in}}%
\pgfusepath{stroke}%
\end{pgfscope}%
\begin{pgfscope}%
\pgfsetrectcap%
\pgfsetmiterjoin%
\pgfsetlinewidth{0.803000pt}%
\definecolor{currentstroke}{rgb}{0.000000,0.000000,0.000000}%
\pgfsetstrokecolor{currentstroke}%
\pgfsetdash{}{0pt}%
\pgfpathmoveto{\pgfqpoint{4.509722in}{5.636667in}}%
\pgfpathlineto{\pgfqpoint{7.815000in}{5.636667in}}%
\pgfusepath{stroke}%
\end{pgfscope}%
\begin{pgfscope}%
\definecolor{textcolor}{rgb}{0.000000,0.000000,0.000000}%
\pgfsetstrokecolor{textcolor}%
\pgfsetfillcolor{textcolor}%
\pgftext[x=6.162361in,y=5.720000in,,base]{\color{textcolor}\sffamily\fontsize{12.000000}{14.400000}\selectfont \(\displaystyle  N = 12 \)}%
\end{pgfscope}%
\begin{pgfscope}%
\pgfsetbuttcap%
\pgfsetmiterjoin%
\definecolor{currentfill}{rgb}{1.000000,1.000000,1.000000}%
\pgfsetfillcolor{currentfill}%
\pgfsetlinewidth{0.000000pt}%
\definecolor{currentstroke}{rgb}{0.000000,0.000000,0.000000}%
\pgfsetstrokecolor{currentstroke}%
\pgfsetstrokeopacity{0.000000}%
\pgfsetdash{}{0pt}%
\pgfpathmoveto{\pgfqpoint{0.584722in}{0.387222in}}%
\pgfpathlineto{\pgfqpoint{3.890000in}{0.387222in}}%
\pgfpathlineto{\pgfqpoint{3.890000in}{2.711667in}}%
\pgfpathlineto{\pgfqpoint{0.584722in}{2.711667in}}%
\pgfpathclose%
\pgfusepath{fill}%
\end{pgfscope}%
\begin{pgfscope}%
\pgfpathrectangle{\pgfqpoint{0.584722in}{0.387222in}}{\pgfqpoint{3.305278in}{2.324444in}}%
\pgfusepath{clip}%
\pgfsetbuttcap%
\pgfsetroundjoin%
\definecolor{currentfill}{rgb}{0.121569,0.466667,0.705882}%
\pgfsetfillcolor{currentfill}%
\pgfsetfillopacity{0.300000}%
\pgfsetlinewidth{1.003750pt}%
\definecolor{currentstroke}{rgb}{0.121569,0.466667,0.705882}%
\pgfsetstrokecolor{currentstroke}%
\pgfsetstrokeopacity{0.300000}%
\pgfsetdash{}{0pt}%
\pgfpathmoveto{\pgfqpoint{0.743210in}{0.500691in}}%
\pgfpathlineto{\pgfqpoint{0.743210in}{0.500607in}}%
\pgfpathlineto{\pgfqpoint{0.817917in}{0.524460in}}%
\pgfpathlineto{\pgfqpoint{0.892625in}{0.551771in}}%
\pgfpathlineto{\pgfqpoint{0.967333in}{0.582920in}}%
\pgfpathlineto{\pgfqpoint{1.042040in}{0.618048in}}%
\pgfpathlineto{\pgfqpoint{1.116748in}{0.657321in}}%
\pgfpathlineto{\pgfqpoint{1.191455in}{0.701138in}}%
\pgfpathlineto{\pgfqpoint{1.266163in}{0.749847in}}%
\pgfpathlineto{\pgfqpoint{1.340870in}{0.803776in}}%
\pgfpathlineto{\pgfqpoint{1.415578in}{0.863030in}}%
\pgfpathlineto{\pgfqpoint{1.490285in}{0.928408in}}%
\pgfpathlineto{\pgfqpoint{1.564993in}{1.000237in}}%
\pgfpathlineto{\pgfqpoint{1.639701in}{1.078530in}}%
\pgfpathlineto{\pgfqpoint{1.714408in}{1.164401in}}%
\pgfpathlineto{\pgfqpoint{1.789116in}{1.259461in}}%
\pgfpathlineto{\pgfqpoint{1.863823in}{1.363452in}}%
\pgfpathlineto{\pgfqpoint{1.938531in}{1.477021in}}%
\pgfpathlineto{\pgfqpoint{2.013238in}{1.604811in}}%
\pgfpathlineto{\pgfqpoint{2.087946in}{1.748193in}}%
\pgfpathlineto{\pgfqpoint{2.162654in}{1.916788in}}%
\pgfpathlineto{\pgfqpoint{2.237361in}{2.116361in}}%
\pgfpathlineto{\pgfqpoint{2.312069in}{2.231656in}}%
\pgfpathlineto{\pgfqpoint{2.386776in}{2.286911in}}%
\pgfpathlineto{\pgfqpoint{2.461484in}{2.324313in}}%
\pgfpathlineto{\pgfqpoint{2.536191in}{2.355788in}}%
\pgfpathlineto{\pgfqpoint{2.610899in}{2.382240in}}%
\pgfpathlineto{\pgfqpoint{2.685606in}{2.405583in}}%
\pgfpathlineto{\pgfqpoint{2.760314in}{2.426964in}}%
\pgfpathlineto{\pgfqpoint{2.835022in}{2.446101in}}%
\pgfpathlineto{\pgfqpoint{2.909729in}{2.463718in}}%
\pgfpathlineto{\pgfqpoint{2.984437in}{2.480344in}}%
\pgfpathlineto{\pgfqpoint{3.059144in}{2.495533in}}%
\pgfpathlineto{\pgfqpoint{3.133852in}{2.509729in}}%
\pgfpathlineto{\pgfqpoint{3.208559in}{2.523024in}}%
\pgfpathlineto{\pgfqpoint{3.283267in}{2.535474in}}%
\pgfpathlineto{\pgfqpoint{3.357975in}{2.547542in}}%
\pgfpathlineto{\pgfqpoint{3.432682in}{2.558473in}}%
\pgfpathlineto{\pgfqpoint{3.507390in}{2.569028in}}%
\pgfpathlineto{\pgfqpoint{3.582097in}{2.579275in}}%
\pgfpathlineto{\pgfqpoint{3.656805in}{2.588913in}}%
\pgfpathlineto{\pgfqpoint{3.731512in}{2.598068in}}%
\pgfpathlineto{\pgfqpoint{3.731512in}{2.598412in}}%
\pgfpathlineto{\pgfqpoint{3.731512in}{2.598412in}}%
\pgfpathlineto{\pgfqpoint{3.656805in}{2.589381in}}%
\pgfpathlineto{\pgfqpoint{3.582097in}{2.579441in}}%
\pgfpathlineto{\pgfqpoint{3.507390in}{2.569786in}}%
\pgfpathlineto{\pgfqpoint{3.432682in}{2.558992in}}%
\pgfpathlineto{\pgfqpoint{3.357975in}{2.547777in}}%
\pgfpathlineto{\pgfqpoint{3.283267in}{2.536063in}}%
\pgfpathlineto{\pgfqpoint{3.208559in}{2.523274in}}%
\pgfpathlineto{\pgfqpoint{3.133852in}{2.510062in}}%
\pgfpathlineto{\pgfqpoint{3.059144in}{2.495985in}}%
\pgfpathlineto{\pgfqpoint{2.984437in}{2.480659in}}%
\pgfpathlineto{\pgfqpoint{2.909729in}{2.464500in}}%
\pgfpathlineto{\pgfqpoint{2.835022in}{2.446595in}}%
\pgfpathlineto{\pgfqpoint{2.760314in}{2.427287in}}%
\pgfpathlineto{\pgfqpoint{2.685606in}{2.406267in}}%
\pgfpathlineto{\pgfqpoint{2.610899in}{2.382718in}}%
\pgfpathlineto{\pgfqpoint{2.536191in}{2.356097in}}%
\pgfpathlineto{\pgfqpoint{2.461484in}{2.325546in}}%
\pgfpathlineto{\pgfqpoint{2.386776in}{2.288318in}}%
\pgfpathlineto{\pgfqpoint{2.312069in}{2.234156in}}%
\pgfpathlineto{\pgfqpoint{2.237361in}{2.118603in}}%
\pgfpathlineto{\pgfqpoint{2.162654in}{1.920512in}}%
\pgfpathlineto{\pgfqpoint{2.087946in}{1.751838in}}%
\pgfpathlineto{\pgfqpoint{2.013238in}{1.607790in}}%
\pgfpathlineto{\pgfqpoint{1.938531in}{1.479254in}}%
\pgfpathlineto{\pgfqpoint{1.863823in}{1.364142in}}%
\pgfpathlineto{\pgfqpoint{1.789116in}{1.260041in}}%
\pgfpathlineto{\pgfqpoint{1.714408in}{1.165529in}}%
\pgfpathlineto{\pgfqpoint{1.639701in}{1.079424in}}%
\pgfpathlineto{\pgfqpoint{1.564993in}{1.000788in}}%
\pgfpathlineto{\pgfqpoint{1.490285in}{0.928855in}}%
\pgfpathlineto{\pgfqpoint{1.415578in}{0.863462in}}%
\pgfpathlineto{\pgfqpoint{1.340870in}{0.804034in}}%
\pgfpathlineto{\pgfqpoint{1.266163in}{0.750199in}}%
\pgfpathlineto{\pgfqpoint{1.191455in}{0.701353in}}%
\pgfpathlineto{\pgfqpoint{1.116748in}{0.657499in}}%
\pgfpathlineto{\pgfqpoint{1.042040in}{0.618111in}}%
\pgfpathlineto{\pgfqpoint{0.967333in}{0.583000in}}%
\pgfpathlineto{\pgfqpoint{0.892625in}{0.551868in}}%
\pgfpathlineto{\pgfqpoint{0.817917in}{0.524505in}}%
\pgfpathlineto{\pgfqpoint{0.743210in}{0.500691in}}%
\pgfpathclose%
\pgfusepath{stroke,fill}%
\end{pgfscope}%
\begin{pgfscope}%
\pgfpathrectangle{\pgfqpoint{0.584722in}{0.387222in}}{\pgfqpoint{3.305278in}{2.324444in}}%
\pgfusepath{clip}%
\pgfsetbuttcap%
\pgfsetroundjoin%
\definecolor{currentfill}{rgb}{0.121569,0.466667,0.705882}%
\pgfsetfillcolor{currentfill}%
\pgfsetlinewidth{1.003750pt}%
\definecolor{currentstroke}{rgb}{0.121569,0.466667,0.705882}%
\pgfsetstrokecolor{currentstroke}%
\pgfsetdash{}{0pt}%
\pgfsys@defobject{currentmarker}{\pgfqpoint{-0.009821in}{-0.009821in}}{\pgfqpoint{0.009821in}{0.009821in}}{%
\pgfpathmoveto{\pgfqpoint{0.000000in}{-0.009821in}}%
\pgfpathcurveto{\pgfqpoint{0.002605in}{-0.009821in}}{\pgfqpoint{0.005103in}{-0.008786in}}{\pgfqpoint{0.006944in}{-0.006944in}}%
\pgfpathcurveto{\pgfqpoint{0.008786in}{-0.005103in}}{\pgfqpoint{0.009821in}{-0.002605in}}{\pgfqpoint{0.009821in}{0.000000in}}%
\pgfpathcurveto{\pgfqpoint{0.009821in}{0.002605in}}{\pgfqpoint{0.008786in}{0.005103in}}{\pgfqpoint{0.006944in}{0.006944in}}%
\pgfpathcurveto{\pgfqpoint{0.005103in}{0.008786in}}{\pgfqpoint{0.002605in}{0.009821in}}{\pgfqpoint{0.000000in}{0.009821in}}%
\pgfpathcurveto{\pgfqpoint{-0.002605in}{0.009821in}}{\pgfqpoint{-0.005103in}{0.008786in}}{\pgfqpoint{-0.006944in}{0.006944in}}%
\pgfpathcurveto{\pgfqpoint{-0.008786in}{0.005103in}}{\pgfqpoint{-0.009821in}{0.002605in}}{\pgfqpoint{-0.009821in}{0.000000in}}%
\pgfpathcurveto{\pgfqpoint{-0.009821in}{-0.002605in}}{\pgfqpoint{-0.008786in}{-0.005103in}}{\pgfqpoint{-0.006944in}{-0.006944in}}%
\pgfpathcurveto{\pgfqpoint{-0.005103in}{-0.008786in}}{\pgfqpoint{-0.002605in}{-0.009821in}}{\pgfqpoint{0.000000in}{-0.009821in}}%
\pgfpathclose%
\pgfusepath{stroke,fill}%
}%
\begin{pgfscope}%
\pgfsys@transformshift{0.743210in}{0.500649in}%
\pgfsys@useobject{currentmarker}{}%
\end{pgfscope}%
\begin{pgfscope}%
\pgfsys@transformshift{0.817917in}{0.524482in}%
\pgfsys@useobject{currentmarker}{}%
\end{pgfscope}%
\begin{pgfscope}%
\pgfsys@transformshift{0.892625in}{0.551819in}%
\pgfsys@useobject{currentmarker}{}%
\end{pgfscope}%
\begin{pgfscope}%
\pgfsys@transformshift{0.967333in}{0.582960in}%
\pgfsys@useobject{currentmarker}{}%
\end{pgfscope}%
\begin{pgfscope}%
\pgfsys@transformshift{1.042040in}{0.618080in}%
\pgfsys@useobject{currentmarker}{}%
\end{pgfscope}%
\begin{pgfscope}%
\pgfsys@transformshift{1.116748in}{0.657410in}%
\pgfsys@useobject{currentmarker}{}%
\end{pgfscope}%
\begin{pgfscope}%
\pgfsys@transformshift{1.191455in}{0.701245in}%
\pgfsys@useobject{currentmarker}{}%
\end{pgfscope}%
\begin{pgfscope}%
\pgfsys@transformshift{1.266163in}{0.750023in}%
\pgfsys@useobject{currentmarker}{}%
\end{pgfscope}%
\begin{pgfscope}%
\pgfsys@transformshift{1.340870in}{0.803905in}%
\pgfsys@useobject{currentmarker}{}%
\end{pgfscope}%
\begin{pgfscope}%
\pgfsys@transformshift{1.415578in}{0.863246in}%
\pgfsys@useobject{currentmarker}{}%
\end{pgfscope}%
\begin{pgfscope}%
\pgfsys@transformshift{1.490285in}{0.928632in}%
\pgfsys@useobject{currentmarker}{}%
\end{pgfscope}%
\begin{pgfscope}%
\pgfsys@transformshift{1.564993in}{1.000513in}%
\pgfsys@useobject{currentmarker}{}%
\end{pgfscope}%
\begin{pgfscope}%
\pgfsys@transformshift{1.639701in}{1.078977in}%
\pgfsys@useobject{currentmarker}{}%
\end{pgfscope}%
\begin{pgfscope}%
\pgfsys@transformshift{1.714408in}{1.164965in}%
\pgfsys@useobject{currentmarker}{}%
\end{pgfscope}%
\begin{pgfscope}%
\pgfsys@transformshift{1.789116in}{1.259751in}%
\pgfsys@useobject{currentmarker}{}%
\end{pgfscope}%
\begin{pgfscope}%
\pgfsys@transformshift{1.863823in}{1.363797in}%
\pgfsys@useobject{currentmarker}{}%
\end{pgfscope}%
\begin{pgfscope}%
\pgfsys@transformshift{1.938531in}{1.478137in}%
\pgfsys@useobject{currentmarker}{}%
\end{pgfscope}%
\begin{pgfscope}%
\pgfsys@transformshift{2.013238in}{1.606301in}%
\pgfsys@useobject{currentmarker}{}%
\end{pgfscope}%
\begin{pgfscope}%
\pgfsys@transformshift{2.087946in}{1.750016in}%
\pgfsys@useobject{currentmarker}{}%
\end{pgfscope}%
\begin{pgfscope}%
\pgfsys@transformshift{2.162654in}{1.918650in}%
\pgfsys@useobject{currentmarker}{}%
\end{pgfscope}%
\begin{pgfscope}%
\pgfsys@transformshift{2.237361in}{2.117482in}%
\pgfsys@useobject{currentmarker}{}%
\end{pgfscope}%
\begin{pgfscope}%
\pgfsys@transformshift{2.312069in}{2.232906in}%
\pgfsys@useobject{currentmarker}{}%
\end{pgfscope}%
\begin{pgfscope}%
\pgfsys@transformshift{2.386776in}{2.287615in}%
\pgfsys@useobject{currentmarker}{}%
\end{pgfscope}%
\begin{pgfscope}%
\pgfsys@transformshift{2.461484in}{2.324929in}%
\pgfsys@useobject{currentmarker}{}%
\end{pgfscope}%
\begin{pgfscope}%
\pgfsys@transformshift{2.536191in}{2.355943in}%
\pgfsys@useobject{currentmarker}{}%
\end{pgfscope}%
\begin{pgfscope}%
\pgfsys@transformshift{2.610899in}{2.382479in}%
\pgfsys@useobject{currentmarker}{}%
\end{pgfscope}%
\begin{pgfscope}%
\pgfsys@transformshift{2.685606in}{2.405925in}%
\pgfsys@useobject{currentmarker}{}%
\end{pgfscope}%
\begin{pgfscope}%
\pgfsys@transformshift{2.760314in}{2.427126in}%
\pgfsys@useobject{currentmarker}{}%
\end{pgfscope}%
\begin{pgfscope}%
\pgfsys@transformshift{2.835022in}{2.446348in}%
\pgfsys@useobject{currentmarker}{}%
\end{pgfscope}%
\begin{pgfscope}%
\pgfsys@transformshift{2.909729in}{2.464109in}%
\pgfsys@useobject{currentmarker}{}%
\end{pgfscope}%
\begin{pgfscope}%
\pgfsys@transformshift{2.984437in}{2.480502in}%
\pgfsys@useobject{currentmarker}{}%
\end{pgfscope}%
\begin{pgfscope}%
\pgfsys@transformshift{3.059144in}{2.495759in}%
\pgfsys@useobject{currentmarker}{}%
\end{pgfscope}%
\begin{pgfscope}%
\pgfsys@transformshift{3.133852in}{2.509895in}%
\pgfsys@useobject{currentmarker}{}%
\end{pgfscope}%
\begin{pgfscope}%
\pgfsys@transformshift{3.208559in}{2.523149in}%
\pgfsys@useobject{currentmarker}{}%
\end{pgfscope}%
\begin{pgfscope}%
\pgfsys@transformshift{3.283267in}{2.535768in}%
\pgfsys@useobject{currentmarker}{}%
\end{pgfscope}%
\begin{pgfscope}%
\pgfsys@transformshift{3.357975in}{2.547660in}%
\pgfsys@useobject{currentmarker}{}%
\end{pgfscope}%
\begin{pgfscope}%
\pgfsys@transformshift{3.432682in}{2.558732in}%
\pgfsys@useobject{currentmarker}{}%
\end{pgfscope}%
\begin{pgfscope}%
\pgfsys@transformshift{3.507390in}{2.569407in}%
\pgfsys@useobject{currentmarker}{}%
\end{pgfscope}%
\begin{pgfscope}%
\pgfsys@transformshift{3.582097in}{2.579358in}%
\pgfsys@useobject{currentmarker}{}%
\end{pgfscope}%
\begin{pgfscope}%
\pgfsys@transformshift{3.656805in}{2.589147in}%
\pgfsys@useobject{currentmarker}{}%
\end{pgfscope}%
\begin{pgfscope}%
\pgfsys@transformshift{3.731512in}{2.598240in}%
\pgfsys@useobject{currentmarker}{}%
\end{pgfscope}%
\end{pgfscope}%
\begin{pgfscope}%
\pgfsetbuttcap%
\pgfsetroundjoin%
\definecolor{currentfill}{rgb}{0.000000,0.000000,0.000000}%
\pgfsetfillcolor{currentfill}%
\pgfsetlinewidth{0.803000pt}%
\definecolor{currentstroke}{rgb}{0.000000,0.000000,0.000000}%
\pgfsetstrokecolor{currentstroke}%
\pgfsetdash{}{0pt}%
\pgfsys@defobject{currentmarker}{\pgfqpoint{0.000000in}{-0.048611in}}{\pgfqpoint{0.000000in}{0.000000in}}{%
\pgfpathmoveto{\pgfqpoint{0.000000in}{0.000000in}}%
\pgfpathlineto{\pgfqpoint{0.000000in}{-0.048611in}}%
\pgfusepath{stroke,fill}%
}%
\begin{pgfscope}%
\pgfsys@transformshift{1.116748in}{0.387222in}%
\pgfsys@useobject{currentmarker}{}%
\end{pgfscope}%
\end{pgfscope}%
\begin{pgfscope}%
\definecolor{textcolor}{rgb}{0.000000,0.000000,0.000000}%
\pgfsetstrokecolor{textcolor}%
\pgfsetfillcolor{textcolor}%
\pgftext[x=1.116748in,y=0.290000in,,top]{\color{textcolor}\sffamily\fontsize{10.000000}{12.000000}\selectfont 3}%
\end{pgfscope}%
\begin{pgfscope}%
\pgfsetbuttcap%
\pgfsetroundjoin%
\definecolor{currentfill}{rgb}{0.000000,0.000000,0.000000}%
\pgfsetfillcolor{currentfill}%
\pgfsetlinewidth{0.803000pt}%
\definecolor{currentstroke}{rgb}{0.000000,0.000000,0.000000}%
\pgfsetstrokecolor{currentstroke}%
\pgfsetdash{}{0pt}%
\pgfsys@defobject{currentmarker}{\pgfqpoint{0.000000in}{-0.048611in}}{\pgfqpoint{0.000000in}{0.000000in}}{%
\pgfpathmoveto{\pgfqpoint{0.000000in}{0.000000in}}%
\pgfpathlineto{\pgfqpoint{0.000000in}{-0.048611in}}%
\pgfusepath{stroke,fill}%
}%
\begin{pgfscope}%
\pgfsys@transformshift{1.863823in}{0.387222in}%
\pgfsys@useobject{currentmarker}{}%
\end{pgfscope}%
\end{pgfscope}%
\begin{pgfscope}%
\definecolor{textcolor}{rgb}{0.000000,0.000000,0.000000}%
\pgfsetstrokecolor{textcolor}%
\pgfsetfillcolor{textcolor}%
\pgftext[x=1.863823in,y=0.290000in,,top]{\color{textcolor}\sffamily\fontsize{10.000000}{12.000000}\selectfont 4}%
\end{pgfscope}%
\begin{pgfscope}%
\pgfsetbuttcap%
\pgfsetroundjoin%
\definecolor{currentfill}{rgb}{0.000000,0.000000,0.000000}%
\pgfsetfillcolor{currentfill}%
\pgfsetlinewidth{0.803000pt}%
\definecolor{currentstroke}{rgb}{0.000000,0.000000,0.000000}%
\pgfsetstrokecolor{currentstroke}%
\pgfsetdash{}{0pt}%
\pgfsys@defobject{currentmarker}{\pgfqpoint{0.000000in}{-0.048611in}}{\pgfqpoint{0.000000in}{0.000000in}}{%
\pgfpathmoveto{\pgfqpoint{0.000000in}{0.000000in}}%
\pgfpathlineto{\pgfqpoint{0.000000in}{-0.048611in}}%
\pgfusepath{stroke,fill}%
}%
\begin{pgfscope}%
\pgfsys@transformshift{2.610899in}{0.387222in}%
\pgfsys@useobject{currentmarker}{}%
\end{pgfscope}%
\end{pgfscope}%
\begin{pgfscope}%
\definecolor{textcolor}{rgb}{0.000000,0.000000,0.000000}%
\pgfsetstrokecolor{textcolor}%
\pgfsetfillcolor{textcolor}%
\pgftext[x=2.610899in,y=0.290000in,,top]{\color{textcolor}\sffamily\fontsize{10.000000}{12.000000}\selectfont 5}%
\end{pgfscope}%
\begin{pgfscope}%
\pgfsetbuttcap%
\pgfsetroundjoin%
\definecolor{currentfill}{rgb}{0.000000,0.000000,0.000000}%
\pgfsetfillcolor{currentfill}%
\pgfsetlinewidth{0.803000pt}%
\definecolor{currentstroke}{rgb}{0.000000,0.000000,0.000000}%
\pgfsetstrokecolor{currentstroke}%
\pgfsetdash{}{0pt}%
\pgfsys@defobject{currentmarker}{\pgfqpoint{0.000000in}{-0.048611in}}{\pgfqpoint{0.000000in}{0.000000in}}{%
\pgfpathmoveto{\pgfqpoint{0.000000in}{0.000000in}}%
\pgfpathlineto{\pgfqpoint{0.000000in}{-0.048611in}}%
\pgfusepath{stroke,fill}%
}%
\begin{pgfscope}%
\pgfsys@transformshift{3.357975in}{0.387222in}%
\pgfsys@useobject{currentmarker}{}%
\end{pgfscope}%
\end{pgfscope}%
\begin{pgfscope}%
\definecolor{textcolor}{rgb}{0.000000,0.000000,0.000000}%
\pgfsetstrokecolor{textcolor}%
\pgfsetfillcolor{textcolor}%
\pgftext[x=3.357975in,y=0.290000in,,top]{\color{textcolor}\sffamily\fontsize{10.000000}{12.000000}\selectfont 6}%
\end{pgfscope}%
\begin{pgfscope}%
\pgfsetbuttcap%
\pgfsetroundjoin%
\definecolor{currentfill}{rgb}{0.000000,0.000000,0.000000}%
\pgfsetfillcolor{currentfill}%
\pgfsetlinewidth{0.803000pt}%
\definecolor{currentstroke}{rgb}{0.000000,0.000000,0.000000}%
\pgfsetstrokecolor{currentstroke}%
\pgfsetdash{}{0pt}%
\pgfsys@defobject{currentmarker}{\pgfqpoint{-0.048611in}{0.000000in}}{\pgfqpoint{0.000000in}{0.000000in}}{%
\pgfpathmoveto{\pgfqpoint{0.000000in}{0.000000in}}%
\pgfpathlineto{\pgfqpoint{-0.048611in}{0.000000in}}%
\pgfusepath{stroke,fill}%
}%
\begin{pgfscope}%
\pgfsys@transformshift{0.584722in}{0.396654in}%
\pgfsys@useobject{currentmarker}{}%
\end{pgfscope}%
\end{pgfscope}%
\begin{pgfscope}%
\definecolor{textcolor}{rgb}{0.000000,0.000000,0.000000}%
\pgfsetstrokecolor{textcolor}%
\pgfsetfillcolor{textcolor}%
\pgftext[x=0.150247in,y=0.343893in,left,base]{\color{textcolor}\sffamily\fontsize{10.000000}{12.000000}\selectfont −3.0}%
\end{pgfscope}%
\begin{pgfscope}%
\pgfsetbuttcap%
\pgfsetroundjoin%
\definecolor{currentfill}{rgb}{0.000000,0.000000,0.000000}%
\pgfsetfillcolor{currentfill}%
\pgfsetlinewidth{0.803000pt}%
\definecolor{currentstroke}{rgb}{0.000000,0.000000,0.000000}%
\pgfsetstrokecolor{currentstroke}%
\pgfsetdash{}{0pt}%
\pgfsys@defobject{currentmarker}{\pgfqpoint{-0.048611in}{0.000000in}}{\pgfqpoint{0.000000in}{0.000000in}}{%
\pgfpathmoveto{\pgfqpoint{0.000000in}{0.000000in}}%
\pgfpathlineto{\pgfqpoint{-0.048611in}{0.000000in}}%
\pgfusepath{stroke,fill}%
}%
\begin{pgfscope}%
\pgfsys@transformshift{0.584722in}{0.839530in}%
\pgfsys@useobject{currentmarker}{}%
\end{pgfscope}%
\end{pgfscope}%
\begin{pgfscope}%
\definecolor{textcolor}{rgb}{0.000000,0.000000,0.000000}%
\pgfsetstrokecolor{textcolor}%
\pgfsetfillcolor{textcolor}%
\pgftext[x=0.150247in,y=0.786769in,left,base]{\color{textcolor}\sffamily\fontsize{10.000000}{12.000000}\selectfont −2.5}%
\end{pgfscope}%
\begin{pgfscope}%
\pgfsetbuttcap%
\pgfsetroundjoin%
\definecolor{currentfill}{rgb}{0.000000,0.000000,0.000000}%
\pgfsetfillcolor{currentfill}%
\pgfsetlinewidth{0.803000pt}%
\definecolor{currentstroke}{rgb}{0.000000,0.000000,0.000000}%
\pgfsetstrokecolor{currentstroke}%
\pgfsetdash{}{0pt}%
\pgfsys@defobject{currentmarker}{\pgfqpoint{-0.048611in}{0.000000in}}{\pgfqpoint{0.000000in}{0.000000in}}{%
\pgfpathmoveto{\pgfqpoint{0.000000in}{0.000000in}}%
\pgfpathlineto{\pgfqpoint{-0.048611in}{0.000000in}}%
\pgfusepath{stroke,fill}%
}%
\begin{pgfscope}%
\pgfsys@transformshift{0.584722in}{1.282406in}%
\pgfsys@useobject{currentmarker}{}%
\end{pgfscope}%
\end{pgfscope}%
\begin{pgfscope}%
\definecolor{textcolor}{rgb}{0.000000,0.000000,0.000000}%
\pgfsetstrokecolor{textcolor}%
\pgfsetfillcolor{textcolor}%
\pgftext[x=0.150247in,y=1.229645in,left,base]{\color{textcolor}\sffamily\fontsize{10.000000}{12.000000}\selectfont −2.0}%
\end{pgfscope}%
\begin{pgfscope}%
\pgfsetbuttcap%
\pgfsetroundjoin%
\definecolor{currentfill}{rgb}{0.000000,0.000000,0.000000}%
\pgfsetfillcolor{currentfill}%
\pgfsetlinewidth{0.803000pt}%
\definecolor{currentstroke}{rgb}{0.000000,0.000000,0.000000}%
\pgfsetstrokecolor{currentstroke}%
\pgfsetdash{}{0pt}%
\pgfsys@defobject{currentmarker}{\pgfqpoint{-0.048611in}{0.000000in}}{\pgfqpoint{0.000000in}{0.000000in}}{%
\pgfpathmoveto{\pgfqpoint{0.000000in}{0.000000in}}%
\pgfpathlineto{\pgfqpoint{-0.048611in}{0.000000in}}%
\pgfusepath{stroke,fill}%
}%
\begin{pgfscope}%
\pgfsys@transformshift{0.584722in}{1.725282in}%
\pgfsys@useobject{currentmarker}{}%
\end{pgfscope}%
\end{pgfscope}%
\begin{pgfscope}%
\definecolor{textcolor}{rgb}{0.000000,0.000000,0.000000}%
\pgfsetstrokecolor{textcolor}%
\pgfsetfillcolor{textcolor}%
\pgftext[x=0.150247in,y=1.672520in,left,base]{\color{textcolor}\sffamily\fontsize{10.000000}{12.000000}\selectfont −1.5}%
\end{pgfscope}%
\begin{pgfscope}%
\pgfsetbuttcap%
\pgfsetroundjoin%
\definecolor{currentfill}{rgb}{0.000000,0.000000,0.000000}%
\pgfsetfillcolor{currentfill}%
\pgfsetlinewidth{0.803000pt}%
\definecolor{currentstroke}{rgb}{0.000000,0.000000,0.000000}%
\pgfsetstrokecolor{currentstroke}%
\pgfsetdash{}{0pt}%
\pgfsys@defobject{currentmarker}{\pgfqpoint{-0.048611in}{0.000000in}}{\pgfqpoint{0.000000in}{0.000000in}}{%
\pgfpathmoveto{\pgfqpoint{0.000000in}{0.000000in}}%
\pgfpathlineto{\pgfqpoint{-0.048611in}{0.000000in}}%
\pgfusepath{stroke,fill}%
}%
\begin{pgfscope}%
\pgfsys@transformshift{0.584722in}{2.168158in}%
\pgfsys@useobject{currentmarker}{}%
\end{pgfscope}%
\end{pgfscope}%
\begin{pgfscope}%
\definecolor{textcolor}{rgb}{0.000000,0.000000,0.000000}%
\pgfsetstrokecolor{textcolor}%
\pgfsetfillcolor{textcolor}%
\pgftext[x=0.150247in,y=2.115396in,left,base]{\color{textcolor}\sffamily\fontsize{10.000000}{12.000000}\selectfont −1.0}%
\end{pgfscope}%
\begin{pgfscope}%
\pgfsetbuttcap%
\pgfsetroundjoin%
\definecolor{currentfill}{rgb}{0.000000,0.000000,0.000000}%
\pgfsetfillcolor{currentfill}%
\pgfsetlinewidth{0.803000pt}%
\definecolor{currentstroke}{rgb}{0.000000,0.000000,0.000000}%
\pgfsetstrokecolor{currentstroke}%
\pgfsetdash{}{0pt}%
\pgfsys@defobject{currentmarker}{\pgfqpoint{-0.048611in}{0.000000in}}{\pgfqpoint{0.000000in}{0.000000in}}{%
\pgfpathmoveto{\pgfqpoint{0.000000in}{0.000000in}}%
\pgfpathlineto{\pgfqpoint{-0.048611in}{0.000000in}}%
\pgfusepath{stroke,fill}%
}%
\begin{pgfscope}%
\pgfsys@transformshift{0.584722in}{2.611033in}%
\pgfsys@useobject{currentmarker}{}%
\end{pgfscope}%
\end{pgfscope}%
\begin{pgfscope}%
\definecolor{textcolor}{rgb}{0.000000,0.000000,0.000000}%
\pgfsetstrokecolor{textcolor}%
\pgfsetfillcolor{textcolor}%
\pgftext[x=0.150247in,y=2.558272in,left,base]{\color{textcolor}\sffamily\fontsize{10.000000}{12.000000}\selectfont −0.5}%
\end{pgfscope}%
\begin{pgfscope}%
\pgfpathrectangle{\pgfqpoint{0.584722in}{0.387222in}}{\pgfqpoint{3.305278in}{2.324444in}}%
\pgfusepath{clip}%
\pgfsetrectcap%
\pgfsetroundjoin%
\pgfsetlinewidth{1.505625pt}%
\definecolor{currentstroke}{rgb}{0.121569,0.466667,0.705882}%
\pgfsetstrokecolor{currentstroke}%
\pgfsetdash{}{0pt}%
\pgfpathmoveto{\pgfqpoint{0.743210in}{0.500649in}}%
\pgfpathlineto{\pgfqpoint{0.817917in}{0.524482in}}%
\pgfpathlineto{\pgfqpoint{0.892625in}{0.551819in}}%
\pgfpathlineto{\pgfqpoint{0.967333in}{0.582960in}}%
\pgfpathlineto{\pgfqpoint{1.042040in}{0.618080in}}%
\pgfpathlineto{\pgfqpoint{1.116748in}{0.657410in}}%
\pgfpathlineto{\pgfqpoint{1.191455in}{0.701245in}}%
\pgfpathlineto{\pgfqpoint{1.266163in}{0.750023in}}%
\pgfpathlineto{\pgfqpoint{1.340870in}{0.803905in}}%
\pgfpathlineto{\pgfqpoint{1.415578in}{0.863246in}}%
\pgfpathlineto{\pgfqpoint{1.490285in}{0.928632in}}%
\pgfpathlineto{\pgfqpoint{1.564993in}{1.000513in}}%
\pgfpathlineto{\pgfqpoint{1.639701in}{1.078977in}}%
\pgfpathlineto{\pgfqpoint{1.714408in}{1.164965in}}%
\pgfpathlineto{\pgfqpoint{1.789116in}{1.259751in}}%
\pgfpathlineto{\pgfqpoint{1.863823in}{1.363797in}}%
\pgfpathlineto{\pgfqpoint{1.938531in}{1.478137in}}%
\pgfpathlineto{\pgfqpoint{2.013238in}{1.606301in}}%
\pgfpathlineto{\pgfqpoint{2.087946in}{1.750016in}}%
\pgfpathlineto{\pgfqpoint{2.162654in}{1.918650in}}%
\pgfpathlineto{\pgfqpoint{2.237361in}{2.117482in}}%
\pgfpathlineto{\pgfqpoint{2.312069in}{2.232906in}}%
\pgfpathlineto{\pgfqpoint{2.386776in}{2.287615in}}%
\pgfpathlineto{\pgfqpoint{2.461484in}{2.324929in}}%
\pgfpathlineto{\pgfqpoint{2.536191in}{2.355943in}}%
\pgfpathlineto{\pgfqpoint{2.610899in}{2.382479in}}%
\pgfpathlineto{\pgfqpoint{2.685606in}{2.405925in}}%
\pgfpathlineto{\pgfqpoint{2.760314in}{2.427126in}}%
\pgfpathlineto{\pgfqpoint{2.835022in}{2.446348in}}%
\pgfpathlineto{\pgfqpoint{2.909729in}{2.464109in}}%
\pgfpathlineto{\pgfqpoint{2.984437in}{2.480502in}}%
\pgfpathlineto{\pgfqpoint{3.059144in}{2.495759in}}%
\pgfpathlineto{\pgfqpoint{3.133852in}{2.509895in}}%
\pgfpathlineto{\pgfqpoint{3.208559in}{2.523149in}}%
\pgfpathlineto{\pgfqpoint{3.283267in}{2.535768in}}%
\pgfpathlineto{\pgfqpoint{3.357975in}{2.547660in}}%
\pgfpathlineto{\pgfqpoint{3.432682in}{2.558732in}}%
\pgfpathlineto{\pgfqpoint{3.507390in}{2.569407in}}%
\pgfpathlineto{\pgfqpoint{3.582097in}{2.579358in}}%
\pgfpathlineto{\pgfqpoint{3.656805in}{2.589147in}}%
\pgfpathlineto{\pgfqpoint{3.731512in}{2.598240in}}%
\pgfusepath{stroke}%
\end{pgfscope}%
\begin{pgfscope}%
\pgfsetrectcap%
\pgfsetmiterjoin%
\pgfsetlinewidth{0.803000pt}%
\definecolor{currentstroke}{rgb}{0.000000,0.000000,0.000000}%
\pgfsetstrokecolor{currentstroke}%
\pgfsetdash{}{0pt}%
\pgfpathmoveto{\pgfqpoint{0.584722in}{0.387222in}}%
\pgfpathlineto{\pgfqpoint{0.584722in}{2.711667in}}%
\pgfusepath{stroke}%
\end{pgfscope}%
\begin{pgfscope}%
\pgfsetrectcap%
\pgfsetmiterjoin%
\pgfsetlinewidth{0.803000pt}%
\definecolor{currentstroke}{rgb}{0.000000,0.000000,0.000000}%
\pgfsetstrokecolor{currentstroke}%
\pgfsetdash{}{0pt}%
\pgfpathmoveto{\pgfqpoint{3.890000in}{0.387222in}}%
\pgfpathlineto{\pgfqpoint{3.890000in}{2.711667in}}%
\pgfusepath{stroke}%
\end{pgfscope}%
\begin{pgfscope}%
\pgfsetrectcap%
\pgfsetmiterjoin%
\pgfsetlinewidth{0.803000pt}%
\definecolor{currentstroke}{rgb}{0.000000,0.000000,0.000000}%
\pgfsetstrokecolor{currentstroke}%
\pgfsetdash{}{0pt}%
\pgfpathmoveto{\pgfqpoint{0.584722in}{0.387222in}}%
\pgfpathlineto{\pgfqpoint{3.890000in}{0.387222in}}%
\pgfusepath{stroke}%
\end{pgfscope}%
\begin{pgfscope}%
\pgfsetrectcap%
\pgfsetmiterjoin%
\pgfsetlinewidth{0.803000pt}%
\definecolor{currentstroke}{rgb}{0.000000,0.000000,0.000000}%
\pgfsetstrokecolor{currentstroke}%
\pgfsetdash{}{0pt}%
\pgfpathmoveto{\pgfqpoint{0.584722in}{2.711667in}}%
\pgfpathlineto{\pgfqpoint{3.890000in}{2.711667in}}%
\pgfusepath{stroke}%
\end{pgfscope}%
\begin{pgfscope}%
\definecolor{textcolor}{rgb}{0.000000,0.000000,0.000000}%
\pgfsetstrokecolor{textcolor}%
\pgfsetfillcolor{textcolor}%
\pgftext[x=2.237361in,y=2.795000in,,base]{\color{textcolor}\sffamily\fontsize{12.000000}{14.400000}\selectfont \(\displaystyle  N = 16 \)}%
\end{pgfscope}%
\begin{pgfscope}%
\pgfsetbuttcap%
\pgfsetmiterjoin%
\definecolor{currentfill}{rgb}{1.000000,1.000000,1.000000}%
\pgfsetfillcolor{currentfill}%
\pgfsetlinewidth{0.000000pt}%
\definecolor{currentstroke}{rgb}{0.000000,0.000000,0.000000}%
\pgfsetstrokecolor{currentstroke}%
\pgfsetstrokeopacity{0.000000}%
\pgfsetdash{}{0pt}%
\pgfpathmoveto{\pgfqpoint{4.509722in}{0.387222in}}%
\pgfpathlineto{\pgfqpoint{7.815000in}{0.387222in}}%
\pgfpathlineto{\pgfqpoint{7.815000in}{2.711667in}}%
\pgfpathlineto{\pgfqpoint{4.509722in}{2.711667in}}%
\pgfpathclose%
\pgfusepath{fill}%
\end{pgfscope}%
\begin{pgfscope}%
\pgfpathrectangle{\pgfqpoint{4.509722in}{0.387222in}}{\pgfqpoint{3.305278in}{2.324444in}}%
\pgfusepath{clip}%
\pgfsetbuttcap%
\pgfsetroundjoin%
\definecolor{currentfill}{rgb}{0.121569,0.466667,0.705882}%
\pgfsetfillcolor{currentfill}%
\pgfsetfillopacity{0.300000}%
\pgfsetlinewidth{1.003750pt}%
\definecolor{currentstroke}{rgb}{0.121569,0.466667,0.705882}%
\pgfsetstrokecolor{currentstroke}%
\pgfsetstrokeopacity{0.300000}%
\pgfsetdash{}{0pt}%
\pgfpathmoveto{\pgfqpoint{4.668210in}{0.500696in}}%
\pgfpathlineto{\pgfqpoint{4.668210in}{0.500602in}}%
\pgfpathlineto{\pgfqpoint{4.742917in}{0.524445in}}%
\pgfpathlineto{\pgfqpoint{4.817625in}{0.551776in}}%
\pgfpathlineto{\pgfqpoint{4.892333in}{0.582958in}}%
\pgfpathlineto{\pgfqpoint{4.967040in}{0.617927in}}%
\pgfpathlineto{\pgfqpoint{5.041748in}{0.657352in}}%
\pgfpathlineto{\pgfqpoint{5.116455in}{0.701188in}}%
\pgfpathlineto{\pgfqpoint{5.191163in}{0.749860in}}%
\pgfpathlineto{\pgfqpoint{5.265870in}{0.803752in}}%
\pgfpathlineto{\pgfqpoint{5.340578in}{0.863051in}}%
\pgfpathlineto{\pgfqpoint{5.415285in}{0.928359in}}%
\pgfpathlineto{\pgfqpoint{5.489993in}{1.000328in}}%
\pgfpathlineto{\pgfqpoint{5.564701in}{1.078933in}}%
\pgfpathlineto{\pgfqpoint{5.639408in}{1.164618in}}%
\pgfpathlineto{\pgfqpoint{5.714116in}{1.259684in}}%
\pgfpathlineto{\pgfqpoint{5.788823in}{1.363807in}}%
\pgfpathlineto{\pgfqpoint{5.863531in}{1.477803in}}%
\pgfpathlineto{\pgfqpoint{5.938238in}{1.604862in}}%
\pgfpathlineto{\pgfqpoint{6.012946in}{1.748715in}}%
\pgfpathlineto{\pgfqpoint{6.087654in}{1.916958in}}%
\pgfpathlineto{\pgfqpoint{6.162361in}{2.128196in}}%
\pgfpathlineto{\pgfqpoint{6.237069in}{2.240589in}}%
\pgfpathlineto{\pgfqpoint{6.311776in}{2.289045in}}%
\pgfpathlineto{\pgfqpoint{6.386484in}{2.325191in}}%
\pgfpathlineto{\pgfqpoint{6.461191in}{2.356044in}}%
\pgfpathlineto{\pgfqpoint{6.535899in}{2.382408in}}%
\pgfpathlineto{\pgfqpoint{6.610606in}{2.405572in}}%
\pgfpathlineto{\pgfqpoint{6.685314in}{2.427001in}}%
\pgfpathlineto{\pgfqpoint{6.760022in}{2.446299in}}%
\pgfpathlineto{\pgfqpoint{6.834729in}{2.463991in}}%
\pgfpathlineto{\pgfqpoint{6.909437in}{2.480226in}}%
\pgfpathlineto{\pgfqpoint{6.984144in}{2.495496in}}%
\pgfpathlineto{\pgfqpoint{7.058852in}{2.509811in}}%
\pgfpathlineto{\pgfqpoint{7.133559in}{2.523306in}}%
\pgfpathlineto{\pgfqpoint{7.208267in}{2.535406in}}%
\pgfpathlineto{\pgfqpoint{7.282975in}{2.547521in}}%
\pgfpathlineto{\pgfqpoint{7.357682in}{2.558538in}}%
\pgfpathlineto{\pgfqpoint{7.432390in}{2.569275in}}%
\pgfpathlineto{\pgfqpoint{7.507097in}{2.579279in}}%
\pgfpathlineto{\pgfqpoint{7.581805in}{2.588974in}}%
\pgfpathlineto{\pgfqpoint{7.656512in}{2.598013in}}%
\pgfpathlineto{\pgfqpoint{7.656512in}{2.598466in}}%
\pgfpathlineto{\pgfqpoint{7.656512in}{2.598466in}}%
\pgfpathlineto{\pgfqpoint{7.581805in}{2.589321in}}%
\pgfpathlineto{\pgfqpoint{7.507097in}{2.579767in}}%
\pgfpathlineto{\pgfqpoint{7.432390in}{2.569806in}}%
\pgfpathlineto{\pgfqpoint{7.357682in}{2.558963in}}%
\pgfpathlineto{\pgfqpoint{7.282975in}{2.547867in}}%
\pgfpathlineto{\pgfqpoint{7.208267in}{2.536049in}}%
\pgfpathlineto{\pgfqpoint{7.133559in}{2.523547in}}%
\pgfpathlineto{\pgfqpoint{7.058852in}{2.510296in}}%
\pgfpathlineto{\pgfqpoint{6.984144in}{2.495783in}}%
\pgfpathlineto{\pgfqpoint{6.909437in}{2.480637in}}%
\pgfpathlineto{\pgfqpoint{6.834729in}{2.464617in}}%
\pgfpathlineto{\pgfqpoint{6.760022in}{2.447020in}}%
\pgfpathlineto{\pgfqpoint{6.685314in}{2.427864in}}%
\pgfpathlineto{\pgfqpoint{6.610606in}{2.406817in}}%
\pgfpathlineto{\pgfqpoint{6.535899in}{2.382987in}}%
\pgfpathlineto{\pgfqpoint{6.461191in}{2.356841in}}%
\pgfpathlineto{\pgfqpoint{6.386484in}{2.326289in}}%
\pgfpathlineto{\pgfqpoint{6.311776in}{2.289951in}}%
\pgfpathlineto{\pgfqpoint{6.237069in}{2.243942in}}%
\pgfpathlineto{\pgfqpoint{6.162361in}{2.133953in}}%
\pgfpathlineto{\pgfqpoint{6.087654in}{1.922132in}}%
\pgfpathlineto{\pgfqpoint{6.012946in}{1.750908in}}%
\pgfpathlineto{\pgfqpoint{5.938238in}{1.606349in}}%
\pgfpathlineto{\pgfqpoint{5.863531in}{1.478786in}}%
\pgfpathlineto{\pgfqpoint{5.788823in}{1.364289in}}%
\pgfpathlineto{\pgfqpoint{5.714116in}{1.260132in}}%
\pgfpathlineto{\pgfqpoint{5.639408in}{1.165196in}}%
\pgfpathlineto{\pgfqpoint{5.564701in}{1.079518in}}%
\pgfpathlineto{\pgfqpoint{5.489993in}{1.000678in}}%
\pgfpathlineto{\pgfqpoint{5.415285in}{0.928600in}}%
\pgfpathlineto{\pgfqpoint{5.340578in}{0.863504in}}%
\pgfpathlineto{\pgfqpoint{5.265870in}{0.804113in}}%
\pgfpathlineto{\pgfqpoint{5.191163in}{0.750162in}}%
\pgfpathlineto{\pgfqpoint{5.116455in}{0.701483in}}%
\pgfpathlineto{\pgfqpoint{5.041748in}{0.657457in}}%
\pgfpathlineto{\pgfqpoint{4.967040in}{0.618134in}}%
\pgfpathlineto{\pgfqpoint{4.892333in}{0.582999in}}%
\pgfpathlineto{\pgfqpoint{4.817625in}{0.551927in}}%
\pgfpathlineto{\pgfqpoint{4.742917in}{0.524529in}}%
\pgfpathlineto{\pgfqpoint{4.668210in}{0.500696in}}%
\pgfpathclose%
\pgfusepath{stroke,fill}%
\end{pgfscope}%
\begin{pgfscope}%
\pgfpathrectangle{\pgfqpoint{4.509722in}{0.387222in}}{\pgfqpoint{3.305278in}{2.324444in}}%
\pgfusepath{clip}%
\pgfsetbuttcap%
\pgfsetroundjoin%
\definecolor{currentfill}{rgb}{0.121569,0.466667,0.705882}%
\pgfsetfillcolor{currentfill}%
\pgfsetlinewidth{1.003750pt}%
\definecolor{currentstroke}{rgb}{0.121569,0.466667,0.705882}%
\pgfsetstrokecolor{currentstroke}%
\pgfsetdash{}{0pt}%
\pgfsys@defobject{currentmarker}{\pgfqpoint{-0.009821in}{-0.009821in}}{\pgfqpoint{0.009821in}{0.009821in}}{%
\pgfpathmoveto{\pgfqpoint{0.000000in}{-0.009821in}}%
\pgfpathcurveto{\pgfqpoint{0.002605in}{-0.009821in}}{\pgfqpoint{0.005103in}{-0.008786in}}{\pgfqpoint{0.006944in}{-0.006944in}}%
\pgfpathcurveto{\pgfqpoint{0.008786in}{-0.005103in}}{\pgfqpoint{0.009821in}{-0.002605in}}{\pgfqpoint{0.009821in}{0.000000in}}%
\pgfpathcurveto{\pgfqpoint{0.009821in}{0.002605in}}{\pgfqpoint{0.008786in}{0.005103in}}{\pgfqpoint{0.006944in}{0.006944in}}%
\pgfpathcurveto{\pgfqpoint{0.005103in}{0.008786in}}{\pgfqpoint{0.002605in}{0.009821in}}{\pgfqpoint{0.000000in}{0.009821in}}%
\pgfpathcurveto{\pgfqpoint{-0.002605in}{0.009821in}}{\pgfqpoint{-0.005103in}{0.008786in}}{\pgfqpoint{-0.006944in}{0.006944in}}%
\pgfpathcurveto{\pgfqpoint{-0.008786in}{0.005103in}}{\pgfqpoint{-0.009821in}{0.002605in}}{\pgfqpoint{-0.009821in}{0.000000in}}%
\pgfpathcurveto{\pgfqpoint{-0.009821in}{-0.002605in}}{\pgfqpoint{-0.008786in}{-0.005103in}}{\pgfqpoint{-0.006944in}{-0.006944in}}%
\pgfpathcurveto{\pgfqpoint{-0.005103in}{-0.008786in}}{\pgfqpoint{-0.002605in}{-0.009821in}}{\pgfqpoint{0.000000in}{-0.009821in}}%
\pgfpathclose%
\pgfusepath{stroke,fill}%
}%
\begin{pgfscope}%
\pgfsys@transformshift{4.668210in}{0.500649in}%
\pgfsys@useobject{currentmarker}{}%
\end{pgfscope}%
\begin{pgfscope}%
\pgfsys@transformshift{4.742917in}{0.524487in}%
\pgfsys@useobject{currentmarker}{}%
\end{pgfscope}%
\begin{pgfscope}%
\pgfsys@transformshift{4.817625in}{0.551851in}%
\pgfsys@useobject{currentmarker}{}%
\end{pgfscope}%
\begin{pgfscope}%
\pgfsys@transformshift{4.892333in}{0.582978in}%
\pgfsys@useobject{currentmarker}{}%
\end{pgfscope}%
\begin{pgfscope}%
\pgfsys@transformshift{4.967040in}{0.618030in}%
\pgfsys@useobject{currentmarker}{}%
\end{pgfscope}%
\begin{pgfscope}%
\pgfsys@transformshift{5.041748in}{0.657404in}%
\pgfsys@useobject{currentmarker}{}%
\end{pgfscope}%
\begin{pgfscope}%
\pgfsys@transformshift{5.116455in}{0.701335in}%
\pgfsys@useobject{currentmarker}{}%
\end{pgfscope}%
\begin{pgfscope}%
\pgfsys@transformshift{5.191163in}{0.750011in}%
\pgfsys@useobject{currentmarker}{}%
\end{pgfscope}%
\begin{pgfscope}%
\pgfsys@transformshift{5.265870in}{0.803933in}%
\pgfsys@useobject{currentmarker}{}%
\end{pgfscope}%
\begin{pgfscope}%
\pgfsys@transformshift{5.340578in}{0.863277in}%
\pgfsys@useobject{currentmarker}{}%
\end{pgfscope}%
\begin{pgfscope}%
\pgfsys@transformshift{5.415285in}{0.928479in}%
\pgfsys@useobject{currentmarker}{}%
\end{pgfscope}%
\begin{pgfscope}%
\pgfsys@transformshift{5.489993in}{1.000503in}%
\pgfsys@useobject{currentmarker}{}%
\end{pgfscope}%
\begin{pgfscope}%
\pgfsys@transformshift{5.564701in}{1.079225in}%
\pgfsys@useobject{currentmarker}{}%
\end{pgfscope}%
\begin{pgfscope}%
\pgfsys@transformshift{5.639408in}{1.164907in}%
\pgfsys@useobject{currentmarker}{}%
\end{pgfscope}%
\begin{pgfscope}%
\pgfsys@transformshift{5.714116in}{1.259908in}%
\pgfsys@useobject{currentmarker}{}%
\end{pgfscope}%
\begin{pgfscope}%
\pgfsys@transformshift{5.788823in}{1.364048in}%
\pgfsys@useobject{currentmarker}{}%
\end{pgfscope}%
\begin{pgfscope}%
\pgfsys@transformshift{5.863531in}{1.478294in}%
\pgfsys@useobject{currentmarker}{}%
\end{pgfscope}%
\begin{pgfscope}%
\pgfsys@transformshift{5.938238in}{1.605606in}%
\pgfsys@useobject{currentmarker}{}%
\end{pgfscope}%
\begin{pgfscope}%
\pgfsys@transformshift{6.012946in}{1.749811in}%
\pgfsys@useobject{currentmarker}{}%
\end{pgfscope}%
\begin{pgfscope}%
\pgfsys@transformshift{6.087654in}{1.919545in}%
\pgfsys@useobject{currentmarker}{}%
\end{pgfscope}%
\begin{pgfscope}%
\pgfsys@transformshift{6.162361in}{2.131074in}%
\pgfsys@useobject{currentmarker}{}%
\end{pgfscope}%
\begin{pgfscope}%
\pgfsys@transformshift{6.237069in}{2.242266in}%
\pgfsys@useobject{currentmarker}{}%
\end{pgfscope}%
\begin{pgfscope}%
\pgfsys@transformshift{6.311776in}{2.289498in}%
\pgfsys@useobject{currentmarker}{}%
\end{pgfscope}%
\begin{pgfscope}%
\pgfsys@transformshift{6.386484in}{2.325740in}%
\pgfsys@useobject{currentmarker}{}%
\end{pgfscope}%
\begin{pgfscope}%
\pgfsys@transformshift{6.461191in}{2.356442in}%
\pgfsys@useobject{currentmarker}{}%
\end{pgfscope}%
\begin{pgfscope}%
\pgfsys@transformshift{6.535899in}{2.382697in}%
\pgfsys@useobject{currentmarker}{}%
\end{pgfscope}%
\begin{pgfscope}%
\pgfsys@transformshift{6.610606in}{2.406194in}%
\pgfsys@useobject{currentmarker}{}%
\end{pgfscope}%
\begin{pgfscope}%
\pgfsys@transformshift{6.685314in}{2.427432in}%
\pgfsys@useobject{currentmarker}{}%
\end{pgfscope}%
\begin{pgfscope}%
\pgfsys@transformshift{6.760022in}{2.446659in}%
\pgfsys@useobject{currentmarker}{}%
\end{pgfscope}%
\begin{pgfscope}%
\pgfsys@transformshift{6.834729in}{2.464304in}%
\pgfsys@useobject{currentmarker}{}%
\end{pgfscope}%
\begin{pgfscope}%
\pgfsys@transformshift{6.909437in}{2.480431in}%
\pgfsys@useobject{currentmarker}{}%
\end{pgfscope}%
\begin{pgfscope}%
\pgfsys@transformshift{6.984144in}{2.495640in}%
\pgfsys@useobject{currentmarker}{}%
\end{pgfscope}%
\begin{pgfscope}%
\pgfsys@transformshift{7.058852in}{2.510054in}%
\pgfsys@useobject{currentmarker}{}%
\end{pgfscope}%
\begin{pgfscope}%
\pgfsys@transformshift{7.133559in}{2.523426in}%
\pgfsys@useobject{currentmarker}{}%
\end{pgfscope}%
\begin{pgfscope}%
\pgfsys@transformshift{7.208267in}{2.535727in}%
\pgfsys@useobject{currentmarker}{}%
\end{pgfscope}%
\begin{pgfscope}%
\pgfsys@transformshift{7.282975in}{2.547694in}%
\pgfsys@useobject{currentmarker}{}%
\end{pgfscope}%
\begin{pgfscope}%
\pgfsys@transformshift{7.357682in}{2.558750in}%
\pgfsys@useobject{currentmarker}{}%
\end{pgfscope}%
\begin{pgfscope}%
\pgfsys@transformshift{7.432390in}{2.569541in}%
\pgfsys@useobject{currentmarker}{}%
\end{pgfscope}%
\begin{pgfscope}%
\pgfsys@transformshift{7.507097in}{2.579523in}%
\pgfsys@useobject{currentmarker}{}%
\end{pgfscope}%
\begin{pgfscope}%
\pgfsys@transformshift{7.581805in}{2.589148in}%
\pgfsys@useobject{currentmarker}{}%
\end{pgfscope}%
\begin{pgfscope}%
\pgfsys@transformshift{7.656512in}{2.598240in}%
\pgfsys@useobject{currentmarker}{}%
\end{pgfscope}%
\end{pgfscope}%
\begin{pgfscope}%
\pgfsetbuttcap%
\pgfsetroundjoin%
\definecolor{currentfill}{rgb}{0.000000,0.000000,0.000000}%
\pgfsetfillcolor{currentfill}%
\pgfsetlinewidth{0.803000pt}%
\definecolor{currentstroke}{rgb}{0.000000,0.000000,0.000000}%
\pgfsetstrokecolor{currentstroke}%
\pgfsetdash{}{0pt}%
\pgfsys@defobject{currentmarker}{\pgfqpoint{0.000000in}{-0.048611in}}{\pgfqpoint{0.000000in}{0.000000in}}{%
\pgfpathmoveto{\pgfqpoint{0.000000in}{0.000000in}}%
\pgfpathlineto{\pgfqpoint{0.000000in}{-0.048611in}}%
\pgfusepath{stroke,fill}%
}%
\begin{pgfscope}%
\pgfsys@transformshift{5.041748in}{0.387222in}%
\pgfsys@useobject{currentmarker}{}%
\end{pgfscope}%
\end{pgfscope}%
\begin{pgfscope}%
\definecolor{textcolor}{rgb}{0.000000,0.000000,0.000000}%
\pgfsetstrokecolor{textcolor}%
\pgfsetfillcolor{textcolor}%
\pgftext[x=5.041748in,y=0.290000in,,top]{\color{textcolor}\sffamily\fontsize{10.000000}{12.000000}\selectfont 3}%
\end{pgfscope}%
\begin{pgfscope}%
\pgfsetbuttcap%
\pgfsetroundjoin%
\definecolor{currentfill}{rgb}{0.000000,0.000000,0.000000}%
\pgfsetfillcolor{currentfill}%
\pgfsetlinewidth{0.803000pt}%
\definecolor{currentstroke}{rgb}{0.000000,0.000000,0.000000}%
\pgfsetstrokecolor{currentstroke}%
\pgfsetdash{}{0pt}%
\pgfsys@defobject{currentmarker}{\pgfqpoint{0.000000in}{-0.048611in}}{\pgfqpoint{0.000000in}{0.000000in}}{%
\pgfpathmoveto{\pgfqpoint{0.000000in}{0.000000in}}%
\pgfpathlineto{\pgfqpoint{0.000000in}{-0.048611in}}%
\pgfusepath{stroke,fill}%
}%
\begin{pgfscope}%
\pgfsys@transformshift{5.788823in}{0.387222in}%
\pgfsys@useobject{currentmarker}{}%
\end{pgfscope}%
\end{pgfscope}%
\begin{pgfscope}%
\definecolor{textcolor}{rgb}{0.000000,0.000000,0.000000}%
\pgfsetstrokecolor{textcolor}%
\pgfsetfillcolor{textcolor}%
\pgftext[x=5.788823in,y=0.290000in,,top]{\color{textcolor}\sffamily\fontsize{10.000000}{12.000000}\selectfont 4}%
\end{pgfscope}%
\begin{pgfscope}%
\pgfsetbuttcap%
\pgfsetroundjoin%
\definecolor{currentfill}{rgb}{0.000000,0.000000,0.000000}%
\pgfsetfillcolor{currentfill}%
\pgfsetlinewidth{0.803000pt}%
\definecolor{currentstroke}{rgb}{0.000000,0.000000,0.000000}%
\pgfsetstrokecolor{currentstroke}%
\pgfsetdash{}{0pt}%
\pgfsys@defobject{currentmarker}{\pgfqpoint{0.000000in}{-0.048611in}}{\pgfqpoint{0.000000in}{0.000000in}}{%
\pgfpathmoveto{\pgfqpoint{0.000000in}{0.000000in}}%
\pgfpathlineto{\pgfqpoint{0.000000in}{-0.048611in}}%
\pgfusepath{stroke,fill}%
}%
\begin{pgfscope}%
\pgfsys@transformshift{6.535899in}{0.387222in}%
\pgfsys@useobject{currentmarker}{}%
\end{pgfscope}%
\end{pgfscope}%
\begin{pgfscope}%
\definecolor{textcolor}{rgb}{0.000000,0.000000,0.000000}%
\pgfsetstrokecolor{textcolor}%
\pgfsetfillcolor{textcolor}%
\pgftext[x=6.535899in,y=0.290000in,,top]{\color{textcolor}\sffamily\fontsize{10.000000}{12.000000}\selectfont 5}%
\end{pgfscope}%
\begin{pgfscope}%
\pgfsetbuttcap%
\pgfsetroundjoin%
\definecolor{currentfill}{rgb}{0.000000,0.000000,0.000000}%
\pgfsetfillcolor{currentfill}%
\pgfsetlinewidth{0.803000pt}%
\definecolor{currentstroke}{rgb}{0.000000,0.000000,0.000000}%
\pgfsetstrokecolor{currentstroke}%
\pgfsetdash{}{0pt}%
\pgfsys@defobject{currentmarker}{\pgfqpoint{0.000000in}{-0.048611in}}{\pgfqpoint{0.000000in}{0.000000in}}{%
\pgfpathmoveto{\pgfqpoint{0.000000in}{0.000000in}}%
\pgfpathlineto{\pgfqpoint{0.000000in}{-0.048611in}}%
\pgfusepath{stroke,fill}%
}%
\begin{pgfscope}%
\pgfsys@transformshift{7.282975in}{0.387222in}%
\pgfsys@useobject{currentmarker}{}%
\end{pgfscope}%
\end{pgfscope}%
\begin{pgfscope}%
\definecolor{textcolor}{rgb}{0.000000,0.000000,0.000000}%
\pgfsetstrokecolor{textcolor}%
\pgfsetfillcolor{textcolor}%
\pgftext[x=7.282975in,y=0.290000in,,top]{\color{textcolor}\sffamily\fontsize{10.000000}{12.000000}\selectfont 6}%
\end{pgfscope}%
\begin{pgfscope}%
\pgfsetbuttcap%
\pgfsetroundjoin%
\definecolor{currentfill}{rgb}{0.000000,0.000000,0.000000}%
\pgfsetfillcolor{currentfill}%
\pgfsetlinewidth{0.803000pt}%
\definecolor{currentstroke}{rgb}{0.000000,0.000000,0.000000}%
\pgfsetstrokecolor{currentstroke}%
\pgfsetdash{}{0pt}%
\pgfsys@defobject{currentmarker}{\pgfqpoint{-0.048611in}{0.000000in}}{\pgfqpoint{0.000000in}{0.000000in}}{%
\pgfpathmoveto{\pgfqpoint{0.000000in}{0.000000in}}%
\pgfpathlineto{\pgfqpoint{-0.048611in}{0.000000in}}%
\pgfusepath{stroke,fill}%
}%
\begin{pgfscope}%
\pgfsys@transformshift{4.509722in}{0.396679in}%
\pgfsys@useobject{currentmarker}{}%
\end{pgfscope}%
\end{pgfscope}%
\begin{pgfscope}%
\definecolor{textcolor}{rgb}{0.000000,0.000000,0.000000}%
\pgfsetstrokecolor{textcolor}%
\pgfsetfillcolor{textcolor}%
\pgftext[x=4.075247in,y=0.343917in,left,base]{\color{textcolor}\sffamily\fontsize{10.000000}{12.000000}\selectfont −3.0}%
\end{pgfscope}%
\begin{pgfscope}%
\pgfsetbuttcap%
\pgfsetroundjoin%
\definecolor{currentfill}{rgb}{0.000000,0.000000,0.000000}%
\pgfsetfillcolor{currentfill}%
\pgfsetlinewidth{0.803000pt}%
\definecolor{currentstroke}{rgb}{0.000000,0.000000,0.000000}%
\pgfsetstrokecolor{currentstroke}%
\pgfsetdash{}{0pt}%
\pgfsys@defobject{currentmarker}{\pgfqpoint{-0.048611in}{0.000000in}}{\pgfqpoint{0.000000in}{0.000000in}}{%
\pgfpathmoveto{\pgfqpoint{0.000000in}{0.000000in}}%
\pgfpathlineto{\pgfqpoint{-0.048611in}{0.000000in}}%
\pgfusepath{stroke,fill}%
}%
\begin{pgfscope}%
\pgfsys@transformshift{4.509722in}{0.839568in}%
\pgfsys@useobject{currentmarker}{}%
\end{pgfscope}%
\end{pgfscope}%
\begin{pgfscope}%
\definecolor{textcolor}{rgb}{0.000000,0.000000,0.000000}%
\pgfsetstrokecolor{textcolor}%
\pgfsetfillcolor{textcolor}%
\pgftext[x=4.075247in,y=0.786806in,left,base]{\color{textcolor}\sffamily\fontsize{10.000000}{12.000000}\selectfont −2.5}%
\end{pgfscope}%
\begin{pgfscope}%
\pgfsetbuttcap%
\pgfsetroundjoin%
\definecolor{currentfill}{rgb}{0.000000,0.000000,0.000000}%
\pgfsetfillcolor{currentfill}%
\pgfsetlinewidth{0.803000pt}%
\definecolor{currentstroke}{rgb}{0.000000,0.000000,0.000000}%
\pgfsetstrokecolor{currentstroke}%
\pgfsetdash{}{0pt}%
\pgfsys@defobject{currentmarker}{\pgfqpoint{-0.048611in}{0.000000in}}{\pgfqpoint{0.000000in}{0.000000in}}{%
\pgfpathmoveto{\pgfqpoint{0.000000in}{0.000000in}}%
\pgfpathlineto{\pgfqpoint{-0.048611in}{0.000000in}}%
\pgfusepath{stroke,fill}%
}%
\begin{pgfscope}%
\pgfsys@transformshift{4.509722in}{1.282457in}%
\pgfsys@useobject{currentmarker}{}%
\end{pgfscope}%
\end{pgfscope}%
\begin{pgfscope}%
\definecolor{textcolor}{rgb}{0.000000,0.000000,0.000000}%
\pgfsetstrokecolor{textcolor}%
\pgfsetfillcolor{textcolor}%
\pgftext[x=4.075247in,y=1.229695in,left,base]{\color{textcolor}\sffamily\fontsize{10.000000}{12.000000}\selectfont −2.0}%
\end{pgfscope}%
\begin{pgfscope}%
\pgfsetbuttcap%
\pgfsetroundjoin%
\definecolor{currentfill}{rgb}{0.000000,0.000000,0.000000}%
\pgfsetfillcolor{currentfill}%
\pgfsetlinewidth{0.803000pt}%
\definecolor{currentstroke}{rgb}{0.000000,0.000000,0.000000}%
\pgfsetstrokecolor{currentstroke}%
\pgfsetdash{}{0pt}%
\pgfsys@defobject{currentmarker}{\pgfqpoint{-0.048611in}{0.000000in}}{\pgfqpoint{0.000000in}{0.000000in}}{%
\pgfpathmoveto{\pgfqpoint{0.000000in}{0.000000in}}%
\pgfpathlineto{\pgfqpoint{-0.048611in}{0.000000in}}%
\pgfusepath{stroke,fill}%
}%
\begin{pgfscope}%
\pgfsys@transformshift{4.509722in}{1.725346in}%
\pgfsys@useobject{currentmarker}{}%
\end{pgfscope}%
\end{pgfscope}%
\begin{pgfscope}%
\definecolor{textcolor}{rgb}{0.000000,0.000000,0.000000}%
\pgfsetstrokecolor{textcolor}%
\pgfsetfillcolor{textcolor}%
\pgftext[x=4.075247in,y=1.672584in,left,base]{\color{textcolor}\sffamily\fontsize{10.000000}{12.000000}\selectfont −1.5}%
\end{pgfscope}%
\begin{pgfscope}%
\pgfsetbuttcap%
\pgfsetroundjoin%
\definecolor{currentfill}{rgb}{0.000000,0.000000,0.000000}%
\pgfsetfillcolor{currentfill}%
\pgfsetlinewidth{0.803000pt}%
\definecolor{currentstroke}{rgb}{0.000000,0.000000,0.000000}%
\pgfsetstrokecolor{currentstroke}%
\pgfsetdash{}{0pt}%
\pgfsys@defobject{currentmarker}{\pgfqpoint{-0.048611in}{0.000000in}}{\pgfqpoint{0.000000in}{0.000000in}}{%
\pgfpathmoveto{\pgfqpoint{0.000000in}{0.000000in}}%
\pgfpathlineto{\pgfqpoint{-0.048611in}{0.000000in}}%
\pgfusepath{stroke,fill}%
}%
\begin{pgfscope}%
\pgfsys@transformshift{4.509722in}{2.168235in}%
\pgfsys@useobject{currentmarker}{}%
\end{pgfscope}%
\end{pgfscope}%
\begin{pgfscope}%
\definecolor{textcolor}{rgb}{0.000000,0.000000,0.000000}%
\pgfsetstrokecolor{textcolor}%
\pgfsetfillcolor{textcolor}%
\pgftext[x=4.075247in,y=2.115473in,left,base]{\color{textcolor}\sffamily\fontsize{10.000000}{12.000000}\selectfont −1.0}%
\end{pgfscope}%
\begin{pgfscope}%
\pgfsetbuttcap%
\pgfsetroundjoin%
\definecolor{currentfill}{rgb}{0.000000,0.000000,0.000000}%
\pgfsetfillcolor{currentfill}%
\pgfsetlinewidth{0.803000pt}%
\definecolor{currentstroke}{rgb}{0.000000,0.000000,0.000000}%
\pgfsetstrokecolor{currentstroke}%
\pgfsetdash{}{0pt}%
\pgfsys@defobject{currentmarker}{\pgfqpoint{-0.048611in}{0.000000in}}{\pgfqpoint{0.000000in}{0.000000in}}{%
\pgfpathmoveto{\pgfqpoint{0.000000in}{0.000000in}}%
\pgfpathlineto{\pgfqpoint{-0.048611in}{0.000000in}}%
\pgfusepath{stroke,fill}%
}%
\begin{pgfscope}%
\pgfsys@transformshift{4.509722in}{2.611124in}%
\pgfsys@useobject{currentmarker}{}%
\end{pgfscope}%
\end{pgfscope}%
\begin{pgfscope}%
\definecolor{textcolor}{rgb}{0.000000,0.000000,0.000000}%
\pgfsetstrokecolor{textcolor}%
\pgfsetfillcolor{textcolor}%
\pgftext[x=4.075247in,y=2.558362in,left,base]{\color{textcolor}\sffamily\fontsize{10.000000}{12.000000}\selectfont −0.5}%
\end{pgfscope}%
\begin{pgfscope}%
\pgfpathrectangle{\pgfqpoint{4.509722in}{0.387222in}}{\pgfqpoint{3.305278in}{2.324444in}}%
\pgfusepath{clip}%
\pgfsetrectcap%
\pgfsetroundjoin%
\pgfsetlinewidth{1.505625pt}%
\definecolor{currentstroke}{rgb}{0.121569,0.466667,0.705882}%
\pgfsetstrokecolor{currentstroke}%
\pgfsetdash{}{0pt}%
\pgfpathmoveto{\pgfqpoint{4.668210in}{0.500649in}}%
\pgfpathlineto{\pgfqpoint{4.742917in}{0.524487in}}%
\pgfpathlineto{\pgfqpoint{4.817625in}{0.551851in}}%
\pgfpathlineto{\pgfqpoint{4.892333in}{0.582978in}}%
\pgfpathlineto{\pgfqpoint{4.967040in}{0.618030in}}%
\pgfpathlineto{\pgfqpoint{5.041748in}{0.657404in}}%
\pgfpathlineto{\pgfqpoint{5.116455in}{0.701335in}}%
\pgfpathlineto{\pgfqpoint{5.191163in}{0.750011in}}%
\pgfpathlineto{\pgfqpoint{5.265870in}{0.803933in}}%
\pgfpathlineto{\pgfqpoint{5.340578in}{0.863277in}}%
\pgfpathlineto{\pgfqpoint{5.415285in}{0.928479in}}%
\pgfpathlineto{\pgfqpoint{5.489993in}{1.000503in}}%
\pgfpathlineto{\pgfqpoint{5.564701in}{1.079225in}}%
\pgfpathlineto{\pgfqpoint{5.639408in}{1.164907in}}%
\pgfpathlineto{\pgfqpoint{5.714116in}{1.259908in}}%
\pgfpathlineto{\pgfqpoint{5.788823in}{1.364048in}}%
\pgfpathlineto{\pgfqpoint{5.863531in}{1.478294in}}%
\pgfpathlineto{\pgfqpoint{5.938238in}{1.605606in}}%
\pgfpathlineto{\pgfqpoint{6.012946in}{1.749811in}}%
\pgfpathlineto{\pgfqpoint{6.087654in}{1.919545in}}%
\pgfpathlineto{\pgfqpoint{6.162361in}{2.131074in}}%
\pgfpathlineto{\pgfqpoint{6.237069in}{2.242266in}}%
\pgfpathlineto{\pgfqpoint{6.311776in}{2.289498in}}%
\pgfpathlineto{\pgfqpoint{6.386484in}{2.325740in}}%
\pgfpathlineto{\pgfqpoint{6.461191in}{2.356442in}}%
\pgfpathlineto{\pgfqpoint{6.535899in}{2.382697in}}%
\pgfpathlineto{\pgfqpoint{6.610606in}{2.406194in}}%
\pgfpathlineto{\pgfqpoint{6.685314in}{2.427432in}}%
\pgfpathlineto{\pgfqpoint{6.760022in}{2.446659in}}%
\pgfpathlineto{\pgfqpoint{6.834729in}{2.464304in}}%
\pgfpathlineto{\pgfqpoint{6.909437in}{2.480431in}}%
\pgfpathlineto{\pgfqpoint{6.984144in}{2.495640in}}%
\pgfpathlineto{\pgfqpoint{7.058852in}{2.510054in}}%
\pgfpathlineto{\pgfqpoint{7.133559in}{2.523426in}}%
\pgfpathlineto{\pgfqpoint{7.208267in}{2.535727in}}%
\pgfpathlineto{\pgfqpoint{7.282975in}{2.547694in}}%
\pgfpathlineto{\pgfqpoint{7.357682in}{2.558750in}}%
\pgfpathlineto{\pgfqpoint{7.432390in}{2.569541in}}%
\pgfpathlineto{\pgfqpoint{7.507097in}{2.579523in}}%
\pgfpathlineto{\pgfqpoint{7.581805in}{2.589148in}}%
\pgfpathlineto{\pgfqpoint{7.656512in}{2.598240in}}%
\pgfusepath{stroke}%
\end{pgfscope}%
\begin{pgfscope}%
\pgfsetrectcap%
\pgfsetmiterjoin%
\pgfsetlinewidth{0.803000pt}%
\definecolor{currentstroke}{rgb}{0.000000,0.000000,0.000000}%
\pgfsetstrokecolor{currentstroke}%
\pgfsetdash{}{0pt}%
\pgfpathmoveto{\pgfqpoint{4.509722in}{0.387222in}}%
\pgfpathlineto{\pgfqpoint{4.509722in}{2.711667in}}%
\pgfusepath{stroke}%
\end{pgfscope}%
\begin{pgfscope}%
\pgfsetrectcap%
\pgfsetmiterjoin%
\pgfsetlinewidth{0.803000pt}%
\definecolor{currentstroke}{rgb}{0.000000,0.000000,0.000000}%
\pgfsetstrokecolor{currentstroke}%
\pgfsetdash{}{0pt}%
\pgfpathmoveto{\pgfqpoint{7.815000in}{0.387222in}}%
\pgfpathlineto{\pgfqpoint{7.815000in}{2.711667in}}%
\pgfusepath{stroke}%
\end{pgfscope}%
\begin{pgfscope}%
\pgfsetrectcap%
\pgfsetmiterjoin%
\pgfsetlinewidth{0.803000pt}%
\definecolor{currentstroke}{rgb}{0.000000,0.000000,0.000000}%
\pgfsetstrokecolor{currentstroke}%
\pgfsetdash{}{0pt}%
\pgfpathmoveto{\pgfqpoint{4.509722in}{0.387222in}}%
\pgfpathlineto{\pgfqpoint{7.815000in}{0.387222in}}%
\pgfusepath{stroke}%
\end{pgfscope}%
\begin{pgfscope}%
\pgfsetrectcap%
\pgfsetmiterjoin%
\pgfsetlinewidth{0.803000pt}%
\definecolor{currentstroke}{rgb}{0.000000,0.000000,0.000000}%
\pgfsetstrokecolor{currentstroke}%
\pgfsetdash{}{0pt}%
\pgfpathmoveto{\pgfqpoint{4.509722in}{2.711667in}}%
\pgfpathlineto{\pgfqpoint{7.815000in}{2.711667in}}%
\pgfusepath{stroke}%
\end{pgfscope}%
\begin{pgfscope}%
\definecolor{textcolor}{rgb}{0.000000,0.000000,0.000000}%
\pgfsetstrokecolor{textcolor}%
\pgfsetfillcolor{textcolor}%
\pgftext[x=6.162361in,y=2.795000in,,base]{\color{textcolor}\sffamily\fontsize{12.000000}{14.400000}\selectfont \(\displaystyle  N = 24 \)}%
\end{pgfscope}%
\end{pgfpicture}%
\makeatother%
\endgroup%
}
\scalebox{0.75}{%% Creator: Matplotlib, PGF backend
%%
%% To include the figure in your LaTeX document, write
%%   \input{<filename>.pgf}
%%
%% Make sure the required packages are loaded in your preamble
%%   \usepackage{pgf}
%%
%% Figures using additional raster images can only be included by \input if
%% they are in the same directory as the main LaTeX file. For loading figures
%% from other directories you can use the `import` package
%%   \usepackage{import}
%% and then include the figures with
%%   \import{<path to file>}{<filename>.pgf}
%%
%% Matplotlib used the following preamble
%%   \usepackage{fontspec}
%%   \setmainfont{DejaVuSerif.ttf}[Path=/home/lzh/anaconda3/envs/numana/lib/python3.7/site-packages/matplotlib/mpl-data/fonts/ttf/]
%%   \setsansfont{DejaVuSans.ttf}[Path=/home/lzh/anaconda3/envs/numana/lib/python3.7/site-packages/matplotlib/mpl-data/fonts/ttf/]
%%   \setmonofont{DejaVuSansMono.ttf}[Path=/home/lzh/anaconda3/envs/numana/lib/python3.7/site-packages/matplotlib/mpl-data/fonts/ttf/]
%%
\begingroup%
\makeatletter%
\begin{pgfpicture}%
\pgfpathrectangle{\pgfpointorigin}{\pgfqpoint{8.000000in}{6.000000in}}%
\pgfusepath{use as bounding box, clip}%
\begin{pgfscope}%
\pgfsetbuttcap%
\pgfsetmiterjoin%
\definecolor{currentfill}{rgb}{1.000000,1.000000,1.000000}%
\pgfsetfillcolor{currentfill}%
\pgfsetlinewidth{0.000000pt}%
\definecolor{currentstroke}{rgb}{1.000000,1.000000,1.000000}%
\pgfsetstrokecolor{currentstroke}%
\pgfsetdash{}{0pt}%
\pgfpathmoveto{\pgfqpoint{0.000000in}{0.000000in}}%
\pgfpathlineto{\pgfqpoint{8.000000in}{0.000000in}}%
\pgfpathlineto{\pgfqpoint{8.000000in}{6.000000in}}%
\pgfpathlineto{\pgfqpoint{0.000000in}{6.000000in}}%
\pgfpathclose%
\pgfusepath{fill}%
\end{pgfscope}%
\begin{pgfscope}%
\pgfsetbuttcap%
\pgfsetmiterjoin%
\definecolor{currentfill}{rgb}{1.000000,1.000000,1.000000}%
\pgfsetfillcolor{currentfill}%
\pgfsetlinewidth{0.000000pt}%
\definecolor{currentstroke}{rgb}{0.000000,0.000000,0.000000}%
\pgfsetstrokecolor{currentstroke}%
\pgfsetstrokeopacity{0.000000}%
\pgfsetdash{}{0pt}%
\pgfpathmoveto{\pgfqpoint{0.469444in}{3.311111in}}%
\pgfpathlineto{\pgfqpoint{3.876389in}{3.311111in}}%
\pgfpathlineto{\pgfqpoint{3.876389in}{5.627778in}}%
\pgfpathlineto{\pgfqpoint{0.469444in}{5.627778in}}%
\pgfpathclose%
\pgfusepath{fill}%
\end{pgfscope}%
\begin{pgfscope}%
\pgfpathrectangle{\pgfqpoint{0.469444in}{3.311111in}}{\pgfqpoint{3.406944in}{2.316667in}}%
\pgfusepath{clip}%
\pgfsetbuttcap%
\pgfsetroundjoin%
\definecolor{currentfill}{rgb}{0.121569,0.466667,0.705882}%
\pgfsetfillcolor{currentfill}%
\pgfsetfillopacity{0.300000}%
\pgfsetlinewidth{1.003750pt}%
\definecolor{currentstroke}{rgb}{0.121569,0.466667,0.705882}%
\pgfsetstrokecolor{currentstroke}%
\pgfsetstrokeopacity{0.300000}%
\pgfsetdash{}{0pt}%
\pgfpathmoveto{\pgfqpoint{0.636088in}{3.610156in}}%
\pgfpathlineto{\pgfqpoint{0.636088in}{3.609783in}}%
\pgfpathlineto{\pgfqpoint{0.712929in}{3.658550in}}%
\pgfpathlineto{\pgfqpoint{0.789771in}{3.710457in}}%
\pgfpathlineto{\pgfqpoint{0.866612in}{3.765426in}}%
\pgfpathlineto{\pgfqpoint{0.943454in}{3.825096in}}%
\pgfpathlineto{\pgfqpoint{1.020295in}{3.887055in}}%
\pgfpathlineto{\pgfqpoint{1.097137in}{3.953694in}}%
\pgfpathlineto{\pgfqpoint{1.173978in}{4.023667in}}%
\pgfpathlineto{\pgfqpoint{1.250819in}{4.099835in}}%
\pgfpathlineto{\pgfqpoint{1.327661in}{4.180591in}}%
\pgfpathlineto{\pgfqpoint{1.404502in}{4.267404in}}%
\pgfpathlineto{\pgfqpoint{1.481344in}{4.359962in}}%
\pgfpathlineto{\pgfqpoint{1.558185in}{4.462407in}}%
\pgfpathlineto{\pgfqpoint{1.635027in}{4.573904in}}%
\pgfpathlineto{\pgfqpoint{1.711868in}{4.702869in}}%
\pgfpathlineto{\pgfqpoint{1.788709in}{4.846102in}}%
\pgfpathlineto{\pgfqpoint{1.865551in}{5.016836in}}%
\pgfpathlineto{\pgfqpoint{1.942392in}{5.256064in}}%
\pgfpathlineto{\pgfqpoint{2.019234in}{5.508532in}}%
\pgfpathlineto{\pgfqpoint{2.096075in}{5.477856in}}%
\pgfpathlineto{\pgfqpoint{2.172917in}{5.096477in}}%
\pgfpathlineto{\pgfqpoint{2.249758in}{4.583898in}}%
\pgfpathlineto{\pgfqpoint{2.326600in}{4.193165in}}%
\pgfpathlineto{\pgfqpoint{2.403441in}{3.955857in}}%
\pgfpathlineto{\pgfqpoint{2.480282in}{3.805642in}}%
\pgfpathlineto{\pgfqpoint{2.557124in}{3.715692in}}%
\pgfpathlineto{\pgfqpoint{2.633965in}{3.653160in}}%
\pgfpathlineto{\pgfqpoint{2.710807in}{3.608795in}}%
\pgfpathlineto{\pgfqpoint{2.787648in}{3.575611in}}%
\pgfpathlineto{\pgfqpoint{2.864490in}{3.551778in}}%
\pgfpathlineto{\pgfqpoint{2.941331in}{3.530173in}}%
\pgfpathlineto{\pgfqpoint{3.018172in}{3.512767in}}%
\pgfpathlineto{\pgfqpoint{3.095014in}{3.498416in}}%
\pgfpathlineto{\pgfqpoint{3.171855in}{3.485508in}}%
\pgfpathlineto{\pgfqpoint{3.248697in}{3.474117in}}%
\pgfpathlineto{\pgfqpoint{3.325538in}{3.464049in}}%
\pgfpathlineto{\pgfqpoint{3.402380in}{3.455666in}}%
\pgfpathlineto{\pgfqpoint{3.479221in}{3.446981in}}%
\pgfpathlineto{\pgfqpoint{3.556063in}{3.439954in}}%
\pgfpathlineto{\pgfqpoint{3.632904in}{3.433387in}}%
\pgfpathlineto{\pgfqpoint{3.709745in}{3.427034in}}%
\pgfpathlineto{\pgfqpoint{3.709745in}{3.427698in}}%
\pgfpathlineto{\pgfqpoint{3.709745in}{3.427698in}}%
\pgfpathlineto{\pgfqpoint{3.632904in}{3.433872in}}%
\pgfpathlineto{\pgfqpoint{3.556063in}{3.440978in}}%
\pgfpathlineto{\pgfqpoint{3.479221in}{3.448583in}}%
\pgfpathlineto{\pgfqpoint{3.402380in}{3.456160in}}%
\pgfpathlineto{\pgfqpoint{3.325538in}{3.464967in}}%
\pgfpathlineto{\pgfqpoint{3.248697in}{3.475409in}}%
\pgfpathlineto{\pgfqpoint{3.171855in}{3.486629in}}%
\pgfpathlineto{\pgfqpoint{3.095014in}{3.499546in}}%
\pgfpathlineto{\pgfqpoint{3.018172in}{3.514373in}}%
\pgfpathlineto{\pgfqpoint{2.941331in}{3.531068in}}%
\pgfpathlineto{\pgfqpoint{2.864490in}{3.553012in}}%
\pgfpathlineto{\pgfqpoint{2.787648in}{3.578905in}}%
\pgfpathlineto{\pgfqpoint{2.710807in}{3.613638in}}%
\pgfpathlineto{\pgfqpoint{2.633965in}{3.656349in}}%
\pgfpathlineto{\pgfqpoint{2.557124in}{3.719629in}}%
\pgfpathlineto{\pgfqpoint{2.480282in}{3.811793in}}%
\pgfpathlineto{\pgfqpoint{2.403441in}{3.965829in}}%
\pgfpathlineto{\pgfqpoint{2.326600in}{4.210293in}}%
\pgfpathlineto{\pgfqpoint{2.249758in}{4.609020in}}%
\pgfpathlineto{\pgfqpoint{2.172917in}{5.111377in}}%
\pgfpathlineto{\pgfqpoint{2.096075in}{5.511263in}}%
\pgfpathlineto{\pgfqpoint{2.019234in}{5.514513in}}%
\pgfpathlineto{\pgfqpoint{1.942392in}{5.272316in}}%
\pgfpathlineto{\pgfqpoint{1.865551in}{5.042453in}}%
\pgfpathlineto{\pgfqpoint{1.788709in}{4.858141in}}%
\pgfpathlineto{\pgfqpoint{1.711868in}{4.705849in}}%
\pgfpathlineto{\pgfqpoint{1.635027in}{4.581248in}}%
\pgfpathlineto{\pgfqpoint{1.558185in}{4.467194in}}%
\pgfpathlineto{\pgfqpoint{1.481344in}{4.363086in}}%
\pgfpathlineto{\pgfqpoint{1.404502in}{4.269772in}}%
\pgfpathlineto{\pgfqpoint{1.327661in}{4.183624in}}%
\pgfpathlineto{\pgfqpoint{1.250819in}{4.102344in}}%
\pgfpathlineto{\pgfqpoint{1.173978in}{4.025920in}}%
\pgfpathlineto{\pgfqpoint{1.097137in}{3.955013in}}%
\pgfpathlineto{\pgfqpoint{1.020295in}{3.888294in}}%
\pgfpathlineto{\pgfqpoint{0.943454in}{3.825524in}}%
\pgfpathlineto{\pgfqpoint{0.866612in}{3.766340in}}%
\pgfpathlineto{\pgfqpoint{0.789771in}{3.711019in}}%
\pgfpathlineto{\pgfqpoint{0.712929in}{3.658956in}}%
\pgfpathlineto{\pgfqpoint{0.636088in}{3.610156in}}%
\pgfpathclose%
\pgfusepath{stroke,fill}%
\end{pgfscope}%
\begin{pgfscope}%
\pgfpathrectangle{\pgfqpoint{0.469444in}{3.311111in}}{\pgfqpoint{3.406944in}{2.316667in}}%
\pgfusepath{clip}%
\pgfsetbuttcap%
\pgfsetroundjoin%
\definecolor{currentfill}{rgb}{0.121569,0.466667,0.705882}%
\pgfsetfillcolor{currentfill}%
\pgfsetlinewidth{1.003750pt}%
\definecolor{currentstroke}{rgb}{0.121569,0.466667,0.705882}%
\pgfsetstrokecolor{currentstroke}%
\pgfsetdash{}{0pt}%
\pgfsys@defobject{currentmarker}{\pgfqpoint{-0.009821in}{-0.009821in}}{\pgfqpoint{0.009821in}{0.009821in}}{%
\pgfpathmoveto{\pgfqpoint{0.000000in}{-0.009821in}}%
\pgfpathcurveto{\pgfqpoint{0.002605in}{-0.009821in}}{\pgfqpoint{0.005103in}{-0.008786in}}{\pgfqpoint{0.006944in}{-0.006944in}}%
\pgfpathcurveto{\pgfqpoint{0.008786in}{-0.005103in}}{\pgfqpoint{0.009821in}{-0.002605in}}{\pgfqpoint{0.009821in}{0.000000in}}%
\pgfpathcurveto{\pgfqpoint{0.009821in}{0.002605in}}{\pgfqpoint{0.008786in}{0.005103in}}{\pgfqpoint{0.006944in}{0.006944in}}%
\pgfpathcurveto{\pgfqpoint{0.005103in}{0.008786in}}{\pgfqpoint{0.002605in}{0.009821in}}{\pgfqpoint{0.000000in}{0.009821in}}%
\pgfpathcurveto{\pgfqpoint{-0.002605in}{0.009821in}}{\pgfqpoint{-0.005103in}{0.008786in}}{\pgfqpoint{-0.006944in}{0.006944in}}%
\pgfpathcurveto{\pgfqpoint{-0.008786in}{0.005103in}}{\pgfqpoint{-0.009821in}{0.002605in}}{\pgfqpoint{-0.009821in}{0.000000in}}%
\pgfpathcurveto{\pgfqpoint{-0.009821in}{-0.002605in}}{\pgfqpoint{-0.008786in}{-0.005103in}}{\pgfqpoint{-0.006944in}{-0.006944in}}%
\pgfpathcurveto{\pgfqpoint{-0.005103in}{-0.008786in}}{\pgfqpoint{-0.002605in}{-0.009821in}}{\pgfqpoint{0.000000in}{-0.009821in}}%
\pgfpathclose%
\pgfusepath{stroke,fill}%
}%
\begin{pgfscope}%
\pgfsys@transformshift{0.636088in}{3.609969in}%
\pgfsys@useobject{currentmarker}{}%
\end{pgfscope}%
\begin{pgfscope}%
\pgfsys@transformshift{0.712929in}{3.658753in}%
\pgfsys@useobject{currentmarker}{}%
\end{pgfscope}%
\begin{pgfscope}%
\pgfsys@transformshift{0.789771in}{3.710738in}%
\pgfsys@useobject{currentmarker}{}%
\end{pgfscope}%
\begin{pgfscope}%
\pgfsys@transformshift{0.866612in}{3.765883in}%
\pgfsys@useobject{currentmarker}{}%
\end{pgfscope}%
\begin{pgfscope}%
\pgfsys@transformshift{0.943454in}{3.825310in}%
\pgfsys@useobject{currentmarker}{}%
\end{pgfscope}%
\begin{pgfscope}%
\pgfsys@transformshift{1.020295in}{3.887675in}%
\pgfsys@useobject{currentmarker}{}%
\end{pgfscope}%
\begin{pgfscope}%
\pgfsys@transformshift{1.097137in}{3.954354in}%
\pgfsys@useobject{currentmarker}{}%
\end{pgfscope}%
\begin{pgfscope}%
\pgfsys@transformshift{1.173978in}{4.024794in}%
\pgfsys@useobject{currentmarker}{}%
\end{pgfscope}%
\begin{pgfscope}%
\pgfsys@transformshift{1.250819in}{4.101090in}%
\pgfsys@useobject{currentmarker}{}%
\end{pgfscope}%
\begin{pgfscope}%
\pgfsys@transformshift{1.327661in}{4.182108in}%
\pgfsys@useobject{currentmarker}{}%
\end{pgfscope}%
\begin{pgfscope}%
\pgfsys@transformshift{1.404502in}{4.268588in}%
\pgfsys@useobject{currentmarker}{}%
\end{pgfscope}%
\begin{pgfscope}%
\pgfsys@transformshift{1.481344in}{4.361524in}%
\pgfsys@useobject{currentmarker}{}%
\end{pgfscope}%
\begin{pgfscope}%
\pgfsys@transformshift{1.558185in}{4.464801in}%
\pgfsys@useobject{currentmarker}{}%
\end{pgfscope}%
\begin{pgfscope}%
\pgfsys@transformshift{1.635027in}{4.577576in}%
\pgfsys@useobject{currentmarker}{}%
\end{pgfscope}%
\begin{pgfscope}%
\pgfsys@transformshift{1.711868in}{4.704359in}%
\pgfsys@useobject{currentmarker}{}%
\end{pgfscope}%
\begin{pgfscope}%
\pgfsys@transformshift{1.788709in}{4.852121in}%
\pgfsys@useobject{currentmarker}{}%
\end{pgfscope}%
\begin{pgfscope}%
\pgfsys@transformshift{1.865551in}{5.029645in}%
\pgfsys@useobject{currentmarker}{}%
\end{pgfscope}%
\begin{pgfscope}%
\pgfsys@transformshift{1.942392in}{5.264190in}%
\pgfsys@useobject{currentmarker}{}%
\end{pgfscope}%
\begin{pgfscope}%
\pgfsys@transformshift{2.019234in}{5.511523in}%
\pgfsys@useobject{currentmarker}{}%
\end{pgfscope}%
\begin{pgfscope}%
\pgfsys@transformshift{2.096075in}{5.494560in}%
\pgfsys@useobject{currentmarker}{}%
\end{pgfscope}%
\begin{pgfscope}%
\pgfsys@transformshift{2.172917in}{5.103927in}%
\pgfsys@useobject{currentmarker}{}%
\end{pgfscope}%
\begin{pgfscope}%
\pgfsys@transformshift{2.249758in}{4.596459in}%
\pgfsys@useobject{currentmarker}{}%
\end{pgfscope}%
\begin{pgfscope}%
\pgfsys@transformshift{2.326600in}{4.201729in}%
\pgfsys@useobject{currentmarker}{}%
\end{pgfscope}%
\begin{pgfscope}%
\pgfsys@transformshift{2.403441in}{3.960843in}%
\pgfsys@useobject{currentmarker}{}%
\end{pgfscope}%
\begin{pgfscope}%
\pgfsys@transformshift{2.480282in}{3.808717in}%
\pgfsys@useobject{currentmarker}{}%
\end{pgfscope}%
\begin{pgfscope}%
\pgfsys@transformshift{2.557124in}{3.717661in}%
\pgfsys@useobject{currentmarker}{}%
\end{pgfscope}%
\begin{pgfscope}%
\pgfsys@transformshift{2.633965in}{3.654754in}%
\pgfsys@useobject{currentmarker}{}%
\end{pgfscope}%
\begin{pgfscope}%
\pgfsys@transformshift{2.710807in}{3.611217in}%
\pgfsys@useobject{currentmarker}{}%
\end{pgfscope}%
\begin{pgfscope}%
\pgfsys@transformshift{2.787648in}{3.577258in}%
\pgfsys@useobject{currentmarker}{}%
\end{pgfscope}%
\begin{pgfscope}%
\pgfsys@transformshift{2.864490in}{3.552395in}%
\pgfsys@useobject{currentmarker}{}%
\end{pgfscope}%
\begin{pgfscope}%
\pgfsys@transformshift{2.941331in}{3.530621in}%
\pgfsys@useobject{currentmarker}{}%
\end{pgfscope}%
\begin{pgfscope}%
\pgfsys@transformshift{3.018172in}{3.513570in}%
\pgfsys@useobject{currentmarker}{}%
\end{pgfscope}%
\begin{pgfscope}%
\pgfsys@transformshift{3.095014in}{3.498981in}%
\pgfsys@useobject{currentmarker}{}%
\end{pgfscope}%
\begin{pgfscope}%
\pgfsys@transformshift{3.171855in}{3.486069in}%
\pgfsys@useobject{currentmarker}{}%
\end{pgfscope}%
\begin{pgfscope}%
\pgfsys@transformshift{3.248697in}{3.474763in}%
\pgfsys@useobject{currentmarker}{}%
\end{pgfscope}%
\begin{pgfscope}%
\pgfsys@transformshift{3.325538in}{3.464508in}%
\pgfsys@useobject{currentmarker}{}%
\end{pgfscope}%
\begin{pgfscope}%
\pgfsys@transformshift{3.402380in}{3.455913in}%
\pgfsys@useobject{currentmarker}{}%
\end{pgfscope}%
\begin{pgfscope}%
\pgfsys@transformshift{3.479221in}{3.447782in}%
\pgfsys@useobject{currentmarker}{}%
\end{pgfscope}%
\begin{pgfscope}%
\pgfsys@transformshift{3.556063in}{3.440466in}%
\pgfsys@useobject{currentmarker}{}%
\end{pgfscope}%
\begin{pgfscope}%
\pgfsys@transformshift{3.632904in}{3.433629in}%
\pgfsys@useobject{currentmarker}{}%
\end{pgfscope}%
\begin{pgfscope}%
\pgfsys@transformshift{3.709745in}{3.427366in}%
\pgfsys@useobject{currentmarker}{}%
\end{pgfscope}%
\end{pgfscope}%
\begin{pgfscope}%
\pgfsetbuttcap%
\pgfsetroundjoin%
\definecolor{currentfill}{rgb}{0.000000,0.000000,0.000000}%
\pgfsetfillcolor{currentfill}%
\pgfsetlinewidth{0.803000pt}%
\definecolor{currentstroke}{rgb}{0.000000,0.000000,0.000000}%
\pgfsetstrokecolor{currentstroke}%
\pgfsetdash{}{0pt}%
\pgfsys@defobject{currentmarker}{\pgfqpoint{0.000000in}{-0.048611in}}{\pgfqpoint{0.000000in}{0.000000in}}{%
\pgfpathmoveto{\pgfqpoint{0.000000in}{0.000000in}}%
\pgfpathlineto{\pgfqpoint{0.000000in}{-0.048611in}}%
\pgfusepath{stroke,fill}%
}%
\begin{pgfscope}%
\pgfsys@transformshift{1.020295in}{3.311111in}%
\pgfsys@useobject{currentmarker}{}%
\end{pgfscope}%
\end{pgfscope}%
\begin{pgfscope}%
\definecolor{textcolor}{rgb}{0.000000,0.000000,0.000000}%
\pgfsetstrokecolor{textcolor}%
\pgfsetfillcolor{textcolor}%
\pgftext[x=1.020295in,y=3.213889in,,top]{\color{textcolor}\sffamily\fontsize{10.000000}{12.000000}\selectfont 3}%
\end{pgfscope}%
\begin{pgfscope}%
\pgfsetbuttcap%
\pgfsetroundjoin%
\definecolor{currentfill}{rgb}{0.000000,0.000000,0.000000}%
\pgfsetfillcolor{currentfill}%
\pgfsetlinewidth{0.803000pt}%
\definecolor{currentstroke}{rgb}{0.000000,0.000000,0.000000}%
\pgfsetstrokecolor{currentstroke}%
\pgfsetdash{}{0pt}%
\pgfsys@defobject{currentmarker}{\pgfqpoint{0.000000in}{-0.048611in}}{\pgfqpoint{0.000000in}{0.000000in}}{%
\pgfpathmoveto{\pgfqpoint{0.000000in}{0.000000in}}%
\pgfpathlineto{\pgfqpoint{0.000000in}{-0.048611in}}%
\pgfusepath{stroke,fill}%
}%
\begin{pgfscope}%
\pgfsys@transformshift{1.788709in}{3.311111in}%
\pgfsys@useobject{currentmarker}{}%
\end{pgfscope}%
\end{pgfscope}%
\begin{pgfscope}%
\definecolor{textcolor}{rgb}{0.000000,0.000000,0.000000}%
\pgfsetstrokecolor{textcolor}%
\pgfsetfillcolor{textcolor}%
\pgftext[x=1.788709in,y=3.213889in,,top]{\color{textcolor}\sffamily\fontsize{10.000000}{12.000000}\selectfont 4}%
\end{pgfscope}%
\begin{pgfscope}%
\pgfsetbuttcap%
\pgfsetroundjoin%
\definecolor{currentfill}{rgb}{0.000000,0.000000,0.000000}%
\pgfsetfillcolor{currentfill}%
\pgfsetlinewidth{0.803000pt}%
\definecolor{currentstroke}{rgb}{0.000000,0.000000,0.000000}%
\pgfsetstrokecolor{currentstroke}%
\pgfsetdash{}{0pt}%
\pgfsys@defobject{currentmarker}{\pgfqpoint{0.000000in}{-0.048611in}}{\pgfqpoint{0.000000in}{0.000000in}}{%
\pgfpathmoveto{\pgfqpoint{0.000000in}{0.000000in}}%
\pgfpathlineto{\pgfqpoint{0.000000in}{-0.048611in}}%
\pgfusepath{stroke,fill}%
}%
\begin{pgfscope}%
\pgfsys@transformshift{2.557124in}{3.311111in}%
\pgfsys@useobject{currentmarker}{}%
\end{pgfscope}%
\end{pgfscope}%
\begin{pgfscope}%
\definecolor{textcolor}{rgb}{0.000000,0.000000,0.000000}%
\pgfsetstrokecolor{textcolor}%
\pgfsetfillcolor{textcolor}%
\pgftext[x=2.557124in,y=3.213889in,,top]{\color{textcolor}\sffamily\fontsize{10.000000}{12.000000}\selectfont 5}%
\end{pgfscope}%
\begin{pgfscope}%
\pgfsetbuttcap%
\pgfsetroundjoin%
\definecolor{currentfill}{rgb}{0.000000,0.000000,0.000000}%
\pgfsetfillcolor{currentfill}%
\pgfsetlinewidth{0.803000pt}%
\definecolor{currentstroke}{rgb}{0.000000,0.000000,0.000000}%
\pgfsetstrokecolor{currentstroke}%
\pgfsetdash{}{0pt}%
\pgfsys@defobject{currentmarker}{\pgfqpoint{0.000000in}{-0.048611in}}{\pgfqpoint{0.000000in}{0.000000in}}{%
\pgfpathmoveto{\pgfqpoint{0.000000in}{0.000000in}}%
\pgfpathlineto{\pgfqpoint{0.000000in}{-0.048611in}}%
\pgfusepath{stroke,fill}%
}%
\begin{pgfscope}%
\pgfsys@transformshift{3.325538in}{3.311111in}%
\pgfsys@useobject{currentmarker}{}%
\end{pgfscope}%
\end{pgfscope}%
\begin{pgfscope}%
\definecolor{textcolor}{rgb}{0.000000,0.000000,0.000000}%
\pgfsetstrokecolor{textcolor}%
\pgfsetfillcolor{textcolor}%
\pgftext[x=3.325538in,y=3.213889in,,top]{\color{textcolor}\sffamily\fontsize{10.000000}{12.000000}\selectfont 6}%
\end{pgfscope}%
\begin{pgfscope}%
\pgfsetbuttcap%
\pgfsetroundjoin%
\definecolor{currentfill}{rgb}{0.000000,0.000000,0.000000}%
\pgfsetfillcolor{currentfill}%
\pgfsetlinewidth{0.803000pt}%
\definecolor{currentstroke}{rgb}{0.000000,0.000000,0.000000}%
\pgfsetstrokecolor{currentstroke}%
\pgfsetdash{}{0pt}%
\pgfsys@defobject{currentmarker}{\pgfqpoint{-0.048611in}{0.000000in}}{\pgfqpoint{0.000000in}{0.000000in}}{%
\pgfpathmoveto{\pgfqpoint{0.000000in}{0.000000in}}%
\pgfpathlineto{\pgfqpoint{-0.048611in}{0.000000in}}%
\pgfusepath{stroke,fill}%
}%
\begin{pgfscope}%
\pgfsys@transformshift{0.469444in}{3.613960in}%
\pgfsys@useobject{currentmarker}{}%
\end{pgfscope}%
\end{pgfscope}%
\begin{pgfscope}%
\definecolor{textcolor}{rgb}{0.000000,0.000000,0.000000}%
\pgfsetstrokecolor{textcolor}%
\pgfsetfillcolor{textcolor}%
\pgftext[x=0.062977in,y=3.561199in,left,base]{\color{textcolor}\sffamily\fontsize{10.000000}{12.000000}\selectfont 0.25}%
\end{pgfscope}%
\begin{pgfscope}%
\pgfsetbuttcap%
\pgfsetroundjoin%
\definecolor{currentfill}{rgb}{0.000000,0.000000,0.000000}%
\pgfsetfillcolor{currentfill}%
\pgfsetlinewidth{0.803000pt}%
\definecolor{currentstroke}{rgb}{0.000000,0.000000,0.000000}%
\pgfsetstrokecolor{currentstroke}%
\pgfsetdash{}{0pt}%
\pgfsys@defobject{currentmarker}{\pgfqpoint{-0.048611in}{0.000000in}}{\pgfqpoint{0.000000in}{0.000000in}}{%
\pgfpathmoveto{\pgfqpoint{0.000000in}{0.000000in}}%
\pgfpathlineto{\pgfqpoint{-0.048611in}{0.000000in}}%
\pgfusepath{stroke,fill}%
}%
\begin{pgfscope}%
\pgfsys@transformshift{0.469444in}{3.927689in}%
\pgfsys@useobject{currentmarker}{}%
\end{pgfscope}%
\end{pgfscope}%
\begin{pgfscope}%
\definecolor{textcolor}{rgb}{0.000000,0.000000,0.000000}%
\pgfsetstrokecolor{textcolor}%
\pgfsetfillcolor{textcolor}%
\pgftext[x=0.062977in,y=3.874927in,left,base]{\color{textcolor}\sffamily\fontsize{10.000000}{12.000000}\selectfont 0.50}%
\end{pgfscope}%
\begin{pgfscope}%
\pgfsetbuttcap%
\pgfsetroundjoin%
\definecolor{currentfill}{rgb}{0.000000,0.000000,0.000000}%
\pgfsetfillcolor{currentfill}%
\pgfsetlinewidth{0.803000pt}%
\definecolor{currentstroke}{rgb}{0.000000,0.000000,0.000000}%
\pgfsetstrokecolor{currentstroke}%
\pgfsetdash{}{0pt}%
\pgfsys@defobject{currentmarker}{\pgfqpoint{-0.048611in}{0.000000in}}{\pgfqpoint{0.000000in}{0.000000in}}{%
\pgfpathmoveto{\pgfqpoint{0.000000in}{0.000000in}}%
\pgfpathlineto{\pgfqpoint{-0.048611in}{0.000000in}}%
\pgfusepath{stroke,fill}%
}%
\begin{pgfscope}%
\pgfsys@transformshift{0.469444in}{4.241417in}%
\pgfsys@useobject{currentmarker}{}%
\end{pgfscope}%
\end{pgfscope}%
\begin{pgfscope}%
\definecolor{textcolor}{rgb}{0.000000,0.000000,0.000000}%
\pgfsetstrokecolor{textcolor}%
\pgfsetfillcolor{textcolor}%
\pgftext[x=0.062977in,y=4.188655in,left,base]{\color{textcolor}\sffamily\fontsize{10.000000}{12.000000}\selectfont 0.75}%
\end{pgfscope}%
\begin{pgfscope}%
\pgfsetbuttcap%
\pgfsetroundjoin%
\definecolor{currentfill}{rgb}{0.000000,0.000000,0.000000}%
\pgfsetfillcolor{currentfill}%
\pgfsetlinewidth{0.803000pt}%
\definecolor{currentstroke}{rgb}{0.000000,0.000000,0.000000}%
\pgfsetstrokecolor{currentstroke}%
\pgfsetdash{}{0pt}%
\pgfsys@defobject{currentmarker}{\pgfqpoint{-0.048611in}{0.000000in}}{\pgfqpoint{0.000000in}{0.000000in}}{%
\pgfpathmoveto{\pgfqpoint{0.000000in}{0.000000in}}%
\pgfpathlineto{\pgfqpoint{-0.048611in}{0.000000in}}%
\pgfusepath{stroke,fill}%
}%
\begin{pgfscope}%
\pgfsys@transformshift{0.469444in}{4.555145in}%
\pgfsys@useobject{currentmarker}{}%
\end{pgfscope}%
\end{pgfscope}%
\begin{pgfscope}%
\definecolor{textcolor}{rgb}{0.000000,0.000000,0.000000}%
\pgfsetstrokecolor{textcolor}%
\pgfsetfillcolor{textcolor}%
\pgftext[x=0.062977in,y=4.502383in,left,base]{\color{textcolor}\sffamily\fontsize{10.000000}{12.000000}\selectfont 1.00}%
\end{pgfscope}%
\begin{pgfscope}%
\pgfsetbuttcap%
\pgfsetroundjoin%
\definecolor{currentfill}{rgb}{0.000000,0.000000,0.000000}%
\pgfsetfillcolor{currentfill}%
\pgfsetlinewidth{0.803000pt}%
\definecolor{currentstroke}{rgb}{0.000000,0.000000,0.000000}%
\pgfsetstrokecolor{currentstroke}%
\pgfsetdash{}{0pt}%
\pgfsys@defobject{currentmarker}{\pgfqpoint{-0.048611in}{0.000000in}}{\pgfqpoint{0.000000in}{0.000000in}}{%
\pgfpathmoveto{\pgfqpoint{0.000000in}{0.000000in}}%
\pgfpathlineto{\pgfqpoint{-0.048611in}{0.000000in}}%
\pgfusepath{stroke,fill}%
}%
\begin{pgfscope}%
\pgfsys@transformshift{0.469444in}{4.868873in}%
\pgfsys@useobject{currentmarker}{}%
\end{pgfscope}%
\end{pgfscope}%
\begin{pgfscope}%
\definecolor{textcolor}{rgb}{0.000000,0.000000,0.000000}%
\pgfsetstrokecolor{textcolor}%
\pgfsetfillcolor{textcolor}%
\pgftext[x=0.062977in,y=4.816112in,left,base]{\color{textcolor}\sffamily\fontsize{10.000000}{12.000000}\selectfont 1.25}%
\end{pgfscope}%
\begin{pgfscope}%
\pgfsetbuttcap%
\pgfsetroundjoin%
\definecolor{currentfill}{rgb}{0.000000,0.000000,0.000000}%
\pgfsetfillcolor{currentfill}%
\pgfsetlinewidth{0.803000pt}%
\definecolor{currentstroke}{rgb}{0.000000,0.000000,0.000000}%
\pgfsetstrokecolor{currentstroke}%
\pgfsetdash{}{0pt}%
\pgfsys@defobject{currentmarker}{\pgfqpoint{-0.048611in}{0.000000in}}{\pgfqpoint{0.000000in}{0.000000in}}{%
\pgfpathmoveto{\pgfqpoint{0.000000in}{0.000000in}}%
\pgfpathlineto{\pgfqpoint{-0.048611in}{0.000000in}}%
\pgfusepath{stroke,fill}%
}%
\begin{pgfscope}%
\pgfsys@transformshift{0.469444in}{5.182601in}%
\pgfsys@useobject{currentmarker}{}%
\end{pgfscope}%
\end{pgfscope}%
\begin{pgfscope}%
\definecolor{textcolor}{rgb}{0.000000,0.000000,0.000000}%
\pgfsetstrokecolor{textcolor}%
\pgfsetfillcolor{textcolor}%
\pgftext[x=0.062977in,y=5.129840in,left,base]{\color{textcolor}\sffamily\fontsize{10.000000}{12.000000}\selectfont 1.50}%
\end{pgfscope}%
\begin{pgfscope}%
\pgfsetbuttcap%
\pgfsetroundjoin%
\definecolor{currentfill}{rgb}{0.000000,0.000000,0.000000}%
\pgfsetfillcolor{currentfill}%
\pgfsetlinewidth{0.803000pt}%
\definecolor{currentstroke}{rgb}{0.000000,0.000000,0.000000}%
\pgfsetstrokecolor{currentstroke}%
\pgfsetdash{}{0pt}%
\pgfsys@defobject{currentmarker}{\pgfqpoint{-0.048611in}{0.000000in}}{\pgfqpoint{0.000000in}{0.000000in}}{%
\pgfpathmoveto{\pgfqpoint{0.000000in}{0.000000in}}%
\pgfpathlineto{\pgfqpoint{-0.048611in}{0.000000in}}%
\pgfusepath{stroke,fill}%
}%
\begin{pgfscope}%
\pgfsys@transformshift{0.469444in}{5.496329in}%
\pgfsys@useobject{currentmarker}{}%
\end{pgfscope}%
\end{pgfscope}%
\begin{pgfscope}%
\definecolor{textcolor}{rgb}{0.000000,0.000000,0.000000}%
\pgfsetstrokecolor{textcolor}%
\pgfsetfillcolor{textcolor}%
\pgftext[x=0.062977in,y=5.443568in,left,base]{\color{textcolor}\sffamily\fontsize{10.000000}{12.000000}\selectfont 1.75}%
\end{pgfscope}%
\begin{pgfscope}%
\pgfpathrectangle{\pgfqpoint{0.469444in}{3.311111in}}{\pgfqpoint{3.406944in}{2.316667in}}%
\pgfusepath{clip}%
\pgfsetrectcap%
\pgfsetroundjoin%
\pgfsetlinewidth{1.505625pt}%
\definecolor{currentstroke}{rgb}{0.121569,0.466667,0.705882}%
\pgfsetstrokecolor{currentstroke}%
\pgfsetdash{}{0pt}%
\pgfpathmoveto{\pgfqpoint{0.636088in}{3.609969in}}%
\pgfpathlineto{\pgfqpoint{0.712929in}{3.658753in}}%
\pgfpathlineto{\pgfqpoint{0.789771in}{3.710738in}}%
\pgfpathlineto{\pgfqpoint{0.866612in}{3.765883in}}%
\pgfpathlineto{\pgfqpoint{0.943454in}{3.825310in}}%
\pgfpathlineto{\pgfqpoint{1.020295in}{3.887675in}}%
\pgfpathlineto{\pgfqpoint{1.097137in}{3.954354in}}%
\pgfpathlineto{\pgfqpoint{1.173978in}{4.024794in}}%
\pgfpathlineto{\pgfqpoint{1.250819in}{4.101090in}}%
\pgfpathlineto{\pgfqpoint{1.327661in}{4.182108in}}%
\pgfpathlineto{\pgfqpoint{1.404502in}{4.268588in}}%
\pgfpathlineto{\pgfqpoint{1.481344in}{4.361524in}}%
\pgfpathlineto{\pgfqpoint{1.558185in}{4.464801in}}%
\pgfpathlineto{\pgfqpoint{1.635027in}{4.577576in}}%
\pgfpathlineto{\pgfqpoint{1.711868in}{4.704359in}}%
\pgfpathlineto{\pgfqpoint{1.788709in}{4.852121in}}%
\pgfpathlineto{\pgfqpoint{1.865551in}{5.029645in}}%
\pgfpathlineto{\pgfqpoint{1.942392in}{5.264190in}}%
\pgfpathlineto{\pgfqpoint{2.019234in}{5.511523in}}%
\pgfpathlineto{\pgfqpoint{2.096075in}{5.494560in}}%
\pgfpathlineto{\pgfqpoint{2.172917in}{5.103927in}}%
\pgfpathlineto{\pgfqpoint{2.249758in}{4.596459in}}%
\pgfpathlineto{\pgfqpoint{2.326600in}{4.201729in}}%
\pgfpathlineto{\pgfqpoint{2.403441in}{3.960843in}}%
\pgfpathlineto{\pgfqpoint{2.480282in}{3.808717in}}%
\pgfpathlineto{\pgfqpoint{2.557124in}{3.717661in}}%
\pgfpathlineto{\pgfqpoint{2.633965in}{3.654754in}}%
\pgfpathlineto{\pgfqpoint{2.710807in}{3.611217in}}%
\pgfpathlineto{\pgfqpoint{2.787648in}{3.577258in}}%
\pgfpathlineto{\pgfqpoint{2.864490in}{3.552395in}}%
\pgfpathlineto{\pgfqpoint{2.941331in}{3.530621in}}%
\pgfpathlineto{\pgfqpoint{3.018172in}{3.513570in}}%
\pgfpathlineto{\pgfqpoint{3.095014in}{3.498981in}}%
\pgfpathlineto{\pgfqpoint{3.171855in}{3.486069in}}%
\pgfpathlineto{\pgfqpoint{3.248697in}{3.474763in}}%
\pgfpathlineto{\pgfqpoint{3.325538in}{3.464508in}}%
\pgfpathlineto{\pgfqpoint{3.402380in}{3.455913in}}%
\pgfpathlineto{\pgfqpoint{3.479221in}{3.447782in}}%
\pgfpathlineto{\pgfqpoint{3.556063in}{3.440466in}}%
\pgfpathlineto{\pgfqpoint{3.632904in}{3.433629in}}%
\pgfpathlineto{\pgfqpoint{3.709745in}{3.427366in}}%
\pgfusepath{stroke}%
\end{pgfscope}%
\begin{pgfscope}%
\pgfsetrectcap%
\pgfsetmiterjoin%
\pgfsetlinewidth{0.803000pt}%
\definecolor{currentstroke}{rgb}{0.000000,0.000000,0.000000}%
\pgfsetstrokecolor{currentstroke}%
\pgfsetdash{}{0pt}%
\pgfpathmoveto{\pgfqpoint{0.469444in}{3.311111in}}%
\pgfpathlineto{\pgfqpoint{0.469444in}{5.627778in}}%
\pgfusepath{stroke}%
\end{pgfscope}%
\begin{pgfscope}%
\pgfsetrectcap%
\pgfsetmiterjoin%
\pgfsetlinewidth{0.803000pt}%
\definecolor{currentstroke}{rgb}{0.000000,0.000000,0.000000}%
\pgfsetstrokecolor{currentstroke}%
\pgfsetdash{}{0pt}%
\pgfpathmoveto{\pgfqpoint{3.876389in}{3.311111in}}%
\pgfpathlineto{\pgfqpoint{3.876389in}{5.627778in}}%
\pgfusepath{stroke}%
\end{pgfscope}%
\begin{pgfscope}%
\pgfsetrectcap%
\pgfsetmiterjoin%
\pgfsetlinewidth{0.803000pt}%
\definecolor{currentstroke}{rgb}{0.000000,0.000000,0.000000}%
\pgfsetstrokecolor{currentstroke}%
\pgfsetdash{}{0pt}%
\pgfpathmoveto{\pgfqpoint{0.469444in}{3.311111in}}%
\pgfpathlineto{\pgfqpoint{3.876389in}{3.311111in}}%
\pgfusepath{stroke}%
\end{pgfscope}%
\begin{pgfscope}%
\pgfsetrectcap%
\pgfsetmiterjoin%
\pgfsetlinewidth{0.803000pt}%
\definecolor{currentstroke}{rgb}{0.000000,0.000000,0.000000}%
\pgfsetstrokecolor{currentstroke}%
\pgfsetdash{}{0pt}%
\pgfpathmoveto{\pgfqpoint{0.469444in}{5.627778in}}%
\pgfpathlineto{\pgfqpoint{3.876389in}{5.627778in}}%
\pgfusepath{stroke}%
\end{pgfscope}%
\begin{pgfscope}%
\definecolor{textcolor}{rgb}{0.000000,0.000000,0.000000}%
\pgfsetstrokecolor{textcolor}%
\pgfsetfillcolor{textcolor}%
\pgftext[x=2.172917in,y=5.711111in,,base]{\color{textcolor}\sffamily\fontsize{12.000000}{14.400000}\selectfont \(\displaystyle  N = 8 \)}%
\end{pgfscope}%
\begin{pgfscope}%
\pgfsetbuttcap%
\pgfsetmiterjoin%
\definecolor{currentfill}{rgb}{1.000000,1.000000,1.000000}%
\pgfsetfillcolor{currentfill}%
\pgfsetlinewidth{0.000000pt}%
\definecolor{currentstroke}{rgb}{0.000000,0.000000,0.000000}%
\pgfsetstrokecolor{currentstroke}%
\pgfsetstrokeopacity{0.000000}%
\pgfsetdash{}{0pt}%
\pgfpathmoveto{\pgfqpoint{4.394444in}{3.311111in}}%
\pgfpathlineto{\pgfqpoint{7.801389in}{3.311111in}}%
\pgfpathlineto{\pgfqpoint{7.801389in}{5.627778in}}%
\pgfpathlineto{\pgfqpoint{4.394444in}{5.627778in}}%
\pgfpathclose%
\pgfusepath{fill}%
\end{pgfscope}%
\begin{pgfscope}%
\pgfpathrectangle{\pgfqpoint{4.394444in}{3.311111in}}{\pgfqpoint{3.406944in}{2.316667in}}%
\pgfusepath{clip}%
\pgfsetbuttcap%
\pgfsetroundjoin%
\definecolor{currentfill}{rgb}{0.121569,0.466667,0.705882}%
\pgfsetfillcolor{currentfill}%
\pgfsetfillopacity{0.300000}%
\pgfsetlinewidth{1.003750pt}%
\definecolor{currentstroke}{rgb}{0.121569,0.466667,0.705882}%
\pgfsetstrokecolor{currentstroke}%
\pgfsetstrokeopacity{0.300000}%
\pgfsetdash{}{0pt}%
\pgfpathmoveto{\pgfqpoint{4.561088in}{3.579531in}}%
\pgfpathlineto{\pgfqpoint{4.561088in}{3.579171in}}%
\pgfpathlineto{\pgfqpoint{4.637929in}{3.617915in}}%
\pgfpathlineto{\pgfqpoint{4.714771in}{3.659261in}}%
\pgfpathlineto{\pgfqpoint{4.791612in}{3.703327in}}%
\pgfpathlineto{\pgfqpoint{4.868454in}{3.750325in}}%
\pgfpathlineto{\pgfqpoint{4.945295in}{3.801436in}}%
\pgfpathlineto{\pgfqpoint{5.022137in}{3.855509in}}%
\pgfpathlineto{\pgfqpoint{5.098978in}{3.912083in}}%
\pgfpathlineto{\pgfqpoint{5.175819in}{3.972931in}}%
\pgfpathlineto{\pgfqpoint{5.252661in}{4.038319in}}%
\pgfpathlineto{\pgfqpoint{5.329502in}{4.107970in}}%
\pgfpathlineto{\pgfqpoint{5.406344in}{4.186771in}}%
\pgfpathlineto{\pgfqpoint{5.483185in}{4.261918in}}%
\pgfpathlineto{\pgfqpoint{5.560027in}{4.359559in}}%
\pgfpathlineto{\pgfqpoint{5.636868in}{4.453322in}}%
\pgfpathlineto{\pgfqpoint{5.713709in}{4.573762in}}%
\pgfpathlineto{\pgfqpoint{5.790551in}{4.694203in}}%
\pgfpathlineto{\pgfqpoint{5.867392in}{4.858976in}}%
\pgfpathlineto{\pgfqpoint{5.944234in}{5.079937in}}%
\pgfpathlineto{\pgfqpoint{6.021075in}{5.434359in}}%
\pgfpathlineto{\pgfqpoint{6.097917in}{5.048524in}}%
\pgfpathlineto{\pgfqpoint{6.174758in}{4.233057in}}%
\pgfpathlineto{\pgfqpoint{6.251600in}{3.890385in}}%
\pgfpathlineto{\pgfqpoint{6.328441in}{3.739865in}}%
\pgfpathlineto{\pgfqpoint{6.405282in}{3.661114in}}%
\pgfpathlineto{\pgfqpoint{6.482124in}{3.614190in}}%
\pgfpathlineto{\pgfqpoint{6.558965in}{3.581810in}}%
\pgfpathlineto{\pgfqpoint{6.635807in}{3.556533in}}%
\pgfpathlineto{\pgfqpoint{6.712648in}{3.535865in}}%
\pgfpathlineto{\pgfqpoint{6.789490in}{3.519538in}}%
\pgfpathlineto{\pgfqpoint{6.866331in}{3.505279in}}%
\pgfpathlineto{\pgfqpoint{6.943172in}{3.493588in}}%
\pgfpathlineto{\pgfqpoint{7.020014in}{3.481956in}}%
\pgfpathlineto{\pgfqpoint{7.096855in}{3.472991in}}%
\pgfpathlineto{\pgfqpoint{7.173697in}{3.464109in}}%
\pgfpathlineto{\pgfqpoint{7.250538in}{3.456059in}}%
\pgfpathlineto{\pgfqpoint{7.327380in}{3.449379in}}%
\pgfpathlineto{\pgfqpoint{7.404221in}{3.443197in}}%
\pgfpathlineto{\pgfqpoint{7.481063in}{3.437359in}}%
\pgfpathlineto{\pgfqpoint{7.557904in}{3.431961in}}%
\pgfpathlineto{\pgfqpoint{7.634745in}{3.427047in}}%
\pgfpathlineto{\pgfqpoint{7.634745in}{3.427769in}}%
\pgfpathlineto{\pgfqpoint{7.634745in}{3.427769in}}%
\pgfpathlineto{\pgfqpoint{7.557904in}{3.433121in}}%
\pgfpathlineto{\pgfqpoint{7.481063in}{3.437843in}}%
\pgfpathlineto{\pgfqpoint{7.404221in}{3.444076in}}%
\pgfpathlineto{\pgfqpoint{7.327380in}{3.450759in}}%
\pgfpathlineto{\pgfqpoint{7.250538in}{3.457172in}}%
\pgfpathlineto{\pgfqpoint{7.173697in}{3.464897in}}%
\pgfpathlineto{\pgfqpoint{7.096855in}{3.474062in}}%
\pgfpathlineto{\pgfqpoint{7.020014in}{3.483713in}}%
\pgfpathlineto{\pgfqpoint{6.943172in}{3.493988in}}%
\pgfpathlineto{\pgfqpoint{6.866331in}{3.507453in}}%
\pgfpathlineto{\pgfqpoint{6.789490in}{3.521533in}}%
\pgfpathlineto{\pgfqpoint{6.712648in}{3.537180in}}%
\pgfpathlineto{\pgfqpoint{6.635807in}{3.557670in}}%
\pgfpathlineto{\pgfqpoint{6.558965in}{3.583764in}}%
\pgfpathlineto{\pgfqpoint{6.482124in}{3.616403in}}%
\pgfpathlineto{\pgfqpoint{6.405282in}{3.665269in}}%
\pgfpathlineto{\pgfqpoint{6.328441in}{3.745149in}}%
\pgfpathlineto{\pgfqpoint{6.251600in}{3.898184in}}%
\pgfpathlineto{\pgfqpoint{6.174758in}{4.290766in}}%
\pgfpathlineto{\pgfqpoint{6.097917in}{5.098754in}}%
\pgfpathlineto{\pgfqpoint{6.021075in}{5.522475in}}%
\pgfpathlineto{\pgfqpoint{5.944234in}{5.150064in}}%
\pgfpathlineto{\pgfqpoint{5.867392in}{4.891632in}}%
\pgfpathlineto{\pgfqpoint{5.790551in}{4.719791in}}%
\pgfpathlineto{\pgfqpoint{5.713709in}{4.582989in}}%
\pgfpathlineto{\pgfqpoint{5.636868in}{4.470141in}}%
\pgfpathlineto{\pgfqpoint{5.560027in}{4.364462in}}%
\pgfpathlineto{\pgfqpoint{5.483185in}{4.273330in}}%
\pgfpathlineto{\pgfqpoint{5.406344in}{4.190383in}}%
\pgfpathlineto{\pgfqpoint{5.329502in}{4.112763in}}%
\pgfpathlineto{\pgfqpoint{5.252661in}{4.039937in}}%
\pgfpathlineto{\pgfqpoint{5.175819in}{3.974892in}}%
\pgfpathlineto{\pgfqpoint{5.098978in}{3.915361in}}%
\pgfpathlineto{\pgfqpoint{5.022137in}{3.855891in}}%
\pgfpathlineto{\pgfqpoint{4.945295in}{3.802471in}}%
\pgfpathlineto{\pgfqpoint{4.868454in}{3.752247in}}%
\pgfpathlineto{\pgfqpoint{4.791612in}{3.704608in}}%
\pgfpathlineto{\pgfqpoint{4.714771in}{3.659974in}}%
\pgfpathlineto{\pgfqpoint{4.637929in}{3.618232in}}%
\pgfpathlineto{\pgfqpoint{4.561088in}{3.579531in}}%
\pgfpathclose%
\pgfusepath{stroke,fill}%
\end{pgfscope}%
\begin{pgfscope}%
\pgfpathrectangle{\pgfqpoint{4.394444in}{3.311111in}}{\pgfqpoint{3.406944in}{2.316667in}}%
\pgfusepath{clip}%
\pgfsetbuttcap%
\pgfsetroundjoin%
\definecolor{currentfill}{rgb}{0.121569,0.466667,0.705882}%
\pgfsetfillcolor{currentfill}%
\pgfsetlinewidth{1.003750pt}%
\definecolor{currentstroke}{rgb}{0.121569,0.466667,0.705882}%
\pgfsetstrokecolor{currentstroke}%
\pgfsetdash{}{0pt}%
\pgfsys@defobject{currentmarker}{\pgfqpoint{-0.009821in}{-0.009821in}}{\pgfqpoint{0.009821in}{0.009821in}}{%
\pgfpathmoveto{\pgfqpoint{0.000000in}{-0.009821in}}%
\pgfpathcurveto{\pgfqpoint{0.002605in}{-0.009821in}}{\pgfqpoint{0.005103in}{-0.008786in}}{\pgfqpoint{0.006944in}{-0.006944in}}%
\pgfpathcurveto{\pgfqpoint{0.008786in}{-0.005103in}}{\pgfqpoint{0.009821in}{-0.002605in}}{\pgfqpoint{0.009821in}{0.000000in}}%
\pgfpathcurveto{\pgfqpoint{0.009821in}{0.002605in}}{\pgfqpoint{0.008786in}{0.005103in}}{\pgfqpoint{0.006944in}{0.006944in}}%
\pgfpathcurveto{\pgfqpoint{0.005103in}{0.008786in}}{\pgfqpoint{0.002605in}{0.009821in}}{\pgfqpoint{0.000000in}{0.009821in}}%
\pgfpathcurveto{\pgfqpoint{-0.002605in}{0.009821in}}{\pgfqpoint{-0.005103in}{0.008786in}}{\pgfqpoint{-0.006944in}{0.006944in}}%
\pgfpathcurveto{\pgfqpoint{-0.008786in}{0.005103in}}{\pgfqpoint{-0.009821in}{0.002605in}}{\pgfqpoint{-0.009821in}{0.000000in}}%
\pgfpathcurveto{\pgfqpoint{-0.009821in}{-0.002605in}}{\pgfqpoint{-0.008786in}{-0.005103in}}{\pgfqpoint{-0.006944in}{-0.006944in}}%
\pgfpathcurveto{\pgfqpoint{-0.005103in}{-0.008786in}}{\pgfqpoint{-0.002605in}{-0.009821in}}{\pgfqpoint{0.000000in}{-0.009821in}}%
\pgfpathclose%
\pgfusepath{stroke,fill}%
}%
\begin{pgfscope}%
\pgfsys@transformshift{4.561088in}{3.579351in}%
\pgfsys@useobject{currentmarker}{}%
\end{pgfscope}%
\begin{pgfscope}%
\pgfsys@transformshift{4.637929in}{3.618073in}%
\pgfsys@useobject{currentmarker}{}%
\end{pgfscope}%
\begin{pgfscope}%
\pgfsys@transformshift{4.714771in}{3.659617in}%
\pgfsys@useobject{currentmarker}{}%
\end{pgfscope}%
\begin{pgfscope}%
\pgfsys@transformshift{4.791612in}{3.703967in}%
\pgfsys@useobject{currentmarker}{}%
\end{pgfscope}%
\begin{pgfscope}%
\pgfsys@transformshift{4.868454in}{3.751286in}%
\pgfsys@useobject{currentmarker}{}%
\end{pgfscope}%
\begin{pgfscope}%
\pgfsys@transformshift{4.945295in}{3.801954in}%
\pgfsys@useobject{currentmarker}{}%
\end{pgfscope}%
\begin{pgfscope}%
\pgfsys@transformshift{5.022137in}{3.855700in}%
\pgfsys@useobject{currentmarker}{}%
\end{pgfscope}%
\begin{pgfscope}%
\pgfsys@transformshift{5.098978in}{3.913722in}%
\pgfsys@useobject{currentmarker}{}%
\end{pgfscope}%
\begin{pgfscope}%
\pgfsys@transformshift{5.175819in}{3.973911in}%
\pgfsys@useobject{currentmarker}{}%
\end{pgfscope}%
\begin{pgfscope}%
\pgfsys@transformshift{5.252661in}{4.039128in}%
\pgfsys@useobject{currentmarker}{}%
\end{pgfscope}%
\begin{pgfscope}%
\pgfsys@transformshift{5.329502in}{4.110366in}%
\pgfsys@useobject{currentmarker}{}%
\end{pgfscope}%
\begin{pgfscope}%
\pgfsys@transformshift{5.406344in}{4.188577in}%
\pgfsys@useobject{currentmarker}{}%
\end{pgfscope}%
\begin{pgfscope}%
\pgfsys@transformshift{5.483185in}{4.267624in}%
\pgfsys@useobject{currentmarker}{}%
\end{pgfscope}%
\begin{pgfscope}%
\pgfsys@transformshift{5.560027in}{4.362011in}%
\pgfsys@useobject{currentmarker}{}%
\end{pgfscope}%
\begin{pgfscope}%
\pgfsys@transformshift{5.636868in}{4.461731in}%
\pgfsys@useobject{currentmarker}{}%
\end{pgfscope}%
\begin{pgfscope}%
\pgfsys@transformshift{5.713709in}{4.578375in}%
\pgfsys@useobject{currentmarker}{}%
\end{pgfscope}%
\begin{pgfscope}%
\pgfsys@transformshift{5.790551in}{4.706997in}%
\pgfsys@useobject{currentmarker}{}%
\end{pgfscope}%
\begin{pgfscope}%
\pgfsys@transformshift{5.867392in}{4.875304in}%
\pgfsys@useobject{currentmarker}{}%
\end{pgfscope}%
\begin{pgfscope}%
\pgfsys@transformshift{5.944234in}{5.115001in}%
\pgfsys@useobject{currentmarker}{}%
\end{pgfscope}%
\begin{pgfscope}%
\pgfsys@transformshift{6.021075in}{5.478417in}%
\pgfsys@useobject{currentmarker}{}%
\end{pgfscope}%
\begin{pgfscope}%
\pgfsys@transformshift{6.097917in}{5.073639in}%
\pgfsys@useobject{currentmarker}{}%
\end{pgfscope}%
\begin{pgfscope}%
\pgfsys@transformshift{6.174758in}{4.261912in}%
\pgfsys@useobject{currentmarker}{}%
\end{pgfscope}%
\begin{pgfscope}%
\pgfsys@transformshift{6.251600in}{3.894284in}%
\pgfsys@useobject{currentmarker}{}%
\end{pgfscope}%
\begin{pgfscope}%
\pgfsys@transformshift{6.328441in}{3.742507in}%
\pgfsys@useobject{currentmarker}{}%
\end{pgfscope}%
\begin{pgfscope}%
\pgfsys@transformshift{6.405282in}{3.663192in}%
\pgfsys@useobject{currentmarker}{}%
\end{pgfscope}%
\begin{pgfscope}%
\pgfsys@transformshift{6.482124in}{3.615297in}%
\pgfsys@useobject{currentmarker}{}%
\end{pgfscope}%
\begin{pgfscope}%
\pgfsys@transformshift{6.558965in}{3.582787in}%
\pgfsys@useobject{currentmarker}{}%
\end{pgfscope}%
\begin{pgfscope}%
\pgfsys@transformshift{6.635807in}{3.557101in}%
\pgfsys@useobject{currentmarker}{}%
\end{pgfscope}%
\begin{pgfscope}%
\pgfsys@transformshift{6.712648in}{3.536523in}%
\pgfsys@useobject{currentmarker}{}%
\end{pgfscope}%
\begin{pgfscope}%
\pgfsys@transformshift{6.789490in}{3.520535in}%
\pgfsys@useobject{currentmarker}{}%
\end{pgfscope}%
\begin{pgfscope}%
\pgfsys@transformshift{6.866331in}{3.506366in}%
\pgfsys@useobject{currentmarker}{}%
\end{pgfscope}%
\begin{pgfscope}%
\pgfsys@transformshift{6.943172in}{3.493788in}%
\pgfsys@useobject{currentmarker}{}%
\end{pgfscope}%
\begin{pgfscope}%
\pgfsys@transformshift{7.020014in}{3.482835in}%
\pgfsys@useobject{currentmarker}{}%
\end{pgfscope}%
\begin{pgfscope}%
\pgfsys@transformshift{7.096855in}{3.473527in}%
\pgfsys@useobject{currentmarker}{}%
\end{pgfscope}%
\begin{pgfscope}%
\pgfsys@transformshift{7.173697in}{3.464503in}%
\pgfsys@useobject{currentmarker}{}%
\end{pgfscope}%
\begin{pgfscope}%
\pgfsys@transformshift{7.250538in}{3.456616in}%
\pgfsys@useobject{currentmarker}{}%
\end{pgfscope}%
\begin{pgfscope}%
\pgfsys@transformshift{7.327380in}{3.450069in}%
\pgfsys@useobject{currentmarker}{}%
\end{pgfscope}%
\begin{pgfscope}%
\pgfsys@transformshift{7.404221in}{3.443636in}%
\pgfsys@useobject{currentmarker}{}%
\end{pgfscope}%
\begin{pgfscope}%
\pgfsys@transformshift{7.481063in}{3.437601in}%
\pgfsys@useobject{currentmarker}{}%
\end{pgfscope}%
\begin{pgfscope}%
\pgfsys@transformshift{7.557904in}{3.432541in}%
\pgfsys@useobject{currentmarker}{}%
\end{pgfscope}%
\begin{pgfscope}%
\pgfsys@transformshift{7.634745in}{3.427408in}%
\pgfsys@useobject{currentmarker}{}%
\end{pgfscope}%
\end{pgfscope}%
\begin{pgfscope}%
\pgfsetbuttcap%
\pgfsetroundjoin%
\definecolor{currentfill}{rgb}{0.000000,0.000000,0.000000}%
\pgfsetfillcolor{currentfill}%
\pgfsetlinewidth{0.803000pt}%
\definecolor{currentstroke}{rgb}{0.000000,0.000000,0.000000}%
\pgfsetstrokecolor{currentstroke}%
\pgfsetdash{}{0pt}%
\pgfsys@defobject{currentmarker}{\pgfqpoint{0.000000in}{-0.048611in}}{\pgfqpoint{0.000000in}{0.000000in}}{%
\pgfpathmoveto{\pgfqpoint{0.000000in}{0.000000in}}%
\pgfpathlineto{\pgfqpoint{0.000000in}{-0.048611in}}%
\pgfusepath{stroke,fill}%
}%
\begin{pgfscope}%
\pgfsys@transformshift{4.945295in}{3.311111in}%
\pgfsys@useobject{currentmarker}{}%
\end{pgfscope}%
\end{pgfscope}%
\begin{pgfscope}%
\definecolor{textcolor}{rgb}{0.000000,0.000000,0.000000}%
\pgfsetstrokecolor{textcolor}%
\pgfsetfillcolor{textcolor}%
\pgftext[x=4.945295in,y=3.213889in,,top]{\color{textcolor}\sffamily\fontsize{10.000000}{12.000000}\selectfont 3}%
\end{pgfscope}%
\begin{pgfscope}%
\pgfsetbuttcap%
\pgfsetroundjoin%
\definecolor{currentfill}{rgb}{0.000000,0.000000,0.000000}%
\pgfsetfillcolor{currentfill}%
\pgfsetlinewidth{0.803000pt}%
\definecolor{currentstroke}{rgb}{0.000000,0.000000,0.000000}%
\pgfsetstrokecolor{currentstroke}%
\pgfsetdash{}{0pt}%
\pgfsys@defobject{currentmarker}{\pgfqpoint{0.000000in}{-0.048611in}}{\pgfqpoint{0.000000in}{0.000000in}}{%
\pgfpathmoveto{\pgfqpoint{0.000000in}{0.000000in}}%
\pgfpathlineto{\pgfqpoint{0.000000in}{-0.048611in}}%
\pgfusepath{stroke,fill}%
}%
\begin{pgfscope}%
\pgfsys@transformshift{5.713709in}{3.311111in}%
\pgfsys@useobject{currentmarker}{}%
\end{pgfscope}%
\end{pgfscope}%
\begin{pgfscope}%
\definecolor{textcolor}{rgb}{0.000000,0.000000,0.000000}%
\pgfsetstrokecolor{textcolor}%
\pgfsetfillcolor{textcolor}%
\pgftext[x=5.713709in,y=3.213889in,,top]{\color{textcolor}\sffamily\fontsize{10.000000}{12.000000}\selectfont 4}%
\end{pgfscope}%
\begin{pgfscope}%
\pgfsetbuttcap%
\pgfsetroundjoin%
\definecolor{currentfill}{rgb}{0.000000,0.000000,0.000000}%
\pgfsetfillcolor{currentfill}%
\pgfsetlinewidth{0.803000pt}%
\definecolor{currentstroke}{rgb}{0.000000,0.000000,0.000000}%
\pgfsetstrokecolor{currentstroke}%
\pgfsetdash{}{0pt}%
\pgfsys@defobject{currentmarker}{\pgfqpoint{0.000000in}{-0.048611in}}{\pgfqpoint{0.000000in}{0.000000in}}{%
\pgfpathmoveto{\pgfqpoint{0.000000in}{0.000000in}}%
\pgfpathlineto{\pgfqpoint{0.000000in}{-0.048611in}}%
\pgfusepath{stroke,fill}%
}%
\begin{pgfscope}%
\pgfsys@transformshift{6.482124in}{3.311111in}%
\pgfsys@useobject{currentmarker}{}%
\end{pgfscope}%
\end{pgfscope}%
\begin{pgfscope}%
\definecolor{textcolor}{rgb}{0.000000,0.000000,0.000000}%
\pgfsetstrokecolor{textcolor}%
\pgfsetfillcolor{textcolor}%
\pgftext[x=6.482124in,y=3.213889in,,top]{\color{textcolor}\sffamily\fontsize{10.000000}{12.000000}\selectfont 5}%
\end{pgfscope}%
\begin{pgfscope}%
\pgfsetbuttcap%
\pgfsetroundjoin%
\definecolor{currentfill}{rgb}{0.000000,0.000000,0.000000}%
\pgfsetfillcolor{currentfill}%
\pgfsetlinewidth{0.803000pt}%
\definecolor{currentstroke}{rgb}{0.000000,0.000000,0.000000}%
\pgfsetstrokecolor{currentstroke}%
\pgfsetdash{}{0pt}%
\pgfsys@defobject{currentmarker}{\pgfqpoint{0.000000in}{-0.048611in}}{\pgfqpoint{0.000000in}{0.000000in}}{%
\pgfpathmoveto{\pgfqpoint{0.000000in}{0.000000in}}%
\pgfpathlineto{\pgfqpoint{0.000000in}{-0.048611in}}%
\pgfusepath{stroke,fill}%
}%
\begin{pgfscope}%
\pgfsys@transformshift{7.250538in}{3.311111in}%
\pgfsys@useobject{currentmarker}{}%
\end{pgfscope}%
\end{pgfscope}%
\begin{pgfscope}%
\definecolor{textcolor}{rgb}{0.000000,0.000000,0.000000}%
\pgfsetstrokecolor{textcolor}%
\pgfsetfillcolor{textcolor}%
\pgftext[x=7.250538in,y=3.213889in,,top]{\color{textcolor}\sffamily\fontsize{10.000000}{12.000000}\selectfont 6}%
\end{pgfscope}%
\begin{pgfscope}%
\pgfsetbuttcap%
\pgfsetroundjoin%
\definecolor{currentfill}{rgb}{0.000000,0.000000,0.000000}%
\pgfsetfillcolor{currentfill}%
\pgfsetlinewidth{0.803000pt}%
\definecolor{currentstroke}{rgb}{0.000000,0.000000,0.000000}%
\pgfsetstrokecolor{currentstroke}%
\pgfsetdash{}{0pt}%
\pgfsys@defobject{currentmarker}{\pgfqpoint{-0.048611in}{0.000000in}}{\pgfqpoint{0.000000in}{0.000000in}}{%
\pgfpathmoveto{\pgfqpoint{0.000000in}{0.000000in}}%
\pgfpathlineto{\pgfqpoint{-0.048611in}{0.000000in}}%
\pgfusepath{stroke,fill}%
}%
\begin{pgfscope}%
\pgfsys@transformshift{4.394444in}{3.324910in}%
\pgfsys@useobject{currentmarker}{}%
\end{pgfscope}%
\end{pgfscope}%
\begin{pgfscope}%
\definecolor{textcolor}{rgb}{0.000000,0.000000,0.000000}%
\pgfsetstrokecolor{textcolor}%
\pgfsetfillcolor{textcolor}%
\pgftext[x=4.076343in,y=3.272148in,left,base]{\color{textcolor}\sffamily\fontsize{10.000000}{12.000000}\selectfont 0.0}%
\end{pgfscope}%
\begin{pgfscope}%
\pgfsetbuttcap%
\pgfsetroundjoin%
\definecolor{currentfill}{rgb}{0.000000,0.000000,0.000000}%
\pgfsetfillcolor{currentfill}%
\pgfsetlinewidth{0.803000pt}%
\definecolor{currentstroke}{rgb}{0.000000,0.000000,0.000000}%
\pgfsetstrokecolor{currentstroke}%
\pgfsetdash{}{0pt}%
\pgfsys@defobject{currentmarker}{\pgfqpoint{-0.048611in}{0.000000in}}{\pgfqpoint{0.000000in}{0.000000in}}{%
\pgfpathmoveto{\pgfqpoint{0.000000in}{0.000000in}}%
\pgfpathlineto{\pgfqpoint{-0.048611in}{0.000000in}}%
\pgfusepath{stroke,fill}%
}%
\begin{pgfscope}%
\pgfsys@transformshift{4.394444in}{3.833185in}%
\pgfsys@useobject{currentmarker}{}%
\end{pgfscope}%
\end{pgfscope}%
\begin{pgfscope}%
\definecolor{textcolor}{rgb}{0.000000,0.000000,0.000000}%
\pgfsetstrokecolor{textcolor}%
\pgfsetfillcolor{textcolor}%
\pgftext[x=4.076343in,y=3.780423in,left,base]{\color{textcolor}\sffamily\fontsize{10.000000}{12.000000}\selectfont 0.5}%
\end{pgfscope}%
\begin{pgfscope}%
\pgfsetbuttcap%
\pgfsetroundjoin%
\definecolor{currentfill}{rgb}{0.000000,0.000000,0.000000}%
\pgfsetfillcolor{currentfill}%
\pgfsetlinewidth{0.803000pt}%
\definecolor{currentstroke}{rgb}{0.000000,0.000000,0.000000}%
\pgfsetstrokecolor{currentstroke}%
\pgfsetdash{}{0pt}%
\pgfsys@defobject{currentmarker}{\pgfqpoint{-0.048611in}{0.000000in}}{\pgfqpoint{0.000000in}{0.000000in}}{%
\pgfpathmoveto{\pgfqpoint{0.000000in}{0.000000in}}%
\pgfpathlineto{\pgfqpoint{-0.048611in}{0.000000in}}%
\pgfusepath{stroke,fill}%
}%
\begin{pgfscope}%
\pgfsys@transformshift{4.394444in}{4.341460in}%
\pgfsys@useobject{currentmarker}{}%
\end{pgfscope}%
\end{pgfscope}%
\begin{pgfscope}%
\definecolor{textcolor}{rgb}{0.000000,0.000000,0.000000}%
\pgfsetstrokecolor{textcolor}%
\pgfsetfillcolor{textcolor}%
\pgftext[x=4.076343in,y=4.288698in,left,base]{\color{textcolor}\sffamily\fontsize{10.000000}{12.000000}\selectfont 1.0}%
\end{pgfscope}%
\begin{pgfscope}%
\pgfsetbuttcap%
\pgfsetroundjoin%
\definecolor{currentfill}{rgb}{0.000000,0.000000,0.000000}%
\pgfsetfillcolor{currentfill}%
\pgfsetlinewidth{0.803000pt}%
\definecolor{currentstroke}{rgb}{0.000000,0.000000,0.000000}%
\pgfsetstrokecolor{currentstroke}%
\pgfsetdash{}{0pt}%
\pgfsys@defobject{currentmarker}{\pgfqpoint{-0.048611in}{0.000000in}}{\pgfqpoint{0.000000in}{0.000000in}}{%
\pgfpathmoveto{\pgfqpoint{0.000000in}{0.000000in}}%
\pgfpathlineto{\pgfqpoint{-0.048611in}{0.000000in}}%
\pgfusepath{stroke,fill}%
}%
\begin{pgfscope}%
\pgfsys@transformshift{4.394444in}{4.849735in}%
\pgfsys@useobject{currentmarker}{}%
\end{pgfscope}%
\end{pgfscope}%
\begin{pgfscope}%
\definecolor{textcolor}{rgb}{0.000000,0.000000,0.000000}%
\pgfsetstrokecolor{textcolor}%
\pgfsetfillcolor{textcolor}%
\pgftext[x=4.076343in,y=4.796973in,left,base]{\color{textcolor}\sffamily\fontsize{10.000000}{12.000000}\selectfont 1.5}%
\end{pgfscope}%
\begin{pgfscope}%
\pgfsetbuttcap%
\pgfsetroundjoin%
\definecolor{currentfill}{rgb}{0.000000,0.000000,0.000000}%
\pgfsetfillcolor{currentfill}%
\pgfsetlinewidth{0.803000pt}%
\definecolor{currentstroke}{rgb}{0.000000,0.000000,0.000000}%
\pgfsetstrokecolor{currentstroke}%
\pgfsetdash{}{0pt}%
\pgfsys@defobject{currentmarker}{\pgfqpoint{-0.048611in}{0.000000in}}{\pgfqpoint{0.000000in}{0.000000in}}{%
\pgfpathmoveto{\pgfqpoint{0.000000in}{0.000000in}}%
\pgfpathlineto{\pgfqpoint{-0.048611in}{0.000000in}}%
\pgfusepath{stroke,fill}%
}%
\begin{pgfscope}%
\pgfsys@transformshift{4.394444in}{5.358010in}%
\pgfsys@useobject{currentmarker}{}%
\end{pgfscope}%
\end{pgfscope}%
\begin{pgfscope}%
\definecolor{textcolor}{rgb}{0.000000,0.000000,0.000000}%
\pgfsetstrokecolor{textcolor}%
\pgfsetfillcolor{textcolor}%
\pgftext[x=4.076343in,y=5.305248in,left,base]{\color{textcolor}\sffamily\fontsize{10.000000}{12.000000}\selectfont 2.0}%
\end{pgfscope}%
\begin{pgfscope}%
\pgfpathrectangle{\pgfqpoint{4.394444in}{3.311111in}}{\pgfqpoint{3.406944in}{2.316667in}}%
\pgfusepath{clip}%
\pgfsetrectcap%
\pgfsetroundjoin%
\pgfsetlinewidth{1.505625pt}%
\definecolor{currentstroke}{rgb}{0.121569,0.466667,0.705882}%
\pgfsetstrokecolor{currentstroke}%
\pgfsetdash{}{0pt}%
\pgfpathmoveto{\pgfqpoint{4.561088in}{3.579351in}}%
\pgfpathlineto{\pgfqpoint{4.637929in}{3.618073in}}%
\pgfpathlineto{\pgfqpoint{4.714771in}{3.659617in}}%
\pgfpathlineto{\pgfqpoint{4.791612in}{3.703967in}}%
\pgfpathlineto{\pgfqpoint{4.868454in}{3.751286in}}%
\pgfpathlineto{\pgfqpoint{4.945295in}{3.801954in}}%
\pgfpathlineto{\pgfqpoint{5.022137in}{3.855700in}}%
\pgfpathlineto{\pgfqpoint{5.098978in}{3.913722in}}%
\pgfpathlineto{\pgfqpoint{5.175819in}{3.973911in}}%
\pgfpathlineto{\pgfqpoint{5.252661in}{4.039128in}}%
\pgfpathlineto{\pgfqpoint{5.329502in}{4.110366in}}%
\pgfpathlineto{\pgfqpoint{5.406344in}{4.188577in}}%
\pgfpathlineto{\pgfqpoint{5.483185in}{4.267624in}}%
\pgfpathlineto{\pgfqpoint{5.560027in}{4.362011in}}%
\pgfpathlineto{\pgfqpoint{5.636868in}{4.461731in}}%
\pgfpathlineto{\pgfqpoint{5.713709in}{4.578375in}}%
\pgfpathlineto{\pgfqpoint{5.790551in}{4.706997in}}%
\pgfpathlineto{\pgfqpoint{5.867392in}{4.875304in}}%
\pgfpathlineto{\pgfqpoint{5.944234in}{5.115001in}}%
\pgfpathlineto{\pgfqpoint{6.021075in}{5.478417in}}%
\pgfpathlineto{\pgfqpoint{6.097917in}{5.073639in}}%
\pgfpathlineto{\pgfqpoint{6.174758in}{4.261912in}}%
\pgfpathlineto{\pgfqpoint{6.251600in}{3.894284in}}%
\pgfpathlineto{\pgfqpoint{6.328441in}{3.742507in}}%
\pgfpathlineto{\pgfqpoint{6.405282in}{3.663192in}}%
\pgfpathlineto{\pgfqpoint{6.482124in}{3.615297in}}%
\pgfpathlineto{\pgfqpoint{6.558965in}{3.582787in}}%
\pgfpathlineto{\pgfqpoint{6.635807in}{3.557101in}}%
\pgfpathlineto{\pgfqpoint{6.712648in}{3.536523in}}%
\pgfpathlineto{\pgfqpoint{6.789490in}{3.520535in}}%
\pgfpathlineto{\pgfqpoint{6.866331in}{3.506366in}}%
\pgfpathlineto{\pgfqpoint{6.943172in}{3.493788in}}%
\pgfpathlineto{\pgfqpoint{7.020014in}{3.482835in}}%
\pgfpathlineto{\pgfqpoint{7.096855in}{3.473527in}}%
\pgfpathlineto{\pgfqpoint{7.173697in}{3.464503in}}%
\pgfpathlineto{\pgfqpoint{7.250538in}{3.456616in}}%
\pgfpathlineto{\pgfqpoint{7.327380in}{3.450069in}}%
\pgfpathlineto{\pgfqpoint{7.404221in}{3.443636in}}%
\pgfpathlineto{\pgfqpoint{7.481063in}{3.437601in}}%
\pgfpathlineto{\pgfqpoint{7.557904in}{3.432541in}}%
\pgfpathlineto{\pgfqpoint{7.634745in}{3.427408in}}%
\pgfusepath{stroke}%
\end{pgfscope}%
\begin{pgfscope}%
\pgfsetrectcap%
\pgfsetmiterjoin%
\pgfsetlinewidth{0.803000pt}%
\definecolor{currentstroke}{rgb}{0.000000,0.000000,0.000000}%
\pgfsetstrokecolor{currentstroke}%
\pgfsetdash{}{0pt}%
\pgfpathmoveto{\pgfqpoint{4.394444in}{3.311111in}}%
\pgfpathlineto{\pgfqpoint{4.394444in}{5.627778in}}%
\pgfusepath{stroke}%
\end{pgfscope}%
\begin{pgfscope}%
\pgfsetrectcap%
\pgfsetmiterjoin%
\pgfsetlinewidth{0.803000pt}%
\definecolor{currentstroke}{rgb}{0.000000,0.000000,0.000000}%
\pgfsetstrokecolor{currentstroke}%
\pgfsetdash{}{0pt}%
\pgfpathmoveto{\pgfqpoint{7.801389in}{3.311111in}}%
\pgfpathlineto{\pgfqpoint{7.801389in}{5.627778in}}%
\pgfusepath{stroke}%
\end{pgfscope}%
\begin{pgfscope}%
\pgfsetrectcap%
\pgfsetmiterjoin%
\pgfsetlinewidth{0.803000pt}%
\definecolor{currentstroke}{rgb}{0.000000,0.000000,0.000000}%
\pgfsetstrokecolor{currentstroke}%
\pgfsetdash{}{0pt}%
\pgfpathmoveto{\pgfqpoint{4.394444in}{3.311111in}}%
\pgfpathlineto{\pgfqpoint{7.801389in}{3.311111in}}%
\pgfusepath{stroke}%
\end{pgfscope}%
\begin{pgfscope}%
\pgfsetrectcap%
\pgfsetmiterjoin%
\pgfsetlinewidth{0.803000pt}%
\definecolor{currentstroke}{rgb}{0.000000,0.000000,0.000000}%
\pgfsetstrokecolor{currentstroke}%
\pgfsetdash{}{0pt}%
\pgfpathmoveto{\pgfqpoint{4.394444in}{5.627778in}}%
\pgfpathlineto{\pgfqpoint{7.801389in}{5.627778in}}%
\pgfusepath{stroke}%
\end{pgfscope}%
\begin{pgfscope}%
\definecolor{textcolor}{rgb}{0.000000,0.000000,0.000000}%
\pgfsetstrokecolor{textcolor}%
\pgfsetfillcolor{textcolor}%
\pgftext[x=6.097917in,y=5.711111in,,base]{\color{textcolor}\sffamily\fontsize{12.000000}{14.400000}\selectfont \(\displaystyle  N = 12 \)}%
\end{pgfscope}%
\begin{pgfscope}%
\pgfsetbuttcap%
\pgfsetmiterjoin%
\definecolor{currentfill}{rgb}{1.000000,1.000000,1.000000}%
\pgfsetfillcolor{currentfill}%
\pgfsetlinewidth{0.000000pt}%
\definecolor{currentstroke}{rgb}{0.000000,0.000000,0.000000}%
\pgfsetstrokecolor{currentstroke}%
\pgfsetstrokeopacity{0.000000}%
\pgfsetdash{}{0pt}%
\pgfpathmoveto{\pgfqpoint{0.469444in}{0.386111in}}%
\pgfpathlineto{\pgfqpoint{3.876389in}{0.386111in}}%
\pgfpathlineto{\pgfqpoint{3.876389in}{2.702778in}}%
\pgfpathlineto{\pgfqpoint{0.469444in}{2.702778in}}%
\pgfpathclose%
\pgfusepath{fill}%
\end{pgfscope}%
\begin{pgfscope}%
\pgfpathrectangle{\pgfqpoint{0.469444in}{0.386111in}}{\pgfqpoint{3.406944in}{2.316667in}}%
\pgfusepath{clip}%
\pgfsetbuttcap%
\pgfsetroundjoin%
\definecolor{currentfill}{rgb}{0.121569,0.466667,0.705882}%
\pgfsetfillcolor{currentfill}%
\pgfsetfillopacity{0.300000}%
\pgfsetlinewidth{1.003750pt}%
\definecolor{currentstroke}{rgb}{0.121569,0.466667,0.705882}%
\pgfsetstrokecolor{currentstroke}%
\pgfsetstrokeopacity{0.300000}%
\pgfsetdash{}{0pt}%
\pgfpathmoveto{\pgfqpoint{0.636088in}{0.652818in}}%
\pgfpathlineto{\pgfqpoint{0.636088in}{0.651804in}}%
\pgfpathlineto{\pgfqpoint{0.712929in}{0.689875in}}%
\pgfpathlineto{\pgfqpoint{0.789771in}{0.730519in}}%
\pgfpathlineto{\pgfqpoint{0.866612in}{0.774091in}}%
\pgfpathlineto{\pgfqpoint{0.943454in}{0.820812in}}%
\pgfpathlineto{\pgfqpoint{1.020295in}{0.869710in}}%
\pgfpathlineto{\pgfqpoint{1.097137in}{0.922196in}}%
\pgfpathlineto{\pgfqpoint{1.173978in}{0.979022in}}%
\pgfpathlineto{\pgfqpoint{1.250819in}{1.039276in}}%
\pgfpathlineto{\pgfqpoint{1.327661in}{1.104463in}}%
\pgfpathlineto{\pgfqpoint{1.404502in}{1.174790in}}%
\pgfpathlineto{\pgfqpoint{1.481344in}{1.247654in}}%
\pgfpathlineto{\pgfqpoint{1.558185in}{1.329336in}}%
\pgfpathlineto{\pgfqpoint{1.635027in}{1.414523in}}%
\pgfpathlineto{\pgfqpoint{1.711868in}{1.506634in}}%
\pgfpathlineto{\pgfqpoint{1.788709in}{1.620823in}}%
\pgfpathlineto{\pgfqpoint{1.865551in}{1.744064in}}%
\pgfpathlineto{\pgfqpoint{1.942392in}{1.921312in}}%
\pgfpathlineto{\pgfqpoint{2.019234in}{2.120245in}}%
\pgfpathlineto{\pgfqpoint{2.096075in}{2.432787in}}%
\pgfpathlineto{\pgfqpoint{2.172917in}{2.379188in}}%
\pgfpathlineto{\pgfqpoint{2.249758in}{1.193585in}}%
\pgfpathlineto{\pgfqpoint{2.326600in}{0.868660in}}%
\pgfpathlineto{\pgfqpoint{2.403441in}{0.780692in}}%
\pgfpathlineto{\pgfqpoint{2.480282in}{0.720258in}}%
\pgfpathlineto{\pgfqpoint{2.557124in}{0.679531in}}%
\pgfpathlineto{\pgfqpoint{2.633965in}{0.652199in}}%
\pgfpathlineto{\pgfqpoint{2.710807in}{0.629569in}}%
\pgfpathlineto{\pgfqpoint{2.787648in}{0.609474in}}%
\pgfpathlineto{\pgfqpoint{2.864490in}{0.592993in}}%
\pgfpathlineto{\pgfqpoint{2.941331in}{0.579617in}}%
\pgfpathlineto{\pgfqpoint{3.018172in}{0.565933in}}%
\pgfpathlineto{\pgfqpoint{3.095014in}{0.554935in}}%
\pgfpathlineto{\pgfqpoint{3.171855in}{0.546983in}}%
\pgfpathlineto{\pgfqpoint{3.248697in}{0.538104in}}%
\pgfpathlineto{\pgfqpoint{3.325538in}{0.530460in}}%
\pgfpathlineto{\pgfqpoint{3.402380in}{0.523919in}}%
\pgfpathlineto{\pgfqpoint{3.479221in}{0.518140in}}%
\pgfpathlineto{\pgfqpoint{3.556063in}{0.512056in}}%
\pgfpathlineto{\pgfqpoint{3.632904in}{0.506877in}}%
\pgfpathlineto{\pgfqpoint{3.709745in}{0.502151in}}%
\pgfpathlineto{\pgfqpoint{3.709745in}{0.502664in}}%
\pgfpathlineto{\pgfqpoint{3.709745in}{0.502664in}}%
\pgfpathlineto{\pgfqpoint{3.632904in}{0.507393in}}%
\pgfpathlineto{\pgfqpoint{3.556063in}{0.513185in}}%
\pgfpathlineto{\pgfqpoint{3.479221in}{0.519248in}}%
\pgfpathlineto{\pgfqpoint{3.402380in}{0.526164in}}%
\pgfpathlineto{\pgfqpoint{3.325538in}{0.531661in}}%
\pgfpathlineto{\pgfqpoint{3.248697in}{0.538906in}}%
\pgfpathlineto{\pgfqpoint{3.171855in}{0.548312in}}%
\pgfpathlineto{\pgfqpoint{3.095014in}{0.557950in}}%
\pgfpathlineto{\pgfqpoint{3.018172in}{0.570064in}}%
\pgfpathlineto{\pgfqpoint{2.941331in}{0.580245in}}%
\pgfpathlineto{\pgfqpoint{2.864490in}{0.594640in}}%
\pgfpathlineto{\pgfqpoint{2.787648in}{0.610705in}}%
\pgfpathlineto{\pgfqpoint{2.710807in}{0.631353in}}%
\pgfpathlineto{\pgfqpoint{2.633965in}{0.656411in}}%
\pgfpathlineto{\pgfqpoint{2.557124in}{0.685704in}}%
\pgfpathlineto{\pgfqpoint{2.480282in}{0.725576in}}%
\pgfpathlineto{\pgfqpoint{2.403441in}{0.788436in}}%
\pgfpathlineto{\pgfqpoint{2.326600in}{0.897026in}}%
\pgfpathlineto{\pgfqpoint{2.249758in}{1.226604in}}%
\pgfpathlineto{\pgfqpoint{2.172917in}{2.434224in}}%
\pgfpathlineto{\pgfqpoint{2.096075in}{2.597475in}}%
\pgfpathlineto{\pgfqpoint{2.019234in}{2.141581in}}%
\pgfpathlineto{\pgfqpoint{1.942392in}{1.958602in}}%
\pgfpathlineto{\pgfqpoint{1.865551in}{1.775694in}}%
\pgfpathlineto{\pgfqpoint{1.788709in}{1.641750in}}%
\pgfpathlineto{\pgfqpoint{1.711868in}{1.525065in}}%
\pgfpathlineto{\pgfqpoint{1.635027in}{1.425411in}}%
\pgfpathlineto{\pgfqpoint{1.558185in}{1.333277in}}%
\pgfpathlineto{\pgfqpoint{1.481344in}{1.252804in}}%
\pgfpathlineto{\pgfqpoint{1.404502in}{1.179008in}}%
\pgfpathlineto{\pgfqpoint{1.327661in}{1.109163in}}%
\pgfpathlineto{\pgfqpoint{1.250819in}{1.041332in}}%
\pgfpathlineto{\pgfqpoint{1.173978in}{0.982745in}}%
\pgfpathlineto{\pgfqpoint{1.097137in}{0.923009in}}%
\pgfpathlineto{\pgfqpoint{1.020295in}{0.872192in}}%
\pgfpathlineto{\pgfqpoint{0.943454in}{0.822713in}}%
\pgfpathlineto{\pgfqpoint{0.866612in}{0.775600in}}%
\pgfpathlineto{\pgfqpoint{0.789771in}{0.731630in}}%
\pgfpathlineto{\pgfqpoint{0.712929in}{0.690447in}}%
\pgfpathlineto{\pgfqpoint{0.636088in}{0.652818in}}%
\pgfpathclose%
\pgfusepath{stroke,fill}%
\end{pgfscope}%
\begin{pgfscope}%
\pgfpathrectangle{\pgfqpoint{0.469444in}{0.386111in}}{\pgfqpoint{3.406944in}{2.316667in}}%
\pgfusepath{clip}%
\pgfsetbuttcap%
\pgfsetroundjoin%
\definecolor{currentfill}{rgb}{0.121569,0.466667,0.705882}%
\pgfsetfillcolor{currentfill}%
\pgfsetlinewidth{1.003750pt}%
\definecolor{currentstroke}{rgb}{0.121569,0.466667,0.705882}%
\pgfsetstrokecolor{currentstroke}%
\pgfsetdash{}{0pt}%
\pgfsys@defobject{currentmarker}{\pgfqpoint{-0.009821in}{-0.009821in}}{\pgfqpoint{0.009821in}{0.009821in}}{%
\pgfpathmoveto{\pgfqpoint{0.000000in}{-0.009821in}}%
\pgfpathcurveto{\pgfqpoint{0.002605in}{-0.009821in}}{\pgfqpoint{0.005103in}{-0.008786in}}{\pgfqpoint{0.006944in}{-0.006944in}}%
\pgfpathcurveto{\pgfqpoint{0.008786in}{-0.005103in}}{\pgfqpoint{0.009821in}{-0.002605in}}{\pgfqpoint{0.009821in}{0.000000in}}%
\pgfpathcurveto{\pgfqpoint{0.009821in}{0.002605in}}{\pgfqpoint{0.008786in}{0.005103in}}{\pgfqpoint{0.006944in}{0.006944in}}%
\pgfpathcurveto{\pgfqpoint{0.005103in}{0.008786in}}{\pgfqpoint{0.002605in}{0.009821in}}{\pgfqpoint{0.000000in}{0.009821in}}%
\pgfpathcurveto{\pgfqpoint{-0.002605in}{0.009821in}}{\pgfqpoint{-0.005103in}{0.008786in}}{\pgfqpoint{-0.006944in}{0.006944in}}%
\pgfpathcurveto{\pgfqpoint{-0.008786in}{0.005103in}}{\pgfqpoint{-0.009821in}{0.002605in}}{\pgfqpoint{-0.009821in}{0.000000in}}%
\pgfpathcurveto{\pgfqpoint{-0.009821in}{-0.002605in}}{\pgfqpoint{-0.008786in}{-0.005103in}}{\pgfqpoint{-0.006944in}{-0.006944in}}%
\pgfpathcurveto{\pgfqpoint{-0.005103in}{-0.008786in}}{\pgfqpoint{-0.002605in}{-0.009821in}}{\pgfqpoint{0.000000in}{-0.009821in}}%
\pgfpathclose%
\pgfusepath{stroke,fill}%
}%
\begin{pgfscope}%
\pgfsys@transformshift{0.636088in}{0.652311in}%
\pgfsys@useobject{currentmarker}{}%
\end{pgfscope}%
\begin{pgfscope}%
\pgfsys@transformshift{0.712929in}{0.690161in}%
\pgfsys@useobject{currentmarker}{}%
\end{pgfscope}%
\begin{pgfscope}%
\pgfsys@transformshift{0.789771in}{0.731075in}%
\pgfsys@useobject{currentmarker}{}%
\end{pgfscope}%
\begin{pgfscope}%
\pgfsys@transformshift{0.866612in}{0.774846in}%
\pgfsys@useobject{currentmarker}{}%
\end{pgfscope}%
\begin{pgfscope}%
\pgfsys@transformshift{0.943454in}{0.821763in}%
\pgfsys@useobject{currentmarker}{}%
\end{pgfscope}%
\begin{pgfscope}%
\pgfsys@transformshift{1.020295in}{0.870951in}%
\pgfsys@useobject{currentmarker}{}%
\end{pgfscope}%
\begin{pgfscope}%
\pgfsys@transformshift{1.097137in}{0.922603in}%
\pgfsys@useobject{currentmarker}{}%
\end{pgfscope}%
\begin{pgfscope}%
\pgfsys@transformshift{1.173978in}{0.980884in}%
\pgfsys@useobject{currentmarker}{}%
\end{pgfscope}%
\begin{pgfscope}%
\pgfsys@transformshift{1.250819in}{1.040304in}%
\pgfsys@useobject{currentmarker}{}%
\end{pgfscope}%
\begin{pgfscope}%
\pgfsys@transformshift{1.327661in}{1.106813in}%
\pgfsys@useobject{currentmarker}{}%
\end{pgfscope}%
\begin{pgfscope}%
\pgfsys@transformshift{1.404502in}{1.176899in}%
\pgfsys@useobject{currentmarker}{}%
\end{pgfscope}%
\begin{pgfscope}%
\pgfsys@transformshift{1.481344in}{1.250229in}%
\pgfsys@useobject{currentmarker}{}%
\end{pgfscope}%
\begin{pgfscope}%
\pgfsys@transformshift{1.558185in}{1.331307in}%
\pgfsys@useobject{currentmarker}{}%
\end{pgfscope}%
\begin{pgfscope}%
\pgfsys@transformshift{1.635027in}{1.419967in}%
\pgfsys@useobject{currentmarker}{}%
\end{pgfscope}%
\begin{pgfscope}%
\pgfsys@transformshift{1.711868in}{1.515850in}%
\pgfsys@useobject{currentmarker}{}%
\end{pgfscope}%
\begin{pgfscope}%
\pgfsys@transformshift{1.788709in}{1.631287in}%
\pgfsys@useobject{currentmarker}{}%
\end{pgfscope}%
\begin{pgfscope}%
\pgfsys@transformshift{1.865551in}{1.759879in}%
\pgfsys@useobject{currentmarker}{}%
\end{pgfscope}%
\begin{pgfscope}%
\pgfsys@transformshift{1.942392in}{1.939957in}%
\pgfsys@useobject{currentmarker}{}%
\end{pgfscope}%
\begin{pgfscope}%
\pgfsys@transformshift{2.019234in}{2.130913in}%
\pgfsys@useobject{currentmarker}{}%
\end{pgfscope}%
\begin{pgfscope}%
\pgfsys@transformshift{2.096075in}{2.515131in}%
\pgfsys@useobject{currentmarker}{}%
\end{pgfscope}%
\begin{pgfscope}%
\pgfsys@transformshift{2.172917in}{2.406706in}%
\pgfsys@useobject{currentmarker}{}%
\end{pgfscope}%
\begin{pgfscope}%
\pgfsys@transformshift{2.249758in}{1.210094in}%
\pgfsys@useobject{currentmarker}{}%
\end{pgfscope}%
\begin{pgfscope}%
\pgfsys@transformshift{2.326600in}{0.882843in}%
\pgfsys@useobject{currentmarker}{}%
\end{pgfscope}%
\begin{pgfscope}%
\pgfsys@transformshift{2.403441in}{0.784564in}%
\pgfsys@useobject{currentmarker}{}%
\end{pgfscope}%
\begin{pgfscope}%
\pgfsys@transformshift{2.480282in}{0.722917in}%
\pgfsys@useobject{currentmarker}{}%
\end{pgfscope}%
\begin{pgfscope}%
\pgfsys@transformshift{2.557124in}{0.682617in}%
\pgfsys@useobject{currentmarker}{}%
\end{pgfscope}%
\begin{pgfscope}%
\pgfsys@transformshift{2.633965in}{0.654305in}%
\pgfsys@useobject{currentmarker}{}%
\end{pgfscope}%
\begin{pgfscope}%
\pgfsys@transformshift{2.710807in}{0.630461in}%
\pgfsys@useobject{currentmarker}{}%
\end{pgfscope}%
\begin{pgfscope}%
\pgfsys@transformshift{2.787648in}{0.610090in}%
\pgfsys@useobject{currentmarker}{}%
\end{pgfscope}%
\begin{pgfscope}%
\pgfsys@transformshift{2.864490in}{0.593816in}%
\pgfsys@useobject{currentmarker}{}%
\end{pgfscope}%
\begin{pgfscope}%
\pgfsys@transformshift{2.941331in}{0.579931in}%
\pgfsys@useobject{currentmarker}{}%
\end{pgfscope}%
\begin{pgfscope}%
\pgfsys@transformshift{3.018172in}{0.567999in}%
\pgfsys@useobject{currentmarker}{}%
\end{pgfscope}%
\begin{pgfscope}%
\pgfsys@transformshift{3.095014in}{0.556442in}%
\pgfsys@useobject{currentmarker}{}%
\end{pgfscope}%
\begin{pgfscope}%
\pgfsys@transformshift{3.171855in}{0.547648in}%
\pgfsys@useobject{currentmarker}{}%
\end{pgfscope}%
\begin{pgfscope}%
\pgfsys@transformshift{3.248697in}{0.538505in}%
\pgfsys@useobject{currentmarker}{}%
\end{pgfscope}%
\begin{pgfscope}%
\pgfsys@transformshift{3.325538in}{0.531060in}%
\pgfsys@useobject{currentmarker}{}%
\end{pgfscope}%
\begin{pgfscope}%
\pgfsys@transformshift{3.402380in}{0.525042in}%
\pgfsys@useobject{currentmarker}{}%
\end{pgfscope}%
\begin{pgfscope}%
\pgfsys@transformshift{3.479221in}{0.518694in}%
\pgfsys@useobject{currentmarker}{}%
\end{pgfscope}%
\begin{pgfscope}%
\pgfsys@transformshift{3.556063in}{0.512620in}%
\pgfsys@useobject{currentmarker}{}%
\end{pgfscope}%
\begin{pgfscope}%
\pgfsys@transformshift{3.632904in}{0.507135in}%
\pgfsys@useobject{currentmarker}{}%
\end{pgfscope}%
\begin{pgfscope}%
\pgfsys@transformshift{3.709745in}{0.502407in}%
\pgfsys@useobject{currentmarker}{}%
\end{pgfscope}%
\end{pgfscope}%
\begin{pgfscope}%
\pgfsetbuttcap%
\pgfsetroundjoin%
\definecolor{currentfill}{rgb}{0.000000,0.000000,0.000000}%
\pgfsetfillcolor{currentfill}%
\pgfsetlinewidth{0.803000pt}%
\definecolor{currentstroke}{rgb}{0.000000,0.000000,0.000000}%
\pgfsetstrokecolor{currentstroke}%
\pgfsetdash{}{0pt}%
\pgfsys@defobject{currentmarker}{\pgfqpoint{0.000000in}{-0.048611in}}{\pgfqpoint{0.000000in}{0.000000in}}{%
\pgfpathmoveto{\pgfqpoint{0.000000in}{0.000000in}}%
\pgfpathlineto{\pgfqpoint{0.000000in}{-0.048611in}}%
\pgfusepath{stroke,fill}%
}%
\begin{pgfscope}%
\pgfsys@transformshift{1.020295in}{0.386111in}%
\pgfsys@useobject{currentmarker}{}%
\end{pgfscope}%
\end{pgfscope}%
\begin{pgfscope}%
\definecolor{textcolor}{rgb}{0.000000,0.000000,0.000000}%
\pgfsetstrokecolor{textcolor}%
\pgfsetfillcolor{textcolor}%
\pgftext[x=1.020295in,y=0.288889in,,top]{\color{textcolor}\sffamily\fontsize{10.000000}{12.000000}\selectfont 3}%
\end{pgfscope}%
\begin{pgfscope}%
\pgfsetbuttcap%
\pgfsetroundjoin%
\definecolor{currentfill}{rgb}{0.000000,0.000000,0.000000}%
\pgfsetfillcolor{currentfill}%
\pgfsetlinewidth{0.803000pt}%
\definecolor{currentstroke}{rgb}{0.000000,0.000000,0.000000}%
\pgfsetstrokecolor{currentstroke}%
\pgfsetdash{}{0pt}%
\pgfsys@defobject{currentmarker}{\pgfqpoint{0.000000in}{-0.048611in}}{\pgfqpoint{0.000000in}{0.000000in}}{%
\pgfpathmoveto{\pgfqpoint{0.000000in}{0.000000in}}%
\pgfpathlineto{\pgfqpoint{0.000000in}{-0.048611in}}%
\pgfusepath{stroke,fill}%
}%
\begin{pgfscope}%
\pgfsys@transformshift{1.788709in}{0.386111in}%
\pgfsys@useobject{currentmarker}{}%
\end{pgfscope}%
\end{pgfscope}%
\begin{pgfscope}%
\definecolor{textcolor}{rgb}{0.000000,0.000000,0.000000}%
\pgfsetstrokecolor{textcolor}%
\pgfsetfillcolor{textcolor}%
\pgftext[x=1.788709in,y=0.288889in,,top]{\color{textcolor}\sffamily\fontsize{10.000000}{12.000000}\selectfont 4}%
\end{pgfscope}%
\begin{pgfscope}%
\pgfsetbuttcap%
\pgfsetroundjoin%
\definecolor{currentfill}{rgb}{0.000000,0.000000,0.000000}%
\pgfsetfillcolor{currentfill}%
\pgfsetlinewidth{0.803000pt}%
\definecolor{currentstroke}{rgb}{0.000000,0.000000,0.000000}%
\pgfsetstrokecolor{currentstroke}%
\pgfsetdash{}{0pt}%
\pgfsys@defobject{currentmarker}{\pgfqpoint{0.000000in}{-0.048611in}}{\pgfqpoint{0.000000in}{0.000000in}}{%
\pgfpathmoveto{\pgfqpoint{0.000000in}{0.000000in}}%
\pgfpathlineto{\pgfqpoint{0.000000in}{-0.048611in}}%
\pgfusepath{stroke,fill}%
}%
\begin{pgfscope}%
\pgfsys@transformshift{2.557124in}{0.386111in}%
\pgfsys@useobject{currentmarker}{}%
\end{pgfscope}%
\end{pgfscope}%
\begin{pgfscope}%
\definecolor{textcolor}{rgb}{0.000000,0.000000,0.000000}%
\pgfsetstrokecolor{textcolor}%
\pgfsetfillcolor{textcolor}%
\pgftext[x=2.557124in,y=0.288889in,,top]{\color{textcolor}\sffamily\fontsize{10.000000}{12.000000}\selectfont 5}%
\end{pgfscope}%
\begin{pgfscope}%
\pgfsetbuttcap%
\pgfsetroundjoin%
\definecolor{currentfill}{rgb}{0.000000,0.000000,0.000000}%
\pgfsetfillcolor{currentfill}%
\pgfsetlinewidth{0.803000pt}%
\definecolor{currentstroke}{rgb}{0.000000,0.000000,0.000000}%
\pgfsetstrokecolor{currentstroke}%
\pgfsetdash{}{0pt}%
\pgfsys@defobject{currentmarker}{\pgfqpoint{0.000000in}{-0.048611in}}{\pgfqpoint{0.000000in}{0.000000in}}{%
\pgfpathmoveto{\pgfqpoint{0.000000in}{0.000000in}}%
\pgfpathlineto{\pgfqpoint{0.000000in}{-0.048611in}}%
\pgfusepath{stroke,fill}%
}%
\begin{pgfscope}%
\pgfsys@transformshift{3.325538in}{0.386111in}%
\pgfsys@useobject{currentmarker}{}%
\end{pgfscope}%
\end{pgfscope}%
\begin{pgfscope}%
\definecolor{textcolor}{rgb}{0.000000,0.000000,0.000000}%
\pgfsetstrokecolor{textcolor}%
\pgfsetfillcolor{textcolor}%
\pgftext[x=3.325538in,y=0.288889in,,top]{\color{textcolor}\sffamily\fontsize{10.000000}{12.000000}\selectfont 6}%
\end{pgfscope}%
\begin{pgfscope}%
\pgfsetbuttcap%
\pgfsetroundjoin%
\definecolor{currentfill}{rgb}{0.000000,0.000000,0.000000}%
\pgfsetfillcolor{currentfill}%
\pgfsetlinewidth{0.803000pt}%
\definecolor{currentstroke}{rgb}{0.000000,0.000000,0.000000}%
\pgfsetstrokecolor{currentstroke}%
\pgfsetdash{}{0pt}%
\pgfsys@defobject{currentmarker}{\pgfqpoint{-0.048611in}{0.000000in}}{\pgfqpoint{0.000000in}{0.000000in}}{%
\pgfpathmoveto{\pgfqpoint{0.000000in}{0.000000in}}%
\pgfpathlineto{\pgfqpoint{-0.048611in}{0.000000in}}%
\pgfusepath{stroke,fill}%
}%
\begin{pgfscope}%
\pgfsys@transformshift{0.469444in}{0.401589in}%
\pgfsys@useobject{currentmarker}{}%
\end{pgfscope}%
\end{pgfscope}%
\begin{pgfscope}%
\definecolor{textcolor}{rgb}{0.000000,0.000000,0.000000}%
\pgfsetstrokecolor{textcolor}%
\pgfsetfillcolor{textcolor}%
\pgftext[x=0.151343in,y=0.348828in,left,base]{\color{textcolor}\sffamily\fontsize{10.000000}{12.000000}\selectfont 0.0}%
\end{pgfscope}%
\begin{pgfscope}%
\pgfsetbuttcap%
\pgfsetroundjoin%
\definecolor{currentfill}{rgb}{0.000000,0.000000,0.000000}%
\pgfsetfillcolor{currentfill}%
\pgfsetlinewidth{0.803000pt}%
\definecolor{currentstroke}{rgb}{0.000000,0.000000,0.000000}%
\pgfsetstrokecolor{currentstroke}%
\pgfsetdash{}{0pt}%
\pgfsys@defobject{currentmarker}{\pgfqpoint{-0.048611in}{0.000000in}}{\pgfqpoint{0.000000in}{0.000000in}}{%
\pgfpathmoveto{\pgfqpoint{0.000000in}{0.000000in}}%
\pgfpathlineto{\pgfqpoint{-0.048611in}{0.000000in}}%
\pgfusepath{stroke,fill}%
}%
\begin{pgfscope}%
\pgfsys@transformshift{0.469444in}{0.902052in}%
\pgfsys@useobject{currentmarker}{}%
\end{pgfscope}%
\end{pgfscope}%
\begin{pgfscope}%
\definecolor{textcolor}{rgb}{0.000000,0.000000,0.000000}%
\pgfsetstrokecolor{textcolor}%
\pgfsetfillcolor{textcolor}%
\pgftext[x=0.151343in,y=0.849291in,left,base]{\color{textcolor}\sffamily\fontsize{10.000000}{12.000000}\selectfont 0.5}%
\end{pgfscope}%
\begin{pgfscope}%
\pgfsetbuttcap%
\pgfsetroundjoin%
\definecolor{currentfill}{rgb}{0.000000,0.000000,0.000000}%
\pgfsetfillcolor{currentfill}%
\pgfsetlinewidth{0.803000pt}%
\definecolor{currentstroke}{rgb}{0.000000,0.000000,0.000000}%
\pgfsetstrokecolor{currentstroke}%
\pgfsetdash{}{0pt}%
\pgfsys@defobject{currentmarker}{\pgfqpoint{-0.048611in}{0.000000in}}{\pgfqpoint{0.000000in}{0.000000in}}{%
\pgfpathmoveto{\pgfqpoint{0.000000in}{0.000000in}}%
\pgfpathlineto{\pgfqpoint{-0.048611in}{0.000000in}}%
\pgfusepath{stroke,fill}%
}%
\begin{pgfscope}%
\pgfsys@transformshift{0.469444in}{1.402515in}%
\pgfsys@useobject{currentmarker}{}%
\end{pgfscope}%
\end{pgfscope}%
\begin{pgfscope}%
\definecolor{textcolor}{rgb}{0.000000,0.000000,0.000000}%
\pgfsetstrokecolor{textcolor}%
\pgfsetfillcolor{textcolor}%
\pgftext[x=0.151343in,y=1.349753in,left,base]{\color{textcolor}\sffamily\fontsize{10.000000}{12.000000}\selectfont 1.0}%
\end{pgfscope}%
\begin{pgfscope}%
\pgfsetbuttcap%
\pgfsetroundjoin%
\definecolor{currentfill}{rgb}{0.000000,0.000000,0.000000}%
\pgfsetfillcolor{currentfill}%
\pgfsetlinewidth{0.803000pt}%
\definecolor{currentstroke}{rgb}{0.000000,0.000000,0.000000}%
\pgfsetstrokecolor{currentstroke}%
\pgfsetdash{}{0pt}%
\pgfsys@defobject{currentmarker}{\pgfqpoint{-0.048611in}{0.000000in}}{\pgfqpoint{0.000000in}{0.000000in}}{%
\pgfpathmoveto{\pgfqpoint{0.000000in}{0.000000in}}%
\pgfpathlineto{\pgfqpoint{-0.048611in}{0.000000in}}%
\pgfusepath{stroke,fill}%
}%
\begin{pgfscope}%
\pgfsys@transformshift{0.469444in}{1.902978in}%
\pgfsys@useobject{currentmarker}{}%
\end{pgfscope}%
\end{pgfscope}%
\begin{pgfscope}%
\definecolor{textcolor}{rgb}{0.000000,0.000000,0.000000}%
\pgfsetstrokecolor{textcolor}%
\pgfsetfillcolor{textcolor}%
\pgftext[x=0.151343in,y=1.850216in,left,base]{\color{textcolor}\sffamily\fontsize{10.000000}{12.000000}\selectfont 1.5}%
\end{pgfscope}%
\begin{pgfscope}%
\pgfsetbuttcap%
\pgfsetroundjoin%
\definecolor{currentfill}{rgb}{0.000000,0.000000,0.000000}%
\pgfsetfillcolor{currentfill}%
\pgfsetlinewidth{0.803000pt}%
\definecolor{currentstroke}{rgb}{0.000000,0.000000,0.000000}%
\pgfsetstrokecolor{currentstroke}%
\pgfsetdash{}{0pt}%
\pgfsys@defobject{currentmarker}{\pgfqpoint{-0.048611in}{0.000000in}}{\pgfqpoint{0.000000in}{0.000000in}}{%
\pgfpathmoveto{\pgfqpoint{0.000000in}{0.000000in}}%
\pgfpathlineto{\pgfqpoint{-0.048611in}{0.000000in}}%
\pgfusepath{stroke,fill}%
}%
\begin{pgfscope}%
\pgfsys@transformshift{0.469444in}{2.403440in}%
\pgfsys@useobject{currentmarker}{}%
\end{pgfscope}%
\end{pgfscope}%
\begin{pgfscope}%
\definecolor{textcolor}{rgb}{0.000000,0.000000,0.000000}%
\pgfsetstrokecolor{textcolor}%
\pgfsetfillcolor{textcolor}%
\pgftext[x=0.151343in,y=2.350679in,left,base]{\color{textcolor}\sffamily\fontsize{10.000000}{12.000000}\selectfont 2.0}%
\end{pgfscope}%
\begin{pgfscope}%
\pgfpathrectangle{\pgfqpoint{0.469444in}{0.386111in}}{\pgfqpoint{3.406944in}{2.316667in}}%
\pgfusepath{clip}%
\pgfsetrectcap%
\pgfsetroundjoin%
\pgfsetlinewidth{1.505625pt}%
\definecolor{currentstroke}{rgb}{0.121569,0.466667,0.705882}%
\pgfsetstrokecolor{currentstroke}%
\pgfsetdash{}{0pt}%
\pgfpathmoveto{\pgfqpoint{0.636088in}{0.652311in}}%
\pgfpathlineto{\pgfqpoint{0.712929in}{0.690161in}}%
\pgfpathlineto{\pgfqpoint{0.789771in}{0.731075in}}%
\pgfpathlineto{\pgfqpoint{0.866612in}{0.774846in}}%
\pgfpathlineto{\pgfqpoint{0.943454in}{0.821763in}}%
\pgfpathlineto{\pgfqpoint{1.020295in}{0.870951in}}%
\pgfpathlineto{\pgfqpoint{1.097137in}{0.922603in}}%
\pgfpathlineto{\pgfqpoint{1.173978in}{0.980884in}}%
\pgfpathlineto{\pgfqpoint{1.250819in}{1.040304in}}%
\pgfpathlineto{\pgfqpoint{1.327661in}{1.106813in}}%
\pgfpathlineto{\pgfqpoint{1.404502in}{1.176899in}}%
\pgfpathlineto{\pgfqpoint{1.481344in}{1.250229in}}%
\pgfpathlineto{\pgfqpoint{1.558185in}{1.331307in}}%
\pgfpathlineto{\pgfqpoint{1.635027in}{1.419967in}}%
\pgfpathlineto{\pgfqpoint{1.711868in}{1.515850in}}%
\pgfpathlineto{\pgfqpoint{1.788709in}{1.631287in}}%
\pgfpathlineto{\pgfqpoint{1.865551in}{1.759879in}}%
\pgfpathlineto{\pgfqpoint{1.942392in}{1.939957in}}%
\pgfpathlineto{\pgfqpoint{2.019234in}{2.130913in}}%
\pgfpathlineto{\pgfqpoint{2.096075in}{2.515131in}}%
\pgfpathlineto{\pgfqpoint{2.172917in}{2.406706in}}%
\pgfpathlineto{\pgfqpoint{2.249758in}{1.210094in}}%
\pgfpathlineto{\pgfqpoint{2.326600in}{0.882843in}}%
\pgfpathlineto{\pgfqpoint{2.403441in}{0.784564in}}%
\pgfpathlineto{\pgfqpoint{2.480282in}{0.722917in}}%
\pgfpathlineto{\pgfqpoint{2.557124in}{0.682617in}}%
\pgfpathlineto{\pgfqpoint{2.633965in}{0.654305in}}%
\pgfpathlineto{\pgfqpoint{2.710807in}{0.630461in}}%
\pgfpathlineto{\pgfqpoint{2.787648in}{0.610090in}}%
\pgfpathlineto{\pgfqpoint{2.864490in}{0.593816in}}%
\pgfpathlineto{\pgfqpoint{2.941331in}{0.579931in}}%
\pgfpathlineto{\pgfqpoint{3.018172in}{0.567999in}}%
\pgfpathlineto{\pgfqpoint{3.095014in}{0.556442in}}%
\pgfpathlineto{\pgfqpoint{3.171855in}{0.547648in}}%
\pgfpathlineto{\pgfqpoint{3.248697in}{0.538505in}}%
\pgfpathlineto{\pgfqpoint{3.325538in}{0.531060in}}%
\pgfpathlineto{\pgfqpoint{3.402380in}{0.525042in}}%
\pgfpathlineto{\pgfqpoint{3.479221in}{0.518694in}}%
\pgfpathlineto{\pgfqpoint{3.556063in}{0.512620in}}%
\pgfpathlineto{\pgfqpoint{3.632904in}{0.507135in}}%
\pgfpathlineto{\pgfqpoint{3.709745in}{0.502407in}}%
\pgfusepath{stroke}%
\end{pgfscope}%
\begin{pgfscope}%
\pgfsetrectcap%
\pgfsetmiterjoin%
\pgfsetlinewidth{0.803000pt}%
\definecolor{currentstroke}{rgb}{0.000000,0.000000,0.000000}%
\pgfsetstrokecolor{currentstroke}%
\pgfsetdash{}{0pt}%
\pgfpathmoveto{\pgfqpoint{0.469444in}{0.386111in}}%
\pgfpathlineto{\pgfqpoint{0.469444in}{2.702778in}}%
\pgfusepath{stroke}%
\end{pgfscope}%
\begin{pgfscope}%
\pgfsetrectcap%
\pgfsetmiterjoin%
\pgfsetlinewidth{0.803000pt}%
\definecolor{currentstroke}{rgb}{0.000000,0.000000,0.000000}%
\pgfsetstrokecolor{currentstroke}%
\pgfsetdash{}{0pt}%
\pgfpathmoveto{\pgfqpoint{3.876389in}{0.386111in}}%
\pgfpathlineto{\pgfqpoint{3.876389in}{2.702778in}}%
\pgfusepath{stroke}%
\end{pgfscope}%
\begin{pgfscope}%
\pgfsetrectcap%
\pgfsetmiterjoin%
\pgfsetlinewidth{0.803000pt}%
\definecolor{currentstroke}{rgb}{0.000000,0.000000,0.000000}%
\pgfsetstrokecolor{currentstroke}%
\pgfsetdash{}{0pt}%
\pgfpathmoveto{\pgfqpoint{0.469444in}{0.386111in}}%
\pgfpathlineto{\pgfqpoint{3.876389in}{0.386111in}}%
\pgfusepath{stroke}%
\end{pgfscope}%
\begin{pgfscope}%
\pgfsetrectcap%
\pgfsetmiterjoin%
\pgfsetlinewidth{0.803000pt}%
\definecolor{currentstroke}{rgb}{0.000000,0.000000,0.000000}%
\pgfsetstrokecolor{currentstroke}%
\pgfsetdash{}{0pt}%
\pgfpathmoveto{\pgfqpoint{0.469444in}{2.702778in}}%
\pgfpathlineto{\pgfqpoint{3.876389in}{2.702778in}}%
\pgfusepath{stroke}%
\end{pgfscope}%
\begin{pgfscope}%
\definecolor{textcolor}{rgb}{0.000000,0.000000,0.000000}%
\pgfsetstrokecolor{textcolor}%
\pgfsetfillcolor{textcolor}%
\pgftext[x=2.172917in,y=2.786111in,,base]{\color{textcolor}\sffamily\fontsize{12.000000}{14.400000}\selectfont \(\displaystyle  N = 16 \)}%
\end{pgfscope}%
\begin{pgfscope}%
\pgfsetbuttcap%
\pgfsetmiterjoin%
\definecolor{currentfill}{rgb}{1.000000,1.000000,1.000000}%
\pgfsetfillcolor{currentfill}%
\pgfsetlinewidth{0.000000pt}%
\definecolor{currentstroke}{rgb}{0.000000,0.000000,0.000000}%
\pgfsetstrokecolor{currentstroke}%
\pgfsetstrokeopacity{0.000000}%
\pgfsetdash{}{0pt}%
\pgfpathmoveto{\pgfqpoint{4.394444in}{0.386111in}}%
\pgfpathlineto{\pgfqpoint{7.801389in}{0.386111in}}%
\pgfpathlineto{\pgfqpoint{7.801389in}{2.702778in}}%
\pgfpathlineto{\pgfqpoint{4.394444in}{2.702778in}}%
\pgfpathclose%
\pgfusepath{fill}%
\end{pgfscope}%
\begin{pgfscope}%
\pgfpathrectangle{\pgfqpoint{4.394444in}{0.386111in}}{\pgfqpoint{3.406944in}{2.316667in}}%
\pgfusepath{clip}%
\pgfsetbuttcap%
\pgfsetroundjoin%
\definecolor{currentfill}{rgb}{0.121569,0.466667,0.705882}%
\pgfsetfillcolor{currentfill}%
\pgfsetfillopacity{0.300000}%
\pgfsetlinewidth{1.003750pt}%
\definecolor{currentstroke}{rgb}{0.121569,0.466667,0.705882}%
\pgfsetstrokecolor{currentstroke}%
\pgfsetstrokeopacity{0.300000}%
\pgfsetdash{}{0pt}%
\pgfpathmoveto{\pgfqpoint{4.561088in}{0.623398in}}%
\pgfpathlineto{\pgfqpoint{4.561088in}{0.622816in}}%
\pgfpathlineto{\pgfqpoint{4.637929in}{0.653528in}}%
\pgfpathlineto{\pgfqpoint{4.714771in}{0.685891in}}%
\pgfpathlineto{\pgfqpoint{4.791612in}{0.720988in}}%
\pgfpathlineto{\pgfqpoint{4.868454in}{0.758337in}}%
\pgfpathlineto{\pgfqpoint{4.945295in}{0.796636in}}%
\pgfpathlineto{\pgfqpoint{5.022137in}{0.840996in}}%
\pgfpathlineto{\pgfqpoint{5.098978in}{0.886247in}}%
\pgfpathlineto{\pgfqpoint{5.175819in}{0.932925in}}%
\pgfpathlineto{\pgfqpoint{5.252661in}{0.985758in}}%
\pgfpathlineto{\pgfqpoint{5.329502in}{1.037290in}}%
\pgfpathlineto{\pgfqpoint{5.406344in}{1.098406in}}%
\pgfpathlineto{\pgfqpoint{5.483185in}{1.163385in}}%
\pgfpathlineto{\pgfqpoint{5.560027in}{1.241841in}}%
\pgfpathlineto{\pgfqpoint{5.636868in}{1.302404in}}%
\pgfpathlineto{\pgfqpoint{5.713709in}{1.399911in}}%
\pgfpathlineto{\pgfqpoint{5.790551in}{1.508073in}}%
\pgfpathlineto{\pgfqpoint{5.867392in}{1.622251in}}%
\pgfpathlineto{\pgfqpoint{5.944234in}{1.777918in}}%
\pgfpathlineto{\pgfqpoint{6.021075in}{2.043728in}}%
\pgfpathlineto{\pgfqpoint{6.097917in}{2.161839in}}%
\pgfpathlineto{\pgfqpoint{6.174758in}{0.924549in}}%
\pgfpathlineto{\pgfqpoint{6.251600in}{0.792944in}}%
\pgfpathlineto{\pgfqpoint{6.328441in}{0.719356in}}%
\pgfpathlineto{\pgfqpoint{6.405282in}{0.675752in}}%
\pgfpathlineto{\pgfqpoint{6.482124in}{0.641404in}}%
\pgfpathlineto{\pgfqpoint{6.558965in}{0.618792in}}%
\pgfpathlineto{\pgfqpoint{6.635807in}{0.601596in}}%
\pgfpathlineto{\pgfqpoint{6.712648in}{0.585420in}}%
\pgfpathlineto{\pgfqpoint{6.789490in}{0.573660in}}%
\pgfpathlineto{\pgfqpoint{6.866331in}{0.561111in}}%
\pgfpathlineto{\pgfqpoint{6.943172in}{0.552349in}}%
\pgfpathlineto{\pgfqpoint{7.020014in}{0.543933in}}%
\pgfpathlineto{\pgfqpoint{7.096855in}{0.537931in}}%
\pgfpathlineto{\pgfqpoint{7.173697in}{0.529750in}}%
\pgfpathlineto{\pgfqpoint{7.250538in}{0.522988in}}%
\pgfpathlineto{\pgfqpoint{7.327380in}{0.518876in}}%
\pgfpathlineto{\pgfqpoint{7.404221in}{0.514225in}}%
\pgfpathlineto{\pgfqpoint{7.481063in}{0.509385in}}%
\pgfpathlineto{\pgfqpoint{7.557904in}{0.504553in}}%
\pgfpathlineto{\pgfqpoint{7.634745in}{0.501717in}}%
\pgfpathlineto{\pgfqpoint{7.634745in}{0.503102in}}%
\pgfpathlineto{\pgfqpoint{7.634745in}{0.503102in}}%
\pgfpathlineto{\pgfqpoint{7.557904in}{0.506899in}}%
\pgfpathlineto{\pgfqpoint{7.481063in}{0.510856in}}%
\pgfpathlineto{\pgfqpoint{7.404221in}{0.516372in}}%
\pgfpathlineto{\pgfqpoint{7.327380in}{0.520404in}}%
\pgfpathlineto{\pgfqpoint{7.250538in}{0.527836in}}%
\pgfpathlineto{\pgfqpoint{7.173697in}{0.531999in}}%
\pgfpathlineto{\pgfqpoint{7.096855in}{0.538449in}}%
\pgfpathlineto{\pgfqpoint{7.020014in}{0.548017in}}%
\pgfpathlineto{\pgfqpoint{6.943172in}{0.554162in}}%
\pgfpathlineto{\pgfqpoint{6.866331in}{0.563398in}}%
\pgfpathlineto{\pgfqpoint{6.789490in}{0.575459in}}%
\pgfpathlineto{\pgfqpoint{6.712648in}{0.587065in}}%
\pgfpathlineto{\pgfqpoint{6.635807in}{0.604641in}}%
\pgfpathlineto{\pgfqpoint{6.558965in}{0.626406in}}%
\pgfpathlineto{\pgfqpoint{6.482124in}{0.647170in}}%
\pgfpathlineto{\pgfqpoint{6.405282in}{0.681654in}}%
\pgfpathlineto{\pgfqpoint{6.328441in}{0.724448in}}%
\pgfpathlineto{\pgfqpoint{6.251600in}{0.799623in}}%
\pgfpathlineto{\pgfqpoint{6.174758in}{0.975178in}}%
\pgfpathlineto{\pgfqpoint{6.097917in}{2.597475in}}%
\pgfpathlineto{\pgfqpoint{6.021075in}{2.122287in}}%
\pgfpathlineto{\pgfqpoint{5.944234in}{1.875042in}}%
\pgfpathlineto{\pgfqpoint{5.867392in}{1.681565in}}%
\pgfpathlineto{\pgfqpoint{5.790551in}{1.531302in}}%
\pgfpathlineto{\pgfqpoint{5.713709in}{1.434003in}}%
\pgfpathlineto{\pgfqpoint{5.636868in}{1.357833in}}%
\pgfpathlineto{\pgfqpoint{5.560027in}{1.251598in}}%
\pgfpathlineto{\pgfqpoint{5.483185in}{1.173119in}}%
\pgfpathlineto{\pgfqpoint{5.406344in}{1.108701in}}%
\pgfpathlineto{\pgfqpoint{5.329502in}{1.049319in}}%
\pgfpathlineto{\pgfqpoint{5.252661in}{0.992009in}}%
\pgfpathlineto{\pgfqpoint{5.175819in}{0.943864in}}%
\pgfpathlineto{\pgfqpoint{5.098978in}{0.889907in}}%
\pgfpathlineto{\pgfqpoint{5.022137in}{0.843821in}}%
\pgfpathlineto{\pgfqpoint{4.945295in}{0.803649in}}%
\pgfpathlineto{\pgfqpoint{4.868454in}{0.761298in}}%
\pgfpathlineto{\pgfqpoint{4.791612in}{0.723454in}}%
\pgfpathlineto{\pgfqpoint{4.714771in}{0.687228in}}%
\pgfpathlineto{\pgfqpoint{4.637929in}{0.654446in}}%
\pgfpathlineto{\pgfqpoint{4.561088in}{0.623398in}}%
\pgfpathclose%
\pgfusepath{stroke,fill}%
\end{pgfscope}%
\begin{pgfscope}%
\pgfpathrectangle{\pgfqpoint{4.394444in}{0.386111in}}{\pgfqpoint{3.406944in}{2.316667in}}%
\pgfusepath{clip}%
\pgfsetbuttcap%
\pgfsetroundjoin%
\definecolor{currentfill}{rgb}{0.121569,0.466667,0.705882}%
\pgfsetfillcolor{currentfill}%
\pgfsetlinewidth{1.003750pt}%
\definecolor{currentstroke}{rgb}{0.121569,0.466667,0.705882}%
\pgfsetstrokecolor{currentstroke}%
\pgfsetdash{}{0pt}%
\pgfsys@defobject{currentmarker}{\pgfqpoint{-0.009821in}{-0.009821in}}{\pgfqpoint{0.009821in}{0.009821in}}{%
\pgfpathmoveto{\pgfqpoint{0.000000in}{-0.009821in}}%
\pgfpathcurveto{\pgfqpoint{0.002605in}{-0.009821in}}{\pgfqpoint{0.005103in}{-0.008786in}}{\pgfqpoint{0.006944in}{-0.006944in}}%
\pgfpathcurveto{\pgfqpoint{0.008786in}{-0.005103in}}{\pgfqpoint{0.009821in}{-0.002605in}}{\pgfqpoint{0.009821in}{0.000000in}}%
\pgfpathcurveto{\pgfqpoint{0.009821in}{0.002605in}}{\pgfqpoint{0.008786in}{0.005103in}}{\pgfqpoint{0.006944in}{0.006944in}}%
\pgfpathcurveto{\pgfqpoint{0.005103in}{0.008786in}}{\pgfqpoint{0.002605in}{0.009821in}}{\pgfqpoint{0.000000in}{0.009821in}}%
\pgfpathcurveto{\pgfqpoint{-0.002605in}{0.009821in}}{\pgfqpoint{-0.005103in}{0.008786in}}{\pgfqpoint{-0.006944in}{0.006944in}}%
\pgfpathcurveto{\pgfqpoint{-0.008786in}{0.005103in}}{\pgfqpoint{-0.009821in}{0.002605in}}{\pgfqpoint{-0.009821in}{0.000000in}}%
\pgfpathcurveto{\pgfqpoint{-0.009821in}{-0.002605in}}{\pgfqpoint{-0.008786in}{-0.005103in}}{\pgfqpoint{-0.006944in}{-0.006944in}}%
\pgfpathcurveto{\pgfqpoint{-0.005103in}{-0.008786in}}{\pgfqpoint{-0.002605in}{-0.009821in}}{\pgfqpoint{0.000000in}{-0.009821in}}%
\pgfpathclose%
\pgfusepath{stroke,fill}%
}%
\begin{pgfscope}%
\pgfsys@transformshift{4.561088in}{0.623107in}%
\pgfsys@useobject{currentmarker}{}%
\end{pgfscope}%
\begin{pgfscope}%
\pgfsys@transformshift{4.637929in}{0.653987in}%
\pgfsys@useobject{currentmarker}{}%
\end{pgfscope}%
\begin{pgfscope}%
\pgfsys@transformshift{4.714771in}{0.686560in}%
\pgfsys@useobject{currentmarker}{}%
\end{pgfscope}%
\begin{pgfscope}%
\pgfsys@transformshift{4.791612in}{0.722221in}%
\pgfsys@useobject{currentmarker}{}%
\end{pgfscope}%
\begin{pgfscope}%
\pgfsys@transformshift{4.868454in}{0.759817in}%
\pgfsys@useobject{currentmarker}{}%
\end{pgfscope}%
\begin{pgfscope}%
\pgfsys@transformshift{4.945295in}{0.800142in}%
\pgfsys@useobject{currentmarker}{}%
\end{pgfscope}%
\begin{pgfscope}%
\pgfsys@transformshift{5.022137in}{0.842408in}%
\pgfsys@useobject{currentmarker}{}%
\end{pgfscope}%
\begin{pgfscope}%
\pgfsys@transformshift{5.098978in}{0.888077in}%
\pgfsys@useobject{currentmarker}{}%
\end{pgfscope}%
\begin{pgfscope}%
\pgfsys@transformshift{5.175819in}{0.938394in}%
\pgfsys@useobject{currentmarker}{}%
\end{pgfscope}%
\begin{pgfscope}%
\pgfsys@transformshift{5.252661in}{0.988884in}%
\pgfsys@useobject{currentmarker}{}%
\end{pgfscope}%
\begin{pgfscope}%
\pgfsys@transformshift{5.329502in}{1.043305in}%
\pgfsys@useobject{currentmarker}{}%
\end{pgfscope}%
\begin{pgfscope}%
\pgfsys@transformshift{5.406344in}{1.103553in}%
\pgfsys@useobject{currentmarker}{}%
\end{pgfscope}%
\begin{pgfscope}%
\pgfsys@transformshift{5.483185in}{1.168252in}%
\pgfsys@useobject{currentmarker}{}%
\end{pgfscope}%
\begin{pgfscope}%
\pgfsys@transformshift{5.560027in}{1.246720in}%
\pgfsys@useobject{currentmarker}{}%
\end{pgfscope}%
\begin{pgfscope}%
\pgfsys@transformshift{5.636868in}{1.330119in}%
\pgfsys@useobject{currentmarker}{}%
\end{pgfscope}%
\begin{pgfscope}%
\pgfsys@transformshift{5.713709in}{1.416957in}%
\pgfsys@useobject{currentmarker}{}%
\end{pgfscope}%
\begin{pgfscope}%
\pgfsys@transformshift{5.790551in}{1.519687in}%
\pgfsys@useobject{currentmarker}{}%
\end{pgfscope}%
\begin{pgfscope}%
\pgfsys@transformshift{5.867392in}{1.651908in}%
\pgfsys@useobject{currentmarker}{}%
\end{pgfscope}%
\begin{pgfscope}%
\pgfsys@transformshift{5.944234in}{1.826480in}%
\pgfsys@useobject{currentmarker}{}%
\end{pgfscope}%
\begin{pgfscope}%
\pgfsys@transformshift{6.021075in}{2.083008in}%
\pgfsys@useobject{currentmarker}{}%
\end{pgfscope}%
\begin{pgfscope}%
\pgfsys@transformshift{6.097917in}{2.379657in}%
\pgfsys@useobject{currentmarker}{}%
\end{pgfscope}%
\begin{pgfscope}%
\pgfsys@transformshift{6.174758in}{0.949864in}%
\pgfsys@useobject{currentmarker}{}%
\end{pgfscope}%
\begin{pgfscope}%
\pgfsys@transformshift{6.251600in}{0.796284in}%
\pgfsys@useobject{currentmarker}{}%
\end{pgfscope}%
\begin{pgfscope}%
\pgfsys@transformshift{6.328441in}{0.721902in}%
\pgfsys@useobject{currentmarker}{}%
\end{pgfscope}%
\begin{pgfscope}%
\pgfsys@transformshift{6.405282in}{0.678703in}%
\pgfsys@useobject{currentmarker}{}%
\end{pgfscope}%
\begin{pgfscope}%
\pgfsys@transformshift{6.482124in}{0.644287in}%
\pgfsys@useobject{currentmarker}{}%
\end{pgfscope}%
\begin{pgfscope}%
\pgfsys@transformshift{6.558965in}{0.622599in}%
\pgfsys@useobject{currentmarker}{}%
\end{pgfscope}%
\begin{pgfscope}%
\pgfsys@transformshift{6.635807in}{0.603119in}%
\pgfsys@useobject{currentmarker}{}%
\end{pgfscope}%
\begin{pgfscope}%
\pgfsys@transformshift{6.712648in}{0.586242in}%
\pgfsys@useobject{currentmarker}{}%
\end{pgfscope}%
\begin{pgfscope}%
\pgfsys@transformshift{6.789490in}{0.574559in}%
\pgfsys@useobject{currentmarker}{}%
\end{pgfscope}%
\begin{pgfscope}%
\pgfsys@transformshift{6.866331in}{0.562255in}%
\pgfsys@useobject{currentmarker}{}%
\end{pgfscope}%
\begin{pgfscope}%
\pgfsys@transformshift{6.943172in}{0.553256in}%
\pgfsys@useobject{currentmarker}{}%
\end{pgfscope}%
\begin{pgfscope}%
\pgfsys@transformshift{7.020014in}{0.545975in}%
\pgfsys@useobject{currentmarker}{}%
\end{pgfscope}%
\begin{pgfscope}%
\pgfsys@transformshift{7.096855in}{0.538190in}%
\pgfsys@useobject{currentmarker}{}%
\end{pgfscope}%
\begin{pgfscope}%
\pgfsys@transformshift{7.173697in}{0.530875in}%
\pgfsys@useobject{currentmarker}{}%
\end{pgfscope}%
\begin{pgfscope}%
\pgfsys@transformshift{7.250538in}{0.525412in}%
\pgfsys@useobject{currentmarker}{}%
\end{pgfscope}%
\begin{pgfscope}%
\pgfsys@transformshift{7.327380in}{0.519640in}%
\pgfsys@useobject{currentmarker}{}%
\end{pgfscope}%
\begin{pgfscope}%
\pgfsys@transformshift{7.404221in}{0.515299in}%
\pgfsys@useobject{currentmarker}{}%
\end{pgfscope}%
\begin{pgfscope}%
\pgfsys@transformshift{7.481063in}{0.510120in}%
\pgfsys@useobject{currentmarker}{}%
\end{pgfscope}%
\begin{pgfscope}%
\pgfsys@transformshift{7.557904in}{0.505726in}%
\pgfsys@useobject{currentmarker}{}%
\end{pgfscope}%
\begin{pgfscope}%
\pgfsys@transformshift{7.634745in}{0.502410in}%
\pgfsys@useobject{currentmarker}{}%
\end{pgfscope}%
\end{pgfscope}%
\begin{pgfscope}%
\pgfsetbuttcap%
\pgfsetroundjoin%
\definecolor{currentfill}{rgb}{0.000000,0.000000,0.000000}%
\pgfsetfillcolor{currentfill}%
\pgfsetlinewidth{0.803000pt}%
\definecolor{currentstroke}{rgb}{0.000000,0.000000,0.000000}%
\pgfsetstrokecolor{currentstroke}%
\pgfsetdash{}{0pt}%
\pgfsys@defobject{currentmarker}{\pgfqpoint{0.000000in}{-0.048611in}}{\pgfqpoint{0.000000in}{0.000000in}}{%
\pgfpathmoveto{\pgfqpoint{0.000000in}{0.000000in}}%
\pgfpathlineto{\pgfqpoint{0.000000in}{-0.048611in}}%
\pgfusepath{stroke,fill}%
}%
\begin{pgfscope}%
\pgfsys@transformshift{4.945295in}{0.386111in}%
\pgfsys@useobject{currentmarker}{}%
\end{pgfscope}%
\end{pgfscope}%
\begin{pgfscope}%
\definecolor{textcolor}{rgb}{0.000000,0.000000,0.000000}%
\pgfsetstrokecolor{textcolor}%
\pgfsetfillcolor{textcolor}%
\pgftext[x=4.945295in,y=0.288889in,,top]{\color{textcolor}\sffamily\fontsize{10.000000}{12.000000}\selectfont 3}%
\end{pgfscope}%
\begin{pgfscope}%
\pgfsetbuttcap%
\pgfsetroundjoin%
\definecolor{currentfill}{rgb}{0.000000,0.000000,0.000000}%
\pgfsetfillcolor{currentfill}%
\pgfsetlinewidth{0.803000pt}%
\definecolor{currentstroke}{rgb}{0.000000,0.000000,0.000000}%
\pgfsetstrokecolor{currentstroke}%
\pgfsetdash{}{0pt}%
\pgfsys@defobject{currentmarker}{\pgfqpoint{0.000000in}{-0.048611in}}{\pgfqpoint{0.000000in}{0.000000in}}{%
\pgfpathmoveto{\pgfqpoint{0.000000in}{0.000000in}}%
\pgfpathlineto{\pgfqpoint{0.000000in}{-0.048611in}}%
\pgfusepath{stroke,fill}%
}%
\begin{pgfscope}%
\pgfsys@transformshift{5.713709in}{0.386111in}%
\pgfsys@useobject{currentmarker}{}%
\end{pgfscope}%
\end{pgfscope}%
\begin{pgfscope}%
\definecolor{textcolor}{rgb}{0.000000,0.000000,0.000000}%
\pgfsetstrokecolor{textcolor}%
\pgfsetfillcolor{textcolor}%
\pgftext[x=5.713709in,y=0.288889in,,top]{\color{textcolor}\sffamily\fontsize{10.000000}{12.000000}\selectfont 4}%
\end{pgfscope}%
\begin{pgfscope}%
\pgfsetbuttcap%
\pgfsetroundjoin%
\definecolor{currentfill}{rgb}{0.000000,0.000000,0.000000}%
\pgfsetfillcolor{currentfill}%
\pgfsetlinewidth{0.803000pt}%
\definecolor{currentstroke}{rgb}{0.000000,0.000000,0.000000}%
\pgfsetstrokecolor{currentstroke}%
\pgfsetdash{}{0pt}%
\pgfsys@defobject{currentmarker}{\pgfqpoint{0.000000in}{-0.048611in}}{\pgfqpoint{0.000000in}{0.000000in}}{%
\pgfpathmoveto{\pgfqpoint{0.000000in}{0.000000in}}%
\pgfpathlineto{\pgfqpoint{0.000000in}{-0.048611in}}%
\pgfusepath{stroke,fill}%
}%
\begin{pgfscope}%
\pgfsys@transformshift{6.482124in}{0.386111in}%
\pgfsys@useobject{currentmarker}{}%
\end{pgfscope}%
\end{pgfscope}%
\begin{pgfscope}%
\definecolor{textcolor}{rgb}{0.000000,0.000000,0.000000}%
\pgfsetstrokecolor{textcolor}%
\pgfsetfillcolor{textcolor}%
\pgftext[x=6.482124in,y=0.288889in,,top]{\color{textcolor}\sffamily\fontsize{10.000000}{12.000000}\selectfont 5}%
\end{pgfscope}%
\begin{pgfscope}%
\pgfsetbuttcap%
\pgfsetroundjoin%
\definecolor{currentfill}{rgb}{0.000000,0.000000,0.000000}%
\pgfsetfillcolor{currentfill}%
\pgfsetlinewidth{0.803000pt}%
\definecolor{currentstroke}{rgb}{0.000000,0.000000,0.000000}%
\pgfsetstrokecolor{currentstroke}%
\pgfsetdash{}{0pt}%
\pgfsys@defobject{currentmarker}{\pgfqpoint{0.000000in}{-0.048611in}}{\pgfqpoint{0.000000in}{0.000000in}}{%
\pgfpathmoveto{\pgfqpoint{0.000000in}{0.000000in}}%
\pgfpathlineto{\pgfqpoint{0.000000in}{-0.048611in}}%
\pgfusepath{stroke,fill}%
}%
\begin{pgfscope}%
\pgfsys@transformshift{7.250538in}{0.386111in}%
\pgfsys@useobject{currentmarker}{}%
\end{pgfscope}%
\end{pgfscope}%
\begin{pgfscope}%
\definecolor{textcolor}{rgb}{0.000000,0.000000,0.000000}%
\pgfsetstrokecolor{textcolor}%
\pgfsetfillcolor{textcolor}%
\pgftext[x=7.250538in,y=0.288889in,,top]{\color{textcolor}\sffamily\fontsize{10.000000}{12.000000}\selectfont 6}%
\end{pgfscope}%
\begin{pgfscope}%
\pgfsetbuttcap%
\pgfsetroundjoin%
\definecolor{currentfill}{rgb}{0.000000,0.000000,0.000000}%
\pgfsetfillcolor{currentfill}%
\pgfsetlinewidth{0.803000pt}%
\definecolor{currentstroke}{rgb}{0.000000,0.000000,0.000000}%
\pgfsetstrokecolor{currentstroke}%
\pgfsetdash{}{0pt}%
\pgfsys@defobject{currentmarker}{\pgfqpoint{-0.048611in}{0.000000in}}{\pgfqpoint{0.000000in}{0.000000in}}{%
\pgfpathmoveto{\pgfqpoint{0.000000in}{0.000000in}}%
\pgfpathlineto{\pgfqpoint{-0.048611in}{0.000000in}}%
\pgfusepath{stroke,fill}%
}%
\begin{pgfscope}%
\pgfsys@transformshift{4.394444in}{0.420327in}%
\pgfsys@useobject{currentmarker}{}%
\end{pgfscope}%
\end{pgfscope}%
\begin{pgfscope}%
\definecolor{textcolor}{rgb}{0.000000,0.000000,0.000000}%
\pgfsetstrokecolor{textcolor}%
\pgfsetfillcolor{textcolor}%
\pgftext[x=4.076343in,y=0.367565in,left,base]{\color{textcolor}\sffamily\fontsize{10.000000}{12.000000}\selectfont 0.0}%
\end{pgfscope}%
\begin{pgfscope}%
\pgfsetbuttcap%
\pgfsetroundjoin%
\definecolor{currentfill}{rgb}{0.000000,0.000000,0.000000}%
\pgfsetfillcolor{currentfill}%
\pgfsetlinewidth{0.803000pt}%
\definecolor{currentstroke}{rgb}{0.000000,0.000000,0.000000}%
\pgfsetstrokecolor{currentstroke}%
\pgfsetdash{}{0pt}%
\pgfsys@defobject{currentmarker}{\pgfqpoint{-0.048611in}{0.000000in}}{\pgfqpoint{0.000000in}{0.000000in}}{%
\pgfpathmoveto{\pgfqpoint{0.000000in}{0.000000in}}%
\pgfpathlineto{\pgfqpoint{-0.048611in}{0.000000in}}%
\pgfusepath{stroke,fill}%
}%
\begin{pgfscope}%
\pgfsys@transformshift{4.394444in}{0.824611in}%
\pgfsys@useobject{currentmarker}{}%
\end{pgfscope}%
\end{pgfscope}%
\begin{pgfscope}%
\definecolor{textcolor}{rgb}{0.000000,0.000000,0.000000}%
\pgfsetstrokecolor{textcolor}%
\pgfsetfillcolor{textcolor}%
\pgftext[x=4.076343in,y=0.771849in,left,base]{\color{textcolor}\sffamily\fontsize{10.000000}{12.000000}\selectfont 0.5}%
\end{pgfscope}%
\begin{pgfscope}%
\pgfsetbuttcap%
\pgfsetroundjoin%
\definecolor{currentfill}{rgb}{0.000000,0.000000,0.000000}%
\pgfsetfillcolor{currentfill}%
\pgfsetlinewidth{0.803000pt}%
\definecolor{currentstroke}{rgb}{0.000000,0.000000,0.000000}%
\pgfsetstrokecolor{currentstroke}%
\pgfsetdash{}{0pt}%
\pgfsys@defobject{currentmarker}{\pgfqpoint{-0.048611in}{0.000000in}}{\pgfqpoint{0.000000in}{0.000000in}}{%
\pgfpathmoveto{\pgfqpoint{0.000000in}{0.000000in}}%
\pgfpathlineto{\pgfqpoint{-0.048611in}{0.000000in}}%
\pgfusepath{stroke,fill}%
}%
\begin{pgfscope}%
\pgfsys@transformshift{4.394444in}{1.228894in}%
\pgfsys@useobject{currentmarker}{}%
\end{pgfscope}%
\end{pgfscope}%
\begin{pgfscope}%
\definecolor{textcolor}{rgb}{0.000000,0.000000,0.000000}%
\pgfsetstrokecolor{textcolor}%
\pgfsetfillcolor{textcolor}%
\pgftext[x=4.076343in,y=1.176133in,left,base]{\color{textcolor}\sffamily\fontsize{10.000000}{12.000000}\selectfont 1.0}%
\end{pgfscope}%
\begin{pgfscope}%
\pgfsetbuttcap%
\pgfsetroundjoin%
\definecolor{currentfill}{rgb}{0.000000,0.000000,0.000000}%
\pgfsetfillcolor{currentfill}%
\pgfsetlinewidth{0.803000pt}%
\definecolor{currentstroke}{rgb}{0.000000,0.000000,0.000000}%
\pgfsetstrokecolor{currentstroke}%
\pgfsetdash{}{0pt}%
\pgfsys@defobject{currentmarker}{\pgfqpoint{-0.048611in}{0.000000in}}{\pgfqpoint{0.000000in}{0.000000in}}{%
\pgfpathmoveto{\pgfqpoint{0.000000in}{0.000000in}}%
\pgfpathlineto{\pgfqpoint{-0.048611in}{0.000000in}}%
\pgfusepath{stroke,fill}%
}%
\begin{pgfscope}%
\pgfsys@transformshift{4.394444in}{1.633178in}%
\pgfsys@useobject{currentmarker}{}%
\end{pgfscope}%
\end{pgfscope}%
\begin{pgfscope}%
\definecolor{textcolor}{rgb}{0.000000,0.000000,0.000000}%
\pgfsetstrokecolor{textcolor}%
\pgfsetfillcolor{textcolor}%
\pgftext[x=4.076343in,y=1.580416in,left,base]{\color{textcolor}\sffamily\fontsize{10.000000}{12.000000}\selectfont 1.5}%
\end{pgfscope}%
\begin{pgfscope}%
\pgfsetbuttcap%
\pgfsetroundjoin%
\definecolor{currentfill}{rgb}{0.000000,0.000000,0.000000}%
\pgfsetfillcolor{currentfill}%
\pgfsetlinewidth{0.803000pt}%
\definecolor{currentstroke}{rgb}{0.000000,0.000000,0.000000}%
\pgfsetstrokecolor{currentstroke}%
\pgfsetdash{}{0pt}%
\pgfsys@defobject{currentmarker}{\pgfqpoint{-0.048611in}{0.000000in}}{\pgfqpoint{0.000000in}{0.000000in}}{%
\pgfpathmoveto{\pgfqpoint{0.000000in}{0.000000in}}%
\pgfpathlineto{\pgfqpoint{-0.048611in}{0.000000in}}%
\pgfusepath{stroke,fill}%
}%
\begin{pgfscope}%
\pgfsys@transformshift{4.394444in}{2.037461in}%
\pgfsys@useobject{currentmarker}{}%
\end{pgfscope}%
\end{pgfscope}%
\begin{pgfscope}%
\definecolor{textcolor}{rgb}{0.000000,0.000000,0.000000}%
\pgfsetstrokecolor{textcolor}%
\pgfsetfillcolor{textcolor}%
\pgftext[x=4.076343in,y=1.984700in,left,base]{\color{textcolor}\sffamily\fontsize{10.000000}{12.000000}\selectfont 2.0}%
\end{pgfscope}%
\begin{pgfscope}%
\pgfsetbuttcap%
\pgfsetroundjoin%
\definecolor{currentfill}{rgb}{0.000000,0.000000,0.000000}%
\pgfsetfillcolor{currentfill}%
\pgfsetlinewidth{0.803000pt}%
\definecolor{currentstroke}{rgb}{0.000000,0.000000,0.000000}%
\pgfsetstrokecolor{currentstroke}%
\pgfsetdash{}{0pt}%
\pgfsys@defobject{currentmarker}{\pgfqpoint{-0.048611in}{0.000000in}}{\pgfqpoint{0.000000in}{0.000000in}}{%
\pgfpathmoveto{\pgfqpoint{0.000000in}{0.000000in}}%
\pgfpathlineto{\pgfqpoint{-0.048611in}{0.000000in}}%
\pgfusepath{stroke,fill}%
}%
\begin{pgfscope}%
\pgfsys@transformshift{4.394444in}{2.441745in}%
\pgfsys@useobject{currentmarker}{}%
\end{pgfscope}%
\end{pgfscope}%
\begin{pgfscope}%
\definecolor{textcolor}{rgb}{0.000000,0.000000,0.000000}%
\pgfsetstrokecolor{textcolor}%
\pgfsetfillcolor{textcolor}%
\pgftext[x=4.076343in,y=2.388983in,left,base]{\color{textcolor}\sffamily\fontsize{10.000000}{12.000000}\selectfont 2.5}%
\end{pgfscope}%
\begin{pgfscope}%
\pgfpathrectangle{\pgfqpoint{4.394444in}{0.386111in}}{\pgfqpoint{3.406944in}{2.316667in}}%
\pgfusepath{clip}%
\pgfsetrectcap%
\pgfsetroundjoin%
\pgfsetlinewidth{1.505625pt}%
\definecolor{currentstroke}{rgb}{0.121569,0.466667,0.705882}%
\pgfsetstrokecolor{currentstroke}%
\pgfsetdash{}{0pt}%
\pgfpathmoveto{\pgfqpoint{4.561088in}{0.623107in}}%
\pgfpathlineto{\pgfqpoint{4.637929in}{0.653987in}}%
\pgfpathlineto{\pgfqpoint{4.714771in}{0.686560in}}%
\pgfpathlineto{\pgfqpoint{4.791612in}{0.722221in}}%
\pgfpathlineto{\pgfqpoint{4.868454in}{0.759817in}}%
\pgfpathlineto{\pgfqpoint{4.945295in}{0.800142in}}%
\pgfpathlineto{\pgfqpoint{5.022137in}{0.842408in}}%
\pgfpathlineto{\pgfqpoint{5.098978in}{0.888077in}}%
\pgfpathlineto{\pgfqpoint{5.175819in}{0.938394in}}%
\pgfpathlineto{\pgfqpoint{5.252661in}{0.988884in}}%
\pgfpathlineto{\pgfqpoint{5.329502in}{1.043305in}}%
\pgfpathlineto{\pgfqpoint{5.406344in}{1.103553in}}%
\pgfpathlineto{\pgfqpoint{5.483185in}{1.168252in}}%
\pgfpathlineto{\pgfqpoint{5.560027in}{1.246720in}}%
\pgfpathlineto{\pgfqpoint{5.636868in}{1.330119in}}%
\pgfpathlineto{\pgfqpoint{5.713709in}{1.416957in}}%
\pgfpathlineto{\pgfqpoint{5.790551in}{1.519687in}}%
\pgfpathlineto{\pgfqpoint{5.867392in}{1.651908in}}%
\pgfpathlineto{\pgfqpoint{5.944234in}{1.826480in}}%
\pgfpathlineto{\pgfqpoint{6.021075in}{2.083008in}}%
\pgfpathlineto{\pgfqpoint{6.097917in}{2.379657in}}%
\pgfpathlineto{\pgfqpoint{6.174758in}{0.949864in}}%
\pgfpathlineto{\pgfqpoint{6.251600in}{0.796284in}}%
\pgfpathlineto{\pgfqpoint{6.328441in}{0.721902in}}%
\pgfpathlineto{\pgfqpoint{6.405282in}{0.678703in}}%
\pgfpathlineto{\pgfqpoint{6.482124in}{0.644287in}}%
\pgfpathlineto{\pgfqpoint{6.558965in}{0.622599in}}%
\pgfpathlineto{\pgfqpoint{6.635807in}{0.603119in}}%
\pgfpathlineto{\pgfqpoint{6.712648in}{0.586242in}}%
\pgfpathlineto{\pgfqpoint{6.789490in}{0.574559in}}%
\pgfpathlineto{\pgfqpoint{6.866331in}{0.562255in}}%
\pgfpathlineto{\pgfqpoint{6.943172in}{0.553256in}}%
\pgfpathlineto{\pgfqpoint{7.020014in}{0.545975in}}%
\pgfpathlineto{\pgfqpoint{7.096855in}{0.538190in}}%
\pgfpathlineto{\pgfqpoint{7.173697in}{0.530875in}}%
\pgfpathlineto{\pgfqpoint{7.250538in}{0.525412in}}%
\pgfpathlineto{\pgfqpoint{7.327380in}{0.519640in}}%
\pgfpathlineto{\pgfqpoint{7.404221in}{0.515299in}}%
\pgfpathlineto{\pgfqpoint{7.481063in}{0.510120in}}%
\pgfpathlineto{\pgfqpoint{7.557904in}{0.505726in}}%
\pgfpathlineto{\pgfqpoint{7.634745in}{0.502410in}}%
\pgfusepath{stroke}%
\end{pgfscope}%
\begin{pgfscope}%
\pgfsetrectcap%
\pgfsetmiterjoin%
\pgfsetlinewidth{0.803000pt}%
\definecolor{currentstroke}{rgb}{0.000000,0.000000,0.000000}%
\pgfsetstrokecolor{currentstroke}%
\pgfsetdash{}{0pt}%
\pgfpathmoveto{\pgfqpoint{4.394444in}{0.386111in}}%
\pgfpathlineto{\pgfqpoint{4.394444in}{2.702778in}}%
\pgfusepath{stroke}%
\end{pgfscope}%
\begin{pgfscope}%
\pgfsetrectcap%
\pgfsetmiterjoin%
\pgfsetlinewidth{0.803000pt}%
\definecolor{currentstroke}{rgb}{0.000000,0.000000,0.000000}%
\pgfsetstrokecolor{currentstroke}%
\pgfsetdash{}{0pt}%
\pgfpathmoveto{\pgfqpoint{7.801389in}{0.386111in}}%
\pgfpathlineto{\pgfqpoint{7.801389in}{2.702778in}}%
\pgfusepath{stroke}%
\end{pgfscope}%
\begin{pgfscope}%
\pgfsetrectcap%
\pgfsetmiterjoin%
\pgfsetlinewidth{0.803000pt}%
\definecolor{currentstroke}{rgb}{0.000000,0.000000,0.000000}%
\pgfsetstrokecolor{currentstroke}%
\pgfsetdash{}{0pt}%
\pgfpathmoveto{\pgfqpoint{4.394444in}{0.386111in}}%
\pgfpathlineto{\pgfqpoint{7.801389in}{0.386111in}}%
\pgfusepath{stroke}%
\end{pgfscope}%
\begin{pgfscope}%
\pgfsetrectcap%
\pgfsetmiterjoin%
\pgfsetlinewidth{0.803000pt}%
\definecolor{currentstroke}{rgb}{0.000000,0.000000,0.000000}%
\pgfsetstrokecolor{currentstroke}%
\pgfsetdash{}{0pt}%
\pgfpathmoveto{\pgfqpoint{4.394444in}{2.702778in}}%
\pgfpathlineto{\pgfqpoint{7.801389in}{2.702778in}}%
\pgfusepath{stroke}%
\end{pgfscope}%
\begin{pgfscope}%
\definecolor{textcolor}{rgb}{0.000000,0.000000,0.000000}%
\pgfsetstrokecolor{textcolor}%
\pgfsetfillcolor{textcolor}%
\pgftext[x=6.097917in,y=2.786111in,,base]{\color{textcolor}\sffamily\fontsize{12.000000}{14.400000}\selectfont \(\displaystyle  N = 24 \)}%
\end{pgfscope}%
\end{pgfpicture}%
\makeatother%
\endgroup%
}
\caption{Figures of the three-dimensional cubatures of $ D = 2 $}
\label{Fig:M3D2Ran}
\end{figure}

\begin{figure}[htbp]
\ContinuedFloat
\centering
\scalebox{0.75}{%% Creator: Matplotlib, PGF backend
%%
%% To include the figure in your LaTeX document, write
%%   \input{<filename>.pgf}
%%
%% Make sure the required packages are loaded in your preamble
%%   \usepackage{pgf}
%%
%% Figures using additional raster images can only be included by \input if
%% they are in the same directory as the main LaTeX file. For loading figures
%% from other directories you can use the `import` package
%%   \usepackage{import}
%% and then include the figures with
%%   \import{<path to file>}{<filename>.pgf}
%%
%% Matplotlib used the following preamble
%%   \usepackage{fontspec}
%%   \setmainfont{DejaVuSerif.ttf}[Path=/home/lzh/anaconda3/envs/numana/lib/python3.7/site-packages/matplotlib/mpl-data/fonts/ttf/]
%%   \setsansfont{DejaVuSans.ttf}[Path=/home/lzh/anaconda3/envs/numana/lib/python3.7/site-packages/matplotlib/mpl-data/fonts/ttf/]
%%   \setmonofont{DejaVuSansMono.ttf}[Path=/home/lzh/anaconda3/envs/numana/lib/python3.7/site-packages/matplotlib/mpl-data/fonts/ttf/]
%%
\begingroup%
\makeatletter%
\begin{pgfpicture}%
\pgfpathrectangle{\pgfpointorigin}{\pgfqpoint{12.000000in}{4.000000in}}%
\pgfusepath{use as bounding box, clip}%
\begin{pgfscope}%
\pgfsetbuttcap%
\pgfsetmiterjoin%
\definecolor{currentfill}{rgb}{1.000000,1.000000,1.000000}%
\pgfsetfillcolor{currentfill}%
\pgfsetlinewidth{0.000000pt}%
\definecolor{currentstroke}{rgb}{1.000000,1.000000,1.000000}%
\pgfsetstrokecolor{currentstroke}%
\pgfsetdash{}{0pt}%
\pgfpathmoveto{\pgfqpoint{0.000000in}{0.000000in}}%
\pgfpathlineto{\pgfqpoint{12.000000in}{0.000000in}}%
\pgfpathlineto{\pgfqpoint{12.000000in}{4.000000in}}%
\pgfpathlineto{\pgfqpoint{0.000000in}{4.000000in}}%
\pgfpathclose%
\pgfusepath{fill}%
\end{pgfscope}%
\begin{pgfscope}%
\pgfsys@transformshift{0.120000in}{0.095000in}%
\pgftext[left,bottom]{\pgfimage[interpolate=true,width=3.830000in,height=3.755000in]{Figure18-img0.png}}%
\end{pgfscope}%
\begin{pgfscope}%
\pgfsys@transformshift{4.100000in}{0.110000in}%
\pgftext[left,bottom]{\pgfimage[interpolate=true,width=3.930000in,height=3.740000in]{Figure18-img1.png}}%
\end{pgfscope}%
\begin{pgfscope}%
\pgfsys@transformshift{8.050000in}{0.150000in}%
\pgftext[left,bottom]{\pgfimage[interpolate=true,width=3.950000in,height=3.720000in]{Figure18-img2.png}}%
\end{pgfscope}%
\end{pgfpicture}%
\makeatother%
\endgroup%
}
\scalebox{0.75}{%% Creator: Matplotlib, PGF backend
%%
%% To include the figure in your LaTeX document, write
%%   \input{<filename>.pgf}
%%
%% Make sure the required packages are loaded in your preamble
%%   \usepackage{pgf}
%%
%% Figures using additional raster images can only be included by \input if
%% they are in the same directory as the main LaTeX file. For loading figures
%% from other directories you can use the `import` package
%%   \usepackage{import}
%% and then include the figures with
%%   \import{<path to file>}{<filename>.pgf}
%%
%% Matplotlib used the following preamble
%%   \usepackage{fontspec}
%%   \setmainfont{DejaVuSerif.ttf}[Path=/home/lzh/anaconda3/envs/numana/lib/python3.7/site-packages/matplotlib/mpl-data/fonts/ttf/]
%%   \setsansfont{DejaVuSans.ttf}[Path=/home/lzh/anaconda3/envs/numana/lib/python3.7/site-packages/matplotlib/mpl-data/fonts/ttf/]
%%   \setmonofont{DejaVuSansMono.ttf}[Path=/home/lzh/anaconda3/envs/numana/lib/python3.7/site-packages/matplotlib/mpl-data/fonts/ttf/]
%%
\begingroup%
\makeatletter%
\begin{pgfpicture}%
\pgfpathrectangle{\pgfpointorigin}{\pgfqpoint{8.000000in}{4.000000in}}%
\pgfusepath{use as bounding box, clip}%
\begin{pgfscope}%
\pgfsetbuttcap%
\pgfsetmiterjoin%
\definecolor{currentfill}{rgb}{1.000000,1.000000,1.000000}%
\pgfsetfillcolor{currentfill}%
\pgfsetlinewidth{0.000000pt}%
\definecolor{currentstroke}{rgb}{1.000000,1.000000,1.000000}%
\pgfsetstrokecolor{currentstroke}%
\pgfsetdash{}{0pt}%
\pgfpathmoveto{\pgfqpoint{0.000000in}{0.000000in}}%
\pgfpathlineto{\pgfqpoint{8.000000in}{0.000000in}}%
\pgfpathlineto{\pgfqpoint{8.000000in}{4.000000in}}%
\pgfpathlineto{\pgfqpoint{0.000000in}{4.000000in}}%
\pgfpathclose%
\pgfusepath{fill}%
\end{pgfscope}%
\begin{pgfscope}%
\pgfsetbuttcap%
\pgfsetmiterjoin%
\definecolor{currentfill}{rgb}{1.000000,1.000000,1.000000}%
\pgfsetfillcolor{currentfill}%
\pgfsetlinewidth{0.000000pt}%
\definecolor{currentstroke}{rgb}{0.000000,0.000000,0.000000}%
\pgfsetstrokecolor{currentstroke}%
\pgfsetstrokeopacity{0.000000}%
\pgfsetdash{}{0pt}%
\pgfpathmoveto{\pgfqpoint{1.000000in}{0.440000in}}%
\pgfpathlineto{\pgfqpoint{3.818182in}{0.440000in}}%
\pgfpathlineto{\pgfqpoint{3.818182in}{3.520000in}}%
\pgfpathlineto{\pgfqpoint{1.000000in}{3.520000in}}%
\pgfpathclose%
\pgfusepath{fill}%
\end{pgfscope}%
\begin{pgfscope}%
\pgfsetbuttcap%
\pgfsetmiterjoin%
\definecolor{currentfill}{rgb}{0.950000,0.950000,0.950000}%
\pgfsetfillcolor{currentfill}%
\pgfsetfillopacity{0.500000}%
\pgfsetlinewidth{1.003750pt}%
\definecolor{currentstroke}{rgb}{0.950000,0.950000,0.950000}%
\pgfsetstrokecolor{currentstroke}%
\pgfsetstrokeopacity{0.500000}%
\pgfsetdash{}{0pt}%
\pgfpathmoveto{\pgfqpoint{2.714177in}{1.884582in}}%
\pgfpathlineto{\pgfqpoint{3.520755in}{1.193628in}}%
\pgfpathlineto{\pgfqpoint{3.579156in}{2.584074in}}%
\pgfpathlineto{\pgfqpoint{2.727606in}{3.204883in}}%
\pgfusepath{stroke,fill}%
\end{pgfscope}%
\begin{pgfscope}%
\pgfsetbuttcap%
\pgfsetmiterjoin%
\definecolor{currentfill}{rgb}{0.900000,0.900000,0.900000}%
\pgfsetfillcolor{currentfill}%
\pgfsetfillopacity{0.500000}%
\pgfsetlinewidth{1.003750pt}%
\definecolor{currentstroke}{rgb}{0.900000,0.900000,0.900000}%
\pgfsetstrokecolor{currentstroke}%
\pgfsetstrokeopacity{0.500000}%
\pgfsetdash{}{0pt}%
\pgfpathmoveto{\pgfqpoint{2.714177in}{1.884582in}}%
\pgfpathlineto{\pgfqpoint{1.407596in}{1.498294in}}%
\pgfpathlineto{\pgfqpoint{1.352930in}{2.858408in}}%
\pgfpathlineto{\pgfqpoint{2.727606in}{3.204883in}}%
\pgfusepath{stroke,fill}%
\end{pgfscope}%
\begin{pgfscope}%
\pgfsetbuttcap%
\pgfsetmiterjoin%
\definecolor{currentfill}{rgb}{0.925000,0.925000,0.925000}%
\pgfsetfillcolor{currentfill}%
\pgfsetfillopacity{0.500000}%
\pgfsetlinewidth{1.003750pt}%
\definecolor{currentstroke}{rgb}{0.925000,0.925000,0.925000}%
\pgfsetstrokecolor{currentstroke}%
\pgfsetstrokeopacity{0.500000}%
\pgfsetdash{}{0pt}%
\pgfpathmoveto{\pgfqpoint{2.714177in}{1.884582in}}%
\pgfpathlineto{\pgfqpoint{1.407596in}{1.498294in}}%
\pgfpathlineto{\pgfqpoint{2.146055in}{0.743733in}}%
\pgfpathlineto{\pgfqpoint{3.520755in}{1.193628in}}%
\pgfusepath{stroke,fill}%
\end{pgfscope}%
\begin{pgfscope}%
\pgfsetrectcap%
\pgfsetroundjoin%
\pgfsetlinewidth{0.803000pt}%
\definecolor{currentstroke}{rgb}{0.000000,0.000000,0.000000}%
\pgfsetstrokecolor{currentstroke}%
\pgfsetdash{}{0pt}%
\pgfpathmoveto{\pgfqpoint{3.520755in}{1.193628in}}%
\pgfpathlineto{\pgfqpoint{2.146055in}{0.743733in}}%
\pgfusepath{stroke}%
\end{pgfscope}%
\begin{pgfscope}%
\pgfsetbuttcap%
\pgfsetroundjoin%
\pgfsetlinewidth{0.803000pt}%
\definecolor{currentstroke}{rgb}{0.690196,0.690196,0.690196}%
\pgfsetstrokecolor{currentstroke}%
\pgfsetdash{}{0pt}%
\pgfpathmoveto{\pgfqpoint{3.436945in}{1.166199in}}%
\pgfpathlineto{\pgfqpoint{2.634268in}{1.860957in}}%
\pgfpathlineto{\pgfqpoint{2.643704in}{3.183736in}}%
\pgfusepath{stroke}%
\end{pgfscope}%
\begin{pgfscope}%
\pgfsetbuttcap%
\pgfsetroundjoin%
\pgfsetlinewidth{0.803000pt}%
\definecolor{currentstroke}{rgb}{0.690196,0.690196,0.690196}%
\pgfsetstrokecolor{currentstroke}%
\pgfsetdash{}{0pt}%
\pgfpathmoveto{\pgfqpoint{3.204592in}{1.090158in}}%
\pgfpathlineto{\pgfqpoint{2.412903in}{1.795511in}}%
\pgfpathlineto{\pgfqpoint{2.411161in}{3.125126in}}%
\pgfusepath{stroke}%
\end{pgfscope}%
\begin{pgfscope}%
\pgfsetbuttcap%
\pgfsetroundjoin%
\pgfsetlinewidth{0.803000pt}%
\definecolor{currentstroke}{rgb}{0.690196,0.690196,0.690196}%
\pgfsetstrokecolor{currentstroke}%
\pgfsetdash{}{0pt}%
\pgfpathmoveto{\pgfqpoint{2.968546in}{1.012908in}}%
\pgfpathlineto{\pgfqpoint{2.188277in}{1.729101in}}%
\pgfpathlineto{\pgfqpoint{2.175018in}{3.065608in}}%
\pgfusepath{stroke}%
\end{pgfscope}%
\begin{pgfscope}%
\pgfsetbuttcap%
\pgfsetroundjoin%
\pgfsetlinewidth{0.803000pt}%
\definecolor{currentstroke}{rgb}{0.690196,0.690196,0.690196}%
\pgfsetstrokecolor{currentstroke}%
\pgfsetdash{}{0pt}%
\pgfpathmoveto{\pgfqpoint{2.728720in}{0.934420in}}%
\pgfpathlineto{\pgfqpoint{1.960319in}{1.661706in}}%
\pgfpathlineto{\pgfqpoint{1.935190in}{3.005161in}}%
\pgfusepath{stroke}%
\end{pgfscope}%
\begin{pgfscope}%
\pgfsetbuttcap%
\pgfsetroundjoin%
\pgfsetlinewidth{0.803000pt}%
\definecolor{currentstroke}{rgb}{0.690196,0.690196,0.690196}%
\pgfsetstrokecolor{currentstroke}%
\pgfsetdash{}{0pt}%
\pgfpathmoveto{\pgfqpoint{2.485022in}{0.854665in}}%
\pgfpathlineto{\pgfqpoint{1.728952in}{1.593303in}}%
\pgfpathlineto{\pgfqpoint{1.691590in}{2.943764in}}%
\pgfusepath{stroke}%
\end{pgfscope}%
\begin{pgfscope}%
\pgfsetbuttcap%
\pgfsetroundjoin%
\pgfsetlinewidth{0.803000pt}%
\definecolor{currentstroke}{rgb}{0.690196,0.690196,0.690196}%
\pgfsetstrokecolor{currentstroke}%
\pgfsetdash{}{0pt}%
\pgfpathmoveto{\pgfqpoint{2.237356in}{0.773612in}}%
\pgfpathlineto{\pgfqpoint{1.494101in}{1.523869in}}%
\pgfpathlineto{\pgfqpoint{1.444129in}{2.881394in}}%
\pgfusepath{stroke}%
\end{pgfscope}%
\begin{pgfscope}%
\pgfsetrectcap%
\pgfsetroundjoin%
\pgfsetlinewidth{0.803000pt}%
\definecolor{currentstroke}{rgb}{0.000000,0.000000,0.000000}%
\pgfsetstrokecolor{currentstroke}%
\pgfsetdash{}{0pt}%
\pgfpathmoveto{\pgfqpoint{3.430029in}{1.172185in}}%
\pgfpathlineto{\pgfqpoint{3.450801in}{1.154206in}}%
\pgfusepath{stroke}%
\end{pgfscope}%
\begin{pgfscope}%
\definecolor{textcolor}{rgb}{0.000000,0.000000,0.000000}%
\pgfsetstrokecolor{textcolor}%
\pgfsetfillcolor{textcolor}%
\pgftext[x=3.525746in,y=0.940342in,,top]{\color{textcolor}\sffamily\fontsize{10.000000}{12.000000}\selectfont 0.0}%
\end{pgfscope}%
\begin{pgfscope}%
\pgfsetrectcap%
\pgfsetroundjoin%
\pgfsetlinewidth{0.803000pt}%
\definecolor{currentstroke}{rgb}{0.000000,0.000000,0.000000}%
\pgfsetstrokecolor{currentstroke}%
\pgfsetdash{}{0pt}%
\pgfpathmoveto{\pgfqpoint{3.197767in}{1.096238in}}%
\pgfpathlineto{\pgfqpoint{3.218267in}{1.077974in}}%
\pgfusepath{stroke}%
\end{pgfscope}%
\begin{pgfscope}%
\definecolor{textcolor}{rgb}{0.000000,0.000000,0.000000}%
\pgfsetstrokecolor{textcolor}%
\pgfsetfillcolor{textcolor}%
\pgftext[x=3.293434in,y=0.862298in,,top]{\color{textcolor}\sffamily\fontsize{10.000000}{12.000000}\selectfont 0.2}%
\end{pgfscope}%
\begin{pgfscope}%
\pgfsetrectcap%
\pgfsetroundjoin%
\pgfsetlinewidth{0.803000pt}%
\definecolor{currentstroke}{rgb}{0.000000,0.000000,0.000000}%
\pgfsetstrokecolor{currentstroke}%
\pgfsetdash{}{0pt}%
\pgfpathmoveto{\pgfqpoint{2.961816in}{1.019085in}}%
\pgfpathlineto{\pgfqpoint{2.982032in}{1.000529in}}%
\pgfusepath{stroke}%
\end{pgfscope}%
\begin{pgfscope}%
\definecolor{textcolor}{rgb}{0.000000,0.000000,0.000000}%
\pgfsetstrokecolor{textcolor}%
\pgfsetfillcolor{textcolor}%
\pgftext[x=3.057419in,y=0.783011in,,top]{\color{textcolor}\sffamily\fontsize{10.000000}{12.000000}\selectfont 0.4}%
\end{pgfscope}%
\begin{pgfscope}%
\pgfsetrectcap%
\pgfsetroundjoin%
\pgfsetlinewidth{0.803000pt}%
\definecolor{currentstroke}{rgb}{0.000000,0.000000,0.000000}%
\pgfsetstrokecolor{currentstroke}%
\pgfsetdash{}{0pt}%
\pgfpathmoveto{\pgfqpoint{2.722089in}{0.940697in}}%
\pgfpathlineto{\pgfqpoint{2.742009in}{0.921843in}}%
\pgfusepath{stroke}%
\end{pgfscope}%
\begin{pgfscope}%
\definecolor{textcolor}{rgb}{0.000000,0.000000,0.000000}%
\pgfsetstrokecolor{textcolor}%
\pgfsetfillcolor{textcolor}%
\pgftext[x=2.817614in,y=0.702450in,,top]{\color{textcolor}\sffamily\fontsize{10.000000}{12.000000}\selectfont 0.6}%
\end{pgfscope}%
\begin{pgfscope}%
\pgfsetrectcap%
\pgfsetroundjoin%
\pgfsetlinewidth{0.803000pt}%
\definecolor{currentstroke}{rgb}{0.000000,0.000000,0.000000}%
\pgfsetstrokecolor{currentstroke}%
\pgfsetdash{}{0pt}%
\pgfpathmoveto{\pgfqpoint{2.478493in}{0.861044in}}%
\pgfpathlineto{\pgfqpoint{2.498105in}{0.841884in}}%
\pgfusepath{stroke}%
\end{pgfscope}%
\begin{pgfscope}%
\definecolor{textcolor}{rgb}{0.000000,0.000000,0.000000}%
\pgfsetstrokecolor{textcolor}%
\pgfsetfillcolor{textcolor}%
\pgftext[x=2.573925in,y=0.620584in,,top]{\color{textcolor}\sffamily\fontsize{10.000000}{12.000000}\selectfont 0.8}%
\end{pgfscope}%
\begin{pgfscope}%
\pgfsetrectcap%
\pgfsetroundjoin%
\pgfsetlinewidth{0.803000pt}%
\definecolor{currentstroke}{rgb}{0.000000,0.000000,0.000000}%
\pgfsetstrokecolor{currentstroke}%
\pgfsetdash{}{0pt}%
\pgfpathmoveto{\pgfqpoint{2.230934in}{0.780095in}}%
\pgfpathlineto{\pgfqpoint{2.250225in}{0.760622in}}%
\pgfusepath{stroke}%
\end{pgfscope}%
\begin{pgfscope}%
\definecolor{textcolor}{rgb}{0.000000,0.000000,0.000000}%
\pgfsetstrokecolor{textcolor}%
\pgfsetfillcolor{textcolor}%
\pgftext[x=2.326257in,y=0.537382in,,top]{\color{textcolor}\sffamily\fontsize{10.000000}{12.000000}\selectfont 1.0}%
\end{pgfscope}%
\begin{pgfscope}%
\pgfsetrectcap%
\pgfsetroundjoin%
\pgfsetlinewidth{0.803000pt}%
\definecolor{currentstroke}{rgb}{0.000000,0.000000,0.000000}%
\pgfsetstrokecolor{currentstroke}%
\pgfsetdash{}{0pt}%
\pgfpathmoveto{\pgfqpoint{1.407596in}{1.498294in}}%
\pgfpathlineto{\pgfqpoint{2.146055in}{0.743733in}}%
\pgfusepath{stroke}%
\end{pgfscope}%
\begin{pgfscope}%
\pgfsetbuttcap%
\pgfsetroundjoin%
\pgfsetlinewidth{0.803000pt}%
\definecolor{currentstroke}{rgb}{0.690196,0.690196,0.690196}%
\pgfsetstrokecolor{currentstroke}%
\pgfsetdash{}{0pt}%
\pgfpathmoveto{\pgfqpoint{2.777949in}{3.168181in}}%
\pgfpathlineto{\pgfqpoint{2.762037in}{1.843583in}}%
\pgfpathlineto{\pgfqpoint{1.451276in}{1.453662in}}%
\pgfusepath{stroke}%
\end{pgfscope}%
\begin{pgfscope}%
\pgfsetbuttcap%
\pgfsetroundjoin%
\pgfsetlinewidth{0.803000pt}%
\definecolor{currentstroke}{rgb}{0.690196,0.690196,0.690196}%
\pgfsetstrokecolor{currentstroke}%
\pgfsetdash{}{0pt}%
\pgfpathmoveto{\pgfqpoint{2.918559in}{3.065671in}}%
\pgfpathlineto{\pgfqpoint{2.895594in}{1.729172in}}%
\pgfpathlineto{\pgfqpoint{1.573260in}{1.329018in}}%
\pgfusepath{stroke}%
\end{pgfscope}%
\begin{pgfscope}%
\pgfsetbuttcap%
\pgfsetroundjoin%
\pgfsetlinewidth{0.803000pt}%
\definecolor{currentstroke}{rgb}{0.690196,0.690196,0.690196}%
\pgfsetstrokecolor{currentstroke}%
\pgfsetdash{}{0pt}%
\pgfpathmoveto{\pgfqpoint{3.062992in}{2.960375in}}%
\pgfpathlineto{\pgfqpoint{3.032602in}{1.611804in}}%
\pgfpathlineto{\pgfqpoint{1.698537in}{1.201009in}}%
\pgfusepath{stroke}%
\end{pgfscope}%
\begin{pgfscope}%
\pgfsetbuttcap%
\pgfsetroundjoin%
\pgfsetlinewidth{0.803000pt}%
\definecolor{currentstroke}{rgb}{0.690196,0.690196,0.690196}%
\pgfsetstrokecolor{currentstroke}%
\pgfsetdash{}{0pt}%
\pgfpathmoveto{\pgfqpoint{3.211405in}{2.852177in}}%
\pgfpathlineto{\pgfqpoint{3.173197in}{1.491363in}}%
\pgfpathlineto{\pgfqpoint{1.827243in}{1.069497in}}%
\pgfusepath{stroke}%
\end{pgfscope}%
\begin{pgfscope}%
\pgfsetbuttcap%
\pgfsetroundjoin%
\pgfsetlinewidth{0.803000pt}%
\definecolor{currentstroke}{rgb}{0.690196,0.690196,0.690196}%
\pgfsetstrokecolor{currentstroke}%
\pgfsetdash{}{0pt}%
\pgfpathmoveto{\pgfqpoint{3.363965in}{2.740956in}}%
\pgfpathlineto{\pgfqpoint{3.317522in}{1.367728in}}%
\pgfpathlineto{\pgfqpoint{1.959519in}{0.934337in}}%
\pgfusepath{stroke}%
\end{pgfscope}%
\begin{pgfscope}%
\pgfsetbuttcap%
\pgfsetroundjoin%
\pgfsetlinewidth{0.803000pt}%
\definecolor{currentstroke}{rgb}{0.690196,0.690196,0.690196}%
\pgfsetstrokecolor{currentstroke}%
\pgfsetdash{}{0pt}%
\pgfpathmoveto{\pgfqpoint{3.520848in}{2.626583in}}%
\pgfpathlineto{\pgfqpoint{3.465727in}{1.240768in}}%
\pgfpathlineto{\pgfqpoint{2.095516in}{0.795374in}}%
\pgfusepath{stroke}%
\end{pgfscope}%
\begin{pgfscope}%
\pgfsetrectcap%
\pgfsetroundjoin%
\pgfsetlinewidth{0.803000pt}%
\definecolor{currentstroke}{rgb}{0.000000,0.000000,0.000000}%
\pgfsetstrokecolor{currentstroke}%
\pgfsetdash{}{0pt}%
\pgfpathmoveto{\pgfqpoint{1.462214in}{1.456916in}}%
\pgfpathlineto{\pgfqpoint{1.429376in}{1.447148in}}%
\pgfusepath{stroke}%
\end{pgfscope}%
\begin{pgfscope}%
\definecolor{textcolor}{rgb}{0.000000,0.000000,0.000000}%
\pgfsetstrokecolor{textcolor}%
\pgfsetfillcolor{textcolor}%
\pgftext[x=1.307986in,y=1.268483in,,top]{\color{textcolor}\sffamily\fontsize{10.000000}{12.000000}\selectfont 0.0}%
\end{pgfscope}%
\begin{pgfscope}%
\pgfsetrectcap%
\pgfsetroundjoin%
\pgfsetlinewidth{0.803000pt}%
\definecolor{currentstroke}{rgb}{0.000000,0.000000,0.000000}%
\pgfsetstrokecolor{currentstroke}%
\pgfsetdash{}{0pt}%
\pgfpathmoveto{\pgfqpoint{1.584301in}{1.332359in}}%
\pgfpathlineto{\pgfqpoint{1.551155in}{1.322329in}}%
\pgfusepath{stroke}%
\end{pgfscope}%
\begin{pgfscope}%
\definecolor{textcolor}{rgb}{0.000000,0.000000,0.000000}%
\pgfsetstrokecolor{textcolor}%
\pgfsetfillcolor{textcolor}%
\pgftext[x=1.427974in,y=1.141467in,,top]{\color{textcolor}\sffamily\fontsize{10.000000}{12.000000}\selectfont 0.2}%
\end{pgfscope}%
\begin{pgfscope}%
\pgfsetrectcap%
\pgfsetroundjoin%
\pgfsetlinewidth{0.803000pt}%
\definecolor{currentstroke}{rgb}{0.000000,0.000000,0.000000}%
\pgfsetstrokecolor{currentstroke}%
\pgfsetdash{}{0pt}%
\pgfpathmoveto{\pgfqpoint{1.709682in}{1.204441in}}%
\pgfpathlineto{\pgfqpoint{1.676223in}{1.194138in}}%
\pgfusepath{stroke}%
\end{pgfscope}%
\begin{pgfscope}%
\definecolor{textcolor}{rgb}{0.000000,0.000000,0.000000}%
\pgfsetstrokecolor{textcolor}%
\pgfsetfillcolor{textcolor}%
\pgftext[x=1.551197in,y=1.011025in,,top]{\color{textcolor}\sffamily\fontsize{10.000000}{12.000000}\selectfont 0.4}%
\end{pgfscope}%
\begin{pgfscope}%
\pgfsetrectcap%
\pgfsetroundjoin%
\pgfsetlinewidth{0.803000pt}%
\definecolor{currentstroke}{rgb}{0.000000,0.000000,0.000000}%
\pgfsetstrokecolor{currentstroke}%
\pgfsetdash{}{0pt}%
\pgfpathmoveto{\pgfqpoint{1.838494in}{1.073023in}}%
\pgfpathlineto{\pgfqpoint{1.804716in}{1.062437in}}%
\pgfusepath{stroke}%
\end{pgfscope}%
\begin{pgfscope}%
\definecolor{textcolor}{rgb}{0.000000,0.000000,0.000000}%
\pgfsetstrokecolor{textcolor}%
\pgfsetfillcolor{textcolor}%
\pgftext[x=1.677790in,y=0.877016in,,top]{\color{textcolor}\sffamily\fontsize{10.000000}{12.000000}\selectfont 0.6}%
\end{pgfscope}%
\begin{pgfscope}%
\pgfsetrectcap%
\pgfsetroundjoin%
\pgfsetlinewidth{0.803000pt}%
\definecolor{currentstroke}{rgb}{0.000000,0.000000,0.000000}%
\pgfsetstrokecolor{currentstroke}%
\pgfsetdash{}{0pt}%
\pgfpathmoveto{\pgfqpoint{1.970877in}{0.937961in}}%
\pgfpathlineto{\pgfqpoint{1.936777in}{0.927079in}}%
\pgfusepath{stroke}%
\end{pgfscope}%
\begin{pgfscope}%
\definecolor{textcolor}{rgb}{0.000000,0.000000,0.000000}%
\pgfsetstrokecolor{textcolor}%
\pgfsetfillcolor{textcolor}%
\pgftext[x=1.807893in,y=0.739291in,,top]{\color{textcolor}\sffamily\fontsize{10.000000}{12.000000}\selectfont 0.8}%
\end{pgfscope}%
\begin{pgfscope}%
\pgfsetrectcap%
\pgfsetroundjoin%
\pgfsetlinewidth{0.803000pt}%
\definecolor{currentstroke}{rgb}{0.000000,0.000000,0.000000}%
\pgfsetstrokecolor{currentstroke}%
\pgfsetdash{}{0pt}%
\pgfpathmoveto{\pgfqpoint{2.106984in}{0.799101in}}%
\pgfpathlineto{\pgfqpoint{2.072556in}{0.787910in}}%
\pgfusepath{stroke}%
\end{pgfscope}%
\begin{pgfscope}%
\definecolor{textcolor}{rgb}{0.000000,0.000000,0.000000}%
\pgfsetstrokecolor{textcolor}%
\pgfsetfillcolor{textcolor}%
\pgftext[x=1.941653in,y=0.597696in,,top]{\color{textcolor}\sffamily\fontsize{10.000000}{12.000000}\selectfont 1.0}%
\end{pgfscope}%
\begin{pgfscope}%
\pgfsetrectcap%
\pgfsetroundjoin%
\pgfsetlinewidth{0.803000pt}%
\definecolor{currentstroke}{rgb}{0.000000,0.000000,0.000000}%
\pgfsetstrokecolor{currentstroke}%
\pgfsetdash{}{0pt}%
\pgfpathmoveto{\pgfqpoint{1.407596in}{1.498294in}}%
\pgfpathlineto{\pgfqpoint{1.352930in}{2.858408in}}%
\pgfusepath{stroke}%
\end{pgfscope}%
\begin{pgfscope}%
\pgfsetbuttcap%
\pgfsetroundjoin%
\pgfsetlinewidth{0.803000pt}%
\definecolor{currentstroke}{rgb}{0.690196,0.690196,0.690196}%
\pgfsetstrokecolor{currentstroke}%
\pgfsetdash{}{0pt}%
\pgfpathmoveto{\pgfqpoint{1.404280in}{1.580785in}}%
\pgfpathlineto{\pgfqpoint{2.714993in}{1.964823in}}%
\pgfpathlineto{\pgfqpoint{3.524292in}{1.277822in}}%
\pgfusepath{stroke}%
\end{pgfscope}%
\begin{pgfscope}%
\pgfsetbuttcap%
\pgfsetroundjoin%
\pgfsetlinewidth{0.803000pt}%
\definecolor{currentstroke}{rgb}{0.690196,0.690196,0.690196}%
\pgfsetstrokecolor{currentstroke}%
\pgfsetdash{}{0pt}%
\pgfpathmoveto{\pgfqpoint{1.395077in}{1.809771in}}%
\pgfpathlineto{\pgfqpoint{2.717257in}{2.187449in}}%
\pgfpathlineto{\pgfqpoint{3.534112in}{1.511629in}}%
\pgfusepath{stroke}%
\end{pgfscope}%
\begin{pgfscope}%
\pgfsetbuttcap%
\pgfsetroundjoin%
\pgfsetlinewidth{0.803000pt}%
\definecolor{currentstroke}{rgb}{0.690196,0.690196,0.690196}%
\pgfsetstrokecolor{currentstroke}%
\pgfsetdash{}{0pt}%
\pgfpathmoveto{\pgfqpoint{1.385710in}{2.042834in}}%
\pgfpathlineto{\pgfqpoint{2.719560in}{2.413873in}}%
\pgfpathlineto{\pgfqpoint{3.544113in}{1.749737in}}%
\pgfusepath{stroke}%
\end{pgfscope}%
\begin{pgfscope}%
\pgfsetbuttcap%
\pgfsetroundjoin%
\pgfsetlinewidth{0.803000pt}%
\definecolor{currentstroke}{rgb}{0.690196,0.690196,0.690196}%
\pgfsetstrokecolor{currentstroke}%
\pgfsetdash{}{0pt}%
\pgfpathmoveto{\pgfqpoint{1.376174in}{2.280084in}}%
\pgfpathlineto{\pgfqpoint{2.721903in}{2.644191in}}%
\pgfpathlineto{\pgfqpoint{3.554299in}{1.992268in}}%
\pgfusepath{stroke}%
\end{pgfscope}%
\begin{pgfscope}%
\pgfsetbuttcap%
\pgfsetroundjoin%
\pgfsetlinewidth{0.803000pt}%
\definecolor{currentstroke}{rgb}{0.690196,0.690196,0.690196}%
\pgfsetstrokecolor{currentstroke}%
\pgfsetdash{}{0pt}%
\pgfpathmoveto{\pgfqpoint{1.366466in}{2.521636in}}%
\pgfpathlineto{\pgfqpoint{2.724286in}{2.878505in}}%
\pgfpathlineto{\pgfqpoint{3.564677in}{2.239346in}}%
\pgfusepath{stroke}%
\end{pgfscope}%
\begin{pgfscope}%
\pgfsetbuttcap%
\pgfsetroundjoin%
\pgfsetlinewidth{0.803000pt}%
\definecolor{currentstroke}{rgb}{0.690196,0.690196,0.690196}%
\pgfsetstrokecolor{currentstroke}%
\pgfsetdash{}{0pt}%
\pgfpathmoveto{\pgfqpoint{1.356579in}{2.767607in}}%
\pgfpathlineto{\pgfqpoint{2.726711in}{3.116919in}}%
\pgfpathlineto{\pgfqpoint{3.575251in}{2.491100in}}%
\pgfusepath{stroke}%
\end{pgfscope}%
\begin{pgfscope}%
\pgfsetrectcap%
\pgfsetroundjoin%
\pgfsetlinewidth{0.803000pt}%
\definecolor{currentstroke}{rgb}{0.000000,0.000000,0.000000}%
\pgfsetstrokecolor{currentstroke}%
\pgfsetdash{}{0pt}%
\pgfpathmoveto{\pgfqpoint{1.415217in}{1.583990in}}%
\pgfpathlineto{\pgfqpoint{1.382384in}{1.574369in}}%
\pgfusepath{stroke}%
\end{pgfscope}%
\begin{pgfscope}%
\definecolor{textcolor}{rgb}{0.000000,0.000000,0.000000}%
\pgfsetstrokecolor{textcolor}%
\pgfsetfillcolor{textcolor}%
\pgftext[x=1.181686in,y=1.612595in,,top]{\color{textcolor}\sffamily\fontsize{10.000000}{12.000000}\selectfont 0.0}%
\end{pgfscope}%
\begin{pgfscope}%
\pgfsetrectcap%
\pgfsetroundjoin%
\pgfsetlinewidth{0.803000pt}%
\definecolor{currentstroke}{rgb}{0.000000,0.000000,0.000000}%
\pgfsetstrokecolor{currentstroke}%
\pgfsetdash{}{0pt}%
\pgfpathmoveto{\pgfqpoint{1.406114in}{1.812923in}}%
\pgfpathlineto{\pgfqpoint{1.372980in}{1.803459in}}%
\pgfusepath{stroke}%
\end{pgfscope}%
\begin{pgfscope}%
\definecolor{textcolor}{rgb}{0.000000,0.000000,0.000000}%
\pgfsetstrokecolor{textcolor}%
\pgfsetfillcolor{textcolor}%
\pgftext[x=1.170557in,y=1.841065in,,top]{\color{textcolor}\sffamily\fontsize{10.000000}{12.000000}\selectfont 0.2}%
\end{pgfscope}%
\begin{pgfscope}%
\pgfsetrectcap%
\pgfsetroundjoin%
\pgfsetlinewidth{0.803000pt}%
\definecolor{currentstroke}{rgb}{0.000000,0.000000,0.000000}%
\pgfsetstrokecolor{currentstroke}%
\pgfsetdash{}{0pt}%
\pgfpathmoveto{\pgfqpoint{1.396848in}{2.045932in}}%
\pgfpathlineto{\pgfqpoint{1.363409in}{2.036631in}}%
\pgfusepath{stroke}%
\end{pgfscope}%
\begin{pgfscope}%
\definecolor{textcolor}{rgb}{0.000000,0.000000,0.000000}%
\pgfsetstrokecolor{textcolor}%
\pgfsetfillcolor{textcolor}%
\pgftext[x=1.159231in,y=2.073588in,,top]{\color{textcolor}\sffamily\fontsize{10.000000}{12.000000}\selectfont 0.4}%
\end{pgfscope}%
\begin{pgfscope}%
\pgfsetrectcap%
\pgfsetroundjoin%
\pgfsetlinewidth{0.803000pt}%
\definecolor{currentstroke}{rgb}{0.000000,0.000000,0.000000}%
\pgfsetstrokecolor{currentstroke}%
\pgfsetdash{}{0pt}%
\pgfpathmoveto{\pgfqpoint{1.387416in}{2.283126in}}%
\pgfpathlineto{\pgfqpoint{1.353667in}{2.273995in}}%
\pgfusepath{stroke}%
\end{pgfscope}%
\begin{pgfscope}%
\definecolor{textcolor}{rgb}{0.000000,0.000000,0.000000}%
\pgfsetstrokecolor{textcolor}%
\pgfsetfillcolor{textcolor}%
\pgftext[x=1.147701in,y=2.310275in,,top]{\color{textcolor}\sffamily\fontsize{10.000000}{12.000000}\selectfont 0.6}%
\end{pgfscope}%
\begin{pgfscope}%
\pgfsetrectcap%
\pgfsetroundjoin%
\pgfsetlinewidth{0.803000pt}%
\definecolor{currentstroke}{rgb}{0.000000,0.000000,0.000000}%
\pgfsetstrokecolor{currentstroke}%
\pgfsetdash{}{0pt}%
\pgfpathmoveto{\pgfqpoint{1.377813in}{2.524619in}}%
\pgfpathlineto{\pgfqpoint{1.343747in}{2.515665in}}%
\pgfusepath{stroke}%
\end{pgfscope}%
\begin{pgfscope}%
\definecolor{textcolor}{rgb}{0.000000,0.000000,0.000000}%
\pgfsetstrokecolor{textcolor}%
\pgfsetfillcolor{textcolor}%
\pgftext[x=1.135964in,y=2.551237in,,top]{\color{textcolor}\sffamily\fontsize{10.000000}{12.000000}\selectfont 0.8}%
\end{pgfscope}%
\begin{pgfscope}%
\pgfsetrectcap%
\pgfsetroundjoin%
\pgfsetlinewidth{0.803000pt}%
\definecolor{currentstroke}{rgb}{0.000000,0.000000,0.000000}%
\pgfsetstrokecolor{currentstroke}%
\pgfsetdash{}{0pt}%
\pgfpathmoveto{\pgfqpoint{1.368034in}{2.770528in}}%
\pgfpathlineto{\pgfqpoint{1.333646in}{2.761761in}}%
\pgfusepath{stroke}%
\end{pgfscope}%
\begin{pgfscope}%
\definecolor{textcolor}{rgb}{0.000000,0.000000,0.000000}%
\pgfsetstrokecolor{textcolor}%
\pgfsetfillcolor{textcolor}%
\pgftext[x=1.124012in,y=2.796592in,,top]{\color{textcolor}\sffamily\fontsize{10.000000}{12.000000}\selectfont 1.0}%
\end{pgfscope}%
\begin{pgfscope}%
\pgfpathrectangle{\pgfqpoint{1.000000in}{0.570909in}}{\pgfqpoint{2.818182in}{2.818182in}}%
\pgfusepath{clip}%
\pgfsetbuttcap%
\pgfsetroundjoin%
\definecolor{currentfill}{rgb}{1.000000,0.498039,0.054902}%
\pgfsetfillcolor{currentfill}%
\pgfsetlinewidth{1.003750pt}%
\definecolor{currentstroke}{rgb}{1.000000,0.498039,0.054902}%
\pgfsetstrokecolor{currentstroke}%
\pgfsetdash{}{0pt}%
\pgfpathmoveto{\pgfqpoint{2.694082in}{2.099771in}}%
\pgfpathcurveto{\pgfqpoint{2.702318in}{2.099771in}}{\pgfqpoint{2.710218in}{2.103043in}}{\pgfqpoint{2.716042in}{2.108867in}}%
\pgfpathcurveto{\pgfqpoint{2.721866in}{2.114691in}}{\pgfqpoint{2.725138in}{2.122591in}}{\pgfqpoint{2.725138in}{2.130827in}}%
\pgfpathcurveto{\pgfqpoint{2.725138in}{2.139063in}}{\pgfqpoint{2.721866in}{2.146963in}}{\pgfqpoint{2.716042in}{2.152787in}}%
\pgfpathcurveto{\pgfqpoint{2.710218in}{2.158611in}}{\pgfqpoint{2.702318in}{2.161884in}}{\pgfqpoint{2.694082in}{2.161884in}}%
\pgfpathcurveto{\pgfqpoint{2.685846in}{2.161884in}}{\pgfqpoint{2.677946in}{2.158611in}}{\pgfqpoint{2.672122in}{2.152787in}}%
\pgfpathcurveto{\pgfqpoint{2.666298in}{2.146963in}}{\pgfqpoint{2.663025in}{2.139063in}}{\pgfqpoint{2.663025in}{2.130827in}}%
\pgfpathcurveto{\pgfqpoint{2.663025in}{2.122591in}}{\pgfqpoint{2.666298in}{2.114691in}}{\pgfqpoint{2.672122in}{2.108867in}}%
\pgfpathcurveto{\pgfqpoint{2.677946in}{2.103043in}}{\pgfqpoint{2.685846in}{2.099771in}}{\pgfqpoint{2.694082in}{2.099771in}}%
\pgfpathclose%
\pgfusepath{stroke,fill}%
\end{pgfscope}%
\begin{pgfscope}%
\definecolor{textcolor}{rgb}{0.000000,0.000000,0.000000}%
\pgfsetstrokecolor{textcolor}%
\pgfsetfillcolor{textcolor}%
\pgftext[x=2.147880in,y=1.692816in,left,base]{\color{textcolor}\sffamily\fontsize{10.000000}{12.000000}\selectfont 0.04167}%
\end{pgfscope}%
\begin{pgfscope}%
\definecolor{textcolor}{rgb}{0.000000,0.000000,0.000000}%
\pgfsetstrokecolor{textcolor}%
\pgfsetfillcolor{textcolor}%
\pgftext[x=2.694082in,y=2.130827in,left,base]{\color{textcolor}\sffamily\fontsize{10.000000}{12.000000}\selectfont 0.04167}%
\end{pgfscope}%
\begin{pgfscope}%
\definecolor{textcolor}{rgb}{0.000000,0.000000,0.000000}%
\pgfsetstrokecolor{textcolor}%
\pgfsetfillcolor{textcolor}%
\pgftext[x=2.921408in,y=1.646413in,left,base]{\color{textcolor}\sffamily\fontsize{10.000000}{12.000000}\selectfont 0.04167}%
\end{pgfscope}%
\begin{pgfscope}%
\definecolor{textcolor}{rgb}{0.000000,0.000000,0.000000}%
\pgfsetstrokecolor{textcolor}%
\pgfsetfillcolor{textcolor}%
\pgftext[x=2.509004in,y=2.350836in,left,base]{\color{textcolor}\sffamily\fontsize{10.000000}{12.000000}\selectfont 0.04167}%
\end{pgfscope}%
\begin{pgfscope}%
\pgfpathrectangle{\pgfqpoint{1.000000in}{0.570909in}}{\pgfqpoint{2.818182in}{2.818182in}}%
\pgfusepath{clip}%
\pgfsetbuttcap%
\pgfsetroundjoin%
\definecolor{currentfill}{rgb}{0.121569,0.466667,0.705882}%
\pgfsetfillcolor{currentfill}%
\pgfsetlinewidth{1.003750pt}%
\definecolor{currentstroke}{rgb}{0.121569,0.466667,0.705882}%
\pgfsetstrokecolor{currentstroke}%
\pgfsetdash{}{0pt}%
\pgfpathmoveto{\pgfqpoint{2.147880in}{1.661760in}}%
\pgfpathcurveto{\pgfqpoint{2.156116in}{1.661760in}}{\pgfqpoint{2.164016in}{1.665032in}}{\pgfqpoint{2.169840in}{1.670856in}}%
\pgfpathcurveto{\pgfqpoint{2.175664in}{1.676680in}}{\pgfqpoint{2.178936in}{1.684580in}}{\pgfqpoint{2.178936in}{1.692816in}}%
\pgfpathcurveto{\pgfqpoint{2.178936in}{1.701053in}}{\pgfqpoint{2.175664in}{1.708953in}}{\pgfqpoint{2.169840in}{1.714777in}}%
\pgfpathcurveto{\pgfqpoint{2.164016in}{1.720601in}}{\pgfqpoint{2.156116in}{1.723873in}}{\pgfqpoint{2.147880in}{1.723873in}}%
\pgfpathcurveto{\pgfqpoint{2.139643in}{1.723873in}}{\pgfqpoint{2.131743in}{1.720601in}}{\pgfqpoint{2.125919in}{1.714777in}}%
\pgfpathcurveto{\pgfqpoint{2.120095in}{1.708953in}}{\pgfqpoint{2.116823in}{1.701053in}}{\pgfqpoint{2.116823in}{1.692816in}}%
\pgfpathcurveto{\pgfqpoint{2.116823in}{1.684580in}}{\pgfqpoint{2.120095in}{1.676680in}}{\pgfqpoint{2.125919in}{1.670856in}}%
\pgfpathcurveto{\pgfqpoint{2.131743in}{1.665032in}}{\pgfqpoint{2.139643in}{1.661760in}}{\pgfqpoint{2.147880in}{1.661760in}}%
\pgfpathclose%
\pgfusepath{stroke,fill}%
\end{pgfscope}%
\begin{pgfscope}%
\pgfpathrectangle{\pgfqpoint{1.000000in}{0.570909in}}{\pgfqpoint{2.818182in}{2.818182in}}%
\pgfusepath{clip}%
\pgfsetbuttcap%
\pgfsetroundjoin%
\definecolor{currentfill}{rgb}{0.839216,0.152941,0.156863}%
\pgfsetfillcolor{currentfill}%
\pgfsetlinewidth{1.003750pt}%
\definecolor{currentstroke}{rgb}{0.839216,0.152941,0.156863}%
\pgfsetstrokecolor{currentstroke}%
\pgfsetdash{}{0pt}%
\pgfpathmoveto{\pgfqpoint{2.509004in}{2.319779in}}%
\pgfpathcurveto{\pgfqpoint{2.517240in}{2.319779in}}{\pgfqpoint{2.525140in}{2.323052in}}{\pgfqpoint{2.530964in}{2.328876in}}%
\pgfpathcurveto{\pgfqpoint{2.536788in}{2.334700in}}{\pgfqpoint{2.540060in}{2.342600in}}{\pgfqpoint{2.540060in}{2.350836in}}%
\pgfpathcurveto{\pgfqpoint{2.540060in}{2.359072in}}{\pgfqpoint{2.536788in}{2.366972in}}{\pgfqpoint{2.530964in}{2.372796in}}%
\pgfpathcurveto{\pgfqpoint{2.525140in}{2.378620in}}{\pgfqpoint{2.517240in}{2.381892in}}{\pgfqpoint{2.509004in}{2.381892in}}%
\pgfpathcurveto{\pgfqpoint{2.500768in}{2.381892in}}{\pgfqpoint{2.492868in}{2.378620in}}{\pgfqpoint{2.487044in}{2.372796in}}%
\pgfpathcurveto{\pgfqpoint{2.481220in}{2.366972in}}{\pgfqpoint{2.477947in}{2.359072in}}{\pgfqpoint{2.477947in}{2.350836in}}%
\pgfpathcurveto{\pgfqpoint{2.477947in}{2.342600in}}{\pgfqpoint{2.481220in}{2.334700in}}{\pgfqpoint{2.487044in}{2.328876in}}%
\pgfpathcurveto{\pgfqpoint{2.492868in}{2.323052in}}{\pgfqpoint{2.500768in}{2.319779in}}{\pgfqpoint{2.509004in}{2.319779in}}%
\pgfpathclose%
\pgfusepath{stroke,fill}%
\end{pgfscope}%
\begin{pgfscope}%
\pgfpathrectangle{\pgfqpoint{1.000000in}{0.570909in}}{\pgfqpoint{2.818182in}{2.818182in}}%
\pgfusepath{clip}%
\pgfsetbuttcap%
\pgfsetroundjoin%
\definecolor{currentfill}{rgb}{0.172549,0.627451,0.172549}%
\pgfsetfillcolor{currentfill}%
\pgfsetlinewidth{1.003750pt}%
\definecolor{currentstroke}{rgb}{0.172549,0.627451,0.172549}%
\pgfsetstrokecolor{currentstroke}%
\pgfsetdash{}{0pt}%
\pgfpathmoveto{\pgfqpoint{2.921408in}{1.615356in}}%
\pgfpathcurveto{\pgfqpoint{2.929644in}{1.615356in}}{\pgfqpoint{2.937544in}{1.618629in}}{\pgfqpoint{2.943368in}{1.624453in}}%
\pgfpathcurveto{\pgfqpoint{2.949192in}{1.630277in}}{\pgfqpoint{2.952465in}{1.638177in}}{\pgfqpoint{2.952465in}{1.646413in}}%
\pgfpathcurveto{\pgfqpoint{2.952465in}{1.654649in}}{\pgfqpoint{2.949192in}{1.662549in}}{\pgfqpoint{2.943368in}{1.668373in}}%
\pgfpathcurveto{\pgfqpoint{2.937544in}{1.674197in}}{\pgfqpoint{2.929644in}{1.677469in}}{\pgfqpoint{2.921408in}{1.677469in}}%
\pgfpathcurveto{\pgfqpoint{2.913172in}{1.677469in}}{\pgfqpoint{2.905272in}{1.674197in}}{\pgfqpoint{2.899448in}{1.668373in}}%
\pgfpathcurveto{\pgfqpoint{2.893624in}{1.662549in}}{\pgfqpoint{2.890352in}{1.654649in}}{\pgfqpoint{2.890352in}{1.646413in}}%
\pgfpathcurveto{\pgfqpoint{2.890352in}{1.638177in}}{\pgfqpoint{2.893624in}{1.630277in}}{\pgfqpoint{2.899448in}{1.624453in}}%
\pgfpathcurveto{\pgfqpoint{2.905272in}{1.618629in}}{\pgfqpoint{2.913172in}{1.615356in}}{\pgfqpoint{2.921408in}{1.615356in}}%
\pgfpathclose%
\pgfusepath{stroke,fill}%
\end{pgfscope}%
\begin{pgfscope}%
\pgfpathrectangle{\pgfqpoint{1.000000in}{0.570909in}}{\pgfqpoint{2.818182in}{2.818182in}}%
\pgfusepath{clip}%
\pgfsetbuttcap%
\pgfsetroundjoin%
\definecolor{currentfill}{rgb}{0.100000,0.100000,0.100000}%
\pgfsetfillcolor{currentfill}%
\pgfsetfillopacity{0.100000}%
\pgfsetlinewidth{0.301125pt}%
\definecolor{currentstroke}{rgb}{0.000000,0.000000,0.000000}%
\pgfsetstrokecolor{currentstroke}%
\pgfsetdash{}{0pt}%
\pgfpathmoveto{\pgfqpoint{2.682610in}{1.900366in}}%
\pgfpathlineto{\pgfqpoint{1.535168in}{1.562080in}}%
\pgfpathlineto{\pgfqpoint{2.692989in}{3.058384in}}%
\pgfpathclose%
\pgfusepath{stroke,fill}%
\end{pgfscope}%
\begin{pgfscope}%
\pgfpathrectangle{\pgfqpoint{1.000000in}{0.570909in}}{\pgfqpoint{2.818182in}{2.818182in}}%
\pgfusepath{clip}%
\pgfsetbuttcap%
\pgfsetroundjoin%
\definecolor{currentfill}{rgb}{0.100000,0.100000,0.100000}%
\pgfsetfillcolor{currentfill}%
\pgfsetfillopacity{0.100000}%
\pgfsetlinewidth{0.301125pt}%
\definecolor{currentstroke}{rgb}{0.000000,0.000000,0.000000}%
\pgfsetstrokecolor{currentstroke}%
\pgfsetdash{}{0pt}%
\pgfpathmoveto{\pgfqpoint{2.682610in}{1.900366in}}%
\pgfpathlineto{\pgfqpoint{3.385245in}{1.297693in}}%
\pgfpathlineto{\pgfqpoint{1.535168in}{1.562080in}}%
\pgfpathclose%
\pgfusepath{stroke,fill}%
\end{pgfscope}%
\begin{pgfscope}%
\pgfpathrectangle{\pgfqpoint{1.000000in}{0.570909in}}{\pgfqpoint{2.818182in}{2.818182in}}%
\pgfusepath{clip}%
\pgfsetbuttcap%
\pgfsetroundjoin%
\definecolor{currentfill}{rgb}{0.100000,0.100000,0.100000}%
\pgfsetfillcolor{currentfill}%
\pgfsetfillopacity{0.100000}%
\pgfsetlinewidth{0.301125pt}%
\definecolor{currentstroke}{rgb}{0.000000,0.000000,0.000000}%
\pgfsetstrokecolor{currentstroke}%
\pgfsetdash{}{0pt}%
\pgfpathmoveto{\pgfqpoint{2.682610in}{1.900366in}}%
\pgfpathlineto{\pgfqpoint{3.385245in}{1.297693in}}%
\pgfpathlineto{\pgfqpoint{2.692989in}{3.058384in}}%
\pgfpathclose%
\pgfusepath{stroke,fill}%
\end{pgfscope}%
\begin{pgfscope}%
\pgfpathrectangle{\pgfqpoint{1.000000in}{0.570909in}}{\pgfqpoint{2.818182in}{2.818182in}}%
\pgfusepath{clip}%
\pgfsetbuttcap%
\pgfsetroundjoin%
\definecolor{currentfill}{rgb}{0.100000,0.100000,0.100000}%
\pgfsetfillcolor{currentfill}%
\pgfsetfillopacity{0.100000}%
\pgfsetlinewidth{0.301125pt}%
\definecolor{currentstroke}{rgb}{0.000000,0.000000,0.000000}%
\pgfsetstrokecolor{currentstroke}%
\pgfsetdash{}{0pt}%
\pgfpathmoveto{\pgfqpoint{2.692989in}{3.058384in}}%
\pgfpathlineto{\pgfqpoint{3.385245in}{1.297693in}}%
\pgfpathlineto{\pgfqpoint{1.535168in}{1.562080in}}%
\pgfpathclose%
\pgfusepath{stroke,fill}%
\end{pgfscope}%
\begin{pgfscope}%
\pgfsetbuttcap%
\pgfsetmiterjoin%
\definecolor{currentfill}{rgb}{1.000000,1.000000,1.000000}%
\pgfsetfillcolor{currentfill}%
\pgfsetlinewidth{0.000000pt}%
\definecolor{currentstroke}{rgb}{0.000000,0.000000,0.000000}%
\pgfsetstrokecolor{currentstroke}%
\pgfsetstrokeopacity{0.000000}%
\pgfsetdash{}{0pt}%
\pgfpathmoveto{\pgfqpoint{4.381818in}{0.440000in}}%
\pgfpathlineto{\pgfqpoint{7.200000in}{0.440000in}}%
\pgfpathlineto{\pgfqpoint{7.200000in}{3.520000in}}%
\pgfpathlineto{\pgfqpoint{4.381818in}{3.520000in}}%
\pgfpathclose%
\pgfusepath{fill}%
\end{pgfscope}%
\begin{pgfscope}%
\pgfsetbuttcap%
\pgfsetmiterjoin%
\definecolor{currentfill}{rgb}{0.950000,0.950000,0.950000}%
\pgfsetfillcolor{currentfill}%
\pgfsetfillopacity{0.500000}%
\pgfsetlinewidth{1.003750pt}%
\definecolor{currentstroke}{rgb}{0.950000,0.950000,0.950000}%
\pgfsetstrokecolor{currentstroke}%
\pgfsetstrokeopacity{0.500000}%
\pgfsetdash{}{0pt}%
\pgfpathmoveto{\pgfqpoint{4.755412in}{1.193628in}}%
\pgfpathlineto{\pgfqpoint{5.561990in}{1.884582in}}%
\pgfpathlineto{\pgfqpoint{5.548561in}{3.204883in}}%
\pgfpathlineto{\pgfqpoint{4.697011in}{2.584074in}}%
\pgfusepath{stroke,fill}%
\end{pgfscope}%
\begin{pgfscope}%
\pgfsetbuttcap%
\pgfsetmiterjoin%
\definecolor{currentfill}{rgb}{0.900000,0.900000,0.900000}%
\pgfsetfillcolor{currentfill}%
\pgfsetfillopacity{0.500000}%
\pgfsetlinewidth{1.003750pt}%
\definecolor{currentstroke}{rgb}{0.900000,0.900000,0.900000}%
\pgfsetstrokecolor{currentstroke}%
\pgfsetstrokeopacity{0.500000}%
\pgfsetdash{}{0pt}%
\pgfpathmoveto{\pgfqpoint{5.561990in}{1.884582in}}%
\pgfpathlineto{\pgfqpoint{6.868571in}{1.498294in}}%
\pgfpathlineto{\pgfqpoint{6.923237in}{2.858408in}}%
\pgfpathlineto{\pgfqpoint{5.548561in}{3.204883in}}%
\pgfusepath{stroke,fill}%
\end{pgfscope}%
\begin{pgfscope}%
\pgfsetbuttcap%
\pgfsetmiterjoin%
\definecolor{currentfill}{rgb}{0.925000,0.925000,0.925000}%
\pgfsetfillcolor{currentfill}%
\pgfsetfillopacity{0.500000}%
\pgfsetlinewidth{1.003750pt}%
\definecolor{currentstroke}{rgb}{0.925000,0.925000,0.925000}%
\pgfsetstrokecolor{currentstroke}%
\pgfsetstrokeopacity{0.500000}%
\pgfsetdash{}{0pt}%
\pgfpathmoveto{\pgfqpoint{4.755412in}{1.193628in}}%
\pgfpathlineto{\pgfqpoint{6.130112in}{0.743733in}}%
\pgfpathlineto{\pgfqpoint{6.868571in}{1.498294in}}%
\pgfpathlineto{\pgfqpoint{5.561990in}{1.884582in}}%
\pgfusepath{stroke,fill}%
\end{pgfscope}%
\begin{pgfscope}%
\pgfsetrectcap%
\pgfsetroundjoin%
\pgfsetlinewidth{0.803000pt}%
\definecolor{currentstroke}{rgb}{0.000000,0.000000,0.000000}%
\pgfsetstrokecolor{currentstroke}%
\pgfsetdash{}{0pt}%
\pgfpathmoveto{\pgfqpoint{4.755412in}{1.193628in}}%
\pgfpathlineto{\pgfqpoint{6.130112in}{0.743733in}}%
\pgfusepath{stroke}%
\end{pgfscope}%
\begin{pgfscope}%
\pgfsetbuttcap%
\pgfsetroundjoin%
\pgfsetlinewidth{0.803000pt}%
\definecolor{currentstroke}{rgb}{0.690196,0.690196,0.690196}%
\pgfsetstrokecolor{currentstroke}%
\pgfsetdash{}{0pt}%
\pgfpathmoveto{\pgfqpoint{4.839222in}{1.166199in}}%
\pgfpathlineto{\pgfqpoint{5.641899in}{1.860957in}}%
\pgfpathlineto{\pgfqpoint{5.632463in}{3.183736in}}%
\pgfusepath{stroke}%
\end{pgfscope}%
\begin{pgfscope}%
\pgfsetbuttcap%
\pgfsetroundjoin%
\pgfsetlinewidth{0.803000pt}%
\definecolor{currentstroke}{rgb}{0.690196,0.690196,0.690196}%
\pgfsetstrokecolor{currentstroke}%
\pgfsetdash{}{0pt}%
\pgfpathmoveto{\pgfqpoint{5.071575in}{1.090158in}}%
\pgfpathlineto{\pgfqpoint{5.863264in}{1.795511in}}%
\pgfpathlineto{\pgfqpoint{5.865006in}{3.125126in}}%
\pgfusepath{stroke}%
\end{pgfscope}%
\begin{pgfscope}%
\pgfsetbuttcap%
\pgfsetroundjoin%
\pgfsetlinewidth{0.803000pt}%
\definecolor{currentstroke}{rgb}{0.690196,0.690196,0.690196}%
\pgfsetstrokecolor{currentstroke}%
\pgfsetdash{}{0pt}%
\pgfpathmoveto{\pgfqpoint{5.307621in}{1.012908in}}%
\pgfpathlineto{\pgfqpoint{6.087890in}{1.729101in}}%
\pgfpathlineto{\pgfqpoint{6.101149in}{3.065608in}}%
\pgfusepath{stroke}%
\end{pgfscope}%
\begin{pgfscope}%
\pgfsetbuttcap%
\pgfsetroundjoin%
\pgfsetlinewidth{0.803000pt}%
\definecolor{currentstroke}{rgb}{0.690196,0.690196,0.690196}%
\pgfsetstrokecolor{currentstroke}%
\pgfsetdash{}{0pt}%
\pgfpathmoveto{\pgfqpoint{5.547447in}{0.934420in}}%
\pgfpathlineto{\pgfqpoint{6.315848in}{1.661706in}}%
\pgfpathlineto{\pgfqpoint{6.340977in}{3.005161in}}%
\pgfusepath{stroke}%
\end{pgfscope}%
\begin{pgfscope}%
\pgfsetbuttcap%
\pgfsetroundjoin%
\pgfsetlinewidth{0.803000pt}%
\definecolor{currentstroke}{rgb}{0.690196,0.690196,0.690196}%
\pgfsetstrokecolor{currentstroke}%
\pgfsetdash{}{0pt}%
\pgfpathmoveto{\pgfqpoint{5.791145in}{0.854665in}}%
\pgfpathlineto{\pgfqpoint{6.547215in}{1.593303in}}%
\pgfpathlineto{\pgfqpoint{6.584577in}{2.943764in}}%
\pgfusepath{stroke}%
\end{pgfscope}%
\begin{pgfscope}%
\pgfsetbuttcap%
\pgfsetroundjoin%
\pgfsetlinewidth{0.803000pt}%
\definecolor{currentstroke}{rgb}{0.690196,0.690196,0.690196}%
\pgfsetstrokecolor{currentstroke}%
\pgfsetdash{}{0pt}%
\pgfpathmoveto{\pgfqpoint{6.038811in}{0.773612in}}%
\pgfpathlineto{\pgfqpoint{6.782066in}{1.523869in}}%
\pgfpathlineto{\pgfqpoint{6.832038in}{2.881394in}}%
\pgfusepath{stroke}%
\end{pgfscope}%
\begin{pgfscope}%
\pgfsetrectcap%
\pgfsetroundjoin%
\pgfsetlinewidth{0.803000pt}%
\definecolor{currentstroke}{rgb}{0.000000,0.000000,0.000000}%
\pgfsetstrokecolor{currentstroke}%
\pgfsetdash{}{0pt}%
\pgfpathmoveto{\pgfqpoint{4.846138in}{1.172185in}}%
\pgfpathlineto{\pgfqpoint{4.825366in}{1.154206in}}%
\pgfusepath{stroke}%
\end{pgfscope}%
\begin{pgfscope}%
\definecolor{textcolor}{rgb}{0.000000,0.000000,0.000000}%
\pgfsetstrokecolor{textcolor}%
\pgfsetfillcolor{textcolor}%
\pgftext[x=4.750421in,y=0.940342in,,top]{\color{textcolor}\sffamily\fontsize{10.000000}{12.000000}\selectfont 0.0}%
\end{pgfscope}%
\begin{pgfscope}%
\pgfsetrectcap%
\pgfsetroundjoin%
\pgfsetlinewidth{0.803000pt}%
\definecolor{currentstroke}{rgb}{0.000000,0.000000,0.000000}%
\pgfsetstrokecolor{currentstroke}%
\pgfsetdash{}{0pt}%
\pgfpathmoveto{\pgfqpoint{5.078400in}{1.096238in}}%
\pgfpathlineto{\pgfqpoint{5.057901in}{1.077974in}}%
\pgfusepath{stroke}%
\end{pgfscope}%
\begin{pgfscope}%
\definecolor{textcolor}{rgb}{0.000000,0.000000,0.000000}%
\pgfsetstrokecolor{textcolor}%
\pgfsetfillcolor{textcolor}%
\pgftext[x=4.982733in,y=0.862298in,,top]{\color{textcolor}\sffamily\fontsize{10.000000}{12.000000}\selectfont 0.2}%
\end{pgfscope}%
\begin{pgfscope}%
\pgfsetrectcap%
\pgfsetroundjoin%
\pgfsetlinewidth{0.803000pt}%
\definecolor{currentstroke}{rgb}{0.000000,0.000000,0.000000}%
\pgfsetstrokecolor{currentstroke}%
\pgfsetdash{}{0pt}%
\pgfpathmoveto{\pgfqpoint{5.314351in}{1.019085in}}%
\pgfpathlineto{\pgfqpoint{5.294135in}{1.000529in}}%
\pgfusepath{stroke}%
\end{pgfscope}%
\begin{pgfscope}%
\definecolor{textcolor}{rgb}{0.000000,0.000000,0.000000}%
\pgfsetstrokecolor{textcolor}%
\pgfsetfillcolor{textcolor}%
\pgftext[x=5.218748in,y=0.783011in,,top]{\color{textcolor}\sffamily\fontsize{10.000000}{12.000000}\selectfont 0.4}%
\end{pgfscope}%
\begin{pgfscope}%
\pgfsetrectcap%
\pgfsetroundjoin%
\pgfsetlinewidth{0.803000pt}%
\definecolor{currentstroke}{rgb}{0.000000,0.000000,0.000000}%
\pgfsetstrokecolor{currentstroke}%
\pgfsetdash{}{0pt}%
\pgfpathmoveto{\pgfqpoint{5.554078in}{0.940697in}}%
\pgfpathlineto{\pgfqpoint{5.534158in}{0.921843in}}%
\pgfusepath{stroke}%
\end{pgfscope}%
\begin{pgfscope}%
\definecolor{textcolor}{rgb}{0.000000,0.000000,0.000000}%
\pgfsetstrokecolor{textcolor}%
\pgfsetfillcolor{textcolor}%
\pgftext[x=5.458553in,y=0.702450in,,top]{\color{textcolor}\sffamily\fontsize{10.000000}{12.000000}\selectfont 0.6}%
\end{pgfscope}%
\begin{pgfscope}%
\pgfsetrectcap%
\pgfsetroundjoin%
\pgfsetlinewidth{0.803000pt}%
\definecolor{currentstroke}{rgb}{0.000000,0.000000,0.000000}%
\pgfsetstrokecolor{currentstroke}%
\pgfsetdash{}{0pt}%
\pgfpathmoveto{\pgfqpoint{5.797674in}{0.861044in}}%
\pgfpathlineto{\pgfqpoint{5.778062in}{0.841884in}}%
\pgfusepath{stroke}%
\end{pgfscope}%
\begin{pgfscope}%
\definecolor{textcolor}{rgb}{0.000000,0.000000,0.000000}%
\pgfsetstrokecolor{textcolor}%
\pgfsetfillcolor{textcolor}%
\pgftext[x=5.702242in,y=0.620584in,,top]{\color{textcolor}\sffamily\fontsize{10.000000}{12.000000}\selectfont 0.8}%
\end{pgfscope}%
\begin{pgfscope}%
\pgfsetrectcap%
\pgfsetroundjoin%
\pgfsetlinewidth{0.803000pt}%
\definecolor{currentstroke}{rgb}{0.000000,0.000000,0.000000}%
\pgfsetstrokecolor{currentstroke}%
\pgfsetdash{}{0pt}%
\pgfpathmoveto{\pgfqpoint{6.045233in}{0.780095in}}%
\pgfpathlineto{\pgfqpoint{6.025942in}{0.760622in}}%
\pgfusepath{stroke}%
\end{pgfscope}%
\begin{pgfscope}%
\definecolor{textcolor}{rgb}{0.000000,0.000000,0.000000}%
\pgfsetstrokecolor{textcolor}%
\pgfsetfillcolor{textcolor}%
\pgftext[x=5.949910in,y=0.537382in,,top]{\color{textcolor}\sffamily\fontsize{10.000000}{12.000000}\selectfont 1.0}%
\end{pgfscope}%
\begin{pgfscope}%
\pgfsetrectcap%
\pgfsetroundjoin%
\pgfsetlinewidth{0.803000pt}%
\definecolor{currentstroke}{rgb}{0.000000,0.000000,0.000000}%
\pgfsetstrokecolor{currentstroke}%
\pgfsetdash{}{0pt}%
\pgfpathmoveto{\pgfqpoint{6.868571in}{1.498294in}}%
\pgfpathlineto{\pgfqpoint{6.130112in}{0.743733in}}%
\pgfusepath{stroke}%
\end{pgfscope}%
\begin{pgfscope}%
\pgfsetbuttcap%
\pgfsetroundjoin%
\pgfsetlinewidth{0.803000pt}%
\definecolor{currentstroke}{rgb}{0.690196,0.690196,0.690196}%
\pgfsetstrokecolor{currentstroke}%
\pgfsetdash{}{0pt}%
\pgfpathmoveto{\pgfqpoint{4.755320in}{2.626583in}}%
\pgfpathlineto{\pgfqpoint{4.810441in}{1.240768in}}%
\pgfpathlineto{\pgfqpoint{6.180651in}{0.795374in}}%
\pgfusepath{stroke}%
\end{pgfscope}%
\begin{pgfscope}%
\pgfsetbuttcap%
\pgfsetroundjoin%
\pgfsetlinewidth{0.803000pt}%
\definecolor{currentstroke}{rgb}{0.690196,0.690196,0.690196}%
\pgfsetstrokecolor{currentstroke}%
\pgfsetdash{}{0pt}%
\pgfpathmoveto{\pgfqpoint{4.912202in}{2.740956in}}%
\pgfpathlineto{\pgfqpoint{4.958645in}{1.367728in}}%
\pgfpathlineto{\pgfqpoint{6.316648in}{0.934337in}}%
\pgfusepath{stroke}%
\end{pgfscope}%
\begin{pgfscope}%
\pgfsetbuttcap%
\pgfsetroundjoin%
\pgfsetlinewidth{0.803000pt}%
\definecolor{currentstroke}{rgb}{0.690196,0.690196,0.690196}%
\pgfsetstrokecolor{currentstroke}%
\pgfsetdash{}{0pt}%
\pgfpathmoveto{\pgfqpoint{5.064762in}{2.852177in}}%
\pgfpathlineto{\pgfqpoint{5.102970in}{1.491363in}}%
\pgfpathlineto{\pgfqpoint{6.448924in}{1.069497in}}%
\pgfusepath{stroke}%
\end{pgfscope}%
\begin{pgfscope}%
\pgfsetbuttcap%
\pgfsetroundjoin%
\pgfsetlinewidth{0.803000pt}%
\definecolor{currentstroke}{rgb}{0.690196,0.690196,0.690196}%
\pgfsetstrokecolor{currentstroke}%
\pgfsetdash{}{0pt}%
\pgfpathmoveto{\pgfqpoint{5.213175in}{2.960375in}}%
\pgfpathlineto{\pgfqpoint{5.243565in}{1.611804in}}%
\pgfpathlineto{\pgfqpoint{6.577630in}{1.201009in}}%
\pgfusepath{stroke}%
\end{pgfscope}%
\begin{pgfscope}%
\pgfsetbuttcap%
\pgfsetroundjoin%
\pgfsetlinewidth{0.803000pt}%
\definecolor{currentstroke}{rgb}{0.690196,0.690196,0.690196}%
\pgfsetstrokecolor{currentstroke}%
\pgfsetdash{}{0pt}%
\pgfpathmoveto{\pgfqpoint{5.357608in}{3.065671in}}%
\pgfpathlineto{\pgfqpoint{5.380573in}{1.729172in}}%
\pgfpathlineto{\pgfqpoint{6.702907in}{1.329018in}}%
\pgfusepath{stroke}%
\end{pgfscope}%
\begin{pgfscope}%
\pgfsetbuttcap%
\pgfsetroundjoin%
\pgfsetlinewidth{0.803000pt}%
\definecolor{currentstroke}{rgb}{0.690196,0.690196,0.690196}%
\pgfsetstrokecolor{currentstroke}%
\pgfsetdash{}{0pt}%
\pgfpathmoveto{\pgfqpoint{5.498218in}{3.168181in}}%
\pgfpathlineto{\pgfqpoint{5.514130in}{1.843583in}}%
\pgfpathlineto{\pgfqpoint{6.824892in}{1.453662in}}%
\pgfusepath{stroke}%
\end{pgfscope}%
\begin{pgfscope}%
\pgfsetrectcap%
\pgfsetroundjoin%
\pgfsetlinewidth{0.803000pt}%
\definecolor{currentstroke}{rgb}{0.000000,0.000000,0.000000}%
\pgfsetstrokecolor{currentstroke}%
\pgfsetdash{}{0pt}%
\pgfpathmoveto{\pgfqpoint{6.169183in}{0.799101in}}%
\pgfpathlineto{\pgfqpoint{6.203611in}{0.787910in}}%
\pgfusepath{stroke}%
\end{pgfscope}%
\begin{pgfscope}%
\definecolor{textcolor}{rgb}{0.000000,0.000000,0.000000}%
\pgfsetstrokecolor{textcolor}%
\pgfsetfillcolor{textcolor}%
\pgftext[x=6.334514in,y=0.597696in,,top]{\color{textcolor}\sffamily\fontsize{10.000000}{12.000000}\selectfont 0.0}%
\end{pgfscope}%
\begin{pgfscope}%
\pgfsetrectcap%
\pgfsetroundjoin%
\pgfsetlinewidth{0.803000pt}%
\definecolor{currentstroke}{rgb}{0.000000,0.000000,0.000000}%
\pgfsetstrokecolor{currentstroke}%
\pgfsetdash{}{0pt}%
\pgfpathmoveto{\pgfqpoint{6.305290in}{0.937961in}}%
\pgfpathlineto{\pgfqpoint{6.339390in}{0.927079in}}%
\pgfusepath{stroke}%
\end{pgfscope}%
\begin{pgfscope}%
\definecolor{textcolor}{rgb}{0.000000,0.000000,0.000000}%
\pgfsetstrokecolor{textcolor}%
\pgfsetfillcolor{textcolor}%
\pgftext[x=6.468274in,y=0.739291in,,top]{\color{textcolor}\sffamily\fontsize{10.000000}{12.000000}\selectfont 0.2}%
\end{pgfscope}%
\begin{pgfscope}%
\pgfsetrectcap%
\pgfsetroundjoin%
\pgfsetlinewidth{0.803000pt}%
\definecolor{currentstroke}{rgb}{0.000000,0.000000,0.000000}%
\pgfsetstrokecolor{currentstroke}%
\pgfsetdash{}{0pt}%
\pgfpathmoveto{\pgfqpoint{6.437674in}{1.073023in}}%
\pgfpathlineto{\pgfqpoint{6.471451in}{1.062437in}}%
\pgfusepath{stroke}%
\end{pgfscope}%
\begin{pgfscope}%
\definecolor{textcolor}{rgb}{0.000000,0.000000,0.000000}%
\pgfsetstrokecolor{textcolor}%
\pgfsetfillcolor{textcolor}%
\pgftext[x=6.598377in,y=0.877016in,,top]{\color{textcolor}\sffamily\fontsize{10.000000}{12.000000}\selectfont 0.4}%
\end{pgfscope}%
\begin{pgfscope}%
\pgfsetrectcap%
\pgfsetroundjoin%
\pgfsetlinewidth{0.803000pt}%
\definecolor{currentstroke}{rgb}{0.000000,0.000000,0.000000}%
\pgfsetstrokecolor{currentstroke}%
\pgfsetdash{}{0pt}%
\pgfpathmoveto{\pgfqpoint{6.566485in}{1.204441in}}%
\pgfpathlineto{\pgfqpoint{6.599944in}{1.194138in}}%
\pgfusepath{stroke}%
\end{pgfscope}%
\begin{pgfscope}%
\definecolor{textcolor}{rgb}{0.000000,0.000000,0.000000}%
\pgfsetstrokecolor{textcolor}%
\pgfsetfillcolor{textcolor}%
\pgftext[x=6.724970in,y=1.011025in,,top]{\color{textcolor}\sffamily\fontsize{10.000000}{12.000000}\selectfont 0.6}%
\end{pgfscope}%
\begin{pgfscope}%
\pgfsetrectcap%
\pgfsetroundjoin%
\pgfsetlinewidth{0.803000pt}%
\definecolor{currentstroke}{rgb}{0.000000,0.000000,0.000000}%
\pgfsetstrokecolor{currentstroke}%
\pgfsetdash{}{0pt}%
\pgfpathmoveto{\pgfqpoint{6.691866in}{1.332359in}}%
\pgfpathlineto{\pgfqpoint{6.725012in}{1.322329in}}%
\pgfusepath{stroke}%
\end{pgfscope}%
\begin{pgfscope}%
\definecolor{textcolor}{rgb}{0.000000,0.000000,0.000000}%
\pgfsetstrokecolor{textcolor}%
\pgfsetfillcolor{textcolor}%
\pgftext[x=6.848194in,y=1.141467in,,top]{\color{textcolor}\sffamily\fontsize{10.000000}{12.000000}\selectfont 0.8}%
\end{pgfscope}%
\begin{pgfscope}%
\pgfsetrectcap%
\pgfsetroundjoin%
\pgfsetlinewidth{0.803000pt}%
\definecolor{currentstroke}{rgb}{0.000000,0.000000,0.000000}%
\pgfsetstrokecolor{currentstroke}%
\pgfsetdash{}{0pt}%
\pgfpathmoveto{\pgfqpoint{6.813953in}{1.456916in}}%
\pgfpathlineto{\pgfqpoint{6.846791in}{1.447148in}}%
\pgfusepath{stroke}%
\end{pgfscope}%
\begin{pgfscope}%
\definecolor{textcolor}{rgb}{0.000000,0.000000,0.000000}%
\pgfsetstrokecolor{textcolor}%
\pgfsetfillcolor{textcolor}%
\pgftext[x=6.968181in,y=1.268483in,,top]{\color{textcolor}\sffamily\fontsize{10.000000}{12.000000}\selectfont 1.0}%
\end{pgfscope}%
\begin{pgfscope}%
\pgfsetrectcap%
\pgfsetroundjoin%
\pgfsetlinewidth{0.803000pt}%
\definecolor{currentstroke}{rgb}{0.000000,0.000000,0.000000}%
\pgfsetstrokecolor{currentstroke}%
\pgfsetdash{}{0pt}%
\pgfpathmoveto{\pgfqpoint{6.868571in}{1.498294in}}%
\pgfpathlineto{\pgfqpoint{6.923237in}{2.858408in}}%
\pgfusepath{stroke}%
\end{pgfscope}%
\begin{pgfscope}%
\pgfsetbuttcap%
\pgfsetroundjoin%
\pgfsetlinewidth{0.803000pt}%
\definecolor{currentstroke}{rgb}{0.690196,0.690196,0.690196}%
\pgfsetstrokecolor{currentstroke}%
\pgfsetdash{}{0pt}%
\pgfpathmoveto{\pgfqpoint{6.871887in}{1.580785in}}%
\pgfpathlineto{\pgfqpoint{5.561174in}{1.964823in}}%
\pgfpathlineto{\pgfqpoint{4.751875in}{1.277822in}}%
\pgfusepath{stroke}%
\end{pgfscope}%
\begin{pgfscope}%
\pgfsetbuttcap%
\pgfsetroundjoin%
\pgfsetlinewidth{0.803000pt}%
\definecolor{currentstroke}{rgb}{0.690196,0.690196,0.690196}%
\pgfsetstrokecolor{currentstroke}%
\pgfsetdash{}{0pt}%
\pgfpathmoveto{\pgfqpoint{6.881090in}{1.809771in}}%
\pgfpathlineto{\pgfqpoint{5.558910in}{2.187449in}}%
\pgfpathlineto{\pgfqpoint{4.742055in}{1.511629in}}%
\pgfusepath{stroke}%
\end{pgfscope}%
\begin{pgfscope}%
\pgfsetbuttcap%
\pgfsetroundjoin%
\pgfsetlinewidth{0.803000pt}%
\definecolor{currentstroke}{rgb}{0.690196,0.690196,0.690196}%
\pgfsetstrokecolor{currentstroke}%
\pgfsetdash{}{0pt}%
\pgfpathmoveto{\pgfqpoint{6.890457in}{2.042834in}}%
\pgfpathlineto{\pgfqpoint{5.556607in}{2.413873in}}%
\pgfpathlineto{\pgfqpoint{4.732054in}{1.749737in}}%
\pgfusepath{stroke}%
\end{pgfscope}%
\begin{pgfscope}%
\pgfsetbuttcap%
\pgfsetroundjoin%
\pgfsetlinewidth{0.803000pt}%
\definecolor{currentstroke}{rgb}{0.690196,0.690196,0.690196}%
\pgfsetstrokecolor{currentstroke}%
\pgfsetdash{}{0pt}%
\pgfpathmoveto{\pgfqpoint{6.899993in}{2.280084in}}%
\pgfpathlineto{\pgfqpoint{5.554264in}{2.644191in}}%
\pgfpathlineto{\pgfqpoint{4.721868in}{1.992268in}}%
\pgfusepath{stroke}%
\end{pgfscope}%
\begin{pgfscope}%
\pgfsetbuttcap%
\pgfsetroundjoin%
\pgfsetlinewidth{0.803000pt}%
\definecolor{currentstroke}{rgb}{0.690196,0.690196,0.690196}%
\pgfsetstrokecolor{currentstroke}%
\pgfsetdash{}{0pt}%
\pgfpathmoveto{\pgfqpoint{6.909702in}{2.521636in}}%
\pgfpathlineto{\pgfqpoint{5.551881in}{2.878505in}}%
\pgfpathlineto{\pgfqpoint{4.711490in}{2.239346in}}%
\pgfusepath{stroke}%
\end{pgfscope}%
\begin{pgfscope}%
\pgfsetbuttcap%
\pgfsetroundjoin%
\pgfsetlinewidth{0.803000pt}%
\definecolor{currentstroke}{rgb}{0.690196,0.690196,0.690196}%
\pgfsetstrokecolor{currentstroke}%
\pgfsetdash{}{0pt}%
\pgfpathmoveto{\pgfqpoint{6.919588in}{2.767607in}}%
\pgfpathlineto{\pgfqpoint{5.549456in}{3.116919in}}%
\pgfpathlineto{\pgfqpoint{4.700916in}{2.491100in}}%
\pgfusepath{stroke}%
\end{pgfscope}%
\begin{pgfscope}%
\pgfsetrectcap%
\pgfsetroundjoin%
\pgfsetlinewidth{0.803000pt}%
\definecolor{currentstroke}{rgb}{0.000000,0.000000,0.000000}%
\pgfsetstrokecolor{currentstroke}%
\pgfsetdash{}{0pt}%
\pgfpathmoveto{\pgfqpoint{6.860950in}{1.583990in}}%
\pgfpathlineto{\pgfqpoint{6.893784in}{1.574369in}}%
\pgfusepath{stroke}%
\end{pgfscope}%
\begin{pgfscope}%
\definecolor{textcolor}{rgb}{0.000000,0.000000,0.000000}%
\pgfsetstrokecolor{textcolor}%
\pgfsetfillcolor{textcolor}%
\pgftext[x=7.094481in,y=1.612595in,,top]{\color{textcolor}\sffamily\fontsize{10.000000}{12.000000}\selectfont 0.0}%
\end{pgfscope}%
\begin{pgfscope}%
\pgfsetrectcap%
\pgfsetroundjoin%
\pgfsetlinewidth{0.803000pt}%
\definecolor{currentstroke}{rgb}{0.000000,0.000000,0.000000}%
\pgfsetstrokecolor{currentstroke}%
\pgfsetdash{}{0pt}%
\pgfpathmoveto{\pgfqpoint{6.870053in}{1.812923in}}%
\pgfpathlineto{\pgfqpoint{6.903187in}{1.803459in}}%
\pgfusepath{stroke}%
\end{pgfscope}%
\begin{pgfscope}%
\definecolor{textcolor}{rgb}{0.000000,0.000000,0.000000}%
\pgfsetstrokecolor{textcolor}%
\pgfsetfillcolor{textcolor}%
\pgftext[x=7.105610in,y=1.841065in,,top]{\color{textcolor}\sffamily\fontsize{10.000000}{12.000000}\selectfont 0.2}%
\end{pgfscope}%
\begin{pgfscope}%
\pgfsetrectcap%
\pgfsetroundjoin%
\pgfsetlinewidth{0.803000pt}%
\definecolor{currentstroke}{rgb}{0.000000,0.000000,0.000000}%
\pgfsetstrokecolor{currentstroke}%
\pgfsetdash{}{0pt}%
\pgfpathmoveto{\pgfqpoint{6.879319in}{2.045932in}}%
\pgfpathlineto{\pgfqpoint{6.912758in}{2.036631in}}%
\pgfusepath{stroke}%
\end{pgfscope}%
\begin{pgfscope}%
\definecolor{textcolor}{rgb}{0.000000,0.000000,0.000000}%
\pgfsetstrokecolor{textcolor}%
\pgfsetfillcolor{textcolor}%
\pgftext[x=7.116936in,y=2.073588in,,top]{\color{textcolor}\sffamily\fontsize{10.000000}{12.000000}\selectfont 0.4}%
\end{pgfscope}%
\begin{pgfscope}%
\pgfsetrectcap%
\pgfsetroundjoin%
\pgfsetlinewidth{0.803000pt}%
\definecolor{currentstroke}{rgb}{0.000000,0.000000,0.000000}%
\pgfsetstrokecolor{currentstroke}%
\pgfsetdash{}{0pt}%
\pgfpathmoveto{\pgfqpoint{6.888751in}{2.283126in}}%
\pgfpathlineto{\pgfqpoint{6.922500in}{2.273995in}}%
\pgfusepath{stroke}%
\end{pgfscope}%
\begin{pgfscope}%
\definecolor{textcolor}{rgb}{0.000000,0.000000,0.000000}%
\pgfsetstrokecolor{textcolor}%
\pgfsetfillcolor{textcolor}%
\pgftext[x=7.128466in,y=2.310275in,,top]{\color{textcolor}\sffamily\fontsize{10.000000}{12.000000}\selectfont 0.6}%
\end{pgfscope}%
\begin{pgfscope}%
\pgfsetrectcap%
\pgfsetroundjoin%
\pgfsetlinewidth{0.803000pt}%
\definecolor{currentstroke}{rgb}{0.000000,0.000000,0.000000}%
\pgfsetstrokecolor{currentstroke}%
\pgfsetdash{}{0pt}%
\pgfpathmoveto{\pgfqpoint{6.898354in}{2.524619in}}%
\pgfpathlineto{\pgfqpoint{6.932420in}{2.515665in}}%
\pgfusepath{stroke}%
\end{pgfscope}%
\begin{pgfscope}%
\definecolor{textcolor}{rgb}{0.000000,0.000000,0.000000}%
\pgfsetstrokecolor{textcolor}%
\pgfsetfillcolor{textcolor}%
\pgftext[x=7.140203in,y=2.551237in,,top]{\color{textcolor}\sffamily\fontsize{10.000000}{12.000000}\selectfont 0.8}%
\end{pgfscope}%
\begin{pgfscope}%
\pgfsetrectcap%
\pgfsetroundjoin%
\pgfsetlinewidth{0.803000pt}%
\definecolor{currentstroke}{rgb}{0.000000,0.000000,0.000000}%
\pgfsetstrokecolor{currentstroke}%
\pgfsetdash{}{0pt}%
\pgfpathmoveto{\pgfqpoint{6.908133in}{2.770528in}}%
\pgfpathlineto{\pgfqpoint{6.942521in}{2.761761in}}%
\pgfusepath{stroke}%
\end{pgfscope}%
\begin{pgfscope}%
\definecolor{textcolor}{rgb}{0.000000,0.000000,0.000000}%
\pgfsetstrokecolor{textcolor}%
\pgfsetfillcolor{textcolor}%
\pgftext[x=7.152155in,y=2.796592in,,top]{\color{textcolor}\sffamily\fontsize{10.000000}{12.000000}\selectfont 1.0}%
\end{pgfscope}%
\begin{pgfscope}%
\pgfpathrectangle{\pgfqpoint{4.381818in}{0.570909in}}{\pgfqpoint{2.818182in}{2.818182in}}%
\pgfusepath{clip}%
\pgfsetbuttcap%
\pgfsetroundjoin%
\definecolor{currentfill}{rgb}{0.172549,0.627451,0.172549}%
\pgfsetfillcolor{currentfill}%
\pgfsetlinewidth{1.003750pt}%
\definecolor{currentstroke}{rgb}{0.172549,0.627451,0.172549}%
\pgfsetstrokecolor{currentstroke}%
\pgfsetdash{}{0pt}%
\pgfpathmoveto{\pgfqpoint{5.463871in}{1.709450in}}%
\pgfpathcurveto{\pgfqpoint{5.472107in}{1.709450in}}{\pgfqpoint{5.480008in}{1.712722in}}{\pgfqpoint{5.485831in}{1.718546in}}%
\pgfpathcurveto{\pgfqpoint{5.491655in}{1.724370in}}{\pgfqpoint{5.494928in}{1.732270in}}{\pgfqpoint{5.494928in}{1.740506in}}%
\pgfpathcurveto{\pgfqpoint{5.494928in}{1.748743in}}{\pgfqpoint{5.491655in}{1.756643in}}{\pgfqpoint{5.485831in}{1.762466in}}%
\pgfpathcurveto{\pgfqpoint{5.480008in}{1.768290in}}{\pgfqpoint{5.472107in}{1.771563in}}{\pgfqpoint{5.463871in}{1.771563in}}%
\pgfpathcurveto{\pgfqpoint{5.455635in}{1.771563in}}{\pgfqpoint{5.447735in}{1.768290in}}{\pgfqpoint{5.441911in}{1.762466in}}%
\pgfpathcurveto{\pgfqpoint{5.436087in}{1.756643in}}{\pgfqpoint{5.432815in}{1.748743in}}{\pgfqpoint{5.432815in}{1.740506in}}%
\pgfpathcurveto{\pgfqpoint{5.432815in}{1.732270in}}{\pgfqpoint{5.436087in}{1.724370in}}{\pgfqpoint{5.441911in}{1.718546in}}%
\pgfpathcurveto{\pgfqpoint{5.447735in}{1.712722in}}{\pgfqpoint{5.455635in}{1.709450in}}{\pgfqpoint{5.463871in}{1.709450in}}%
\pgfpathclose%
\pgfusepath{stroke,fill}%
\end{pgfscope}%
\begin{pgfscope}%
\definecolor{textcolor}{rgb}{0.000000,0.000000,0.000000}%
\pgfsetstrokecolor{textcolor}%
\pgfsetfillcolor{textcolor}%
\pgftext[x=5.627727in,y=1.229636in,left,base]{\color{textcolor}\sffamily\fontsize{10.000000}{12.000000}\selectfont 0.04167}%
\end{pgfscope}%
\begin{pgfscope}%
\definecolor{textcolor}{rgb}{0.000000,0.000000,0.000000}%
\pgfsetstrokecolor{textcolor}%
\pgfsetfillcolor{textcolor}%
\pgftext[x=5.036742in,y=1.675666in,left,base]{\color{textcolor}\sffamily\fontsize{10.000000}{12.000000}\selectfont 0.04167}%
\end{pgfscope}%
\begin{pgfscope}%
\definecolor{textcolor}{rgb}{0.000000,0.000000,0.000000}%
\pgfsetstrokecolor{textcolor}%
\pgfsetfillcolor{textcolor}%
\pgftext[x=5.463871in,y=1.740506in,left,base]{\color{textcolor}\sffamily\fontsize{10.000000}{12.000000}\selectfont 0.04167}%
\end{pgfscope}%
\begin{pgfscope}%
\definecolor{textcolor}{rgb}{0.000000,0.000000,0.000000}%
\pgfsetstrokecolor{textcolor}%
\pgfsetfillcolor{textcolor}%
\pgftext[x=5.308989in,y=1.971368in,left,base]{\color{textcolor}\sffamily\fontsize{10.000000}{12.000000}\selectfont 0.04167}%
\end{pgfscope}%
\begin{pgfscope}%
\pgfpathrectangle{\pgfqpoint{4.381818in}{0.570909in}}{\pgfqpoint{2.818182in}{2.818182in}}%
\pgfusepath{clip}%
\pgfsetbuttcap%
\pgfsetroundjoin%
\definecolor{currentfill}{rgb}{1.000000,0.498039,0.054902}%
\pgfsetfillcolor{currentfill}%
\pgfsetlinewidth{1.003750pt}%
\definecolor{currentstroke}{rgb}{1.000000,0.498039,0.054902}%
\pgfsetstrokecolor{currentstroke}%
\pgfsetdash{}{0pt}%
\pgfpathmoveto{\pgfqpoint{5.036742in}{1.644610in}}%
\pgfpathcurveto{\pgfqpoint{5.044978in}{1.644610in}}{\pgfqpoint{5.052878in}{1.647882in}}{\pgfqpoint{5.058702in}{1.653706in}}%
\pgfpathcurveto{\pgfqpoint{5.064526in}{1.659530in}}{\pgfqpoint{5.067799in}{1.667430in}}{\pgfqpoint{5.067799in}{1.675666in}}%
\pgfpathcurveto{\pgfqpoint{5.067799in}{1.683903in}}{\pgfqpoint{5.064526in}{1.691803in}}{\pgfqpoint{5.058702in}{1.697627in}}%
\pgfpathcurveto{\pgfqpoint{5.052878in}{1.703450in}}{\pgfqpoint{5.044978in}{1.706723in}}{\pgfqpoint{5.036742in}{1.706723in}}%
\pgfpathcurveto{\pgfqpoint{5.028506in}{1.706723in}}{\pgfqpoint{5.020606in}{1.703450in}}{\pgfqpoint{5.014782in}{1.697627in}}%
\pgfpathcurveto{\pgfqpoint{5.008958in}{1.691803in}}{\pgfqpoint{5.005686in}{1.683903in}}{\pgfqpoint{5.005686in}{1.675666in}}%
\pgfpathcurveto{\pgfqpoint{5.005686in}{1.667430in}}{\pgfqpoint{5.008958in}{1.659530in}}{\pgfqpoint{5.014782in}{1.653706in}}%
\pgfpathcurveto{\pgfqpoint{5.020606in}{1.647882in}}{\pgfqpoint{5.028506in}{1.644610in}}{\pgfqpoint{5.036742in}{1.644610in}}%
\pgfpathclose%
\pgfusepath{stroke,fill}%
\end{pgfscope}%
\begin{pgfscope}%
\pgfpathrectangle{\pgfqpoint{4.381818in}{0.570909in}}{\pgfqpoint{2.818182in}{2.818182in}}%
\pgfusepath{clip}%
\pgfsetbuttcap%
\pgfsetroundjoin%
\definecolor{currentfill}{rgb}{0.121569,0.466667,0.705882}%
\pgfsetfillcolor{currentfill}%
\pgfsetlinewidth{1.003750pt}%
\definecolor{currentstroke}{rgb}{0.121569,0.466667,0.705882}%
\pgfsetstrokecolor{currentstroke}%
\pgfsetdash{}{0pt}%
\pgfpathmoveto{\pgfqpoint{5.627727in}{1.198579in}}%
\pgfpathcurveto{\pgfqpoint{5.635963in}{1.198579in}}{\pgfqpoint{5.643863in}{1.201852in}}{\pgfqpoint{5.649687in}{1.207676in}}%
\pgfpathcurveto{\pgfqpoint{5.655511in}{1.213500in}}{\pgfqpoint{5.658784in}{1.221400in}}{\pgfqpoint{5.658784in}{1.229636in}}%
\pgfpathcurveto{\pgfqpoint{5.658784in}{1.237872in}}{\pgfqpoint{5.655511in}{1.245772in}}{\pgfqpoint{5.649687in}{1.251596in}}%
\pgfpathcurveto{\pgfqpoint{5.643863in}{1.257420in}}{\pgfqpoint{5.635963in}{1.260692in}}{\pgfqpoint{5.627727in}{1.260692in}}%
\pgfpathcurveto{\pgfqpoint{5.619491in}{1.260692in}}{\pgfqpoint{5.611591in}{1.257420in}}{\pgfqpoint{5.605767in}{1.251596in}}%
\pgfpathcurveto{\pgfqpoint{5.599943in}{1.245772in}}{\pgfqpoint{5.596671in}{1.237872in}}{\pgfqpoint{5.596671in}{1.229636in}}%
\pgfpathcurveto{\pgfqpoint{5.596671in}{1.221400in}}{\pgfqpoint{5.599943in}{1.213500in}}{\pgfqpoint{5.605767in}{1.207676in}}%
\pgfpathcurveto{\pgfqpoint{5.611591in}{1.201852in}}{\pgfqpoint{5.619491in}{1.198579in}}{\pgfqpoint{5.627727in}{1.198579in}}%
\pgfpathclose%
\pgfusepath{stroke,fill}%
\end{pgfscope}%
\begin{pgfscope}%
\pgfpathrectangle{\pgfqpoint{4.381818in}{0.570909in}}{\pgfqpoint{2.818182in}{2.818182in}}%
\pgfusepath{clip}%
\pgfsetbuttcap%
\pgfsetroundjoin%
\definecolor{currentfill}{rgb}{0.839216,0.152941,0.156863}%
\pgfsetfillcolor{currentfill}%
\pgfsetlinewidth{1.003750pt}%
\definecolor{currentstroke}{rgb}{0.839216,0.152941,0.156863}%
\pgfsetstrokecolor{currentstroke}%
\pgfsetdash{}{0pt}%
\pgfpathmoveto{\pgfqpoint{5.308989in}{1.940311in}}%
\pgfpathcurveto{\pgfqpoint{5.317225in}{1.940311in}}{\pgfqpoint{5.325125in}{1.943584in}}{\pgfqpoint{5.330949in}{1.949408in}}%
\pgfpathcurveto{\pgfqpoint{5.336773in}{1.955232in}}{\pgfqpoint{5.340045in}{1.963132in}}{\pgfqpoint{5.340045in}{1.971368in}}%
\pgfpathcurveto{\pgfqpoint{5.340045in}{1.979604in}}{\pgfqpoint{5.336773in}{1.987504in}}{\pgfqpoint{5.330949in}{1.993328in}}%
\pgfpathcurveto{\pgfqpoint{5.325125in}{1.999152in}}{\pgfqpoint{5.317225in}{2.002424in}}{\pgfqpoint{5.308989in}{2.002424in}}%
\pgfpathcurveto{\pgfqpoint{5.300753in}{2.002424in}}{\pgfqpoint{5.292853in}{1.999152in}}{\pgfqpoint{5.287029in}{1.993328in}}%
\pgfpathcurveto{\pgfqpoint{5.281205in}{1.987504in}}{\pgfqpoint{5.277932in}{1.979604in}}{\pgfqpoint{5.277932in}{1.971368in}}%
\pgfpathcurveto{\pgfqpoint{5.277932in}{1.963132in}}{\pgfqpoint{5.281205in}{1.955232in}}{\pgfqpoint{5.287029in}{1.949408in}}%
\pgfpathcurveto{\pgfqpoint{5.292853in}{1.943584in}}{\pgfqpoint{5.300753in}{1.940311in}}{\pgfqpoint{5.308989in}{1.940311in}}%
\pgfpathclose%
\pgfusepath{stroke,fill}%
\end{pgfscope}%
\begin{pgfscope}%
\pgfpathrectangle{\pgfqpoint{4.381818in}{0.570909in}}{\pgfqpoint{2.818182in}{2.818182in}}%
\pgfusepath{clip}%
\pgfsetbuttcap%
\pgfsetroundjoin%
\definecolor{currentfill}{rgb}{0.100000,0.100000,0.100000}%
\pgfsetfillcolor{currentfill}%
\pgfsetfillopacity{0.100000}%
\pgfsetlinewidth{0.301125pt}%
\definecolor{currentstroke}{rgb}{0.000000,0.000000,0.000000}%
\pgfsetstrokecolor{currentstroke}%
\pgfsetdash{}{0pt}%
\pgfpathmoveto{\pgfqpoint{4.890922in}{1.297693in}}%
\pgfpathlineto{\pgfqpoint{5.593557in}{1.900366in}}%
\pgfpathlineto{\pgfqpoint{6.090559in}{0.911119in}}%
\pgfpathclose%
\pgfusepath{stroke,fill}%
\end{pgfscope}%
\begin{pgfscope}%
\pgfpathrectangle{\pgfqpoint{4.381818in}{0.570909in}}{\pgfqpoint{2.818182in}{2.818182in}}%
\pgfusepath{clip}%
\pgfsetbuttcap%
\pgfsetroundjoin%
\definecolor{currentfill}{rgb}{0.100000,0.100000,0.100000}%
\pgfsetfillcolor{currentfill}%
\pgfsetfillopacity{0.100000}%
\pgfsetlinewidth{0.301125pt}%
\definecolor{currentstroke}{rgb}{0.000000,0.000000,0.000000}%
\pgfsetstrokecolor{currentstroke}%
\pgfsetdash{}{0pt}%
\pgfpathmoveto{\pgfqpoint{4.890922in}{1.297693in}}%
\pgfpathlineto{\pgfqpoint{5.593557in}{1.900366in}}%
\pgfpathlineto{\pgfqpoint{4.846638in}{2.509261in}}%
\pgfpathclose%
\pgfusepath{stroke,fill}%
\end{pgfscope}%
\begin{pgfscope}%
\pgfpathrectangle{\pgfqpoint{4.381818in}{0.570909in}}{\pgfqpoint{2.818182in}{2.818182in}}%
\pgfusepath{clip}%
\pgfsetbuttcap%
\pgfsetroundjoin%
\definecolor{currentfill}{rgb}{0.100000,0.100000,0.100000}%
\pgfsetfillcolor{currentfill}%
\pgfsetfillopacity{0.100000}%
\pgfsetlinewidth{0.301125pt}%
\definecolor{currentstroke}{rgb}{0.000000,0.000000,0.000000}%
\pgfsetstrokecolor{currentstroke}%
\pgfsetdash{}{0pt}%
\pgfpathmoveto{\pgfqpoint{4.846638in}{2.509261in}}%
\pgfpathlineto{\pgfqpoint{5.593557in}{1.900366in}}%
\pgfpathlineto{\pgfqpoint{6.090559in}{0.911119in}}%
\pgfpathclose%
\pgfusepath{stroke,fill}%
\end{pgfscope}%
\begin{pgfscope}%
\pgfpathrectangle{\pgfqpoint{4.381818in}{0.570909in}}{\pgfqpoint{2.818182in}{2.818182in}}%
\pgfusepath{clip}%
\pgfsetbuttcap%
\pgfsetroundjoin%
\definecolor{currentfill}{rgb}{0.100000,0.100000,0.100000}%
\pgfsetfillcolor{currentfill}%
\pgfsetfillopacity{0.100000}%
\pgfsetlinewidth{0.301125pt}%
\definecolor{currentstroke}{rgb}{0.000000,0.000000,0.000000}%
\pgfsetstrokecolor{currentstroke}%
\pgfsetdash{}{0pt}%
\pgfpathmoveto{\pgfqpoint{4.890922in}{1.297693in}}%
\pgfpathlineto{\pgfqpoint{6.090559in}{0.911119in}}%
\pgfpathlineto{\pgfqpoint{4.846638in}{2.509261in}}%
\pgfpathclose%
\pgfusepath{stroke,fill}%
\end{pgfscope}%
\end{pgfpicture}%
\makeatother%
\endgroup%
}
\scalebox{0.75}{%% Creator: Matplotlib, PGF backend
%%
%% To include the figure in your LaTeX document, write
%%   \input{<filename>.pgf}
%%
%% Make sure the required packages are loaded in your preamble
%%   \usepackage{pgf}
%%
%% Figures using additional raster images can only be included by \input if
%% they are in the same directory as the main LaTeX file. For loading figures
%% from other directories you can use the `import` package
%%   \usepackage{import}
%% and then include the figures with
%%   \import{<path to file>}{<filename>.pgf}
%%
%% Matplotlib used the following preamble
%%   \usepackage{fontspec}
%%   \setmainfont{DejaVuSerif.ttf}[Path=/home/lzh/anaconda3/envs/numana/lib/python3.7/site-packages/matplotlib/mpl-data/fonts/ttf/]
%%   \setsansfont{DejaVuSans.ttf}[Path=/home/lzh/anaconda3/envs/numana/lib/python3.7/site-packages/matplotlib/mpl-data/fonts/ttf/]
%%   \setmonofont{DejaVuSansMono.ttf}[Path=/home/lzh/anaconda3/envs/numana/lib/python3.7/site-packages/matplotlib/mpl-data/fonts/ttf/]
%%
\begingroup%
\makeatletter%
\begin{pgfpicture}%
\pgfpathrectangle{\pgfpointorigin}{\pgfqpoint{8.000000in}{6.000000in}}%
\pgfusepath{use as bounding box, clip}%
\begin{pgfscope}%
\pgfsetbuttcap%
\pgfsetmiterjoin%
\definecolor{currentfill}{rgb}{1.000000,1.000000,1.000000}%
\pgfsetfillcolor{currentfill}%
\pgfsetlinewidth{0.000000pt}%
\definecolor{currentstroke}{rgb}{1.000000,1.000000,1.000000}%
\pgfsetstrokecolor{currentstroke}%
\pgfsetdash{}{0pt}%
\pgfpathmoveto{\pgfqpoint{0.000000in}{0.000000in}}%
\pgfpathlineto{\pgfqpoint{8.000000in}{0.000000in}}%
\pgfpathlineto{\pgfqpoint{8.000000in}{6.000000in}}%
\pgfpathlineto{\pgfqpoint{0.000000in}{6.000000in}}%
\pgfpathclose%
\pgfusepath{fill}%
\end{pgfscope}%
\begin{pgfscope}%
\pgfsetbuttcap%
\pgfsetmiterjoin%
\definecolor{currentfill}{rgb}{1.000000,1.000000,1.000000}%
\pgfsetfillcolor{currentfill}%
\pgfsetlinewidth{0.000000pt}%
\definecolor{currentstroke}{rgb}{0.000000,0.000000,0.000000}%
\pgfsetstrokecolor{currentstroke}%
\pgfsetstrokeopacity{0.000000}%
\pgfsetdash{}{0pt}%
\pgfpathmoveto{\pgfqpoint{0.557222in}{3.312222in}}%
\pgfpathlineto{\pgfqpoint{3.878555in}{3.312222in}}%
\pgfpathlineto{\pgfqpoint{3.878555in}{5.636667in}}%
\pgfpathlineto{\pgfqpoint{0.557222in}{5.636667in}}%
\pgfpathclose%
\pgfusepath{fill}%
\end{pgfscope}%
\begin{pgfscope}%
\pgfpathrectangle{\pgfqpoint{0.557222in}{3.312222in}}{\pgfqpoint{3.321333in}{2.324444in}}%
\pgfusepath{clip}%
\pgfsetbuttcap%
\pgfsetroundjoin%
\definecolor{currentfill}{rgb}{0.121569,0.466667,0.705882}%
\pgfsetfillcolor{currentfill}%
\pgfsetfillopacity{0.300000}%
\pgfsetlinewidth{1.003750pt}%
\definecolor{currentstroke}{rgb}{0.121569,0.466667,0.705882}%
\pgfsetstrokecolor{currentstroke}%
\pgfsetstrokeopacity{0.300000}%
\pgfsetdash{}{0pt}%
\pgfpathmoveto{\pgfqpoint{0.716480in}{3.672916in}}%
\pgfpathlineto{\pgfqpoint{0.716480in}{3.417879in}}%
\pgfpathlineto{\pgfqpoint{0.866621in}{3.626120in}}%
\pgfpathlineto{\pgfqpoint{1.016761in}{3.871857in}}%
\pgfpathlineto{\pgfqpoint{1.166902in}{3.987270in}}%
\pgfpathlineto{\pgfqpoint{1.317043in}{4.211289in}}%
\pgfpathlineto{\pgfqpoint{1.467184in}{4.414802in}}%
\pgfpathlineto{\pgfqpoint{1.617325in}{4.507409in}}%
\pgfpathlineto{\pgfqpoint{1.767466in}{4.662197in}}%
\pgfpathlineto{\pgfqpoint{1.917607in}{4.770355in}}%
\pgfpathlineto{\pgfqpoint{2.067748in}{4.908875in}}%
\pgfpathlineto{\pgfqpoint{2.217889in}{5.023077in}}%
\pgfpathlineto{\pgfqpoint{2.368030in}{5.106082in}}%
\pgfpathlineto{\pgfqpoint{2.518171in}{5.180012in}}%
\pgfpathlineto{\pgfqpoint{2.668311in}{5.263007in}}%
\pgfpathlineto{\pgfqpoint{2.818452in}{5.401152in}}%
\pgfpathlineto{\pgfqpoint{2.968593in}{5.305616in}}%
\pgfpathlineto{\pgfqpoint{3.118734in}{5.309602in}}%
\pgfpathlineto{\pgfqpoint{3.268875in}{5.081954in}}%
\pgfpathlineto{\pgfqpoint{3.419016in}{5.066298in}}%
\pgfpathlineto{\pgfqpoint{3.569157in}{4.915523in}}%
\pgfpathlineto{\pgfqpoint{3.719298in}{4.758516in}}%
\pgfpathlineto{\pgfqpoint{3.719298in}{4.877711in}}%
\pgfpathlineto{\pgfqpoint{3.719298in}{4.877711in}}%
\pgfpathlineto{\pgfqpoint{3.569157in}{5.029667in}}%
\pgfpathlineto{\pgfqpoint{3.419016in}{5.231818in}}%
\pgfpathlineto{\pgfqpoint{3.268875in}{5.385278in}}%
\pgfpathlineto{\pgfqpoint{3.118734in}{5.358833in}}%
\pgfpathlineto{\pgfqpoint{2.968593in}{5.428529in}}%
\pgfpathlineto{\pgfqpoint{2.818452in}{5.531010in}}%
\pgfpathlineto{\pgfqpoint{2.668311in}{5.491378in}}%
\pgfpathlineto{\pgfqpoint{2.518171in}{5.409003in}}%
\pgfpathlineto{\pgfqpoint{2.368030in}{5.378136in}}%
\pgfpathlineto{\pgfqpoint{2.217889in}{5.294052in}}%
\pgfpathlineto{\pgfqpoint{2.067748in}{5.153425in}}%
\pgfpathlineto{\pgfqpoint{1.917607in}{5.154178in}}%
\pgfpathlineto{\pgfqpoint{1.767466in}{4.925890in}}%
\pgfpathlineto{\pgfqpoint{1.617325in}{4.769365in}}%
\pgfpathlineto{\pgfqpoint{1.467184in}{4.534989in}}%
\pgfpathlineto{\pgfqpoint{1.317043in}{4.399321in}}%
\pgfpathlineto{\pgfqpoint{1.166902in}{4.234998in}}%
\pgfpathlineto{\pgfqpoint{1.016761in}{3.937211in}}%
\pgfpathlineto{\pgfqpoint{0.866621in}{3.801449in}}%
\pgfpathlineto{\pgfqpoint{0.716480in}{3.672916in}}%
\pgfpathclose%
\pgfusepath{stroke,fill}%
\end{pgfscope}%
\begin{pgfscope}%
\pgfpathrectangle{\pgfqpoint{0.557222in}{3.312222in}}{\pgfqpoint{3.321333in}{2.324444in}}%
\pgfusepath{clip}%
\pgfsetbuttcap%
\pgfsetroundjoin%
\definecolor{currentfill}{rgb}{0.121569,0.466667,0.705882}%
\pgfsetfillcolor{currentfill}%
\pgfsetlinewidth{1.003750pt}%
\definecolor{currentstroke}{rgb}{0.121569,0.466667,0.705882}%
\pgfsetstrokecolor{currentstroke}%
\pgfsetdash{}{0pt}%
\pgfsys@defobject{currentmarker}{\pgfqpoint{-0.009821in}{-0.009821in}}{\pgfqpoint{0.009821in}{0.009821in}}{%
\pgfpathmoveto{\pgfqpoint{0.000000in}{-0.009821in}}%
\pgfpathcurveto{\pgfqpoint{0.002605in}{-0.009821in}}{\pgfqpoint{0.005103in}{-0.008786in}}{\pgfqpoint{0.006944in}{-0.006944in}}%
\pgfpathcurveto{\pgfqpoint{0.008786in}{-0.005103in}}{\pgfqpoint{0.009821in}{-0.002605in}}{\pgfqpoint{0.009821in}{0.000000in}}%
\pgfpathcurveto{\pgfqpoint{0.009821in}{0.002605in}}{\pgfqpoint{0.008786in}{0.005103in}}{\pgfqpoint{0.006944in}{0.006944in}}%
\pgfpathcurveto{\pgfqpoint{0.005103in}{0.008786in}}{\pgfqpoint{0.002605in}{0.009821in}}{\pgfqpoint{0.000000in}{0.009821in}}%
\pgfpathcurveto{\pgfqpoint{-0.002605in}{0.009821in}}{\pgfqpoint{-0.005103in}{0.008786in}}{\pgfqpoint{-0.006944in}{0.006944in}}%
\pgfpathcurveto{\pgfqpoint{-0.008786in}{0.005103in}}{\pgfqpoint{-0.009821in}{0.002605in}}{\pgfqpoint{-0.009821in}{0.000000in}}%
\pgfpathcurveto{\pgfqpoint{-0.009821in}{-0.002605in}}{\pgfqpoint{-0.008786in}{-0.005103in}}{\pgfqpoint{-0.006944in}{-0.006944in}}%
\pgfpathcurveto{\pgfqpoint{-0.005103in}{-0.008786in}}{\pgfqpoint{-0.002605in}{-0.009821in}}{\pgfqpoint{0.000000in}{-0.009821in}}%
\pgfpathclose%
\pgfusepath{stroke,fill}%
}%
\begin{pgfscope}%
\pgfsys@transformshift{0.716480in}{3.545397in}%
\pgfsys@useobject{currentmarker}{}%
\end{pgfscope}%
\begin{pgfscope}%
\pgfsys@transformshift{0.866621in}{3.713785in}%
\pgfsys@useobject{currentmarker}{}%
\end{pgfscope}%
\begin{pgfscope}%
\pgfsys@transformshift{1.016761in}{3.904534in}%
\pgfsys@useobject{currentmarker}{}%
\end{pgfscope}%
\begin{pgfscope}%
\pgfsys@transformshift{1.166902in}{4.111134in}%
\pgfsys@useobject{currentmarker}{}%
\end{pgfscope}%
\begin{pgfscope}%
\pgfsys@transformshift{1.317043in}{4.305305in}%
\pgfsys@useobject{currentmarker}{}%
\end{pgfscope}%
\begin{pgfscope}%
\pgfsys@transformshift{1.467184in}{4.474895in}%
\pgfsys@useobject{currentmarker}{}%
\end{pgfscope}%
\begin{pgfscope}%
\pgfsys@transformshift{1.617325in}{4.638387in}%
\pgfsys@useobject{currentmarker}{}%
\end{pgfscope}%
\begin{pgfscope}%
\pgfsys@transformshift{1.767466in}{4.794043in}%
\pgfsys@useobject{currentmarker}{}%
\end{pgfscope}%
\begin{pgfscope}%
\pgfsys@transformshift{1.917607in}{4.962267in}%
\pgfsys@useobject{currentmarker}{}%
\end{pgfscope}%
\begin{pgfscope}%
\pgfsys@transformshift{2.067748in}{5.031150in}%
\pgfsys@useobject{currentmarker}{}%
\end{pgfscope}%
\begin{pgfscope}%
\pgfsys@transformshift{2.217889in}{5.158565in}%
\pgfsys@useobject{currentmarker}{}%
\end{pgfscope}%
\begin{pgfscope}%
\pgfsys@transformshift{2.368030in}{5.242109in}%
\pgfsys@useobject{currentmarker}{}%
\end{pgfscope}%
\begin{pgfscope}%
\pgfsys@transformshift{2.518171in}{5.294508in}%
\pgfsys@useobject{currentmarker}{}%
\end{pgfscope}%
\begin{pgfscope}%
\pgfsys@transformshift{2.668311in}{5.377193in}%
\pgfsys@useobject{currentmarker}{}%
\end{pgfscope}%
\begin{pgfscope}%
\pgfsys@transformshift{2.818452in}{5.466081in}%
\pgfsys@useobject{currentmarker}{}%
\end{pgfscope}%
\begin{pgfscope}%
\pgfsys@transformshift{2.968593in}{5.367072in}%
\pgfsys@useobject{currentmarker}{}%
\end{pgfscope}%
\begin{pgfscope}%
\pgfsys@transformshift{3.118734in}{5.334217in}%
\pgfsys@useobject{currentmarker}{}%
\end{pgfscope}%
\begin{pgfscope}%
\pgfsys@transformshift{3.268875in}{5.233616in}%
\pgfsys@useobject{currentmarker}{}%
\end{pgfscope}%
\begin{pgfscope}%
\pgfsys@transformshift{3.419016in}{5.149058in}%
\pgfsys@useobject{currentmarker}{}%
\end{pgfscope}%
\begin{pgfscope}%
\pgfsys@transformshift{3.569157in}{4.972595in}%
\pgfsys@useobject{currentmarker}{}%
\end{pgfscope}%
\begin{pgfscope}%
\pgfsys@transformshift{3.719298in}{4.818114in}%
\pgfsys@useobject{currentmarker}{}%
\end{pgfscope}%
\end{pgfscope}%
\begin{pgfscope}%
\pgfsetbuttcap%
\pgfsetroundjoin%
\definecolor{currentfill}{rgb}{0.000000,0.000000,0.000000}%
\pgfsetfillcolor{currentfill}%
\pgfsetlinewidth{0.803000pt}%
\definecolor{currentstroke}{rgb}{0.000000,0.000000,0.000000}%
\pgfsetstrokecolor{currentstroke}%
\pgfsetdash{}{0pt}%
\pgfsys@defobject{currentmarker}{\pgfqpoint{0.000000in}{-0.048611in}}{\pgfqpoint{0.000000in}{0.000000in}}{%
\pgfpathmoveto{\pgfqpoint{0.000000in}{0.000000in}}%
\pgfpathlineto{\pgfqpoint{0.000000in}{-0.048611in}}%
\pgfusepath{stroke,fill}%
}%
\begin{pgfscope}%
\pgfsys@transformshift{0.566339in}{3.312222in}%
\pgfsys@useobject{currentmarker}{}%
\end{pgfscope}%
\end{pgfscope}%
\begin{pgfscope}%
\definecolor{textcolor}{rgb}{0.000000,0.000000,0.000000}%
\pgfsetstrokecolor{textcolor}%
\pgfsetfillcolor{textcolor}%
\pgftext[x=0.566339in,y=3.215000in,,top]{\color{textcolor}\sffamily\fontsize{10.000000}{12.000000}\selectfont 4.20}%
\end{pgfscope}%
\begin{pgfscope}%
\pgfsetbuttcap%
\pgfsetroundjoin%
\definecolor{currentfill}{rgb}{0.000000,0.000000,0.000000}%
\pgfsetfillcolor{currentfill}%
\pgfsetlinewidth{0.803000pt}%
\definecolor{currentstroke}{rgb}{0.000000,0.000000,0.000000}%
\pgfsetstrokecolor{currentstroke}%
\pgfsetdash{}{0pt}%
\pgfsys@defobject{currentmarker}{\pgfqpoint{0.000000in}{-0.048611in}}{\pgfqpoint{0.000000in}{0.000000in}}{%
\pgfpathmoveto{\pgfqpoint{0.000000in}{0.000000in}}%
\pgfpathlineto{\pgfqpoint{0.000000in}{-0.048611in}}%
\pgfusepath{stroke,fill}%
}%
\begin{pgfscope}%
\pgfsys@transformshift{1.317043in}{3.312222in}%
\pgfsys@useobject{currentmarker}{}%
\end{pgfscope}%
\end{pgfscope}%
\begin{pgfscope}%
\definecolor{textcolor}{rgb}{0.000000,0.000000,0.000000}%
\pgfsetstrokecolor{textcolor}%
\pgfsetfillcolor{textcolor}%
\pgftext[x=1.317043in,y=3.215000in,,top]{\color{textcolor}\sffamily\fontsize{10.000000}{12.000000}\selectfont 4.25}%
\end{pgfscope}%
\begin{pgfscope}%
\pgfsetbuttcap%
\pgfsetroundjoin%
\definecolor{currentfill}{rgb}{0.000000,0.000000,0.000000}%
\pgfsetfillcolor{currentfill}%
\pgfsetlinewidth{0.803000pt}%
\definecolor{currentstroke}{rgb}{0.000000,0.000000,0.000000}%
\pgfsetstrokecolor{currentstroke}%
\pgfsetdash{}{0pt}%
\pgfsys@defobject{currentmarker}{\pgfqpoint{0.000000in}{-0.048611in}}{\pgfqpoint{0.000000in}{0.000000in}}{%
\pgfpathmoveto{\pgfqpoint{0.000000in}{0.000000in}}%
\pgfpathlineto{\pgfqpoint{0.000000in}{-0.048611in}}%
\pgfusepath{stroke,fill}%
}%
\begin{pgfscope}%
\pgfsys@transformshift{2.067748in}{3.312222in}%
\pgfsys@useobject{currentmarker}{}%
\end{pgfscope}%
\end{pgfscope}%
\begin{pgfscope}%
\definecolor{textcolor}{rgb}{0.000000,0.000000,0.000000}%
\pgfsetstrokecolor{textcolor}%
\pgfsetfillcolor{textcolor}%
\pgftext[x=2.067748in,y=3.215000in,,top]{\color{textcolor}\sffamily\fontsize{10.000000}{12.000000}\selectfont 4.30}%
\end{pgfscope}%
\begin{pgfscope}%
\pgfsetbuttcap%
\pgfsetroundjoin%
\definecolor{currentfill}{rgb}{0.000000,0.000000,0.000000}%
\pgfsetfillcolor{currentfill}%
\pgfsetlinewidth{0.803000pt}%
\definecolor{currentstroke}{rgb}{0.000000,0.000000,0.000000}%
\pgfsetstrokecolor{currentstroke}%
\pgfsetdash{}{0pt}%
\pgfsys@defobject{currentmarker}{\pgfqpoint{0.000000in}{-0.048611in}}{\pgfqpoint{0.000000in}{0.000000in}}{%
\pgfpathmoveto{\pgfqpoint{0.000000in}{0.000000in}}%
\pgfpathlineto{\pgfqpoint{0.000000in}{-0.048611in}}%
\pgfusepath{stroke,fill}%
}%
\begin{pgfscope}%
\pgfsys@transformshift{2.818452in}{3.312222in}%
\pgfsys@useobject{currentmarker}{}%
\end{pgfscope}%
\end{pgfscope}%
\begin{pgfscope}%
\definecolor{textcolor}{rgb}{0.000000,0.000000,0.000000}%
\pgfsetstrokecolor{textcolor}%
\pgfsetfillcolor{textcolor}%
\pgftext[x=2.818452in,y=3.215000in,,top]{\color{textcolor}\sffamily\fontsize{10.000000}{12.000000}\selectfont 4.35}%
\end{pgfscope}%
\begin{pgfscope}%
\pgfsetbuttcap%
\pgfsetroundjoin%
\definecolor{currentfill}{rgb}{0.000000,0.000000,0.000000}%
\pgfsetfillcolor{currentfill}%
\pgfsetlinewidth{0.803000pt}%
\definecolor{currentstroke}{rgb}{0.000000,0.000000,0.000000}%
\pgfsetstrokecolor{currentstroke}%
\pgfsetdash{}{0pt}%
\pgfsys@defobject{currentmarker}{\pgfqpoint{0.000000in}{-0.048611in}}{\pgfqpoint{0.000000in}{0.000000in}}{%
\pgfpathmoveto{\pgfqpoint{0.000000in}{0.000000in}}%
\pgfpathlineto{\pgfqpoint{0.000000in}{-0.048611in}}%
\pgfusepath{stroke,fill}%
}%
\begin{pgfscope}%
\pgfsys@transformshift{3.569157in}{3.312222in}%
\pgfsys@useobject{currentmarker}{}%
\end{pgfscope}%
\end{pgfscope}%
\begin{pgfscope}%
\definecolor{textcolor}{rgb}{0.000000,0.000000,0.000000}%
\pgfsetstrokecolor{textcolor}%
\pgfsetfillcolor{textcolor}%
\pgftext[x=3.569157in,y=3.215000in,,top]{\color{textcolor}\sffamily\fontsize{10.000000}{12.000000}\selectfont 4.40}%
\end{pgfscope}%
\begin{pgfscope}%
\pgfsetbuttcap%
\pgfsetroundjoin%
\definecolor{currentfill}{rgb}{0.000000,0.000000,0.000000}%
\pgfsetfillcolor{currentfill}%
\pgfsetlinewidth{0.803000pt}%
\definecolor{currentstroke}{rgb}{0.000000,0.000000,0.000000}%
\pgfsetstrokecolor{currentstroke}%
\pgfsetdash{}{0pt}%
\pgfsys@defobject{currentmarker}{\pgfqpoint{-0.048611in}{0.000000in}}{\pgfqpoint{0.000000in}{0.000000in}}{%
\pgfpathmoveto{\pgfqpoint{0.000000in}{0.000000in}}%
\pgfpathlineto{\pgfqpoint{-0.048611in}{0.000000in}}%
\pgfusepath{stroke,fill}%
}%
\begin{pgfscope}%
\pgfsys@transformshift{0.557222in}{3.612294in}%
\pgfsys@useobject{currentmarker}{}%
\end{pgfscope}%
\end{pgfscope}%
\begin{pgfscope}%
\definecolor{textcolor}{rgb}{0.000000,0.000000,0.000000}%
\pgfsetstrokecolor{textcolor}%
\pgfsetfillcolor{textcolor}%
\pgftext[x=0.150755in,y=3.559532in,left,base]{\color{textcolor}\sffamily\fontsize{10.000000}{12.000000}\selectfont 1.60}%
\end{pgfscope}%
\begin{pgfscope}%
\pgfsetbuttcap%
\pgfsetroundjoin%
\definecolor{currentfill}{rgb}{0.000000,0.000000,0.000000}%
\pgfsetfillcolor{currentfill}%
\pgfsetlinewidth{0.803000pt}%
\definecolor{currentstroke}{rgb}{0.000000,0.000000,0.000000}%
\pgfsetstrokecolor{currentstroke}%
\pgfsetdash{}{0pt}%
\pgfsys@defobject{currentmarker}{\pgfqpoint{-0.048611in}{0.000000in}}{\pgfqpoint{0.000000in}{0.000000in}}{%
\pgfpathmoveto{\pgfqpoint{0.000000in}{0.000000in}}%
\pgfpathlineto{\pgfqpoint{-0.048611in}{0.000000in}}%
\pgfusepath{stroke,fill}%
}%
\begin{pgfscope}%
\pgfsys@transformshift{0.557222in}{4.082278in}%
\pgfsys@useobject{currentmarker}{}%
\end{pgfscope}%
\end{pgfscope}%
\begin{pgfscope}%
\definecolor{textcolor}{rgb}{0.000000,0.000000,0.000000}%
\pgfsetstrokecolor{textcolor}%
\pgfsetfillcolor{textcolor}%
\pgftext[x=0.150755in,y=4.029517in,left,base]{\color{textcolor}\sffamily\fontsize{10.000000}{12.000000}\selectfont 1.65}%
\end{pgfscope}%
\begin{pgfscope}%
\pgfsetbuttcap%
\pgfsetroundjoin%
\definecolor{currentfill}{rgb}{0.000000,0.000000,0.000000}%
\pgfsetfillcolor{currentfill}%
\pgfsetlinewidth{0.803000pt}%
\definecolor{currentstroke}{rgb}{0.000000,0.000000,0.000000}%
\pgfsetstrokecolor{currentstroke}%
\pgfsetdash{}{0pt}%
\pgfsys@defobject{currentmarker}{\pgfqpoint{-0.048611in}{0.000000in}}{\pgfqpoint{0.000000in}{0.000000in}}{%
\pgfpathmoveto{\pgfqpoint{0.000000in}{0.000000in}}%
\pgfpathlineto{\pgfqpoint{-0.048611in}{0.000000in}}%
\pgfusepath{stroke,fill}%
}%
\begin{pgfscope}%
\pgfsys@transformshift{0.557222in}{4.552263in}%
\pgfsys@useobject{currentmarker}{}%
\end{pgfscope}%
\end{pgfscope}%
\begin{pgfscope}%
\definecolor{textcolor}{rgb}{0.000000,0.000000,0.000000}%
\pgfsetstrokecolor{textcolor}%
\pgfsetfillcolor{textcolor}%
\pgftext[x=0.150755in,y=4.499501in,left,base]{\color{textcolor}\sffamily\fontsize{10.000000}{12.000000}\selectfont 1.70}%
\end{pgfscope}%
\begin{pgfscope}%
\pgfsetbuttcap%
\pgfsetroundjoin%
\definecolor{currentfill}{rgb}{0.000000,0.000000,0.000000}%
\pgfsetfillcolor{currentfill}%
\pgfsetlinewidth{0.803000pt}%
\definecolor{currentstroke}{rgb}{0.000000,0.000000,0.000000}%
\pgfsetstrokecolor{currentstroke}%
\pgfsetdash{}{0pt}%
\pgfsys@defobject{currentmarker}{\pgfqpoint{-0.048611in}{0.000000in}}{\pgfqpoint{0.000000in}{0.000000in}}{%
\pgfpathmoveto{\pgfqpoint{0.000000in}{0.000000in}}%
\pgfpathlineto{\pgfqpoint{-0.048611in}{0.000000in}}%
\pgfusepath{stroke,fill}%
}%
\begin{pgfscope}%
\pgfsys@transformshift{0.557222in}{5.022247in}%
\pgfsys@useobject{currentmarker}{}%
\end{pgfscope}%
\end{pgfscope}%
\begin{pgfscope}%
\definecolor{textcolor}{rgb}{0.000000,0.000000,0.000000}%
\pgfsetstrokecolor{textcolor}%
\pgfsetfillcolor{textcolor}%
\pgftext[x=0.150755in,y=4.969485in,left,base]{\color{textcolor}\sffamily\fontsize{10.000000}{12.000000}\selectfont 1.75}%
\end{pgfscope}%
\begin{pgfscope}%
\pgfsetbuttcap%
\pgfsetroundjoin%
\definecolor{currentfill}{rgb}{0.000000,0.000000,0.000000}%
\pgfsetfillcolor{currentfill}%
\pgfsetlinewidth{0.803000pt}%
\definecolor{currentstroke}{rgb}{0.000000,0.000000,0.000000}%
\pgfsetstrokecolor{currentstroke}%
\pgfsetdash{}{0pt}%
\pgfsys@defobject{currentmarker}{\pgfqpoint{-0.048611in}{0.000000in}}{\pgfqpoint{0.000000in}{0.000000in}}{%
\pgfpathmoveto{\pgfqpoint{0.000000in}{0.000000in}}%
\pgfpathlineto{\pgfqpoint{-0.048611in}{0.000000in}}%
\pgfusepath{stroke,fill}%
}%
\begin{pgfscope}%
\pgfsys@transformshift{0.557222in}{5.492231in}%
\pgfsys@useobject{currentmarker}{}%
\end{pgfscope}%
\end{pgfscope}%
\begin{pgfscope}%
\definecolor{textcolor}{rgb}{0.000000,0.000000,0.000000}%
\pgfsetstrokecolor{textcolor}%
\pgfsetfillcolor{textcolor}%
\pgftext[x=0.150755in,y=5.439470in,left,base]{\color{textcolor}\sffamily\fontsize{10.000000}{12.000000}\selectfont 1.80}%
\end{pgfscope}%
\begin{pgfscope}%
\pgfpathrectangle{\pgfqpoint{0.557222in}{3.312222in}}{\pgfqpoint{3.321333in}{2.324444in}}%
\pgfusepath{clip}%
\pgfsetrectcap%
\pgfsetroundjoin%
\pgfsetlinewidth{1.505625pt}%
\definecolor{currentstroke}{rgb}{0.121569,0.466667,0.705882}%
\pgfsetstrokecolor{currentstroke}%
\pgfsetdash{}{0pt}%
\pgfpathmoveto{\pgfqpoint{0.716480in}{3.545397in}}%
\pgfpathlineto{\pgfqpoint{0.866621in}{3.713785in}}%
\pgfpathlineto{\pgfqpoint{1.016761in}{3.904534in}}%
\pgfpathlineto{\pgfqpoint{1.166902in}{4.111134in}}%
\pgfpathlineto{\pgfqpoint{1.317043in}{4.305305in}}%
\pgfpathlineto{\pgfqpoint{1.467184in}{4.474895in}}%
\pgfpathlineto{\pgfqpoint{1.617325in}{4.638387in}}%
\pgfpathlineto{\pgfqpoint{1.767466in}{4.794043in}}%
\pgfpathlineto{\pgfqpoint{1.917607in}{4.962267in}}%
\pgfpathlineto{\pgfqpoint{2.067748in}{5.031150in}}%
\pgfpathlineto{\pgfqpoint{2.217889in}{5.158565in}}%
\pgfpathlineto{\pgfqpoint{2.368030in}{5.242109in}}%
\pgfpathlineto{\pgfqpoint{2.518171in}{5.294508in}}%
\pgfpathlineto{\pgfqpoint{2.668311in}{5.377193in}}%
\pgfpathlineto{\pgfqpoint{2.818452in}{5.466081in}}%
\pgfpathlineto{\pgfqpoint{2.968593in}{5.367072in}}%
\pgfpathlineto{\pgfqpoint{3.118734in}{5.334217in}}%
\pgfpathlineto{\pgfqpoint{3.268875in}{5.233616in}}%
\pgfpathlineto{\pgfqpoint{3.419016in}{5.149058in}}%
\pgfpathlineto{\pgfqpoint{3.569157in}{4.972595in}}%
\pgfpathlineto{\pgfqpoint{3.719298in}{4.818114in}}%
\pgfusepath{stroke}%
\end{pgfscope}%
\begin{pgfscope}%
\pgfsetrectcap%
\pgfsetmiterjoin%
\pgfsetlinewidth{0.803000pt}%
\definecolor{currentstroke}{rgb}{0.000000,0.000000,0.000000}%
\pgfsetstrokecolor{currentstroke}%
\pgfsetdash{}{0pt}%
\pgfpathmoveto{\pgfqpoint{0.557222in}{3.312222in}}%
\pgfpathlineto{\pgfqpoint{0.557222in}{5.636667in}}%
\pgfusepath{stroke}%
\end{pgfscope}%
\begin{pgfscope}%
\pgfsetrectcap%
\pgfsetmiterjoin%
\pgfsetlinewidth{0.803000pt}%
\definecolor{currentstroke}{rgb}{0.000000,0.000000,0.000000}%
\pgfsetstrokecolor{currentstroke}%
\pgfsetdash{}{0pt}%
\pgfpathmoveto{\pgfqpoint{3.878555in}{3.312222in}}%
\pgfpathlineto{\pgfqpoint{3.878555in}{5.636667in}}%
\pgfusepath{stroke}%
\end{pgfscope}%
\begin{pgfscope}%
\pgfsetrectcap%
\pgfsetmiterjoin%
\pgfsetlinewidth{0.803000pt}%
\definecolor{currentstroke}{rgb}{0.000000,0.000000,0.000000}%
\pgfsetstrokecolor{currentstroke}%
\pgfsetdash{}{0pt}%
\pgfpathmoveto{\pgfqpoint{0.557222in}{3.312222in}}%
\pgfpathlineto{\pgfqpoint{3.878555in}{3.312222in}}%
\pgfusepath{stroke}%
\end{pgfscope}%
\begin{pgfscope}%
\pgfsetrectcap%
\pgfsetmiterjoin%
\pgfsetlinewidth{0.803000pt}%
\definecolor{currentstroke}{rgb}{0.000000,0.000000,0.000000}%
\pgfsetstrokecolor{currentstroke}%
\pgfsetdash{}{0pt}%
\pgfpathmoveto{\pgfqpoint{0.557222in}{5.636667in}}%
\pgfpathlineto{\pgfqpoint{3.878555in}{5.636667in}}%
\pgfusepath{stroke}%
\end{pgfscope}%
\begin{pgfscope}%
\definecolor{textcolor}{rgb}{0.000000,0.000000,0.000000}%
\pgfsetstrokecolor{textcolor}%
\pgfsetfillcolor{textcolor}%
\pgftext[x=2.217889in,y=5.720000in,,base]{\color{textcolor}\sffamily\fontsize{12.000000}{14.400000}\selectfont \(\displaystyle  N = 8 \)}%
\end{pgfscope}%
\begin{pgfscope}%
\pgfsetbuttcap%
\pgfsetmiterjoin%
\definecolor{currentfill}{rgb}{1.000000,1.000000,1.000000}%
\pgfsetfillcolor{currentfill}%
\pgfsetlinewidth{0.000000pt}%
\definecolor{currentstroke}{rgb}{0.000000,0.000000,0.000000}%
\pgfsetstrokecolor{currentstroke}%
\pgfsetstrokeopacity{0.000000}%
\pgfsetdash{}{0pt}%
\pgfpathmoveto{\pgfqpoint{4.493667in}{3.312222in}}%
\pgfpathlineto{\pgfqpoint{7.815000in}{3.312222in}}%
\pgfpathlineto{\pgfqpoint{7.815000in}{5.636667in}}%
\pgfpathlineto{\pgfqpoint{4.493667in}{5.636667in}}%
\pgfpathclose%
\pgfusepath{fill}%
\end{pgfscope}%
\begin{pgfscope}%
\pgfpathrectangle{\pgfqpoint{4.493667in}{3.312222in}}{\pgfqpoint{3.321333in}{2.324444in}}%
\pgfusepath{clip}%
\pgfsetbuttcap%
\pgfsetroundjoin%
\definecolor{currentfill}{rgb}{0.121569,0.466667,0.705882}%
\pgfsetfillcolor{currentfill}%
\pgfsetfillopacity{0.300000}%
\pgfsetlinewidth{1.003750pt}%
\definecolor{currentstroke}{rgb}{0.121569,0.466667,0.705882}%
\pgfsetstrokecolor{currentstroke}%
\pgfsetstrokeopacity{0.300000}%
\pgfsetdash{}{0pt}%
\pgfpathmoveto{\pgfqpoint{4.652924in}{4.130337in}}%
\pgfpathlineto{\pgfqpoint{4.652924in}{3.831880in}}%
\pgfpathlineto{\pgfqpoint{4.803065in}{4.081670in}}%
\pgfpathlineto{\pgfqpoint{4.953206in}{4.185650in}}%
\pgfpathlineto{\pgfqpoint{5.103347in}{4.326550in}}%
\pgfpathlineto{\pgfqpoint{5.253488in}{4.412782in}}%
\pgfpathlineto{\pgfqpoint{5.403629in}{4.694028in}}%
\pgfpathlineto{\pgfqpoint{5.553770in}{4.672016in}}%
\pgfpathlineto{\pgfqpoint{5.703911in}{4.732707in}}%
\pgfpathlineto{\pgfqpoint{5.854052in}{5.004509in}}%
\pgfpathlineto{\pgfqpoint{6.004193in}{4.813395in}}%
\pgfpathlineto{\pgfqpoint{6.154333in}{5.188264in}}%
\pgfpathlineto{\pgfqpoint{6.304474in}{5.162685in}}%
\pgfpathlineto{\pgfqpoint{6.454615in}{5.217063in}}%
\pgfpathlineto{\pgfqpoint{6.604756in}{5.048375in}}%
\pgfpathlineto{\pgfqpoint{6.754897in}{4.969868in}}%
\pgfpathlineto{\pgfqpoint{6.905038in}{4.741271in}}%
\pgfpathlineto{\pgfqpoint{7.055179in}{4.656925in}}%
\pgfpathlineto{\pgfqpoint{7.205320in}{4.323229in}}%
\pgfpathlineto{\pgfqpoint{7.355461in}{4.067751in}}%
\pgfpathlineto{\pgfqpoint{7.505602in}{3.697392in}}%
\pgfpathlineto{\pgfqpoint{7.655743in}{3.417879in}}%
\pgfpathlineto{\pgfqpoint{7.655743in}{3.554916in}}%
\pgfpathlineto{\pgfqpoint{7.655743in}{3.554916in}}%
\pgfpathlineto{\pgfqpoint{7.505602in}{3.873930in}}%
\pgfpathlineto{\pgfqpoint{7.355461in}{4.202160in}}%
\pgfpathlineto{\pgfqpoint{7.205320in}{4.592441in}}%
\pgfpathlineto{\pgfqpoint{7.055179in}{4.722552in}}%
\pgfpathlineto{\pgfqpoint{6.905038in}{5.031308in}}%
\pgfpathlineto{\pgfqpoint{6.754897in}{5.140423in}}%
\pgfpathlineto{\pgfqpoint{6.604756in}{5.219895in}}%
\pgfpathlineto{\pgfqpoint{6.454615in}{5.531010in}}%
\pgfpathlineto{\pgfqpoint{6.304474in}{5.370828in}}%
\pgfpathlineto{\pgfqpoint{6.154333in}{5.424746in}}%
\pgfpathlineto{\pgfqpoint{6.004193in}{5.293037in}}%
\pgfpathlineto{\pgfqpoint{5.854052in}{5.328443in}}%
\pgfpathlineto{\pgfqpoint{5.703911in}{5.153756in}}%
\pgfpathlineto{\pgfqpoint{5.553770in}{4.987146in}}%
\pgfpathlineto{\pgfqpoint{5.403629in}{4.923502in}}%
\pgfpathlineto{\pgfqpoint{5.253488in}{4.593892in}}%
\pgfpathlineto{\pgfqpoint{5.103347in}{4.445594in}}%
\pgfpathlineto{\pgfqpoint{4.953206in}{4.698401in}}%
\pgfpathlineto{\pgfqpoint{4.803065in}{4.282898in}}%
\pgfpathlineto{\pgfqpoint{4.652924in}{4.130337in}}%
\pgfpathclose%
\pgfusepath{stroke,fill}%
\end{pgfscope}%
\begin{pgfscope}%
\pgfpathrectangle{\pgfqpoint{4.493667in}{3.312222in}}{\pgfqpoint{3.321333in}{2.324444in}}%
\pgfusepath{clip}%
\pgfsetbuttcap%
\pgfsetroundjoin%
\definecolor{currentfill}{rgb}{0.121569,0.466667,0.705882}%
\pgfsetfillcolor{currentfill}%
\pgfsetlinewidth{1.003750pt}%
\definecolor{currentstroke}{rgb}{0.121569,0.466667,0.705882}%
\pgfsetstrokecolor{currentstroke}%
\pgfsetdash{}{0pt}%
\pgfsys@defobject{currentmarker}{\pgfqpoint{-0.009821in}{-0.009821in}}{\pgfqpoint{0.009821in}{0.009821in}}{%
\pgfpathmoveto{\pgfqpoint{0.000000in}{-0.009821in}}%
\pgfpathcurveto{\pgfqpoint{0.002605in}{-0.009821in}}{\pgfqpoint{0.005103in}{-0.008786in}}{\pgfqpoint{0.006944in}{-0.006944in}}%
\pgfpathcurveto{\pgfqpoint{0.008786in}{-0.005103in}}{\pgfqpoint{0.009821in}{-0.002605in}}{\pgfqpoint{0.009821in}{0.000000in}}%
\pgfpathcurveto{\pgfqpoint{0.009821in}{0.002605in}}{\pgfqpoint{0.008786in}{0.005103in}}{\pgfqpoint{0.006944in}{0.006944in}}%
\pgfpathcurveto{\pgfqpoint{0.005103in}{0.008786in}}{\pgfqpoint{0.002605in}{0.009821in}}{\pgfqpoint{0.000000in}{0.009821in}}%
\pgfpathcurveto{\pgfqpoint{-0.002605in}{0.009821in}}{\pgfqpoint{-0.005103in}{0.008786in}}{\pgfqpoint{-0.006944in}{0.006944in}}%
\pgfpathcurveto{\pgfqpoint{-0.008786in}{0.005103in}}{\pgfqpoint{-0.009821in}{0.002605in}}{\pgfqpoint{-0.009821in}{0.000000in}}%
\pgfpathcurveto{\pgfqpoint{-0.009821in}{-0.002605in}}{\pgfqpoint{-0.008786in}{-0.005103in}}{\pgfqpoint{-0.006944in}{-0.006944in}}%
\pgfpathcurveto{\pgfqpoint{-0.005103in}{-0.008786in}}{\pgfqpoint{-0.002605in}{-0.009821in}}{\pgfqpoint{0.000000in}{-0.009821in}}%
\pgfpathclose%
\pgfusepath{stroke,fill}%
}%
\begin{pgfscope}%
\pgfsys@transformshift{4.652924in}{3.981108in}%
\pgfsys@useobject{currentmarker}{}%
\end{pgfscope}%
\begin{pgfscope}%
\pgfsys@transformshift{4.803065in}{4.182284in}%
\pgfsys@useobject{currentmarker}{}%
\end{pgfscope}%
\begin{pgfscope}%
\pgfsys@transformshift{4.953206in}{4.442026in}%
\pgfsys@useobject{currentmarker}{}%
\end{pgfscope}%
\begin{pgfscope}%
\pgfsys@transformshift{5.103347in}{4.386072in}%
\pgfsys@useobject{currentmarker}{}%
\end{pgfscope}%
\begin{pgfscope}%
\pgfsys@transformshift{5.253488in}{4.503337in}%
\pgfsys@useobject{currentmarker}{}%
\end{pgfscope}%
\begin{pgfscope}%
\pgfsys@transformshift{5.403629in}{4.808765in}%
\pgfsys@useobject{currentmarker}{}%
\end{pgfscope}%
\begin{pgfscope}%
\pgfsys@transformshift{5.553770in}{4.829581in}%
\pgfsys@useobject{currentmarker}{}%
\end{pgfscope}%
\begin{pgfscope}%
\pgfsys@transformshift{5.703911in}{4.943231in}%
\pgfsys@useobject{currentmarker}{}%
\end{pgfscope}%
\begin{pgfscope}%
\pgfsys@transformshift{5.854052in}{5.166476in}%
\pgfsys@useobject{currentmarker}{}%
\end{pgfscope}%
\begin{pgfscope}%
\pgfsys@transformshift{6.004193in}{5.053216in}%
\pgfsys@useobject{currentmarker}{}%
\end{pgfscope}%
\begin{pgfscope}%
\pgfsys@transformshift{6.154333in}{5.306505in}%
\pgfsys@useobject{currentmarker}{}%
\end{pgfscope}%
\begin{pgfscope}%
\pgfsys@transformshift{6.304474in}{5.266757in}%
\pgfsys@useobject{currentmarker}{}%
\end{pgfscope}%
\begin{pgfscope}%
\pgfsys@transformshift{6.454615in}{5.374036in}%
\pgfsys@useobject{currentmarker}{}%
\end{pgfscope}%
\begin{pgfscope}%
\pgfsys@transformshift{6.604756in}{5.134135in}%
\pgfsys@useobject{currentmarker}{}%
\end{pgfscope}%
\begin{pgfscope}%
\pgfsys@transformshift{6.754897in}{5.055146in}%
\pgfsys@useobject{currentmarker}{}%
\end{pgfscope}%
\begin{pgfscope}%
\pgfsys@transformshift{6.905038in}{4.886289in}%
\pgfsys@useobject{currentmarker}{}%
\end{pgfscope}%
\begin{pgfscope}%
\pgfsys@transformshift{7.055179in}{4.689739in}%
\pgfsys@useobject{currentmarker}{}%
\end{pgfscope}%
\begin{pgfscope}%
\pgfsys@transformshift{7.205320in}{4.457835in}%
\pgfsys@useobject{currentmarker}{}%
\end{pgfscope}%
\begin{pgfscope}%
\pgfsys@transformshift{7.355461in}{4.134956in}%
\pgfsys@useobject{currentmarker}{}%
\end{pgfscope}%
\begin{pgfscope}%
\pgfsys@transformshift{7.505602in}{3.785661in}%
\pgfsys@useobject{currentmarker}{}%
\end{pgfscope}%
\begin{pgfscope}%
\pgfsys@transformshift{7.655743in}{3.486397in}%
\pgfsys@useobject{currentmarker}{}%
\end{pgfscope}%
\end{pgfscope}%
\begin{pgfscope}%
\pgfsetbuttcap%
\pgfsetroundjoin%
\definecolor{currentfill}{rgb}{0.000000,0.000000,0.000000}%
\pgfsetfillcolor{currentfill}%
\pgfsetlinewidth{0.803000pt}%
\definecolor{currentstroke}{rgb}{0.000000,0.000000,0.000000}%
\pgfsetstrokecolor{currentstroke}%
\pgfsetdash{}{0pt}%
\pgfsys@defobject{currentmarker}{\pgfqpoint{0.000000in}{-0.048611in}}{\pgfqpoint{0.000000in}{0.000000in}}{%
\pgfpathmoveto{\pgfqpoint{0.000000in}{0.000000in}}%
\pgfpathlineto{\pgfqpoint{0.000000in}{-0.048611in}}%
\pgfusepath{stroke,fill}%
}%
\begin{pgfscope}%
\pgfsys@transformshift{4.502784in}{3.312222in}%
\pgfsys@useobject{currentmarker}{}%
\end{pgfscope}%
\end{pgfscope}%
\begin{pgfscope}%
\definecolor{textcolor}{rgb}{0.000000,0.000000,0.000000}%
\pgfsetstrokecolor{textcolor}%
\pgfsetfillcolor{textcolor}%
\pgftext[x=4.502784in,y=3.215000in,,top]{\color{textcolor}\sffamily\fontsize{10.000000}{12.000000}\selectfont 4.30}%
\end{pgfscope}%
\begin{pgfscope}%
\pgfsetbuttcap%
\pgfsetroundjoin%
\definecolor{currentfill}{rgb}{0.000000,0.000000,0.000000}%
\pgfsetfillcolor{currentfill}%
\pgfsetlinewidth{0.803000pt}%
\definecolor{currentstroke}{rgb}{0.000000,0.000000,0.000000}%
\pgfsetstrokecolor{currentstroke}%
\pgfsetdash{}{0pt}%
\pgfsys@defobject{currentmarker}{\pgfqpoint{0.000000in}{-0.048611in}}{\pgfqpoint{0.000000in}{0.000000in}}{%
\pgfpathmoveto{\pgfqpoint{0.000000in}{0.000000in}}%
\pgfpathlineto{\pgfqpoint{0.000000in}{-0.048611in}}%
\pgfusepath{stroke,fill}%
}%
\begin{pgfscope}%
\pgfsys@transformshift{5.253488in}{3.312222in}%
\pgfsys@useobject{currentmarker}{}%
\end{pgfscope}%
\end{pgfscope}%
\begin{pgfscope}%
\definecolor{textcolor}{rgb}{0.000000,0.000000,0.000000}%
\pgfsetstrokecolor{textcolor}%
\pgfsetfillcolor{textcolor}%
\pgftext[x=5.253488in,y=3.215000in,,top]{\color{textcolor}\sffamily\fontsize{10.000000}{12.000000}\selectfont 4.35}%
\end{pgfscope}%
\begin{pgfscope}%
\pgfsetbuttcap%
\pgfsetroundjoin%
\definecolor{currentfill}{rgb}{0.000000,0.000000,0.000000}%
\pgfsetfillcolor{currentfill}%
\pgfsetlinewidth{0.803000pt}%
\definecolor{currentstroke}{rgb}{0.000000,0.000000,0.000000}%
\pgfsetstrokecolor{currentstroke}%
\pgfsetdash{}{0pt}%
\pgfsys@defobject{currentmarker}{\pgfqpoint{0.000000in}{-0.048611in}}{\pgfqpoint{0.000000in}{0.000000in}}{%
\pgfpathmoveto{\pgfqpoint{0.000000in}{0.000000in}}%
\pgfpathlineto{\pgfqpoint{0.000000in}{-0.048611in}}%
\pgfusepath{stroke,fill}%
}%
\begin{pgfscope}%
\pgfsys@transformshift{6.004193in}{3.312222in}%
\pgfsys@useobject{currentmarker}{}%
\end{pgfscope}%
\end{pgfscope}%
\begin{pgfscope}%
\definecolor{textcolor}{rgb}{0.000000,0.000000,0.000000}%
\pgfsetstrokecolor{textcolor}%
\pgfsetfillcolor{textcolor}%
\pgftext[x=6.004193in,y=3.215000in,,top]{\color{textcolor}\sffamily\fontsize{10.000000}{12.000000}\selectfont 4.40}%
\end{pgfscope}%
\begin{pgfscope}%
\pgfsetbuttcap%
\pgfsetroundjoin%
\definecolor{currentfill}{rgb}{0.000000,0.000000,0.000000}%
\pgfsetfillcolor{currentfill}%
\pgfsetlinewidth{0.803000pt}%
\definecolor{currentstroke}{rgb}{0.000000,0.000000,0.000000}%
\pgfsetstrokecolor{currentstroke}%
\pgfsetdash{}{0pt}%
\pgfsys@defobject{currentmarker}{\pgfqpoint{0.000000in}{-0.048611in}}{\pgfqpoint{0.000000in}{0.000000in}}{%
\pgfpathmoveto{\pgfqpoint{0.000000in}{0.000000in}}%
\pgfpathlineto{\pgfqpoint{0.000000in}{-0.048611in}}%
\pgfusepath{stroke,fill}%
}%
\begin{pgfscope}%
\pgfsys@transformshift{6.754897in}{3.312222in}%
\pgfsys@useobject{currentmarker}{}%
\end{pgfscope}%
\end{pgfscope}%
\begin{pgfscope}%
\definecolor{textcolor}{rgb}{0.000000,0.000000,0.000000}%
\pgfsetstrokecolor{textcolor}%
\pgfsetfillcolor{textcolor}%
\pgftext[x=6.754897in,y=3.215000in,,top]{\color{textcolor}\sffamily\fontsize{10.000000}{12.000000}\selectfont 4.45}%
\end{pgfscope}%
\begin{pgfscope}%
\pgfsetbuttcap%
\pgfsetroundjoin%
\definecolor{currentfill}{rgb}{0.000000,0.000000,0.000000}%
\pgfsetfillcolor{currentfill}%
\pgfsetlinewidth{0.803000pt}%
\definecolor{currentstroke}{rgb}{0.000000,0.000000,0.000000}%
\pgfsetstrokecolor{currentstroke}%
\pgfsetdash{}{0pt}%
\pgfsys@defobject{currentmarker}{\pgfqpoint{0.000000in}{-0.048611in}}{\pgfqpoint{0.000000in}{0.000000in}}{%
\pgfpathmoveto{\pgfqpoint{0.000000in}{0.000000in}}%
\pgfpathlineto{\pgfqpoint{0.000000in}{-0.048611in}}%
\pgfusepath{stroke,fill}%
}%
\begin{pgfscope}%
\pgfsys@transformshift{7.505602in}{3.312222in}%
\pgfsys@useobject{currentmarker}{}%
\end{pgfscope}%
\end{pgfscope}%
\begin{pgfscope}%
\definecolor{textcolor}{rgb}{0.000000,0.000000,0.000000}%
\pgfsetstrokecolor{textcolor}%
\pgfsetfillcolor{textcolor}%
\pgftext[x=7.505602in,y=3.215000in,,top]{\color{textcolor}\sffamily\fontsize{10.000000}{12.000000}\selectfont 4.50}%
\end{pgfscope}%
\begin{pgfscope}%
\pgfsetbuttcap%
\pgfsetroundjoin%
\definecolor{currentfill}{rgb}{0.000000,0.000000,0.000000}%
\pgfsetfillcolor{currentfill}%
\pgfsetlinewidth{0.803000pt}%
\definecolor{currentstroke}{rgb}{0.000000,0.000000,0.000000}%
\pgfsetstrokecolor{currentstroke}%
\pgfsetdash{}{0pt}%
\pgfsys@defobject{currentmarker}{\pgfqpoint{-0.048611in}{0.000000in}}{\pgfqpoint{0.000000in}{0.000000in}}{%
\pgfpathmoveto{\pgfqpoint{0.000000in}{0.000000in}}%
\pgfpathlineto{\pgfqpoint{-0.048611in}{0.000000in}}%
\pgfusepath{stroke,fill}%
}%
\begin{pgfscope}%
\pgfsys@transformshift{4.493667in}{3.314775in}%
\pgfsys@useobject{currentmarker}{}%
\end{pgfscope}%
\end{pgfscope}%
\begin{pgfscope}%
\definecolor{textcolor}{rgb}{0.000000,0.000000,0.000000}%
\pgfsetstrokecolor{textcolor}%
\pgfsetfillcolor{textcolor}%
\pgftext[x=4.175565in,y=3.262014in,left,base]{\color{textcolor}\sffamily\fontsize{10.000000}{12.000000}\selectfont 1.6}%
\end{pgfscope}%
\begin{pgfscope}%
\pgfsetbuttcap%
\pgfsetroundjoin%
\definecolor{currentfill}{rgb}{0.000000,0.000000,0.000000}%
\pgfsetfillcolor{currentfill}%
\pgfsetlinewidth{0.803000pt}%
\definecolor{currentstroke}{rgb}{0.000000,0.000000,0.000000}%
\pgfsetstrokecolor{currentstroke}%
\pgfsetdash{}{0pt}%
\pgfsys@defobject{currentmarker}{\pgfqpoint{-0.048611in}{0.000000in}}{\pgfqpoint{0.000000in}{0.000000in}}{%
\pgfpathmoveto{\pgfqpoint{0.000000in}{0.000000in}}%
\pgfpathlineto{\pgfqpoint{-0.048611in}{0.000000in}}%
\pgfusepath{stroke,fill}%
}%
\begin{pgfscope}%
\pgfsys@transformshift{4.493667in}{3.693134in}%
\pgfsys@useobject{currentmarker}{}%
\end{pgfscope}%
\end{pgfscope}%
\begin{pgfscope}%
\definecolor{textcolor}{rgb}{0.000000,0.000000,0.000000}%
\pgfsetstrokecolor{textcolor}%
\pgfsetfillcolor{textcolor}%
\pgftext[x=4.175565in,y=3.640373in,left,base]{\color{textcolor}\sffamily\fontsize{10.000000}{12.000000}\selectfont 1.7}%
\end{pgfscope}%
\begin{pgfscope}%
\pgfsetbuttcap%
\pgfsetroundjoin%
\definecolor{currentfill}{rgb}{0.000000,0.000000,0.000000}%
\pgfsetfillcolor{currentfill}%
\pgfsetlinewidth{0.803000pt}%
\definecolor{currentstroke}{rgb}{0.000000,0.000000,0.000000}%
\pgfsetstrokecolor{currentstroke}%
\pgfsetdash{}{0pt}%
\pgfsys@defobject{currentmarker}{\pgfqpoint{-0.048611in}{0.000000in}}{\pgfqpoint{0.000000in}{0.000000in}}{%
\pgfpathmoveto{\pgfqpoint{0.000000in}{0.000000in}}%
\pgfpathlineto{\pgfqpoint{-0.048611in}{0.000000in}}%
\pgfusepath{stroke,fill}%
}%
\begin{pgfscope}%
\pgfsys@transformshift{4.493667in}{4.071494in}%
\pgfsys@useobject{currentmarker}{}%
\end{pgfscope}%
\end{pgfscope}%
\begin{pgfscope}%
\definecolor{textcolor}{rgb}{0.000000,0.000000,0.000000}%
\pgfsetstrokecolor{textcolor}%
\pgfsetfillcolor{textcolor}%
\pgftext[x=4.175565in,y=4.018732in,left,base]{\color{textcolor}\sffamily\fontsize{10.000000}{12.000000}\selectfont 1.8}%
\end{pgfscope}%
\begin{pgfscope}%
\pgfsetbuttcap%
\pgfsetroundjoin%
\definecolor{currentfill}{rgb}{0.000000,0.000000,0.000000}%
\pgfsetfillcolor{currentfill}%
\pgfsetlinewidth{0.803000pt}%
\definecolor{currentstroke}{rgb}{0.000000,0.000000,0.000000}%
\pgfsetstrokecolor{currentstroke}%
\pgfsetdash{}{0pt}%
\pgfsys@defobject{currentmarker}{\pgfqpoint{-0.048611in}{0.000000in}}{\pgfqpoint{0.000000in}{0.000000in}}{%
\pgfpathmoveto{\pgfqpoint{0.000000in}{0.000000in}}%
\pgfpathlineto{\pgfqpoint{-0.048611in}{0.000000in}}%
\pgfusepath{stroke,fill}%
}%
\begin{pgfscope}%
\pgfsys@transformshift{4.493667in}{4.449853in}%
\pgfsys@useobject{currentmarker}{}%
\end{pgfscope}%
\end{pgfscope}%
\begin{pgfscope}%
\definecolor{textcolor}{rgb}{0.000000,0.000000,0.000000}%
\pgfsetstrokecolor{textcolor}%
\pgfsetfillcolor{textcolor}%
\pgftext[x=4.175565in,y=4.397091in,left,base]{\color{textcolor}\sffamily\fontsize{10.000000}{12.000000}\selectfont 1.9}%
\end{pgfscope}%
\begin{pgfscope}%
\pgfsetbuttcap%
\pgfsetroundjoin%
\definecolor{currentfill}{rgb}{0.000000,0.000000,0.000000}%
\pgfsetfillcolor{currentfill}%
\pgfsetlinewidth{0.803000pt}%
\definecolor{currentstroke}{rgb}{0.000000,0.000000,0.000000}%
\pgfsetstrokecolor{currentstroke}%
\pgfsetdash{}{0pt}%
\pgfsys@defobject{currentmarker}{\pgfqpoint{-0.048611in}{0.000000in}}{\pgfqpoint{0.000000in}{0.000000in}}{%
\pgfpathmoveto{\pgfqpoint{0.000000in}{0.000000in}}%
\pgfpathlineto{\pgfqpoint{-0.048611in}{0.000000in}}%
\pgfusepath{stroke,fill}%
}%
\begin{pgfscope}%
\pgfsys@transformshift{4.493667in}{4.828212in}%
\pgfsys@useobject{currentmarker}{}%
\end{pgfscope}%
\end{pgfscope}%
\begin{pgfscope}%
\definecolor{textcolor}{rgb}{0.000000,0.000000,0.000000}%
\pgfsetstrokecolor{textcolor}%
\pgfsetfillcolor{textcolor}%
\pgftext[x=4.175565in,y=4.775451in,left,base]{\color{textcolor}\sffamily\fontsize{10.000000}{12.000000}\selectfont 2.0}%
\end{pgfscope}%
\begin{pgfscope}%
\pgfsetbuttcap%
\pgfsetroundjoin%
\definecolor{currentfill}{rgb}{0.000000,0.000000,0.000000}%
\pgfsetfillcolor{currentfill}%
\pgfsetlinewidth{0.803000pt}%
\definecolor{currentstroke}{rgb}{0.000000,0.000000,0.000000}%
\pgfsetstrokecolor{currentstroke}%
\pgfsetdash{}{0pt}%
\pgfsys@defobject{currentmarker}{\pgfqpoint{-0.048611in}{0.000000in}}{\pgfqpoint{0.000000in}{0.000000in}}{%
\pgfpathmoveto{\pgfqpoint{0.000000in}{0.000000in}}%
\pgfpathlineto{\pgfqpoint{-0.048611in}{0.000000in}}%
\pgfusepath{stroke,fill}%
}%
\begin{pgfscope}%
\pgfsys@transformshift{4.493667in}{5.206571in}%
\pgfsys@useobject{currentmarker}{}%
\end{pgfscope}%
\end{pgfscope}%
\begin{pgfscope}%
\definecolor{textcolor}{rgb}{0.000000,0.000000,0.000000}%
\pgfsetstrokecolor{textcolor}%
\pgfsetfillcolor{textcolor}%
\pgftext[x=4.175565in,y=5.153810in,left,base]{\color{textcolor}\sffamily\fontsize{10.000000}{12.000000}\selectfont 2.1}%
\end{pgfscope}%
\begin{pgfscope}%
\pgfsetbuttcap%
\pgfsetroundjoin%
\definecolor{currentfill}{rgb}{0.000000,0.000000,0.000000}%
\pgfsetfillcolor{currentfill}%
\pgfsetlinewidth{0.803000pt}%
\definecolor{currentstroke}{rgb}{0.000000,0.000000,0.000000}%
\pgfsetstrokecolor{currentstroke}%
\pgfsetdash{}{0pt}%
\pgfsys@defobject{currentmarker}{\pgfqpoint{-0.048611in}{0.000000in}}{\pgfqpoint{0.000000in}{0.000000in}}{%
\pgfpathmoveto{\pgfqpoint{0.000000in}{0.000000in}}%
\pgfpathlineto{\pgfqpoint{-0.048611in}{0.000000in}}%
\pgfusepath{stroke,fill}%
}%
\begin{pgfscope}%
\pgfsys@transformshift{4.493667in}{5.584931in}%
\pgfsys@useobject{currentmarker}{}%
\end{pgfscope}%
\end{pgfscope}%
\begin{pgfscope}%
\definecolor{textcolor}{rgb}{0.000000,0.000000,0.000000}%
\pgfsetstrokecolor{textcolor}%
\pgfsetfillcolor{textcolor}%
\pgftext[x=4.175565in,y=5.532169in,left,base]{\color{textcolor}\sffamily\fontsize{10.000000}{12.000000}\selectfont 2.2}%
\end{pgfscope}%
\begin{pgfscope}%
\pgfpathrectangle{\pgfqpoint{4.493667in}{3.312222in}}{\pgfqpoint{3.321333in}{2.324444in}}%
\pgfusepath{clip}%
\pgfsetrectcap%
\pgfsetroundjoin%
\pgfsetlinewidth{1.505625pt}%
\definecolor{currentstroke}{rgb}{0.121569,0.466667,0.705882}%
\pgfsetstrokecolor{currentstroke}%
\pgfsetdash{}{0pt}%
\pgfpathmoveto{\pgfqpoint{4.652924in}{3.981108in}}%
\pgfpathlineto{\pgfqpoint{4.803065in}{4.182284in}}%
\pgfpathlineto{\pgfqpoint{4.953206in}{4.442026in}}%
\pgfpathlineto{\pgfqpoint{5.103347in}{4.386072in}}%
\pgfpathlineto{\pgfqpoint{5.253488in}{4.503337in}}%
\pgfpathlineto{\pgfqpoint{5.403629in}{4.808765in}}%
\pgfpathlineto{\pgfqpoint{5.553770in}{4.829581in}}%
\pgfpathlineto{\pgfqpoint{5.703911in}{4.943231in}}%
\pgfpathlineto{\pgfqpoint{5.854052in}{5.166476in}}%
\pgfpathlineto{\pgfqpoint{6.004193in}{5.053216in}}%
\pgfpathlineto{\pgfqpoint{6.154333in}{5.306505in}}%
\pgfpathlineto{\pgfqpoint{6.304474in}{5.266757in}}%
\pgfpathlineto{\pgfqpoint{6.454615in}{5.374036in}}%
\pgfpathlineto{\pgfqpoint{6.604756in}{5.134135in}}%
\pgfpathlineto{\pgfqpoint{6.754897in}{5.055146in}}%
\pgfpathlineto{\pgfqpoint{6.905038in}{4.886289in}}%
\pgfpathlineto{\pgfqpoint{7.055179in}{4.689739in}}%
\pgfpathlineto{\pgfqpoint{7.205320in}{4.457835in}}%
\pgfpathlineto{\pgfqpoint{7.355461in}{4.134956in}}%
\pgfpathlineto{\pgfqpoint{7.505602in}{3.785661in}}%
\pgfpathlineto{\pgfqpoint{7.655743in}{3.486397in}}%
\pgfusepath{stroke}%
\end{pgfscope}%
\begin{pgfscope}%
\pgfsetrectcap%
\pgfsetmiterjoin%
\pgfsetlinewidth{0.803000pt}%
\definecolor{currentstroke}{rgb}{0.000000,0.000000,0.000000}%
\pgfsetstrokecolor{currentstroke}%
\pgfsetdash{}{0pt}%
\pgfpathmoveto{\pgfqpoint{4.493667in}{3.312222in}}%
\pgfpathlineto{\pgfqpoint{4.493667in}{5.636667in}}%
\pgfusepath{stroke}%
\end{pgfscope}%
\begin{pgfscope}%
\pgfsetrectcap%
\pgfsetmiterjoin%
\pgfsetlinewidth{0.803000pt}%
\definecolor{currentstroke}{rgb}{0.000000,0.000000,0.000000}%
\pgfsetstrokecolor{currentstroke}%
\pgfsetdash{}{0pt}%
\pgfpathmoveto{\pgfqpoint{7.815000in}{3.312222in}}%
\pgfpathlineto{\pgfqpoint{7.815000in}{5.636667in}}%
\pgfusepath{stroke}%
\end{pgfscope}%
\begin{pgfscope}%
\pgfsetrectcap%
\pgfsetmiterjoin%
\pgfsetlinewidth{0.803000pt}%
\definecolor{currentstroke}{rgb}{0.000000,0.000000,0.000000}%
\pgfsetstrokecolor{currentstroke}%
\pgfsetdash{}{0pt}%
\pgfpathmoveto{\pgfqpoint{4.493667in}{3.312222in}}%
\pgfpathlineto{\pgfqpoint{7.815000in}{3.312222in}}%
\pgfusepath{stroke}%
\end{pgfscope}%
\begin{pgfscope}%
\pgfsetrectcap%
\pgfsetmiterjoin%
\pgfsetlinewidth{0.803000pt}%
\definecolor{currentstroke}{rgb}{0.000000,0.000000,0.000000}%
\pgfsetstrokecolor{currentstroke}%
\pgfsetdash{}{0pt}%
\pgfpathmoveto{\pgfqpoint{4.493667in}{5.636667in}}%
\pgfpathlineto{\pgfqpoint{7.815000in}{5.636667in}}%
\pgfusepath{stroke}%
\end{pgfscope}%
\begin{pgfscope}%
\definecolor{textcolor}{rgb}{0.000000,0.000000,0.000000}%
\pgfsetstrokecolor{textcolor}%
\pgfsetfillcolor{textcolor}%
\pgftext[x=6.154333in,y=5.720000in,,base]{\color{textcolor}\sffamily\fontsize{12.000000}{14.400000}\selectfont \(\displaystyle  N = 12 \)}%
\end{pgfscope}%
\begin{pgfscope}%
\pgfsetbuttcap%
\pgfsetmiterjoin%
\definecolor{currentfill}{rgb}{1.000000,1.000000,1.000000}%
\pgfsetfillcolor{currentfill}%
\pgfsetlinewidth{0.000000pt}%
\definecolor{currentstroke}{rgb}{0.000000,0.000000,0.000000}%
\pgfsetstrokecolor{currentstroke}%
\pgfsetstrokeopacity{0.000000}%
\pgfsetdash{}{0pt}%
\pgfpathmoveto{\pgfqpoint{0.557222in}{0.387222in}}%
\pgfpathlineto{\pgfqpoint{3.878555in}{0.387222in}}%
\pgfpathlineto{\pgfqpoint{3.878555in}{2.711667in}}%
\pgfpathlineto{\pgfqpoint{0.557222in}{2.711667in}}%
\pgfpathclose%
\pgfusepath{fill}%
\end{pgfscope}%
\begin{pgfscope}%
\pgfpathrectangle{\pgfqpoint{0.557222in}{0.387222in}}{\pgfqpoint{3.321333in}{2.324444in}}%
\pgfusepath{clip}%
\pgfsetbuttcap%
\pgfsetroundjoin%
\definecolor{currentfill}{rgb}{0.121569,0.466667,0.705882}%
\pgfsetfillcolor{currentfill}%
\pgfsetfillopacity{0.300000}%
\pgfsetlinewidth{1.003750pt}%
\definecolor{currentstroke}{rgb}{0.121569,0.466667,0.705882}%
\pgfsetstrokecolor{currentstroke}%
\pgfsetstrokeopacity{0.300000}%
\pgfsetdash{}{0pt}%
\pgfpathmoveto{\pgfqpoint{0.716480in}{1.646430in}}%
\pgfpathlineto{\pgfqpoint{0.716480in}{1.438942in}}%
\pgfpathlineto{\pgfqpoint{0.866621in}{1.501684in}}%
\pgfpathlineto{\pgfqpoint{1.016761in}{1.495716in}}%
\pgfpathlineto{\pgfqpoint{1.166902in}{1.652535in}}%
\pgfpathlineto{\pgfqpoint{1.317043in}{1.745961in}}%
\pgfpathlineto{\pgfqpoint{1.467184in}{1.917216in}}%
\pgfpathlineto{\pgfqpoint{1.617325in}{2.111955in}}%
\pgfpathlineto{\pgfqpoint{1.767466in}{1.903064in}}%
\pgfpathlineto{\pgfqpoint{1.917607in}{2.096764in}}%
\pgfpathlineto{\pgfqpoint{2.067748in}{2.093664in}}%
\pgfpathlineto{\pgfqpoint{2.217889in}{2.352322in}}%
\pgfpathlineto{\pgfqpoint{2.368030in}{2.325127in}}%
\pgfpathlineto{\pgfqpoint{2.518171in}{2.360365in}}%
\pgfpathlineto{\pgfqpoint{2.668311in}{2.245398in}}%
\pgfpathlineto{\pgfqpoint{2.818452in}{2.035883in}}%
\pgfpathlineto{\pgfqpoint{2.968593in}{1.897238in}}%
\pgfpathlineto{\pgfqpoint{3.118734in}{1.675899in}}%
\pgfpathlineto{\pgfqpoint{3.268875in}{1.409605in}}%
\pgfpathlineto{\pgfqpoint{3.419016in}{1.028249in}}%
\pgfpathlineto{\pgfqpoint{3.569157in}{0.644246in}}%
\pgfpathlineto{\pgfqpoint{3.719298in}{0.492879in}}%
\pgfpathlineto{\pgfqpoint{3.719298in}{0.718029in}}%
\pgfpathlineto{\pgfqpoint{3.719298in}{0.718029in}}%
\pgfpathlineto{\pgfqpoint{3.569157in}{1.002063in}}%
\pgfpathlineto{\pgfqpoint{3.419016in}{1.363585in}}%
\pgfpathlineto{\pgfqpoint{3.268875in}{1.553831in}}%
\pgfpathlineto{\pgfqpoint{3.118734in}{1.886544in}}%
\pgfpathlineto{\pgfqpoint{2.968593in}{2.126970in}}%
\pgfpathlineto{\pgfqpoint{2.818452in}{2.340972in}}%
\pgfpathlineto{\pgfqpoint{2.668311in}{2.506038in}}%
\pgfpathlineto{\pgfqpoint{2.518171in}{2.549044in}}%
\pgfpathlineto{\pgfqpoint{2.368030in}{2.606010in}}%
\pgfpathlineto{\pgfqpoint{2.217889in}{2.531533in}}%
\pgfpathlineto{\pgfqpoint{2.067748in}{2.505356in}}%
\pgfpathlineto{\pgfqpoint{1.917607in}{2.297037in}}%
\pgfpathlineto{\pgfqpoint{1.767466in}{2.184593in}}%
\pgfpathlineto{\pgfqpoint{1.617325in}{2.265799in}}%
\pgfpathlineto{\pgfqpoint{1.467184in}{2.093140in}}%
\pgfpathlineto{\pgfqpoint{1.317043in}{1.994826in}}%
\pgfpathlineto{\pgfqpoint{1.166902in}{1.767629in}}%
\pgfpathlineto{\pgfqpoint{1.016761in}{1.790397in}}%
\pgfpathlineto{\pgfqpoint{0.866621in}{1.773855in}}%
\pgfpathlineto{\pgfqpoint{0.716480in}{1.646430in}}%
\pgfpathclose%
\pgfusepath{stroke,fill}%
\end{pgfscope}%
\begin{pgfscope}%
\pgfpathrectangle{\pgfqpoint{0.557222in}{0.387222in}}{\pgfqpoint{3.321333in}{2.324444in}}%
\pgfusepath{clip}%
\pgfsetbuttcap%
\pgfsetroundjoin%
\definecolor{currentfill}{rgb}{0.121569,0.466667,0.705882}%
\pgfsetfillcolor{currentfill}%
\pgfsetlinewidth{1.003750pt}%
\definecolor{currentstroke}{rgb}{0.121569,0.466667,0.705882}%
\pgfsetstrokecolor{currentstroke}%
\pgfsetdash{}{0pt}%
\pgfsys@defobject{currentmarker}{\pgfqpoint{-0.009821in}{-0.009821in}}{\pgfqpoint{0.009821in}{0.009821in}}{%
\pgfpathmoveto{\pgfqpoint{0.000000in}{-0.009821in}}%
\pgfpathcurveto{\pgfqpoint{0.002605in}{-0.009821in}}{\pgfqpoint{0.005103in}{-0.008786in}}{\pgfqpoint{0.006944in}{-0.006944in}}%
\pgfpathcurveto{\pgfqpoint{0.008786in}{-0.005103in}}{\pgfqpoint{0.009821in}{-0.002605in}}{\pgfqpoint{0.009821in}{0.000000in}}%
\pgfpathcurveto{\pgfqpoint{0.009821in}{0.002605in}}{\pgfqpoint{0.008786in}{0.005103in}}{\pgfqpoint{0.006944in}{0.006944in}}%
\pgfpathcurveto{\pgfqpoint{0.005103in}{0.008786in}}{\pgfqpoint{0.002605in}{0.009821in}}{\pgfqpoint{0.000000in}{0.009821in}}%
\pgfpathcurveto{\pgfqpoint{-0.002605in}{0.009821in}}{\pgfqpoint{-0.005103in}{0.008786in}}{\pgfqpoint{-0.006944in}{0.006944in}}%
\pgfpathcurveto{\pgfqpoint{-0.008786in}{0.005103in}}{\pgfqpoint{-0.009821in}{0.002605in}}{\pgfqpoint{-0.009821in}{0.000000in}}%
\pgfpathcurveto{\pgfqpoint{-0.009821in}{-0.002605in}}{\pgfqpoint{-0.008786in}{-0.005103in}}{\pgfqpoint{-0.006944in}{-0.006944in}}%
\pgfpathcurveto{\pgfqpoint{-0.005103in}{-0.008786in}}{\pgfqpoint{-0.002605in}{-0.009821in}}{\pgfqpoint{0.000000in}{-0.009821in}}%
\pgfpathclose%
\pgfusepath{stroke,fill}%
}%
\begin{pgfscope}%
\pgfsys@transformshift{0.716480in}{1.542686in}%
\pgfsys@useobject{currentmarker}{}%
\end{pgfscope}%
\begin{pgfscope}%
\pgfsys@transformshift{0.866621in}{1.637770in}%
\pgfsys@useobject{currentmarker}{}%
\end{pgfscope}%
\begin{pgfscope}%
\pgfsys@transformshift{1.016761in}{1.643057in}%
\pgfsys@useobject{currentmarker}{}%
\end{pgfscope}%
\begin{pgfscope}%
\pgfsys@transformshift{1.166902in}{1.710082in}%
\pgfsys@useobject{currentmarker}{}%
\end{pgfscope}%
\begin{pgfscope}%
\pgfsys@transformshift{1.317043in}{1.870393in}%
\pgfsys@useobject{currentmarker}{}%
\end{pgfscope}%
\begin{pgfscope}%
\pgfsys@transformshift{1.467184in}{2.005178in}%
\pgfsys@useobject{currentmarker}{}%
\end{pgfscope}%
\begin{pgfscope}%
\pgfsys@transformshift{1.617325in}{2.188877in}%
\pgfsys@useobject{currentmarker}{}%
\end{pgfscope}%
\begin{pgfscope}%
\pgfsys@transformshift{1.767466in}{2.043828in}%
\pgfsys@useobject{currentmarker}{}%
\end{pgfscope}%
\begin{pgfscope}%
\pgfsys@transformshift{1.917607in}{2.196901in}%
\pgfsys@useobject{currentmarker}{}%
\end{pgfscope}%
\begin{pgfscope}%
\pgfsys@transformshift{2.067748in}{2.299510in}%
\pgfsys@useobject{currentmarker}{}%
\end{pgfscope}%
\begin{pgfscope}%
\pgfsys@transformshift{2.217889in}{2.441928in}%
\pgfsys@useobject{currentmarker}{}%
\end{pgfscope}%
\begin{pgfscope}%
\pgfsys@transformshift{2.368030in}{2.465569in}%
\pgfsys@useobject{currentmarker}{}%
\end{pgfscope}%
\begin{pgfscope}%
\pgfsys@transformshift{2.518171in}{2.454705in}%
\pgfsys@useobject{currentmarker}{}%
\end{pgfscope}%
\begin{pgfscope}%
\pgfsys@transformshift{2.668311in}{2.375718in}%
\pgfsys@useobject{currentmarker}{}%
\end{pgfscope}%
\begin{pgfscope}%
\pgfsys@transformshift{2.818452in}{2.188428in}%
\pgfsys@useobject{currentmarker}{}%
\end{pgfscope}%
\begin{pgfscope}%
\pgfsys@transformshift{2.968593in}{2.012104in}%
\pgfsys@useobject{currentmarker}{}%
\end{pgfscope}%
\begin{pgfscope}%
\pgfsys@transformshift{3.118734in}{1.781221in}%
\pgfsys@useobject{currentmarker}{}%
\end{pgfscope}%
\begin{pgfscope}%
\pgfsys@transformshift{3.268875in}{1.481718in}%
\pgfsys@useobject{currentmarker}{}%
\end{pgfscope}%
\begin{pgfscope}%
\pgfsys@transformshift{3.419016in}{1.195917in}%
\pgfsys@useobject{currentmarker}{}%
\end{pgfscope}%
\begin{pgfscope}%
\pgfsys@transformshift{3.569157in}{0.823155in}%
\pgfsys@useobject{currentmarker}{}%
\end{pgfscope}%
\begin{pgfscope}%
\pgfsys@transformshift{3.719298in}{0.605454in}%
\pgfsys@useobject{currentmarker}{}%
\end{pgfscope}%
\end{pgfscope}%
\begin{pgfscope}%
\pgfsetbuttcap%
\pgfsetroundjoin%
\definecolor{currentfill}{rgb}{0.000000,0.000000,0.000000}%
\pgfsetfillcolor{currentfill}%
\pgfsetlinewidth{0.803000pt}%
\definecolor{currentstroke}{rgb}{0.000000,0.000000,0.000000}%
\pgfsetstrokecolor{currentstroke}%
\pgfsetdash{}{0pt}%
\pgfsys@defobject{currentmarker}{\pgfqpoint{0.000000in}{-0.048611in}}{\pgfqpoint{0.000000in}{0.000000in}}{%
\pgfpathmoveto{\pgfqpoint{0.000000in}{0.000000in}}%
\pgfpathlineto{\pgfqpoint{0.000000in}{-0.048611in}}%
\pgfusepath{stroke,fill}%
}%
\begin{pgfscope}%
\pgfsys@transformshift{0.866621in}{0.387222in}%
\pgfsys@useobject{currentmarker}{}%
\end{pgfscope}%
\end{pgfscope}%
\begin{pgfscope}%
\definecolor{textcolor}{rgb}{0.000000,0.000000,0.000000}%
\pgfsetstrokecolor{textcolor}%
\pgfsetfillcolor{textcolor}%
\pgftext[x=0.866621in,y=0.290000in,,top]{\color{textcolor}\sffamily\fontsize{10.000000}{12.000000}\selectfont 4.35}%
\end{pgfscope}%
\begin{pgfscope}%
\pgfsetbuttcap%
\pgfsetroundjoin%
\definecolor{currentfill}{rgb}{0.000000,0.000000,0.000000}%
\pgfsetfillcolor{currentfill}%
\pgfsetlinewidth{0.803000pt}%
\definecolor{currentstroke}{rgb}{0.000000,0.000000,0.000000}%
\pgfsetstrokecolor{currentstroke}%
\pgfsetdash{}{0pt}%
\pgfsys@defobject{currentmarker}{\pgfqpoint{0.000000in}{-0.048611in}}{\pgfqpoint{0.000000in}{0.000000in}}{%
\pgfpathmoveto{\pgfqpoint{0.000000in}{0.000000in}}%
\pgfpathlineto{\pgfqpoint{0.000000in}{-0.048611in}}%
\pgfusepath{stroke,fill}%
}%
\begin{pgfscope}%
\pgfsys@transformshift{1.617325in}{0.387222in}%
\pgfsys@useobject{currentmarker}{}%
\end{pgfscope}%
\end{pgfscope}%
\begin{pgfscope}%
\definecolor{textcolor}{rgb}{0.000000,0.000000,0.000000}%
\pgfsetstrokecolor{textcolor}%
\pgfsetfillcolor{textcolor}%
\pgftext[x=1.617325in,y=0.290000in,,top]{\color{textcolor}\sffamily\fontsize{10.000000}{12.000000}\selectfont 4.40}%
\end{pgfscope}%
\begin{pgfscope}%
\pgfsetbuttcap%
\pgfsetroundjoin%
\definecolor{currentfill}{rgb}{0.000000,0.000000,0.000000}%
\pgfsetfillcolor{currentfill}%
\pgfsetlinewidth{0.803000pt}%
\definecolor{currentstroke}{rgb}{0.000000,0.000000,0.000000}%
\pgfsetstrokecolor{currentstroke}%
\pgfsetdash{}{0pt}%
\pgfsys@defobject{currentmarker}{\pgfqpoint{0.000000in}{-0.048611in}}{\pgfqpoint{0.000000in}{0.000000in}}{%
\pgfpathmoveto{\pgfqpoint{0.000000in}{0.000000in}}%
\pgfpathlineto{\pgfqpoint{0.000000in}{-0.048611in}}%
\pgfusepath{stroke,fill}%
}%
\begin{pgfscope}%
\pgfsys@transformshift{2.368030in}{0.387222in}%
\pgfsys@useobject{currentmarker}{}%
\end{pgfscope}%
\end{pgfscope}%
\begin{pgfscope}%
\definecolor{textcolor}{rgb}{0.000000,0.000000,0.000000}%
\pgfsetstrokecolor{textcolor}%
\pgfsetfillcolor{textcolor}%
\pgftext[x=2.368030in,y=0.290000in,,top]{\color{textcolor}\sffamily\fontsize{10.000000}{12.000000}\selectfont 4.45}%
\end{pgfscope}%
\begin{pgfscope}%
\pgfsetbuttcap%
\pgfsetroundjoin%
\definecolor{currentfill}{rgb}{0.000000,0.000000,0.000000}%
\pgfsetfillcolor{currentfill}%
\pgfsetlinewidth{0.803000pt}%
\definecolor{currentstroke}{rgb}{0.000000,0.000000,0.000000}%
\pgfsetstrokecolor{currentstroke}%
\pgfsetdash{}{0pt}%
\pgfsys@defobject{currentmarker}{\pgfqpoint{0.000000in}{-0.048611in}}{\pgfqpoint{0.000000in}{0.000000in}}{%
\pgfpathmoveto{\pgfqpoint{0.000000in}{0.000000in}}%
\pgfpathlineto{\pgfqpoint{0.000000in}{-0.048611in}}%
\pgfusepath{stroke,fill}%
}%
\begin{pgfscope}%
\pgfsys@transformshift{3.118734in}{0.387222in}%
\pgfsys@useobject{currentmarker}{}%
\end{pgfscope}%
\end{pgfscope}%
\begin{pgfscope}%
\definecolor{textcolor}{rgb}{0.000000,0.000000,0.000000}%
\pgfsetstrokecolor{textcolor}%
\pgfsetfillcolor{textcolor}%
\pgftext[x=3.118734in,y=0.290000in,,top]{\color{textcolor}\sffamily\fontsize{10.000000}{12.000000}\selectfont 4.50}%
\end{pgfscope}%
\begin{pgfscope}%
\pgfsetbuttcap%
\pgfsetroundjoin%
\definecolor{currentfill}{rgb}{0.000000,0.000000,0.000000}%
\pgfsetfillcolor{currentfill}%
\pgfsetlinewidth{0.803000pt}%
\definecolor{currentstroke}{rgb}{0.000000,0.000000,0.000000}%
\pgfsetstrokecolor{currentstroke}%
\pgfsetdash{}{0pt}%
\pgfsys@defobject{currentmarker}{\pgfqpoint{0.000000in}{-0.048611in}}{\pgfqpoint{0.000000in}{0.000000in}}{%
\pgfpathmoveto{\pgfqpoint{0.000000in}{0.000000in}}%
\pgfpathlineto{\pgfqpoint{0.000000in}{-0.048611in}}%
\pgfusepath{stroke,fill}%
}%
\begin{pgfscope}%
\pgfsys@transformshift{3.869439in}{0.387222in}%
\pgfsys@useobject{currentmarker}{}%
\end{pgfscope}%
\end{pgfscope}%
\begin{pgfscope}%
\definecolor{textcolor}{rgb}{0.000000,0.000000,0.000000}%
\pgfsetstrokecolor{textcolor}%
\pgfsetfillcolor{textcolor}%
\pgftext[x=3.869439in,y=0.290000in,,top]{\color{textcolor}\sffamily\fontsize{10.000000}{12.000000}\selectfont 4.55}%
\end{pgfscope}%
\begin{pgfscope}%
\pgfsetbuttcap%
\pgfsetroundjoin%
\definecolor{currentfill}{rgb}{0.000000,0.000000,0.000000}%
\pgfsetfillcolor{currentfill}%
\pgfsetlinewidth{0.803000pt}%
\definecolor{currentstroke}{rgb}{0.000000,0.000000,0.000000}%
\pgfsetstrokecolor{currentstroke}%
\pgfsetdash{}{0pt}%
\pgfsys@defobject{currentmarker}{\pgfqpoint{-0.048611in}{0.000000in}}{\pgfqpoint{0.000000in}{0.000000in}}{%
\pgfpathmoveto{\pgfqpoint{0.000000in}{0.000000in}}%
\pgfpathlineto{\pgfqpoint{-0.048611in}{0.000000in}}%
\pgfusepath{stroke,fill}%
}%
\begin{pgfscope}%
\pgfsys@transformshift{0.557222in}{0.627205in}%
\pgfsys@useobject{currentmarker}{}%
\end{pgfscope}%
\end{pgfscope}%
\begin{pgfscope}%
\definecolor{textcolor}{rgb}{0.000000,0.000000,0.000000}%
\pgfsetstrokecolor{textcolor}%
\pgfsetfillcolor{textcolor}%
\pgftext[x=0.239121in,y=0.574443in,left,base]{\color{textcolor}\sffamily\fontsize{10.000000}{12.000000}\selectfont 1.4}%
\end{pgfscope}%
\begin{pgfscope}%
\pgfsetbuttcap%
\pgfsetroundjoin%
\definecolor{currentfill}{rgb}{0.000000,0.000000,0.000000}%
\pgfsetfillcolor{currentfill}%
\pgfsetlinewidth{0.803000pt}%
\definecolor{currentstroke}{rgb}{0.000000,0.000000,0.000000}%
\pgfsetstrokecolor{currentstroke}%
\pgfsetdash{}{0pt}%
\pgfsys@defobject{currentmarker}{\pgfqpoint{-0.048611in}{0.000000in}}{\pgfqpoint{0.000000in}{0.000000in}}{%
\pgfpathmoveto{\pgfqpoint{0.000000in}{0.000000in}}%
\pgfpathlineto{\pgfqpoint{-0.048611in}{0.000000in}}%
\pgfusepath{stroke,fill}%
}%
\begin{pgfscope}%
\pgfsys@transformshift{0.557222in}{1.020238in}%
\pgfsys@useobject{currentmarker}{}%
\end{pgfscope}%
\end{pgfscope}%
\begin{pgfscope}%
\definecolor{textcolor}{rgb}{0.000000,0.000000,0.000000}%
\pgfsetstrokecolor{textcolor}%
\pgfsetfillcolor{textcolor}%
\pgftext[x=0.239121in,y=0.967476in,left,base]{\color{textcolor}\sffamily\fontsize{10.000000}{12.000000}\selectfont 1.6}%
\end{pgfscope}%
\begin{pgfscope}%
\pgfsetbuttcap%
\pgfsetroundjoin%
\definecolor{currentfill}{rgb}{0.000000,0.000000,0.000000}%
\pgfsetfillcolor{currentfill}%
\pgfsetlinewidth{0.803000pt}%
\definecolor{currentstroke}{rgb}{0.000000,0.000000,0.000000}%
\pgfsetstrokecolor{currentstroke}%
\pgfsetdash{}{0pt}%
\pgfsys@defobject{currentmarker}{\pgfqpoint{-0.048611in}{0.000000in}}{\pgfqpoint{0.000000in}{0.000000in}}{%
\pgfpathmoveto{\pgfqpoint{0.000000in}{0.000000in}}%
\pgfpathlineto{\pgfqpoint{-0.048611in}{0.000000in}}%
\pgfusepath{stroke,fill}%
}%
\begin{pgfscope}%
\pgfsys@transformshift{0.557222in}{1.413270in}%
\pgfsys@useobject{currentmarker}{}%
\end{pgfscope}%
\end{pgfscope}%
\begin{pgfscope}%
\definecolor{textcolor}{rgb}{0.000000,0.000000,0.000000}%
\pgfsetstrokecolor{textcolor}%
\pgfsetfillcolor{textcolor}%
\pgftext[x=0.239121in,y=1.360509in,left,base]{\color{textcolor}\sffamily\fontsize{10.000000}{12.000000}\selectfont 1.8}%
\end{pgfscope}%
\begin{pgfscope}%
\pgfsetbuttcap%
\pgfsetroundjoin%
\definecolor{currentfill}{rgb}{0.000000,0.000000,0.000000}%
\pgfsetfillcolor{currentfill}%
\pgfsetlinewidth{0.803000pt}%
\definecolor{currentstroke}{rgb}{0.000000,0.000000,0.000000}%
\pgfsetstrokecolor{currentstroke}%
\pgfsetdash{}{0pt}%
\pgfsys@defobject{currentmarker}{\pgfqpoint{-0.048611in}{0.000000in}}{\pgfqpoint{0.000000in}{0.000000in}}{%
\pgfpathmoveto{\pgfqpoint{0.000000in}{0.000000in}}%
\pgfpathlineto{\pgfqpoint{-0.048611in}{0.000000in}}%
\pgfusepath{stroke,fill}%
}%
\begin{pgfscope}%
\pgfsys@transformshift{0.557222in}{1.806303in}%
\pgfsys@useobject{currentmarker}{}%
\end{pgfscope}%
\end{pgfscope}%
\begin{pgfscope}%
\definecolor{textcolor}{rgb}{0.000000,0.000000,0.000000}%
\pgfsetstrokecolor{textcolor}%
\pgfsetfillcolor{textcolor}%
\pgftext[x=0.239121in,y=1.753541in,left,base]{\color{textcolor}\sffamily\fontsize{10.000000}{12.000000}\selectfont 2.0}%
\end{pgfscope}%
\begin{pgfscope}%
\pgfsetbuttcap%
\pgfsetroundjoin%
\definecolor{currentfill}{rgb}{0.000000,0.000000,0.000000}%
\pgfsetfillcolor{currentfill}%
\pgfsetlinewidth{0.803000pt}%
\definecolor{currentstroke}{rgb}{0.000000,0.000000,0.000000}%
\pgfsetstrokecolor{currentstroke}%
\pgfsetdash{}{0pt}%
\pgfsys@defobject{currentmarker}{\pgfqpoint{-0.048611in}{0.000000in}}{\pgfqpoint{0.000000in}{0.000000in}}{%
\pgfpathmoveto{\pgfqpoint{0.000000in}{0.000000in}}%
\pgfpathlineto{\pgfqpoint{-0.048611in}{0.000000in}}%
\pgfusepath{stroke,fill}%
}%
\begin{pgfscope}%
\pgfsys@transformshift{0.557222in}{2.199335in}%
\pgfsys@useobject{currentmarker}{}%
\end{pgfscope}%
\end{pgfscope}%
\begin{pgfscope}%
\definecolor{textcolor}{rgb}{0.000000,0.000000,0.000000}%
\pgfsetstrokecolor{textcolor}%
\pgfsetfillcolor{textcolor}%
\pgftext[x=0.239121in,y=2.146574in,left,base]{\color{textcolor}\sffamily\fontsize{10.000000}{12.000000}\selectfont 2.2}%
\end{pgfscope}%
\begin{pgfscope}%
\pgfsetbuttcap%
\pgfsetroundjoin%
\definecolor{currentfill}{rgb}{0.000000,0.000000,0.000000}%
\pgfsetfillcolor{currentfill}%
\pgfsetlinewidth{0.803000pt}%
\definecolor{currentstroke}{rgb}{0.000000,0.000000,0.000000}%
\pgfsetstrokecolor{currentstroke}%
\pgfsetdash{}{0pt}%
\pgfsys@defobject{currentmarker}{\pgfqpoint{-0.048611in}{0.000000in}}{\pgfqpoint{0.000000in}{0.000000in}}{%
\pgfpathmoveto{\pgfqpoint{0.000000in}{0.000000in}}%
\pgfpathlineto{\pgfqpoint{-0.048611in}{0.000000in}}%
\pgfusepath{stroke,fill}%
}%
\begin{pgfscope}%
\pgfsys@transformshift{0.557222in}{2.592368in}%
\pgfsys@useobject{currentmarker}{}%
\end{pgfscope}%
\end{pgfscope}%
\begin{pgfscope}%
\definecolor{textcolor}{rgb}{0.000000,0.000000,0.000000}%
\pgfsetstrokecolor{textcolor}%
\pgfsetfillcolor{textcolor}%
\pgftext[x=0.239121in,y=2.539607in,left,base]{\color{textcolor}\sffamily\fontsize{10.000000}{12.000000}\selectfont 2.4}%
\end{pgfscope}%
\begin{pgfscope}%
\pgfpathrectangle{\pgfqpoint{0.557222in}{0.387222in}}{\pgfqpoint{3.321333in}{2.324444in}}%
\pgfusepath{clip}%
\pgfsetrectcap%
\pgfsetroundjoin%
\pgfsetlinewidth{1.505625pt}%
\definecolor{currentstroke}{rgb}{0.121569,0.466667,0.705882}%
\pgfsetstrokecolor{currentstroke}%
\pgfsetdash{}{0pt}%
\pgfpathmoveto{\pgfqpoint{0.716480in}{1.542686in}}%
\pgfpathlineto{\pgfqpoint{0.866621in}{1.637770in}}%
\pgfpathlineto{\pgfqpoint{1.016761in}{1.643057in}}%
\pgfpathlineto{\pgfqpoint{1.166902in}{1.710082in}}%
\pgfpathlineto{\pgfqpoint{1.317043in}{1.870393in}}%
\pgfpathlineto{\pgfqpoint{1.467184in}{2.005178in}}%
\pgfpathlineto{\pgfqpoint{1.617325in}{2.188877in}}%
\pgfpathlineto{\pgfqpoint{1.767466in}{2.043828in}}%
\pgfpathlineto{\pgfqpoint{1.917607in}{2.196901in}}%
\pgfpathlineto{\pgfqpoint{2.067748in}{2.299510in}}%
\pgfpathlineto{\pgfqpoint{2.217889in}{2.441928in}}%
\pgfpathlineto{\pgfqpoint{2.368030in}{2.465569in}}%
\pgfpathlineto{\pgfqpoint{2.518171in}{2.454705in}}%
\pgfpathlineto{\pgfqpoint{2.668311in}{2.375718in}}%
\pgfpathlineto{\pgfqpoint{2.818452in}{2.188428in}}%
\pgfpathlineto{\pgfqpoint{2.968593in}{2.012104in}}%
\pgfpathlineto{\pgfqpoint{3.118734in}{1.781221in}}%
\pgfpathlineto{\pgfqpoint{3.268875in}{1.481718in}}%
\pgfpathlineto{\pgfqpoint{3.419016in}{1.195917in}}%
\pgfpathlineto{\pgfqpoint{3.569157in}{0.823155in}}%
\pgfpathlineto{\pgfqpoint{3.719298in}{0.605454in}}%
\pgfusepath{stroke}%
\end{pgfscope}%
\begin{pgfscope}%
\pgfsetrectcap%
\pgfsetmiterjoin%
\pgfsetlinewidth{0.803000pt}%
\definecolor{currentstroke}{rgb}{0.000000,0.000000,0.000000}%
\pgfsetstrokecolor{currentstroke}%
\pgfsetdash{}{0pt}%
\pgfpathmoveto{\pgfqpoint{0.557222in}{0.387222in}}%
\pgfpathlineto{\pgfqpoint{0.557222in}{2.711667in}}%
\pgfusepath{stroke}%
\end{pgfscope}%
\begin{pgfscope}%
\pgfsetrectcap%
\pgfsetmiterjoin%
\pgfsetlinewidth{0.803000pt}%
\definecolor{currentstroke}{rgb}{0.000000,0.000000,0.000000}%
\pgfsetstrokecolor{currentstroke}%
\pgfsetdash{}{0pt}%
\pgfpathmoveto{\pgfqpoint{3.878555in}{0.387222in}}%
\pgfpathlineto{\pgfqpoint{3.878555in}{2.711667in}}%
\pgfusepath{stroke}%
\end{pgfscope}%
\begin{pgfscope}%
\pgfsetrectcap%
\pgfsetmiterjoin%
\pgfsetlinewidth{0.803000pt}%
\definecolor{currentstroke}{rgb}{0.000000,0.000000,0.000000}%
\pgfsetstrokecolor{currentstroke}%
\pgfsetdash{}{0pt}%
\pgfpathmoveto{\pgfqpoint{0.557222in}{0.387222in}}%
\pgfpathlineto{\pgfqpoint{3.878555in}{0.387222in}}%
\pgfusepath{stroke}%
\end{pgfscope}%
\begin{pgfscope}%
\pgfsetrectcap%
\pgfsetmiterjoin%
\pgfsetlinewidth{0.803000pt}%
\definecolor{currentstroke}{rgb}{0.000000,0.000000,0.000000}%
\pgfsetstrokecolor{currentstroke}%
\pgfsetdash{}{0pt}%
\pgfpathmoveto{\pgfqpoint{0.557222in}{2.711667in}}%
\pgfpathlineto{\pgfqpoint{3.878555in}{2.711667in}}%
\pgfusepath{stroke}%
\end{pgfscope}%
\begin{pgfscope}%
\definecolor{textcolor}{rgb}{0.000000,0.000000,0.000000}%
\pgfsetstrokecolor{textcolor}%
\pgfsetfillcolor{textcolor}%
\pgftext[x=2.217889in,y=2.795000in,,base]{\color{textcolor}\sffamily\fontsize{12.000000}{14.400000}\selectfont \(\displaystyle  N = 16 \)}%
\end{pgfscope}%
\begin{pgfscope}%
\pgfsetbuttcap%
\pgfsetmiterjoin%
\definecolor{currentfill}{rgb}{1.000000,1.000000,1.000000}%
\pgfsetfillcolor{currentfill}%
\pgfsetlinewidth{0.000000pt}%
\definecolor{currentstroke}{rgb}{0.000000,0.000000,0.000000}%
\pgfsetstrokecolor{currentstroke}%
\pgfsetstrokeopacity{0.000000}%
\pgfsetdash{}{0pt}%
\pgfpathmoveto{\pgfqpoint{4.493667in}{0.387222in}}%
\pgfpathlineto{\pgfqpoint{7.815000in}{0.387222in}}%
\pgfpathlineto{\pgfqpoint{7.815000in}{2.711667in}}%
\pgfpathlineto{\pgfqpoint{4.493667in}{2.711667in}}%
\pgfpathclose%
\pgfusepath{fill}%
\end{pgfscope}%
\begin{pgfscope}%
\pgfpathrectangle{\pgfqpoint{4.493667in}{0.387222in}}{\pgfqpoint{3.321333in}{2.324444in}}%
\pgfusepath{clip}%
\pgfsetbuttcap%
\pgfsetroundjoin%
\definecolor{currentfill}{rgb}{0.121569,0.466667,0.705882}%
\pgfsetfillcolor{currentfill}%
\pgfsetfillopacity{0.300000}%
\pgfsetlinewidth{1.003750pt}%
\definecolor{currentstroke}{rgb}{0.121569,0.466667,0.705882}%
\pgfsetstrokecolor{currentstroke}%
\pgfsetstrokeopacity{0.300000}%
\pgfsetdash{}{0pt}%
\pgfpathmoveto{\pgfqpoint{4.652924in}{1.637828in}}%
\pgfpathlineto{\pgfqpoint{4.652924in}{1.474225in}}%
\pgfpathlineto{\pgfqpoint{4.803065in}{1.568553in}}%
\pgfpathlineto{\pgfqpoint{4.953206in}{1.621694in}}%
\pgfpathlineto{\pgfqpoint{5.103347in}{1.602059in}}%
\pgfpathlineto{\pgfqpoint{5.253488in}{1.556349in}}%
\pgfpathlineto{\pgfqpoint{5.403629in}{1.817882in}}%
\pgfpathlineto{\pgfqpoint{5.553770in}{1.759891in}}%
\pgfpathlineto{\pgfqpoint{5.703911in}{1.670415in}}%
\pgfpathlineto{\pgfqpoint{5.854052in}{1.913314in}}%
\pgfpathlineto{\pgfqpoint{6.004193in}{1.971687in}}%
\pgfpathlineto{\pgfqpoint{6.154333in}{1.877971in}}%
\pgfpathlineto{\pgfqpoint{6.304474in}{2.035620in}}%
\pgfpathlineto{\pgfqpoint{6.454615in}{1.749108in}}%
\pgfpathlineto{\pgfqpoint{6.604756in}{1.525652in}}%
\pgfpathlineto{\pgfqpoint{6.754897in}{1.224774in}}%
\pgfpathlineto{\pgfqpoint{6.905038in}{1.109751in}}%
\pgfpathlineto{\pgfqpoint{7.055179in}{0.861425in}}%
\pgfpathlineto{\pgfqpoint{7.205320in}{0.678074in}}%
\pgfpathlineto{\pgfqpoint{7.355461in}{0.624139in}}%
\pgfpathlineto{\pgfqpoint{7.505602in}{0.595311in}}%
\pgfpathlineto{\pgfqpoint{7.655743in}{0.492879in}}%
\pgfpathlineto{\pgfqpoint{7.655743in}{0.645701in}}%
\pgfpathlineto{\pgfqpoint{7.655743in}{0.645701in}}%
\pgfpathlineto{\pgfqpoint{7.505602in}{0.653598in}}%
\pgfpathlineto{\pgfqpoint{7.355461in}{0.756707in}}%
\pgfpathlineto{\pgfqpoint{7.205320in}{0.782929in}}%
\pgfpathlineto{\pgfqpoint{7.055179in}{0.995652in}}%
\pgfpathlineto{\pgfqpoint{6.905038in}{1.328929in}}%
\pgfpathlineto{\pgfqpoint{6.754897in}{1.533202in}}%
\pgfpathlineto{\pgfqpoint{6.604756in}{1.837038in}}%
\pgfpathlineto{\pgfqpoint{6.454615in}{2.144026in}}%
\pgfpathlineto{\pgfqpoint{6.304474in}{2.276200in}}%
\pgfpathlineto{\pgfqpoint{6.154333in}{2.524013in}}%
\pgfpathlineto{\pgfqpoint{6.004193in}{2.606010in}}%
\pgfpathlineto{\pgfqpoint{5.854052in}{2.303113in}}%
\pgfpathlineto{\pgfqpoint{5.703911in}{1.932022in}}%
\pgfpathlineto{\pgfqpoint{5.553770in}{2.039326in}}%
\pgfpathlineto{\pgfqpoint{5.403629in}{1.919021in}}%
\pgfpathlineto{\pgfqpoint{5.253488in}{1.846578in}}%
\pgfpathlineto{\pgfqpoint{5.103347in}{1.762105in}}%
\pgfpathlineto{\pgfqpoint{4.953206in}{1.725028in}}%
\pgfpathlineto{\pgfqpoint{4.803065in}{1.747686in}}%
\pgfpathlineto{\pgfqpoint{4.652924in}{1.637828in}}%
\pgfpathclose%
\pgfusepath{stroke,fill}%
\end{pgfscope}%
\begin{pgfscope}%
\pgfpathrectangle{\pgfqpoint{4.493667in}{0.387222in}}{\pgfqpoint{3.321333in}{2.324444in}}%
\pgfusepath{clip}%
\pgfsetbuttcap%
\pgfsetroundjoin%
\definecolor{currentfill}{rgb}{0.121569,0.466667,0.705882}%
\pgfsetfillcolor{currentfill}%
\pgfsetlinewidth{1.003750pt}%
\definecolor{currentstroke}{rgb}{0.121569,0.466667,0.705882}%
\pgfsetstrokecolor{currentstroke}%
\pgfsetdash{}{0pt}%
\pgfsys@defobject{currentmarker}{\pgfqpoint{-0.009821in}{-0.009821in}}{\pgfqpoint{0.009821in}{0.009821in}}{%
\pgfpathmoveto{\pgfqpoint{0.000000in}{-0.009821in}}%
\pgfpathcurveto{\pgfqpoint{0.002605in}{-0.009821in}}{\pgfqpoint{0.005103in}{-0.008786in}}{\pgfqpoint{0.006944in}{-0.006944in}}%
\pgfpathcurveto{\pgfqpoint{0.008786in}{-0.005103in}}{\pgfqpoint{0.009821in}{-0.002605in}}{\pgfqpoint{0.009821in}{0.000000in}}%
\pgfpathcurveto{\pgfqpoint{0.009821in}{0.002605in}}{\pgfqpoint{0.008786in}{0.005103in}}{\pgfqpoint{0.006944in}{0.006944in}}%
\pgfpathcurveto{\pgfqpoint{0.005103in}{0.008786in}}{\pgfqpoint{0.002605in}{0.009821in}}{\pgfqpoint{0.000000in}{0.009821in}}%
\pgfpathcurveto{\pgfqpoint{-0.002605in}{0.009821in}}{\pgfqpoint{-0.005103in}{0.008786in}}{\pgfqpoint{-0.006944in}{0.006944in}}%
\pgfpathcurveto{\pgfqpoint{-0.008786in}{0.005103in}}{\pgfqpoint{-0.009821in}{0.002605in}}{\pgfqpoint{-0.009821in}{0.000000in}}%
\pgfpathcurveto{\pgfqpoint{-0.009821in}{-0.002605in}}{\pgfqpoint{-0.008786in}{-0.005103in}}{\pgfqpoint{-0.006944in}{-0.006944in}}%
\pgfpathcurveto{\pgfqpoint{-0.005103in}{-0.008786in}}{\pgfqpoint{-0.002605in}{-0.009821in}}{\pgfqpoint{0.000000in}{-0.009821in}}%
\pgfpathclose%
\pgfusepath{stroke,fill}%
}%
\begin{pgfscope}%
\pgfsys@transformshift{4.652924in}{1.556027in}%
\pgfsys@useobject{currentmarker}{}%
\end{pgfscope}%
\begin{pgfscope}%
\pgfsys@transformshift{4.803065in}{1.658119in}%
\pgfsys@useobject{currentmarker}{}%
\end{pgfscope}%
\begin{pgfscope}%
\pgfsys@transformshift{4.953206in}{1.673361in}%
\pgfsys@useobject{currentmarker}{}%
\end{pgfscope}%
\begin{pgfscope}%
\pgfsys@transformshift{5.103347in}{1.682082in}%
\pgfsys@useobject{currentmarker}{}%
\end{pgfscope}%
\begin{pgfscope}%
\pgfsys@transformshift{5.253488in}{1.701464in}%
\pgfsys@useobject{currentmarker}{}%
\end{pgfscope}%
\begin{pgfscope}%
\pgfsys@transformshift{5.403629in}{1.868451in}%
\pgfsys@useobject{currentmarker}{}%
\end{pgfscope}%
\begin{pgfscope}%
\pgfsys@transformshift{5.553770in}{1.899609in}%
\pgfsys@useobject{currentmarker}{}%
\end{pgfscope}%
\begin{pgfscope}%
\pgfsys@transformshift{5.703911in}{1.801219in}%
\pgfsys@useobject{currentmarker}{}%
\end{pgfscope}%
\begin{pgfscope}%
\pgfsys@transformshift{5.854052in}{2.108213in}%
\pgfsys@useobject{currentmarker}{}%
\end{pgfscope}%
\begin{pgfscope}%
\pgfsys@transformshift{6.004193in}{2.288848in}%
\pgfsys@useobject{currentmarker}{}%
\end{pgfscope}%
\begin{pgfscope}%
\pgfsys@transformshift{6.154333in}{2.200992in}%
\pgfsys@useobject{currentmarker}{}%
\end{pgfscope}%
\begin{pgfscope}%
\pgfsys@transformshift{6.304474in}{2.155910in}%
\pgfsys@useobject{currentmarker}{}%
\end{pgfscope}%
\begin{pgfscope}%
\pgfsys@transformshift{6.454615in}{1.946567in}%
\pgfsys@useobject{currentmarker}{}%
\end{pgfscope}%
\begin{pgfscope}%
\pgfsys@transformshift{6.604756in}{1.681345in}%
\pgfsys@useobject{currentmarker}{}%
\end{pgfscope}%
\begin{pgfscope}%
\pgfsys@transformshift{6.754897in}{1.378988in}%
\pgfsys@useobject{currentmarker}{}%
\end{pgfscope}%
\begin{pgfscope}%
\pgfsys@transformshift{6.905038in}{1.219340in}%
\pgfsys@useobject{currentmarker}{}%
\end{pgfscope}%
\begin{pgfscope}%
\pgfsys@transformshift{7.055179in}{0.928539in}%
\pgfsys@useobject{currentmarker}{}%
\end{pgfscope}%
\begin{pgfscope}%
\pgfsys@transformshift{7.205320in}{0.730501in}%
\pgfsys@useobject{currentmarker}{}%
\end{pgfscope}%
\begin{pgfscope}%
\pgfsys@transformshift{7.355461in}{0.690423in}%
\pgfsys@useobject{currentmarker}{}%
\end{pgfscope}%
\begin{pgfscope}%
\pgfsys@transformshift{7.505602in}{0.624454in}%
\pgfsys@useobject{currentmarker}{}%
\end{pgfscope}%
\begin{pgfscope}%
\pgfsys@transformshift{7.655743in}{0.569290in}%
\pgfsys@useobject{currentmarker}{}%
\end{pgfscope}%
\end{pgfscope}%
\begin{pgfscope}%
\pgfsetbuttcap%
\pgfsetroundjoin%
\definecolor{currentfill}{rgb}{0.000000,0.000000,0.000000}%
\pgfsetfillcolor{currentfill}%
\pgfsetlinewidth{0.803000pt}%
\definecolor{currentstroke}{rgb}{0.000000,0.000000,0.000000}%
\pgfsetstrokecolor{currentstroke}%
\pgfsetdash{}{0pt}%
\pgfsys@defobject{currentmarker}{\pgfqpoint{0.000000in}{-0.048611in}}{\pgfqpoint{0.000000in}{0.000000in}}{%
\pgfpathmoveto{\pgfqpoint{0.000000in}{0.000000in}}%
\pgfpathlineto{\pgfqpoint{0.000000in}{-0.048611in}}%
\pgfusepath{stroke,fill}%
}%
\begin{pgfscope}%
\pgfsys@transformshift{4.953206in}{0.387222in}%
\pgfsys@useobject{currentmarker}{}%
\end{pgfscope}%
\end{pgfscope}%
\begin{pgfscope}%
\definecolor{textcolor}{rgb}{0.000000,0.000000,0.000000}%
\pgfsetstrokecolor{textcolor}%
\pgfsetfillcolor{textcolor}%
\pgftext[x=4.953206in,y=0.290000in,,top]{\color{textcolor}\sffamily\fontsize{10.000000}{12.000000}\selectfont 4.40}%
\end{pgfscope}%
\begin{pgfscope}%
\pgfsetbuttcap%
\pgfsetroundjoin%
\definecolor{currentfill}{rgb}{0.000000,0.000000,0.000000}%
\pgfsetfillcolor{currentfill}%
\pgfsetlinewidth{0.803000pt}%
\definecolor{currentstroke}{rgb}{0.000000,0.000000,0.000000}%
\pgfsetstrokecolor{currentstroke}%
\pgfsetdash{}{0pt}%
\pgfsys@defobject{currentmarker}{\pgfqpoint{0.000000in}{-0.048611in}}{\pgfqpoint{0.000000in}{0.000000in}}{%
\pgfpathmoveto{\pgfqpoint{0.000000in}{0.000000in}}%
\pgfpathlineto{\pgfqpoint{0.000000in}{-0.048611in}}%
\pgfusepath{stroke,fill}%
}%
\begin{pgfscope}%
\pgfsys@transformshift{5.703911in}{0.387222in}%
\pgfsys@useobject{currentmarker}{}%
\end{pgfscope}%
\end{pgfscope}%
\begin{pgfscope}%
\definecolor{textcolor}{rgb}{0.000000,0.000000,0.000000}%
\pgfsetstrokecolor{textcolor}%
\pgfsetfillcolor{textcolor}%
\pgftext[x=5.703911in,y=0.290000in,,top]{\color{textcolor}\sffamily\fontsize{10.000000}{12.000000}\selectfont 4.45}%
\end{pgfscope}%
\begin{pgfscope}%
\pgfsetbuttcap%
\pgfsetroundjoin%
\definecolor{currentfill}{rgb}{0.000000,0.000000,0.000000}%
\pgfsetfillcolor{currentfill}%
\pgfsetlinewidth{0.803000pt}%
\definecolor{currentstroke}{rgb}{0.000000,0.000000,0.000000}%
\pgfsetstrokecolor{currentstroke}%
\pgfsetdash{}{0pt}%
\pgfsys@defobject{currentmarker}{\pgfqpoint{0.000000in}{-0.048611in}}{\pgfqpoint{0.000000in}{0.000000in}}{%
\pgfpathmoveto{\pgfqpoint{0.000000in}{0.000000in}}%
\pgfpathlineto{\pgfqpoint{0.000000in}{-0.048611in}}%
\pgfusepath{stroke,fill}%
}%
\begin{pgfscope}%
\pgfsys@transformshift{6.454615in}{0.387222in}%
\pgfsys@useobject{currentmarker}{}%
\end{pgfscope}%
\end{pgfscope}%
\begin{pgfscope}%
\definecolor{textcolor}{rgb}{0.000000,0.000000,0.000000}%
\pgfsetstrokecolor{textcolor}%
\pgfsetfillcolor{textcolor}%
\pgftext[x=6.454615in,y=0.290000in,,top]{\color{textcolor}\sffamily\fontsize{10.000000}{12.000000}\selectfont 4.50}%
\end{pgfscope}%
\begin{pgfscope}%
\pgfsetbuttcap%
\pgfsetroundjoin%
\definecolor{currentfill}{rgb}{0.000000,0.000000,0.000000}%
\pgfsetfillcolor{currentfill}%
\pgfsetlinewidth{0.803000pt}%
\definecolor{currentstroke}{rgb}{0.000000,0.000000,0.000000}%
\pgfsetstrokecolor{currentstroke}%
\pgfsetdash{}{0pt}%
\pgfsys@defobject{currentmarker}{\pgfqpoint{0.000000in}{-0.048611in}}{\pgfqpoint{0.000000in}{0.000000in}}{%
\pgfpathmoveto{\pgfqpoint{0.000000in}{0.000000in}}%
\pgfpathlineto{\pgfqpoint{0.000000in}{-0.048611in}}%
\pgfusepath{stroke,fill}%
}%
\begin{pgfscope}%
\pgfsys@transformshift{7.205320in}{0.387222in}%
\pgfsys@useobject{currentmarker}{}%
\end{pgfscope}%
\end{pgfscope}%
\begin{pgfscope}%
\definecolor{textcolor}{rgb}{0.000000,0.000000,0.000000}%
\pgfsetstrokecolor{textcolor}%
\pgfsetfillcolor{textcolor}%
\pgftext[x=7.205320in,y=0.290000in,,top]{\color{textcolor}\sffamily\fontsize{10.000000}{12.000000}\selectfont 4.55}%
\end{pgfscope}%
\begin{pgfscope}%
\pgfsetbuttcap%
\pgfsetroundjoin%
\definecolor{currentfill}{rgb}{0.000000,0.000000,0.000000}%
\pgfsetfillcolor{currentfill}%
\pgfsetlinewidth{0.803000pt}%
\definecolor{currentstroke}{rgb}{0.000000,0.000000,0.000000}%
\pgfsetstrokecolor{currentstroke}%
\pgfsetdash{}{0pt}%
\pgfsys@defobject{currentmarker}{\pgfqpoint{-0.048611in}{0.000000in}}{\pgfqpoint{0.000000in}{0.000000in}}{%
\pgfpathmoveto{\pgfqpoint{0.000000in}{0.000000in}}%
\pgfpathlineto{\pgfqpoint{-0.048611in}{0.000000in}}%
\pgfusepath{stroke,fill}%
}%
\begin{pgfscope}%
\pgfsys@transformshift{4.493667in}{0.761223in}%
\pgfsys@useobject{currentmarker}{}%
\end{pgfscope}%
\end{pgfscope}%
\begin{pgfscope}%
\definecolor{textcolor}{rgb}{0.000000,0.000000,0.000000}%
\pgfsetstrokecolor{textcolor}%
\pgfsetfillcolor{textcolor}%
\pgftext[x=4.175565in,y=0.708461in,left,base]{\color{textcolor}\sffamily\fontsize{10.000000}{12.000000}\selectfont 1.0}%
\end{pgfscope}%
\begin{pgfscope}%
\pgfsetbuttcap%
\pgfsetroundjoin%
\definecolor{currentfill}{rgb}{0.000000,0.000000,0.000000}%
\pgfsetfillcolor{currentfill}%
\pgfsetlinewidth{0.803000pt}%
\definecolor{currentstroke}{rgb}{0.000000,0.000000,0.000000}%
\pgfsetstrokecolor{currentstroke}%
\pgfsetdash{}{0pt}%
\pgfsys@defobject{currentmarker}{\pgfqpoint{-0.048611in}{0.000000in}}{\pgfqpoint{0.000000in}{0.000000in}}{%
\pgfpathmoveto{\pgfqpoint{0.000000in}{0.000000in}}%
\pgfpathlineto{\pgfqpoint{-0.048611in}{0.000000in}}%
\pgfusepath{stroke,fill}%
}%
\begin{pgfscope}%
\pgfsys@transformshift{4.493667in}{1.167089in}%
\pgfsys@useobject{currentmarker}{}%
\end{pgfscope}%
\end{pgfscope}%
\begin{pgfscope}%
\definecolor{textcolor}{rgb}{0.000000,0.000000,0.000000}%
\pgfsetstrokecolor{textcolor}%
\pgfsetfillcolor{textcolor}%
\pgftext[x=4.175565in,y=1.114328in,left,base]{\color{textcolor}\sffamily\fontsize{10.000000}{12.000000}\selectfont 1.5}%
\end{pgfscope}%
\begin{pgfscope}%
\pgfsetbuttcap%
\pgfsetroundjoin%
\definecolor{currentfill}{rgb}{0.000000,0.000000,0.000000}%
\pgfsetfillcolor{currentfill}%
\pgfsetlinewidth{0.803000pt}%
\definecolor{currentstroke}{rgb}{0.000000,0.000000,0.000000}%
\pgfsetstrokecolor{currentstroke}%
\pgfsetdash{}{0pt}%
\pgfsys@defobject{currentmarker}{\pgfqpoint{-0.048611in}{0.000000in}}{\pgfqpoint{0.000000in}{0.000000in}}{%
\pgfpathmoveto{\pgfqpoint{0.000000in}{0.000000in}}%
\pgfpathlineto{\pgfqpoint{-0.048611in}{0.000000in}}%
\pgfusepath{stroke,fill}%
}%
\begin{pgfscope}%
\pgfsys@transformshift{4.493667in}{1.572955in}%
\pgfsys@useobject{currentmarker}{}%
\end{pgfscope}%
\end{pgfscope}%
\begin{pgfscope}%
\definecolor{textcolor}{rgb}{0.000000,0.000000,0.000000}%
\pgfsetstrokecolor{textcolor}%
\pgfsetfillcolor{textcolor}%
\pgftext[x=4.175565in,y=1.520194in,left,base]{\color{textcolor}\sffamily\fontsize{10.000000}{12.000000}\selectfont 2.0}%
\end{pgfscope}%
\begin{pgfscope}%
\pgfsetbuttcap%
\pgfsetroundjoin%
\definecolor{currentfill}{rgb}{0.000000,0.000000,0.000000}%
\pgfsetfillcolor{currentfill}%
\pgfsetlinewidth{0.803000pt}%
\definecolor{currentstroke}{rgb}{0.000000,0.000000,0.000000}%
\pgfsetstrokecolor{currentstroke}%
\pgfsetdash{}{0pt}%
\pgfsys@defobject{currentmarker}{\pgfqpoint{-0.048611in}{0.000000in}}{\pgfqpoint{0.000000in}{0.000000in}}{%
\pgfpathmoveto{\pgfqpoint{0.000000in}{0.000000in}}%
\pgfpathlineto{\pgfqpoint{-0.048611in}{0.000000in}}%
\pgfusepath{stroke,fill}%
}%
\begin{pgfscope}%
\pgfsys@transformshift{4.493667in}{1.978822in}%
\pgfsys@useobject{currentmarker}{}%
\end{pgfscope}%
\end{pgfscope}%
\begin{pgfscope}%
\definecolor{textcolor}{rgb}{0.000000,0.000000,0.000000}%
\pgfsetstrokecolor{textcolor}%
\pgfsetfillcolor{textcolor}%
\pgftext[x=4.175565in,y=1.926060in,left,base]{\color{textcolor}\sffamily\fontsize{10.000000}{12.000000}\selectfont 2.5}%
\end{pgfscope}%
\begin{pgfscope}%
\pgfsetbuttcap%
\pgfsetroundjoin%
\definecolor{currentfill}{rgb}{0.000000,0.000000,0.000000}%
\pgfsetfillcolor{currentfill}%
\pgfsetlinewidth{0.803000pt}%
\definecolor{currentstroke}{rgb}{0.000000,0.000000,0.000000}%
\pgfsetstrokecolor{currentstroke}%
\pgfsetdash{}{0pt}%
\pgfsys@defobject{currentmarker}{\pgfqpoint{-0.048611in}{0.000000in}}{\pgfqpoint{0.000000in}{0.000000in}}{%
\pgfpathmoveto{\pgfqpoint{0.000000in}{0.000000in}}%
\pgfpathlineto{\pgfqpoint{-0.048611in}{0.000000in}}%
\pgfusepath{stroke,fill}%
}%
\begin{pgfscope}%
\pgfsys@transformshift{4.493667in}{2.384688in}%
\pgfsys@useobject{currentmarker}{}%
\end{pgfscope}%
\end{pgfscope}%
\begin{pgfscope}%
\definecolor{textcolor}{rgb}{0.000000,0.000000,0.000000}%
\pgfsetstrokecolor{textcolor}%
\pgfsetfillcolor{textcolor}%
\pgftext[x=4.175565in,y=2.331926in,left,base]{\color{textcolor}\sffamily\fontsize{10.000000}{12.000000}\selectfont 3.0}%
\end{pgfscope}%
\begin{pgfscope}%
\pgfpathrectangle{\pgfqpoint{4.493667in}{0.387222in}}{\pgfqpoint{3.321333in}{2.324444in}}%
\pgfusepath{clip}%
\pgfsetrectcap%
\pgfsetroundjoin%
\pgfsetlinewidth{1.505625pt}%
\definecolor{currentstroke}{rgb}{0.121569,0.466667,0.705882}%
\pgfsetstrokecolor{currentstroke}%
\pgfsetdash{}{0pt}%
\pgfpathmoveto{\pgfqpoint{4.652924in}{1.556027in}}%
\pgfpathlineto{\pgfqpoint{4.803065in}{1.658119in}}%
\pgfpathlineto{\pgfqpoint{4.953206in}{1.673361in}}%
\pgfpathlineto{\pgfqpoint{5.103347in}{1.682082in}}%
\pgfpathlineto{\pgfqpoint{5.253488in}{1.701464in}}%
\pgfpathlineto{\pgfqpoint{5.403629in}{1.868451in}}%
\pgfpathlineto{\pgfqpoint{5.553770in}{1.899609in}}%
\pgfpathlineto{\pgfqpoint{5.703911in}{1.801219in}}%
\pgfpathlineto{\pgfqpoint{5.854052in}{2.108213in}}%
\pgfpathlineto{\pgfqpoint{6.004193in}{2.288848in}}%
\pgfpathlineto{\pgfqpoint{6.154333in}{2.200992in}}%
\pgfpathlineto{\pgfqpoint{6.304474in}{2.155910in}}%
\pgfpathlineto{\pgfqpoint{6.454615in}{1.946567in}}%
\pgfpathlineto{\pgfqpoint{6.604756in}{1.681345in}}%
\pgfpathlineto{\pgfqpoint{6.754897in}{1.378988in}}%
\pgfpathlineto{\pgfqpoint{6.905038in}{1.219340in}}%
\pgfpathlineto{\pgfqpoint{7.055179in}{0.928539in}}%
\pgfpathlineto{\pgfqpoint{7.205320in}{0.730501in}}%
\pgfpathlineto{\pgfqpoint{7.355461in}{0.690423in}}%
\pgfpathlineto{\pgfqpoint{7.505602in}{0.624454in}}%
\pgfpathlineto{\pgfqpoint{7.655743in}{0.569290in}}%
\pgfusepath{stroke}%
\end{pgfscope}%
\begin{pgfscope}%
\pgfsetrectcap%
\pgfsetmiterjoin%
\pgfsetlinewidth{0.803000pt}%
\definecolor{currentstroke}{rgb}{0.000000,0.000000,0.000000}%
\pgfsetstrokecolor{currentstroke}%
\pgfsetdash{}{0pt}%
\pgfpathmoveto{\pgfqpoint{4.493667in}{0.387222in}}%
\pgfpathlineto{\pgfqpoint{4.493667in}{2.711667in}}%
\pgfusepath{stroke}%
\end{pgfscope}%
\begin{pgfscope}%
\pgfsetrectcap%
\pgfsetmiterjoin%
\pgfsetlinewidth{0.803000pt}%
\definecolor{currentstroke}{rgb}{0.000000,0.000000,0.000000}%
\pgfsetstrokecolor{currentstroke}%
\pgfsetdash{}{0pt}%
\pgfpathmoveto{\pgfqpoint{7.815000in}{0.387222in}}%
\pgfpathlineto{\pgfqpoint{7.815000in}{2.711667in}}%
\pgfusepath{stroke}%
\end{pgfscope}%
\begin{pgfscope}%
\pgfsetrectcap%
\pgfsetmiterjoin%
\pgfsetlinewidth{0.803000pt}%
\definecolor{currentstroke}{rgb}{0.000000,0.000000,0.000000}%
\pgfsetstrokecolor{currentstroke}%
\pgfsetdash{}{0pt}%
\pgfpathmoveto{\pgfqpoint{4.493667in}{0.387222in}}%
\pgfpathlineto{\pgfqpoint{7.815000in}{0.387222in}}%
\pgfusepath{stroke}%
\end{pgfscope}%
\begin{pgfscope}%
\pgfsetrectcap%
\pgfsetmiterjoin%
\pgfsetlinewidth{0.803000pt}%
\definecolor{currentstroke}{rgb}{0.000000,0.000000,0.000000}%
\pgfsetstrokecolor{currentstroke}%
\pgfsetdash{}{0pt}%
\pgfpathmoveto{\pgfqpoint{4.493667in}{2.711667in}}%
\pgfpathlineto{\pgfqpoint{7.815000in}{2.711667in}}%
\pgfusepath{stroke}%
\end{pgfscope}%
\begin{pgfscope}%
\definecolor{textcolor}{rgb}{0.000000,0.000000,0.000000}%
\pgfsetstrokecolor{textcolor}%
\pgfsetfillcolor{textcolor}%
\pgftext[x=6.154333in,y=2.795000in,,base]{\color{textcolor}\sffamily\fontsize{12.000000}{14.400000}\selectfont \(\displaystyle  N = 24 \)}%
\end{pgfscope}%
\end{pgfpicture}%
\makeatother%
\endgroup%
}
\caption{Figures of the three-dimensional cubatures of $ D = 2 $ (cont.)}
\end{figure}

We may also add extra constraints. Using $ \mathcal{I}_2 \cup \cbr{ \pbr{ 0, 0, 3 }, \pbr{ 0, 3, 0 }, \pbr{ 3, 0, 0 } } $ as constraints, we have $ L = 2 N + C = 13 $. We use $ \text{FPG} \rbr{ 10^4, 10^{-2} } + \text{N} \rbr{30} $ to optimize and get the results in Figure \ref{Fig:M3D2333}.

\begin{figure}[htbp]
\centering
\scalebox{0.75}{%% Creator: Matplotlib, PGF backend
%%
%% To include the figure in your LaTeX document, write
%%   \input{<filename>.pgf}
%%
%% Make sure the required packages are loaded in your preamble
%%   \usepackage{pgf}
%%
%% Figures using additional raster images can only be included by \input if
%% they are in the same directory as the main LaTeX file. For loading figures
%% from other directories you can use the `import` package
%%   \usepackage{import}
%% and then include the figures with
%%   \import{<path to file>}{<filename>.pgf}
%%
%% Matplotlib used the following preamble
%%   \usepackage{fontspec}
%%   \setmainfont{DejaVuSerif.ttf}[Path=/home/lzh/anaconda3/envs/numana/lib/python3.7/site-packages/matplotlib/mpl-data/fonts/ttf/]
%%   \setsansfont{DejaVuSans.ttf}[Path=/home/lzh/anaconda3/envs/numana/lib/python3.7/site-packages/matplotlib/mpl-data/fonts/ttf/]
%%   \setmonofont{DejaVuSansMono.ttf}[Path=/home/lzh/anaconda3/envs/numana/lib/python3.7/site-packages/matplotlib/mpl-data/fonts/ttf/]
%%
\begingroup%
\makeatletter%
\begin{pgfpicture}%
\pgfpathrectangle{\pgfpointorigin}{\pgfqpoint{8.000000in}{6.000000in}}%
\pgfusepath{use as bounding box, clip}%
\begin{pgfscope}%
\pgfsetbuttcap%
\pgfsetmiterjoin%
\definecolor{currentfill}{rgb}{1.000000,1.000000,1.000000}%
\pgfsetfillcolor{currentfill}%
\pgfsetlinewidth{0.000000pt}%
\definecolor{currentstroke}{rgb}{1.000000,1.000000,1.000000}%
\pgfsetstrokecolor{currentstroke}%
\pgfsetdash{}{0pt}%
\pgfpathmoveto{\pgfqpoint{0.000000in}{0.000000in}}%
\pgfpathlineto{\pgfqpoint{8.000000in}{0.000000in}}%
\pgfpathlineto{\pgfqpoint{8.000000in}{6.000000in}}%
\pgfpathlineto{\pgfqpoint{0.000000in}{6.000000in}}%
\pgfpathclose%
\pgfusepath{fill}%
\end{pgfscope}%
\begin{pgfscope}%
\pgfsetbuttcap%
\pgfsetmiterjoin%
\definecolor{currentfill}{rgb}{1.000000,1.000000,1.000000}%
\pgfsetfillcolor{currentfill}%
\pgfsetlinewidth{0.000000pt}%
\definecolor{currentstroke}{rgb}{0.000000,0.000000,0.000000}%
\pgfsetstrokecolor{currentstroke}%
\pgfsetstrokeopacity{0.000000}%
\pgfsetdash{}{0pt}%
\pgfpathmoveto{\pgfqpoint{0.555972in}{3.312222in}}%
\pgfpathlineto{\pgfqpoint{3.878555in}{3.312222in}}%
\pgfpathlineto{\pgfqpoint{3.878555in}{5.636667in}}%
\pgfpathlineto{\pgfqpoint{0.555972in}{5.636667in}}%
\pgfpathclose%
\pgfusepath{fill}%
\end{pgfscope}%
\begin{pgfscope}%
\pgfpathrectangle{\pgfqpoint{0.555972in}{3.312222in}}{\pgfqpoint{3.322583in}{2.324444in}}%
\pgfusepath{clip}%
\pgfsetbuttcap%
\pgfsetroundjoin%
\definecolor{currentfill}{rgb}{0.121569,0.466667,0.705882}%
\pgfsetfillcolor{currentfill}%
\pgfsetfillopacity{0.300000}%
\pgfsetlinewidth{1.003750pt}%
\definecolor{currentstroke}{rgb}{0.121569,0.466667,0.705882}%
\pgfsetstrokecolor{currentstroke}%
\pgfsetstrokeopacity{0.300000}%
\pgfsetdash{}{0pt}%
\pgfpathmoveto{\pgfqpoint{0.715290in}{5.531010in}}%
\pgfpathlineto{\pgfqpoint{0.715290in}{5.511605in}}%
\pgfpathlineto{\pgfqpoint{0.865487in}{5.440431in}}%
\pgfpathlineto{\pgfqpoint{1.015684in}{5.349840in}}%
\pgfpathlineto{\pgfqpoint{1.165882in}{5.268395in}}%
\pgfpathlineto{\pgfqpoint{1.316079in}{5.185211in}}%
\pgfpathlineto{\pgfqpoint{1.466277in}{5.083360in}}%
\pgfpathlineto{\pgfqpoint{1.616474in}{4.997688in}}%
\pgfpathlineto{\pgfqpoint{1.766671in}{4.898672in}}%
\pgfpathlineto{\pgfqpoint{1.916869in}{4.789061in}}%
\pgfpathlineto{\pgfqpoint{2.067066in}{4.698190in}}%
\pgfpathlineto{\pgfqpoint{2.217264in}{4.590857in}}%
\pgfpathlineto{\pgfqpoint{2.367461in}{4.474640in}}%
\pgfpathlineto{\pgfqpoint{2.517659in}{4.388162in}}%
\pgfpathlineto{\pgfqpoint{2.667856in}{4.259915in}}%
\pgfpathlineto{\pgfqpoint{2.818053in}{4.146677in}}%
\pgfpathlineto{\pgfqpoint{2.968251in}{4.030397in}}%
\pgfpathlineto{\pgfqpoint{3.118448in}{3.912067in}}%
\pgfpathlineto{\pgfqpoint{3.268646in}{3.777380in}}%
\pgfpathlineto{\pgfqpoint{3.418843in}{3.673431in}}%
\pgfpathlineto{\pgfqpoint{3.569040in}{3.558551in}}%
\pgfpathlineto{\pgfqpoint{3.719238in}{3.417879in}}%
\pgfpathlineto{\pgfqpoint{3.719238in}{3.450248in}}%
\pgfpathlineto{\pgfqpoint{3.719238in}{3.450248in}}%
\pgfpathlineto{\pgfqpoint{3.569040in}{3.581866in}}%
\pgfpathlineto{\pgfqpoint{3.418843in}{3.681569in}}%
\pgfpathlineto{\pgfqpoint{3.268646in}{3.828696in}}%
\pgfpathlineto{\pgfqpoint{3.118448in}{3.923040in}}%
\pgfpathlineto{\pgfqpoint{2.968251in}{4.055642in}}%
\pgfpathlineto{\pgfqpoint{2.818053in}{4.166993in}}%
\pgfpathlineto{\pgfqpoint{2.667856in}{4.288349in}}%
\pgfpathlineto{\pgfqpoint{2.517659in}{4.398945in}}%
\pgfpathlineto{\pgfqpoint{2.367461in}{4.512501in}}%
\pgfpathlineto{\pgfqpoint{2.217264in}{4.614087in}}%
\pgfpathlineto{\pgfqpoint{2.067066in}{4.726033in}}%
\pgfpathlineto{\pgfqpoint{1.916869in}{4.818645in}}%
\pgfpathlineto{\pgfqpoint{1.766671in}{4.911875in}}%
\pgfpathlineto{\pgfqpoint{1.616474in}{5.009523in}}%
\pgfpathlineto{\pgfqpoint{1.466277in}{5.104291in}}%
\pgfpathlineto{\pgfqpoint{1.316079in}{5.194185in}}%
\pgfpathlineto{\pgfqpoint{1.165882in}{5.287266in}}%
\pgfpathlineto{\pgfqpoint{1.015684in}{5.371328in}}%
\pgfpathlineto{\pgfqpoint{0.865487in}{5.450498in}}%
\pgfpathlineto{\pgfqpoint{0.715290in}{5.531010in}}%
\pgfpathclose%
\pgfusepath{stroke,fill}%
\end{pgfscope}%
\begin{pgfscope}%
\pgfpathrectangle{\pgfqpoint{0.555972in}{3.312222in}}{\pgfqpoint{3.322583in}{2.324444in}}%
\pgfusepath{clip}%
\pgfsetbuttcap%
\pgfsetroundjoin%
\definecolor{currentfill}{rgb}{0.121569,0.466667,0.705882}%
\pgfsetfillcolor{currentfill}%
\pgfsetlinewidth{1.003750pt}%
\definecolor{currentstroke}{rgb}{0.121569,0.466667,0.705882}%
\pgfsetstrokecolor{currentstroke}%
\pgfsetdash{}{0pt}%
\pgfsys@defobject{currentmarker}{\pgfqpoint{-0.009821in}{-0.009821in}}{\pgfqpoint{0.009821in}{0.009821in}}{%
\pgfpathmoveto{\pgfqpoint{0.000000in}{-0.009821in}}%
\pgfpathcurveto{\pgfqpoint{0.002605in}{-0.009821in}}{\pgfqpoint{0.005103in}{-0.008786in}}{\pgfqpoint{0.006944in}{-0.006944in}}%
\pgfpathcurveto{\pgfqpoint{0.008786in}{-0.005103in}}{\pgfqpoint{0.009821in}{-0.002605in}}{\pgfqpoint{0.009821in}{0.000000in}}%
\pgfpathcurveto{\pgfqpoint{0.009821in}{0.002605in}}{\pgfqpoint{0.008786in}{0.005103in}}{\pgfqpoint{0.006944in}{0.006944in}}%
\pgfpathcurveto{\pgfqpoint{0.005103in}{0.008786in}}{\pgfqpoint{0.002605in}{0.009821in}}{\pgfqpoint{0.000000in}{0.009821in}}%
\pgfpathcurveto{\pgfqpoint{-0.002605in}{0.009821in}}{\pgfqpoint{-0.005103in}{0.008786in}}{\pgfqpoint{-0.006944in}{0.006944in}}%
\pgfpathcurveto{\pgfqpoint{-0.008786in}{0.005103in}}{\pgfqpoint{-0.009821in}{0.002605in}}{\pgfqpoint{-0.009821in}{0.000000in}}%
\pgfpathcurveto{\pgfqpoint{-0.009821in}{-0.002605in}}{\pgfqpoint{-0.008786in}{-0.005103in}}{\pgfqpoint{-0.006944in}{-0.006944in}}%
\pgfpathcurveto{\pgfqpoint{-0.005103in}{-0.008786in}}{\pgfqpoint{-0.002605in}{-0.009821in}}{\pgfqpoint{0.000000in}{-0.009821in}}%
\pgfpathclose%
\pgfusepath{stroke,fill}%
}%
\begin{pgfscope}%
\pgfsys@transformshift{0.715290in}{5.521307in}%
\pgfsys@useobject{currentmarker}{}%
\end{pgfscope}%
\begin{pgfscope}%
\pgfsys@transformshift{0.865487in}{5.445464in}%
\pgfsys@useobject{currentmarker}{}%
\end{pgfscope}%
\begin{pgfscope}%
\pgfsys@transformshift{1.015684in}{5.360584in}%
\pgfsys@useobject{currentmarker}{}%
\end{pgfscope}%
\begin{pgfscope}%
\pgfsys@transformshift{1.165882in}{5.277830in}%
\pgfsys@useobject{currentmarker}{}%
\end{pgfscope}%
\begin{pgfscope}%
\pgfsys@transformshift{1.316079in}{5.189698in}%
\pgfsys@useobject{currentmarker}{}%
\end{pgfscope}%
\begin{pgfscope}%
\pgfsys@transformshift{1.466277in}{5.093825in}%
\pgfsys@useobject{currentmarker}{}%
\end{pgfscope}%
\begin{pgfscope}%
\pgfsys@transformshift{1.616474in}{5.003605in}%
\pgfsys@useobject{currentmarker}{}%
\end{pgfscope}%
\begin{pgfscope}%
\pgfsys@transformshift{1.766671in}{4.905274in}%
\pgfsys@useobject{currentmarker}{}%
\end{pgfscope}%
\begin{pgfscope}%
\pgfsys@transformshift{1.916869in}{4.803853in}%
\pgfsys@useobject{currentmarker}{}%
\end{pgfscope}%
\begin{pgfscope}%
\pgfsys@transformshift{2.067066in}{4.712111in}%
\pgfsys@useobject{currentmarker}{}%
\end{pgfscope}%
\begin{pgfscope}%
\pgfsys@transformshift{2.217264in}{4.602472in}%
\pgfsys@useobject{currentmarker}{}%
\end{pgfscope}%
\begin{pgfscope}%
\pgfsys@transformshift{2.367461in}{4.493570in}%
\pgfsys@useobject{currentmarker}{}%
\end{pgfscope}%
\begin{pgfscope}%
\pgfsys@transformshift{2.517659in}{4.393553in}%
\pgfsys@useobject{currentmarker}{}%
\end{pgfscope}%
\begin{pgfscope}%
\pgfsys@transformshift{2.667856in}{4.274132in}%
\pgfsys@useobject{currentmarker}{}%
\end{pgfscope}%
\begin{pgfscope}%
\pgfsys@transformshift{2.818053in}{4.156835in}%
\pgfsys@useobject{currentmarker}{}%
\end{pgfscope}%
\begin{pgfscope}%
\pgfsys@transformshift{2.968251in}{4.043020in}%
\pgfsys@useobject{currentmarker}{}%
\end{pgfscope}%
\begin{pgfscope}%
\pgfsys@transformshift{3.118448in}{3.917554in}%
\pgfsys@useobject{currentmarker}{}%
\end{pgfscope}%
\begin{pgfscope}%
\pgfsys@transformshift{3.268646in}{3.803038in}%
\pgfsys@useobject{currentmarker}{}%
\end{pgfscope}%
\begin{pgfscope}%
\pgfsys@transformshift{3.418843in}{3.677500in}%
\pgfsys@useobject{currentmarker}{}%
\end{pgfscope}%
\begin{pgfscope}%
\pgfsys@transformshift{3.569040in}{3.570209in}%
\pgfsys@useobject{currentmarker}{}%
\end{pgfscope}%
\begin{pgfscope}%
\pgfsys@transformshift{3.719238in}{3.434063in}%
\pgfsys@useobject{currentmarker}{}%
\end{pgfscope}%
\end{pgfscope}%
\begin{pgfscope}%
\pgfsetbuttcap%
\pgfsetroundjoin%
\definecolor{currentfill}{rgb}{0.000000,0.000000,0.000000}%
\pgfsetfillcolor{currentfill}%
\pgfsetlinewidth{0.803000pt}%
\definecolor{currentstroke}{rgb}{0.000000,0.000000,0.000000}%
\pgfsetstrokecolor{currentstroke}%
\pgfsetdash{}{0pt}%
\pgfsys@defobject{currentmarker}{\pgfqpoint{0.000000in}{-0.048611in}}{\pgfqpoint{0.000000in}{0.000000in}}{%
\pgfpathmoveto{\pgfqpoint{0.000000in}{0.000000in}}%
\pgfpathlineto{\pgfqpoint{0.000000in}{-0.048611in}}%
\pgfusepath{stroke,fill}%
}%
\begin{pgfscope}%
\pgfsys@transformshift{0.565092in}{3.312222in}%
\pgfsys@useobject{currentmarker}{}%
\end{pgfscope}%
\end{pgfscope}%
\begin{pgfscope}%
\definecolor{textcolor}{rgb}{0.000000,0.000000,0.000000}%
\pgfsetstrokecolor{textcolor}%
\pgfsetfillcolor{textcolor}%
\pgftext[x=0.565092in,y=3.215000in,,top]{\color{textcolor}\sffamily\fontsize{10.000000}{12.000000}\selectfont 4.20}%
\end{pgfscope}%
\begin{pgfscope}%
\pgfsetbuttcap%
\pgfsetroundjoin%
\definecolor{currentfill}{rgb}{0.000000,0.000000,0.000000}%
\pgfsetfillcolor{currentfill}%
\pgfsetlinewidth{0.803000pt}%
\definecolor{currentstroke}{rgb}{0.000000,0.000000,0.000000}%
\pgfsetstrokecolor{currentstroke}%
\pgfsetdash{}{0pt}%
\pgfsys@defobject{currentmarker}{\pgfqpoint{0.000000in}{-0.048611in}}{\pgfqpoint{0.000000in}{0.000000in}}{%
\pgfpathmoveto{\pgfqpoint{0.000000in}{0.000000in}}%
\pgfpathlineto{\pgfqpoint{0.000000in}{-0.048611in}}%
\pgfusepath{stroke,fill}%
}%
\begin{pgfscope}%
\pgfsys@transformshift{1.316079in}{3.312222in}%
\pgfsys@useobject{currentmarker}{}%
\end{pgfscope}%
\end{pgfscope}%
\begin{pgfscope}%
\definecolor{textcolor}{rgb}{0.000000,0.000000,0.000000}%
\pgfsetstrokecolor{textcolor}%
\pgfsetfillcolor{textcolor}%
\pgftext[x=1.316079in,y=3.215000in,,top]{\color{textcolor}\sffamily\fontsize{10.000000}{12.000000}\selectfont 4.25}%
\end{pgfscope}%
\begin{pgfscope}%
\pgfsetbuttcap%
\pgfsetroundjoin%
\definecolor{currentfill}{rgb}{0.000000,0.000000,0.000000}%
\pgfsetfillcolor{currentfill}%
\pgfsetlinewidth{0.803000pt}%
\definecolor{currentstroke}{rgb}{0.000000,0.000000,0.000000}%
\pgfsetstrokecolor{currentstroke}%
\pgfsetdash{}{0pt}%
\pgfsys@defobject{currentmarker}{\pgfqpoint{0.000000in}{-0.048611in}}{\pgfqpoint{0.000000in}{0.000000in}}{%
\pgfpathmoveto{\pgfqpoint{0.000000in}{0.000000in}}%
\pgfpathlineto{\pgfqpoint{0.000000in}{-0.048611in}}%
\pgfusepath{stroke,fill}%
}%
\begin{pgfscope}%
\pgfsys@transformshift{2.067066in}{3.312222in}%
\pgfsys@useobject{currentmarker}{}%
\end{pgfscope}%
\end{pgfscope}%
\begin{pgfscope}%
\definecolor{textcolor}{rgb}{0.000000,0.000000,0.000000}%
\pgfsetstrokecolor{textcolor}%
\pgfsetfillcolor{textcolor}%
\pgftext[x=2.067066in,y=3.215000in,,top]{\color{textcolor}\sffamily\fontsize{10.000000}{12.000000}\selectfont 4.30}%
\end{pgfscope}%
\begin{pgfscope}%
\pgfsetbuttcap%
\pgfsetroundjoin%
\definecolor{currentfill}{rgb}{0.000000,0.000000,0.000000}%
\pgfsetfillcolor{currentfill}%
\pgfsetlinewidth{0.803000pt}%
\definecolor{currentstroke}{rgb}{0.000000,0.000000,0.000000}%
\pgfsetstrokecolor{currentstroke}%
\pgfsetdash{}{0pt}%
\pgfsys@defobject{currentmarker}{\pgfqpoint{0.000000in}{-0.048611in}}{\pgfqpoint{0.000000in}{0.000000in}}{%
\pgfpathmoveto{\pgfqpoint{0.000000in}{0.000000in}}%
\pgfpathlineto{\pgfqpoint{0.000000in}{-0.048611in}}%
\pgfusepath{stroke,fill}%
}%
\begin{pgfscope}%
\pgfsys@transformshift{2.818053in}{3.312222in}%
\pgfsys@useobject{currentmarker}{}%
\end{pgfscope}%
\end{pgfscope}%
\begin{pgfscope}%
\definecolor{textcolor}{rgb}{0.000000,0.000000,0.000000}%
\pgfsetstrokecolor{textcolor}%
\pgfsetfillcolor{textcolor}%
\pgftext[x=2.818053in,y=3.215000in,,top]{\color{textcolor}\sffamily\fontsize{10.000000}{12.000000}\selectfont 4.35}%
\end{pgfscope}%
\begin{pgfscope}%
\pgfsetbuttcap%
\pgfsetroundjoin%
\definecolor{currentfill}{rgb}{0.000000,0.000000,0.000000}%
\pgfsetfillcolor{currentfill}%
\pgfsetlinewidth{0.803000pt}%
\definecolor{currentstroke}{rgb}{0.000000,0.000000,0.000000}%
\pgfsetstrokecolor{currentstroke}%
\pgfsetdash{}{0pt}%
\pgfsys@defobject{currentmarker}{\pgfqpoint{0.000000in}{-0.048611in}}{\pgfqpoint{0.000000in}{0.000000in}}{%
\pgfpathmoveto{\pgfqpoint{0.000000in}{0.000000in}}%
\pgfpathlineto{\pgfqpoint{0.000000in}{-0.048611in}}%
\pgfusepath{stroke,fill}%
}%
\begin{pgfscope}%
\pgfsys@transformshift{3.569040in}{3.312222in}%
\pgfsys@useobject{currentmarker}{}%
\end{pgfscope}%
\end{pgfscope}%
\begin{pgfscope}%
\definecolor{textcolor}{rgb}{0.000000,0.000000,0.000000}%
\pgfsetstrokecolor{textcolor}%
\pgfsetfillcolor{textcolor}%
\pgftext[x=3.569040in,y=3.215000in,,top]{\color{textcolor}\sffamily\fontsize{10.000000}{12.000000}\selectfont 4.40}%
\end{pgfscope}%
\begin{pgfscope}%
\pgfsetbuttcap%
\pgfsetroundjoin%
\definecolor{currentfill}{rgb}{0.000000,0.000000,0.000000}%
\pgfsetfillcolor{currentfill}%
\pgfsetlinewidth{0.803000pt}%
\definecolor{currentstroke}{rgb}{0.000000,0.000000,0.000000}%
\pgfsetstrokecolor{currentstroke}%
\pgfsetdash{}{0pt}%
\pgfsys@defobject{currentmarker}{\pgfqpoint{-0.048611in}{0.000000in}}{\pgfqpoint{0.000000in}{0.000000in}}{%
\pgfpathmoveto{\pgfqpoint{0.000000in}{0.000000in}}%
\pgfpathlineto{\pgfqpoint{-0.048611in}{0.000000in}}%
\pgfusepath{stroke,fill}%
}%
\begin{pgfscope}%
\pgfsys@transformshift{0.555972in}{3.430873in}%
\pgfsys@useobject{currentmarker}{}%
\end{pgfscope}%
\end{pgfscope}%
\begin{pgfscope}%
\definecolor{textcolor}{rgb}{0.000000,0.000000,0.000000}%
\pgfsetstrokecolor{textcolor}%
\pgfsetfillcolor{textcolor}%
\pgftext[x=0.061140in,y=3.378111in,left,base]{\color{textcolor}\sffamily\fontsize{10.000000}{12.000000}\selectfont 0.475}%
\end{pgfscope}%
\begin{pgfscope}%
\pgfsetbuttcap%
\pgfsetroundjoin%
\definecolor{currentfill}{rgb}{0.000000,0.000000,0.000000}%
\pgfsetfillcolor{currentfill}%
\pgfsetlinewidth{0.803000pt}%
\definecolor{currentstroke}{rgb}{0.000000,0.000000,0.000000}%
\pgfsetstrokecolor{currentstroke}%
\pgfsetdash{}{0pt}%
\pgfsys@defobject{currentmarker}{\pgfqpoint{-0.048611in}{0.000000in}}{\pgfqpoint{0.000000in}{0.000000in}}{%
\pgfpathmoveto{\pgfqpoint{0.000000in}{0.000000in}}%
\pgfpathlineto{\pgfqpoint{-0.048611in}{0.000000in}}%
\pgfusepath{stroke,fill}%
}%
\begin{pgfscope}%
\pgfsys@transformshift{0.555972in}{3.735526in}%
\pgfsys@useobject{currentmarker}{}%
\end{pgfscope}%
\end{pgfscope}%
\begin{pgfscope}%
\definecolor{textcolor}{rgb}{0.000000,0.000000,0.000000}%
\pgfsetstrokecolor{textcolor}%
\pgfsetfillcolor{textcolor}%
\pgftext[x=0.061140in,y=3.682764in,left,base]{\color{textcolor}\sffamily\fontsize{10.000000}{12.000000}\selectfont 0.500}%
\end{pgfscope}%
\begin{pgfscope}%
\pgfsetbuttcap%
\pgfsetroundjoin%
\definecolor{currentfill}{rgb}{0.000000,0.000000,0.000000}%
\pgfsetfillcolor{currentfill}%
\pgfsetlinewidth{0.803000pt}%
\definecolor{currentstroke}{rgb}{0.000000,0.000000,0.000000}%
\pgfsetstrokecolor{currentstroke}%
\pgfsetdash{}{0pt}%
\pgfsys@defobject{currentmarker}{\pgfqpoint{-0.048611in}{0.000000in}}{\pgfqpoint{0.000000in}{0.000000in}}{%
\pgfpathmoveto{\pgfqpoint{0.000000in}{0.000000in}}%
\pgfpathlineto{\pgfqpoint{-0.048611in}{0.000000in}}%
\pgfusepath{stroke,fill}%
}%
\begin{pgfscope}%
\pgfsys@transformshift{0.555972in}{4.040178in}%
\pgfsys@useobject{currentmarker}{}%
\end{pgfscope}%
\end{pgfscope}%
\begin{pgfscope}%
\definecolor{textcolor}{rgb}{0.000000,0.000000,0.000000}%
\pgfsetstrokecolor{textcolor}%
\pgfsetfillcolor{textcolor}%
\pgftext[x=0.061140in,y=3.987417in,left,base]{\color{textcolor}\sffamily\fontsize{10.000000}{12.000000}\selectfont 0.525}%
\end{pgfscope}%
\begin{pgfscope}%
\pgfsetbuttcap%
\pgfsetroundjoin%
\definecolor{currentfill}{rgb}{0.000000,0.000000,0.000000}%
\pgfsetfillcolor{currentfill}%
\pgfsetlinewidth{0.803000pt}%
\definecolor{currentstroke}{rgb}{0.000000,0.000000,0.000000}%
\pgfsetstrokecolor{currentstroke}%
\pgfsetdash{}{0pt}%
\pgfsys@defobject{currentmarker}{\pgfqpoint{-0.048611in}{0.000000in}}{\pgfqpoint{0.000000in}{0.000000in}}{%
\pgfpathmoveto{\pgfqpoint{0.000000in}{0.000000in}}%
\pgfpathlineto{\pgfqpoint{-0.048611in}{0.000000in}}%
\pgfusepath{stroke,fill}%
}%
\begin{pgfscope}%
\pgfsys@transformshift{0.555972in}{4.344831in}%
\pgfsys@useobject{currentmarker}{}%
\end{pgfscope}%
\end{pgfscope}%
\begin{pgfscope}%
\definecolor{textcolor}{rgb}{0.000000,0.000000,0.000000}%
\pgfsetstrokecolor{textcolor}%
\pgfsetfillcolor{textcolor}%
\pgftext[x=0.061140in,y=4.292069in,left,base]{\color{textcolor}\sffamily\fontsize{10.000000}{12.000000}\selectfont 0.550}%
\end{pgfscope}%
\begin{pgfscope}%
\pgfsetbuttcap%
\pgfsetroundjoin%
\definecolor{currentfill}{rgb}{0.000000,0.000000,0.000000}%
\pgfsetfillcolor{currentfill}%
\pgfsetlinewidth{0.803000pt}%
\definecolor{currentstroke}{rgb}{0.000000,0.000000,0.000000}%
\pgfsetstrokecolor{currentstroke}%
\pgfsetdash{}{0pt}%
\pgfsys@defobject{currentmarker}{\pgfqpoint{-0.048611in}{0.000000in}}{\pgfqpoint{0.000000in}{0.000000in}}{%
\pgfpathmoveto{\pgfqpoint{0.000000in}{0.000000in}}%
\pgfpathlineto{\pgfqpoint{-0.048611in}{0.000000in}}%
\pgfusepath{stroke,fill}%
}%
\begin{pgfscope}%
\pgfsys@transformshift{0.555972in}{4.649483in}%
\pgfsys@useobject{currentmarker}{}%
\end{pgfscope}%
\end{pgfscope}%
\begin{pgfscope}%
\definecolor{textcolor}{rgb}{0.000000,0.000000,0.000000}%
\pgfsetstrokecolor{textcolor}%
\pgfsetfillcolor{textcolor}%
\pgftext[x=0.061140in,y=4.596722in,left,base]{\color{textcolor}\sffamily\fontsize{10.000000}{12.000000}\selectfont 0.575}%
\end{pgfscope}%
\begin{pgfscope}%
\pgfsetbuttcap%
\pgfsetroundjoin%
\definecolor{currentfill}{rgb}{0.000000,0.000000,0.000000}%
\pgfsetfillcolor{currentfill}%
\pgfsetlinewidth{0.803000pt}%
\definecolor{currentstroke}{rgb}{0.000000,0.000000,0.000000}%
\pgfsetstrokecolor{currentstroke}%
\pgfsetdash{}{0pt}%
\pgfsys@defobject{currentmarker}{\pgfqpoint{-0.048611in}{0.000000in}}{\pgfqpoint{0.000000in}{0.000000in}}{%
\pgfpathmoveto{\pgfqpoint{0.000000in}{0.000000in}}%
\pgfpathlineto{\pgfqpoint{-0.048611in}{0.000000in}}%
\pgfusepath{stroke,fill}%
}%
\begin{pgfscope}%
\pgfsys@transformshift{0.555972in}{4.954136in}%
\pgfsys@useobject{currentmarker}{}%
\end{pgfscope}%
\end{pgfscope}%
\begin{pgfscope}%
\definecolor{textcolor}{rgb}{0.000000,0.000000,0.000000}%
\pgfsetstrokecolor{textcolor}%
\pgfsetfillcolor{textcolor}%
\pgftext[x=0.061140in,y=4.901374in,left,base]{\color{textcolor}\sffamily\fontsize{10.000000}{12.000000}\selectfont 0.600}%
\end{pgfscope}%
\begin{pgfscope}%
\pgfsetbuttcap%
\pgfsetroundjoin%
\definecolor{currentfill}{rgb}{0.000000,0.000000,0.000000}%
\pgfsetfillcolor{currentfill}%
\pgfsetlinewidth{0.803000pt}%
\definecolor{currentstroke}{rgb}{0.000000,0.000000,0.000000}%
\pgfsetstrokecolor{currentstroke}%
\pgfsetdash{}{0pt}%
\pgfsys@defobject{currentmarker}{\pgfqpoint{-0.048611in}{0.000000in}}{\pgfqpoint{0.000000in}{0.000000in}}{%
\pgfpathmoveto{\pgfqpoint{0.000000in}{0.000000in}}%
\pgfpathlineto{\pgfqpoint{-0.048611in}{0.000000in}}%
\pgfusepath{stroke,fill}%
}%
\begin{pgfscope}%
\pgfsys@transformshift{0.555972in}{5.258789in}%
\pgfsys@useobject{currentmarker}{}%
\end{pgfscope}%
\end{pgfscope}%
\begin{pgfscope}%
\definecolor{textcolor}{rgb}{0.000000,0.000000,0.000000}%
\pgfsetstrokecolor{textcolor}%
\pgfsetfillcolor{textcolor}%
\pgftext[x=0.061140in,y=5.206027in,left,base]{\color{textcolor}\sffamily\fontsize{10.000000}{12.000000}\selectfont 0.625}%
\end{pgfscope}%
\begin{pgfscope}%
\pgfsetbuttcap%
\pgfsetroundjoin%
\definecolor{currentfill}{rgb}{0.000000,0.000000,0.000000}%
\pgfsetfillcolor{currentfill}%
\pgfsetlinewidth{0.803000pt}%
\definecolor{currentstroke}{rgb}{0.000000,0.000000,0.000000}%
\pgfsetstrokecolor{currentstroke}%
\pgfsetdash{}{0pt}%
\pgfsys@defobject{currentmarker}{\pgfqpoint{-0.048611in}{0.000000in}}{\pgfqpoint{0.000000in}{0.000000in}}{%
\pgfpathmoveto{\pgfqpoint{0.000000in}{0.000000in}}%
\pgfpathlineto{\pgfqpoint{-0.048611in}{0.000000in}}%
\pgfusepath{stroke,fill}%
}%
\begin{pgfscope}%
\pgfsys@transformshift{0.555972in}{5.563441in}%
\pgfsys@useobject{currentmarker}{}%
\end{pgfscope}%
\end{pgfscope}%
\begin{pgfscope}%
\definecolor{textcolor}{rgb}{0.000000,0.000000,0.000000}%
\pgfsetstrokecolor{textcolor}%
\pgfsetfillcolor{textcolor}%
\pgftext[x=0.061140in,y=5.510680in,left,base]{\color{textcolor}\sffamily\fontsize{10.000000}{12.000000}\selectfont 0.650}%
\end{pgfscope}%
\begin{pgfscope}%
\pgfpathrectangle{\pgfqpoint{0.555972in}{3.312222in}}{\pgfqpoint{3.322583in}{2.324444in}}%
\pgfusepath{clip}%
\pgfsetrectcap%
\pgfsetroundjoin%
\pgfsetlinewidth{1.505625pt}%
\definecolor{currentstroke}{rgb}{0.121569,0.466667,0.705882}%
\pgfsetstrokecolor{currentstroke}%
\pgfsetdash{}{0pt}%
\pgfpathmoveto{\pgfqpoint{0.715290in}{5.521307in}}%
\pgfpathlineto{\pgfqpoint{0.865487in}{5.445464in}}%
\pgfpathlineto{\pgfqpoint{1.015684in}{5.360584in}}%
\pgfpathlineto{\pgfqpoint{1.165882in}{5.277830in}}%
\pgfpathlineto{\pgfqpoint{1.316079in}{5.189698in}}%
\pgfpathlineto{\pgfqpoint{1.466277in}{5.093825in}}%
\pgfpathlineto{\pgfqpoint{1.616474in}{5.003605in}}%
\pgfpathlineto{\pgfqpoint{1.766671in}{4.905274in}}%
\pgfpathlineto{\pgfqpoint{1.916869in}{4.803853in}}%
\pgfpathlineto{\pgfqpoint{2.067066in}{4.712111in}}%
\pgfpathlineto{\pgfqpoint{2.217264in}{4.602472in}}%
\pgfpathlineto{\pgfqpoint{2.367461in}{4.493570in}}%
\pgfpathlineto{\pgfqpoint{2.517659in}{4.393553in}}%
\pgfpathlineto{\pgfqpoint{2.667856in}{4.274132in}}%
\pgfpathlineto{\pgfqpoint{2.818053in}{4.156835in}}%
\pgfpathlineto{\pgfqpoint{2.968251in}{4.043020in}}%
\pgfpathlineto{\pgfqpoint{3.118448in}{3.917554in}}%
\pgfpathlineto{\pgfqpoint{3.268646in}{3.803038in}}%
\pgfpathlineto{\pgfqpoint{3.418843in}{3.677500in}}%
\pgfpathlineto{\pgfqpoint{3.569040in}{3.570209in}}%
\pgfpathlineto{\pgfqpoint{3.719238in}{3.434063in}}%
\pgfusepath{stroke}%
\end{pgfscope}%
\begin{pgfscope}%
\pgfsetrectcap%
\pgfsetmiterjoin%
\pgfsetlinewidth{0.803000pt}%
\definecolor{currentstroke}{rgb}{0.000000,0.000000,0.000000}%
\pgfsetstrokecolor{currentstroke}%
\pgfsetdash{}{0pt}%
\pgfpathmoveto{\pgfqpoint{0.555972in}{3.312222in}}%
\pgfpathlineto{\pgfqpoint{0.555972in}{5.636667in}}%
\pgfusepath{stroke}%
\end{pgfscope}%
\begin{pgfscope}%
\pgfsetrectcap%
\pgfsetmiterjoin%
\pgfsetlinewidth{0.803000pt}%
\definecolor{currentstroke}{rgb}{0.000000,0.000000,0.000000}%
\pgfsetstrokecolor{currentstroke}%
\pgfsetdash{}{0pt}%
\pgfpathmoveto{\pgfqpoint{3.878555in}{3.312222in}}%
\pgfpathlineto{\pgfqpoint{3.878555in}{5.636667in}}%
\pgfusepath{stroke}%
\end{pgfscope}%
\begin{pgfscope}%
\pgfsetrectcap%
\pgfsetmiterjoin%
\pgfsetlinewidth{0.803000pt}%
\definecolor{currentstroke}{rgb}{0.000000,0.000000,0.000000}%
\pgfsetstrokecolor{currentstroke}%
\pgfsetdash{}{0pt}%
\pgfpathmoveto{\pgfqpoint{0.555972in}{3.312222in}}%
\pgfpathlineto{\pgfqpoint{3.878555in}{3.312222in}}%
\pgfusepath{stroke}%
\end{pgfscope}%
\begin{pgfscope}%
\pgfsetrectcap%
\pgfsetmiterjoin%
\pgfsetlinewidth{0.803000pt}%
\definecolor{currentstroke}{rgb}{0.000000,0.000000,0.000000}%
\pgfsetstrokecolor{currentstroke}%
\pgfsetdash{}{0pt}%
\pgfpathmoveto{\pgfqpoint{0.555972in}{5.636667in}}%
\pgfpathlineto{\pgfqpoint{3.878555in}{5.636667in}}%
\pgfusepath{stroke}%
\end{pgfscope}%
\begin{pgfscope}%
\definecolor{textcolor}{rgb}{0.000000,0.000000,0.000000}%
\pgfsetstrokecolor{textcolor}%
\pgfsetfillcolor{textcolor}%
\pgftext[x=2.217264in,y=5.720000in,,base]{\color{textcolor}\sffamily\fontsize{12.000000}{14.400000}\selectfont \(\displaystyle  N = 8 \)}%
\end{pgfscope}%
\begin{pgfscope}%
\pgfsetbuttcap%
\pgfsetmiterjoin%
\definecolor{currentfill}{rgb}{1.000000,1.000000,1.000000}%
\pgfsetfillcolor{currentfill}%
\pgfsetlinewidth{0.000000pt}%
\definecolor{currentstroke}{rgb}{0.000000,0.000000,0.000000}%
\pgfsetstrokecolor{currentstroke}%
\pgfsetstrokeopacity{0.000000}%
\pgfsetdash{}{0pt}%
\pgfpathmoveto{\pgfqpoint{4.492417in}{3.312222in}}%
\pgfpathlineto{\pgfqpoint{7.815000in}{3.312222in}}%
\pgfpathlineto{\pgfqpoint{7.815000in}{5.636667in}}%
\pgfpathlineto{\pgfqpoint{4.492417in}{5.636667in}}%
\pgfpathclose%
\pgfusepath{fill}%
\end{pgfscope}%
\begin{pgfscope}%
\pgfpathrectangle{\pgfqpoint{4.492417in}{3.312222in}}{\pgfqpoint{3.322583in}{2.324444in}}%
\pgfusepath{clip}%
\pgfsetbuttcap%
\pgfsetroundjoin%
\definecolor{currentfill}{rgb}{0.121569,0.466667,0.705882}%
\pgfsetfillcolor{currentfill}%
\pgfsetfillopacity{0.300000}%
\pgfsetlinewidth{1.003750pt}%
\definecolor{currentstroke}{rgb}{0.121569,0.466667,0.705882}%
\pgfsetstrokecolor{currentstroke}%
\pgfsetstrokeopacity{0.300000}%
\pgfsetdash{}{0pt}%
\pgfpathmoveto{\pgfqpoint{4.651734in}{5.531010in}}%
\pgfpathlineto{\pgfqpoint{4.651734in}{5.510067in}}%
\pgfpathlineto{\pgfqpoint{4.801932in}{5.432364in}}%
\pgfpathlineto{\pgfqpoint{4.952129in}{5.345794in}}%
\pgfpathlineto{\pgfqpoint{5.102327in}{5.285522in}}%
\pgfpathlineto{\pgfqpoint{5.252524in}{5.209687in}}%
\pgfpathlineto{\pgfqpoint{5.402721in}{5.106223in}}%
\pgfpathlineto{\pgfqpoint{5.552919in}{5.038321in}}%
\pgfpathlineto{\pgfqpoint{5.703116in}{4.934893in}}%
\pgfpathlineto{\pgfqpoint{5.853314in}{4.828058in}}%
\pgfpathlineto{\pgfqpoint{6.003511in}{4.738035in}}%
\pgfpathlineto{\pgfqpoint{6.153708in}{4.621380in}}%
\pgfpathlineto{\pgfqpoint{6.303906in}{4.508049in}}%
\pgfpathlineto{\pgfqpoint{6.454103in}{4.359312in}}%
\pgfpathlineto{\pgfqpoint{6.604301in}{4.286396in}}%
\pgfpathlineto{\pgfqpoint{6.754498in}{4.143292in}}%
\pgfpathlineto{\pgfqpoint{6.904696in}{4.024149in}}%
\pgfpathlineto{\pgfqpoint{7.054893in}{3.879464in}}%
\pgfpathlineto{\pgfqpoint{7.205090in}{3.749283in}}%
\pgfpathlineto{\pgfqpoint{7.355288in}{3.670456in}}%
\pgfpathlineto{\pgfqpoint{7.505485in}{3.541117in}}%
\pgfpathlineto{\pgfqpoint{7.655683in}{3.417879in}}%
\pgfpathlineto{\pgfqpoint{7.655683in}{3.496912in}}%
\pgfpathlineto{\pgfqpoint{7.655683in}{3.496912in}}%
\pgfpathlineto{\pgfqpoint{7.505485in}{3.586917in}}%
\pgfpathlineto{\pgfqpoint{7.355288in}{3.684989in}}%
\pgfpathlineto{\pgfqpoint{7.205090in}{3.803047in}}%
\pgfpathlineto{\pgfqpoint{7.054893in}{3.957739in}}%
\pgfpathlineto{\pgfqpoint{6.904696in}{4.095644in}}%
\pgfpathlineto{\pgfqpoint{6.754498in}{4.196719in}}%
\pgfpathlineto{\pgfqpoint{6.604301in}{4.317837in}}%
\pgfpathlineto{\pgfqpoint{6.454103in}{4.423195in}}%
\pgfpathlineto{\pgfqpoint{6.303906in}{4.537561in}}%
\pgfpathlineto{\pgfqpoint{6.153708in}{4.635690in}}%
\pgfpathlineto{\pgfqpoint{6.003511in}{4.780569in}}%
\pgfpathlineto{\pgfqpoint{5.853314in}{4.866208in}}%
\pgfpathlineto{\pgfqpoint{5.703116in}{4.959398in}}%
\pgfpathlineto{\pgfqpoint{5.552919in}{5.059418in}}%
\pgfpathlineto{\pgfqpoint{5.402721in}{5.146321in}}%
\pgfpathlineto{\pgfqpoint{5.252524in}{5.228964in}}%
\pgfpathlineto{\pgfqpoint{5.102327in}{5.309877in}}%
\pgfpathlineto{\pgfqpoint{4.952129in}{5.381861in}}%
\pgfpathlineto{\pgfqpoint{4.801932in}{5.448673in}}%
\pgfpathlineto{\pgfqpoint{4.651734in}{5.531010in}}%
\pgfpathclose%
\pgfusepath{stroke,fill}%
\end{pgfscope}%
\begin{pgfscope}%
\pgfpathrectangle{\pgfqpoint{4.492417in}{3.312222in}}{\pgfqpoint{3.322583in}{2.324444in}}%
\pgfusepath{clip}%
\pgfsetbuttcap%
\pgfsetroundjoin%
\definecolor{currentfill}{rgb}{0.121569,0.466667,0.705882}%
\pgfsetfillcolor{currentfill}%
\pgfsetlinewidth{1.003750pt}%
\definecolor{currentstroke}{rgb}{0.121569,0.466667,0.705882}%
\pgfsetstrokecolor{currentstroke}%
\pgfsetdash{}{0pt}%
\pgfsys@defobject{currentmarker}{\pgfqpoint{-0.009821in}{-0.009821in}}{\pgfqpoint{0.009821in}{0.009821in}}{%
\pgfpathmoveto{\pgfqpoint{0.000000in}{-0.009821in}}%
\pgfpathcurveto{\pgfqpoint{0.002605in}{-0.009821in}}{\pgfqpoint{0.005103in}{-0.008786in}}{\pgfqpoint{0.006944in}{-0.006944in}}%
\pgfpathcurveto{\pgfqpoint{0.008786in}{-0.005103in}}{\pgfqpoint{0.009821in}{-0.002605in}}{\pgfqpoint{0.009821in}{0.000000in}}%
\pgfpathcurveto{\pgfqpoint{0.009821in}{0.002605in}}{\pgfqpoint{0.008786in}{0.005103in}}{\pgfqpoint{0.006944in}{0.006944in}}%
\pgfpathcurveto{\pgfqpoint{0.005103in}{0.008786in}}{\pgfqpoint{0.002605in}{0.009821in}}{\pgfqpoint{0.000000in}{0.009821in}}%
\pgfpathcurveto{\pgfqpoint{-0.002605in}{0.009821in}}{\pgfqpoint{-0.005103in}{0.008786in}}{\pgfqpoint{-0.006944in}{0.006944in}}%
\pgfpathcurveto{\pgfqpoint{-0.008786in}{0.005103in}}{\pgfqpoint{-0.009821in}{0.002605in}}{\pgfqpoint{-0.009821in}{0.000000in}}%
\pgfpathcurveto{\pgfqpoint{-0.009821in}{-0.002605in}}{\pgfqpoint{-0.008786in}{-0.005103in}}{\pgfqpoint{-0.006944in}{-0.006944in}}%
\pgfpathcurveto{\pgfqpoint{-0.005103in}{-0.008786in}}{\pgfqpoint{-0.002605in}{-0.009821in}}{\pgfqpoint{0.000000in}{-0.009821in}}%
\pgfpathclose%
\pgfusepath{stroke,fill}%
}%
\begin{pgfscope}%
\pgfsys@transformshift{4.651734in}{5.520538in}%
\pgfsys@useobject{currentmarker}{}%
\end{pgfscope}%
\begin{pgfscope}%
\pgfsys@transformshift{4.801932in}{5.440519in}%
\pgfsys@useobject{currentmarker}{}%
\end{pgfscope}%
\begin{pgfscope}%
\pgfsys@transformshift{4.952129in}{5.363827in}%
\pgfsys@useobject{currentmarker}{}%
\end{pgfscope}%
\begin{pgfscope}%
\pgfsys@transformshift{5.102327in}{5.297699in}%
\pgfsys@useobject{currentmarker}{}%
\end{pgfscope}%
\begin{pgfscope}%
\pgfsys@transformshift{5.252524in}{5.219326in}%
\pgfsys@useobject{currentmarker}{}%
\end{pgfscope}%
\begin{pgfscope}%
\pgfsys@transformshift{5.402721in}{5.126272in}%
\pgfsys@useobject{currentmarker}{}%
\end{pgfscope}%
\begin{pgfscope}%
\pgfsys@transformshift{5.552919in}{5.048869in}%
\pgfsys@useobject{currentmarker}{}%
\end{pgfscope}%
\begin{pgfscope}%
\pgfsys@transformshift{5.703116in}{4.947146in}%
\pgfsys@useobject{currentmarker}{}%
\end{pgfscope}%
\begin{pgfscope}%
\pgfsys@transformshift{5.853314in}{4.847133in}%
\pgfsys@useobject{currentmarker}{}%
\end{pgfscope}%
\begin{pgfscope}%
\pgfsys@transformshift{6.003511in}{4.759302in}%
\pgfsys@useobject{currentmarker}{}%
\end{pgfscope}%
\begin{pgfscope}%
\pgfsys@transformshift{6.153708in}{4.628535in}%
\pgfsys@useobject{currentmarker}{}%
\end{pgfscope}%
\begin{pgfscope}%
\pgfsys@transformshift{6.303906in}{4.522805in}%
\pgfsys@useobject{currentmarker}{}%
\end{pgfscope}%
\begin{pgfscope}%
\pgfsys@transformshift{6.454103in}{4.391254in}%
\pgfsys@useobject{currentmarker}{}%
\end{pgfscope}%
\begin{pgfscope}%
\pgfsys@transformshift{6.604301in}{4.302116in}%
\pgfsys@useobject{currentmarker}{}%
\end{pgfscope}%
\begin{pgfscope}%
\pgfsys@transformshift{6.754498in}{4.170006in}%
\pgfsys@useobject{currentmarker}{}%
\end{pgfscope}%
\begin{pgfscope}%
\pgfsys@transformshift{6.904696in}{4.059897in}%
\pgfsys@useobject{currentmarker}{}%
\end{pgfscope}%
\begin{pgfscope}%
\pgfsys@transformshift{7.054893in}{3.918601in}%
\pgfsys@useobject{currentmarker}{}%
\end{pgfscope}%
\begin{pgfscope}%
\pgfsys@transformshift{7.205090in}{3.776165in}%
\pgfsys@useobject{currentmarker}{}%
\end{pgfscope}%
\begin{pgfscope}%
\pgfsys@transformshift{7.355288in}{3.677723in}%
\pgfsys@useobject{currentmarker}{}%
\end{pgfscope}%
\begin{pgfscope}%
\pgfsys@transformshift{7.505485in}{3.564017in}%
\pgfsys@useobject{currentmarker}{}%
\end{pgfscope}%
\begin{pgfscope}%
\pgfsys@transformshift{7.655683in}{3.457395in}%
\pgfsys@useobject{currentmarker}{}%
\end{pgfscope}%
\end{pgfscope}%
\begin{pgfscope}%
\pgfsetbuttcap%
\pgfsetroundjoin%
\definecolor{currentfill}{rgb}{0.000000,0.000000,0.000000}%
\pgfsetfillcolor{currentfill}%
\pgfsetlinewidth{0.803000pt}%
\definecolor{currentstroke}{rgb}{0.000000,0.000000,0.000000}%
\pgfsetstrokecolor{currentstroke}%
\pgfsetdash{}{0pt}%
\pgfsys@defobject{currentmarker}{\pgfqpoint{0.000000in}{-0.048611in}}{\pgfqpoint{0.000000in}{0.000000in}}{%
\pgfpathmoveto{\pgfqpoint{0.000000in}{0.000000in}}%
\pgfpathlineto{\pgfqpoint{0.000000in}{-0.048611in}}%
\pgfusepath{stroke,fill}%
}%
\begin{pgfscope}%
\pgfsys@transformshift{4.501537in}{3.312222in}%
\pgfsys@useobject{currentmarker}{}%
\end{pgfscope}%
\end{pgfscope}%
\begin{pgfscope}%
\definecolor{textcolor}{rgb}{0.000000,0.000000,0.000000}%
\pgfsetstrokecolor{textcolor}%
\pgfsetfillcolor{textcolor}%
\pgftext[x=4.501537in,y=3.215000in,,top]{\color{textcolor}\sffamily\fontsize{10.000000}{12.000000}\selectfont 4.30}%
\end{pgfscope}%
\begin{pgfscope}%
\pgfsetbuttcap%
\pgfsetroundjoin%
\definecolor{currentfill}{rgb}{0.000000,0.000000,0.000000}%
\pgfsetfillcolor{currentfill}%
\pgfsetlinewidth{0.803000pt}%
\definecolor{currentstroke}{rgb}{0.000000,0.000000,0.000000}%
\pgfsetstrokecolor{currentstroke}%
\pgfsetdash{}{0pt}%
\pgfsys@defobject{currentmarker}{\pgfqpoint{0.000000in}{-0.048611in}}{\pgfqpoint{0.000000in}{0.000000in}}{%
\pgfpathmoveto{\pgfqpoint{0.000000in}{0.000000in}}%
\pgfpathlineto{\pgfqpoint{0.000000in}{-0.048611in}}%
\pgfusepath{stroke,fill}%
}%
\begin{pgfscope}%
\pgfsys@transformshift{5.252524in}{3.312222in}%
\pgfsys@useobject{currentmarker}{}%
\end{pgfscope}%
\end{pgfscope}%
\begin{pgfscope}%
\definecolor{textcolor}{rgb}{0.000000,0.000000,0.000000}%
\pgfsetstrokecolor{textcolor}%
\pgfsetfillcolor{textcolor}%
\pgftext[x=5.252524in,y=3.215000in,,top]{\color{textcolor}\sffamily\fontsize{10.000000}{12.000000}\selectfont 4.35}%
\end{pgfscope}%
\begin{pgfscope}%
\pgfsetbuttcap%
\pgfsetroundjoin%
\definecolor{currentfill}{rgb}{0.000000,0.000000,0.000000}%
\pgfsetfillcolor{currentfill}%
\pgfsetlinewidth{0.803000pt}%
\definecolor{currentstroke}{rgb}{0.000000,0.000000,0.000000}%
\pgfsetstrokecolor{currentstroke}%
\pgfsetdash{}{0pt}%
\pgfsys@defobject{currentmarker}{\pgfqpoint{0.000000in}{-0.048611in}}{\pgfqpoint{0.000000in}{0.000000in}}{%
\pgfpathmoveto{\pgfqpoint{0.000000in}{0.000000in}}%
\pgfpathlineto{\pgfqpoint{0.000000in}{-0.048611in}}%
\pgfusepath{stroke,fill}%
}%
\begin{pgfscope}%
\pgfsys@transformshift{6.003511in}{3.312222in}%
\pgfsys@useobject{currentmarker}{}%
\end{pgfscope}%
\end{pgfscope}%
\begin{pgfscope}%
\definecolor{textcolor}{rgb}{0.000000,0.000000,0.000000}%
\pgfsetstrokecolor{textcolor}%
\pgfsetfillcolor{textcolor}%
\pgftext[x=6.003511in,y=3.215000in,,top]{\color{textcolor}\sffamily\fontsize{10.000000}{12.000000}\selectfont 4.40}%
\end{pgfscope}%
\begin{pgfscope}%
\pgfsetbuttcap%
\pgfsetroundjoin%
\definecolor{currentfill}{rgb}{0.000000,0.000000,0.000000}%
\pgfsetfillcolor{currentfill}%
\pgfsetlinewidth{0.803000pt}%
\definecolor{currentstroke}{rgb}{0.000000,0.000000,0.000000}%
\pgfsetstrokecolor{currentstroke}%
\pgfsetdash{}{0pt}%
\pgfsys@defobject{currentmarker}{\pgfqpoint{0.000000in}{-0.048611in}}{\pgfqpoint{0.000000in}{0.000000in}}{%
\pgfpathmoveto{\pgfqpoint{0.000000in}{0.000000in}}%
\pgfpathlineto{\pgfqpoint{0.000000in}{-0.048611in}}%
\pgfusepath{stroke,fill}%
}%
\begin{pgfscope}%
\pgfsys@transformshift{6.754498in}{3.312222in}%
\pgfsys@useobject{currentmarker}{}%
\end{pgfscope}%
\end{pgfscope}%
\begin{pgfscope}%
\definecolor{textcolor}{rgb}{0.000000,0.000000,0.000000}%
\pgfsetstrokecolor{textcolor}%
\pgfsetfillcolor{textcolor}%
\pgftext[x=6.754498in,y=3.215000in,,top]{\color{textcolor}\sffamily\fontsize{10.000000}{12.000000}\selectfont 4.45}%
\end{pgfscope}%
\begin{pgfscope}%
\pgfsetbuttcap%
\pgfsetroundjoin%
\definecolor{currentfill}{rgb}{0.000000,0.000000,0.000000}%
\pgfsetfillcolor{currentfill}%
\pgfsetlinewidth{0.803000pt}%
\definecolor{currentstroke}{rgb}{0.000000,0.000000,0.000000}%
\pgfsetstrokecolor{currentstroke}%
\pgfsetdash{}{0pt}%
\pgfsys@defobject{currentmarker}{\pgfqpoint{0.000000in}{-0.048611in}}{\pgfqpoint{0.000000in}{0.000000in}}{%
\pgfpathmoveto{\pgfqpoint{0.000000in}{0.000000in}}%
\pgfpathlineto{\pgfqpoint{0.000000in}{-0.048611in}}%
\pgfusepath{stroke,fill}%
}%
\begin{pgfscope}%
\pgfsys@transformshift{7.505485in}{3.312222in}%
\pgfsys@useobject{currentmarker}{}%
\end{pgfscope}%
\end{pgfscope}%
\begin{pgfscope}%
\definecolor{textcolor}{rgb}{0.000000,0.000000,0.000000}%
\pgfsetstrokecolor{textcolor}%
\pgfsetfillcolor{textcolor}%
\pgftext[x=7.505485in,y=3.215000in,,top]{\color{textcolor}\sffamily\fontsize{10.000000}{12.000000}\selectfont 4.50}%
\end{pgfscope}%
\begin{pgfscope}%
\pgfsetbuttcap%
\pgfsetroundjoin%
\definecolor{currentfill}{rgb}{0.000000,0.000000,0.000000}%
\pgfsetfillcolor{currentfill}%
\pgfsetlinewidth{0.803000pt}%
\definecolor{currentstroke}{rgb}{0.000000,0.000000,0.000000}%
\pgfsetstrokecolor{currentstroke}%
\pgfsetdash{}{0pt}%
\pgfsys@defobject{currentmarker}{\pgfqpoint{-0.048611in}{0.000000in}}{\pgfqpoint{0.000000in}{0.000000in}}{%
\pgfpathmoveto{\pgfqpoint{0.000000in}{0.000000in}}%
\pgfpathlineto{\pgfqpoint{-0.048611in}{0.000000in}}%
\pgfusepath{stroke,fill}%
}%
\begin{pgfscope}%
\pgfsys@transformshift{4.492417in}{3.387190in}%
\pgfsys@useobject{currentmarker}{}%
\end{pgfscope}%
\end{pgfscope}%
\begin{pgfscope}%
\definecolor{textcolor}{rgb}{0.000000,0.000000,0.000000}%
\pgfsetstrokecolor{textcolor}%
\pgfsetfillcolor{textcolor}%
\pgftext[x=4.085950in,y=3.334429in,left,base]{\color{textcolor}\sffamily\fontsize{10.000000}{12.000000}\selectfont 0.30}%
\end{pgfscope}%
\begin{pgfscope}%
\pgfsetbuttcap%
\pgfsetroundjoin%
\definecolor{currentfill}{rgb}{0.000000,0.000000,0.000000}%
\pgfsetfillcolor{currentfill}%
\pgfsetlinewidth{0.803000pt}%
\definecolor{currentstroke}{rgb}{0.000000,0.000000,0.000000}%
\pgfsetstrokecolor{currentstroke}%
\pgfsetdash{}{0pt}%
\pgfsys@defobject{currentmarker}{\pgfqpoint{-0.048611in}{0.000000in}}{\pgfqpoint{0.000000in}{0.000000in}}{%
\pgfpathmoveto{\pgfqpoint{0.000000in}{0.000000in}}%
\pgfpathlineto{\pgfqpoint{-0.048611in}{0.000000in}}%
\pgfusepath{stroke,fill}%
}%
\begin{pgfscope}%
\pgfsys@transformshift{4.492417in}{3.768242in}%
\pgfsys@useobject{currentmarker}{}%
\end{pgfscope}%
\end{pgfscope}%
\begin{pgfscope}%
\definecolor{textcolor}{rgb}{0.000000,0.000000,0.000000}%
\pgfsetstrokecolor{textcolor}%
\pgfsetfillcolor{textcolor}%
\pgftext[x=4.085950in,y=3.715480in,left,base]{\color{textcolor}\sffamily\fontsize{10.000000}{12.000000}\selectfont 0.35}%
\end{pgfscope}%
\begin{pgfscope}%
\pgfsetbuttcap%
\pgfsetroundjoin%
\definecolor{currentfill}{rgb}{0.000000,0.000000,0.000000}%
\pgfsetfillcolor{currentfill}%
\pgfsetlinewidth{0.803000pt}%
\definecolor{currentstroke}{rgb}{0.000000,0.000000,0.000000}%
\pgfsetstrokecolor{currentstroke}%
\pgfsetdash{}{0pt}%
\pgfsys@defobject{currentmarker}{\pgfqpoint{-0.048611in}{0.000000in}}{\pgfqpoint{0.000000in}{0.000000in}}{%
\pgfpathmoveto{\pgfqpoint{0.000000in}{0.000000in}}%
\pgfpathlineto{\pgfqpoint{-0.048611in}{0.000000in}}%
\pgfusepath{stroke,fill}%
}%
\begin{pgfscope}%
\pgfsys@transformshift{4.492417in}{4.149293in}%
\pgfsys@useobject{currentmarker}{}%
\end{pgfscope}%
\end{pgfscope}%
\begin{pgfscope}%
\definecolor{textcolor}{rgb}{0.000000,0.000000,0.000000}%
\pgfsetstrokecolor{textcolor}%
\pgfsetfillcolor{textcolor}%
\pgftext[x=4.085950in,y=4.096531in,left,base]{\color{textcolor}\sffamily\fontsize{10.000000}{12.000000}\selectfont 0.40}%
\end{pgfscope}%
\begin{pgfscope}%
\pgfsetbuttcap%
\pgfsetroundjoin%
\definecolor{currentfill}{rgb}{0.000000,0.000000,0.000000}%
\pgfsetfillcolor{currentfill}%
\pgfsetlinewidth{0.803000pt}%
\definecolor{currentstroke}{rgb}{0.000000,0.000000,0.000000}%
\pgfsetstrokecolor{currentstroke}%
\pgfsetdash{}{0pt}%
\pgfsys@defobject{currentmarker}{\pgfqpoint{-0.048611in}{0.000000in}}{\pgfqpoint{0.000000in}{0.000000in}}{%
\pgfpathmoveto{\pgfqpoint{0.000000in}{0.000000in}}%
\pgfpathlineto{\pgfqpoint{-0.048611in}{0.000000in}}%
\pgfusepath{stroke,fill}%
}%
\begin{pgfscope}%
\pgfsys@transformshift{4.492417in}{4.530344in}%
\pgfsys@useobject{currentmarker}{}%
\end{pgfscope}%
\end{pgfscope}%
\begin{pgfscope}%
\definecolor{textcolor}{rgb}{0.000000,0.000000,0.000000}%
\pgfsetstrokecolor{textcolor}%
\pgfsetfillcolor{textcolor}%
\pgftext[x=4.085950in,y=4.477583in,left,base]{\color{textcolor}\sffamily\fontsize{10.000000}{12.000000}\selectfont 0.45}%
\end{pgfscope}%
\begin{pgfscope}%
\pgfsetbuttcap%
\pgfsetroundjoin%
\definecolor{currentfill}{rgb}{0.000000,0.000000,0.000000}%
\pgfsetfillcolor{currentfill}%
\pgfsetlinewidth{0.803000pt}%
\definecolor{currentstroke}{rgb}{0.000000,0.000000,0.000000}%
\pgfsetstrokecolor{currentstroke}%
\pgfsetdash{}{0pt}%
\pgfsys@defobject{currentmarker}{\pgfqpoint{-0.048611in}{0.000000in}}{\pgfqpoint{0.000000in}{0.000000in}}{%
\pgfpathmoveto{\pgfqpoint{0.000000in}{0.000000in}}%
\pgfpathlineto{\pgfqpoint{-0.048611in}{0.000000in}}%
\pgfusepath{stroke,fill}%
}%
\begin{pgfscope}%
\pgfsys@transformshift{4.492417in}{4.911395in}%
\pgfsys@useobject{currentmarker}{}%
\end{pgfscope}%
\end{pgfscope}%
\begin{pgfscope}%
\definecolor{textcolor}{rgb}{0.000000,0.000000,0.000000}%
\pgfsetstrokecolor{textcolor}%
\pgfsetfillcolor{textcolor}%
\pgftext[x=4.085950in,y=4.858634in,left,base]{\color{textcolor}\sffamily\fontsize{10.000000}{12.000000}\selectfont 0.50}%
\end{pgfscope}%
\begin{pgfscope}%
\pgfsetbuttcap%
\pgfsetroundjoin%
\definecolor{currentfill}{rgb}{0.000000,0.000000,0.000000}%
\pgfsetfillcolor{currentfill}%
\pgfsetlinewidth{0.803000pt}%
\definecolor{currentstroke}{rgb}{0.000000,0.000000,0.000000}%
\pgfsetstrokecolor{currentstroke}%
\pgfsetdash{}{0pt}%
\pgfsys@defobject{currentmarker}{\pgfqpoint{-0.048611in}{0.000000in}}{\pgfqpoint{0.000000in}{0.000000in}}{%
\pgfpathmoveto{\pgfqpoint{0.000000in}{0.000000in}}%
\pgfpathlineto{\pgfqpoint{-0.048611in}{0.000000in}}%
\pgfusepath{stroke,fill}%
}%
\begin{pgfscope}%
\pgfsys@transformshift{4.492417in}{5.292446in}%
\pgfsys@useobject{currentmarker}{}%
\end{pgfscope}%
\end{pgfscope}%
\begin{pgfscope}%
\definecolor{textcolor}{rgb}{0.000000,0.000000,0.000000}%
\pgfsetstrokecolor{textcolor}%
\pgfsetfillcolor{textcolor}%
\pgftext[x=4.085950in,y=5.239685in,left,base]{\color{textcolor}\sffamily\fontsize{10.000000}{12.000000}\selectfont 0.55}%
\end{pgfscope}%
\begin{pgfscope}%
\pgfpathrectangle{\pgfqpoint{4.492417in}{3.312222in}}{\pgfqpoint{3.322583in}{2.324444in}}%
\pgfusepath{clip}%
\pgfsetrectcap%
\pgfsetroundjoin%
\pgfsetlinewidth{1.505625pt}%
\definecolor{currentstroke}{rgb}{0.121569,0.466667,0.705882}%
\pgfsetstrokecolor{currentstroke}%
\pgfsetdash{}{0pt}%
\pgfpathmoveto{\pgfqpoint{4.651734in}{5.520538in}}%
\pgfpathlineto{\pgfqpoint{4.801932in}{5.440519in}}%
\pgfpathlineto{\pgfqpoint{4.952129in}{5.363827in}}%
\pgfpathlineto{\pgfqpoint{5.102327in}{5.297699in}}%
\pgfpathlineto{\pgfqpoint{5.252524in}{5.219326in}}%
\pgfpathlineto{\pgfqpoint{5.402721in}{5.126272in}}%
\pgfpathlineto{\pgfqpoint{5.552919in}{5.048869in}}%
\pgfpathlineto{\pgfqpoint{5.703116in}{4.947146in}}%
\pgfpathlineto{\pgfqpoint{5.853314in}{4.847133in}}%
\pgfpathlineto{\pgfqpoint{6.003511in}{4.759302in}}%
\pgfpathlineto{\pgfqpoint{6.153708in}{4.628535in}}%
\pgfpathlineto{\pgfqpoint{6.303906in}{4.522805in}}%
\pgfpathlineto{\pgfqpoint{6.454103in}{4.391254in}}%
\pgfpathlineto{\pgfqpoint{6.604301in}{4.302116in}}%
\pgfpathlineto{\pgfqpoint{6.754498in}{4.170006in}}%
\pgfpathlineto{\pgfqpoint{6.904696in}{4.059897in}}%
\pgfpathlineto{\pgfqpoint{7.054893in}{3.918601in}}%
\pgfpathlineto{\pgfqpoint{7.205090in}{3.776165in}}%
\pgfpathlineto{\pgfqpoint{7.355288in}{3.677723in}}%
\pgfpathlineto{\pgfqpoint{7.505485in}{3.564017in}}%
\pgfpathlineto{\pgfqpoint{7.655683in}{3.457395in}}%
\pgfusepath{stroke}%
\end{pgfscope}%
\begin{pgfscope}%
\pgfsetrectcap%
\pgfsetmiterjoin%
\pgfsetlinewidth{0.803000pt}%
\definecolor{currentstroke}{rgb}{0.000000,0.000000,0.000000}%
\pgfsetstrokecolor{currentstroke}%
\pgfsetdash{}{0pt}%
\pgfpathmoveto{\pgfqpoint{4.492417in}{3.312222in}}%
\pgfpathlineto{\pgfqpoint{4.492417in}{5.636667in}}%
\pgfusepath{stroke}%
\end{pgfscope}%
\begin{pgfscope}%
\pgfsetrectcap%
\pgfsetmiterjoin%
\pgfsetlinewidth{0.803000pt}%
\definecolor{currentstroke}{rgb}{0.000000,0.000000,0.000000}%
\pgfsetstrokecolor{currentstroke}%
\pgfsetdash{}{0pt}%
\pgfpathmoveto{\pgfqpoint{7.815000in}{3.312222in}}%
\pgfpathlineto{\pgfqpoint{7.815000in}{5.636667in}}%
\pgfusepath{stroke}%
\end{pgfscope}%
\begin{pgfscope}%
\pgfsetrectcap%
\pgfsetmiterjoin%
\pgfsetlinewidth{0.803000pt}%
\definecolor{currentstroke}{rgb}{0.000000,0.000000,0.000000}%
\pgfsetstrokecolor{currentstroke}%
\pgfsetdash{}{0pt}%
\pgfpathmoveto{\pgfqpoint{4.492417in}{3.312222in}}%
\pgfpathlineto{\pgfqpoint{7.815000in}{3.312222in}}%
\pgfusepath{stroke}%
\end{pgfscope}%
\begin{pgfscope}%
\pgfsetrectcap%
\pgfsetmiterjoin%
\pgfsetlinewidth{0.803000pt}%
\definecolor{currentstroke}{rgb}{0.000000,0.000000,0.000000}%
\pgfsetstrokecolor{currentstroke}%
\pgfsetdash{}{0pt}%
\pgfpathmoveto{\pgfqpoint{4.492417in}{5.636667in}}%
\pgfpathlineto{\pgfqpoint{7.815000in}{5.636667in}}%
\pgfusepath{stroke}%
\end{pgfscope}%
\begin{pgfscope}%
\definecolor{textcolor}{rgb}{0.000000,0.000000,0.000000}%
\pgfsetstrokecolor{textcolor}%
\pgfsetfillcolor{textcolor}%
\pgftext[x=6.153708in,y=5.720000in,,base]{\color{textcolor}\sffamily\fontsize{12.000000}{14.400000}\selectfont \(\displaystyle  N = 12 \)}%
\end{pgfscope}%
\begin{pgfscope}%
\pgfsetbuttcap%
\pgfsetmiterjoin%
\definecolor{currentfill}{rgb}{1.000000,1.000000,1.000000}%
\pgfsetfillcolor{currentfill}%
\pgfsetlinewidth{0.000000pt}%
\definecolor{currentstroke}{rgb}{0.000000,0.000000,0.000000}%
\pgfsetstrokecolor{currentstroke}%
\pgfsetstrokeopacity{0.000000}%
\pgfsetdash{}{0pt}%
\pgfpathmoveto{\pgfqpoint{0.555972in}{0.387222in}}%
\pgfpathlineto{\pgfqpoint{3.878555in}{0.387222in}}%
\pgfpathlineto{\pgfqpoint{3.878555in}{2.711667in}}%
\pgfpathlineto{\pgfqpoint{0.555972in}{2.711667in}}%
\pgfpathclose%
\pgfusepath{fill}%
\end{pgfscope}%
\begin{pgfscope}%
\pgfpathrectangle{\pgfqpoint{0.555972in}{0.387222in}}{\pgfqpoint{3.322583in}{2.324444in}}%
\pgfusepath{clip}%
\pgfsetbuttcap%
\pgfsetroundjoin%
\definecolor{currentfill}{rgb}{0.121569,0.466667,0.705882}%
\pgfsetfillcolor{currentfill}%
\pgfsetfillopacity{0.300000}%
\pgfsetlinewidth{1.003750pt}%
\definecolor{currentstroke}{rgb}{0.121569,0.466667,0.705882}%
\pgfsetstrokecolor{currentstroke}%
\pgfsetstrokeopacity{0.300000}%
\pgfsetdash{}{0pt}%
\pgfpathmoveto{\pgfqpoint{0.715290in}{2.606010in}}%
\pgfpathlineto{\pgfqpoint{0.715290in}{2.580402in}}%
\pgfpathlineto{\pgfqpoint{0.865487in}{2.527109in}}%
\pgfpathlineto{\pgfqpoint{1.015684in}{2.461032in}}%
\pgfpathlineto{\pgfqpoint{1.165882in}{2.393286in}}%
\pgfpathlineto{\pgfqpoint{1.316079in}{2.329547in}}%
\pgfpathlineto{\pgfqpoint{1.466277in}{2.248239in}}%
\pgfpathlineto{\pgfqpoint{1.616474in}{2.150988in}}%
\pgfpathlineto{\pgfqpoint{1.766671in}{2.084149in}}%
\pgfpathlineto{\pgfqpoint{1.916869in}{1.986896in}}%
\pgfpathlineto{\pgfqpoint{2.067066in}{1.878383in}}%
\pgfpathlineto{\pgfqpoint{2.217264in}{1.737534in}}%
\pgfpathlineto{\pgfqpoint{2.367461in}{1.629259in}}%
\pgfpathlineto{\pgfqpoint{2.517659in}{1.471218in}}%
\pgfpathlineto{\pgfqpoint{2.667856in}{1.382235in}}%
\pgfpathlineto{\pgfqpoint{2.818053in}{1.242006in}}%
\pgfpathlineto{\pgfqpoint{2.968251in}{1.104723in}}%
\pgfpathlineto{\pgfqpoint{3.118448in}{0.944310in}}%
\pgfpathlineto{\pgfqpoint{3.268646in}{0.827024in}}%
\pgfpathlineto{\pgfqpoint{3.418843in}{0.725387in}}%
\pgfpathlineto{\pgfqpoint{3.569040in}{0.596541in}}%
\pgfpathlineto{\pgfqpoint{3.719238in}{0.492879in}}%
\pgfpathlineto{\pgfqpoint{3.719238in}{0.553955in}}%
\pgfpathlineto{\pgfqpoint{3.719238in}{0.553955in}}%
\pgfpathlineto{\pgfqpoint{3.569040in}{0.658777in}}%
\pgfpathlineto{\pgfqpoint{3.418843in}{0.801373in}}%
\pgfpathlineto{\pgfqpoint{3.268646in}{0.915453in}}%
\pgfpathlineto{\pgfqpoint{3.118448in}{1.059672in}}%
\pgfpathlineto{\pgfqpoint{2.968251in}{1.183340in}}%
\pgfpathlineto{\pgfqpoint{2.818053in}{1.289740in}}%
\pgfpathlineto{\pgfqpoint{2.667856in}{1.476683in}}%
\pgfpathlineto{\pgfqpoint{2.517659in}{1.579797in}}%
\pgfpathlineto{\pgfqpoint{2.367461in}{1.657037in}}%
\pgfpathlineto{\pgfqpoint{2.217264in}{1.815880in}}%
\pgfpathlineto{\pgfqpoint{2.067066in}{1.923416in}}%
\pgfpathlineto{\pgfqpoint{1.916869in}{2.010197in}}%
\pgfpathlineto{\pgfqpoint{1.766671in}{2.117666in}}%
\pgfpathlineto{\pgfqpoint{1.616474in}{2.175486in}}%
\pgfpathlineto{\pgfqpoint{1.466277in}{2.258266in}}%
\pgfpathlineto{\pgfqpoint{1.316079in}{2.349182in}}%
\pgfpathlineto{\pgfqpoint{1.165882in}{2.421358in}}%
\pgfpathlineto{\pgfqpoint{1.015684in}{2.483160in}}%
\pgfpathlineto{\pgfqpoint{0.865487in}{2.545380in}}%
\pgfpathlineto{\pgfqpoint{0.715290in}{2.606010in}}%
\pgfpathclose%
\pgfusepath{stroke,fill}%
\end{pgfscope}%
\begin{pgfscope}%
\pgfpathrectangle{\pgfqpoint{0.555972in}{0.387222in}}{\pgfqpoint{3.322583in}{2.324444in}}%
\pgfusepath{clip}%
\pgfsetbuttcap%
\pgfsetroundjoin%
\definecolor{currentfill}{rgb}{0.121569,0.466667,0.705882}%
\pgfsetfillcolor{currentfill}%
\pgfsetlinewidth{1.003750pt}%
\definecolor{currentstroke}{rgb}{0.121569,0.466667,0.705882}%
\pgfsetstrokecolor{currentstroke}%
\pgfsetdash{}{0pt}%
\pgfsys@defobject{currentmarker}{\pgfqpoint{-0.009821in}{-0.009821in}}{\pgfqpoint{0.009821in}{0.009821in}}{%
\pgfpathmoveto{\pgfqpoint{0.000000in}{-0.009821in}}%
\pgfpathcurveto{\pgfqpoint{0.002605in}{-0.009821in}}{\pgfqpoint{0.005103in}{-0.008786in}}{\pgfqpoint{0.006944in}{-0.006944in}}%
\pgfpathcurveto{\pgfqpoint{0.008786in}{-0.005103in}}{\pgfqpoint{0.009821in}{-0.002605in}}{\pgfqpoint{0.009821in}{0.000000in}}%
\pgfpathcurveto{\pgfqpoint{0.009821in}{0.002605in}}{\pgfqpoint{0.008786in}{0.005103in}}{\pgfqpoint{0.006944in}{0.006944in}}%
\pgfpathcurveto{\pgfqpoint{0.005103in}{0.008786in}}{\pgfqpoint{0.002605in}{0.009821in}}{\pgfqpoint{0.000000in}{0.009821in}}%
\pgfpathcurveto{\pgfqpoint{-0.002605in}{0.009821in}}{\pgfqpoint{-0.005103in}{0.008786in}}{\pgfqpoint{-0.006944in}{0.006944in}}%
\pgfpathcurveto{\pgfqpoint{-0.008786in}{0.005103in}}{\pgfqpoint{-0.009821in}{0.002605in}}{\pgfqpoint{-0.009821in}{0.000000in}}%
\pgfpathcurveto{\pgfqpoint{-0.009821in}{-0.002605in}}{\pgfqpoint{-0.008786in}{-0.005103in}}{\pgfqpoint{-0.006944in}{-0.006944in}}%
\pgfpathcurveto{\pgfqpoint{-0.005103in}{-0.008786in}}{\pgfqpoint{-0.002605in}{-0.009821in}}{\pgfqpoint{0.000000in}{-0.009821in}}%
\pgfpathclose%
\pgfusepath{stroke,fill}%
}%
\begin{pgfscope}%
\pgfsys@transformshift{0.715290in}{2.593206in}%
\pgfsys@useobject{currentmarker}{}%
\end{pgfscope}%
\begin{pgfscope}%
\pgfsys@transformshift{0.865487in}{2.536245in}%
\pgfsys@useobject{currentmarker}{}%
\end{pgfscope}%
\begin{pgfscope}%
\pgfsys@transformshift{1.015684in}{2.472096in}%
\pgfsys@useobject{currentmarker}{}%
\end{pgfscope}%
\begin{pgfscope}%
\pgfsys@transformshift{1.165882in}{2.407322in}%
\pgfsys@useobject{currentmarker}{}%
\end{pgfscope}%
\begin{pgfscope}%
\pgfsys@transformshift{1.316079in}{2.339364in}%
\pgfsys@useobject{currentmarker}{}%
\end{pgfscope}%
\begin{pgfscope}%
\pgfsys@transformshift{1.466277in}{2.253253in}%
\pgfsys@useobject{currentmarker}{}%
\end{pgfscope}%
\begin{pgfscope}%
\pgfsys@transformshift{1.616474in}{2.163237in}%
\pgfsys@useobject{currentmarker}{}%
\end{pgfscope}%
\begin{pgfscope}%
\pgfsys@transformshift{1.766671in}{2.100907in}%
\pgfsys@useobject{currentmarker}{}%
\end{pgfscope}%
\begin{pgfscope}%
\pgfsys@transformshift{1.916869in}{1.998547in}%
\pgfsys@useobject{currentmarker}{}%
\end{pgfscope}%
\begin{pgfscope}%
\pgfsys@transformshift{2.067066in}{1.900900in}%
\pgfsys@useobject{currentmarker}{}%
\end{pgfscope}%
\begin{pgfscope}%
\pgfsys@transformshift{2.217264in}{1.776707in}%
\pgfsys@useobject{currentmarker}{}%
\end{pgfscope}%
\begin{pgfscope}%
\pgfsys@transformshift{2.367461in}{1.643148in}%
\pgfsys@useobject{currentmarker}{}%
\end{pgfscope}%
\begin{pgfscope}%
\pgfsys@transformshift{2.517659in}{1.525508in}%
\pgfsys@useobject{currentmarker}{}%
\end{pgfscope}%
\begin{pgfscope}%
\pgfsys@transformshift{2.667856in}{1.429459in}%
\pgfsys@useobject{currentmarker}{}%
\end{pgfscope}%
\begin{pgfscope}%
\pgfsys@transformshift{2.818053in}{1.265873in}%
\pgfsys@useobject{currentmarker}{}%
\end{pgfscope}%
\begin{pgfscope}%
\pgfsys@transformshift{2.968251in}{1.144032in}%
\pgfsys@useobject{currentmarker}{}%
\end{pgfscope}%
\begin{pgfscope}%
\pgfsys@transformshift{3.118448in}{1.001991in}%
\pgfsys@useobject{currentmarker}{}%
\end{pgfscope}%
\begin{pgfscope}%
\pgfsys@transformshift{3.268646in}{0.871239in}%
\pgfsys@useobject{currentmarker}{}%
\end{pgfscope}%
\begin{pgfscope}%
\pgfsys@transformshift{3.418843in}{0.763380in}%
\pgfsys@useobject{currentmarker}{}%
\end{pgfscope}%
\begin{pgfscope}%
\pgfsys@transformshift{3.569040in}{0.627659in}%
\pgfsys@useobject{currentmarker}{}%
\end{pgfscope}%
\begin{pgfscope}%
\pgfsys@transformshift{3.719238in}{0.523417in}%
\pgfsys@useobject{currentmarker}{}%
\end{pgfscope}%
\end{pgfscope}%
\begin{pgfscope}%
\pgfsetbuttcap%
\pgfsetroundjoin%
\definecolor{currentfill}{rgb}{0.000000,0.000000,0.000000}%
\pgfsetfillcolor{currentfill}%
\pgfsetlinewidth{0.803000pt}%
\definecolor{currentstroke}{rgb}{0.000000,0.000000,0.000000}%
\pgfsetstrokecolor{currentstroke}%
\pgfsetdash{}{0pt}%
\pgfsys@defobject{currentmarker}{\pgfqpoint{0.000000in}{-0.048611in}}{\pgfqpoint{0.000000in}{0.000000in}}{%
\pgfpathmoveto{\pgfqpoint{0.000000in}{0.000000in}}%
\pgfpathlineto{\pgfqpoint{0.000000in}{-0.048611in}}%
\pgfusepath{stroke,fill}%
}%
\begin{pgfscope}%
\pgfsys@transformshift{0.865487in}{0.387222in}%
\pgfsys@useobject{currentmarker}{}%
\end{pgfscope}%
\end{pgfscope}%
\begin{pgfscope}%
\definecolor{textcolor}{rgb}{0.000000,0.000000,0.000000}%
\pgfsetstrokecolor{textcolor}%
\pgfsetfillcolor{textcolor}%
\pgftext[x=0.865487in,y=0.290000in,,top]{\color{textcolor}\sffamily\fontsize{10.000000}{12.000000}\selectfont 4.35}%
\end{pgfscope}%
\begin{pgfscope}%
\pgfsetbuttcap%
\pgfsetroundjoin%
\definecolor{currentfill}{rgb}{0.000000,0.000000,0.000000}%
\pgfsetfillcolor{currentfill}%
\pgfsetlinewidth{0.803000pt}%
\definecolor{currentstroke}{rgb}{0.000000,0.000000,0.000000}%
\pgfsetstrokecolor{currentstroke}%
\pgfsetdash{}{0pt}%
\pgfsys@defobject{currentmarker}{\pgfqpoint{0.000000in}{-0.048611in}}{\pgfqpoint{0.000000in}{0.000000in}}{%
\pgfpathmoveto{\pgfqpoint{0.000000in}{0.000000in}}%
\pgfpathlineto{\pgfqpoint{0.000000in}{-0.048611in}}%
\pgfusepath{stroke,fill}%
}%
\begin{pgfscope}%
\pgfsys@transformshift{1.616474in}{0.387222in}%
\pgfsys@useobject{currentmarker}{}%
\end{pgfscope}%
\end{pgfscope}%
\begin{pgfscope}%
\definecolor{textcolor}{rgb}{0.000000,0.000000,0.000000}%
\pgfsetstrokecolor{textcolor}%
\pgfsetfillcolor{textcolor}%
\pgftext[x=1.616474in,y=0.290000in,,top]{\color{textcolor}\sffamily\fontsize{10.000000}{12.000000}\selectfont 4.40}%
\end{pgfscope}%
\begin{pgfscope}%
\pgfsetbuttcap%
\pgfsetroundjoin%
\definecolor{currentfill}{rgb}{0.000000,0.000000,0.000000}%
\pgfsetfillcolor{currentfill}%
\pgfsetlinewidth{0.803000pt}%
\definecolor{currentstroke}{rgb}{0.000000,0.000000,0.000000}%
\pgfsetstrokecolor{currentstroke}%
\pgfsetdash{}{0pt}%
\pgfsys@defobject{currentmarker}{\pgfqpoint{0.000000in}{-0.048611in}}{\pgfqpoint{0.000000in}{0.000000in}}{%
\pgfpathmoveto{\pgfqpoint{0.000000in}{0.000000in}}%
\pgfpathlineto{\pgfqpoint{0.000000in}{-0.048611in}}%
\pgfusepath{stroke,fill}%
}%
\begin{pgfscope}%
\pgfsys@transformshift{2.367461in}{0.387222in}%
\pgfsys@useobject{currentmarker}{}%
\end{pgfscope}%
\end{pgfscope}%
\begin{pgfscope}%
\definecolor{textcolor}{rgb}{0.000000,0.000000,0.000000}%
\pgfsetstrokecolor{textcolor}%
\pgfsetfillcolor{textcolor}%
\pgftext[x=2.367461in,y=0.290000in,,top]{\color{textcolor}\sffamily\fontsize{10.000000}{12.000000}\selectfont 4.45}%
\end{pgfscope}%
\begin{pgfscope}%
\pgfsetbuttcap%
\pgfsetroundjoin%
\definecolor{currentfill}{rgb}{0.000000,0.000000,0.000000}%
\pgfsetfillcolor{currentfill}%
\pgfsetlinewidth{0.803000pt}%
\definecolor{currentstroke}{rgb}{0.000000,0.000000,0.000000}%
\pgfsetstrokecolor{currentstroke}%
\pgfsetdash{}{0pt}%
\pgfsys@defobject{currentmarker}{\pgfqpoint{0.000000in}{-0.048611in}}{\pgfqpoint{0.000000in}{0.000000in}}{%
\pgfpathmoveto{\pgfqpoint{0.000000in}{0.000000in}}%
\pgfpathlineto{\pgfqpoint{0.000000in}{-0.048611in}}%
\pgfusepath{stroke,fill}%
}%
\begin{pgfscope}%
\pgfsys@transformshift{3.118448in}{0.387222in}%
\pgfsys@useobject{currentmarker}{}%
\end{pgfscope}%
\end{pgfscope}%
\begin{pgfscope}%
\definecolor{textcolor}{rgb}{0.000000,0.000000,0.000000}%
\pgfsetstrokecolor{textcolor}%
\pgfsetfillcolor{textcolor}%
\pgftext[x=3.118448in,y=0.290000in,,top]{\color{textcolor}\sffamily\fontsize{10.000000}{12.000000}\selectfont 4.50}%
\end{pgfscope}%
\begin{pgfscope}%
\pgfsetbuttcap%
\pgfsetroundjoin%
\definecolor{currentfill}{rgb}{0.000000,0.000000,0.000000}%
\pgfsetfillcolor{currentfill}%
\pgfsetlinewidth{0.803000pt}%
\definecolor{currentstroke}{rgb}{0.000000,0.000000,0.000000}%
\pgfsetstrokecolor{currentstroke}%
\pgfsetdash{}{0pt}%
\pgfsys@defobject{currentmarker}{\pgfqpoint{0.000000in}{-0.048611in}}{\pgfqpoint{0.000000in}{0.000000in}}{%
\pgfpathmoveto{\pgfqpoint{0.000000in}{0.000000in}}%
\pgfpathlineto{\pgfqpoint{0.000000in}{-0.048611in}}%
\pgfusepath{stroke,fill}%
}%
\begin{pgfscope}%
\pgfsys@transformshift{3.869435in}{0.387222in}%
\pgfsys@useobject{currentmarker}{}%
\end{pgfscope}%
\end{pgfscope}%
\begin{pgfscope}%
\definecolor{textcolor}{rgb}{0.000000,0.000000,0.000000}%
\pgfsetstrokecolor{textcolor}%
\pgfsetfillcolor{textcolor}%
\pgftext[x=3.869435in,y=0.290000in,,top]{\color{textcolor}\sffamily\fontsize{10.000000}{12.000000}\selectfont 4.55}%
\end{pgfscope}%
\begin{pgfscope}%
\pgfsetbuttcap%
\pgfsetroundjoin%
\definecolor{currentfill}{rgb}{0.000000,0.000000,0.000000}%
\pgfsetfillcolor{currentfill}%
\pgfsetlinewidth{0.803000pt}%
\definecolor{currentstroke}{rgb}{0.000000,0.000000,0.000000}%
\pgfsetstrokecolor{currentstroke}%
\pgfsetdash{}{0pt}%
\pgfsys@defobject{currentmarker}{\pgfqpoint{-0.048611in}{0.000000in}}{\pgfqpoint{0.000000in}{0.000000in}}{%
\pgfpathmoveto{\pgfqpoint{0.000000in}{0.000000in}}%
\pgfpathlineto{\pgfqpoint{-0.048611in}{0.000000in}}%
\pgfusepath{stroke,fill}%
}%
\begin{pgfscope}%
\pgfsys@transformshift{0.555972in}{0.465375in}%
\pgfsys@useobject{currentmarker}{}%
\end{pgfscope}%
\end{pgfscope}%
\begin{pgfscope}%
\definecolor{textcolor}{rgb}{0.000000,0.000000,0.000000}%
\pgfsetstrokecolor{textcolor}%
\pgfsetfillcolor{textcolor}%
\pgftext[x=0.149505in,y=0.412614in,left,base]{\color{textcolor}\sffamily\fontsize{10.000000}{12.000000}\selectfont 0.20}%
\end{pgfscope}%
\begin{pgfscope}%
\pgfsetbuttcap%
\pgfsetroundjoin%
\definecolor{currentfill}{rgb}{0.000000,0.000000,0.000000}%
\pgfsetfillcolor{currentfill}%
\pgfsetlinewidth{0.803000pt}%
\definecolor{currentstroke}{rgb}{0.000000,0.000000,0.000000}%
\pgfsetstrokecolor{currentstroke}%
\pgfsetdash{}{0pt}%
\pgfsys@defobject{currentmarker}{\pgfqpoint{-0.048611in}{0.000000in}}{\pgfqpoint{0.000000in}{0.000000in}}{%
\pgfpathmoveto{\pgfqpoint{0.000000in}{0.000000in}}%
\pgfpathlineto{\pgfqpoint{-0.048611in}{0.000000in}}%
\pgfusepath{stroke,fill}%
}%
\begin{pgfscope}%
\pgfsys@transformshift{0.555972in}{0.766906in}%
\pgfsys@useobject{currentmarker}{}%
\end{pgfscope}%
\end{pgfscope}%
\begin{pgfscope}%
\definecolor{textcolor}{rgb}{0.000000,0.000000,0.000000}%
\pgfsetstrokecolor{textcolor}%
\pgfsetfillcolor{textcolor}%
\pgftext[x=0.149505in,y=0.714144in,left,base]{\color{textcolor}\sffamily\fontsize{10.000000}{12.000000}\selectfont 0.25}%
\end{pgfscope}%
\begin{pgfscope}%
\pgfsetbuttcap%
\pgfsetroundjoin%
\definecolor{currentfill}{rgb}{0.000000,0.000000,0.000000}%
\pgfsetfillcolor{currentfill}%
\pgfsetlinewidth{0.803000pt}%
\definecolor{currentstroke}{rgb}{0.000000,0.000000,0.000000}%
\pgfsetstrokecolor{currentstroke}%
\pgfsetdash{}{0pt}%
\pgfsys@defobject{currentmarker}{\pgfqpoint{-0.048611in}{0.000000in}}{\pgfqpoint{0.000000in}{0.000000in}}{%
\pgfpathmoveto{\pgfqpoint{0.000000in}{0.000000in}}%
\pgfpathlineto{\pgfqpoint{-0.048611in}{0.000000in}}%
\pgfusepath{stroke,fill}%
}%
\begin{pgfscope}%
\pgfsys@transformshift{0.555972in}{1.068436in}%
\pgfsys@useobject{currentmarker}{}%
\end{pgfscope}%
\end{pgfscope}%
\begin{pgfscope}%
\definecolor{textcolor}{rgb}{0.000000,0.000000,0.000000}%
\pgfsetstrokecolor{textcolor}%
\pgfsetfillcolor{textcolor}%
\pgftext[x=0.149505in,y=1.015675in,left,base]{\color{textcolor}\sffamily\fontsize{10.000000}{12.000000}\selectfont 0.30}%
\end{pgfscope}%
\begin{pgfscope}%
\pgfsetbuttcap%
\pgfsetroundjoin%
\definecolor{currentfill}{rgb}{0.000000,0.000000,0.000000}%
\pgfsetfillcolor{currentfill}%
\pgfsetlinewidth{0.803000pt}%
\definecolor{currentstroke}{rgb}{0.000000,0.000000,0.000000}%
\pgfsetstrokecolor{currentstroke}%
\pgfsetdash{}{0pt}%
\pgfsys@defobject{currentmarker}{\pgfqpoint{-0.048611in}{0.000000in}}{\pgfqpoint{0.000000in}{0.000000in}}{%
\pgfpathmoveto{\pgfqpoint{0.000000in}{0.000000in}}%
\pgfpathlineto{\pgfqpoint{-0.048611in}{0.000000in}}%
\pgfusepath{stroke,fill}%
}%
\begin{pgfscope}%
\pgfsys@transformshift{0.555972in}{1.369967in}%
\pgfsys@useobject{currentmarker}{}%
\end{pgfscope}%
\end{pgfscope}%
\begin{pgfscope}%
\definecolor{textcolor}{rgb}{0.000000,0.000000,0.000000}%
\pgfsetstrokecolor{textcolor}%
\pgfsetfillcolor{textcolor}%
\pgftext[x=0.149505in,y=1.317206in,left,base]{\color{textcolor}\sffamily\fontsize{10.000000}{12.000000}\selectfont 0.35}%
\end{pgfscope}%
\begin{pgfscope}%
\pgfsetbuttcap%
\pgfsetroundjoin%
\definecolor{currentfill}{rgb}{0.000000,0.000000,0.000000}%
\pgfsetfillcolor{currentfill}%
\pgfsetlinewidth{0.803000pt}%
\definecolor{currentstroke}{rgb}{0.000000,0.000000,0.000000}%
\pgfsetstrokecolor{currentstroke}%
\pgfsetdash{}{0pt}%
\pgfsys@defobject{currentmarker}{\pgfqpoint{-0.048611in}{0.000000in}}{\pgfqpoint{0.000000in}{0.000000in}}{%
\pgfpathmoveto{\pgfqpoint{0.000000in}{0.000000in}}%
\pgfpathlineto{\pgfqpoint{-0.048611in}{0.000000in}}%
\pgfusepath{stroke,fill}%
}%
\begin{pgfscope}%
\pgfsys@transformshift{0.555972in}{1.671498in}%
\pgfsys@useobject{currentmarker}{}%
\end{pgfscope}%
\end{pgfscope}%
\begin{pgfscope}%
\definecolor{textcolor}{rgb}{0.000000,0.000000,0.000000}%
\pgfsetstrokecolor{textcolor}%
\pgfsetfillcolor{textcolor}%
\pgftext[x=0.149505in,y=1.618736in,left,base]{\color{textcolor}\sffamily\fontsize{10.000000}{12.000000}\selectfont 0.40}%
\end{pgfscope}%
\begin{pgfscope}%
\pgfsetbuttcap%
\pgfsetroundjoin%
\definecolor{currentfill}{rgb}{0.000000,0.000000,0.000000}%
\pgfsetfillcolor{currentfill}%
\pgfsetlinewidth{0.803000pt}%
\definecolor{currentstroke}{rgb}{0.000000,0.000000,0.000000}%
\pgfsetstrokecolor{currentstroke}%
\pgfsetdash{}{0pt}%
\pgfsys@defobject{currentmarker}{\pgfqpoint{-0.048611in}{0.000000in}}{\pgfqpoint{0.000000in}{0.000000in}}{%
\pgfpathmoveto{\pgfqpoint{0.000000in}{0.000000in}}%
\pgfpathlineto{\pgfqpoint{-0.048611in}{0.000000in}}%
\pgfusepath{stroke,fill}%
}%
\begin{pgfscope}%
\pgfsys@transformshift{0.555972in}{1.973028in}%
\pgfsys@useobject{currentmarker}{}%
\end{pgfscope}%
\end{pgfscope}%
\begin{pgfscope}%
\definecolor{textcolor}{rgb}{0.000000,0.000000,0.000000}%
\pgfsetstrokecolor{textcolor}%
\pgfsetfillcolor{textcolor}%
\pgftext[x=0.149505in,y=1.920267in,left,base]{\color{textcolor}\sffamily\fontsize{10.000000}{12.000000}\selectfont 0.45}%
\end{pgfscope}%
\begin{pgfscope}%
\pgfsetbuttcap%
\pgfsetroundjoin%
\definecolor{currentfill}{rgb}{0.000000,0.000000,0.000000}%
\pgfsetfillcolor{currentfill}%
\pgfsetlinewidth{0.803000pt}%
\definecolor{currentstroke}{rgb}{0.000000,0.000000,0.000000}%
\pgfsetstrokecolor{currentstroke}%
\pgfsetdash{}{0pt}%
\pgfsys@defobject{currentmarker}{\pgfqpoint{-0.048611in}{0.000000in}}{\pgfqpoint{0.000000in}{0.000000in}}{%
\pgfpathmoveto{\pgfqpoint{0.000000in}{0.000000in}}%
\pgfpathlineto{\pgfqpoint{-0.048611in}{0.000000in}}%
\pgfusepath{stroke,fill}%
}%
\begin{pgfscope}%
\pgfsys@transformshift{0.555972in}{2.274559in}%
\pgfsys@useobject{currentmarker}{}%
\end{pgfscope}%
\end{pgfscope}%
\begin{pgfscope}%
\definecolor{textcolor}{rgb}{0.000000,0.000000,0.000000}%
\pgfsetstrokecolor{textcolor}%
\pgfsetfillcolor{textcolor}%
\pgftext[x=0.149505in,y=2.221798in,left,base]{\color{textcolor}\sffamily\fontsize{10.000000}{12.000000}\selectfont 0.50}%
\end{pgfscope}%
\begin{pgfscope}%
\pgfsetbuttcap%
\pgfsetroundjoin%
\definecolor{currentfill}{rgb}{0.000000,0.000000,0.000000}%
\pgfsetfillcolor{currentfill}%
\pgfsetlinewidth{0.803000pt}%
\definecolor{currentstroke}{rgb}{0.000000,0.000000,0.000000}%
\pgfsetstrokecolor{currentstroke}%
\pgfsetdash{}{0pt}%
\pgfsys@defobject{currentmarker}{\pgfqpoint{-0.048611in}{0.000000in}}{\pgfqpoint{0.000000in}{0.000000in}}{%
\pgfpathmoveto{\pgfqpoint{0.000000in}{0.000000in}}%
\pgfpathlineto{\pgfqpoint{-0.048611in}{0.000000in}}%
\pgfusepath{stroke,fill}%
}%
\begin{pgfscope}%
\pgfsys@transformshift{0.555972in}{2.576090in}%
\pgfsys@useobject{currentmarker}{}%
\end{pgfscope}%
\end{pgfscope}%
\begin{pgfscope}%
\definecolor{textcolor}{rgb}{0.000000,0.000000,0.000000}%
\pgfsetstrokecolor{textcolor}%
\pgfsetfillcolor{textcolor}%
\pgftext[x=0.149505in,y=2.523328in,left,base]{\color{textcolor}\sffamily\fontsize{10.000000}{12.000000}\selectfont 0.55}%
\end{pgfscope}%
\begin{pgfscope}%
\pgfpathrectangle{\pgfqpoint{0.555972in}{0.387222in}}{\pgfqpoint{3.322583in}{2.324444in}}%
\pgfusepath{clip}%
\pgfsetrectcap%
\pgfsetroundjoin%
\pgfsetlinewidth{1.505625pt}%
\definecolor{currentstroke}{rgb}{0.121569,0.466667,0.705882}%
\pgfsetstrokecolor{currentstroke}%
\pgfsetdash{}{0pt}%
\pgfpathmoveto{\pgfqpoint{0.715290in}{2.593206in}}%
\pgfpathlineto{\pgfqpoint{0.865487in}{2.536245in}}%
\pgfpathlineto{\pgfqpoint{1.015684in}{2.472096in}}%
\pgfpathlineto{\pgfqpoint{1.165882in}{2.407322in}}%
\pgfpathlineto{\pgfqpoint{1.316079in}{2.339364in}}%
\pgfpathlineto{\pgfqpoint{1.466277in}{2.253253in}}%
\pgfpathlineto{\pgfqpoint{1.616474in}{2.163237in}}%
\pgfpathlineto{\pgfqpoint{1.766671in}{2.100907in}}%
\pgfpathlineto{\pgfqpoint{1.916869in}{1.998547in}}%
\pgfpathlineto{\pgfqpoint{2.067066in}{1.900900in}}%
\pgfpathlineto{\pgfqpoint{2.217264in}{1.776707in}}%
\pgfpathlineto{\pgfqpoint{2.367461in}{1.643148in}}%
\pgfpathlineto{\pgfqpoint{2.517659in}{1.525508in}}%
\pgfpathlineto{\pgfqpoint{2.667856in}{1.429459in}}%
\pgfpathlineto{\pgfqpoint{2.818053in}{1.265873in}}%
\pgfpathlineto{\pgfqpoint{2.968251in}{1.144032in}}%
\pgfpathlineto{\pgfqpoint{3.118448in}{1.001991in}}%
\pgfpathlineto{\pgfqpoint{3.268646in}{0.871239in}}%
\pgfpathlineto{\pgfqpoint{3.418843in}{0.763380in}}%
\pgfpathlineto{\pgfqpoint{3.569040in}{0.627659in}}%
\pgfpathlineto{\pgfqpoint{3.719238in}{0.523417in}}%
\pgfusepath{stroke}%
\end{pgfscope}%
\begin{pgfscope}%
\pgfsetrectcap%
\pgfsetmiterjoin%
\pgfsetlinewidth{0.803000pt}%
\definecolor{currentstroke}{rgb}{0.000000,0.000000,0.000000}%
\pgfsetstrokecolor{currentstroke}%
\pgfsetdash{}{0pt}%
\pgfpathmoveto{\pgfqpoint{0.555972in}{0.387222in}}%
\pgfpathlineto{\pgfqpoint{0.555972in}{2.711667in}}%
\pgfusepath{stroke}%
\end{pgfscope}%
\begin{pgfscope}%
\pgfsetrectcap%
\pgfsetmiterjoin%
\pgfsetlinewidth{0.803000pt}%
\definecolor{currentstroke}{rgb}{0.000000,0.000000,0.000000}%
\pgfsetstrokecolor{currentstroke}%
\pgfsetdash{}{0pt}%
\pgfpathmoveto{\pgfqpoint{3.878555in}{0.387222in}}%
\pgfpathlineto{\pgfqpoint{3.878555in}{2.711667in}}%
\pgfusepath{stroke}%
\end{pgfscope}%
\begin{pgfscope}%
\pgfsetrectcap%
\pgfsetmiterjoin%
\pgfsetlinewidth{0.803000pt}%
\definecolor{currentstroke}{rgb}{0.000000,0.000000,0.000000}%
\pgfsetstrokecolor{currentstroke}%
\pgfsetdash{}{0pt}%
\pgfpathmoveto{\pgfqpoint{0.555972in}{0.387222in}}%
\pgfpathlineto{\pgfqpoint{3.878555in}{0.387222in}}%
\pgfusepath{stroke}%
\end{pgfscope}%
\begin{pgfscope}%
\pgfsetrectcap%
\pgfsetmiterjoin%
\pgfsetlinewidth{0.803000pt}%
\definecolor{currentstroke}{rgb}{0.000000,0.000000,0.000000}%
\pgfsetstrokecolor{currentstroke}%
\pgfsetdash{}{0pt}%
\pgfpathmoveto{\pgfqpoint{0.555972in}{2.711667in}}%
\pgfpathlineto{\pgfqpoint{3.878555in}{2.711667in}}%
\pgfusepath{stroke}%
\end{pgfscope}%
\begin{pgfscope}%
\definecolor{textcolor}{rgb}{0.000000,0.000000,0.000000}%
\pgfsetstrokecolor{textcolor}%
\pgfsetfillcolor{textcolor}%
\pgftext[x=2.217264in,y=2.795000in,,base]{\color{textcolor}\sffamily\fontsize{12.000000}{14.400000}\selectfont \(\displaystyle  N = 16 \)}%
\end{pgfscope}%
\begin{pgfscope}%
\pgfsetbuttcap%
\pgfsetmiterjoin%
\definecolor{currentfill}{rgb}{1.000000,1.000000,1.000000}%
\pgfsetfillcolor{currentfill}%
\pgfsetlinewidth{0.000000pt}%
\definecolor{currentstroke}{rgb}{0.000000,0.000000,0.000000}%
\pgfsetstrokecolor{currentstroke}%
\pgfsetstrokeopacity{0.000000}%
\pgfsetdash{}{0pt}%
\pgfpathmoveto{\pgfqpoint{4.492417in}{0.387222in}}%
\pgfpathlineto{\pgfqpoint{7.815000in}{0.387222in}}%
\pgfpathlineto{\pgfqpoint{7.815000in}{2.711667in}}%
\pgfpathlineto{\pgfqpoint{4.492417in}{2.711667in}}%
\pgfpathclose%
\pgfusepath{fill}%
\end{pgfscope}%
\begin{pgfscope}%
\pgfpathrectangle{\pgfqpoint{4.492417in}{0.387222in}}{\pgfqpoint{3.322583in}{2.324444in}}%
\pgfusepath{clip}%
\pgfsetbuttcap%
\pgfsetroundjoin%
\definecolor{currentfill}{rgb}{0.121569,0.466667,0.705882}%
\pgfsetfillcolor{currentfill}%
\pgfsetfillopacity{0.300000}%
\pgfsetlinewidth{1.003750pt}%
\definecolor{currentstroke}{rgb}{0.121569,0.466667,0.705882}%
\pgfsetstrokecolor{currentstroke}%
\pgfsetstrokeopacity{0.300000}%
\pgfsetdash{}{0pt}%
\pgfpathmoveto{\pgfqpoint{4.651734in}{2.604432in}}%
\pgfpathlineto{\pgfqpoint{4.651734in}{2.591946in}}%
\pgfpathlineto{\pgfqpoint{4.801932in}{2.524048in}}%
\pgfpathlineto{\pgfqpoint{4.952129in}{2.468748in}}%
\pgfpathlineto{\pgfqpoint{5.102327in}{2.404510in}}%
\pgfpathlineto{\pgfqpoint{5.252524in}{2.311050in}}%
\pgfpathlineto{\pgfqpoint{5.402721in}{2.222722in}}%
\pgfpathlineto{\pgfqpoint{5.552919in}{2.145262in}}%
\pgfpathlineto{\pgfqpoint{5.703116in}{2.059815in}}%
\pgfpathlineto{\pgfqpoint{5.853314in}{1.882681in}}%
\pgfpathlineto{\pgfqpoint{6.003511in}{1.659306in}}%
\pgfpathlineto{\pgfqpoint{6.153708in}{1.532207in}}%
\pgfpathlineto{\pgfqpoint{6.303906in}{1.431786in}}%
\pgfpathlineto{\pgfqpoint{6.454103in}{1.224044in}}%
\pgfpathlineto{\pgfqpoint{6.604301in}{1.059552in}}%
\pgfpathlineto{\pgfqpoint{6.754498in}{0.807475in}}%
\pgfpathlineto{\pgfqpoint{6.904696in}{0.794919in}}%
\pgfpathlineto{\pgfqpoint{7.054893in}{0.712201in}}%
\pgfpathlineto{\pgfqpoint{7.205090in}{0.591630in}}%
\pgfpathlineto{\pgfqpoint{7.355288in}{0.569899in}}%
\pgfpathlineto{\pgfqpoint{7.505485in}{0.548511in}}%
\pgfpathlineto{\pgfqpoint{7.655683in}{0.492879in}}%
\pgfpathlineto{\pgfqpoint{7.655683in}{0.587738in}}%
\pgfpathlineto{\pgfqpoint{7.655683in}{0.587738in}}%
\pgfpathlineto{\pgfqpoint{7.505485in}{0.624698in}}%
\pgfpathlineto{\pgfqpoint{7.355288in}{0.704551in}}%
\pgfpathlineto{\pgfqpoint{7.205090in}{0.681423in}}%
\pgfpathlineto{\pgfqpoint{7.054893in}{0.822050in}}%
\pgfpathlineto{\pgfqpoint{6.904696in}{0.968468in}}%
\pgfpathlineto{\pgfqpoint{6.754498in}{1.138743in}}%
\pgfpathlineto{\pgfqpoint{6.604301in}{1.281809in}}%
\pgfpathlineto{\pgfqpoint{6.454103in}{1.287274in}}%
\pgfpathlineto{\pgfqpoint{6.303906in}{1.505983in}}%
\pgfpathlineto{\pgfqpoint{6.153708in}{1.754038in}}%
\pgfpathlineto{\pgfqpoint{6.003511in}{1.813558in}}%
\pgfpathlineto{\pgfqpoint{5.853314in}{1.971883in}}%
\pgfpathlineto{\pgfqpoint{5.703116in}{2.099858in}}%
\pgfpathlineto{\pgfqpoint{5.552919in}{2.212103in}}%
\pgfpathlineto{\pgfqpoint{5.402721in}{2.261699in}}%
\pgfpathlineto{\pgfqpoint{5.252524in}{2.364106in}}%
\pgfpathlineto{\pgfqpoint{5.102327in}{2.424678in}}%
\pgfpathlineto{\pgfqpoint{4.952129in}{2.495980in}}%
\pgfpathlineto{\pgfqpoint{4.801932in}{2.553097in}}%
\pgfpathlineto{\pgfqpoint{4.651734in}{2.604432in}}%
\pgfpathclose%
\pgfusepath{stroke,fill}%
\end{pgfscope}%
\begin{pgfscope}%
\pgfpathrectangle{\pgfqpoint{4.492417in}{0.387222in}}{\pgfqpoint{3.322583in}{2.324444in}}%
\pgfusepath{clip}%
\pgfsetbuttcap%
\pgfsetroundjoin%
\definecolor{currentfill}{rgb}{0.121569,0.466667,0.705882}%
\pgfsetfillcolor{currentfill}%
\pgfsetlinewidth{1.003750pt}%
\definecolor{currentstroke}{rgb}{0.121569,0.466667,0.705882}%
\pgfsetstrokecolor{currentstroke}%
\pgfsetdash{}{0pt}%
\pgfsys@defobject{currentmarker}{\pgfqpoint{-0.009821in}{-0.009821in}}{\pgfqpoint{0.009821in}{0.009821in}}{%
\pgfpathmoveto{\pgfqpoint{0.000000in}{-0.009821in}}%
\pgfpathcurveto{\pgfqpoint{0.002605in}{-0.009821in}}{\pgfqpoint{0.005103in}{-0.008786in}}{\pgfqpoint{0.006944in}{-0.006944in}}%
\pgfpathcurveto{\pgfqpoint{0.008786in}{-0.005103in}}{\pgfqpoint{0.009821in}{-0.002605in}}{\pgfqpoint{0.009821in}{0.000000in}}%
\pgfpathcurveto{\pgfqpoint{0.009821in}{0.002605in}}{\pgfqpoint{0.008786in}{0.005103in}}{\pgfqpoint{0.006944in}{0.006944in}}%
\pgfpathcurveto{\pgfqpoint{0.005103in}{0.008786in}}{\pgfqpoint{0.002605in}{0.009821in}}{\pgfqpoint{0.000000in}{0.009821in}}%
\pgfpathcurveto{\pgfqpoint{-0.002605in}{0.009821in}}{\pgfqpoint{-0.005103in}{0.008786in}}{\pgfqpoint{-0.006944in}{0.006944in}}%
\pgfpathcurveto{\pgfqpoint{-0.008786in}{0.005103in}}{\pgfqpoint{-0.009821in}{0.002605in}}{\pgfqpoint{-0.009821in}{0.000000in}}%
\pgfpathcurveto{\pgfqpoint{-0.009821in}{-0.002605in}}{\pgfqpoint{-0.008786in}{-0.005103in}}{\pgfqpoint{-0.006944in}{-0.006944in}}%
\pgfpathcurveto{\pgfqpoint{-0.005103in}{-0.008786in}}{\pgfqpoint{-0.002605in}{-0.009821in}}{\pgfqpoint{0.000000in}{-0.009821in}}%
\pgfpathclose%
\pgfusepath{stroke,fill}%
}%
\begin{pgfscope}%
\pgfsys@transformshift{4.651734in}{2.598189in}%
\pgfsys@useobject{currentmarker}{}%
\end{pgfscope}%
\begin{pgfscope}%
\pgfsys@transformshift{4.801932in}{2.538573in}%
\pgfsys@useobject{currentmarker}{}%
\end{pgfscope}%
\begin{pgfscope}%
\pgfsys@transformshift{4.952129in}{2.482364in}%
\pgfsys@useobject{currentmarker}{}%
\end{pgfscope}%
\begin{pgfscope}%
\pgfsys@transformshift{5.102327in}{2.414594in}%
\pgfsys@useobject{currentmarker}{}%
\end{pgfscope}%
\begin{pgfscope}%
\pgfsys@transformshift{5.252524in}{2.337578in}%
\pgfsys@useobject{currentmarker}{}%
\end{pgfscope}%
\begin{pgfscope}%
\pgfsys@transformshift{5.402721in}{2.242210in}%
\pgfsys@useobject{currentmarker}{}%
\end{pgfscope}%
\begin{pgfscope}%
\pgfsys@transformshift{5.552919in}{2.178683in}%
\pgfsys@useobject{currentmarker}{}%
\end{pgfscope}%
\begin{pgfscope}%
\pgfsys@transformshift{5.703116in}{2.079836in}%
\pgfsys@useobject{currentmarker}{}%
\end{pgfscope}%
\begin{pgfscope}%
\pgfsys@transformshift{5.853314in}{1.927282in}%
\pgfsys@useobject{currentmarker}{}%
\end{pgfscope}%
\begin{pgfscope}%
\pgfsys@transformshift{6.003511in}{1.736432in}%
\pgfsys@useobject{currentmarker}{}%
\end{pgfscope}%
\begin{pgfscope}%
\pgfsys@transformshift{6.153708in}{1.643123in}%
\pgfsys@useobject{currentmarker}{}%
\end{pgfscope}%
\begin{pgfscope}%
\pgfsys@transformshift{6.303906in}{1.468885in}%
\pgfsys@useobject{currentmarker}{}%
\end{pgfscope}%
\begin{pgfscope}%
\pgfsys@transformshift{6.454103in}{1.255659in}%
\pgfsys@useobject{currentmarker}{}%
\end{pgfscope}%
\begin{pgfscope}%
\pgfsys@transformshift{6.604301in}{1.170680in}%
\pgfsys@useobject{currentmarker}{}%
\end{pgfscope}%
\begin{pgfscope}%
\pgfsys@transformshift{6.754498in}{0.973109in}%
\pgfsys@useobject{currentmarker}{}%
\end{pgfscope}%
\begin{pgfscope}%
\pgfsys@transformshift{6.904696in}{0.881693in}%
\pgfsys@useobject{currentmarker}{}%
\end{pgfscope}%
\begin{pgfscope}%
\pgfsys@transformshift{7.054893in}{0.767126in}%
\pgfsys@useobject{currentmarker}{}%
\end{pgfscope}%
\begin{pgfscope}%
\pgfsys@transformshift{7.205090in}{0.636527in}%
\pgfsys@useobject{currentmarker}{}%
\end{pgfscope}%
\begin{pgfscope}%
\pgfsys@transformshift{7.355288in}{0.637225in}%
\pgfsys@useobject{currentmarker}{}%
\end{pgfscope}%
\begin{pgfscope}%
\pgfsys@transformshift{7.505485in}{0.586604in}%
\pgfsys@useobject{currentmarker}{}%
\end{pgfscope}%
\begin{pgfscope}%
\pgfsys@transformshift{7.655683in}{0.540309in}%
\pgfsys@useobject{currentmarker}{}%
\end{pgfscope}%
\end{pgfscope}%
\begin{pgfscope}%
\pgfsetbuttcap%
\pgfsetroundjoin%
\definecolor{currentfill}{rgb}{0.000000,0.000000,0.000000}%
\pgfsetfillcolor{currentfill}%
\pgfsetlinewidth{0.803000pt}%
\definecolor{currentstroke}{rgb}{0.000000,0.000000,0.000000}%
\pgfsetstrokecolor{currentstroke}%
\pgfsetdash{}{0pt}%
\pgfsys@defobject{currentmarker}{\pgfqpoint{0.000000in}{-0.048611in}}{\pgfqpoint{0.000000in}{0.000000in}}{%
\pgfpathmoveto{\pgfqpoint{0.000000in}{0.000000in}}%
\pgfpathlineto{\pgfqpoint{0.000000in}{-0.048611in}}%
\pgfusepath{stroke,fill}%
}%
\begin{pgfscope}%
\pgfsys@transformshift{4.952129in}{0.387222in}%
\pgfsys@useobject{currentmarker}{}%
\end{pgfscope}%
\end{pgfscope}%
\begin{pgfscope}%
\definecolor{textcolor}{rgb}{0.000000,0.000000,0.000000}%
\pgfsetstrokecolor{textcolor}%
\pgfsetfillcolor{textcolor}%
\pgftext[x=4.952129in,y=0.290000in,,top]{\color{textcolor}\sffamily\fontsize{10.000000}{12.000000}\selectfont 4.40}%
\end{pgfscope}%
\begin{pgfscope}%
\pgfsetbuttcap%
\pgfsetroundjoin%
\definecolor{currentfill}{rgb}{0.000000,0.000000,0.000000}%
\pgfsetfillcolor{currentfill}%
\pgfsetlinewidth{0.803000pt}%
\definecolor{currentstroke}{rgb}{0.000000,0.000000,0.000000}%
\pgfsetstrokecolor{currentstroke}%
\pgfsetdash{}{0pt}%
\pgfsys@defobject{currentmarker}{\pgfqpoint{0.000000in}{-0.048611in}}{\pgfqpoint{0.000000in}{0.000000in}}{%
\pgfpathmoveto{\pgfqpoint{0.000000in}{0.000000in}}%
\pgfpathlineto{\pgfqpoint{0.000000in}{-0.048611in}}%
\pgfusepath{stroke,fill}%
}%
\begin{pgfscope}%
\pgfsys@transformshift{5.703116in}{0.387222in}%
\pgfsys@useobject{currentmarker}{}%
\end{pgfscope}%
\end{pgfscope}%
\begin{pgfscope}%
\definecolor{textcolor}{rgb}{0.000000,0.000000,0.000000}%
\pgfsetstrokecolor{textcolor}%
\pgfsetfillcolor{textcolor}%
\pgftext[x=5.703116in,y=0.290000in,,top]{\color{textcolor}\sffamily\fontsize{10.000000}{12.000000}\selectfont 4.45}%
\end{pgfscope}%
\begin{pgfscope}%
\pgfsetbuttcap%
\pgfsetroundjoin%
\definecolor{currentfill}{rgb}{0.000000,0.000000,0.000000}%
\pgfsetfillcolor{currentfill}%
\pgfsetlinewidth{0.803000pt}%
\definecolor{currentstroke}{rgb}{0.000000,0.000000,0.000000}%
\pgfsetstrokecolor{currentstroke}%
\pgfsetdash{}{0pt}%
\pgfsys@defobject{currentmarker}{\pgfqpoint{0.000000in}{-0.048611in}}{\pgfqpoint{0.000000in}{0.000000in}}{%
\pgfpathmoveto{\pgfqpoint{0.000000in}{0.000000in}}%
\pgfpathlineto{\pgfqpoint{0.000000in}{-0.048611in}}%
\pgfusepath{stroke,fill}%
}%
\begin{pgfscope}%
\pgfsys@transformshift{6.454103in}{0.387222in}%
\pgfsys@useobject{currentmarker}{}%
\end{pgfscope}%
\end{pgfscope}%
\begin{pgfscope}%
\definecolor{textcolor}{rgb}{0.000000,0.000000,0.000000}%
\pgfsetstrokecolor{textcolor}%
\pgfsetfillcolor{textcolor}%
\pgftext[x=6.454103in,y=0.290000in,,top]{\color{textcolor}\sffamily\fontsize{10.000000}{12.000000}\selectfont 4.50}%
\end{pgfscope}%
\begin{pgfscope}%
\pgfsetbuttcap%
\pgfsetroundjoin%
\definecolor{currentfill}{rgb}{0.000000,0.000000,0.000000}%
\pgfsetfillcolor{currentfill}%
\pgfsetlinewidth{0.803000pt}%
\definecolor{currentstroke}{rgb}{0.000000,0.000000,0.000000}%
\pgfsetstrokecolor{currentstroke}%
\pgfsetdash{}{0pt}%
\pgfsys@defobject{currentmarker}{\pgfqpoint{0.000000in}{-0.048611in}}{\pgfqpoint{0.000000in}{0.000000in}}{%
\pgfpathmoveto{\pgfqpoint{0.000000in}{0.000000in}}%
\pgfpathlineto{\pgfqpoint{0.000000in}{-0.048611in}}%
\pgfusepath{stroke,fill}%
}%
\begin{pgfscope}%
\pgfsys@transformshift{7.205090in}{0.387222in}%
\pgfsys@useobject{currentmarker}{}%
\end{pgfscope}%
\end{pgfscope}%
\begin{pgfscope}%
\definecolor{textcolor}{rgb}{0.000000,0.000000,0.000000}%
\pgfsetstrokecolor{textcolor}%
\pgfsetfillcolor{textcolor}%
\pgftext[x=7.205090in,y=0.290000in,,top]{\color{textcolor}\sffamily\fontsize{10.000000}{12.000000}\selectfont 4.55}%
\end{pgfscope}%
\begin{pgfscope}%
\pgfsetbuttcap%
\pgfsetroundjoin%
\definecolor{currentfill}{rgb}{0.000000,0.000000,0.000000}%
\pgfsetfillcolor{currentfill}%
\pgfsetlinewidth{0.803000pt}%
\definecolor{currentstroke}{rgb}{0.000000,0.000000,0.000000}%
\pgfsetstrokecolor{currentstroke}%
\pgfsetdash{}{0pt}%
\pgfsys@defobject{currentmarker}{\pgfqpoint{-0.048611in}{0.000000in}}{\pgfqpoint{0.000000in}{0.000000in}}{%
\pgfpathmoveto{\pgfqpoint{0.000000in}{0.000000in}}%
\pgfpathlineto{\pgfqpoint{-0.048611in}{0.000000in}}%
\pgfusepath{stroke,fill}%
}%
\begin{pgfscope}%
\pgfsys@transformshift{4.492417in}{0.559943in}%
\pgfsys@useobject{currentmarker}{}%
\end{pgfscope}%
\end{pgfscope}%
\begin{pgfscope}%
\definecolor{textcolor}{rgb}{0.000000,0.000000,0.000000}%
\pgfsetstrokecolor{textcolor}%
\pgfsetfillcolor{textcolor}%
\pgftext[x=4.174315in,y=0.507182in,left,base]{\color{textcolor}\sffamily\fontsize{10.000000}{12.000000}\selectfont 0.1}%
\end{pgfscope}%
\begin{pgfscope}%
\pgfsetbuttcap%
\pgfsetroundjoin%
\definecolor{currentfill}{rgb}{0.000000,0.000000,0.000000}%
\pgfsetfillcolor{currentfill}%
\pgfsetlinewidth{0.803000pt}%
\definecolor{currentstroke}{rgb}{0.000000,0.000000,0.000000}%
\pgfsetstrokecolor{currentstroke}%
\pgfsetdash{}{0pt}%
\pgfsys@defobject{currentmarker}{\pgfqpoint{-0.048611in}{0.000000in}}{\pgfqpoint{0.000000in}{0.000000in}}{%
\pgfpathmoveto{\pgfqpoint{0.000000in}{0.000000in}}%
\pgfpathlineto{\pgfqpoint{-0.048611in}{0.000000in}}%
\pgfusepath{stroke,fill}%
}%
\begin{pgfscope}%
\pgfsys@transformshift{4.492417in}{1.057152in}%
\pgfsys@useobject{currentmarker}{}%
\end{pgfscope}%
\end{pgfscope}%
\begin{pgfscope}%
\definecolor{textcolor}{rgb}{0.000000,0.000000,0.000000}%
\pgfsetstrokecolor{textcolor}%
\pgfsetfillcolor{textcolor}%
\pgftext[x=4.174315in,y=1.004391in,left,base]{\color{textcolor}\sffamily\fontsize{10.000000}{12.000000}\selectfont 0.2}%
\end{pgfscope}%
\begin{pgfscope}%
\pgfsetbuttcap%
\pgfsetroundjoin%
\definecolor{currentfill}{rgb}{0.000000,0.000000,0.000000}%
\pgfsetfillcolor{currentfill}%
\pgfsetlinewidth{0.803000pt}%
\definecolor{currentstroke}{rgb}{0.000000,0.000000,0.000000}%
\pgfsetstrokecolor{currentstroke}%
\pgfsetdash{}{0pt}%
\pgfsys@defobject{currentmarker}{\pgfqpoint{-0.048611in}{0.000000in}}{\pgfqpoint{0.000000in}{0.000000in}}{%
\pgfpathmoveto{\pgfqpoint{0.000000in}{0.000000in}}%
\pgfpathlineto{\pgfqpoint{-0.048611in}{0.000000in}}%
\pgfusepath{stroke,fill}%
}%
\begin{pgfscope}%
\pgfsys@transformshift{4.492417in}{1.554361in}%
\pgfsys@useobject{currentmarker}{}%
\end{pgfscope}%
\end{pgfscope}%
\begin{pgfscope}%
\definecolor{textcolor}{rgb}{0.000000,0.000000,0.000000}%
\pgfsetstrokecolor{textcolor}%
\pgfsetfillcolor{textcolor}%
\pgftext[x=4.174315in,y=1.501600in,left,base]{\color{textcolor}\sffamily\fontsize{10.000000}{12.000000}\selectfont 0.3}%
\end{pgfscope}%
\begin{pgfscope}%
\pgfsetbuttcap%
\pgfsetroundjoin%
\definecolor{currentfill}{rgb}{0.000000,0.000000,0.000000}%
\pgfsetfillcolor{currentfill}%
\pgfsetlinewidth{0.803000pt}%
\definecolor{currentstroke}{rgb}{0.000000,0.000000,0.000000}%
\pgfsetstrokecolor{currentstroke}%
\pgfsetdash{}{0pt}%
\pgfsys@defobject{currentmarker}{\pgfqpoint{-0.048611in}{0.000000in}}{\pgfqpoint{0.000000in}{0.000000in}}{%
\pgfpathmoveto{\pgfqpoint{0.000000in}{0.000000in}}%
\pgfpathlineto{\pgfqpoint{-0.048611in}{0.000000in}}%
\pgfusepath{stroke,fill}%
}%
\begin{pgfscope}%
\pgfsys@transformshift{4.492417in}{2.051570in}%
\pgfsys@useobject{currentmarker}{}%
\end{pgfscope}%
\end{pgfscope}%
\begin{pgfscope}%
\definecolor{textcolor}{rgb}{0.000000,0.000000,0.000000}%
\pgfsetstrokecolor{textcolor}%
\pgfsetfillcolor{textcolor}%
\pgftext[x=4.174315in,y=1.998809in,left,base]{\color{textcolor}\sffamily\fontsize{10.000000}{12.000000}\selectfont 0.4}%
\end{pgfscope}%
\begin{pgfscope}%
\pgfsetbuttcap%
\pgfsetroundjoin%
\definecolor{currentfill}{rgb}{0.000000,0.000000,0.000000}%
\pgfsetfillcolor{currentfill}%
\pgfsetlinewidth{0.803000pt}%
\definecolor{currentstroke}{rgb}{0.000000,0.000000,0.000000}%
\pgfsetstrokecolor{currentstroke}%
\pgfsetdash{}{0pt}%
\pgfsys@defobject{currentmarker}{\pgfqpoint{-0.048611in}{0.000000in}}{\pgfqpoint{0.000000in}{0.000000in}}{%
\pgfpathmoveto{\pgfqpoint{0.000000in}{0.000000in}}%
\pgfpathlineto{\pgfqpoint{-0.048611in}{0.000000in}}%
\pgfusepath{stroke,fill}%
}%
\begin{pgfscope}%
\pgfsys@transformshift{4.492417in}{2.548779in}%
\pgfsys@useobject{currentmarker}{}%
\end{pgfscope}%
\end{pgfscope}%
\begin{pgfscope}%
\definecolor{textcolor}{rgb}{0.000000,0.000000,0.000000}%
\pgfsetstrokecolor{textcolor}%
\pgfsetfillcolor{textcolor}%
\pgftext[x=4.174315in,y=2.496018in,left,base]{\color{textcolor}\sffamily\fontsize{10.000000}{12.000000}\selectfont 0.5}%
\end{pgfscope}%
\begin{pgfscope}%
\pgfpathrectangle{\pgfqpoint{4.492417in}{0.387222in}}{\pgfqpoint{3.322583in}{2.324444in}}%
\pgfusepath{clip}%
\pgfsetrectcap%
\pgfsetroundjoin%
\pgfsetlinewidth{1.505625pt}%
\definecolor{currentstroke}{rgb}{0.121569,0.466667,0.705882}%
\pgfsetstrokecolor{currentstroke}%
\pgfsetdash{}{0pt}%
\pgfpathmoveto{\pgfqpoint{4.651734in}{2.598189in}}%
\pgfpathlineto{\pgfqpoint{4.801932in}{2.538573in}}%
\pgfpathlineto{\pgfqpoint{4.952129in}{2.482364in}}%
\pgfpathlineto{\pgfqpoint{5.102327in}{2.414594in}}%
\pgfpathlineto{\pgfqpoint{5.252524in}{2.337578in}}%
\pgfpathlineto{\pgfqpoint{5.402721in}{2.242210in}}%
\pgfpathlineto{\pgfqpoint{5.552919in}{2.178683in}}%
\pgfpathlineto{\pgfqpoint{5.703116in}{2.079836in}}%
\pgfpathlineto{\pgfqpoint{5.853314in}{1.927282in}}%
\pgfpathlineto{\pgfqpoint{6.003511in}{1.736432in}}%
\pgfpathlineto{\pgfqpoint{6.153708in}{1.643123in}}%
\pgfpathlineto{\pgfqpoint{6.303906in}{1.468885in}}%
\pgfpathlineto{\pgfqpoint{6.454103in}{1.255659in}}%
\pgfpathlineto{\pgfqpoint{6.604301in}{1.170680in}}%
\pgfpathlineto{\pgfqpoint{6.754498in}{0.973109in}}%
\pgfpathlineto{\pgfqpoint{6.904696in}{0.881693in}}%
\pgfpathlineto{\pgfqpoint{7.054893in}{0.767126in}}%
\pgfpathlineto{\pgfqpoint{7.205090in}{0.636527in}}%
\pgfpathlineto{\pgfqpoint{7.355288in}{0.637225in}}%
\pgfpathlineto{\pgfqpoint{7.505485in}{0.586604in}}%
\pgfpathlineto{\pgfqpoint{7.655683in}{0.540309in}}%
\pgfusepath{stroke}%
\end{pgfscope}%
\begin{pgfscope}%
\pgfsetrectcap%
\pgfsetmiterjoin%
\pgfsetlinewidth{0.803000pt}%
\definecolor{currentstroke}{rgb}{0.000000,0.000000,0.000000}%
\pgfsetstrokecolor{currentstroke}%
\pgfsetdash{}{0pt}%
\pgfpathmoveto{\pgfqpoint{4.492417in}{0.387222in}}%
\pgfpathlineto{\pgfqpoint{4.492417in}{2.711667in}}%
\pgfusepath{stroke}%
\end{pgfscope}%
\begin{pgfscope}%
\pgfsetrectcap%
\pgfsetmiterjoin%
\pgfsetlinewidth{0.803000pt}%
\definecolor{currentstroke}{rgb}{0.000000,0.000000,0.000000}%
\pgfsetstrokecolor{currentstroke}%
\pgfsetdash{}{0pt}%
\pgfpathmoveto{\pgfqpoint{7.815000in}{0.387222in}}%
\pgfpathlineto{\pgfqpoint{7.815000in}{2.711667in}}%
\pgfusepath{stroke}%
\end{pgfscope}%
\begin{pgfscope}%
\pgfsetrectcap%
\pgfsetmiterjoin%
\pgfsetlinewidth{0.803000pt}%
\definecolor{currentstroke}{rgb}{0.000000,0.000000,0.000000}%
\pgfsetstrokecolor{currentstroke}%
\pgfsetdash{}{0pt}%
\pgfpathmoveto{\pgfqpoint{4.492417in}{0.387222in}}%
\pgfpathlineto{\pgfqpoint{7.815000in}{0.387222in}}%
\pgfusepath{stroke}%
\end{pgfscope}%
\begin{pgfscope}%
\pgfsetrectcap%
\pgfsetmiterjoin%
\pgfsetlinewidth{0.803000pt}%
\definecolor{currentstroke}{rgb}{0.000000,0.000000,0.000000}%
\pgfsetstrokecolor{currentstroke}%
\pgfsetdash{}{0pt}%
\pgfpathmoveto{\pgfqpoint{4.492417in}{2.711667in}}%
\pgfpathlineto{\pgfqpoint{7.815000in}{2.711667in}}%
\pgfusepath{stroke}%
\end{pgfscope}%
\begin{pgfscope}%
\definecolor{textcolor}{rgb}{0.000000,0.000000,0.000000}%
\pgfsetstrokecolor{textcolor}%
\pgfsetfillcolor{textcolor}%
\pgftext[x=6.153708in,y=2.795000in,,base]{\color{textcolor}\sffamily\fontsize{12.000000}{14.400000}\selectfont \(\displaystyle  N = 24 \)}%
\end{pgfscope}%
\end{pgfpicture}%
\makeatother%
\endgroup%
}
\scalebox{0.75}{%% Creator: Matplotlib, PGF backend
%%
%% To include the figure in your LaTeX document, write
%%   \input{<filename>.pgf}
%%
%% Make sure the required packages are loaded in your preamble
%%   \usepackage{pgf}
%%
%% Figures using additional raster images can only be included by \input if
%% they are in the same directory as the main LaTeX file. For loading figures
%% from other directories you can use the `import` package
%%   \usepackage{import}
%% and then include the figures with
%%   \import{<path to file>}{<filename>.pgf}
%%
%% Matplotlib used the following preamble
%%   \usepackage{fontspec}
%%   \setmainfont{DejaVuSerif.ttf}[Path=/home/lzh/anaconda3/envs/numana/lib/python3.7/site-packages/matplotlib/mpl-data/fonts/ttf/]
%%   \setsansfont{DejaVuSans.ttf}[Path=/home/lzh/anaconda3/envs/numana/lib/python3.7/site-packages/matplotlib/mpl-data/fonts/ttf/]
%%   \setmonofont{DejaVuSansMono.ttf}[Path=/home/lzh/anaconda3/envs/numana/lib/python3.7/site-packages/matplotlib/mpl-data/fonts/ttf/]
%%
\begingroup%
\makeatletter%
\begin{pgfpicture}%
\pgfpathrectangle{\pgfpointorigin}{\pgfqpoint{8.000000in}{4.000000in}}%
\pgfusepath{use as bounding box, clip}%
\begin{pgfscope}%
\pgfsetbuttcap%
\pgfsetmiterjoin%
\definecolor{currentfill}{rgb}{1.000000,1.000000,1.000000}%
\pgfsetfillcolor{currentfill}%
\pgfsetlinewidth{0.000000pt}%
\definecolor{currentstroke}{rgb}{1.000000,1.000000,1.000000}%
\pgfsetstrokecolor{currentstroke}%
\pgfsetdash{}{0pt}%
\pgfpathmoveto{\pgfqpoint{0.000000in}{0.000000in}}%
\pgfpathlineto{\pgfqpoint{8.000000in}{0.000000in}}%
\pgfpathlineto{\pgfqpoint{8.000000in}{4.000000in}}%
\pgfpathlineto{\pgfqpoint{0.000000in}{4.000000in}}%
\pgfpathclose%
\pgfusepath{fill}%
\end{pgfscope}%
\begin{pgfscope}%
\pgfsetbuttcap%
\pgfsetmiterjoin%
\definecolor{currentfill}{rgb}{1.000000,1.000000,1.000000}%
\pgfsetfillcolor{currentfill}%
\pgfsetlinewidth{0.000000pt}%
\definecolor{currentstroke}{rgb}{0.000000,0.000000,0.000000}%
\pgfsetstrokecolor{currentstroke}%
\pgfsetstrokeopacity{0.000000}%
\pgfsetdash{}{0pt}%
\pgfpathmoveto{\pgfqpoint{1.000000in}{0.440000in}}%
\pgfpathlineto{\pgfqpoint{3.818182in}{0.440000in}}%
\pgfpathlineto{\pgfqpoint{3.818182in}{3.520000in}}%
\pgfpathlineto{\pgfqpoint{1.000000in}{3.520000in}}%
\pgfpathclose%
\pgfusepath{fill}%
\end{pgfscope}%
\begin{pgfscope}%
\pgfsetbuttcap%
\pgfsetmiterjoin%
\definecolor{currentfill}{rgb}{0.950000,0.950000,0.950000}%
\pgfsetfillcolor{currentfill}%
\pgfsetfillopacity{0.500000}%
\pgfsetlinewidth{1.003750pt}%
\definecolor{currentstroke}{rgb}{0.950000,0.950000,0.950000}%
\pgfsetstrokecolor{currentstroke}%
\pgfsetstrokeopacity{0.500000}%
\pgfsetdash{}{0pt}%
\pgfpathmoveto{\pgfqpoint{2.714177in}{1.884582in}}%
\pgfpathlineto{\pgfqpoint{3.520755in}{1.193628in}}%
\pgfpathlineto{\pgfqpoint{3.579156in}{2.584074in}}%
\pgfpathlineto{\pgfqpoint{2.727606in}{3.204883in}}%
\pgfusepath{stroke,fill}%
\end{pgfscope}%
\begin{pgfscope}%
\pgfsetbuttcap%
\pgfsetmiterjoin%
\definecolor{currentfill}{rgb}{0.900000,0.900000,0.900000}%
\pgfsetfillcolor{currentfill}%
\pgfsetfillopacity{0.500000}%
\pgfsetlinewidth{1.003750pt}%
\definecolor{currentstroke}{rgb}{0.900000,0.900000,0.900000}%
\pgfsetstrokecolor{currentstroke}%
\pgfsetstrokeopacity{0.500000}%
\pgfsetdash{}{0pt}%
\pgfpathmoveto{\pgfqpoint{2.714177in}{1.884582in}}%
\pgfpathlineto{\pgfqpoint{1.407596in}{1.498294in}}%
\pgfpathlineto{\pgfqpoint{1.352930in}{2.858408in}}%
\pgfpathlineto{\pgfqpoint{2.727606in}{3.204883in}}%
\pgfusepath{stroke,fill}%
\end{pgfscope}%
\begin{pgfscope}%
\pgfsetbuttcap%
\pgfsetmiterjoin%
\definecolor{currentfill}{rgb}{0.925000,0.925000,0.925000}%
\pgfsetfillcolor{currentfill}%
\pgfsetfillopacity{0.500000}%
\pgfsetlinewidth{1.003750pt}%
\definecolor{currentstroke}{rgb}{0.925000,0.925000,0.925000}%
\pgfsetstrokecolor{currentstroke}%
\pgfsetstrokeopacity{0.500000}%
\pgfsetdash{}{0pt}%
\pgfpathmoveto{\pgfqpoint{2.714177in}{1.884582in}}%
\pgfpathlineto{\pgfqpoint{1.407596in}{1.498294in}}%
\pgfpathlineto{\pgfqpoint{2.146055in}{0.743733in}}%
\pgfpathlineto{\pgfqpoint{3.520755in}{1.193628in}}%
\pgfusepath{stroke,fill}%
\end{pgfscope}%
\begin{pgfscope}%
\pgfsetrectcap%
\pgfsetroundjoin%
\pgfsetlinewidth{0.803000pt}%
\definecolor{currentstroke}{rgb}{0.000000,0.000000,0.000000}%
\pgfsetstrokecolor{currentstroke}%
\pgfsetdash{}{0pt}%
\pgfpathmoveto{\pgfqpoint{3.520755in}{1.193628in}}%
\pgfpathlineto{\pgfqpoint{2.146055in}{0.743733in}}%
\pgfusepath{stroke}%
\end{pgfscope}%
\begin{pgfscope}%
\pgfsetbuttcap%
\pgfsetroundjoin%
\pgfsetlinewidth{0.803000pt}%
\definecolor{currentstroke}{rgb}{0.690196,0.690196,0.690196}%
\pgfsetstrokecolor{currentstroke}%
\pgfsetdash{}{0pt}%
\pgfpathmoveto{\pgfqpoint{3.436945in}{1.166199in}}%
\pgfpathlineto{\pgfqpoint{2.634268in}{1.860957in}}%
\pgfpathlineto{\pgfqpoint{2.643704in}{3.183736in}}%
\pgfusepath{stroke}%
\end{pgfscope}%
\begin{pgfscope}%
\pgfsetbuttcap%
\pgfsetroundjoin%
\pgfsetlinewidth{0.803000pt}%
\definecolor{currentstroke}{rgb}{0.690196,0.690196,0.690196}%
\pgfsetstrokecolor{currentstroke}%
\pgfsetdash{}{0pt}%
\pgfpathmoveto{\pgfqpoint{3.204592in}{1.090158in}}%
\pgfpathlineto{\pgfqpoint{2.412903in}{1.795511in}}%
\pgfpathlineto{\pgfqpoint{2.411161in}{3.125126in}}%
\pgfusepath{stroke}%
\end{pgfscope}%
\begin{pgfscope}%
\pgfsetbuttcap%
\pgfsetroundjoin%
\pgfsetlinewidth{0.803000pt}%
\definecolor{currentstroke}{rgb}{0.690196,0.690196,0.690196}%
\pgfsetstrokecolor{currentstroke}%
\pgfsetdash{}{0pt}%
\pgfpathmoveto{\pgfqpoint{2.968546in}{1.012908in}}%
\pgfpathlineto{\pgfqpoint{2.188277in}{1.729101in}}%
\pgfpathlineto{\pgfqpoint{2.175018in}{3.065608in}}%
\pgfusepath{stroke}%
\end{pgfscope}%
\begin{pgfscope}%
\pgfsetbuttcap%
\pgfsetroundjoin%
\pgfsetlinewidth{0.803000pt}%
\definecolor{currentstroke}{rgb}{0.690196,0.690196,0.690196}%
\pgfsetstrokecolor{currentstroke}%
\pgfsetdash{}{0pt}%
\pgfpathmoveto{\pgfqpoint{2.728720in}{0.934420in}}%
\pgfpathlineto{\pgfqpoint{1.960319in}{1.661706in}}%
\pgfpathlineto{\pgfqpoint{1.935190in}{3.005161in}}%
\pgfusepath{stroke}%
\end{pgfscope}%
\begin{pgfscope}%
\pgfsetbuttcap%
\pgfsetroundjoin%
\pgfsetlinewidth{0.803000pt}%
\definecolor{currentstroke}{rgb}{0.690196,0.690196,0.690196}%
\pgfsetstrokecolor{currentstroke}%
\pgfsetdash{}{0pt}%
\pgfpathmoveto{\pgfqpoint{2.485022in}{0.854665in}}%
\pgfpathlineto{\pgfqpoint{1.728952in}{1.593303in}}%
\pgfpathlineto{\pgfqpoint{1.691590in}{2.943764in}}%
\pgfusepath{stroke}%
\end{pgfscope}%
\begin{pgfscope}%
\pgfsetbuttcap%
\pgfsetroundjoin%
\pgfsetlinewidth{0.803000pt}%
\definecolor{currentstroke}{rgb}{0.690196,0.690196,0.690196}%
\pgfsetstrokecolor{currentstroke}%
\pgfsetdash{}{0pt}%
\pgfpathmoveto{\pgfqpoint{2.237356in}{0.773612in}}%
\pgfpathlineto{\pgfqpoint{1.494101in}{1.523869in}}%
\pgfpathlineto{\pgfqpoint{1.444129in}{2.881394in}}%
\pgfusepath{stroke}%
\end{pgfscope}%
\begin{pgfscope}%
\pgfsetrectcap%
\pgfsetroundjoin%
\pgfsetlinewidth{0.803000pt}%
\definecolor{currentstroke}{rgb}{0.000000,0.000000,0.000000}%
\pgfsetstrokecolor{currentstroke}%
\pgfsetdash{}{0pt}%
\pgfpathmoveto{\pgfqpoint{3.430029in}{1.172185in}}%
\pgfpathlineto{\pgfqpoint{3.450801in}{1.154206in}}%
\pgfusepath{stroke}%
\end{pgfscope}%
\begin{pgfscope}%
\definecolor{textcolor}{rgb}{0.000000,0.000000,0.000000}%
\pgfsetstrokecolor{textcolor}%
\pgfsetfillcolor{textcolor}%
\pgftext[x=3.525746in,y=0.940342in,,top]{\color{textcolor}\sffamily\fontsize{10.000000}{12.000000}\selectfont 0.0}%
\end{pgfscope}%
\begin{pgfscope}%
\pgfsetrectcap%
\pgfsetroundjoin%
\pgfsetlinewidth{0.803000pt}%
\definecolor{currentstroke}{rgb}{0.000000,0.000000,0.000000}%
\pgfsetstrokecolor{currentstroke}%
\pgfsetdash{}{0pt}%
\pgfpathmoveto{\pgfqpoint{3.197767in}{1.096238in}}%
\pgfpathlineto{\pgfqpoint{3.218267in}{1.077974in}}%
\pgfusepath{stroke}%
\end{pgfscope}%
\begin{pgfscope}%
\definecolor{textcolor}{rgb}{0.000000,0.000000,0.000000}%
\pgfsetstrokecolor{textcolor}%
\pgfsetfillcolor{textcolor}%
\pgftext[x=3.293434in,y=0.862298in,,top]{\color{textcolor}\sffamily\fontsize{10.000000}{12.000000}\selectfont 0.2}%
\end{pgfscope}%
\begin{pgfscope}%
\pgfsetrectcap%
\pgfsetroundjoin%
\pgfsetlinewidth{0.803000pt}%
\definecolor{currentstroke}{rgb}{0.000000,0.000000,0.000000}%
\pgfsetstrokecolor{currentstroke}%
\pgfsetdash{}{0pt}%
\pgfpathmoveto{\pgfqpoint{2.961816in}{1.019085in}}%
\pgfpathlineto{\pgfqpoint{2.982032in}{1.000529in}}%
\pgfusepath{stroke}%
\end{pgfscope}%
\begin{pgfscope}%
\definecolor{textcolor}{rgb}{0.000000,0.000000,0.000000}%
\pgfsetstrokecolor{textcolor}%
\pgfsetfillcolor{textcolor}%
\pgftext[x=3.057419in,y=0.783011in,,top]{\color{textcolor}\sffamily\fontsize{10.000000}{12.000000}\selectfont 0.4}%
\end{pgfscope}%
\begin{pgfscope}%
\pgfsetrectcap%
\pgfsetroundjoin%
\pgfsetlinewidth{0.803000pt}%
\definecolor{currentstroke}{rgb}{0.000000,0.000000,0.000000}%
\pgfsetstrokecolor{currentstroke}%
\pgfsetdash{}{0pt}%
\pgfpathmoveto{\pgfqpoint{2.722089in}{0.940697in}}%
\pgfpathlineto{\pgfqpoint{2.742009in}{0.921843in}}%
\pgfusepath{stroke}%
\end{pgfscope}%
\begin{pgfscope}%
\definecolor{textcolor}{rgb}{0.000000,0.000000,0.000000}%
\pgfsetstrokecolor{textcolor}%
\pgfsetfillcolor{textcolor}%
\pgftext[x=2.817614in,y=0.702450in,,top]{\color{textcolor}\sffamily\fontsize{10.000000}{12.000000}\selectfont 0.6}%
\end{pgfscope}%
\begin{pgfscope}%
\pgfsetrectcap%
\pgfsetroundjoin%
\pgfsetlinewidth{0.803000pt}%
\definecolor{currentstroke}{rgb}{0.000000,0.000000,0.000000}%
\pgfsetstrokecolor{currentstroke}%
\pgfsetdash{}{0pt}%
\pgfpathmoveto{\pgfqpoint{2.478493in}{0.861044in}}%
\pgfpathlineto{\pgfqpoint{2.498105in}{0.841884in}}%
\pgfusepath{stroke}%
\end{pgfscope}%
\begin{pgfscope}%
\definecolor{textcolor}{rgb}{0.000000,0.000000,0.000000}%
\pgfsetstrokecolor{textcolor}%
\pgfsetfillcolor{textcolor}%
\pgftext[x=2.573925in,y=0.620584in,,top]{\color{textcolor}\sffamily\fontsize{10.000000}{12.000000}\selectfont 0.8}%
\end{pgfscope}%
\begin{pgfscope}%
\pgfsetrectcap%
\pgfsetroundjoin%
\pgfsetlinewidth{0.803000pt}%
\definecolor{currentstroke}{rgb}{0.000000,0.000000,0.000000}%
\pgfsetstrokecolor{currentstroke}%
\pgfsetdash{}{0pt}%
\pgfpathmoveto{\pgfqpoint{2.230934in}{0.780095in}}%
\pgfpathlineto{\pgfqpoint{2.250225in}{0.760622in}}%
\pgfusepath{stroke}%
\end{pgfscope}%
\begin{pgfscope}%
\definecolor{textcolor}{rgb}{0.000000,0.000000,0.000000}%
\pgfsetstrokecolor{textcolor}%
\pgfsetfillcolor{textcolor}%
\pgftext[x=2.326257in,y=0.537382in,,top]{\color{textcolor}\sffamily\fontsize{10.000000}{12.000000}\selectfont 1.0}%
\end{pgfscope}%
\begin{pgfscope}%
\pgfsetrectcap%
\pgfsetroundjoin%
\pgfsetlinewidth{0.803000pt}%
\definecolor{currentstroke}{rgb}{0.000000,0.000000,0.000000}%
\pgfsetstrokecolor{currentstroke}%
\pgfsetdash{}{0pt}%
\pgfpathmoveto{\pgfqpoint{1.407596in}{1.498294in}}%
\pgfpathlineto{\pgfqpoint{2.146055in}{0.743733in}}%
\pgfusepath{stroke}%
\end{pgfscope}%
\begin{pgfscope}%
\pgfsetbuttcap%
\pgfsetroundjoin%
\pgfsetlinewidth{0.803000pt}%
\definecolor{currentstroke}{rgb}{0.690196,0.690196,0.690196}%
\pgfsetstrokecolor{currentstroke}%
\pgfsetdash{}{0pt}%
\pgfpathmoveto{\pgfqpoint{2.777949in}{3.168181in}}%
\pgfpathlineto{\pgfqpoint{2.762037in}{1.843583in}}%
\pgfpathlineto{\pgfqpoint{1.451276in}{1.453662in}}%
\pgfusepath{stroke}%
\end{pgfscope}%
\begin{pgfscope}%
\pgfsetbuttcap%
\pgfsetroundjoin%
\pgfsetlinewidth{0.803000pt}%
\definecolor{currentstroke}{rgb}{0.690196,0.690196,0.690196}%
\pgfsetstrokecolor{currentstroke}%
\pgfsetdash{}{0pt}%
\pgfpathmoveto{\pgfqpoint{2.918559in}{3.065671in}}%
\pgfpathlineto{\pgfqpoint{2.895594in}{1.729172in}}%
\pgfpathlineto{\pgfqpoint{1.573260in}{1.329018in}}%
\pgfusepath{stroke}%
\end{pgfscope}%
\begin{pgfscope}%
\pgfsetbuttcap%
\pgfsetroundjoin%
\pgfsetlinewidth{0.803000pt}%
\definecolor{currentstroke}{rgb}{0.690196,0.690196,0.690196}%
\pgfsetstrokecolor{currentstroke}%
\pgfsetdash{}{0pt}%
\pgfpathmoveto{\pgfqpoint{3.062992in}{2.960375in}}%
\pgfpathlineto{\pgfqpoint{3.032602in}{1.611804in}}%
\pgfpathlineto{\pgfqpoint{1.698537in}{1.201009in}}%
\pgfusepath{stroke}%
\end{pgfscope}%
\begin{pgfscope}%
\pgfsetbuttcap%
\pgfsetroundjoin%
\pgfsetlinewidth{0.803000pt}%
\definecolor{currentstroke}{rgb}{0.690196,0.690196,0.690196}%
\pgfsetstrokecolor{currentstroke}%
\pgfsetdash{}{0pt}%
\pgfpathmoveto{\pgfqpoint{3.211405in}{2.852177in}}%
\pgfpathlineto{\pgfqpoint{3.173197in}{1.491363in}}%
\pgfpathlineto{\pgfqpoint{1.827243in}{1.069497in}}%
\pgfusepath{stroke}%
\end{pgfscope}%
\begin{pgfscope}%
\pgfsetbuttcap%
\pgfsetroundjoin%
\pgfsetlinewidth{0.803000pt}%
\definecolor{currentstroke}{rgb}{0.690196,0.690196,0.690196}%
\pgfsetstrokecolor{currentstroke}%
\pgfsetdash{}{0pt}%
\pgfpathmoveto{\pgfqpoint{3.363965in}{2.740956in}}%
\pgfpathlineto{\pgfqpoint{3.317522in}{1.367728in}}%
\pgfpathlineto{\pgfqpoint{1.959519in}{0.934337in}}%
\pgfusepath{stroke}%
\end{pgfscope}%
\begin{pgfscope}%
\pgfsetbuttcap%
\pgfsetroundjoin%
\pgfsetlinewidth{0.803000pt}%
\definecolor{currentstroke}{rgb}{0.690196,0.690196,0.690196}%
\pgfsetstrokecolor{currentstroke}%
\pgfsetdash{}{0pt}%
\pgfpathmoveto{\pgfqpoint{3.520848in}{2.626583in}}%
\pgfpathlineto{\pgfqpoint{3.465727in}{1.240768in}}%
\pgfpathlineto{\pgfqpoint{2.095516in}{0.795374in}}%
\pgfusepath{stroke}%
\end{pgfscope}%
\begin{pgfscope}%
\pgfsetrectcap%
\pgfsetroundjoin%
\pgfsetlinewidth{0.803000pt}%
\definecolor{currentstroke}{rgb}{0.000000,0.000000,0.000000}%
\pgfsetstrokecolor{currentstroke}%
\pgfsetdash{}{0pt}%
\pgfpathmoveto{\pgfqpoint{1.462214in}{1.456916in}}%
\pgfpathlineto{\pgfqpoint{1.429376in}{1.447148in}}%
\pgfusepath{stroke}%
\end{pgfscope}%
\begin{pgfscope}%
\definecolor{textcolor}{rgb}{0.000000,0.000000,0.000000}%
\pgfsetstrokecolor{textcolor}%
\pgfsetfillcolor{textcolor}%
\pgftext[x=1.307986in,y=1.268483in,,top]{\color{textcolor}\sffamily\fontsize{10.000000}{12.000000}\selectfont 0.0}%
\end{pgfscope}%
\begin{pgfscope}%
\pgfsetrectcap%
\pgfsetroundjoin%
\pgfsetlinewidth{0.803000pt}%
\definecolor{currentstroke}{rgb}{0.000000,0.000000,0.000000}%
\pgfsetstrokecolor{currentstroke}%
\pgfsetdash{}{0pt}%
\pgfpathmoveto{\pgfqpoint{1.584301in}{1.332359in}}%
\pgfpathlineto{\pgfqpoint{1.551155in}{1.322329in}}%
\pgfusepath{stroke}%
\end{pgfscope}%
\begin{pgfscope}%
\definecolor{textcolor}{rgb}{0.000000,0.000000,0.000000}%
\pgfsetstrokecolor{textcolor}%
\pgfsetfillcolor{textcolor}%
\pgftext[x=1.427974in,y=1.141467in,,top]{\color{textcolor}\sffamily\fontsize{10.000000}{12.000000}\selectfont 0.2}%
\end{pgfscope}%
\begin{pgfscope}%
\pgfsetrectcap%
\pgfsetroundjoin%
\pgfsetlinewidth{0.803000pt}%
\definecolor{currentstroke}{rgb}{0.000000,0.000000,0.000000}%
\pgfsetstrokecolor{currentstroke}%
\pgfsetdash{}{0pt}%
\pgfpathmoveto{\pgfqpoint{1.709682in}{1.204441in}}%
\pgfpathlineto{\pgfqpoint{1.676223in}{1.194138in}}%
\pgfusepath{stroke}%
\end{pgfscope}%
\begin{pgfscope}%
\definecolor{textcolor}{rgb}{0.000000,0.000000,0.000000}%
\pgfsetstrokecolor{textcolor}%
\pgfsetfillcolor{textcolor}%
\pgftext[x=1.551197in,y=1.011025in,,top]{\color{textcolor}\sffamily\fontsize{10.000000}{12.000000}\selectfont 0.4}%
\end{pgfscope}%
\begin{pgfscope}%
\pgfsetrectcap%
\pgfsetroundjoin%
\pgfsetlinewidth{0.803000pt}%
\definecolor{currentstroke}{rgb}{0.000000,0.000000,0.000000}%
\pgfsetstrokecolor{currentstroke}%
\pgfsetdash{}{0pt}%
\pgfpathmoveto{\pgfqpoint{1.838494in}{1.073023in}}%
\pgfpathlineto{\pgfqpoint{1.804716in}{1.062437in}}%
\pgfusepath{stroke}%
\end{pgfscope}%
\begin{pgfscope}%
\definecolor{textcolor}{rgb}{0.000000,0.000000,0.000000}%
\pgfsetstrokecolor{textcolor}%
\pgfsetfillcolor{textcolor}%
\pgftext[x=1.677790in,y=0.877016in,,top]{\color{textcolor}\sffamily\fontsize{10.000000}{12.000000}\selectfont 0.6}%
\end{pgfscope}%
\begin{pgfscope}%
\pgfsetrectcap%
\pgfsetroundjoin%
\pgfsetlinewidth{0.803000pt}%
\definecolor{currentstroke}{rgb}{0.000000,0.000000,0.000000}%
\pgfsetstrokecolor{currentstroke}%
\pgfsetdash{}{0pt}%
\pgfpathmoveto{\pgfqpoint{1.970877in}{0.937961in}}%
\pgfpathlineto{\pgfqpoint{1.936777in}{0.927079in}}%
\pgfusepath{stroke}%
\end{pgfscope}%
\begin{pgfscope}%
\definecolor{textcolor}{rgb}{0.000000,0.000000,0.000000}%
\pgfsetstrokecolor{textcolor}%
\pgfsetfillcolor{textcolor}%
\pgftext[x=1.807893in,y=0.739291in,,top]{\color{textcolor}\sffamily\fontsize{10.000000}{12.000000}\selectfont 0.8}%
\end{pgfscope}%
\begin{pgfscope}%
\pgfsetrectcap%
\pgfsetroundjoin%
\pgfsetlinewidth{0.803000pt}%
\definecolor{currentstroke}{rgb}{0.000000,0.000000,0.000000}%
\pgfsetstrokecolor{currentstroke}%
\pgfsetdash{}{0pt}%
\pgfpathmoveto{\pgfqpoint{2.106984in}{0.799101in}}%
\pgfpathlineto{\pgfqpoint{2.072556in}{0.787910in}}%
\pgfusepath{stroke}%
\end{pgfscope}%
\begin{pgfscope}%
\definecolor{textcolor}{rgb}{0.000000,0.000000,0.000000}%
\pgfsetstrokecolor{textcolor}%
\pgfsetfillcolor{textcolor}%
\pgftext[x=1.941653in,y=0.597696in,,top]{\color{textcolor}\sffamily\fontsize{10.000000}{12.000000}\selectfont 1.0}%
\end{pgfscope}%
\begin{pgfscope}%
\pgfsetrectcap%
\pgfsetroundjoin%
\pgfsetlinewidth{0.803000pt}%
\definecolor{currentstroke}{rgb}{0.000000,0.000000,0.000000}%
\pgfsetstrokecolor{currentstroke}%
\pgfsetdash{}{0pt}%
\pgfpathmoveto{\pgfqpoint{1.407596in}{1.498294in}}%
\pgfpathlineto{\pgfqpoint{1.352930in}{2.858408in}}%
\pgfusepath{stroke}%
\end{pgfscope}%
\begin{pgfscope}%
\pgfsetbuttcap%
\pgfsetroundjoin%
\pgfsetlinewidth{0.803000pt}%
\definecolor{currentstroke}{rgb}{0.690196,0.690196,0.690196}%
\pgfsetstrokecolor{currentstroke}%
\pgfsetdash{}{0pt}%
\pgfpathmoveto{\pgfqpoint{1.404280in}{1.580785in}}%
\pgfpathlineto{\pgfqpoint{2.714993in}{1.964823in}}%
\pgfpathlineto{\pgfqpoint{3.524292in}{1.277822in}}%
\pgfusepath{stroke}%
\end{pgfscope}%
\begin{pgfscope}%
\pgfsetbuttcap%
\pgfsetroundjoin%
\pgfsetlinewidth{0.803000pt}%
\definecolor{currentstroke}{rgb}{0.690196,0.690196,0.690196}%
\pgfsetstrokecolor{currentstroke}%
\pgfsetdash{}{0pt}%
\pgfpathmoveto{\pgfqpoint{1.395077in}{1.809771in}}%
\pgfpathlineto{\pgfqpoint{2.717257in}{2.187449in}}%
\pgfpathlineto{\pgfqpoint{3.534112in}{1.511629in}}%
\pgfusepath{stroke}%
\end{pgfscope}%
\begin{pgfscope}%
\pgfsetbuttcap%
\pgfsetroundjoin%
\pgfsetlinewidth{0.803000pt}%
\definecolor{currentstroke}{rgb}{0.690196,0.690196,0.690196}%
\pgfsetstrokecolor{currentstroke}%
\pgfsetdash{}{0pt}%
\pgfpathmoveto{\pgfqpoint{1.385710in}{2.042834in}}%
\pgfpathlineto{\pgfqpoint{2.719560in}{2.413873in}}%
\pgfpathlineto{\pgfqpoint{3.544113in}{1.749737in}}%
\pgfusepath{stroke}%
\end{pgfscope}%
\begin{pgfscope}%
\pgfsetbuttcap%
\pgfsetroundjoin%
\pgfsetlinewidth{0.803000pt}%
\definecolor{currentstroke}{rgb}{0.690196,0.690196,0.690196}%
\pgfsetstrokecolor{currentstroke}%
\pgfsetdash{}{0pt}%
\pgfpathmoveto{\pgfqpoint{1.376174in}{2.280084in}}%
\pgfpathlineto{\pgfqpoint{2.721903in}{2.644191in}}%
\pgfpathlineto{\pgfqpoint{3.554299in}{1.992268in}}%
\pgfusepath{stroke}%
\end{pgfscope}%
\begin{pgfscope}%
\pgfsetbuttcap%
\pgfsetroundjoin%
\pgfsetlinewidth{0.803000pt}%
\definecolor{currentstroke}{rgb}{0.690196,0.690196,0.690196}%
\pgfsetstrokecolor{currentstroke}%
\pgfsetdash{}{0pt}%
\pgfpathmoveto{\pgfqpoint{1.366466in}{2.521636in}}%
\pgfpathlineto{\pgfqpoint{2.724286in}{2.878505in}}%
\pgfpathlineto{\pgfqpoint{3.564677in}{2.239346in}}%
\pgfusepath{stroke}%
\end{pgfscope}%
\begin{pgfscope}%
\pgfsetbuttcap%
\pgfsetroundjoin%
\pgfsetlinewidth{0.803000pt}%
\definecolor{currentstroke}{rgb}{0.690196,0.690196,0.690196}%
\pgfsetstrokecolor{currentstroke}%
\pgfsetdash{}{0pt}%
\pgfpathmoveto{\pgfqpoint{1.356579in}{2.767607in}}%
\pgfpathlineto{\pgfqpoint{2.726711in}{3.116919in}}%
\pgfpathlineto{\pgfqpoint{3.575251in}{2.491100in}}%
\pgfusepath{stroke}%
\end{pgfscope}%
\begin{pgfscope}%
\pgfsetrectcap%
\pgfsetroundjoin%
\pgfsetlinewidth{0.803000pt}%
\definecolor{currentstroke}{rgb}{0.000000,0.000000,0.000000}%
\pgfsetstrokecolor{currentstroke}%
\pgfsetdash{}{0pt}%
\pgfpathmoveto{\pgfqpoint{1.415217in}{1.583990in}}%
\pgfpathlineto{\pgfqpoint{1.382384in}{1.574369in}}%
\pgfusepath{stroke}%
\end{pgfscope}%
\begin{pgfscope}%
\definecolor{textcolor}{rgb}{0.000000,0.000000,0.000000}%
\pgfsetstrokecolor{textcolor}%
\pgfsetfillcolor{textcolor}%
\pgftext[x=1.181686in,y=1.612595in,,top]{\color{textcolor}\sffamily\fontsize{10.000000}{12.000000}\selectfont 0.0}%
\end{pgfscope}%
\begin{pgfscope}%
\pgfsetrectcap%
\pgfsetroundjoin%
\pgfsetlinewidth{0.803000pt}%
\definecolor{currentstroke}{rgb}{0.000000,0.000000,0.000000}%
\pgfsetstrokecolor{currentstroke}%
\pgfsetdash{}{0pt}%
\pgfpathmoveto{\pgfqpoint{1.406114in}{1.812923in}}%
\pgfpathlineto{\pgfqpoint{1.372980in}{1.803459in}}%
\pgfusepath{stroke}%
\end{pgfscope}%
\begin{pgfscope}%
\definecolor{textcolor}{rgb}{0.000000,0.000000,0.000000}%
\pgfsetstrokecolor{textcolor}%
\pgfsetfillcolor{textcolor}%
\pgftext[x=1.170557in,y=1.841065in,,top]{\color{textcolor}\sffamily\fontsize{10.000000}{12.000000}\selectfont 0.2}%
\end{pgfscope}%
\begin{pgfscope}%
\pgfsetrectcap%
\pgfsetroundjoin%
\pgfsetlinewidth{0.803000pt}%
\definecolor{currentstroke}{rgb}{0.000000,0.000000,0.000000}%
\pgfsetstrokecolor{currentstroke}%
\pgfsetdash{}{0pt}%
\pgfpathmoveto{\pgfqpoint{1.396848in}{2.045932in}}%
\pgfpathlineto{\pgfqpoint{1.363409in}{2.036631in}}%
\pgfusepath{stroke}%
\end{pgfscope}%
\begin{pgfscope}%
\definecolor{textcolor}{rgb}{0.000000,0.000000,0.000000}%
\pgfsetstrokecolor{textcolor}%
\pgfsetfillcolor{textcolor}%
\pgftext[x=1.159231in,y=2.073588in,,top]{\color{textcolor}\sffamily\fontsize{10.000000}{12.000000}\selectfont 0.4}%
\end{pgfscope}%
\begin{pgfscope}%
\pgfsetrectcap%
\pgfsetroundjoin%
\pgfsetlinewidth{0.803000pt}%
\definecolor{currentstroke}{rgb}{0.000000,0.000000,0.000000}%
\pgfsetstrokecolor{currentstroke}%
\pgfsetdash{}{0pt}%
\pgfpathmoveto{\pgfqpoint{1.387416in}{2.283126in}}%
\pgfpathlineto{\pgfqpoint{1.353667in}{2.273995in}}%
\pgfusepath{stroke}%
\end{pgfscope}%
\begin{pgfscope}%
\definecolor{textcolor}{rgb}{0.000000,0.000000,0.000000}%
\pgfsetstrokecolor{textcolor}%
\pgfsetfillcolor{textcolor}%
\pgftext[x=1.147701in,y=2.310275in,,top]{\color{textcolor}\sffamily\fontsize{10.000000}{12.000000}\selectfont 0.6}%
\end{pgfscope}%
\begin{pgfscope}%
\pgfsetrectcap%
\pgfsetroundjoin%
\pgfsetlinewidth{0.803000pt}%
\definecolor{currentstroke}{rgb}{0.000000,0.000000,0.000000}%
\pgfsetstrokecolor{currentstroke}%
\pgfsetdash{}{0pt}%
\pgfpathmoveto{\pgfqpoint{1.377813in}{2.524619in}}%
\pgfpathlineto{\pgfqpoint{1.343747in}{2.515665in}}%
\pgfusepath{stroke}%
\end{pgfscope}%
\begin{pgfscope}%
\definecolor{textcolor}{rgb}{0.000000,0.000000,0.000000}%
\pgfsetstrokecolor{textcolor}%
\pgfsetfillcolor{textcolor}%
\pgftext[x=1.135964in,y=2.551237in,,top]{\color{textcolor}\sffamily\fontsize{10.000000}{12.000000}\selectfont 0.8}%
\end{pgfscope}%
\begin{pgfscope}%
\pgfsetrectcap%
\pgfsetroundjoin%
\pgfsetlinewidth{0.803000pt}%
\definecolor{currentstroke}{rgb}{0.000000,0.000000,0.000000}%
\pgfsetstrokecolor{currentstroke}%
\pgfsetdash{}{0pt}%
\pgfpathmoveto{\pgfqpoint{1.368034in}{2.770528in}}%
\pgfpathlineto{\pgfqpoint{1.333646in}{2.761761in}}%
\pgfusepath{stroke}%
\end{pgfscope}%
\begin{pgfscope}%
\definecolor{textcolor}{rgb}{0.000000,0.000000,0.000000}%
\pgfsetstrokecolor{textcolor}%
\pgfsetfillcolor{textcolor}%
\pgftext[x=1.124012in,y=2.796592in,,top]{\color{textcolor}\sffamily\fontsize{10.000000}{12.000000}\selectfont 1.0}%
\end{pgfscope}%
\begin{pgfscope}%
\pgfpathrectangle{\pgfqpoint{1.000000in}{0.570909in}}{\pgfqpoint{2.818182in}{2.818182in}}%
\pgfusepath{clip}%
\pgfsetbuttcap%
\pgfsetroundjoin%
\definecolor{currentfill}{rgb}{1.000000,0.498039,0.054902}%
\pgfsetfillcolor{currentfill}%
\pgfsetlinewidth{1.003750pt}%
\definecolor{currentstroke}{rgb}{1.000000,0.498039,0.054902}%
\pgfsetstrokecolor{currentstroke}%
\pgfsetdash{}{0pt}%
\pgfpathmoveto{\pgfqpoint{2.539981in}{1.774825in}}%
\pgfpathcurveto{\pgfqpoint{2.548218in}{1.774825in}}{\pgfqpoint{2.556118in}{1.778097in}}{\pgfqpoint{2.561942in}{1.783921in}}%
\pgfpathcurveto{\pgfqpoint{2.567766in}{1.789745in}}{\pgfqpoint{2.571038in}{1.797645in}}{\pgfqpoint{2.571038in}{1.805881in}}%
\pgfpathcurveto{\pgfqpoint{2.571038in}{1.814117in}}{\pgfqpoint{2.567766in}{1.822017in}}{\pgfqpoint{2.561942in}{1.827841in}}%
\pgfpathcurveto{\pgfqpoint{2.556118in}{1.833665in}}{\pgfqpoint{2.548218in}{1.836938in}}{\pgfqpoint{2.539981in}{1.836938in}}%
\pgfpathcurveto{\pgfqpoint{2.531745in}{1.836938in}}{\pgfqpoint{2.523845in}{1.833665in}}{\pgfqpoint{2.518021in}{1.827841in}}%
\pgfpathcurveto{\pgfqpoint{2.512197in}{1.822017in}}{\pgfqpoint{2.508925in}{1.814117in}}{\pgfqpoint{2.508925in}{1.805881in}}%
\pgfpathcurveto{\pgfqpoint{2.508925in}{1.797645in}}{\pgfqpoint{2.512197in}{1.789745in}}{\pgfqpoint{2.518021in}{1.783921in}}%
\pgfpathcurveto{\pgfqpoint{2.523845in}{1.778097in}}{\pgfqpoint{2.531745in}{1.774825in}}{\pgfqpoint{2.539981in}{1.774825in}}%
\pgfpathclose%
\pgfusepath{stroke,fill}%
\end{pgfscope}%
\begin{pgfscope}%
\definecolor{textcolor}{rgb}{0.000000,0.000000,0.000000}%
\pgfsetstrokecolor{textcolor}%
\pgfsetfillcolor{textcolor}%
\pgftext[x=2.553362in,y=2.469525in,left,base]{\color{textcolor}\sffamily\fontsize{10.000000}{12.000000}\selectfont 0.04167}%
\end{pgfscope}%
\begin{pgfscope}%
\definecolor{textcolor}{rgb}{0.000000,0.000000,0.000000}%
\pgfsetstrokecolor{textcolor}%
\pgfsetfillcolor{textcolor}%
\pgftext[x=2.539981in,y=1.805881in,left,base]{\color{textcolor}\sffamily\fontsize{10.000000}{12.000000}\selectfont 0.04167}%
\end{pgfscope}%
\begin{pgfscope}%
\definecolor{textcolor}{rgb}{0.000000,0.000000,0.000000}%
\pgfsetstrokecolor{textcolor}%
\pgfsetfillcolor{textcolor}%
\pgftext[x=2.175861in,y=1.754937in,left,base]{\color{textcolor}\sffamily\fontsize{10.000000}{12.000000}\selectfont 0.04167}%
\end{pgfscope}%
\begin{pgfscope}%
\definecolor{textcolor}{rgb}{0.000000,0.000000,0.000000}%
\pgfsetstrokecolor{textcolor}%
\pgfsetfillcolor{textcolor}%
\pgftext[x=3.011410in,y=1.795988in,left,base]{\color{textcolor}\sffamily\fontsize{10.000000}{12.000000}\selectfont 0.04167}%
\end{pgfscope}%
\begin{pgfscope}%
\pgfpathrectangle{\pgfqpoint{1.000000in}{0.570909in}}{\pgfqpoint{2.818182in}{2.818182in}}%
\pgfusepath{clip}%
\pgfsetbuttcap%
\pgfsetroundjoin%
\definecolor{currentfill}{rgb}{0.121569,0.466667,0.705882}%
\pgfsetfillcolor{currentfill}%
\pgfsetlinewidth{1.003750pt}%
\definecolor{currentstroke}{rgb}{0.121569,0.466667,0.705882}%
\pgfsetstrokecolor{currentstroke}%
\pgfsetdash{}{0pt}%
\pgfpathmoveto{\pgfqpoint{2.553362in}{2.438468in}}%
\pgfpathcurveto{\pgfqpoint{2.561599in}{2.438468in}}{\pgfqpoint{2.569499in}{2.441741in}}{\pgfqpoint{2.575323in}{2.447564in}}%
\pgfpathcurveto{\pgfqpoint{2.581147in}{2.453388in}}{\pgfqpoint{2.584419in}{2.461288in}}{\pgfqpoint{2.584419in}{2.469525in}}%
\pgfpathcurveto{\pgfqpoint{2.584419in}{2.477761in}}{\pgfqpoint{2.581147in}{2.485661in}}{\pgfqpoint{2.575323in}{2.491485in}}%
\pgfpathcurveto{\pgfqpoint{2.569499in}{2.497309in}}{\pgfqpoint{2.561599in}{2.500581in}}{\pgfqpoint{2.553362in}{2.500581in}}%
\pgfpathcurveto{\pgfqpoint{2.545126in}{2.500581in}}{\pgfqpoint{2.537226in}{2.497309in}}{\pgfqpoint{2.531402in}{2.491485in}}%
\pgfpathcurveto{\pgfqpoint{2.525578in}{2.485661in}}{\pgfqpoint{2.522306in}{2.477761in}}{\pgfqpoint{2.522306in}{2.469525in}}%
\pgfpathcurveto{\pgfqpoint{2.522306in}{2.461288in}}{\pgfqpoint{2.525578in}{2.453388in}}{\pgfqpoint{2.531402in}{2.447564in}}%
\pgfpathcurveto{\pgfqpoint{2.537226in}{2.441741in}}{\pgfqpoint{2.545126in}{2.438468in}}{\pgfqpoint{2.553362in}{2.438468in}}%
\pgfpathclose%
\pgfusepath{stroke,fill}%
\end{pgfscope}%
\begin{pgfscope}%
\pgfpathrectangle{\pgfqpoint{1.000000in}{0.570909in}}{\pgfqpoint{2.818182in}{2.818182in}}%
\pgfusepath{clip}%
\pgfsetbuttcap%
\pgfsetroundjoin%
\definecolor{currentfill}{rgb}{0.172549,0.627451,0.172549}%
\pgfsetfillcolor{currentfill}%
\pgfsetlinewidth{1.003750pt}%
\definecolor{currentstroke}{rgb}{0.172549,0.627451,0.172549}%
\pgfsetstrokecolor{currentstroke}%
\pgfsetdash{}{0pt}%
\pgfpathmoveto{\pgfqpoint{2.175861in}{1.723881in}}%
\pgfpathcurveto{\pgfqpoint{2.184097in}{1.723881in}}{\pgfqpoint{2.191997in}{1.727153in}}{\pgfqpoint{2.197821in}{1.732977in}}%
\pgfpathcurveto{\pgfqpoint{2.203645in}{1.738801in}}{\pgfqpoint{2.206917in}{1.746701in}}{\pgfqpoint{2.206917in}{1.754937in}}%
\pgfpathcurveto{\pgfqpoint{2.206917in}{1.763174in}}{\pgfqpoint{2.203645in}{1.771074in}}{\pgfqpoint{2.197821in}{1.776898in}}%
\pgfpathcurveto{\pgfqpoint{2.191997in}{1.782721in}}{\pgfqpoint{2.184097in}{1.785994in}}{\pgfqpoint{2.175861in}{1.785994in}}%
\pgfpathcurveto{\pgfqpoint{2.167624in}{1.785994in}}{\pgfqpoint{2.159724in}{1.782721in}}{\pgfqpoint{2.153900in}{1.776898in}}%
\pgfpathcurveto{\pgfqpoint{2.148076in}{1.771074in}}{\pgfqpoint{2.144804in}{1.763174in}}{\pgfqpoint{2.144804in}{1.754937in}}%
\pgfpathcurveto{\pgfqpoint{2.144804in}{1.746701in}}{\pgfqpoint{2.148076in}{1.738801in}}{\pgfqpoint{2.153900in}{1.732977in}}%
\pgfpathcurveto{\pgfqpoint{2.159724in}{1.727153in}}{\pgfqpoint{2.167624in}{1.723881in}}{\pgfqpoint{2.175861in}{1.723881in}}%
\pgfpathclose%
\pgfusepath{stroke,fill}%
\end{pgfscope}%
\begin{pgfscope}%
\pgfpathrectangle{\pgfqpoint{1.000000in}{0.570909in}}{\pgfqpoint{2.818182in}{2.818182in}}%
\pgfusepath{clip}%
\pgfsetbuttcap%
\pgfsetroundjoin%
\definecolor{currentfill}{rgb}{0.839216,0.152941,0.156863}%
\pgfsetfillcolor{currentfill}%
\pgfsetlinewidth{1.003750pt}%
\definecolor{currentstroke}{rgb}{0.839216,0.152941,0.156863}%
\pgfsetstrokecolor{currentstroke}%
\pgfsetdash{}{0pt}%
\pgfpathmoveto{\pgfqpoint{3.011410in}{1.764932in}}%
\pgfpathcurveto{\pgfqpoint{3.019647in}{1.764932in}}{\pgfqpoint{3.027547in}{1.768204in}}{\pgfqpoint{3.033371in}{1.774028in}}%
\pgfpathcurveto{\pgfqpoint{3.039195in}{1.779852in}}{\pgfqpoint{3.042467in}{1.787752in}}{\pgfqpoint{3.042467in}{1.795988in}}%
\pgfpathcurveto{\pgfqpoint{3.042467in}{1.804225in}}{\pgfqpoint{3.039195in}{1.812125in}}{\pgfqpoint{3.033371in}{1.817949in}}%
\pgfpathcurveto{\pgfqpoint{3.027547in}{1.823773in}}{\pgfqpoint{3.019647in}{1.827045in}}{\pgfqpoint{3.011410in}{1.827045in}}%
\pgfpathcurveto{\pgfqpoint{3.003174in}{1.827045in}}{\pgfqpoint{2.995274in}{1.823773in}}{\pgfqpoint{2.989450in}{1.817949in}}%
\pgfpathcurveto{\pgfqpoint{2.983626in}{1.812125in}}{\pgfqpoint{2.980354in}{1.804225in}}{\pgfqpoint{2.980354in}{1.795988in}}%
\pgfpathcurveto{\pgfqpoint{2.980354in}{1.787752in}}{\pgfqpoint{2.983626in}{1.779852in}}{\pgfqpoint{2.989450in}{1.774028in}}%
\pgfpathcurveto{\pgfqpoint{2.995274in}{1.768204in}}{\pgfqpoint{3.003174in}{1.764932in}}{\pgfqpoint{3.011410in}{1.764932in}}%
\pgfpathclose%
\pgfusepath{stroke,fill}%
\end{pgfscope}%
\begin{pgfscope}%
\pgfpathrectangle{\pgfqpoint{1.000000in}{0.570909in}}{\pgfqpoint{2.818182in}{2.818182in}}%
\pgfusepath{clip}%
\pgfsetbuttcap%
\pgfsetroundjoin%
\definecolor{currentfill}{rgb}{0.100000,0.100000,0.100000}%
\pgfsetfillcolor{currentfill}%
\pgfsetfillopacity{0.100000}%
\pgfsetlinewidth{0.301125pt}%
\definecolor{currentstroke}{rgb}{0.000000,0.000000,0.000000}%
\pgfsetstrokecolor{currentstroke}%
\pgfsetdash{}{0pt}%
\pgfpathmoveto{\pgfqpoint{2.682610in}{1.900366in}}%
\pgfpathlineto{\pgfqpoint{1.535168in}{1.562080in}}%
\pgfpathlineto{\pgfqpoint{2.692989in}{3.058384in}}%
\pgfpathclose%
\pgfusepath{stroke,fill}%
\end{pgfscope}%
\begin{pgfscope}%
\pgfpathrectangle{\pgfqpoint{1.000000in}{0.570909in}}{\pgfqpoint{2.818182in}{2.818182in}}%
\pgfusepath{clip}%
\pgfsetbuttcap%
\pgfsetroundjoin%
\definecolor{currentfill}{rgb}{0.100000,0.100000,0.100000}%
\pgfsetfillcolor{currentfill}%
\pgfsetfillopacity{0.100000}%
\pgfsetlinewidth{0.301125pt}%
\definecolor{currentstroke}{rgb}{0.000000,0.000000,0.000000}%
\pgfsetstrokecolor{currentstroke}%
\pgfsetdash{}{0pt}%
\pgfpathmoveto{\pgfqpoint{2.682610in}{1.900366in}}%
\pgfpathlineto{\pgfqpoint{3.385245in}{1.297693in}}%
\pgfpathlineto{\pgfqpoint{1.535168in}{1.562080in}}%
\pgfpathclose%
\pgfusepath{stroke,fill}%
\end{pgfscope}%
\begin{pgfscope}%
\pgfpathrectangle{\pgfqpoint{1.000000in}{0.570909in}}{\pgfqpoint{2.818182in}{2.818182in}}%
\pgfusepath{clip}%
\pgfsetbuttcap%
\pgfsetroundjoin%
\definecolor{currentfill}{rgb}{0.100000,0.100000,0.100000}%
\pgfsetfillcolor{currentfill}%
\pgfsetfillopacity{0.100000}%
\pgfsetlinewidth{0.301125pt}%
\definecolor{currentstroke}{rgb}{0.000000,0.000000,0.000000}%
\pgfsetstrokecolor{currentstroke}%
\pgfsetdash{}{0pt}%
\pgfpathmoveto{\pgfqpoint{2.682610in}{1.900366in}}%
\pgfpathlineto{\pgfqpoint{3.385245in}{1.297693in}}%
\pgfpathlineto{\pgfqpoint{2.692989in}{3.058384in}}%
\pgfpathclose%
\pgfusepath{stroke,fill}%
\end{pgfscope}%
\begin{pgfscope}%
\pgfpathrectangle{\pgfqpoint{1.000000in}{0.570909in}}{\pgfqpoint{2.818182in}{2.818182in}}%
\pgfusepath{clip}%
\pgfsetbuttcap%
\pgfsetroundjoin%
\definecolor{currentfill}{rgb}{0.100000,0.100000,0.100000}%
\pgfsetfillcolor{currentfill}%
\pgfsetfillopacity{0.100000}%
\pgfsetlinewidth{0.301125pt}%
\definecolor{currentstroke}{rgb}{0.000000,0.000000,0.000000}%
\pgfsetstrokecolor{currentstroke}%
\pgfsetdash{}{0pt}%
\pgfpathmoveto{\pgfqpoint{2.692989in}{3.058384in}}%
\pgfpathlineto{\pgfqpoint{3.385245in}{1.297693in}}%
\pgfpathlineto{\pgfqpoint{1.535168in}{1.562080in}}%
\pgfpathclose%
\pgfusepath{stroke,fill}%
\end{pgfscope}%
\begin{pgfscope}%
\pgfsetbuttcap%
\pgfsetmiterjoin%
\definecolor{currentfill}{rgb}{1.000000,1.000000,1.000000}%
\pgfsetfillcolor{currentfill}%
\pgfsetlinewidth{0.000000pt}%
\definecolor{currentstroke}{rgb}{0.000000,0.000000,0.000000}%
\pgfsetstrokecolor{currentstroke}%
\pgfsetstrokeopacity{0.000000}%
\pgfsetdash{}{0pt}%
\pgfpathmoveto{\pgfqpoint{4.381818in}{0.440000in}}%
\pgfpathlineto{\pgfqpoint{7.200000in}{0.440000in}}%
\pgfpathlineto{\pgfqpoint{7.200000in}{3.520000in}}%
\pgfpathlineto{\pgfqpoint{4.381818in}{3.520000in}}%
\pgfpathclose%
\pgfusepath{fill}%
\end{pgfscope}%
\begin{pgfscope}%
\pgfsetbuttcap%
\pgfsetmiterjoin%
\definecolor{currentfill}{rgb}{0.950000,0.950000,0.950000}%
\pgfsetfillcolor{currentfill}%
\pgfsetfillopacity{0.500000}%
\pgfsetlinewidth{1.003750pt}%
\definecolor{currentstroke}{rgb}{0.950000,0.950000,0.950000}%
\pgfsetstrokecolor{currentstroke}%
\pgfsetstrokeopacity{0.500000}%
\pgfsetdash{}{0pt}%
\pgfpathmoveto{\pgfqpoint{4.755412in}{1.193628in}}%
\pgfpathlineto{\pgfqpoint{5.561990in}{1.884582in}}%
\pgfpathlineto{\pgfqpoint{5.548561in}{3.204883in}}%
\pgfpathlineto{\pgfqpoint{4.697011in}{2.584074in}}%
\pgfusepath{stroke,fill}%
\end{pgfscope}%
\begin{pgfscope}%
\pgfsetbuttcap%
\pgfsetmiterjoin%
\definecolor{currentfill}{rgb}{0.900000,0.900000,0.900000}%
\pgfsetfillcolor{currentfill}%
\pgfsetfillopacity{0.500000}%
\pgfsetlinewidth{1.003750pt}%
\definecolor{currentstroke}{rgb}{0.900000,0.900000,0.900000}%
\pgfsetstrokecolor{currentstroke}%
\pgfsetstrokeopacity{0.500000}%
\pgfsetdash{}{0pt}%
\pgfpathmoveto{\pgfqpoint{5.561990in}{1.884582in}}%
\pgfpathlineto{\pgfqpoint{6.868571in}{1.498294in}}%
\pgfpathlineto{\pgfqpoint{6.923237in}{2.858408in}}%
\pgfpathlineto{\pgfqpoint{5.548561in}{3.204883in}}%
\pgfusepath{stroke,fill}%
\end{pgfscope}%
\begin{pgfscope}%
\pgfsetbuttcap%
\pgfsetmiterjoin%
\definecolor{currentfill}{rgb}{0.925000,0.925000,0.925000}%
\pgfsetfillcolor{currentfill}%
\pgfsetfillopacity{0.500000}%
\pgfsetlinewidth{1.003750pt}%
\definecolor{currentstroke}{rgb}{0.925000,0.925000,0.925000}%
\pgfsetstrokecolor{currentstroke}%
\pgfsetstrokeopacity{0.500000}%
\pgfsetdash{}{0pt}%
\pgfpathmoveto{\pgfqpoint{4.755412in}{1.193628in}}%
\pgfpathlineto{\pgfqpoint{6.130112in}{0.743733in}}%
\pgfpathlineto{\pgfqpoint{6.868571in}{1.498294in}}%
\pgfpathlineto{\pgfqpoint{5.561990in}{1.884582in}}%
\pgfusepath{stroke,fill}%
\end{pgfscope}%
\begin{pgfscope}%
\pgfsetrectcap%
\pgfsetroundjoin%
\pgfsetlinewidth{0.803000pt}%
\definecolor{currentstroke}{rgb}{0.000000,0.000000,0.000000}%
\pgfsetstrokecolor{currentstroke}%
\pgfsetdash{}{0pt}%
\pgfpathmoveto{\pgfqpoint{4.755412in}{1.193628in}}%
\pgfpathlineto{\pgfqpoint{6.130112in}{0.743733in}}%
\pgfusepath{stroke}%
\end{pgfscope}%
\begin{pgfscope}%
\pgfsetbuttcap%
\pgfsetroundjoin%
\pgfsetlinewidth{0.803000pt}%
\definecolor{currentstroke}{rgb}{0.690196,0.690196,0.690196}%
\pgfsetstrokecolor{currentstroke}%
\pgfsetdash{}{0pt}%
\pgfpathmoveto{\pgfqpoint{4.839222in}{1.166199in}}%
\pgfpathlineto{\pgfqpoint{5.641899in}{1.860957in}}%
\pgfpathlineto{\pgfqpoint{5.632463in}{3.183736in}}%
\pgfusepath{stroke}%
\end{pgfscope}%
\begin{pgfscope}%
\pgfsetbuttcap%
\pgfsetroundjoin%
\pgfsetlinewidth{0.803000pt}%
\definecolor{currentstroke}{rgb}{0.690196,0.690196,0.690196}%
\pgfsetstrokecolor{currentstroke}%
\pgfsetdash{}{0pt}%
\pgfpathmoveto{\pgfqpoint{5.071575in}{1.090158in}}%
\pgfpathlineto{\pgfqpoint{5.863264in}{1.795511in}}%
\pgfpathlineto{\pgfqpoint{5.865006in}{3.125126in}}%
\pgfusepath{stroke}%
\end{pgfscope}%
\begin{pgfscope}%
\pgfsetbuttcap%
\pgfsetroundjoin%
\pgfsetlinewidth{0.803000pt}%
\definecolor{currentstroke}{rgb}{0.690196,0.690196,0.690196}%
\pgfsetstrokecolor{currentstroke}%
\pgfsetdash{}{0pt}%
\pgfpathmoveto{\pgfqpoint{5.307621in}{1.012908in}}%
\pgfpathlineto{\pgfqpoint{6.087890in}{1.729101in}}%
\pgfpathlineto{\pgfqpoint{6.101149in}{3.065608in}}%
\pgfusepath{stroke}%
\end{pgfscope}%
\begin{pgfscope}%
\pgfsetbuttcap%
\pgfsetroundjoin%
\pgfsetlinewidth{0.803000pt}%
\definecolor{currentstroke}{rgb}{0.690196,0.690196,0.690196}%
\pgfsetstrokecolor{currentstroke}%
\pgfsetdash{}{0pt}%
\pgfpathmoveto{\pgfqpoint{5.547447in}{0.934420in}}%
\pgfpathlineto{\pgfqpoint{6.315848in}{1.661706in}}%
\pgfpathlineto{\pgfqpoint{6.340977in}{3.005161in}}%
\pgfusepath{stroke}%
\end{pgfscope}%
\begin{pgfscope}%
\pgfsetbuttcap%
\pgfsetroundjoin%
\pgfsetlinewidth{0.803000pt}%
\definecolor{currentstroke}{rgb}{0.690196,0.690196,0.690196}%
\pgfsetstrokecolor{currentstroke}%
\pgfsetdash{}{0pt}%
\pgfpathmoveto{\pgfqpoint{5.791145in}{0.854665in}}%
\pgfpathlineto{\pgfqpoint{6.547215in}{1.593303in}}%
\pgfpathlineto{\pgfqpoint{6.584577in}{2.943764in}}%
\pgfusepath{stroke}%
\end{pgfscope}%
\begin{pgfscope}%
\pgfsetbuttcap%
\pgfsetroundjoin%
\pgfsetlinewidth{0.803000pt}%
\definecolor{currentstroke}{rgb}{0.690196,0.690196,0.690196}%
\pgfsetstrokecolor{currentstroke}%
\pgfsetdash{}{0pt}%
\pgfpathmoveto{\pgfqpoint{6.038811in}{0.773612in}}%
\pgfpathlineto{\pgfqpoint{6.782066in}{1.523869in}}%
\pgfpathlineto{\pgfqpoint{6.832038in}{2.881394in}}%
\pgfusepath{stroke}%
\end{pgfscope}%
\begin{pgfscope}%
\pgfsetrectcap%
\pgfsetroundjoin%
\pgfsetlinewidth{0.803000pt}%
\definecolor{currentstroke}{rgb}{0.000000,0.000000,0.000000}%
\pgfsetstrokecolor{currentstroke}%
\pgfsetdash{}{0pt}%
\pgfpathmoveto{\pgfqpoint{4.846138in}{1.172185in}}%
\pgfpathlineto{\pgfqpoint{4.825366in}{1.154206in}}%
\pgfusepath{stroke}%
\end{pgfscope}%
\begin{pgfscope}%
\definecolor{textcolor}{rgb}{0.000000,0.000000,0.000000}%
\pgfsetstrokecolor{textcolor}%
\pgfsetfillcolor{textcolor}%
\pgftext[x=4.750421in,y=0.940342in,,top]{\color{textcolor}\sffamily\fontsize{10.000000}{12.000000}\selectfont 0.0}%
\end{pgfscope}%
\begin{pgfscope}%
\pgfsetrectcap%
\pgfsetroundjoin%
\pgfsetlinewidth{0.803000pt}%
\definecolor{currentstroke}{rgb}{0.000000,0.000000,0.000000}%
\pgfsetstrokecolor{currentstroke}%
\pgfsetdash{}{0pt}%
\pgfpathmoveto{\pgfqpoint{5.078400in}{1.096238in}}%
\pgfpathlineto{\pgfqpoint{5.057901in}{1.077974in}}%
\pgfusepath{stroke}%
\end{pgfscope}%
\begin{pgfscope}%
\definecolor{textcolor}{rgb}{0.000000,0.000000,0.000000}%
\pgfsetstrokecolor{textcolor}%
\pgfsetfillcolor{textcolor}%
\pgftext[x=4.982733in,y=0.862298in,,top]{\color{textcolor}\sffamily\fontsize{10.000000}{12.000000}\selectfont 0.2}%
\end{pgfscope}%
\begin{pgfscope}%
\pgfsetrectcap%
\pgfsetroundjoin%
\pgfsetlinewidth{0.803000pt}%
\definecolor{currentstroke}{rgb}{0.000000,0.000000,0.000000}%
\pgfsetstrokecolor{currentstroke}%
\pgfsetdash{}{0pt}%
\pgfpathmoveto{\pgfqpoint{5.314351in}{1.019085in}}%
\pgfpathlineto{\pgfqpoint{5.294135in}{1.000529in}}%
\pgfusepath{stroke}%
\end{pgfscope}%
\begin{pgfscope}%
\definecolor{textcolor}{rgb}{0.000000,0.000000,0.000000}%
\pgfsetstrokecolor{textcolor}%
\pgfsetfillcolor{textcolor}%
\pgftext[x=5.218748in,y=0.783011in,,top]{\color{textcolor}\sffamily\fontsize{10.000000}{12.000000}\selectfont 0.4}%
\end{pgfscope}%
\begin{pgfscope}%
\pgfsetrectcap%
\pgfsetroundjoin%
\pgfsetlinewidth{0.803000pt}%
\definecolor{currentstroke}{rgb}{0.000000,0.000000,0.000000}%
\pgfsetstrokecolor{currentstroke}%
\pgfsetdash{}{0pt}%
\pgfpathmoveto{\pgfqpoint{5.554078in}{0.940697in}}%
\pgfpathlineto{\pgfqpoint{5.534158in}{0.921843in}}%
\pgfusepath{stroke}%
\end{pgfscope}%
\begin{pgfscope}%
\definecolor{textcolor}{rgb}{0.000000,0.000000,0.000000}%
\pgfsetstrokecolor{textcolor}%
\pgfsetfillcolor{textcolor}%
\pgftext[x=5.458553in,y=0.702450in,,top]{\color{textcolor}\sffamily\fontsize{10.000000}{12.000000}\selectfont 0.6}%
\end{pgfscope}%
\begin{pgfscope}%
\pgfsetrectcap%
\pgfsetroundjoin%
\pgfsetlinewidth{0.803000pt}%
\definecolor{currentstroke}{rgb}{0.000000,0.000000,0.000000}%
\pgfsetstrokecolor{currentstroke}%
\pgfsetdash{}{0pt}%
\pgfpathmoveto{\pgfqpoint{5.797674in}{0.861044in}}%
\pgfpathlineto{\pgfqpoint{5.778062in}{0.841884in}}%
\pgfusepath{stroke}%
\end{pgfscope}%
\begin{pgfscope}%
\definecolor{textcolor}{rgb}{0.000000,0.000000,0.000000}%
\pgfsetstrokecolor{textcolor}%
\pgfsetfillcolor{textcolor}%
\pgftext[x=5.702242in,y=0.620584in,,top]{\color{textcolor}\sffamily\fontsize{10.000000}{12.000000}\selectfont 0.8}%
\end{pgfscope}%
\begin{pgfscope}%
\pgfsetrectcap%
\pgfsetroundjoin%
\pgfsetlinewidth{0.803000pt}%
\definecolor{currentstroke}{rgb}{0.000000,0.000000,0.000000}%
\pgfsetstrokecolor{currentstroke}%
\pgfsetdash{}{0pt}%
\pgfpathmoveto{\pgfqpoint{6.045233in}{0.780095in}}%
\pgfpathlineto{\pgfqpoint{6.025942in}{0.760622in}}%
\pgfusepath{stroke}%
\end{pgfscope}%
\begin{pgfscope}%
\definecolor{textcolor}{rgb}{0.000000,0.000000,0.000000}%
\pgfsetstrokecolor{textcolor}%
\pgfsetfillcolor{textcolor}%
\pgftext[x=5.949910in,y=0.537382in,,top]{\color{textcolor}\sffamily\fontsize{10.000000}{12.000000}\selectfont 1.0}%
\end{pgfscope}%
\begin{pgfscope}%
\pgfsetrectcap%
\pgfsetroundjoin%
\pgfsetlinewidth{0.803000pt}%
\definecolor{currentstroke}{rgb}{0.000000,0.000000,0.000000}%
\pgfsetstrokecolor{currentstroke}%
\pgfsetdash{}{0pt}%
\pgfpathmoveto{\pgfqpoint{6.868571in}{1.498294in}}%
\pgfpathlineto{\pgfqpoint{6.130112in}{0.743733in}}%
\pgfusepath{stroke}%
\end{pgfscope}%
\begin{pgfscope}%
\pgfsetbuttcap%
\pgfsetroundjoin%
\pgfsetlinewidth{0.803000pt}%
\definecolor{currentstroke}{rgb}{0.690196,0.690196,0.690196}%
\pgfsetstrokecolor{currentstroke}%
\pgfsetdash{}{0pt}%
\pgfpathmoveto{\pgfqpoint{4.755320in}{2.626583in}}%
\pgfpathlineto{\pgfqpoint{4.810441in}{1.240768in}}%
\pgfpathlineto{\pgfqpoint{6.180651in}{0.795374in}}%
\pgfusepath{stroke}%
\end{pgfscope}%
\begin{pgfscope}%
\pgfsetbuttcap%
\pgfsetroundjoin%
\pgfsetlinewidth{0.803000pt}%
\definecolor{currentstroke}{rgb}{0.690196,0.690196,0.690196}%
\pgfsetstrokecolor{currentstroke}%
\pgfsetdash{}{0pt}%
\pgfpathmoveto{\pgfqpoint{4.912202in}{2.740956in}}%
\pgfpathlineto{\pgfqpoint{4.958645in}{1.367728in}}%
\pgfpathlineto{\pgfqpoint{6.316648in}{0.934337in}}%
\pgfusepath{stroke}%
\end{pgfscope}%
\begin{pgfscope}%
\pgfsetbuttcap%
\pgfsetroundjoin%
\pgfsetlinewidth{0.803000pt}%
\definecolor{currentstroke}{rgb}{0.690196,0.690196,0.690196}%
\pgfsetstrokecolor{currentstroke}%
\pgfsetdash{}{0pt}%
\pgfpathmoveto{\pgfqpoint{5.064762in}{2.852177in}}%
\pgfpathlineto{\pgfqpoint{5.102970in}{1.491363in}}%
\pgfpathlineto{\pgfqpoint{6.448924in}{1.069497in}}%
\pgfusepath{stroke}%
\end{pgfscope}%
\begin{pgfscope}%
\pgfsetbuttcap%
\pgfsetroundjoin%
\pgfsetlinewidth{0.803000pt}%
\definecolor{currentstroke}{rgb}{0.690196,0.690196,0.690196}%
\pgfsetstrokecolor{currentstroke}%
\pgfsetdash{}{0pt}%
\pgfpathmoveto{\pgfqpoint{5.213175in}{2.960375in}}%
\pgfpathlineto{\pgfqpoint{5.243565in}{1.611804in}}%
\pgfpathlineto{\pgfqpoint{6.577630in}{1.201009in}}%
\pgfusepath{stroke}%
\end{pgfscope}%
\begin{pgfscope}%
\pgfsetbuttcap%
\pgfsetroundjoin%
\pgfsetlinewidth{0.803000pt}%
\definecolor{currentstroke}{rgb}{0.690196,0.690196,0.690196}%
\pgfsetstrokecolor{currentstroke}%
\pgfsetdash{}{0pt}%
\pgfpathmoveto{\pgfqpoint{5.357608in}{3.065671in}}%
\pgfpathlineto{\pgfqpoint{5.380573in}{1.729172in}}%
\pgfpathlineto{\pgfqpoint{6.702907in}{1.329018in}}%
\pgfusepath{stroke}%
\end{pgfscope}%
\begin{pgfscope}%
\pgfsetbuttcap%
\pgfsetroundjoin%
\pgfsetlinewidth{0.803000pt}%
\definecolor{currentstroke}{rgb}{0.690196,0.690196,0.690196}%
\pgfsetstrokecolor{currentstroke}%
\pgfsetdash{}{0pt}%
\pgfpathmoveto{\pgfqpoint{5.498218in}{3.168181in}}%
\pgfpathlineto{\pgfqpoint{5.514130in}{1.843583in}}%
\pgfpathlineto{\pgfqpoint{6.824892in}{1.453662in}}%
\pgfusepath{stroke}%
\end{pgfscope}%
\begin{pgfscope}%
\pgfsetrectcap%
\pgfsetroundjoin%
\pgfsetlinewidth{0.803000pt}%
\definecolor{currentstroke}{rgb}{0.000000,0.000000,0.000000}%
\pgfsetstrokecolor{currentstroke}%
\pgfsetdash{}{0pt}%
\pgfpathmoveto{\pgfqpoint{6.169183in}{0.799101in}}%
\pgfpathlineto{\pgfqpoint{6.203611in}{0.787910in}}%
\pgfusepath{stroke}%
\end{pgfscope}%
\begin{pgfscope}%
\definecolor{textcolor}{rgb}{0.000000,0.000000,0.000000}%
\pgfsetstrokecolor{textcolor}%
\pgfsetfillcolor{textcolor}%
\pgftext[x=6.334514in,y=0.597696in,,top]{\color{textcolor}\sffamily\fontsize{10.000000}{12.000000}\selectfont 0.0}%
\end{pgfscope}%
\begin{pgfscope}%
\pgfsetrectcap%
\pgfsetroundjoin%
\pgfsetlinewidth{0.803000pt}%
\definecolor{currentstroke}{rgb}{0.000000,0.000000,0.000000}%
\pgfsetstrokecolor{currentstroke}%
\pgfsetdash{}{0pt}%
\pgfpathmoveto{\pgfqpoint{6.305290in}{0.937961in}}%
\pgfpathlineto{\pgfqpoint{6.339390in}{0.927079in}}%
\pgfusepath{stroke}%
\end{pgfscope}%
\begin{pgfscope}%
\definecolor{textcolor}{rgb}{0.000000,0.000000,0.000000}%
\pgfsetstrokecolor{textcolor}%
\pgfsetfillcolor{textcolor}%
\pgftext[x=6.468274in,y=0.739291in,,top]{\color{textcolor}\sffamily\fontsize{10.000000}{12.000000}\selectfont 0.2}%
\end{pgfscope}%
\begin{pgfscope}%
\pgfsetrectcap%
\pgfsetroundjoin%
\pgfsetlinewidth{0.803000pt}%
\definecolor{currentstroke}{rgb}{0.000000,0.000000,0.000000}%
\pgfsetstrokecolor{currentstroke}%
\pgfsetdash{}{0pt}%
\pgfpathmoveto{\pgfqpoint{6.437674in}{1.073023in}}%
\pgfpathlineto{\pgfqpoint{6.471451in}{1.062437in}}%
\pgfusepath{stroke}%
\end{pgfscope}%
\begin{pgfscope}%
\definecolor{textcolor}{rgb}{0.000000,0.000000,0.000000}%
\pgfsetstrokecolor{textcolor}%
\pgfsetfillcolor{textcolor}%
\pgftext[x=6.598377in,y=0.877016in,,top]{\color{textcolor}\sffamily\fontsize{10.000000}{12.000000}\selectfont 0.4}%
\end{pgfscope}%
\begin{pgfscope}%
\pgfsetrectcap%
\pgfsetroundjoin%
\pgfsetlinewidth{0.803000pt}%
\definecolor{currentstroke}{rgb}{0.000000,0.000000,0.000000}%
\pgfsetstrokecolor{currentstroke}%
\pgfsetdash{}{0pt}%
\pgfpathmoveto{\pgfqpoint{6.566485in}{1.204441in}}%
\pgfpathlineto{\pgfqpoint{6.599944in}{1.194138in}}%
\pgfusepath{stroke}%
\end{pgfscope}%
\begin{pgfscope}%
\definecolor{textcolor}{rgb}{0.000000,0.000000,0.000000}%
\pgfsetstrokecolor{textcolor}%
\pgfsetfillcolor{textcolor}%
\pgftext[x=6.724970in,y=1.011025in,,top]{\color{textcolor}\sffamily\fontsize{10.000000}{12.000000}\selectfont 0.6}%
\end{pgfscope}%
\begin{pgfscope}%
\pgfsetrectcap%
\pgfsetroundjoin%
\pgfsetlinewidth{0.803000pt}%
\definecolor{currentstroke}{rgb}{0.000000,0.000000,0.000000}%
\pgfsetstrokecolor{currentstroke}%
\pgfsetdash{}{0pt}%
\pgfpathmoveto{\pgfqpoint{6.691866in}{1.332359in}}%
\pgfpathlineto{\pgfqpoint{6.725012in}{1.322329in}}%
\pgfusepath{stroke}%
\end{pgfscope}%
\begin{pgfscope}%
\definecolor{textcolor}{rgb}{0.000000,0.000000,0.000000}%
\pgfsetstrokecolor{textcolor}%
\pgfsetfillcolor{textcolor}%
\pgftext[x=6.848194in,y=1.141467in,,top]{\color{textcolor}\sffamily\fontsize{10.000000}{12.000000}\selectfont 0.8}%
\end{pgfscope}%
\begin{pgfscope}%
\pgfsetrectcap%
\pgfsetroundjoin%
\pgfsetlinewidth{0.803000pt}%
\definecolor{currentstroke}{rgb}{0.000000,0.000000,0.000000}%
\pgfsetstrokecolor{currentstroke}%
\pgfsetdash{}{0pt}%
\pgfpathmoveto{\pgfqpoint{6.813953in}{1.456916in}}%
\pgfpathlineto{\pgfqpoint{6.846791in}{1.447148in}}%
\pgfusepath{stroke}%
\end{pgfscope}%
\begin{pgfscope}%
\definecolor{textcolor}{rgb}{0.000000,0.000000,0.000000}%
\pgfsetstrokecolor{textcolor}%
\pgfsetfillcolor{textcolor}%
\pgftext[x=6.968181in,y=1.268483in,,top]{\color{textcolor}\sffamily\fontsize{10.000000}{12.000000}\selectfont 1.0}%
\end{pgfscope}%
\begin{pgfscope}%
\pgfsetrectcap%
\pgfsetroundjoin%
\pgfsetlinewidth{0.803000pt}%
\definecolor{currentstroke}{rgb}{0.000000,0.000000,0.000000}%
\pgfsetstrokecolor{currentstroke}%
\pgfsetdash{}{0pt}%
\pgfpathmoveto{\pgfqpoint{6.868571in}{1.498294in}}%
\pgfpathlineto{\pgfqpoint{6.923237in}{2.858408in}}%
\pgfusepath{stroke}%
\end{pgfscope}%
\begin{pgfscope}%
\pgfsetbuttcap%
\pgfsetroundjoin%
\pgfsetlinewidth{0.803000pt}%
\definecolor{currentstroke}{rgb}{0.690196,0.690196,0.690196}%
\pgfsetstrokecolor{currentstroke}%
\pgfsetdash{}{0pt}%
\pgfpathmoveto{\pgfqpoint{6.871887in}{1.580785in}}%
\pgfpathlineto{\pgfqpoint{5.561174in}{1.964823in}}%
\pgfpathlineto{\pgfqpoint{4.751875in}{1.277822in}}%
\pgfusepath{stroke}%
\end{pgfscope}%
\begin{pgfscope}%
\pgfsetbuttcap%
\pgfsetroundjoin%
\pgfsetlinewidth{0.803000pt}%
\definecolor{currentstroke}{rgb}{0.690196,0.690196,0.690196}%
\pgfsetstrokecolor{currentstroke}%
\pgfsetdash{}{0pt}%
\pgfpathmoveto{\pgfqpoint{6.881090in}{1.809771in}}%
\pgfpathlineto{\pgfqpoint{5.558910in}{2.187449in}}%
\pgfpathlineto{\pgfqpoint{4.742055in}{1.511629in}}%
\pgfusepath{stroke}%
\end{pgfscope}%
\begin{pgfscope}%
\pgfsetbuttcap%
\pgfsetroundjoin%
\pgfsetlinewidth{0.803000pt}%
\definecolor{currentstroke}{rgb}{0.690196,0.690196,0.690196}%
\pgfsetstrokecolor{currentstroke}%
\pgfsetdash{}{0pt}%
\pgfpathmoveto{\pgfqpoint{6.890457in}{2.042834in}}%
\pgfpathlineto{\pgfqpoint{5.556607in}{2.413873in}}%
\pgfpathlineto{\pgfqpoint{4.732054in}{1.749737in}}%
\pgfusepath{stroke}%
\end{pgfscope}%
\begin{pgfscope}%
\pgfsetbuttcap%
\pgfsetroundjoin%
\pgfsetlinewidth{0.803000pt}%
\definecolor{currentstroke}{rgb}{0.690196,0.690196,0.690196}%
\pgfsetstrokecolor{currentstroke}%
\pgfsetdash{}{0pt}%
\pgfpathmoveto{\pgfqpoint{6.899993in}{2.280084in}}%
\pgfpathlineto{\pgfqpoint{5.554264in}{2.644191in}}%
\pgfpathlineto{\pgfqpoint{4.721868in}{1.992268in}}%
\pgfusepath{stroke}%
\end{pgfscope}%
\begin{pgfscope}%
\pgfsetbuttcap%
\pgfsetroundjoin%
\pgfsetlinewidth{0.803000pt}%
\definecolor{currentstroke}{rgb}{0.690196,0.690196,0.690196}%
\pgfsetstrokecolor{currentstroke}%
\pgfsetdash{}{0pt}%
\pgfpathmoveto{\pgfqpoint{6.909702in}{2.521636in}}%
\pgfpathlineto{\pgfqpoint{5.551881in}{2.878505in}}%
\pgfpathlineto{\pgfqpoint{4.711490in}{2.239346in}}%
\pgfusepath{stroke}%
\end{pgfscope}%
\begin{pgfscope}%
\pgfsetbuttcap%
\pgfsetroundjoin%
\pgfsetlinewidth{0.803000pt}%
\definecolor{currentstroke}{rgb}{0.690196,0.690196,0.690196}%
\pgfsetstrokecolor{currentstroke}%
\pgfsetdash{}{0pt}%
\pgfpathmoveto{\pgfqpoint{6.919588in}{2.767607in}}%
\pgfpathlineto{\pgfqpoint{5.549456in}{3.116919in}}%
\pgfpathlineto{\pgfqpoint{4.700916in}{2.491100in}}%
\pgfusepath{stroke}%
\end{pgfscope}%
\begin{pgfscope}%
\pgfsetrectcap%
\pgfsetroundjoin%
\pgfsetlinewidth{0.803000pt}%
\definecolor{currentstroke}{rgb}{0.000000,0.000000,0.000000}%
\pgfsetstrokecolor{currentstroke}%
\pgfsetdash{}{0pt}%
\pgfpathmoveto{\pgfqpoint{6.860950in}{1.583990in}}%
\pgfpathlineto{\pgfqpoint{6.893784in}{1.574369in}}%
\pgfusepath{stroke}%
\end{pgfscope}%
\begin{pgfscope}%
\definecolor{textcolor}{rgb}{0.000000,0.000000,0.000000}%
\pgfsetstrokecolor{textcolor}%
\pgfsetfillcolor{textcolor}%
\pgftext[x=7.094481in,y=1.612595in,,top]{\color{textcolor}\sffamily\fontsize{10.000000}{12.000000}\selectfont 0.0}%
\end{pgfscope}%
\begin{pgfscope}%
\pgfsetrectcap%
\pgfsetroundjoin%
\pgfsetlinewidth{0.803000pt}%
\definecolor{currentstroke}{rgb}{0.000000,0.000000,0.000000}%
\pgfsetstrokecolor{currentstroke}%
\pgfsetdash{}{0pt}%
\pgfpathmoveto{\pgfqpoint{6.870053in}{1.812923in}}%
\pgfpathlineto{\pgfqpoint{6.903187in}{1.803459in}}%
\pgfusepath{stroke}%
\end{pgfscope}%
\begin{pgfscope}%
\definecolor{textcolor}{rgb}{0.000000,0.000000,0.000000}%
\pgfsetstrokecolor{textcolor}%
\pgfsetfillcolor{textcolor}%
\pgftext[x=7.105610in,y=1.841065in,,top]{\color{textcolor}\sffamily\fontsize{10.000000}{12.000000}\selectfont 0.2}%
\end{pgfscope}%
\begin{pgfscope}%
\pgfsetrectcap%
\pgfsetroundjoin%
\pgfsetlinewidth{0.803000pt}%
\definecolor{currentstroke}{rgb}{0.000000,0.000000,0.000000}%
\pgfsetstrokecolor{currentstroke}%
\pgfsetdash{}{0pt}%
\pgfpathmoveto{\pgfqpoint{6.879319in}{2.045932in}}%
\pgfpathlineto{\pgfqpoint{6.912758in}{2.036631in}}%
\pgfusepath{stroke}%
\end{pgfscope}%
\begin{pgfscope}%
\definecolor{textcolor}{rgb}{0.000000,0.000000,0.000000}%
\pgfsetstrokecolor{textcolor}%
\pgfsetfillcolor{textcolor}%
\pgftext[x=7.116936in,y=2.073588in,,top]{\color{textcolor}\sffamily\fontsize{10.000000}{12.000000}\selectfont 0.4}%
\end{pgfscope}%
\begin{pgfscope}%
\pgfsetrectcap%
\pgfsetroundjoin%
\pgfsetlinewidth{0.803000pt}%
\definecolor{currentstroke}{rgb}{0.000000,0.000000,0.000000}%
\pgfsetstrokecolor{currentstroke}%
\pgfsetdash{}{0pt}%
\pgfpathmoveto{\pgfqpoint{6.888751in}{2.283126in}}%
\pgfpathlineto{\pgfqpoint{6.922500in}{2.273995in}}%
\pgfusepath{stroke}%
\end{pgfscope}%
\begin{pgfscope}%
\definecolor{textcolor}{rgb}{0.000000,0.000000,0.000000}%
\pgfsetstrokecolor{textcolor}%
\pgfsetfillcolor{textcolor}%
\pgftext[x=7.128466in,y=2.310275in,,top]{\color{textcolor}\sffamily\fontsize{10.000000}{12.000000}\selectfont 0.6}%
\end{pgfscope}%
\begin{pgfscope}%
\pgfsetrectcap%
\pgfsetroundjoin%
\pgfsetlinewidth{0.803000pt}%
\definecolor{currentstroke}{rgb}{0.000000,0.000000,0.000000}%
\pgfsetstrokecolor{currentstroke}%
\pgfsetdash{}{0pt}%
\pgfpathmoveto{\pgfqpoint{6.898354in}{2.524619in}}%
\pgfpathlineto{\pgfqpoint{6.932420in}{2.515665in}}%
\pgfusepath{stroke}%
\end{pgfscope}%
\begin{pgfscope}%
\definecolor{textcolor}{rgb}{0.000000,0.000000,0.000000}%
\pgfsetstrokecolor{textcolor}%
\pgfsetfillcolor{textcolor}%
\pgftext[x=7.140203in,y=2.551237in,,top]{\color{textcolor}\sffamily\fontsize{10.000000}{12.000000}\selectfont 0.8}%
\end{pgfscope}%
\begin{pgfscope}%
\pgfsetrectcap%
\pgfsetroundjoin%
\pgfsetlinewidth{0.803000pt}%
\definecolor{currentstroke}{rgb}{0.000000,0.000000,0.000000}%
\pgfsetstrokecolor{currentstroke}%
\pgfsetdash{}{0pt}%
\pgfpathmoveto{\pgfqpoint{6.908133in}{2.770528in}}%
\pgfpathlineto{\pgfqpoint{6.942521in}{2.761761in}}%
\pgfusepath{stroke}%
\end{pgfscope}%
\begin{pgfscope}%
\definecolor{textcolor}{rgb}{0.000000,0.000000,0.000000}%
\pgfsetstrokecolor{textcolor}%
\pgfsetfillcolor{textcolor}%
\pgftext[x=7.152155in,y=2.796592in,,top]{\color{textcolor}\sffamily\fontsize{10.000000}{12.000000}\selectfont 1.0}%
\end{pgfscope}%
\begin{pgfscope}%
\pgfpathrectangle{\pgfqpoint{4.381818in}{0.570909in}}{\pgfqpoint{2.818182in}{2.818182in}}%
\pgfusepath{clip}%
\pgfsetbuttcap%
\pgfsetroundjoin%
\definecolor{currentfill}{rgb}{0.839216,0.152941,0.156863}%
\pgfsetfillcolor{currentfill}%
\pgfsetlinewidth{1.003750pt}%
\definecolor{currentstroke}{rgb}{0.839216,0.152941,0.156863}%
\pgfsetstrokecolor{currentstroke}%
\pgfsetdash{}{0pt}%
\pgfpathmoveto{\pgfqpoint{5.364595in}{1.848913in}}%
\pgfpathcurveto{\pgfqpoint{5.372832in}{1.848913in}}{\pgfqpoint{5.380732in}{1.852185in}}{\pgfqpoint{5.386556in}{1.858009in}}%
\pgfpathcurveto{\pgfqpoint{5.392380in}{1.863833in}}{\pgfqpoint{5.395652in}{1.871733in}}{\pgfqpoint{5.395652in}{1.879969in}}%
\pgfpathcurveto{\pgfqpoint{5.395652in}{1.888205in}}{\pgfqpoint{5.392380in}{1.896105in}}{\pgfqpoint{5.386556in}{1.901929in}}%
\pgfpathcurveto{\pgfqpoint{5.380732in}{1.907753in}}{\pgfqpoint{5.372832in}{1.911026in}}{\pgfqpoint{5.364595in}{1.911026in}}%
\pgfpathcurveto{\pgfqpoint{5.356359in}{1.911026in}}{\pgfqpoint{5.348459in}{1.907753in}}{\pgfqpoint{5.342635in}{1.901929in}}%
\pgfpathcurveto{\pgfqpoint{5.336811in}{1.896105in}}{\pgfqpoint{5.333539in}{1.888205in}}{\pgfqpoint{5.333539in}{1.879969in}}%
\pgfpathcurveto{\pgfqpoint{5.333539in}{1.871733in}}{\pgfqpoint{5.336811in}{1.863833in}}{\pgfqpoint{5.342635in}{1.858009in}}%
\pgfpathcurveto{\pgfqpoint{5.348459in}{1.852185in}}{\pgfqpoint{5.356359in}{1.848913in}}{\pgfqpoint{5.364595in}{1.848913in}}%
\pgfpathclose%
\pgfusepath{stroke,fill}%
\end{pgfscope}%
\begin{pgfscope}%
\definecolor{textcolor}{rgb}{0.000000,0.000000,0.000000}%
\pgfsetstrokecolor{textcolor}%
\pgfsetfillcolor{textcolor}%
\pgftext[x=5.092479in,y=1.957025in,left,base]{\color{textcolor}\sffamily\fontsize{10.000000}{12.000000}\selectfont 0.04167}%
\end{pgfscope}%
\begin{pgfscope}%
\definecolor{textcolor}{rgb}{0.000000,0.000000,0.000000}%
\pgfsetstrokecolor{textcolor}%
\pgfsetfillcolor{textcolor}%
\pgftext[x=5.254747in,y=1.382750in,left,base]{\color{textcolor}\sffamily\fontsize{10.000000}{12.000000}\selectfont 0.04167}%
\end{pgfscope}%
\begin{pgfscope}%
\definecolor{textcolor}{rgb}{0.000000,0.000000,0.000000}%
\pgfsetstrokecolor{textcolor}%
\pgfsetfillcolor{textcolor}%
\pgftext[x=5.717805in,y=1.404405in,left,base]{\color{textcolor}\sffamily\fontsize{10.000000}{12.000000}\selectfont 0.04167}%
\end{pgfscope}%
\begin{pgfscope}%
\definecolor{textcolor}{rgb}{0.000000,0.000000,0.000000}%
\pgfsetstrokecolor{textcolor}%
\pgfsetfillcolor{textcolor}%
\pgftext[x=5.364595in,y=1.879969in,left,base]{\color{textcolor}\sffamily\fontsize{10.000000}{12.000000}\selectfont 0.04167}%
\end{pgfscope}%
\begin{pgfscope}%
\pgfpathrectangle{\pgfqpoint{4.381818in}{0.570909in}}{\pgfqpoint{2.818182in}{2.818182in}}%
\pgfusepath{clip}%
\pgfsetbuttcap%
\pgfsetroundjoin%
\definecolor{currentfill}{rgb}{1.000000,0.498039,0.054902}%
\pgfsetfillcolor{currentfill}%
\pgfsetlinewidth{1.003750pt}%
\definecolor{currentstroke}{rgb}{1.000000,0.498039,0.054902}%
\pgfsetstrokecolor{currentstroke}%
\pgfsetdash{}{0pt}%
\pgfpathmoveto{\pgfqpoint{5.254747in}{1.351694in}}%
\pgfpathcurveto{\pgfqpoint{5.262983in}{1.351694in}}{\pgfqpoint{5.270883in}{1.354966in}}{\pgfqpoint{5.276707in}{1.360790in}}%
\pgfpathcurveto{\pgfqpoint{5.282531in}{1.366614in}}{\pgfqpoint{5.285804in}{1.374514in}}{\pgfqpoint{5.285804in}{1.382750in}}%
\pgfpathcurveto{\pgfqpoint{5.285804in}{1.390986in}}{\pgfqpoint{5.282531in}{1.398887in}}{\pgfqpoint{5.276707in}{1.404710in}}%
\pgfpathcurveto{\pgfqpoint{5.270883in}{1.410534in}}{\pgfqpoint{5.262983in}{1.413807in}}{\pgfqpoint{5.254747in}{1.413807in}}%
\pgfpathcurveto{\pgfqpoint{5.246511in}{1.413807in}}{\pgfqpoint{5.238611in}{1.410534in}}{\pgfqpoint{5.232787in}{1.404710in}}%
\pgfpathcurveto{\pgfqpoint{5.226963in}{1.398887in}}{\pgfqpoint{5.223691in}{1.390986in}}{\pgfqpoint{5.223691in}{1.382750in}}%
\pgfpathcurveto{\pgfqpoint{5.223691in}{1.374514in}}{\pgfqpoint{5.226963in}{1.366614in}}{\pgfqpoint{5.232787in}{1.360790in}}%
\pgfpathcurveto{\pgfqpoint{5.238611in}{1.354966in}}{\pgfqpoint{5.246511in}{1.351694in}}{\pgfqpoint{5.254747in}{1.351694in}}%
\pgfpathclose%
\pgfusepath{stroke,fill}%
\end{pgfscope}%
\begin{pgfscope}%
\pgfpathrectangle{\pgfqpoint{4.381818in}{0.570909in}}{\pgfqpoint{2.818182in}{2.818182in}}%
\pgfusepath{clip}%
\pgfsetbuttcap%
\pgfsetroundjoin%
\definecolor{currentfill}{rgb}{0.172549,0.627451,0.172549}%
\pgfsetfillcolor{currentfill}%
\pgfsetlinewidth{1.003750pt}%
\definecolor{currentstroke}{rgb}{0.172549,0.627451,0.172549}%
\pgfsetstrokecolor{currentstroke}%
\pgfsetdash{}{0pt}%
\pgfpathmoveto{\pgfqpoint{5.717805in}{1.373349in}}%
\pgfpathcurveto{\pgfqpoint{5.726041in}{1.373349in}}{\pgfqpoint{5.733941in}{1.376621in}}{\pgfqpoint{5.739765in}{1.382445in}}%
\pgfpathcurveto{\pgfqpoint{5.745589in}{1.388269in}}{\pgfqpoint{5.748861in}{1.396169in}}{\pgfqpoint{5.748861in}{1.404405in}}%
\pgfpathcurveto{\pgfqpoint{5.748861in}{1.412641in}}{\pgfqpoint{5.745589in}{1.420541in}}{\pgfqpoint{5.739765in}{1.426365in}}%
\pgfpathcurveto{\pgfqpoint{5.733941in}{1.432189in}}{\pgfqpoint{5.726041in}{1.435462in}}{\pgfqpoint{5.717805in}{1.435462in}}%
\pgfpathcurveto{\pgfqpoint{5.709568in}{1.435462in}}{\pgfqpoint{5.701668in}{1.432189in}}{\pgfqpoint{5.695844in}{1.426365in}}%
\pgfpathcurveto{\pgfqpoint{5.690021in}{1.420541in}}{\pgfqpoint{5.686748in}{1.412641in}}{\pgfqpoint{5.686748in}{1.404405in}}%
\pgfpathcurveto{\pgfqpoint{5.686748in}{1.396169in}}{\pgfqpoint{5.690021in}{1.388269in}}{\pgfqpoint{5.695844in}{1.382445in}}%
\pgfpathcurveto{\pgfqpoint{5.701668in}{1.376621in}}{\pgfqpoint{5.709568in}{1.373349in}}{\pgfqpoint{5.717805in}{1.373349in}}%
\pgfpathclose%
\pgfusepath{stroke,fill}%
\end{pgfscope}%
\begin{pgfscope}%
\pgfpathrectangle{\pgfqpoint{4.381818in}{0.570909in}}{\pgfqpoint{2.818182in}{2.818182in}}%
\pgfusepath{clip}%
\pgfsetbuttcap%
\pgfsetroundjoin%
\definecolor{currentfill}{rgb}{0.121569,0.466667,0.705882}%
\pgfsetfillcolor{currentfill}%
\pgfsetlinewidth{1.003750pt}%
\definecolor{currentstroke}{rgb}{0.121569,0.466667,0.705882}%
\pgfsetstrokecolor{currentstroke}%
\pgfsetdash{}{0pt}%
\pgfpathmoveto{\pgfqpoint{5.092479in}{1.925968in}}%
\pgfpathcurveto{\pgfqpoint{5.100715in}{1.925968in}}{\pgfqpoint{5.108615in}{1.929240in}}{\pgfqpoint{5.114439in}{1.935064in}}%
\pgfpathcurveto{\pgfqpoint{5.120263in}{1.940888in}}{\pgfqpoint{5.123535in}{1.948788in}}{\pgfqpoint{5.123535in}{1.957025in}}%
\pgfpathcurveto{\pgfqpoint{5.123535in}{1.965261in}}{\pgfqpoint{5.120263in}{1.973161in}}{\pgfqpoint{5.114439in}{1.978985in}}%
\pgfpathcurveto{\pgfqpoint{5.108615in}{1.984809in}}{\pgfqpoint{5.100715in}{1.988081in}}{\pgfqpoint{5.092479in}{1.988081in}}%
\pgfpathcurveto{\pgfqpoint{5.084242in}{1.988081in}}{\pgfqpoint{5.076342in}{1.984809in}}{\pgfqpoint{5.070518in}{1.978985in}}%
\pgfpathcurveto{\pgfqpoint{5.064695in}{1.973161in}}{\pgfqpoint{5.061422in}{1.965261in}}{\pgfqpoint{5.061422in}{1.957025in}}%
\pgfpathcurveto{\pgfqpoint{5.061422in}{1.948788in}}{\pgfqpoint{5.064695in}{1.940888in}}{\pgfqpoint{5.070518in}{1.935064in}}%
\pgfpathcurveto{\pgfqpoint{5.076342in}{1.929240in}}{\pgfqpoint{5.084242in}{1.925968in}}{\pgfqpoint{5.092479in}{1.925968in}}%
\pgfpathclose%
\pgfusepath{stroke,fill}%
\end{pgfscope}%
\begin{pgfscope}%
\pgfpathrectangle{\pgfqpoint{4.381818in}{0.570909in}}{\pgfqpoint{2.818182in}{2.818182in}}%
\pgfusepath{clip}%
\pgfsetbuttcap%
\pgfsetroundjoin%
\definecolor{currentfill}{rgb}{0.100000,0.100000,0.100000}%
\pgfsetfillcolor{currentfill}%
\pgfsetfillopacity{0.100000}%
\pgfsetlinewidth{0.301125pt}%
\definecolor{currentstroke}{rgb}{0.000000,0.000000,0.000000}%
\pgfsetstrokecolor{currentstroke}%
\pgfsetdash{}{0pt}%
\pgfpathmoveto{\pgfqpoint{4.890922in}{1.297693in}}%
\pgfpathlineto{\pgfqpoint{5.593557in}{1.900366in}}%
\pgfpathlineto{\pgfqpoint{6.090559in}{0.911119in}}%
\pgfpathclose%
\pgfusepath{stroke,fill}%
\end{pgfscope}%
\begin{pgfscope}%
\pgfpathrectangle{\pgfqpoint{4.381818in}{0.570909in}}{\pgfqpoint{2.818182in}{2.818182in}}%
\pgfusepath{clip}%
\pgfsetbuttcap%
\pgfsetroundjoin%
\definecolor{currentfill}{rgb}{0.100000,0.100000,0.100000}%
\pgfsetfillcolor{currentfill}%
\pgfsetfillopacity{0.100000}%
\pgfsetlinewidth{0.301125pt}%
\definecolor{currentstroke}{rgb}{0.000000,0.000000,0.000000}%
\pgfsetstrokecolor{currentstroke}%
\pgfsetdash{}{0pt}%
\pgfpathmoveto{\pgfqpoint{4.890922in}{1.297693in}}%
\pgfpathlineto{\pgfqpoint{5.593557in}{1.900366in}}%
\pgfpathlineto{\pgfqpoint{4.846638in}{2.509261in}}%
\pgfpathclose%
\pgfusepath{stroke,fill}%
\end{pgfscope}%
\begin{pgfscope}%
\pgfpathrectangle{\pgfqpoint{4.381818in}{0.570909in}}{\pgfqpoint{2.818182in}{2.818182in}}%
\pgfusepath{clip}%
\pgfsetbuttcap%
\pgfsetroundjoin%
\definecolor{currentfill}{rgb}{0.100000,0.100000,0.100000}%
\pgfsetfillcolor{currentfill}%
\pgfsetfillopacity{0.100000}%
\pgfsetlinewidth{0.301125pt}%
\definecolor{currentstroke}{rgb}{0.000000,0.000000,0.000000}%
\pgfsetstrokecolor{currentstroke}%
\pgfsetdash{}{0pt}%
\pgfpathmoveto{\pgfqpoint{4.846638in}{2.509261in}}%
\pgfpathlineto{\pgfqpoint{5.593557in}{1.900366in}}%
\pgfpathlineto{\pgfqpoint{6.090559in}{0.911119in}}%
\pgfpathclose%
\pgfusepath{stroke,fill}%
\end{pgfscope}%
\begin{pgfscope}%
\pgfpathrectangle{\pgfqpoint{4.381818in}{0.570909in}}{\pgfqpoint{2.818182in}{2.818182in}}%
\pgfusepath{clip}%
\pgfsetbuttcap%
\pgfsetroundjoin%
\definecolor{currentfill}{rgb}{0.100000,0.100000,0.100000}%
\pgfsetfillcolor{currentfill}%
\pgfsetfillopacity{0.100000}%
\pgfsetlinewidth{0.301125pt}%
\definecolor{currentstroke}{rgb}{0.000000,0.000000,0.000000}%
\pgfsetstrokecolor{currentstroke}%
\pgfsetdash{}{0pt}%
\pgfpathmoveto{\pgfqpoint{4.890922in}{1.297693in}}%
\pgfpathlineto{\pgfqpoint{6.090559in}{0.911119in}}%
\pgfpathlineto{\pgfqpoint{4.846638in}{2.509261in}}%
\pgfpathclose%
\pgfusepath{stroke,fill}%
\end{pgfscope}%
\end{pgfpicture}%
\makeatother%
\endgroup%
}
\caption{Figures of the three-dimensional cubatures of $ D = 2 $ with extra constraints $ \pbr{ 0, 0, 3 }, \pbr{ 0, 3, 0 }, \pbr{ 3, 0, 0 } $}
\label{Fig:M3D2333}
\end{figure}

In this case, the smallest $N$ is $4$.

\subsubsection{The case $ D = 3 $}

We optimize for $ D = 3 $, $ N = 5 $ with $ \text{FPG} \rbr{ 10^6, 10^{-2} } $ and get the residual $ F ( \mathbf{x}_{\cdot}^{\rbr{K}}, u_{\cdot}^{\rbr{K}} ) = \text{\input{Text4.txt}} $. This implies $ N \le 5 $ is impossible.

We turn to try $ N = 6 $. In this case, since $ \abs{\mathcal{I}_3} = 20 $, we may group with $ C = 2 $ and thus try $ \sbr{ 1, 1, 1, 2, 2, 2 } $. Using $\mathcal{I}_3$ for constraints, we have $ L = 2 N + C = 20 $. We use $ \text{FPG} \rbr{ 10^4, 10^{-2} } + \text{N} \rbr{30} $ to optimize and get the results in Figure \ref{Fig:M3D3}.

\begin{figure}[htbp]
\centering
\scalebox{0.75}{%% Creator: Matplotlib, PGF backend
%%
%% To include the figure in your LaTeX document, write
%%   \input{<filename>.pgf}
%%
%% Make sure the required packages are loaded in your preamble
%%   \usepackage{pgf}
%%
%% Figures using additional raster images can only be included by \input if
%% they are in the same directory as the main LaTeX file. For loading figures
%% from other directories you can use the `import` package
%%   \usepackage{import}
%% and then include the figures with
%%   \import{<path to file>}{<filename>.pgf}
%%
%% Matplotlib used the following preamble
%%   \usepackage{fontspec}
%%   \setmainfont{DejaVuSerif.ttf}[Path=/home/lzh/anaconda3/envs/numana/lib/python3.7/site-packages/matplotlib/mpl-data/fonts/ttf/]
%%   \setsansfont{DejaVuSans.ttf}[Path=/home/lzh/anaconda3/envs/numana/lib/python3.7/site-packages/matplotlib/mpl-data/fonts/ttf/]
%%   \setmonofont{DejaVuSansMono.ttf}[Path=/home/lzh/anaconda3/envs/numana/lib/python3.7/site-packages/matplotlib/mpl-data/fonts/ttf/]
%%
\begingroup%
\makeatletter%
\begin{pgfpicture}%
\pgfpathrectangle{\pgfpointorigin}{\pgfqpoint{6.000000in}{6.000000in}}%
\pgfusepath{use as bounding box, clip}%
\begin{pgfscope}%
\pgfsetbuttcap%
\pgfsetmiterjoin%
\definecolor{currentfill}{rgb}{1.000000,1.000000,1.000000}%
\pgfsetfillcolor{currentfill}%
\pgfsetlinewidth{0.000000pt}%
\definecolor{currentstroke}{rgb}{1.000000,1.000000,1.000000}%
\pgfsetstrokecolor{currentstroke}%
\pgfsetdash{}{0pt}%
\pgfpathmoveto{\pgfqpoint{0.000000in}{0.000000in}}%
\pgfpathlineto{\pgfqpoint{6.000000in}{0.000000in}}%
\pgfpathlineto{\pgfqpoint{6.000000in}{6.000000in}}%
\pgfpathlineto{\pgfqpoint{0.000000in}{6.000000in}}%
\pgfpathclose%
\pgfusepath{fill}%
\end{pgfscope}%
\begin{pgfscope}%
\pgfsetbuttcap%
\pgfsetmiterjoin%
\definecolor{currentfill}{rgb}{1.000000,1.000000,1.000000}%
\pgfsetfillcolor{currentfill}%
\pgfsetlinewidth{0.000000pt}%
\definecolor{currentstroke}{rgb}{0.000000,0.000000,0.000000}%
\pgfsetstrokecolor{currentstroke}%
\pgfsetstrokeopacity{0.000000}%
\pgfsetdash{}{0pt}%
\pgfpathmoveto{\pgfqpoint{0.600000in}{0.600000in}}%
\pgfpathlineto{\pgfqpoint{4.500000in}{0.600000in}}%
\pgfpathlineto{\pgfqpoint{4.500000in}{4.500000in}}%
\pgfpathlineto{\pgfqpoint{0.600000in}{4.500000in}}%
\pgfpathclose%
\pgfusepath{fill}%
\end{pgfscope}%
\begin{pgfscope}%
\pgfpathrectangle{\pgfqpoint{0.600000in}{0.600000in}}{\pgfqpoint{3.900000in}{3.900000in}}%
\pgfusepath{clip}%
\pgfsetbuttcap%
\pgfsetroundjoin%
\definecolor{currentfill}{rgb}{0.121569,0.466667,0.705882}%
\pgfsetfillcolor{currentfill}%
\pgfsetlinewidth{1.003750pt}%
\definecolor{currentstroke}{rgb}{0.121569,0.466667,0.705882}%
\pgfsetstrokecolor{currentstroke}%
\pgfsetdash{}{0pt}%
\pgfpathmoveto{\pgfqpoint{2.193249in}{2.779173in}}%
\pgfpathcurveto{\pgfqpoint{2.204299in}{2.779173in}}{\pgfqpoint{2.214898in}{2.783563in}}{\pgfqpoint{2.222712in}{2.791377in}}%
\pgfpathcurveto{\pgfqpoint{2.230526in}{2.799190in}}{\pgfqpoint{2.234916in}{2.809789in}}{\pgfqpoint{2.234916in}{2.820839in}}%
\pgfpathcurveto{\pgfqpoint{2.234916in}{2.831890in}}{\pgfqpoint{2.230526in}{2.842489in}}{\pgfqpoint{2.222712in}{2.850302in}}%
\pgfpathcurveto{\pgfqpoint{2.214898in}{2.858116in}}{\pgfqpoint{2.204299in}{2.862506in}}{\pgfqpoint{2.193249in}{2.862506in}}%
\pgfpathcurveto{\pgfqpoint{2.182199in}{2.862506in}}{\pgfqpoint{2.171600in}{2.858116in}}{\pgfqpoint{2.163786in}{2.850302in}}%
\pgfpathcurveto{\pgfqpoint{2.155973in}{2.842489in}}{\pgfqpoint{2.151582in}{2.831890in}}{\pgfqpoint{2.151582in}{2.820839in}}%
\pgfpathcurveto{\pgfqpoint{2.151582in}{2.809789in}}{\pgfqpoint{2.155973in}{2.799190in}}{\pgfqpoint{2.163786in}{2.791377in}}%
\pgfpathcurveto{\pgfqpoint{2.171600in}{2.783563in}}{\pgfqpoint{2.182199in}{2.779173in}}{\pgfqpoint{2.193249in}{2.779173in}}%
\pgfpathclose%
\pgfusepath{stroke,fill}%
\end{pgfscope}%
\begin{pgfscope}%
\pgfpathrectangle{\pgfqpoint{0.600000in}{0.600000in}}{\pgfqpoint{3.900000in}{3.900000in}}%
\pgfusepath{clip}%
\pgfsetbuttcap%
\pgfsetroundjoin%
\definecolor{currentfill}{rgb}{0.121569,0.466667,0.705882}%
\pgfsetfillcolor{currentfill}%
\pgfsetlinewidth{1.003750pt}%
\definecolor{currentstroke}{rgb}{0.121569,0.466667,0.705882}%
\pgfsetstrokecolor{currentstroke}%
\pgfsetdash{}{0pt}%
\pgfpathmoveto{\pgfqpoint{2.743981in}{3.225500in}}%
\pgfpathcurveto{\pgfqpoint{2.755031in}{3.225500in}}{\pgfqpoint{2.765630in}{3.229891in}}{\pgfqpoint{2.773443in}{3.237704in}}%
\pgfpathcurveto{\pgfqpoint{2.781257in}{3.245518in}}{\pgfqpoint{2.785647in}{3.256117in}}{\pgfqpoint{2.785647in}{3.267167in}}%
\pgfpathcurveto{\pgfqpoint{2.785647in}{3.278217in}}{\pgfqpoint{2.781257in}{3.288816in}}{\pgfqpoint{2.773443in}{3.296630in}}%
\pgfpathcurveto{\pgfqpoint{2.765630in}{3.304443in}}{\pgfqpoint{2.755031in}{3.308834in}}{\pgfqpoint{2.743981in}{3.308834in}}%
\pgfpathcurveto{\pgfqpoint{2.732931in}{3.308834in}}{\pgfqpoint{2.722331in}{3.304443in}}{\pgfqpoint{2.714518in}{3.296630in}}%
\pgfpathcurveto{\pgfqpoint{2.706704in}{3.288816in}}{\pgfqpoint{2.702314in}{3.278217in}}{\pgfqpoint{2.702314in}{3.267167in}}%
\pgfpathcurveto{\pgfqpoint{2.702314in}{3.256117in}}{\pgfqpoint{2.706704in}{3.245518in}}{\pgfqpoint{2.714518in}{3.237704in}}%
\pgfpathcurveto{\pgfqpoint{2.722331in}{3.229891in}}{\pgfqpoint{2.732931in}{3.225500in}}{\pgfqpoint{2.743981in}{3.225500in}}%
\pgfpathclose%
\pgfusepath{stroke,fill}%
\end{pgfscope}%
\begin{pgfscope}%
\pgfpathrectangle{\pgfqpoint{0.600000in}{0.600000in}}{\pgfqpoint{3.900000in}{3.900000in}}%
\pgfusepath{clip}%
\pgfsetbuttcap%
\pgfsetroundjoin%
\definecolor{currentfill}{rgb}{0.121569,0.466667,0.705882}%
\pgfsetfillcolor{currentfill}%
\pgfsetlinewidth{1.003750pt}%
\definecolor{currentstroke}{rgb}{0.121569,0.466667,0.705882}%
\pgfsetstrokecolor{currentstroke}%
\pgfsetdash{}{0pt}%
\pgfpathmoveto{\pgfqpoint{3.291205in}{2.002225in}}%
\pgfpathcurveto{\pgfqpoint{3.302255in}{2.002225in}}{\pgfqpoint{3.312854in}{2.006615in}}{\pgfqpoint{3.320668in}{2.014429in}}%
\pgfpathcurveto{\pgfqpoint{3.328481in}{2.022242in}}{\pgfqpoint{3.332872in}{2.032841in}}{\pgfqpoint{3.332872in}{2.043892in}}%
\pgfpathcurveto{\pgfqpoint{3.332872in}{2.054942in}}{\pgfqpoint{3.328481in}{2.065541in}}{\pgfqpoint{3.320668in}{2.073354in}}%
\pgfpathcurveto{\pgfqpoint{3.312854in}{2.081168in}}{\pgfqpoint{3.302255in}{2.085558in}}{\pgfqpoint{3.291205in}{2.085558in}}%
\pgfpathcurveto{\pgfqpoint{3.280155in}{2.085558in}}{\pgfqpoint{3.269556in}{2.081168in}}{\pgfqpoint{3.261742in}{2.073354in}}%
\pgfpathcurveto{\pgfqpoint{3.253929in}{2.065541in}}{\pgfqpoint{3.249538in}{2.054942in}}{\pgfqpoint{3.249538in}{2.043892in}}%
\pgfpathcurveto{\pgfqpoint{3.249538in}{2.032841in}}{\pgfqpoint{3.253929in}{2.022242in}}{\pgfqpoint{3.261742in}{2.014429in}}%
\pgfpathcurveto{\pgfqpoint{3.269556in}{2.006615in}}{\pgfqpoint{3.280155in}{2.002225in}}{\pgfqpoint{3.291205in}{2.002225in}}%
\pgfpathclose%
\pgfusepath{stroke,fill}%
\end{pgfscope}%
\begin{pgfscope}%
\pgfpathrectangle{\pgfqpoint{0.600000in}{0.600000in}}{\pgfqpoint{3.900000in}{3.900000in}}%
\pgfusepath{clip}%
\pgfsetbuttcap%
\pgfsetroundjoin%
\definecolor{currentfill}{rgb}{0.121569,0.466667,0.705882}%
\pgfsetfillcolor{currentfill}%
\pgfsetlinewidth{1.003750pt}%
\definecolor{currentstroke}{rgb}{0.121569,0.466667,0.705882}%
\pgfsetstrokecolor{currentstroke}%
\pgfsetdash{}{0pt}%
\pgfpathmoveto{\pgfqpoint{2.837106in}{2.696532in}}%
\pgfpathcurveto{\pgfqpoint{2.848156in}{2.696532in}}{\pgfqpoint{2.858755in}{2.700922in}}{\pgfqpoint{2.866568in}{2.708736in}}%
\pgfpathcurveto{\pgfqpoint{2.874382in}{2.716549in}}{\pgfqpoint{2.878772in}{2.727148in}}{\pgfqpoint{2.878772in}{2.738198in}}%
\pgfpathcurveto{\pgfqpoint{2.878772in}{2.749249in}}{\pgfqpoint{2.874382in}{2.759848in}}{\pgfqpoint{2.866568in}{2.767661in}}%
\pgfpathcurveto{\pgfqpoint{2.858755in}{2.775475in}}{\pgfqpoint{2.848156in}{2.779865in}}{\pgfqpoint{2.837106in}{2.779865in}}%
\pgfpathcurveto{\pgfqpoint{2.826055in}{2.779865in}}{\pgfqpoint{2.815456in}{2.775475in}}{\pgfqpoint{2.807643in}{2.767661in}}%
\pgfpathcurveto{\pgfqpoint{2.799829in}{2.759848in}}{\pgfqpoint{2.795439in}{2.749249in}}{\pgfqpoint{2.795439in}{2.738198in}}%
\pgfpathcurveto{\pgfqpoint{2.795439in}{2.727148in}}{\pgfqpoint{2.799829in}{2.716549in}}{\pgfqpoint{2.807643in}{2.708736in}}%
\pgfpathcurveto{\pgfqpoint{2.815456in}{2.700922in}}{\pgfqpoint{2.826055in}{2.696532in}}{\pgfqpoint{2.837106in}{2.696532in}}%
\pgfpathclose%
\pgfusepath{stroke,fill}%
\end{pgfscope}%
\begin{pgfscope}%
\pgfpathrectangle{\pgfqpoint{0.600000in}{0.600000in}}{\pgfqpoint{3.900000in}{3.900000in}}%
\pgfusepath{clip}%
\pgfsetbuttcap%
\pgfsetroundjoin%
\definecolor{currentfill}{rgb}{0.121569,0.466667,0.705882}%
\pgfsetfillcolor{currentfill}%
\pgfsetlinewidth{1.003750pt}%
\definecolor{currentstroke}{rgb}{0.121569,0.466667,0.705882}%
\pgfsetstrokecolor{currentstroke}%
\pgfsetdash{}{0pt}%
\pgfpathmoveto{\pgfqpoint{2.166581in}{1.468867in}}%
\pgfpathcurveto{\pgfqpoint{2.177631in}{1.468867in}}{\pgfqpoint{2.188230in}{1.473257in}}{\pgfqpoint{2.196043in}{1.481071in}}%
\pgfpathcurveto{\pgfqpoint{2.203857in}{1.488885in}}{\pgfqpoint{2.208247in}{1.499484in}}{\pgfqpoint{2.208247in}{1.510534in}}%
\pgfpathcurveto{\pgfqpoint{2.208247in}{1.521584in}}{\pgfqpoint{2.203857in}{1.532183in}}{\pgfqpoint{2.196043in}{1.539997in}}%
\pgfpathcurveto{\pgfqpoint{2.188230in}{1.547810in}}{\pgfqpoint{2.177631in}{1.552200in}}{\pgfqpoint{2.166581in}{1.552200in}}%
\pgfpathcurveto{\pgfqpoint{2.155530in}{1.552200in}}{\pgfqpoint{2.144931in}{1.547810in}}{\pgfqpoint{2.137118in}{1.539997in}}%
\pgfpathcurveto{\pgfqpoint{2.129304in}{1.532183in}}{\pgfqpoint{2.124914in}{1.521584in}}{\pgfqpoint{2.124914in}{1.510534in}}%
\pgfpathcurveto{\pgfqpoint{2.124914in}{1.499484in}}{\pgfqpoint{2.129304in}{1.488885in}}{\pgfqpoint{2.137118in}{1.481071in}}%
\pgfpathcurveto{\pgfqpoint{2.144931in}{1.473257in}}{\pgfqpoint{2.155530in}{1.468867in}}{\pgfqpoint{2.166581in}{1.468867in}}%
\pgfpathclose%
\pgfusepath{stroke,fill}%
\end{pgfscope}%
\begin{pgfscope}%
\pgfpathrectangle{\pgfqpoint{0.600000in}{0.600000in}}{\pgfqpoint{3.900000in}{3.900000in}}%
\pgfusepath{clip}%
\pgfsetbuttcap%
\pgfsetroundjoin%
\definecolor{currentfill}{rgb}{0.121569,0.466667,0.705882}%
\pgfsetfillcolor{currentfill}%
\pgfsetlinewidth{1.003750pt}%
\definecolor{currentstroke}{rgb}{0.121569,0.466667,0.705882}%
\pgfsetstrokecolor{currentstroke}%
\pgfsetdash{}{0pt}%
\pgfpathmoveto{\pgfqpoint{3.003822in}{2.040345in}}%
\pgfpathcurveto{\pgfqpoint{3.014872in}{2.040345in}}{\pgfqpoint{3.025471in}{2.044735in}}{\pgfqpoint{3.033285in}{2.052548in}}%
\pgfpathcurveto{\pgfqpoint{3.041099in}{2.060362in}}{\pgfqpoint{3.045489in}{2.070961in}}{\pgfqpoint{3.045489in}{2.082011in}}%
\pgfpathcurveto{\pgfqpoint{3.045489in}{2.093061in}}{\pgfqpoint{3.041099in}{2.103660in}}{\pgfqpoint{3.033285in}{2.111474in}}%
\pgfpathcurveto{\pgfqpoint{3.025471in}{2.119288in}}{\pgfqpoint{3.014872in}{2.123678in}}{\pgfqpoint{3.003822in}{2.123678in}}%
\pgfpathcurveto{\pgfqpoint{2.992772in}{2.123678in}}{\pgfqpoint{2.982173in}{2.119288in}}{\pgfqpoint{2.974359in}{2.111474in}}%
\pgfpathcurveto{\pgfqpoint{2.966546in}{2.103660in}}{\pgfqpoint{2.962156in}{2.093061in}}{\pgfqpoint{2.962156in}{2.082011in}}%
\pgfpathcurveto{\pgfqpoint{2.962156in}{2.070961in}}{\pgfqpoint{2.966546in}{2.060362in}}{\pgfqpoint{2.974359in}{2.052548in}}%
\pgfpathcurveto{\pgfqpoint{2.982173in}{2.044735in}}{\pgfqpoint{2.992772in}{2.040345in}}{\pgfqpoint{3.003822in}{2.040345in}}%
\pgfpathclose%
\pgfusepath{stroke,fill}%
\end{pgfscope}%
\begin{pgfscope}%
\pgfpathrectangle{\pgfqpoint{0.600000in}{0.600000in}}{\pgfqpoint{3.900000in}{3.900000in}}%
\pgfusepath{clip}%
\pgfsetbuttcap%
\pgfsetroundjoin%
\definecolor{currentfill}{rgb}{0.121569,0.466667,0.705882}%
\pgfsetfillcolor{currentfill}%
\pgfsetlinewidth{1.003750pt}%
\definecolor{currentstroke}{rgb}{0.121569,0.466667,0.705882}%
\pgfsetstrokecolor{currentstroke}%
\pgfsetdash{}{0pt}%
\pgfpathmoveto{\pgfqpoint{2.285130in}{3.201892in}}%
\pgfpathcurveto{\pgfqpoint{2.296180in}{3.201892in}}{\pgfqpoint{2.306779in}{3.206282in}}{\pgfqpoint{2.314592in}{3.214096in}}%
\pgfpathcurveto{\pgfqpoint{2.322406in}{3.221909in}}{\pgfqpoint{2.326796in}{3.232509in}}{\pgfqpoint{2.326796in}{3.243559in}}%
\pgfpathcurveto{\pgfqpoint{2.326796in}{3.254609in}}{\pgfqpoint{2.322406in}{3.265208in}}{\pgfqpoint{2.314592in}{3.273021in}}%
\pgfpathcurveto{\pgfqpoint{2.306779in}{3.280835in}}{\pgfqpoint{2.296180in}{3.285225in}}{\pgfqpoint{2.285130in}{3.285225in}}%
\pgfpathcurveto{\pgfqpoint{2.274079in}{3.285225in}}{\pgfqpoint{2.263480in}{3.280835in}}{\pgfqpoint{2.255667in}{3.273021in}}%
\pgfpathcurveto{\pgfqpoint{2.247853in}{3.265208in}}{\pgfqpoint{2.243463in}{3.254609in}}{\pgfqpoint{2.243463in}{3.243559in}}%
\pgfpathcurveto{\pgfqpoint{2.243463in}{3.232509in}}{\pgfqpoint{2.247853in}{3.221909in}}{\pgfqpoint{2.255667in}{3.214096in}}%
\pgfpathcurveto{\pgfqpoint{2.263480in}{3.206282in}}{\pgfqpoint{2.274079in}{3.201892in}}{\pgfqpoint{2.285130in}{3.201892in}}%
\pgfpathclose%
\pgfusepath{stroke,fill}%
\end{pgfscope}%
\begin{pgfscope}%
\pgfpathrectangle{\pgfqpoint{0.600000in}{0.600000in}}{\pgfqpoint{3.900000in}{3.900000in}}%
\pgfusepath{clip}%
\pgfsetbuttcap%
\pgfsetroundjoin%
\definecolor{currentfill}{rgb}{0.121569,0.466667,0.705882}%
\pgfsetfillcolor{currentfill}%
\pgfsetlinewidth{1.003750pt}%
\definecolor{currentstroke}{rgb}{0.121569,0.466667,0.705882}%
\pgfsetstrokecolor{currentstroke}%
\pgfsetdash{}{0pt}%
\pgfpathmoveto{\pgfqpoint{3.140834in}{2.638112in}}%
\pgfpathcurveto{\pgfqpoint{3.151884in}{2.638112in}}{\pgfqpoint{3.162483in}{2.642502in}}{\pgfqpoint{3.170296in}{2.650316in}}%
\pgfpathcurveto{\pgfqpoint{3.178110in}{2.658130in}}{\pgfqpoint{3.182500in}{2.668729in}}{\pgfqpoint{3.182500in}{2.679779in}}%
\pgfpathcurveto{\pgfqpoint{3.182500in}{2.690829in}}{\pgfqpoint{3.178110in}{2.701428in}}{\pgfqpoint{3.170296in}{2.709241in}}%
\pgfpathcurveto{\pgfqpoint{3.162483in}{2.717055in}}{\pgfqpoint{3.151884in}{2.721445in}}{\pgfqpoint{3.140834in}{2.721445in}}%
\pgfpathcurveto{\pgfqpoint{3.129783in}{2.721445in}}{\pgfqpoint{3.119184in}{2.717055in}}{\pgfqpoint{3.111371in}{2.709241in}}%
\pgfpathcurveto{\pgfqpoint{3.103557in}{2.701428in}}{\pgfqpoint{3.099167in}{2.690829in}}{\pgfqpoint{3.099167in}{2.679779in}}%
\pgfpathcurveto{\pgfqpoint{3.099167in}{2.668729in}}{\pgfqpoint{3.103557in}{2.658130in}}{\pgfqpoint{3.111371in}{2.650316in}}%
\pgfpathcurveto{\pgfqpoint{3.119184in}{2.642502in}}{\pgfqpoint{3.129783in}{2.638112in}}{\pgfqpoint{3.140834in}{2.638112in}}%
\pgfpathclose%
\pgfusepath{stroke,fill}%
\end{pgfscope}%
\begin{pgfscope}%
\pgfpathrectangle{\pgfqpoint{0.600000in}{0.600000in}}{\pgfqpoint{3.900000in}{3.900000in}}%
\pgfusepath{clip}%
\pgfsetbuttcap%
\pgfsetroundjoin%
\definecolor{currentfill}{rgb}{0.121569,0.466667,0.705882}%
\pgfsetfillcolor{currentfill}%
\pgfsetlinewidth{1.003750pt}%
\definecolor{currentstroke}{rgb}{0.121569,0.466667,0.705882}%
\pgfsetstrokecolor{currentstroke}%
\pgfsetdash{}{0pt}%
\pgfpathmoveto{\pgfqpoint{2.668415in}{2.279240in}}%
\pgfpathcurveto{\pgfqpoint{2.679465in}{2.279240in}}{\pgfqpoint{2.690064in}{2.283630in}}{\pgfqpoint{2.697878in}{2.291443in}}%
\pgfpathcurveto{\pgfqpoint{2.705691in}{2.299257in}}{\pgfqpoint{2.710082in}{2.309856in}}{\pgfqpoint{2.710082in}{2.320906in}}%
\pgfpathcurveto{\pgfqpoint{2.710082in}{2.331956in}}{\pgfqpoint{2.705691in}{2.342555in}}{\pgfqpoint{2.697878in}{2.350369in}}%
\pgfpathcurveto{\pgfqpoint{2.690064in}{2.358183in}}{\pgfqpoint{2.679465in}{2.362573in}}{\pgfqpoint{2.668415in}{2.362573in}}%
\pgfpathcurveto{\pgfqpoint{2.657365in}{2.362573in}}{\pgfqpoint{2.646766in}{2.358183in}}{\pgfqpoint{2.638952in}{2.350369in}}%
\pgfpathcurveto{\pgfqpoint{2.631139in}{2.342555in}}{\pgfqpoint{2.626748in}{2.331956in}}{\pgfqpoint{2.626748in}{2.320906in}}%
\pgfpathcurveto{\pgfqpoint{2.626748in}{2.309856in}}{\pgfqpoint{2.631139in}{2.299257in}}{\pgfqpoint{2.638952in}{2.291443in}}%
\pgfpathcurveto{\pgfqpoint{2.646766in}{2.283630in}}{\pgfqpoint{2.657365in}{2.279240in}}{\pgfqpoint{2.668415in}{2.279240in}}%
\pgfpathclose%
\pgfusepath{stroke,fill}%
\end{pgfscope}%
\begin{pgfscope}%
\pgfpathrectangle{\pgfqpoint{0.600000in}{0.600000in}}{\pgfqpoint{3.900000in}{3.900000in}}%
\pgfusepath{clip}%
\pgfsetbuttcap%
\pgfsetroundjoin%
\definecolor{currentfill}{rgb}{0.121569,0.466667,0.705882}%
\pgfsetfillcolor{currentfill}%
\pgfsetlinewidth{1.003750pt}%
\definecolor{currentstroke}{rgb}{0.121569,0.466667,0.705882}%
\pgfsetstrokecolor{currentstroke}%
\pgfsetdash{}{0pt}%
\pgfpathmoveto{\pgfqpoint{2.493189in}{2.006709in}}%
\pgfpathcurveto{\pgfqpoint{2.504239in}{2.006709in}}{\pgfqpoint{2.514838in}{2.011099in}}{\pgfqpoint{2.522652in}{2.018913in}}%
\pgfpathcurveto{\pgfqpoint{2.530466in}{2.026726in}}{\pgfqpoint{2.534856in}{2.037325in}}{\pgfqpoint{2.534856in}{2.048376in}}%
\pgfpathcurveto{\pgfqpoint{2.534856in}{2.059426in}}{\pgfqpoint{2.530466in}{2.070025in}}{\pgfqpoint{2.522652in}{2.077838in}}%
\pgfpathcurveto{\pgfqpoint{2.514838in}{2.085652in}}{\pgfqpoint{2.504239in}{2.090042in}}{\pgfqpoint{2.493189in}{2.090042in}}%
\pgfpathcurveto{\pgfqpoint{2.482139in}{2.090042in}}{\pgfqpoint{2.471540in}{2.085652in}}{\pgfqpoint{2.463727in}{2.077838in}}%
\pgfpathcurveto{\pgfqpoint{2.455913in}{2.070025in}}{\pgfqpoint{2.451523in}{2.059426in}}{\pgfqpoint{2.451523in}{2.048376in}}%
\pgfpathcurveto{\pgfqpoint{2.451523in}{2.037325in}}{\pgfqpoint{2.455913in}{2.026726in}}{\pgfqpoint{2.463727in}{2.018913in}}%
\pgfpathcurveto{\pgfqpoint{2.471540in}{2.011099in}}{\pgfqpoint{2.482139in}{2.006709in}}{\pgfqpoint{2.493189in}{2.006709in}}%
\pgfpathclose%
\pgfusepath{stroke,fill}%
\end{pgfscope}%
\begin{pgfscope}%
\pgfpathrectangle{\pgfqpoint{0.600000in}{0.600000in}}{\pgfqpoint{3.900000in}{3.900000in}}%
\pgfusepath{clip}%
\pgfsetbuttcap%
\pgfsetroundjoin%
\definecolor{currentfill}{rgb}{0.121569,0.466667,0.705882}%
\pgfsetfillcolor{currentfill}%
\pgfsetlinewidth{1.003750pt}%
\definecolor{currentstroke}{rgb}{0.121569,0.466667,0.705882}%
\pgfsetstrokecolor{currentstroke}%
\pgfsetdash{}{0pt}%
\pgfpathmoveto{\pgfqpoint{3.021077in}{1.369769in}}%
\pgfpathcurveto{\pgfqpoint{3.032127in}{1.369769in}}{\pgfqpoint{3.042726in}{1.374160in}}{\pgfqpoint{3.050539in}{1.381973in}}%
\pgfpathcurveto{\pgfqpoint{3.058353in}{1.389787in}}{\pgfqpoint{3.062743in}{1.400386in}}{\pgfqpoint{3.062743in}{1.411436in}}%
\pgfpathcurveto{\pgfqpoint{3.062743in}{1.422486in}}{\pgfqpoint{3.058353in}{1.433085in}}{\pgfqpoint{3.050539in}{1.440899in}}%
\pgfpathcurveto{\pgfqpoint{3.042726in}{1.448713in}}{\pgfqpoint{3.032127in}{1.453103in}}{\pgfqpoint{3.021077in}{1.453103in}}%
\pgfpathcurveto{\pgfqpoint{3.010026in}{1.453103in}}{\pgfqpoint{2.999427in}{1.448713in}}{\pgfqpoint{2.991614in}{1.440899in}}%
\pgfpathcurveto{\pgfqpoint{2.983800in}{1.433085in}}{\pgfqpoint{2.979410in}{1.422486in}}{\pgfqpoint{2.979410in}{1.411436in}}%
\pgfpathcurveto{\pgfqpoint{2.979410in}{1.400386in}}{\pgfqpoint{2.983800in}{1.389787in}}{\pgfqpoint{2.991614in}{1.381973in}}%
\pgfpathcurveto{\pgfqpoint{2.999427in}{1.374160in}}{\pgfqpoint{3.010026in}{1.369769in}}{\pgfqpoint{3.021077in}{1.369769in}}%
\pgfpathclose%
\pgfusepath{stroke,fill}%
\end{pgfscope}%
\begin{pgfscope}%
\pgfpathrectangle{\pgfqpoint{0.600000in}{0.600000in}}{\pgfqpoint{3.900000in}{3.900000in}}%
\pgfusepath{clip}%
\pgfsetbuttcap%
\pgfsetroundjoin%
\definecolor{currentfill}{rgb}{0.121569,0.466667,0.705882}%
\pgfsetfillcolor{currentfill}%
\pgfsetlinewidth{1.003750pt}%
\definecolor{currentstroke}{rgb}{0.121569,0.466667,0.705882}%
\pgfsetstrokecolor{currentstroke}%
\pgfsetdash{}{0pt}%
\pgfpathmoveto{\pgfqpoint{2.591358in}{2.785156in}}%
\pgfpathcurveto{\pgfqpoint{2.602408in}{2.785156in}}{\pgfqpoint{2.613007in}{2.789546in}}{\pgfqpoint{2.620821in}{2.797360in}}%
\pgfpathcurveto{\pgfqpoint{2.628634in}{2.805174in}}{\pgfqpoint{2.633025in}{2.815773in}}{\pgfqpoint{2.633025in}{2.826823in}}%
\pgfpathcurveto{\pgfqpoint{2.633025in}{2.837873in}}{\pgfqpoint{2.628634in}{2.848472in}}{\pgfqpoint{2.620821in}{2.856286in}}%
\pgfpathcurveto{\pgfqpoint{2.613007in}{2.864099in}}{\pgfqpoint{2.602408in}{2.868490in}}{\pgfqpoint{2.591358in}{2.868490in}}%
\pgfpathcurveto{\pgfqpoint{2.580308in}{2.868490in}}{\pgfqpoint{2.569709in}{2.864099in}}{\pgfqpoint{2.561895in}{2.856286in}}%
\pgfpathcurveto{\pgfqpoint{2.554081in}{2.848472in}}{\pgfqpoint{2.549691in}{2.837873in}}{\pgfqpoint{2.549691in}{2.826823in}}%
\pgfpathcurveto{\pgfqpoint{2.549691in}{2.815773in}}{\pgfqpoint{2.554081in}{2.805174in}}{\pgfqpoint{2.561895in}{2.797360in}}%
\pgfpathcurveto{\pgfqpoint{2.569709in}{2.789546in}}{\pgfqpoint{2.580308in}{2.785156in}}{\pgfqpoint{2.591358in}{2.785156in}}%
\pgfpathclose%
\pgfusepath{stroke,fill}%
\end{pgfscope}%
\begin{pgfscope}%
\pgfpathrectangle{\pgfqpoint{0.600000in}{0.600000in}}{\pgfqpoint{3.900000in}{3.900000in}}%
\pgfusepath{clip}%
\pgfsetbuttcap%
\pgfsetroundjoin%
\definecolor{currentfill}{rgb}{0.121569,0.466667,0.705882}%
\pgfsetfillcolor{currentfill}%
\pgfsetlinewidth{1.003750pt}%
\definecolor{currentstroke}{rgb}{0.121569,0.466667,0.705882}%
\pgfsetstrokecolor{currentstroke}%
\pgfsetdash{}{0pt}%
\pgfpathmoveto{\pgfqpoint{2.589252in}{2.472380in}}%
\pgfpathcurveto{\pgfqpoint{2.600303in}{2.472380in}}{\pgfqpoint{2.610902in}{2.476770in}}{\pgfqpoint{2.618715in}{2.484584in}}%
\pgfpathcurveto{\pgfqpoint{2.626529in}{2.492398in}}{\pgfqpoint{2.630919in}{2.502997in}}{\pgfqpoint{2.630919in}{2.514047in}}%
\pgfpathcurveto{\pgfqpoint{2.630919in}{2.525097in}}{\pgfqpoint{2.626529in}{2.535696in}}{\pgfqpoint{2.618715in}{2.543510in}}%
\pgfpathcurveto{\pgfqpoint{2.610902in}{2.551323in}}{\pgfqpoint{2.600303in}{2.555714in}}{\pgfqpoint{2.589252in}{2.555714in}}%
\pgfpathcurveto{\pgfqpoint{2.578202in}{2.555714in}}{\pgfqpoint{2.567603in}{2.551323in}}{\pgfqpoint{2.559790in}{2.543510in}}%
\pgfpathcurveto{\pgfqpoint{2.551976in}{2.535696in}}{\pgfqpoint{2.547586in}{2.525097in}}{\pgfqpoint{2.547586in}{2.514047in}}%
\pgfpathcurveto{\pgfqpoint{2.547586in}{2.502997in}}{\pgfqpoint{2.551976in}{2.492398in}}{\pgfqpoint{2.559790in}{2.484584in}}%
\pgfpathcurveto{\pgfqpoint{2.567603in}{2.476770in}}{\pgfqpoint{2.578202in}{2.472380in}}{\pgfqpoint{2.589252in}{2.472380in}}%
\pgfpathclose%
\pgfusepath{stroke,fill}%
\end{pgfscope}%
\begin{pgfscope}%
\pgfpathrectangle{\pgfqpoint{0.600000in}{0.600000in}}{\pgfqpoint{3.900000in}{3.900000in}}%
\pgfusepath{clip}%
\pgfsetbuttcap%
\pgfsetroundjoin%
\definecolor{currentfill}{rgb}{0.121569,0.466667,0.705882}%
\pgfsetfillcolor{currentfill}%
\pgfsetlinewidth{1.003750pt}%
\definecolor{currentstroke}{rgb}{0.121569,0.466667,0.705882}%
\pgfsetstrokecolor{currentstroke}%
\pgfsetdash{}{0pt}%
\pgfpathmoveto{\pgfqpoint{2.095013in}{2.022645in}}%
\pgfpathcurveto{\pgfqpoint{2.106063in}{2.022645in}}{\pgfqpoint{2.116662in}{2.027035in}}{\pgfqpoint{2.124475in}{2.034849in}}%
\pgfpathcurveto{\pgfqpoint{2.132289in}{2.042663in}}{\pgfqpoint{2.136679in}{2.053262in}}{\pgfqpoint{2.136679in}{2.064312in}}%
\pgfpathcurveto{\pgfqpoint{2.136679in}{2.075362in}}{\pgfqpoint{2.132289in}{2.085961in}}{\pgfqpoint{2.124475in}{2.093774in}}%
\pgfpathcurveto{\pgfqpoint{2.116662in}{2.101588in}}{\pgfqpoint{2.106063in}{2.105978in}}{\pgfqpoint{2.095013in}{2.105978in}}%
\pgfpathcurveto{\pgfqpoint{2.083962in}{2.105978in}}{\pgfqpoint{2.073363in}{2.101588in}}{\pgfqpoint{2.065550in}{2.093774in}}%
\pgfpathcurveto{\pgfqpoint{2.057736in}{2.085961in}}{\pgfqpoint{2.053346in}{2.075362in}}{\pgfqpoint{2.053346in}{2.064312in}}%
\pgfpathcurveto{\pgfqpoint{2.053346in}{2.053262in}}{\pgfqpoint{2.057736in}{2.042663in}}{\pgfqpoint{2.065550in}{2.034849in}}%
\pgfpathcurveto{\pgfqpoint{2.073363in}{2.027035in}}{\pgfqpoint{2.083962in}{2.022645in}}{\pgfqpoint{2.095013in}{2.022645in}}%
\pgfpathclose%
\pgfusepath{stroke,fill}%
\end{pgfscope}%
\begin{pgfscope}%
\pgfpathrectangle{\pgfqpoint{0.600000in}{0.600000in}}{\pgfqpoint{3.900000in}{3.900000in}}%
\pgfusepath{clip}%
\pgfsetbuttcap%
\pgfsetroundjoin%
\definecolor{currentfill}{rgb}{0.121569,0.466667,0.705882}%
\pgfsetfillcolor{currentfill}%
\pgfsetlinewidth{1.003750pt}%
\definecolor{currentstroke}{rgb}{0.121569,0.466667,0.705882}%
\pgfsetstrokecolor{currentstroke}%
\pgfsetdash{}{0pt}%
\pgfpathmoveto{\pgfqpoint{2.947414in}{2.448287in}}%
\pgfpathcurveto{\pgfqpoint{2.958464in}{2.448287in}}{\pgfqpoint{2.969063in}{2.452678in}}{\pgfqpoint{2.976877in}{2.460491in}}%
\pgfpathcurveto{\pgfqpoint{2.984690in}{2.468305in}}{\pgfqpoint{2.989080in}{2.478904in}}{\pgfqpoint{2.989080in}{2.489954in}}%
\pgfpathcurveto{\pgfqpoint{2.989080in}{2.501004in}}{\pgfqpoint{2.984690in}{2.511603in}}{\pgfqpoint{2.976877in}{2.519417in}}%
\pgfpathcurveto{\pgfqpoint{2.969063in}{2.527230in}}{\pgfqpoint{2.958464in}{2.531621in}}{\pgfqpoint{2.947414in}{2.531621in}}%
\pgfpathcurveto{\pgfqpoint{2.936364in}{2.531621in}}{\pgfqpoint{2.925765in}{2.527230in}}{\pgfqpoint{2.917951in}{2.519417in}}%
\pgfpathcurveto{\pgfqpoint{2.910137in}{2.511603in}}{\pgfqpoint{2.905747in}{2.501004in}}{\pgfqpoint{2.905747in}{2.489954in}}%
\pgfpathcurveto{\pgfqpoint{2.905747in}{2.478904in}}{\pgfqpoint{2.910137in}{2.468305in}}{\pgfqpoint{2.917951in}{2.460491in}}%
\pgfpathcurveto{\pgfqpoint{2.925765in}{2.452678in}}{\pgfqpoint{2.936364in}{2.448287in}}{\pgfqpoint{2.947414in}{2.448287in}}%
\pgfpathclose%
\pgfusepath{stroke,fill}%
\end{pgfscope}%
\begin{pgfscope}%
\pgfpathrectangle{\pgfqpoint{0.600000in}{0.600000in}}{\pgfqpoint{3.900000in}{3.900000in}}%
\pgfusepath{clip}%
\pgfsetbuttcap%
\pgfsetroundjoin%
\definecolor{currentfill}{rgb}{0.121569,0.466667,0.705882}%
\pgfsetfillcolor{currentfill}%
\pgfsetlinewidth{1.003750pt}%
\definecolor{currentstroke}{rgb}{0.121569,0.466667,0.705882}%
\pgfsetstrokecolor{currentstroke}%
\pgfsetdash{}{0pt}%
\pgfpathmoveto{\pgfqpoint{2.702167in}{2.312574in}}%
\pgfpathcurveto{\pgfqpoint{2.713218in}{2.312574in}}{\pgfqpoint{2.723817in}{2.316965in}}{\pgfqpoint{2.731630in}{2.324778in}}%
\pgfpathcurveto{\pgfqpoint{2.739444in}{2.332592in}}{\pgfqpoint{2.743834in}{2.343191in}}{\pgfqpoint{2.743834in}{2.354241in}}%
\pgfpathcurveto{\pgfqpoint{2.743834in}{2.365291in}}{\pgfqpoint{2.739444in}{2.375890in}}{\pgfqpoint{2.731630in}{2.383704in}}%
\pgfpathcurveto{\pgfqpoint{2.723817in}{2.391517in}}{\pgfqpoint{2.713218in}{2.395908in}}{\pgfqpoint{2.702167in}{2.395908in}}%
\pgfpathcurveto{\pgfqpoint{2.691117in}{2.395908in}}{\pgfqpoint{2.680518in}{2.391517in}}{\pgfqpoint{2.672705in}{2.383704in}}%
\pgfpathcurveto{\pgfqpoint{2.664891in}{2.375890in}}{\pgfqpoint{2.660501in}{2.365291in}}{\pgfqpoint{2.660501in}{2.354241in}}%
\pgfpathcurveto{\pgfqpoint{2.660501in}{2.343191in}}{\pgfqpoint{2.664891in}{2.332592in}}{\pgfqpoint{2.672705in}{2.324778in}}%
\pgfpathcurveto{\pgfqpoint{2.680518in}{2.316965in}}{\pgfqpoint{2.691117in}{2.312574in}}{\pgfqpoint{2.702167in}{2.312574in}}%
\pgfpathclose%
\pgfusepath{stroke,fill}%
\end{pgfscope}%
\begin{pgfscope}%
\pgfpathrectangle{\pgfqpoint{0.600000in}{0.600000in}}{\pgfqpoint{3.900000in}{3.900000in}}%
\pgfusepath{clip}%
\pgfsetbuttcap%
\pgfsetroundjoin%
\definecolor{currentfill}{rgb}{0.121569,0.466667,0.705882}%
\pgfsetfillcolor{currentfill}%
\pgfsetlinewidth{1.003750pt}%
\definecolor{currentstroke}{rgb}{0.121569,0.466667,0.705882}%
\pgfsetstrokecolor{currentstroke}%
\pgfsetdash{}{0pt}%
\pgfpathmoveto{\pgfqpoint{2.363906in}{1.962986in}}%
\pgfpathcurveto{\pgfqpoint{2.374956in}{1.962986in}}{\pgfqpoint{2.385555in}{1.967376in}}{\pgfqpoint{2.393368in}{1.975189in}}%
\pgfpathcurveto{\pgfqpoint{2.401182in}{1.983003in}}{\pgfqpoint{2.405572in}{1.993602in}}{\pgfqpoint{2.405572in}{2.004652in}}%
\pgfpathcurveto{\pgfqpoint{2.405572in}{2.015702in}}{\pgfqpoint{2.401182in}{2.026301in}}{\pgfqpoint{2.393368in}{2.034115in}}%
\pgfpathcurveto{\pgfqpoint{2.385555in}{2.041929in}}{\pgfqpoint{2.374956in}{2.046319in}}{\pgfqpoint{2.363906in}{2.046319in}}%
\pgfpathcurveto{\pgfqpoint{2.352855in}{2.046319in}}{\pgfqpoint{2.342256in}{2.041929in}}{\pgfqpoint{2.334443in}{2.034115in}}%
\pgfpathcurveto{\pgfqpoint{2.326629in}{2.026301in}}{\pgfqpoint{2.322239in}{2.015702in}}{\pgfqpoint{2.322239in}{2.004652in}}%
\pgfpathcurveto{\pgfqpoint{2.322239in}{1.993602in}}{\pgfqpoint{2.326629in}{1.983003in}}{\pgfqpoint{2.334443in}{1.975189in}}%
\pgfpathcurveto{\pgfqpoint{2.342256in}{1.967376in}}{\pgfqpoint{2.352855in}{1.962986in}}{\pgfqpoint{2.363906in}{1.962986in}}%
\pgfpathclose%
\pgfusepath{stroke,fill}%
\end{pgfscope}%
\begin{pgfscope}%
\pgfpathrectangle{\pgfqpoint{0.600000in}{0.600000in}}{\pgfqpoint{3.900000in}{3.900000in}}%
\pgfusepath{clip}%
\pgfsetbuttcap%
\pgfsetroundjoin%
\definecolor{currentfill}{rgb}{0.121569,0.466667,0.705882}%
\pgfsetfillcolor{currentfill}%
\pgfsetlinewidth{1.003750pt}%
\definecolor{currentstroke}{rgb}{0.121569,0.466667,0.705882}%
\pgfsetstrokecolor{currentstroke}%
\pgfsetdash{}{0pt}%
\pgfpathmoveto{\pgfqpoint{2.592126in}{3.049524in}}%
\pgfpathcurveto{\pgfqpoint{2.603176in}{3.049524in}}{\pgfqpoint{2.613775in}{3.053914in}}{\pgfqpoint{2.621588in}{3.061728in}}%
\pgfpathcurveto{\pgfqpoint{2.629402in}{3.069541in}}{\pgfqpoint{2.633792in}{3.080140in}}{\pgfqpoint{2.633792in}{3.091190in}}%
\pgfpathcurveto{\pgfqpoint{2.633792in}{3.102241in}}{\pgfqpoint{2.629402in}{3.112840in}}{\pgfqpoint{2.621588in}{3.120653in}}%
\pgfpathcurveto{\pgfqpoint{2.613775in}{3.128467in}}{\pgfqpoint{2.603176in}{3.132857in}}{\pgfqpoint{2.592126in}{3.132857in}}%
\pgfpathcurveto{\pgfqpoint{2.581075in}{3.132857in}}{\pgfqpoint{2.570476in}{3.128467in}}{\pgfqpoint{2.562663in}{3.120653in}}%
\pgfpathcurveto{\pgfqpoint{2.554849in}{3.112840in}}{\pgfqpoint{2.550459in}{3.102241in}}{\pgfqpoint{2.550459in}{3.091190in}}%
\pgfpathcurveto{\pgfqpoint{2.550459in}{3.080140in}}{\pgfqpoint{2.554849in}{3.069541in}}{\pgfqpoint{2.562663in}{3.061728in}}%
\pgfpathcurveto{\pgfqpoint{2.570476in}{3.053914in}}{\pgfqpoint{2.581075in}{3.049524in}}{\pgfqpoint{2.592126in}{3.049524in}}%
\pgfpathclose%
\pgfusepath{stroke,fill}%
\end{pgfscope}%
\begin{pgfscope}%
\pgfpathrectangle{\pgfqpoint{0.600000in}{0.600000in}}{\pgfqpoint{3.900000in}{3.900000in}}%
\pgfusepath{clip}%
\pgfsetbuttcap%
\pgfsetroundjoin%
\definecolor{currentfill}{rgb}{0.121569,0.466667,0.705882}%
\pgfsetfillcolor{currentfill}%
\pgfsetlinewidth{1.003750pt}%
\definecolor{currentstroke}{rgb}{0.121569,0.466667,0.705882}%
\pgfsetstrokecolor{currentstroke}%
\pgfsetdash{}{0pt}%
\pgfpathmoveto{\pgfqpoint{2.306330in}{1.679248in}}%
\pgfpathcurveto{\pgfqpoint{2.317380in}{1.679248in}}{\pgfqpoint{2.327979in}{1.683638in}}{\pgfqpoint{2.335792in}{1.691451in}}%
\pgfpathcurveto{\pgfqpoint{2.343606in}{1.699265in}}{\pgfqpoint{2.347996in}{1.709864in}}{\pgfqpoint{2.347996in}{1.720914in}}%
\pgfpathcurveto{\pgfqpoint{2.347996in}{1.731964in}}{\pgfqpoint{2.343606in}{1.742563in}}{\pgfqpoint{2.335792in}{1.750377in}}%
\pgfpathcurveto{\pgfqpoint{2.327979in}{1.758191in}}{\pgfqpoint{2.317380in}{1.762581in}}{\pgfqpoint{2.306330in}{1.762581in}}%
\pgfpathcurveto{\pgfqpoint{2.295279in}{1.762581in}}{\pgfqpoint{2.284680in}{1.758191in}}{\pgfqpoint{2.276867in}{1.750377in}}%
\pgfpathcurveto{\pgfqpoint{2.269053in}{1.742563in}}{\pgfqpoint{2.264663in}{1.731964in}}{\pgfqpoint{2.264663in}{1.720914in}}%
\pgfpathcurveto{\pgfqpoint{2.264663in}{1.709864in}}{\pgfqpoint{2.269053in}{1.699265in}}{\pgfqpoint{2.276867in}{1.691451in}}%
\pgfpathcurveto{\pgfqpoint{2.284680in}{1.683638in}}{\pgfqpoint{2.295279in}{1.679248in}}{\pgfqpoint{2.306330in}{1.679248in}}%
\pgfpathclose%
\pgfusepath{stroke,fill}%
\end{pgfscope}%
\begin{pgfscope}%
\pgfpathrectangle{\pgfqpoint{0.600000in}{0.600000in}}{\pgfqpoint{3.900000in}{3.900000in}}%
\pgfusepath{clip}%
\pgfsetbuttcap%
\pgfsetroundjoin%
\definecolor{currentfill}{rgb}{0.121569,0.466667,0.705882}%
\pgfsetfillcolor{currentfill}%
\pgfsetlinewidth{1.003750pt}%
\definecolor{currentstroke}{rgb}{0.121569,0.466667,0.705882}%
\pgfsetstrokecolor{currentstroke}%
\pgfsetdash{}{0pt}%
\pgfpathmoveto{\pgfqpoint{2.171717in}{3.056445in}}%
\pgfpathcurveto{\pgfqpoint{2.182767in}{3.056445in}}{\pgfqpoint{2.193366in}{3.060836in}}{\pgfqpoint{2.201180in}{3.068649in}}%
\pgfpathcurveto{\pgfqpoint{2.208993in}{3.076463in}}{\pgfqpoint{2.213384in}{3.087062in}}{\pgfqpoint{2.213384in}{3.098112in}}%
\pgfpathcurveto{\pgfqpoint{2.213384in}{3.109162in}}{\pgfqpoint{2.208993in}{3.119761in}}{\pgfqpoint{2.201180in}{3.127575in}}%
\pgfpathcurveto{\pgfqpoint{2.193366in}{3.135389in}}{\pgfqpoint{2.182767in}{3.139779in}}{\pgfqpoint{2.171717in}{3.139779in}}%
\pgfpathcurveto{\pgfqpoint{2.160667in}{3.139779in}}{\pgfqpoint{2.150068in}{3.135389in}}{\pgfqpoint{2.142254in}{3.127575in}}%
\pgfpathcurveto{\pgfqpoint{2.134440in}{3.119761in}}{\pgfqpoint{2.130050in}{3.109162in}}{\pgfqpoint{2.130050in}{3.098112in}}%
\pgfpathcurveto{\pgfqpoint{2.130050in}{3.087062in}}{\pgfqpoint{2.134440in}{3.076463in}}{\pgfqpoint{2.142254in}{3.068649in}}%
\pgfpathcurveto{\pgfqpoint{2.150068in}{3.060836in}}{\pgfqpoint{2.160667in}{3.056445in}}{\pgfqpoint{2.171717in}{3.056445in}}%
\pgfpathclose%
\pgfusepath{stroke,fill}%
\end{pgfscope}%
\begin{pgfscope}%
\pgfpathrectangle{\pgfqpoint{0.600000in}{0.600000in}}{\pgfqpoint{3.900000in}{3.900000in}}%
\pgfusepath{clip}%
\pgfsetbuttcap%
\pgfsetroundjoin%
\definecolor{currentfill}{rgb}{0.121569,0.466667,0.705882}%
\pgfsetfillcolor{currentfill}%
\pgfsetlinewidth{1.003750pt}%
\definecolor{currentstroke}{rgb}{0.121569,0.466667,0.705882}%
\pgfsetstrokecolor{currentstroke}%
\pgfsetdash{}{0pt}%
\pgfpathmoveto{\pgfqpoint{2.367046in}{2.634638in}}%
\pgfpathcurveto{\pgfqpoint{2.378096in}{2.634638in}}{\pgfqpoint{2.388695in}{2.639028in}}{\pgfqpoint{2.396509in}{2.646842in}}%
\pgfpathcurveto{\pgfqpoint{2.404322in}{2.654655in}}{\pgfqpoint{2.408713in}{2.665254in}}{\pgfqpoint{2.408713in}{2.676304in}}%
\pgfpathcurveto{\pgfqpoint{2.408713in}{2.687354in}}{\pgfqpoint{2.404322in}{2.697954in}}{\pgfqpoint{2.396509in}{2.705767in}}%
\pgfpathcurveto{\pgfqpoint{2.388695in}{2.713581in}}{\pgfqpoint{2.378096in}{2.717971in}}{\pgfqpoint{2.367046in}{2.717971in}}%
\pgfpathcurveto{\pgfqpoint{2.355996in}{2.717971in}}{\pgfqpoint{2.345397in}{2.713581in}}{\pgfqpoint{2.337583in}{2.705767in}}%
\pgfpathcurveto{\pgfqpoint{2.329770in}{2.697954in}}{\pgfqpoint{2.325379in}{2.687354in}}{\pgfqpoint{2.325379in}{2.676304in}}%
\pgfpathcurveto{\pgfqpoint{2.325379in}{2.665254in}}{\pgfqpoint{2.329770in}{2.654655in}}{\pgfqpoint{2.337583in}{2.646842in}}%
\pgfpathcurveto{\pgfqpoint{2.345397in}{2.639028in}}{\pgfqpoint{2.355996in}{2.634638in}}{\pgfqpoint{2.367046in}{2.634638in}}%
\pgfpathclose%
\pgfusepath{stroke,fill}%
\end{pgfscope}%
\begin{pgfscope}%
\pgfpathrectangle{\pgfqpoint{0.600000in}{0.600000in}}{\pgfqpoint{3.900000in}{3.900000in}}%
\pgfusepath{clip}%
\pgfsetbuttcap%
\pgfsetroundjoin%
\definecolor{currentfill}{rgb}{0.121569,0.466667,0.705882}%
\pgfsetfillcolor{currentfill}%
\pgfsetlinewidth{1.003750pt}%
\definecolor{currentstroke}{rgb}{0.121569,0.466667,0.705882}%
\pgfsetstrokecolor{currentstroke}%
\pgfsetdash{}{0pt}%
\pgfpathmoveto{\pgfqpoint{2.913717in}{2.009611in}}%
\pgfpathcurveto{\pgfqpoint{2.924767in}{2.009611in}}{\pgfqpoint{2.935366in}{2.014001in}}{\pgfqpoint{2.943180in}{2.021815in}}%
\pgfpathcurveto{\pgfqpoint{2.950994in}{2.029629in}}{\pgfqpoint{2.955384in}{2.040228in}}{\pgfqpoint{2.955384in}{2.051278in}}%
\pgfpathcurveto{\pgfqpoint{2.955384in}{2.062328in}}{\pgfqpoint{2.950994in}{2.072927in}}{\pgfqpoint{2.943180in}{2.080741in}}%
\pgfpathcurveto{\pgfqpoint{2.935366in}{2.088554in}}{\pgfqpoint{2.924767in}{2.092944in}}{\pgfqpoint{2.913717in}{2.092944in}}%
\pgfpathcurveto{\pgfqpoint{2.902667in}{2.092944in}}{\pgfqpoint{2.892068in}{2.088554in}}{\pgfqpoint{2.884254in}{2.080741in}}%
\pgfpathcurveto{\pgfqpoint{2.876441in}{2.072927in}}{\pgfqpoint{2.872051in}{2.062328in}}{\pgfqpoint{2.872051in}{2.051278in}}%
\pgfpathcurveto{\pgfqpoint{2.872051in}{2.040228in}}{\pgfqpoint{2.876441in}{2.029629in}}{\pgfqpoint{2.884254in}{2.021815in}}%
\pgfpathcurveto{\pgfqpoint{2.892068in}{2.014001in}}{\pgfqpoint{2.902667in}{2.009611in}}{\pgfqpoint{2.913717in}{2.009611in}}%
\pgfpathclose%
\pgfusepath{stroke,fill}%
\end{pgfscope}%
\begin{pgfscope}%
\pgfpathrectangle{\pgfqpoint{0.600000in}{0.600000in}}{\pgfqpoint{3.900000in}{3.900000in}}%
\pgfusepath{clip}%
\pgfsetbuttcap%
\pgfsetroundjoin%
\definecolor{currentfill}{rgb}{0.121569,0.466667,0.705882}%
\pgfsetfillcolor{currentfill}%
\pgfsetlinewidth{1.003750pt}%
\definecolor{currentstroke}{rgb}{0.121569,0.466667,0.705882}%
\pgfsetstrokecolor{currentstroke}%
\pgfsetdash{}{0pt}%
\pgfpathmoveto{\pgfqpoint{3.299013in}{2.255738in}}%
\pgfpathcurveto{\pgfqpoint{3.310063in}{2.255738in}}{\pgfqpoint{3.320662in}{2.260128in}}{\pgfqpoint{3.328475in}{2.267942in}}%
\pgfpathcurveto{\pgfqpoint{3.336289in}{2.275755in}}{\pgfqpoint{3.340679in}{2.286354in}}{\pgfqpoint{3.340679in}{2.297404in}}%
\pgfpathcurveto{\pgfqpoint{3.340679in}{2.308454in}}{\pgfqpoint{3.336289in}{2.319053in}}{\pgfqpoint{3.328475in}{2.326867in}}%
\pgfpathcurveto{\pgfqpoint{3.320662in}{2.334681in}}{\pgfqpoint{3.310063in}{2.339071in}}{\pgfqpoint{3.299013in}{2.339071in}}%
\pgfpathcurveto{\pgfqpoint{3.287962in}{2.339071in}}{\pgfqpoint{3.277363in}{2.334681in}}{\pgfqpoint{3.269550in}{2.326867in}}%
\pgfpathcurveto{\pgfqpoint{3.261736in}{2.319053in}}{\pgfqpoint{3.257346in}{2.308454in}}{\pgfqpoint{3.257346in}{2.297404in}}%
\pgfpathcurveto{\pgfqpoint{3.257346in}{2.286354in}}{\pgfqpoint{3.261736in}{2.275755in}}{\pgfqpoint{3.269550in}{2.267942in}}%
\pgfpathcurveto{\pgfqpoint{3.277363in}{2.260128in}}{\pgfqpoint{3.287962in}{2.255738in}}{\pgfqpoint{3.299013in}{2.255738in}}%
\pgfpathclose%
\pgfusepath{stroke,fill}%
\end{pgfscope}%
\begin{pgfscope}%
\pgfpathrectangle{\pgfqpoint{0.600000in}{0.600000in}}{\pgfqpoint{3.900000in}{3.900000in}}%
\pgfusepath{clip}%
\pgfsetbuttcap%
\pgfsetroundjoin%
\definecolor{currentfill}{rgb}{0.121569,0.466667,0.705882}%
\pgfsetfillcolor{currentfill}%
\pgfsetlinewidth{1.003750pt}%
\definecolor{currentstroke}{rgb}{0.121569,0.466667,0.705882}%
\pgfsetstrokecolor{currentstroke}%
\pgfsetdash{}{0pt}%
\pgfpathmoveto{\pgfqpoint{2.653114in}{2.806227in}}%
\pgfpathcurveto{\pgfqpoint{2.664164in}{2.806227in}}{\pgfqpoint{2.674763in}{2.810617in}}{\pgfqpoint{2.682577in}{2.818431in}}%
\pgfpathcurveto{\pgfqpoint{2.690391in}{2.826245in}}{\pgfqpoint{2.694781in}{2.836844in}}{\pgfqpoint{2.694781in}{2.847894in}}%
\pgfpathcurveto{\pgfqpoint{2.694781in}{2.858944in}}{\pgfqpoint{2.690391in}{2.869543in}}{\pgfqpoint{2.682577in}{2.877357in}}%
\pgfpathcurveto{\pgfqpoint{2.674763in}{2.885170in}}{\pgfqpoint{2.664164in}{2.889561in}}{\pgfqpoint{2.653114in}{2.889561in}}%
\pgfpathcurveto{\pgfqpoint{2.642064in}{2.889561in}}{\pgfqpoint{2.631465in}{2.885170in}}{\pgfqpoint{2.623651in}{2.877357in}}%
\pgfpathcurveto{\pgfqpoint{2.615838in}{2.869543in}}{\pgfqpoint{2.611448in}{2.858944in}}{\pgfqpoint{2.611448in}{2.847894in}}%
\pgfpathcurveto{\pgfqpoint{2.611448in}{2.836844in}}{\pgfqpoint{2.615838in}{2.826245in}}{\pgfqpoint{2.623651in}{2.818431in}}%
\pgfpathcurveto{\pgfqpoint{2.631465in}{2.810617in}}{\pgfqpoint{2.642064in}{2.806227in}}{\pgfqpoint{2.653114in}{2.806227in}}%
\pgfpathclose%
\pgfusepath{stroke,fill}%
\end{pgfscope}%
\begin{pgfscope}%
\pgfpathrectangle{\pgfqpoint{0.600000in}{0.600000in}}{\pgfqpoint{3.900000in}{3.900000in}}%
\pgfusepath{clip}%
\pgfsetbuttcap%
\pgfsetroundjoin%
\definecolor{currentfill}{rgb}{0.121569,0.466667,0.705882}%
\pgfsetfillcolor{currentfill}%
\pgfsetlinewidth{1.003750pt}%
\definecolor{currentstroke}{rgb}{0.121569,0.466667,0.705882}%
\pgfsetstrokecolor{currentstroke}%
\pgfsetdash{}{0pt}%
\pgfpathmoveto{\pgfqpoint{2.563116in}{3.524800in}}%
\pgfpathcurveto{\pgfqpoint{2.574166in}{3.524800in}}{\pgfqpoint{2.584765in}{3.529190in}}{\pgfqpoint{2.592578in}{3.537004in}}%
\pgfpathcurveto{\pgfqpoint{2.600392in}{3.544818in}}{\pgfqpoint{2.604782in}{3.555417in}}{\pgfqpoint{2.604782in}{3.566467in}}%
\pgfpathcurveto{\pgfqpoint{2.604782in}{3.577517in}}{\pgfqpoint{2.600392in}{3.588116in}}{\pgfqpoint{2.592578in}{3.595930in}}%
\pgfpathcurveto{\pgfqpoint{2.584765in}{3.603743in}}{\pgfqpoint{2.574166in}{3.608133in}}{\pgfqpoint{2.563116in}{3.608133in}}%
\pgfpathcurveto{\pgfqpoint{2.552065in}{3.608133in}}{\pgfqpoint{2.541466in}{3.603743in}}{\pgfqpoint{2.533653in}{3.595930in}}%
\pgfpathcurveto{\pgfqpoint{2.525839in}{3.588116in}}{\pgfqpoint{2.521449in}{3.577517in}}{\pgfqpoint{2.521449in}{3.566467in}}%
\pgfpathcurveto{\pgfqpoint{2.521449in}{3.555417in}}{\pgfqpoint{2.525839in}{3.544818in}}{\pgfqpoint{2.533653in}{3.537004in}}%
\pgfpathcurveto{\pgfqpoint{2.541466in}{3.529190in}}{\pgfqpoint{2.552065in}{3.524800in}}{\pgfqpoint{2.563116in}{3.524800in}}%
\pgfpathclose%
\pgfusepath{stroke,fill}%
\end{pgfscope}%
\begin{pgfscope}%
\pgfpathrectangle{\pgfqpoint{0.600000in}{0.600000in}}{\pgfqpoint{3.900000in}{3.900000in}}%
\pgfusepath{clip}%
\pgfsetbuttcap%
\pgfsetroundjoin%
\definecolor{currentfill}{rgb}{0.121569,0.466667,0.705882}%
\pgfsetfillcolor{currentfill}%
\pgfsetlinewidth{1.003750pt}%
\definecolor{currentstroke}{rgb}{0.121569,0.466667,0.705882}%
\pgfsetstrokecolor{currentstroke}%
\pgfsetdash{}{0pt}%
\pgfpathmoveto{\pgfqpoint{1.937203in}{3.546785in}}%
\pgfpathcurveto{\pgfqpoint{1.948253in}{3.546785in}}{\pgfqpoint{1.958852in}{3.551176in}}{\pgfqpoint{1.966666in}{3.558989in}}%
\pgfpathcurveto{\pgfqpoint{1.974479in}{3.566803in}}{\pgfqpoint{1.978869in}{3.577402in}}{\pgfqpoint{1.978869in}{3.588452in}}%
\pgfpathcurveto{\pgfqpoint{1.978869in}{3.599502in}}{\pgfqpoint{1.974479in}{3.610101in}}{\pgfqpoint{1.966666in}{3.617915in}}%
\pgfpathcurveto{\pgfqpoint{1.958852in}{3.625729in}}{\pgfqpoint{1.948253in}{3.630119in}}{\pgfqpoint{1.937203in}{3.630119in}}%
\pgfpathcurveto{\pgfqpoint{1.926153in}{3.630119in}}{\pgfqpoint{1.915554in}{3.625729in}}{\pgfqpoint{1.907740in}{3.617915in}}%
\pgfpathcurveto{\pgfqpoint{1.899926in}{3.610101in}}{\pgfqpoint{1.895536in}{3.599502in}}{\pgfqpoint{1.895536in}{3.588452in}}%
\pgfpathcurveto{\pgfqpoint{1.895536in}{3.577402in}}{\pgfqpoint{1.899926in}{3.566803in}}{\pgfqpoint{1.907740in}{3.558989in}}%
\pgfpathcurveto{\pgfqpoint{1.915554in}{3.551176in}}{\pgfqpoint{1.926153in}{3.546785in}}{\pgfqpoint{1.937203in}{3.546785in}}%
\pgfpathclose%
\pgfusepath{stroke,fill}%
\end{pgfscope}%
\begin{pgfscope}%
\pgfpathrectangle{\pgfqpoint{0.600000in}{0.600000in}}{\pgfqpoint{3.900000in}{3.900000in}}%
\pgfusepath{clip}%
\pgfsetbuttcap%
\pgfsetroundjoin%
\definecolor{currentfill}{rgb}{0.121569,0.466667,0.705882}%
\pgfsetfillcolor{currentfill}%
\pgfsetlinewidth{1.003750pt}%
\definecolor{currentstroke}{rgb}{0.121569,0.466667,0.705882}%
\pgfsetstrokecolor{currentstroke}%
\pgfsetdash{}{0pt}%
\pgfpathmoveto{\pgfqpoint{3.108154in}{3.475217in}}%
\pgfpathcurveto{\pgfqpoint{3.119204in}{3.475217in}}{\pgfqpoint{3.129803in}{3.479607in}}{\pgfqpoint{3.137616in}{3.487421in}}%
\pgfpathcurveto{\pgfqpoint{3.145430in}{3.495234in}}{\pgfqpoint{3.149820in}{3.505833in}}{\pgfqpoint{3.149820in}{3.516884in}}%
\pgfpathcurveto{\pgfqpoint{3.149820in}{3.527934in}}{\pgfqpoint{3.145430in}{3.538533in}}{\pgfqpoint{3.137616in}{3.546346in}}%
\pgfpathcurveto{\pgfqpoint{3.129803in}{3.554160in}}{\pgfqpoint{3.119204in}{3.558550in}}{\pgfqpoint{3.108154in}{3.558550in}}%
\pgfpathcurveto{\pgfqpoint{3.097103in}{3.558550in}}{\pgfqpoint{3.086504in}{3.554160in}}{\pgfqpoint{3.078691in}{3.546346in}}%
\pgfpathcurveto{\pgfqpoint{3.070877in}{3.538533in}}{\pgfqpoint{3.066487in}{3.527934in}}{\pgfqpoint{3.066487in}{3.516884in}}%
\pgfpathcurveto{\pgfqpoint{3.066487in}{3.505833in}}{\pgfqpoint{3.070877in}{3.495234in}}{\pgfqpoint{3.078691in}{3.487421in}}%
\pgfpathcurveto{\pgfqpoint{3.086504in}{3.479607in}}{\pgfqpoint{3.097103in}{3.475217in}}{\pgfqpoint{3.108154in}{3.475217in}}%
\pgfpathclose%
\pgfusepath{stroke,fill}%
\end{pgfscope}%
\begin{pgfscope}%
\pgfpathrectangle{\pgfqpoint{0.600000in}{0.600000in}}{\pgfqpoint{3.900000in}{3.900000in}}%
\pgfusepath{clip}%
\pgfsetbuttcap%
\pgfsetroundjoin%
\definecolor{currentfill}{rgb}{0.121569,0.466667,0.705882}%
\pgfsetfillcolor{currentfill}%
\pgfsetlinewidth{1.003750pt}%
\definecolor{currentstroke}{rgb}{0.121569,0.466667,0.705882}%
\pgfsetstrokecolor{currentstroke}%
\pgfsetdash{}{0pt}%
\pgfpathmoveto{\pgfqpoint{2.238904in}{3.633181in}}%
\pgfpathcurveto{\pgfqpoint{2.249954in}{3.633181in}}{\pgfqpoint{2.260553in}{3.637571in}}{\pgfqpoint{2.268366in}{3.645384in}}%
\pgfpathcurveto{\pgfqpoint{2.276180in}{3.653198in}}{\pgfqpoint{2.280570in}{3.663797in}}{\pgfqpoint{2.280570in}{3.674847in}}%
\pgfpathcurveto{\pgfqpoint{2.280570in}{3.685897in}}{\pgfqpoint{2.276180in}{3.696496in}}{\pgfqpoint{2.268366in}{3.704310in}}%
\pgfpathcurveto{\pgfqpoint{2.260553in}{3.712124in}}{\pgfqpoint{2.249954in}{3.716514in}}{\pgfqpoint{2.238904in}{3.716514in}}%
\pgfpathcurveto{\pgfqpoint{2.227854in}{3.716514in}}{\pgfqpoint{2.217255in}{3.712124in}}{\pgfqpoint{2.209441in}{3.704310in}}%
\pgfpathcurveto{\pgfqpoint{2.201627in}{3.696496in}}{\pgfqpoint{2.197237in}{3.685897in}}{\pgfqpoint{2.197237in}{3.674847in}}%
\pgfpathcurveto{\pgfqpoint{2.197237in}{3.663797in}}{\pgfqpoint{2.201627in}{3.653198in}}{\pgfqpoint{2.209441in}{3.645384in}}%
\pgfpathcurveto{\pgfqpoint{2.217255in}{3.637571in}}{\pgfqpoint{2.227854in}{3.633181in}}{\pgfqpoint{2.238904in}{3.633181in}}%
\pgfpathclose%
\pgfusepath{stroke,fill}%
\end{pgfscope}%
\begin{pgfscope}%
\pgfpathrectangle{\pgfqpoint{0.600000in}{0.600000in}}{\pgfqpoint{3.900000in}{3.900000in}}%
\pgfusepath{clip}%
\pgfsetbuttcap%
\pgfsetroundjoin%
\definecolor{currentfill}{rgb}{0.121569,0.466667,0.705882}%
\pgfsetfillcolor{currentfill}%
\pgfsetlinewidth{1.003750pt}%
\definecolor{currentstroke}{rgb}{0.121569,0.466667,0.705882}%
\pgfsetstrokecolor{currentstroke}%
\pgfsetdash{}{0pt}%
\pgfpathmoveto{\pgfqpoint{1.967281in}{2.420969in}}%
\pgfpathcurveto{\pgfqpoint{1.978331in}{2.420969in}}{\pgfqpoint{1.988930in}{2.425360in}}{\pgfqpoint{1.996744in}{2.433173in}}%
\pgfpathcurveto{\pgfqpoint{2.004558in}{2.440987in}}{\pgfqpoint{2.008948in}{2.451586in}}{\pgfqpoint{2.008948in}{2.462636in}}%
\pgfpathcurveto{\pgfqpoint{2.008948in}{2.473686in}}{\pgfqpoint{2.004558in}{2.484285in}}{\pgfqpoint{1.996744in}{2.492099in}}%
\pgfpathcurveto{\pgfqpoint{1.988930in}{2.499913in}}{\pgfqpoint{1.978331in}{2.504303in}}{\pgfqpoint{1.967281in}{2.504303in}}%
\pgfpathcurveto{\pgfqpoint{1.956231in}{2.504303in}}{\pgfqpoint{1.945632in}{2.499913in}}{\pgfqpoint{1.937818in}{2.492099in}}%
\pgfpathcurveto{\pgfqpoint{1.930005in}{2.484285in}}{\pgfqpoint{1.925615in}{2.473686in}}{\pgfqpoint{1.925615in}{2.462636in}}%
\pgfpathcurveto{\pgfqpoint{1.925615in}{2.451586in}}{\pgfqpoint{1.930005in}{2.440987in}}{\pgfqpoint{1.937818in}{2.433173in}}%
\pgfpathcurveto{\pgfqpoint{1.945632in}{2.425360in}}{\pgfqpoint{1.956231in}{2.420969in}}{\pgfqpoint{1.967281in}{2.420969in}}%
\pgfpathclose%
\pgfusepath{stroke,fill}%
\end{pgfscope}%
\begin{pgfscope}%
\pgfpathrectangle{\pgfqpoint{0.600000in}{0.600000in}}{\pgfqpoint{3.900000in}{3.900000in}}%
\pgfusepath{clip}%
\pgfsetbuttcap%
\pgfsetroundjoin%
\definecolor{currentfill}{rgb}{0.121569,0.466667,0.705882}%
\pgfsetfillcolor{currentfill}%
\pgfsetlinewidth{1.003750pt}%
\definecolor{currentstroke}{rgb}{0.121569,0.466667,0.705882}%
\pgfsetstrokecolor{currentstroke}%
\pgfsetdash{}{0pt}%
\pgfpathmoveto{\pgfqpoint{2.486127in}{2.673130in}}%
\pgfpathcurveto{\pgfqpoint{2.497177in}{2.673130in}}{\pgfqpoint{2.507776in}{2.677521in}}{\pgfqpoint{2.515590in}{2.685334in}}%
\pgfpathcurveto{\pgfqpoint{2.523403in}{2.693148in}}{\pgfqpoint{2.527794in}{2.703747in}}{\pgfqpoint{2.527794in}{2.714797in}}%
\pgfpathcurveto{\pgfqpoint{2.527794in}{2.725847in}}{\pgfqpoint{2.523403in}{2.736446in}}{\pgfqpoint{2.515590in}{2.744260in}}%
\pgfpathcurveto{\pgfqpoint{2.507776in}{2.752073in}}{\pgfqpoint{2.497177in}{2.756464in}}{\pgfqpoint{2.486127in}{2.756464in}}%
\pgfpathcurveto{\pgfqpoint{2.475077in}{2.756464in}}{\pgfqpoint{2.464478in}{2.752073in}}{\pgfqpoint{2.456664in}{2.744260in}}%
\pgfpathcurveto{\pgfqpoint{2.448850in}{2.736446in}}{\pgfqpoint{2.444460in}{2.725847in}}{\pgfqpoint{2.444460in}{2.714797in}}%
\pgfpathcurveto{\pgfqpoint{2.444460in}{2.703747in}}{\pgfqpoint{2.448850in}{2.693148in}}{\pgfqpoint{2.456664in}{2.685334in}}%
\pgfpathcurveto{\pgfqpoint{2.464478in}{2.677521in}}{\pgfqpoint{2.475077in}{2.673130in}}{\pgfqpoint{2.486127in}{2.673130in}}%
\pgfpathclose%
\pgfusepath{stroke,fill}%
\end{pgfscope}%
\begin{pgfscope}%
\pgfpathrectangle{\pgfqpoint{0.600000in}{0.600000in}}{\pgfqpoint{3.900000in}{3.900000in}}%
\pgfusepath{clip}%
\pgfsetbuttcap%
\pgfsetroundjoin%
\definecolor{currentfill}{rgb}{0.121569,0.466667,0.705882}%
\pgfsetfillcolor{currentfill}%
\pgfsetlinewidth{1.003750pt}%
\definecolor{currentstroke}{rgb}{0.121569,0.466667,0.705882}%
\pgfsetstrokecolor{currentstroke}%
\pgfsetdash{}{0pt}%
\pgfpathmoveto{\pgfqpoint{1.694703in}{1.894424in}}%
\pgfpathcurveto{\pgfqpoint{1.705753in}{1.894424in}}{\pgfqpoint{1.716352in}{1.898815in}}{\pgfqpoint{1.724166in}{1.906628in}}%
\pgfpathcurveto{\pgfqpoint{1.731980in}{1.914442in}}{\pgfqpoint{1.736370in}{1.925041in}}{\pgfqpoint{1.736370in}{1.936091in}}%
\pgfpathcurveto{\pgfqpoint{1.736370in}{1.947141in}}{\pgfqpoint{1.731980in}{1.957740in}}{\pgfqpoint{1.724166in}{1.965554in}}%
\pgfpathcurveto{\pgfqpoint{1.716352in}{1.973367in}}{\pgfqpoint{1.705753in}{1.977758in}}{\pgfqpoint{1.694703in}{1.977758in}}%
\pgfpathcurveto{\pgfqpoint{1.683653in}{1.977758in}}{\pgfqpoint{1.673054in}{1.973367in}}{\pgfqpoint{1.665240in}{1.965554in}}%
\pgfpathcurveto{\pgfqpoint{1.657427in}{1.957740in}}{\pgfqpoint{1.653036in}{1.947141in}}{\pgfqpoint{1.653036in}{1.936091in}}%
\pgfpathcurveto{\pgfqpoint{1.653036in}{1.925041in}}{\pgfqpoint{1.657427in}{1.914442in}}{\pgfqpoint{1.665240in}{1.906628in}}%
\pgfpathcurveto{\pgfqpoint{1.673054in}{1.898815in}}{\pgfqpoint{1.683653in}{1.894424in}}{\pgfqpoint{1.694703in}{1.894424in}}%
\pgfpathclose%
\pgfusepath{stroke,fill}%
\end{pgfscope}%
\begin{pgfscope}%
\pgfpathrectangle{\pgfqpoint{0.600000in}{0.600000in}}{\pgfqpoint{3.900000in}{3.900000in}}%
\pgfusepath{clip}%
\pgfsetbuttcap%
\pgfsetroundjoin%
\definecolor{currentfill}{rgb}{0.121569,0.466667,0.705882}%
\pgfsetfillcolor{currentfill}%
\pgfsetlinewidth{1.003750pt}%
\definecolor{currentstroke}{rgb}{0.121569,0.466667,0.705882}%
\pgfsetstrokecolor{currentstroke}%
\pgfsetdash{}{0pt}%
\pgfpathmoveto{\pgfqpoint{2.690167in}{2.530306in}}%
\pgfpathcurveto{\pgfqpoint{2.701217in}{2.530306in}}{\pgfqpoint{2.711816in}{2.534697in}}{\pgfqpoint{2.719630in}{2.542510in}}%
\pgfpathcurveto{\pgfqpoint{2.727443in}{2.550324in}}{\pgfqpoint{2.731834in}{2.560923in}}{\pgfqpoint{2.731834in}{2.571973in}}%
\pgfpathcurveto{\pgfqpoint{2.731834in}{2.583023in}}{\pgfqpoint{2.727443in}{2.593622in}}{\pgfqpoint{2.719630in}{2.601436in}}%
\pgfpathcurveto{\pgfqpoint{2.711816in}{2.609249in}}{\pgfqpoint{2.701217in}{2.613640in}}{\pgfqpoint{2.690167in}{2.613640in}}%
\pgfpathcurveto{\pgfqpoint{2.679117in}{2.613640in}}{\pgfqpoint{2.668518in}{2.609249in}}{\pgfqpoint{2.660704in}{2.601436in}}%
\pgfpathcurveto{\pgfqpoint{2.652891in}{2.593622in}}{\pgfqpoint{2.648500in}{2.583023in}}{\pgfqpoint{2.648500in}{2.571973in}}%
\pgfpathcurveto{\pgfqpoint{2.648500in}{2.560923in}}{\pgfqpoint{2.652891in}{2.550324in}}{\pgfqpoint{2.660704in}{2.542510in}}%
\pgfpathcurveto{\pgfqpoint{2.668518in}{2.534697in}}{\pgfqpoint{2.679117in}{2.530306in}}{\pgfqpoint{2.690167in}{2.530306in}}%
\pgfpathclose%
\pgfusepath{stroke,fill}%
\end{pgfscope}%
\begin{pgfscope}%
\pgfpathrectangle{\pgfqpoint{0.600000in}{0.600000in}}{\pgfqpoint{3.900000in}{3.900000in}}%
\pgfusepath{clip}%
\pgfsetbuttcap%
\pgfsetroundjoin%
\definecolor{currentfill}{rgb}{0.121569,0.466667,0.705882}%
\pgfsetfillcolor{currentfill}%
\pgfsetlinewidth{1.003750pt}%
\definecolor{currentstroke}{rgb}{0.121569,0.466667,0.705882}%
\pgfsetstrokecolor{currentstroke}%
\pgfsetdash{}{0pt}%
\pgfpathmoveto{\pgfqpoint{2.905630in}{2.639801in}}%
\pgfpathcurveto{\pgfqpoint{2.916680in}{2.639801in}}{\pgfqpoint{2.927279in}{2.644191in}}{\pgfqpoint{2.935092in}{2.652004in}}%
\pgfpathcurveto{\pgfqpoint{2.942906in}{2.659818in}}{\pgfqpoint{2.947296in}{2.670417in}}{\pgfqpoint{2.947296in}{2.681467in}}%
\pgfpathcurveto{\pgfqpoint{2.947296in}{2.692517in}}{\pgfqpoint{2.942906in}{2.703116in}}{\pgfqpoint{2.935092in}{2.710930in}}%
\pgfpathcurveto{\pgfqpoint{2.927279in}{2.718744in}}{\pgfqpoint{2.916680in}{2.723134in}}{\pgfqpoint{2.905630in}{2.723134in}}%
\pgfpathcurveto{\pgfqpoint{2.894579in}{2.723134in}}{\pgfqpoint{2.883980in}{2.718744in}}{\pgfqpoint{2.876167in}{2.710930in}}%
\pgfpathcurveto{\pgfqpoint{2.868353in}{2.703116in}}{\pgfqpoint{2.863963in}{2.692517in}}{\pgfqpoint{2.863963in}{2.681467in}}%
\pgfpathcurveto{\pgfqpoint{2.863963in}{2.670417in}}{\pgfqpoint{2.868353in}{2.659818in}}{\pgfqpoint{2.876167in}{2.652004in}}%
\pgfpathcurveto{\pgfqpoint{2.883980in}{2.644191in}}{\pgfqpoint{2.894579in}{2.639801in}}{\pgfqpoint{2.905630in}{2.639801in}}%
\pgfpathclose%
\pgfusepath{stroke,fill}%
\end{pgfscope}%
\begin{pgfscope}%
\pgfpathrectangle{\pgfqpoint{0.600000in}{0.600000in}}{\pgfqpoint{3.900000in}{3.900000in}}%
\pgfusepath{clip}%
\pgfsetbuttcap%
\pgfsetroundjoin%
\definecolor{currentfill}{rgb}{0.121569,0.466667,0.705882}%
\pgfsetfillcolor{currentfill}%
\pgfsetlinewidth{1.003750pt}%
\definecolor{currentstroke}{rgb}{0.121569,0.466667,0.705882}%
\pgfsetstrokecolor{currentstroke}%
\pgfsetdash{}{0pt}%
\pgfpathmoveto{\pgfqpoint{3.233143in}{2.667842in}}%
\pgfpathcurveto{\pgfqpoint{3.244193in}{2.667842in}}{\pgfqpoint{3.254792in}{2.672232in}}{\pgfqpoint{3.262606in}{2.680046in}}%
\pgfpathcurveto{\pgfqpoint{3.270420in}{2.687860in}}{\pgfqpoint{3.274810in}{2.698459in}}{\pgfqpoint{3.274810in}{2.709509in}}%
\pgfpathcurveto{\pgfqpoint{3.274810in}{2.720559in}}{\pgfqpoint{3.270420in}{2.731158in}}{\pgfqpoint{3.262606in}{2.738972in}}%
\pgfpathcurveto{\pgfqpoint{3.254792in}{2.746785in}}{\pgfqpoint{3.244193in}{2.751175in}}{\pgfqpoint{3.233143in}{2.751175in}}%
\pgfpathcurveto{\pgfqpoint{3.222093in}{2.751175in}}{\pgfqpoint{3.211494in}{2.746785in}}{\pgfqpoint{3.203680in}{2.738972in}}%
\pgfpathcurveto{\pgfqpoint{3.195867in}{2.731158in}}{\pgfqpoint{3.191476in}{2.720559in}}{\pgfqpoint{3.191476in}{2.709509in}}%
\pgfpathcurveto{\pgfqpoint{3.191476in}{2.698459in}}{\pgfqpoint{3.195867in}{2.687860in}}{\pgfqpoint{3.203680in}{2.680046in}}%
\pgfpathcurveto{\pgfqpoint{3.211494in}{2.672232in}}{\pgfqpoint{3.222093in}{2.667842in}}{\pgfqpoint{3.233143in}{2.667842in}}%
\pgfpathclose%
\pgfusepath{stroke,fill}%
\end{pgfscope}%
\begin{pgfscope}%
\pgfpathrectangle{\pgfqpoint{0.600000in}{0.600000in}}{\pgfqpoint{3.900000in}{3.900000in}}%
\pgfusepath{clip}%
\pgfsetbuttcap%
\pgfsetroundjoin%
\definecolor{currentfill}{rgb}{0.121569,0.466667,0.705882}%
\pgfsetfillcolor{currentfill}%
\pgfsetlinewidth{1.003750pt}%
\definecolor{currentstroke}{rgb}{0.121569,0.466667,0.705882}%
\pgfsetstrokecolor{currentstroke}%
\pgfsetdash{}{0pt}%
\pgfpathmoveto{\pgfqpoint{3.564712in}{1.702185in}}%
\pgfpathcurveto{\pgfqpoint{3.575762in}{1.702185in}}{\pgfqpoint{3.586361in}{1.706575in}}{\pgfqpoint{3.594175in}{1.714389in}}%
\pgfpathcurveto{\pgfqpoint{3.601989in}{1.722202in}}{\pgfqpoint{3.606379in}{1.732801in}}{\pgfqpoint{3.606379in}{1.743851in}}%
\pgfpathcurveto{\pgfqpoint{3.606379in}{1.754901in}}{\pgfqpoint{3.601989in}{1.765501in}}{\pgfqpoint{3.594175in}{1.773314in}}%
\pgfpathcurveto{\pgfqpoint{3.586361in}{1.781128in}}{\pgfqpoint{3.575762in}{1.785518in}}{\pgfqpoint{3.564712in}{1.785518in}}%
\pgfpathcurveto{\pgfqpoint{3.553662in}{1.785518in}}{\pgfqpoint{3.543063in}{1.781128in}}{\pgfqpoint{3.535249in}{1.773314in}}%
\pgfpathcurveto{\pgfqpoint{3.527436in}{1.765501in}}{\pgfqpoint{3.523046in}{1.754901in}}{\pgfqpoint{3.523046in}{1.743851in}}%
\pgfpathcurveto{\pgfqpoint{3.523046in}{1.732801in}}{\pgfqpoint{3.527436in}{1.722202in}}{\pgfqpoint{3.535249in}{1.714389in}}%
\pgfpathcurveto{\pgfqpoint{3.543063in}{1.706575in}}{\pgfqpoint{3.553662in}{1.702185in}}{\pgfqpoint{3.564712in}{1.702185in}}%
\pgfpathclose%
\pgfusepath{stroke,fill}%
\end{pgfscope}%
\begin{pgfscope}%
\pgfpathrectangle{\pgfqpoint{0.600000in}{0.600000in}}{\pgfqpoint{3.900000in}{3.900000in}}%
\pgfusepath{clip}%
\pgfsetbuttcap%
\pgfsetroundjoin%
\definecolor{currentfill}{rgb}{0.121569,0.466667,0.705882}%
\pgfsetfillcolor{currentfill}%
\pgfsetlinewidth{1.003750pt}%
\definecolor{currentstroke}{rgb}{0.121569,0.466667,0.705882}%
\pgfsetstrokecolor{currentstroke}%
\pgfsetdash{}{0pt}%
\pgfpathmoveto{\pgfqpoint{2.447684in}{2.856117in}}%
\pgfpathcurveto{\pgfqpoint{2.458734in}{2.856117in}}{\pgfqpoint{2.469333in}{2.860508in}}{\pgfqpoint{2.477147in}{2.868321in}}%
\pgfpathcurveto{\pgfqpoint{2.484961in}{2.876135in}}{\pgfqpoint{2.489351in}{2.886734in}}{\pgfqpoint{2.489351in}{2.897784in}}%
\pgfpathcurveto{\pgfqpoint{2.489351in}{2.908834in}}{\pgfqpoint{2.484961in}{2.919433in}}{\pgfqpoint{2.477147in}{2.927247in}}%
\pgfpathcurveto{\pgfqpoint{2.469333in}{2.935060in}}{\pgfqpoint{2.458734in}{2.939451in}}{\pgfqpoint{2.447684in}{2.939451in}}%
\pgfpathcurveto{\pgfqpoint{2.436634in}{2.939451in}}{\pgfqpoint{2.426035in}{2.935060in}}{\pgfqpoint{2.418221in}{2.927247in}}%
\pgfpathcurveto{\pgfqpoint{2.410408in}{2.919433in}}{\pgfqpoint{2.406017in}{2.908834in}}{\pgfqpoint{2.406017in}{2.897784in}}%
\pgfpathcurveto{\pgfqpoint{2.406017in}{2.886734in}}{\pgfqpoint{2.410408in}{2.876135in}}{\pgfqpoint{2.418221in}{2.868321in}}%
\pgfpathcurveto{\pgfqpoint{2.426035in}{2.860508in}}{\pgfqpoint{2.436634in}{2.856117in}}{\pgfqpoint{2.447684in}{2.856117in}}%
\pgfpathclose%
\pgfusepath{stroke,fill}%
\end{pgfscope}%
\begin{pgfscope}%
\pgfpathrectangle{\pgfqpoint{0.600000in}{0.600000in}}{\pgfqpoint{3.900000in}{3.900000in}}%
\pgfusepath{clip}%
\pgfsetbuttcap%
\pgfsetroundjoin%
\definecolor{currentfill}{rgb}{0.121569,0.466667,0.705882}%
\pgfsetfillcolor{currentfill}%
\pgfsetlinewidth{1.003750pt}%
\definecolor{currentstroke}{rgb}{0.121569,0.466667,0.705882}%
\pgfsetstrokecolor{currentstroke}%
\pgfsetdash{}{0pt}%
\pgfpathmoveto{\pgfqpoint{3.362016in}{2.839424in}}%
\pgfpathcurveto{\pgfqpoint{3.373066in}{2.839424in}}{\pgfqpoint{3.383665in}{2.843814in}}{\pgfqpoint{3.391478in}{2.851628in}}%
\pgfpathcurveto{\pgfqpoint{3.399292in}{2.859441in}}{\pgfqpoint{3.403682in}{2.870040in}}{\pgfqpoint{3.403682in}{2.881090in}}%
\pgfpathcurveto{\pgfqpoint{3.403682in}{2.892141in}}{\pgfqpoint{3.399292in}{2.902740in}}{\pgfqpoint{3.391478in}{2.910553in}}%
\pgfpathcurveto{\pgfqpoint{3.383665in}{2.918367in}}{\pgfqpoint{3.373066in}{2.922757in}}{\pgfqpoint{3.362016in}{2.922757in}}%
\pgfpathcurveto{\pgfqpoint{3.350965in}{2.922757in}}{\pgfqpoint{3.340366in}{2.918367in}}{\pgfqpoint{3.332553in}{2.910553in}}%
\pgfpathcurveto{\pgfqpoint{3.324739in}{2.902740in}}{\pgfqpoint{3.320349in}{2.892141in}}{\pgfqpoint{3.320349in}{2.881090in}}%
\pgfpathcurveto{\pgfqpoint{3.320349in}{2.870040in}}{\pgfqpoint{3.324739in}{2.859441in}}{\pgfqpoint{3.332553in}{2.851628in}}%
\pgfpathcurveto{\pgfqpoint{3.340366in}{2.843814in}}{\pgfqpoint{3.350965in}{2.839424in}}{\pgfqpoint{3.362016in}{2.839424in}}%
\pgfpathclose%
\pgfusepath{stroke,fill}%
\end{pgfscope}%
\begin{pgfscope}%
\pgfpathrectangle{\pgfqpoint{0.600000in}{0.600000in}}{\pgfqpoint{3.900000in}{3.900000in}}%
\pgfusepath{clip}%
\pgfsetbuttcap%
\pgfsetroundjoin%
\definecolor{currentfill}{rgb}{0.121569,0.466667,0.705882}%
\pgfsetfillcolor{currentfill}%
\pgfsetlinewidth{1.003750pt}%
\definecolor{currentstroke}{rgb}{0.121569,0.466667,0.705882}%
\pgfsetstrokecolor{currentstroke}%
\pgfsetdash{}{0pt}%
\pgfpathmoveto{\pgfqpoint{2.043934in}{3.478949in}}%
\pgfpathcurveto{\pgfqpoint{2.054984in}{3.478949in}}{\pgfqpoint{2.065583in}{3.483340in}}{\pgfqpoint{2.073397in}{3.491153in}}%
\pgfpathcurveto{\pgfqpoint{2.081210in}{3.498967in}}{\pgfqpoint{2.085601in}{3.509566in}}{\pgfqpoint{2.085601in}{3.520616in}}%
\pgfpathcurveto{\pgfqpoint{2.085601in}{3.531666in}}{\pgfqpoint{2.081210in}{3.542265in}}{\pgfqpoint{2.073397in}{3.550079in}}%
\pgfpathcurveto{\pgfqpoint{2.065583in}{3.557893in}}{\pgfqpoint{2.054984in}{3.562283in}}{\pgfqpoint{2.043934in}{3.562283in}}%
\pgfpathcurveto{\pgfqpoint{2.032884in}{3.562283in}}{\pgfqpoint{2.022285in}{3.557893in}}{\pgfqpoint{2.014471in}{3.550079in}}%
\pgfpathcurveto{\pgfqpoint{2.006658in}{3.542265in}}{\pgfqpoint{2.002267in}{3.531666in}}{\pgfqpoint{2.002267in}{3.520616in}}%
\pgfpathcurveto{\pgfqpoint{2.002267in}{3.509566in}}{\pgfqpoint{2.006658in}{3.498967in}}{\pgfqpoint{2.014471in}{3.491153in}}%
\pgfpathcurveto{\pgfqpoint{2.022285in}{3.483340in}}{\pgfqpoint{2.032884in}{3.478949in}}{\pgfqpoint{2.043934in}{3.478949in}}%
\pgfpathclose%
\pgfusepath{stroke,fill}%
\end{pgfscope}%
\begin{pgfscope}%
\pgfpathrectangle{\pgfqpoint{0.600000in}{0.600000in}}{\pgfqpoint{3.900000in}{3.900000in}}%
\pgfusepath{clip}%
\pgfsetbuttcap%
\pgfsetroundjoin%
\definecolor{currentfill}{rgb}{0.121569,0.466667,0.705882}%
\pgfsetfillcolor{currentfill}%
\pgfsetlinewidth{1.003750pt}%
\definecolor{currentstroke}{rgb}{0.121569,0.466667,0.705882}%
\pgfsetstrokecolor{currentstroke}%
\pgfsetdash{}{0pt}%
\pgfpathmoveto{\pgfqpoint{2.591965in}{2.435040in}}%
\pgfpathcurveto{\pgfqpoint{2.603015in}{2.435040in}}{\pgfqpoint{2.613614in}{2.439430in}}{\pgfqpoint{2.621428in}{2.447243in}}%
\pgfpathcurveto{\pgfqpoint{2.629242in}{2.455057in}}{\pgfqpoint{2.633632in}{2.465656in}}{\pgfqpoint{2.633632in}{2.476706in}}%
\pgfpathcurveto{\pgfqpoint{2.633632in}{2.487756in}}{\pgfqpoint{2.629242in}{2.498355in}}{\pgfqpoint{2.621428in}{2.506169in}}%
\pgfpathcurveto{\pgfqpoint{2.613614in}{2.513983in}}{\pgfqpoint{2.603015in}{2.518373in}}{\pgfqpoint{2.591965in}{2.518373in}}%
\pgfpathcurveto{\pgfqpoint{2.580915in}{2.518373in}}{\pgfqpoint{2.570316in}{2.513983in}}{\pgfqpoint{2.562502in}{2.506169in}}%
\pgfpathcurveto{\pgfqpoint{2.554689in}{2.498355in}}{\pgfqpoint{2.550298in}{2.487756in}}{\pgfqpoint{2.550298in}{2.476706in}}%
\pgfpathcurveto{\pgfqpoint{2.550298in}{2.465656in}}{\pgfqpoint{2.554689in}{2.455057in}}{\pgfqpoint{2.562502in}{2.447243in}}%
\pgfpathcurveto{\pgfqpoint{2.570316in}{2.439430in}}{\pgfqpoint{2.580915in}{2.435040in}}{\pgfqpoint{2.591965in}{2.435040in}}%
\pgfpathclose%
\pgfusepath{stroke,fill}%
\end{pgfscope}%
\begin{pgfscope}%
\pgfpathrectangle{\pgfqpoint{0.600000in}{0.600000in}}{\pgfqpoint{3.900000in}{3.900000in}}%
\pgfusepath{clip}%
\pgfsetbuttcap%
\pgfsetroundjoin%
\definecolor{currentfill}{rgb}{0.121569,0.466667,0.705882}%
\pgfsetfillcolor{currentfill}%
\pgfsetlinewidth{1.003750pt}%
\definecolor{currentstroke}{rgb}{0.121569,0.466667,0.705882}%
\pgfsetstrokecolor{currentstroke}%
\pgfsetdash{}{0pt}%
\pgfpathmoveto{\pgfqpoint{2.738737in}{2.157081in}}%
\pgfpathcurveto{\pgfqpoint{2.749787in}{2.157081in}}{\pgfqpoint{2.760386in}{2.161471in}}{\pgfqpoint{2.768200in}{2.169285in}}%
\pgfpathcurveto{\pgfqpoint{2.776014in}{2.177098in}}{\pgfqpoint{2.780404in}{2.187697in}}{\pgfqpoint{2.780404in}{2.198748in}}%
\pgfpathcurveto{\pgfqpoint{2.780404in}{2.209798in}}{\pgfqpoint{2.776014in}{2.220397in}}{\pgfqpoint{2.768200in}{2.228210in}}%
\pgfpathcurveto{\pgfqpoint{2.760386in}{2.236024in}}{\pgfqpoint{2.749787in}{2.240414in}}{\pgfqpoint{2.738737in}{2.240414in}}%
\pgfpathcurveto{\pgfqpoint{2.727687in}{2.240414in}}{\pgfqpoint{2.717088in}{2.236024in}}{\pgfqpoint{2.709274in}{2.228210in}}%
\pgfpathcurveto{\pgfqpoint{2.701461in}{2.220397in}}{\pgfqpoint{2.697070in}{2.209798in}}{\pgfqpoint{2.697070in}{2.198748in}}%
\pgfpathcurveto{\pgfqpoint{2.697070in}{2.187697in}}{\pgfqpoint{2.701461in}{2.177098in}}{\pgfqpoint{2.709274in}{2.169285in}}%
\pgfpathcurveto{\pgfqpoint{2.717088in}{2.161471in}}{\pgfqpoint{2.727687in}{2.157081in}}{\pgfqpoint{2.738737in}{2.157081in}}%
\pgfpathclose%
\pgfusepath{stroke,fill}%
\end{pgfscope}%
\begin{pgfscope}%
\pgfpathrectangle{\pgfqpoint{0.600000in}{0.600000in}}{\pgfqpoint{3.900000in}{3.900000in}}%
\pgfusepath{clip}%
\pgfsetbuttcap%
\pgfsetroundjoin%
\definecolor{currentfill}{rgb}{0.121569,0.466667,0.705882}%
\pgfsetfillcolor{currentfill}%
\pgfsetlinewidth{1.003750pt}%
\definecolor{currentstroke}{rgb}{0.121569,0.466667,0.705882}%
\pgfsetstrokecolor{currentstroke}%
\pgfsetdash{}{0pt}%
\pgfpathmoveto{\pgfqpoint{2.440567in}{2.271448in}}%
\pgfpathcurveto{\pgfqpoint{2.451617in}{2.271448in}}{\pgfqpoint{2.462216in}{2.275838in}}{\pgfqpoint{2.470030in}{2.283652in}}%
\pgfpathcurveto{\pgfqpoint{2.477843in}{2.291466in}}{\pgfqpoint{2.482234in}{2.302065in}}{\pgfqpoint{2.482234in}{2.313115in}}%
\pgfpathcurveto{\pgfqpoint{2.482234in}{2.324165in}}{\pgfqpoint{2.477843in}{2.334764in}}{\pgfqpoint{2.470030in}{2.342578in}}%
\pgfpathcurveto{\pgfqpoint{2.462216in}{2.350391in}}{\pgfqpoint{2.451617in}{2.354782in}}{\pgfqpoint{2.440567in}{2.354782in}}%
\pgfpathcurveto{\pgfqpoint{2.429517in}{2.354782in}}{\pgfqpoint{2.418918in}{2.350391in}}{\pgfqpoint{2.411104in}{2.342578in}}%
\pgfpathcurveto{\pgfqpoint{2.403291in}{2.334764in}}{\pgfqpoint{2.398900in}{2.324165in}}{\pgfqpoint{2.398900in}{2.313115in}}%
\pgfpathcurveto{\pgfqpoint{2.398900in}{2.302065in}}{\pgfqpoint{2.403291in}{2.291466in}}{\pgfqpoint{2.411104in}{2.283652in}}%
\pgfpathcurveto{\pgfqpoint{2.418918in}{2.275838in}}{\pgfqpoint{2.429517in}{2.271448in}}{\pgfqpoint{2.440567in}{2.271448in}}%
\pgfpathclose%
\pgfusepath{stroke,fill}%
\end{pgfscope}%
\begin{pgfscope}%
\pgfpathrectangle{\pgfqpoint{0.600000in}{0.600000in}}{\pgfqpoint{3.900000in}{3.900000in}}%
\pgfusepath{clip}%
\pgfsetbuttcap%
\pgfsetroundjoin%
\definecolor{currentfill}{rgb}{0.121569,0.466667,0.705882}%
\pgfsetfillcolor{currentfill}%
\pgfsetlinewidth{1.003750pt}%
\definecolor{currentstroke}{rgb}{0.121569,0.466667,0.705882}%
\pgfsetstrokecolor{currentstroke}%
\pgfsetdash{}{0pt}%
\pgfpathmoveto{\pgfqpoint{3.281553in}{1.982453in}}%
\pgfpathcurveto{\pgfqpoint{3.292603in}{1.982453in}}{\pgfqpoint{3.303202in}{1.986843in}}{\pgfqpoint{3.311016in}{1.994657in}}%
\pgfpathcurveto{\pgfqpoint{3.318829in}{2.002471in}}{\pgfqpoint{3.323220in}{2.013070in}}{\pgfqpoint{3.323220in}{2.024120in}}%
\pgfpathcurveto{\pgfqpoint{3.323220in}{2.035170in}}{\pgfqpoint{3.318829in}{2.045769in}}{\pgfqpoint{3.311016in}{2.053583in}}%
\pgfpathcurveto{\pgfqpoint{3.303202in}{2.061396in}}{\pgfqpoint{3.292603in}{2.065786in}}{\pgfqpoint{3.281553in}{2.065786in}}%
\pgfpathcurveto{\pgfqpoint{3.270503in}{2.065786in}}{\pgfqpoint{3.259904in}{2.061396in}}{\pgfqpoint{3.252090in}{2.053583in}}%
\pgfpathcurveto{\pgfqpoint{3.244276in}{2.045769in}}{\pgfqpoint{3.239886in}{2.035170in}}{\pgfqpoint{3.239886in}{2.024120in}}%
\pgfpathcurveto{\pgfqpoint{3.239886in}{2.013070in}}{\pgfqpoint{3.244276in}{2.002471in}}{\pgfqpoint{3.252090in}{1.994657in}}%
\pgfpathcurveto{\pgfqpoint{3.259904in}{1.986843in}}{\pgfqpoint{3.270503in}{1.982453in}}{\pgfqpoint{3.281553in}{1.982453in}}%
\pgfpathclose%
\pgfusepath{stroke,fill}%
\end{pgfscope}%
\begin{pgfscope}%
\pgfpathrectangle{\pgfqpoint{0.600000in}{0.600000in}}{\pgfqpoint{3.900000in}{3.900000in}}%
\pgfusepath{clip}%
\pgfsetbuttcap%
\pgfsetroundjoin%
\definecolor{currentfill}{rgb}{0.121569,0.466667,0.705882}%
\pgfsetfillcolor{currentfill}%
\pgfsetlinewidth{1.003750pt}%
\definecolor{currentstroke}{rgb}{0.121569,0.466667,0.705882}%
\pgfsetstrokecolor{currentstroke}%
\pgfsetdash{}{0pt}%
\pgfpathmoveto{\pgfqpoint{2.869838in}{2.082596in}}%
\pgfpathcurveto{\pgfqpoint{2.880888in}{2.082596in}}{\pgfqpoint{2.891487in}{2.086987in}}{\pgfqpoint{2.899300in}{2.094800in}}%
\pgfpathcurveto{\pgfqpoint{2.907114in}{2.102614in}}{\pgfqpoint{2.911504in}{2.113213in}}{\pgfqpoint{2.911504in}{2.124263in}}%
\pgfpathcurveto{\pgfqpoint{2.911504in}{2.135313in}}{\pgfqpoint{2.907114in}{2.145912in}}{\pgfqpoint{2.899300in}{2.153726in}}%
\pgfpathcurveto{\pgfqpoint{2.891487in}{2.161539in}}{\pgfqpoint{2.880888in}{2.165930in}}{\pgfqpoint{2.869838in}{2.165930in}}%
\pgfpathcurveto{\pgfqpoint{2.858787in}{2.165930in}}{\pgfqpoint{2.848188in}{2.161539in}}{\pgfqpoint{2.840375in}{2.153726in}}%
\pgfpathcurveto{\pgfqpoint{2.832561in}{2.145912in}}{\pgfqpoint{2.828171in}{2.135313in}}{\pgfqpoint{2.828171in}{2.124263in}}%
\pgfpathcurveto{\pgfqpoint{2.828171in}{2.113213in}}{\pgfqpoint{2.832561in}{2.102614in}}{\pgfqpoint{2.840375in}{2.094800in}}%
\pgfpathcurveto{\pgfqpoint{2.848188in}{2.086987in}}{\pgfqpoint{2.858787in}{2.082596in}}{\pgfqpoint{2.869838in}{2.082596in}}%
\pgfpathclose%
\pgfusepath{stroke,fill}%
\end{pgfscope}%
\begin{pgfscope}%
\pgfpathrectangle{\pgfqpoint{0.600000in}{0.600000in}}{\pgfqpoint{3.900000in}{3.900000in}}%
\pgfusepath{clip}%
\pgfsetbuttcap%
\pgfsetroundjoin%
\definecolor{currentfill}{rgb}{0.121569,0.466667,0.705882}%
\pgfsetfillcolor{currentfill}%
\pgfsetlinewidth{1.003750pt}%
\definecolor{currentstroke}{rgb}{0.121569,0.466667,0.705882}%
\pgfsetstrokecolor{currentstroke}%
\pgfsetdash{}{0pt}%
\pgfpathmoveto{\pgfqpoint{2.415319in}{2.376460in}}%
\pgfpathcurveto{\pgfqpoint{2.426370in}{2.376460in}}{\pgfqpoint{2.436969in}{2.380851in}}{\pgfqpoint{2.444782in}{2.388664in}}%
\pgfpathcurveto{\pgfqpoint{2.452596in}{2.396478in}}{\pgfqpoint{2.456986in}{2.407077in}}{\pgfqpoint{2.456986in}{2.418127in}}%
\pgfpathcurveto{\pgfqpoint{2.456986in}{2.429177in}}{\pgfqpoint{2.452596in}{2.439776in}}{\pgfqpoint{2.444782in}{2.447590in}}%
\pgfpathcurveto{\pgfqpoint{2.436969in}{2.455403in}}{\pgfqpoint{2.426370in}{2.459794in}}{\pgfqpoint{2.415319in}{2.459794in}}%
\pgfpathcurveto{\pgfqpoint{2.404269in}{2.459794in}}{\pgfqpoint{2.393670in}{2.455403in}}{\pgfqpoint{2.385857in}{2.447590in}}%
\pgfpathcurveto{\pgfqpoint{2.378043in}{2.439776in}}{\pgfqpoint{2.373653in}{2.429177in}}{\pgfqpoint{2.373653in}{2.418127in}}%
\pgfpathcurveto{\pgfqpoint{2.373653in}{2.407077in}}{\pgfqpoint{2.378043in}{2.396478in}}{\pgfqpoint{2.385857in}{2.388664in}}%
\pgfpathcurveto{\pgfqpoint{2.393670in}{2.380851in}}{\pgfqpoint{2.404269in}{2.376460in}}{\pgfqpoint{2.415319in}{2.376460in}}%
\pgfpathclose%
\pgfusepath{stroke,fill}%
\end{pgfscope}%
\begin{pgfscope}%
\pgfpathrectangle{\pgfqpoint{0.600000in}{0.600000in}}{\pgfqpoint{3.900000in}{3.900000in}}%
\pgfusepath{clip}%
\pgfsetbuttcap%
\pgfsetroundjoin%
\definecolor{currentfill}{rgb}{0.121569,0.466667,0.705882}%
\pgfsetfillcolor{currentfill}%
\pgfsetlinewidth{1.003750pt}%
\definecolor{currentstroke}{rgb}{0.121569,0.466667,0.705882}%
\pgfsetstrokecolor{currentstroke}%
\pgfsetdash{}{0pt}%
\pgfpathmoveto{\pgfqpoint{3.189316in}{2.606351in}}%
\pgfpathcurveto{\pgfqpoint{3.200366in}{2.606351in}}{\pgfqpoint{3.210965in}{2.610742in}}{\pgfqpoint{3.218779in}{2.618555in}}%
\pgfpathcurveto{\pgfqpoint{3.226593in}{2.626369in}}{\pgfqpoint{3.230983in}{2.636968in}}{\pgfqpoint{3.230983in}{2.648018in}}%
\pgfpathcurveto{\pgfqpoint{3.230983in}{2.659068in}}{\pgfqpoint{3.226593in}{2.669667in}}{\pgfqpoint{3.218779in}{2.677481in}}%
\pgfpathcurveto{\pgfqpoint{3.210965in}{2.685294in}}{\pgfqpoint{3.200366in}{2.689685in}}{\pgfqpoint{3.189316in}{2.689685in}}%
\pgfpathcurveto{\pgfqpoint{3.178266in}{2.689685in}}{\pgfqpoint{3.167667in}{2.685294in}}{\pgfqpoint{3.159853in}{2.677481in}}%
\pgfpathcurveto{\pgfqpoint{3.152040in}{2.669667in}}{\pgfqpoint{3.147650in}{2.659068in}}{\pgfqpoint{3.147650in}{2.648018in}}%
\pgfpathcurveto{\pgfqpoint{3.147650in}{2.636968in}}{\pgfqpoint{3.152040in}{2.626369in}}{\pgfqpoint{3.159853in}{2.618555in}}%
\pgfpathcurveto{\pgfqpoint{3.167667in}{2.610742in}}{\pgfqpoint{3.178266in}{2.606351in}}{\pgfqpoint{3.189316in}{2.606351in}}%
\pgfpathclose%
\pgfusepath{stroke,fill}%
\end{pgfscope}%
\begin{pgfscope}%
\pgfpathrectangle{\pgfqpoint{0.600000in}{0.600000in}}{\pgfqpoint{3.900000in}{3.900000in}}%
\pgfusepath{clip}%
\pgfsetbuttcap%
\pgfsetroundjoin%
\definecolor{currentfill}{rgb}{0.121569,0.466667,0.705882}%
\pgfsetfillcolor{currentfill}%
\pgfsetlinewidth{1.003750pt}%
\definecolor{currentstroke}{rgb}{0.121569,0.466667,0.705882}%
\pgfsetstrokecolor{currentstroke}%
\pgfsetdash{}{0pt}%
\pgfpathmoveto{\pgfqpoint{1.538070in}{2.661826in}}%
\pgfpathcurveto{\pgfqpoint{1.549120in}{2.661826in}}{\pgfqpoint{1.559719in}{2.666217in}}{\pgfqpoint{1.567533in}{2.674030in}}%
\pgfpathcurveto{\pgfqpoint{1.575346in}{2.681844in}}{\pgfqpoint{1.579737in}{2.692443in}}{\pgfqpoint{1.579737in}{2.703493in}}%
\pgfpathcurveto{\pgfqpoint{1.579737in}{2.714543in}}{\pgfqpoint{1.575346in}{2.725142in}}{\pgfqpoint{1.567533in}{2.732956in}}%
\pgfpathcurveto{\pgfqpoint{1.559719in}{2.740769in}}{\pgfqpoint{1.549120in}{2.745160in}}{\pgfqpoint{1.538070in}{2.745160in}}%
\pgfpathcurveto{\pgfqpoint{1.527020in}{2.745160in}}{\pgfqpoint{1.516421in}{2.740769in}}{\pgfqpoint{1.508607in}{2.732956in}}%
\pgfpathcurveto{\pgfqpoint{1.500793in}{2.725142in}}{\pgfqpoint{1.496403in}{2.714543in}}{\pgfqpoint{1.496403in}{2.703493in}}%
\pgfpathcurveto{\pgfqpoint{1.496403in}{2.692443in}}{\pgfqpoint{1.500793in}{2.681844in}}{\pgfqpoint{1.508607in}{2.674030in}}%
\pgfpathcurveto{\pgfqpoint{1.516421in}{2.666217in}}{\pgfqpoint{1.527020in}{2.661826in}}{\pgfqpoint{1.538070in}{2.661826in}}%
\pgfpathclose%
\pgfusepath{stroke,fill}%
\end{pgfscope}%
\begin{pgfscope}%
\pgfpathrectangle{\pgfqpoint{0.600000in}{0.600000in}}{\pgfqpoint{3.900000in}{3.900000in}}%
\pgfusepath{clip}%
\pgfsetbuttcap%
\pgfsetroundjoin%
\definecolor{currentfill}{rgb}{0.121569,0.466667,0.705882}%
\pgfsetfillcolor{currentfill}%
\pgfsetlinewidth{1.003750pt}%
\definecolor{currentstroke}{rgb}{0.121569,0.466667,0.705882}%
\pgfsetstrokecolor{currentstroke}%
\pgfsetdash{}{0pt}%
\pgfpathmoveto{\pgfqpoint{2.380133in}{1.780000in}}%
\pgfpathcurveto{\pgfqpoint{2.391183in}{1.780000in}}{\pgfqpoint{2.401782in}{1.784390in}}{\pgfqpoint{2.409595in}{1.792204in}}%
\pgfpathcurveto{\pgfqpoint{2.417409in}{1.800017in}}{\pgfqpoint{2.421799in}{1.810616in}}{\pgfqpoint{2.421799in}{1.821666in}}%
\pgfpathcurveto{\pgfqpoint{2.421799in}{1.832717in}}{\pgfqpoint{2.417409in}{1.843316in}}{\pgfqpoint{2.409595in}{1.851129in}}%
\pgfpathcurveto{\pgfqpoint{2.401782in}{1.858943in}}{\pgfqpoint{2.391183in}{1.863333in}}{\pgfqpoint{2.380133in}{1.863333in}}%
\pgfpathcurveto{\pgfqpoint{2.369082in}{1.863333in}}{\pgfqpoint{2.358483in}{1.858943in}}{\pgfqpoint{2.350670in}{1.851129in}}%
\pgfpathcurveto{\pgfqpoint{2.342856in}{1.843316in}}{\pgfqpoint{2.338466in}{1.832717in}}{\pgfqpoint{2.338466in}{1.821666in}}%
\pgfpathcurveto{\pgfqpoint{2.338466in}{1.810616in}}{\pgfqpoint{2.342856in}{1.800017in}}{\pgfqpoint{2.350670in}{1.792204in}}%
\pgfpathcurveto{\pgfqpoint{2.358483in}{1.784390in}}{\pgfqpoint{2.369082in}{1.780000in}}{\pgfqpoint{2.380133in}{1.780000in}}%
\pgfpathclose%
\pgfusepath{stroke,fill}%
\end{pgfscope}%
\begin{pgfscope}%
\pgfpathrectangle{\pgfqpoint{0.600000in}{0.600000in}}{\pgfqpoint{3.900000in}{3.900000in}}%
\pgfusepath{clip}%
\pgfsetbuttcap%
\pgfsetroundjoin%
\definecolor{currentfill}{rgb}{0.121569,0.466667,0.705882}%
\pgfsetfillcolor{currentfill}%
\pgfsetlinewidth{1.003750pt}%
\definecolor{currentstroke}{rgb}{0.121569,0.466667,0.705882}%
\pgfsetstrokecolor{currentstroke}%
\pgfsetdash{}{0pt}%
\pgfpathmoveto{\pgfqpoint{3.853940in}{2.167058in}}%
\pgfpathcurveto{\pgfqpoint{3.864990in}{2.167058in}}{\pgfqpoint{3.875589in}{2.171448in}}{\pgfqpoint{3.883403in}{2.179261in}}%
\pgfpathcurveto{\pgfqpoint{3.891216in}{2.187075in}}{\pgfqpoint{3.895607in}{2.197674in}}{\pgfqpoint{3.895607in}{2.208724in}}%
\pgfpathcurveto{\pgfqpoint{3.895607in}{2.219774in}}{\pgfqpoint{3.891216in}{2.230373in}}{\pgfqpoint{3.883403in}{2.238187in}}%
\pgfpathcurveto{\pgfqpoint{3.875589in}{2.246001in}}{\pgfqpoint{3.864990in}{2.250391in}}{\pgfqpoint{3.853940in}{2.250391in}}%
\pgfpathcurveto{\pgfqpoint{3.842890in}{2.250391in}}{\pgfqpoint{3.832291in}{2.246001in}}{\pgfqpoint{3.824477in}{2.238187in}}%
\pgfpathcurveto{\pgfqpoint{3.816664in}{2.230373in}}{\pgfqpoint{3.812273in}{2.219774in}}{\pgfqpoint{3.812273in}{2.208724in}}%
\pgfpathcurveto{\pgfqpoint{3.812273in}{2.197674in}}{\pgfqpoint{3.816664in}{2.187075in}}{\pgfqpoint{3.824477in}{2.179261in}}%
\pgfpathcurveto{\pgfqpoint{3.832291in}{2.171448in}}{\pgfqpoint{3.842890in}{2.167058in}}{\pgfqpoint{3.853940in}{2.167058in}}%
\pgfpathclose%
\pgfusepath{stroke,fill}%
\end{pgfscope}%
\begin{pgfscope}%
\pgfpathrectangle{\pgfqpoint{0.600000in}{0.600000in}}{\pgfqpoint{3.900000in}{3.900000in}}%
\pgfusepath{clip}%
\pgfsetbuttcap%
\pgfsetroundjoin%
\definecolor{currentfill}{rgb}{0.121569,0.466667,0.705882}%
\pgfsetfillcolor{currentfill}%
\pgfsetlinewidth{1.003750pt}%
\definecolor{currentstroke}{rgb}{0.121569,0.466667,0.705882}%
\pgfsetstrokecolor{currentstroke}%
\pgfsetdash{}{0pt}%
\pgfpathmoveto{\pgfqpoint{2.549988in}{3.134071in}}%
\pgfpathcurveto{\pgfqpoint{2.561038in}{3.134071in}}{\pgfqpoint{2.571637in}{3.138462in}}{\pgfqpoint{2.579451in}{3.146275in}}%
\pgfpathcurveto{\pgfqpoint{2.587264in}{3.154089in}}{\pgfqpoint{2.591655in}{3.164688in}}{\pgfqpoint{2.591655in}{3.175738in}}%
\pgfpathcurveto{\pgfqpoint{2.591655in}{3.186788in}}{\pgfqpoint{2.587264in}{3.197387in}}{\pgfqpoint{2.579451in}{3.205201in}}%
\pgfpathcurveto{\pgfqpoint{2.571637in}{3.213014in}}{\pgfqpoint{2.561038in}{3.217405in}}{\pgfqpoint{2.549988in}{3.217405in}}%
\pgfpathcurveto{\pgfqpoint{2.538938in}{3.217405in}}{\pgfqpoint{2.528339in}{3.213014in}}{\pgfqpoint{2.520525in}{3.205201in}}%
\pgfpathcurveto{\pgfqpoint{2.512712in}{3.197387in}}{\pgfqpoint{2.508321in}{3.186788in}}{\pgfqpoint{2.508321in}{3.175738in}}%
\pgfpathcurveto{\pgfqpoint{2.508321in}{3.164688in}}{\pgfqpoint{2.512712in}{3.154089in}}{\pgfqpoint{2.520525in}{3.146275in}}%
\pgfpathcurveto{\pgfqpoint{2.528339in}{3.138462in}}{\pgfqpoint{2.538938in}{3.134071in}}{\pgfqpoint{2.549988in}{3.134071in}}%
\pgfpathclose%
\pgfusepath{stroke,fill}%
\end{pgfscope}%
\begin{pgfscope}%
\pgfpathrectangle{\pgfqpoint{0.600000in}{0.600000in}}{\pgfqpoint{3.900000in}{3.900000in}}%
\pgfusepath{clip}%
\pgfsetbuttcap%
\pgfsetroundjoin%
\definecolor{currentfill}{rgb}{0.121569,0.466667,0.705882}%
\pgfsetfillcolor{currentfill}%
\pgfsetlinewidth{1.003750pt}%
\definecolor{currentstroke}{rgb}{0.121569,0.466667,0.705882}%
\pgfsetstrokecolor{currentstroke}%
\pgfsetdash{}{0pt}%
\pgfpathmoveto{\pgfqpoint{2.780980in}{2.694881in}}%
\pgfpathcurveto{\pgfqpoint{2.792030in}{2.694881in}}{\pgfqpoint{2.802629in}{2.699271in}}{\pgfqpoint{2.810442in}{2.707085in}}%
\pgfpathcurveto{\pgfqpoint{2.818256in}{2.714898in}}{\pgfqpoint{2.822646in}{2.725498in}}{\pgfqpoint{2.822646in}{2.736548in}}%
\pgfpathcurveto{\pgfqpoint{2.822646in}{2.747598in}}{\pgfqpoint{2.818256in}{2.758197in}}{\pgfqpoint{2.810442in}{2.766010in}}%
\pgfpathcurveto{\pgfqpoint{2.802629in}{2.773824in}}{\pgfqpoint{2.792030in}{2.778214in}}{\pgfqpoint{2.780980in}{2.778214in}}%
\pgfpathcurveto{\pgfqpoint{2.769930in}{2.778214in}}{\pgfqpoint{2.759330in}{2.773824in}}{\pgfqpoint{2.751517in}{2.766010in}}%
\pgfpathcurveto{\pgfqpoint{2.743703in}{2.758197in}}{\pgfqpoint{2.739313in}{2.747598in}}{\pgfqpoint{2.739313in}{2.736548in}}%
\pgfpathcurveto{\pgfqpoint{2.739313in}{2.725498in}}{\pgfqpoint{2.743703in}{2.714898in}}{\pgfqpoint{2.751517in}{2.707085in}}%
\pgfpathcurveto{\pgfqpoint{2.759330in}{2.699271in}}{\pgfqpoint{2.769930in}{2.694881in}}{\pgfqpoint{2.780980in}{2.694881in}}%
\pgfpathclose%
\pgfusepath{stroke,fill}%
\end{pgfscope}%
\begin{pgfscope}%
\pgfpathrectangle{\pgfqpoint{0.600000in}{0.600000in}}{\pgfqpoint{3.900000in}{3.900000in}}%
\pgfusepath{clip}%
\pgfsetbuttcap%
\pgfsetroundjoin%
\definecolor{currentfill}{rgb}{0.121569,0.466667,0.705882}%
\pgfsetfillcolor{currentfill}%
\pgfsetlinewidth{1.003750pt}%
\definecolor{currentstroke}{rgb}{0.121569,0.466667,0.705882}%
\pgfsetstrokecolor{currentstroke}%
\pgfsetdash{}{0pt}%
\pgfpathmoveto{\pgfqpoint{2.638921in}{2.054517in}}%
\pgfpathcurveto{\pgfqpoint{2.649971in}{2.054517in}}{\pgfqpoint{2.660570in}{2.058907in}}{\pgfqpoint{2.668384in}{2.066720in}}%
\pgfpathcurveto{\pgfqpoint{2.676197in}{2.074534in}}{\pgfqpoint{2.680588in}{2.085133in}}{\pgfqpoint{2.680588in}{2.096183in}}%
\pgfpathcurveto{\pgfqpoint{2.680588in}{2.107233in}}{\pgfqpoint{2.676197in}{2.117832in}}{\pgfqpoint{2.668384in}{2.125646in}}%
\pgfpathcurveto{\pgfqpoint{2.660570in}{2.133460in}}{\pgfqpoint{2.649971in}{2.137850in}}{\pgfqpoint{2.638921in}{2.137850in}}%
\pgfpathcurveto{\pgfqpoint{2.627871in}{2.137850in}}{\pgfqpoint{2.617272in}{2.133460in}}{\pgfqpoint{2.609458in}{2.125646in}}%
\pgfpathcurveto{\pgfqpoint{2.601645in}{2.117832in}}{\pgfqpoint{2.597254in}{2.107233in}}{\pgfqpoint{2.597254in}{2.096183in}}%
\pgfpathcurveto{\pgfqpoint{2.597254in}{2.085133in}}{\pgfqpoint{2.601645in}{2.074534in}}{\pgfqpoint{2.609458in}{2.066720in}}%
\pgfpathcurveto{\pgfqpoint{2.617272in}{2.058907in}}{\pgfqpoint{2.627871in}{2.054517in}}{\pgfqpoint{2.638921in}{2.054517in}}%
\pgfpathclose%
\pgfusepath{stroke,fill}%
\end{pgfscope}%
\begin{pgfscope}%
\pgfpathrectangle{\pgfqpoint{0.600000in}{0.600000in}}{\pgfqpoint{3.900000in}{3.900000in}}%
\pgfusepath{clip}%
\pgfsetbuttcap%
\pgfsetroundjoin%
\definecolor{currentfill}{rgb}{0.121569,0.466667,0.705882}%
\pgfsetfillcolor{currentfill}%
\pgfsetlinewidth{1.003750pt}%
\definecolor{currentstroke}{rgb}{0.121569,0.466667,0.705882}%
\pgfsetstrokecolor{currentstroke}%
\pgfsetdash{}{0pt}%
\pgfpathmoveto{\pgfqpoint{2.913271in}{2.781972in}}%
\pgfpathcurveto{\pgfqpoint{2.924321in}{2.781972in}}{\pgfqpoint{2.934920in}{2.786362in}}{\pgfqpoint{2.942734in}{2.794176in}}%
\pgfpathcurveto{\pgfqpoint{2.950547in}{2.801989in}}{\pgfqpoint{2.954938in}{2.812588in}}{\pgfqpoint{2.954938in}{2.823639in}}%
\pgfpathcurveto{\pgfqpoint{2.954938in}{2.834689in}}{\pgfqpoint{2.950547in}{2.845288in}}{\pgfqpoint{2.942734in}{2.853101in}}%
\pgfpathcurveto{\pgfqpoint{2.934920in}{2.860915in}}{\pgfqpoint{2.924321in}{2.865305in}}{\pgfqpoint{2.913271in}{2.865305in}}%
\pgfpathcurveto{\pgfqpoint{2.902221in}{2.865305in}}{\pgfqpoint{2.891622in}{2.860915in}}{\pgfqpoint{2.883808in}{2.853101in}}%
\pgfpathcurveto{\pgfqpoint{2.875995in}{2.845288in}}{\pgfqpoint{2.871604in}{2.834689in}}{\pgfqpoint{2.871604in}{2.823639in}}%
\pgfpathcurveto{\pgfqpoint{2.871604in}{2.812588in}}{\pgfqpoint{2.875995in}{2.801989in}}{\pgfqpoint{2.883808in}{2.794176in}}%
\pgfpathcurveto{\pgfqpoint{2.891622in}{2.786362in}}{\pgfqpoint{2.902221in}{2.781972in}}{\pgfqpoint{2.913271in}{2.781972in}}%
\pgfpathclose%
\pgfusepath{stroke,fill}%
\end{pgfscope}%
\begin{pgfscope}%
\pgfpathrectangle{\pgfqpoint{0.600000in}{0.600000in}}{\pgfqpoint{3.900000in}{3.900000in}}%
\pgfusepath{clip}%
\pgfsetbuttcap%
\pgfsetroundjoin%
\definecolor{currentfill}{rgb}{0.121569,0.466667,0.705882}%
\pgfsetfillcolor{currentfill}%
\pgfsetlinewidth{1.003750pt}%
\definecolor{currentstroke}{rgb}{0.121569,0.466667,0.705882}%
\pgfsetstrokecolor{currentstroke}%
\pgfsetdash{}{0pt}%
\pgfpathmoveto{\pgfqpoint{3.011055in}{2.522042in}}%
\pgfpathcurveto{\pgfqpoint{3.022105in}{2.522042in}}{\pgfqpoint{3.032704in}{2.526432in}}{\pgfqpoint{3.040518in}{2.534246in}}%
\pgfpathcurveto{\pgfqpoint{3.048331in}{2.542060in}}{\pgfqpoint{3.052722in}{2.552659in}}{\pgfqpoint{3.052722in}{2.563709in}}%
\pgfpathcurveto{\pgfqpoint{3.052722in}{2.574759in}}{\pgfqpoint{3.048331in}{2.585358in}}{\pgfqpoint{3.040518in}{2.593172in}}%
\pgfpathcurveto{\pgfqpoint{3.032704in}{2.600985in}}{\pgfqpoint{3.022105in}{2.605376in}}{\pgfqpoint{3.011055in}{2.605376in}}%
\pgfpathcurveto{\pgfqpoint{3.000005in}{2.605376in}}{\pgfqpoint{2.989406in}{2.600985in}}{\pgfqpoint{2.981592in}{2.593172in}}%
\pgfpathcurveto{\pgfqpoint{2.973779in}{2.585358in}}{\pgfqpoint{2.969388in}{2.574759in}}{\pgfqpoint{2.969388in}{2.563709in}}%
\pgfpathcurveto{\pgfqpoint{2.969388in}{2.552659in}}{\pgfqpoint{2.973779in}{2.542060in}}{\pgfqpoint{2.981592in}{2.534246in}}%
\pgfpathcurveto{\pgfqpoint{2.989406in}{2.526432in}}{\pgfqpoint{3.000005in}{2.522042in}}{\pgfqpoint{3.011055in}{2.522042in}}%
\pgfpathclose%
\pgfusepath{stroke,fill}%
\end{pgfscope}%
\begin{pgfscope}%
\pgfpathrectangle{\pgfqpoint{0.600000in}{0.600000in}}{\pgfqpoint{3.900000in}{3.900000in}}%
\pgfusepath{clip}%
\pgfsetbuttcap%
\pgfsetroundjoin%
\definecolor{currentfill}{rgb}{0.121569,0.466667,0.705882}%
\pgfsetfillcolor{currentfill}%
\pgfsetlinewidth{1.003750pt}%
\definecolor{currentstroke}{rgb}{0.121569,0.466667,0.705882}%
\pgfsetstrokecolor{currentstroke}%
\pgfsetdash{}{0pt}%
\pgfpathmoveto{\pgfqpoint{3.882548in}{1.985225in}}%
\pgfpathcurveto{\pgfqpoint{3.893598in}{1.985225in}}{\pgfqpoint{3.904197in}{1.989615in}}{\pgfqpoint{3.912011in}{1.997429in}}%
\pgfpathcurveto{\pgfqpoint{3.919825in}{2.005242in}}{\pgfqpoint{3.924215in}{2.015841in}}{\pgfqpoint{3.924215in}{2.026891in}}%
\pgfpathcurveto{\pgfqpoint{3.924215in}{2.037941in}}{\pgfqpoint{3.919825in}{2.048540in}}{\pgfqpoint{3.912011in}{2.056354in}}%
\pgfpathcurveto{\pgfqpoint{3.904197in}{2.064168in}}{\pgfqpoint{3.893598in}{2.068558in}}{\pgfqpoint{3.882548in}{2.068558in}}%
\pgfpathcurveto{\pgfqpoint{3.871498in}{2.068558in}}{\pgfqpoint{3.860899in}{2.064168in}}{\pgfqpoint{3.853085in}{2.056354in}}%
\pgfpathcurveto{\pgfqpoint{3.845272in}{2.048540in}}{\pgfqpoint{3.840882in}{2.037941in}}{\pgfqpoint{3.840882in}{2.026891in}}%
\pgfpathcurveto{\pgfqpoint{3.840882in}{2.015841in}}{\pgfqpoint{3.845272in}{2.005242in}}{\pgfqpoint{3.853085in}{1.997429in}}%
\pgfpathcurveto{\pgfqpoint{3.860899in}{1.989615in}}{\pgfqpoint{3.871498in}{1.985225in}}{\pgfqpoint{3.882548in}{1.985225in}}%
\pgfpathclose%
\pgfusepath{stroke,fill}%
\end{pgfscope}%
\begin{pgfscope}%
\pgfpathrectangle{\pgfqpoint{0.600000in}{0.600000in}}{\pgfqpoint{3.900000in}{3.900000in}}%
\pgfusepath{clip}%
\pgfsetbuttcap%
\pgfsetroundjoin%
\definecolor{currentfill}{rgb}{0.121569,0.466667,0.705882}%
\pgfsetfillcolor{currentfill}%
\pgfsetlinewidth{1.003750pt}%
\definecolor{currentstroke}{rgb}{0.121569,0.466667,0.705882}%
\pgfsetstrokecolor{currentstroke}%
\pgfsetdash{}{0pt}%
\pgfpathmoveto{\pgfqpoint{2.039501in}{3.439291in}}%
\pgfpathcurveto{\pgfqpoint{2.050551in}{3.439291in}}{\pgfqpoint{2.061150in}{3.443682in}}{\pgfqpoint{2.068963in}{3.451495in}}%
\pgfpathcurveto{\pgfqpoint{2.076777in}{3.459309in}}{\pgfqpoint{2.081167in}{3.469908in}}{\pgfqpoint{2.081167in}{3.480958in}}%
\pgfpathcurveto{\pgfqpoint{2.081167in}{3.492008in}}{\pgfqpoint{2.076777in}{3.502607in}}{\pgfqpoint{2.068963in}{3.510421in}}%
\pgfpathcurveto{\pgfqpoint{2.061150in}{3.518235in}}{\pgfqpoint{2.050551in}{3.522625in}}{\pgfqpoint{2.039501in}{3.522625in}}%
\pgfpathcurveto{\pgfqpoint{2.028450in}{3.522625in}}{\pgfqpoint{2.017851in}{3.518235in}}{\pgfqpoint{2.010038in}{3.510421in}}%
\pgfpathcurveto{\pgfqpoint{2.002224in}{3.502607in}}{\pgfqpoint{1.997834in}{3.492008in}}{\pgfqpoint{1.997834in}{3.480958in}}%
\pgfpathcurveto{\pgfqpoint{1.997834in}{3.469908in}}{\pgfqpoint{2.002224in}{3.459309in}}{\pgfqpoint{2.010038in}{3.451495in}}%
\pgfpathcurveto{\pgfqpoint{2.017851in}{3.443682in}}{\pgfqpoint{2.028450in}{3.439291in}}{\pgfqpoint{2.039501in}{3.439291in}}%
\pgfpathclose%
\pgfusepath{stroke,fill}%
\end{pgfscope}%
\begin{pgfscope}%
\pgfpathrectangle{\pgfqpoint{0.600000in}{0.600000in}}{\pgfqpoint{3.900000in}{3.900000in}}%
\pgfusepath{clip}%
\pgfsetbuttcap%
\pgfsetroundjoin%
\definecolor{currentfill}{rgb}{0.121569,0.466667,0.705882}%
\pgfsetfillcolor{currentfill}%
\pgfsetlinewidth{1.003750pt}%
\definecolor{currentstroke}{rgb}{0.121569,0.466667,0.705882}%
\pgfsetstrokecolor{currentstroke}%
\pgfsetdash{}{0pt}%
\pgfpathmoveto{\pgfqpoint{3.049524in}{2.621238in}}%
\pgfpathcurveto{\pgfqpoint{3.060574in}{2.621238in}}{\pgfqpoint{3.071173in}{2.625628in}}{\pgfqpoint{3.078987in}{2.633442in}}%
\pgfpathcurveto{\pgfqpoint{3.086800in}{2.641255in}}{\pgfqpoint{3.091191in}{2.651854in}}{\pgfqpoint{3.091191in}{2.662904in}}%
\pgfpathcurveto{\pgfqpoint{3.091191in}{2.673955in}}{\pgfqpoint{3.086800in}{2.684554in}}{\pgfqpoint{3.078987in}{2.692367in}}%
\pgfpathcurveto{\pgfqpoint{3.071173in}{2.700181in}}{\pgfqpoint{3.060574in}{2.704571in}}{\pgfqpoint{3.049524in}{2.704571in}}%
\pgfpathcurveto{\pgfqpoint{3.038474in}{2.704571in}}{\pgfqpoint{3.027875in}{2.700181in}}{\pgfqpoint{3.020061in}{2.692367in}}%
\pgfpathcurveto{\pgfqpoint{3.012248in}{2.684554in}}{\pgfqpoint{3.007857in}{2.673955in}}{\pgfqpoint{3.007857in}{2.662904in}}%
\pgfpathcurveto{\pgfqpoint{3.007857in}{2.651854in}}{\pgfqpoint{3.012248in}{2.641255in}}{\pgfqpoint{3.020061in}{2.633442in}}%
\pgfpathcurveto{\pgfqpoint{3.027875in}{2.625628in}}{\pgfqpoint{3.038474in}{2.621238in}}{\pgfqpoint{3.049524in}{2.621238in}}%
\pgfpathclose%
\pgfusepath{stroke,fill}%
\end{pgfscope}%
\begin{pgfscope}%
\pgfpathrectangle{\pgfqpoint{0.600000in}{0.600000in}}{\pgfqpoint{3.900000in}{3.900000in}}%
\pgfusepath{clip}%
\pgfsetbuttcap%
\pgfsetroundjoin%
\definecolor{currentfill}{rgb}{0.121569,0.466667,0.705882}%
\pgfsetfillcolor{currentfill}%
\pgfsetlinewidth{1.003750pt}%
\definecolor{currentstroke}{rgb}{0.121569,0.466667,0.705882}%
\pgfsetstrokecolor{currentstroke}%
\pgfsetdash{}{0pt}%
\pgfpathmoveto{\pgfqpoint{2.113102in}{2.759069in}}%
\pgfpathcurveto{\pgfqpoint{2.124153in}{2.759069in}}{\pgfqpoint{2.134752in}{2.763460in}}{\pgfqpoint{2.142565in}{2.771273in}}%
\pgfpathcurveto{\pgfqpoint{2.150379in}{2.779087in}}{\pgfqpoint{2.154769in}{2.789686in}}{\pgfqpoint{2.154769in}{2.800736in}}%
\pgfpathcurveto{\pgfqpoint{2.154769in}{2.811786in}}{\pgfqpoint{2.150379in}{2.822385in}}{\pgfqpoint{2.142565in}{2.830199in}}%
\pgfpathcurveto{\pgfqpoint{2.134752in}{2.838013in}}{\pgfqpoint{2.124153in}{2.842403in}}{\pgfqpoint{2.113102in}{2.842403in}}%
\pgfpathcurveto{\pgfqpoint{2.102052in}{2.842403in}}{\pgfqpoint{2.091453in}{2.838013in}}{\pgfqpoint{2.083640in}{2.830199in}}%
\pgfpathcurveto{\pgfqpoint{2.075826in}{2.822385in}}{\pgfqpoint{2.071436in}{2.811786in}}{\pgfqpoint{2.071436in}{2.800736in}}%
\pgfpathcurveto{\pgfqpoint{2.071436in}{2.789686in}}{\pgfqpoint{2.075826in}{2.779087in}}{\pgfqpoint{2.083640in}{2.771273in}}%
\pgfpathcurveto{\pgfqpoint{2.091453in}{2.763460in}}{\pgfqpoint{2.102052in}{2.759069in}}{\pgfqpoint{2.113102in}{2.759069in}}%
\pgfpathclose%
\pgfusepath{stroke,fill}%
\end{pgfscope}%
\begin{pgfscope}%
\pgfpathrectangle{\pgfqpoint{0.600000in}{0.600000in}}{\pgfqpoint{3.900000in}{3.900000in}}%
\pgfusepath{clip}%
\pgfsetbuttcap%
\pgfsetroundjoin%
\definecolor{currentfill}{rgb}{0.121569,0.466667,0.705882}%
\pgfsetfillcolor{currentfill}%
\pgfsetlinewidth{1.003750pt}%
\definecolor{currentstroke}{rgb}{0.121569,0.466667,0.705882}%
\pgfsetstrokecolor{currentstroke}%
\pgfsetdash{}{0pt}%
\pgfpathmoveto{\pgfqpoint{3.203408in}{3.189369in}}%
\pgfpathcurveto{\pgfqpoint{3.214458in}{3.189369in}}{\pgfqpoint{3.225057in}{3.193759in}}{\pgfqpoint{3.232871in}{3.201573in}}%
\pgfpathcurveto{\pgfqpoint{3.240684in}{3.209386in}}{\pgfqpoint{3.245075in}{3.219985in}}{\pgfqpoint{3.245075in}{3.231036in}}%
\pgfpathcurveto{\pgfqpoint{3.245075in}{3.242086in}}{\pgfqpoint{3.240684in}{3.252685in}}{\pgfqpoint{3.232871in}{3.260498in}}%
\pgfpathcurveto{\pgfqpoint{3.225057in}{3.268312in}}{\pgfqpoint{3.214458in}{3.272702in}}{\pgfqpoint{3.203408in}{3.272702in}}%
\pgfpathcurveto{\pgfqpoint{3.192358in}{3.272702in}}{\pgfqpoint{3.181759in}{3.268312in}}{\pgfqpoint{3.173945in}{3.260498in}}%
\pgfpathcurveto{\pgfqpoint{3.166132in}{3.252685in}}{\pgfqpoint{3.161741in}{3.242086in}}{\pgfqpoint{3.161741in}{3.231036in}}%
\pgfpathcurveto{\pgfqpoint{3.161741in}{3.219985in}}{\pgfqpoint{3.166132in}{3.209386in}}{\pgfqpoint{3.173945in}{3.201573in}}%
\pgfpathcurveto{\pgfqpoint{3.181759in}{3.193759in}}{\pgfqpoint{3.192358in}{3.189369in}}{\pgfqpoint{3.203408in}{3.189369in}}%
\pgfpathclose%
\pgfusepath{stroke,fill}%
\end{pgfscope}%
\begin{pgfscope}%
\pgfpathrectangle{\pgfqpoint{0.600000in}{0.600000in}}{\pgfqpoint{3.900000in}{3.900000in}}%
\pgfusepath{clip}%
\pgfsetbuttcap%
\pgfsetroundjoin%
\definecolor{currentfill}{rgb}{0.121569,0.466667,0.705882}%
\pgfsetfillcolor{currentfill}%
\pgfsetlinewidth{1.003750pt}%
\definecolor{currentstroke}{rgb}{0.121569,0.466667,0.705882}%
\pgfsetstrokecolor{currentstroke}%
\pgfsetdash{}{0pt}%
\pgfpathmoveto{\pgfqpoint{3.004118in}{2.748147in}}%
\pgfpathcurveto{\pgfqpoint{3.015168in}{2.748147in}}{\pgfqpoint{3.025767in}{2.752538in}}{\pgfqpoint{3.033581in}{2.760351in}}%
\pgfpathcurveto{\pgfqpoint{3.041394in}{2.768165in}}{\pgfqpoint{3.045785in}{2.778764in}}{\pgfqpoint{3.045785in}{2.789814in}}%
\pgfpathcurveto{\pgfqpoint{3.045785in}{2.800864in}}{\pgfqpoint{3.041394in}{2.811463in}}{\pgfqpoint{3.033581in}{2.819277in}}%
\pgfpathcurveto{\pgfqpoint{3.025767in}{2.827090in}}{\pgfqpoint{3.015168in}{2.831481in}}{\pgfqpoint{3.004118in}{2.831481in}}%
\pgfpathcurveto{\pgfqpoint{2.993068in}{2.831481in}}{\pgfqpoint{2.982469in}{2.827090in}}{\pgfqpoint{2.974655in}{2.819277in}}%
\pgfpathcurveto{\pgfqpoint{2.966842in}{2.811463in}}{\pgfqpoint{2.962451in}{2.800864in}}{\pgfqpoint{2.962451in}{2.789814in}}%
\pgfpathcurveto{\pgfqpoint{2.962451in}{2.778764in}}{\pgfqpoint{2.966842in}{2.768165in}}{\pgfqpoint{2.974655in}{2.760351in}}%
\pgfpathcurveto{\pgfqpoint{2.982469in}{2.752538in}}{\pgfqpoint{2.993068in}{2.748147in}}{\pgfqpoint{3.004118in}{2.748147in}}%
\pgfpathclose%
\pgfusepath{stroke,fill}%
\end{pgfscope}%
\begin{pgfscope}%
\pgfpathrectangle{\pgfqpoint{0.600000in}{0.600000in}}{\pgfqpoint{3.900000in}{3.900000in}}%
\pgfusepath{clip}%
\pgfsetbuttcap%
\pgfsetroundjoin%
\definecolor{currentfill}{rgb}{0.121569,0.466667,0.705882}%
\pgfsetfillcolor{currentfill}%
\pgfsetlinewidth{1.003750pt}%
\definecolor{currentstroke}{rgb}{0.121569,0.466667,0.705882}%
\pgfsetstrokecolor{currentstroke}%
\pgfsetdash{}{0pt}%
\pgfpathmoveto{\pgfqpoint{2.584094in}{2.496126in}}%
\pgfpathcurveto{\pgfqpoint{2.595144in}{2.496126in}}{\pgfqpoint{2.605743in}{2.500516in}}{\pgfqpoint{2.613557in}{2.508330in}}%
\pgfpathcurveto{\pgfqpoint{2.621370in}{2.516143in}}{\pgfqpoint{2.625760in}{2.526742in}}{\pgfqpoint{2.625760in}{2.537792in}}%
\pgfpathcurveto{\pgfqpoint{2.625760in}{2.548842in}}{\pgfqpoint{2.621370in}{2.559442in}}{\pgfqpoint{2.613557in}{2.567255in}}%
\pgfpathcurveto{\pgfqpoint{2.605743in}{2.575069in}}{\pgfqpoint{2.595144in}{2.579459in}}{\pgfqpoint{2.584094in}{2.579459in}}%
\pgfpathcurveto{\pgfqpoint{2.573044in}{2.579459in}}{\pgfqpoint{2.562445in}{2.575069in}}{\pgfqpoint{2.554631in}{2.567255in}}%
\pgfpathcurveto{\pgfqpoint{2.546817in}{2.559442in}}{\pgfqpoint{2.542427in}{2.548842in}}{\pgfqpoint{2.542427in}{2.537792in}}%
\pgfpathcurveto{\pgfqpoint{2.542427in}{2.526742in}}{\pgfqpoint{2.546817in}{2.516143in}}{\pgfqpoint{2.554631in}{2.508330in}}%
\pgfpathcurveto{\pgfqpoint{2.562445in}{2.500516in}}{\pgfqpoint{2.573044in}{2.496126in}}{\pgfqpoint{2.584094in}{2.496126in}}%
\pgfpathclose%
\pgfusepath{stroke,fill}%
\end{pgfscope}%
\begin{pgfscope}%
\pgfpathrectangle{\pgfqpoint{0.600000in}{0.600000in}}{\pgfqpoint{3.900000in}{3.900000in}}%
\pgfusepath{clip}%
\pgfsetbuttcap%
\pgfsetroundjoin%
\definecolor{currentfill}{rgb}{0.121569,0.466667,0.705882}%
\pgfsetfillcolor{currentfill}%
\pgfsetlinewidth{1.003750pt}%
\definecolor{currentstroke}{rgb}{0.121569,0.466667,0.705882}%
\pgfsetstrokecolor{currentstroke}%
\pgfsetdash{}{0pt}%
\pgfpathmoveto{\pgfqpoint{2.760850in}{2.268988in}}%
\pgfpathcurveto{\pgfqpoint{2.771901in}{2.268988in}}{\pgfqpoint{2.782500in}{2.273378in}}{\pgfqpoint{2.790313in}{2.281192in}}%
\pgfpathcurveto{\pgfqpoint{2.798127in}{2.289006in}}{\pgfqpoint{2.802517in}{2.299605in}}{\pgfqpoint{2.802517in}{2.310655in}}%
\pgfpathcurveto{\pgfqpoint{2.802517in}{2.321705in}}{\pgfqpoint{2.798127in}{2.332304in}}{\pgfqpoint{2.790313in}{2.340118in}}%
\pgfpathcurveto{\pgfqpoint{2.782500in}{2.347931in}}{\pgfqpoint{2.771901in}{2.352322in}}{\pgfqpoint{2.760850in}{2.352322in}}%
\pgfpathcurveto{\pgfqpoint{2.749800in}{2.352322in}}{\pgfqpoint{2.739201in}{2.347931in}}{\pgfqpoint{2.731388in}{2.340118in}}%
\pgfpathcurveto{\pgfqpoint{2.723574in}{2.332304in}}{\pgfqpoint{2.719184in}{2.321705in}}{\pgfqpoint{2.719184in}{2.310655in}}%
\pgfpathcurveto{\pgfqpoint{2.719184in}{2.299605in}}{\pgfqpoint{2.723574in}{2.289006in}}{\pgfqpoint{2.731388in}{2.281192in}}%
\pgfpathcurveto{\pgfqpoint{2.739201in}{2.273378in}}{\pgfqpoint{2.749800in}{2.268988in}}{\pgfqpoint{2.760850in}{2.268988in}}%
\pgfpathclose%
\pgfusepath{stroke,fill}%
\end{pgfscope}%
\begin{pgfscope}%
\pgfpathrectangle{\pgfqpoint{0.600000in}{0.600000in}}{\pgfqpoint{3.900000in}{3.900000in}}%
\pgfusepath{clip}%
\pgfsetbuttcap%
\pgfsetroundjoin%
\definecolor{currentfill}{rgb}{0.121569,0.466667,0.705882}%
\pgfsetfillcolor{currentfill}%
\pgfsetlinewidth{1.003750pt}%
\definecolor{currentstroke}{rgb}{0.121569,0.466667,0.705882}%
\pgfsetstrokecolor{currentstroke}%
\pgfsetdash{}{0pt}%
\pgfpathmoveto{\pgfqpoint{3.073460in}{2.656399in}}%
\pgfpathcurveto{\pgfqpoint{3.084510in}{2.656399in}}{\pgfqpoint{3.095109in}{2.660789in}}{\pgfqpoint{3.102923in}{2.668602in}}%
\pgfpathcurveto{\pgfqpoint{3.110736in}{2.676416in}}{\pgfqpoint{3.115126in}{2.687015in}}{\pgfqpoint{3.115126in}{2.698065in}}%
\pgfpathcurveto{\pgfqpoint{3.115126in}{2.709115in}}{\pgfqpoint{3.110736in}{2.719714in}}{\pgfqpoint{3.102923in}{2.727528in}}%
\pgfpathcurveto{\pgfqpoint{3.095109in}{2.735342in}}{\pgfqpoint{3.084510in}{2.739732in}}{\pgfqpoint{3.073460in}{2.739732in}}%
\pgfpathcurveto{\pgfqpoint{3.062410in}{2.739732in}}{\pgfqpoint{3.051811in}{2.735342in}}{\pgfqpoint{3.043997in}{2.727528in}}%
\pgfpathcurveto{\pgfqpoint{3.036183in}{2.719714in}}{\pgfqpoint{3.031793in}{2.709115in}}{\pgfqpoint{3.031793in}{2.698065in}}%
\pgfpathcurveto{\pgfqpoint{3.031793in}{2.687015in}}{\pgfqpoint{3.036183in}{2.676416in}}{\pgfqpoint{3.043997in}{2.668602in}}%
\pgfpathcurveto{\pgfqpoint{3.051811in}{2.660789in}}{\pgfqpoint{3.062410in}{2.656399in}}{\pgfqpoint{3.073460in}{2.656399in}}%
\pgfpathclose%
\pgfusepath{stroke,fill}%
\end{pgfscope}%
\begin{pgfscope}%
\pgfpathrectangle{\pgfqpoint{0.600000in}{0.600000in}}{\pgfqpoint{3.900000in}{3.900000in}}%
\pgfusepath{clip}%
\pgfsetbuttcap%
\pgfsetroundjoin%
\definecolor{currentfill}{rgb}{0.121569,0.466667,0.705882}%
\pgfsetfillcolor{currentfill}%
\pgfsetlinewidth{1.003750pt}%
\definecolor{currentstroke}{rgb}{0.121569,0.466667,0.705882}%
\pgfsetstrokecolor{currentstroke}%
\pgfsetdash{}{0pt}%
\pgfpathmoveto{\pgfqpoint{2.523724in}{2.666353in}}%
\pgfpathcurveto{\pgfqpoint{2.534774in}{2.666353in}}{\pgfqpoint{2.545373in}{2.670743in}}{\pgfqpoint{2.553187in}{2.678557in}}%
\pgfpathcurveto{\pgfqpoint{2.561001in}{2.686370in}}{\pgfqpoint{2.565391in}{2.696969in}}{\pgfqpoint{2.565391in}{2.708020in}}%
\pgfpathcurveto{\pgfqpoint{2.565391in}{2.719070in}}{\pgfqpoint{2.561001in}{2.729669in}}{\pgfqpoint{2.553187in}{2.737482in}}%
\pgfpathcurveto{\pgfqpoint{2.545373in}{2.745296in}}{\pgfqpoint{2.534774in}{2.749686in}}{\pgfqpoint{2.523724in}{2.749686in}}%
\pgfpathcurveto{\pgfqpoint{2.512674in}{2.749686in}}{\pgfqpoint{2.502075in}{2.745296in}}{\pgfqpoint{2.494261in}{2.737482in}}%
\pgfpathcurveto{\pgfqpoint{2.486448in}{2.729669in}}{\pgfqpoint{2.482058in}{2.719070in}}{\pgfqpoint{2.482058in}{2.708020in}}%
\pgfpathcurveto{\pgfqpoint{2.482058in}{2.696969in}}{\pgfqpoint{2.486448in}{2.686370in}}{\pgfqpoint{2.494261in}{2.678557in}}%
\pgfpathcurveto{\pgfqpoint{2.502075in}{2.670743in}}{\pgfqpoint{2.512674in}{2.666353in}}{\pgfqpoint{2.523724in}{2.666353in}}%
\pgfpathclose%
\pgfusepath{stroke,fill}%
\end{pgfscope}%
\begin{pgfscope}%
\pgfpathrectangle{\pgfqpoint{0.600000in}{0.600000in}}{\pgfqpoint{3.900000in}{3.900000in}}%
\pgfusepath{clip}%
\pgfsetbuttcap%
\pgfsetroundjoin%
\definecolor{currentfill}{rgb}{0.121569,0.466667,0.705882}%
\pgfsetfillcolor{currentfill}%
\pgfsetlinewidth{1.003750pt}%
\definecolor{currentstroke}{rgb}{0.121569,0.466667,0.705882}%
\pgfsetstrokecolor{currentstroke}%
\pgfsetdash{}{0pt}%
\pgfpathmoveto{\pgfqpoint{3.108888in}{2.453849in}}%
\pgfpathcurveto{\pgfqpoint{3.119938in}{2.453849in}}{\pgfqpoint{3.130537in}{2.458239in}}{\pgfqpoint{3.138351in}{2.466052in}}%
\pgfpathcurveto{\pgfqpoint{3.146165in}{2.473866in}}{\pgfqpoint{3.150555in}{2.484465in}}{\pgfqpoint{3.150555in}{2.495515in}}%
\pgfpathcurveto{\pgfqpoint{3.150555in}{2.506565in}}{\pgfqpoint{3.146165in}{2.517164in}}{\pgfqpoint{3.138351in}{2.524978in}}%
\pgfpathcurveto{\pgfqpoint{3.130537in}{2.532792in}}{\pgfqpoint{3.119938in}{2.537182in}}{\pgfqpoint{3.108888in}{2.537182in}}%
\pgfpathcurveto{\pgfqpoint{3.097838in}{2.537182in}}{\pgfqpoint{3.087239in}{2.532792in}}{\pgfqpoint{3.079425in}{2.524978in}}%
\pgfpathcurveto{\pgfqpoint{3.071612in}{2.517164in}}{\pgfqpoint{3.067222in}{2.506565in}}{\pgfqpoint{3.067222in}{2.495515in}}%
\pgfpathcurveto{\pgfqpoint{3.067222in}{2.484465in}}{\pgfqpoint{3.071612in}{2.473866in}}{\pgfqpoint{3.079425in}{2.466052in}}%
\pgfpathcurveto{\pgfqpoint{3.087239in}{2.458239in}}{\pgfqpoint{3.097838in}{2.453849in}}{\pgfqpoint{3.108888in}{2.453849in}}%
\pgfpathclose%
\pgfusepath{stroke,fill}%
\end{pgfscope}%
\begin{pgfscope}%
\pgfpathrectangle{\pgfqpoint{0.600000in}{0.600000in}}{\pgfqpoint{3.900000in}{3.900000in}}%
\pgfusepath{clip}%
\pgfsetbuttcap%
\pgfsetroundjoin%
\definecolor{currentfill}{rgb}{0.121569,0.466667,0.705882}%
\pgfsetfillcolor{currentfill}%
\pgfsetlinewidth{1.003750pt}%
\definecolor{currentstroke}{rgb}{0.121569,0.466667,0.705882}%
\pgfsetstrokecolor{currentstroke}%
\pgfsetdash{}{0pt}%
\pgfpathmoveto{\pgfqpoint{2.309747in}{2.835405in}}%
\pgfpathcurveto{\pgfqpoint{2.320798in}{2.835405in}}{\pgfqpoint{2.331397in}{2.839795in}}{\pgfqpoint{2.339210in}{2.847609in}}%
\pgfpathcurveto{\pgfqpoint{2.347024in}{2.855422in}}{\pgfqpoint{2.351414in}{2.866022in}}{\pgfqpoint{2.351414in}{2.877072in}}%
\pgfpathcurveto{\pgfqpoint{2.351414in}{2.888122in}}{\pgfqpoint{2.347024in}{2.898721in}}{\pgfqpoint{2.339210in}{2.906534in}}%
\pgfpathcurveto{\pgfqpoint{2.331397in}{2.914348in}}{\pgfqpoint{2.320798in}{2.918738in}}{\pgfqpoint{2.309747in}{2.918738in}}%
\pgfpathcurveto{\pgfqpoint{2.298697in}{2.918738in}}{\pgfqpoint{2.288098in}{2.914348in}}{\pgfqpoint{2.280285in}{2.906534in}}%
\pgfpathcurveto{\pgfqpoint{2.272471in}{2.898721in}}{\pgfqpoint{2.268081in}{2.888122in}}{\pgfqpoint{2.268081in}{2.877072in}}%
\pgfpathcurveto{\pgfqpoint{2.268081in}{2.866022in}}{\pgfqpoint{2.272471in}{2.855422in}}{\pgfqpoint{2.280285in}{2.847609in}}%
\pgfpathcurveto{\pgfqpoint{2.288098in}{2.839795in}}{\pgfqpoint{2.298697in}{2.835405in}}{\pgfqpoint{2.309747in}{2.835405in}}%
\pgfpathclose%
\pgfusepath{stroke,fill}%
\end{pgfscope}%
\begin{pgfscope}%
\pgfpathrectangle{\pgfqpoint{0.600000in}{0.600000in}}{\pgfqpoint{3.900000in}{3.900000in}}%
\pgfusepath{clip}%
\pgfsetbuttcap%
\pgfsetroundjoin%
\definecolor{currentfill}{rgb}{0.121569,0.466667,0.705882}%
\pgfsetfillcolor{currentfill}%
\pgfsetlinewidth{1.003750pt}%
\definecolor{currentstroke}{rgb}{0.121569,0.466667,0.705882}%
\pgfsetstrokecolor{currentstroke}%
\pgfsetdash{}{0pt}%
\pgfpathmoveto{\pgfqpoint{2.910461in}{2.756704in}}%
\pgfpathcurveto{\pgfqpoint{2.921511in}{2.756704in}}{\pgfqpoint{2.932110in}{2.761094in}}{\pgfqpoint{2.939923in}{2.768908in}}%
\pgfpathcurveto{\pgfqpoint{2.947737in}{2.776721in}}{\pgfqpoint{2.952127in}{2.787320in}}{\pgfqpoint{2.952127in}{2.798370in}}%
\pgfpathcurveto{\pgfqpoint{2.952127in}{2.809421in}}{\pgfqpoint{2.947737in}{2.820020in}}{\pgfqpoint{2.939923in}{2.827833in}}%
\pgfpathcurveto{\pgfqpoint{2.932110in}{2.835647in}}{\pgfqpoint{2.921511in}{2.840037in}}{\pgfqpoint{2.910461in}{2.840037in}}%
\pgfpathcurveto{\pgfqpoint{2.899411in}{2.840037in}}{\pgfqpoint{2.888812in}{2.835647in}}{\pgfqpoint{2.880998in}{2.827833in}}%
\pgfpathcurveto{\pgfqpoint{2.873184in}{2.820020in}}{\pgfqpoint{2.868794in}{2.809421in}}{\pgfqpoint{2.868794in}{2.798370in}}%
\pgfpathcurveto{\pgfqpoint{2.868794in}{2.787320in}}{\pgfqpoint{2.873184in}{2.776721in}}{\pgfqpoint{2.880998in}{2.768908in}}%
\pgfpathcurveto{\pgfqpoint{2.888812in}{2.761094in}}{\pgfqpoint{2.899411in}{2.756704in}}{\pgfqpoint{2.910461in}{2.756704in}}%
\pgfpathclose%
\pgfusepath{stroke,fill}%
\end{pgfscope}%
\begin{pgfscope}%
\pgfpathrectangle{\pgfqpoint{0.600000in}{0.600000in}}{\pgfqpoint{3.900000in}{3.900000in}}%
\pgfusepath{clip}%
\pgfsetbuttcap%
\pgfsetroundjoin%
\definecolor{currentfill}{rgb}{0.121569,0.466667,0.705882}%
\pgfsetfillcolor{currentfill}%
\pgfsetlinewidth{1.003750pt}%
\definecolor{currentstroke}{rgb}{0.121569,0.466667,0.705882}%
\pgfsetstrokecolor{currentstroke}%
\pgfsetdash{}{0pt}%
\pgfpathmoveto{\pgfqpoint{1.984257in}{2.472656in}}%
\pgfpathcurveto{\pgfqpoint{1.995307in}{2.472656in}}{\pgfqpoint{2.005906in}{2.477046in}}{\pgfqpoint{2.013720in}{2.484860in}}%
\pgfpathcurveto{\pgfqpoint{2.021534in}{2.492673in}}{\pgfqpoint{2.025924in}{2.503272in}}{\pgfqpoint{2.025924in}{2.514322in}}%
\pgfpathcurveto{\pgfqpoint{2.025924in}{2.525372in}}{\pgfqpoint{2.021534in}{2.535972in}}{\pgfqpoint{2.013720in}{2.543785in}}%
\pgfpathcurveto{\pgfqpoint{2.005906in}{2.551599in}}{\pgfqpoint{1.995307in}{2.555989in}}{\pgfqpoint{1.984257in}{2.555989in}}%
\pgfpathcurveto{\pgfqpoint{1.973207in}{2.555989in}}{\pgfqpoint{1.962608in}{2.551599in}}{\pgfqpoint{1.954795in}{2.543785in}}%
\pgfpathcurveto{\pgfqpoint{1.946981in}{2.535972in}}{\pgfqpoint{1.942591in}{2.525372in}}{\pgfqpoint{1.942591in}{2.514322in}}%
\pgfpathcurveto{\pgfqpoint{1.942591in}{2.503272in}}{\pgfqpoint{1.946981in}{2.492673in}}{\pgfqpoint{1.954795in}{2.484860in}}%
\pgfpathcurveto{\pgfqpoint{1.962608in}{2.477046in}}{\pgfqpoint{1.973207in}{2.472656in}}{\pgfqpoint{1.984257in}{2.472656in}}%
\pgfpathclose%
\pgfusepath{stroke,fill}%
\end{pgfscope}%
\begin{pgfscope}%
\pgfpathrectangle{\pgfqpoint{0.600000in}{0.600000in}}{\pgfqpoint{3.900000in}{3.900000in}}%
\pgfusepath{clip}%
\pgfsetbuttcap%
\pgfsetroundjoin%
\definecolor{currentfill}{rgb}{0.121569,0.466667,0.705882}%
\pgfsetfillcolor{currentfill}%
\pgfsetlinewidth{1.003750pt}%
\definecolor{currentstroke}{rgb}{0.121569,0.466667,0.705882}%
\pgfsetstrokecolor{currentstroke}%
\pgfsetdash{}{0pt}%
\pgfpathmoveto{\pgfqpoint{3.321016in}{2.501806in}}%
\pgfpathcurveto{\pgfqpoint{3.332067in}{2.501806in}}{\pgfqpoint{3.342666in}{2.506196in}}{\pgfqpoint{3.350479in}{2.514009in}}%
\pgfpathcurveto{\pgfqpoint{3.358293in}{2.521823in}}{\pgfqpoint{3.362683in}{2.532422in}}{\pgfqpoint{3.362683in}{2.543472in}}%
\pgfpathcurveto{\pgfqpoint{3.362683in}{2.554522in}}{\pgfqpoint{3.358293in}{2.565121in}}{\pgfqpoint{3.350479in}{2.572935in}}%
\pgfpathcurveto{\pgfqpoint{3.342666in}{2.580749in}}{\pgfqpoint{3.332067in}{2.585139in}}{\pgfqpoint{3.321016in}{2.585139in}}%
\pgfpathcurveto{\pgfqpoint{3.309966in}{2.585139in}}{\pgfqpoint{3.299367in}{2.580749in}}{\pgfqpoint{3.291554in}{2.572935in}}%
\pgfpathcurveto{\pgfqpoint{3.283740in}{2.565121in}}{\pgfqpoint{3.279350in}{2.554522in}}{\pgfqpoint{3.279350in}{2.543472in}}%
\pgfpathcurveto{\pgfqpoint{3.279350in}{2.532422in}}{\pgfqpoint{3.283740in}{2.521823in}}{\pgfqpoint{3.291554in}{2.514009in}}%
\pgfpathcurveto{\pgfqpoint{3.299367in}{2.506196in}}{\pgfqpoint{3.309966in}{2.501806in}}{\pgfqpoint{3.321016in}{2.501806in}}%
\pgfpathclose%
\pgfusepath{stroke,fill}%
\end{pgfscope}%
\begin{pgfscope}%
\pgfpathrectangle{\pgfqpoint{0.600000in}{0.600000in}}{\pgfqpoint{3.900000in}{3.900000in}}%
\pgfusepath{clip}%
\pgfsetbuttcap%
\pgfsetroundjoin%
\definecolor{currentfill}{rgb}{0.121569,0.466667,0.705882}%
\pgfsetfillcolor{currentfill}%
\pgfsetlinewidth{1.003750pt}%
\definecolor{currentstroke}{rgb}{0.121569,0.466667,0.705882}%
\pgfsetstrokecolor{currentstroke}%
\pgfsetdash{}{0pt}%
\pgfpathmoveto{\pgfqpoint{3.597349in}{1.656988in}}%
\pgfpathcurveto{\pgfqpoint{3.608399in}{1.656988in}}{\pgfqpoint{3.618999in}{1.661378in}}{\pgfqpoint{3.626812in}{1.669191in}}%
\pgfpathcurveto{\pgfqpoint{3.634626in}{1.677005in}}{\pgfqpoint{3.639016in}{1.687604in}}{\pgfqpoint{3.639016in}{1.698654in}}%
\pgfpathcurveto{\pgfqpoint{3.639016in}{1.709704in}}{\pgfqpoint{3.634626in}{1.720303in}}{\pgfqpoint{3.626812in}{1.728117in}}%
\pgfpathcurveto{\pgfqpoint{3.618999in}{1.735931in}}{\pgfqpoint{3.608399in}{1.740321in}}{\pgfqpoint{3.597349in}{1.740321in}}%
\pgfpathcurveto{\pgfqpoint{3.586299in}{1.740321in}}{\pgfqpoint{3.575700in}{1.735931in}}{\pgfqpoint{3.567887in}{1.728117in}}%
\pgfpathcurveto{\pgfqpoint{3.560073in}{1.720303in}}{\pgfqpoint{3.555683in}{1.709704in}}{\pgfqpoint{3.555683in}{1.698654in}}%
\pgfpathcurveto{\pgfqpoint{3.555683in}{1.687604in}}{\pgfqpoint{3.560073in}{1.677005in}}{\pgfqpoint{3.567887in}{1.669191in}}%
\pgfpathcurveto{\pgfqpoint{3.575700in}{1.661378in}}{\pgfqpoint{3.586299in}{1.656988in}}{\pgfqpoint{3.597349in}{1.656988in}}%
\pgfpathclose%
\pgfusepath{stroke,fill}%
\end{pgfscope}%
\begin{pgfscope}%
\pgfpathrectangle{\pgfqpoint{0.600000in}{0.600000in}}{\pgfqpoint{3.900000in}{3.900000in}}%
\pgfusepath{clip}%
\pgfsetbuttcap%
\pgfsetroundjoin%
\definecolor{currentfill}{rgb}{0.121569,0.466667,0.705882}%
\pgfsetfillcolor{currentfill}%
\pgfsetlinewidth{1.003750pt}%
\definecolor{currentstroke}{rgb}{0.121569,0.466667,0.705882}%
\pgfsetstrokecolor{currentstroke}%
\pgfsetdash{}{0pt}%
\pgfpathmoveto{\pgfqpoint{2.446245in}{2.470795in}}%
\pgfpathcurveto{\pgfqpoint{2.457295in}{2.470795in}}{\pgfqpoint{2.467894in}{2.475186in}}{\pgfqpoint{2.475708in}{2.482999in}}%
\pgfpathcurveto{\pgfqpoint{2.483522in}{2.490813in}}{\pgfqpoint{2.487912in}{2.501412in}}{\pgfqpoint{2.487912in}{2.512462in}}%
\pgfpathcurveto{\pgfqpoint{2.487912in}{2.523512in}}{\pgfqpoint{2.483522in}{2.534111in}}{\pgfqpoint{2.475708in}{2.541925in}}%
\pgfpathcurveto{\pgfqpoint{2.467894in}{2.549739in}}{\pgfqpoint{2.457295in}{2.554129in}}{\pgfqpoint{2.446245in}{2.554129in}}%
\pgfpathcurveto{\pgfqpoint{2.435195in}{2.554129in}}{\pgfqpoint{2.424596in}{2.549739in}}{\pgfqpoint{2.416783in}{2.541925in}}%
\pgfpathcurveto{\pgfqpoint{2.408969in}{2.534111in}}{\pgfqpoint{2.404579in}{2.523512in}}{\pgfqpoint{2.404579in}{2.512462in}}%
\pgfpathcurveto{\pgfqpoint{2.404579in}{2.501412in}}{\pgfqpoint{2.408969in}{2.490813in}}{\pgfqpoint{2.416783in}{2.482999in}}%
\pgfpathcurveto{\pgfqpoint{2.424596in}{2.475186in}}{\pgfqpoint{2.435195in}{2.470795in}}{\pgfqpoint{2.446245in}{2.470795in}}%
\pgfpathclose%
\pgfusepath{stroke,fill}%
\end{pgfscope}%
\begin{pgfscope}%
\pgfpathrectangle{\pgfqpoint{0.600000in}{0.600000in}}{\pgfqpoint{3.900000in}{3.900000in}}%
\pgfusepath{clip}%
\pgfsetbuttcap%
\pgfsetroundjoin%
\definecolor{currentfill}{rgb}{0.121569,0.466667,0.705882}%
\pgfsetfillcolor{currentfill}%
\pgfsetlinewidth{1.003750pt}%
\definecolor{currentstroke}{rgb}{0.121569,0.466667,0.705882}%
\pgfsetstrokecolor{currentstroke}%
\pgfsetdash{}{0pt}%
\pgfpathmoveto{\pgfqpoint{1.712304in}{3.451071in}}%
\pgfpathcurveto{\pgfqpoint{1.723354in}{3.451071in}}{\pgfqpoint{1.733953in}{3.455461in}}{\pgfqpoint{1.741766in}{3.463275in}}%
\pgfpathcurveto{\pgfqpoint{1.749580in}{3.471088in}}{\pgfqpoint{1.753970in}{3.481687in}}{\pgfqpoint{1.753970in}{3.492737in}}%
\pgfpathcurveto{\pgfqpoint{1.753970in}{3.503788in}}{\pgfqpoint{1.749580in}{3.514387in}}{\pgfqpoint{1.741766in}{3.522200in}}%
\pgfpathcurveto{\pgfqpoint{1.733953in}{3.530014in}}{\pgfqpoint{1.723354in}{3.534404in}}{\pgfqpoint{1.712304in}{3.534404in}}%
\pgfpathcurveto{\pgfqpoint{1.701254in}{3.534404in}}{\pgfqpoint{1.690654in}{3.530014in}}{\pgfqpoint{1.682841in}{3.522200in}}%
\pgfpathcurveto{\pgfqpoint{1.675027in}{3.514387in}}{\pgfqpoint{1.670637in}{3.503788in}}{\pgfqpoint{1.670637in}{3.492737in}}%
\pgfpathcurveto{\pgfqpoint{1.670637in}{3.481687in}}{\pgfqpoint{1.675027in}{3.471088in}}{\pgfqpoint{1.682841in}{3.463275in}}%
\pgfpathcurveto{\pgfqpoint{1.690654in}{3.455461in}}{\pgfqpoint{1.701254in}{3.451071in}}{\pgfqpoint{1.712304in}{3.451071in}}%
\pgfpathclose%
\pgfusepath{stroke,fill}%
\end{pgfscope}%
\begin{pgfscope}%
\pgfpathrectangle{\pgfqpoint{0.600000in}{0.600000in}}{\pgfqpoint{3.900000in}{3.900000in}}%
\pgfusepath{clip}%
\pgfsetbuttcap%
\pgfsetroundjoin%
\definecolor{currentfill}{rgb}{0.121569,0.466667,0.705882}%
\pgfsetfillcolor{currentfill}%
\pgfsetlinewidth{1.003750pt}%
\definecolor{currentstroke}{rgb}{0.121569,0.466667,0.705882}%
\pgfsetstrokecolor{currentstroke}%
\pgfsetdash{}{0pt}%
\pgfpathmoveto{\pgfqpoint{2.235431in}{2.641884in}}%
\pgfpathcurveto{\pgfqpoint{2.246481in}{2.641884in}}{\pgfqpoint{2.257080in}{2.646274in}}{\pgfqpoint{2.264893in}{2.654088in}}%
\pgfpathcurveto{\pgfqpoint{2.272707in}{2.661902in}}{\pgfqpoint{2.277097in}{2.672501in}}{\pgfqpoint{2.277097in}{2.683551in}}%
\pgfpathcurveto{\pgfqpoint{2.277097in}{2.694601in}}{\pgfqpoint{2.272707in}{2.705200in}}{\pgfqpoint{2.264893in}{2.713013in}}%
\pgfpathcurveto{\pgfqpoint{2.257080in}{2.720827in}}{\pgfqpoint{2.246481in}{2.725217in}}{\pgfqpoint{2.235431in}{2.725217in}}%
\pgfpathcurveto{\pgfqpoint{2.224381in}{2.725217in}}{\pgfqpoint{2.213782in}{2.720827in}}{\pgfqpoint{2.205968in}{2.713013in}}%
\pgfpathcurveto{\pgfqpoint{2.198154in}{2.705200in}}{\pgfqpoint{2.193764in}{2.694601in}}{\pgfqpoint{2.193764in}{2.683551in}}%
\pgfpathcurveto{\pgfqpoint{2.193764in}{2.672501in}}{\pgfqpoint{2.198154in}{2.661902in}}{\pgfqpoint{2.205968in}{2.654088in}}%
\pgfpathcurveto{\pgfqpoint{2.213782in}{2.646274in}}{\pgfqpoint{2.224381in}{2.641884in}}{\pgfqpoint{2.235431in}{2.641884in}}%
\pgfpathclose%
\pgfusepath{stroke,fill}%
\end{pgfscope}%
\begin{pgfscope}%
\pgfpathrectangle{\pgfqpoint{0.600000in}{0.600000in}}{\pgfqpoint{3.900000in}{3.900000in}}%
\pgfusepath{clip}%
\pgfsetbuttcap%
\pgfsetroundjoin%
\definecolor{currentfill}{rgb}{0.121569,0.466667,0.705882}%
\pgfsetfillcolor{currentfill}%
\pgfsetlinewidth{1.003750pt}%
\definecolor{currentstroke}{rgb}{0.121569,0.466667,0.705882}%
\pgfsetstrokecolor{currentstroke}%
\pgfsetdash{}{0pt}%
\pgfpathmoveto{\pgfqpoint{2.796685in}{1.762682in}}%
\pgfpathcurveto{\pgfqpoint{2.807735in}{1.762682in}}{\pgfqpoint{2.818334in}{1.767073in}}{\pgfqpoint{2.826148in}{1.774886in}}%
\pgfpathcurveto{\pgfqpoint{2.833961in}{1.782700in}}{\pgfqpoint{2.838351in}{1.793299in}}{\pgfqpoint{2.838351in}{1.804349in}}%
\pgfpathcurveto{\pgfqpoint{2.838351in}{1.815399in}}{\pgfqpoint{2.833961in}{1.825998in}}{\pgfqpoint{2.826148in}{1.833812in}}%
\pgfpathcurveto{\pgfqpoint{2.818334in}{1.841625in}}{\pgfqpoint{2.807735in}{1.846016in}}{\pgfqpoint{2.796685in}{1.846016in}}%
\pgfpathcurveto{\pgfqpoint{2.785635in}{1.846016in}}{\pgfqpoint{2.775036in}{1.841625in}}{\pgfqpoint{2.767222in}{1.833812in}}%
\pgfpathcurveto{\pgfqpoint{2.759408in}{1.825998in}}{\pgfqpoint{2.755018in}{1.815399in}}{\pgfqpoint{2.755018in}{1.804349in}}%
\pgfpathcurveto{\pgfqpoint{2.755018in}{1.793299in}}{\pgfqpoint{2.759408in}{1.782700in}}{\pgfqpoint{2.767222in}{1.774886in}}%
\pgfpathcurveto{\pgfqpoint{2.775036in}{1.767073in}}{\pgfqpoint{2.785635in}{1.762682in}}{\pgfqpoint{2.796685in}{1.762682in}}%
\pgfpathclose%
\pgfusepath{stroke,fill}%
\end{pgfscope}%
\begin{pgfscope}%
\pgfpathrectangle{\pgfqpoint{0.600000in}{0.600000in}}{\pgfqpoint{3.900000in}{3.900000in}}%
\pgfusepath{clip}%
\pgfsetbuttcap%
\pgfsetroundjoin%
\definecolor{currentfill}{rgb}{0.121569,0.466667,0.705882}%
\pgfsetfillcolor{currentfill}%
\pgfsetlinewidth{1.003750pt}%
\definecolor{currentstroke}{rgb}{0.121569,0.466667,0.705882}%
\pgfsetstrokecolor{currentstroke}%
\pgfsetdash{}{0pt}%
\pgfpathmoveto{\pgfqpoint{2.940090in}{1.360062in}}%
\pgfpathcurveto{\pgfqpoint{2.951140in}{1.360062in}}{\pgfqpoint{2.961739in}{1.364452in}}{\pgfqpoint{2.969553in}{1.372266in}}%
\pgfpathcurveto{\pgfqpoint{2.977366in}{1.380080in}}{\pgfqpoint{2.981756in}{1.390679in}}{\pgfqpoint{2.981756in}{1.401729in}}%
\pgfpathcurveto{\pgfqpoint{2.981756in}{1.412779in}}{\pgfqpoint{2.977366in}{1.423378in}}{\pgfqpoint{2.969553in}{1.431191in}}%
\pgfpathcurveto{\pgfqpoint{2.961739in}{1.439005in}}{\pgfqpoint{2.951140in}{1.443395in}}{\pgfqpoint{2.940090in}{1.443395in}}%
\pgfpathcurveto{\pgfqpoint{2.929040in}{1.443395in}}{\pgfqpoint{2.918441in}{1.439005in}}{\pgfqpoint{2.910627in}{1.431191in}}%
\pgfpathcurveto{\pgfqpoint{2.902813in}{1.423378in}}{\pgfqpoint{2.898423in}{1.412779in}}{\pgfqpoint{2.898423in}{1.401729in}}%
\pgfpathcurveto{\pgfqpoint{2.898423in}{1.390679in}}{\pgfqpoint{2.902813in}{1.380080in}}{\pgfqpoint{2.910627in}{1.372266in}}%
\pgfpathcurveto{\pgfqpoint{2.918441in}{1.364452in}}{\pgfqpoint{2.929040in}{1.360062in}}{\pgfqpoint{2.940090in}{1.360062in}}%
\pgfpathclose%
\pgfusepath{stroke,fill}%
\end{pgfscope}%
\begin{pgfscope}%
\pgfpathrectangle{\pgfqpoint{0.600000in}{0.600000in}}{\pgfqpoint{3.900000in}{3.900000in}}%
\pgfusepath{clip}%
\pgfsetbuttcap%
\pgfsetroundjoin%
\definecolor{currentfill}{rgb}{0.121569,0.466667,0.705882}%
\pgfsetfillcolor{currentfill}%
\pgfsetlinewidth{1.003750pt}%
\definecolor{currentstroke}{rgb}{0.121569,0.466667,0.705882}%
\pgfsetstrokecolor{currentstroke}%
\pgfsetdash{}{0pt}%
\pgfpathmoveto{\pgfqpoint{3.230126in}{2.847554in}}%
\pgfpathcurveto{\pgfqpoint{3.241176in}{2.847554in}}{\pgfqpoint{3.251775in}{2.851944in}}{\pgfqpoint{3.259588in}{2.859758in}}%
\pgfpathcurveto{\pgfqpoint{3.267402in}{2.867572in}}{\pgfqpoint{3.271792in}{2.878171in}}{\pgfqpoint{3.271792in}{2.889221in}}%
\pgfpathcurveto{\pgfqpoint{3.271792in}{2.900271in}}{\pgfqpoint{3.267402in}{2.910870in}}{\pgfqpoint{3.259588in}{2.918684in}}%
\pgfpathcurveto{\pgfqpoint{3.251775in}{2.926497in}}{\pgfqpoint{3.241176in}{2.930888in}}{\pgfqpoint{3.230126in}{2.930888in}}%
\pgfpathcurveto{\pgfqpoint{3.219076in}{2.930888in}}{\pgfqpoint{3.208477in}{2.926497in}}{\pgfqpoint{3.200663in}{2.918684in}}%
\pgfpathcurveto{\pgfqpoint{3.192849in}{2.910870in}}{\pgfqpoint{3.188459in}{2.900271in}}{\pgfqpoint{3.188459in}{2.889221in}}%
\pgfpathcurveto{\pgfqpoint{3.188459in}{2.878171in}}{\pgfqpoint{3.192849in}{2.867572in}}{\pgfqpoint{3.200663in}{2.859758in}}%
\pgfpathcurveto{\pgfqpoint{3.208477in}{2.851944in}}{\pgfqpoint{3.219076in}{2.847554in}}{\pgfqpoint{3.230126in}{2.847554in}}%
\pgfpathclose%
\pgfusepath{stroke,fill}%
\end{pgfscope}%
\begin{pgfscope}%
\pgfpathrectangle{\pgfqpoint{0.600000in}{0.600000in}}{\pgfqpoint{3.900000in}{3.900000in}}%
\pgfusepath{clip}%
\pgfsetbuttcap%
\pgfsetroundjoin%
\definecolor{currentfill}{rgb}{0.121569,0.466667,0.705882}%
\pgfsetfillcolor{currentfill}%
\pgfsetlinewidth{1.003750pt}%
\definecolor{currentstroke}{rgb}{0.121569,0.466667,0.705882}%
\pgfsetstrokecolor{currentstroke}%
\pgfsetdash{}{0pt}%
\pgfpathmoveto{\pgfqpoint{2.934807in}{2.551134in}}%
\pgfpathcurveto{\pgfqpoint{2.945857in}{2.551134in}}{\pgfqpoint{2.956456in}{2.555525in}}{\pgfqpoint{2.964270in}{2.563338in}}%
\pgfpathcurveto{\pgfqpoint{2.972083in}{2.571152in}}{\pgfqpoint{2.976473in}{2.581751in}}{\pgfqpoint{2.976473in}{2.592801in}}%
\pgfpathcurveto{\pgfqpoint{2.976473in}{2.603851in}}{\pgfqpoint{2.972083in}{2.614450in}}{\pgfqpoint{2.964270in}{2.622264in}}%
\pgfpathcurveto{\pgfqpoint{2.956456in}{2.630078in}}{\pgfqpoint{2.945857in}{2.634468in}}{\pgfqpoint{2.934807in}{2.634468in}}%
\pgfpathcurveto{\pgfqpoint{2.923757in}{2.634468in}}{\pgfqpoint{2.913158in}{2.630078in}}{\pgfqpoint{2.905344in}{2.622264in}}%
\pgfpathcurveto{\pgfqpoint{2.897530in}{2.614450in}}{\pgfqpoint{2.893140in}{2.603851in}}{\pgfqpoint{2.893140in}{2.592801in}}%
\pgfpathcurveto{\pgfqpoint{2.893140in}{2.581751in}}{\pgfqpoint{2.897530in}{2.571152in}}{\pgfqpoint{2.905344in}{2.563338in}}%
\pgfpathcurveto{\pgfqpoint{2.913158in}{2.555525in}}{\pgfqpoint{2.923757in}{2.551134in}}{\pgfqpoint{2.934807in}{2.551134in}}%
\pgfpathclose%
\pgfusepath{stroke,fill}%
\end{pgfscope}%
\begin{pgfscope}%
\pgfpathrectangle{\pgfqpoint{0.600000in}{0.600000in}}{\pgfqpoint{3.900000in}{3.900000in}}%
\pgfusepath{clip}%
\pgfsetbuttcap%
\pgfsetroundjoin%
\definecolor{currentfill}{rgb}{0.121569,0.466667,0.705882}%
\pgfsetfillcolor{currentfill}%
\pgfsetlinewidth{1.003750pt}%
\definecolor{currentstroke}{rgb}{0.121569,0.466667,0.705882}%
\pgfsetstrokecolor{currentstroke}%
\pgfsetdash{}{0pt}%
\pgfpathmoveto{\pgfqpoint{1.819360in}{1.598593in}}%
\pgfpathcurveto{\pgfqpoint{1.830410in}{1.598593in}}{\pgfqpoint{1.841009in}{1.602983in}}{\pgfqpoint{1.848822in}{1.610797in}}%
\pgfpathcurveto{\pgfqpoint{1.856636in}{1.618610in}}{\pgfqpoint{1.861026in}{1.629210in}}{\pgfqpoint{1.861026in}{1.640260in}}%
\pgfpathcurveto{\pgfqpoint{1.861026in}{1.651310in}}{\pgfqpoint{1.856636in}{1.661909in}}{\pgfqpoint{1.848822in}{1.669722in}}%
\pgfpathcurveto{\pgfqpoint{1.841009in}{1.677536in}}{\pgfqpoint{1.830410in}{1.681926in}}{\pgfqpoint{1.819360in}{1.681926in}}%
\pgfpathcurveto{\pgfqpoint{1.808309in}{1.681926in}}{\pgfqpoint{1.797710in}{1.677536in}}{\pgfqpoint{1.789897in}{1.669722in}}%
\pgfpathcurveto{\pgfqpoint{1.782083in}{1.661909in}}{\pgfqpoint{1.777693in}{1.651310in}}{\pgfqpoint{1.777693in}{1.640260in}}%
\pgfpathcurveto{\pgfqpoint{1.777693in}{1.629210in}}{\pgfqpoint{1.782083in}{1.618610in}}{\pgfqpoint{1.789897in}{1.610797in}}%
\pgfpathcurveto{\pgfqpoint{1.797710in}{1.602983in}}{\pgfqpoint{1.808309in}{1.598593in}}{\pgfqpoint{1.819360in}{1.598593in}}%
\pgfpathclose%
\pgfusepath{stroke,fill}%
\end{pgfscope}%
\begin{pgfscope}%
\pgfpathrectangle{\pgfqpoint{0.600000in}{0.600000in}}{\pgfqpoint{3.900000in}{3.900000in}}%
\pgfusepath{clip}%
\pgfsetbuttcap%
\pgfsetroundjoin%
\definecolor{currentfill}{rgb}{0.121569,0.466667,0.705882}%
\pgfsetfillcolor{currentfill}%
\pgfsetlinewidth{1.003750pt}%
\definecolor{currentstroke}{rgb}{0.121569,0.466667,0.705882}%
\pgfsetstrokecolor{currentstroke}%
\pgfsetdash{}{0pt}%
\pgfpathmoveto{\pgfqpoint{2.049506in}{2.190720in}}%
\pgfpathcurveto{\pgfqpoint{2.060556in}{2.190720in}}{\pgfqpoint{2.071155in}{2.195111in}}{\pgfqpoint{2.078969in}{2.202924in}}%
\pgfpathcurveto{\pgfqpoint{2.086782in}{2.210738in}}{\pgfqpoint{2.091173in}{2.221337in}}{\pgfqpoint{2.091173in}{2.232387in}}%
\pgfpathcurveto{\pgfqpoint{2.091173in}{2.243437in}}{\pgfqpoint{2.086782in}{2.254036in}}{\pgfqpoint{2.078969in}{2.261850in}}%
\pgfpathcurveto{\pgfqpoint{2.071155in}{2.269664in}}{\pgfqpoint{2.060556in}{2.274054in}}{\pgfqpoint{2.049506in}{2.274054in}}%
\pgfpathcurveto{\pgfqpoint{2.038456in}{2.274054in}}{\pgfqpoint{2.027857in}{2.269664in}}{\pgfqpoint{2.020043in}{2.261850in}}%
\pgfpathcurveto{\pgfqpoint{2.012229in}{2.254036in}}{\pgfqpoint{2.007839in}{2.243437in}}{\pgfqpoint{2.007839in}{2.232387in}}%
\pgfpathcurveto{\pgfqpoint{2.007839in}{2.221337in}}{\pgfqpoint{2.012229in}{2.210738in}}{\pgfqpoint{2.020043in}{2.202924in}}%
\pgfpathcurveto{\pgfqpoint{2.027857in}{2.195111in}}{\pgfqpoint{2.038456in}{2.190720in}}{\pgfqpoint{2.049506in}{2.190720in}}%
\pgfpathclose%
\pgfusepath{stroke,fill}%
\end{pgfscope}%
\begin{pgfscope}%
\pgfpathrectangle{\pgfqpoint{0.600000in}{0.600000in}}{\pgfqpoint{3.900000in}{3.900000in}}%
\pgfusepath{clip}%
\pgfsetbuttcap%
\pgfsetroundjoin%
\definecolor{currentfill}{rgb}{0.121569,0.466667,0.705882}%
\pgfsetfillcolor{currentfill}%
\pgfsetlinewidth{1.003750pt}%
\definecolor{currentstroke}{rgb}{0.121569,0.466667,0.705882}%
\pgfsetstrokecolor{currentstroke}%
\pgfsetdash{}{0pt}%
\pgfpathmoveto{\pgfqpoint{1.889938in}{2.760533in}}%
\pgfpathcurveto{\pgfqpoint{1.900988in}{2.760533in}}{\pgfqpoint{1.911587in}{2.764924in}}{\pgfqpoint{1.919401in}{2.772737in}}%
\pgfpathcurveto{\pgfqpoint{1.927215in}{2.780551in}}{\pgfqpoint{1.931605in}{2.791150in}}{\pgfqpoint{1.931605in}{2.802200in}}%
\pgfpathcurveto{\pgfqpoint{1.931605in}{2.813250in}}{\pgfqpoint{1.927215in}{2.823849in}}{\pgfqpoint{1.919401in}{2.831663in}}%
\pgfpathcurveto{\pgfqpoint{1.911587in}{2.839476in}}{\pgfqpoint{1.900988in}{2.843867in}}{\pgfqpoint{1.889938in}{2.843867in}}%
\pgfpathcurveto{\pgfqpoint{1.878888in}{2.843867in}}{\pgfqpoint{1.868289in}{2.839476in}}{\pgfqpoint{1.860475in}{2.831663in}}%
\pgfpathcurveto{\pgfqpoint{1.852662in}{2.823849in}}{\pgfqpoint{1.848272in}{2.813250in}}{\pgfqpoint{1.848272in}{2.802200in}}%
\pgfpathcurveto{\pgfqpoint{1.848272in}{2.791150in}}{\pgfqpoint{1.852662in}{2.780551in}}{\pgfqpoint{1.860475in}{2.772737in}}%
\pgfpathcurveto{\pgfqpoint{1.868289in}{2.764924in}}{\pgfqpoint{1.878888in}{2.760533in}}{\pgfqpoint{1.889938in}{2.760533in}}%
\pgfpathclose%
\pgfusepath{stroke,fill}%
\end{pgfscope}%
\begin{pgfscope}%
\pgfpathrectangle{\pgfqpoint{0.600000in}{0.600000in}}{\pgfqpoint{3.900000in}{3.900000in}}%
\pgfusepath{clip}%
\pgfsetbuttcap%
\pgfsetroundjoin%
\definecolor{currentfill}{rgb}{0.121569,0.466667,0.705882}%
\pgfsetfillcolor{currentfill}%
\pgfsetlinewidth{1.003750pt}%
\definecolor{currentstroke}{rgb}{0.121569,0.466667,0.705882}%
\pgfsetstrokecolor{currentstroke}%
\pgfsetdash{}{0pt}%
\pgfpathmoveto{\pgfqpoint{2.344697in}{1.964297in}}%
\pgfpathcurveto{\pgfqpoint{2.355747in}{1.964297in}}{\pgfqpoint{2.366346in}{1.968687in}}{\pgfqpoint{2.374160in}{1.976500in}}%
\pgfpathcurveto{\pgfqpoint{2.381974in}{1.984314in}}{\pgfqpoint{2.386364in}{1.994913in}}{\pgfqpoint{2.386364in}{2.005963in}}%
\pgfpathcurveto{\pgfqpoint{2.386364in}{2.017013in}}{\pgfqpoint{2.381974in}{2.027612in}}{\pgfqpoint{2.374160in}{2.035426in}}%
\pgfpathcurveto{\pgfqpoint{2.366346in}{2.043240in}}{\pgfqpoint{2.355747in}{2.047630in}}{\pgfqpoint{2.344697in}{2.047630in}}%
\pgfpathcurveto{\pgfqpoint{2.333647in}{2.047630in}}{\pgfqpoint{2.323048in}{2.043240in}}{\pgfqpoint{2.315234in}{2.035426in}}%
\pgfpathcurveto{\pgfqpoint{2.307421in}{2.027612in}}{\pgfqpoint{2.303031in}{2.017013in}}{\pgfqpoint{2.303031in}{2.005963in}}%
\pgfpathcurveto{\pgfqpoint{2.303031in}{1.994913in}}{\pgfqpoint{2.307421in}{1.984314in}}{\pgfqpoint{2.315234in}{1.976500in}}%
\pgfpathcurveto{\pgfqpoint{2.323048in}{1.968687in}}{\pgfqpoint{2.333647in}{1.964297in}}{\pgfqpoint{2.344697in}{1.964297in}}%
\pgfpathclose%
\pgfusepath{stroke,fill}%
\end{pgfscope}%
\begin{pgfscope}%
\pgfpathrectangle{\pgfqpoint{0.600000in}{0.600000in}}{\pgfqpoint{3.900000in}{3.900000in}}%
\pgfusepath{clip}%
\pgfsetbuttcap%
\pgfsetroundjoin%
\definecolor{currentfill}{rgb}{0.121569,0.466667,0.705882}%
\pgfsetfillcolor{currentfill}%
\pgfsetlinewidth{1.003750pt}%
\definecolor{currentstroke}{rgb}{0.121569,0.466667,0.705882}%
\pgfsetstrokecolor{currentstroke}%
\pgfsetdash{}{0pt}%
\pgfpathmoveto{\pgfqpoint{2.456257in}{3.222382in}}%
\pgfpathcurveto{\pgfqpoint{2.467307in}{3.222382in}}{\pgfqpoint{2.477906in}{3.226772in}}{\pgfqpoint{2.485719in}{3.234586in}}%
\pgfpathcurveto{\pgfqpoint{2.493533in}{3.242400in}}{\pgfqpoint{2.497923in}{3.252999in}}{\pgfqpoint{2.497923in}{3.264049in}}%
\pgfpathcurveto{\pgfqpoint{2.497923in}{3.275099in}}{\pgfqpoint{2.493533in}{3.285698in}}{\pgfqpoint{2.485719in}{3.293512in}}%
\pgfpathcurveto{\pgfqpoint{2.477906in}{3.301325in}}{\pgfqpoint{2.467307in}{3.305715in}}{\pgfqpoint{2.456257in}{3.305715in}}%
\pgfpathcurveto{\pgfqpoint{2.445206in}{3.305715in}}{\pgfqpoint{2.434607in}{3.301325in}}{\pgfqpoint{2.426794in}{3.293512in}}%
\pgfpathcurveto{\pgfqpoint{2.418980in}{3.285698in}}{\pgfqpoint{2.414590in}{3.275099in}}{\pgfqpoint{2.414590in}{3.264049in}}%
\pgfpathcurveto{\pgfqpoint{2.414590in}{3.252999in}}{\pgfqpoint{2.418980in}{3.242400in}}{\pgfqpoint{2.426794in}{3.234586in}}%
\pgfpathcurveto{\pgfqpoint{2.434607in}{3.226772in}}{\pgfqpoint{2.445206in}{3.222382in}}{\pgfqpoint{2.456257in}{3.222382in}}%
\pgfpathclose%
\pgfusepath{stroke,fill}%
\end{pgfscope}%
\begin{pgfscope}%
\pgfpathrectangle{\pgfqpoint{0.600000in}{0.600000in}}{\pgfqpoint{3.900000in}{3.900000in}}%
\pgfusepath{clip}%
\pgfsetbuttcap%
\pgfsetroundjoin%
\definecolor{currentfill}{rgb}{0.121569,0.466667,0.705882}%
\pgfsetfillcolor{currentfill}%
\pgfsetlinewidth{1.003750pt}%
\definecolor{currentstroke}{rgb}{0.121569,0.466667,0.705882}%
\pgfsetstrokecolor{currentstroke}%
\pgfsetdash{}{0pt}%
\pgfpathmoveto{\pgfqpoint{2.116398in}{2.806836in}}%
\pgfpathcurveto{\pgfqpoint{2.127448in}{2.806836in}}{\pgfqpoint{2.138047in}{2.811226in}}{\pgfqpoint{2.145861in}{2.819040in}}%
\pgfpathcurveto{\pgfqpoint{2.153675in}{2.826854in}}{\pgfqpoint{2.158065in}{2.837453in}}{\pgfqpoint{2.158065in}{2.848503in}}%
\pgfpathcurveto{\pgfqpoint{2.158065in}{2.859553in}}{\pgfqpoint{2.153675in}{2.870152in}}{\pgfqpoint{2.145861in}{2.877966in}}%
\pgfpathcurveto{\pgfqpoint{2.138047in}{2.885779in}}{\pgfqpoint{2.127448in}{2.890170in}}{\pgfqpoint{2.116398in}{2.890170in}}%
\pgfpathcurveto{\pgfqpoint{2.105348in}{2.890170in}}{\pgfqpoint{2.094749in}{2.885779in}}{\pgfqpoint{2.086935in}{2.877966in}}%
\pgfpathcurveto{\pgfqpoint{2.079122in}{2.870152in}}{\pgfqpoint{2.074731in}{2.859553in}}{\pgfqpoint{2.074731in}{2.848503in}}%
\pgfpathcurveto{\pgfqpoint{2.074731in}{2.837453in}}{\pgfqpoint{2.079122in}{2.826854in}}{\pgfqpoint{2.086935in}{2.819040in}}%
\pgfpathcurveto{\pgfqpoint{2.094749in}{2.811226in}}{\pgfqpoint{2.105348in}{2.806836in}}{\pgfqpoint{2.116398in}{2.806836in}}%
\pgfpathclose%
\pgfusepath{stroke,fill}%
\end{pgfscope}%
\begin{pgfscope}%
\pgfpathrectangle{\pgfqpoint{0.600000in}{0.600000in}}{\pgfqpoint{3.900000in}{3.900000in}}%
\pgfusepath{clip}%
\pgfsetbuttcap%
\pgfsetroundjoin%
\definecolor{currentfill}{rgb}{0.121569,0.466667,0.705882}%
\pgfsetfillcolor{currentfill}%
\pgfsetlinewidth{1.003750pt}%
\definecolor{currentstroke}{rgb}{0.121569,0.466667,0.705882}%
\pgfsetstrokecolor{currentstroke}%
\pgfsetdash{}{0pt}%
\pgfpathmoveto{\pgfqpoint{2.511639in}{1.918868in}}%
\pgfpathcurveto{\pgfqpoint{2.522690in}{1.918868in}}{\pgfqpoint{2.533289in}{1.923258in}}{\pgfqpoint{2.541102in}{1.931072in}}%
\pgfpathcurveto{\pgfqpoint{2.548916in}{1.938885in}}{\pgfqpoint{2.553306in}{1.949484in}}{\pgfqpoint{2.553306in}{1.960535in}}%
\pgfpathcurveto{\pgfqpoint{2.553306in}{1.971585in}}{\pgfqpoint{2.548916in}{1.982184in}}{\pgfqpoint{2.541102in}{1.989997in}}%
\pgfpathcurveto{\pgfqpoint{2.533289in}{1.997811in}}{\pgfqpoint{2.522690in}{2.002201in}}{\pgfqpoint{2.511639in}{2.002201in}}%
\pgfpathcurveto{\pgfqpoint{2.500589in}{2.002201in}}{\pgfqpoint{2.489990in}{1.997811in}}{\pgfqpoint{2.482177in}{1.989997in}}%
\pgfpathcurveto{\pgfqpoint{2.474363in}{1.982184in}}{\pgfqpoint{2.469973in}{1.971585in}}{\pgfqpoint{2.469973in}{1.960535in}}%
\pgfpathcurveto{\pgfqpoint{2.469973in}{1.949484in}}{\pgfqpoint{2.474363in}{1.938885in}}{\pgfqpoint{2.482177in}{1.931072in}}%
\pgfpathcurveto{\pgfqpoint{2.489990in}{1.923258in}}{\pgfqpoint{2.500589in}{1.918868in}}{\pgfqpoint{2.511639in}{1.918868in}}%
\pgfpathclose%
\pgfusepath{stroke,fill}%
\end{pgfscope}%
\begin{pgfscope}%
\pgfpathrectangle{\pgfqpoint{0.600000in}{0.600000in}}{\pgfqpoint{3.900000in}{3.900000in}}%
\pgfusepath{clip}%
\pgfsetbuttcap%
\pgfsetroundjoin%
\definecolor{currentfill}{rgb}{0.121569,0.466667,0.705882}%
\pgfsetfillcolor{currentfill}%
\pgfsetlinewidth{1.003750pt}%
\definecolor{currentstroke}{rgb}{0.121569,0.466667,0.705882}%
\pgfsetstrokecolor{currentstroke}%
\pgfsetdash{}{0pt}%
\pgfpathmoveto{\pgfqpoint{2.898963in}{1.836203in}}%
\pgfpathcurveto{\pgfqpoint{2.910014in}{1.836203in}}{\pgfqpoint{2.920613in}{1.840594in}}{\pgfqpoint{2.928426in}{1.848407in}}%
\pgfpathcurveto{\pgfqpoint{2.936240in}{1.856221in}}{\pgfqpoint{2.940630in}{1.866820in}}{\pgfqpoint{2.940630in}{1.877870in}}%
\pgfpathcurveto{\pgfqpoint{2.940630in}{1.888920in}}{\pgfqpoint{2.936240in}{1.899519in}}{\pgfqpoint{2.928426in}{1.907333in}}%
\pgfpathcurveto{\pgfqpoint{2.920613in}{1.915146in}}{\pgfqpoint{2.910014in}{1.919537in}}{\pgfqpoint{2.898963in}{1.919537in}}%
\pgfpathcurveto{\pgfqpoint{2.887913in}{1.919537in}}{\pgfqpoint{2.877314in}{1.915146in}}{\pgfqpoint{2.869501in}{1.907333in}}%
\pgfpathcurveto{\pgfqpoint{2.861687in}{1.899519in}}{\pgfqpoint{2.857297in}{1.888920in}}{\pgfqpoint{2.857297in}{1.877870in}}%
\pgfpathcurveto{\pgfqpoint{2.857297in}{1.866820in}}{\pgfqpoint{2.861687in}{1.856221in}}{\pgfqpoint{2.869501in}{1.848407in}}%
\pgfpathcurveto{\pgfqpoint{2.877314in}{1.840594in}}{\pgfqpoint{2.887913in}{1.836203in}}{\pgfqpoint{2.898963in}{1.836203in}}%
\pgfpathclose%
\pgfusepath{stroke,fill}%
\end{pgfscope}%
\begin{pgfscope}%
\pgfpathrectangle{\pgfqpoint{0.600000in}{0.600000in}}{\pgfqpoint{3.900000in}{3.900000in}}%
\pgfusepath{clip}%
\pgfsetbuttcap%
\pgfsetroundjoin%
\definecolor{currentfill}{rgb}{0.121569,0.466667,0.705882}%
\pgfsetfillcolor{currentfill}%
\pgfsetlinewidth{1.003750pt}%
\definecolor{currentstroke}{rgb}{0.121569,0.466667,0.705882}%
\pgfsetstrokecolor{currentstroke}%
\pgfsetdash{}{0pt}%
\pgfpathmoveto{\pgfqpoint{2.339501in}{2.643366in}}%
\pgfpathcurveto{\pgfqpoint{2.350551in}{2.643366in}}{\pgfqpoint{2.361150in}{2.647756in}}{\pgfqpoint{2.368964in}{2.655570in}}%
\pgfpathcurveto{\pgfqpoint{2.376777in}{2.663383in}}{\pgfqpoint{2.381167in}{2.673982in}}{\pgfqpoint{2.381167in}{2.685032in}}%
\pgfpathcurveto{\pgfqpoint{2.381167in}{2.696083in}}{\pgfqpoint{2.376777in}{2.706682in}}{\pgfqpoint{2.368964in}{2.714495in}}%
\pgfpathcurveto{\pgfqpoint{2.361150in}{2.722309in}}{\pgfqpoint{2.350551in}{2.726699in}}{\pgfqpoint{2.339501in}{2.726699in}}%
\pgfpathcurveto{\pgfqpoint{2.328451in}{2.726699in}}{\pgfqpoint{2.317852in}{2.722309in}}{\pgfqpoint{2.310038in}{2.714495in}}%
\pgfpathcurveto{\pgfqpoint{2.302224in}{2.706682in}}{\pgfqpoint{2.297834in}{2.696083in}}{\pgfqpoint{2.297834in}{2.685032in}}%
\pgfpathcurveto{\pgfqpoint{2.297834in}{2.673982in}}{\pgfqpoint{2.302224in}{2.663383in}}{\pgfqpoint{2.310038in}{2.655570in}}%
\pgfpathcurveto{\pgfqpoint{2.317852in}{2.647756in}}{\pgfqpoint{2.328451in}{2.643366in}}{\pgfqpoint{2.339501in}{2.643366in}}%
\pgfpathclose%
\pgfusepath{stroke,fill}%
\end{pgfscope}%
\begin{pgfscope}%
\pgfpathrectangle{\pgfqpoint{0.600000in}{0.600000in}}{\pgfqpoint{3.900000in}{3.900000in}}%
\pgfusepath{clip}%
\pgfsetbuttcap%
\pgfsetroundjoin%
\definecolor{currentfill}{rgb}{0.121569,0.466667,0.705882}%
\pgfsetfillcolor{currentfill}%
\pgfsetlinewidth{1.003750pt}%
\definecolor{currentstroke}{rgb}{0.121569,0.466667,0.705882}%
\pgfsetstrokecolor{currentstroke}%
\pgfsetdash{}{0pt}%
\pgfpathmoveto{\pgfqpoint{1.480833in}{3.016287in}}%
\pgfpathcurveto{\pgfqpoint{1.491884in}{3.016287in}}{\pgfqpoint{1.502483in}{3.020678in}}{\pgfqpoint{1.510296in}{3.028491in}}%
\pgfpathcurveto{\pgfqpoint{1.518110in}{3.036305in}}{\pgfqpoint{1.522500in}{3.046904in}}{\pgfqpoint{1.522500in}{3.057954in}}%
\pgfpathcurveto{\pgfqpoint{1.522500in}{3.069004in}}{\pgfqpoint{1.518110in}{3.079603in}}{\pgfqpoint{1.510296in}{3.087417in}}%
\pgfpathcurveto{\pgfqpoint{1.502483in}{3.095231in}}{\pgfqpoint{1.491884in}{3.099621in}}{\pgfqpoint{1.480833in}{3.099621in}}%
\pgfpathcurveto{\pgfqpoint{1.469783in}{3.099621in}}{\pgfqpoint{1.459184in}{3.095231in}}{\pgfqpoint{1.451371in}{3.087417in}}%
\pgfpathcurveto{\pgfqpoint{1.443557in}{3.079603in}}{\pgfqpoint{1.439167in}{3.069004in}}{\pgfqpoint{1.439167in}{3.057954in}}%
\pgfpathcurveto{\pgfqpoint{1.439167in}{3.046904in}}{\pgfqpoint{1.443557in}{3.036305in}}{\pgfqpoint{1.451371in}{3.028491in}}%
\pgfpathcurveto{\pgfqpoint{1.459184in}{3.020678in}}{\pgfqpoint{1.469783in}{3.016287in}}{\pgfqpoint{1.480833in}{3.016287in}}%
\pgfpathclose%
\pgfusepath{stroke,fill}%
\end{pgfscope}%
\begin{pgfscope}%
\pgfpathrectangle{\pgfqpoint{0.600000in}{0.600000in}}{\pgfqpoint{3.900000in}{3.900000in}}%
\pgfusepath{clip}%
\pgfsetbuttcap%
\pgfsetroundjoin%
\definecolor{currentfill}{rgb}{0.121569,0.466667,0.705882}%
\pgfsetfillcolor{currentfill}%
\pgfsetlinewidth{1.003750pt}%
\definecolor{currentstroke}{rgb}{0.121569,0.466667,0.705882}%
\pgfsetstrokecolor{currentstroke}%
\pgfsetdash{}{0pt}%
\pgfpathmoveto{\pgfqpoint{2.195227in}{2.405185in}}%
\pgfpathcurveto{\pgfqpoint{2.206277in}{2.405185in}}{\pgfqpoint{2.216876in}{2.409575in}}{\pgfqpoint{2.224689in}{2.417389in}}%
\pgfpathcurveto{\pgfqpoint{2.232503in}{2.425202in}}{\pgfqpoint{2.236893in}{2.435801in}}{\pgfqpoint{2.236893in}{2.446852in}}%
\pgfpathcurveto{\pgfqpoint{2.236893in}{2.457902in}}{\pgfqpoint{2.232503in}{2.468501in}}{\pgfqpoint{2.224689in}{2.476314in}}%
\pgfpathcurveto{\pgfqpoint{2.216876in}{2.484128in}}{\pgfqpoint{2.206277in}{2.488518in}}{\pgfqpoint{2.195227in}{2.488518in}}%
\pgfpathcurveto{\pgfqpoint{2.184177in}{2.488518in}}{\pgfqpoint{2.173578in}{2.484128in}}{\pgfqpoint{2.165764in}{2.476314in}}%
\pgfpathcurveto{\pgfqpoint{2.157950in}{2.468501in}}{\pgfqpoint{2.153560in}{2.457902in}}{\pgfqpoint{2.153560in}{2.446852in}}%
\pgfpathcurveto{\pgfqpoint{2.153560in}{2.435801in}}{\pgfqpoint{2.157950in}{2.425202in}}{\pgfqpoint{2.165764in}{2.417389in}}%
\pgfpathcurveto{\pgfqpoint{2.173578in}{2.409575in}}{\pgfqpoint{2.184177in}{2.405185in}}{\pgfqpoint{2.195227in}{2.405185in}}%
\pgfpathclose%
\pgfusepath{stroke,fill}%
\end{pgfscope}%
\begin{pgfscope}%
\pgfpathrectangle{\pgfqpoint{0.600000in}{0.600000in}}{\pgfqpoint{3.900000in}{3.900000in}}%
\pgfusepath{clip}%
\pgfsetbuttcap%
\pgfsetroundjoin%
\definecolor{currentfill}{rgb}{0.121569,0.466667,0.705882}%
\pgfsetfillcolor{currentfill}%
\pgfsetlinewidth{1.003750pt}%
\definecolor{currentstroke}{rgb}{0.121569,0.466667,0.705882}%
\pgfsetstrokecolor{currentstroke}%
\pgfsetdash{}{0pt}%
\pgfpathmoveto{\pgfqpoint{2.826293in}{3.147554in}}%
\pgfpathcurveto{\pgfqpoint{2.837343in}{3.147554in}}{\pgfqpoint{2.847942in}{3.151945in}}{\pgfqpoint{2.855756in}{3.159758in}}%
\pgfpathcurveto{\pgfqpoint{2.863569in}{3.167572in}}{\pgfqpoint{2.867959in}{3.178171in}}{\pgfqpoint{2.867959in}{3.189221in}}%
\pgfpathcurveto{\pgfqpoint{2.867959in}{3.200271in}}{\pgfqpoint{2.863569in}{3.210870in}}{\pgfqpoint{2.855756in}{3.218684in}}%
\pgfpathcurveto{\pgfqpoint{2.847942in}{3.226497in}}{\pgfqpoint{2.837343in}{3.230888in}}{\pgfqpoint{2.826293in}{3.230888in}}%
\pgfpathcurveto{\pgfqpoint{2.815243in}{3.230888in}}{\pgfqpoint{2.804644in}{3.226497in}}{\pgfqpoint{2.796830in}{3.218684in}}%
\pgfpathcurveto{\pgfqpoint{2.789016in}{3.210870in}}{\pgfqpoint{2.784626in}{3.200271in}}{\pgfqpoint{2.784626in}{3.189221in}}%
\pgfpathcurveto{\pgfqpoint{2.784626in}{3.178171in}}{\pgfqpoint{2.789016in}{3.167572in}}{\pgfqpoint{2.796830in}{3.159758in}}%
\pgfpathcurveto{\pgfqpoint{2.804644in}{3.151945in}}{\pgfqpoint{2.815243in}{3.147554in}}{\pgfqpoint{2.826293in}{3.147554in}}%
\pgfpathclose%
\pgfusepath{stroke,fill}%
\end{pgfscope}%
\begin{pgfscope}%
\pgfpathrectangle{\pgfqpoint{0.600000in}{0.600000in}}{\pgfqpoint{3.900000in}{3.900000in}}%
\pgfusepath{clip}%
\pgfsetbuttcap%
\pgfsetroundjoin%
\definecolor{currentfill}{rgb}{0.121569,0.466667,0.705882}%
\pgfsetfillcolor{currentfill}%
\pgfsetlinewidth{1.003750pt}%
\definecolor{currentstroke}{rgb}{0.121569,0.466667,0.705882}%
\pgfsetstrokecolor{currentstroke}%
\pgfsetdash{}{0pt}%
\pgfpathmoveto{\pgfqpoint{2.396142in}{3.142709in}}%
\pgfpathcurveto{\pgfqpoint{2.407193in}{3.142709in}}{\pgfqpoint{2.417792in}{3.147100in}}{\pgfqpoint{2.425605in}{3.154913in}}%
\pgfpathcurveto{\pgfqpoint{2.433419in}{3.162727in}}{\pgfqpoint{2.437809in}{3.173326in}}{\pgfqpoint{2.437809in}{3.184376in}}%
\pgfpathcurveto{\pgfqpoint{2.437809in}{3.195426in}}{\pgfqpoint{2.433419in}{3.206025in}}{\pgfqpoint{2.425605in}{3.213839in}}%
\pgfpathcurveto{\pgfqpoint{2.417792in}{3.221653in}}{\pgfqpoint{2.407193in}{3.226043in}}{\pgfqpoint{2.396142in}{3.226043in}}%
\pgfpathcurveto{\pgfqpoint{2.385092in}{3.226043in}}{\pgfqpoint{2.374493in}{3.221653in}}{\pgfqpoint{2.366680in}{3.213839in}}%
\pgfpathcurveto{\pgfqpoint{2.358866in}{3.206025in}}{\pgfqpoint{2.354476in}{3.195426in}}{\pgfqpoint{2.354476in}{3.184376in}}%
\pgfpathcurveto{\pgfqpoint{2.354476in}{3.173326in}}{\pgfqpoint{2.358866in}{3.162727in}}{\pgfqpoint{2.366680in}{3.154913in}}%
\pgfpathcurveto{\pgfqpoint{2.374493in}{3.147100in}}{\pgfqpoint{2.385092in}{3.142709in}}{\pgfqpoint{2.396142in}{3.142709in}}%
\pgfpathclose%
\pgfusepath{stroke,fill}%
\end{pgfscope}%
\begin{pgfscope}%
\pgfpathrectangle{\pgfqpoint{0.600000in}{0.600000in}}{\pgfqpoint{3.900000in}{3.900000in}}%
\pgfusepath{clip}%
\pgfsetbuttcap%
\pgfsetroundjoin%
\definecolor{currentfill}{rgb}{0.121569,0.466667,0.705882}%
\pgfsetfillcolor{currentfill}%
\pgfsetlinewidth{1.003750pt}%
\definecolor{currentstroke}{rgb}{0.121569,0.466667,0.705882}%
\pgfsetstrokecolor{currentstroke}%
\pgfsetdash{}{0pt}%
\pgfpathmoveto{\pgfqpoint{2.748370in}{3.380275in}}%
\pgfpathcurveto{\pgfqpoint{2.759420in}{3.380275in}}{\pgfqpoint{2.770019in}{3.384665in}}{\pgfqpoint{2.777833in}{3.392479in}}%
\pgfpathcurveto{\pgfqpoint{2.785647in}{3.400292in}}{\pgfqpoint{2.790037in}{3.410891in}}{\pgfqpoint{2.790037in}{3.421942in}}%
\pgfpathcurveto{\pgfqpoint{2.790037in}{3.432992in}}{\pgfqpoint{2.785647in}{3.443591in}}{\pgfqpoint{2.777833in}{3.451404in}}%
\pgfpathcurveto{\pgfqpoint{2.770019in}{3.459218in}}{\pgfqpoint{2.759420in}{3.463608in}}{\pgfqpoint{2.748370in}{3.463608in}}%
\pgfpathcurveto{\pgfqpoint{2.737320in}{3.463608in}}{\pgfqpoint{2.726721in}{3.459218in}}{\pgfqpoint{2.718907in}{3.451404in}}%
\pgfpathcurveto{\pgfqpoint{2.711094in}{3.443591in}}{\pgfqpoint{2.706704in}{3.432992in}}{\pgfqpoint{2.706704in}{3.421942in}}%
\pgfpathcurveto{\pgfqpoint{2.706704in}{3.410891in}}{\pgfqpoint{2.711094in}{3.400292in}}{\pgfqpoint{2.718907in}{3.392479in}}%
\pgfpathcurveto{\pgfqpoint{2.726721in}{3.384665in}}{\pgfqpoint{2.737320in}{3.380275in}}{\pgfqpoint{2.748370in}{3.380275in}}%
\pgfpathclose%
\pgfusepath{stroke,fill}%
\end{pgfscope}%
\begin{pgfscope}%
\pgfpathrectangle{\pgfqpoint{0.600000in}{0.600000in}}{\pgfqpoint{3.900000in}{3.900000in}}%
\pgfusepath{clip}%
\pgfsetbuttcap%
\pgfsetroundjoin%
\definecolor{currentfill}{rgb}{0.121569,0.466667,0.705882}%
\pgfsetfillcolor{currentfill}%
\pgfsetlinewidth{1.003750pt}%
\definecolor{currentstroke}{rgb}{0.121569,0.466667,0.705882}%
\pgfsetstrokecolor{currentstroke}%
\pgfsetdash{}{0pt}%
\pgfpathmoveto{\pgfqpoint{2.508462in}{1.625243in}}%
\pgfpathcurveto{\pgfqpoint{2.519513in}{1.625243in}}{\pgfqpoint{2.530112in}{1.629633in}}{\pgfqpoint{2.537925in}{1.637447in}}%
\pgfpathcurveto{\pgfqpoint{2.545739in}{1.645260in}}{\pgfqpoint{2.550129in}{1.655859in}}{\pgfqpoint{2.550129in}{1.666910in}}%
\pgfpathcurveto{\pgfqpoint{2.550129in}{1.677960in}}{\pgfqpoint{2.545739in}{1.688559in}}{\pgfqpoint{2.537925in}{1.696372in}}%
\pgfpathcurveto{\pgfqpoint{2.530112in}{1.704186in}}{\pgfqpoint{2.519513in}{1.708576in}}{\pgfqpoint{2.508462in}{1.708576in}}%
\pgfpathcurveto{\pgfqpoint{2.497412in}{1.708576in}}{\pgfqpoint{2.486813in}{1.704186in}}{\pgfqpoint{2.479000in}{1.696372in}}%
\pgfpathcurveto{\pgfqpoint{2.471186in}{1.688559in}}{\pgfqpoint{2.466796in}{1.677960in}}{\pgfqpoint{2.466796in}{1.666910in}}%
\pgfpathcurveto{\pgfqpoint{2.466796in}{1.655859in}}{\pgfqpoint{2.471186in}{1.645260in}}{\pgfqpoint{2.479000in}{1.637447in}}%
\pgfpathcurveto{\pgfqpoint{2.486813in}{1.629633in}}{\pgfqpoint{2.497412in}{1.625243in}}{\pgfqpoint{2.508462in}{1.625243in}}%
\pgfpathclose%
\pgfusepath{stroke,fill}%
\end{pgfscope}%
\begin{pgfscope}%
\pgfpathrectangle{\pgfqpoint{0.600000in}{0.600000in}}{\pgfqpoint{3.900000in}{3.900000in}}%
\pgfusepath{clip}%
\pgfsetbuttcap%
\pgfsetroundjoin%
\definecolor{currentfill}{rgb}{0.121569,0.466667,0.705882}%
\pgfsetfillcolor{currentfill}%
\pgfsetlinewidth{1.003750pt}%
\definecolor{currentstroke}{rgb}{0.121569,0.466667,0.705882}%
\pgfsetstrokecolor{currentstroke}%
\pgfsetdash{}{0pt}%
\pgfpathmoveto{\pgfqpoint{2.154594in}{0.533753in}}%
\pgfpathcurveto{\pgfqpoint{2.165644in}{0.533753in}}{\pgfqpoint{2.176243in}{0.538144in}}{\pgfqpoint{2.184057in}{0.545957in}}%
\pgfpathcurveto{\pgfqpoint{2.191871in}{0.553771in}}{\pgfqpoint{2.196261in}{0.564370in}}{\pgfqpoint{2.196261in}{0.575420in}}%
\pgfpathcurveto{\pgfqpoint{2.196261in}{0.586470in}}{\pgfqpoint{2.191871in}{0.597069in}}{\pgfqpoint{2.184057in}{0.604883in}}%
\pgfpathcurveto{\pgfqpoint{2.176243in}{0.612697in}}{\pgfqpoint{2.165644in}{0.617087in}}{\pgfqpoint{2.154594in}{0.617087in}}%
\pgfpathcurveto{\pgfqpoint{2.143544in}{0.617087in}}{\pgfqpoint{2.132945in}{0.612697in}}{\pgfqpoint{2.125131in}{0.604883in}}%
\pgfpathcurveto{\pgfqpoint{2.117318in}{0.597069in}}{\pgfqpoint{2.112927in}{0.586470in}}{\pgfqpoint{2.112927in}{0.575420in}}%
\pgfpathcurveto{\pgfqpoint{2.112927in}{0.564370in}}{\pgfqpoint{2.117318in}{0.553771in}}{\pgfqpoint{2.125131in}{0.545957in}}%
\pgfpathcurveto{\pgfqpoint{2.132945in}{0.538144in}}{\pgfqpoint{2.143544in}{0.533753in}}{\pgfqpoint{2.154594in}{0.533753in}}%
\pgfpathclose%
\pgfusepath{stroke,fill}%
\end{pgfscope}%
\begin{pgfscope}%
\pgfpathrectangle{\pgfqpoint{0.600000in}{0.600000in}}{\pgfqpoint{3.900000in}{3.900000in}}%
\pgfusepath{clip}%
\pgfsetbuttcap%
\pgfsetroundjoin%
\definecolor{currentfill}{rgb}{0.121569,0.466667,0.705882}%
\pgfsetfillcolor{currentfill}%
\pgfsetlinewidth{1.003750pt}%
\definecolor{currentstroke}{rgb}{0.121569,0.466667,0.705882}%
\pgfsetstrokecolor{currentstroke}%
\pgfsetdash{}{0pt}%
\pgfpathmoveto{\pgfqpoint{2.535084in}{3.121889in}}%
\pgfpathcurveto{\pgfqpoint{2.546134in}{3.121889in}}{\pgfqpoint{2.556733in}{3.126279in}}{\pgfqpoint{2.564546in}{3.134092in}}%
\pgfpathcurveto{\pgfqpoint{2.572360in}{3.141906in}}{\pgfqpoint{2.576750in}{3.152505in}}{\pgfqpoint{2.576750in}{3.163555in}}%
\pgfpathcurveto{\pgfqpoint{2.576750in}{3.174605in}}{\pgfqpoint{2.572360in}{3.185204in}}{\pgfqpoint{2.564546in}{3.193018in}}%
\pgfpathcurveto{\pgfqpoint{2.556733in}{3.200832in}}{\pgfqpoint{2.546134in}{3.205222in}}{\pgfqpoint{2.535084in}{3.205222in}}%
\pgfpathcurveto{\pgfqpoint{2.524034in}{3.205222in}}{\pgfqpoint{2.513434in}{3.200832in}}{\pgfqpoint{2.505621in}{3.193018in}}%
\pgfpathcurveto{\pgfqpoint{2.497807in}{3.185204in}}{\pgfqpoint{2.493417in}{3.174605in}}{\pgfqpoint{2.493417in}{3.163555in}}%
\pgfpathcurveto{\pgfqpoint{2.493417in}{3.152505in}}{\pgfqpoint{2.497807in}{3.141906in}}{\pgfqpoint{2.505621in}{3.134092in}}%
\pgfpathcurveto{\pgfqpoint{2.513434in}{3.126279in}}{\pgfqpoint{2.524034in}{3.121889in}}{\pgfqpoint{2.535084in}{3.121889in}}%
\pgfpathclose%
\pgfusepath{stroke,fill}%
\end{pgfscope}%
\begin{pgfscope}%
\pgfpathrectangle{\pgfqpoint{0.600000in}{0.600000in}}{\pgfqpoint{3.900000in}{3.900000in}}%
\pgfusepath{clip}%
\pgfsetbuttcap%
\pgfsetroundjoin%
\definecolor{currentfill}{rgb}{0.121569,0.466667,0.705882}%
\pgfsetfillcolor{currentfill}%
\pgfsetlinewidth{1.003750pt}%
\definecolor{currentstroke}{rgb}{0.121569,0.466667,0.705882}%
\pgfsetstrokecolor{currentstroke}%
\pgfsetdash{}{0pt}%
\pgfpathmoveto{\pgfqpoint{2.895226in}{2.394570in}}%
\pgfpathcurveto{\pgfqpoint{2.906276in}{2.394570in}}{\pgfqpoint{2.916875in}{2.398960in}}{\pgfqpoint{2.924689in}{2.406774in}}%
\pgfpathcurveto{\pgfqpoint{2.932503in}{2.414587in}}{\pgfqpoint{2.936893in}{2.425186in}}{\pgfqpoint{2.936893in}{2.436236in}}%
\pgfpathcurveto{\pgfqpoint{2.936893in}{2.447286in}}{\pgfqpoint{2.932503in}{2.457886in}}{\pgfqpoint{2.924689in}{2.465699in}}%
\pgfpathcurveto{\pgfqpoint{2.916875in}{2.473513in}}{\pgfqpoint{2.906276in}{2.477903in}}{\pgfqpoint{2.895226in}{2.477903in}}%
\pgfpathcurveto{\pgfqpoint{2.884176in}{2.477903in}}{\pgfqpoint{2.873577in}{2.473513in}}{\pgfqpoint{2.865764in}{2.465699in}}%
\pgfpathcurveto{\pgfqpoint{2.857950in}{2.457886in}}{\pgfqpoint{2.853560in}{2.447286in}}{\pgfqpoint{2.853560in}{2.436236in}}%
\pgfpathcurveto{\pgfqpoint{2.853560in}{2.425186in}}{\pgfqpoint{2.857950in}{2.414587in}}{\pgfqpoint{2.865764in}{2.406774in}}%
\pgfpathcurveto{\pgfqpoint{2.873577in}{2.398960in}}{\pgfqpoint{2.884176in}{2.394570in}}{\pgfqpoint{2.895226in}{2.394570in}}%
\pgfpathclose%
\pgfusepath{stroke,fill}%
\end{pgfscope}%
\begin{pgfscope}%
\pgfpathrectangle{\pgfqpoint{0.600000in}{0.600000in}}{\pgfqpoint{3.900000in}{3.900000in}}%
\pgfusepath{clip}%
\pgfsetbuttcap%
\pgfsetroundjoin%
\definecolor{currentfill}{rgb}{0.121569,0.466667,0.705882}%
\pgfsetfillcolor{currentfill}%
\pgfsetlinewidth{1.003750pt}%
\definecolor{currentstroke}{rgb}{0.121569,0.466667,0.705882}%
\pgfsetstrokecolor{currentstroke}%
\pgfsetdash{}{0pt}%
\pgfpathmoveto{\pgfqpoint{3.050534in}{3.650957in}}%
\pgfpathcurveto{\pgfqpoint{3.061584in}{3.650957in}}{\pgfqpoint{3.072183in}{3.655347in}}{\pgfqpoint{3.079996in}{3.663161in}}%
\pgfpathcurveto{\pgfqpoint{3.087810in}{3.670974in}}{\pgfqpoint{3.092200in}{3.681573in}}{\pgfqpoint{3.092200in}{3.692623in}}%
\pgfpathcurveto{\pgfqpoint{3.092200in}{3.703673in}}{\pgfqpoint{3.087810in}{3.714272in}}{\pgfqpoint{3.079996in}{3.722086in}}%
\pgfpathcurveto{\pgfqpoint{3.072183in}{3.729900in}}{\pgfqpoint{3.061584in}{3.734290in}}{\pgfqpoint{3.050534in}{3.734290in}}%
\pgfpathcurveto{\pgfqpoint{3.039483in}{3.734290in}}{\pgfqpoint{3.028884in}{3.729900in}}{\pgfqpoint{3.021071in}{3.722086in}}%
\pgfpathcurveto{\pgfqpoint{3.013257in}{3.714272in}}{\pgfqpoint{3.008867in}{3.703673in}}{\pgfqpoint{3.008867in}{3.692623in}}%
\pgfpathcurveto{\pgfqpoint{3.008867in}{3.681573in}}{\pgfqpoint{3.013257in}{3.670974in}}{\pgfqpoint{3.021071in}{3.663161in}}%
\pgfpathcurveto{\pgfqpoint{3.028884in}{3.655347in}}{\pgfqpoint{3.039483in}{3.650957in}}{\pgfqpoint{3.050534in}{3.650957in}}%
\pgfpathclose%
\pgfusepath{stroke,fill}%
\end{pgfscope}%
\begin{pgfscope}%
\pgfpathrectangle{\pgfqpoint{0.600000in}{0.600000in}}{\pgfqpoint{3.900000in}{3.900000in}}%
\pgfusepath{clip}%
\pgfsetbuttcap%
\pgfsetroundjoin%
\definecolor{currentfill}{rgb}{0.121569,0.466667,0.705882}%
\pgfsetfillcolor{currentfill}%
\pgfsetlinewidth{1.003750pt}%
\definecolor{currentstroke}{rgb}{0.121569,0.466667,0.705882}%
\pgfsetstrokecolor{currentstroke}%
\pgfsetdash{}{0pt}%
\pgfpathmoveto{\pgfqpoint{3.614621in}{1.933785in}}%
\pgfpathcurveto{\pgfqpoint{3.625672in}{1.933785in}}{\pgfqpoint{3.636271in}{1.938176in}}{\pgfqpoint{3.644084in}{1.945989in}}%
\pgfpathcurveto{\pgfqpoint{3.651898in}{1.953803in}}{\pgfqpoint{3.656288in}{1.964402in}}{\pgfqpoint{3.656288in}{1.975452in}}%
\pgfpathcurveto{\pgfqpoint{3.656288in}{1.986502in}}{\pgfqpoint{3.651898in}{1.997101in}}{\pgfqpoint{3.644084in}{2.004915in}}%
\pgfpathcurveto{\pgfqpoint{3.636271in}{2.012728in}}{\pgfqpoint{3.625672in}{2.017119in}}{\pgfqpoint{3.614621in}{2.017119in}}%
\pgfpathcurveto{\pgfqpoint{3.603571in}{2.017119in}}{\pgfqpoint{3.592972in}{2.012728in}}{\pgfqpoint{3.585159in}{2.004915in}}%
\pgfpathcurveto{\pgfqpoint{3.577345in}{1.997101in}}{\pgfqpoint{3.572955in}{1.986502in}}{\pgfqpoint{3.572955in}{1.975452in}}%
\pgfpathcurveto{\pgfqpoint{3.572955in}{1.964402in}}{\pgfqpoint{3.577345in}{1.953803in}}{\pgfqpoint{3.585159in}{1.945989in}}%
\pgfpathcurveto{\pgfqpoint{3.592972in}{1.938176in}}{\pgfqpoint{3.603571in}{1.933785in}}{\pgfqpoint{3.614621in}{1.933785in}}%
\pgfpathclose%
\pgfusepath{stroke,fill}%
\end{pgfscope}%
\begin{pgfscope}%
\pgfpathrectangle{\pgfqpoint{0.600000in}{0.600000in}}{\pgfqpoint{3.900000in}{3.900000in}}%
\pgfusepath{clip}%
\pgfsetbuttcap%
\pgfsetroundjoin%
\definecolor{currentfill}{rgb}{0.121569,0.466667,0.705882}%
\pgfsetfillcolor{currentfill}%
\pgfsetlinewidth{1.003750pt}%
\definecolor{currentstroke}{rgb}{0.121569,0.466667,0.705882}%
\pgfsetstrokecolor{currentstroke}%
\pgfsetdash{}{0pt}%
\pgfpathmoveto{\pgfqpoint{2.909182in}{2.015141in}}%
\pgfpathcurveto{\pgfqpoint{2.920232in}{2.015141in}}{\pgfqpoint{2.930831in}{2.019531in}}{\pgfqpoint{2.938644in}{2.027345in}}%
\pgfpathcurveto{\pgfqpoint{2.946458in}{2.035158in}}{\pgfqpoint{2.950848in}{2.045757in}}{\pgfqpoint{2.950848in}{2.056807in}}%
\pgfpathcurveto{\pgfqpoint{2.950848in}{2.067858in}}{\pgfqpoint{2.946458in}{2.078457in}}{\pgfqpoint{2.938644in}{2.086270in}}%
\pgfpathcurveto{\pgfqpoint{2.930831in}{2.094084in}}{\pgfqpoint{2.920232in}{2.098474in}}{\pgfqpoint{2.909182in}{2.098474in}}%
\pgfpathcurveto{\pgfqpoint{2.898131in}{2.098474in}}{\pgfqpoint{2.887532in}{2.094084in}}{\pgfqpoint{2.879719in}{2.086270in}}%
\pgfpathcurveto{\pgfqpoint{2.871905in}{2.078457in}}{\pgfqpoint{2.867515in}{2.067858in}}{\pgfqpoint{2.867515in}{2.056807in}}%
\pgfpathcurveto{\pgfqpoint{2.867515in}{2.045757in}}{\pgfqpoint{2.871905in}{2.035158in}}{\pgfqpoint{2.879719in}{2.027345in}}%
\pgfpathcurveto{\pgfqpoint{2.887532in}{2.019531in}}{\pgfqpoint{2.898131in}{2.015141in}}{\pgfqpoint{2.909182in}{2.015141in}}%
\pgfpathclose%
\pgfusepath{stroke,fill}%
\end{pgfscope}%
\begin{pgfscope}%
\pgfpathrectangle{\pgfqpoint{0.600000in}{0.600000in}}{\pgfqpoint{3.900000in}{3.900000in}}%
\pgfusepath{clip}%
\pgfsetbuttcap%
\pgfsetroundjoin%
\definecolor{currentfill}{rgb}{0.121569,0.466667,0.705882}%
\pgfsetfillcolor{currentfill}%
\pgfsetlinewidth{1.003750pt}%
\definecolor{currentstroke}{rgb}{0.121569,0.466667,0.705882}%
\pgfsetstrokecolor{currentstroke}%
\pgfsetdash{}{0pt}%
\pgfpathmoveto{\pgfqpoint{3.087110in}{2.116916in}}%
\pgfpathcurveto{\pgfqpoint{3.098160in}{2.116916in}}{\pgfqpoint{3.108759in}{2.121306in}}{\pgfqpoint{3.116573in}{2.129120in}}%
\pgfpathcurveto{\pgfqpoint{3.124387in}{2.136933in}}{\pgfqpoint{3.128777in}{2.147532in}}{\pgfqpoint{3.128777in}{2.158582in}}%
\pgfpathcurveto{\pgfqpoint{3.128777in}{2.169633in}}{\pgfqpoint{3.124387in}{2.180232in}}{\pgfqpoint{3.116573in}{2.188045in}}%
\pgfpathcurveto{\pgfqpoint{3.108759in}{2.195859in}}{\pgfqpoint{3.098160in}{2.200249in}}{\pgfqpoint{3.087110in}{2.200249in}}%
\pgfpathcurveto{\pgfqpoint{3.076060in}{2.200249in}}{\pgfqpoint{3.065461in}{2.195859in}}{\pgfqpoint{3.057648in}{2.188045in}}%
\pgfpathcurveto{\pgfqpoint{3.049834in}{2.180232in}}{\pgfqpoint{3.045444in}{2.169633in}}{\pgfqpoint{3.045444in}{2.158582in}}%
\pgfpathcurveto{\pgfqpoint{3.045444in}{2.147532in}}{\pgfqpoint{3.049834in}{2.136933in}}{\pgfqpoint{3.057648in}{2.129120in}}%
\pgfpathcurveto{\pgfqpoint{3.065461in}{2.121306in}}{\pgfqpoint{3.076060in}{2.116916in}}{\pgfqpoint{3.087110in}{2.116916in}}%
\pgfpathclose%
\pgfusepath{stroke,fill}%
\end{pgfscope}%
\begin{pgfscope}%
\pgfpathrectangle{\pgfqpoint{0.600000in}{0.600000in}}{\pgfqpoint{3.900000in}{3.900000in}}%
\pgfusepath{clip}%
\pgfsetbuttcap%
\pgfsetroundjoin%
\definecolor{currentfill}{rgb}{0.121569,0.466667,0.705882}%
\pgfsetfillcolor{currentfill}%
\pgfsetlinewidth{1.003750pt}%
\definecolor{currentstroke}{rgb}{0.121569,0.466667,0.705882}%
\pgfsetstrokecolor{currentstroke}%
\pgfsetdash{}{0pt}%
\pgfpathmoveto{\pgfqpoint{2.739504in}{2.556399in}}%
\pgfpathcurveto{\pgfqpoint{2.750554in}{2.556399in}}{\pgfqpoint{2.761153in}{2.560789in}}{\pgfqpoint{2.768966in}{2.568603in}}%
\pgfpathcurveto{\pgfqpoint{2.776780in}{2.576416in}}{\pgfqpoint{2.781170in}{2.587015in}}{\pgfqpoint{2.781170in}{2.598065in}}%
\pgfpathcurveto{\pgfqpoint{2.781170in}{2.609115in}}{\pgfqpoint{2.776780in}{2.619715in}}{\pgfqpoint{2.768966in}{2.627528in}}%
\pgfpathcurveto{\pgfqpoint{2.761153in}{2.635342in}}{\pgfqpoint{2.750554in}{2.639732in}}{\pgfqpoint{2.739504in}{2.639732in}}%
\pgfpathcurveto{\pgfqpoint{2.728453in}{2.639732in}}{\pgfqpoint{2.717854in}{2.635342in}}{\pgfqpoint{2.710041in}{2.627528in}}%
\pgfpathcurveto{\pgfqpoint{2.702227in}{2.619715in}}{\pgfqpoint{2.697837in}{2.609115in}}{\pgfqpoint{2.697837in}{2.598065in}}%
\pgfpathcurveto{\pgfqpoint{2.697837in}{2.587015in}}{\pgfqpoint{2.702227in}{2.576416in}}{\pgfqpoint{2.710041in}{2.568603in}}%
\pgfpathcurveto{\pgfqpoint{2.717854in}{2.560789in}}{\pgfqpoint{2.728453in}{2.556399in}}{\pgfqpoint{2.739504in}{2.556399in}}%
\pgfpathclose%
\pgfusepath{stroke,fill}%
\end{pgfscope}%
\begin{pgfscope}%
\pgfpathrectangle{\pgfqpoint{0.600000in}{0.600000in}}{\pgfqpoint{3.900000in}{3.900000in}}%
\pgfusepath{clip}%
\pgfsetbuttcap%
\pgfsetroundjoin%
\definecolor{currentfill}{rgb}{0.121569,0.466667,0.705882}%
\pgfsetfillcolor{currentfill}%
\pgfsetlinewidth{1.003750pt}%
\definecolor{currentstroke}{rgb}{0.121569,0.466667,0.705882}%
\pgfsetstrokecolor{currentstroke}%
\pgfsetdash{}{0pt}%
\pgfpathmoveto{\pgfqpoint{2.652401in}{2.174623in}}%
\pgfpathcurveto{\pgfqpoint{2.663452in}{2.174623in}}{\pgfqpoint{2.674051in}{2.179014in}}{\pgfqpoint{2.681864in}{2.186827in}}%
\pgfpathcurveto{\pgfqpoint{2.689678in}{2.194641in}}{\pgfqpoint{2.694068in}{2.205240in}}{\pgfqpoint{2.694068in}{2.216290in}}%
\pgfpathcurveto{\pgfqpoint{2.694068in}{2.227340in}}{\pgfqpoint{2.689678in}{2.237939in}}{\pgfqpoint{2.681864in}{2.245753in}}%
\pgfpathcurveto{\pgfqpoint{2.674051in}{2.253566in}}{\pgfqpoint{2.663452in}{2.257957in}}{\pgfqpoint{2.652401in}{2.257957in}}%
\pgfpathcurveto{\pgfqpoint{2.641351in}{2.257957in}}{\pgfqpoint{2.630752in}{2.253566in}}{\pgfqpoint{2.622939in}{2.245753in}}%
\pgfpathcurveto{\pgfqpoint{2.615125in}{2.237939in}}{\pgfqpoint{2.610735in}{2.227340in}}{\pgfqpoint{2.610735in}{2.216290in}}%
\pgfpathcurveto{\pgfqpoint{2.610735in}{2.205240in}}{\pgfqpoint{2.615125in}{2.194641in}}{\pgfqpoint{2.622939in}{2.186827in}}%
\pgfpathcurveto{\pgfqpoint{2.630752in}{2.179014in}}{\pgfqpoint{2.641351in}{2.174623in}}{\pgfqpoint{2.652401in}{2.174623in}}%
\pgfpathclose%
\pgfusepath{stroke,fill}%
\end{pgfscope}%
\begin{pgfscope}%
\pgfpathrectangle{\pgfqpoint{0.600000in}{0.600000in}}{\pgfqpoint{3.900000in}{3.900000in}}%
\pgfusepath{clip}%
\pgfsetbuttcap%
\pgfsetroundjoin%
\definecolor{currentfill}{rgb}{0.121569,0.466667,0.705882}%
\pgfsetfillcolor{currentfill}%
\pgfsetlinewidth{1.003750pt}%
\definecolor{currentstroke}{rgb}{0.121569,0.466667,0.705882}%
\pgfsetstrokecolor{currentstroke}%
\pgfsetdash{}{0pt}%
\pgfpathmoveto{\pgfqpoint{2.719097in}{1.725411in}}%
\pgfpathcurveto{\pgfqpoint{2.730147in}{1.725411in}}{\pgfqpoint{2.740746in}{1.729801in}}{\pgfqpoint{2.748560in}{1.737615in}}%
\pgfpathcurveto{\pgfqpoint{2.756374in}{1.745429in}}{\pgfqpoint{2.760764in}{1.756028in}}{\pgfqpoint{2.760764in}{1.767078in}}%
\pgfpathcurveto{\pgfqpoint{2.760764in}{1.778128in}}{\pgfqpoint{2.756374in}{1.788727in}}{\pgfqpoint{2.748560in}{1.796541in}}%
\pgfpathcurveto{\pgfqpoint{2.740746in}{1.804354in}}{\pgfqpoint{2.730147in}{1.808745in}}{\pgfqpoint{2.719097in}{1.808745in}}%
\pgfpathcurveto{\pgfqpoint{2.708047in}{1.808745in}}{\pgfqpoint{2.697448in}{1.804354in}}{\pgfqpoint{2.689634in}{1.796541in}}%
\pgfpathcurveto{\pgfqpoint{2.681821in}{1.788727in}}{\pgfqpoint{2.677431in}{1.778128in}}{\pgfqpoint{2.677431in}{1.767078in}}%
\pgfpathcurveto{\pgfqpoint{2.677431in}{1.756028in}}{\pgfqpoint{2.681821in}{1.745429in}}{\pgfqpoint{2.689634in}{1.737615in}}%
\pgfpathcurveto{\pgfqpoint{2.697448in}{1.729801in}}{\pgfqpoint{2.708047in}{1.725411in}}{\pgfqpoint{2.719097in}{1.725411in}}%
\pgfpathclose%
\pgfusepath{stroke,fill}%
\end{pgfscope}%
\begin{pgfscope}%
\pgfpathrectangle{\pgfqpoint{0.600000in}{0.600000in}}{\pgfqpoint{3.900000in}{3.900000in}}%
\pgfusepath{clip}%
\pgfsetbuttcap%
\pgfsetroundjoin%
\definecolor{currentfill}{rgb}{0.121569,0.466667,0.705882}%
\pgfsetfillcolor{currentfill}%
\pgfsetlinewidth{1.003750pt}%
\definecolor{currentstroke}{rgb}{0.121569,0.466667,0.705882}%
\pgfsetstrokecolor{currentstroke}%
\pgfsetdash{}{0pt}%
\pgfpathmoveto{\pgfqpoint{2.634213in}{2.835183in}}%
\pgfpathcurveto{\pgfqpoint{2.645263in}{2.835183in}}{\pgfqpoint{2.655862in}{2.839573in}}{\pgfqpoint{2.663676in}{2.847387in}}%
\pgfpathcurveto{\pgfqpoint{2.671489in}{2.855201in}}{\pgfqpoint{2.675879in}{2.865800in}}{\pgfqpoint{2.675879in}{2.876850in}}%
\pgfpathcurveto{\pgfqpoint{2.675879in}{2.887900in}}{\pgfqpoint{2.671489in}{2.898499in}}{\pgfqpoint{2.663676in}{2.906313in}}%
\pgfpathcurveto{\pgfqpoint{2.655862in}{2.914126in}}{\pgfqpoint{2.645263in}{2.918517in}}{\pgfqpoint{2.634213in}{2.918517in}}%
\pgfpathcurveto{\pgfqpoint{2.623163in}{2.918517in}}{\pgfqpoint{2.612564in}{2.914126in}}{\pgfqpoint{2.604750in}{2.906313in}}%
\pgfpathcurveto{\pgfqpoint{2.596936in}{2.898499in}}{\pgfqpoint{2.592546in}{2.887900in}}{\pgfqpoint{2.592546in}{2.876850in}}%
\pgfpathcurveto{\pgfqpoint{2.592546in}{2.865800in}}{\pgfqpoint{2.596936in}{2.855201in}}{\pgfqpoint{2.604750in}{2.847387in}}%
\pgfpathcurveto{\pgfqpoint{2.612564in}{2.839573in}}{\pgfqpoint{2.623163in}{2.835183in}}{\pgfqpoint{2.634213in}{2.835183in}}%
\pgfpathclose%
\pgfusepath{stroke,fill}%
\end{pgfscope}%
\begin{pgfscope}%
\pgfpathrectangle{\pgfqpoint{0.600000in}{0.600000in}}{\pgfqpoint{3.900000in}{3.900000in}}%
\pgfusepath{clip}%
\pgfsetbuttcap%
\pgfsetroundjoin%
\definecolor{currentfill}{rgb}{0.121569,0.466667,0.705882}%
\pgfsetfillcolor{currentfill}%
\pgfsetlinewidth{1.003750pt}%
\definecolor{currentstroke}{rgb}{0.121569,0.466667,0.705882}%
\pgfsetstrokecolor{currentstroke}%
\pgfsetdash{}{0pt}%
\pgfpathmoveto{\pgfqpoint{3.445439in}{3.435699in}}%
\pgfpathcurveto{\pgfqpoint{3.456489in}{3.435699in}}{\pgfqpoint{3.467088in}{3.440089in}}{\pgfqpoint{3.474902in}{3.447902in}}%
\pgfpathcurveto{\pgfqpoint{3.482716in}{3.455716in}}{\pgfqpoint{3.487106in}{3.466315in}}{\pgfqpoint{3.487106in}{3.477365in}}%
\pgfpathcurveto{\pgfqpoint{3.487106in}{3.488415in}}{\pgfqpoint{3.482716in}{3.499014in}}{\pgfqpoint{3.474902in}{3.506828in}}%
\pgfpathcurveto{\pgfqpoint{3.467088in}{3.514642in}}{\pgfqpoint{3.456489in}{3.519032in}}{\pgfqpoint{3.445439in}{3.519032in}}%
\pgfpathcurveto{\pgfqpoint{3.434389in}{3.519032in}}{\pgfqpoint{3.423790in}{3.514642in}}{\pgfqpoint{3.415976in}{3.506828in}}%
\pgfpathcurveto{\pgfqpoint{3.408163in}{3.499014in}}{\pgfqpoint{3.403772in}{3.488415in}}{\pgfqpoint{3.403772in}{3.477365in}}%
\pgfpathcurveto{\pgfqpoint{3.403772in}{3.466315in}}{\pgfqpoint{3.408163in}{3.455716in}}{\pgfqpoint{3.415976in}{3.447902in}}%
\pgfpathcurveto{\pgfqpoint{3.423790in}{3.440089in}}{\pgfqpoint{3.434389in}{3.435699in}}{\pgfqpoint{3.445439in}{3.435699in}}%
\pgfpathclose%
\pgfusepath{stroke,fill}%
\end{pgfscope}%
\begin{pgfscope}%
\pgfpathrectangle{\pgfqpoint{0.600000in}{0.600000in}}{\pgfqpoint{3.900000in}{3.900000in}}%
\pgfusepath{clip}%
\pgfsetbuttcap%
\pgfsetroundjoin%
\definecolor{currentfill}{rgb}{0.121569,0.466667,0.705882}%
\pgfsetfillcolor{currentfill}%
\pgfsetlinewidth{1.003750pt}%
\definecolor{currentstroke}{rgb}{0.121569,0.466667,0.705882}%
\pgfsetstrokecolor{currentstroke}%
\pgfsetdash{}{0pt}%
\pgfpathmoveto{\pgfqpoint{2.499978in}{1.975112in}}%
\pgfpathcurveto{\pgfqpoint{2.511028in}{1.975112in}}{\pgfqpoint{2.521627in}{1.979503in}}{\pgfqpoint{2.529441in}{1.987316in}}%
\pgfpathcurveto{\pgfqpoint{2.537254in}{1.995130in}}{\pgfqpoint{2.541644in}{2.005729in}}{\pgfqpoint{2.541644in}{2.016779in}}%
\pgfpathcurveto{\pgfqpoint{2.541644in}{2.027829in}}{\pgfqpoint{2.537254in}{2.038428in}}{\pgfqpoint{2.529441in}{2.046242in}}%
\pgfpathcurveto{\pgfqpoint{2.521627in}{2.054056in}}{\pgfqpoint{2.511028in}{2.058446in}}{\pgfqpoint{2.499978in}{2.058446in}}%
\pgfpathcurveto{\pgfqpoint{2.488928in}{2.058446in}}{\pgfqpoint{2.478329in}{2.054056in}}{\pgfqpoint{2.470515in}{2.046242in}}%
\pgfpathcurveto{\pgfqpoint{2.462701in}{2.038428in}}{\pgfqpoint{2.458311in}{2.027829in}}{\pgfqpoint{2.458311in}{2.016779in}}%
\pgfpathcurveto{\pgfqpoint{2.458311in}{2.005729in}}{\pgfqpoint{2.462701in}{1.995130in}}{\pgfqpoint{2.470515in}{1.987316in}}%
\pgfpathcurveto{\pgfqpoint{2.478329in}{1.979503in}}{\pgfqpoint{2.488928in}{1.975112in}}{\pgfqpoint{2.499978in}{1.975112in}}%
\pgfpathclose%
\pgfusepath{stroke,fill}%
\end{pgfscope}%
\begin{pgfscope}%
\pgfpathrectangle{\pgfqpoint{0.600000in}{0.600000in}}{\pgfqpoint{3.900000in}{3.900000in}}%
\pgfusepath{clip}%
\pgfsetbuttcap%
\pgfsetroundjoin%
\definecolor{currentfill}{rgb}{0.121569,0.466667,0.705882}%
\pgfsetfillcolor{currentfill}%
\pgfsetlinewidth{1.003750pt}%
\definecolor{currentstroke}{rgb}{0.121569,0.466667,0.705882}%
\pgfsetstrokecolor{currentstroke}%
\pgfsetdash{}{0pt}%
\pgfpathmoveto{\pgfqpoint{2.638193in}{3.069022in}}%
\pgfpathcurveto{\pgfqpoint{2.649243in}{3.069022in}}{\pgfqpoint{2.659842in}{3.073412in}}{\pgfqpoint{2.667655in}{3.081226in}}%
\pgfpathcurveto{\pgfqpoint{2.675469in}{3.089040in}}{\pgfqpoint{2.679859in}{3.099639in}}{\pgfqpoint{2.679859in}{3.110689in}}%
\pgfpathcurveto{\pgfqpoint{2.679859in}{3.121739in}}{\pgfqpoint{2.675469in}{3.132338in}}{\pgfqpoint{2.667655in}{3.140151in}}%
\pgfpathcurveto{\pgfqpoint{2.659842in}{3.147965in}}{\pgfqpoint{2.649243in}{3.152355in}}{\pgfqpoint{2.638193in}{3.152355in}}%
\pgfpathcurveto{\pgfqpoint{2.627142in}{3.152355in}}{\pgfqpoint{2.616543in}{3.147965in}}{\pgfqpoint{2.608730in}{3.140151in}}%
\pgfpathcurveto{\pgfqpoint{2.600916in}{3.132338in}}{\pgfqpoint{2.596526in}{3.121739in}}{\pgfqpoint{2.596526in}{3.110689in}}%
\pgfpathcurveto{\pgfqpoint{2.596526in}{3.099639in}}{\pgfqpoint{2.600916in}{3.089040in}}{\pgfqpoint{2.608730in}{3.081226in}}%
\pgfpathcurveto{\pgfqpoint{2.616543in}{3.073412in}}{\pgfqpoint{2.627142in}{3.069022in}}{\pgfqpoint{2.638193in}{3.069022in}}%
\pgfpathclose%
\pgfusepath{stroke,fill}%
\end{pgfscope}%
\begin{pgfscope}%
\pgfpathrectangle{\pgfqpoint{0.600000in}{0.600000in}}{\pgfqpoint{3.900000in}{3.900000in}}%
\pgfusepath{clip}%
\pgfsetbuttcap%
\pgfsetroundjoin%
\definecolor{currentfill}{rgb}{0.121569,0.466667,0.705882}%
\pgfsetfillcolor{currentfill}%
\pgfsetlinewidth{1.003750pt}%
\definecolor{currentstroke}{rgb}{0.121569,0.466667,0.705882}%
\pgfsetstrokecolor{currentstroke}%
\pgfsetdash{}{0pt}%
\pgfpathmoveto{\pgfqpoint{2.073393in}{3.023143in}}%
\pgfpathcurveto{\pgfqpoint{2.084443in}{3.023143in}}{\pgfqpoint{2.095042in}{3.027534in}}{\pgfqpoint{2.102856in}{3.035347in}}%
\pgfpathcurveto{\pgfqpoint{2.110670in}{3.043161in}}{\pgfqpoint{2.115060in}{3.053760in}}{\pgfqpoint{2.115060in}{3.064810in}}%
\pgfpathcurveto{\pgfqpoint{2.115060in}{3.075860in}}{\pgfqpoint{2.110670in}{3.086459in}}{\pgfqpoint{2.102856in}{3.094273in}}%
\pgfpathcurveto{\pgfqpoint{2.095042in}{3.102087in}}{\pgfqpoint{2.084443in}{3.106477in}}{\pgfqpoint{2.073393in}{3.106477in}}%
\pgfpathcurveto{\pgfqpoint{2.062343in}{3.106477in}}{\pgfqpoint{2.051744in}{3.102087in}}{\pgfqpoint{2.043930in}{3.094273in}}%
\pgfpathcurveto{\pgfqpoint{2.036117in}{3.086459in}}{\pgfqpoint{2.031727in}{3.075860in}}{\pgfqpoint{2.031727in}{3.064810in}}%
\pgfpathcurveto{\pgfqpoint{2.031727in}{3.053760in}}{\pgfqpoint{2.036117in}{3.043161in}}{\pgfqpoint{2.043930in}{3.035347in}}%
\pgfpathcurveto{\pgfqpoint{2.051744in}{3.027534in}}{\pgfqpoint{2.062343in}{3.023143in}}{\pgfqpoint{2.073393in}{3.023143in}}%
\pgfpathclose%
\pgfusepath{stroke,fill}%
\end{pgfscope}%
\begin{pgfscope}%
\pgfpathrectangle{\pgfqpoint{0.600000in}{0.600000in}}{\pgfqpoint{3.900000in}{3.900000in}}%
\pgfusepath{clip}%
\pgfsetbuttcap%
\pgfsetroundjoin%
\definecolor{currentfill}{rgb}{0.121569,0.466667,0.705882}%
\pgfsetfillcolor{currentfill}%
\pgfsetlinewidth{1.003750pt}%
\definecolor{currentstroke}{rgb}{0.121569,0.466667,0.705882}%
\pgfsetstrokecolor{currentstroke}%
\pgfsetdash{}{0pt}%
\pgfpathmoveto{\pgfqpoint{3.231714in}{2.484425in}}%
\pgfpathcurveto{\pgfqpoint{3.242764in}{2.484425in}}{\pgfqpoint{3.253363in}{2.488815in}}{\pgfqpoint{3.261177in}{2.496629in}}%
\pgfpathcurveto{\pgfqpoint{3.268990in}{2.504443in}}{\pgfqpoint{3.273380in}{2.515042in}}{\pgfqpoint{3.273380in}{2.526092in}}%
\pgfpathcurveto{\pgfqpoint{3.273380in}{2.537142in}}{\pgfqpoint{3.268990in}{2.547741in}}{\pgfqpoint{3.261177in}{2.555554in}}%
\pgfpathcurveto{\pgfqpoint{3.253363in}{2.563368in}}{\pgfqpoint{3.242764in}{2.567758in}}{\pgfqpoint{3.231714in}{2.567758in}}%
\pgfpathcurveto{\pgfqpoint{3.220664in}{2.567758in}}{\pgfqpoint{3.210065in}{2.563368in}}{\pgfqpoint{3.202251in}{2.555554in}}%
\pgfpathcurveto{\pgfqpoint{3.194437in}{2.547741in}}{\pgfqpoint{3.190047in}{2.537142in}}{\pgfqpoint{3.190047in}{2.526092in}}%
\pgfpathcurveto{\pgfqpoint{3.190047in}{2.515042in}}{\pgfqpoint{3.194437in}{2.504443in}}{\pgfqpoint{3.202251in}{2.496629in}}%
\pgfpathcurveto{\pgfqpoint{3.210065in}{2.488815in}}{\pgfqpoint{3.220664in}{2.484425in}}{\pgfqpoint{3.231714in}{2.484425in}}%
\pgfpathclose%
\pgfusepath{stroke,fill}%
\end{pgfscope}%
\begin{pgfscope}%
\pgfpathrectangle{\pgfqpoint{0.600000in}{0.600000in}}{\pgfqpoint{3.900000in}{3.900000in}}%
\pgfusepath{clip}%
\pgfsetbuttcap%
\pgfsetroundjoin%
\definecolor{currentfill}{rgb}{0.121569,0.466667,0.705882}%
\pgfsetfillcolor{currentfill}%
\pgfsetlinewidth{1.003750pt}%
\definecolor{currentstroke}{rgb}{0.121569,0.466667,0.705882}%
\pgfsetstrokecolor{currentstroke}%
\pgfsetdash{}{0pt}%
\pgfpathmoveto{\pgfqpoint{2.663834in}{2.033854in}}%
\pgfpathcurveto{\pgfqpoint{2.674884in}{2.033854in}}{\pgfqpoint{2.685483in}{2.038244in}}{\pgfqpoint{2.693296in}{2.046058in}}%
\pgfpathcurveto{\pgfqpoint{2.701110in}{2.053871in}}{\pgfqpoint{2.705500in}{2.064470in}}{\pgfqpoint{2.705500in}{2.075520in}}%
\pgfpathcurveto{\pgfqpoint{2.705500in}{2.086571in}}{\pgfqpoint{2.701110in}{2.097170in}}{\pgfqpoint{2.693296in}{2.104983in}}%
\pgfpathcurveto{\pgfqpoint{2.685483in}{2.112797in}}{\pgfqpoint{2.674884in}{2.117187in}}{\pgfqpoint{2.663834in}{2.117187in}}%
\pgfpathcurveto{\pgfqpoint{2.652783in}{2.117187in}}{\pgfqpoint{2.642184in}{2.112797in}}{\pgfqpoint{2.634371in}{2.104983in}}%
\pgfpathcurveto{\pgfqpoint{2.626557in}{2.097170in}}{\pgfqpoint{2.622167in}{2.086571in}}{\pgfqpoint{2.622167in}{2.075520in}}%
\pgfpathcurveto{\pgfqpoint{2.622167in}{2.064470in}}{\pgfqpoint{2.626557in}{2.053871in}}{\pgfqpoint{2.634371in}{2.046058in}}%
\pgfpathcurveto{\pgfqpoint{2.642184in}{2.038244in}}{\pgfqpoint{2.652783in}{2.033854in}}{\pgfqpoint{2.663834in}{2.033854in}}%
\pgfpathclose%
\pgfusepath{stroke,fill}%
\end{pgfscope}%
\begin{pgfscope}%
\pgfpathrectangle{\pgfqpoint{0.600000in}{0.600000in}}{\pgfqpoint{3.900000in}{3.900000in}}%
\pgfusepath{clip}%
\pgfsetbuttcap%
\pgfsetroundjoin%
\definecolor{currentfill}{rgb}{0.121569,0.466667,0.705882}%
\pgfsetfillcolor{currentfill}%
\pgfsetlinewidth{1.003750pt}%
\definecolor{currentstroke}{rgb}{0.121569,0.466667,0.705882}%
\pgfsetstrokecolor{currentstroke}%
\pgfsetdash{}{0pt}%
\pgfpathmoveto{\pgfqpoint{2.120506in}{2.758547in}}%
\pgfpathcurveto{\pgfqpoint{2.131556in}{2.758547in}}{\pgfqpoint{2.142155in}{2.762937in}}{\pgfqpoint{2.149969in}{2.770751in}}%
\pgfpathcurveto{\pgfqpoint{2.157782in}{2.778564in}}{\pgfqpoint{2.162173in}{2.789163in}}{\pgfqpoint{2.162173in}{2.800213in}}%
\pgfpathcurveto{\pgfqpoint{2.162173in}{2.811264in}}{\pgfqpoint{2.157782in}{2.821863in}}{\pgfqpoint{2.149969in}{2.829676in}}%
\pgfpathcurveto{\pgfqpoint{2.142155in}{2.837490in}}{\pgfqpoint{2.131556in}{2.841880in}}{\pgfqpoint{2.120506in}{2.841880in}}%
\pgfpathcurveto{\pgfqpoint{2.109456in}{2.841880in}}{\pgfqpoint{2.098857in}{2.837490in}}{\pgfqpoint{2.091043in}{2.829676in}}%
\pgfpathcurveto{\pgfqpoint{2.083229in}{2.821863in}}{\pgfqpoint{2.078839in}{2.811264in}}{\pgfqpoint{2.078839in}{2.800213in}}%
\pgfpathcurveto{\pgfqpoint{2.078839in}{2.789163in}}{\pgfqpoint{2.083229in}{2.778564in}}{\pgfqpoint{2.091043in}{2.770751in}}%
\pgfpathcurveto{\pgfqpoint{2.098857in}{2.762937in}}{\pgfqpoint{2.109456in}{2.758547in}}{\pgfqpoint{2.120506in}{2.758547in}}%
\pgfpathclose%
\pgfusepath{stroke,fill}%
\end{pgfscope}%
\begin{pgfscope}%
\pgfpathrectangle{\pgfqpoint{0.600000in}{0.600000in}}{\pgfqpoint{3.900000in}{3.900000in}}%
\pgfusepath{clip}%
\pgfsetbuttcap%
\pgfsetroundjoin%
\definecolor{currentfill}{rgb}{0.121569,0.466667,0.705882}%
\pgfsetfillcolor{currentfill}%
\pgfsetlinewidth{1.003750pt}%
\definecolor{currentstroke}{rgb}{0.121569,0.466667,0.705882}%
\pgfsetstrokecolor{currentstroke}%
\pgfsetdash{}{0pt}%
\pgfpathmoveto{\pgfqpoint{2.670035in}{2.993170in}}%
\pgfpathcurveto{\pgfqpoint{2.681086in}{2.993170in}}{\pgfqpoint{2.691685in}{2.997561in}}{\pgfqpoint{2.699498in}{3.005374in}}%
\pgfpathcurveto{\pgfqpoint{2.707312in}{3.013188in}}{\pgfqpoint{2.711702in}{3.023787in}}{\pgfqpoint{2.711702in}{3.034837in}}%
\pgfpathcurveto{\pgfqpoint{2.711702in}{3.045887in}}{\pgfqpoint{2.707312in}{3.056486in}}{\pgfqpoint{2.699498in}{3.064300in}}%
\pgfpathcurveto{\pgfqpoint{2.691685in}{3.072113in}}{\pgfqpoint{2.681086in}{3.076504in}}{\pgfqpoint{2.670035in}{3.076504in}}%
\pgfpathcurveto{\pgfqpoint{2.658985in}{3.076504in}}{\pgfqpoint{2.648386in}{3.072113in}}{\pgfqpoint{2.640573in}{3.064300in}}%
\pgfpathcurveto{\pgfqpoint{2.632759in}{3.056486in}}{\pgfqpoint{2.628369in}{3.045887in}}{\pgfqpoint{2.628369in}{3.034837in}}%
\pgfpathcurveto{\pgfqpoint{2.628369in}{3.023787in}}{\pgfqpoint{2.632759in}{3.013188in}}{\pgfqpoint{2.640573in}{3.005374in}}%
\pgfpathcurveto{\pgfqpoint{2.648386in}{2.997561in}}{\pgfqpoint{2.658985in}{2.993170in}}{\pgfqpoint{2.670035in}{2.993170in}}%
\pgfpathclose%
\pgfusepath{stroke,fill}%
\end{pgfscope}%
\begin{pgfscope}%
\pgfpathrectangle{\pgfqpoint{0.600000in}{0.600000in}}{\pgfqpoint{3.900000in}{3.900000in}}%
\pgfusepath{clip}%
\pgfsetbuttcap%
\pgfsetroundjoin%
\definecolor{currentfill}{rgb}{0.121569,0.466667,0.705882}%
\pgfsetfillcolor{currentfill}%
\pgfsetlinewidth{1.003750pt}%
\definecolor{currentstroke}{rgb}{0.121569,0.466667,0.705882}%
\pgfsetstrokecolor{currentstroke}%
\pgfsetdash{}{0pt}%
\pgfpathmoveto{\pgfqpoint{2.379297in}{1.790012in}}%
\pgfpathcurveto{\pgfqpoint{2.390347in}{1.790012in}}{\pgfqpoint{2.400946in}{1.794402in}}{\pgfqpoint{2.408760in}{1.802216in}}%
\pgfpathcurveto{\pgfqpoint{2.416574in}{1.810029in}}{\pgfqpoint{2.420964in}{1.820628in}}{\pgfqpoint{2.420964in}{1.831678in}}%
\pgfpathcurveto{\pgfqpoint{2.420964in}{1.842729in}}{\pgfqpoint{2.416574in}{1.853328in}}{\pgfqpoint{2.408760in}{1.861141in}}%
\pgfpathcurveto{\pgfqpoint{2.400946in}{1.868955in}}{\pgfqpoint{2.390347in}{1.873345in}}{\pgfqpoint{2.379297in}{1.873345in}}%
\pgfpathcurveto{\pgfqpoint{2.368247in}{1.873345in}}{\pgfqpoint{2.357648in}{1.868955in}}{\pgfqpoint{2.349834in}{1.861141in}}%
\pgfpathcurveto{\pgfqpoint{2.342021in}{1.853328in}}{\pgfqpoint{2.337631in}{1.842729in}}{\pgfqpoint{2.337631in}{1.831678in}}%
\pgfpathcurveto{\pgfqpoint{2.337631in}{1.820628in}}{\pgfqpoint{2.342021in}{1.810029in}}{\pgfqpoint{2.349834in}{1.802216in}}%
\pgfpathcurveto{\pgfqpoint{2.357648in}{1.794402in}}{\pgfqpoint{2.368247in}{1.790012in}}{\pgfqpoint{2.379297in}{1.790012in}}%
\pgfpathclose%
\pgfusepath{stroke,fill}%
\end{pgfscope}%
\begin{pgfscope}%
\pgfpathrectangle{\pgfqpoint{0.600000in}{0.600000in}}{\pgfqpoint{3.900000in}{3.900000in}}%
\pgfusepath{clip}%
\pgfsetbuttcap%
\pgfsetroundjoin%
\definecolor{currentfill}{rgb}{0.121569,0.466667,0.705882}%
\pgfsetfillcolor{currentfill}%
\pgfsetlinewidth{1.003750pt}%
\definecolor{currentstroke}{rgb}{0.121569,0.466667,0.705882}%
\pgfsetstrokecolor{currentstroke}%
\pgfsetdash{}{0pt}%
\pgfpathmoveto{\pgfqpoint{2.062767in}{2.795754in}}%
\pgfpathcurveto{\pgfqpoint{2.073817in}{2.795754in}}{\pgfqpoint{2.084416in}{2.800144in}}{\pgfqpoint{2.092230in}{2.807957in}}%
\pgfpathcurveto{\pgfqpoint{2.100043in}{2.815771in}}{\pgfqpoint{2.104434in}{2.826370in}}{\pgfqpoint{2.104434in}{2.837420in}}%
\pgfpathcurveto{\pgfqpoint{2.104434in}{2.848470in}}{\pgfqpoint{2.100043in}{2.859069in}}{\pgfqpoint{2.092230in}{2.866883in}}%
\pgfpathcurveto{\pgfqpoint{2.084416in}{2.874697in}}{\pgfqpoint{2.073817in}{2.879087in}}{\pgfqpoint{2.062767in}{2.879087in}}%
\pgfpathcurveto{\pgfqpoint{2.051717in}{2.879087in}}{\pgfqpoint{2.041118in}{2.874697in}}{\pgfqpoint{2.033304in}{2.866883in}}%
\pgfpathcurveto{\pgfqpoint{2.025491in}{2.859069in}}{\pgfqpoint{2.021100in}{2.848470in}}{\pgfqpoint{2.021100in}{2.837420in}}%
\pgfpathcurveto{\pgfqpoint{2.021100in}{2.826370in}}{\pgfqpoint{2.025491in}{2.815771in}}{\pgfqpoint{2.033304in}{2.807957in}}%
\pgfpathcurveto{\pgfqpoint{2.041118in}{2.800144in}}{\pgfqpoint{2.051717in}{2.795754in}}{\pgfqpoint{2.062767in}{2.795754in}}%
\pgfpathclose%
\pgfusepath{stroke,fill}%
\end{pgfscope}%
\begin{pgfscope}%
\pgfpathrectangle{\pgfqpoint{0.600000in}{0.600000in}}{\pgfqpoint{3.900000in}{3.900000in}}%
\pgfusepath{clip}%
\pgfsetbuttcap%
\pgfsetroundjoin%
\definecolor{currentfill}{rgb}{0.121569,0.466667,0.705882}%
\pgfsetfillcolor{currentfill}%
\pgfsetlinewidth{1.003750pt}%
\definecolor{currentstroke}{rgb}{0.121569,0.466667,0.705882}%
\pgfsetstrokecolor{currentstroke}%
\pgfsetdash{}{0pt}%
\pgfpathmoveto{\pgfqpoint{2.112679in}{2.142903in}}%
\pgfpathcurveto{\pgfqpoint{2.123729in}{2.142903in}}{\pgfqpoint{2.134328in}{2.147293in}}{\pgfqpoint{2.142142in}{2.155107in}}%
\pgfpathcurveto{\pgfqpoint{2.149955in}{2.162921in}}{\pgfqpoint{2.154345in}{2.173520in}}{\pgfqpoint{2.154345in}{2.184570in}}%
\pgfpathcurveto{\pgfqpoint{2.154345in}{2.195620in}}{\pgfqpoint{2.149955in}{2.206219in}}{\pgfqpoint{2.142142in}{2.214032in}}%
\pgfpathcurveto{\pgfqpoint{2.134328in}{2.221846in}}{\pgfqpoint{2.123729in}{2.226236in}}{\pgfqpoint{2.112679in}{2.226236in}}%
\pgfpathcurveto{\pgfqpoint{2.101629in}{2.226236in}}{\pgfqpoint{2.091030in}{2.221846in}}{\pgfqpoint{2.083216in}{2.214032in}}%
\pgfpathcurveto{\pgfqpoint{2.075402in}{2.206219in}}{\pgfqpoint{2.071012in}{2.195620in}}{\pgfqpoint{2.071012in}{2.184570in}}%
\pgfpathcurveto{\pgfqpoint{2.071012in}{2.173520in}}{\pgfqpoint{2.075402in}{2.162921in}}{\pgfqpoint{2.083216in}{2.155107in}}%
\pgfpathcurveto{\pgfqpoint{2.091030in}{2.147293in}}{\pgfqpoint{2.101629in}{2.142903in}}{\pgfqpoint{2.112679in}{2.142903in}}%
\pgfpathclose%
\pgfusepath{stroke,fill}%
\end{pgfscope}%
\begin{pgfscope}%
\pgfpathrectangle{\pgfqpoint{0.600000in}{0.600000in}}{\pgfqpoint{3.900000in}{3.900000in}}%
\pgfusepath{clip}%
\pgfsetbuttcap%
\pgfsetroundjoin%
\definecolor{currentfill}{rgb}{0.121569,0.466667,0.705882}%
\pgfsetfillcolor{currentfill}%
\pgfsetlinewidth{1.003750pt}%
\definecolor{currentstroke}{rgb}{0.121569,0.466667,0.705882}%
\pgfsetstrokecolor{currentstroke}%
\pgfsetdash{}{0pt}%
\pgfpathmoveto{\pgfqpoint{1.761548in}{2.487422in}}%
\pgfpathcurveto{\pgfqpoint{1.772598in}{2.487422in}}{\pgfqpoint{1.783197in}{2.491812in}}{\pgfqpoint{1.791011in}{2.499626in}}%
\pgfpathcurveto{\pgfqpoint{1.798824in}{2.507439in}}{\pgfqpoint{1.803214in}{2.518038in}}{\pgfqpoint{1.803214in}{2.529088in}}%
\pgfpathcurveto{\pgfqpoint{1.803214in}{2.540138in}}{\pgfqpoint{1.798824in}{2.550737in}}{\pgfqpoint{1.791011in}{2.558551in}}%
\pgfpathcurveto{\pgfqpoint{1.783197in}{2.566365in}}{\pgfqpoint{1.772598in}{2.570755in}}{\pgfqpoint{1.761548in}{2.570755in}}%
\pgfpathcurveto{\pgfqpoint{1.750498in}{2.570755in}}{\pgfqpoint{1.739899in}{2.566365in}}{\pgfqpoint{1.732085in}{2.558551in}}%
\pgfpathcurveto{\pgfqpoint{1.724271in}{2.550737in}}{\pgfqpoint{1.719881in}{2.540138in}}{\pgfqpoint{1.719881in}{2.529088in}}%
\pgfpathcurveto{\pgfqpoint{1.719881in}{2.518038in}}{\pgfqpoint{1.724271in}{2.507439in}}{\pgfqpoint{1.732085in}{2.499626in}}%
\pgfpathcurveto{\pgfqpoint{1.739899in}{2.491812in}}{\pgfqpoint{1.750498in}{2.487422in}}{\pgfqpoint{1.761548in}{2.487422in}}%
\pgfpathclose%
\pgfusepath{stroke,fill}%
\end{pgfscope}%
\begin{pgfscope}%
\pgfpathrectangle{\pgfqpoint{0.600000in}{0.600000in}}{\pgfqpoint{3.900000in}{3.900000in}}%
\pgfusepath{clip}%
\pgfsetbuttcap%
\pgfsetroundjoin%
\definecolor{currentfill}{rgb}{0.121569,0.466667,0.705882}%
\pgfsetfillcolor{currentfill}%
\pgfsetlinewidth{1.003750pt}%
\definecolor{currentstroke}{rgb}{0.121569,0.466667,0.705882}%
\pgfsetstrokecolor{currentstroke}%
\pgfsetdash{}{0pt}%
\pgfpathmoveto{\pgfqpoint{1.689573in}{2.308423in}}%
\pgfpathcurveto{\pgfqpoint{1.700623in}{2.308423in}}{\pgfqpoint{1.711222in}{2.312813in}}{\pgfqpoint{1.719036in}{2.320627in}}%
\pgfpathcurveto{\pgfqpoint{1.726849in}{2.328441in}}{\pgfqpoint{1.731240in}{2.339040in}}{\pgfqpoint{1.731240in}{2.350090in}}%
\pgfpathcurveto{\pgfqpoint{1.731240in}{2.361140in}}{\pgfqpoint{1.726849in}{2.371739in}}{\pgfqpoint{1.719036in}{2.379553in}}%
\pgfpathcurveto{\pgfqpoint{1.711222in}{2.387366in}}{\pgfqpoint{1.700623in}{2.391756in}}{\pgfqpoint{1.689573in}{2.391756in}}%
\pgfpathcurveto{\pgfqpoint{1.678523in}{2.391756in}}{\pgfqpoint{1.667924in}{2.387366in}}{\pgfqpoint{1.660110in}{2.379553in}}%
\pgfpathcurveto{\pgfqpoint{1.652296in}{2.371739in}}{\pgfqpoint{1.647906in}{2.361140in}}{\pgfqpoint{1.647906in}{2.350090in}}%
\pgfpathcurveto{\pgfqpoint{1.647906in}{2.339040in}}{\pgfqpoint{1.652296in}{2.328441in}}{\pgfqpoint{1.660110in}{2.320627in}}%
\pgfpathcurveto{\pgfqpoint{1.667924in}{2.312813in}}{\pgfqpoint{1.678523in}{2.308423in}}{\pgfqpoint{1.689573in}{2.308423in}}%
\pgfpathclose%
\pgfusepath{stroke,fill}%
\end{pgfscope}%
\begin{pgfscope}%
\pgfpathrectangle{\pgfqpoint{0.600000in}{0.600000in}}{\pgfqpoint{3.900000in}{3.900000in}}%
\pgfusepath{clip}%
\pgfsetbuttcap%
\pgfsetroundjoin%
\definecolor{currentfill}{rgb}{0.121569,0.466667,0.705882}%
\pgfsetfillcolor{currentfill}%
\pgfsetlinewidth{1.003750pt}%
\definecolor{currentstroke}{rgb}{0.121569,0.466667,0.705882}%
\pgfsetstrokecolor{currentstroke}%
\pgfsetdash{}{0pt}%
\pgfpathmoveto{\pgfqpoint{2.031592in}{2.108918in}}%
\pgfpathcurveto{\pgfqpoint{2.042642in}{2.108918in}}{\pgfqpoint{2.053241in}{2.113308in}}{\pgfqpoint{2.061055in}{2.121122in}}%
\pgfpathcurveto{\pgfqpoint{2.068869in}{2.128936in}}{\pgfqpoint{2.073259in}{2.139535in}}{\pgfqpoint{2.073259in}{2.150585in}}%
\pgfpathcurveto{\pgfqpoint{2.073259in}{2.161635in}}{\pgfqpoint{2.068869in}{2.172234in}}{\pgfqpoint{2.061055in}{2.180048in}}%
\pgfpathcurveto{\pgfqpoint{2.053241in}{2.187861in}}{\pgfqpoint{2.042642in}{2.192251in}}{\pgfqpoint{2.031592in}{2.192251in}}%
\pgfpathcurveto{\pgfqpoint{2.020542in}{2.192251in}}{\pgfqpoint{2.009943in}{2.187861in}}{\pgfqpoint{2.002129in}{2.180048in}}%
\pgfpathcurveto{\pgfqpoint{1.994316in}{2.172234in}}{\pgfqpoint{1.989926in}{2.161635in}}{\pgfqpoint{1.989926in}{2.150585in}}%
\pgfpathcurveto{\pgfqpoint{1.989926in}{2.139535in}}{\pgfqpoint{1.994316in}{2.128936in}}{\pgfqpoint{2.002129in}{2.121122in}}%
\pgfpathcurveto{\pgfqpoint{2.009943in}{2.113308in}}{\pgfqpoint{2.020542in}{2.108918in}}{\pgfqpoint{2.031592in}{2.108918in}}%
\pgfpathclose%
\pgfusepath{stroke,fill}%
\end{pgfscope}%
\begin{pgfscope}%
\pgfpathrectangle{\pgfqpoint{0.600000in}{0.600000in}}{\pgfqpoint{3.900000in}{3.900000in}}%
\pgfusepath{clip}%
\pgfsetbuttcap%
\pgfsetroundjoin%
\definecolor{currentfill}{rgb}{0.121569,0.466667,0.705882}%
\pgfsetfillcolor{currentfill}%
\pgfsetlinewidth{1.003750pt}%
\definecolor{currentstroke}{rgb}{0.121569,0.466667,0.705882}%
\pgfsetstrokecolor{currentstroke}%
\pgfsetdash{}{0pt}%
\pgfpathmoveto{\pgfqpoint{2.939125in}{3.153551in}}%
\pgfpathcurveto{\pgfqpoint{2.950175in}{3.153551in}}{\pgfqpoint{2.960774in}{3.157941in}}{\pgfqpoint{2.968588in}{3.165755in}}%
\pgfpathcurveto{\pgfqpoint{2.976401in}{3.173568in}}{\pgfqpoint{2.980791in}{3.184167in}}{\pgfqpoint{2.980791in}{3.195217in}}%
\pgfpathcurveto{\pgfqpoint{2.980791in}{3.206268in}}{\pgfqpoint{2.976401in}{3.216867in}}{\pgfqpoint{2.968588in}{3.224680in}}%
\pgfpathcurveto{\pgfqpoint{2.960774in}{3.232494in}}{\pgfqpoint{2.950175in}{3.236884in}}{\pgfqpoint{2.939125in}{3.236884in}}%
\pgfpathcurveto{\pgfqpoint{2.928075in}{3.236884in}}{\pgfqpoint{2.917476in}{3.232494in}}{\pgfqpoint{2.909662in}{3.224680in}}%
\pgfpathcurveto{\pgfqpoint{2.901848in}{3.216867in}}{\pgfqpoint{2.897458in}{3.206268in}}{\pgfqpoint{2.897458in}{3.195217in}}%
\pgfpathcurveto{\pgfqpoint{2.897458in}{3.184167in}}{\pgfqpoint{2.901848in}{3.173568in}}{\pgfqpoint{2.909662in}{3.165755in}}%
\pgfpathcurveto{\pgfqpoint{2.917476in}{3.157941in}}{\pgfqpoint{2.928075in}{3.153551in}}{\pgfqpoint{2.939125in}{3.153551in}}%
\pgfpathclose%
\pgfusepath{stroke,fill}%
\end{pgfscope}%
\begin{pgfscope}%
\pgfpathrectangle{\pgfqpoint{0.600000in}{0.600000in}}{\pgfqpoint{3.900000in}{3.900000in}}%
\pgfusepath{clip}%
\pgfsetbuttcap%
\pgfsetroundjoin%
\definecolor{currentfill}{rgb}{0.121569,0.466667,0.705882}%
\pgfsetfillcolor{currentfill}%
\pgfsetlinewidth{1.003750pt}%
\definecolor{currentstroke}{rgb}{0.121569,0.466667,0.705882}%
\pgfsetstrokecolor{currentstroke}%
\pgfsetdash{}{0pt}%
\pgfpathmoveto{\pgfqpoint{3.448627in}{2.157873in}}%
\pgfpathcurveto{\pgfqpoint{3.459677in}{2.157873in}}{\pgfqpoint{3.470276in}{2.162263in}}{\pgfqpoint{3.478089in}{2.170077in}}%
\pgfpathcurveto{\pgfqpoint{3.485903in}{2.177890in}}{\pgfqpoint{3.490293in}{2.188489in}}{\pgfqpoint{3.490293in}{2.199539in}}%
\pgfpathcurveto{\pgfqpoint{3.490293in}{2.210590in}}{\pgfqpoint{3.485903in}{2.221189in}}{\pgfqpoint{3.478089in}{2.229002in}}%
\pgfpathcurveto{\pgfqpoint{3.470276in}{2.236816in}}{\pgfqpoint{3.459677in}{2.241206in}}{\pgfqpoint{3.448627in}{2.241206in}}%
\pgfpathcurveto{\pgfqpoint{3.437577in}{2.241206in}}{\pgfqpoint{3.426978in}{2.236816in}}{\pgfqpoint{3.419164in}{2.229002in}}%
\pgfpathcurveto{\pgfqpoint{3.411350in}{2.221189in}}{\pgfqpoint{3.406960in}{2.210590in}}{\pgfqpoint{3.406960in}{2.199539in}}%
\pgfpathcurveto{\pgfqpoint{3.406960in}{2.188489in}}{\pgfqpoint{3.411350in}{2.177890in}}{\pgfqpoint{3.419164in}{2.170077in}}%
\pgfpathcurveto{\pgfqpoint{3.426978in}{2.162263in}}{\pgfqpoint{3.437577in}{2.157873in}}{\pgfqpoint{3.448627in}{2.157873in}}%
\pgfpathclose%
\pgfusepath{stroke,fill}%
\end{pgfscope}%
\begin{pgfscope}%
\pgfpathrectangle{\pgfqpoint{0.600000in}{0.600000in}}{\pgfqpoint{3.900000in}{3.900000in}}%
\pgfusepath{clip}%
\pgfsetbuttcap%
\pgfsetroundjoin%
\definecolor{currentfill}{rgb}{0.121569,0.466667,0.705882}%
\pgfsetfillcolor{currentfill}%
\pgfsetlinewidth{1.003750pt}%
\definecolor{currentstroke}{rgb}{0.121569,0.466667,0.705882}%
\pgfsetstrokecolor{currentstroke}%
\pgfsetdash{}{0pt}%
\pgfpathmoveto{\pgfqpoint{3.280564in}{2.222351in}}%
\pgfpathcurveto{\pgfqpoint{3.291614in}{2.222351in}}{\pgfqpoint{3.302213in}{2.226741in}}{\pgfqpoint{3.310027in}{2.234555in}}%
\pgfpathcurveto{\pgfqpoint{3.317840in}{2.242368in}}{\pgfqpoint{3.322231in}{2.252967in}}{\pgfqpoint{3.322231in}{2.264017in}}%
\pgfpathcurveto{\pgfqpoint{3.322231in}{2.275068in}}{\pgfqpoint{3.317840in}{2.285667in}}{\pgfqpoint{3.310027in}{2.293480in}}%
\pgfpathcurveto{\pgfqpoint{3.302213in}{2.301294in}}{\pgfqpoint{3.291614in}{2.305684in}}{\pgfqpoint{3.280564in}{2.305684in}}%
\pgfpathcurveto{\pgfqpoint{3.269514in}{2.305684in}}{\pgfqpoint{3.258915in}{2.301294in}}{\pgfqpoint{3.251101in}{2.293480in}}%
\pgfpathcurveto{\pgfqpoint{3.243287in}{2.285667in}}{\pgfqpoint{3.238897in}{2.275068in}}{\pgfqpoint{3.238897in}{2.264017in}}%
\pgfpathcurveto{\pgfqpoint{3.238897in}{2.252967in}}{\pgfqpoint{3.243287in}{2.242368in}}{\pgfqpoint{3.251101in}{2.234555in}}%
\pgfpathcurveto{\pgfqpoint{3.258915in}{2.226741in}}{\pgfqpoint{3.269514in}{2.222351in}}{\pgfqpoint{3.280564in}{2.222351in}}%
\pgfpathclose%
\pgfusepath{stroke,fill}%
\end{pgfscope}%
\begin{pgfscope}%
\pgfpathrectangle{\pgfqpoint{0.600000in}{0.600000in}}{\pgfqpoint{3.900000in}{3.900000in}}%
\pgfusepath{clip}%
\pgfsetbuttcap%
\pgfsetroundjoin%
\definecolor{currentfill}{rgb}{0.121569,0.466667,0.705882}%
\pgfsetfillcolor{currentfill}%
\pgfsetlinewidth{1.003750pt}%
\definecolor{currentstroke}{rgb}{0.121569,0.466667,0.705882}%
\pgfsetstrokecolor{currentstroke}%
\pgfsetdash{}{0pt}%
\pgfpathmoveto{\pgfqpoint{1.876026in}{3.263007in}}%
\pgfpathcurveto{\pgfqpoint{1.887077in}{3.263007in}}{\pgfqpoint{1.897676in}{3.267397in}}{\pgfqpoint{1.905489in}{3.275211in}}%
\pgfpathcurveto{\pgfqpoint{1.913303in}{3.283025in}}{\pgfqpoint{1.917693in}{3.293624in}}{\pgfqpoint{1.917693in}{3.304674in}}%
\pgfpathcurveto{\pgfqpoint{1.917693in}{3.315724in}}{\pgfqpoint{1.913303in}{3.326323in}}{\pgfqpoint{1.905489in}{3.334137in}}%
\pgfpathcurveto{\pgfqpoint{1.897676in}{3.341950in}}{\pgfqpoint{1.887077in}{3.346341in}}{\pgfqpoint{1.876026in}{3.346341in}}%
\pgfpathcurveto{\pgfqpoint{1.864976in}{3.346341in}}{\pgfqpoint{1.854377in}{3.341950in}}{\pgfqpoint{1.846564in}{3.334137in}}%
\pgfpathcurveto{\pgfqpoint{1.838750in}{3.326323in}}{\pgfqpoint{1.834360in}{3.315724in}}{\pgfqpoint{1.834360in}{3.304674in}}%
\pgfpathcurveto{\pgfqpoint{1.834360in}{3.293624in}}{\pgfqpoint{1.838750in}{3.283025in}}{\pgfqpoint{1.846564in}{3.275211in}}%
\pgfpathcurveto{\pgfqpoint{1.854377in}{3.267397in}}{\pgfqpoint{1.864976in}{3.263007in}}{\pgfqpoint{1.876026in}{3.263007in}}%
\pgfpathclose%
\pgfusepath{stroke,fill}%
\end{pgfscope}%
\begin{pgfscope}%
\pgfpathrectangle{\pgfqpoint{0.600000in}{0.600000in}}{\pgfqpoint{3.900000in}{3.900000in}}%
\pgfusepath{clip}%
\pgfsetbuttcap%
\pgfsetroundjoin%
\definecolor{currentfill}{rgb}{0.121569,0.466667,0.705882}%
\pgfsetfillcolor{currentfill}%
\pgfsetlinewidth{1.003750pt}%
\definecolor{currentstroke}{rgb}{0.121569,0.466667,0.705882}%
\pgfsetstrokecolor{currentstroke}%
\pgfsetdash{}{0pt}%
\pgfpathmoveto{\pgfqpoint{2.590784in}{2.989740in}}%
\pgfpathcurveto{\pgfqpoint{2.601834in}{2.989740in}}{\pgfqpoint{2.612433in}{2.994131in}}{\pgfqpoint{2.620247in}{3.001944in}}%
\pgfpathcurveto{\pgfqpoint{2.628060in}{3.009758in}}{\pgfqpoint{2.632451in}{3.020357in}}{\pgfqpoint{2.632451in}{3.031407in}}%
\pgfpathcurveto{\pgfqpoint{2.632451in}{3.042457in}}{\pgfqpoint{2.628060in}{3.053056in}}{\pgfqpoint{2.620247in}{3.060870in}}%
\pgfpathcurveto{\pgfqpoint{2.612433in}{3.068683in}}{\pgfqpoint{2.601834in}{3.073074in}}{\pgfqpoint{2.590784in}{3.073074in}}%
\pgfpathcurveto{\pgfqpoint{2.579734in}{3.073074in}}{\pgfqpoint{2.569135in}{3.068683in}}{\pgfqpoint{2.561321in}{3.060870in}}%
\pgfpathcurveto{\pgfqpoint{2.553508in}{3.053056in}}{\pgfqpoint{2.549117in}{3.042457in}}{\pgfqpoint{2.549117in}{3.031407in}}%
\pgfpathcurveto{\pgfqpoint{2.549117in}{3.020357in}}{\pgfqpoint{2.553508in}{3.009758in}}{\pgfqpoint{2.561321in}{3.001944in}}%
\pgfpathcurveto{\pgfqpoint{2.569135in}{2.994131in}}{\pgfqpoint{2.579734in}{2.989740in}}{\pgfqpoint{2.590784in}{2.989740in}}%
\pgfpathclose%
\pgfusepath{stroke,fill}%
\end{pgfscope}%
\begin{pgfscope}%
\pgfpathrectangle{\pgfqpoint{0.600000in}{0.600000in}}{\pgfqpoint{3.900000in}{3.900000in}}%
\pgfusepath{clip}%
\pgfsetbuttcap%
\pgfsetroundjoin%
\definecolor{currentfill}{rgb}{0.121569,0.466667,0.705882}%
\pgfsetfillcolor{currentfill}%
\pgfsetlinewidth{1.003750pt}%
\definecolor{currentstroke}{rgb}{0.121569,0.466667,0.705882}%
\pgfsetstrokecolor{currentstroke}%
\pgfsetdash{}{0pt}%
\pgfpathmoveto{\pgfqpoint{2.653379in}{2.794848in}}%
\pgfpathcurveto{\pgfqpoint{2.664429in}{2.794848in}}{\pgfqpoint{2.675028in}{2.799238in}}{\pgfqpoint{2.682842in}{2.807052in}}%
\pgfpathcurveto{\pgfqpoint{2.690655in}{2.814866in}}{\pgfqpoint{2.695045in}{2.825465in}}{\pgfqpoint{2.695045in}{2.836515in}}%
\pgfpathcurveto{\pgfqpoint{2.695045in}{2.847565in}}{\pgfqpoint{2.690655in}{2.858164in}}{\pgfqpoint{2.682842in}{2.865978in}}%
\pgfpathcurveto{\pgfqpoint{2.675028in}{2.873791in}}{\pgfqpoint{2.664429in}{2.878182in}}{\pgfqpoint{2.653379in}{2.878182in}}%
\pgfpathcurveto{\pgfqpoint{2.642329in}{2.878182in}}{\pgfqpoint{2.631730in}{2.873791in}}{\pgfqpoint{2.623916in}{2.865978in}}%
\pgfpathcurveto{\pgfqpoint{2.616102in}{2.858164in}}{\pgfqpoint{2.611712in}{2.847565in}}{\pgfqpoint{2.611712in}{2.836515in}}%
\pgfpathcurveto{\pgfqpoint{2.611712in}{2.825465in}}{\pgfqpoint{2.616102in}{2.814866in}}{\pgfqpoint{2.623916in}{2.807052in}}%
\pgfpathcurveto{\pgfqpoint{2.631730in}{2.799238in}}{\pgfqpoint{2.642329in}{2.794848in}}{\pgfqpoint{2.653379in}{2.794848in}}%
\pgfpathclose%
\pgfusepath{stroke,fill}%
\end{pgfscope}%
\begin{pgfscope}%
\pgfpathrectangle{\pgfqpoint{0.600000in}{0.600000in}}{\pgfqpoint{3.900000in}{3.900000in}}%
\pgfusepath{clip}%
\pgfsetbuttcap%
\pgfsetroundjoin%
\definecolor{currentfill}{rgb}{0.121569,0.466667,0.705882}%
\pgfsetfillcolor{currentfill}%
\pgfsetlinewidth{1.003750pt}%
\definecolor{currentstroke}{rgb}{0.121569,0.466667,0.705882}%
\pgfsetstrokecolor{currentstroke}%
\pgfsetdash{}{0pt}%
\pgfpathmoveto{\pgfqpoint{2.895494in}{3.149380in}}%
\pgfpathcurveto{\pgfqpoint{2.906544in}{3.149380in}}{\pgfqpoint{2.917143in}{3.153771in}}{\pgfqpoint{2.924957in}{3.161584in}}%
\pgfpathcurveto{\pgfqpoint{2.932771in}{3.169398in}}{\pgfqpoint{2.937161in}{3.179997in}}{\pgfqpoint{2.937161in}{3.191047in}}%
\pgfpathcurveto{\pgfqpoint{2.937161in}{3.202097in}}{\pgfqpoint{2.932771in}{3.212696in}}{\pgfqpoint{2.924957in}{3.220510in}}%
\pgfpathcurveto{\pgfqpoint{2.917143in}{3.228324in}}{\pgfqpoint{2.906544in}{3.232714in}}{\pgfqpoint{2.895494in}{3.232714in}}%
\pgfpathcurveto{\pgfqpoint{2.884444in}{3.232714in}}{\pgfqpoint{2.873845in}{3.228324in}}{\pgfqpoint{2.866031in}{3.220510in}}%
\pgfpathcurveto{\pgfqpoint{2.858218in}{3.212696in}}{\pgfqpoint{2.853828in}{3.202097in}}{\pgfqpoint{2.853828in}{3.191047in}}%
\pgfpathcurveto{\pgfqpoint{2.853828in}{3.179997in}}{\pgfqpoint{2.858218in}{3.169398in}}{\pgfqpoint{2.866031in}{3.161584in}}%
\pgfpathcurveto{\pgfqpoint{2.873845in}{3.153771in}}{\pgfqpoint{2.884444in}{3.149380in}}{\pgfqpoint{2.895494in}{3.149380in}}%
\pgfpathclose%
\pgfusepath{stroke,fill}%
\end{pgfscope}%
\begin{pgfscope}%
\pgfpathrectangle{\pgfqpoint{0.600000in}{0.600000in}}{\pgfqpoint{3.900000in}{3.900000in}}%
\pgfusepath{clip}%
\pgfsetbuttcap%
\pgfsetroundjoin%
\definecolor{currentfill}{rgb}{0.121569,0.466667,0.705882}%
\pgfsetfillcolor{currentfill}%
\pgfsetlinewidth{1.003750pt}%
\definecolor{currentstroke}{rgb}{0.121569,0.466667,0.705882}%
\pgfsetstrokecolor{currentstroke}%
\pgfsetdash{}{0pt}%
\pgfpathmoveto{\pgfqpoint{2.920640in}{3.076874in}}%
\pgfpathcurveto{\pgfqpoint{2.931690in}{3.076874in}}{\pgfqpoint{2.942289in}{3.081265in}}{\pgfqpoint{2.950103in}{3.089078in}}%
\pgfpathcurveto{\pgfqpoint{2.957917in}{3.096892in}}{\pgfqpoint{2.962307in}{3.107491in}}{\pgfqpoint{2.962307in}{3.118541in}}%
\pgfpathcurveto{\pgfqpoint{2.962307in}{3.129591in}}{\pgfqpoint{2.957917in}{3.140190in}}{\pgfqpoint{2.950103in}{3.148004in}}%
\pgfpathcurveto{\pgfqpoint{2.942289in}{3.155817in}}{\pgfqpoint{2.931690in}{3.160208in}}{\pgfqpoint{2.920640in}{3.160208in}}%
\pgfpathcurveto{\pgfqpoint{2.909590in}{3.160208in}}{\pgfqpoint{2.898991in}{3.155817in}}{\pgfqpoint{2.891177in}{3.148004in}}%
\pgfpathcurveto{\pgfqpoint{2.883364in}{3.140190in}}{\pgfqpoint{2.878974in}{3.129591in}}{\pgfqpoint{2.878974in}{3.118541in}}%
\pgfpathcurveto{\pgfqpoint{2.878974in}{3.107491in}}{\pgfqpoint{2.883364in}{3.096892in}}{\pgfqpoint{2.891177in}{3.089078in}}%
\pgfpathcurveto{\pgfqpoint{2.898991in}{3.081265in}}{\pgfqpoint{2.909590in}{3.076874in}}{\pgfqpoint{2.920640in}{3.076874in}}%
\pgfpathclose%
\pgfusepath{stroke,fill}%
\end{pgfscope}%
\begin{pgfscope}%
\pgfpathrectangle{\pgfqpoint{0.600000in}{0.600000in}}{\pgfqpoint{3.900000in}{3.900000in}}%
\pgfusepath{clip}%
\pgfsetbuttcap%
\pgfsetroundjoin%
\definecolor{currentfill}{rgb}{0.121569,0.466667,0.705882}%
\pgfsetfillcolor{currentfill}%
\pgfsetlinewidth{1.003750pt}%
\definecolor{currentstroke}{rgb}{0.121569,0.466667,0.705882}%
\pgfsetstrokecolor{currentstroke}%
\pgfsetdash{}{0pt}%
\pgfpathmoveto{\pgfqpoint{2.685519in}{2.718356in}}%
\pgfpathcurveto{\pgfqpoint{2.696569in}{2.718356in}}{\pgfqpoint{2.707168in}{2.722746in}}{\pgfqpoint{2.714982in}{2.730560in}}%
\pgfpathcurveto{\pgfqpoint{2.722795in}{2.738373in}}{\pgfqpoint{2.727186in}{2.748972in}}{\pgfqpoint{2.727186in}{2.760023in}}%
\pgfpathcurveto{\pgfqpoint{2.727186in}{2.771073in}}{\pgfqpoint{2.722795in}{2.781672in}}{\pgfqpoint{2.714982in}{2.789485in}}%
\pgfpathcurveto{\pgfqpoint{2.707168in}{2.797299in}}{\pgfqpoint{2.696569in}{2.801689in}}{\pgfqpoint{2.685519in}{2.801689in}}%
\pgfpathcurveto{\pgfqpoint{2.674469in}{2.801689in}}{\pgfqpoint{2.663870in}{2.797299in}}{\pgfqpoint{2.656056in}{2.789485in}}%
\pgfpathcurveto{\pgfqpoint{2.648242in}{2.781672in}}{\pgfqpoint{2.643852in}{2.771073in}}{\pgfqpoint{2.643852in}{2.760023in}}%
\pgfpathcurveto{\pgfqpoint{2.643852in}{2.748972in}}{\pgfqpoint{2.648242in}{2.738373in}}{\pgfqpoint{2.656056in}{2.730560in}}%
\pgfpathcurveto{\pgfqpoint{2.663870in}{2.722746in}}{\pgfqpoint{2.674469in}{2.718356in}}{\pgfqpoint{2.685519in}{2.718356in}}%
\pgfpathclose%
\pgfusepath{stroke,fill}%
\end{pgfscope}%
\begin{pgfscope}%
\pgfpathrectangle{\pgfqpoint{0.600000in}{0.600000in}}{\pgfqpoint{3.900000in}{3.900000in}}%
\pgfusepath{clip}%
\pgfsetbuttcap%
\pgfsetroundjoin%
\definecolor{currentfill}{rgb}{0.121569,0.466667,0.705882}%
\pgfsetfillcolor{currentfill}%
\pgfsetlinewidth{1.003750pt}%
\definecolor{currentstroke}{rgb}{0.121569,0.466667,0.705882}%
\pgfsetstrokecolor{currentstroke}%
\pgfsetdash{}{0pt}%
\pgfpathmoveto{\pgfqpoint{1.918873in}{2.813843in}}%
\pgfpathcurveto{\pgfqpoint{1.929923in}{2.813843in}}{\pgfqpoint{1.940522in}{2.818234in}}{\pgfqpoint{1.948336in}{2.826047in}}%
\pgfpathcurveto{\pgfqpoint{1.956149in}{2.833861in}}{\pgfqpoint{1.960540in}{2.844460in}}{\pgfqpoint{1.960540in}{2.855510in}}%
\pgfpathcurveto{\pgfqpoint{1.960540in}{2.866560in}}{\pgfqpoint{1.956149in}{2.877159in}}{\pgfqpoint{1.948336in}{2.884973in}}%
\pgfpathcurveto{\pgfqpoint{1.940522in}{2.892787in}}{\pgfqpoint{1.929923in}{2.897177in}}{\pgfqpoint{1.918873in}{2.897177in}}%
\pgfpathcurveto{\pgfqpoint{1.907823in}{2.897177in}}{\pgfqpoint{1.897224in}{2.892787in}}{\pgfqpoint{1.889410in}{2.884973in}}%
\pgfpathcurveto{\pgfqpoint{1.881596in}{2.877159in}}{\pgfqpoint{1.877206in}{2.866560in}}{\pgfqpoint{1.877206in}{2.855510in}}%
\pgfpathcurveto{\pgfqpoint{1.877206in}{2.844460in}}{\pgfqpoint{1.881596in}{2.833861in}}{\pgfqpoint{1.889410in}{2.826047in}}%
\pgfpathcurveto{\pgfqpoint{1.897224in}{2.818234in}}{\pgfqpoint{1.907823in}{2.813843in}}{\pgfqpoint{1.918873in}{2.813843in}}%
\pgfpathclose%
\pgfusepath{stroke,fill}%
\end{pgfscope}%
\begin{pgfscope}%
\pgfpathrectangle{\pgfqpoint{0.600000in}{0.600000in}}{\pgfqpoint{3.900000in}{3.900000in}}%
\pgfusepath{clip}%
\pgfsetbuttcap%
\pgfsetroundjoin%
\definecolor{currentfill}{rgb}{0.121569,0.466667,0.705882}%
\pgfsetfillcolor{currentfill}%
\pgfsetlinewidth{1.003750pt}%
\definecolor{currentstroke}{rgb}{0.121569,0.466667,0.705882}%
\pgfsetstrokecolor{currentstroke}%
\pgfsetdash{}{0pt}%
\pgfpathmoveto{\pgfqpoint{3.238621in}{3.375004in}}%
\pgfpathcurveto{\pgfqpoint{3.249671in}{3.375004in}}{\pgfqpoint{3.260270in}{3.379394in}}{\pgfqpoint{3.268084in}{3.387208in}}%
\pgfpathcurveto{\pgfqpoint{3.275898in}{3.395022in}}{\pgfqpoint{3.280288in}{3.405621in}}{\pgfqpoint{3.280288in}{3.416671in}}%
\pgfpathcurveto{\pgfqpoint{3.280288in}{3.427721in}}{\pgfqpoint{3.275898in}{3.438320in}}{\pgfqpoint{3.268084in}{3.446134in}}%
\pgfpathcurveto{\pgfqpoint{3.260270in}{3.453947in}}{\pgfqpoint{3.249671in}{3.458337in}}{\pgfqpoint{3.238621in}{3.458337in}}%
\pgfpathcurveto{\pgfqpoint{3.227571in}{3.458337in}}{\pgfqpoint{3.216972in}{3.453947in}}{\pgfqpoint{3.209158in}{3.446134in}}%
\pgfpathcurveto{\pgfqpoint{3.201345in}{3.438320in}}{\pgfqpoint{3.196955in}{3.427721in}}{\pgfqpoint{3.196955in}{3.416671in}}%
\pgfpathcurveto{\pgfqpoint{3.196955in}{3.405621in}}{\pgfqpoint{3.201345in}{3.395022in}}{\pgfqpoint{3.209158in}{3.387208in}}%
\pgfpathcurveto{\pgfqpoint{3.216972in}{3.379394in}}{\pgfqpoint{3.227571in}{3.375004in}}{\pgfqpoint{3.238621in}{3.375004in}}%
\pgfpathclose%
\pgfusepath{stroke,fill}%
\end{pgfscope}%
\begin{pgfscope}%
\pgfpathrectangle{\pgfqpoint{0.600000in}{0.600000in}}{\pgfqpoint{3.900000in}{3.900000in}}%
\pgfusepath{clip}%
\pgfsetbuttcap%
\pgfsetroundjoin%
\definecolor{currentfill}{rgb}{0.121569,0.466667,0.705882}%
\pgfsetfillcolor{currentfill}%
\pgfsetlinewidth{1.003750pt}%
\definecolor{currentstroke}{rgb}{0.121569,0.466667,0.705882}%
\pgfsetstrokecolor{currentstroke}%
\pgfsetdash{}{0pt}%
\pgfpathmoveto{\pgfqpoint{2.449897in}{2.189706in}}%
\pgfpathcurveto{\pgfqpoint{2.460947in}{2.189706in}}{\pgfqpoint{2.471546in}{2.194096in}}{\pgfqpoint{2.479360in}{2.201910in}}%
\pgfpathcurveto{\pgfqpoint{2.487173in}{2.209723in}}{\pgfqpoint{2.491564in}{2.220322in}}{\pgfqpoint{2.491564in}{2.231372in}}%
\pgfpathcurveto{\pgfqpoint{2.491564in}{2.242422in}}{\pgfqpoint{2.487173in}{2.253021in}}{\pgfqpoint{2.479360in}{2.260835in}}%
\pgfpathcurveto{\pgfqpoint{2.471546in}{2.268649in}}{\pgfqpoint{2.460947in}{2.273039in}}{\pgfqpoint{2.449897in}{2.273039in}}%
\pgfpathcurveto{\pgfqpoint{2.438847in}{2.273039in}}{\pgfqpoint{2.428248in}{2.268649in}}{\pgfqpoint{2.420434in}{2.260835in}}%
\pgfpathcurveto{\pgfqpoint{2.412621in}{2.253021in}}{\pgfqpoint{2.408230in}{2.242422in}}{\pgfqpoint{2.408230in}{2.231372in}}%
\pgfpathcurveto{\pgfqpoint{2.408230in}{2.220322in}}{\pgfqpoint{2.412621in}{2.209723in}}{\pgfqpoint{2.420434in}{2.201910in}}%
\pgfpathcurveto{\pgfqpoint{2.428248in}{2.194096in}}{\pgfqpoint{2.438847in}{2.189706in}}{\pgfqpoint{2.449897in}{2.189706in}}%
\pgfpathclose%
\pgfusepath{stroke,fill}%
\end{pgfscope}%
\begin{pgfscope}%
\pgfpathrectangle{\pgfqpoint{0.600000in}{0.600000in}}{\pgfqpoint{3.900000in}{3.900000in}}%
\pgfusepath{clip}%
\pgfsetbuttcap%
\pgfsetroundjoin%
\definecolor{currentfill}{rgb}{0.121569,0.466667,0.705882}%
\pgfsetfillcolor{currentfill}%
\pgfsetlinewidth{1.003750pt}%
\definecolor{currentstroke}{rgb}{0.121569,0.466667,0.705882}%
\pgfsetstrokecolor{currentstroke}%
\pgfsetdash{}{0pt}%
\pgfpathmoveto{\pgfqpoint{1.876498in}{2.130377in}}%
\pgfpathcurveto{\pgfqpoint{1.887548in}{2.130377in}}{\pgfqpoint{1.898147in}{2.134767in}}{\pgfqpoint{1.905960in}{2.142580in}}%
\pgfpathcurveto{\pgfqpoint{1.913774in}{2.150394in}}{\pgfqpoint{1.918164in}{2.160993in}}{\pgfqpoint{1.918164in}{2.172043in}}%
\pgfpathcurveto{\pgfqpoint{1.918164in}{2.183093in}}{\pgfqpoint{1.913774in}{2.193692in}}{\pgfqpoint{1.905960in}{2.201506in}}%
\pgfpathcurveto{\pgfqpoint{1.898147in}{2.209320in}}{\pgfqpoint{1.887548in}{2.213710in}}{\pgfqpoint{1.876498in}{2.213710in}}%
\pgfpathcurveto{\pgfqpoint{1.865447in}{2.213710in}}{\pgfqpoint{1.854848in}{2.209320in}}{\pgfqpoint{1.847035in}{2.201506in}}%
\pgfpathcurveto{\pgfqpoint{1.839221in}{2.193692in}}{\pgfqpoint{1.834831in}{2.183093in}}{\pgfqpoint{1.834831in}{2.172043in}}%
\pgfpathcurveto{\pgfqpoint{1.834831in}{2.160993in}}{\pgfqpoint{1.839221in}{2.150394in}}{\pgfqpoint{1.847035in}{2.142580in}}%
\pgfpathcurveto{\pgfqpoint{1.854848in}{2.134767in}}{\pgfqpoint{1.865447in}{2.130377in}}{\pgfqpoint{1.876498in}{2.130377in}}%
\pgfpathclose%
\pgfusepath{stroke,fill}%
\end{pgfscope}%
\begin{pgfscope}%
\pgfpathrectangle{\pgfqpoint{0.600000in}{0.600000in}}{\pgfqpoint{3.900000in}{3.900000in}}%
\pgfusepath{clip}%
\pgfsetbuttcap%
\pgfsetroundjoin%
\definecolor{currentfill}{rgb}{0.121569,0.466667,0.705882}%
\pgfsetfillcolor{currentfill}%
\pgfsetlinewidth{1.003750pt}%
\definecolor{currentstroke}{rgb}{0.121569,0.466667,0.705882}%
\pgfsetstrokecolor{currentstroke}%
\pgfsetdash{}{0pt}%
\pgfpathmoveto{\pgfqpoint{0.705857in}{2.925807in}}%
\pgfpathcurveto{\pgfqpoint{0.716907in}{2.925807in}}{\pgfqpoint{0.727506in}{2.930197in}}{\pgfqpoint{0.735319in}{2.938011in}}%
\pgfpathcurveto{\pgfqpoint{0.743133in}{2.945824in}}{\pgfqpoint{0.747523in}{2.956423in}}{\pgfqpoint{0.747523in}{2.967473in}}%
\pgfpathcurveto{\pgfqpoint{0.747523in}{2.978523in}}{\pgfqpoint{0.743133in}{2.989123in}}{\pgfqpoint{0.735319in}{2.996936in}}%
\pgfpathcurveto{\pgfqpoint{0.727506in}{3.004750in}}{\pgfqpoint{0.716907in}{3.009140in}}{\pgfqpoint{0.705857in}{3.009140in}}%
\pgfpathcurveto{\pgfqpoint{0.694807in}{3.009140in}}{\pgfqpoint{0.684208in}{3.004750in}}{\pgfqpoint{0.676394in}{2.996936in}}%
\pgfpathcurveto{\pgfqpoint{0.668580in}{2.989123in}}{\pgfqpoint{0.664190in}{2.978523in}}{\pgfqpoint{0.664190in}{2.967473in}}%
\pgfpathcurveto{\pgfqpoint{0.664190in}{2.956423in}}{\pgfqpoint{0.668580in}{2.945824in}}{\pgfqpoint{0.676394in}{2.938011in}}%
\pgfpathcurveto{\pgfqpoint{0.684208in}{2.930197in}}{\pgfqpoint{0.694807in}{2.925807in}}{\pgfqpoint{0.705857in}{2.925807in}}%
\pgfpathclose%
\pgfusepath{stroke,fill}%
\end{pgfscope}%
\begin{pgfscope}%
\pgfpathrectangle{\pgfqpoint{0.600000in}{0.600000in}}{\pgfqpoint{3.900000in}{3.900000in}}%
\pgfusepath{clip}%
\pgfsetbuttcap%
\pgfsetroundjoin%
\definecolor{currentfill}{rgb}{0.121569,0.466667,0.705882}%
\pgfsetfillcolor{currentfill}%
\pgfsetlinewidth{1.003750pt}%
\definecolor{currentstroke}{rgb}{0.121569,0.466667,0.705882}%
\pgfsetstrokecolor{currentstroke}%
\pgfsetdash{}{0pt}%
\pgfpathmoveto{\pgfqpoint{2.143436in}{2.002690in}}%
\pgfpathcurveto{\pgfqpoint{2.154487in}{2.002690in}}{\pgfqpoint{2.165086in}{2.007080in}}{\pgfqpoint{2.172899in}{2.014893in}}%
\pgfpathcurveto{\pgfqpoint{2.180713in}{2.022707in}}{\pgfqpoint{2.185103in}{2.033306in}}{\pgfqpoint{2.185103in}{2.044356in}}%
\pgfpathcurveto{\pgfqpoint{2.185103in}{2.055406in}}{\pgfqpoint{2.180713in}{2.066005in}}{\pgfqpoint{2.172899in}{2.073819in}}%
\pgfpathcurveto{\pgfqpoint{2.165086in}{2.081633in}}{\pgfqpoint{2.154487in}{2.086023in}}{\pgfqpoint{2.143436in}{2.086023in}}%
\pgfpathcurveto{\pgfqpoint{2.132386in}{2.086023in}}{\pgfqpoint{2.121787in}{2.081633in}}{\pgfqpoint{2.113974in}{2.073819in}}%
\pgfpathcurveto{\pgfqpoint{2.106160in}{2.066005in}}{\pgfqpoint{2.101770in}{2.055406in}}{\pgfqpoint{2.101770in}{2.044356in}}%
\pgfpathcurveto{\pgfqpoint{2.101770in}{2.033306in}}{\pgfqpoint{2.106160in}{2.022707in}}{\pgfqpoint{2.113974in}{2.014893in}}%
\pgfpathcurveto{\pgfqpoint{2.121787in}{2.007080in}}{\pgfqpoint{2.132386in}{2.002690in}}{\pgfqpoint{2.143436in}{2.002690in}}%
\pgfpathclose%
\pgfusepath{stroke,fill}%
\end{pgfscope}%
\begin{pgfscope}%
\pgfpathrectangle{\pgfqpoint{0.600000in}{0.600000in}}{\pgfqpoint{3.900000in}{3.900000in}}%
\pgfusepath{clip}%
\pgfsetbuttcap%
\pgfsetroundjoin%
\definecolor{currentfill}{rgb}{0.121569,0.466667,0.705882}%
\pgfsetfillcolor{currentfill}%
\pgfsetlinewidth{1.003750pt}%
\definecolor{currentstroke}{rgb}{0.121569,0.466667,0.705882}%
\pgfsetstrokecolor{currentstroke}%
\pgfsetdash{}{0pt}%
\pgfpathmoveto{\pgfqpoint{2.701206in}{2.878420in}}%
\pgfpathcurveto{\pgfqpoint{2.712256in}{2.878420in}}{\pgfqpoint{2.722855in}{2.882810in}}{\pgfqpoint{2.730669in}{2.890624in}}%
\pgfpathcurveto{\pgfqpoint{2.738483in}{2.898437in}}{\pgfqpoint{2.742873in}{2.909036in}}{\pgfqpoint{2.742873in}{2.920086in}}%
\pgfpathcurveto{\pgfqpoint{2.742873in}{2.931137in}}{\pgfqpoint{2.738483in}{2.941736in}}{\pgfqpoint{2.730669in}{2.949549in}}%
\pgfpathcurveto{\pgfqpoint{2.722855in}{2.957363in}}{\pgfqpoint{2.712256in}{2.961753in}}{\pgfqpoint{2.701206in}{2.961753in}}%
\pgfpathcurveto{\pgfqpoint{2.690156in}{2.961753in}}{\pgfqpoint{2.679557in}{2.957363in}}{\pgfqpoint{2.671743in}{2.949549in}}%
\pgfpathcurveto{\pgfqpoint{2.663930in}{2.941736in}}{\pgfqpoint{2.659539in}{2.931137in}}{\pgfqpoint{2.659539in}{2.920086in}}%
\pgfpathcurveto{\pgfqpoint{2.659539in}{2.909036in}}{\pgfqpoint{2.663930in}{2.898437in}}{\pgfqpoint{2.671743in}{2.890624in}}%
\pgfpathcurveto{\pgfqpoint{2.679557in}{2.882810in}}{\pgfqpoint{2.690156in}{2.878420in}}{\pgfqpoint{2.701206in}{2.878420in}}%
\pgfpathclose%
\pgfusepath{stroke,fill}%
\end{pgfscope}%
\begin{pgfscope}%
\pgfpathrectangle{\pgfqpoint{0.600000in}{0.600000in}}{\pgfqpoint{3.900000in}{3.900000in}}%
\pgfusepath{clip}%
\pgfsetbuttcap%
\pgfsetroundjoin%
\definecolor{currentfill}{rgb}{0.121569,0.466667,0.705882}%
\pgfsetfillcolor{currentfill}%
\pgfsetlinewidth{1.003750pt}%
\definecolor{currentstroke}{rgb}{0.121569,0.466667,0.705882}%
\pgfsetstrokecolor{currentstroke}%
\pgfsetdash{}{0pt}%
\pgfpathmoveto{\pgfqpoint{2.606368in}{2.358144in}}%
\pgfpathcurveto{\pgfqpoint{2.617418in}{2.358144in}}{\pgfqpoint{2.628017in}{2.362535in}}{\pgfqpoint{2.635831in}{2.370348in}}%
\pgfpathcurveto{\pgfqpoint{2.643644in}{2.378162in}}{\pgfqpoint{2.648035in}{2.388761in}}{\pgfqpoint{2.648035in}{2.399811in}}%
\pgfpathcurveto{\pgfqpoint{2.648035in}{2.410861in}}{\pgfqpoint{2.643644in}{2.421460in}}{\pgfqpoint{2.635831in}{2.429274in}}%
\pgfpathcurveto{\pgfqpoint{2.628017in}{2.437088in}}{\pgfqpoint{2.617418in}{2.441478in}}{\pgfqpoint{2.606368in}{2.441478in}}%
\pgfpathcurveto{\pgfqpoint{2.595318in}{2.441478in}}{\pgfqpoint{2.584719in}{2.437088in}}{\pgfqpoint{2.576905in}{2.429274in}}%
\pgfpathcurveto{\pgfqpoint{2.569092in}{2.421460in}}{\pgfqpoint{2.564701in}{2.410861in}}{\pgfqpoint{2.564701in}{2.399811in}}%
\pgfpathcurveto{\pgfqpoint{2.564701in}{2.388761in}}{\pgfqpoint{2.569092in}{2.378162in}}{\pgfqpoint{2.576905in}{2.370348in}}%
\pgfpathcurveto{\pgfqpoint{2.584719in}{2.362535in}}{\pgfqpoint{2.595318in}{2.358144in}}{\pgfqpoint{2.606368in}{2.358144in}}%
\pgfpathclose%
\pgfusepath{stroke,fill}%
\end{pgfscope}%
\begin{pgfscope}%
\pgfpathrectangle{\pgfqpoint{0.600000in}{0.600000in}}{\pgfqpoint{3.900000in}{3.900000in}}%
\pgfusepath{clip}%
\pgfsetbuttcap%
\pgfsetroundjoin%
\definecolor{currentfill}{rgb}{0.121569,0.466667,0.705882}%
\pgfsetfillcolor{currentfill}%
\pgfsetlinewidth{1.003750pt}%
\definecolor{currentstroke}{rgb}{0.121569,0.466667,0.705882}%
\pgfsetstrokecolor{currentstroke}%
\pgfsetdash{}{0pt}%
\pgfpathmoveto{\pgfqpoint{3.160923in}{2.638668in}}%
\pgfpathcurveto{\pgfqpoint{3.171973in}{2.638668in}}{\pgfqpoint{3.182572in}{2.643059in}}{\pgfqpoint{3.190385in}{2.650872in}}%
\pgfpathcurveto{\pgfqpoint{3.198199in}{2.658686in}}{\pgfqpoint{3.202589in}{2.669285in}}{\pgfqpoint{3.202589in}{2.680335in}}%
\pgfpathcurveto{\pgfqpoint{3.202589in}{2.691385in}}{\pgfqpoint{3.198199in}{2.701984in}}{\pgfqpoint{3.190385in}{2.709798in}}%
\pgfpathcurveto{\pgfqpoint{3.182572in}{2.717612in}}{\pgfqpoint{3.171973in}{2.722002in}}{\pgfqpoint{3.160923in}{2.722002in}}%
\pgfpathcurveto{\pgfqpoint{3.149873in}{2.722002in}}{\pgfqpoint{3.139274in}{2.717612in}}{\pgfqpoint{3.131460in}{2.709798in}}%
\pgfpathcurveto{\pgfqpoint{3.123646in}{2.701984in}}{\pgfqpoint{3.119256in}{2.691385in}}{\pgfqpoint{3.119256in}{2.680335in}}%
\pgfpathcurveto{\pgfqpoint{3.119256in}{2.669285in}}{\pgfqpoint{3.123646in}{2.658686in}}{\pgfqpoint{3.131460in}{2.650872in}}%
\pgfpathcurveto{\pgfqpoint{3.139274in}{2.643059in}}{\pgfqpoint{3.149873in}{2.638668in}}{\pgfqpoint{3.160923in}{2.638668in}}%
\pgfpathclose%
\pgfusepath{stroke,fill}%
\end{pgfscope}%
\begin{pgfscope}%
\pgfpathrectangle{\pgfqpoint{0.600000in}{0.600000in}}{\pgfqpoint{3.900000in}{3.900000in}}%
\pgfusepath{clip}%
\pgfsetbuttcap%
\pgfsetroundjoin%
\definecolor{currentfill}{rgb}{0.121569,0.466667,0.705882}%
\pgfsetfillcolor{currentfill}%
\pgfsetlinewidth{1.003750pt}%
\definecolor{currentstroke}{rgb}{0.121569,0.466667,0.705882}%
\pgfsetstrokecolor{currentstroke}%
\pgfsetdash{}{0pt}%
\pgfpathmoveto{\pgfqpoint{2.022017in}{3.859460in}}%
\pgfpathcurveto{\pgfqpoint{2.033067in}{3.859460in}}{\pgfqpoint{2.043666in}{3.863851in}}{\pgfqpoint{2.051479in}{3.871664in}}%
\pgfpathcurveto{\pgfqpoint{2.059293in}{3.879478in}}{\pgfqpoint{2.063683in}{3.890077in}}{\pgfqpoint{2.063683in}{3.901127in}}%
\pgfpathcurveto{\pgfqpoint{2.063683in}{3.912177in}}{\pgfqpoint{2.059293in}{3.922776in}}{\pgfqpoint{2.051479in}{3.930590in}}%
\pgfpathcurveto{\pgfqpoint{2.043666in}{3.938404in}}{\pgfqpoint{2.033067in}{3.942794in}}{\pgfqpoint{2.022017in}{3.942794in}}%
\pgfpathcurveto{\pgfqpoint{2.010967in}{3.942794in}}{\pgfqpoint{2.000367in}{3.938404in}}{\pgfqpoint{1.992554in}{3.930590in}}%
\pgfpathcurveto{\pgfqpoint{1.984740in}{3.922776in}}{\pgfqpoint{1.980350in}{3.912177in}}{\pgfqpoint{1.980350in}{3.901127in}}%
\pgfpathcurveto{\pgfqpoint{1.980350in}{3.890077in}}{\pgfqpoint{1.984740in}{3.879478in}}{\pgfqpoint{1.992554in}{3.871664in}}%
\pgfpathcurveto{\pgfqpoint{2.000367in}{3.863851in}}{\pgfqpoint{2.010967in}{3.859460in}}{\pgfqpoint{2.022017in}{3.859460in}}%
\pgfpathclose%
\pgfusepath{stroke,fill}%
\end{pgfscope}%
\begin{pgfscope}%
\pgfpathrectangle{\pgfqpoint{0.600000in}{0.600000in}}{\pgfqpoint{3.900000in}{3.900000in}}%
\pgfusepath{clip}%
\pgfsetbuttcap%
\pgfsetroundjoin%
\definecolor{currentfill}{rgb}{0.121569,0.466667,0.705882}%
\pgfsetfillcolor{currentfill}%
\pgfsetlinewidth{1.003750pt}%
\definecolor{currentstroke}{rgb}{0.121569,0.466667,0.705882}%
\pgfsetstrokecolor{currentstroke}%
\pgfsetdash{}{0pt}%
\pgfpathmoveto{\pgfqpoint{1.968342in}{2.701762in}}%
\pgfpathcurveto{\pgfqpoint{1.979392in}{2.701762in}}{\pgfqpoint{1.989991in}{2.706152in}}{\pgfqpoint{1.997805in}{2.713965in}}%
\pgfpathcurveto{\pgfqpoint{2.005618in}{2.721779in}}{\pgfqpoint{2.010009in}{2.732378in}}{\pgfqpoint{2.010009in}{2.743428in}}%
\pgfpathcurveto{\pgfqpoint{2.010009in}{2.754478in}}{\pgfqpoint{2.005618in}{2.765077in}}{\pgfqpoint{1.997805in}{2.772891in}}%
\pgfpathcurveto{\pgfqpoint{1.989991in}{2.780705in}}{\pgfqpoint{1.979392in}{2.785095in}}{\pgfqpoint{1.968342in}{2.785095in}}%
\pgfpathcurveto{\pgfqpoint{1.957292in}{2.785095in}}{\pgfqpoint{1.946693in}{2.780705in}}{\pgfqpoint{1.938879in}{2.772891in}}%
\pgfpathcurveto{\pgfqpoint{1.931065in}{2.765077in}}{\pgfqpoint{1.926675in}{2.754478in}}{\pgfqpoint{1.926675in}{2.743428in}}%
\pgfpathcurveto{\pgfqpoint{1.926675in}{2.732378in}}{\pgfqpoint{1.931065in}{2.721779in}}{\pgfqpoint{1.938879in}{2.713965in}}%
\pgfpathcurveto{\pgfqpoint{1.946693in}{2.706152in}}{\pgfqpoint{1.957292in}{2.701762in}}{\pgfqpoint{1.968342in}{2.701762in}}%
\pgfpathclose%
\pgfusepath{stroke,fill}%
\end{pgfscope}%
\begin{pgfscope}%
\pgfpathrectangle{\pgfqpoint{0.600000in}{0.600000in}}{\pgfqpoint{3.900000in}{3.900000in}}%
\pgfusepath{clip}%
\pgfsetbuttcap%
\pgfsetroundjoin%
\definecolor{currentfill}{rgb}{0.121569,0.466667,0.705882}%
\pgfsetfillcolor{currentfill}%
\pgfsetlinewidth{1.003750pt}%
\definecolor{currentstroke}{rgb}{0.121569,0.466667,0.705882}%
\pgfsetstrokecolor{currentstroke}%
\pgfsetdash{}{0pt}%
\pgfpathmoveto{\pgfqpoint{3.070033in}{1.799848in}}%
\pgfpathcurveto{\pgfqpoint{3.081083in}{1.799848in}}{\pgfqpoint{3.091682in}{1.804238in}}{\pgfqpoint{3.099495in}{1.812052in}}%
\pgfpathcurveto{\pgfqpoint{3.107309in}{1.819865in}}{\pgfqpoint{3.111699in}{1.830464in}}{\pgfqpoint{3.111699in}{1.841515in}}%
\pgfpathcurveto{\pgfqpoint{3.111699in}{1.852565in}}{\pgfqpoint{3.107309in}{1.863164in}}{\pgfqpoint{3.099495in}{1.870977in}}%
\pgfpathcurveto{\pgfqpoint{3.091682in}{1.878791in}}{\pgfqpoint{3.081083in}{1.883181in}}{\pgfqpoint{3.070033in}{1.883181in}}%
\pgfpathcurveto{\pgfqpoint{3.058982in}{1.883181in}}{\pgfqpoint{3.048383in}{1.878791in}}{\pgfqpoint{3.040570in}{1.870977in}}%
\pgfpathcurveto{\pgfqpoint{3.032756in}{1.863164in}}{\pgfqpoint{3.028366in}{1.852565in}}{\pgfqpoint{3.028366in}{1.841515in}}%
\pgfpathcurveto{\pgfqpoint{3.028366in}{1.830464in}}{\pgfqpoint{3.032756in}{1.819865in}}{\pgfqpoint{3.040570in}{1.812052in}}%
\pgfpathcurveto{\pgfqpoint{3.048383in}{1.804238in}}{\pgfqpoint{3.058982in}{1.799848in}}{\pgfqpoint{3.070033in}{1.799848in}}%
\pgfpathclose%
\pgfusepath{stroke,fill}%
\end{pgfscope}%
\begin{pgfscope}%
\pgfpathrectangle{\pgfqpoint{0.600000in}{0.600000in}}{\pgfqpoint{3.900000in}{3.900000in}}%
\pgfusepath{clip}%
\pgfsetbuttcap%
\pgfsetroundjoin%
\definecolor{currentfill}{rgb}{0.121569,0.466667,0.705882}%
\pgfsetfillcolor{currentfill}%
\pgfsetlinewidth{1.003750pt}%
\definecolor{currentstroke}{rgb}{0.121569,0.466667,0.705882}%
\pgfsetstrokecolor{currentstroke}%
\pgfsetdash{}{0pt}%
\pgfpathmoveto{\pgfqpoint{2.647799in}{2.198174in}}%
\pgfpathcurveto{\pgfqpoint{2.658849in}{2.198174in}}{\pgfqpoint{2.669448in}{2.202564in}}{\pgfqpoint{2.677262in}{2.210378in}}%
\pgfpathcurveto{\pgfqpoint{2.685076in}{2.218192in}}{\pgfqpoint{2.689466in}{2.228791in}}{\pgfqpoint{2.689466in}{2.239841in}}%
\pgfpathcurveto{\pgfqpoint{2.689466in}{2.250891in}}{\pgfqpoint{2.685076in}{2.261490in}}{\pgfqpoint{2.677262in}{2.269304in}}%
\pgfpathcurveto{\pgfqpoint{2.669448in}{2.277117in}}{\pgfqpoint{2.658849in}{2.281508in}}{\pgfqpoint{2.647799in}{2.281508in}}%
\pgfpathcurveto{\pgfqpoint{2.636749in}{2.281508in}}{\pgfqpoint{2.626150in}{2.277117in}}{\pgfqpoint{2.618337in}{2.269304in}}%
\pgfpathcurveto{\pgfqpoint{2.610523in}{2.261490in}}{\pgfqpoint{2.606133in}{2.250891in}}{\pgfqpoint{2.606133in}{2.239841in}}%
\pgfpathcurveto{\pgfqpoint{2.606133in}{2.228791in}}{\pgfqpoint{2.610523in}{2.218192in}}{\pgfqpoint{2.618337in}{2.210378in}}%
\pgfpathcurveto{\pgfqpoint{2.626150in}{2.202564in}}{\pgfqpoint{2.636749in}{2.198174in}}{\pgfqpoint{2.647799in}{2.198174in}}%
\pgfpathclose%
\pgfusepath{stroke,fill}%
\end{pgfscope}%
\begin{pgfscope}%
\pgfpathrectangle{\pgfqpoint{0.600000in}{0.600000in}}{\pgfqpoint{3.900000in}{3.900000in}}%
\pgfusepath{clip}%
\pgfsetbuttcap%
\pgfsetroundjoin%
\definecolor{currentfill}{rgb}{0.121569,0.466667,0.705882}%
\pgfsetfillcolor{currentfill}%
\pgfsetlinewidth{1.003750pt}%
\definecolor{currentstroke}{rgb}{0.121569,0.466667,0.705882}%
\pgfsetstrokecolor{currentstroke}%
\pgfsetdash{}{0pt}%
\pgfpathmoveto{\pgfqpoint{2.276689in}{2.698601in}}%
\pgfpathcurveto{\pgfqpoint{2.287739in}{2.698601in}}{\pgfqpoint{2.298338in}{2.702992in}}{\pgfqpoint{2.306152in}{2.710805in}}%
\pgfpathcurveto{\pgfqpoint{2.313965in}{2.718619in}}{\pgfqpoint{2.318356in}{2.729218in}}{\pgfqpoint{2.318356in}{2.740268in}}%
\pgfpathcurveto{\pgfqpoint{2.318356in}{2.751318in}}{\pgfqpoint{2.313965in}{2.761917in}}{\pgfqpoint{2.306152in}{2.769731in}}%
\pgfpathcurveto{\pgfqpoint{2.298338in}{2.777544in}}{\pgfqpoint{2.287739in}{2.781935in}}{\pgfqpoint{2.276689in}{2.781935in}}%
\pgfpathcurveto{\pgfqpoint{2.265639in}{2.781935in}}{\pgfqpoint{2.255040in}{2.777544in}}{\pgfqpoint{2.247226in}{2.769731in}}%
\pgfpathcurveto{\pgfqpoint{2.239413in}{2.761917in}}{\pgfqpoint{2.235022in}{2.751318in}}{\pgfqpoint{2.235022in}{2.740268in}}%
\pgfpathcurveto{\pgfqpoint{2.235022in}{2.729218in}}{\pgfqpoint{2.239413in}{2.718619in}}{\pgfqpoint{2.247226in}{2.710805in}}%
\pgfpathcurveto{\pgfqpoint{2.255040in}{2.702992in}}{\pgfqpoint{2.265639in}{2.698601in}}{\pgfqpoint{2.276689in}{2.698601in}}%
\pgfpathclose%
\pgfusepath{stroke,fill}%
\end{pgfscope}%
\begin{pgfscope}%
\pgfpathrectangle{\pgfqpoint{0.600000in}{0.600000in}}{\pgfqpoint{3.900000in}{3.900000in}}%
\pgfusepath{clip}%
\pgfsetbuttcap%
\pgfsetroundjoin%
\definecolor{currentfill}{rgb}{0.121569,0.466667,0.705882}%
\pgfsetfillcolor{currentfill}%
\pgfsetlinewidth{1.003750pt}%
\definecolor{currentstroke}{rgb}{0.121569,0.466667,0.705882}%
\pgfsetstrokecolor{currentstroke}%
\pgfsetdash{}{0pt}%
\pgfpathmoveto{\pgfqpoint{2.464068in}{3.445712in}}%
\pgfpathcurveto{\pgfqpoint{2.475118in}{3.445712in}}{\pgfqpoint{2.485717in}{3.450102in}}{\pgfqpoint{2.493531in}{3.457916in}}%
\pgfpathcurveto{\pgfqpoint{2.501344in}{3.465729in}}{\pgfqpoint{2.505735in}{3.476328in}}{\pgfqpoint{2.505735in}{3.487379in}}%
\pgfpathcurveto{\pgfqpoint{2.505735in}{3.498429in}}{\pgfqpoint{2.501344in}{3.509028in}}{\pgfqpoint{2.493531in}{3.516841in}}%
\pgfpathcurveto{\pgfqpoint{2.485717in}{3.524655in}}{\pgfqpoint{2.475118in}{3.529045in}}{\pgfqpoint{2.464068in}{3.529045in}}%
\pgfpathcurveto{\pgfqpoint{2.453018in}{3.529045in}}{\pgfqpoint{2.442419in}{3.524655in}}{\pgfqpoint{2.434605in}{3.516841in}}%
\pgfpathcurveto{\pgfqpoint{2.426792in}{3.509028in}}{\pgfqpoint{2.422401in}{3.498429in}}{\pgfqpoint{2.422401in}{3.487379in}}%
\pgfpathcurveto{\pgfqpoint{2.422401in}{3.476328in}}{\pgfqpoint{2.426792in}{3.465729in}}{\pgfqpoint{2.434605in}{3.457916in}}%
\pgfpathcurveto{\pgfqpoint{2.442419in}{3.450102in}}{\pgfqpoint{2.453018in}{3.445712in}}{\pgfqpoint{2.464068in}{3.445712in}}%
\pgfpathclose%
\pgfusepath{stroke,fill}%
\end{pgfscope}%
\begin{pgfscope}%
\pgfpathrectangle{\pgfqpoint{0.600000in}{0.600000in}}{\pgfqpoint{3.900000in}{3.900000in}}%
\pgfusepath{clip}%
\pgfsetbuttcap%
\pgfsetroundjoin%
\definecolor{currentfill}{rgb}{0.121569,0.466667,0.705882}%
\pgfsetfillcolor{currentfill}%
\pgfsetlinewidth{1.003750pt}%
\definecolor{currentstroke}{rgb}{0.121569,0.466667,0.705882}%
\pgfsetstrokecolor{currentstroke}%
\pgfsetdash{}{0pt}%
\pgfpathmoveto{\pgfqpoint{2.119661in}{2.665676in}}%
\pgfpathcurveto{\pgfqpoint{2.130711in}{2.665676in}}{\pgfqpoint{2.141310in}{2.670066in}}{\pgfqpoint{2.149124in}{2.677879in}}%
\pgfpathcurveto{\pgfqpoint{2.156937in}{2.685693in}}{\pgfqpoint{2.161328in}{2.696292in}}{\pgfqpoint{2.161328in}{2.707342in}}%
\pgfpathcurveto{\pgfqpoint{2.161328in}{2.718392in}}{\pgfqpoint{2.156937in}{2.728991in}}{\pgfqpoint{2.149124in}{2.736805in}}%
\pgfpathcurveto{\pgfqpoint{2.141310in}{2.744619in}}{\pgfqpoint{2.130711in}{2.749009in}}{\pgfqpoint{2.119661in}{2.749009in}}%
\pgfpathcurveto{\pgfqpoint{2.108611in}{2.749009in}}{\pgfqpoint{2.098012in}{2.744619in}}{\pgfqpoint{2.090198in}{2.736805in}}%
\pgfpathcurveto{\pgfqpoint{2.082385in}{2.728991in}}{\pgfqpoint{2.077994in}{2.718392in}}{\pgfqpoint{2.077994in}{2.707342in}}%
\pgfpathcurveto{\pgfqpoint{2.077994in}{2.696292in}}{\pgfqpoint{2.082385in}{2.685693in}}{\pgfqpoint{2.090198in}{2.677879in}}%
\pgfpathcurveto{\pgfqpoint{2.098012in}{2.670066in}}{\pgfqpoint{2.108611in}{2.665676in}}{\pgfqpoint{2.119661in}{2.665676in}}%
\pgfpathclose%
\pgfusepath{stroke,fill}%
\end{pgfscope}%
\begin{pgfscope}%
\pgfpathrectangle{\pgfqpoint{0.600000in}{0.600000in}}{\pgfqpoint{3.900000in}{3.900000in}}%
\pgfusepath{clip}%
\pgfsetbuttcap%
\pgfsetroundjoin%
\definecolor{currentfill}{rgb}{0.121569,0.466667,0.705882}%
\pgfsetfillcolor{currentfill}%
\pgfsetlinewidth{1.003750pt}%
\definecolor{currentstroke}{rgb}{0.121569,0.466667,0.705882}%
\pgfsetstrokecolor{currentstroke}%
\pgfsetdash{}{0pt}%
\pgfpathmoveto{\pgfqpoint{2.150228in}{3.319642in}}%
\pgfpathcurveto{\pgfqpoint{2.161278in}{3.319642in}}{\pgfqpoint{2.171877in}{3.324033in}}{\pgfqpoint{2.179691in}{3.331846in}}%
\pgfpathcurveto{\pgfqpoint{2.187505in}{3.339660in}}{\pgfqpoint{2.191895in}{3.350259in}}{\pgfqpoint{2.191895in}{3.361309in}}%
\pgfpathcurveto{\pgfqpoint{2.191895in}{3.372359in}}{\pgfqpoint{2.187505in}{3.382958in}}{\pgfqpoint{2.179691in}{3.390772in}}%
\pgfpathcurveto{\pgfqpoint{2.171877in}{3.398585in}}{\pgfqpoint{2.161278in}{3.402976in}}{\pgfqpoint{2.150228in}{3.402976in}}%
\pgfpathcurveto{\pgfqpoint{2.139178in}{3.402976in}}{\pgfqpoint{2.128579in}{3.398585in}}{\pgfqpoint{2.120766in}{3.390772in}}%
\pgfpathcurveto{\pgfqpoint{2.112952in}{3.382958in}}{\pgfqpoint{2.108562in}{3.372359in}}{\pgfqpoint{2.108562in}{3.361309in}}%
\pgfpathcurveto{\pgfqpoint{2.108562in}{3.350259in}}{\pgfqpoint{2.112952in}{3.339660in}}{\pgfqpoint{2.120766in}{3.331846in}}%
\pgfpathcurveto{\pgfqpoint{2.128579in}{3.324033in}}{\pgfqpoint{2.139178in}{3.319642in}}{\pgfqpoint{2.150228in}{3.319642in}}%
\pgfpathclose%
\pgfusepath{stroke,fill}%
\end{pgfscope}%
\begin{pgfscope}%
\pgfpathrectangle{\pgfqpoint{0.600000in}{0.600000in}}{\pgfqpoint{3.900000in}{3.900000in}}%
\pgfusepath{clip}%
\pgfsetbuttcap%
\pgfsetroundjoin%
\definecolor{currentfill}{rgb}{0.121569,0.466667,0.705882}%
\pgfsetfillcolor{currentfill}%
\pgfsetlinewidth{1.003750pt}%
\definecolor{currentstroke}{rgb}{0.121569,0.466667,0.705882}%
\pgfsetstrokecolor{currentstroke}%
\pgfsetdash{}{0pt}%
\pgfpathmoveto{\pgfqpoint{3.046382in}{2.866798in}}%
\pgfpathcurveto{\pgfqpoint{3.057432in}{2.866798in}}{\pgfqpoint{3.068031in}{2.871188in}}{\pgfqpoint{3.075844in}{2.879002in}}%
\pgfpathcurveto{\pgfqpoint{3.083658in}{2.886816in}}{\pgfqpoint{3.088048in}{2.897415in}}{\pgfqpoint{3.088048in}{2.908465in}}%
\pgfpathcurveto{\pgfqpoint{3.088048in}{2.919515in}}{\pgfqpoint{3.083658in}{2.930114in}}{\pgfqpoint{3.075844in}{2.937928in}}%
\pgfpathcurveto{\pgfqpoint{3.068031in}{2.945741in}}{\pgfqpoint{3.057432in}{2.950132in}}{\pgfqpoint{3.046382in}{2.950132in}}%
\pgfpathcurveto{\pgfqpoint{3.035331in}{2.950132in}}{\pgfqpoint{3.024732in}{2.945741in}}{\pgfqpoint{3.016919in}{2.937928in}}%
\pgfpathcurveto{\pgfqpoint{3.009105in}{2.930114in}}{\pgfqpoint{3.004715in}{2.919515in}}{\pgfqpoint{3.004715in}{2.908465in}}%
\pgfpathcurveto{\pgfqpoint{3.004715in}{2.897415in}}{\pgfqpoint{3.009105in}{2.886816in}}{\pgfqpoint{3.016919in}{2.879002in}}%
\pgfpathcurveto{\pgfqpoint{3.024732in}{2.871188in}}{\pgfqpoint{3.035331in}{2.866798in}}{\pgfqpoint{3.046382in}{2.866798in}}%
\pgfpathclose%
\pgfusepath{stroke,fill}%
\end{pgfscope}%
\begin{pgfscope}%
\pgfpathrectangle{\pgfqpoint{0.600000in}{0.600000in}}{\pgfqpoint{3.900000in}{3.900000in}}%
\pgfusepath{clip}%
\pgfsetbuttcap%
\pgfsetroundjoin%
\definecolor{currentfill}{rgb}{0.121569,0.466667,0.705882}%
\pgfsetfillcolor{currentfill}%
\pgfsetlinewidth{1.003750pt}%
\definecolor{currentstroke}{rgb}{0.121569,0.466667,0.705882}%
\pgfsetstrokecolor{currentstroke}%
\pgfsetdash{}{0pt}%
\pgfpathmoveto{\pgfqpoint{2.181384in}{2.955012in}}%
\pgfpathcurveto{\pgfqpoint{2.192434in}{2.955012in}}{\pgfqpoint{2.203034in}{2.959403in}}{\pgfqpoint{2.210847in}{2.967216in}}%
\pgfpathcurveto{\pgfqpoint{2.218661in}{2.975030in}}{\pgfqpoint{2.223051in}{2.985629in}}{\pgfqpoint{2.223051in}{2.996679in}}%
\pgfpathcurveto{\pgfqpoint{2.223051in}{3.007729in}}{\pgfqpoint{2.218661in}{3.018328in}}{\pgfqpoint{2.210847in}{3.026142in}}%
\pgfpathcurveto{\pgfqpoint{2.203034in}{3.033955in}}{\pgfqpoint{2.192434in}{3.038346in}}{\pgfqpoint{2.181384in}{3.038346in}}%
\pgfpathcurveto{\pgfqpoint{2.170334in}{3.038346in}}{\pgfqpoint{2.159735in}{3.033955in}}{\pgfqpoint{2.151922in}{3.026142in}}%
\pgfpathcurveto{\pgfqpoint{2.144108in}{3.018328in}}{\pgfqpoint{2.139718in}{3.007729in}}{\pgfqpoint{2.139718in}{2.996679in}}%
\pgfpathcurveto{\pgfqpoint{2.139718in}{2.985629in}}{\pgfqpoint{2.144108in}{2.975030in}}{\pgfqpoint{2.151922in}{2.967216in}}%
\pgfpathcurveto{\pgfqpoint{2.159735in}{2.959403in}}{\pgfqpoint{2.170334in}{2.955012in}}{\pgfqpoint{2.181384in}{2.955012in}}%
\pgfpathclose%
\pgfusepath{stroke,fill}%
\end{pgfscope}%
\begin{pgfscope}%
\pgfpathrectangle{\pgfqpoint{0.600000in}{0.600000in}}{\pgfqpoint{3.900000in}{3.900000in}}%
\pgfusepath{clip}%
\pgfsetbuttcap%
\pgfsetroundjoin%
\definecolor{currentfill}{rgb}{0.121569,0.466667,0.705882}%
\pgfsetfillcolor{currentfill}%
\pgfsetlinewidth{1.003750pt}%
\definecolor{currentstroke}{rgb}{0.121569,0.466667,0.705882}%
\pgfsetstrokecolor{currentstroke}%
\pgfsetdash{}{0pt}%
\pgfpathmoveto{\pgfqpoint{2.205784in}{2.298450in}}%
\pgfpathcurveto{\pgfqpoint{2.216834in}{2.298450in}}{\pgfqpoint{2.227433in}{2.302841in}}{\pgfqpoint{2.235247in}{2.310654in}}%
\pgfpathcurveto{\pgfqpoint{2.243061in}{2.318468in}}{\pgfqpoint{2.247451in}{2.329067in}}{\pgfqpoint{2.247451in}{2.340117in}}%
\pgfpathcurveto{\pgfqpoint{2.247451in}{2.351167in}}{\pgfqpoint{2.243061in}{2.361766in}}{\pgfqpoint{2.235247in}{2.369580in}}%
\pgfpathcurveto{\pgfqpoint{2.227433in}{2.377394in}}{\pgfqpoint{2.216834in}{2.381784in}}{\pgfqpoint{2.205784in}{2.381784in}}%
\pgfpathcurveto{\pgfqpoint{2.194734in}{2.381784in}}{\pgfqpoint{2.184135in}{2.377394in}}{\pgfqpoint{2.176321in}{2.369580in}}%
\pgfpathcurveto{\pgfqpoint{2.168508in}{2.361766in}}{\pgfqpoint{2.164117in}{2.351167in}}{\pgfqpoint{2.164117in}{2.340117in}}%
\pgfpathcurveto{\pgfqpoint{2.164117in}{2.329067in}}{\pgfqpoint{2.168508in}{2.318468in}}{\pgfqpoint{2.176321in}{2.310654in}}%
\pgfpathcurveto{\pgfqpoint{2.184135in}{2.302841in}}{\pgfqpoint{2.194734in}{2.298450in}}{\pgfqpoint{2.205784in}{2.298450in}}%
\pgfpathclose%
\pgfusepath{stroke,fill}%
\end{pgfscope}%
\begin{pgfscope}%
\pgfpathrectangle{\pgfqpoint{0.600000in}{0.600000in}}{\pgfqpoint{3.900000in}{3.900000in}}%
\pgfusepath{clip}%
\pgfsetbuttcap%
\pgfsetroundjoin%
\definecolor{currentfill}{rgb}{0.121569,0.466667,0.705882}%
\pgfsetfillcolor{currentfill}%
\pgfsetlinewidth{1.003750pt}%
\definecolor{currentstroke}{rgb}{0.121569,0.466667,0.705882}%
\pgfsetstrokecolor{currentstroke}%
\pgfsetdash{}{0pt}%
\pgfpathmoveto{\pgfqpoint{1.926159in}{2.753032in}}%
\pgfpathcurveto{\pgfqpoint{1.937209in}{2.753032in}}{\pgfqpoint{1.947809in}{2.757423in}}{\pgfqpoint{1.955622in}{2.765236in}}%
\pgfpathcurveto{\pgfqpoint{1.963436in}{2.773050in}}{\pgfqpoint{1.967826in}{2.783649in}}{\pgfqpoint{1.967826in}{2.794699in}}%
\pgfpathcurveto{\pgfqpoint{1.967826in}{2.805749in}}{\pgfqpoint{1.963436in}{2.816348in}}{\pgfqpoint{1.955622in}{2.824162in}}%
\pgfpathcurveto{\pgfqpoint{1.947809in}{2.831976in}}{\pgfqpoint{1.937209in}{2.836366in}}{\pgfqpoint{1.926159in}{2.836366in}}%
\pgfpathcurveto{\pgfqpoint{1.915109in}{2.836366in}}{\pgfqpoint{1.904510in}{2.831976in}}{\pgfqpoint{1.896697in}{2.824162in}}%
\pgfpathcurveto{\pgfqpoint{1.888883in}{2.816348in}}{\pgfqpoint{1.884493in}{2.805749in}}{\pgfqpoint{1.884493in}{2.794699in}}%
\pgfpathcurveto{\pgfqpoint{1.884493in}{2.783649in}}{\pgfqpoint{1.888883in}{2.773050in}}{\pgfqpoint{1.896697in}{2.765236in}}%
\pgfpathcurveto{\pgfqpoint{1.904510in}{2.757423in}}{\pgfqpoint{1.915109in}{2.753032in}}{\pgfqpoint{1.926159in}{2.753032in}}%
\pgfpathclose%
\pgfusepath{stroke,fill}%
\end{pgfscope}%
\begin{pgfscope}%
\pgfpathrectangle{\pgfqpoint{0.600000in}{0.600000in}}{\pgfqpoint{3.900000in}{3.900000in}}%
\pgfusepath{clip}%
\pgfsetbuttcap%
\pgfsetroundjoin%
\definecolor{currentfill}{rgb}{0.121569,0.466667,0.705882}%
\pgfsetfillcolor{currentfill}%
\pgfsetlinewidth{1.003750pt}%
\definecolor{currentstroke}{rgb}{0.121569,0.466667,0.705882}%
\pgfsetstrokecolor{currentstroke}%
\pgfsetdash{}{0pt}%
\pgfpathmoveto{\pgfqpoint{3.670989in}{3.026233in}}%
\pgfpathcurveto{\pgfqpoint{3.682039in}{3.026233in}}{\pgfqpoint{3.692638in}{3.030623in}}{\pgfqpoint{3.700452in}{3.038437in}}%
\pgfpathcurveto{\pgfqpoint{3.708265in}{3.046250in}}{\pgfqpoint{3.712656in}{3.056849in}}{\pgfqpoint{3.712656in}{3.067899in}}%
\pgfpathcurveto{\pgfqpoint{3.712656in}{3.078950in}}{\pgfqpoint{3.708265in}{3.089549in}}{\pgfqpoint{3.700452in}{3.097362in}}%
\pgfpathcurveto{\pgfqpoint{3.692638in}{3.105176in}}{\pgfqpoint{3.682039in}{3.109566in}}{\pgfqpoint{3.670989in}{3.109566in}}%
\pgfpathcurveto{\pgfqpoint{3.659939in}{3.109566in}}{\pgfqpoint{3.649340in}{3.105176in}}{\pgfqpoint{3.641526in}{3.097362in}}%
\pgfpathcurveto{\pgfqpoint{3.633713in}{3.089549in}}{\pgfqpoint{3.629322in}{3.078950in}}{\pgfqpoint{3.629322in}{3.067899in}}%
\pgfpathcurveto{\pgfqpoint{3.629322in}{3.056849in}}{\pgfqpoint{3.633713in}{3.046250in}}{\pgfqpoint{3.641526in}{3.038437in}}%
\pgfpathcurveto{\pgfqpoint{3.649340in}{3.030623in}}{\pgfqpoint{3.659939in}{3.026233in}}{\pgfqpoint{3.670989in}{3.026233in}}%
\pgfpathclose%
\pgfusepath{stroke,fill}%
\end{pgfscope}%
\begin{pgfscope}%
\pgfpathrectangle{\pgfqpoint{0.600000in}{0.600000in}}{\pgfqpoint{3.900000in}{3.900000in}}%
\pgfusepath{clip}%
\pgfsetbuttcap%
\pgfsetroundjoin%
\definecolor{currentfill}{rgb}{0.121569,0.466667,0.705882}%
\pgfsetfillcolor{currentfill}%
\pgfsetlinewidth{1.003750pt}%
\definecolor{currentstroke}{rgb}{0.121569,0.466667,0.705882}%
\pgfsetstrokecolor{currentstroke}%
\pgfsetdash{}{0pt}%
\pgfpathmoveto{\pgfqpoint{2.645077in}{1.980924in}}%
\pgfpathcurveto{\pgfqpoint{2.656127in}{1.980924in}}{\pgfqpoint{2.666726in}{1.985314in}}{\pgfqpoint{2.674539in}{1.993128in}}%
\pgfpathcurveto{\pgfqpoint{2.682353in}{2.000941in}}{\pgfqpoint{2.686743in}{2.011541in}}{\pgfqpoint{2.686743in}{2.022591in}}%
\pgfpathcurveto{\pgfqpoint{2.686743in}{2.033641in}}{\pgfqpoint{2.682353in}{2.044240in}}{\pgfqpoint{2.674539in}{2.052053in}}%
\pgfpathcurveto{\pgfqpoint{2.666726in}{2.059867in}}{\pgfqpoint{2.656127in}{2.064257in}}{\pgfqpoint{2.645077in}{2.064257in}}%
\pgfpathcurveto{\pgfqpoint{2.634026in}{2.064257in}}{\pgfqpoint{2.623427in}{2.059867in}}{\pgfqpoint{2.615614in}{2.052053in}}%
\pgfpathcurveto{\pgfqpoint{2.607800in}{2.044240in}}{\pgfqpoint{2.603410in}{2.033641in}}{\pgfqpoint{2.603410in}{2.022591in}}%
\pgfpathcurveto{\pgfqpoint{2.603410in}{2.011541in}}{\pgfqpoint{2.607800in}{2.000941in}}{\pgfqpoint{2.615614in}{1.993128in}}%
\pgfpathcurveto{\pgfqpoint{2.623427in}{1.985314in}}{\pgfqpoint{2.634026in}{1.980924in}}{\pgfqpoint{2.645077in}{1.980924in}}%
\pgfpathclose%
\pgfusepath{stroke,fill}%
\end{pgfscope}%
\begin{pgfscope}%
\pgfpathrectangle{\pgfqpoint{0.600000in}{0.600000in}}{\pgfqpoint{3.900000in}{3.900000in}}%
\pgfusepath{clip}%
\pgfsetbuttcap%
\pgfsetroundjoin%
\definecolor{currentfill}{rgb}{0.121569,0.466667,0.705882}%
\pgfsetfillcolor{currentfill}%
\pgfsetlinewidth{1.003750pt}%
\definecolor{currentstroke}{rgb}{0.121569,0.466667,0.705882}%
\pgfsetstrokecolor{currentstroke}%
\pgfsetdash{}{0pt}%
\pgfpathmoveto{\pgfqpoint{1.954827in}{2.621244in}}%
\pgfpathcurveto{\pgfqpoint{1.965877in}{2.621244in}}{\pgfqpoint{1.976477in}{2.625634in}}{\pgfqpoint{1.984290in}{2.633448in}}%
\pgfpathcurveto{\pgfqpoint{1.992104in}{2.641261in}}{\pgfqpoint{1.996494in}{2.651860in}}{\pgfqpoint{1.996494in}{2.662910in}}%
\pgfpathcurveto{\pgfqpoint{1.996494in}{2.673961in}}{\pgfqpoint{1.992104in}{2.684560in}}{\pgfqpoint{1.984290in}{2.692373in}}%
\pgfpathcurveto{\pgfqpoint{1.976477in}{2.700187in}}{\pgfqpoint{1.965877in}{2.704577in}}{\pgfqpoint{1.954827in}{2.704577in}}%
\pgfpathcurveto{\pgfqpoint{1.943777in}{2.704577in}}{\pgfqpoint{1.933178in}{2.700187in}}{\pgfqpoint{1.925365in}{2.692373in}}%
\pgfpathcurveto{\pgfqpoint{1.917551in}{2.684560in}}{\pgfqpoint{1.913161in}{2.673961in}}{\pgfqpoint{1.913161in}{2.662910in}}%
\pgfpathcurveto{\pgfqpoint{1.913161in}{2.651860in}}{\pgfqpoint{1.917551in}{2.641261in}}{\pgfqpoint{1.925365in}{2.633448in}}%
\pgfpathcurveto{\pgfqpoint{1.933178in}{2.625634in}}{\pgfqpoint{1.943777in}{2.621244in}}{\pgfqpoint{1.954827in}{2.621244in}}%
\pgfpathclose%
\pgfusepath{stroke,fill}%
\end{pgfscope}%
\begin{pgfscope}%
\pgfpathrectangle{\pgfqpoint{0.600000in}{0.600000in}}{\pgfqpoint{3.900000in}{3.900000in}}%
\pgfusepath{clip}%
\pgfsetbuttcap%
\pgfsetroundjoin%
\definecolor{currentfill}{rgb}{0.121569,0.466667,0.705882}%
\pgfsetfillcolor{currentfill}%
\pgfsetlinewidth{1.003750pt}%
\definecolor{currentstroke}{rgb}{0.121569,0.466667,0.705882}%
\pgfsetstrokecolor{currentstroke}%
\pgfsetdash{}{0pt}%
\pgfpathmoveto{\pgfqpoint{1.909329in}{1.931156in}}%
\pgfpathcurveto{\pgfqpoint{1.920379in}{1.931156in}}{\pgfqpoint{1.930978in}{1.935546in}}{\pgfqpoint{1.938792in}{1.943360in}}%
\pgfpathcurveto{\pgfqpoint{1.946606in}{1.951174in}}{\pgfqpoint{1.950996in}{1.961773in}}{\pgfqpoint{1.950996in}{1.972823in}}%
\pgfpathcurveto{\pgfqpoint{1.950996in}{1.983873in}}{\pgfqpoint{1.946606in}{1.994472in}}{\pgfqpoint{1.938792in}{2.002286in}}%
\pgfpathcurveto{\pgfqpoint{1.930978in}{2.010099in}}{\pgfqpoint{1.920379in}{2.014489in}}{\pgfqpoint{1.909329in}{2.014489in}}%
\pgfpathcurveto{\pgfqpoint{1.898279in}{2.014489in}}{\pgfqpoint{1.887680in}{2.010099in}}{\pgfqpoint{1.879867in}{2.002286in}}%
\pgfpathcurveto{\pgfqpoint{1.872053in}{1.994472in}}{\pgfqpoint{1.867663in}{1.983873in}}{\pgfqpoint{1.867663in}{1.972823in}}%
\pgfpathcurveto{\pgfqpoint{1.867663in}{1.961773in}}{\pgfqpoint{1.872053in}{1.951174in}}{\pgfqpoint{1.879867in}{1.943360in}}%
\pgfpathcurveto{\pgfqpoint{1.887680in}{1.935546in}}{\pgfqpoint{1.898279in}{1.931156in}}{\pgfqpoint{1.909329in}{1.931156in}}%
\pgfpathclose%
\pgfusepath{stroke,fill}%
\end{pgfscope}%
\begin{pgfscope}%
\pgfpathrectangle{\pgfqpoint{0.600000in}{0.600000in}}{\pgfqpoint{3.900000in}{3.900000in}}%
\pgfusepath{clip}%
\pgfsetbuttcap%
\pgfsetroundjoin%
\definecolor{currentfill}{rgb}{0.121569,0.466667,0.705882}%
\pgfsetfillcolor{currentfill}%
\pgfsetlinewidth{1.003750pt}%
\definecolor{currentstroke}{rgb}{0.121569,0.466667,0.705882}%
\pgfsetstrokecolor{currentstroke}%
\pgfsetdash{}{0pt}%
\pgfpathmoveto{\pgfqpoint{2.009718in}{2.831809in}}%
\pgfpathcurveto{\pgfqpoint{2.020768in}{2.831809in}}{\pgfqpoint{2.031367in}{2.836199in}}{\pgfqpoint{2.039181in}{2.844012in}}%
\pgfpathcurveto{\pgfqpoint{2.046994in}{2.851826in}}{\pgfqpoint{2.051385in}{2.862425in}}{\pgfqpoint{2.051385in}{2.873475in}}%
\pgfpathcurveto{\pgfqpoint{2.051385in}{2.884525in}}{\pgfqpoint{2.046994in}{2.895124in}}{\pgfqpoint{2.039181in}{2.902938in}}%
\pgfpathcurveto{\pgfqpoint{2.031367in}{2.910752in}}{\pgfqpoint{2.020768in}{2.915142in}}{\pgfqpoint{2.009718in}{2.915142in}}%
\pgfpathcurveto{\pgfqpoint{1.998668in}{2.915142in}}{\pgfqpoint{1.988069in}{2.910752in}}{\pgfqpoint{1.980255in}{2.902938in}}%
\pgfpathcurveto{\pgfqpoint{1.972441in}{2.895124in}}{\pgfqpoint{1.968051in}{2.884525in}}{\pgfqpoint{1.968051in}{2.873475in}}%
\pgfpathcurveto{\pgfqpoint{1.968051in}{2.862425in}}{\pgfqpoint{1.972441in}{2.851826in}}{\pgfqpoint{1.980255in}{2.844012in}}%
\pgfpathcurveto{\pgfqpoint{1.988069in}{2.836199in}}{\pgfqpoint{1.998668in}{2.831809in}}{\pgfqpoint{2.009718in}{2.831809in}}%
\pgfpathclose%
\pgfusepath{stroke,fill}%
\end{pgfscope}%
\begin{pgfscope}%
\pgfpathrectangle{\pgfqpoint{0.600000in}{0.600000in}}{\pgfqpoint{3.900000in}{3.900000in}}%
\pgfusepath{clip}%
\pgfsetbuttcap%
\pgfsetroundjoin%
\definecolor{currentfill}{rgb}{0.121569,0.466667,0.705882}%
\pgfsetfillcolor{currentfill}%
\pgfsetlinewidth{1.003750pt}%
\definecolor{currentstroke}{rgb}{0.121569,0.466667,0.705882}%
\pgfsetstrokecolor{currentstroke}%
\pgfsetdash{}{0pt}%
\pgfpathmoveto{\pgfqpoint{3.386995in}{2.143866in}}%
\pgfpathcurveto{\pgfqpoint{3.398045in}{2.143866in}}{\pgfqpoint{3.408644in}{2.148256in}}{\pgfqpoint{3.416458in}{2.156070in}}%
\pgfpathcurveto{\pgfqpoint{3.424271in}{2.163884in}}{\pgfqpoint{3.428662in}{2.174483in}}{\pgfqpoint{3.428662in}{2.185533in}}%
\pgfpathcurveto{\pgfqpoint{3.428662in}{2.196583in}}{\pgfqpoint{3.424271in}{2.207182in}}{\pgfqpoint{3.416458in}{2.214996in}}%
\pgfpathcurveto{\pgfqpoint{3.408644in}{2.222809in}}{\pgfqpoint{3.398045in}{2.227199in}}{\pgfqpoint{3.386995in}{2.227199in}}%
\pgfpathcurveto{\pgfqpoint{3.375945in}{2.227199in}}{\pgfqpoint{3.365346in}{2.222809in}}{\pgfqpoint{3.357532in}{2.214996in}}%
\pgfpathcurveto{\pgfqpoint{3.349719in}{2.207182in}}{\pgfqpoint{3.345328in}{2.196583in}}{\pgfqpoint{3.345328in}{2.185533in}}%
\pgfpathcurveto{\pgfqpoint{3.345328in}{2.174483in}}{\pgfqpoint{3.349719in}{2.163884in}}{\pgfqpoint{3.357532in}{2.156070in}}%
\pgfpathcurveto{\pgfqpoint{3.365346in}{2.148256in}}{\pgfqpoint{3.375945in}{2.143866in}}{\pgfqpoint{3.386995in}{2.143866in}}%
\pgfpathclose%
\pgfusepath{stroke,fill}%
\end{pgfscope}%
\begin{pgfscope}%
\pgfpathrectangle{\pgfqpoint{0.600000in}{0.600000in}}{\pgfqpoint{3.900000in}{3.900000in}}%
\pgfusepath{clip}%
\pgfsetbuttcap%
\pgfsetroundjoin%
\definecolor{currentfill}{rgb}{0.121569,0.466667,0.705882}%
\pgfsetfillcolor{currentfill}%
\pgfsetlinewidth{1.003750pt}%
\definecolor{currentstroke}{rgb}{0.121569,0.466667,0.705882}%
\pgfsetstrokecolor{currentstroke}%
\pgfsetdash{}{0pt}%
\pgfpathmoveto{\pgfqpoint{2.528258in}{2.581697in}}%
\pgfpathcurveto{\pgfqpoint{2.539309in}{2.581697in}}{\pgfqpoint{2.549908in}{2.586087in}}{\pgfqpoint{2.557721in}{2.593901in}}%
\pgfpathcurveto{\pgfqpoint{2.565535in}{2.601714in}}{\pgfqpoint{2.569925in}{2.612313in}}{\pgfqpoint{2.569925in}{2.623363in}}%
\pgfpathcurveto{\pgfqpoint{2.569925in}{2.634413in}}{\pgfqpoint{2.565535in}{2.645013in}}{\pgfqpoint{2.557721in}{2.652826in}}%
\pgfpathcurveto{\pgfqpoint{2.549908in}{2.660640in}}{\pgfqpoint{2.539309in}{2.665030in}}{\pgfqpoint{2.528258in}{2.665030in}}%
\pgfpathcurveto{\pgfqpoint{2.517208in}{2.665030in}}{\pgfqpoint{2.506609in}{2.660640in}}{\pgfqpoint{2.498796in}{2.652826in}}%
\pgfpathcurveto{\pgfqpoint{2.490982in}{2.645013in}}{\pgfqpoint{2.486592in}{2.634413in}}{\pgfqpoint{2.486592in}{2.623363in}}%
\pgfpathcurveto{\pgfqpoint{2.486592in}{2.612313in}}{\pgfqpoint{2.490982in}{2.601714in}}{\pgfqpoint{2.498796in}{2.593901in}}%
\pgfpathcurveto{\pgfqpoint{2.506609in}{2.586087in}}{\pgfqpoint{2.517208in}{2.581697in}}{\pgfqpoint{2.528258in}{2.581697in}}%
\pgfpathclose%
\pgfusepath{stroke,fill}%
\end{pgfscope}%
\begin{pgfscope}%
\pgfpathrectangle{\pgfqpoint{0.600000in}{0.600000in}}{\pgfqpoint{3.900000in}{3.900000in}}%
\pgfusepath{clip}%
\pgfsetbuttcap%
\pgfsetroundjoin%
\definecolor{currentfill}{rgb}{0.121569,0.466667,0.705882}%
\pgfsetfillcolor{currentfill}%
\pgfsetlinewidth{1.003750pt}%
\definecolor{currentstroke}{rgb}{0.121569,0.466667,0.705882}%
\pgfsetstrokecolor{currentstroke}%
\pgfsetdash{}{0pt}%
\pgfpathmoveto{\pgfqpoint{2.522559in}{2.499127in}}%
\pgfpathcurveto{\pgfqpoint{2.533609in}{2.499127in}}{\pgfqpoint{2.544208in}{2.503517in}}{\pgfqpoint{2.552022in}{2.511331in}}%
\pgfpathcurveto{\pgfqpoint{2.559835in}{2.519144in}}{\pgfqpoint{2.564225in}{2.529743in}}{\pgfqpoint{2.564225in}{2.540793in}}%
\pgfpathcurveto{\pgfqpoint{2.564225in}{2.551844in}}{\pgfqpoint{2.559835in}{2.562443in}}{\pgfqpoint{2.552022in}{2.570256in}}%
\pgfpathcurveto{\pgfqpoint{2.544208in}{2.578070in}}{\pgfqpoint{2.533609in}{2.582460in}}{\pgfqpoint{2.522559in}{2.582460in}}%
\pgfpathcurveto{\pgfqpoint{2.511509in}{2.582460in}}{\pgfqpoint{2.500910in}{2.578070in}}{\pgfqpoint{2.493096in}{2.570256in}}%
\pgfpathcurveto{\pgfqpoint{2.485282in}{2.562443in}}{\pgfqpoint{2.480892in}{2.551844in}}{\pgfqpoint{2.480892in}{2.540793in}}%
\pgfpathcurveto{\pgfqpoint{2.480892in}{2.529743in}}{\pgfqpoint{2.485282in}{2.519144in}}{\pgfqpoint{2.493096in}{2.511331in}}%
\pgfpathcurveto{\pgfqpoint{2.500910in}{2.503517in}}{\pgfqpoint{2.511509in}{2.499127in}}{\pgfqpoint{2.522559in}{2.499127in}}%
\pgfpathclose%
\pgfusepath{stroke,fill}%
\end{pgfscope}%
\begin{pgfscope}%
\pgfpathrectangle{\pgfqpoint{0.600000in}{0.600000in}}{\pgfqpoint{3.900000in}{3.900000in}}%
\pgfusepath{clip}%
\pgfsetbuttcap%
\pgfsetroundjoin%
\definecolor{currentfill}{rgb}{0.121569,0.466667,0.705882}%
\pgfsetfillcolor{currentfill}%
\pgfsetlinewidth{1.003750pt}%
\definecolor{currentstroke}{rgb}{0.121569,0.466667,0.705882}%
\pgfsetstrokecolor{currentstroke}%
\pgfsetdash{}{0pt}%
\pgfpathmoveto{\pgfqpoint{3.149376in}{2.401367in}}%
\pgfpathcurveto{\pgfqpoint{3.160426in}{2.401367in}}{\pgfqpoint{3.171025in}{2.405757in}}{\pgfqpoint{3.178839in}{2.413570in}}%
\pgfpathcurveto{\pgfqpoint{3.186652in}{2.421384in}}{\pgfqpoint{3.191043in}{2.431983in}}{\pgfqpoint{3.191043in}{2.443033in}}%
\pgfpathcurveto{\pgfqpoint{3.191043in}{2.454083in}}{\pgfqpoint{3.186652in}{2.464682in}}{\pgfqpoint{3.178839in}{2.472496in}}%
\pgfpathcurveto{\pgfqpoint{3.171025in}{2.480310in}}{\pgfqpoint{3.160426in}{2.484700in}}{\pgfqpoint{3.149376in}{2.484700in}}%
\pgfpathcurveto{\pgfqpoint{3.138326in}{2.484700in}}{\pgfqpoint{3.127727in}{2.480310in}}{\pgfqpoint{3.119913in}{2.472496in}}%
\pgfpathcurveto{\pgfqpoint{3.112100in}{2.464682in}}{\pgfqpoint{3.107709in}{2.454083in}}{\pgfqpoint{3.107709in}{2.443033in}}%
\pgfpathcurveto{\pgfqpoint{3.107709in}{2.431983in}}{\pgfqpoint{3.112100in}{2.421384in}}{\pgfqpoint{3.119913in}{2.413570in}}%
\pgfpathcurveto{\pgfqpoint{3.127727in}{2.405757in}}{\pgfqpoint{3.138326in}{2.401367in}}{\pgfqpoint{3.149376in}{2.401367in}}%
\pgfpathclose%
\pgfusepath{stroke,fill}%
\end{pgfscope}%
\begin{pgfscope}%
\pgfpathrectangle{\pgfqpoint{0.600000in}{0.600000in}}{\pgfqpoint{3.900000in}{3.900000in}}%
\pgfusepath{clip}%
\pgfsetbuttcap%
\pgfsetroundjoin%
\definecolor{currentfill}{rgb}{0.121569,0.466667,0.705882}%
\pgfsetfillcolor{currentfill}%
\pgfsetlinewidth{1.003750pt}%
\definecolor{currentstroke}{rgb}{0.121569,0.466667,0.705882}%
\pgfsetstrokecolor{currentstroke}%
\pgfsetdash{}{0pt}%
\pgfpathmoveto{\pgfqpoint{3.439749in}{2.816991in}}%
\pgfpathcurveto{\pgfqpoint{3.450799in}{2.816991in}}{\pgfqpoint{3.461398in}{2.821381in}}{\pgfqpoint{3.469212in}{2.829195in}}%
\pgfpathcurveto{\pgfqpoint{3.477025in}{2.837008in}}{\pgfqpoint{3.481416in}{2.847607in}}{\pgfqpoint{3.481416in}{2.858657in}}%
\pgfpathcurveto{\pgfqpoint{3.481416in}{2.869708in}}{\pgfqpoint{3.477025in}{2.880307in}}{\pgfqpoint{3.469212in}{2.888120in}}%
\pgfpathcurveto{\pgfqpoint{3.461398in}{2.895934in}}{\pgfqpoint{3.450799in}{2.900324in}}{\pgfqpoint{3.439749in}{2.900324in}}%
\pgfpathcurveto{\pgfqpoint{3.428699in}{2.900324in}}{\pgfqpoint{3.418100in}{2.895934in}}{\pgfqpoint{3.410286in}{2.888120in}}%
\pgfpathcurveto{\pgfqpoint{3.402473in}{2.880307in}}{\pgfqpoint{3.398082in}{2.869708in}}{\pgfqpoint{3.398082in}{2.858657in}}%
\pgfpathcurveto{\pgfqpoint{3.398082in}{2.847607in}}{\pgfqpoint{3.402473in}{2.837008in}}{\pgfqpoint{3.410286in}{2.829195in}}%
\pgfpathcurveto{\pgfqpoint{3.418100in}{2.821381in}}{\pgfqpoint{3.428699in}{2.816991in}}{\pgfqpoint{3.439749in}{2.816991in}}%
\pgfpathclose%
\pgfusepath{stroke,fill}%
\end{pgfscope}%
\begin{pgfscope}%
\pgfpathrectangle{\pgfqpoint{0.600000in}{0.600000in}}{\pgfqpoint{3.900000in}{3.900000in}}%
\pgfusepath{clip}%
\pgfsetbuttcap%
\pgfsetroundjoin%
\definecolor{currentfill}{rgb}{0.121569,0.466667,0.705882}%
\pgfsetfillcolor{currentfill}%
\pgfsetlinewidth{1.003750pt}%
\definecolor{currentstroke}{rgb}{0.121569,0.466667,0.705882}%
\pgfsetstrokecolor{currentstroke}%
\pgfsetdash{}{0pt}%
\pgfpathmoveto{\pgfqpoint{1.987789in}{2.907088in}}%
\pgfpathcurveto{\pgfqpoint{1.998839in}{2.907088in}}{\pgfqpoint{2.009438in}{2.911478in}}{\pgfqpoint{2.017252in}{2.919292in}}%
\pgfpathcurveto{\pgfqpoint{2.025065in}{2.927106in}}{\pgfqpoint{2.029456in}{2.937705in}}{\pgfqpoint{2.029456in}{2.948755in}}%
\pgfpathcurveto{\pgfqpoint{2.029456in}{2.959805in}}{\pgfqpoint{2.025065in}{2.970404in}}{\pgfqpoint{2.017252in}{2.978217in}}%
\pgfpathcurveto{\pgfqpoint{2.009438in}{2.986031in}}{\pgfqpoint{1.998839in}{2.990421in}}{\pgfqpoint{1.987789in}{2.990421in}}%
\pgfpathcurveto{\pgfqpoint{1.976739in}{2.990421in}}{\pgfqpoint{1.966140in}{2.986031in}}{\pgfqpoint{1.958326in}{2.978217in}}%
\pgfpathcurveto{\pgfqpoint{1.950512in}{2.970404in}}{\pgfqpoint{1.946122in}{2.959805in}}{\pgfqpoint{1.946122in}{2.948755in}}%
\pgfpathcurveto{\pgfqpoint{1.946122in}{2.937705in}}{\pgfqpoint{1.950512in}{2.927106in}}{\pgfqpoint{1.958326in}{2.919292in}}%
\pgfpathcurveto{\pgfqpoint{1.966140in}{2.911478in}}{\pgfqpoint{1.976739in}{2.907088in}}{\pgfqpoint{1.987789in}{2.907088in}}%
\pgfpathclose%
\pgfusepath{stroke,fill}%
\end{pgfscope}%
\begin{pgfscope}%
\pgfpathrectangle{\pgfqpoint{0.600000in}{0.600000in}}{\pgfqpoint{3.900000in}{3.900000in}}%
\pgfusepath{clip}%
\pgfsetbuttcap%
\pgfsetroundjoin%
\definecolor{currentfill}{rgb}{0.121569,0.466667,0.705882}%
\pgfsetfillcolor{currentfill}%
\pgfsetlinewidth{1.003750pt}%
\definecolor{currentstroke}{rgb}{0.121569,0.466667,0.705882}%
\pgfsetstrokecolor{currentstroke}%
\pgfsetdash{}{0pt}%
\pgfpathmoveto{\pgfqpoint{2.727235in}{2.513305in}}%
\pgfpathcurveto{\pgfqpoint{2.738286in}{2.513305in}}{\pgfqpoint{2.748885in}{2.517695in}}{\pgfqpoint{2.756698in}{2.525509in}}%
\pgfpathcurveto{\pgfqpoint{2.764512in}{2.533323in}}{\pgfqpoint{2.768902in}{2.543922in}}{\pgfqpoint{2.768902in}{2.554972in}}%
\pgfpathcurveto{\pgfqpoint{2.768902in}{2.566022in}}{\pgfqpoint{2.764512in}{2.576621in}}{\pgfqpoint{2.756698in}{2.584435in}}%
\pgfpathcurveto{\pgfqpoint{2.748885in}{2.592248in}}{\pgfqpoint{2.738286in}{2.596638in}}{\pgfqpoint{2.727235in}{2.596638in}}%
\pgfpathcurveto{\pgfqpoint{2.716185in}{2.596638in}}{\pgfqpoint{2.705586in}{2.592248in}}{\pgfqpoint{2.697773in}{2.584435in}}%
\pgfpathcurveto{\pgfqpoint{2.689959in}{2.576621in}}{\pgfqpoint{2.685569in}{2.566022in}}{\pgfqpoint{2.685569in}{2.554972in}}%
\pgfpathcurveto{\pgfqpoint{2.685569in}{2.543922in}}{\pgfqpoint{2.689959in}{2.533323in}}{\pgfqpoint{2.697773in}{2.525509in}}%
\pgfpathcurveto{\pgfqpoint{2.705586in}{2.517695in}}{\pgfqpoint{2.716185in}{2.513305in}}{\pgfqpoint{2.727235in}{2.513305in}}%
\pgfpathclose%
\pgfusepath{stroke,fill}%
\end{pgfscope}%
\begin{pgfscope}%
\pgfpathrectangle{\pgfqpoint{0.600000in}{0.600000in}}{\pgfqpoint{3.900000in}{3.900000in}}%
\pgfusepath{clip}%
\pgfsetbuttcap%
\pgfsetroundjoin%
\definecolor{currentfill}{rgb}{0.121569,0.466667,0.705882}%
\pgfsetfillcolor{currentfill}%
\pgfsetlinewidth{1.003750pt}%
\definecolor{currentstroke}{rgb}{0.121569,0.466667,0.705882}%
\pgfsetstrokecolor{currentstroke}%
\pgfsetdash{}{0pt}%
\pgfpathmoveto{\pgfqpoint{3.027003in}{2.468852in}}%
\pgfpathcurveto{\pgfqpoint{3.038053in}{2.468852in}}{\pgfqpoint{3.048652in}{2.473242in}}{\pgfqpoint{3.056466in}{2.481056in}}%
\pgfpathcurveto{\pgfqpoint{3.064279in}{2.488870in}}{\pgfqpoint{3.068670in}{2.499469in}}{\pgfqpoint{3.068670in}{2.510519in}}%
\pgfpathcurveto{\pgfqpoint{3.068670in}{2.521569in}}{\pgfqpoint{3.064279in}{2.532168in}}{\pgfqpoint{3.056466in}{2.539982in}}%
\pgfpathcurveto{\pgfqpoint{3.048652in}{2.547795in}}{\pgfqpoint{3.038053in}{2.552186in}}{\pgfqpoint{3.027003in}{2.552186in}}%
\pgfpathcurveto{\pgfqpoint{3.015953in}{2.552186in}}{\pgfqpoint{3.005354in}{2.547795in}}{\pgfqpoint{2.997540in}{2.539982in}}%
\pgfpathcurveto{\pgfqpoint{2.989727in}{2.532168in}}{\pgfqpoint{2.985336in}{2.521569in}}{\pgfqpoint{2.985336in}{2.510519in}}%
\pgfpathcurveto{\pgfqpoint{2.985336in}{2.499469in}}{\pgfqpoint{2.989727in}{2.488870in}}{\pgfqpoint{2.997540in}{2.481056in}}%
\pgfpathcurveto{\pgfqpoint{3.005354in}{2.473242in}}{\pgfqpoint{3.015953in}{2.468852in}}{\pgfqpoint{3.027003in}{2.468852in}}%
\pgfpathclose%
\pgfusepath{stroke,fill}%
\end{pgfscope}%
\begin{pgfscope}%
\pgfpathrectangle{\pgfqpoint{0.600000in}{0.600000in}}{\pgfqpoint{3.900000in}{3.900000in}}%
\pgfusepath{clip}%
\pgfsetbuttcap%
\pgfsetroundjoin%
\definecolor{currentfill}{rgb}{0.121569,0.466667,0.705882}%
\pgfsetfillcolor{currentfill}%
\pgfsetlinewidth{1.003750pt}%
\definecolor{currentstroke}{rgb}{0.121569,0.466667,0.705882}%
\pgfsetstrokecolor{currentstroke}%
\pgfsetdash{}{0pt}%
\pgfpathmoveto{\pgfqpoint{2.197640in}{1.237675in}}%
\pgfpathcurveto{\pgfqpoint{2.208690in}{1.237675in}}{\pgfqpoint{2.219289in}{1.242066in}}{\pgfqpoint{2.227103in}{1.249879in}}%
\pgfpathcurveto{\pgfqpoint{2.234917in}{1.257693in}}{\pgfqpoint{2.239307in}{1.268292in}}{\pgfqpoint{2.239307in}{1.279342in}}%
\pgfpathcurveto{\pgfqpoint{2.239307in}{1.290392in}}{\pgfqpoint{2.234917in}{1.300991in}}{\pgfqpoint{2.227103in}{1.308805in}}%
\pgfpathcurveto{\pgfqpoint{2.219289in}{1.316618in}}{\pgfqpoint{2.208690in}{1.321009in}}{\pgfqpoint{2.197640in}{1.321009in}}%
\pgfpathcurveto{\pgfqpoint{2.186590in}{1.321009in}}{\pgfqpoint{2.175991in}{1.316618in}}{\pgfqpoint{2.168177in}{1.308805in}}%
\pgfpathcurveto{\pgfqpoint{2.160364in}{1.300991in}}{\pgfqpoint{2.155973in}{1.290392in}}{\pgfqpoint{2.155973in}{1.279342in}}%
\pgfpathcurveto{\pgfqpoint{2.155973in}{1.268292in}}{\pgfqpoint{2.160364in}{1.257693in}}{\pgfqpoint{2.168177in}{1.249879in}}%
\pgfpathcurveto{\pgfqpoint{2.175991in}{1.242066in}}{\pgfqpoint{2.186590in}{1.237675in}}{\pgfqpoint{2.197640in}{1.237675in}}%
\pgfpathclose%
\pgfusepath{stroke,fill}%
\end{pgfscope}%
\begin{pgfscope}%
\pgfpathrectangle{\pgfqpoint{0.600000in}{0.600000in}}{\pgfqpoint{3.900000in}{3.900000in}}%
\pgfusepath{clip}%
\pgfsetbuttcap%
\pgfsetroundjoin%
\definecolor{currentfill}{rgb}{0.121569,0.466667,0.705882}%
\pgfsetfillcolor{currentfill}%
\pgfsetlinewidth{1.003750pt}%
\definecolor{currentstroke}{rgb}{0.121569,0.466667,0.705882}%
\pgfsetstrokecolor{currentstroke}%
\pgfsetdash{}{0pt}%
\pgfpathmoveto{\pgfqpoint{2.707051in}{2.274434in}}%
\pgfpathcurveto{\pgfqpoint{2.718101in}{2.274434in}}{\pgfqpoint{2.728700in}{2.278825in}}{\pgfqpoint{2.736514in}{2.286638in}}%
\pgfpathcurveto{\pgfqpoint{2.744328in}{2.294452in}}{\pgfqpoint{2.748718in}{2.305051in}}{\pgfqpoint{2.748718in}{2.316101in}}%
\pgfpathcurveto{\pgfqpoint{2.748718in}{2.327151in}}{\pgfqpoint{2.744328in}{2.337750in}}{\pgfqpoint{2.736514in}{2.345564in}}%
\pgfpathcurveto{\pgfqpoint{2.728700in}{2.353377in}}{\pgfqpoint{2.718101in}{2.357768in}}{\pgfqpoint{2.707051in}{2.357768in}}%
\pgfpathcurveto{\pgfqpoint{2.696001in}{2.357768in}}{\pgfqpoint{2.685402in}{2.353377in}}{\pgfqpoint{2.677588in}{2.345564in}}%
\pgfpathcurveto{\pgfqpoint{2.669775in}{2.337750in}}{\pgfqpoint{2.665385in}{2.327151in}}{\pgfqpoint{2.665385in}{2.316101in}}%
\pgfpathcurveto{\pgfqpoint{2.665385in}{2.305051in}}{\pgfqpoint{2.669775in}{2.294452in}}{\pgfqpoint{2.677588in}{2.286638in}}%
\pgfpathcurveto{\pgfqpoint{2.685402in}{2.278825in}}{\pgfqpoint{2.696001in}{2.274434in}}{\pgfqpoint{2.707051in}{2.274434in}}%
\pgfpathclose%
\pgfusepath{stroke,fill}%
\end{pgfscope}%
\begin{pgfscope}%
\pgfpathrectangle{\pgfqpoint{0.600000in}{0.600000in}}{\pgfqpoint{3.900000in}{3.900000in}}%
\pgfusepath{clip}%
\pgfsetbuttcap%
\pgfsetroundjoin%
\definecolor{currentfill}{rgb}{0.121569,0.466667,0.705882}%
\pgfsetfillcolor{currentfill}%
\pgfsetlinewidth{1.003750pt}%
\definecolor{currentstroke}{rgb}{0.121569,0.466667,0.705882}%
\pgfsetstrokecolor{currentstroke}%
\pgfsetdash{}{0pt}%
\pgfpathmoveto{\pgfqpoint{2.142437in}{2.737511in}}%
\pgfpathcurveto{\pgfqpoint{2.153487in}{2.737511in}}{\pgfqpoint{2.164086in}{2.741901in}}{\pgfqpoint{2.171900in}{2.749715in}}%
\pgfpathcurveto{\pgfqpoint{2.179713in}{2.757528in}}{\pgfqpoint{2.184104in}{2.768127in}}{\pgfqpoint{2.184104in}{2.779177in}}%
\pgfpathcurveto{\pgfqpoint{2.184104in}{2.790228in}}{\pgfqpoint{2.179713in}{2.800827in}}{\pgfqpoint{2.171900in}{2.808640in}}%
\pgfpathcurveto{\pgfqpoint{2.164086in}{2.816454in}}{\pgfqpoint{2.153487in}{2.820844in}}{\pgfqpoint{2.142437in}{2.820844in}}%
\pgfpathcurveto{\pgfqpoint{2.131387in}{2.820844in}}{\pgfqpoint{2.120788in}{2.816454in}}{\pgfqpoint{2.112974in}{2.808640in}}%
\pgfpathcurveto{\pgfqpoint{2.105161in}{2.800827in}}{\pgfqpoint{2.100770in}{2.790228in}}{\pgfqpoint{2.100770in}{2.779177in}}%
\pgfpathcurveto{\pgfqpoint{2.100770in}{2.768127in}}{\pgfqpoint{2.105161in}{2.757528in}}{\pgfqpoint{2.112974in}{2.749715in}}%
\pgfpathcurveto{\pgfqpoint{2.120788in}{2.741901in}}{\pgfqpoint{2.131387in}{2.737511in}}{\pgfqpoint{2.142437in}{2.737511in}}%
\pgfpathclose%
\pgfusepath{stroke,fill}%
\end{pgfscope}%
\begin{pgfscope}%
\pgfpathrectangle{\pgfqpoint{0.600000in}{0.600000in}}{\pgfqpoint{3.900000in}{3.900000in}}%
\pgfusepath{clip}%
\pgfsetbuttcap%
\pgfsetroundjoin%
\definecolor{currentfill}{rgb}{0.121569,0.466667,0.705882}%
\pgfsetfillcolor{currentfill}%
\pgfsetlinewidth{1.003750pt}%
\definecolor{currentstroke}{rgb}{0.121569,0.466667,0.705882}%
\pgfsetstrokecolor{currentstroke}%
\pgfsetdash{}{0pt}%
\pgfpathmoveto{\pgfqpoint{1.681839in}{2.617690in}}%
\pgfpathcurveto{\pgfqpoint{1.692889in}{2.617690in}}{\pgfqpoint{1.703488in}{2.622081in}}{\pgfqpoint{1.711302in}{2.629894in}}%
\pgfpathcurveto{\pgfqpoint{1.719115in}{2.637708in}}{\pgfqpoint{1.723506in}{2.648307in}}{\pgfqpoint{1.723506in}{2.659357in}}%
\pgfpathcurveto{\pgfqpoint{1.723506in}{2.670407in}}{\pgfqpoint{1.719115in}{2.681006in}}{\pgfqpoint{1.711302in}{2.688820in}}%
\pgfpathcurveto{\pgfqpoint{1.703488in}{2.696634in}}{\pgfqpoint{1.692889in}{2.701024in}}{\pgfqpoint{1.681839in}{2.701024in}}%
\pgfpathcurveto{\pgfqpoint{1.670789in}{2.701024in}}{\pgfqpoint{1.660190in}{2.696634in}}{\pgfqpoint{1.652376in}{2.688820in}}%
\pgfpathcurveto{\pgfqpoint{1.644562in}{2.681006in}}{\pgfqpoint{1.640172in}{2.670407in}}{\pgfqpoint{1.640172in}{2.659357in}}%
\pgfpathcurveto{\pgfqpoint{1.640172in}{2.648307in}}{\pgfqpoint{1.644562in}{2.637708in}}{\pgfqpoint{1.652376in}{2.629894in}}%
\pgfpathcurveto{\pgfqpoint{1.660190in}{2.622081in}}{\pgfqpoint{1.670789in}{2.617690in}}{\pgfqpoint{1.681839in}{2.617690in}}%
\pgfpathclose%
\pgfusepath{stroke,fill}%
\end{pgfscope}%
\begin{pgfscope}%
\pgfpathrectangle{\pgfqpoint{0.600000in}{0.600000in}}{\pgfqpoint{3.900000in}{3.900000in}}%
\pgfusepath{clip}%
\pgfsetbuttcap%
\pgfsetroundjoin%
\definecolor{currentfill}{rgb}{0.121569,0.466667,0.705882}%
\pgfsetfillcolor{currentfill}%
\pgfsetlinewidth{1.003750pt}%
\definecolor{currentstroke}{rgb}{0.121569,0.466667,0.705882}%
\pgfsetstrokecolor{currentstroke}%
\pgfsetdash{}{0pt}%
\pgfpathmoveto{\pgfqpoint{2.462221in}{2.107991in}}%
\pgfpathcurveto{\pgfqpoint{2.473271in}{2.107991in}}{\pgfqpoint{2.483870in}{2.112381in}}{\pgfqpoint{2.491684in}{2.120195in}}%
\pgfpathcurveto{\pgfqpoint{2.499497in}{2.128009in}}{\pgfqpoint{2.503887in}{2.138608in}}{\pgfqpoint{2.503887in}{2.149658in}}%
\pgfpathcurveto{\pgfqpoint{2.503887in}{2.160708in}}{\pgfqpoint{2.499497in}{2.171307in}}{\pgfqpoint{2.491684in}{2.179121in}}%
\pgfpathcurveto{\pgfqpoint{2.483870in}{2.186934in}}{\pgfqpoint{2.473271in}{2.191325in}}{\pgfqpoint{2.462221in}{2.191325in}}%
\pgfpathcurveto{\pgfqpoint{2.451171in}{2.191325in}}{\pgfqpoint{2.440572in}{2.186934in}}{\pgfqpoint{2.432758in}{2.179121in}}%
\pgfpathcurveto{\pgfqpoint{2.424944in}{2.171307in}}{\pgfqpoint{2.420554in}{2.160708in}}{\pgfqpoint{2.420554in}{2.149658in}}%
\pgfpathcurveto{\pgfqpoint{2.420554in}{2.138608in}}{\pgfqpoint{2.424944in}{2.128009in}}{\pgfqpoint{2.432758in}{2.120195in}}%
\pgfpathcurveto{\pgfqpoint{2.440572in}{2.112381in}}{\pgfqpoint{2.451171in}{2.107991in}}{\pgfqpoint{2.462221in}{2.107991in}}%
\pgfpathclose%
\pgfusepath{stroke,fill}%
\end{pgfscope}%
\begin{pgfscope}%
\pgfpathrectangle{\pgfqpoint{0.600000in}{0.600000in}}{\pgfqpoint{3.900000in}{3.900000in}}%
\pgfusepath{clip}%
\pgfsetbuttcap%
\pgfsetroundjoin%
\definecolor{currentfill}{rgb}{0.121569,0.466667,0.705882}%
\pgfsetfillcolor{currentfill}%
\pgfsetlinewidth{1.003750pt}%
\definecolor{currentstroke}{rgb}{0.121569,0.466667,0.705882}%
\pgfsetstrokecolor{currentstroke}%
\pgfsetdash{}{0pt}%
\pgfpathmoveto{\pgfqpoint{1.785120in}{2.743828in}}%
\pgfpathcurveto{\pgfqpoint{1.796170in}{2.743828in}}{\pgfqpoint{1.806769in}{2.748219in}}{\pgfqpoint{1.814582in}{2.756032in}}%
\pgfpathcurveto{\pgfqpoint{1.822396in}{2.763846in}}{\pgfqpoint{1.826786in}{2.774445in}}{\pgfqpoint{1.826786in}{2.785495in}}%
\pgfpathcurveto{\pgfqpoint{1.826786in}{2.796545in}}{\pgfqpoint{1.822396in}{2.807144in}}{\pgfqpoint{1.814582in}{2.814958in}}%
\pgfpathcurveto{\pgfqpoint{1.806769in}{2.822771in}}{\pgfqpoint{1.796170in}{2.827162in}}{\pgfqpoint{1.785120in}{2.827162in}}%
\pgfpathcurveto{\pgfqpoint{1.774070in}{2.827162in}}{\pgfqpoint{1.763470in}{2.822771in}}{\pgfqpoint{1.755657in}{2.814958in}}%
\pgfpathcurveto{\pgfqpoint{1.747843in}{2.807144in}}{\pgfqpoint{1.743453in}{2.796545in}}{\pgfqpoint{1.743453in}{2.785495in}}%
\pgfpathcurveto{\pgfqpoint{1.743453in}{2.774445in}}{\pgfqpoint{1.747843in}{2.763846in}}{\pgfqpoint{1.755657in}{2.756032in}}%
\pgfpathcurveto{\pgfqpoint{1.763470in}{2.748219in}}{\pgfqpoint{1.774070in}{2.743828in}}{\pgfqpoint{1.785120in}{2.743828in}}%
\pgfpathclose%
\pgfusepath{stroke,fill}%
\end{pgfscope}%
\begin{pgfscope}%
\pgfpathrectangle{\pgfqpoint{0.600000in}{0.600000in}}{\pgfqpoint{3.900000in}{3.900000in}}%
\pgfusepath{clip}%
\pgfsetbuttcap%
\pgfsetroundjoin%
\definecolor{currentfill}{rgb}{0.121569,0.466667,0.705882}%
\pgfsetfillcolor{currentfill}%
\pgfsetlinewidth{1.003750pt}%
\definecolor{currentstroke}{rgb}{0.121569,0.466667,0.705882}%
\pgfsetstrokecolor{currentstroke}%
\pgfsetdash{}{0pt}%
\pgfpathmoveto{\pgfqpoint{2.180705in}{2.251920in}}%
\pgfpathcurveto{\pgfqpoint{2.191755in}{2.251920in}}{\pgfqpoint{2.202354in}{2.256310in}}{\pgfqpoint{2.210168in}{2.264124in}}%
\pgfpathcurveto{\pgfqpoint{2.217982in}{2.271937in}}{\pgfqpoint{2.222372in}{2.282537in}}{\pgfqpoint{2.222372in}{2.293587in}}%
\pgfpathcurveto{\pgfqpoint{2.222372in}{2.304637in}}{\pgfqpoint{2.217982in}{2.315236in}}{\pgfqpoint{2.210168in}{2.323049in}}%
\pgfpathcurveto{\pgfqpoint{2.202354in}{2.330863in}}{\pgfqpoint{2.191755in}{2.335253in}}{\pgfqpoint{2.180705in}{2.335253in}}%
\pgfpathcurveto{\pgfqpoint{2.169655in}{2.335253in}}{\pgfqpoint{2.159056in}{2.330863in}}{\pgfqpoint{2.151242in}{2.323049in}}%
\pgfpathcurveto{\pgfqpoint{2.143429in}{2.315236in}}{\pgfqpoint{2.139039in}{2.304637in}}{\pgfqpoint{2.139039in}{2.293587in}}%
\pgfpathcurveto{\pgfqpoint{2.139039in}{2.282537in}}{\pgfqpoint{2.143429in}{2.271937in}}{\pgfqpoint{2.151242in}{2.264124in}}%
\pgfpathcurveto{\pgfqpoint{2.159056in}{2.256310in}}{\pgfqpoint{2.169655in}{2.251920in}}{\pgfqpoint{2.180705in}{2.251920in}}%
\pgfpathclose%
\pgfusepath{stroke,fill}%
\end{pgfscope}%
\begin{pgfscope}%
\pgfpathrectangle{\pgfqpoint{0.600000in}{0.600000in}}{\pgfqpoint{3.900000in}{3.900000in}}%
\pgfusepath{clip}%
\pgfsetbuttcap%
\pgfsetroundjoin%
\definecolor{currentfill}{rgb}{0.121569,0.466667,0.705882}%
\pgfsetfillcolor{currentfill}%
\pgfsetlinewidth{1.003750pt}%
\definecolor{currentstroke}{rgb}{0.121569,0.466667,0.705882}%
\pgfsetstrokecolor{currentstroke}%
\pgfsetdash{}{0pt}%
\pgfpathmoveto{\pgfqpoint{2.417965in}{3.417535in}}%
\pgfpathcurveto{\pgfqpoint{2.429015in}{3.417535in}}{\pgfqpoint{2.439614in}{3.421926in}}{\pgfqpoint{2.447428in}{3.429739in}}%
\pgfpathcurveto{\pgfqpoint{2.455241in}{3.437553in}}{\pgfqpoint{2.459632in}{3.448152in}}{\pgfqpoint{2.459632in}{3.459202in}}%
\pgfpathcurveto{\pgfqpoint{2.459632in}{3.470252in}}{\pgfqpoint{2.455241in}{3.480851in}}{\pgfqpoint{2.447428in}{3.488665in}}%
\pgfpathcurveto{\pgfqpoint{2.439614in}{3.496478in}}{\pgfqpoint{2.429015in}{3.500869in}}{\pgfqpoint{2.417965in}{3.500869in}}%
\pgfpathcurveto{\pgfqpoint{2.406915in}{3.500869in}}{\pgfqpoint{2.396316in}{3.496478in}}{\pgfqpoint{2.388502in}{3.488665in}}%
\pgfpathcurveto{\pgfqpoint{2.380689in}{3.480851in}}{\pgfqpoint{2.376298in}{3.470252in}}{\pgfqpoint{2.376298in}{3.459202in}}%
\pgfpathcurveto{\pgfqpoint{2.376298in}{3.448152in}}{\pgfqpoint{2.380689in}{3.437553in}}{\pgfqpoint{2.388502in}{3.429739in}}%
\pgfpathcurveto{\pgfqpoint{2.396316in}{3.421926in}}{\pgfqpoint{2.406915in}{3.417535in}}{\pgfqpoint{2.417965in}{3.417535in}}%
\pgfpathclose%
\pgfusepath{stroke,fill}%
\end{pgfscope}%
\begin{pgfscope}%
\pgfpathrectangle{\pgfqpoint{0.600000in}{0.600000in}}{\pgfqpoint{3.900000in}{3.900000in}}%
\pgfusepath{clip}%
\pgfsetbuttcap%
\pgfsetroundjoin%
\definecolor{currentfill}{rgb}{0.121569,0.466667,0.705882}%
\pgfsetfillcolor{currentfill}%
\pgfsetlinewidth{1.003750pt}%
\definecolor{currentstroke}{rgb}{0.121569,0.466667,0.705882}%
\pgfsetstrokecolor{currentstroke}%
\pgfsetdash{}{0pt}%
\pgfpathmoveto{\pgfqpoint{2.014598in}{3.888033in}}%
\pgfpathcurveto{\pgfqpoint{2.025648in}{3.888033in}}{\pgfqpoint{2.036247in}{3.892423in}}{\pgfqpoint{2.044061in}{3.900237in}}%
\pgfpathcurveto{\pgfqpoint{2.051875in}{3.908051in}}{\pgfqpoint{2.056265in}{3.918650in}}{\pgfqpoint{2.056265in}{3.929700in}}%
\pgfpathcurveto{\pgfqpoint{2.056265in}{3.940750in}}{\pgfqpoint{2.051875in}{3.951349in}}{\pgfqpoint{2.044061in}{3.959163in}}%
\pgfpathcurveto{\pgfqpoint{2.036247in}{3.966976in}}{\pgfqpoint{2.025648in}{3.971367in}}{\pgfqpoint{2.014598in}{3.971367in}}%
\pgfpathcurveto{\pgfqpoint{2.003548in}{3.971367in}}{\pgfqpoint{1.992949in}{3.966976in}}{\pgfqpoint{1.985135in}{3.959163in}}%
\pgfpathcurveto{\pgfqpoint{1.977322in}{3.951349in}}{\pgfqpoint{1.972931in}{3.940750in}}{\pgfqpoint{1.972931in}{3.929700in}}%
\pgfpathcurveto{\pgfqpoint{1.972931in}{3.918650in}}{\pgfqpoint{1.977322in}{3.908051in}}{\pgfqpoint{1.985135in}{3.900237in}}%
\pgfpathcurveto{\pgfqpoint{1.992949in}{3.892423in}}{\pgfqpoint{2.003548in}{3.888033in}}{\pgfqpoint{2.014598in}{3.888033in}}%
\pgfpathclose%
\pgfusepath{stroke,fill}%
\end{pgfscope}%
\begin{pgfscope}%
\pgfpathrectangle{\pgfqpoint{0.600000in}{0.600000in}}{\pgfqpoint{3.900000in}{3.900000in}}%
\pgfusepath{clip}%
\pgfsetbuttcap%
\pgfsetroundjoin%
\definecolor{currentfill}{rgb}{0.121569,0.466667,0.705882}%
\pgfsetfillcolor{currentfill}%
\pgfsetlinewidth{1.003750pt}%
\definecolor{currentstroke}{rgb}{0.121569,0.466667,0.705882}%
\pgfsetstrokecolor{currentstroke}%
\pgfsetdash{}{0pt}%
\pgfpathmoveto{\pgfqpoint{2.546254in}{2.077267in}}%
\pgfpathcurveto{\pgfqpoint{2.557304in}{2.077267in}}{\pgfqpoint{2.567903in}{2.081657in}}{\pgfqpoint{2.575717in}{2.089471in}}%
\pgfpathcurveto{\pgfqpoint{2.583530in}{2.097284in}}{\pgfqpoint{2.587921in}{2.107883in}}{\pgfqpoint{2.587921in}{2.118933in}}%
\pgfpathcurveto{\pgfqpoint{2.587921in}{2.129983in}}{\pgfqpoint{2.583530in}{2.140582in}}{\pgfqpoint{2.575717in}{2.148396in}}%
\pgfpathcurveto{\pgfqpoint{2.567903in}{2.156210in}}{\pgfqpoint{2.557304in}{2.160600in}}{\pgfqpoint{2.546254in}{2.160600in}}%
\pgfpathcurveto{\pgfqpoint{2.535204in}{2.160600in}}{\pgfqpoint{2.524605in}{2.156210in}}{\pgfqpoint{2.516791in}{2.148396in}}%
\pgfpathcurveto{\pgfqpoint{2.508977in}{2.140582in}}{\pgfqpoint{2.504587in}{2.129983in}}{\pgfqpoint{2.504587in}{2.118933in}}%
\pgfpathcurveto{\pgfqpoint{2.504587in}{2.107883in}}{\pgfqpoint{2.508977in}{2.097284in}}{\pgfqpoint{2.516791in}{2.089471in}}%
\pgfpathcurveto{\pgfqpoint{2.524605in}{2.081657in}}{\pgfqpoint{2.535204in}{2.077267in}}{\pgfqpoint{2.546254in}{2.077267in}}%
\pgfpathclose%
\pgfusepath{stroke,fill}%
\end{pgfscope}%
\begin{pgfscope}%
\pgfpathrectangle{\pgfqpoint{0.600000in}{0.600000in}}{\pgfqpoint{3.900000in}{3.900000in}}%
\pgfusepath{clip}%
\pgfsetbuttcap%
\pgfsetroundjoin%
\definecolor{currentfill}{rgb}{0.121569,0.466667,0.705882}%
\pgfsetfillcolor{currentfill}%
\pgfsetlinewidth{1.003750pt}%
\definecolor{currentstroke}{rgb}{0.121569,0.466667,0.705882}%
\pgfsetstrokecolor{currentstroke}%
\pgfsetdash{}{0pt}%
\pgfpathmoveto{\pgfqpoint{2.013318in}{1.973569in}}%
\pgfpathcurveto{\pgfqpoint{2.024368in}{1.973569in}}{\pgfqpoint{2.034967in}{1.977959in}}{\pgfqpoint{2.042780in}{1.985773in}}%
\pgfpathcurveto{\pgfqpoint{2.050594in}{1.993586in}}{\pgfqpoint{2.054984in}{2.004185in}}{\pgfqpoint{2.054984in}{2.015235in}}%
\pgfpathcurveto{\pgfqpoint{2.054984in}{2.026285in}}{\pgfqpoint{2.050594in}{2.036885in}}{\pgfqpoint{2.042780in}{2.044698in}}%
\pgfpathcurveto{\pgfqpoint{2.034967in}{2.052512in}}{\pgfqpoint{2.024368in}{2.056902in}}{\pgfqpoint{2.013318in}{2.056902in}}%
\pgfpathcurveto{\pgfqpoint{2.002268in}{2.056902in}}{\pgfqpoint{1.991668in}{2.052512in}}{\pgfqpoint{1.983855in}{2.044698in}}%
\pgfpathcurveto{\pgfqpoint{1.976041in}{2.036885in}}{\pgfqpoint{1.971651in}{2.026285in}}{\pgfqpoint{1.971651in}{2.015235in}}%
\pgfpathcurveto{\pgfqpoint{1.971651in}{2.004185in}}{\pgfqpoint{1.976041in}{1.993586in}}{\pgfqpoint{1.983855in}{1.985773in}}%
\pgfpathcurveto{\pgfqpoint{1.991668in}{1.977959in}}{\pgfqpoint{2.002268in}{1.973569in}}{\pgfqpoint{2.013318in}{1.973569in}}%
\pgfpathclose%
\pgfusepath{stroke,fill}%
\end{pgfscope}%
\begin{pgfscope}%
\pgfpathrectangle{\pgfqpoint{0.600000in}{0.600000in}}{\pgfqpoint{3.900000in}{3.900000in}}%
\pgfusepath{clip}%
\pgfsetbuttcap%
\pgfsetroundjoin%
\definecolor{currentfill}{rgb}{0.121569,0.466667,0.705882}%
\pgfsetfillcolor{currentfill}%
\pgfsetlinewidth{1.003750pt}%
\definecolor{currentstroke}{rgb}{0.121569,0.466667,0.705882}%
\pgfsetstrokecolor{currentstroke}%
\pgfsetdash{}{0pt}%
\pgfpathmoveto{\pgfqpoint{2.422468in}{2.193672in}}%
\pgfpathcurveto{\pgfqpoint{2.433518in}{2.193672in}}{\pgfqpoint{2.444117in}{2.198062in}}{\pgfqpoint{2.451930in}{2.205876in}}%
\pgfpathcurveto{\pgfqpoint{2.459744in}{2.213690in}}{\pgfqpoint{2.464134in}{2.224289in}}{\pgfqpoint{2.464134in}{2.235339in}}%
\pgfpathcurveto{\pgfqpoint{2.464134in}{2.246389in}}{\pgfqpoint{2.459744in}{2.256988in}}{\pgfqpoint{2.451930in}{2.264802in}}%
\pgfpathcurveto{\pgfqpoint{2.444117in}{2.272615in}}{\pgfqpoint{2.433518in}{2.277005in}}{\pgfqpoint{2.422468in}{2.277005in}}%
\pgfpathcurveto{\pgfqpoint{2.411418in}{2.277005in}}{\pgfqpoint{2.400819in}{2.272615in}}{\pgfqpoint{2.393005in}{2.264802in}}%
\pgfpathcurveto{\pgfqpoint{2.385191in}{2.256988in}}{\pgfqpoint{2.380801in}{2.246389in}}{\pgfqpoint{2.380801in}{2.235339in}}%
\pgfpathcurveto{\pgfqpoint{2.380801in}{2.224289in}}{\pgfqpoint{2.385191in}{2.213690in}}{\pgfqpoint{2.393005in}{2.205876in}}%
\pgfpathcurveto{\pgfqpoint{2.400819in}{2.198062in}}{\pgfqpoint{2.411418in}{2.193672in}}{\pgfqpoint{2.422468in}{2.193672in}}%
\pgfpathclose%
\pgfusepath{stroke,fill}%
\end{pgfscope}%
\begin{pgfscope}%
\pgfpathrectangle{\pgfqpoint{0.600000in}{0.600000in}}{\pgfqpoint{3.900000in}{3.900000in}}%
\pgfusepath{clip}%
\pgfsetbuttcap%
\pgfsetroundjoin%
\definecolor{currentfill}{rgb}{0.121569,0.466667,0.705882}%
\pgfsetfillcolor{currentfill}%
\pgfsetlinewidth{1.003750pt}%
\definecolor{currentstroke}{rgb}{0.121569,0.466667,0.705882}%
\pgfsetstrokecolor{currentstroke}%
\pgfsetdash{}{0pt}%
\pgfpathmoveto{\pgfqpoint{2.517933in}{2.484529in}}%
\pgfpathcurveto{\pgfqpoint{2.528984in}{2.484529in}}{\pgfqpoint{2.539583in}{2.488919in}}{\pgfqpoint{2.547396in}{2.496733in}}%
\pgfpathcurveto{\pgfqpoint{2.555210in}{2.504547in}}{\pgfqpoint{2.559600in}{2.515146in}}{\pgfqpoint{2.559600in}{2.526196in}}%
\pgfpathcurveto{\pgfqpoint{2.559600in}{2.537246in}}{\pgfqpoint{2.555210in}{2.547845in}}{\pgfqpoint{2.547396in}{2.555659in}}%
\pgfpathcurveto{\pgfqpoint{2.539583in}{2.563472in}}{\pgfqpoint{2.528984in}{2.567863in}}{\pgfqpoint{2.517933in}{2.567863in}}%
\pgfpathcurveto{\pgfqpoint{2.506883in}{2.567863in}}{\pgfqpoint{2.496284in}{2.563472in}}{\pgfqpoint{2.488471in}{2.555659in}}%
\pgfpathcurveto{\pgfqpoint{2.480657in}{2.547845in}}{\pgfqpoint{2.476267in}{2.537246in}}{\pgfqpoint{2.476267in}{2.526196in}}%
\pgfpathcurveto{\pgfqpoint{2.476267in}{2.515146in}}{\pgfqpoint{2.480657in}{2.504547in}}{\pgfqpoint{2.488471in}{2.496733in}}%
\pgfpathcurveto{\pgfqpoint{2.496284in}{2.488919in}}{\pgfqpoint{2.506883in}{2.484529in}}{\pgfqpoint{2.517933in}{2.484529in}}%
\pgfpathclose%
\pgfusepath{stroke,fill}%
\end{pgfscope}%
\begin{pgfscope}%
\pgfpathrectangle{\pgfqpoint{0.600000in}{0.600000in}}{\pgfqpoint{3.900000in}{3.900000in}}%
\pgfusepath{clip}%
\pgfsetbuttcap%
\pgfsetroundjoin%
\definecolor{currentfill}{rgb}{0.121569,0.466667,0.705882}%
\pgfsetfillcolor{currentfill}%
\pgfsetlinewidth{1.003750pt}%
\definecolor{currentstroke}{rgb}{0.121569,0.466667,0.705882}%
\pgfsetstrokecolor{currentstroke}%
\pgfsetdash{}{0pt}%
\pgfpathmoveto{\pgfqpoint{2.738214in}{2.691716in}}%
\pgfpathcurveto{\pgfqpoint{2.749264in}{2.691716in}}{\pgfqpoint{2.759863in}{2.696106in}}{\pgfqpoint{2.767677in}{2.703920in}}%
\pgfpathcurveto{\pgfqpoint{2.775490in}{2.711733in}}{\pgfqpoint{2.779881in}{2.722332in}}{\pgfqpoint{2.779881in}{2.733382in}}%
\pgfpathcurveto{\pgfqpoint{2.779881in}{2.744433in}}{\pgfqpoint{2.775490in}{2.755032in}}{\pgfqpoint{2.767677in}{2.762845in}}%
\pgfpathcurveto{\pgfqpoint{2.759863in}{2.770659in}}{\pgfqpoint{2.749264in}{2.775049in}}{\pgfqpoint{2.738214in}{2.775049in}}%
\pgfpathcurveto{\pgfqpoint{2.727164in}{2.775049in}}{\pgfqpoint{2.716565in}{2.770659in}}{\pgfqpoint{2.708751in}{2.762845in}}%
\pgfpathcurveto{\pgfqpoint{2.700938in}{2.755032in}}{\pgfqpoint{2.696547in}{2.744433in}}{\pgfqpoint{2.696547in}{2.733382in}}%
\pgfpathcurveto{\pgfqpoint{2.696547in}{2.722332in}}{\pgfqpoint{2.700938in}{2.711733in}}{\pgfqpoint{2.708751in}{2.703920in}}%
\pgfpathcurveto{\pgfqpoint{2.716565in}{2.696106in}}{\pgfqpoint{2.727164in}{2.691716in}}{\pgfqpoint{2.738214in}{2.691716in}}%
\pgfpathclose%
\pgfusepath{stroke,fill}%
\end{pgfscope}%
\begin{pgfscope}%
\pgfpathrectangle{\pgfqpoint{0.600000in}{0.600000in}}{\pgfqpoint{3.900000in}{3.900000in}}%
\pgfusepath{clip}%
\pgfsetbuttcap%
\pgfsetroundjoin%
\definecolor{currentfill}{rgb}{0.121569,0.466667,0.705882}%
\pgfsetfillcolor{currentfill}%
\pgfsetlinewidth{1.003750pt}%
\definecolor{currentstroke}{rgb}{0.121569,0.466667,0.705882}%
\pgfsetstrokecolor{currentstroke}%
\pgfsetdash{}{0pt}%
\pgfpathmoveto{\pgfqpoint{2.956733in}{2.542486in}}%
\pgfpathcurveto{\pgfqpoint{2.967783in}{2.542486in}}{\pgfqpoint{2.978382in}{2.546876in}}{\pgfqpoint{2.986196in}{2.554690in}}%
\pgfpathcurveto{\pgfqpoint{2.994010in}{2.562504in}}{\pgfqpoint{2.998400in}{2.573103in}}{\pgfqpoint{2.998400in}{2.584153in}}%
\pgfpathcurveto{\pgfqpoint{2.998400in}{2.595203in}}{\pgfqpoint{2.994010in}{2.605802in}}{\pgfqpoint{2.986196in}{2.613616in}}%
\pgfpathcurveto{\pgfqpoint{2.978382in}{2.621429in}}{\pgfqpoint{2.967783in}{2.625820in}}{\pgfqpoint{2.956733in}{2.625820in}}%
\pgfpathcurveto{\pgfqpoint{2.945683in}{2.625820in}}{\pgfqpoint{2.935084in}{2.621429in}}{\pgfqpoint{2.927270in}{2.613616in}}%
\pgfpathcurveto{\pgfqpoint{2.919457in}{2.605802in}}{\pgfqpoint{2.915067in}{2.595203in}}{\pgfqpoint{2.915067in}{2.584153in}}%
\pgfpathcurveto{\pgfqpoint{2.915067in}{2.573103in}}{\pgfqpoint{2.919457in}{2.562504in}}{\pgfqpoint{2.927270in}{2.554690in}}%
\pgfpathcurveto{\pgfqpoint{2.935084in}{2.546876in}}{\pgfqpoint{2.945683in}{2.542486in}}{\pgfqpoint{2.956733in}{2.542486in}}%
\pgfpathclose%
\pgfusepath{stroke,fill}%
\end{pgfscope}%
\begin{pgfscope}%
\pgfpathrectangle{\pgfqpoint{0.600000in}{0.600000in}}{\pgfqpoint{3.900000in}{3.900000in}}%
\pgfusepath{clip}%
\pgfsetbuttcap%
\pgfsetroundjoin%
\definecolor{currentfill}{rgb}{0.121569,0.466667,0.705882}%
\pgfsetfillcolor{currentfill}%
\pgfsetlinewidth{1.003750pt}%
\definecolor{currentstroke}{rgb}{0.121569,0.466667,0.705882}%
\pgfsetstrokecolor{currentstroke}%
\pgfsetdash{}{0pt}%
\pgfpathmoveto{\pgfqpoint{2.411492in}{3.472821in}}%
\pgfpathcurveto{\pgfqpoint{2.422542in}{3.472821in}}{\pgfqpoint{2.433141in}{3.477211in}}{\pgfqpoint{2.440955in}{3.485025in}}%
\pgfpathcurveto{\pgfqpoint{2.448768in}{3.492838in}}{\pgfqpoint{2.453158in}{3.503437in}}{\pgfqpoint{2.453158in}{3.514488in}}%
\pgfpathcurveto{\pgfqpoint{2.453158in}{3.525538in}}{\pgfqpoint{2.448768in}{3.536137in}}{\pgfqpoint{2.440955in}{3.543950in}}%
\pgfpathcurveto{\pgfqpoint{2.433141in}{3.551764in}}{\pgfqpoint{2.422542in}{3.556154in}}{\pgfqpoint{2.411492in}{3.556154in}}%
\pgfpathcurveto{\pgfqpoint{2.400442in}{3.556154in}}{\pgfqpoint{2.389843in}{3.551764in}}{\pgfqpoint{2.382029in}{3.543950in}}%
\pgfpathcurveto{\pgfqpoint{2.374215in}{3.536137in}}{\pgfqpoint{2.369825in}{3.525538in}}{\pgfqpoint{2.369825in}{3.514488in}}%
\pgfpathcurveto{\pgfqpoint{2.369825in}{3.503437in}}{\pgfqpoint{2.374215in}{3.492838in}}{\pgfqpoint{2.382029in}{3.485025in}}%
\pgfpathcurveto{\pgfqpoint{2.389843in}{3.477211in}}{\pgfqpoint{2.400442in}{3.472821in}}{\pgfqpoint{2.411492in}{3.472821in}}%
\pgfpathclose%
\pgfusepath{stroke,fill}%
\end{pgfscope}%
\begin{pgfscope}%
\pgfpathrectangle{\pgfqpoint{0.600000in}{0.600000in}}{\pgfqpoint{3.900000in}{3.900000in}}%
\pgfusepath{clip}%
\pgfsetbuttcap%
\pgfsetroundjoin%
\definecolor{currentfill}{rgb}{0.121569,0.466667,0.705882}%
\pgfsetfillcolor{currentfill}%
\pgfsetlinewidth{1.003750pt}%
\definecolor{currentstroke}{rgb}{0.121569,0.466667,0.705882}%
\pgfsetstrokecolor{currentstroke}%
\pgfsetdash{}{0pt}%
\pgfpathmoveto{\pgfqpoint{2.575520in}{3.318769in}}%
\pgfpathcurveto{\pgfqpoint{2.586570in}{3.318769in}}{\pgfqpoint{2.597169in}{3.323159in}}{\pgfqpoint{2.604983in}{3.330973in}}%
\pgfpathcurveto{\pgfqpoint{2.612796in}{3.338787in}}{\pgfqpoint{2.617187in}{3.349386in}}{\pgfqpoint{2.617187in}{3.360436in}}%
\pgfpathcurveto{\pgfqpoint{2.617187in}{3.371486in}}{\pgfqpoint{2.612796in}{3.382085in}}{\pgfqpoint{2.604983in}{3.389899in}}%
\pgfpathcurveto{\pgfqpoint{2.597169in}{3.397712in}}{\pgfqpoint{2.586570in}{3.402102in}}{\pgfqpoint{2.575520in}{3.402102in}}%
\pgfpathcurveto{\pgfqpoint{2.564470in}{3.402102in}}{\pgfqpoint{2.553871in}{3.397712in}}{\pgfqpoint{2.546057in}{3.389899in}}%
\pgfpathcurveto{\pgfqpoint{2.538244in}{3.382085in}}{\pgfqpoint{2.533853in}{3.371486in}}{\pgfqpoint{2.533853in}{3.360436in}}%
\pgfpathcurveto{\pgfqpoint{2.533853in}{3.349386in}}{\pgfqpoint{2.538244in}{3.338787in}}{\pgfqpoint{2.546057in}{3.330973in}}%
\pgfpathcurveto{\pgfqpoint{2.553871in}{3.323159in}}{\pgfqpoint{2.564470in}{3.318769in}}{\pgfqpoint{2.575520in}{3.318769in}}%
\pgfpathclose%
\pgfusepath{stroke,fill}%
\end{pgfscope}%
\begin{pgfscope}%
\pgfpathrectangle{\pgfqpoint{0.600000in}{0.600000in}}{\pgfqpoint{3.900000in}{3.900000in}}%
\pgfusepath{clip}%
\pgfsetbuttcap%
\pgfsetroundjoin%
\definecolor{currentfill}{rgb}{0.121569,0.466667,0.705882}%
\pgfsetfillcolor{currentfill}%
\pgfsetlinewidth{1.003750pt}%
\definecolor{currentstroke}{rgb}{0.121569,0.466667,0.705882}%
\pgfsetstrokecolor{currentstroke}%
\pgfsetdash{}{0pt}%
\pgfpathmoveto{\pgfqpoint{2.732706in}{2.494586in}}%
\pgfpathcurveto{\pgfqpoint{2.743756in}{2.494586in}}{\pgfqpoint{2.754355in}{2.498977in}}{\pgfqpoint{2.762169in}{2.506790in}}%
\pgfpathcurveto{\pgfqpoint{2.769983in}{2.514604in}}{\pgfqpoint{2.774373in}{2.525203in}}{\pgfqpoint{2.774373in}{2.536253in}}%
\pgfpathcurveto{\pgfqpoint{2.774373in}{2.547303in}}{\pgfqpoint{2.769983in}{2.557902in}}{\pgfqpoint{2.762169in}{2.565716in}}%
\pgfpathcurveto{\pgfqpoint{2.754355in}{2.573529in}}{\pgfqpoint{2.743756in}{2.577920in}}{\pgfqpoint{2.732706in}{2.577920in}}%
\pgfpathcurveto{\pgfqpoint{2.721656in}{2.577920in}}{\pgfqpoint{2.711057in}{2.573529in}}{\pgfqpoint{2.703243in}{2.565716in}}%
\pgfpathcurveto{\pgfqpoint{2.695430in}{2.557902in}}{\pgfqpoint{2.691039in}{2.547303in}}{\pgfqpoint{2.691039in}{2.536253in}}%
\pgfpathcurveto{\pgfqpoint{2.691039in}{2.525203in}}{\pgfqpoint{2.695430in}{2.514604in}}{\pgfqpoint{2.703243in}{2.506790in}}%
\pgfpathcurveto{\pgfqpoint{2.711057in}{2.498977in}}{\pgfqpoint{2.721656in}{2.494586in}}{\pgfqpoint{2.732706in}{2.494586in}}%
\pgfpathclose%
\pgfusepath{stroke,fill}%
\end{pgfscope}%
\begin{pgfscope}%
\pgfpathrectangle{\pgfqpoint{0.600000in}{0.600000in}}{\pgfqpoint{3.900000in}{3.900000in}}%
\pgfusepath{clip}%
\pgfsetbuttcap%
\pgfsetroundjoin%
\definecolor{currentfill}{rgb}{0.121569,0.466667,0.705882}%
\pgfsetfillcolor{currentfill}%
\pgfsetlinewidth{1.003750pt}%
\definecolor{currentstroke}{rgb}{0.121569,0.466667,0.705882}%
\pgfsetstrokecolor{currentstroke}%
\pgfsetdash{}{0pt}%
\pgfpathmoveto{\pgfqpoint{2.713864in}{2.883196in}}%
\pgfpathcurveto{\pgfqpoint{2.724914in}{2.883196in}}{\pgfqpoint{2.735513in}{2.887586in}}{\pgfqpoint{2.743327in}{2.895400in}}%
\pgfpathcurveto{\pgfqpoint{2.751141in}{2.903214in}}{\pgfqpoint{2.755531in}{2.913813in}}{\pgfqpoint{2.755531in}{2.924863in}}%
\pgfpathcurveto{\pgfqpoint{2.755531in}{2.935913in}}{\pgfqpoint{2.751141in}{2.946512in}}{\pgfqpoint{2.743327in}{2.954326in}}%
\pgfpathcurveto{\pgfqpoint{2.735513in}{2.962139in}}{\pgfqpoint{2.724914in}{2.966529in}}{\pgfqpoint{2.713864in}{2.966529in}}%
\pgfpathcurveto{\pgfqpoint{2.702814in}{2.966529in}}{\pgfqpoint{2.692215in}{2.962139in}}{\pgfqpoint{2.684402in}{2.954326in}}%
\pgfpathcurveto{\pgfqpoint{2.676588in}{2.946512in}}{\pgfqpoint{2.672198in}{2.935913in}}{\pgfqpoint{2.672198in}{2.924863in}}%
\pgfpathcurveto{\pgfqpoint{2.672198in}{2.913813in}}{\pgfqpoint{2.676588in}{2.903214in}}{\pgfqpoint{2.684402in}{2.895400in}}%
\pgfpathcurveto{\pgfqpoint{2.692215in}{2.887586in}}{\pgfqpoint{2.702814in}{2.883196in}}{\pgfqpoint{2.713864in}{2.883196in}}%
\pgfpathclose%
\pgfusepath{stroke,fill}%
\end{pgfscope}%
\begin{pgfscope}%
\pgfpathrectangle{\pgfqpoint{0.600000in}{0.600000in}}{\pgfqpoint{3.900000in}{3.900000in}}%
\pgfusepath{clip}%
\pgfsetbuttcap%
\pgfsetroundjoin%
\definecolor{currentfill}{rgb}{0.121569,0.466667,0.705882}%
\pgfsetfillcolor{currentfill}%
\pgfsetlinewidth{1.003750pt}%
\definecolor{currentstroke}{rgb}{0.121569,0.466667,0.705882}%
\pgfsetstrokecolor{currentstroke}%
\pgfsetdash{}{0pt}%
\pgfpathmoveto{\pgfqpoint{2.862017in}{2.731493in}}%
\pgfpathcurveto{\pgfqpoint{2.873067in}{2.731493in}}{\pgfqpoint{2.883666in}{2.735884in}}{\pgfqpoint{2.891480in}{2.743697in}}%
\pgfpathcurveto{\pgfqpoint{2.899294in}{2.751511in}}{\pgfqpoint{2.903684in}{2.762110in}}{\pgfqpoint{2.903684in}{2.773160in}}%
\pgfpathcurveto{\pgfqpoint{2.903684in}{2.784210in}}{\pgfqpoint{2.899294in}{2.794809in}}{\pgfqpoint{2.891480in}{2.802623in}}%
\pgfpathcurveto{\pgfqpoint{2.883666in}{2.810437in}}{\pgfqpoint{2.873067in}{2.814827in}}{\pgfqpoint{2.862017in}{2.814827in}}%
\pgfpathcurveto{\pgfqpoint{2.850967in}{2.814827in}}{\pgfqpoint{2.840368in}{2.810437in}}{\pgfqpoint{2.832554in}{2.802623in}}%
\pgfpathcurveto{\pgfqpoint{2.824741in}{2.794809in}}{\pgfqpoint{2.820350in}{2.784210in}}{\pgfqpoint{2.820350in}{2.773160in}}%
\pgfpathcurveto{\pgfqpoint{2.820350in}{2.762110in}}{\pgfqpoint{2.824741in}{2.751511in}}{\pgfqpoint{2.832554in}{2.743697in}}%
\pgfpathcurveto{\pgfqpoint{2.840368in}{2.735884in}}{\pgfqpoint{2.850967in}{2.731493in}}{\pgfqpoint{2.862017in}{2.731493in}}%
\pgfpathclose%
\pgfusepath{stroke,fill}%
\end{pgfscope}%
\begin{pgfscope}%
\pgfpathrectangle{\pgfqpoint{0.600000in}{0.600000in}}{\pgfqpoint{3.900000in}{3.900000in}}%
\pgfusepath{clip}%
\pgfsetbuttcap%
\pgfsetroundjoin%
\definecolor{currentfill}{rgb}{0.121569,0.466667,0.705882}%
\pgfsetfillcolor{currentfill}%
\pgfsetlinewidth{1.003750pt}%
\definecolor{currentstroke}{rgb}{0.121569,0.466667,0.705882}%
\pgfsetstrokecolor{currentstroke}%
\pgfsetdash{}{0pt}%
\pgfpathmoveto{\pgfqpoint{2.676109in}{2.267709in}}%
\pgfpathcurveto{\pgfqpoint{2.687159in}{2.267709in}}{\pgfqpoint{2.697758in}{2.272099in}}{\pgfqpoint{2.705572in}{2.279913in}}%
\pgfpathcurveto{\pgfqpoint{2.713385in}{2.287726in}}{\pgfqpoint{2.717776in}{2.298325in}}{\pgfqpoint{2.717776in}{2.309375in}}%
\pgfpathcurveto{\pgfqpoint{2.717776in}{2.320426in}}{\pgfqpoint{2.713385in}{2.331025in}}{\pgfqpoint{2.705572in}{2.338838in}}%
\pgfpathcurveto{\pgfqpoint{2.697758in}{2.346652in}}{\pgfqpoint{2.687159in}{2.351042in}}{\pgfqpoint{2.676109in}{2.351042in}}%
\pgfpathcurveto{\pgfqpoint{2.665059in}{2.351042in}}{\pgfqpoint{2.654460in}{2.346652in}}{\pgfqpoint{2.646646in}{2.338838in}}%
\pgfpathcurveto{\pgfqpoint{2.638833in}{2.331025in}}{\pgfqpoint{2.634442in}{2.320426in}}{\pgfqpoint{2.634442in}{2.309375in}}%
\pgfpathcurveto{\pgfqpoint{2.634442in}{2.298325in}}{\pgfqpoint{2.638833in}{2.287726in}}{\pgfqpoint{2.646646in}{2.279913in}}%
\pgfpathcurveto{\pgfqpoint{2.654460in}{2.272099in}}{\pgfqpoint{2.665059in}{2.267709in}}{\pgfqpoint{2.676109in}{2.267709in}}%
\pgfpathclose%
\pgfusepath{stroke,fill}%
\end{pgfscope}%
\begin{pgfscope}%
\pgfpathrectangle{\pgfqpoint{0.600000in}{0.600000in}}{\pgfqpoint{3.900000in}{3.900000in}}%
\pgfusepath{clip}%
\pgfsetbuttcap%
\pgfsetroundjoin%
\definecolor{currentfill}{rgb}{0.121569,0.466667,0.705882}%
\pgfsetfillcolor{currentfill}%
\pgfsetlinewidth{1.003750pt}%
\definecolor{currentstroke}{rgb}{0.121569,0.466667,0.705882}%
\pgfsetstrokecolor{currentstroke}%
\pgfsetdash{}{0pt}%
\pgfpathmoveto{\pgfqpoint{3.173056in}{2.246793in}}%
\pgfpathcurveto{\pgfqpoint{3.184106in}{2.246793in}}{\pgfqpoint{3.194705in}{2.251183in}}{\pgfqpoint{3.202519in}{2.258997in}}%
\pgfpathcurveto{\pgfqpoint{3.210333in}{2.266810in}}{\pgfqpoint{3.214723in}{2.277409in}}{\pgfqpoint{3.214723in}{2.288459in}}%
\pgfpathcurveto{\pgfqpoint{3.214723in}{2.299509in}}{\pgfqpoint{3.210333in}{2.310108in}}{\pgfqpoint{3.202519in}{2.317922in}}%
\pgfpathcurveto{\pgfqpoint{3.194705in}{2.325736in}}{\pgfqpoint{3.184106in}{2.330126in}}{\pgfqpoint{3.173056in}{2.330126in}}%
\pgfpathcurveto{\pgfqpoint{3.162006in}{2.330126in}}{\pgfqpoint{3.151407in}{2.325736in}}{\pgfqpoint{3.143593in}{2.317922in}}%
\pgfpathcurveto{\pgfqpoint{3.135780in}{2.310108in}}{\pgfqpoint{3.131389in}{2.299509in}}{\pgfqpoint{3.131389in}{2.288459in}}%
\pgfpathcurveto{\pgfqpoint{3.131389in}{2.277409in}}{\pgfqpoint{3.135780in}{2.266810in}}{\pgfqpoint{3.143593in}{2.258997in}}%
\pgfpathcurveto{\pgfqpoint{3.151407in}{2.251183in}}{\pgfqpoint{3.162006in}{2.246793in}}{\pgfqpoint{3.173056in}{2.246793in}}%
\pgfpathclose%
\pgfusepath{stroke,fill}%
\end{pgfscope}%
\begin{pgfscope}%
\pgfpathrectangle{\pgfqpoint{0.600000in}{0.600000in}}{\pgfqpoint{3.900000in}{3.900000in}}%
\pgfusepath{clip}%
\pgfsetbuttcap%
\pgfsetroundjoin%
\definecolor{currentfill}{rgb}{0.121569,0.466667,0.705882}%
\pgfsetfillcolor{currentfill}%
\pgfsetlinewidth{1.003750pt}%
\definecolor{currentstroke}{rgb}{0.121569,0.466667,0.705882}%
\pgfsetstrokecolor{currentstroke}%
\pgfsetdash{}{0pt}%
\pgfpathmoveto{\pgfqpoint{2.205860in}{1.412840in}}%
\pgfpathcurveto{\pgfqpoint{2.216910in}{1.412840in}}{\pgfqpoint{2.227509in}{1.417230in}}{\pgfqpoint{2.235323in}{1.425044in}}%
\pgfpathcurveto{\pgfqpoint{2.243136in}{1.432857in}}{\pgfqpoint{2.247527in}{1.443456in}}{\pgfqpoint{2.247527in}{1.454506in}}%
\pgfpathcurveto{\pgfqpoint{2.247527in}{1.465556in}}{\pgfqpoint{2.243136in}{1.476156in}}{\pgfqpoint{2.235323in}{1.483969in}}%
\pgfpathcurveto{\pgfqpoint{2.227509in}{1.491783in}}{\pgfqpoint{2.216910in}{1.496173in}}{\pgfqpoint{2.205860in}{1.496173in}}%
\pgfpathcurveto{\pgfqpoint{2.194810in}{1.496173in}}{\pgfqpoint{2.184211in}{1.491783in}}{\pgfqpoint{2.176397in}{1.483969in}}%
\pgfpathcurveto{\pgfqpoint{2.168583in}{1.476156in}}{\pgfqpoint{2.164193in}{1.465556in}}{\pgfqpoint{2.164193in}{1.454506in}}%
\pgfpathcurveto{\pgfqpoint{2.164193in}{1.443456in}}{\pgfqpoint{2.168583in}{1.432857in}}{\pgfqpoint{2.176397in}{1.425044in}}%
\pgfpathcurveto{\pgfqpoint{2.184211in}{1.417230in}}{\pgfqpoint{2.194810in}{1.412840in}}{\pgfqpoint{2.205860in}{1.412840in}}%
\pgfpathclose%
\pgfusepath{stroke,fill}%
\end{pgfscope}%
\begin{pgfscope}%
\pgfpathrectangle{\pgfqpoint{0.600000in}{0.600000in}}{\pgfqpoint{3.900000in}{3.900000in}}%
\pgfusepath{clip}%
\pgfsetbuttcap%
\pgfsetroundjoin%
\definecolor{currentfill}{rgb}{0.121569,0.466667,0.705882}%
\pgfsetfillcolor{currentfill}%
\pgfsetlinewidth{1.003750pt}%
\definecolor{currentstroke}{rgb}{0.121569,0.466667,0.705882}%
\pgfsetstrokecolor{currentstroke}%
\pgfsetdash{}{0pt}%
\pgfpathmoveto{\pgfqpoint{3.258112in}{2.757168in}}%
\pgfpathcurveto{\pgfqpoint{3.269162in}{2.757168in}}{\pgfqpoint{3.279761in}{2.761558in}}{\pgfqpoint{3.287575in}{2.769372in}}%
\pgfpathcurveto{\pgfqpoint{3.295388in}{2.777186in}}{\pgfqpoint{3.299779in}{2.787785in}}{\pgfqpoint{3.299779in}{2.798835in}}%
\pgfpathcurveto{\pgfqpoint{3.299779in}{2.809885in}}{\pgfqpoint{3.295388in}{2.820484in}}{\pgfqpoint{3.287575in}{2.828298in}}%
\pgfpathcurveto{\pgfqpoint{3.279761in}{2.836111in}}{\pgfqpoint{3.269162in}{2.840502in}}{\pgfqpoint{3.258112in}{2.840502in}}%
\pgfpathcurveto{\pgfqpoint{3.247062in}{2.840502in}}{\pgfqpoint{3.236463in}{2.836111in}}{\pgfqpoint{3.228649in}{2.828298in}}%
\pgfpathcurveto{\pgfqpoint{3.220836in}{2.820484in}}{\pgfqpoint{3.216445in}{2.809885in}}{\pgfqpoint{3.216445in}{2.798835in}}%
\pgfpathcurveto{\pgfqpoint{3.216445in}{2.787785in}}{\pgfqpoint{3.220836in}{2.777186in}}{\pgfqpoint{3.228649in}{2.769372in}}%
\pgfpathcurveto{\pgfqpoint{3.236463in}{2.761558in}}{\pgfqpoint{3.247062in}{2.757168in}}{\pgfqpoint{3.258112in}{2.757168in}}%
\pgfpathclose%
\pgfusepath{stroke,fill}%
\end{pgfscope}%
\begin{pgfscope}%
\pgfpathrectangle{\pgfqpoint{0.600000in}{0.600000in}}{\pgfqpoint{3.900000in}{3.900000in}}%
\pgfusepath{clip}%
\pgfsetbuttcap%
\pgfsetroundjoin%
\definecolor{currentfill}{rgb}{0.121569,0.466667,0.705882}%
\pgfsetfillcolor{currentfill}%
\pgfsetlinewidth{1.003750pt}%
\definecolor{currentstroke}{rgb}{0.121569,0.466667,0.705882}%
\pgfsetstrokecolor{currentstroke}%
\pgfsetdash{}{0pt}%
\pgfpathmoveto{\pgfqpoint{2.937787in}{3.508910in}}%
\pgfpathcurveto{\pgfqpoint{2.948837in}{3.508910in}}{\pgfqpoint{2.959436in}{3.513300in}}{\pgfqpoint{2.967250in}{3.521113in}}%
\pgfpathcurveto{\pgfqpoint{2.975063in}{3.528927in}}{\pgfqpoint{2.979454in}{3.539526in}}{\pgfqpoint{2.979454in}{3.550576in}}%
\pgfpathcurveto{\pgfqpoint{2.979454in}{3.561626in}}{\pgfqpoint{2.975063in}{3.572225in}}{\pgfqpoint{2.967250in}{3.580039in}}%
\pgfpathcurveto{\pgfqpoint{2.959436in}{3.587853in}}{\pgfqpoint{2.948837in}{3.592243in}}{\pgfqpoint{2.937787in}{3.592243in}}%
\pgfpathcurveto{\pgfqpoint{2.926737in}{3.592243in}}{\pgfqpoint{2.916138in}{3.587853in}}{\pgfqpoint{2.908324in}{3.580039in}}%
\pgfpathcurveto{\pgfqpoint{2.900511in}{3.572225in}}{\pgfqpoint{2.896120in}{3.561626in}}{\pgfqpoint{2.896120in}{3.550576in}}%
\pgfpathcurveto{\pgfqpoint{2.896120in}{3.539526in}}{\pgfqpoint{2.900511in}{3.528927in}}{\pgfqpoint{2.908324in}{3.521113in}}%
\pgfpathcurveto{\pgfqpoint{2.916138in}{3.513300in}}{\pgfqpoint{2.926737in}{3.508910in}}{\pgfqpoint{2.937787in}{3.508910in}}%
\pgfpathclose%
\pgfusepath{stroke,fill}%
\end{pgfscope}%
\begin{pgfscope}%
\pgfpathrectangle{\pgfqpoint{0.600000in}{0.600000in}}{\pgfqpoint{3.900000in}{3.900000in}}%
\pgfusepath{clip}%
\pgfsetbuttcap%
\pgfsetroundjoin%
\definecolor{currentfill}{rgb}{0.121569,0.466667,0.705882}%
\pgfsetfillcolor{currentfill}%
\pgfsetlinewidth{1.003750pt}%
\definecolor{currentstroke}{rgb}{0.121569,0.466667,0.705882}%
\pgfsetstrokecolor{currentstroke}%
\pgfsetdash{}{0pt}%
\pgfpathmoveto{\pgfqpoint{2.174630in}{2.527710in}}%
\pgfpathcurveto{\pgfqpoint{2.185680in}{2.527710in}}{\pgfqpoint{2.196279in}{2.532100in}}{\pgfqpoint{2.204092in}{2.539914in}}%
\pgfpathcurveto{\pgfqpoint{2.211906in}{2.547727in}}{\pgfqpoint{2.216296in}{2.558326in}}{\pgfqpoint{2.216296in}{2.569377in}}%
\pgfpathcurveto{\pgfqpoint{2.216296in}{2.580427in}}{\pgfqpoint{2.211906in}{2.591026in}}{\pgfqpoint{2.204092in}{2.598839in}}%
\pgfpathcurveto{\pgfqpoint{2.196279in}{2.606653in}}{\pgfqpoint{2.185680in}{2.611043in}}{\pgfqpoint{2.174630in}{2.611043in}}%
\pgfpathcurveto{\pgfqpoint{2.163580in}{2.611043in}}{\pgfqpoint{2.152981in}{2.606653in}}{\pgfqpoint{2.145167in}{2.598839in}}%
\pgfpathcurveto{\pgfqpoint{2.137353in}{2.591026in}}{\pgfqpoint{2.132963in}{2.580427in}}{\pgfqpoint{2.132963in}{2.569377in}}%
\pgfpathcurveto{\pgfqpoint{2.132963in}{2.558326in}}{\pgfqpoint{2.137353in}{2.547727in}}{\pgfqpoint{2.145167in}{2.539914in}}%
\pgfpathcurveto{\pgfqpoint{2.152981in}{2.532100in}}{\pgfqpoint{2.163580in}{2.527710in}}{\pgfqpoint{2.174630in}{2.527710in}}%
\pgfpathclose%
\pgfusepath{stroke,fill}%
\end{pgfscope}%
\begin{pgfscope}%
\pgfpathrectangle{\pgfqpoint{0.600000in}{0.600000in}}{\pgfqpoint{3.900000in}{3.900000in}}%
\pgfusepath{clip}%
\pgfsetbuttcap%
\pgfsetroundjoin%
\definecolor{currentfill}{rgb}{0.121569,0.466667,0.705882}%
\pgfsetfillcolor{currentfill}%
\pgfsetlinewidth{1.003750pt}%
\definecolor{currentstroke}{rgb}{0.121569,0.466667,0.705882}%
\pgfsetstrokecolor{currentstroke}%
\pgfsetdash{}{0pt}%
\pgfpathmoveto{\pgfqpoint{2.676482in}{2.382370in}}%
\pgfpathcurveto{\pgfqpoint{2.687532in}{2.382370in}}{\pgfqpoint{2.698131in}{2.386760in}}{\pgfqpoint{2.705945in}{2.394574in}}%
\pgfpathcurveto{\pgfqpoint{2.713758in}{2.402388in}}{\pgfqpoint{2.718148in}{2.412987in}}{\pgfqpoint{2.718148in}{2.424037in}}%
\pgfpathcurveto{\pgfqpoint{2.718148in}{2.435087in}}{\pgfqpoint{2.713758in}{2.445686in}}{\pgfqpoint{2.705945in}{2.453499in}}%
\pgfpathcurveto{\pgfqpoint{2.698131in}{2.461313in}}{\pgfqpoint{2.687532in}{2.465703in}}{\pgfqpoint{2.676482in}{2.465703in}}%
\pgfpathcurveto{\pgfqpoint{2.665432in}{2.465703in}}{\pgfqpoint{2.654833in}{2.461313in}}{\pgfqpoint{2.647019in}{2.453499in}}%
\pgfpathcurveto{\pgfqpoint{2.639205in}{2.445686in}}{\pgfqpoint{2.634815in}{2.435087in}}{\pgfqpoint{2.634815in}{2.424037in}}%
\pgfpathcurveto{\pgfqpoint{2.634815in}{2.412987in}}{\pgfqpoint{2.639205in}{2.402388in}}{\pgfqpoint{2.647019in}{2.394574in}}%
\pgfpathcurveto{\pgfqpoint{2.654833in}{2.386760in}}{\pgfqpoint{2.665432in}{2.382370in}}{\pgfqpoint{2.676482in}{2.382370in}}%
\pgfpathclose%
\pgfusepath{stroke,fill}%
\end{pgfscope}%
\begin{pgfscope}%
\pgfpathrectangle{\pgfqpoint{0.600000in}{0.600000in}}{\pgfqpoint{3.900000in}{3.900000in}}%
\pgfusepath{clip}%
\pgfsetbuttcap%
\pgfsetroundjoin%
\definecolor{currentfill}{rgb}{0.121569,0.466667,0.705882}%
\pgfsetfillcolor{currentfill}%
\pgfsetlinewidth{1.003750pt}%
\definecolor{currentstroke}{rgb}{0.121569,0.466667,0.705882}%
\pgfsetstrokecolor{currentstroke}%
\pgfsetdash{}{0pt}%
\pgfpathmoveto{\pgfqpoint{2.265964in}{2.194176in}}%
\pgfpathcurveto{\pgfqpoint{2.277015in}{2.194176in}}{\pgfqpoint{2.287614in}{2.198566in}}{\pgfqpoint{2.295427in}{2.206380in}}%
\pgfpathcurveto{\pgfqpoint{2.303241in}{2.214193in}}{\pgfqpoint{2.307631in}{2.224792in}}{\pgfqpoint{2.307631in}{2.235843in}}%
\pgfpathcurveto{\pgfqpoint{2.307631in}{2.246893in}}{\pgfqpoint{2.303241in}{2.257492in}}{\pgfqpoint{2.295427in}{2.265305in}}%
\pgfpathcurveto{\pgfqpoint{2.287614in}{2.273119in}}{\pgfqpoint{2.277015in}{2.277509in}}{\pgfqpoint{2.265964in}{2.277509in}}%
\pgfpathcurveto{\pgfqpoint{2.254914in}{2.277509in}}{\pgfqpoint{2.244315in}{2.273119in}}{\pgfqpoint{2.236502in}{2.265305in}}%
\pgfpathcurveto{\pgfqpoint{2.228688in}{2.257492in}}{\pgfqpoint{2.224298in}{2.246893in}}{\pgfqpoint{2.224298in}{2.235843in}}%
\pgfpathcurveto{\pgfqpoint{2.224298in}{2.224792in}}{\pgfqpoint{2.228688in}{2.214193in}}{\pgfqpoint{2.236502in}{2.206380in}}%
\pgfpathcurveto{\pgfqpoint{2.244315in}{2.198566in}}{\pgfqpoint{2.254914in}{2.194176in}}{\pgfqpoint{2.265964in}{2.194176in}}%
\pgfpathclose%
\pgfusepath{stroke,fill}%
\end{pgfscope}%
\begin{pgfscope}%
\pgfpathrectangle{\pgfqpoint{0.600000in}{0.600000in}}{\pgfqpoint{3.900000in}{3.900000in}}%
\pgfusepath{clip}%
\pgfsetbuttcap%
\pgfsetroundjoin%
\definecolor{currentfill}{rgb}{0.121569,0.466667,0.705882}%
\pgfsetfillcolor{currentfill}%
\pgfsetlinewidth{1.003750pt}%
\definecolor{currentstroke}{rgb}{0.121569,0.466667,0.705882}%
\pgfsetstrokecolor{currentstroke}%
\pgfsetdash{}{0pt}%
\pgfpathmoveto{\pgfqpoint{3.389484in}{2.859633in}}%
\pgfpathcurveto{\pgfqpoint{3.400534in}{2.859633in}}{\pgfqpoint{3.411133in}{2.864023in}}{\pgfqpoint{3.418947in}{2.871837in}}%
\pgfpathcurveto{\pgfqpoint{3.426761in}{2.879651in}}{\pgfqpoint{3.431151in}{2.890250in}}{\pgfqpoint{3.431151in}{2.901300in}}%
\pgfpathcurveto{\pgfqpoint{3.431151in}{2.912350in}}{\pgfqpoint{3.426761in}{2.922949in}}{\pgfqpoint{3.418947in}{2.930763in}}%
\pgfpathcurveto{\pgfqpoint{3.411133in}{2.938576in}}{\pgfqpoint{3.400534in}{2.942967in}}{\pgfqpoint{3.389484in}{2.942967in}}%
\pgfpathcurveto{\pgfqpoint{3.378434in}{2.942967in}}{\pgfqpoint{3.367835in}{2.938576in}}{\pgfqpoint{3.360021in}{2.930763in}}%
\pgfpathcurveto{\pgfqpoint{3.352208in}{2.922949in}}{\pgfqpoint{3.347818in}{2.912350in}}{\pgfqpoint{3.347818in}{2.901300in}}%
\pgfpathcurveto{\pgfqpoint{3.347818in}{2.890250in}}{\pgfqpoint{3.352208in}{2.879651in}}{\pgfqpoint{3.360021in}{2.871837in}}%
\pgfpathcurveto{\pgfqpoint{3.367835in}{2.864023in}}{\pgfqpoint{3.378434in}{2.859633in}}{\pgfqpoint{3.389484in}{2.859633in}}%
\pgfpathclose%
\pgfusepath{stroke,fill}%
\end{pgfscope}%
\begin{pgfscope}%
\pgfpathrectangle{\pgfqpoint{0.600000in}{0.600000in}}{\pgfqpoint{3.900000in}{3.900000in}}%
\pgfusepath{clip}%
\pgfsetbuttcap%
\pgfsetroundjoin%
\definecolor{currentfill}{rgb}{0.121569,0.466667,0.705882}%
\pgfsetfillcolor{currentfill}%
\pgfsetlinewidth{1.003750pt}%
\definecolor{currentstroke}{rgb}{0.121569,0.466667,0.705882}%
\pgfsetstrokecolor{currentstroke}%
\pgfsetdash{}{0pt}%
\pgfpathmoveto{\pgfqpoint{2.917358in}{3.063690in}}%
\pgfpathcurveto{\pgfqpoint{2.928408in}{3.063690in}}{\pgfqpoint{2.939007in}{3.068080in}}{\pgfqpoint{2.946821in}{3.075894in}}%
\pgfpathcurveto{\pgfqpoint{2.954634in}{3.083708in}}{\pgfqpoint{2.959025in}{3.094307in}}{\pgfqpoint{2.959025in}{3.105357in}}%
\pgfpathcurveto{\pgfqpoint{2.959025in}{3.116407in}}{\pgfqpoint{2.954634in}{3.127006in}}{\pgfqpoint{2.946821in}{3.134819in}}%
\pgfpathcurveto{\pgfqpoint{2.939007in}{3.142633in}}{\pgfqpoint{2.928408in}{3.147023in}}{\pgfqpoint{2.917358in}{3.147023in}}%
\pgfpathcurveto{\pgfqpoint{2.906308in}{3.147023in}}{\pgfqpoint{2.895709in}{3.142633in}}{\pgfqpoint{2.887895in}{3.134819in}}%
\pgfpathcurveto{\pgfqpoint{2.880082in}{3.127006in}}{\pgfqpoint{2.875691in}{3.116407in}}{\pgfqpoint{2.875691in}{3.105357in}}%
\pgfpathcurveto{\pgfqpoint{2.875691in}{3.094307in}}{\pgfqpoint{2.880082in}{3.083708in}}{\pgfqpoint{2.887895in}{3.075894in}}%
\pgfpathcurveto{\pgfqpoint{2.895709in}{3.068080in}}{\pgfqpoint{2.906308in}{3.063690in}}{\pgfqpoint{2.917358in}{3.063690in}}%
\pgfpathclose%
\pgfusepath{stroke,fill}%
\end{pgfscope}%
\begin{pgfscope}%
\pgfpathrectangle{\pgfqpoint{0.600000in}{0.600000in}}{\pgfqpoint{3.900000in}{3.900000in}}%
\pgfusepath{clip}%
\pgfsetbuttcap%
\pgfsetroundjoin%
\definecolor{currentfill}{rgb}{0.121569,0.466667,0.705882}%
\pgfsetfillcolor{currentfill}%
\pgfsetlinewidth{1.003750pt}%
\definecolor{currentstroke}{rgb}{0.121569,0.466667,0.705882}%
\pgfsetstrokecolor{currentstroke}%
\pgfsetdash{}{0pt}%
\pgfpathmoveto{\pgfqpoint{2.194438in}{2.133738in}}%
\pgfpathcurveto{\pgfqpoint{2.205488in}{2.133738in}}{\pgfqpoint{2.216087in}{2.138129in}}{\pgfqpoint{2.223901in}{2.145942in}}%
\pgfpathcurveto{\pgfqpoint{2.231715in}{2.153756in}}{\pgfqpoint{2.236105in}{2.164355in}}{\pgfqpoint{2.236105in}{2.175405in}}%
\pgfpathcurveto{\pgfqpoint{2.236105in}{2.186455in}}{\pgfqpoint{2.231715in}{2.197054in}}{\pgfqpoint{2.223901in}{2.204868in}}%
\pgfpathcurveto{\pgfqpoint{2.216087in}{2.212681in}}{\pgfqpoint{2.205488in}{2.217072in}}{\pgfqpoint{2.194438in}{2.217072in}}%
\pgfpathcurveto{\pgfqpoint{2.183388in}{2.217072in}}{\pgfqpoint{2.172789in}{2.212681in}}{\pgfqpoint{2.164975in}{2.204868in}}%
\pgfpathcurveto{\pgfqpoint{2.157162in}{2.197054in}}{\pgfqpoint{2.152772in}{2.186455in}}{\pgfqpoint{2.152772in}{2.175405in}}%
\pgfpathcurveto{\pgfqpoint{2.152772in}{2.164355in}}{\pgfqpoint{2.157162in}{2.153756in}}{\pgfqpoint{2.164975in}{2.145942in}}%
\pgfpathcurveto{\pgfqpoint{2.172789in}{2.138129in}}{\pgfqpoint{2.183388in}{2.133738in}}{\pgfqpoint{2.194438in}{2.133738in}}%
\pgfpathclose%
\pgfusepath{stroke,fill}%
\end{pgfscope}%
\begin{pgfscope}%
\pgfpathrectangle{\pgfqpoint{0.600000in}{0.600000in}}{\pgfqpoint{3.900000in}{3.900000in}}%
\pgfusepath{clip}%
\pgfsetbuttcap%
\pgfsetroundjoin%
\definecolor{currentfill}{rgb}{0.121569,0.466667,0.705882}%
\pgfsetfillcolor{currentfill}%
\pgfsetlinewidth{1.003750pt}%
\definecolor{currentstroke}{rgb}{0.121569,0.466667,0.705882}%
\pgfsetstrokecolor{currentstroke}%
\pgfsetdash{}{0pt}%
\pgfpathmoveto{\pgfqpoint{2.324069in}{3.083420in}}%
\pgfpathcurveto{\pgfqpoint{2.335119in}{3.083420in}}{\pgfqpoint{2.345718in}{3.087810in}}{\pgfqpoint{2.353532in}{3.095624in}}%
\pgfpathcurveto{\pgfqpoint{2.361346in}{3.103438in}}{\pgfqpoint{2.365736in}{3.114037in}}{\pgfqpoint{2.365736in}{3.125087in}}%
\pgfpathcurveto{\pgfqpoint{2.365736in}{3.136137in}}{\pgfqpoint{2.361346in}{3.146736in}}{\pgfqpoint{2.353532in}{3.154550in}}%
\pgfpathcurveto{\pgfqpoint{2.345718in}{3.162363in}}{\pgfqpoint{2.335119in}{3.166754in}}{\pgfqpoint{2.324069in}{3.166754in}}%
\pgfpathcurveto{\pgfqpoint{2.313019in}{3.166754in}}{\pgfqpoint{2.302420in}{3.162363in}}{\pgfqpoint{2.294606in}{3.154550in}}%
\pgfpathcurveto{\pgfqpoint{2.286793in}{3.146736in}}{\pgfqpoint{2.282403in}{3.136137in}}{\pgfqpoint{2.282403in}{3.125087in}}%
\pgfpathcurveto{\pgfqpoint{2.282403in}{3.114037in}}{\pgfqpoint{2.286793in}{3.103438in}}{\pgfqpoint{2.294606in}{3.095624in}}%
\pgfpathcurveto{\pgfqpoint{2.302420in}{3.087810in}}{\pgfqpoint{2.313019in}{3.083420in}}{\pgfqpoint{2.324069in}{3.083420in}}%
\pgfpathclose%
\pgfusepath{stroke,fill}%
\end{pgfscope}%
\begin{pgfscope}%
\pgfpathrectangle{\pgfqpoint{0.600000in}{0.600000in}}{\pgfqpoint{3.900000in}{3.900000in}}%
\pgfusepath{clip}%
\pgfsetbuttcap%
\pgfsetroundjoin%
\definecolor{currentfill}{rgb}{0.121569,0.466667,0.705882}%
\pgfsetfillcolor{currentfill}%
\pgfsetlinewidth{1.003750pt}%
\definecolor{currentstroke}{rgb}{0.121569,0.466667,0.705882}%
\pgfsetstrokecolor{currentstroke}%
\pgfsetdash{}{0pt}%
\pgfpathmoveto{\pgfqpoint{1.980778in}{2.697735in}}%
\pgfpathcurveto{\pgfqpoint{1.991828in}{2.697735in}}{\pgfqpoint{2.002427in}{2.702125in}}{\pgfqpoint{2.010241in}{2.709938in}}%
\pgfpathcurveto{\pgfqpoint{2.018054in}{2.717752in}}{\pgfqpoint{2.022445in}{2.728351in}}{\pgfqpoint{2.022445in}{2.739401in}}%
\pgfpathcurveto{\pgfqpoint{2.022445in}{2.750451in}}{\pgfqpoint{2.018054in}{2.761050in}}{\pgfqpoint{2.010241in}{2.768864in}}%
\pgfpathcurveto{\pgfqpoint{2.002427in}{2.776678in}}{\pgfqpoint{1.991828in}{2.781068in}}{\pgfqpoint{1.980778in}{2.781068in}}%
\pgfpathcurveto{\pgfqpoint{1.969728in}{2.781068in}}{\pgfqpoint{1.959129in}{2.776678in}}{\pgfqpoint{1.951315in}{2.768864in}}%
\pgfpathcurveto{\pgfqpoint{1.943502in}{2.761050in}}{\pgfqpoint{1.939111in}{2.750451in}}{\pgfqpoint{1.939111in}{2.739401in}}%
\pgfpathcurveto{\pgfqpoint{1.939111in}{2.728351in}}{\pgfqpoint{1.943502in}{2.717752in}}{\pgfqpoint{1.951315in}{2.709938in}}%
\pgfpathcurveto{\pgfqpoint{1.959129in}{2.702125in}}{\pgfqpoint{1.969728in}{2.697735in}}{\pgfqpoint{1.980778in}{2.697735in}}%
\pgfpathclose%
\pgfusepath{stroke,fill}%
\end{pgfscope}%
\begin{pgfscope}%
\pgfpathrectangle{\pgfqpoint{0.600000in}{0.600000in}}{\pgfqpoint{3.900000in}{3.900000in}}%
\pgfusepath{clip}%
\pgfsetbuttcap%
\pgfsetroundjoin%
\definecolor{currentfill}{rgb}{0.121569,0.466667,0.705882}%
\pgfsetfillcolor{currentfill}%
\pgfsetlinewidth{1.003750pt}%
\definecolor{currentstroke}{rgb}{0.121569,0.466667,0.705882}%
\pgfsetstrokecolor{currentstroke}%
\pgfsetdash{}{0pt}%
\pgfpathmoveto{\pgfqpoint{2.026997in}{2.419048in}}%
\pgfpathcurveto{\pgfqpoint{2.038048in}{2.419048in}}{\pgfqpoint{2.048647in}{2.423438in}}{\pgfqpoint{2.056460in}{2.431252in}}%
\pgfpathcurveto{\pgfqpoint{2.064274in}{2.439066in}}{\pgfqpoint{2.068664in}{2.449665in}}{\pgfqpoint{2.068664in}{2.460715in}}%
\pgfpathcurveto{\pgfqpoint{2.068664in}{2.471765in}}{\pgfqpoint{2.064274in}{2.482364in}}{\pgfqpoint{2.056460in}{2.490177in}}%
\pgfpathcurveto{\pgfqpoint{2.048647in}{2.497991in}}{\pgfqpoint{2.038048in}{2.502381in}}{\pgfqpoint{2.026997in}{2.502381in}}%
\pgfpathcurveto{\pgfqpoint{2.015947in}{2.502381in}}{\pgfqpoint{2.005348in}{2.497991in}}{\pgfqpoint{1.997535in}{2.490177in}}%
\pgfpathcurveto{\pgfqpoint{1.989721in}{2.482364in}}{\pgfqpoint{1.985331in}{2.471765in}}{\pgfqpoint{1.985331in}{2.460715in}}%
\pgfpathcurveto{\pgfqpoint{1.985331in}{2.449665in}}{\pgfqpoint{1.989721in}{2.439066in}}{\pgfqpoint{1.997535in}{2.431252in}}%
\pgfpathcurveto{\pgfqpoint{2.005348in}{2.423438in}}{\pgfqpoint{2.015947in}{2.419048in}}{\pgfqpoint{2.026997in}{2.419048in}}%
\pgfpathclose%
\pgfusepath{stroke,fill}%
\end{pgfscope}%
\begin{pgfscope}%
\pgfpathrectangle{\pgfqpoint{0.600000in}{0.600000in}}{\pgfqpoint{3.900000in}{3.900000in}}%
\pgfusepath{clip}%
\pgfsetbuttcap%
\pgfsetroundjoin%
\definecolor{currentfill}{rgb}{0.121569,0.466667,0.705882}%
\pgfsetfillcolor{currentfill}%
\pgfsetlinewidth{1.003750pt}%
\definecolor{currentstroke}{rgb}{0.121569,0.466667,0.705882}%
\pgfsetstrokecolor{currentstroke}%
\pgfsetdash{}{0pt}%
\pgfpathmoveto{\pgfqpoint{2.510542in}{2.509793in}}%
\pgfpathcurveto{\pgfqpoint{2.521592in}{2.509793in}}{\pgfqpoint{2.532191in}{2.514183in}}{\pgfqpoint{2.540005in}{2.521997in}}%
\pgfpathcurveto{\pgfqpoint{2.547819in}{2.529811in}}{\pgfqpoint{2.552209in}{2.540410in}}{\pgfqpoint{2.552209in}{2.551460in}}%
\pgfpathcurveto{\pgfqpoint{2.552209in}{2.562510in}}{\pgfqpoint{2.547819in}{2.573109in}}{\pgfqpoint{2.540005in}{2.580923in}}%
\pgfpathcurveto{\pgfqpoint{2.532191in}{2.588736in}}{\pgfqpoint{2.521592in}{2.593126in}}{\pgfqpoint{2.510542in}{2.593126in}}%
\pgfpathcurveto{\pgfqpoint{2.499492in}{2.593126in}}{\pgfqpoint{2.488893in}{2.588736in}}{\pgfqpoint{2.481079in}{2.580923in}}%
\pgfpathcurveto{\pgfqpoint{2.473266in}{2.573109in}}{\pgfqpoint{2.468876in}{2.562510in}}{\pgfqpoint{2.468876in}{2.551460in}}%
\pgfpathcurveto{\pgfqpoint{2.468876in}{2.540410in}}{\pgfqpoint{2.473266in}{2.529811in}}{\pgfqpoint{2.481079in}{2.521997in}}%
\pgfpathcurveto{\pgfqpoint{2.488893in}{2.514183in}}{\pgfqpoint{2.499492in}{2.509793in}}{\pgfqpoint{2.510542in}{2.509793in}}%
\pgfpathclose%
\pgfusepath{stroke,fill}%
\end{pgfscope}%
\begin{pgfscope}%
\pgfpathrectangle{\pgfqpoint{0.600000in}{0.600000in}}{\pgfqpoint{3.900000in}{3.900000in}}%
\pgfusepath{clip}%
\pgfsetbuttcap%
\pgfsetroundjoin%
\definecolor{currentfill}{rgb}{0.121569,0.466667,0.705882}%
\pgfsetfillcolor{currentfill}%
\pgfsetlinewidth{1.003750pt}%
\definecolor{currentstroke}{rgb}{0.121569,0.466667,0.705882}%
\pgfsetstrokecolor{currentstroke}%
\pgfsetdash{}{0pt}%
\pgfpathmoveto{\pgfqpoint{2.347178in}{3.491787in}}%
\pgfpathcurveto{\pgfqpoint{2.358229in}{3.491787in}}{\pgfqpoint{2.368828in}{3.496177in}}{\pgfqpoint{2.376641in}{3.503990in}}%
\pgfpathcurveto{\pgfqpoint{2.384455in}{3.511804in}}{\pgfqpoint{2.388845in}{3.522403in}}{\pgfqpoint{2.388845in}{3.533453in}}%
\pgfpathcurveto{\pgfqpoint{2.388845in}{3.544503in}}{\pgfqpoint{2.384455in}{3.555102in}}{\pgfqpoint{2.376641in}{3.562916in}}%
\pgfpathcurveto{\pgfqpoint{2.368828in}{3.570730in}}{\pgfqpoint{2.358229in}{3.575120in}}{\pgfqpoint{2.347178in}{3.575120in}}%
\pgfpathcurveto{\pgfqpoint{2.336128in}{3.575120in}}{\pgfqpoint{2.325529in}{3.570730in}}{\pgfqpoint{2.317716in}{3.562916in}}%
\pgfpathcurveto{\pgfqpoint{2.309902in}{3.555102in}}{\pgfqpoint{2.305512in}{3.544503in}}{\pgfqpoint{2.305512in}{3.533453in}}%
\pgfpathcurveto{\pgfqpoint{2.305512in}{3.522403in}}{\pgfqpoint{2.309902in}{3.511804in}}{\pgfqpoint{2.317716in}{3.503990in}}%
\pgfpathcurveto{\pgfqpoint{2.325529in}{3.496177in}}{\pgfqpoint{2.336128in}{3.491787in}}{\pgfqpoint{2.347178in}{3.491787in}}%
\pgfpathclose%
\pgfusepath{stroke,fill}%
\end{pgfscope}%
\begin{pgfscope}%
\pgfpathrectangle{\pgfqpoint{0.600000in}{0.600000in}}{\pgfqpoint{3.900000in}{3.900000in}}%
\pgfusepath{clip}%
\pgfsetbuttcap%
\pgfsetroundjoin%
\definecolor{currentfill}{rgb}{0.121569,0.466667,0.705882}%
\pgfsetfillcolor{currentfill}%
\pgfsetlinewidth{1.003750pt}%
\definecolor{currentstroke}{rgb}{0.121569,0.466667,0.705882}%
\pgfsetstrokecolor{currentstroke}%
\pgfsetdash{}{0pt}%
\pgfpathmoveto{\pgfqpoint{2.332013in}{2.464084in}}%
\pgfpathcurveto{\pgfqpoint{2.343063in}{2.464084in}}{\pgfqpoint{2.353662in}{2.468474in}}{\pgfqpoint{2.361475in}{2.476288in}}%
\pgfpathcurveto{\pgfqpoint{2.369289in}{2.484102in}}{\pgfqpoint{2.373679in}{2.494701in}}{\pgfqpoint{2.373679in}{2.505751in}}%
\pgfpathcurveto{\pgfqpoint{2.373679in}{2.516801in}}{\pgfqpoint{2.369289in}{2.527400in}}{\pgfqpoint{2.361475in}{2.535214in}}%
\pgfpathcurveto{\pgfqpoint{2.353662in}{2.543027in}}{\pgfqpoint{2.343063in}{2.547418in}}{\pgfqpoint{2.332013in}{2.547418in}}%
\pgfpathcurveto{\pgfqpoint{2.320962in}{2.547418in}}{\pgfqpoint{2.310363in}{2.543027in}}{\pgfqpoint{2.302550in}{2.535214in}}%
\pgfpathcurveto{\pgfqpoint{2.294736in}{2.527400in}}{\pgfqpoint{2.290346in}{2.516801in}}{\pgfqpoint{2.290346in}{2.505751in}}%
\pgfpathcurveto{\pgfqpoint{2.290346in}{2.494701in}}{\pgfqpoint{2.294736in}{2.484102in}}{\pgfqpoint{2.302550in}{2.476288in}}%
\pgfpathcurveto{\pgfqpoint{2.310363in}{2.468474in}}{\pgfqpoint{2.320962in}{2.464084in}}{\pgfqpoint{2.332013in}{2.464084in}}%
\pgfpathclose%
\pgfusepath{stroke,fill}%
\end{pgfscope}%
\begin{pgfscope}%
\pgfpathrectangle{\pgfqpoint{0.600000in}{0.600000in}}{\pgfqpoint{3.900000in}{3.900000in}}%
\pgfusepath{clip}%
\pgfsetbuttcap%
\pgfsetroundjoin%
\definecolor{currentfill}{rgb}{0.121569,0.466667,0.705882}%
\pgfsetfillcolor{currentfill}%
\pgfsetlinewidth{1.003750pt}%
\definecolor{currentstroke}{rgb}{0.121569,0.466667,0.705882}%
\pgfsetstrokecolor{currentstroke}%
\pgfsetdash{}{0pt}%
\pgfpathmoveto{\pgfqpoint{2.566145in}{1.579019in}}%
\pgfpathcurveto{\pgfqpoint{2.577195in}{1.579019in}}{\pgfqpoint{2.587794in}{1.583409in}}{\pgfqpoint{2.595608in}{1.591223in}}%
\pgfpathcurveto{\pgfqpoint{2.603421in}{1.599036in}}{\pgfqpoint{2.607812in}{1.609635in}}{\pgfqpoint{2.607812in}{1.620685in}}%
\pgfpathcurveto{\pgfqpoint{2.607812in}{1.631736in}}{\pgfqpoint{2.603421in}{1.642335in}}{\pgfqpoint{2.595608in}{1.650148in}}%
\pgfpathcurveto{\pgfqpoint{2.587794in}{1.657962in}}{\pgfqpoint{2.577195in}{1.662352in}}{\pgfqpoint{2.566145in}{1.662352in}}%
\pgfpathcurveto{\pgfqpoint{2.555095in}{1.662352in}}{\pgfqpoint{2.544496in}{1.657962in}}{\pgfqpoint{2.536682in}{1.650148in}}%
\pgfpathcurveto{\pgfqpoint{2.528869in}{1.642335in}}{\pgfqpoint{2.524478in}{1.631736in}}{\pgfqpoint{2.524478in}{1.620685in}}%
\pgfpathcurveto{\pgfqpoint{2.524478in}{1.609635in}}{\pgfqpoint{2.528869in}{1.599036in}}{\pgfqpoint{2.536682in}{1.591223in}}%
\pgfpathcurveto{\pgfqpoint{2.544496in}{1.583409in}}{\pgfqpoint{2.555095in}{1.579019in}}{\pgfqpoint{2.566145in}{1.579019in}}%
\pgfpathclose%
\pgfusepath{stroke,fill}%
\end{pgfscope}%
\begin{pgfscope}%
\pgfpathrectangle{\pgfqpoint{0.600000in}{0.600000in}}{\pgfqpoint{3.900000in}{3.900000in}}%
\pgfusepath{clip}%
\pgfsetbuttcap%
\pgfsetroundjoin%
\definecolor{currentfill}{rgb}{0.121569,0.466667,0.705882}%
\pgfsetfillcolor{currentfill}%
\pgfsetlinewidth{1.003750pt}%
\definecolor{currentstroke}{rgb}{0.121569,0.466667,0.705882}%
\pgfsetstrokecolor{currentstroke}%
\pgfsetdash{}{0pt}%
\pgfpathmoveto{\pgfqpoint{2.478698in}{2.756626in}}%
\pgfpathcurveto{\pgfqpoint{2.489748in}{2.756626in}}{\pgfqpoint{2.500347in}{2.761016in}}{\pgfqpoint{2.508161in}{2.768830in}}%
\pgfpathcurveto{\pgfqpoint{2.515974in}{2.776643in}}{\pgfqpoint{2.520364in}{2.787242in}}{\pgfqpoint{2.520364in}{2.798292in}}%
\pgfpathcurveto{\pgfqpoint{2.520364in}{2.809343in}}{\pgfqpoint{2.515974in}{2.819942in}}{\pgfqpoint{2.508161in}{2.827755in}}%
\pgfpathcurveto{\pgfqpoint{2.500347in}{2.835569in}}{\pgfqpoint{2.489748in}{2.839959in}}{\pgfqpoint{2.478698in}{2.839959in}}%
\pgfpathcurveto{\pgfqpoint{2.467648in}{2.839959in}}{\pgfqpoint{2.457049in}{2.835569in}}{\pgfqpoint{2.449235in}{2.827755in}}%
\pgfpathcurveto{\pgfqpoint{2.441421in}{2.819942in}}{\pgfqpoint{2.437031in}{2.809343in}}{\pgfqpoint{2.437031in}{2.798292in}}%
\pgfpathcurveto{\pgfqpoint{2.437031in}{2.787242in}}{\pgfqpoint{2.441421in}{2.776643in}}{\pgfqpoint{2.449235in}{2.768830in}}%
\pgfpathcurveto{\pgfqpoint{2.457049in}{2.761016in}}{\pgfqpoint{2.467648in}{2.756626in}}{\pgfqpoint{2.478698in}{2.756626in}}%
\pgfpathclose%
\pgfusepath{stroke,fill}%
\end{pgfscope}%
\begin{pgfscope}%
\pgfpathrectangle{\pgfqpoint{0.600000in}{0.600000in}}{\pgfqpoint{3.900000in}{3.900000in}}%
\pgfusepath{clip}%
\pgfsetbuttcap%
\pgfsetroundjoin%
\definecolor{currentfill}{rgb}{0.121569,0.466667,0.705882}%
\pgfsetfillcolor{currentfill}%
\pgfsetlinewidth{1.003750pt}%
\definecolor{currentstroke}{rgb}{0.121569,0.466667,0.705882}%
\pgfsetstrokecolor{currentstroke}%
\pgfsetdash{}{0pt}%
\pgfpathmoveto{\pgfqpoint{2.176512in}{2.610620in}}%
\pgfpathcurveto{\pgfqpoint{2.187562in}{2.610620in}}{\pgfqpoint{2.198161in}{2.615011in}}{\pgfqpoint{2.205975in}{2.622824in}}%
\pgfpathcurveto{\pgfqpoint{2.213788in}{2.630638in}}{\pgfqpoint{2.218179in}{2.641237in}}{\pgfqpoint{2.218179in}{2.652287in}}%
\pgfpathcurveto{\pgfqpoint{2.218179in}{2.663337in}}{\pgfqpoint{2.213788in}{2.673936in}}{\pgfqpoint{2.205975in}{2.681750in}}%
\pgfpathcurveto{\pgfqpoint{2.198161in}{2.689564in}}{\pgfqpoint{2.187562in}{2.693954in}}{\pgfqpoint{2.176512in}{2.693954in}}%
\pgfpathcurveto{\pgfqpoint{2.165462in}{2.693954in}}{\pgfqpoint{2.154863in}{2.689564in}}{\pgfqpoint{2.147049in}{2.681750in}}%
\pgfpathcurveto{\pgfqpoint{2.139236in}{2.673936in}}{\pgfqpoint{2.134845in}{2.663337in}}{\pgfqpoint{2.134845in}{2.652287in}}%
\pgfpathcurveto{\pgfqpoint{2.134845in}{2.641237in}}{\pgfqpoint{2.139236in}{2.630638in}}{\pgfqpoint{2.147049in}{2.622824in}}%
\pgfpathcurveto{\pgfqpoint{2.154863in}{2.615011in}}{\pgfqpoint{2.165462in}{2.610620in}}{\pgfqpoint{2.176512in}{2.610620in}}%
\pgfpathclose%
\pgfusepath{stroke,fill}%
\end{pgfscope}%
\begin{pgfscope}%
\pgfpathrectangle{\pgfqpoint{0.600000in}{0.600000in}}{\pgfqpoint{3.900000in}{3.900000in}}%
\pgfusepath{clip}%
\pgfsetbuttcap%
\pgfsetroundjoin%
\definecolor{currentfill}{rgb}{0.121569,0.466667,0.705882}%
\pgfsetfillcolor{currentfill}%
\pgfsetlinewidth{1.003750pt}%
\definecolor{currentstroke}{rgb}{0.121569,0.466667,0.705882}%
\pgfsetstrokecolor{currentstroke}%
\pgfsetdash{}{0pt}%
\pgfpathmoveto{\pgfqpoint{3.151215in}{2.213233in}}%
\pgfpathcurveto{\pgfqpoint{3.162265in}{2.213233in}}{\pgfqpoint{3.172864in}{2.217623in}}{\pgfqpoint{3.180678in}{2.225437in}}%
\pgfpathcurveto{\pgfqpoint{3.188491in}{2.233250in}}{\pgfqpoint{3.192882in}{2.243849in}}{\pgfqpoint{3.192882in}{2.254899in}}%
\pgfpathcurveto{\pgfqpoint{3.192882in}{2.265949in}}{\pgfqpoint{3.188491in}{2.276548in}}{\pgfqpoint{3.180678in}{2.284362in}}%
\pgfpathcurveto{\pgfqpoint{3.172864in}{2.292176in}}{\pgfqpoint{3.162265in}{2.296566in}}{\pgfqpoint{3.151215in}{2.296566in}}%
\pgfpathcurveto{\pgfqpoint{3.140165in}{2.296566in}}{\pgfqpoint{3.129566in}{2.292176in}}{\pgfqpoint{3.121752in}{2.284362in}}%
\pgfpathcurveto{\pgfqpoint{3.113938in}{2.276548in}}{\pgfqpoint{3.109548in}{2.265949in}}{\pgfqpoint{3.109548in}{2.254899in}}%
\pgfpathcurveto{\pgfqpoint{3.109548in}{2.243849in}}{\pgfqpoint{3.113938in}{2.233250in}}{\pgfqpoint{3.121752in}{2.225437in}}%
\pgfpathcurveto{\pgfqpoint{3.129566in}{2.217623in}}{\pgfqpoint{3.140165in}{2.213233in}}{\pgfqpoint{3.151215in}{2.213233in}}%
\pgfpathclose%
\pgfusepath{stroke,fill}%
\end{pgfscope}%
\begin{pgfscope}%
\pgfpathrectangle{\pgfqpoint{0.600000in}{0.600000in}}{\pgfqpoint{3.900000in}{3.900000in}}%
\pgfusepath{clip}%
\pgfsetbuttcap%
\pgfsetroundjoin%
\definecolor{currentfill}{rgb}{0.121569,0.466667,0.705882}%
\pgfsetfillcolor{currentfill}%
\pgfsetlinewidth{1.003750pt}%
\definecolor{currentstroke}{rgb}{0.121569,0.466667,0.705882}%
\pgfsetstrokecolor{currentstroke}%
\pgfsetdash{}{0pt}%
\pgfpathmoveto{\pgfqpoint{2.371308in}{3.234702in}}%
\pgfpathcurveto{\pgfqpoint{2.382359in}{3.234702in}}{\pgfqpoint{2.392958in}{3.239093in}}{\pgfqpoint{2.400771in}{3.246906in}}%
\pgfpathcurveto{\pgfqpoint{2.408585in}{3.254720in}}{\pgfqpoint{2.412975in}{3.265319in}}{\pgfqpoint{2.412975in}{3.276369in}}%
\pgfpathcurveto{\pgfqpoint{2.412975in}{3.287419in}}{\pgfqpoint{2.408585in}{3.298018in}}{\pgfqpoint{2.400771in}{3.305832in}}%
\pgfpathcurveto{\pgfqpoint{2.392958in}{3.313645in}}{\pgfqpoint{2.382359in}{3.318036in}}{\pgfqpoint{2.371308in}{3.318036in}}%
\pgfpathcurveto{\pgfqpoint{2.360258in}{3.318036in}}{\pgfqpoint{2.349659in}{3.313645in}}{\pgfqpoint{2.341846in}{3.305832in}}%
\pgfpathcurveto{\pgfqpoint{2.334032in}{3.298018in}}{\pgfqpoint{2.329642in}{3.287419in}}{\pgfqpoint{2.329642in}{3.276369in}}%
\pgfpathcurveto{\pgfqpoint{2.329642in}{3.265319in}}{\pgfqpoint{2.334032in}{3.254720in}}{\pgfqpoint{2.341846in}{3.246906in}}%
\pgfpathcurveto{\pgfqpoint{2.349659in}{3.239093in}}{\pgfqpoint{2.360258in}{3.234702in}}{\pgfqpoint{2.371308in}{3.234702in}}%
\pgfpathclose%
\pgfusepath{stroke,fill}%
\end{pgfscope}%
\begin{pgfscope}%
\pgfpathrectangle{\pgfqpoint{0.600000in}{0.600000in}}{\pgfqpoint{3.900000in}{3.900000in}}%
\pgfusepath{clip}%
\pgfsetbuttcap%
\pgfsetroundjoin%
\definecolor{currentfill}{rgb}{0.121569,0.466667,0.705882}%
\pgfsetfillcolor{currentfill}%
\pgfsetlinewidth{1.003750pt}%
\definecolor{currentstroke}{rgb}{0.121569,0.466667,0.705882}%
\pgfsetstrokecolor{currentstroke}%
\pgfsetdash{}{0pt}%
\pgfpathmoveto{\pgfqpoint{2.024668in}{2.554704in}}%
\pgfpathcurveto{\pgfqpoint{2.035718in}{2.554704in}}{\pgfqpoint{2.046317in}{2.559094in}}{\pgfqpoint{2.054130in}{2.566908in}}%
\pgfpathcurveto{\pgfqpoint{2.061944in}{2.574722in}}{\pgfqpoint{2.066334in}{2.585321in}}{\pgfqpoint{2.066334in}{2.596371in}}%
\pgfpathcurveto{\pgfqpoint{2.066334in}{2.607421in}}{\pgfqpoint{2.061944in}{2.618020in}}{\pgfqpoint{2.054130in}{2.625833in}}%
\pgfpathcurveto{\pgfqpoint{2.046317in}{2.633647in}}{\pgfqpoint{2.035718in}{2.638037in}}{\pgfqpoint{2.024668in}{2.638037in}}%
\pgfpathcurveto{\pgfqpoint{2.013617in}{2.638037in}}{\pgfqpoint{2.003018in}{2.633647in}}{\pgfqpoint{1.995205in}{2.625833in}}%
\pgfpathcurveto{\pgfqpoint{1.987391in}{2.618020in}}{\pgfqpoint{1.983001in}{2.607421in}}{\pgfqpoint{1.983001in}{2.596371in}}%
\pgfpathcurveto{\pgfqpoint{1.983001in}{2.585321in}}{\pgfqpoint{1.987391in}{2.574722in}}{\pgfqpoint{1.995205in}{2.566908in}}%
\pgfpathcurveto{\pgfqpoint{2.003018in}{2.559094in}}{\pgfqpoint{2.013617in}{2.554704in}}{\pgfqpoint{2.024668in}{2.554704in}}%
\pgfpathclose%
\pgfusepath{stroke,fill}%
\end{pgfscope}%
\begin{pgfscope}%
\pgfpathrectangle{\pgfqpoint{0.600000in}{0.600000in}}{\pgfqpoint{3.900000in}{3.900000in}}%
\pgfusepath{clip}%
\pgfsetbuttcap%
\pgfsetroundjoin%
\definecolor{currentfill}{rgb}{0.121569,0.466667,0.705882}%
\pgfsetfillcolor{currentfill}%
\pgfsetlinewidth{1.003750pt}%
\definecolor{currentstroke}{rgb}{0.121569,0.466667,0.705882}%
\pgfsetstrokecolor{currentstroke}%
\pgfsetdash{}{0pt}%
\pgfpathmoveto{\pgfqpoint{1.843518in}{2.828345in}}%
\pgfpathcurveto{\pgfqpoint{1.854568in}{2.828345in}}{\pgfqpoint{1.865167in}{2.832736in}}{\pgfqpoint{1.872980in}{2.840549in}}%
\pgfpathcurveto{\pgfqpoint{1.880794in}{2.848363in}}{\pgfqpoint{1.885184in}{2.858962in}}{\pgfqpoint{1.885184in}{2.870012in}}%
\pgfpathcurveto{\pgfqpoint{1.885184in}{2.881062in}}{\pgfqpoint{1.880794in}{2.891661in}}{\pgfqpoint{1.872980in}{2.899475in}}%
\pgfpathcurveto{\pgfqpoint{1.865167in}{2.907288in}}{\pgfqpoint{1.854568in}{2.911679in}}{\pgfqpoint{1.843518in}{2.911679in}}%
\pgfpathcurveto{\pgfqpoint{1.832467in}{2.911679in}}{\pgfqpoint{1.821868in}{2.907288in}}{\pgfqpoint{1.814055in}{2.899475in}}%
\pgfpathcurveto{\pgfqpoint{1.806241in}{2.891661in}}{\pgfqpoint{1.801851in}{2.881062in}}{\pgfqpoint{1.801851in}{2.870012in}}%
\pgfpathcurveto{\pgfqpoint{1.801851in}{2.858962in}}{\pgfqpoint{1.806241in}{2.848363in}}{\pgfqpoint{1.814055in}{2.840549in}}%
\pgfpathcurveto{\pgfqpoint{1.821868in}{2.832736in}}{\pgfqpoint{1.832467in}{2.828345in}}{\pgfqpoint{1.843518in}{2.828345in}}%
\pgfpathclose%
\pgfusepath{stroke,fill}%
\end{pgfscope}%
\begin{pgfscope}%
\pgfpathrectangle{\pgfqpoint{0.600000in}{0.600000in}}{\pgfqpoint{3.900000in}{3.900000in}}%
\pgfusepath{clip}%
\pgfsetbuttcap%
\pgfsetroundjoin%
\definecolor{currentfill}{rgb}{0.121569,0.466667,0.705882}%
\pgfsetfillcolor{currentfill}%
\pgfsetlinewidth{1.003750pt}%
\definecolor{currentstroke}{rgb}{0.121569,0.466667,0.705882}%
\pgfsetstrokecolor{currentstroke}%
\pgfsetdash{}{0pt}%
\pgfpathmoveto{\pgfqpoint{2.830396in}{2.187968in}}%
\pgfpathcurveto{\pgfqpoint{2.841446in}{2.187968in}}{\pgfqpoint{2.852045in}{2.192358in}}{\pgfqpoint{2.859858in}{2.200172in}}%
\pgfpathcurveto{\pgfqpoint{2.867672in}{2.207985in}}{\pgfqpoint{2.872062in}{2.218584in}}{\pgfqpoint{2.872062in}{2.229635in}}%
\pgfpathcurveto{\pgfqpoint{2.872062in}{2.240685in}}{\pgfqpoint{2.867672in}{2.251284in}}{\pgfqpoint{2.859858in}{2.259097in}}%
\pgfpathcurveto{\pgfqpoint{2.852045in}{2.266911in}}{\pgfqpoint{2.841446in}{2.271301in}}{\pgfqpoint{2.830396in}{2.271301in}}%
\pgfpathcurveto{\pgfqpoint{2.819345in}{2.271301in}}{\pgfqpoint{2.808746in}{2.266911in}}{\pgfqpoint{2.800933in}{2.259097in}}%
\pgfpathcurveto{\pgfqpoint{2.793119in}{2.251284in}}{\pgfqpoint{2.788729in}{2.240685in}}{\pgfqpoint{2.788729in}{2.229635in}}%
\pgfpathcurveto{\pgfqpoint{2.788729in}{2.218584in}}{\pgfqpoint{2.793119in}{2.207985in}}{\pgfqpoint{2.800933in}{2.200172in}}%
\pgfpathcurveto{\pgfqpoint{2.808746in}{2.192358in}}{\pgfqpoint{2.819345in}{2.187968in}}{\pgfqpoint{2.830396in}{2.187968in}}%
\pgfpathclose%
\pgfusepath{stroke,fill}%
\end{pgfscope}%
\begin{pgfscope}%
\pgfpathrectangle{\pgfqpoint{0.600000in}{0.600000in}}{\pgfqpoint{3.900000in}{3.900000in}}%
\pgfusepath{clip}%
\pgfsetbuttcap%
\pgfsetroundjoin%
\definecolor{currentfill}{rgb}{0.121569,0.466667,0.705882}%
\pgfsetfillcolor{currentfill}%
\pgfsetlinewidth{1.003750pt}%
\definecolor{currentstroke}{rgb}{0.121569,0.466667,0.705882}%
\pgfsetstrokecolor{currentstroke}%
\pgfsetdash{}{0pt}%
\pgfpathmoveto{\pgfqpoint{2.541448in}{2.892067in}}%
\pgfpathcurveto{\pgfqpoint{2.552498in}{2.892067in}}{\pgfqpoint{2.563097in}{2.896457in}}{\pgfqpoint{2.570910in}{2.904271in}}%
\pgfpathcurveto{\pgfqpoint{2.578724in}{2.912084in}}{\pgfqpoint{2.583114in}{2.922684in}}{\pgfqpoint{2.583114in}{2.933734in}}%
\pgfpathcurveto{\pgfqpoint{2.583114in}{2.944784in}}{\pgfqpoint{2.578724in}{2.955383in}}{\pgfqpoint{2.570910in}{2.963196in}}%
\pgfpathcurveto{\pgfqpoint{2.563097in}{2.971010in}}{\pgfqpoint{2.552498in}{2.975400in}}{\pgfqpoint{2.541448in}{2.975400in}}%
\pgfpathcurveto{\pgfqpoint{2.530397in}{2.975400in}}{\pgfqpoint{2.519798in}{2.971010in}}{\pgfqpoint{2.511985in}{2.963196in}}%
\pgfpathcurveto{\pgfqpoint{2.504171in}{2.955383in}}{\pgfqpoint{2.499781in}{2.944784in}}{\pgfqpoint{2.499781in}{2.933734in}}%
\pgfpathcurveto{\pgfqpoint{2.499781in}{2.922684in}}{\pgfqpoint{2.504171in}{2.912084in}}{\pgfqpoint{2.511985in}{2.904271in}}%
\pgfpathcurveto{\pgfqpoint{2.519798in}{2.896457in}}{\pgfqpoint{2.530397in}{2.892067in}}{\pgfqpoint{2.541448in}{2.892067in}}%
\pgfpathclose%
\pgfusepath{stroke,fill}%
\end{pgfscope}%
\begin{pgfscope}%
\pgfpathrectangle{\pgfqpoint{0.600000in}{0.600000in}}{\pgfqpoint{3.900000in}{3.900000in}}%
\pgfusepath{clip}%
\pgfsetbuttcap%
\pgfsetroundjoin%
\definecolor{currentfill}{rgb}{0.121569,0.466667,0.705882}%
\pgfsetfillcolor{currentfill}%
\pgfsetlinewidth{1.003750pt}%
\definecolor{currentstroke}{rgb}{0.121569,0.466667,0.705882}%
\pgfsetstrokecolor{currentstroke}%
\pgfsetdash{}{0pt}%
\pgfpathmoveto{\pgfqpoint{1.833908in}{1.964824in}}%
\pgfpathcurveto{\pgfqpoint{1.844959in}{1.964824in}}{\pgfqpoint{1.855558in}{1.969214in}}{\pgfqpoint{1.863371in}{1.977028in}}%
\pgfpathcurveto{\pgfqpoint{1.871185in}{1.984841in}}{\pgfqpoint{1.875575in}{1.995441in}}{\pgfqpoint{1.875575in}{2.006491in}}%
\pgfpathcurveto{\pgfqpoint{1.875575in}{2.017541in}}{\pgfqpoint{1.871185in}{2.028140in}}{\pgfqpoint{1.863371in}{2.035953in}}%
\pgfpathcurveto{\pgfqpoint{1.855558in}{2.043767in}}{\pgfqpoint{1.844959in}{2.048157in}}{\pgfqpoint{1.833908in}{2.048157in}}%
\pgfpathcurveto{\pgfqpoint{1.822858in}{2.048157in}}{\pgfqpoint{1.812259in}{2.043767in}}{\pgfqpoint{1.804446in}{2.035953in}}%
\pgfpathcurveto{\pgfqpoint{1.796632in}{2.028140in}}{\pgfqpoint{1.792242in}{2.017541in}}{\pgfqpoint{1.792242in}{2.006491in}}%
\pgfpathcurveto{\pgfqpoint{1.792242in}{1.995441in}}{\pgfqpoint{1.796632in}{1.984841in}}{\pgfqpoint{1.804446in}{1.977028in}}%
\pgfpathcurveto{\pgfqpoint{1.812259in}{1.969214in}}{\pgfqpoint{1.822858in}{1.964824in}}{\pgfqpoint{1.833908in}{1.964824in}}%
\pgfpathclose%
\pgfusepath{stroke,fill}%
\end{pgfscope}%
\begin{pgfscope}%
\pgfpathrectangle{\pgfqpoint{0.600000in}{0.600000in}}{\pgfqpoint{3.900000in}{3.900000in}}%
\pgfusepath{clip}%
\pgfsetbuttcap%
\pgfsetroundjoin%
\definecolor{currentfill}{rgb}{0.121569,0.466667,0.705882}%
\pgfsetfillcolor{currentfill}%
\pgfsetlinewidth{1.003750pt}%
\definecolor{currentstroke}{rgb}{0.121569,0.466667,0.705882}%
\pgfsetstrokecolor{currentstroke}%
\pgfsetdash{}{0pt}%
\pgfpathmoveto{\pgfqpoint{2.137793in}{2.376362in}}%
\pgfpathcurveto{\pgfqpoint{2.148843in}{2.376362in}}{\pgfqpoint{2.159442in}{2.380752in}}{\pgfqpoint{2.167256in}{2.388565in}}%
\pgfpathcurveto{\pgfqpoint{2.175069in}{2.396379in}}{\pgfqpoint{2.179459in}{2.406978in}}{\pgfqpoint{2.179459in}{2.418028in}}%
\pgfpathcurveto{\pgfqpoint{2.179459in}{2.429078in}}{\pgfqpoint{2.175069in}{2.439677in}}{\pgfqpoint{2.167256in}{2.447491in}}%
\pgfpathcurveto{\pgfqpoint{2.159442in}{2.455305in}}{\pgfqpoint{2.148843in}{2.459695in}}{\pgfqpoint{2.137793in}{2.459695in}}%
\pgfpathcurveto{\pgfqpoint{2.126743in}{2.459695in}}{\pgfqpoint{2.116144in}{2.455305in}}{\pgfqpoint{2.108330in}{2.447491in}}%
\pgfpathcurveto{\pgfqpoint{2.100516in}{2.439677in}}{\pgfqpoint{2.096126in}{2.429078in}}{\pgfqpoint{2.096126in}{2.418028in}}%
\pgfpathcurveto{\pgfqpoint{2.096126in}{2.406978in}}{\pgfqpoint{2.100516in}{2.396379in}}{\pgfqpoint{2.108330in}{2.388565in}}%
\pgfpathcurveto{\pgfqpoint{2.116144in}{2.380752in}}{\pgfqpoint{2.126743in}{2.376362in}}{\pgfqpoint{2.137793in}{2.376362in}}%
\pgfpathclose%
\pgfusepath{stroke,fill}%
\end{pgfscope}%
\begin{pgfscope}%
\pgfpathrectangle{\pgfqpoint{0.600000in}{0.600000in}}{\pgfqpoint{3.900000in}{3.900000in}}%
\pgfusepath{clip}%
\pgfsetbuttcap%
\pgfsetroundjoin%
\definecolor{currentfill}{rgb}{0.121569,0.466667,0.705882}%
\pgfsetfillcolor{currentfill}%
\pgfsetlinewidth{1.003750pt}%
\definecolor{currentstroke}{rgb}{0.121569,0.466667,0.705882}%
\pgfsetstrokecolor{currentstroke}%
\pgfsetdash{}{0pt}%
\pgfpathmoveto{\pgfqpoint{2.701134in}{2.708943in}}%
\pgfpathcurveto{\pgfqpoint{2.712184in}{2.708943in}}{\pgfqpoint{2.722783in}{2.713334in}}{\pgfqpoint{2.730597in}{2.721147in}}%
\pgfpathcurveto{\pgfqpoint{2.738411in}{2.728961in}}{\pgfqpoint{2.742801in}{2.739560in}}{\pgfqpoint{2.742801in}{2.750610in}}%
\pgfpathcurveto{\pgfqpoint{2.742801in}{2.761660in}}{\pgfqpoint{2.738411in}{2.772259in}}{\pgfqpoint{2.730597in}{2.780073in}}%
\pgfpathcurveto{\pgfqpoint{2.722783in}{2.787886in}}{\pgfqpoint{2.712184in}{2.792277in}}{\pgfqpoint{2.701134in}{2.792277in}}%
\pgfpathcurveto{\pgfqpoint{2.690084in}{2.792277in}}{\pgfqpoint{2.679485in}{2.787886in}}{\pgfqpoint{2.671671in}{2.780073in}}%
\pgfpathcurveto{\pgfqpoint{2.663858in}{2.772259in}}{\pgfqpoint{2.659468in}{2.761660in}}{\pgfqpoint{2.659468in}{2.750610in}}%
\pgfpathcurveto{\pgfqpoint{2.659468in}{2.739560in}}{\pgfqpoint{2.663858in}{2.728961in}}{\pgfqpoint{2.671671in}{2.721147in}}%
\pgfpathcurveto{\pgfqpoint{2.679485in}{2.713334in}}{\pgfqpoint{2.690084in}{2.708943in}}{\pgfqpoint{2.701134in}{2.708943in}}%
\pgfpathclose%
\pgfusepath{stroke,fill}%
\end{pgfscope}%
\begin{pgfscope}%
\pgfpathrectangle{\pgfqpoint{0.600000in}{0.600000in}}{\pgfqpoint{3.900000in}{3.900000in}}%
\pgfusepath{clip}%
\pgfsetbuttcap%
\pgfsetroundjoin%
\definecolor{currentfill}{rgb}{0.121569,0.466667,0.705882}%
\pgfsetfillcolor{currentfill}%
\pgfsetlinewidth{1.003750pt}%
\definecolor{currentstroke}{rgb}{0.121569,0.466667,0.705882}%
\pgfsetstrokecolor{currentstroke}%
\pgfsetdash{}{0pt}%
\pgfpathmoveto{\pgfqpoint{2.358292in}{2.207533in}}%
\pgfpathcurveto{\pgfqpoint{2.369342in}{2.207533in}}{\pgfqpoint{2.379941in}{2.211923in}}{\pgfqpoint{2.387754in}{2.219737in}}%
\pgfpathcurveto{\pgfqpoint{2.395568in}{2.227550in}}{\pgfqpoint{2.399958in}{2.238149in}}{\pgfqpoint{2.399958in}{2.249199in}}%
\pgfpathcurveto{\pgfqpoint{2.399958in}{2.260250in}}{\pgfqpoint{2.395568in}{2.270849in}}{\pgfqpoint{2.387754in}{2.278662in}}%
\pgfpathcurveto{\pgfqpoint{2.379941in}{2.286476in}}{\pgfqpoint{2.369342in}{2.290866in}}{\pgfqpoint{2.358292in}{2.290866in}}%
\pgfpathcurveto{\pgfqpoint{2.347242in}{2.290866in}}{\pgfqpoint{2.336642in}{2.286476in}}{\pgfqpoint{2.328829in}{2.278662in}}%
\pgfpathcurveto{\pgfqpoint{2.321015in}{2.270849in}}{\pgfqpoint{2.316625in}{2.260250in}}{\pgfqpoint{2.316625in}{2.249199in}}%
\pgfpathcurveto{\pgfqpoint{2.316625in}{2.238149in}}{\pgfqpoint{2.321015in}{2.227550in}}{\pgfqpoint{2.328829in}{2.219737in}}%
\pgfpathcurveto{\pgfqpoint{2.336642in}{2.211923in}}{\pgfqpoint{2.347242in}{2.207533in}}{\pgfqpoint{2.358292in}{2.207533in}}%
\pgfpathclose%
\pgfusepath{stroke,fill}%
\end{pgfscope}%
\begin{pgfscope}%
\pgfpathrectangle{\pgfqpoint{0.600000in}{0.600000in}}{\pgfqpoint{3.900000in}{3.900000in}}%
\pgfusepath{clip}%
\pgfsetbuttcap%
\pgfsetroundjoin%
\definecolor{currentfill}{rgb}{0.121569,0.466667,0.705882}%
\pgfsetfillcolor{currentfill}%
\pgfsetlinewidth{1.003750pt}%
\definecolor{currentstroke}{rgb}{0.121569,0.466667,0.705882}%
\pgfsetstrokecolor{currentstroke}%
\pgfsetdash{}{0pt}%
\pgfpathmoveto{\pgfqpoint{2.820690in}{1.439367in}}%
\pgfpathcurveto{\pgfqpoint{2.831740in}{1.439367in}}{\pgfqpoint{2.842339in}{1.443757in}}{\pgfqpoint{2.850152in}{1.451571in}}%
\pgfpathcurveto{\pgfqpoint{2.857966in}{1.459385in}}{\pgfqpoint{2.862356in}{1.469984in}}{\pgfqpoint{2.862356in}{1.481034in}}%
\pgfpathcurveto{\pgfqpoint{2.862356in}{1.492084in}}{\pgfqpoint{2.857966in}{1.502683in}}{\pgfqpoint{2.850152in}{1.510496in}}%
\pgfpathcurveto{\pgfqpoint{2.842339in}{1.518310in}}{\pgfqpoint{2.831740in}{1.522700in}}{\pgfqpoint{2.820690in}{1.522700in}}%
\pgfpathcurveto{\pgfqpoint{2.809639in}{1.522700in}}{\pgfqpoint{2.799040in}{1.518310in}}{\pgfqpoint{2.791227in}{1.510496in}}%
\pgfpathcurveto{\pgfqpoint{2.783413in}{1.502683in}}{\pgfqpoint{2.779023in}{1.492084in}}{\pgfqpoint{2.779023in}{1.481034in}}%
\pgfpathcurveto{\pgfqpoint{2.779023in}{1.469984in}}{\pgfqpoint{2.783413in}{1.459385in}}{\pgfqpoint{2.791227in}{1.451571in}}%
\pgfpathcurveto{\pgfqpoint{2.799040in}{1.443757in}}{\pgfqpoint{2.809639in}{1.439367in}}{\pgfqpoint{2.820690in}{1.439367in}}%
\pgfpathclose%
\pgfusepath{stroke,fill}%
\end{pgfscope}%
\begin{pgfscope}%
\pgfpathrectangle{\pgfqpoint{0.600000in}{0.600000in}}{\pgfqpoint{3.900000in}{3.900000in}}%
\pgfusepath{clip}%
\pgfsetbuttcap%
\pgfsetroundjoin%
\definecolor{currentfill}{rgb}{0.121569,0.466667,0.705882}%
\pgfsetfillcolor{currentfill}%
\pgfsetlinewidth{1.003750pt}%
\definecolor{currentstroke}{rgb}{0.121569,0.466667,0.705882}%
\pgfsetstrokecolor{currentstroke}%
\pgfsetdash{}{0pt}%
\pgfpathmoveto{\pgfqpoint{1.851422in}{1.715831in}}%
\pgfpathcurveto{\pgfqpoint{1.862472in}{1.715831in}}{\pgfqpoint{1.873071in}{1.720221in}}{\pgfqpoint{1.880884in}{1.728035in}}%
\pgfpathcurveto{\pgfqpoint{1.888698in}{1.735848in}}{\pgfqpoint{1.893088in}{1.746447in}}{\pgfqpoint{1.893088in}{1.757497in}}%
\pgfpathcurveto{\pgfqpoint{1.893088in}{1.768548in}}{\pgfqpoint{1.888698in}{1.779147in}}{\pgfqpoint{1.880884in}{1.786960in}}%
\pgfpathcurveto{\pgfqpoint{1.873071in}{1.794774in}}{\pgfqpoint{1.862472in}{1.799164in}}{\pgfqpoint{1.851422in}{1.799164in}}%
\pgfpathcurveto{\pgfqpoint{1.840371in}{1.799164in}}{\pgfqpoint{1.829772in}{1.794774in}}{\pgfqpoint{1.821959in}{1.786960in}}%
\pgfpathcurveto{\pgfqpoint{1.814145in}{1.779147in}}{\pgfqpoint{1.809755in}{1.768548in}}{\pgfqpoint{1.809755in}{1.757497in}}%
\pgfpathcurveto{\pgfqpoint{1.809755in}{1.746447in}}{\pgfqpoint{1.814145in}{1.735848in}}{\pgfqpoint{1.821959in}{1.728035in}}%
\pgfpathcurveto{\pgfqpoint{1.829772in}{1.720221in}}{\pgfqpoint{1.840371in}{1.715831in}}{\pgfqpoint{1.851422in}{1.715831in}}%
\pgfpathclose%
\pgfusepath{stroke,fill}%
\end{pgfscope}%
\begin{pgfscope}%
\pgfpathrectangle{\pgfqpoint{0.600000in}{0.600000in}}{\pgfqpoint{3.900000in}{3.900000in}}%
\pgfusepath{clip}%
\pgfsetbuttcap%
\pgfsetroundjoin%
\definecolor{currentfill}{rgb}{0.121569,0.466667,0.705882}%
\pgfsetfillcolor{currentfill}%
\pgfsetlinewidth{1.003750pt}%
\definecolor{currentstroke}{rgb}{0.121569,0.466667,0.705882}%
\pgfsetstrokecolor{currentstroke}%
\pgfsetdash{}{0pt}%
\pgfpathmoveto{\pgfqpoint{2.449684in}{3.403705in}}%
\pgfpathcurveto{\pgfqpoint{2.460734in}{3.403705in}}{\pgfqpoint{2.471333in}{3.408095in}}{\pgfqpoint{2.479147in}{3.415909in}}%
\pgfpathcurveto{\pgfqpoint{2.486960in}{3.423722in}}{\pgfqpoint{2.491351in}{3.434322in}}{\pgfqpoint{2.491351in}{3.445372in}}%
\pgfpathcurveto{\pgfqpoint{2.491351in}{3.456422in}}{\pgfqpoint{2.486960in}{3.467021in}}{\pgfqpoint{2.479147in}{3.474834in}}%
\pgfpathcurveto{\pgfqpoint{2.471333in}{3.482648in}}{\pgfqpoint{2.460734in}{3.487038in}}{\pgfqpoint{2.449684in}{3.487038in}}%
\pgfpathcurveto{\pgfqpoint{2.438634in}{3.487038in}}{\pgfqpoint{2.428035in}{3.482648in}}{\pgfqpoint{2.420221in}{3.474834in}}%
\pgfpathcurveto{\pgfqpoint{2.412408in}{3.467021in}}{\pgfqpoint{2.408017in}{3.456422in}}{\pgfqpoint{2.408017in}{3.445372in}}%
\pgfpathcurveto{\pgfqpoint{2.408017in}{3.434322in}}{\pgfqpoint{2.412408in}{3.423722in}}{\pgfqpoint{2.420221in}{3.415909in}}%
\pgfpathcurveto{\pgfqpoint{2.428035in}{3.408095in}}{\pgfqpoint{2.438634in}{3.403705in}}{\pgfqpoint{2.449684in}{3.403705in}}%
\pgfpathclose%
\pgfusepath{stroke,fill}%
\end{pgfscope}%
\begin{pgfscope}%
\pgfpathrectangle{\pgfqpoint{0.600000in}{0.600000in}}{\pgfqpoint{3.900000in}{3.900000in}}%
\pgfusepath{clip}%
\pgfsetbuttcap%
\pgfsetroundjoin%
\definecolor{currentfill}{rgb}{0.121569,0.466667,0.705882}%
\pgfsetfillcolor{currentfill}%
\pgfsetlinewidth{1.003750pt}%
\definecolor{currentstroke}{rgb}{0.121569,0.466667,0.705882}%
\pgfsetstrokecolor{currentstroke}%
\pgfsetdash{}{0pt}%
\pgfpathmoveto{\pgfqpoint{2.945735in}{2.504478in}}%
\pgfpathcurveto{\pgfqpoint{2.956785in}{2.504478in}}{\pgfqpoint{2.967384in}{2.508869in}}{\pgfqpoint{2.975198in}{2.516682in}}%
\pgfpathcurveto{\pgfqpoint{2.983012in}{2.524496in}}{\pgfqpoint{2.987402in}{2.535095in}}{\pgfqpoint{2.987402in}{2.546145in}}%
\pgfpathcurveto{\pgfqpoint{2.987402in}{2.557195in}}{\pgfqpoint{2.983012in}{2.567794in}}{\pgfqpoint{2.975198in}{2.575608in}}%
\pgfpathcurveto{\pgfqpoint{2.967384in}{2.583421in}}{\pgfqpoint{2.956785in}{2.587812in}}{\pgfqpoint{2.945735in}{2.587812in}}%
\pgfpathcurveto{\pgfqpoint{2.934685in}{2.587812in}}{\pgfqpoint{2.924086in}{2.583421in}}{\pgfqpoint{2.916272in}{2.575608in}}%
\pgfpathcurveto{\pgfqpoint{2.908459in}{2.567794in}}{\pgfqpoint{2.904068in}{2.557195in}}{\pgfqpoint{2.904068in}{2.546145in}}%
\pgfpathcurveto{\pgfqpoint{2.904068in}{2.535095in}}{\pgfqpoint{2.908459in}{2.524496in}}{\pgfqpoint{2.916272in}{2.516682in}}%
\pgfpathcurveto{\pgfqpoint{2.924086in}{2.508869in}}{\pgfqpoint{2.934685in}{2.504478in}}{\pgfqpoint{2.945735in}{2.504478in}}%
\pgfpathclose%
\pgfusepath{stroke,fill}%
\end{pgfscope}%
\begin{pgfscope}%
\pgfpathrectangle{\pgfqpoint{0.600000in}{0.600000in}}{\pgfqpoint{3.900000in}{3.900000in}}%
\pgfusepath{clip}%
\pgfsetbuttcap%
\pgfsetroundjoin%
\definecolor{currentfill}{rgb}{0.121569,0.466667,0.705882}%
\pgfsetfillcolor{currentfill}%
\pgfsetlinewidth{1.003750pt}%
\definecolor{currentstroke}{rgb}{0.121569,0.466667,0.705882}%
\pgfsetstrokecolor{currentstroke}%
\pgfsetdash{}{0pt}%
\pgfpathmoveto{\pgfqpoint{2.532023in}{2.814718in}}%
\pgfpathcurveto{\pgfqpoint{2.543073in}{2.814718in}}{\pgfqpoint{2.553672in}{2.819108in}}{\pgfqpoint{2.561485in}{2.826922in}}%
\pgfpathcurveto{\pgfqpoint{2.569299in}{2.834735in}}{\pgfqpoint{2.573689in}{2.845334in}}{\pgfqpoint{2.573689in}{2.856384in}}%
\pgfpathcurveto{\pgfqpoint{2.573689in}{2.867435in}}{\pgfqpoint{2.569299in}{2.878034in}}{\pgfqpoint{2.561485in}{2.885847in}}%
\pgfpathcurveto{\pgfqpoint{2.553672in}{2.893661in}}{\pgfqpoint{2.543073in}{2.898051in}}{\pgfqpoint{2.532023in}{2.898051in}}%
\pgfpathcurveto{\pgfqpoint{2.520972in}{2.898051in}}{\pgfqpoint{2.510373in}{2.893661in}}{\pgfqpoint{2.502560in}{2.885847in}}%
\pgfpathcurveto{\pgfqpoint{2.494746in}{2.878034in}}{\pgfqpoint{2.490356in}{2.867435in}}{\pgfqpoint{2.490356in}{2.856384in}}%
\pgfpathcurveto{\pgfqpoint{2.490356in}{2.845334in}}{\pgfqpoint{2.494746in}{2.834735in}}{\pgfqpoint{2.502560in}{2.826922in}}%
\pgfpathcurveto{\pgfqpoint{2.510373in}{2.819108in}}{\pgfqpoint{2.520972in}{2.814718in}}{\pgfqpoint{2.532023in}{2.814718in}}%
\pgfpathclose%
\pgfusepath{stroke,fill}%
\end{pgfscope}%
\begin{pgfscope}%
\pgfpathrectangle{\pgfqpoint{0.600000in}{0.600000in}}{\pgfqpoint{3.900000in}{3.900000in}}%
\pgfusepath{clip}%
\pgfsetbuttcap%
\pgfsetroundjoin%
\definecolor{currentfill}{rgb}{0.121569,0.466667,0.705882}%
\pgfsetfillcolor{currentfill}%
\pgfsetlinewidth{1.003750pt}%
\definecolor{currentstroke}{rgb}{0.121569,0.466667,0.705882}%
\pgfsetstrokecolor{currentstroke}%
\pgfsetdash{}{0pt}%
\pgfpathmoveto{\pgfqpoint{1.619219in}{2.266945in}}%
\pgfpathcurveto{\pgfqpoint{1.630269in}{2.266945in}}{\pgfqpoint{1.640868in}{2.271335in}}{\pgfqpoint{1.648682in}{2.279149in}}%
\pgfpathcurveto{\pgfqpoint{1.656495in}{2.286963in}}{\pgfqpoint{1.660886in}{2.297562in}}{\pgfqpoint{1.660886in}{2.308612in}}%
\pgfpathcurveto{\pgfqpoint{1.660886in}{2.319662in}}{\pgfqpoint{1.656495in}{2.330261in}}{\pgfqpoint{1.648682in}{2.338075in}}%
\pgfpathcurveto{\pgfqpoint{1.640868in}{2.345888in}}{\pgfqpoint{1.630269in}{2.350278in}}{\pgfqpoint{1.619219in}{2.350278in}}%
\pgfpathcurveto{\pgfqpoint{1.608169in}{2.350278in}}{\pgfqpoint{1.597570in}{2.345888in}}{\pgfqpoint{1.589756in}{2.338075in}}%
\pgfpathcurveto{\pgfqpoint{1.581943in}{2.330261in}}{\pgfqpoint{1.577552in}{2.319662in}}{\pgfqpoint{1.577552in}{2.308612in}}%
\pgfpathcurveto{\pgfqpoint{1.577552in}{2.297562in}}{\pgfqpoint{1.581943in}{2.286963in}}{\pgfqpoint{1.589756in}{2.279149in}}%
\pgfpathcurveto{\pgfqpoint{1.597570in}{2.271335in}}{\pgfqpoint{1.608169in}{2.266945in}}{\pgfqpoint{1.619219in}{2.266945in}}%
\pgfpathclose%
\pgfusepath{stroke,fill}%
\end{pgfscope}%
\begin{pgfscope}%
\pgfpathrectangle{\pgfqpoint{0.600000in}{0.600000in}}{\pgfqpoint{3.900000in}{3.900000in}}%
\pgfusepath{clip}%
\pgfsetbuttcap%
\pgfsetroundjoin%
\definecolor{currentfill}{rgb}{0.121569,0.466667,0.705882}%
\pgfsetfillcolor{currentfill}%
\pgfsetlinewidth{1.003750pt}%
\definecolor{currentstroke}{rgb}{0.121569,0.466667,0.705882}%
\pgfsetstrokecolor{currentstroke}%
\pgfsetdash{}{0pt}%
\pgfpathmoveto{\pgfqpoint{3.142097in}{2.291924in}}%
\pgfpathcurveto{\pgfqpoint{3.153147in}{2.291924in}}{\pgfqpoint{3.163746in}{2.296314in}}{\pgfqpoint{3.171560in}{2.304128in}}%
\pgfpathcurveto{\pgfqpoint{3.179373in}{2.311942in}}{\pgfqpoint{3.183764in}{2.322541in}}{\pgfqpoint{3.183764in}{2.333591in}}%
\pgfpathcurveto{\pgfqpoint{3.183764in}{2.344641in}}{\pgfqpoint{3.179373in}{2.355240in}}{\pgfqpoint{3.171560in}{2.363054in}}%
\pgfpathcurveto{\pgfqpoint{3.163746in}{2.370867in}}{\pgfqpoint{3.153147in}{2.375258in}}{\pgfqpoint{3.142097in}{2.375258in}}%
\pgfpathcurveto{\pgfqpoint{3.131047in}{2.375258in}}{\pgfqpoint{3.120448in}{2.370867in}}{\pgfqpoint{3.112634in}{2.363054in}}%
\pgfpathcurveto{\pgfqpoint{3.104821in}{2.355240in}}{\pgfqpoint{3.100430in}{2.344641in}}{\pgfqpoint{3.100430in}{2.333591in}}%
\pgfpathcurveto{\pgfqpoint{3.100430in}{2.322541in}}{\pgfqpoint{3.104821in}{2.311942in}}{\pgfqpoint{3.112634in}{2.304128in}}%
\pgfpathcurveto{\pgfqpoint{3.120448in}{2.296314in}}{\pgfqpoint{3.131047in}{2.291924in}}{\pgfqpoint{3.142097in}{2.291924in}}%
\pgfpathclose%
\pgfusepath{stroke,fill}%
\end{pgfscope}%
\begin{pgfscope}%
\pgfpathrectangle{\pgfqpoint{0.600000in}{0.600000in}}{\pgfqpoint{3.900000in}{3.900000in}}%
\pgfusepath{clip}%
\pgfsetbuttcap%
\pgfsetroundjoin%
\definecolor{currentfill}{rgb}{0.121569,0.466667,0.705882}%
\pgfsetfillcolor{currentfill}%
\pgfsetlinewidth{1.003750pt}%
\definecolor{currentstroke}{rgb}{0.121569,0.466667,0.705882}%
\pgfsetstrokecolor{currentstroke}%
\pgfsetdash{}{0pt}%
\pgfpathmoveto{\pgfqpoint{3.213072in}{2.065202in}}%
\pgfpathcurveto{\pgfqpoint{3.224122in}{2.065202in}}{\pgfqpoint{3.234721in}{2.069592in}}{\pgfqpoint{3.242535in}{2.077405in}}%
\pgfpathcurveto{\pgfqpoint{3.250348in}{2.085219in}}{\pgfqpoint{3.254739in}{2.095818in}}{\pgfqpoint{3.254739in}{2.106868in}}%
\pgfpathcurveto{\pgfqpoint{3.254739in}{2.117918in}}{\pgfqpoint{3.250348in}{2.128517in}}{\pgfqpoint{3.242535in}{2.136331in}}%
\pgfpathcurveto{\pgfqpoint{3.234721in}{2.144145in}}{\pgfqpoint{3.224122in}{2.148535in}}{\pgfqpoint{3.213072in}{2.148535in}}%
\pgfpathcurveto{\pgfqpoint{3.202022in}{2.148535in}}{\pgfqpoint{3.191423in}{2.144145in}}{\pgfqpoint{3.183609in}{2.136331in}}%
\pgfpathcurveto{\pgfqpoint{3.175796in}{2.128517in}}{\pgfqpoint{3.171405in}{2.117918in}}{\pgfqpoint{3.171405in}{2.106868in}}%
\pgfpathcurveto{\pgfqpoint{3.171405in}{2.095818in}}{\pgfqpoint{3.175796in}{2.085219in}}{\pgfqpoint{3.183609in}{2.077405in}}%
\pgfpathcurveto{\pgfqpoint{3.191423in}{2.069592in}}{\pgfqpoint{3.202022in}{2.065202in}}{\pgfqpoint{3.213072in}{2.065202in}}%
\pgfpathclose%
\pgfusepath{stroke,fill}%
\end{pgfscope}%
\begin{pgfscope}%
\pgfpathrectangle{\pgfqpoint{0.600000in}{0.600000in}}{\pgfqpoint{3.900000in}{3.900000in}}%
\pgfusepath{clip}%
\pgfsetbuttcap%
\pgfsetroundjoin%
\definecolor{currentfill}{rgb}{0.121569,0.466667,0.705882}%
\pgfsetfillcolor{currentfill}%
\pgfsetlinewidth{1.003750pt}%
\definecolor{currentstroke}{rgb}{0.121569,0.466667,0.705882}%
\pgfsetstrokecolor{currentstroke}%
\pgfsetdash{}{0pt}%
\pgfpathmoveto{\pgfqpoint{2.751051in}{2.960998in}}%
\pgfpathcurveto{\pgfqpoint{2.762101in}{2.960998in}}{\pgfqpoint{2.772700in}{2.965388in}}{\pgfqpoint{2.780513in}{2.973202in}}%
\pgfpathcurveto{\pgfqpoint{2.788327in}{2.981015in}}{\pgfqpoint{2.792717in}{2.991614in}}{\pgfqpoint{2.792717in}{3.002664in}}%
\pgfpathcurveto{\pgfqpoint{2.792717in}{3.013715in}}{\pgfqpoint{2.788327in}{3.024314in}}{\pgfqpoint{2.780513in}{3.032127in}}%
\pgfpathcurveto{\pgfqpoint{2.772700in}{3.039941in}}{\pgfqpoint{2.762101in}{3.044331in}}{\pgfqpoint{2.751051in}{3.044331in}}%
\pgfpathcurveto{\pgfqpoint{2.740001in}{3.044331in}}{\pgfqpoint{2.729402in}{3.039941in}}{\pgfqpoint{2.721588in}{3.032127in}}%
\pgfpathcurveto{\pgfqpoint{2.713774in}{3.024314in}}{\pgfqpoint{2.709384in}{3.013715in}}{\pgfqpoint{2.709384in}{3.002664in}}%
\pgfpathcurveto{\pgfqpoint{2.709384in}{2.991614in}}{\pgfqpoint{2.713774in}{2.981015in}}{\pgfqpoint{2.721588in}{2.973202in}}%
\pgfpathcurveto{\pgfqpoint{2.729402in}{2.965388in}}{\pgfqpoint{2.740001in}{2.960998in}}{\pgfqpoint{2.751051in}{2.960998in}}%
\pgfpathclose%
\pgfusepath{stroke,fill}%
\end{pgfscope}%
\begin{pgfscope}%
\pgfpathrectangle{\pgfqpoint{0.600000in}{0.600000in}}{\pgfqpoint{3.900000in}{3.900000in}}%
\pgfusepath{clip}%
\pgfsetbuttcap%
\pgfsetroundjoin%
\definecolor{currentfill}{rgb}{0.121569,0.466667,0.705882}%
\pgfsetfillcolor{currentfill}%
\pgfsetlinewidth{1.003750pt}%
\definecolor{currentstroke}{rgb}{0.121569,0.466667,0.705882}%
\pgfsetstrokecolor{currentstroke}%
\pgfsetdash{}{0pt}%
\pgfpathmoveto{\pgfqpoint{2.845533in}{3.054398in}}%
\pgfpathcurveto{\pgfqpoint{2.856583in}{3.054398in}}{\pgfqpoint{2.867182in}{3.058788in}}{\pgfqpoint{2.874996in}{3.066602in}}%
\pgfpathcurveto{\pgfqpoint{2.882809in}{3.074416in}}{\pgfqpoint{2.887200in}{3.085015in}}{\pgfqpoint{2.887200in}{3.096065in}}%
\pgfpathcurveto{\pgfqpoint{2.887200in}{3.107115in}}{\pgfqpoint{2.882809in}{3.117714in}}{\pgfqpoint{2.874996in}{3.125528in}}%
\pgfpathcurveto{\pgfqpoint{2.867182in}{3.133341in}}{\pgfqpoint{2.856583in}{3.137731in}}{\pgfqpoint{2.845533in}{3.137731in}}%
\pgfpathcurveto{\pgfqpoint{2.834483in}{3.137731in}}{\pgfqpoint{2.823884in}{3.133341in}}{\pgfqpoint{2.816070in}{3.125528in}}%
\pgfpathcurveto{\pgfqpoint{2.808256in}{3.117714in}}{\pgfqpoint{2.803866in}{3.107115in}}{\pgfqpoint{2.803866in}{3.096065in}}%
\pgfpathcurveto{\pgfqpoint{2.803866in}{3.085015in}}{\pgfqpoint{2.808256in}{3.074416in}}{\pgfqpoint{2.816070in}{3.066602in}}%
\pgfpathcurveto{\pgfqpoint{2.823884in}{3.058788in}}{\pgfqpoint{2.834483in}{3.054398in}}{\pgfqpoint{2.845533in}{3.054398in}}%
\pgfpathclose%
\pgfusepath{stroke,fill}%
\end{pgfscope}%
\begin{pgfscope}%
\pgfpathrectangle{\pgfqpoint{0.600000in}{0.600000in}}{\pgfqpoint{3.900000in}{3.900000in}}%
\pgfusepath{clip}%
\pgfsetbuttcap%
\pgfsetroundjoin%
\definecolor{currentfill}{rgb}{0.121569,0.466667,0.705882}%
\pgfsetfillcolor{currentfill}%
\pgfsetlinewidth{1.003750pt}%
\definecolor{currentstroke}{rgb}{0.121569,0.466667,0.705882}%
\pgfsetstrokecolor{currentstroke}%
\pgfsetdash{}{0pt}%
\pgfpathmoveto{\pgfqpoint{2.817420in}{2.396318in}}%
\pgfpathcurveto{\pgfqpoint{2.828470in}{2.396318in}}{\pgfqpoint{2.839069in}{2.400708in}}{\pgfqpoint{2.846883in}{2.408522in}}%
\pgfpathcurveto{\pgfqpoint{2.854696in}{2.416335in}}{\pgfqpoint{2.859086in}{2.426934in}}{\pgfqpoint{2.859086in}{2.437985in}}%
\pgfpathcurveto{\pgfqpoint{2.859086in}{2.449035in}}{\pgfqpoint{2.854696in}{2.459634in}}{\pgfqpoint{2.846883in}{2.467447in}}%
\pgfpathcurveto{\pgfqpoint{2.839069in}{2.475261in}}{\pgfqpoint{2.828470in}{2.479651in}}{\pgfqpoint{2.817420in}{2.479651in}}%
\pgfpathcurveto{\pgfqpoint{2.806370in}{2.479651in}}{\pgfqpoint{2.795771in}{2.475261in}}{\pgfqpoint{2.787957in}{2.467447in}}%
\pgfpathcurveto{\pgfqpoint{2.780143in}{2.459634in}}{\pgfqpoint{2.775753in}{2.449035in}}{\pgfqpoint{2.775753in}{2.437985in}}%
\pgfpathcurveto{\pgfqpoint{2.775753in}{2.426934in}}{\pgfqpoint{2.780143in}{2.416335in}}{\pgfqpoint{2.787957in}{2.408522in}}%
\pgfpathcurveto{\pgfqpoint{2.795771in}{2.400708in}}{\pgfqpoint{2.806370in}{2.396318in}}{\pgfqpoint{2.817420in}{2.396318in}}%
\pgfpathclose%
\pgfusepath{stroke,fill}%
\end{pgfscope}%
\begin{pgfscope}%
\pgfpathrectangle{\pgfqpoint{0.600000in}{0.600000in}}{\pgfqpoint{3.900000in}{3.900000in}}%
\pgfusepath{clip}%
\pgfsetbuttcap%
\pgfsetroundjoin%
\definecolor{currentfill}{rgb}{0.121569,0.466667,0.705882}%
\pgfsetfillcolor{currentfill}%
\pgfsetlinewidth{1.003750pt}%
\definecolor{currentstroke}{rgb}{0.121569,0.466667,0.705882}%
\pgfsetstrokecolor{currentstroke}%
\pgfsetdash{}{0pt}%
\pgfpathmoveto{\pgfqpoint{2.102669in}{3.862408in}}%
\pgfpathcurveto{\pgfqpoint{2.113719in}{3.862408in}}{\pgfqpoint{2.124318in}{3.866798in}}{\pgfqpoint{2.132132in}{3.874611in}}%
\pgfpathcurveto{\pgfqpoint{2.139946in}{3.882425in}}{\pgfqpoint{2.144336in}{3.893024in}}{\pgfqpoint{2.144336in}{3.904074in}}%
\pgfpathcurveto{\pgfqpoint{2.144336in}{3.915124in}}{\pgfqpoint{2.139946in}{3.925723in}}{\pgfqpoint{2.132132in}{3.933537in}}%
\pgfpathcurveto{\pgfqpoint{2.124318in}{3.941351in}}{\pgfqpoint{2.113719in}{3.945741in}}{\pgfqpoint{2.102669in}{3.945741in}}%
\pgfpathcurveto{\pgfqpoint{2.091619in}{3.945741in}}{\pgfqpoint{2.081020in}{3.941351in}}{\pgfqpoint{2.073206in}{3.933537in}}%
\pgfpathcurveto{\pgfqpoint{2.065393in}{3.925723in}}{\pgfqpoint{2.061003in}{3.915124in}}{\pgfqpoint{2.061003in}{3.904074in}}%
\pgfpathcurveto{\pgfqpoint{2.061003in}{3.893024in}}{\pgfqpoint{2.065393in}{3.882425in}}{\pgfqpoint{2.073206in}{3.874611in}}%
\pgfpathcurveto{\pgfqpoint{2.081020in}{3.866798in}}{\pgfqpoint{2.091619in}{3.862408in}}{\pgfqpoint{2.102669in}{3.862408in}}%
\pgfpathclose%
\pgfusepath{stroke,fill}%
\end{pgfscope}%
\begin{pgfscope}%
\pgfpathrectangle{\pgfqpoint{0.600000in}{0.600000in}}{\pgfqpoint{3.900000in}{3.900000in}}%
\pgfusepath{clip}%
\pgfsetbuttcap%
\pgfsetroundjoin%
\definecolor{currentfill}{rgb}{0.121569,0.466667,0.705882}%
\pgfsetfillcolor{currentfill}%
\pgfsetlinewidth{1.003750pt}%
\definecolor{currentstroke}{rgb}{0.121569,0.466667,0.705882}%
\pgfsetstrokecolor{currentstroke}%
\pgfsetdash{}{0pt}%
\pgfpathmoveto{\pgfqpoint{2.496547in}{2.828508in}}%
\pgfpathcurveto{\pgfqpoint{2.507597in}{2.828508in}}{\pgfqpoint{2.518196in}{2.832898in}}{\pgfqpoint{2.526009in}{2.840712in}}%
\pgfpathcurveto{\pgfqpoint{2.533823in}{2.848526in}}{\pgfqpoint{2.538213in}{2.859125in}}{\pgfqpoint{2.538213in}{2.870175in}}%
\pgfpathcurveto{\pgfqpoint{2.538213in}{2.881225in}}{\pgfqpoint{2.533823in}{2.891824in}}{\pgfqpoint{2.526009in}{2.899637in}}%
\pgfpathcurveto{\pgfqpoint{2.518196in}{2.907451in}}{\pgfqpoint{2.507597in}{2.911841in}}{\pgfqpoint{2.496547in}{2.911841in}}%
\pgfpathcurveto{\pgfqpoint{2.485497in}{2.911841in}}{\pgfqpoint{2.474898in}{2.907451in}}{\pgfqpoint{2.467084in}{2.899637in}}%
\pgfpathcurveto{\pgfqpoint{2.459270in}{2.891824in}}{\pgfqpoint{2.454880in}{2.881225in}}{\pgfqpoint{2.454880in}{2.870175in}}%
\pgfpathcurveto{\pgfqpoint{2.454880in}{2.859125in}}{\pgfqpoint{2.459270in}{2.848526in}}{\pgfqpoint{2.467084in}{2.840712in}}%
\pgfpathcurveto{\pgfqpoint{2.474898in}{2.832898in}}{\pgfqpoint{2.485497in}{2.828508in}}{\pgfqpoint{2.496547in}{2.828508in}}%
\pgfpathclose%
\pgfusepath{stroke,fill}%
\end{pgfscope}%
\begin{pgfscope}%
\pgfpathrectangle{\pgfqpoint{0.600000in}{0.600000in}}{\pgfqpoint{3.900000in}{3.900000in}}%
\pgfusepath{clip}%
\pgfsetbuttcap%
\pgfsetroundjoin%
\definecolor{currentfill}{rgb}{0.121569,0.466667,0.705882}%
\pgfsetfillcolor{currentfill}%
\pgfsetlinewidth{1.003750pt}%
\definecolor{currentstroke}{rgb}{0.121569,0.466667,0.705882}%
\pgfsetstrokecolor{currentstroke}%
\pgfsetdash{}{0pt}%
\pgfpathmoveto{\pgfqpoint{2.121102in}{2.430832in}}%
\pgfpathcurveto{\pgfqpoint{2.132152in}{2.430832in}}{\pgfqpoint{2.142751in}{2.435223in}}{\pgfqpoint{2.150565in}{2.443036in}}%
\pgfpathcurveto{\pgfqpoint{2.158379in}{2.450850in}}{\pgfqpoint{2.162769in}{2.461449in}}{\pgfqpoint{2.162769in}{2.472499in}}%
\pgfpathcurveto{\pgfqpoint{2.162769in}{2.483549in}}{\pgfqpoint{2.158379in}{2.494148in}}{\pgfqpoint{2.150565in}{2.501962in}}%
\pgfpathcurveto{\pgfqpoint{2.142751in}{2.509775in}}{\pgfqpoint{2.132152in}{2.514166in}}{\pgfqpoint{2.121102in}{2.514166in}}%
\pgfpathcurveto{\pgfqpoint{2.110052in}{2.514166in}}{\pgfqpoint{2.099453in}{2.509775in}}{\pgfqpoint{2.091640in}{2.501962in}}%
\pgfpathcurveto{\pgfqpoint{2.083826in}{2.494148in}}{\pgfqpoint{2.079436in}{2.483549in}}{\pgfqpoint{2.079436in}{2.472499in}}%
\pgfpathcurveto{\pgfqpoint{2.079436in}{2.461449in}}{\pgfqpoint{2.083826in}{2.450850in}}{\pgfqpoint{2.091640in}{2.443036in}}%
\pgfpathcurveto{\pgfqpoint{2.099453in}{2.435223in}}{\pgfqpoint{2.110052in}{2.430832in}}{\pgfqpoint{2.121102in}{2.430832in}}%
\pgfpathclose%
\pgfusepath{stroke,fill}%
\end{pgfscope}%
\begin{pgfscope}%
\pgfpathrectangle{\pgfqpoint{0.600000in}{0.600000in}}{\pgfqpoint{3.900000in}{3.900000in}}%
\pgfusepath{clip}%
\pgfsetbuttcap%
\pgfsetroundjoin%
\definecolor{currentfill}{rgb}{0.121569,0.466667,0.705882}%
\pgfsetfillcolor{currentfill}%
\pgfsetlinewidth{1.003750pt}%
\definecolor{currentstroke}{rgb}{0.121569,0.466667,0.705882}%
\pgfsetstrokecolor{currentstroke}%
\pgfsetdash{}{0pt}%
\pgfpathmoveto{\pgfqpoint{2.002991in}{3.079599in}}%
\pgfpathcurveto{\pgfqpoint{2.014041in}{3.079599in}}{\pgfqpoint{2.024640in}{3.083989in}}{\pgfqpoint{2.032454in}{3.091803in}}%
\pgfpathcurveto{\pgfqpoint{2.040267in}{3.099616in}}{\pgfqpoint{2.044657in}{3.110215in}}{\pgfqpoint{2.044657in}{3.121265in}}%
\pgfpathcurveto{\pgfqpoint{2.044657in}{3.132315in}}{\pgfqpoint{2.040267in}{3.142914in}}{\pgfqpoint{2.032454in}{3.150728in}}%
\pgfpathcurveto{\pgfqpoint{2.024640in}{3.158542in}}{\pgfqpoint{2.014041in}{3.162932in}}{\pgfqpoint{2.002991in}{3.162932in}}%
\pgfpathcurveto{\pgfqpoint{1.991941in}{3.162932in}}{\pgfqpoint{1.981342in}{3.158542in}}{\pgfqpoint{1.973528in}{3.150728in}}%
\pgfpathcurveto{\pgfqpoint{1.965714in}{3.142914in}}{\pgfqpoint{1.961324in}{3.132315in}}{\pgfqpoint{1.961324in}{3.121265in}}%
\pgfpathcurveto{\pgfqpoint{1.961324in}{3.110215in}}{\pgfqpoint{1.965714in}{3.099616in}}{\pgfqpoint{1.973528in}{3.091803in}}%
\pgfpathcurveto{\pgfqpoint{1.981342in}{3.083989in}}{\pgfqpoint{1.991941in}{3.079599in}}{\pgfqpoint{2.002991in}{3.079599in}}%
\pgfpathclose%
\pgfusepath{stroke,fill}%
\end{pgfscope}%
\begin{pgfscope}%
\pgfpathrectangle{\pgfqpoint{0.600000in}{0.600000in}}{\pgfqpoint{3.900000in}{3.900000in}}%
\pgfusepath{clip}%
\pgfsetbuttcap%
\pgfsetroundjoin%
\definecolor{currentfill}{rgb}{0.121569,0.466667,0.705882}%
\pgfsetfillcolor{currentfill}%
\pgfsetlinewidth{1.003750pt}%
\definecolor{currentstroke}{rgb}{0.121569,0.466667,0.705882}%
\pgfsetstrokecolor{currentstroke}%
\pgfsetdash{}{0pt}%
\pgfpathmoveto{\pgfqpoint{1.456021in}{2.771399in}}%
\pgfpathcurveto{\pgfqpoint{1.467071in}{2.771399in}}{\pgfqpoint{1.477670in}{2.775789in}}{\pgfqpoint{1.485483in}{2.783603in}}%
\pgfpathcurveto{\pgfqpoint{1.493297in}{2.791416in}}{\pgfqpoint{1.497687in}{2.802016in}}{\pgfqpoint{1.497687in}{2.813066in}}%
\pgfpathcurveto{\pgfqpoint{1.497687in}{2.824116in}}{\pgfqpoint{1.493297in}{2.834715in}}{\pgfqpoint{1.485483in}{2.842528in}}%
\pgfpathcurveto{\pgfqpoint{1.477670in}{2.850342in}}{\pgfqpoint{1.467071in}{2.854732in}}{\pgfqpoint{1.456021in}{2.854732in}}%
\pgfpathcurveto{\pgfqpoint{1.444971in}{2.854732in}}{\pgfqpoint{1.434371in}{2.850342in}}{\pgfqpoint{1.426558in}{2.842528in}}%
\pgfpathcurveto{\pgfqpoint{1.418744in}{2.834715in}}{\pgfqpoint{1.414354in}{2.824116in}}{\pgfqpoint{1.414354in}{2.813066in}}%
\pgfpathcurveto{\pgfqpoint{1.414354in}{2.802016in}}{\pgfqpoint{1.418744in}{2.791416in}}{\pgfqpoint{1.426558in}{2.783603in}}%
\pgfpathcurveto{\pgfqpoint{1.434371in}{2.775789in}}{\pgfqpoint{1.444971in}{2.771399in}}{\pgfqpoint{1.456021in}{2.771399in}}%
\pgfpathclose%
\pgfusepath{stroke,fill}%
\end{pgfscope}%
\begin{pgfscope}%
\pgfpathrectangle{\pgfqpoint{0.600000in}{0.600000in}}{\pgfqpoint{3.900000in}{3.900000in}}%
\pgfusepath{clip}%
\pgfsetbuttcap%
\pgfsetroundjoin%
\definecolor{currentfill}{rgb}{0.121569,0.466667,0.705882}%
\pgfsetfillcolor{currentfill}%
\pgfsetlinewidth{1.003750pt}%
\definecolor{currentstroke}{rgb}{0.121569,0.466667,0.705882}%
\pgfsetstrokecolor{currentstroke}%
\pgfsetdash{}{0pt}%
\pgfpathmoveto{\pgfqpoint{2.275969in}{2.676610in}}%
\pgfpathcurveto{\pgfqpoint{2.287020in}{2.676610in}}{\pgfqpoint{2.297619in}{2.681000in}}{\pgfqpoint{2.305432in}{2.688814in}}%
\pgfpathcurveto{\pgfqpoint{2.313246in}{2.696627in}}{\pgfqpoint{2.317636in}{2.707226in}}{\pgfqpoint{2.317636in}{2.718277in}}%
\pgfpathcurveto{\pgfqpoint{2.317636in}{2.729327in}}{\pgfqpoint{2.313246in}{2.739926in}}{\pgfqpoint{2.305432in}{2.747739in}}%
\pgfpathcurveto{\pgfqpoint{2.297619in}{2.755553in}}{\pgfqpoint{2.287020in}{2.759943in}}{\pgfqpoint{2.275969in}{2.759943in}}%
\pgfpathcurveto{\pgfqpoint{2.264919in}{2.759943in}}{\pgfqpoint{2.254320in}{2.755553in}}{\pgfqpoint{2.246507in}{2.747739in}}%
\pgfpathcurveto{\pgfqpoint{2.238693in}{2.739926in}}{\pgfqpoint{2.234303in}{2.729327in}}{\pgfqpoint{2.234303in}{2.718277in}}%
\pgfpathcurveto{\pgfqpoint{2.234303in}{2.707226in}}{\pgfqpoint{2.238693in}{2.696627in}}{\pgfqpoint{2.246507in}{2.688814in}}%
\pgfpathcurveto{\pgfqpoint{2.254320in}{2.681000in}}{\pgfqpoint{2.264919in}{2.676610in}}{\pgfqpoint{2.275969in}{2.676610in}}%
\pgfpathclose%
\pgfusepath{stroke,fill}%
\end{pgfscope}%
\begin{pgfscope}%
\pgfpathrectangle{\pgfqpoint{0.600000in}{0.600000in}}{\pgfqpoint{3.900000in}{3.900000in}}%
\pgfusepath{clip}%
\pgfsetbuttcap%
\pgfsetroundjoin%
\definecolor{currentfill}{rgb}{0.121569,0.466667,0.705882}%
\pgfsetfillcolor{currentfill}%
\pgfsetlinewidth{1.003750pt}%
\definecolor{currentstroke}{rgb}{0.121569,0.466667,0.705882}%
\pgfsetstrokecolor{currentstroke}%
\pgfsetdash{}{0pt}%
\pgfpathmoveto{\pgfqpoint{2.333466in}{1.846557in}}%
\pgfpathcurveto{\pgfqpoint{2.344516in}{1.846557in}}{\pgfqpoint{2.355115in}{1.850947in}}{\pgfqpoint{2.362928in}{1.858760in}}%
\pgfpathcurveto{\pgfqpoint{2.370742in}{1.866574in}}{\pgfqpoint{2.375132in}{1.877173in}}{\pgfqpoint{2.375132in}{1.888223in}}%
\pgfpathcurveto{\pgfqpoint{2.375132in}{1.899273in}}{\pgfqpoint{2.370742in}{1.909872in}}{\pgfqpoint{2.362928in}{1.917686in}}%
\pgfpathcurveto{\pgfqpoint{2.355115in}{1.925500in}}{\pgfqpoint{2.344516in}{1.929890in}}{\pgfqpoint{2.333466in}{1.929890in}}%
\pgfpathcurveto{\pgfqpoint{2.322416in}{1.929890in}}{\pgfqpoint{2.311816in}{1.925500in}}{\pgfqpoint{2.304003in}{1.917686in}}%
\pgfpathcurveto{\pgfqpoint{2.296189in}{1.909872in}}{\pgfqpoint{2.291799in}{1.899273in}}{\pgfqpoint{2.291799in}{1.888223in}}%
\pgfpathcurveto{\pgfqpoint{2.291799in}{1.877173in}}{\pgfqpoint{2.296189in}{1.866574in}}{\pgfqpoint{2.304003in}{1.858760in}}%
\pgfpathcurveto{\pgfqpoint{2.311816in}{1.850947in}}{\pgfqpoint{2.322416in}{1.846557in}}{\pgfqpoint{2.333466in}{1.846557in}}%
\pgfpathclose%
\pgfusepath{stroke,fill}%
\end{pgfscope}%
\begin{pgfscope}%
\pgfpathrectangle{\pgfqpoint{0.600000in}{0.600000in}}{\pgfqpoint{3.900000in}{3.900000in}}%
\pgfusepath{clip}%
\pgfsetbuttcap%
\pgfsetroundjoin%
\definecolor{currentfill}{rgb}{0.121569,0.466667,0.705882}%
\pgfsetfillcolor{currentfill}%
\pgfsetlinewidth{1.003750pt}%
\definecolor{currentstroke}{rgb}{0.121569,0.466667,0.705882}%
\pgfsetstrokecolor{currentstroke}%
\pgfsetdash{}{0pt}%
\pgfpathmoveto{\pgfqpoint{1.466738in}{1.855187in}}%
\pgfpathcurveto{\pgfqpoint{1.477788in}{1.855187in}}{\pgfqpoint{1.488387in}{1.859578in}}{\pgfqpoint{1.496201in}{1.867391in}}%
\pgfpathcurveto{\pgfqpoint{1.504015in}{1.875205in}}{\pgfqpoint{1.508405in}{1.885804in}}{\pgfqpoint{1.508405in}{1.896854in}}%
\pgfpathcurveto{\pgfqpoint{1.508405in}{1.907904in}}{\pgfqpoint{1.504015in}{1.918503in}}{\pgfqpoint{1.496201in}{1.926317in}}%
\pgfpathcurveto{\pgfqpoint{1.488387in}{1.934131in}}{\pgfqpoint{1.477788in}{1.938521in}}{\pgfqpoint{1.466738in}{1.938521in}}%
\pgfpathcurveto{\pgfqpoint{1.455688in}{1.938521in}}{\pgfqpoint{1.445089in}{1.934131in}}{\pgfqpoint{1.437275in}{1.926317in}}%
\pgfpathcurveto{\pgfqpoint{1.429462in}{1.918503in}}{\pgfqpoint{1.425072in}{1.907904in}}{\pgfqpoint{1.425072in}{1.896854in}}%
\pgfpathcurveto{\pgfqpoint{1.425072in}{1.885804in}}{\pgfqpoint{1.429462in}{1.875205in}}{\pgfqpoint{1.437275in}{1.867391in}}%
\pgfpathcurveto{\pgfqpoint{1.445089in}{1.859578in}}{\pgfqpoint{1.455688in}{1.855187in}}{\pgfqpoint{1.466738in}{1.855187in}}%
\pgfpathclose%
\pgfusepath{stroke,fill}%
\end{pgfscope}%
\begin{pgfscope}%
\pgfpathrectangle{\pgfqpoint{0.600000in}{0.600000in}}{\pgfqpoint{3.900000in}{3.900000in}}%
\pgfusepath{clip}%
\pgfsetbuttcap%
\pgfsetroundjoin%
\definecolor{currentfill}{rgb}{0.121569,0.466667,0.705882}%
\pgfsetfillcolor{currentfill}%
\pgfsetlinewidth{1.003750pt}%
\definecolor{currentstroke}{rgb}{0.121569,0.466667,0.705882}%
\pgfsetstrokecolor{currentstroke}%
\pgfsetdash{}{0pt}%
\pgfpathmoveto{\pgfqpoint{2.987809in}{1.929818in}}%
\pgfpathcurveto{\pgfqpoint{2.998859in}{1.929818in}}{\pgfqpoint{3.009458in}{1.934208in}}{\pgfqpoint{3.017272in}{1.942022in}}%
\pgfpathcurveto{\pgfqpoint{3.025085in}{1.949835in}}{\pgfqpoint{3.029476in}{1.960434in}}{\pgfqpoint{3.029476in}{1.971484in}}%
\pgfpathcurveto{\pgfqpoint{3.029476in}{1.982534in}}{\pgfqpoint{3.025085in}{1.993134in}}{\pgfqpoint{3.017272in}{2.000947in}}%
\pgfpathcurveto{\pgfqpoint{3.009458in}{2.008761in}}{\pgfqpoint{2.998859in}{2.013151in}}{\pgfqpoint{2.987809in}{2.013151in}}%
\pgfpathcurveto{\pgfqpoint{2.976759in}{2.013151in}}{\pgfqpoint{2.966160in}{2.008761in}}{\pgfqpoint{2.958346in}{2.000947in}}%
\pgfpathcurveto{\pgfqpoint{2.950533in}{1.993134in}}{\pgfqpoint{2.946142in}{1.982534in}}{\pgfqpoint{2.946142in}{1.971484in}}%
\pgfpathcurveto{\pgfqpoint{2.946142in}{1.960434in}}{\pgfqpoint{2.950533in}{1.949835in}}{\pgfqpoint{2.958346in}{1.942022in}}%
\pgfpathcurveto{\pgfqpoint{2.966160in}{1.934208in}}{\pgfqpoint{2.976759in}{1.929818in}}{\pgfqpoint{2.987809in}{1.929818in}}%
\pgfpathclose%
\pgfusepath{stroke,fill}%
\end{pgfscope}%
\begin{pgfscope}%
\pgfpathrectangle{\pgfqpoint{0.600000in}{0.600000in}}{\pgfqpoint{3.900000in}{3.900000in}}%
\pgfusepath{clip}%
\pgfsetbuttcap%
\pgfsetroundjoin%
\definecolor{currentfill}{rgb}{0.121569,0.466667,0.705882}%
\pgfsetfillcolor{currentfill}%
\pgfsetlinewidth{1.003750pt}%
\definecolor{currentstroke}{rgb}{0.121569,0.466667,0.705882}%
\pgfsetstrokecolor{currentstroke}%
\pgfsetdash{}{0pt}%
\pgfpathmoveto{\pgfqpoint{3.678947in}{2.592438in}}%
\pgfpathcurveto{\pgfqpoint{3.689997in}{2.592438in}}{\pgfqpoint{3.700596in}{2.596828in}}{\pgfqpoint{3.708409in}{2.604642in}}%
\pgfpathcurveto{\pgfqpoint{3.716223in}{2.612455in}}{\pgfqpoint{3.720613in}{2.623054in}}{\pgfqpoint{3.720613in}{2.634104in}}%
\pgfpathcurveto{\pgfqpoint{3.720613in}{2.645155in}}{\pgfqpoint{3.716223in}{2.655754in}}{\pgfqpoint{3.708409in}{2.663567in}}%
\pgfpathcurveto{\pgfqpoint{3.700596in}{2.671381in}}{\pgfqpoint{3.689997in}{2.675771in}}{\pgfqpoint{3.678947in}{2.675771in}}%
\pgfpathcurveto{\pgfqpoint{3.667897in}{2.675771in}}{\pgfqpoint{3.657298in}{2.671381in}}{\pgfqpoint{3.649484in}{2.663567in}}%
\pgfpathcurveto{\pgfqpoint{3.641670in}{2.655754in}}{\pgfqpoint{3.637280in}{2.645155in}}{\pgfqpoint{3.637280in}{2.634104in}}%
\pgfpathcurveto{\pgfqpoint{3.637280in}{2.623054in}}{\pgfqpoint{3.641670in}{2.612455in}}{\pgfqpoint{3.649484in}{2.604642in}}%
\pgfpathcurveto{\pgfqpoint{3.657298in}{2.596828in}}{\pgfqpoint{3.667897in}{2.592438in}}{\pgfqpoint{3.678947in}{2.592438in}}%
\pgfpathclose%
\pgfusepath{stroke,fill}%
\end{pgfscope}%
\begin{pgfscope}%
\pgfpathrectangle{\pgfqpoint{0.600000in}{0.600000in}}{\pgfqpoint{3.900000in}{3.900000in}}%
\pgfusepath{clip}%
\pgfsetbuttcap%
\pgfsetroundjoin%
\definecolor{currentfill}{rgb}{0.121569,0.466667,0.705882}%
\pgfsetfillcolor{currentfill}%
\pgfsetlinewidth{1.003750pt}%
\definecolor{currentstroke}{rgb}{0.121569,0.466667,0.705882}%
\pgfsetstrokecolor{currentstroke}%
\pgfsetdash{}{0pt}%
\pgfpathmoveto{\pgfqpoint{2.549384in}{1.882341in}}%
\pgfpathcurveto{\pgfqpoint{2.560434in}{1.882341in}}{\pgfqpoint{2.571033in}{1.886731in}}{\pgfqpoint{2.578847in}{1.894545in}}%
\pgfpathcurveto{\pgfqpoint{2.586661in}{1.902358in}}{\pgfqpoint{2.591051in}{1.912957in}}{\pgfqpoint{2.591051in}{1.924008in}}%
\pgfpathcurveto{\pgfqpoint{2.591051in}{1.935058in}}{\pgfqpoint{2.586661in}{1.945657in}}{\pgfqpoint{2.578847in}{1.953470in}}%
\pgfpathcurveto{\pgfqpoint{2.571033in}{1.961284in}}{\pgfqpoint{2.560434in}{1.965674in}}{\pgfqpoint{2.549384in}{1.965674in}}%
\pgfpathcurveto{\pgfqpoint{2.538334in}{1.965674in}}{\pgfqpoint{2.527735in}{1.961284in}}{\pgfqpoint{2.519921in}{1.953470in}}%
\pgfpathcurveto{\pgfqpoint{2.512108in}{1.945657in}}{\pgfqpoint{2.507718in}{1.935058in}}{\pgfqpoint{2.507718in}{1.924008in}}%
\pgfpathcurveto{\pgfqpoint{2.507718in}{1.912957in}}{\pgfqpoint{2.512108in}{1.902358in}}{\pgfqpoint{2.519921in}{1.894545in}}%
\pgfpathcurveto{\pgfqpoint{2.527735in}{1.886731in}}{\pgfqpoint{2.538334in}{1.882341in}}{\pgfqpoint{2.549384in}{1.882341in}}%
\pgfpathclose%
\pgfusepath{stroke,fill}%
\end{pgfscope}%
\begin{pgfscope}%
\pgfpathrectangle{\pgfqpoint{0.600000in}{0.600000in}}{\pgfqpoint{3.900000in}{3.900000in}}%
\pgfusepath{clip}%
\pgfsetbuttcap%
\pgfsetroundjoin%
\definecolor{currentfill}{rgb}{0.121569,0.466667,0.705882}%
\pgfsetfillcolor{currentfill}%
\pgfsetlinewidth{1.003750pt}%
\definecolor{currentstroke}{rgb}{0.121569,0.466667,0.705882}%
\pgfsetstrokecolor{currentstroke}%
\pgfsetdash{}{0pt}%
\pgfpathmoveto{\pgfqpoint{2.692829in}{2.087625in}}%
\pgfpathcurveto{\pgfqpoint{2.703879in}{2.087625in}}{\pgfqpoint{2.714478in}{2.092015in}}{\pgfqpoint{2.722292in}{2.099828in}}%
\pgfpathcurveto{\pgfqpoint{2.730105in}{2.107642in}}{\pgfqpoint{2.734496in}{2.118241in}}{\pgfqpoint{2.734496in}{2.129291in}}%
\pgfpathcurveto{\pgfqpoint{2.734496in}{2.140341in}}{\pgfqpoint{2.730105in}{2.150940in}}{\pgfqpoint{2.722292in}{2.158754in}}%
\pgfpathcurveto{\pgfqpoint{2.714478in}{2.166568in}}{\pgfqpoint{2.703879in}{2.170958in}}{\pgfqpoint{2.692829in}{2.170958in}}%
\pgfpathcurveto{\pgfqpoint{2.681779in}{2.170958in}}{\pgfqpoint{2.671180in}{2.166568in}}{\pgfqpoint{2.663366in}{2.158754in}}%
\pgfpathcurveto{\pgfqpoint{2.655552in}{2.150940in}}{\pgfqpoint{2.651162in}{2.140341in}}{\pgfqpoint{2.651162in}{2.129291in}}%
\pgfpathcurveto{\pgfqpoint{2.651162in}{2.118241in}}{\pgfqpoint{2.655552in}{2.107642in}}{\pgfqpoint{2.663366in}{2.099828in}}%
\pgfpathcurveto{\pgfqpoint{2.671180in}{2.092015in}}{\pgfqpoint{2.681779in}{2.087625in}}{\pgfqpoint{2.692829in}{2.087625in}}%
\pgfpathclose%
\pgfusepath{stroke,fill}%
\end{pgfscope}%
\begin{pgfscope}%
\pgfpathrectangle{\pgfqpoint{0.600000in}{0.600000in}}{\pgfqpoint{3.900000in}{3.900000in}}%
\pgfusepath{clip}%
\pgfsetbuttcap%
\pgfsetroundjoin%
\definecolor{currentfill}{rgb}{0.121569,0.466667,0.705882}%
\pgfsetfillcolor{currentfill}%
\pgfsetlinewidth{1.003750pt}%
\definecolor{currentstroke}{rgb}{0.121569,0.466667,0.705882}%
\pgfsetstrokecolor{currentstroke}%
\pgfsetdash{}{0pt}%
\pgfpathmoveto{\pgfqpoint{2.429441in}{2.671017in}}%
\pgfpathcurveto{\pgfqpoint{2.440491in}{2.671017in}}{\pgfqpoint{2.451090in}{2.675407in}}{\pgfqpoint{2.458904in}{2.683221in}}%
\pgfpathcurveto{\pgfqpoint{2.466717in}{2.691035in}}{\pgfqpoint{2.471108in}{2.701634in}}{\pgfqpoint{2.471108in}{2.712684in}}%
\pgfpathcurveto{\pgfqpoint{2.471108in}{2.723734in}}{\pgfqpoint{2.466717in}{2.734333in}}{\pgfqpoint{2.458904in}{2.742146in}}%
\pgfpathcurveto{\pgfqpoint{2.451090in}{2.749960in}}{\pgfqpoint{2.440491in}{2.754350in}}{\pgfqpoint{2.429441in}{2.754350in}}%
\pgfpathcurveto{\pgfqpoint{2.418391in}{2.754350in}}{\pgfqpoint{2.407792in}{2.749960in}}{\pgfqpoint{2.399978in}{2.742146in}}%
\pgfpathcurveto{\pgfqpoint{2.392165in}{2.734333in}}{\pgfqpoint{2.387774in}{2.723734in}}{\pgfqpoint{2.387774in}{2.712684in}}%
\pgfpathcurveto{\pgfqpoint{2.387774in}{2.701634in}}{\pgfqpoint{2.392165in}{2.691035in}}{\pgfqpoint{2.399978in}{2.683221in}}%
\pgfpathcurveto{\pgfqpoint{2.407792in}{2.675407in}}{\pgfqpoint{2.418391in}{2.671017in}}{\pgfqpoint{2.429441in}{2.671017in}}%
\pgfpathclose%
\pgfusepath{stroke,fill}%
\end{pgfscope}%
\begin{pgfscope}%
\pgfpathrectangle{\pgfqpoint{0.600000in}{0.600000in}}{\pgfqpoint{3.900000in}{3.900000in}}%
\pgfusepath{clip}%
\pgfsetbuttcap%
\pgfsetroundjoin%
\definecolor{currentfill}{rgb}{0.121569,0.466667,0.705882}%
\pgfsetfillcolor{currentfill}%
\pgfsetlinewidth{1.003750pt}%
\definecolor{currentstroke}{rgb}{0.121569,0.466667,0.705882}%
\pgfsetstrokecolor{currentstroke}%
\pgfsetdash{}{0pt}%
\pgfpathmoveto{\pgfqpoint{2.004675in}{2.612218in}}%
\pgfpathcurveto{\pgfqpoint{2.015725in}{2.612218in}}{\pgfqpoint{2.026324in}{2.616608in}}{\pgfqpoint{2.034138in}{2.624422in}}%
\pgfpathcurveto{\pgfqpoint{2.041952in}{2.632235in}}{\pgfqpoint{2.046342in}{2.642834in}}{\pgfqpoint{2.046342in}{2.653884in}}%
\pgfpathcurveto{\pgfqpoint{2.046342in}{2.664935in}}{\pgfqpoint{2.041952in}{2.675534in}}{\pgfqpoint{2.034138in}{2.683347in}}%
\pgfpathcurveto{\pgfqpoint{2.026324in}{2.691161in}}{\pgfqpoint{2.015725in}{2.695551in}}{\pgfqpoint{2.004675in}{2.695551in}}%
\pgfpathcurveto{\pgfqpoint{1.993625in}{2.695551in}}{\pgfqpoint{1.983026in}{2.691161in}}{\pgfqpoint{1.975212in}{2.683347in}}%
\pgfpathcurveto{\pgfqpoint{1.967399in}{2.675534in}}{\pgfqpoint{1.963009in}{2.664935in}}{\pgfqpoint{1.963009in}{2.653884in}}%
\pgfpathcurveto{\pgfqpoint{1.963009in}{2.642834in}}{\pgfqpoint{1.967399in}{2.632235in}}{\pgfqpoint{1.975212in}{2.624422in}}%
\pgfpathcurveto{\pgfqpoint{1.983026in}{2.616608in}}{\pgfqpoint{1.993625in}{2.612218in}}{\pgfqpoint{2.004675in}{2.612218in}}%
\pgfpathclose%
\pgfusepath{stroke,fill}%
\end{pgfscope}%
\begin{pgfscope}%
\pgfpathrectangle{\pgfqpoint{0.600000in}{0.600000in}}{\pgfqpoint{3.900000in}{3.900000in}}%
\pgfusepath{clip}%
\pgfsetbuttcap%
\pgfsetroundjoin%
\definecolor{currentfill}{rgb}{0.121569,0.466667,0.705882}%
\pgfsetfillcolor{currentfill}%
\pgfsetlinewidth{1.003750pt}%
\definecolor{currentstroke}{rgb}{0.121569,0.466667,0.705882}%
\pgfsetstrokecolor{currentstroke}%
\pgfsetdash{}{0pt}%
\pgfpathmoveto{\pgfqpoint{3.438261in}{1.390930in}}%
\pgfpathcurveto{\pgfqpoint{3.449311in}{1.390930in}}{\pgfqpoint{3.459910in}{1.395320in}}{\pgfqpoint{3.467723in}{1.403134in}}%
\pgfpathcurveto{\pgfqpoint{3.475537in}{1.410947in}}{\pgfqpoint{3.479927in}{1.421547in}}{\pgfqpoint{3.479927in}{1.432597in}}%
\pgfpathcurveto{\pgfqpoint{3.479927in}{1.443647in}}{\pgfqpoint{3.475537in}{1.454246in}}{\pgfqpoint{3.467723in}{1.462059in}}%
\pgfpathcurveto{\pgfqpoint{3.459910in}{1.469873in}}{\pgfqpoint{3.449311in}{1.474263in}}{\pgfqpoint{3.438261in}{1.474263in}}%
\pgfpathcurveto{\pgfqpoint{3.427210in}{1.474263in}}{\pgfqpoint{3.416611in}{1.469873in}}{\pgfqpoint{3.408798in}{1.462059in}}%
\pgfpathcurveto{\pgfqpoint{3.400984in}{1.454246in}}{\pgfqpoint{3.396594in}{1.443647in}}{\pgfqpoint{3.396594in}{1.432597in}}%
\pgfpathcurveto{\pgfqpoint{3.396594in}{1.421547in}}{\pgfqpoint{3.400984in}{1.410947in}}{\pgfqpoint{3.408798in}{1.403134in}}%
\pgfpathcurveto{\pgfqpoint{3.416611in}{1.395320in}}{\pgfqpoint{3.427210in}{1.390930in}}{\pgfqpoint{3.438261in}{1.390930in}}%
\pgfpathclose%
\pgfusepath{stroke,fill}%
\end{pgfscope}%
\begin{pgfscope}%
\pgfpathrectangle{\pgfqpoint{0.600000in}{0.600000in}}{\pgfqpoint{3.900000in}{3.900000in}}%
\pgfusepath{clip}%
\pgfsetbuttcap%
\pgfsetroundjoin%
\definecolor{currentfill}{rgb}{0.121569,0.466667,0.705882}%
\pgfsetfillcolor{currentfill}%
\pgfsetlinewidth{1.003750pt}%
\definecolor{currentstroke}{rgb}{0.121569,0.466667,0.705882}%
\pgfsetstrokecolor{currentstroke}%
\pgfsetdash{}{0pt}%
\pgfpathmoveto{\pgfqpoint{1.617885in}{2.156395in}}%
\pgfpathcurveto{\pgfqpoint{1.628935in}{2.156395in}}{\pgfqpoint{1.639534in}{2.160785in}}{\pgfqpoint{1.647348in}{2.168599in}}%
\pgfpathcurveto{\pgfqpoint{1.655161in}{2.176412in}}{\pgfqpoint{1.659552in}{2.187011in}}{\pgfqpoint{1.659552in}{2.198062in}}%
\pgfpathcurveto{\pgfqpoint{1.659552in}{2.209112in}}{\pgfqpoint{1.655161in}{2.219711in}}{\pgfqpoint{1.647348in}{2.227524in}}%
\pgfpathcurveto{\pgfqpoint{1.639534in}{2.235338in}}{\pgfqpoint{1.628935in}{2.239728in}}{\pgfqpoint{1.617885in}{2.239728in}}%
\pgfpathcurveto{\pgfqpoint{1.606835in}{2.239728in}}{\pgfqpoint{1.596236in}{2.235338in}}{\pgfqpoint{1.588422in}{2.227524in}}%
\pgfpathcurveto{\pgfqpoint{1.580609in}{2.219711in}}{\pgfqpoint{1.576218in}{2.209112in}}{\pgfqpoint{1.576218in}{2.198062in}}%
\pgfpathcurveto{\pgfqpoint{1.576218in}{2.187011in}}{\pgfqpoint{1.580609in}{2.176412in}}{\pgfqpoint{1.588422in}{2.168599in}}%
\pgfpathcurveto{\pgfqpoint{1.596236in}{2.160785in}}{\pgfqpoint{1.606835in}{2.156395in}}{\pgfqpoint{1.617885in}{2.156395in}}%
\pgfpathclose%
\pgfusepath{stroke,fill}%
\end{pgfscope}%
\begin{pgfscope}%
\pgfpathrectangle{\pgfqpoint{0.600000in}{0.600000in}}{\pgfqpoint{3.900000in}{3.900000in}}%
\pgfusepath{clip}%
\pgfsetbuttcap%
\pgfsetroundjoin%
\definecolor{currentfill}{rgb}{0.121569,0.466667,0.705882}%
\pgfsetfillcolor{currentfill}%
\pgfsetlinewidth{1.003750pt}%
\definecolor{currentstroke}{rgb}{0.121569,0.466667,0.705882}%
\pgfsetstrokecolor{currentstroke}%
\pgfsetdash{}{0pt}%
\pgfpathmoveto{\pgfqpoint{2.795882in}{2.892169in}}%
\pgfpathcurveto{\pgfqpoint{2.806932in}{2.892169in}}{\pgfqpoint{2.817531in}{2.896559in}}{\pgfqpoint{2.825345in}{2.904372in}}%
\pgfpathcurveto{\pgfqpoint{2.833159in}{2.912186in}}{\pgfqpoint{2.837549in}{2.922785in}}{\pgfqpoint{2.837549in}{2.933835in}}%
\pgfpathcurveto{\pgfqpoint{2.837549in}{2.944885in}}{\pgfqpoint{2.833159in}{2.955484in}}{\pgfqpoint{2.825345in}{2.963298in}}%
\pgfpathcurveto{\pgfqpoint{2.817531in}{2.971112in}}{\pgfqpoint{2.806932in}{2.975502in}}{\pgfqpoint{2.795882in}{2.975502in}}%
\pgfpathcurveto{\pgfqpoint{2.784832in}{2.975502in}}{\pgfqpoint{2.774233in}{2.971112in}}{\pgfqpoint{2.766419in}{2.963298in}}%
\pgfpathcurveto{\pgfqpoint{2.758606in}{2.955484in}}{\pgfqpoint{2.754216in}{2.944885in}}{\pgfqpoint{2.754216in}{2.933835in}}%
\pgfpathcurveto{\pgfqpoint{2.754216in}{2.922785in}}{\pgfqpoint{2.758606in}{2.912186in}}{\pgfqpoint{2.766419in}{2.904372in}}%
\pgfpathcurveto{\pgfqpoint{2.774233in}{2.896559in}}{\pgfqpoint{2.784832in}{2.892169in}}{\pgfqpoint{2.795882in}{2.892169in}}%
\pgfpathclose%
\pgfusepath{stroke,fill}%
\end{pgfscope}%
\begin{pgfscope}%
\pgfpathrectangle{\pgfqpoint{0.600000in}{0.600000in}}{\pgfqpoint{3.900000in}{3.900000in}}%
\pgfusepath{clip}%
\pgfsetbuttcap%
\pgfsetroundjoin%
\definecolor{currentfill}{rgb}{0.121569,0.466667,0.705882}%
\pgfsetfillcolor{currentfill}%
\pgfsetlinewidth{1.003750pt}%
\definecolor{currentstroke}{rgb}{0.121569,0.466667,0.705882}%
\pgfsetstrokecolor{currentstroke}%
\pgfsetdash{}{0pt}%
\pgfpathmoveto{\pgfqpoint{2.167665in}{1.319039in}}%
\pgfpathcurveto{\pgfqpoint{2.178715in}{1.319039in}}{\pgfqpoint{2.189314in}{1.323430in}}{\pgfqpoint{2.197127in}{1.331243in}}%
\pgfpathcurveto{\pgfqpoint{2.204941in}{1.339057in}}{\pgfqpoint{2.209331in}{1.349656in}}{\pgfqpoint{2.209331in}{1.360706in}}%
\pgfpathcurveto{\pgfqpoint{2.209331in}{1.371756in}}{\pgfqpoint{2.204941in}{1.382355in}}{\pgfqpoint{2.197127in}{1.390169in}}%
\pgfpathcurveto{\pgfqpoint{2.189314in}{1.397983in}}{\pgfqpoint{2.178715in}{1.402373in}}{\pgfqpoint{2.167665in}{1.402373in}}%
\pgfpathcurveto{\pgfqpoint{2.156614in}{1.402373in}}{\pgfqpoint{2.146015in}{1.397983in}}{\pgfqpoint{2.138202in}{1.390169in}}%
\pgfpathcurveto{\pgfqpoint{2.130388in}{1.382355in}}{\pgfqpoint{2.125998in}{1.371756in}}{\pgfqpoint{2.125998in}{1.360706in}}%
\pgfpathcurveto{\pgfqpoint{2.125998in}{1.349656in}}{\pgfqpoint{2.130388in}{1.339057in}}{\pgfqpoint{2.138202in}{1.331243in}}%
\pgfpathcurveto{\pgfqpoint{2.146015in}{1.323430in}}{\pgfqpoint{2.156614in}{1.319039in}}{\pgfqpoint{2.167665in}{1.319039in}}%
\pgfpathclose%
\pgfusepath{stroke,fill}%
\end{pgfscope}%
\begin{pgfscope}%
\pgfpathrectangle{\pgfqpoint{0.600000in}{0.600000in}}{\pgfqpoint{3.900000in}{3.900000in}}%
\pgfusepath{clip}%
\pgfsetbuttcap%
\pgfsetroundjoin%
\definecolor{currentfill}{rgb}{0.121569,0.466667,0.705882}%
\pgfsetfillcolor{currentfill}%
\pgfsetlinewidth{1.003750pt}%
\definecolor{currentstroke}{rgb}{0.121569,0.466667,0.705882}%
\pgfsetstrokecolor{currentstroke}%
\pgfsetdash{}{0pt}%
\pgfpathmoveto{\pgfqpoint{2.202699in}{1.794113in}}%
\pgfpathcurveto{\pgfqpoint{2.213749in}{1.794113in}}{\pgfqpoint{2.224349in}{1.798504in}}{\pgfqpoint{2.232162in}{1.806317in}}%
\pgfpathcurveto{\pgfqpoint{2.239976in}{1.814131in}}{\pgfqpoint{2.244366in}{1.824730in}}{\pgfqpoint{2.244366in}{1.835780in}}%
\pgfpathcurveto{\pgfqpoint{2.244366in}{1.846830in}}{\pgfqpoint{2.239976in}{1.857429in}}{\pgfqpoint{2.232162in}{1.865243in}}%
\pgfpathcurveto{\pgfqpoint{2.224349in}{1.873056in}}{\pgfqpoint{2.213749in}{1.877447in}}{\pgfqpoint{2.202699in}{1.877447in}}%
\pgfpathcurveto{\pgfqpoint{2.191649in}{1.877447in}}{\pgfqpoint{2.181050in}{1.873056in}}{\pgfqpoint{2.173237in}{1.865243in}}%
\pgfpathcurveto{\pgfqpoint{2.165423in}{1.857429in}}{\pgfqpoint{2.161033in}{1.846830in}}{\pgfqpoint{2.161033in}{1.835780in}}%
\pgfpathcurveto{\pgfqpoint{2.161033in}{1.824730in}}{\pgfqpoint{2.165423in}{1.814131in}}{\pgfqpoint{2.173237in}{1.806317in}}%
\pgfpathcurveto{\pgfqpoint{2.181050in}{1.798504in}}{\pgfqpoint{2.191649in}{1.794113in}}{\pgfqpoint{2.202699in}{1.794113in}}%
\pgfpathclose%
\pgfusepath{stroke,fill}%
\end{pgfscope}%
\begin{pgfscope}%
\pgfpathrectangle{\pgfqpoint{0.600000in}{0.600000in}}{\pgfqpoint{3.900000in}{3.900000in}}%
\pgfusepath{clip}%
\pgfsetbuttcap%
\pgfsetroundjoin%
\definecolor{currentfill}{rgb}{0.121569,0.466667,0.705882}%
\pgfsetfillcolor{currentfill}%
\pgfsetlinewidth{1.003750pt}%
\definecolor{currentstroke}{rgb}{0.121569,0.466667,0.705882}%
\pgfsetstrokecolor{currentstroke}%
\pgfsetdash{}{0pt}%
\pgfpathmoveto{\pgfqpoint{2.056682in}{2.675053in}}%
\pgfpathcurveto{\pgfqpoint{2.067732in}{2.675053in}}{\pgfqpoint{2.078331in}{2.679443in}}{\pgfqpoint{2.086145in}{2.687257in}}%
\pgfpathcurveto{\pgfqpoint{2.093959in}{2.695071in}}{\pgfqpoint{2.098349in}{2.705670in}}{\pgfqpoint{2.098349in}{2.716720in}}%
\pgfpathcurveto{\pgfqpoint{2.098349in}{2.727770in}}{\pgfqpoint{2.093959in}{2.738369in}}{\pgfqpoint{2.086145in}{2.746183in}}%
\pgfpathcurveto{\pgfqpoint{2.078331in}{2.753996in}}{\pgfqpoint{2.067732in}{2.758387in}}{\pgfqpoint{2.056682in}{2.758387in}}%
\pgfpathcurveto{\pgfqpoint{2.045632in}{2.758387in}}{\pgfqpoint{2.035033in}{2.753996in}}{\pgfqpoint{2.027219in}{2.746183in}}%
\pgfpathcurveto{\pgfqpoint{2.019406in}{2.738369in}}{\pgfqpoint{2.015015in}{2.727770in}}{\pgfqpoint{2.015015in}{2.716720in}}%
\pgfpathcurveto{\pgfqpoint{2.015015in}{2.705670in}}{\pgfqpoint{2.019406in}{2.695071in}}{\pgfqpoint{2.027219in}{2.687257in}}%
\pgfpathcurveto{\pgfqpoint{2.035033in}{2.679443in}}{\pgfqpoint{2.045632in}{2.675053in}}{\pgfqpoint{2.056682in}{2.675053in}}%
\pgfpathclose%
\pgfusepath{stroke,fill}%
\end{pgfscope}%
\begin{pgfscope}%
\pgfpathrectangle{\pgfqpoint{0.600000in}{0.600000in}}{\pgfqpoint{3.900000in}{3.900000in}}%
\pgfusepath{clip}%
\pgfsetbuttcap%
\pgfsetroundjoin%
\definecolor{currentfill}{rgb}{0.121569,0.466667,0.705882}%
\pgfsetfillcolor{currentfill}%
\pgfsetlinewidth{1.003750pt}%
\definecolor{currentstroke}{rgb}{0.121569,0.466667,0.705882}%
\pgfsetstrokecolor{currentstroke}%
\pgfsetdash{}{0pt}%
\pgfpathmoveto{\pgfqpoint{3.772428in}{2.394056in}}%
\pgfpathcurveto{\pgfqpoint{3.783478in}{2.394056in}}{\pgfqpoint{3.794077in}{2.398446in}}{\pgfqpoint{3.801891in}{2.406260in}}%
\pgfpathcurveto{\pgfqpoint{3.809704in}{2.414073in}}{\pgfqpoint{3.814095in}{2.424672in}}{\pgfqpoint{3.814095in}{2.435723in}}%
\pgfpathcurveto{\pgfqpoint{3.814095in}{2.446773in}}{\pgfqpoint{3.809704in}{2.457372in}}{\pgfqpoint{3.801891in}{2.465185in}}%
\pgfpathcurveto{\pgfqpoint{3.794077in}{2.472999in}}{\pgfqpoint{3.783478in}{2.477389in}}{\pgfqpoint{3.772428in}{2.477389in}}%
\pgfpathcurveto{\pgfqpoint{3.761378in}{2.477389in}}{\pgfqpoint{3.750779in}{2.472999in}}{\pgfqpoint{3.742965in}{2.465185in}}%
\pgfpathcurveto{\pgfqpoint{3.735152in}{2.457372in}}{\pgfqpoint{3.730761in}{2.446773in}}{\pgfqpoint{3.730761in}{2.435723in}}%
\pgfpathcurveto{\pgfqpoint{3.730761in}{2.424672in}}{\pgfqpoint{3.735152in}{2.414073in}}{\pgfqpoint{3.742965in}{2.406260in}}%
\pgfpathcurveto{\pgfqpoint{3.750779in}{2.398446in}}{\pgfqpoint{3.761378in}{2.394056in}}{\pgfqpoint{3.772428in}{2.394056in}}%
\pgfpathclose%
\pgfusepath{stroke,fill}%
\end{pgfscope}%
\begin{pgfscope}%
\pgfpathrectangle{\pgfqpoint{0.600000in}{0.600000in}}{\pgfqpoint{3.900000in}{3.900000in}}%
\pgfusepath{clip}%
\pgfsetbuttcap%
\pgfsetroundjoin%
\definecolor{currentfill}{rgb}{0.121569,0.466667,0.705882}%
\pgfsetfillcolor{currentfill}%
\pgfsetlinewidth{1.003750pt}%
\definecolor{currentstroke}{rgb}{0.121569,0.466667,0.705882}%
\pgfsetstrokecolor{currentstroke}%
\pgfsetdash{}{0pt}%
\pgfpathmoveto{\pgfqpoint{2.598766in}{3.283507in}}%
\pgfpathcurveto{\pgfqpoint{2.609816in}{3.283507in}}{\pgfqpoint{2.620415in}{3.287897in}}{\pgfqpoint{2.628228in}{3.295711in}}%
\pgfpathcurveto{\pgfqpoint{2.636042in}{3.303524in}}{\pgfqpoint{2.640432in}{3.314123in}}{\pgfqpoint{2.640432in}{3.325173in}}%
\pgfpathcurveto{\pgfqpoint{2.640432in}{3.336224in}}{\pgfqpoint{2.636042in}{3.346823in}}{\pgfqpoint{2.628228in}{3.354636in}}%
\pgfpathcurveto{\pgfqpoint{2.620415in}{3.362450in}}{\pgfqpoint{2.609816in}{3.366840in}}{\pgfqpoint{2.598766in}{3.366840in}}%
\pgfpathcurveto{\pgfqpoint{2.587715in}{3.366840in}}{\pgfqpoint{2.577116in}{3.362450in}}{\pgfqpoint{2.569303in}{3.354636in}}%
\pgfpathcurveto{\pgfqpoint{2.561489in}{3.346823in}}{\pgfqpoint{2.557099in}{3.336224in}}{\pgfqpoint{2.557099in}{3.325173in}}%
\pgfpathcurveto{\pgfqpoint{2.557099in}{3.314123in}}{\pgfqpoint{2.561489in}{3.303524in}}{\pgfqpoint{2.569303in}{3.295711in}}%
\pgfpathcurveto{\pgfqpoint{2.577116in}{3.287897in}}{\pgfqpoint{2.587715in}{3.283507in}}{\pgfqpoint{2.598766in}{3.283507in}}%
\pgfpathclose%
\pgfusepath{stroke,fill}%
\end{pgfscope}%
\begin{pgfscope}%
\pgfpathrectangle{\pgfqpoint{0.600000in}{0.600000in}}{\pgfqpoint{3.900000in}{3.900000in}}%
\pgfusepath{clip}%
\pgfsetbuttcap%
\pgfsetroundjoin%
\definecolor{currentfill}{rgb}{0.121569,0.466667,0.705882}%
\pgfsetfillcolor{currentfill}%
\pgfsetlinewidth{1.003750pt}%
\definecolor{currentstroke}{rgb}{0.121569,0.466667,0.705882}%
\pgfsetstrokecolor{currentstroke}%
\pgfsetdash{}{0pt}%
\pgfpathmoveto{\pgfqpoint{1.701297in}{2.223999in}}%
\pgfpathcurveto{\pgfqpoint{1.712347in}{2.223999in}}{\pgfqpoint{1.722946in}{2.228389in}}{\pgfqpoint{1.730760in}{2.236203in}}%
\pgfpathcurveto{\pgfqpoint{1.738574in}{2.244017in}}{\pgfqpoint{1.742964in}{2.254616in}}{\pgfqpoint{1.742964in}{2.265666in}}%
\pgfpathcurveto{\pgfqpoint{1.742964in}{2.276716in}}{\pgfqpoint{1.738574in}{2.287315in}}{\pgfqpoint{1.730760in}{2.295129in}}%
\pgfpathcurveto{\pgfqpoint{1.722946in}{2.302942in}}{\pgfqpoint{1.712347in}{2.307332in}}{\pgfqpoint{1.701297in}{2.307332in}}%
\pgfpathcurveto{\pgfqpoint{1.690247in}{2.307332in}}{\pgfqpoint{1.679648in}{2.302942in}}{\pgfqpoint{1.671834in}{2.295129in}}%
\pgfpathcurveto{\pgfqpoint{1.664021in}{2.287315in}}{\pgfqpoint{1.659631in}{2.276716in}}{\pgfqpoint{1.659631in}{2.265666in}}%
\pgfpathcurveto{\pgfqpoint{1.659631in}{2.254616in}}{\pgfqpoint{1.664021in}{2.244017in}}{\pgfqpoint{1.671834in}{2.236203in}}%
\pgfpathcurveto{\pgfqpoint{1.679648in}{2.228389in}}{\pgfqpoint{1.690247in}{2.223999in}}{\pgfqpoint{1.701297in}{2.223999in}}%
\pgfpathclose%
\pgfusepath{stroke,fill}%
\end{pgfscope}%
\begin{pgfscope}%
\pgfpathrectangle{\pgfqpoint{0.600000in}{0.600000in}}{\pgfqpoint{3.900000in}{3.900000in}}%
\pgfusepath{clip}%
\pgfsetbuttcap%
\pgfsetroundjoin%
\definecolor{currentfill}{rgb}{0.121569,0.466667,0.705882}%
\pgfsetfillcolor{currentfill}%
\pgfsetlinewidth{1.003750pt}%
\definecolor{currentstroke}{rgb}{0.121569,0.466667,0.705882}%
\pgfsetstrokecolor{currentstroke}%
\pgfsetdash{}{0pt}%
\pgfpathmoveto{\pgfqpoint{2.357867in}{2.377657in}}%
\pgfpathcurveto{\pgfqpoint{2.368917in}{2.377657in}}{\pgfqpoint{2.379516in}{2.382048in}}{\pgfqpoint{2.387330in}{2.389861in}}%
\pgfpathcurveto{\pgfqpoint{2.395143in}{2.397675in}}{\pgfqpoint{2.399534in}{2.408274in}}{\pgfqpoint{2.399534in}{2.419324in}}%
\pgfpathcurveto{\pgfqpoint{2.399534in}{2.430374in}}{\pgfqpoint{2.395143in}{2.440973in}}{\pgfqpoint{2.387330in}{2.448787in}}%
\pgfpathcurveto{\pgfqpoint{2.379516in}{2.456601in}}{\pgfqpoint{2.368917in}{2.460991in}}{\pgfqpoint{2.357867in}{2.460991in}}%
\pgfpathcurveto{\pgfqpoint{2.346817in}{2.460991in}}{\pgfqpoint{2.336218in}{2.456601in}}{\pgfqpoint{2.328404in}{2.448787in}}%
\pgfpathcurveto{\pgfqpoint{2.320591in}{2.440973in}}{\pgfqpoint{2.316200in}{2.430374in}}{\pgfqpoint{2.316200in}{2.419324in}}%
\pgfpathcurveto{\pgfqpoint{2.316200in}{2.408274in}}{\pgfqpoint{2.320591in}{2.397675in}}{\pgfqpoint{2.328404in}{2.389861in}}%
\pgfpathcurveto{\pgfqpoint{2.336218in}{2.382048in}}{\pgfqpoint{2.346817in}{2.377657in}}{\pgfqpoint{2.357867in}{2.377657in}}%
\pgfpathclose%
\pgfusepath{stroke,fill}%
\end{pgfscope}%
\begin{pgfscope}%
\pgfpathrectangle{\pgfqpoint{0.600000in}{0.600000in}}{\pgfqpoint{3.900000in}{3.900000in}}%
\pgfusepath{clip}%
\pgfsetbuttcap%
\pgfsetroundjoin%
\definecolor{currentfill}{rgb}{0.121569,0.466667,0.705882}%
\pgfsetfillcolor{currentfill}%
\pgfsetlinewidth{1.003750pt}%
\definecolor{currentstroke}{rgb}{0.121569,0.466667,0.705882}%
\pgfsetstrokecolor{currentstroke}%
\pgfsetdash{}{0pt}%
\pgfpathmoveto{\pgfqpoint{2.998011in}{2.598426in}}%
\pgfpathcurveto{\pgfqpoint{3.009061in}{2.598426in}}{\pgfqpoint{3.019660in}{2.602817in}}{\pgfqpoint{3.027474in}{2.610630in}}%
\pgfpathcurveto{\pgfqpoint{3.035287in}{2.618444in}}{\pgfqpoint{3.039678in}{2.629043in}}{\pgfqpoint{3.039678in}{2.640093in}}%
\pgfpathcurveto{\pgfqpoint{3.039678in}{2.651143in}}{\pgfqpoint{3.035287in}{2.661742in}}{\pgfqpoint{3.027474in}{2.669556in}}%
\pgfpathcurveto{\pgfqpoint{3.019660in}{2.677369in}}{\pgfqpoint{3.009061in}{2.681760in}}{\pgfqpoint{2.998011in}{2.681760in}}%
\pgfpathcurveto{\pgfqpoint{2.986961in}{2.681760in}}{\pgfqpoint{2.976362in}{2.677369in}}{\pgfqpoint{2.968548in}{2.669556in}}%
\pgfpathcurveto{\pgfqpoint{2.960735in}{2.661742in}}{\pgfqpoint{2.956344in}{2.651143in}}{\pgfqpoint{2.956344in}{2.640093in}}%
\pgfpathcurveto{\pgfqpoint{2.956344in}{2.629043in}}{\pgfqpoint{2.960735in}{2.618444in}}{\pgfqpoint{2.968548in}{2.610630in}}%
\pgfpathcurveto{\pgfqpoint{2.976362in}{2.602817in}}{\pgfqpoint{2.986961in}{2.598426in}}{\pgfqpoint{2.998011in}{2.598426in}}%
\pgfpathclose%
\pgfusepath{stroke,fill}%
\end{pgfscope}%
\begin{pgfscope}%
\pgfpathrectangle{\pgfqpoint{0.600000in}{0.600000in}}{\pgfqpoint{3.900000in}{3.900000in}}%
\pgfusepath{clip}%
\pgfsetbuttcap%
\pgfsetroundjoin%
\definecolor{currentfill}{rgb}{0.121569,0.466667,0.705882}%
\pgfsetfillcolor{currentfill}%
\pgfsetlinewidth{1.003750pt}%
\definecolor{currentstroke}{rgb}{0.121569,0.466667,0.705882}%
\pgfsetstrokecolor{currentstroke}%
\pgfsetdash{}{0pt}%
\pgfpathmoveto{\pgfqpoint{2.864999in}{2.792074in}}%
\pgfpathcurveto{\pgfqpoint{2.876049in}{2.792074in}}{\pgfqpoint{2.886648in}{2.796464in}}{\pgfqpoint{2.894461in}{2.804278in}}%
\pgfpathcurveto{\pgfqpoint{2.902275in}{2.812092in}}{\pgfqpoint{2.906665in}{2.822691in}}{\pgfqpoint{2.906665in}{2.833741in}}%
\pgfpathcurveto{\pgfqpoint{2.906665in}{2.844791in}}{\pgfqpoint{2.902275in}{2.855390in}}{\pgfqpoint{2.894461in}{2.863203in}}%
\pgfpathcurveto{\pgfqpoint{2.886648in}{2.871017in}}{\pgfqpoint{2.876049in}{2.875407in}}{\pgfqpoint{2.864999in}{2.875407in}}%
\pgfpathcurveto{\pgfqpoint{2.853948in}{2.875407in}}{\pgfqpoint{2.843349in}{2.871017in}}{\pgfqpoint{2.835536in}{2.863203in}}%
\pgfpathcurveto{\pgfqpoint{2.827722in}{2.855390in}}{\pgfqpoint{2.823332in}{2.844791in}}{\pgfqpoint{2.823332in}{2.833741in}}%
\pgfpathcurveto{\pgfqpoint{2.823332in}{2.822691in}}{\pgfqpoint{2.827722in}{2.812092in}}{\pgfqpoint{2.835536in}{2.804278in}}%
\pgfpathcurveto{\pgfqpoint{2.843349in}{2.796464in}}{\pgfqpoint{2.853948in}{2.792074in}}{\pgfqpoint{2.864999in}{2.792074in}}%
\pgfpathclose%
\pgfusepath{stroke,fill}%
\end{pgfscope}%
\begin{pgfscope}%
\pgfpathrectangle{\pgfqpoint{0.600000in}{0.600000in}}{\pgfqpoint{3.900000in}{3.900000in}}%
\pgfusepath{clip}%
\pgfsetbuttcap%
\pgfsetroundjoin%
\definecolor{currentfill}{rgb}{0.121569,0.466667,0.705882}%
\pgfsetfillcolor{currentfill}%
\pgfsetlinewidth{1.003750pt}%
\definecolor{currentstroke}{rgb}{0.121569,0.466667,0.705882}%
\pgfsetstrokecolor{currentstroke}%
\pgfsetdash{}{0pt}%
\pgfpathmoveto{\pgfqpoint{3.211485in}{2.006794in}}%
\pgfpathcurveto{\pgfqpoint{3.222535in}{2.006794in}}{\pgfqpoint{3.233134in}{2.011185in}}{\pgfqpoint{3.240948in}{2.018998in}}%
\pgfpathcurveto{\pgfqpoint{3.248761in}{2.026812in}}{\pgfqpoint{3.253152in}{2.037411in}}{\pgfqpoint{3.253152in}{2.048461in}}%
\pgfpathcurveto{\pgfqpoint{3.253152in}{2.059511in}}{\pgfqpoint{3.248761in}{2.070110in}}{\pgfqpoint{3.240948in}{2.077924in}}%
\pgfpathcurveto{\pgfqpoint{3.233134in}{2.085737in}}{\pgfqpoint{3.222535in}{2.090128in}}{\pgfqpoint{3.211485in}{2.090128in}}%
\pgfpathcurveto{\pgfqpoint{3.200435in}{2.090128in}}{\pgfqpoint{3.189836in}{2.085737in}}{\pgfqpoint{3.182022in}{2.077924in}}%
\pgfpathcurveto{\pgfqpoint{3.174209in}{2.070110in}}{\pgfqpoint{3.169818in}{2.059511in}}{\pgfqpoint{3.169818in}{2.048461in}}%
\pgfpathcurveto{\pgfqpoint{3.169818in}{2.037411in}}{\pgfqpoint{3.174209in}{2.026812in}}{\pgfqpoint{3.182022in}{2.018998in}}%
\pgfpathcurveto{\pgfqpoint{3.189836in}{2.011185in}}{\pgfqpoint{3.200435in}{2.006794in}}{\pgfqpoint{3.211485in}{2.006794in}}%
\pgfpathclose%
\pgfusepath{stroke,fill}%
\end{pgfscope}%
\begin{pgfscope}%
\pgfpathrectangle{\pgfqpoint{0.600000in}{0.600000in}}{\pgfqpoint{3.900000in}{3.900000in}}%
\pgfusepath{clip}%
\pgfsetbuttcap%
\pgfsetroundjoin%
\definecolor{currentfill}{rgb}{0.121569,0.466667,0.705882}%
\pgfsetfillcolor{currentfill}%
\pgfsetlinewidth{1.003750pt}%
\definecolor{currentstroke}{rgb}{0.121569,0.466667,0.705882}%
\pgfsetstrokecolor{currentstroke}%
\pgfsetdash{}{0pt}%
\pgfpathmoveto{\pgfqpoint{2.814335in}{2.905679in}}%
\pgfpathcurveto{\pgfqpoint{2.825385in}{2.905679in}}{\pgfqpoint{2.835984in}{2.910069in}}{\pgfqpoint{2.843798in}{2.917883in}}%
\pgfpathcurveto{\pgfqpoint{2.851612in}{2.925697in}}{\pgfqpoint{2.856002in}{2.936296in}}{\pgfqpoint{2.856002in}{2.947346in}}%
\pgfpathcurveto{\pgfqpoint{2.856002in}{2.958396in}}{\pgfqpoint{2.851612in}{2.968995in}}{\pgfqpoint{2.843798in}{2.976809in}}%
\pgfpathcurveto{\pgfqpoint{2.835984in}{2.984622in}}{\pgfqpoint{2.825385in}{2.989012in}}{\pgfqpoint{2.814335in}{2.989012in}}%
\pgfpathcurveto{\pgfqpoint{2.803285in}{2.989012in}}{\pgfqpoint{2.792686in}{2.984622in}}{\pgfqpoint{2.784872in}{2.976809in}}%
\pgfpathcurveto{\pgfqpoint{2.777059in}{2.968995in}}{\pgfqpoint{2.772669in}{2.958396in}}{\pgfqpoint{2.772669in}{2.947346in}}%
\pgfpathcurveto{\pgfqpoint{2.772669in}{2.936296in}}{\pgfqpoint{2.777059in}{2.925697in}}{\pgfqpoint{2.784872in}{2.917883in}}%
\pgfpathcurveto{\pgfqpoint{2.792686in}{2.910069in}}{\pgfqpoint{2.803285in}{2.905679in}}{\pgfqpoint{2.814335in}{2.905679in}}%
\pgfpathclose%
\pgfusepath{stroke,fill}%
\end{pgfscope}%
\begin{pgfscope}%
\pgfpathrectangle{\pgfqpoint{0.600000in}{0.600000in}}{\pgfqpoint{3.900000in}{3.900000in}}%
\pgfusepath{clip}%
\pgfsetbuttcap%
\pgfsetroundjoin%
\definecolor{currentfill}{rgb}{0.121569,0.466667,0.705882}%
\pgfsetfillcolor{currentfill}%
\pgfsetlinewidth{1.003750pt}%
\definecolor{currentstroke}{rgb}{0.121569,0.466667,0.705882}%
\pgfsetstrokecolor{currentstroke}%
\pgfsetdash{}{0pt}%
\pgfpathmoveto{\pgfqpoint{1.949587in}{1.843121in}}%
\pgfpathcurveto{\pgfqpoint{1.960637in}{1.843121in}}{\pgfqpoint{1.971236in}{1.847511in}}{\pgfqpoint{1.979050in}{1.855325in}}%
\pgfpathcurveto{\pgfqpoint{1.986863in}{1.863138in}}{\pgfqpoint{1.991254in}{1.873738in}}{\pgfqpoint{1.991254in}{1.884788in}}%
\pgfpathcurveto{\pgfqpoint{1.991254in}{1.895838in}}{\pgfqpoint{1.986863in}{1.906437in}}{\pgfqpoint{1.979050in}{1.914250in}}%
\pgfpathcurveto{\pgfqpoint{1.971236in}{1.922064in}}{\pgfqpoint{1.960637in}{1.926454in}}{\pgfqpoint{1.949587in}{1.926454in}}%
\pgfpathcurveto{\pgfqpoint{1.938537in}{1.926454in}}{\pgfqpoint{1.927938in}{1.922064in}}{\pgfqpoint{1.920124in}{1.914250in}}%
\pgfpathcurveto{\pgfqpoint{1.912311in}{1.906437in}}{\pgfqpoint{1.907920in}{1.895838in}}{\pgfqpoint{1.907920in}{1.884788in}}%
\pgfpathcurveto{\pgfqpoint{1.907920in}{1.873738in}}{\pgfqpoint{1.912311in}{1.863138in}}{\pgfqpoint{1.920124in}{1.855325in}}%
\pgfpathcurveto{\pgfqpoint{1.927938in}{1.847511in}}{\pgfqpoint{1.938537in}{1.843121in}}{\pgfqpoint{1.949587in}{1.843121in}}%
\pgfpathclose%
\pgfusepath{stroke,fill}%
\end{pgfscope}%
\begin{pgfscope}%
\pgfpathrectangle{\pgfqpoint{0.600000in}{0.600000in}}{\pgfqpoint{3.900000in}{3.900000in}}%
\pgfusepath{clip}%
\pgfsetbuttcap%
\pgfsetroundjoin%
\definecolor{currentfill}{rgb}{0.121569,0.466667,0.705882}%
\pgfsetfillcolor{currentfill}%
\pgfsetlinewidth{1.003750pt}%
\definecolor{currentstroke}{rgb}{0.121569,0.466667,0.705882}%
\pgfsetstrokecolor{currentstroke}%
\pgfsetdash{}{0pt}%
\pgfpathmoveto{\pgfqpoint{1.485956in}{2.711652in}}%
\pgfpathcurveto{\pgfqpoint{1.497006in}{2.711652in}}{\pgfqpoint{1.507605in}{2.716043in}}{\pgfqpoint{1.515418in}{2.723856in}}%
\pgfpathcurveto{\pgfqpoint{1.523232in}{2.731670in}}{\pgfqpoint{1.527622in}{2.742269in}}{\pgfqpoint{1.527622in}{2.753319in}}%
\pgfpathcurveto{\pgfqpoint{1.527622in}{2.764369in}}{\pgfqpoint{1.523232in}{2.774968in}}{\pgfqpoint{1.515418in}{2.782782in}}%
\pgfpathcurveto{\pgfqpoint{1.507605in}{2.790596in}}{\pgfqpoint{1.497006in}{2.794986in}}{\pgfqpoint{1.485956in}{2.794986in}}%
\pgfpathcurveto{\pgfqpoint{1.474905in}{2.794986in}}{\pgfqpoint{1.464306in}{2.790596in}}{\pgfqpoint{1.456493in}{2.782782in}}%
\pgfpathcurveto{\pgfqpoint{1.448679in}{2.774968in}}{\pgfqpoint{1.444289in}{2.764369in}}{\pgfqpoint{1.444289in}{2.753319in}}%
\pgfpathcurveto{\pgfqpoint{1.444289in}{2.742269in}}{\pgfqpoint{1.448679in}{2.731670in}}{\pgfqpoint{1.456493in}{2.723856in}}%
\pgfpathcurveto{\pgfqpoint{1.464306in}{2.716043in}}{\pgfqpoint{1.474905in}{2.711652in}}{\pgfqpoint{1.485956in}{2.711652in}}%
\pgfpathclose%
\pgfusepath{stroke,fill}%
\end{pgfscope}%
\begin{pgfscope}%
\pgfpathrectangle{\pgfqpoint{0.600000in}{0.600000in}}{\pgfqpoint{3.900000in}{3.900000in}}%
\pgfusepath{clip}%
\pgfsetbuttcap%
\pgfsetroundjoin%
\definecolor{currentfill}{rgb}{0.121569,0.466667,0.705882}%
\pgfsetfillcolor{currentfill}%
\pgfsetlinewidth{1.003750pt}%
\definecolor{currentstroke}{rgb}{0.121569,0.466667,0.705882}%
\pgfsetstrokecolor{currentstroke}%
\pgfsetdash{}{0pt}%
\pgfpathmoveto{\pgfqpoint{2.568371in}{2.344890in}}%
\pgfpathcurveto{\pgfqpoint{2.579421in}{2.344890in}}{\pgfqpoint{2.590020in}{2.349280in}}{\pgfqpoint{2.597834in}{2.357094in}}%
\pgfpathcurveto{\pgfqpoint{2.605648in}{2.364908in}}{\pgfqpoint{2.610038in}{2.375507in}}{\pgfqpoint{2.610038in}{2.386557in}}%
\pgfpathcurveto{\pgfqpoint{2.610038in}{2.397607in}}{\pgfqpoint{2.605648in}{2.408206in}}{\pgfqpoint{2.597834in}{2.416020in}}%
\pgfpathcurveto{\pgfqpoint{2.590020in}{2.423833in}}{\pgfqpoint{2.579421in}{2.428223in}}{\pgfqpoint{2.568371in}{2.428223in}}%
\pgfpathcurveto{\pgfqpoint{2.557321in}{2.428223in}}{\pgfqpoint{2.546722in}{2.423833in}}{\pgfqpoint{2.538908in}{2.416020in}}%
\pgfpathcurveto{\pgfqpoint{2.531095in}{2.408206in}}{\pgfqpoint{2.526705in}{2.397607in}}{\pgfqpoint{2.526705in}{2.386557in}}%
\pgfpathcurveto{\pgfqpoint{2.526705in}{2.375507in}}{\pgfqpoint{2.531095in}{2.364908in}}{\pgfqpoint{2.538908in}{2.357094in}}%
\pgfpathcurveto{\pgfqpoint{2.546722in}{2.349280in}}{\pgfqpoint{2.557321in}{2.344890in}}{\pgfqpoint{2.568371in}{2.344890in}}%
\pgfpathclose%
\pgfusepath{stroke,fill}%
\end{pgfscope}%
\begin{pgfscope}%
\pgfpathrectangle{\pgfqpoint{0.600000in}{0.600000in}}{\pgfqpoint{3.900000in}{3.900000in}}%
\pgfusepath{clip}%
\pgfsetbuttcap%
\pgfsetroundjoin%
\definecolor{currentfill}{rgb}{0.121569,0.466667,0.705882}%
\pgfsetfillcolor{currentfill}%
\pgfsetlinewidth{1.003750pt}%
\definecolor{currentstroke}{rgb}{0.121569,0.466667,0.705882}%
\pgfsetstrokecolor{currentstroke}%
\pgfsetdash{}{0pt}%
\pgfpathmoveto{\pgfqpoint{0.957648in}{2.497043in}}%
\pgfpathcurveto{\pgfqpoint{0.968698in}{2.497043in}}{\pgfqpoint{0.979297in}{2.501433in}}{\pgfqpoint{0.987111in}{2.509247in}}%
\pgfpathcurveto{\pgfqpoint{0.994924in}{2.517060in}}{\pgfqpoint{0.999315in}{2.527659in}}{\pgfqpoint{0.999315in}{2.538709in}}%
\pgfpathcurveto{\pgfqpoint{0.999315in}{2.549759in}}{\pgfqpoint{0.994924in}{2.560359in}}{\pgfqpoint{0.987111in}{2.568172in}}%
\pgfpathcurveto{\pgfqpoint{0.979297in}{2.575986in}}{\pgfqpoint{0.968698in}{2.580376in}}{\pgfqpoint{0.957648in}{2.580376in}}%
\pgfpathcurveto{\pgfqpoint{0.946598in}{2.580376in}}{\pgfqpoint{0.935999in}{2.575986in}}{\pgfqpoint{0.928185in}{2.568172in}}%
\pgfpathcurveto{\pgfqpoint{0.920372in}{2.560359in}}{\pgfqpoint{0.915981in}{2.549759in}}{\pgfqpoint{0.915981in}{2.538709in}}%
\pgfpathcurveto{\pgfqpoint{0.915981in}{2.527659in}}{\pgfqpoint{0.920372in}{2.517060in}}{\pgfqpoint{0.928185in}{2.509247in}}%
\pgfpathcurveto{\pgfqpoint{0.935999in}{2.501433in}}{\pgfqpoint{0.946598in}{2.497043in}}{\pgfqpoint{0.957648in}{2.497043in}}%
\pgfpathclose%
\pgfusepath{stroke,fill}%
\end{pgfscope}%
\begin{pgfscope}%
\pgfpathrectangle{\pgfqpoint{0.600000in}{0.600000in}}{\pgfqpoint{3.900000in}{3.900000in}}%
\pgfusepath{clip}%
\pgfsetbuttcap%
\pgfsetroundjoin%
\definecolor{currentfill}{rgb}{0.121569,0.466667,0.705882}%
\pgfsetfillcolor{currentfill}%
\pgfsetlinewidth{1.003750pt}%
\definecolor{currentstroke}{rgb}{0.121569,0.466667,0.705882}%
\pgfsetstrokecolor{currentstroke}%
\pgfsetdash{}{0pt}%
\pgfpathmoveto{\pgfqpoint{1.693801in}{2.544567in}}%
\pgfpathcurveto{\pgfqpoint{1.704851in}{2.544567in}}{\pgfqpoint{1.715450in}{2.548957in}}{\pgfqpoint{1.723264in}{2.556771in}}%
\pgfpathcurveto{\pgfqpoint{1.731078in}{2.564585in}}{\pgfqpoint{1.735468in}{2.575184in}}{\pgfqpoint{1.735468in}{2.586234in}}%
\pgfpathcurveto{\pgfqpoint{1.735468in}{2.597284in}}{\pgfqpoint{1.731078in}{2.607883in}}{\pgfqpoint{1.723264in}{2.615696in}}%
\pgfpathcurveto{\pgfqpoint{1.715450in}{2.623510in}}{\pgfqpoint{1.704851in}{2.627900in}}{\pgfqpoint{1.693801in}{2.627900in}}%
\pgfpathcurveto{\pgfqpoint{1.682751in}{2.627900in}}{\pgfqpoint{1.672152in}{2.623510in}}{\pgfqpoint{1.664338in}{2.615696in}}%
\pgfpathcurveto{\pgfqpoint{1.656525in}{2.607883in}}{\pgfqpoint{1.652135in}{2.597284in}}{\pgfqpoint{1.652135in}{2.586234in}}%
\pgfpathcurveto{\pgfqpoint{1.652135in}{2.575184in}}{\pgfqpoint{1.656525in}{2.564585in}}{\pgfqpoint{1.664338in}{2.556771in}}%
\pgfpathcurveto{\pgfqpoint{1.672152in}{2.548957in}}{\pgfqpoint{1.682751in}{2.544567in}}{\pgfqpoint{1.693801in}{2.544567in}}%
\pgfpathclose%
\pgfusepath{stroke,fill}%
\end{pgfscope}%
\begin{pgfscope}%
\pgfpathrectangle{\pgfqpoint{0.600000in}{0.600000in}}{\pgfqpoint{3.900000in}{3.900000in}}%
\pgfusepath{clip}%
\pgfsetbuttcap%
\pgfsetroundjoin%
\definecolor{currentfill}{rgb}{0.121569,0.466667,0.705882}%
\pgfsetfillcolor{currentfill}%
\pgfsetlinewidth{1.003750pt}%
\definecolor{currentstroke}{rgb}{0.121569,0.466667,0.705882}%
\pgfsetstrokecolor{currentstroke}%
\pgfsetdash{}{0pt}%
\pgfpathmoveto{\pgfqpoint{1.830231in}{2.816593in}}%
\pgfpathcurveto{\pgfqpoint{1.841281in}{2.816593in}}{\pgfqpoint{1.851880in}{2.820983in}}{\pgfqpoint{1.859694in}{2.828797in}}%
\pgfpathcurveto{\pgfqpoint{1.867507in}{2.836611in}}{\pgfqpoint{1.871898in}{2.847210in}}{\pgfqpoint{1.871898in}{2.858260in}}%
\pgfpathcurveto{\pgfqpoint{1.871898in}{2.869310in}}{\pgfqpoint{1.867507in}{2.879909in}}{\pgfqpoint{1.859694in}{2.887723in}}%
\pgfpathcurveto{\pgfqpoint{1.851880in}{2.895536in}}{\pgfqpoint{1.841281in}{2.899926in}}{\pgfqpoint{1.830231in}{2.899926in}}%
\pgfpathcurveto{\pgfqpoint{1.819181in}{2.899926in}}{\pgfqpoint{1.808582in}{2.895536in}}{\pgfqpoint{1.800768in}{2.887723in}}%
\pgfpathcurveto{\pgfqpoint{1.792955in}{2.879909in}}{\pgfqpoint{1.788564in}{2.869310in}}{\pgfqpoint{1.788564in}{2.858260in}}%
\pgfpathcurveto{\pgfqpoint{1.788564in}{2.847210in}}{\pgfqpoint{1.792955in}{2.836611in}}{\pgfqpoint{1.800768in}{2.828797in}}%
\pgfpathcurveto{\pgfqpoint{1.808582in}{2.820983in}}{\pgfqpoint{1.819181in}{2.816593in}}{\pgfqpoint{1.830231in}{2.816593in}}%
\pgfpathclose%
\pgfusepath{stroke,fill}%
\end{pgfscope}%
\begin{pgfscope}%
\pgfpathrectangle{\pgfqpoint{0.600000in}{0.600000in}}{\pgfqpoint{3.900000in}{3.900000in}}%
\pgfusepath{clip}%
\pgfsetbuttcap%
\pgfsetroundjoin%
\definecolor{currentfill}{rgb}{0.121569,0.466667,0.705882}%
\pgfsetfillcolor{currentfill}%
\pgfsetlinewidth{1.003750pt}%
\definecolor{currentstroke}{rgb}{0.121569,0.466667,0.705882}%
\pgfsetstrokecolor{currentstroke}%
\pgfsetdash{}{0pt}%
\pgfpathmoveto{\pgfqpoint{2.422612in}{2.768336in}}%
\pgfpathcurveto{\pgfqpoint{2.433662in}{2.768336in}}{\pgfqpoint{2.444262in}{2.772727in}}{\pgfqpoint{2.452075in}{2.780540in}}%
\pgfpathcurveto{\pgfqpoint{2.459889in}{2.788354in}}{\pgfqpoint{2.464279in}{2.798953in}}{\pgfqpoint{2.464279in}{2.810003in}}%
\pgfpathcurveto{\pgfqpoint{2.464279in}{2.821053in}}{\pgfqpoint{2.459889in}{2.831652in}}{\pgfqpoint{2.452075in}{2.839466in}}%
\pgfpathcurveto{\pgfqpoint{2.444262in}{2.847279in}}{\pgfqpoint{2.433662in}{2.851670in}}{\pgfqpoint{2.422612in}{2.851670in}}%
\pgfpathcurveto{\pgfqpoint{2.411562in}{2.851670in}}{\pgfqpoint{2.400963in}{2.847279in}}{\pgfqpoint{2.393150in}{2.839466in}}%
\pgfpathcurveto{\pgfqpoint{2.385336in}{2.831652in}}{\pgfqpoint{2.380946in}{2.821053in}}{\pgfqpoint{2.380946in}{2.810003in}}%
\pgfpathcurveto{\pgfqpoint{2.380946in}{2.798953in}}{\pgfqpoint{2.385336in}{2.788354in}}{\pgfqpoint{2.393150in}{2.780540in}}%
\pgfpathcurveto{\pgfqpoint{2.400963in}{2.772727in}}{\pgfqpoint{2.411562in}{2.768336in}}{\pgfqpoint{2.422612in}{2.768336in}}%
\pgfpathclose%
\pgfusepath{stroke,fill}%
\end{pgfscope}%
\begin{pgfscope}%
\pgfpathrectangle{\pgfqpoint{0.600000in}{0.600000in}}{\pgfqpoint{3.900000in}{3.900000in}}%
\pgfusepath{clip}%
\pgfsetbuttcap%
\pgfsetroundjoin%
\definecolor{currentfill}{rgb}{0.121569,0.466667,0.705882}%
\pgfsetfillcolor{currentfill}%
\pgfsetlinewidth{1.003750pt}%
\definecolor{currentstroke}{rgb}{0.121569,0.466667,0.705882}%
\pgfsetstrokecolor{currentstroke}%
\pgfsetdash{}{0pt}%
\pgfpathmoveto{\pgfqpoint{2.854035in}{1.619117in}}%
\pgfpathcurveto{\pgfqpoint{2.865085in}{1.619117in}}{\pgfqpoint{2.875684in}{1.623507in}}{\pgfqpoint{2.883498in}{1.631321in}}%
\pgfpathcurveto{\pgfqpoint{2.891312in}{1.639134in}}{\pgfqpoint{2.895702in}{1.649733in}}{\pgfqpoint{2.895702in}{1.660783in}}%
\pgfpathcurveto{\pgfqpoint{2.895702in}{1.671833in}}{\pgfqpoint{2.891312in}{1.682433in}}{\pgfqpoint{2.883498in}{1.690246in}}%
\pgfpathcurveto{\pgfqpoint{2.875684in}{1.698060in}}{\pgfqpoint{2.865085in}{1.702450in}}{\pgfqpoint{2.854035in}{1.702450in}}%
\pgfpathcurveto{\pgfqpoint{2.842985in}{1.702450in}}{\pgfqpoint{2.832386in}{1.698060in}}{\pgfqpoint{2.824573in}{1.690246in}}%
\pgfpathcurveto{\pgfqpoint{2.816759in}{1.682433in}}{\pgfqpoint{2.812369in}{1.671833in}}{\pgfqpoint{2.812369in}{1.660783in}}%
\pgfpathcurveto{\pgfqpoint{2.812369in}{1.649733in}}{\pgfqpoint{2.816759in}{1.639134in}}{\pgfqpoint{2.824573in}{1.631321in}}%
\pgfpathcurveto{\pgfqpoint{2.832386in}{1.623507in}}{\pgfqpoint{2.842985in}{1.619117in}}{\pgfqpoint{2.854035in}{1.619117in}}%
\pgfpathclose%
\pgfusepath{stroke,fill}%
\end{pgfscope}%
\begin{pgfscope}%
\pgfpathrectangle{\pgfqpoint{0.600000in}{0.600000in}}{\pgfqpoint{3.900000in}{3.900000in}}%
\pgfusepath{clip}%
\pgfsetbuttcap%
\pgfsetroundjoin%
\definecolor{currentfill}{rgb}{0.121569,0.466667,0.705882}%
\pgfsetfillcolor{currentfill}%
\pgfsetlinewidth{1.003750pt}%
\definecolor{currentstroke}{rgb}{0.121569,0.466667,0.705882}%
\pgfsetstrokecolor{currentstroke}%
\pgfsetdash{}{0pt}%
\pgfpathmoveto{\pgfqpoint{2.525314in}{1.913745in}}%
\pgfpathcurveto{\pgfqpoint{2.536364in}{1.913745in}}{\pgfqpoint{2.546963in}{1.918136in}}{\pgfqpoint{2.554777in}{1.925949in}}%
\pgfpathcurveto{\pgfqpoint{2.562590in}{1.933763in}}{\pgfqpoint{2.566981in}{1.944362in}}{\pgfqpoint{2.566981in}{1.955412in}}%
\pgfpathcurveto{\pgfqpoint{2.566981in}{1.966462in}}{\pgfqpoint{2.562590in}{1.977061in}}{\pgfqpoint{2.554777in}{1.984875in}}%
\pgfpathcurveto{\pgfqpoint{2.546963in}{1.992688in}}{\pgfqpoint{2.536364in}{1.997079in}}{\pgfqpoint{2.525314in}{1.997079in}}%
\pgfpathcurveto{\pgfqpoint{2.514264in}{1.997079in}}{\pgfqpoint{2.503665in}{1.992688in}}{\pgfqpoint{2.495851in}{1.984875in}}%
\pgfpathcurveto{\pgfqpoint{2.488038in}{1.977061in}}{\pgfqpoint{2.483647in}{1.966462in}}{\pgfqpoint{2.483647in}{1.955412in}}%
\pgfpathcurveto{\pgfqpoint{2.483647in}{1.944362in}}{\pgfqpoint{2.488038in}{1.933763in}}{\pgfqpoint{2.495851in}{1.925949in}}%
\pgfpathcurveto{\pgfqpoint{2.503665in}{1.918136in}}{\pgfqpoint{2.514264in}{1.913745in}}{\pgfqpoint{2.525314in}{1.913745in}}%
\pgfpathclose%
\pgfusepath{stroke,fill}%
\end{pgfscope}%
\begin{pgfscope}%
\pgfpathrectangle{\pgfqpoint{0.600000in}{0.600000in}}{\pgfqpoint{3.900000in}{3.900000in}}%
\pgfusepath{clip}%
\pgfsetbuttcap%
\pgfsetroundjoin%
\definecolor{currentfill}{rgb}{0.121569,0.466667,0.705882}%
\pgfsetfillcolor{currentfill}%
\pgfsetlinewidth{1.003750pt}%
\definecolor{currentstroke}{rgb}{0.121569,0.466667,0.705882}%
\pgfsetstrokecolor{currentstroke}%
\pgfsetdash{}{0pt}%
\pgfpathmoveto{\pgfqpoint{2.732038in}{2.042385in}}%
\pgfpathcurveto{\pgfqpoint{2.743089in}{2.042385in}}{\pgfqpoint{2.753688in}{2.046775in}}{\pgfqpoint{2.761501in}{2.054589in}}%
\pgfpathcurveto{\pgfqpoint{2.769315in}{2.062403in}}{\pgfqpoint{2.773705in}{2.073002in}}{\pgfqpoint{2.773705in}{2.084052in}}%
\pgfpathcurveto{\pgfqpoint{2.773705in}{2.095102in}}{\pgfqpoint{2.769315in}{2.105701in}}{\pgfqpoint{2.761501in}{2.113514in}}%
\pgfpathcurveto{\pgfqpoint{2.753688in}{2.121328in}}{\pgfqpoint{2.743089in}{2.125718in}}{\pgfqpoint{2.732038in}{2.125718in}}%
\pgfpathcurveto{\pgfqpoint{2.720988in}{2.125718in}}{\pgfqpoint{2.710389in}{2.121328in}}{\pgfqpoint{2.702576in}{2.113514in}}%
\pgfpathcurveto{\pgfqpoint{2.694762in}{2.105701in}}{\pgfqpoint{2.690372in}{2.095102in}}{\pgfqpoint{2.690372in}{2.084052in}}%
\pgfpathcurveto{\pgfqpoint{2.690372in}{2.073002in}}{\pgfqpoint{2.694762in}{2.062403in}}{\pgfqpoint{2.702576in}{2.054589in}}%
\pgfpathcurveto{\pgfqpoint{2.710389in}{2.046775in}}{\pgfqpoint{2.720988in}{2.042385in}}{\pgfqpoint{2.732038in}{2.042385in}}%
\pgfpathclose%
\pgfusepath{stroke,fill}%
\end{pgfscope}%
\begin{pgfscope}%
\pgfpathrectangle{\pgfqpoint{0.600000in}{0.600000in}}{\pgfqpoint{3.900000in}{3.900000in}}%
\pgfusepath{clip}%
\pgfsetbuttcap%
\pgfsetroundjoin%
\definecolor{currentfill}{rgb}{0.121569,0.466667,0.705882}%
\pgfsetfillcolor{currentfill}%
\pgfsetlinewidth{1.003750pt}%
\definecolor{currentstroke}{rgb}{0.121569,0.466667,0.705882}%
\pgfsetstrokecolor{currentstroke}%
\pgfsetdash{}{0pt}%
\pgfpathmoveto{\pgfqpoint{3.223040in}{2.537506in}}%
\pgfpathcurveto{\pgfqpoint{3.234090in}{2.537506in}}{\pgfqpoint{3.244689in}{2.541896in}}{\pgfqpoint{3.252503in}{2.549710in}}%
\pgfpathcurveto{\pgfqpoint{3.260316in}{2.557523in}}{\pgfqpoint{3.264707in}{2.568123in}}{\pgfqpoint{3.264707in}{2.579173in}}%
\pgfpathcurveto{\pgfqpoint{3.264707in}{2.590223in}}{\pgfqpoint{3.260316in}{2.600822in}}{\pgfqpoint{3.252503in}{2.608635in}}%
\pgfpathcurveto{\pgfqpoint{3.244689in}{2.616449in}}{\pgfqpoint{3.234090in}{2.620839in}}{\pgfqpoint{3.223040in}{2.620839in}}%
\pgfpathcurveto{\pgfqpoint{3.211990in}{2.620839in}}{\pgfqpoint{3.201391in}{2.616449in}}{\pgfqpoint{3.193577in}{2.608635in}}%
\pgfpathcurveto{\pgfqpoint{3.185764in}{2.600822in}}{\pgfqpoint{3.181373in}{2.590223in}}{\pgfqpoint{3.181373in}{2.579173in}}%
\pgfpathcurveto{\pgfqpoint{3.181373in}{2.568123in}}{\pgfqpoint{3.185764in}{2.557523in}}{\pgfqpoint{3.193577in}{2.549710in}}%
\pgfpathcurveto{\pgfqpoint{3.201391in}{2.541896in}}{\pgfqpoint{3.211990in}{2.537506in}}{\pgfqpoint{3.223040in}{2.537506in}}%
\pgfpathclose%
\pgfusepath{stroke,fill}%
\end{pgfscope}%
\begin{pgfscope}%
\pgfpathrectangle{\pgfqpoint{0.600000in}{0.600000in}}{\pgfqpoint{3.900000in}{3.900000in}}%
\pgfusepath{clip}%
\pgfsetbuttcap%
\pgfsetroundjoin%
\definecolor{currentfill}{rgb}{0.121569,0.466667,0.705882}%
\pgfsetfillcolor{currentfill}%
\pgfsetlinewidth{1.003750pt}%
\definecolor{currentstroke}{rgb}{0.121569,0.466667,0.705882}%
\pgfsetstrokecolor{currentstroke}%
\pgfsetdash{}{0pt}%
\pgfpathmoveto{\pgfqpoint{2.390205in}{1.928450in}}%
\pgfpathcurveto{\pgfqpoint{2.401255in}{1.928450in}}{\pgfqpoint{2.411854in}{1.932840in}}{\pgfqpoint{2.419668in}{1.940654in}}%
\pgfpathcurveto{\pgfqpoint{2.427482in}{1.948467in}}{\pgfqpoint{2.431872in}{1.959066in}}{\pgfqpoint{2.431872in}{1.970116in}}%
\pgfpathcurveto{\pgfqpoint{2.431872in}{1.981166in}}{\pgfqpoint{2.427482in}{1.991765in}}{\pgfqpoint{2.419668in}{1.999579in}}%
\pgfpathcurveto{\pgfqpoint{2.411854in}{2.007393in}}{\pgfqpoint{2.401255in}{2.011783in}}{\pgfqpoint{2.390205in}{2.011783in}}%
\pgfpathcurveto{\pgfqpoint{2.379155in}{2.011783in}}{\pgfqpoint{2.368556in}{2.007393in}}{\pgfqpoint{2.360743in}{1.999579in}}%
\pgfpathcurveto{\pgfqpoint{2.352929in}{1.991765in}}{\pgfqpoint{2.348539in}{1.981166in}}{\pgfqpoint{2.348539in}{1.970116in}}%
\pgfpathcurveto{\pgfqpoint{2.348539in}{1.959066in}}{\pgfqpoint{2.352929in}{1.948467in}}{\pgfqpoint{2.360743in}{1.940654in}}%
\pgfpathcurveto{\pgfqpoint{2.368556in}{1.932840in}}{\pgfqpoint{2.379155in}{1.928450in}}{\pgfqpoint{2.390205in}{1.928450in}}%
\pgfpathclose%
\pgfusepath{stroke,fill}%
\end{pgfscope}%
\begin{pgfscope}%
\pgfpathrectangle{\pgfqpoint{0.600000in}{0.600000in}}{\pgfqpoint{3.900000in}{3.900000in}}%
\pgfusepath{clip}%
\pgfsetbuttcap%
\pgfsetroundjoin%
\definecolor{currentfill}{rgb}{0.121569,0.466667,0.705882}%
\pgfsetfillcolor{currentfill}%
\pgfsetlinewidth{1.003750pt}%
\definecolor{currentstroke}{rgb}{0.121569,0.466667,0.705882}%
\pgfsetstrokecolor{currentstroke}%
\pgfsetdash{}{0pt}%
\pgfpathmoveto{\pgfqpoint{3.483900in}{2.430277in}}%
\pgfpathcurveto{\pgfqpoint{3.494950in}{2.430277in}}{\pgfqpoint{3.505549in}{2.434667in}}{\pgfqpoint{3.513363in}{2.442481in}}%
\pgfpathcurveto{\pgfqpoint{3.521176in}{2.450294in}}{\pgfqpoint{3.525567in}{2.460893in}}{\pgfqpoint{3.525567in}{2.471944in}}%
\pgfpathcurveto{\pgfqpoint{3.525567in}{2.482994in}}{\pgfqpoint{3.521176in}{2.493593in}}{\pgfqpoint{3.513363in}{2.501406in}}%
\pgfpathcurveto{\pgfqpoint{3.505549in}{2.509220in}}{\pgfqpoint{3.494950in}{2.513610in}}{\pgfqpoint{3.483900in}{2.513610in}}%
\pgfpathcurveto{\pgfqpoint{3.472850in}{2.513610in}}{\pgfqpoint{3.462251in}{2.509220in}}{\pgfqpoint{3.454437in}{2.501406in}}%
\pgfpathcurveto{\pgfqpoint{3.446624in}{2.493593in}}{\pgfqpoint{3.442233in}{2.482994in}}{\pgfqpoint{3.442233in}{2.471944in}}%
\pgfpathcurveto{\pgfqpoint{3.442233in}{2.460893in}}{\pgfqpoint{3.446624in}{2.450294in}}{\pgfqpoint{3.454437in}{2.442481in}}%
\pgfpathcurveto{\pgfqpoint{3.462251in}{2.434667in}}{\pgfqpoint{3.472850in}{2.430277in}}{\pgfqpoint{3.483900in}{2.430277in}}%
\pgfpathclose%
\pgfusepath{stroke,fill}%
\end{pgfscope}%
\begin{pgfscope}%
\pgfpathrectangle{\pgfqpoint{0.600000in}{0.600000in}}{\pgfqpoint{3.900000in}{3.900000in}}%
\pgfusepath{clip}%
\pgfsetbuttcap%
\pgfsetroundjoin%
\definecolor{currentfill}{rgb}{0.121569,0.466667,0.705882}%
\pgfsetfillcolor{currentfill}%
\pgfsetlinewidth{1.003750pt}%
\definecolor{currentstroke}{rgb}{0.121569,0.466667,0.705882}%
\pgfsetstrokecolor{currentstroke}%
\pgfsetdash{}{0pt}%
\pgfpathmoveto{\pgfqpoint{2.778823in}{2.879302in}}%
\pgfpathcurveto{\pgfqpoint{2.789873in}{2.879302in}}{\pgfqpoint{2.800472in}{2.883692in}}{\pgfqpoint{2.808285in}{2.891505in}}%
\pgfpathcurveto{\pgfqpoint{2.816099in}{2.899319in}}{\pgfqpoint{2.820489in}{2.909918in}}{\pgfqpoint{2.820489in}{2.920968in}}%
\pgfpathcurveto{\pgfqpoint{2.820489in}{2.932018in}}{\pgfqpoint{2.816099in}{2.942617in}}{\pgfqpoint{2.808285in}{2.950431in}}%
\pgfpathcurveto{\pgfqpoint{2.800472in}{2.958245in}}{\pgfqpoint{2.789873in}{2.962635in}}{\pgfqpoint{2.778823in}{2.962635in}}%
\pgfpathcurveto{\pgfqpoint{2.767772in}{2.962635in}}{\pgfqpoint{2.757173in}{2.958245in}}{\pgfqpoint{2.749360in}{2.950431in}}%
\pgfpathcurveto{\pgfqpoint{2.741546in}{2.942617in}}{\pgfqpoint{2.737156in}{2.932018in}}{\pgfqpoint{2.737156in}{2.920968in}}%
\pgfpathcurveto{\pgfqpoint{2.737156in}{2.909918in}}{\pgfqpoint{2.741546in}{2.899319in}}{\pgfqpoint{2.749360in}{2.891505in}}%
\pgfpathcurveto{\pgfqpoint{2.757173in}{2.883692in}}{\pgfqpoint{2.767772in}{2.879302in}}{\pgfqpoint{2.778823in}{2.879302in}}%
\pgfpathclose%
\pgfusepath{stroke,fill}%
\end{pgfscope}%
\begin{pgfscope}%
\pgfpathrectangle{\pgfqpoint{0.600000in}{0.600000in}}{\pgfqpoint{3.900000in}{3.900000in}}%
\pgfusepath{clip}%
\pgfsetbuttcap%
\pgfsetroundjoin%
\definecolor{currentfill}{rgb}{0.121569,0.466667,0.705882}%
\pgfsetfillcolor{currentfill}%
\pgfsetlinewidth{1.003750pt}%
\definecolor{currentstroke}{rgb}{0.121569,0.466667,0.705882}%
\pgfsetstrokecolor{currentstroke}%
\pgfsetdash{}{0pt}%
\pgfpathmoveto{\pgfqpoint{3.189939in}{2.456410in}}%
\pgfpathcurveto{\pgfqpoint{3.200989in}{2.456410in}}{\pgfqpoint{3.211588in}{2.460800in}}{\pgfqpoint{3.219402in}{2.468614in}}%
\pgfpathcurveto{\pgfqpoint{3.227216in}{2.476428in}}{\pgfqpoint{3.231606in}{2.487027in}}{\pgfqpoint{3.231606in}{2.498077in}}%
\pgfpathcurveto{\pgfqpoint{3.231606in}{2.509127in}}{\pgfqpoint{3.227216in}{2.519726in}}{\pgfqpoint{3.219402in}{2.527540in}}%
\pgfpathcurveto{\pgfqpoint{3.211588in}{2.535353in}}{\pgfqpoint{3.200989in}{2.539744in}}{\pgfqpoint{3.189939in}{2.539744in}}%
\pgfpathcurveto{\pgfqpoint{3.178889in}{2.539744in}}{\pgfqpoint{3.168290in}{2.535353in}}{\pgfqpoint{3.160477in}{2.527540in}}%
\pgfpathcurveto{\pgfqpoint{3.152663in}{2.519726in}}{\pgfqpoint{3.148273in}{2.509127in}}{\pgfqpoint{3.148273in}{2.498077in}}%
\pgfpathcurveto{\pgfqpoint{3.148273in}{2.487027in}}{\pgfqpoint{3.152663in}{2.476428in}}{\pgfqpoint{3.160477in}{2.468614in}}%
\pgfpathcurveto{\pgfqpoint{3.168290in}{2.460800in}}{\pgfqpoint{3.178889in}{2.456410in}}{\pgfqpoint{3.189939in}{2.456410in}}%
\pgfpathclose%
\pgfusepath{stroke,fill}%
\end{pgfscope}%
\begin{pgfscope}%
\pgfpathrectangle{\pgfqpoint{0.600000in}{0.600000in}}{\pgfqpoint{3.900000in}{3.900000in}}%
\pgfusepath{clip}%
\pgfsetbuttcap%
\pgfsetroundjoin%
\definecolor{currentfill}{rgb}{0.121569,0.466667,0.705882}%
\pgfsetfillcolor{currentfill}%
\pgfsetlinewidth{1.003750pt}%
\definecolor{currentstroke}{rgb}{0.121569,0.466667,0.705882}%
\pgfsetstrokecolor{currentstroke}%
\pgfsetdash{}{0pt}%
\pgfpathmoveto{\pgfqpoint{2.071655in}{2.762374in}}%
\pgfpathcurveto{\pgfqpoint{2.082705in}{2.762374in}}{\pgfqpoint{2.093304in}{2.766764in}}{\pgfqpoint{2.101118in}{2.774578in}}%
\pgfpathcurveto{\pgfqpoint{2.108931in}{2.782391in}}{\pgfqpoint{2.113322in}{2.792990in}}{\pgfqpoint{2.113322in}{2.804040in}}%
\pgfpathcurveto{\pgfqpoint{2.113322in}{2.815091in}}{\pgfqpoint{2.108931in}{2.825690in}}{\pgfqpoint{2.101118in}{2.833503in}}%
\pgfpathcurveto{\pgfqpoint{2.093304in}{2.841317in}}{\pgfqpoint{2.082705in}{2.845707in}}{\pgfqpoint{2.071655in}{2.845707in}}%
\pgfpathcurveto{\pgfqpoint{2.060605in}{2.845707in}}{\pgfqpoint{2.050006in}{2.841317in}}{\pgfqpoint{2.042192in}{2.833503in}}%
\pgfpathcurveto{\pgfqpoint{2.034379in}{2.825690in}}{\pgfqpoint{2.029988in}{2.815091in}}{\pgfqpoint{2.029988in}{2.804040in}}%
\pgfpathcurveto{\pgfqpoint{2.029988in}{2.792990in}}{\pgfqpoint{2.034379in}{2.782391in}}{\pgfqpoint{2.042192in}{2.774578in}}%
\pgfpathcurveto{\pgfqpoint{2.050006in}{2.766764in}}{\pgfqpoint{2.060605in}{2.762374in}}{\pgfqpoint{2.071655in}{2.762374in}}%
\pgfpathclose%
\pgfusepath{stroke,fill}%
\end{pgfscope}%
\begin{pgfscope}%
\pgfpathrectangle{\pgfqpoint{0.600000in}{0.600000in}}{\pgfqpoint{3.900000in}{3.900000in}}%
\pgfusepath{clip}%
\pgfsetbuttcap%
\pgfsetroundjoin%
\definecolor{currentfill}{rgb}{0.121569,0.466667,0.705882}%
\pgfsetfillcolor{currentfill}%
\pgfsetlinewidth{1.003750pt}%
\definecolor{currentstroke}{rgb}{0.121569,0.466667,0.705882}%
\pgfsetstrokecolor{currentstroke}%
\pgfsetdash{}{0pt}%
\pgfpathmoveto{\pgfqpoint{2.628623in}{2.628929in}}%
\pgfpathcurveto{\pgfqpoint{2.639673in}{2.628929in}}{\pgfqpoint{2.650272in}{2.633319in}}{\pgfqpoint{2.658086in}{2.641133in}}%
\pgfpathcurveto{\pgfqpoint{2.665899in}{2.648947in}}{\pgfqpoint{2.670290in}{2.659546in}}{\pgfqpoint{2.670290in}{2.670596in}}%
\pgfpathcurveto{\pgfqpoint{2.670290in}{2.681646in}}{\pgfqpoint{2.665899in}{2.692245in}}{\pgfqpoint{2.658086in}{2.700058in}}%
\pgfpathcurveto{\pgfqpoint{2.650272in}{2.707872in}}{\pgfqpoint{2.639673in}{2.712262in}}{\pgfqpoint{2.628623in}{2.712262in}}%
\pgfpathcurveto{\pgfqpoint{2.617573in}{2.712262in}}{\pgfqpoint{2.606974in}{2.707872in}}{\pgfqpoint{2.599160in}{2.700058in}}%
\pgfpathcurveto{\pgfqpoint{2.591347in}{2.692245in}}{\pgfqpoint{2.586956in}{2.681646in}}{\pgfqpoint{2.586956in}{2.670596in}}%
\pgfpathcurveto{\pgfqpoint{2.586956in}{2.659546in}}{\pgfqpoint{2.591347in}{2.648947in}}{\pgfqpoint{2.599160in}{2.641133in}}%
\pgfpathcurveto{\pgfqpoint{2.606974in}{2.633319in}}{\pgfqpoint{2.617573in}{2.628929in}}{\pgfqpoint{2.628623in}{2.628929in}}%
\pgfpathclose%
\pgfusepath{stroke,fill}%
\end{pgfscope}%
\begin{pgfscope}%
\pgfpathrectangle{\pgfqpoint{0.600000in}{0.600000in}}{\pgfqpoint{3.900000in}{3.900000in}}%
\pgfusepath{clip}%
\pgfsetbuttcap%
\pgfsetroundjoin%
\definecolor{currentfill}{rgb}{0.121569,0.466667,0.705882}%
\pgfsetfillcolor{currentfill}%
\pgfsetlinewidth{1.003750pt}%
\definecolor{currentstroke}{rgb}{0.121569,0.466667,0.705882}%
\pgfsetstrokecolor{currentstroke}%
\pgfsetdash{}{0pt}%
\pgfpathmoveto{\pgfqpoint{2.436044in}{3.075665in}}%
\pgfpathcurveto{\pgfqpoint{2.447094in}{3.075665in}}{\pgfqpoint{2.457693in}{3.080055in}}{\pgfqpoint{2.465507in}{3.087869in}}%
\pgfpathcurveto{\pgfqpoint{2.473321in}{3.095683in}}{\pgfqpoint{2.477711in}{3.106282in}}{\pgfqpoint{2.477711in}{3.117332in}}%
\pgfpathcurveto{\pgfqpoint{2.477711in}{3.128382in}}{\pgfqpoint{2.473321in}{3.138981in}}{\pgfqpoint{2.465507in}{3.146795in}}%
\pgfpathcurveto{\pgfqpoint{2.457693in}{3.154608in}}{\pgfqpoint{2.447094in}{3.158998in}}{\pgfqpoint{2.436044in}{3.158998in}}%
\pgfpathcurveto{\pgfqpoint{2.424994in}{3.158998in}}{\pgfqpoint{2.414395in}{3.154608in}}{\pgfqpoint{2.406582in}{3.146795in}}%
\pgfpathcurveto{\pgfqpoint{2.398768in}{3.138981in}}{\pgfqpoint{2.394378in}{3.128382in}}{\pgfqpoint{2.394378in}{3.117332in}}%
\pgfpathcurveto{\pgfqpoint{2.394378in}{3.106282in}}{\pgfqpoint{2.398768in}{3.095683in}}{\pgfqpoint{2.406582in}{3.087869in}}%
\pgfpathcurveto{\pgfqpoint{2.414395in}{3.080055in}}{\pgfqpoint{2.424994in}{3.075665in}}{\pgfqpoint{2.436044in}{3.075665in}}%
\pgfpathclose%
\pgfusepath{stroke,fill}%
\end{pgfscope}%
\begin{pgfscope}%
\pgfpathrectangle{\pgfqpoint{0.600000in}{0.600000in}}{\pgfqpoint{3.900000in}{3.900000in}}%
\pgfusepath{clip}%
\pgfsetbuttcap%
\pgfsetroundjoin%
\definecolor{currentfill}{rgb}{0.121569,0.466667,0.705882}%
\pgfsetfillcolor{currentfill}%
\pgfsetlinewidth{1.003750pt}%
\definecolor{currentstroke}{rgb}{0.121569,0.466667,0.705882}%
\pgfsetstrokecolor{currentstroke}%
\pgfsetdash{}{0pt}%
\pgfpathmoveto{\pgfqpoint{1.948635in}{3.196435in}}%
\pgfpathcurveto{\pgfqpoint{1.959685in}{3.196435in}}{\pgfqpoint{1.970284in}{3.200825in}}{\pgfqpoint{1.978098in}{3.208639in}}%
\pgfpathcurveto{\pgfqpoint{1.985911in}{3.216452in}}{\pgfqpoint{1.990302in}{3.227051in}}{\pgfqpoint{1.990302in}{3.238101in}}%
\pgfpathcurveto{\pgfqpoint{1.990302in}{3.249152in}}{\pgfqpoint{1.985911in}{3.259751in}}{\pgfqpoint{1.978098in}{3.267564in}}%
\pgfpathcurveto{\pgfqpoint{1.970284in}{3.275378in}}{\pgfqpoint{1.959685in}{3.279768in}}{\pgfqpoint{1.948635in}{3.279768in}}%
\pgfpathcurveto{\pgfqpoint{1.937585in}{3.279768in}}{\pgfqpoint{1.926986in}{3.275378in}}{\pgfqpoint{1.919172in}{3.267564in}}%
\pgfpathcurveto{\pgfqpoint{1.911359in}{3.259751in}}{\pgfqpoint{1.906968in}{3.249152in}}{\pgfqpoint{1.906968in}{3.238101in}}%
\pgfpathcurveto{\pgfqpoint{1.906968in}{3.227051in}}{\pgfqpoint{1.911359in}{3.216452in}}{\pgfqpoint{1.919172in}{3.208639in}}%
\pgfpathcurveto{\pgfqpoint{1.926986in}{3.200825in}}{\pgfqpoint{1.937585in}{3.196435in}}{\pgfqpoint{1.948635in}{3.196435in}}%
\pgfpathclose%
\pgfusepath{stroke,fill}%
\end{pgfscope}%
\begin{pgfscope}%
\pgfpathrectangle{\pgfqpoint{0.600000in}{0.600000in}}{\pgfqpoint{3.900000in}{3.900000in}}%
\pgfusepath{clip}%
\pgfsetbuttcap%
\pgfsetroundjoin%
\definecolor{currentfill}{rgb}{0.121569,0.466667,0.705882}%
\pgfsetfillcolor{currentfill}%
\pgfsetlinewidth{1.003750pt}%
\definecolor{currentstroke}{rgb}{0.121569,0.466667,0.705882}%
\pgfsetstrokecolor{currentstroke}%
\pgfsetdash{}{0pt}%
\pgfpathmoveto{\pgfqpoint{1.774775in}{2.475316in}}%
\pgfpathcurveto{\pgfqpoint{1.785825in}{2.475316in}}{\pgfqpoint{1.796424in}{2.479706in}}{\pgfqpoint{1.804237in}{2.487519in}}%
\pgfpathcurveto{\pgfqpoint{1.812051in}{2.495333in}}{\pgfqpoint{1.816441in}{2.505932in}}{\pgfqpoint{1.816441in}{2.516982in}}%
\pgfpathcurveto{\pgfqpoint{1.816441in}{2.528032in}}{\pgfqpoint{1.812051in}{2.538631in}}{\pgfqpoint{1.804237in}{2.546445in}}%
\pgfpathcurveto{\pgfqpoint{1.796424in}{2.554259in}}{\pgfqpoint{1.785825in}{2.558649in}}{\pgfqpoint{1.774775in}{2.558649in}}%
\pgfpathcurveto{\pgfqpoint{1.763725in}{2.558649in}}{\pgfqpoint{1.753126in}{2.554259in}}{\pgfqpoint{1.745312in}{2.546445in}}%
\pgfpathcurveto{\pgfqpoint{1.737498in}{2.538631in}}{\pgfqpoint{1.733108in}{2.528032in}}{\pgfqpoint{1.733108in}{2.516982in}}%
\pgfpathcurveto{\pgfqpoint{1.733108in}{2.505932in}}{\pgfqpoint{1.737498in}{2.495333in}}{\pgfqpoint{1.745312in}{2.487519in}}%
\pgfpathcurveto{\pgfqpoint{1.753126in}{2.479706in}}{\pgfqpoint{1.763725in}{2.475316in}}{\pgfqpoint{1.774775in}{2.475316in}}%
\pgfpathclose%
\pgfusepath{stroke,fill}%
\end{pgfscope}%
\begin{pgfscope}%
\pgfpathrectangle{\pgfqpoint{0.600000in}{0.600000in}}{\pgfqpoint{3.900000in}{3.900000in}}%
\pgfusepath{clip}%
\pgfsetbuttcap%
\pgfsetroundjoin%
\definecolor{currentfill}{rgb}{0.121569,0.466667,0.705882}%
\pgfsetfillcolor{currentfill}%
\pgfsetlinewidth{1.003750pt}%
\definecolor{currentstroke}{rgb}{0.121569,0.466667,0.705882}%
\pgfsetstrokecolor{currentstroke}%
\pgfsetdash{}{0pt}%
\pgfpathmoveto{\pgfqpoint{1.878353in}{2.308565in}}%
\pgfpathcurveto{\pgfqpoint{1.889403in}{2.308565in}}{\pgfqpoint{1.900002in}{2.312955in}}{\pgfqpoint{1.907816in}{2.320769in}}%
\pgfpathcurveto{\pgfqpoint{1.915630in}{2.328583in}}{\pgfqpoint{1.920020in}{2.339182in}}{\pgfqpoint{1.920020in}{2.350232in}}%
\pgfpathcurveto{\pgfqpoint{1.920020in}{2.361282in}}{\pgfqpoint{1.915630in}{2.371881in}}{\pgfqpoint{1.907816in}{2.379694in}}%
\pgfpathcurveto{\pgfqpoint{1.900002in}{2.387508in}}{\pgfqpoint{1.889403in}{2.391898in}}{\pgfqpoint{1.878353in}{2.391898in}}%
\pgfpathcurveto{\pgfqpoint{1.867303in}{2.391898in}}{\pgfqpoint{1.856704in}{2.387508in}}{\pgfqpoint{1.848891in}{2.379694in}}%
\pgfpathcurveto{\pgfqpoint{1.841077in}{2.371881in}}{\pgfqpoint{1.836687in}{2.361282in}}{\pgfqpoint{1.836687in}{2.350232in}}%
\pgfpathcurveto{\pgfqpoint{1.836687in}{2.339182in}}{\pgfqpoint{1.841077in}{2.328583in}}{\pgfqpoint{1.848891in}{2.320769in}}%
\pgfpathcurveto{\pgfqpoint{1.856704in}{2.312955in}}{\pgfqpoint{1.867303in}{2.308565in}}{\pgfqpoint{1.878353in}{2.308565in}}%
\pgfpathclose%
\pgfusepath{stroke,fill}%
\end{pgfscope}%
\begin{pgfscope}%
\pgfpathrectangle{\pgfqpoint{0.600000in}{0.600000in}}{\pgfqpoint{3.900000in}{3.900000in}}%
\pgfusepath{clip}%
\pgfsetbuttcap%
\pgfsetroundjoin%
\definecolor{currentfill}{rgb}{0.121569,0.466667,0.705882}%
\pgfsetfillcolor{currentfill}%
\pgfsetlinewidth{1.003750pt}%
\definecolor{currentstroke}{rgb}{0.121569,0.466667,0.705882}%
\pgfsetstrokecolor{currentstroke}%
\pgfsetdash{}{0pt}%
\pgfpathmoveto{\pgfqpoint{2.396927in}{3.002217in}}%
\pgfpathcurveto{\pgfqpoint{2.407977in}{3.002217in}}{\pgfqpoint{2.418576in}{3.006608in}}{\pgfqpoint{2.426390in}{3.014421in}}%
\pgfpathcurveto{\pgfqpoint{2.434204in}{3.022235in}}{\pgfqpoint{2.438594in}{3.032834in}}{\pgfqpoint{2.438594in}{3.043884in}}%
\pgfpathcurveto{\pgfqpoint{2.438594in}{3.054934in}}{\pgfqpoint{2.434204in}{3.065533in}}{\pgfqpoint{2.426390in}{3.073347in}}%
\pgfpathcurveto{\pgfqpoint{2.418576in}{3.081161in}}{\pgfqpoint{2.407977in}{3.085551in}}{\pgfqpoint{2.396927in}{3.085551in}}%
\pgfpathcurveto{\pgfqpoint{2.385877in}{3.085551in}}{\pgfqpoint{2.375278in}{3.081161in}}{\pgfqpoint{2.367464in}{3.073347in}}%
\pgfpathcurveto{\pgfqpoint{2.359651in}{3.065533in}}{\pgfqpoint{2.355261in}{3.054934in}}{\pgfqpoint{2.355261in}{3.043884in}}%
\pgfpathcurveto{\pgfqpoint{2.355261in}{3.032834in}}{\pgfqpoint{2.359651in}{3.022235in}}{\pgfqpoint{2.367464in}{3.014421in}}%
\pgfpathcurveto{\pgfqpoint{2.375278in}{3.006608in}}{\pgfqpoint{2.385877in}{3.002217in}}{\pgfqpoint{2.396927in}{3.002217in}}%
\pgfpathclose%
\pgfusepath{stroke,fill}%
\end{pgfscope}%
\begin{pgfscope}%
\pgfpathrectangle{\pgfqpoint{0.600000in}{0.600000in}}{\pgfqpoint{3.900000in}{3.900000in}}%
\pgfusepath{clip}%
\pgfsetbuttcap%
\pgfsetroundjoin%
\definecolor{currentfill}{rgb}{0.121569,0.466667,0.705882}%
\pgfsetfillcolor{currentfill}%
\pgfsetlinewidth{1.003750pt}%
\definecolor{currentstroke}{rgb}{0.121569,0.466667,0.705882}%
\pgfsetstrokecolor{currentstroke}%
\pgfsetdash{}{0pt}%
\pgfpathmoveto{\pgfqpoint{2.261025in}{3.211620in}}%
\pgfpathcurveto{\pgfqpoint{2.272076in}{3.211620in}}{\pgfqpoint{2.282675in}{3.216010in}}{\pgfqpoint{2.290488in}{3.223824in}}%
\pgfpathcurveto{\pgfqpoint{2.298302in}{3.231638in}}{\pgfqpoint{2.302692in}{3.242237in}}{\pgfqpoint{2.302692in}{3.253287in}}%
\pgfpathcurveto{\pgfqpoint{2.302692in}{3.264337in}}{\pgfqpoint{2.298302in}{3.274936in}}{\pgfqpoint{2.290488in}{3.282750in}}%
\pgfpathcurveto{\pgfqpoint{2.282675in}{3.290563in}}{\pgfqpoint{2.272076in}{3.294954in}}{\pgfqpoint{2.261025in}{3.294954in}}%
\pgfpathcurveto{\pgfqpoint{2.249975in}{3.294954in}}{\pgfqpoint{2.239376in}{3.290563in}}{\pgfqpoint{2.231563in}{3.282750in}}%
\pgfpathcurveto{\pgfqpoint{2.223749in}{3.274936in}}{\pgfqpoint{2.219359in}{3.264337in}}{\pgfqpoint{2.219359in}{3.253287in}}%
\pgfpathcurveto{\pgfqpoint{2.219359in}{3.242237in}}{\pgfqpoint{2.223749in}{3.231638in}}{\pgfqpoint{2.231563in}{3.223824in}}%
\pgfpathcurveto{\pgfqpoint{2.239376in}{3.216010in}}{\pgfqpoint{2.249975in}{3.211620in}}{\pgfqpoint{2.261025in}{3.211620in}}%
\pgfpathclose%
\pgfusepath{stroke,fill}%
\end{pgfscope}%
\begin{pgfscope}%
\pgfpathrectangle{\pgfqpoint{0.600000in}{0.600000in}}{\pgfqpoint{3.900000in}{3.900000in}}%
\pgfusepath{clip}%
\pgfsetbuttcap%
\pgfsetroundjoin%
\definecolor{currentfill}{rgb}{0.121569,0.466667,0.705882}%
\pgfsetfillcolor{currentfill}%
\pgfsetlinewidth{1.003750pt}%
\definecolor{currentstroke}{rgb}{0.121569,0.466667,0.705882}%
\pgfsetstrokecolor{currentstroke}%
\pgfsetdash{}{0pt}%
\pgfpathmoveto{\pgfqpoint{3.092748in}{1.490711in}}%
\pgfpathcurveto{\pgfqpoint{3.103798in}{1.490711in}}{\pgfqpoint{3.114397in}{1.495101in}}{\pgfqpoint{3.122211in}{1.502915in}}%
\pgfpathcurveto{\pgfqpoint{3.130024in}{1.510729in}}{\pgfqpoint{3.134415in}{1.521328in}}{\pgfqpoint{3.134415in}{1.532378in}}%
\pgfpathcurveto{\pgfqpoint{3.134415in}{1.543428in}}{\pgfqpoint{3.130024in}{1.554027in}}{\pgfqpoint{3.122211in}{1.561841in}}%
\pgfpathcurveto{\pgfqpoint{3.114397in}{1.569654in}}{\pgfqpoint{3.103798in}{1.574045in}}{\pgfqpoint{3.092748in}{1.574045in}}%
\pgfpathcurveto{\pgfqpoint{3.081698in}{1.574045in}}{\pgfqpoint{3.071099in}{1.569654in}}{\pgfqpoint{3.063285in}{1.561841in}}%
\pgfpathcurveto{\pgfqpoint{3.055472in}{1.554027in}}{\pgfqpoint{3.051081in}{1.543428in}}{\pgfqpoint{3.051081in}{1.532378in}}%
\pgfpathcurveto{\pgfqpoint{3.051081in}{1.521328in}}{\pgfqpoint{3.055472in}{1.510729in}}{\pgfqpoint{3.063285in}{1.502915in}}%
\pgfpathcurveto{\pgfqpoint{3.071099in}{1.495101in}}{\pgfqpoint{3.081698in}{1.490711in}}{\pgfqpoint{3.092748in}{1.490711in}}%
\pgfpathclose%
\pgfusepath{stroke,fill}%
\end{pgfscope}%
\begin{pgfscope}%
\pgfpathrectangle{\pgfqpoint{0.600000in}{0.600000in}}{\pgfqpoint{3.900000in}{3.900000in}}%
\pgfusepath{clip}%
\pgfsetbuttcap%
\pgfsetroundjoin%
\definecolor{currentfill}{rgb}{0.121569,0.466667,0.705882}%
\pgfsetfillcolor{currentfill}%
\pgfsetlinewidth{1.003750pt}%
\definecolor{currentstroke}{rgb}{0.121569,0.466667,0.705882}%
\pgfsetstrokecolor{currentstroke}%
\pgfsetdash{}{0pt}%
\pgfpathmoveto{\pgfqpoint{2.328942in}{2.875449in}}%
\pgfpathcurveto{\pgfqpoint{2.339992in}{2.875449in}}{\pgfqpoint{2.350591in}{2.879840in}}{\pgfqpoint{2.358404in}{2.887653in}}%
\pgfpathcurveto{\pgfqpoint{2.366218in}{2.895467in}}{\pgfqpoint{2.370608in}{2.906066in}}{\pgfqpoint{2.370608in}{2.917116in}}%
\pgfpathcurveto{\pgfqpoint{2.370608in}{2.928166in}}{\pgfqpoint{2.366218in}{2.938765in}}{\pgfqpoint{2.358404in}{2.946579in}}%
\pgfpathcurveto{\pgfqpoint{2.350591in}{2.954393in}}{\pgfqpoint{2.339992in}{2.958783in}}{\pgfqpoint{2.328942in}{2.958783in}}%
\pgfpathcurveto{\pgfqpoint{2.317891in}{2.958783in}}{\pgfqpoint{2.307292in}{2.954393in}}{\pgfqpoint{2.299479in}{2.946579in}}%
\pgfpathcurveto{\pgfqpoint{2.291665in}{2.938765in}}{\pgfqpoint{2.287275in}{2.928166in}}{\pgfqpoint{2.287275in}{2.917116in}}%
\pgfpathcurveto{\pgfqpoint{2.287275in}{2.906066in}}{\pgfqpoint{2.291665in}{2.895467in}}{\pgfqpoint{2.299479in}{2.887653in}}%
\pgfpathcurveto{\pgfqpoint{2.307292in}{2.879840in}}{\pgfqpoint{2.317891in}{2.875449in}}{\pgfqpoint{2.328942in}{2.875449in}}%
\pgfpathclose%
\pgfusepath{stroke,fill}%
\end{pgfscope}%
\begin{pgfscope}%
\pgfpathrectangle{\pgfqpoint{0.600000in}{0.600000in}}{\pgfqpoint{3.900000in}{3.900000in}}%
\pgfusepath{clip}%
\pgfsetbuttcap%
\pgfsetroundjoin%
\definecolor{currentfill}{rgb}{0.121569,0.466667,0.705882}%
\pgfsetfillcolor{currentfill}%
\pgfsetlinewidth{1.003750pt}%
\definecolor{currentstroke}{rgb}{0.121569,0.466667,0.705882}%
\pgfsetstrokecolor{currentstroke}%
\pgfsetdash{}{0pt}%
\pgfpathmoveto{\pgfqpoint{2.347469in}{2.317939in}}%
\pgfpathcurveto{\pgfqpoint{2.358519in}{2.317939in}}{\pgfqpoint{2.369118in}{2.322329in}}{\pgfqpoint{2.376932in}{2.330143in}}%
\pgfpathcurveto{\pgfqpoint{2.384746in}{2.337956in}}{\pgfqpoint{2.389136in}{2.348556in}}{\pgfqpoint{2.389136in}{2.359606in}}%
\pgfpathcurveto{\pgfqpoint{2.389136in}{2.370656in}}{\pgfqpoint{2.384746in}{2.381255in}}{\pgfqpoint{2.376932in}{2.389068in}}%
\pgfpathcurveto{\pgfqpoint{2.369118in}{2.396882in}}{\pgfqpoint{2.358519in}{2.401272in}}{\pgfqpoint{2.347469in}{2.401272in}}%
\pgfpathcurveto{\pgfqpoint{2.336419in}{2.401272in}}{\pgfqpoint{2.325820in}{2.396882in}}{\pgfqpoint{2.318006in}{2.389068in}}%
\pgfpathcurveto{\pgfqpoint{2.310193in}{2.381255in}}{\pgfqpoint{2.305802in}{2.370656in}}{\pgfqpoint{2.305802in}{2.359606in}}%
\pgfpathcurveto{\pgfqpoint{2.305802in}{2.348556in}}{\pgfqpoint{2.310193in}{2.337956in}}{\pgfqpoint{2.318006in}{2.330143in}}%
\pgfpathcurveto{\pgfqpoint{2.325820in}{2.322329in}}{\pgfqpoint{2.336419in}{2.317939in}}{\pgfqpoint{2.347469in}{2.317939in}}%
\pgfpathclose%
\pgfusepath{stroke,fill}%
\end{pgfscope}%
\begin{pgfscope}%
\pgfpathrectangle{\pgfqpoint{0.600000in}{0.600000in}}{\pgfqpoint{3.900000in}{3.900000in}}%
\pgfusepath{clip}%
\pgfsetbuttcap%
\pgfsetroundjoin%
\definecolor{currentfill}{rgb}{0.121569,0.466667,0.705882}%
\pgfsetfillcolor{currentfill}%
\pgfsetlinewidth{1.003750pt}%
\definecolor{currentstroke}{rgb}{0.121569,0.466667,0.705882}%
\pgfsetstrokecolor{currentstroke}%
\pgfsetdash{}{0pt}%
\pgfpathmoveto{\pgfqpoint{2.652533in}{1.502739in}}%
\pgfpathcurveto{\pgfqpoint{2.663583in}{1.502739in}}{\pgfqpoint{2.674182in}{1.507129in}}{\pgfqpoint{2.681996in}{1.514943in}}%
\pgfpathcurveto{\pgfqpoint{2.689809in}{1.522757in}}{\pgfqpoint{2.694200in}{1.533356in}}{\pgfqpoint{2.694200in}{1.544406in}}%
\pgfpathcurveto{\pgfqpoint{2.694200in}{1.555456in}}{\pgfqpoint{2.689809in}{1.566055in}}{\pgfqpoint{2.681996in}{1.573869in}}%
\pgfpathcurveto{\pgfqpoint{2.674182in}{1.581682in}}{\pgfqpoint{2.663583in}{1.586073in}}{\pgfqpoint{2.652533in}{1.586073in}}%
\pgfpathcurveto{\pgfqpoint{2.641483in}{1.586073in}}{\pgfqpoint{2.630884in}{1.581682in}}{\pgfqpoint{2.623070in}{1.573869in}}%
\pgfpathcurveto{\pgfqpoint{2.615256in}{1.566055in}}{\pgfqpoint{2.610866in}{1.555456in}}{\pgfqpoint{2.610866in}{1.544406in}}%
\pgfpathcurveto{\pgfqpoint{2.610866in}{1.533356in}}{\pgfqpoint{2.615256in}{1.522757in}}{\pgfqpoint{2.623070in}{1.514943in}}%
\pgfpathcurveto{\pgfqpoint{2.630884in}{1.507129in}}{\pgfqpoint{2.641483in}{1.502739in}}{\pgfqpoint{2.652533in}{1.502739in}}%
\pgfpathclose%
\pgfusepath{stroke,fill}%
\end{pgfscope}%
\begin{pgfscope}%
\pgfpathrectangle{\pgfqpoint{0.600000in}{0.600000in}}{\pgfqpoint{3.900000in}{3.900000in}}%
\pgfusepath{clip}%
\pgfsetbuttcap%
\pgfsetroundjoin%
\definecolor{currentfill}{rgb}{0.121569,0.466667,0.705882}%
\pgfsetfillcolor{currentfill}%
\pgfsetlinewidth{1.003750pt}%
\definecolor{currentstroke}{rgb}{0.121569,0.466667,0.705882}%
\pgfsetstrokecolor{currentstroke}%
\pgfsetdash{}{0pt}%
\pgfpathmoveto{\pgfqpoint{3.438427in}{1.629790in}}%
\pgfpathcurveto{\pgfqpoint{3.449478in}{1.629790in}}{\pgfqpoint{3.460077in}{1.634180in}}{\pgfqpoint{3.467890in}{1.641994in}}%
\pgfpathcurveto{\pgfqpoint{3.475704in}{1.649808in}}{\pgfqpoint{3.480094in}{1.660407in}}{\pgfqpoint{3.480094in}{1.671457in}}%
\pgfpathcurveto{\pgfqpoint{3.480094in}{1.682507in}}{\pgfqpoint{3.475704in}{1.693106in}}{\pgfqpoint{3.467890in}{1.700919in}}%
\pgfpathcurveto{\pgfqpoint{3.460077in}{1.708733in}}{\pgfqpoint{3.449478in}{1.713123in}}{\pgfqpoint{3.438427in}{1.713123in}}%
\pgfpathcurveto{\pgfqpoint{3.427377in}{1.713123in}}{\pgfqpoint{3.416778in}{1.708733in}}{\pgfqpoint{3.408965in}{1.700919in}}%
\pgfpathcurveto{\pgfqpoint{3.401151in}{1.693106in}}{\pgfqpoint{3.396761in}{1.682507in}}{\pgfqpoint{3.396761in}{1.671457in}}%
\pgfpathcurveto{\pgfqpoint{3.396761in}{1.660407in}}{\pgfqpoint{3.401151in}{1.649808in}}{\pgfqpoint{3.408965in}{1.641994in}}%
\pgfpathcurveto{\pgfqpoint{3.416778in}{1.634180in}}{\pgfqpoint{3.427377in}{1.629790in}}{\pgfqpoint{3.438427in}{1.629790in}}%
\pgfpathclose%
\pgfusepath{stroke,fill}%
\end{pgfscope}%
\begin{pgfscope}%
\pgfpathrectangle{\pgfqpoint{0.600000in}{0.600000in}}{\pgfqpoint{3.900000in}{3.900000in}}%
\pgfusepath{clip}%
\pgfsetbuttcap%
\pgfsetroundjoin%
\definecolor{currentfill}{rgb}{0.121569,0.466667,0.705882}%
\pgfsetfillcolor{currentfill}%
\pgfsetlinewidth{1.003750pt}%
\definecolor{currentstroke}{rgb}{0.121569,0.466667,0.705882}%
\pgfsetstrokecolor{currentstroke}%
\pgfsetdash{}{0pt}%
\pgfpathmoveto{\pgfqpoint{2.353389in}{1.815506in}}%
\pgfpathcurveto{\pgfqpoint{2.364439in}{1.815506in}}{\pgfqpoint{2.375038in}{1.819896in}}{\pgfqpoint{2.382852in}{1.827710in}}%
\pgfpathcurveto{\pgfqpoint{2.390665in}{1.835523in}}{\pgfqpoint{2.395056in}{1.846122in}}{\pgfqpoint{2.395056in}{1.857172in}}%
\pgfpathcurveto{\pgfqpoint{2.395056in}{1.868223in}}{\pgfqpoint{2.390665in}{1.878822in}}{\pgfqpoint{2.382852in}{1.886635in}}%
\pgfpathcurveto{\pgfqpoint{2.375038in}{1.894449in}}{\pgfqpoint{2.364439in}{1.898839in}}{\pgfqpoint{2.353389in}{1.898839in}}%
\pgfpathcurveto{\pgfqpoint{2.342339in}{1.898839in}}{\pgfqpoint{2.331740in}{1.894449in}}{\pgfqpoint{2.323926in}{1.886635in}}%
\pgfpathcurveto{\pgfqpoint{2.316112in}{1.878822in}}{\pgfqpoint{2.311722in}{1.868223in}}{\pgfqpoint{2.311722in}{1.857172in}}%
\pgfpathcurveto{\pgfqpoint{2.311722in}{1.846122in}}{\pgfqpoint{2.316112in}{1.835523in}}{\pgfqpoint{2.323926in}{1.827710in}}%
\pgfpathcurveto{\pgfqpoint{2.331740in}{1.819896in}}{\pgfqpoint{2.342339in}{1.815506in}}{\pgfqpoint{2.353389in}{1.815506in}}%
\pgfpathclose%
\pgfusepath{stroke,fill}%
\end{pgfscope}%
\begin{pgfscope}%
\pgfpathrectangle{\pgfqpoint{0.600000in}{0.600000in}}{\pgfqpoint{3.900000in}{3.900000in}}%
\pgfusepath{clip}%
\pgfsetbuttcap%
\pgfsetroundjoin%
\definecolor{currentfill}{rgb}{0.121569,0.466667,0.705882}%
\pgfsetfillcolor{currentfill}%
\pgfsetlinewidth{1.003750pt}%
\definecolor{currentstroke}{rgb}{0.121569,0.466667,0.705882}%
\pgfsetstrokecolor{currentstroke}%
\pgfsetdash{}{0pt}%
\pgfpathmoveto{\pgfqpoint{2.346739in}{2.872651in}}%
\pgfpathcurveto{\pgfqpoint{2.357789in}{2.872651in}}{\pgfqpoint{2.368388in}{2.877041in}}{\pgfqpoint{2.376202in}{2.884855in}}%
\pgfpathcurveto{\pgfqpoint{2.384016in}{2.892668in}}{\pgfqpoint{2.388406in}{2.903267in}}{\pgfqpoint{2.388406in}{2.914318in}}%
\pgfpathcurveto{\pgfqpoint{2.388406in}{2.925368in}}{\pgfqpoint{2.384016in}{2.935967in}}{\pgfqpoint{2.376202in}{2.943780in}}%
\pgfpathcurveto{\pgfqpoint{2.368388in}{2.951594in}}{\pgfqpoint{2.357789in}{2.955984in}}{\pgfqpoint{2.346739in}{2.955984in}}%
\pgfpathcurveto{\pgfqpoint{2.335689in}{2.955984in}}{\pgfqpoint{2.325090in}{2.951594in}}{\pgfqpoint{2.317276in}{2.943780in}}%
\pgfpathcurveto{\pgfqpoint{2.309463in}{2.935967in}}{\pgfqpoint{2.305073in}{2.925368in}}{\pgfqpoint{2.305073in}{2.914318in}}%
\pgfpathcurveto{\pgfqpoint{2.305073in}{2.903267in}}{\pgfqpoint{2.309463in}{2.892668in}}{\pgfqpoint{2.317276in}{2.884855in}}%
\pgfpathcurveto{\pgfqpoint{2.325090in}{2.877041in}}{\pgfqpoint{2.335689in}{2.872651in}}{\pgfqpoint{2.346739in}{2.872651in}}%
\pgfpathclose%
\pgfusepath{stroke,fill}%
\end{pgfscope}%
\begin{pgfscope}%
\pgfpathrectangle{\pgfqpoint{0.600000in}{0.600000in}}{\pgfqpoint{3.900000in}{3.900000in}}%
\pgfusepath{clip}%
\pgfsetbuttcap%
\pgfsetroundjoin%
\definecolor{currentfill}{rgb}{0.121569,0.466667,0.705882}%
\pgfsetfillcolor{currentfill}%
\pgfsetlinewidth{1.003750pt}%
\definecolor{currentstroke}{rgb}{0.121569,0.466667,0.705882}%
\pgfsetstrokecolor{currentstroke}%
\pgfsetdash{}{0pt}%
\pgfpathmoveto{\pgfqpoint{2.631365in}{2.532697in}}%
\pgfpathcurveto{\pgfqpoint{2.642415in}{2.532697in}}{\pgfqpoint{2.653014in}{2.537087in}}{\pgfqpoint{2.660828in}{2.544900in}}%
\pgfpathcurveto{\pgfqpoint{2.668642in}{2.552714in}}{\pgfqpoint{2.673032in}{2.563313in}}{\pgfqpoint{2.673032in}{2.574363in}}%
\pgfpathcurveto{\pgfqpoint{2.673032in}{2.585413in}}{\pgfqpoint{2.668642in}{2.596012in}}{\pgfqpoint{2.660828in}{2.603826in}}%
\pgfpathcurveto{\pgfqpoint{2.653014in}{2.611640in}}{\pgfqpoint{2.642415in}{2.616030in}}{\pgfqpoint{2.631365in}{2.616030in}}%
\pgfpathcurveto{\pgfqpoint{2.620315in}{2.616030in}}{\pgfqpoint{2.609716in}{2.611640in}}{\pgfqpoint{2.601902in}{2.603826in}}%
\pgfpathcurveto{\pgfqpoint{2.594089in}{2.596012in}}{\pgfqpoint{2.589698in}{2.585413in}}{\pgfqpoint{2.589698in}{2.574363in}}%
\pgfpathcurveto{\pgfqpoint{2.589698in}{2.563313in}}{\pgfqpoint{2.594089in}{2.552714in}}{\pgfqpoint{2.601902in}{2.544900in}}%
\pgfpathcurveto{\pgfqpoint{2.609716in}{2.537087in}}{\pgfqpoint{2.620315in}{2.532697in}}{\pgfqpoint{2.631365in}{2.532697in}}%
\pgfpathclose%
\pgfusepath{stroke,fill}%
\end{pgfscope}%
\begin{pgfscope}%
\pgfpathrectangle{\pgfqpoint{0.600000in}{0.600000in}}{\pgfqpoint{3.900000in}{3.900000in}}%
\pgfusepath{clip}%
\pgfsetbuttcap%
\pgfsetroundjoin%
\definecolor{currentfill}{rgb}{0.121569,0.466667,0.705882}%
\pgfsetfillcolor{currentfill}%
\pgfsetlinewidth{1.003750pt}%
\definecolor{currentstroke}{rgb}{0.121569,0.466667,0.705882}%
\pgfsetstrokecolor{currentstroke}%
\pgfsetdash{}{0pt}%
\pgfpathmoveto{\pgfqpoint{2.435906in}{3.040068in}}%
\pgfpathcurveto{\pgfqpoint{2.446956in}{3.040068in}}{\pgfqpoint{2.457555in}{3.044458in}}{\pgfqpoint{2.465369in}{3.052272in}}%
\pgfpathcurveto{\pgfqpoint{2.473183in}{3.060085in}}{\pgfqpoint{2.477573in}{3.070684in}}{\pgfqpoint{2.477573in}{3.081734in}}%
\pgfpathcurveto{\pgfqpoint{2.477573in}{3.092785in}}{\pgfqpoint{2.473183in}{3.103384in}}{\pgfqpoint{2.465369in}{3.111197in}}%
\pgfpathcurveto{\pgfqpoint{2.457555in}{3.119011in}}{\pgfqpoint{2.446956in}{3.123401in}}{\pgfqpoint{2.435906in}{3.123401in}}%
\pgfpathcurveto{\pgfqpoint{2.424856in}{3.123401in}}{\pgfqpoint{2.414257in}{3.119011in}}{\pgfqpoint{2.406443in}{3.111197in}}%
\pgfpathcurveto{\pgfqpoint{2.398630in}{3.103384in}}{\pgfqpoint{2.394239in}{3.092785in}}{\pgfqpoint{2.394239in}{3.081734in}}%
\pgfpathcurveto{\pgfqpoint{2.394239in}{3.070684in}}{\pgfqpoint{2.398630in}{3.060085in}}{\pgfqpoint{2.406443in}{3.052272in}}%
\pgfpathcurveto{\pgfqpoint{2.414257in}{3.044458in}}{\pgfqpoint{2.424856in}{3.040068in}}{\pgfqpoint{2.435906in}{3.040068in}}%
\pgfpathclose%
\pgfusepath{stroke,fill}%
\end{pgfscope}%
\begin{pgfscope}%
\pgfpathrectangle{\pgfqpoint{0.600000in}{0.600000in}}{\pgfqpoint{3.900000in}{3.900000in}}%
\pgfusepath{clip}%
\pgfsetbuttcap%
\pgfsetroundjoin%
\definecolor{currentfill}{rgb}{0.121569,0.466667,0.705882}%
\pgfsetfillcolor{currentfill}%
\pgfsetlinewidth{1.003750pt}%
\definecolor{currentstroke}{rgb}{0.121569,0.466667,0.705882}%
\pgfsetstrokecolor{currentstroke}%
\pgfsetdash{}{0pt}%
\pgfpathmoveto{\pgfqpoint{2.366726in}{2.411136in}}%
\pgfpathcurveto{\pgfqpoint{2.377776in}{2.411136in}}{\pgfqpoint{2.388375in}{2.415526in}}{\pgfqpoint{2.396189in}{2.423340in}}%
\pgfpathcurveto{\pgfqpoint{2.404003in}{2.431153in}}{\pgfqpoint{2.408393in}{2.441752in}}{\pgfqpoint{2.408393in}{2.452802in}}%
\pgfpathcurveto{\pgfqpoint{2.408393in}{2.463853in}}{\pgfqpoint{2.404003in}{2.474452in}}{\pgfqpoint{2.396189in}{2.482265in}}%
\pgfpathcurveto{\pgfqpoint{2.388375in}{2.490079in}}{\pgfqpoint{2.377776in}{2.494469in}}{\pgfqpoint{2.366726in}{2.494469in}}%
\pgfpathcurveto{\pgfqpoint{2.355676in}{2.494469in}}{\pgfqpoint{2.345077in}{2.490079in}}{\pgfqpoint{2.337263in}{2.482265in}}%
\pgfpathcurveto{\pgfqpoint{2.329450in}{2.474452in}}{\pgfqpoint{2.325060in}{2.463853in}}{\pgfqpoint{2.325060in}{2.452802in}}%
\pgfpathcurveto{\pgfqpoint{2.325060in}{2.441752in}}{\pgfqpoint{2.329450in}{2.431153in}}{\pgfqpoint{2.337263in}{2.423340in}}%
\pgfpathcurveto{\pgfqpoint{2.345077in}{2.415526in}}{\pgfqpoint{2.355676in}{2.411136in}}{\pgfqpoint{2.366726in}{2.411136in}}%
\pgfpathclose%
\pgfusepath{stroke,fill}%
\end{pgfscope}%
\begin{pgfscope}%
\pgfpathrectangle{\pgfqpoint{0.600000in}{0.600000in}}{\pgfqpoint{3.900000in}{3.900000in}}%
\pgfusepath{clip}%
\pgfsetbuttcap%
\pgfsetroundjoin%
\definecolor{currentfill}{rgb}{0.121569,0.466667,0.705882}%
\pgfsetfillcolor{currentfill}%
\pgfsetlinewidth{1.003750pt}%
\definecolor{currentstroke}{rgb}{0.121569,0.466667,0.705882}%
\pgfsetstrokecolor{currentstroke}%
\pgfsetdash{}{0pt}%
\pgfpathmoveto{\pgfqpoint{3.249858in}{2.197757in}}%
\pgfpathcurveto{\pgfqpoint{3.260908in}{2.197757in}}{\pgfqpoint{3.271507in}{2.202147in}}{\pgfqpoint{3.279321in}{2.209961in}}%
\pgfpathcurveto{\pgfqpoint{3.287134in}{2.217775in}}{\pgfqpoint{3.291525in}{2.228374in}}{\pgfqpoint{3.291525in}{2.239424in}}%
\pgfpathcurveto{\pgfqpoint{3.291525in}{2.250474in}}{\pgfqpoint{3.287134in}{2.261073in}}{\pgfqpoint{3.279321in}{2.268887in}}%
\pgfpathcurveto{\pgfqpoint{3.271507in}{2.276700in}}{\pgfqpoint{3.260908in}{2.281090in}}{\pgfqpoint{3.249858in}{2.281090in}}%
\pgfpathcurveto{\pgfqpoint{3.238808in}{2.281090in}}{\pgfqpoint{3.228209in}{2.276700in}}{\pgfqpoint{3.220395in}{2.268887in}}%
\pgfpathcurveto{\pgfqpoint{3.212581in}{2.261073in}}{\pgfqpoint{3.208191in}{2.250474in}}{\pgfqpoint{3.208191in}{2.239424in}}%
\pgfpathcurveto{\pgfqpoint{3.208191in}{2.228374in}}{\pgfqpoint{3.212581in}{2.217775in}}{\pgfqpoint{3.220395in}{2.209961in}}%
\pgfpathcurveto{\pgfqpoint{3.228209in}{2.202147in}}{\pgfqpoint{3.238808in}{2.197757in}}{\pgfqpoint{3.249858in}{2.197757in}}%
\pgfpathclose%
\pgfusepath{stroke,fill}%
\end{pgfscope}%
\begin{pgfscope}%
\pgfpathrectangle{\pgfqpoint{0.600000in}{0.600000in}}{\pgfqpoint{3.900000in}{3.900000in}}%
\pgfusepath{clip}%
\pgfsetbuttcap%
\pgfsetroundjoin%
\definecolor{currentfill}{rgb}{0.121569,0.466667,0.705882}%
\pgfsetfillcolor{currentfill}%
\pgfsetlinewidth{1.003750pt}%
\definecolor{currentstroke}{rgb}{0.121569,0.466667,0.705882}%
\pgfsetstrokecolor{currentstroke}%
\pgfsetdash{}{0pt}%
\pgfpathmoveto{\pgfqpoint{3.309359in}{2.807066in}}%
\pgfpathcurveto{\pgfqpoint{3.320409in}{2.807066in}}{\pgfqpoint{3.331008in}{2.811457in}}{\pgfqpoint{3.338822in}{2.819270in}}%
\pgfpathcurveto{\pgfqpoint{3.346635in}{2.827084in}}{\pgfqpoint{3.351026in}{2.837683in}}{\pgfqpoint{3.351026in}{2.848733in}}%
\pgfpathcurveto{\pgfqpoint{3.351026in}{2.859783in}}{\pgfqpoint{3.346635in}{2.870382in}}{\pgfqpoint{3.338822in}{2.878196in}}%
\pgfpathcurveto{\pgfqpoint{3.331008in}{2.886009in}}{\pgfqpoint{3.320409in}{2.890400in}}{\pgfqpoint{3.309359in}{2.890400in}}%
\pgfpathcurveto{\pgfqpoint{3.298309in}{2.890400in}}{\pgfqpoint{3.287710in}{2.886009in}}{\pgfqpoint{3.279896in}{2.878196in}}%
\pgfpathcurveto{\pgfqpoint{3.272083in}{2.870382in}}{\pgfqpoint{3.267692in}{2.859783in}}{\pgfqpoint{3.267692in}{2.848733in}}%
\pgfpathcurveto{\pgfqpoint{3.267692in}{2.837683in}}{\pgfqpoint{3.272083in}{2.827084in}}{\pgfqpoint{3.279896in}{2.819270in}}%
\pgfpathcurveto{\pgfqpoint{3.287710in}{2.811457in}}{\pgfqpoint{3.298309in}{2.807066in}}{\pgfqpoint{3.309359in}{2.807066in}}%
\pgfpathclose%
\pgfusepath{stroke,fill}%
\end{pgfscope}%
\begin{pgfscope}%
\pgfpathrectangle{\pgfqpoint{0.600000in}{0.600000in}}{\pgfqpoint{3.900000in}{3.900000in}}%
\pgfusepath{clip}%
\pgfsetbuttcap%
\pgfsetroundjoin%
\definecolor{currentfill}{rgb}{0.121569,0.466667,0.705882}%
\pgfsetfillcolor{currentfill}%
\pgfsetlinewidth{1.003750pt}%
\definecolor{currentstroke}{rgb}{0.121569,0.466667,0.705882}%
\pgfsetstrokecolor{currentstroke}%
\pgfsetdash{}{0pt}%
\pgfpathmoveto{\pgfqpoint{3.343691in}{3.238966in}}%
\pgfpathcurveto{\pgfqpoint{3.354741in}{3.238966in}}{\pgfqpoint{3.365340in}{3.243356in}}{\pgfqpoint{3.373153in}{3.251170in}}%
\pgfpathcurveto{\pgfqpoint{3.380967in}{3.258984in}}{\pgfqpoint{3.385357in}{3.269583in}}{\pgfqpoint{3.385357in}{3.280633in}}%
\pgfpathcurveto{\pgfqpoint{3.385357in}{3.291683in}}{\pgfqpoint{3.380967in}{3.302282in}}{\pgfqpoint{3.373153in}{3.310095in}}%
\pgfpathcurveto{\pgfqpoint{3.365340in}{3.317909in}}{\pgfqpoint{3.354741in}{3.322299in}}{\pgfqpoint{3.343691in}{3.322299in}}%
\pgfpathcurveto{\pgfqpoint{3.332640in}{3.322299in}}{\pgfqpoint{3.322041in}{3.317909in}}{\pgfqpoint{3.314228in}{3.310095in}}%
\pgfpathcurveto{\pgfqpoint{3.306414in}{3.302282in}}{\pgfqpoint{3.302024in}{3.291683in}}{\pgfqpoint{3.302024in}{3.280633in}}%
\pgfpathcurveto{\pgfqpoint{3.302024in}{3.269583in}}{\pgfqpoint{3.306414in}{3.258984in}}{\pgfqpoint{3.314228in}{3.251170in}}%
\pgfpathcurveto{\pgfqpoint{3.322041in}{3.243356in}}{\pgfqpoint{3.332640in}{3.238966in}}{\pgfqpoint{3.343691in}{3.238966in}}%
\pgfpathclose%
\pgfusepath{stroke,fill}%
\end{pgfscope}%
\begin{pgfscope}%
\pgfpathrectangle{\pgfqpoint{0.600000in}{0.600000in}}{\pgfqpoint{3.900000in}{3.900000in}}%
\pgfusepath{clip}%
\pgfsetbuttcap%
\pgfsetroundjoin%
\definecolor{currentfill}{rgb}{0.121569,0.466667,0.705882}%
\pgfsetfillcolor{currentfill}%
\pgfsetlinewidth{1.003750pt}%
\definecolor{currentstroke}{rgb}{0.121569,0.466667,0.705882}%
\pgfsetstrokecolor{currentstroke}%
\pgfsetdash{}{0pt}%
\pgfpathmoveto{\pgfqpoint{2.958869in}{1.559885in}}%
\pgfpathcurveto{\pgfqpoint{2.969919in}{1.559885in}}{\pgfqpoint{2.980518in}{1.564276in}}{\pgfqpoint{2.988332in}{1.572089in}}%
\pgfpathcurveto{\pgfqpoint{2.996145in}{1.579903in}}{\pgfqpoint{3.000536in}{1.590502in}}{\pgfqpoint{3.000536in}{1.601552in}}%
\pgfpathcurveto{\pgfqpoint{3.000536in}{1.612602in}}{\pgfqpoint{2.996145in}{1.623201in}}{\pgfqpoint{2.988332in}{1.631015in}}%
\pgfpathcurveto{\pgfqpoint{2.980518in}{1.638828in}}{\pgfqpoint{2.969919in}{1.643219in}}{\pgfqpoint{2.958869in}{1.643219in}}%
\pgfpathcurveto{\pgfqpoint{2.947819in}{1.643219in}}{\pgfqpoint{2.937220in}{1.638828in}}{\pgfqpoint{2.929406in}{1.631015in}}%
\pgfpathcurveto{\pgfqpoint{2.921593in}{1.623201in}}{\pgfqpoint{2.917202in}{1.612602in}}{\pgfqpoint{2.917202in}{1.601552in}}%
\pgfpathcurveto{\pgfqpoint{2.917202in}{1.590502in}}{\pgfqpoint{2.921593in}{1.579903in}}{\pgfqpoint{2.929406in}{1.572089in}}%
\pgfpathcurveto{\pgfqpoint{2.937220in}{1.564276in}}{\pgfqpoint{2.947819in}{1.559885in}}{\pgfqpoint{2.958869in}{1.559885in}}%
\pgfpathclose%
\pgfusepath{stroke,fill}%
\end{pgfscope}%
\begin{pgfscope}%
\pgfpathrectangle{\pgfqpoint{0.600000in}{0.600000in}}{\pgfqpoint{3.900000in}{3.900000in}}%
\pgfusepath{clip}%
\pgfsetbuttcap%
\pgfsetroundjoin%
\definecolor{currentfill}{rgb}{0.121569,0.466667,0.705882}%
\pgfsetfillcolor{currentfill}%
\pgfsetlinewidth{1.003750pt}%
\definecolor{currentstroke}{rgb}{0.121569,0.466667,0.705882}%
\pgfsetstrokecolor{currentstroke}%
\pgfsetdash{}{0pt}%
\pgfpathmoveto{\pgfqpoint{3.383646in}{1.768491in}}%
\pgfpathcurveto{\pgfqpoint{3.394696in}{1.768491in}}{\pgfqpoint{3.405295in}{1.772881in}}{\pgfqpoint{3.413109in}{1.780694in}}%
\pgfpathcurveto{\pgfqpoint{3.420923in}{1.788508in}}{\pgfqpoint{3.425313in}{1.799107in}}{\pgfqpoint{3.425313in}{1.810157in}}%
\pgfpathcurveto{\pgfqpoint{3.425313in}{1.821207in}}{\pgfqpoint{3.420923in}{1.831806in}}{\pgfqpoint{3.413109in}{1.839620in}}%
\pgfpathcurveto{\pgfqpoint{3.405295in}{1.847434in}}{\pgfqpoint{3.394696in}{1.851824in}}{\pgfqpoint{3.383646in}{1.851824in}}%
\pgfpathcurveto{\pgfqpoint{3.372596in}{1.851824in}}{\pgfqpoint{3.361997in}{1.847434in}}{\pgfqpoint{3.354183in}{1.839620in}}%
\pgfpathcurveto{\pgfqpoint{3.346370in}{1.831806in}}{\pgfqpoint{3.341980in}{1.821207in}}{\pgfqpoint{3.341980in}{1.810157in}}%
\pgfpathcurveto{\pgfqpoint{3.341980in}{1.799107in}}{\pgfqpoint{3.346370in}{1.788508in}}{\pgfqpoint{3.354183in}{1.780694in}}%
\pgfpathcurveto{\pgfqpoint{3.361997in}{1.772881in}}{\pgfqpoint{3.372596in}{1.768491in}}{\pgfqpoint{3.383646in}{1.768491in}}%
\pgfpathclose%
\pgfusepath{stroke,fill}%
\end{pgfscope}%
\begin{pgfscope}%
\pgfpathrectangle{\pgfqpoint{0.600000in}{0.600000in}}{\pgfqpoint{3.900000in}{3.900000in}}%
\pgfusepath{clip}%
\pgfsetbuttcap%
\pgfsetroundjoin%
\definecolor{currentfill}{rgb}{0.121569,0.466667,0.705882}%
\pgfsetfillcolor{currentfill}%
\pgfsetlinewidth{1.003750pt}%
\definecolor{currentstroke}{rgb}{0.121569,0.466667,0.705882}%
\pgfsetstrokecolor{currentstroke}%
\pgfsetdash{}{0pt}%
\pgfpathmoveto{\pgfqpoint{1.784444in}{3.286910in}}%
\pgfpathcurveto{\pgfqpoint{1.795494in}{3.286910in}}{\pgfqpoint{1.806093in}{3.291300in}}{\pgfqpoint{1.813906in}{3.299114in}}%
\pgfpathcurveto{\pgfqpoint{1.821720in}{3.306928in}}{\pgfqpoint{1.826110in}{3.317527in}}{\pgfqpoint{1.826110in}{3.328577in}}%
\pgfpathcurveto{\pgfqpoint{1.826110in}{3.339627in}}{\pgfqpoint{1.821720in}{3.350226in}}{\pgfqpoint{1.813906in}{3.358040in}}%
\pgfpathcurveto{\pgfqpoint{1.806093in}{3.365853in}}{\pgfqpoint{1.795494in}{3.370243in}}{\pgfqpoint{1.784444in}{3.370243in}}%
\pgfpathcurveto{\pgfqpoint{1.773393in}{3.370243in}}{\pgfqpoint{1.762794in}{3.365853in}}{\pgfqpoint{1.754981in}{3.358040in}}%
\pgfpathcurveto{\pgfqpoint{1.747167in}{3.350226in}}{\pgfqpoint{1.742777in}{3.339627in}}{\pgfqpoint{1.742777in}{3.328577in}}%
\pgfpathcurveto{\pgfqpoint{1.742777in}{3.317527in}}{\pgfqpoint{1.747167in}{3.306928in}}{\pgfqpoint{1.754981in}{3.299114in}}%
\pgfpathcurveto{\pgfqpoint{1.762794in}{3.291300in}}{\pgfqpoint{1.773393in}{3.286910in}}{\pgfqpoint{1.784444in}{3.286910in}}%
\pgfpathclose%
\pgfusepath{stroke,fill}%
\end{pgfscope}%
\begin{pgfscope}%
\pgfpathrectangle{\pgfqpoint{0.600000in}{0.600000in}}{\pgfqpoint{3.900000in}{3.900000in}}%
\pgfusepath{clip}%
\pgfsetbuttcap%
\pgfsetroundjoin%
\definecolor{currentfill}{rgb}{0.121569,0.466667,0.705882}%
\pgfsetfillcolor{currentfill}%
\pgfsetlinewidth{1.003750pt}%
\definecolor{currentstroke}{rgb}{0.121569,0.466667,0.705882}%
\pgfsetstrokecolor{currentstroke}%
\pgfsetdash{}{0pt}%
\pgfpathmoveto{\pgfqpoint{2.622028in}{2.830842in}}%
\pgfpathcurveto{\pgfqpoint{2.633078in}{2.830842in}}{\pgfqpoint{2.643677in}{2.835232in}}{\pgfqpoint{2.651491in}{2.843045in}}%
\pgfpathcurveto{\pgfqpoint{2.659304in}{2.850859in}}{\pgfqpoint{2.663694in}{2.861458in}}{\pgfqpoint{2.663694in}{2.872508in}}%
\pgfpathcurveto{\pgfqpoint{2.663694in}{2.883558in}}{\pgfqpoint{2.659304in}{2.894157in}}{\pgfqpoint{2.651491in}{2.901971in}}%
\pgfpathcurveto{\pgfqpoint{2.643677in}{2.909785in}}{\pgfqpoint{2.633078in}{2.914175in}}{\pgfqpoint{2.622028in}{2.914175in}}%
\pgfpathcurveto{\pgfqpoint{2.610978in}{2.914175in}}{\pgfqpoint{2.600379in}{2.909785in}}{\pgfqpoint{2.592565in}{2.901971in}}%
\pgfpathcurveto{\pgfqpoint{2.584751in}{2.894157in}}{\pgfqpoint{2.580361in}{2.883558in}}{\pgfqpoint{2.580361in}{2.872508in}}%
\pgfpathcurveto{\pgfqpoint{2.580361in}{2.861458in}}{\pgfqpoint{2.584751in}{2.850859in}}{\pgfqpoint{2.592565in}{2.843045in}}%
\pgfpathcurveto{\pgfqpoint{2.600379in}{2.835232in}}{\pgfqpoint{2.610978in}{2.830842in}}{\pgfqpoint{2.622028in}{2.830842in}}%
\pgfpathclose%
\pgfusepath{stroke,fill}%
\end{pgfscope}%
\begin{pgfscope}%
\pgfpathrectangle{\pgfqpoint{0.600000in}{0.600000in}}{\pgfqpoint{3.900000in}{3.900000in}}%
\pgfusepath{clip}%
\pgfsetbuttcap%
\pgfsetroundjoin%
\definecolor{currentfill}{rgb}{0.121569,0.466667,0.705882}%
\pgfsetfillcolor{currentfill}%
\pgfsetlinewidth{1.003750pt}%
\definecolor{currentstroke}{rgb}{0.121569,0.466667,0.705882}%
\pgfsetstrokecolor{currentstroke}%
\pgfsetdash{}{0pt}%
\pgfpathmoveto{\pgfqpoint{2.813804in}{2.801155in}}%
\pgfpathcurveto{\pgfqpoint{2.824854in}{2.801155in}}{\pgfqpoint{2.835453in}{2.805545in}}{\pgfqpoint{2.843267in}{2.813359in}}%
\pgfpathcurveto{\pgfqpoint{2.851081in}{2.821172in}}{\pgfqpoint{2.855471in}{2.831772in}}{\pgfqpoint{2.855471in}{2.842822in}}%
\pgfpathcurveto{\pgfqpoint{2.855471in}{2.853872in}}{\pgfqpoint{2.851081in}{2.864471in}}{\pgfqpoint{2.843267in}{2.872284in}}%
\pgfpathcurveto{\pgfqpoint{2.835453in}{2.880098in}}{\pgfqpoint{2.824854in}{2.884488in}}{\pgfqpoint{2.813804in}{2.884488in}}%
\pgfpathcurveto{\pgfqpoint{2.802754in}{2.884488in}}{\pgfqpoint{2.792155in}{2.880098in}}{\pgfqpoint{2.784342in}{2.872284in}}%
\pgfpathcurveto{\pgfqpoint{2.776528in}{2.864471in}}{\pgfqpoint{2.772138in}{2.853872in}}{\pgfqpoint{2.772138in}{2.842822in}}%
\pgfpathcurveto{\pgfqpoint{2.772138in}{2.831772in}}{\pgfqpoint{2.776528in}{2.821172in}}{\pgfqpoint{2.784342in}{2.813359in}}%
\pgfpathcurveto{\pgfqpoint{2.792155in}{2.805545in}}{\pgfqpoint{2.802754in}{2.801155in}}{\pgfqpoint{2.813804in}{2.801155in}}%
\pgfpathclose%
\pgfusepath{stroke,fill}%
\end{pgfscope}%
\begin{pgfscope}%
\pgfpathrectangle{\pgfqpoint{0.600000in}{0.600000in}}{\pgfqpoint{3.900000in}{3.900000in}}%
\pgfusepath{clip}%
\pgfsetbuttcap%
\pgfsetroundjoin%
\definecolor{currentfill}{rgb}{0.121569,0.466667,0.705882}%
\pgfsetfillcolor{currentfill}%
\pgfsetlinewidth{1.003750pt}%
\definecolor{currentstroke}{rgb}{0.121569,0.466667,0.705882}%
\pgfsetstrokecolor{currentstroke}%
\pgfsetdash{}{0pt}%
\pgfpathmoveto{\pgfqpoint{3.145554in}{2.724187in}}%
\pgfpathcurveto{\pgfqpoint{3.156604in}{2.724187in}}{\pgfqpoint{3.167203in}{2.728577in}}{\pgfqpoint{3.175017in}{2.736391in}}%
\pgfpathcurveto{\pgfqpoint{3.182830in}{2.744205in}}{\pgfqpoint{3.187220in}{2.754804in}}{\pgfqpoint{3.187220in}{2.765854in}}%
\pgfpathcurveto{\pgfqpoint{3.187220in}{2.776904in}}{\pgfqpoint{3.182830in}{2.787503in}}{\pgfqpoint{3.175017in}{2.795317in}}%
\pgfpathcurveto{\pgfqpoint{3.167203in}{2.803130in}}{\pgfqpoint{3.156604in}{2.807520in}}{\pgfqpoint{3.145554in}{2.807520in}}%
\pgfpathcurveto{\pgfqpoint{3.134504in}{2.807520in}}{\pgfqpoint{3.123905in}{2.803130in}}{\pgfqpoint{3.116091in}{2.795317in}}%
\pgfpathcurveto{\pgfqpoint{3.108277in}{2.787503in}}{\pgfqpoint{3.103887in}{2.776904in}}{\pgfqpoint{3.103887in}{2.765854in}}%
\pgfpathcurveto{\pgfqpoint{3.103887in}{2.754804in}}{\pgfqpoint{3.108277in}{2.744205in}}{\pgfqpoint{3.116091in}{2.736391in}}%
\pgfpathcurveto{\pgfqpoint{3.123905in}{2.728577in}}{\pgfqpoint{3.134504in}{2.724187in}}{\pgfqpoint{3.145554in}{2.724187in}}%
\pgfpathclose%
\pgfusepath{stroke,fill}%
\end{pgfscope}%
\begin{pgfscope}%
\pgfpathrectangle{\pgfqpoint{0.600000in}{0.600000in}}{\pgfqpoint{3.900000in}{3.900000in}}%
\pgfusepath{clip}%
\pgfsetbuttcap%
\pgfsetroundjoin%
\definecolor{currentfill}{rgb}{0.121569,0.466667,0.705882}%
\pgfsetfillcolor{currentfill}%
\pgfsetlinewidth{1.003750pt}%
\definecolor{currentstroke}{rgb}{0.121569,0.466667,0.705882}%
\pgfsetstrokecolor{currentstroke}%
\pgfsetdash{}{0pt}%
\pgfpathmoveto{\pgfqpoint{2.265348in}{2.986168in}}%
\pgfpathcurveto{\pgfqpoint{2.276398in}{2.986168in}}{\pgfqpoint{2.286997in}{2.990558in}}{\pgfqpoint{2.294811in}{2.998372in}}%
\pgfpathcurveto{\pgfqpoint{2.302624in}{3.006186in}}{\pgfqpoint{2.307015in}{3.016785in}}{\pgfqpoint{2.307015in}{3.027835in}}%
\pgfpathcurveto{\pgfqpoint{2.307015in}{3.038885in}}{\pgfqpoint{2.302624in}{3.049484in}}{\pgfqpoint{2.294811in}{3.057298in}}%
\pgfpathcurveto{\pgfqpoint{2.286997in}{3.065111in}}{\pgfqpoint{2.276398in}{3.069501in}}{\pgfqpoint{2.265348in}{3.069501in}}%
\pgfpathcurveto{\pgfqpoint{2.254298in}{3.069501in}}{\pgfqpoint{2.243699in}{3.065111in}}{\pgfqpoint{2.235885in}{3.057298in}}%
\pgfpathcurveto{\pgfqpoint{2.228072in}{3.049484in}}{\pgfqpoint{2.223681in}{3.038885in}}{\pgfqpoint{2.223681in}{3.027835in}}%
\pgfpathcurveto{\pgfqpoint{2.223681in}{3.016785in}}{\pgfqpoint{2.228072in}{3.006186in}}{\pgfqpoint{2.235885in}{2.998372in}}%
\pgfpathcurveto{\pgfqpoint{2.243699in}{2.990558in}}{\pgfqpoint{2.254298in}{2.986168in}}{\pgfqpoint{2.265348in}{2.986168in}}%
\pgfpathclose%
\pgfusepath{stroke,fill}%
\end{pgfscope}%
\begin{pgfscope}%
\pgfpathrectangle{\pgfqpoint{0.600000in}{0.600000in}}{\pgfqpoint{3.900000in}{3.900000in}}%
\pgfusepath{clip}%
\pgfsetbuttcap%
\pgfsetroundjoin%
\definecolor{currentfill}{rgb}{0.121569,0.466667,0.705882}%
\pgfsetfillcolor{currentfill}%
\pgfsetlinewidth{1.003750pt}%
\definecolor{currentstroke}{rgb}{0.121569,0.466667,0.705882}%
\pgfsetstrokecolor{currentstroke}%
\pgfsetdash{}{0pt}%
\pgfpathmoveto{\pgfqpoint{2.717517in}{1.878586in}}%
\pgfpathcurveto{\pgfqpoint{2.728567in}{1.878586in}}{\pgfqpoint{2.739166in}{1.882976in}}{\pgfqpoint{2.746980in}{1.890790in}}%
\pgfpathcurveto{\pgfqpoint{2.754794in}{1.898603in}}{\pgfqpoint{2.759184in}{1.909202in}}{\pgfqpoint{2.759184in}{1.920252in}}%
\pgfpathcurveto{\pgfqpoint{2.759184in}{1.931303in}}{\pgfqpoint{2.754794in}{1.941902in}}{\pgfqpoint{2.746980in}{1.949715in}}%
\pgfpathcurveto{\pgfqpoint{2.739166in}{1.957529in}}{\pgfqpoint{2.728567in}{1.961919in}}{\pgfqpoint{2.717517in}{1.961919in}}%
\pgfpathcurveto{\pgfqpoint{2.706467in}{1.961919in}}{\pgfqpoint{2.695868in}{1.957529in}}{\pgfqpoint{2.688054in}{1.949715in}}%
\pgfpathcurveto{\pgfqpoint{2.680241in}{1.941902in}}{\pgfqpoint{2.675851in}{1.931303in}}{\pgfqpoint{2.675851in}{1.920252in}}%
\pgfpathcurveto{\pgfqpoint{2.675851in}{1.909202in}}{\pgfqpoint{2.680241in}{1.898603in}}{\pgfqpoint{2.688054in}{1.890790in}}%
\pgfpathcurveto{\pgfqpoint{2.695868in}{1.882976in}}{\pgfqpoint{2.706467in}{1.878586in}}{\pgfqpoint{2.717517in}{1.878586in}}%
\pgfpathclose%
\pgfusepath{stroke,fill}%
\end{pgfscope}%
\begin{pgfscope}%
\pgfpathrectangle{\pgfqpoint{0.600000in}{0.600000in}}{\pgfqpoint{3.900000in}{3.900000in}}%
\pgfusepath{clip}%
\pgfsetbuttcap%
\pgfsetroundjoin%
\definecolor{currentfill}{rgb}{0.121569,0.466667,0.705882}%
\pgfsetfillcolor{currentfill}%
\pgfsetlinewidth{1.003750pt}%
\definecolor{currentstroke}{rgb}{0.121569,0.466667,0.705882}%
\pgfsetstrokecolor{currentstroke}%
\pgfsetdash{}{0pt}%
\pgfpathmoveto{\pgfqpoint{1.064963in}{2.592167in}}%
\pgfpathcurveto{\pgfqpoint{1.076013in}{2.592167in}}{\pgfqpoint{1.086612in}{2.596557in}}{\pgfqpoint{1.094425in}{2.604371in}}%
\pgfpathcurveto{\pgfqpoint{1.102239in}{2.612184in}}{\pgfqpoint{1.106629in}{2.622783in}}{\pgfqpoint{1.106629in}{2.633833in}}%
\pgfpathcurveto{\pgfqpoint{1.106629in}{2.644884in}}{\pgfqpoint{1.102239in}{2.655483in}}{\pgfqpoint{1.094425in}{2.663296in}}%
\pgfpathcurveto{\pgfqpoint{1.086612in}{2.671110in}}{\pgfqpoint{1.076013in}{2.675500in}}{\pgfqpoint{1.064963in}{2.675500in}}%
\pgfpathcurveto{\pgfqpoint{1.053913in}{2.675500in}}{\pgfqpoint{1.043313in}{2.671110in}}{\pgfqpoint{1.035500in}{2.663296in}}%
\pgfpathcurveto{\pgfqpoint{1.027686in}{2.655483in}}{\pgfqpoint{1.023296in}{2.644884in}}{\pgfqpoint{1.023296in}{2.633833in}}%
\pgfpathcurveto{\pgfqpoint{1.023296in}{2.622783in}}{\pgfqpoint{1.027686in}{2.612184in}}{\pgfqpoint{1.035500in}{2.604371in}}%
\pgfpathcurveto{\pgfqpoint{1.043313in}{2.596557in}}{\pgfqpoint{1.053913in}{2.592167in}}{\pgfqpoint{1.064963in}{2.592167in}}%
\pgfpathclose%
\pgfusepath{stroke,fill}%
\end{pgfscope}%
\begin{pgfscope}%
\pgfpathrectangle{\pgfqpoint{0.600000in}{0.600000in}}{\pgfqpoint{3.900000in}{3.900000in}}%
\pgfusepath{clip}%
\pgfsetbuttcap%
\pgfsetroundjoin%
\definecolor{currentfill}{rgb}{0.121569,0.466667,0.705882}%
\pgfsetfillcolor{currentfill}%
\pgfsetlinewidth{1.003750pt}%
\definecolor{currentstroke}{rgb}{0.121569,0.466667,0.705882}%
\pgfsetstrokecolor{currentstroke}%
\pgfsetdash{}{0pt}%
\pgfpathmoveto{\pgfqpoint{3.012220in}{0.901228in}}%
\pgfpathcurveto{\pgfqpoint{3.023270in}{0.901228in}}{\pgfqpoint{3.033869in}{0.905618in}}{\pgfqpoint{3.041682in}{0.913432in}}%
\pgfpathcurveto{\pgfqpoint{3.049496in}{0.921245in}}{\pgfqpoint{3.053886in}{0.931844in}}{\pgfqpoint{3.053886in}{0.942895in}}%
\pgfpathcurveto{\pgfqpoint{3.053886in}{0.953945in}}{\pgfqpoint{3.049496in}{0.964544in}}{\pgfqpoint{3.041682in}{0.972357in}}%
\pgfpathcurveto{\pgfqpoint{3.033869in}{0.980171in}}{\pgfqpoint{3.023270in}{0.984561in}}{\pgfqpoint{3.012220in}{0.984561in}}%
\pgfpathcurveto{\pgfqpoint{3.001169in}{0.984561in}}{\pgfqpoint{2.990570in}{0.980171in}}{\pgfqpoint{2.982757in}{0.972357in}}%
\pgfpathcurveto{\pgfqpoint{2.974943in}{0.964544in}}{\pgfqpoint{2.970553in}{0.953945in}}{\pgfqpoint{2.970553in}{0.942895in}}%
\pgfpathcurveto{\pgfqpoint{2.970553in}{0.931844in}}{\pgfqpoint{2.974943in}{0.921245in}}{\pgfqpoint{2.982757in}{0.913432in}}%
\pgfpathcurveto{\pgfqpoint{2.990570in}{0.905618in}}{\pgfqpoint{3.001169in}{0.901228in}}{\pgfqpoint{3.012220in}{0.901228in}}%
\pgfpathclose%
\pgfusepath{stroke,fill}%
\end{pgfscope}%
\begin{pgfscope}%
\pgfpathrectangle{\pgfqpoint{0.600000in}{0.600000in}}{\pgfqpoint{3.900000in}{3.900000in}}%
\pgfusepath{clip}%
\pgfsetbuttcap%
\pgfsetroundjoin%
\definecolor{currentfill}{rgb}{0.121569,0.466667,0.705882}%
\pgfsetfillcolor{currentfill}%
\pgfsetlinewidth{1.003750pt}%
\definecolor{currentstroke}{rgb}{0.121569,0.466667,0.705882}%
\pgfsetstrokecolor{currentstroke}%
\pgfsetdash{}{0pt}%
\pgfpathmoveto{\pgfqpoint{2.293510in}{2.047704in}}%
\pgfpathcurveto{\pgfqpoint{2.304561in}{2.047704in}}{\pgfqpoint{2.315160in}{2.052094in}}{\pgfqpoint{2.322973in}{2.059908in}}%
\pgfpathcurveto{\pgfqpoint{2.330787in}{2.067721in}}{\pgfqpoint{2.335177in}{2.078320in}}{\pgfqpoint{2.335177in}{2.089370in}}%
\pgfpathcurveto{\pgfqpoint{2.335177in}{2.100421in}}{\pgfqpoint{2.330787in}{2.111020in}}{\pgfqpoint{2.322973in}{2.118833in}}%
\pgfpathcurveto{\pgfqpoint{2.315160in}{2.126647in}}{\pgfqpoint{2.304561in}{2.131037in}}{\pgfqpoint{2.293510in}{2.131037in}}%
\pgfpathcurveto{\pgfqpoint{2.282460in}{2.131037in}}{\pgfqpoint{2.271861in}{2.126647in}}{\pgfqpoint{2.264048in}{2.118833in}}%
\pgfpathcurveto{\pgfqpoint{2.256234in}{2.111020in}}{\pgfqpoint{2.251844in}{2.100421in}}{\pgfqpoint{2.251844in}{2.089370in}}%
\pgfpathcurveto{\pgfqpoint{2.251844in}{2.078320in}}{\pgfqpoint{2.256234in}{2.067721in}}{\pgfqpoint{2.264048in}{2.059908in}}%
\pgfpathcurveto{\pgfqpoint{2.271861in}{2.052094in}}{\pgfqpoint{2.282460in}{2.047704in}}{\pgfqpoint{2.293510in}{2.047704in}}%
\pgfpathclose%
\pgfusepath{stroke,fill}%
\end{pgfscope}%
\begin{pgfscope}%
\pgfpathrectangle{\pgfqpoint{0.600000in}{0.600000in}}{\pgfqpoint{3.900000in}{3.900000in}}%
\pgfusepath{clip}%
\pgfsetbuttcap%
\pgfsetroundjoin%
\definecolor{currentfill}{rgb}{0.121569,0.466667,0.705882}%
\pgfsetfillcolor{currentfill}%
\pgfsetlinewidth{1.003750pt}%
\definecolor{currentstroke}{rgb}{0.121569,0.466667,0.705882}%
\pgfsetstrokecolor{currentstroke}%
\pgfsetdash{}{0pt}%
\pgfpathmoveto{\pgfqpoint{3.338843in}{2.275163in}}%
\pgfpathcurveto{\pgfqpoint{3.349893in}{2.275163in}}{\pgfqpoint{3.360492in}{2.279554in}}{\pgfqpoint{3.368306in}{2.287367in}}%
\pgfpathcurveto{\pgfqpoint{3.376119in}{2.295181in}}{\pgfqpoint{3.380510in}{2.305780in}}{\pgfqpoint{3.380510in}{2.316830in}}%
\pgfpathcurveto{\pgfqpoint{3.380510in}{2.327880in}}{\pgfqpoint{3.376119in}{2.338479in}}{\pgfqpoint{3.368306in}{2.346293in}}%
\pgfpathcurveto{\pgfqpoint{3.360492in}{2.354106in}}{\pgfqpoint{3.349893in}{2.358497in}}{\pgfqpoint{3.338843in}{2.358497in}}%
\pgfpathcurveto{\pgfqpoint{3.327793in}{2.358497in}}{\pgfqpoint{3.317194in}{2.354106in}}{\pgfqpoint{3.309380in}{2.346293in}}%
\pgfpathcurveto{\pgfqpoint{3.301567in}{2.338479in}}{\pgfqpoint{3.297176in}{2.327880in}}{\pgfqpoint{3.297176in}{2.316830in}}%
\pgfpathcurveto{\pgfqpoint{3.297176in}{2.305780in}}{\pgfqpoint{3.301567in}{2.295181in}}{\pgfqpoint{3.309380in}{2.287367in}}%
\pgfpathcurveto{\pgfqpoint{3.317194in}{2.279554in}}{\pgfqpoint{3.327793in}{2.275163in}}{\pgfqpoint{3.338843in}{2.275163in}}%
\pgfpathclose%
\pgfusepath{stroke,fill}%
\end{pgfscope}%
\begin{pgfscope}%
\pgfpathrectangle{\pgfqpoint{0.600000in}{0.600000in}}{\pgfqpoint{3.900000in}{3.900000in}}%
\pgfusepath{clip}%
\pgfsetbuttcap%
\pgfsetroundjoin%
\definecolor{currentfill}{rgb}{0.121569,0.466667,0.705882}%
\pgfsetfillcolor{currentfill}%
\pgfsetlinewidth{1.003750pt}%
\definecolor{currentstroke}{rgb}{0.121569,0.466667,0.705882}%
\pgfsetstrokecolor{currentstroke}%
\pgfsetdash{}{0pt}%
\pgfpathmoveto{\pgfqpoint{2.675084in}{3.095147in}}%
\pgfpathcurveto{\pgfqpoint{2.686134in}{3.095147in}}{\pgfqpoint{2.696733in}{3.099538in}}{\pgfqpoint{2.704547in}{3.107351in}}%
\pgfpathcurveto{\pgfqpoint{2.712360in}{3.115165in}}{\pgfqpoint{2.716751in}{3.125764in}}{\pgfqpoint{2.716751in}{3.136814in}}%
\pgfpathcurveto{\pgfqpoint{2.716751in}{3.147864in}}{\pgfqpoint{2.712360in}{3.158463in}}{\pgfqpoint{2.704547in}{3.166277in}}%
\pgfpathcurveto{\pgfqpoint{2.696733in}{3.174091in}}{\pgfqpoint{2.686134in}{3.178481in}}{\pgfqpoint{2.675084in}{3.178481in}}%
\pgfpathcurveto{\pgfqpoint{2.664034in}{3.178481in}}{\pgfqpoint{2.653435in}{3.174091in}}{\pgfqpoint{2.645621in}{3.166277in}}%
\pgfpathcurveto{\pgfqpoint{2.637807in}{3.158463in}}{\pgfqpoint{2.633417in}{3.147864in}}{\pgfqpoint{2.633417in}{3.136814in}}%
\pgfpathcurveto{\pgfqpoint{2.633417in}{3.125764in}}{\pgfqpoint{2.637807in}{3.115165in}}{\pgfqpoint{2.645621in}{3.107351in}}%
\pgfpathcurveto{\pgfqpoint{2.653435in}{3.099538in}}{\pgfqpoint{2.664034in}{3.095147in}}{\pgfqpoint{2.675084in}{3.095147in}}%
\pgfpathclose%
\pgfusepath{stroke,fill}%
\end{pgfscope}%
\begin{pgfscope}%
\pgfpathrectangle{\pgfqpoint{0.600000in}{0.600000in}}{\pgfqpoint{3.900000in}{3.900000in}}%
\pgfusepath{clip}%
\pgfsetbuttcap%
\pgfsetroundjoin%
\definecolor{currentfill}{rgb}{0.121569,0.466667,0.705882}%
\pgfsetfillcolor{currentfill}%
\pgfsetlinewidth{1.003750pt}%
\definecolor{currentstroke}{rgb}{0.121569,0.466667,0.705882}%
\pgfsetstrokecolor{currentstroke}%
\pgfsetdash{}{0pt}%
\pgfpathmoveto{\pgfqpoint{2.689567in}{2.329469in}}%
\pgfpathcurveto{\pgfqpoint{2.700617in}{2.329469in}}{\pgfqpoint{2.711216in}{2.333859in}}{\pgfqpoint{2.719029in}{2.341673in}}%
\pgfpathcurveto{\pgfqpoint{2.726843in}{2.349487in}}{\pgfqpoint{2.731233in}{2.360086in}}{\pgfqpoint{2.731233in}{2.371136in}}%
\pgfpathcurveto{\pgfqpoint{2.731233in}{2.382186in}}{\pgfqpoint{2.726843in}{2.392785in}}{\pgfqpoint{2.719029in}{2.400599in}}%
\pgfpathcurveto{\pgfqpoint{2.711216in}{2.408412in}}{\pgfqpoint{2.700617in}{2.412802in}}{\pgfqpoint{2.689567in}{2.412802in}}%
\pgfpathcurveto{\pgfqpoint{2.678517in}{2.412802in}}{\pgfqpoint{2.667918in}{2.408412in}}{\pgfqpoint{2.660104in}{2.400599in}}%
\pgfpathcurveto{\pgfqpoint{2.652290in}{2.392785in}}{\pgfqpoint{2.647900in}{2.382186in}}{\pgfqpoint{2.647900in}{2.371136in}}%
\pgfpathcurveto{\pgfqpoint{2.647900in}{2.360086in}}{\pgfqpoint{2.652290in}{2.349487in}}{\pgfqpoint{2.660104in}{2.341673in}}%
\pgfpathcurveto{\pgfqpoint{2.667918in}{2.333859in}}{\pgfqpoint{2.678517in}{2.329469in}}{\pgfqpoint{2.689567in}{2.329469in}}%
\pgfpathclose%
\pgfusepath{stroke,fill}%
\end{pgfscope}%
\begin{pgfscope}%
\pgfpathrectangle{\pgfqpoint{0.600000in}{0.600000in}}{\pgfqpoint{3.900000in}{3.900000in}}%
\pgfusepath{clip}%
\pgfsetbuttcap%
\pgfsetroundjoin%
\definecolor{currentfill}{rgb}{0.121569,0.466667,0.705882}%
\pgfsetfillcolor{currentfill}%
\pgfsetlinewidth{1.003750pt}%
\definecolor{currentstroke}{rgb}{0.121569,0.466667,0.705882}%
\pgfsetstrokecolor{currentstroke}%
\pgfsetdash{}{0pt}%
\pgfpathmoveto{\pgfqpoint{2.929305in}{2.578207in}}%
\pgfpathcurveto{\pgfqpoint{2.940355in}{2.578207in}}{\pgfqpoint{2.950955in}{2.582597in}}{\pgfqpoint{2.958768in}{2.590411in}}%
\pgfpathcurveto{\pgfqpoint{2.966582in}{2.598224in}}{\pgfqpoint{2.970972in}{2.608823in}}{\pgfqpoint{2.970972in}{2.619874in}}%
\pgfpathcurveto{\pgfqpoint{2.970972in}{2.630924in}}{\pgfqpoint{2.966582in}{2.641523in}}{\pgfqpoint{2.958768in}{2.649336in}}%
\pgfpathcurveto{\pgfqpoint{2.950955in}{2.657150in}}{\pgfqpoint{2.940355in}{2.661540in}}{\pgfqpoint{2.929305in}{2.661540in}}%
\pgfpathcurveto{\pgfqpoint{2.918255in}{2.661540in}}{\pgfqpoint{2.907656in}{2.657150in}}{\pgfqpoint{2.899843in}{2.649336in}}%
\pgfpathcurveto{\pgfqpoint{2.892029in}{2.641523in}}{\pgfqpoint{2.887639in}{2.630924in}}{\pgfqpoint{2.887639in}{2.619874in}}%
\pgfpathcurveto{\pgfqpoint{2.887639in}{2.608823in}}{\pgfqpoint{2.892029in}{2.598224in}}{\pgfqpoint{2.899843in}{2.590411in}}%
\pgfpathcurveto{\pgfqpoint{2.907656in}{2.582597in}}{\pgfqpoint{2.918255in}{2.578207in}}{\pgfqpoint{2.929305in}{2.578207in}}%
\pgfpathclose%
\pgfusepath{stroke,fill}%
\end{pgfscope}%
\begin{pgfscope}%
\pgfpathrectangle{\pgfqpoint{0.600000in}{0.600000in}}{\pgfqpoint{3.900000in}{3.900000in}}%
\pgfusepath{clip}%
\pgfsetbuttcap%
\pgfsetroundjoin%
\definecolor{currentfill}{rgb}{0.121569,0.466667,0.705882}%
\pgfsetfillcolor{currentfill}%
\pgfsetlinewidth{1.003750pt}%
\definecolor{currentstroke}{rgb}{0.121569,0.466667,0.705882}%
\pgfsetstrokecolor{currentstroke}%
\pgfsetdash{}{0pt}%
\pgfpathmoveto{\pgfqpoint{2.193651in}{3.513124in}}%
\pgfpathcurveto{\pgfqpoint{2.204701in}{3.513124in}}{\pgfqpoint{2.215300in}{3.517515in}}{\pgfqpoint{2.223114in}{3.525328in}}%
\pgfpathcurveto{\pgfqpoint{2.230928in}{3.533142in}}{\pgfqpoint{2.235318in}{3.543741in}}{\pgfqpoint{2.235318in}{3.554791in}}%
\pgfpathcurveto{\pgfqpoint{2.235318in}{3.565841in}}{\pgfqpoint{2.230928in}{3.576440in}}{\pgfqpoint{2.223114in}{3.584254in}}%
\pgfpathcurveto{\pgfqpoint{2.215300in}{3.592067in}}{\pgfqpoint{2.204701in}{3.596458in}}{\pgfqpoint{2.193651in}{3.596458in}}%
\pgfpathcurveto{\pgfqpoint{2.182601in}{3.596458in}}{\pgfqpoint{2.172002in}{3.592067in}}{\pgfqpoint{2.164188in}{3.584254in}}%
\pgfpathcurveto{\pgfqpoint{2.156375in}{3.576440in}}{\pgfqpoint{2.151985in}{3.565841in}}{\pgfqpoint{2.151985in}{3.554791in}}%
\pgfpathcurveto{\pgfqpoint{2.151985in}{3.543741in}}{\pgfqpoint{2.156375in}{3.533142in}}{\pgfqpoint{2.164188in}{3.525328in}}%
\pgfpathcurveto{\pgfqpoint{2.172002in}{3.517515in}}{\pgfqpoint{2.182601in}{3.513124in}}{\pgfqpoint{2.193651in}{3.513124in}}%
\pgfpathclose%
\pgfusepath{stroke,fill}%
\end{pgfscope}%
\begin{pgfscope}%
\pgfpathrectangle{\pgfqpoint{0.600000in}{0.600000in}}{\pgfqpoint{3.900000in}{3.900000in}}%
\pgfusepath{clip}%
\pgfsetbuttcap%
\pgfsetroundjoin%
\definecolor{currentfill}{rgb}{0.121569,0.466667,0.705882}%
\pgfsetfillcolor{currentfill}%
\pgfsetlinewidth{1.003750pt}%
\definecolor{currentstroke}{rgb}{0.121569,0.466667,0.705882}%
\pgfsetstrokecolor{currentstroke}%
\pgfsetdash{}{0pt}%
\pgfpathmoveto{\pgfqpoint{2.407534in}{2.167548in}}%
\pgfpathcurveto{\pgfqpoint{2.418584in}{2.167548in}}{\pgfqpoint{2.429183in}{2.171938in}}{\pgfqpoint{2.436997in}{2.179752in}}%
\pgfpathcurveto{\pgfqpoint{2.444811in}{2.187565in}}{\pgfqpoint{2.449201in}{2.198164in}}{\pgfqpoint{2.449201in}{2.209215in}}%
\pgfpathcurveto{\pgfqpoint{2.449201in}{2.220265in}}{\pgfqpoint{2.444811in}{2.230864in}}{\pgfqpoint{2.436997in}{2.238677in}}%
\pgfpathcurveto{\pgfqpoint{2.429183in}{2.246491in}}{\pgfqpoint{2.418584in}{2.250881in}}{\pgfqpoint{2.407534in}{2.250881in}}%
\pgfpathcurveto{\pgfqpoint{2.396484in}{2.250881in}}{\pgfqpoint{2.385885in}{2.246491in}}{\pgfqpoint{2.378071in}{2.238677in}}%
\pgfpathcurveto{\pgfqpoint{2.370258in}{2.230864in}}{\pgfqpoint{2.365867in}{2.220265in}}{\pgfqpoint{2.365867in}{2.209215in}}%
\pgfpathcurveto{\pgfqpoint{2.365867in}{2.198164in}}{\pgfqpoint{2.370258in}{2.187565in}}{\pgfqpoint{2.378071in}{2.179752in}}%
\pgfpathcurveto{\pgfqpoint{2.385885in}{2.171938in}}{\pgfqpoint{2.396484in}{2.167548in}}{\pgfqpoint{2.407534in}{2.167548in}}%
\pgfpathclose%
\pgfusepath{stroke,fill}%
\end{pgfscope}%
\begin{pgfscope}%
\pgfpathrectangle{\pgfqpoint{0.600000in}{0.600000in}}{\pgfqpoint{3.900000in}{3.900000in}}%
\pgfusepath{clip}%
\pgfsetbuttcap%
\pgfsetroundjoin%
\definecolor{currentfill}{rgb}{0.121569,0.466667,0.705882}%
\pgfsetfillcolor{currentfill}%
\pgfsetlinewidth{1.003750pt}%
\definecolor{currentstroke}{rgb}{0.121569,0.466667,0.705882}%
\pgfsetstrokecolor{currentstroke}%
\pgfsetdash{}{0pt}%
\pgfpathmoveto{\pgfqpoint{2.103676in}{3.392226in}}%
\pgfpathcurveto{\pgfqpoint{2.114726in}{3.392226in}}{\pgfqpoint{2.125325in}{3.396616in}}{\pgfqpoint{2.133138in}{3.404429in}}%
\pgfpathcurveto{\pgfqpoint{2.140952in}{3.412243in}}{\pgfqpoint{2.145342in}{3.422842in}}{\pgfqpoint{2.145342in}{3.433892in}}%
\pgfpathcurveto{\pgfqpoint{2.145342in}{3.444942in}}{\pgfqpoint{2.140952in}{3.455541in}}{\pgfqpoint{2.133138in}{3.463355in}}%
\pgfpathcurveto{\pgfqpoint{2.125325in}{3.471169in}}{\pgfqpoint{2.114726in}{3.475559in}}{\pgfqpoint{2.103676in}{3.475559in}}%
\pgfpathcurveto{\pgfqpoint{2.092625in}{3.475559in}}{\pgfqpoint{2.082026in}{3.471169in}}{\pgfqpoint{2.074213in}{3.463355in}}%
\pgfpathcurveto{\pgfqpoint{2.066399in}{3.455541in}}{\pgfqpoint{2.062009in}{3.444942in}}{\pgfqpoint{2.062009in}{3.433892in}}%
\pgfpathcurveto{\pgfqpoint{2.062009in}{3.422842in}}{\pgfqpoint{2.066399in}{3.412243in}}{\pgfqpoint{2.074213in}{3.404429in}}%
\pgfpathcurveto{\pgfqpoint{2.082026in}{3.396616in}}{\pgfqpoint{2.092625in}{3.392226in}}{\pgfqpoint{2.103676in}{3.392226in}}%
\pgfpathclose%
\pgfusepath{stroke,fill}%
\end{pgfscope}%
\begin{pgfscope}%
\pgfpathrectangle{\pgfqpoint{0.600000in}{0.600000in}}{\pgfqpoint{3.900000in}{3.900000in}}%
\pgfusepath{clip}%
\pgfsetbuttcap%
\pgfsetroundjoin%
\definecolor{currentfill}{rgb}{0.121569,0.466667,0.705882}%
\pgfsetfillcolor{currentfill}%
\pgfsetlinewidth{1.003750pt}%
\definecolor{currentstroke}{rgb}{0.121569,0.466667,0.705882}%
\pgfsetstrokecolor{currentstroke}%
\pgfsetdash{}{0pt}%
\pgfpathmoveto{\pgfqpoint{2.578257in}{2.122417in}}%
\pgfpathcurveto{\pgfqpoint{2.589308in}{2.122417in}}{\pgfqpoint{2.599907in}{2.126807in}}{\pgfqpoint{2.607720in}{2.134621in}}%
\pgfpathcurveto{\pgfqpoint{2.615534in}{2.142434in}}{\pgfqpoint{2.619924in}{2.153033in}}{\pgfqpoint{2.619924in}{2.164083in}}%
\pgfpathcurveto{\pgfqpoint{2.619924in}{2.175133in}}{\pgfqpoint{2.615534in}{2.185733in}}{\pgfqpoint{2.607720in}{2.193546in}}%
\pgfpathcurveto{\pgfqpoint{2.599907in}{2.201360in}}{\pgfqpoint{2.589308in}{2.205750in}}{\pgfqpoint{2.578257in}{2.205750in}}%
\pgfpathcurveto{\pgfqpoint{2.567207in}{2.205750in}}{\pgfqpoint{2.556608in}{2.201360in}}{\pgfqpoint{2.548795in}{2.193546in}}%
\pgfpathcurveto{\pgfqpoint{2.540981in}{2.185733in}}{\pgfqpoint{2.536591in}{2.175133in}}{\pgfqpoint{2.536591in}{2.164083in}}%
\pgfpathcurveto{\pgfqpoint{2.536591in}{2.153033in}}{\pgfqpoint{2.540981in}{2.142434in}}{\pgfqpoint{2.548795in}{2.134621in}}%
\pgfpathcurveto{\pgfqpoint{2.556608in}{2.126807in}}{\pgfqpoint{2.567207in}{2.122417in}}{\pgfqpoint{2.578257in}{2.122417in}}%
\pgfpathclose%
\pgfusepath{stroke,fill}%
\end{pgfscope}%
\begin{pgfscope}%
\pgfpathrectangle{\pgfqpoint{0.600000in}{0.600000in}}{\pgfqpoint{3.900000in}{3.900000in}}%
\pgfusepath{clip}%
\pgfsetbuttcap%
\pgfsetroundjoin%
\definecolor{currentfill}{rgb}{0.121569,0.466667,0.705882}%
\pgfsetfillcolor{currentfill}%
\pgfsetlinewidth{1.003750pt}%
\definecolor{currentstroke}{rgb}{0.121569,0.466667,0.705882}%
\pgfsetstrokecolor{currentstroke}%
\pgfsetdash{}{0pt}%
\pgfpathmoveto{\pgfqpoint{2.390092in}{2.841992in}}%
\pgfpathcurveto{\pgfqpoint{2.401142in}{2.841992in}}{\pgfqpoint{2.411741in}{2.846383in}}{\pgfqpoint{2.419555in}{2.854196in}}%
\pgfpathcurveto{\pgfqpoint{2.427368in}{2.862010in}}{\pgfqpoint{2.431759in}{2.872609in}}{\pgfqpoint{2.431759in}{2.883659in}}%
\pgfpathcurveto{\pgfqpoint{2.431759in}{2.894709in}}{\pgfqpoint{2.427368in}{2.905308in}}{\pgfqpoint{2.419555in}{2.913122in}}%
\pgfpathcurveto{\pgfqpoint{2.411741in}{2.920935in}}{\pgfqpoint{2.401142in}{2.925326in}}{\pgfqpoint{2.390092in}{2.925326in}}%
\pgfpathcurveto{\pgfqpoint{2.379042in}{2.925326in}}{\pgfqpoint{2.368443in}{2.920935in}}{\pgfqpoint{2.360629in}{2.913122in}}%
\pgfpathcurveto{\pgfqpoint{2.352816in}{2.905308in}}{\pgfqpoint{2.348425in}{2.894709in}}{\pgfqpoint{2.348425in}{2.883659in}}%
\pgfpathcurveto{\pgfqpoint{2.348425in}{2.872609in}}{\pgfqpoint{2.352816in}{2.862010in}}{\pgfqpoint{2.360629in}{2.854196in}}%
\pgfpathcurveto{\pgfqpoint{2.368443in}{2.846383in}}{\pgfqpoint{2.379042in}{2.841992in}}{\pgfqpoint{2.390092in}{2.841992in}}%
\pgfpathclose%
\pgfusepath{stroke,fill}%
\end{pgfscope}%
\begin{pgfscope}%
\pgfpathrectangle{\pgfqpoint{0.600000in}{0.600000in}}{\pgfqpoint{3.900000in}{3.900000in}}%
\pgfusepath{clip}%
\pgfsetbuttcap%
\pgfsetroundjoin%
\definecolor{currentfill}{rgb}{0.121569,0.466667,0.705882}%
\pgfsetfillcolor{currentfill}%
\pgfsetlinewidth{1.003750pt}%
\definecolor{currentstroke}{rgb}{0.121569,0.466667,0.705882}%
\pgfsetstrokecolor{currentstroke}%
\pgfsetdash{}{0pt}%
\pgfpathmoveto{\pgfqpoint{2.427169in}{2.391461in}}%
\pgfpathcurveto{\pgfqpoint{2.438219in}{2.391461in}}{\pgfqpoint{2.448818in}{2.395851in}}{\pgfqpoint{2.456632in}{2.403665in}}%
\pgfpathcurveto{\pgfqpoint{2.464445in}{2.411478in}}{\pgfqpoint{2.468836in}{2.422077in}}{\pgfqpoint{2.468836in}{2.433128in}}%
\pgfpathcurveto{\pgfqpoint{2.468836in}{2.444178in}}{\pgfqpoint{2.464445in}{2.454777in}}{\pgfqpoint{2.456632in}{2.462590in}}%
\pgfpathcurveto{\pgfqpoint{2.448818in}{2.470404in}}{\pgfqpoint{2.438219in}{2.474794in}}{\pgfqpoint{2.427169in}{2.474794in}}%
\pgfpathcurveto{\pgfqpoint{2.416119in}{2.474794in}}{\pgfqpoint{2.405520in}{2.470404in}}{\pgfqpoint{2.397706in}{2.462590in}}%
\pgfpathcurveto{\pgfqpoint{2.389892in}{2.454777in}}{\pgfqpoint{2.385502in}{2.444178in}}{\pgfqpoint{2.385502in}{2.433128in}}%
\pgfpathcurveto{\pgfqpoint{2.385502in}{2.422077in}}{\pgfqpoint{2.389892in}{2.411478in}}{\pgfqpoint{2.397706in}{2.403665in}}%
\pgfpathcurveto{\pgfqpoint{2.405520in}{2.395851in}}{\pgfqpoint{2.416119in}{2.391461in}}{\pgfqpoint{2.427169in}{2.391461in}}%
\pgfpathclose%
\pgfusepath{stroke,fill}%
\end{pgfscope}%
\begin{pgfscope}%
\pgfpathrectangle{\pgfqpoint{0.600000in}{0.600000in}}{\pgfqpoint{3.900000in}{3.900000in}}%
\pgfusepath{clip}%
\pgfsetbuttcap%
\pgfsetroundjoin%
\definecolor{currentfill}{rgb}{0.121569,0.466667,0.705882}%
\pgfsetfillcolor{currentfill}%
\pgfsetlinewidth{1.003750pt}%
\definecolor{currentstroke}{rgb}{0.121569,0.466667,0.705882}%
\pgfsetstrokecolor{currentstroke}%
\pgfsetdash{}{0pt}%
\pgfpathmoveto{\pgfqpoint{2.633393in}{2.518673in}}%
\pgfpathcurveto{\pgfqpoint{2.644443in}{2.518673in}}{\pgfqpoint{2.655042in}{2.523063in}}{\pgfqpoint{2.662856in}{2.530877in}}%
\pgfpathcurveto{\pgfqpoint{2.670669in}{2.538690in}}{\pgfqpoint{2.675060in}{2.549289in}}{\pgfqpoint{2.675060in}{2.560339in}}%
\pgfpathcurveto{\pgfqpoint{2.675060in}{2.571390in}}{\pgfqpoint{2.670669in}{2.581989in}}{\pgfqpoint{2.662856in}{2.589802in}}%
\pgfpathcurveto{\pgfqpoint{2.655042in}{2.597616in}}{\pgfqpoint{2.644443in}{2.602006in}}{\pgfqpoint{2.633393in}{2.602006in}}%
\pgfpathcurveto{\pgfqpoint{2.622343in}{2.602006in}}{\pgfqpoint{2.611744in}{2.597616in}}{\pgfqpoint{2.603930in}{2.589802in}}%
\pgfpathcurveto{\pgfqpoint{2.596117in}{2.581989in}}{\pgfqpoint{2.591726in}{2.571390in}}{\pgfqpoint{2.591726in}{2.560339in}}%
\pgfpathcurveto{\pgfqpoint{2.591726in}{2.549289in}}{\pgfqpoint{2.596117in}{2.538690in}}{\pgfqpoint{2.603930in}{2.530877in}}%
\pgfpathcurveto{\pgfqpoint{2.611744in}{2.523063in}}{\pgfqpoint{2.622343in}{2.518673in}}{\pgfqpoint{2.633393in}{2.518673in}}%
\pgfpathclose%
\pgfusepath{stroke,fill}%
\end{pgfscope}%
\begin{pgfscope}%
\pgfpathrectangle{\pgfqpoint{0.600000in}{0.600000in}}{\pgfqpoint{3.900000in}{3.900000in}}%
\pgfusepath{clip}%
\pgfsetbuttcap%
\pgfsetroundjoin%
\definecolor{currentfill}{rgb}{0.121569,0.466667,0.705882}%
\pgfsetfillcolor{currentfill}%
\pgfsetlinewidth{1.003750pt}%
\definecolor{currentstroke}{rgb}{0.121569,0.466667,0.705882}%
\pgfsetstrokecolor{currentstroke}%
\pgfsetdash{}{0pt}%
\pgfpathmoveto{\pgfqpoint{2.934915in}{1.640069in}}%
\pgfpathcurveto{\pgfqpoint{2.945965in}{1.640069in}}{\pgfqpoint{2.956564in}{1.644460in}}{\pgfqpoint{2.964377in}{1.652273in}}%
\pgfpathcurveto{\pgfqpoint{2.972191in}{1.660087in}}{\pgfqpoint{2.976581in}{1.670686in}}{\pgfqpoint{2.976581in}{1.681736in}}%
\pgfpathcurveto{\pgfqpoint{2.976581in}{1.692786in}}{\pgfqpoint{2.972191in}{1.703385in}}{\pgfqpoint{2.964377in}{1.711199in}}%
\pgfpathcurveto{\pgfqpoint{2.956564in}{1.719013in}}{\pgfqpoint{2.945965in}{1.723403in}}{\pgfqpoint{2.934915in}{1.723403in}}%
\pgfpathcurveto{\pgfqpoint{2.923864in}{1.723403in}}{\pgfqpoint{2.913265in}{1.719013in}}{\pgfqpoint{2.905452in}{1.711199in}}%
\pgfpathcurveto{\pgfqpoint{2.897638in}{1.703385in}}{\pgfqpoint{2.893248in}{1.692786in}}{\pgfqpoint{2.893248in}{1.681736in}}%
\pgfpathcurveto{\pgfqpoint{2.893248in}{1.670686in}}{\pgfqpoint{2.897638in}{1.660087in}}{\pgfqpoint{2.905452in}{1.652273in}}%
\pgfpathcurveto{\pgfqpoint{2.913265in}{1.644460in}}{\pgfqpoint{2.923864in}{1.640069in}}{\pgfqpoint{2.934915in}{1.640069in}}%
\pgfpathclose%
\pgfusepath{stroke,fill}%
\end{pgfscope}%
\begin{pgfscope}%
\pgfpathrectangle{\pgfqpoint{0.600000in}{0.600000in}}{\pgfqpoint{3.900000in}{3.900000in}}%
\pgfusepath{clip}%
\pgfsetbuttcap%
\pgfsetroundjoin%
\definecolor{currentfill}{rgb}{0.121569,0.466667,0.705882}%
\pgfsetfillcolor{currentfill}%
\pgfsetlinewidth{1.003750pt}%
\definecolor{currentstroke}{rgb}{0.121569,0.466667,0.705882}%
\pgfsetstrokecolor{currentstroke}%
\pgfsetdash{}{0pt}%
\pgfpathmoveto{\pgfqpoint{2.936112in}{1.331790in}}%
\pgfpathcurveto{\pgfqpoint{2.947163in}{1.331790in}}{\pgfqpoint{2.957762in}{1.336180in}}{\pgfqpoint{2.965575in}{1.343994in}}%
\pgfpathcurveto{\pgfqpoint{2.973389in}{1.351807in}}{\pgfqpoint{2.977779in}{1.362406in}}{\pgfqpoint{2.977779in}{1.373456in}}%
\pgfpathcurveto{\pgfqpoint{2.977779in}{1.384507in}}{\pgfqpoint{2.973389in}{1.395106in}}{\pgfqpoint{2.965575in}{1.402919in}}%
\pgfpathcurveto{\pgfqpoint{2.957762in}{1.410733in}}{\pgfqpoint{2.947163in}{1.415123in}}{\pgfqpoint{2.936112in}{1.415123in}}%
\pgfpathcurveto{\pgfqpoint{2.925062in}{1.415123in}}{\pgfqpoint{2.914463in}{1.410733in}}{\pgfqpoint{2.906650in}{1.402919in}}%
\pgfpathcurveto{\pgfqpoint{2.898836in}{1.395106in}}{\pgfqpoint{2.894446in}{1.384507in}}{\pgfqpoint{2.894446in}{1.373456in}}%
\pgfpathcurveto{\pgfqpoint{2.894446in}{1.362406in}}{\pgfqpoint{2.898836in}{1.351807in}}{\pgfqpoint{2.906650in}{1.343994in}}%
\pgfpathcurveto{\pgfqpoint{2.914463in}{1.336180in}}{\pgfqpoint{2.925062in}{1.331790in}}{\pgfqpoint{2.936112in}{1.331790in}}%
\pgfpathclose%
\pgfusepath{stroke,fill}%
\end{pgfscope}%
\begin{pgfscope}%
\pgfpathrectangle{\pgfqpoint{0.600000in}{0.600000in}}{\pgfqpoint{3.900000in}{3.900000in}}%
\pgfusepath{clip}%
\pgfsetbuttcap%
\pgfsetroundjoin%
\definecolor{currentfill}{rgb}{0.121569,0.466667,0.705882}%
\pgfsetfillcolor{currentfill}%
\pgfsetlinewidth{1.003750pt}%
\definecolor{currentstroke}{rgb}{0.121569,0.466667,0.705882}%
\pgfsetstrokecolor{currentstroke}%
\pgfsetdash{}{0pt}%
\pgfpathmoveto{\pgfqpoint{1.867725in}{2.249974in}}%
\pgfpathcurveto{\pgfqpoint{1.878775in}{2.249974in}}{\pgfqpoint{1.889374in}{2.254364in}}{\pgfqpoint{1.897188in}{2.262178in}}%
\pgfpathcurveto{\pgfqpoint{1.905001in}{2.269991in}}{\pgfqpoint{1.909391in}{2.280591in}}{\pgfqpoint{1.909391in}{2.291641in}}%
\pgfpathcurveto{\pgfqpoint{1.909391in}{2.302691in}}{\pgfqpoint{1.905001in}{2.313290in}}{\pgfqpoint{1.897188in}{2.321103in}}%
\pgfpathcurveto{\pgfqpoint{1.889374in}{2.328917in}}{\pgfqpoint{1.878775in}{2.333307in}}{\pgfqpoint{1.867725in}{2.333307in}}%
\pgfpathcurveto{\pgfqpoint{1.856675in}{2.333307in}}{\pgfqpoint{1.846076in}{2.328917in}}{\pgfqpoint{1.838262in}{2.321103in}}%
\pgfpathcurveto{\pgfqpoint{1.830448in}{2.313290in}}{\pgfqpoint{1.826058in}{2.302691in}}{\pgfqpoint{1.826058in}{2.291641in}}%
\pgfpathcurveto{\pgfqpoint{1.826058in}{2.280591in}}{\pgfqpoint{1.830448in}{2.269991in}}{\pgfqpoint{1.838262in}{2.262178in}}%
\pgfpathcurveto{\pgfqpoint{1.846076in}{2.254364in}}{\pgfqpoint{1.856675in}{2.249974in}}{\pgfqpoint{1.867725in}{2.249974in}}%
\pgfpathclose%
\pgfusepath{stroke,fill}%
\end{pgfscope}%
\begin{pgfscope}%
\pgfpathrectangle{\pgfqpoint{0.600000in}{0.600000in}}{\pgfqpoint{3.900000in}{3.900000in}}%
\pgfusepath{clip}%
\pgfsetbuttcap%
\pgfsetroundjoin%
\definecolor{currentfill}{rgb}{0.121569,0.466667,0.705882}%
\pgfsetfillcolor{currentfill}%
\pgfsetlinewidth{1.003750pt}%
\definecolor{currentstroke}{rgb}{0.121569,0.466667,0.705882}%
\pgfsetstrokecolor{currentstroke}%
\pgfsetdash{}{0pt}%
\pgfpathmoveto{\pgfqpoint{2.738339in}{3.436833in}}%
\pgfpathcurveto{\pgfqpoint{2.749389in}{3.436833in}}{\pgfqpoint{2.759988in}{3.441223in}}{\pgfqpoint{2.767802in}{3.449036in}}%
\pgfpathcurveto{\pgfqpoint{2.775615in}{3.456850in}}{\pgfqpoint{2.780005in}{3.467449in}}{\pgfqpoint{2.780005in}{3.478499in}}%
\pgfpathcurveto{\pgfqpoint{2.780005in}{3.489549in}}{\pgfqpoint{2.775615in}{3.500148in}}{\pgfqpoint{2.767802in}{3.507962in}}%
\pgfpathcurveto{\pgfqpoint{2.759988in}{3.515776in}}{\pgfqpoint{2.749389in}{3.520166in}}{\pgfqpoint{2.738339in}{3.520166in}}%
\pgfpathcurveto{\pgfqpoint{2.727289in}{3.520166in}}{\pgfqpoint{2.716690in}{3.515776in}}{\pgfqpoint{2.708876in}{3.507962in}}%
\pgfpathcurveto{\pgfqpoint{2.701062in}{3.500148in}}{\pgfqpoint{2.696672in}{3.489549in}}{\pgfqpoint{2.696672in}{3.478499in}}%
\pgfpathcurveto{\pgfqpoint{2.696672in}{3.467449in}}{\pgfqpoint{2.701062in}{3.456850in}}{\pgfqpoint{2.708876in}{3.449036in}}%
\pgfpathcurveto{\pgfqpoint{2.716690in}{3.441223in}}{\pgfqpoint{2.727289in}{3.436833in}}{\pgfqpoint{2.738339in}{3.436833in}}%
\pgfpathclose%
\pgfusepath{stroke,fill}%
\end{pgfscope}%
\begin{pgfscope}%
\pgfpathrectangle{\pgfqpoint{0.600000in}{0.600000in}}{\pgfqpoint{3.900000in}{3.900000in}}%
\pgfusepath{clip}%
\pgfsetbuttcap%
\pgfsetroundjoin%
\definecolor{currentfill}{rgb}{0.121569,0.466667,0.705882}%
\pgfsetfillcolor{currentfill}%
\pgfsetlinewidth{1.003750pt}%
\definecolor{currentstroke}{rgb}{0.121569,0.466667,0.705882}%
\pgfsetstrokecolor{currentstroke}%
\pgfsetdash{}{0pt}%
\pgfpathmoveto{\pgfqpoint{2.287639in}{3.023506in}}%
\pgfpathcurveto{\pgfqpoint{2.298689in}{3.023506in}}{\pgfqpoint{2.309288in}{3.027896in}}{\pgfqpoint{2.317101in}{3.035710in}}%
\pgfpathcurveto{\pgfqpoint{2.324915in}{3.043523in}}{\pgfqpoint{2.329305in}{3.054122in}}{\pgfqpoint{2.329305in}{3.065172in}}%
\pgfpathcurveto{\pgfqpoint{2.329305in}{3.076223in}}{\pgfqpoint{2.324915in}{3.086822in}}{\pgfqpoint{2.317101in}{3.094635in}}%
\pgfpathcurveto{\pgfqpoint{2.309288in}{3.102449in}}{\pgfqpoint{2.298689in}{3.106839in}}{\pgfqpoint{2.287639in}{3.106839in}}%
\pgfpathcurveto{\pgfqpoint{2.276588in}{3.106839in}}{\pgfqpoint{2.265989in}{3.102449in}}{\pgfqpoint{2.258176in}{3.094635in}}%
\pgfpathcurveto{\pgfqpoint{2.250362in}{3.086822in}}{\pgfqpoint{2.245972in}{3.076223in}}{\pgfqpoint{2.245972in}{3.065172in}}%
\pgfpathcurveto{\pgfqpoint{2.245972in}{3.054122in}}{\pgfqpoint{2.250362in}{3.043523in}}{\pgfqpoint{2.258176in}{3.035710in}}%
\pgfpathcurveto{\pgfqpoint{2.265989in}{3.027896in}}{\pgfqpoint{2.276588in}{3.023506in}}{\pgfqpoint{2.287639in}{3.023506in}}%
\pgfpathclose%
\pgfusepath{stroke,fill}%
\end{pgfscope}%
\begin{pgfscope}%
\pgfpathrectangle{\pgfqpoint{0.600000in}{0.600000in}}{\pgfqpoint{3.900000in}{3.900000in}}%
\pgfusepath{clip}%
\pgfsetbuttcap%
\pgfsetroundjoin%
\definecolor{currentfill}{rgb}{0.121569,0.466667,0.705882}%
\pgfsetfillcolor{currentfill}%
\pgfsetlinewidth{1.003750pt}%
\definecolor{currentstroke}{rgb}{0.121569,0.466667,0.705882}%
\pgfsetstrokecolor{currentstroke}%
\pgfsetdash{}{0pt}%
\pgfpathmoveto{\pgfqpoint{2.662503in}{3.286119in}}%
\pgfpathcurveto{\pgfqpoint{2.673553in}{3.286119in}}{\pgfqpoint{2.684152in}{3.290509in}}{\pgfqpoint{2.691966in}{3.298323in}}%
\pgfpathcurveto{\pgfqpoint{2.699780in}{3.306136in}}{\pgfqpoint{2.704170in}{3.316735in}}{\pgfqpoint{2.704170in}{3.327785in}}%
\pgfpathcurveto{\pgfqpoint{2.704170in}{3.338836in}}{\pgfqpoint{2.699780in}{3.349435in}}{\pgfqpoint{2.691966in}{3.357248in}}%
\pgfpathcurveto{\pgfqpoint{2.684152in}{3.365062in}}{\pgfqpoint{2.673553in}{3.369452in}}{\pgfqpoint{2.662503in}{3.369452in}}%
\pgfpathcurveto{\pgfqpoint{2.651453in}{3.369452in}}{\pgfqpoint{2.640854in}{3.365062in}}{\pgfqpoint{2.633041in}{3.357248in}}%
\pgfpathcurveto{\pgfqpoint{2.625227in}{3.349435in}}{\pgfqpoint{2.620837in}{3.338836in}}{\pgfqpoint{2.620837in}{3.327785in}}%
\pgfpathcurveto{\pgfqpoint{2.620837in}{3.316735in}}{\pgfqpoint{2.625227in}{3.306136in}}{\pgfqpoint{2.633041in}{3.298323in}}%
\pgfpathcurveto{\pgfqpoint{2.640854in}{3.290509in}}{\pgfqpoint{2.651453in}{3.286119in}}{\pgfqpoint{2.662503in}{3.286119in}}%
\pgfpathclose%
\pgfusepath{stroke,fill}%
\end{pgfscope}%
\begin{pgfscope}%
\pgfpathrectangle{\pgfqpoint{0.600000in}{0.600000in}}{\pgfqpoint{3.900000in}{3.900000in}}%
\pgfusepath{clip}%
\pgfsetbuttcap%
\pgfsetroundjoin%
\definecolor{currentfill}{rgb}{0.121569,0.466667,0.705882}%
\pgfsetfillcolor{currentfill}%
\pgfsetlinewidth{1.003750pt}%
\definecolor{currentstroke}{rgb}{0.121569,0.466667,0.705882}%
\pgfsetstrokecolor{currentstroke}%
\pgfsetdash{}{0pt}%
\pgfpathmoveto{\pgfqpoint{3.313454in}{2.386059in}}%
\pgfpathcurveto{\pgfqpoint{3.324504in}{2.386059in}}{\pgfqpoint{3.335103in}{2.390449in}}{\pgfqpoint{3.342916in}{2.398263in}}%
\pgfpathcurveto{\pgfqpoint{3.350730in}{2.406077in}}{\pgfqpoint{3.355120in}{2.416676in}}{\pgfqpoint{3.355120in}{2.427726in}}%
\pgfpathcurveto{\pgfqpoint{3.355120in}{2.438776in}}{\pgfqpoint{3.350730in}{2.449375in}}{\pgfqpoint{3.342916in}{2.457189in}}%
\pgfpathcurveto{\pgfqpoint{3.335103in}{2.465002in}}{\pgfqpoint{3.324504in}{2.469393in}}{\pgfqpoint{3.313454in}{2.469393in}}%
\pgfpathcurveto{\pgfqpoint{3.302403in}{2.469393in}}{\pgfqpoint{3.291804in}{2.465002in}}{\pgfqpoint{3.283991in}{2.457189in}}%
\pgfpathcurveto{\pgfqpoint{3.276177in}{2.449375in}}{\pgfqpoint{3.271787in}{2.438776in}}{\pgfqpoint{3.271787in}{2.427726in}}%
\pgfpathcurveto{\pgfqpoint{3.271787in}{2.416676in}}{\pgfqpoint{3.276177in}{2.406077in}}{\pgfqpoint{3.283991in}{2.398263in}}%
\pgfpathcurveto{\pgfqpoint{3.291804in}{2.390449in}}{\pgfqpoint{3.302403in}{2.386059in}}{\pgfqpoint{3.313454in}{2.386059in}}%
\pgfpathclose%
\pgfusepath{stroke,fill}%
\end{pgfscope}%
\begin{pgfscope}%
\pgfpathrectangle{\pgfqpoint{0.600000in}{0.600000in}}{\pgfqpoint{3.900000in}{3.900000in}}%
\pgfusepath{clip}%
\pgfsetbuttcap%
\pgfsetroundjoin%
\definecolor{currentfill}{rgb}{0.121569,0.466667,0.705882}%
\pgfsetfillcolor{currentfill}%
\pgfsetlinewidth{1.003750pt}%
\definecolor{currentstroke}{rgb}{0.121569,0.466667,0.705882}%
\pgfsetstrokecolor{currentstroke}%
\pgfsetdash{}{0pt}%
\pgfpathmoveto{\pgfqpoint{2.406919in}{2.743590in}}%
\pgfpathcurveto{\pgfqpoint{2.417970in}{2.743590in}}{\pgfqpoint{2.428569in}{2.747980in}}{\pgfqpoint{2.436382in}{2.755794in}}%
\pgfpathcurveto{\pgfqpoint{2.444196in}{2.763607in}}{\pgfqpoint{2.448586in}{2.774207in}}{\pgfqpoint{2.448586in}{2.785257in}}%
\pgfpathcurveto{\pgfqpoint{2.448586in}{2.796307in}}{\pgfqpoint{2.444196in}{2.806906in}}{\pgfqpoint{2.436382in}{2.814719in}}%
\pgfpathcurveto{\pgfqpoint{2.428569in}{2.822533in}}{\pgfqpoint{2.417970in}{2.826923in}}{\pgfqpoint{2.406919in}{2.826923in}}%
\pgfpathcurveto{\pgfqpoint{2.395869in}{2.826923in}}{\pgfqpoint{2.385270in}{2.822533in}}{\pgfqpoint{2.377457in}{2.814719in}}%
\pgfpathcurveto{\pgfqpoint{2.369643in}{2.806906in}}{\pgfqpoint{2.365253in}{2.796307in}}{\pgfqpoint{2.365253in}{2.785257in}}%
\pgfpathcurveto{\pgfqpoint{2.365253in}{2.774207in}}{\pgfqpoint{2.369643in}{2.763607in}}{\pgfqpoint{2.377457in}{2.755794in}}%
\pgfpathcurveto{\pgfqpoint{2.385270in}{2.747980in}}{\pgfqpoint{2.395869in}{2.743590in}}{\pgfqpoint{2.406919in}{2.743590in}}%
\pgfpathclose%
\pgfusepath{stroke,fill}%
\end{pgfscope}%
\begin{pgfscope}%
\pgfpathrectangle{\pgfqpoint{0.600000in}{0.600000in}}{\pgfqpoint{3.900000in}{3.900000in}}%
\pgfusepath{clip}%
\pgfsetbuttcap%
\pgfsetroundjoin%
\definecolor{currentfill}{rgb}{0.121569,0.466667,0.705882}%
\pgfsetfillcolor{currentfill}%
\pgfsetlinewidth{1.003750pt}%
\definecolor{currentstroke}{rgb}{0.121569,0.466667,0.705882}%
\pgfsetstrokecolor{currentstroke}%
\pgfsetdash{}{0pt}%
\pgfpathmoveto{\pgfqpoint{2.539891in}{2.468647in}}%
\pgfpathcurveto{\pgfqpoint{2.550941in}{2.468647in}}{\pgfqpoint{2.561540in}{2.473037in}}{\pgfqpoint{2.569354in}{2.480851in}}%
\pgfpathcurveto{\pgfqpoint{2.577167in}{2.488664in}}{\pgfqpoint{2.581558in}{2.499263in}}{\pgfqpoint{2.581558in}{2.510313in}}%
\pgfpathcurveto{\pgfqpoint{2.581558in}{2.521363in}}{\pgfqpoint{2.577167in}{2.531962in}}{\pgfqpoint{2.569354in}{2.539776in}}%
\pgfpathcurveto{\pgfqpoint{2.561540in}{2.547590in}}{\pgfqpoint{2.550941in}{2.551980in}}{\pgfqpoint{2.539891in}{2.551980in}}%
\pgfpathcurveto{\pgfqpoint{2.528841in}{2.551980in}}{\pgfqpoint{2.518242in}{2.547590in}}{\pgfqpoint{2.510428in}{2.539776in}}%
\pgfpathcurveto{\pgfqpoint{2.502615in}{2.531962in}}{\pgfqpoint{2.498224in}{2.521363in}}{\pgfqpoint{2.498224in}{2.510313in}}%
\pgfpathcurveto{\pgfqpoint{2.498224in}{2.499263in}}{\pgfqpoint{2.502615in}{2.488664in}}{\pgfqpoint{2.510428in}{2.480851in}}%
\pgfpathcurveto{\pgfqpoint{2.518242in}{2.473037in}}{\pgfqpoint{2.528841in}{2.468647in}}{\pgfqpoint{2.539891in}{2.468647in}}%
\pgfpathclose%
\pgfusepath{stroke,fill}%
\end{pgfscope}%
\begin{pgfscope}%
\pgfpathrectangle{\pgfqpoint{0.600000in}{0.600000in}}{\pgfqpoint{3.900000in}{3.900000in}}%
\pgfusepath{clip}%
\pgfsetbuttcap%
\pgfsetroundjoin%
\definecolor{currentfill}{rgb}{0.121569,0.466667,0.705882}%
\pgfsetfillcolor{currentfill}%
\pgfsetlinewidth{1.003750pt}%
\definecolor{currentstroke}{rgb}{0.121569,0.466667,0.705882}%
\pgfsetstrokecolor{currentstroke}%
\pgfsetdash{}{0pt}%
\pgfpathmoveto{\pgfqpoint{3.264213in}{3.452470in}}%
\pgfpathcurveto{\pgfqpoint{3.275263in}{3.452470in}}{\pgfqpoint{3.285862in}{3.456860in}}{\pgfqpoint{3.293675in}{3.464674in}}%
\pgfpathcurveto{\pgfqpoint{3.301489in}{3.472488in}}{\pgfqpoint{3.305879in}{3.483087in}}{\pgfqpoint{3.305879in}{3.494137in}}%
\pgfpathcurveto{\pgfqpoint{3.305879in}{3.505187in}}{\pgfqpoint{3.301489in}{3.515786in}}{\pgfqpoint{3.293675in}{3.523599in}}%
\pgfpathcurveto{\pgfqpoint{3.285862in}{3.531413in}}{\pgfqpoint{3.275263in}{3.535803in}}{\pgfqpoint{3.264213in}{3.535803in}}%
\pgfpathcurveto{\pgfqpoint{3.253162in}{3.535803in}}{\pgfqpoint{3.242563in}{3.531413in}}{\pgfqpoint{3.234750in}{3.523599in}}%
\pgfpathcurveto{\pgfqpoint{3.226936in}{3.515786in}}{\pgfqpoint{3.222546in}{3.505187in}}{\pgfqpoint{3.222546in}{3.494137in}}%
\pgfpathcurveto{\pgfqpoint{3.222546in}{3.483087in}}{\pgfqpoint{3.226936in}{3.472488in}}{\pgfqpoint{3.234750in}{3.464674in}}%
\pgfpathcurveto{\pgfqpoint{3.242563in}{3.456860in}}{\pgfqpoint{3.253162in}{3.452470in}}{\pgfqpoint{3.264213in}{3.452470in}}%
\pgfpathclose%
\pgfusepath{stroke,fill}%
\end{pgfscope}%
\begin{pgfscope}%
\pgfpathrectangle{\pgfqpoint{0.600000in}{0.600000in}}{\pgfqpoint{3.900000in}{3.900000in}}%
\pgfusepath{clip}%
\pgfsetbuttcap%
\pgfsetroundjoin%
\definecolor{currentfill}{rgb}{0.121569,0.466667,0.705882}%
\pgfsetfillcolor{currentfill}%
\pgfsetlinewidth{1.003750pt}%
\definecolor{currentstroke}{rgb}{0.121569,0.466667,0.705882}%
\pgfsetstrokecolor{currentstroke}%
\pgfsetdash{}{0pt}%
\pgfpathmoveto{\pgfqpoint{2.181259in}{2.474339in}}%
\pgfpathcurveto{\pgfqpoint{2.192309in}{2.474339in}}{\pgfqpoint{2.202908in}{2.478729in}}{\pgfqpoint{2.210722in}{2.486542in}}%
\pgfpathcurveto{\pgfqpoint{2.218536in}{2.494356in}}{\pgfqpoint{2.222926in}{2.504955in}}{\pgfqpoint{2.222926in}{2.516005in}}%
\pgfpathcurveto{\pgfqpoint{2.222926in}{2.527055in}}{\pgfqpoint{2.218536in}{2.537654in}}{\pgfqpoint{2.210722in}{2.545468in}}%
\pgfpathcurveto{\pgfqpoint{2.202908in}{2.553282in}}{\pgfqpoint{2.192309in}{2.557672in}}{\pgfqpoint{2.181259in}{2.557672in}}%
\pgfpathcurveto{\pgfqpoint{2.170209in}{2.557672in}}{\pgfqpoint{2.159610in}{2.553282in}}{\pgfqpoint{2.151796in}{2.545468in}}%
\pgfpathcurveto{\pgfqpoint{2.143983in}{2.537654in}}{\pgfqpoint{2.139593in}{2.527055in}}{\pgfqpoint{2.139593in}{2.516005in}}%
\pgfpathcurveto{\pgfqpoint{2.139593in}{2.504955in}}{\pgfqpoint{2.143983in}{2.494356in}}{\pgfqpoint{2.151796in}{2.486542in}}%
\pgfpathcurveto{\pgfqpoint{2.159610in}{2.478729in}}{\pgfqpoint{2.170209in}{2.474339in}}{\pgfqpoint{2.181259in}{2.474339in}}%
\pgfpathclose%
\pgfusepath{stroke,fill}%
\end{pgfscope}%
\begin{pgfscope}%
\pgfpathrectangle{\pgfqpoint{0.600000in}{0.600000in}}{\pgfqpoint{3.900000in}{3.900000in}}%
\pgfusepath{clip}%
\pgfsetbuttcap%
\pgfsetroundjoin%
\definecolor{currentfill}{rgb}{0.121569,0.466667,0.705882}%
\pgfsetfillcolor{currentfill}%
\pgfsetlinewidth{1.003750pt}%
\definecolor{currentstroke}{rgb}{0.121569,0.466667,0.705882}%
\pgfsetstrokecolor{currentstroke}%
\pgfsetdash{}{0pt}%
\pgfpathmoveto{\pgfqpoint{3.281477in}{2.330788in}}%
\pgfpathcurveto{\pgfqpoint{3.292527in}{2.330788in}}{\pgfqpoint{3.303126in}{2.335179in}}{\pgfqpoint{3.310940in}{2.342992in}}%
\pgfpathcurveto{\pgfqpoint{3.318753in}{2.350806in}}{\pgfqpoint{3.323144in}{2.361405in}}{\pgfqpoint{3.323144in}{2.372455in}}%
\pgfpathcurveto{\pgfqpoint{3.323144in}{2.383505in}}{\pgfqpoint{3.318753in}{2.394104in}}{\pgfqpoint{3.310940in}{2.401918in}}%
\pgfpathcurveto{\pgfqpoint{3.303126in}{2.409731in}}{\pgfqpoint{3.292527in}{2.414122in}}{\pgfqpoint{3.281477in}{2.414122in}}%
\pgfpathcurveto{\pgfqpoint{3.270427in}{2.414122in}}{\pgfqpoint{3.259828in}{2.409731in}}{\pgfqpoint{3.252014in}{2.401918in}}%
\pgfpathcurveto{\pgfqpoint{3.244200in}{2.394104in}}{\pgfqpoint{3.239810in}{2.383505in}}{\pgfqpoint{3.239810in}{2.372455in}}%
\pgfpathcurveto{\pgfqpoint{3.239810in}{2.361405in}}{\pgfqpoint{3.244200in}{2.350806in}}{\pgfqpoint{3.252014in}{2.342992in}}%
\pgfpathcurveto{\pgfqpoint{3.259828in}{2.335179in}}{\pgfqpoint{3.270427in}{2.330788in}}{\pgfqpoint{3.281477in}{2.330788in}}%
\pgfpathclose%
\pgfusepath{stroke,fill}%
\end{pgfscope}%
\begin{pgfscope}%
\pgfpathrectangle{\pgfqpoint{0.600000in}{0.600000in}}{\pgfqpoint{3.900000in}{3.900000in}}%
\pgfusepath{clip}%
\pgfsetbuttcap%
\pgfsetroundjoin%
\definecolor{currentfill}{rgb}{0.121569,0.466667,0.705882}%
\pgfsetfillcolor{currentfill}%
\pgfsetlinewidth{1.003750pt}%
\definecolor{currentstroke}{rgb}{0.121569,0.466667,0.705882}%
\pgfsetstrokecolor{currentstroke}%
\pgfsetdash{}{0pt}%
\pgfpathmoveto{\pgfqpoint{2.327106in}{2.323448in}}%
\pgfpathcurveto{\pgfqpoint{2.338156in}{2.323448in}}{\pgfqpoint{2.348755in}{2.327838in}}{\pgfqpoint{2.356569in}{2.335651in}}%
\pgfpathcurveto{\pgfqpoint{2.364382in}{2.343465in}}{\pgfqpoint{2.368773in}{2.354064in}}{\pgfqpoint{2.368773in}{2.365114in}}%
\pgfpathcurveto{\pgfqpoint{2.368773in}{2.376164in}}{\pgfqpoint{2.364382in}{2.386763in}}{\pgfqpoint{2.356569in}{2.394577in}}%
\pgfpathcurveto{\pgfqpoint{2.348755in}{2.402391in}}{\pgfqpoint{2.338156in}{2.406781in}}{\pgfqpoint{2.327106in}{2.406781in}}%
\pgfpathcurveto{\pgfqpoint{2.316056in}{2.406781in}}{\pgfqpoint{2.305457in}{2.402391in}}{\pgfqpoint{2.297643in}{2.394577in}}%
\pgfpathcurveto{\pgfqpoint{2.289829in}{2.386763in}}{\pgfqpoint{2.285439in}{2.376164in}}{\pgfqpoint{2.285439in}{2.365114in}}%
\pgfpathcurveto{\pgfqpoint{2.285439in}{2.354064in}}{\pgfqpoint{2.289829in}{2.343465in}}{\pgfqpoint{2.297643in}{2.335651in}}%
\pgfpathcurveto{\pgfqpoint{2.305457in}{2.327838in}}{\pgfqpoint{2.316056in}{2.323448in}}{\pgfqpoint{2.327106in}{2.323448in}}%
\pgfpathclose%
\pgfusepath{stroke,fill}%
\end{pgfscope}%
\begin{pgfscope}%
\pgfpathrectangle{\pgfqpoint{0.600000in}{0.600000in}}{\pgfqpoint{3.900000in}{3.900000in}}%
\pgfusepath{clip}%
\pgfsetbuttcap%
\pgfsetroundjoin%
\definecolor{currentfill}{rgb}{0.121569,0.466667,0.705882}%
\pgfsetfillcolor{currentfill}%
\pgfsetlinewidth{1.003750pt}%
\definecolor{currentstroke}{rgb}{0.121569,0.466667,0.705882}%
\pgfsetstrokecolor{currentstroke}%
\pgfsetdash{}{0pt}%
\pgfpathmoveto{\pgfqpoint{2.731642in}{2.611655in}}%
\pgfpathcurveto{\pgfqpoint{2.742692in}{2.611655in}}{\pgfqpoint{2.753291in}{2.616045in}}{\pgfqpoint{2.761105in}{2.623859in}}%
\pgfpathcurveto{\pgfqpoint{2.768919in}{2.631673in}}{\pgfqpoint{2.773309in}{2.642272in}}{\pgfqpoint{2.773309in}{2.653322in}}%
\pgfpathcurveto{\pgfqpoint{2.773309in}{2.664372in}}{\pgfqpoint{2.768919in}{2.674971in}}{\pgfqpoint{2.761105in}{2.682785in}}%
\pgfpathcurveto{\pgfqpoint{2.753291in}{2.690598in}}{\pgfqpoint{2.742692in}{2.694988in}}{\pgfqpoint{2.731642in}{2.694988in}}%
\pgfpathcurveto{\pgfqpoint{2.720592in}{2.694988in}}{\pgfqpoint{2.709993in}{2.690598in}}{\pgfqpoint{2.702180in}{2.682785in}}%
\pgfpathcurveto{\pgfqpoint{2.694366in}{2.674971in}}{\pgfqpoint{2.689976in}{2.664372in}}{\pgfqpoint{2.689976in}{2.653322in}}%
\pgfpathcurveto{\pgfqpoint{2.689976in}{2.642272in}}{\pgfqpoint{2.694366in}{2.631673in}}{\pgfqpoint{2.702180in}{2.623859in}}%
\pgfpathcurveto{\pgfqpoint{2.709993in}{2.616045in}}{\pgfqpoint{2.720592in}{2.611655in}}{\pgfqpoint{2.731642in}{2.611655in}}%
\pgfpathclose%
\pgfusepath{stroke,fill}%
\end{pgfscope}%
\begin{pgfscope}%
\pgfpathrectangle{\pgfqpoint{0.600000in}{0.600000in}}{\pgfqpoint{3.900000in}{3.900000in}}%
\pgfusepath{clip}%
\pgfsetbuttcap%
\pgfsetroundjoin%
\definecolor{currentfill}{rgb}{0.121569,0.466667,0.705882}%
\pgfsetfillcolor{currentfill}%
\pgfsetlinewidth{1.003750pt}%
\definecolor{currentstroke}{rgb}{0.121569,0.466667,0.705882}%
\pgfsetstrokecolor{currentstroke}%
\pgfsetdash{}{0pt}%
\pgfpathmoveto{\pgfqpoint{2.030544in}{2.072886in}}%
\pgfpathcurveto{\pgfqpoint{2.041594in}{2.072886in}}{\pgfqpoint{2.052193in}{2.077276in}}{\pgfqpoint{2.060007in}{2.085089in}}%
\pgfpathcurveto{\pgfqpoint{2.067820in}{2.092903in}}{\pgfqpoint{2.072211in}{2.103502in}}{\pgfqpoint{2.072211in}{2.114552in}}%
\pgfpathcurveto{\pgfqpoint{2.072211in}{2.125602in}}{\pgfqpoint{2.067820in}{2.136201in}}{\pgfqpoint{2.060007in}{2.144015in}}%
\pgfpathcurveto{\pgfqpoint{2.052193in}{2.151829in}}{\pgfqpoint{2.041594in}{2.156219in}}{\pgfqpoint{2.030544in}{2.156219in}}%
\pgfpathcurveto{\pgfqpoint{2.019494in}{2.156219in}}{\pgfqpoint{2.008895in}{2.151829in}}{\pgfqpoint{2.001081in}{2.144015in}}%
\pgfpathcurveto{\pgfqpoint{1.993268in}{2.136201in}}{\pgfqpoint{1.988877in}{2.125602in}}{\pgfqpoint{1.988877in}{2.114552in}}%
\pgfpathcurveto{\pgfqpoint{1.988877in}{2.103502in}}{\pgfqpoint{1.993268in}{2.092903in}}{\pgfqpoint{2.001081in}{2.085089in}}%
\pgfpathcurveto{\pgfqpoint{2.008895in}{2.077276in}}{\pgfqpoint{2.019494in}{2.072886in}}{\pgfqpoint{2.030544in}{2.072886in}}%
\pgfpathclose%
\pgfusepath{stroke,fill}%
\end{pgfscope}%
\begin{pgfscope}%
\pgfpathrectangle{\pgfqpoint{0.600000in}{0.600000in}}{\pgfqpoint{3.900000in}{3.900000in}}%
\pgfusepath{clip}%
\pgfsetbuttcap%
\pgfsetroundjoin%
\definecolor{currentfill}{rgb}{0.121569,0.466667,0.705882}%
\pgfsetfillcolor{currentfill}%
\pgfsetlinewidth{1.003750pt}%
\definecolor{currentstroke}{rgb}{0.121569,0.466667,0.705882}%
\pgfsetstrokecolor{currentstroke}%
\pgfsetdash{}{0pt}%
\pgfpathmoveto{\pgfqpoint{2.737306in}{2.334276in}}%
\pgfpathcurveto{\pgfqpoint{2.748356in}{2.334276in}}{\pgfqpoint{2.758955in}{2.338666in}}{\pgfqpoint{2.766768in}{2.346480in}}%
\pgfpathcurveto{\pgfqpoint{2.774582in}{2.354293in}}{\pgfqpoint{2.778972in}{2.364892in}}{\pgfqpoint{2.778972in}{2.375943in}}%
\pgfpathcurveto{\pgfqpoint{2.778972in}{2.386993in}}{\pgfqpoint{2.774582in}{2.397592in}}{\pgfqpoint{2.766768in}{2.405405in}}%
\pgfpathcurveto{\pgfqpoint{2.758955in}{2.413219in}}{\pgfqpoint{2.748356in}{2.417609in}}{\pgfqpoint{2.737306in}{2.417609in}}%
\pgfpathcurveto{\pgfqpoint{2.726255in}{2.417609in}}{\pgfqpoint{2.715656in}{2.413219in}}{\pgfqpoint{2.707843in}{2.405405in}}%
\pgfpathcurveto{\pgfqpoint{2.700029in}{2.397592in}}{\pgfqpoint{2.695639in}{2.386993in}}{\pgfqpoint{2.695639in}{2.375943in}}%
\pgfpathcurveto{\pgfqpoint{2.695639in}{2.364892in}}{\pgfqpoint{2.700029in}{2.354293in}}{\pgfqpoint{2.707843in}{2.346480in}}%
\pgfpathcurveto{\pgfqpoint{2.715656in}{2.338666in}}{\pgfqpoint{2.726255in}{2.334276in}}{\pgfqpoint{2.737306in}{2.334276in}}%
\pgfpathclose%
\pgfusepath{stroke,fill}%
\end{pgfscope}%
\begin{pgfscope}%
\pgfpathrectangle{\pgfqpoint{0.600000in}{0.600000in}}{\pgfqpoint{3.900000in}{3.900000in}}%
\pgfusepath{clip}%
\pgfsetbuttcap%
\pgfsetroundjoin%
\definecolor{currentfill}{rgb}{0.121569,0.466667,0.705882}%
\pgfsetfillcolor{currentfill}%
\pgfsetlinewidth{1.003750pt}%
\definecolor{currentstroke}{rgb}{0.121569,0.466667,0.705882}%
\pgfsetstrokecolor{currentstroke}%
\pgfsetdash{}{0pt}%
\pgfpathmoveto{\pgfqpoint{3.155854in}{1.904272in}}%
\pgfpathcurveto{\pgfqpoint{3.166904in}{1.904272in}}{\pgfqpoint{3.177503in}{1.908662in}}{\pgfqpoint{3.185317in}{1.916476in}}%
\pgfpathcurveto{\pgfqpoint{3.193130in}{1.924289in}}{\pgfqpoint{3.197521in}{1.934888in}}{\pgfqpoint{3.197521in}{1.945939in}}%
\pgfpathcurveto{\pgfqpoint{3.197521in}{1.956989in}}{\pgfqpoint{3.193130in}{1.967588in}}{\pgfqpoint{3.185317in}{1.975401in}}%
\pgfpathcurveto{\pgfqpoint{3.177503in}{1.983215in}}{\pgfqpoint{3.166904in}{1.987605in}}{\pgfqpoint{3.155854in}{1.987605in}}%
\pgfpathcurveto{\pgfqpoint{3.144804in}{1.987605in}}{\pgfqpoint{3.134205in}{1.983215in}}{\pgfqpoint{3.126391in}{1.975401in}}%
\pgfpathcurveto{\pgfqpoint{3.118578in}{1.967588in}}{\pgfqpoint{3.114187in}{1.956989in}}{\pgfqpoint{3.114187in}{1.945939in}}%
\pgfpathcurveto{\pgfqpoint{3.114187in}{1.934888in}}{\pgfqpoint{3.118578in}{1.924289in}}{\pgfqpoint{3.126391in}{1.916476in}}%
\pgfpathcurveto{\pgfqpoint{3.134205in}{1.908662in}}{\pgfqpoint{3.144804in}{1.904272in}}{\pgfqpoint{3.155854in}{1.904272in}}%
\pgfpathclose%
\pgfusepath{stroke,fill}%
\end{pgfscope}%
\begin{pgfscope}%
\pgfpathrectangle{\pgfqpoint{0.600000in}{0.600000in}}{\pgfqpoint{3.900000in}{3.900000in}}%
\pgfusepath{clip}%
\pgfsetbuttcap%
\pgfsetroundjoin%
\definecolor{currentfill}{rgb}{0.121569,0.466667,0.705882}%
\pgfsetfillcolor{currentfill}%
\pgfsetlinewidth{1.003750pt}%
\definecolor{currentstroke}{rgb}{0.121569,0.466667,0.705882}%
\pgfsetstrokecolor{currentstroke}%
\pgfsetdash{}{0pt}%
\pgfpathmoveto{\pgfqpoint{2.285388in}{2.741055in}}%
\pgfpathcurveto{\pgfqpoint{2.296438in}{2.741055in}}{\pgfqpoint{2.307037in}{2.745445in}}{\pgfqpoint{2.314850in}{2.753259in}}%
\pgfpathcurveto{\pgfqpoint{2.322664in}{2.761073in}}{\pgfqpoint{2.327054in}{2.771672in}}{\pgfqpoint{2.327054in}{2.782722in}}%
\pgfpathcurveto{\pgfqpoint{2.327054in}{2.793772in}}{\pgfqpoint{2.322664in}{2.804371in}}{\pgfqpoint{2.314850in}{2.812185in}}%
\pgfpathcurveto{\pgfqpoint{2.307037in}{2.819998in}}{\pgfqpoint{2.296438in}{2.824389in}}{\pgfqpoint{2.285388in}{2.824389in}}%
\pgfpathcurveto{\pgfqpoint{2.274338in}{2.824389in}}{\pgfqpoint{2.263739in}{2.819998in}}{\pgfqpoint{2.255925in}{2.812185in}}%
\pgfpathcurveto{\pgfqpoint{2.248111in}{2.804371in}}{\pgfqpoint{2.243721in}{2.793772in}}{\pgfqpoint{2.243721in}{2.782722in}}%
\pgfpathcurveto{\pgfqpoint{2.243721in}{2.771672in}}{\pgfqpoint{2.248111in}{2.761073in}}{\pgfqpoint{2.255925in}{2.753259in}}%
\pgfpathcurveto{\pgfqpoint{2.263739in}{2.745445in}}{\pgfqpoint{2.274338in}{2.741055in}}{\pgfqpoint{2.285388in}{2.741055in}}%
\pgfpathclose%
\pgfusepath{stroke,fill}%
\end{pgfscope}%
\begin{pgfscope}%
\pgfpathrectangle{\pgfqpoint{0.600000in}{0.600000in}}{\pgfqpoint{3.900000in}{3.900000in}}%
\pgfusepath{clip}%
\pgfsetbuttcap%
\pgfsetroundjoin%
\definecolor{currentfill}{rgb}{0.121569,0.466667,0.705882}%
\pgfsetfillcolor{currentfill}%
\pgfsetlinewidth{1.003750pt}%
\definecolor{currentstroke}{rgb}{0.121569,0.466667,0.705882}%
\pgfsetstrokecolor{currentstroke}%
\pgfsetdash{}{0pt}%
\pgfpathmoveto{\pgfqpoint{3.143508in}{3.115451in}}%
\pgfpathcurveto{\pgfqpoint{3.154558in}{3.115451in}}{\pgfqpoint{3.165157in}{3.119842in}}{\pgfqpoint{3.172971in}{3.127655in}}%
\pgfpathcurveto{\pgfqpoint{3.180784in}{3.135469in}}{\pgfqpoint{3.185174in}{3.146068in}}{\pgfqpoint{3.185174in}{3.157118in}}%
\pgfpathcurveto{\pgfqpoint{3.185174in}{3.168168in}}{\pgfqpoint{3.180784in}{3.178767in}}{\pgfqpoint{3.172971in}{3.186581in}}%
\pgfpathcurveto{\pgfqpoint{3.165157in}{3.194395in}}{\pgfqpoint{3.154558in}{3.198785in}}{\pgfqpoint{3.143508in}{3.198785in}}%
\pgfpathcurveto{\pgfqpoint{3.132458in}{3.198785in}}{\pgfqpoint{3.121859in}{3.194395in}}{\pgfqpoint{3.114045in}{3.186581in}}%
\pgfpathcurveto{\pgfqpoint{3.106231in}{3.178767in}}{\pgfqpoint{3.101841in}{3.168168in}}{\pgfqpoint{3.101841in}{3.157118in}}%
\pgfpathcurveto{\pgfqpoint{3.101841in}{3.146068in}}{\pgfqpoint{3.106231in}{3.135469in}}{\pgfqpoint{3.114045in}{3.127655in}}%
\pgfpathcurveto{\pgfqpoint{3.121859in}{3.119842in}}{\pgfqpoint{3.132458in}{3.115451in}}{\pgfqpoint{3.143508in}{3.115451in}}%
\pgfpathclose%
\pgfusepath{stroke,fill}%
\end{pgfscope}%
\begin{pgfscope}%
\pgfpathrectangle{\pgfqpoint{0.600000in}{0.600000in}}{\pgfqpoint{3.900000in}{3.900000in}}%
\pgfusepath{clip}%
\pgfsetbuttcap%
\pgfsetroundjoin%
\definecolor{currentfill}{rgb}{0.121569,0.466667,0.705882}%
\pgfsetfillcolor{currentfill}%
\pgfsetlinewidth{1.003750pt}%
\definecolor{currentstroke}{rgb}{0.121569,0.466667,0.705882}%
\pgfsetstrokecolor{currentstroke}%
\pgfsetdash{}{0pt}%
\pgfpathmoveto{\pgfqpoint{2.164017in}{2.094099in}}%
\pgfpathcurveto{\pgfqpoint{2.175067in}{2.094099in}}{\pgfqpoint{2.185666in}{2.098489in}}{\pgfqpoint{2.193480in}{2.106303in}}%
\pgfpathcurveto{\pgfqpoint{2.201293in}{2.114116in}}{\pgfqpoint{2.205683in}{2.124715in}}{\pgfqpoint{2.205683in}{2.135765in}}%
\pgfpathcurveto{\pgfqpoint{2.205683in}{2.146816in}}{\pgfqpoint{2.201293in}{2.157415in}}{\pgfqpoint{2.193480in}{2.165228in}}%
\pgfpathcurveto{\pgfqpoint{2.185666in}{2.173042in}}{\pgfqpoint{2.175067in}{2.177432in}}{\pgfqpoint{2.164017in}{2.177432in}}%
\pgfpathcurveto{\pgfqpoint{2.152967in}{2.177432in}}{\pgfqpoint{2.142368in}{2.173042in}}{\pgfqpoint{2.134554in}{2.165228in}}%
\pgfpathcurveto{\pgfqpoint{2.126740in}{2.157415in}}{\pgfqpoint{2.122350in}{2.146816in}}{\pgfqpoint{2.122350in}{2.135765in}}%
\pgfpathcurveto{\pgfqpoint{2.122350in}{2.124715in}}{\pgfqpoint{2.126740in}{2.114116in}}{\pgfqpoint{2.134554in}{2.106303in}}%
\pgfpathcurveto{\pgfqpoint{2.142368in}{2.098489in}}{\pgfqpoint{2.152967in}{2.094099in}}{\pgfqpoint{2.164017in}{2.094099in}}%
\pgfpathclose%
\pgfusepath{stroke,fill}%
\end{pgfscope}%
\begin{pgfscope}%
\pgfpathrectangle{\pgfqpoint{0.600000in}{0.600000in}}{\pgfqpoint{3.900000in}{3.900000in}}%
\pgfusepath{clip}%
\pgfsetbuttcap%
\pgfsetroundjoin%
\definecolor{currentfill}{rgb}{0.121569,0.466667,0.705882}%
\pgfsetfillcolor{currentfill}%
\pgfsetlinewidth{1.003750pt}%
\definecolor{currentstroke}{rgb}{0.121569,0.466667,0.705882}%
\pgfsetstrokecolor{currentstroke}%
\pgfsetdash{}{0pt}%
\pgfpathmoveto{\pgfqpoint{3.119266in}{1.916314in}}%
\pgfpathcurveto{\pgfqpoint{3.130316in}{1.916314in}}{\pgfqpoint{3.140915in}{1.920705in}}{\pgfqpoint{3.148728in}{1.928518in}}%
\pgfpathcurveto{\pgfqpoint{3.156542in}{1.936332in}}{\pgfqpoint{3.160932in}{1.946931in}}{\pgfqpoint{3.160932in}{1.957981in}}%
\pgfpathcurveto{\pgfqpoint{3.160932in}{1.969031in}}{\pgfqpoint{3.156542in}{1.979630in}}{\pgfqpoint{3.148728in}{1.987444in}}%
\pgfpathcurveto{\pgfqpoint{3.140915in}{1.995257in}}{\pgfqpoint{3.130316in}{1.999648in}}{\pgfqpoint{3.119266in}{1.999648in}}%
\pgfpathcurveto{\pgfqpoint{3.108215in}{1.999648in}}{\pgfqpoint{3.097616in}{1.995257in}}{\pgfqpoint{3.089803in}{1.987444in}}%
\pgfpathcurveto{\pgfqpoint{3.081989in}{1.979630in}}{\pgfqpoint{3.077599in}{1.969031in}}{\pgfqpoint{3.077599in}{1.957981in}}%
\pgfpathcurveto{\pgfqpoint{3.077599in}{1.946931in}}{\pgfqpoint{3.081989in}{1.936332in}}{\pgfqpoint{3.089803in}{1.928518in}}%
\pgfpathcurveto{\pgfqpoint{3.097616in}{1.920705in}}{\pgfqpoint{3.108215in}{1.916314in}}{\pgfqpoint{3.119266in}{1.916314in}}%
\pgfpathclose%
\pgfusepath{stroke,fill}%
\end{pgfscope}%
\begin{pgfscope}%
\pgfpathrectangle{\pgfqpoint{0.600000in}{0.600000in}}{\pgfqpoint{3.900000in}{3.900000in}}%
\pgfusepath{clip}%
\pgfsetbuttcap%
\pgfsetroundjoin%
\definecolor{currentfill}{rgb}{0.121569,0.466667,0.705882}%
\pgfsetfillcolor{currentfill}%
\pgfsetlinewidth{1.003750pt}%
\definecolor{currentstroke}{rgb}{0.121569,0.466667,0.705882}%
\pgfsetstrokecolor{currentstroke}%
\pgfsetdash{}{0pt}%
\pgfpathmoveto{\pgfqpoint{2.348910in}{3.027258in}}%
\pgfpathcurveto{\pgfqpoint{2.359960in}{3.027258in}}{\pgfqpoint{2.370559in}{3.031648in}}{\pgfqpoint{2.378373in}{3.039462in}}%
\pgfpathcurveto{\pgfqpoint{2.386187in}{3.047276in}}{\pgfqpoint{2.390577in}{3.057875in}}{\pgfqpoint{2.390577in}{3.068925in}}%
\pgfpathcurveto{\pgfqpoint{2.390577in}{3.079975in}}{\pgfqpoint{2.386187in}{3.090574in}}{\pgfqpoint{2.378373in}{3.098388in}}%
\pgfpathcurveto{\pgfqpoint{2.370559in}{3.106201in}}{\pgfqpoint{2.359960in}{3.110591in}}{\pgfqpoint{2.348910in}{3.110591in}}%
\pgfpathcurveto{\pgfqpoint{2.337860in}{3.110591in}}{\pgfqpoint{2.327261in}{3.106201in}}{\pgfqpoint{2.319447in}{3.098388in}}%
\pgfpathcurveto{\pgfqpoint{2.311634in}{3.090574in}}{\pgfqpoint{2.307243in}{3.079975in}}{\pgfqpoint{2.307243in}{3.068925in}}%
\pgfpathcurveto{\pgfqpoint{2.307243in}{3.057875in}}{\pgfqpoint{2.311634in}{3.047276in}}{\pgfqpoint{2.319447in}{3.039462in}}%
\pgfpathcurveto{\pgfqpoint{2.327261in}{3.031648in}}{\pgfqpoint{2.337860in}{3.027258in}}{\pgfqpoint{2.348910in}{3.027258in}}%
\pgfpathclose%
\pgfusepath{stroke,fill}%
\end{pgfscope}%
\begin{pgfscope}%
\pgfpathrectangle{\pgfqpoint{0.600000in}{0.600000in}}{\pgfqpoint{3.900000in}{3.900000in}}%
\pgfusepath{clip}%
\pgfsetbuttcap%
\pgfsetroundjoin%
\definecolor{currentfill}{rgb}{0.121569,0.466667,0.705882}%
\pgfsetfillcolor{currentfill}%
\pgfsetlinewidth{1.003750pt}%
\definecolor{currentstroke}{rgb}{0.121569,0.466667,0.705882}%
\pgfsetstrokecolor{currentstroke}%
\pgfsetdash{}{0pt}%
\pgfpathmoveto{\pgfqpoint{2.837656in}{2.710316in}}%
\pgfpathcurveto{\pgfqpoint{2.848706in}{2.710316in}}{\pgfqpoint{2.859305in}{2.714707in}}{\pgfqpoint{2.867119in}{2.722520in}}%
\pgfpathcurveto{\pgfqpoint{2.874933in}{2.730334in}}{\pgfqpoint{2.879323in}{2.740933in}}{\pgfqpoint{2.879323in}{2.751983in}}%
\pgfpathcurveto{\pgfqpoint{2.879323in}{2.763033in}}{\pgfqpoint{2.874933in}{2.773632in}}{\pgfqpoint{2.867119in}{2.781446in}}%
\pgfpathcurveto{\pgfqpoint{2.859305in}{2.789259in}}{\pgfqpoint{2.848706in}{2.793650in}}{\pgfqpoint{2.837656in}{2.793650in}}%
\pgfpathcurveto{\pgfqpoint{2.826606in}{2.793650in}}{\pgfqpoint{2.816007in}{2.789259in}}{\pgfqpoint{2.808193in}{2.781446in}}%
\pgfpathcurveto{\pgfqpoint{2.800380in}{2.773632in}}{\pgfqpoint{2.795989in}{2.763033in}}{\pgfqpoint{2.795989in}{2.751983in}}%
\pgfpathcurveto{\pgfqpoint{2.795989in}{2.740933in}}{\pgfqpoint{2.800380in}{2.730334in}}{\pgfqpoint{2.808193in}{2.722520in}}%
\pgfpathcurveto{\pgfqpoint{2.816007in}{2.714707in}}{\pgfqpoint{2.826606in}{2.710316in}}{\pgfqpoint{2.837656in}{2.710316in}}%
\pgfpathclose%
\pgfusepath{stroke,fill}%
\end{pgfscope}%
\begin{pgfscope}%
\pgfpathrectangle{\pgfqpoint{0.600000in}{0.600000in}}{\pgfqpoint{3.900000in}{3.900000in}}%
\pgfusepath{clip}%
\pgfsetbuttcap%
\pgfsetroundjoin%
\definecolor{currentfill}{rgb}{0.121569,0.466667,0.705882}%
\pgfsetfillcolor{currentfill}%
\pgfsetlinewidth{1.003750pt}%
\definecolor{currentstroke}{rgb}{0.121569,0.466667,0.705882}%
\pgfsetstrokecolor{currentstroke}%
\pgfsetdash{}{0pt}%
\pgfpathmoveto{\pgfqpoint{2.343777in}{2.358805in}}%
\pgfpathcurveto{\pgfqpoint{2.354827in}{2.358805in}}{\pgfqpoint{2.365426in}{2.363195in}}{\pgfqpoint{2.373239in}{2.371009in}}%
\pgfpathcurveto{\pgfqpoint{2.381053in}{2.378822in}}{\pgfqpoint{2.385443in}{2.389421in}}{\pgfqpoint{2.385443in}{2.400471in}}%
\pgfpathcurveto{\pgfqpoint{2.385443in}{2.411522in}}{\pgfqpoint{2.381053in}{2.422121in}}{\pgfqpoint{2.373239in}{2.429934in}}%
\pgfpathcurveto{\pgfqpoint{2.365426in}{2.437748in}}{\pgfqpoint{2.354827in}{2.442138in}}{\pgfqpoint{2.343777in}{2.442138in}}%
\pgfpathcurveto{\pgfqpoint{2.332727in}{2.442138in}}{\pgfqpoint{2.322128in}{2.437748in}}{\pgfqpoint{2.314314in}{2.429934in}}%
\pgfpathcurveto{\pgfqpoint{2.306500in}{2.422121in}}{\pgfqpoint{2.302110in}{2.411522in}}{\pgfqpoint{2.302110in}{2.400471in}}%
\pgfpathcurveto{\pgfqpoint{2.302110in}{2.389421in}}{\pgfqpoint{2.306500in}{2.378822in}}{\pgfqpoint{2.314314in}{2.371009in}}%
\pgfpathcurveto{\pgfqpoint{2.322128in}{2.363195in}}{\pgfqpoint{2.332727in}{2.358805in}}{\pgfqpoint{2.343777in}{2.358805in}}%
\pgfpathclose%
\pgfusepath{stroke,fill}%
\end{pgfscope}%
\begin{pgfscope}%
\pgfpathrectangle{\pgfqpoint{0.600000in}{0.600000in}}{\pgfqpoint{3.900000in}{3.900000in}}%
\pgfusepath{clip}%
\pgfsetbuttcap%
\pgfsetroundjoin%
\definecolor{currentfill}{rgb}{0.121569,0.466667,0.705882}%
\pgfsetfillcolor{currentfill}%
\pgfsetlinewidth{1.003750pt}%
\definecolor{currentstroke}{rgb}{0.121569,0.466667,0.705882}%
\pgfsetstrokecolor{currentstroke}%
\pgfsetdash{}{0pt}%
\pgfpathmoveto{\pgfqpoint{3.821703in}{2.758879in}}%
\pgfpathcurveto{\pgfqpoint{3.832753in}{2.758879in}}{\pgfqpoint{3.843352in}{2.763269in}}{\pgfqpoint{3.851165in}{2.771083in}}%
\pgfpathcurveto{\pgfqpoint{3.858979in}{2.778897in}}{\pgfqpoint{3.863369in}{2.789496in}}{\pgfqpoint{3.863369in}{2.800546in}}%
\pgfpathcurveto{\pgfqpoint{3.863369in}{2.811596in}}{\pgfqpoint{3.858979in}{2.822195in}}{\pgfqpoint{3.851165in}{2.830008in}}%
\pgfpathcurveto{\pgfqpoint{3.843352in}{2.837822in}}{\pgfqpoint{3.832753in}{2.842212in}}{\pgfqpoint{3.821703in}{2.842212in}}%
\pgfpathcurveto{\pgfqpoint{3.810652in}{2.842212in}}{\pgfqpoint{3.800053in}{2.837822in}}{\pgfqpoint{3.792240in}{2.830008in}}%
\pgfpathcurveto{\pgfqpoint{3.784426in}{2.822195in}}{\pgfqpoint{3.780036in}{2.811596in}}{\pgfqpoint{3.780036in}{2.800546in}}%
\pgfpathcurveto{\pgfqpoint{3.780036in}{2.789496in}}{\pgfqpoint{3.784426in}{2.778897in}}{\pgfqpoint{3.792240in}{2.771083in}}%
\pgfpathcurveto{\pgfqpoint{3.800053in}{2.763269in}}{\pgfqpoint{3.810652in}{2.758879in}}{\pgfqpoint{3.821703in}{2.758879in}}%
\pgfpathclose%
\pgfusepath{stroke,fill}%
\end{pgfscope}%
\begin{pgfscope}%
\pgfpathrectangle{\pgfqpoint{0.600000in}{0.600000in}}{\pgfqpoint{3.900000in}{3.900000in}}%
\pgfusepath{clip}%
\pgfsetbuttcap%
\pgfsetroundjoin%
\definecolor{currentfill}{rgb}{0.121569,0.466667,0.705882}%
\pgfsetfillcolor{currentfill}%
\pgfsetlinewidth{1.003750pt}%
\definecolor{currentstroke}{rgb}{0.121569,0.466667,0.705882}%
\pgfsetstrokecolor{currentstroke}%
\pgfsetdash{}{0pt}%
\pgfpathmoveto{\pgfqpoint{2.341749in}{1.851609in}}%
\pgfpathcurveto{\pgfqpoint{2.352799in}{1.851609in}}{\pgfqpoint{2.363398in}{1.855999in}}{\pgfqpoint{2.371212in}{1.863813in}}%
\pgfpathcurveto{\pgfqpoint{2.379025in}{1.871627in}}{\pgfqpoint{2.383416in}{1.882226in}}{\pgfqpoint{2.383416in}{1.893276in}}%
\pgfpathcurveto{\pgfqpoint{2.383416in}{1.904326in}}{\pgfqpoint{2.379025in}{1.914925in}}{\pgfqpoint{2.371212in}{1.922739in}}%
\pgfpathcurveto{\pgfqpoint{2.363398in}{1.930552in}}{\pgfqpoint{2.352799in}{1.934943in}}{\pgfqpoint{2.341749in}{1.934943in}}%
\pgfpathcurveto{\pgfqpoint{2.330699in}{1.934943in}}{\pgfqpoint{2.320100in}{1.930552in}}{\pgfqpoint{2.312286in}{1.922739in}}%
\pgfpathcurveto{\pgfqpoint{2.304472in}{1.914925in}}{\pgfqpoint{2.300082in}{1.904326in}}{\pgfqpoint{2.300082in}{1.893276in}}%
\pgfpathcurveto{\pgfqpoint{2.300082in}{1.882226in}}{\pgfqpoint{2.304472in}{1.871627in}}{\pgfqpoint{2.312286in}{1.863813in}}%
\pgfpathcurveto{\pgfqpoint{2.320100in}{1.855999in}}{\pgfqpoint{2.330699in}{1.851609in}}{\pgfqpoint{2.341749in}{1.851609in}}%
\pgfpathclose%
\pgfusepath{stroke,fill}%
\end{pgfscope}%
\begin{pgfscope}%
\pgfpathrectangle{\pgfqpoint{0.600000in}{0.600000in}}{\pgfqpoint{3.900000in}{3.900000in}}%
\pgfusepath{clip}%
\pgfsetbuttcap%
\pgfsetroundjoin%
\definecolor{currentfill}{rgb}{0.121569,0.466667,0.705882}%
\pgfsetfillcolor{currentfill}%
\pgfsetlinewidth{1.003750pt}%
\definecolor{currentstroke}{rgb}{0.121569,0.466667,0.705882}%
\pgfsetstrokecolor{currentstroke}%
\pgfsetdash{}{0pt}%
\pgfpathmoveto{\pgfqpoint{3.601943in}{2.604358in}}%
\pgfpathcurveto{\pgfqpoint{3.612993in}{2.604358in}}{\pgfqpoint{3.623593in}{2.608748in}}{\pgfqpoint{3.631406in}{2.616562in}}%
\pgfpathcurveto{\pgfqpoint{3.639220in}{2.624375in}}{\pgfqpoint{3.643610in}{2.634975in}}{\pgfqpoint{3.643610in}{2.646025in}}%
\pgfpathcurveto{\pgfqpoint{3.643610in}{2.657075in}}{\pgfqpoint{3.639220in}{2.667674in}}{\pgfqpoint{3.631406in}{2.675487in}}%
\pgfpathcurveto{\pgfqpoint{3.623593in}{2.683301in}}{\pgfqpoint{3.612993in}{2.687691in}}{\pgfqpoint{3.601943in}{2.687691in}}%
\pgfpathcurveto{\pgfqpoint{3.590893in}{2.687691in}}{\pgfqpoint{3.580294in}{2.683301in}}{\pgfqpoint{3.572481in}{2.675487in}}%
\pgfpathcurveto{\pgfqpoint{3.564667in}{2.667674in}}{\pgfqpoint{3.560277in}{2.657075in}}{\pgfqpoint{3.560277in}{2.646025in}}%
\pgfpathcurveto{\pgfqpoint{3.560277in}{2.634975in}}{\pgfqpoint{3.564667in}{2.624375in}}{\pgfqpoint{3.572481in}{2.616562in}}%
\pgfpathcurveto{\pgfqpoint{3.580294in}{2.608748in}}{\pgfqpoint{3.590893in}{2.604358in}}{\pgfqpoint{3.601943in}{2.604358in}}%
\pgfpathclose%
\pgfusepath{stroke,fill}%
\end{pgfscope}%
\begin{pgfscope}%
\pgfpathrectangle{\pgfqpoint{0.600000in}{0.600000in}}{\pgfqpoint{3.900000in}{3.900000in}}%
\pgfusepath{clip}%
\pgfsetbuttcap%
\pgfsetroundjoin%
\definecolor{currentfill}{rgb}{0.121569,0.466667,0.705882}%
\pgfsetfillcolor{currentfill}%
\pgfsetlinewidth{1.003750pt}%
\definecolor{currentstroke}{rgb}{0.121569,0.466667,0.705882}%
\pgfsetstrokecolor{currentstroke}%
\pgfsetdash{}{0pt}%
\pgfpathmoveto{\pgfqpoint{3.318823in}{2.422786in}}%
\pgfpathcurveto{\pgfqpoint{3.329874in}{2.422786in}}{\pgfqpoint{3.340473in}{2.427176in}}{\pgfqpoint{3.348286in}{2.434990in}}%
\pgfpathcurveto{\pgfqpoint{3.356100in}{2.442803in}}{\pgfqpoint{3.360490in}{2.453402in}}{\pgfqpoint{3.360490in}{2.464453in}}%
\pgfpathcurveto{\pgfqpoint{3.360490in}{2.475503in}}{\pgfqpoint{3.356100in}{2.486102in}}{\pgfqpoint{3.348286in}{2.493915in}}%
\pgfpathcurveto{\pgfqpoint{3.340473in}{2.501729in}}{\pgfqpoint{3.329874in}{2.506119in}}{\pgfqpoint{3.318823in}{2.506119in}}%
\pgfpathcurveto{\pgfqpoint{3.307773in}{2.506119in}}{\pgfqpoint{3.297174in}{2.501729in}}{\pgfqpoint{3.289361in}{2.493915in}}%
\pgfpathcurveto{\pgfqpoint{3.281547in}{2.486102in}}{\pgfqpoint{3.277157in}{2.475503in}}{\pgfqpoint{3.277157in}{2.464453in}}%
\pgfpathcurveto{\pgfqpoint{3.277157in}{2.453402in}}{\pgfqpoint{3.281547in}{2.442803in}}{\pgfqpoint{3.289361in}{2.434990in}}%
\pgfpathcurveto{\pgfqpoint{3.297174in}{2.427176in}}{\pgfqpoint{3.307773in}{2.422786in}}{\pgfqpoint{3.318823in}{2.422786in}}%
\pgfpathclose%
\pgfusepath{stroke,fill}%
\end{pgfscope}%
\begin{pgfscope}%
\pgfpathrectangle{\pgfqpoint{0.600000in}{0.600000in}}{\pgfqpoint{3.900000in}{3.900000in}}%
\pgfusepath{clip}%
\pgfsetbuttcap%
\pgfsetroundjoin%
\definecolor{currentfill}{rgb}{0.121569,0.466667,0.705882}%
\pgfsetfillcolor{currentfill}%
\pgfsetlinewidth{1.003750pt}%
\definecolor{currentstroke}{rgb}{0.121569,0.466667,0.705882}%
\pgfsetstrokecolor{currentstroke}%
\pgfsetdash{}{0pt}%
\pgfpathmoveto{\pgfqpoint{2.372911in}{2.235155in}}%
\pgfpathcurveto{\pgfqpoint{2.383961in}{2.235155in}}{\pgfqpoint{2.394560in}{2.239545in}}{\pgfqpoint{2.402374in}{2.247359in}}%
\pgfpathcurveto{\pgfqpoint{2.410187in}{2.255172in}}{\pgfqpoint{2.414578in}{2.265771in}}{\pgfqpoint{2.414578in}{2.276821in}}%
\pgfpathcurveto{\pgfqpoint{2.414578in}{2.287871in}}{\pgfqpoint{2.410187in}{2.298470in}}{\pgfqpoint{2.402374in}{2.306284in}}%
\pgfpathcurveto{\pgfqpoint{2.394560in}{2.314098in}}{\pgfqpoint{2.383961in}{2.318488in}}{\pgfqpoint{2.372911in}{2.318488in}}%
\pgfpathcurveto{\pgfqpoint{2.361861in}{2.318488in}}{\pgfqpoint{2.351262in}{2.314098in}}{\pgfqpoint{2.343448in}{2.306284in}}%
\pgfpathcurveto{\pgfqpoint{2.335635in}{2.298470in}}{\pgfqpoint{2.331244in}{2.287871in}}{\pgfqpoint{2.331244in}{2.276821in}}%
\pgfpathcurveto{\pgfqpoint{2.331244in}{2.265771in}}{\pgfqpoint{2.335635in}{2.255172in}}{\pgfqpoint{2.343448in}{2.247359in}}%
\pgfpathcurveto{\pgfqpoint{2.351262in}{2.239545in}}{\pgfqpoint{2.361861in}{2.235155in}}{\pgfqpoint{2.372911in}{2.235155in}}%
\pgfpathclose%
\pgfusepath{stroke,fill}%
\end{pgfscope}%
\begin{pgfscope}%
\pgfpathrectangle{\pgfqpoint{0.600000in}{0.600000in}}{\pgfqpoint{3.900000in}{3.900000in}}%
\pgfusepath{clip}%
\pgfsetbuttcap%
\pgfsetroundjoin%
\definecolor{currentfill}{rgb}{0.121569,0.466667,0.705882}%
\pgfsetfillcolor{currentfill}%
\pgfsetlinewidth{1.003750pt}%
\definecolor{currentstroke}{rgb}{0.121569,0.466667,0.705882}%
\pgfsetstrokecolor{currentstroke}%
\pgfsetdash{}{0pt}%
\pgfpathmoveto{\pgfqpoint{3.339684in}{3.196580in}}%
\pgfpathcurveto{\pgfqpoint{3.350734in}{3.196580in}}{\pgfqpoint{3.361333in}{3.200971in}}{\pgfqpoint{3.369146in}{3.208784in}}%
\pgfpathcurveto{\pgfqpoint{3.376960in}{3.216598in}}{\pgfqpoint{3.381350in}{3.227197in}}{\pgfqpoint{3.381350in}{3.238247in}}%
\pgfpathcurveto{\pgfqpoint{3.381350in}{3.249297in}}{\pgfqpoint{3.376960in}{3.259896in}}{\pgfqpoint{3.369146in}{3.267710in}}%
\pgfpathcurveto{\pgfqpoint{3.361333in}{3.275523in}}{\pgfqpoint{3.350734in}{3.279914in}}{\pgfqpoint{3.339684in}{3.279914in}}%
\pgfpathcurveto{\pgfqpoint{3.328633in}{3.279914in}}{\pgfqpoint{3.318034in}{3.275523in}}{\pgfqpoint{3.310221in}{3.267710in}}%
\pgfpathcurveto{\pgfqpoint{3.302407in}{3.259896in}}{\pgfqpoint{3.298017in}{3.249297in}}{\pgfqpoint{3.298017in}{3.238247in}}%
\pgfpathcurveto{\pgfqpoint{3.298017in}{3.227197in}}{\pgfqpoint{3.302407in}{3.216598in}}{\pgfqpoint{3.310221in}{3.208784in}}%
\pgfpathcurveto{\pgfqpoint{3.318034in}{3.200971in}}{\pgfqpoint{3.328633in}{3.196580in}}{\pgfqpoint{3.339684in}{3.196580in}}%
\pgfpathclose%
\pgfusepath{stroke,fill}%
\end{pgfscope}%
\begin{pgfscope}%
\pgfpathrectangle{\pgfqpoint{0.600000in}{0.600000in}}{\pgfqpoint{3.900000in}{3.900000in}}%
\pgfusepath{clip}%
\pgfsetbuttcap%
\pgfsetroundjoin%
\definecolor{currentfill}{rgb}{0.121569,0.466667,0.705882}%
\pgfsetfillcolor{currentfill}%
\pgfsetlinewidth{1.003750pt}%
\definecolor{currentstroke}{rgb}{0.121569,0.466667,0.705882}%
\pgfsetstrokecolor{currentstroke}%
\pgfsetdash{}{0pt}%
\pgfpathmoveto{\pgfqpoint{2.598868in}{1.974572in}}%
\pgfpathcurveto{\pgfqpoint{2.609918in}{1.974572in}}{\pgfqpoint{2.620517in}{1.978962in}}{\pgfqpoint{2.628331in}{1.986775in}}%
\pgfpathcurveto{\pgfqpoint{2.636144in}{1.994589in}}{\pgfqpoint{2.640535in}{2.005188in}}{\pgfqpoint{2.640535in}{2.016238in}}%
\pgfpathcurveto{\pgfqpoint{2.640535in}{2.027288in}}{\pgfqpoint{2.636144in}{2.037887in}}{\pgfqpoint{2.628331in}{2.045701in}}%
\pgfpathcurveto{\pgfqpoint{2.620517in}{2.053515in}}{\pgfqpoint{2.609918in}{2.057905in}}{\pgfqpoint{2.598868in}{2.057905in}}%
\pgfpathcurveto{\pgfqpoint{2.587818in}{2.057905in}}{\pgfqpoint{2.577219in}{2.053515in}}{\pgfqpoint{2.569405in}{2.045701in}}%
\pgfpathcurveto{\pgfqpoint{2.561592in}{2.037887in}}{\pgfqpoint{2.557201in}{2.027288in}}{\pgfqpoint{2.557201in}{2.016238in}}%
\pgfpathcurveto{\pgfqpoint{2.557201in}{2.005188in}}{\pgfqpoint{2.561592in}{1.994589in}}{\pgfqpoint{2.569405in}{1.986775in}}%
\pgfpathcurveto{\pgfqpoint{2.577219in}{1.978962in}}{\pgfqpoint{2.587818in}{1.974572in}}{\pgfqpoint{2.598868in}{1.974572in}}%
\pgfpathclose%
\pgfusepath{stroke,fill}%
\end{pgfscope}%
\begin{pgfscope}%
\pgfpathrectangle{\pgfqpoint{0.600000in}{0.600000in}}{\pgfqpoint{3.900000in}{3.900000in}}%
\pgfusepath{clip}%
\pgfsetbuttcap%
\pgfsetroundjoin%
\definecolor{currentfill}{rgb}{0.121569,0.466667,0.705882}%
\pgfsetfillcolor{currentfill}%
\pgfsetlinewidth{1.003750pt}%
\definecolor{currentstroke}{rgb}{0.121569,0.466667,0.705882}%
\pgfsetstrokecolor{currentstroke}%
\pgfsetdash{}{0pt}%
\pgfpathmoveto{\pgfqpoint{2.201616in}{1.799724in}}%
\pgfpathcurveto{\pgfqpoint{2.212666in}{1.799724in}}{\pgfqpoint{2.223266in}{1.804114in}}{\pgfqpoint{2.231079in}{1.811928in}}%
\pgfpathcurveto{\pgfqpoint{2.238893in}{1.819741in}}{\pgfqpoint{2.243283in}{1.830340in}}{\pgfqpoint{2.243283in}{1.841390in}}%
\pgfpathcurveto{\pgfqpoint{2.243283in}{1.852441in}}{\pgfqpoint{2.238893in}{1.863040in}}{\pgfqpoint{2.231079in}{1.870853in}}%
\pgfpathcurveto{\pgfqpoint{2.223266in}{1.878667in}}{\pgfqpoint{2.212666in}{1.883057in}}{\pgfqpoint{2.201616in}{1.883057in}}%
\pgfpathcurveto{\pgfqpoint{2.190566in}{1.883057in}}{\pgfqpoint{2.179967in}{1.878667in}}{\pgfqpoint{2.172154in}{1.870853in}}%
\pgfpathcurveto{\pgfqpoint{2.164340in}{1.863040in}}{\pgfqpoint{2.159950in}{1.852441in}}{\pgfqpoint{2.159950in}{1.841390in}}%
\pgfpathcurveto{\pgfqpoint{2.159950in}{1.830340in}}{\pgfqpoint{2.164340in}{1.819741in}}{\pgfqpoint{2.172154in}{1.811928in}}%
\pgfpathcurveto{\pgfqpoint{2.179967in}{1.804114in}}{\pgfqpoint{2.190566in}{1.799724in}}{\pgfqpoint{2.201616in}{1.799724in}}%
\pgfpathclose%
\pgfusepath{stroke,fill}%
\end{pgfscope}%
\begin{pgfscope}%
\pgfpathrectangle{\pgfqpoint{0.600000in}{0.600000in}}{\pgfqpoint{3.900000in}{3.900000in}}%
\pgfusepath{clip}%
\pgfsetbuttcap%
\pgfsetroundjoin%
\definecolor{currentfill}{rgb}{0.121569,0.466667,0.705882}%
\pgfsetfillcolor{currentfill}%
\pgfsetlinewidth{1.003750pt}%
\definecolor{currentstroke}{rgb}{0.121569,0.466667,0.705882}%
\pgfsetstrokecolor{currentstroke}%
\pgfsetdash{}{0pt}%
\pgfpathmoveto{\pgfqpoint{2.579801in}{2.376751in}}%
\pgfpathcurveto{\pgfqpoint{2.590851in}{2.376751in}}{\pgfqpoint{2.601450in}{2.381141in}}{\pgfqpoint{2.609264in}{2.388955in}}%
\pgfpathcurveto{\pgfqpoint{2.617077in}{2.396768in}}{\pgfqpoint{2.621468in}{2.407368in}}{\pgfqpoint{2.621468in}{2.418418in}}%
\pgfpathcurveto{\pgfqpoint{2.621468in}{2.429468in}}{\pgfqpoint{2.617077in}{2.440067in}}{\pgfqpoint{2.609264in}{2.447880in}}%
\pgfpathcurveto{\pgfqpoint{2.601450in}{2.455694in}}{\pgfqpoint{2.590851in}{2.460084in}}{\pgfqpoint{2.579801in}{2.460084in}}%
\pgfpathcurveto{\pgfqpoint{2.568751in}{2.460084in}}{\pgfqpoint{2.558152in}{2.455694in}}{\pgfqpoint{2.550338in}{2.447880in}}%
\pgfpathcurveto{\pgfqpoint{2.542525in}{2.440067in}}{\pgfqpoint{2.538134in}{2.429468in}}{\pgfqpoint{2.538134in}{2.418418in}}%
\pgfpathcurveto{\pgfqpoint{2.538134in}{2.407368in}}{\pgfqpoint{2.542525in}{2.396768in}}{\pgfqpoint{2.550338in}{2.388955in}}%
\pgfpathcurveto{\pgfqpoint{2.558152in}{2.381141in}}{\pgfqpoint{2.568751in}{2.376751in}}{\pgfqpoint{2.579801in}{2.376751in}}%
\pgfpathclose%
\pgfusepath{stroke,fill}%
\end{pgfscope}%
\begin{pgfscope}%
\pgfpathrectangle{\pgfqpoint{0.600000in}{0.600000in}}{\pgfqpoint{3.900000in}{3.900000in}}%
\pgfusepath{clip}%
\pgfsetbuttcap%
\pgfsetroundjoin%
\definecolor{currentfill}{rgb}{0.121569,0.466667,0.705882}%
\pgfsetfillcolor{currentfill}%
\pgfsetlinewidth{1.003750pt}%
\definecolor{currentstroke}{rgb}{0.121569,0.466667,0.705882}%
\pgfsetstrokecolor{currentstroke}%
\pgfsetdash{}{0pt}%
\pgfpathmoveto{\pgfqpoint{2.595135in}{1.634046in}}%
\pgfpathcurveto{\pgfqpoint{2.606185in}{1.634046in}}{\pgfqpoint{2.616784in}{1.638436in}}{\pgfqpoint{2.624597in}{1.646250in}}%
\pgfpathcurveto{\pgfqpoint{2.632411in}{1.654064in}}{\pgfqpoint{2.636801in}{1.664663in}}{\pgfqpoint{2.636801in}{1.675713in}}%
\pgfpathcurveto{\pgfqpoint{2.636801in}{1.686763in}}{\pgfqpoint{2.632411in}{1.697362in}}{\pgfqpoint{2.624597in}{1.705176in}}%
\pgfpathcurveto{\pgfqpoint{2.616784in}{1.712989in}}{\pgfqpoint{2.606185in}{1.717379in}}{\pgfqpoint{2.595135in}{1.717379in}}%
\pgfpathcurveto{\pgfqpoint{2.584084in}{1.717379in}}{\pgfqpoint{2.573485in}{1.712989in}}{\pgfqpoint{2.565672in}{1.705176in}}%
\pgfpathcurveto{\pgfqpoint{2.557858in}{1.697362in}}{\pgfqpoint{2.553468in}{1.686763in}}{\pgfqpoint{2.553468in}{1.675713in}}%
\pgfpathcurveto{\pgfqpoint{2.553468in}{1.664663in}}{\pgfqpoint{2.557858in}{1.654064in}}{\pgfqpoint{2.565672in}{1.646250in}}%
\pgfpathcurveto{\pgfqpoint{2.573485in}{1.638436in}}{\pgfqpoint{2.584084in}{1.634046in}}{\pgfqpoint{2.595135in}{1.634046in}}%
\pgfpathclose%
\pgfusepath{stroke,fill}%
\end{pgfscope}%
\begin{pgfscope}%
\pgfpathrectangle{\pgfqpoint{0.600000in}{0.600000in}}{\pgfqpoint{3.900000in}{3.900000in}}%
\pgfusepath{clip}%
\pgfsetbuttcap%
\pgfsetroundjoin%
\definecolor{currentfill}{rgb}{0.121569,0.466667,0.705882}%
\pgfsetfillcolor{currentfill}%
\pgfsetlinewidth{1.003750pt}%
\definecolor{currentstroke}{rgb}{0.121569,0.466667,0.705882}%
\pgfsetstrokecolor{currentstroke}%
\pgfsetdash{}{0pt}%
\pgfpathmoveto{\pgfqpoint{2.781500in}{1.477706in}}%
\pgfpathcurveto{\pgfqpoint{2.792551in}{1.477706in}}{\pgfqpoint{2.803150in}{1.482096in}}{\pgfqpoint{2.810963in}{1.489910in}}%
\pgfpathcurveto{\pgfqpoint{2.818777in}{1.497723in}}{\pgfqpoint{2.823167in}{1.508322in}}{\pgfqpoint{2.823167in}{1.519372in}}%
\pgfpathcurveto{\pgfqpoint{2.823167in}{1.530423in}}{\pgfqpoint{2.818777in}{1.541022in}}{\pgfqpoint{2.810963in}{1.548835in}}%
\pgfpathcurveto{\pgfqpoint{2.803150in}{1.556649in}}{\pgfqpoint{2.792551in}{1.561039in}}{\pgfqpoint{2.781500in}{1.561039in}}%
\pgfpathcurveto{\pgfqpoint{2.770450in}{1.561039in}}{\pgfqpoint{2.759851in}{1.556649in}}{\pgfqpoint{2.752038in}{1.548835in}}%
\pgfpathcurveto{\pgfqpoint{2.744224in}{1.541022in}}{\pgfqpoint{2.739834in}{1.530423in}}{\pgfqpoint{2.739834in}{1.519372in}}%
\pgfpathcurveto{\pgfqpoint{2.739834in}{1.508322in}}{\pgfqpoint{2.744224in}{1.497723in}}{\pgfqpoint{2.752038in}{1.489910in}}%
\pgfpathcurveto{\pgfqpoint{2.759851in}{1.482096in}}{\pgfqpoint{2.770450in}{1.477706in}}{\pgfqpoint{2.781500in}{1.477706in}}%
\pgfpathclose%
\pgfusepath{stroke,fill}%
\end{pgfscope}%
\begin{pgfscope}%
\pgfpathrectangle{\pgfqpoint{0.600000in}{0.600000in}}{\pgfqpoint{3.900000in}{3.900000in}}%
\pgfusepath{clip}%
\pgfsetbuttcap%
\pgfsetroundjoin%
\definecolor{currentfill}{rgb}{0.121569,0.466667,0.705882}%
\pgfsetfillcolor{currentfill}%
\pgfsetlinewidth{1.003750pt}%
\definecolor{currentstroke}{rgb}{0.121569,0.466667,0.705882}%
\pgfsetstrokecolor{currentstroke}%
\pgfsetdash{}{0pt}%
\pgfpathmoveto{\pgfqpoint{2.387844in}{2.086054in}}%
\pgfpathcurveto{\pgfqpoint{2.398894in}{2.086054in}}{\pgfqpoint{2.409493in}{2.090445in}}{\pgfqpoint{2.417307in}{2.098258in}}%
\pgfpathcurveto{\pgfqpoint{2.425121in}{2.106072in}}{\pgfqpoint{2.429511in}{2.116671in}}{\pgfqpoint{2.429511in}{2.127721in}}%
\pgfpathcurveto{\pgfqpoint{2.429511in}{2.138771in}}{\pgfqpoint{2.425121in}{2.149370in}}{\pgfqpoint{2.417307in}{2.157184in}}%
\pgfpathcurveto{\pgfqpoint{2.409493in}{2.164997in}}{\pgfqpoint{2.398894in}{2.169388in}}{\pgfqpoint{2.387844in}{2.169388in}}%
\pgfpathcurveto{\pgfqpoint{2.376794in}{2.169388in}}{\pgfqpoint{2.366195in}{2.164997in}}{\pgfqpoint{2.358382in}{2.157184in}}%
\pgfpathcurveto{\pgfqpoint{2.350568in}{2.149370in}}{\pgfqpoint{2.346178in}{2.138771in}}{\pgfqpoint{2.346178in}{2.127721in}}%
\pgfpathcurveto{\pgfqpoint{2.346178in}{2.116671in}}{\pgfqpoint{2.350568in}{2.106072in}}{\pgfqpoint{2.358382in}{2.098258in}}%
\pgfpathcurveto{\pgfqpoint{2.366195in}{2.090445in}}{\pgfqpoint{2.376794in}{2.086054in}}{\pgfqpoint{2.387844in}{2.086054in}}%
\pgfpathclose%
\pgfusepath{stroke,fill}%
\end{pgfscope}%
\begin{pgfscope}%
\pgfpathrectangle{\pgfqpoint{0.600000in}{0.600000in}}{\pgfqpoint{3.900000in}{3.900000in}}%
\pgfusepath{clip}%
\pgfsetbuttcap%
\pgfsetroundjoin%
\definecolor{currentfill}{rgb}{0.121569,0.466667,0.705882}%
\pgfsetfillcolor{currentfill}%
\pgfsetlinewidth{1.003750pt}%
\definecolor{currentstroke}{rgb}{0.121569,0.466667,0.705882}%
\pgfsetstrokecolor{currentstroke}%
\pgfsetdash{}{0pt}%
\pgfpathmoveto{\pgfqpoint{3.131908in}{2.705871in}}%
\pgfpathcurveto{\pgfqpoint{3.142958in}{2.705871in}}{\pgfqpoint{3.153557in}{2.710262in}}{\pgfqpoint{3.161371in}{2.718075in}}%
\pgfpathcurveto{\pgfqpoint{3.169184in}{2.725889in}}{\pgfqpoint{3.173574in}{2.736488in}}{\pgfqpoint{3.173574in}{2.747538in}}%
\pgfpathcurveto{\pgfqpoint{3.173574in}{2.758588in}}{\pgfqpoint{3.169184in}{2.769187in}}{\pgfqpoint{3.161371in}{2.777001in}}%
\pgfpathcurveto{\pgfqpoint{3.153557in}{2.784814in}}{\pgfqpoint{3.142958in}{2.789205in}}{\pgfqpoint{3.131908in}{2.789205in}}%
\pgfpathcurveto{\pgfqpoint{3.120858in}{2.789205in}}{\pgfqpoint{3.110259in}{2.784814in}}{\pgfqpoint{3.102445in}{2.777001in}}%
\pgfpathcurveto{\pgfqpoint{3.094631in}{2.769187in}}{\pgfqpoint{3.090241in}{2.758588in}}{\pgfqpoint{3.090241in}{2.747538in}}%
\pgfpathcurveto{\pgfqpoint{3.090241in}{2.736488in}}{\pgfqpoint{3.094631in}{2.725889in}}{\pgfqpoint{3.102445in}{2.718075in}}%
\pgfpathcurveto{\pgfqpoint{3.110259in}{2.710262in}}{\pgfqpoint{3.120858in}{2.705871in}}{\pgfqpoint{3.131908in}{2.705871in}}%
\pgfpathclose%
\pgfusepath{stroke,fill}%
\end{pgfscope}%
\begin{pgfscope}%
\pgfpathrectangle{\pgfqpoint{0.600000in}{0.600000in}}{\pgfqpoint{3.900000in}{3.900000in}}%
\pgfusepath{clip}%
\pgfsetbuttcap%
\pgfsetroundjoin%
\definecolor{currentfill}{rgb}{0.121569,0.466667,0.705882}%
\pgfsetfillcolor{currentfill}%
\pgfsetlinewidth{1.003750pt}%
\definecolor{currentstroke}{rgb}{0.121569,0.466667,0.705882}%
\pgfsetstrokecolor{currentstroke}%
\pgfsetdash{}{0pt}%
\pgfpathmoveto{\pgfqpoint{2.279361in}{2.468726in}}%
\pgfpathcurveto{\pgfqpoint{2.290411in}{2.468726in}}{\pgfqpoint{2.301010in}{2.473116in}}{\pgfqpoint{2.308823in}{2.480929in}}%
\pgfpathcurveto{\pgfqpoint{2.316637in}{2.488743in}}{\pgfqpoint{2.321027in}{2.499342in}}{\pgfqpoint{2.321027in}{2.510392in}}%
\pgfpathcurveto{\pgfqpoint{2.321027in}{2.521442in}}{\pgfqpoint{2.316637in}{2.532041in}}{\pgfqpoint{2.308823in}{2.539855in}}%
\pgfpathcurveto{\pgfqpoint{2.301010in}{2.547669in}}{\pgfqpoint{2.290411in}{2.552059in}}{\pgfqpoint{2.279361in}{2.552059in}}%
\pgfpathcurveto{\pgfqpoint{2.268311in}{2.552059in}}{\pgfqpoint{2.257712in}{2.547669in}}{\pgfqpoint{2.249898in}{2.539855in}}%
\pgfpathcurveto{\pgfqpoint{2.242084in}{2.532041in}}{\pgfqpoint{2.237694in}{2.521442in}}{\pgfqpoint{2.237694in}{2.510392in}}%
\pgfpathcurveto{\pgfqpoint{2.237694in}{2.499342in}}{\pgfqpoint{2.242084in}{2.488743in}}{\pgfqpoint{2.249898in}{2.480929in}}%
\pgfpathcurveto{\pgfqpoint{2.257712in}{2.473116in}}{\pgfqpoint{2.268311in}{2.468726in}}{\pgfqpoint{2.279361in}{2.468726in}}%
\pgfpathclose%
\pgfusepath{stroke,fill}%
\end{pgfscope}%
\begin{pgfscope}%
\pgfpathrectangle{\pgfqpoint{0.600000in}{0.600000in}}{\pgfqpoint{3.900000in}{3.900000in}}%
\pgfusepath{clip}%
\pgfsetbuttcap%
\pgfsetroundjoin%
\definecolor{currentfill}{rgb}{0.121569,0.466667,0.705882}%
\pgfsetfillcolor{currentfill}%
\pgfsetlinewidth{1.003750pt}%
\definecolor{currentstroke}{rgb}{0.121569,0.466667,0.705882}%
\pgfsetstrokecolor{currentstroke}%
\pgfsetdash{}{0pt}%
\pgfpathmoveto{\pgfqpoint{2.931550in}{2.926806in}}%
\pgfpathcurveto{\pgfqpoint{2.942600in}{2.926806in}}{\pgfqpoint{2.953199in}{2.931196in}}{\pgfqpoint{2.961013in}{2.939009in}}%
\pgfpathcurveto{\pgfqpoint{2.968827in}{2.946823in}}{\pgfqpoint{2.973217in}{2.957422in}}{\pgfqpoint{2.973217in}{2.968472in}}%
\pgfpathcurveto{\pgfqpoint{2.973217in}{2.979522in}}{\pgfqpoint{2.968827in}{2.990121in}}{\pgfqpoint{2.961013in}{2.997935in}}%
\pgfpathcurveto{\pgfqpoint{2.953199in}{3.005749in}}{\pgfqpoint{2.942600in}{3.010139in}}{\pgfqpoint{2.931550in}{3.010139in}}%
\pgfpathcurveto{\pgfqpoint{2.920500in}{3.010139in}}{\pgfqpoint{2.909901in}{3.005749in}}{\pgfqpoint{2.902088in}{2.997935in}}%
\pgfpathcurveto{\pgfqpoint{2.894274in}{2.990121in}}{\pgfqpoint{2.889884in}{2.979522in}}{\pgfqpoint{2.889884in}{2.968472in}}%
\pgfpathcurveto{\pgfqpoint{2.889884in}{2.957422in}}{\pgfqpoint{2.894274in}{2.946823in}}{\pgfqpoint{2.902088in}{2.939009in}}%
\pgfpathcurveto{\pgfqpoint{2.909901in}{2.931196in}}{\pgfqpoint{2.920500in}{2.926806in}}{\pgfqpoint{2.931550in}{2.926806in}}%
\pgfpathclose%
\pgfusepath{stroke,fill}%
\end{pgfscope}%
\begin{pgfscope}%
\pgfpathrectangle{\pgfqpoint{0.600000in}{0.600000in}}{\pgfqpoint{3.900000in}{3.900000in}}%
\pgfusepath{clip}%
\pgfsetbuttcap%
\pgfsetroundjoin%
\definecolor{currentfill}{rgb}{0.121569,0.466667,0.705882}%
\pgfsetfillcolor{currentfill}%
\pgfsetlinewidth{1.003750pt}%
\definecolor{currentstroke}{rgb}{0.121569,0.466667,0.705882}%
\pgfsetstrokecolor{currentstroke}%
\pgfsetdash{}{0pt}%
\pgfpathmoveto{\pgfqpoint{2.605433in}{2.293384in}}%
\pgfpathcurveto{\pgfqpoint{2.616483in}{2.293384in}}{\pgfqpoint{2.627082in}{2.297774in}}{\pgfqpoint{2.634895in}{2.305588in}}%
\pgfpathcurveto{\pgfqpoint{2.642709in}{2.313401in}}{\pgfqpoint{2.647099in}{2.324000in}}{\pgfqpoint{2.647099in}{2.335051in}}%
\pgfpathcurveto{\pgfqpoint{2.647099in}{2.346101in}}{\pgfqpoint{2.642709in}{2.356700in}}{\pgfqpoint{2.634895in}{2.364513in}}%
\pgfpathcurveto{\pgfqpoint{2.627082in}{2.372327in}}{\pgfqpoint{2.616483in}{2.376717in}}{\pgfqpoint{2.605433in}{2.376717in}}%
\pgfpathcurveto{\pgfqpoint{2.594383in}{2.376717in}}{\pgfqpoint{2.583783in}{2.372327in}}{\pgfqpoint{2.575970in}{2.364513in}}%
\pgfpathcurveto{\pgfqpoint{2.568156in}{2.356700in}}{\pgfqpoint{2.563766in}{2.346101in}}{\pgfqpoint{2.563766in}{2.335051in}}%
\pgfpathcurveto{\pgfqpoint{2.563766in}{2.324000in}}{\pgfqpoint{2.568156in}{2.313401in}}{\pgfqpoint{2.575970in}{2.305588in}}%
\pgfpathcurveto{\pgfqpoint{2.583783in}{2.297774in}}{\pgfqpoint{2.594383in}{2.293384in}}{\pgfqpoint{2.605433in}{2.293384in}}%
\pgfpathclose%
\pgfusepath{stroke,fill}%
\end{pgfscope}%
\begin{pgfscope}%
\pgfpathrectangle{\pgfqpoint{0.600000in}{0.600000in}}{\pgfqpoint{3.900000in}{3.900000in}}%
\pgfusepath{clip}%
\pgfsetbuttcap%
\pgfsetroundjoin%
\definecolor{currentfill}{rgb}{0.121569,0.466667,0.705882}%
\pgfsetfillcolor{currentfill}%
\pgfsetlinewidth{1.003750pt}%
\definecolor{currentstroke}{rgb}{0.121569,0.466667,0.705882}%
\pgfsetstrokecolor{currentstroke}%
\pgfsetdash{}{0pt}%
\pgfpathmoveto{\pgfqpoint{1.634971in}{3.216109in}}%
\pgfpathcurveto{\pgfqpoint{1.646021in}{3.216109in}}{\pgfqpoint{1.656620in}{3.220499in}}{\pgfqpoint{1.664433in}{3.228312in}}%
\pgfpathcurveto{\pgfqpoint{1.672247in}{3.236126in}}{\pgfqpoint{1.676637in}{3.246725in}}{\pgfqpoint{1.676637in}{3.257775in}}%
\pgfpathcurveto{\pgfqpoint{1.676637in}{3.268825in}}{\pgfqpoint{1.672247in}{3.279424in}}{\pgfqpoint{1.664433in}{3.287238in}}%
\pgfpathcurveto{\pgfqpoint{1.656620in}{3.295052in}}{\pgfqpoint{1.646021in}{3.299442in}}{\pgfqpoint{1.634971in}{3.299442in}}%
\pgfpathcurveto{\pgfqpoint{1.623920in}{3.299442in}}{\pgfqpoint{1.613321in}{3.295052in}}{\pgfqpoint{1.605508in}{3.287238in}}%
\pgfpathcurveto{\pgfqpoint{1.597694in}{3.279424in}}{\pgfqpoint{1.593304in}{3.268825in}}{\pgfqpoint{1.593304in}{3.257775in}}%
\pgfpathcurveto{\pgfqpoint{1.593304in}{3.246725in}}{\pgfqpoint{1.597694in}{3.236126in}}{\pgfqpoint{1.605508in}{3.228312in}}%
\pgfpathcurveto{\pgfqpoint{1.613321in}{3.220499in}}{\pgfqpoint{1.623920in}{3.216109in}}{\pgfqpoint{1.634971in}{3.216109in}}%
\pgfpathclose%
\pgfusepath{stroke,fill}%
\end{pgfscope}%
\begin{pgfscope}%
\pgfpathrectangle{\pgfqpoint{0.600000in}{0.600000in}}{\pgfqpoint{3.900000in}{3.900000in}}%
\pgfusepath{clip}%
\pgfsetbuttcap%
\pgfsetroundjoin%
\definecolor{currentfill}{rgb}{0.121569,0.466667,0.705882}%
\pgfsetfillcolor{currentfill}%
\pgfsetlinewidth{1.003750pt}%
\definecolor{currentstroke}{rgb}{0.121569,0.466667,0.705882}%
\pgfsetstrokecolor{currentstroke}%
\pgfsetdash{}{0pt}%
\pgfpathmoveto{\pgfqpoint{2.415162in}{2.638950in}}%
\pgfpathcurveto{\pgfqpoint{2.426212in}{2.638950in}}{\pgfqpoint{2.436811in}{2.643340in}}{\pgfqpoint{2.444625in}{2.651154in}}%
\pgfpathcurveto{\pgfqpoint{2.452438in}{2.658968in}}{\pgfqpoint{2.456829in}{2.669567in}}{\pgfqpoint{2.456829in}{2.680617in}}%
\pgfpathcurveto{\pgfqpoint{2.456829in}{2.691667in}}{\pgfqpoint{2.452438in}{2.702266in}}{\pgfqpoint{2.444625in}{2.710080in}}%
\pgfpathcurveto{\pgfqpoint{2.436811in}{2.717893in}}{\pgfqpoint{2.426212in}{2.722284in}}{\pgfqpoint{2.415162in}{2.722284in}}%
\pgfpathcurveto{\pgfqpoint{2.404112in}{2.722284in}}{\pgfqpoint{2.393513in}{2.717893in}}{\pgfqpoint{2.385699in}{2.710080in}}%
\pgfpathcurveto{\pgfqpoint{2.377885in}{2.702266in}}{\pgfqpoint{2.373495in}{2.691667in}}{\pgfqpoint{2.373495in}{2.680617in}}%
\pgfpathcurveto{\pgfqpoint{2.373495in}{2.669567in}}{\pgfqpoint{2.377885in}{2.658968in}}{\pgfqpoint{2.385699in}{2.651154in}}%
\pgfpathcurveto{\pgfqpoint{2.393513in}{2.643340in}}{\pgfqpoint{2.404112in}{2.638950in}}{\pgfqpoint{2.415162in}{2.638950in}}%
\pgfpathclose%
\pgfusepath{stroke,fill}%
\end{pgfscope}%
\begin{pgfscope}%
\pgfpathrectangle{\pgfqpoint{0.600000in}{0.600000in}}{\pgfqpoint{3.900000in}{3.900000in}}%
\pgfusepath{clip}%
\pgfsetbuttcap%
\pgfsetroundjoin%
\definecolor{currentfill}{rgb}{0.121569,0.466667,0.705882}%
\pgfsetfillcolor{currentfill}%
\pgfsetlinewidth{1.003750pt}%
\definecolor{currentstroke}{rgb}{0.121569,0.466667,0.705882}%
\pgfsetstrokecolor{currentstroke}%
\pgfsetdash{}{0pt}%
\pgfpathmoveto{\pgfqpoint{2.482389in}{2.171571in}}%
\pgfpathcurveto{\pgfqpoint{2.493439in}{2.171571in}}{\pgfqpoint{2.504038in}{2.175961in}}{\pgfqpoint{2.511852in}{2.183775in}}%
\pgfpathcurveto{\pgfqpoint{2.519665in}{2.191589in}}{\pgfqpoint{2.524056in}{2.202188in}}{\pgfqpoint{2.524056in}{2.213238in}}%
\pgfpathcurveto{\pgfqpoint{2.524056in}{2.224288in}}{\pgfqpoint{2.519665in}{2.234887in}}{\pgfqpoint{2.511852in}{2.242701in}}%
\pgfpathcurveto{\pgfqpoint{2.504038in}{2.250514in}}{\pgfqpoint{2.493439in}{2.254904in}}{\pgfqpoint{2.482389in}{2.254904in}}%
\pgfpathcurveto{\pgfqpoint{2.471339in}{2.254904in}}{\pgfqpoint{2.460740in}{2.250514in}}{\pgfqpoint{2.452926in}{2.242701in}}%
\pgfpathcurveto{\pgfqpoint{2.445113in}{2.234887in}}{\pgfqpoint{2.440722in}{2.224288in}}{\pgfqpoint{2.440722in}{2.213238in}}%
\pgfpathcurveto{\pgfqpoint{2.440722in}{2.202188in}}{\pgfqpoint{2.445113in}{2.191589in}}{\pgfqpoint{2.452926in}{2.183775in}}%
\pgfpathcurveto{\pgfqpoint{2.460740in}{2.175961in}}{\pgfqpoint{2.471339in}{2.171571in}}{\pgfqpoint{2.482389in}{2.171571in}}%
\pgfpathclose%
\pgfusepath{stroke,fill}%
\end{pgfscope}%
\begin{pgfscope}%
\pgfpathrectangle{\pgfqpoint{0.600000in}{0.600000in}}{\pgfqpoint{3.900000in}{3.900000in}}%
\pgfusepath{clip}%
\pgfsetbuttcap%
\pgfsetroundjoin%
\definecolor{currentfill}{rgb}{0.121569,0.466667,0.705882}%
\pgfsetfillcolor{currentfill}%
\pgfsetlinewidth{1.003750pt}%
\definecolor{currentstroke}{rgb}{0.121569,0.466667,0.705882}%
\pgfsetstrokecolor{currentstroke}%
\pgfsetdash{}{0pt}%
\pgfpathmoveto{\pgfqpoint{3.487204in}{2.250570in}}%
\pgfpathcurveto{\pgfqpoint{3.498254in}{2.250570in}}{\pgfqpoint{3.508853in}{2.254960in}}{\pgfqpoint{3.516667in}{2.262774in}}%
\pgfpathcurveto{\pgfqpoint{3.524481in}{2.270587in}}{\pgfqpoint{3.528871in}{2.281186in}}{\pgfqpoint{3.528871in}{2.292236in}}%
\pgfpathcurveto{\pgfqpoint{3.528871in}{2.303287in}}{\pgfqpoint{3.524481in}{2.313886in}}{\pgfqpoint{3.516667in}{2.321699in}}%
\pgfpathcurveto{\pgfqpoint{3.508853in}{2.329513in}}{\pgfqpoint{3.498254in}{2.333903in}}{\pgfqpoint{3.487204in}{2.333903in}}%
\pgfpathcurveto{\pgfqpoint{3.476154in}{2.333903in}}{\pgfqpoint{3.465555in}{2.329513in}}{\pgfqpoint{3.457742in}{2.321699in}}%
\pgfpathcurveto{\pgfqpoint{3.449928in}{2.313886in}}{\pgfqpoint{3.445538in}{2.303287in}}{\pgfqpoint{3.445538in}{2.292236in}}%
\pgfpathcurveto{\pgfqpoint{3.445538in}{2.281186in}}{\pgfqpoint{3.449928in}{2.270587in}}{\pgfqpoint{3.457742in}{2.262774in}}%
\pgfpathcurveto{\pgfqpoint{3.465555in}{2.254960in}}{\pgfqpoint{3.476154in}{2.250570in}}{\pgfqpoint{3.487204in}{2.250570in}}%
\pgfpathclose%
\pgfusepath{stroke,fill}%
\end{pgfscope}%
\begin{pgfscope}%
\pgfpathrectangle{\pgfqpoint{0.600000in}{0.600000in}}{\pgfqpoint{3.900000in}{3.900000in}}%
\pgfusepath{clip}%
\pgfsetbuttcap%
\pgfsetroundjoin%
\definecolor{currentfill}{rgb}{0.121569,0.466667,0.705882}%
\pgfsetfillcolor{currentfill}%
\pgfsetlinewidth{1.003750pt}%
\definecolor{currentstroke}{rgb}{0.121569,0.466667,0.705882}%
\pgfsetstrokecolor{currentstroke}%
\pgfsetdash{}{0pt}%
\pgfpathmoveto{\pgfqpoint{2.234737in}{2.110513in}}%
\pgfpathcurveto{\pgfqpoint{2.245787in}{2.110513in}}{\pgfqpoint{2.256386in}{2.114904in}}{\pgfqpoint{2.264200in}{2.122717in}}%
\pgfpathcurveto{\pgfqpoint{2.272013in}{2.130531in}}{\pgfqpoint{2.276403in}{2.141130in}}{\pgfqpoint{2.276403in}{2.152180in}}%
\pgfpathcurveto{\pgfqpoint{2.276403in}{2.163230in}}{\pgfqpoint{2.272013in}{2.173829in}}{\pgfqpoint{2.264200in}{2.181643in}}%
\pgfpathcurveto{\pgfqpoint{2.256386in}{2.189456in}}{\pgfqpoint{2.245787in}{2.193847in}}{\pgfqpoint{2.234737in}{2.193847in}}%
\pgfpathcurveto{\pgfqpoint{2.223687in}{2.193847in}}{\pgfqpoint{2.213088in}{2.189456in}}{\pgfqpoint{2.205274in}{2.181643in}}%
\pgfpathcurveto{\pgfqpoint{2.197460in}{2.173829in}}{\pgfqpoint{2.193070in}{2.163230in}}{\pgfqpoint{2.193070in}{2.152180in}}%
\pgfpathcurveto{\pgfqpoint{2.193070in}{2.141130in}}{\pgfqpoint{2.197460in}{2.130531in}}{\pgfqpoint{2.205274in}{2.122717in}}%
\pgfpathcurveto{\pgfqpoint{2.213088in}{2.114904in}}{\pgfqpoint{2.223687in}{2.110513in}}{\pgfqpoint{2.234737in}{2.110513in}}%
\pgfpathclose%
\pgfusepath{stroke,fill}%
\end{pgfscope}%
\begin{pgfscope}%
\pgfpathrectangle{\pgfqpoint{0.600000in}{0.600000in}}{\pgfqpoint{3.900000in}{3.900000in}}%
\pgfusepath{clip}%
\pgfsetbuttcap%
\pgfsetroundjoin%
\definecolor{currentfill}{rgb}{0.121569,0.466667,0.705882}%
\pgfsetfillcolor{currentfill}%
\pgfsetlinewidth{1.003750pt}%
\definecolor{currentstroke}{rgb}{0.121569,0.466667,0.705882}%
\pgfsetstrokecolor{currentstroke}%
\pgfsetdash{}{0pt}%
\pgfpathmoveto{\pgfqpoint{2.118960in}{3.148794in}}%
\pgfpathcurveto{\pgfqpoint{2.130010in}{3.148794in}}{\pgfqpoint{2.140609in}{3.153184in}}{\pgfqpoint{2.148423in}{3.160998in}}%
\pgfpathcurveto{\pgfqpoint{2.156237in}{3.168811in}}{\pgfqpoint{2.160627in}{3.179410in}}{\pgfqpoint{2.160627in}{3.190460in}}%
\pgfpathcurveto{\pgfqpoint{2.160627in}{3.201511in}}{\pgfqpoint{2.156237in}{3.212110in}}{\pgfqpoint{2.148423in}{3.219923in}}%
\pgfpathcurveto{\pgfqpoint{2.140609in}{3.227737in}}{\pgfqpoint{2.130010in}{3.232127in}}{\pgfqpoint{2.118960in}{3.232127in}}%
\pgfpathcurveto{\pgfqpoint{2.107910in}{3.232127in}}{\pgfqpoint{2.097311in}{3.227737in}}{\pgfqpoint{2.089497in}{3.219923in}}%
\pgfpathcurveto{\pgfqpoint{2.081684in}{3.212110in}}{\pgfqpoint{2.077294in}{3.201511in}}{\pgfqpoint{2.077294in}{3.190460in}}%
\pgfpathcurveto{\pgfqpoint{2.077294in}{3.179410in}}{\pgfqpoint{2.081684in}{3.168811in}}{\pgfqpoint{2.089497in}{3.160998in}}%
\pgfpathcurveto{\pgfqpoint{2.097311in}{3.153184in}}{\pgfqpoint{2.107910in}{3.148794in}}{\pgfqpoint{2.118960in}{3.148794in}}%
\pgfpathclose%
\pgfusepath{stroke,fill}%
\end{pgfscope}%
\begin{pgfscope}%
\pgfpathrectangle{\pgfqpoint{0.600000in}{0.600000in}}{\pgfqpoint{3.900000in}{3.900000in}}%
\pgfusepath{clip}%
\pgfsetbuttcap%
\pgfsetroundjoin%
\definecolor{currentfill}{rgb}{0.121569,0.466667,0.705882}%
\pgfsetfillcolor{currentfill}%
\pgfsetlinewidth{1.003750pt}%
\definecolor{currentstroke}{rgb}{0.121569,0.466667,0.705882}%
\pgfsetstrokecolor{currentstroke}%
\pgfsetdash{}{0pt}%
\pgfpathmoveto{\pgfqpoint{2.334550in}{3.001392in}}%
\pgfpathcurveto{\pgfqpoint{2.345600in}{3.001392in}}{\pgfqpoint{2.356199in}{3.005783in}}{\pgfqpoint{2.364013in}{3.013596in}}%
\pgfpathcurveto{\pgfqpoint{2.371826in}{3.021410in}}{\pgfqpoint{2.376217in}{3.032009in}}{\pgfqpoint{2.376217in}{3.043059in}}%
\pgfpathcurveto{\pgfqpoint{2.376217in}{3.054109in}}{\pgfqpoint{2.371826in}{3.064708in}}{\pgfqpoint{2.364013in}{3.072522in}}%
\pgfpathcurveto{\pgfqpoint{2.356199in}{3.080335in}}{\pgfqpoint{2.345600in}{3.084726in}}{\pgfqpoint{2.334550in}{3.084726in}}%
\pgfpathcurveto{\pgfqpoint{2.323500in}{3.084726in}}{\pgfqpoint{2.312901in}{3.080335in}}{\pgfqpoint{2.305087in}{3.072522in}}%
\pgfpathcurveto{\pgfqpoint{2.297274in}{3.064708in}}{\pgfqpoint{2.292883in}{3.054109in}}{\pgfqpoint{2.292883in}{3.043059in}}%
\pgfpathcurveto{\pgfqpoint{2.292883in}{3.032009in}}{\pgfqpoint{2.297274in}{3.021410in}}{\pgfqpoint{2.305087in}{3.013596in}}%
\pgfpathcurveto{\pgfqpoint{2.312901in}{3.005783in}}{\pgfqpoint{2.323500in}{3.001392in}}{\pgfqpoint{2.334550in}{3.001392in}}%
\pgfpathclose%
\pgfusepath{stroke,fill}%
\end{pgfscope}%
\begin{pgfscope}%
\pgfpathrectangle{\pgfqpoint{0.600000in}{0.600000in}}{\pgfqpoint{3.900000in}{3.900000in}}%
\pgfusepath{clip}%
\pgfsetbuttcap%
\pgfsetroundjoin%
\definecolor{currentfill}{rgb}{0.121569,0.466667,0.705882}%
\pgfsetfillcolor{currentfill}%
\pgfsetlinewidth{1.003750pt}%
\definecolor{currentstroke}{rgb}{0.121569,0.466667,0.705882}%
\pgfsetstrokecolor{currentstroke}%
\pgfsetdash{}{0pt}%
\pgfpathmoveto{\pgfqpoint{1.814522in}{2.878544in}}%
\pgfpathcurveto{\pgfqpoint{1.825572in}{2.878544in}}{\pgfqpoint{1.836171in}{2.882934in}}{\pgfqpoint{1.843985in}{2.890748in}}%
\pgfpathcurveto{\pgfqpoint{1.851798in}{2.898562in}}{\pgfqpoint{1.856188in}{2.909161in}}{\pgfqpoint{1.856188in}{2.920211in}}%
\pgfpathcurveto{\pgfqpoint{1.856188in}{2.931261in}}{\pgfqpoint{1.851798in}{2.941860in}}{\pgfqpoint{1.843985in}{2.949674in}}%
\pgfpathcurveto{\pgfqpoint{1.836171in}{2.957487in}}{\pgfqpoint{1.825572in}{2.961877in}}{\pgfqpoint{1.814522in}{2.961877in}}%
\pgfpathcurveto{\pgfqpoint{1.803472in}{2.961877in}}{\pgfqpoint{1.792873in}{2.957487in}}{\pgfqpoint{1.785059in}{2.949674in}}%
\pgfpathcurveto{\pgfqpoint{1.777245in}{2.941860in}}{\pgfqpoint{1.772855in}{2.931261in}}{\pgfqpoint{1.772855in}{2.920211in}}%
\pgfpathcurveto{\pgfqpoint{1.772855in}{2.909161in}}{\pgfqpoint{1.777245in}{2.898562in}}{\pgfqpoint{1.785059in}{2.890748in}}%
\pgfpathcurveto{\pgfqpoint{1.792873in}{2.882934in}}{\pgfqpoint{1.803472in}{2.878544in}}{\pgfqpoint{1.814522in}{2.878544in}}%
\pgfpathclose%
\pgfusepath{stroke,fill}%
\end{pgfscope}%
\begin{pgfscope}%
\pgfpathrectangle{\pgfqpoint{0.600000in}{0.600000in}}{\pgfqpoint{3.900000in}{3.900000in}}%
\pgfusepath{clip}%
\pgfsetbuttcap%
\pgfsetroundjoin%
\definecolor{currentfill}{rgb}{0.121569,0.466667,0.705882}%
\pgfsetfillcolor{currentfill}%
\pgfsetlinewidth{1.003750pt}%
\definecolor{currentstroke}{rgb}{0.121569,0.466667,0.705882}%
\pgfsetstrokecolor{currentstroke}%
\pgfsetdash{}{0pt}%
\pgfpathmoveto{\pgfqpoint{2.262481in}{2.679662in}}%
\pgfpathcurveto{\pgfqpoint{2.273531in}{2.679662in}}{\pgfqpoint{2.284130in}{2.684052in}}{\pgfqpoint{2.291944in}{2.691866in}}%
\pgfpathcurveto{\pgfqpoint{2.299757in}{2.699679in}}{\pgfqpoint{2.304148in}{2.710278in}}{\pgfqpoint{2.304148in}{2.721329in}}%
\pgfpathcurveto{\pgfqpoint{2.304148in}{2.732379in}}{\pgfqpoint{2.299757in}{2.742978in}}{\pgfqpoint{2.291944in}{2.750791in}}%
\pgfpathcurveto{\pgfqpoint{2.284130in}{2.758605in}}{\pgfqpoint{2.273531in}{2.762995in}}{\pgfqpoint{2.262481in}{2.762995in}}%
\pgfpathcurveto{\pgfqpoint{2.251431in}{2.762995in}}{\pgfqpoint{2.240832in}{2.758605in}}{\pgfqpoint{2.233018in}{2.750791in}}%
\pgfpathcurveto{\pgfqpoint{2.225205in}{2.742978in}}{\pgfqpoint{2.220814in}{2.732379in}}{\pgfqpoint{2.220814in}{2.721329in}}%
\pgfpathcurveto{\pgfqpoint{2.220814in}{2.710278in}}{\pgfqpoint{2.225205in}{2.699679in}}{\pgfqpoint{2.233018in}{2.691866in}}%
\pgfpathcurveto{\pgfqpoint{2.240832in}{2.684052in}}{\pgfqpoint{2.251431in}{2.679662in}}{\pgfqpoint{2.262481in}{2.679662in}}%
\pgfpathclose%
\pgfusepath{stroke,fill}%
\end{pgfscope}%
\begin{pgfscope}%
\pgfpathrectangle{\pgfqpoint{0.600000in}{0.600000in}}{\pgfqpoint{3.900000in}{3.900000in}}%
\pgfusepath{clip}%
\pgfsetbuttcap%
\pgfsetroundjoin%
\definecolor{currentfill}{rgb}{0.121569,0.466667,0.705882}%
\pgfsetfillcolor{currentfill}%
\pgfsetlinewidth{1.003750pt}%
\definecolor{currentstroke}{rgb}{0.121569,0.466667,0.705882}%
\pgfsetstrokecolor{currentstroke}%
\pgfsetdash{}{0pt}%
\pgfpathmoveto{\pgfqpoint{2.582012in}{1.856708in}}%
\pgfpathcurveto{\pgfqpoint{2.593062in}{1.856708in}}{\pgfqpoint{2.603661in}{1.861099in}}{\pgfqpoint{2.611475in}{1.868912in}}%
\pgfpathcurveto{\pgfqpoint{2.619288in}{1.876726in}}{\pgfqpoint{2.623679in}{1.887325in}}{\pgfqpoint{2.623679in}{1.898375in}}%
\pgfpathcurveto{\pgfqpoint{2.623679in}{1.909425in}}{\pgfqpoint{2.619288in}{1.920024in}}{\pgfqpoint{2.611475in}{1.927838in}}%
\pgfpathcurveto{\pgfqpoint{2.603661in}{1.935651in}}{\pgfqpoint{2.593062in}{1.940042in}}{\pgfqpoint{2.582012in}{1.940042in}}%
\pgfpathcurveto{\pgfqpoint{2.570962in}{1.940042in}}{\pgfqpoint{2.560363in}{1.935651in}}{\pgfqpoint{2.552549in}{1.927838in}}%
\pgfpathcurveto{\pgfqpoint{2.544736in}{1.920024in}}{\pgfqpoint{2.540345in}{1.909425in}}{\pgfqpoint{2.540345in}{1.898375in}}%
\pgfpathcurveto{\pgfqpoint{2.540345in}{1.887325in}}{\pgfqpoint{2.544736in}{1.876726in}}{\pgfqpoint{2.552549in}{1.868912in}}%
\pgfpathcurveto{\pgfqpoint{2.560363in}{1.861099in}}{\pgfqpoint{2.570962in}{1.856708in}}{\pgfqpoint{2.582012in}{1.856708in}}%
\pgfpathclose%
\pgfusepath{stroke,fill}%
\end{pgfscope}%
\begin{pgfscope}%
\pgfpathrectangle{\pgfqpoint{0.600000in}{0.600000in}}{\pgfqpoint{3.900000in}{3.900000in}}%
\pgfusepath{clip}%
\pgfsetbuttcap%
\pgfsetroundjoin%
\definecolor{currentfill}{rgb}{0.121569,0.466667,0.705882}%
\pgfsetfillcolor{currentfill}%
\pgfsetlinewidth{1.003750pt}%
\definecolor{currentstroke}{rgb}{0.121569,0.466667,0.705882}%
\pgfsetstrokecolor{currentstroke}%
\pgfsetdash{}{0pt}%
\pgfpathmoveto{\pgfqpoint{3.246809in}{1.868561in}}%
\pgfpathcurveto{\pgfqpoint{3.257859in}{1.868561in}}{\pgfqpoint{3.268458in}{1.872951in}}{\pgfqpoint{3.276271in}{1.880765in}}%
\pgfpathcurveto{\pgfqpoint{3.284085in}{1.888578in}}{\pgfqpoint{3.288475in}{1.899177in}}{\pgfqpoint{3.288475in}{1.910227in}}%
\pgfpathcurveto{\pgfqpoint{3.288475in}{1.921277in}}{\pgfqpoint{3.284085in}{1.931877in}}{\pgfqpoint{3.276271in}{1.939690in}}%
\pgfpathcurveto{\pgfqpoint{3.268458in}{1.947504in}}{\pgfqpoint{3.257859in}{1.951894in}}{\pgfqpoint{3.246809in}{1.951894in}}%
\pgfpathcurveto{\pgfqpoint{3.235759in}{1.951894in}}{\pgfqpoint{3.225159in}{1.947504in}}{\pgfqpoint{3.217346in}{1.939690in}}%
\pgfpathcurveto{\pgfqpoint{3.209532in}{1.931877in}}{\pgfqpoint{3.205142in}{1.921277in}}{\pgfqpoint{3.205142in}{1.910227in}}%
\pgfpathcurveto{\pgfqpoint{3.205142in}{1.899177in}}{\pgfqpoint{3.209532in}{1.888578in}}{\pgfqpoint{3.217346in}{1.880765in}}%
\pgfpathcurveto{\pgfqpoint{3.225159in}{1.872951in}}{\pgfqpoint{3.235759in}{1.868561in}}{\pgfqpoint{3.246809in}{1.868561in}}%
\pgfpathclose%
\pgfusepath{stroke,fill}%
\end{pgfscope}%
\begin{pgfscope}%
\pgfpathrectangle{\pgfqpoint{0.600000in}{0.600000in}}{\pgfqpoint{3.900000in}{3.900000in}}%
\pgfusepath{clip}%
\pgfsetbuttcap%
\pgfsetroundjoin%
\definecolor{currentfill}{rgb}{0.121569,0.466667,0.705882}%
\pgfsetfillcolor{currentfill}%
\pgfsetlinewidth{1.003750pt}%
\definecolor{currentstroke}{rgb}{0.121569,0.466667,0.705882}%
\pgfsetstrokecolor{currentstroke}%
\pgfsetdash{}{0pt}%
\pgfpathmoveto{\pgfqpoint{1.505964in}{3.786409in}}%
\pgfpathcurveto{\pgfqpoint{1.517014in}{3.786409in}}{\pgfqpoint{1.527613in}{3.790799in}}{\pgfqpoint{1.535427in}{3.798613in}}%
\pgfpathcurveto{\pgfqpoint{1.543240in}{3.806427in}}{\pgfqpoint{1.547631in}{3.817026in}}{\pgfqpoint{1.547631in}{3.828076in}}%
\pgfpathcurveto{\pgfqpoint{1.547631in}{3.839126in}}{\pgfqpoint{1.543240in}{3.849725in}}{\pgfqpoint{1.535427in}{3.857539in}}%
\pgfpathcurveto{\pgfqpoint{1.527613in}{3.865352in}}{\pgfqpoint{1.517014in}{3.869742in}}{\pgfqpoint{1.505964in}{3.869742in}}%
\pgfpathcurveto{\pgfqpoint{1.494914in}{3.869742in}}{\pgfqpoint{1.484315in}{3.865352in}}{\pgfqpoint{1.476501in}{3.857539in}}%
\pgfpathcurveto{\pgfqpoint{1.468688in}{3.849725in}}{\pgfqpoint{1.464297in}{3.839126in}}{\pgfqpoint{1.464297in}{3.828076in}}%
\pgfpathcurveto{\pgfqpoint{1.464297in}{3.817026in}}{\pgfqpoint{1.468688in}{3.806427in}}{\pgfqpoint{1.476501in}{3.798613in}}%
\pgfpathcurveto{\pgfqpoint{1.484315in}{3.790799in}}{\pgfqpoint{1.494914in}{3.786409in}}{\pgfqpoint{1.505964in}{3.786409in}}%
\pgfpathclose%
\pgfusepath{stroke,fill}%
\end{pgfscope}%
\begin{pgfscope}%
\pgfpathrectangle{\pgfqpoint{0.600000in}{0.600000in}}{\pgfqpoint{3.900000in}{3.900000in}}%
\pgfusepath{clip}%
\pgfsetbuttcap%
\pgfsetroundjoin%
\definecolor{currentfill}{rgb}{0.121569,0.466667,0.705882}%
\pgfsetfillcolor{currentfill}%
\pgfsetlinewidth{1.003750pt}%
\definecolor{currentstroke}{rgb}{0.121569,0.466667,0.705882}%
\pgfsetstrokecolor{currentstroke}%
\pgfsetdash{}{0pt}%
\pgfpathmoveto{\pgfqpoint{2.257965in}{2.957945in}}%
\pgfpathcurveto{\pgfqpoint{2.269016in}{2.957945in}}{\pgfqpoint{2.279615in}{2.962336in}}{\pgfqpoint{2.287428in}{2.970149in}}%
\pgfpathcurveto{\pgfqpoint{2.295242in}{2.977963in}}{\pgfqpoint{2.299632in}{2.988562in}}{\pgfqpoint{2.299632in}{2.999612in}}%
\pgfpathcurveto{\pgfqpoint{2.299632in}{3.010662in}}{\pgfqpoint{2.295242in}{3.021261in}}{\pgfqpoint{2.287428in}{3.029075in}}%
\pgfpathcurveto{\pgfqpoint{2.279615in}{3.036888in}}{\pgfqpoint{2.269016in}{3.041279in}}{\pgfqpoint{2.257965in}{3.041279in}}%
\pgfpathcurveto{\pgfqpoint{2.246915in}{3.041279in}}{\pgfqpoint{2.236316in}{3.036888in}}{\pgfqpoint{2.228503in}{3.029075in}}%
\pgfpathcurveto{\pgfqpoint{2.220689in}{3.021261in}}{\pgfqpoint{2.216299in}{3.010662in}}{\pgfqpoint{2.216299in}{2.999612in}}%
\pgfpathcurveto{\pgfqpoint{2.216299in}{2.988562in}}{\pgfqpoint{2.220689in}{2.977963in}}{\pgfqpoint{2.228503in}{2.970149in}}%
\pgfpathcurveto{\pgfqpoint{2.236316in}{2.962336in}}{\pgfqpoint{2.246915in}{2.957945in}}{\pgfqpoint{2.257965in}{2.957945in}}%
\pgfpathclose%
\pgfusepath{stroke,fill}%
\end{pgfscope}%
\begin{pgfscope}%
\pgfpathrectangle{\pgfqpoint{0.600000in}{0.600000in}}{\pgfqpoint{3.900000in}{3.900000in}}%
\pgfusepath{clip}%
\pgfsetbuttcap%
\pgfsetroundjoin%
\definecolor{currentfill}{rgb}{0.121569,0.466667,0.705882}%
\pgfsetfillcolor{currentfill}%
\pgfsetlinewidth{1.003750pt}%
\definecolor{currentstroke}{rgb}{0.121569,0.466667,0.705882}%
\pgfsetstrokecolor{currentstroke}%
\pgfsetdash{}{0pt}%
\pgfpathmoveto{\pgfqpoint{2.321193in}{2.061145in}}%
\pgfpathcurveto{\pgfqpoint{2.332243in}{2.061145in}}{\pgfqpoint{2.342842in}{2.065535in}}{\pgfqpoint{2.350655in}{2.073349in}}%
\pgfpathcurveto{\pgfqpoint{2.358469in}{2.081162in}}{\pgfqpoint{2.362859in}{2.091761in}}{\pgfqpoint{2.362859in}{2.102811in}}%
\pgfpathcurveto{\pgfqpoint{2.362859in}{2.113861in}}{\pgfqpoint{2.358469in}{2.124460in}}{\pgfqpoint{2.350655in}{2.132274in}}%
\pgfpathcurveto{\pgfqpoint{2.342842in}{2.140088in}}{\pgfqpoint{2.332243in}{2.144478in}}{\pgfqpoint{2.321193in}{2.144478in}}%
\pgfpathcurveto{\pgfqpoint{2.310143in}{2.144478in}}{\pgfqpoint{2.299544in}{2.140088in}}{\pgfqpoint{2.291730in}{2.132274in}}%
\pgfpathcurveto{\pgfqpoint{2.283916in}{2.124460in}}{\pgfqpoint{2.279526in}{2.113861in}}{\pgfqpoint{2.279526in}{2.102811in}}%
\pgfpathcurveto{\pgfqpoint{2.279526in}{2.091761in}}{\pgfqpoint{2.283916in}{2.081162in}}{\pgfqpoint{2.291730in}{2.073349in}}%
\pgfpathcurveto{\pgfqpoint{2.299544in}{2.065535in}}{\pgfqpoint{2.310143in}{2.061145in}}{\pgfqpoint{2.321193in}{2.061145in}}%
\pgfpathclose%
\pgfusepath{stroke,fill}%
\end{pgfscope}%
\begin{pgfscope}%
\pgfpathrectangle{\pgfqpoint{0.600000in}{0.600000in}}{\pgfqpoint{3.900000in}{3.900000in}}%
\pgfusepath{clip}%
\pgfsetbuttcap%
\pgfsetroundjoin%
\definecolor{currentfill}{rgb}{0.121569,0.466667,0.705882}%
\pgfsetfillcolor{currentfill}%
\pgfsetlinewidth{1.003750pt}%
\definecolor{currentstroke}{rgb}{0.121569,0.466667,0.705882}%
\pgfsetstrokecolor{currentstroke}%
\pgfsetdash{}{0pt}%
\pgfpathmoveto{\pgfqpoint{3.024836in}{3.084634in}}%
\pgfpathcurveto{\pgfqpoint{3.035886in}{3.084634in}}{\pgfqpoint{3.046485in}{3.089024in}}{\pgfqpoint{3.054299in}{3.096838in}}%
\pgfpathcurveto{\pgfqpoint{3.062112in}{3.104652in}}{\pgfqpoint{3.066503in}{3.115251in}}{\pgfqpoint{3.066503in}{3.126301in}}%
\pgfpathcurveto{\pgfqpoint{3.066503in}{3.137351in}}{\pgfqpoint{3.062112in}{3.147950in}}{\pgfqpoint{3.054299in}{3.155763in}}%
\pgfpathcurveto{\pgfqpoint{3.046485in}{3.163577in}}{\pgfqpoint{3.035886in}{3.167967in}}{\pgfqpoint{3.024836in}{3.167967in}}%
\pgfpathcurveto{\pgfqpoint{3.013786in}{3.167967in}}{\pgfqpoint{3.003187in}{3.163577in}}{\pgfqpoint{2.995373in}{3.155763in}}%
\pgfpathcurveto{\pgfqpoint{2.987559in}{3.147950in}}{\pgfqpoint{2.983169in}{3.137351in}}{\pgfqpoint{2.983169in}{3.126301in}}%
\pgfpathcurveto{\pgfqpoint{2.983169in}{3.115251in}}{\pgfqpoint{2.987559in}{3.104652in}}{\pgfqpoint{2.995373in}{3.096838in}}%
\pgfpathcurveto{\pgfqpoint{3.003187in}{3.089024in}}{\pgfqpoint{3.013786in}{3.084634in}}{\pgfqpoint{3.024836in}{3.084634in}}%
\pgfpathclose%
\pgfusepath{stroke,fill}%
\end{pgfscope}%
\begin{pgfscope}%
\pgfpathrectangle{\pgfqpoint{0.600000in}{0.600000in}}{\pgfqpoint{3.900000in}{3.900000in}}%
\pgfusepath{clip}%
\pgfsetbuttcap%
\pgfsetroundjoin%
\definecolor{currentfill}{rgb}{0.121569,0.466667,0.705882}%
\pgfsetfillcolor{currentfill}%
\pgfsetlinewidth{1.003750pt}%
\definecolor{currentstroke}{rgb}{0.121569,0.466667,0.705882}%
\pgfsetstrokecolor{currentstroke}%
\pgfsetdash{}{0pt}%
\pgfpathmoveto{\pgfqpoint{2.446062in}{2.568517in}}%
\pgfpathcurveto{\pgfqpoint{2.457112in}{2.568517in}}{\pgfqpoint{2.467712in}{2.572908in}}{\pgfqpoint{2.475525in}{2.580721in}}%
\pgfpathcurveto{\pgfqpoint{2.483339in}{2.588535in}}{\pgfqpoint{2.487729in}{2.599134in}}{\pgfqpoint{2.487729in}{2.610184in}}%
\pgfpathcurveto{\pgfqpoint{2.487729in}{2.621234in}}{\pgfqpoint{2.483339in}{2.631833in}}{\pgfqpoint{2.475525in}{2.639647in}}%
\pgfpathcurveto{\pgfqpoint{2.467712in}{2.647460in}}{\pgfqpoint{2.457112in}{2.651851in}}{\pgfqpoint{2.446062in}{2.651851in}}%
\pgfpathcurveto{\pgfqpoint{2.435012in}{2.651851in}}{\pgfqpoint{2.424413in}{2.647460in}}{\pgfqpoint{2.416600in}{2.639647in}}%
\pgfpathcurveto{\pgfqpoint{2.408786in}{2.631833in}}{\pgfqpoint{2.404396in}{2.621234in}}{\pgfqpoint{2.404396in}{2.610184in}}%
\pgfpathcurveto{\pgfqpoint{2.404396in}{2.599134in}}{\pgfqpoint{2.408786in}{2.588535in}}{\pgfqpoint{2.416600in}{2.580721in}}%
\pgfpathcurveto{\pgfqpoint{2.424413in}{2.572908in}}{\pgfqpoint{2.435012in}{2.568517in}}{\pgfqpoint{2.446062in}{2.568517in}}%
\pgfpathclose%
\pgfusepath{stroke,fill}%
\end{pgfscope}%
\begin{pgfscope}%
\pgfpathrectangle{\pgfqpoint{0.600000in}{0.600000in}}{\pgfqpoint{3.900000in}{3.900000in}}%
\pgfusepath{clip}%
\pgfsetbuttcap%
\pgfsetroundjoin%
\definecolor{currentfill}{rgb}{0.121569,0.466667,0.705882}%
\pgfsetfillcolor{currentfill}%
\pgfsetlinewidth{1.003750pt}%
\definecolor{currentstroke}{rgb}{0.121569,0.466667,0.705882}%
\pgfsetstrokecolor{currentstroke}%
\pgfsetdash{}{0pt}%
\pgfpathmoveto{\pgfqpoint{2.518153in}{2.648264in}}%
\pgfpathcurveto{\pgfqpoint{2.529203in}{2.648264in}}{\pgfqpoint{2.539802in}{2.652654in}}{\pgfqpoint{2.547616in}{2.660468in}}%
\pgfpathcurveto{\pgfqpoint{2.555429in}{2.668281in}}{\pgfqpoint{2.559820in}{2.678880in}}{\pgfqpoint{2.559820in}{2.689931in}}%
\pgfpathcurveto{\pgfqpoint{2.559820in}{2.700981in}}{\pgfqpoint{2.555429in}{2.711580in}}{\pgfqpoint{2.547616in}{2.719393in}}%
\pgfpathcurveto{\pgfqpoint{2.539802in}{2.727207in}}{\pgfqpoint{2.529203in}{2.731597in}}{\pgfqpoint{2.518153in}{2.731597in}}%
\pgfpathcurveto{\pgfqpoint{2.507103in}{2.731597in}}{\pgfqpoint{2.496504in}{2.727207in}}{\pgfqpoint{2.488690in}{2.719393in}}%
\pgfpathcurveto{\pgfqpoint{2.480877in}{2.711580in}}{\pgfqpoint{2.476486in}{2.700981in}}{\pgfqpoint{2.476486in}{2.689931in}}%
\pgfpathcurveto{\pgfqpoint{2.476486in}{2.678880in}}{\pgfqpoint{2.480877in}{2.668281in}}{\pgfqpoint{2.488690in}{2.660468in}}%
\pgfpathcurveto{\pgfqpoint{2.496504in}{2.652654in}}{\pgfqpoint{2.507103in}{2.648264in}}{\pgfqpoint{2.518153in}{2.648264in}}%
\pgfpathclose%
\pgfusepath{stroke,fill}%
\end{pgfscope}%
\begin{pgfscope}%
\pgfpathrectangle{\pgfqpoint{0.600000in}{0.600000in}}{\pgfqpoint{3.900000in}{3.900000in}}%
\pgfusepath{clip}%
\pgfsetbuttcap%
\pgfsetroundjoin%
\definecolor{currentfill}{rgb}{0.121569,0.466667,0.705882}%
\pgfsetfillcolor{currentfill}%
\pgfsetlinewidth{1.003750pt}%
\definecolor{currentstroke}{rgb}{0.121569,0.466667,0.705882}%
\pgfsetstrokecolor{currentstroke}%
\pgfsetdash{}{0pt}%
\pgfpathmoveto{\pgfqpoint{2.808753in}{1.226177in}}%
\pgfpathcurveto{\pgfqpoint{2.819804in}{1.226177in}}{\pgfqpoint{2.830403in}{1.230567in}}{\pgfqpoint{2.838216in}{1.238381in}}%
\pgfpathcurveto{\pgfqpoint{2.846030in}{1.246194in}}{\pgfqpoint{2.850420in}{1.256793in}}{\pgfqpoint{2.850420in}{1.267844in}}%
\pgfpathcurveto{\pgfqpoint{2.850420in}{1.278894in}}{\pgfqpoint{2.846030in}{1.289493in}}{\pgfqpoint{2.838216in}{1.297306in}}%
\pgfpathcurveto{\pgfqpoint{2.830403in}{1.305120in}}{\pgfqpoint{2.819804in}{1.309510in}}{\pgfqpoint{2.808753in}{1.309510in}}%
\pgfpathcurveto{\pgfqpoint{2.797703in}{1.309510in}}{\pgfqpoint{2.787104in}{1.305120in}}{\pgfqpoint{2.779291in}{1.297306in}}%
\pgfpathcurveto{\pgfqpoint{2.771477in}{1.289493in}}{\pgfqpoint{2.767087in}{1.278894in}}{\pgfqpoint{2.767087in}{1.267844in}}%
\pgfpathcurveto{\pgfqpoint{2.767087in}{1.256793in}}{\pgfqpoint{2.771477in}{1.246194in}}{\pgfqpoint{2.779291in}{1.238381in}}%
\pgfpathcurveto{\pgfqpoint{2.787104in}{1.230567in}}{\pgfqpoint{2.797703in}{1.226177in}}{\pgfqpoint{2.808753in}{1.226177in}}%
\pgfpathclose%
\pgfusepath{stroke,fill}%
\end{pgfscope}%
\begin{pgfscope}%
\pgfpathrectangle{\pgfqpoint{0.600000in}{0.600000in}}{\pgfqpoint{3.900000in}{3.900000in}}%
\pgfusepath{clip}%
\pgfsetbuttcap%
\pgfsetroundjoin%
\definecolor{currentfill}{rgb}{0.121569,0.466667,0.705882}%
\pgfsetfillcolor{currentfill}%
\pgfsetlinewidth{1.003750pt}%
\definecolor{currentstroke}{rgb}{0.121569,0.466667,0.705882}%
\pgfsetstrokecolor{currentstroke}%
\pgfsetdash{}{0pt}%
\pgfpathmoveto{\pgfqpoint{2.766669in}{0.700898in}}%
\pgfpathcurveto{\pgfqpoint{2.777720in}{0.700898in}}{\pgfqpoint{2.788319in}{0.705289in}}{\pgfqpoint{2.796132in}{0.713102in}}%
\pgfpathcurveto{\pgfqpoint{2.803946in}{0.720916in}}{\pgfqpoint{2.808336in}{0.731515in}}{\pgfqpoint{2.808336in}{0.742565in}}%
\pgfpathcurveto{\pgfqpoint{2.808336in}{0.753615in}}{\pgfqpoint{2.803946in}{0.764214in}}{\pgfqpoint{2.796132in}{0.772028in}}%
\pgfpathcurveto{\pgfqpoint{2.788319in}{0.779841in}}{\pgfqpoint{2.777720in}{0.784232in}}{\pgfqpoint{2.766669in}{0.784232in}}%
\pgfpathcurveto{\pgfqpoint{2.755619in}{0.784232in}}{\pgfqpoint{2.745020in}{0.779841in}}{\pgfqpoint{2.737207in}{0.772028in}}%
\pgfpathcurveto{\pgfqpoint{2.729393in}{0.764214in}}{\pgfqpoint{2.725003in}{0.753615in}}{\pgfqpoint{2.725003in}{0.742565in}}%
\pgfpathcurveto{\pgfqpoint{2.725003in}{0.731515in}}{\pgfqpoint{2.729393in}{0.720916in}}{\pgfqpoint{2.737207in}{0.713102in}}%
\pgfpathcurveto{\pgfqpoint{2.745020in}{0.705289in}}{\pgfqpoint{2.755619in}{0.700898in}}{\pgfqpoint{2.766669in}{0.700898in}}%
\pgfpathclose%
\pgfusepath{stroke,fill}%
\end{pgfscope}%
\begin{pgfscope}%
\pgfpathrectangle{\pgfqpoint{0.600000in}{0.600000in}}{\pgfqpoint{3.900000in}{3.900000in}}%
\pgfusepath{clip}%
\pgfsetbuttcap%
\pgfsetroundjoin%
\definecolor{currentfill}{rgb}{0.121569,0.466667,0.705882}%
\pgfsetfillcolor{currentfill}%
\pgfsetlinewidth{1.003750pt}%
\definecolor{currentstroke}{rgb}{0.121569,0.466667,0.705882}%
\pgfsetstrokecolor{currentstroke}%
\pgfsetdash{}{0pt}%
\pgfpathmoveto{\pgfqpoint{2.744669in}{2.764797in}}%
\pgfpathcurveto{\pgfqpoint{2.755719in}{2.764797in}}{\pgfqpoint{2.766318in}{2.769188in}}{\pgfqpoint{2.774132in}{2.777001in}}%
\pgfpathcurveto{\pgfqpoint{2.781946in}{2.784815in}}{\pgfqpoint{2.786336in}{2.795414in}}{\pgfqpoint{2.786336in}{2.806464in}}%
\pgfpathcurveto{\pgfqpoint{2.786336in}{2.817514in}}{\pgfqpoint{2.781946in}{2.828113in}}{\pgfqpoint{2.774132in}{2.835927in}}%
\pgfpathcurveto{\pgfqpoint{2.766318in}{2.843741in}}{\pgfqpoint{2.755719in}{2.848131in}}{\pgfqpoint{2.744669in}{2.848131in}}%
\pgfpathcurveto{\pgfqpoint{2.733619in}{2.848131in}}{\pgfqpoint{2.723020in}{2.843741in}}{\pgfqpoint{2.715206in}{2.835927in}}%
\pgfpathcurveto{\pgfqpoint{2.707393in}{2.828113in}}{\pgfqpoint{2.703003in}{2.817514in}}{\pgfqpoint{2.703003in}{2.806464in}}%
\pgfpathcurveto{\pgfqpoint{2.703003in}{2.795414in}}{\pgfqpoint{2.707393in}{2.784815in}}{\pgfqpoint{2.715206in}{2.777001in}}%
\pgfpathcurveto{\pgfqpoint{2.723020in}{2.769188in}}{\pgfqpoint{2.733619in}{2.764797in}}{\pgfqpoint{2.744669in}{2.764797in}}%
\pgfpathclose%
\pgfusepath{stroke,fill}%
\end{pgfscope}%
\begin{pgfscope}%
\pgfpathrectangle{\pgfqpoint{0.600000in}{0.600000in}}{\pgfqpoint{3.900000in}{3.900000in}}%
\pgfusepath{clip}%
\pgfsetbuttcap%
\pgfsetroundjoin%
\definecolor{currentfill}{rgb}{0.121569,0.466667,0.705882}%
\pgfsetfillcolor{currentfill}%
\pgfsetlinewidth{1.003750pt}%
\definecolor{currentstroke}{rgb}{0.121569,0.466667,0.705882}%
\pgfsetstrokecolor{currentstroke}%
\pgfsetdash{}{0pt}%
\pgfpathmoveto{\pgfqpoint{2.862212in}{2.931837in}}%
\pgfpathcurveto{\pgfqpoint{2.873262in}{2.931837in}}{\pgfqpoint{2.883861in}{2.936228in}}{\pgfqpoint{2.891675in}{2.944041in}}%
\pgfpathcurveto{\pgfqpoint{2.899488in}{2.951855in}}{\pgfqpoint{2.903878in}{2.962454in}}{\pgfqpoint{2.903878in}{2.973504in}}%
\pgfpathcurveto{\pgfqpoint{2.903878in}{2.984554in}}{\pgfqpoint{2.899488in}{2.995153in}}{\pgfqpoint{2.891675in}{3.002967in}}%
\pgfpathcurveto{\pgfqpoint{2.883861in}{3.010780in}}{\pgfqpoint{2.873262in}{3.015171in}}{\pgfqpoint{2.862212in}{3.015171in}}%
\pgfpathcurveto{\pgfqpoint{2.851162in}{3.015171in}}{\pgfqpoint{2.840563in}{3.010780in}}{\pgfqpoint{2.832749in}{3.002967in}}%
\pgfpathcurveto{\pgfqpoint{2.824935in}{2.995153in}}{\pgfqpoint{2.820545in}{2.984554in}}{\pgfqpoint{2.820545in}{2.973504in}}%
\pgfpathcurveto{\pgfqpoint{2.820545in}{2.962454in}}{\pgfqpoint{2.824935in}{2.951855in}}{\pgfqpoint{2.832749in}{2.944041in}}%
\pgfpathcurveto{\pgfqpoint{2.840563in}{2.936228in}}{\pgfqpoint{2.851162in}{2.931837in}}{\pgfqpoint{2.862212in}{2.931837in}}%
\pgfpathclose%
\pgfusepath{stroke,fill}%
\end{pgfscope}%
\begin{pgfscope}%
\pgfpathrectangle{\pgfqpoint{0.600000in}{0.600000in}}{\pgfqpoint{3.900000in}{3.900000in}}%
\pgfusepath{clip}%
\pgfsetbuttcap%
\pgfsetroundjoin%
\definecolor{currentfill}{rgb}{0.121569,0.466667,0.705882}%
\pgfsetfillcolor{currentfill}%
\pgfsetlinewidth{1.003750pt}%
\definecolor{currentstroke}{rgb}{0.121569,0.466667,0.705882}%
\pgfsetstrokecolor{currentstroke}%
\pgfsetdash{}{0pt}%
\pgfpathmoveto{\pgfqpoint{3.400722in}{1.111478in}}%
\pgfpathcurveto{\pgfqpoint{3.411772in}{1.111478in}}{\pgfqpoint{3.422371in}{1.115868in}}{\pgfqpoint{3.430185in}{1.123682in}}%
\pgfpathcurveto{\pgfqpoint{3.437998in}{1.131496in}}{\pgfqpoint{3.442389in}{1.142095in}}{\pgfqpoint{3.442389in}{1.153145in}}%
\pgfpathcurveto{\pgfqpoint{3.442389in}{1.164195in}}{\pgfqpoint{3.437998in}{1.174794in}}{\pgfqpoint{3.430185in}{1.182608in}}%
\pgfpathcurveto{\pgfqpoint{3.422371in}{1.190421in}}{\pgfqpoint{3.411772in}{1.194811in}}{\pgfqpoint{3.400722in}{1.194811in}}%
\pgfpathcurveto{\pgfqpoint{3.389672in}{1.194811in}}{\pgfqpoint{3.379073in}{1.190421in}}{\pgfqpoint{3.371259in}{1.182608in}}%
\pgfpathcurveto{\pgfqpoint{3.363446in}{1.174794in}}{\pgfqpoint{3.359055in}{1.164195in}}{\pgfqpoint{3.359055in}{1.153145in}}%
\pgfpathcurveto{\pgfqpoint{3.359055in}{1.142095in}}{\pgfqpoint{3.363446in}{1.131496in}}{\pgfqpoint{3.371259in}{1.123682in}}%
\pgfpathcurveto{\pgfqpoint{3.379073in}{1.115868in}}{\pgfqpoint{3.389672in}{1.111478in}}{\pgfqpoint{3.400722in}{1.111478in}}%
\pgfpathclose%
\pgfusepath{stroke,fill}%
\end{pgfscope}%
\begin{pgfscope}%
\pgfpathrectangle{\pgfqpoint{0.600000in}{0.600000in}}{\pgfqpoint{3.900000in}{3.900000in}}%
\pgfusepath{clip}%
\pgfsetbuttcap%
\pgfsetroundjoin%
\definecolor{currentfill}{rgb}{0.121569,0.466667,0.705882}%
\pgfsetfillcolor{currentfill}%
\pgfsetlinewidth{1.003750pt}%
\definecolor{currentstroke}{rgb}{0.121569,0.466667,0.705882}%
\pgfsetstrokecolor{currentstroke}%
\pgfsetdash{}{0pt}%
\pgfpathmoveto{\pgfqpoint{2.256593in}{3.217093in}}%
\pgfpathcurveto{\pgfqpoint{2.267643in}{3.217093in}}{\pgfqpoint{2.278242in}{3.221483in}}{\pgfqpoint{2.286056in}{3.229297in}}%
\pgfpathcurveto{\pgfqpoint{2.293869in}{3.237110in}}{\pgfqpoint{2.298260in}{3.247709in}}{\pgfqpoint{2.298260in}{3.258759in}}%
\pgfpathcurveto{\pgfqpoint{2.298260in}{3.269810in}}{\pgfqpoint{2.293869in}{3.280409in}}{\pgfqpoint{2.286056in}{3.288222in}}%
\pgfpathcurveto{\pgfqpoint{2.278242in}{3.296036in}}{\pgfqpoint{2.267643in}{3.300426in}}{\pgfqpoint{2.256593in}{3.300426in}}%
\pgfpathcurveto{\pgfqpoint{2.245543in}{3.300426in}}{\pgfqpoint{2.234944in}{3.296036in}}{\pgfqpoint{2.227130in}{3.288222in}}%
\pgfpathcurveto{\pgfqpoint{2.219316in}{3.280409in}}{\pgfqpoint{2.214926in}{3.269810in}}{\pgfqpoint{2.214926in}{3.258759in}}%
\pgfpathcurveto{\pgfqpoint{2.214926in}{3.247709in}}{\pgfqpoint{2.219316in}{3.237110in}}{\pgfqpoint{2.227130in}{3.229297in}}%
\pgfpathcurveto{\pgfqpoint{2.234944in}{3.221483in}}{\pgfqpoint{2.245543in}{3.217093in}}{\pgfqpoint{2.256593in}{3.217093in}}%
\pgfpathclose%
\pgfusepath{stroke,fill}%
\end{pgfscope}%
\begin{pgfscope}%
\pgfpathrectangle{\pgfqpoint{0.600000in}{0.600000in}}{\pgfqpoint{3.900000in}{3.900000in}}%
\pgfusepath{clip}%
\pgfsetbuttcap%
\pgfsetroundjoin%
\definecolor{currentfill}{rgb}{0.121569,0.466667,0.705882}%
\pgfsetfillcolor{currentfill}%
\pgfsetlinewidth{1.003750pt}%
\definecolor{currentstroke}{rgb}{0.121569,0.466667,0.705882}%
\pgfsetstrokecolor{currentstroke}%
\pgfsetdash{}{0pt}%
\pgfpathmoveto{\pgfqpoint{2.482460in}{3.088221in}}%
\pgfpathcurveto{\pgfqpoint{2.493511in}{3.088221in}}{\pgfqpoint{2.504110in}{3.092611in}}{\pgfqpoint{2.511923in}{3.100425in}}%
\pgfpathcurveto{\pgfqpoint{2.519737in}{3.108239in}}{\pgfqpoint{2.524127in}{3.118838in}}{\pgfqpoint{2.524127in}{3.129888in}}%
\pgfpathcurveto{\pgfqpoint{2.524127in}{3.140938in}}{\pgfqpoint{2.519737in}{3.151537in}}{\pgfqpoint{2.511923in}{3.159351in}}%
\pgfpathcurveto{\pgfqpoint{2.504110in}{3.167164in}}{\pgfqpoint{2.493511in}{3.171555in}}{\pgfqpoint{2.482460in}{3.171555in}}%
\pgfpathcurveto{\pgfqpoint{2.471410in}{3.171555in}}{\pgfqpoint{2.460811in}{3.167164in}}{\pgfqpoint{2.452998in}{3.159351in}}%
\pgfpathcurveto{\pgfqpoint{2.445184in}{3.151537in}}{\pgfqpoint{2.440794in}{3.140938in}}{\pgfqpoint{2.440794in}{3.129888in}}%
\pgfpathcurveto{\pgfqpoint{2.440794in}{3.118838in}}{\pgfqpoint{2.445184in}{3.108239in}}{\pgfqpoint{2.452998in}{3.100425in}}%
\pgfpathcurveto{\pgfqpoint{2.460811in}{3.092611in}}{\pgfqpoint{2.471410in}{3.088221in}}{\pgfqpoint{2.482460in}{3.088221in}}%
\pgfpathclose%
\pgfusepath{stroke,fill}%
\end{pgfscope}%
\begin{pgfscope}%
\pgfpathrectangle{\pgfqpoint{0.600000in}{0.600000in}}{\pgfqpoint{3.900000in}{3.900000in}}%
\pgfusepath{clip}%
\pgfsetbuttcap%
\pgfsetroundjoin%
\definecolor{currentfill}{rgb}{0.121569,0.466667,0.705882}%
\pgfsetfillcolor{currentfill}%
\pgfsetlinewidth{1.003750pt}%
\definecolor{currentstroke}{rgb}{0.121569,0.466667,0.705882}%
\pgfsetstrokecolor{currentstroke}%
\pgfsetdash{}{0pt}%
\pgfpathmoveto{\pgfqpoint{2.200407in}{1.429614in}}%
\pgfpathcurveto{\pgfqpoint{2.211457in}{1.429614in}}{\pgfqpoint{2.222056in}{1.434004in}}{\pgfqpoint{2.229870in}{1.441818in}}%
\pgfpathcurveto{\pgfqpoint{2.237684in}{1.449632in}}{\pgfqpoint{2.242074in}{1.460231in}}{\pgfqpoint{2.242074in}{1.471281in}}%
\pgfpathcurveto{\pgfqpoint{2.242074in}{1.482331in}}{\pgfqpoint{2.237684in}{1.492930in}}{\pgfqpoint{2.229870in}{1.500744in}}%
\pgfpathcurveto{\pgfqpoint{2.222056in}{1.508557in}}{\pgfqpoint{2.211457in}{1.512948in}}{\pgfqpoint{2.200407in}{1.512948in}}%
\pgfpathcurveto{\pgfqpoint{2.189357in}{1.512948in}}{\pgfqpoint{2.178758in}{1.508557in}}{\pgfqpoint{2.170945in}{1.500744in}}%
\pgfpathcurveto{\pgfqpoint{2.163131in}{1.492930in}}{\pgfqpoint{2.158741in}{1.482331in}}{\pgfqpoint{2.158741in}{1.471281in}}%
\pgfpathcurveto{\pgfqpoint{2.158741in}{1.460231in}}{\pgfqpoint{2.163131in}{1.449632in}}{\pgfqpoint{2.170945in}{1.441818in}}%
\pgfpathcurveto{\pgfqpoint{2.178758in}{1.434004in}}{\pgfqpoint{2.189357in}{1.429614in}}{\pgfqpoint{2.200407in}{1.429614in}}%
\pgfpathclose%
\pgfusepath{stroke,fill}%
\end{pgfscope}%
\begin{pgfscope}%
\pgfpathrectangle{\pgfqpoint{0.600000in}{0.600000in}}{\pgfqpoint{3.900000in}{3.900000in}}%
\pgfusepath{clip}%
\pgfsetbuttcap%
\pgfsetroundjoin%
\definecolor{currentfill}{rgb}{0.121569,0.466667,0.705882}%
\pgfsetfillcolor{currentfill}%
\pgfsetlinewidth{1.003750pt}%
\definecolor{currentstroke}{rgb}{0.121569,0.466667,0.705882}%
\pgfsetstrokecolor{currentstroke}%
\pgfsetdash{}{0pt}%
\pgfpathmoveto{\pgfqpoint{2.402964in}{2.347725in}}%
\pgfpathcurveto{\pgfqpoint{2.414014in}{2.347725in}}{\pgfqpoint{2.424613in}{2.352115in}}{\pgfqpoint{2.432427in}{2.359928in}}%
\pgfpathcurveto{\pgfqpoint{2.440241in}{2.367742in}}{\pgfqpoint{2.444631in}{2.378341in}}{\pgfqpoint{2.444631in}{2.389391in}}%
\pgfpathcurveto{\pgfqpoint{2.444631in}{2.400441in}}{\pgfqpoint{2.440241in}{2.411040in}}{\pgfqpoint{2.432427in}{2.418854in}}%
\pgfpathcurveto{\pgfqpoint{2.424613in}{2.426668in}}{\pgfqpoint{2.414014in}{2.431058in}}{\pgfqpoint{2.402964in}{2.431058in}}%
\pgfpathcurveto{\pgfqpoint{2.391914in}{2.431058in}}{\pgfqpoint{2.381315in}{2.426668in}}{\pgfqpoint{2.373502in}{2.418854in}}%
\pgfpathcurveto{\pgfqpoint{2.365688in}{2.411040in}}{\pgfqpoint{2.361298in}{2.400441in}}{\pgfqpoint{2.361298in}{2.389391in}}%
\pgfpathcurveto{\pgfqpoint{2.361298in}{2.378341in}}{\pgfqpoint{2.365688in}{2.367742in}}{\pgfqpoint{2.373502in}{2.359928in}}%
\pgfpathcurveto{\pgfqpoint{2.381315in}{2.352115in}}{\pgfqpoint{2.391914in}{2.347725in}}{\pgfqpoint{2.402964in}{2.347725in}}%
\pgfpathclose%
\pgfusepath{stroke,fill}%
\end{pgfscope}%
\begin{pgfscope}%
\pgfpathrectangle{\pgfqpoint{0.600000in}{0.600000in}}{\pgfqpoint{3.900000in}{3.900000in}}%
\pgfusepath{clip}%
\pgfsetbuttcap%
\pgfsetroundjoin%
\definecolor{currentfill}{rgb}{0.121569,0.466667,0.705882}%
\pgfsetfillcolor{currentfill}%
\pgfsetlinewidth{1.003750pt}%
\definecolor{currentstroke}{rgb}{0.121569,0.466667,0.705882}%
\pgfsetstrokecolor{currentstroke}%
\pgfsetdash{}{0pt}%
\pgfpathmoveto{\pgfqpoint{1.780297in}{2.712284in}}%
\pgfpathcurveto{\pgfqpoint{1.791348in}{2.712284in}}{\pgfqpoint{1.801947in}{2.716674in}}{\pgfqpoint{1.809760in}{2.724488in}}%
\pgfpathcurveto{\pgfqpoint{1.817574in}{2.732301in}}{\pgfqpoint{1.821964in}{2.742900in}}{\pgfqpoint{1.821964in}{2.753950in}}%
\pgfpathcurveto{\pgfqpoint{1.821964in}{2.765001in}}{\pgfqpoint{1.817574in}{2.775600in}}{\pgfqpoint{1.809760in}{2.783413in}}%
\pgfpathcurveto{\pgfqpoint{1.801947in}{2.791227in}}{\pgfqpoint{1.791348in}{2.795617in}}{\pgfqpoint{1.780297in}{2.795617in}}%
\pgfpathcurveto{\pgfqpoint{1.769247in}{2.795617in}}{\pgfqpoint{1.758648in}{2.791227in}}{\pgfqpoint{1.750835in}{2.783413in}}%
\pgfpathcurveto{\pgfqpoint{1.743021in}{2.775600in}}{\pgfqpoint{1.738631in}{2.765001in}}{\pgfqpoint{1.738631in}{2.753950in}}%
\pgfpathcurveto{\pgfqpoint{1.738631in}{2.742900in}}{\pgfqpoint{1.743021in}{2.732301in}}{\pgfqpoint{1.750835in}{2.724488in}}%
\pgfpathcurveto{\pgfqpoint{1.758648in}{2.716674in}}{\pgfqpoint{1.769247in}{2.712284in}}{\pgfqpoint{1.780297in}{2.712284in}}%
\pgfpathclose%
\pgfusepath{stroke,fill}%
\end{pgfscope}%
\begin{pgfscope}%
\pgfpathrectangle{\pgfqpoint{0.600000in}{0.600000in}}{\pgfqpoint{3.900000in}{3.900000in}}%
\pgfusepath{clip}%
\pgfsetbuttcap%
\pgfsetroundjoin%
\definecolor{currentfill}{rgb}{0.121569,0.466667,0.705882}%
\pgfsetfillcolor{currentfill}%
\pgfsetlinewidth{1.003750pt}%
\definecolor{currentstroke}{rgb}{0.121569,0.466667,0.705882}%
\pgfsetstrokecolor{currentstroke}%
\pgfsetdash{}{0pt}%
\pgfpathmoveto{\pgfqpoint{2.529412in}{3.325821in}}%
\pgfpathcurveto{\pgfqpoint{2.540462in}{3.325821in}}{\pgfqpoint{2.551061in}{3.330211in}}{\pgfqpoint{2.558874in}{3.338025in}}%
\pgfpathcurveto{\pgfqpoint{2.566688in}{3.345838in}}{\pgfqpoint{2.571078in}{3.356437in}}{\pgfqpoint{2.571078in}{3.367487in}}%
\pgfpathcurveto{\pgfqpoint{2.571078in}{3.378537in}}{\pgfqpoint{2.566688in}{3.389137in}}{\pgfqpoint{2.558874in}{3.396950in}}%
\pgfpathcurveto{\pgfqpoint{2.551061in}{3.404764in}}{\pgfqpoint{2.540462in}{3.409154in}}{\pgfqpoint{2.529412in}{3.409154in}}%
\pgfpathcurveto{\pgfqpoint{2.518362in}{3.409154in}}{\pgfqpoint{2.507762in}{3.404764in}}{\pgfqpoint{2.499949in}{3.396950in}}%
\pgfpathcurveto{\pgfqpoint{2.492135in}{3.389137in}}{\pgfqpoint{2.487745in}{3.378537in}}{\pgfqpoint{2.487745in}{3.367487in}}%
\pgfpathcurveto{\pgfqpoint{2.487745in}{3.356437in}}{\pgfqpoint{2.492135in}{3.345838in}}{\pgfqpoint{2.499949in}{3.338025in}}%
\pgfpathcurveto{\pgfqpoint{2.507762in}{3.330211in}}{\pgfqpoint{2.518362in}{3.325821in}}{\pgfqpoint{2.529412in}{3.325821in}}%
\pgfpathclose%
\pgfusepath{stroke,fill}%
\end{pgfscope}%
\begin{pgfscope}%
\pgfpathrectangle{\pgfqpoint{0.600000in}{0.600000in}}{\pgfqpoint{3.900000in}{3.900000in}}%
\pgfusepath{clip}%
\pgfsetbuttcap%
\pgfsetroundjoin%
\definecolor{currentfill}{rgb}{0.121569,0.466667,0.705882}%
\pgfsetfillcolor{currentfill}%
\pgfsetlinewidth{1.003750pt}%
\definecolor{currentstroke}{rgb}{0.121569,0.466667,0.705882}%
\pgfsetstrokecolor{currentstroke}%
\pgfsetdash{}{0pt}%
\pgfpathmoveto{\pgfqpoint{2.300901in}{3.106153in}}%
\pgfpathcurveto{\pgfqpoint{2.311952in}{3.106153in}}{\pgfqpoint{2.322551in}{3.110543in}}{\pgfqpoint{2.330364in}{3.118357in}}%
\pgfpathcurveto{\pgfqpoint{2.338178in}{3.126171in}}{\pgfqpoint{2.342568in}{3.136770in}}{\pgfqpoint{2.342568in}{3.147820in}}%
\pgfpathcurveto{\pgfqpoint{2.342568in}{3.158870in}}{\pgfqpoint{2.338178in}{3.169469in}}{\pgfqpoint{2.330364in}{3.177282in}}%
\pgfpathcurveto{\pgfqpoint{2.322551in}{3.185096in}}{\pgfqpoint{2.311952in}{3.189486in}}{\pgfqpoint{2.300901in}{3.189486in}}%
\pgfpathcurveto{\pgfqpoint{2.289851in}{3.189486in}}{\pgfqpoint{2.279252in}{3.185096in}}{\pgfqpoint{2.271439in}{3.177282in}}%
\pgfpathcurveto{\pgfqpoint{2.263625in}{3.169469in}}{\pgfqpoint{2.259235in}{3.158870in}}{\pgfqpoint{2.259235in}{3.147820in}}%
\pgfpathcurveto{\pgfqpoint{2.259235in}{3.136770in}}{\pgfqpoint{2.263625in}{3.126171in}}{\pgfqpoint{2.271439in}{3.118357in}}%
\pgfpathcurveto{\pgfqpoint{2.279252in}{3.110543in}}{\pgfqpoint{2.289851in}{3.106153in}}{\pgfqpoint{2.300901in}{3.106153in}}%
\pgfpathclose%
\pgfusepath{stroke,fill}%
\end{pgfscope}%
\begin{pgfscope}%
\pgfpathrectangle{\pgfqpoint{0.600000in}{0.600000in}}{\pgfqpoint{3.900000in}{3.900000in}}%
\pgfusepath{clip}%
\pgfsetbuttcap%
\pgfsetroundjoin%
\definecolor{currentfill}{rgb}{0.121569,0.466667,0.705882}%
\pgfsetfillcolor{currentfill}%
\pgfsetlinewidth{1.003750pt}%
\definecolor{currentstroke}{rgb}{0.121569,0.466667,0.705882}%
\pgfsetstrokecolor{currentstroke}%
\pgfsetdash{}{0pt}%
\pgfpathmoveto{\pgfqpoint{3.038394in}{2.969006in}}%
\pgfpathcurveto{\pgfqpoint{3.049444in}{2.969006in}}{\pgfqpoint{3.060043in}{2.973396in}}{\pgfqpoint{3.067857in}{2.981210in}}%
\pgfpathcurveto{\pgfqpoint{3.075670in}{2.989023in}}{\pgfqpoint{3.080061in}{2.999622in}}{\pgfqpoint{3.080061in}{3.010672in}}%
\pgfpathcurveto{\pgfqpoint{3.080061in}{3.021723in}}{\pgfqpoint{3.075670in}{3.032322in}}{\pgfqpoint{3.067857in}{3.040135in}}%
\pgfpathcurveto{\pgfqpoint{3.060043in}{3.047949in}}{\pgfqpoint{3.049444in}{3.052339in}}{\pgfqpoint{3.038394in}{3.052339in}}%
\pgfpathcurveto{\pgfqpoint{3.027344in}{3.052339in}}{\pgfqpoint{3.016745in}{3.047949in}}{\pgfqpoint{3.008931in}{3.040135in}}%
\pgfpathcurveto{\pgfqpoint{3.001118in}{3.032322in}}{\pgfqpoint{2.996727in}{3.021723in}}{\pgfqpoint{2.996727in}{3.010672in}}%
\pgfpathcurveto{\pgfqpoint{2.996727in}{2.999622in}}{\pgfqpoint{3.001118in}{2.989023in}}{\pgfqpoint{3.008931in}{2.981210in}}%
\pgfpathcurveto{\pgfqpoint{3.016745in}{2.973396in}}{\pgfqpoint{3.027344in}{2.969006in}}{\pgfqpoint{3.038394in}{2.969006in}}%
\pgfpathclose%
\pgfusepath{stroke,fill}%
\end{pgfscope}%
\begin{pgfscope}%
\pgfpathrectangle{\pgfqpoint{0.600000in}{0.600000in}}{\pgfqpoint{3.900000in}{3.900000in}}%
\pgfusepath{clip}%
\pgfsetbuttcap%
\pgfsetroundjoin%
\definecolor{currentfill}{rgb}{0.121569,0.466667,0.705882}%
\pgfsetfillcolor{currentfill}%
\pgfsetlinewidth{1.003750pt}%
\definecolor{currentstroke}{rgb}{0.121569,0.466667,0.705882}%
\pgfsetstrokecolor{currentstroke}%
\pgfsetdash{}{0pt}%
\pgfpathmoveto{\pgfqpoint{2.432912in}{2.069833in}}%
\pgfpathcurveto{\pgfqpoint{2.443962in}{2.069833in}}{\pgfqpoint{2.454561in}{2.074223in}}{\pgfqpoint{2.462375in}{2.082037in}}%
\pgfpathcurveto{\pgfqpoint{2.470188in}{2.089851in}}{\pgfqpoint{2.474579in}{2.100450in}}{\pgfqpoint{2.474579in}{2.111500in}}%
\pgfpathcurveto{\pgfqpoint{2.474579in}{2.122550in}}{\pgfqpoint{2.470188in}{2.133149in}}{\pgfqpoint{2.462375in}{2.140962in}}%
\pgfpathcurveto{\pgfqpoint{2.454561in}{2.148776in}}{\pgfqpoint{2.443962in}{2.153166in}}{\pgfqpoint{2.432912in}{2.153166in}}%
\pgfpathcurveto{\pgfqpoint{2.421862in}{2.153166in}}{\pgfqpoint{2.411263in}{2.148776in}}{\pgfqpoint{2.403449in}{2.140962in}}%
\pgfpathcurveto{\pgfqpoint{2.395636in}{2.133149in}}{\pgfqpoint{2.391245in}{2.122550in}}{\pgfqpoint{2.391245in}{2.111500in}}%
\pgfpathcurveto{\pgfqpoint{2.391245in}{2.100450in}}{\pgfqpoint{2.395636in}{2.089851in}}{\pgfqpoint{2.403449in}{2.082037in}}%
\pgfpathcurveto{\pgfqpoint{2.411263in}{2.074223in}}{\pgfqpoint{2.421862in}{2.069833in}}{\pgfqpoint{2.432912in}{2.069833in}}%
\pgfpathclose%
\pgfusepath{stroke,fill}%
\end{pgfscope}%
\begin{pgfscope}%
\pgfpathrectangle{\pgfqpoint{0.600000in}{0.600000in}}{\pgfqpoint{3.900000in}{3.900000in}}%
\pgfusepath{clip}%
\pgfsetbuttcap%
\pgfsetroundjoin%
\definecolor{currentfill}{rgb}{0.121569,0.466667,0.705882}%
\pgfsetfillcolor{currentfill}%
\pgfsetlinewidth{1.003750pt}%
\definecolor{currentstroke}{rgb}{0.121569,0.466667,0.705882}%
\pgfsetstrokecolor{currentstroke}%
\pgfsetdash{}{0pt}%
\pgfpathmoveto{\pgfqpoint{2.310007in}{2.290941in}}%
\pgfpathcurveto{\pgfqpoint{2.321057in}{2.290941in}}{\pgfqpoint{2.331657in}{2.295332in}}{\pgfqpoint{2.339470in}{2.303145in}}%
\pgfpathcurveto{\pgfqpoint{2.347284in}{2.310959in}}{\pgfqpoint{2.351674in}{2.321558in}}{\pgfqpoint{2.351674in}{2.332608in}}%
\pgfpathcurveto{\pgfqpoint{2.351674in}{2.343658in}}{\pgfqpoint{2.347284in}{2.354257in}}{\pgfqpoint{2.339470in}{2.362071in}}%
\pgfpathcurveto{\pgfqpoint{2.331657in}{2.369884in}}{\pgfqpoint{2.321057in}{2.374275in}}{\pgfqpoint{2.310007in}{2.374275in}}%
\pgfpathcurveto{\pgfqpoint{2.298957in}{2.374275in}}{\pgfqpoint{2.288358in}{2.369884in}}{\pgfqpoint{2.280545in}{2.362071in}}%
\pgfpathcurveto{\pgfqpoint{2.272731in}{2.354257in}}{\pgfqpoint{2.268341in}{2.343658in}}{\pgfqpoint{2.268341in}{2.332608in}}%
\pgfpathcurveto{\pgfqpoint{2.268341in}{2.321558in}}{\pgfqpoint{2.272731in}{2.310959in}}{\pgfqpoint{2.280545in}{2.303145in}}%
\pgfpathcurveto{\pgfqpoint{2.288358in}{2.295332in}}{\pgfqpoint{2.298957in}{2.290941in}}{\pgfqpoint{2.310007in}{2.290941in}}%
\pgfpathclose%
\pgfusepath{stroke,fill}%
\end{pgfscope}%
\begin{pgfscope}%
\pgfpathrectangle{\pgfqpoint{0.600000in}{0.600000in}}{\pgfqpoint{3.900000in}{3.900000in}}%
\pgfusepath{clip}%
\pgfsetbuttcap%
\pgfsetroundjoin%
\definecolor{currentfill}{rgb}{0.121569,0.466667,0.705882}%
\pgfsetfillcolor{currentfill}%
\pgfsetlinewidth{1.003750pt}%
\definecolor{currentstroke}{rgb}{0.121569,0.466667,0.705882}%
\pgfsetstrokecolor{currentstroke}%
\pgfsetdash{}{0pt}%
\pgfpathmoveto{\pgfqpoint{2.735809in}{2.885062in}}%
\pgfpathcurveto{\pgfqpoint{2.746859in}{2.885062in}}{\pgfqpoint{2.757458in}{2.889453in}}{\pgfqpoint{2.765272in}{2.897266in}}%
\pgfpathcurveto{\pgfqpoint{2.773085in}{2.905080in}}{\pgfqpoint{2.777476in}{2.915679in}}{\pgfqpoint{2.777476in}{2.926729in}}%
\pgfpathcurveto{\pgfqpoint{2.777476in}{2.937779in}}{\pgfqpoint{2.773085in}{2.948378in}}{\pgfqpoint{2.765272in}{2.956192in}}%
\pgfpathcurveto{\pgfqpoint{2.757458in}{2.964005in}}{\pgfqpoint{2.746859in}{2.968396in}}{\pgfqpoint{2.735809in}{2.968396in}}%
\pgfpathcurveto{\pgfqpoint{2.724759in}{2.968396in}}{\pgfqpoint{2.714160in}{2.964005in}}{\pgfqpoint{2.706346in}{2.956192in}}%
\pgfpathcurveto{\pgfqpoint{2.698533in}{2.948378in}}{\pgfqpoint{2.694142in}{2.937779in}}{\pgfqpoint{2.694142in}{2.926729in}}%
\pgfpathcurveto{\pgfqpoint{2.694142in}{2.915679in}}{\pgfqpoint{2.698533in}{2.905080in}}{\pgfqpoint{2.706346in}{2.897266in}}%
\pgfpathcurveto{\pgfqpoint{2.714160in}{2.889453in}}{\pgfqpoint{2.724759in}{2.885062in}}{\pgfqpoint{2.735809in}{2.885062in}}%
\pgfpathclose%
\pgfusepath{stroke,fill}%
\end{pgfscope}%
\begin{pgfscope}%
\pgfpathrectangle{\pgfqpoint{0.600000in}{0.600000in}}{\pgfqpoint{3.900000in}{3.900000in}}%
\pgfusepath{clip}%
\pgfsetbuttcap%
\pgfsetroundjoin%
\definecolor{currentfill}{rgb}{0.121569,0.466667,0.705882}%
\pgfsetfillcolor{currentfill}%
\pgfsetlinewidth{1.003750pt}%
\definecolor{currentstroke}{rgb}{0.121569,0.466667,0.705882}%
\pgfsetstrokecolor{currentstroke}%
\pgfsetdash{}{0pt}%
\pgfpathmoveto{\pgfqpoint{2.808046in}{2.886783in}}%
\pgfpathcurveto{\pgfqpoint{2.819096in}{2.886783in}}{\pgfqpoint{2.829695in}{2.891174in}}{\pgfqpoint{2.837509in}{2.898987in}}%
\pgfpathcurveto{\pgfqpoint{2.845323in}{2.906801in}}{\pgfqpoint{2.849713in}{2.917400in}}{\pgfqpoint{2.849713in}{2.928450in}}%
\pgfpathcurveto{\pgfqpoint{2.849713in}{2.939500in}}{\pgfqpoint{2.845323in}{2.950099in}}{\pgfqpoint{2.837509in}{2.957913in}}%
\pgfpathcurveto{\pgfqpoint{2.829695in}{2.965726in}}{\pgfqpoint{2.819096in}{2.970117in}}{\pgfqpoint{2.808046in}{2.970117in}}%
\pgfpathcurveto{\pgfqpoint{2.796996in}{2.970117in}}{\pgfqpoint{2.786397in}{2.965726in}}{\pgfqpoint{2.778583in}{2.957913in}}%
\pgfpathcurveto{\pgfqpoint{2.770770in}{2.950099in}}{\pgfqpoint{2.766380in}{2.939500in}}{\pgfqpoint{2.766380in}{2.928450in}}%
\pgfpathcurveto{\pgfqpoint{2.766380in}{2.917400in}}{\pgfqpoint{2.770770in}{2.906801in}}{\pgfqpoint{2.778583in}{2.898987in}}%
\pgfpathcurveto{\pgfqpoint{2.786397in}{2.891174in}}{\pgfqpoint{2.796996in}{2.886783in}}{\pgfqpoint{2.808046in}{2.886783in}}%
\pgfpathclose%
\pgfusepath{stroke,fill}%
\end{pgfscope}%
\begin{pgfscope}%
\pgfpathrectangle{\pgfqpoint{0.600000in}{0.600000in}}{\pgfqpoint{3.900000in}{3.900000in}}%
\pgfusepath{clip}%
\pgfsetbuttcap%
\pgfsetroundjoin%
\definecolor{currentfill}{rgb}{0.121569,0.466667,0.705882}%
\pgfsetfillcolor{currentfill}%
\pgfsetlinewidth{1.003750pt}%
\definecolor{currentstroke}{rgb}{0.121569,0.466667,0.705882}%
\pgfsetstrokecolor{currentstroke}%
\pgfsetdash{}{0pt}%
\pgfpathmoveto{\pgfqpoint{2.861003in}{3.055992in}}%
\pgfpathcurveto{\pgfqpoint{2.872053in}{3.055992in}}{\pgfqpoint{2.882652in}{3.060383in}}{\pgfqpoint{2.890465in}{3.068196in}}%
\pgfpathcurveto{\pgfqpoint{2.898279in}{3.076010in}}{\pgfqpoint{2.902669in}{3.086609in}}{\pgfqpoint{2.902669in}{3.097659in}}%
\pgfpathcurveto{\pgfqpoint{2.902669in}{3.108709in}}{\pgfqpoint{2.898279in}{3.119308in}}{\pgfqpoint{2.890465in}{3.127122in}}%
\pgfpathcurveto{\pgfqpoint{2.882652in}{3.134935in}}{\pgfqpoint{2.872053in}{3.139326in}}{\pgfqpoint{2.861003in}{3.139326in}}%
\pgfpathcurveto{\pgfqpoint{2.849952in}{3.139326in}}{\pgfqpoint{2.839353in}{3.134935in}}{\pgfqpoint{2.831540in}{3.127122in}}%
\pgfpathcurveto{\pgfqpoint{2.823726in}{3.119308in}}{\pgfqpoint{2.819336in}{3.108709in}}{\pgfqpoint{2.819336in}{3.097659in}}%
\pgfpathcurveto{\pgfqpoint{2.819336in}{3.086609in}}{\pgfqpoint{2.823726in}{3.076010in}}{\pgfqpoint{2.831540in}{3.068196in}}%
\pgfpathcurveto{\pgfqpoint{2.839353in}{3.060383in}}{\pgfqpoint{2.849952in}{3.055992in}}{\pgfqpoint{2.861003in}{3.055992in}}%
\pgfpathclose%
\pgfusepath{stroke,fill}%
\end{pgfscope}%
\begin{pgfscope}%
\pgfpathrectangle{\pgfqpoint{0.600000in}{0.600000in}}{\pgfqpoint{3.900000in}{3.900000in}}%
\pgfusepath{clip}%
\pgfsetbuttcap%
\pgfsetroundjoin%
\definecolor{currentfill}{rgb}{0.121569,0.466667,0.705882}%
\pgfsetfillcolor{currentfill}%
\pgfsetlinewidth{1.003750pt}%
\definecolor{currentstroke}{rgb}{0.121569,0.466667,0.705882}%
\pgfsetstrokecolor{currentstroke}%
\pgfsetdash{}{0pt}%
\pgfpathmoveto{\pgfqpoint{2.510271in}{3.179938in}}%
\pgfpathcurveto{\pgfqpoint{2.521322in}{3.179938in}}{\pgfqpoint{2.531921in}{3.184328in}}{\pgfqpoint{2.539734in}{3.192142in}}%
\pgfpathcurveto{\pgfqpoint{2.547548in}{3.199956in}}{\pgfqpoint{2.551938in}{3.210555in}}{\pgfqpoint{2.551938in}{3.221605in}}%
\pgfpathcurveto{\pgfqpoint{2.551938in}{3.232655in}}{\pgfqpoint{2.547548in}{3.243254in}}{\pgfqpoint{2.539734in}{3.251068in}}%
\pgfpathcurveto{\pgfqpoint{2.531921in}{3.258881in}}{\pgfqpoint{2.521322in}{3.263272in}}{\pgfqpoint{2.510271in}{3.263272in}}%
\pgfpathcurveto{\pgfqpoint{2.499221in}{3.263272in}}{\pgfqpoint{2.488622in}{3.258881in}}{\pgfqpoint{2.480809in}{3.251068in}}%
\pgfpathcurveto{\pgfqpoint{2.472995in}{3.243254in}}{\pgfqpoint{2.468605in}{3.232655in}}{\pgfqpoint{2.468605in}{3.221605in}}%
\pgfpathcurveto{\pgfqpoint{2.468605in}{3.210555in}}{\pgfqpoint{2.472995in}{3.199956in}}{\pgfqpoint{2.480809in}{3.192142in}}%
\pgfpathcurveto{\pgfqpoint{2.488622in}{3.184328in}}{\pgfqpoint{2.499221in}{3.179938in}}{\pgfqpoint{2.510271in}{3.179938in}}%
\pgfpathclose%
\pgfusepath{stroke,fill}%
\end{pgfscope}%
\begin{pgfscope}%
\pgfpathrectangle{\pgfqpoint{0.600000in}{0.600000in}}{\pgfqpoint{3.900000in}{3.900000in}}%
\pgfusepath{clip}%
\pgfsetbuttcap%
\pgfsetroundjoin%
\definecolor{currentfill}{rgb}{0.121569,0.466667,0.705882}%
\pgfsetfillcolor{currentfill}%
\pgfsetlinewidth{1.003750pt}%
\definecolor{currentstroke}{rgb}{0.121569,0.466667,0.705882}%
\pgfsetstrokecolor{currentstroke}%
\pgfsetdash{}{0pt}%
\pgfpathmoveto{\pgfqpoint{2.962222in}{2.929740in}}%
\pgfpathcurveto{\pgfqpoint{2.973272in}{2.929740in}}{\pgfqpoint{2.983871in}{2.934131in}}{\pgfqpoint{2.991685in}{2.941944in}}%
\pgfpathcurveto{\pgfqpoint{2.999499in}{2.949758in}}{\pgfqpoint{3.003889in}{2.960357in}}{\pgfqpoint{3.003889in}{2.971407in}}%
\pgfpathcurveto{\pgfqpoint{3.003889in}{2.982457in}}{\pgfqpoint{2.999499in}{2.993056in}}{\pgfqpoint{2.991685in}{3.000870in}}%
\pgfpathcurveto{\pgfqpoint{2.983871in}{3.008684in}}{\pgfqpoint{2.973272in}{3.013074in}}{\pgfqpoint{2.962222in}{3.013074in}}%
\pgfpathcurveto{\pgfqpoint{2.951172in}{3.013074in}}{\pgfqpoint{2.940573in}{3.008684in}}{\pgfqpoint{2.932760in}{3.000870in}}%
\pgfpathcurveto{\pgfqpoint{2.924946in}{2.993056in}}{\pgfqpoint{2.920556in}{2.982457in}}{\pgfqpoint{2.920556in}{2.971407in}}%
\pgfpathcurveto{\pgfqpoint{2.920556in}{2.960357in}}{\pgfqpoint{2.924946in}{2.949758in}}{\pgfqpoint{2.932760in}{2.941944in}}%
\pgfpathcurveto{\pgfqpoint{2.940573in}{2.934131in}}{\pgfqpoint{2.951172in}{2.929740in}}{\pgfqpoint{2.962222in}{2.929740in}}%
\pgfpathclose%
\pgfusepath{stroke,fill}%
\end{pgfscope}%
\begin{pgfscope}%
\pgfpathrectangle{\pgfqpoint{0.600000in}{0.600000in}}{\pgfqpoint{3.900000in}{3.900000in}}%
\pgfusepath{clip}%
\pgfsetbuttcap%
\pgfsetroundjoin%
\definecolor{currentfill}{rgb}{0.121569,0.466667,0.705882}%
\pgfsetfillcolor{currentfill}%
\pgfsetlinewidth{1.003750pt}%
\definecolor{currentstroke}{rgb}{0.121569,0.466667,0.705882}%
\pgfsetstrokecolor{currentstroke}%
\pgfsetdash{}{0pt}%
\pgfpathmoveto{\pgfqpoint{2.067481in}{2.092513in}}%
\pgfpathcurveto{\pgfqpoint{2.078531in}{2.092513in}}{\pgfqpoint{2.089130in}{2.096904in}}{\pgfqpoint{2.096943in}{2.104717in}}%
\pgfpathcurveto{\pgfqpoint{2.104757in}{2.112531in}}{\pgfqpoint{2.109147in}{2.123130in}}{\pgfqpoint{2.109147in}{2.134180in}}%
\pgfpathcurveto{\pgfqpoint{2.109147in}{2.145230in}}{\pgfqpoint{2.104757in}{2.155829in}}{\pgfqpoint{2.096943in}{2.163643in}}%
\pgfpathcurveto{\pgfqpoint{2.089130in}{2.171457in}}{\pgfqpoint{2.078531in}{2.175847in}}{\pgfqpoint{2.067481in}{2.175847in}}%
\pgfpathcurveto{\pgfqpoint{2.056430in}{2.175847in}}{\pgfqpoint{2.045831in}{2.171457in}}{\pgfqpoint{2.038018in}{2.163643in}}%
\pgfpathcurveto{\pgfqpoint{2.030204in}{2.155829in}}{\pgfqpoint{2.025814in}{2.145230in}}{\pgfqpoint{2.025814in}{2.134180in}}%
\pgfpathcurveto{\pgfqpoint{2.025814in}{2.123130in}}{\pgfqpoint{2.030204in}{2.112531in}}{\pgfqpoint{2.038018in}{2.104717in}}%
\pgfpathcurveto{\pgfqpoint{2.045831in}{2.096904in}}{\pgfqpoint{2.056430in}{2.092513in}}{\pgfqpoint{2.067481in}{2.092513in}}%
\pgfpathclose%
\pgfusepath{stroke,fill}%
\end{pgfscope}%
\begin{pgfscope}%
\pgfpathrectangle{\pgfqpoint{0.600000in}{0.600000in}}{\pgfqpoint{3.900000in}{3.900000in}}%
\pgfusepath{clip}%
\pgfsetbuttcap%
\pgfsetroundjoin%
\definecolor{currentfill}{rgb}{0.121569,0.466667,0.705882}%
\pgfsetfillcolor{currentfill}%
\pgfsetlinewidth{1.003750pt}%
\definecolor{currentstroke}{rgb}{0.121569,0.466667,0.705882}%
\pgfsetstrokecolor{currentstroke}%
\pgfsetdash{}{0pt}%
\pgfpathmoveto{\pgfqpoint{2.020087in}{3.307732in}}%
\pgfpathcurveto{\pgfqpoint{2.031137in}{3.307732in}}{\pgfqpoint{2.041736in}{3.312122in}}{\pgfqpoint{2.049550in}{3.319936in}}%
\pgfpathcurveto{\pgfqpoint{2.057363in}{3.327749in}}{\pgfqpoint{2.061753in}{3.338348in}}{\pgfqpoint{2.061753in}{3.349398in}}%
\pgfpathcurveto{\pgfqpoint{2.061753in}{3.360449in}}{\pgfqpoint{2.057363in}{3.371048in}}{\pgfqpoint{2.049550in}{3.378861in}}%
\pgfpathcurveto{\pgfqpoint{2.041736in}{3.386675in}}{\pgfqpoint{2.031137in}{3.391065in}}{\pgfqpoint{2.020087in}{3.391065in}}%
\pgfpathcurveto{\pgfqpoint{2.009037in}{3.391065in}}{\pgfqpoint{1.998438in}{3.386675in}}{\pgfqpoint{1.990624in}{3.378861in}}%
\pgfpathcurveto{\pgfqpoint{1.982810in}{3.371048in}}{\pgfqpoint{1.978420in}{3.360449in}}{\pgfqpoint{1.978420in}{3.349398in}}%
\pgfpathcurveto{\pgfqpoint{1.978420in}{3.338348in}}{\pgfqpoint{1.982810in}{3.327749in}}{\pgfqpoint{1.990624in}{3.319936in}}%
\pgfpathcurveto{\pgfqpoint{1.998438in}{3.312122in}}{\pgfqpoint{2.009037in}{3.307732in}}{\pgfqpoint{2.020087in}{3.307732in}}%
\pgfpathclose%
\pgfusepath{stroke,fill}%
\end{pgfscope}%
\begin{pgfscope}%
\pgfpathrectangle{\pgfqpoint{0.600000in}{0.600000in}}{\pgfqpoint{3.900000in}{3.900000in}}%
\pgfusepath{clip}%
\pgfsetbuttcap%
\pgfsetroundjoin%
\definecolor{currentfill}{rgb}{0.121569,0.466667,0.705882}%
\pgfsetfillcolor{currentfill}%
\pgfsetlinewidth{1.003750pt}%
\definecolor{currentstroke}{rgb}{0.121569,0.466667,0.705882}%
\pgfsetstrokecolor{currentstroke}%
\pgfsetdash{}{0pt}%
\pgfpathmoveto{\pgfqpoint{2.886961in}{3.078133in}}%
\pgfpathcurveto{\pgfqpoint{2.898011in}{3.078133in}}{\pgfqpoint{2.908610in}{3.082523in}}{\pgfqpoint{2.916424in}{3.090337in}}%
\pgfpathcurveto{\pgfqpoint{2.924238in}{3.098151in}}{\pgfqpoint{2.928628in}{3.108750in}}{\pgfqpoint{2.928628in}{3.119800in}}%
\pgfpathcurveto{\pgfqpoint{2.928628in}{3.130850in}}{\pgfqpoint{2.924238in}{3.141449in}}{\pgfqpoint{2.916424in}{3.149263in}}%
\pgfpathcurveto{\pgfqpoint{2.908610in}{3.157076in}}{\pgfqpoint{2.898011in}{3.161467in}}{\pgfqpoint{2.886961in}{3.161467in}}%
\pgfpathcurveto{\pgfqpoint{2.875911in}{3.161467in}}{\pgfqpoint{2.865312in}{3.157076in}}{\pgfqpoint{2.857498in}{3.149263in}}%
\pgfpathcurveto{\pgfqpoint{2.849685in}{3.141449in}}{\pgfqpoint{2.845294in}{3.130850in}}{\pgfqpoint{2.845294in}{3.119800in}}%
\pgfpathcurveto{\pgfqpoint{2.845294in}{3.108750in}}{\pgfqpoint{2.849685in}{3.098151in}}{\pgfqpoint{2.857498in}{3.090337in}}%
\pgfpathcurveto{\pgfqpoint{2.865312in}{3.082523in}}{\pgfqpoint{2.875911in}{3.078133in}}{\pgfqpoint{2.886961in}{3.078133in}}%
\pgfpathclose%
\pgfusepath{stroke,fill}%
\end{pgfscope}%
\begin{pgfscope}%
\pgfpathrectangle{\pgfqpoint{0.600000in}{0.600000in}}{\pgfqpoint{3.900000in}{3.900000in}}%
\pgfusepath{clip}%
\pgfsetbuttcap%
\pgfsetroundjoin%
\definecolor{currentfill}{rgb}{0.121569,0.466667,0.705882}%
\pgfsetfillcolor{currentfill}%
\pgfsetlinewidth{1.003750pt}%
\definecolor{currentstroke}{rgb}{0.121569,0.466667,0.705882}%
\pgfsetstrokecolor{currentstroke}%
\pgfsetdash{}{0pt}%
\pgfpathmoveto{\pgfqpoint{2.083066in}{3.120750in}}%
\pgfpathcurveto{\pgfqpoint{2.094116in}{3.120750in}}{\pgfqpoint{2.104715in}{3.125141in}}{\pgfqpoint{2.112529in}{3.132954in}}%
\pgfpathcurveto{\pgfqpoint{2.120342in}{3.140768in}}{\pgfqpoint{2.124733in}{3.151367in}}{\pgfqpoint{2.124733in}{3.162417in}}%
\pgfpathcurveto{\pgfqpoint{2.124733in}{3.173467in}}{\pgfqpoint{2.120342in}{3.184066in}}{\pgfqpoint{2.112529in}{3.191880in}}%
\pgfpathcurveto{\pgfqpoint{2.104715in}{3.199693in}}{\pgfqpoint{2.094116in}{3.204084in}}{\pgfqpoint{2.083066in}{3.204084in}}%
\pgfpathcurveto{\pgfqpoint{2.072016in}{3.204084in}}{\pgfqpoint{2.061417in}{3.199693in}}{\pgfqpoint{2.053603in}{3.191880in}}%
\pgfpathcurveto{\pgfqpoint{2.045790in}{3.184066in}}{\pgfqpoint{2.041399in}{3.173467in}}{\pgfqpoint{2.041399in}{3.162417in}}%
\pgfpathcurveto{\pgfqpoint{2.041399in}{3.151367in}}{\pgfqpoint{2.045790in}{3.140768in}}{\pgfqpoint{2.053603in}{3.132954in}}%
\pgfpathcurveto{\pgfqpoint{2.061417in}{3.125141in}}{\pgfqpoint{2.072016in}{3.120750in}}{\pgfqpoint{2.083066in}{3.120750in}}%
\pgfpathclose%
\pgfusepath{stroke,fill}%
\end{pgfscope}%
\begin{pgfscope}%
\pgfpathrectangle{\pgfqpoint{0.600000in}{0.600000in}}{\pgfqpoint{3.900000in}{3.900000in}}%
\pgfusepath{clip}%
\pgfsetbuttcap%
\pgfsetroundjoin%
\definecolor{currentfill}{rgb}{0.121569,0.466667,0.705882}%
\pgfsetfillcolor{currentfill}%
\pgfsetlinewidth{1.003750pt}%
\definecolor{currentstroke}{rgb}{0.121569,0.466667,0.705882}%
\pgfsetstrokecolor{currentstroke}%
\pgfsetdash{}{0pt}%
\pgfpathmoveto{\pgfqpoint{1.473883in}{2.974966in}}%
\pgfpathcurveto{\pgfqpoint{1.484933in}{2.974966in}}{\pgfqpoint{1.495532in}{2.979356in}}{\pgfqpoint{1.503345in}{2.987170in}}%
\pgfpathcurveto{\pgfqpoint{1.511159in}{2.994983in}}{\pgfqpoint{1.515549in}{3.005582in}}{\pgfqpoint{1.515549in}{3.016633in}}%
\pgfpathcurveto{\pgfqpoint{1.515549in}{3.027683in}}{\pgfqpoint{1.511159in}{3.038282in}}{\pgfqpoint{1.503345in}{3.046095in}}%
\pgfpathcurveto{\pgfqpoint{1.495532in}{3.053909in}}{\pgfqpoint{1.484933in}{3.058299in}}{\pgfqpoint{1.473883in}{3.058299in}}%
\pgfpathcurveto{\pgfqpoint{1.462832in}{3.058299in}}{\pgfqpoint{1.452233in}{3.053909in}}{\pgfqpoint{1.444420in}{3.046095in}}%
\pgfpathcurveto{\pgfqpoint{1.436606in}{3.038282in}}{\pgfqpoint{1.432216in}{3.027683in}}{\pgfqpoint{1.432216in}{3.016633in}}%
\pgfpathcurveto{\pgfqpoint{1.432216in}{3.005582in}}{\pgfqpoint{1.436606in}{2.994983in}}{\pgfqpoint{1.444420in}{2.987170in}}%
\pgfpathcurveto{\pgfqpoint{1.452233in}{2.979356in}}{\pgfqpoint{1.462832in}{2.974966in}}{\pgfqpoint{1.473883in}{2.974966in}}%
\pgfpathclose%
\pgfusepath{stroke,fill}%
\end{pgfscope}%
\begin{pgfscope}%
\pgfpathrectangle{\pgfqpoint{0.600000in}{0.600000in}}{\pgfqpoint{3.900000in}{3.900000in}}%
\pgfusepath{clip}%
\pgfsetbuttcap%
\pgfsetroundjoin%
\definecolor{currentfill}{rgb}{0.121569,0.466667,0.705882}%
\pgfsetfillcolor{currentfill}%
\pgfsetlinewidth{1.003750pt}%
\definecolor{currentstroke}{rgb}{0.121569,0.466667,0.705882}%
\pgfsetstrokecolor{currentstroke}%
\pgfsetdash{}{0pt}%
\pgfpathmoveto{\pgfqpoint{2.843993in}{3.498773in}}%
\pgfpathcurveto{\pgfqpoint{2.855043in}{3.498773in}}{\pgfqpoint{2.865642in}{3.503163in}}{\pgfqpoint{2.873456in}{3.510977in}}%
\pgfpathcurveto{\pgfqpoint{2.881269in}{3.518790in}}{\pgfqpoint{2.885660in}{3.529389in}}{\pgfqpoint{2.885660in}{3.540439in}}%
\pgfpathcurveto{\pgfqpoint{2.885660in}{3.551490in}}{\pgfqpoint{2.881269in}{3.562089in}}{\pgfqpoint{2.873456in}{3.569902in}}%
\pgfpathcurveto{\pgfqpoint{2.865642in}{3.577716in}}{\pgfqpoint{2.855043in}{3.582106in}}{\pgfqpoint{2.843993in}{3.582106in}}%
\pgfpathcurveto{\pgfqpoint{2.832943in}{3.582106in}}{\pgfqpoint{2.822344in}{3.577716in}}{\pgfqpoint{2.814530in}{3.569902in}}%
\pgfpathcurveto{\pgfqpoint{2.806716in}{3.562089in}}{\pgfqpoint{2.802326in}{3.551490in}}{\pgfqpoint{2.802326in}{3.540439in}}%
\pgfpathcurveto{\pgfqpoint{2.802326in}{3.529389in}}{\pgfqpoint{2.806716in}{3.518790in}}{\pgfqpoint{2.814530in}{3.510977in}}%
\pgfpathcurveto{\pgfqpoint{2.822344in}{3.503163in}}{\pgfqpoint{2.832943in}{3.498773in}}{\pgfqpoint{2.843993in}{3.498773in}}%
\pgfpathclose%
\pgfusepath{stroke,fill}%
\end{pgfscope}%
\begin{pgfscope}%
\pgfpathrectangle{\pgfqpoint{0.600000in}{0.600000in}}{\pgfqpoint{3.900000in}{3.900000in}}%
\pgfusepath{clip}%
\pgfsetbuttcap%
\pgfsetroundjoin%
\definecolor{currentfill}{rgb}{0.121569,0.466667,0.705882}%
\pgfsetfillcolor{currentfill}%
\pgfsetlinewidth{1.003750pt}%
\definecolor{currentstroke}{rgb}{0.121569,0.466667,0.705882}%
\pgfsetstrokecolor{currentstroke}%
\pgfsetdash{}{0pt}%
\pgfpathmoveto{\pgfqpoint{1.797856in}{3.612449in}}%
\pgfpathcurveto{\pgfqpoint{1.808906in}{3.612449in}}{\pgfqpoint{1.819505in}{3.616840in}}{\pgfqpoint{1.827319in}{3.624653in}}%
\pgfpathcurveto{\pgfqpoint{1.835133in}{3.632467in}}{\pgfqpoint{1.839523in}{3.643066in}}{\pgfqpoint{1.839523in}{3.654116in}}%
\pgfpathcurveto{\pgfqpoint{1.839523in}{3.665166in}}{\pgfqpoint{1.835133in}{3.675765in}}{\pgfqpoint{1.827319in}{3.683579in}}%
\pgfpathcurveto{\pgfqpoint{1.819505in}{3.691392in}}{\pgfqpoint{1.808906in}{3.695783in}}{\pgfqpoint{1.797856in}{3.695783in}}%
\pgfpathcurveto{\pgfqpoint{1.786806in}{3.695783in}}{\pgfqpoint{1.776207in}{3.691392in}}{\pgfqpoint{1.768393in}{3.683579in}}%
\pgfpathcurveto{\pgfqpoint{1.760580in}{3.675765in}}{\pgfqpoint{1.756189in}{3.665166in}}{\pgfqpoint{1.756189in}{3.654116in}}%
\pgfpathcurveto{\pgfqpoint{1.756189in}{3.643066in}}{\pgfqpoint{1.760580in}{3.632467in}}{\pgfqpoint{1.768393in}{3.624653in}}%
\pgfpathcurveto{\pgfqpoint{1.776207in}{3.616840in}}{\pgfqpoint{1.786806in}{3.612449in}}{\pgfqpoint{1.797856in}{3.612449in}}%
\pgfpathclose%
\pgfusepath{stroke,fill}%
\end{pgfscope}%
\begin{pgfscope}%
\pgfpathrectangle{\pgfqpoint{0.600000in}{0.600000in}}{\pgfqpoint{3.900000in}{3.900000in}}%
\pgfusepath{clip}%
\pgfsetbuttcap%
\pgfsetroundjoin%
\definecolor{currentfill}{rgb}{0.121569,0.466667,0.705882}%
\pgfsetfillcolor{currentfill}%
\pgfsetlinewidth{1.003750pt}%
\definecolor{currentstroke}{rgb}{0.121569,0.466667,0.705882}%
\pgfsetstrokecolor{currentstroke}%
\pgfsetdash{}{0pt}%
\pgfpathmoveto{\pgfqpoint{2.508971in}{2.649790in}}%
\pgfpathcurveto{\pgfqpoint{2.520021in}{2.649790in}}{\pgfqpoint{2.530620in}{2.654180in}}{\pgfqpoint{2.538433in}{2.661994in}}%
\pgfpathcurveto{\pgfqpoint{2.546247in}{2.669807in}}{\pgfqpoint{2.550637in}{2.680406in}}{\pgfqpoint{2.550637in}{2.691456in}}%
\pgfpathcurveto{\pgfqpoint{2.550637in}{2.702507in}}{\pgfqpoint{2.546247in}{2.713106in}}{\pgfqpoint{2.538433in}{2.720919in}}%
\pgfpathcurveto{\pgfqpoint{2.530620in}{2.728733in}}{\pgfqpoint{2.520021in}{2.733123in}}{\pgfqpoint{2.508971in}{2.733123in}}%
\pgfpathcurveto{\pgfqpoint{2.497920in}{2.733123in}}{\pgfqpoint{2.487321in}{2.728733in}}{\pgfqpoint{2.479508in}{2.720919in}}%
\pgfpathcurveto{\pgfqpoint{2.471694in}{2.713106in}}{\pgfqpoint{2.467304in}{2.702507in}}{\pgfqpoint{2.467304in}{2.691456in}}%
\pgfpathcurveto{\pgfqpoint{2.467304in}{2.680406in}}{\pgfqpoint{2.471694in}{2.669807in}}{\pgfqpoint{2.479508in}{2.661994in}}%
\pgfpathcurveto{\pgfqpoint{2.487321in}{2.654180in}}{\pgfqpoint{2.497920in}{2.649790in}}{\pgfqpoint{2.508971in}{2.649790in}}%
\pgfpathclose%
\pgfusepath{stroke,fill}%
\end{pgfscope}%
\begin{pgfscope}%
\pgfpathrectangle{\pgfqpoint{0.600000in}{0.600000in}}{\pgfqpoint{3.900000in}{3.900000in}}%
\pgfusepath{clip}%
\pgfsetbuttcap%
\pgfsetroundjoin%
\definecolor{currentfill}{rgb}{0.121569,0.466667,0.705882}%
\pgfsetfillcolor{currentfill}%
\pgfsetlinewidth{1.003750pt}%
\definecolor{currentstroke}{rgb}{0.121569,0.466667,0.705882}%
\pgfsetstrokecolor{currentstroke}%
\pgfsetdash{}{0pt}%
\pgfpathmoveto{\pgfqpoint{2.672795in}{3.029942in}}%
\pgfpathcurveto{\pgfqpoint{2.683845in}{3.029942in}}{\pgfqpoint{2.694444in}{3.034332in}}{\pgfqpoint{2.702258in}{3.042146in}}%
\pgfpathcurveto{\pgfqpoint{2.710071in}{3.049959in}}{\pgfqpoint{2.714462in}{3.060558in}}{\pgfqpoint{2.714462in}{3.071609in}}%
\pgfpathcurveto{\pgfqpoint{2.714462in}{3.082659in}}{\pgfqpoint{2.710071in}{3.093258in}}{\pgfqpoint{2.702258in}{3.101071in}}%
\pgfpathcurveto{\pgfqpoint{2.694444in}{3.108885in}}{\pgfqpoint{2.683845in}{3.113275in}}{\pgfqpoint{2.672795in}{3.113275in}}%
\pgfpathcurveto{\pgfqpoint{2.661745in}{3.113275in}}{\pgfqpoint{2.651146in}{3.108885in}}{\pgfqpoint{2.643332in}{3.101071in}}%
\pgfpathcurveto{\pgfqpoint{2.635519in}{3.093258in}}{\pgfqpoint{2.631128in}{3.082659in}}{\pgfqpoint{2.631128in}{3.071609in}}%
\pgfpathcurveto{\pgfqpoint{2.631128in}{3.060558in}}{\pgfqpoint{2.635519in}{3.049959in}}{\pgfqpoint{2.643332in}{3.042146in}}%
\pgfpathcurveto{\pgfqpoint{2.651146in}{3.034332in}}{\pgfqpoint{2.661745in}{3.029942in}}{\pgfqpoint{2.672795in}{3.029942in}}%
\pgfpathclose%
\pgfusepath{stroke,fill}%
\end{pgfscope}%
\begin{pgfscope}%
\pgfpathrectangle{\pgfqpoint{0.600000in}{0.600000in}}{\pgfqpoint{3.900000in}{3.900000in}}%
\pgfusepath{clip}%
\pgfsetbuttcap%
\pgfsetroundjoin%
\definecolor{currentfill}{rgb}{0.121569,0.466667,0.705882}%
\pgfsetfillcolor{currentfill}%
\pgfsetlinewidth{1.003750pt}%
\definecolor{currentstroke}{rgb}{0.121569,0.466667,0.705882}%
\pgfsetstrokecolor{currentstroke}%
\pgfsetdash{}{0pt}%
\pgfpathmoveto{\pgfqpoint{2.888378in}{2.107674in}}%
\pgfpathcurveto{\pgfqpoint{2.899428in}{2.107674in}}{\pgfqpoint{2.910027in}{2.112064in}}{\pgfqpoint{2.917841in}{2.119877in}}%
\pgfpathcurveto{\pgfqpoint{2.925654in}{2.127691in}}{\pgfqpoint{2.930045in}{2.138290in}}{\pgfqpoint{2.930045in}{2.149340in}}%
\pgfpathcurveto{\pgfqpoint{2.930045in}{2.160390in}}{\pgfqpoint{2.925654in}{2.170989in}}{\pgfqpoint{2.917841in}{2.178803in}}%
\pgfpathcurveto{\pgfqpoint{2.910027in}{2.186617in}}{\pgfqpoint{2.899428in}{2.191007in}}{\pgfqpoint{2.888378in}{2.191007in}}%
\pgfpathcurveto{\pgfqpoint{2.877328in}{2.191007in}}{\pgfqpoint{2.866729in}{2.186617in}}{\pgfqpoint{2.858915in}{2.178803in}}%
\pgfpathcurveto{\pgfqpoint{2.851102in}{2.170989in}}{\pgfqpoint{2.846711in}{2.160390in}}{\pgfqpoint{2.846711in}{2.149340in}}%
\pgfpathcurveto{\pgfqpoint{2.846711in}{2.138290in}}{\pgfqpoint{2.851102in}{2.127691in}}{\pgfqpoint{2.858915in}{2.119877in}}%
\pgfpathcurveto{\pgfqpoint{2.866729in}{2.112064in}}{\pgfqpoint{2.877328in}{2.107674in}}{\pgfqpoint{2.888378in}{2.107674in}}%
\pgfpathclose%
\pgfusepath{stroke,fill}%
\end{pgfscope}%
\begin{pgfscope}%
\pgfpathrectangle{\pgfqpoint{0.600000in}{0.600000in}}{\pgfqpoint{3.900000in}{3.900000in}}%
\pgfusepath{clip}%
\pgfsetbuttcap%
\pgfsetroundjoin%
\definecolor{currentfill}{rgb}{0.121569,0.466667,0.705882}%
\pgfsetfillcolor{currentfill}%
\pgfsetlinewidth{1.003750pt}%
\definecolor{currentstroke}{rgb}{0.121569,0.466667,0.705882}%
\pgfsetstrokecolor{currentstroke}%
\pgfsetdash{}{0pt}%
\pgfpathmoveto{\pgfqpoint{1.873086in}{1.812507in}}%
\pgfpathcurveto{\pgfqpoint{1.884136in}{1.812507in}}{\pgfqpoint{1.894735in}{1.816897in}}{\pgfqpoint{1.902549in}{1.824711in}}%
\pgfpathcurveto{\pgfqpoint{1.910362in}{1.832524in}}{\pgfqpoint{1.914753in}{1.843123in}}{\pgfqpoint{1.914753in}{1.854174in}}%
\pgfpathcurveto{\pgfqpoint{1.914753in}{1.865224in}}{\pgfqpoint{1.910362in}{1.875823in}}{\pgfqpoint{1.902549in}{1.883636in}}%
\pgfpathcurveto{\pgfqpoint{1.894735in}{1.891450in}}{\pgfqpoint{1.884136in}{1.895840in}}{\pgfqpoint{1.873086in}{1.895840in}}%
\pgfpathcurveto{\pgfqpoint{1.862036in}{1.895840in}}{\pgfqpoint{1.851437in}{1.891450in}}{\pgfqpoint{1.843623in}{1.883636in}}%
\pgfpathcurveto{\pgfqpoint{1.835810in}{1.875823in}}{\pgfqpoint{1.831419in}{1.865224in}}{\pgfqpoint{1.831419in}{1.854174in}}%
\pgfpathcurveto{\pgfqpoint{1.831419in}{1.843123in}}{\pgfqpoint{1.835810in}{1.832524in}}{\pgfqpoint{1.843623in}{1.824711in}}%
\pgfpathcurveto{\pgfqpoint{1.851437in}{1.816897in}}{\pgfqpoint{1.862036in}{1.812507in}}{\pgfqpoint{1.873086in}{1.812507in}}%
\pgfpathclose%
\pgfusepath{stroke,fill}%
\end{pgfscope}%
\begin{pgfscope}%
\pgfpathrectangle{\pgfqpoint{0.600000in}{0.600000in}}{\pgfqpoint{3.900000in}{3.900000in}}%
\pgfusepath{clip}%
\pgfsetbuttcap%
\pgfsetroundjoin%
\definecolor{currentfill}{rgb}{0.121569,0.466667,0.705882}%
\pgfsetfillcolor{currentfill}%
\pgfsetlinewidth{1.003750pt}%
\definecolor{currentstroke}{rgb}{0.121569,0.466667,0.705882}%
\pgfsetstrokecolor{currentstroke}%
\pgfsetdash{}{0pt}%
\pgfpathmoveto{\pgfqpoint{2.728251in}{2.948475in}}%
\pgfpathcurveto{\pgfqpoint{2.739301in}{2.948475in}}{\pgfqpoint{2.749900in}{2.952865in}}{\pgfqpoint{2.757714in}{2.960679in}}%
\pgfpathcurveto{\pgfqpoint{2.765527in}{2.968493in}}{\pgfqpoint{2.769918in}{2.979092in}}{\pgfqpoint{2.769918in}{2.990142in}}%
\pgfpathcurveto{\pgfqpoint{2.769918in}{3.001192in}}{\pgfqpoint{2.765527in}{3.011791in}}{\pgfqpoint{2.757714in}{3.019604in}}%
\pgfpathcurveto{\pgfqpoint{2.749900in}{3.027418in}}{\pgfqpoint{2.739301in}{3.031808in}}{\pgfqpoint{2.728251in}{3.031808in}}%
\pgfpathcurveto{\pgfqpoint{2.717201in}{3.031808in}}{\pgfqpoint{2.706602in}{3.027418in}}{\pgfqpoint{2.698788in}{3.019604in}}%
\pgfpathcurveto{\pgfqpoint{2.690975in}{3.011791in}}{\pgfqpoint{2.686584in}{3.001192in}}{\pgfqpoint{2.686584in}{2.990142in}}%
\pgfpathcurveto{\pgfqpoint{2.686584in}{2.979092in}}{\pgfqpoint{2.690975in}{2.968493in}}{\pgfqpoint{2.698788in}{2.960679in}}%
\pgfpathcurveto{\pgfqpoint{2.706602in}{2.952865in}}{\pgfqpoint{2.717201in}{2.948475in}}{\pgfqpoint{2.728251in}{2.948475in}}%
\pgfpathclose%
\pgfusepath{stroke,fill}%
\end{pgfscope}%
\begin{pgfscope}%
\pgfpathrectangle{\pgfqpoint{0.600000in}{0.600000in}}{\pgfqpoint{3.900000in}{3.900000in}}%
\pgfusepath{clip}%
\pgfsetbuttcap%
\pgfsetroundjoin%
\definecolor{currentfill}{rgb}{0.121569,0.466667,0.705882}%
\pgfsetfillcolor{currentfill}%
\pgfsetlinewidth{1.003750pt}%
\definecolor{currentstroke}{rgb}{0.121569,0.466667,0.705882}%
\pgfsetstrokecolor{currentstroke}%
\pgfsetdash{}{0pt}%
\pgfpathmoveto{\pgfqpoint{2.117095in}{3.770131in}}%
\pgfpathcurveto{\pgfqpoint{2.128146in}{3.770131in}}{\pgfqpoint{2.138745in}{3.774521in}}{\pgfqpoint{2.146558in}{3.782334in}}%
\pgfpathcurveto{\pgfqpoint{2.154372in}{3.790148in}}{\pgfqpoint{2.158762in}{3.800747in}}{\pgfqpoint{2.158762in}{3.811797in}}%
\pgfpathcurveto{\pgfqpoint{2.158762in}{3.822847in}}{\pgfqpoint{2.154372in}{3.833446in}}{\pgfqpoint{2.146558in}{3.841260in}}%
\pgfpathcurveto{\pgfqpoint{2.138745in}{3.849074in}}{\pgfqpoint{2.128146in}{3.853464in}}{\pgfqpoint{2.117095in}{3.853464in}}%
\pgfpathcurveto{\pgfqpoint{2.106045in}{3.853464in}}{\pgfqpoint{2.095446in}{3.849074in}}{\pgfqpoint{2.087633in}{3.841260in}}%
\pgfpathcurveto{\pgfqpoint{2.079819in}{3.833446in}}{\pgfqpoint{2.075429in}{3.822847in}}{\pgfqpoint{2.075429in}{3.811797in}}%
\pgfpathcurveto{\pgfqpoint{2.075429in}{3.800747in}}{\pgfqpoint{2.079819in}{3.790148in}}{\pgfqpoint{2.087633in}{3.782334in}}%
\pgfpathcurveto{\pgfqpoint{2.095446in}{3.774521in}}{\pgfqpoint{2.106045in}{3.770131in}}{\pgfqpoint{2.117095in}{3.770131in}}%
\pgfpathclose%
\pgfusepath{stroke,fill}%
\end{pgfscope}%
\begin{pgfscope}%
\pgfpathrectangle{\pgfqpoint{0.600000in}{0.600000in}}{\pgfqpoint{3.900000in}{3.900000in}}%
\pgfusepath{clip}%
\pgfsetbuttcap%
\pgfsetroundjoin%
\definecolor{currentfill}{rgb}{0.121569,0.466667,0.705882}%
\pgfsetfillcolor{currentfill}%
\pgfsetlinewidth{1.003750pt}%
\definecolor{currentstroke}{rgb}{0.121569,0.466667,0.705882}%
\pgfsetstrokecolor{currentstroke}%
\pgfsetdash{}{0pt}%
\pgfpathmoveto{\pgfqpoint{2.067315in}{2.849689in}}%
\pgfpathcurveto{\pgfqpoint{2.078366in}{2.849689in}}{\pgfqpoint{2.088965in}{2.854079in}}{\pgfqpoint{2.096778in}{2.861893in}}%
\pgfpathcurveto{\pgfqpoint{2.104592in}{2.869707in}}{\pgfqpoint{2.108982in}{2.880306in}}{\pgfqpoint{2.108982in}{2.891356in}}%
\pgfpathcurveto{\pgfqpoint{2.108982in}{2.902406in}}{\pgfqpoint{2.104592in}{2.913005in}}{\pgfqpoint{2.096778in}{2.920819in}}%
\pgfpathcurveto{\pgfqpoint{2.088965in}{2.928632in}}{\pgfqpoint{2.078366in}{2.933023in}}{\pgfqpoint{2.067315in}{2.933023in}}%
\pgfpathcurveto{\pgfqpoint{2.056265in}{2.933023in}}{\pgfqpoint{2.045666in}{2.928632in}}{\pgfqpoint{2.037853in}{2.920819in}}%
\pgfpathcurveto{\pgfqpoint{2.030039in}{2.913005in}}{\pgfqpoint{2.025649in}{2.902406in}}{\pgfqpoint{2.025649in}{2.891356in}}%
\pgfpathcurveto{\pgfqpoint{2.025649in}{2.880306in}}{\pgfqpoint{2.030039in}{2.869707in}}{\pgfqpoint{2.037853in}{2.861893in}}%
\pgfpathcurveto{\pgfqpoint{2.045666in}{2.854079in}}{\pgfqpoint{2.056265in}{2.849689in}}{\pgfqpoint{2.067315in}{2.849689in}}%
\pgfpathclose%
\pgfusepath{stroke,fill}%
\end{pgfscope}%
\begin{pgfscope}%
\pgfpathrectangle{\pgfqpoint{0.600000in}{0.600000in}}{\pgfqpoint{3.900000in}{3.900000in}}%
\pgfusepath{clip}%
\pgfsetbuttcap%
\pgfsetroundjoin%
\definecolor{currentfill}{rgb}{0.121569,0.466667,0.705882}%
\pgfsetfillcolor{currentfill}%
\pgfsetlinewidth{1.003750pt}%
\definecolor{currentstroke}{rgb}{0.121569,0.466667,0.705882}%
\pgfsetstrokecolor{currentstroke}%
\pgfsetdash{}{0pt}%
\pgfpathmoveto{\pgfqpoint{3.206659in}{2.065376in}}%
\pgfpathcurveto{\pgfqpoint{3.217709in}{2.065376in}}{\pgfqpoint{3.228308in}{2.069766in}}{\pgfqpoint{3.236122in}{2.077580in}}%
\pgfpathcurveto{\pgfqpoint{3.243935in}{2.085393in}}{\pgfqpoint{3.248325in}{2.095992in}}{\pgfqpoint{3.248325in}{2.107042in}}%
\pgfpathcurveto{\pgfqpoint{3.248325in}{2.118092in}}{\pgfqpoint{3.243935in}{2.128691in}}{\pgfqpoint{3.236122in}{2.136505in}}%
\pgfpathcurveto{\pgfqpoint{3.228308in}{2.144319in}}{\pgfqpoint{3.217709in}{2.148709in}}{\pgfqpoint{3.206659in}{2.148709in}}%
\pgfpathcurveto{\pgfqpoint{3.195609in}{2.148709in}}{\pgfqpoint{3.185010in}{2.144319in}}{\pgfqpoint{3.177196in}{2.136505in}}%
\pgfpathcurveto{\pgfqpoint{3.169382in}{2.128691in}}{\pgfqpoint{3.164992in}{2.118092in}}{\pgfqpoint{3.164992in}{2.107042in}}%
\pgfpathcurveto{\pgfqpoint{3.164992in}{2.095992in}}{\pgfqpoint{3.169382in}{2.085393in}}{\pgfqpoint{3.177196in}{2.077580in}}%
\pgfpathcurveto{\pgfqpoint{3.185010in}{2.069766in}}{\pgfqpoint{3.195609in}{2.065376in}}{\pgfqpoint{3.206659in}{2.065376in}}%
\pgfpathclose%
\pgfusepath{stroke,fill}%
\end{pgfscope}%
\begin{pgfscope}%
\pgfpathrectangle{\pgfqpoint{0.600000in}{0.600000in}}{\pgfqpoint{3.900000in}{3.900000in}}%
\pgfusepath{clip}%
\pgfsetbuttcap%
\pgfsetroundjoin%
\definecolor{currentfill}{rgb}{0.121569,0.466667,0.705882}%
\pgfsetfillcolor{currentfill}%
\pgfsetlinewidth{1.003750pt}%
\definecolor{currentstroke}{rgb}{0.121569,0.466667,0.705882}%
\pgfsetstrokecolor{currentstroke}%
\pgfsetdash{}{0pt}%
\pgfpathmoveto{\pgfqpoint{3.224653in}{1.677079in}}%
\pgfpathcurveto{\pgfqpoint{3.235703in}{1.677079in}}{\pgfqpoint{3.246302in}{1.681469in}}{\pgfqpoint{3.254115in}{1.689283in}}%
\pgfpathcurveto{\pgfqpoint{3.261929in}{1.697096in}}{\pgfqpoint{3.266319in}{1.707695in}}{\pgfqpoint{3.266319in}{1.718745in}}%
\pgfpathcurveto{\pgfqpoint{3.266319in}{1.729795in}}{\pgfqpoint{3.261929in}{1.740395in}}{\pgfqpoint{3.254115in}{1.748208in}}%
\pgfpathcurveto{\pgfqpoint{3.246302in}{1.756022in}}{\pgfqpoint{3.235703in}{1.760412in}}{\pgfqpoint{3.224653in}{1.760412in}}%
\pgfpathcurveto{\pgfqpoint{3.213602in}{1.760412in}}{\pgfqpoint{3.203003in}{1.756022in}}{\pgfqpoint{3.195190in}{1.748208in}}%
\pgfpathcurveto{\pgfqpoint{3.187376in}{1.740395in}}{\pgfqpoint{3.182986in}{1.729795in}}{\pgfqpoint{3.182986in}{1.718745in}}%
\pgfpathcurveto{\pgfqpoint{3.182986in}{1.707695in}}{\pgfqpoint{3.187376in}{1.697096in}}{\pgfqpoint{3.195190in}{1.689283in}}%
\pgfpathcurveto{\pgfqpoint{3.203003in}{1.681469in}}{\pgfqpoint{3.213602in}{1.677079in}}{\pgfqpoint{3.224653in}{1.677079in}}%
\pgfpathclose%
\pgfusepath{stroke,fill}%
\end{pgfscope}%
\begin{pgfscope}%
\pgfpathrectangle{\pgfqpoint{0.600000in}{0.600000in}}{\pgfqpoint{3.900000in}{3.900000in}}%
\pgfusepath{clip}%
\pgfsetbuttcap%
\pgfsetroundjoin%
\definecolor{currentfill}{rgb}{0.121569,0.466667,0.705882}%
\pgfsetfillcolor{currentfill}%
\pgfsetlinewidth{1.003750pt}%
\definecolor{currentstroke}{rgb}{0.121569,0.466667,0.705882}%
\pgfsetstrokecolor{currentstroke}%
\pgfsetdash{}{0pt}%
\pgfpathmoveto{\pgfqpoint{1.809254in}{1.947131in}}%
\pgfpathcurveto{\pgfqpoint{1.820304in}{1.947131in}}{\pgfqpoint{1.830903in}{1.951521in}}{\pgfqpoint{1.838716in}{1.959334in}}%
\pgfpathcurveto{\pgfqpoint{1.846530in}{1.967148in}}{\pgfqpoint{1.850920in}{1.977747in}}{\pgfqpoint{1.850920in}{1.988797in}}%
\pgfpathcurveto{\pgfqpoint{1.850920in}{1.999847in}}{\pgfqpoint{1.846530in}{2.010446in}}{\pgfqpoint{1.838716in}{2.018260in}}%
\pgfpathcurveto{\pgfqpoint{1.830903in}{2.026074in}}{\pgfqpoint{1.820304in}{2.030464in}}{\pgfqpoint{1.809254in}{2.030464in}}%
\pgfpathcurveto{\pgfqpoint{1.798203in}{2.030464in}}{\pgfqpoint{1.787604in}{2.026074in}}{\pgfqpoint{1.779791in}{2.018260in}}%
\pgfpathcurveto{\pgfqpoint{1.771977in}{2.010446in}}{\pgfqpoint{1.767587in}{1.999847in}}{\pgfqpoint{1.767587in}{1.988797in}}%
\pgfpathcurveto{\pgfqpoint{1.767587in}{1.977747in}}{\pgfqpoint{1.771977in}{1.967148in}}{\pgfqpoint{1.779791in}{1.959334in}}%
\pgfpathcurveto{\pgfqpoint{1.787604in}{1.951521in}}{\pgfqpoint{1.798203in}{1.947131in}}{\pgfqpoint{1.809254in}{1.947131in}}%
\pgfpathclose%
\pgfusepath{stroke,fill}%
\end{pgfscope}%
\begin{pgfscope}%
\pgfpathrectangle{\pgfqpoint{0.600000in}{0.600000in}}{\pgfqpoint{3.900000in}{3.900000in}}%
\pgfusepath{clip}%
\pgfsetbuttcap%
\pgfsetroundjoin%
\definecolor{currentfill}{rgb}{0.121569,0.466667,0.705882}%
\pgfsetfillcolor{currentfill}%
\pgfsetlinewidth{1.003750pt}%
\definecolor{currentstroke}{rgb}{0.121569,0.466667,0.705882}%
\pgfsetstrokecolor{currentstroke}%
\pgfsetdash{}{0pt}%
\pgfpathmoveto{\pgfqpoint{2.266033in}{3.220636in}}%
\pgfpathcurveto{\pgfqpoint{2.277083in}{3.220636in}}{\pgfqpoint{2.287682in}{3.225026in}}{\pgfqpoint{2.295496in}{3.232840in}}%
\pgfpathcurveto{\pgfqpoint{2.303310in}{3.240653in}}{\pgfqpoint{2.307700in}{3.251252in}}{\pgfqpoint{2.307700in}{3.262302in}}%
\pgfpathcurveto{\pgfqpoint{2.307700in}{3.273353in}}{\pgfqpoint{2.303310in}{3.283952in}}{\pgfqpoint{2.295496in}{3.291765in}}%
\pgfpathcurveto{\pgfqpoint{2.287682in}{3.299579in}}{\pgfqpoint{2.277083in}{3.303969in}}{\pgfqpoint{2.266033in}{3.303969in}}%
\pgfpathcurveto{\pgfqpoint{2.254983in}{3.303969in}}{\pgfqpoint{2.244384in}{3.299579in}}{\pgfqpoint{2.236570in}{3.291765in}}%
\pgfpathcurveto{\pgfqpoint{2.228757in}{3.283952in}}{\pgfqpoint{2.224367in}{3.273353in}}{\pgfqpoint{2.224367in}{3.262302in}}%
\pgfpathcurveto{\pgfqpoint{2.224367in}{3.251252in}}{\pgfqpoint{2.228757in}{3.240653in}}{\pgfqpoint{2.236570in}{3.232840in}}%
\pgfpathcurveto{\pgfqpoint{2.244384in}{3.225026in}}{\pgfqpoint{2.254983in}{3.220636in}}{\pgfqpoint{2.266033in}{3.220636in}}%
\pgfpathclose%
\pgfusepath{stroke,fill}%
\end{pgfscope}%
\begin{pgfscope}%
\pgfpathrectangle{\pgfqpoint{0.600000in}{0.600000in}}{\pgfqpoint{3.900000in}{3.900000in}}%
\pgfusepath{clip}%
\pgfsetbuttcap%
\pgfsetroundjoin%
\definecolor{currentfill}{rgb}{0.121569,0.466667,0.705882}%
\pgfsetfillcolor{currentfill}%
\pgfsetlinewidth{1.003750pt}%
\definecolor{currentstroke}{rgb}{0.121569,0.466667,0.705882}%
\pgfsetstrokecolor{currentstroke}%
\pgfsetdash{}{0pt}%
\pgfpathmoveto{\pgfqpoint{2.951188in}{2.542297in}}%
\pgfpathcurveto{\pgfqpoint{2.962238in}{2.542297in}}{\pgfqpoint{2.972837in}{2.546688in}}{\pgfqpoint{2.980651in}{2.554501in}}%
\pgfpathcurveto{\pgfqpoint{2.988464in}{2.562315in}}{\pgfqpoint{2.992854in}{2.572914in}}{\pgfqpoint{2.992854in}{2.583964in}}%
\pgfpathcurveto{\pgfqpoint{2.992854in}{2.595014in}}{\pgfqpoint{2.988464in}{2.605613in}}{\pgfqpoint{2.980651in}{2.613427in}}%
\pgfpathcurveto{\pgfqpoint{2.972837in}{2.621240in}}{\pgfqpoint{2.962238in}{2.625631in}}{\pgfqpoint{2.951188in}{2.625631in}}%
\pgfpathcurveto{\pgfqpoint{2.940138in}{2.625631in}}{\pgfqpoint{2.929539in}{2.621240in}}{\pgfqpoint{2.921725in}{2.613427in}}%
\pgfpathcurveto{\pgfqpoint{2.913911in}{2.605613in}}{\pgfqpoint{2.909521in}{2.595014in}}{\pgfqpoint{2.909521in}{2.583964in}}%
\pgfpathcurveto{\pgfqpoint{2.909521in}{2.572914in}}{\pgfqpoint{2.913911in}{2.562315in}}{\pgfqpoint{2.921725in}{2.554501in}}%
\pgfpathcurveto{\pgfqpoint{2.929539in}{2.546688in}}{\pgfqpoint{2.940138in}{2.542297in}}{\pgfqpoint{2.951188in}{2.542297in}}%
\pgfpathclose%
\pgfusepath{stroke,fill}%
\end{pgfscope}%
\begin{pgfscope}%
\pgfpathrectangle{\pgfqpoint{0.600000in}{0.600000in}}{\pgfqpoint{3.900000in}{3.900000in}}%
\pgfusepath{clip}%
\pgfsetbuttcap%
\pgfsetroundjoin%
\definecolor{currentfill}{rgb}{0.121569,0.466667,0.705882}%
\pgfsetfillcolor{currentfill}%
\pgfsetlinewidth{1.003750pt}%
\definecolor{currentstroke}{rgb}{0.121569,0.466667,0.705882}%
\pgfsetstrokecolor{currentstroke}%
\pgfsetdash{}{0pt}%
\pgfpathmoveto{\pgfqpoint{2.565336in}{2.271103in}}%
\pgfpathcurveto{\pgfqpoint{2.576386in}{2.271103in}}{\pgfqpoint{2.586985in}{2.275493in}}{\pgfqpoint{2.594799in}{2.283307in}}%
\pgfpathcurveto{\pgfqpoint{2.602613in}{2.291120in}}{\pgfqpoint{2.607003in}{2.301719in}}{\pgfqpoint{2.607003in}{2.312770in}}%
\pgfpathcurveto{\pgfqpoint{2.607003in}{2.323820in}}{\pgfqpoint{2.602613in}{2.334419in}}{\pgfqpoint{2.594799in}{2.342232in}}%
\pgfpathcurveto{\pgfqpoint{2.586985in}{2.350046in}}{\pgfqpoint{2.576386in}{2.354436in}}{\pgfqpoint{2.565336in}{2.354436in}}%
\pgfpathcurveto{\pgfqpoint{2.554286in}{2.354436in}}{\pgfqpoint{2.543687in}{2.350046in}}{\pgfqpoint{2.535873in}{2.342232in}}%
\pgfpathcurveto{\pgfqpoint{2.528060in}{2.334419in}}{\pgfqpoint{2.523669in}{2.323820in}}{\pgfqpoint{2.523669in}{2.312770in}}%
\pgfpathcurveto{\pgfqpoint{2.523669in}{2.301719in}}{\pgfqpoint{2.528060in}{2.291120in}}{\pgfqpoint{2.535873in}{2.283307in}}%
\pgfpathcurveto{\pgfqpoint{2.543687in}{2.275493in}}{\pgfqpoint{2.554286in}{2.271103in}}{\pgfqpoint{2.565336in}{2.271103in}}%
\pgfpathclose%
\pgfusepath{stroke,fill}%
\end{pgfscope}%
\begin{pgfscope}%
\pgfpathrectangle{\pgfqpoint{0.600000in}{0.600000in}}{\pgfqpoint{3.900000in}{3.900000in}}%
\pgfusepath{clip}%
\pgfsetbuttcap%
\pgfsetroundjoin%
\definecolor{currentfill}{rgb}{0.121569,0.466667,0.705882}%
\pgfsetfillcolor{currentfill}%
\pgfsetlinewidth{1.003750pt}%
\definecolor{currentstroke}{rgb}{0.121569,0.466667,0.705882}%
\pgfsetstrokecolor{currentstroke}%
\pgfsetdash{}{0pt}%
\pgfpathmoveto{\pgfqpoint{2.047042in}{2.416352in}}%
\pgfpathcurveto{\pgfqpoint{2.058093in}{2.416352in}}{\pgfqpoint{2.068692in}{2.420742in}}{\pgfqpoint{2.076505in}{2.428556in}}%
\pgfpathcurveto{\pgfqpoint{2.084319in}{2.436369in}}{\pgfqpoint{2.088709in}{2.446968in}}{\pgfqpoint{2.088709in}{2.458018in}}%
\pgfpathcurveto{\pgfqpoint{2.088709in}{2.469068in}}{\pgfqpoint{2.084319in}{2.479667in}}{\pgfqpoint{2.076505in}{2.487481in}}%
\pgfpathcurveto{\pgfqpoint{2.068692in}{2.495295in}}{\pgfqpoint{2.058093in}{2.499685in}}{\pgfqpoint{2.047042in}{2.499685in}}%
\pgfpathcurveto{\pgfqpoint{2.035992in}{2.499685in}}{\pgfqpoint{2.025393in}{2.495295in}}{\pgfqpoint{2.017580in}{2.487481in}}%
\pgfpathcurveto{\pgfqpoint{2.009766in}{2.479667in}}{\pgfqpoint{2.005376in}{2.469068in}}{\pgfqpoint{2.005376in}{2.458018in}}%
\pgfpathcurveto{\pgfqpoint{2.005376in}{2.446968in}}{\pgfqpoint{2.009766in}{2.436369in}}{\pgfqpoint{2.017580in}{2.428556in}}%
\pgfpathcurveto{\pgfqpoint{2.025393in}{2.420742in}}{\pgfqpoint{2.035992in}{2.416352in}}{\pgfqpoint{2.047042in}{2.416352in}}%
\pgfpathclose%
\pgfusepath{stroke,fill}%
\end{pgfscope}%
\begin{pgfscope}%
\pgfpathrectangle{\pgfqpoint{0.600000in}{0.600000in}}{\pgfqpoint{3.900000in}{3.900000in}}%
\pgfusepath{clip}%
\pgfsetbuttcap%
\pgfsetroundjoin%
\definecolor{currentfill}{rgb}{0.121569,0.466667,0.705882}%
\pgfsetfillcolor{currentfill}%
\pgfsetlinewidth{1.003750pt}%
\definecolor{currentstroke}{rgb}{0.121569,0.466667,0.705882}%
\pgfsetstrokecolor{currentstroke}%
\pgfsetdash{}{0pt}%
\pgfpathmoveto{\pgfqpoint{2.304455in}{1.464484in}}%
\pgfpathcurveto{\pgfqpoint{2.315505in}{1.464484in}}{\pgfqpoint{2.326104in}{1.468875in}}{\pgfqpoint{2.333917in}{1.476688in}}%
\pgfpathcurveto{\pgfqpoint{2.341731in}{1.484502in}}{\pgfqpoint{2.346121in}{1.495101in}}{\pgfqpoint{2.346121in}{1.506151in}}%
\pgfpathcurveto{\pgfqpoint{2.346121in}{1.517201in}}{\pgfqpoint{2.341731in}{1.527800in}}{\pgfqpoint{2.333917in}{1.535614in}}%
\pgfpathcurveto{\pgfqpoint{2.326104in}{1.543428in}}{\pgfqpoint{2.315505in}{1.547818in}}{\pgfqpoint{2.304455in}{1.547818in}}%
\pgfpathcurveto{\pgfqpoint{2.293404in}{1.547818in}}{\pgfqpoint{2.282805in}{1.543428in}}{\pgfqpoint{2.274992in}{1.535614in}}%
\pgfpathcurveto{\pgfqpoint{2.267178in}{1.527800in}}{\pgfqpoint{2.262788in}{1.517201in}}{\pgfqpoint{2.262788in}{1.506151in}}%
\pgfpathcurveto{\pgfqpoint{2.262788in}{1.495101in}}{\pgfqpoint{2.267178in}{1.484502in}}{\pgfqpoint{2.274992in}{1.476688in}}%
\pgfpathcurveto{\pgfqpoint{2.282805in}{1.468875in}}{\pgfqpoint{2.293404in}{1.464484in}}{\pgfqpoint{2.304455in}{1.464484in}}%
\pgfpathclose%
\pgfusepath{stroke,fill}%
\end{pgfscope}%
\begin{pgfscope}%
\pgfpathrectangle{\pgfqpoint{0.600000in}{0.600000in}}{\pgfqpoint{3.900000in}{3.900000in}}%
\pgfusepath{clip}%
\pgfsetbuttcap%
\pgfsetroundjoin%
\definecolor{currentfill}{rgb}{0.121569,0.466667,0.705882}%
\pgfsetfillcolor{currentfill}%
\pgfsetlinewidth{1.003750pt}%
\definecolor{currentstroke}{rgb}{0.121569,0.466667,0.705882}%
\pgfsetstrokecolor{currentstroke}%
\pgfsetdash{}{0pt}%
\pgfpathmoveto{\pgfqpoint{3.049990in}{2.413843in}}%
\pgfpathcurveto{\pgfqpoint{3.061040in}{2.413843in}}{\pgfqpoint{3.071639in}{2.418234in}}{\pgfqpoint{3.079453in}{2.426047in}}%
\pgfpathcurveto{\pgfqpoint{3.087267in}{2.433861in}}{\pgfqpoint{3.091657in}{2.444460in}}{\pgfqpoint{3.091657in}{2.455510in}}%
\pgfpathcurveto{\pgfqpoint{3.091657in}{2.466560in}}{\pgfqpoint{3.087267in}{2.477159in}}{\pgfqpoint{3.079453in}{2.484973in}}%
\pgfpathcurveto{\pgfqpoint{3.071639in}{2.492786in}}{\pgfqpoint{3.061040in}{2.497177in}}{\pgfqpoint{3.049990in}{2.497177in}}%
\pgfpathcurveto{\pgfqpoint{3.038940in}{2.497177in}}{\pgfqpoint{3.028341in}{2.492786in}}{\pgfqpoint{3.020527in}{2.484973in}}%
\pgfpathcurveto{\pgfqpoint{3.012714in}{2.477159in}}{\pgfqpoint{3.008323in}{2.466560in}}{\pgfqpoint{3.008323in}{2.455510in}}%
\pgfpathcurveto{\pgfqpoint{3.008323in}{2.444460in}}{\pgfqpoint{3.012714in}{2.433861in}}{\pgfqpoint{3.020527in}{2.426047in}}%
\pgfpathcurveto{\pgfqpoint{3.028341in}{2.418234in}}{\pgfqpoint{3.038940in}{2.413843in}}{\pgfqpoint{3.049990in}{2.413843in}}%
\pgfpathclose%
\pgfusepath{stroke,fill}%
\end{pgfscope}%
\begin{pgfscope}%
\pgfpathrectangle{\pgfqpoint{0.600000in}{0.600000in}}{\pgfqpoint{3.900000in}{3.900000in}}%
\pgfusepath{clip}%
\pgfsetbuttcap%
\pgfsetroundjoin%
\definecolor{currentfill}{rgb}{0.121569,0.466667,0.705882}%
\pgfsetfillcolor{currentfill}%
\pgfsetlinewidth{1.003750pt}%
\definecolor{currentstroke}{rgb}{0.121569,0.466667,0.705882}%
\pgfsetstrokecolor{currentstroke}%
\pgfsetdash{}{0pt}%
\pgfpathmoveto{\pgfqpoint{3.432106in}{3.427221in}}%
\pgfpathcurveto{\pgfqpoint{3.443156in}{3.427221in}}{\pgfqpoint{3.453755in}{3.431611in}}{\pgfqpoint{3.461569in}{3.439425in}}%
\pgfpathcurveto{\pgfqpoint{3.469382in}{3.447239in}}{\pgfqpoint{3.473773in}{3.457838in}}{\pgfqpoint{3.473773in}{3.468888in}}%
\pgfpathcurveto{\pgfqpoint{3.473773in}{3.479938in}}{\pgfqpoint{3.469382in}{3.490537in}}{\pgfqpoint{3.461569in}{3.498351in}}%
\pgfpathcurveto{\pgfqpoint{3.453755in}{3.506164in}}{\pgfqpoint{3.443156in}{3.510555in}}{\pgfqpoint{3.432106in}{3.510555in}}%
\pgfpathcurveto{\pgfqpoint{3.421056in}{3.510555in}}{\pgfqpoint{3.410457in}{3.506164in}}{\pgfqpoint{3.402643in}{3.498351in}}%
\pgfpathcurveto{\pgfqpoint{3.394830in}{3.490537in}}{\pgfqpoint{3.390439in}{3.479938in}}{\pgfqpoint{3.390439in}{3.468888in}}%
\pgfpathcurveto{\pgfqpoint{3.390439in}{3.457838in}}{\pgfqpoint{3.394830in}{3.447239in}}{\pgfqpoint{3.402643in}{3.439425in}}%
\pgfpathcurveto{\pgfqpoint{3.410457in}{3.431611in}}{\pgfqpoint{3.421056in}{3.427221in}}{\pgfqpoint{3.432106in}{3.427221in}}%
\pgfpathclose%
\pgfusepath{stroke,fill}%
\end{pgfscope}%
\begin{pgfscope}%
\pgfpathrectangle{\pgfqpoint{0.600000in}{0.600000in}}{\pgfqpoint{3.900000in}{3.900000in}}%
\pgfusepath{clip}%
\pgfsetbuttcap%
\pgfsetroundjoin%
\definecolor{currentfill}{rgb}{0.121569,0.466667,0.705882}%
\pgfsetfillcolor{currentfill}%
\pgfsetlinewidth{1.003750pt}%
\definecolor{currentstroke}{rgb}{0.121569,0.466667,0.705882}%
\pgfsetstrokecolor{currentstroke}%
\pgfsetdash{}{0pt}%
\pgfpathmoveto{\pgfqpoint{1.860450in}{1.760592in}}%
\pgfpathcurveto{\pgfqpoint{1.871500in}{1.760592in}}{\pgfqpoint{1.882099in}{1.764983in}}{\pgfqpoint{1.889913in}{1.772796in}}%
\pgfpathcurveto{\pgfqpoint{1.897726in}{1.780610in}}{\pgfqpoint{1.902117in}{1.791209in}}{\pgfqpoint{1.902117in}{1.802259in}}%
\pgfpathcurveto{\pgfqpoint{1.902117in}{1.813309in}}{\pgfqpoint{1.897726in}{1.823908in}}{\pgfqpoint{1.889913in}{1.831722in}}%
\pgfpathcurveto{\pgfqpoint{1.882099in}{1.839536in}}{\pgfqpoint{1.871500in}{1.843926in}}{\pgfqpoint{1.860450in}{1.843926in}}%
\pgfpathcurveto{\pgfqpoint{1.849400in}{1.843926in}}{\pgfqpoint{1.838801in}{1.839536in}}{\pgfqpoint{1.830987in}{1.831722in}}%
\pgfpathcurveto{\pgfqpoint{1.823174in}{1.823908in}}{\pgfqpoint{1.818783in}{1.813309in}}{\pgfqpoint{1.818783in}{1.802259in}}%
\pgfpathcurveto{\pgfqpoint{1.818783in}{1.791209in}}{\pgfqpoint{1.823174in}{1.780610in}}{\pgfqpoint{1.830987in}{1.772796in}}%
\pgfpathcurveto{\pgfqpoint{1.838801in}{1.764983in}}{\pgfqpoint{1.849400in}{1.760592in}}{\pgfqpoint{1.860450in}{1.760592in}}%
\pgfpathclose%
\pgfusepath{stroke,fill}%
\end{pgfscope}%
\begin{pgfscope}%
\pgfpathrectangle{\pgfqpoint{0.600000in}{0.600000in}}{\pgfqpoint{3.900000in}{3.900000in}}%
\pgfusepath{clip}%
\pgfsetbuttcap%
\pgfsetroundjoin%
\definecolor{currentfill}{rgb}{0.121569,0.466667,0.705882}%
\pgfsetfillcolor{currentfill}%
\pgfsetlinewidth{1.003750pt}%
\definecolor{currentstroke}{rgb}{0.121569,0.466667,0.705882}%
\pgfsetstrokecolor{currentstroke}%
\pgfsetdash{}{0pt}%
\pgfpathmoveto{\pgfqpoint{2.147005in}{2.688419in}}%
\pgfpathcurveto{\pgfqpoint{2.158055in}{2.688419in}}{\pgfqpoint{2.168654in}{2.692809in}}{\pgfqpoint{2.176467in}{2.700623in}}%
\pgfpathcurveto{\pgfqpoint{2.184281in}{2.708436in}}{\pgfqpoint{2.188671in}{2.719035in}}{\pgfqpoint{2.188671in}{2.730086in}}%
\pgfpathcurveto{\pgfqpoint{2.188671in}{2.741136in}}{\pgfqpoint{2.184281in}{2.751735in}}{\pgfqpoint{2.176467in}{2.759548in}}%
\pgfpathcurveto{\pgfqpoint{2.168654in}{2.767362in}}{\pgfqpoint{2.158055in}{2.771752in}}{\pgfqpoint{2.147005in}{2.771752in}}%
\pgfpathcurveto{\pgfqpoint{2.135954in}{2.771752in}}{\pgfqpoint{2.125355in}{2.767362in}}{\pgfqpoint{2.117542in}{2.759548in}}%
\pgfpathcurveto{\pgfqpoint{2.109728in}{2.751735in}}{\pgfqpoint{2.105338in}{2.741136in}}{\pgfqpoint{2.105338in}{2.730086in}}%
\pgfpathcurveto{\pgfqpoint{2.105338in}{2.719035in}}{\pgfqpoint{2.109728in}{2.708436in}}{\pgfqpoint{2.117542in}{2.700623in}}%
\pgfpathcurveto{\pgfqpoint{2.125355in}{2.692809in}}{\pgfqpoint{2.135954in}{2.688419in}}{\pgfqpoint{2.147005in}{2.688419in}}%
\pgfpathclose%
\pgfusepath{stroke,fill}%
\end{pgfscope}%
\begin{pgfscope}%
\pgfpathrectangle{\pgfqpoint{0.600000in}{0.600000in}}{\pgfqpoint{3.900000in}{3.900000in}}%
\pgfusepath{clip}%
\pgfsetbuttcap%
\pgfsetroundjoin%
\definecolor{currentfill}{rgb}{0.121569,0.466667,0.705882}%
\pgfsetfillcolor{currentfill}%
\pgfsetlinewidth{1.003750pt}%
\definecolor{currentstroke}{rgb}{0.121569,0.466667,0.705882}%
\pgfsetstrokecolor{currentstroke}%
\pgfsetdash{}{0pt}%
\pgfpathmoveto{\pgfqpoint{2.857496in}{3.323003in}}%
\pgfpathcurveto{\pgfqpoint{2.868546in}{3.323003in}}{\pgfqpoint{2.879145in}{3.327393in}}{\pgfqpoint{2.886958in}{3.335207in}}%
\pgfpathcurveto{\pgfqpoint{2.894772in}{3.343020in}}{\pgfqpoint{2.899162in}{3.353619in}}{\pgfqpoint{2.899162in}{3.364669in}}%
\pgfpathcurveto{\pgfqpoint{2.899162in}{3.375719in}}{\pgfqpoint{2.894772in}{3.386318in}}{\pgfqpoint{2.886958in}{3.394132in}}%
\pgfpathcurveto{\pgfqpoint{2.879145in}{3.401946in}}{\pgfqpoint{2.868546in}{3.406336in}}{\pgfqpoint{2.857496in}{3.406336in}}%
\pgfpathcurveto{\pgfqpoint{2.846445in}{3.406336in}}{\pgfqpoint{2.835846in}{3.401946in}}{\pgfqpoint{2.828033in}{3.394132in}}%
\pgfpathcurveto{\pgfqpoint{2.820219in}{3.386318in}}{\pgfqpoint{2.815829in}{3.375719in}}{\pgfqpoint{2.815829in}{3.364669in}}%
\pgfpathcurveto{\pgfqpoint{2.815829in}{3.353619in}}{\pgfqpoint{2.820219in}{3.343020in}}{\pgfqpoint{2.828033in}{3.335207in}}%
\pgfpathcurveto{\pgfqpoint{2.835846in}{3.327393in}}{\pgfqpoint{2.846445in}{3.323003in}}{\pgfqpoint{2.857496in}{3.323003in}}%
\pgfpathclose%
\pgfusepath{stroke,fill}%
\end{pgfscope}%
\begin{pgfscope}%
\pgfpathrectangle{\pgfqpoint{0.600000in}{0.600000in}}{\pgfqpoint{3.900000in}{3.900000in}}%
\pgfusepath{clip}%
\pgfsetbuttcap%
\pgfsetroundjoin%
\definecolor{currentfill}{rgb}{0.121569,0.466667,0.705882}%
\pgfsetfillcolor{currentfill}%
\pgfsetlinewidth{1.003750pt}%
\definecolor{currentstroke}{rgb}{0.121569,0.466667,0.705882}%
\pgfsetstrokecolor{currentstroke}%
\pgfsetdash{}{0pt}%
\pgfpathmoveto{\pgfqpoint{1.854809in}{1.947489in}}%
\pgfpathcurveto{\pgfqpoint{1.865859in}{1.947489in}}{\pgfqpoint{1.876458in}{1.951879in}}{\pgfqpoint{1.884271in}{1.959693in}}%
\pgfpathcurveto{\pgfqpoint{1.892085in}{1.967506in}}{\pgfqpoint{1.896475in}{1.978105in}}{\pgfqpoint{1.896475in}{1.989155in}}%
\pgfpathcurveto{\pgfqpoint{1.896475in}{2.000206in}}{\pgfqpoint{1.892085in}{2.010805in}}{\pgfqpoint{1.884271in}{2.018618in}}%
\pgfpathcurveto{\pgfqpoint{1.876458in}{2.026432in}}{\pgfqpoint{1.865859in}{2.030822in}}{\pgfqpoint{1.854809in}{2.030822in}}%
\pgfpathcurveto{\pgfqpoint{1.843759in}{2.030822in}}{\pgfqpoint{1.833159in}{2.026432in}}{\pgfqpoint{1.825346in}{2.018618in}}%
\pgfpathcurveto{\pgfqpoint{1.817532in}{2.010805in}}{\pgfqpoint{1.813142in}{2.000206in}}{\pgfqpoint{1.813142in}{1.989155in}}%
\pgfpathcurveto{\pgfqpoint{1.813142in}{1.978105in}}{\pgfqpoint{1.817532in}{1.967506in}}{\pgfqpoint{1.825346in}{1.959693in}}%
\pgfpathcurveto{\pgfqpoint{1.833159in}{1.951879in}}{\pgfqpoint{1.843759in}{1.947489in}}{\pgfqpoint{1.854809in}{1.947489in}}%
\pgfpathclose%
\pgfusepath{stroke,fill}%
\end{pgfscope}%
\begin{pgfscope}%
\pgfpathrectangle{\pgfqpoint{0.600000in}{0.600000in}}{\pgfqpoint{3.900000in}{3.900000in}}%
\pgfusepath{clip}%
\pgfsetbuttcap%
\pgfsetroundjoin%
\definecolor{currentfill}{rgb}{0.121569,0.466667,0.705882}%
\pgfsetfillcolor{currentfill}%
\pgfsetlinewidth{1.003750pt}%
\definecolor{currentstroke}{rgb}{0.121569,0.466667,0.705882}%
\pgfsetstrokecolor{currentstroke}%
\pgfsetdash{}{0pt}%
\pgfpathmoveto{\pgfqpoint{2.015725in}{1.371002in}}%
\pgfpathcurveto{\pgfqpoint{2.026775in}{1.371002in}}{\pgfqpoint{2.037374in}{1.375393in}}{\pgfqpoint{2.045188in}{1.383206in}}%
\pgfpathcurveto{\pgfqpoint{2.053001in}{1.391020in}}{\pgfqpoint{2.057391in}{1.401619in}}{\pgfqpoint{2.057391in}{1.412669in}}%
\pgfpathcurveto{\pgfqpoint{2.057391in}{1.423719in}}{\pgfqpoint{2.053001in}{1.434318in}}{\pgfqpoint{2.045188in}{1.442132in}}%
\pgfpathcurveto{\pgfqpoint{2.037374in}{1.449945in}}{\pgfqpoint{2.026775in}{1.454336in}}{\pgfqpoint{2.015725in}{1.454336in}}%
\pgfpathcurveto{\pgfqpoint{2.004675in}{1.454336in}}{\pgfqpoint{1.994076in}{1.449945in}}{\pgfqpoint{1.986262in}{1.442132in}}%
\pgfpathcurveto{\pgfqpoint{1.978448in}{1.434318in}}{\pgfqpoint{1.974058in}{1.423719in}}{\pgfqpoint{1.974058in}{1.412669in}}%
\pgfpathcurveto{\pgfqpoint{1.974058in}{1.401619in}}{\pgfqpoint{1.978448in}{1.391020in}}{\pgfqpoint{1.986262in}{1.383206in}}%
\pgfpathcurveto{\pgfqpoint{1.994076in}{1.375393in}}{\pgfqpoint{2.004675in}{1.371002in}}{\pgfqpoint{2.015725in}{1.371002in}}%
\pgfpathclose%
\pgfusepath{stroke,fill}%
\end{pgfscope}%
\begin{pgfscope}%
\pgfpathrectangle{\pgfqpoint{0.600000in}{0.600000in}}{\pgfqpoint{3.900000in}{3.900000in}}%
\pgfusepath{clip}%
\pgfsetbuttcap%
\pgfsetroundjoin%
\definecolor{currentfill}{rgb}{0.121569,0.466667,0.705882}%
\pgfsetfillcolor{currentfill}%
\pgfsetlinewidth{1.003750pt}%
\definecolor{currentstroke}{rgb}{0.121569,0.466667,0.705882}%
\pgfsetstrokecolor{currentstroke}%
\pgfsetdash{}{0pt}%
\pgfpathmoveto{\pgfqpoint{1.780197in}{2.023907in}}%
\pgfpathcurveto{\pgfqpoint{1.791247in}{2.023907in}}{\pgfqpoint{1.801846in}{2.028297in}}{\pgfqpoint{1.809660in}{2.036111in}}%
\pgfpathcurveto{\pgfqpoint{1.817473in}{2.043925in}}{\pgfqpoint{1.821864in}{2.054524in}}{\pgfqpoint{1.821864in}{2.065574in}}%
\pgfpathcurveto{\pgfqpoint{1.821864in}{2.076624in}}{\pgfqpoint{1.817473in}{2.087223in}}{\pgfqpoint{1.809660in}{2.095037in}}%
\pgfpathcurveto{\pgfqpoint{1.801846in}{2.102850in}}{\pgfqpoint{1.791247in}{2.107241in}}{\pgfqpoint{1.780197in}{2.107241in}}%
\pgfpathcurveto{\pgfqpoint{1.769147in}{2.107241in}}{\pgfqpoint{1.758548in}{2.102850in}}{\pgfqpoint{1.750734in}{2.095037in}}%
\pgfpathcurveto{\pgfqpoint{1.742920in}{2.087223in}}{\pgfqpoint{1.738530in}{2.076624in}}{\pgfqpoint{1.738530in}{2.065574in}}%
\pgfpathcurveto{\pgfqpoint{1.738530in}{2.054524in}}{\pgfqpoint{1.742920in}{2.043925in}}{\pgfqpoint{1.750734in}{2.036111in}}%
\pgfpathcurveto{\pgfqpoint{1.758548in}{2.028297in}}{\pgfqpoint{1.769147in}{2.023907in}}{\pgfqpoint{1.780197in}{2.023907in}}%
\pgfpathclose%
\pgfusepath{stroke,fill}%
\end{pgfscope}%
\begin{pgfscope}%
\pgfpathrectangle{\pgfqpoint{0.600000in}{0.600000in}}{\pgfqpoint{3.900000in}{3.900000in}}%
\pgfusepath{clip}%
\pgfsetbuttcap%
\pgfsetroundjoin%
\definecolor{currentfill}{rgb}{0.121569,0.466667,0.705882}%
\pgfsetfillcolor{currentfill}%
\pgfsetlinewidth{1.003750pt}%
\definecolor{currentstroke}{rgb}{0.121569,0.466667,0.705882}%
\pgfsetstrokecolor{currentstroke}%
\pgfsetdash{}{0pt}%
\pgfpathmoveto{\pgfqpoint{2.254194in}{2.137140in}}%
\pgfpathcurveto{\pgfqpoint{2.265244in}{2.137140in}}{\pgfqpoint{2.275843in}{2.141531in}}{\pgfqpoint{2.283657in}{2.149344in}}%
\pgfpathcurveto{\pgfqpoint{2.291470in}{2.157158in}}{\pgfqpoint{2.295861in}{2.167757in}}{\pgfqpoint{2.295861in}{2.178807in}}%
\pgfpathcurveto{\pgfqpoint{2.295861in}{2.189857in}}{\pgfqpoint{2.291470in}{2.200456in}}{\pgfqpoint{2.283657in}{2.208270in}}%
\pgfpathcurveto{\pgfqpoint{2.275843in}{2.216083in}}{\pgfqpoint{2.265244in}{2.220474in}}{\pgfqpoint{2.254194in}{2.220474in}}%
\pgfpathcurveto{\pgfqpoint{2.243144in}{2.220474in}}{\pgfqpoint{2.232545in}{2.216083in}}{\pgfqpoint{2.224731in}{2.208270in}}%
\pgfpathcurveto{\pgfqpoint{2.216918in}{2.200456in}}{\pgfqpoint{2.212527in}{2.189857in}}{\pgfqpoint{2.212527in}{2.178807in}}%
\pgfpathcurveto{\pgfqpoint{2.212527in}{2.167757in}}{\pgfqpoint{2.216918in}{2.157158in}}{\pgfqpoint{2.224731in}{2.149344in}}%
\pgfpathcurveto{\pgfqpoint{2.232545in}{2.141531in}}{\pgfqpoint{2.243144in}{2.137140in}}{\pgfqpoint{2.254194in}{2.137140in}}%
\pgfpathclose%
\pgfusepath{stroke,fill}%
\end{pgfscope}%
\begin{pgfscope}%
\pgfpathrectangle{\pgfqpoint{0.600000in}{0.600000in}}{\pgfqpoint{3.900000in}{3.900000in}}%
\pgfusepath{clip}%
\pgfsetbuttcap%
\pgfsetroundjoin%
\definecolor{currentfill}{rgb}{0.121569,0.466667,0.705882}%
\pgfsetfillcolor{currentfill}%
\pgfsetlinewidth{1.003750pt}%
\definecolor{currentstroke}{rgb}{0.121569,0.466667,0.705882}%
\pgfsetstrokecolor{currentstroke}%
\pgfsetdash{}{0pt}%
\pgfpathmoveto{\pgfqpoint{3.475322in}{3.160574in}}%
\pgfpathcurveto{\pgfqpoint{3.486372in}{3.160574in}}{\pgfqpoint{3.496971in}{3.164965in}}{\pgfqpoint{3.504785in}{3.172778in}}%
\pgfpathcurveto{\pgfqpoint{3.512599in}{3.180592in}}{\pgfqpoint{3.516989in}{3.191191in}}{\pgfqpoint{3.516989in}{3.202241in}}%
\pgfpathcurveto{\pgfqpoint{3.516989in}{3.213291in}}{\pgfqpoint{3.512599in}{3.223890in}}{\pgfqpoint{3.504785in}{3.231704in}}%
\pgfpathcurveto{\pgfqpoint{3.496971in}{3.239517in}}{\pgfqpoint{3.486372in}{3.243908in}}{\pgfqpoint{3.475322in}{3.243908in}}%
\pgfpathcurveto{\pgfqpoint{3.464272in}{3.243908in}}{\pgfqpoint{3.453673in}{3.239517in}}{\pgfqpoint{3.445860in}{3.231704in}}%
\pgfpathcurveto{\pgfqpoint{3.438046in}{3.223890in}}{\pgfqpoint{3.433656in}{3.213291in}}{\pgfqpoint{3.433656in}{3.202241in}}%
\pgfpathcurveto{\pgfqpoint{3.433656in}{3.191191in}}{\pgfqpoint{3.438046in}{3.180592in}}{\pgfqpoint{3.445860in}{3.172778in}}%
\pgfpathcurveto{\pgfqpoint{3.453673in}{3.164965in}}{\pgfqpoint{3.464272in}{3.160574in}}{\pgfqpoint{3.475322in}{3.160574in}}%
\pgfpathclose%
\pgfusepath{stroke,fill}%
\end{pgfscope}%
\begin{pgfscope}%
\pgfpathrectangle{\pgfqpoint{0.600000in}{0.600000in}}{\pgfqpoint{3.900000in}{3.900000in}}%
\pgfusepath{clip}%
\pgfsetbuttcap%
\pgfsetroundjoin%
\definecolor{currentfill}{rgb}{0.121569,0.466667,0.705882}%
\pgfsetfillcolor{currentfill}%
\pgfsetlinewidth{1.003750pt}%
\definecolor{currentstroke}{rgb}{0.121569,0.466667,0.705882}%
\pgfsetstrokecolor{currentstroke}%
\pgfsetdash{}{0pt}%
\pgfpathmoveto{\pgfqpoint{1.803324in}{2.397068in}}%
\pgfpathcurveto{\pgfqpoint{1.814374in}{2.397068in}}{\pgfqpoint{1.824973in}{2.401458in}}{\pgfqpoint{1.832787in}{2.409272in}}%
\pgfpathcurveto{\pgfqpoint{1.840601in}{2.417085in}}{\pgfqpoint{1.844991in}{2.427684in}}{\pgfqpoint{1.844991in}{2.438735in}}%
\pgfpathcurveto{\pgfqpoint{1.844991in}{2.449785in}}{\pgfqpoint{1.840601in}{2.460384in}}{\pgfqpoint{1.832787in}{2.468197in}}%
\pgfpathcurveto{\pgfqpoint{1.824973in}{2.476011in}}{\pgfqpoint{1.814374in}{2.480401in}}{\pgfqpoint{1.803324in}{2.480401in}}%
\pgfpathcurveto{\pgfqpoint{1.792274in}{2.480401in}}{\pgfqpoint{1.781675in}{2.476011in}}{\pgfqpoint{1.773861in}{2.468197in}}%
\pgfpathcurveto{\pgfqpoint{1.766048in}{2.460384in}}{\pgfqpoint{1.761657in}{2.449785in}}{\pgfqpoint{1.761657in}{2.438735in}}%
\pgfpathcurveto{\pgfqpoint{1.761657in}{2.427684in}}{\pgfqpoint{1.766048in}{2.417085in}}{\pgfqpoint{1.773861in}{2.409272in}}%
\pgfpathcurveto{\pgfqpoint{1.781675in}{2.401458in}}{\pgfqpoint{1.792274in}{2.397068in}}{\pgfqpoint{1.803324in}{2.397068in}}%
\pgfpathclose%
\pgfusepath{stroke,fill}%
\end{pgfscope}%
\begin{pgfscope}%
\pgfpathrectangle{\pgfqpoint{0.600000in}{0.600000in}}{\pgfqpoint{3.900000in}{3.900000in}}%
\pgfusepath{clip}%
\pgfsetbuttcap%
\pgfsetroundjoin%
\definecolor{currentfill}{rgb}{0.121569,0.466667,0.705882}%
\pgfsetfillcolor{currentfill}%
\pgfsetlinewidth{1.003750pt}%
\definecolor{currentstroke}{rgb}{0.121569,0.466667,0.705882}%
\pgfsetstrokecolor{currentstroke}%
\pgfsetdash{}{0pt}%
\pgfpathmoveto{\pgfqpoint{2.633994in}{2.845184in}}%
\pgfpathcurveto{\pgfqpoint{2.645044in}{2.845184in}}{\pgfqpoint{2.655643in}{2.849574in}}{\pgfqpoint{2.663456in}{2.857388in}}%
\pgfpathcurveto{\pgfqpoint{2.671270in}{2.865201in}}{\pgfqpoint{2.675660in}{2.875800in}}{\pgfqpoint{2.675660in}{2.886850in}}%
\pgfpathcurveto{\pgfqpoint{2.675660in}{2.897900in}}{\pgfqpoint{2.671270in}{2.908500in}}{\pgfqpoint{2.663456in}{2.916313in}}%
\pgfpathcurveto{\pgfqpoint{2.655643in}{2.924127in}}{\pgfqpoint{2.645044in}{2.928517in}}{\pgfqpoint{2.633994in}{2.928517in}}%
\pgfpathcurveto{\pgfqpoint{2.622943in}{2.928517in}}{\pgfqpoint{2.612344in}{2.924127in}}{\pgfqpoint{2.604531in}{2.916313in}}%
\pgfpathcurveto{\pgfqpoint{2.596717in}{2.908500in}}{\pgfqpoint{2.592327in}{2.897900in}}{\pgfqpoint{2.592327in}{2.886850in}}%
\pgfpathcurveto{\pgfqpoint{2.592327in}{2.875800in}}{\pgfqpoint{2.596717in}{2.865201in}}{\pgfqpoint{2.604531in}{2.857388in}}%
\pgfpathcurveto{\pgfqpoint{2.612344in}{2.849574in}}{\pgfqpoint{2.622943in}{2.845184in}}{\pgfqpoint{2.633994in}{2.845184in}}%
\pgfpathclose%
\pgfusepath{stroke,fill}%
\end{pgfscope}%
\begin{pgfscope}%
\pgfpathrectangle{\pgfqpoint{0.600000in}{0.600000in}}{\pgfqpoint{3.900000in}{3.900000in}}%
\pgfusepath{clip}%
\pgfsetbuttcap%
\pgfsetroundjoin%
\definecolor{currentfill}{rgb}{0.121569,0.466667,0.705882}%
\pgfsetfillcolor{currentfill}%
\pgfsetlinewidth{1.003750pt}%
\definecolor{currentstroke}{rgb}{0.121569,0.466667,0.705882}%
\pgfsetstrokecolor{currentstroke}%
\pgfsetdash{}{0pt}%
\pgfpathmoveto{\pgfqpoint{2.226986in}{2.440665in}}%
\pgfpathcurveto{\pgfqpoint{2.238037in}{2.440665in}}{\pgfqpoint{2.248636in}{2.445055in}}{\pgfqpoint{2.256449in}{2.452868in}}%
\pgfpathcurveto{\pgfqpoint{2.264263in}{2.460682in}}{\pgfqpoint{2.268653in}{2.471281in}}{\pgfqpoint{2.268653in}{2.482331in}}%
\pgfpathcurveto{\pgfqpoint{2.268653in}{2.493381in}}{\pgfqpoint{2.264263in}{2.503980in}}{\pgfqpoint{2.256449in}{2.511794in}}%
\pgfpathcurveto{\pgfqpoint{2.248636in}{2.519608in}}{\pgfqpoint{2.238037in}{2.523998in}}{\pgfqpoint{2.226986in}{2.523998in}}%
\pgfpathcurveto{\pgfqpoint{2.215936in}{2.523998in}}{\pgfqpoint{2.205337in}{2.519608in}}{\pgfqpoint{2.197524in}{2.511794in}}%
\pgfpathcurveto{\pgfqpoint{2.189710in}{2.503980in}}{\pgfqpoint{2.185320in}{2.493381in}}{\pgfqpoint{2.185320in}{2.482331in}}%
\pgfpathcurveto{\pgfqpoint{2.185320in}{2.471281in}}{\pgfqpoint{2.189710in}{2.460682in}}{\pgfqpoint{2.197524in}{2.452868in}}%
\pgfpathcurveto{\pgfqpoint{2.205337in}{2.445055in}}{\pgfqpoint{2.215936in}{2.440665in}}{\pgfqpoint{2.226986in}{2.440665in}}%
\pgfpathclose%
\pgfusepath{stroke,fill}%
\end{pgfscope}%
\begin{pgfscope}%
\pgfpathrectangle{\pgfqpoint{0.600000in}{0.600000in}}{\pgfqpoint{3.900000in}{3.900000in}}%
\pgfusepath{clip}%
\pgfsetbuttcap%
\pgfsetroundjoin%
\definecolor{currentfill}{rgb}{0.121569,0.466667,0.705882}%
\pgfsetfillcolor{currentfill}%
\pgfsetlinewidth{1.003750pt}%
\definecolor{currentstroke}{rgb}{0.121569,0.466667,0.705882}%
\pgfsetstrokecolor{currentstroke}%
\pgfsetdash{}{0pt}%
\pgfpathmoveto{\pgfqpoint{2.471120in}{0.845301in}}%
\pgfpathcurveto{\pgfqpoint{2.482171in}{0.845301in}}{\pgfqpoint{2.492770in}{0.849692in}}{\pgfqpoint{2.500583in}{0.857505in}}%
\pgfpathcurveto{\pgfqpoint{2.508397in}{0.865319in}}{\pgfqpoint{2.512787in}{0.875918in}}{\pgfqpoint{2.512787in}{0.886968in}}%
\pgfpathcurveto{\pgfqpoint{2.512787in}{0.898018in}}{\pgfqpoint{2.508397in}{0.908617in}}{\pgfqpoint{2.500583in}{0.916431in}}%
\pgfpathcurveto{\pgfqpoint{2.492770in}{0.924244in}}{\pgfqpoint{2.482171in}{0.928635in}}{\pgfqpoint{2.471120in}{0.928635in}}%
\pgfpathcurveto{\pgfqpoint{2.460070in}{0.928635in}}{\pgfqpoint{2.449471in}{0.924244in}}{\pgfqpoint{2.441658in}{0.916431in}}%
\pgfpathcurveto{\pgfqpoint{2.433844in}{0.908617in}}{\pgfqpoint{2.429454in}{0.898018in}}{\pgfqpoint{2.429454in}{0.886968in}}%
\pgfpathcurveto{\pgfqpoint{2.429454in}{0.875918in}}{\pgfqpoint{2.433844in}{0.865319in}}{\pgfqpoint{2.441658in}{0.857505in}}%
\pgfpathcurveto{\pgfqpoint{2.449471in}{0.849692in}}{\pgfqpoint{2.460070in}{0.845301in}}{\pgfqpoint{2.471120in}{0.845301in}}%
\pgfpathclose%
\pgfusepath{stroke,fill}%
\end{pgfscope}%
\begin{pgfscope}%
\pgfpathrectangle{\pgfqpoint{0.600000in}{0.600000in}}{\pgfqpoint{3.900000in}{3.900000in}}%
\pgfusepath{clip}%
\pgfsetbuttcap%
\pgfsetroundjoin%
\definecolor{currentfill}{rgb}{0.121569,0.466667,0.705882}%
\pgfsetfillcolor{currentfill}%
\pgfsetlinewidth{1.003750pt}%
\definecolor{currentstroke}{rgb}{0.121569,0.466667,0.705882}%
\pgfsetstrokecolor{currentstroke}%
\pgfsetdash{}{0pt}%
\pgfpathmoveto{\pgfqpoint{2.708041in}{2.775305in}}%
\pgfpathcurveto{\pgfqpoint{2.719091in}{2.775305in}}{\pgfqpoint{2.729690in}{2.779695in}}{\pgfqpoint{2.737504in}{2.787509in}}%
\pgfpathcurveto{\pgfqpoint{2.745317in}{2.795323in}}{\pgfqpoint{2.749708in}{2.805922in}}{\pgfqpoint{2.749708in}{2.816972in}}%
\pgfpathcurveto{\pgfqpoint{2.749708in}{2.828022in}}{\pgfqpoint{2.745317in}{2.838621in}}{\pgfqpoint{2.737504in}{2.846434in}}%
\pgfpathcurveto{\pgfqpoint{2.729690in}{2.854248in}}{\pgfqpoint{2.719091in}{2.858638in}}{\pgfqpoint{2.708041in}{2.858638in}}%
\pgfpathcurveto{\pgfqpoint{2.696991in}{2.858638in}}{\pgfqpoint{2.686392in}{2.854248in}}{\pgfqpoint{2.678578in}{2.846434in}}%
\pgfpathcurveto{\pgfqpoint{2.670765in}{2.838621in}}{\pgfqpoint{2.666374in}{2.828022in}}{\pgfqpoint{2.666374in}{2.816972in}}%
\pgfpathcurveto{\pgfqpoint{2.666374in}{2.805922in}}{\pgfqpoint{2.670765in}{2.795323in}}{\pgfqpoint{2.678578in}{2.787509in}}%
\pgfpathcurveto{\pgfqpoint{2.686392in}{2.779695in}}{\pgfqpoint{2.696991in}{2.775305in}}{\pgfqpoint{2.708041in}{2.775305in}}%
\pgfpathclose%
\pgfusepath{stroke,fill}%
\end{pgfscope}%
\begin{pgfscope}%
\pgfpathrectangle{\pgfqpoint{0.600000in}{0.600000in}}{\pgfqpoint{3.900000in}{3.900000in}}%
\pgfusepath{clip}%
\pgfsetbuttcap%
\pgfsetroundjoin%
\definecolor{currentfill}{rgb}{0.121569,0.466667,0.705882}%
\pgfsetfillcolor{currentfill}%
\pgfsetlinewidth{1.003750pt}%
\definecolor{currentstroke}{rgb}{0.121569,0.466667,0.705882}%
\pgfsetstrokecolor{currentstroke}%
\pgfsetdash{}{0pt}%
\pgfpathmoveto{\pgfqpoint{2.646331in}{2.307093in}}%
\pgfpathcurveto{\pgfqpoint{2.657381in}{2.307093in}}{\pgfqpoint{2.667980in}{2.311484in}}{\pgfqpoint{2.675794in}{2.319297in}}%
\pgfpathcurveto{\pgfqpoint{2.683607in}{2.327111in}}{\pgfqpoint{2.687998in}{2.337710in}}{\pgfqpoint{2.687998in}{2.348760in}}%
\pgfpathcurveto{\pgfqpoint{2.687998in}{2.359810in}}{\pgfqpoint{2.683607in}{2.370409in}}{\pgfqpoint{2.675794in}{2.378223in}}%
\pgfpathcurveto{\pgfqpoint{2.667980in}{2.386037in}}{\pgfqpoint{2.657381in}{2.390427in}}{\pgfqpoint{2.646331in}{2.390427in}}%
\pgfpathcurveto{\pgfqpoint{2.635281in}{2.390427in}}{\pgfqpoint{2.624682in}{2.386037in}}{\pgfqpoint{2.616868in}{2.378223in}}%
\pgfpathcurveto{\pgfqpoint{2.609055in}{2.370409in}}{\pgfqpoint{2.604664in}{2.359810in}}{\pgfqpoint{2.604664in}{2.348760in}}%
\pgfpathcurveto{\pgfqpoint{2.604664in}{2.337710in}}{\pgfqpoint{2.609055in}{2.327111in}}{\pgfqpoint{2.616868in}{2.319297in}}%
\pgfpathcurveto{\pgfqpoint{2.624682in}{2.311484in}}{\pgfqpoint{2.635281in}{2.307093in}}{\pgfqpoint{2.646331in}{2.307093in}}%
\pgfpathclose%
\pgfusepath{stroke,fill}%
\end{pgfscope}%
\begin{pgfscope}%
\pgfpathrectangle{\pgfqpoint{0.600000in}{0.600000in}}{\pgfqpoint{3.900000in}{3.900000in}}%
\pgfusepath{clip}%
\pgfsetbuttcap%
\pgfsetroundjoin%
\definecolor{currentfill}{rgb}{0.121569,0.466667,0.705882}%
\pgfsetfillcolor{currentfill}%
\pgfsetlinewidth{1.003750pt}%
\definecolor{currentstroke}{rgb}{0.121569,0.466667,0.705882}%
\pgfsetstrokecolor{currentstroke}%
\pgfsetdash{}{0pt}%
\pgfpathmoveto{\pgfqpoint{3.946406in}{3.756072in}}%
\pgfpathcurveto{\pgfqpoint{3.957456in}{3.756072in}}{\pgfqpoint{3.968055in}{3.760462in}}{\pgfqpoint{3.975869in}{3.768276in}}%
\pgfpathcurveto{\pgfqpoint{3.983682in}{3.776090in}}{\pgfqpoint{3.988073in}{3.786689in}}{\pgfqpoint{3.988073in}{3.797739in}}%
\pgfpathcurveto{\pgfqpoint{3.988073in}{3.808789in}}{\pgfqpoint{3.983682in}{3.819388in}}{\pgfqpoint{3.975869in}{3.827201in}}%
\pgfpathcurveto{\pgfqpoint{3.968055in}{3.835015in}}{\pgfqpoint{3.957456in}{3.839405in}}{\pgfqpoint{3.946406in}{3.839405in}}%
\pgfpathcurveto{\pgfqpoint{3.935356in}{3.839405in}}{\pgfqpoint{3.924757in}{3.835015in}}{\pgfqpoint{3.916943in}{3.827201in}}%
\pgfpathcurveto{\pgfqpoint{3.909130in}{3.819388in}}{\pgfqpoint{3.904739in}{3.808789in}}{\pgfqpoint{3.904739in}{3.797739in}}%
\pgfpathcurveto{\pgfqpoint{3.904739in}{3.786689in}}{\pgfqpoint{3.909130in}{3.776090in}}{\pgfqpoint{3.916943in}{3.768276in}}%
\pgfpathcurveto{\pgfqpoint{3.924757in}{3.760462in}}{\pgfqpoint{3.935356in}{3.756072in}}{\pgfqpoint{3.946406in}{3.756072in}}%
\pgfpathclose%
\pgfusepath{stroke,fill}%
\end{pgfscope}%
\begin{pgfscope}%
\pgfpathrectangle{\pgfqpoint{0.600000in}{0.600000in}}{\pgfqpoint{3.900000in}{3.900000in}}%
\pgfusepath{clip}%
\pgfsetbuttcap%
\pgfsetroundjoin%
\definecolor{currentfill}{rgb}{0.121569,0.466667,0.705882}%
\pgfsetfillcolor{currentfill}%
\pgfsetlinewidth{1.003750pt}%
\definecolor{currentstroke}{rgb}{0.121569,0.466667,0.705882}%
\pgfsetstrokecolor{currentstroke}%
\pgfsetdash{}{0pt}%
\pgfpathmoveto{\pgfqpoint{2.248640in}{2.349503in}}%
\pgfpathcurveto{\pgfqpoint{2.259690in}{2.349503in}}{\pgfqpoint{2.270289in}{2.353893in}}{\pgfqpoint{2.278102in}{2.361706in}}%
\pgfpathcurveto{\pgfqpoint{2.285916in}{2.369520in}}{\pgfqpoint{2.290306in}{2.380119in}}{\pgfqpoint{2.290306in}{2.391169in}}%
\pgfpathcurveto{\pgfqpoint{2.290306in}{2.402219in}}{\pgfqpoint{2.285916in}{2.412818in}}{\pgfqpoint{2.278102in}{2.420632in}}%
\pgfpathcurveto{\pgfqpoint{2.270289in}{2.428446in}}{\pgfqpoint{2.259690in}{2.432836in}}{\pgfqpoint{2.248640in}{2.432836in}}%
\pgfpathcurveto{\pgfqpoint{2.237590in}{2.432836in}}{\pgfqpoint{2.226991in}{2.428446in}}{\pgfqpoint{2.219177in}{2.420632in}}%
\pgfpathcurveto{\pgfqpoint{2.211363in}{2.412818in}}{\pgfqpoint{2.206973in}{2.402219in}}{\pgfqpoint{2.206973in}{2.391169in}}%
\pgfpathcurveto{\pgfqpoint{2.206973in}{2.380119in}}{\pgfqpoint{2.211363in}{2.369520in}}{\pgfqpoint{2.219177in}{2.361706in}}%
\pgfpathcurveto{\pgfqpoint{2.226991in}{2.353893in}}{\pgfqpoint{2.237590in}{2.349503in}}{\pgfqpoint{2.248640in}{2.349503in}}%
\pgfpathclose%
\pgfusepath{stroke,fill}%
\end{pgfscope}%
\begin{pgfscope}%
\pgfpathrectangle{\pgfqpoint{0.600000in}{0.600000in}}{\pgfqpoint{3.900000in}{3.900000in}}%
\pgfusepath{clip}%
\pgfsetbuttcap%
\pgfsetroundjoin%
\definecolor{currentfill}{rgb}{0.121569,0.466667,0.705882}%
\pgfsetfillcolor{currentfill}%
\pgfsetlinewidth{1.003750pt}%
\definecolor{currentstroke}{rgb}{0.121569,0.466667,0.705882}%
\pgfsetstrokecolor{currentstroke}%
\pgfsetdash{}{0pt}%
\pgfpathmoveto{\pgfqpoint{2.514300in}{2.376492in}}%
\pgfpathcurveto{\pgfqpoint{2.525350in}{2.376492in}}{\pgfqpoint{2.535949in}{2.380882in}}{\pgfqpoint{2.543762in}{2.388696in}}%
\pgfpathcurveto{\pgfqpoint{2.551576in}{2.396509in}}{\pgfqpoint{2.555966in}{2.407108in}}{\pgfqpoint{2.555966in}{2.418158in}}%
\pgfpathcurveto{\pgfqpoint{2.555966in}{2.429208in}}{\pgfqpoint{2.551576in}{2.439807in}}{\pgfqpoint{2.543762in}{2.447621in}}%
\pgfpathcurveto{\pgfqpoint{2.535949in}{2.455435in}}{\pgfqpoint{2.525350in}{2.459825in}}{\pgfqpoint{2.514300in}{2.459825in}}%
\pgfpathcurveto{\pgfqpoint{2.503250in}{2.459825in}}{\pgfqpoint{2.492651in}{2.455435in}}{\pgfqpoint{2.484837in}{2.447621in}}%
\pgfpathcurveto{\pgfqpoint{2.477023in}{2.439807in}}{\pgfqpoint{2.472633in}{2.429208in}}{\pgfqpoint{2.472633in}{2.418158in}}%
\pgfpathcurveto{\pgfqpoint{2.472633in}{2.407108in}}{\pgfqpoint{2.477023in}{2.396509in}}{\pgfqpoint{2.484837in}{2.388696in}}%
\pgfpathcurveto{\pgfqpoint{2.492651in}{2.380882in}}{\pgfqpoint{2.503250in}{2.376492in}}{\pgfqpoint{2.514300in}{2.376492in}}%
\pgfpathclose%
\pgfusepath{stroke,fill}%
\end{pgfscope}%
\begin{pgfscope}%
\pgfpathrectangle{\pgfqpoint{0.600000in}{0.600000in}}{\pgfqpoint{3.900000in}{3.900000in}}%
\pgfusepath{clip}%
\pgfsetbuttcap%
\pgfsetroundjoin%
\definecolor{currentfill}{rgb}{0.121569,0.466667,0.705882}%
\pgfsetfillcolor{currentfill}%
\pgfsetlinewidth{1.003750pt}%
\definecolor{currentstroke}{rgb}{0.121569,0.466667,0.705882}%
\pgfsetstrokecolor{currentstroke}%
\pgfsetdash{}{0pt}%
\pgfpathmoveto{\pgfqpoint{2.597022in}{2.403167in}}%
\pgfpathcurveto{\pgfqpoint{2.608072in}{2.403167in}}{\pgfqpoint{2.618671in}{2.407557in}}{\pgfqpoint{2.626485in}{2.415370in}}%
\pgfpathcurveto{\pgfqpoint{2.634298in}{2.423184in}}{\pgfqpoint{2.638689in}{2.433783in}}{\pgfqpoint{2.638689in}{2.444833in}}%
\pgfpathcurveto{\pgfqpoint{2.638689in}{2.455883in}}{\pgfqpoint{2.634298in}{2.466482in}}{\pgfqpoint{2.626485in}{2.474296in}}%
\pgfpathcurveto{\pgfqpoint{2.618671in}{2.482110in}}{\pgfqpoint{2.608072in}{2.486500in}}{\pgfqpoint{2.597022in}{2.486500in}}%
\pgfpathcurveto{\pgfqpoint{2.585972in}{2.486500in}}{\pgfqpoint{2.575373in}{2.482110in}}{\pgfqpoint{2.567559in}{2.474296in}}%
\pgfpathcurveto{\pgfqpoint{2.559745in}{2.466482in}}{\pgfqpoint{2.555355in}{2.455883in}}{\pgfqpoint{2.555355in}{2.444833in}}%
\pgfpathcurveto{\pgfqpoint{2.555355in}{2.433783in}}{\pgfqpoint{2.559745in}{2.423184in}}{\pgfqpoint{2.567559in}{2.415370in}}%
\pgfpathcurveto{\pgfqpoint{2.575373in}{2.407557in}}{\pgfqpoint{2.585972in}{2.403167in}}{\pgfqpoint{2.597022in}{2.403167in}}%
\pgfpathclose%
\pgfusepath{stroke,fill}%
\end{pgfscope}%
\begin{pgfscope}%
\pgfpathrectangle{\pgfqpoint{0.600000in}{0.600000in}}{\pgfqpoint{3.900000in}{3.900000in}}%
\pgfusepath{clip}%
\pgfsetbuttcap%
\pgfsetroundjoin%
\definecolor{currentfill}{rgb}{0.121569,0.466667,0.705882}%
\pgfsetfillcolor{currentfill}%
\pgfsetlinewidth{1.003750pt}%
\definecolor{currentstroke}{rgb}{0.121569,0.466667,0.705882}%
\pgfsetstrokecolor{currentstroke}%
\pgfsetdash{}{0pt}%
\pgfpathmoveto{\pgfqpoint{2.612347in}{2.836044in}}%
\pgfpathcurveto{\pgfqpoint{2.623398in}{2.836044in}}{\pgfqpoint{2.633997in}{2.840434in}}{\pgfqpoint{2.641810in}{2.848247in}}%
\pgfpathcurveto{\pgfqpoint{2.649624in}{2.856061in}}{\pgfqpoint{2.654014in}{2.866660in}}{\pgfqpoint{2.654014in}{2.877710in}}%
\pgfpathcurveto{\pgfqpoint{2.654014in}{2.888760in}}{\pgfqpoint{2.649624in}{2.899359in}}{\pgfqpoint{2.641810in}{2.907173in}}%
\pgfpathcurveto{\pgfqpoint{2.633997in}{2.914987in}}{\pgfqpoint{2.623398in}{2.919377in}}{\pgfqpoint{2.612347in}{2.919377in}}%
\pgfpathcurveto{\pgfqpoint{2.601297in}{2.919377in}}{\pgfqpoint{2.590698in}{2.914987in}}{\pgfqpoint{2.582885in}{2.907173in}}%
\pgfpathcurveto{\pgfqpoint{2.575071in}{2.899359in}}{\pgfqpoint{2.570681in}{2.888760in}}{\pgfqpoint{2.570681in}{2.877710in}}%
\pgfpathcurveto{\pgfqpoint{2.570681in}{2.866660in}}{\pgfqpoint{2.575071in}{2.856061in}}{\pgfqpoint{2.582885in}{2.848247in}}%
\pgfpathcurveto{\pgfqpoint{2.590698in}{2.840434in}}{\pgfqpoint{2.601297in}{2.836044in}}{\pgfqpoint{2.612347in}{2.836044in}}%
\pgfpathclose%
\pgfusepath{stroke,fill}%
\end{pgfscope}%
\begin{pgfscope}%
\pgfpathrectangle{\pgfqpoint{0.600000in}{0.600000in}}{\pgfqpoint{3.900000in}{3.900000in}}%
\pgfusepath{clip}%
\pgfsetbuttcap%
\pgfsetroundjoin%
\definecolor{currentfill}{rgb}{0.121569,0.466667,0.705882}%
\pgfsetfillcolor{currentfill}%
\pgfsetlinewidth{1.003750pt}%
\definecolor{currentstroke}{rgb}{0.121569,0.466667,0.705882}%
\pgfsetstrokecolor{currentstroke}%
\pgfsetdash{}{0pt}%
\pgfpathmoveto{\pgfqpoint{3.142442in}{2.420736in}}%
\pgfpathcurveto{\pgfqpoint{3.153492in}{2.420736in}}{\pgfqpoint{3.164091in}{2.425126in}}{\pgfqpoint{3.171905in}{2.432940in}}%
\pgfpathcurveto{\pgfqpoint{3.179718in}{2.440754in}}{\pgfqpoint{3.184108in}{2.451353in}}{\pgfqpoint{3.184108in}{2.462403in}}%
\pgfpathcurveto{\pgfqpoint{3.184108in}{2.473453in}}{\pgfqpoint{3.179718in}{2.484052in}}{\pgfqpoint{3.171905in}{2.491866in}}%
\pgfpathcurveto{\pgfqpoint{3.164091in}{2.499679in}}{\pgfqpoint{3.153492in}{2.504069in}}{\pgfqpoint{3.142442in}{2.504069in}}%
\pgfpathcurveto{\pgfqpoint{3.131392in}{2.504069in}}{\pgfqpoint{3.120793in}{2.499679in}}{\pgfqpoint{3.112979in}{2.491866in}}%
\pgfpathcurveto{\pgfqpoint{3.105165in}{2.484052in}}{\pgfqpoint{3.100775in}{2.473453in}}{\pgfqpoint{3.100775in}{2.462403in}}%
\pgfpathcurveto{\pgfqpoint{3.100775in}{2.451353in}}{\pgfqpoint{3.105165in}{2.440754in}}{\pgfqpoint{3.112979in}{2.432940in}}%
\pgfpathcurveto{\pgfqpoint{3.120793in}{2.425126in}}{\pgfqpoint{3.131392in}{2.420736in}}{\pgfqpoint{3.142442in}{2.420736in}}%
\pgfpathclose%
\pgfusepath{stroke,fill}%
\end{pgfscope}%
\begin{pgfscope}%
\pgfpathrectangle{\pgfqpoint{0.600000in}{0.600000in}}{\pgfqpoint{3.900000in}{3.900000in}}%
\pgfusepath{clip}%
\pgfsetbuttcap%
\pgfsetroundjoin%
\definecolor{currentfill}{rgb}{0.121569,0.466667,0.705882}%
\pgfsetfillcolor{currentfill}%
\pgfsetlinewidth{1.003750pt}%
\definecolor{currentstroke}{rgb}{0.121569,0.466667,0.705882}%
\pgfsetstrokecolor{currentstroke}%
\pgfsetdash{}{0pt}%
\pgfpathmoveto{\pgfqpoint{2.700381in}{3.193044in}}%
\pgfpathcurveto{\pgfqpoint{2.711431in}{3.193044in}}{\pgfqpoint{2.722030in}{3.197434in}}{\pgfqpoint{2.729844in}{3.205248in}}%
\pgfpathcurveto{\pgfqpoint{2.737657in}{3.213061in}}{\pgfqpoint{2.742048in}{3.223660in}}{\pgfqpoint{2.742048in}{3.234710in}}%
\pgfpathcurveto{\pgfqpoint{2.742048in}{3.245760in}}{\pgfqpoint{2.737657in}{3.256360in}}{\pgfqpoint{2.729844in}{3.264173in}}%
\pgfpathcurveto{\pgfqpoint{2.722030in}{3.271987in}}{\pgfqpoint{2.711431in}{3.276377in}}{\pgfqpoint{2.700381in}{3.276377in}}%
\pgfpathcurveto{\pgfqpoint{2.689331in}{3.276377in}}{\pgfqpoint{2.678732in}{3.271987in}}{\pgfqpoint{2.670918in}{3.264173in}}%
\pgfpathcurveto{\pgfqpoint{2.663105in}{3.256360in}}{\pgfqpoint{2.658714in}{3.245760in}}{\pgfqpoint{2.658714in}{3.234710in}}%
\pgfpathcurveto{\pgfqpoint{2.658714in}{3.223660in}}{\pgfqpoint{2.663105in}{3.213061in}}{\pgfqpoint{2.670918in}{3.205248in}}%
\pgfpathcurveto{\pgfqpoint{2.678732in}{3.197434in}}{\pgfqpoint{2.689331in}{3.193044in}}{\pgfqpoint{2.700381in}{3.193044in}}%
\pgfpathclose%
\pgfusepath{stroke,fill}%
\end{pgfscope}%
\begin{pgfscope}%
\pgfpathrectangle{\pgfqpoint{0.600000in}{0.600000in}}{\pgfqpoint{3.900000in}{3.900000in}}%
\pgfusepath{clip}%
\pgfsetbuttcap%
\pgfsetroundjoin%
\definecolor{currentfill}{rgb}{0.121569,0.466667,0.705882}%
\pgfsetfillcolor{currentfill}%
\pgfsetlinewidth{1.003750pt}%
\definecolor{currentstroke}{rgb}{0.121569,0.466667,0.705882}%
\pgfsetstrokecolor{currentstroke}%
\pgfsetdash{}{0pt}%
\pgfpathmoveto{\pgfqpoint{2.967071in}{1.749008in}}%
\pgfpathcurveto{\pgfqpoint{2.978121in}{1.749008in}}{\pgfqpoint{2.988720in}{1.753399in}}{\pgfqpoint{2.996534in}{1.761212in}}%
\pgfpathcurveto{\pgfqpoint{3.004347in}{1.769026in}}{\pgfqpoint{3.008738in}{1.779625in}}{\pgfqpoint{3.008738in}{1.790675in}}%
\pgfpathcurveto{\pgfqpoint{3.008738in}{1.801725in}}{\pgfqpoint{3.004347in}{1.812324in}}{\pgfqpoint{2.996534in}{1.820138in}}%
\pgfpathcurveto{\pgfqpoint{2.988720in}{1.827951in}}{\pgfqpoint{2.978121in}{1.832342in}}{\pgfqpoint{2.967071in}{1.832342in}}%
\pgfpathcurveto{\pgfqpoint{2.956021in}{1.832342in}}{\pgfqpoint{2.945422in}{1.827951in}}{\pgfqpoint{2.937608in}{1.820138in}}%
\pgfpathcurveto{\pgfqpoint{2.929795in}{1.812324in}}{\pgfqpoint{2.925404in}{1.801725in}}{\pgfqpoint{2.925404in}{1.790675in}}%
\pgfpathcurveto{\pgfqpoint{2.925404in}{1.779625in}}{\pgfqpoint{2.929795in}{1.769026in}}{\pgfqpoint{2.937608in}{1.761212in}}%
\pgfpathcurveto{\pgfqpoint{2.945422in}{1.753399in}}{\pgfqpoint{2.956021in}{1.749008in}}{\pgfqpoint{2.967071in}{1.749008in}}%
\pgfpathclose%
\pgfusepath{stroke,fill}%
\end{pgfscope}%
\begin{pgfscope}%
\pgfpathrectangle{\pgfqpoint{0.600000in}{0.600000in}}{\pgfqpoint{3.900000in}{3.900000in}}%
\pgfusepath{clip}%
\pgfsetbuttcap%
\pgfsetroundjoin%
\definecolor{currentfill}{rgb}{0.121569,0.466667,0.705882}%
\pgfsetfillcolor{currentfill}%
\pgfsetlinewidth{1.003750pt}%
\definecolor{currentstroke}{rgb}{0.121569,0.466667,0.705882}%
\pgfsetstrokecolor{currentstroke}%
\pgfsetdash{}{0pt}%
\pgfpathmoveto{\pgfqpoint{2.638817in}{3.283247in}}%
\pgfpathcurveto{\pgfqpoint{2.649867in}{3.283247in}}{\pgfqpoint{2.660466in}{3.287638in}}{\pgfqpoint{2.668280in}{3.295451in}}%
\pgfpathcurveto{\pgfqpoint{2.676093in}{3.303265in}}{\pgfqpoint{2.680483in}{3.313864in}}{\pgfqpoint{2.680483in}{3.324914in}}%
\pgfpathcurveto{\pgfqpoint{2.680483in}{3.335964in}}{\pgfqpoint{2.676093in}{3.346563in}}{\pgfqpoint{2.668280in}{3.354377in}}%
\pgfpathcurveto{\pgfqpoint{2.660466in}{3.362191in}}{\pgfqpoint{2.649867in}{3.366581in}}{\pgfqpoint{2.638817in}{3.366581in}}%
\pgfpathcurveto{\pgfqpoint{2.627767in}{3.366581in}}{\pgfqpoint{2.617168in}{3.362191in}}{\pgfqpoint{2.609354in}{3.354377in}}%
\pgfpathcurveto{\pgfqpoint{2.601540in}{3.346563in}}{\pgfqpoint{2.597150in}{3.335964in}}{\pgfqpoint{2.597150in}{3.324914in}}%
\pgfpathcurveto{\pgfqpoint{2.597150in}{3.313864in}}{\pgfqpoint{2.601540in}{3.303265in}}{\pgfqpoint{2.609354in}{3.295451in}}%
\pgfpathcurveto{\pgfqpoint{2.617168in}{3.287638in}}{\pgfqpoint{2.627767in}{3.283247in}}{\pgfqpoint{2.638817in}{3.283247in}}%
\pgfpathclose%
\pgfusepath{stroke,fill}%
\end{pgfscope}%
\begin{pgfscope}%
\pgfpathrectangle{\pgfqpoint{0.600000in}{0.600000in}}{\pgfqpoint{3.900000in}{3.900000in}}%
\pgfusepath{clip}%
\pgfsetbuttcap%
\pgfsetroundjoin%
\definecolor{currentfill}{rgb}{0.121569,0.466667,0.705882}%
\pgfsetfillcolor{currentfill}%
\pgfsetlinewidth{1.003750pt}%
\definecolor{currentstroke}{rgb}{0.121569,0.466667,0.705882}%
\pgfsetstrokecolor{currentstroke}%
\pgfsetdash{}{0pt}%
\pgfpathmoveto{\pgfqpoint{2.260629in}{2.873341in}}%
\pgfpathcurveto{\pgfqpoint{2.271679in}{2.873341in}}{\pgfqpoint{2.282278in}{2.877731in}}{\pgfqpoint{2.290092in}{2.885545in}}%
\pgfpathcurveto{\pgfqpoint{2.297906in}{2.893358in}}{\pgfqpoint{2.302296in}{2.903957in}}{\pgfqpoint{2.302296in}{2.915008in}}%
\pgfpathcurveto{\pgfqpoint{2.302296in}{2.926058in}}{\pgfqpoint{2.297906in}{2.936657in}}{\pgfqpoint{2.290092in}{2.944470in}}%
\pgfpathcurveto{\pgfqpoint{2.282278in}{2.952284in}}{\pgfqpoint{2.271679in}{2.956674in}}{\pgfqpoint{2.260629in}{2.956674in}}%
\pgfpathcurveto{\pgfqpoint{2.249579in}{2.956674in}}{\pgfqpoint{2.238980in}{2.952284in}}{\pgfqpoint{2.231166in}{2.944470in}}%
\pgfpathcurveto{\pgfqpoint{2.223353in}{2.936657in}}{\pgfqpoint{2.218963in}{2.926058in}}{\pgfqpoint{2.218963in}{2.915008in}}%
\pgfpathcurveto{\pgfqpoint{2.218963in}{2.903957in}}{\pgfqpoint{2.223353in}{2.893358in}}{\pgfqpoint{2.231166in}{2.885545in}}%
\pgfpathcurveto{\pgfqpoint{2.238980in}{2.877731in}}{\pgfqpoint{2.249579in}{2.873341in}}{\pgfqpoint{2.260629in}{2.873341in}}%
\pgfpathclose%
\pgfusepath{stroke,fill}%
\end{pgfscope}%
\begin{pgfscope}%
\pgfpathrectangle{\pgfqpoint{0.600000in}{0.600000in}}{\pgfqpoint{3.900000in}{3.900000in}}%
\pgfusepath{clip}%
\pgfsetbuttcap%
\pgfsetroundjoin%
\definecolor{currentfill}{rgb}{0.121569,0.466667,0.705882}%
\pgfsetfillcolor{currentfill}%
\pgfsetlinewidth{1.003750pt}%
\definecolor{currentstroke}{rgb}{0.121569,0.466667,0.705882}%
\pgfsetstrokecolor{currentstroke}%
\pgfsetdash{}{0pt}%
\pgfpathmoveto{\pgfqpoint{2.430648in}{2.194738in}}%
\pgfpathcurveto{\pgfqpoint{2.441698in}{2.194738in}}{\pgfqpoint{2.452297in}{2.199128in}}{\pgfqpoint{2.460111in}{2.206942in}}%
\pgfpathcurveto{\pgfqpoint{2.467924in}{2.214756in}}{\pgfqpoint{2.472315in}{2.225355in}}{\pgfqpoint{2.472315in}{2.236405in}}%
\pgfpathcurveto{\pgfqpoint{2.472315in}{2.247455in}}{\pgfqpoint{2.467924in}{2.258054in}}{\pgfqpoint{2.460111in}{2.265867in}}%
\pgfpathcurveto{\pgfqpoint{2.452297in}{2.273681in}}{\pgfqpoint{2.441698in}{2.278071in}}{\pgfqpoint{2.430648in}{2.278071in}}%
\pgfpathcurveto{\pgfqpoint{2.419598in}{2.278071in}}{\pgfqpoint{2.408999in}{2.273681in}}{\pgfqpoint{2.401185in}{2.265867in}}%
\pgfpathcurveto{\pgfqpoint{2.393372in}{2.258054in}}{\pgfqpoint{2.388981in}{2.247455in}}{\pgfqpoint{2.388981in}{2.236405in}}%
\pgfpathcurveto{\pgfqpoint{2.388981in}{2.225355in}}{\pgfqpoint{2.393372in}{2.214756in}}{\pgfqpoint{2.401185in}{2.206942in}}%
\pgfpathcurveto{\pgfqpoint{2.408999in}{2.199128in}}{\pgfqpoint{2.419598in}{2.194738in}}{\pgfqpoint{2.430648in}{2.194738in}}%
\pgfpathclose%
\pgfusepath{stroke,fill}%
\end{pgfscope}%
\begin{pgfscope}%
\pgfpathrectangle{\pgfqpoint{0.600000in}{0.600000in}}{\pgfqpoint{3.900000in}{3.900000in}}%
\pgfusepath{clip}%
\pgfsetbuttcap%
\pgfsetroundjoin%
\definecolor{currentfill}{rgb}{0.121569,0.466667,0.705882}%
\pgfsetfillcolor{currentfill}%
\pgfsetlinewidth{1.003750pt}%
\definecolor{currentstroke}{rgb}{0.121569,0.466667,0.705882}%
\pgfsetstrokecolor{currentstroke}%
\pgfsetdash{}{0pt}%
\pgfpathmoveto{\pgfqpoint{1.943206in}{2.293288in}}%
\pgfpathcurveto{\pgfqpoint{1.954256in}{2.293288in}}{\pgfqpoint{1.964855in}{2.297678in}}{\pgfqpoint{1.972668in}{2.305491in}}%
\pgfpathcurveto{\pgfqpoint{1.980482in}{2.313305in}}{\pgfqpoint{1.984872in}{2.323904in}}{\pgfqpoint{1.984872in}{2.334954in}}%
\pgfpathcurveto{\pgfqpoint{1.984872in}{2.346004in}}{\pgfqpoint{1.980482in}{2.356603in}}{\pgfqpoint{1.972668in}{2.364417in}}%
\pgfpathcurveto{\pgfqpoint{1.964855in}{2.372231in}}{\pgfqpoint{1.954256in}{2.376621in}}{\pgfqpoint{1.943206in}{2.376621in}}%
\pgfpathcurveto{\pgfqpoint{1.932155in}{2.376621in}}{\pgfqpoint{1.921556in}{2.372231in}}{\pgfqpoint{1.913743in}{2.364417in}}%
\pgfpathcurveto{\pgfqpoint{1.905929in}{2.356603in}}{\pgfqpoint{1.901539in}{2.346004in}}{\pgfqpoint{1.901539in}{2.334954in}}%
\pgfpathcurveto{\pgfqpoint{1.901539in}{2.323904in}}{\pgfqpoint{1.905929in}{2.313305in}}{\pgfqpoint{1.913743in}{2.305491in}}%
\pgfpathcurveto{\pgfqpoint{1.921556in}{2.297678in}}{\pgfqpoint{1.932155in}{2.293288in}}{\pgfqpoint{1.943206in}{2.293288in}}%
\pgfpathclose%
\pgfusepath{stroke,fill}%
\end{pgfscope}%
\begin{pgfscope}%
\pgfpathrectangle{\pgfqpoint{0.600000in}{0.600000in}}{\pgfqpoint{3.900000in}{3.900000in}}%
\pgfusepath{clip}%
\pgfsetbuttcap%
\pgfsetroundjoin%
\definecolor{currentfill}{rgb}{0.121569,0.466667,0.705882}%
\pgfsetfillcolor{currentfill}%
\pgfsetlinewidth{1.003750pt}%
\definecolor{currentstroke}{rgb}{0.121569,0.466667,0.705882}%
\pgfsetstrokecolor{currentstroke}%
\pgfsetdash{}{0pt}%
\pgfpathmoveto{\pgfqpoint{2.681377in}{3.045860in}}%
\pgfpathcurveto{\pgfqpoint{2.692427in}{3.045860in}}{\pgfqpoint{2.703026in}{3.050250in}}{\pgfqpoint{2.710840in}{3.058064in}}%
\pgfpathcurveto{\pgfqpoint{2.718654in}{3.065877in}}{\pgfqpoint{2.723044in}{3.076476in}}{\pgfqpoint{2.723044in}{3.087526in}}%
\pgfpathcurveto{\pgfqpoint{2.723044in}{3.098577in}}{\pgfqpoint{2.718654in}{3.109176in}}{\pgfqpoint{2.710840in}{3.116989in}}%
\pgfpathcurveto{\pgfqpoint{2.703026in}{3.124803in}}{\pgfqpoint{2.692427in}{3.129193in}}{\pgfqpoint{2.681377in}{3.129193in}}%
\pgfpathcurveto{\pgfqpoint{2.670327in}{3.129193in}}{\pgfqpoint{2.659728in}{3.124803in}}{\pgfqpoint{2.651915in}{3.116989in}}%
\pgfpathcurveto{\pgfqpoint{2.644101in}{3.109176in}}{\pgfqpoint{2.639711in}{3.098577in}}{\pgfqpoint{2.639711in}{3.087526in}}%
\pgfpathcurveto{\pgfqpoint{2.639711in}{3.076476in}}{\pgfqpoint{2.644101in}{3.065877in}}{\pgfqpoint{2.651915in}{3.058064in}}%
\pgfpathcurveto{\pgfqpoint{2.659728in}{3.050250in}}{\pgfqpoint{2.670327in}{3.045860in}}{\pgfqpoint{2.681377in}{3.045860in}}%
\pgfpathclose%
\pgfusepath{stroke,fill}%
\end{pgfscope}%
\begin{pgfscope}%
\pgfpathrectangle{\pgfqpoint{0.600000in}{0.600000in}}{\pgfqpoint{3.900000in}{3.900000in}}%
\pgfusepath{clip}%
\pgfsetbuttcap%
\pgfsetroundjoin%
\definecolor{currentfill}{rgb}{0.121569,0.466667,0.705882}%
\pgfsetfillcolor{currentfill}%
\pgfsetlinewidth{1.003750pt}%
\definecolor{currentstroke}{rgb}{0.121569,0.466667,0.705882}%
\pgfsetstrokecolor{currentstroke}%
\pgfsetdash{}{0pt}%
\pgfpathmoveto{\pgfqpoint{2.174694in}{2.089207in}}%
\pgfpathcurveto{\pgfqpoint{2.185744in}{2.089207in}}{\pgfqpoint{2.196343in}{2.093598in}}{\pgfqpoint{2.204157in}{2.101411in}}%
\pgfpathcurveto{\pgfqpoint{2.211970in}{2.109225in}}{\pgfqpoint{2.216361in}{2.119824in}}{\pgfqpoint{2.216361in}{2.130874in}}%
\pgfpathcurveto{\pgfqpoint{2.216361in}{2.141924in}}{\pgfqpoint{2.211970in}{2.152523in}}{\pgfqpoint{2.204157in}{2.160337in}}%
\pgfpathcurveto{\pgfqpoint{2.196343in}{2.168150in}}{\pgfqpoint{2.185744in}{2.172541in}}{\pgfqpoint{2.174694in}{2.172541in}}%
\pgfpathcurveto{\pgfqpoint{2.163644in}{2.172541in}}{\pgfqpoint{2.153045in}{2.168150in}}{\pgfqpoint{2.145231in}{2.160337in}}%
\pgfpathcurveto{\pgfqpoint{2.137418in}{2.152523in}}{\pgfqpoint{2.133027in}{2.141924in}}{\pgfqpoint{2.133027in}{2.130874in}}%
\pgfpathcurveto{\pgfqpoint{2.133027in}{2.119824in}}{\pgfqpoint{2.137418in}{2.109225in}}{\pgfqpoint{2.145231in}{2.101411in}}%
\pgfpathcurveto{\pgfqpoint{2.153045in}{2.093598in}}{\pgfqpoint{2.163644in}{2.089207in}}{\pgfqpoint{2.174694in}{2.089207in}}%
\pgfpathclose%
\pgfusepath{stroke,fill}%
\end{pgfscope}%
\begin{pgfscope}%
\pgfpathrectangle{\pgfqpoint{0.600000in}{0.600000in}}{\pgfqpoint{3.900000in}{3.900000in}}%
\pgfusepath{clip}%
\pgfsetbuttcap%
\pgfsetroundjoin%
\definecolor{currentfill}{rgb}{0.121569,0.466667,0.705882}%
\pgfsetfillcolor{currentfill}%
\pgfsetlinewidth{1.003750pt}%
\definecolor{currentstroke}{rgb}{0.121569,0.466667,0.705882}%
\pgfsetstrokecolor{currentstroke}%
\pgfsetdash{}{0pt}%
\pgfpathmoveto{\pgfqpoint{1.644674in}{2.876291in}}%
\pgfpathcurveto{\pgfqpoint{1.655724in}{2.876291in}}{\pgfqpoint{1.666323in}{2.880681in}}{\pgfqpoint{1.674136in}{2.888495in}}%
\pgfpathcurveto{\pgfqpoint{1.681950in}{2.896308in}}{\pgfqpoint{1.686340in}{2.906907in}}{\pgfqpoint{1.686340in}{2.917957in}}%
\pgfpathcurveto{\pgfqpoint{1.686340in}{2.929007in}}{\pgfqpoint{1.681950in}{2.939606in}}{\pgfqpoint{1.674136in}{2.947420in}}%
\pgfpathcurveto{\pgfqpoint{1.666323in}{2.955234in}}{\pgfqpoint{1.655724in}{2.959624in}}{\pgfqpoint{1.644674in}{2.959624in}}%
\pgfpathcurveto{\pgfqpoint{1.633623in}{2.959624in}}{\pgfqpoint{1.623024in}{2.955234in}}{\pgfqpoint{1.615211in}{2.947420in}}%
\pgfpathcurveto{\pgfqpoint{1.607397in}{2.939606in}}{\pgfqpoint{1.603007in}{2.929007in}}{\pgfqpoint{1.603007in}{2.917957in}}%
\pgfpathcurveto{\pgfqpoint{1.603007in}{2.906907in}}{\pgfqpoint{1.607397in}{2.896308in}}{\pgfqpoint{1.615211in}{2.888495in}}%
\pgfpathcurveto{\pgfqpoint{1.623024in}{2.880681in}}{\pgfqpoint{1.633623in}{2.876291in}}{\pgfqpoint{1.644674in}{2.876291in}}%
\pgfpathclose%
\pgfusepath{stroke,fill}%
\end{pgfscope}%
\begin{pgfscope}%
\pgfpathrectangle{\pgfqpoint{0.600000in}{0.600000in}}{\pgfqpoint{3.900000in}{3.900000in}}%
\pgfusepath{clip}%
\pgfsetbuttcap%
\pgfsetroundjoin%
\definecolor{currentfill}{rgb}{0.121569,0.466667,0.705882}%
\pgfsetfillcolor{currentfill}%
\pgfsetlinewidth{1.003750pt}%
\definecolor{currentstroke}{rgb}{0.121569,0.466667,0.705882}%
\pgfsetstrokecolor{currentstroke}%
\pgfsetdash{}{0pt}%
\pgfpathmoveto{\pgfqpoint{2.690959in}{2.577605in}}%
\pgfpathcurveto{\pgfqpoint{2.702009in}{2.577605in}}{\pgfqpoint{2.712608in}{2.581995in}}{\pgfqpoint{2.720421in}{2.589809in}}%
\pgfpathcurveto{\pgfqpoint{2.728235in}{2.597622in}}{\pgfqpoint{2.732625in}{2.608222in}}{\pgfqpoint{2.732625in}{2.619272in}}%
\pgfpathcurveto{\pgfqpoint{2.732625in}{2.630322in}}{\pgfqpoint{2.728235in}{2.640921in}}{\pgfqpoint{2.720421in}{2.648734in}}%
\pgfpathcurveto{\pgfqpoint{2.712608in}{2.656548in}}{\pgfqpoint{2.702009in}{2.660938in}}{\pgfqpoint{2.690959in}{2.660938in}}%
\pgfpathcurveto{\pgfqpoint{2.679909in}{2.660938in}}{\pgfqpoint{2.669310in}{2.656548in}}{\pgfqpoint{2.661496in}{2.648734in}}%
\pgfpathcurveto{\pgfqpoint{2.653682in}{2.640921in}}{\pgfqpoint{2.649292in}{2.630322in}}{\pgfqpoint{2.649292in}{2.619272in}}%
\pgfpathcurveto{\pgfqpoint{2.649292in}{2.608222in}}{\pgfqpoint{2.653682in}{2.597622in}}{\pgfqpoint{2.661496in}{2.589809in}}%
\pgfpathcurveto{\pgfqpoint{2.669310in}{2.581995in}}{\pgfqpoint{2.679909in}{2.577605in}}{\pgfqpoint{2.690959in}{2.577605in}}%
\pgfpathclose%
\pgfusepath{stroke,fill}%
\end{pgfscope}%
\begin{pgfscope}%
\pgfpathrectangle{\pgfqpoint{0.600000in}{0.600000in}}{\pgfqpoint{3.900000in}{3.900000in}}%
\pgfusepath{clip}%
\pgfsetbuttcap%
\pgfsetroundjoin%
\definecolor{currentfill}{rgb}{0.121569,0.466667,0.705882}%
\pgfsetfillcolor{currentfill}%
\pgfsetlinewidth{1.003750pt}%
\definecolor{currentstroke}{rgb}{0.121569,0.466667,0.705882}%
\pgfsetstrokecolor{currentstroke}%
\pgfsetdash{}{0pt}%
\pgfpathmoveto{\pgfqpoint{2.443731in}{2.325920in}}%
\pgfpathcurveto{\pgfqpoint{2.454781in}{2.325920in}}{\pgfqpoint{2.465380in}{2.330310in}}{\pgfqpoint{2.473194in}{2.338124in}}%
\pgfpathcurveto{\pgfqpoint{2.481008in}{2.345937in}}{\pgfqpoint{2.485398in}{2.356536in}}{\pgfqpoint{2.485398in}{2.367586in}}%
\pgfpathcurveto{\pgfqpoint{2.485398in}{2.378636in}}{\pgfqpoint{2.481008in}{2.389235in}}{\pgfqpoint{2.473194in}{2.397049in}}%
\pgfpathcurveto{\pgfqpoint{2.465380in}{2.404863in}}{\pgfqpoint{2.454781in}{2.409253in}}{\pgfqpoint{2.443731in}{2.409253in}}%
\pgfpathcurveto{\pgfqpoint{2.432681in}{2.409253in}}{\pgfqpoint{2.422082in}{2.404863in}}{\pgfqpoint{2.414268in}{2.397049in}}%
\pgfpathcurveto{\pgfqpoint{2.406455in}{2.389235in}}{\pgfqpoint{2.402064in}{2.378636in}}{\pgfqpoint{2.402064in}{2.367586in}}%
\pgfpathcurveto{\pgfqpoint{2.402064in}{2.356536in}}{\pgfqpoint{2.406455in}{2.345937in}}{\pgfqpoint{2.414268in}{2.338124in}}%
\pgfpathcurveto{\pgfqpoint{2.422082in}{2.330310in}}{\pgfqpoint{2.432681in}{2.325920in}}{\pgfqpoint{2.443731in}{2.325920in}}%
\pgfpathclose%
\pgfusepath{stroke,fill}%
\end{pgfscope}%
\begin{pgfscope}%
\pgfpathrectangle{\pgfqpoint{0.600000in}{0.600000in}}{\pgfqpoint{3.900000in}{3.900000in}}%
\pgfusepath{clip}%
\pgfsetbuttcap%
\pgfsetroundjoin%
\definecolor{currentfill}{rgb}{0.121569,0.466667,0.705882}%
\pgfsetfillcolor{currentfill}%
\pgfsetlinewidth{1.003750pt}%
\definecolor{currentstroke}{rgb}{0.121569,0.466667,0.705882}%
\pgfsetstrokecolor{currentstroke}%
\pgfsetdash{}{0pt}%
\pgfpathmoveto{\pgfqpoint{2.770262in}{2.515915in}}%
\pgfpathcurveto{\pgfqpoint{2.781312in}{2.515915in}}{\pgfqpoint{2.791911in}{2.520305in}}{\pgfqpoint{2.799725in}{2.528119in}}%
\pgfpathcurveto{\pgfqpoint{2.807538in}{2.535932in}}{\pgfqpoint{2.811928in}{2.546531in}}{\pgfqpoint{2.811928in}{2.557581in}}%
\pgfpathcurveto{\pgfqpoint{2.811928in}{2.568632in}}{\pgfqpoint{2.807538in}{2.579231in}}{\pgfqpoint{2.799725in}{2.587044in}}%
\pgfpathcurveto{\pgfqpoint{2.791911in}{2.594858in}}{\pgfqpoint{2.781312in}{2.599248in}}{\pgfqpoint{2.770262in}{2.599248in}}%
\pgfpathcurveto{\pgfqpoint{2.759212in}{2.599248in}}{\pgfqpoint{2.748613in}{2.594858in}}{\pgfqpoint{2.740799in}{2.587044in}}%
\pgfpathcurveto{\pgfqpoint{2.732985in}{2.579231in}}{\pgfqpoint{2.728595in}{2.568632in}}{\pgfqpoint{2.728595in}{2.557581in}}%
\pgfpathcurveto{\pgfqpoint{2.728595in}{2.546531in}}{\pgfqpoint{2.732985in}{2.535932in}}{\pgfqpoint{2.740799in}{2.528119in}}%
\pgfpathcurveto{\pgfqpoint{2.748613in}{2.520305in}}{\pgfqpoint{2.759212in}{2.515915in}}{\pgfqpoint{2.770262in}{2.515915in}}%
\pgfpathclose%
\pgfusepath{stroke,fill}%
\end{pgfscope}%
\begin{pgfscope}%
\pgfpathrectangle{\pgfqpoint{0.600000in}{0.600000in}}{\pgfqpoint{3.900000in}{3.900000in}}%
\pgfusepath{clip}%
\pgfsetbuttcap%
\pgfsetroundjoin%
\definecolor{currentfill}{rgb}{0.121569,0.466667,0.705882}%
\pgfsetfillcolor{currentfill}%
\pgfsetlinewidth{1.003750pt}%
\definecolor{currentstroke}{rgb}{0.121569,0.466667,0.705882}%
\pgfsetstrokecolor{currentstroke}%
\pgfsetdash{}{0pt}%
\pgfpathmoveto{\pgfqpoint{1.471564in}{2.228881in}}%
\pgfpathcurveto{\pgfqpoint{1.482615in}{2.228881in}}{\pgfqpoint{1.493214in}{2.233271in}}{\pgfqpoint{1.501027in}{2.241085in}}%
\pgfpathcurveto{\pgfqpoint{1.508841in}{2.248899in}}{\pgfqpoint{1.513231in}{2.259498in}}{\pgfqpoint{1.513231in}{2.270548in}}%
\pgfpathcurveto{\pgfqpoint{1.513231in}{2.281598in}}{\pgfqpoint{1.508841in}{2.292197in}}{\pgfqpoint{1.501027in}{2.300011in}}%
\pgfpathcurveto{\pgfqpoint{1.493214in}{2.307824in}}{\pgfqpoint{1.482615in}{2.312214in}}{\pgfqpoint{1.471564in}{2.312214in}}%
\pgfpathcurveto{\pgfqpoint{1.460514in}{2.312214in}}{\pgfqpoint{1.449915in}{2.307824in}}{\pgfqpoint{1.442102in}{2.300011in}}%
\pgfpathcurveto{\pgfqpoint{1.434288in}{2.292197in}}{\pgfqpoint{1.429898in}{2.281598in}}{\pgfqpoint{1.429898in}{2.270548in}}%
\pgfpathcurveto{\pgfqpoint{1.429898in}{2.259498in}}{\pgfqpoint{1.434288in}{2.248899in}}{\pgfqpoint{1.442102in}{2.241085in}}%
\pgfpathcurveto{\pgfqpoint{1.449915in}{2.233271in}}{\pgfqpoint{1.460514in}{2.228881in}}{\pgfqpoint{1.471564in}{2.228881in}}%
\pgfpathclose%
\pgfusepath{stroke,fill}%
\end{pgfscope}%
\begin{pgfscope}%
\pgfpathrectangle{\pgfqpoint{0.600000in}{0.600000in}}{\pgfqpoint{3.900000in}{3.900000in}}%
\pgfusepath{clip}%
\pgfsetbuttcap%
\pgfsetroundjoin%
\definecolor{currentfill}{rgb}{0.121569,0.466667,0.705882}%
\pgfsetfillcolor{currentfill}%
\pgfsetlinewidth{1.003750pt}%
\definecolor{currentstroke}{rgb}{0.121569,0.466667,0.705882}%
\pgfsetstrokecolor{currentstroke}%
\pgfsetdash{}{0pt}%
\pgfpathmoveto{\pgfqpoint{2.260831in}{3.100720in}}%
\pgfpathcurveto{\pgfqpoint{2.271881in}{3.100720in}}{\pgfqpoint{2.282480in}{3.105110in}}{\pgfqpoint{2.290294in}{3.112923in}}%
\pgfpathcurveto{\pgfqpoint{2.298107in}{3.120737in}}{\pgfqpoint{2.302497in}{3.131336in}}{\pgfqpoint{2.302497in}{3.142386in}}%
\pgfpathcurveto{\pgfqpoint{2.302497in}{3.153436in}}{\pgfqpoint{2.298107in}{3.164035in}}{\pgfqpoint{2.290294in}{3.171849in}}%
\pgfpathcurveto{\pgfqpoint{2.282480in}{3.179663in}}{\pgfqpoint{2.271881in}{3.184053in}}{\pgfqpoint{2.260831in}{3.184053in}}%
\pgfpathcurveto{\pgfqpoint{2.249781in}{3.184053in}}{\pgfqpoint{2.239182in}{3.179663in}}{\pgfqpoint{2.231368in}{3.171849in}}%
\pgfpathcurveto{\pgfqpoint{2.223554in}{3.164035in}}{\pgfqpoint{2.219164in}{3.153436in}}{\pgfqpoint{2.219164in}{3.142386in}}%
\pgfpathcurveto{\pgfqpoint{2.219164in}{3.131336in}}{\pgfqpoint{2.223554in}{3.120737in}}{\pgfqpoint{2.231368in}{3.112923in}}%
\pgfpathcurveto{\pgfqpoint{2.239182in}{3.105110in}}{\pgfqpoint{2.249781in}{3.100720in}}{\pgfqpoint{2.260831in}{3.100720in}}%
\pgfpathclose%
\pgfusepath{stroke,fill}%
\end{pgfscope}%
\begin{pgfscope}%
\pgfpathrectangle{\pgfqpoint{0.600000in}{0.600000in}}{\pgfqpoint{3.900000in}{3.900000in}}%
\pgfusepath{clip}%
\pgfsetbuttcap%
\pgfsetroundjoin%
\definecolor{currentfill}{rgb}{0.121569,0.466667,0.705882}%
\pgfsetfillcolor{currentfill}%
\pgfsetlinewidth{1.003750pt}%
\definecolor{currentstroke}{rgb}{0.121569,0.466667,0.705882}%
\pgfsetstrokecolor{currentstroke}%
\pgfsetdash{}{0pt}%
\pgfpathmoveto{\pgfqpoint{1.897610in}{2.810162in}}%
\pgfpathcurveto{\pgfqpoint{1.908660in}{2.810162in}}{\pgfqpoint{1.919259in}{2.814552in}}{\pgfqpoint{1.927073in}{2.822366in}}%
\pgfpathcurveto{\pgfqpoint{1.934886in}{2.830180in}}{\pgfqpoint{1.939277in}{2.840779in}}{\pgfqpoint{1.939277in}{2.851829in}}%
\pgfpathcurveto{\pgfqpoint{1.939277in}{2.862879in}}{\pgfqpoint{1.934886in}{2.873478in}}{\pgfqpoint{1.927073in}{2.881291in}}%
\pgfpathcurveto{\pgfqpoint{1.919259in}{2.889105in}}{\pgfqpoint{1.908660in}{2.893495in}}{\pgfqpoint{1.897610in}{2.893495in}}%
\pgfpathcurveto{\pgfqpoint{1.886560in}{2.893495in}}{\pgfqpoint{1.875961in}{2.889105in}}{\pgfqpoint{1.868147in}{2.881291in}}%
\pgfpathcurveto{\pgfqpoint{1.860334in}{2.873478in}}{\pgfqpoint{1.855943in}{2.862879in}}{\pgfqpoint{1.855943in}{2.851829in}}%
\pgfpathcurveto{\pgfqpoint{1.855943in}{2.840779in}}{\pgfqpoint{1.860334in}{2.830180in}}{\pgfqpoint{1.868147in}{2.822366in}}%
\pgfpathcurveto{\pgfqpoint{1.875961in}{2.814552in}}{\pgfqpoint{1.886560in}{2.810162in}}{\pgfqpoint{1.897610in}{2.810162in}}%
\pgfpathclose%
\pgfusepath{stroke,fill}%
\end{pgfscope}%
\begin{pgfscope}%
\pgfpathrectangle{\pgfqpoint{0.600000in}{0.600000in}}{\pgfqpoint{3.900000in}{3.900000in}}%
\pgfusepath{clip}%
\pgfsetbuttcap%
\pgfsetroundjoin%
\definecolor{currentfill}{rgb}{0.121569,0.466667,0.705882}%
\pgfsetfillcolor{currentfill}%
\pgfsetlinewidth{1.003750pt}%
\definecolor{currentstroke}{rgb}{0.121569,0.466667,0.705882}%
\pgfsetstrokecolor{currentstroke}%
\pgfsetdash{}{0pt}%
\pgfpathmoveto{\pgfqpoint{1.912208in}{2.452033in}}%
\pgfpathcurveto{\pgfqpoint{1.923258in}{2.452033in}}{\pgfqpoint{1.933857in}{2.456423in}}{\pgfqpoint{1.941671in}{2.464237in}}%
\pgfpathcurveto{\pgfqpoint{1.949484in}{2.472050in}}{\pgfqpoint{1.953874in}{2.482649in}}{\pgfqpoint{1.953874in}{2.493700in}}%
\pgfpathcurveto{\pgfqpoint{1.953874in}{2.504750in}}{\pgfqpoint{1.949484in}{2.515349in}}{\pgfqpoint{1.941671in}{2.523162in}}%
\pgfpathcurveto{\pgfqpoint{1.933857in}{2.530976in}}{\pgfqpoint{1.923258in}{2.535366in}}{\pgfqpoint{1.912208in}{2.535366in}}%
\pgfpathcurveto{\pgfqpoint{1.901158in}{2.535366in}}{\pgfqpoint{1.890559in}{2.530976in}}{\pgfqpoint{1.882745in}{2.523162in}}%
\pgfpathcurveto{\pgfqpoint{1.874931in}{2.515349in}}{\pgfqpoint{1.870541in}{2.504750in}}{\pgfqpoint{1.870541in}{2.493700in}}%
\pgfpathcurveto{\pgfqpoint{1.870541in}{2.482649in}}{\pgfqpoint{1.874931in}{2.472050in}}{\pgfqpoint{1.882745in}{2.464237in}}%
\pgfpathcurveto{\pgfqpoint{1.890559in}{2.456423in}}{\pgfqpoint{1.901158in}{2.452033in}}{\pgfqpoint{1.912208in}{2.452033in}}%
\pgfpathclose%
\pgfusepath{stroke,fill}%
\end{pgfscope}%
\begin{pgfscope}%
\pgfpathrectangle{\pgfqpoint{0.600000in}{0.600000in}}{\pgfqpoint{3.900000in}{3.900000in}}%
\pgfusepath{clip}%
\pgfsetbuttcap%
\pgfsetroundjoin%
\definecolor{currentfill}{rgb}{0.121569,0.466667,0.705882}%
\pgfsetfillcolor{currentfill}%
\pgfsetlinewidth{1.003750pt}%
\definecolor{currentstroke}{rgb}{0.121569,0.466667,0.705882}%
\pgfsetstrokecolor{currentstroke}%
\pgfsetdash{}{0pt}%
\pgfpathmoveto{\pgfqpoint{1.994777in}{1.694453in}}%
\pgfpathcurveto{\pgfqpoint{2.005827in}{1.694453in}}{\pgfqpoint{2.016426in}{1.698843in}}{\pgfqpoint{2.024239in}{1.706657in}}%
\pgfpathcurveto{\pgfqpoint{2.032053in}{1.714471in}}{\pgfqpoint{2.036443in}{1.725070in}}{\pgfqpoint{2.036443in}{1.736120in}}%
\pgfpathcurveto{\pgfqpoint{2.036443in}{1.747170in}}{\pgfqpoint{2.032053in}{1.757769in}}{\pgfqpoint{2.024239in}{1.765583in}}%
\pgfpathcurveto{\pgfqpoint{2.016426in}{1.773396in}}{\pgfqpoint{2.005827in}{1.777786in}}{\pgfqpoint{1.994777in}{1.777786in}}%
\pgfpathcurveto{\pgfqpoint{1.983727in}{1.777786in}}{\pgfqpoint{1.973127in}{1.773396in}}{\pgfqpoint{1.965314in}{1.765583in}}%
\pgfpathcurveto{\pgfqpoint{1.957500in}{1.757769in}}{\pgfqpoint{1.953110in}{1.747170in}}{\pgfqpoint{1.953110in}{1.736120in}}%
\pgfpathcurveto{\pgfqpoint{1.953110in}{1.725070in}}{\pgfqpoint{1.957500in}{1.714471in}}{\pgfqpoint{1.965314in}{1.706657in}}%
\pgfpathcurveto{\pgfqpoint{1.973127in}{1.698843in}}{\pgfqpoint{1.983727in}{1.694453in}}{\pgfqpoint{1.994777in}{1.694453in}}%
\pgfpathclose%
\pgfusepath{stroke,fill}%
\end{pgfscope}%
\begin{pgfscope}%
\pgfpathrectangle{\pgfqpoint{0.600000in}{0.600000in}}{\pgfqpoint{3.900000in}{3.900000in}}%
\pgfusepath{clip}%
\pgfsetbuttcap%
\pgfsetroundjoin%
\definecolor{currentfill}{rgb}{0.121569,0.466667,0.705882}%
\pgfsetfillcolor{currentfill}%
\pgfsetlinewidth{1.003750pt}%
\definecolor{currentstroke}{rgb}{0.121569,0.466667,0.705882}%
\pgfsetstrokecolor{currentstroke}%
\pgfsetdash{}{0pt}%
\pgfpathmoveto{\pgfqpoint{2.828837in}{3.424893in}}%
\pgfpathcurveto{\pgfqpoint{2.839888in}{3.424893in}}{\pgfqpoint{2.850487in}{3.429283in}}{\pgfqpoint{2.858300in}{3.437097in}}%
\pgfpathcurveto{\pgfqpoint{2.866114in}{3.444910in}}{\pgfqpoint{2.870504in}{3.455509in}}{\pgfqpoint{2.870504in}{3.466559in}}%
\pgfpathcurveto{\pgfqpoint{2.870504in}{3.477609in}}{\pgfqpoint{2.866114in}{3.488209in}}{\pgfqpoint{2.858300in}{3.496022in}}%
\pgfpathcurveto{\pgfqpoint{2.850487in}{3.503836in}}{\pgfqpoint{2.839888in}{3.508226in}}{\pgfqpoint{2.828837in}{3.508226in}}%
\pgfpathcurveto{\pgfqpoint{2.817787in}{3.508226in}}{\pgfqpoint{2.807188in}{3.503836in}}{\pgfqpoint{2.799375in}{3.496022in}}%
\pgfpathcurveto{\pgfqpoint{2.791561in}{3.488209in}}{\pgfqpoint{2.787171in}{3.477609in}}{\pgfqpoint{2.787171in}{3.466559in}}%
\pgfpathcurveto{\pgfqpoint{2.787171in}{3.455509in}}{\pgfqpoint{2.791561in}{3.444910in}}{\pgfqpoint{2.799375in}{3.437097in}}%
\pgfpathcurveto{\pgfqpoint{2.807188in}{3.429283in}}{\pgfqpoint{2.817787in}{3.424893in}}{\pgfqpoint{2.828837in}{3.424893in}}%
\pgfpathclose%
\pgfusepath{stroke,fill}%
\end{pgfscope}%
\begin{pgfscope}%
\pgfpathrectangle{\pgfqpoint{0.600000in}{0.600000in}}{\pgfqpoint{3.900000in}{3.900000in}}%
\pgfusepath{clip}%
\pgfsetbuttcap%
\pgfsetroundjoin%
\definecolor{currentfill}{rgb}{0.121569,0.466667,0.705882}%
\pgfsetfillcolor{currentfill}%
\pgfsetlinewidth{1.003750pt}%
\definecolor{currentstroke}{rgb}{0.121569,0.466667,0.705882}%
\pgfsetstrokecolor{currentstroke}%
\pgfsetdash{}{0pt}%
\pgfpathmoveto{\pgfqpoint{2.725628in}{2.261384in}}%
\pgfpathcurveto{\pgfqpoint{2.736679in}{2.261384in}}{\pgfqpoint{2.747278in}{2.265774in}}{\pgfqpoint{2.755091in}{2.273587in}}%
\pgfpathcurveto{\pgfqpoint{2.762905in}{2.281401in}}{\pgfqpoint{2.767295in}{2.292000in}}{\pgfqpoint{2.767295in}{2.303050in}}%
\pgfpathcurveto{\pgfqpoint{2.767295in}{2.314100in}}{\pgfqpoint{2.762905in}{2.324699in}}{\pgfqpoint{2.755091in}{2.332513in}}%
\pgfpathcurveto{\pgfqpoint{2.747278in}{2.340327in}}{\pgfqpoint{2.736679in}{2.344717in}}{\pgfqpoint{2.725628in}{2.344717in}}%
\pgfpathcurveto{\pgfqpoint{2.714578in}{2.344717in}}{\pgfqpoint{2.703979in}{2.340327in}}{\pgfqpoint{2.696166in}{2.332513in}}%
\pgfpathcurveto{\pgfqpoint{2.688352in}{2.324699in}}{\pgfqpoint{2.683962in}{2.314100in}}{\pgfqpoint{2.683962in}{2.303050in}}%
\pgfpathcurveto{\pgfqpoint{2.683962in}{2.292000in}}{\pgfqpoint{2.688352in}{2.281401in}}{\pgfqpoint{2.696166in}{2.273587in}}%
\pgfpathcurveto{\pgfqpoint{2.703979in}{2.265774in}}{\pgfqpoint{2.714578in}{2.261384in}}{\pgfqpoint{2.725628in}{2.261384in}}%
\pgfpathclose%
\pgfusepath{stroke,fill}%
\end{pgfscope}%
\begin{pgfscope}%
\pgfpathrectangle{\pgfqpoint{0.600000in}{0.600000in}}{\pgfqpoint{3.900000in}{3.900000in}}%
\pgfusepath{clip}%
\pgfsetbuttcap%
\pgfsetroundjoin%
\definecolor{currentfill}{rgb}{0.121569,0.466667,0.705882}%
\pgfsetfillcolor{currentfill}%
\pgfsetlinewidth{1.003750pt}%
\definecolor{currentstroke}{rgb}{0.121569,0.466667,0.705882}%
\pgfsetstrokecolor{currentstroke}%
\pgfsetdash{}{0pt}%
\pgfpathmoveto{\pgfqpoint{2.996552in}{2.105993in}}%
\pgfpathcurveto{\pgfqpoint{3.007602in}{2.105993in}}{\pgfqpoint{3.018202in}{2.110384in}}{\pgfqpoint{3.026015in}{2.118197in}}%
\pgfpathcurveto{\pgfqpoint{3.033829in}{2.126011in}}{\pgfqpoint{3.038219in}{2.136610in}}{\pgfqpoint{3.038219in}{2.147660in}}%
\pgfpathcurveto{\pgfqpoint{3.038219in}{2.158710in}}{\pgfqpoint{3.033829in}{2.169309in}}{\pgfqpoint{3.026015in}{2.177123in}}%
\pgfpathcurveto{\pgfqpoint{3.018202in}{2.184936in}}{\pgfqpoint{3.007602in}{2.189327in}}{\pgfqpoint{2.996552in}{2.189327in}}%
\pgfpathcurveto{\pgfqpoint{2.985502in}{2.189327in}}{\pgfqpoint{2.974903in}{2.184936in}}{\pgfqpoint{2.967090in}{2.177123in}}%
\pgfpathcurveto{\pgfqpoint{2.959276in}{2.169309in}}{\pgfqpoint{2.954886in}{2.158710in}}{\pgfqpoint{2.954886in}{2.147660in}}%
\pgfpathcurveto{\pgfqpoint{2.954886in}{2.136610in}}{\pgfqpoint{2.959276in}{2.126011in}}{\pgfqpoint{2.967090in}{2.118197in}}%
\pgfpathcurveto{\pgfqpoint{2.974903in}{2.110384in}}{\pgfqpoint{2.985502in}{2.105993in}}{\pgfqpoint{2.996552in}{2.105993in}}%
\pgfpathclose%
\pgfusepath{stroke,fill}%
\end{pgfscope}%
\begin{pgfscope}%
\pgfpathrectangle{\pgfqpoint{0.600000in}{0.600000in}}{\pgfqpoint{3.900000in}{3.900000in}}%
\pgfusepath{clip}%
\pgfsetbuttcap%
\pgfsetroundjoin%
\definecolor{currentfill}{rgb}{0.121569,0.466667,0.705882}%
\pgfsetfillcolor{currentfill}%
\pgfsetlinewidth{1.003750pt}%
\definecolor{currentstroke}{rgb}{0.121569,0.466667,0.705882}%
\pgfsetstrokecolor{currentstroke}%
\pgfsetdash{}{0pt}%
\pgfpathmoveto{\pgfqpoint{2.107360in}{2.300721in}}%
\pgfpathcurveto{\pgfqpoint{2.118410in}{2.300721in}}{\pgfqpoint{2.129009in}{2.305111in}}{\pgfqpoint{2.136823in}{2.312925in}}%
\pgfpathcurveto{\pgfqpoint{2.144636in}{2.320739in}}{\pgfqpoint{2.149027in}{2.331338in}}{\pgfqpoint{2.149027in}{2.342388in}}%
\pgfpathcurveto{\pgfqpoint{2.149027in}{2.353438in}}{\pgfqpoint{2.144636in}{2.364037in}}{\pgfqpoint{2.136823in}{2.371851in}}%
\pgfpathcurveto{\pgfqpoint{2.129009in}{2.379664in}}{\pgfqpoint{2.118410in}{2.384054in}}{\pgfqpoint{2.107360in}{2.384054in}}%
\pgfpathcurveto{\pgfqpoint{2.096310in}{2.384054in}}{\pgfqpoint{2.085711in}{2.379664in}}{\pgfqpoint{2.077897in}{2.371851in}}%
\pgfpathcurveto{\pgfqpoint{2.070084in}{2.364037in}}{\pgfqpoint{2.065693in}{2.353438in}}{\pgfqpoint{2.065693in}{2.342388in}}%
\pgfpathcurveto{\pgfqpoint{2.065693in}{2.331338in}}{\pgfqpoint{2.070084in}{2.320739in}}{\pgfqpoint{2.077897in}{2.312925in}}%
\pgfpathcurveto{\pgfqpoint{2.085711in}{2.305111in}}{\pgfqpoint{2.096310in}{2.300721in}}{\pgfqpoint{2.107360in}{2.300721in}}%
\pgfpathclose%
\pgfusepath{stroke,fill}%
\end{pgfscope}%
\begin{pgfscope}%
\pgfpathrectangle{\pgfqpoint{0.600000in}{0.600000in}}{\pgfqpoint{3.900000in}{3.900000in}}%
\pgfusepath{clip}%
\pgfsetbuttcap%
\pgfsetroundjoin%
\definecolor{currentfill}{rgb}{0.121569,0.466667,0.705882}%
\pgfsetfillcolor{currentfill}%
\pgfsetlinewidth{1.003750pt}%
\definecolor{currentstroke}{rgb}{0.121569,0.466667,0.705882}%
\pgfsetstrokecolor{currentstroke}%
\pgfsetdash{}{0pt}%
\pgfpathmoveto{\pgfqpoint{2.923413in}{2.135281in}}%
\pgfpathcurveto{\pgfqpoint{2.934463in}{2.135281in}}{\pgfqpoint{2.945062in}{2.139671in}}{\pgfqpoint{2.952876in}{2.147485in}}%
\pgfpathcurveto{\pgfqpoint{2.960690in}{2.155298in}}{\pgfqpoint{2.965080in}{2.165897in}}{\pgfqpoint{2.965080in}{2.176947in}}%
\pgfpathcurveto{\pgfqpoint{2.965080in}{2.187997in}}{\pgfqpoint{2.960690in}{2.198596in}}{\pgfqpoint{2.952876in}{2.206410in}}%
\pgfpathcurveto{\pgfqpoint{2.945062in}{2.214224in}}{\pgfqpoint{2.934463in}{2.218614in}}{\pgfqpoint{2.923413in}{2.218614in}}%
\pgfpathcurveto{\pgfqpoint{2.912363in}{2.218614in}}{\pgfqpoint{2.901764in}{2.214224in}}{\pgfqpoint{2.893950in}{2.206410in}}%
\pgfpathcurveto{\pgfqpoint{2.886137in}{2.198596in}}{\pgfqpoint{2.881747in}{2.187997in}}{\pgfqpoint{2.881747in}{2.176947in}}%
\pgfpathcurveto{\pgfqpoint{2.881747in}{2.165897in}}{\pgfqpoint{2.886137in}{2.155298in}}{\pgfqpoint{2.893950in}{2.147485in}}%
\pgfpathcurveto{\pgfqpoint{2.901764in}{2.139671in}}{\pgfqpoint{2.912363in}{2.135281in}}{\pgfqpoint{2.923413in}{2.135281in}}%
\pgfpathclose%
\pgfusepath{stroke,fill}%
\end{pgfscope}%
\begin{pgfscope}%
\pgfpathrectangle{\pgfqpoint{0.600000in}{0.600000in}}{\pgfqpoint{3.900000in}{3.900000in}}%
\pgfusepath{clip}%
\pgfsetbuttcap%
\pgfsetroundjoin%
\definecolor{currentfill}{rgb}{0.121569,0.466667,0.705882}%
\pgfsetfillcolor{currentfill}%
\pgfsetlinewidth{1.003750pt}%
\definecolor{currentstroke}{rgb}{0.121569,0.466667,0.705882}%
\pgfsetstrokecolor{currentstroke}%
\pgfsetdash{}{0pt}%
\pgfpathmoveto{\pgfqpoint{2.962459in}{2.648907in}}%
\pgfpathcurveto{\pgfqpoint{2.973510in}{2.648907in}}{\pgfqpoint{2.984109in}{2.653298in}}{\pgfqpoint{2.991922in}{2.661111in}}%
\pgfpathcurveto{\pgfqpoint{2.999736in}{2.668925in}}{\pgfqpoint{3.004126in}{2.679524in}}{\pgfqpoint{3.004126in}{2.690574in}}%
\pgfpathcurveto{\pgfqpoint{3.004126in}{2.701624in}}{\pgfqpoint{2.999736in}{2.712223in}}{\pgfqpoint{2.991922in}{2.720037in}}%
\pgfpathcurveto{\pgfqpoint{2.984109in}{2.727851in}}{\pgfqpoint{2.973510in}{2.732241in}}{\pgfqpoint{2.962459in}{2.732241in}}%
\pgfpathcurveto{\pgfqpoint{2.951409in}{2.732241in}}{\pgfqpoint{2.940810in}{2.727851in}}{\pgfqpoint{2.932997in}{2.720037in}}%
\pgfpathcurveto{\pgfqpoint{2.925183in}{2.712223in}}{\pgfqpoint{2.920793in}{2.701624in}}{\pgfqpoint{2.920793in}{2.690574in}}%
\pgfpathcurveto{\pgfqpoint{2.920793in}{2.679524in}}{\pgfqpoint{2.925183in}{2.668925in}}{\pgfqpoint{2.932997in}{2.661111in}}%
\pgfpathcurveto{\pgfqpoint{2.940810in}{2.653298in}}{\pgfqpoint{2.951409in}{2.648907in}}{\pgfqpoint{2.962459in}{2.648907in}}%
\pgfpathclose%
\pgfusepath{stroke,fill}%
\end{pgfscope}%
\begin{pgfscope}%
\pgfpathrectangle{\pgfqpoint{0.600000in}{0.600000in}}{\pgfqpoint{3.900000in}{3.900000in}}%
\pgfusepath{clip}%
\pgfsetbuttcap%
\pgfsetroundjoin%
\definecolor{currentfill}{rgb}{0.121569,0.466667,0.705882}%
\pgfsetfillcolor{currentfill}%
\pgfsetlinewidth{1.003750pt}%
\definecolor{currentstroke}{rgb}{0.121569,0.466667,0.705882}%
\pgfsetstrokecolor{currentstroke}%
\pgfsetdash{}{0pt}%
\pgfpathmoveto{\pgfqpoint{3.004579in}{1.543376in}}%
\pgfpathcurveto{\pgfqpoint{3.015629in}{1.543376in}}{\pgfqpoint{3.026228in}{1.547766in}}{\pgfqpoint{3.034041in}{1.555580in}}%
\pgfpathcurveto{\pgfqpoint{3.041855in}{1.563393in}}{\pgfqpoint{3.046245in}{1.573992in}}{\pgfqpoint{3.046245in}{1.585043in}}%
\pgfpathcurveto{\pgfqpoint{3.046245in}{1.596093in}}{\pgfqpoint{3.041855in}{1.606692in}}{\pgfqpoint{3.034041in}{1.614505in}}%
\pgfpathcurveto{\pgfqpoint{3.026228in}{1.622319in}}{\pgfqpoint{3.015629in}{1.626709in}}{\pgfqpoint{3.004579in}{1.626709in}}%
\pgfpathcurveto{\pgfqpoint{2.993528in}{1.626709in}}{\pgfqpoint{2.982929in}{1.622319in}}{\pgfqpoint{2.975116in}{1.614505in}}%
\pgfpathcurveto{\pgfqpoint{2.967302in}{1.606692in}}{\pgfqpoint{2.962912in}{1.596093in}}{\pgfqpoint{2.962912in}{1.585043in}}%
\pgfpathcurveto{\pgfqpoint{2.962912in}{1.573992in}}{\pgfqpoint{2.967302in}{1.563393in}}{\pgfqpoint{2.975116in}{1.555580in}}%
\pgfpathcurveto{\pgfqpoint{2.982929in}{1.547766in}}{\pgfqpoint{2.993528in}{1.543376in}}{\pgfqpoint{3.004579in}{1.543376in}}%
\pgfpathclose%
\pgfusepath{stroke,fill}%
\end{pgfscope}%
\begin{pgfscope}%
\pgfpathrectangle{\pgfqpoint{0.600000in}{0.600000in}}{\pgfqpoint{3.900000in}{3.900000in}}%
\pgfusepath{clip}%
\pgfsetbuttcap%
\pgfsetroundjoin%
\definecolor{currentfill}{rgb}{0.121569,0.466667,0.705882}%
\pgfsetfillcolor{currentfill}%
\pgfsetlinewidth{1.003750pt}%
\definecolor{currentstroke}{rgb}{0.121569,0.466667,0.705882}%
\pgfsetstrokecolor{currentstroke}%
\pgfsetdash{}{0pt}%
\pgfpathmoveto{\pgfqpoint{2.082261in}{3.226196in}}%
\pgfpathcurveto{\pgfqpoint{2.093312in}{3.226196in}}{\pgfqpoint{2.103911in}{3.230586in}}{\pgfqpoint{2.111724in}{3.238400in}}%
\pgfpathcurveto{\pgfqpoint{2.119538in}{3.246214in}}{\pgfqpoint{2.123928in}{3.256813in}}{\pgfqpoint{2.123928in}{3.267863in}}%
\pgfpathcurveto{\pgfqpoint{2.123928in}{3.278913in}}{\pgfqpoint{2.119538in}{3.289512in}}{\pgfqpoint{2.111724in}{3.297326in}}%
\pgfpathcurveto{\pgfqpoint{2.103911in}{3.305139in}}{\pgfqpoint{2.093312in}{3.309530in}}{\pgfqpoint{2.082261in}{3.309530in}}%
\pgfpathcurveto{\pgfqpoint{2.071211in}{3.309530in}}{\pgfqpoint{2.060612in}{3.305139in}}{\pgfqpoint{2.052799in}{3.297326in}}%
\pgfpathcurveto{\pgfqpoint{2.044985in}{3.289512in}}{\pgfqpoint{2.040595in}{3.278913in}}{\pgfqpoint{2.040595in}{3.267863in}}%
\pgfpathcurveto{\pgfqpoint{2.040595in}{3.256813in}}{\pgfqpoint{2.044985in}{3.246214in}}{\pgfqpoint{2.052799in}{3.238400in}}%
\pgfpathcurveto{\pgfqpoint{2.060612in}{3.230586in}}{\pgfqpoint{2.071211in}{3.226196in}}{\pgfqpoint{2.082261in}{3.226196in}}%
\pgfpathclose%
\pgfusepath{stroke,fill}%
\end{pgfscope}%
\begin{pgfscope}%
\pgfpathrectangle{\pgfqpoint{0.600000in}{0.600000in}}{\pgfqpoint{3.900000in}{3.900000in}}%
\pgfusepath{clip}%
\pgfsetbuttcap%
\pgfsetroundjoin%
\definecolor{currentfill}{rgb}{0.121569,0.466667,0.705882}%
\pgfsetfillcolor{currentfill}%
\pgfsetlinewidth{1.003750pt}%
\definecolor{currentstroke}{rgb}{0.121569,0.466667,0.705882}%
\pgfsetstrokecolor{currentstroke}%
\pgfsetdash{}{0pt}%
\pgfpathmoveto{\pgfqpoint{2.259370in}{2.393775in}}%
\pgfpathcurveto{\pgfqpoint{2.270420in}{2.393775in}}{\pgfqpoint{2.281019in}{2.398166in}}{\pgfqpoint{2.288833in}{2.405979in}}%
\pgfpathcurveto{\pgfqpoint{2.296646in}{2.413793in}}{\pgfqpoint{2.301037in}{2.424392in}}{\pgfqpoint{2.301037in}{2.435442in}}%
\pgfpathcurveto{\pgfqpoint{2.301037in}{2.446492in}}{\pgfqpoint{2.296646in}{2.457091in}}{\pgfqpoint{2.288833in}{2.464905in}}%
\pgfpathcurveto{\pgfqpoint{2.281019in}{2.472718in}}{\pgfqpoint{2.270420in}{2.477109in}}{\pgfqpoint{2.259370in}{2.477109in}}%
\pgfpathcurveto{\pgfqpoint{2.248320in}{2.477109in}}{\pgfqpoint{2.237721in}{2.472718in}}{\pgfqpoint{2.229907in}{2.464905in}}%
\pgfpathcurveto{\pgfqpoint{2.222094in}{2.457091in}}{\pgfqpoint{2.217703in}{2.446492in}}{\pgfqpoint{2.217703in}{2.435442in}}%
\pgfpathcurveto{\pgfqpoint{2.217703in}{2.424392in}}{\pgfqpoint{2.222094in}{2.413793in}}{\pgfqpoint{2.229907in}{2.405979in}}%
\pgfpathcurveto{\pgfqpoint{2.237721in}{2.398166in}}{\pgfqpoint{2.248320in}{2.393775in}}{\pgfqpoint{2.259370in}{2.393775in}}%
\pgfpathclose%
\pgfusepath{stroke,fill}%
\end{pgfscope}%
\begin{pgfscope}%
\pgfpathrectangle{\pgfqpoint{0.600000in}{0.600000in}}{\pgfqpoint{3.900000in}{3.900000in}}%
\pgfusepath{clip}%
\pgfsetbuttcap%
\pgfsetroundjoin%
\definecolor{currentfill}{rgb}{0.121569,0.466667,0.705882}%
\pgfsetfillcolor{currentfill}%
\pgfsetlinewidth{1.003750pt}%
\definecolor{currentstroke}{rgb}{0.121569,0.466667,0.705882}%
\pgfsetstrokecolor{currentstroke}%
\pgfsetdash{}{0pt}%
\pgfpathmoveto{\pgfqpoint{1.710501in}{2.271512in}}%
\pgfpathcurveto{\pgfqpoint{1.721551in}{2.271512in}}{\pgfqpoint{1.732150in}{2.275902in}}{\pgfqpoint{1.739964in}{2.283716in}}%
\pgfpathcurveto{\pgfqpoint{1.747777in}{2.291530in}}{\pgfqpoint{1.752168in}{2.302129in}}{\pgfqpoint{1.752168in}{2.313179in}}%
\pgfpathcurveto{\pgfqpoint{1.752168in}{2.324229in}}{\pgfqpoint{1.747777in}{2.334828in}}{\pgfqpoint{1.739964in}{2.342642in}}%
\pgfpathcurveto{\pgfqpoint{1.732150in}{2.350455in}}{\pgfqpoint{1.721551in}{2.354846in}}{\pgfqpoint{1.710501in}{2.354846in}}%
\pgfpathcurveto{\pgfqpoint{1.699451in}{2.354846in}}{\pgfqpoint{1.688852in}{2.350455in}}{\pgfqpoint{1.681038in}{2.342642in}}%
\pgfpathcurveto{\pgfqpoint{1.673225in}{2.334828in}}{\pgfqpoint{1.668834in}{2.324229in}}{\pgfqpoint{1.668834in}{2.313179in}}%
\pgfpathcurveto{\pgfqpoint{1.668834in}{2.302129in}}{\pgfqpoint{1.673225in}{2.291530in}}{\pgfqpoint{1.681038in}{2.283716in}}%
\pgfpathcurveto{\pgfqpoint{1.688852in}{2.275902in}}{\pgfqpoint{1.699451in}{2.271512in}}{\pgfqpoint{1.710501in}{2.271512in}}%
\pgfpathclose%
\pgfusepath{stroke,fill}%
\end{pgfscope}%
\begin{pgfscope}%
\pgfpathrectangle{\pgfqpoint{0.600000in}{0.600000in}}{\pgfqpoint{3.900000in}{3.900000in}}%
\pgfusepath{clip}%
\pgfsetbuttcap%
\pgfsetroundjoin%
\definecolor{currentfill}{rgb}{0.121569,0.466667,0.705882}%
\pgfsetfillcolor{currentfill}%
\pgfsetlinewidth{1.003750pt}%
\definecolor{currentstroke}{rgb}{0.121569,0.466667,0.705882}%
\pgfsetstrokecolor{currentstroke}%
\pgfsetdash{}{0pt}%
\pgfpathmoveto{\pgfqpoint{2.209684in}{2.943348in}}%
\pgfpathcurveto{\pgfqpoint{2.220734in}{2.943348in}}{\pgfqpoint{2.231333in}{2.947738in}}{\pgfqpoint{2.239147in}{2.955552in}}%
\pgfpathcurveto{\pgfqpoint{2.246960in}{2.963366in}}{\pgfqpoint{2.251351in}{2.973965in}}{\pgfqpoint{2.251351in}{2.985015in}}%
\pgfpathcurveto{\pgfqpoint{2.251351in}{2.996065in}}{\pgfqpoint{2.246960in}{3.006664in}}{\pgfqpoint{2.239147in}{3.014478in}}%
\pgfpathcurveto{\pgfqpoint{2.231333in}{3.022291in}}{\pgfqpoint{2.220734in}{3.026681in}}{\pgfqpoint{2.209684in}{3.026681in}}%
\pgfpathcurveto{\pgfqpoint{2.198634in}{3.026681in}}{\pgfqpoint{2.188035in}{3.022291in}}{\pgfqpoint{2.180221in}{3.014478in}}%
\pgfpathcurveto{\pgfqpoint{2.172407in}{3.006664in}}{\pgfqpoint{2.168017in}{2.996065in}}{\pgfqpoint{2.168017in}{2.985015in}}%
\pgfpathcurveto{\pgfqpoint{2.168017in}{2.973965in}}{\pgfqpoint{2.172407in}{2.963366in}}{\pgfqpoint{2.180221in}{2.955552in}}%
\pgfpathcurveto{\pgfqpoint{2.188035in}{2.947738in}}{\pgfqpoint{2.198634in}{2.943348in}}{\pgfqpoint{2.209684in}{2.943348in}}%
\pgfpathclose%
\pgfusepath{stroke,fill}%
\end{pgfscope}%
\begin{pgfscope}%
\pgfpathrectangle{\pgfqpoint{0.600000in}{0.600000in}}{\pgfqpoint{3.900000in}{3.900000in}}%
\pgfusepath{clip}%
\pgfsetbuttcap%
\pgfsetroundjoin%
\definecolor{currentfill}{rgb}{0.121569,0.466667,0.705882}%
\pgfsetfillcolor{currentfill}%
\pgfsetlinewidth{1.003750pt}%
\definecolor{currentstroke}{rgb}{0.121569,0.466667,0.705882}%
\pgfsetstrokecolor{currentstroke}%
\pgfsetdash{}{0pt}%
\pgfpathmoveto{\pgfqpoint{2.532145in}{2.443906in}}%
\pgfpathcurveto{\pgfqpoint{2.543195in}{2.443906in}}{\pgfqpoint{2.553794in}{2.448297in}}{\pgfqpoint{2.561607in}{2.456110in}}%
\pgfpathcurveto{\pgfqpoint{2.569421in}{2.463924in}}{\pgfqpoint{2.573811in}{2.474523in}}{\pgfqpoint{2.573811in}{2.485573in}}%
\pgfpathcurveto{\pgfqpoint{2.573811in}{2.496623in}}{\pgfqpoint{2.569421in}{2.507222in}}{\pgfqpoint{2.561607in}{2.515036in}}%
\pgfpathcurveto{\pgfqpoint{2.553794in}{2.522849in}}{\pgfqpoint{2.543195in}{2.527240in}}{\pgfqpoint{2.532145in}{2.527240in}}%
\pgfpathcurveto{\pgfqpoint{2.521094in}{2.527240in}}{\pgfqpoint{2.510495in}{2.522849in}}{\pgfqpoint{2.502682in}{2.515036in}}%
\pgfpathcurveto{\pgfqpoint{2.494868in}{2.507222in}}{\pgfqpoint{2.490478in}{2.496623in}}{\pgfqpoint{2.490478in}{2.485573in}}%
\pgfpathcurveto{\pgfqpoint{2.490478in}{2.474523in}}{\pgfqpoint{2.494868in}{2.463924in}}{\pgfqpoint{2.502682in}{2.456110in}}%
\pgfpathcurveto{\pgfqpoint{2.510495in}{2.448297in}}{\pgfqpoint{2.521094in}{2.443906in}}{\pgfqpoint{2.532145in}{2.443906in}}%
\pgfpathclose%
\pgfusepath{stroke,fill}%
\end{pgfscope}%
\begin{pgfscope}%
\pgfpathrectangle{\pgfqpoint{0.600000in}{0.600000in}}{\pgfqpoint{3.900000in}{3.900000in}}%
\pgfusepath{clip}%
\pgfsetbuttcap%
\pgfsetroundjoin%
\definecolor{currentfill}{rgb}{0.121569,0.466667,0.705882}%
\pgfsetfillcolor{currentfill}%
\pgfsetlinewidth{1.003750pt}%
\definecolor{currentstroke}{rgb}{0.121569,0.466667,0.705882}%
\pgfsetstrokecolor{currentstroke}%
\pgfsetdash{}{0pt}%
\pgfpathmoveto{\pgfqpoint{2.583798in}{2.101759in}}%
\pgfpathcurveto{\pgfqpoint{2.594848in}{2.101759in}}{\pgfqpoint{2.605447in}{2.106149in}}{\pgfqpoint{2.613261in}{2.113963in}}%
\pgfpathcurveto{\pgfqpoint{2.621075in}{2.121776in}}{\pgfqpoint{2.625465in}{2.132375in}}{\pgfqpoint{2.625465in}{2.143425in}}%
\pgfpathcurveto{\pgfqpoint{2.625465in}{2.154476in}}{\pgfqpoint{2.621075in}{2.165075in}}{\pgfqpoint{2.613261in}{2.172888in}}%
\pgfpathcurveto{\pgfqpoint{2.605447in}{2.180702in}}{\pgfqpoint{2.594848in}{2.185092in}}{\pgfqpoint{2.583798in}{2.185092in}}%
\pgfpathcurveto{\pgfqpoint{2.572748in}{2.185092in}}{\pgfqpoint{2.562149in}{2.180702in}}{\pgfqpoint{2.554335in}{2.172888in}}%
\pgfpathcurveto{\pgfqpoint{2.546522in}{2.165075in}}{\pgfqpoint{2.542132in}{2.154476in}}{\pgfqpoint{2.542132in}{2.143425in}}%
\pgfpathcurveto{\pgfqpoint{2.542132in}{2.132375in}}{\pgfqpoint{2.546522in}{2.121776in}}{\pgfqpoint{2.554335in}{2.113963in}}%
\pgfpathcurveto{\pgfqpoint{2.562149in}{2.106149in}}{\pgfqpoint{2.572748in}{2.101759in}}{\pgfqpoint{2.583798in}{2.101759in}}%
\pgfpathclose%
\pgfusepath{stroke,fill}%
\end{pgfscope}%
\begin{pgfscope}%
\pgfpathrectangle{\pgfqpoint{0.600000in}{0.600000in}}{\pgfqpoint{3.900000in}{3.900000in}}%
\pgfusepath{clip}%
\pgfsetbuttcap%
\pgfsetroundjoin%
\definecolor{currentfill}{rgb}{0.121569,0.466667,0.705882}%
\pgfsetfillcolor{currentfill}%
\pgfsetlinewidth{1.003750pt}%
\definecolor{currentstroke}{rgb}{0.121569,0.466667,0.705882}%
\pgfsetstrokecolor{currentstroke}%
\pgfsetdash{}{0pt}%
\pgfpathmoveto{\pgfqpoint{3.229823in}{2.713790in}}%
\pgfpathcurveto{\pgfqpoint{3.240873in}{2.713790in}}{\pgfqpoint{3.251472in}{2.718181in}}{\pgfqpoint{3.259286in}{2.725994in}}%
\pgfpathcurveto{\pgfqpoint{3.267099in}{2.733808in}}{\pgfqpoint{3.271490in}{2.744407in}}{\pgfqpoint{3.271490in}{2.755457in}}%
\pgfpathcurveto{\pgfqpoint{3.271490in}{2.766507in}}{\pgfqpoint{3.267099in}{2.777106in}}{\pgfqpoint{3.259286in}{2.784920in}}%
\pgfpathcurveto{\pgfqpoint{3.251472in}{2.792733in}}{\pgfqpoint{3.240873in}{2.797124in}}{\pgfqpoint{3.229823in}{2.797124in}}%
\pgfpathcurveto{\pgfqpoint{3.218773in}{2.797124in}}{\pgfqpoint{3.208174in}{2.792733in}}{\pgfqpoint{3.200360in}{2.784920in}}%
\pgfpathcurveto{\pgfqpoint{3.192547in}{2.777106in}}{\pgfqpoint{3.188156in}{2.766507in}}{\pgfqpoint{3.188156in}{2.755457in}}%
\pgfpathcurveto{\pgfqpoint{3.188156in}{2.744407in}}{\pgfqpoint{3.192547in}{2.733808in}}{\pgfqpoint{3.200360in}{2.725994in}}%
\pgfpathcurveto{\pgfqpoint{3.208174in}{2.718181in}}{\pgfqpoint{3.218773in}{2.713790in}}{\pgfqpoint{3.229823in}{2.713790in}}%
\pgfpathclose%
\pgfusepath{stroke,fill}%
\end{pgfscope}%
\begin{pgfscope}%
\pgfpathrectangle{\pgfqpoint{0.600000in}{0.600000in}}{\pgfqpoint{3.900000in}{3.900000in}}%
\pgfusepath{clip}%
\pgfsetbuttcap%
\pgfsetroundjoin%
\definecolor{currentfill}{rgb}{0.121569,0.466667,0.705882}%
\pgfsetfillcolor{currentfill}%
\pgfsetlinewidth{1.003750pt}%
\definecolor{currentstroke}{rgb}{0.121569,0.466667,0.705882}%
\pgfsetstrokecolor{currentstroke}%
\pgfsetdash{}{0pt}%
\pgfpathmoveto{\pgfqpoint{2.322073in}{3.219038in}}%
\pgfpathcurveto{\pgfqpoint{2.333123in}{3.219038in}}{\pgfqpoint{2.343722in}{3.223428in}}{\pgfqpoint{2.351536in}{3.231242in}}%
\pgfpathcurveto{\pgfqpoint{2.359349in}{3.239055in}}{\pgfqpoint{2.363740in}{3.249654in}}{\pgfqpoint{2.363740in}{3.260704in}}%
\pgfpathcurveto{\pgfqpoint{2.363740in}{3.271755in}}{\pgfqpoint{2.359349in}{3.282354in}}{\pgfqpoint{2.351536in}{3.290167in}}%
\pgfpathcurveto{\pgfqpoint{2.343722in}{3.297981in}}{\pgfqpoint{2.333123in}{3.302371in}}{\pgfqpoint{2.322073in}{3.302371in}}%
\pgfpathcurveto{\pgfqpoint{2.311023in}{3.302371in}}{\pgfqpoint{2.300424in}{3.297981in}}{\pgfqpoint{2.292610in}{3.290167in}}%
\pgfpathcurveto{\pgfqpoint{2.284797in}{3.282354in}}{\pgfqpoint{2.280406in}{3.271755in}}{\pgfqpoint{2.280406in}{3.260704in}}%
\pgfpathcurveto{\pgfqpoint{2.280406in}{3.249654in}}{\pgfqpoint{2.284797in}{3.239055in}}{\pgfqpoint{2.292610in}{3.231242in}}%
\pgfpathcurveto{\pgfqpoint{2.300424in}{3.223428in}}{\pgfqpoint{2.311023in}{3.219038in}}{\pgfqpoint{2.322073in}{3.219038in}}%
\pgfpathclose%
\pgfusepath{stroke,fill}%
\end{pgfscope}%
\begin{pgfscope}%
\pgfpathrectangle{\pgfqpoint{0.600000in}{0.600000in}}{\pgfqpoint{3.900000in}{3.900000in}}%
\pgfusepath{clip}%
\pgfsetbuttcap%
\pgfsetroundjoin%
\definecolor{currentfill}{rgb}{0.121569,0.466667,0.705882}%
\pgfsetfillcolor{currentfill}%
\pgfsetlinewidth{1.003750pt}%
\definecolor{currentstroke}{rgb}{0.121569,0.466667,0.705882}%
\pgfsetstrokecolor{currentstroke}%
\pgfsetdash{}{0pt}%
\pgfpathmoveto{\pgfqpoint{2.629757in}{3.087707in}}%
\pgfpathcurveto{\pgfqpoint{2.640807in}{3.087707in}}{\pgfqpoint{2.651406in}{3.092097in}}{\pgfqpoint{2.659219in}{3.099911in}}%
\pgfpathcurveto{\pgfqpoint{2.667033in}{3.107725in}}{\pgfqpoint{2.671423in}{3.118324in}}{\pgfqpoint{2.671423in}{3.129374in}}%
\pgfpathcurveto{\pgfqpoint{2.671423in}{3.140424in}}{\pgfqpoint{2.667033in}{3.151023in}}{\pgfqpoint{2.659219in}{3.158837in}}%
\pgfpathcurveto{\pgfqpoint{2.651406in}{3.166650in}}{\pgfqpoint{2.640807in}{3.171040in}}{\pgfqpoint{2.629757in}{3.171040in}}%
\pgfpathcurveto{\pgfqpoint{2.618706in}{3.171040in}}{\pgfqpoint{2.608107in}{3.166650in}}{\pgfqpoint{2.600294in}{3.158837in}}%
\pgfpathcurveto{\pgfqpoint{2.592480in}{3.151023in}}{\pgfqpoint{2.588090in}{3.140424in}}{\pgfqpoint{2.588090in}{3.129374in}}%
\pgfpathcurveto{\pgfqpoint{2.588090in}{3.118324in}}{\pgfqpoint{2.592480in}{3.107725in}}{\pgfqpoint{2.600294in}{3.099911in}}%
\pgfpathcurveto{\pgfqpoint{2.608107in}{3.092097in}}{\pgfqpoint{2.618706in}{3.087707in}}{\pgfqpoint{2.629757in}{3.087707in}}%
\pgfpathclose%
\pgfusepath{stroke,fill}%
\end{pgfscope}%
\begin{pgfscope}%
\pgfpathrectangle{\pgfqpoint{0.600000in}{0.600000in}}{\pgfqpoint{3.900000in}{3.900000in}}%
\pgfusepath{clip}%
\pgfsetbuttcap%
\pgfsetroundjoin%
\definecolor{currentfill}{rgb}{0.121569,0.466667,0.705882}%
\pgfsetfillcolor{currentfill}%
\pgfsetlinewidth{1.003750pt}%
\definecolor{currentstroke}{rgb}{0.121569,0.466667,0.705882}%
\pgfsetstrokecolor{currentstroke}%
\pgfsetdash{}{0pt}%
\pgfpathmoveto{\pgfqpoint{2.608280in}{2.294671in}}%
\pgfpathcurveto{\pgfqpoint{2.619330in}{2.294671in}}{\pgfqpoint{2.629929in}{2.299061in}}{\pgfqpoint{2.637742in}{2.306874in}}%
\pgfpathcurveto{\pgfqpoint{2.645556in}{2.314688in}}{\pgfqpoint{2.649946in}{2.325287in}}{\pgfqpoint{2.649946in}{2.336337in}}%
\pgfpathcurveto{\pgfqpoint{2.649946in}{2.347387in}}{\pgfqpoint{2.645556in}{2.357986in}}{\pgfqpoint{2.637742in}{2.365800in}}%
\pgfpathcurveto{\pgfqpoint{2.629929in}{2.373614in}}{\pgfqpoint{2.619330in}{2.378004in}}{\pgfqpoint{2.608280in}{2.378004in}}%
\pgfpathcurveto{\pgfqpoint{2.597229in}{2.378004in}}{\pgfqpoint{2.586630in}{2.373614in}}{\pgfqpoint{2.578817in}{2.365800in}}%
\pgfpathcurveto{\pgfqpoint{2.571003in}{2.357986in}}{\pgfqpoint{2.566613in}{2.347387in}}{\pgfqpoint{2.566613in}{2.336337in}}%
\pgfpathcurveto{\pgfqpoint{2.566613in}{2.325287in}}{\pgfqpoint{2.571003in}{2.314688in}}{\pgfqpoint{2.578817in}{2.306874in}}%
\pgfpathcurveto{\pgfqpoint{2.586630in}{2.299061in}}{\pgfqpoint{2.597229in}{2.294671in}}{\pgfqpoint{2.608280in}{2.294671in}}%
\pgfpathclose%
\pgfusepath{stroke,fill}%
\end{pgfscope}%
\begin{pgfscope}%
\pgfpathrectangle{\pgfqpoint{0.600000in}{0.600000in}}{\pgfqpoint{3.900000in}{3.900000in}}%
\pgfusepath{clip}%
\pgfsetbuttcap%
\pgfsetroundjoin%
\definecolor{currentfill}{rgb}{0.121569,0.466667,0.705882}%
\pgfsetfillcolor{currentfill}%
\pgfsetlinewidth{1.003750pt}%
\definecolor{currentstroke}{rgb}{0.121569,0.466667,0.705882}%
\pgfsetstrokecolor{currentstroke}%
\pgfsetdash{}{0pt}%
\pgfpathmoveto{\pgfqpoint{1.375353in}{1.570115in}}%
\pgfpathcurveto{\pgfqpoint{1.386404in}{1.570115in}}{\pgfqpoint{1.397003in}{1.574505in}}{\pgfqpoint{1.404816in}{1.582319in}}%
\pgfpathcurveto{\pgfqpoint{1.412630in}{1.590132in}}{\pgfqpoint{1.417020in}{1.600731in}}{\pgfqpoint{1.417020in}{1.611781in}}%
\pgfpathcurveto{\pgfqpoint{1.417020in}{1.622832in}}{\pgfqpoint{1.412630in}{1.633431in}}{\pgfqpoint{1.404816in}{1.641244in}}%
\pgfpathcurveto{\pgfqpoint{1.397003in}{1.649058in}}{\pgfqpoint{1.386404in}{1.653448in}}{\pgfqpoint{1.375353in}{1.653448in}}%
\pgfpathcurveto{\pgfqpoint{1.364303in}{1.653448in}}{\pgfqpoint{1.353704in}{1.649058in}}{\pgfqpoint{1.345891in}{1.641244in}}%
\pgfpathcurveto{\pgfqpoint{1.338077in}{1.633431in}}{\pgfqpoint{1.333687in}{1.622832in}}{\pgfqpoint{1.333687in}{1.611781in}}%
\pgfpathcurveto{\pgfqpoint{1.333687in}{1.600731in}}{\pgfqpoint{1.338077in}{1.590132in}}{\pgfqpoint{1.345891in}{1.582319in}}%
\pgfpathcurveto{\pgfqpoint{1.353704in}{1.574505in}}{\pgfqpoint{1.364303in}{1.570115in}}{\pgfqpoint{1.375353in}{1.570115in}}%
\pgfpathclose%
\pgfusepath{stroke,fill}%
\end{pgfscope}%
\begin{pgfscope}%
\pgfpathrectangle{\pgfqpoint{0.600000in}{0.600000in}}{\pgfqpoint{3.900000in}{3.900000in}}%
\pgfusepath{clip}%
\pgfsetbuttcap%
\pgfsetroundjoin%
\definecolor{currentfill}{rgb}{0.121569,0.466667,0.705882}%
\pgfsetfillcolor{currentfill}%
\pgfsetlinewidth{1.003750pt}%
\definecolor{currentstroke}{rgb}{0.121569,0.466667,0.705882}%
\pgfsetstrokecolor{currentstroke}%
\pgfsetdash{}{0pt}%
\pgfpathmoveto{\pgfqpoint{2.303793in}{2.833666in}}%
\pgfpathcurveto{\pgfqpoint{2.314843in}{2.833666in}}{\pgfqpoint{2.325442in}{2.838056in}}{\pgfqpoint{2.333256in}{2.845870in}}%
\pgfpathcurveto{\pgfqpoint{2.341070in}{2.853683in}}{\pgfqpoint{2.345460in}{2.864282in}}{\pgfqpoint{2.345460in}{2.875332in}}%
\pgfpathcurveto{\pgfqpoint{2.345460in}{2.886383in}}{\pgfqpoint{2.341070in}{2.896982in}}{\pgfqpoint{2.333256in}{2.904795in}}%
\pgfpathcurveto{\pgfqpoint{2.325442in}{2.912609in}}{\pgfqpoint{2.314843in}{2.916999in}}{\pgfqpoint{2.303793in}{2.916999in}}%
\pgfpathcurveto{\pgfqpoint{2.292743in}{2.916999in}}{\pgfqpoint{2.282144in}{2.912609in}}{\pgfqpoint{2.274330in}{2.904795in}}%
\pgfpathcurveto{\pgfqpoint{2.266517in}{2.896982in}}{\pgfqpoint{2.262127in}{2.886383in}}{\pgfqpoint{2.262127in}{2.875332in}}%
\pgfpathcurveto{\pgfqpoint{2.262127in}{2.864282in}}{\pgfqpoint{2.266517in}{2.853683in}}{\pgfqpoint{2.274330in}{2.845870in}}%
\pgfpathcurveto{\pgfqpoint{2.282144in}{2.838056in}}{\pgfqpoint{2.292743in}{2.833666in}}{\pgfqpoint{2.303793in}{2.833666in}}%
\pgfpathclose%
\pgfusepath{stroke,fill}%
\end{pgfscope}%
\begin{pgfscope}%
\pgfpathrectangle{\pgfqpoint{0.600000in}{0.600000in}}{\pgfqpoint{3.900000in}{3.900000in}}%
\pgfusepath{clip}%
\pgfsetbuttcap%
\pgfsetroundjoin%
\definecolor{currentfill}{rgb}{0.121569,0.466667,0.705882}%
\pgfsetfillcolor{currentfill}%
\pgfsetlinewidth{1.003750pt}%
\definecolor{currentstroke}{rgb}{0.121569,0.466667,0.705882}%
\pgfsetstrokecolor{currentstroke}%
\pgfsetdash{}{0pt}%
\pgfpathmoveto{\pgfqpoint{1.879565in}{3.076711in}}%
\pgfpathcurveto{\pgfqpoint{1.890615in}{3.076711in}}{\pgfqpoint{1.901214in}{3.081101in}}{\pgfqpoint{1.909028in}{3.088915in}}%
\pgfpathcurveto{\pgfqpoint{1.916841in}{3.096728in}}{\pgfqpoint{1.921232in}{3.107328in}}{\pgfqpoint{1.921232in}{3.118378in}}%
\pgfpathcurveto{\pgfqpoint{1.921232in}{3.129428in}}{\pgfqpoint{1.916841in}{3.140027in}}{\pgfqpoint{1.909028in}{3.147840in}}%
\pgfpathcurveto{\pgfqpoint{1.901214in}{3.155654in}}{\pgfqpoint{1.890615in}{3.160044in}}{\pgfqpoint{1.879565in}{3.160044in}}%
\pgfpathcurveto{\pgfqpoint{1.868515in}{3.160044in}}{\pgfqpoint{1.857916in}{3.155654in}}{\pgfqpoint{1.850102in}{3.147840in}}%
\pgfpathcurveto{\pgfqpoint{1.842288in}{3.140027in}}{\pgfqpoint{1.837898in}{3.129428in}}{\pgfqpoint{1.837898in}{3.118378in}}%
\pgfpathcurveto{\pgfqpoint{1.837898in}{3.107328in}}{\pgfqpoint{1.842288in}{3.096728in}}{\pgfqpoint{1.850102in}{3.088915in}}%
\pgfpathcurveto{\pgfqpoint{1.857916in}{3.081101in}}{\pgfqpoint{1.868515in}{3.076711in}}{\pgfqpoint{1.879565in}{3.076711in}}%
\pgfpathclose%
\pgfusepath{stroke,fill}%
\end{pgfscope}%
\begin{pgfscope}%
\pgfpathrectangle{\pgfqpoint{0.600000in}{0.600000in}}{\pgfqpoint{3.900000in}{3.900000in}}%
\pgfusepath{clip}%
\pgfsetbuttcap%
\pgfsetroundjoin%
\definecolor{currentfill}{rgb}{0.121569,0.466667,0.705882}%
\pgfsetfillcolor{currentfill}%
\pgfsetlinewidth{1.003750pt}%
\definecolor{currentstroke}{rgb}{0.121569,0.466667,0.705882}%
\pgfsetstrokecolor{currentstroke}%
\pgfsetdash{}{0pt}%
\pgfpathmoveto{\pgfqpoint{2.663925in}{2.869285in}}%
\pgfpathcurveto{\pgfqpoint{2.674975in}{2.869285in}}{\pgfqpoint{2.685574in}{2.873675in}}{\pgfqpoint{2.693387in}{2.881489in}}%
\pgfpathcurveto{\pgfqpoint{2.701201in}{2.889303in}}{\pgfqpoint{2.705591in}{2.899902in}}{\pgfqpoint{2.705591in}{2.910952in}}%
\pgfpathcurveto{\pgfqpoint{2.705591in}{2.922002in}}{\pgfqpoint{2.701201in}{2.932601in}}{\pgfqpoint{2.693387in}{2.940415in}}%
\pgfpathcurveto{\pgfqpoint{2.685574in}{2.948228in}}{\pgfqpoint{2.674975in}{2.952619in}}{\pgfqpoint{2.663925in}{2.952619in}}%
\pgfpathcurveto{\pgfqpoint{2.652874in}{2.952619in}}{\pgfqpoint{2.642275in}{2.948228in}}{\pgfqpoint{2.634462in}{2.940415in}}%
\pgfpathcurveto{\pgfqpoint{2.626648in}{2.932601in}}{\pgfqpoint{2.622258in}{2.922002in}}{\pgfqpoint{2.622258in}{2.910952in}}%
\pgfpathcurveto{\pgfqpoint{2.622258in}{2.899902in}}{\pgfqpoint{2.626648in}{2.889303in}}{\pgfqpoint{2.634462in}{2.881489in}}%
\pgfpathcurveto{\pgfqpoint{2.642275in}{2.873675in}}{\pgfqpoint{2.652874in}{2.869285in}}{\pgfqpoint{2.663925in}{2.869285in}}%
\pgfpathclose%
\pgfusepath{stroke,fill}%
\end{pgfscope}%
\begin{pgfscope}%
\pgfpathrectangle{\pgfqpoint{0.600000in}{0.600000in}}{\pgfqpoint{3.900000in}{3.900000in}}%
\pgfusepath{clip}%
\pgfsetbuttcap%
\pgfsetroundjoin%
\definecolor{currentfill}{rgb}{0.121569,0.466667,0.705882}%
\pgfsetfillcolor{currentfill}%
\pgfsetlinewidth{1.003750pt}%
\definecolor{currentstroke}{rgb}{0.121569,0.466667,0.705882}%
\pgfsetstrokecolor{currentstroke}%
\pgfsetdash{}{0pt}%
\pgfpathmoveto{\pgfqpoint{3.670881in}{2.639921in}}%
\pgfpathcurveto{\pgfqpoint{3.681931in}{2.639921in}}{\pgfqpoint{3.692530in}{2.644311in}}{\pgfqpoint{3.700344in}{2.652125in}}%
\pgfpathcurveto{\pgfqpoint{3.708158in}{2.659938in}}{\pgfqpoint{3.712548in}{2.670537in}}{\pgfqpoint{3.712548in}{2.681587in}}%
\pgfpathcurveto{\pgfqpoint{3.712548in}{2.692637in}}{\pgfqpoint{3.708158in}{2.703237in}}{\pgfqpoint{3.700344in}{2.711050in}}%
\pgfpathcurveto{\pgfqpoint{3.692530in}{2.718864in}}{\pgfqpoint{3.681931in}{2.723254in}}{\pgfqpoint{3.670881in}{2.723254in}}%
\pgfpathcurveto{\pgfqpoint{3.659831in}{2.723254in}}{\pgfqpoint{3.649232in}{2.718864in}}{\pgfqpoint{3.641418in}{2.711050in}}%
\pgfpathcurveto{\pgfqpoint{3.633605in}{2.703237in}}{\pgfqpoint{3.629215in}{2.692637in}}{\pgfqpoint{3.629215in}{2.681587in}}%
\pgfpathcurveto{\pgfqpoint{3.629215in}{2.670537in}}{\pgfqpoint{3.633605in}{2.659938in}}{\pgfqpoint{3.641418in}{2.652125in}}%
\pgfpathcurveto{\pgfqpoint{3.649232in}{2.644311in}}{\pgfqpoint{3.659831in}{2.639921in}}{\pgfqpoint{3.670881in}{2.639921in}}%
\pgfpathclose%
\pgfusepath{stroke,fill}%
\end{pgfscope}%
\begin{pgfscope}%
\pgfpathrectangle{\pgfqpoint{0.600000in}{0.600000in}}{\pgfqpoint{3.900000in}{3.900000in}}%
\pgfusepath{clip}%
\pgfsetbuttcap%
\pgfsetroundjoin%
\definecolor{currentfill}{rgb}{0.121569,0.466667,0.705882}%
\pgfsetfillcolor{currentfill}%
\pgfsetlinewidth{1.003750pt}%
\definecolor{currentstroke}{rgb}{0.121569,0.466667,0.705882}%
\pgfsetstrokecolor{currentstroke}%
\pgfsetdash{}{0pt}%
\pgfpathmoveto{\pgfqpoint{2.748224in}{2.606563in}}%
\pgfpathcurveto{\pgfqpoint{2.759274in}{2.606563in}}{\pgfqpoint{2.769873in}{2.610953in}}{\pgfqpoint{2.777687in}{2.618767in}}%
\pgfpathcurveto{\pgfqpoint{2.785501in}{2.626580in}}{\pgfqpoint{2.789891in}{2.637180in}}{\pgfqpoint{2.789891in}{2.648230in}}%
\pgfpathcurveto{\pgfqpoint{2.789891in}{2.659280in}}{\pgfqpoint{2.785501in}{2.669879in}}{\pgfqpoint{2.777687in}{2.677692in}}%
\pgfpathcurveto{\pgfqpoint{2.769873in}{2.685506in}}{\pgfqpoint{2.759274in}{2.689896in}}{\pgfqpoint{2.748224in}{2.689896in}}%
\pgfpathcurveto{\pgfqpoint{2.737174in}{2.689896in}}{\pgfqpoint{2.726575in}{2.685506in}}{\pgfqpoint{2.718762in}{2.677692in}}%
\pgfpathcurveto{\pgfqpoint{2.710948in}{2.669879in}}{\pgfqpoint{2.706558in}{2.659280in}}{\pgfqpoint{2.706558in}{2.648230in}}%
\pgfpathcurveto{\pgfqpoint{2.706558in}{2.637180in}}{\pgfqpoint{2.710948in}{2.626580in}}{\pgfqpoint{2.718762in}{2.618767in}}%
\pgfpathcurveto{\pgfqpoint{2.726575in}{2.610953in}}{\pgfqpoint{2.737174in}{2.606563in}}{\pgfqpoint{2.748224in}{2.606563in}}%
\pgfpathclose%
\pgfusepath{stroke,fill}%
\end{pgfscope}%
\begin{pgfscope}%
\pgfpathrectangle{\pgfqpoint{0.600000in}{0.600000in}}{\pgfqpoint{3.900000in}{3.900000in}}%
\pgfusepath{clip}%
\pgfsetbuttcap%
\pgfsetroundjoin%
\definecolor{currentfill}{rgb}{0.121569,0.466667,0.705882}%
\pgfsetfillcolor{currentfill}%
\pgfsetlinewidth{1.003750pt}%
\definecolor{currentstroke}{rgb}{0.121569,0.466667,0.705882}%
\pgfsetstrokecolor{currentstroke}%
\pgfsetdash{}{0pt}%
\pgfpathmoveto{\pgfqpoint{4.032498in}{2.968297in}}%
\pgfpathcurveto{\pgfqpoint{4.043548in}{2.968297in}}{\pgfqpoint{4.054147in}{2.972688in}}{\pgfqpoint{4.061961in}{2.980501in}}%
\pgfpathcurveto{\pgfqpoint{4.069775in}{2.988315in}}{\pgfqpoint{4.074165in}{2.998914in}}{\pgfqpoint{4.074165in}{3.009964in}}%
\pgfpathcurveto{\pgfqpoint{4.074165in}{3.021014in}}{\pgfqpoint{4.069775in}{3.031613in}}{\pgfqpoint{4.061961in}{3.039427in}}%
\pgfpathcurveto{\pgfqpoint{4.054147in}{3.047240in}}{\pgfqpoint{4.043548in}{3.051631in}}{\pgfqpoint{4.032498in}{3.051631in}}%
\pgfpathcurveto{\pgfqpoint{4.021448in}{3.051631in}}{\pgfqpoint{4.010849in}{3.047240in}}{\pgfqpoint{4.003035in}{3.039427in}}%
\pgfpathcurveto{\pgfqpoint{3.995222in}{3.031613in}}{\pgfqpoint{3.990832in}{3.021014in}}{\pgfqpoint{3.990832in}{3.009964in}}%
\pgfpathcurveto{\pgfqpoint{3.990832in}{2.998914in}}{\pgfqpoint{3.995222in}{2.988315in}}{\pgfqpoint{4.003035in}{2.980501in}}%
\pgfpathcurveto{\pgfqpoint{4.010849in}{2.972688in}}{\pgfqpoint{4.021448in}{2.968297in}}{\pgfqpoint{4.032498in}{2.968297in}}%
\pgfpathclose%
\pgfusepath{stroke,fill}%
\end{pgfscope}%
\begin{pgfscope}%
\pgfpathrectangle{\pgfqpoint{0.600000in}{0.600000in}}{\pgfqpoint{3.900000in}{3.900000in}}%
\pgfusepath{clip}%
\pgfsetbuttcap%
\pgfsetroundjoin%
\definecolor{currentfill}{rgb}{0.121569,0.466667,0.705882}%
\pgfsetfillcolor{currentfill}%
\pgfsetlinewidth{1.003750pt}%
\definecolor{currentstroke}{rgb}{0.121569,0.466667,0.705882}%
\pgfsetstrokecolor{currentstroke}%
\pgfsetdash{}{0pt}%
\pgfpathmoveto{\pgfqpoint{2.308159in}{1.324175in}}%
\pgfpathcurveto{\pgfqpoint{2.319209in}{1.324175in}}{\pgfqpoint{2.329808in}{1.328566in}}{\pgfqpoint{2.337622in}{1.336379in}}%
\pgfpathcurveto{\pgfqpoint{2.345435in}{1.344193in}}{\pgfqpoint{2.349825in}{1.354792in}}{\pgfqpoint{2.349825in}{1.365842in}}%
\pgfpathcurveto{\pgfqpoint{2.349825in}{1.376892in}}{\pgfqpoint{2.345435in}{1.387491in}}{\pgfqpoint{2.337622in}{1.395305in}}%
\pgfpathcurveto{\pgfqpoint{2.329808in}{1.403118in}}{\pgfqpoint{2.319209in}{1.407509in}}{\pgfqpoint{2.308159in}{1.407509in}}%
\pgfpathcurveto{\pgfqpoint{2.297109in}{1.407509in}}{\pgfqpoint{2.286510in}{1.403118in}}{\pgfqpoint{2.278696in}{1.395305in}}%
\pgfpathcurveto{\pgfqpoint{2.270882in}{1.387491in}}{\pgfqpoint{2.266492in}{1.376892in}}{\pgfqpoint{2.266492in}{1.365842in}}%
\pgfpathcurveto{\pgfqpoint{2.266492in}{1.354792in}}{\pgfqpoint{2.270882in}{1.344193in}}{\pgfqpoint{2.278696in}{1.336379in}}%
\pgfpathcurveto{\pgfqpoint{2.286510in}{1.328566in}}{\pgfqpoint{2.297109in}{1.324175in}}{\pgfqpoint{2.308159in}{1.324175in}}%
\pgfpathclose%
\pgfusepath{stroke,fill}%
\end{pgfscope}%
\begin{pgfscope}%
\pgfpathrectangle{\pgfqpoint{0.600000in}{0.600000in}}{\pgfqpoint{3.900000in}{3.900000in}}%
\pgfusepath{clip}%
\pgfsetbuttcap%
\pgfsetroundjoin%
\definecolor{currentfill}{rgb}{0.121569,0.466667,0.705882}%
\pgfsetfillcolor{currentfill}%
\pgfsetlinewidth{1.003750pt}%
\definecolor{currentstroke}{rgb}{0.121569,0.466667,0.705882}%
\pgfsetstrokecolor{currentstroke}%
\pgfsetdash{}{0pt}%
\pgfpathmoveto{\pgfqpoint{1.889163in}{2.479756in}}%
\pgfpathcurveto{\pgfqpoint{1.900213in}{2.479756in}}{\pgfqpoint{1.910812in}{2.484146in}}{\pgfqpoint{1.918626in}{2.491960in}}%
\pgfpathcurveto{\pgfqpoint{1.926440in}{2.499774in}}{\pgfqpoint{1.930830in}{2.510373in}}{\pgfqpoint{1.930830in}{2.521423in}}%
\pgfpathcurveto{\pgfqpoint{1.930830in}{2.532473in}}{\pgfqpoint{1.926440in}{2.543072in}}{\pgfqpoint{1.918626in}{2.550886in}}%
\pgfpathcurveto{\pgfqpoint{1.910812in}{2.558699in}}{\pgfqpoint{1.900213in}{2.563090in}}{\pgfqpoint{1.889163in}{2.563090in}}%
\pgfpathcurveto{\pgfqpoint{1.878113in}{2.563090in}}{\pgfqpoint{1.867514in}{2.558699in}}{\pgfqpoint{1.859700in}{2.550886in}}%
\pgfpathcurveto{\pgfqpoint{1.851887in}{2.543072in}}{\pgfqpoint{1.847496in}{2.532473in}}{\pgfqpoint{1.847496in}{2.521423in}}%
\pgfpathcurveto{\pgfqpoint{1.847496in}{2.510373in}}{\pgfqpoint{1.851887in}{2.499774in}}{\pgfqpoint{1.859700in}{2.491960in}}%
\pgfpathcurveto{\pgfqpoint{1.867514in}{2.484146in}}{\pgfqpoint{1.878113in}{2.479756in}}{\pgfqpoint{1.889163in}{2.479756in}}%
\pgfpathclose%
\pgfusepath{stroke,fill}%
\end{pgfscope}%
\begin{pgfscope}%
\pgfpathrectangle{\pgfqpoint{0.600000in}{0.600000in}}{\pgfqpoint{3.900000in}{3.900000in}}%
\pgfusepath{clip}%
\pgfsetbuttcap%
\pgfsetroundjoin%
\definecolor{currentfill}{rgb}{0.121569,0.466667,0.705882}%
\pgfsetfillcolor{currentfill}%
\pgfsetlinewidth{1.003750pt}%
\definecolor{currentstroke}{rgb}{0.121569,0.466667,0.705882}%
\pgfsetstrokecolor{currentstroke}%
\pgfsetdash{}{0pt}%
\pgfpathmoveto{\pgfqpoint{2.112660in}{2.403920in}}%
\pgfpathcurveto{\pgfqpoint{2.123710in}{2.403920in}}{\pgfqpoint{2.134309in}{2.408310in}}{\pgfqpoint{2.142122in}{2.416124in}}%
\pgfpathcurveto{\pgfqpoint{2.149936in}{2.423937in}}{\pgfqpoint{2.154326in}{2.434536in}}{\pgfqpoint{2.154326in}{2.445586in}}%
\pgfpathcurveto{\pgfqpoint{2.154326in}{2.456637in}}{\pgfqpoint{2.149936in}{2.467236in}}{\pgfqpoint{2.142122in}{2.475049in}}%
\pgfpathcurveto{\pgfqpoint{2.134309in}{2.482863in}}{\pgfqpoint{2.123710in}{2.487253in}}{\pgfqpoint{2.112660in}{2.487253in}}%
\pgfpathcurveto{\pgfqpoint{2.101610in}{2.487253in}}{\pgfqpoint{2.091011in}{2.482863in}}{\pgfqpoint{2.083197in}{2.475049in}}%
\pgfpathcurveto{\pgfqpoint{2.075383in}{2.467236in}}{\pgfqpoint{2.070993in}{2.456637in}}{\pgfqpoint{2.070993in}{2.445586in}}%
\pgfpathcurveto{\pgfqpoint{2.070993in}{2.434536in}}{\pgfqpoint{2.075383in}{2.423937in}}{\pgfqpoint{2.083197in}{2.416124in}}%
\pgfpathcurveto{\pgfqpoint{2.091011in}{2.408310in}}{\pgfqpoint{2.101610in}{2.403920in}}{\pgfqpoint{2.112660in}{2.403920in}}%
\pgfpathclose%
\pgfusepath{stroke,fill}%
\end{pgfscope}%
\begin{pgfscope}%
\pgfpathrectangle{\pgfqpoint{0.600000in}{0.600000in}}{\pgfqpoint{3.900000in}{3.900000in}}%
\pgfusepath{clip}%
\pgfsetbuttcap%
\pgfsetroundjoin%
\definecolor{currentfill}{rgb}{0.121569,0.466667,0.705882}%
\pgfsetfillcolor{currentfill}%
\pgfsetlinewidth{1.003750pt}%
\definecolor{currentstroke}{rgb}{0.121569,0.466667,0.705882}%
\pgfsetstrokecolor{currentstroke}%
\pgfsetdash{}{0pt}%
\pgfpathmoveto{\pgfqpoint{1.793261in}{3.462733in}}%
\pgfpathcurveto{\pgfqpoint{1.804311in}{3.462733in}}{\pgfqpoint{1.814910in}{3.467124in}}{\pgfqpoint{1.822724in}{3.474937in}}%
\pgfpathcurveto{\pgfqpoint{1.830537in}{3.482751in}}{\pgfqpoint{1.834928in}{3.493350in}}{\pgfqpoint{1.834928in}{3.504400in}}%
\pgfpathcurveto{\pgfqpoint{1.834928in}{3.515450in}}{\pgfqpoint{1.830537in}{3.526049in}}{\pgfqpoint{1.822724in}{3.533863in}}%
\pgfpathcurveto{\pgfqpoint{1.814910in}{3.541676in}}{\pgfqpoint{1.804311in}{3.546067in}}{\pgfqpoint{1.793261in}{3.546067in}}%
\pgfpathcurveto{\pgfqpoint{1.782211in}{3.546067in}}{\pgfqpoint{1.771612in}{3.541676in}}{\pgfqpoint{1.763798in}{3.533863in}}%
\pgfpathcurveto{\pgfqpoint{1.755985in}{3.526049in}}{\pgfqpoint{1.751594in}{3.515450in}}{\pgfqpoint{1.751594in}{3.504400in}}%
\pgfpathcurveto{\pgfqpoint{1.751594in}{3.493350in}}{\pgfqpoint{1.755985in}{3.482751in}}{\pgfqpoint{1.763798in}{3.474937in}}%
\pgfpathcurveto{\pgfqpoint{1.771612in}{3.467124in}}{\pgfqpoint{1.782211in}{3.462733in}}{\pgfqpoint{1.793261in}{3.462733in}}%
\pgfpathclose%
\pgfusepath{stroke,fill}%
\end{pgfscope}%
\begin{pgfscope}%
\pgfpathrectangle{\pgfqpoint{0.600000in}{0.600000in}}{\pgfqpoint{3.900000in}{3.900000in}}%
\pgfusepath{clip}%
\pgfsetbuttcap%
\pgfsetroundjoin%
\definecolor{currentfill}{rgb}{0.121569,0.466667,0.705882}%
\pgfsetfillcolor{currentfill}%
\pgfsetlinewidth{1.003750pt}%
\definecolor{currentstroke}{rgb}{0.121569,0.466667,0.705882}%
\pgfsetstrokecolor{currentstroke}%
\pgfsetdash{}{0pt}%
\pgfpathmoveto{\pgfqpoint{1.874910in}{2.310104in}}%
\pgfpathcurveto{\pgfqpoint{1.885960in}{2.310104in}}{\pgfqpoint{1.896559in}{2.314494in}}{\pgfqpoint{1.904373in}{2.322308in}}%
\pgfpathcurveto{\pgfqpoint{1.912187in}{2.330121in}}{\pgfqpoint{1.916577in}{2.340720in}}{\pgfqpoint{1.916577in}{2.351771in}}%
\pgfpathcurveto{\pgfqpoint{1.916577in}{2.362821in}}{\pgfqpoint{1.912187in}{2.373420in}}{\pgfqpoint{1.904373in}{2.381233in}}%
\pgfpathcurveto{\pgfqpoint{1.896559in}{2.389047in}}{\pgfqpoint{1.885960in}{2.393437in}}{\pgfqpoint{1.874910in}{2.393437in}}%
\pgfpathcurveto{\pgfqpoint{1.863860in}{2.393437in}}{\pgfqpoint{1.853261in}{2.389047in}}{\pgfqpoint{1.845447in}{2.381233in}}%
\pgfpathcurveto{\pgfqpoint{1.837634in}{2.373420in}}{\pgfqpoint{1.833243in}{2.362821in}}{\pgfqpoint{1.833243in}{2.351771in}}%
\pgfpathcurveto{\pgfqpoint{1.833243in}{2.340720in}}{\pgfqpoint{1.837634in}{2.330121in}}{\pgfqpoint{1.845447in}{2.322308in}}%
\pgfpathcurveto{\pgfqpoint{1.853261in}{2.314494in}}{\pgfqpoint{1.863860in}{2.310104in}}{\pgfqpoint{1.874910in}{2.310104in}}%
\pgfpathclose%
\pgfusepath{stroke,fill}%
\end{pgfscope}%
\begin{pgfscope}%
\pgfpathrectangle{\pgfqpoint{0.600000in}{0.600000in}}{\pgfqpoint{3.900000in}{3.900000in}}%
\pgfusepath{clip}%
\pgfsetbuttcap%
\pgfsetroundjoin%
\definecolor{currentfill}{rgb}{0.121569,0.466667,0.705882}%
\pgfsetfillcolor{currentfill}%
\pgfsetlinewidth{1.003750pt}%
\definecolor{currentstroke}{rgb}{0.121569,0.466667,0.705882}%
\pgfsetstrokecolor{currentstroke}%
\pgfsetdash{}{0pt}%
\pgfpathmoveto{\pgfqpoint{3.372492in}{1.539851in}}%
\pgfpathcurveto{\pgfqpoint{3.383542in}{1.539851in}}{\pgfqpoint{3.394141in}{1.544241in}}{\pgfqpoint{3.401954in}{1.552055in}}%
\pgfpathcurveto{\pgfqpoint{3.409768in}{1.559869in}}{\pgfqpoint{3.414158in}{1.570468in}}{\pgfqpoint{3.414158in}{1.581518in}}%
\pgfpathcurveto{\pgfqpoint{3.414158in}{1.592568in}}{\pgfqpoint{3.409768in}{1.603167in}}{\pgfqpoint{3.401954in}{1.610980in}}%
\pgfpathcurveto{\pgfqpoint{3.394141in}{1.618794in}}{\pgfqpoint{3.383542in}{1.623184in}}{\pgfqpoint{3.372492in}{1.623184in}}%
\pgfpathcurveto{\pgfqpoint{3.361441in}{1.623184in}}{\pgfqpoint{3.350842in}{1.618794in}}{\pgfqpoint{3.343029in}{1.610980in}}%
\pgfpathcurveto{\pgfqpoint{3.335215in}{1.603167in}}{\pgfqpoint{3.330825in}{1.592568in}}{\pgfqpoint{3.330825in}{1.581518in}}%
\pgfpathcurveto{\pgfqpoint{3.330825in}{1.570468in}}{\pgfqpoint{3.335215in}{1.559869in}}{\pgfqpoint{3.343029in}{1.552055in}}%
\pgfpathcurveto{\pgfqpoint{3.350842in}{1.544241in}}{\pgfqpoint{3.361441in}{1.539851in}}{\pgfqpoint{3.372492in}{1.539851in}}%
\pgfpathclose%
\pgfusepath{stroke,fill}%
\end{pgfscope}%
\begin{pgfscope}%
\pgfpathrectangle{\pgfqpoint{0.600000in}{0.600000in}}{\pgfqpoint{3.900000in}{3.900000in}}%
\pgfusepath{clip}%
\pgfsetbuttcap%
\pgfsetroundjoin%
\definecolor{currentfill}{rgb}{0.121569,0.466667,0.705882}%
\pgfsetfillcolor{currentfill}%
\pgfsetlinewidth{1.003750pt}%
\definecolor{currentstroke}{rgb}{0.121569,0.466667,0.705882}%
\pgfsetstrokecolor{currentstroke}%
\pgfsetdash{}{0pt}%
\pgfpathmoveto{\pgfqpoint{2.727225in}{2.381020in}}%
\pgfpathcurveto{\pgfqpoint{2.738275in}{2.381020in}}{\pgfqpoint{2.748874in}{2.385410in}}{\pgfqpoint{2.756688in}{2.393224in}}%
\pgfpathcurveto{\pgfqpoint{2.764502in}{2.401037in}}{\pgfqpoint{2.768892in}{2.411636in}}{\pgfqpoint{2.768892in}{2.422687in}}%
\pgfpathcurveto{\pgfqpoint{2.768892in}{2.433737in}}{\pgfqpoint{2.764502in}{2.444336in}}{\pgfqpoint{2.756688in}{2.452149in}}%
\pgfpathcurveto{\pgfqpoint{2.748874in}{2.459963in}}{\pgfqpoint{2.738275in}{2.464353in}}{\pgfqpoint{2.727225in}{2.464353in}}%
\pgfpathcurveto{\pgfqpoint{2.716175in}{2.464353in}}{\pgfqpoint{2.705576in}{2.459963in}}{\pgfqpoint{2.697762in}{2.452149in}}%
\pgfpathcurveto{\pgfqpoint{2.689949in}{2.444336in}}{\pgfqpoint{2.685559in}{2.433737in}}{\pgfqpoint{2.685559in}{2.422687in}}%
\pgfpathcurveto{\pgfqpoint{2.685559in}{2.411636in}}{\pgfqpoint{2.689949in}{2.401037in}}{\pgfqpoint{2.697762in}{2.393224in}}%
\pgfpathcurveto{\pgfqpoint{2.705576in}{2.385410in}}{\pgfqpoint{2.716175in}{2.381020in}}{\pgfqpoint{2.727225in}{2.381020in}}%
\pgfpathclose%
\pgfusepath{stroke,fill}%
\end{pgfscope}%
\begin{pgfscope}%
\pgfpathrectangle{\pgfqpoint{0.600000in}{0.600000in}}{\pgfqpoint{3.900000in}{3.900000in}}%
\pgfusepath{clip}%
\pgfsetbuttcap%
\pgfsetroundjoin%
\definecolor{currentfill}{rgb}{0.121569,0.466667,0.705882}%
\pgfsetfillcolor{currentfill}%
\pgfsetlinewidth{1.003750pt}%
\definecolor{currentstroke}{rgb}{0.121569,0.466667,0.705882}%
\pgfsetstrokecolor{currentstroke}%
\pgfsetdash{}{0pt}%
\pgfpathmoveto{\pgfqpoint{2.850614in}{3.042710in}}%
\pgfpathcurveto{\pgfqpoint{2.861665in}{3.042710in}}{\pgfqpoint{2.872264in}{3.047100in}}{\pgfqpoint{2.880077in}{3.054914in}}%
\pgfpathcurveto{\pgfqpoint{2.887891in}{3.062727in}}{\pgfqpoint{2.892281in}{3.073326in}}{\pgfqpoint{2.892281in}{3.084377in}}%
\pgfpathcurveto{\pgfqpoint{2.892281in}{3.095427in}}{\pgfqpoint{2.887891in}{3.106026in}}{\pgfqpoint{2.880077in}{3.113839in}}%
\pgfpathcurveto{\pgfqpoint{2.872264in}{3.121653in}}{\pgfqpoint{2.861665in}{3.126043in}}{\pgfqpoint{2.850614in}{3.126043in}}%
\pgfpathcurveto{\pgfqpoint{2.839564in}{3.126043in}}{\pgfqpoint{2.828965in}{3.121653in}}{\pgfqpoint{2.821152in}{3.113839in}}%
\pgfpathcurveto{\pgfqpoint{2.813338in}{3.106026in}}{\pgfqpoint{2.808948in}{3.095427in}}{\pgfqpoint{2.808948in}{3.084377in}}%
\pgfpathcurveto{\pgfqpoint{2.808948in}{3.073326in}}{\pgfqpoint{2.813338in}{3.062727in}}{\pgfqpoint{2.821152in}{3.054914in}}%
\pgfpathcurveto{\pgfqpoint{2.828965in}{3.047100in}}{\pgfqpoint{2.839564in}{3.042710in}}{\pgfqpoint{2.850614in}{3.042710in}}%
\pgfpathclose%
\pgfusepath{stroke,fill}%
\end{pgfscope}%
\begin{pgfscope}%
\pgfpathrectangle{\pgfqpoint{0.600000in}{0.600000in}}{\pgfqpoint{3.900000in}{3.900000in}}%
\pgfusepath{clip}%
\pgfsetbuttcap%
\pgfsetroundjoin%
\definecolor{currentfill}{rgb}{0.121569,0.466667,0.705882}%
\pgfsetfillcolor{currentfill}%
\pgfsetlinewidth{1.003750pt}%
\definecolor{currentstroke}{rgb}{0.121569,0.466667,0.705882}%
\pgfsetstrokecolor{currentstroke}%
\pgfsetdash{}{0pt}%
\pgfpathmoveto{\pgfqpoint{2.588791in}{2.720269in}}%
\pgfpathcurveto{\pgfqpoint{2.599841in}{2.720269in}}{\pgfqpoint{2.610440in}{2.724659in}}{\pgfqpoint{2.618253in}{2.732473in}}%
\pgfpathcurveto{\pgfqpoint{2.626067in}{2.740286in}}{\pgfqpoint{2.630457in}{2.750886in}}{\pgfqpoint{2.630457in}{2.761936in}}%
\pgfpathcurveto{\pgfqpoint{2.630457in}{2.772986in}}{\pgfqpoint{2.626067in}{2.783585in}}{\pgfqpoint{2.618253in}{2.791398in}}%
\pgfpathcurveto{\pgfqpoint{2.610440in}{2.799212in}}{\pgfqpoint{2.599841in}{2.803602in}}{\pgfqpoint{2.588791in}{2.803602in}}%
\pgfpathcurveto{\pgfqpoint{2.577740in}{2.803602in}}{\pgfqpoint{2.567141in}{2.799212in}}{\pgfqpoint{2.559328in}{2.791398in}}%
\pgfpathcurveto{\pgfqpoint{2.551514in}{2.783585in}}{\pgfqpoint{2.547124in}{2.772986in}}{\pgfqpoint{2.547124in}{2.761936in}}%
\pgfpathcurveto{\pgfqpoint{2.547124in}{2.750886in}}{\pgfqpoint{2.551514in}{2.740286in}}{\pgfqpoint{2.559328in}{2.732473in}}%
\pgfpathcurveto{\pgfqpoint{2.567141in}{2.724659in}}{\pgfqpoint{2.577740in}{2.720269in}}{\pgfqpoint{2.588791in}{2.720269in}}%
\pgfpathclose%
\pgfusepath{stroke,fill}%
\end{pgfscope}%
\begin{pgfscope}%
\pgfpathrectangle{\pgfqpoint{0.600000in}{0.600000in}}{\pgfqpoint{3.900000in}{3.900000in}}%
\pgfusepath{clip}%
\pgfsetbuttcap%
\pgfsetroundjoin%
\definecolor{currentfill}{rgb}{0.121569,0.466667,0.705882}%
\pgfsetfillcolor{currentfill}%
\pgfsetlinewidth{1.003750pt}%
\definecolor{currentstroke}{rgb}{0.121569,0.466667,0.705882}%
\pgfsetstrokecolor{currentstroke}%
\pgfsetdash{}{0pt}%
\pgfpathmoveto{\pgfqpoint{2.557788in}{3.192470in}}%
\pgfpathcurveto{\pgfqpoint{2.568838in}{3.192470in}}{\pgfqpoint{2.579437in}{3.196861in}}{\pgfqpoint{2.587251in}{3.204674in}}%
\pgfpathcurveto{\pgfqpoint{2.595064in}{3.212488in}}{\pgfqpoint{2.599454in}{3.223087in}}{\pgfqpoint{2.599454in}{3.234137in}}%
\pgfpathcurveto{\pgfqpoint{2.599454in}{3.245187in}}{\pgfqpoint{2.595064in}{3.255786in}}{\pgfqpoint{2.587251in}{3.263600in}}%
\pgfpathcurveto{\pgfqpoint{2.579437in}{3.271413in}}{\pgfqpoint{2.568838in}{3.275804in}}{\pgfqpoint{2.557788in}{3.275804in}}%
\pgfpathcurveto{\pgfqpoint{2.546738in}{3.275804in}}{\pgfqpoint{2.536139in}{3.271413in}}{\pgfqpoint{2.528325in}{3.263600in}}%
\pgfpathcurveto{\pgfqpoint{2.520511in}{3.255786in}}{\pgfqpoint{2.516121in}{3.245187in}}{\pgfqpoint{2.516121in}{3.234137in}}%
\pgfpathcurveto{\pgfqpoint{2.516121in}{3.223087in}}{\pgfqpoint{2.520511in}{3.212488in}}{\pgfqpoint{2.528325in}{3.204674in}}%
\pgfpathcurveto{\pgfqpoint{2.536139in}{3.196861in}}{\pgfqpoint{2.546738in}{3.192470in}}{\pgfqpoint{2.557788in}{3.192470in}}%
\pgfpathclose%
\pgfusepath{stroke,fill}%
\end{pgfscope}%
\begin{pgfscope}%
\pgfpathrectangle{\pgfqpoint{0.600000in}{0.600000in}}{\pgfqpoint{3.900000in}{3.900000in}}%
\pgfusepath{clip}%
\pgfsetbuttcap%
\pgfsetroundjoin%
\definecolor{currentfill}{rgb}{0.121569,0.466667,0.705882}%
\pgfsetfillcolor{currentfill}%
\pgfsetlinewidth{1.003750pt}%
\definecolor{currentstroke}{rgb}{0.121569,0.466667,0.705882}%
\pgfsetstrokecolor{currentstroke}%
\pgfsetdash{}{0pt}%
\pgfpathmoveto{\pgfqpoint{3.013149in}{3.314992in}}%
\pgfpathcurveto{\pgfqpoint{3.024199in}{3.314992in}}{\pgfqpoint{3.034798in}{3.319382in}}{\pgfqpoint{3.042612in}{3.327196in}}%
\pgfpathcurveto{\pgfqpoint{3.050425in}{3.335009in}}{\pgfqpoint{3.054815in}{3.345608in}}{\pgfqpoint{3.054815in}{3.356659in}}%
\pgfpathcurveto{\pgfqpoint{3.054815in}{3.367709in}}{\pgfqpoint{3.050425in}{3.378308in}}{\pgfqpoint{3.042612in}{3.386121in}}%
\pgfpathcurveto{\pgfqpoint{3.034798in}{3.393935in}}{\pgfqpoint{3.024199in}{3.398325in}}{\pgfqpoint{3.013149in}{3.398325in}}%
\pgfpathcurveto{\pgfqpoint{3.002099in}{3.398325in}}{\pgfqpoint{2.991500in}{3.393935in}}{\pgfqpoint{2.983686in}{3.386121in}}%
\pgfpathcurveto{\pgfqpoint{2.975872in}{3.378308in}}{\pgfqpoint{2.971482in}{3.367709in}}{\pgfqpoint{2.971482in}{3.356659in}}%
\pgfpathcurveto{\pgfqpoint{2.971482in}{3.345608in}}{\pgfqpoint{2.975872in}{3.335009in}}{\pgfqpoint{2.983686in}{3.327196in}}%
\pgfpathcurveto{\pgfqpoint{2.991500in}{3.319382in}}{\pgfqpoint{3.002099in}{3.314992in}}{\pgfqpoint{3.013149in}{3.314992in}}%
\pgfpathclose%
\pgfusepath{stroke,fill}%
\end{pgfscope}%
\begin{pgfscope}%
\pgfpathrectangle{\pgfqpoint{0.600000in}{0.600000in}}{\pgfqpoint{3.900000in}{3.900000in}}%
\pgfusepath{clip}%
\pgfsetbuttcap%
\pgfsetroundjoin%
\definecolor{currentfill}{rgb}{0.121569,0.466667,0.705882}%
\pgfsetfillcolor{currentfill}%
\pgfsetlinewidth{1.003750pt}%
\definecolor{currentstroke}{rgb}{0.121569,0.466667,0.705882}%
\pgfsetstrokecolor{currentstroke}%
\pgfsetdash{}{0pt}%
\pgfpathmoveto{\pgfqpoint{2.015669in}{2.716600in}}%
\pgfpathcurveto{\pgfqpoint{2.026719in}{2.716600in}}{\pgfqpoint{2.037318in}{2.720990in}}{\pgfqpoint{2.045132in}{2.728804in}}%
\pgfpathcurveto{\pgfqpoint{2.052946in}{2.736617in}}{\pgfqpoint{2.057336in}{2.747216in}}{\pgfqpoint{2.057336in}{2.758266in}}%
\pgfpathcurveto{\pgfqpoint{2.057336in}{2.769316in}}{\pgfqpoint{2.052946in}{2.779915in}}{\pgfqpoint{2.045132in}{2.787729in}}%
\pgfpathcurveto{\pgfqpoint{2.037318in}{2.795543in}}{\pgfqpoint{2.026719in}{2.799933in}}{\pgfqpoint{2.015669in}{2.799933in}}%
\pgfpathcurveto{\pgfqpoint{2.004619in}{2.799933in}}{\pgfqpoint{1.994020in}{2.795543in}}{\pgfqpoint{1.986207in}{2.787729in}}%
\pgfpathcurveto{\pgfqpoint{1.978393in}{2.779915in}}{\pgfqpoint{1.974003in}{2.769316in}}{\pgfqpoint{1.974003in}{2.758266in}}%
\pgfpathcurveto{\pgfqpoint{1.974003in}{2.747216in}}{\pgfqpoint{1.978393in}{2.736617in}}{\pgfqpoint{1.986207in}{2.728804in}}%
\pgfpathcurveto{\pgfqpoint{1.994020in}{2.720990in}}{\pgfqpoint{2.004619in}{2.716600in}}{\pgfqpoint{2.015669in}{2.716600in}}%
\pgfpathclose%
\pgfusepath{stroke,fill}%
\end{pgfscope}%
\begin{pgfscope}%
\pgfpathrectangle{\pgfqpoint{0.600000in}{0.600000in}}{\pgfqpoint{3.900000in}{3.900000in}}%
\pgfusepath{clip}%
\pgfsetbuttcap%
\pgfsetroundjoin%
\definecolor{currentfill}{rgb}{0.121569,0.466667,0.705882}%
\pgfsetfillcolor{currentfill}%
\pgfsetlinewidth{1.003750pt}%
\definecolor{currentstroke}{rgb}{0.121569,0.466667,0.705882}%
\pgfsetstrokecolor{currentstroke}%
\pgfsetdash{}{0pt}%
\pgfpathmoveto{\pgfqpoint{2.646890in}{2.476848in}}%
\pgfpathcurveto{\pgfqpoint{2.657940in}{2.476848in}}{\pgfqpoint{2.668539in}{2.481238in}}{\pgfqpoint{2.676352in}{2.489052in}}%
\pgfpathcurveto{\pgfqpoint{2.684166in}{2.496865in}}{\pgfqpoint{2.688556in}{2.507464in}}{\pgfqpoint{2.688556in}{2.518514in}}%
\pgfpathcurveto{\pgfqpoint{2.688556in}{2.529564in}}{\pgfqpoint{2.684166in}{2.540164in}}{\pgfqpoint{2.676352in}{2.547977in}}%
\pgfpathcurveto{\pgfqpoint{2.668539in}{2.555791in}}{\pgfqpoint{2.657940in}{2.560181in}}{\pgfqpoint{2.646890in}{2.560181in}}%
\pgfpathcurveto{\pgfqpoint{2.635839in}{2.560181in}}{\pgfqpoint{2.625240in}{2.555791in}}{\pgfqpoint{2.617427in}{2.547977in}}%
\pgfpathcurveto{\pgfqpoint{2.609613in}{2.540164in}}{\pgfqpoint{2.605223in}{2.529564in}}{\pgfqpoint{2.605223in}{2.518514in}}%
\pgfpathcurveto{\pgfqpoint{2.605223in}{2.507464in}}{\pgfqpoint{2.609613in}{2.496865in}}{\pgfqpoint{2.617427in}{2.489052in}}%
\pgfpathcurveto{\pgfqpoint{2.625240in}{2.481238in}}{\pgfqpoint{2.635839in}{2.476848in}}{\pgfqpoint{2.646890in}{2.476848in}}%
\pgfpathclose%
\pgfusepath{stroke,fill}%
\end{pgfscope}%
\begin{pgfscope}%
\pgfpathrectangle{\pgfqpoint{0.600000in}{0.600000in}}{\pgfqpoint{3.900000in}{3.900000in}}%
\pgfusepath{clip}%
\pgfsetbuttcap%
\pgfsetroundjoin%
\definecolor{currentfill}{rgb}{0.121569,0.466667,0.705882}%
\pgfsetfillcolor{currentfill}%
\pgfsetlinewidth{1.003750pt}%
\definecolor{currentstroke}{rgb}{0.121569,0.466667,0.705882}%
\pgfsetstrokecolor{currentstroke}%
\pgfsetdash{}{0pt}%
\pgfpathmoveto{\pgfqpoint{3.636662in}{2.738895in}}%
\pgfpathcurveto{\pgfqpoint{3.647712in}{2.738895in}}{\pgfqpoint{3.658311in}{2.743285in}}{\pgfqpoint{3.666125in}{2.751098in}}%
\pgfpathcurveto{\pgfqpoint{3.673938in}{2.758912in}}{\pgfqpoint{3.678329in}{2.769511in}}{\pgfqpoint{3.678329in}{2.780561in}}%
\pgfpathcurveto{\pgfqpoint{3.678329in}{2.791611in}}{\pgfqpoint{3.673938in}{2.802210in}}{\pgfqpoint{3.666125in}{2.810024in}}%
\pgfpathcurveto{\pgfqpoint{3.658311in}{2.817838in}}{\pgfqpoint{3.647712in}{2.822228in}}{\pgfqpoint{3.636662in}{2.822228in}}%
\pgfpathcurveto{\pgfqpoint{3.625612in}{2.822228in}}{\pgfqpoint{3.615013in}{2.817838in}}{\pgfqpoint{3.607199in}{2.810024in}}%
\pgfpathcurveto{\pgfqpoint{3.599386in}{2.802210in}}{\pgfqpoint{3.594995in}{2.791611in}}{\pgfqpoint{3.594995in}{2.780561in}}%
\pgfpathcurveto{\pgfqpoint{3.594995in}{2.769511in}}{\pgfqpoint{3.599386in}{2.758912in}}{\pgfqpoint{3.607199in}{2.751098in}}%
\pgfpathcurveto{\pgfqpoint{3.615013in}{2.743285in}}{\pgfqpoint{3.625612in}{2.738895in}}{\pgfqpoint{3.636662in}{2.738895in}}%
\pgfpathclose%
\pgfusepath{stroke,fill}%
\end{pgfscope}%
\begin{pgfscope}%
\pgfpathrectangle{\pgfqpoint{0.600000in}{0.600000in}}{\pgfqpoint{3.900000in}{3.900000in}}%
\pgfusepath{clip}%
\pgfsetbuttcap%
\pgfsetroundjoin%
\definecolor{currentfill}{rgb}{0.121569,0.466667,0.705882}%
\pgfsetfillcolor{currentfill}%
\pgfsetlinewidth{1.003750pt}%
\definecolor{currentstroke}{rgb}{0.121569,0.466667,0.705882}%
\pgfsetstrokecolor{currentstroke}%
\pgfsetdash{}{0pt}%
\pgfpathmoveto{\pgfqpoint{2.799360in}{2.393757in}}%
\pgfpathcurveto{\pgfqpoint{2.810410in}{2.393757in}}{\pgfqpoint{2.821009in}{2.398147in}}{\pgfqpoint{2.828823in}{2.405961in}}%
\pgfpathcurveto{\pgfqpoint{2.836636in}{2.413774in}}{\pgfqpoint{2.841027in}{2.424373in}}{\pgfqpoint{2.841027in}{2.435423in}}%
\pgfpathcurveto{\pgfqpoint{2.841027in}{2.446474in}}{\pgfqpoint{2.836636in}{2.457073in}}{\pgfqpoint{2.828823in}{2.464886in}}%
\pgfpathcurveto{\pgfqpoint{2.821009in}{2.472700in}}{\pgfqpoint{2.810410in}{2.477090in}}{\pgfqpoint{2.799360in}{2.477090in}}%
\pgfpathcurveto{\pgfqpoint{2.788310in}{2.477090in}}{\pgfqpoint{2.777711in}{2.472700in}}{\pgfqpoint{2.769897in}{2.464886in}}%
\pgfpathcurveto{\pgfqpoint{2.762084in}{2.457073in}}{\pgfqpoint{2.757693in}{2.446474in}}{\pgfqpoint{2.757693in}{2.435423in}}%
\pgfpathcurveto{\pgfqpoint{2.757693in}{2.424373in}}{\pgfqpoint{2.762084in}{2.413774in}}{\pgfqpoint{2.769897in}{2.405961in}}%
\pgfpathcurveto{\pgfqpoint{2.777711in}{2.398147in}}{\pgfqpoint{2.788310in}{2.393757in}}{\pgfqpoint{2.799360in}{2.393757in}}%
\pgfpathclose%
\pgfusepath{stroke,fill}%
\end{pgfscope}%
\begin{pgfscope}%
\pgfpathrectangle{\pgfqpoint{0.600000in}{0.600000in}}{\pgfqpoint{3.900000in}{3.900000in}}%
\pgfusepath{clip}%
\pgfsetbuttcap%
\pgfsetroundjoin%
\definecolor{currentfill}{rgb}{0.121569,0.466667,0.705882}%
\pgfsetfillcolor{currentfill}%
\pgfsetlinewidth{1.003750pt}%
\definecolor{currentstroke}{rgb}{0.121569,0.466667,0.705882}%
\pgfsetstrokecolor{currentstroke}%
\pgfsetdash{}{0pt}%
\pgfpathmoveto{\pgfqpoint{2.983704in}{1.412858in}}%
\pgfpathcurveto{\pgfqpoint{2.994754in}{1.412858in}}{\pgfqpoint{3.005353in}{1.417249in}}{\pgfqpoint{3.013166in}{1.425062in}}%
\pgfpathcurveto{\pgfqpoint{3.020980in}{1.432876in}}{\pgfqpoint{3.025370in}{1.443475in}}{\pgfqpoint{3.025370in}{1.454525in}}%
\pgfpathcurveto{\pgfqpoint{3.025370in}{1.465575in}}{\pgfqpoint{3.020980in}{1.476174in}}{\pgfqpoint{3.013166in}{1.483988in}}%
\pgfpathcurveto{\pgfqpoint{3.005353in}{1.491801in}}{\pgfqpoint{2.994754in}{1.496192in}}{\pgfqpoint{2.983704in}{1.496192in}}%
\pgfpathcurveto{\pgfqpoint{2.972653in}{1.496192in}}{\pgfqpoint{2.962054in}{1.491801in}}{\pgfqpoint{2.954241in}{1.483988in}}%
\pgfpathcurveto{\pgfqpoint{2.946427in}{1.476174in}}{\pgfqpoint{2.942037in}{1.465575in}}{\pgfqpoint{2.942037in}{1.454525in}}%
\pgfpathcurveto{\pgfqpoint{2.942037in}{1.443475in}}{\pgfqpoint{2.946427in}{1.432876in}}{\pgfqpoint{2.954241in}{1.425062in}}%
\pgfpathcurveto{\pgfqpoint{2.962054in}{1.417249in}}{\pgfqpoint{2.972653in}{1.412858in}}{\pgfqpoint{2.983704in}{1.412858in}}%
\pgfpathclose%
\pgfusepath{stroke,fill}%
\end{pgfscope}%
\begin{pgfscope}%
\pgfpathrectangle{\pgfqpoint{0.600000in}{0.600000in}}{\pgfqpoint{3.900000in}{3.900000in}}%
\pgfusepath{clip}%
\pgfsetbuttcap%
\pgfsetroundjoin%
\definecolor{currentfill}{rgb}{0.121569,0.466667,0.705882}%
\pgfsetfillcolor{currentfill}%
\pgfsetlinewidth{1.003750pt}%
\definecolor{currentstroke}{rgb}{0.121569,0.466667,0.705882}%
\pgfsetstrokecolor{currentstroke}%
\pgfsetdash{}{0pt}%
\pgfpathmoveto{\pgfqpoint{2.085265in}{2.418631in}}%
\pgfpathcurveto{\pgfqpoint{2.096315in}{2.418631in}}{\pgfqpoint{2.106915in}{2.423022in}}{\pgfqpoint{2.114728in}{2.430835in}}%
\pgfpathcurveto{\pgfqpoint{2.122542in}{2.438649in}}{\pgfqpoint{2.126932in}{2.449248in}}{\pgfqpoint{2.126932in}{2.460298in}}%
\pgfpathcurveto{\pgfqpoint{2.126932in}{2.471348in}}{\pgfqpoint{2.122542in}{2.481947in}}{\pgfqpoint{2.114728in}{2.489761in}}%
\pgfpathcurveto{\pgfqpoint{2.106915in}{2.497574in}}{\pgfqpoint{2.096315in}{2.501965in}}{\pgfqpoint{2.085265in}{2.501965in}}%
\pgfpathcurveto{\pgfqpoint{2.074215in}{2.501965in}}{\pgfqpoint{2.063616in}{2.497574in}}{\pgfqpoint{2.055803in}{2.489761in}}%
\pgfpathcurveto{\pgfqpoint{2.047989in}{2.481947in}}{\pgfqpoint{2.043599in}{2.471348in}}{\pgfqpoint{2.043599in}{2.460298in}}%
\pgfpathcurveto{\pgfqpoint{2.043599in}{2.449248in}}{\pgfqpoint{2.047989in}{2.438649in}}{\pgfqpoint{2.055803in}{2.430835in}}%
\pgfpathcurveto{\pgfqpoint{2.063616in}{2.423022in}}{\pgfqpoint{2.074215in}{2.418631in}}{\pgfqpoint{2.085265in}{2.418631in}}%
\pgfpathclose%
\pgfusepath{stroke,fill}%
\end{pgfscope}%
\begin{pgfscope}%
\pgfpathrectangle{\pgfqpoint{0.600000in}{0.600000in}}{\pgfqpoint{3.900000in}{3.900000in}}%
\pgfusepath{clip}%
\pgfsetbuttcap%
\pgfsetroundjoin%
\definecolor{currentfill}{rgb}{0.121569,0.466667,0.705882}%
\pgfsetfillcolor{currentfill}%
\pgfsetlinewidth{1.003750pt}%
\definecolor{currentstroke}{rgb}{0.121569,0.466667,0.705882}%
\pgfsetstrokecolor{currentstroke}%
\pgfsetdash{}{0pt}%
\pgfpathmoveto{\pgfqpoint{2.552462in}{3.580362in}}%
\pgfpathcurveto{\pgfqpoint{2.563512in}{3.580362in}}{\pgfqpoint{2.574111in}{3.584752in}}{\pgfqpoint{2.581925in}{3.592566in}}%
\pgfpathcurveto{\pgfqpoint{2.589738in}{3.600379in}}{\pgfqpoint{2.594129in}{3.610978in}}{\pgfqpoint{2.594129in}{3.622029in}}%
\pgfpathcurveto{\pgfqpoint{2.594129in}{3.633079in}}{\pgfqpoint{2.589738in}{3.643678in}}{\pgfqpoint{2.581925in}{3.651491in}}%
\pgfpathcurveto{\pgfqpoint{2.574111in}{3.659305in}}{\pgfqpoint{2.563512in}{3.663695in}}{\pgfqpoint{2.552462in}{3.663695in}}%
\pgfpathcurveto{\pgfqpoint{2.541412in}{3.663695in}}{\pgfqpoint{2.530813in}{3.659305in}}{\pgfqpoint{2.522999in}{3.651491in}}%
\pgfpathcurveto{\pgfqpoint{2.515185in}{3.643678in}}{\pgfqpoint{2.510795in}{3.633079in}}{\pgfqpoint{2.510795in}{3.622029in}}%
\pgfpathcurveto{\pgfqpoint{2.510795in}{3.610978in}}{\pgfqpoint{2.515185in}{3.600379in}}{\pgfqpoint{2.522999in}{3.592566in}}%
\pgfpathcurveto{\pgfqpoint{2.530813in}{3.584752in}}{\pgfqpoint{2.541412in}{3.580362in}}{\pgfqpoint{2.552462in}{3.580362in}}%
\pgfpathclose%
\pgfusepath{stroke,fill}%
\end{pgfscope}%
\begin{pgfscope}%
\pgfpathrectangle{\pgfqpoint{0.600000in}{0.600000in}}{\pgfqpoint{3.900000in}{3.900000in}}%
\pgfusepath{clip}%
\pgfsetbuttcap%
\pgfsetroundjoin%
\definecolor{currentfill}{rgb}{0.121569,0.466667,0.705882}%
\pgfsetfillcolor{currentfill}%
\pgfsetlinewidth{1.003750pt}%
\definecolor{currentstroke}{rgb}{0.121569,0.466667,0.705882}%
\pgfsetstrokecolor{currentstroke}%
\pgfsetdash{}{0pt}%
\pgfpathmoveto{\pgfqpoint{3.302047in}{2.079070in}}%
\pgfpathcurveto{\pgfqpoint{3.313097in}{2.079070in}}{\pgfqpoint{3.323696in}{2.083461in}}{\pgfqpoint{3.331510in}{2.091274in}}%
\pgfpathcurveto{\pgfqpoint{3.339324in}{2.099088in}}{\pgfqpoint{3.343714in}{2.109687in}}{\pgfqpoint{3.343714in}{2.120737in}}%
\pgfpathcurveto{\pgfqpoint{3.343714in}{2.131787in}}{\pgfqpoint{3.339324in}{2.142386in}}{\pgfqpoint{3.331510in}{2.150200in}}%
\pgfpathcurveto{\pgfqpoint{3.323696in}{2.158013in}}{\pgfqpoint{3.313097in}{2.162404in}}{\pgfqpoint{3.302047in}{2.162404in}}%
\pgfpathcurveto{\pgfqpoint{3.290997in}{2.162404in}}{\pgfqpoint{3.280398in}{2.158013in}}{\pgfqpoint{3.272584in}{2.150200in}}%
\pgfpathcurveto{\pgfqpoint{3.264771in}{2.142386in}}{\pgfqpoint{3.260381in}{2.131787in}}{\pgfqpoint{3.260381in}{2.120737in}}%
\pgfpathcurveto{\pgfqpoint{3.260381in}{2.109687in}}{\pgfqpoint{3.264771in}{2.099088in}}{\pgfqpoint{3.272584in}{2.091274in}}%
\pgfpathcurveto{\pgfqpoint{3.280398in}{2.083461in}}{\pgfqpoint{3.290997in}{2.079070in}}{\pgfqpoint{3.302047in}{2.079070in}}%
\pgfpathclose%
\pgfusepath{stroke,fill}%
\end{pgfscope}%
\begin{pgfscope}%
\pgfpathrectangle{\pgfqpoint{0.600000in}{0.600000in}}{\pgfqpoint{3.900000in}{3.900000in}}%
\pgfusepath{clip}%
\pgfsetbuttcap%
\pgfsetroundjoin%
\definecolor{currentfill}{rgb}{0.121569,0.466667,0.705882}%
\pgfsetfillcolor{currentfill}%
\pgfsetlinewidth{1.003750pt}%
\definecolor{currentstroke}{rgb}{0.121569,0.466667,0.705882}%
\pgfsetstrokecolor{currentstroke}%
\pgfsetdash{}{0pt}%
\pgfpathmoveto{\pgfqpoint{2.909663in}{2.686164in}}%
\pgfpathcurveto{\pgfqpoint{2.920713in}{2.686164in}}{\pgfqpoint{2.931312in}{2.690554in}}{\pgfqpoint{2.939126in}{2.698368in}}%
\pgfpathcurveto{\pgfqpoint{2.946939in}{2.706181in}}{\pgfqpoint{2.951330in}{2.716780in}}{\pgfqpoint{2.951330in}{2.727831in}}%
\pgfpathcurveto{\pgfqpoint{2.951330in}{2.738881in}}{\pgfqpoint{2.946939in}{2.749480in}}{\pgfqpoint{2.939126in}{2.757293in}}%
\pgfpathcurveto{\pgfqpoint{2.931312in}{2.765107in}}{\pgfqpoint{2.920713in}{2.769497in}}{\pgfqpoint{2.909663in}{2.769497in}}%
\pgfpathcurveto{\pgfqpoint{2.898613in}{2.769497in}}{\pgfqpoint{2.888014in}{2.765107in}}{\pgfqpoint{2.880200in}{2.757293in}}%
\pgfpathcurveto{\pgfqpoint{2.872387in}{2.749480in}}{\pgfqpoint{2.867996in}{2.738881in}}{\pgfqpoint{2.867996in}{2.727831in}}%
\pgfpathcurveto{\pgfqpoint{2.867996in}{2.716780in}}{\pgfqpoint{2.872387in}{2.706181in}}{\pgfqpoint{2.880200in}{2.698368in}}%
\pgfpathcurveto{\pgfqpoint{2.888014in}{2.690554in}}{\pgfqpoint{2.898613in}{2.686164in}}{\pgfqpoint{2.909663in}{2.686164in}}%
\pgfpathclose%
\pgfusepath{stroke,fill}%
\end{pgfscope}%
\begin{pgfscope}%
\pgfpathrectangle{\pgfqpoint{0.600000in}{0.600000in}}{\pgfqpoint{3.900000in}{3.900000in}}%
\pgfusepath{clip}%
\pgfsetbuttcap%
\pgfsetroundjoin%
\definecolor{currentfill}{rgb}{0.121569,0.466667,0.705882}%
\pgfsetfillcolor{currentfill}%
\pgfsetlinewidth{1.003750pt}%
\definecolor{currentstroke}{rgb}{0.121569,0.466667,0.705882}%
\pgfsetstrokecolor{currentstroke}%
\pgfsetdash{}{0pt}%
\pgfpathmoveto{\pgfqpoint{2.985586in}{2.268794in}}%
\pgfpathcurveto{\pgfqpoint{2.996636in}{2.268794in}}{\pgfqpoint{3.007235in}{2.273184in}}{\pgfqpoint{3.015049in}{2.280998in}}%
\pgfpathcurveto{\pgfqpoint{3.022862in}{2.288811in}}{\pgfqpoint{3.027253in}{2.299410in}}{\pgfqpoint{3.027253in}{2.310460in}}%
\pgfpathcurveto{\pgfqpoint{3.027253in}{2.321511in}}{\pgfqpoint{3.022862in}{2.332110in}}{\pgfqpoint{3.015049in}{2.339923in}}%
\pgfpathcurveto{\pgfqpoint{3.007235in}{2.347737in}}{\pgfqpoint{2.996636in}{2.352127in}}{\pgfqpoint{2.985586in}{2.352127in}}%
\pgfpathcurveto{\pgfqpoint{2.974536in}{2.352127in}}{\pgfqpoint{2.963937in}{2.347737in}}{\pgfqpoint{2.956123in}{2.339923in}}%
\pgfpathcurveto{\pgfqpoint{2.948310in}{2.332110in}}{\pgfqpoint{2.943919in}{2.321511in}}{\pgfqpoint{2.943919in}{2.310460in}}%
\pgfpathcurveto{\pgfqpoint{2.943919in}{2.299410in}}{\pgfqpoint{2.948310in}{2.288811in}}{\pgfqpoint{2.956123in}{2.280998in}}%
\pgfpathcurveto{\pgfqpoint{2.963937in}{2.273184in}}{\pgfqpoint{2.974536in}{2.268794in}}{\pgfqpoint{2.985586in}{2.268794in}}%
\pgfpathclose%
\pgfusepath{stroke,fill}%
\end{pgfscope}%
\begin{pgfscope}%
\pgfpathrectangle{\pgfqpoint{0.600000in}{0.600000in}}{\pgfqpoint{3.900000in}{3.900000in}}%
\pgfusepath{clip}%
\pgfsetbuttcap%
\pgfsetroundjoin%
\definecolor{currentfill}{rgb}{0.121569,0.466667,0.705882}%
\pgfsetfillcolor{currentfill}%
\pgfsetlinewidth{1.003750pt}%
\definecolor{currentstroke}{rgb}{0.121569,0.466667,0.705882}%
\pgfsetstrokecolor{currentstroke}%
\pgfsetdash{}{0pt}%
\pgfpathmoveto{\pgfqpoint{1.167140in}{2.045700in}}%
\pgfpathcurveto{\pgfqpoint{1.178190in}{2.045700in}}{\pgfqpoint{1.188789in}{2.050091in}}{\pgfqpoint{1.196603in}{2.057904in}}%
\pgfpathcurveto{\pgfqpoint{1.204417in}{2.065718in}}{\pgfqpoint{1.208807in}{2.076317in}}{\pgfqpoint{1.208807in}{2.087367in}}%
\pgfpathcurveto{\pgfqpoint{1.208807in}{2.098417in}}{\pgfqpoint{1.204417in}{2.109016in}}{\pgfqpoint{1.196603in}{2.116830in}}%
\pgfpathcurveto{\pgfqpoint{1.188789in}{2.124643in}}{\pgfqpoint{1.178190in}{2.129034in}}{\pgfqpoint{1.167140in}{2.129034in}}%
\pgfpathcurveto{\pgfqpoint{1.156090in}{2.129034in}}{\pgfqpoint{1.145491in}{2.124643in}}{\pgfqpoint{1.137678in}{2.116830in}}%
\pgfpathcurveto{\pgfqpoint{1.129864in}{2.109016in}}{\pgfqpoint{1.125474in}{2.098417in}}{\pgfqpoint{1.125474in}{2.087367in}}%
\pgfpathcurveto{\pgfqpoint{1.125474in}{2.076317in}}{\pgfqpoint{1.129864in}{2.065718in}}{\pgfqpoint{1.137678in}{2.057904in}}%
\pgfpathcurveto{\pgfqpoint{1.145491in}{2.050091in}}{\pgfqpoint{1.156090in}{2.045700in}}{\pgfqpoint{1.167140in}{2.045700in}}%
\pgfpathclose%
\pgfusepath{stroke,fill}%
\end{pgfscope}%
\begin{pgfscope}%
\pgfpathrectangle{\pgfqpoint{0.600000in}{0.600000in}}{\pgfqpoint{3.900000in}{3.900000in}}%
\pgfusepath{clip}%
\pgfsetbuttcap%
\pgfsetroundjoin%
\definecolor{currentfill}{rgb}{0.121569,0.466667,0.705882}%
\pgfsetfillcolor{currentfill}%
\pgfsetlinewidth{1.003750pt}%
\definecolor{currentstroke}{rgb}{0.121569,0.466667,0.705882}%
\pgfsetstrokecolor{currentstroke}%
\pgfsetdash{}{0pt}%
\pgfpathmoveto{\pgfqpoint{3.004651in}{2.606402in}}%
\pgfpathcurveto{\pgfqpoint{3.015701in}{2.606402in}}{\pgfqpoint{3.026300in}{2.610792in}}{\pgfqpoint{3.034113in}{2.618606in}}%
\pgfpathcurveto{\pgfqpoint{3.041927in}{2.626419in}}{\pgfqpoint{3.046317in}{2.637018in}}{\pgfqpoint{3.046317in}{2.648069in}}%
\pgfpathcurveto{\pgfqpoint{3.046317in}{2.659119in}}{\pgfqpoint{3.041927in}{2.669718in}}{\pgfqpoint{3.034113in}{2.677531in}}%
\pgfpathcurveto{\pgfqpoint{3.026300in}{2.685345in}}{\pgfqpoint{3.015701in}{2.689735in}}{\pgfqpoint{3.004651in}{2.689735in}}%
\pgfpathcurveto{\pgfqpoint{2.993601in}{2.689735in}}{\pgfqpoint{2.983002in}{2.685345in}}{\pgfqpoint{2.975188in}{2.677531in}}%
\pgfpathcurveto{\pgfqpoint{2.967374in}{2.669718in}}{\pgfqpoint{2.962984in}{2.659119in}}{\pgfqpoint{2.962984in}{2.648069in}}%
\pgfpathcurveto{\pgfqpoint{2.962984in}{2.637018in}}{\pgfqpoint{2.967374in}{2.626419in}}{\pgfqpoint{2.975188in}{2.618606in}}%
\pgfpathcurveto{\pgfqpoint{2.983002in}{2.610792in}}{\pgfqpoint{2.993601in}{2.606402in}}{\pgfqpoint{3.004651in}{2.606402in}}%
\pgfpathclose%
\pgfusepath{stroke,fill}%
\end{pgfscope}%
\begin{pgfscope}%
\pgfpathrectangle{\pgfqpoint{0.600000in}{0.600000in}}{\pgfqpoint{3.900000in}{3.900000in}}%
\pgfusepath{clip}%
\pgfsetbuttcap%
\pgfsetroundjoin%
\definecolor{currentfill}{rgb}{0.121569,0.466667,0.705882}%
\pgfsetfillcolor{currentfill}%
\pgfsetlinewidth{1.003750pt}%
\definecolor{currentstroke}{rgb}{0.121569,0.466667,0.705882}%
\pgfsetstrokecolor{currentstroke}%
\pgfsetdash{}{0pt}%
\pgfpathmoveto{\pgfqpoint{2.602747in}{2.079977in}}%
\pgfpathcurveto{\pgfqpoint{2.613797in}{2.079977in}}{\pgfqpoint{2.624396in}{2.084367in}}{\pgfqpoint{2.632210in}{2.092181in}}%
\pgfpathcurveto{\pgfqpoint{2.640024in}{2.099994in}}{\pgfqpoint{2.644414in}{2.110593in}}{\pgfqpoint{2.644414in}{2.121643in}}%
\pgfpathcurveto{\pgfqpoint{2.644414in}{2.132693in}}{\pgfqpoint{2.640024in}{2.143293in}}{\pgfqpoint{2.632210in}{2.151106in}}%
\pgfpathcurveto{\pgfqpoint{2.624396in}{2.158920in}}{\pgfqpoint{2.613797in}{2.163310in}}{\pgfqpoint{2.602747in}{2.163310in}}%
\pgfpathcurveto{\pgfqpoint{2.591697in}{2.163310in}}{\pgfqpoint{2.581098in}{2.158920in}}{\pgfqpoint{2.573285in}{2.151106in}}%
\pgfpathcurveto{\pgfqpoint{2.565471in}{2.143293in}}{\pgfqpoint{2.561081in}{2.132693in}}{\pgfqpoint{2.561081in}{2.121643in}}%
\pgfpathcurveto{\pgfqpoint{2.561081in}{2.110593in}}{\pgfqpoint{2.565471in}{2.099994in}}{\pgfqpoint{2.573285in}{2.092181in}}%
\pgfpathcurveto{\pgfqpoint{2.581098in}{2.084367in}}{\pgfqpoint{2.591697in}{2.079977in}}{\pgfqpoint{2.602747in}{2.079977in}}%
\pgfpathclose%
\pgfusepath{stroke,fill}%
\end{pgfscope}%
\begin{pgfscope}%
\pgfpathrectangle{\pgfqpoint{0.600000in}{0.600000in}}{\pgfqpoint{3.900000in}{3.900000in}}%
\pgfusepath{clip}%
\pgfsetbuttcap%
\pgfsetroundjoin%
\definecolor{currentfill}{rgb}{0.121569,0.466667,0.705882}%
\pgfsetfillcolor{currentfill}%
\pgfsetlinewidth{1.003750pt}%
\definecolor{currentstroke}{rgb}{0.121569,0.466667,0.705882}%
\pgfsetstrokecolor{currentstroke}%
\pgfsetdash{}{0pt}%
\pgfpathmoveto{\pgfqpoint{2.124123in}{2.238426in}}%
\pgfpathcurveto{\pgfqpoint{2.135173in}{2.238426in}}{\pgfqpoint{2.145773in}{2.242817in}}{\pgfqpoint{2.153586in}{2.250630in}}%
\pgfpathcurveto{\pgfqpoint{2.161400in}{2.258444in}}{\pgfqpoint{2.165790in}{2.269043in}}{\pgfqpoint{2.165790in}{2.280093in}}%
\pgfpathcurveto{\pgfqpoint{2.165790in}{2.291143in}}{\pgfqpoint{2.161400in}{2.301742in}}{\pgfqpoint{2.153586in}{2.309556in}}%
\pgfpathcurveto{\pgfqpoint{2.145773in}{2.317369in}}{\pgfqpoint{2.135173in}{2.321760in}}{\pgfqpoint{2.124123in}{2.321760in}}%
\pgfpathcurveto{\pgfqpoint{2.113073in}{2.321760in}}{\pgfqpoint{2.102474in}{2.317369in}}{\pgfqpoint{2.094661in}{2.309556in}}%
\pgfpathcurveto{\pgfqpoint{2.086847in}{2.301742in}}{\pgfqpoint{2.082457in}{2.291143in}}{\pgfqpoint{2.082457in}{2.280093in}}%
\pgfpathcurveto{\pgfqpoint{2.082457in}{2.269043in}}{\pgfqpoint{2.086847in}{2.258444in}}{\pgfqpoint{2.094661in}{2.250630in}}%
\pgfpathcurveto{\pgfqpoint{2.102474in}{2.242817in}}{\pgfqpoint{2.113073in}{2.238426in}}{\pgfqpoint{2.124123in}{2.238426in}}%
\pgfpathclose%
\pgfusepath{stroke,fill}%
\end{pgfscope}%
\begin{pgfscope}%
\pgfpathrectangle{\pgfqpoint{0.600000in}{0.600000in}}{\pgfqpoint{3.900000in}{3.900000in}}%
\pgfusepath{clip}%
\pgfsetbuttcap%
\pgfsetroundjoin%
\definecolor{currentfill}{rgb}{0.121569,0.466667,0.705882}%
\pgfsetfillcolor{currentfill}%
\pgfsetlinewidth{1.003750pt}%
\definecolor{currentstroke}{rgb}{0.121569,0.466667,0.705882}%
\pgfsetstrokecolor{currentstroke}%
\pgfsetdash{}{0pt}%
\pgfpathmoveto{\pgfqpoint{3.035640in}{1.273137in}}%
\pgfpathcurveto{\pgfqpoint{3.046691in}{1.273137in}}{\pgfqpoint{3.057290in}{1.277528in}}{\pgfqpoint{3.065103in}{1.285341in}}%
\pgfpathcurveto{\pgfqpoint{3.072917in}{1.293155in}}{\pgfqpoint{3.077307in}{1.303754in}}{\pgfqpoint{3.077307in}{1.314804in}}%
\pgfpathcurveto{\pgfqpoint{3.077307in}{1.325854in}}{\pgfqpoint{3.072917in}{1.336453in}}{\pgfqpoint{3.065103in}{1.344267in}}%
\pgfpathcurveto{\pgfqpoint{3.057290in}{1.352080in}}{\pgfqpoint{3.046691in}{1.356471in}}{\pgfqpoint{3.035640in}{1.356471in}}%
\pgfpathcurveto{\pgfqpoint{3.024590in}{1.356471in}}{\pgfqpoint{3.013991in}{1.352080in}}{\pgfqpoint{3.006178in}{1.344267in}}%
\pgfpathcurveto{\pgfqpoint{2.998364in}{1.336453in}}{\pgfqpoint{2.993974in}{1.325854in}}{\pgfqpoint{2.993974in}{1.314804in}}%
\pgfpathcurveto{\pgfqpoint{2.993974in}{1.303754in}}{\pgfqpoint{2.998364in}{1.293155in}}{\pgfqpoint{3.006178in}{1.285341in}}%
\pgfpathcurveto{\pgfqpoint{3.013991in}{1.277528in}}{\pgfqpoint{3.024590in}{1.273137in}}{\pgfqpoint{3.035640in}{1.273137in}}%
\pgfpathclose%
\pgfusepath{stroke,fill}%
\end{pgfscope}%
\begin{pgfscope}%
\pgfpathrectangle{\pgfqpoint{0.600000in}{0.600000in}}{\pgfqpoint{3.900000in}{3.900000in}}%
\pgfusepath{clip}%
\pgfsetbuttcap%
\pgfsetroundjoin%
\definecolor{currentfill}{rgb}{0.121569,0.466667,0.705882}%
\pgfsetfillcolor{currentfill}%
\pgfsetlinewidth{1.003750pt}%
\definecolor{currentstroke}{rgb}{0.121569,0.466667,0.705882}%
\pgfsetstrokecolor{currentstroke}%
\pgfsetdash{}{0pt}%
\pgfpathmoveto{\pgfqpoint{3.567502in}{2.557401in}}%
\pgfpathcurveto{\pgfqpoint{3.578552in}{2.557401in}}{\pgfqpoint{3.589151in}{2.561791in}}{\pgfqpoint{3.596965in}{2.569605in}}%
\pgfpathcurveto{\pgfqpoint{3.604778in}{2.577419in}}{\pgfqpoint{3.609168in}{2.588018in}}{\pgfqpoint{3.609168in}{2.599068in}}%
\pgfpathcurveto{\pgfqpoint{3.609168in}{2.610118in}}{\pgfqpoint{3.604778in}{2.620717in}}{\pgfqpoint{3.596965in}{2.628531in}}%
\pgfpathcurveto{\pgfqpoint{3.589151in}{2.636344in}}{\pgfqpoint{3.578552in}{2.640734in}}{\pgfqpoint{3.567502in}{2.640734in}}%
\pgfpathcurveto{\pgfqpoint{3.556452in}{2.640734in}}{\pgfqpoint{3.545853in}{2.636344in}}{\pgfqpoint{3.538039in}{2.628531in}}%
\pgfpathcurveto{\pgfqpoint{3.530225in}{2.620717in}}{\pgfqpoint{3.525835in}{2.610118in}}{\pgfqpoint{3.525835in}{2.599068in}}%
\pgfpathcurveto{\pgfqpoint{3.525835in}{2.588018in}}{\pgfqpoint{3.530225in}{2.577419in}}{\pgfqpoint{3.538039in}{2.569605in}}%
\pgfpathcurveto{\pgfqpoint{3.545853in}{2.561791in}}{\pgfqpoint{3.556452in}{2.557401in}}{\pgfqpoint{3.567502in}{2.557401in}}%
\pgfpathclose%
\pgfusepath{stroke,fill}%
\end{pgfscope}%
\begin{pgfscope}%
\pgfpathrectangle{\pgfqpoint{0.600000in}{0.600000in}}{\pgfqpoint{3.900000in}{3.900000in}}%
\pgfusepath{clip}%
\pgfsetbuttcap%
\pgfsetroundjoin%
\definecolor{currentfill}{rgb}{0.121569,0.466667,0.705882}%
\pgfsetfillcolor{currentfill}%
\pgfsetlinewidth{1.003750pt}%
\definecolor{currentstroke}{rgb}{0.121569,0.466667,0.705882}%
\pgfsetstrokecolor{currentstroke}%
\pgfsetdash{}{0pt}%
\pgfpathmoveto{\pgfqpoint{2.756722in}{3.029864in}}%
\pgfpathcurveto{\pgfqpoint{2.767772in}{3.029864in}}{\pgfqpoint{2.778371in}{3.034254in}}{\pgfqpoint{2.786184in}{3.042068in}}%
\pgfpathcurveto{\pgfqpoint{2.793998in}{3.049881in}}{\pgfqpoint{2.798388in}{3.060480in}}{\pgfqpoint{2.798388in}{3.071531in}}%
\pgfpathcurveto{\pgfqpoint{2.798388in}{3.082581in}}{\pgfqpoint{2.793998in}{3.093180in}}{\pgfqpoint{2.786184in}{3.100993in}}%
\pgfpathcurveto{\pgfqpoint{2.778371in}{3.108807in}}{\pgfqpoint{2.767772in}{3.113197in}}{\pgfqpoint{2.756722in}{3.113197in}}%
\pgfpathcurveto{\pgfqpoint{2.745671in}{3.113197in}}{\pgfqpoint{2.735072in}{3.108807in}}{\pgfqpoint{2.727259in}{3.100993in}}%
\pgfpathcurveto{\pgfqpoint{2.719445in}{3.093180in}}{\pgfqpoint{2.715055in}{3.082581in}}{\pgfqpoint{2.715055in}{3.071531in}}%
\pgfpathcurveto{\pgfqpoint{2.715055in}{3.060480in}}{\pgfqpoint{2.719445in}{3.049881in}}{\pgfqpoint{2.727259in}{3.042068in}}%
\pgfpathcurveto{\pgfqpoint{2.735072in}{3.034254in}}{\pgfqpoint{2.745671in}{3.029864in}}{\pgfqpoint{2.756722in}{3.029864in}}%
\pgfpathclose%
\pgfusepath{stroke,fill}%
\end{pgfscope}%
\begin{pgfscope}%
\pgfpathrectangle{\pgfqpoint{0.600000in}{0.600000in}}{\pgfqpoint{3.900000in}{3.900000in}}%
\pgfusepath{clip}%
\pgfsetbuttcap%
\pgfsetroundjoin%
\definecolor{currentfill}{rgb}{0.121569,0.466667,0.705882}%
\pgfsetfillcolor{currentfill}%
\pgfsetlinewidth{1.003750pt}%
\definecolor{currentstroke}{rgb}{0.121569,0.466667,0.705882}%
\pgfsetstrokecolor{currentstroke}%
\pgfsetdash{}{0pt}%
\pgfpathmoveto{\pgfqpoint{2.234798in}{3.098021in}}%
\pgfpathcurveto{\pgfqpoint{2.245848in}{3.098021in}}{\pgfqpoint{2.256448in}{3.102411in}}{\pgfqpoint{2.264261in}{3.110224in}}%
\pgfpathcurveto{\pgfqpoint{2.272075in}{3.118038in}}{\pgfqpoint{2.276465in}{3.128637in}}{\pgfqpoint{2.276465in}{3.139687in}}%
\pgfpathcurveto{\pgfqpoint{2.276465in}{3.150737in}}{\pgfqpoint{2.272075in}{3.161336in}}{\pgfqpoint{2.264261in}{3.169150in}}%
\pgfpathcurveto{\pgfqpoint{2.256448in}{3.176964in}}{\pgfqpoint{2.245848in}{3.181354in}}{\pgfqpoint{2.234798in}{3.181354in}}%
\pgfpathcurveto{\pgfqpoint{2.223748in}{3.181354in}}{\pgfqpoint{2.213149in}{3.176964in}}{\pgfqpoint{2.205336in}{3.169150in}}%
\pgfpathcurveto{\pgfqpoint{2.197522in}{3.161336in}}{\pgfqpoint{2.193132in}{3.150737in}}{\pgfqpoint{2.193132in}{3.139687in}}%
\pgfpathcurveto{\pgfqpoint{2.193132in}{3.128637in}}{\pgfqpoint{2.197522in}{3.118038in}}{\pgfqpoint{2.205336in}{3.110224in}}%
\pgfpathcurveto{\pgfqpoint{2.213149in}{3.102411in}}{\pgfqpoint{2.223748in}{3.098021in}}{\pgfqpoint{2.234798in}{3.098021in}}%
\pgfpathclose%
\pgfusepath{stroke,fill}%
\end{pgfscope}%
\begin{pgfscope}%
\pgfpathrectangle{\pgfqpoint{0.600000in}{0.600000in}}{\pgfqpoint{3.900000in}{3.900000in}}%
\pgfusepath{clip}%
\pgfsetbuttcap%
\pgfsetroundjoin%
\definecolor{currentfill}{rgb}{0.121569,0.466667,0.705882}%
\pgfsetfillcolor{currentfill}%
\pgfsetlinewidth{1.003750pt}%
\definecolor{currentstroke}{rgb}{0.121569,0.466667,0.705882}%
\pgfsetstrokecolor{currentstroke}%
\pgfsetdash{}{0pt}%
\pgfpathmoveto{\pgfqpoint{2.775965in}{1.789277in}}%
\pgfpathcurveto{\pgfqpoint{2.787015in}{1.789277in}}{\pgfqpoint{2.797614in}{1.793667in}}{\pgfqpoint{2.805428in}{1.801481in}}%
\pgfpathcurveto{\pgfqpoint{2.813242in}{1.809294in}}{\pgfqpoint{2.817632in}{1.819893in}}{\pgfqpoint{2.817632in}{1.830944in}}%
\pgfpathcurveto{\pgfqpoint{2.817632in}{1.841994in}}{\pgfqpoint{2.813242in}{1.852593in}}{\pgfqpoint{2.805428in}{1.860406in}}%
\pgfpathcurveto{\pgfqpoint{2.797614in}{1.868220in}}{\pgfqpoint{2.787015in}{1.872610in}}{\pgfqpoint{2.775965in}{1.872610in}}%
\pgfpathcurveto{\pgfqpoint{2.764915in}{1.872610in}}{\pgfqpoint{2.754316in}{1.868220in}}{\pgfqpoint{2.746502in}{1.860406in}}%
\pgfpathcurveto{\pgfqpoint{2.738689in}{1.852593in}}{\pgfqpoint{2.734298in}{1.841994in}}{\pgfqpoint{2.734298in}{1.830944in}}%
\pgfpathcurveto{\pgfqpoint{2.734298in}{1.819893in}}{\pgfqpoint{2.738689in}{1.809294in}}{\pgfqpoint{2.746502in}{1.801481in}}%
\pgfpathcurveto{\pgfqpoint{2.754316in}{1.793667in}}{\pgfqpoint{2.764915in}{1.789277in}}{\pgfqpoint{2.775965in}{1.789277in}}%
\pgfpathclose%
\pgfusepath{stroke,fill}%
\end{pgfscope}%
\begin{pgfscope}%
\pgfpathrectangle{\pgfqpoint{0.600000in}{0.600000in}}{\pgfqpoint{3.900000in}{3.900000in}}%
\pgfusepath{clip}%
\pgfsetbuttcap%
\pgfsetroundjoin%
\definecolor{currentfill}{rgb}{0.121569,0.466667,0.705882}%
\pgfsetfillcolor{currentfill}%
\pgfsetlinewidth{1.003750pt}%
\definecolor{currentstroke}{rgb}{0.121569,0.466667,0.705882}%
\pgfsetstrokecolor{currentstroke}%
\pgfsetdash{}{0pt}%
\pgfpathmoveto{\pgfqpoint{2.641058in}{1.705390in}}%
\pgfpathcurveto{\pgfqpoint{2.652108in}{1.705390in}}{\pgfqpoint{2.662707in}{1.709780in}}{\pgfqpoint{2.670521in}{1.717594in}}%
\pgfpathcurveto{\pgfqpoint{2.678335in}{1.725408in}}{\pgfqpoint{2.682725in}{1.736007in}}{\pgfqpoint{2.682725in}{1.747057in}}%
\pgfpathcurveto{\pgfqpoint{2.682725in}{1.758107in}}{\pgfqpoint{2.678335in}{1.768706in}}{\pgfqpoint{2.670521in}{1.776520in}}%
\pgfpathcurveto{\pgfqpoint{2.662707in}{1.784333in}}{\pgfqpoint{2.652108in}{1.788723in}}{\pgfqpoint{2.641058in}{1.788723in}}%
\pgfpathcurveto{\pgfqpoint{2.630008in}{1.788723in}}{\pgfqpoint{2.619409in}{1.784333in}}{\pgfqpoint{2.611596in}{1.776520in}}%
\pgfpathcurveto{\pgfqpoint{2.603782in}{1.768706in}}{\pgfqpoint{2.599392in}{1.758107in}}{\pgfqpoint{2.599392in}{1.747057in}}%
\pgfpathcurveto{\pgfqpoint{2.599392in}{1.736007in}}{\pgfqpoint{2.603782in}{1.725408in}}{\pgfqpoint{2.611596in}{1.717594in}}%
\pgfpathcurveto{\pgfqpoint{2.619409in}{1.709780in}}{\pgfqpoint{2.630008in}{1.705390in}}{\pgfqpoint{2.641058in}{1.705390in}}%
\pgfpathclose%
\pgfusepath{stroke,fill}%
\end{pgfscope}%
\begin{pgfscope}%
\pgfpathrectangle{\pgfqpoint{0.600000in}{0.600000in}}{\pgfqpoint{3.900000in}{3.900000in}}%
\pgfusepath{clip}%
\pgfsetbuttcap%
\pgfsetroundjoin%
\definecolor{currentfill}{rgb}{0.121569,0.466667,0.705882}%
\pgfsetfillcolor{currentfill}%
\pgfsetlinewidth{1.003750pt}%
\definecolor{currentstroke}{rgb}{0.121569,0.466667,0.705882}%
\pgfsetstrokecolor{currentstroke}%
\pgfsetdash{}{0pt}%
\pgfpathmoveto{\pgfqpoint{2.546604in}{2.441136in}}%
\pgfpathcurveto{\pgfqpoint{2.557654in}{2.441136in}}{\pgfqpoint{2.568253in}{2.445526in}}{\pgfqpoint{2.576067in}{2.453340in}}%
\pgfpathcurveto{\pgfqpoint{2.583881in}{2.461154in}}{\pgfqpoint{2.588271in}{2.471753in}}{\pgfqpoint{2.588271in}{2.482803in}}%
\pgfpathcurveto{\pgfqpoint{2.588271in}{2.493853in}}{\pgfqpoint{2.583881in}{2.504452in}}{\pgfqpoint{2.576067in}{2.512265in}}%
\pgfpathcurveto{\pgfqpoint{2.568253in}{2.520079in}}{\pgfqpoint{2.557654in}{2.524469in}}{\pgfqpoint{2.546604in}{2.524469in}}%
\pgfpathcurveto{\pgfqpoint{2.535554in}{2.524469in}}{\pgfqpoint{2.524955in}{2.520079in}}{\pgfqpoint{2.517141in}{2.512265in}}%
\pgfpathcurveto{\pgfqpoint{2.509328in}{2.504452in}}{\pgfqpoint{2.504937in}{2.493853in}}{\pgfqpoint{2.504937in}{2.482803in}}%
\pgfpathcurveto{\pgfqpoint{2.504937in}{2.471753in}}{\pgfqpoint{2.509328in}{2.461154in}}{\pgfqpoint{2.517141in}{2.453340in}}%
\pgfpathcurveto{\pgfqpoint{2.524955in}{2.445526in}}{\pgfqpoint{2.535554in}{2.441136in}}{\pgfqpoint{2.546604in}{2.441136in}}%
\pgfpathclose%
\pgfusepath{stroke,fill}%
\end{pgfscope}%
\begin{pgfscope}%
\pgfpathrectangle{\pgfqpoint{0.600000in}{0.600000in}}{\pgfqpoint{3.900000in}{3.900000in}}%
\pgfusepath{clip}%
\pgfsetbuttcap%
\pgfsetroundjoin%
\definecolor{currentfill}{rgb}{0.121569,0.466667,0.705882}%
\pgfsetfillcolor{currentfill}%
\pgfsetlinewidth{1.003750pt}%
\definecolor{currentstroke}{rgb}{0.121569,0.466667,0.705882}%
\pgfsetstrokecolor{currentstroke}%
\pgfsetdash{}{0pt}%
\pgfpathmoveto{\pgfqpoint{1.976268in}{1.906236in}}%
\pgfpathcurveto{\pgfqpoint{1.987318in}{1.906236in}}{\pgfqpoint{1.997918in}{1.910627in}}{\pgfqpoint{2.005731in}{1.918440in}}%
\pgfpathcurveto{\pgfqpoint{2.013545in}{1.926254in}}{\pgfqpoint{2.017935in}{1.936853in}}{\pgfqpoint{2.017935in}{1.947903in}}%
\pgfpathcurveto{\pgfqpoint{2.017935in}{1.958953in}}{\pgfqpoint{2.013545in}{1.969552in}}{\pgfqpoint{2.005731in}{1.977366in}}%
\pgfpathcurveto{\pgfqpoint{1.997918in}{1.985179in}}{\pgfqpoint{1.987318in}{1.989570in}}{\pgfqpoint{1.976268in}{1.989570in}}%
\pgfpathcurveto{\pgfqpoint{1.965218in}{1.989570in}}{\pgfqpoint{1.954619in}{1.985179in}}{\pgfqpoint{1.946806in}{1.977366in}}%
\pgfpathcurveto{\pgfqpoint{1.938992in}{1.969552in}}{\pgfqpoint{1.934602in}{1.958953in}}{\pgfqpoint{1.934602in}{1.947903in}}%
\pgfpathcurveto{\pgfqpoint{1.934602in}{1.936853in}}{\pgfqpoint{1.938992in}{1.926254in}}{\pgfqpoint{1.946806in}{1.918440in}}%
\pgfpathcurveto{\pgfqpoint{1.954619in}{1.910627in}}{\pgfqpoint{1.965218in}{1.906236in}}{\pgfqpoint{1.976268in}{1.906236in}}%
\pgfpathclose%
\pgfusepath{stroke,fill}%
\end{pgfscope}%
\begin{pgfscope}%
\pgfpathrectangle{\pgfqpoint{0.600000in}{0.600000in}}{\pgfqpoint{3.900000in}{3.900000in}}%
\pgfusepath{clip}%
\pgfsetbuttcap%
\pgfsetroundjoin%
\definecolor{currentfill}{rgb}{0.121569,0.466667,0.705882}%
\pgfsetfillcolor{currentfill}%
\pgfsetlinewidth{1.003750pt}%
\definecolor{currentstroke}{rgb}{0.121569,0.466667,0.705882}%
\pgfsetstrokecolor{currentstroke}%
\pgfsetdash{}{0pt}%
\pgfpathmoveto{\pgfqpoint{2.324783in}{0.742163in}}%
\pgfpathcurveto{\pgfqpoint{2.335833in}{0.742163in}}{\pgfqpoint{2.346432in}{0.746553in}}{\pgfqpoint{2.354246in}{0.754367in}}%
\pgfpathcurveto{\pgfqpoint{2.362059in}{0.762180in}}{\pgfqpoint{2.366450in}{0.772779in}}{\pgfqpoint{2.366450in}{0.783830in}}%
\pgfpathcurveto{\pgfqpoint{2.366450in}{0.794880in}}{\pgfqpoint{2.362059in}{0.805479in}}{\pgfqpoint{2.354246in}{0.813292in}}%
\pgfpathcurveto{\pgfqpoint{2.346432in}{0.821106in}}{\pgfqpoint{2.335833in}{0.825496in}}{\pgfqpoint{2.324783in}{0.825496in}}%
\pgfpathcurveto{\pgfqpoint{2.313733in}{0.825496in}}{\pgfqpoint{2.303134in}{0.821106in}}{\pgfqpoint{2.295320in}{0.813292in}}%
\pgfpathcurveto{\pgfqpoint{2.287506in}{0.805479in}}{\pgfqpoint{2.283116in}{0.794880in}}{\pgfqpoint{2.283116in}{0.783830in}}%
\pgfpathcurveto{\pgfqpoint{2.283116in}{0.772779in}}{\pgfqpoint{2.287506in}{0.762180in}}{\pgfqpoint{2.295320in}{0.754367in}}%
\pgfpathcurveto{\pgfqpoint{2.303134in}{0.746553in}}{\pgfqpoint{2.313733in}{0.742163in}}{\pgfqpoint{2.324783in}{0.742163in}}%
\pgfpathclose%
\pgfusepath{stroke,fill}%
\end{pgfscope}%
\begin{pgfscope}%
\pgfpathrectangle{\pgfqpoint{0.600000in}{0.600000in}}{\pgfqpoint{3.900000in}{3.900000in}}%
\pgfusepath{clip}%
\pgfsetbuttcap%
\pgfsetroundjoin%
\definecolor{currentfill}{rgb}{0.121569,0.466667,0.705882}%
\pgfsetfillcolor{currentfill}%
\pgfsetlinewidth{1.003750pt}%
\definecolor{currentstroke}{rgb}{0.121569,0.466667,0.705882}%
\pgfsetstrokecolor{currentstroke}%
\pgfsetdash{}{0pt}%
\pgfpathmoveto{\pgfqpoint{2.661087in}{3.032261in}}%
\pgfpathcurveto{\pgfqpoint{2.672137in}{3.032261in}}{\pgfqpoint{2.682736in}{3.036651in}}{\pgfqpoint{2.690549in}{3.044464in}}%
\pgfpathcurveto{\pgfqpoint{2.698363in}{3.052278in}}{\pgfqpoint{2.702753in}{3.062877in}}{\pgfqpoint{2.702753in}{3.073927in}}%
\pgfpathcurveto{\pgfqpoint{2.702753in}{3.084977in}}{\pgfqpoint{2.698363in}{3.095576in}}{\pgfqpoint{2.690549in}{3.103390in}}%
\pgfpathcurveto{\pgfqpoint{2.682736in}{3.111204in}}{\pgfqpoint{2.672137in}{3.115594in}}{\pgfqpoint{2.661087in}{3.115594in}}%
\pgfpathcurveto{\pgfqpoint{2.650036in}{3.115594in}}{\pgfqpoint{2.639437in}{3.111204in}}{\pgfqpoint{2.631624in}{3.103390in}}%
\pgfpathcurveto{\pgfqpoint{2.623810in}{3.095576in}}{\pgfqpoint{2.619420in}{3.084977in}}{\pgfqpoint{2.619420in}{3.073927in}}%
\pgfpathcurveto{\pgfqpoint{2.619420in}{3.062877in}}{\pgfqpoint{2.623810in}{3.052278in}}{\pgfqpoint{2.631624in}{3.044464in}}%
\pgfpathcurveto{\pgfqpoint{2.639437in}{3.036651in}}{\pgfqpoint{2.650036in}{3.032261in}}{\pgfqpoint{2.661087in}{3.032261in}}%
\pgfpathclose%
\pgfusepath{stroke,fill}%
\end{pgfscope}%
\begin{pgfscope}%
\pgfpathrectangle{\pgfqpoint{0.600000in}{0.600000in}}{\pgfqpoint{3.900000in}{3.900000in}}%
\pgfusepath{clip}%
\pgfsetbuttcap%
\pgfsetroundjoin%
\definecolor{currentfill}{rgb}{0.121569,0.466667,0.705882}%
\pgfsetfillcolor{currentfill}%
\pgfsetlinewidth{1.003750pt}%
\definecolor{currentstroke}{rgb}{0.121569,0.466667,0.705882}%
\pgfsetstrokecolor{currentstroke}%
\pgfsetdash{}{0pt}%
\pgfpathmoveto{\pgfqpoint{2.034400in}{1.911716in}}%
\pgfpathcurveto{\pgfqpoint{2.045450in}{1.911716in}}{\pgfqpoint{2.056049in}{1.916107in}}{\pgfqpoint{2.063863in}{1.923920in}}%
\pgfpathcurveto{\pgfqpoint{2.071676in}{1.931734in}}{\pgfqpoint{2.076067in}{1.942333in}}{\pgfqpoint{2.076067in}{1.953383in}}%
\pgfpathcurveto{\pgfqpoint{2.076067in}{1.964433in}}{\pgfqpoint{2.071676in}{1.975032in}}{\pgfqpoint{2.063863in}{1.982846in}}%
\pgfpathcurveto{\pgfqpoint{2.056049in}{1.990659in}}{\pgfqpoint{2.045450in}{1.995050in}}{\pgfqpoint{2.034400in}{1.995050in}}%
\pgfpathcurveto{\pgfqpoint{2.023350in}{1.995050in}}{\pgfqpoint{2.012751in}{1.990659in}}{\pgfqpoint{2.004937in}{1.982846in}}%
\pgfpathcurveto{\pgfqpoint{1.997124in}{1.975032in}}{\pgfqpoint{1.992733in}{1.964433in}}{\pgfqpoint{1.992733in}{1.953383in}}%
\pgfpathcurveto{\pgfqpoint{1.992733in}{1.942333in}}{\pgfqpoint{1.997124in}{1.931734in}}{\pgfqpoint{2.004937in}{1.923920in}}%
\pgfpathcurveto{\pgfqpoint{2.012751in}{1.916107in}}{\pgfqpoint{2.023350in}{1.911716in}}{\pgfqpoint{2.034400in}{1.911716in}}%
\pgfpathclose%
\pgfusepath{stroke,fill}%
\end{pgfscope}%
\begin{pgfscope}%
\pgfpathrectangle{\pgfqpoint{0.600000in}{0.600000in}}{\pgfqpoint{3.900000in}{3.900000in}}%
\pgfusepath{clip}%
\pgfsetbuttcap%
\pgfsetroundjoin%
\definecolor{currentfill}{rgb}{0.121569,0.466667,0.705882}%
\pgfsetfillcolor{currentfill}%
\pgfsetlinewidth{1.003750pt}%
\definecolor{currentstroke}{rgb}{0.121569,0.466667,0.705882}%
\pgfsetstrokecolor{currentstroke}%
\pgfsetdash{}{0pt}%
\pgfpathmoveto{\pgfqpoint{2.120558in}{2.476017in}}%
\pgfpathcurveto{\pgfqpoint{2.131608in}{2.476017in}}{\pgfqpoint{2.142207in}{2.480407in}}{\pgfqpoint{2.150021in}{2.488221in}}%
\pgfpathcurveto{\pgfqpoint{2.157835in}{2.496035in}}{\pgfqpoint{2.162225in}{2.506634in}}{\pgfqpoint{2.162225in}{2.517684in}}%
\pgfpathcurveto{\pgfqpoint{2.162225in}{2.528734in}}{\pgfqpoint{2.157835in}{2.539333in}}{\pgfqpoint{2.150021in}{2.547147in}}%
\pgfpathcurveto{\pgfqpoint{2.142207in}{2.554960in}}{\pgfqpoint{2.131608in}{2.559351in}}{\pgfqpoint{2.120558in}{2.559351in}}%
\pgfpathcurveto{\pgfqpoint{2.109508in}{2.559351in}}{\pgfqpoint{2.098909in}{2.554960in}}{\pgfqpoint{2.091095in}{2.547147in}}%
\pgfpathcurveto{\pgfqpoint{2.083282in}{2.539333in}}{\pgfqpoint{2.078891in}{2.528734in}}{\pgfqpoint{2.078891in}{2.517684in}}%
\pgfpathcurveto{\pgfqpoint{2.078891in}{2.506634in}}{\pgfqpoint{2.083282in}{2.496035in}}{\pgfqpoint{2.091095in}{2.488221in}}%
\pgfpathcurveto{\pgfqpoint{2.098909in}{2.480407in}}{\pgfqpoint{2.109508in}{2.476017in}}{\pgfqpoint{2.120558in}{2.476017in}}%
\pgfpathclose%
\pgfusepath{stroke,fill}%
\end{pgfscope}%
\begin{pgfscope}%
\pgfpathrectangle{\pgfqpoint{0.600000in}{0.600000in}}{\pgfqpoint{3.900000in}{3.900000in}}%
\pgfusepath{clip}%
\pgfsetbuttcap%
\pgfsetroundjoin%
\definecolor{currentfill}{rgb}{0.121569,0.466667,0.705882}%
\pgfsetfillcolor{currentfill}%
\pgfsetlinewidth{1.003750pt}%
\definecolor{currentstroke}{rgb}{0.121569,0.466667,0.705882}%
\pgfsetstrokecolor{currentstroke}%
\pgfsetdash{}{0pt}%
\pgfpathmoveto{\pgfqpoint{1.859083in}{2.602008in}}%
\pgfpathcurveto{\pgfqpoint{1.870133in}{2.602008in}}{\pgfqpoint{1.880732in}{2.606399in}}{\pgfqpoint{1.888545in}{2.614212in}}%
\pgfpathcurveto{\pgfqpoint{1.896359in}{2.622026in}}{\pgfqpoint{1.900749in}{2.632625in}}{\pgfqpoint{1.900749in}{2.643675in}}%
\pgfpathcurveto{\pgfqpoint{1.900749in}{2.654725in}}{\pgfqpoint{1.896359in}{2.665324in}}{\pgfqpoint{1.888545in}{2.673138in}}%
\pgfpathcurveto{\pgfqpoint{1.880732in}{2.680951in}}{\pgfqpoint{1.870133in}{2.685342in}}{\pgfqpoint{1.859083in}{2.685342in}}%
\pgfpathcurveto{\pgfqpoint{1.848033in}{2.685342in}}{\pgfqpoint{1.837433in}{2.680951in}}{\pgfqpoint{1.829620in}{2.673138in}}%
\pgfpathcurveto{\pgfqpoint{1.821806in}{2.665324in}}{\pgfqpoint{1.817416in}{2.654725in}}{\pgfqpoint{1.817416in}{2.643675in}}%
\pgfpathcurveto{\pgfqpoint{1.817416in}{2.632625in}}{\pgfqpoint{1.821806in}{2.622026in}}{\pgfqpoint{1.829620in}{2.614212in}}%
\pgfpathcurveto{\pgfqpoint{1.837433in}{2.606399in}}{\pgfqpoint{1.848033in}{2.602008in}}{\pgfqpoint{1.859083in}{2.602008in}}%
\pgfpathclose%
\pgfusepath{stroke,fill}%
\end{pgfscope}%
\begin{pgfscope}%
\pgfpathrectangle{\pgfqpoint{0.600000in}{0.600000in}}{\pgfqpoint{3.900000in}{3.900000in}}%
\pgfusepath{clip}%
\pgfsetbuttcap%
\pgfsetroundjoin%
\definecolor{currentfill}{rgb}{0.121569,0.466667,0.705882}%
\pgfsetfillcolor{currentfill}%
\pgfsetlinewidth{1.003750pt}%
\definecolor{currentstroke}{rgb}{0.121569,0.466667,0.705882}%
\pgfsetstrokecolor{currentstroke}%
\pgfsetdash{}{0pt}%
\pgfpathmoveto{\pgfqpoint{2.791513in}{2.599960in}}%
\pgfpathcurveto{\pgfqpoint{2.802563in}{2.599960in}}{\pgfqpoint{2.813163in}{2.604351in}}{\pgfqpoint{2.820976in}{2.612164in}}%
\pgfpathcurveto{\pgfqpoint{2.828790in}{2.619978in}}{\pgfqpoint{2.833180in}{2.630577in}}{\pgfqpoint{2.833180in}{2.641627in}}%
\pgfpathcurveto{\pgfqpoint{2.833180in}{2.652677in}}{\pgfqpoint{2.828790in}{2.663276in}}{\pgfqpoint{2.820976in}{2.671090in}}%
\pgfpathcurveto{\pgfqpoint{2.813163in}{2.678904in}}{\pgfqpoint{2.802563in}{2.683294in}}{\pgfqpoint{2.791513in}{2.683294in}}%
\pgfpathcurveto{\pgfqpoint{2.780463in}{2.683294in}}{\pgfqpoint{2.769864in}{2.678904in}}{\pgfqpoint{2.762051in}{2.671090in}}%
\pgfpathcurveto{\pgfqpoint{2.754237in}{2.663276in}}{\pgfqpoint{2.749847in}{2.652677in}}{\pgfqpoint{2.749847in}{2.641627in}}%
\pgfpathcurveto{\pgfqpoint{2.749847in}{2.630577in}}{\pgfqpoint{2.754237in}{2.619978in}}{\pgfqpoint{2.762051in}{2.612164in}}%
\pgfpathcurveto{\pgfqpoint{2.769864in}{2.604351in}}{\pgfqpoint{2.780463in}{2.599960in}}{\pgfqpoint{2.791513in}{2.599960in}}%
\pgfpathclose%
\pgfusepath{stroke,fill}%
\end{pgfscope}%
\begin{pgfscope}%
\pgfpathrectangle{\pgfqpoint{0.600000in}{0.600000in}}{\pgfqpoint{3.900000in}{3.900000in}}%
\pgfusepath{clip}%
\pgfsetbuttcap%
\pgfsetroundjoin%
\definecolor{currentfill}{rgb}{0.121569,0.466667,0.705882}%
\pgfsetfillcolor{currentfill}%
\pgfsetlinewidth{1.003750pt}%
\definecolor{currentstroke}{rgb}{0.121569,0.466667,0.705882}%
\pgfsetstrokecolor{currentstroke}%
\pgfsetdash{}{0pt}%
\pgfpathmoveto{\pgfqpoint{2.815357in}{2.252148in}}%
\pgfpathcurveto{\pgfqpoint{2.826407in}{2.252148in}}{\pgfqpoint{2.837006in}{2.256538in}}{\pgfqpoint{2.844819in}{2.264352in}}%
\pgfpathcurveto{\pgfqpoint{2.852633in}{2.272166in}}{\pgfqpoint{2.857023in}{2.282765in}}{\pgfqpoint{2.857023in}{2.293815in}}%
\pgfpathcurveto{\pgfqpoint{2.857023in}{2.304865in}}{\pgfqpoint{2.852633in}{2.315464in}}{\pgfqpoint{2.844819in}{2.323278in}}%
\pgfpathcurveto{\pgfqpoint{2.837006in}{2.331091in}}{\pgfqpoint{2.826407in}{2.335481in}}{\pgfqpoint{2.815357in}{2.335481in}}%
\pgfpathcurveto{\pgfqpoint{2.804307in}{2.335481in}}{\pgfqpoint{2.793708in}{2.331091in}}{\pgfqpoint{2.785894in}{2.323278in}}%
\pgfpathcurveto{\pgfqpoint{2.778080in}{2.315464in}}{\pgfqpoint{2.773690in}{2.304865in}}{\pgfqpoint{2.773690in}{2.293815in}}%
\pgfpathcurveto{\pgfqpoint{2.773690in}{2.282765in}}{\pgfqpoint{2.778080in}{2.272166in}}{\pgfqpoint{2.785894in}{2.264352in}}%
\pgfpathcurveto{\pgfqpoint{2.793708in}{2.256538in}}{\pgfqpoint{2.804307in}{2.252148in}}{\pgfqpoint{2.815357in}{2.252148in}}%
\pgfpathclose%
\pgfusepath{stroke,fill}%
\end{pgfscope}%
\begin{pgfscope}%
\pgfpathrectangle{\pgfqpoint{0.600000in}{0.600000in}}{\pgfqpoint{3.900000in}{3.900000in}}%
\pgfusepath{clip}%
\pgfsetbuttcap%
\pgfsetroundjoin%
\definecolor{currentfill}{rgb}{0.121569,0.466667,0.705882}%
\pgfsetfillcolor{currentfill}%
\pgfsetlinewidth{1.003750pt}%
\definecolor{currentstroke}{rgb}{0.121569,0.466667,0.705882}%
\pgfsetstrokecolor{currentstroke}%
\pgfsetdash{}{0pt}%
\pgfpathmoveto{\pgfqpoint{2.612261in}{2.615681in}}%
\pgfpathcurveto{\pgfqpoint{2.623311in}{2.615681in}}{\pgfqpoint{2.633910in}{2.620071in}}{\pgfqpoint{2.641724in}{2.627885in}}%
\pgfpathcurveto{\pgfqpoint{2.649537in}{2.635699in}}{\pgfqpoint{2.653928in}{2.646298in}}{\pgfqpoint{2.653928in}{2.657348in}}%
\pgfpathcurveto{\pgfqpoint{2.653928in}{2.668398in}}{\pgfqpoint{2.649537in}{2.678997in}}{\pgfqpoint{2.641724in}{2.686811in}}%
\pgfpathcurveto{\pgfqpoint{2.633910in}{2.694624in}}{\pgfqpoint{2.623311in}{2.699015in}}{\pgfqpoint{2.612261in}{2.699015in}}%
\pgfpathcurveto{\pgfqpoint{2.601211in}{2.699015in}}{\pgfqpoint{2.590612in}{2.694624in}}{\pgfqpoint{2.582798in}{2.686811in}}%
\pgfpathcurveto{\pgfqpoint{2.574984in}{2.678997in}}{\pgfqpoint{2.570594in}{2.668398in}}{\pgfqpoint{2.570594in}{2.657348in}}%
\pgfpathcurveto{\pgfqpoint{2.570594in}{2.646298in}}{\pgfqpoint{2.574984in}{2.635699in}}{\pgfqpoint{2.582798in}{2.627885in}}%
\pgfpathcurveto{\pgfqpoint{2.590612in}{2.620071in}}{\pgfqpoint{2.601211in}{2.615681in}}{\pgfqpoint{2.612261in}{2.615681in}}%
\pgfpathclose%
\pgfusepath{stroke,fill}%
\end{pgfscope}%
\begin{pgfscope}%
\pgfpathrectangle{\pgfqpoint{0.600000in}{0.600000in}}{\pgfqpoint{3.900000in}{3.900000in}}%
\pgfusepath{clip}%
\pgfsetbuttcap%
\pgfsetroundjoin%
\definecolor{currentfill}{rgb}{0.121569,0.466667,0.705882}%
\pgfsetfillcolor{currentfill}%
\pgfsetlinewidth{1.003750pt}%
\definecolor{currentstroke}{rgb}{0.121569,0.466667,0.705882}%
\pgfsetstrokecolor{currentstroke}%
\pgfsetdash{}{0pt}%
\pgfpathmoveto{\pgfqpoint{2.033180in}{1.707666in}}%
\pgfpathcurveto{\pgfqpoint{2.044230in}{1.707666in}}{\pgfqpoint{2.054829in}{1.712056in}}{\pgfqpoint{2.062643in}{1.719870in}}%
\pgfpathcurveto{\pgfqpoint{2.070456in}{1.727684in}}{\pgfqpoint{2.074846in}{1.738283in}}{\pgfqpoint{2.074846in}{1.749333in}}%
\pgfpathcurveto{\pgfqpoint{2.074846in}{1.760383in}}{\pgfqpoint{2.070456in}{1.770982in}}{\pgfqpoint{2.062643in}{1.778796in}}%
\pgfpathcurveto{\pgfqpoint{2.054829in}{1.786609in}}{\pgfqpoint{2.044230in}{1.791000in}}{\pgfqpoint{2.033180in}{1.791000in}}%
\pgfpathcurveto{\pgfqpoint{2.022130in}{1.791000in}}{\pgfqpoint{2.011531in}{1.786609in}}{\pgfqpoint{2.003717in}{1.778796in}}%
\pgfpathcurveto{\pgfqpoint{1.995903in}{1.770982in}}{\pgfqpoint{1.991513in}{1.760383in}}{\pgfqpoint{1.991513in}{1.749333in}}%
\pgfpathcurveto{\pgfqpoint{1.991513in}{1.738283in}}{\pgfqpoint{1.995903in}{1.727684in}}{\pgfqpoint{2.003717in}{1.719870in}}%
\pgfpathcurveto{\pgfqpoint{2.011531in}{1.712056in}}{\pgfqpoint{2.022130in}{1.707666in}}{\pgfqpoint{2.033180in}{1.707666in}}%
\pgfpathclose%
\pgfusepath{stroke,fill}%
\end{pgfscope}%
\begin{pgfscope}%
\pgfpathrectangle{\pgfqpoint{0.600000in}{0.600000in}}{\pgfqpoint{3.900000in}{3.900000in}}%
\pgfusepath{clip}%
\pgfsetbuttcap%
\pgfsetroundjoin%
\definecolor{currentfill}{rgb}{0.121569,0.466667,0.705882}%
\pgfsetfillcolor{currentfill}%
\pgfsetlinewidth{1.003750pt}%
\definecolor{currentstroke}{rgb}{0.121569,0.466667,0.705882}%
\pgfsetstrokecolor{currentstroke}%
\pgfsetdash{}{0pt}%
\pgfpathmoveto{\pgfqpoint{2.880687in}{2.192538in}}%
\pgfpathcurveto{\pgfqpoint{2.891737in}{2.192538in}}{\pgfqpoint{2.902336in}{2.196928in}}{\pgfqpoint{2.910150in}{2.204742in}}%
\pgfpathcurveto{\pgfqpoint{2.917964in}{2.212556in}}{\pgfqpoint{2.922354in}{2.223155in}}{\pgfqpoint{2.922354in}{2.234205in}}%
\pgfpathcurveto{\pgfqpoint{2.922354in}{2.245255in}}{\pgfqpoint{2.917964in}{2.255854in}}{\pgfqpoint{2.910150in}{2.263667in}}%
\pgfpathcurveto{\pgfqpoint{2.902336in}{2.271481in}}{\pgfqpoint{2.891737in}{2.275871in}}{\pgfqpoint{2.880687in}{2.275871in}}%
\pgfpathcurveto{\pgfqpoint{2.869637in}{2.275871in}}{\pgfqpoint{2.859038in}{2.271481in}}{\pgfqpoint{2.851224in}{2.263667in}}%
\pgfpathcurveto{\pgfqpoint{2.843411in}{2.255854in}}{\pgfqpoint{2.839021in}{2.245255in}}{\pgfqpoint{2.839021in}{2.234205in}}%
\pgfpathcurveto{\pgfqpoint{2.839021in}{2.223155in}}{\pgfqpoint{2.843411in}{2.212556in}}{\pgfqpoint{2.851224in}{2.204742in}}%
\pgfpathcurveto{\pgfqpoint{2.859038in}{2.196928in}}{\pgfqpoint{2.869637in}{2.192538in}}{\pgfqpoint{2.880687in}{2.192538in}}%
\pgfpathclose%
\pgfusepath{stroke,fill}%
\end{pgfscope}%
\begin{pgfscope}%
\pgfpathrectangle{\pgfqpoint{0.600000in}{0.600000in}}{\pgfqpoint{3.900000in}{3.900000in}}%
\pgfusepath{clip}%
\pgfsetbuttcap%
\pgfsetroundjoin%
\definecolor{currentfill}{rgb}{0.121569,0.466667,0.705882}%
\pgfsetfillcolor{currentfill}%
\pgfsetlinewidth{1.003750pt}%
\definecolor{currentstroke}{rgb}{0.121569,0.466667,0.705882}%
\pgfsetstrokecolor{currentstroke}%
\pgfsetdash{}{0pt}%
\pgfpathmoveto{\pgfqpoint{2.468614in}{1.733225in}}%
\pgfpathcurveto{\pgfqpoint{2.479665in}{1.733225in}}{\pgfqpoint{2.490264in}{1.737616in}}{\pgfqpoint{2.498077in}{1.745429in}}%
\pgfpathcurveto{\pgfqpoint{2.505891in}{1.753243in}}{\pgfqpoint{2.510281in}{1.763842in}}{\pgfqpoint{2.510281in}{1.774892in}}%
\pgfpathcurveto{\pgfqpoint{2.510281in}{1.785942in}}{\pgfqpoint{2.505891in}{1.796541in}}{\pgfqpoint{2.498077in}{1.804355in}}%
\pgfpathcurveto{\pgfqpoint{2.490264in}{1.812169in}}{\pgfqpoint{2.479665in}{1.816559in}}{\pgfqpoint{2.468614in}{1.816559in}}%
\pgfpathcurveto{\pgfqpoint{2.457564in}{1.816559in}}{\pgfqpoint{2.446965in}{1.812169in}}{\pgfqpoint{2.439152in}{1.804355in}}%
\pgfpathcurveto{\pgfqpoint{2.431338in}{1.796541in}}{\pgfqpoint{2.426948in}{1.785942in}}{\pgfqpoint{2.426948in}{1.774892in}}%
\pgfpathcurveto{\pgfqpoint{2.426948in}{1.763842in}}{\pgfqpoint{2.431338in}{1.753243in}}{\pgfqpoint{2.439152in}{1.745429in}}%
\pgfpathcurveto{\pgfqpoint{2.446965in}{1.737616in}}{\pgfqpoint{2.457564in}{1.733225in}}{\pgfqpoint{2.468614in}{1.733225in}}%
\pgfpathclose%
\pgfusepath{stroke,fill}%
\end{pgfscope}%
\begin{pgfscope}%
\pgfpathrectangle{\pgfqpoint{0.600000in}{0.600000in}}{\pgfqpoint{3.900000in}{3.900000in}}%
\pgfusepath{clip}%
\pgfsetbuttcap%
\pgfsetroundjoin%
\definecolor{currentfill}{rgb}{0.121569,0.466667,0.705882}%
\pgfsetfillcolor{currentfill}%
\pgfsetlinewidth{1.003750pt}%
\definecolor{currentstroke}{rgb}{0.121569,0.466667,0.705882}%
\pgfsetstrokecolor{currentstroke}%
\pgfsetdash{}{0pt}%
\pgfpathmoveto{\pgfqpoint{2.827309in}{3.392437in}}%
\pgfpathcurveto{\pgfqpoint{2.838359in}{3.392437in}}{\pgfqpoint{2.848958in}{3.396827in}}{\pgfqpoint{2.856772in}{3.404641in}}%
\pgfpathcurveto{\pgfqpoint{2.864585in}{3.412454in}}{\pgfqpoint{2.868976in}{3.423053in}}{\pgfqpoint{2.868976in}{3.434103in}}%
\pgfpathcurveto{\pgfqpoint{2.868976in}{3.445154in}}{\pgfqpoint{2.864585in}{3.455753in}}{\pgfqpoint{2.856772in}{3.463566in}}%
\pgfpathcurveto{\pgfqpoint{2.848958in}{3.471380in}}{\pgfqpoint{2.838359in}{3.475770in}}{\pgfqpoint{2.827309in}{3.475770in}}%
\pgfpathcurveto{\pgfqpoint{2.816259in}{3.475770in}}{\pgfqpoint{2.805660in}{3.471380in}}{\pgfqpoint{2.797846in}{3.463566in}}%
\pgfpathcurveto{\pgfqpoint{2.790033in}{3.455753in}}{\pgfqpoint{2.785642in}{3.445154in}}{\pgfqpoint{2.785642in}{3.434103in}}%
\pgfpathcurveto{\pgfqpoint{2.785642in}{3.423053in}}{\pgfqpoint{2.790033in}{3.412454in}}{\pgfqpoint{2.797846in}{3.404641in}}%
\pgfpathcurveto{\pgfqpoint{2.805660in}{3.396827in}}{\pgfqpoint{2.816259in}{3.392437in}}{\pgfqpoint{2.827309in}{3.392437in}}%
\pgfpathclose%
\pgfusepath{stroke,fill}%
\end{pgfscope}%
\begin{pgfscope}%
\pgfpathrectangle{\pgfqpoint{0.600000in}{0.600000in}}{\pgfqpoint{3.900000in}{3.900000in}}%
\pgfusepath{clip}%
\pgfsetbuttcap%
\pgfsetroundjoin%
\definecolor{currentfill}{rgb}{0.121569,0.466667,0.705882}%
\pgfsetfillcolor{currentfill}%
\pgfsetlinewidth{1.003750pt}%
\definecolor{currentstroke}{rgb}{0.121569,0.466667,0.705882}%
\pgfsetstrokecolor{currentstroke}%
\pgfsetdash{}{0pt}%
\pgfpathmoveto{\pgfqpoint{2.396509in}{2.663754in}}%
\pgfpathcurveto{\pgfqpoint{2.407559in}{2.663754in}}{\pgfqpoint{2.418158in}{2.668144in}}{\pgfqpoint{2.425972in}{2.675957in}}%
\pgfpathcurveto{\pgfqpoint{2.433785in}{2.683771in}}{\pgfqpoint{2.438176in}{2.694370in}}{\pgfqpoint{2.438176in}{2.705420in}}%
\pgfpathcurveto{\pgfqpoint{2.438176in}{2.716470in}}{\pgfqpoint{2.433785in}{2.727069in}}{\pgfqpoint{2.425972in}{2.734883in}}%
\pgfpathcurveto{\pgfqpoint{2.418158in}{2.742697in}}{\pgfqpoint{2.407559in}{2.747087in}}{\pgfqpoint{2.396509in}{2.747087in}}%
\pgfpathcurveto{\pgfqpoint{2.385459in}{2.747087in}}{\pgfqpoint{2.374860in}{2.742697in}}{\pgfqpoint{2.367046in}{2.734883in}}%
\pgfpathcurveto{\pgfqpoint{2.359233in}{2.727069in}}{\pgfqpoint{2.354842in}{2.716470in}}{\pgfqpoint{2.354842in}{2.705420in}}%
\pgfpathcurveto{\pgfqpoint{2.354842in}{2.694370in}}{\pgfqpoint{2.359233in}{2.683771in}}{\pgfqpoint{2.367046in}{2.675957in}}%
\pgfpathcurveto{\pgfqpoint{2.374860in}{2.668144in}}{\pgfqpoint{2.385459in}{2.663754in}}{\pgfqpoint{2.396509in}{2.663754in}}%
\pgfpathclose%
\pgfusepath{stroke,fill}%
\end{pgfscope}%
\begin{pgfscope}%
\pgfpathrectangle{\pgfqpoint{0.600000in}{0.600000in}}{\pgfqpoint{3.900000in}{3.900000in}}%
\pgfusepath{clip}%
\pgfsetbuttcap%
\pgfsetroundjoin%
\definecolor{currentfill}{rgb}{0.121569,0.466667,0.705882}%
\pgfsetfillcolor{currentfill}%
\pgfsetlinewidth{1.003750pt}%
\definecolor{currentstroke}{rgb}{0.121569,0.466667,0.705882}%
\pgfsetstrokecolor{currentstroke}%
\pgfsetdash{}{0pt}%
\pgfpathmoveto{\pgfqpoint{2.103356in}{2.932325in}}%
\pgfpathcurveto{\pgfqpoint{2.114406in}{2.932325in}}{\pgfqpoint{2.125005in}{2.936715in}}{\pgfqpoint{2.132819in}{2.944529in}}%
\pgfpathcurveto{\pgfqpoint{2.140632in}{2.952342in}}{\pgfqpoint{2.145022in}{2.962941in}}{\pgfqpoint{2.145022in}{2.973991in}}%
\pgfpathcurveto{\pgfqpoint{2.145022in}{2.985041in}}{\pgfqpoint{2.140632in}{2.995640in}}{\pgfqpoint{2.132819in}{3.003454in}}%
\pgfpathcurveto{\pgfqpoint{2.125005in}{3.011268in}}{\pgfqpoint{2.114406in}{3.015658in}}{\pgfqpoint{2.103356in}{3.015658in}}%
\pgfpathcurveto{\pgfqpoint{2.092306in}{3.015658in}}{\pgfqpoint{2.081707in}{3.011268in}}{\pgfqpoint{2.073893in}{3.003454in}}%
\pgfpathcurveto{\pgfqpoint{2.066079in}{2.995640in}}{\pgfqpoint{2.061689in}{2.985041in}}{\pgfqpoint{2.061689in}{2.973991in}}%
\pgfpathcurveto{\pgfqpoint{2.061689in}{2.962941in}}{\pgfqpoint{2.066079in}{2.952342in}}{\pgfqpoint{2.073893in}{2.944529in}}%
\pgfpathcurveto{\pgfqpoint{2.081707in}{2.936715in}}{\pgfqpoint{2.092306in}{2.932325in}}{\pgfqpoint{2.103356in}{2.932325in}}%
\pgfpathclose%
\pgfusepath{stroke,fill}%
\end{pgfscope}%
\begin{pgfscope}%
\pgfpathrectangle{\pgfqpoint{0.600000in}{0.600000in}}{\pgfqpoint{3.900000in}{3.900000in}}%
\pgfusepath{clip}%
\pgfsetbuttcap%
\pgfsetroundjoin%
\definecolor{currentfill}{rgb}{0.121569,0.466667,0.705882}%
\pgfsetfillcolor{currentfill}%
\pgfsetlinewidth{1.003750pt}%
\definecolor{currentstroke}{rgb}{0.121569,0.466667,0.705882}%
\pgfsetstrokecolor{currentstroke}%
\pgfsetdash{}{0pt}%
\pgfpathmoveto{\pgfqpoint{1.359951in}{3.868414in}}%
\pgfpathcurveto{\pgfqpoint{1.371002in}{3.868414in}}{\pgfqpoint{1.381601in}{3.872804in}}{\pgfqpoint{1.389414in}{3.880617in}}%
\pgfpathcurveto{\pgfqpoint{1.397228in}{3.888431in}}{\pgfqpoint{1.401618in}{3.899030in}}{\pgfqpoint{1.401618in}{3.910080in}}%
\pgfpathcurveto{\pgfqpoint{1.401618in}{3.921130in}}{\pgfqpoint{1.397228in}{3.931729in}}{\pgfqpoint{1.389414in}{3.939543in}}%
\pgfpathcurveto{\pgfqpoint{1.381601in}{3.947357in}}{\pgfqpoint{1.371002in}{3.951747in}}{\pgfqpoint{1.359951in}{3.951747in}}%
\pgfpathcurveto{\pgfqpoint{1.348901in}{3.951747in}}{\pgfqpoint{1.338302in}{3.947357in}}{\pgfqpoint{1.330489in}{3.939543in}}%
\pgfpathcurveto{\pgfqpoint{1.322675in}{3.931729in}}{\pgfqpoint{1.318285in}{3.921130in}}{\pgfqpoint{1.318285in}{3.910080in}}%
\pgfpathcurveto{\pgfqpoint{1.318285in}{3.899030in}}{\pgfqpoint{1.322675in}{3.888431in}}{\pgfqpoint{1.330489in}{3.880617in}}%
\pgfpathcurveto{\pgfqpoint{1.338302in}{3.872804in}}{\pgfqpoint{1.348901in}{3.868414in}}{\pgfqpoint{1.359951in}{3.868414in}}%
\pgfpathclose%
\pgfusepath{stroke,fill}%
\end{pgfscope}%
\begin{pgfscope}%
\pgfpathrectangle{\pgfqpoint{0.600000in}{0.600000in}}{\pgfqpoint{3.900000in}{3.900000in}}%
\pgfusepath{clip}%
\pgfsetbuttcap%
\pgfsetroundjoin%
\definecolor{currentfill}{rgb}{0.121569,0.466667,0.705882}%
\pgfsetfillcolor{currentfill}%
\pgfsetlinewidth{1.003750pt}%
\definecolor{currentstroke}{rgb}{0.121569,0.466667,0.705882}%
\pgfsetstrokecolor{currentstroke}%
\pgfsetdash{}{0pt}%
\pgfpathmoveto{\pgfqpoint{2.645577in}{2.999519in}}%
\pgfpathcurveto{\pgfqpoint{2.656627in}{2.999519in}}{\pgfqpoint{2.667226in}{3.003909in}}{\pgfqpoint{2.675040in}{3.011723in}}%
\pgfpathcurveto{\pgfqpoint{2.682853in}{3.019536in}}{\pgfqpoint{2.687244in}{3.030135in}}{\pgfqpoint{2.687244in}{3.041185in}}%
\pgfpathcurveto{\pgfqpoint{2.687244in}{3.052235in}}{\pgfqpoint{2.682853in}{3.062834in}}{\pgfqpoint{2.675040in}{3.070648in}}%
\pgfpathcurveto{\pgfqpoint{2.667226in}{3.078462in}}{\pgfqpoint{2.656627in}{3.082852in}}{\pgfqpoint{2.645577in}{3.082852in}}%
\pgfpathcurveto{\pgfqpoint{2.634527in}{3.082852in}}{\pgfqpoint{2.623928in}{3.078462in}}{\pgfqpoint{2.616114in}{3.070648in}}%
\pgfpathcurveto{\pgfqpoint{2.608301in}{3.062834in}}{\pgfqpoint{2.603910in}{3.052235in}}{\pgfqpoint{2.603910in}{3.041185in}}%
\pgfpathcurveto{\pgfqpoint{2.603910in}{3.030135in}}{\pgfqpoint{2.608301in}{3.019536in}}{\pgfqpoint{2.616114in}{3.011723in}}%
\pgfpathcurveto{\pgfqpoint{2.623928in}{3.003909in}}{\pgfqpoint{2.634527in}{2.999519in}}{\pgfqpoint{2.645577in}{2.999519in}}%
\pgfpathclose%
\pgfusepath{stroke,fill}%
\end{pgfscope}%
\begin{pgfscope}%
\pgfpathrectangle{\pgfqpoint{0.600000in}{0.600000in}}{\pgfqpoint{3.900000in}{3.900000in}}%
\pgfusepath{clip}%
\pgfsetbuttcap%
\pgfsetroundjoin%
\definecolor{currentfill}{rgb}{0.121569,0.466667,0.705882}%
\pgfsetfillcolor{currentfill}%
\pgfsetlinewidth{1.003750pt}%
\definecolor{currentstroke}{rgb}{0.121569,0.466667,0.705882}%
\pgfsetstrokecolor{currentstroke}%
\pgfsetdash{}{0pt}%
\pgfpathmoveto{\pgfqpoint{2.957575in}{2.954597in}}%
\pgfpathcurveto{\pgfqpoint{2.968626in}{2.954597in}}{\pgfqpoint{2.979225in}{2.958987in}}{\pgfqpoint{2.987038in}{2.966801in}}%
\pgfpathcurveto{\pgfqpoint{2.994852in}{2.974614in}}{\pgfqpoint{2.999242in}{2.985213in}}{\pgfqpoint{2.999242in}{2.996263in}}%
\pgfpathcurveto{\pgfqpoint{2.999242in}{3.007313in}}{\pgfqpoint{2.994852in}{3.017912in}}{\pgfqpoint{2.987038in}{3.025726in}}%
\pgfpathcurveto{\pgfqpoint{2.979225in}{3.033540in}}{\pgfqpoint{2.968626in}{3.037930in}}{\pgfqpoint{2.957575in}{3.037930in}}%
\pgfpathcurveto{\pgfqpoint{2.946525in}{3.037930in}}{\pgfqpoint{2.935926in}{3.033540in}}{\pgfqpoint{2.928113in}{3.025726in}}%
\pgfpathcurveto{\pgfqpoint{2.920299in}{3.017912in}}{\pgfqpoint{2.915909in}{3.007313in}}{\pgfqpoint{2.915909in}{2.996263in}}%
\pgfpathcurveto{\pgfqpoint{2.915909in}{2.985213in}}{\pgfqpoint{2.920299in}{2.974614in}}{\pgfqpoint{2.928113in}{2.966801in}}%
\pgfpathcurveto{\pgfqpoint{2.935926in}{2.958987in}}{\pgfqpoint{2.946525in}{2.954597in}}{\pgfqpoint{2.957575in}{2.954597in}}%
\pgfpathclose%
\pgfusepath{stroke,fill}%
\end{pgfscope}%
\begin{pgfscope}%
\pgfpathrectangle{\pgfqpoint{0.600000in}{0.600000in}}{\pgfqpoint{3.900000in}{3.900000in}}%
\pgfusepath{clip}%
\pgfsetbuttcap%
\pgfsetroundjoin%
\definecolor{currentfill}{rgb}{0.121569,0.466667,0.705882}%
\pgfsetfillcolor{currentfill}%
\pgfsetlinewidth{1.003750pt}%
\definecolor{currentstroke}{rgb}{0.121569,0.466667,0.705882}%
\pgfsetstrokecolor{currentstroke}%
\pgfsetdash{}{0pt}%
\pgfpathmoveto{\pgfqpoint{2.925760in}{2.485238in}}%
\pgfpathcurveto{\pgfqpoint{2.936810in}{2.485238in}}{\pgfqpoint{2.947409in}{2.489628in}}{\pgfqpoint{2.955223in}{2.497442in}}%
\pgfpathcurveto{\pgfqpoint{2.963036in}{2.505255in}}{\pgfqpoint{2.967427in}{2.515854in}}{\pgfqpoint{2.967427in}{2.526905in}}%
\pgfpathcurveto{\pgfqpoint{2.967427in}{2.537955in}}{\pgfqpoint{2.963036in}{2.548554in}}{\pgfqpoint{2.955223in}{2.556367in}}%
\pgfpathcurveto{\pgfqpoint{2.947409in}{2.564181in}}{\pgfqpoint{2.936810in}{2.568571in}}{\pgfqpoint{2.925760in}{2.568571in}}%
\pgfpathcurveto{\pgfqpoint{2.914710in}{2.568571in}}{\pgfqpoint{2.904111in}{2.564181in}}{\pgfqpoint{2.896297in}{2.556367in}}%
\pgfpathcurveto{\pgfqpoint{2.888484in}{2.548554in}}{\pgfqpoint{2.884093in}{2.537955in}}{\pgfqpoint{2.884093in}{2.526905in}}%
\pgfpathcurveto{\pgfqpoint{2.884093in}{2.515854in}}{\pgfqpoint{2.888484in}{2.505255in}}{\pgfqpoint{2.896297in}{2.497442in}}%
\pgfpathcurveto{\pgfqpoint{2.904111in}{2.489628in}}{\pgfqpoint{2.914710in}{2.485238in}}{\pgfqpoint{2.925760in}{2.485238in}}%
\pgfpathclose%
\pgfusepath{stroke,fill}%
\end{pgfscope}%
\begin{pgfscope}%
\pgfpathrectangle{\pgfqpoint{0.600000in}{0.600000in}}{\pgfqpoint{3.900000in}{3.900000in}}%
\pgfusepath{clip}%
\pgfsetbuttcap%
\pgfsetroundjoin%
\definecolor{currentfill}{rgb}{0.121569,0.466667,0.705882}%
\pgfsetfillcolor{currentfill}%
\pgfsetlinewidth{1.003750pt}%
\definecolor{currentstroke}{rgb}{0.121569,0.466667,0.705882}%
\pgfsetstrokecolor{currentstroke}%
\pgfsetdash{}{0pt}%
\pgfpathmoveto{\pgfqpoint{2.197381in}{3.037494in}}%
\pgfpathcurveto{\pgfqpoint{2.208431in}{3.037494in}}{\pgfqpoint{2.219030in}{3.041884in}}{\pgfqpoint{2.226843in}{3.049697in}}%
\pgfpathcurveto{\pgfqpoint{2.234657in}{3.057511in}}{\pgfqpoint{2.239047in}{3.068110in}}{\pgfqpoint{2.239047in}{3.079160in}}%
\pgfpathcurveto{\pgfqpoint{2.239047in}{3.090210in}}{\pgfqpoint{2.234657in}{3.100809in}}{\pgfqpoint{2.226843in}{3.108623in}}%
\pgfpathcurveto{\pgfqpoint{2.219030in}{3.116437in}}{\pgfqpoint{2.208431in}{3.120827in}}{\pgfqpoint{2.197381in}{3.120827in}}%
\pgfpathcurveto{\pgfqpoint{2.186330in}{3.120827in}}{\pgfqpoint{2.175731in}{3.116437in}}{\pgfqpoint{2.167918in}{3.108623in}}%
\pgfpathcurveto{\pgfqpoint{2.160104in}{3.100809in}}{\pgfqpoint{2.155714in}{3.090210in}}{\pgfqpoint{2.155714in}{3.079160in}}%
\pgfpathcurveto{\pgfqpoint{2.155714in}{3.068110in}}{\pgfqpoint{2.160104in}{3.057511in}}{\pgfqpoint{2.167918in}{3.049697in}}%
\pgfpathcurveto{\pgfqpoint{2.175731in}{3.041884in}}{\pgfqpoint{2.186330in}{3.037494in}}{\pgfqpoint{2.197381in}{3.037494in}}%
\pgfpathclose%
\pgfusepath{stroke,fill}%
\end{pgfscope}%
\begin{pgfscope}%
\pgfpathrectangle{\pgfqpoint{0.600000in}{0.600000in}}{\pgfqpoint{3.900000in}{3.900000in}}%
\pgfusepath{clip}%
\pgfsetbuttcap%
\pgfsetroundjoin%
\definecolor{currentfill}{rgb}{0.121569,0.466667,0.705882}%
\pgfsetfillcolor{currentfill}%
\pgfsetlinewidth{1.003750pt}%
\definecolor{currentstroke}{rgb}{0.121569,0.466667,0.705882}%
\pgfsetstrokecolor{currentstroke}%
\pgfsetdash{}{0pt}%
\pgfpathmoveto{\pgfqpoint{2.790913in}{2.449806in}}%
\pgfpathcurveto{\pgfqpoint{2.801963in}{2.449806in}}{\pgfqpoint{2.812562in}{2.454197in}}{\pgfqpoint{2.820376in}{2.462010in}}%
\pgfpathcurveto{\pgfqpoint{2.828190in}{2.469824in}}{\pgfqpoint{2.832580in}{2.480423in}}{\pgfqpoint{2.832580in}{2.491473in}}%
\pgfpathcurveto{\pgfqpoint{2.832580in}{2.502523in}}{\pgfqpoint{2.828190in}{2.513122in}}{\pgfqpoint{2.820376in}{2.520936in}}%
\pgfpathcurveto{\pgfqpoint{2.812562in}{2.528750in}}{\pgfqpoint{2.801963in}{2.533140in}}{\pgfqpoint{2.790913in}{2.533140in}}%
\pgfpathcurveto{\pgfqpoint{2.779863in}{2.533140in}}{\pgfqpoint{2.769264in}{2.528750in}}{\pgfqpoint{2.761450in}{2.520936in}}%
\pgfpathcurveto{\pgfqpoint{2.753637in}{2.513122in}}{\pgfqpoint{2.749247in}{2.502523in}}{\pgfqpoint{2.749247in}{2.491473in}}%
\pgfpathcurveto{\pgfqpoint{2.749247in}{2.480423in}}{\pgfqpoint{2.753637in}{2.469824in}}{\pgfqpoint{2.761450in}{2.462010in}}%
\pgfpathcurveto{\pgfqpoint{2.769264in}{2.454197in}}{\pgfqpoint{2.779863in}{2.449806in}}{\pgfqpoint{2.790913in}{2.449806in}}%
\pgfpathclose%
\pgfusepath{stroke,fill}%
\end{pgfscope}%
\begin{pgfscope}%
\pgfpathrectangle{\pgfqpoint{0.600000in}{0.600000in}}{\pgfqpoint{3.900000in}{3.900000in}}%
\pgfusepath{clip}%
\pgfsetbuttcap%
\pgfsetroundjoin%
\definecolor{currentfill}{rgb}{0.121569,0.466667,0.705882}%
\pgfsetfillcolor{currentfill}%
\pgfsetlinewidth{1.003750pt}%
\definecolor{currentstroke}{rgb}{0.121569,0.466667,0.705882}%
\pgfsetstrokecolor{currentstroke}%
\pgfsetdash{}{0pt}%
\pgfpathmoveto{\pgfqpoint{2.458211in}{3.519869in}}%
\pgfpathcurveto{\pgfqpoint{2.469261in}{3.519869in}}{\pgfqpoint{2.479860in}{3.524259in}}{\pgfqpoint{2.487673in}{3.532073in}}%
\pgfpathcurveto{\pgfqpoint{2.495487in}{3.539887in}}{\pgfqpoint{2.499877in}{3.550486in}}{\pgfqpoint{2.499877in}{3.561536in}}%
\pgfpathcurveto{\pgfqpoint{2.499877in}{3.572586in}}{\pgfqpoint{2.495487in}{3.583185in}}{\pgfqpoint{2.487673in}{3.590999in}}%
\pgfpathcurveto{\pgfqpoint{2.479860in}{3.598812in}}{\pgfqpoint{2.469261in}{3.603203in}}{\pgfqpoint{2.458211in}{3.603203in}}%
\pgfpathcurveto{\pgfqpoint{2.447160in}{3.603203in}}{\pgfqpoint{2.436561in}{3.598812in}}{\pgfqpoint{2.428748in}{3.590999in}}%
\pgfpathcurveto{\pgfqpoint{2.420934in}{3.583185in}}{\pgfqpoint{2.416544in}{3.572586in}}{\pgfqpoint{2.416544in}{3.561536in}}%
\pgfpathcurveto{\pgfqpoint{2.416544in}{3.550486in}}{\pgfqpoint{2.420934in}{3.539887in}}{\pgfqpoint{2.428748in}{3.532073in}}%
\pgfpathcurveto{\pgfqpoint{2.436561in}{3.524259in}}{\pgfqpoint{2.447160in}{3.519869in}}{\pgfqpoint{2.458211in}{3.519869in}}%
\pgfpathclose%
\pgfusepath{stroke,fill}%
\end{pgfscope}%
\begin{pgfscope}%
\pgfpathrectangle{\pgfqpoint{0.600000in}{0.600000in}}{\pgfqpoint{3.900000in}{3.900000in}}%
\pgfusepath{clip}%
\pgfsetbuttcap%
\pgfsetroundjoin%
\definecolor{currentfill}{rgb}{0.121569,0.466667,0.705882}%
\pgfsetfillcolor{currentfill}%
\pgfsetlinewidth{1.003750pt}%
\definecolor{currentstroke}{rgb}{0.121569,0.466667,0.705882}%
\pgfsetstrokecolor{currentstroke}%
\pgfsetdash{}{0pt}%
\pgfpathmoveto{\pgfqpoint{2.499709in}{2.311574in}}%
\pgfpathcurveto{\pgfqpoint{2.510759in}{2.311574in}}{\pgfqpoint{2.521358in}{2.315965in}}{\pgfqpoint{2.529172in}{2.323778in}}%
\pgfpathcurveto{\pgfqpoint{2.536986in}{2.331592in}}{\pgfqpoint{2.541376in}{2.342191in}}{\pgfqpoint{2.541376in}{2.353241in}}%
\pgfpathcurveto{\pgfqpoint{2.541376in}{2.364291in}}{\pgfqpoint{2.536986in}{2.374890in}}{\pgfqpoint{2.529172in}{2.382704in}}%
\pgfpathcurveto{\pgfqpoint{2.521358in}{2.390517in}}{\pgfqpoint{2.510759in}{2.394908in}}{\pgfqpoint{2.499709in}{2.394908in}}%
\pgfpathcurveto{\pgfqpoint{2.488659in}{2.394908in}}{\pgfqpoint{2.478060in}{2.390517in}}{\pgfqpoint{2.470246in}{2.382704in}}%
\pgfpathcurveto{\pgfqpoint{2.462433in}{2.374890in}}{\pgfqpoint{2.458042in}{2.364291in}}{\pgfqpoint{2.458042in}{2.353241in}}%
\pgfpathcurveto{\pgfqpoint{2.458042in}{2.342191in}}{\pgfqpoint{2.462433in}{2.331592in}}{\pgfqpoint{2.470246in}{2.323778in}}%
\pgfpathcurveto{\pgfqpoint{2.478060in}{2.315965in}}{\pgfqpoint{2.488659in}{2.311574in}}{\pgfqpoint{2.499709in}{2.311574in}}%
\pgfpathclose%
\pgfusepath{stroke,fill}%
\end{pgfscope}%
\begin{pgfscope}%
\pgfpathrectangle{\pgfqpoint{0.600000in}{0.600000in}}{\pgfqpoint{3.900000in}{3.900000in}}%
\pgfusepath{clip}%
\pgfsetbuttcap%
\pgfsetroundjoin%
\definecolor{currentfill}{rgb}{0.121569,0.466667,0.705882}%
\pgfsetfillcolor{currentfill}%
\pgfsetlinewidth{1.003750pt}%
\definecolor{currentstroke}{rgb}{0.121569,0.466667,0.705882}%
\pgfsetstrokecolor{currentstroke}%
\pgfsetdash{}{0pt}%
\pgfpathmoveto{\pgfqpoint{1.975856in}{2.694493in}}%
\pgfpathcurveto{\pgfqpoint{1.986906in}{2.694493in}}{\pgfqpoint{1.997505in}{2.698883in}}{\pgfqpoint{2.005318in}{2.706697in}}%
\pgfpathcurveto{\pgfqpoint{2.013132in}{2.714510in}}{\pgfqpoint{2.017522in}{2.725109in}}{\pgfqpoint{2.017522in}{2.736159in}}%
\pgfpathcurveto{\pgfqpoint{2.017522in}{2.747209in}}{\pgfqpoint{2.013132in}{2.757809in}}{\pgfqpoint{2.005318in}{2.765622in}}%
\pgfpathcurveto{\pgfqpoint{1.997505in}{2.773436in}}{\pgfqpoint{1.986906in}{2.777826in}}{\pgfqpoint{1.975856in}{2.777826in}}%
\pgfpathcurveto{\pgfqpoint{1.964805in}{2.777826in}}{\pgfqpoint{1.954206in}{2.773436in}}{\pgfqpoint{1.946393in}{2.765622in}}%
\pgfpathcurveto{\pgfqpoint{1.938579in}{2.757809in}}{\pgfqpoint{1.934189in}{2.747209in}}{\pgfqpoint{1.934189in}{2.736159in}}%
\pgfpathcurveto{\pgfqpoint{1.934189in}{2.725109in}}{\pgfqpoint{1.938579in}{2.714510in}}{\pgfqpoint{1.946393in}{2.706697in}}%
\pgfpathcurveto{\pgfqpoint{1.954206in}{2.698883in}}{\pgfqpoint{1.964805in}{2.694493in}}{\pgfqpoint{1.975856in}{2.694493in}}%
\pgfpathclose%
\pgfusepath{stroke,fill}%
\end{pgfscope}%
\begin{pgfscope}%
\pgfpathrectangle{\pgfqpoint{0.600000in}{0.600000in}}{\pgfqpoint{3.900000in}{3.900000in}}%
\pgfusepath{clip}%
\pgfsetbuttcap%
\pgfsetroundjoin%
\definecolor{currentfill}{rgb}{0.121569,0.466667,0.705882}%
\pgfsetfillcolor{currentfill}%
\pgfsetlinewidth{1.003750pt}%
\definecolor{currentstroke}{rgb}{0.121569,0.466667,0.705882}%
\pgfsetstrokecolor{currentstroke}%
\pgfsetdash{}{0pt}%
\pgfpathmoveto{\pgfqpoint{2.829603in}{3.347742in}}%
\pgfpathcurveto{\pgfqpoint{2.840653in}{3.347742in}}{\pgfqpoint{2.851252in}{3.352132in}}{\pgfqpoint{2.859066in}{3.359946in}}%
\pgfpathcurveto{\pgfqpoint{2.866879in}{3.367760in}}{\pgfqpoint{2.871269in}{3.378359in}}{\pgfqpoint{2.871269in}{3.389409in}}%
\pgfpathcurveto{\pgfqpoint{2.871269in}{3.400459in}}{\pgfqpoint{2.866879in}{3.411058in}}{\pgfqpoint{2.859066in}{3.418872in}}%
\pgfpathcurveto{\pgfqpoint{2.851252in}{3.426685in}}{\pgfqpoint{2.840653in}{3.431075in}}{\pgfqpoint{2.829603in}{3.431075in}}%
\pgfpathcurveto{\pgfqpoint{2.818553in}{3.431075in}}{\pgfqpoint{2.807954in}{3.426685in}}{\pgfqpoint{2.800140in}{3.418872in}}%
\pgfpathcurveto{\pgfqpoint{2.792326in}{3.411058in}}{\pgfqpoint{2.787936in}{3.400459in}}{\pgfqpoint{2.787936in}{3.389409in}}%
\pgfpathcurveto{\pgfqpoint{2.787936in}{3.378359in}}{\pgfqpoint{2.792326in}{3.367760in}}{\pgfqpoint{2.800140in}{3.359946in}}%
\pgfpathcurveto{\pgfqpoint{2.807954in}{3.352132in}}{\pgfqpoint{2.818553in}{3.347742in}}{\pgfqpoint{2.829603in}{3.347742in}}%
\pgfpathclose%
\pgfusepath{stroke,fill}%
\end{pgfscope}%
\begin{pgfscope}%
\pgfpathrectangle{\pgfqpoint{0.600000in}{0.600000in}}{\pgfqpoint{3.900000in}{3.900000in}}%
\pgfusepath{clip}%
\pgfsetbuttcap%
\pgfsetroundjoin%
\definecolor{currentfill}{rgb}{0.121569,0.466667,0.705882}%
\pgfsetfillcolor{currentfill}%
\pgfsetlinewidth{1.003750pt}%
\definecolor{currentstroke}{rgb}{0.121569,0.466667,0.705882}%
\pgfsetstrokecolor{currentstroke}%
\pgfsetdash{}{0pt}%
\pgfpathmoveto{\pgfqpoint{3.049479in}{2.624754in}}%
\pgfpathcurveto{\pgfqpoint{3.060529in}{2.624754in}}{\pgfqpoint{3.071128in}{2.629145in}}{\pgfqpoint{3.078942in}{2.636958in}}%
\pgfpathcurveto{\pgfqpoint{3.086755in}{2.644772in}}{\pgfqpoint{3.091146in}{2.655371in}}{\pgfqpoint{3.091146in}{2.666421in}}%
\pgfpathcurveto{\pgfqpoint{3.091146in}{2.677471in}}{\pgfqpoint{3.086755in}{2.688070in}}{\pgfqpoint{3.078942in}{2.695884in}}%
\pgfpathcurveto{\pgfqpoint{3.071128in}{2.703697in}}{\pgfqpoint{3.060529in}{2.708088in}}{\pgfqpoint{3.049479in}{2.708088in}}%
\pgfpathcurveto{\pgfqpoint{3.038429in}{2.708088in}}{\pgfqpoint{3.027830in}{2.703697in}}{\pgfqpoint{3.020016in}{2.695884in}}%
\pgfpathcurveto{\pgfqpoint{3.012202in}{2.688070in}}{\pgfqpoint{3.007812in}{2.677471in}}{\pgfqpoint{3.007812in}{2.666421in}}%
\pgfpathcurveto{\pgfqpoint{3.007812in}{2.655371in}}{\pgfqpoint{3.012202in}{2.644772in}}{\pgfqpoint{3.020016in}{2.636958in}}%
\pgfpathcurveto{\pgfqpoint{3.027830in}{2.629145in}}{\pgfqpoint{3.038429in}{2.624754in}}{\pgfqpoint{3.049479in}{2.624754in}}%
\pgfpathclose%
\pgfusepath{stroke,fill}%
\end{pgfscope}%
\begin{pgfscope}%
\pgfpathrectangle{\pgfqpoint{0.600000in}{0.600000in}}{\pgfqpoint{3.900000in}{3.900000in}}%
\pgfusepath{clip}%
\pgfsetbuttcap%
\pgfsetroundjoin%
\definecolor{currentfill}{rgb}{0.121569,0.466667,0.705882}%
\pgfsetfillcolor{currentfill}%
\pgfsetlinewidth{1.003750pt}%
\definecolor{currentstroke}{rgb}{0.121569,0.466667,0.705882}%
\pgfsetstrokecolor{currentstroke}%
\pgfsetdash{}{0pt}%
\pgfpathmoveto{\pgfqpoint{2.423311in}{3.203253in}}%
\pgfpathcurveto{\pgfqpoint{2.434361in}{3.203253in}}{\pgfqpoint{2.444960in}{3.207644in}}{\pgfqpoint{2.452774in}{3.215457in}}%
\pgfpathcurveto{\pgfqpoint{2.460587in}{3.223271in}}{\pgfqpoint{2.464978in}{3.233870in}}{\pgfqpoint{2.464978in}{3.244920in}}%
\pgfpathcurveto{\pgfqpoint{2.464978in}{3.255970in}}{\pgfqpoint{2.460587in}{3.266569in}}{\pgfqpoint{2.452774in}{3.274383in}}%
\pgfpathcurveto{\pgfqpoint{2.444960in}{3.282196in}}{\pgfqpoint{2.434361in}{3.286587in}}{\pgfqpoint{2.423311in}{3.286587in}}%
\pgfpathcurveto{\pgfqpoint{2.412261in}{3.286587in}}{\pgfqpoint{2.401662in}{3.282196in}}{\pgfqpoint{2.393848in}{3.274383in}}%
\pgfpathcurveto{\pgfqpoint{2.386034in}{3.266569in}}{\pgfqpoint{2.381644in}{3.255970in}}{\pgfqpoint{2.381644in}{3.244920in}}%
\pgfpathcurveto{\pgfqpoint{2.381644in}{3.233870in}}{\pgfqpoint{2.386034in}{3.223271in}}{\pgfqpoint{2.393848in}{3.215457in}}%
\pgfpathcurveto{\pgfqpoint{2.401662in}{3.207644in}}{\pgfqpoint{2.412261in}{3.203253in}}{\pgfqpoint{2.423311in}{3.203253in}}%
\pgfpathclose%
\pgfusepath{stroke,fill}%
\end{pgfscope}%
\begin{pgfscope}%
\pgfpathrectangle{\pgfqpoint{0.600000in}{0.600000in}}{\pgfqpoint{3.900000in}{3.900000in}}%
\pgfusepath{clip}%
\pgfsetbuttcap%
\pgfsetroundjoin%
\definecolor{currentfill}{rgb}{0.121569,0.466667,0.705882}%
\pgfsetfillcolor{currentfill}%
\pgfsetlinewidth{1.003750pt}%
\definecolor{currentstroke}{rgb}{0.121569,0.466667,0.705882}%
\pgfsetstrokecolor{currentstroke}%
\pgfsetdash{}{0pt}%
\pgfpathmoveto{\pgfqpoint{2.469201in}{2.546223in}}%
\pgfpathcurveto{\pgfqpoint{2.480251in}{2.546223in}}{\pgfqpoint{2.490850in}{2.550613in}}{\pgfqpoint{2.498663in}{2.558427in}}%
\pgfpathcurveto{\pgfqpoint{2.506477in}{2.566241in}}{\pgfqpoint{2.510867in}{2.576840in}}{\pgfqpoint{2.510867in}{2.587890in}}%
\pgfpathcurveto{\pgfqpoint{2.510867in}{2.598940in}}{\pgfqpoint{2.506477in}{2.609539in}}{\pgfqpoint{2.498663in}{2.617353in}}%
\pgfpathcurveto{\pgfqpoint{2.490850in}{2.625166in}}{\pgfqpoint{2.480251in}{2.629556in}}{\pgfqpoint{2.469201in}{2.629556in}}%
\pgfpathcurveto{\pgfqpoint{2.458151in}{2.629556in}}{\pgfqpoint{2.447552in}{2.625166in}}{\pgfqpoint{2.439738in}{2.617353in}}%
\pgfpathcurveto{\pgfqpoint{2.431924in}{2.609539in}}{\pgfqpoint{2.427534in}{2.598940in}}{\pgfqpoint{2.427534in}{2.587890in}}%
\pgfpathcurveto{\pgfqpoint{2.427534in}{2.576840in}}{\pgfqpoint{2.431924in}{2.566241in}}{\pgfqpoint{2.439738in}{2.558427in}}%
\pgfpathcurveto{\pgfqpoint{2.447552in}{2.550613in}}{\pgfqpoint{2.458151in}{2.546223in}}{\pgfqpoint{2.469201in}{2.546223in}}%
\pgfpathclose%
\pgfusepath{stroke,fill}%
\end{pgfscope}%
\begin{pgfscope}%
\pgfpathrectangle{\pgfqpoint{0.600000in}{0.600000in}}{\pgfqpoint{3.900000in}{3.900000in}}%
\pgfusepath{clip}%
\pgfsetbuttcap%
\pgfsetroundjoin%
\definecolor{currentfill}{rgb}{0.121569,0.466667,0.705882}%
\pgfsetfillcolor{currentfill}%
\pgfsetlinewidth{1.003750pt}%
\definecolor{currentstroke}{rgb}{0.121569,0.466667,0.705882}%
\pgfsetstrokecolor{currentstroke}%
\pgfsetdash{}{0pt}%
\pgfpathmoveto{\pgfqpoint{2.954975in}{2.685123in}}%
\pgfpathcurveto{\pgfqpoint{2.966025in}{2.685123in}}{\pgfqpoint{2.976624in}{2.689513in}}{\pgfqpoint{2.984437in}{2.697327in}}%
\pgfpathcurveto{\pgfqpoint{2.992251in}{2.705140in}}{\pgfqpoint{2.996641in}{2.715739in}}{\pgfqpoint{2.996641in}{2.726789in}}%
\pgfpathcurveto{\pgfqpoint{2.996641in}{2.737839in}}{\pgfqpoint{2.992251in}{2.748439in}}{\pgfqpoint{2.984437in}{2.756252in}}%
\pgfpathcurveto{\pgfqpoint{2.976624in}{2.764066in}}{\pgfqpoint{2.966025in}{2.768456in}}{\pgfqpoint{2.954975in}{2.768456in}}%
\pgfpathcurveto{\pgfqpoint{2.943925in}{2.768456in}}{\pgfqpoint{2.933325in}{2.764066in}}{\pgfqpoint{2.925512in}{2.756252in}}%
\pgfpathcurveto{\pgfqpoint{2.917698in}{2.748439in}}{\pgfqpoint{2.913308in}{2.737839in}}{\pgfqpoint{2.913308in}{2.726789in}}%
\pgfpathcurveto{\pgfqpoint{2.913308in}{2.715739in}}{\pgfqpoint{2.917698in}{2.705140in}}{\pgfqpoint{2.925512in}{2.697327in}}%
\pgfpathcurveto{\pgfqpoint{2.933325in}{2.689513in}}{\pgfqpoint{2.943925in}{2.685123in}}{\pgfqpoint{2.954975in}{2.685123in}}%
\pgfpathclose%
\pgfusepath{stroke,fill}%
\end{pgfscope}%
\begin{pgfscope}%
\pgfpathrectangle{\pgfqpoint{0.600000in}{0.600000in}}{\pgfqpoint{3.900000in}{3.900000in}}%
\pgfusepath{clip}%
\pgfsetbuttcap%
\pgfsetroundjoin%
\definecolor{currentfill}{rgb}{0.121569,0.466667,0.705882}%
\pgfsetfillcolor{currentfill}%
\pgfsetlinewidth{1.003750pt}%
\definecolor{currentstroke}{rgb}{0.121569,0.466667,0.705882}%
\pgfsetstrokecolor{currentstroke}%
\pgfsetdash{}{0pt}%
\pgfpathmoveto{\pgfqpoint{2.543162in}{2.897787in}}%
\pgfpathcurveto{\pgfqpoint{2.554212in}{2.897787in}}{\pgfqpoint{2.564811in}{2.902177in}}{\pgfqpoint{2.572625in}{2.909991in}}%
\pgfpathcurveto{\pgfqpoint{2.580438in}{2.917804in}}{\pgfqpoint{2.584829in}{2.928403in}}{\pgfqpoint{2.584829in}{2.939454in}}%
\pgfpathcurveto{\pgfqpoint{2.584829in}{2.950504in}}{\pgfqpoint{2.580438in}{2.961103in}}{\pgfqpoint{2.572625in}{2.968916in}}%
\pgfpathcurveto{\pgfqpoint{2.564811in}{2.976730in}}{\pgfqpoint{2.554212in}{2.981120in}}{\pgfqpoint{2.543162in}{2.981120in}}%
\pgfpathcurveto{\pgfqpoint{2.532112in}{2.981120in}}{\pgfqpoint{2.521513in}{2.976730in}}{\pgfqpoint{2.513699in}{2.968916in}}%
\pgfpathcurveto{\pgfqpoint{2.505886in}{2.961103in}}{\pgfqpoint{2.501495in}{2.950504in}}{\pgfqpoint{2.501495in}{2.939454in}}%
\pgfpathcurveto{\pgfqpoint{2.501495in}{2.928403in}}{\pgfqpoint{2.505886in}{2.917804in}}{\pgfqpoint{2.513699in}{2.909991in}}%
\pgfpathcurveto{\pgfqpoint{2.521513in}{2.902177in}}{\pgfqpoint{2.532112in}{2.897787in}}{\pgfqpoint{2.543162in}{2.897787in}}%
\pgfpathclose%
\pgfusepath{stroke,fill}%
\end{pgfscope}%
\begin{pgfscope}%
\pgfpathrectangle{\pgfqpoint{0.600000in}{0.600000in}}{\pgfqpoint{3.900000in}{3.900000in}}%
\pgfusepath{clip}%
\pgfsetbuttcap%
\pgfsetroundjoin%
\definecolor{currentfill}{rgb}{0.121569,0.466667,0.705882}%
\pgfsetfillcolor{currentfill}%
\pgfsetlinewidth{1.003750pt}%
\definecolor{currentstroke}{rgb}{0.121569,0.466667,0.705882}%
\pgfsetstrokecolor{currentstroke}%
\pgfsetdash{}{0pt}%
\pgfpathmoveto{\pgfqpoint{2.511432in}{2.119748in}}%
\pgfpathcurveto{\pgfqpoint{2.522482in}{2.119748in}}{\pgfqpoint{2.533081in}{2.124138in}}{\pgfqpoint{2.540895in}{2.131952in}}%
\pgfpathcurveto{\pgfqpoint{2.548709in}{2.139765in}}{\pgfqpoint{2.553099in}{2.150364in}}{\pgfqpoint{2.553099in}{2.161414in}}%
\pgfpathcurveto{\pgfqpoint{2.553099in}{2.172465in}}{\pgfqpoint{2.548709in}{2.183064in}}{\pgfqpoint{2.540895in}{2.190877in}}%
\pgfpathcurveto{\pgfqpoint{2.533081in}{2.198691in}}{\pgfqpoint{2.522482in}{2.203081in}}{\pgfqpoint{2.511432in}{2.203081in}}%
\pgfpathcurveto{\pgfqpoint{2.500382in}{2.203081in}}{\pgfqpoint{2.489783in}{2.198691in}}{\pgfqpoint{2.481969in}{2.190877in}}%
\pgfpathcurveto{\pgfqpoint{2.474156in}{2.183064in}}{\pgfqpoint{2.469765in}{2.172465in}}{\pgfqpoint{2.469765in}{2.161414in}}%
\pgfpathcurveto{\pgfqpoint{2.469765in}{2.150364in}}{\pgfqpoint{2.474156in}{2.139765in}}{\pgfqpoint{2.481969in}{2.131952in}}%
\pgfpathcurveto{\pgfqpoint{2.489783in}{2.124138in}}{\pgfqpoint{2.500382in}{2.119748in}}{\pgfqpoint{2.511432in}{2.119748in}}%
\pgfpathclose%
\pgfusepath{stroke,fill}%
\end{pgfscope}%
\begin{pgfscope}%
\pgfpathrectangle{\pgfqpoint{0.600000in}{0.600000in}}{\pgfqpoint{3.900000in}{3.900000in}}%
\pgfusepath{clip}%
\pgfsetbuttcap%
\pgfsetroundjoin%
\definecolor{currentfill}{rgb}{0.121569,0.466667,0.705882}%
\pgfsetfillcolor{currentfill}%
\pgfsetlinewidth{1.003750pt}%
\definecolor{currentstroke}{rgb}{0.121569,0.466667,0.705882}%
\pgfsetstrokecolor{currentstroke}%
\pgfsetdash{}{0pt}%
\pgfpathmoveto{\pgfqpoint{2.757405in}{3.329076in}}%
\pgfpathcurveto{\pgfqpoint{2.768455in}{3.329076in}}{\pgfqpoint{2.779054in}{3.333466in}}{\pgfqpoint{2.786867in}{3.341280in}}%
\pgfpathcurveto{\pgfqpoint{2.794681in}{3.349094in}}{\pgfqpoint{2.799071in}{3.359693in}}{\pgfqpoint{2.799071in}{3.370743in}}%
\pgfpathcurveto{\pgfqpoint{2.799071in}{3.381793in}}{\pgfqpoint{2.794681in}{3.392392in}}{\pgfqpoint{2.786867in}{3.400205in}}%
\pgfpathcurveto{\pgfqpoint{2.779054in}{3.408019in}}{\pgfqpoint{2.768455in}{3.412409in}}{\pgfqpoint{2.757405in}{3.412409in}}%
\pgfpathcurveto{\pgfqpoint{2.746354in}{3.412409in}}{\pgfqpoint{2.735755in}{3.408019in}}{\pgfqpoint{2.727942in}{3.400205in}}%
\pgfpathcurveto{\pgfqpoint{2.720128in}{3.392392in}}{\pgfqpoint{2.715738in}{3.381793in}}{\pgfqpoint{2.715738in}{3.370743in}}%
\pgfpathcurveto{\pgfqpoint{2.715738in}{3.359693in}}{\pgfqpoint{2.720128in}{3.349094in}}{\pgfqpoint{2.727942in}{3.341280in}}%
\pgfpathcurveto{\pgfqpoint{2.735755in}{3.333466in}}{\pgfqpoint{2.746354in}{3.329076in}}{\pgfqpoint{2.757405in}{3.329076in}}%
\pgfpathclose%
\pgfusepath{stroke,fill}%
\end{pgfscope}%
\begin{pgfscope}%
\pgfpathrectangle{\pgfqpoint{0.600000in}{0.600000in}}{\pgfqpoint{3.900000in}{3.900000in}}%
\pgfusepath{clip}%
\pgfsetbuttcap%
\pgfsetroundjoin%
\definecolor{currentfill}{rgb}{0.121569,0.466667,0.705882}%
\pgfsetfillcolor{currentfill}%
\pgfsetlinewidth{1.003750pt}%
\definecolor{currentstroke}{rgb}{0.121569,0.466667,0.705882}%
\pgfsetstrokecolor{currentstroke}%
\pgfsetdash{}{0pt}%
\pgfpathmoveto{\pgfqpoint{1.931718in}{2.162387in}}%
\pgfpathcurveto{\pgfqpoint{1.942768in}{2.162387in}}{\pgfqpoint{1.953367in}{2.166777in}}{\pgfqpoint{1.961181in}{2.174590in}}%
\pgfpathcurveto{\pgfqpoint{1.968994in}{2.182404in}}{\pgfqpoint{1.973384in}{2.193003in}}{\pgfqpoint{1.973384in}{2.204053in}}%
\pgfpathcurveto{\pgfqpoint{1.973384in}{2.215103in}}{\pgfqpoint{1.968994in}{2.225702in}}{\pgfqpoint{1.961181in}{2.233516in}}%
\pgfpathcurveto{\pgfqpoint{1.953367in}{2.241330in}}{\pgfqpoint{1.942768in}{2.245720in}}{\pgfqpoint{1.931718in}{2.245720in}}%
\pgfpathcurveto{\pgfqpoint{1.920668in}{2.245720in}}{\pgfqpoint{1.910069in}{2.241330in}}{\pgfqpoint{1.902255in}{2.233516in}}%
\pgfpathcurveto{\pgfqpoint{1.894441in}{2.225702in}}{\pgfqpoint{1.890051in}{2.215103in}}{\pgfqpoint{1.890051in}{2.204053in}}%
\pgfpathcurveto{\pgfqpoint{1.890051in}{2.193003in}}{\pgfqpoint{1.894441in}{2.182404in}}{\pgfqpoint{1.902255in}{2.174590in}}%
\pgfpathcurveto{\pgfqpoint{1.910069in}{2.166777in}}{\pgfqpoint{1.920668in}{2.162387in}}{\pgfqpoint{1.931718in}{2.162387in}}%
\pgfpathclose%
\pgfusepath{stroke,fill}%
\end{pgfscope}%
\begin{pgfscope}%
\pgfpathrectangle{\pgfqpoint{0.600000in}{0.600000in}}{\pgfqpoint{3.900000in}{3.900000in}}%
\pgfusepath{clip}%
\pgfsetbuttcap%
\pgfsetroundjoin%
\definecolor{currentfill}{rgb}{0.121569,0.466667,0.705882}%
\pgfsetfillcolor{currentfill}%
\pgfsetlinewidth{1.003750pt}%
\definecolor{currentstroke}{rgb}{0.121569,0.466667,0.705882}%
\pgfsetstrokecolor{currentstroke}%
\pgfsetdash{}{0pt}%
\pgfpathmoveto{\pgfqpoint{1.815101in}{2.909307in}}%
\pgfpathcurveto{\pgfqpoint{1.826152in}{2.909307in}}{\pgfqpoint{1.836751in}{2.913697in}}{\pgfqpoint{1.844564in}{2.921511in}}%
\pgfpathcurveto{\pgfqpoint{1.852378in}{2.929325in}}{\pgfqpoint{1.856768in}{2.939924in}}{\pgfqpoint{1.856768in}{2.950974in}}%
\pgfpathcurveto{\pgfqpoint{1.856768in}{2.962024in}}{\pgfqpoint{1.852378in}{2.972623in}}{\pgfqpoint{1.844564in}{2.980437in}}%
\pgfpathcurveto{\pgfqpoint{1.836751in}{2.988250in}}{\pgfqpoint{1.826152in}{2.992641in}}{\pgfqpoint{1.815101in}{2.992641in}}%
\pgfpathcurveto{\pgfqpoint{1.804051in}{2.992641in}}{\pgfqpoint{1.793452in}{2.988250in}}{\pgfqpoint{1.785639in}{2.980437in}}%
\pgfpathcurveto{\pgfqpoint{1.777825in}{2.972623in}}{\pgfqpoint{1.773435in}{2.962024in}}{\pgfqpoint{1.773435in}{2.950974in}}%
\pgfpathcurveto{\pgfqpoint{1.773435in}{2.939924in}}{\pgfqpoint{1.777825in}{2.929325in}}{\pgfqpoint{1.785639in}{2.921511in}}%
\pgfpathcurveto{\pgfqpoint{1.793452in}{2.913697in}}{\pgfqpoint{1.804051in}{2.909307in}}{\pgfqpoint{1.815101in}{2.909307in}}%
\pgfpathclose%
\pgfusepath{stroke,fill}%
\end{pgfscope}%
\begin{pgfscope}%
\pgfpathrectangle{\pgfqpoint{0.600000in}{0.600000in}}{\pgfqpoint{3.900000in}{3.900000in}}%
\pgfusepath{clip}%
\pgfsetbuttcap%
\pgfsetroundjoin%
\definecolor{currentfill}{rgb}{0.121569,0.466667,0.705882}%
\pgfsetfillcolor{currentfill}%
\pgfsetlinewidth{1.003750pt}%
\definecolor{currentstroke}{rgb}{0.121569,0.466667,0.705882}%
\pgfsetstrokecolor{currentstroke}%
\pgfsetdash{}{0pt}%
\pgfpathmoveto{\pgfqpoint{2.501207in}{3.121290in}}%
\pgfpathcurveto{\pgfqpoint{2.512257in}{3.121290in}}{\pgfqpoint{2.522856in}{3.125680in}}{\pgfqpoint{2.530669in}{3.133493in}}%
\pgfpathcurveto{\pgfqpoint{2.538483in}{3.141307in}}{\pgfqpoint{2.542873in}{3.151906in}}{\pgfqpoint{2.542873in}{3.162956in}}%
\pgfpathcurveto{\pgfqpoint{2.542873in}{3.174006in}}{\pgfqpoint{2.538483in}{3.184605in}}{\pgfqpoint{2.530669in}{3.192419in}}%
\pgfpathcurveto{\pgfqpoint{2.522856in}{3.200233in}}{\pgfqpoint{2.512257in}{3.204623in}}{\pgfqpoint{2.501207in}{3.204623in}}%
\pgfpathcurveto{\pgfqpoint{2.490157in}{3.204623in}}{\pgfqpoint{2.479557in}{3.200233in}}{\pgfqpoint{2.471744in}{3.192419in}}%
\pgfpathcurveto{\pgfqpoint{2.463930in}{3.184605in}}{\pgfqpoint{2.459540in}{3.174006in}}{\pgfqpoint{2.459540in}{3.162956in}}%
\pgfpathcurveto{\pgfqpoint{2.459540in}{3.151906in}}{\pgfqpoint{2.463930in}{3.141307in}}{\pgfqpoint{2.471744in}{3.133493in}}%
\pgfpathcurveto{\pgfqpoint{2.479557in}{3.125680in}}{\pgfqpoint{2.490157in}{3.121290in}}{\pgfqpoint{2.501207in}{3.121290in}}%
\pgfpathclose%
\pgfusepath{stroke,fill}%
\end{pgfscope}%
\begin{pgfscope}%
\pgfpathrectangle{\pgfqpoint{0.600000in}{0.600000in}}{\pgfqpoint{3.900000in}{3.900000in}}%
\pgfusepath{clip}%
\pgfsetbuttcap%
\pgfsetroundjoin%
\definecolor{currentfill}{rgb}{0.121569,0.466667,0.705882}%
\pgfsetfillcolor{currentfill}%
\pgfsetlinewidth{1.003750pt}%
\definecolor{currentstroke}{rgb}{0.121569,0.466667,0.705882}%
\pgfsetstrokecolor{currentstroke}%
\pgfsetdash{}{0pt}%
\pgfpathmoveto{\pgfqpoint{1.809636in}{2.079536in}}%
\pgfpathcurveto{\pgfqpoint{1.820686in}{2.079536in}}{\pgfqpoint{1.831285in}{2.083926in}}{\pgfqpoint{1.839099in}{2.091740in}}%
\pgfpathcurveto{\pgfqpoint{1.846913in}{2.099554in}}{\pgfqpoint{1.851303in}{2.110153in}}{\pgfqpoint{1.851303in}{2.121203in}}%
\pgfpathcurveto{\pgfqpoint{1.851303in}{2.132253in}}{\pgfqpoint{1.846913in}{2.142852in}}{\pgfqpoint{1.839099in}{2.150666in}}%
\pgfpathcurveto{\pgfqpoint{1.831285in}{2.158479in}}{\pgfqpoint{1.820686in}{2.162870in}}{\pgfqpoint{1.809636in}{2.162870in}}%
\pgfpathcurveto{\pgfqpoint{1.798586in}{2.162870in}}{\pgfqpoint{1.787987in}{2.158479in}}{\pgfqpoint{1.780173in}{2.150666in}}%
\pgfpathcurveto{\pgfqpoint{1.772360in}{2.142852in}}{\pgfqpoint{1.767970in}{2.132253in}}{\pgfqpoint{1.767970in}{2.121203in}}%
\pgfpathcurveto{\pgfqpoint{1.767970in}{2.110153in}}{\pgfqpoint{1.772360in}{2.099554in}}{\pgfqpoint{1.780173in}{2.091740in}}%
\pgfpathcurveto{\pgfqpoint{1.787987in}{2.083926in}}{\pgfqpoint{1.798586in}{2.079536in}}{\pgfqpoint{1.809636in}{2.079536in}}%
\pgfpathclose%
\pgfusepath{stroke,fill}%
\end{pgfscope}%
\begin{pgfscope}%
\pgfpathrectangle{\pgfqpoint{0.600000in}{0.600000in}}{\pgfqpoint{3.900000in}{3.900000in}}%
\pgfusepath{clip}%
\pgfsetbuttcap%
\pgfsetroundjoin%
\definecolor{currentfill}{rgb}{0.121569,0.466667,0.705882}%
\pgfsetfillcolor{currentfill}%
\pgfsetlinewidth{1.003750pt}%
\definecolor{currentstroke}{rgb}{0.121569,0.466667,0.705882}%
\pgfsetstrokecolor{currentstroke}%
\pgfsetdash{}{0pt}%
\pgfpathmoveto{\pgfqpoint{3.290910in}{2.710987in}}%
\pgfpathcurveto{\pgfqpoint{3.301960in}{2.710987in}}{\pgfqpoint{3.312559in}{2.715378in}}{\pgfqpoint{3.320373in}{2.723191in}}%
\pgfpathcurveto{\pgfqpoint{3.328187in}{2.731005in}}{\pgfqpoint{3.332577in}{2.741604in}}{\pgfqpoint{3.332577in}{2.752654in}}%
\pgfpathcurveto{\pgfqpoint{3.332577in}{2.763704in}}{\pgfqpoint{3.328187in}{2.774303in}}{\pgfqpoint{3.320373in}{2.782117in}}%
\pgfpathcurveto{\pgfqpoint{3.312559in}{2.789930in}}{\pgfqpoint{3.301960in}{2.794321in}}{\pgfqpoint{3.290910in}{2.794321in}}%
\pgfpathcurveto{\pgfqpoint{3.279860in}{2.794321in}}{\pgfqpoint{3.269261in}{2.789930in}}{\pgfqpoint{3.261447in}{2.782117in}}%
\pgfpathcurveto{\pgfqpoint{3.253634in}{2.774303in}}{\pgfqpoint{3.249244in}{2.763704in}}{\pgfqpoint{3.249244in}{2.752654in}}%
\pgfpathcurveto{\pgfqpoint{3.249244in}{2.741604in}}{\pgfqpoint{3.253634in}{2.731005in}}{\pgfqpoint{3.261447in}{2.723191in}}%
\pgfpathcurveto{\pgfqpoint{3.269261in}{2.715378in}}{\pgfqpoint{3.279860in}{2.710987in}}{\pgfqpoint{3.290910in}{2.710987in}}%
\pgfpathclose%
\pgfusepath{stroke,fill}%
\end{pgfscope}%
\begin{pgfscope}%
\pgfpathrectangle{\pgfqpoint{0.600000in}{0.600000in}}{\pgfqpoint{3.900000in}{3.900000in}}%
\pgfusepath{clip}%
\pgfsetbuttcap%
\pgfsetroundjoin%
\definecolor{currentfill}{rgb}{0.121569,0.466667,0.705882}%
\pgfsetfillcolor{currentfill}%
\pgfsetlinewidth{1.003750pt}%
\definecolor{currentstroke}{rgb}{0.121569,0.466667,0.705882}%
\pgfsetstrokecolor{currentstroke}%
\pgfsetdash{}{0pt}%
\pgfpathmoveto{\pgfqpoint{1.625868in}{3.347244in}}%
\pgfpathcurveto{\pgfqpoint{1.636918in}{3.347244in}}{\pgfqpoint{1.647517in}{3.351635in}}{\pgfqpoint{1.655330in}{3.359448in}}%
\pgfpathcurveto{\pgfqpoint{1.663144in}{3.367262in}}{\pgfqpoint{1.667534in}{3.377861in}}{\pgfqpoint{1.667534in}{3.388911in}}%
\pgfpathcurveto{\pgfqpoint{1.667534in}{3.399961in}}{\pgfqpoint{1.663144in}{3.410560in}}{\pgfqpoint{1.655330in}{3.418374in}}%
\pgfpathcurveto{\pgfqpoint{1.647517in}{3.426187in}}{\pgfqpoint{1.636918in}{3.430578in}}{\pgfqpoint{1.625868in}{3.430578in}}%
\pgfpathcurveto{\pgfqpoint{1.614817in}{3.430578in}}{\pgfqpoint{1.604218in}{3.426187in}}{\pgfqpoint{1.596405in}{3.418374in}}%
\pgfpathcurveto{\pgfqpoint{1.588591in}{3.410560in}}{\pgfqpoint{1.584201in}{3.399961in}}{\pgfqpoint{1.584201in}{3.388911in}}%
\pgfpathcurveto{\pgfqpoint{1.584201in}{3.377861in}}{\pgfqpoint{1.588591in}{3.367262in}}{\pgfqpoint{1.596405in}{3.359448in}}%
\pgfpathcurveto{\pgfqpoint{1.604218in}{3.351635in}}{\pgfqpoint{1.614817in}{3.347244in}}{\pgfqpoint{1.625868in}{3.347244in}}%
\pgfpathclose%
\pgfusepath{stroke,fill}%
\end{pgfscope}%
\begin{pgfscope}%
\pgfpathrectangle{\pgfqpoint{0.600000in}{0.600000in}}{\pgfqpoint{3.900000in}{3.900000in}}%
\pgfusepath{clip}%
\pgfsetbuttcap%
\pgfsetroundjoin%
\definecolor{currentfill}{rgb}{0.121569,0.466667,0.705882}%
\pgfsetfillcolor{currentfill}%
\pgfsetlinewidth{1.003750pt}%
\definecolor{currentstroke}{rgb}{0.121569,0.466667,0.705882}%
\pgfsetstrokecolor{currentstroke}%
\pgfsetdash{}{0pt}%
\pgfpathmoveto{\pgfqpoint{2.095572in}{1.694221in}}%
\pgfpathcurveto{\pgfqpoint{2.106622in}{1.694221in}}{\pgfqpoint{2.117221in}{1.698612in}}{\pgfqpoint{2.125034in}{1.706425in}}%
\pgfpathcurveto{\pgfqpoint{2.132848in}{1.714239in}}{\pgfqpoint{2.137238in}{1.724838in}}{\pgfqpoint{2.137238in}{1.735888in}}%
\pgfpathcurveto{\pgfqpoint{2.137238in}{1.746938in}}{\pgfqpoint{2.132848in}{1.757537in}}{\pgfqpoint{2.125034in}{1.765351in}}%
\pgfpathcurveto{\pgfqpoint{2.117221in}{1.773164in}}{\pgfqpoint{2.106622in}{1.777555in}}{\pgfqpoint{2.095572in}{1.777555in}}%
\pgfpathcurveto{\pgfqpoint{2.084522in}{1.777555in}}{\pgfqpoint{2.073923in}{1.773164in}}{\pgfqpoint{2.066109in}{1.765351in}}%
\pgfpathcurveto{\pgfqpoint{2.058295in}{1.757537in}}{\pgfqpoint{2.053905in}{1.746938in}}{\pgfqpoint{2.053905in}{1.735888in}}%
\pgfpathcurveto{\pgfqpoint{2.053905in}{1.724838in}}{\pgfqpoint{2.058295in}{1.714239in}}{\pgfqpoint{2.066109in}{1.706425in}}%
\pgfpathcurveto{\pgfqpoint{2.073923in}{1.698612in}}{\pgfqpoint{2.084522in}{1.694221in}}{\pgfqpoint{2.095572in}{1.694221in}}%
\pgfpathclose%
\pgfusepath{stroke,fill}%
\end{pgfscope}%
\begin{pgfscope}%
\pgfpathrectangle{\pgfqpoint{0.600000in}{0.600000in}}{\pgfqpoint{3.900000in}{3.900000in}}%
\pgfusepath{clip}%
\pgfsetbuttcap%
\pgfsetroundjoin%
\definecolor{currentfill}{rgb}{0.121569,0.466667,0.705882}%
\pgfsetfillcolor{currentfill}%
\pgfsetlinewidth{1.003750pt}%
\definecolor{currentstroke}{rgb}{0.121569,0.466667,0.705882}%
\pgfsetstrokecolor{currentstroke}%
\pgfsetdash{}{0pt}%
\pgfpathmoveto{\pgfqpoint{1.838331in}{2.868886in}}%
\pgfpathcurveto{\pgfqpoint{1.849382in}{2.868886in}}{\pgfqpoint{1.859981in}{2.873276in}}{\pgfqpoint{1.867794in}{2.881090in}}%
\pgfpathcurveto{\pgfqpoint{1.875608in}{2.888903in}}{\pgfqpoint{1.879998in}{2.899502in}}{\pgfqpoint{1.879998in}{2.910553in}}%
\pgfpathcurveto{\pgfqpoint{1.879998in}{2.921603in}}{\pgfqpoint{1.875608in}{2.932202in}}{\pgfqpoint{1.867794in}{2.940015in}}%
\pgfpathcurveto{\pgfqpoint{1.859981in}{2.947829in}}{\pgfqpoint{1.849382in}{2.952219in}}{\pgfqpoint{1.838331in}{2.952219in}}%
\pgfpathcurveto{\pgfqpoint{1.827281in}{2.952219in}}{\pgfqpoint{1.816682in}{2.947829in}}{\pgfqpoint{1.808869in}{2.940015in}}%
\pgfpathcurveto{\pgfqpoint{1.801055in}{2.932202in}}{\pgfqpoint{1.796665in}{2.921603in}}{\pgfqpoint{1.796665in}{2.910553in}}%
\pgfpathcurveto{\pgfqpoint{1.796665in}{2.899502in}}{\pgfqpoint{1.801055in}{2.888903in}}{\pgfqpoint{1.808869in}{2.881090in}}%
\pgfpathcurveto{\pgfqpoint{1.816682in}{2.873276in}}{\pgfqpoint{1.827281in}{2.868886in}}{\pgfqpoint{1.838331in}{2.868886in}}%
\pgfpathclose%
\pgfusepath{stroke,fill}%
\end{pgfscope}%
\begin{pgfscope}%
\pgfpathrectangle{\pgfqpoint{0.600000in}{0.600000in}}{\pgfqpoint{3.900000in}{3.900000in}}%
\pgfusepath{clip}%
\pgfsetbuttcap%
\pgfsetroundjoin%
\definecolor{currentfill}{rgb}{0.121569,0.466667,0.705882}%
\pgfsetfillcolor{currentfill}%
\pgfsetlinewidth{1.003750pt}%
\definecolor{currentstroke}{rgb}{0.121569,0.466667,0.705882}%
\pgfsetstrokecolor{currentstroke}%
\pgfsetdash{}{0pt}%
\pgfpathmoveto{\pgfqpoint{3.118634in}{1.641119in}}%
\pgfpathcurveto{\pgfqpoint{3.129684in}{1.641119in}}{\pgfqpoint{3.140283in}{1.645509in}}{\pgfqpoint{3.148097in}{1.653323in}}%
\pgfpathcurveto{\pgfqpoint{3.155910in}{1.661136in}}{\pgfqpoint{3.160301in}{1.671735in}}{\pgfqpoint{3.160301in}{1.682786in}}%
\pgfpathcurveto{\pgfqpoint{3.160301in}{1.693836in}}{\pgfqpoint{3.155910in}{1.704435in}}{\pgfqpoint{3.148097in}{1.712248in}}%
\pgfpathcurveto{\pgfqpoint{3.140283in}{1.720062in}}{\pgfqpoint{3.129684in}{1.724452in}}{\pgfqpoint{3.118634in}{1.724452in}}%
\pgfpathcurveto{\pgfqpoint{3.107584in}{1.724452in}}{\pgfqpoint{3.096985in}{1.720062in}}{\pgfqpoint{3.089171in}{1.712248in}}%
\pgfpathcurveto{\pgfqpoint{3.081357in}{1.704435in}}{\pgfqpoint{3.076967in}{1.693836in}}{\pgfqpoint{3.076967in}{1.682786in}}%
\pgfpathcurveto{\pgfqpoint{3.076967in}{1.671735in}}{\pgfqpoint{3.081357in}{1.661136in}}{\pgfqpoint{3.089171in}{1.653323in}}%
\pgfpathcurveto{\pgfqpoint{3.096985in}{1.645509in}}{\pgfqpoint{3.107584in}{1.641119in}}{\pgfqpoint{3.118634in}{1.641119in}}%
\pgfpathclose%
\pgfusepath{stroke,fill}%
\end{pgfscope}%
\begin{pgfscope}%
\pgfpathrectangle{\pgfqpoint{0.600000in}{0.600000in}}{\pgfqpoint{3.900000in}{3.900000in}}%
\pgfusepath{clip}%
\pgfsetbuttcap%
\pgfsetroundjoin%
\definecolor{currentfill}{rgb}{0.121569,0.466667,0.705882}%
\pgfsetfillcolor{currentfill}%
\pgfsetlinewidth{1.003750pt}%
\definecolor{currentstroke}{rgb}{0.121569,0.466667,0.705882}%
\pgfsetstrokecolor{currentstroke}%
\pgfsetdash{}{0pt}%
\pgfpathmoveto{\pgfqpoint{2.249026in}{2.533771in}}%
\pgfpathcurveto{\pgfqpoint{2.260076in}{2.533771in}}{\pgfqpoint{2.270675in}{2.538161in}}{\pgfqpoint{2.278488in}{2.545975in}}%
\pgfpathcurveto{\pgfqpoint{2.286302in}{2.553788in}}{\pgfqpoint{2.290692in}{2.564387in}}{\pgfqpoint{2.290692in}{2.575438in}}%
\pgfpathcurveto{\pgfqpoint{2.290692in}{2.586488in}}{\pgfqpoint{2.286302in}{2.597087in}}{\pgfqpoint{2.278488in}{2.604900in}}%
\pgfpathcurveto{\pgfqpoint{2.270675in}{2.612714in}}{\pgfqpoint{2.260076in}{2.617104in}}{\pgfqpoint{2.249026in}{2.617104in}}%
\pgfpathcurveto{\pgfqpoint{2.237976in}{2.617104in}}{\pgfqpoint{2.227377in}{2.612714in}}{\pgfqpoint{2.219563in}{2.604900in}}%
\pgfpathcurveto{\pgfqpoint{2.211749in}{2.597087in}}{\pgfqpoint{2.207359in}{2.586488in}}{\pgfqpoint{2.207359in}{2.575438in}}%
\pgfpathcurveto{\pgfqpoint{2.207359in}{2.564387in}}{\pgfqpoint{2.211749in}{2.553788in}}{\pgfqpoint{2.219563in}{2.545975in}}%
\pgfpathcurveto{\pgfqpoint{2.227377in}{2.538161in}}{\pgfqpoint{2.237976in}{2.533771in}}{\pgfqpoint{2.249026in}{2.533771in}}%
\pgfpathclose%
\pgfusepath{stroke,fill}%
\end{pgfscope}%
\begin{pgfscope}%
\pgfpathrectangle{\pgfqpoint{0.600000in}{0.600000in}}{\pgfqpoint{3.900000in}{3.900000in}}%
\pgfusepath{clip}%
\pgfsetbuttcap%
\pgfsetroundjoin%
\definecolor{currentfill}{rgb}{0.121569,0.466667,0.705882}%
\pgfsetfillcolor{currentfill}%
\pgfsetlinewidth{1.003750pt}%
\definecolor{currentstroke}{rgb}{0.121569,0.466667,0.705882}%
\pgfsetstrokecolor{currentstroke}%
\pgfsetdash{}{0pt}%
\pgfpathmoveto{\pgfqpoint{2.258523in}{2.354558in}}%
\pgfpathcurveto{\pgfqpoint{2.269574in}{2.354558in}}{\pgfqpoint{2.280173in}{2.358948in}}{\pgfqpoint{2.287986in}{2.366762in}}%
\pgfpathcurveto{\pgfqpoint{2.295800in}{2.374575in}}{\pgfqpoint{2.300190in}{2.385174in}}{\pgfqpoint{2.300190in}{2.396224in}}%
\pgfpathcurveto{\pgfqpoint{2.300190in}{2.407274in}}{\pgfqpoint{2.295800in}{2.417873in}}{\pgfqpoint{2.287986in}{2.425687in}}%
\pgfpathcurveto{\pgfqpoint{2.280173in}{2.433501in}}{\pgfqpoint{2.269574in}{2.437891in}}{\pgfqpoint{2.258523in}{2.437891in}}%
\pgfpathcurveto{\pgfqpoint{2.247473in}{2.437891in}}{\pgfqpoint{2.236874in}{2.433501in}}{\pgfqpoint{2.229061in}{2.425687in}}%
\pgfpathcurveto{\pgfqpoint{2.221247in}{2.417873in}}{\pgfqpoint{2.216857in}{2.407274in}}{\pgfqpoint{2.216857in}{2.396224in}}%
\pgfpathcurveto{\pgfqpoint{2.216857in}{2.385174in}}{\pgfqpoint{2.221247in}{2.374575in}}{\pgfqpoint{2.229061in}{2.366762in}}%
\pgfpathcurveto{\pgfqpoint{2.236874in}{2.358948in}}{\pgfqpoint{2.247473in}{2.354558in}}{\pgfqpoint{2.258523in}{2.354558in}}%
\pgfpathclose%
\pgfusepath{stroke,fill}%
\end{pgfscope}%
\begin{pgfscope}%
\pgfpathrectangle{\pgfqpoint{0.600000in}{0.600000in}}{\pgfqpoint{3.900000in}{3.900000in}}%
\pgfusepath{clip}%
\pgfsetbuttcap%
\pgfsetroundjoin%
\definecolor{currentfill}{rgb}{0.121569,0.466667,0.705882}%
\pgfsetfillcolor{currentfill}%
\pgfsetlinewidth{1.003750pt}%
\definecolor{currentstroke}{rgb}{0.121569,0.466667,0.705882}%
\pgfsetstrokecolor{currentstroke}%
\pgfsetdash{}{0pt}%
\pgfpathmoveto{\pgfqpoint{1.255907in}{2.030893in}}%
\pgfpathcurveto{\pgfqpoint{1.266957in}{2.030893in}}{\pgfqpoint{1.277557in}{2.035283in}}{\pgfqpoint{1.285370in}{2.043096in}}%
\pgfpathcurveto{\pgfqpoint{1.293184in}{2.050910in}}{\pgfqpoint{1.297574in}{2.061509in}}{\pgfqpoint{1.297574in}{2.072559in}}%
\pgfpathcurveto{\pgfqpoint{1.297574in}{2.083609in}}{\pgfqpoint{1.293184in}{2.094208in}}{\pgfqpoint{1.285370in}{2.102022in}}%
\pgfpathcurveto{\pgfqpoint{1.277557in}{2.109836in}}{\pgfqpoint{1.266957in}{2.114226in}}{\pgfqpoint{1.255907in}{2.114226in}}%
\pgfpathcurveto{\pgfqpoint{1.244857in}{2.114226in}}{\pgfqpoint{1.234258in}{2.109836in}}{\pgfqpoint{1.226445in}{2.102022in}}%
\pgfpathcurveto{\pgfqpoint{1.218631in}{2.094208in}}{\pgfqpoint{1.214241in}{2.083609in}}{\pgfqpoint{1.214241in}{2.072559in}}%
\pgfpathcurveto{\pgfqpoint{1.214241in}{2.061509in}}{\pgfqpoint{1.218631in}{2.050910in}}{\pgfqpoint{1.226445in}{2.043096in}}%
\pgfpathcurveto{\pgfqpoint{1.234258in}{2.035283in}}{\pgfqpoint{1.244857in}{2.030893in}}{\pgfqpoint{1.255907in}{2.030893in}}%
\pgfpathclose%
\pgfusepath{stroke,fill}%
\end{pgfscope}%
\begin{pgfscope}%
\pgfpathrectangle{\pgfqpoint{0.600000in}{0.600000in}}{\pgfqpoint{3.900000in}{3.900000in}}%
\pgfusepath{clip}%
\pgfsetbuttcap%
\pgfsetroundjoin%
\definecolor{currentfill}{rgb}{0.121569,0.466667,0.705882}%
\pgfsetfillcolor{currentfill}%
\pgfsetlinewidth{1.003750pt}%
\definecolor{currentstroke}{rgb}{0.121569,0.466667,0.705882}%
\pgfsetstrokecolor{currentstroke}%
\pgfsetdash{}{0pt}%
\pgfpathmoveto{\pgfqpoint{1.964353in}{2.326708in}}%
\pgfpathcurveto{\pgfqpoint{1.975403in}{2.326708in}}{\pgfqpoint{1.986002in}{2.331098in}}{\pgfqpoint{1.993815in}{2.338911in}}%
\pgfpathcurveto{\pgfqpoint{2.001629in}{2.346725in}}{\pgfqpoint{2.006019in}{2.357324in}}{\pgfqpoint{2.006019in}{2.368374in}}%
\pgfpathcurveto{\pgfqpoint{2.006019in}{2.379424in}}{\pgfqpoint{2.001629in}{2.390023in}}{\pgfqpoint{1.993815in}{2.397837in}}%
\pgfpathcurveto{\pgfqpoint{1.986002in}{2.405651in}}{\pgfqpoint{1.975403in}{2.410041in}}{\pgfqpoint{1.964353in}{2.410041in}}%
\pgfpathcurveto{\pgfqpoint{1.953303in}{2.410041in}}{\pgfqpoint{1.942703in}{2.405651in}}{\pgfqpoint{1.934890in}{2.397837in}}%
\pgfpathcurveto{\pgfqpoint{1.927076in}{2.390023in}}{\pgfqpoint{1.922686in}{2.379424in}}{\pgfqpoint{1.922686in}{2.368374in}}%
\pgfpathcurveto{\pgfqpoint{1.922686in}{2.357324in}}{\pgfqpoint{1.927076in}{2.346725in}}{\pgfqpoint{1.934890in}{2.338911in}}%
\pgfpathcurveto{\pgfqpoint{1.942703in}{2.331098in}}{\pgfqpoint{1.953303in}{2.326708in}}{\pgfqpoint{1.964353in}{2.326708in}}%
\pgfpathclose%
\pgfusepath{stroke,fill}%
\end{pgfscope}%
\begin{pgfscope}%
\pgfpathrectangle{\pgfqpoint{0.600000in}{0.600000in}}{\pgfqpoint{3.900000in}{3.900000in}}%
\pgfusepath{clip}%
\pgfsetbuttcap%
\pgfsetroundjoin%
\definecolor{currentfill}{rgb}{0.121569,0.466667,0.705882}%
\pgfsetfillcolor{currentfill}%
\pgfsetlinewidth{1.003750pt}%
\definecolor{currentstroke}{rgb}{0.121569,0.466667,0.705882}%
\pgfsetstrokecolor{currentstroke}%
\pgfsetdash{}{0pt}%
\pgfpathmoveto{\pgfqpoint{3.303068in}{2.666130in}}%
\pgfpathcurveto{\pgfqpoint{3.314118in}{2.666130in}}{\pgfqpoint{3.324717in}{2.670520in}}{\pgfqpoint{3.332531in}{2.678334in}}%
\pgfpathcurveto{\pgfqpoint{3.340345in}{2.686147in}}{\pgfqpoint{3.344735in}{2.696746in}}{\pgfqpoint{3.344735in}{2.707796in}}%
\pgfpathcurveto{\pgfqpoint{3.344735in}{2.718847in}}{\pgfqpoint{3.340345in}{2.729446in}}{\pgfqpoint{3.332531in}{2.737259in}}%
\pgfpathcurveto{\pgfqpoint{3.324717in}{2.745073in}}{\pgfqpoint{3.314118in}{2.749463in}}{\pgfqpoint{3.303068in}{2.749463in}}%
\pgfpathcurveto{\pgfqpoint{3.292018in}{2.749463in}}{\pgfqpoint{3.281419in}{2.745073in}}{\pgfqpoint{3.273606in}{2.737259in}}%
\pgfpathcurveto{\pgfqpoint{3.265792in}{2.729446in}}{\pgfqpoint{3.261402in}{2.718847in}}{\pgfqpoint{3.261402in}{2.707796in}}%
\pgfpathcurveto{\pgfqpoint{3.261402in}{2.696746in}}{\pgfqpoint{3.265792in}{2.686147in}}{\pgfqpoint{3.273606in}{2.678334in}}%
\pgfpathcurveto{\pgfqpoint{3.281419in}{2.670520in}}{\pgfqpoint{3.292018in}{2.666130in}}{\pgfqpoint{3.303068in}{2.666130in}}%
\pgfpathclose%
\pgfusepath{stroke,fill}%
\end{pgfscope}%
\begin{pgfscope}%
\pgfpathrectangle{\pgfqpoint{0.600000in}{0.600000in}}{\pgfqpoint{3.900000in}{3.900000in}}%
\pgfusepath{clip}%
\pgfsetbuttcap%
\pgfsetroundjoin%
\definecolor{currentfill}{rgb}{0.121569,0.466667,0.705882}%
\pgfsetfillcolor{currentfill}%
\pgfsetlinewidth{1.003750pt}%
\definecolor{currentstroke}{rgb}{0.121569,0.466667,0.705882}%
\pgfsetstrokecolor{currentstroke}%
\pgfsetdash{}{0pt}%
\pgfpathmoveto{\pgfqpoint{3.064836in}{2.056747in}}%
\pgfpathcurveto{\pgfqpoint{3.075886in}{2.056747in}}{\pgfqpoint{3.086485in}{2.061137in}}{\pgfqpoint{3.094298in}{2.068950in}}%
\pgfpathcurveto{\pgfqpoint{3.102112in}{2.076764in}}{\pgfqpoint{3.106502in}{2.087363in}}{\pgfqpoint{3.106502in}{2.098413in}}%
\pgfpathcurveto{\pgfqpoint{3.106502in}{2.109463in}}{\pgfqpoint{3.102112in}{2.120062in}}{\pgfqpoint{3.094298in}{2.127876in}}%
\pgfpathcurveto{\pgfqpoint{3.086485in}{2.135690in}}{\pgfqpoint{3.075886in}{2.140080in}}{\pgfqpoint{3.064836in}{2.140080in}}%
\pgfpathcurveto{\pgfqpoint{3.053785in}{2.140080in}}{\pgfqpoint{3.043186in}{2.135690in}}{\pgfqpoint{3.035373in}{2.127876in}}%
\pgfpathcurveto{\pgfqpoint{3.027559in}{2.120062in}}{\pgfqpoint{3.023169in}{2.109463in}}{\pgfqpoint{3.023169in}{2.098413in}}%
\pgfpathcurveto{\pgfqpoint{3.023169in}{2.087363in}}{\pgfqpoint{3.027559in}{2.076764in}}{\pgfqpoint{3.035373in}{2.068950in}}%
\pgfpathcurveto{\pgfqpoint{3.043186in}{2.061137in}}{\pgfqpoint{3.053785in}{2.056747in}}{\pgfqpoint{3.064836in}{2.056747in}}%
\pgfpathclose%
\pgfusepath{stroke,fill}%
\end{pgfscope}%
\begin{pgfscope}%
\pgfpathrectangle{\pgfqpoint{0.600000in}{0.600000in}}{\pgfqpoint{3.900000in}{3.900000in}}%
\pgfusepath{clip}%
\pgfsetbuttcap%
\pgfsetroundjoin%
\definecolor{currentfill}{rgb}{0.121569,0.466667,0.705882}%
\pgfsetfillcolor{currentfill}%
\pgfsetlinewidth{1.003750pt}%
\definecolor{currentstroke}{rgb}{0.121569,0.466667,0.705882}%
\pgfsetstrokecolor{currentstroke}%
\pgfsetdash{}{0pt}%
\pgfpathmoveto{\pgfqpoint{3.413907in}{2.075069in}}%
\pgfpathcurveto{\pgfqpoint{3.424957in}{2.075069in}}{\pgfqpoint{3.435556in}{2.079459in}}{\pgfqpoint{3.443369in}{2.087273in}}%
\pgfpathcurveto{\pgfqpoint{3.451183in}{2.095087in}}{\pgfqpoint{3.455573in}{2.105686in}}{\pgfqpoint{3.455573in}{2.116736in}}%
\pgfpathcurveto{\pgfqpoint{3.455573in}{2.127786in}}{\pgfqpoint{3.451183in}{2.138385in}}{\pgfqpoint{3.443369in}{2.146199in}}%
\pgfpathcurveto{\pgfqpoint{3.435556in}{2.154012in}}{\pgfqpoint{3.424957in}{2.158402in}}{\pgfqpoint{3.413907in}{2.158402in}}%
\pgfpathcurveto{\pgfqpoint{3.402856in}{2.158402in}}{\pgfqpoint{3.392257in}{2.154012in}}{\pgfqpoint{3.384444in}{2.146199in}}%
\pgfpathcurveto{\pgfqpoint{3.376630in}{2.138385in}}{\pgfqpoint{3.372240in}{2.127786in}}{\pgfqpoint{3.372240in}{2.116736in}}%
\pgfpathcurveto{\pgfqpoint{3.372240in}{2.105686in}}{\pgfqpoint{3.376630in}{2.095087in}}{\pgfqpoint{3.384444in}{2.087273in}}%
\pgfpathcurveto{\pgfqpoint{3.392257in}{2.079459in}}{\pgfqpoint{3.402856in}{2.075069in}}{\pgfqpoint{3.413907in}{2.075069in}}%
\pgfpathclose%
\pgfusepath{stroke,fill}%
\end{pgfscope}%
\begin{pgfscope}%
\pgfpathrectangle{\pgfqpoint{0.600000in}{0.600000in}}{\pgfqpoint{3.900000in}{3.900000in}}%
\pgfusepath{clip}%
\pgfsetbuttcap%
\pgfsetroundjoin%
\definecolor{currentfill}{rgb}{0.121569,0.466667,0.705882}%
\pgfsetfillcolor{currentfill}%
\pgfsetlinewidth{1.003750pt}%
\definecolor{currentstroke}{rgb}{0.121569,0.466667,0.705882}%
\pgfsetstrokecolor{currentstroke}%
\pgfsetdash{}{0pt}%
\pgfpathmoveto{\pgfqpoint{1.722307in}{1.541817in}}%
\pgfpathcurveto{\pgfqpoint{1.733357in}{1.541817in}}{\pgfqpoint{1.743956in}{1.546207in}}{\pgfqpoint{1.751770in}{1.554021in}}%
\pgfpathcurveto{\pgfqpoint{1.759584in}{1.561834in}}{\pgfqpoint{1.763974in}{1.572433in}}{\pgfqpoint{1.763974in}{1.583484in}}%
\pgfpathcurveto{\pgfqpoint{1.763974in}{1.594534in}}{\pgfqpoint{1.759584in}{1.605133in}}{\pgfqpoint{1.751770in}{1.612946in}}%
\pgfpathcurveto{\pgfqpoint{1.743956in}{1.620760in}}{\pgfqpoint{1.733357in}{1.625150in}}{\pgfqpoint{1.722307in}{1.625150in}}%
\pgfpathcurveto{\pgfqpoint{1.711257in}{1.625150in}}{\pgfqpoint{1.700658in}{1.620760in}}{\pgfqpoint{1.692845in}{1.612946in}}%
\pgfpathcurveto{\pgfqpoint{1.685031in}{1.605133in}}{\pgfqpoint{1.680641in}{1.594534in}}{\pgfqpoint{1.680641in}{1.583484in}}%
\pgfpathcurveto{\pgfqpoint{1.680641in}{1.572433in}}{\pgfqpoint{1.685031in}{1.561834in}}{\pgfqpoint{1.692845in}{1.554021in}}%
\pgfpathcurveto{\pgfqpoint{1.700658in}{1.546207in}}{\pgfqpoint{1.711257in}{1.541817in}}{\pgfqpoint{1.722307in}{1.541817in}}%
\pgfpathclose%
\pgfusepath{stroke,fill}%
\end{pgfscope}%
\begin{pgfscope}%
\pgfpathrectangle{\pgfqpoint{0.600000in}{0.600000in}}{\pgfqpoint{3.900000in}{3.900000in}}%
\pgfusepath{clip}%
\pgfsetbuttcap%
\pgfsetroundjoin%
\definecolor{currentfill}{rgb}{0.121569,0.466667,0.705882}%
\pgfsetfillcolor{currentfill}%
\pgfsetlinewidth{1.003750pt}%
\definecolor{currentstroke}{rgb}{0.121569,0.466667,0.705882}%
\pgfsetstrokecolor{currentstroke}%
\pgfsetdash{}{0pt}%
\pgfpathmoveto{\pgfqpoint{2.353624in}{2.403904in}}%
\pgfpathcurveto{\pgfqpoint{2.364674in}{2.403904in}}{\pgfqpoint{2.375273in}{2.408294in}}{\pgfqpoint{2.383087in}{2.416108in}}%
\pgfpathcurveto{\pgfqpoint{2.390900in}{2.423921in}}{\pgfqpoint{2.395291in}{2.434520in}}{\pgfqpoint{2.395291in}{2.445570in}}%
\pgfpathcurveto{\pgfqpoint{2.395291in}{2.456620in}}{\pgfqpoint{2.390900in}{2.467219in}}{\pgfqpoint{2.383087in}{2.475033in}}%
\pgfpathcurveto{\pgfqpoint{2.375273in}{2.482847in}}{\pgfqpoint{2.364674in}{2.487237in}}{\pgfqpoint{2.353624in}{2.487237in}}%
\pgfpathcurveto{\pgfqpoint{2.342574in}{2.487237in}}{\pgfqpoint{2.331975in}{2.482847in}}{\pgfqpoint{2.324161in}{2.475033in}}%
\pgfpathcurveto{\pgfqpoint{2.316348in}{2.467219in}}{\pgfqpoint{2.311957in}{2.456620in}}{\pgfqpoint{2.311957in}{2.445570in}}%
\pgfpathcurveto{\pgfqpoint{2.311957in}{2.434520in}}{\pgfqpoint{2.316348in}{2.423921in}}{\pgfqpoint{2.324161in}{2.416108in}}%
\pgfpathcurveto{\pgfqpoint{2.331975in}{2.408294in}}{\pgfqpoint{2.342574in}{2.403904in}}{\pgfqpoint{2.353624in}{2.403904in}}%
\pgfpathclose%
\pgfusepath{stroke,fill}%
\end{pgfscope}%
\begin{pgfscope}%
\pgfpathrectangle{\pgfqpoint{0.600000in}{0.600000in}}{\pgfqpoint{3.900000in}{3.900000in}}%
\pgfusepath{clip}%
\pgfsetbuttcap%
\pgfsetroundjoin%
\definecolor{currentfill}{rgb}{0.121569,0.466667,0.705882}%
\pgfsetfillcolor{currentfill}%
\pgfsetlinewidth{1.003750pt}%
\definecolor{currentstroke}{rgb}{0.121569,0.466667,0.705882}%
\pgfsetstrokecolor{currentstroke}%
\pgfsetdash{}{0pt}%
\pgfpathmoveto{\pgfqpoint{3.192503in}{2.750098in}}%
\pgfpathcurveto{\pgfqpoint{3.203553in}{2.750098in}}{\pgfqpoint{3.214152in}{2.754488in}}{\pgfqpoint{3.221966in}{2.762302in}}%
\pgfpathcurveto{\pgfqpoint{3.229779in}{2.770115in}}{\pgfqpoint{3.234170in}{2.780714in}}{\pgfqpoint{3.234170in}{2.791764in}}%
\pgfpathcurveto{\pgfqpoint{3.234170in}{2.802815in}}{\pgfqpoint{3.229779in}{2.813414in}}{\pgfqpoint{3.221966in}{2.821227in}}%
\pgfpathcurveto{\pgfqpoint{3.214152in}{2.829041in}}{\pgfqpoint{3.203553in}{2.833431in}}{\pgfqpoint{3.192503in}{2.833431in}}%
\pgfpathcurveto{\pgfqpoint{3.181453in}{2.833431in}}{\pgfqpoint{3.170854in}{2.829041in}}{\pgfqpoint{3.163040in}{2.821227in}}%
\pgfpathcurveto{\pgfqpoint{3.155227in}{2.813414in}}{\pgfqpoint{3.150836in}{2.802815in}}{\pgfqpoint{3.150836in}{2.791764in}}%
\pgfpathcurveto{\pgfqpoint{3.150836in}{2.780714in}}{\pgfqpoint{3.155227in}{2.770115in}}{\pgfqpoint{3.163040in}{2.762302in}}%
\pgfpathcurveto{\pgfqpoint{3.170854in}{2.754488in}}{\pgfqpoint{3.181453in}{2.750098in}}{\pgfqpoint{3.192503in}{2.750098in}}%
\pgfpathclose%
\pgfusepath{stroke,fill}%
\end{pgfscope}%
\begin{pgfscope}%
\pgfpathrectangle{\pgfqpoint{0.600000in}{0.600000in}}{\pgfqpoint{3.900000in}{3.900000in}}%
\pgfusepath{clip}%
\pgfsetbuttcap%
\pgfsetroundjoin%
\definecolor{currentfill}{rgb}{0.121569,0.466667,0.705882}%
\pgfsetfillcolor{currentfill}%
\pgfsetlinewidth{1.003750pt}%
\definecolor{currentstroke}{rgb}{0.121569,0.466667,0.705882}%
\pgfsetstrokecolor{currentstroke}%
\pgfsetdash{}{0pt}%
\pgfpathmoveto{\pgfqpoint{2.047470in}{1.926070in}}%
\pgfpathcurveto{\pgfqpoint{2.058520in}{1.926070in}}{\pgfqpoint{2.069119in}{1.930460in}}{\pgfqpoint{2.076932in}{1.938273in}}%
\pgfpathcurveto{\pgfqpoint{2.084746in}{1.946087in}}{\pgfqpoint{2.089136in}{1.956686in}}{\pgfqpoint{2.089136in}{1.967736in}}%
\pgfpathcurveto{\pgfqpoint{2.089136in}{1.978786in}}{\pgfqpoint{2.084746in}{1.989385in}}{\pgfqpoint{2.076932in}{1.997199in}}%
\pgfpathcurveto{\pgfqpoint{2.069119in}{2.005013in}}{\pgfqpoint{2.058520in}{2.009403in}}{\pgfqpoint{2.047470in}{2.009403in}}%
\pgfpathcurveto{\pgfqpoint{2.036419in}{2.009403in}}{\pgfqpoint{2.025820in}{2.005013in}}{\pgfqpoint{2.018007in}{1.997199in}}%
\pgfpathcurveto{\pgfqpoint{2.010193in}{1.989385in}}{\pgfqpoint{2.005803in}{1.978786in}}{\pgfqpoint{2.005803in}{1.967736in}}%
\pgfpathcurveto{\pgfqpoint{2.005803in}{1.956686in}}{\pgfqpoint{2.010193in}{1.946087in}}{\pgfqpoint{2.018007in}{1.938273in}}%
\pgfpathcurveto{\pgfqpoint{2.025820in}{1.930460in}}{\pgfqpoint{2.036419in}{1.926070in}}{\pgfqpoint{2.047470in}{1.926070in}}%
\pgfpathclose%
\pgfusepath{stroke,fill}%
\end{pgfscope}%
\begin{pgfscope}%
\pgfpathrectangle{\pgfqpoint{0.600000in}{0.600000in}}{\pgfqpoint{3.900000in}{3.900000in}}%
\pgfusepath{clip}%
\pgfsetbuttcap%
\pgfsetroundjoin%
\definecolor{currentfill}{rgb}{0.121569,0.466667,0.705882}%
\pgfsetfillcolor{currentfill}%
\pgfsetlinewidth{1.003750pt}%
\definecolor{currentstroke}{rgb}{0.121569,0.466667,0.705882}%
\pgfsetstrokecolor{currentstroke}%
\pgfsetdash{}{0pt}%
\pgfpathmoveto{\pgfqpoint{2.113929in}{2.087791in}}%
\pgfpathcurveto{\pgfqpoint{2.124979in}{2.087791in}}{\pgfqpoint{2.135579in}{2.092181in}}{\pgfqpoint{2.143392in}{2.099994in}}%
\pgfpathcurveto{\pgfqpoint{2.151206in}{2.107808in}}{\pgfqpoint{2.155596in}{2.118407in}}{\pgfqpoint{2.155596in}{2.129457in}}%
\pgfpathcurveto{\pgfqpoint{2.155596in}{2.140507in}}{\pgfqpoint{2.151206in}{2.151106in}}{\pgfqpoint{2.143392in}{2.158920in}}%
\pgfpathcurveto{\pgfqpoint{2.135579in}{2.166734in}}{\pgfqpoint{2.124979in}{2.171124in}}{\pgfqpoint{2.113929in}{2.171124in}}%
\pgfpathcurveto{\pgfqpoint{2.102879in}{2.171124in}}{\pgfqpoint{2.092280in}{2.166734in}}{\pgfqpoint{2.084467in}{2.158920in}}%
\pgfpathcurveto{\pgfqpoint{2.076653in}{2.151106in}}{\pgfqpoint{2.072263in}{2.140507in}}{\pgfqpoint{2.072263in}{2.129457in}}%
\pgfpathcurveto{\pgfqpoint{2.072263in}{2.118407in}}{\pgfqpoint{2.076653in}{2.107808in}}{\pgfqpoint{2.084467in}{2.099994in}}%
\pgfpathcurveto{\pgfqpoint{2.092280in}{2.092181in}}{\pgfqpoint{2.102879in}{2.087791in}}{\pgfqpoint{2.113929in}{2.087791in}}%
\pgfpathclose%
\pgfusepath{stroke,fill}%
\end{pgfscope}%
\begin{pgfscope}%
\pgfpathrectangle{\pgfqpoint{0.600000in}{0.600000in}}{\pgfqpoint{3.900000in}{3.900000in}}%
\pgfusepath{clip}%
\pgfsetbuttcap%
\pgfsetroundjoin%
\definecolor{currentfill}{rgb}{0.121569,0.466667,0.705882}%
\pgfsetfillcolor{currentfill}%
\pgfsetlinewidth{1.003750pt}%
\definecolor{currentstroke}{rgb}{0.121569,0.466667,0.705882}%
\pgfsetstrokecolor{currentstroke}%
\pgfsetdash{}{0pt}%
\pgfpathmoveto{\pgfqpoint{2.877536in}{2.718031in}}%
\pgfpathcurveto{\pgfqpoint{2.888586in}{2.718031in}}{\pgfqpoint{2.899185in}{2.722421in}}{\pgfqpoint{2.906999in}{2.730235in}}%
\pgfpathcurveto{\pgfqpoint{2.914812in}{2.738048in}}{\pgfqpoint{2.919202in}{2.748648in}}{\pgfqpoint{2.919202in}{2.759698in}}%
\pgfpathcurveto{\pgfqpoint{2.919202in}{2.770748in}}{\pgfqpoint{2.914812in}{2.781347in}}{\pgfqpoint{2.906999in}{2.789160in}}%
\pgfpathcurveto{\pgfqpoint{2.899185in}{2.796974in}}{\pgfqpoint{2.888586in}{2.801364in}}{\pgfqpoint{2.877536in}{2.801364in}}%
\pgfpathcurveto{\pgfqpoint{2.866486in}{2.801364in}}{\pgfqpoint{2.855887in}{2.796974in}}{\pgfqpoint{2.848073in}{2.789160in}}%
\pgfpathcurveto{\pgfqpoint{2.840259in}{2.781347in}}{\pgfqpoint{2.835869in}{2.770748in}}{\pgfqpoint{2.835869in}{2.759698in}}%
\pgfpathcurveto{\pgfqpoint{2.835869in}{2.748648in}}{\pgfqpoint{2.840259in}{2.738048in}}{\pgfqpoint{2.848073in}{2.730235in}}%
\pgfpathcurveto{\pgfqpoint{2.855887in}{2.722421in}}{\pgfqpoint{2.866486in}{2.718031in}}{\pgfqpoint{2.877536in}{2.718031in}}%
\pgfpathclose%
\pgfusepath{stroke,fill}%
\end{pgfscope}%
\begin{pgfscope}%
\pgfpathrectangle{\pgfqpoint{0.600000in}{0.600000in}}{\pgfqpoint{3.900000in}{3.900000in}}%
\pgfusepath{clip}%
\pgfsetbuttcap%
\pgfsetroundjoin%
\definecolor{currentfill}{rgb}{0.121569,0.466667,0.705882}%
\pgfsetfillcolor{currentfill}%
\pgfsetlinewidth{1.003750pt}%
\definecolor{currentstroke}{rgb}{0.121569,0.466667,0.705882}%
\pgfsetstrokecolor{currentstroke}%
\pgfsetdash{}{0pt}%
\pgfpathmoveto{\pgfqpoint{2.633618in}{2.167733in}}%
\pgfpathcurveto{\pgfqpoint{2.644668in}{2.167733in}}{\pgfqpoint{2.655267in}{2.172123in}}{\pgfqpoint{2.663081in}{2.179936in}}%
\pgfpathcurveto{\pgfqpoint{2.670895in}{2.187750in}}{\pgfqpoint{2.675285in}{2.198349in}}{\pgfqpoint{2.675285in}{2.209399in}}%
\pgfpathcurveto{\pgfqpoint{2.675285in}{2.220449in}}{\pgfqpoint{2.670895in}{2.231048in}}{\pgfqpoint{2.663081in}{2.238862in}}%
\pgfpathcurveto{\pgfqpoint{2.655267in}{2.246676in}}{\pgfqpoint{2.644668in}{2.251066in}}{\pgfqpoint{2.633618in}{2.251066in}}%
\pgfpathcurveto{\pgfqpoint{2.622568in}{2.251066in}}{\pgfqpoint{2.611969in}{2.246676in}}{\pgfqpoint{2.604155in}{2.238862in}}%
\pgfpathcurveto{\pgfqpoint{2.596342in}{2.231048in}}{\pgfqpoint{2.591952in}{2.220449in}}{\pgfqpoint{2.591952in}{2.209399in}}%
\pgfpathcurveto{\pgfqpoint{2.591952in}{2.198349in}}{\pgfqpoint{2.596342in}{2.187750in}}{\pgfqpoint{2.604155in}{2.179936in}}%
\pgfpathcurveto{\pgfqpoint{2.611969in}{2.172123in}}{\pgfqpoint{2.622568in}{2.167733in}}{\pgfqpoint{2.633618in}{2.167733in}}%
\pgfpathclose%
\pgfusepath{stroke,fill}%
\end{pgfscope}%
\begin{pgfscope}%
\pgfpathrectangle{\pgfqpoint{0.600000in}{0.600000in}}{\pgfqpoint{3.900000in}{3.900000in}}%
\pgfusepath{clip}%
\pgfsetbuttcap%
\pgfsetroundjoin%
\definecolor{currentfill}{rgb}{0.121569,0.466667,0.705882}%
\pgfsetfillcolor{currentfill}%
\pgfsetlinewidth{1.003750pt}%
\definecolor{currentstroke}{rgb}{0.121569,0.466667,0.705882}%
\pgfsetstrokecolor{currentstroke}%
\pgfsetdash{}{0pt}%
\pgfpathmoveto{\pgfqpoint{2.749823in}{2.388910in}}%
\pgfpathcurveto{\pgfqpoint{2.760873in}{2.388910in}}{\pgfqpoint{2.771472in}{2.393300in}}{\pgfqpoint{2.779286in}{2.401114in}}%
\pgfpathcurveto{\pgfqpoint{2.787099in}{2.408927in}}{\pgfqpoint{2.791490in}{2.419526in}}{\pgfqpoint{2.791490in}{2.430576in}}%
\pgfpathcurveto{\pgfqpoint{2.791490in}{2.441626in}}{\pgfqpoint{2.787099in}{2.452225in}}{\pgfqpoint{2.779286in}{2.460039in}}%
\pgfpathcurveto{\pgfqpoint{2.771472in}{2.467853in}}{\pgfqpoint{2.760873in}{2.472243in}}{\pgfqpoint{2.749823in}{2.472243in}}%
\pgfpathcurveto{\pgfqpoint{2.738773in}{2.472243in}}{\pgfqpoint{2.728174in}{2.467853in}}{\pgfqpoint{2.720360in}{2.460039in}}%
\pgfpathcurveto{\pgfqpoint{2.712547in}{2.452225in}}{\pgfqpoint{2.708156in}{2.441626in}}{\pgfqpoint{2.708156in}{2.430576in}}%
\pgfpathcurveto{\pgfqpoint{2.708156in}{2.419526in}}{\pgfqpoint{2.712547in}{2.408927in}}{\pgfqpoint{2.720360in}{2.401114in}}%
\pgfpathcurveto{\pgfqpoint{2.728174in}{2.393300in}}{\pgfqpoint{2.738773in}{2.388910in}}{\pgfqpoint{2.749823in}{2.388910in}}%
\pgfpathclose%
\pgfusepath{stroke,fill}%
\end{pgfscope}%
\begin{pgfscope}%
\pgfpathrectangle{\pgfqpoint{0.600000in}{0.600000in}}{\pgfqpoint{3.900000in}{3.900000in}}%
\pgfusepath{clip}%
\pgfsetbuttcap%
\pgfsetroundjoin%
\definecolor{currentfill}{rgb}{0.121569,0.466667,0.705882}%
\pgfsetfillcolor{currentfill}%
\pgfsetlinewidth{1.003750pt}%
\definecolor{currentstroke}{rgb}{0.121569,0.466667,0.705882}%
\pgfsetstrokecolor{currentstroke}%
\pgfsetdash{}{0pt}%
\pgfpathmoveto{\pgfqpoint{2.472172in}{1.739492in}}%
\pgfpathcurveto{\pgfqpoint{2.483222in}{1.739492in}}{\pgfqpoint{2.493821in}{1.743882in}}{\pgfqpoint{2.501635in}{1.751696in}}%
\pgfpathcurveto{\pgfqpoint{2.509448in}{1.759509in}}{\pgfqpoint{2.513839in}{1.770108in}}{\pgfqpoint{2.513839in}{1.781158in}}%
\pgfpathcurveto{\pgfqpoint{2.513839in}{1.792209in}}{\pgfqpoint{2.509448in}{1.802808in}}{\pgfqpoint{2.501635in}{1.810621in}}%
\pgfpathcurveto{\pgfqpoint{2.493821in}{1.818435in}}{\pgfqpoint{2.483222in}{1.822825in}}{\pgfqpoint{2.472172in}{1.822825in}}%
\pgfpathcurveto{\pgfqpoint{2.461122in}{1.822825in}}{\pgfqpoint{2.450523in}{1.818435in}}{\pgfqpoint{2.442709in}{1.810621in}}%
\pgfpathcurveto{\pgfqpoint{2.434896in}{1.802808in}}{\pgfqpoint{2.430505in}{1.792209in}}{\pgfqpoint{2.430505in}{1.781158in}}%
\pgfpathcurveto{\pgfqpoint{2.430505in}{1.770108in}}{\pgfqpoint{2.434896in}{1.759509in}}{\pgfqpoint{2.442709in}{1.751696in}}%
\pgfpathcurveto{\pgfqpoint{2.450523in}{1.743882in}}{\pgfqpoint{2.461122in}{1.739492in}}{\pgfqpoint{2.472172in}{1.739492in}}%
\pgfpathclose%
\pgfusepath{stroke,fill}%
\end{pgfscope}%
\begin{pgfscope}%
\pgfpathrectangle{\pgfqpoint{0.600000in}{0.600000in}}{\pgfqpoint{3.900000in}{3.900000in}}%
\pgfusepath{clip}%
\pgfsetbuttcap%
\pgfsetroundjoin%
\definecolor{currentfill}{rgb}{0.121569,0.466667,0.705882}%
\pgfsetfillcolor{currentfill}%
\pgfsetlinewidth{1.003750pt}%
\definecolor{currentstroke}{rgb}{0.121569,0.466667,0.705882}%
\pgfsetstrokecolor{currentstroke}%
\pgfsetdash{}{0pt}%
\pgfpathmoveto{\pgfqpoint{2.798850in}{2.424136in}}%
\pgfpathcurveto{\pgfqpoint{2.809900in}{2.424136in}}{\pgfqpoint{2.820499in}{2.428527in}}{\pgfqpoint{2.828312in}{2.436340in}}%
\pgfpathcurveto{\pgfqpoint{2.836126in}{2.444154in}}{\pgfqpoint{2.840516in}{2.454753in}}{\pgfqpoint{2.840516in}{2.465803in}}%
\pgfpathcurveto{\pgfqpoint{2.840516in}{2.476853in}}{\pgfqpoint{2.836126in}{2.487452in}}{\pgfqpoint{2.828312in}{2.495266in}}%
\pgfpathcurveto{\pgfqpoint{2.820499in}{2.503079in}}{\pgfqpoint{2.809900in}{2.507470in}}{\pgfqpoint{2.798850in}{2.507470in}}%
\pgfpathcurveto{\pgfqpoint{2.787800in}{2.507470in}}{\pgfqpoint{2.777201in}{2.503079in}}{\pgfqpoint{2.769387in}{2.495266in}}%
\pgfpathcurveto{\pgfqpoint{2.761573in}{2.487452in}}{\pgfqpoint{2.757183in}{2.476853in}}{\pgfqpoint{2.757183in}{2.465803in}}%
\pgfpathcurveto{\pgfqpoint{2.757183in}{2.454753in}}{\pgfqpoint{2.761573in}{2.444154in}}{\pgfqpoint{2.769387in}{2.436340in}}%
\pgfpathcurveto{\pgfqpoint{2.777201in}{2.428527in}}{\pgfqpoint{2.787800in}{2.424136in}}{\pgfqpoint{2.798850in}{2.424136in}}%
\pgfpathclose%
\pgfusepath{stroke,fill}%
\end{pgfscope}%
\begin{pgfscope}%
\pgfpathrectangle{\pgfqpoint{0.600000in}{0.600000in}}{\pgfqpoint{3.900000in}{3.900000in}}%
\pgfusepath{clip}%
\pgfsetbuttcap%
\pgfsetroundjoin%
\definecolor{currentfill}{rgb}{0.121569,0.466667,0.705882}%
\pgfsetfillcolor{currentfill}%
\pgfsetlinewidth{1.003750pt}%
\definecolor{currentstroke}{rgb}{0.121569,0.466667,0.705882}%
\pgfsetstrokecolor{currentstroke}%
\pgfsetdash{}{0pt}%
\pgfpathmoveto{\pgfqpoint{2.443224in}{2.317170in}}%
\pgfpathcurveto{\pgfqpoint{2.454274in}{2.317170in}}{\pgfqpoint{2.464873in}{2.321560in}}{\pgfqpoint{2.472687in}{2.329373in}}%
\pgfpathcurveto{\pgfqpoint{2.480500in}{2.337187in}}{\pgfqpoint{2.484891in}{2.347786in}}{\pgfqpoint{2.484891in}{2.358836in}}%
\pgfpathcurveto{\pgfqpoint{2.484891in}{2.369886in}}{\pgfqpoint{2.480500in}{2.380485in}}{\pgfqpoint{2.472687in}{2.388299in}}%
\pgfpathcurveto{\pgfqpoint{2.464873in}{2.396113in}}{\pgfqpoint{2.454274in}{2.400503in}}{\pgfqpoint{2.443224in}{2.400503in}}%
\pgfpathcurveto{\pgfqpoint{2.432174in}{2.400503in}}{\pgfqpoint{2.421575in}{2.396113in}}{\pgfqpoint{2.413761in}{2.388299in}}%
\pgfpathcurveto{\pgfqpoint{2.405948in}{2.380485in}}{\pgfqpoint{2.401557in}{2.369886in}}{\pgfqpoint{2.401557in}{2.358836in}}%
\pgfpathcurveto{\pgfqpoint{2.401557in}{2.347786in}}{\pgfqpoint{2.405948in}{2.337187in}}{\pgfqpoint{2.413761in}{2.329373in}}%
\pgfpathcurveto{\pgfqpoint{2.421575in}{2.321560in}}{\pgfqpoint{2.432174in}{2.317170in}}{\pgfqpoint{2.443224in}{2.317170in}}%
\pgfpathclose%
\pgfusepath{stroke,fill}%
\end{pgfscope}%
\begin{pgfscope}%
\pgfpathrectangle{\pgfqpoint{0.600000in}{0.600000in}}{\pgfqpoint{3.900000in}{3.900000in}}%
\pgfusepath{clip}%
\pgfsetbuttcap%
\pgfsetroundjoin%
\definecolor{currentfill}{rgb}{0.121569,0.466667,0.705882}%
\pgfsetfillcolor{currentfill}%
\pgfsetlinewidth{1.003750pt}%
\definecolor{currentstroke}{rgb}{0.121569,0.466667,0.705882}%
\pgfsetstrokecolor{currentstroke}%
\pgfsetdash{}{0pt}%
\pgfpathmoveto{\pgfqpoint{2.473318in}{2.991295in}}%
\pgfpathcurveto{\pgfqpoint{2.484369in}{2.991295in}}{\pgfqpoint{2.494968in}{2.995685in}}{\pgfqpoint{2.502781in}{3.003499in}}%
\pgfpathcurveto{\pgfqpoint{2.510595in}{3.011312in}}{\pgfqpoint{2.514985in}{3.021911in}}{\pgfqpoint{2.514985in}{3.032961in}}%
\pgfpathcurveto{\pgfqpoint{2.514985in}{3.044011in}}{\pgfqpoint{2.510595in}{3.054610in}}{\pgfqpoint{2.502781in}{3.062424in}}%
\pgfpathcurveto{\pgfqpoint{2.494968in}{3.070238in}}{\pgfqpoint{2.484369in}{3.074628in}}{\pgfqpoint{2.473318in}{3.074628in}}%
\pgfpathcurveto{\pgfqpoint{2.462268in}{3.074628in}}{\pgfqpoint{2.451669in}{3.070238in}}{\pgfqpoint{2.443856in}{3.062424in}}%
\pgfpathcurveto{\pgfqpoint{2.436042in}{3.054610in}}{\pgfqpoint{2.431652in}{3.044011in}}{\pgfqpoint{2.431652in}{3.032961in}}%
\pgfpathcurveto{\pgfqpoint{2.431652in}{3.021911in}}{\pgfqpoint{2.436042in}{3.011312in}}{\pgfqpoint{2.443856in}{3.003499in}}%
\pgfpathcurveto{\pgfqpoint{2.451669in}{2.995685in}}{\pgfqpoint{2.462268in}{2.991295in}}{\pgfqpoint{2.473318in}{2.991295in}}%
\pgfpathclose%
\pgfusepath{stroke,fill}%
\end{pgfscope}%
\begin{pgfscope}%
\pgfpathrectangle{\pgfqpoint{0.600000in}{0.600000in}}{\pgfqpoint{3.900000in}{3.900000in}}%
\pgfusepath{clip}%
\pgfsetbuttcap%
\pgfsetroundjoin%
\definecolor{currentfill}{rgb}{0.121569,0.466667,0.705882}%
\pgfsetfillcolor{currentfill}%
\pgfsetlinewidth{1.003750pt}%
\definecolor{currentstroke}{rgb}{0.121569,0.466667,0.705882}%
\pgfsetstrokecolor{currentstroke}%
\pgfsetdash{}{0pt}%
\pgfpathmoveto{\pgfqpoint{1.767157in}{1.533437in}}%
\pgfpathcurveto{\pgfqpoint{1.778207in}{1.533437in}}{\pgfqpoint{1.788806in}{1.537828in}}{\pgfqpoint{1.796620in}{1.545641in}}%
\pgfpathcurveto{\pgfqpoint{1.804434in}{1.553455in}}{\pgfqpoint{1.808824in}{1.564054in}}{\pgfqpoint{1.808824in}{1.575104in}}%
\pgfpathcurveto{\pgfqpoint{1.808824in}{1.586154in}}{\pgfqpoint{1.804434in}{1.596753in}}{\pgfqpoint{1.796620in}{1.604567in}}%
\pgfpathcurveto{\pgfqpoint{1.788806in}{1.612380in}}{\pgfqpoint{1.778207in}{1.616771in}}{\pgfqpoint{1.767157in}{1.616771in}}%
\pgfpathcurveto{\pgfqpoint{1.756107in}{1.616771in}}{\pgfqpoint{1.745508in}{1.612380in}}{\pgfqpoint{1.737694in}{1.604567in}}%
\pgfpathcurveto{\pgfqpoint{1.729881in}{1.596753in}}{\pgfqpoint{1.725490in}{1.586154in}}{\pgfqpoint{1.725490in}{1.575104in}}%
\pgfpathcurveto{\pgfqpoint{1.725490in}{1.564054in}}{\pgfqpoint{1.729881in}{1.553455in}}{\pgfqpoint{1.737694in}{1.545641in}}%
\pgfpathcurveto{\pgfqpoint{1.745508in}{1.537828in}}{\pgfqpoint{1.756107in}{1.533437in}}{\pgfqpoint{1.767157in}{1.533437in}}%
\pgfpathclose%
\pgfusepath{stroke,fill}%
\end{pgfscope}%
\begin{pgfscope}%
\pgfpathrectangle{\pgfqpoint{0.600000in}{0.600000in}}{\pgfqpoint{3.900000in}{3.900000in}}%
\pgfusepath{clip}%
\pgfsetbuttcap%
\pgfsetroundjoin%
\definecolor{currentfill}{rgb}{0.121569,0.466667,0.705882}%
\pgfsetfillcolor{currentfill}%
\pgfsetlinewidth{1.003750pt}%
\definecolor{currentstroke}{rgb}{0.121569,0.466667,0.705882}%
\pgfsetstrokecolor{currentstroke}%
\pgfsetdash{}{0pt}%
\pgfpathmoveto{\pgfqpoint{2.355360in}{2.716992in}}%
\pgfpathcurveto{\pgfqpoint{2.366410in}{2.716992in}}{\pgfqpoint{2.377009in}{2.721383in}}{\pgfqpoint{2.384822in}{2.729196in}}%
\pgfpathcurveto{\pgfqpoint{2.392636in}{2.737010in}}{\pgfqpoint{2.397026in}{2.747609in}}{\pgfqpoint{2.397026in}{2.758659in}}%
\pgfpathcurveto{\pgfqpoint{2.397026in}{2.769709in}}{\pgfqpoint{2.392636in}{2.780308in}}{\pgfqpoint{2.384822in}{2.788122in}}%
\pgfpathcurveto{\pgfqpoint{2.377009in}{2.795935in}}{\pgfqpoint{2.366410in}{2.800326in}}{\pgfqpoint{2.355360in}{2.800326in}}%
\pgfpathcurveto{\pgfqpoint{2.344309in}{2.800326in}}{\pgfqpoint{2.333710in}{2.795935in}}{\pgfqpoint{2.325897in}{2.788122in}}%
\pgfpathcurveto{\pgfqpoint{2.318083in}{2.780308in}}{\pgfqpoint{2.313693in}{2.769709in}}{\pgfqpoint{2.313693in}{2.758659in}}%
\pgfpathcurveto{\pgfqpoint{2.313693in}{2.747609in}}{\pgfqpoint{2.318083in}{2.737010in}}{\pgfqpoint{2.325897in}{2.729196in}}%
\pgfpathcurveto{\pgfqpoint{2.333710in}{2.721383in}}{\pgfqpoint{2.344309in}{2.716992in}}{\pgfqpoint{2.355360in}{2.716992in}}%
\pgfpathclose%
\pgfusepath{stroke,fill}%
\end{pgfscope}%
\begin{pgfscope}%
\pgfpathrectangle{\pgfqpoint{0.600000in}{0.600000in}}{\pgfqpoint{3.900000in}{3.900000in}}%
\pgfusepath{clip}%
\pgfsetbuttcap%
\pgfsetroundjoin%
\definecolor{currentfill}{rgb}{0.121569,0.466667,0.705882}%
\pgfsetfillcolor{currentfill}%
\pgfsetlinewidth{1.003750pt}%
\definecolor{currentstroke}{rgb}{0.121569,0.466667,0.705882}%
\pgfsetstrokecolor{currentstroke}%
\pgfsetdash{}{0pt}%
\pgfpathmoveto{\pgfqpoint{1.798904in}{2.516407in}}%
\pgfpathcurveto{\pgfqpoint{1.809954in}{2.516407in}}{\pgfqpoint{1.820553in}{2.520798in}}{\pgfqpoint{1.828367in}{2.528611in}}%
\pgfpathcurveto{\pgfqpoint{1.836180in}{2.536425in}}{\pgfqpoint{1.840571in}{2.547024in}}{\pgfqpoint{1.840571in}{2.558074in}}%
\pgfpathcurveto{\pgfqpoint{1.840571in}{2.569124in}}{\pgfqpoint{1.836180in}{2.579723in}}{\pgfqpoint{1.828367in}{2.587537in}}%
\pgfpathcurveto{\pgfqpoint{1.820553in}{2.595350in}}{\pgfqpoint{1.809954in}{2.599741in}}{\pgfqpoint{1.798904in}{2.599741in}}%
\pgfpathcurveto{\pgfqpoint{1.787854in}{2.599741in}}{\pgfqpoint{1.777255in}{2.595350in}}{\pgfqpoint{1.769441in}{2.587537in}}%
\pgfpathcurveto{\pgfqpoint{1.761628in}{2.579723in}}{\pgfqpoint{1.757237in}{2.569124in}}{\pgfqpoint{1.757237in}{2.558074in}}%
\pgfpathcurveto{\pgfqpoint{1.757237in}{2.547024in}}{\pgfqpoint{1.761628in}{2.536425in}}{\pgfqpoint{1.769441in}{2.528611in}}%
\pgfpathcurveto{\pgfqpoint{1.777255in}{2.520798in}}{\pgfqpoint{1.787854in}{2.516407in}}{\pgfqpoint{1.798904in}{2.516407in}}%
\pgfpathclose%
\pgfusepath{stroke,fill}%
\end{pgfscope}%
\begin{pgfscope}%
\pgfpathrectangle{\pgfqpoint{0.600000in}{0.600000in}}{\pgfqpoint{3.900000in}{3.900000in}}%
\pgfusepath{clip}%
\pgfsetbuttcap%
\pgfsetroundjoin%
\definecolor{currentfill}{rgb}{0.121569,0.466667,0.705882}%
\pgfsetfillcolor{currentfill}%
\pgfsetlinewidth{1.003750pt}%
\definecolor{currentstroke}{rgb}{0.121569,0.466667,0.705882}%
\pgfsetstrokecolor{currentstroke}%
\pgfsetdash{}{0pt}%
\pgfpathmoveto{\pgfqpoint{1.631334in}{2.748692in}}%
\pgfpathcurveto{\pgfqpoint{1.642384in}{2.748692in}}{\pgfqpoint{1.652983in}{2.753082in}}{\pgfqpoint{1.660797in}{2.760896in}}%
\pgfpathcurveto{\pgfqpoint{1.668611in}{2.768710in}}{\pgfqpoint{1.673001in}{2.779309in}}{\pgfqpoint{1.673001in}{2.790359in}}%
\pgfpathcurveto{\pgfqpoint{1.673001in}{2.801409in}}{\pgfqpoint{1.668611in}{2.812008in}}{\pgfqpoint{1.660797in}{2.819822in}}%
\pgfpathcurveto{\pgfqpoint{1.652983in}{2.827635in}}{\pgfqpoint{1.642384in}{2.832025in}}{\pgfqpoint{1.631334in}{2.832025in}}%
\pgfpathcurveto{\pgfqpoint{1.620284in}{2.832025in}}{\pgfqpoint{1.609685in}{2.827635in}}{\pgfqpoint{1.601871in}{2.819822in}}%
\pgfpathcurveto{\pgfqpoint{1.594058in}{2.812008in}}{\pgfqpoint{1.589668in}{2.801409in}}{\pgfqpoint{1.589668in}{2.790359in}}%
\pgfpathcurveto{\pgfqpoint{1.589668in}{2.779309in}}{\pgfqpoint{1.594058in}{2.768710in}}{\pgfqpoint{1.601871in}{2.760896in}}%
\pgfpathcurveto{\pgfqpoint{1.609685in}{2.753082in}}{\pgfqpoint{1.620284in}{2.748692in}}{\pgfqpoint{1.631334in}{2.748692in}}%
\pgfpathclose%
\pgfusepath{stroke,fill}%
\end{pgfscope}%
\begin{pgfscope}%
\pgfpathrectangle{\pgfqpoint{0.600000in}{0.600000in}}{\pgfqpoint{3.900000in}{3.900000in}}%
\pgfusepath{clip}%
\pgfsetbuttcap%
\pgfsetroundjoin%
\definecolor{currentfill}{rgb}{0.121569,0.466667,0.705882}%
\pgfsetfillcolor{currentfill}%
\pgfsetlinewidth{1.003750pt}%
\definecolor{currentstroke}{rgb}{0.121569,0.466667,0.705882}%
\pgfsetstrokecolor{currentstroke}%
\pgfsetdash{}{0pt}%
\pgfpathmoveto{\pgfqpoint{2.323539in}{2.689161in}}%
\pgfpathcurveto{\pgfqpoint{2.334589in}{2.689161in}}{\pgfqpoint{2.345188in}{2.693551in}}{\pgfqpoint{2.353002in}{2.701365in}}%
\pgfpathcurveto{\pgfqpoint{2.360815in}{2.709178in}}{\pgfqpoint{2.365206in}{2.719777in}}{\pgfqpoint{2.365206in}{2.730827in}}%
\pgfpathcurveto{\pgfqpoint{2.365206in}{2.741878in}}{\pgfqpoint{2.360815in}{2.752477in}}{\pgfqpoint{2.353002in}{2.760290in}}%
\pgfpathcurveto{\pgfqpoint{2.345188in}{2.768104in}}{\pgfqpoint{2.334589in}{2.772494in}}{\pgfqpoint{2.323539in}{2.772494in}}%
\pgfpathcurveto{\pgfqpoint{2.312489in}{2.772494in}}{\pgfqpoint{2.301890in}{2.768104in}}{\pgfqpoint{2.294076in}{2.760290in}}%
\pgfpathcurveto{\pgfqpoint{2.286263in}{2.752477in}}{\pgfqpoint{2.281872in}{2.741878in}}{\pgfqpoint{2.281872in}{2.730827in}}%
\pgfpathcurveto{\pgfqpoint{2.281872in}{2.719777in}}{\pgfqpoint{2.286263in}{2.709178in}}{\pgfqpoint{2.294076in}{2.701365in}}%
\pgfpathcurveto{\pgfqpoint{2.301890in}{2.693551in}}{\pgfqpoint{2.312489in}{2.689161in}}{\pgfqpoint{2.323539in}{2.689161in}}%
\pgfpathclose%
\pgfusepath{stroke,fill}%
\end{pgfscope}%
\begin{pgfscope}%
\pgfpathrectangle{\pgfqpoint{0.600000in}{0.600000in}}{\pgfqpoint{3.900000in}{3.900000in}}%
\pgfusepath{clip}%
\pgfsetbuttcap%
\pgfsetroundjoin%
\definecolor{currentfill}{rgb}{0.121569,0.466667,0.705882}%
\pgfsetfillcolor{currentfill}%
\pgfsetlinewidth{1.003750pt}%
\definecolor{currentstroke}{rgb}{0.121569,0.466667,0.705882}%
\pgfsetstrokecolor{currentstroke}%
\pgfsetdash{}{0pt}%
\pgfpathmoveto{\pgfqpoint{3.641858in}{2.725739in}}%
\pgfpathcurveto{\pgfqpoint{3.652908in}{2.725739in}}{\pgfqpoint{3.663507in}{2.730130in}}{\pgfqpoint{3.671321in}{2.737943in}}%
\pgfpathcurveto{\pgfqpoint{3.679134in}{2.745757in}}{\pgfqpoint{3.683524in}{2.756356in}}{\pgfqpoint{3.683524in}{2.767406in}}%
\pgfpathcurveto{\pgfqpoint{3.683524in}{2.778456in}}{\pgfqpoint{3.679134in}{2.789055in}}{\pgfqpoint{3.671321in}{2.796869in}}%
\pgfpathcurveto{\pgfqpoint{3.663507in}{2.804682in}}{\pgfqpoint{3.652908in}{2.809073in}}{\pgfqpoint{3.641858in}{2.809073in}}%
\pgfpathcurveto{\pgfqpoint{3.630808in}{2.809073in}}{\pgfqpoint{3.620209in}{2.804682in}}{\pgfqpoint{3.612395in}{2.796869in}}%
\pgfpathcurveto{\pgfqpoint{3.604581in}{2.789055in}}{\pgfqpoint{3.600191in}{2.778456in}}{\pgfqpoint{3.600191in}{2.767406in}}%
\pgfpathcurveto{\pgfqpoint{3.600191in}{2.756356in}}{\pgfqpoint{3.604581in}{2.745757in}}{\pgfqpoint{3.612395in}{2.737943in}}%
\pgfpathcurveto{\pgfqpoint{3.620209in}{2.730130in}}{\pgfqpoint{3.630808in}{2.725739in}}{\pgfqpoint{3.641858in}{2.725739in}}%
\pgfpathclose%
\pgfusepath{stroke,fill}%
\end{pgfscope}%
\begin{pgfscope}%
\pgfpathrectangle{\pgfqpoint{0.600000in}{0.600000in}}{\pgfqpoint{3.900000in}{3.900000in}}%
\pgfusepath{clip}%
\pgfsetbuttcap%
\pgfsetroundjoin%
\definecolor{currentfill}{rgb}{0.121569,0.466667,0.705882}%
\pgfsetfillcolor{currentfill}%
\pgfsetlinewidth{1.003750pt}%
\definecolor{currentstroke}{rgb}{0.121569,0.466667,0.705882}%
\pgfsetstrokecolor{currentstroke}%
\pgfsetdash{}{0pt}%
\pgfpathmoveto{\pgfqpoint{1.897351in}{2.699830in}}%
\pgfpathcurveto{\pgfqpoint{1.908402in}{2.699830in}}{\pgfqpoint{1.919001in}{2.704220in}}{\pgfqpoint{1.926814in}{2.712034in}}%
\pgfpathcurveto{\pgfqpoint{1.934628in}{2.719847in}}{\pgfqpoint{1.939018in}{2.730446in}}{\pgfqpoint{1.939018in}{2.741496in}}%
\pgfpathcurveto{\pgfqpoint{1.939018in}{2.752547in}}{\pgfqpoint{1.934628in}{2.763146in}}{\pgfqpoint{1.926814in}{2.770959in}}%
\pgfpathcurveto{\pgfqpoint{1.919001in}{2.778773in}}{\pgfqpoint{1.908402in}{2.783163in}}{\pgfqpoint{1.897351in}{2.783163in}}%
\pgfpathcurveto{\pgfqpoint{1.886301in}{2.783163in}}{\pgfqpoint{1.875702in}{2.778773in}}{\pgfqpoint{1.867889in}{2.770959in}}%
\pgfpathcurveto{\pgfqpoint{1.860075in}{2.763146in}}{\pgfqpoint{1.855685in}{2.752547in}}{\pgfqpoint{1.855685in}{2.741496in}}%
\pgfpathcurveto{\pgfqpoint{1.855685in}{2.730446in}}{\pgfqpoint{1.860075in}{2.719847in}}{\pgfqpoint{1.867889in}{2.712034in}}%
\pgfpathcurveto{\pgfqpoint{1.875702in}{2.704220in}}{\pgfqpoint{1.886301in}{2.699830in}}{\pgfqpoint{1.897351in}{2.699830in}}%
\pgfpathclose%
\pgfusepath{stroke,fill}%
\end{pgfscope}%
\begin{pgfscope}%
\pgfpathrectangle{\pgfqpoint{0.600000in}{0.600000in}}{\pgfqpoint{3.900000in}{3.900000in}}%
\pgfusepath{clip}%
\pgfsetbuttcap%
\pgfsetroundjoin%
\definecolor{currentfill}{rgb}{0.121569,0.466667,0.705882}%
\pgfsetfillcolor{currentfill}%
\pgfsetlinewidth{1.003750pt}%
\definecolor{currentstroke}{rgb}{0.121569,0.466667,0.705882}%
\pgfsetstrokecolor{currentstroke}%
\pgfsetdash{}{0pt}%
\pgfpathmoveto{\pgfqpoint{1.938440in}{2.703013in}}%
\pgfpathcurveto{\pgfqpoint{1.949490in}{2.703013in}}{\pgfqpoint{1.960089in}{2.707404in}}{\pgfqpoint{1.967902in}{2.715217in}}%
\pgfpathcurveto{\pgfqpoint{1.975716in}{2.723031in}}{\pgfqpoint{1.980106in}{2.733630in}}{\pgfqpoint{1.980106in}{2.744680in}}%
\pgfpathcurveto{\pgfqpoint{1.980106in}{2.755730in}}{\pgfqpoint{1.975716in}{2.766329in}}{\pgfqpoint{1.967902in}{2.774143in}}%
\pgfpathcurveto{\pgfqpoint{1.960089in}{2.781956in}}{\pgfqpoint{1.949490in}{2.786347in}}{\pgfqpoint{1.938440in}{2.786347in}}%
\pgfpathcurveto{\pgfqpoint{1.927390in}{2.786347in}}{\pgfqpoint{1.916791in}{2.781956in}}{\pgfqpoint{1.908977in}{2.774143in}}%
\pgfpathcurveto{\pgfqpoint{1.901163in}{2.766329in}}{\pgfqpoint{1.896773in}{2.755730in}}{\pgfqpoint{1.896773in}{2.744680in}}%
\pgfpathcurveto{\pgfqpoint{1.896773in}{2.733630in}}{\pgfqpoint{1.901163in}{2.723031in}}{\pgfqpoint{1.908977in}{2.715217in}}%
\pgfpathcurveto{\pgfqpoint{1.916791in}{2.707404in}}{\pgfqpoint{1.927390in}{2.703013in}}{\pgfqpoint{1.938440in}{2.703013in}}%
\pgfpathclose%
\pgfusepath{stroke,fill}%
\end{pgfscope}%
\begin{pgfscope}%
\pgfpathrectangle{\pgfqpoint{0.600000in}{0.600000in}}{\pgfqpoint{3.900000in}{3.900000in}}%
\pgfusepath{clip}%
\pgfsetbuttcap%
\pgfsetroundjoin%
\definecolor{currentfill}{rgb}{0.121569,0.466667,0.705882}%
\pgfsetfillcolor{currentfill}%
\pgfsetlinewidth{1.003750pt}%
\definecolor{currentstroke}{rgb}{0.121569,0.466667,0.705882}%
\pgfsetstrokecolor{currentstroke}%
\pgfsetdash{}{0pt}%
\pgfpathmoveto{\pgfqpoint{1.580400in}{2.055189in}}%
\pgfpathcurveto{\pgfqpoint{1.591450in}{2.055189in}}{\pgfqpoint{1.602049in}{2.059579in}}{\pgfqpoint{1.609863in}{2.067393in}}%
\pgfpathcurveto{\pgfqpoint{1.617676in}{2.075206in}}{\pgfqpoint{1.622067in}{2.085805in}}{\pgfqpoint{1.622067in}{2.096855in}}%
\pgfpathcurveto{\pgfqpoint{1.622067in}{2.107906in}}{\pgfqpoint{1.617676in}{2.118505in}}{\pgfqpoint{1.609863in}{2.126318in}}%
\pgfpathcurveto{\pgfqpoint{1.602049in}{2.134132in}}{\pgfqpoint{1.591450in}{2.138522in}}{\pgfqpoint{1.580400in}{2.138522in}}%
\pgfpathcurveto{\pgfqpoint{1.569350in}{2.138522in}}{\pgfqpoint{1.558751in}{2.134132in}}{\pgfqpoint{1.550937in}{2.126318in}}%
\pgfpathcurveto{\pgfqpoint{1.543124in}{2.118505in}}{\pgfqpoint{1.538733in}{2.107906in}}{\pgfqpoint{1.538733in}{2.096855in}}%
\pgfpathcurveto{\pgfqpoint{1.538733in}{2.085805in}}{\pgfqpoint{1.543124in}{2.075206in}}{\pgfqpoint{1.550937in}{2.067393in}}%
\pgfpathcurveto{\pgfqpoint{1.558751in}{2.059579in}}{\pgfqpoint{1.569350in}{2.055189in}}{\pgfqpoint{1.580400in}{2.055189in}}%
\pgfpathclose%
\pgfusepath{stroke,fill}%
\end{pgfscope}%
\begin{pgfscope}%
\pgfpathrectangle{\pgfqpoint{0.600000in}{0.600000in}}{\pgfqpoint{3.900000in}{3.900000in}}%
\pgfusepath{clip}%
\pgfsetbuttcap%
\pgfsetroundjoin%
\definecolor{currentfill}{rgb}{0.121569,0.466667,0.705882}%
\pgfsetfillcolor{currentfill}%
\pgfsetlinewidth{1.003750pt}%
\definecolor{currentstroke}{rgb}{0.121569,0.466667,0.705882}%
\pgfsetstrokecolor{currentstroke}%
\pgfsetdash{}{0pt}%
\pgfpathmoveto{\pgfqpoint{3.663099in}{2.966451in}}%
\pgfpathcurveto{\pgfqpoint{3.674149in}{2.966451in}}{\pgfqpoint{3.684748in}{2.970841in}}{\pgfqpoint{3.692561in}{2.978655in}}%
\pgfpathcurveto{\pgfqpoint{3.700375in}{2.986469in}}{\pgfqpoint{3.704765in}{2.997068in}}{\pgfqpoint{3.704765in}{3.008118in}}%
\pgfpathcurveto{\pgfqpoint{3.704765in}{3.019168in}}{\pgfqpoint{3.700375in}{3.029767in}}{\pgfqpoint{3.692561in}{3.037581in}}%
\pgfpathcurveto{\pgfqpoint{3.684748in}{3.045394in}}{\pgfqpoint{3.674149in}{3.049785in}}{\pgfqpoint{3.663099in}{3.049785in}}%
\pgfpathcurveto{\pgfqpoint{3.652048in}{3.049785in}}{\pgfqpoint{3.641449in}{3.045394in}}{\pgfqpoint{3.633636in}{3.037581in}}%
\pgfpathcurveto{\pgfqpoint{3.625822in}{3.029767in}}{\pgfqpoint{3.621432in}{3.019168in}}{\pgfqpoint{3.621432in}{3.008118in}}%
\pgfpathcurveto{\pgfqpoint{3.621432in}{2.997068in}}{\pgfqpoint{3.625822in}{2.986469in}}{\pgfqpoint{3.633636in}{2.978655in}}%
\pgfpathcurveto{\pgfqpoint{3.641449in}{2.970841in}}{\pgfqpoint{3.652048in}{2.966451in}}{\pgfqpoint{3.663099in}{2.966451in}}%
\pgfpathclose%
\pgfusepath{stroke,fill}%
\end{pgfscope}%
\begin{pgfscope}%
\pgfpathrectangle{\pgfqpoint{0.600000in}{0.600000in}}{\pgfqpoint{3.900000in}{3.900000in}}%
\pgfusepath{clip}%
\pgfsetbuttcap%
\pgfsetroundjoin%
\definecolor{currentfill}{rgb}{0.121569,0.466667,0.705882}%
\pgfsetfillcolor{currentfill}%
\pgfsetlinewidth{1.003750pt}%
\definecolor{currentstroke}{rgb}{0.121569,0.466667,0.705882}%
\pgfsetstrokecolor{currentstroke}%
\pgfsetdash{}{0pt}%
\pgfpathmoveto{\pgfqpoint{3.063888in}{2.675654in}}%
\pgfpathcurveto{\pgfqpoint{3.074938in}{2.675654in}}{\pgfqpoint{3.085538in}{2.680044in}}{\pgfqpoint{3.093351in}{2.687858in}}%
\pgfpathcurveto{\pgfqpoint{3.101165in}{2.695671in}}{\pgfqpoint{3.105555in}{2.706270in}}{\pgfqpoint{3.105555in}{2.717320in}}%
\pgfpathcurveto{\pgfqpoint{3.105555in}{2.728370in}}{\pgfqpoint{3.101165in}{2.738969in}}{\pgfqpoint{3.093351in}{2.746783in}}%
\pgfpathcurveto{\pgfqpoint{3.085538in}{2.754597in}}{\pgfqpoint{3.074938in}{2.758987in}}{\pgfqpoint{3.063888in}{2.758987in}}%
\pgfpathcurveto{\pgfqpoint{3.052838in}{2.758987in}}{\pgfqpoint{3.042239in}{2.754597in}}{\pgfqpoint{3.034426in}{2.746783in}}%
\pgfpathcurveto{\pgfqpoint{3.026612in}{2.738969in}}{\pgfqpoint{3.022222in}{2.728370in}}{\pgfqpoint{3.022222in}{2.717320in}}%
\pgfpathcurveto{\pgfqpoint{3.022222in}{2.706270in}}{\pgfqpoint{3.026612in}{2.695671in}}{\pgfqpoint{3.034426in}{2.687858in}}%
\pgfpathcurveto{\pgfqpoint{3.042239in}{2.680044in}}{\pgfqpoint{3.052838in}{2.675654in}}{\pgfqpoint{3.063888in}{2.675654in}}%
\pgfpathclose%
\pgfusepath{stroke,fill}%
\end{pgfscope}%
\begin{pgfscope}%
\pgfpathrectangle{\pgfqpoint{0.600000in}{0.600000in}}{\pgfqpoint{3.900000in}{3.900000in}}%
\pgfusepath{clip}%
\pgfsetbuttcap%
\pgfsetroundjoin%
\definecolor{currentfill}{rgb}{0.121569,0.466667,0.705882}%
\pgfsetfillcolor{currentfill}%
\pgfsetlinewidth{1.003750pt}%
\definecolor{currentstroke}{rgb}{0.121569,0.466667,0.705882}%
\pgfsetstrokecolor{currentstroke}%
\pgfsetdash{}{0pt}%
\pgfpathmoveto{\pgfqpoint{2.485315in}{1.699251in}}%
\pgfpathcurveto{\pgfqpoint{2.496365in}{1.699251in}}{\pgfqpoint{2.506964in}{1.703641in}}{\pgfqpoint{2.514777in}{1.711455in}}%
\pgfpathcurveto{\pgfqpoint{2.522591in}{1.719268in}}{\pgfqpoint{2.526981in}{1.729867in}}{\pgfqpoint{2.526981in}{1.740917in}}%
\pgfpathcurveto{\pgfqpoint{2.526981in}{1.751967in}}{\pgfqpoint{2.522591in}{1.762567in}}{\pgfqpoint{2.514777in}{1.770380in}}%
\pgfpathcurveto{\pgfqpoint{2.506964in}{1.778194in}}{\pgfqpoint{2.496365in}{1.782584in}}{\pgfqpoint{2.485315in}{1.782584in}}%
\pgfpathcurveto{\pgfqpoint{2.474264in}{1.782584in}}{\pgfqpoint{2.463665in}{1.778194in}}{\pgfqpoint{2.455852in}{1.770380in}}%
\pgfpathcurveto{\pgfqpoint{2.448038in}{1.762567in}}{\pgfqpoint{2.443648in}{1.751967in}}{\pgfqpoint{2.443648in}{1.740917in}}%
\pgfpathcurveto{\pgfqpoint{2.443648in}{1.729867in}}{\pgfqpoint{2.448038in}{1.719268in}}{\pgfqpoint{2.455852in}{1.711455in}}%
\pgfpathcurveto{\pgfqpoint{2.463665in}{1.703641in}}{\pgfqpoint{2.474264in}{1.699251in}}{\pgfqpoint{2.485315in}{1.699251in}}%
\pgfpathclose%
\pgfusepath{stroke,fill}%
\end{pgfscope}%
\begin{pgfscope}%
\pgfpathrectangle{\pgfqpoint{0.600000in}{0.600000in}}{\pgfqpoint{3.900000in}{3.900000in}}%
\pgfusepath{clip}%
\pgfsetbuttcap%
\pgfsetroundjoin%
\definecolor{currentfill}{rgb}{0.121569,0.466667,0.705882}%
\pgfsetfillcolor{currentfill}%
\pgfsetlinewidth{1.003750pt}%
\definecolor{currentstroke}{rgb}{0.121569,0.466667,0.705882}%
\pgfsetstrokecolor{currentstroke}%
\pgfsetdash{}{0pt}%
\pgfpathmoveto{\pgfqpoint{2.364221in}{1.319184in}}%
\pgfpathcurveto{\pgfqpoint{2.375271in}{1.319184in}}{\pgfqpoint{2.385870in}{1.323574in}}{\pgfqpoint{2.393684in}{1.331388in}}%
\pgfpathcurveto{\pgfqpoint{2.401497in}{1.339201in}}{\pgfqpoint{2.405887in}{1.349800in}}{\pgfqpoint{2.405887in}{1.360850in}}%
\pgfpathcurveto{\pgfqpoint{2.405887in}{1.371901in}}{\pgfqpoint{2.401497in}{1.382500in}}{\pgfqpoint{2.393684in}{1.390313in}}%
\pgfpathcurveto{\pgfqpoint{2.385870in}{1.398127in}}{\pgfqpoint{2.375271in}{1.402517in}}{\pgfqpoint{2.364221in}{1.402517in}}%
\pgfpathcurveto{\pgfqpoint{2.353171in}{1.402517in}}{\pgfqpoint{2.342572in}{1.398127in}}{\pgfqpoint{2.334758in}{1.390313in}}%
\pgfpathcurveto{\pgfqpoint{2.326944in}{1.382500in}}{\pgfqpoint{2.322554in}{1.371901in}}{\pgfqpoint{2.322554in}{1.360850in}}%
\pgfpathcurveto{\pgfqpoint{2.322554in}{1.349800in}}{\pgfqpoint{2.326944in}{1.339201in}}{\pgfqpoint{2.334758in}{1.331388in}}%
\pgfpathcurveto{\pgfqpoint{2.342572in}{1.323574in}}{\pgfqpoint{2.353171in}{1.319184in}}{\pgfqpoint{2.364221in}{1.319184in}}%
\pgfpathclose%
\pgfusepath{stroke,fill}%
\end{pgfscope}%
\begin{pgfscope}%
\pgfpathrectangle{\pgfqpoint{0.600000in}{0.600000in}}{\pgfqpoint{3.900000in}{3.900000in}}%
\pgfusepath{clip}%
\pgfsetbuttcap%
\pgfsetroundjoin%
\definecolor{currentfill}{rgb}{0.121569,0.466667,0.705882}%
\pgfsetfillcolor{currentfill}%
\pgfsetlinewidth{1.003750pt}%
\definecolor{currentstroke}{rgb}{0.121569,0.466667,0.705882}%
\pgfsetstrokecolor{currentstroke}%
\pgfsetdash{}{0pt}%
\pgfpathmoveto{\pgfqpoint{1.947295in}{2.482852in}}%
\pgfpathcurveto{\pgfqpoint{1.958345in}{2.482852in}}{\pgfqpoint{1.968944in}{2.487242in}}{\pgfqpoint{1.976758in}{2.495056in}}%
\pgfpathcurveto{\pgfqpoint{1.984572in}{2.502870in}}{\pgfqpoint{1.988962in}{2.513469in}}{\pgfqpoint{1.988962in}{2.524519in}}%
\pgfpathcurveto{\pgfqpoint{1.988962in}{2.535569in}}{\pgfqpoint{1.984572in}{2.546168in}}{\pgfqpoint{1.976758in}{2.553982in}}%
\pgfpathcurveto{\pgfqpoint{1.968944in}{2.561795in}}{\pgfqpoint{1.958345in}{2.566185in}}{\pgfqpoint{1.947295in}{2.566185in}}%
\pgfpathcurveto{\pgfqpoint{1.936245in}{2.566185in}}{\pgfqpoint{1.925646in}{2.561795in}}{\pgfqpoint{1.917832in}{2.553982in}}%
\pgfpathcurveto{\pgfqpoint{1.910019in}{2.546168in}}{\pgfqpoint{1.905629in}{2.535569in}}{\pgfqpoint{1.905629in}{2.524519in}}%
\pgfpathcurveto{\pgfqpoint{1.905629in}{2.513469in}}{\pgfqpoint{1.910019in}{2.502870in}}{\pgfqpoint{1.917832in}{2.495056in}}%
\pgfpathcurveto{\pgfqpoint{1.925646in}{2.487242in}}{\pgfqpoint{1.936245in}{2.482852in}}{\pgfqpoint{1.947295in}{2.482852in}}%
\pgfpathclose%
\pgfusepath{stroke,fill}%
\end{pgfscope}%
\begin{pgfscope}%
\pgfpathrectangle{\pgfqpoint{0.600000in}{0.600000in}}{\pgfqpoint{3.900000in}{3.900000in}}%
\pgfusepath{clip}%
\pgfsetbuttcap%
\pgfsetroundjoin%
\definecolor{currentfill}{rgb}{0.121569,0.466667,0.705882}%
\pgfsetfillcolor{currentfill}%
\pgfsetlinewidth{1.003750pt}%
\definecolor{currentstroke}{rgb}{0.121569,0.466667,0.705882}%
\pgfsetstrokecolor{currentstroke}%
\pgfsetdash{}{0pt}%
\pgfpathmoveto{\pgfqpoint{2.048375in}{2.400249in}}%
\pgfpathcurveto{\pgfqpoint{2.059425in}{2.400249in}}{\pgfqpoint{2.070024in}{2.404640in}}{\pgfqpoint{2.077837in}{2.412453in}}%
\pgfpathcurveto{\pgfqpoint{2.085651in}{2.420267in}}{\pgfqpoint{2.090041in}{2.430866in}}{\pgfqpoint{2.090041in}{2.441916in}}%
\pgfpathcurveto{\pgfqpoint{2.090041in}{2.452966in}}{\pgfqpoint{2.085651in}{2.463565in}}{\pgfqpoint{2.077837in}{2.471379in}}%
\pgfpathcurveto{\pgfqpoint{2.070024in}{2.479192in}}{\pgfqpoint{2.059425in}{2.483583in}}{\pgfqpoint{2.048375in}{2.483583in}}%
\pgfpathcurveto{\pgfqpoint{2.037324in}{2.483583in}}{\pgfqpoint{2.026725in}{2.479192in}}{\pgfqpoint{2.018912in}{2.471379in}}%
\pgfpathcurveto{\pgfqpoint{2.011098in}{2.463565in}}{\pgfqpoint{2.006708in}{2.452966in}}{\pgfqpoint{2.006708in}{2.441916in}}%
\pgfpathcurveto{\pgfqpoint{2.006708in}{2.430866in}}{\pgfqpoint{2.011098in}{2.420267in}}{\pgfqpoint{2.018912in}{2.412453in}}%
\pgfpathcurveto{\pgfqpoint{2.026725in}{2.404640in}}{\pgfqpoint{2.037324in}{2.400249in}}{\pgfqpoint{2.048375in}{2.400249in}}%
\pgfpathclose%
\pgfusepath{stroke,fill}%
\end{pgfscope}%
\begin{pgfscope}%
\pgfpathrectangle{\pgfqpoint{0.600000in}{0.600000in}}{\pgfqpoint{3.900000in}{3.900000in}}%
\pgfusepath{clip}%
\pgfsetbuttcap%
\pgfsetroundjoin%
\definecolor{currentfill}{rgb}{0.121569,0.466667,0.705882}%
\pgfsetfillcolor{currentfill}%
\pgfsetlinewidth{1.003750pt}%
\definecolor{currentstroke}{rgb}{0.121569,0.466667,0.705882}%
\pgfsetstrokecolor{currentstroke}%
\pgfsetdash{}{0pt}%
\pgfpathmoveto{\pgfqpoint{2.405716in}{3.564827in}}%
\pgfpathcurveto{\pgfqpoint{2.416766in}{3.564827in}}{\pgfqpoint{2.427365in}{3.569217in}}{\pgfqpoint{2.435179in}{3.577031in}}%
\pgfpathcurveto{\pgfqpoint{2.442992in}{3.584844in}}{\pgfqpoint{2.447383in}{3.595443in}}{\pgfqpoint{2.447383in}{3.606493in}}%
\pgfpathcurveto{\pgfqpoint{2.447383in}{3.617544in}}{\pgfqpoint{2.442992in}{3.628143in}}{\pgfqpoint{2.435179in}{3.635956in}}%
\pgfpathcurveto{\pgfqpoint{2.427365in}{3.643770in}}{\pgfqpoint{2.416766in}{3.648160in}}{\pgfqpoint{2.405716in}{3.648160in}}%
\pgfpathcurveto{\pgfqpoint{2.394666in}{3.648160in}}{\pgfqpoint{2.384067in}{3.643770in}}{\pgfqpoint{2.376253in}{3.635956in}}%
\pgfpathcurveto{\pgfqpoint{2.368440in}{3.628143in}}{\pgfqpoint{2.364049in}{3.617544in}}{\pgfqpoint{2.364049in}{3.606493in}}%
\pgfpathcurveto{\pgfqpoint{2.364049in}{3.595443in}}{\pgfqpoint{2.368440in}{3.584844in}}{\pgfqpoint{2.376253in}{3.577031in}}%
\pgfpathcurveto{\pgfqpoint{2.384067in}{3.569217in}}{\pgfqpoint{2.394666in}{3.564827in}}{\pgfqpoint{2.405716in}{3.564827in}}%
\pgfpathclose%
\pgfusepath{stroke,fill}%
\end{pgfscope}%
\begin{pgfscope}%
\pgfpathrectangle{\pgfqpoint{0.600000in}{0.600000in}}{\pgfqpoint{3.900000in}{3.900000in}}%
\pgfusepath{clip}%
\pgfsetbuttcap%
\pgfsetroundjoin%
\definecolor{currentfill}{rgb}{0.121569,0.466667,0.705882}%
\pgfsetfillcolor{currentfill}%
\pgfsetlinewidth{1.003750pt}%
\definecolor{currentstroke}{rgb}{0.121569,0.466667,0.705882}%
\pgfsetstrokecolor{currentstroke}%
\pgfsetdash{}{0pt}%
\pgfpathmoveto{\pgfqpoint{2.315443in}{3.376432in}}%
\pgfpathcurveto{\pgfqpoint{2.326493in}{3.376432in}}{\pgfqpoint{2.337092in}{3.380822in}}{\pgfqpoint{2.344905in}{3.388636in}}%
\pgfpathcurveto{\pgfqpoint{2.352719in}{3.396449in}}{\pgfqpoint{2.357109in}{3.407048in}}{\pgfqpoint{2.357109in}{3.418099in}}%
\pgfpathcurveto{\pgfqpoint{2.357109in}{3.429149in}}{\pgfqpoint{2.352719in}{3.439748in}}{\pgfqpoint{2.344905in}{3.447561in}}%
\pgfpathcurveto{\pgfqpoint{2.337092in}{3.455375in}}{\pgfqpoint{2.326493in}{3.459765in}}{\pgfqpoint{2.315443in}{3.459765in}}%
\pgfpathcurveto{\pgfqpoint{2.304393in}{3.459765in}}{\pgfqpoint{2.293793in}{3.455375in}}{\pgfqpoint{2.285980in}{3.447561in}}%
\pgfpathcurveto{\pgfqpoint{2.278166in}{3.439748in}}{\pgfqpoint{2.273776in}{3.429149in}}{\pgfqpoint{2.273776in}{3.418099in}}%
\pgfpathcurveto{\pgfqpoint{2.273776in}{3.407048in}}{\pgfqpoint{2.278166in}{3.396449in}}{\pgfqpoint{2.285980in}{3.388636in}}%
\pgfpathcurveto{\pgfqpoint{2.293793in}{3.380822in}}{\pgfqpoint{2.304393in}{3.376432in}}{\pgfqpoint{2.315443in}{3.376432in}}%
\pgfpathclose%
\pgfusepath{stroke,fill}%
\end{pgfscope}%
\begin{pgfscope}%
\pgfpathrectangle{\pgfqpoint{0.600000in}{0.600000in}}{\pgfqpoint{3.900000in}{3.900000in}}%
\pgfusepath{clip}%
\pgfsetbuttcap%
\pgfsetroundjoin%
\definecolor{currentfill}{rgb}{0.121569,0.466667,0.705882}%
\pgfsetfillcolor{currentfill}%
\pgfsetlinewidth{1.003750pt}%
\definecolor{currentstroke}{rgb}{0.121569,0.466667,0.705882}%
\pgfsetstrokecolor{currentstroke}%
\pgfsetdash{}{0pt}%
\pgfpathmoveto{\pgfqpoint{2.521294in}{2.542640in}}%
\pgfpathcurveto{\pgfqpoint{2.532344in}{2.542640in}}{\pgfqpoint{2.542943in}{2.547030in}}{\pgfqpoint{2.550756in}{2.554844in}}%
\pgfpathcurveto{\pgfqpoint{2.558570in}{2.562657in}}{\pgfqpoint{2.562960in}{2.573256in}}{\pgfqpoint{2.562960in}{2.584306in}}%
\pgfpathcurveto{\pgfqpoint{2.562960in}{2.595356in}}{\pgfqpoint{2.558570in}{2.605955in}}{\pgfqpoint{2.550756in}{2.613769in}}%
\pgfpathcurveto{\pgfqpoint{2.542943in}{2.621583in}}{\pgfqpoint{2.532344in}{2.625973in}}{\pgfqpoint{2.521294in}{2.625973in}}%
\pgfpathcurveto{\pgfqpoint{2.510243in}{2.625973in}}{\pgfqpoint{2.499644in}{2.621583in}}{\pgfqpoint{2.491831in}{2.613769in}}%
\pgfpathcurveto{\pgfqpoint{2.484017in}{2.605955in}}{\pgfqpoint{2.479627in}{2.595356in}}{\pgfqpoint{2.479627in}{2.584306in}}%
\pgfpathcurveto{\pgfqpoint{2.479627in}{2.573256in}}{\pgfqpoint{2.484017in}{2.562657in}}{\pgfqpoint{2.491831in}{2.554844in}}%
\pgfpathcurveto{\pgfqpoint{2.499644in}{2.547030in}}{\pgfqpoint{2.510243in}{2.542640in}}{\pgfqpoint{2.521294in}{2.542640in}}%
\pgfpathclose%
\pgfusepath{stroke,fill}%
\end{pgfscope}%
\begin{pgfscope}%
\pgfpathrectangle{\pgfqpoint{0.600000in}{0.600000in}}{\pgfqpoint{3.900000in}{3.900000in}}%
\pgfusepath{clip}%
\pgfsetbuttcap%
\pgfsetroundjoin%
\definecolor{currentfill}{rgb}{0.121569,0.466667,0.705882}%
\pgfsetfillcolor{currentfill}%
\pgfsetlinewidth{1.003750pt}%
\definecolor{currentstroke}{rgb}{0.121569,0.466667,0.705882}%
\pgfsetstrokecolor{currentstroke}%
\pgfsetdash{}{0pt}%
\pgfpathmoveto{\pgfqpoint{2.710846in}{2.199142in}}%
\pgfpathcurveto{\pgfqpoint{2.721897in}{2.199142in}}{\pgfqpoint{2.732496in}{2.203532in}}{\pgfqpoint{2.740309in}{2.211346in}}%
\pgfpathcurveto{\pgfqpoint{2.748123in}{2.219160in}}{\pgfqpoint{2.752513in}{2.229759in}}{\pgfqpoint{2.752513in}{2.240809in}}%
\pgfpathcurveto{\pgfqpoint{2.752513in}{2.251859in}}{\pgfqpoint{2.748123in}{2.262458in}}{\pgfqpoint{2.740309in}{2.270272in}}%
\pgfpathcurveto{\pgfqpoint{2.732496in}{2.278085in}}{\pgfqpoint{2.721897in}{2.282475in}}{\pgfqpoint{2.710846in}{2.282475in}}%
\pgfpathcurveto{\pgfqpoint{2.699796in}{2.282475in}}{\pgfqpoint{2.689197in}{2.278085in}}{\pgfqpoint{2.681384in}{2.270272in}}%
\pgfpathcurveto{\pgfqpoint{2.673570in}{2.262458in}}{\pgfqpoint{2.669180in}{2.251859in}}{\pgfqpoint{2.669180in}{2.240809in}}%
\pgfpathcurveto{\pgfqpoint{2.669180in}{2.229759in}}{\pgfqpoint{2.673570in}{2.219160in}}{\pgfqpoint{2.681384in}{2.211346in}}%
\pgfpathcurveto{\pgfqpoint{2.689197in}{2.203532in}}{\pgfqpoint{2.699796in}{2.199142in}}{\pgfqpoint{2.710846in}{2.199142in}}%
\pgfpathclose%
\pgfusepath{stroke,fill}%
\end{pgfscope}%
\begin{pgfscope}%
\pgfpathrectangle{\pgfqpoint{0.600000in}{0.600000in}}{\pgfqpoint{3.900000in}{3.900000in}}%
\pgfusepath{clip}%
\pgfsetbuttcap%
\pgfsetroundjoin%
\definecolor{currentfill}{rgb}{0.121569,0.466667,0.705882}%
\pgfsetfillcolor{currentfill}%
\pgfsetlinewidth{1.003750pt}%
\definecolor{currentstroke}{rgb}{0.121569,0.466667,0.705882}%
\pgfsetstrokecolor{currentstroke}%
\pgfsetdash{}{0pt}%
\pgfpathmoveto{\pgfqpoint{1.945023in}{4.151712in}}%
\pgfpathcurveto{\pgfqpoint{1.956073in}{4.151712in}}{\pgfqpoint{1.966672in}{4.156102in}}{\pgfqpoint{1.974486in}{4.163916in}}%
\pgfpathcurveto{\pgfqpoint{1.982299in}{4.171730in}}{\pgfqpoint{1.986690in}{4.182329in}}{\pgfqpoint{1.986690in}{4.193379in}}%
\pgfpathcurveto{\pgfqpoint{1.986690in}{4.204429in}}{\pgfqpoint{1.982299in}{4.215028in}}{\pgfqpoint{1.974486in}{4.222842in}}%
\pgfpathcurveto{\pgfqpoint{1.966672in}{4.230655in}}{\pgfqpoint{1.956073in}{4.235046in}}{\pgfqpoint{1.945023in}{4.235046in}}%
\pgfpathcurveto{\pgfqpoint{1.933973in}{4.235046in}}{\pgfqpoint{1.923374in}{4.230655in}}{\pgfqpoint{1.915560in}{4.222842in}}%
\pgfpathcurveto{\pgfqpoint{1.907746in}{4.215028in}}{\pgfqpoint{1.903356in}{4.204429in}}{\pgfqpoint{1.903356in}{4.193379in}}%
\pgfpathcurveto{\pgfqpoint{1.903356in}{4.182329in}}{\pgfqpoint{1.907746in}{4.171730in}}{\pgfqpoint{1.915560in}{4.163916in}}%
\pgfpathcurveto{\pgfqpoint{1.923374in}{4.156102in}}{\pgfqpoint{1.933973in}{4.151712in}}{\pgfqpoint{1.945023in}{4.151712in}}%
\pgfpathclose%
\pgfusepath{stroke,fill}%
\end{pgfscope}%
\begin{pgfscope}%
\pgfpathrectangle{\pgfqpoint{0.600000in}{0.600000in}}{\pgfqpoint{3.900000in}{3.900000in}}%
\pgfusepath{clip}%
\pgfsetbuttcap%
\pgfsetroundjoin%
\definecolor{currentfill}{rgb}{0.121569,0.466667,0.705882}%
\pgfsetfillcolor{currentfill}%
\pgfsetlinewidth{1.003750pt}%
\definecolor{currentstroke}{rgb}{0.121569,0.466667,0.705882}%
\pgfsetstrokecolor{currentstroke}%
\pgfsetdash{}{0pt}%
\pgfpathmoveto{\pgfqpoint{3.115384in}{2.040518in}}%
\pgfpathcurveto{\pgfqpoint{3.126435in}{2.040518in}}{\pgfqpoint{3.137034in}{2.044908in}}{\pgfqpoint{3.144847in}{2.052722in}}%
\pgfpathcurveto{\pgfqpoint{3.152661in}{2.060535in}}{\pgfqpoint{3.157051in}{2.071134in}}{\pgfqpoint{3.157051in}{2.082184in}}%
\pgfpathcurveto{\pgfqpoint{3.157051in}{2.093235in}}{\pgfqpoint{3.152661in}{2.103834in}}{\pgfqpoint{3.144847in}{2.111647in}}%
\pgfpathcurveto{\pgfqpoint{3.137034in}{2.119461in}}{\pgfqpoint{3.126435in}{2.123851in}}{\pgfqpoint{3.115384in}{2.123851in}}%
\pgfpathcurveto{\pgfqpoint{3.104334in}{2.123851in}}{\pgfqpoint{3.093735in}{2.119461in}}{\pgfqpoint{3.085922in}{2.111647in}}%
\pgfpathcurveto{\pgfqpoint{3.078108in}{2.103834in}}{\pgfqpoint{3.073718in}{2.093235in}}{\pgfqpoint{3.073718in}{2.082184in}}%
\pgfpathcurveto{\pgfqpoint{3.073718in}{2.071134in}}{\pgfqpoint{3.078108in}{2.060535in}}{\pgfqpoint{3.085922in}{2.052722in}}%
\pgfpathcurveto{\pgfqpoint{3.093735in}{2.044908in}}{\pgfqpoint{3.104334in}{2.040518in}}{\pgfqpoint{3.115384in}{2.040518in}}%
\pgfpathclose%
\pgfusepath{stroke,fill}%
\end{pgfscope}%
\begin{pgfscope}%
\pgfpathrectangle{\pgfqpoint{0.600000in}{0.600000in}}{\pgfqpoint{3.900000in}{3.900000in}}%
\pgfusepath{clip}%
\pgfsetbuttcap%
\pgfsetroundjoin%
\definecolor{currentfill}{rgb}{0.121569,0.466667,0.705882}%
\pgfsetfillcolor{currentfill}%
\pgfsetlinewidth{1.003750pt}%
\definecolor{currentstroke}{rgb}{0.121569,0.466667,0.705882}%
\pgfsetstrokecolor{currentstroke}%
\pgfsetdash{}{0pt}%
\pgfpathmoveto{\pgfqpoint{2.929895in}{2.960483in}}%
\pgfpathcurveto{\pgfqpoint{2.940945in}{2.960483in}}{\pgfqpoint{2.951544in}{2.964873in}}{\pgfqpoint{2.959358in}{2.972687in}}%
\pgfpathcurveto{\pgfqpoint{2.967172in}{2.980501in}}{\pgfqpoint{2.971562in}{2.991100in}}{\pgfqpoint{2.971562in}{3.002150in}}%
\pgfpathcurveto{\pgfqpoint{2.971562in}{3.013200in}}{\pgfqpoint{2.967172in}{3.023799in}}{\pgfqpoint{2.959358in}{3.031613in}}%
\pgfpathcurveto{\pgfqpoint{2.951544in}{3.039426in}}{\pgfqpoint{2.940945in}{3.043817in}}{\pgfqpoint{2.929895in}{3.043817in}}%
\pgfpathcurveto{\pgfqpoint{2.918845in}{3.043817in}}{\pgfqpoint{2.908246in}{3.039426in}}{\pgfqpoint{2.900433in}{3.031613in}}%
\pgfpathcurveto{\pgfqpoint{2.892619in}{3.023799in}}{\pgfqpoint{2.888229in}{3.013200in}}{\pgfqpoint{2.888229in}{3.002150in}}%
\pgfpathcurveto{\pgfqpoint{2.888229in}{2.991100in}}{\pgfqpoint{2.892619in}{2.980501in}}{\pgfqpoint{2.900433in}{2.972687in}}%
\pgfpathcurveto{\pgfqpoint{2.908246in}{2.964873in}}{\pgfqpoint{2.918845in}{2.960483in}}{\pgfqpoint{2.929895in}{2.960483in}}%
\pgfpathclose%
\pgfusepath{stroke,fill}%
\end{pgfscope}%
\begin{pgfscope}%
\pgfpathrectangle{\pgfqpoint{0.600000in}{0.600000in}}{\pgfqpoint{3.900000in}{3.900000in}}%
\pgfusepath{clip}%
\pgfsetbuttcap%
\pgfsetroundjoin%
\definecolor{currentfill}{rgb}{0.121569,0.466667,0.705882}%
\pgfsetfillcolor{currentfill}%
\pgfsetlinewidth{1.003750pt}%
\definecolor{currentstroke}{rgb}{0.121569,0.466667,0.705882}%
\pgfsetstrokecolor{currentstroke}%
\pgfsetdash{}{0pt}%
\pgfpathmoveto{\pgfqpoint{3.237358in}{3.186259in}}%
\pgfpathcurveto{\pgfqpoint{3.248408in}{3.186259in}}{\pgfqpoint{3.259007in}{3.190650in}}{\pgfqpoint{3.266820in}{3.198463in}}%
\pgfpathcurveto{\pgfqpoint{3.274634in}{3.206277in}}{\pgfqpoint{3.279024in}{3.216876in}}{\pgfqpoint{3.279024in}{3.227926in}}%
\pgfpathcurveto{\pgfqpoint{3.279024in}{3.238976in}}{\pgfqpoint{3.274634in}{3.249575in}}{\pgfqpoint{3.266820in}{3.257389in}}%
\pgfpathcurveto{\pgfqpoint{3.259007in}{3.265202in}}{\pgfqpoint{3.248408in}{3.269593in}}{\pgfqpoint{3.237358in}{3.269593in}}%
\pgfpathcurveto{\pgfqpoint{3.226307in}{3.269593in}}{\pgfqpoint{3.215708in}{3.265202in}}{\pgfqpoint{3.207895in}{3.257389in}}%
\pgfpathcurveto{\pgfqpoint{3.200081in}{3.249575in}}{\pgfqpoint{3.195691in}{3.238976in}}{\pgfqpoint{3.195691in}{3.227926in}}%
\pgfpathcurveto{\pgfqpoint{3.195691in}{3.216876in}}{\pgfqpoint{3.200081in}{3.206277in}}{\pgfqpoint{3.207895in}{3.198463in}}%
\pgfpathcurveto{\pgfqpoint{3.215708in}{3.190650in}}{\pgfqpoint{3.226307in}{3.186259in}}{\pgfqpoint{3.237358in}{3.186259in}}%
\pgfpathclose%
\pgfusepath{stroke,fill}%
\end{pgfscope}%
\begin{pgfscope}%
\pgfpathrectangle{\pgfqpoint{0.600000in}{0.600000in}}{\pgfqpoint{3.900000in}{3.900000in}}%
\pgfusepath{clip}%
\pgfsetbuttcap%
\pgfsetroundjoin%
\definecolor{currentfill}{rgb}{0.121569,0.466667,0.705882}%
\pgfsetfillcolor{currentfill}%
\pgfsetlinewidth{1.003750pt}%
\definecolor{currentstroke}{rgb}{0.121569,0.466667,0.705882}%
\pgfsetstrokecolor{currentstroke}%
\pgfsetdash{}{0pt}%
\pgfpathmoveto{\pgfqpoint{2.378451in}{2.077515in}}%
\pgfpathcurveto{\pgfqpoint{2.389501in}{2.077515in}}{\pgfqpoint{2.400100in}{2.081906in}}{\pgfqpoint{2.407914in}{2.089719in}}%
\pgfpathcurveto{\pgfqpoint{2.415727in}{2.097533in}}{\pgfqpoint{2.420117in}{2.108132in}}{\pgfqpoint{2.420117in}{2.119182in}}%
\pgfpathcurveto{\pgfqpoint{2.420117in}{2.130232in}}{\pgfqpoint{2.415727in}{2.140831in}}{\pgfqpoint{2.407914in}{2.148645in}}%
\pgfpathcurveto{\pgfqpoint{2.400100in}{2.156459in}}{\pgfqpoint{2.389501in}{2.160849in}}{\pgfqpoint{2.378451in}{2.160849in}}%
\pgfpathcurveto{\pgfqpoint{2.367401in}{2.160849in}}{\pgfqpoint{2.356802in}{2.156459in}}{\pgfqpoint{2.348988in}{2.148645in}}%
\pgfpathcurveto{\pgfqpoint{2.341174in}{2.140831in}}{\pgfqpoint{2.336784in}{2.130232in}}{\pgfqpoint{2.336784in}{2.119182in}}%
\pgfpathcurveto{\pgfqpoint{2.336784in}{2.108132in}}{\pgfqpoint{2.341174in}{2.097533in}}{\pgfqpoint{2.348988in}{2.089719in}}%
\pgfpathcurveto{\pgfqpoint{2.356802in}{2.081906in}}{\pgfqpoint{2.367401in}{2.077515in}}{\pgfqpoint{2.378451in}{2.077515in}}%
\pgfpathclose%
\pgfusepath{stroke,fill}%
\end{pgfscope}%
\begin{pgfscope}%
\pgfpathrectangle{\pgfqpoint{0.600000in}{0.600000in}}{\pgfqpoint{3.900000in}{3.900000in}}%
\pgfusepath{clip}%
\pgfsetbuttcap%
\pgfsetroundjoin%
\definecolor{currentfill}{rgb}{0.121569,0.466667,0.705882}%
\pgfsetfillcolor{currentfill}%
\pgfsetlinewidth{1.003750pt}%
\definecolor{currentstroke}{rgb}{0.121569,0.466667,0.705882}%
\pgfsetstrokecolor{currentstroke}%
\pgfsetdash{}{0pt}%
\pgfpathmoveto{\pgfqpoint{3.903117in}{1.881328in}}%
\pgfpathcurveto{\pgfqpoint{3.914167in}{1.881328in}}{\pgfqpoint{3.924766in}{1.885718in}}{\pgfqpoint{3.932580in}{1.893532in}}%
\pgfpathcurveto{\pgfqpoint{3.940394in}{1.901345in}}{\pgfqpoint{3.944784in}{1.911944in}}{\pgfqpoint{3.944784in}{1.922995in}}%
\pgfpathcurveto{\pgfqpoint{3.944784in}{1.934045in}}{\pgfqpoint{3.940394in}{1.944644in}}{\pgfqpoint{3.932580in}{1.952457in}}%
\pgfpathcurveto{\pgfqpoint{3.924766in}{1.960271in}}{\pgfqpoint{3.914167in}{1.964661in}}{\pgfqpoint{3.903117in}{1.964661in}}%
\pgfpathcurveto{\pgfqpoint{3.892067in}{1.964661in}}{\pgfqpoint{3.881468in}{1.960271in}}{\pgfqpoint{3.873654in}{1.952457in}}%
\pgfpathcurveto{\pgfqpoint{3.865841in}{1.944644in}}{\pgfqpoint{3.861451in}{1.934045in}}{\pgfqpoint{3.861451in}{1.922995in}}%
\pgfpathcurveto{\pgfqpoint{3.861451in}{1.911944in}}{\pgfqpoint{3.865841in}{1.901345in}}{\pgfqpoint{3.873654in}{1.893532in}}%
\pgfpathcurveto{\pgfqpoint{3.881468in}{1.885718in}}{\pgfqpoint{3.892067in}{1.881328in}}{\pgfqpoint{3.903117in}{1.881328in}}%
\pgfpathclose%
\pgfusepath{stroke,fill}%
\end{pgfscope}%
\begin{pgfscope}%
\pgfpathrectangle{\pgfqpoint{0.600000in}{0.600000in}}{\pgfqpoint{3.900000in}{3.900000in}}%
\pgfusepath{clip}%
\pgfsetbuttcap%
\pgfsetroundjoin%
\definecolor{currentfill}{rgb}{0.121569,0.466667,0.705882}%
\pgfsetfillcolor{currentfill}%
\pgfsetlinewidth{1.003750pt}%
\definecolor{currentstroke}{rgb}{0.121569,0.466667,0.705882}%
\pgfsetstrokecolor{currentstroke}%
\pgfsetdash{}{0pt}%
\pgfpathmoveto{\pgfqpoint{2.431344in}{1.743748in}}%
\pgfpathcurveto{\pgfqpoint{2.442394in}{1.743748in}}{\pgfqpoint{2.452993in}{1.748138in}}{\pgfqpoint{2.460807in}{1.755952in}}%
\pgfpathcurveto{\pgfqpoint{2.468620in}{1.763766in}}{\pgfqpoint{2.473011in}{1.774365in}}{\pgfqpoint{2.473011in}{1.785415in}}%
\pgfpathcurveto{\pgfqpoint{2.473011in}{1.796465in}}{\pgfqpoint{2.468620in}{1.807064in}}{\pgfqpoint{2.460807in}{1.814878in}}%
\pgfpathcurveto{\pgfqpoint{2.452993in}{1.822691in}}{\pgfqpoint{2.442394in}{1.827081in}}{\pgfqpoint{2.431344in}{1.827081in}}%
\pgfpathcurveto{\pgfqpoint{2.420294in}{1.827081in}}{\pgfqpoint{2.409695in}{1.822691in}}{\pgfqpoint{2.401881in}{1.814878in}}%
\pgfpathcurveto{\pgfqpoint{2.394068in}{1.807064in}}{\pgfqpoint{2.389677in}{1.796465in}}{\pgfqpoint{2.389677in}{1.785415in}}%
\pgfpathcurveto{\pgfqpoint{2.389677in}{1.774365in}}{\pgfqpoint{2.394068in}{1.763766in}}{\pgfqpoint{2.401881in}{1.755952in}}%
\pgfpathcurveto{\pgfqpoint{2.409695in}{1.748138in}}{\pgfqpoint{2.420294in}{1.743748in}}{\pgfqpoint{2.431344in}{1.743748in}}%
\pgfpathclose%
\pgfusepath{stroke,fill}%
\end{pgfscope}%
\begin{pgfscope}%
\pgfpathrectangle{\pgfqpoint{0.600000in}{0.600000in}}{\pgfqpoint{3.900000in}{3.900000in}}%
\pgfusepath{clip}%
\pgfsetbuttcap%
\pgfsetroundjoin%
\definecolor{currentfill}{rgb}{0.121569,0.466667,0.705882}%
\pgfsetfillcolor{currentfill}%
\pgfsetlinewidth{1.003750pt}%
\definecolor{currentstroke}{rgb}{0.121569,0.466667,0.705882}%
\pgfsetstrokecolor{currentstroke}%
\pgfsetdash{}{0pt}%
\pgfpathmoveto{\pgfqpoint{3.011068in}{2.126608in}}%
\pgfpathcurveto{\pgfqpoint{3.022119in}{2.126608in}}{\pgfqpoint{3.032718in}{2.130998in}}{\pgfqpoint{3.040531in}{2.138812in}}%
\pgfpathcurveto{\pgfqpoint{3.048345in}{2.146625in}}{\pgfqpoint{3.052735in}{2.157224in}}{\pgfqpoint{3.052735in}{2.168274in}}%
\pgfpathcurveto{\pgfqpoint{3.052735in}{2.179325in}}{\pgfqpoint{3.048345in}{2.189924in}}{\pgfqpoint{3.040531in}{2.197737in}}%
\pgfpathcurveto{\pgfqpoint{3.032718in}{2.205551in}}{\pgfqpoint{3.022119in}{2.209941in}}{\pgfqpoint{3.011068in}{2.209941in}}%
\pgfpathcurveto{\pgfqpoint{3.000018in}{2.209941in}}{\pgfqpoint{2.989419in}{2.205551in}}{\pgfqpoint{2.981606in}{2.197737in}}%
\pgfpathcurveto{\pgfqpoint{2.973792in}{2.189924in}}{\pgfqpoint{2.969402in}{2.179325in}}{\pgfqpoint{2.969402in}{2.168274in}}%
\pgfpathcurveto{\pgfqpoint{2.969402in}{2.157224in}}{\pgfqpoint{2.973792in}{2.146625in}}{\pgfqpoint{2.981606in}{2.138812in}}%
\pgfpathcurveto{\pgfqpoint{2.989419in}{2.130998in}}{\pgfqpoint{3.000018in}{2.126608in}}{\pgfqpoint{3.011068in}{2.126608in}}%
\pgfpathclose%
\pgfusepath{stroke,fill}%
\end{pgfscope}%
\begin{pgfscope}%
\pgfpathrectangle{\pgfqpoint{0.600000in}{0.600000in}}{\pgfqpoint{3.900000in}{3.900000in}}%
\pgfusepath{clip}%
\pgfsetbuttcap%
\pgfsetroundjoin%
\definecolor{currentfill}{rgb}{0.121569,0.466667,0.705882}%
\pgfsetfillcolor{currentfill}%
\pgfsetlinewidth{1.003750pt}%
\definecolor{currentstroke}{rgb}{0.121569,0.466667,0.705882}%
\pgfsetstrokecolor{currentstroke}%
\pgfsetdash{}{0pt}%
\pgfpathmoveto{\pgfqpoint{3.472733in}{3.346474in}}%
\pgfpathcurveto{\pgfqpoint{3.483783in}{3.346474in}}{\pgfqpoint{3.494382in}{3.350864in}}{\pgfqpoint{3.502196in}{3.358678in}}%
\pgfpathcurveto{\pgfqpoint{3.510009in}{3.366492in}}{\pgfqpoint{3.514400in}{3.377091in}}{\pgfqpoint{3.514400in}{3.388141in}}%
\pgfpathcurveto{\pgfqpoint{3.514400in}{3.399191in}}{\pgfqpoint{3.510009in}{3.409790in}}{\pgfqpoint{3.502196in}{3.417604in}}%
\pgfpathcurveto{\pgfqpoint{3.494382in}{3.425417in}}{\pgfqpoint{3.483783in}{3.429807in}}{\pgfqpoint{3.472733in}{3.429807in}}%
\pgfpathcurveto{\pgfqpoint{3.461683in}{3.429807in}}{\pgfqpoint{3.451084in}{3.425417in}}{\pgfqpoint{3.443270in}{3.417604in}}%
\pgfpathcurveto{\pgfqpoint{3.435457in}{3.409790in}}{\pgfqpoint{3.431066in}{3.399191in}}{\pgfqpoint{3.431066in}{3.388141in}}%
\pgfpathcurveto{\pgfqpoint{3.431066in}{3.377091in}}{\pgfqpoint{3.435457in}{3.366492in}}{\pgfqpoint{3.443270in}{3.358678in}}%
\pgfpathcurveto{\pgfqpoint{3.451084in}{3.350864in}}{\pgfqpoint{3.461683in}{3.346474in}}{\pgfqpoint{3.472733in}{3.346474in}}%
\pgfpathclose%
\pgfusepath{stroke,fill}%
\end{pgfscope}%
\begin{pgfscope}%
\pgfpathrectangle{\pgfqpoint{0.600000in}{0.600000in}}{\pgfqpoint{3.900000in}{3.900000in}}%
\pgfusepath{clip}%
\pgfsetbuttcap%
\pgfsetroundjoin%
\definecolor{currentfill}{rgb}{0.121569,0.466667,0.705882}%
\pgfsetfillcolor{currentfill}%
\pgfsetlinewidth{1.003750pt}%
\definecolor{currentstroke}{rgb}{0.121569,0.466667,0.705882}%
\pgfsetstrokecolor{currentstroke}%
\pgfsetdash{}{0pt}%
\pgfpathmoveto{\pgfqpoint{2.732208in}{2.282830in}}%
\pgfpathcurveto{\pgfqpoint{2.743258in}{2.282830in}}{\pgfqpoint{2.753857in}{2.287220in}}{\pgfqpoint{2.761670in}{2.295034in}}%
\pgfpathcurveto{\pgfqpoint{2.769484in}{2.302847in}}{\pgfqpoint{2.773874in}{2.313446in}}{\pgfqpoint{2.773874in}{2.324496in}}%
\pgfpathcurveto{\pgfqpoint{2.773874in}{2.335547in}}{\pgfqpoint{2.769484in}{2.346146in}}{\pgfqpoint{2.761670in}{2.353959in}}%
\pgfpathcurveto{\pgfqpoint{2.753857in}{2.361773in}}{\pgfqpoint{2.743258in}{2.366163in}}{\pgfqpoint{2.732208in}{2.366163in}}%
\pgfpathcurveto{\pgfqpoint{2.721157in}{2.366163in}}{\pgfqpoint{2.710558in}{2.361773in}}{\pgfqpoint{2.702745in}{2.353959in}}%
\pgfpathcurveto{\pgfqpoint{2.694931in}{2.346146in}}{\pgfqpoint{2.690541in}{2.335547in}}{\pgfqpoint{2.690541in}{2.324496in}}%
\pgfpathcurveto{\pgfqpoint{2.690541in}{2.313446in}}{\pgfqpoint{2.694931in}{2.302847in}}{\pgfqpoint{2.702745in}{2.295034in}}%
\pgfpathcurveto{\pgfqpoint{2.710558in}{2.287220in}}{\pgfqpoint{2.721157in}{2.282830in}}{\pgfqpoint{2.732208in}{2.282830in}}%
\pgfpathclose%
\pgfusepath{stroke,fill}%
\end{pgfscope}%
\begin{pgfscope}%
\pgfpathrectangle{\pgfqpoint{0.600000in}{0.600000in}}{\pgfqpoint{3.900000in}{3.900000in}}%
\pgfusepath{clip}%
\pgfsetbuttcap%
\pgfsetroundjoin%
\definecolor{currentfill}{rgb}{0.121569,0.466667,0.705882}%
\pgfsetfillcolor{currentfill}%
\pgfsetlinewidth{1.003750pt}%
\definecolor{currentstroke}{rgb}{0.121569,0.466667,0.705882}%
\pgfsetstrokecolor{currentstroke}%
\pgfsetdash{}{0pt}%
\pgfpathmoveto{\pgfqpoint{2.648343in}{1.683935in}}%
\pgfpathcurveto{\pgfqpoint{2.659393in}{1.683935in}}{\pgfqpoint{2.669992in}{1.688325in}}{\pgfqpoint{2.677806in}{1.696139in}}%
\pgfpathcurveto{\pgfqpoint{2.685619in}{1.703952in}}{\pgfqpoint{2.690009in}{1.714551in}}{\pgfqpoint{2.690009in}{1.725601in}}%
\pgfpathcurveto{\pgfqpoint{2.690009in}{1.736651in}}{\pgfqpoint{2.685619in}{1.747250in}}{\pgfqpoint{2.677806in}{1.755064in}}%
\pgfpathcurveto{\pgfqpoint{2.669992in}{1.762878in}}{\pgfqpoint{2.659393in}{1.767268in}}{\pgfqpoint{2.648343in}{1.767268in}}%
\pgfpathcurveto{\pgfqpoint{2.637293in}{1.767268in}}{\pgfqpoint{2.626694in}{1.762878in}}{\pgfqpoint{2.618880in}{1.755064in}}%
\pgfpathcurveto{\pgfqpoint{2.611066in}{1.747250in}}{\pgfqpoint{2.606676in}{1.736651in}}{\pgfqpoint{2.606676in}{1.725601in}}%
\pgfpathcurveto{\pgfqpoint{2.606676in}{1.714551in}}{\pgfqpoint{2.611066in}{1.703952in}}{\pgfqpoint{2.618880in}{1.696139in}}%
\pgfpathcurveto{\pgfqpoint{2.626694in}{1.688325in}}{\pgfqpoint{2.637293in}{1.683935in}}{\pgfqpoint{2.648343in}{1.683935in}}%
\pgfpathclose%
\pgfusepath{stroke,fill}%
\end{pgfscope}%
\begin{pgfscope}%
\pgfpathrectangle{\pgfqpoint{0.600000in}{0.600000in}}{\pgfqpoint{3.900000in}{3.900000in}}%
\pgfusepath{clip}%
\pgfsetbuttcap%
\pgfsetroundjoin%
\definecolor{currentfill}{rgb}{0.121569,0.466667,0.705882}%
\pgfsetfillcolor{currentfill}%
\pgfsetlinewidth{1.003750pt}%
\definecolor{currentstroke}{rgb}{0.121569,0.466667,0.705882}%
\pgfsetstrokecolor{currentstroke}%
\pgfsetdash{}{0pt}%
\pgfpathmoveto{\pgfqpoint{2.341440in}{2.402060in}}%
\pgfpathcurveto{\pgfqpoint{2.352490in}{2.402060in}}{\pgfqpoint{2.363089in}{2.406450in}}{\pgfqpoint{2.370902in}{2.414263in}}%
\pgfpathcurveto{\pgfqpoint{2.378716in}{2.422077in}}{\pgfqpoint{2.383106in}{2.432676in}}{\pgfqpoint{2.383106in}{2.443726in}}%
\pgfpathcurveto{\pgfqpoint{2.383106in}{2.454776in}}{\pgfqpoint{2.378716in}{2.465375in}}{\pgfqpoint{2.370902in}{2.473189in}}%
\pgfpathcurveto{\pgfqpoint{2.363089in}{2.481003in}}{\pgfqpoint{2.352490in}{2.485393in}}{\pgfqpoint{2.341440in}{2.485393in}}%
\pgfpathcurveto{\pgfqpoint{2.330390in}{2.485393in}}{\pgfqpoint{2.319791in}{2.481003in}}{\pgfqpoint{2.311977in}{2.473189in}}%
\pgfpathcurveto{\pgfqpoint{2.304163in}{2.465375in}}{\pgfqpoint{2.299773in}{2.454776in}}{\pgfqpoint{2.299773in}{2.443726in}}%
\pgfpathcurveto{\pgfqpoint{2.299773in}{2.432676in}}{\pgfqpoint{2.304163in}{2.422077in}}{\pgfqpoint{2.311977in}{2.414263in}}%
\pgfpathcurveto{\pgfqpoint{2.319791in}{2.406450in}}{\pgfqpoint{2.330390in}{2.402060in}}{\pgfqpoint{2.341440in}{2.402060in}}%
\pgfpathclose%
\pgfusepath{stroke,fill}%
\end{pgfscope}%
\begin{pgfscope}%
\pgfpathrectangle{\pgfqpoint{0.600000in}{0.600000in}}{\pgfqpoint{3.900000in}{3.900000in}}%
\pgfusepath{clip}%
\pgfsetbuttcap%
\pgfsetroundjoin%
\definecolor{currentfill}{rgb}{0.121569,0.466667,0.705882}%
\pgfsetfillcolor{currentfill}%
\pgfsetlinewidth{1.003750pt}%
\definecolor{currentstroke}{rgb}{0.121569,0.466667,0.705882}%
\pgfsetstrokecolor{currentstroke}%
\pgfsetdash{}{0pt}%
\pgfpathmoveto{\pgfqpoint{2.871915in}{1.672130in}}%
\pgfpathcurveto{\pgfqpoint{2.882965in}{1.672130in}}{\pgfqpoint{2.893564in}{1.676520in}}{\pgfqpoint{2.901378in}{1.684334in}}%
\pgfpathcurveto{\pgfqpoint{2.909192in}{1.692147in}}{\pgfqpoint{2.913582in}{1.702746in}}{\pgfqpoint{2.913582in}{1.713796in}}%
\pgfpathcurveto{\pgfqpoint{2.913582in}{1.724847in}}{\pgfqpoint{2.909192in}{1.735446in}}{\pgfqpoint{2.901378in}{1.743259in}}%
\pgfpathcurveto{\pgfqpoint{2.893564in}{1.751073in}}{\pgfqpoint{2.882965in}{1.755463in}}{\pgfqpoint{2.871915in}{1.755463in}}%
\pgfpathcurveto{\pgfqpoint{2.860865in}{1.755463in}}{\pgfqpoint{2.850266in}{1.751073in}}{\pgfqpoint{2.842453in}{1.743259in}}%
\pgfpathcurveto{\pgfqpoint{2.834639in}{1.735446in}}{\pgfqpoint{2.830249in}{1.724847in}}{\pgfqpoint{2.830249in}{1.713796in}}%
\pgfpathcurveto{\pgfqpoint{2.830249in}{1.702746in}}{\pgfqpoint{2.834639in}{1.692147in}}{\pgfqpoint{2.842453in}{1.684334in}}%
\pgfpathcurveto{\pgfqpoint{2.850266in}{1.676520in}}{\pgfqpoint{2.860865in}{1.672130in}}{\pgfqpoint{2.871915in}{1.672130in}}%
\pgfpathclose%
\pgfusepath{stroke,fill}%
\end{pgfscope}%
\begin{pgfscope}%
\pgfpathrectangle{\pgfqpoint{0.600000in}{0.600000in}}{\pgfqpoint{3.900000in}{3.900000in}}%
\pgfusepath{clip}%
\pgfsetbuttcap%
\pgfsetroundjoin%
\definecolor{currentfill}{rgb}{0.121569,0.466667,0.705882}%
\pgfsetfillcolor{currentfill}%
\pgfsetlinewidth{1.003750pt}%
\definecolor{currentstroke}{rgb}{0.121569,0.466667,0.705882}%
\pgfsetstrokecolor{currentstroke}%
\pgfsetdash{}{0pt}%
\pgfpathmoveto{\pgfqpoint{2.629398in}{1.959561in}}%
\pgfpathcurveto{\pgfqpoint{2.640449in}{1.959561in}}{\pgfqpoint{2.651048in}{1.963952in}}{\pgfqpoint{2.658861in}{1.971765in}}%
\pgfpathcurveto{\pgfqpoint{2.666675in}{1.979579in}}{\pgfqpoint{2.671065in}{1.990178in}}{\pgfqpoint{2.671065in}{2.001228in}}%
\pgfpathcurveto{\pgfqpoint{2.671065in}{2.012278in}}{\pgfqpoint{2.666675in}{2.022877in}}{\pgfqpoint{2.658861in}{2.030691in}}%
\pgfpathcurveto{\pgfqpoint{2.651048in}{2.038504in}}{\pgfqpoint{2.640449in}{2.042895in}}{\pgfqpoint{2.629398in}{2.042895in}}%
\pgfpathcurveto{\pgfqpoint{2.618348in}{2.042895in}}{\pgfqpoint{2.607749in}{2.038504in}}{\pgfqpoint{2.599936in}{2.030691in}}%
\pgfpathcurveto{\pgfqpoint{2.592122in}{2.022877in}}{\pgfqpoint{2.587732in}{2.012278in}}{\pgfqpoint{2.587732in}{2.001228in}}%
\pgfpathcurveto{\pgfqpoint{2.587732in}{1.990178in}}{\pgfqpoint{2.592122in}{1.979579in}}{\pgfqpoint{2.599936in}{1.971765in}}%
\pgfpathcurveto{\pgfqpoint{2.607749in}{1.963952in}}{\pgfqpoint{2.618348in}{1.959561in}}{\pgfqpoint{2.629398in}{1.959561in}}%
\pgfpathclose%
\pgfusepath{stroke,fill}%
\end{pgfscope}%
\begin{pgfscope}%
\pgfpathrectangle{\pgfqpoint{0.600000in}{0.600000in}}{\pgfqpoint{3.900000in}{3.900000in}}%
\pgfusepath{clip}%
\pgfsetbuttcap%
\pgfsetroundjoin%
\definecolor{currentfill}{rgb}{0.121569,0.466667,0.705882}%
\pgfsetfillcolor{currentfill}%
\pgfsetlinewidth{1.003750pt}%
\definecolor{currentstroke}{rgb}{0.121569,0.466667,0.705882}%
\pgfsetstrokecolor{currentstroke}%
\pgfsetdash{}{0pt}%
\pgfpathmoveto{\pgfqpoint{2.487699in}{1.833387in}}%
\pgfpathcurveto{\pgfqpoint{2.498749in}{1.833387in}}{\pgfqpoint{2.509348in}{1.837777in}}{\pgfqpoint{2.517162in}{1.845591in}}%
\pgfpathcurveto{\pgfqpoint{2.524976in}{1.853405in}}{\pgfqpoint{2.529366in}{1.864004in}}{\pgfqpoint{2.529366in}{1.875054in}}%
\pgfpathcurveto{\pgfqpoint{2.529366in}{1.886104in}}{\pgfqpoint{2.524976in}{1.896703in}}{\pgfqpoint{2.517162in}{1.904517in}}%
\pgfpathcurveto{\pgfqpoint{2.509348in}{1.912330in}}{\pgfqpoint{2.498749in}{1.916721in}}{\pgfqpoint{2.487699in}{1.916721in}}%
\pgfpathcurveto{\pgfqpoint{2.476649in}{1.916721in}}{\pgfqpoint{2.466050in}{1.912330in}}{\pgfqpoint{2.458236in}{1.904517in}}%
\pgfpathcurveto{\pgfqpoint{2.450423in}{1.896703in}}{\pgfqpoint{2.446032in}{1.886104in}}{\pgfqpoint{2.446032in}{1.875054in}}%
\pgfpathcurveto{\pgfqpoint{2.446032in}{1.864004in}}{\pgfqpoint{2.450423in}{1.853405in}}{\pgfqpoint{2.458236in}{1.845591in}}%
\pgfpathcurveto{\pgfqpoint{2.466050in}{1.837777in}}{\pgfqpoint{2.476649in}{1.833387in}}{\pgfqpoint{2.487699in}{1.833387in}}%
\pgfpathclose%
\pgfusepath{stroke,fill}%
\end{pgfscope}%
\begin{pgfscope}%
\pgfpathrectangle{\pgfqpoint{0.600000in}{0.600000in}}{\pgfqpoint{3.900000in}{3.900000in}}%
\pgfusepath{clip}%
\pgfsetbuttcap%
\pgfsetroundjoin%
\definecolor{currentfill}{rgb}{0.121569,0.466667,0.705882}%
\pgfsetfillcolor{currentfill}%
\pgfsetlinewidth{1.003750pt}%
\definecolor{currentstroke}{rgb}{0.121569,0.466667,0.705882}%
\pgfsetstrokecolor{currentstroke}%
\pgfsetdash{}{0pt}%
\pgfpathmoveto{\pgfqpoint{2.412838in}{1.701823in}}%
\pgfpathcurveto{\pgfqpoint{2.423888in}{1.701823in}}{\pgfqpoint{2.434487in}{1.706213in}}{\pgfqpoint{2.442300in}{1.714027in}}%
\pgfpathcurveto{\pgfqpoint{2.450114in}{1.721840in}}{\pgfqpoint{2.454504in}{1.732439in}}{\pgfqpoint{2.454504in}{1.743489in}}%
\pgfpathcurveto{\pgfqpoint{2.454504in}{1.754539in}}{\pgfqpoint{2.450114in}{1.765139in}}{\pgfqpoint{2.442300in}{1.772952in}}%
\pgfpathcurveto{\pgfqpoint{2.434487in}{1.780766in}}{\pgfqpoint{2.423888in}{1.785156in}}{\pgfqpoint{2.412838in}{1.785156in}}%
\pgfpathcurveto{\pgfqpoint{2.401788in}{1.785156in}}{\pgfqpoint{2.391189in}{1.780766in}}{\pgfqpoint{2.383375in}{1.772952in}}%
\pgfpathcurveto{\pgfqpoint{2.375561in}{1.765139in}}{\pgfqpoint{2.371171in}{1.754539in}}{\pgfqpoint{2.371171in}{1.743489in}}%
\pgfpathcurveto{\pgfqpoint{2.371171in}{1.732439in}}{\pgfqpoint{2.375561in}{1.721840in}}{\pgfqpoint{2.383375in}{1.714027in}}%
\pgfpathcurveto{\pgfqpoint{2.391189in}{1.706213in}}{\pgfqpoint{2.401788in}{1.701823in}}{\pgfqpoint{2.412838in}{1.701823in}}%
\pgfpathclose%
\pgfusepath{stroke,fill}%
\end{pgfscope}%
\begin{pgfscope}%
\pgfpathrectangle{\pgfqpoint{0.600000in}{0.600000in}}{\pgfqpoint{3.900000in}{3.900000in}}%
\pgfusepath{clip}%
\pgfsetbuttcap%
\pgfsetroundjoin%
\definecolor{currentfill}{rgb}{0.121569,0.466667,0.705882}%
\pgfsetfillcolor{currentfill}%
\pgfsetlinewidth{1.003750pt}%
\definecolor{currentstroke}{rgb}{0.121569,0.466667,0.705882}%
\pgfsetstrokecolor{currentstroke}%
\pgfsetdash{}{0pt}%
\pgfpathmoveto{\pgfqpoint{3.096027in}{2.661023in}}%
\pgfpathcurveto{\pgfqpoint{3.107077in}{2.661023in}}{\pgfqpoint{3.117676in}{2.665413in}}{\pgfqpoint{3.125490in}{2.673227in}}%
\pgfpathcurveto{\pgfqpoint{3.133303in}{2.681040in}}{\pgfqpoint{3.137694in}{2.691640in}}{\pgfqpoint{3.137694in}{2.702690in}}%
\pgfpathcurveto{\pgfqpoint{3.137694in}{2.713740in}}{\pgfqpoint{3.133303in}{2.724339in}}{\pgfqpoint{3.125490in}{2.732152in}}%
\pgfpathcurveto{\pgfqpoint{3.117676in}{2.739966in}}{\pgfqpoint{3.107077in}{2.744356in}}{\pgfqpoint{3.096027in}{2.744356in}}%
\pgfpathcurveto{\pgfqpoint{3.084977in}{2.744356in}}{\pgfqpoint{3.074378in}{2.739966in}}{\pgfqpoint{3.066564in}{2.732152in}}%
\pgfpathcurveto{\pgfqpoint{3.058751in}{2.724339in}}{\pgfqpoint{3.054360in}{2.713740in}}{\pgfqpoint{3.054360in}{2.702690in}}%
\pgfpathcurveto{\pgfqpoint{3.054360in}{2.691640in}}{\pgfqpoint{3.058751in}{2.681040in}}{\pgfqpoint{3.066564in}{2.673227in}}%
\pgfpathcurveto{\pgfqpoint{3.074378in}{2.665413in}}{\pgfqpoint{3.084977in}{2.661023in}}{\pgfqpoint{3.096027in}{2.661023in}}%
\pgfpathclose%
\pgfusepath{stroke,fill}%
\end{pgfscope}%
\begin{pgfscope}%
\pgfpathrectangle{\pgfqpoint{0.600000in}{0.600000in}}{\pgfqpoint{3.900000in}{3.900000in}}%
\pgfusepath{clip}%
\pgfsetbuttcap%
\pgfsetroundjoin%
\definecolor{currentfill}{rgb}{0.121569,0.466667,0.705882}%
\pgfsetfillcolor{currentfill}%
\pgfsetlinewidth{1.003750pt}%
\definecolor{currentstroke}{rgb}{0.121569,0.466667,0.705882}%
\pgfsetstrokecolor{currentstroke}%
\pgfsetdash{}{0pt}%
\pgfpathmoveto{\pgfqpoint{3.223429in}{1.856211in}}%
\pgfpathcurveto{\pgfqpoint{3.234479in}{1.856211in}}{\pgfqpoint{3.245078in}{1.860601in}}{\pgfqpoint{3.252892in}{1.868414in}}%
\pgfpathcurveto{\pgfqpoint{3.260705in}{1.876228in}}{\pgfqpoint{3.265096in}{1.886827in}}{\pgfqpoint{3.265096in}{1.897877in}}%
\pgfpathcurveto{\pgfqpoint{3.265096in}{1.908927in}}{\pgfqpoint{3.260705in}{1.919526in}}{\pgfqpoint{3.252892in}{1.927340in}}%
\pgfpathcurveto{\pgfqpoint{3.245078in}{1.935154in}}{\pgfqpoint{3.234479in}{1.939544in}}{\pgfqpoint{3.223429in}{1.939544in}}%
\pgfpathcurveto{\pgfqpoint{3.212379in}{1.939544in}}{\pgfqpoint{3.201780in}{1.935154in}}{\pgfqpoint{3.193966in}{1.927340in}}%
\pgfpathcurveto{\pgfqpoint{3.186152in}{1.919526in}}{\pgfqpoint{3.181762in}{1.908927in}}{\pgfqpoint{3.181762in}{1.897877in}}%
\pgfpathcurveto{\pgfqpoint{3.181762in}{1.886827in}}{\pgfqpoint{3.186152in}{1.876228in}}{\pgfqpoint{3.193966in}{1.868414in}}%
\pgfpathcurveto{\pgfqpoint{3.201780in}{1.860601in}}{\pgfqpoint{3.212379in}{1.856211in}}{\pgfqpoint{3.223429in}{1.856211in}}%
\pgfpathclose%
\pgfusepath{stroke,fill}%
\end{pgfscope}%
\begin{pgfscope}%
\pgfpathrectangle{\pgfqpoint{0.600000in}{0.600000in}}{\pgfqpoint{3.900000in}{3.900000in}}%
\pgfusepath{clip}%
\pgfsetbuttcap%
\pgfsetroundjoin%
\definecolor{currentfill}{rgb}{0.121569,0.466667,0.705882}%
\pgfsetfillcolor{currentfill}%
\pgfsetlinewidth{1.003750pt}%
\definecolor{currentstroke}{rgb}{0.121569,0.466667,0.705882}%
\pgfsetstrokecolor{currentstroke}%
\pgfsetdash{}{0pt}%
\pgfpathmoveto{\pgfqpoint{2.115152in}{2.883256in}}%
\pgfpathcurveto{\pgfqpoint{2.126202in}{2.883256in}}{\pgfqpoint{2.136801in}{2.887647in}}{\pgfqpoint{2.144615in}{2.895460in}}%
\pgfpathcurveto{\pgfqpoint{2.152428in}{2.903274in}}{\pgfqpoint{2.156819in}{2.913873in}}{\pgfqpoint{2.156819in}{2.924923in}}%
\pgfpathcurveto{\pgfqpoint{2.156819in}{2.935973in}}{\pgfqpoint{2.152428in}{2.946572in}}{\pgfqpoint{2.144615in}{2.954386in}}%
\pgfpathcurveto{\pgfqpoint{2.136801in}{2.962199in}}{\pgfqpoint{2.126202in}{2.966590in}}{\pgfqpoint{2.115152in}{2.966590in}}%
\pgfpathcurveto{\pgfqpoint{2.104102in}{2.966590in}}{\pgfqpoint{2.093503in}{2.962199in}}{\pgfqpoint{2.085689in}{2.954386in}}%
\pgfpathcurveto{\pgfqpoint{2.077876in}{2.946572in}}{\pgfqpoint{2.073485in}{2.935973in}}{\pgfqpoint{2.073485in}{2.924923in}}%
\pgfpathcurveto{\pgfqpoint{2.073485in}{2.913873in}}{\pgfqpoint{2.077876in}{2.903274in}}{\pgfqpoint{2.085689in}{2.895460in}}%
\pgfpathcurveto{\pgfqpoint{2.093503in}{2.887647in}}{\pgfqpoint{2.104102in}{2.883256in}}{\pgfqpoint{2.115152in}{2.883256in}}%
\pgfpathclose%
\pgfusepath{stroke,fill}%
\end{pgfscope}%
\begin{pgfscope}%
\pgfpathrectangle{\pgfqpoint{0.600000in}{0.600000in}}{\pgfqpoint{3.900000in}{3.900000in}}%
\pgfusepath{clip}%
\pgfsetbuttcap%
\pgfsetroundjoin%
\definecolor{currentfill}{rgb}{0.121569,0.466667,0.705882}%
\pgfsetfillcolor{currentfill}%
\pgfsetlinewidth{1.003750pt}%
\definecolor{currentstroke}{rgb}{0.121569,0.466667,0.705882}%
\pgfsetstrokecolor{currentstroke}%
\pgfsetdash{}{0pt}%
\pgfpathmoveto{\pgfqpoint{2.623321in}{2.596300in}}%
\pgfpathcurveto{\pgfqpoint{2.634371in}{2.596300in}}{\pgfqpoint{2.644970in}{2.600690in}}{\pgfqpoint{2.652784in}{2.608504in}}%
\pgfpathcurveto{\pgfqpoint{2.660597in}{2.616317in}}{\pgfqpoint{2.664988in}{2.626916in}}{\pgfqpoint{2.664988in}{2.637966in}}%
\pgfpathcurveto{\pgfqpoint{2.664988in}{2.649017in}}{\pgfqpoint{2.660597in}{2.659616in}}{\pgfqpoint{2.652784in}{2.667429in}}%
\pgfpathcurveto{\pgfqpoint{2.644970in}{2.675243in}}{\pgfqpoint{2.634371in}{2.679633in}}{\pgfqpoint{2.623321in}{2.679633in}}%
\pgfpathcurveto{\pgfqpoint{2.612271in}{2.679633in}}{\pgfqpoint{2.601672in}{2.675243in}}{\pgfqpoint{2.593858in}{2.667429in}}%
\pgfpathcurveto{\pgfqpoint{2.586044in}{2.659616in}}{\pgfqpoint{2.581654in}{2.649017in}}{\pgfqpoint{2.581654in}{2.637966in}}%
\pgfpathcurveto{\pgfqpoint{2.581654in}{2.626916in}}{\pgfqpoint{2.586044in}{2.616317in}}{\pgfqpoint{2.593858in}{2.608504in}}%
\pgfpathcurveto{\pgfqpoint{2.601672in}{2.600690in}}{\pgfqpoint{2.612271in}{2.596300in}}{\pgfqpoint{2.623321in}{2.596300in}}%
\pgfpathclose%
\pgfusepath{stroke,fill}%
\end{pgfscope}%
\begin{pgfscope}%
\pgfpathrectangle{\pgfqpoint{0.600000in}{0.600000in}}{\pgfqpoint{3.900000in}{3.900000in}}%
\pgfusepath{clip}%
\pgfsetbuttcap%
\pgfsetroundjoin%
\definecolor{currentfill}{rgb}{0.121569,0.466667,0.705882}%
\pgfsetfillcolor{currentfill}%
\pgfsetlinewidth{1.003750pt}%
\definecolor{currentstroke}{rgb}{0.121569,0.466667,0.705882}%
\pgfsetstrokecolor{currentstroke}%
\pgfsetdash{}{0pt}%
\pgfpathmoveto{\pgfqpoint{1.572516in}{2.808087in}}%
\pgfpathcurveto{\pgfqpoint{1.583566in}{2.808087in}}{\pgfqpoint{1.594165in}{2.812477in}}{\pgfqpoint{1.601979in}{2.820291in}}%
\pgfpathcurveto{\pgfqpoint{1.609792in}{2.828104in}}{\pgfqpoint{1.614183in}{2.838703in}}{\pgfqpoint{1.614183in}{2.849754in}}%
\pgfpathcurveto{\pgfqpoint{1.614183in}{2.860804in}}{\pgfqpoint{1.609792in}{2.871403in}}{\pgfqpoint{1.601979in}{2.879216in}}%
\pgfpathcurveto{\pgfqpoint{1.594165in}{2.887030in}}{\pgfqpoint{1.583566in}{2.891420in}}{\pgfqpoint{1.572516in}{2.891420in}}%
\pgfpathcurveto{\pgfqpoint{1.561466in}{2.891420in}}{\pgfqpoint{1.550867in}{2.887030in}}{\pgfqpoint{1.543053in}{2.879216in}}%
\pgfpathcurveto{\pgfqpoint{1.535240in}{2.871403in}}{\pgfqpoint{1.530849in}{2.860804in}}{\pgfqpoint{1.530849in}{2.849754in}}%
\pgfpathcurveto{\pgfqpoint{1.530849in}{2.838703in}}{\pgfqpoint{1.535240in}{2.828104in}}{\pgfqpoint{1.543053in}{2.820291in}}%
\pgfpathcurveto{\pgfqpoint{1.550867in}{2.812477in}}{\pgfqpoint{1.561466in}{2.808087in}}{\pgfqpoint{1.572516in}{2.808087in}}%
\pgfpathclose%
\pgfusepath{stroke,fill}%
\end{pgfscope}%
\begin{pgfscope}%
\pgfpathrectangle{\pgfqpoint{0.600000in}{0.600000in}}{\pgfqpoint{3.900000in}{3.900000in}}%
\pgfusepath{clip}%
\pgfsetbuttcap%
\pgfsetroundjoin%
\definecolor{currentfill}{rgb}{0.121569,0.466667,0.705882}%
\pgfsetfillcolor{currentfill}%
\pgfsetlinewidth{1.003750pt}%
\definecolor{currentstroke}{rgb}{0.121569,0.466667,0.705882}%
\pgfsetstrokecolor{currentstroke}%
\pgfsetdash{}{0pt}%
\pgfpathmoveto{\pgfqpoint{2.034209in}{2.253496in}}%
\pgfpathcurveto{\pgfqpoint{2.045259in}{2.253496in}}{\pgfqpoint{2.055858in}{2.257887in}}{\pgfqpoint{2.063672in}{2.265700in}}%
\pgfpathcurveto{\pgfqpoint{2.071485in}{2.273514in}}{\pgfqpoint{2.075875in}{2.284113in}}{\pgfqpoint{2.075875in}{2.295163in}}%
\pgfpathcurveto{\pgfqpoint{2.075875in}{2.306213in}}{\pgfqpoint{2.071485in}{2.316812in}}{\pgfqpoint{2.063672in}{2.324626in}}%
\pgfpathcurveto{\pgfqpoint{2.055858in}{2.332439in}}{\pgfqpoint{2.045259in}{2.336830in}}{\pgfqpoint{2.034209in}{2.336830in}}%
\pgfpathcurveto{\pgfqpoint{2.023159in}{2.336830in}}{\pgfqpoint{2.012560in}{2.332439in}}{\pgfqpoint{2.004746in}{2.324626in}}%
\pgfpathcurveto{\pgfqpoint{1.996932in}{2.316812in}}{\pgfqpoint{1.992542in}{2.306213in}}{\pgfqpoint{1.992542in}{2.295163in}}%
\pgfpathcurveto{\pgfqpoint{1.992542in}{2.284113in}}{\pgfqpoint{1.996932in}{2.273514in}}{\pgfqpoint{2.004746in}{2.265700in}}%
\pgfpathcurveto{\pgfqpoint{2.012560in}{2.257887in}}{\pgfqpoint{2.023159in}{2.253496in}}{\pgfqpoint{2.034209in}{2.253496in}}%
\pgfpathclose%
\pgfusepath{stroke,fill}%
\end{pgfscope}%
\begin{pgfscope}%
\pgfpathrectangle{\pgfqpoint{0.600000in}{0.600000in}}{\pgfqpoint{3.900000in}{3.900000in}}%
\pgfusepath{clip}%
\pgfsetbuttcap%
\pgfsetroundjoin%
\definecolor{currentfill}{rgb}{0.121569,0.466667,0.705882}%
\pgfsetfillcolor{currentfill}%
\pgfsetlinewidth{1.003750pt}%
\definecolor{currentstroke}{rgb}{0.121569,0.466667,0.705882}%
\pgfsetstrokecolor{currentstroke}%
\pgfsetdash{}{0pt}%
\pgfpathmoveto{\pgfqpoint{2.497193in}{2.921445in}}%
\pgfpathcurveto{\pgfqpoint{2.508243in}{2.921445in}}{\pgfqpoint{2.518842in}{2.925835in}}{\pgfqpoint{2.526655in}{2.933649in}}%
\pgfpathcurveto{\pgfqpoint{2.534469in}{2.941462in}}{\pgfqpoint{2.538859in}{2.952061in}}{\pgfqpoint{2.538859in}{2.963111in}}%
\pgfpathcurveto{\pgfqpoint{2.538859in}{2.974162in}}{\pgfqpoint{2.534469in}{2.984761in}}{\pgfqpoint{2.526655in}{2.992574in}}%
\pgfpathcurveto{\pgfqpoint{2.518842in}{3.000388in}}{\pgfqpoint{2.508243in}{3.004778in}}{\pgfqpoint{2.497193in}{3.004778in}}%
\pgfpathcurveto{\pgfqpoint{2.486143in}{3.004778in}}{\pgfqpoint{2.475544in}{3.000388in}}{\pgfqpoint{2.467730in}{2.992574in}}%
\pgfpathcurveto{\pgfqpoint{2.459916in}{2.984761in}}{\pgfqpoint{2.455526in}{2.974162in}}{\pgfqpoint{2.455526in}{2.963111in}}%
\pgfpathcurveto{\pgfqpoint{2.455526in}{2.952061in}}{\pgfqpoint{2.459916in}{2.941462in}}{\pgfqpoint{2.467730in}{2.933649in}}%
\pgfpathcurveto{\pgfqpoint{2.475544in}{2.925835in}}{\pgfqpoint{2.486143in}{2.921445in}}{\pgfqpoint{2.497193in}{2.921445in}}%
\pgfpathclose%
\pgfusepath{stroke,fill}%
\end{pgfscope}%
\begin{pgfscope}%
\pgfpathrectangle{\pgfqpoint{0.600000in}{0.600000in}}{\pgfqpoint{3.900000in}{3.900000in}}%
\pgfusepath{clip}%
\pgfsetbuttcap%
\pgfsetroundjoin%
\definecolor{currentfill}{rgb}{0.121569,0.466667,0.705882}%
\pgfsetfillcolor{currentfill}%
\pgfsetlinewidth{1.003750pt}%
\definecolor{currentstroke}{rgb}{0.121569,0.466667,0.705882}%
\pgfsetstrokecolor{currentstroke}%
\pgfsetdash{}{0pt}%
\pgfpathmoveto{\pgfqpoint{2.686770in}{2.839834in}}%
\pgfpathcurveto{\pgfqpoint{2.697820in}{2.839834in}}{\pgfqpoint{2.708420in}{2.844225in}}{\pgfqpoint{2.716233in}{2.852038in}}%
\pgfpathcurveto{\pgfqpoint{2.724047in}{2.859852in}}{\pgfqpoint{2.728437in}{2.870451in}}{\pgfqpoint{2.728437in}{2.881501in}}%
\pgfpathcurveto{\pgfqpoint{2.728437in}{2.892551in}}{\pgfqpoint{2.724047in}{2.903150in}}{\pgfqpoint{2.716233in}{2.910964in}}%
\pgfpathcurveto{\pgfqpoint{2.708420in}{2.918777in}}{\pgfqpoint{2.697820in}{2.923168in}}{\pgfqpoint{2.686770in}{2.923168in}}%
\pgfpathcurveto{\pgfqpoint{2.675720in}{2.923168in}}{\pgfqpoint{2.665121in}{2.918777in}}{\pgfqpoint{2.657308in}{2.910964in}}%
\pgfpathcurveto{\pgfqpoint{2.649494in}{2.903150in}}{\pgfqpoint{2.645104in}{2.892551in}}{\pgfqpoint{2.645104in}{2.881501in}}%
\pgfpathcurveto{\pgfqpoint{2.645104in}{2.870451in}}{\pgfqpoint{2.649494in}{2.859852in}}{\pgfqpoint{2.657308in}{2.852038in}}%
\pgfpathcurveto{\pgfqpoint{2.665121in}{2.844225in}}{\pgfqpoint{2.675720in}{2.839834in}}{\pgfqpoint{2.686770in}{2.839834in}}%
\pgfpathclose%
\pgfusepath{stroke,fill}%
\end{pgfscope}%
\begin{pgfscope}%
\pgfpathrectangle{\pgfqpoint{0.600000in}{0.600000in}}{\pgfqpoint{3.900000in}{3.900000in}}%
\pgfusepath{clip}%
\pgfsetbuttcap%
\pgfsetroundjoin%
\definecolor{currentfill}{rgb}{0.121569,0.466667,0.705882}%
\pgfsetfillcolor{currentfill}%
\pgfsetlinewidth{1.003750pt}%
\definecolor{currentstroke}{rgb}{0.121569,0.466667,0.705882}%
\pgfsetstrokecolor{currentstroke}%
\pgfsetdash{}{0pt}%
\pgfpathmoveto{\pgfqpoint{3.082972in}{2.096433in}}%
\pgfpathcurveto{\pgfqpoint{3.094022in}{2.096433in}}{\pgfqpoint{3.104621in}{2.100823in}}{\pgfqpoint{3.112434in}{2.108637in}}%
\pgfpathcurveto{\pgfqpoint{3.120248in}{2.116451in}}{\pgfqpoint{3.124638in}{2.127050in}}{\pgfqpoint{3.124638in}{2.138100in}}%
\pgfpathcurveto{\pgfqpoint{3.124638in}{2.149150in}}{\pgfqpoint{3.120248in}{2.159749in}}{\pgfqpoint{3.112434in}{2.167563in}}%
\pgfpathcurveto{\pgfqpoint{3.104621in}{2.175376in}}{\pgfqpoint{3.094022in}{2.179766in}}{\pgfqpoint{3.082972in}{2.179766in}}%
\pgfpathcurveto{\pgfqpoint{3.071921in}{2.179766in}}{\pgfqpoint{3.061322in}{2.175376in}}{\pgfqpoint{3.053509in}{2.167563in}}%
\pgfpathcurveto{\pgfqpoint{3.045695in}{2.159749in}}{\pgfqpoint{3.041305in}{2.149150in}}{\pgfqpoint{3.041305in}{2.138100in}}%
\pgfpathcurveto{\pgfqpoint{3.041305in}{2.127050in}}{\pgfqpoint{3.045695in}{2.116451in}}{\pgfqpoint{3.053509in}{2.108637in}}%
\pgfpathcurveto{\pgfqpoint{3.061322in}{2.100823in}}{\pgfqpoint{3.071921in}{2.096433in}}{\pgfqpoint{3.082972in}{2.096433in}}%
\pgfpathclose%
\pgfusepath{stroke,fill}%
\end{pgfscope}%
\begin{pgfscope}%
\pgfpathrectangle{\pgfqpoint{0.600000in}{0.600000in}}{\pgfqpoint{3.900000in}{3.900000in}}%
\pgfusepath{clip}%
\pgfsetbuttcap%
\pgfsetroundjoin%
\definecolor{currentfill}{rgb}{0.121569,0.466667,0.705882}%
\pgfsetfillcolor{currentfill}%
\pgfsetlinewidth{1.003750pt}%
\definecolor{currentstroke}{rgb}{0.121569,0.466667,0.705882}%
\pgfsetstrokecolor{currentstroke}%
\pgfsetdash{}{0pt}%
\pgfpathmoveto{\pgfqpoint{3.027977in}{2.737434in}}%
\pgfpathcurveto{\pgfqpoint{3.039027in}{2.737434in}}{\pgfqpoint{3.049626in}{2.741825in}}{\pgfqpoint{3.057440in}{2.749638in}}%
\pgfpathcurveto{\pgfqpoint{3.065253in}{2.757452in}}{\pgfqpoint{3.069643in}{2.768051in}}{\pgfqpoint{3.069643in}{2.779101in}}%
\pgfpathcurveto{\pgfqpoint{3.069643in}{2.790151in}}{\pgfqpoint{3.065253in}{2.800750in}}{\pgfqpoint{3.057440in}{2.808564in}}%
\pgfpathcurveto{\pgfqpoint{3.049626in}{2.816377in}}{\pgfqpoint{3.039027in}{2.820768in}}{\pgfqpoint{3.027977in}{2.820768in}}%
\pgfpathcurveto{\pgfqpoint{3.016927in}{2.820768in}}{\pgfqpoint{3.006328in}{2.816377in}}{\pgfqpoint{2.998514in}{2.808564in}}%
\pgfpathcurveto{\pgfqpoint{2.990700in}{2.800750in}}{\pgfqpoint{2.986310in}{2.790151in}}{\pgfqpoint{2.986310in}{2.779101in}}%
\pgfpathcurveto{\pgfqpoint{2.986310in}{2.768051in}}{\pgfqpoint{2.990700in}{2.757452in}}{\pgfqpoint{2.998514in}{2.749638in}}%
\pgfpathcurveto{\pgfqpoint{3.006328in}{2.741825in}}{\pgfqpoint{3.016927in}{2.737434in}}{\pgfqpoint{3.027977in}{2.737434in}}%
\pgfpathclose%
\pgfusepath{stroke,fill}%
\end{pgfscope}%
\begin{pgfscope}%
\pgfpathrectangle{\pgfqpoint{0.600000in}{0.600000in}}{\pgfqpoint{3.900000in}{3.900000in}}%
\pgfusepath{clip}%
\pgfsetbuttcap%
\pgfsetroundjoin%
\definecolor{currentfill}{rgb}{0.121569,0.466667,0.705882}%
\pgfsetfillcolor{currentfill}%
\pgfsetlinewidth{1.003750pt}%
\definecolor{currentstroke}{rgb}{0.121569,0.466667,0.705882}%
\pgfsetstrokecolor{currentstroke}%
\pgfsetdash{}{0pt}%
\pgfpathmoveto{\pgfqpoint{2.779322in}{2.309754in}}%
\pgfpathcurveto{\pgfqpoint{2.790372in}{2.309754in}}{\pgfqpoint{2.800971in}{2.314144in}}{\pgfqpoint{2.808785in}{2.321958in}}%
\pgfpathcurveto{\pgfqpoint{2.816598in}{2.329772in}}{\pgfqpoint{2.820989in}{2.340371in}}{\pgfqpoint{2.820989in}{2.351421in}}%
\pgfpathcurveto{\pgfqpoint{2.820989in}{2.362471in}}{\pgfqpoint{2.816598in}{2.373070in}}{\pgfqpoint{2.808785in}{2.380883in}}%
\pgfpathcurveto{\pgfqpoint{2.800971in}{2.388697in}}{\pgfqpoint{2.790372in}{2.393087in}}{\pgfqpoint{2.779322in}{2.393087in}}%
\pgfpathcurveto{\pgfqpoint{2.768272in}{2.393087in}}{\pgfqpoint{2.757673in}{2.388697in}}{\pgfqpoint{2.749859in}{2.380883in}}%
\pgfpathcurveto{\pgfqpoint{2.742046in}{2.373070in}}{\pgfqpoint{2.737655in}{2.362471in}}{\pgfqpoint{2.737655in}{2.351421in}}%
\pgfpathcurveto{\pgfqpoint{2.737655in}{2.340371in}}{\pgfqpoint{2.742046in}{2.329772in}}{\pgfqpoint{2.749859in}{2.321958in}}%
\pgfpathcurveto{\pgfqpoint{2.757673in}{2.314144in}}{\pgfqpoint{2.768272in}{2.309754in}}{\pgfqpoint{2.779322in}{2.309754in}}%
\pgfpathclose%
\pgfusepath{stroke,fill}%
\end{pgfscope}%
\begin{pgfscope}%
\pgfpathrectangle{\pgfqpoint{0.600000in}{0.600000in}}{\pgfqpoint{3.900000in}{3.900000in}}%
\pgfusepath{clip}%
\pgfsetbuttcap%
\pgfsetroundjoin%
\definecolor{currentfill}{rgb}{0.121569,0.466667,0.705882}%
\pgfsetfillcolor{currentfill}%
\pgfsetlinewidth{1.003750pt}%
\definecolor{currentstroke}{rgb}{0.121569,0.466667,0.705882}%
\pgfsetstrokecolor{currentstroke}%
\pgfsetdash{}{0pt}%
\pgfpathmoveto{\pgfqpoint{2.606363in}{2.317735in}}%
\pgfpathcurveto{\pgfqpoint{2.617413in}{2.317735in}}{\pgfqpoint{2.628012in}{2.322126in}}{\pgfqpoint{2.635826in}{2.329939in}}%
\pgfpathcurveto{\pgfqpoint{2.643639in}{2.337753in}}{\pgfqpoint{2.648030in}{2.348352in}}{\pgfqpoint{2.648030in}{2.359402in}}%
\pgfpathcurveto{\pgfqpoint{2.648030in}{2.370452in}}{\pgfqpoint{2.643639in}{2.381051in}}{\pgfqpoint{2.635826in}{2.388865in}}%
\pgfpathcurveto{\pgfqpoint{2.628012in}{2.396678in}}{\pgfqpoint{2.617413in}{2.401069in}}{\pgfqpoint{2.606363in}{2.401069in}}%
\pgfpathcurveto{\pgfqpoint{2.595313in}{2.401069in}}{\pgfqpoint{2.584714in}{2.396678in}}{\pgfqpoint{2.576900in}{2.388865in}}%
\pgfpathcurveto{\pgfqpoint{2.569086in}{2.381051in}}{\pgfqpoint{2.564696in}{2.370452in}}{\pgfqpoint{2.564696in}{2.359402in}}%
\pgfpathcurveto{\pgfqpoint{2.564696in}{2.348352in}}{\pgfqpoint{2.569086in}{2.337753in}}{\pgfqpoint{2.576900in}{2.329939in}}%
\pgfpathcurveto{\pgfqpoint{2.584714in}{2.322126in}}{\pgfqpoint{2.595313in}{2.317735in}}{\pgfqpoint{2.606363in}{2.317735in}}%
\pgfpathclose%
\pgfusepath{stroke,fill}%
\end{pgfscope}%
\begin{pgfscope}%
\pgfpathrectangle{\pgfqpoint{0.600000in}{0.600000in}}{\pgfqpoint{3.900000in}{3.900000in}}%
\pgfusepath{clip}%
\pgfsetbuttcap%
\pgfsetroundjoin%
\definecolor{currentfill}{rgb}{0.121569,0.466667,0.705882}%
\pgfsetfillcolor{currentfill}%
\pgfsetlinewidth{1.003750pt}%
\definecolor{currentstroke}{rgb}{0.121569,0.466667,0.705882}%
\pgfsetstrokecolor{currentstroke}%
\pgfsetdash{}{0pt}%
\pgfpathmoveto{\pgfqpoint{2.302912in}{1.876699in}}%
\pgfpathcurveto{\pgfqpoint{2.313962in}{1.876699in}}{\pgfqpoint{2.324561in}{1.881089in}}{\pgfqpoint{2.332375in}{1.888903in}}%
\pgfpathcurveto{\pgfqpoint{2.340188in}{1.896716in}}{\pgfqpoint{2.344579in}{1.907315in}}{\pgfqpoint{2.344579in}{1.918365in}}%
\pgfpathcurveto{\pgfqpoint{2.344579in}{1.929416in}}{\pgfqpoint{2.340188in}{1.940015in}}{\pgfqpoint{2.332375in}{1.947828in}}%
\pgfpathcurveto{\pgfqpoint{2.324561in}{1.955642in}}{\pgfqpoint{2.313962in}{1.960032in}}{\pgfqpoint{2.302912in}{1.960032in}}%
\pgfpathcurveto{\pgfqpoint{2.291862in}{1.960032in}}{\pgfqpoint{2.281263in}{1.955642in}}{\pgfqpoint{2.273449in}{1.947828in}}%
\pgfpathcurveto{\pgfqpoint{2.265636in}{1.940015in}}{\pgfqpoint{2.261245in}{1.929416in}}{\pgfqpoint{2.261245in}{1.918365in}}%
\pgfpathcurveto{\pgfqpoint{2.261245in}{1.907315in}}{\pgfqpoint{2.265636in}{1.896716in}}{\pgfqpoint{2.273449in}{1.888903in}}%
\pgfpathcurveto{\pgfqpoint{2.281263in}{1.881089in}}{\pgfqpoint{2.291862in}{1.876699in}}{\pgfqpoint{2.302912in}{1.876699in}}%
\pgfpathclose%
\pgfusepath{stroke,fill}%
\end{pgfscope}%
\begin{pgfscope}%
\pgfpathrectangle{\pgfqpoint{0.600000in}{0.600000in}}{\pgfqpoint{3.900000in}{3.900000in}}%
\pgfusepath{clip}%
\pgfsetbuttcap%
\pgfsetroundjoin%
\definecolor{currentfill}{rgb}{0.121569,0.466667,0.705882}%
\pgfsetfillcolor{currentfill}%
\pgfsetlinewidth{1.003750pt}%
\definecolor{currentstroke}{rgb}{0.121569,0.466667,0.705882}%
\pgfsetstrokecolor{currentstroke}%
\pgfsetdash{}{0pt}%
\pgfpathmoveto{\pgfqpoint{2.109043in}{2.492733in}}%
\pgfpathcurveto{\pgfqpoint{2.120093in}{2.492733in}}{\pgfqpoint{2.130692in}{2.497123in}}{\pgfqpoint{2.138505in}{2.504936in}}%
\pgfpathcurveto{\pgfqpoint{2.146319in}{2.512750in}}{\pgfqpoint{2.150709in}{2.523349in}}{\pgfqpoint{2.150709in}{2.534399in}}%
\pgfpathcurveto{\pgfqpoint{2.150709in}{2.545449in}}{\pgfqpoint{2.146319in}{2.556048in}}{\pgfqpoint{2.138505in}{2.563862in}}%
\pgfpathcurveto{\pgfqpoint{2.130692in}{2.571676in}}{\pgfqpoint{2.120093in}{2.576066in}}{\pgfqpoint{2.109043in}{2.576066in}}%
\pgfpathcurveto{\pgfqpoint{2.097992in}{2.576066in}}{\pgfqpoint{2.087393in}{2.571676in}}{\pgfqpoint{2.079580in}{2.563862in}}%
\pgfpathcurveto{\pgfqpoint{2.071766in}{2.556048in}}{\pgfqpoint{2.067376in}{2.545449in}}{\pgfqpoint{2.067376in}{2.534399in}}%
\pgfpathcurveto{\pgfqpoint{2.067376in}{2.523349in}}{\pgfqpoint{2.071766in}{2.512750in}}{\pgfqpoint{2.079580in}{2.504936in}}%
\pgfpathcurveto{\pgfqpoint{2.087393in}{2.497123in}}{\pgfqpoint{2.097992in}{2.492733in}}{\pgfqpoint{2.109043in}{2.492733in}}%
\pgfpathclose%
\pgfusepath{stroke,fill}%
\end{pgfscope}%
\begin{pgfscope}%
\pgfpathrectangle{\pgfqpoint{0.600000in}{0.600000in}}{\pgfqpoint{3.900000in}{3.900000in}}%
\pgfusepath{clip}%
\pgfsetbuttcap%
\pgfsetroundjoin%
\definecolor{currentfill}{rgb}{0.121569,0.466667,0.705882}%
\pgfsetfillcolor{currentfill}%
\pgfsetlinewidth{1.003750pt}%
\definecolor{currentstroke}{rgb}{0.121569,0.466667,0.705882}%
\pgfsetstrokecolor{currentstroke}%
\pgfsetdash{}{0pt}%
\pgfpathmoveto{\pgfqpoint{2.798677in}{1.649943in}}%
\pgfpathcurveto{\pgfqpoint{2.809727in}{1.649943in}}{\pgfqpoint{2.820326in}{1.654334in}}{\pgfqpoint{2.828139in}{1.662147in}}%
\pgfpathcurveto{\pgfqpoint{2.835953in}{1.669961in}}{\pgfqpoint{2.840343in}{1.680560in}}{\pgfqpoint{2.840343in}{1.691610in}}%
\pgfpathcurveto{\pgfqpoint{2.840343in}{1.702660in}}{\pgfqpoint{2.835953in}{1.713259in}}{\pgfqpoint{2.828139in}{1.721073in}}%
\pgfpathcurveto{\pgfqpoint{2.820326in}{1.728886in}}{\pgfqpoint{2.809727in}{1.733277in}}{\pgfqpoint{2.798677in}{1.733277in}}%
\pgfpathcurveto{\pgfqpoint{2.787626in}{1.733277in}}{\pgfqpoint{2.777027in}{1.728886in}}{\pgfqpoint{2.769214in}{1.721073in}}%
\pgfpathcurveto{\pgfqpoint{2.761400in}{1.713259in}}{\pgfqpoint{2.757010in}{1.702660in}}{\pgfqpoint{2.757010in}{1.691610in}}%
\pgfpathcurveto{\pgfqpoint{2.757010in}{1.680560in}}{\pgfqpoint{2.761400in}{1.669961in}}{\pgfqpoint{2.769214in}{1.662147in}}%
\pgfpathcurveto{\pgfqpoint{2.777027in}{1.654334in}}{\pgfqpoint{2.787626in}{1.649943in}}{\pgfqpoint{2.798677in}{1.649943in}}%
\pgfpathclose%
\pgfusepath{stroke,fill}%
\end{pgfscope}%
\begin{pgfscope}%
\pgfpathrectangle{\pgfqpoint{0.600000in}{0.600000in}}{\pgfqpoint{3.900000in}{3.900000in}}%
\pgfusepath{clip}%
\pgfsetbuttcap%
\pgfsetroundjoin%
\definecolor{currentfill}{rgb}{0.121569,0.466667,0.705882}%
\pgfsetfillcolor{currentfill}%
\pgfsetlinewidth{1.003750pt}%
\definecolor{currentstroke}{rgb}{0.121569,0.466667,0.705882}%
\pgfsetstrokecolor{currentstroke}%
\pgfsetdash{}{0pt}%
\pgfpathmoveto{\pgfqpoint{1.962506in}{2.289599in}}%
\pgfpathcurveto{\pgfqpoint{1.973556in}{2.289599in}}{\pgfqpoint{1.984155in}{2.293989in}}{\pgfqpoint{1.991969in}{2.301803in}}%
\pgfpathcurveto{\pgfqpoint{1.999782in}{2.309616in}}{\pgfqpoint{2.004173in}{2.320215in}}{\pgfqpoint{2.004173in}{2.331266in}}%
\pgfpathcurveto{\pgfqpoint{2.004173in}{2.342316in}}{\pgfqpoint{1.999782in}{2.352915in}}{\pgfqpoint{1.991969in}{2.360728in}}%
\pgfpathcurveto{\pgfqpoint{1.984155in}{2.368542in}}{\pgfqpoint{1.973556in}{2.372932in}}{\pgfqpoint{1.962506in}{2.372932in}}%
\pgfpathcurveto{\pgfqpoint{1.951456in}{2.372932in}}{\pgfqpoint{1.940857in}{2.368542in}}{\pgfqpoint{1.933043in}{2.360728in}}%
\pgfpathcurveto{\pgfqpoint{1.925230in}{2.352915in}}{\pgfqpoint{1.920839in}{2.342316in}}{\pgfqpoint{1.920839in}{2.331266in}}%
\pgfpathcurveto{\pgfqpoint{1.920839in}{2.320215in}}{\pgfqpoint{1.925230in}{2.309616in}}{\pgfqpoint{1.933043in}{2.301803in}}%
\pgfpathcurveto{\pgfqpoint{1.940857in}{2.293989in}}{\pgfqpoint{1.951456in}{2.289599in}}{\pgfqpoint{1.962506in}{2.289599in}}%
\pgfpathclose%
\pgfusepath{stroke,fill}%
\end{pgfscope}%
\begin{pgfscope}%
\pgfpathrectangle{\pgfqpoint{0.600000in}{0.600000in}}{\pgfqpoint{3.900000in}{3.900000in}}%
\pgfusepath{clip}%
\pgfsetbuttcap%
\pgfsetroundjoin%
\definecolor{currentfill}{rgb}{0.121569,0.466667,0.705882}%
\pgfsetfillcolor{currentfill}%
\pgfsetlinewidth{1.003750pt}%
\definecolor{currentstroke}{rgb}{0.121569,0.466667,0.705882}%
\pgfsetstrokecolor{currentstroke}%
\pgfsetdash{}{0pt}%
\pgfpathmoveto{\pgfqpoint{1.284633in}{2.090281in}}%
\pgfpathcurveto{\pgfqpoint{1.295683in}{2.090281in}}{\pgfqpoint{1.306282in}{2.094672in}}{\pgfqpoint{1.314096in}{2.102485in}}%
\pgfpathcurveto{\pgfqpoint{1.321910in}{2.110299in}}{\pgfqpoint{1.326300in}{2.120898in}}{\pgfqpoint{1.326300in}{2.131948in}}%
\pgfpathcurveto{\pgfqpoint{1.326300in}{2.142998in}}{\pgfqpoint{1.321910in}{2.153597in}}{\pgfqpoint{1.314096in}{2.161411in}}%
\pgfpathcurveto{\pgfqpoint{1.306282in}{2.169224in}}{\pgfqpoint{1.295683in}{2.173615in}}{\pgfqpoint{1.284633in}{2.173615in}}%
\pgfpathcurveto{\pgfqpoint{1.273583in}{2.173615in}}{\pgfqpoint{1.262984in}{2.169224in}}{\pgfqpoint{1.255170in}{2.161411in}}%
\pgfpathcurveto{\pgfqpoint{1.247357in}{2.153597in}}{\pgfqpoint{1.242967in}{2.142998in}}{\pgfqpoint{1.242967in}{2.131948in}}%
\pgfpathcurveto{\pgfqpoint{1.242967in}{2.120898in}}{\pgfqpoint{1.247357in}{2.110299in}}{\pgfqpoint{1.255170in}{2.102485in}}%
\pgfpathcurveto{\pgfqpoint{1.262984in}{2.094672in}}{\pgfqpoint{1.273583in}{2.090281in}}{\pgfqpoint{1.284633in}{2.090281in}}%
\pgfpathclose%
\pgfusepath{stroke,fill}%
\end{pgfscope}%
\begin{pgfscope}%
\pgfpathrectangle{\pgfqpoint{0.600000in}{0.600000in}}{\pgfqpoint{3.900000in}{3.900000in}}%
\pgfusepath{clip}%
\pgfsetbuttcap%
\pgfsetroundjoin%
\definecolor{currentfill}{rgb}{0.121569,0.466667,0.705882}%
\pgfsetfillcolor{currentfill}%
\pgfsetlinewidth{1.003750pt}%
\definecolor{currentstroke}{rgb}{0.121569,0.466667,0.705882}%
\pgfsetstrokecolor{currentstroke}%
\pgfsetdash{}{0pt}%
\pgfpathmoveto{\pgfqpoint{2.282165in}{2.725220in}}%
\pgfpathcurveto{\pgfqpoint{2.293215in}{2.725220in}}{\pgfqpoint{2.303814in}{2.729610in}}{\pgfqpoint{2.311627in}{2.737424in}}%
\pgfpathcurveto{\pgfqpoint{2.319441in}{2.745238in}}{\pgfqpoint{2.323831in}{2.755837in}}{\pgfqpoint{2.323831in}{2.766887in}}%
\pgfpathcurveto{\pgfqpoint{2.323831in}{2.777937in}}{\pgfqpoint{2.319441in}{2.788536in}}{\pgfqpoint{2.311627in}{2.796350in}}%
\pgfpathcurveto{\pgfqpoint{2.303814in}{2.804163in}}{\pgfqpoint{2.293215in}{2.808554in}}{\pgfqpoint{2.282165in}{2.808554in}}%
\pgfpathcurveto{\pgfqpoint{2.271115in}{2.808554in}}{\pgfqpoint{2.260515in}{2.804163in}}{\pgfqpoint{2.252702in}{2.796350in}}%
\pgfpathcurveto{\pgfqpoint{2.244888in}{2.788536in}}{\pgfqpoint{2.240498in}{2.777937in}}{\pgfqpoint{2.240498in}{2.766887in}}%
\pgfpathcurveto{\pgfqpoint{2.240498in}{2.755837in}}{\pgfqpoint{2.244888in}{2.745238in}}{\pgfqpoint{2.252702in}{2.737424in}}%
\pgfpathcurveto{\pgfqpoint{2.260515in}{2.729610in}}{\pgfqpoint{2.271115in}{2.725220in}}{\pgfqpoint{2.282165in}{2.725220in}}%
\pgfpathclose%
\pgfusepath{stroke,fill}%
\end{pgfscope}%
\begin{pgfscope}%
\pgfpathrectangle{\pgfqpoint{0.600000in}{0.600000in}}{\pgfqpoint{3.900000in}{3.900000in}}%
\pgfusepath{clip}%
\pgfsetbuttcap%
\pgfsetroundjoin%
\definecolor{currentfill}{rgb}{0.121569,0.466667,0.705882}%
\pgfsetfillcolor{currentfill}%
\pgfsetlinewidth{1.003750pt}%
\definecolor{currentstroke}{rgb}{0.121569,0.466667,0.705882}%
\pgfsetstrokecolor{currentstroke}%
\pgfsetdash{}{0pt}%
\pgfpathmoveto{\pgfqpoint{2.925136in}{2.891959in}}%
\pgfpathcurveto{\pgfqpoint{2.936186in}{2.891959in}}{\pgfqpoint{2.946785in}{2.896349in}}{\pgfqpoint{2.954599in}{2.904163in}}%
\pgfpathcurveto{\pgfqpoint{2.962412in}{2.911976in}}{\pgfqpoint{2.966802in}{2.922575in}}{\pgfqpoint{2.966802in}{2.933625in}}%
\pgfpathcurveto{\pgfqpoint{2.966802in}{2.944675in}}{\pgfqpoint{2.962412in}{2.955275in}}{\pgfqpoint{2.954599in}{2.963088in}}%
\pgfpathcurveto{\pgfqpoint{2.946785in}{2.970902in}}{\pgfqpoint{2.936186in}{2.975292in}}{\pgfqpoint{2.925136in}{2.975292in}}%
\pgfpathcurveto{\pgfqpoint{2.914086in}{2.975292in}}{\pgfqpoint{2.903487in}{2.970902in}}{\pgfqpoint{2.895673in}{2.963088in}}%
\pgfpathcurveto{\pgfqpoint{2.887859in}{2.955275in}}{\pgfqpoint{2.883469in}{2.944675in}}{\pgfqpoint{2.883469in}{2.933625in}}%
\pgfpathcurveto{\pgfqpoint{2.883469in}{2.922575in}}{\pgfqpoint{2.887859in}{2.911976in}}{\pgfqpoint{2.895673in}{2.904163in}}%
\pgfpathcurveto{\pgfqpoint{2.903487in}{2.896349in}}{\pgfqpoint{2.914086in}{2.891959in}}{\pgfqpoint{2.925136in}{2.891959in}}%
\pgfpathclose%
\pgfusepath{stroke,fill}%
\end{pgfscope}%
\begin{pgfscope}%
\pgfpathrectangle{\pgfqpoint{0.600000in}{0.600000in}}{\pgfqpoint{3.900000in}{3.900000in}}%
\pgfusepath{clip}%
\pgfsetbuttcap%
\pgfsetroundjoin%
\definecolor{currentfill}{rgb}{0.121569,0.466667,0.705882}%
\pgfsetfillcolor{currentfill}%
\pgfsetlinewidth{1.003750pt}%
\definecolor{currentstroke}{rgb}{0.121569,0.466667,0.705882}%
\pgfsetstrokecolor{currentstroke}%
\pgfsetdash{}{0pt}%
\pgfpathmoveto{\pgfqpoint{1.739357in}{2.675047in}}%
\pgfpathcurveto{\pgfqpoint{1.750407in}{2.675047in}}{\pgfqpoint{1.761006in}{2.679438in}}{\pgfqpoint{1.768820in}{2.687251in}}%
\pgfpathcurveto{\pgfqpoint{1.776634in}{2.695065in}}{\pgfqpoint{1.781024in}{2.705664in}}{\pgfqpoint{1.781024in}{2.716714in}}%
\pgfpathcurveto{\pgfqpoint{1.781024in}{2.727764in}}{\pgfqpoint{1.776634in}{2.738363in}}{\pgfqpoint{1.768820in}{2.746177in}}%
\pgfpathcurveto{\pgfqpoint{1.761006in}{2.753990in}}{\pgfqpoint{1.750407in}{2.758381in}}{\pgfqpoint{1.739357in}{2.758381in}}%
\pgfpathcurveto{\pgfqpoint{1.728307in}{2.758381in}}{\pgfqpoint{1.717708in}{2.753990in}}{\pgfqpoint{1.709895in}{2.746177in}}%
\pgfpathcurveto{\pgfqpoint{1.702081in}{2.738363in}}{\pgfqpoint{1.697691in}{2.727764in}}{\pgfqpoint{1.697691in}{2.716714in}}%
\pgfpathcurveto{\pgfqpoint{1.697691in}{2.705664in}}{\pgfqpoint{1.702081in}{2.695065in}}{\pgfqpoint{1.709895in}{2.687251in}}%
\pgfpathcurveto{\pgfqpoint{1.717708in}{2.679438in}}{\pgfqpoint{1.728307in}{2.675047in}}{\pgfqpoint{1.739357in}{2.675047in}}%
\pgfpathclose%
\pgfusepath{stroke,fill}%
\end{pgfscope}%
\begin{pgfscope}%
\pgfpathrectangle{\pgfqpoint{0.600000in}{0.600000in}}{\pgfqpoint{3.900000in}{3.900000in}}%
\pgfusepath{clip}%
\pgfsetbuttcap%
\pgfsetroundjoin%
\definecolor{currentfill}{rgb}{0.121569,0.466667,0.705882}%
\pgfsetfillcolor{currentfill}%
\pgfsetlinewidth{1.003750pt}%
\definecolor{currentstroke}{rgb}{0.121569,0.466667,0.705882}%
\pgfsetstrokecolor{currentstroke}%
\pgfsetdash{}{0pt}%
\pgfpathmoveto{\pgfqpoint{2.051813in}{2.684807in}}%
\pgfpathcurveto{\pgfqpoint{2.062863in}{2.684807in}}{\pgfqpoint{2.073462in}{2.689197in}}{\pgfqpoint{2.081276in}{2.697010in}}%
\pgfpathcurveto{\pgfqpoint{2.089089in}{2.704824in}}{\pgfqpoint{2.093479in}{2.715423in}}{\pgfqpoint{2.093479in}{2.726473in}}%
\pgfpathcurveto{\pgfqpoint{2.093479in}{2.737523in}}{\pgfqpoint{2.089089in}{2.748122in}}{\pgfqpoint{2.081276in}{2.755936in}}%
\pgfpathcurveto{\pgfqpoint{2.073462in}{2.763750in}}{\pgfqpoint{2.062863in}{2.768140in}}{\pgfqpoint{2.051813in}{2.768140in}}%
\pgfpathcurveto{\pgfqpoint{2.040763in}{2.768140in}}{\pgfqpoint{2.030164in}{2.763750in}}{\pgfqpoint{2.022350in}{2.755936in}}%
\pgfpathcurveto{\pgfqpoint{2.014536in}{2.748122in}}{\pgfqpoint{2.010146in}{2.737523in}}{\pgfqpoint{2.010146in}{2.726473in}}%
\pgfpathcurveto{\pgfqpoint{2.010146in}{2.715423in}}{\pgfqpoint{2.014536in}{2.704824in}}{\pgfqpoint{2.022350in}{2.697010in}}%
\pgfpathcurveto{\pgfqpoint{2.030164in}{2.689197in}}{\pgfqpoint{2.040763in}{2.684807in}}{\pgfqpoint{2.051813in}{2.684807in}}%
\pgfpathclose%
\pgfusepath{stroke,fill}%
\end{pgfscope}%
\begin{pgfscope}%
\pgfpathrectangle{\pgfqpoint{0.600000in}{0.600000in}}{\pgfqpoint{3.900000in}{3.900000in}}%
\pgfusepath{clip}%
\pgfsetbuttcap%
\pgfsetroundjoin%
\definecolor{currentfill}{rgb}{0.121569,0.466667,0.705882}%
\pgfsetfillcolor{currentfill}%
\pgfsetlinewidth{1.003750pt}%
\definecolor{currentstroke}{rgb}{0.121569,0.466667,0.705882}%
\pgfsetstrokecolor{currentstroke}%
\pgfsetdash{}{0pt}%
\pgfpathmoveto{\pgfqpoint{2.545201in}{3.107685in}}%
\pgfpathcurveto{\pgfqpoint{2.556251in}{3.107685in}}{\pgfqpoint{2.566850in}{3.112075in}}{\pgfqpoint{2.574664in}{3.119889in}}%
\pgfpathcurveto{\pgfqpoint{2.582477in}{3.127703in}}{\pgfqpoint{2.586868in}{3.138302in}}{\pgfqpoint{2.586868in}{3.149352in}}%
\pgfpathcurveto{\pgfqpoint{2.586868in}{3.160402in}}{\pgfqpoint{2.582477in}{3.171001in}}{\pgfqpoint{2.574664in}{3.178815in}}%
\pgfpathcurveto{\pgfqpoint{2.566850in}{3.186628in}}{\pgfqpoint{2.556251in}{3.191019in}}{\pgfqpoint{2.545201in}{3.191019in}}%
\pgfpathcurveto{\pgfqpoint{2.534151in}{3.191019in}}{\pgfqpoint{2.523552in}{3.186628in}}{\pgfqpoint{2.515738in}{3.178815in}}%
\pgfpathcurveto{\pgfqpoint{2.507925in}{3.171001in}}{\pgfqpoint{2.503534in}{3.160402in}}{\pgfqpoint{2.503534in}{3.149352in}}%
\pgfpathcurveto{\pgfqpoint{2.503534in}{3.138302in}}{\pgfqpoint{2.507925in}{3.127703in}}{\pgfqpoint{2.515738in}{3.119889in}}%
\pgfpathcurveto{\pgfqpoint{2.523552in}{3.112075in}}{\pgfqpoint{2.534151in}{3.107685in}}{\pgfqpoint{2.545201in}{3.107685in}}%
\pgfpathclose%
\pgfusepath{stroke,fill}%
\end{pgfscope}%
\begin{pgfscope}%
\pgfpathrectangle{\pgfqpoint{0.600000in}{0.600000in}}{\pgfqpoint{3.900000in}{3.900000in}}%
\pgfusepath{clip}%
\pgfsetbuttcap%
\pgfsetroundjoin%
\definecolor{currentfill}{rgb}{0.121569,0.466667,0.705882}%
\pgfsetfillcolor{currentfill}%
\pgfsetlinewidth{1.003750pt}%
\definecolor{currentstroke}{rgb}{0.121569,0.466667,0.705882}%
\pgfsetstrokecolor{currentstroke}%
\pgfsetdash{}{0pt}%
\pgfpathmoveto{\pgfqpoint{2.077927in}{3.064868in}}%
\pgfpathcurveto{\pgfqpoint{2.088977in}{3.064868in}}{\pgfqpoint{2.099576in}{3.069259in}}{\pgfqpoint{2.107390in}{3.077072in}}%
\pgfpathcurveto{\pgfqpoint{2.115204in}{3.084886in}}{\pgfqpoint{2.119594in}{3.095485in}}{\pgfqpoint{2.119594in}{3.106535in}}%
\pgfpathcurveto{\pgfqpoint{2.119594in}{3.117585in}}{\pgfqpoint{2.115204in}{3.128184in}}{\pgfqpoint{2.107390in}{3.135998in}}%
\pgfpathcurveto{\pgfqpoint{2.099576in}{3.143811in}}{\pgfqpoint{2.088977in}{3.148202in}}{\pgfqpoint{2.077927in}{3.148202in}}%
\pgfpathcurveto{\pgfqpoint{2.066877in}{3.148202in}}{\pgfqpoint{2.056278in}{3.143811in}}{\pgfqpoint{2.048464in}{3.135998in}}%
\pgfpathcurveto{\pgfqpoint{2.040651in}{3.128184in}}{\pgfqpoint{2.036260in}{3.117585in}}{\pgfqpoint{2.036260in}{3.106535in}}%
\pgfpathcurveto{\pgfqpoint{2.036260in}{3.095485in}}{\pgfqpoint{2.040651in}{3.084886in}}{\pgfqpoint{2.048464in}{3.077072in}}%
\pgfpathcurveto{\pgfqpoint{2.056278in}{3.069259in}}{\pgfqpoint{2.066877in}{3.064868in}}{\pgfqpoint{2.077927in}{3.064868in}}%
\pgfpathclose%
\pgfusepath{stroke,fill}%
\end{pgfscope}%
\begin{pgfscope}%
\pgfpathrectangle{\pgfqpoint{0.600000in}{0.600000in}}{\pgfqpoint{3.900000in}{3.900000in}}%
\pgfusepath{clip}%
\pgfsetbuttcap%
\pgfsetroundjoin%
\definecolor{currentfill}{rgb}{0.121569,0.466667,0.705882}%
\pgfsetfillcolor{currentfill}%
\pgfsetlinewidth{1.003750pt}%
\definecolor{currentstroke}{rgb}{0.121569,0.466667,0.705882}%
\pgfsetstrokecolor{currentstroke}%
\pgfsetdash{}{0pt}%
\pgfpathmoveto{\pgfqpoint{3.424826in}{2.344525in}}%
\pgfpathcurveto{\pgfqpoint{3.435876in}{2.344525in}}{\pgfqpoint{3.446475in}{2.348916in}}{\pgfqpoint{3.454289in}{2.356729in}}%
\pgfpathcurveto{\pgfqpoint{3.462102in}{2.364543in}}{\pgfqpoint{3.466493in}{2.375142in}}{\pgfqpoint{3.466493in}{2.386192in}}%
\pgfpathcurveto{\pgfqpoint{3.466493in}{2.397242in}}{\pgfqpoint{3.462102in}{2.407841in}}{\pgfqpoint{3.454289in}{2.415655in}}%
\pgfpathcurveto{\pgfqpoint{3.446475in}{2.423468in}}{\pgfqpoint{3.435876in}{2.427859in}}{\pgfqpoint{3.424826in}{2.427859in}}%
\pgfpathcurveto{\pgfqpoint{3.413776in}{2.427859in}}{\pgfqpoint{3.403177in}{2.423468in}}{\pgfqpoint{3.395363in}{2.415655in}}%
\pgfpathcurveto{\pgfqpoint{3.387550in}{2.407841in}}{\pgfqpoint{3.383159in}{2.397242in}}{\pgfqpoint{3.383159in}{2.386192in}}%
\pgfpathcurveto{\pgfqpoint{3.383159in}{2.375142in}}{\pgfqpoint{3.387550in}{2.364543in}}{\pgfqpoint{3.395363in}{2.356729in}}%
\pgfpathcurveto{\pgfqpoint{3.403177in}{2.348916in}}{\pgfqpoint{3.413776in}{2.344525in}}{\pgfqpoint{3.424826in}{2.344525in}}%
\pgfpathclose%
\pgfusepath{stroke,fill}%
\end{pgfscope}%
\begin{pgfscope}%
\pgfpathrectangle{\pgfqpoint{0.600000in}{0.600000in}}{\pgfqpoint{3.900000in}{3.900000in}}%
\pgfusepath{clip}%
\pgfsetbuttcap%
\pgfsetroundjoin%
\definecolor{currentfill}{rgb}{0.121569,0.466667,0.705882}%
\pgfsetfillcolor{currentfill}%
\pgfsetlinewidth{1.003750pt}%
\definecolor{currentstroke}{rgb}{0.121569,0.466667,0.705882}%
\pgfsetstrokecolor{currentstroke}%
\pgfsetdash{}{0pt}%
\pgfpathmoveto{\pgfqpoint{2.753858in}{3.197299in}}%
\pgfpathcurveto{\pgfqpoint{2.764908in}{3.197299in}}{\pgfqpoint{2.775507in}{3.201689in}}{\pgfqpoint{2.783320in}{3.209503in}}%
\pgfpathcurveto{\pgfqpoint{2.791134in}{3.217317in}}{\pgfqpoint{2.795524in}{3.227916in}}{\pgfqpoint{2.795524in}{3.238966in}}%
\pgfpathcurveto{\pgfqpoint{2.795524in}{3.250016in}}{\pgfqpoint{2.791134in}{3.260615in}}{\pgfqpoint{2.783320in}{3.268429in}}%
\pgfpathcurveto{\pgfqpoint{2.775507in}{3.276242in}}{\pgfqpoint{2.764908in}{3.280633in}}{\pgfqpoint{2.753858in}{3.280633in}}%
\pgfpathcurveto{\pgfqpoint{2.742807in}{3.280633in}}{\pgfqpoint{2.732208in}{3.276242in}}{\pgfqpoint{2.724395in}{3.268429in}}%
\pgfpathcurveto{\pgfqpoint{2.716581in}{3.260615in}}{\pgfqpoint{2.712191in}{3.250016in}}{\pgfqpoint{2.712191in}{3.238966in}}%
\pgfpathcurveto{\pgfqpoint{2.712191in}{3.227916in}}{\pgfqpoint{2.716581in}{3.217317in}}{\pgfqpoint{2.724395in}{3.209503in}}%
\pgfpathcurveto{\pgfqpoint{2.732208in}{3.201689in}}{\pgfqpoint{2.742807in}{3.197299in}}{\pgfqpoint{2.753858in}{3.197299in}}%
\pgfpathclose%
\pgfusepath{stroke,fill}%
\end{pgfscope}%
\begin{pgfscope}%
\pgfpathrectangle{\pgfqpoint{0.600000in}{0.600000in}}{\pgfqpoint{3.900000in}{3.900000in}}%
\pgfusepath{clip}%
\pgfsetbuttcap%
\pgfsetroundjoin%
\definecolor{currentfill}{rgb}{0.121569,0.466667,0.705882}%
\pgfsetfillcolor{currentfill}%
\pgfsetlinewidth{1.003750pt}%
\definecolor{currentstroke}{rgb}{0.121569,0.466667,0.705882}%
\pgfsetstrokecolor{currentstroke}%
\pgfsetdash{}{0pt}%
\pgfpathmoveto{\pgfqpoint{2.926224in}{2.704946in}}%
\pgfpathcurveto{\pgfqpoint{2.937274in}{2.704946in}}{\pgfqpoint{2.947873in}{2.709337in}}{\pgfqpoint{2.955687in}{2.717150in}}%
\pgfpathcurveto{\pgfqpoint{2.963501in}{2.724964in}}{\pgfqpoint{2.967891in}{2.735563in}}{\pgfqpoint{2.967891in}{2.746613in}}%
\pgfpathcurveto{\pgfqpoint{2.967891in}{2.757663in}}{\pgfqpoint{2.963501in}{2.768262in}}{\pgfqpoint{2.955687in}{2.776076in}}%
\pgfpathcurveto{\pgfqpoint{2.947873in}{2.783890in}}{\pgfqpoint{2.937274in}{2.788280in}}{\pgfqpoint{2.926224in}{2.788280in}}%
\pgfpathcurveto{\pgfqpoint{2.915174in}{2.788280in}}{\pgfqpoint{2.904575in}{2.783890in}}{\pgfqpoint{2.896762in}{2.776076in}}%
\pgfpathcurveto{\pgfqpoint{2.888948in}{2.768262in}}{\pgfqpoint{2.884558in}{2.757663in}}{\pgfqpoint{2.884558in}{2.746613in}}%
\pgfpathcurveto{\pgfqpoint{2.884558in}{2.735563in}}{\pgfqpoint{2.888948in}{2.724964in}}{\pgfqpoint{2.896762in}{2.717150in}}%
\pgfpathcurveto{\pgfqpoint{2.904575in}{2.709337in}}{\pgfqpoint{2.915174in}{2.704946in}}{\pgfqpoint{2.926224in}{2.704946in}}%
\pgfpathclose%
\pgfusepath{stroke,fill}%
\end{pgfscope}%
\begin{pgfscope}%
\pgfpathrectangle{\pgfqpoint{0.600000in}{0.600000in}}{\pgfqpoint{3.900000in}{3.900000in}}%
\pgfusepath{clip}%
\pgfsetbuttcap%
\pgfsetroundjoin%
\definecolor{currentfill}{rgb}{0.121569,0.466667,0.705882}%
\pgfsetfillcolor{currentfill}%
\pgfsetlinewidth{1.003750pt}%
\definecolor{currentstroke}{rgb}{0.121569,0.466667,0.705882}%
\pgfsetstrokecolor{currentstroke}%
\pgfsetdash{}{0pt}%
\pgfpathmoveto{\pgfqpoint{2.305508in}{3.172425in}}%
\pgfpathcurveto{\pgfqpoint{2.316558in}{3.172425in}}{\pgfqpoint{2.327157in}{3.176815in}}{\pgfqpoint{2.334971in}{3.184629in}}%
\pgfpathcurveto{\pgfqpoint{2.342784in}{3.192442in}}{\pgfqpoint{2.347175in}{3.203041in}}{\pgfqpoint{2.347175in}{3.214091in}}%
\pgfpathcurveto{\pgfqpoint{2.347175in}{3.225142in}}{\pgfqpoint{2.342784in}{3.235741in}}{\pgfqpoint{2.334971in}{3.243554in}}%
\pgfpathcurveto{\pgfqpoint{2.327157in}{3.251368in}}{\pgfqpoint{2.316558in}{3.255758in}}{\pgfqpoint{2.305508in}{3.255758in}}%
\pgfpathcurveto{\pgfqpoint{2.294458in}{3.255758in}}{\pgfqpoint{2.283859in}{3.251368in}}{\pgfqpoint{2.276045in}{3.243554in}}%
\pgfpathcurveto{\pgfqpoint{2.268232in}{3.235741in}}{\pgfqpoint{2.263841in}{3.225142in}}{\pgfqpoint{2.263841in}{3.214091in}}%
\pgfpathcurveto{\pgfqpoint{2.263841in}{3.203041in}}{\pgfqpoint{2.268232in}{3.192442in}}{\pgfqpoint{2.276045in}{3.184629in}}%
\pgfpathcurveto{\pgfqpoint{2.283859in}{3.176815in}}{\pgfqpoint{2.294458in}{3.172425in}}{\pgfqpoint{2.305508in}{3.172425in}}%
\pgfpathclose%
\pgfusepath{stroke,fill}%
\end{pgfscope}%
\begin{pgfscope}%
\pgfpathrectangle{\pgfqpoint{0.600000in}{0.600000in}}{\pgfqpoint{3.900000in}{3.900000in}}%
\pgfusepath{clip}%
\pgfsetbuttcap%
\pgfsetroundjoin%
\definecolor{currentfill}{rgb}{0.121569,0.466667,0.705882}%
\pgfsetfillcolor{currentfill}%
\pgfsetlinewidth{1.003750pt}%
\definecolor{currentstroke}{rgb}{0.121569,0.466667,0.705882}%
\pgfsetstrokecolor{currentstroke}%
\pgfsetdash{}{0pt}%
\pgfpathmoveto{\pgfqpoint{2.863895in}{2.102648in}}%
\pgfpathcurveto{\pgfqpoint{2.874945in}{2.102648in}}{\pgfqpoint{2.885544in}{2.107038in}}{\pgfqpoint{2.893357in}{2.114852in}}%
\pgfpathcurveto{\pgfqpoint{2.901171in}{2.122666in}}{\pgfqpoint{2.905561in}{2.133265in}}{\pgfqpoint{2.905561in}{2.144315in}}%
\pgfpathcurveto{\pgfqpoint{2.905561in}{2.155365in}}{\pgfqpoint{2.901171in}{2.165964in}}{\pgfqpoint{2.893357in}{2.173778in}}%
\pgfpathcurveto{\pgfqpoint{2.885544in}{2.181591in}}{\pgfqpoint{2.874945in}{2.185981in}}{\pgfqpoint{2.863895in}{2.185981in}}%
\pgfpathcurveto{\pgfqpoint{2.852845in}{2.185981in}}{\pgfqpoint{2.842246in}{2.181591in}}{\pgfqpoint{2.834432in}{2.173778in}}%
\pgfpathcurveto{\pgfqpoint{2.826618in}{2.165964in}}{\pgfqpoint{2.822228in}{2.155365in}}{\pgfqpoint{2.822228in}{2.144315in}}%
\pgfpathcurveto{\pgfqpoint{2.822228in}{2.133265in}}{\pgfqpoint{2.826618in}{2.122666in}}{\pgfqpoint{2.834432in}{2.114852in}}%
\pgfpathcurveto{\pgfqpoint{2.842246in}{2.107038in}}{\pgfqpoint{2.852845in}{2.102648in}}{\pgfqpoint{2.863895in}{2.102648in}}%
\pgfpathclose%
\pgfusepath{stroke,fill}%
\end{pgfscope}%
\begin{pgfscope}%
\pgfpathrectangle{\pgfqpoint{0.600000in}{0.600000in}}{\pgfqpoint{3.900000in}{3.900000in}}%
\pgfusepath{clip}%
\pgfsetbuttcap%
\pgfsetroundjoin%
\definecolor{currentfill}{rgb}{0.121569,0.466667,0.705882}%
\pgfsetfillcolor{currentfill}%
\pgfsetlinewidth{1.003750pt}%
\definecolor{currentstroke}{rgb}{0.121569,0.466667,0.705882}%
\pgfsetstrokecolor{currentstroke}%
\pgfsetdash{}{0pt}%
\pgfpathmoveto{\pgfqpoint{1.965453in}{2.375873in}}%
\pgfpathcurveto{\pgfqpoint{1.976503in}{2.375873in}}{\pgfqpoint{1.987102in}{2.380264in}}{\pgfqpoint{1.994916in}{2.388077in}}%
\pgfpathcurveto{\pgfqpoint{2.002729in}{2.395891in}}{\pgfqpoint{2.007120in}{2.406490in}}{\pgfqpoint{2.007120in}{2.417540in}}%
\pgfpathcurveto{\pgfqpoint{2.007120in}{2.428590in}}{\pgfqpoint{2.002729in}{2.439189in}}{\pgfqpoint{1.994916in}{2.447003in}}%
\pgfpathcurveto{\pgfqpoint{1.987102in}{2.454817in}}{\pgfqpoint{1.976503in}{2.459207in}}{\pgfqpoint{1.965453in}{2.459207in}}%
\pgfpathcurveto{\pgfqpoint{1.954403in}{2.459207in}}{\pgfqpoint{1.943804in}{2.454817in}}{\pgfqpoint{1.935990in}{2.447003in}}%
\pgfpathcurveto{\pgfqpoint{1.928177in}{2.439189in}}{\pgfqpoint{1.923786in}{2.428590in}}{\pgfqpoint{1.923786in}{2.417540in}}%
\pgfpathcurveto{\pgfqpoint{1.923786in}{2.406490in}}{\pgfqpoint{1.928177in}{2.395891in}}{\pgfqpoint{1.935990in}{2.388077in}}%
\pgfpathcurveto{\pgfqpoint{1.943804in}{2.380264in}}{\pgfqpoint{1.954403in}{2.375873in}}{\pgfqpoint{1.965453in}{2.375873in}}%
\pgfpathclose%
\pgfusepath{stroke,fill}%
\end{pgfscope}%
\begin{pgfscope}%
\pgfpathrectangle{\pgfqpoint{0.600000in}{0.600000in}}{\pgfqpoint{3.900000in}{3.900000in}}%
\pgfusepath{clip}%
\pgfsetbuttcap%
\pgfsetroundjoin%
\definecolor{currentfill}{rgb}{0.121569,0.466667,0.705882}%
\pgfsetfillcolor{currentfill}%
\pgfsetlinewidth{1.003750pt}%
\definecolor{currentstroke}{rgb}{0.121569,0.466667,0.705882}%
\pgfsetstrokecolor{currentstroke}%
\pgfsetdash{}{0pt}%
\pgfpathmoveto{\pgfqpoint{2.632149in}{2.091518in}}%
\pgfpathcurveto{\pgfqpoint{2.643200in}{2.091518in}}{\pgfqpoint{2.653799in}{2.095908in}}{\pgfqpoint{2.661612in}{2.103722in}}%
\pgfpathcurveto{\pgfqpoint{2.669426in}{2.111535in}}{\pgfqpoint{2.673816in}{2.122134in}}{\pgfqpoint{2.673816in}{2.133185in}}%
\pgfpathcurveto{\pgfqpoint{2.673816in}{2.144235in}}{\pgfqpoint{2.669426in}{2.154834in}}{\pgfqpoint{2.661612in}{2.162647in}}%
\pgfpathcurveto{\pgfqpoint{2.653799in}{2.170461in}}{\pgfqpoint{2.643200in}{2.174851in}}{\pgfqpoint{2.632149in}{2.174851in}}%
\pgfpathcurveto{\pgfqpoint{2.621099in}{2.174851in}}{\pgfqpoint{2.610500in}{2.170461in}}{\pgfqpoint{2.602687in}{2.162647in}}%
\pgfpathcurveto{\pgfqpoint{2.594873in}{2.154834in}}{\pgfqpoint{2.590483in}{2.144235in}}{\pgfqpoint{2.590483in}{2.133185in}}%
\pgfpathcurveto{\pgfqpoint{2.590483in}{2.122134in}}{\pgfqpoint{2.594873in}{2.111535in}}{\pgfqpoint{2.602687in}{2.103722in}}%
\pgfpathcurveto{\pgfqpoint{2.610500in}{2.095908in}}{\pgfqpoint{2.621099in}{2.091518in}}{\pgfqpoint{2.632149in}{2.091518in}}%
\pgfpathclose%
\pgfusepath{stroke,fill}%
\end{pgfscope}%
\begin{pgfscope}%
\pgfpathrectangle{\pgfqpoint{0.600000in}{0.600000in}}{\pgfqpoint{3.900000in}{3.900000in}}%
\pgfusepath{clip}%
\pgfsetbuttcap%
\pgfsetroundjoin%
\definecolor{currentfill}{rgb}{0.121569,0.466667,0.705882}%
\pgfsetfillcolor{currentfill}%
\pgfsetlinewidth{1.003750pt}%
\definecolor{currentstroke}{rgb}{0.121569,0.466667,0.705882}%
\pgfsetstrokecolor{currentstroke}%
\pgfsetdash{}{0pt}%
\pgfpathmoveto{\pgfqpoint{2.731719in}{2.457542in}}%
\pgfpathcurveto{\pgfqpoint{2.742769in}{2.457542in}}{\pgfqpoint{2.753368in}{2.461932in}}{\pgfqpoint{2.761181in}{2.469746in}}%
\pgfpathcurveto{\pgfqpoint{2.768995in}{2.477559in}}{\pgfqpoint{2.773385in}{2.488158in}}{\pgfqpoint{2.773385in}{2.499208in}}%
\pgfpathcurveto{\pgfqpoint{2.773385in}{2.510259in}}{\pgfqpoint{2.768995in}{2.520858in}}{\pgfqpoint{2.761181in}{2.528671in}}%
\pgfpathcurveto{\pgfqpoint{2.753368in}{2.536485in}}{\pgfqpoint{2.742769in}{2.540875in}}{\pgfqpoint{2.731719in}{2.540875in}}%
\pgfpathcurveto{\pgfqpoint{2.720668in}{2.540875in}}{\pgfqpoint{2.710069in}{2.536485in}}{\pgfqpoint{2.702256in}{2.528671in}}%
\pgfpathcurveto{\pgfqpoint{2.694442in}{2.520858in}}{\pgfqpoint{2.690052in}{2.510259in}}{\pgfqpoint{2.690052in}{2.499208in}}%
\pgfpathcurveto{\pgfqpoint{2.690052in}{2.488158in}}{\pgfqpoint{2.694442in}{2.477559in}}{\pgfqpoint{2.702256in}{2.469746in}}%
\pgfpathcurveto{\pgfqpoint{2.710069in}{2.461932in}}{\pgfqpoint{2.720668in}{2.457542in}}{\pgfqpoint{2.731719in}{2.457542in}}%
\pgfpathclose%
\pgfusepath{stroke,fill}%
\end{pgfscope}%
\begin{pgfscope}%
\pgfpathrectangle{\pgfqpoint{0.600000in}{0.600000in}}{\pgfqpoint{3.900000in}{3.900000in}}%
\pgfusepath{clip}%
\pgfsetbuttcap%
\pgfsetroundjoin%
\definecolor{currentfill}{rgb}{0.121569,0.466667,0.705882}%
\pgfsetfillcolor{currentfill}%
\pgfsetlinewidth{1.003750pt}%
\definecolor{currentstroke}{rgb}{0.121569,0.466667,0.705882}%
\pgfsetstrokecolor{currentstroke}%
\pgfsetdash{}{0pt}%
\pgfpathmoveto{\pgfqpoint{2.917177in}{3.064061in}}%
\pgfpathcurveto{\pgfqpoint{2.928227in}{3.064061in}}{\pgfqpoint{2.938826in}{3.068452in}}{\pgfqpoint{2.946639in}{3.076265in}}%
\pgfpathcurveto{\pgfqpoint{2.954453in}{3.084079in}}{\pgfqpoint{2.958843in}{3.094678in}}{\pgfqpoint{2.958843in}{3.105728in}}%
\pgfpathcurveto{\pgfqpoint{2.958843in}{3.116778in}}{\pgfqpoint{2.954453in}{3.127377in}}{\pgfqpoint{2.946639in}{3.135191in}}%
\pgfpathcurveto{\pgfqpoint{2.938826in}{3.143004in}}{\pgfqpoint{2.928227in}{3.147395in}}{\pgfqpoint{2.917177in}{3.147395in}}%
\pgfpathcurveto{\pgfqpoint{2.906126in}{3.147395in}}{\pgfqpoint{2.895527in}{3.143004in}}{\pgfqpoint{2.887714in}{3.135191in}}%
\pgfpathcurveto{\pgfqpoint{2.879900in}{3.127377in}}{\pgfqpoint{2.875510in}{3.116778in}}{\pgfqpoint{2.875510in}{3.105728in}}%
\pgfpathcurveto{\pgfqpoint{2.875510in}{3.094678in}}{\pgfqpoint{2.879900in}{3.084079in}}{\pgfqpoint{2.887714in}{3.076265in}}%
\pgfpathcurveto{\pgfqpoint{2.895527in}{3.068452in}}{\pgfqpoint{2.906126in}{3.064061in}}{\pgfqpoint{2.917177in}{3.064061in}}%
\pgfpathclose%
\pgfusepath{stroke,fill}%
\end{pgfscope}%
\begin{pgfscope}%
\pgfpathrectangle{\pgfqpoint{0.600000in}{0.600000in}}{\pgfqpoint{3.900000in}{3.900000in}}%
\pgfusepath{clip}%
\pgfsetbuttcap%
\pgfsetroundjoin%
\definecolor{currentfill}{rgb}{0.121569,0.466667,0.705882}%
\pgfsetfillcolor{currentfill}%
\pgfsetlinewidth{1.003750pt}%
\definecolor{currentstroke}{rgb}{0.121569,0.466667,0.705882}%
\pgfsetstrokecolor{currentstroke}%
\pgfsetdash{}{0pt}%
\pgfpathmoveto{\pgfqpoint{1.336588in}{3.511151in}}%
\pgfpathcurveto{\pgfqpoint{1.347638in}{3.511151in}}{\pgfqpoint{1.358237in}{3.515541in}}{\pgfqpoint{1.366051in}{3.523355in}}%
\pgfpathcurveto{\pgfqpoint{1.373864in}{3.531168in}}{\pgfqpoint{1.378255in}{3.541767in}}{\pgfqpoint{1.378255in}{3.552817in}}%
\pgfpathcurveto{\pgfqpoint{1.378255in}{3.563867in}}{\pgfqpoint{1.373864in}{3.574467in}}{\pgfqpoint{1.366051in}{3.582280in}}%
\pgfpathcurveto{\pgfqpoint{1.358237in}{3.590094in}}{\pgfqpoint{1.347638in}{3.594484in}}{\pgfqpoint{1.336588in}{3.594484in}}%
\pgfpathcurveto{\pgfqpoint{1.325538in}{3.594484in}}{\pgfqpoint{1.314939in}{3.590094in}}{\pgfqpoint{1.307125in}{3.582280in}}%
\pgfpathcurveto{\pgfqpoint{1.299312in}{3.574467in}}{\pgfqpoint{1.294921in}{3.563867in}}{\pgfqpoint{1.294921in}{3.552817in}}%
\pgfpathcurveto{\pgfqpoint{1.294921in}{3.541767in}}{\pgfqpoint{1.299312in}{3.531168in}}{\pgfqpoint{1.307125in}{3.523355in}}%
\pgfpathcurveto{\pgfqpoint{1.314939in}{3.515541in}}{\pgfqpoint{1.325538in}{3.511151in}}{\pgfqpoint{1.336588in}{3.511151in}}%
\pgfpathclose%
\pgfusepath{stroke,fill}%
\end{pgfscope}%
\begin{pgfscope}%
\pgfpathrectangle{\pgfqpoint{0.600000in}{0.600000in}}{\pgfqpoint{3.900000in}{3.900000in}}%
\pgfusepath{clip}%
\pgfsetbuttcap%
\pgfsetroundjoin%
\definecolor{currentfill}{rgb}{0.121569,0.466667,0.705882}%
\pgfsetfillcolor{currentfill}%
\pgfsetlinewidth{1.003750pt}%
\definecolor{currentstroke}{rgb}{0.121569,0.466667,0.705882}%
\pgfsetstrokecolor{currentstroke}%
\pgfsetdash{}{0pt}%
\pgfpathmoveto{\pgfqpoint{2.815073in}{2.653351in}}%
\pgfpathcurveto{\pgfqpoint{2.826123in}{2.653351in}}{\pgfqpoint{2.836722in}{2.657742in}}{\pgfqpoint{2.844535in}{2.665555in}}%
\pgfpathcurveto{\pgfqpoint{2.852349in}{2.673369in}}{\pgfqpoint{2.856739in}{2.683968in}}{\pgfqpoint{2.856739in}{2.695018in}}%
\pgfpathcurveto{\pgfqpoint{2.856739in}{2.706068in}}{\pgfqpoint{2.852349in}{2.716667in}}{\pgfqpoint{2.844535in}{2.724481in}}%
\pgfpathcurveto{\pgfqpoint{2.836722in}{2.732294in}}{\pgfqpoint{2.826123in}{2.736685in}}{\pgfqpoint{2.815073in}{2.736685in}}%
\pgfpathcurveto{\pgfqpoint{2.804022in}{2.736685in}}{\pgfqpoint{2.793423in}{2.732294in}}{\pgfqpoint{2.785610in}{2.724481in}}%
\pgfpathcurveto{\pgfqpoint{2.777796in}{2.716667in}}{\pgfqpoint{2.773406in}{2.706068in}}{\pgfqpoint{2.773406in}{2.695018in}}%
\pgfpathcurveto{\pgfqpoint{2.773406in}{2.683968in}}{\pgfqpoint{2.777796in}{2.673369in}}{\pgfqpoint{2.785610in}{2.665555in}}%
\pgfpathcurveto{\pgfqpoint{2.793423in}{2.657742in}}{\pgfqpoint{2.804022in}{2.653351in}}{\pgfqpoint{2.815073in}{2.653351in}}%
\pgfpathclose%
\pgfusepath{stroke,fill}%
\end{pgfscope}%
\begin{pgfscope}%
\pgfpathrectangle{\pgfqpoint{0.600000in}{0.600000in}}{\pgfqpoint{3.900000in}{3.900000in}}%
\pgfusepath{clip}%
\pgfsetbuttcap%
\pgfsetroundjoin%
\definecolor{currentfill}{rgb}{0.121569,0.466667,0.705882}%
\pgfsetfillcolor{currentfill}%
\pgfsetlinewidth{1.003750pt}%
\definecolor{currentstroke}{rgb}{0.121569,0.466667,0.705882}%
\pgfsetstrokecolor{currentstroke}%
\pgfsetdash{}{0pt}%
\pgfpathmoveto{\pgfqpoint{2.644697in}{2.090244in}}%
\pgfpathcurveto{\pgfqpoint{2.655747in}{2.090244in}}{\pgfqpoint{2.666346in}{2.094634in}}{\pgfqpoint{2.674160in}{2.102448in}}%
\pgfpathcurveto{\pgfqpoint{2.681973in}{2.110262in}}{\pgfqpoint{2.686363in}{2.120861in}}{\pgfqpoint{2.686363in}{2.131911in}}%
\pgfpathcurveto{\pgfqpoint{2.686363in}{2.142961in}}{\pgfqpoint{2.681973in}{2.153560in}}{\pgfqpoint{2.674160in}{2.161374in}}%
\pgfpathcurveto{\pgfqpoint{2.666346in}{2.169187in}}{\pgfqpoint{2.655747in}{2.173577in}}{\pgfqpoint{2.644697in}{2.173577in}}%
\pgfpathcurveto{\pgfqpoint{2.633647in}{2.173577in}}{\pgfqpoint{2.623048in}{2.169187in}}{\pgfqpoint{2.615234in}{2.161374in}}%
\pgfpathcurveto{\pgfqpoint{2.607420in}{2.153560in}}{\pgfqpoint{2.603030in}{2.142961in}}{\pgfqpoint{2.603030in}{2.131911in}}%
\pgfpathcurveto{\pgfqpoint{2.603030in}{2.120861in}}{\pgfqpoint{2.607420in}{2.110262in}}{\pgfqpoint{2.615234in}{2.102448in}}%
\pgfpathcurveto{\pgfqpoint{2.623048in}{2.094634in}}{\pgfqpoint{2.633647in}{2.090244in}}{\pgfqpoint{2.644697in}{2.090244in}}%
\pgfpathclose%
\pgfusepath{stroke,fill}%
\end{pgfscope}%
\begin{pgfscope}%
\pgfpathrectangle{\pgfqpoint{0.600000in}{0.600000in}}{\pgfqpoint{3.900000in}{3.900000in}}%
\pgfusepath{clip}%
\pgfsetbuttcap%
\pgfsetroundjoin%
\definecolor{currentfill}{rgb}{0.121569,0.466667,0.705882}%
\pgfsetfillcolor{currentfill}%
\pgfsetlinewidth{1.003750pt}%
\definecolor{currentstroke}{rgb}{0.121569,0.466667,0.705882}%
\pgfsetstrokecolor{currentstroke}%
\pgfsetdash{}{0pt}%
\pgfpathmoveto{\pgfqpoint{2.366243in}{2.777021in}}%
\pgfpathcurveto{\pgfqpoint{2.377293in}{2.777021in}}{\pgfqpoint{2.387892in}{2.781411in}}{\pgfqpoint{2.395706in}{2.789225in}}%
\pgfpathcurveto{\pgfqpoint{2.403519in}{2.797039in}}{\pgfqpoint{2.407910in}{2.807638in}}{\pgfqpoint{2.407910in}{2.818688in}}%
\pgfpathcurveto{\pgfqpoint{2.407910in}{2.829738in}}{\pgfqpoint{2.403519in}{2.840337in}}{\pgfqpoint{2.395706in}{2.848150in}}%
\pgfpathcurveto{\pgfqpoint{2.387892in}{2.855964in}}{\pgfqpoint{2.377293in}{2.860354in}}{\pgfqpoint{2.366243in}{2.860354in}}%
\pgfpathcurveto{\pgfqpoint{2.355193in}{2.860354in}}{\pgfqpoint{2.344594in}{2.855964in}}{\pgfqpoint{2.336780in}{2.848150in}}%
\pgfpathcurveto{\pgfqpoint{2.328967in}{2.840337in}}{\pgfqpoint{2.324576in}{2.829738in}}{\pgfqpoint{2.324576in}{2.818688in}}%
\pgfpathcurveto{\pgfqpoint{2.324576in}{2.807638in}}{\pgfqpoint{2.328967in}{2.797039in}}{\pgfqpoint{2.336780in}{2.789225in}}%
\pgfpathcurveto{\pgfqpoint{2.344594in}{2.781411in}}{\pgfqpoint{2.355193in}{2.777021in}}{\pgfqpoint{2.366243in}{2.777021in}}%
\pgfpathclose%
\pgfusepath{stroke,fill}%
\end{pgfscope}%
\begin{pgfscope}%
\pgfpathrectangle{\pgfqpoint{0.600000in}{0.600000in}}{\pgfqpoint{3.900000in}{3.900000in}}%
\pgfusepath{clip}%
\pgfsetbuttcap%
\pgfsetroundjoin%
\definecolor{currentfill}{rgb}{0.121569,0.466667,0.705882}%
\pgfsetfillcolor{currentfill}%
\pgfsetlinewidth{1.003750pt}%
\definecolor{currentstroke}{rgb}{0.121569,0.466667,0.705882}%
\pgfsetstrokecolor{currentstroke}%
\pgfsetdash{}{0pt}%
\pgfpathmoveto{\pgfqpoint{2.603718in}{2.076709in}}%
\pgfpathcurveto{\pgfqpoint{2.614768in}{2.076709in}}{\pgfqpoint{2.625367in}{2.081099in}}{\pgfqpoint{2.633180in}{2.088913in}}%
\pgfpathcurveto{\pgfqpoint{2.640994in}{2.096727in}}{\pgfqpoint{2.645384in}{2.107326in}}{\pgfqpoint{2.645384in}{2.118376in}}%
\pgfpathcurveto{\pgfqpoint{2.645384in}{2.129426in}}{\pgfqpoint{2.640994in}{2.140025in}}{\pgfqpoint{2.633180in}{2.147839in}}%
\pgfpathcurveto{\pgfqpoint{2.625367in}{2.155652in}}{\pgfqpoint{2.614768in}{2.160042in}}{\pgfqpoint{2.603718in}{2.160042in}}%
\pgfpathcurveto{\pgfqpoint{2.592667in}{2.160042in}}{\pgfqpoint{2.582068in}{2.155652in}}{\pgfqpoint{2.574255in}{2.147839in}}%
\pgfpathcurveto{\pgfqpoint{2.566441in}{2.140025in}}{\pgfqpoint{2.562051in}{2.129426in}}{\pgfqpoint{2.562051in}{2.118376in}}%
\pgfpathcurveto{\pgfqpoint{2.562051in}{2.107326in}}{\pgfqpoint{2.566441in}{2.096727in}}{\pgfqpoint{2.574255in}{2.088913in}}%
\pgfpathcurveto{\pgfqpoint{2.582068in}{2.081099in}}{\pgfqpoint{2.592667in}{2.076709in}}{\pgfqpoint{2.603718in}{2.076709in}}%
\pgfpathclose%
\pgfusepath{stroke,fill}%
\end{pgfscope}%
\begin{pgfscope}%
\pgfpathrectangle{\pgfqpoint{0.600000in}{0.600000in}}{\pgfqpoint{3.900000in}{3.900000in}}%
\pgfusepath{clip}%
\pgfsetbuttcap%
\pgfsetroundjoin%
\definecolor{currentfill}{rgb}{0.121569,0.466667,0.705882}%
\pgfsetfillcolor{currentfill}%
\pgfsetlinewidth{1.003750pt}%
\definecolor{currentstroke}{rgb}{0.121569,0.466667,0.705882}%
\pgfsetstrokecolor{currentstroke}%
\pgfsetdash{}{0pt}%
\pgfpathmoveto{\pgfqpoint{2.779576in}{2.667071in}}%
\pgfpathcurveto{\pgfqpoint{2.790626in}{2.667071in}}{\pgfqpoint{2.801226in}{2.671461in}}{\pgfqpoint{2.809039in}{2.679275in}}%
\pgfpathcurveto{\pgfqpoint{2.816853in}{2.687089in}}{\pgfqpoint{2.821243in}{2.697688in}}{\pgfqpoint{2.821243in}{2.708738in}}%
\pgfpathcurveto{\pgfqpoint{2.821243in}{2.719788in}}{\pgfqpoint{2.816853in}{2.730387in}}{\pgfqpoint{2.809039in}{2.738201in}}%
\pgfpathcurveto{\pgfqpoint{2.801226in}{2.746014in}}{\pgfqpoint{2.790626in}{2.750405in}}{\pgfqpoint{2.779576in}{2.750405in}}%
\pgfpathcurveto{\pgfqpoint{2.768526in}{2.750405in}}{\pgfqpoint{2.757927in}{2.746014in}}{\pgfqpoint{2.750114in}{2.738201in}}%
\pgfpathcurveto{\pgfqpoint{2.742300in}{2.730387in}}{\pgfqpoint{2.737910in}{2.719788in}}{\pgfqpoint{2.737910in}{2.708738in}}%
\pgfpathcurveto{\pgfqpoint{2.737910in}{2.697688in}}{\pgfqpoint{2.742300in}{2.687089in}}{\pgfqpoint{2.750114in}{2.679275in}}%
\pgfpathcurveto{\pgfqpoint{2.757927in}{2.671461in}}{\pgfqpoint{2.768526in}{2.667071in}}{\pgfqpoint{2.779576in}{2.667071in}}%
\pgfpathclose%
\pgfusepath{stroke,fill}%
\end{pgfscope}%
\begin{pgfscope}%
\pgfpathrectangle{\pgfqpoint{0.600000in}{0.600000in}}{\pgfqpoint{3.900000in}{3.900000in}}%
\pgfusepath{clip}%
\pgfsetbuttcap%
\pgfsetroundjoin%
\definecolor{currentfill}{rgb}{0.121569,0.466667,0.705882}%
\pgfsetfillcolor{currentfill}%
\pgfsetlinewidth{1.003750pt}%
\definecolor{currentstroke}{rgb}{0.121569,0.466667,0.705882}%
\pgfsetstrokecolor{currentstroke}%
\pgfsetdash{}{0pt}%
\pgfpathmoveto{\pgfqpoint{2.721067in}{2.763368in}}%
\pgfpathcurveto{\pgfqpoint{2.732117in}{2.763368in}}{\pgfqpoint{2.742716in}{2.767758in}}{\pgfqpoint{2.750530in}{2.775572in}}%
\pgfpathcurveto{\pgfqpoint{2.758344in}{2.783385in}}{\pgfqpoint{2.762734in}{2.793984in}}{\pgfqpoint{2.762734in}{2.805035in}}%
\pgfpathcurveto{\pgfqpoint{2.762734in}{2.816085in}}{\pgfqpoint{2.758344in}{2.826684in}}{\pgfqpoint{2.750530in}{2.834497in}}%
\pgfpathcurveto{\pgfqpoint{2.742716in}{2.842311in}}{\pgfqpoint{2.732117in}{2.846701in}}{\pgfqpoint{2.721067in}{2.846701in}}%
\pgfpathcurveto{\pgfqpoint{2.710017in}{2.846701in}}{\pgfqpoint{2.699418in}{2.842311in}}{\pgfqpoint{2.691605in}{2.834497in}}%
\pgfpathcurveto{\pgfqpoint{2.683791in}{2.826684in}}{\pgfqpoint{2.679401in}{2.816085in}}{\pgfqpoint{2.679401in}{2.805035in}}%
\pgfpathcurveto{\pgfqpoint{2.679401in}{2.793984in}}{\pgfqpoint{2.683791in}{2.783385in}}{\pgfqpoint{2.691605in}{2.775572in}}%
\pgfpathcurveto{\pgfqpoint{2.699418in}{2.767758in}}{\pgfqpoint{2.710017in}{2.763368in}}{\pgfqpoint{2.721067in}{2.763368in}}%
\pgfpathclose%
\pgfusepath{stroke,fill}%
\end{pgfscope}%
\begin{pgfscope}%
\pgfpathrectangle{\pgfqpoint{0.600000in}{0.600000in}}{\pgfqpoint{3.900000in}{3.900000in}}%
\pgfusepath{clip}%
\pgfsetbuttcap%
\pgfsetroundjoin%
\definecolor{currentfill}{rgb}{0.121569,0.466667,0.705882}%
\pgfsetfillcolor{currentfill}%
\pgfsetlinewidth{1.003750pt}%
\definecolor{currentstroke}{rgb}{0.121569,0.466667,0.705882}%
\pgfsetstrokecolor{currentstroke}%
\pgfsetdash{}{0pt}%
\pgfpathmoveto{\pgfqpoint{3.510090in}{2.231243in}}%
\pgfpathcurveto{\pgfqpoint{3.521141in}{2.231243in}}{\pgfqpoint{3.531740in}{2.235633in}}{\pgfqpoint{3.539553in}{2.243447in}}%
\pgfpathcurveto{\pgfqpoint{3.547367in}{2.251260in}}{\pgfqpoint{3.551757in}{2.261859in}}{\pgfqpoint{3.551757in}{2.272909in}}%
\pgfpathcurveto{\pgfqpoint{3.551757in}{2.283960in}}{\pgfqpoint{3.547367in}{2.294559in}}{\pgfqpoint{3.539553in}{2.302372in}}%
\pgfpathcurveto{\pgfqpoint{3.531740in}{2.310186in}}{\pgfqpoint{3.521141in}{2.314576in}}{\pgfqpoint{3.510090in}{2.314576in}}%
\pgfpathcurveto{\pgfqpoint{3.499040in}{2.314576in}}{\pgfqpoint{3.488441in}{2.310186in}}{\pgfqpoint{3.480628in}{2.302372in}}%
\pgfpathcurveto{\pgfqpoint{3.472814in}{2.294559in}}{\pgfqpoint{3.468424in}{2.283960in}}{\pgfqpoint{3.468424in}{2.272909in}}%
\pgfpathcurveto{\pgfqpoint{3.468424in}{2.261859in}}{\pgfqpoint{3.472814in}{2.251260in}}{\pgfqpoint{3.480628in}{2.243447in}}%
\pgfpathcurveto{\pgfqpoint{3.488441in}{2.235633in}}{\pgfqpoint{3.499040in}{2.231243in}}{\pgfqpoint{3.510090in}{2.231243in}}%
\pgfpathclose%
\pgfusepath{stroke,fill}%
\end{pgfscope}%
\begin{pgfscope}%
\pgfpathrectangle{\pgfqpoint{0.600000in}{0.600000in}}{\pgfqpoint{3.900000in}{3.900000in}}%
\pgfusepath{clip}%
\pgfsetbuttcap%
\pgfsetroundjoin%
\definecolor{currentfill}{rgb}{0.121569,0.466667,0.705882}%
\pgfsetfillcolor{currentfill}%
\pgfsetlinewidth{1.003750pt}%
\definecolor{currentstroke}{rgb}{0.121569,0.466667,0.705882}%
\pgfsetstrokecolor{currentstroke}%
\pgfsetdash{}{0pt}%
\pgfpathmoveto{\pgfqpoint{2.859968in}{2.352795in}}%
\pgfpathcurveto{\pgfqpoint{2.871018in}{2.352795in}}{\pgfqpoint{2.881617in}{2.357186in}}{\pgfqpoint{2.889430in}{2.364999in}}%
\pgfpathcurveto{\pgfqpoint{2.897244in}{2.372813in}}{\pgfqpoint{2.901634in}{2.383412in}}{\pgfqpoint{2.901634in}{2.394462in}}%
\pgfpathcurveto{\pgfqpoint{2.901634in}{2.405512in}}{\pgfqpoint{2.897244in}{2.416111in}}{\pgfqpoint{2.889430in}{2.423925in}}%
\pgfpathcurveto{\pgfqpoint{2.881617in}{2.431738in}}{\pgfqpoint{2.871018in}{2.436129in}}{\pgfqpoint{2.859968in}{2.436129in}}%
\pgfpathcurveto{\pgfqpoint{2.848917in}{2.436129in}}{\pgfqpoint{2.838318in}{2.431738in}}{\pgfqpoint{2.830505in}{2.423925in}}%
\pgfpathcurveto{\pgfqpoint{2.822691in}{2.416111in}}{\pgfqpoint{2.818301in}{2.405512in}}{\pgfqpoint{2.818301in}{2.394462in}}%
\pgfpathcurveto{\pgfqpoint{2.818301in}{2.383412in}}{\pgfqpoint{2.822691in}{2.372813in}}{\pgfqpoint{2.830505in}{2.364999in}}%
\pgfpathcurveto{\pgfqpoint{2.838318in}{2.357186in}}{\pgfqpoint{2.848917in}{2.352795in}}{\pgfqpoint{2.859968in}{2.352795in}}%
\pgfpathclose%
\pgfusepath{stroke,fill}%
\end{pgfscope}%
\begin{pgfscope}%
\pgfpathrectangle{\pgfqpoint{0.600000in}{0.600000in}}{\pgfqpoint{3.900000in}{3.900000in}}%
\pgfusepath{clip}%
\pgfsetbuttcap%
\pgfsetroundjoin%
\definecolor{currentfill}{rgb}{0.121569,0.466667,0.705882}%
\pgfsetfillcolor{currentfill}%
\pgfsetlinewidth{1.003750pt}%
\definecolor{currentstroke}{rgb}{0.121569,0.466667,0.705882}%
\pgfsetstrokecolor{currentstroke}%
\pgfsetdash{}{0pt}%
\pgfpathmoveto{\pgfqpoint{3.035476in}{1.928331in}}%
\pgfpathcurveto{\pgfqpoint{3.046526in}{1.928331in}}{\pgfqpoint{3.057125in}{1.932722in}}{\pgfqpoint{3.064938in}{1.940535in}}%
\pgfpathcurveto{\pgfqpoint{3.072752in}{1.948349in}}{\pgfqpoint{3.077142in}{1.958948in}}{\pgfqpoint{3.077142in}{1.969998in}}%
\pgfpathcurveto{\pgfqpoint{3.077142in}{1.981048in}}{\pgfqpoint{3.072752in}{1.991647in}}{\pgfqpoint{3.064938in}{1.999461in}}%
\pgfpathcurveto{\pgfqpoint{3.057125in}{2.007274in}}{\pgfqpoint{3.046526in}{2.011665in}}{\pgfqpoint{3.035476in}{2.011665in}}%
\pgfpathcurveto{\pgfqpoint{3.024426in}{2.011665in}}{\pgfqpoint{3.013827in}{2.007274in}}{\pgfqpoint{3.006013in}{1.999461in}}%
\pgfpathcurveto{\pgfqpoint{2.998199in}{1.991647in}}{\pgfqpoint{2.993809in}{1.981048in}}{\pgfqpoint{2.993809in}{1.969998in}}%
\pgfpathcurveto{\pgfqpoint{2.993809in}{1.958948in}}{\pgfqpoint{2.998199in}{1.948349in}}{\pgfqpoint{3.006013in}{1.940535in}}%
\pgfpathcurveto{\pgfqpoint{3.013827in}{1.932722in}}{\pgfqpoint{3.024426in}{1.928331in}}{\pgfqpoint{3.035476in}{1.928331in}}%
\pgfpathclose%
\pgfusepath{stroke,fill}%
\end{pgfscope}%
\begin{pgfscope}%
\pgfpathrectangle{\pgfqpoint{0.600000in}{0.600000in}}{\pgfqpoint{3.900000in}{3.900000in}}%
\pgfusepath{clip}%
\pgfsetbuttcap%
\pgfsetroundjoin%
\definecolor{currentfill}{rgb}{0.121569,0.466667,0.705882}%
\pgfsetfillcolor{currentfill}%
\pgfsetlinewidth{1.003750pt}%
\definecolor{currentstroke}{rgb}{0.121569,0.466667,0.705882}%
\pgfsetstrokecolor{currentstroke}%
\pgfsetdash{}{0pt}%
\pgfpathmoveto{\pgfqpoint{2.472750in}{2.072975in}}%
\pgfpathcurveto{\pgfqpoint{2.483800in}{2.072975in}}{\pgfqpoint{2.494399in}{2.077365in}}{\pgfqpoint{2.502213in}{2.085179in}}%
\pgfpathcurveto{\pgfqpoint{2.510026in}{2.092992in}}{\pgfqpoint{2.514416in}{2.103591in}}{\pgfqpoint{2.514416in}{2.114641in}}%
\pgfpathcurveto{\pgfqpoint{2.514416in}{2.125692in}}{\pgfqpoint{2.510026in}{2.136291in}}{\pgfqpoint{2.502213in}{2.144104in}}%
\pgfpathcurveto{\pgfqpoint{2.494399in}{2.151918in}}{\pgfqpoint{2.483800in}{2.156308in}}{\pgfqpoint{2.472750in}{2.156308in}}%
\pgfpathcurveto{\pgfqpoint{2.461700in}{2.156308in}}{\pgfqpoint{2.451101in}{2.151918in}}{\pgfqpoint{2.443287in}{2.144104in}}%
\pgfpathcurveto{\pgfqpoint{2.435473in}{2.136291in}}{\pgfqpoint{2.431083in}{2.125692in}}{\pgfqpoint{2.431083in}{2.114641in}}%
\pgfpathcurveto{\pgfqpoint{2.431083in}{2.103591in}}{\pgfqpoint{2.435473in}{2.092992in}}{\pgfqpoint{2.443287in}{2.085179in}}%
\pgfpathcurveto{\pgfqpoint{2.451101in}{2.077365in}}{\pgfqpoint{2.461700in}{2.072975in}}{\pgfqpoint{2.472750in}{2.072975in}}%
\pgfpathclose%
\pgfusepath{stroke,fill}%
\end{pgfscope}%
\begin{pgfscope}%
\pgfpathrectangle{\pgfqpoint{0.600000in}{0.600000in}}{\pgfqpoint{3.900000in}{3.900000in}}%
\pgfusepath{clip}%
\pgfsetbuttcap%
\pgfsetroundjoin%
\definecolor{currentfill}{rgb}{0.121569,0.466667,0.705882}%
\pgfsetfillcolor{currentfill}%
\pgfsetlinewidth{1.003750pt}%
\definecolor{currentstroke}{rgb}{0.121569,0.466667,0.705882}%
\pgfsetstrokecolor{currentstroke}%
\pgfsetdash{}{0pt}%
\pgfpathmoveto{\pgfqpoint{2.604297in}{2.662337in}}%
\pgfpathcurveto{\pgfqpoint{2.615347in}{2.662337in}}{\pgfqpoint{2.625946in}{2.666728in}}{\pgfqpoint{2.633760in}{2.674541in}}%
\pgfpathcurveto{\pgfqpoint{2.641574in}{2.682355in}}{\pgfqpoint{2.645964in}{2.692954in}}{\pgfqpoint{2.645964in}{2.704004in}}%
\pgfpathcurveto{\pgfqpoint{2.645964in}{2.715054in}}{\pgfqpoint{2.641574in}{2.725653in}}{\pgfqpoint{2.633760in}{2.733467in}}%
\pgfpathcurveto{\pgfqpoint{2.625946in}{2.741281in}}{\pgfqpoint{2.615347in}{2.745671in}}{\pgfqpoint{2.604297in}{2.745671in}}%
\pgfpathcurveto{\pgfqpoint{2.593247in}{2.745671in}}{\pgfqpoint{2.582648in}{2.741281in}}{\pgfqpoint{2.574834in}{2.733467in}}%
\pgfpathcurveto{\pgfqpoint{2.567021in}{2.725653in}}{\pgfqpoint{2.562630in}{2.715054in}}{\pgfqpoint{2.562630in}{2.704004in}}%
\pgfpathcurveto{\pgfqpoint{2.562630in}{2.692954in}}{\pgfqpoint{2.567021in}{2.682355in}}{\pgfqpoint{2.574834in}{2.674541in}}%
\pgfpathcurveto{\pgfqpoint{2.582648in}{2.666728in}}{\pgfqpoint{2.593247in}{2.662337in}}{\pgfqpoint{2.604297in}{2.662337in}}%
\pgfpathclose%
\pgfusepath{stroke,fill}%
\end{pgfscope}%
\begin{pgfscope}%
\pgfpathrectangle{\pgfqpoint{0.600000in}{0.600000in}}{\pgfqpoint{3.900000in}{3.900000in}}%
\pgfusepath{clip}%
\pgfsetbuttcap%
\pgfsetroundjoin%
\definecolor{currentfill}{rgb}{0.121569,0.466667,0.705882}%
\pgfsetfillcolor{currentfill}%
\pgfsetlinewidth{1.003750pt}%
\definecolor{currentstroke}{rgb}{0.121569,0.466667,0.705882}%
\pgfsetstrokecolor{currentstroke}%
\pgfsetdash{}{0pt}%
\pgfpathmoveto{\pgfqpoint{2.357841in}{2.738627in}}%
\pgfpathcurveto{\pgfqpoint{2.368891in}{2.738627in}}{\pgfqpoint{2.379490in}{2.743017in}}{\pgfqpoint{2.387304in}{2.750831in}}%
\pgfpathcurveto{\pgfqpoint{2.395118in}{2.758644in}}{\pgfqpoint{2.399508in}{2.769243in}}{\pgfqpoint{2.399508in}{2.780294in}}%
\pgfpathcurveto{\pgfqpoint{2.399508in}{2.791344in}}{\pgfqpoint{2.395118in}{2.801943in}}{\pgfqpoint{2.387304in}{2.809756in}}%
\pgfpathcurveto{\pgfqpoint{2.379490in}{2.817570in}}{\pgfqpoint{2.368891in}{2.821960in}}{\pgfqpoint{2.357841in}{2.821960in}}%
\pgfpathcurveto{\pgfqpoint{2.346791in}{2.821960in}}{\pgfqpoint{2.336192in}{2.817570in}}{\pgfqpoint{2.328378in}{2.809756in}}%
\pgfpathcurveto{\pgfqpoint{2.320565in}{2.801943in}}{\pgfqpoint{2.316175in}{2.791344in}}{\pgfqpoint{2.316175in}{2.780294in}}%
\pgfpathcurveto{\pgfqpoint{2.316175in}{2.769243in}}{\pgfqpoint{2.320565in}{2.758644in}}{\pgfqpoint{2.328378in}{2.750831in}}%
\pgfpathcurveto{\pgfqpoint{2.336192in}{2.743017in}}{\pgfqpoint{2.346791in}{2.738627in}}{\pgfqpoint{2.357841in}{2.738627in}}%
\pgfpathclose%
\pgfusepath{stroke,fill}%
\end{pgfscope}%
\begin{pgfscope}%
\pgfpathrectangle{\pgfqpoint{0.600000in}{0.600000in}}{\pgfqpoint{3.900000in}{3.900000in}}%
\pgfusepath{clip}%
\pgfsetbuttcap%
\pgfsetroundjoin%
\definecolor{currentfill}{rgb}{0.121569,0.466667,0.705882}%
\pgfsetfillcolor{currentfill}%
\pgfsetlinewidth{1.003750pt}%
\definecolor{currentstroke}{rgb}{0.121569,0.466667,0.705882}%
\pgfsetstrokecolor{currentstroke}%
\pgfsetdash{}{0pt}%
\pgfpathmoveto{\pgfqpoint{2.867698in}{3.085266in}}%
\pgfpathcurveto{\pgfqpoint{2.878748in}{3.085266in}}{\pgfqpoint{2.889347in}{3.089656in}}{\pgfqpoint{2.897161in}{3.097469in}}%
\pgfpathcurveto{\pgfqpoint{2.904975in}{3.105283in}}{\pgfqpoint{2.909365in}{3.115882in}}{\pgfqpoint{2.909365in}{3.126932in}}%
\pgfpathcurveto{\pgfqpoint{2.909365in}{3.137982in}}{\pgfqpoint{2.904975in}{3.148581in}}{\pgfqpoint{2.897161in}{3.156395in}}%
\pgfpathcurveto{\pgfqpoint{2.889347in}{3.164209in}}{\pgfqpoint{2.878748in}{3.168599in}}{\pgfqpoint{2.867698in}{3.168599in}}%
\pgfpathcurveto{\pgfqpoint{2.856648in}{3.168599in}}{\pgfqpoint{2.846049in}{3.164209in}}{\pgfqpoint{2.838235in}{3.156395in}}%
\pgfpathcurveto{\pgfqpoint{2.830422in}{3.148581in}}{\pgfqpoint{2.826031in}{3.137982in}}{\pgfqpoint{2.826031in}{3.126932in}}%
\pgfpathcurveto{\pgfqpoint{2.826031in}{3.115882in}}{\pgfqpoint{2.830422in}{3.105283in}}{\pgfqpoint{2.838235in}{3.097469in}}%
\pgfpathcurveto{\pgfqpoint{2.846049in}{3.089656in}}{\pgfqpoint{2.856648in}{3.085266in}}{\pgfqpoint{2.867698in}{3.085266in}}%
\pgfpathclose%
\pgfusepath{stroke,fill}%
\end{pgfscope}%
\begin{pgfscope}%
\pgfpathrectangle{\pgfqpoint{0.600000in}{0.600000in}}{\pgfqpoint{3.900000in}{3.900000in}}%
\pgfusepath{clip}%
\pgfsetbuttcap%
\pgfsetroundjoin%
\definecolor{currentfill}{rgb}{0.121569,0.466667,0.705882}%
\pgfsetfillcolor{currentfill}%
\pgfsetlinewidth{1.003750pt}%
\definecolor{currentstroke}{rgb}{0.121569,0.466667,0.705882}%
\pgfsetstrokecolor{currentstroke}%
\pgfsetdash{}{0pt}%
\pgfpathmoveto{\pgfqpoint{2.928439in}{2.125400in}}%
\pgfpathcurveto{\pgfqpoint{2.939489in}{2.125400in}}{\pgfqpoint{2.950088in}{2.129790in}}{\pgfqpoint{2.957902in}{2.137604in}}%
\pgfpathcurveto{\pgfqpoint{2.965716in}{2.145417in}}{\pgfqpoint{2.970106in}{2.156016in}}{\pgfqpoint{2.970106in}{2.167066in}}%
\pgfpathcurveto{\pgfqpoint{2.970106in}{2.178117in}}{\pgfqpoint{2.965716in}{2.188716in}}{\pgfqpoint{2.957902in}{2.196529in}}%
\pgfpathcurveto{\pgfqpoint{2.950088in}{2.204343in}}{\pgfqpoint{2.939489in}{2.208733in}}{\pgfqpoint{2.928439in}{2.208733in}}%
\pgfpathcurveto{\pgfqpoint{2.917389in}{2.208733in}}{\pgfqpoint{2.906790in}{2.204343in}}{\pgfqpoint{2.898976in}{2.196529in}}%
\pgfpathcurveto{\pgfqpoint{2.891163in}{2.188716in}}{\pgfqpoint{2.886773in}{2.178117in}}{\pgfqpoint{2.886773in}{2.167066in}}%
\pgfpathcurveto{\pgfqpoint{2.886773in}{2.156016in}}{\pgfqpoint{2.891163in}{2.145417in}}{\pgfqpoint{2.898976in}{2.137604in}}%
\pgfpathcurveto{\pgfqpoint{2.906790in}{2.129790in}}{\pgfqpoint{2.917389in}{2.125400in}}{\pgfqpoint{2.928439in}{2.125400in}}%
\pgfpathclose%
\pgfusepath{stroke,fill}%
\end{pgfscope}%
\begin{pgfscope}%
\pgfpathrectangle{\pgfqpoint{0.600000in}{0.600000in}}{\pgfqpoint{3.900000in}{3.900000in}}%
\pgfusepath{clip}%
\pgfsetbuttcap%
\pgfsetroundjoin%
\definecolor{currentfill}{rgb}{0.121569,0.466667,0.705882}%
\pgfsetfillcolor{currentfill}%
\pgfsetlinewidth{1.003750pt}%
\definecolor{currentstroke}{rgb}{0.121569,0.466667,0.705882}%
\pgfsetstrokecolor{currentstroke}%
\pgfsetdash{}{0pt}%
\pgfpathmoveto{\pgfqpoint{2.519311in}{2.528132in}}%
\pgfpathcurveto{\pgfqpoint{2.530361in}{2.528132in}}{\pgfqpoint{2.540960in}{2.532522in}}{\pgfqpoint{2.548774in}{2.540336in}}%
\pgfpathcurveto{\pgfqpoint{2.556588in}{2.548149in}}{\pgfqpoint{2.560978in}{2.558748in}}{\pgfqpoint{2.560978in}{2.569798in}}%
\pgfpathcurveto{\pgfqpoint{2.560978in}{2.580849in}}{\pgfqpoint{2.556588in}{2.591448in}}{\pgfqpoint{2.548774in}{2.599261in}}%
\pgfpathcurveto{\pgfqpoint{2.540960in}{2.607075in}}{\pgfqpoint{2.530361in}{2.611465in}}{\pgfqpoint{2.519311in}{2.611465in}}%
\pgfpathcurveto{\pgfqpoint{2.508261in}{2.611465in}}{\pgfqpoint{2.497662in}{2.607075in}}{\pgfqpoint{2.489849in}{2.599261in}}%
\pgfpathcurveto{\pgfqpoint{2.482035in}{2.591448in}}{\pgfqpoint{2.477645in}{2.580849in}}{\pgfqpoint{2.477645in}{2.569798in}}%
\pgfpathcurveto{\pgfqpoint{2.477645in}{2.558748in}}{\pgfqpoint{2.482035in}{2.548149in}}{\pgfqpoint{2.489849in}{2.540336in}}%
\pgfpathcurveto{\pgfqpoint{2.497662in}{2.532522in}}{\pgfqpoint{2.508261in}{2.528132in}}{\pgfqpoint{2.519311in}{2.528132in}}%
\pgfpathclose%
\pgfusepath{stroke,fill}%
\end{pgfscope}%
\begin{pgfscope}%
\pgfpathrectangle{\pgfqpoint{0.600000in}{0.600000in}}{\pgfqpoint{3.900000in}{3.900000in}}%
\pgfusepath{clip}%
\pgfsetbuttcap%
\pgfsetroundjoin%
\definecolor{currentfill}{rgb}{0.121569,0.466667,0.705882}%
\pgfsetfillcolor{currentfill}%
\pgfsetlinewidth{1.003750pt}%
\definecolor{currentstroke}{rgb}{0.121569,0.466667,0.705882}%
\pgfsetstrokecolor{currentstroke}%
\pgfsetdash{}{0pt}%
\pgfpathmoveto{\pgfqpoint{2.472146in}{2.154469in}}%
\pgfpathcurveto{\pgfqpoint{2.483196in}{2.154469in}}{\pgfqpoint{2.493795in}{2.158859in}}{\pgfqpoint{2.501609in}{2.166673in}}%
\pgfpathcurveto{\pgfqpoint{2.509423in}{2.174487in}}{\pgfqpoint{2.513813in}{2.185086in}}{\pgfqpoint{2.513813in}{2.196136in}}%
\pgfpathcurveto{\pgfqpoint{2.513813in}{2.207186in}}{\pgfqpoint{2.509423in}{2.217785in}}{\pgfqpoint{2.501609in}{2.225599in}}%
\pgfpathcurveto{\pgfqpoint{2.493795in}{2.233412in}}{\pgfqpoint{2.483196in}{2.237802in}}{\pgfqpoint{2.472146in}{2.237802in}}%
\pgfpathcurveto{\pgfqpoint{2.461096in}{2.237802in}}{\pgfqpoint{2.450497in}{2.233412in}}{\pgfqpoint{2.442683in}{2.225599in}}%
\pgfpathcurveto{\pgfqpoint{2.434870in}{2.217785in}}{\pgfqpoint{2.430480in}{2.207186in}}{\pgfqpoint{2.430480in}{2.196136in}}%
\pgfpathcurveto{\pgfqpoint{2.430480in}{2.185086in}}{\pgfqpoint{2.434870in}{2.174487in}}{\pgfqpoint{2.442683in}{2.166673in}}%
\pgfpathcurveto{\pgfqpoint{2.450497in}{2.158859in}}{\pgfqpoint{2.461096in}{2.154469in}}{\pgfqpoint{2.472146in}{2.154469in}}%
\pgfpathclose%
\pgfusepath{stroke,fill}%
\end{pgfscope}%
\begin{pgfscope}%
\pgfpathrectangle{\pgfqpoint{0.600000in}{0.600000in}}{\pgfqpoint{3.900000in}{3.900000in}}%
\pgfusepath{clip}%
\pgfsetbuttcap%
\pgfsetroundjoin%
\definecolor{currentfill}{rgb}{0.121569,0.466667,0.705882}%
\pgfsetfillcolor{currentfill}%
\pgfsetlinewidth{1.003750pt}%
\definecolor{currentstroke}{rgb}{0.121569,0.466667,0.705882}%
\pgfsetstrokecolor{currentstroke}%
\pgfsetdash{}{0pt}%
\pgfpathmoveto{\pgfqpoint{2.728341in}{2.900063in}}%
\pgfpathcurveto{\pgfqpoint{2.739391in}{2.900063in}}{\pgfqpoint{2.749990in}{2.904453in}}{\pgfqpoint{2.757803in}{2.912267in}}%
\pgfpathcurveto{\pgfqpoint{2.765617in}{2.920080in}}{\pgfqpoint{2.770007in}{2.930679in}}{\pgfqpoint{2.770007in}{2.941730in}}%
\pgfpathcurveto{\pgfqpoint{2.770007in}{2.952780in}}{\pgfqpoint{2.765617in}{2.963379in}}{\pgfqpoint{2.757803in}{2.971192in}}%
\pgfpathcurveto{\pgfqpoint{2.749990in}{2.979006in}}{\pgfqpoint{2.739391in}{2.983396in}}{\pgfqpoint{2.728341in}{2.983396in}}%
\pgfpathcurveto{\pgfqpoint{2.717290in}{2.983396in}}{\pgfqpoint{2.706691in}{2.979006in}}{\pgfqpoint{2.698878in}{2.971192in}}%
\pgfpathcurveto{\pgfqpoint{2.691064in}{2.963379in}}{\pgfqpoint{2.686674in}{2.952780in}}{\pgfqpoint{2.686674in}{2.941730in}}%
\pgfpathcurveto{\pgfqpoint{2.686674in}{2.930679in}}{\pgfqpoint{2.691064in}{2.920080in}}{\pgfqpoint{2.698878in}{2.912267in}}%
\pgfpathcurveto{\pgfqpoint{2.706691in}{2.904453in}}{\pgfqpoint{2.717290in}{2.900063in}}{\pgfqpoint{2.728341in}{2.900063in}}%
\pgfpathclose%
\pgfusepath{stroke,fill}%
\end{pgfscope}%
\begin{pgfscope}%
\pgfpathrectangle{\pgfqpoint{0.600000in}{0.600000in}}{\pgfqpoint{3.900000in}{3.900000in}}%
\pgfusepath{clip}%
\pgfsetbuttcap%
\pgfsetroundjoin%
\definecolor{currentfill}{rgb}{0.121569,0.466667,0.705882}%
\pgfsetfillcolor{currentfill}%
\pgfsetlinewidth{1.003750pt}%
\definecolor{currentstroke}{rgb}{0.121569,0.466667,0.705882}%
\pgfsetstrokecolor{currentstroke}%
\pgfsetdash{}{0pt}%
\pgfpathmoveto{\pgfqpoint{2.729908in}{3.015695in}}%
\pgfpathcurveto{\pgfqpoint{2.740958in}{3.015695in}}{\pgfqpoint{2.751557in}{3.020085in}}{\pgfqpoint{2.759371in}{3.027899in}}%
\pgfpathcurveto{\pgfqpoint{2.767185in}{3.035713in}}{\pgfqpoint{2.771575in}{3.046312in}}{\pgfqpoint{2.771575in}{3.057362in}}%
\pgfpathcurveto{\pgfqpoint{2.771575in}{3.068412in}}{\pgfqpoint{2.767185in}{3.079011in}}{\pgfqpoint{2.759371in}{3.086825in}}%
\pgfpathcurveto{\pgfqpoint{2.751557in}{3.094638in}}{\pgfqpoint{2.740958in}{3.099029in}}{\pgfqpoint{2.729908in}{3.099029in}}%
\pgfpathcurveto{\pgfqpoint{2.718858in}{3.099029in}}{\pgfqpoint{2.708259in}{3.094638in}}{\pgfqpoint{2.700446in}{3.086825in}}%
\pgfpathcurveto{\pgfqpoint{2.692632in}{3.079011in}}{\pgfqpoint{2.688242in}{3.068412in}}{\pgfqpoint{2.688242in}{3.057362in}}%
\pgfpathcurveto{\pgfqpoint{2.688242in}{3.046312in}}{\pgfqpoint{2.692632in}{3.035713in}}{\pgfqpoint{2.700446in}{3.027899in}}%
\pgfpathcurveto{\pgfqpoint{2.708259in}{3.020085in}}{\pgfqpoint{2.718858in}{3.015695in}}{\pgfqpoint{2.729908in}{3.015695in}}%
\pgfpathclose%
\pgfusepath{stroke,fill}%
\end{pgfscope}%
\begin{pgfscope}%
\pgfpathrectangle{\pgfqpoint{0.600000in}{0.600000in}}{\pgfqpoint{3.900000in}{3.900000in}}%
\pgfusepath{clip}%
\pgfsetbuttcap%
\pgfsetroundjoin%
\definecolor{currentfill}{rgb}{0.121569,0.466667,0.705882}%
\pgfsetfillcolor{currentfill}%
\pgfsetlinewidth{1.003750pt}%
\definecolor{currentstroke}{rgb}{0.121569,0.466667,0.705882}%
\pgfsetstrokecolor{currentstroke}%
\pgfsetdash{}{0pt}%
\pgfpathmoveto{\pgfqpoint{2.176826in}{1.863775in}}%
\pgfpathcurveto{\pgfqpoint{2.187876in}{1.863775in}}{\pgfqpoint{2.198475in}{1.868166in}}{\pgfqpoint{2.206289in}{1.875979in}}%
\pgfpathcurveto{\pgfqpoint{2.214102in}{1.883793in}}{\pgfqpoint{2.218492in}{1.894392in}}{\pgfqpoint{2.218492in}{1.905442in}}%
\pgfpathcurveto{\pgfqpoint{2.218492in}{1.916492in}}{\pgfqpoint{2.214102in}{1.927091in}}{\pgfqpoint{2.206289in}{1.934905in}}%
\pgfpathcurveto{\pgfqpoint{2.198475in}{1.942718in}}{\pgfqpoint{2.187876in}{1.947109in}}{\pgfqpoint{2.176826in}{1.947109in}}%
\pgfpathcurveto{\pgfqpoint{2.165776in}{1.947109in}}{\pgfqpoint{2.155177in}{1.942718in}}{\pgfqpoint{2.147363in}{1.934905in}}%
\pgfpathcurveto{\pgfqpoint{2.139549in}{1.927091in}}{\pgfqpoint{2.135159in}{1.916492in}}{\pgfqpoint{2.135159in}{1.905442in}}%
\pgfpathcurveto{\pgfqpoint{2.135159in}{1.894392in}}{\pgfqpoint{2.139549in}{1.883793in}}{\pgfqpoint{2.147363in}{1.875979in}}%
\pgfpathcurveto{\pgfqpoint{2.155177in}{1.868166in}}{\pgfqpoint{2.165776in}{1.863775in}}{\pgfqpoint{2.176826in}{1.863775in}}%
\pgfpathclose%
\pgfusepath{stroke,fill}%
\end{pgfscope}%
\begin{pgfscope}%
\pgfpathrectangle{\pgfqpoint{0.600000in}{0.600000in}}{\pgfqpoint{3.900000in}{3.900000in}}%
\pgfusepath{clip}%
\pgfsetbuttcap%
\pgfsetroundjoin%
\definecolor{currentfill}{rgb}{0.121569,0.466667,0.705882}%
\pgfsetfillcolor{currentfill}%
\pgfsetlinewidth{1.003750pt}%
\definecolor{currentstroke}{rgb}{0.121569,0.466667,0.705882}%
\pgfsetstrokecolor{currentstroke}%
\pgfsetdash{}{0pt}%
\pgfpathmoveto{\pgfqpoint{2.967461in}{2.160641in}}%
\pgfpathcurveto{\pgfqpoint{2.978511in}{2.160641in}}{\pgfqpoint{2.989110in}{2.165031in}}{\pgfqpoint{2.996924in}{2.172845in}}%
\pgfpathcurveto{\pgfqpoint{3.004737in}{2.180658in}}{\pgfqpoint{3.009128in}{2.191257in}}{\pgfqpoint{3.009128in}{2.202308in}}%
\pgfpathcurveto{\pgfqpoint{3.009128in}{2.213358in}}{\pgfqpoint{3.004737in}{2.223957in}}{\pgfqpoint{2.996924in}{2.231770in}}%
\pgfpathcurveto{\pgfqpoint{2.989110in}{2.239584in}}{\pgfqpoint{2.978511in}{2.243974in}}{\pgfqpoint{2.967461in}{2.243974in}}%
\pgfpathcurveto{\pgfqpoint{2.956411in}{2.243974in}}{\pgfqpoint{2.945812in}{2.239584in}}{\pgfqpoint{2.937998in}{2.231770in}}%
\pgfpathcurveto{\pgfqpoint{2.930185in}{2.223957in}}{\pgfqpoint{2.925794in}{2.213358in}}{\pgfqpoint{2.925794in}{2.202308in}}%
\pgfpathcurveto{\pgfqpoint{2.925794in}{2.191257in}}{\pgfqpoint{2.930185in}{2.180658in}}{\pgfqpoint{2.937998in}{2.172845in}}%
\pgfpathcurveto{\pgfqpoint{2.945812in}{2.165031in}}{\pgfqpoint{2.956411in}{2.160641in}}{\pgfqpoint{2.967461in}{2.160641in}}%
\pgfpathclose%
\pgfusepath{stroke,fill}%
\end{pgfscope}%
\begin{pgfscope}%
\pgfpathrectangle{\pgfqpoint{0.600000in}{0.600000in}}{\pgfqpoint{3.900000in}{3.900000in}}%
\pgfusepath{clip}%
\pgfsetbuttcap%
\pgfsetroundjoin%
\definecolor{currentfill}{rgb}{0.121569,0.466667,0.705882}%
\pgfsetfillcolor{currentfill}%
\pgfsetlinewidth{1.003750pt}%
\definecolor{currentstroke}{rgb}{0.121569,0.466667,0.705882}%
\pgfsetstrokecolor{currentstroke}%
\pgfsetdash{}{0pt}%
\pgfpathmoveto{\pgfqpoint{1.916781in}{2.013683in}}%
\pgfpathcurveto{\pgfqpoint{1.927831in}{2.013683in}}{\pgfqpoint{1.938430in}{2.018073in}}{\pgfqpoint{1.946244in}{2.025886in}}%
\pgfpathcurveto{\pgfqpoint{1.954057in}{2.033700in}}{\pgfqpoint{1.958448in}{2.044299in}}{\pgfqpoint{1.958448in}{2.055349in}}%
\pgfpathcurveto{\pgfqpoint{1.958448in}{2.066399in}}{\pgfqpoint{1.954057in}{2.076998in}}{\pgfqpoint{1.946244in}{2.084812in}}%
\pgfpathcurveto{\pgfqpoint{1.938430in}{2.092626in}}{\pgfqpoint{1.927831in}{2.097016in}}{\pgfqpoint{1.916781in}{2.097016in}}%
\pgfpathcurveto{\pgfqpoint{1.905731in}{2.097016in}}{\pgfqpoint{1.895132in}{2.092626in}}{\pgfqpoint{1.887318in}{2.084812in}}%
\pgfpathcurveto{\pgfqpoint{1.879505in}{2.076998in}}{\pgfqpoint{1.875114in}{2.066399in}}{\pgfqpoint{1.875114in}{2.055349in}}%
\pgfpathcurveto{\pgfqpoint{1.875114in}{2.044299in}}{\pgfqpoint{1.879505in}{2.033700in}}{\pgfqpoint{1.887318in}{2.025886in}}%
\pgfpathcurveto{\pgfqpoint{1.895132in}{2.018073in}}{\pgfqpoint{1.905731in}{2.013683in}}{\pgfqpoint{1.916781in}{2.013683in}}%
\pgfpathclose%
\pgfusepath{stroke,fill}%
\end{pgfscope}%
\begin{pgfscope}%
\pgfpathrectangle{\pgfqpoint{0.600000in}{0.600000in}}{\pgfqpoint{3.900000in}{3.900000in}}%
\pgfusepath{clip}%
\pgfsetbuttcap%
\pgfsetroundjoin%
\definecolor{currentfill}{rgb}{0.121569,0.466667,0.705882}%
\pgfsetfillcolor{currentfill}%
\pgfsetlinewidth{1.003750pt}%
\definecolor{currentstroke}{rgb}{0.121569,0.466667,0.705882}%
\pgfsetstrokecolor{currentstroke}%
\pgfsetdash{}{0pt}%
\pgfpathmoveto{\pgfqpoint{2.025558in}{1.873871in}}%
\pgfpathcurveto{\pgfqpoint{2.036608in}{1.873871in}}{\pgfqpoint{2.047207in}{1.878262in}}{\pgfqpoint{2.055021in}{1.886075in}}%
\pgfpathcurveto{\pgfqpoint{2.062834in}{1.893889in}}{\pgfqpoint{2.067224in}{1.904488in}}{\pgfqpoint{2.067224in}{1.915538in}}%
\pgfpathcurveto{\pgfqpoint{2.067224in}{1.926588in}}{\pgfqpoint{2.062834in}{1.937187in}}{\pgfqpoint{2.055021in}{1.945001in}}%
\pgfpathcurveto{\pgfqpoint{2.047207in}{1.952814in}}{\pgfqpoint{2.036608in}{1.957205in}}{\pgfqpoint{2.025558in}{1.957205in}}%
\pgfpathcurveto{\pgfqpoint{2.014508in}{1.957205in}}{\pgfqpoint{2.003909in}{1.952814in}}{\pgfqpoint{1.996095in}{1.945001in}}%
\pgfpathcurveto{\pgfqpoint{1.988281in}{1.937187in}}{\pgfqpoint{1.983891in}{1.926588in}}{\pgfqpoint{1.983891in}{1.915538in}}%
\pgfpathcurveto{\pgfqpoint{1.983891in}{1.904488in}}{\pgfqpoint{1.988281in}{1.893889in}}{\pgfqpoint{1.996095in}{1.886075in}}%
\pgfpathcurveto{\pgfqpoint{2.003909in}{1.878262in}}{\pgfqpoint{2.014508in}{1.873871in}}{\pgfqpoint{2.025558in}{1.873871in}}%
\pgfpathclose%
\pgfusepath{stroke,fill}%
\end{pgfscope}%
\begin{pgfscope}%
\pgfpathrectangle{\pgfqpoint{0.600000in}{0.600000in}}{\pgfqpoint{3.900000in}{3.900000in}}%
\pgfusepath{clip}%
\pgfsetbuttcap%
\pgfsetroundjoin%
\definecolor{currentfill}{rgb}{0.121569,0.466667,0.705882}%
\pgfsetfillcolor{currentfill}%
\pgfsetlinewidth{1.003750pt}%
\definecolor{currentstroke}{rgb}{0.121569,0.466667,0.705882}%
\pgfsetstrokecolor{currentstroke}%
\pgfsetdash{}{0pt}%
\pgfpathmoveto{\pgfqpoint{1.705962in}{1.634244in}}%
\pgfpathcurveto{\pgfqpoint{1.717012in}{1.634244in}}{\pgfqpoint{1.727611in}{1.638634in}}{\pgfqpoint{1.735424in}{1.646448in}}%
\pgfpathcurveto{\pgfqpoint{1.743238in}{1.654262in}}{\pgfqpoint{1.747628in}{1.664861in}}{\pgfqpoint{1.747628in}{1.675911in}}%
\pgfpathcurveto{\pgfqpoint{1.747628in}{1.686961in}}{\pgfqpoint{1.743238in}{1.697560in}}{\pgfqpoint{1.735424in}{1.705374in}}%
\pgfpathcurveto{\pgfqpoint{1.727611in}{1.713187in}}{\pgfqpoint{1.717012in}{1.717577in}}{\pgfqpoint{1.705962in}{1.717577in}}%
\pgfpathcurveto{\pgfqpoint{1.694912in}{1.717577in}}{\pgfqpoint{1.684313in}{1.713187in}}{\pgfqpoint{1.676499in}{1.705374in}}%
\pgfpathcurveto{\pgfqpoint{1.668685in}{1.697560in}}{\pgfqpoint{1.664295in}{1.686961in}}{\pgfqpoint{1.664295in}{1.675911in}}%
\pgfpathcurveto{\pgfqpoint{1.664295in}{1.664861in}}{\pgfqpoint{1.668685in}{1.654262in}}{\pgfqpoint{1.676499in}{1.646448in}}%
\pgfpathcurveto{\pgfqpoint{1.684313in}{1.638634in}}{\pgfqpoint{1.694912in}{1.634244in}}{\pgfqpoint{1.705962in}{1.634244in}}%
\pgfpathclose%
\pgfusepath{stroke,fill}%
\end{pgfscope}%
\begin{pgfscope}%
\pgfpathrectangle{\pgfqpoint{0.600000in}{0.600000in}}{\pgfqpoint{3.900000in}{3.900000in}}%
\pgfusepath{clip}%
\pgfsetbuttcap%
\pgfsetroundjoin%
\definecolor{currentfill}{rgb}{0.121569,0.466667,0.705882}%
\pgfsetfillcolor{currentfill}%
\pgfsetlinewidth{1.003750pt}%
\definecolor{currentstroke}{rgb}{0.121569,0.466667,0.705882}%
\pgfsetstrokecolor{currentstroke}%
\pgfsetdash{}{0pt}%
\pgfpathmoveto{\pgfqpoint{3.104864in}{1.901105in}}%
\pgfpathcurveto{\pgfqpoint{3.115914in}{1.901105in}}{\pgfqpoint{3.126513in}{1.905495in}}{\pgfqpoint{3.134327in}{1.913309in}}%
\pgfpathcurveto{\pgfqpoint{3.142140in}{1.921122in}}{\pgfqpoint{3.146531in}{1.931721in}}{\pgfqpoint{3.146531in}{1.942771in}}%
\pgfpathcurveto{\pgfqpoint{3.146531in}{1.953821in}}{\pgfqpoint{3.142140in}{1.964421in}}{\pgfqpoint{3.134327in}{1.972234in}}%
\pgfpathcurveto{\pgfqpoint{3.126513in}{1.980048in}}{\pgfqpoint{3.115914in}{1.984438in}}{\pgfqpoint{3.104864in}{1.984438in}}%
\pgfpathcurveto{\pgfqpoint{3.093814in}{1.984438in}}{\pgfqpoint{3.083215in}{1.980048in}}{\pgfqpoint{3.075401in}{1.972234in}}%
\pgfpathcurveto{\pgfqpoint{3.067587in}{1.964421in}}{\pgfqpoint{3.063197in}{1.953821in}}{\pgfqpoint{3.063197in}{1.942771in}}%
\pgfpathcurveto{\pgfqpoint{3.063197in}{1.931721in}}{\pgfqpoint{3.067587in}{1.921122in}}{\pgfqpoint{3.075401in}{1.913309in}}%
\pgfpathcurveto{\pgfqpoint{3.083215in}{1.905495in}}{\pgfqpoint{3.093814in}{1.901105in}}{\pgfqpoint{3.104864in}{1.901105in}}%
\pgfpathclose%
\pgfusepath{stroke,fill}%
\end{pgfscope}%
\begin{pgfscope}%
\pgfpathrectangle{\pgfqpoint{0.600000in}{0.600000in}}{\pgfqpoint{3.900000in}{3.900000in}}%
\pgfusepath{clip}%
\pgfsetbuttcap%
\pgfsetroundjoin%
\definecolor{currentfill}{rgb}{0.121569,0.466667,0.705882}%
\pgfsetfillcolor{currentfill}%
\pgfsetlinewidth{1.003750pt}%
\definecolor{currentstroke}{rgb}{0.121569,0.466667,0.705882}%
\pgfsetstrokecolor{currentstroke}%
\pgfsetdash{}{0pt}%
\pgfpathmoveto{\pgfqpoint{2.959575in}{3.447463in}}%
\pgfpathcurveto{\pgfqpoint{2.970625in}{3.447463in}}{\pgfqpoint{2.981224in}{3.451854in}}{\pgfqpoint{2.989037in}{3.459667in}}%
\pgfpathcurveto{\pgfqpoint{2.996851in}{3.467481in}}{\pgfqpoint{3.001241in}{3.478080in}}{\pgfqpoint{3.001241in}{3.489130in}}%
\pgfpathcurveto{\pgfqpoint{3.001241in}{3.500180in}}{\pgfqpoint{2.996851in}{3.510779in}}{\pgfqpoint{2.989037in}{3.518593in}}%
\pgfpathcurveto{\pgfqpoint{2.981224in}{3.526406in}}{\pgfqpoint{2.970625in}{3.530797in}}{\pgfqpoint{2.959575in}{3.530797in}}%
\pgfpathcurveto{\pgfqpoint{2.948524in}{3.530797in}}{\pgfqpoint{2.937925in}{3.526406in}}{\pgfqpoint{2.930112in}{3.518593in}}%
\pgfpathcurveto{\pgfqpoint{2.922298in}{3.510779in}}{\pgfqpoint{2.917908in}{3.500180in}}{\pgfqpoint{2.917908in}{3.489130in}}%
\pgfpathcurveto{\pgfqpoint{2.917908in}{3.478080in}}{\pgfqpoint{2.922298in}{3.467481in}}{\pgfqpoint{2.930112in}{3.459667in}}%
\pgfpathcurveto{\pgfqpoint{2.937925in}{3.451854in}}{\pgfqpoint{2.948524in}{3.447463in}}{\pgfqpoint{2.959575in}{3.447463in}}%
\pgfpathclose%
\pgfusepath{stroke,fill}%
\end{pgfscope}%
\begin{pgfscope}%
\pgfpathrectangle{\pgfqpoint{0.600000in}{0.600000in}}{\pgfqpoint{3.900000in}{3.900000in}}%
\pgfusepath{clip}%
\pgfsetbuttcap%
\pgfsetroundjoin%
\definecolor{currentfill}{rgb}{0.121569,0.466667,0.705882}%
\pgfsetfillcolor{currentfill}%
\pgfsetlinewidth{1.003750pt}%
\definecolor{currentstroke}{rgb}{0.121569,0.466667,0.705882}%
\pgfsetstrokecolor{currentstroke}%
\pgfsetdash{}{0pt}%
\pgfpathmoveto{\pgfqpoint{2.766573in}{3.047857in}}%
\pgfpathcurveto{\pgfqpoint{2.777623in}{3.047857in}}{\pgfqpoint{2.788222in}{3.052248in}}{\pgfqpoint{2.796036in}{3.060061in}}%
\pgfpathcurveto{\pgfqpoint{2.803849in}{3.067875in}}{\pgfqpoint{2.808240in}{3.078474in}}{\pgfqpoint{2.808240in}{3.089524in}}%
\pgfpathcurveto{\pgfqpoint{2.808240in}{3.100574in}}{\pgfqpoint{2.803849in}{3.111173in}}{\pgfqpoint{2.796036in}{3.118987in}}%
\pgfpathcurveto{\pgfqpoint{2.788222in}{3.126800in}}{\pgfqpoint{2.777623in}{3.131191in}}{\pgfqpoint{2.766573in}{3.131191in}}%
\pgfpathcurveto{\pgfqpoint{2.755523in}{3.131191in}}{\pgfqpoint{2.744924in}{3.126800in}}{\pgfqpoint{2.737110in}{3.118987in}}%
\pgfpathcurveto{\pgfqpoint{2.729296in}{3.111173in}}{\pgfqpoint{2.724906in}{3.100574in}}{\pgfqpoint{2.724906in}{3.089524in}}%
\pgfpathcurveto{\pgfqpoint{2.724906in}{3.078474in}}{\pgfqpoint{2.729296in}{3.067875in}}{\pgfqpoint{2.737110in}{3.060061in}}%
\pgfpathcurveto{\pgfqpoint{2.744924in}{3.052248in}}{\pgfqpoint{2.755523in}{3.047857in}}{\pgfqpoint{2.766573in}{3.047857in}}%
\pgfpathclose%
\pgfusepath{stroke,fill}%
\end{pgfscope}%
\begin{pgfscope}%
\pgfpathrectangle{\pgfqpoint{0.600000in}{0.600000in}}{\pgfqpoint{3.900000in}{3.900000in}}%
\pgfusepath{clip}%
\pgfsetbuttcap%
\pgfsetroundjoin%
\definecolor{currentfill}{rgb}{0.121569,0.466667,0.705882}%
\pgfsetfillcolor{currentfill}%
\pgfsetlinewidth{1.003750pt}%
\definecolor{currentstroke}{rgb}{0.121569,0.466667,0.705882}%
\pgfsetstrokecolor{currentstroke}%
\pgfsetdash{}{0pt}%
\pgfpathmoveto{\pgfqpoint{2.761031in}{1.652918in}}%
\pgfpathcurveto{\pgfqpoint{2.772081in}{1.652918in}}{\pgfqpoint{2.782680in}{1.657309in}}{\pgfqpoint{2.790493in}{1.665122in}}%
\pgfpathcurveto{\pgfqpoint{2.798307in}{1.672936in}}{\pgfqpoint{2.802697in}{1.683535in}}{\pgfqpoint{2.802697in}{1.694585in}}%
\pgfpathcurveto{\pgfqpoint{2.802697in}{1.705635in}}{\pgfqpoint{2.798307in}{1.716234in}}{\pgfqpoint{2.790493in}{1.724048in}}%
\pgfpathcurveto{\pgfqpoint{2.782680in}{1.731862in}}{\pgfqpoint{2.772081in}{1.736252in}}{\pgfqpoint{2.761031in}{1.736252in}}%
\pgfpathcurveto{\pgfqpoint{2.749981in}{1.736252in}}{\pgfqpoint{2.739382in}{1.731862in}}{\pgfqpoint{2.731568in}{1.724048in}}%
\pgfpathcurveto{\pgfqpoint{2.723754in}{1.716234in}}{\pgfqpoint{2.719364in}{1.705635in}}{\pgfqpoint{2.719364in}{1.694585in}}%
\pgfpathcurveto{\pgfqpoint{2.719364in}{1.683535in}}{\pgfqpoint{2.723754in}{1.672936in}}{\pgfqpoint{2.731568in}{1.665122in}}%
\pgfpathcurveto{\pgfqpoint{2.739382in}{1.657309in}}{\pgfqpoint{2.749981in}{1.652918in}}{\pgfqpoint{2.761031in}{1.652918in}}%
\pgfpathclose%
\pgfusepath{stroke,fill}%
\end{pgfscope}%
\begin{pgfscope}%
\pgfpathrectangle{\pgfqpoint{0.600000in}{0.600000in}}{\pgfqpoint{3.900000in}{3.900000in}}%
\pgfusepath{clip}%
\pgfsetbuttcap%
\pgfsetroundjoin%
\definecolor{currentfill}{rgb}{0.121569,0.466667,0.705882}%
\pgfsetfillcolor{currentfill}%
\pgfsetlinewidth{1.003750pt}%
\definecolor{currentstroke}{rgb}{0.121569,0.466667,0.705882}%
\pgfsetstrokecolor{currentstroke}%
\pgfsetdash{}{0pt}%
\pgfpathmoveto{\pgfqpoint{2.809807in}{2.504863in}}%
\pgfpathcurveto{\pgfqpoint{2.820857in}{2.504863in}}{\pgfqpoint{2.831456in}{2.509253in}}{\pgfqpoint{2.839269in}{2.517067in}}%
\pgfpathcurveto{\pgfqpoint{2.847083in}{2.524881in}}{\pgfqpoint{2.851473in}{2.535480in}}{\pgfqpoint{2.851473in}{2.546530in}}%
\pgfpathcurveto{\pgfqpoint{2.851473in}{2.557580in}}{\pgfqpoint{2.847083in}{2.568179in}}{\pgfqpoint{2.839269in}{2.575993in}}%
\pgfpathcurveto{\pgfqpoint{2.831456in}{2.583806in}}{\pgfqpoint{2.820857in}{2.588197in}}{\pgfqpoint{2.809807in}{2.588197in}}%
\pgfpathcurveto{\pgfqpoint{2.798757in}{2.588197in}}{\pgfqpoint{2.788158in}{2.583806in}}{\pgfqpoint{2.780344in}{2.575993in}}%
\pgfpathcurveto{\pgfqpoint{2.772530in}{2.568179in}}{\pgfqpoint{2.768140in}{2.557580in}}{\pgfqpoint{2.768140in}{2.546530in}}%
\pgfpathcurveto{\pgfqpoint{2.768140in}{2.535480in}}{\pgfqpoint{2.772530in}{2.524881in}}{\pgfqpoint{2.780344in}{2.517067in}}%
\pgfpathcurveto{\pgfqpoint{2.788158in}{2.509253in}}{\pgfqpoint{2.798757in}{2.504863in}}{\pgfqpoint{2.809807in}{2.504863in}}%
\pgfpathclose%
\pgfusepath{stroke,fill}%
\end{pgfscope}%
\begin{pgfscope}%
\pgfpathrectangle{\pgfqpoint{0.600000in}{0.600000in}}{\pgfqpoint{3.900000in}{3.900000in}}%
\pgfusepath{clip}%
\pgfsetbuttcap%
\pgfsetroundjoin%
\definecolor{currentfill}{rgb}{0.121569,0.466667,0.705882}%
\pgfsetfillcolor{currentfill}%
\pgfsetlinewidth{1.003750pt}%
\definecolor{currentstroke}{rgb}{0.121569,0.466667,0.705882}%
\pgfsetstrokecolor{currentstroke}%
\pgfsetdash{}{0pt}%
\pgfpathmoveto{\pgfqpoint{2.510760in}{2.553268in}}%
\pgfpathcurveto{\pgfqpoint{2.521810in}{2.553268in}}{\pgfqpoint{2.532409in}{2.557658in}}{\pgfqpoint{2.540223in}{2.565472in}}%
\pgfpathcurveto{\pgfqpoint{2.548036in}{2.573286in}}{\pgfqpoint{2.552426in}{2.583885in}}{\pgfqpoint{2.552426in}{2.594935in}}%
\pgfpathcurveto{\pgfqpoint{2.552426in}{2.605985in}}{\pgfqpoint{2.548036in}{2.616584in}}{\pgfqpoint{2.540223in}{2.624398in}}%
\pgfpathcurveto{\pgfqpoint{2.532409in}{2.632211in}}{\pgfqpoint{2.521810in}{2.636602in}}{\pgfqpoint{2.510760in}{2.636602in}}%
\pgfpathcurveto{\pgfqpoint{2.499710in}{2.636602in}}{\pgfqpoint{2.489111in}{2.632211in}}{\pgfqpoint{2.481297in}{2.624398in}}%
\pgfpathcurveto{\pgfqpoint{2.473483in}{2.616584in}}{\pgfqpoint{2.469093in}{2.605985in}}{\pgfqpoint{2.469093in}{2.594935in}}%
\pgfpathcurveto{\pgfqpoint{2.469093in}{2.583885in}}{\pgfqpoint{2.473483in}{2.573286in}}{\pgfqpoint{2.481297in}{2.565472in}}%
\pgfpathcurveto{\pgfqpoint{2.489111in}{2.557658in}}{\pgfqpoint{2.499710in}{2.553268in}}{\pgfqpoint{2.510760in}{2.553268in}}%
\pgfpathclose%
\pgfusepath{stroke,fill}%
\end{pgfscope}%
\begin{pgfscope}%
\pgfpathrectangle{\pgfqpoint{0.600000in}{0.600000in}}{\pgfqpoint{3.900000in}{3.900000in}}%
\pgfusepath{clip}%
\pgfsetbuttcap%
\pgfsetroundjoin%
\definecolor{currentfill}{rgb}{0.121569,0.466667,0.705882}%
\pgfsetfillcolor{currentfill}%
\pgfsetlinewidth{1.003750pt}%
\definecolor{currentstroke}{rgb}{0.121569,0.466667,0.705882}%
\pgfsetstrokecolor{currentstroke}%
\pgfsetdash{}{0pt}%
\pgfpathmoveto{\pgfqpoint{2.976418in}{3.412021in}}%
\pgfpathcurveto{\pgfqpoint{2.987468in}{3.412021in}}{\pgfqpoint{2.998067in}{3.416411in}}{\pgfqpoint{3.005880in}{3.424225in}}%
\pgfpathcurveto{\pgfqpoint{3.013694in}{3.432038in}}{\pgfqpoint{3.018084in}{3.442637in}}{\pgfqpoint{3.018084in}{3.453687in}}%
\pgfpathcurveto{\pgfqpoint{3.018084in}{3.464738in}}{\pgfqpoint{3.013694in}{3.475337in}}{\pgfqpoint{3.005880in}{3.483150in}}%
\pgfpathcurveto{\pgfqpoint{2.998067in}{3.490964in}}{\pgfqpoint{2.987468in}{3.495354in}}{\pgfqpoint{2.976418in}{3.495354in}}%
\pgfpathcurveto{\pgfqpoint{2.965367in}{3.495354in}}{\pgfqpoint{2.954768in}{3.490964in}}{\pgfqpoint{2.946955in}{3.483150in}}%
\pgfpathcurveto{\pgfqpoint{2.939141in}{3.475337in}}{\pgfqpoint{2.934751in}{3.464738in}}{\pgfqpoint{2.934751in}{3.453687in}}%
\pgfpathcurveto{\pgfqpoint{2.934751in}{3.442637in}}{\pgfqpoint{2.939141in}{3.432038in}}{\pgfqpoint{2.946955in}{3.424225in}}%
\pgfpathcurveto{\pgfqpoint{2.954768in}{3.416411in}}{\pgfqpoint{2.965367in}{3.412021in}}{\pgfqpoint{2.976418in}{3.412021in}}%
\pgfpathclose%
\pgfusepath{stroke,fill}%
\end{pgfscope}%
\begin{pgfscope}%
\pgfpathrectangle{\pgfqpoint{0.600000in}{0.600000in}}{\pgfqpoint{3.900000in}{3.900000in}}%
\pgfusepath{clip}%
\pgfsetbuttcap%
\pgfsetroundjoin%
\definecolor{currentfill}{rgb}{0.121569,0.466667,0.705882}%
\pgfsetfillcolor{currentfill}%
\pgfsetlinewidth{1.003750pt}%
\definecolor{currentstroke}{rgb}{0.121569,0.466667,0.705882}%
\pgfsetstrokecolor{currentstroke}%
\pgfsetdash{}{0pt}%
\pgfpathmoveto{\pgfqpoint{2.253513in}{2.350737in}}%
\pgfpathcurveto{\pgfqpoint{2.264563in}{2.350737in}}{\pgfqpoint{2.275162in}{2.355127in}}{\pgfqpoint{2.282976in}{2.362941in}}%
\pgfpathcurveto{\pgfqpoint{2.290790in}{2.370754in}}{\pgfqpoint{2.295180in}{2.381353in}}{\pgfqpoint{2.295180in}{2.392403in}}%
\pgfpathcurveto{\pgfqpoint{2.295180in}{2.403453in}}{\pgfqpoint{2.290790in}{2.414052in}}{\pgfqpoint{2.282976in}{2.421866in}}%
\pgfpathcurveto{\pgfqpoint{2.275162in}{2.429680in}}{\pgfqpoint{2.264563in}{2.434070in}}{\pgfqpoint{2.253513in}{2.434070in}}%
\pgfpathcurveto{\pgfqpoint{2.242463in}{2.434070in}}{\pgfqpoint{2.231864in}{2.429680in}}{\pgfqpoint{2.224050in}{2.421866in}}%
\pgfpathcurveto{\pgfqpoint{2.216237in}{2.414052in}}{\pgfqpoint{2.211847in}{2.403453in}}{\pgfqpoint{2.211847in}{2.392403in}}%
\pgfpathcurveto{\pgfqpoint{2.211847in}{2.381353in}}{\pgfqpoint{2.216237in}{2.370754in}}{\pgfqpoint{2.224050in}{2.362941in}}%
\pgfpathcurveto{\pgfqpoint{2.231864in}{2.355127in}}{\pgfqpoint{2.242463in}{2.350737in}}{\pgfqpoint{2.253513in}{2.350737in}}%
\pgfpathclose%
\pgfusepath{stroke,fill}%
\end{pgfscope}%
\begin{pgfscope}%
\pgfpathrectangle{\pgfqpoint{0.600000in}{0.600000in}}{\pgfqpoint{3.900000in}{3.900000in}}%
\pgfusepath{clip}%
\pgfsetbuttcap%
\pgfsetroundjoin%
\definecolor{currentfill}{rgb}{0.121569,0.466667,0.705882}%
\pgfsetfillcolor{currentfill}%
\pgfsetlinewidth{1.003750pt}%
\definecolor{currentstroke}{rgb}{0.121569,0.466667,0.705882}%
\pgfsetstrokecolor{currentstroke}%
\pgfsetdash{}{0pt}%
\pgfpathmoveto{\pgfqpoint{2.549145in}{2.161689in}}%
\pgfpathcurveto{\pgfqpoint{2.560195in}{2.161689in}}{\pgfqpoint{2.570794in}{2.166079in}}{\pgfqpoint{2.578608in}{2.173893in}}%
\pgfpathcurveto{\pgfqpoint{2.586422in}{2.181707in}}{\pgfqpoint{2.590812in}{2.192306in}}{\pgfqpoint{2.590812in}{2.203356in}}%
\pgfpathcurveto{\pgfqpoint{2.590812in}{2.214406in}}{\pgfqpoint{2.586422in}{2.225005in}}{\pgfqpoint{2.578608in}{2.232819in}}%
\pgfpathcurveto{\pgfqpoint{2.570794in}{2.240632in}}{\pgfqpoint{2.560195in}{2.245022in}}{\pgfqpoint{2.549145in}{2.245022in}}%
\pgfpathcurveto{\pgfqpoint{2.538095in}{2.245022in}}{\pgfqpoint{2.527496in}{2.240632in}}{\pgfqpoint{2.519682in}{2.232819in}}%
\pgfpathcurveto{\pgfqpoint{2.511869in}{2.225005in}}{\pgfqpoint{2.507478in}{2.214406in}}{\pgfqpoint{2.507478in}{2.203356in}}%
\pgfpathcurveto{\pgfqpoint{2.507478in}{2.192306in}}{\pgfqpoint{2.511869in}{2.181707in}}{\pgfqpoint{2.519682in}{2.173893in}}%
\pgfpathcurveto{\pgfqpoint{2.527496in}{2.166079in}}{\pgfqpoint{2.538095in}{2.161689in}}{\pgfqpoint{2.549145in}{2.161689in}}%
\pgfpathclose%
\pgfusepath{stroke,fill}%
\end{pgfscope}%
\begin{pgfscope}%
\pgfpathrectangle{\pgfqpoint{0.600000in}{0.600000in}}{\pgfqpoint{3.900000in}{3.900000in}}%
\pgfusepath{clip}%
\pgfsetbuttcap%
\pgfsetroundjoin%
\definecolor{currentfill}{rgb}{0.121569,0.466667,0.705882}%
\pgfsetfillcolor{currentfill}%
\pgfsetlinewidth{1.003750pt}%
\definecolor{currentstroke}{rgb}{0.121569,0.466667,0.705882}%
\pgfsetstrokecolor{currentstroke}%
\pgfsetdash{}{0pt}%
\pgfpathmoveto{\pgfqpoint{3.046188in}{2.150308in}}%
\pgfpathcurveto{\pgfqpoint{3.057238in}{2.150308in}}{\pgfqpoint{3.067837in}{2.154699in}}{\pgfqpoint{3.075650in}{2.162512in}}%
\pgfpathcurveto{\pgfqpoint{3.083464in}{2.170326in}}{\pgfqpoint{3.087854in}{2.180925in}}{\pgfqpoint{3.087854in}{2.191975in}}%
\pgfpathcurveto{\pgfqpoint{3.087854in}{2.203025in}}{\pgfqpoint{3.083464in}{2.213624in}}{\pgfqpoint{3.075650in}{2.221438in}}%
\pgfpathcurveto{\pgfqpoint{3.067837in}{2.229251in}}{\pgfqpoint{3.057238in}{2.233642in}}{\pgfqpoint{3.046188in}{2.233642in}}%
\pgfpathcurveto{\pgfqpoint{3.035137in}{2.233642in}}{\pgfqpoint{3.024538in}{2.229251in}}{\pgfqpoint{3.016725in}{2.221438in}}%
\pgfpathcurveto{\pgfqpoint{3.008911in}{2.213624in}}{\pgfqpoint{3.004521in}{2.203025in}}{\pgfqpoint{3.004521in}{2.191975in}}%
\pgfpathcurveto{\pgfqpoint{3.004521in}{2.180925in}}{\pgfqpoint{3.008911in}{2.170326in}}{\pgfqpoint{3.016725in}{2.162512in}}%
\pgfpathcurveto{\pgfqpoint{3.024538in}{2.154699in}}{\pgfqpoint{3.035137in}{2.150308in}}{\pgfqpoint{3.046188in}{2.150308in}}%
\pgfpathclose%
\pgfusepath{stroke,fill}%
\end{pgfscope}%
\begin{pgfscope}%
\pgfpathrectangle{\pgfqpoint{0.600000in}{0.600000in}}{\pgfqpoint{3.900000in}{3.900000in}}%
\pgfusepath{clip}%
\pgfsetbuttcap%
\pgfsetroundjoin%
\definecolor{currentfill}{rgb}{0.121569,0.466667,0.705882}%
\pgfsetfillcolor{currentfill}%
\pgfsetlinewidth{1.003750pt}%
\definecolor{currentstroke}{rgb}{0.121569,0.466667,0.705882}%
\pgfsetstrokecolor{currentstroke}%
\pgfsetdash{}{0pt}%
\pgfpathmoveto{\pgfqpoint{2.723803in}{2.385172in}}%
\pgfpathcurveto{\pgfqpoint{2.734853in}{2.385172in}}{\pgfqpoint{2.745452in}{2.389563in}}{\pgfqpoint{2.753265in}{2.397376in}}%
\pgfpathcurveto{\pgfqpoint{2.761079in}{2.405190in}}{\pgfqpoint{2.765469in}{2.415789in}}{\pgfqpoint{2.765469in}{2.426839in}}%
\pgfpathcurveto{\pgfqpoint{2.765469in}{2.437889in}}{\pgfqpoint{2.761079in}{2.448488in}}{\pgfqpoint{2.753265in}{2.456302in}}%
\pgfpathcurveto{\pgfqpoint{2.745452in}{2.464116in}}{\pgfqpoint{2.734853in}{2.468506in}}{\pgfqpoint{2.723803in}{2.468506in}}%
\pgfpathcurveto{\pgfqpoint{2.712752in}{2.468506in}}{\pgfqpoint{2.702153in}{2.464116in}}{\pgfqpoint{2.694340in}{2.456302in}}%
\pgfpathcurveto{\pgfqpoint{2.686526in}{2.448488in}}{\pgfqpoint{2.682136in}{2.437889in}}{\pgfqpoint{2.682136in}{2.426839in}}%
\pgfpathcurveto{\pgfqpoint{2.682136in}{2.415789in}}{\pgfqpoint{2.686526in}{2.405190in}}{\pgfqpoint{2.694340in}{2.397376in}}%
\pgfpathcurveto{\pgfqpoint{2.702153in}{2.389563in}}{\pgfqpoint{2.712752in}{2.385172in}}{\pgfqpoint{2.723803in}{2.385172in}}%
\pgfpathclose%
\pgfusepath{stroke,fill}%
\end{pgfscope}%
\begin{pgfscope}%
\pgfpathrectangle{\pgfqpoint{0.600000in}{0.600000in}}{\pgfqpoint{3.900000in}{3.900000in}}%
\pgfusepath{clip}%
\pgfsetbuttcap%
\pgfsetroundjoin%
\definecolor{currentfill}{rgb}{0.121569,0.466667,0.705882}%
\pgfsetfillcolor{currentfill}%
\pgfsetlinewidth{1.003750pt}%
\definecolor{currentstroke}{rgb}{0.121569,0.466667,0.705882}%
\pgfsetstrokecolor{currentstroke}%
\pgfsetdash{}{0pt}%
\pgfpathmoveto{\pgfqpoint{2.798549in}{2.739115in}}%
\pgfpathcurveto{\pgfqpoint{2.809599in}{2.739115in}}{\pgfqpoint{2.820198in}{2.743505in}}{\pgfqpoint{2.828012in}{2.751319in}}%
\pgfpathcurveto{\pgfqpoint{2.835826in}{2.759132in}}{\pgfqpoint{2.840216in}{2.769731in}}{\pgfqpoint{2.840216in}{2.780782in}}%
\pgfpathcurveto{\pgfqpoint{2.840216in}{2.791832in}}{\pgfqpoint{2.835826in}{2.802431in}}{\pgfqpoint{2.828012in}{2.810244in}}%
\pgfpathcurveto{\pgfqpoint{2.820198in}{2.818058in}}{\pgfqpoint{2.809599in}{2.822448in}}{\pgfqpoint{2.798549in}{2.822448in}}%
\pgfpathcurveto{\pgfqpoint{2.787499in}{2.822448in}}{\pgfqpoint{2.776900in}{2.818058in}}{\pgfqpoint{2.769086in}{2.810244in}}%
\pgfpathcurveto{\pgfqpoint{2.761273in}{2.802431in}}{\pgfqpoint{2.756882in}{2.791832in}}{\pgfqpoint{2.756882in}{2.780782in}}%
\pgfpathcurveto{\pgfqpoint{2.756882in}{2.769731in}}{\pgfqpoint{2.761273in}{2.759132in}}{\pgfqpoint{2.769086in}{2.751319in}}%
\pgfpathcurveto{\pgfqpoint{2.776900in}{2.743505in}}{\pgfqpoint{2.787499in}{2.739115in}}{\pgfqpoint{2.798549in}{2.739115in}}%
\pgfpathclose%
\pgfusepath{stroke,fill}%
\end{pgfscope}%
\begin{pgfscope}%
\pgfpathrectangle{\pgfqpoint{0.600000in}{0.600000in}}{\pgfqpoint{3.900000in}{3.900000in}}%
\pgfusepath{clip}%
\pgfsetbuttcap%
\pgfsetroundjoin%
\definecolor{currentfill}{rgb}{0.121569,0.466667,0.705882}%
\pgfsetfillcolor{currentfill}%
\pgfsetlinewidth{1.003750pt}%
\definecolor{currentstroke}{rgb}{0.121569,0.466667,0.705882}%
\pgfsetstrokecolor{currentstroke}%
\pgfsetdash{}{0pt}%
\pgfpathmoveto{\pgfqpoint{2.331413in}{2.328752in}}%
\pgfpathcurveto{\pgfqpoint{2.342463in}{2.328752in}}{\pgfqpoint{2.353062in}{2.333142in}}{\pgfqpoint{2.360876in}{2.340956in}}%
\pgfpathcurveto{\pgfqpoint{2.368689in}{2.348769in}}{\pgfqpoint{2.373080in}{2.359368in}}{\pgfqpoint{2.373080in}{2.370418in}}%
\pgfpathcurveto{\pgfqpoint{2.373080in}{2.381468in}}{\pgfqpoint{2.368689in}{2.392068in}}{\pgfqpoint{2.360876in}{2.399881in}}%
\pgfpathcurveto{\pgfqpoint{2.353062in}{2.407695in}}{\pgfqpoint{2.342463in}{2.412085in}}{\pgfqpoint{2.331413in}{2.412085in}}%
\pgfpathcurveto{\pgfqpoint{2.320363in}{2.412085in}}{\pgfqpoint{2.309764in}{2.407695in}}{\pgfqpoint{2.301950in}{2.399881in}}%
\pgfpathcurveto{\pgfqpoint{2.294137in}{2.392068in}}{\pgfqpoint{2.289746in}{2.381468in}}{\pgfqpoint{2.289746in}{2.370418in}}%
\pgfpathcurveto{\pgfqpoint{2.289746in}{2.359368in}}{\pgfqpoint{2.294137in}{2.348769in}}{\pgfqpoint{2.301950in}{2.340956in}}%
\pgfpathcurveto{\pgfqpoint{2.309764in}{2.333142in}}{\pgfqpoint{2.320363in}{2.328752in}}{\pgfqpoint{2.331413in}{2.328752in}}%
\pgfpathclose%
\pgfusepath{stroke,fill}%
\end{pgfscope}%
\begin{pgfscope}%
\pgfpathrectangle{\pgfqpoint{0.600000in}{0.600000in}}{\pgfqpoint{3.900000in}{3.900000in}}%
\pgfusepath{clip}%
\pgfsetbuttcap%
\pgfsetroundjoin%
\definecolor{currentfill}{rgb}{0.121569,0.466667,0.705882}%
\pgfsetfillcolor{currentfill}%
\pgfsetlinewidth{1.003750pt}%
\definecolor{currentstroke}{rgb}{0.121569,0.466667,0.705882}%
\pgfsetstrokecolor{currentstroke}%
\pgfsetdash{}{0pt}%
\pgfpathmoveto{\pgfqpoint{3.202557in}{2.854469in}}%
\pgfpathcurveto{\pgfqpoint{3.213607in}{2.854469in}}{\pgfqpoint{3.224206in}{2.858859in}}{\pgfqpoint{3.232020in}{2.866673in}}%
\pgfpathcurveto{\pgfqpoint{3.239833in}{2.874486in}}{\pgfqpoint{3.244224in}{2.885085in}}{\pgfqpoint{3.244224in}{2.896135in}}%
\pgfpathcurveto{\pgfqpoint{3.244224in}{2.907186in}}{\pgfqpoint{3.239833in}{2.917785in}}{\pgfqpoint{3.232020in}{2.925598in}}%
\pgfpathcurveto{\pgfqpoint{3.224206in}{2.933412in}}{\pgfqpoint{3.213607in}{2.937802in}}{\pgfqpoint{3.202557in}{2.937802in}}%
\pgfpathcurveto{\pgfqpoint{3.191507in}{2.937802in}}{\pgfqpoint{3.180908in}{2.933412in}}{\pgfqpoint{3.173094in}{2.925598in}}%
\pgfpathcurveto{\pgfqpoint{3.165281in}{2.917785in}}{\pgfqpoint{3.160890in}{2.907186in}}{\pgfqpoint{3.160890in}{2.896135in}}%
\pgfpathcurveto{\pgfqpoint{3.160890in}{2.885085in}}{\pgfqpoint{3.165281in}{2.874486in}}{\pgfqpoint{3.173094in}{2.866673in}}%
\pgfpathcurveto{\pgfqpoint{3.180908in}{2.858859in}}{\pgfqpoint{3.191507in}{2.854469in}}{\pgfqpoint{3.202557in}{2.854469in}}%
\pgfpathclose%
\pgfusepath{stroke,fill}%
\end{pgfscope}%
\begin{pgfscope}%
\pgfpathrectangle{\pgfqpoint{0.600000in}{0.600000in}}{\pgfqpoint{3.900000in}{3.900000in}}%
\pgfusepath{clip}%
\pgfsetbuttcap%
\pgfsetroundjoin%
\definecolor{currentfill}{rgb}{0.121569,0.466667,0.705882}%
\pgfsetfillcolor{currentfill}%
\pgfsetlinewidth{1.003750pt}%
\definecolor{currentstroke}{rgb}{0.121569,0.466667,0.705882}%
\pgfsetstrokecolor{currentstroke}%
\pgfsetdash{}{0pt}%
\pgfpathmoveto{\pgfqpoint{1.015268in}{1.528967in}}%
\pgfpathcurveto{\pgfqpoint{1.026318in}{1.528967in}}{\pgfqpoint{1.036917in}{1.533357in}}{\pgfqpoint{1.044730in}{1.541170in}}%
\pgfpathcurveto{\pgfqpoint{1.052544in}{1.548984in}}{\pgfqpoint{1.056934in}{1.559583in}}{\pgfqpoint{1.056934in}{1.570633in}}%
\pgfpathcurveto{\pgfqpoint{1.056934in}{1.581683in}}{\pgfqpoint{1.052544in}{1.592282in}}{\pgfqpoint{1.044730in}{1.600096in}}%
\pgfpathcurveto{\pgfqpoint{1.036917in}{1.607910in}}{\pgfqpoint{1.026318in}{1.612300in}}{\pgfqpoint{1.015268in}{1.612300in}}%
\pgfpathcurveto{\pgfqpoint{1.004217in}{1.612300in}}{\pgfqpoint{0.993618in}{1.607910in}}{\pgfqpoint{0.985805in}{1.600096in}}%
\pgfpathcurveto{\pgfqpoint{0.977991in}{1.592282in}}{\pgfqpoint{0.973601in}{1.581683in}}{\pgfqpoint{0.973601in}{1.570633in}}%
\pgfpathcurveto{\pgfqpoint{0.973601in}{1.559583in}}{\pgfqpoint{0.977991in}{1.548984in}}{\pgfqpoint{0.985805in}{1.541170in}}%
\pgfpathcurveto{\pgfqpoint{0.993618in}{1.533357in}}{\pgfqpoint{1.004217in}{1.528967in}}{\pgfqpoint{1.015268in}{1.528967in}}%
\pgfpathclose%
\pgfusepath{stroke,fill}%
\end{pgfscope}%
\begin{pgfscope}%
\pgfpathrectangle{\pgfqpoint{0.600000in}{0.600000in}}{\pgfqpoint{3.900000in}{3.900000in}}%
\pgfusepath{clip}%
\pgfsetbuttcap%
\pgfsetroundjoin%
\definecolor{currentfill}{rgb}{0.121569,0.466667,0.705882}%
\pgfsetfillcolor{currentfill}%
\pgfsetlinewidth{1.003750pt}%
\definecolor{currentstroke}{rgb}{0.121569,0.466667,0.705882}%
\pgfsetstrokecolor{currentstroke}%
\pgfsetdash{}{0pt}%
\pgfpathmoveto{\pgfqpoint{2.570531in}{2.233245in}}%
\pgfpathcurveto{\pgfqpoint{2.581581in}{2.233245in}}{\pgfqpoint{2.592180in}{2.237635in}}{\pgfqpoint{2.599994in}{2.245449in}}%
\pgfpathcurveto{\pgfqpoint{2.607807in}{2.253262in}}{\pgfqpoint{2.612197in}{2.263861in}}{\pgfqpoint{2.612197in}{2.274911in}}%
\pgfpathcurveto{\pgfqpoint{2.612197in}{2.285962in}}{\pgfqpoint{2.607807in}{2.296561in}}{\pgfqpoint{2.599994in}{2.304374in}}%
\pgfpathcurveto{\pgfqpoint{2.592180in}{2.312188in}}{\pgfqpoint{2.581581in}{2.316578in}}{\pgfqpoint{2.570531in}{2.316578in}}%
\pgfpathcurveto{\pgfqpoint{2.559481in}{2.316578in}}{\pgfqpoint{2.548882in}{2.312188in}}{\pgfqpoint{2.541068in}{2.304374in}}%
\pgfpathcurveto{\pgfqpoint{2.533254in}{2.296561in}}{\pgfqpoint{2.528864in}{2.285962in}}{\pgfqpoint{2.528864in}{2.274911in}}%
\pgfpathcurveto{\pgfqpoint{2.528864in}{2.263861in}}{\pgfqpoint{2.533254in}{2.253262in}}{\pgfqpoint{2.541068in}{2.245449in}}%
\pgfpathcurveto{\pgfqpoint{2.548882in}{2.237635in}}{\pgfqpoint{2.559481in}{2.233245in}}{\pgfqpoint{2.570531in}{2.233245in}}%
\pgfpathclose%
\pgfusepath{stroke,fill}%
\end{pgfscope}%
\begin{pgfscope}%
\pgfpathrectangle{\pgfqpoint{0.600000in}{0.600000in}}{\pgfqpoint{3.900000in}{3.900000in}}%
\pgfusepath{clip}%
\pgfsetbuttcap%
\pgfsetroundjoin%
\definecolor{currentfill}{rgb}{0.121569,0.466667,0.705882}%
\pgfsetfillcolor{currentfill}%
\pgfsetlinewidth{1.003750pt}%
\definecolor{currentstroke}{rgb}{0.121569,0.466667,0.705882}%
\pgfsetstrokecolor{currentstroke}%
\pgfsetdash{}{0pt}%
\pgfpathmoveto{\pgfqpoint{2.260983in}{2.635910in}}%
\pgfpathcurveto{\pgfqpoint{2.272033in}{2.635910in}}{\pgfqpoint{2.282632in}{2.640300in}}{\pgfqpoint{2.290446in}{2.648113in}}%
\pgfpathcurveto{\pgfqpoint{2.298259in}{2.655927in}}{\pgfqpoint{2.302650in}{2.666526in}}{\pgfqpoint{2.302650in}{2.677576in}}%
\pgfpathcurveto{\pgfqpoint{2.302650in}{2.688626in}}{\pgfqpoint{2.298259in}{2.699225in}}{\pgfqpoint{2.290446in}{2.707039in}}%
\pgfpathcurveto{\pgfqpoint{2.282632in}{2.714853in}}{\pgfqpoint{2.272033in}{2.719243in}}{\pgfqpoint{2.260983in}{2.719243in}}%
\pgfpathcurveto{\pgfqpoint{2.249933in}{2.719243in}}{\pgfqpoint{2.239334in}{2.714853in}}{\pgfqpoint{2.231520in}{2.707039in}}%
\pgfpathcurveto{\pgfqpoint{2.223707in}{2.699225in}}{\pgfqpoint{2.219316in}{2.688626in}}{\pgfqpoint{2.219316in}{2.677576in}}%
\pgfpathcurveto{\pgfqpoint{2.219316in}{2.666526in}}{\pgfqpoint{2.223707in}{2.655927in}}{\pgfqpoint{2.231520in}{2.648113in}}%
\pgfpathcurveto{\pgfqpoint{2.239334in}{2.640300in}}{\pgfqpoint{2.249933in}{2.635910in}}{\pgfqpoint{2.260983in}{2.635910in}}%
\pgfpathclose%
\pgfusepath{stroke,fill}%
\end{pgfscope}%
\begin{pgfscope}%
\pgfpathrectangle{\pgfqpoint{0.600000in}{0.600000in}}{\pgfqpoint{3.900000in}{3.900000in}}%
\pgfusepath{clip}%
\pgfsetbuttcap%
\pgfsetroundjoin%
\definecolor{currentfill}{rgb}{0.121569,0.466667,0.705882}%
\pgfsetfillcolor{currentfill}%
\pgfsetlinewidth{1.003750pt}%
\definecolor{currentstroke}{rgb}{0.121569,0.466667,0.705882}%
\pgfsetstrokecolor{currentstroke}%
\pgfsetdash{}{0pt}%
\pgfpathmoveto{\pgfqpoint{2.478370in}{2.229041in}}%
\pgfpathcurveto{\pgfqpoint{2.489420in}{2.229041in}}{\pgfqpoint{2.500019in}{2.233432in}}{\pgfqpoint{2.507832in}{2.241245in}}%
\pgfpathcurveto{\pgfqpoint{2.515646in}{2.249059in}}{\pgfqpoint{2.520036in}{2.259658in}}{\pgfqpoint{2.520036in}{2.270708in}}%
\pgfpathcurveto{\pgfqpoint{2.520036in}{2.281758in}}{\pgfqpoint{2.515646in}{2.292357in}}{\pgfqpoint{2.507832in}{2.300171in}}%
\pgfpathcurveto{\pgfqpoint{2.500019in}{2.307985in}}{\pgfqpoint{2.489420in}{2.312375in}}{\pgfqpoint{2.478370in}{2.312375in}}%
\pgfpathcurveto{\pgfqpoint{2.467320in}{2.312375in}}{\pgfqpoint{2.456721in}{2.307985in}}{\pgfqpoint{2.448907in}{2.300171in}}%
\pgfpathcurveto{\pgfqpoint{2.441093in}{2.292357in}}{\pgfqpoint{2.436703in}{2.281758in}}{\pgfqpoint{2.436703in}{2.270708in}}%
\pgfpathcurveto{\pgfqpoint{2.436703in}{2.259658in}}{\pgfqpoint{2.441093in}{2.249059in}}{\pgfqpoint{2.448907in}{2.241245in}}%
\pgfpathcurveto{\pgfqpoint{2.456721in}{2.233432in}}{\pgfqpoint{2.467320in}{2.229041in}}{\pgfqpoint{2.478370in}{2.229041in}}%
\pgfpathclose%
\pgfusepath{stroke,fill}%
\end{pgfscope}%
\begin{pgfscope}%
\pgfpathrectangle{\pgfqpoint{0.600000in}{0.600000in}}{\pgfqpoint{3.900000in}{3.900000in}}%
\pgfusepath{clip}%
\pgfsetbuttcap%
\pgfsetroundjoin%
\definecolor{currentfill}{rgb}{0.121569,0.466667,0.705882}%
\pgfsetfillcolor{currentfill}%
\pgfsetlinewidth{1.003750pt}%
\definecolor{currentstroke}{rgb}{0.121569,0.466667,0.705882}%
\pgfsetstrokecolor{currentstroke}%
\pgfsetdash{}{0pt}%
\pgfpathmoveto{\pgfqpoint{3.036024in}{2.631907in}}%
\pgfpathcurveto{\pgfqpoint{3.047074in}{2.631907in}}{\pgfqpoint{3.057673in}{2.636297in}}{\pgfqpoint{3.065487in}{2.644110in}}%
\pgfpathcurveto{\pgfqpoint{3.073300in}{2.651924in}}{\pgfqpoint{3.077691in}{2.662523in}}{\pgfqpoint{3.077691in}{2.673573in}}%
\pgfpathcurveto{\pgfqpoint{3.077691in}{2.684623in}}{\pgfqpoint{3.073300in}{2.695222in}}{\pgfqpoint{3.065487in}{2.703036in}}%
\pgfpathcurveto{\pgfqpoint{3.057673in}{2.710850in}}{\pgfqpoint{3.047074in}{2.715240in}}{\pgfqpoint{3.036024in}{2.715240in}}%
\pgfpathcurveto{\pgfqpoint{3.024974in}{2.715240in}}{\pgfqpoint{3.014375in}{2.710850in}}{\pgfqpoint{3.006561in}{2.703036in}}%
\pgfpathcurveto{\pgfqpoint{2.998747in}{2.695222in}}{\pgfqpoint{2.994357in}{2.684623in}}{\pgfqpoint{2.994357in}{2.673573in}}%
\pgfpathcurveto{\pgfqpoint{2.994357in}{2.662523in}}{\pgfqpoint{2.998747in}{2.651924in}}{\pgfqpoint{3.006561in}{2.644110in}}%
\pgfpathcurveto{\pgfqpoint{3.014375in}{2.636297in}}{\pgfqpoint{3.024974in}{2.631907in}}{\pgfqpoint{3.036024in}{2.631907in}}%
\pgfpathclose%
\pgfusepath{stroke,fill}%
\end{pgfscope}%
\begin{pgfscope}%
\pgfpathrectangle{\pgfqpoint{0.600000in}{0.600000in}}{\pgfqpoint{3.900000in}{3.900000in}}%
\pgfusepath{clip}%
\pgfsetbuttcap%
\pgfsetroundjoin%
\definecolor{currentfill}{rgb}{0.121569,0.466667,0.705882}%
\pgfsetfillcolor{currentfill}%
\pgfsetlinewidth{1.003750pt}%
\definecolor{currentstroke}{rgb}{0.121569,0.466667,0.705882}%
\pgfsetstrokecolor{currentstroke}%
\pgfsetdash{}{0pt}%
\pgfpathmoveto{\pgfqpoint{1.918180in}{3.109409in}}%
\pgfpathcurveto{\pgfqpoint{1.929230in}{3.109409in}}{\pgfqpoint{1.939829in}{3.113800in}}{\pgfqpoint{1.947643in}{3.121613in}}%
\pgfpathcurveto{\pgfqpoint{1.955456in}{3.129427in}}{\pgfqpoint{1.959846in}{3.140026in}}{\pgfqpoint{1.959846in}{3.151076in}}%
\pgfpathcurveto{\pgfqpoint{1.959846in}{3.162126in}}{\pgfqpoint{1.955456in}{3.172725in}}{\pgfqpoint{1.947643in}{3.180539in}}%
\pgfpathcurveto{\pgfqpoint{1.939829in}{3.188352in}}{\pgfqpoint{1.929230in}{3.192743in}}{\pgfqpoint{1.918180in}{3.192743in}}%
\pgfpathcurveto{\pgfqpoint{1.907130in}{3.192743in}}{\pgfqpoint{1.896531in}{3.188352in}}{\pgfqpoint{1.888717in}{3.180539in}}%
\pgfpathcurveto{\pgfqpoint{1.880903in}{3.172725in}}{\pgfqpoint{1.876513in}{3.162126in}}{\pgfqpoint{1.876513in}{3.151076in}}%
\pgfpathcurveto{\pgfqpoint{1.876513in}{3.140026in}}{\pgfqpoint{1.880903in}{3.129427in}}{\pgfqpoint{1.888717in}{3.121613in}}%
\pgfpathcurveto{\pgfqpoint{1.896531in}{3.113800in}}{\pgfqpoint{1.907130in}{3.109409in}}{\pgfqpoint{1.918180in}{3.109409in}}%
\pgfpathclose%
\pgfusepath{stroke,fill}%
\end{pgfscope}%
\begin{pgfscope}%
\pgfpathrectangle{\pgfqpoint{0.600000in}{0.600000in}}{\pgfqpoint{3.900000in}{3.900000in}}%
\pgfusepath{clip}%
\pgfsetbuttcap%
\pgfsetroundjoin%
\definecolor{currentfill}{rgb}{0.121569,0.466667,0.705882}%
\pgfsetfillcolor{currentfill}%
\pgfsetlinewidth{1.003750pt}%
\definecolor{currentstroke}{rgb}{0.121569,0.466667,0.705882}%
\pgfsetstrokecolor{currentstroke}%
\pgfsetdash{}{0pt}%
\pgfpathmoveto{\pgfqpoint{2.218995in}{1.777554in}}%
\pgfpathcurveto{\pgfqpoint{2.230045in}{1.777554in}}{\pgfqpoint{2.240644in}{1.781944in}}{\pgfqpoint{2.248457in}{1.789758in}}%
\pgfpathcurveto{\pgfqpoint{2.256271in}{1.797572in}}{\pgfqpoint{2.260661in}{1.808171in}}{\pgfqpoint{2.260661in}{1.819221in}}%
\pgfpathcurveto{\pgfqpoint{2.260661in}{1.830271in}}{\pgfqpoint{2.256271in}{1.840870in}}{\pgfqpoint{2.248457in}{1.848684in}}%
\pgfpathcurveto{\pgfqpoint{2.240644in}{1.856497in}}{\pgfqpoint{2.230045in}{1.860888in}}{\pgfqpoint{2.218995in}{1.860888in}}%
\pgfpathcurveto{\pgfqpoint{2.207944in}{1.860888in}}{\pgfqpoint{2.197345in}{1.856497in}}{\pgfqpoint{2.189532in}{1.848684in}}%
\pgfpathcurveto{\pgfqpoint{2.181718in}{1.840870in}}{\pgfqpoint{2.177328in}{1.830271in}}{\pgfqpoint{2.177328in}{1.819221in}}%
\pgfpathcurveto{\pgfqpoint{2.177328in}{1.808171in}}{\pgfqpoint{2.181718in}{1.797572in}}{\pgfqpoint{2.189532in}{1.789758in}}%
\pgfpathcurveto{\pgfqpoint{2.197345in}{1.781944in}}{\pgfqpoint{2.207944in}{1.777554in}}{\pgfqpoint{2.218995in}{1.777554in}}%
\pgfpathclose%
\pgfusepath{stroke,fill}%
\end{pgfscope}%
\begin{pgfscope}%
\pgfpathrectangle{\pgfqpoint{0.600000in}{0.600000in}}{\pgfqpoint{3.900000in}{3.900000in}}%
\pgfusepath{clip}%
\pgfsetbuttcap%
\pgfsetroundjoin%
\definecolor{currentfill}{rgb}{0.121569,0.466667,0.705882}%
\pgfsetfillcolor{currentfill}%
\pgfsetlinewidth{1.003750pt}%
\definecolor{currentstroke}{rgb}{0.121569,0.466667,0.705882}%
\pgfsetstrokecolor{currentstroke}%
\pgfsetdash{}{0pt}%
\pgfpathmoveto{\pgfqpoint{2.943812in}{3.135509in}}%
\pgfpathcurveto{\pgfqpoint{2.954862in}{3.135509in}}{\pgfqpoint{2.965461in}{3.139899in}}{\pgfqpoint{2.973275in}{3.147713in}}%
\pgfpathcurveto{\pgfqpoint{2.981088in}{3.155526in}}{\pgfqpoint{2.985479in}{3.166126in}}{\pgfqpoint{2.985479in}{3.177176in}}%
\pgfpathcurveto{\pgfqpoint{2.985479in}{3.188226in}}{\pgfqpoint{2.981088in}{3.198825in}}{\pgfqpoint{2.973275in}{3.206638in}}%
\pgfpathcurveto{\pgfqpoint{2.965461in}{3.214452in}}{\pgfqpoint{2.954862in}{3.218842in}}{\pgfqpoint{2.943812in}{3.218842in}}%
\pgfpathcurveto{\pgfqpoint{2.932762in}{3.218842in}}{\pgfqpoint{2.922163in}{3.214452in}}{\pgfqpoint{2.914349in}{3.206638in}}%
\pgfpathcurveto{\pgfqpoint{2.906536in}{3.198825in}}{\pgfqpoint{2.902145in}{3.188226in}}{\pgfqpoint{2.902145in}{3.177176in}}%
\pgfpathcurveto{\pgfqpoint{2.902145in}{3.166126in}}{\pgfqpoint{2.906536in}{3.155526in}}{\pgfqpoint{2.914349in}{3.147713in}}%
\pgfpathcurveto{\pgfqpoint{2.922163in}{3.139899in}}{\pgfqpoint{2.932762in}{3.135509in}}{\pgfqpoint{2.943812in}{3.135509in}}%
\pgfpathclose%
\pgfusepath{stroke,fill}%
\end{pgfscope}%
\begin{pgfscope}%
\pgfpathrectangle{\pgfqpoint{0.600000in}{0.600000in}}{\pgfqpoint{3.900000in}{3.900000in}}%
\pgfusepath{clip}%
\pgfsetbuttcap%
\pgfsetroundjoin%
\definecolor{currentfill}{rgb}{0.121569,0.466667,0.705882}%
\pgfsetfillcolor{currentfill}%
\pgfsetlinewidth{1.003750pt}%
\definecolor{currentstroke}{rgb}{0.121569,0.466667,0.705882}%
\pgfsetstrokecolor{currentstroke}%
\pgfsetdash{}{0pt}%
\pgfpathmoveto{\pgfqpoint{2.829193in}{3.438922in}}%
\pgfpathcurveto{\pgfqpoint{2.840243in}{3.438922in}}{\pgfqpoint{2.850842in}{3.443312in}}{\pgfqpoint{2.858656in}{3.451126in}}%
\pgfpathcurveto{\pgfqpoint{2.866469in}{3.458939in}}{\pgfqpoint{2.870859in}{3.469538in}}{\pgfqpoint{2.870859in}{3.480588in}}%
\pgfpathcurveto{\pgfqpoint{2.870859in}{3.491639in}}{\pgfqpoint{2.866469in}{3.502238in}}{\pgfqpoint{2.858656in}{3.510051in}}%
\pgfpathcurveto{\pgfqpoint{2.850842in}{3.517865in}}{\pgfqpoint{2.840243in}{3.522255in}}{\pgfqpoint{2.829193in}{3.522255in}}%
\pgfpathcurveto{\pgfqpoint{2.818143in}{3.522255in}}{\pgfqpoint{2.807544in}{3.517865in}}{\pgfqpoint{2.799730in}{3.510051in}}%
\pgfpathcurveto{\pgfqpoint{2.791916in}{3.502238in}}{\pgfqpoint{2.787526in}{3.491639in}}{\pgfqpoint{2.787526in}{3.480588in}}%
\pgfpathcurveto{\pgfqpoint{2.787526in}{3.469538in}}{\pgfqpoint{2.791916in}{3.458939in}}{\pgfqpoint{2.799730in}{3.451126in}}%
\pgfpathcurveto{\pgfqpoint{2.807544in}{3.443312in}}{\pgfqpoint{2.818143in}{3.438922in}}{\pgfqpoint{2.829193in}{3.438922in}}%
\pgfpathclose%
\pgfusepath{stroke,fill}%
\end{pgfscope}%
\begin{pgfscope}%
\pgfpathrectangle{\pgfqpoint{0.600000in}{0.600000in}}{\pgfqpoint{3.900000in}{3.900000in}}%
\pgfusepath{clip}%
\pgfsetbuttcap%
\pgfsetroundjoin%
\definecolor{currentfill}{rgb}{0.121569,0.466667,0.705882}%
\pgfsetfillcolor{currentfill}%
\pgfsetlinewidth{1.003750pt}%
\definecolor{currentstroke}{rgb}{0.121569,0.466667,0.705882}%
\pgfsetstrokecolor{currentstroke}%
\pgfsetdash{}{0pt}%
\pgfpathmoveto{\pgfqpoint{3.316869in}{2.448605in}}%
\pgfpathcurveto{\pgfqpoint{3.327920in}{2.448605in}}{\pgfqpoint{3.338519in}{2.452995in}}{\pgfqpoint{3.346332in}{2.460808in}}%
\pgfpathcurveto{\pgfqpoint{3.354146in}{2.468622in}}{\pgfqpoint{3.358536in}{2.479221in}}{\pgfqpoint{3.358536in}{2.490271in}}%
\pgfpathcurveto{\pgfqpoint{3.358536in}{2.501321in}}{\pgfqpoint{3.354146in}{2.511920in}}{\pgfqpoint{3.346332in}{2.519734in}}%
\pgfpathcurveto{\pgfqpoint{3.338519in}{2.527548in}}{\pgfqpoint{3.327920in}{2.531938in}}{\pgfqpoint{3.316869in}{2.531938in}}%
\pgfpathcurveto{\pgfqpoint{3.305819in}{2.531938in}}{\pgfqpoint{3.295220in}{2.527548in}}{\pgfqpoint{3.287407in}{2.519734in}}%
\pgfpathcurveto{\pgfqpoint{3.279593in}{2.511920in}}{\pgfqpoint{3.275203in}{2.501321in}}{\pgfqpoint{3.275203in}{2.490271in}}%
\pgfpathcurveto{\pgfqpoint{3.275203in}{2.479221in}}{\pgfqpoint{3.279593in}{2.468622in}}{\pgfqpoint{3.287407in}{2.460808in}}%
\pgfpathcurveto{\pgfqpoint{3.295220in}{2.452995in}}{\pgfqpoint{3.305819in}{2.448605in}}{\pgfqpoint{3.316869in}{2.448605in}}%
\pgfpathclose%
\pgfusepath{stroke,fill}%
\end{pgfscope}%
\begin{pgfscope}%
\pgfpathrectangle{\pgfqpoint{0.600000in}{0.600000in}}{\pgfqpoint{3.900000in}{3.900000in}}%
\pgfusepath{clip}%
\pgfsetbuttcap%
\pgfsetroundjoin%
\definecolor{currentfill}{rgb}{0.121569,0.466667,0.705882}%
\pgfsetfillcolor{currentfill}%
\pgfsetlinewidth{1.003750pt}%
\definecolor{currentstroke}{rgb}{0.121569,0.466667,0.705882}%
\pgfsetstrokecolor{currentstroke}%
\pgfsetdash{}{0pt}%
\pgfpathmoveto{\pgfqpoint{3.041906in}{2.711754in}}%
\pgfpathcurveto{\pgfqpoint{3.052956in}{2.711754in}}{\pgfqpoint{3.063555in}{2.716144in}}{\pgfqpoint{3.071369in}{2.723958in}}%
\pgfpathcurveto{\pgfqpoint{3.079182in}{2.731772in}}{\pgfqpoint{3.083573in}{2.742371in}}{\pgfqpoint{3.083573in}{2.753421in}}%
\pgfpathcurveto{\pgfqpoint{3.083573in}{2.764471in}}{\pgfqpoint{3.079182in}{2.775070in}}{\pgfqpoint{3.071369in}{2.782884in}}%
\pgfpathcurveto{\pgfqpoint{3.063555in}{2.790697in}}{\pgfqpoint{3.052956in}{2.795088in}}{\pgfqpoint{3.041906in}{2.795088in}}%
\pgfpathcurveto{\pgfqpoint{3.030856in}{2.795088in}}{\pgfqpoint{3.020257in}{2.790697in}}{\pgfqpoint{3.012443in}{2.782884in}}%
\pgfpathcurveto{\pgfqpoint{3.004630in}{2.775070in}}{\pgfqpoint{3.000239in}{2.764471in}}{\pgfqpoint{3.000239in}{2.753421in}}%
\pgfpathcurveto{\pgfqpoint{3.000239in}{2.742371in}}{\pgfqpoint{3.004630in}{2.731772in}}{\pgfqpoint{3.012443in}{2.723958in}}%
\pgfpathcurveto{\pgfqpoint{3.020257in}{2.716144in}}{\pgfqpoint{3.030856in}{2.711754in}}{\pgfqpoint{3.041906in}{2.711754in}}%
\pgfpathclose%
\pgfusepath{stroke,fill}%
\end{pgfscope}%
\begin{pgfscope}%
\pgfpathrectangle{\pgfqpoint{0.600000in}{0.600000in}}{\pgfqpoint{3.900000in}{3.900000in}}%
\pgfusepath{clip}%
\pgfsetbuttcap%
\pgfsetroundjoin%
\definecolor{currentfill}{rgb}{0.121569,0.466667,0.705882}%
\pgfsetfillcolor{currentfill}%
\pgfsetlinewidth{1.003750pt}%
\definecolor{currentstroke}{rgb}{0.121569,0.466667,0.705882}%
\pgfsetstrokecolor{currentstroke}%
\pgfsetdash{}{0pt}%
\pgfpathmoveto{\pgfqpoint{3.044720in}{1.837483in}}%
\pgfpathcurveto{\pgfqpoint{3.055770in}{1.837483in}}{\pgfqpoint{3.066369in}{1.841873in}}{\pgfqpoint{3.074182in}{1.849687in}}%
\pgfpathcurveto{\pgfqpoint{3.081996in}{1.857500in}}{\pgfqpoint{3.086386in}{1.868100in}}{\pgfqpoint{3.086386in}{1.879150in}}%
\pgfpathcurveto{\pgfqpoint{3.086386in}{1.890200in}}{\pgfqpoint{3.081996in}{1.900799in}}{\pgfqpoint{3.074182in}{1.908612in}}%
\pgfpathcurveto{\pgfqpoint{3.066369in}{1.916426in}}{\pgfqpoint{3.055770in}{1.920816in}}{\pgfqpoint{3.044720in}{1.920816in}}%
\pgfpathcurveto{\pgfqpoint{3.033670in}{1.920816in}}{\pgfqpoint{3.023070in}{1.916426in}}{\pgfqpoint{3.015257in}{1.908612in}}%
\pgfpathcurveto{\pgfqpoint{3.007443in}{1.900799in}}{\pgfqpoint{3.003053in}{1.890200in}}{\pgfqpoint{3.003053in}{1.879150in}}%
\pgfpathcurveto{\pgfqpoint{3.003053in}{1.868100in}}{\pgfqpoint{3.007443in}{1.857500in}}{\pgfqpoint{3.015257in}{1.849687in}}%
\pgfpathcurveto{\pgfqpoint{3.023070in}{1.841873in}}{\pgfqpoint{3.033670in}{1.837483in}}{\pgfqpoint{3.044720in}{1.837483in}}%
\pgfpathclose%
\pgfusepath{stroke,fill}%
\end{pgfscope}%
\begin{pgfscope}%
\pgfpathrectangle{\pgfqpoint{0.600000in}{0.600000in}}{\pgfqpoint{3.900000in}{3.900000in}}%
\pgfusepath{clip}%
\pgfsetbuttcap%
\pgfsetroundjoin%
\definecolor{currentfill}{rgb}{0.121569,0.466667,0.705882}%
\pgfsetfillcolor{currentfill}%
\pgfsetlinewidth{1.003750pt}%
\definecolor{currentstroke}{rgb}{0.121569,0.466667,0.705882}%
\pgfsetstrokecolor{currentstroke}%
\pgfsetdash{}{0pt}%
\pgfpathmoveto{\pgfqpoint{2.975348in}{2.862046in}}%
\pgfpathcurveto{\pgfqpoint{2.986399in}{2.862046in}}{\pgfqpoint{2.996998in}{2.866436in}}{\pgfqpoint{3.004811in}{2.874250in}}%
\pgfpathcurveto{\pgfqpoint{3.012625in}{2.882063in}}{\pgfqpoint{3.017015in}{2.892662in}}{\pgfqpoint{3.017015in}{2.903712in}}%
\pgfpathcurveto{\pgfqpoint{3.017015in}{2.914762in}}{\pgfqpoint{3.012625in}{2.925361in}}{\pgfqpoint{3.004811in}{2.933175in}}%
\pgfpathcurveto{\pgfqpoint{2.996998in}{2.940989in}}{\pgfqpoint{2.986399in}{2.945379in}}{\pgfqpoint{2.975348in}{2.945379in}}%
\pgfpathcurveto{\pgfqpoint{2.964298in}{2.945379in}}{\pgfqpoint{2.953699in}{2.940989in}}{\pgfqpoint{2.945886in}{2.933175in}}%
\pgfpathcurveto{\pgfqpoint{2.938072in}{2.925361in}}{\pgfqpoint{2.933682in}{2.914762in}}{\pgfqpoint{2.933682in}{2.903712in}}%
\pgfpathcurveto{\pgfqpoint{2.933682in}{2.892662in}}{\pgfqpoint{2.938072in}{2.882063in}}{\pgfqpoint{2.945886in}{2.874250in}}%
\pgfpathcurveto{\pgfqpoint{2.953699in}{2.866436in}}{\pgfqpoint{2.964298in}{2.862046in}}{\pgfqpoint{2.975348in}{2.862046in}}%
\pgfpathclose%
\pgfusepath{stroke,fill}%
\end{pgfscope}%
\begin{pgfscope}%
\pgfpathrectangle{\pgfqpoint{0.600000in}{0.600000in}}{\pgfqpoint{3.900000in}{3.900000in}}%
\pgfusepath{clip}%
\pgfsetbuttcap%
\pgfsetroundjoin%
\definecolor{currentfill}{rgb}{0.121569,0.466667,0.705882}%
\pgfsetfillcolor{currentfill}%
\pgfsetlinewidth{1.003750pt}%
\definecolor{currentstroke}{rgb}{0.121569,0.466667,0.705882}%
\pgfsetstrokecolor{currentstroke}%
\pgfsetdash{}{0pt}%
\pgfpathmoveto{\pgfqpoint{3.213294in}{3.243818in}}%
\pgfpathcurveto{\pgfqpoint{3.224344in}{3.243818in}}{\pgfqpoint{3.234943in}{3.248209in}}{\pgfqpoint{3.242756in}{3.256022in}}%
\pgfpathcurveto{\pgfqpoint{3.250570in}{3.263836in}}{\pgfqpoint{3.254960in}{3.274435in}}{\pgfqpoint{3.254960in}{3.285485in}}%
\pgfpathcurveto{\pgfqpoint{3.254960in}{3.296535in}}{\pgfqpoint{3.250570in}{3.307134in}}{\pgfqpoint{3.242756in}{3.314948in}}%
\pgfpathcurveto{\pgfqpoint{3.234943in}{3.322761in}}{\pgfqpoint{3.224344in}{3.327152in}}{\pgfqpoint{3.213294in}{3.327152in}}%
\pgfpathcurveto{\pgfqpoint{3.202243in}{3.327152in}}{\pgfqpoint{3.191644in}{3.322761in}}{\pgfqpoint{3.183831in}{3.314948in}}%
\pgfpathcurveto{\pgfqpoint{3.176017in}{3.307134in}}{\pgfqpoint{3.171627in}{3.296535in}}{\pgfqpoint{3.171627in}{3.285485in}}%
\pgfpathcurveto{\pgfqpoint{3.171627in}{3.274435in}}{\pgfqpoint{3.176017in}{3.263836in}}{\pgfqpoint{3.183831in}{3.256022in}}%
\pgfpathcurveto{\pgfqpoint{3.191644in}{3.248209in}}{\pgfqpoint{3.202243in}{3.243818in}}{\pgfqpoint{3.213294in}{3.243818in}}%
\pgfpathclose%
\pgfusepath{stroke,fill}%
\end{pgfscope}%
\begin{pgfscope}%
\pgfpathrectangle{\pgfqpoint{0.600000in}{0.600000in}}{\pgfqpoint{3.900000in}{3.900000in}}%
\pgfusepath{clip}%
\pgfsetbuttcap%
\pgfsetroundjoin%
\definecolor{currentfill}{rgb}{0.121569,0.466667,0.705882}%
\pgfsetfillcolor{currentfill}%
\pgfsetlinewidth{1.003750pt}%
\definecolor{currentstroke}{rgb}{0.121569,0.466667,0.705882}%
\pgfsetstrokecolor{currentstroke}%
\pgfsetdash{}{0pt}%
\pgfpathmoveto{\pgfqpoint{2.809758in}{2.455203in}}%
\pgfpathcurveto{\pgfqpoint{2.820808in}{2.455203in}}{\pgfqpoint{2.831407in}{2.459593in}}{\pgfqpoint{2.839221in}{2.467407in}}%
\pgfpathcurveto{\pgfqpoint{2.847034in}{2.475221in}}{\pgfqpoint{2.851425in}{2.485820in}}{\pgfqpoint{2.851425in}{2.496870in}}%
\pgfpathcurveto{\pgfqpoint{2.851425in}{2.507920in}}{\pgfqpoint{2.847034in}{2.518519in}}{\pgfqpoint{2.839221in}{2.526333in}}%
\pgfpathcurveto{\pgfqpoint{2.831407in}{2.534146in}}{\pgfqpoint{2.820808in}{2.538536in}}{\pgfqpoint{2.809758in}{2.538536in}}%
\pgfpathcurveto{\pgfqpoint{2.798708in}{2.538536in}}{\pgfqpoint{2.788109in}{2.534146in}}{\pgfqpoint{2.780295in}{2.526333in}}%
\pgfpathcurveto{\pgfqpoint{2.772482in}{2.518519in}}{\pgfqpoint{2.768091in}{2.507920in}}{\pgfqpoint{2.768091in}{2.496870in}}%
\pgfpathcurveto{\pgfqpoint{2.768091in}{2.485820in}}{\pgfqpoint{2.772482in}{2.475221in}}{\pgfqpoint{2.780295in}{2.467407in}}%
\pgfpathcurveto{\pgfqpoint{2.788109in}{2.459593in}}{\pgfqpoint{2.798708in}{2.455203in}}{\pgfqpoint{2.809758in}{2.455203in}}%
\pgfpathclose%
\pgfusepath{stroke,fill}%
\end{pgfscope}%
\begin{pgfscope}%
\pgfpathrectangle{\pgfqpoint{0.600000in}{0.600000in}}{\pgfqpoint{3.900000in}{3.900000in}}%
\pgfusepath{clip}%
\pgfsetbuttcap%
\pgfsetroundjoin%
\definecolor{currentfill}{rgb}{0.121569,0.466667,0.705882}%
\pgfsetfillcolor{currentfill}%
\pgfsetlinewidth{1.003750pt}%
\definecolor{currentstroke}{rgb}{0.121569,0.466667,0.705882}%
\pgfsetstrokecolor{currentstroke}%
\pgfsetdash{}{0pt}%
\pgfpathmoveto{\pgfqpoint{3.245647in}{2.436035in}}%
\pgfpathcurveto{\pgfqpoint{3.256697in}{2.436035in}}{\pgfqpoint{3.267296in}{2.440425in}}{\pgfqpoint{3.275110in}{2.448239in}}%
\pgfpathcurveto{\pgfqpoint{3.282923in}{2.456052in}}{\pgfqpoint{3.287314in}{2.466651in}}{\pgfqpoint{3.287314in}{2.477702in}}%
\pgfpathcurveto{\pgfqpoint{3.287314in}{2.488752in}}{\pgfqpoint{3.282923in}{2.499351in}}{\pgfqpoint{3.275110in}{2.507164in}}%
\pgfpathcurveto{\pgfqpoint{3.267296in}{2.514978in}}{\pgfqpoint{3.256697in}{2.519368in}}{\pgfqpoint{3.245647in}{2.519368in}}%
\pgfpathcurveto{\pgfqpoint{3.234597in}{2.519368in}}{\pgfqpoint{3.223998in}{2.514978in}}{\pgfqpoint{3.216184in}{2.507164in}}%
\pgfpathcurveto{\pgfqpoint{3.208371in}{2.499351in}}{\pgfqpoint{3.203980in}{2.488752in}}{\pgfqpoint{3.203980in}{2.477702in}}%
\pgfpathcurveto{\pgfqpoint{3.203980in}{2.466651in}}{\pgfqpoint{3.208371in}{2.456052in}}{\pgfqpoint{3.216184in}{2.448239in}}%
\pgfpathcurveto{\pgfqpoint{3.223998in}{2.440425in}}{\pgfqpoint{3.234597in}{2.436035in}}{\pgfqpoint{3.245647in}{2.436035in}}%
\pgfpathclose%
\pgfusepath{stroke,fill}%
\end{pgfscope}%
\begin{pgfscope}%
\pgfpathrectangle{\pgfqpoint{0.600000in}{0.600000in}}{\pgfqpoint{3.900000in}{3.900000in}}%
\pgfusepath{clip}%
\pgfsetbuttcap%
\pgfsetroundjoin%
\definecolor{currentfill}{rgb}{0.121569,0.466667,0.705882}%
\pgfsetfillcolor{currentfill}%
\pgfsetlinewidth{1.003750pt}%
\definecolor{currentstroke}{rgb}{0.121569,0.466667,0.705882}%
\pgfsetstrokecolor{currentstroke}%
\pgfsetdash{}{0pt}%
\pgfpathmoveto{\pgfqpoint{2.073707in}{2.715606in}}%
\pgfpathcurveto{\pgfqpoint{2.084757in}{2.715606in}}{\pgfqpoint{2.095356in}{2.719996in}}{\pgfqpoint{2.103170in}{2.727810in}}%
\pgfpathcurveto{\pgfqpoint{2.110984in}{2.735624in}}{\pgfqpoint{2.115374in}{2.746223in}}{\pgfqpoint{2.115374in}{2.757273in}}%
\pgfpathcurveto{\pgfqpoint{2.115374in}{2.768323in}}{\pgfqpoint{2.110984in}{2.778922in}}{\pgfqpoint{2.103170in}{2.786736in}}%
\pgfpathcurveto{\pgfqpoint{2.095356in}{2.794549in}}{\pgfqpoint{2.084757in}{2.798939in}}{\pgfqpoint{2.073707in}{2.798939in}}%
\pgfpathcurveto{\pgfqpoint{2.062657in}{2.798939in}}{\pgfqpoint{2.052058in}{2.794549in}}{\pgfqpoint{2.044244in}{2.786736in}}%
\pgfpathcurveto{\pgfqpoint{2.036431in}{2.778922in}}{\pgfqpoint{2.032041in}{2.768323in}}{\pgfqpoint{2.032041in}{2.757273in}}%
\pgfpathcurveto{\pgfqpoint{2.032041in}{2.746223in}}{\pgfqpoint{2.036431in}{2.735624in}}{\pgfqpoint{2.044244in}{2.727810in}}%
\pgfpathcurveto{\pgfqpoint{2.052058in}{2.719996in}}{\pgfqpoint{2.062657in}{2.715606in}}{\pgfqpoint{2.073707in}{2.715606in}}%
\pgfpathclose%
\pgfusepath{stroke,fill}%
\end{pgfscope}%
\begin{pgfscope}%
\pgfpathrectangle{\pgfqpoint{0.600000in}{0.600000in}}{\pgfqpoint{3.900000in}{3.900000in}}%
\pgfusepath{clip}%
\pgfsetbuttcap%
\pgfsetroundjoin%
\definecolor{currentfill}{rgb}{0.121569,0.466667,0.705882}%
\pgfsetfillcolor{currentfill}%
\pgfsetlinewidth{1.003750pt}%
\definecolor{currentstroke}{rgb}{0.121569,0.466667,0.705882}%
\pgfsetstrokecolor{currentstroke}%
\pgfsetdash{}{0pt}%
\pgfpathmoveto{\pgfqpoint{2.570371in}{2.460047in}}%
\pgfpathcurveto{\pgfqpoint{2.581421in}{2.460047in}}{\pgfqpoint{2.592020in}{2.464437in}}{\pgfqpoint{2.599834in}{2.472251in}}%
\pgfpathcurveto{\pgfqpoint{2.607647in}{2.480065in}}{\pgfqpoint{2.612038in}{2.490664in}}{\pgfqpoint{2.612038in}{2.501714in}}%
\pgfpathcurveto{\pgfqpoint{2.612038in}{2.512764in}}{\pgfqpoint{2.607647in}{2.523363in}}{\pgfqpoint{2.599834in}{2.531177in}}%
\pgfpathcurveto{\pgfqpoint{2.592020in}{2.538990in}}{\pgfqpoint{2.581421in}{2.543381in}}{\pgfqpoint{2.570371in}{2.543381in}}%
\pgfpathcurveto{\pgfqpoint{2.559321in}{2.543381in}}{\pgfqpoint{2.548722in}{2.538990in}}{\pgfqpoint{2.540908in}{2.531177in}}%
\pgfpathcurveto{\pgfqpoint{2.533095in}{2.523363in}}{\pgfqpoint{2.528704in}{2.512764in}}{\pgfqpoint{2.528704in}{2.501714in}}%
\pgfpathcurveto{\pgfqpoint{2.528704in}{2.490664in}}{\pgfqpoint{2.533095in}{2.480065in}}{\pgfqpoint{2.540908in}{2.472251in}}%
\pgfpathcurveto{\pgfqpoint{2.548722in}{2.464437in}}{\pgfqpoint{2.559321in}{2.460047in}}{\pgfqpoint{2.570371in}{2.460047in}}%
\pgfpathclose%
\pgfusepath{stroke,fill}%
\end{pgfscope}%
\begin{pgfscope}%
\pgfpathrectangle{\pgfqpoint{0.600000in}{0.600000in}}{\pgfqpoint{3.900000in}{3.900000in}}%
\pgfusepath{clip}%
\pgfsetbuttcap%
\pgfsetroundjoin%
\definecolor{currentfill}{rgb}{0.121569,0.466667,0.705882}%
\pgfsetfillcolor{currentfill}%
\pgfsetlinewidth{1.003750pt}%
\definecolor{currentstroke}{rgb}{0.121569,0.466667,0.705882}%
\pgfsetstrokecolor{currentstroke}%
\pgfsetdash{}{0pt}%
\pgfpathmoveto{\pgfqpoint{3.004173in}{2.419274in}}%
\pgfpathcurveto{\pgfqpoint{3.015223in}{2.419274in}}{\pgfqpoint{3.025822in}{2.423665in}}{\pgfqpoint{3.033636in}{2.431478in}}%
\pgfpathcurveto{\pgfqpoint{3.041449in}{2.439292in}}{\pgfqpoint{3.045840in}{2.449891in}}{\pgfqpoint{3.045840in}{2.460941in}}%
\pgfpathcurveto{\pgfqpoint{3.045840in}{2.471991in}}{\pgfqpoint{3.041449in}{2.482590in}}{\pgfqpoint{3.033636in}{2.490404in}}%
\pgfpathcurveto{\pgfqpoint{3.025822in}{2.498217in}}{\pgfqpoint{3.015223in}{2.502608in}}{\pgfqpoint{3.004173in}{2.502608in}}%
\pgfpathcurveto{\pgfqpoint{2.993123in}{2.502608in}}{\pgfqpoint{2.982524in}{2.498217in}}{\pgfqpoint{2.974710in}{2.490404in}}%
\pgfpathcurveto{\pgfqpoint{2.966896in}{2.482590in}}{\pgfqpoint{2.962506in}{2.471991in}}{\pgfqpoint{2.962506in}{2.460941in}}%
\pgfpathcurveto{\pgfqpoint{2.962506in}{2.449891in}}{\pgfqpoint{2.966896in}{2.439292in}}{\pgfqpoint{2.974710in}{2.431478in}}%
\pgfpathcurveto{\pgfqpoint{2.982524in}{2.423665in}}{\pgfqpoint{2.993123in}{2.419274in}}{\pgfqpoint{3.004173in}{2.419274in}}%
\pgfpathclose%
\pgfusepath{stroke,fill}%
\end{pgfscope}%
\begin{pgfscope}%
\pgfpathrectangle{\pgfqpoint{0.600000in}{0.600000in}}{\pgfqpoint{3.900000in}{3.900000in}}%
\pgfusepath{clip}%
\pgfsetbuttcap%
\pgfsetroundjoin%
\definecolor{currentfill}{rgb}{0.121569,0.466667,0.705882}%
\pgfsetfillcolor{currentfill}%
\pgfsetlinewidth{1.003750pt}%
\definecolor{currentstroke}{rgb}{0.121569,0.466667,0.705882}%
\pgfsetstrokecolor{currentstroke}%
\pgfsetdash{}{0pt}%
\pgfpathmoveto{\pgfqpoint{1.743811in}{1.973818in}}%
\pgfpathcurveto{\pgfqpoint{1.754861in}{1.973818in}}{\pgfqpoint{1.765460in}{1.978208in}}{\pgfqpoint{1.773273in}{1.986022in}}%
\pgfpathcurveto{\pgfqpoint{1.781087in}{1.993836in}}{\pgfqpoint{1.785477in}{2.004435in}}{\pgfqpoint{1.785477in}{2.015485in}}%
\pgfpathcurveto{\pgfqpoint{1.785477in}{2.026535in}}{\pgfqpoint{1.781087in}{2.037134in}}{\pgfqpoint{1.773273in}{2.044948in}}%
\pgfpathcurveto{\pgfqpoint{1.765460in}{2.052761in}}{\pgfqpoint{1.754861in}{2.057152in}}{\pgfqpoint{1.743811in}{2.057152in}}%
\pgfpathcurveto{\pgfqpoint{1.732761in}{2.057152in}}{\pgfqpoint{1.722161in}{2.052761in}}{\pgfqpoint{1.714348in}{2.044948in}}%
\pgfpathcurveto{\pgfqpoint{1.706534in}{2.037134in}}{\pgfqpoint{1.702144in}{2.026535in}}{\pgfqpoint{1.702144in}{2.015485in}}%
\pgfpathcurveto{\pgfqpoint{1.702144in}{2.004435in}}{\pgfqpoint{1.706534in}{1.993836in}}{\pgfqpoint{1.714348in}{1.986022in}}%
\pgfpathcurveto{\pgfqpoint{1.722161in}{1.978208in}}{\pgfqpoint{1.732761in}{1.973818in}}{\pgfqpoint{1.743811in}{1.973818in}}%
\pgfpathclose%
\pgfusepath{stroke,fill}%
\end{pgfscope}%
\begin{pgfscope}%
\pgfpathrectangle{\pgfqpoint{0.600000in}{0.600000in}}{\pgfqpoint{3.900000in}{3.900000in}}%
\pgfusepath{clip}%
\pgfsetbuttcap%
\pgfsetroundjoin%
\definecolor{currentfill}{rgb}{0.121569,0.466667,0.705882}%
\pgfsetfillcolor{currentfill}%
\pgfsetlinewidth{1.003750pt}%
\definecolor{currentstroke}{rgb}{0.121569,0.466667,0.705882}%
\pgfsetstrokecolor{currentstroke}%
\pgfsetdash{}{0pt}%
\pgfpathmoveto{\pgfqpoint{2.552735in}{2.114780in}}%
\pgfpathcurveto{\pgfqpoint{2.563785in}{2.114780in}}{\pgfqpoint{2.574384in}{2.119170in}}{\pgfqpoint{2.582198in}{2.126983in}}%
\pgfpathcurveto{\pgfqpoint{2.590012in}{2.134797in}}{\pgfqpoint{2.594402in}{2.145396in}}{\pgfqpoint{2.594402in}{2.156446in}}%
\pgfpathcurveto{\pgfqpoint{2.594402in}{2.167496in}}{\pgfqpoint{2.590012in}{2.178095in}}{\pgfqpoint{2.582198in}{2.185909in}}%
\pgfpathcurveto{\pgfqpoint{2.574384in}{2.193723in}}{\pgfqpoint{2.563785in}{2.198113in}}{\pgfqpoint{2.552735in}{2.198113in}}%
\pgfpathcurveto{\pgfqpoint{2.541685in}{2.198113in}}{\pgfqpoint{2.531086in}{2.193723in}}{\pgfqpoint{2.523272in}{2.185909in}}%
\pgfpathcurveto{\pgfqpoint{2.515459in}{2.178095in}}{\pgfqpoint{2.511068in}{2.167496in}}{\pgfqpoint{2.511068in}{2.156446in}}%
\pgfpathcurveto{\pgfqpoint{2.511068in}{2.145396in}}{\pgfqpoint{2.515459in}{2.134797in}}{\pgfqpoint{2.523272in}{2.126983in}}%
\pgfpathcurveto{\pgfqpoint{2.531086in}{2.119170in}}{\pgfqpoint{2.541685in}{2.114780in}}{\pgfqpoint{2.552735in}{2.114780in}}%
\pgfpathclose%
\pgfusepath{stroke,fill}%
\end{pgfscope}%
\begin{pgfscope}%
\pgfpathrectangle{\pgfqpoint{0.600000in}{0.600000in}}{\pgfqpoint{3.900000in}{3.900000in}}%
\pgfusepath{clip}%
\pgfsetbuttcap%
\pgfsetroundjoin%
\definecolor{currentfill}{rgb}{0.121569,0.466667,0.705882}%
\pgfsetfillcolor{currentfill}%
\pgfsetlinewidth{1.003750pt}%
\definecolor{currentstroke}{rgb}{0.121569,0.466667,0.705882}%
\pgfsetstrokecolor{currentstroke}%
\pgfsetdash{}{0pt}%
\pgfpathmoveto{\pgfqpoint{1.577778in}{2.247640in}}%
\pgfpathcurveto{\pgfqpoint{1.588828in}{2.247640in}}{\pgfqpoint{1.599427in}{2.252031in}}{\pgfqpoint{1.607241in}{2.259844in}}%
\pgfpathcurveto{\pgfqpoint{1.615055in}{2.267658in}}{\pgfqpoint{1.619445in}{2.278257in}}{\pgfqpoint{1.619445in}{2.289307in}}%
\pgfpathcurveto{\pgfqpoint{1.619445in}{2.300357in}}{\pgfqpoint{1.615055in}{2.310956in}}{\pgfqpoint{1.607241in}{2.318770in}}%
\pgfpathcurveto{\pgfqpoint{1.599427in}{2.326583in}}{\pgfqpoint{1.588828in}{2.330974in}}{\pgfqpoint{1.577778in}{2.330974in}}%
\pgfpathcurveto{\pgfqpoint{1.566728in}{2.330974in}}{\pgfqpoint{1.556129in}{2.326583in}}{\pgfqpoint{1.548315in}{2.318770in}}%
\pgfpathcurveto{\pgfqpoint{1.540502in}{2.310956in}}{\pgfqpoint{1.536111in}{2.300357in}}{\pgfqpoint{1.536111in}{2.289307in}}%
\pgfpathcurveto{\pgfqpoint{1.536111in}{2.278257in}}{\pgfqpoint{1.540502in}{2.267658in}}{\pgfqpoint{1.548315in}{2.259844in}}%
\pgfpathcurveto{\pgfqpoint{1.556129in}{2.252031in}}{\pgfqpoint{1.566728in}{2.247640in}}{\pgfqpoint{1.577778in}{2.247640in}}%
\pgfpathclose%
\pgfusepath{stroke,fill}%
\end{pgfscope}%
\begin{pgfscope}%
\pgfpathrectangle{\pgfqpoint{0.600000in}{0.600000in}}{\pgfqpoint{3.900000in}{3.900000in}}%
\pgfusepath{clip}%
\pgfsetbuttcap%
\pgfsetroundjoin%
\definecolor{currentfill}{rgb}{0.121569,0.466667,0.705882}%
\pgfsetfillcolor{currentfill}%
\pgfsetlinewidth{1.003750pt}%
\definecolor{currentstroke}{rgb}{0.121569,0.466667,0.705882}%
\pgfsetstrokecolor{currentstroke}%
\pgfsetdash{}{0pt}%
\pgfpathmoveto{\pgfqpoint{2.702710in}{2.159191in}}%
\pgfpathcurveto{\pgfqpoint{2.713760in}{2.159191in}}{\pgfqpoint{2.724359in}{2.163582in}}{\pgfqpoint{2.732173in}{2.171395in}}%
\pgfpathcurveto{\pgfqpoint{2.739987in}{2.179209in}}{\pgfqpoint{2.744377in}{2.189808in}}{\pgfqpoint{2.744377in}{2.200858in}}%
\pgfpathcurveto{\pgfqpoint{2.744377in}{2.211908in}}{\pgfqpoint{2.739987in}{2.222507in}}{\pgfqpoint{2.732173in}{2.230321in}}%
\pgfpathcurveto{\pgfqpoint{2.724359in}{2.238134in}}{\pgfqpoint{2.713760in}{2.242525in}}{\pgfqpoint{2.702710in}{2.242525in}}%
\pgfpathcurveto{\pgfqpoint{2.691660in}{2.242525in}}{\pgfqpoint{2.681061in}{2.238134in}}{\pgfqpoint{2.673247in}{2.230321in}}%
\pgfpathcurveto{\pgfqpoint{2.665434in}{2.222507in}}{\pgfqpoint{2.661043in}{2.211908in}}{\pgfqpoint{2.661043in}{2.200858in}}%
\pgfpathcurveto{\pgfqpoint{2.661043in}{2.189808in}}{\pgfqpoint{2.665434in}{2.179209in}}{\pgfqpoint{2.673247in}{2.171395in}}%
\pgfpathcurveto{\pgfqpoint{2.681061in}{2.163582in}}{\pgfqpoint{2.691660in}{2.159191in}}{\pgfqpoint{2.702710in}{2.159191in}}%
\pgfpathclose%
\pgfusepath{stroke,fill}%
\end{pgfscope}%
\begin{pgfscope}%
\pgfpathrectangle{\pgfqpoint{0.600000in}{0.600000in}}{\pgfqpoint{3.900000in}{3.900000in}}%
\pgfusepath{clip}%
\pgfsetbuttcap%
\pgfsetroundjoin%
\definecolor{currentfill}{rgb}{0.121569,0.466667,0.705882}%
\pgfsetfillcolor{currentfill}%
\pgfsetlinewidth{1.003750pt}%
\definecolor{currentstroke}{rgb}{0.121569,0.466667,0.705882}%
\pgfsetstrokecolor{currentstroke}%
\pgfsetdash{}{0pt}%
\pgfpathmoveto{\pgfqpoint{1.941093in}{2.365416in}}%
\pgfpathcurveto{\pgfqpoint{1.952144in}{2.365416in}}{\pgfqpoint{1.962743in}{2.369806in}}{\pgfqpoint{1.970556in}{2.377620in}}%
\pgfpathcurveto{\pgfqpoint{1.978370in}{2.385434in}}{\pgfqpoint{1.982760in}{2.396033in}}{\pgfqpoint{1.982760in}{2.407083in}}%
\pgfpathcurveto{\pgfqpoint{1.982760in}{2.418133in}}{\pgfqpoint{1.978370in}{2.428732in}}{\pgfqpoint{1.970556in}{2.436545in}}%
\pgfpathcurveto{\pgfqpoint{1.962743in}{2.444359in}}{\pgfqpoint{1.952144in}{2.448749in}}{\pgfqpoint{1.941093in}{2.448749in}}%
\pgfpathcurveto{\pgfqpoint{1.930043in}{2.448749in}}{\pgfqpoint{1.919444in}{2.444359in}}{\pgfqpoint{1.911631in}{2.436545in}}%
\pgfpathcurveto{\pgfqpoint{1.903817in}{2.428732in}}{\pgfqpoint{1.899427in}{2.418133in}}{\pgfqpoint{1.899427in}{2.407083in}}%
\pgfpathcurveto{\pgfqpoint{1.899427in}{2.396033in}}{\pgfqpoint{1.903817in}{2.385434in}}{\pgfqpoint{1.911631in}{2.377620in}}%
\pgfpathcurveto{\pgfqpoint{1.919444in}{2.369806in}}{\pgfqpoint{1.930043in}{2.365416in}}{\pgfqpoint{1.941093in}{2.365416in}}%
\pgfpathclose%
\pgfusepath{stroke,fill}%
\end{pgfscope}%
\begin{pgfscope}%
\pgfpathrectangle{\pgfqpoint{0.600000in}{0.600000in}}{\pgfqpoint{3.900000in}{3.900000in}}%
\pgfusepath{clip}%
\pgfsetbuttcap%
\pgfsetroundjoin%
\definecolor{currentfill}{rgb}{0.121569,0.466667,0.705882}%
\pgfsetfillcolor{currentfill}%
\pgfsetlinewidth{1.003750pt}%
\definecolor{currentstroke}{rgb}{0.121569,0.466667,0.705882}%
\pgfsetstrokecolor{currentstroke}%
\pgfsetdash{}{0pt}%
\pgfpathmoveto{\pgfqpoint{2.206998in}{3.404552in}}%
\pgfpathcurveto{\pgfqpoint{2.218048in}{3.404552in}}{\pgfqpoint{2.228647in}{3.408943in}}{\pgfqpoint{2.236460in}{3.416756in}}%
\pgfpathcurveto{\pgfqpoint{2.244274in}{3.424570in}}{\pgfqpoint{2.248664in}{3.435169in}}{\pgfqpoint{2.248664in}{3.446219in}}%
\pgfpathcurveto{\pgfqpoint{2.248664in}{3.457269in}}{\pgfqpoint{2.244274in}{3.467868in}}{\pgfqpoint{2.236460in}{3.475682in}}%
\pgfpathcurveto{\pgfqpoint{2.228647in}{3.483495in}}{\pgfqpoint{2.218048in}{3.487886in}}{\pgfqpoint{2.206998in}{3.487886in}}%
\pgfpathcurveto{\pgfqpoint{2.195947in}{3.487886in}}{\pgfqpoint{2.185348in}{3.483495in}}{\pgfqpoint{2.177535in}{3.475682in}}%
\pgfpathcurveto{\pgfqpoint{2.169721in}{3.467868in}}{\pgfqpoint{2.165331in}{3.457269in}}{\pgfqpoint{2.165331in}{3.446219in}}%
\pgfpathcurveto{\pgfqpoint{2.165331in}{3.435169in}}{\pgfqpoint{2.169721in}{3.424570in}}{\pgfqpoint{2.177535in}{3.416756in}}%
\pgfpathcurveto{\pgfqpoint{2.185348in}{3.408943in}}{\pgfqpoint{2.195947in}{3.404552in}}{\pgfqpoint{2.206998in}{3.404552in}}%
\pgfpathclose%
\pgfusepath{stroke,fill}%
\end{pgfscope}%
\begin{pgfscope}%
\pgfpathrectangle{\pgfqpoint{0.600000in}{0.600000in}}{\pgfqpoint{3.900000in}{3.900000in}}%
\pgfusepath{clip}%
\pgfsetbuttcap%
\pgfsetroundjoin%
\definecolor{currentfill}{rgb}{0.121569,0.466667,0.705882}%
\pgfsetfillcolor{currentfill}%
\pgfsetlinewidth{1.003750pt}%
\definecolor{currentstroke}{rgb}{0.121569,0.466667,0.705882}%
\pgfsetstrokecolor{currentstroke}%
\pgfsetdash{}{0pt}%
\pgfpathmoveto{\pgfqpoint{2.802644in}{2.705319in}}%
\pgfpathcurveto{\pgfqpoint{2.813694in}{2.705319in}}{\pgfqpoint{2.824293in}{2.709709in}}{\pgfqpoint{2.832107in}{2.717523in}}%
\pgfpathcurveto{\pgfqpoint{2.839920in}{2.725336in}}{\pgfqpoint{2.844311in}{2.735935in}}{\pgfqpoint{2.844311in}{2.746985in}}%
\pgfpathcurveto{\pgfqpoint{2.844311in}{2.758035in}}{\pgfqpoint{2.839920in}{2.768634in}}{\pgfqpoint{2.832107in}{2.776448in}}%
\pgfpathcurveto{\pgfqpoint{2.824293in}{2.784262in}}{\pgfqpoint{2.813694in}{2.788652in}}{\pgfqpoint{2.802644in}{2.788652in}}%
\pgfpathcurveto{\pgfqpoint{2.791594in}{2.788652in}}{\pgfqpoint{2.780995in}{2.784262in}}{\pgfqpoint{2.773181in}{2.776448in}}%
\pgfpathcurveto{\pgfqpoint{2.765367in}{2.768634in}}{\pgfqpoint{2.760977in}{2.758035in}}{\pgfqpoint{2.760977in}{2.746985in}}%
\pgfpathcurveto{\pgfqpoint{2.760977in}{2.735935in}}{\pgfqpoint{2.765367in}{2.725336in}}{\pgfqpoint{2.773181in}{2.717523in}}%
\pgfpathcurveto{\pgfqpoint{2.780995in}{2.709709in}}{\pgfqpoint{2.791594in}{2.705319in}}{\pgfqpoint{2.802644in}{2.705319in}}%
\pgfpathclose%
\pgfusepath{stroke,fill}%
\end{pgfscope}%
\begin{pgfscope}%
\pgfpathrectangle{\pgfqpoint{0.600000in}{0.600000in}}{\pgfqpoint{3.900000in}{3.900000in}}%
\pgfusepath{clip}%
\pgfsetbuttcap%
\pgfsetroundjoin%
\definecolor{currentfill}{rgb}{0.121569,0.466667,0.705882}%
\pgfsetfillcolor{currentfill}%
\pgfsetlinewidth{1.003750pt}%
\definecolor{currentstroke}{rgb}{0.121569,0.466667,0.705882}%
\pgfsetstrokecolor{currentstroke}%
\pgfsetdash{}{0pt}%
\pgfpathmoveto{\pgfqpoint{2.329074in}{2.474279in}}%
\pgfpathcurveto{\pgfqpoint{2.340124in}{2.474279in}}{\pgfqpoint{2.350723in}{2.478669in}}{\pgfqpoint{2.358537in}{2.486482in}}%
\pgfpathcurveto{\pgfqpoint{2.366351in}{2.494296in}}{\pgfqpoint{2.370741in}{2.504895in}}{\pgfqpoint{2.370741in}{2.515945in}}%
\pgfpathcurveto{\pgfqpoint{2.370741in}{2.526995in}}{\pgfqpoint{2.366351in}{2.537594in}}{\pgfqpoint{2.358537in}{2.545408in}}%
\pgfpathcurveto{\pgfqpoint{2.350723in}{2.553222in}}{\pgfqpoint{2.340124in}{2.557612in}}{\pgfqpoint{2.329074in}{2.557612in}}%
\pgfpathcurveto{\pgfqpoint{2.318024in}{2.557612in}}{\pgfqpoint{2.307425in}{2.553222in}}{\pgfqpoint{2.299611in}{2.545408in}}%
\pgfpathcurveto{\pgfqpoint{2.291798in}{2.537594in}}{\pgfqpoint{2.287408in}{2.526995in}}{\pgfqpoint{2.287408in}{2.515945in}}%
\pgfpathcurveto{\pgfqpoint{2.287408in}{2.504895in}}{\pgfqpoint{2.291798in}{2.494296in}}{\pgfqpoint{2.299611in}{2.486482in}}%
\pgfpathcurveto{\pgfqpoint{2.307425in}{2.478669in}}{\pgfqpoint{2.318024in}{2.474279in}}{\pgfqpoint{2.329074in}{2.474279in}}%
\pgfpathclose%
\pgfusepath{stroke,fill}%
\end{pgfscope}%
\begin{pgfscope}%
\pgfpathrectangle{\pgfqpoint{0.600000in}{0.600000in}}{\pgfqpoint{3.900000in}{3.900000in}}%
\pgfusepath{clip}%
\pgfsetbuttcap%
\pgfsetroundjoin%
\definecolor{currentfill}{rgb}{0.121569,0.466667,0.705882}%
\pgfsetfillcolor{currentfill}%
\pgfsetlinewidth{1.003750pt}%
\definecolor{currentstroke}{rgb}{0.121569,0.466667,0.705882}%
\pgfsetstrokecolor{currentstroke}%
\pgfsetdash{}{0pt}%
\pgfpathmoveto{\pgfqpoint{2.244834in}{2.856864in}}%
\pgfpathcurveto{\pgfqpoint{2.255884in}{2.856864in}}{\pgfqpoint{2.266483in}{2.861255in}}{\pgfqpoint{2.274297in}{2.869068in}}%
\pgfpathcurveto{\pgfqpoint{2.282110in}{2.876882in}}{\pgfqpoint{2.286500in}{2.887481in}}{\pgfqpoint{2.286500in}{2.898531in}}%
\pgfpathcurveto{\pgfqpoint{2.286500in}{2.909581in}}{\pgfqpoint{2.282110in}{2.920180in}}{\pgfqpoint{2.274297in}{2.927994in}}%
\pgfpathcurveto{\pgfqpoint{2.266483in}{2.935807in}}{\pgfqpoint{2.255884in}{2.940198in}}{\pgfqpoint{2.244834in}{2.940198in}}%
\pgfpathcurveto{\pgfqpoint{2.233784in}{2.940198in}}{\pgfqpoint{2.223185in}{2.935807in}}{\pgfqpoint{2.215371in}{2.927994in}}%
\pgfpathcurveto{\pgfqpoint{2.207557in}{2.920180in}}{\pgfqpoint{2.203167in}{2.909581in}}{\pgfqpoint{2.203167in}{2.898531in}}%
\pgfpathcurveto{\pgfqpoint{2.203167in}{2.887481in}}{\pgfqpoint{2.207557in}{2.876882in}}{\pgfqpoint{2.215371in}{2.869068in}}%
\pgfpathcurveto{\pgfqpoint{2.223185in}{2.861255in}}{\pgfqpoint{2.233784in}{2.856864in}}{\pgfqpoint{2.244834in}{2.856864in}}%
\pgfpathclose%
\pgfusepath{stroke,fill}%
\end{pgfscope}%
\begin{pgfscope}%
\pgfpathrectangle{\pgfqpoint{0.600000in}{0.600000in}}{\pgfqpoint{3.900000in}{3.900000in}}%
\pgfusepath{clip}%
\pgfsetbuttcap%
\pgfsetroundjoin%
\definecolor{currentfill}{rgb}{0.121569,0.466667,0.705882}%
\pgfsetfillcolor{currentfill}%
\pgfsetlinewidth{1.003750pt}%
\definecolor{currentstroke}{rgb}{0.121569,0.466667,0.705882}%
\pgfsetstrokecolor{currentstroke}%
\pgfsetdash{}{0pt}%
\pgfpathmoveto{\pgfqpoint{2.409908in}{2.610766in}}%
\pgfpathcurveto{\pgfqpoint{2.420958in}{2.610766in}}{\pgfqpoint{2.431558in}{2.615156in}}{\pgfqpoint{2.439371in}{2.622970in}}%
\pgfpathcurveto{\pgfqpoint{2.447185in}{2.630783in}}{\pgfqpoint{2.451575in}{2.641382in}}{\pgfqpoint{2.451575in}{2.652433in}}%
\pgfpathcurveto{\pgfqpoint{2.451575in}{2.663483in}}{\pgfqpoint{2.447185in}{2.674082in}}{\pgfqpoint{2.439371in}{2.681895in}}%
\pgfpathcurveto{\pgfqpoint{2.431558in}{2.689709in}}{\pgfqpoint{2.420958in}{2.694099in}}{\pgfqpoint{2.409908in}{2.694099in}}%
\pgfpathcurveto{\pgfqpoint{2.398858in}{2.694099in}}{\pgfqpoint{2.388259in}{2.689709in}}{\pgfqpoint{2.380446in}{2.681895in}}%
\pgfpathcurveto{\pgfqpoint{2.372632in}{2.674082in}}{\pgfqpoint{2.368242in}{2.663483in}}{\pgfqpoint{2.368242in}{2.652433in}}%
\pgfpathcurveto{\pgfqpoint{2.368242in}{2.641382in}}{\pgfqpoint{2.372632in}{2.630783in}}{\pgfqpoint{2.380446in}{2.622970in}}%
\pgfpathcurveto{\pgfqpoint{2.388259in}{2.615156in}}{\pgfqpoint{2.398858in}{2.610766in}}{\pgfqpoint{2.409908in}{2.610766in}}%
\pgfpathclose%
\pgfusepath{stroke,fill}%
\end{pgfscope}%
\begin{pgfscope}%
\pgfpathrectangle{\pgfqpoint{0.600000in}{0.600000in}}{\pgfqpoint{3.900000in}{3.900000in}}%
\pgfusepath{clip}%
\pgfsetbuttcap%
\pgfsetroundjoin%
\definecolor{currentfill}{rgb}{0.121569,0.466667,0.705882}%
\pgfsetfillcolor{currentfill}%
\pgfsetlinewidth{1.003750pt}%
\definecolor{currentstroke}{rgb}{0.121569,0.466667,0.705882}%
\pgfsetstrokecolor{currentstroke}%
\pgfsetdash{}{0pt}%
\pgfpathmoveto{\pgfqpoint{2.634243in}{2.421003in}}%
\pgfpathcurveto{\pgfqpoint{2.645293in}{2.421003in}}{\pgfqpoint{2.655892in}{2.425393in}}{\pgfqpoint{2.663706in}{2.433207in}}%
\pgfpathcurveto{\pgfqpoint{2.671519in}{2.441021in}}{\pgfqpoint{2.675910in}{2.451620in}}{\pgfqpoint{2.675910in}{2.462670in}}%
\pgfpathcurveto{\pgfqpoint{2.675910in}{2.473720in}}{\pgfqpoint{2.671519in}{2.484319in}}{\pgfqpoint{2.663706in}{2.492133in}}%
\pgfpathcurveto{\pgfqpoint{2.655892in}{2.499946in}}{\pgfqpoint{2.645293in}{2.504337in}}{\pgfqpoint{2.634243in}{2.504337in}}%
\pgfpathcurveto{\pgfqpoint{2.623193in}{2.504337in}}{\pgfqpoint{2.612594in}{2.499946in}}{\pgfqpoint{2.604780in}{2.492133in}}%
\pgfpathcurveto{\pgfqpoint{2.596967in}{2.484319in}}{\pgfqpoint{2.592576in}{2.473720in}}{\pgfqpoint{2.592576in}{2.462670in}}%
\pgfpathcurveto{\pgfqpoint{2.592576in}{2.451620in}}{\pgfqpoint{2.596967in}{2.441021in}}{\pgfqpoint{2.604780in}{2.433207in}}%
\pgfpathcurveto{\pgfqpoint{2.612594in}{2.425393in}}{\pgfqpoint{2.623193in}{2.421003in}}{\pgfqpoint{2.634243in}{2.421003in}}%
\pgfpathclose%
\pgfusepath{stroke,fill}%
\end{pgfscope}%
\begin{pgfscope}%
\pgfpathrectangle{\pgfqpoint{0.600000in}{0.600000in}}{\pgfqpoint{3.900000in}{3.900000in}}%
\pgfusepath{clip}%
\pgfsetbuttcap%
\pgfsetroundjoin%
\definecolor{currentfill}{rgb}{0.121569,0.466667,0.705882}%
\pgfsetfillcolor{currentfill}%
\pgfsetlinewidth{1.003750pt}%
\definecolor{currentstroke}{rgb}{0.121569,0.466667,0.705882}%
\pgfsetstrokecolor{currentstroke}%
\pgfsetdash{}{0pt}%
\pgfpathmoveto{\pgfqpoint{2.088997in}{1.870956in}}%
\pgfpathcurveto{\pgfqpoint{2.100047in}{1.870956in}}{\pgfqpoint{2.110646in}{1.875347in}}{\pgfqpoint{2.118460in}{1.883160in}}%
\pgfpathcurveto{\pgfqpoint{2.126273in}{1.890974in}}{\pgfqpoint{2.130664in}{1.901573in}}{\pgfqpoint{2.130664in}{1.912623in}}%
\pgfpathcurveto{\pgfqpoint{2.130664in}{1.923673in}}{\pgfqpoint{2.126273in}{1.934272in}}{\pgfqpoint{2.118460in}{1.942086in}}%
\pgfpathcurveto{\pgfqpoint{2.110646in}{1.949900in}}{\pgfqpoint{2.100047in}{1.954290in}}{\pgfqpoint{2.088997in}{1.954290in}}%
\pgfpathcurveto{\pgfqpoint{2.077947in}{1.954290in}}{\pgfqpoint{2.067348in}{1.949900in}}{\pgfqpoint{2.059534in}{1.942086in}}%
\pgfpathcurveto{\pgfqpoint{2.051721in}{1.934272in}}{\pgfqpoint{2.047330in}{1.923673in}}{\pgfqpoint{2.047330in}{1.912623in}}%
\pgfpathcurveto{\pgfqpoint{2.047330in}{1.901573in}}{\pgfqpoint{2.051721in}{1.890974in}}{\pgfqpoint{2.059534in}{1.883160in}}%
\pgfpathcurveto{\pgfqpoint{2.067348in}{1.875347in}}{\pgfqpoint{2.077947in}{1.870956in}}{\pgfqpoint{2.088997in}{1.870956in}}%
\pgfpathclose%
\pgfusepath{stroke,fill}%
\end{pgfscope}%
\begin{pgfscope}%
\pgfpathrectangle{\pgfqpoint{0.600000in}{0.600000in}}{\pgfqpoint{3.900000in}{3.900000in}}%
\pgfusepath{clip}%
\pgfsetbuttcap%
\pgfsetroundjoin%
\definecolor{currentfill}{rgb}{0.121569,0.466667,0.705882}%
\pgfsetfillcolor{currentfill}%
\pgfsetlinewidth{1.003750pt}%
\definecolor{currentstroke}{rgb}{0.121569,0.466667,0.705882}%
\pgfsetstrokecolor{currentstroke}%
\pgfsetdash{}{0pt}%
\pgfpathmoveto{\pgfqpoint{2.632426in}{2.958992in}}%
\pgfpathcurveto{\pgfqpoint{2.643476in}{2.958992in}}{\pgfqpoint{2.654075in}{2.963382in}}{\pgfqpoint{2.661889in}{2.971196in}}%
\pgfpathcurveto{\pgfqpoint{2.669702in}{2.979009in}}{\pgfqpoint{2.674093in}{2.989608in}}{\pgfqpoint{2.674093in}{3.000659in}}%
\pgfpathcurveto{\pgfqpoint{2.674093in}{3.011709in}}{\pgfqpoint{2.669702in}{3.022308in}}{\pgfqpoint{2.661889in}{3.030121in}}%
\pgfpathcurveto{\pgfqpoint{2.654075in}{3.037935in}}{\pgfqpoint{2.643476in}{3.042325in}}{\pgfqpoint{2.632426in}{3.042325in}}%
\pgfpathcurveto{\pgfqpoint{2.621376in}{3.042325in}}{\pgfqpoint{2.610777in}{3.037935in}}{\pgfqpoint{2.602963in}{3.030121in}}%
\pgfpathcurveto{\pgfqpoint{2.595150in}{3.022308in}}{\pgfqpoint{2.590759in}{3.011709in}}{\pgfqpoint{2.590759in}{3.000659in}}%
\pgfpathcurveto{\pgfqpoint{2.590759in}{2.989608in}}{\pgfqpoint{2.595150in}{2.979009in}}{\pgfqpoint{2.602963in}{2.971196in}}%
\pgfpathcurveto{\pgfqpoint{2.610777in}{2.963382in}}{\pgfqpoint{2.621376in}{2.958992in}}{\pgfqpoint{2.632426in}{2.958992in}}%
\pgfpathclose%
\pgfusepath{stroke,fill}%
\end{pgfscope}%
\begin{pgfscope}%
\pgfpathrectangle{\pgfqpoint{0.600000in}{0.600000in}}{\pgfqpoint{3.900000in}{3.900000in}}%
\pgfusepath{clip}%
\pgfsetbuttcap%
\pgfsetroundjoin%
\definecolor{currentfill}{rgb}{0.121569,0.466667,0.705882}%
\pgfsetfillcolor{currentfill}%
\pgfsetlinewidth{1.003750pt}%
\definecolor{currentstroke}{rgb}{0.121569,0.466667,0.705882}%
\pgfsetstrokecolor{currentstroke}%
\pgfsetdash{}{0pt}%
\pgfpathmoveto{\pgfqpoint{2.376777in}{2.516249in}}%
\pgfpathcurveto{\pgfqpoint{2.387827in}{2.516249in}}{\pgfqpoint{2.398426in}{2.520640in}}{\pgfqpoint{2.406239in}{2.528453in}}%
\pgfpathcurveto{\pgfqpoint{2.414053in}{2.536267in}}{\pgfqpoint{2.418443in}{2.546866in}}{\pgfqpoint{2.418443in}{2.557916in}}%
\pgfpathcurveto{\pgfqpoint{2.418443in}{2.568966in}}{\pgfqpoint{2.414053in}{2.579565in}}{\pgfqpoint{2.406239in}{2.587379in}}%
\pgfpathcurveto{\pgfqpoint{2.398426in}{2.595192in}}{\pgfqpoint{2.387827in}{2.599583in}}{\pgfqpoint{2.376777in}{2.599583in}}%
\pgfpathcurveto{\pgfqpoint{2.365726in}{2.599583in}}{\pgfqpoint{2.355127in}{2.595192in}}{\pgfqpoint{2.347314in}{2.587379in}}%
\pgfpathcurveto{\pgfqpoint{2.339500in}{2.579565in}}{\pgfqpoint{2.335110in}{2.568966in}}{\pgfqpoint{2.335110in}{2.557916in}}%
\pgfpathcurveto{\pgfqpoint{2.335110in}{2.546866in}}{\pgfqpoint{2.339500in}{2.536267in}}{\pgfqpoint{2.347314in}{2.528453in}}%
\pgfpathcurveto{\pgfqpoint{2.355127in}{2.520640in}}{\pgfqpoint{2.365726in}{2.516249in}}{\pgfqpoint{2.376777in}{2.516249in}}%
\pgfpathclose%
\pgfusepath{stroke,fill}%
\end{pgfscope}%
\begin{pgfscope}%
\pgfpathrectangle{\pgfqpoint{0.600000in}{0.600000in}}{\pgfqpoint{3.900000in}{3.900000in}}%
\pgfusepath{clip}%
\pgfsetbuttcap%
\pgfsetroundjoin%
\definecolor{currentfill}{rgb}{0.121569,0.466667,0.705882}%
\pgfsetfillcolor{currentfill}%
\pgfsetlinewidth{1.003750pt}%
\definecolor{currentstroke}{rgb}{0.121569,0.466667,0.705882}%
\pgfsetstrokecolor{currentstroke}%
\pgfsetdash{}{0pt}%
\pgfpathmoveto{\pgfqpoint{1.988460in}{3.214800in}}%
\pgfpathcurveto{\pgfqpoint{1.999510in}{3.214800in}}{\pgfqpoint{2.010109in}{3.219190in}}{\pgfqpoint{2.017923in}{3.227004in}}%
\pgfpathcurveto{\pgfqpoint{2.025736in}{3.234817in}}{\pgfqpoint{2.030127in}{3.245416in}}{\pgfqpoint{2.030127in}{3.256466in}}%
\pgfpathcurveto{\pgfqpoint{2.030127in}{3.267516in}}{\pgfqpoint{2.025736in}{3.278116in}}{\pgfqpoint{2.017923in}{3.285929in}}%
\pgfpathcurveto{\pgfqpoint{2.010109in}{3.293743in}}{\pgfqpoint{1.999510in}{3.298133in}}{\pgfqpoint{1.988460in}{3.298133in}}%
\pgfpathcurveto{\pgfqpoint{1.977410in}{3.298133in}}{\pgfqpoint{1.966811in}{3.293743in}}{\pgfqpoint{1.958997in}{3.285929in}}%
\pgfpathcurveto{\pgfqpoint{1.951184in}{3.278116in}}{\pgfqpoint{1.946793in}{3.267516in}}{\pgfqpoint{1.946793in}{3.256466in}}%
\pgfpathcurveto{\pgfqpoint{1.946793in}{3.245416in}}{\pgfqpoint{1.951184in}{3.234817in}}{\pgfqpoint{1.958997in}{3.227004in}}%
\pgfpathcurveto{\pgfqpoint{1.966811in}{3.219190in}}{\pgfqpoint{1.977410in}{3.214800in}}{\pgfqpoint{1.988460in}{3.214800in}}%
\pgfpathclose%
\pgfusepath{stroke,fill}%
\end{pgfscope}%
\begin{pgfscope}%
\pgfpathrectangle{\pgfqpoint{0.600000in}{0.600000in}}{\pgfqpoint{3.900000in}{3.900000in}}%
\pgfusepath{clip}%
\pgfsetbuttcap%
\pgfsetroundjoin%
\definecolor{currentfill}{rgb}{0.121569,0.466667,0.705882}%
\pgfsetfillcolor{currentfill}%
\pgfsetlinewidth{1.003750pt}%
\definecolor{currentstroke}{rgb}{0.121569,0.466667,0.705882}%
\pgfsetstrokecolor{currentstroke}%
\pgfsetdash{}{0pt}%
\pgfpathmoveto{\pgfqpoint{2.628410in}{1.722489in}}%
\pgfpathcurveto{\pgfqpoint{2.639460in}{1.722489in}}{\pgfqpoint{2.650059in}{1.726879in}}{\pgfqpoint{2.657873in}{1.734693in}}%
\pgfpathcurveto{\pgfqpoint{2.665687in}{1.742506in}}{\pgfqpoint{2.670077in}{1.753105in}}{\pgfqpoint{2.670077in}{1.764155in}}%
\pgfpathcurveto{\pgfqpoint{2.670077in}{1.775205in}}{\pgfqpoint{2.665687in}{1.785804in}}{\pgfqpoint{2.657873in}{1.793618in}}%
\pgfpathcurveto{\pgfqpoint{2.650059in}{1.801432in}}{\pgfqpoint{2.639460in}{1.805822in}}{\pgfqpoint{2.628410in}{1.805822in}}%
\pgfpathcurveto{\pgfqpoint{2.617360in}{1.805822in}}{\pgfqpoint{2.606761in}{1.801432in}}{\pgfqpoint{2.598947in}{1.793618in}}%
\pgfpathcurveto{\pgfqpoint{2.591134in}{1.785804in}}{\pgfqpoint{2.586744in}{1.775205in}}{\pgfqpoint{2.586744in}{1.764155in}}%
\pgfpathcurveto{\pgfqpoint{2.586744in}{1.753105in}}{\pgfqpoint{2.591134in}{1.742506in}}{\pgfqpoint{2.598947in}{1.734693in}}%
\pgfpathcurveto{\pgfqpoint{2.606761in}{1.726879in}}{\pgfqpoint{2.617360in}{1.722489in}}{\pgfqpoint{2.628410in}{1.722489in}}%
\pgfpathclose%
\pgfusepath{stroke,fill}%
\end{pgfscope}%
\begin{pgfscope}%
\pgfpathrectangle{\pgfqpoint{0.600000in}{0.600000in}}{\pgfqpoint{3.900000in}{3.900000in}}%
\pgfusepath{clip}%
\pgfsetbuttcap%
\pgfsetroundjoin%
\definecolor{currentfill}{rgb}{0.121569,0.466667,0.705882}%
\pgfsetfillcolor{currentfill}%
\pgfsetlinewidth{1.003750pt}%
\definecolor{currentstroke}{rgb}{0.121569,0.466667,0.705882}%
\pgfsetstrokecolor{currentstroke}%
\pgfsetdash{}{0pt}%
\pgfpathmoveto{\pgfqpoint{1.791413in}{2.175317in}}%
\pgfpathcurveto{\pgfqpoint{1.802463in}{2.175317in}}{\pgfqpoint{1.813062in}{2.179707in}}{\pgfqpoint{1.820876in}{2.187521in}}%
\pgfpathcurveto{\pgfqpoint{1.828690in}{2.195334in}}{\pgfqpoint{1.833080in}{2.205933in}}{\pgfqpoint{1.833080in}{2.216984in}}%
\pgfpathcurveto{\pgfqpoint{1.833080in}{2.228034in}}{\pgfqpoint{1.828690in}{2.238633in}}{\pgfqpoint{1.820876in}{2.246446in}}%
\pgfpathcurveto{\pgfqpoint{1.813062in}{2.254260in}}{\pgfqpoint{1.802463in}{2.258650in}}{\pgfqpoint{1.791413in}{2.258650in}}%
\pgfpathcurveto{\pgfqpoint{1.780363in}{2.258650in}}{\pgfqpoint{1.769764in}{2.254260in}}{\pgfqpoint{1.761950in}{2.246446in}}%
\pgfpathcurveto{\pgfqpoint{1.754137in}{2.238633in}}{\pgfqpoint{1.749746in}{2.228034in}}{\pgfqpoint{1.749746in}{2.216984in}}%
\pgfpathcurveto{\pgfqpoint{1.749746in}{2.205933in}}{\pgfqpoint{1.754137in}{2.195334in}}{\pgfqpoint{1.761950in}{2.187521in}}%
\pgfpathcurveto{\pgfqpoint{1.769764in}{2.179707in}}{\pgfqpoint{1.780363in}{2.175317in}}{\pgfqpoint{1.791413in}{2.175317in}}%
\pgfpathclose%
\pgfusepath{stroke,fill}%
\end{pgfscope}%
\begin{pgfscope}%
\pgfpathrectangle{\pgfqpoint{0.600000in}{0.600000in}}{\pgfqpoint{3.900000in}{3.900000in}}%
\pgfusepath{clip}%
\pgfsetbuttcap%
\pgfsetroundjoin%
\definecolor{currentfill}{rgb}{0.121569,0.466667,0.705882}%
\pgfsetfillcolor{currentfill}%
\pgfsetlinewidth{1.003750pt}%
\definecolor{currentstroke}{rgb}{0.121569,0.466667,0.705882}%
\pgfsetstrokecolor{currentstroke}%
\pgfsetdash{}{0pt}%
\pgfpathmoveto{\pgfqpoint{2.976249in}{2.377625in}}%
\pgfpathcurveto{\pgfqpoint{2.987299in}{2.377625in}}{\pgfqpoint{2.997898in}{2.382016in}}{\pgfqpoint{3.005711in}{2.389829in}}%
\pgfpathcurveto{\pgfqpoint{3.013525in}{2.397643in}}{\pgfqpoint{3.017915in}{2.408242in}}{\pgfqpoint{3.017915in}{2.419292in}}%
\pgfpathcurveto{\pgfqpoint{3.017915in}{2.430342in}}{\pgfqpoint{3.013525in}{2.440941in}}{\pgfqpoint{3.005711in}{2.448755in}}%
\pgfpathcurveto{\pgfqpoint{2.997898in}{2.456568in}}{\pgfqpoint{2.987299in}{2.460959in}}{\pgfqpoint{2.976249in}{2.460959in}}%
\pgfpathcurveto{\pgfqpoint{2.965198in}{2.460959in}}{\pgfqpoint{2.954599in}{2.456568in}}{\pgfqpoint{2.946786in}{2.448755in}}%
\pgfpathcurveto{\pgfqpoint{2.938972in}{2.440941in}}{\pgfqpoint{2.934582in}{2.430342in}}{\pgfqpoint{2.934582in}{2.419292in}}%
\pgfpathcurveto{\pgfqpoint{2.934582in}{2.408242in}}{\pgfqpoint{2.938972in}{2.397643in}}{\pgfqpoint{2.946786in}{2.389829in}}%
\pgfpathcurveto{\pgfqpoint{2.954599in}{2.382016in}}{\pgfqpoint{2.965198in}{2.377625in}}{\pgfqpoint{2.976249in}{2.377625in}}%
\pgfpathclose%
\pgfusepath{stroke,fill}%
\end{pgfscope}%
\begin{pgfscope}%
\pgfpathrectangle{\pgfqpoint{0.600000in}{0.600000in}}{\pgfqpoint{3.900000in}{3.900000in}}%
\pgfusepath{clip}%
\pgfsetbuttcap%
\pgfsetroundjoin%
\definecolor{currentfill}{rgb}{0.121569,0.466667,0.705882}%
\pgfsetfillcolor{currentfill}%
\pgfsetlinewidth{1.003750pt}%
\definecolor{currentstroke}{rgb}{0.121569,0.466667,0.705882}%
\pgfsetstrokecolor{currentstroke}%
\pgfsetdash{}{0pt}%
\pgfpathmoveto{\pgfqpoint{1.831969in}{2.836848in}}%
\pgfpathcurveto{\pgfqpoint{1.843019in}{2.836848in}}{\pgfqpoint{1.853618in}{2.841238in}}{\pgfqpoint{1.861431in}{2.849052in}}%
\pgfpathcurveto{\pgfqpoint{1.869245in}{2.856865in}}{\pgfqpoint{1.873635in}{2.867464in}}{\pgfqpoint{1.873635in}{2.878515in}}%
\pgfpathcurveto{\pgfqpoint{1.873635in}{2.889565in}}{\pgfqpoint{1.869245in}{2.900164in}}{\pgfqpoint{1.861431in}{2.907977in}}%
\pgfpathcurveto{\pgfqpoint{1.853618in}{2.915791in}}{\pgfqpoint{1.843019in}{2.920181in}}{\pgfqpoint{1.831969in}{2.920181in}}%
\pgfpathcurveto{\pgfqpoint{1.820918in}{2.920181in}}{\pgfqpoint{1.810319in}{2.915791in}}{\pgfqpoint{1.802506in}{2.907977in}}%
\pgfpathcurveto{\pgfqpoint{1.794692in}{2.900164in}}{\pgfqpoint{1.790302in}{2.889565in}}{\pgfqpoint{1.790302in}{2.878515in}}%
\pgfpathcurveto{\pgfqpoint{1.790302in}{2.867464in}}{\pgfqpoint{1.794692in}{2.856865in}}{\pgfqpoint{1.802506in}{2.849052in}}%
\pgfpathcurveto{\pgfqpoint{1.810319in}{2.841238in}}{\pgfqpoint{1.820918in}{2.836848in}}{\pgfqpoint{1.831969in}{2.836848in}}%
\pgfpathclose%
\pgfusepath{stroke,fill}%
\end{pgfscope}%
\begin{pgfscope}%
\pgfpathrectangle{\pgfqpoint{0.600000in}{0.600000in}}{\pgfqpoint{3.900000in}{3.900000in}}%
\pgfusepath{clip}%
\pgfsetbuttcap%
\pgfsetroundjoin%
\definecolor{currentfill}{rgb}{0.121569,0.466667,0.705882}%
\pgfsetfillcolor{currentfill}%
\pgfsetlinewidth{1.003750pt}%
\definecolor{currentstroke}{rgb}{0.121569,0.466667,0.705882}%
\pgfsetstrokecolor{currentstroke}%
\pgfsetdash{}{0pt}%
\pgfpathmoveto{\pgfqpoint{1.500358in}{2.732096in}}%
\pgfpathcurveto{\pgfqpoint{1.511408in}{2.732096in}}{\pgfqpoint{1.522007in}{2.736487in}}{\pgfqpoint{1.529821in}{2.744300in}}%
\pgfpathcurveto{\pgfqpoint{1.537635in}{2.752114in}}{\pgfqpoint{1.542025in}{2.762713in}}{\pgfqpoint{1.542025in}{2.773763in}}%
\pgfpathcurveto{\pgfqpoint{1.542025in}{2.784813in}}{\pgfqpoint{1.537635in}{2.795412in}}{\pgfqpoint{1.529821in}{2.803226in}}%
\pgfpathcurveto{\pgfqpoint{1.522007in}{2.811039in}}{\pgfqpoint{1.511408in}{2.815430in}}{\pgfqpoint{1.500358in}{2.815430in}}%
\pgfpathcurveto{\pgfqpoint{1.489308in}{2.815430in}}{\pgfqpoint{1.478709in}{2.811039in}}{\pgfqpoint{1.470895in}{2.803226in}}%
\pgfpathcurveto{\pgfqpoint{1.463082in}{2.795412in}}{\pgfqpoint{1.458691in}{2.784813in}}{\pgfqpoint{1.458691in}{2.773763in}}%
\pgfpathcurveto{\pgfqpoint{1.458691in}{2.762713in}}{\pgfqpoint{1.463082in}{2.752114in}}{\pgfqpoint{1.470895in}{2.744300in}}%
\pgfpathcurveto{\pgfqpoint{1.478709in}{2.736487in}}{\pgfqpoint{1.489308in}{2.732096in}}{\pgfqpoint{1.500358in}{2.732096in}}%
\pgfpathclose%
\pgfusepath{stroke,fill}%
\end{pgfscope}%
\begin{pgfscope}%
\pgfpathrectangle{\pgfqpoint{0.600000in}{0.600000in}}{\pgfqpoint{3.900000in}{3.900000in}}%
\pgfusepath{clip}%
\pgfsetbuttcap%
\pgfsetroundjoin%
\definecolor{currentfill}{rgb}{0.121569,0.466667,0.705882}%
\pgfsetfillcolor{currentfill}%
\pgfsetlinewidth{1.003750pt}%
\definecolor{currentstroke}{rgb}{0.121569,0.466667,0.705882}%
\pgfsetstrokecolor{currentstroke}%
\pgfsetdash{}{0pt}%
\pgfpathmoveto{\pgfqpoint{3.080512in}{3.065341in}}%
\pgfpathcurveto{\pgfqpoint{3.091562in}{3.065341in}}{\pgfqpoint{3.102161in}{3.069731in}}{\pgfqpoint{3.109975in}{3.077545in}}%
\pgfpathcurveto{\pgfqpoint{3.117789in}{3.085359in}}{\pgfqpoint{3.122179in}{3.095958in}}{\pgfqpoint{3.122179in}{3.107008in}}%
\pgfpathcurveto{\pgfqpoint{3.122179in}{3.118058in}}{\pgfqpoint{3.117789in}{3.128657in}}{\pgfqpoint{3.109975in}{3.136470in}}%
\pgfpathcurveto{\pgfqpoint{3.102161in}{3.144284in}}{\pgfqpoint{3.091562in}{3.148674in}}{\pgfqpoint{3.080512in}{3.148674in}}%
\pgfpathcurveto{\pgfqpoint{3.069462in}{3.148674in}}{\pgfqpoint{3.058863in}{3.144284in}}{\pgfqpoint{3.051049in}{3.136470in}}%
\pgfpathcurveto{\pgfqpoint{3.043236in}{3.128657in}}{\pgfqpoint{3.038845in}{3.118058in}}{\pgfqpoint{3.038845in}{3.107008in}}%
\pgfpathcurveto{\pgfqpoint{3.038845in}{3.095958in}}{\pgfqpoint{3.043236in}{3.085359in}}{\pgfqpoint{3.051049in}{3.077545in}}%
\pgfpathcurveto{\pgfqpoint{3.058863in}{3.069731in}}{\pgfqpoint{3.069462in}{3.065341in}}{\pgfqpoint{3.080512in}{3.065341in}}%
\pgfpathclose%
\pgfusepath{stroke,fill}%
\end{pgfscope}%
\begin{pgfscope}%
\pgfpathrectangle{\pgfqpoint{0.600000in}{0.600000in}}{\pgfqpoint{3.900000in}{3.900000in}}%
\pgfusepath{clip}%
\pgfsetbuttcap%
\pgfsetroundjoin%
\definecolor{currentfill}{rgb}{0.121569,0.466667,0.705882}%
\pgfsetfillcolor{currentfill}%
\pgfsetlinewidth{1.003750pt}%
\definecolor{currentstroke}{rgb}{0.121569,0.466667,0.705882}%
\pgfsetstrokecolor{currentstroke}%
\pgfsetdash{}{0pt}%
\pgfpathmoveto{\pgfqpoint{2.490592in}{2.783567in}}%
\pgfpathcurveto{\pgfqpoint{2.501642in}{2.783567in}}{\pgfqpoint{2.512241in}{2.787958in}}{\pgfqpoint{2.520055in}{2.795771in}}%
\pgfpathcurveto{\pgfqpoint{2.527869in}{2.803585in}}{\pgfqpoint{2.532259in}{2.814184in}}{\pgfqpoint{2.532259in}{2.825234in}}%
\pgfpathcurveto{\pgfqpoint{2.532259in}{2.836284in}}{\pgfqpoint{2.527869in}{2.846883in}}{\pgfqpoint{2.520055in}{2.854697in}}%
\pgfpathcurveto{\pgfqpoint{2.512241in}{2.862511in}}{\pgfqpoint{2.501642in}{2.866901in}}{\pgfqpoint{2.490592in}{2.866901in}}%
\pgfpathcurveto{\pgfqpoint{2.479542in}{2.866901in}}{\pgfqpoint{2.468943in}{2.862511in}}{\pgfqpoint{2.461129in}{2.854697in}}%
\pgfpathcurveto{\pgfqpoint{2.453316in}{2.846883in}}{\pgfqpoint{2.448926in}{2.836284in}}{\pgfqpoint{2.448926in}{2.825234in}}%
\pgfpathcurveto{\pgfqpoint{2.448926in}{2.814184in}}{\pgfqpoint{2.453316in}{2.803585in}}{\pgfqpoint{2.461129in}{2.795771in}}%
\pgfpathcurveto{\pgfqpoint{2.468943in}{2.787958in}}{\pgfqpoint{2.479542in}{2.783567in}}{\pgfqpoint{2.490592in}{2.783567in}}%
\pgfpathclose%
\pgfusepath{stroke,fill}%
\end{pgfscope}%
\begin{pgfscope}%
\pgfpathrectangle{\pgfqpoint{0.600000in}{0.600000in}}{\pgfqpoint{3.900000in}{3.900000in}}%
\pgfusepath{clip}%
\pgfsetbuttcap%
\pgfsetroundjoin%
\definecolor{currentfill}{rgb}{0.121569,0.466667,0.705882}%
\pgfsetfillcolor{currentfill}%
\pgfsetlinewidth{1.003750pt}%
\definecolor{currentstroke}{rgb}{0.121569,0.466667,0.705882}%
\pgfsetstrokecolor{currentstroke}%
\pgfsetdash{}{0pt}%
\pgfpathmoveto{\pgfqpoint{3.075774in}{2.607709in}}%
\pgfpathcurveto{\pgfqpoint{3.086824in}{2.607709in}}{\pgfqpoint{3.097423in}{2.612099in}}{\pgfqpoint{3.105237in}{2.619913in}}%
\pgfpathcurveto{\pgfqpoint{3.113051in}{2.627726in}}{\pgfqpoint{3.117441in}{2.638325in}}{\pgfqpoint{3.117441in}{2.649375in}}%
\pgfpathcurveto{\pgfqpoint{3.117441in}{2.660426in}}{\pgfqpoint{3.113051in}{2.671025in}}{\pgfqpoint{3.105237in}{2.678838in}}%
\pgfpathcurveto{\pgfqpoint{3.097423in}{2.686652in}}{\pgfqpoint{3.086824in}{2.691042in}}{\pgfqpoint{3.075774in}{2.691042in}}%
\pgfpathcurveto{\pgfqpoint{3.064724in}{2.691042in}}{\pgfqpoint{3.054125in}{2.686652in}}{\pgfqpoint{3.046312in}{2.678838in}}%
\pgfpathcurveto{\pgfqpoint{3.038498in}{2.671025in}}{\pgfqpoint{3.034108in}{2.660426in}}{\pgfqpoint{3.034108in}{2.649375in}}%
\pgfpathcurveto{\pgfqpoint{3.034108in}{2.638325in}}{\pgfqpoint{3.038498in}{2.627726in}}{\pgfqpoint{3.046312in}{2.619913in}}%
\pgfpathcurveto{\pgfqpoint{3.054125in}{2.612099in}}{\pgfqpoint{3.064724in}{2.607709in}}{\pgfqpoint{3.075774in}{2.607709in}}%
\pgfpathclose%
\pgfusepath{stroke,fill}%
\end{pgfscope}%
\begin{pgfscope}%
\pgfpathrectangle{\pgfqpoint{0.600000in}{0.600000in}}{\pgfqpoint{3.900000in}{3.900000in}}%
\pgfusepath{clip}%
\pgfsetbuttcap%
\pgfsetroundjoin%
\definecolor{currentfill}{rgb}{0.121569,0.466667,0.705882}%
\pgfsetfillcolor{currentfill}%
\pgfsetlinewidth{1.003750pt}%
\definecolor{currentstroke}{rgb}{0.121569,0.466667,0.705882}%
\pgfsetstrokecolor{currentstroke}%
\pgfsetdash{}{0pt}%
\pgfpathmoveto{\pgfqpoint{3.129707in}{2.659375in}}%
\pgfpathcurveto{\pgfqpoint{3.140757in}{2.659375in}}{\pgfqpoint{3.151356in}{2.663766in}}{\pgfqpoint{3.159169in}{2.671579in}}%
\pgfpathcurveto{\pgfqpoint{3.166983in}{2.679393in}}{\pgfqpoint{3.171373in}{2.689992in}}{\pgfqpoint{3.171373in}{2.701042in}}%
\pgfpathcurveto{\pgfqpoint{3.171373in}{2.712092in}}{\pgfqpoint{3.166983in}{2.722691in}}{\pgfqpoint{3.159169in}{2.730505in}}%
\pgfpathcurveto{\pgfqpoint{3.151356in}{2.738319in}}{\pgfqpoint{3.140757in}{2.742709in}}{\pgfqpoint{3.129707in}{2.742709in}}%
\pgfpathcurveto{\pgfqpoint{3.118656in}{2.742709in}}{\pgfqpoint{3.108057in}{2.738319in}}{\pgfqpoint{3.100244in}{2.730505in}}%
\pgfpathcurveto{\pgfqpoint{3.092430in}{2.722691in}}{\pgfqpoint{3.088040in}{2.712092in}}{\pgfqpoint{3.088040in}{2.701042in}}%
\pgfpathcurveto{\pgfqpoint{3.088040in}{2.689992in}}{\pgfqpoint{3.092430in}{2.679393in}}{\pgfqpoint{3.100244in}{2.671579in}}%
\pgfpathcurveto{\pgfqpoint{3.108057in}{2.663766in}}{\pgfqpoint{3.118656in}{2.659375in}}{\pgfqpoint{3.129707in}{2.659375in}}%
\pgfpathclose%
\pgfusepath{stroke,fill}%
\end{pgfscope}%
\begin{pgfscope}%
\pgfpathrectangle{\pgfqpoint{0.600000in}{0.600000in}}{\pgfqpoint{3.900000in}{3.900000in}}%
\pgfusepath{clip}%
\pgfsetbuttcap%
\pgfsetroundjoin%
\definecolor{currentfill}{rgb}{0.121569,0.466667,0.705882}%
\pgfsetfillcolor{currentfill}%
\pgfsetlinewidth{1.003750pt}%
\definecolor{currentstroke}{rgb}{0.121569,0.466667,0.705882}%
\pgfsetstrokecolor{currentstroke}%
\pgfsetdash{}{0pt}%
\pgfpathmoveto{\pgfqpoint{3.890042in}{2.283569in}}%
\pgfpathcurveto{\pgfqpoint{3.901092in}{2.283569in}}{\pgfqpoint{3.911691in}{2.287959in}}{\pgfqpoint{3.919505in}{2.295773in}}%
\pgfpathcurveto{\pgfqpoint{3.927319in}{2.303586in}}{\pgfqpoint{3.931709in}{2.314185in}}{\pgfqpoint{3.931709in}{2.325236in}}%
\pgfpathcurveto{\pgfqpoint{3.931709in}{2.336286in}}{\pgfqpoint{3.927319in}{2.346885in}}{\pgfqpoint{3.919505in}{2.354698in}}%
\pgfpathcurveto{\pgfqpoint{3.911691in}{2.362512in}}{\pgfqpoint{3.901092in}{2.366902in}}{\pgfqpoint{3.890042in}{2.366902in}}%
\pgfpathcurveto{\pgfqpoint{3.878992in}{2.366902in}}{\pgfqpoint{3.868393in}{2.362512in}}{\pgfqpoint{3.860579in}{2.354698in}}%
\pgfpathcurveto{\pgfqpoint{3.852766in}{2.346885in}}{\pgfqpoint{3.848376in}{2.336286in}}{\pgfqpoint{3.848376in}{2.325236in}}%
\pgfpathcurveto{\pgfqpoint{3.848376in}{2.314185in}}{\pgfqpoint{3.852766in}{2.303586in}}{\pgfqpoint{3.860579in}{2.295773in}}%
\pgfpathcurveto{\pgfqpoint{3.868393in}{2.287959in}}{\pgfqpoint{3.878992in}{2.283569in}}{\pgfqpoint{3.890042in}{2.283569in}}%
\pgfpathclose%
\pgfusepath{stroke,fill}%
\end{pgfscope}%
\begin{pgfscope}%
\pgfpathrectangle{\pgfqpoint{0.600000in}{0.600000in}}{\pgfqpoint{3.900000in}{3.900000in}}%
\pgfusepath{clip}%
\pgfsetbuttcap%
\pgfsetroundjoin%
\definecolor{currentfill}{rgb}{0.121569,0.466667,0.705882}%
\pgfsetfillcolor{currentfill}%
\pgfsetlinewidth{1.003750pt}%
\definecolor{currentstroke}{rgb}{0.121569,0.466667,0.705882}%
\pgfsetstrokecolor{currentstroke}%
\pgfsetdash{}{0pt}%
\pgfpathmoveto{\pgfqpoint{2.385576in}{1.141793in}}%
\pgfpathcurveto{\pgfqpoint{2.396626in}{1.141793in}}{\pgfqpoint{2.407225in}{1.146184in}}{\pgfqpoint{2.415039in}{1.153997in}}%
\pgfpathcurveto{\pgfqpoint{2.422853in}{1.161811in}}{\pgfqpoint{2.427243in}{1.172410in}}{\pgfqpoint{2.427243in}{1.183460in}}%
\pgfpathcurveto{\pgfqpoint{2.427243in}{1.194510in}}{\pgfqpoint{2.422853in}{1.205109in}}{\pgfqpoint{2.415039in}{1.212923in}}%
\pgfpathcurveto{\pgfqpoint{2.407225in}{1.220736in}}{\pgfqpoint{2.396626in}{1.225127in}}{\pgfqpoint{2.385576in}{1.225127in}}%
\pgfpathcurveto{\pgfqpoint{2.374526in}{1.225127in}}{\pgfqpoint{2.363927in}{1.220736in}}{\pgfqpoint{2.356113in}{1.212923in}}%
\pgfpathcurveto{\pgfqpoint{2.348300in}{1.205109in}}{\pgfqpoint{2.343910in}{1.194510in}}{\pgfqpoint{2.343910in}{1.183460in}}%
\pgfpathcurveto{\pgfqpoint{2.343910in}{1.172410in}}{\pgfqpoint{2.348300in}{1.161811in}}{\pgfqpoint{2.356113in}{1.153997in}}%
\pgfpathcurveto{\pgfqpoint{2.363927in}{1.146184in}}{\pgfqpoint{2.374526in}{1.141793in}}{\pgfqpoint{2.385576in}{1.141793in}}%
\pgfpathclose%
\pgfusepath{stroke,fill}%
\end{pgfscope}%
\begin{pgfscope}%
\pgfpathrectangle{\pgfqpoint{0.600000in}{0.600000in}}{\pgfqpoint{3.900000in}{3.900000in}}%
\pgfusepath{clip}%
\pgfsetbuttcap%
\pgfsetroundjoin%
\definecolor{currentfill}{rgb}{0.121569,0.466667,0.705882}%
\pgfsetfillcolor{currentfill}%
\pgfsetlinewidth{1.003750pt}%
\definecolor{currentstroke}{rgb}{0.121569,0.466667,0.705882}%
\pgfsetstrokecolor{currentstroke}%
\pgfsetdash{}{0pt}%
\pgfpathmoveto{\pgfqpoint{2.455478in}{2.846959in}}%
\pgfpathcurveto{\pgfqpoint{2.466528in}{2.846959in}}{\pgfqpoint{2.477127in}{2.851349in}}{\pgfqpoint{2.484941in}{2.859163in}}%
\pgfpathcurveto{\pgfqpoint{2.492755in}{2.866976in}}{\pgfqpoint{2.497145in}{2.877575in}}{\pgfqpoint{2.497145in}{2.888626in}}%
\pgfpathcurveto{\pgfqpoint{2.497145in}{2.899676in}}{\pgfqpoint{2.492755in}{2.910275in}}{\pgfqpoint{2.484941in}{2.918088in}}%
\pgfpathcurveto{\pgfqpoint{2.477127in}{2.925902in}}{\pgfqpoint{2.466528in}{2.930292in}}{\pgfqpoint{2.455478in}{2.930292in}}%
\pgfpathcurveto{\pgfqpoint{2.444428in}{2.930292in}}{\pgfqpoint{2.433829in}{2.925902in}}{\pgfqpoint{2.426015in}{2.918088in}}%
\pgfpathcurveto{\pgfqpoint{2.418202in}{2.910275in}}{\pgfqpoint{2.413811in}{2.899676in}}{\pgfqpoint{2.413811in}{2.888626in}}%
\pgfpathcurveto{\pgfqpoint{2.413811in}{2.877575in}}{\pgfqpoint{2.418202in}{2.866976in}}{\pgfqpoint{2.426015in}{2.859163in}}%
\pgfpathcurveto{\pgfqpoint{2.433829in}{2.851349in}}{\pgfqpoint{2.444428in}{2.846959in}}{\pgfqpoint{2.455478in}{2.846959in}}%
\pgfpathclose%
\pgfusepath{stroke,fill}%
\end{pgfscope}%
\begin{pgfscope}%
\pgfpathrectangle{\pgfqpoint{0.600000in}{0.600000in}}{\pgfqpoint{3.900000in}{3.900000in}}%
\pgfusepath{clip}%
\pgfsetbuttcap%
\pgfsetroundjoin%
\definecolor{currentfill}{rgb}{0.121569,0.466667,0.705882}%
\pgfsetfillcolor{currentfill}%
\pgfsetlinewidth{1.003750pt}%
\definecolor{currentstroke}{rgb}{0.121569,0.466667,0.705882}%
\pgfsetstrokecolor{currentstroke}%
\pgfsetdash{}{0pt}%
\pgfpathmoveto{\pgfqpoint{2.149604in}{2.927228in}}%
\pgfpathcurveto{\pgfqpoint{2.160654in}{2.927228in}}{\pgfqpoint{2.171253in}{2.931618in}}{\pgfqpoint{2.179066in}{2.939432in}}%
\pgfpathcurveto{\pgfqpoint{2.186880in}{2.947246in}}{\pgfqpoint{2.191270in}{2.957845in}}{\pgfqpoint{2.191270in}{2.968895in}}%
\pgfpathcurveto{\pgfqpoint{2.191270in}{2.979945in}}{\pgfqpoint{2.186880in}{2.990544in}}{\pgfqpoint{2.179066in}{2.998358in}}%
\pgfpathcurveto{\pgfqpoint{2.171253in}{3.006171in}}{\pgfqpoint{2.160654in}{3.010561in}}{\pgfqpoint{2.149604in}{3.010561in}}%
\pgfpathcurveto{\pgfqpoint{2.138553in}{3.010561in}}{\pgfqpoint{2.127954in}{3.006171in}}{\pgfqpoint{2.120141in}{2.998358in}}%
\pgfpathcurveto{\pgfqpoint{2.112327in}{2.990544in}}{\pgfqpoint{2.107937in}{2.979945in}}{\pgfqpoint{2.107937in}{2.968895in}}%
\pgfpathcurveto{\pgfqpoint{2.107937in}{2.957845in}}{\pgfqpoint{2.112327in}{2.947246in}}{\pgfqpoint{2.120141in}{2.939432in}}%
\pgfpathcurveto{\pgfqpoint{2.127954in}{2.931618in}}{\pgfqpoint{2.138553in}{2.927228in}}{\pgfqpoint{2.149604in}{2.927228in}}%
\pgfpathclose%
\pgfusepath{stroke,fill}%
\end{pgfscope}%
\begin{pgfscope}%
\pgfpathrectangle{\pgfqpoint{0.600000in}{0.600000in}}{\pgfqpoint{3.900000in}{3.900000in}}%
\pgfusepath{clip}%
\pgfsetbuttcap%
\pgfsetroundjoin%
\definecolor{currentfill}{rgb}{0.121569,0.466667,0.705882}%
\pgfsetfillcolor{currentfill}%
\pgfsetlinewidth{1.003750pt}%
\definecolor{currentstroke}{rgb}{0.121569,0.466667,0.705882}%
\pgfsetstrokecolor{currentstroke}%
\pgfsetdash{}{0pt}%
\pgfpathmoveto{\pgfqpoint{2.572891in}{2.887566in}}%
\pgfpathcurveto{\pgfqpoint{2.583941in}{2.887566in}}{\pgfqpoint{2.594540in}{2.891956in}}{\pgfqpoint{2.602354in}{2.899770in}}%
\pgfpathcurveto{\pgfqpoint{2.610167in}{2.907584in}}{\pgfqpoint{2.614557in}{2.918183in}}{\pgfqpoint{2.614557in}{2.929233in}}%
\pgfpathcurveto{\pgfqpoint{2.614557in}{2.940283in}}{\pgfqpoint{2.610167in}{2.950882in}}{\pgfqpoint{2.602354in}{2.958696in}}%
\pgfpathcurveto{\pgfqpoint{2.594540in}{2.966509in}}{\pgfqpoint{2.583941in}{2.970899in}}{\pgfqpoint{2.572891in}{2.970899in}}%
\pgfpathcurveto{\pgfqpoint{2.561841in}{2.970899in}}{\pgfqpoint{2.551242in}{2.966509in}}{\pgfqpoint{2.543428in}{2.958696in}}%
\pgfpathcurveto{\pgfqpoint{2.535614in}{2.950882in}}{\pgfqpoint{2.531224in}{2.940283in}}{\pgfqpoint{2.531224in}{2.929233in}}%
\pgfpathcurveto{\pgfqpoint{2.531224in}{2.918183in}}{\pgfqpoint{2.535614in}{2.907584in}}{\pgfqpoint{2.543428in}{2.899770in}}%
\pgfpathcurveto{\pgfqpoint{2.551242in}{2.891956in}}{\pgfqpoint{2.561841in}{2.887566in}}{\pgfqpoint{2.572891in}{2.887566in}}%
\pgfpathclose%
\pgfusepath{stroke,fill}%
\end{pgfscope}%
\begin{pgfscope}%
\pgfpathrectangle{\pgfqpoint{0.600000in}{0.600000in}}{\pgfqpoint{3.900000in}{3.900000in}}%
\pgfusepath{clip}%
\pgfsetbuttcap%
\pgfsetroundjoin%
\definecolor{currentfill}{rgb}{0.121569,0.466667,0.705882}%
\pgfsetfillcolor{currentfill}%
\pgfsetlinewidth{1.003750pt}%
\definecolor{currentstroke}{rgb}{0.121569,0.466667,0.705882}%
\pgfsetstrokecolor{currentstroke}%
\pgfsetdash{}{0pt}%
\pgfpathmoveto{\pgfqpoint{2.895883in}{2.142828in}}%
\pgfpathcurveto{\pgfqpoint{2.906933in}{2.142828in}}{\pgfqpoint{2.917532in}{2.147218in}}{\pgfqpoint{2.925346in}{2.155032in}}%
\pgfpathcurveto{\pgfqpoint{2.933159in}{2.162846in}}{\pgfqpoint{2.937549in}{2.173445in}}{\pgfqpoint{2.937549in}{2.184495in}}%
\pgfpathcurveto{\pgfqpoint{2.937549in}{2.195545in}}{\pgfqpoint{2.933159in}{2.206144in}}{\pgfqpoint{2.925346in}{2.213958in}}%
\pgfpathcurveto{\pgfqpoint{2.917532in}{2.221771in}}{\pgfqpoint{2.906933in}{2.226161in}}{\pgfqpoint{2.895883in}{2.226161in}}%
\pgfpathcurveto{\pgfqpoint{2.884833in}{2.226161in}}{\pgfqpoint{2.874234in}{2.221771in}}{\pgfqpoint{2.866420in}{2.213958in}}%
\pgfpathcurveto{\pgfqpoint{2.858606in}{2.206144in}}{\pgfqpoint{2.854216in}{2.195545in}}{\pgfqpoint{2.854216in}{2.184495in}}%
\pgfpathcurveto{\pgfqpoint{2.854216in}{2.173445in}}{\pgfqpoint{2.858606in}{2.162846in}}{\pgfqpoint{2.866420in}{2.155032in}}%
\pgfpathcurveto{\pgfqpoint{2.874234in}{2.147218in}}{\pgfqpoint{2.884833in}{2.142828in}}{\pgfqpoint{2.895883in}{2.142828in}}%
\pgfpathclose%
\pgfusepath{stroke,fill}%
\end{pgfscope}%
\begin{pgfscope}%
\pgfpathrectangle{\pgfqpoint{0.600000in}{0.600000in}}{\pgfqpoint{3.900000in}{3.900000in}}%
\pgfusepath{clip}%
\pgfsetbuttcap%
\pgfsetroundjoin%
\definecolor{currentfill}{rgb}{0.121569,0.466667,0.705882}%
\pgfsetfillcolor{currentfill}%
\pgfsetlinewidth{1.003750pt}%
\definecolor{currentstroke}{rgb}{0.121569,0.466667,0.705882}%
\pgfsetstrokecolor{currentstroke}%
\pgfsetdash{}{0pt}%
\pgfpathmoveto{\pgfqpoint{2.434955in}{3.227913in}}%
\pgfpathcurveto{\pgfqpoint{2.446005in}{3.227913in}}{\pgfqpoint{2.456604in}{3.232303in}}{\pgfqpoint{2.464417in}{3.240117in}}%
\pgfpathcurveto{\pgfqpoint{2.472231in}{3.247930in}}{\pgfqpoint{2.476621in}{3.258529in}}{\pgfqpoint{2.476621in}{3.269579in}}%
\pgfpathcurveto{\pgfqpoint{2.476621in}{3.280629in}}{\pgfqpoint{2.472231in}{3.291229in}}{\pgfqpoint{2.464417in}{3.299042in}}%
\pgfpathcurveto{\pgfqpoint{2.456604in}{3.306856in}}{\pgfqpoint{2.446005in}{3.311246in}}{\pgfqpoint{2.434955in}{3.311246in}}%
\pgfpathcurveto{\pgfqpoint{2.423905in}{3.311246in}}{\pgfqpoint{2.413306in}{3.306856in}}{\pgfqpoint{2.405492in}{3.299042in}}%
\pgfpathcurveto{\pgfqpoint{2.397678in}{3.291229in}}{\pgfqpoint{2.393288in}{3.280629in}}{\pgfqpoint{2.393288in}{3.269579in}}%
\pgfpathcurveto{\pgfqpoint{2.393288in}{3.258529in}}{\pgfqpoint{2.397678in}{3.247930in}}{\pgfqpoint{2.405492in}{3.240117in}}%
\pgfpathcurveto{\pgfqpoint{2.413306in}{3.232303in}}{\pgfqpoint{2.423905in}{3.227913in}}{\pgfqpoint{2.434955in}{3.227913in}}%
\pgfpathclose%
\pgfusepath{stroke,fill}%
\end{pgfscope}%
\begin{pgfscope}%
\pgfpathrectangle{\pgfqpoint{0.600000in}{0.600000in}}{\pgfqpoint{3.900000in}{3.900000in}}%
\pgfusepath{clip}%
\pgfsetbuttcap%
\pgfsetroundjoin%
\definecolor{currentfill}{rgb}{0.121569,0.466667,0.705882}%
\pgfsetfillcolor{currentfill}%
\pgfsetlinewidth{1.003750pt}%
\definecolor{currentstroke}{rgb}{0.121569,0.466667,0.705882}%
\pgfsetstrokecolor{currentstroke}%
\pgfsetdash{}{0pt}%
\pgfpathmoveto{\pgfqpoint{2.550378in}{1.706995in}}%
\pgfpathcurveto{\pgfqpoint{2.561428in}{1.706995in}}{\pgfqpoint{2.572027in}{1.711385in}}{\pgfqpoint{2.579841in}{1.719199in}}%
\pgfpathcurveto{\pgfqpoint{2.587654in}{1.727012in}}{\pgfqpoint{2.592045in}{1.737611in}}{\pgfqpoint{2.592045in}{1.748661in}}%
\pgfpathcurveto{\pgfqpoint{2.592045in}{1.759711in}}{\pgfqpoint{2.587654in}{1.770310in}}{\pgfqpoint{2.579841in}{1.778124in}}%
\pgfpathcurveto{\pgfqpoint{2.572027in}{1.785938in}}{\pgfqpoint{2.561428in}{1.790328in}}{\pgfqpoint{2.550378in}{1.790328in}}%
\pgfpathcurveto{\pgfqpoint{2.539328in}{1.790328in}}{\pgfqpoint{2.528729in}{1.785938in}}{\pgfqpoint{2.520915in}{1.778124in}}%
\pgfpathcurveto{\pgfqpoint{2.513102in}{1.770310in}}{\pgfqpoint{2.508711in}{1.759711in}}{\pgfqpoint{2.508711in}{1.748661in}}%
\pgfpathcurveto{\pgfqpoint{2.508711in}{1.737611in}}{\pgfqpoint{2.513102in}{1.727012in}}{\pgfqpoint{2.520915in}{1.719199in}}%
\pgfpathcurveto{\pgfqpoint{2.528729in}{1.711385in}}{\pgfqpoint{2.539328in}{1.706995in}}{\pgfqpoint{2.550378in}{1.706995in}}%
\pgfpathclose%
\pgfusepath{stroke,fill}%
\end{pgfscope}%
\begin{pgfscope}%
\pgfpathrectangle{\pgfqpoint{0.600000in}{0.600000in}}{\pgfqpoint{3.900000in}{3.900000in}}%
\pgfusepath{clip}%
\pgfsetbuttcap%
\pgfsetroundjoin%
\definecolor{currentfill}{rgb}{0.121569,0.466667,0.705882}%
\pgfsetfillcolor{currentfill}%
\pgfsetlinewidth{1.003750pt}%
\definecolor{currentstroke}{rgb}{0.121569,0.466667,0.705882}%
\pgfsetstrokecolor{currentstroke}%
\pgfsetdash{}{0pt}%
\pgfpathmoveto{\pgfqpoint{2.221879in}{2.193721in}}%
\pgfpathcurveto{\pgfqpoint{2.232929in}{2.193721in}}{\pgfqpoint{2.243529in}{2.198111in}}{\pgfqpoint{2.251342in}{2.205925in}}%
\pgfpathcurveto{\pgfqpoint{2.259156in}{2.213738in}}{\pgfqpoint{2.263546in}{2.224337in}}{\pgfqpoint{2.263546in}{2.235388in}}%
\pgfpathcurveto{\pgfqpoint{2.263546in}{2.246438in}}{\pgfqpoint{2.259156in}{2.257037in}}{\pgfqpoint{2.251342in}{2.264850in}}%
\pgfpathcurveto{\pgfqpoint{2.243529in}{2.272664in}}{\pgfqpoint{2.232929in}{2.277054in}}{\pgfqpoint{2.221879in}{2.277054in}}%
\pgfpathcurveto{\pgfqpoint{2.210829in}{2.277054in}}{\pgfqpoint{2.200230in}{2.272664in}}{\pgfqpoint{2.192417in}{2.264850in}}%
\pgfpathcurveto{\pgfqpoint{2.184603in}{2.257037in}}{\pgfqpoint{2.180213in}{2.246438in}}{\pgfqpoint{2.180213in}{2.235388in}}%
\pgfpathcurveto{\pgfqpoint{2.180213in}{2.224337in}}{\pgfqpoint{2.184603in}{2.213738in}}{\pgfqpoint{2.192417in}{2.205925in}}%
\pgfpathcurveto{\pgfqpoint{2.200230in}{2.198111in}}{\pgfqpoint{2.210829in}{2.193721in}}{\pgfqpoint{2.221879in}{2.193721in}}%
\pgfpathclose%
\pgfusepath{stroke,fill}%
\end{pgfscope}%
\begin{pgfscope}%
\pgfpathrectangle{\pgfqpoint{0.600000in}{0.600000in}}{\pgfqpoint{3.900000in}{3.900000in}}%
\pgfusepath{clip}%
\pgfsetbuttcap%
\pgfsetroundjoin%
\definecolor{currentfill}{rgb}{0.121569,0.466667,0.705882}%
\pgfsetfillcolor{currentfill}%
\pgfsetlinewidth{1.003750pt}%
\definecolor{currentstroke}{rgb}{0.121569,0.466667,0.705882}%
\pgfsetstrokecolor{currentstroke}%
\pgfsetdash{}{0pt}%
\pgfpathmoveto{\pgfqpoint{2.682498in}{3.065247in}}%
\pgfpathcurveto{\pgfqpoint{2.693549in}{3.065247in}}{\pgfqpoint{2.704148in}{3.069638in}}{\pgfqpoint{2.711961in}{3.077451in}}%
\pgfpathcurveto{\pgfqpoint{2.719775in}{3.085265in}}{\pgfqpoint{2.724165in}{3.095864in}}{\pgfqpoint{2.724165in}{3.106914in}}%
\pgfpathcurveto{\pgfqpoint{2.724165in}{3.117964in}}{\pgfqpoint{2.719775in}{3.128563in}}{\pgfqpoint{2.711961in}{3.136377in}}%
\pgfpathcurveto{\pgfqpoint{2.704148in}{3.144190in}}{\pgfqpoint{2.693549in}{3.148581in}}{\pgfqpoint{2.682498in}{3.148581in}}%
\pgfpathcurveto{\pgfqpoint{2.671448in}{3.148581in}}{\pgfqpoint{2.660849in}{3.144190in}}{\pgfqpoint{2.653036in}{3.136377in}}%
\pgfpathcurveto{\pgfqpoint{2.645222in}{3.128563in}}{\pgfqpoint{2.640832in}{3.117964in}}{\pgfqpoint{2.640832in}{3.106914in}}%
\pgfpathcurveto{\pgfqpoint{2.640832in}{3.095864in}}{\pgfqpoint{2.645222in}{3.085265in}}{\pgfqpoint{2.653036in}{3.077451in}}%
\pgfpathcurveto{\pgfqpoint{2.660849in}{3.069638in}}{\pgfqpoint{2.671448in}{3.065247in}}{\pgfqpoint{2.682498in}{3.065247in}}%
\pgfpathclose%
\pgfusepath{stroke,fill}%
\end{pgfscope}%
\begin{pgfscope}%
\pgfpathrectangle{\pgfqpoint{0.600000in}{0.600000in}}{\pgfqpoint{3.900000in}{3.900000in}}%
\pgfusepath{clip}%
\pgfsetbuttcap%
\pgfsetroundjoin%
\definecolor{currentfill}{rgb}{0.121569,0.466667,0.705882}%
\pgfsetfillcolor{currentfill}%
\pgfsetlinewidth{1.003750pt}%
\definecolor{currentstroke}{rgb}{0.121569,0.466667,0.705882}%
\pgfsetstrokecolor{currentstroke}%
\pgfsetdash{}{0pt}%
\pgfpathmoveto{\pgfqpoint{1.667513in}{2.268255in}}%
\pgfpathcurveto{\pgfqpoint{1.678563in}{2.268255in}}{\pgfqpoint{1.689162in}{2.272645in}}{\pgfqpoint{1.696975in}{2.280459in}}%
\pgfpathcurveto{\pgfqpoint{1.704789in}{2.288273in}}{\pgfqpoint{1.709179in}{2.298872in}}{\pgfqpoint{1.709179in}{2.309922in}}%
\pgfpathcurveto{\pgfqpoint{1.709179in}{2.320972in}}{\pgfqpoint{1.704789in}{2.331571in}}{\pgfqpoint{1.696975in}{2.339385in}}%
\pgfpathcurveto{\pgfqpoint{1.689162in}{2.347198in}}{\pgfqpoint{1.678563in}{2.351588in}}{\pgfqpoint{1.667513in}{2.351588in}}%
\pgfpathcurveto{\pgfqpoint{1.656462in}{2.351588in}}{\pgfqpoint{1.645863in}{2.347198in}}{\pgfqpoint{1.638050in}{2.339385in}}%
\pgfpathcurveto{\pgfqpoint{1.630236in}{2.331571in}}{\pgfqpoint{1.625846in}{2.320972in}}{\pgfqpoint{1.625846in}{2.309922in}}%
\pgfpathcurveto{\pgfqpoint{1.625846in}{2.298872in}}{\pgfqpoint{1.630236in}{2.288273in}}{\pgfqpoint{1.638050in}{2.280459in}}%
\pgfpathcurveto{\pgfqpoint{1.645863in}{2.272645in}}{\pgfqpoint{1.656462in}{2.268255in}}{\pgfqpoint{1.667513in}{2.268255in}}%
\pgfpathclose%
\pgfusepath{stroke,fill}%
\end{pgfscope}%
\begin{pgfscope}%
\pgfpathrectangle{\pgfqpoint{0.600000in}{0.600000in}}{\pgfqpoint{3.900000in}{3.900000in}}%
\pgfusepath{clip}%
\pgfsetbuttcap%
\pgfsetroundjoin%
\definecolor{currentfill}{rgb}{0.121569,0.466667,0.705882}%
\pgfsetfillcolor{currentfill}%
\pgfsetlinewidth{1.003750pt}%
\definecolor{currentstroke}{rgb}{0.121569,0.466667,0.705882}%
\pgfsetstrokecolor{currentstroke}%
\pgfsetdash{}{0pt}%
\pgfpathmoveto{\pgfqpoint{2.988631in}{2.584381in}}%
\pgfpathcurveto{\pgfqpoint{2.999681in}{2.584381in}}{\pgfqpoint{3.010280in}{2.588771in}}{\pgfqpoint{3.018094in}{2.596585in}}%
\pgfpathcurveto{\pgfqpoint{3.025907in}{2.604398in}}{\pgfqpoint{3.030298in}{2.614997in}}{\pgfqpoint{3.030298in}{2.626048in}}%
\pgfpathcurveto{\pgfqpoint{3.030298in}{2.637098in}}{\pgfqpoint{3.025907in}{2.647697in}}{\pgfqpoint{3.018094in}{2.655510in}}%
\pgfpathcurveto{\pgfqpoint{3.010280in}{2.663324in}}{\pgfqpoint{2.999681in}{2.667714in}}{\pgfqpoint{2.988631in}{2.667714in}}%
\pgfpathcurveto{\pgfqpoint{2.977581in}{2.667714in}}{\pgfqpoint{2.966982in}{2.663324in}}{\pgfqpoint{2.959168in}{2.655510in}}%
\pgfpathcurveto{\pgfqpoint{2.951355in}{2.647697in}}{\pgfqpoint{2.946964in}{2.637098in}}{\pgfqpoint{2.946964in}{2.626048in}}%
\pgfpathcurveto{\pgfqpoint{2.946964in}{2.614997in}}{\pgfqpoint{2.951355in}{2.604398in}}{\pgfqpoint{2.959168in}{2.596585in}}%
\pgfpathcurveto{\pgfqpoint{2.966982in}{2.588771in}}{\pgfqpoint{2.977581in}{2.584381in}}{\pgfqpoint{2.988631in}{2.584381in}}%
\pgfpathclose%
\pgfusepath{stroke,fill}%
\end{pgfscope}%
\begin{pgfscope}%
\pgfpathrectangle{\pgfqpoint{0.600000in}{0.600000in}}{\pgfqpoint{3.900000in}{3.900000in}}%
\pgfusepath{clip}%
\pgfsetbuttcap%
\pgfsetroundjoin%
\definecolor{currentfill}{rgb}{0.121569,0.466667,0.705882}%
\pgfsetfillcolor{currentfill}%
\pgfsetlinewidth{1.003750pt}%
\definecolor{currentstroke}{rgb}{0.121569,0.466667,0.705882}%
\pgfsetstrokecolor{currentstroke}%
\pgfsetdash{}{0pt}%
\pgfpathmoveto{\pgfqpoint{2.681672in}{3.300446in}}%
\pgfpathcurveto{\pgfqpoint{2.692722in}{3.300446in}}{\pgfqpoint{2.703321in}{3.304836in}}{\pgfqpoint{2.711135in}{3.312650in}}%
\pgfpathcurveto{\pgfqpoint{2.718949in}{3.320463in}}{\pgfqpoint{2.723339in}{3.331063in}}{\pgfqpoint{2.723339in}{3.342113in}}%
\pgfpathcurveto{\pgfqpoint{2.723339in}{3.353163in}}{\pgfqpoint{2.718949in}{3.363762in}}{\pgfqpoint{2.711135in}{3.371575in}}%
\pgfpathcurveto{\pgfqpoint{2.703321in}{3.379389in}}{\pgfqpoint{2.692722in}{3.383779in}}{\pgfqpoint{2.681672in}{3.383779in}}%
\pgfpathcurveto{\pgfqpoint{2.670622in}{3.383779in}}{\pgfqpoint{2.660023in}{3.379389in}}{\pgfqpoint{2.652209in}{3.371575in}}%
\pgfpathcurveto{\pgfqpoint{2.644396in}{3.363762in}}{\pgfqpoint{2.640006in}{3.353163in}}{\pgfqpoint{2.640006in}{3.342113in}}%
\pgfpathcurveto{\pgfqpoint{2.640006in}{3.331063in}}{\pgfqpoint{2.644396in}{3.320463in}}{\pgfqpoint{2.652209in}{3.312650in}}%
\pgfpathcurveto{\pgfqpoint{2.660023in}{3.304836in}}{\pgfqpoint{2.670622in}{3.300446in}}{\pgfqpoint{2.681672in}{3.300446in}}%
\pgfpathclose%
\pgfusepath{stroke,fill}%
\end{pgfscope}%
\begin{pgfscope}%
\pgfpathrectangle{\pgfqpoint{0.600000in}{0.600000in}}{\pgfqpoint{3.900000in}{3.900000in}}%
\pgfusepath{clip}%
\pgfsetbuttcap%
\pgfsetroundjoin%
\definecolor{currentfill}{rgb}{0.121569,0.466667,0.705882}%
\pgfsetfillcolor{currentfill}%
\pgfsetlinewidth{1.003750pt}%
\definecolor{currentstroke}{rgb}{0.121569,0.466667,0.705882}%
\pgfsetstrokecolor{currentstroke}%
\pgfsetdash{}{0pt}%
\pgfpathmoveto{\pgfqpoint{2.482792in}{2.762550in}}%
\pgfpathcurveto{\pgfqpoint{2.493842in}{2.762550in}}{\pgfqpoint{2.504441in}{2.766941in}}{\pgfqpoint{2.512255in}{2.774754in}}%
\pgfpathcurveto{\pgfqpoint{2.520068in}{2.782568in}}{\pgfqpoint{2.524459in}{2.793167in}}{\pgfqpoint{2.524459in}{2.804217in}}%
\pgfpathcurveto{\pgfqpoint{2.524459in}{2.815267in}}{\pgfqpoint{2.520068in}{2.825866in}}{\pgfqpoint{2.512255in}{2.833680in}}%
\pgfpathcurveto{\pgfqpoint{2.504441in}{2.841493in}}{\pgfqpoint{2.493842in}{2.845884in}}{\pgfqpoint{2.482792in}{2.845884in}}%
\pgfpathcurveto{\pgfqpoint{2.471742in}{2.845884in}}{\pgfqpoint{2.461143in}{2.841493in}}{\pgfqpoint{2.453329in}{2.833680in}}%
\pgfpathcurveto{\pgfqpoint{2.445516in}{2.825866in}}{\pgfqpoint{2.441125in}{2.815267in}}{\pgfqpoint{2.441125in}{2.804217in}}%
\pgfpathcurveto{\pgfqpoint{2.441125in}{2.793167in}}{\pgfqpoint{2.445516in}{2.782568in}}{\pgfqpoint{2.453329in}{2.774754in}}%
\pgfpathcurveto{\pgfqpoint{2.461143in}{2.766941in}}{\pgfqpoint{2.471742in}{2.762550in}}{\pgfqpoint{2.482792in}{2.762550in}}%
\pgfpathclose%
\pgfusepath{stroke,fill}%
\end{pgfscope}%
\begin{pgfscope}%
\pgfpathrectangle{\pgfqpoint{0.600000in}{0.600000in}}{\pgfqpoint{3.900000in}{3.900000in}}%
\pgfusepath{clip}%
\pgfsetbuttcap%
\pgfsetroundjoin%
\definecolor{currentfill}{rgb}{0.121569,0.466667,0.705882}%
\pgfsetfillcolor{currentfill}%
\pgfsetlinewidth{1.003750pt}%
\definecolor{currentstroke}{rgb}{0.121569,0.466667,0.705882}%
\pgfsetstrokecolor{currentstroke}%
\pgfsetdash{}{0pt}%
\pgfpathmoveto{\pgfqpoint{3.087820in}{2.721562in}}%
\pgfpathcurveto{\pgfqpoint{3.098870in}{2.721562in}}{\pgfqpoint{3.109469in}{2.725953in}}{\pgfqpoint{3.117283in}{2.733766in}}%
\pgfpathcurveto{\pgfqpoint{3.125096in}{2.741580in}}{\pgfqpoint{3.129487in}{2.752179in}}{\pgfqpoint{3.129487in}{2.763229in}}%
\pgfpathcurveto{\pgfqpoint{3.129487in}{2.774279in}}{\pgfqpoint{3.125096in}{2.784878in}}{\pgfqpoint{3.117283in}{2.792692in}}%
\pgfpathcurveto{\pgfqpoint{3.109469in}{2.800506in}}{\pgfqpoint{3.098870in}{2.804896in}}{\pgfqpoint{3.087820in}{2.804896in}}%
\pgfpathcurveto{\pgfqpoint{3.076770in}{2.804896in}}{\pgfqpoint{3.066171in}{2.800506in}}{\pgfqpoint{3.058357in}{2.792692in}}%
\pgfpathcurveto{\pgfqpoint{3.050544in}{2.784878in}}{\pgfqpoint{3.046153in}{2.774279in}}{\pgfqpoint{3.046153in}{2.763229in}}%
\pgfpathcurveto{\pgfqpoint{3.046153in}{2.752179in}}{\pgfqpoint{3.050544in}{2.741580in}}{\pgfqpoint{3.058357in}{2.733766in}}%
\pgfpathcurveto{\pgfqpoint{3.066171in}{2.725953in}}{\pgfqpoint{3.076770in}{2.721562in}}{\pgfqpoint{3.087820in}{2.721562in}}%
\pgfpathclose%
\pgfusepath{stroke,fill}%
\end{pgfscope}%
\begin{pgfscope}%
\pgfpathrectangle{\pgfqpoint{0.600000in}{0.600000in}}{\pgfqpoint{3.900000in}{3.900000in}}%
\pgfusepath{clip}%
\pgfsetbuttcap%
\pgfsetroundjoin%
\definecolor{currentfill}{rgb}{0.121569,0.466667,0.705882}%
\pgfsetfillcolor{currentfill}%
\pgfsetlinewidth{1.003750pt}%
\definecolor{currentstroke}{rgb}{0.121569,0.466667,0.705882}%
\pgfsetstrokecolor{currentstroke}%
\pgfsetdash{}{0pt}%
\pgfpathmoveto{\pgfqpoint{2.257149in}{2.208348in}}%
\pgfpathcurveto{\pgfqpoint{2.268199in}{2.208348in}}{\pgfqpoint{2.278798in}{2.212738in}}{\pgfqpoint{2.286612in}{2.220551in}}%
\pgfpathcurveto{\pgfqpoint{2.294425in}{2.228365in}}{\pgfqpoint{2.298816in}{2.238964in}}{\pgfqpoint{2.298816in}{2.250014in}}%
\pgfpathcurveto{\pgfqpoint{2.298816in}{2.261064in}}{\pgfqpoint{2.294425in}{2.271663in}}{\pgfqpoint{2.286612in}{2.279477in}}%
\pgfpathcurveto{\pgfqpoint{2.278798in}{2.287291in}}{\pgfqpoint{2.268199in}{2.291681in}}{\pgfqpoint{2.257149in}{2.291681in}}%
\pgfpathcurveto{\pgfqpoint{2.246099in}{2.291681in}}{\pgfqpoint{2.235500in}{2.287291in}}{\pgfqpoint{2.227686in}{2.279477in}}%
\pgfpathcurveto{\pgfqpoint{2.219872in}{2.271663in}}{\pgfqpoint{2.215482in}{2.261064in}}{\pgfqpoint{2.215482in}{2.250014in}}%
\pgfpathcurveto{\pgfqpoint{2.215482in}{2.238964in}}{\pgfqpoint{2.219872in}{2.228365in}}{\pgfqpoint{2.227686in}{2.220551in}}%
\pgfpathcurveto{\pgfqpoint{2.235500in}{2.212738in}}{\pgfqpoint{2.246099in}{2.208348in}}{\pgfqpoint{2.257149in}{2.208348in}}%
\pgfpathclose%
\pgfusepath{stroke,fill}%
\end{pgfscope}%
\begin{pgfscope}%
\pgfpathrectangle{\pgfqpoint{0.600000in}{0.600000in}}{\pgfqpoint{3.900000in}{3.900000in}}%
\pgfusepath{clip}%
\pgfsetbuttcap%
\pgfsetroundjoin%
\definecolor{currentfill}{rgb}{0.121569,0.466667,0.705882}%
\pgfsetfillcolor{currentfill}%
\pgfsetlinewidth{1.003750pt}%
\definecolor{currentstroke}{rgb}{0.121569,0.466667,0.705882}%
\pgfsetstrokecolor{currentstroke}%
\pgfsetdash{}{0pt}%
\pgfpathmoveto{\pgfqpoint{1.843278in}{2.660176in}}%
\pgfpathcurveto{\pgfqpoint{1.854328in}{2.660176in}}{\pgfqpoint{1.864927in}{2.664566in}}{\pgfqpoint{1.872741in}{2.672380in}}%
\pgfpathcurveto{\pgfqpoint{1.880555in}{2.680193in}}{\pgfqpoint{1.884945in}{2.690792in}}{\pgfqpoint{1.884945in}{2.701842in}}%
\pgfpathcurveto{\pgfqpoint{1.884945in}{2.712892in}}{\pgfqpoint{1.880555in}{2.723492in}}{\pgfqpoint{1.872741in}{2.731305in}}%
\pgfpathcurveto{\pgfqpoint{1.864927in}{2.739119in}}{\pgfqpoint{1.854328in}{2.743509in}}{\pgfqpoint{1.843278in}{2.743509in}}%
\pgfpathcurveto{\pgfqpoint{1.832228in}{2.743509in}}{\pgfqpoint{1.821629in}{2.739119in}}{\pgfqpoint{1.813815in}{2.731305in}}%
\pgfpathcurveto{\pgfqpoint{1.806002in}{2.723492in}}{\pgfqpoint{1.801612in}{2.712892in}}{\pgfqpoint{1.801612in}{2.701842in}}%
\pgfpathcurveto{\pgfqpoint{1.801612in}{2.690792in}}{\pgfqpoint{1.806002in}{2.680193in}}{\pgfqpoint{1.813815in}{2.672380in}}%
\pgfpathcurveto{\pgfqpoint{1.821629in}{2.664566in}}{\pgfqpoint{1.832228in}{2.660176in}}{\pgfqpoint{1.843278in}{2.660176in}}%
\pgfpathclose%
\pgfusepath{stroke,fill}%
\end{pgfscope}%
\begin{pgfscope}%
\pgfpathrectangle{\pgfqpoint{0.600000in}{0.600000in}}{\pgfqpoint{3.900000in}{3.900000in}}%
\pgfusepath{clip}%
\pgfsetbuttcap%
\pgfsetroundjoin%
\definecolor{currentfill}{rgb}{0.121569,0.466667,0.705882}%
\pgfsetfillcolor{currentfill}%
\pgfsetlinewidth{1.003750pt}%
\definecolor{currentstroke}{rgb}{0.121569,0.466667,0.705882}%
\pgfsetstrokecolor{currentstroke}%
\pgfsetdash{}{0pt}%
\pgfpathmoveto{\pgfqpoint{3.064994in}{1.985429in}}%
\pgfpathcurveto{\pgfqpoint{3.076044in}{1.985429in}}{\pgfqpoint{3.086643in}{1.989819in}}{\pgfqpoint{3.094457in}{1.997633in}}%
\pgfpathcurveto{\pgfqpoint{3.102270in}{2.005447in}}{\pgfqpoint{3.106660in}{2.016046in}}{\pgfqpoint{3.106660in}{2.027096in}}%
\pgfpathcurveto{\pgfqpoint{3.106660in}{2.038146in}}{\pgfqpoint{3.102270in}{2.048745in}}{\pgfqpoint{3.094457in}{2.056559in}}%
\pgfpathcurveto{\pgfqpoint{3.086643in}{2.064372in}}{\pgfqpoint{3.076044in}{2.068762in}}{\pgfqpoint{3.064994in}{2.068762in}}%
\pgfpathcurveto{\pgfqpoint{3.053944in}{2.068762in}}{\pgfqpoint{3.043345in}{2.064372in}}{\pgfqpoint{3.035531in}{2.056559in}}%
\pgfpathcurveto{\pgfqpoint{3.027717in}{2.048745in}}{\pgfqpoint{3.023327in}{2.038146in}}{\pgfqpoint{3.023327in}{2.027096in}}%
\pgfpathcurveto{\pgfqpoint{3.023327in}{2.016046in}}{\pgfqpoint{3.027717in}{2.005447in}}{\pgfqpoint{3.035531in}{1.997633in}}%
\pgfpathcurveto{\pgfqpoint{3.043345in}{1.989819in}}{\pgfqpoint{3.053944in}{1.985429in}}{\pgfqpoint{3.064994in}{1.985429in}}%
\pgfpathclose%
\pgfusepath{stroke,fill}%
\end{pgfscope}%
\begin{pgfscope}%
\pgfpathrectangle{\pgfqpoint{0.600000in}{0.600000in}}{\pgfqpoint{3.900000in}{3.900000in}}%
\pgfusepath{clip}%
\pgfsetbuttcap%
\pgfsetroundjoin%
\definecolor{currentfill}{rgb}{0.121569,0.466667,0.705882}%
\pgfsetfillcolor{currentfill}%
\pgfsetlinewidth{1.003750pt}%
\definecolor{currentstroke}{rgb}{0.121569,0.466667,0.705882}%
\pgfsetstrokecolor{currentstroke}%
\pgfsetdash{}{0pt}%
\pgfpathmoveto{\pgfqpoint{1.462584in}{2.009504in}}%
\pgfpathcurveto{\pgfqpoint{1.473634in}{2.009504in}}{\pgfqpoint{1.484233in}{2.013894in}}{\pgfqpoint{1.492047in}{2.021708in}}%
\pgfpathcurveto{\pgfqpoint{1.499861in}{2.029522in}}{\pgfqpoint{1.504251in}{2.040121in}}{\pgfqpoint{1.504251in}{2.051171in}}%
\pgfpathcurveto{\pgfqpoint{1.504251in}{2.062221in}}{\pgfqpoint{1.499861in}{2.072820in}}{\pgfqpoint{1.492047in}{2.080634in}}%
\pgfpathcurveto{\pgfqpoint{1.484233in}{2.088447in}}{\pgfqpoint{1.473634in}{2.092837in}}{\pgfqpoint{1.462584in}{2.092837in}}%
\pgfpathcurveto{\pgfqpoint{1.451534in}{2.092837in}}{\pgfqpoint{1.440935in}{2.088447in}}{\pgfqpoint{1.433121in}{2.080634in}}%
\pgfpathcurveto{\pgfqpoint{1.425308in}{2.072820in}}{\pgfqpoint{1.420917in}{2.062221in}}{\pgfqpoint{1.420917in}{2.051171in}}%
\pgfpathcurveto{\pgfqpoint{1.420917in}{2.040121in}}{\pgfqpoint{1.425308in}{2.029522in}}{\pgfqpoint{1.433121in}{2.021708in}}%
\pgfpathcurveto{\pgfqpoint{1.440935in}{2.013894in}}{\pgfqpoint{1.451534in}{2.009504in}}{\pgfqpoint{1.462584in}{2.009504in}}%
\pgfpathclose%
\pgfusepath{stroke,fill}%
\end{pgfscope}%
\begin{pgfscope}%
\pgfpathrectangle{\pgfqpoint{0.600000in}{0.600000in}}{\pgfqpoint{3.900000in}{3.900000in}}%
\pgfusepath{clip}%
\pgfsetbuttcap%
\pgfsetroundjoin%
\definecolor{currentfill}{rgb}{0.121569,0.466667,0.705882}%
\pgfsetfillcolor{currentfill}%
\pgfsetlinewidth{1.003750pt}%
\definecolor{currentstroke}{rgb}{0.121569,0.466667,0.705882}%
\pgfsetstrokecolor{currentstroke}%
\pgfsetdash{}{0pt}%
\pgfpathmoveto{\pgfqpoint{2.492606in}{4.017793in}}%
\pgfpathcurveto{\pgfqpoint{2.503656in}{4.017793in}}{\pgfqpoint{2.514255in}{4.022184in}}{\pgfqpoint{2.522069in}{4.029997in}}%
\pgfpathcurveto{\pgfqpoint{2.529882in}{4.037811in}}{\pgfqpoint{2.534273in}{4.048410in}}{\pgfqpoint{2.534273in}{4.059460in}}%
\pgfpathcurveto{\pgfqpoint{2.534273in}{4.070510in}}{\pgfqpoint{2.529882in}{4.081109in}}{\pgfqpoint{2.522069in}{4.088923in}}%
\pgfpathcurveto{\pgfqpoint{2.514255in}{4.096736in}}{\pgfqpoint{2.503656in}{4.101127in}}{\pgfqpoint{2.492606in}{4.101127in}}%
\pgfpathcurveto{\pgfqpoint{2.481556in}{4.101127in}}{\pgfqpoint{2.470957in}{4.096736in}}{\pgfqpoint{2.463143in}{4.088923in}}%
\pgfpathcurveto{\pgfqpoint{2.455330in}{4.081109in}}{\pgfqpoint{2.450939in}{4.070510in}}{\pgfqpoint{2.450939in}{4.059460in}}%
\pgfpathcurveto{\pgfqpoint{2.450939in}{4.048410in}}{\pgfqpoint{2.455330in}{4.037811in}}{\pgfqpoint{2.463143in}{4.029997in}}%
\pgfpathcurveto{\pgfqpoint{2.470957in}{4.022184in}}{\pgfqpoint{2.481556in}{4.017793in}}{\pgfqpoint{2.492606in}{4.017793in}}%
\pgfpathclose%
\pgfusepath{stroke,fill}%
\end{pgfscope}%
\begin{pgfscope}%
\pgfpathrectangle{\pgfqpoint{0.600000in}{0.600000in}}{\pgfqpoint{3.900000in}{3.900000in}}%
\pgfusepath{clip}%
\pgfsetbuttcap%
\pgfsetroundjoin%
\definecolor{currentfill}{rgb}{0.121569,0.466667,0.705882}%
\pgfsetfillcolor{currentfill}%
\pgfsetlinewidth{1.003750pt}%
\definecolor{currentstroke}{rgb}{0.121569,0.466667,0.705882}%
\pgfsetstrokecolor{currentstroke}%
\pgfsetdash{}{0pt}%
\pgfpathmoveto{\pgfqpoint{2.804012in}{1.894448in}}%
\pgfpathcurveto{\pgfqpoint{2.815062in}{1.894448in}}{\pgfqpoint{2.825661in}{1.898839in}}{\pgfqpoint{2.833475in}{1.906652in}}%
\pgfpathcurveto{\pgfqpoint{2.841288in}{1.914466in}}{\pgfqpoint{2.845679in}{1.925065in}}{\pgfqpoint{2.845679in}{1.936115in}}%
\pgfpathcurveto{\pgfqpoint{2.845679in}{1.947165in}}{\pgfqpoint{2.841288in}{1.957764in}}{\pgfqpoint{2.833475in}{1.965578in}}%
\pgfpathcurveto{\pgfqpoint{2.825661in}{1.973391in}}{\pgfqpoint{2.815062in}{1.977782in}}{\pgfqpoint{2.804012in}{1.977782in}}%
\pgfpathcurveto{\pgfqpoint{2.792962in}{1.977782in}}{\pgfqpoint{2.782363in}{1.973391in}}{\pgfqpoint{2.774549in}{1.965578in}}%
\pgfpathcurveto{\pgfqpoint{2.766736in}{1.957764in}}{\pgfqpoint{2.762345in}{1.947165in}}{\pgfqpoint{2.762345in}{1.936115in}}%
\pgfpathcurveto{\pgfqpoint{2.762345in}{1.925065in}}{\pgfqpoint{2.766736in}{1.914466in}}{\pgfqpoint{2.774549in}{1.906652in}}%
\pgfpathcurveto{\pgfqpoint{2.782363in}{1.898839in}}{\pgfqpoint{2.792962in}{1.894448in}}{\pgfqpoint{2.804012in}{1.894448in}}%
\pgfpathclose%
\pgfusepath{stroke,fill}%
\end{pgfscope}%
\begin{pgfscope}%
\pgfpathrectangle{\pgfqpoint{0.600000in}{0.600000in}}{\pgfqpoint{3.900000in}{3.900000in}}%
\pgfusepath{clip}%
\pgfsetbuttcap%
\pgfsetroundjoin%
\definecolor{currentfill}{rgb}{0.121569,0.466667,0.705882}%
\pgfsetfillcolor{currentfill}%
\pgfsetlinewidth{1.003750pt}%
\definecolor{currentstroke}{rgb}{0.121569,0.466667,0.705882}%
\pgfsetstrokecolor{currentstroke}%
\pgfsetdash{}{0pt}%
\pgfpathmoveto{\pgfqpoint{2.374861in}{2.180582in}}%
\pgfpathcurveto{\pgfqpoint{2.385911in}{2.180582in}}{\pgfqpoint{2.396510in}{2.184972in}}{\pgfqpoint{2.404324in}{2.192786in}}%
\pgfpathcurveto{\pgfqpoint{2.412137in}{2.200599in}}{\pgfqpoint{2.416528in}{2.211198in}}{\pgfqpoint{2.416528in}{2.222248in}}%
\pgfpathcurveto{\pgfqpoint{2.416528in}{2.233299in}}{\pgfqpoint{2.412137in}{2.243898in}}{\pgfqpoint{2.404324in}{2.251711in}}%
\pgfpathcurveto{\pgfqpoint{2.396510in}{2.259525in}}{\pgfqpoint{2.385911in}{2.263915in}}{\pgfqpoint{2.374861in}{2.263915in}}%
\pgfpathcurveto{\pgfqpoint{2.363811in}{2.263915in}}{\pgfqpoint{2.353212in}{2.259525in}}{\pgfqpoint{2.345398in}{2.251711in}}%
\pgfpathcurveto{\pgfqpoint{2.337585in}{2.243898in}}{\pgfqpoint{2.333194in}{2.233299in}}{\pgfqpoint{2.333194in}{2.222248in}}%
\pgfpathcurveto{\pgfqpoint{2.333194in}{2.211198in}}{\pgfqpoint{2.337585in}{2.200599in}}{\pgfqpoint{2.345398in}{2.192786in}}%
\pgfpathcurveto{\pgfqpoint{2.353212in}{2.184972in}}{\pgfqpoint{2.363811in}{2.180582in}}{\pgfqpoint{2.374861in}{2.180582in}}%
\pgfpathclose%
\pgfusepath{stroke,fill}%
\end{pgfscope}%
\begin{pgfscope}%
\pgfpathrectangle{\pgfqpoint{0.600000in}{0.600000in}}{\pgfqpoint{3.900000in}{3.900000in}}%
\pgfusepath{clip}%
\pgfsetbuttcap%
\pgfsetroundjoin%
\definecolor{currentfill}{rgb}{0.121569,0.466667,0.705882}%
\pgfsetfillcolor{currentfill}%
\pgfsetlinewidth{1.003750pt}%
\definecolor{currentstroke}{rgb}{0.121569,0.466667,0.705882}%
\pgfsetstrokecolor{currentstroke}%
\pgfsetdash{}{0pt}%
\pgfpathmoveto{\pgfqpoint{3.244392in}{2.828717in}}%
\pgfpathcurveto{\pgfqpoint{3.255442in}{2.828717in}}{\pgfqpoint{3.266041in}{2.833107in}}{\pgfqpoint{3.273854in}{2.840921in}}%
\pgfpathcurveto{\pgfqpoint{3.281668in}{2.848734in}}{\pgfqpoint{3.286058in}{2.859333in}}{\pgfqpoint{3.286058in}{2.870384in}}%
\pgfpathcurveto{\pgfqpoint{3.286058in}{2.881434in}}{\pgfqpoint{3.281668in}{2.892033in}}{\pgfqpoint{3.273854in}{2.899846in}}%
\pgfpathcurveto{\pgfqpoint{3.266041in}{2.907660in}}{\pgfqpoint{3.255442in}{2.912050in}}{\pgfqpoint{3.244392in}{2.912050in}}%
\pgfpathcurveto{\pgfqpoint{3.233341in}{2.912050in}}{\pgfqpoint{3.222742in}{2.907660in}}{\pgfqpoint{3.214929in}{2.899846in}}%
\pgfpathcurveto{\pgfqpoint{3.207115in}{2.892033in}}{\pgfqpoint{3.202725in}{2.881434in}}{\pgfqpoint{3.202725in}{2.870384in}}%
\pgfpathcurveto{\pgfqpoint{3.202725in}{2.859333in}}{\pgfqpoint{3.207115in}{2.848734in}}{\pgfqpoint{3.214929in}{2.840921in}}%
\pgfpathcurveto{\pgfqpoint{3.222742in}{2.833107in}}{\pgfqpoint{3.233341in}{2.828717in}}{\pgfqpoint{3.244392in}{2.828717in}}%
\pgfpathclose%
\pgfusepath{stroke,fill}%
\end{pgfscope}%
\begin{pgfscope}%
\pgfpathrectangle{\pgfqpoint{0.600000in}{0.600000in}}{\pgfqpoint{3.900000in}{3.900000in}}%
\pgfusepath{clip}%
\pgfsetbuttcap%
\pgfsetroundjoin%
\definecolor{currentfill}{rgb}{0.121569,0.466667,0.705882}%
\pgfsetfillcolor{currentfill}%
\pgfsetlinewidth{1.003750pt}%
\definecolor{currentstroke}{rgb}{0.121569,0.466667,0.705882}%
\pgfsetstrokecolor{currentstroke}%
\pgfsetdash{}{0pt}%
\pgfpathmoveto{\pgfqpoint{2.696430in}{2.143164in}}%
\pgfpathcurveto{\pgfqpoint{2.707480in}{2.143164in}}{\pgfqpoint{2.718079in}{2.147554in}}{\pgfqpoint{2.725893in}{2.155368in}}%
\pgfpathcurveto{\pgfqpoint{2.733707in}{2.163181in}}{\pgfqpoint{2.738097in}{2.173780in}}{\pgfqpoint{2.738097in}{2.184831in}}%
\pgfpathcurveto{\pgfqpoint{2.738097in}{2.195881in}}{\pgfqpoint{2.733707in}{2.206480in}}{\pgfqpoint{2.725893in}{2.214293in}}%
\pgfpathcurveto{\pgfqpoint{2.718079in}{2.222107in}}{\pgfqpoint{2.707480in}{2.226497in}}{\pgfqpoint{2.696430in}{2.226497in}}%
\pgfpathcurveto{\pgfqpoint{2.685380in}{2.226497in}}{\pgfqpoint{2.674781in}{2.222107in}}{\pgfqpoint{2.666967in}{2.214293in}}%
\pgfpathcurveto{\pgfqpoint{2.659154in}{2.206480in}}{\pgfqpoint{2.654764in}{2.195881in}}{\pgfqpoint{2.654764in}{2.184831in}}%
\pgfpathcurveto{\pgfqpoint{2.654764in}{2.173780in}}{\pgfqpoint{2.659154in}{2.163181in}}{\pgfqpoint{2.666967in}{2.155368in}}%
\pgfpathcurveto{\pgfqpoint{2.674781in}{2.147554in}}{\pgfqpoint{2.685380in}{2.143164in}}{\pgfqpoint{2.696430in}{2.143164in}}%
\pgfpathclose%
\pgfusepath{stroke,fill}%
\end{pgfscope}%
\begin{pgfscope}%
\pgfpathrectangle{\pgfqpoint{0.600000in}{0.600000in}}{\pgfqpoint{3.900000in}{3.900000in}}%
\pgfusepath{clip}%
\pgfsetbuttcap%
\pgfsetroundjoin%
\definecolor{currentfill}{rgb}{0.121569,0.466667,0.705882}%
\pgfsetfillcolor{currentfill}%
\pgfsetlinewidth{1.003750pt}%
\definecolor{currentstroke}{rgb}{0.121569,0.466667,0.705882}%
\pgfsetstrokecolor{currentstroke}%
\pgfsetdash{}{0pt}%
\pgfpathmoveto{\pgfqpoint{2.600932in}{2.499050in}}%
\pgfpathcurveto{\pgfqpoint{2.611982in}{2.499050in}}{\pgfqpoint{2.622581in}{2.503440in}}{\pgfqpoint{2.630395in}{2.511254in}}%
\pgfpathcurveto{\pgfqpoint{2.638209in}{2.519068in}}{\pgfqpoint{2.642599in}{2.529667in}}{\pgfqpoint{2.642599in}{2.540717in}}%
\pgfpathcurveto{\pgfqpoint{2.642599in}{2.551767in}}{\pgfqpoint{2.638209in}{2.562366in}}{\pgfqpoint{2.630395in}{2.570180in}}%
\pgfpathcurveto{\pgfqpoint{2.622581in}{2.577993in}}{\pgfqpoint{2.611982in}{2.582383in}}{\pgfqpoint{2.600932in}{2.582383in}}%
\pgfpathcurveto{\pgfqpoint{2.589882in}{2.582383in}}{\pgfqpoint{2.579283in}{2.577993in}}{\pgfqpoint{2.571469in}{2.570180in}}%
\pgfpathcurveto{\pgfqpoint{2.563656in}{2.562366in}}{\pgfqpoint{2.559266in}{2.551767in}}{\pgfqpoint{2.559266in}{2.540717in}}%
\pgfpathcurveto{\pgfqpoint{2.559266in}{2.529667in}}{\pgfqpoint{2.563656in}{2.519068in}}{\pgfqpoint{2.571469in}{2.511254in}}%
\pgfpathcurveto{\pgfqpoint{2.579283in}{2.503440in}}{\pgfqpoint{2.589882in}{2.499050in}}{\pgfqpoint{2.600932in}{2.499050in}}%
\pgfpathclose%
\pgfusepath{stroke,fill}%
\end{pgfscope}%
\begin{pgfscope}%
\pgfpathrectangle{\pgfqpoint{0.600000in}{0.600000in}}{\pgfqpoint{3.900000in}{3.900000in}}%
\pgfusepath{clip}%
\pgfsetbuttcap%
\pgfsetroundjoin%
\definecolor{currentfill}{rgb}{0.121569,0.466667,0.705882}%
\pgfsetfillcolor{currentfill}%
\pgfsetlinewidth{1.003750pt}%
\definecolor{currentstroke}{rgb}{0.121569,0.466667,0.705882}%
\pgfsetstrokecolor{currentstroke}%
\pgfsetdash{}{0pt}%
\pgfpathmoveto{\pgfqpoint{2.691990in}{2.487353in}}%
\pgfpathcurveto{\pgfqpoint{2.703041in}{2.487353in}}{\pgfqpoint{2.713640in}{2.491743in}}{\pgfqpoint{2.721453in}{2.499556in}}%
\pgfpathcurveto{\pgfqpoint{2.729267in}{2.507370in}}{\pgfqpoint{2.733657in}{2.517969in}}{\pgfqpoint{2.733657in}{2.529019in}}%
\pgfpathcurveto{\pgfqpoint{2.733657in}{2.540069in}}{\pgfqpoint{2.729267in}{2.550668in}}{\pgfqpoint{2.721453in}{2.558482in}}%
\pgfpathcurveto{\pgfqpoint{2.713640in}{2.566296in}}{\pgfqpoint{2.703041in}{2.570686in}}{\pgfqpoint{2.691990in}{2.570686in}}%
\pgfpathcurveto{\pgfqpoint{2.680940in}{2.570686in}}{\pgfqpoint{2.670341in}{2.566296in}}{\pgfqpoint{2.662528in}{2.558482in}}%
\pgfpathcurveto{\pgfqpoint{2.654714in}{2.550668in}}{\pgfqpoint{2.650324in}{2.540069in}}{\pgfqpoint{2.650324in}{2.529019in}}%
\pgfpathcurveto{\pgfqpoint{2.650324in}{2.517969in}}{\pgfqpoint{2.654714in}{2.507370in}}{\pgfqpoint{2.662528in}{2.499556in}}%
\pgfpathcurveto{\pgfqpoint{2.670341in}{2.491743in}}{\pgfqpoint{2.680940in}{2.487353in}}{\pgfqpoint{2.691990in}{2.487353in}}%
\pgfpathclose%
\pgfusepath{stroke,fill}%
\end{pgfscope}%
\begin{pgfscope}%
\pgfpathrectangle{\pgfqpoint{0.600000in}{0.600000in}}{\pgfqpoint{3.900000in}{3.900000in}}%
\pgfusepath{clip}%
\pgfsetbuttcap%
\pgfsetroundjoin%
\definecolor{currentfill}{rgb}{0.121569,0.466667,0.705882}%
\pgfsetfillcolor{currentfill}%
\pgfsetlinewidth{1.003750pt}%
\definecolor{currentstroke}{rgb}{0.121569,0.466667,0.705882}%
\pgfsetstrokecolor{currentstroke}%
\pgfsetdash{}{0pt}%
\pgfpathmoveto{\pgfqpoint{2.830932in}{2.136189in}}%
\pgfpathcurveto{\pgfqpoint{2.841982in}{2.136189in}}{\pgfqpoint{2.852581in}{2.140579in}}{\pgfqpoint{2.860394in}{2.148393in}}%
\pgfpathcurveto{\pgfqpoint{2.868208in}{2.156207in}}{\pgfqpoint{2.872598in}{2.166806in}}{\pgfqpoint{2.872598in}{2.177856in}}%
\pgfpathcurveto{\pgfqpoint{2.872598in}{2.188906in}}{\pgfqpoint{2.868208in}{2.199505in}}{\pgfqpoint{2.860394in}{2.207319in}}%
\pgfpathcurveto{\pgfqpoint{2.852581in}{2.215132in}}{\pgfqpoint{2.841982in}{2.219523in}}{\pgfqpoint{2.830932in}{2.219523in}}%
\pgfpathcurveto{\pgfqpoint{2.819882in}{2.219523in}}{\pgfqpoint{2.809283in}{2.215132in}}{\pgfqpoint{2.801469in}{2.207319in}}%
\pgfpathcurveto{\pgfqpoint{2.793655in}{2.199505in}}{\pgfqpoint{2.789265in}{2.188906in}}{\pgfqpoint{2.789265in}{2.177856in}}%
\pgfpathcurveto{\pgfqpoint{2.789265in}{2.166806in}}{\pgfqpoint{2.793655in}{2.156207in}}{\pgfqpoint{2.801469in}{2.148393in}}%
\pgfpathcurveto{\pgfqpoint{2.809283in}{2.140579in}}{\pgfqpoint{2.819882in}{2.136189in}}{\pgfqpoint{2.830932in}{2.136189in}}%
\pgfpathclose%
\pgfusepath{stroke,fill}%
\end{pgfscope}%
\begin{pgfscope}%
\pgfpathrectangle{\pgfqpoint{0.600000in}{0.600000in}}{\pgfqpoint{3.900000in}{3.900000in}}%
\pgfusepath{clip}%
\pgfsetbuttcap%
\pgfsetroundjoin%
\definecolor{currentfill}{rgb}{0.121569,0.466667,0.705882}%
\pgfsetfillcolor{currentfill}%
\pgfsetlinewidth{1.003750pt}%
\definecolor{currentstroke}{rgb}{0.121569,0.466667,0.705882}%
\pgfsetstrokecolor{currentstroke}%
\pgfsetdash{}{0pt}%
\pgfpathmoveto{\pgfqpoint{2.360143in}{2.347428in}}%
\pgfpathcurveto{\pgfqpoint{2.371193in}{2.347428in}}{\pgfqpoint{2.381792in}{2.351818in}}{\pgfqpoint{2.389606in}{2.359632in}}%
\pgfpathcurveto{\pgfqpoint{2.397420in}{2.367445in}}{\pgfqpoint{2.401810in}{2.378044in}}{\pgfqpoint{2.401810in}{2.389094in}}%
\pgfpathcurveto{\pgfqpoint{2.401810in}{2.400145in}}{\pgfqpoint{2.397420in}{2.410744in}}{\pgfqpoint{2.389606in}{2.418557in}}%
\pgfpathcurveto{\pgfqpoint{2.381792in}{2.426371in}}{\pgfqpoint{2.371193in}{2.430761in}}{\pgfqpoint{2.360143in}{2.430761in}}%
\pgfpathcurveto{\pgfqpoint{2.349093in}{2.430761in}}{\pgfqpoint{2.338494in}{2.426371in}}{\pgfqpoint{2.330680in}{2.418557in}}%
\pgfpathcurveto{\pgfqpoint{2.322867in}{2.410744in}}{\pgfqpoint{2.318477in}{2.400145in}}{\pgfqpoint{2.318477in}{2.389094in}}%
\pgfpathcurveto{\pgfqpoint{2.318477in}{2.378044in}}{\pgfqpoint{2.322867in}{2.367445in}}{\pgfqpoint{2.330680in}{2.359632in}}%
\pgfpathcurveto{\pgfqpoint{2.338494in}{2.351818in}}{\pgfqpoint{2.349093in}{2.347428in}}{\pgfqpoint{2.360143in}{2.347428in}}%
\pgfpathclose%
\pgfusepath{stroke,fill}%
\end{pgfscope}%
\begin{pgfscope}%
\pgfpathrectangle{\pgfqpoint{0.600000in}{0.600000in}}{\pgfqpoint{3.900000in}{3.900000in}}%
\pgfusepath{clip}%
\pgfsetbuttcap%
\pgfsetroundjoin%
\definecolor{currentfill}{rgb}{0.121569,0.466667,0.705882}%
\pgfsetfillcolor{currentfill}%
\pgfsetlinewidth{1.003750pt}%
\definecolor{currentstroke}{rgb}{0.121569,0.466667,0.705882}%
\pgfsetstrokecolor{currentstroke}%
\pgfsetdash{}{0pt}%
\pgfpathmoveto{\pgfqpoint{2.956523in}{1.707627in}}%
\pgfpathcurveto{\pgfqpoint{2.967574in}{1.707627in}}{\pgfqpoint{2.978173in}{1.712018in}}{\pgfqpoint{2.985986in}{1.719831in}}%
\pgfpathcurveto{\pgfqpoint{2.993800in}{1.727645in}}{\pgfqpoint{2.998190in}{1.738244in}}{\pgfqpoint{2.998190in}{1.749294in}}%
\pgfpathcurveto{\pgfqpoint{2.998190in}{1.760344in}}{\pgfqpoint{2.993800in}{1.770943in}}{\pgfqpoint{2.985986in}{1.778757in}}%
\pgfpathcurveto{\pgfqpoint{2.978173in}{1.786570in}}{\pgfqpoint{2.967574in}{1.790961in}}{\pgfqpoint{2.956523in}{1.790961in}}%
\pgfpathcurveto{\pgfqpoint{2.945473in}{1.790961in}}{\pgfqpoint{2.934874in}{1.786570in}}{\pgfqpoint{2.927061in}{1.778757in}}%
\pgfpathcurveto{\pgfqpoint{2.919247in}{1.770943in}}{\pgfqpoint{2.914857in}{1.760344in}}{\pgfqpoint{2.914857in}{1.749294in}}%
\pgfpathcurveto{\pgfqpoint{2.914857in}{1.738244in}}{\pgfqpoint{2.919247in}{1.727645in}}{\pgfqpoint{2.927061in}{1.719831in}}%
\pgfpathcurveto{\pgfqpoint{2.934874in}{1.712018in}}{\pgfqpoint{2.945473in}{1.707627in}}{\pgfqpoint{2.956523in}{1.707627in}}%
\pgfpathclose%
\pgfusepath{stroke,fill}%
\end{pgfscope}%
\begin{pgfscope}%
\pgfpathrectangle{\pgfqpoint{0.600000in}{0.600000in}}{\pgfqpoint{3.900000in}{3.900000in}}%
\pgfusepath{clip}%
\pgfsetbuttcap%
\pgfsetroundjoin%
\definecolor{currentfill}{rgb}{0.121569,0.466667,0.705882}%
\pgfsetfillcolor{currentfill}%
\pgfsetlinewidth{1.003750pt}%
\definecolor{currentstroke}{rgb}{0.121569,0.466667,0.705882}%
\pgfsetstrokecolor{currentstroke}%
\pgfsetdash{}{0pt}%
\pgfpathmoveto{\pgfqpoint{2.882958in}{2.744763in}}%
\pgfpathcurveto{\pgfqpoint{2.894008in}{2.744763in}}{\pgfqpoint{2.904607in}{2.749153in}}{\pgfqpoint{2.912421in}{2.756967in}}%
\pgfpathcurveto{\pgfqpoint{2.920235in}{2.764781in}}{\pgfqpoint{2.924625in}{2.775380in}}{\pgfqpoint{2.924625in}{2.786430in}}%
\pgfpathcurveto{\pgfqpoint{2.924625in}{2.797480in}}{\pgfqpoint{2.920235in}{2.808079in}}{\pgfqpoint{2.912421in}{2.815893in}}%
\pgfpathcurveto{\pgfqpoint{2.904607in}{2.823706in}}{\pgfqpoint{2.894008in}{2.828097in}}{\pgfqpoint{2.882958in}{2.828097in}}%
\pgfpathcurveto{\pgfqpoint{2.871908in}{2.828097in}}{\pgfqpoint{2.861309in}{2.823706in}}{\pgfqpoint{2.853495in}{2.815893in}}%
\pgfpathcurveto{\pgfqpoint{2.845682in}{2.808079in}}{\pgfqpoint{2.841292in}{2.797480in}}{\pgfqpoint{2.841292in}{2.786430in}}%
\pgfpathcurveto{\pgfqpoint{2.841292in}{2.775380in}}{\pgfqpoint{2.845682in}{2.764781in}}{\pgfqpoint{2.853495in}{2.756967in}}%
\pgfpathcurveto{\pgfqpoint{2.861309in}{2.749153in}}{\pgfqpoint{2.871908in}{2.744763in}}{\pgfqpoint{2.882958in}{2.744763in}}%
\pgfpathclose%
\pgfusepath{stroke,fill}%
\end{pgfscope}%
\begin{pgfscope}%
\pgfpathrectangle{\pgfqpoint{0.600000in}{0.600000in}}{\pgfqpoint{3.900000in}{3.900000in}}%
\pgfusepath{clip}%
\pgfsetbuttcap%
\pgfsetroundjoin%
\definecolor{currentfill}{rgb}{0.121569,0.466667,0.705882}%
\pgfsetfillcolor{currentfill}%
\pgfsetlinewidth{1.003750pt}%
\definecolor{currentstroke}{rgb}{0.121569,0.466667,0.705882}%
\pgfsetstrokecolor{currentstroke}%
\pgfsetdash{}{0pt}%
\pgfpathmoveto{\pgfqpoint{3.008964in}{2.261472in}}%
\pgfpathcurveto{\pgfqpoint{3.020014in}{2.261472in}}{\pgfqpoint{3.030613in}{2.265863in}}{\pgfqpoint{3.038427in}{2.273676in}}%
\pgfpathcurveto{\pgfqpoint{3.046240in}{2.281490in}}{\pgfqpoint{3.050631in}{2.292089in}}{\pgfqpoint{3.050631in}{2.303139in}}%
\pgfpathcurveto{\pgfqpoint{3.050631in}{2.314189in}}{\pgfqpoint{3.046240in}{2.324788in}}{\pgfqpoint{3.038427in}{2.332602in}}%
\pgfpathcurveto{\pgfqpoint{3.030613in}{2.340415in}}{\pgfqpoint{3.020014in}{2.344806in}}{\pgfqpoint{3.008964in}{2.344806in}}%
\pgfpathcurveto{\pgfqpoint{2.997914in}{2.344806in}}{\pgfqpoint{2.987315in}{2.340415in}}{\pgfqpoint{2.979501in}{2.332602in}}%
\pgfpathcurveto{\pgfqpoint{2.971688in}{2.324788in}}{\pgfqpoint{2.967297in}{2.314189in}}{\pgfqpoint{2.967297in}{2.303139in}}%
\pgfpathcurveto{\pgfqpoint{2.967297in}{2.292089in}}{\pgfqpoint{2.971688in}{2.281490in}}{\pgfqpoint{2.979501in}{2.273676in}}%
\pgfpathcurveto{\pgfqpoint{2.987315in}{2.265863in}}{\pgfqpoint{2.997914in}{2.261472in}}{\pgfqpoint{3.008964in}{2.261472in}}%
\pgfpathclose%
\pgfusepath{stroke,fill}%
\end{pgfscope}%
\begin{pgfscope}%
\pgfpathrectangle{\pgfqpoint{0.600000in}{0.600000in}}{\pgfqpoint{3.900000in}{3.900000in}}%
\pgfusepath{clip}%
\pgfsetbuttcap%
\pgfsetroundjoin%
\definecolor{currentfill}{rgb}{0.121569,0.466667,0.705882}%
\pgfsetfillcolor{currentfill}%
\pgfsetlinewidth{1.003750pt}%
\definecolor{currentstroke}{rgb}{0.121569,0.466667,0.705882}%
\pgfsetstrokecolor{currentstroke}%
\pgfsetdash{}{0pt}%
\pgfpathmoveto{\pgfqpoint{2.549380in}{2.154897in}}%
\pgfpathcurveto{\pgfqpoint{2.560430in}{2.154897in}}{\pgfqpoint{2.571029in}{2.159287in}}{\pgfqpoint{2.578843in}{2.167101in}}%
\pgfpathcurveto{\pgfqpoint{2.586656in}{2.174914in}}{\pgfqpoint{2.591047in}{2.185513in}}{\pgfqpoint{2.591047in}{2.196564in}}%
\pgfpathcurveto{\pgfqpoint{2.591047in}{2.207614in}}{\pgfqpoint{2.586656in}{2.218213in}}{\pgfqpoint{2.578843in}{2.226026in}}%
\pgfpathcurveto{\pgfqpoint{2.571029in}{2.233840in}}{\pgfqpoint{2.560430in}{2.238230in}}{\pgfqpoint{2.549380in}{2.238230in}}%
\pgfpathcurveto{\pgfqpoint{2.538330in}{2.238230in}}{\pgfqpoint{2.527731in}{2.233840in}}{\pgfqpoint{2.519917in}{2.226026in}}%
\pgfpathcurveto{\pgfqpoint{2.512104in}{2.218213in}}{\pgfqpoint{2.507713in}{2.207614in}}{\pgfqpoint{2.507713in}{2.196564in}}%
\pgfpathcurveto{\pgfqpoint{2.507713in}{2.185513in}}{\pgfqpoint{2.512104in}{2.174914in}}{\pgfqpoint{2.519917in}{2.167101in}}%
\pgfpathcurveto{\pgfqpoint{2.527731in}{2.159287in}}{\pgfqpoint{2.538330in}{2.154897in}}{\pgfqpoint{2.549380in}{2.154897in}}%
\pgfpathclose%
\pgfusepath{stroke,fill}%
\end{pgfscope}%
\begin{pgfscope}%
\pgfpathrectangle{\pgfqpoint{0.600000in}{0.600000in}}{\pgfqpoint{3.900000in}{3.900000in}}%
\pgfusepath{clip}%
\pgfsetbuttcap%
\pgfsetroundjoin%
\definecolor{currentfill}{rgb}{0.121569,0.466667,0.705882}%
\pgfsetfillcolor{currentfill}%
\pgfsetlinewidth{1.003750pt}%
\definecolor{currentstroke}{rgb}{0.121569,0.466667,0.705882}%
\pgfsetstrokecolor{currentstroke}%
\pgfsetdash{}{0pt}%
\pgfpathmoveto{\pgfqpoint{2.374100in}{1.933828in}}%
\pgfpathcurveto{\pgfqpoint{2.385150in}{1.933828in}}{\pgfqpoint{2.395749in}{1.938219in}}{\pgfqpoint{2.403563in}{1.946032in}}%
\pgfpathcurveto{\pgfqpoint{2.411376in}{1.953846in}}{\pgfqpoint{2.415767in}{1.964445in}}{\pgfqpoint{2.415767in}{1.975495in}}%
\pgfpathcurveto{\pgfqpoint{2.415767in}{1.986545in}}{\pgfqpoint{2.411376in}{1.997144in}}{\pgfqpoint{2.403563in}{2.004958in}}%
\pgfpathcurveto{\pgfqpoint{2.395749in}{2.012771in}}{\pgfqpoint{2.385150in}{2.017162in}}{\pgfqpoint{2.374100in}{2.017162in}}%
\pgfpathcurveto{\pgfqpoint{2.363050in}{2.017162in}}{\pgfqpoint{2.352451in}{2.012771in}}{\pgfqpoint{2.344637in}{2.004958in}}%
\pgfpathcurveto{\pgfqpoint{2.336823in}{1.997144in}}{\pgfqpoint{2.332433in}{1.986545in}}{\pgfqpoint{2.332433in}{1.975495in}}%
\pgfpathcurveto{\pgfqpoint{2.332433in}{1.964445in}}{\pgfqpoint{2.336823in}{1.953846in}}{\pgfqpoint{2.344637in}{1.946032in}}%
\pgfpathcurveto{\pgfqpoint{2.352451in}{1.938219in}}{\pgfqpoint{2.363050in}{1.933828in}}{\pgfqpoint{2.374100in}{1.933828in}}%
\pgfpathclose%
\pgfusepath{stroke,fill}%
\end{pgfscope}%
\begin{pgfscope}%
\pgfpathrectangle{\pgfqpoint{0.600000in}{0.600000in}}{\pgfqpoint{3.900000in}{3.900000in}}%
\pgfusepath{clip}%
\pgfsetbuttcap%
\pgfsetroundjoin%
\definecolor{currentfill}{rgb}{0.121569,0.466667,0.705882}%
\pgfsetfillcolor{currentfill}%
\pgfsetlinewidth{1.003750pt}%
\definecolor{currentstroke}{rgb}{0.121569,0.466667,0.705882}%
\pgfsetstrokecolor{currentstroke}%
\pgfsetdash{}{0pt}%
\pgfpathmoveto{\pgfqpoint{2.408041in}{3.857752in}}%
\pgfpathcurveto{\pgfqpoint{2.419091in}{3.857752in}}{\pgfqpoint{2.429690in}{3.862142in}}{\pgfqpoint{2.437504in}{3.869956in}}%
\pgfpathcurveto{\pgfqpoint{2.445318in}{3.877770in}}{\pgfqpoint{2.449708in}{3.888369in}}{\pgfqpoint{2.449708in}{3.899419in}}%
\pgfpathcurveto{\pgfqpoint{2.449708in}{3.910469in}}{\pgfqpoint{2.445318in}{3.921068in}}{\pgfqpoint{2.437504in}{3.928881in}}%
\pgfpathcurveto{\pgfqpoint{2.429690in}{3.936695in}}{\pgfqpoint{2.419091in}{3.941085in}}{\pgfqpoint{2.408041in}{3.941085in}}%
\pgfpathcurveto{\pgfqpoint{2.396991in}{3.941085in}}{\pgfqpoint{2.386392in}{3.936695in}}{\pgfqpoint{2.378579in}{3.928881in}}%
\pgfpathcurveto{\pgfqpoint{2.370765in}{3.921068in}}{\pgfqpoint{2.366375in}{3.910469in}}{\pgfqpoint{2.366375in}{3.899419in}}%
\pgfpathcurveto{\pgfqpoint{2.366375in}{3.888369in}}{\pgfqpoint{2.370765in}{3.877770in}}{\pgfqpoint{2.378579in}{3.869956in}}%
\pgfpathcurveto{\pgfqpoint{2.386392in}{3.862142in}}{\pgfqpoint{2.396991in}{3.857752in}}{\pgfqpoint{2.408041in}{3.857752in}}%
\pgfpathclose%
\pgfusepath{stroke,fill}%
\end{pgfscope}%
\begin{pgfscope}%
\pgfpathrectangle{\pgfqpoint{0.600000in}{0.600000in}}{\pgfqpoint{3.900000in}{3.900000in}}%
\pgfusepath{clip}%
\pgfsetbuttcap%
\pgfsetroundjoin%
\definecolor{currentfill}{rgb}{0.121569,0.466667,0.705882}%
\pgfsetfillcolor{currentfill}%
\pgfsetlinewidth{1.003750pt}%
\definecolor{currentstroke}{rgb}{0.121569,0.466667,0.705882}%
\pgfsetstrokecolor{currentstroke}%
\pgfsetdash{}{0pt}%
\pgfpathmoveto{\pgfqpoint{2.309648in}{1.803187in}}%
\pgfpathcurveto{\pgfqpoint{2.320698in}{1.803187in}}{\pgfqpoint{2.331297in}{1.807578in}}{\pgfqpoint{2.339111in}{1.815391in}}%
\pgfpathcurveto{\pgfqpoint{2.346925in}{1.823205in}}{\pgfqpoint{2.351315in}{1.833804in}}{\pgfqpoint{2.351315in}{1.844854in}}%
\pgfpathcurveto{\pgfqpoint{2.351315in}{1.855904in}}{\pgfqpoint{2.346925in}{1.866503in}}{\pgfqpoint{2.339111in}{1.874317in}}%
\pgfpathcurveto{\pgfqpoint{2.331297in}{1.882130in}}{\pgfqpoint{2.320698in}{1.886521in}}{\pgfqpoint{2.309648in}{1.886521in}}%
\pgfpathcurveto{\pgfqpoint{2.298598in}{1.886521in}}{\pgfqpoint{2.287999in}{1.882130in}}{\pgfqpoint{2.280185in}{1.874317in}}%
\pgfpathcurveto{\pgfqpoint{2.272372in}{1.866503in}}{\pgfqpoint{2.267981in}{1.855904in}}{\pgfqpoint{2.267981in}{1.844854in}}%
\pgfpathcurveto{\pgfqpoint{2.267981in}{1.833804in}}{\pgfqpoint{2.272372in}{1.823205in}}{\pgfqpoint{2.280185in}{1.815391in}}%
\pgfpathcurveto{\pgfqpoint{2.287999in}{1.807578in}}{\pgfqpoint{2.298598in}{1.803187in}}{\pgfqpoint{2.309648in}{1.803187in}}%
\pgfpathclose%
\pgfusepath{stroke,fill}%
\end{pgfscope}%
\begin{pgfscope}%
\pgfpathrectangle{\pgfqpoint{0.600000in}{0.600000in}}{\pgfqpoint{3.900000in}{3.900000in}}%
\pgfusepath{clip}%
\pgfsetbuttcap%
\pgfsetroundjoin%
\definecolor{currentfill}{rgb}{0.121569,0.466667,0.705882}%
\pgfsetfillcolor{currentfill}%
\pgfsetlinewidth{1.003750pt}%
\definecolor{currentstroke}{rgb}{0.121569,0.466667,0.705882}%
\pgfsetstrokecolor{currentstroke}%
\pgfsetdash{}{0pt}%
\pgfpathmoveto{\pgfqpoint{3.368370in}{2.064268in}}%
\pgfpathcurveto{\pgfqpoint{3.379420in}{2.064268in}}{\pgfqpoint{3.390019in}{2.068658in}}{\pgfqpoint{3.397833in}{2.076472in}}%
\pgfpathcurveto{\pgfqpoint{3.405646in}{2.084286in}}{\pgfqpoint{3.410036in}{2.094885in}}{\pgfqpoint{3.410036in}{2.105935in}}%
\pgfpathcurveto{\pgfqpoint{3.410036in}{2.116985in}}{\pgfqpoint{3.405646in}{2.127584in}}{\pgfqpoint{3.397833in}{2.135398in}}%
\pgfpathcurveto{\pgfqpoint{3.390019in}{2.143211in}}{\pgfqpoint{3.379420in}{2.147602in}}{\pgfqpoint{3.368370in}{2.147602in}}%
\pgfpathcurveto{\pgfqpoint{3.357320in}{2.147602in}}{\pgfqpoint{3.346721in}{2.143211in}}{\pgfqpoint{3.338907in}{2.135398in}}%
\pgfpathcurveto{\pgfqpoint{3.331093in}{2.127584in}}{\pgfqpoint{3.326703in}{2.116985in}}{\pgfqpoint{3.326703in}{2.105935in}}%
\pgfpathcurveto{\pgfqpoint{3.326703in}{2.094885in}}{\pgfqpoint{3.331093in}{2.084286in}}{\pgfqpoint{3.338907in}{2.076472in}}%
\pgfpathcurveto{\pgfqpoint{3.346721in}{2.068658in}}{\pgfqpoint{3.357320in}{2.064268in}}{\pgfqpoint{3.368370in}{2.064268in}}%
\pgfpathclose%
\pgfusepath{stroke,fill}%
\end{pgfscope}%
\begin{pgfscope}%
\pgfpathrectangle{\pgfqpoint{0.600000in}{0.600000in}}{\pgfqpoint{3.900000in}{3.900000in}}%
\pgfusepath{clip}%
\pgfsetbuttcap%
\pgfsetroundjoin%
\definecolor{currentfill}{rgb}{0.121569,0.466667,0.705882}%
\pgfsetfillcolor{currentfill}%
\pgfsetlinewidth{1.003750pt}%
\definecolor{currentstroke}{rgb}{0.121569,0.466667,0.705882}%
\pgfsetstrokecolor{currentstroke}%
\pgfsetdash{}{0pt}%
\pgfpathmoveto{\pgfqpoint{1.816298in}{2.618911in}}%
\pgfpathcurveto{\pgfqpoint{1.827348in}{2.618911in}}{\pgfqpoint{1.837947in}{2.623301in}}{\pgfqpoint{1.845760in}{2.631115in}}%
\pgfpathcurveto{\pgfqpoint{1.853574in}{2.638928in}}{\pgfqpoint{1.857964in}{2.649527in}}{\pgfqpoint{1.857964in}{2.660578in}}%
\pgfpathcurveto{\pgfqpoint{1.857964in}{2.671628in}}{\pgfqpoint{1.853574in}{2.682227in}}{\pgfqpoint{1.845760in}{2.690040in}}%
\pgfpathcurveto{\pgfqpoint{1.837947in}{2.697854in}}{\pgfqpoint{1.827348in}{2.702244in}}{\pgfqpoint{1.816298in}{2.702244in}}%
\pgfpathcurveto{\pgfqpoint{1.805248in}{2.702244in}}{\pgfqpoint{1.794649in}{2.697854in}}{\pgfqpoint{1.786835in}{2.690040in}}%
\pgfpathcurveto{\pgfqpoint{1.779021in}{2.682227in}}{\pgfqpoint{1.774631in}{2.671628in}}{\pgfqpoint{1.774631in}{2.660578in}}%
\pgfpathcurveto{\pgfqpoint{1.774631in}{2.649527in}}{\pgfqpoint{1.779021in}{2.638928in}}{\pgfqpoint{1.786835in}{2.631115in}}%
\pgfpathcurveto{\pgfqpoint{1.794649in}{2.623301in}}{\pgfqpoint{1.805248in}{2.618911in}}{\pgfqpoint{1.816298in}{2.618911in}}%
\pgfpathclose%
\pgfusepath{stroke,fill}%
\end{pgfscope}%
\begin{pgfscope}%
\pgfpathrectangle{\pgfqpoint{0.600000in}{0.600000in}}{\pgfqpoint{3.900000in}{3.900000in}}%
\pgfusepath{clip}%
\pgfsetbuttcap%
\pgfsetroundjoin%
\definecolor{currentfill}{rgb}{0.121569,0.466667,0.705882}%
\pgfsetfillcolor{currentfill}%
\pgfsetlinewidth{1.003750pt}%
\definecolor{currentstroke}{rgb}{0.121569,0.466667,0.705882}%
\pgfsetstrokecolor{currentstroke}%
\pgfsetdash{}{0pt}%
\pgfpathmoveto{\pgfqpoint{2.551412in}{2.270823in}}%
\pgfpathcurveto{\pgfqpoint{2.562462in}{2.270823in}}{\pgfqpoint{2.573061in}{2.275213in}}{\pgfqpoint{2.580874in}{2.283027in}}%
\pgfpathcurveto{\pgfqpoint{2.588688in}{2.290840in}}{\pgfqpoint{2.593078in}{2.301439in}}{\pgfqpoint{2.593078in}{2.312489in}}%
\pgfpathcurveto{\pgfqpoint{2.593078in}{2.323540in}}{\pgfqpoint{2.588688in}{2.334139in}}{\pgfqpoint{2.580874in}{2.341952in}}%
\pgfpathcurveto{\pgfqpoint{2.573061in}{2.349766in}}{\pgfqpoint{2.562462in}{2.354156in}}{\pgfqpoint{2.551412in}{2.354156in}}%
\pgfpathcurveto{\pgfqpoint{2.540362in}{2.354156in}}{\pgfqpoint{2.529762in}{2.349766in}}{\pgfqpoint{2.521949in}{2.341952in}}%
\pgfpathcurveto{\pgfqpoint{2.514135in}{2.334139in}}{\pgfqpoint{2.509745in}{2.323540in}}{\pgfqpoint{2.509745in}{2.312489in}}%
\pgfpathcurveto{\pgfqpoint{2.509745in}{2.301439in}}{\pgfqpoint{2.514135in}{2.290840in}}{\pgfqpoint{2.521949in}{2.283027in}}%
\pgfpathcurveto{\pgfqpoint{2.529762in}{2.275213in}}{\pgfqpoint{2.540362in}{2.270823in}}{\pgfqpoint{2.551412in}{2.270823in}}%
\pgfpathclose%
\pgfusepath{stroke,fill}%
\end{pgfscope}%
\begin{pgfscope}%
\pgfpathrectangle{\pgfqpoint{0.600000in}{0.600000in}}{\pgfqpoint{3.900000in}{3.900000in}}%
\pgfusepath{clip}%
\pgfsetbuttcap%
\pgfsetroundjoin%
\definecolor{currentfill}{rgb}{0.121569,0.466667,0.705882}%
\pgfsetfillcolor{currentfill}%
\pgfsetlinewidth{1.003750pt}%
\definecolor{currentstroke}{rgb}{0.121569,0.466667,0.705882}%
\pgfsetstrokecolor{currentstroke}%
\pgfsetdash{}{0pt}%
\pgfpathmoveto{\pgfqpoint{3.169285in}{2.757576in}}%
\pgfpathcurveto{\pgfqpoint{3.180335in}{2.757576in}}{\pgfqpoint{3.190934in}{2.761967in}}{\pgfqpoint{3.198748in}{2.769780in}}%
\pgfpathcurveto{\pgfqpoint{3.206561in}{2.777594in}}{\pgfqpoint{3.210952in}{2.788193in}}{\pgfqpoint{3.210952in}{2.799243in}}%
\pgfpathcurveto{\pgfqpoint{3.210952in}{2.810293in}}{\pgfqpoint{3.206561in}{2.820892in}}{\pgfqpoint{3.198748in}{2.828706in}}%
\pgfpathcurveto{\pgfqpoint{3.190934in}{2.836519in}}{\pgfqpoint{3.180335in}{2.840910in}}{\pgfqpoint{3.169285in}{2.840910in}}%
\pgfpathcurveto{\pgfqpoint{3.158235in}{2.840910in}}{\pgfqpoint{3.147636in}{2.836519in}}{\pgfqpoint{3.139822in}{2.828706in}}%
\pgfpathcurveto{\pgfqpoint{3.132008in}{2.820892in}}{\pgfqpoint{3.127618in}{2.810293in}}{\pgfqpoint{3.127618in}{2.799243in}}%
\pgfpathcurveto{\pgfqpoint{3.127618in}{2.788193in}}{\pgfqpoint{3.132008in}{2.777594in}}{\pgfqpoint{3.139822in}{2.769780in}}%
\pgfpathcurveto{\pgfqpoint{3.147636in}{2.761967in}}{\pgfqpoint{3.158235in}{2.757576in}}{\pgfqpoint{3.169285in}{2.757576in}}%
\pgfpathclose%
\pgfusepath{stroke,fill}%
\end{pgfscope}%
\begin{pgfscope}%
\pgfpathrectangle{\pgfqpoint{0.600000in}{0.600000in}}{\pgfqpoint{3.900000in}{3.900000in}}%
\pgfusepath{clip}%
\pgfsetbuttcap%
\pgfsetroundjoin%
\definecolor{currentfill}{rgb}{0.121569,0.466667,0.705882}%
\pgfsetfillcolor{currentfill}%
\pgfsetlinewidth{1.003750pt}%
\definecolor{currentstroke}{rgb}{0.121569,0.466667,0.705882}%
\pgfsetstrokecolor{currentstroke}%
\pgfsetdash{}{0pt}%
\pgfpathmoveto{\pgfqpoint{2.720777in}{2.345661in}}%
\pgfpathcurveto{\pgfqpoint{2.731827in}{2.345661in}}{\pgfqpoint{2.742426in}{2.350051in}}{\pgfqpoint{2.750240in}{2.357865in}}%
\pgfpathcurveto{\pgfqpoint{2.758054in}{2.365679in}}{\pgfqpoint{2.762444in}{2.376278in}}{\pgfqpoint{2.762444in}{2.387328in}}%
\pgfpathcurveto{\pgfqpoint{2.762444in}{2.398378in}}{\pgfqpoint{2.758054in}{2.408977in}}{\pgfqpoint{2.750240in}{2.416790in}}%
\pgfpathcurveto{\pgfqpoint{2.742426in}{2.424604in}}{\pgfqpoint{2.731827in}{2.428994in}}{\pgfqpoint{2.720777in}{2.428994in}}%
\pgfpathcurveto{\pgfqpoint{2.709727in}{2.428994in}}{\pgfqpoint{2.699128in}{2.424604in}}{\pgfqpoint{2.691314in}{2.416790in}}%
\pgfpathcurveto{\pgfqpoint{2.683501in}{2.408977in}}{\pgfqpoint{2.679110in}{2.398378in}}{\pgfqpoint{2.679110in}{2.387328in}}%
\pgfpathcurveto{\pgfqpoint{2.679110in}{2.376278in}}{\pgfqpoint{2.683501in}{2.365679in}}{\pgfqpoint{2.691314in}{2.357865in}}%
\pgfpathcurveto{\pgfqpoint{2.699128in}{2.350051in}}{\pgfqpoint{2.709727in}{2.345661in}}{\pgfqpoint{2.720777in}{2.345661in}}%
\pgfpathclose%
\pgfusepath{stroke,fill}%
\end{pgfscope}%
\begin{pgfscope}%
\pgfpathrectangle{\pgfqpoint{0.600000in}{0.600000in}}{\pgfqpoint{3.900000in}{3.900000in}}%
\pgfusepath{clip}%
\pgfsetbuttcap%
\pgfsetroundjoin%
\definecolor{currentfill}{rgb}{0.121569,0.466667,0.705882}%
\pgfsetfillcolor{currentfill}%
\pgfsetlinewidth{1.003750pt}%
\definecolor{currentstroke}{rgb}{0.121569,0.466667,0.705882}%
\pgfsetstrokecolor{currentstroke}%
\pgfsetdash{}{0pt}%
\pgfpathmoveto{\pgfqpoint{2.889135in}{2.327058in}}%
\pgfpathcurveto{\pgfqpoint{2.900185in}{2.327058in}}{\pgfqpoint{2.910785in}{2.331448in}}{\pgfqpoint{2.918598in}{2.339262in}}%
\pgfpathcurveto{\pgfqpoint{2.926412in}{2.347075in}}{\pgfqpoint{2.930802in}{2.357674in}}{\pgfqpoint{2.930802in}{2.368725in}}%
\pgfpathcurveto{\pgfqpoint{2.930802in}{2.379775in}}{\pgfqpoint{2.926412in}{2.390374in}}{\pgfqpoint{2.918598in}{2.398187in}}%
\pgfpathcurveto{\pgfqpoint{2.910785in}{2.406001in}}{\pgfqpoint{2.900185in}{2.410391in}}{\pgfqpoint{2.889135in}{2.410391in}}%
\pgfpathcurveto{\pgfqpoint{2.878085in}{2.410391in}}{\pgfqpoint{2.867486in}{2.406001in}}{\pgfqpoint{2.859673in}{2.398187in}}%
\pgfpathcurveto{\pgfqpoint{2.851859in}{2.390374in}}{\pgfqpoint{2.847469in}{2.379775in}}{\pgfqpoint{2.847469in}{2.368725in}}%
\pgfpathcurveto{\pgfqpoint{2.847469in}{2.357674in}}{\pgfqpoint{2.851859in}{2.347075in}}{\pgfqpoint{2.859673in}{2.339262in}}%
\pgfpathcurveto{\pgfqpoint{2.867486in}{2.331448in}}{\pgfqpoint{2.878085in}{2.327058in}}{\pgfqpoint{2.889135in}{2.327058in}}%
\pgfpathclose%
\pgfusepath{stroke,fill}%
\end{pgfscope}%
\begin{pgfscope}%
\pgfpathrectangle{\pgfqpoint{0.600000in}{0.600000in}}{\pgfqpoint{3.900000in}{3.900000in}}%
\pgfusepath{clip}%
\pgfsetbuttcap%
\pgfsetroundjoin%
\definecolor{currentfill}{rgb}{0.121569,0.466667,0.705882}%
\pgfsetfillcolor{currentfill}%
\pgfsetlinewidth{1.003750pt}%
\definecolor{currentstroke}{rgb}{0.121569,0.466667,0.705882}%
\pgfsetstrokecolor{currentstroke}%
\pgfsetdash{}{0pt}%
\pgfpathmoveto{\pgfqpoint{1.852681in}{2.887175in}}%
\pgfpathcurveto{\pgfqpoint{1.863731in}{2.887175in}}{\pgfqpoint{1.874331in}{2.891566in}}{\pgfqpoint{1.882144in}{2.899379in}}%
\pgfpathcurveto{\pgfqpoint{1.889958in}{2.907193in}}{\pgfqpoint{1.894348in}{2.917792in}}{\pgfqpoint{1.894348in}{2.928842in}}%
\pgfpathcurveto{\pgfqpoint{1.894348in}{2.939892in}}{\pgfqpoint{1.889958in}{2.950491in}}{\pgfqpoint{1.882144in}{2.958305in}}%
\pgfpathcurveto{\pgfqpoint{1.874331in}{2.966118in}}{\pgfqpoint{1.863731in}{2.970509in}}{\pgfqpoint{1.852681in}{2.970509in}}%
\pgfpathcurveto{\pgfqpoint{1.841631in}{2.970509in}}{\pgfqpoint{1.831032in}{2.966118in}}{\pgfqpoint{1.823219in}{2.958305in}}%
\pgfpathcurveto{\pgfqpoint{1.815405in}{2.950491in}}{\pgfqpoint{1.811015in}{2.939892in}}{\pgfqpoint{1.811015in}{2.928842in}}%
\pgfpathcurveto{\pgfqpoint{1.811015in}{2.917792in}}{\pgfqpoint{1.815405in}{2.907193in}}{\pgfqpoint{1.823219in}{2.899379in}}%
\pgfpathcurveto{\pgfqpoint{1.831032in}{2.891566in}}{\pgfqpoint{1.841631in}{2.887175in}}{\pgfqpoint{1.852681in}{2.887175in}}%
\pgfpathclose%
\pgfusepath{stroke,fill}%
\end{pgfscope}%
\begin{pgfscope}%
\pgfpathrectangle{\pgfqpoint{0.600000in}{0.600000in}}{\pgfqpoint{3.900000in}{3.900000in}}%
\pgfusepath{clip}%
\pgfsetbuttcap%
\pgfsetroundjoin%
\definecolor{currentfill}{rgb}{0.121569,0.466667,0.705882}%
\pgfsetfillcolor{currentfill}%
\pgfsetlinewidth{1.003750pt}%
\definecolor{currentstroke}{rgb}{0.121569,0.466667,0.705882}%
\pgfsetstrokecolor{currentstroke}%
\pgfsetdash{}{0pt}%
\pgfpathmoveto{\pgfqpoint{2.537795in}{3.510113in}}%
\pgfpathcurveto{\pgfqpoint{2.548845in}{3.510113in}}{\pgfqpoint{2.559444in}{3.514503in}}{\pgfqpoint{2.567258in}{3.522317in}}%
\pgfpathcurveto{\pgfqpoint{2.575071in}{3.530131in}}{\pgfqpoint{2.579462in}{3.540730in}}{\pgfqpoint{2.579462in}{3.551780in}}%
\pgfpathcurveto{\pgfqpoint{2.579462in}{3.562830in}}{\pgfqpoint{2.575071in}{3.573429in}}{\pgfqpoint{2.567258in}{3.581242in}}%
\pgfpathcurveto{\pgfqpoint{2.559444in}{3.589056in}}{\pgfqpoint{2.548845in}{3.593446in}}{\pgfqpoint{2.537795in}{3.593446in}}%
\pgfpathcurveto{\pgfqpoint{2.526745in}{3.593446in}}{\pgfqpoint{2.516146in}{3.589056in}}{\pgfqpoint{2.508332in}{3.581242in}}%
\pgfpathcurveto{\pgfqpoint{2.500519in}{3.573429in}}{\pgfqpoint{2.496128in}{3.562830in}}{\pgfqpoint{2.496128in}{3.551780in}}%
\pgfpathcurveto{\pgfqpoint{2.496128in}{3.540730in}}{\pgfqpoint{2.500519in}{3.530131in}}{\pgfqpoint{2.508332in}{3.522317in}}%
\pgfpathcurveto{\pgfqpoint{2.516146in}{3.514503in}}{\pgfqpoint{2.526745in}{3.510113in}}{\pgfqpoint{2.537795in}{3.510113in}}%
\pgfpathclose%
\pgfusepath{stroke,fill}%
\end{pgfscope}%
\begin{pgfscope}%
\pgfpathrectangle{\pgfqpoint{0.600000in}{0.600000in}}{\pgfqpoint{3.900000in}{3.900000in}}%
\pgfusepath{clip}%
\pgfsetbuttcap%
\pgfsetroundjoin%
\definecolor{currentfill}{rgb}{0.121569,0.466667,0.705882}%
\pgfsetfillcolor{currentfill}%
\pgfsetlinewidth{1.003750pt}%
\definecolor{currentstroke}{rgb}{0.121569,0.466667,0.705882}%
\pgfsetstrokecolor{currentstroke}%
\pgfsetdash{}{0pt}%
\pgfpathmoveto{\pgfqpoint{3.114058in}{2.076004in}}%
\pgfpathcurveto{\pgfqpoint{3.125108in}{2.076004in}}{\pgfqpoint{3.135707in}{2.080395in}}{\pgfqpoint{3.143520in}{2.088208in}}%
\pgfpathcurveto{\pgfqpoint{3.151334in}{2.096022in}}{\pgfqpoint{3.155724in}{2.106621in}}{\pgfqpoint{3.155724in}{2.117671in}}%
\pgfpathcurveto{\pgfqpoint{3.155724in}{2.128721in}}{\pgfqpoint{3.151334in}{2.139320in}}{\pgfqpoint{3.143520in}{2.147134in}}%
\pgfpathcurveto{\pgfqpoint{3.135707in}{2.154948in}}{\pgfqpoint{3.125108in}{2.159338in}}{\pgfqpoint{3.114058in}{2.159338in}}%
\pgfpathcurveto{\pgfqpoint{3.103007in}{2.159338in}}{\pgfqpoint{3.092408in}{2.154948in}}{\pgfqpoint{3.084595in}{2.147134in}}%
\pgfpathcurveto{\pgfqpoint{3.076781in}{2.139320in}}{\pgfqpoint{3.072391in}{2.128721in}}{\pgfqpoint{3.072391in}{2.117671in}}%
\pgfpathcurveto{\pgfqpoint{3.072391in}{2.106621in}}{\pgfqpoint{3.076781in}{2.096022in}}{\pgfqpoint{3.084595in}{2.088208in}}%
\pgfpathcurveto{\pgfqpoint{3.092408in}{2.080395in}}{\pgfqpoint{3.103007in}{2.076004in}}{\pgfqpoint{3.114058in}{2.076004in}}%
\pgfpathclose%
\pgfusepath{stroke,fill}%
\end{pgfscope}%
\begin{pgfscope}%
\pgfpathrectangle{\pgfqpoint{0.600000in}{0.600000in}}{\pgfqpoint{3.900000in}{3.900000in}}%
\pgfusepath{clip}%
\pgfsetbuttcap%
\pgfsetroundjoin%
\definecolor{currentfill}{rgb}{0.121569,0.466667,0.705882}%
\pgfsetfillcolor{currentfill}%
\pgfsetlinewidth{1.003750pt}%
\definecolor{currentstroke}{rgb}{0.121569,0.466667,0.705882}%
\pgfsetstrokecolor{currentstroke}%
\pgfsetdash{}{0pt}%
\pgfpathmoveto{\pgfqpoint{2.250964in}{2.226380in}}%
\pgfpathcurveto{\pgfqpoint{2.262014in}{2.226380in}}{\pgfqpoint{2.272613in}{2.230771in}}{\pgfqpoint{2.280427in}{2.238584in}}%
\pgfpathcurveto{\pgfqpoint{2.288240in}{2.246398in}}{\pgfqpoint{2.292630in}{2.256997in}}{\pgfqpoint{2.292630in}{2.268047in}}%
\pgfpathcurveto{\pgfqpoint{2.292630in}{2.279097in}}{\pgfqpoint{2.288240in}{2.289696in}}{\pgfqpoint{2.280427in}{2.297510in}}%
\pgfpathcurveto{\pgfqpoint{2.272613in}{2.305323in}}{\pgfqpoint{2.262014in}{2.309714in}}{\pgfqpoint{2.250964in}{2.309714in}}%
\pgfpathcurveto{\pgfqpoint{2.239914in}{2.309714in}}{\pgfqpoint{2.229315in}{2.305323in}}{\pgfqpoint{2.221501in}{2.297510in}}%
\pgfpathcurveto{\pgfqpoint{2.213687in}{2.289696in}}{\pgfqpoint{2.209297in}{2.279097in}}{\pgfqpoint{2.209297in}{2.268047in}}%
\pgfpathcurveto{\pgfqpoint{2.209297in}{2.256997in}}{\pgfqpoint{2.213687in}{2.246398in}}{\pgfqpoint{2.221501in}{2.238584in}}%
\pgfpathcurveto{\pgfqpoint{2.229315in}{2.230771in}}{\pgfqpoint{2.239914in}{2.226380in}}{\pgfqpoint{2.250964in}{2.226380in}}%
\pgfpathclose%
\pgfusepath{stroke,fill}%
\end{pgfscope}%
\begin{pgfscope}%
\pgfpathrectangle{\pgfqpoint{0.600000in}{0.600000in}}{\pgfqpoint{3.900000in}{3.900000in}}%
\pgfusepath{clip}%
\pgfsetbuttcap%
\pgfsetroundjoin%
\definecolor{currentfill}{rgb}{0.121569,0.466667,0.705882}%
\pgfsetfillcolor{currentfill}%
\pgfsetlinewidth{1.003750pt}%
\definecolor{currentstroke}{rgb}{0.121569,0.466667,0.705882}%
\pgfsetstrokecolor{currentstroke}%
\pgfsetdash{}{0pt}%
\pgfpathmoveto{\pgfqpoint{2.624439in}{1.795083in}}%
\pgfpathcurveto{\pgfqpoint{2.635489in}{1.795083in}}{\pgfqpoint{2.646088in}{1.799474in}}{\pgfqpoint{2.653902in}{1.807287in}}%
\pgfpathcurveto{\pgfqpoint{2.661716in}{1.815101in}}{\pgfqpoint{2.666106in}{1.825700in}}{\pgfqpoint{2.666106in}{1.836750in}}%
\pgfpathcurveto{\pgfqpoint{2.666106in}{1.847800in}}{\pgfqpoint{2.661716in}{1.858399in}}{\pgfqpoint{2.653902in}{1.866213in}}%
\pgfpathcurveto{\pgfqpoint{2.646088in}{1.874027in}}{\pgfqpoint{2.635489in}{1.878417in}}{\pgfqpoint{2.624439in}{1.878417in}}%
\pgfpathcurveto{\pgfqpoint{2.613389in}{1.878417in}}{\pgfqpoint{2.602790in}{1.874027in}}{\pgfqpoint{2.594976in}{1.866213in}}%
\pgfpathcurveto{\pgfqpoint{2.587163in}{1.858399in}}{\pgfqpoint{2.582772in}{1.847800in}}{\pgfqpoint{2.582772in}{1.836750in}}%
\pgfpathcurveto{\pgfqpoint{2.582772in}{1.825700in}}{\pgfqpoint{2.587163in}{1.815101in}}{\pgfqpoint{2.594976in}{1.807287in}}%
\pgfpathcurveto{\pgfqpoint{2.602790in}{1.799474in}}{\pgfqpoint{2.613389in}{1.795083in}}{\pgfqpoint{2.624439in}{1.795083in}}%
\pgfpathclose%
\pgfusepath{stroke,fill}%
\end{pgfscope}%
\begin{pgfscope}%
\pgfpathrectangle{\pgfqpoint{0.600000in}{0.600000in}}{\pgfqpoint{3.900000in}{3.900000in}}%
\pgfusepath{clip}%
\pgfsetbuttcap%
\pgfsetroundjoin%
\definecolor{currentfill}{rgb}{0.121569,0.466667,0.705882}%
\pgfsetfillcolor{currentfill}%
\pgfsetlinewidth{1.003750pt}%
\definecolor{currentstroke}{rgb}{0.121569,0.466667,0.705882}%
\pgfsetstrokecolor{currentstroke}%
\pgfsetdash{}{0pt}%
\pgfpathmoveto{\pgfqpoint{2.786490in}{1.858451in}}%
\pgfpathcurveto{\pgfqpoint{2.797540in}{1.858451in}}{\pgfqpoint{2.808139in}{1.862841in}}{\pgfqpoint{2.815953in}{1.870655in}}%
\pgfpathcurveto{\pgfqpoint{2.823767in}{1.878469in}}{\pgfqpoint{2.828157in}{1.889068in}}{\pgfqpoint{2.828157in}{1.900118in}}%
\pgfpathcurveto{\pgfqpoint{2.828157in}{1.911168in}}{\pgfqpoint{2.823767in}{1.921767in}}{\pgfqpoint{2.815953in}{1.929581in}}%
\pgfpathcurveto{\pgfqpoint{2.808139in}{1.937394in}}{\pgfqpoint{2.797540in}{1.941784in}}{\pgfqpoint{2.786490in}{1.941784in}}%
\pgfpathcurveto{\pgfqpoint{2.775440in}{1.941784in}}{\pgfqpoint{2.764841in}{1.937394in}}{\pgfqpoint{2.757027in}{1.929581in}}%
\pgfpathcurveto{\pgfqpoint{2.749214in}{1.921767in}}{\pgfqpoint{2.744824in}{1.911168in}}{\pgfqpoint{2.744824in}{1.900118in}}%
\pgfpathcurveto{\pgfqpoint{2.744824in}{1.889068in}}{\pgfqpoint{2.749214in}{1.878469in}}{\pgfqpoint{2.757027in}{1.870655in}}%
\pgfpathcurveto{\pgfqpoint{2.764841in}{1.862841in}}{\pgfqpoint{2.775440in}{1.858451in}}{\pgfqpoint{2.786490in}{1.858451in}}%
\pgfpathclose%
\pgfusepath{stroke,fill}%
\end{pgfscope}%
\begin{pgfscope}%
\pgfpathrectangle{\pgfqpoint{0.600000in}{0.600000in}}{\pgfqpoint{3.900000in}{3.900000in}}%
\pgfusepath{clip}%
\pgfsetbuttcap%
\pgfsetroundjoin%
\definecolor{currentfill}{rgb}{0.121569,0.466667,0.705882}%
\pgfsetfillcolor{currentfill}%
\pgfsetlinewidth{1.003750pt}%
\definecolor{currentstroke}{rgb}{0.121569,0.466667,0.705882}%
\pgfsetstrokecolor{currentstroke}%
\pgfsetdash{}{0pt}%
\pgfpathmoveto{\pgfqpoint{2.740007in}{2.464529in}}%
\pgfpathcurveto{\pgfqpoint{2.751057in}{2.464529in}}{\pgfqpoint{2.761656in}{2.468920in}}{\pgfqpoint{2.769469in}{2.476733in}}%
\pgfpathcurveto{\pgfqpoint{2.777283in}{2.484547in}}{\pgfqpoint{2.781673in}{2.495146in}}{\pgfqpoint{2.781673in}{2.506196in}}%
\pgfpathcurveto{\pgfqpoint{2.781673in}{2.517246in}}{\pgfqpoint{2.777283in}{2.527845in}}{\pgfqpoint{2.769469in}{2.535659in}}%
\pgfpathcurveto{\pgfqpoint{2.761656in}{2.543472in}}{\pgfqpoint{2.751057in}{2.547863in}}{\pgfqpoint{2.740007in}{2.547863in}}%
\pgfpathcurveto{\pgfqpoint{2.728956in}{2.547863in}}{\pgfqpoint{2.718357in}{2.543472in}}{\pgfqpoint{2.710544in}{2.535659in}}%
\pgfpathcurveto{\pgfqpoint{2.702730in}{2.527845in}}{\pgfqpoint{2.698340in}{2.517246in}}{\pgfqpoint{2.698340in}{2.506196in}}%
\pgfpathcurveto{\pgfqpoint{2.698340in}{2.495146in}}{\pgfqpoint{2.702730in}{2.484547in}}{\pgfqpoint{2.710544in}{2.476733in}}%
\pgfpathcurveto{\pgfqpoint{2.718357in}{2.468920in}}{\pgfqpoint{2.728956in}{2.464529in}}{\pgfqpoint{2.740007in}{2.464529in}}%
\pgfpathclose%
\pgfusepath{stroke,fill}%
\end{pgfscope}%
\begin{pgfscope}%
\pgfpathrectangle{\pgfqpoint{0.600000in}{0.600000in}}{\pgfqpoint{3.900000in}{3.900000in}}%
\pgfusepath{clip}%
\pgfsetbuttcap%
\pgfsetroundjoin%
\definecolor{currentfill}{rgb}{0.121569,0.466667,0.705882}%
\pgfsetfillcolor{currentfill}%
\pgfsetlinewidth{1.003750pt}%
\definecolor{currentstroke}{rgb}{0.121569,0.466667,0.705882}%
\pgfsetstrokecolor{currentstroke}%
\pgfsetdash{}{0pt}%
\pgfpathmoveto{\pgfqpoint{2.759030in}{1.669752in}}%
\pgfpathcurveto{\pgfqpoint{2.770080in}{1.669752in}}{\pgfqpoint{2.780679in}{1.674142in}}{\pgfqpoint{2.788492in}{1.681956in}}%
\pgfpathcurveto{\pgfqpoint{2.796306in}{1.689769in}}{\pgfqpoint{2.800696in}{1.700368in}}{\pgfqpoint{2.800696in}{1.711419in}}%
\pgfpathcurveto{\pgfqpoint{2.800696in}{1.722469in}}{\pgfqpoint{2.796306in}{1.733068in}}{\pgfqpoint{2.788492in}{1.740881in}}%
\pgfpathcurveto{\pgfqpoint{2.780679in}{1.748695in}}{\pgfqpoint{2.770080in}{1.753085in}}{\pgfqpoint{2.759030in}{1.753085in}}%
\pgfpathcurveto{\pgfqpoint{2.747980in}{1.753085in}}{\pgfqpoint{2.737380in}{1.748695in}}{\pgfqpoint{2.729567in}{1.740881in}}%
\pgfpathcurveto{\pgfqpoint{2.721753in}{1.733068in}}{\pgfqpoint{2.717363in}{1.722469in}}{\pgfqpoint{2.717363in}{1.711419in}}%
\pgfpathcurveto{\pgfqpoint{2.717363in}{1.700368in}}{\pgfqpoint{2.721753in}{1.689769in}}{\pgfqpoint{2.729567in}{1.681956in}}%
\pgfpathcurveto{\pgfqpoint{2.737380in}{1.674142in}}{\pgfqpoint{2.747980in}{1.669752in}}{\pgfqpoint{2.759030in}{1.669752in}}%
\pgfpathclose%
\pgfusepath{stroke,fill}%
\end{pgfscope}%
\begin{pgfscope}%
\pgfpathrectangle{\pgfqpoint{0.600000in}{0.600000in}}{\pgfqpoint{3.900000in}{3.900000in}}%
\pgfusepath{clip}%
\pgfsetbuttcap%
\pgfsetroundjoin%
\definecolor{currentfill}{rgb}{0.121569,0.466667,0.705882}%
\pgfsetfillcolor{currentfill}%
\pgfsetlinewidth{1.003750pt}%
\definecolor{currentstroke}{rgb}{0.121569,0.466667,0.705882}%
\pgfsetstrokecolor{currentstroke}%
\pgfsetdash{}{0pt}%
\pgfpathmoveto{\pgfqpoint{3.514030in}{2.721817in}}%
\pgfpathcurveto{\pgfqpoint{3.525080in}{2.721817in}}{\pgfqpoint{3.535679in}{2.726207in}}{\pgfqpoint{3.543493in}{2.734021in}}%
\pgfpathcurveto{\pgfqpoint{3.551306in}{2.741834in}}{\pgfqpoint{3.555697in}{2.752433in}}{\pgfqpoint{3.555697in}{2.763483in}}%
\pgfpathcurveto{\pgfqpoint{3.555697in}{2.774534in}}{\pgfqpoint{3.551306in}{2.785133in}}{\pgfqpoint{3.543493in}{2.792946in}}%
\pgfpathcurveto{\pgfqpoint{3.535679in}{2.800760in}}{\pgfqpoint{3.525080in}{2.805150in}}{\pgfqpoint{3.514030in}{2.805150in}}%
\pgfpathcurveto{\pgfqpoint{3.502980in}{2.805150in}}{\pgfqpoint{3.492381in}{2.800760in}}{\pgfqpoint{3.484567in}{2.792946in}}%
\pgfpathcurveto{\pgfqpoint{3.476754in}{2.785133in}}{\pgfqpoint{3.472363in}{2.774534in}}{\pgfqpoint{3.472363in}{2.763483in}}%
\pgfpathcurveto{\pgfqpoint{3.472363in}{2.752433in}}{\pgfqpoint{3.476754in}{2.741834in}}{\pgfqpoint{3.484567in}{2.734021in}}%
\pgfpathcurveto{\pgfqpoint{3.492381in}{2.726207in}}{\pgfqpoint{3.502980in}{2.721817in}}{\pgfqpoint{3.514030in}{2.721817in}}%
\pgfpathclose%
\pgfusepath{stroke,fill}%
\end{pgfscope}%
\begin{pgfscope}%
\pgfpathrectangle{\pgfqpoint{0.600000in}{0.600000in}}{\pgfqpoint{3.900000in}{3.900000in}}%
\pgfusepath{clip}%
\pgfsetbuttcap%
\pgfsetroundjoin%
\definecolor{currentfill}{rgb}{0.121569,0.466667,0.705882}%
\pgfsetfillcolor{currentfill}%
\pgfsetlinewidth{1.003750pt}%
\definecolor{currentstroke}{rgb}{0.121569,0.466667,0.705882}%
\pgfsetstrokecolor{currentstroke}%
\pgfsetdash{}{0pt}%
\pgfpathmoveto{\pgfqpoint{3.073832in}{3.129304in}}%
\pgfpathcurveto{\pgfqpoint{3.084882in}{3.129304in}}{\pgfqpoint{3.095481in}{3.133694in}}{\pgfqpoint{3.103295in}{3.141508in}}%
\pgfpathcurveto{\pgfqpoint{3.111109in}{3.149321in}}{\pgfqpoint{3.115499in}{3.159921in}}{\pgfqpoint{3.115499in}{3.170971in}}%
\pgfpathcurveto{\pgfqpoint{3.115499in}{3.182021in}}{\pgfqpoint{3.111109in}{3.192620in}}{\pgfqpoint{3.103295in}{3.200433in}}%
\pgfpathcurveto{\pgfqpoint{3.095481in}{3.208247in}}{\pgfqpoint{3.084882in}{3.212637in}}{\pgfqpoint{3.073832in}{3.212637in}}%
\pgfpathcurveto{\pgfqpoint{3.062782in}{3.212637in}}{\pgfqpoint{3.052183in}{3.208247in}}{\pgfqpoint{3.044369in}{3.200433in}}%
\pgfpathcurveto{\pgfqpoint{3.036556in}{3.192620in}}{\pgfqpoint{3.032166in}{3.182021in}}{\pgfqpoint{3.032166in}{3.170971in}}%
\pgfpathcurveto{\pgfqpoint{3.032166in}{3.159921in}}{\pgfqpoint{3.036556in}{3.149321in}}{\pgfqpoint{3.044369in}{3.141508in}}%
\pgfpathcurveto{\pgfqpoint{3.052183in}{3.133694in}}{\pgfqpoint{3.062782in}{3.129304in}}{\pgfqpoint{3.073832in}{3.129304in}}%
\pgfpathclose%
\pgfusepath{stroke,fill}%
\end{pgfscope}%
\begin{pgfscope}%
\pgfpathrectangle{\pgfqpoint{0.600000in}{0.600000in}}{\pgfqpoint{3.900000in}{3.900000in}}%
\pgfusepath{clip}%
\pgfsetbuttcap%
\pgfsetroundjoin%
\definecolor{currentfill}{rgb}{0.121569,0.466667,0.705882}%
\pgfsetfillcolor{currentfill}%
\pgfsetlinewidth{1.003750pt}%
\definecolor{currentstroke}{rgb}{0.121569,0.466667,0.705882}%
\pgfsetstrokecolor{currentstroke}%
\pgfsetdash{}{0pt}%
\pgfpathmoveto{\pgfqpoint{2.922965in}{3.360547in}}%
\pgfpathcurveto{\pgfqpoint{2.934015in}{3.360547in}}{\pgfqpoint{2.944614in}{3.364937in}}{\pgfqpoint{2.952427in}{3.372750in}}%
\pgfpathcurveto{\pgfqpoint{2.960241in}{3.380564in}}{\pgfqpoint{2.964631in}{3.391163in}}{\pgfqpoint{2.964631in}{3.402213in}}%
\pgfpathcurveto{\pgfqpoint{2.964631in}{3.413263in}}{\pgfqpoint{2.960241in}{3.423862in}}{\pgfqpoint{2.952427in}{3.431676in}}%
\pgfpathcurveto{\pgfqpoint{2.944614in}{3.439490in}}{\pgfqpoint{2.934015in}{3.443880in}}{\pgfqpoint{2.922965in}{3.443880in}}%
\pgfpathcurveto{\pgfqpoint{2.911914in}{3.443880in}}{\pgfqpoint{2.901315in}{3.439490in}}{\pgfqpoint{2.893502in}{3.431676in}}%
\pgfpathcurveto{\pgfqpoint{2.885688in}{3.423862in}}{\pgfqpoint{2.881298in}{3.413263in}}{\pgfqpoint{2.881298in}{3.402213in}}%
\pgfpathcurveto{\pgfqpoint{2.881298in}{3.391163in}}{\pgfqpoint{2.885688in}{3.380564in}}{\pgfqpoint{2.893502in}{3.372750in}}%
\pgfpathcurveto{\pgfqpoint{2.901315in}{3.364937in}}{\pgfqpoint{2.911914in}{3.360547in}}{\pgfqpoint{2.922965in}{3.360547in}}%
\pgfpathclose%
\pgfusepath{stroke,fill}%
\end{pgfscope}%
\begin{pgfscope}%
\pgfpathrectangle{\pgfqpoint{0.600000in}{0.600000in}}{\pgfqpoint{3.900000in}{3.900000in}}%
\pgfusepath{clip}%
\pgfsetbuttcap%
\pgfsetroundjoin%
\definecolor{currentfill}{rgb}{0.121569,0.466667,0.705882}%
\pgfsetfillcolor{currentfill}%
\pgfsetlinewidth{1.003750pt}%
\definecolor{currentstroke}{rgb}{0.121569,0.466667,0.705882}%
\pgfsetstrokecolor{currentstroke}%
\pgfsetdash{}{0pt}%
\pgfpathmoveto{\pgfqpoint{2.735621in}{2.731797in}}%
\pgfpathcurveto{\pgfqpoint{2.746672in}{2.731797in}}{\pgfqpoint{2.757271in}{2.736188in}}{\pgfqpoint{2.765084in}{2.744001in}}%
\pgfpathcurveto{\pgfqpoint{2.772898in}{2.751815in}}{\pgfqpoint{2.777288in}{2.762414in}}{\pgfqpoint{2.777288in}{2.773464in}}%
\pgfpathcurveto{\pgfqpoint{2.777288in}{2.784514in}}{\pgfqpoint{2.772898in}{2.795113in}}{\pgfqpoint{2.765084in}{2.802927in}}%
\pgfpathcurveto{\pgfqpoint{2.757271in}{2.810740in}}{\pgfqpoint{2.746672in}{2.815131in}}{\pgfqpoint{2.735621in}{2.815131in}}%
\pgfpathcurveto{\pgfqpoint{2.724571in}{2.815131in}}{\pgfqpoint{2.713972in}{2.810740in}}{\pgfqpoint{2.706159in}{2.802927in}}%
\pgfpathcurveto{\pgfqpoint{2.698345in}{2.795113in}}{\pgfqpoint{2.693955in}{2.784514in}}{\pgfqpoint{2.693955in}{2.773464in}}%
\pgfpathcurveto{\pgfqpoint{2.693955in}{2.762414in}}{\pgfqpoint{2.698345in}{2.751815in}}{\pgfqpoint{2.706159in}{2.744001in}}%
\pgfpathcurveto{\pgfqpoint{2.713972in}{2.736188in}}{\pgfqpoint{2.724571in}{2.731797in}}{\pgfqpoint{2.735621in}{2.731797in}}%
\pgfpathclose%
\pgfusepath{stroke,fill}%
\end{pgfscope}%
\begin{pgfscope}%
\pgfpathrectangle{\pgfqpoint{0.600000in}{0.600000in}}{\pgfqpoint{3.900000in}{3.900000in}}%
\pgfusepath{clip}%
\pgfsetbuttcap%
\pgfsetroundjoin%
\definecolor{currentfill}{rgb}{0.121569,0.466667,0.705882}%
\pgfsetfillcolor{currentfill}%
\pgfsetlinewidth{1.003750pt}%
\definecolor{currentstroke}{rgb}{0.121569,0.466667,0.705882}%
\pgfsetstrokecolor{currentstroke}%
\pgfsetdash{}{0pt}%
\pgfpathmoveto{\pgfqpoint{2.507250in}{2.689772in}}%
\pgfpathcurveto{\pgfqpoint{2.518300in}{2.689772in}}{\pgfqpoint{2.528899in}{2.694163in}}{\pgfqpoint{2.536713in}{2.701976in}}%
\pgfpathcurveto{\pgfqpoint{2.544527in}{2.709790in}}{\pgfqpoint{2.548917in}{2.720389in}}{\pgfqpoint{2.548917in}{2.731439in}}%
\pgfpathcurveto{\pgfqpoint{2.548917in}{2.742489in}}{\pgfqpoint{2.544527in}{2.753088in}}{\pgfqpoint{2.536713in}{2.760902in}}%
\pgfpathcurveto{\pgfqpoint{2.528899in}{2.768715in}}{\pgfqpoint{2.518300in}{2.773106in}}{\pgfqpoint{2.507250in}{2.773106in}}%
\pgfpathcurveto{\pgfqpoint{2.496200in}{2.773106in}}{\pgfqpoint{2.485601in}{2.768715in}}{\pgfqpoint{2.477787in}{2.760902in}}%
\pgfpathcurveto{\pgfqpoint{2.469974in}{2.753088in}}{\pgfqpoint{2.465583in}{2.742489in}}{\pgfqpoint{2.465583in}{2.731439in}}%
\pgfpathcurveto{\pgfqpoint{2.465583in}{2.720389in}}{\pgfqpoint{2.469974in}{2.709790in}}{\pgfqpoint{2.477787in}{2.701976in}}%
\pgfpathcurveto{\pgfqpoint{2.485601in}{2.694163in}}{\pgfqpoint{2.496200in}{2.689772in}}{\pgfqpoint{2.507250in}{2.689772in}}%
\pgfpathclose%
\pgfusepath{stroke,fill}%
\end{pgfscope}%
\begin{pgfscope}%
\pgfpathrectangle{\pgfqpoint{0.600000in}{0.600000in}}{\pgfqpoint{3.900000in}{3.900000in}}%
\pgfusepath{clip}%
\pgfsetbuttcap%
\pgfsetroundjoin%
\definecolor{currentfill}{rgb}{0.121569,0.466667,0.705882}%
\pgfsetfillcolor{currentfill}%
\pgfsetlinewidth{1.003750pt}%
\definecolor{currentstroke}{rgb}{0.121569,0.466667,0.705882}%
\pgfsetstrokecolor{currentstroke}%
\pgfsetdash{}{0pt}%
\pgfpathmoveto{\pgfqpoint{2.766927in}{3.307643in}}%
\pgfpathcurveto{\pgfqpoint{2.777978in}{3.307643in}}{\pgfqpoint{2.788577in}{3.312033in}}{\pgfqpoint{2.796390in}{3.319847in}}%
\pgfpathcurveto{\pgfqpoint{2.804204in}{3.327661in}}{\pgfqpoint{2.808594in}{3.338260in}}{\pgfqpoint{2.808594in}{3.349310in}}%
\pgfpathcurveto{\pgfqpoint{2.808594in}{3.360360in}}{\pgfqpoint{2.804204in}{3.370959in}}{\pgfqpoint{2.796390in}{3.378772in}}%
\pgfpathcurveto{\pgfqpoint{2.788577in}{3.386586in}}{\pgfqpoint{2.777978in}{3.390976in}}{\pgfqpoint{2.766927in}{3.390976in}}%
\pgfpathcurveto{\pgfqpoint{2.755877in}{3.390976in}}{\pgfqpoint{2.745278in}{3.386586in}}{\pgfqpoint{2.737465in}{3.378772in}}%
\pgfpathcurveto{\pgfqpoint{2.729651in}{3.370959in}}{\pgfqpoint{2.725261in}{3.360360in}}{\pgfqpoint{2.725261in}{3.349310in}}%
\pgfpathcurveto{\pgfqpoint{2.725261in}{3.338260in}}{\pgfqpoint{2.729651in}{3.327661in}}{\pgfqpoint{2.737465in}{3.319847in}}%
\pgfpathcurveto{\pgfqpoint{2.745278in}{3.312033in}}{\pgfqpoint{2.755877in}{3.307643in}}{\pgfqpoint{2.766927in}{3.307643in}}%
\pgfpathclose%
\pgfusepath{stroke,fill}%
\end{pgfscope}%
\begin{pgfscope}%
\pgfpathrectangle{\pgfqpoint{0.600000in}{0.600000in}}{\pgfqpoint{3.900000in}{3.900000in}}%
\pgfusepath{clip}%
\pgfsetbuttcap%
\pgfsetroundjoin%
\definecolor{currentfill}{rgb}{0.121569,0.466667,0.705882}%
\pgfsetfillcolor{currentfill}%
\pgfsetlinewidth{1.003750pt}%
\definecolor{currentstroke}{rgb}{0.121569,0.466667,0.705882}%
\pgfsetstrokecolor{currentstroke}%
\pgfsetdash{}{0pt}%
\pgfpathmoveto{\pgfqpoint{2.446631in}{1.949974in}}%
\pgfpathcurveto{\pgfqpoint{2.457681in}{1.949974in}}{\pgfqpoint{2.468281in}{1.954364in}}{\pgfqpoint{2.476094in}{1.962178in}}%
\pgfpathcurveto{\pgfqpoint{2.483908in}{1.969991in}}{\pgfqpoint{2.488298in}{1.980590in}}{\pgfqpoint{2.488298in}{1.991641in}}%
\pgfpathcurveto{\pgfqpoint{2.488298in}{2.002691in}}{\pgfqpoint{2.483908in}{2.013290in}}{\pgfqpoint{2.476094in}{2.021103in}}%
\pgfpathcurveto{\pgfqpoint{2.468281in}{2.028917in}}{\pgfqpoint{2.457681in}{2.033307in}}{\pgfqpoint{2.446631in}{2.033307in}}%
\pgfpathcurveto{\pgfqpoint{2.435581in}{2.033307in}}{\pgfqpoint{2.424982in}{2.028917in}}{\pgfqpoint{2.417169in}{2.021103in}}%
\pgfpathcurveto{\pgfqpoint{2.409355in}{2.013290in}}{\pgfqpoint{2.404965in}{2.002691in}}{\pgfqpoint{2.404965in}{1.991641in}}%
\pgfpathcurveto{\pgfqpoint{2.404965in}{1.980590in}}{\pgfqpoint{2.409355in}{1.969991in}}{\pgfqpoint{2.417169in}{1.962178in}}%
\pgfpathcurveto{\pgfqpoint{2.424982in}{1.954364in}}{\pgfqpoint{2.435581in}{1.949974in}}{\pgfqpoint{2.446631in}{1.949974in}}%
\pgfpathclose%
\pgfusepath{stroke,fill}%
\end{pgfscope}%
\begin{pgfscope}%
\pgfpathrectangle{\pgfqpoint{0.600000in}{0.600000in}}{\pgfqpoint{3.900000in}{3.900000in}}%
\pgfusepath{clip}%
\pgfsetbuttcap%
\pgfsetroundjoin%
\definecolor{currentfill}{rgb}{0.121569,0.466667,0.705882}%
\pgfsetfillcolor{currentfill}%
\pgfsetlinewidth{1.003750pt}%
\definecolor{currentstroke}{rgb}{0.121569,0.466667,0.705882}%
\pgfsetstrokecolor{currentstroke}%
\pgfsetdash{}{0pt}%
\pgfpathmoveto{\pgfqpoint{2.287512in}{2.567636in}}%
\pgfpathcurveto{\pgfqpoint{2.298562in}{2.567636in}}{\pgfqpoint{2.309161in}{2.572026in}}{\pgfqpoint{2.316974in}{2.579840in}}%
\pgfpathcurveto{\pgfqpoint{2.324788in}{2.587654in}}{\pgfqpoint{2.329178in}{2.598253in}}{\pgfqpoint{2.329178in}{2.609303in}}%
\pgfpathcurveto{\pgfqpoint{2.329178in}{2.620353in}}{\pgfqpoint{2.324788in}{2.630952in}}{\pgfqpoint{2.316974in}{2.638765in}}%
\pgfpathcurveto{\pgfqpoint{2.309161in}{2.646579in}}{\pgfqpoint{2.298562in}{2.650969in}}{\pgfqpoint{2.287512in}{2.650969in}}%
\pgfpathcurveto{\pgfqpoint{2.276461in}{2.650969in}}{\pgfqpoint{2.265862in}{2.646579in}}{\pgfqpoint{2.258049in}{2.638765in}}%
\pgfpathcurveto{\pgfqpoint{2.250235in}{2.630952in}}{\pgfqpoint{2.245845in}{2.620353in}}{\pgfqpoint{2.245845in}{2.609303in}}%
\pgfpathcurveto{\pgfqpoint{2.245845in}{2.598253in}}{\pgfqpoint{2.250235in}{2.587654in}}{\pgfqpoint{2.258049in}{2.579840in}}%
\pgfpathcurveto{\pgfqpoint{2.265862in}{2.572026in}}{\pgfqpoint{2.276461in}{2.567636in}}{\pgfqpoint{2.287512in}{2.567636in}}%
\pgfpathclose%
\pgfusepath{stroke,fill}%
\end{pgfscope}%
\begin{pgfscope}%
\pgfpathrectangle{\pgfqpoint{0.600000in}{0.600000in}}{\pgfqpoint{3.900000in}{3.900000in}}%
\pgfusepath{clip}%
\pgfsetbuttcap%
\pgfsetroundjoin%
\definecolor{currentfill}{rgb}{0.121569,0.466667,0.705882}%
\pgfsetfillcolor{currentfill}%
\pgfsetlinewidth{1.003750pt}%
\definecolor{currentstroke}{rgb}{0.121569,0.466667,0.705882}%
\pgfsetstrokecolor{currentstroke}%
\pgfsetdash{}{0pt}%
\pgfpathmoveto{\pgfqpoint{2.808365in}{1.766670in}}%
\pgfpathcurveto{\pgfqpoint{2.819415in}{1.766670in}}{\pgfqpoint{2.830014in}{1.771060in}}{\pgfqpoint{2.837828in}{1.778874in}}%
\pgfpathcurveto{\pgfqpoint{2.845642in}{1.786688in}}{\pgfqpoint{2.850032in}{1.797287in}}{\pgfqpoint{2.850032in}{1.808337in}}%
\pgfpathcurveto{\pgfqpoint{2.850032in}{1.819387in}}{\pgfqpoint{2.845642in}{1.829986in}}{\pgfqpoint{2.837828in}{1.837799in}}%
\pgfpathcurveto{\pgfqpoint{2.830014in}{1.845613in}}{\pgfqpoint{2.819415in}{1.850003in}}{\pgfqpoint{2.808365in}{1.850003in}}%
\pgfpathcurveto{\pgfqpoint{2.797315in}{1.850003in}}{\pgfqpoint{2.786716in}{1.845613in}}{\pgfqpoint{2.778902in}{1.837799in}}%
\pgfpathcurveto{\pgfqpoint{2.771089in}{1.829986in}}{\pgfqpoint{2.766698in}{1.819387in}}{\pgfqpoint{2.766698in}{1.808337in}}%
\pgfpathcurveto{\pgfqpoint{2.766698in}{1.797287in}}{\pgfqpoint{2.771089in}{1.786688in}}{\pgfqpoint{2.778902in}{1.778874in}}%
\pgfpathcurveto{\pgfqpoint{2.786716in}{1.771060in}}{\pgfqpoint{2.797315in}{1.766670in}}{\pgfqpoint{2.808365in}{1.766670in}}%
\pgfpathclose%
\pgfusepath{stroke,fill}%
\end{pgfscope}%
\begin{pgfscope}%
\pgfpathrectangle{\pgfqpoint{0.600000in}{0.600000in}}{\pgfqpoint{3.900000in}{3.900000in}}%
\pgfusepath{clip}%
\pgfsetbuttcap%
\pgfsetroundjoin%
\definecolor{currentfill}{rgb}{0.121569,0.466667,0.705882}%
\pgfsetfillcolor{currentfill}%
\pgfsetlinewidth{1.003750pt}%
\definecolor{currentstroke}{rgb}{0.121569,0.466667,0.705882}%
\pgfsetstrokecolor{currentstroke}%
\pgfsetdash{}{0pt}%
\pgfpathmoveto{\pgfqpoint{1.618654in}{1.680215in}}%
\pgfpathcurveto{\pgfqpoint{1.629704in}{1.680215in}}{\pgfqpoint{1.640303in}{1.684605in}}{\pgfqpoint{1.648117in}{1.692418in}}%
\pgfpathcurveto{\pgfqpoint{1.655930in}{1.700232in}}{\pgfqpoint{1.660320in}{1.710831in}}{\pgfqpoint{1.660320in}{1.721881in}}%
\pgfpathcurveto{\pgfqpoint{1.660320in}{1.732931in}}{\pgfqpoint{1.655930in}{1.743530in}}{\pgfqpoint{1.648117in}{1.751344in}}%
\pgfpathcurveto{\pgfqpoint{1.640303in}{1.759158in}}{\pgfqpoint{1.629704in}{1.763548in}}{\pgfqpoint{1.618654in}{1.763548in}}%
\pgfpathcurveto{\pgfqpoint{1.607604in}{1.763548in}}{\pgfqpoint{1.597005in}{1.759158in}}{\pgfqpoint{1.589191in}{1.751344in}}%
\pgfpathcurveto{\pgfqpoint{1.581377in}{1.743530in}}{\pgfqpoint{1.576987in}{1.732931in}}{\pgfqpoint{1.576987in}{1.721881in}}%
\pgfpathcurveto{\pgfqpoint{1.576987in}{1.710831in}}{\pgfqpoint{1.581377in}{1.700232in}}{\pgfqpoint{1.589191in}{1.692418in}}%
\pgfpathcurveto{\pgfqpoint{1.597005in}{1.684605in}}{\pgfqpoint{1.607604in}{1.680215in}}{\pgfqpoint{1.618654in}{1.680215in}}%
\pgfpathclose%
\pgfusepath{stroke,fill}%
\end{pgfscope}%
\begin{pgfscope}%
\pgfpathrectangle{\pgfqpoint{0.600000in}{0.600000in}}{\pgfqpoint{3.900000in}{3.900000in}}%
\pgfusepath{clip}%
\pgfsetbuttcap%
\pgfsetroundjoin%
\definecolor{currentfill}{rgb}{0.121569,0.466667,0.705882}%
\pgfsetfillcolor{currentfill}%
\pgfsetlinewidth{1.003750pt}%
\definecolor{currentstroke}{rgb}{0.121569,0.466667,0.705882}%
\pgfsetstrokecolor{currentstroke}%
\pgfsetdash{}{0pt}%
\pgfpathmoveto{\pgfqpoint{2.710923in}{3.037433in}}%
\pgfpathcurveto{\pgfqpoint{2.721973in}{3.037433in}}{\pgfqpoint{2.732572in}{3.041823in}}{\pgfqpoint{2.740386in}{3.049637in}}%
\pgfpathcurveto{\pgfqpoint{2.748200in}{3.057450in}}{\pgfqpoint{2.752590in}{3.068050in}}{\pgfqpoint{2.752590in}{3.079100in}}%
\pgfpathcurveto{\pgfqpoint{2.752590in}{3.090150in}}{\pgfqpoint{2.748200in}{3.100749in}}{\pgfqpoint{2.740386in}{3.108562in}}%
\pgfpathcurveto{\pgfqpoint{2.732572in}{3.116376in}}{\pgfqpoint{2.721973in}{3.120766in}}{\pgfqpoint{2.710923in}{3.120766in}}%
\pgfpathcurveto{\pgfqpoint{2.699873in}{3.120766in}}{\pgfqpoint{2.689274in}{3.116376in}}{\pgfqpoint{2.681460in}{3.108562in}}%
\pgfpathcurveto{\pgfqpoint{2.673647in}{3.100749in}}{\pgfqpoint{2.669256in}{3.090150in}}{\pgfqpoint{2.669256in}{3.079100in}}%
\pgfpathcurveto{\pgfqpoint{2.669256in}{3.068050in}}{\pgfqpoint{2.673647in}{3.057450in}}{\pgfqpoint{2.681460in}{3.049637in}}%
\pgfpathcurveto{\pgfqpoint{2.689274in}{3.041823in}}{\pgfqpoint{2.699873in}{3.037433in}}{\pgfqpoint{2.710923in}{3.037433in}}%
\pgfpathclose%
\pgfusepath{stroke,fill}%
\end{pgfscope}%
\begin{pgfscope}%
\pgfpathrectangle{\pgfqpoint{0.600000in}{0.600000in}}{\pgfqpoint{3.900000in}{3.900000in}}%
\pgfusepath{clip}%
\pgfsetbuttcap%
\pgfsetroundjoin%
\definecolor{currentfill}{rgb}{0.121569,0.466667,0.705882}%
\pgfsetfillcolor{currentfill}%
\pgfsetlinewidth{1.003750pt}%
\definecolor{currentstroke}{rgb}{0.121569,0.466667,0.705882}%
\pgfsetstrokecolor{currentstroke}%
\pgfsetdash{}{0pt}%
\pgfpathmoveto{\pgfqpoint{2.539640in}{2.073808in}}%
\pgfpathcurveto{\pgfqpoint{2.550690in}{2.073808in}}{\pgfqpoint{2.561289in}{2.078198in}}{\pgfqpoint{2.569103in}{2.086012in}}%
\pgfpathcurveto{\pgfqpoint{2.576916in}{2.093825in}}{\pgfqpoint{2.581306in}{2.104424in}}{\pgfqpoint{2.581306in}{2.115474in}}%
\pgfpathcurveto{\pgfqpoint{2.581306in}{2.126525in}}{\pgfqpoint{2.576916in}{2.137124in}}{\pgfqpoint{2.569103in}{2.144937in}}%
\pgfpathcurveto{\pgfqpoint{2.561289in}{2.152751in}}{\pgfqpoint{2.550690in}{2.157141in}}{\pgfqpoint{2.539640in}{2.157141in}}%
\pgfpathcurveto{\pgfqpoint{2.528590in}{2.157141in}}{\pgfqpoint{2.517991in}{2.152751in}}{\pgfqpoint{2.510177in}{2.144937in}}%
\pgfpathcurveto{\pgfqpoint{2.502363in}{2.137124in}}{\pgfqpoint{2.497973in}{2.126525in}}{\pgfqpoint{2.497973in}{2.115474in}}%
\pgfpathcurveto{\pgfqpoint{2.497973in}{2.104424in}}{\pgfqpoint{2.502363in}{2.093825in}}{\pgfqpoint{2.510177in}{2.086012in}}%
\pgfpathcurveto{\pgfqpoint{2.517991in}{2.078198in}}{\pgfqpoint{2.528590in}{2.073808in}}{\pgfqpoint{2.539640in}{2.073808in}}%
\pgfpathclose%
\pgfusepath{stroke,fill}%
\end{pgfscope}%
\begin{pgfscope}%
\pgfpathrectangle{\pgfqpoint{0.600000in}{0.600000in}}{\pgfqpoint{3.900000in}{3.900000in}}%
\pgfusepath{clip}%
\pgfsetbuttcap%
\pgfsetroundjoin%
\definecolor{currentfill}{rgb}{0.121569,0.466667,0.705882}%
\pgfsetfillcolor{currentfill}%
\pgfsetlinewidth{1.003750pt}%
\definecolor{currentstroke}{rgb}{0.121569,0.466667,0.705882}%
\pgfsetstrokecolor{currentstroke}%
\pgfsetdash{}{0pt}%
\pgfpathmoveto{\pgfqpoint{2.581136in}{3.528507in}}%
\pgfpathcurveto{\pgfqpoint{2.592187in}{3.528507in}}{\pgfqpoint{2.602786in}{3.532897in}}{\pgfqpoint{2.610599in}{3.540711in}}%
\pgfpathcurveto{\pgfqpoint{2.618413in}{3.548524in}}{\pgfqpoint{2.622803in}{3.559123in}}{\pgfqpoint{2.622803in}{3.570174in}}%
\pgfpathcurveto{\pgfqpoint{2.622803in}{3.581224in}}{\pgfqpoint{2.618413in}{3.591823in}}{\pgfqpoint{2.610599in}{3.599636in}}%
\pgfpathcurveto{\pgfqpoint{2.602786in}{3.607450in}}{\pgfqpoint{2.592187in}{3.611840in}}{\pgfqpoint{2.581136in}{3.611840in}}%
\pgfpathcurveto{\pgfqpoint{2.570086in}{3.611840in}}{\pgfqpoint{2.559487in}{3.607450in}}{\pgfqpoint{2.551674in}{3.599636in}}%
\pgfpathcurveto{\pgfqpoint{2.543860in}{3.591823in}}{\pgfqpoint{2.539470in}{3.581224in}}{\pgfqpoint{2.539470in}{3.570174in}}%
\pgfpathcurveto{\pgfqpoint{2.539470in}{3.559123in}}{\pgfqpoint{2.543860in}{3.548524in}}{\pgfqpoint{2.551674in}{3.540711in}}%
\pgfpathcurveto{\pgfqpoint{2.559487in}{3.532897in}}{\pgfqpoint{2.570086in}{3.528507in}}{\pgfqpoint{2.581136in}{3.528507in}}%
\pgfpathclose%
\pgfusepath{stroke,fill}%
\end{pgfscope}%
\begin{pgfscope}%
\pgfpathrectangle{\pgfqpoint{0.600000in}{0.600000in}}{\pgfqpoint{3.900000in}{3.900000in}}%
\pgfusepath{clip}%
\pgfsetbuttcap%
\pgfsetroundjoin%
\definecolor{currentfill}{rgb}{0.121569,0.466667,0.705882}%
\pgfsetfillcolor{currentfill}%
\pgfsetlinewidth{1.003750pt}%
\definecolor{currentstroke}{rgb}{0.121569,0.466667,0.705882}%
\pgfsetstrokecolor{currentstroke}%
\pgfsetdash{}{0pt}%
\pgfpathmoveto{\pgfqpoint{3.040009in}{2.434920in}}%
\pgfpathcurveto{\pgfqpoint{3.051059in}{2.434920in}}{\pgfqpoint{3.061659in}{2.439310in}}{\pgfqpoint{3.069472in}{2.447124in}}%
\pgfpathcurveto{\pgfqpoint{3.077286in}{2.454937in}}{\pgfqpoint{3.081676in}{2.465536in}}{\pgfqpoint{3.081676in}{2.476586in}}%
\pgfpathcurveto{\pgfqpoint{3.081676in}{2.487636in}}{\pgfqpoint{3.077286in}{2.498235in}}{\pgfqpoint{3.069472in}{2.506049in}}%
\pgfpathcurveto{\pgfqpoint{3.061659in}{2.513863in}}{\pgfqpoint{3.051059in}{2.518253in}}{\pgfqpoint{3.040009in}{2.518253in}}%
\pgfpathcurveto{\pgfqpoint{3.028959in}{2.518253in}}{\pgfqpoint{3.018360in}{2.513863in}}{\pgfqpoint{3.010547in}{2.506049in}}%
\pgfpathcurveto{\pgfqpoint{3.002733in}{2.498235in}}{\pgfqpoint{2.998343in}{2.487636in}}{\pgfqpoint{2.998343in}{2.476586in}}%
\pgfpathcurveto{\pgfqpoint{2.998343in}{2.465536in}}{\pgfqpoint{3.002733in}{2.454937in}}{\pgfqpoint{3.010547in}{2.447124in}}%
\pgfpathcurveto{\pgfqpoint{3.018360in}{2.439310in}}{\pgfqpoint{3.028959in}{2.434920in}}{\pgfqpoint{3.040009in}{2.434920in}}%
\pgfpathclose%
\pgfusepath{stroke,fill}%
\end{pgfscope}%
\begin{pgfscope}%
\pgfpathrectangle{\pgfqpoint{0.600000in}{0.600000in}}{\pgfqpoint{3.900000in}{3.900000in}}%
\pgfusepath{clip}%
\pgfsetbuttcap%
\pgfsetroundjoin%
\definecolor{currentfill}{rgb}{0.121569,0.466667,0.705882}%
\pgfsetfillcolor{currentfill}%
\pgfsetlinewidth{1.003750pt}%
\definecolor{currentstroke}{rgb}{0.121569,0.466667,0.705882}%
\pgfsetstrokecolor{currentstroke}%
\pgfsetdash{}{0pt}%
\pgfpathmoveto{\pgfqpoint{2.635443in}{3.132789in}}%
\pgfpathcurveto{\pgfqpoint{2.646493in}{3.132789in}}{\pgfqpoint{2.657092in}{3.137179in}}{\pgfqpoint{2.664906in}{3.144993in}}%
\pgfpathcurveto{\pgfqpoint{2.672719in}{3.152806in}}{\pgfqpoint{2.677110in}{3.163405in}}{\pgfqpoint{2.677110in}{3.174456in}}%
\pgfpathcurveto{\pgfqpoint{2.677110in}{3.185506in}}{\pgfqpoint{2.672719in}{3.196105in}}{\pgfqpoint{2.664906in}{3.203918in}}%
\pgfpathcurveto{\pgfqpoint{2.657092in}{3.211732in}}{\pgfqpoint{2.646493in}{3.216122in}}{\pgfqpoint{2.635443in}{3.216122in}}%
\pgfpathcurveto{\pgfqpoint{2.624393in}{3.216122in}}{\pgfqpoint{2.613794in}{3.211732in}}{\pgfqpoint{2.605980in}{3.203918in}}%
\pgfpathcurveto{\pgfqpoint{2.598166in}{3.196105in}}{\pgfqpoint{2.593776in}{3.185506in}}{\pgfqpoint{2.593776in}{3.174456in}}%
\pgfpathcurveto{\pgfqpoint{2.593776in}{3.163405in}}{\pgfqpoint{2.598166in}{3.152806in}}{\pgfqpoint{2.605980in}{3.144993in}}%
\pgfpathcurveto{\pgfqpoint{2.613794in}{3.137179in}}{\pgfqpoint{2.624393in}{3.132789in}}{\pgfqpoint{2.635443in}{3.132789in}}%
\pgfpathclose%
\pgfusepath{stroke,fill}%
\end{pgfscope}%
\begin{pgfscope}%
\pgfpathrectangle{\pgfqpoint{0.600000in}{0.600000in}}{\pgfqpoint{3.900000in}{3.900000in}}%
\pgfusepath{clip}%
\pgfsetbuttcap%
\pgfsetroundjoin%
\definecolor{currentfill}{rgb}{0.121569,0.466667,0.705882}%
\pgfsetfillcolor{currentfill}%
\pgfsetlinewidth{1.003750pt}%
\definecolor{currentstroke}{rgb}{0.121569,0.466667,0.705882}%
\pgfsetstrokecolor{currentstroke}%
\pgfsetdash{}{0pt}%
\pgfpathmoveto{\pgfqpoint{2.960465in}{2.026199in}}%
\pgfpathcurveto{\pgfqpoint{2.971515in}{2.026199in}}{\pgfqpoint{2.982114in}{2.030590in}}{\pgfqpoint{2.989928in}{2.038403in}}%
\pgfpathcurveto{\pgfqpoint{2.997741in}{2.046217in}}{\pgfqpoint{3.002132in}{2.056816in}}{\pgfqpoint{3.002132in}{2.067866in}}%
\pgfpathcurveto{\pgfqpoint{3.002132in}{2.078916in}}{\pgfqpoint{2.997741in}{2.089515in}}{\pgfqpoint{2.989928in}{2.097329in}}%
\pgfpathcurveto{\pgfqpoint{2.982114in}{2.105142in}}{\pgfqpoint{2.971515in}{2.109533in}}{\pgfqpoint{2.960465in}{2.109533in}}%
\pgfpathcurveto{\pgfqpoint{2.949415in}{2.109533in}}{\pgfqpoint{2.938816in}{2.105142in}}{\pgfqpoint{2.931002in}{2.097329in}}%
\pgfpathcurveto{\pgfqpoint{2.923189in}{2.089515in}}{\pgfqpoint{2.918798in}{2.078916in}}{\pgfqpoint{2.918798in}{2.067866in}}%
\pgfpathcurveto{\pgfqpoint{2.918798in}{2.056816in}}{\pgfqpoint{2.923189in}{2.046217in}}{\pgfqpoint{2.931002in}{2.038403in}}%
\pgfpathcurveto{\pgfqpoint{2.938816in}{2.030590in}}{\pgfqpoint{2.949415in}{2.026199in}}{\pgfqpoint{2.960465in}{2.026199in}}%
\pgfpathclose%
\pgfusepath{stroke,fill}%
\end{pgfscope}%
\begin{pgfscope}%
\pgfpathrectangle{\pgfqpoint{0.600000in}{0.600000in}}{\pgfqpoint{3.900000in}{3.900000in}}%
\pgfusepath{clip}%
\pgfsetbuttcap%
\pgfsetroundjoin%
\definecolor{currentfill}{rgb}{0.121569,0.466667,0.705882}%
\pgfsetfillcolor{currentfill}%
\pgfsetlinewidth{1.003750pt}%
\definecolor{currentstroke}{rgb}{0.121569,0.466667,0.705882}%
\pgfsetstrokecolor{currentstroke}%
\pgfsetdash{}{0pt}%
\pgfpathmoveto{\pgfqpoint{2.602960in}{2.278709in}}%
\pgfpathcurveto{\pgfqpoint{2.614010in}{2.278709in}}{\pgfqpoint{2.624609in}{2.283099in}}{\pgfqpoint{2.632423in}{2.290913in}}%
\pgfpathcurveto{\pgfqpoint{2.640237in}{2.298727in}}{\pgfqpoint{2.644627in}{2.309326in}}{\pgfqpoint{2.644627in}{2.320376in}}%
\pgfpathcurveto{\pgfqpoint{2.644627in}{2.331426in}}{\pgfqpoint{2.640237in}{2.342025in}}{\pgfqpoint{2.632423in}{2.349838in}}%
\pgfpathcurveto{\pgfqpoint{2.624609in}{2.357652in}}{\pgfqpoint{2.614010in}{2.362042in}}{\pgfqpoint{2.602960in}{2.362042in}}%
\pgfpathcurveto{\pgfqpoint{2.591910in}{2.362042in}}{\pgfqpoint{2.581311in}{2.357652in}}{\pgfqpoint{2.573497in}{2.349838in}}%
\pgfpathcurveto{\pgfqpoint{2.565684in}{2.342025in}}{\pgfqpoint{2.561293in}{2.331426in}}{\pgfqpoint{2.561293in}{2.320376in}}%
\pgfpathcurveto{\pgfqpoint{2.561293in}{2.309326in}}{\pgfqpoint{2.565684in}{2.298727in}}{\pgfqpoint{2.573497in}{2.290913in}}%
\pgfpathcurveto{\pgfqpoint{2.581311in}{2.283099in}}{\pgfqpoint{2.591910in}{2.278709in}}{\pgfqpoint{2.602960in}{2.278709in}}%
\pgfpathclose%
\pgfusepath{stroke,fill}%
\end{pgfscope}%
\begin{pgfscope}%
\pgfpathrectangle{\pgfqpoint{0.600000in}{0.600000in}}{\pgfqpoint{3.900000in}{3.900000in}}%
\pgfusepath{clip}%
\pgfsetbuttcap%
\pgfsetroundjoin%
\definecolor{currentfill}{rgb}{0.121569,0.466667,0.705882}%
\pgfsetfillcolor{currentfill}%
\pgfsetlinewidth{1.003750pt}%
\definecolor{currentstroke}{rgb}{0.121569,0.466667,0.705882}%
\pgfsetstrokecolor{currentstroke}%
\pgfsetdash{}{0pt}%
\pgfpathmoveto{\pgfqpoint{2.602629in}{2.231664in}}%
\pgfpathcurveto{\pgfqpoint{2.613679in}{2.231664in}}{\pgfqpoint{2.624279in}{2.236055in}}{\pgfqpoint{2.632092in}{2.243868in}}%
\pgfpathcurveto{\pgfqpoint{2.639906in}{2.251682in}}{\pgfqpoint{2.644296in}{2.262281in}}{\pgfqpoint{2.644296in}{2.273331in}}%
\pgfpathcurveto{\pgfqpoint{2.644296in}{2.284381in}}{\pgfqpoint{2.639906in}{2.294980in}}{\pgfqpoint{2.632092in}{2.302794in}}%
\pgfpathcurveto{\pgfqpoint{2.624279in}{2.310607in}}{\pgfqpoint{2.613679in}{2.314998in}}{\pgfqpoint{2.602629in}{2.314998in}}%
\pgfpathcurveto{\pgfqpoint{2.591579in}{2.314998in}}{\pgfqpoint{2.580980in}{2.310607in}}{\pgfqpoint{2.573167in}{2.302794in}}%
\pgfpathcurveto{\pgfqpoint{2.565353in}{2.294980in}}{\pgfqpoint{2.560963in}{2.284381in}}{\pgfqpoint{2.560963in}{2.273331in}}%
\pgfpathcurveto{\pgfqpoint{2.560963in}{2.262281in}}{\pgfqpoint{2.565353in}{2.251682in}}{\pgfqpoint{2.573167in}{2.243868in}}%
\pgfpathcurveto{\pgfqpoint{2.580980in}{2.236055in}}{\pgfqpoint{2.591579in}{2.231664in}}{\pgfqpoint{2.602629in}{2.231664in}}%
\pgfpathclose%
\pgfusepath{stroke,fill}%
\end{pgfscope}%
\begin{pgfscope}%
\pgfpathrectangle{\pgfqpoint{0.600000in}{0.600000in}}{\pgfqpoint{3.900000in}{3.900000in}}%
\pgfusepath{clip}%
\pgfsetbuttcap%
\pgfsetroundjoin%
\definecolor{currentfill}{rgb}{0.121569,0.466667,0.705882}%
\pgfsetfillcolor{currentfill}%
\pgfsetlinewidth{1.003750pt}%
\definecolor{currentstroke}{rgb}{0.121569,0.466667,0.705882}%
\pgfsetstrokecolor{currentstroke}%
\pgfsetdash{}{0pt}%
\pgfpathmoveto{\pgfqpoint{2.648596in}{2.424986in}}%
\pgfpathcurveto{\pgfqpoint{2.659647in}{2.424986in}}{\pgfqpoint{2.670246in}{2.429376in}}{\pgfqpoint{2.678059in}{2.437190in}}%
\pgfpathcurveto{\pgfqpoint{2.685873in}{2.445004in}}{\pgfqpoint{2.690263in}{2.455603in}}{\pgfqpoint{2.690263in}{2.466653in}}%
\pgfpathcurveto{\pgfqpoint{2.690263in}{2.477703in}}{\pgfqpoint{2.685873in}{2.488302in}}{\pgfqpoint{2.678059in}{2.496116in}}%
\pgfpathcurveto{\pgfqpoint{2.670246in}{2.503929in}}{\pgfqpoint{2.659647in}{2.508319in}}{\pgfqpoint{2.648596in}{2.508319in}}%
\pgfpathcurveto{\pgfqpoint{2.637546in}{2.508319in}}{\pgfqpoint{2.626947in}{2.503929in}}{\pgfqpoint{2.619134in}{2.496116in}}%
\pgfpathcurveto{\pgfqpoint{2.611320in}{2.488302in}}{\pgfqpoint{2.606930in}{2.477703in}}{\pgfqpoint{2.606930in}{2.466653in}}%
\pgfpathcurveto{\pgfqpoint{2.606930in}{2.455603in}}{\pgfqpoint{2.611320in}{2.445004in}}{\pgfqpoint{2.619134in}{2.437190in}}%
\pgfpathcurveto{\pgfqpoint{2.626947in}{2.429376in}}{\pgfqpoint{2.637546in}{2.424986in}}{\pgfqpoint{2.648596in}{2.424986in}}%
\pgfpathclose%
\pgfusepath{stroke,fill}%
\end{pgfscope}%
\begin{pgfscope}%
\pgfpathrectangle{\pgfqpoint{0.600000in}{0.600000in}}{\pgfqpoint{3.900000in}{3.900000in}}%
\pgfusepath{clip}%
\pgfsetbuttcap%
\pgfsetroundjoin%
\definecolor{currentfill}{rgb}{0.121569,0.466667,0.705882}%
\pgfsetfillcolor{currentfill}%
\pgfsetlinewidth{1.003750pt}%
\definecolor{currentstroke}{rgb}{0.121569,0.466667,0.705882}%
\pgfsetstrokecolor{currentstroke}%
\pgfsetdash{}{0pt}%
\pgfpathmoveto{\pgfqpoint{2.400509in}{2.109513in}}%
\pgfpathcurveto{\pgfqpoint{2.411559in}{2.109513in}}{\pgfqpoint{2.422158in}{2.113903in}}{\pgfqpoint{2.429971in}{2.121717in}}%
\pgfpathcurveto{\pgfqpoint{2.437785in}{2.129531in}}{\pgfqpoint{2.442175in}{2.140130in}}{\pgfqpoint{2.442175in}{2.151180in}}%
\pgfpathcurveto{\pgfqpoint{2.442175in}{2.162230in}}{\pgfqpoint{2.437785in}{2.172829in}}{\pgfqpoint{2.429971in}{2.180643in}}%
\pgfpathcurveto{\pgfqpoint{2.422158in}{2.188456in}}{\pgfqpoint{2.411559in}{2.192846in}}{\pgfqpoint{2.400509in}{2.192846in}}%
\pgfpathcurveto{\pgfqpoint{2.389458in}{2.192846in}}{\pgfqpoint{2.378859in}{2.188456in}}{\pgfqpoint{2.371046in}{2.180643in}}%
\pgfpathcurveto{\pgfqpoint{2.363232in}{2.172829in}}{\pgfqpoint{2.358842in}{2.162230in}}{\pgfqpoint{2.358842in}{2.151180in}}%
\pgfpathcurveto{\pgfqpoint{2.358842in}{2.140130in}}{\pgfqpoint{2.363232in}{2.129531in}}{\pgfqpoint{2.371046in}{2.121717in}}%
\pgfpathcurveto{\pgfqpoint{2.378859in}{2.113903in}}{\pgfqpoint{2.389458in}{2.109513in}}{\pgfqpoint{2.400509in}{2.109513in}}%
\pgfpathclose%
\pgfusepath{stroke,fill}%
\end{pgfscope}%
\begin{pgfscope}%
\pgfpathrectangle{\pgfqpoint{0.600000in}{0.600000in}}{\pgfqpoint{3.900000in}{3.900000in}}%
\pgfusepath{clip}%
\pgfsetbuttcap%
\pgfsetroundjoin%
\definecolor{currentfill}{rgb}{0.121569,0.466667,0.705882}%
\pgfsetfillcolor{currentfill}%
\pgfsetlinewidth{1.003750pt}%
\definecolor{currentstroke}{rgb}{0.121569,0.466667,0.705882}%
\pgfsetstrokecolor{currentstroke}%
\pgfsetdash{}{0pt}%
\pgfpathmoveto{\pgfqpoint{2.694652in}{2.767584in}}%
\pgfpathcurveto{\pgfqpoint{2.705702in}{2.767584in}}{\pgfqpoint{2.716301in}{2.771974in}}{\pgfqpoint{2.724115in}{2.779788in}}%
\pgfpathcurveto{\pgfqpoint{2.731929in}{2.787602in}}{\pgfqpoint{2.736319in}{2.798201in}}{\pgfqpoint{2.736319in}{2.809251in}}%
\pgfpathcurveto{\pgfqpoint{2.736319in}{2.820301in}}{\pgfqpoint{2.731929in}{2.830900in}}{\pgfqpoint{2.724115in}{2.838714in}}%
\pgfpathcurveto{\pgfqpoint{2.716301in}{2.846527in}}{\pgfqpoint{2.705702in}{2.850918in}}{\pgfqpoint{2.694652in}{2.850918in}}%
\pgfpathcurveto{\pgfqpoint{2.683602in}{2.850918in}}{\pgfqpoint{2.673003in}{2.846527in}}{\pgfqpoint{2.665189in}{2.838714in}}%
\pgfpathcurveto{\pgfqpoint{2.657376in}{2.830900in}}{\pgfqpoint{2.652986in}{2.820301in}}{\pgfqpoint{2.652986in}{2.809251in}}%
\pgfpathcurveto{\pgfqpoint{2.652986in}{2.798201in}}{\pgfqpoint{2.657376in}{2.787602in}}{\pgfqpoint{2.665189in}{2.779788in}}%
\pgfpathcurveto{\pgfqpoint{2.673003in}{2.771974in}}{\pgfqpoint{2.683602in}{2.767584in}}{\pgfqpoint{2.694652in}{2.767584in}}%
\pgfpathclose%
\pgfusepath{stroke,fill}%
\end{pgfscope}%
\begin{pgfscope}%
\pgfpathrectangle{\pgfqpoint{0.600000in}{0.600000in}}{\pgfqpoint{3.900000in}{3.900000in}}%
\pgfusepath{clip}%
\pgfsetbuttcap%
\pgfsetroundjoin%
\definecolor{currentfill}{rgb}{0.121569,0.466667,0.705882}%
\pgfsetfillcolor{currentfill}%
\pgfsetlinewidth{1.003750pt}%
\definecolor{currentstroke}{rgb}{0.121569,0.466667,0.705882}%
\pgfsetstrokecolor{currentstroke}%
\pgfsetdash{}{0pt}%
\pgfpathmoveto{\pgfqpoint{2.498254in}{1.706300in}}%
\pgfpathcurveto{\pgfqpoint{2.509304in}{1.706300in}}{\pgfqpoint{2.519903in}{1.710690in}}{\pgfqpoint{2.527717in}{1.718504in}}%
\pgfpathcurveto{\pgfqpoint{2.535531in}{1.726317in}}{\pgfqpoint{2.539921in}{1.736916in}}{\pgfqpoint{2.539921in}{1.747966in}}%
\pgfpathcurveto{\pgfqpoint{2.539921in}{1.759017in}}{\pgfqpoint{2.535531in}{1.769616in}}{\pgfqpoint{2.527717in}{1.777429in}}%
\pgfpathcurveto{\pgfqpoint{2.519903in}{1.785243in}}{\pgfqpoint{2.509304in}{1.789633in}}{\pgfqpoint{2.498254in}{1.789633in}}%
\pgfpathcurveto{\pgfqpoint{2.487204in}{1.789633in}}{\pgfqpoint{2.476605in}{1.785243in}}{\pgfqpoint{2.468791in}{1.777429in}}%
\pgfpathcurveto{\pgfqpoint{2.460978in}{1.769616in}}{\pgfqpoint{2.456588in}{1.759017in}}{\pgfqpoint{2.456588in}{1.747966in}}%
\pgfpathcurveto{\pgfqpoint{2.456588in}{1.736916in}}{\pgfqpoint{2.460978in}{1.726317in}}{\pgfqpoint{2.468791in}{1.718504in}}%
\pgfpathcurveto{\pgfqpoint{2.476605in}{1.710690in}}{\pgfqpoint{2.487204in}{1.706300in}}{\pgfqpoint{2.498254in}{1.706300in}}%
\pgfpathclose%
\pgfusepath{stroke,fill}%
\end{pgfscope}%
\begin{pgfscope}%
\pgfpathrectangle{\pgfqpoint{0.600000in}{0.600000in}}{\pgfqpoint{3.900000in}{3.900000in}}%
\pgfusepath{clip}%
\pgfsetbuttcap%
\pgfsetroundjoin%
\definecolor{currentfill}{rgb}{0.121569,0.466667,0.705882}%
\pgfsetfillcolor{currentfill}%
\pgfsetlinewidth{1.003750pt}%
\definecolor{currentstroke}{rgb}{0.121569,0.466667,0.705882}%
\pgfsetstrokecolor{currentstroke}%
\pgfsetdash{}{0pt}%
\pgfpathmoveto{\pgfqpoint{2.315140in}{2.745724in}}%
\pgfpathcurveto{\pgfqpoint{2.326190in}{2.745724in}}{\pgfqpoint{2.336789in}{2.750114in}}{\pgfqpoint{2.344602in}{2.757928in}}%
\pgfpathcurveto{\pgfqpoint{2.352416in}{2.765741in}}{\pgfqpoint{2.356806in}{2.776340in}}{\pgfqpoint{2.356806in}{2.787390in}}%
\pgfpathcurveto{\pgfqpoint{2.356806in}{2.798441in}}{\pgfqpoint{2.352416in}{2.809040in}}{\pgfqpoint{2.344602in}{2.816853in}}%
\pgfpathcurveto{\pgfqpoint{2.336789in}{2.824667in}}{\pgfqpoint{2.326190in}{2.829057in}}{\pgfqpoint{2.315140in}{2.829057in}}%
\pgfpathcurveto{\pgfqpoint{2.304089in}{2.829057in}}{\pgfqpoint{2.293490in}{2.824667in}}{\pgfqpoint{2.285677in}{2.816853in}}%
\pgfpathcurveto{\pgfqpoint{2.277863in}{2.809040in}}{\pgfqpoint{2.273473in}{2.798441in}}{\pgfqpoint{2.273473in}{2.787390in}}%
\pgfpathcurveto{\pgfqpoint{2.273473in}{2.776340in}}{\pgfqpoint{2.277863in}{2.765741in}}{\pgfqpoint{2.285677in}{2.757928in}}%
\pgfpathcurveto{\pgfqpoint{2.293490in}{2.750114in}}{\pgfqpoint{2.304089in}{2.745724in}}{\pgfqpoint{2.315140in}{2.745724in}}%
\pgfpathclose%
\pgfusepath{stroke,fill}%
\end{pgfscope}%
\begin{pgfscope}%
\pgfpathrectangle{\pgfqpoint{0.600000in}{0.600000in}}{\pgfqpoint{3.900000in}{3.900000in}}%
\pgfusepath{clip}%
\pgfsetbuttcap%
\pgfsetroundjoin%
\definecolor{currentfill}{rgb}{0.121569,0.466667,0.705882}%
\pgfsetfillcolor{currentfill}%
\pgfsetlinewidth{1.003750pt}%
\definecolor{currentstroke}{rgb}{0.121569,0.466667,0.705882}%
\pgfsetstrokecolor{currentstroke}%
\pgfsetdash{}{0pt}%
\pgfpathmoveto{\pgfqpoint{2.731780in}{3.226457in}}%
\pgfpathcurveto{\pgfqpoint{2.742830in}{3.226457in}}{\pgfqpoint{2.753429in}{3.230847in}}{\pgfqpoint{2.761243in}{3.238661in}}%
\pgfpathcurveto{\pgfqpoint{2.769057in}{3.246475in}}{\pgfqpoint{2.773447in}{3.257074in}}{\pgfqpoint{2.773447in}{3.268124in}}%
\pgfpathcurveto{\pgfqpoint{2.773447in}{3.279174in}}{\pgfqpoint{2.769057in}{3.289773in}}{\pgfqpoint{2.761243in}{3.297587in}}%
\pgfpathcurveto{\pgfqpoint{2.753429in}{3.305400in}}{\pgfqpoint{2.742830in}{3.309791in}}{\pgfqpoint{2.731780in}{3.309791in}}%
\pgfpathcurveto{\pgfqpoint{2.720730in}{3.309791in}}{\pgfqpoint{2.710131in}{3.305400in}}{\pgfqpoint{2.702317in}{3.297587in}}%
\pgfpathcurveto{\pgfqpoint{2.694504in}{3.289773in}}{\pgfqpoint{2.690113in}{3.279174in}}{\pgfqpoint{2.690113in}{3.268124in}}%
\pgfpathcurveto{\pgfqpoint{2.690113in}{3.257074in}}{\pgfqpoint{2.694504in}{3.246475in}}{\pgfqpoint{2.702317in}{3.238661in}}%
\pgfpathcurveto{\pgfqpoint{2.710131in}{3.230847in}}{\pgfqpoint{2.720730in}{3.226457in}}{\pgfqpoint{2.731780in}{3.226457in}}%
\pgfpathclose%
\pgfusepath{stroke,fill}%
\end{pgfscope}%
\begin{pgfscope}%
\pgfpathrectangle{\pgfqpoint{0.600000in}{0.600000in}}{\pgfqpoint{3.900000in}{3.900000in}}%
\pgfusepath{clip}%
\pgfsetbuttcap%
\pgfsetroundjoin%
\definecolor{currentfill}{rgb}{0.121569,0.466667,0.705882}%
\pgfsetfillcolor{currentfill}%
\pgfsetlinewidth{1.003750pt}%
\definecolor{currentstroke}{rgb}{0.121569,0.466667,0.705882}%
\pgfsetstrokecolor{currentstroke}%
\pgfsetdash{}{0pt}%
\pgfpathmoveto{\pgfqpoint{3.256057in}{2.203584in}}%
\pgfpathcurveto{\pgfqpoint{3.267107in}{2.203584in}}{\pgfqpoint{3.277706in}{2.207974in}}{\pgfqpoint{3.285519in}{2.215788in}}%
\pgfpathcurveto{\pgfqpoint{3.293333in}{2.223601in}}{\pgfqpoint{3.297723in}{2.234200in}}{\pgfqpoint{3.297723in}{2.245251in}}%
\pgfpathcurveto{\pgfqpoint{3.297723in}{2.256301in}}{\pgfqpoint{3.293333in}{2.266900in}}{\pgfqpoint{3.285519in}{2.274713in}}%
\pgfpathcurveto{\pgfqpoint{3.277706in}{2.282527in}}{\pgfqpoint{3.267107in}{2.286917in}}{\pgfqpoint{3.256057in}{2.286917in}}%
\pgfpathcurveto{\pgfqpoint{3.245006in}{2.286917in}}{\pgfqpoint{3.234407in}{2.282527in}}{\pgfqpoint{3.226594in}{2.274713in}}%
\pgfpathcurveto{\pgfqpoint{3.218780in}{2.266900in}}{\pgfqpoint{3.214390in}{2.256301in}}{\pgfqpoint{3.214390in}{2.245251in}}%
\pgfpathcurveto{\pgfqpoint{3.214390in}{2.234200in}}{\pgfqpoint{3.218780in}{2.223601in}}{\pgfqpoint{3.226594in}{2.215788in}}%
\pgfpathcurveto{\pgfqpoint{3.234407in}{2.207974in}}{\pgfqpoint{3.245006in}{2.203584in}}{\pgfqpoint{3.256057in}{2.203584in}}%
\pgfpathclose%
\pgfusepath{stroke,fill}%
\end{pgfscope}%
\begin{pgfscope}%
\pgfpathrectangle{\pgfqpoint{0.600000in}{0.600000in}}{\pgfqpoint{3.900000in}{3.900000in}}%
\pgfusepath{clip}%
\pgfsetbuttcap%
\pgfsetroundjoin%
\definecolor{currentfill}{rgb}{0.121569,0.466667,0.705882}%
\pgfsetfillcolor{currentfill}%
\pgfsetlinewidth{1.003750pt}%
\definecolor{currentstroke}{rgb}{0.121569,0.466667,0.705882}%
\pgfsetstrokecolor{currentstroke}%
\pgfsetdash{}{0pt}%
\pgfpathmoveto{\pgfqpoint{2.367027in}{2.430338in}}%
\pgfpathcurveto{\pgfqpoint{2.378077in}{2.430338in}}{\pgfqpoint{2.388676in}{2.434729in}}{\pgfqpoint{2.396489in}{2.442542in}}%
\pgfpathcurveto{\pgfqpoint{2.404303in}{2.450356in}}{\pgfqpoint{2.408693in}{2.460955in}}{\pgfqpoint{2.408693in}{2.472005in}}%
\pgfpathcurveto{\pgfqpoint{2.408693in}{2.483055in}}{\pgfqpoint{2.404303in}{2.493654in}}{\pgfqpoint{2.396489in}{2.501468in}}%
\pgfpathcurveto{\pgfqpoint{2.388676in}{2.509282in}}{\pgfqpoint{2.378077in}{2.513672in}}{\pgfqpoint{2.367027in}{2.513672in}}%
\pgfpathcurveto{\pgfqpoint{2.355977in}{2.513672in}}{\pgfqpoint{2.345378in}{2.509282in}}{\pgfqpoint{2.337564in}{2.501468in}}%
\pgfpathcurveto{\pgfqpoint{2.329750in}{2.493654in}}{\pgfqpoint{2.325360in}{2.483055in}}{\pgfqpoint{2.325360in}{2.472005in}}%
\pgfpathcurveto{\pgfqpoint{2.325360in}{2.460955in}}{\pgfqpoint{2.329750in}{2.450356in}}{\pgfqpoint{2.337564in}{2.442542in}}%
\pgfpathcurveto{\pgfqpoint{2.345378in}{2.434729in}}{\pgfqpoint{2.355977in}{2.430338in}}{\pgfqpoint{2.367027in}{2.430338in}}%
\pgfpathclose%
\pgfusepath{stroke,fill}%
\end{pgfscope}%
\begin{pgfscope}%
\pgfpathrectangle{\pgfqpoint{0.600000in}{0.600000in}}{\pgfqpoint{3.900000in}{3.900000in}}%
\pgfusepath{clip}%
\pgfsetbuttcap%
\pgfsetroundjoin%
\definecolor{currentfill}{rgb}{0.121569,0.466667,0.705882}%
\pgfsetfillcolor{currentfill}%
\pgfsetlinewidth{1.003750pt}%
\definecolor{currentstroke}{rgb}{0.121569,0.466667,0.705882}%
\pgfsetstrokecolor{currentstroke}%
\pgfsetdash{}{0pt}%
\pgfpathmoveto{\pgfqpoint{2.431637in}{2.403370in}}%
\pgfpathcurveto{\pgfqpoint{2.442688in}{2.403370in}}{\pgfqpoint{2.453287in}{2.407760in}}{\pgfqpoint{2.461100in}{2.415574in}}%
\pgfpathcurveto{\pgfqpoint{2.468914in}{2.423388in}}{\pgfqpoint{2.473304in}{2.433987in}}{\pgfqpoint{2.473304in}{2.445037in}}%
\pgfpathcurveto{\pgfqpoint{2.473304in}{2.456087in}}{\pgfqpoint{2.468914in}{2.466686in}}{\pgfqpoint{2.461100in}{2.474499in}}%
\pgfpathcurveto{\pgfqpoint{2.453287in}{2.482313in}}{\pgfqpoint{2.442688in}{2.486703in}}{\pgfqpoint{2.431637in}{2.486703in}}%
\pgfpathcurveto{\pgfqpoint{2.420587in}{2.486703in}}{\pgfqpoint{2.409988in}{2.482313in}}{\pgfqpoint{2.402175in}{2.474499in}}%
\pgfpathcurveto{\pgfqpoint{2.394361in}{2.466686in}}{\pgfqpoint{2.389971in}{2.456087in}}{\pgfqpoint{2.389971in}{2.445037in}}%
\pgfpathcurveto{\pgfqpoint{2.389971in}{2.433987in}}{\pgfqpoint{2.394361in}{2.423388in}}{\pgfqpoint{2.402175in}{2.415574in}}%
\pgfpathcurveto{\pgfqpoint{2.409988in}{2.407760in}}{\pgfqpoint{2.420587in}{2.403370in}}{\pgfqpoint{2.431637in}{2.403370in}}%
\pgfpathclose%
\pgfusepath{stroke,fill}%
\end{pgfscope}%
\begin{pgfscope}%
\pgfpathrectangle{\pgfqpoint{0.600000in}{0.600000in}}{\pgfqpoint{3.900000in}{3.900000in}}%
\pgfusepath{clip}%
\pgfsetbuttcap%
\pgfsetroundjoin%
\definecolor{currentfill}{rgb}{0.121569,0.466667,0.705882}%
\pgfsetfillcolor{currentfill}%
\pgfsetlinewidth{1.003750pt}%
\definecolor{currentstroke}{rgb}{0.121569,0.466667,0.705882}%
\pgfsetstrokecolor{currentstroke}%
\pgfsetdash{}{0pt}%
\pgfpathmoveto{\pgfqpoint{2.338007in}{2.203575in}}%
\pgfpathcurveto{\pgfqpoint{2.349057in}{2.203575in}}{\pgfqpoint{2.359656in}{2.207965in}}{\pgfqpoint{2.367470in}{2.215779in}}%
\pgfpathcurveto{\pgfqpoint{2.375283in}{2.223592in}}{\pgfqpoint{2.379674in}{2.234191in}}{\pgfqpoint{2.379674in}{2.245242in}}%
\pgfpathcurveto{\pgfqpoint{2.379674in}{2.256292in}}{\pgfqpoint{2.375283in}{2.266891in}}{\pgfqpoint{2.367470in}{2.274704in}}%
\pgfpathcurveto{\pgfqpoint{2.359656in}{2.282518in}}{\pgfqpoint{2.349057in}{2.286908in}}{\pgfqpoint{2.338007in}{2.286908in}}%
\pgfpathcurveto{\pgfqpoint{2.326957in}{2.286908in}}{\pgfqpoint{2.316358in}{2.282518in}}{\pgfqpoint{2.308544in}{2.274704in}}%
\pgfpathcurveto{\pgfqpoint{2.300731in}{2.266891in}}{\pgfqpoint{2.296340in}{2.256292in}}{\pgfqpoint{2.296340in}{2.245242in}}%
\pgfpathcurveto{\pgfqpoint{2.296340in}{2.234191in}}{\pgfqpoint{2.300731in}{2.223592in}}{\pgfqpoint{2.308544in}{2.215779in}}%
\pgfpathcurveto{\pgfqpoint{2.316358in}{2.207965in}}{\pgfqpoint{2.326957in}{2.203575in}}{\pgfqpoint{2.338007in}{2.203575in}}%
\pgfpathclose%
\pgfusepath{stroke,fill}%
\end{pgfscope}%
\begin{pgfscope}%
\pgfpathrectangle{\pgfqpoint{0.600000in}{0.600000in}}{\pgfqpoint{3.900000in}{3.900000in}}%
\pgfusepath{clip}%
\pgfsetbuttcap%
\pgfsetroundjoin%
\definecolor{currentfill}{rgb}{0.121569,0.466667,0.705882}%
\pgfsetfillcolor{currentfill}%
\pgfsetlinewidth{1.003750pt}%
\definecolor{currentstroke}{rgb}{0.121569,0.466667,0.705882}%
\pgfsetstrokecolor{currentstroke}%
\pgfsetdash{}{0pt}%
\pgfpathmoveto{\pgfqpoint{2.640252in}{2.679155in}}%
\pgfpathcurveto{\pgfqpoint{2.651302in}{2.679155in}}{\pgfqpoint{2.661901in}{2.683545in}}{\pgfqpoint{2.669715in}{2.691359in}}%
\pgfpathcurveto{\pgfqpoint{2.677528in}{2.699172in}}{\pgfqpoint{2.681918in}{2.709772in}}{\pgfqpoint{2.681918in}{2.720822in}}%
\pgfpathcurveto{\pgfqpoint{2.681918in}{2.731872in}}{\pgfqpoint{2.677528in}{2.742471in}}{\pgfqpoint{2.669715in}{2.750284in}}%
\pgfpathcurveto{\pgfqpoint{2.661901in}{2.758098in}}{\pgfqpoint{2.651302in}{2.762488in}}{\pgfqpoint{2.640252in}{2.762488in}}%
\pgfpathcurveto{\pgfqpoint{2.629202in}{2.762488in}}{\pgfqpoint{2.618603in}{2.758098in}}{\pgfqpoint{2.610789in}{2.750284in}}%
\pgfpathcurveto{\pgfqpoint{2.602975in}{2.742471in}}{\pgfqpoint{2.598585in}{2.731872in}}{\pgfqpoint{2.598585in}{2.720822in}}%
\pgfpathcurveto{\pgfqpoint{2.598585in}{2.709772in}}{\pgfqpoint{2.602975in}{2.699172in}}{\pgfqpoint{2.610789in}{2.691359in}}%
\pgfpathcurveto{\pgfqpoint{2.618603in}{2.683545in}}{\pgfqpoint{2.629202in}{2.679155in}}{\pgfqpoint{2.640252in}{2.679155in}}%
\pgfpathclose%
\pgfusepath{stroke,fill}%
\end{pgfscope}%
\begin{pgfscope}%
\pgfpathrectangle{\pgfqpoint{0.600000in}{0.600000in}}{\pgfqpoint{3.900000in}{3.900000in}}%
\pgfusepath{clip}%
\pgfsetbuttcap%
\pgfsetroundjoin%
\definecolor{currentfill}{rgb}{0.121569,0.466667,0.705882}%
\pgfsetfillcolor{currentfill}%
\pgfsetlinewidth{1.003750pt}%
\definecolor{currentstroke}{rgb}{0.121569,0.466667,0.705882}%
\pgfsetstrokecolor{currentstroke}%
\pgfsetdash{}{0pt}%
\pgfpathmoveto{\pgfqpoint{2.108834in}{2.990215in}}%
\pgfpathcurveto{\pgfqpoint{2.119884in}{2.990215in}}{\pgfqpoint{2.130483in}{2.994605in}}{\pgfqpoint{2.138297in}{3.002418in}}%
\pgfpathcurveto{\pgfqpoint{2.146110in}{3.010232in}}{\pgfqpoint{2.150500in}{3.020831in}}{\pgfqpoint{2.150500in}{3.031881in}}%
\pgfpathcurveto{\pgfqpoint{2.150500in}{3.042931in}}{\pgfqpoint{2.146110in}{3.053530in}}{\pgfqpoint{2.138297in}{3.061344in}}%
\pgfpathcurveto{\pgfqpoint{2.130483in}{3.069158in}}{\pgfqpoint{2.119884in}{3.073548in}}{\pgfqpoint{2.108834in}{3.073548in}}%
\pgfpathcurveto{\pgfqpoint{2.097784in}{3.073548in}}{\pgfqpoint{2.087185in}{3.069158in}}{\pgfqpoint{2.079371in}{3.061344in}}%
\pgfpathcurveto{\pgfqpoint{2.071557in}{3.053530in}}{\pgfqpoint{2.067167in}{3.042931in}}{\pgfqpoint{2.067167in}{3.031881in}}%
\pgfpathcurveto{\pgfqpoint{2.067167in}{3.020831in}}{\pgfqpoint{2.071557in}{3.010232in}}{\pgfqpoint{2.079371in}{3.002418in}}%
\pgfpathcurveto{\pgfqpoint{2.087185in}{2.994605in}}{\pgfqpoint{2.097784in}{2.990215in}}{\pgfqpoint{2.108834in}{2.990215in}}%
\pgfpathclose%
\pgfusepath{stroke,fill}%
\end{pgfscope}%
\begin{pgfscope}%
\pgfpathrectangle{\pgfqpoint{0.600000in}{0.600000in}}{\pgfqpoint{3.900000in}{3.900000in}}%
\pgfusepath{clip}%
\pgfsetbuttcap%
\pgfsetroundjoin%
\definecolor{currentfill}{rgb}{0.121569,0.466667,0.705882}%
\pgfsetfillcolor{currentfill}%
\pgfsetlinewidth{1.003750pt}%
\definecolor{currentstroke}{rgb}{0.121569,0.466667,0.705882}%
\pgfsetstrokecolor{currentstroke}%
\pgfsetdash{}{0pt}%
\pgfpathmoveto{\pgfqpoint{2.712015in}{2.752670in}}%
\pgfpathcurveto{\pgfqpoint{2.723065in}{2.752670in}}{\pgfqpoint{2.733664in}{2.757061in}}{\pgfqpoint{2.741478in}{2.764874in}}%
\pgfpathcurveto{\pgfqpoint{2.749292in}{2.772688in}}{\pgfqpoint{2.753682in}{2.783287in}}{\pgfqpoint{2.753682in}{2.794337in}}%
\pgfpathcurveto{\pgfqpoint{2.753682in}{2.805387in}}{\pgfqpoint{2.749292in}{2.815986in}}{\pgfqpoint{2.741478in}{2.823800in}}%
\pgfpathcurveto{\pgfqpoint{2.733664in}{2.831614in}}{\pgfqpoint{2.723065in}{2.836004in}}{\pgfqpoint{2.712015in}{2.836004in}}%
\pgfpathcurveto{\pgfqpoint{2.700965in}{2.836004in}}{\pgfqpoint{2.690366in}{2.831614in}}{\pgfqpoint{2.682553in}{2.823800in}}%
\pgfpathcurveto{\pgfqpoint{2.674739in}{2.815986in}}{\pgfqpoint{2.670349in}{2.805387in}}{\pgfqpoint{2.670349in}{2.794337in}}%
\pgfpathcurveto{\pgfqpoint{2.670349in}{2.783287in}}{\pgfqpoint{2.674739in}{2.772688in}}{\pgfqpoint{2.682553in}{2.764874in}}%
\pgfpathcurveto{\pgfqpoint{2.690366in}{2.757061in}}{\pgfqpoint{2.700965in}{2.752670in}}{\pgfqpoint{2.712015in}{2.752670in}}%
\pgfpathclose%
\pgfusepath{stroke,fill}%
\end{pgfscope}%
\begin{pgfscope}%
\pgfpathrectangle{\pgfqpoint{0.600000in}{0.600000in}}{\pgfqpoint{3.900000in}{3.900000in}}%
\pgfusepath{clip}%
\pgfsetbuttcap%
\pgfsetroundjoin%
\definecolor{currentfill}{rgb}{0.121569,0.466667,0.705882}%
\pgfsetfillcolor{currentfill}%
\pgfsetlinewidth{1.003750pt}%
\definecolor{currentstroke}{rgb}{0.121569,0.466667,0.705882}%
\pgfsetstrokecolor{currentstroke}%
\pgfsetdash{}{0pt}%
\pgfpathmoveto{\pgfqpoint{2.210795in}{2.799451in}}%
\pgfpathcurveto{\pgfqpoint{2.221846in}{2.799451in}}{\pgfqpoint{2.232445in}{2.803841in}}{\pgfqpoint{2.240258in}{2.811655in}}%
\pgfpathcurveto{\pgfqpoint{2.248072in}{2.819468in}}{\pgfqpoint{2.252462in}{2.830067in}}{\pgfqpoint{2.252462in}{2.841118in}}%
\pgfpathcurveto{\pgfqpoint{2.252462in}{2.852168in}}{\pgfqpoint{2.248072in}{2.862767in}}{\pgfqpoint{2.240258in}{2.870580in}}%
\pgfpathcurveto{\pgfqpoint{2.232445in}{2.878394in}}{\pgfqpoint{2.221846in}{2.882784in}}{\pgfqpoint{2.210795in}{2.882784in}}%
\pgfpathcurveto{\pgfqpoint{2.199745in}{2.882784in}}{\pgfqpoint{2.189146in}{2.878394in}}{\pgfqpoint{2.181333in}{2.870580in}}%
\pgfpathcurveto{\pgfqpoint{2.173519in}{2.862767in}}{\pgfqpoint{2.169129in}{2.852168in}}{\pgfqpoint{2.169129in}{2.841118in}}%
\pgfpathcurveto{\pgfqpoint{2.169129in}{2.830067in}}{\pgfqpoint{2.173519in}{2.819468in}}{\pgfqpoint{2.181333in}{2.811655in}}%
\pgfpathcurveto{\pgfqpoint{2.189146in}{2.803841in}}{\pgfqpoint{2.199745in}{2.799451in}}{\pgfqpoint{2.210795in}{2.799451in}}%
\pgfpathclose%
\pgfusepath{stroke,fill}%
\end{pgfscope}%
\begin{pgfscope}%
\pgfpathrectangle{\pgfqpoint{0.600000in}{0.600000in}}{\pgfqpoint{3.900000in}{3.900000in}}%
\pgfusepath{clip}%
\pgfsetbuttcap%
\pgfsetroundjoin%
\definecolor{currentfill}{rgb}{0.121569,0.466667,0.705882}%
\pgfsetfillcolor{currentfill}%
\pgfsetlinewidth{1.003750pt}%
\definecolor{currentstroke}{rgb}{0.121569,0.466667,0.705882}%
\pgfsetstrokecolor{currentstroke}%
\pgfsetdash{}{0pt}%
\pgfpathmoveto{\pgfqpoint{2.527780in}{2.679699in}}%
\pgfpathcurveto{\pgfqpoint{2.538830in}{2.679699in}}{\pgfqpoint{2.549429in}{2.684089in}}{\pgfqpoint{2.557242in}{2.691903in}}%
\pgfpathcurveto{\pgfqpoint{2.565056in}{2.699716in}}{\pgfqpoint{2.569446in}{2.710315in}}{\pgfqpoint{2.569446in}{2.721365in}}%
\pgfpathcurveto{\pgfqpoint{2.569446in}{2.732416in}}{\pgfqpoint{2.565056in}{2.743015in}}{\pgfqpoint{2.557242in}{2.750828in}}%
\pgfpathcurveto{\pgfqpoint{2.549429in}{2.758642in}}{\pgfqpoint{2.538830in}{2.763032in}}{\pgfqpoint{2.527780in}{2.763032in}}%
\pgfpathcurveto{\pgfqpoint{2.516730in}{2.763032in}}{\pgfqpoint{2.506131in}{2.758642in}}{\pgfqpoint{2.498317in}{2.750828in}}%
\pgfpathcurveto{\pgfqpoint{2.490503in}{2.743015in}}{\pgfqpoint{2.486113in}{2.732416in}}{\pgfqpoint{2.486113in}{2.721365in}}%
\pgfpathcurveto{\pgfqpoint{2.486113in}{2.710315in}}{\pgfqpoint{2.490503in}{2.699716in}}{\pgfqpoint{2.498317in}{2.691903in}}%
\pgfpathcurveto{\pgfqpoint{2.506131in}{2.684089in}}{\pgfqpoint{2.516730in}{2.679699in}}{\pgfqpoint{2.527780in}{2.679699in}}%
\pgfpathclose%
\pgfusepath{stroke,fill}%
\end{pgfscope}%
\begin{pgfscope}%
\pgfpathrectangle{\pgfqpoint{0.600000in}{0.600000in}}{\pgfqpoint{3.900000in}{3.900000in}}%
\pgfusepath{clip}%
\pgfsetbuttcap%
\pgfsetroundjoin%
\definecolor{currentfill}{rgb}{0.121569,0.466667,0.705882}%
\pgfsetfillcolor{currentfill}%
\pgfsetlinewidth{1.003750pt}%
\definecolor{currentstroke}{rgb}{0.121569,0.466667,0.705882}%
\pgfsetstrokecolor{currentstroke}%
\pgfsetdash{}{0pt}%
\pgfpathmoveto{\pgfqpoint{2.597884in}{2.659018in}}%
\pgfpathcurveto{\pgfqpoint{2.608934in}{2.659018in}}{\pgfqpoint{2.619533in}{2.663408in}}{\pgfqpoint{2.627347in}{2.671222in}}%
\pgfpathcurveto{\pgfqpoint{2.635161in}{2.679035in}}{\pgfqpoint{2.639551in}{2.689634in}}{\pgfqpoint{2.639551in}{2.700685in}}%
\pgfpathcurveto{\pgfqpoint{2.639551in}{2.711735in}}{\pgfqpoint{2.635161in}{2.722334in}}{\pgfqpoint{2.627347in}{2.730147in}}%
\pgfpathcurveto{\pgfqpoint{2.619533in}{2.737961in}}{\pgfqpoint{2.608934in}{2.742351in}}{\pgfqpoint{2.597884in}{2.742351in}}%
\pgfpathcurveto{\pgfqpoint{2.586834in}{2.742351in}}{\pgfqpoint{2.576235in}{2.737961in}}{\pgfqpoint{2.568421in}{2.730147in}}%
\pgfpathcurveto{\pgfqpoint{2.560608in}{2.722334in}}{\pgfqpoint{2.556218in}{2.711735in}}{\pgfqpoint{2.556218in}{2.700685in}}%
\pgfpathcurveto{\pgfqpoint{2.556218in}{2.689634in}}{\pgfqpoint{2.560608in}{2.679035in}}{\pgfqpoint{2.568421in}{2.671222in}}%
\pgfpathcurveto{\pgfqpoint{2.576235in}{2.663408in}}{\pgfqpoint{2.586834in}{2.659018in}}{\pgfqpoint{2.597884in}{2.659018in}}%
\pgfpathclose%
\pgfusepath{stroke,fill}%
\end{pgfscope}%
\begin{pgfscope}%
\pgfpathrectangle{\pgfqpoint{0.600000in}{0.600000in}}{\pgfqpoint{3.900000in}{3.900000in}}%
\pgfusepath{clip}%
\pgfsetbuttcap%
\pgfsetroundjoin%
\definecolor{currentfill}{rgb}{0.121569,0.466667,0.705882}%
\pgfsetfillcolor{currentfill}%
\pgfsetlinewidth{1.003750pt}%
\definecolor{currentstroke}{rgb}{0.121569,0.466667,0.705882}%
\pgfsetstrokecolor{currentstroke}%
\pgfsetdash{}{0pt}%
\pgfpathmoveto{\pgfqpoint{2.142332in}{2.933299in}}%
\pgfpathcurveto{\pgfqpoint{2.153382in}{2.933299in}}{\pgfqpoint{2.163981in}{2.937689in}}{\pgfqpoint{2.171795in}{2.945503in}}%
\pgfpathcurveto{\pgfqpoint{2.179608in}{2.953316in}}{\pgfqpoint{2.183999in}{2.963916in}}{\pgfqpoint{2.183999in}{2.974966in}}%
\pgfpathcurveto{\pgfqpoint{2.183999in}{2.986016in}}{\pgfqpoint{2.179608in}{2.996615in}}{\pgfqpoint{2.171795in}{3.004428in}}%
\pgfpathcurveto{\pgfqpoint{2.163981in}{3.012242in}}{\pgfqpoint{2.153382in}{3.016632in}}{\pgfqpoint{2.142332in}{3.016632in}}%
\pgfpathcurveto{\pgfqpoint{2.131282in}{3.016632in}}{\pgfqpoint{2.120683in}{3.012242in}}{\pgfqpoint{2.112869in}{3.004428in}}%
\pgfpathcurveto{\pgfqpoint{2.105056in}{2.996615in}}{\pgfqpoint{2.100665in}{2.986016in}}{\pgfqpoint{2.100665in}{2.974966in}}%
\pgfpathcurveto{\pgfqpoint{2.100665in}{2.963916in}}{\pgfqpoint{2.105056in}{2.953316in}}{\pgfqpoint{2.112869in}{2.945503in}}%
\pgfpathcurveto{\pgfqpoint{2.120683in}{2.937689in}}{\pgfqpoint{2.131282in}{2.933299in}}{\pgfqpoint{2.142332in}{2.933299in}}%
\pgfpathclose%
\pgfusepath{stroke,fill}%
\end{pgfscope}%
\begin{pgfscope}%
\pgfpathrectangle{\pgfqpoint{0.600000in}{0.600000in}}{\pgfqpoint{3.900000in}{3.900000in}}%
\pgfusepath{clip}%
\pgfsetbuttcap%
\pgfsetroundjoin%
\definecolor{currentfill}{rgb}{0.121569,0.466667,0.705882}%
\pgfsetfillcolor{currentfill}%
\pgfsetlinewidth{1.003750pt}%
\definecolor{currentstroke}{rgb}{0.121569,0.466667,0.705882}%
\pgfsetstrokecolor{currentstroke}%
\pgfsetdash{}{0pt}%
\pgfpathmoveto{\pgfqpoint{2.640282in}{2.521288in}}%
\pgfpathcurveto{\pgfqpoint{2.651332in}{2.521288in}}{\pgfqpoint{2.661931in}{2.525678in}}{\pgfqpoint{2.669745in}{2.533492in}}%
\pgfpathcurveto{\pgfqpoint{2.677558in}{2.541305in}}{\pgfqpoint{2.681949in}{2.551904in}}{\pgfqpoint{2.681949in}{2.562954in}}%
\pgfpathcurveto{\pgfqpoint{2.681949in}{2.574004in}}{\pgfqpoint{2.677558in}{2.584603in}}{\pgfqpoint{2.669745in}{2.592417in}}%
\pgfpathcurveto{\pgfqpoint{2.661931in}{2.600231in}}{\pgfqpoint{2.651332in}{2.604621in}}{\pgfqpoint{2.640282in}{2.604621in}}%
\pgfpathcurveto{\pgfqpoint{2.629232in}{2.604621in}}{\pgfqpoint{2.618633in}{2.600231in}}{\pgfqpoint{2.610819in}{2.592417in}}%
\pgfpathcurveto{\pgfqpoint{2.603006in}{2.584603in}}{\pgfqpoint{2.598615in}{2.574004in}}{\pgfqpoint{2.598615in}{2.562954in}}%
\pgfpathcurveto{\pgfqpoint{2.598615in}{2.551904in}}{\pgfqpoint{2.603006in}{2.541305in}}{\pgfqpoint{2.610819in}{2.533492in}}%
\pgfpathcurveto{\pgfqpoint{2.618633in}{2.525678in}}{\pgfqpoint{2.629232in}{2.521288in}}{\pgfqpoint{2.640282in}{2.521288in}}%
\pgfpathclose%
\pgfusepath{stroke,fill}%
\end{pgfscope}%
\begin{pgfscope}%
\pgfpathrectangle{\pgfqpoint{0.600000in}{0.600000in}}{\pgfqpoint{3.900000in}{3.900000in}}%
\pgfusepath{clip}%
\pgfsetbuttcap%
\pgfsetroundjoin%
\definecolor{currentfill}{rgb}{0.121569,0.466667,0.705882}%
\pgfsetfillcolor{currentfill}%
\pgfsetlinewidth{1.003750pt}%
\definecolor{currentstroke}{rgb}{0.121569,0.466667,0.705882}%
\pgfsetstrokecolor{currentstroke}%
\pgfsetdash{}{0pt}%
\pgfpathmoveto{\pgfqpoint{2.078857in}{2.665589in}}%
\pgfpathcurveto{\pgfqpoint{2.089907in}{2.665589in}}{\pgfqpoint{2.100506in}{2.669979in}}{\pgfqpoint{2.108319in}{2.677793in}}%
\pgfpathcurveto{\pgfqpoint{2.116133in}{2.685607in}}{\pgfqpoint{2.120523in}{2.696206in}}{\pgfqpoint{2.120523in}{2.707256in}}%
\pgfpathcurveto{\pgfqpoint{2.120523in}{2.718306in}}{\pgfqpoint{2.116133in}{2.728905in}}{\pgfqpoint{2.108319in}{2.736719in}}%
\pgfpathcurveto{\pgfqpoint{2.100506in}{2.744532in}}{\pgfqpoint{2.089907in}{2.748922in}}{\pgfqpoint{2.078857in}{2.748922in}}%
\pgfpathcurveto{\pgfqpoint{2.067806in}{2.748922in}}{\pgfqpoint{2.057207in}{2.744532in}}{\pgfqpoint{2.049394in}{2.736719in}}%
\pgfpathcurveto{\pgfqpoint{2.041580in}{2.728905in}}{\pgfqpoint{2.037190in}{2.718306in}}{\pgfqpoint{2.037190in}{2.707256in}}%
\pgfpathcurveto{\pgfqpoint{2.037190in}{2.696206in}}{\pgfqpoint{2.041580in}{2.685607in}}{\pgfqpoint{2.049394in}{2.677793in}}%
\pgfpathcurveto{\pgfqpoint{2.057207in}{2.669979in}}{\pgfqpoint{2.067806in}{2.665589in}}{\pgfqpoint{2.078857in}{2.665589in}}%
\pgfpathclose%
\pgfusepath{stroke,fill}%
\end{pgfscope}%
\begin{pgfscope}%
\pgfpathrectangle{\pgfqpoint{0.600000in}{0.600000in}}{\pgfqpoint{3.900000in}{3.900000in}}%
\pgfusepath{clip}%
\pgfsetbuttcap%
\pgfsetroundjoin%
\definecolor{currentfill}{rgb}{0.121569,0.466667,0.705882}%
\pgfsetfillcolor{currentfill}%
\pgfsetlinewidth{1.003750pt}%
\definecolor{currentstroke}{rgb}{0.121569,0.466667,0.705882}%
\pgfsetstrokecolor{currentstroke}%
\pgfsetdash{}{0pt}%
\pgfpathmoveto{\pgfqpoint{2.815857in}{1.642146in}}%
\pgfpathcurveto{\pgfqpoint{2.826907in}{1.642146in}}{\pgfqpoint{2.837506in}{1.646536in}}{\pgfqpoint{2.845320in}{1.654349in}}%
\pgfpathcurveto{\pgfqpoint{2.853133in}{1.662163in}}{\pgfqpoint{2.857524in}{1.672762in}}{\pgfqpoint{2.857524in}{1.683812in}}%
\pgfpathcurveto{\pgfqpoint{2.857524in}{1.694862in}}{\pgfqpoint{2.853133in}{1.705461in}}{\pgfqpoint{2.845320in}{1.713275in}}%
\pgfpathcurveto{\pgfqpoint{2.837506in}{1.721089in}}{\pgfqpoint{2.826907in}{1.725479in}}{\pgfqpoint{2.815857in}{1.725479in}}%
\pgfpathcurveto{\pgfqpoint{2.804807in}{1.725479in}}{\pgfqpoint{2.794208in}{1.721089in}}{\pgfqpoint{2.786394in}{1.713275in}}%
\pgfpathcurveto{\pgfqpoint{2.778581in}{1.705461in}}{\pgfqpoint{2.774190in}{1.694862in}}{\pgfqpoint{2.774190in}{1.683812in}}%
\pgfpathcurveto{\pgfqpoint{2.774190in}{1.672762in}}{\pgfqpoint{2.778581in}{1.662163in}}{\pgfqpoint{2.786394in}{1.654349in}}%
\pgfpathcurveto{\pgfqpoint{2.794208in}{1.646536in}}{\pgfqpoint{2.804807in}{1.642146in}}{\pgfqpoint{2.815857in}{1.642146in}}%
\pgfpathclose%
\pgfusepath{stroke,fill}%
\end{pgfscope}%
\begin{pgfscope}%
\pgfpathrectangle{\pgfqpoint{0.600000in}{0.600000in}}{\pgfqpoint{3.900000in}{3.900000in}}%
\pgfusepath{clip}%
\pgfsetbuttcap%
\pgfsetroundjoin%
\definecolor{currentfill}{rgb}{0.121569,0.466667,0.705882}%
\pgfsetfillcolor{currentfill}%
\pgfsetlinewidth{1.003750pt}%
\definecolor{currentstroke}{rgb}{0.121569,0.466667,0.705882}%
\pgfsetstrokecolor{currentstroke}%
\pgfsetdash{}{0pt}%
\pgfpathmoveto{\pgfqpoint{2.804284in}{3.065622in}}%
\pgfpathcurveto{\pgfqpoint{2.815334in}{3.065622in}}{\pgfqpoint{2.825933in}{3.070012in}}{\pgfqpoint{2.833747in}{3.077826in}}%
\pgfpathcurveto{\pgfqpoint{2.841561in}{3.085639in}}{\pgfqpoint{2.845951in}{3.096238in}}{\pgfqpoint{2.845951in}{3.107288in}}%
\pgfpathcurveto{\pgfqpoint{2.845951in}{3.118339in}}{\pgfqpoint{2.841561in}{3.128938in}}{\pgfqpoint{2.833747in}{3.136751in}}%
\pgfpathcurveto{\pgfqpoint{2.825933in}{3.144565in}}{\pgfqpoint{2.815334in}{3.148955in}}{\pgfqpoint{2.804284in}{3.148955in}}%
\pgfpathcurveto{\pgfqpoint{2.793234in}{3.148955in}}{\pgfqpoint{2.782635in}{3.144565in}}{\pgfqpoint{2.774822in}{3.136751in}}%
\pgfpathcurveto{\pgfqpoint{2.767008in}{3.128938in}}{\pgfqpoint{2.762618in}{3.118339in}}{\pgfqpoint{2.762618in}{3.107288in}}%
\pgfpathcurveto{\pgfqpoint{2.762618in}{3.096238in}}{\pgfqpoint{2.767008in}{3.085639in}}{\pgfqpoint{2.774822in}{3.077826in}}%
\pgfpathcurveto{\pgfqpoint{2.782635in}{3.070012in}}{\pgfqpoint{2.793234in}{3.065622in}}{\pgfqpoint{2.804284in}{3.065622in}}%
\pgfpathclose%
\pgfusepath{stroke,fill}%
\end{pgfscope}%
\begin{pgfscope}%
\pgfpathrectangle{\pgfqpoint{0.600000in}{0.600000in}}{\pgfqpoint{3.900000in}{3.900000in}}%
\pgfusepath{clip}%
\pgfsetbuttcap%
\pgfsetroundjoin%
\definecolor{currentfill}{rgb}{0.121569,0.466667,0.705882}%
\pgfsetfillcolor{currentfill}%
\pgfsetlinewidth{1.003750pt}%
\definecolor{currentstroke}{rgb}{0.121569,0.466667,0.705882}%
\pgfsetstrokecolor{currentstroke}%
\pgfsetdash{}{0pt}%
\pgfpathmoveto{\pgfqpoint{2.850718in}{2.858154in}}%
\pgfpathcurveto{\pgfqpoint{2.861768in}{2.858154in}}{\pgfqpoint{2.872367in}{2.862544in}}{\pgfqpoint{2.880181in}{2.870358in}}%
\pgfpathcurveto{\pgfqpoint{2.887994in}{2.878171in}}{\pgfqpoint{2.892384in}{2.888770in}}{\pgfqpoint{2.892384in}{2.899820in}}%
\pgfpathcurveto{\pgfqpoint{2.892384in}{2.910871in}}{\pgfqpoint{2.887994in}{2.921470in}}{\pgfqpoint{2.880181in}{2.929283in}}%
\pgfpathcurveto{\pgfqpoint{2.872367in}{2.937097in}}{\pgfqpoint{2.861768in}{2.941487in}}{\pgfqpoint{2.850718in}{2.941487in}}%
\pgfpathcurveto{\pgfqpoint{2.839668in}{2.941487in}}{\pgfqpoint{2.829069in}{2.937097in}}{\pgfqpoint{2.821255in}{2.929283in}}%
\pgfpathcurveto{\pgfqpoint{2.813441in}{2.921470in}}{\pgfqpoint{2.809051in}{2.910871in}}{\pgfqpoint{2.809051in}{2.899820in}}%
\pgfpathcurveto{\pgfqpoint{2.809051in}{2.888770in}}{\pgfqpoint{2.813441in}{2.878171in}}{\pgfqpoint{2.821255in}{2.870358in}}%
\pgfpathcurveto{\pgfqpoint{2.829069in}{2.862544in}}{\pgfqpoint{2.839668in}{2.858154in}}{\pgfqpoint{2.850718in}{2.858154in}}%
\pgfpathclose%
\pgfusepath{stroke,fill}%
\end{pgfscope}%
\begin{pgfscope}%
\pgfpathrectangle{\pgfqpoint{0.600000in}{0.600000in}}{\pgfqpoint{3.900000in}{3.900000in}}%
\pgfusepath{clip}%
\pgfsetbuttcap%
\pgfsetroundjoin%
\definecolor{currentfill}{rgb}{0.121569,0.466667,0.705882}%
\pgfsetfillcolor{currentfill}%
\pgfsetlinewidth{1.003750pt}%
\definecolor{currentstroke}{rgb}{0.121569,0.466667,0.705882}%
\pgfsetstrokecolor{currentstroke}%
\pgfsetdash{}{0pt}%
\pgfpathmoveto{\pgfqpoint{2.788044in}{2.893233in}}%
\pgfpathcurveto{\pgfqpoint{2.799094in}{2.893233in}}{\pgfqpoint{2.809693in}{2.897623in}}{\pgfqpoint{2.817507in}{2.905437in}}%
\pgfpathcurveto{\pgfqpoint{2.825320in}{2.913250in}}{\pgfqpoint{2.829711in}{2.923850in}}{\pgfqpoint{2.829711in}{2.934900in}}%
\pgfpathcurveto{\pgfqpoint{2.829711in}{2.945950in}}{\pgfqpoint{2.825320in}{2.956549in}}{\pgfqpoint{2.817507in}{2.964362in}}%
\pgfpathcurveto{\pgfqpoint{2.809693in}{2.972176in}}{\pgfqpoint{2.799094in}{2.976566in}}{\pgfqpoint{2.788044in}{2.976566in}}%
\pgfpathcurveto{\pgfqpoint{2.776994in}{2.976566in}}{\pgfqpoint{2.766395in}{2.972176in}}{\pgfqpoint{2.758581in}{2.964362in}}%
\pgfpathcurveto{\pgfqpoint{2.750768in}{2.956549in}}{\pgfqpoint{2.746377in}{2.945950in}}{\pgfqpoint{2.746377in}{2.934900in}}%
\pgfpathcurveto{\pgfqpoint{2.746377in}{2.923850in}}{\pgfqpoint{2.750768in}{2.913250in}}{\pgfqpoint{2.758581in}{2.905437in}}%
\pgfpathcurveto{\pgfqpoint{2.766395in}{2.897623in}}{\pgfqpoint{2.776994in}{2.893233in}}{\pgfqpoint{2.788044in}{2.893233in}}%
\pgfpathclose%
\pgfusepath{stroke,fill}%
\end{pgfscope}%
\begin{pgfscope}%
\pgfpathrectangle{\pgfqpoint{0.600000in}{0.600000in}}{\pgfqpoint{3.900000in}{3.900000in}}%
\pgfusepath{clip}%
\pgfsetbuttcap%
\pgfsetroundjoin%
\definecolor{currentfill}{rgb}{0.121569,0.466667,0.705882}%
\pgfsetfillcolor{currentfill}%
\pgfsetlinewidth{1.003750pt}%
\definecolor{currentstroke}{rgb}{0.121569,0.466667,0.705882}%
\pgfsetstrokecolor{currentstroke}%
\pgfsetdash{}{0pt}%
\pgfpathmoveto{\pgfqpoint{1.902132in}{3.121544in}}%
\pgfpathcurveto{\pgfqpoint{1.913182in}{3.121544in}}{\pgfqpoint{1.923781in}{3.125935in}}{\pgfqpoint{1.931594in}{3.133748in}}%
\pgfpathcurveto{\pgfqpoint{1.939408in}{3.141562in}}{\pgfqpoint{1.943798in}{3.152161in}}{\pgfqpoint{1.943798in}{3.163211in}}%
\pgfpathcurveto{\pgfqpoint{1.943798in}{3.174261in}}{\pgfqpoint{1.939408in}{3.184860in}}{\pgfqpoint{1.931594in}{3.192674in}}%
\pgfpathcurveto{\pgfqpoint{1.923781in}{3.200487in}}{\pgfqpoint{1.913182in}{3.204878in}}{\pgfqpoint{1.902132in}{3.204878in}}%
\pgfpathcurveto{\pgfqpoint{1.891081in}{3.204878in}}{\pgfqpoint{1.880482in}{3.200487in}}{\pgfqpoint{1.872669in}{3.192674in}}%
\pgfpathcurveto{\pgfqpoint{1.864855in}{3.184860in}}{\pgfqpoint{1.860465in}{3.174261in}}{\pgfqpoint{1.860465in}{3.163211in}}%
\pgfpathcurveto{\pgfqpoint{1.860465in}{3.152161in}}{\pgfqpoint{1.864855in}{3.141562in}}{\pgfqpoint{1.872669in}{3.133748in}}%
\pgfpathcurveto{\pgfqpoint{1.880482in}{3.125935in}}{\pgfqpoint{1.891081in}{3.121544in}}{\pgfqpoint{1.902132in}{3.121544in}}%
\pgfpathclose%
\pgfusepath{stroke,fill}%
\end{pgfscope}%
\begin{pgfscope}%
\pgfpathrectangle{\pgfqpoint{0.600000in}{0.600000in}}{\pgfqpoint{3.900000in}{3.900000in}}%
\pgfusepath{clip}%
\pgfsetbuttcap%
\pgfsetroundjoin%
\definecolor{currentfill}{rgb}{0.121569,0.466667,0.705882}%
\pgfsetfillcolor{currentfill}%
\pgfsetlinewidth{1.003750pt}%
\definecolor{currentstroke}{rgb}{0.121569,0.466667,0.705882}%
\pgfsetstrokecolor{currentstroke}%
\pgfsetdash{}{0pt}%
\pgfpathmoveto{\pgfqpoint{2.529519in}{2.745051in}}%
\pgfpathcurveto{\pgfqpoint{2.540570in}{2.745051in}}{\pgfqpoint{2.551169in}{2.749442in}}{\pgfqpoint{2.558982in}{2.757255in}}%
\pgfpathcurveto{\pgfqpoint{2.566796in}{2.765069in}}{\pgfqpoint{2.571186in}{2.775668in}}{\pgfqpoint{2.571186in}{2.786718in}}%
\pgfpathcurveto{\pgfqpoint{2.571186in}{2.797768in}}{\pgfqpoint{2.566796in}{2.808367in}}{\pgfqpoint{2.558982in}{2.816181in}}%
\pgfpathcurveto{\pgfqpoint{2.551169in}{2.823995in}}{\pgfqpoint{2.540570in}{2.828385in}}{\pgfqpoint{2.529519in}{2.828385in}}%
\pgfpathcurveto{\pgfqpoint{2.518469in}{2.828385in}}{\pgfqpoint{2.507870in}{2.823995in}}{\pgfqpoint{2.500057in}{2.816181in}}%
\pgfpathcurveto{\pgfqpoint{2.492243in}{2.808367in}}{\pgfqpoint{2.487853in}{2.797768in}}{\pgfqpoint{2.487853in}{2.786718in}}%
\pgfpathcurveto{\pgfqpoint{2.487853in}{2.775668in}}{\pgfqpoint{2.492243in}{2.765069in}}{\pgfqpoint{2.500057in}{2.757255in}}%
\pgfpathcurveto{\pgfqpoint{2.507870in}{2.749442in}}{\pgfqpoint{2.518469in}{2.745051in}}{\pgfqpoint{2.529519in}{2.745051in}}%
\pgfpathclose%
\pgfusepath{stroke,fill}%
\end{pgfscope}%
\begin{pgfscope}%
\pgfpathrectangle{\pgfqpoint{0.600000in}{0.600000in}}{\pgfqpoint{3.900000in}{3.900000in}}%
\pgfusepath{clip}%
\pgfsetbuttcap%
\pgfsetroundjoin%
\definecolor{currentfill}{rgb}{0.121569,0.466667,0.705882}%
\pgfsetfillcolor{currentfill}%
\pgfsetlinewidth{1.003750pt}%
\definecolor{currentstroke}{rgb}{0.121569,0.466667,0.705882}%
\pgfsetstrokecolor{currentstroke}%
\pgfsetdash{}{0pt}%
\pgfpathmoveto{\pgfqpoint{2.469980in}{1.982760in}}%
\pgfpathcurveto{\pgfqpoint{2.481030in}{1.982760in}}{\pgfqpoint{2.491629in}{1.987151in}}{\pgfqpoint{2.499443in}{1.994964in}}%
\pgfpathcurveto{\pgfqpoint{2.507257in}{2.002778in}}{\pgfqpoint{2.511647in}{2.013377in}}{\pgfqpoint{2.511647in}{2.024427in}}%
\pgfpathcurveto{\pgfqpoint{2.511647in}{2.035477in}}{\pgfqpoint{2.507257in}{2.046076in}}{\pgfqpoint{2.499443in}{2.053890in}}%
\pgfpathcurveto{\pgfqpoint{2.491629in}{2.061703in}}{\pgfqpoint{2.481030in}{2.066094in}}{\pgfqpoint{2.469980in}{2.066094in}}%
\pgfpathcurveto{\pgfqpoint{2.458930in}{2.066094in}}{\pgfqpoint{2.448331in}{2.061703in}}{\pgfqpoint{2.440517in}{2.053890in}}%
\pgfpathcurveto{\pgfqpoint{2.432704in}{2.046076in}}{\pgfqpoint{2.428314in}{2.035477in}}{\pgfqpoint{2.428314in}{2.024427in}}%
\pgfpathcurveto{\pgfqpoint{2.428314in}{2.013377in}}{\pgfqpoint{2.432704in}{2.002778in}}{\pgfqpoint{2.440517in}{1.994964in}}%
\pgfpathcurveto{\pgfqpoint{2.448331in}{1.987151in}}{\pgfqpoint{2.458930in}{1.982760in}}{\pgfqpoint{2.469980in}{1.982760in}}%
\pgfpathclose%
\pgfusepath{stroke,fill}%
\end{pgfscope}%
\begin{pgfscope}%
\pgfpathrectangle{\pgfqpoint{0.600000in}{0.600000in}}{\pgfqpoint{3.900000in}{3.900000in}}%
\pgfusepath{clip}%
\pgfsetbuttcap%
\pgfsetroundjoin%
\definecolor{currentfill}{rgb}{0.121569,0.466667,0.705882}%
\pgfsetfillcolor{currentfill}%
\pgfsetlinewidth{1.003750pt}%
\definecolor{currentstroke}{rgb}{0.121569,0.466667,0.705882}%
\pgfsetstrokecolor{currentstroke}%
\pgfsetdash{}{0pt}%
\pgfpathmoveto{\pgfqpoint{2.967720in}{2.037552in}}%
\pgfpathcurveto{\pgfqpoint{2.978771in}{2.037552in}}{\pgfqpoint{2.989370in}{2.041942in}}{\pgfqpoint{2.997183in}{2.049756in}}%
\pgfpathcurveto{\pgfqpoint{3.004997in}{2.057569in}}{\pgfqpoint{3.009387in}{2.068169in}}{\pgfqpoint{3.009387in}{2.079219in}}%
\pgfpathcurveto{\pgfqpoint{3.009387in}{2.090269in}}{\pgfqpoint{3.004997in}{2.100868in}}{\pgfqpoint{2.997183in}{2.108681in}}%
\pgfpathcurveto{\pgfqpoint{2.989370in}{2.116495in}}{\pgfqpoint{2.978771in}{2.120885in}}{\pgfqpoint{2.967720in}{2.120885in}}%
\pgfpathcurveto{\pgfqpoint{2.956670in}{2.120885in}}{\pgfqpoint{2.946071in}{2.116495in}}{\pgfqpoint{2.938258in}{2.108681in}}%
\pgfpathcurveto{\pgfqpoint{2.930444in}{2.100868in}}{\pgfqpoint{2.926054in}{2.090269in}}{\pgfqpoint{2.926054in}{2.079219in}}%
\pgfpathcurveto{\pgfqpoint{2.926054in}{2.068169in}}{\pgfqpoint{2.930444in}{2.057569in}}{\pgfqpoint{2.938258in}{2.049756in}}%
\pgfpathcurveto{\pgfqpoint{2.946071in}{2.041942in}}{\pgfqpoint{2.956670in}{2.037552in}}{\pgfqpoint{2.967720in}{2.037552in}}%
\pgfpathclose%
\pgfusepath{stroke,fill}%
\end{pgfscope}%
\begin{pgfscope}%
\pgfpathrectangle{\pgfqpoint{0.600000in}{0.600000in}}{\pgfqpoint{3.900000in}{3.900000in}}%
\pgfusepath{clip}%
\pgfsetbuttcap%
\pgfsetroundjoin%
\definecolor{currentfill}{rgb}{0.121569,0.466667,0.705882}%
\pgfsetfillcolor{currentfill}%
\pgfsetlinewidth{1.003750pt}%
\definecolor{currentstroke}{rgb}{0.121569,0.466667,0.705882}%
\pgfsetstrokecolor{currentstroke}%
\pgfsetdash{}{0pt}%
\pgfpathmoveto{\pgfqpoint{3.097865in}{3.341986in}}%
\pgfpathcurveto{\pgfqpoint{3.108915in}{3.341986in}}{\pgfqpoint{3.119514in}{3.346376in}}{\pgfqpoint{3.127328in}{3.354190in}}%
\pgfpathcurveto{\pgfqpoint{3.135142in}{3.362003in}}{\pgfqpoint{3.139532in}{3.372602in}}{\pgfqpoint{3.139532in}{3.383652in}}%
\pgfpathcurveto{\pgfqpoint{3.139532in}{3.394702in}}{\pgfqpoint{3.135142in}{3.405301in}}{\pgfqpoint{3.127328in}{3.413115in}}%
\pgfpathcurveto{\pgfqpoint{3.119514in}{3.420929in}}{\pgfqpoint{3.108915in}{3.425319in}}{\pgfqpoint{3.097865in}{3.425319in}}%
\pgfpathcurveto{\pgfqpoint{3.086815in}{3.425319in}}{\pgfqpoint{3.076216in}{3.420929in}}{\pgfqpoint{3.068402in}{3.413115in}}%
\pgfpathcurveto{\pgfqpoint{3.060589in}{3.405301in}}{\pgfqpoint{3.056199in}{3.394702in}}{\pgfqpoint{3.056199in}{3.383652in}}%
\pgfpathcurveto{\pgfqpoint{3.056199in}{3.372602in}}{\pgfqpoint{3.060589in}{3.362003in}}{\pgfqpoint{3.068402in}{3.354190in}}%
\pgfpathcurveto{\pgfqpoint{3.076216in}{3.346376in}}{\pgfqpoint{3.086815in}{3.341986in}}{\pgfqpoint{3.097865in}{3.341986in}}%
\pgfpathclose%
\pgfusepath{stroke,fill}%
\end{pgfscope}%
\begin{pgfscope}%
\pgfpathrectangle{\pgfqpoint{0.600000in}{0.600000in}}{\pgfqpoint{3.900000in}{3.900000in}}%
\pgfusepath{clip}%
\pgfsetbuttcap%
\pgfsetroundjoin%
\definecolor{currentfill}{rgb}{0.121569,0.466667,0.705882}%
\pgfsetfillcolor{currentfill}%
\pgfsetlinewidth{1.003750pt}%
\definecolor{currentstroke}{rgb}{0.121569,0.466667,0.705882}%
\pgfsetstrokecolor{currentstroke}%
\pgfsetdash{}{0pt}%
\pgfpathmoveto{\pgfqpoint{2.187070in}{2.914441in}}%
\pgfpathcurveto{\pgfqpoint{2.198120in}{2.914441in}}{\pgfqpoint{2.208719in}{2.918831in}}{\pgfqpoint{2.216533in}{2.926645in}}%
\pgfpathcurveto{\pgfqpoint{2.224347in}{2.934458in}}{\pgfqpoint{2.228737in}{2.945057in}}{\pgfqpoint{2.228737in}{2.956107in}}%
\pgfpathcurveto{\pgfqpoint{2.228737in}{2.967158in}}{\pgfqpoint{2.224347in}{2.977757in}}{\pgfqpoint{2.216533in}{2.985570in}}%
\pgfpathcurveto{\pgfqpoint{2.208719in}{2.993384in}}{\pgfqpoint{2.198120in}{2.997774in}}{\pgfqpoint{2.187070in}{2.997774in}}%
\pgfpathcurveto{\pgfqpoint{2.176020in}{2.997774in}}{\pgfqpoint{2.165421in}{2.993384in}}{\pgfqpoint{2.157607in}{2.985570in}}%
\pgfpathcurveto{\pgfqpoint{2.149794in}{2.977757in}}{\pgfqpoint{2.145404in}{2.967158in}}{\pgfqpoint{2.145404in}{2.956107in}}%
\pgfpathcurveto{\pgfqpoint{2.145404in}{2.945057in}}{\pgfqpoint{2.149794in}{2.934458in}}{\pgfqpoint{2.157607in}{2.926645in}}%
\pgfpathcurveto{\pgfqpoint{2.165421in}{2.918831in}}{\pgfqpoint{2.176020in}{2.914441in}}{\pgfqpoint{2.187070in}{2.914441in}}%
\pgfpathclose%
\pgfusepath{stroke,fill}%
\end{pgfscope}%
\begin{pgfscope}%
\pgfpathrectangle{\pgfqpoint{0.600000in}{0.600000in}}{\pgfqpoint{3.900000in}{3.900000in}}%
\pgfusepath{clip}%
\pgfsetbuttcap%
\pgfsetroundjoin%
\definecolor{currentfill}{rgb}{0.121569,0.466667,0.705882}%
\pgfsetfillcolor{currentfill}%
\pgfsetlinewidth{1.003750pt}%
\definecolor{currentstroke}{rgb}{0.121569,0.466667,0.705882}%
\pgfsetstrokecolor{currentstroke}%
\pgfsetdash{}{0pt}%
\pgfpathmoveto{\pgfqpoint{2.155464in}{2.800434in}}%
\pgfpathcurveto{\pgfqpoint{2.166515in}{2.800434in}}{\pgfqpoint{2.177114in}{2.804825in}}{\pgfqpoint{2.184927in}{2.812638in}}%
\pgfpathcurveto{\pgfqpoint{2.192741in}{2.820452in}}{\pgfqpoint{2.197131in}{2.831051in}}{\pgfqpoint{2.197131in}{2.842101in}}%
\pgfpathcurveto{\pgfqpoint{2.197131in}{2.853151in}}{\pgfqpoint{2.192741in}{2.863750in}}{\pgfqpoint{2.184927in}{2.871564in}}%
\pgfpathcurveto{\pgfqpoint{2.177114in}{2.879377in}}{\pgfqpoint{2.166515in}{2.883768in}}{\pgfqpoint{2.155464in}{2.883768in}}%
\pgfpathcurveto{\pgfqpoint{2.144414in}{2.883768in}}{\pgfqpoint{2.133815in}{2.879377in}}{\pgfqpoint{2.126002in}{2.871564in}}%
\pgfpathcurveto{\pgfqpoint{2.118188in}{2.863750in}}{\pgfqpoint{2.113798in}{2.853151in}}{\pgfqpoint{2.113798in}{2.842101in}}%
\pgfpathcurveto{\pgfqpoint{2.113798in}{2.831051in}}{\pgfqpoint{2.118188in}{2.820452in}}{\pgfqpoint{2.126002in}{2.812638in}}%
\pgfpathcurveto{\pgfqpoint{2.133815in}{2.804825in}}{\pgfqpoint{2.144414in}{2.800434in}}{\pgfqpoint{2.155464in}{2.800434in}}%
\pgfpathclose%
\pgfusepath{stroke,fill}%
\end{pgfscope}%
\begin{pgfscope}%
\pgfpathrectangle{\pgfqpoint{0.600000in}{0.600000in}}{\pgfqpoint{3.900000in}{3.900000in}}%
\pgfusepath{clip}%
\pgfsetbuttcap%
\pgfsetroundjoin%
\definecolor{currentfill}{rgb}{0.121569,0.466667,0.705882}%
\pgfsetfillcolor{currentfill}%
\pgfsetlinewidth{1.003750pt}%
\definecolor{currentstroke}{rgb}{0.121569,0.466667,0.705882}%
\pgfsetstrokecolor{currentstroke}%
\pgfsetdash{}{0pt}%
\pgfpathmoveto{\pgfqpoint{2.673006in}{3.343241in}}%
\pgfpathcurveto{\pgfqpoint{2.684056in}{3.343241in}}{\pgfqpoint{2.694655in}{3.347631in}}{\pgfqpoint{2.702468in}{3.355444in}}%
\pgfpathcurveto{\pgfqpoint{2.710282in}{3.363258in}}{\pgfqpoint{2.714672in}{3.373857in}}{\pgfqpoint{2.714672in}{3.384907in}}%
\pgfpathcurveto{\pgfqpoint{2.714672in}{3.395957in}}{\pgfqpoint{2.710282in}{3.406556in}}{\pgfqpoint{2.702468in}{3.414370in}}%
\pgfpathcurveto{\pgfqpoint{2.694655in}{3.422184in}}{\pgfqpoint{2.684056in}{3.426574in}}{\pgfqpoint{2.673006in}{3.426574in}}%
\pgfpathcurveto{\pgfqpoint{2.661956in}{3.426574in}}{\pgfqpoint{2.651357in}{3.422184in}}{\pgfqpoint{2.643543in}{3.414370in}}%
\pgfpathcurveto{\pgfqpoint{2.635729in}{3.406556in}}{\pgfqpoint{2.631339in}{3.395957in}}{\pgfqpoint{2.631339in}{3.384907in}}%
\pgfpathcurveto{\pgfqpoint{2.631339in}{3.373857in}}{\pgfqpoint{2.635729in}{3.363258in}}{\pgfqpoint{2.643543in}{3.355444in}}%
\pgfpathcurveto{\pgfqpoint{2.651357in}{3.347631in}}{\pgfqpoint{2.661956in}{3.343241in}}{\pgfqpoint{2.673006in}{3.343241in}}%
\pgfpathclose%
\pgfusepath{stroke,fill}%
\end{pgfscope}%
\begin{pgfscope}%
\pgfpathrectangle{\pgfqpoint{0.600000in}{0.600000in}}{\pgfqpoint{3.900000in}{3.900000in}}%
\pgfusepath{clip}%
\pgfsetbuttcap%
\pgfsetroundjoin%
\definecolor{currentfill}{rgb}{0.121569,0.466667,0.705882}%
\pgfsetfillcolor{currentfill}%
\pgfsetlinewidth{1.003750pt}%
\definecolor{currentstroke}{rgb}{0.121569,0.466667,0.705882}%
\pgfsetstrokecolor{currentstroke}%
\pgfsetdash{}{0pt}%
\pgfpathmoveto{\pgfqpoint{2.330668in}{3.106521in}}%
\pgfpathcurveto{\pgfqpoint{2.341718in}{3.106521in}}{\pgfqpoint{2.352317in}{3.110911in}}{\pgfqpoint{2.360130in}{3.118725in}}%
\pgfpathcurveto{\pgfqpoint{2.367944in}{3.126538in}}{\pgfqpoint{2.372334in}{3.137137in}}{\pgfqpoint{2.372334in}{3.148187in}}%
\pgfpathcurveto{\pgfqpoint{2.372334in}{3.159238in}}{\pgfqpoint{2.367944in}{3.169837in}}{\pgfqpoint{2.360130in}{3.177650in}}%
\pgfpathcurveto{\pgfqpoint{2.352317in}{3.185464in}}{\pgfqpoint{2.341718in}{3.189854in}}{\pgfqpoint{2.330668in}{3.189854in}}%
\pgfpathcurveto{\pgfqpoint{2.319618in}{3.189854in}}{\pgfqpoint{2.309019in}{3.185464in}}{\pgfqpoint{2.301205in}{3.177650in}}%
\pgfpathcurveto{\pgfqpoint{2.293391in}{3.169837in}}{\pgfqpoint{2.289001in}{3.159238in}}{\pgfqpoint{2.289001in}{3.148187in}}%
\pgfpathcurveto{\pgfqpoint{2.289001in}{3.137137in}}{\pgfqpoint{2.293391in}{3.126538in}}{\pgfqpoint{2.301205in}{3.118725in}}%
\pgfpathcurveto{\pgfqpoint{2.309019in}{3.110911in}}{\pgfqpoint{2.319618in}{3.106521in}}{\pgfqpoint{2.330668in}{3.106521in}}%
\pgfpathclose%
\pgfusepath{stroke,fill}%
\end{pgfscope}%
\begin{pgfscope}%
\pgfpathrectangle{\pgfqpoint{0.600000in}{0.600000in}}{\pgfqpoint{3.900000in}{3.900000in}}%
\pgfusepath{clip}%
\pgfsetbuttcap%
\pgfsetroundjoin%
\definecolor{currentfill}{rgb}{0.121569,0.466667,0.705882}%
\pgfsetfillcolor{currentfill}%
\pgfsetlinewidth{1.003750pt}%
\definecolor{currentstroke}{rgb}{0.121569,0.466667,0.705882}%
\pgfsetstrokecolor{currentstroke}%
\pgfsetdash{}{0pt}%
\pgfpathmoveto{\pgfqpoint{1.851549in}{2.738464in}}%
\pgfpathcurveto{\pgfqpoint{1.862599in}{2.738464in}}{\pgfqpoint{1.873198in}{2.742855in}}{\pgfqpoint{1.881012in}{2.750668in}}%
\pgfpathcurveto{\pgfqpoint{1.888826in}{2.758482in}}{\pgfqpoint{1.893216in}{2.769081in}}{\pgfqpoint{1.893216in}{2.780131in}}%
\pgfpathcurveto{\pgfqpoint{1.893216in}{2.791181in}}{\pgfqpoint{1.888826in}{2.801780in}}{\pgfqpoint{1.881012in}{2.809594in}}%
\pgfpathcurveto{\pgfqpoint{1.873198in}{2.817407in}}{\pgfqpoint{1.862599in}{2.821798in}}{\pgfqpoint{1.851549in}{2.821798in}}%
\pgfpathcurveto{\pgfqpoint{1.840499in}{2.821798in}}{\pgfqpoint{1.829900in}{2.817407in}}{\pgfqpoint{1.822086in}{2.809594in}}%
\pgfpathcurveto{\pgfqpoint{1.814273in}{2.801780in}}{\pgfqpoint{1.809883in}{2.791181in}}{\pgfqpoint{1.809883in}{2.780131in}}%
\pgfpathcurveto{\pgfqpoint{1.809883in}{2.769081in}}{\pgfqpoint{1.814273in}{2.758482in}}{\pgfqpoint{1.822086in}{2.750668in}}%
\pgfpathcurveto{\pgfqpoint{1.829900in}{2.742855in}}{\pgfqpoint{1.840499in}{2.738464in}}{\pgfqpoint{1.851549in}{2.738464in}}%
\pgfpathclose%
\pgfusepath{stroke,fill}%
\end{pgfscope}%
\begin{pgfscope}%
\pgfpathrectangle{\pgfqpoint{0.600000in}{0.600000in}}{\pgfqpoint{3.900000in}{3.900000in}}%
\pgfusepath{clip}%
\pgfsetbuttcap%
\pgfsetroundjoin%
\definecolor{currentfill}{rgb}{0.121569,0.466667,0.705882}%
\pgfsetfillcolor{currentfill}%
\pgfsetlinewidth{1.003750pt}%
\definecolor{currentstroke}{rgb}{0.121569,0.466667,0.705882}%
\pgfsetstrokecolor{currentstroke}%
\pgfsetdash{}{0pt}%
\pgfpathmoveto{\pgfqpoint{2.617281in}{1.430180in}}%
\pgfpathcurveto{\pgfqpoint{2.628331in}{1.430180in}}{\pgfqpoint{2.638930in}{1.434570in}}{\pgfqpoint{2.646744in}{1.442384in}}%
\pgfpathcurveto{\pgfqpoint{2.654557in}{1.450197in}}{\pgfqpoint{2.658948in}{1.460797in}}{\pgfqpoint{2.658948in}{1.471847in}}%
\pgfpathcurveto{\pgfqpoint{2.658948in}{1.482897in}}{\pgfqpoint{2.654557in}{1.493496in}}{\pgfqpoint{2.646744in}{1.501309in}}%
\pgfpathcurveto{\pgfqpoint{2.638930in}{1.509123in}}{\pgfqpoint{2.628331in}{1.513513in}}{\pgfqpoint{2.617281in}{1.513513in}}%
\pgfpathcurveto{\pgfqpoint{2.606231in}{1.513513in}}{\pgfqpoint{2.595632in}{1.509123in}}{\pgfqpoint{2.587818in}{1.501309in}}%
\pgfpathcurveto{\pgfqpoint{2.580004in}{1.493496in}}{\pgfqpoint{2.575614in}{1.482897in}}{\pgfqpoint{2.575614in}{1.471847in}}%
\pgfpathcurveto{\pgfqpoint{2.575614in}{1.460797in}}{\pgfqpoint{2.580004in}{1.450197in}}{\pgfqpoint{2.587818in}{1.442384in}}%
\pgfpathcurveto{\pgfqpoint{2.595632in}{1.434570in}}{\pgfqpoint{2.606231in}{1.430180in}}{\pgfqpoint{2.617281in}{1.430180in}}%
\pgfpathclose%
\pgfusepath{stroke,fill}%
\end{pgfscope}%
\begin{pgfscope}%
\pgfpathrectangle{\pgfqpoint{0.600000in}{0.600000in}}{\pgfqpoint{3.900000in}{3.900000in}}%
\pgfusepath{clip}%
\pgfsetbuttcap%
\pgfsetroundjoin%
\definecolor{currentfill}{rgb}{0.121569,0.466667,0.705882}%
\pgfsetfillcolor{currentfill}%
\pgfsetlinewidth{1.003750pt}%
\definecolor{currentstroke}{rgb}{0.121569,0.466667,0.705882}%
\pgfsetstrokecolor{currentstroke}%
\pgfsetdash{}{0pt}%
\pgfpathmoveto{\pgfqpoint{3.052766in}{3.023322in}}%
\pgfpathcurveto{\pgfqpoint{3.063816in}{3.023322in}}{\pgfqpoint{3.074415in}{3.027712in}}{\pgfqpoint{3.082229in}{3.035526in}}%
\pgfpathcurveto{\pgfqpoint{3.090042in}{3.043340in}}{\pgfqpoint{3.094433in}{3.053939in}}{\pgfqpoint{3.094433in}{3.064989in}}%
\pgfpathcurveto{\pgfqpoint{3.094433in}{3.076039in}}{\pgfqpoint{3.090042in}{3.086638in}}{\pgfqpoint{3.082229in}{3.094451in}}%
\pgfpathcurveto{\pgfqpoint{3.074415in}{3.102265in}}{\pgfqpoint{3.063816in}{3.106655in}}{\pgfqpoint{3.052766in}{3.106655in}}%
\pgfpathcurveto{\pgfqpoint{3.041716in}{3.106655in}}{\pgfqpoint{3.031117in}{3.102265in}}{\pgfqpoint{3.023303in}{3.094451in}}%
\pgfpathcurveto{\pgfqpoint{3.015490in}{3.086638in}}{\pgfqpoint{3.011099in}{3.076039in}}{\pgfqpoint{3.011099in}{3.064989in}}%
\pgfpathcurveto{\pgfqpoint{3.011099in}{3.053939in}}{\pgfqpoint{3.015490in}{3.043340in}}{\pgfqpoint{3.023303in}{3.035526in}}%
\pgfpathcurveto{\pgfqpoint{3.031117in}{3.027712in}}{\pgfqpoint{3.041716in}{3.023322in}}{\pgfqpoint{3.052766in}{3.023322in}}%
\pgfpathclose%
\pgfusepath{stroke,fill}%
\end{pgfscope}%
\begin{pgfscope}%
\pgfpathrectangle{\pgfqpoint{0.600000in}{0.600000in}}{\pgfqpoint{3.900000in}{3.900000in}}%
\pgfusepath{clip}%
\pgfsetbuttcap%
\pgfsetroundjoin%
\definecolor{currentfill}{rgb}{0.121569,0.466667,0.705882}%
\pgfsetfillcolor{currentfill}%
\pgfsetlinewidth{1.003750pt}%
\definecolor{currentstroke}{rgb}{0.121569,0.466667,0.705882}%
\pgfsetstrokecolor{currentstroke}%
\pgfsetdash{}{0pt}%
\pgfpathmoveto{\pgfqpoint{2.780046in}{1.236263in}}%
\pgfpathcurveto{\pgfqpoint{2.791096in}{1.236263in}}{\pgfqpoint{2.801695in}{1.240653in}}{\pgfqpoint{2.809509in}{1.248467in}}%
\pgfpathcurveto{\pgfqpoint{2.817322in}{1.256280in}}{\pgfqpoint{2.821713in}{1.266879in}}{\pgfqpoint{2.821713in}{1.277929in}}%
\pgfpathcurveto{\pgfqpoint{2.821713in}{1.288980in}}{\pgfqpoint{2.817322in}{1.299579in}}{\pgfqpoint{2.809509in}{1.307392in}}%
\pgfpathcurveto{\pgfqpoint{2.801695in}{1.315206in}}{\pgfqpoint{2.791096in}{1.319596in}}{\pgfqpoint{2.780046in}{1.319596in}}%
\pgfpathcurveto{\pgfqpoint{2.768996in}{1.319596in}}{\pgfqpoint{2.758397in}{1.315206in}}{\pgfqpoint{2.750583in}{1.307392in}}%
\pgfpathcurveto{\pgfqpoint{2.742770in}{1.299579in}}{\pgfqpoint{2.738379in}{1.288980in}}{\pgfqpoint{2.738379in}{1.277929in}}%
\pgfpathcurveto{\pgfqpoint{2.738379in}{1.266879in}}{\pgfqpoint{2.742770in}{1.256280in}}{\pgfqpoint{2.750583in}{1.248467in}}%
\pgfpathcurveto{\pgfqpoint{2.758397in}{1.240653in}}{\pgfqpoint{2.768996in}{1.236263in}}{\pgfqpoint{2.780046in}{1.236263in}}%
\pgfpathclose%
\pgfusepath{stroke,fill}%
\end{pgfscope}%
\begin{pgfscope}%
\pgfpathrectangle{\pgfqpoint{0.600000in}{0.600000in}}{\pgfqpoint{3.900000in}{3.900000in}}%
\pgfusepath{clip}%
\pgfsetbuttcap%
\pgfsetroundjoin%
\definecolor{currentfill}{rgb}{0.121569,0.466667,0.705882}%
\pgfsetfillcolor{currentfill}%
\pgfsetlinewidth{1.003750pt}%
\definecolor{currentstroke}{rgb}{0.121569,0.466667,0.705882}%
\pgfsetstrokecolor{currentstroke}%
\pgfsetdash{}{0pt}%
\pgfpathmoveto{\pgfqpoint{2.395423in}{2.620986in}}%
\pgfpathcurveto{\pgfqpoint{2.406473in}{2.620986in}}{\pgfqpoint{2.417072in}{2.625376in}}{\pgfqpoint{2.424886in}{2.633190in}}%
\pgfpathcurveto{\pgfqpoint{2.432700in}{2.641004in}}{\pgfqpoint{2.437090in}{2.651603in}}{\pgfqpoint{2.437090in}{2.662653in}}%
\pgfpathcurveto{\pgfqpoint{2.437090in}{2.673703in}}{\pgfqpoint{2.432700in}{2.684302in}}{\pgfqpoint{2.424886in}{2.692116in}}%
\pgfpathcurveto{\pgfqpoint{2.417072in}{2.699929in}}{\pgfqpoint{2.406473in}{2.704319in}}{\pgfqpoint{2.395423in}{2.704319in}}%
\pgfpathcurveto{\pgfqpoint{2.384373in}{2.704319in}}{\pgfqpoint{2.373774in}{2.699929in}}{\pgfqpoint{2.365960in}{2.692116in}}%
\pgfpathcurveto{\pgfqpoint{2.358147in}{2.684302in}}{\pgfqpoint{2.353756in}{2.673703in}}{\pgfqpoint{2.353756in}{2.662653in}}%
\pgfpathcurveto{\pgfqpoint{2.353756in}{2.651603in}}{\pgfqpoint{2.358147in}{2.641004in}}{\pgfqpoint{2.365960in}{2.633190in}}%
\pgfpathcurveto{\pgfqpoint{2.373774in}{2.625376in}}{\pgfqpoint{2.384373in}{2.620986in}}{\pgfqpoint{2.395423in}{2.620986in}}%
\pgfpathclose%
\pgfusepath{stroke,fill}%
\end{pgfscope}%
\begin{pgfscope}%
\pgfpathrectangle{\pgfqpoint{0.600000in}{0.600000in}}{\pgfqpoint{3.900000in}{3.900000in}}%
\pgfusepath{clip}%
\pgfsetbuttcap%
\pgfsetroundjoin%
\definecolor{currentfill}{rgb}{0.121569,0.466667,0.705882}%
\pgfsetfillcolor{currentfill}%
\pgfsetlinewidth{1.003750pt}%
\definecolor{currentstroke}{rgb}{0.121569,0.466667,0.705882}%
\pgfsetstrokecolor{currentstroke}%
\pgfsetdash{}{0pt}%
\pgfpathmoveto{\pgfqpoint{2.269620in}{3.076055in}}%
\pgfpathcurveto{\pgfqpoint{2.280670in}{3.076055in}}{\pgfqpoint{2.291269in}{3.080445in}}{\pgfqpoint{2.299083in}{3.088259in}}%
\pgfpathcurveto{\pgfqpoint{2.306897in}{3.096073in}}{\pgfqpoint{2.311287in}{3.106672in}}{\pgfqpoint{2.311287in}{3.117722in}}%
\pgfpathcurveto{\pgfqpoint{2.311287in}{3.128772in}}{\pgfqpoint{2.306897in}{3.139371in}}{\pgfqpoint{2.299083in}{3.147185in}}%
\pgfpathcurveto{\pgfqpoint{2.291269in}{3.154998in}}{\pgfqpoint{2.280670in}{3.159388in}}{\pgfqpoint{2.269620in}{3.159388in}}%
\pgfpathcurveto{\pgfqpoint{2.258570in}{3.159388in}}{\pgfqpoint{2.247971in}{3.154998in}}{\pgfqpoint{2.240158in}{3.147185in}}%
\pgfpathcurveto{\pgfqpoint{2.232344in}{3.139371in}}{\pgfqpoint{2.227954in}{3.128772in}}{\pgfqpoint{2.227954in}{3.117722in}}%
\pgfpathcurveto{\pgfqpoint{2.227954in}{3.106672in}}{\pgfqpoint{2.232344in}{3.096073in}}{\pgfqpoint{2.240158in}{3.088259in}}%
\pgfpathcurveto{\pgfqpoint{2.247971in}{3.080445in}}{\pgfqpoint{2.258570in}{3.076055in}}{\pgfqpoint{2.269620in}{3.076055in}}%
\pgfpathclose%
\pgfusepath{stroke,fill}%
\end{pgfscope}%
\begin{pgfscope}%
\pgfpathrectangle{\pgfqpoint{0.600000in}{0.600000in}}{\pgfqpoint{3.900000in}{3.900000in}}%
\pgfusepath{clip}%
\pgfsetbuttcap%
\pgfsetroundjoin%
\definecolor{currentfill}{rgb}{0.121569,0.466667,0.705882}%
\pgfsetfillcolor{currentfill}%
\pgfsetlinewidth{1.003750pt}%
\definecolor{currentstroke}{rgb}{0.121569,0.466667,0.705882}%
\pgfsetstrokecolor{currentstroke}%
\pgfsetdash{}{0pt}%
\pgfpathmoveto{\pgfqpoint{2.582977in}{2.811086in}}%
\pgfpathcurveto{\pgfqpoint{2.594027in}{2.811086in}}{\pgfqpoint{2.604626in}{2.815477in}}{\pgfqpoint{2.612440in}{2.823290in}}%
\pgfpathcurveto{\pgfqpoint{2.620253in}{2.831104in}}{\pgfqpoint{2.624643in}{2.841703in}}{\pgfqpoint{2.624643in}{2.852753in}}%
\pgfpathcurveto{\pgfqpoint{2.624643in}{2.863803in}}{\pgfqpoint{2.620253in}{2.874402in}}{\pgfqpoint{2.612440in}{2.882216in}}%
\pgfpathcurveto{\pgfqpoint{2.604626in}{2.890030in}}{\pgfqpoint{2.594027in}{2.894420in}}{\pgfqpoint{2.582977in}{2.894420in}}%
\pgfpathcurveto{\pgfqpoint{2.571927in}{2.894420in}}{\pgfqpoint{2.561328in}{2.890030in}}{\pgfqpoint{2.553514in}{2.882216in}}%
\pgfpathcurveto{\pgfqpoint{2.545700in}{2.874402in}}{\pgfqpoint{2.541310in}{2.863803in}}{\pgfqpoint{2.541310in}{2.852753in}}%
\pgfpathcurveto{\pgfqpoint{2.541310in}{2.841703in}}{\pgfqpoint{2.545700in}{2.831104in}}{\pgfqpoint{2.553514in}{2.823290in}}%
\pgfpathcurveto{\pgfqpoint{2.561328in}{2.815477in}}{\pgfqpoint{2.571927in}{2.811086in}}{\pgfqpoint{2.582977in}{2.811086in}}%
\pgfpathclose%
\pgfusepath{stroke,fill}%
\end{pgfscope}%
\begin{pgfscope}%
\pgfpathrectangle{\pgfqpoint{0.600000in}{0.600000in}}{\pgfqpoint{3.900000in}{3.900000in}}%
\pgfusepath{clip}%
\pgfsetbuttcap%
\pgfsetroundjoin%
\definecolor{currentfill}{rgb}{0.121569,0.466667,0.705882}%
\pgfsetfillcolor{currentfill}%
\pgfsetlinewidth{1.003750pt}%
\definecolor{currentstroke}{rgb}{0.121569,0.466667,0.705882}%
\pgfsetstrokecolor{currentstroke}%
\pgfsetdash{}{0pt}%
\pgfpathmoveto{\pgfqpoint{2.856406in}{2.825007in}}%
\pgfpathcurveto{\pgfqpoint{2.867456in}{2.825007in}}{\pgfqpoint{2.878055in}{2.829397in}}{\pgfqpoint{2.885869in}{2.837211in}}%
\pgfpathcurveto{\pgfqpoint{2.893683in}{2.845025in}}{\pgfqpoint{2.898073in}{2.855624in}}{\pgfqpoint{2.898073in}{2.866674in}}%
\pgfpathcurveto{\pgfqpoint{2.898073in}{2.877724in}}{\pgfqpoint{2.893683in}{2.888323in}}{\pgfqpoint{2.885869in}{2.896137in}}%
\pgfpathcurveto{\pgfqpoint{2.878055in}{2.903950in}}{\pgfqpoint{2.867456in}{2.908340in}}{\pgfqpoint{2.856406in}{2.908340in}}%
\pgfpathcurveto{\pgfqpoint{2.845356in}{2.908340in}}{\pgfqpoint{2.834757in}{2.903950in}}{\pgfqpoint{2.826943in}{2.896137in}}%
\pgfpathcurveto{\pgfqpoint{2.819130in}{2.888323in}}{\pgfqpoint{2.814740in}{2.877724in}}{\pgfqpoint{2.814740in}{2.866674in}}%
\pgfpathcurveto{\pgfqpoint{2.814740in}{2.855624in}}{\pgfqpoint{2.819130in}{2.845025in}}{\pgfqpoint{2.826943in}{2.837211in}}%
\pgfpathcurveto{\pgfqpoint{2.834757in}{2.829397in}}{\pgfqpoint{2.845356in}{2.825007in}}{\pgfqpoint{2.856406in}{2.825007in}}%
\pgfpathclose%
\pgfusepath{stroke,fill}%
\end{pgfscope}%
\begin{pgfscope}%
\pgfpathrectangle{\pgfqpoint{0.600000in}{0.600000in}}{\pgfqpoint{3.900000in}{3.900000in}}%
\pgfusepath{clip}%
\pgfsetbuttcap%
\pgfsetroundjoin%
\definecolor{currentfill}{rgb}{0.121569,0.466667,0.705882}%
\pgfsetfillcolor{currentfill}%
\pgfsetlinewidth{1.003750pt}%
\definecolor{currentstroke}{rgb}{0.121569,0.466667,0.705882}%
\pgfsetstrokecolor{currentstroke}%
\pgfsetdash{}{0pt}%
\pgfpathmoveto{\pgfqpoint{2.915066in}{2.706491in}}%
\pgfpathcurveto{\pgfqpoint{2.926116in}{2.706491in}}{\pgfqpoint{2.936715in}{2.710881in}}{\pgfqpoint{2.944529in}{2.718695in}}%
\pgfpathcurveto{\pgfqpoint{2.952342in}{2.726508in}}{\pgfqpoint{2.956733in}{2.737107in}}{\pgfqpoint{2.956733in}{2.748158in}}%
\pgfpathcurveto{\pgfqpoint{2.956733in}{2.759208in}}{\pgfqpoint{2.952342in}{2.769807in}}{\pgfqpoint{2.944529in}{2.777620in}}%
\pgfpathcurveto{\pgfqpoint{2.936715in}{2.785434in}}{\pgfqpoint{2.926116in}{2.789824in}}{\pgfqpoint{2.915066in}{2.789824in}}%
\pgfpathcurveto{\pgfqpoint{2.904016in}{2.789824in}}{\pgfqpoint{2.893417in}{2.785434in}}{\pgfqpoint{2.885603in}{2.777620in}}%
\pgfpathcurveto{\pgfqpoint{2.877789in}{2.769807in}}{\pgfqpoint{2.873399in}{2.759208in}}{\pgfqpoint{2.873399in}{2.748158in}}%
\pgfpathcurveto{\pgfqpoint{2.873399in}{2.737107in}}{\pgfqpoint{2.877789in}{2.726508in}}{\pgfqpoint{2.885603in}{2.718695in}}%
\pgfpathcurveto{\pgfqpoint{2.893417in}{2.710881in}}{\pgfqpoint{2.904016in}{2.706491in}}{\pgfqpoint{2.915066in}{2.706491in}}%
\pgfpathclose%
\pgfusepath{stroke,fill}%
\end{pgfscope}%
\begin{pgfscope}%
\pgfpathrectangle{\pgfqpoint{0.600000in}{0.600000in}}{\pgfqpoint{3.900000in}{3.900000in}}%
\pgfusepath{clip}%
\pgfsetbuttcap%
\pgfsetroundjoin%
\definecolor{currentfill}{rgb}{0.121569,0.466667,0.705882}%
\pgfsetfillcolor{currentfill}%
\pgfsetlinewidth{1.003750pt}%
\definecolor{currentstroke}{rgb}{0.121569,0.466667,0.705882}%
\pgfsetstrokecolor{currentstroke}%
\pgfsetdash{}{0pt}%
\pgfpathmoveto{\pgfqpoint{2.728782in}{2.502206in}}%
\pgfpathcurveto{\pgfqpoint{2.739832in}{2.502206in}}{\pgfqpoint{2.750431in}{2.506596in}}{\pgfqpoint{2.758245in}{2.514410in}}%
\pgfpathcurveto{\pgfqpoint{2.766058in}{2.522223in}}{\pgfqpoint{2.770449in}{2.532823in}}{\pgfqpoint{2.770449in}{2.543873in}}%
\pgfpathcurveto{\pgfqpoint{2.770449in}{2.554923in}}{\pgfqpoint{2.766058in}{2.565522in}}{\pgfqpoint{2.758245in}{2.573335in}}%
\pgfpathcurveto{\pgfqpoint{2.750431in}{2.581149in}}{\pgfqpoint{2.739832in}{2.585539in}}{\pgfqpoint{2.728782in}{2.585539in}}%
\pgfpathcurveto{\pgfqpoint{2.717732in}{2.585539in}}{\pgfqpoint{2.707133in}{2.581149in}}{\pgfqpoint{2.699319in}{2.573335in}}%
\pgfpathcurveto{\pgfqpoint{2.691506in}{2.565522in}}{\pgfqpoint{2.687115in}{2.554923in}}{\pgfqpoint{2.687115in}{2.543873in}}%
\pgfpathcurveto{\pgfqpoint{2.687115in}{2.532823in}}{\pgfqpoint{2.691506in}{2.522223in}}{\pgfqpoint{2.699319in}{2.514410in}}%
\pgfpathcurveto{\pgfqpoint{2.707133in}{2.506596in}}{\pgfqpoint{2.717732in}{2.502206in}}{\pgfqpoint{2.728782in}{2.502206in}}%
\pgfpathclose%
\pgfusepath{stroke,fill}%
\end{pgfscope}%
\begin{pgfscope}%
\pgfpathrectangle{\pgfqpoint{0.600000in}{0.600000in}}{\pgfqpoint{3.900000in}{3.900000in}}%
\pgfusepath{clip}%
\pgfsetbuttcap%
\pgfsetroundjoin%
\definecolor{currentfill}{rgb}{0.121569,0.466667,0.705882}%
\pgfsetfillcolor{currentfill}%
\pgfsetlinewidth{1.003750pt}%
\definecolor{currentstroke}{rgb}{0.121569,0.466667,0.705882}%
\pgfsetstrokecolor{currentstroke}%
\pgfsetdash{}{0pt}%
\pgfpathmoveto{\pgfqpoint{3.448690in}{1.945204in}}%
\pgfpathcurveto{\pgfqpoint{3.459740in}{1.945204in}}{\pgfqpoint{3.470339in}{1.949595in}}{\pgfqpoint{3.478153in}{1.957408in}}%
\pgfpathcurveto{\pgfqpoint{3.485967in}{1.965222in}}{\pgfqpoint{3.490357in}{1.975821in}}{\pgfqpoint{3.490357in}{1.986871in}}%
\pgfpathcurveto{\pgfqpoint{3.490357in}{1.997921in}}{\pgfqpoint{3.485967in}{2.008520in}}{\pgfqpoint{3.478153in}{2.016334in}}%
\pgfpathcurveto{\pgfqpoint{3.470339in}{2.024147in}}{\pgfqpoint{3.459740in}{2.028538in}}{\pgfqpoint{3.448690in}{2.028538in}}%
\pgfpathcurveto{\pgfqpoint{3.437640in}{2.028538in}}{\pgfqpoint{3.427041in}{2.024147in}}{\pgfqpoint{3.419227in}{2.016334in}}%
\pgfpathcurveto{\pgfqpoint{3.411414in}{2.008520in}}{\pgfqpoint{3.407024in}{1.997921in}}{\pgfqpoint{3.407024in}{1.986871in}}%
\pgfpathcurveto{\pgfqpoint{3.407024in}{1.975821in}}{\pgfqpoint{3.411414in}{1.965222in}}{\pgfqpoint{3.419227in}{1.957408in}}%
\pgfpathcurveto{\pgfqpoint{3.427041in}{1.949595in}}{\pgfqpoint{3.437640in}{1.945204in}}{\pgfqpoint{3.448690in}{1.945204in}}%
\pgfpathclose%
\pgfusepath{stroke,fill}%
\end{pgfscope}%
\begin{pgfscope}%
\pgfpathrectangle{\pgfqpoint{0.600000in}{0.600000in}}{\pgfqpoint{3.900000in}{3.900000in}}%
\pgfusepath{clip}%
\pgfsetbuttcap%
\pgfsetroundjoin%
\definecolor{currentfill}{rgb}{0.121569,0.466667,0.705882}%
\pgfsetfillcolor{currentfill}%
\pgfsetlinewidth{1.003750pt}%
\definecolor{currentstroke}{rgb}{0.121569,0.466667,0.705882}%
\pgfsetstrokecolor{currentstroke}%
\pgfsetdash{}{0pt}%
\pgfpathmoveto{\pgfqpoint{3.076762in}{2.534780in}}%
\pgfpathcurveto{\pgfqpoint{3.087812in}{2.534780in}}{\pgfqpoint{3.098411in}{2.539171in}}{\pgfqpoint{3.106225in}{2.546984in}}%
\pgfpathcurveto{\pgfqpoint{3.114039in}{2.554798in}}{\pgfqpoint{3.118429in}{2.565397in}}{\pgfqpoint{3.118429in}{2.576447in}}%
\pgfpathcurveto{\pgfqpoint{3.118429in}{2.587497in}}{\pgfqpoint{3.114039in}{2.598096in}}{\pgfqpoint{3.106225in}{2.605910in}}%
\pgfpathcurveto{\pgfqpoint{3.098411in}{2.613723in}}{\pgfqpoint{3.087812in}{2.618114in}}{\pgfqpoint{3.076762in}{2.618114in}}%
\pgfpathcurveto{\pgfqpoint{3.065712in}{2.618114in}}{\pgfqpoint{3.055113in}{2.613723in}}{\pgfqpoint{3.047300in}{2.605910in}}%
\pgfpathcurveto{\pgfqpoint{3.039486in}{2.598096in}}{\pgfqpoint{3.035096in}{2.587497in}}{\pgfqpoint{3.035096in}{2.576447in}}%
\pgfpathcurveto{\pgfqpoint{3.035096in}{2.565397in}}{\pgfqpoint{3.039486in}{2.554798in}}{\pgfqpoint{3.047300in}{2.546984in}}%
\pgfpathcurveto{\pgfqpoint{3.055113in}{2.539171in}}{\pgfqpoint{3.065712in}{2.534780in}}{\pgfqpoint{3.076762in}{2.534780in}}%
\pgfpathclose%
\pgfusepath{stroke,fill}%
\end{pgfscope}%
\begin{pgfscope}%
\pgfpathrectangle{\pgfqpoint{0.600000in}{0.600000in}}{\pgfqpoint{3.900000in}{3.900000in}}%
\pgfusepath{clip}%
\pgfsetbuttcap%
\pgfsetroundjoin%
\definecolor{currentfill}{rgb}{0.121569,0.466667,0.705882}%
\pgfsetfillcolor{currentfill}%
\pgfsetlinewidth{1.003750pt}%
\definecolor{currentstroke}{rgb}{0.121569,0.466667,0.705882}%
\pgfsetstrokecolor{currentstroke}%
\pgfsetdash{}{0pt}%
\pgfpathmoveto{\pgfqpoint{2.346854in}{2.903010in}}%
\pgfpathcurveto{\pgfqpoint{2.357904in}{2.903010in}}{\pgfqpoint{2.368503in}{2.907400in}}{\pgfqpoint{2.376316in}{2.915214in}}%
\pgfpathcurveto{\pgfqpoint{2.384130in}{2.923028in}}{\pgfqpoint{2.388520in}{2.933627in}}{\pgfqpoint{2.388520in}{2.944677in}}%
\pgfpathcurveto{\pgfqpoint{2.388520in}{2.955727in}}{\pgfqpoint{2.384130in}{2.966326in}}{\pgfqpoint{2.376316in}{2.974140in}}%
\pgfpathcurveto{\pgfqpoint{2.368503in}{2.981953in}}{\pgfqpoint{2.357904in}{2.986344in}}{\pgfqpoint{2.346854in}{2.986344in}}%
\pgfpathcurveto{\pgfqpoint{2.335803in}{2.986344in}}{\pgfqpoint{2.325204in}{2.981953in}}{\pgfqpoint{2.317391in}{2.974140in}}%
\pgfpathcurveto{\pgfqpoint{2.309577in}{2.966326in}}{\pgfqpoint{2.305187in}{2.955727in}}{\pgfqpoint{2.305187in}{2.944677in}}%
\pgfpathcurveto{\pgfqpoint{2.305187in}{2.933627in}}{\pgfqpoint{2.309577in}{2.923028in}}{\pgfqpoint{2.317391in}{2.915214in}}%
\pgfpathcurveto{\pgfqpoint{2.325204in}{2.907400in}}{\pgfqpoint{2.335803in}{2.903010in}}{\pgfqpoint{2.346854in}{2.903010in}}%
\pgfpathclose%
\pgfusepath{stroke,fill}%
\end{pgfscope}%
\begin{pgfscope}%
\pgfpathrectangle{\pgfqpoint{0.600000in}{0.600000in}}{\pgfqpoint{3.900000in}{3.900000in}}%
\pgfusepath{clip}%
\pgfsetbuttcap%
\pgfsetroundjoin%
\definecolor{currentfill}{rgb}{0.121569,0.466667,0.705882}%
\pgfsetfillcolor{currentfill}%
\pgfsetlinewidth{1.003750pt}%
\definecolor{currentstroke}{rgb}{0.121569,0.466667,0.705882}%
\pgfsetstrokecolor{currentstroke}%
\pgfsetdash{}{0pt}%
\pgfpathmoveto{\pgfqpoint{2.912972in}{2.371037in}}%
\pgfpathcurveto{\pgfqpoint{2.924022in}{2.371037in}}{\pgfqpoint{2.934621in}{2.375428in}}{\pgfqpoint{2.942435in}{2.383241in}}%
\pgfpathcurveto{\pgfqpoint{2.950249in}{2.391055in}}{\pgfqpoint{2.954639in}{2.401654in}}{\pgfqpoint{2.954639in}{2.412704in}}%
\pgfpathcurveto{\pgfqpoint{2.954639in}{2.423754in}}{\pgfqpoint{2.950249in}{2.434353in}}{\pgfqpoint{2.942435in}{2.442167in}}%
\pgfpathcurveto{\pgfqpoint{2.934621in}{2.449981in}}{\pgfqpoint{2.924022in}{2.454371in}}{\pgfqpoint{2.912972in}{2.454371in}}%
\pgfpathcurveto{\pgfqpoint{2.901922in}{2.454371in}}{\pgfqpoint{2.891323in}{2.449981in}}{\pgfqpoint{2.883509in}{2.442167in}}%
\pgfpathcurveto{\pgfqpoint{2.875696in}{2.434353in}}{\pgfqpoint{2.871305in}{2.423754in}}{\pgfqpoint{2.871305in}{2.412704in}}%
\pgfpathcurveto{\pgfqpoint{2.871305in}{2.401654in}}{\pgfqpoint{2.875696in}{2.391055in}}{\pgfqpoint{2.883509in}{2.383241in}}%
\pgfpathcurveto{\pgfqpoint{2.891323in}{2.375428in}}{\pgfqpoint{2.901922in}{2.371037in}}{\pgfqpoint{2.912972in}{2.371037in}}%
\pgfpathclose%
\pgfusepath{stroke,fill}%
\end{pgfscope}%
\begin{pgfscope}%
\pgfpathrectangle{\pgfqpoint{0.600000in}{0.600000in}}{\pgfqpoint{3.900000in}{3.900000in}}%
\pgfusepath{clip}%
\pgfsetbuttcap%
\pgfsetroundjoin%
\definecolor{currentfill}{rgb}{0.121569,0.466667,0.705882}%
\pgfsetfillcolor{currentfill}%
\pgfsetlinewidth{1.003750pt}%
\definecolor{currentstroke}{rgb}{0.121569,0.466667,0.705882}%
\pgfsetstrokecolor{currentstroke}%
\pgfsetdash{}{0pt}%
\pgfpathmoveto{\pgfqpoint{2.763695in}{2.912411in}}%
\pgfpathcurveto{\pgfqpoint{2.774745in}{2.912411in}}{\pgfqpoint{2.785344in}{2.916801in}}{\pgfqpoint{2.793157in}{2.924615in}}%
\pgfpathcurveto{\pgfqpoint{2.800971in}{2.932429in}}{\pgfqpoint{2.805361in}{2.943028in}}{\pgfqpoint{2.805361in}{2.954078in}}%
\pgfpathcurveto{\pgfqpoint{2.805361in}{2.965128in}}{\pgfqpoint{2.800971in}{2.975727in}}{\pgfqpoint{2.793157in}{2.983541in}}%
\pgfpathcurveto{\pgfqpoint{2.785344in}{2.991354in}}{\pgfqpoint{2.774745in}{2.995745in}}{\pgfqpoint{2.763695in}{2.995745in}}%
\pgfpathcurveto{\pgfqpoint{2.752644in}{2.995745in}}{\pgfqpoint{2.742045in}{2.991354in}}{\pgfqpoint{2.734232in}{2.983541in}}%
\pgfpathcurveto{\pgfqpoint{2.726418in}{2.975727in}}{\pgfqpoint{2.722028in}{2.965128in}}{\pgfqpoint{2.722028in}{2.954078in}}%
\pgfpathcurveto{\pgfqpoint{2.722028in}{2.943028in}}{\pgfqpoint{2.726418in}{2.932429in}}{\pgfqpoint{2.734232in}{2.924615in}}%
\pgfpathcurveto{\pgfqpoint{2.742045in}{2.916801in}}{\pgfqpoint{2.752644in}{2.912411in}}{\pgfqpoint{2.763695in}{2.912411in}}%
\pgfpathclose%
\pgfusepath{stroke,fill}%
\end{pgfscope}%
\begin{pgfscope}%
\pgfpathrectangle{\pgfqpoint{0.600000in}{0.600000in}}{\pgfqpoint{3.900000in}{3.900000in}}%
\pgfusepath{clip}%
\pgfsetbuttcap%
\pgfsetroundjoin%
\definecolor{currentfill}{rgb}{0.121569,0.466667,0.705882}%
\pgfsetfillcolor{currentfill}%
\pgfsetlinewidth{1.003750pt}%
\definecolor{currentstroke}{rgb}{0.121569,0.466667,0.705882}%
\pgfsetstrokecolor{currentstroke}%
\pgfsetdash{}{0pt}%
\pgfpathmoveto{\pgfqpoint{1.774264in}{2.618469in}}%
\pgfpathcurveto{\pgfqpoint{1.785314in}{2.618469in}}{\pgfqpoint{1.795913in}{2.622859in}}{\pgfqpoint{1.803726in}{2.630672in}}%
\pgfpathcurveto{\pgfqpoint{1.811540in}{2.638486in}}{\pgfqpoint{1.815930in}{2.649085in}}{\pgfqpoint{1.815930in}{2.660135in}}%
\pgfpathcurveto{\pgfqpoint{1.815930in}{2.671185in}}{\pgfqpoint{1.811540in}{2.681784in}}{\pgfqpoint{1.803726in}{2.689598in}}%
\pgfpathcurveto{\pgfqpoint{1.795913in}{2.697412in}}{\pgfqpoint{1.785314in}{2.701802in}}{\pgfqpoint{1.774264in}{2.701802in}}%
\pgfpathcurveto{\pgfqpoint{1.763214in}{2.701802in}}{\pgfqpoint{1.752614in}{2.697412in}}{\pgfqpoint{1.744801in}{2.689598in}}%
\pgfpathcurveto{\pgfqpoint{1.736987in}{2.681784in}}{\pgfqpoint{1.732597in}{2.671185in}}{\pgfqpoint{1.732597in}{2.660135in}}%
\pgfpathcurveto{\pgfqpoint{1.732597in}{2.649085in}}{\pgfqpoint{1.736987in}{2.638486in}}{\pgfqpoint{1.744801in}{2.630672in}}%
\pgfpathcurveto{\pgfqpoint{1.752614in}{2.622859in}}{\pgfqpoint{1.763214in}{2.618469in}}{\pgfqpoint{1.774264in}{2.618469in}}%
\pgfpathclose%
\pgfusepath{stroke,fill}%
\end{pgfscope}%
\begin{pgfscope}%
\pgfpathrectangle{\pgfqpoint{0.600000in}{0.600000in}}{\pgfqpoint{3.900000in}{3.900000in}}%
\pgfusepath{clip}%
\pgfsetbuttcap%
\pgfsetroundjoin%
\definecolor{currentfill}{rgb}{0.121569,0.466667,0.705882}%
\pgfsetfillcolor{currentfill}%
\pgfsetlinewidth{1.003750pt}%
\definecolor{currentstroke}{rgb}{0.121569,0.466667,0.705882}%
\pgfsetstrokecolor{currentstroke}%
\pgfsetdash{}{0pt}%
\pgfpathmoveto{\pgfqpoint{2.361019in}{3.046348in}}%
\pgfpathcurveto{\pgfqpoint{2.372069in}{3.046348in}}{\pgfqpoint{2.382668in}{3.050738in}}{\pgfqpoint{2.390482in}{3.058552in}}%
\pgfpathcurveto{\pgfqpoint{2.398295in}{3.066365in}}{\pgfqpoint{2.402686in}{3.076964in}}{\pgfqpoint{2.402686in}{3.088014in}}%
\pgfpathcurveto{\pgfqpoint{2.402686in}{3.099065in}}{\pgfqpoint{2.398295in}{3.109664in}}{\pgfqpoint{2.390482in}{3.117477in}}%
\pgfpathcurveto{\pgfqpoint{2.382668in}{3.125291in}}{\pgfqpoint{2.372069in}{3.129681in}}{\pgfqpoint{2.361019in}{3.129681in}}%
\pgfpathcurveto{\pgfqpoint{2.349969in}{3.129681in}}{\pgfqpoint{2.339370in}{3.125291in}}{\pgfqpoint{2.331556in}{3.117477in}}%
\pgfpathcurveto{\pgfqpoint{2.323742in}{3.109664in}}{\pgfqpoint{2.319352in}{3.099065in}}{\pgfqpoint{2.319352in}{3.088014in}}%
\pgfpathcurveto{\pgfqpoint{2.319352in}{3.076964in}}{\pgfqpoint{2.323742in}{3.066365in}}{\pgfqpoint{2.331556in}{3.058552in}}%
\pgfpathcurveto{\pgfqpoint{2.339370in}{3.050738in}}{\pgfqpoint{2.349969in}{3.046348in}}{\pgfqpoint{2.361019in}{3.046348in}}%
\pgfpathclose%
\pgfusepath{stroke,fill}%
\end{pgfscope}%
\begin{pgfscope}%
\pgfpathrectangle{\pgfqpoint{0.600000in}{0.600000in}}{\pgfqpoint{3.900000in}{3.900000in}}%
\pgfusepath{clip}%
\pgfsetbuttcap%
\pgfsetroundjoin%
\definecolor{currentfill}{rgb}{0.121569,0.466667,0.705882}%
\pgfsetfillcolor{currentfill}%
\pgfsetlinewidth{1.003750pt}%
\definecolor{currentstroke}{rgb}{0.121569,0.466667,0.705882}%
\pgfsetstrokecolor{currentstroke}%
\pgfsetdash{}{0pt}%
\pgfpathmoveto{\pgfqpoint{2.537826in}{1.583497in}}%
\pgfpathcurveto{\pgfqpoint{2.548877in}{1.583497in}}{\pgfqpoint{2.559476in}{1.587887in}}{\pgfqpoint{2.567289in}{1.595700in}}%
\pgfpathcurveto{\pgfqpoint{2.575103in}{1.603514in}}{\pgfqpoint{2.579493in}{1.614113in}}{\pgfqpoint{2.579493in}{1.625163in}}%
\pgfpathcurveto{\pgfqpoint{2.579493in}{1.636213in}}{\pgfqpoint{2.575103in}{1.646812in}}{\pgfqpoint{2.567289in}{1.654626in}}%
\pgfpathcurveto{\pgfqpoint{2.559476in}{1.662440in}}{\pgfqpoint{2.548877in}{1.666830in}}{\pgfqpoint{2.537826in}{1.666830in}}%
\pgfpathcurveto{\pgfqpoint{2.526776in}{1.666830in}}{\pgfqpoint{2.516177in}{1.662440in}}{\pgfqpoint{2.508364in}{1.654626in}}%
\pgfpathcurveto{\pgfqpoint{2.500550in}{1.646812in}}{\pgfqpoint{2.496160in}{1.636213in}}{\pgfqpoint{2.496160in}{1.625163in}}%
\pgfpathcurveto{\pgfqpoint{2.496160in}{1.614113in}}{\pgfqpoint{2.500550in}{1.603514in}}{\pgfqpoint{2.508364in}{1.595700in}}%
\pgfpathcurveto{\pgfqpoint{2.516177in}{1.587887in}}{\pgfqpoint{2.526776in}{1.583497in}}{\pgfqpoint{2.537826in}{1.583497in}}%
\pgfpathclose%
\pgfusepath{stroke,fill}%
\end{pgfscope}%
\begin{pgfscope}%
\pgfpathrectangle{\pgfqpoint{0.600000in}{0.600000in}}{\pgfqpoint{3.900000in}{3.900000in}}%
\pgfusepath{clip}%
\pgfsetbuttcap%
\pgfsetroundjoin%
\definecolor{currentfill}{rgb}{0.121569,0.466667,0.705882}%
\pgfsetfillcolor{currentfill}%
\pgfsetlinewidth{1.003750pt}%
\definecolor{currentstroke}{rgb}{0.121569,0.466667,0.705882}%
\pgfsetstrokecolor{currentstroke}%
\pgfsetdash{}{0pt}%
\pgfpathmoveto{\pgfqpoint{2.192903in}{2.707316in}}%
\pgfpathcurveto{\pgfqpoint{2.203953in}{2.707316in}}{\pgfqpoint{2.214552in}{2.711706in}}{\pgfqpoint{2.222365in}{2.719520in}}%
\pgfpathcurveto{\pgfqpoint{2.230179in}{2.727333in}}{\pgfqpoint{2.234569in}{2.737932in}}{\pgfqpoint{2.234569in}{2.748982in}}%
\pgfpathcurveto{\pgfqpoint{2.234569in}{2.760033in}}{\pgfqpoint{2.230179in}{2.770632in}}{\pgfqpoint{2.222365in}{2.778445in}}%
\pgfpathcurveto{\pgfqpoint{2.214552in}{2.786259in}}{\pgfqpoint{2.203953in}{2.790649in}}{\pgfqpoint{2.192903in}{2.790649in}}%
\pgfpathcurveto{\pgfqpoint{2.181853in}{2.790649in}}{\pgfqpoint{2.171253in}{2.786259in}}{\pgfqpoint{2.163440in}{2.778445in}}%
\pgfpathcurveto{\pgfqpoint{2.155626in}{2.770632in}}{\pgfqpoint{2.151236in}{2.760033in}}{\pgfqpoint{2.151236in}{2.748982in}}%
\pgfpathcurveto{\pgfqpoint{2.151236in}{2.737932in}}{\pgfqpoint{2.155626in}{2.727333in}}{\pgfqpoint{2.163440in}{2.719520in}}%
\pgfpathcurveto{\pgfqpoint{2.171253in}{2.711706in}}{\pgfqpoint{2.181853in}{2.707316in}}{\pgfqpoint{2.192903in}{2.707316in}}%
\pgfpathclose%
\pgfusepath{stroke,fill}%
\end{pgfscope}%
\begin{pgfscope}%
\pgfpathrectangle{\pgfqpoint{0.600000in}{0.600000in}}{\pgfqpoint{3.900000in}{3.900000in}}%
\pgfusepath{clip}%
\pgfsetbuttcap%
\pgfsetroundjoin%
\definecolor{currentfill}{rgb}{0.121569,0.466667,0.705882}%
\pgfsetfillcolor{currentfill}%
\pgfsetlinewidth{1.003750pt}%
\definecolor{currentstroke}{rgb}{0.121569,0.466667,0.705882}%
\pgfsetstrokecolor{currentstroke}%
\pgfsetdash{}{0pt}%
\pgfpathmoveto{\pgfqpoint{2.681177in}{2.586788in}}%
\pgfpathcurveto{\pgfqpoint{2.692228in}{2.586788in}}{\pgfqpoint{2.702827in}{2.591178in}}{\pgfqpoint{2.710640in}{2.598992in}}%
\pgfpathcurveto{\pgfqpoint{2.718454in}{2.606806in}}{\pgfqpoint{2.722844in}{2.617405in}}{\pgfqpoint{2.722844in}{2.628455in}}%
\pgfpathcurveto{\pgfqpoint{2.722844in}{2.639505in}}{\pgfqpoint{2.718454in}{2.650104in}}{\pgfqpoint{2.710640in}{2.657918in}}%
\pgfpathcurveto{\pgfqpoint{2.702827in}{2.665731in}}{\pgfqpoint{2.692228in}{2.670122in}}{\pgfqpoint{2.681177in}{2.670122in}}%
\pgfpathcurveto{\pgfqpoint{2.670127in}{2.670122in}}{\pgfqpoint{2.659528in}{2.665731in}}{\pgfqpoint{2.651715in}{2.657918in}}%
\pgfpathcurveto{\pgfqpoint{2.643901in}{2.650104in}}{\pgfqpoint{2.639511in}{2.639505in}}{\pgfqpoint{2.639511in}{2.628455in}}%
\pgfpathcurveto{\pgfqpoint{2.639511in}{2.617405in}}{\pgfqpoint{2.643901in}{2.606806in}}{\pgfqpoint{2.651715in}{2.598992in}}%
\pgfpathcurveto{\pgfqpoint{2.659528in}{2.591178in}}{\pgfqpoint{2.670127in}{2.586788in}}{\pgfqpoint{2.681177in}{2.586788in}}%
\pgfpathclose%
\pgfusepath{stroke,fill}%
\end{pgfscope}%
\begin{pgfscope}%
\pgfpathrectangle{\pgfqpoint{0.600000in}{0.600000in}}{\pgfqpoint{3.900000in}{3.900000in}}%
\pgfusepath{clip}%
\pgfsetbuttcap%
\pgfsetroundjoin%
\definecolor{currentfill}{rgb}{0.121569,0.466667,0.705882}%
\pgfsetfillcolor{currentfill}%
\pgfsetlinewidth{1.003750pt}%
\definecolor{currentstroke}{rgb}{0.121569,0.466667,0.705882}%
\pgfsetstrokecolor{currentstroke}%
\pgfsetdash{}{0pt}%
\pgfpathmoveto{\pgfqpoint{2.405121in}{2.905021in}}%
\pgfpathcurveto{\pgfqpoint{2.416171in}{2.905021in}}{\pgfqpoint{2.426770in}{2.909411in}}{\pgfqpoint{2.434584in}{2.917225in}}%
\pgfpathcurveto{\pgfqpoint{2.442397in}{2.925038in}}{\pgfqpoint{2.446788in}{2.935637in}}{\pgfqpoint{2.446788in}{2.946687in}}%
\pgfpathcurveto{\pgfqpoint{2.446788in}{2.957737in}}{\pgfqpoint{2.442397in}{2.968336in}}{\pgfqpoint{2.434584in}{2.976150in}}%
\pgfpathcurveto{\pgfqpoint{2.426770in}{2.983964in}}{\pgfqpoint{2.416171in}{2.988354in}}{\pgfqpoint{2.405121in}{2.988354in}}%
\pgfpathcurveto{\pgfqpoint{2.394071in}{2.988354in}}{\pgfqpoint{2.383472in}{2.983964in}}{\pgfqpoint{2.375658in}{2.976150in}}%
\pgfpathcurveto{\pgfqpoint{2.367845in}{2.968336in}}{\pgfqpoint{2.363454in}{2.957737in}}{\pgfqpoint{2.363454in}{2.946687in}}%
\pgfpathcurveto{\pgfqpoint{2.363454in}{2.935637in}}{\pgfqpoint{2.367845in}{2.925038in}}{\pgfqpoint{2.375658in}{2.917225in}}%
\pgfpathcurveto{\pgfqpoint{2.383472in}{2.909411in}}{\pgfqpoint{2.394071in}{2.905021in}}{\pgfqpoint{2.405121in}{2.905021in}}%
\pgfpathclose%
\pgfusepath{stroke,fill}%
\end{pgfscope}%
\begin{pgfscope}%
\pgfpathrectangle{\pgfqpoint{0.600000in}{0.600000in}}{\pgfqpoint{3.900000in}{3.900000in}}%
\pgfusepath{clip}%
\pgfsetbuttcap%
\pgfsetroundjoin%
\definecolor{currentfill}{rgb}{0.121569,0.466667,0.705882}%
\pgfsetfillcolor{currentfill}%
\pgfsetlinewidth{1.003750pt}%
\definecolor{currentstroke}{rgb}{0.121569,0.466667,0.705882}%
\pgfsetstrokecolor{currentstroke}%
\pgfsetdash{}{0pt}%
\pgfpathmoveto{\pgfqpoint{2.433168in}{2.357686in}}%
\pgfpathcurveto{\pgfqpoint{2.444218in}{2.357686in}}{\pgfqpoint{2.454817in}{2.362076in}}{\pgfqpoint{2.462631in}{2.369890in}}%
\pgfpathcurveto{\pgfqpoint{2.470444in}{2.377703in}}{\pgfqpoint{2.474835in}{2.388302in}}{\pgfqpoint{2.474835in}{2.399352in}}%
\pgfpathcurveto{\pgfqpoint{2.474835in}{2.410403in}}{\pgfqpoint{2.470444in}{2.421002in}}{\pgfqpoint{2.462631in}{2.428815in}}%
\pgfpathcurveto{\pgfqpoint{2.454817in}{2.436629in}}{\pgfqpoint{2.444218in}{2.441019in}}{\pgfqpoint{2.433168in}{2.441019in}}%
\pgfpathcurveto{\pgfqpoint{2.422118in}{2.441019in}}{\pgfqpoint{2.411519in}{2.436629in}}{\pgfqpoint{2.403705in}{2.428815in}}%
\pgfpathcurveto{\pgfqpoint{2.395892in}{2.421002in}}{\pgfqpoint{2.391501in}{2.410403in}}{\pgfqpoint{2.391501in}{2.399352in}}%
\pgfpathcurveto{\pgfqpoint{2.391501in}{2.388302in}}{\pgfqpoint{2.395892in}{2.377703in}}{\pgfqpoint{2.403705in}{2.369890in}}%
\pgfpathcurveto{\pgfqpoint{2.411519in}{2.362076in}}{\pgfqpoint{2.422118in}{2.357686in}}{\pgfqpoint{2.433168in}{2.357686in}}%
\pgfpathclose%
\pgfusepath{stroke,fill}%
\end{pgfscope}%
\begin{pgfscope}%
\pgfpathrectangle{\pgfqpoint{0.600000in}{0.600000in}}{\pgfqpoint{3.900000in}{3.900000in}}%
\pgfusepath{clip}%
\pgfsetbuttcap%
\pgfsetroundjoin%
\definecolor{currentfill}{rgb}{0.121569,0.466667,0.705882}%
\pgfsetfillcolor{currentfill}%
\pgfsetlinewidth{1.003750pt}%
\definecolor{currentstroke}{rgb}{0.121569,0.466667,0.705882}%
\pgfsetstrokecolor{currentstroke}%
\pgfsetdash{}{0pt}%
\pgfpathmoveto{\pgfqpoint{1.659890in}{2.566939in}}%
\pgfpathcurveto{\pgfqpoint{1.670940in}{2.566939in}}{\pgfqpoint{1.681539in}{2.571329in}}{\pgfqpoint{1.689353in}{2.579143in}}%
\pgfpathcurveto{\pgfqpoint{1.697167in}{2.586956in}}{\pgfqpoint{1.701557in}{2.597555in}}{\pgfqpoint{1.701557in}{2.608605in}}%
\pgfpathcurveto{\pgfqpoint{1.701557in}{2.619655in}}{\pgfqpoint{1.697167in}{2.630254in}}{\pgfqpoint{1.689353in}{2.638068in}}%
\pgfpathcurveto{\pgfqpoint{1.681539in}{2.645882in}}{\pgfqpoint{1.670940in}{2.650272in}}{\pgfqpoint{1.659890in}{2.650272in}}%
\pgfpathcurveto{\pgfqpoint{1.648840in}{2.650272in}}{\pgfqpoint{1.638241in}{2.645882in}}{\pgfqpoint{1.630428in}{2.638068in}}%
\pgfpathcurveto{\pgfqpoint{1.622614in}{2.630254in}}{\pgfqpoint{1.618224in}{2.619655in}}{\pgfqpoint{1.618224in}{2.608605in}}%
\pgfpathcurveto{\pgfqpoint{1.618224in}{2.597555in}}{\pgfqpoint{1.622614in}{2.586956in}}{\pgfqpoint{1.630428in}{2.579143in}}%
\pgfpathcurveto{\pgfqpoint{1.638241in}{2.571329in}}{\pgfqpoint{1.648840in}{2.566939in}}{\pgfqpoint{1.659890in}{2.566939in}}%
\pgfpathclose%
\pgfusepath{stroke,fill}%
\end{pgfscope}%
\begin{pgfscope}%
\pgfpathrectangle{\pgfqpoint{0.600000in}{0.600000in}}{\pgfqpoint{3.900000in}{3.900000in}}%
\pgfusepath{clip}%
\pgfsetbuttcap%
\pgfsetroundjoin%
\definecolor{currentfill}{rgb}{0.121569,0.466667,0.705882}%
\pgfsetfillcolor{currentfill}%
\pgfsetlinewidth{1.003750pt}%
\definecolor{currentstroke}{rgb}{0.121569,0.466667,0.705882}%
\pgfsetstrokecolor{currentstroke}%
\pgfsetdash{}{0pt}%
\pgfpathmoveto{\pgfqpoint{2.777714in}{2.374463in}}%
\pgfpathcurveto{\pgfqpoint{2.788764in}{2.374463in}}{\pgfqpoint{2.799363in}{2.378853in}}{\pgfqpoint{2.807177in}{2.386667in}}%
\pgfpathcurveto{\pgfqpoint{2.814990in}{2.394480in}}{\pgfqpoint{2.819381in}{2.405079in}}{\pgfqpoint{2.819381in}{2.416130in}}%
\pgfpathcurveto{\pgfqpoint{2.819381in}{2.427180in}}{\pgfqpoint{2.814990in}{2.437779in}}{\pgfqpoint{2.807177in}{2.445592in}}%
\pgfpathcurveto{\pgfqpoint{2.799363in}{2.453406in}}{\pgfqpoint{2.788764in}{2.457796in}}{\pgfqpoint{2.777714in}{2.457796in}}%
\pgfpathcurveto{\pgfqpoint{2.766664in}{2.457796in}}{\pgfqpoint{2.756065in}{2.453406in}}{\pgfqpoint{2.748251in}{2.445592in}}%
\pgfpathcurveto{\pgfqpoint{2.740437in}{2.437779in}}{\pgfqpoint{2.736047in}{2.427180in}}{\pgfqpoint{2.736047in}{2.416130in}}%
\pgfpathcurveto{\pgfqpoint{2.736047in}{2.405079in}}{\pgfqpoint{2.740437in}{2.394480in}}{\pgfqpoint{2.748251in}{2.386667in}}%
\pgfpathcurveto{\pgfqpoint{2.756065in}{2.378853in}}{\pgfqpoint{2.766664in}{2.374463in}}{\pgfqpoint{2.777714in}{2.374463in}}%
\pgfpathclose%
\pgfusepath{stroke,fill}%
\end{pgfscope}%
\begin{pgfscope}%
\pgfpathrectangle{\pgfqpoint{0.600000in}{0.600000in}}{\pgfqpoint{3.900000in}{3.900000in}}%
\pgfusepath{clip}%
\pgfsetbuttcap%
\pgfsetroundjoin%
\definecolor{currentfill}{rgb}{0.121569,0.466667,0.705882}%
\pgfsetfillcolor{currentfill}%
\pgfsetlinewidth{1.003750pt}%
\definecolor{currentstroke}{rgb}{0.121569,0.466667,0.705882}%
\pgfsetstrokecolor{currentstroke}%
\pgfsetdash{}{0pt}%
\pgfpathmoveto{\pgfqpoint{2.300912in}{2.913695in}}%
\pgfpathcurveto{\pgfqpoint{2.311962in}{2.913695in}}{\pgfqpoint{2.322561in}{2.918086in}}{\pgfqpoint{2.330374in}{2.925899in}}%
\pgfpathcurveto{\pgfqpoint{2.338188in}{2.933713in}}{\pgfqpoint{2.342578in}{2.944312in}}{\pgfqpoint{2.342578in}{2.955362in}}%
\pgfpathcurveto{\pgfqpoint{2.342578in}{2.966412in}}{\pgfqpoint{2.338188in}{2.977011in}}{\pgfqpoint{2.330374in}{2.984825in}}%
\pgfpathcurveto{\pgfqpoint{2.322561in}{2.992639in}}{\pgfqpoint{2.311962in}{2.997029in}}{\pgfqpoint{2.300912in}{2.997029in}}%
\pgfpathcurveto{\pgfqpoint{2.289861in}{2.997029in}}{\pgfqpoint{2.279262in}{2.992639in}}{\pgfqpoint{2.271449in}{2.984825in}}%
\pgfpathcurveto{\pgfqpoint{2.263635in}{2.977011in}}{\pgfqpoint{2.259245in}{2.966412in}}{\pgfqpoint{2.259245in}{2.955362in}}%
\pgfpathcurveto{\pgfqpoint{2.259245in}{2.944312in}}{\pgfqpoint{2.263635in}{2.933713in}}{\pgfqpoint{2.271449in}{2.925899in}}%
\pgfpathcurveto{\pgfqpoint{2.279262in}{2.918086in}}{\pgfqpoint{2.289861in}{2.913695in}}{\pgfqpoint{2.300912in}{2.913695in}}%
\pgfpathclose%
\pgfusepath{stroke,fill}%
\end{pgfscope}%
\begin{pgfscope}%
\pgfpathrectangle{\pgfqpoint{0.600000in}{0.600000in}}{\pgfqpoint{3.900000in}{3.900000in}}%
\pgfusepath{clip}%
\pgfsetbuttcap%
\pgfsetroundjoin%
\definecolor{currentfill}{rgb}{0.121569,0.466667,0.705882}%
\pgfsetfillcolor{currentfill}%
\pgfsetlinewidth{1.003750pt}%
\definecolor{currentstroke}{rgb}{0.121569,0.466667,0.705882}%
\pgfsetstrokecolor{currentstroke}%
\pgfsetdash{}{0pt}%
\pgfpathmoveto{\pgfqpoint{2.521032in}{2.618966in}}%
\pgfpathcurveto{\pgfqpoint{2.532083in}{2.618966in}}{\pgfqpoint{2.542682in}{2.623357in}}{\pgfqpoint{2.550495in}{2.631170in}}%
\pgfpathcurveto{\pgfqpoint{2.558309in}{2.638984in}}{\pgfqpoint{2.562699in}{2.649583in}}{\pgfqpoint{2.562699in}{2.660633in}}%
\pgfpathcurveto{\pgfqpoint{2.562699in}{2.671683in}}{\pgfqpoint{2.558309in}{2.682282in}}{\pgfqpoint{2.550495in}{2.690096in}}%
\pgfpathcurveto{\pgfqpoint{2.542682in}{2.697910in}}{\pgfqpoint{2.532083in}{2.702300in}}{\pgfqpoint{2.521032in}{2.702300in}}%
\pgfpathcurveto{\pgfqpoint{2.509982in}{2.702300in}}{\pgfqpoint{2.499383in}{2.697910in}}{\pgfqpoint{2.491570in}{2.690096in}}%
\pgfpathcurveto{\pgfqpoint{2.483756in}{2.682282in}}{\pgfqpoint{2.479366in}{2.671683in}}{\pgfqpoint{2.479366in}{2.660633in}}%
\pgfpathcurveto{\pgfqpoint{2.479366in}{2.649583in}}{\pgfqpoint{2.483756in}{2.638984in}}{\pgfqpoint{2.491570in}{2.631170in}}%
\pgfpathcurveto{\pgfqpoint{2.499383in}{2.623357in}}{\pgfqpoint{2.509982in}{2.618966in}}{\pgfqpoint{2.521032in}{2.618966in}}%
\pgfpathclose%
\pgfusepath{stroke,fill}%
\end{pgfscope}%
\begin{pgfscope}%
\pgfpathrectangle{\pgfqpoint{0.600000in}{0.600000in}}{\pgfqpoint{3.900000in}{3.900000in}}%
\pgfusepath{clip}%
\pgfsetbuttcap%
\pgfsetroundjoin%
\definecolor{currentfill}{rgb}{0.121569,0.466667,0.705882}%
\pgfsetfillcolor{currentfill}%
\pgfsetlinewidth{1.003750pt}%
\definecolor{currentstroke}{rgb}{0.121569,0.466667,0.705882}%
\pgfsetstrokecolor{currentstroke}%
\pgfsetdash{}{0pt}%
\pgfpathmoveto{\pgfqpoint{2.555793in}{2.370070in}}%
\pgfpathcurveto{\pgfqpoint{2.566843in}{2.370070in}}{\pgfqpoint{2.577442in}{2.374460in}}{\pgfqpoint{2.585256in}{2.382274in}}%
\pgfpathcurveto{\pgfqpoint{2.593069in}{2.390087in}}{\pgfqpoint{2.597459in}{2.400686in}}{\pgfqpoint{2.597459in}{2.411736in}}%
\pgfpathcurveto{\pgfqpoint{2.597459in}{2.422787in}}{\pgfqpoint{2.593069in}{2.433386in}}{\pgfqpoint{2.585256in}{2.441199in}}%
\pgfpathcurveto{\pgfqpoint{2.577442in}{2.449013in}}{\pgfqpoint{2.566843in}{2.453403in}}{\pgfqpoint{2.555793in}{2.453403in}}%
\pgfpathcurveto{\pgfqpoint{2.544743in}{2.453403in}}{\pgfqpoint{2.534144in}{2.449013in}}{\pgfqpoint{2.526330in}{2.441199in}}%
\pgfpathcurveto{\pgfqpoint{2.518516in}{2.433386in}}{\pgfqpoint{2.514126in}{2.422787in}}{\pgfqpoint{2.514126in}{2.411736in}}%
\pgfpathcurveto{\pgfqpoint{2.514126in}{2.400686in}}{\pgfqpoint{2.518516in}{2.390087in}}{\pgfqpoint{2.526330in}{2.382274in}}%
\pgfpathcurveto{\pgfqpoint{2.534144in}{2.374460in}}{\pgfqpoint{2.544743in}{2.370070in}}{\pgfqpoint{2.555793in}{2.370070in}}%
\pgfpathclose%
\pgfusepath{stroke,fill}%
\end{pgfscope}%
\begin{pgfscope}%
\pgfpathrectangle{\pgfqpoint{0.600000in}{0.600000in}}{\pgfqpoint{3.900000in}{3.900000in}}%
\pgfusepath{clip}%
\pgfsetbuttcap%
\pgfsetroundjoin%
\definecolor{currentfill}{rgb}{0.121569,0.466667,0.705882}%
\pgfsetfillcolor{currentfill}%
\pgfsetlinewidth{1.003750pt}%
\definecolor{currentstroke}{rgb}{0.121569,0.466667,0.705882}%
\pgfsetstrokecolor{currentstroke}%
\pgfsetdash{}{0pt}%
\pgfpathmoveto{\pgfqpoint{3.001480in}{2.354635in}}%
\pgfpathcurveto{\pgfqpoint{3.012531in}{2.354635in}}{\pgfqpoint{3.023130in}{2.359026in}}{\pgfqpoint{3.030943in}{2.366839in}}%
\pgfpathcurveto{\pgfqpoint{3.038757in}{2.374653in}}{\pgfqpoint{3.043147in}{2.385252in}}{\pgfqpoint{3.043147in}{2.396302in}}%
\pgfpathcurveto{\pgfqpoint{3.043147in}{2.407352in}}{\pgfqpoint{3.038757in}{2.417951in}}{\pgfqpoint{3.030943in}{2.425765in}}%
\pgfpathcurveto{\pgfqpoint{3.023130in}{2.433578in}}{\pgfqpoint{3.012531in}{2.437969in}}{\pgfqpoint{3.001480in}{2.437969in}}%
\pgfpathcurveto{\pgfqpoint{2.990430in}{2.437969in}}{\pgfqpoint{2.979831in}{2.433578in}}{\pgfqpoint{2.972018in}{2.425765in}}%
\pgfpathcurveto{\pgfqpoint{2.964204in}{2.417951in}}{\pgfqpoint{2.959814in}{2.407352in}}{\pgfqpoint{2.959814in}{2.396302in}}%
\pgfpathcurveto{\pgfqpoint{2.959814in}{2.385252in}}{\pgfqpoint{2.964204in}{2.374653in}}{\pgfqpoint{2.972018in}{2.366839in}}%
\pgfpathcurveto{\pgfqpoint{2.979831in}{2.359026in}}{\pgfqpoint{2.990430in}{2.354635in}}{\pgfqpoint{3.001480in}{2.354635in}}%
\pgfpathclose%
\pgfusepath{stroke,fill}%
\end{pgfscope}%
\begin{pgfscope}%
\pgfpathrectangle{\pgfqpoint{0.600000in}{0.600000in}}{\pgfqpoint{3.900000in}{3.900000in}}%
\pgfusepath{clip}%
\pgfsetbuttcap%
\pgfsetroundjoin%
\definecolor{currentfill}{rgb}{0.121569,0.466667,0.705882}%
\pgfsetfillcolor{currentfill}%
\pgfsetlinewidth{1.003750pt}%
\definecolor{currentstroke}{rgb}{0.121569,0.466667,0.705882}%
\pgfsetstrokecolor{currentstroke}%
\pgfsetdash{}{0pt}%
\pgfpathmoveto{\pgfqpoint{1.738138in}{1.448082in}}%
\pgfpathcurveto{\pgfqpoint{1.749189in}{1.448082in}}{\pgfqpoint{1.759788in}{1.452473in}}{\pgfqpoint{1.767601in}{1.460286in}}%
\pgfpathcurveto{\pgfqpoint{1.775415in}{1.468100in}}{\pgfqpoint{1.779805in}{1.478699in}}{\pgfqpoint{1.779805in}{1.489749in}}%
\pgfpathcurveto{\pgfqpoint{1.779805in}{1.500799in}}{\pgfqpoint{1.775415in}{1.511398in}}{\pgfqpoint{1.767601in}{1.519212in}}%
\pgfpathcurveto{\pgfqpoint{1.759788in}{1.527025in}}{\pgfqpoint{1.749189in}{1.531416in}}{\pgfqpoint{1.738138in}{1.531416in}}%
\pgfpathcurveto{\pgfqpoint{1.727088in}{1.531416in}}{\pgfqpoint{1.716489in}{1.527025in}}{\pgfqpoint{1.708676in}{1.519212in}}%
\pgfpathcurveto{\pgfqpoint{1.700862in}{1.511398in}}{\pgfqpoint{1.696472in}{1.500799in}}{\pgfqpoint{1.696472in}{1.489749in}}%
\pgfpathcurveto{\pgfqpoint{1.696472in}{1.478699in}}{\pgfqpoint{1.700862in}{1.468100in}}{\pgfqpoint{1.708676in}{1.460286in}}%
\pgfpathcurveto{\pgfqpoint{1.716489in}{1.452473in}}{\pgfqpoint{1.727088in}{1.448082in}}{\pgfqpoint{1.738138in}{1.448082in}}%
\pgfpathclose%
\pgfusepath{stroke,fill}%
\end{pgfscope}%
\begin{pgfscope}%
\pgfpathrectangle{\pgfqpoint{0.600000in}{0.600000in}}{\pgfqpoint{3.900000in}{3.900000in}}%
\pgfusepath{clip}%
\pgfsetbuttcap%
\pgfsetroundjoin%
\definecolor{currentfill}{rgb}{0.121569,0.466667,0.705882}%
\pgfsetfillcolor{currentfill}%
\pgfsetlinewidth{1.003750pt}%
\definecolor{currentstroke}{rgb}{0.121569,0.466667,0.705882}%
\pgfsetstrokecolor{currentstroke}%
\pgfsetdash{}{0pt}%
\pgfpathmoveto{\pgfqpoint{2.042661in}{1.188180in}}%
\pgfpathcurveto{\pgfqpoint{2.053711in}{1.188180in}}{\pgfqpoint{2.064310in}{1.192570in}}{\pgfqpoint{2.072124in}{1.200384in}}%
\pgfpathcurveto{\pgfqpoint{2.079937in}{1.208197in}}{\pgfqpoint{2.084328in}{1.218796in}}{\pgfqpoint{2.084328in}{1.229847in}}%
\pgfpathcurveto{\pgfqpoint{2.084328in}{1.240897in}}{\pgfqpoint{2.079937in}{1.251496in}}{\pgfqpoint{2.072124in}{1.259309in}}%
\pgfpathcurveto{\pgfqpoint{2.064310in}{1.267123in}}{\pgfqpoint{2.053711in}{1.271513in}}{\pgfqpoint{2.042661in}{1.271513in}}%
\pgfpathcurveto{\pgfqpoint{2.031611in}{1.271513in}}{\pgfqpoint{2.021012in}{1.267123in}}{\pgfqpoint{2.013198in}{1.259309in}}%
\pgfpathcurveto{\pgfqpoint{2.005385in}{1.251496in}}{\pgfqpoint{2.000994in}{1.240897in}}{\pgfqpoint{2.000994in}{1.229847in}}%
\pgfpathcurveto{\pgfqpoint{2.000994in}{1.218796in}}{\pgfqpoint{2.005385in}{1.208197in}}{\pgfqpoint{2.013198in}{1.200384in}}%
\pgfpathcurveto{\pgfqpoint{2.021012in}{1.192570in}}{\pgfqpoint{2.031611in}{1.188180in}}{\pgfqpoint{2.042661in}{1.188180in}}%
\pgfpathclose%
\pgfusepath{stroke,fill}%
\end{pgfscope}%
\begin{pgfscope}%
\pgfpathrectangle{\pgfqpoint{0.600000in}{0.600000in}}{\pgfqpoint{3.900000in}{3.900000in}}%
\pgfusepath{clip}%
\pgfsetbuttcap%
\pgfsetroundjoin%
\definecolor{currentfill}{rgb}{0.121569,0.466667,0.705882}%
\pgfsetfillcolor{currentfill}%
\pgfsetlinewidth{1.003750pt}%
\definecolor{currentstroke}{rgb}{0.121569,0.466667,0.705882}%
\pgfsetstrokecolor{currentstroke}%
\pgfsetdash{}{0pt}%
\pgfpathmoveto{\pgfqpoint{2.889526in}{1.551588in}}%
\pgfpathcurveto{\pgfqpoint{2.900576in}{1.551588in}}{\pgfqpoint{2.911175in}{1.555978in}}{\pgfqpoint{2.918988in}{1.563792in}}%
\pgfpathcurveto{\pgfqpoint{2.926802in}{1.571605in}}{\pgfqpoint{2.931192in}{1.582204in}}{\pgfqpoint{2.931192in}{1.593254in}}%
\pgfpathcurveto{\pgfqpoint{2.931192in}{1.604304in}}{\pgfqpoint{2.926802in}{1.614903in}}{\pgfqpoint{2.918988in}{1.622717in}}%
\pgfpathcurveto{\pgfqpoint{2.911175in}{1.630531in}}{\pgfqpoint{2.900576in}{1.634921in}}{\pgfqpoint{2.889526in}{1.634921in}}%
\pgfpathcurveto{\pgfqpoint{2.878475in}{1.634921in}}{\pgfqpoint{2.867876in}{1.630531in}}{\pgfqpoint{2.860063in}{1.622717in}}%
\pgfpathcurveto{\pgfqpoint{2.852249in}{1.614903in}}{\pgfqpoint{2.847859in}{1.604304in}}{\pgfqpoint{2.847859in}{1.593254in}}%
\pgfpathcurveto{\pgfqpoint{2.847859in}{1.582204in}}{\pgfqpoint{2.852249in}{1.571605in}}{\pgfqpoint{2.860063in}{1.563792in}}%
\pgfpathcurveto{\pgfqpoint{2.867876in}{1.555978in}}{\pgfqpoint{2.878475in}{1.551588in}}{\pgfqpoint{2.889526in}{1.551588in}}%
\pgfpathclose%
\pgfusepath{stroke,fill}%
\end{pgfscope}%
\begin{pgfscope}%
\pgfpathrectangle{\pgfqpoint{0.600000in}{0.600000in}}{\pgfqpoint{3.900000in}{3.900000in}}%
\pgfusepath{clip}%
\pgfsetbuttcap%
\pgfsetroundjoin%
\definecolor{currentfill}{rgb}{0.121569,0.466667,0.705882}%
\pgfsetfillcolor{currentfill}%
\pgfsetlinewidth{1.003750pt}%
\definecolor{currentstroke}{rgb}{0.121569,0.466667,0.705882}%
\pgfsetstrokecolor{currentstroke}%
\pgfsetdash{}{0pt}%
\pgfpathmoveto{\pgfqpoint{2.754131in}{1.617706in}}%
\pgfpathcurveto{\pgfqpoint{2.765181in}{1.617706in}}{\pgfqpoint{2.775780in}{1.622096in}}{\pgfqpoint{2.783594in}{1.629910in}}%
\pgfpathcurveto{\pgfqpoint{2.791407in}{1.637724in}}{\pgfqpoint{2.795798in}{1.648323in}}{\pgfqpoint{2.795798in}{1.659373in}}%
\pgfpathcurveto{\pgfqpoint{2.795798in}{1.670423in}}{\pgfqpoint{2.791407in}{1.681022in}}{\pgfqpoint{2.783594in}{1.688836in}}%
\pgfpathcurveto{\pgfqpoint{2.775780in}{1.696649in}}{\pgfqpoint{2.765181in}{1.701040in}}{\pgfqpoint{2.754131in}{1.701040in}}%
\pgfpathcurveto{\pgfqpoint{2.743081in}{1.701040in}}{\pgfqpoint{2.732482in}{1.696649in}}{\pgfqpoint{2.724668in}{1.688836in}}%
\pgfpathcurveto{\pgfqpoint{2.716854in}{1.681022in}}{\pgfqpoint{2.712464in}{1.670423in}}{\pgfqpoint{2.712464in}{1.659373in}}%
\pgfpathcurveto{\pgfqpoint{2.712464in}{1.648323in}}{\pgfqpoint{2.716854in}{1.637724in}}{\pgfqpoint{2.724668in}{1.629910in}}%
\pgfpathcurveto{\pgfqpoint{2.732482in}{1.622096in}}{\pgfqpoint{2.743081in}{1.617706in}}{\pgfqpoint{2.754131in}{1.617706in}}%
\pgfpathclose%
\pgfusepath{stroke,fill}%
\end{pgfscope}%
\begin{pgfscope}%
\pgfpathrectangle{\pgfqpoint{0.600000in}{0.600000in}}{\pgfqpoint{3.900000in}{3.900000in}}%
\pgfusepath{clip}%
\pgfsetbuttcap%
\pgfsetroundjoin%
\definecolor{currentfill}{rgb}{0.121569,0.466667,0.705882}%
\pgfsetfillcolor{currentfill}%
\pgfsetlinewidth{1.003750pt}%
\definecolor{currentstroke}{rgb}{0.121569,0.466667,0.705882}%
\pgfsetstrokecolor{currentstroke}%
\pgfsetdash{}{0pt}%
\pgfpathmoveto{\pgfqpoint{2.299831in}{2.813646in}}%
\pgfpathcurveto{\pgfqpoint{2.310881in}{2.813646in}}{\pgfqpoint{2.321480in}{2.818036in}}{\pgfqpoint{2.329293in}{2.825850in}}%
\pgfpathcurveto{\pgfqpoint{2.337107in}{2.833664in}}{\pgfqpoint{2.341497in}{2.844263in}}{\pgfqpoint{2.341497in}{2.855313in}}%
\pgfpathcurveto{\pgfqpoint{2.341497in}{2.866363in}}{\pgfqpoint{2.337107in}{2.876962in}}{\pgfqpoint{2.329293in}{2.884776in}}%
\pgfpathcurveto{\pgfqpoint{2.321480in}{2.892589in}}{\pgfqpoint{2.310881in}{2.896980in}}{\pgfqpoint{2.299831in}{2.896980in}}%
\pgfpathcurveto{\pgfqpoint{2.288780in}{2.896980in}}{\pgfqpoint{2.278181in}{2.892589in}}{\pgfqpoint{2.270368in}{2.884776in}}%
\pgfpathcurveto{\pgfqpoint{2.262554in}{2.876962in}}{\pgfqpoint{2.258164in}{2.866363in}}{\pgfqpoint{2.258164in}{2.855313in}}%
\pgfpathcurveto{\pgfqpoint{2.258164in}{2.844263in}}{\pgfqpoint{2.262554in}{2.833664in}}{\pgfqpoint{2.270368in}{2.825850in}}%
\pgfpathcurveto{\pgfqpoint{2.278181in}{2.818036in}}{\pgfqpoint{2.288780in}{2.813646in}}{\pgfqpoint{2.299831in}{2.813646in}}%
\pgfpathclose%
\pgfusepath{stroke,fill}%
\end{pgfscope}%
\begin{pgfscope}%
\pgfpathrectangle{\pgfqpoint{0.600000in}{0.600000in}}{\pgfqpoint{3.900000in}{3.900000in}}%
\pgfusepath{clip}%
\pgfsetbuttcap%
\pgfsetroundjoin%
\definecolor{currentfill}{rgb}{0.121569,0.466667,0.705882}%
\pgfsetfillcolor{currentfill}%
\pgfsetlinewidth{1.003750pt}%
\definecolor{currentstroke}{rgb}{0.121569,0.466667,0.705882}%
\pgfsetstrokecolor{currentstroke}%
\pgfsetdash{}{0pt}%
\pgfpathmoveto{\pgfqpoint{1.681927in}{2.524120in}}%
\pgfpathcurveto{\pgfqpoint{1.692977in}{2.524120in}}{\pgfqpoint{1.703576in}{2.528511in}}{\pgfqpoint{1.711389in}{2.536324in}}%
\pgfpathcurveto{\pgfqpoint{1.719203in}{2.544138in}}{\pgfqpoint{1.723593in}{2.554737in}}{\pgfqpoint{1.723593in}{2.565787in}}%
\pgfpathcurveto{\pgfqpoint{1.723593in}{2.576837in}}{\pgfqpoint{1.719203in}{2.587436in}}{\pgfqpoint{1.711389in}{2.595250in}}%
\pgfpathcurveto{\pgfqpoint{1.703576in}{2.603063in}}{\pgfqpoint{1.692977in}{2.607454in}}{\pgfqpoint{1.681927in}{2.607454in}}%
\pgfpathcurveto{\pgfqpoint{1.670877in}{2.607454in}}{\pgfqpoint{1.660278in}{2.603063in}}{\pgfqpoint{1.652464in}{2.595250in}}%
\pgfpathcurveto{\pgfqpoint{1.644650in}{2.587436in}}{\pgfqpoint{1.640260in}{2.576837in}}{\pgfqpoint{1.640260in}{2.565787in}}%
\pgfpathcurveto{\pgfqpoint{1.640260in}{2.554737in}}{\pgfqpoint{1.644650in}{2.544138in}}{\pgfqpoint{1.652464in}{2.536324in}}%
\pgfpathcurveto{\pgfqpoint{1.660278in}{2.528511in}}{\pgfqpoint{1.670877in}{2.524120in}}{\pgfqpoint{1.681927in}{2.524120in}}%
\pgfpathclose%
\pgfusepath{stroke,fill}%
\end{pgfscope}%
\begin{pgfscope}%
\pgfpathrectangle{\pgfqpoint{0.600000in}{0.600000in}}{\pgfqpoint{3.900000in}{3.900000in}}%
\pgfusepath{clip}%
\pgfsetbuttcap%
\pgfsetroundjoin%
\definecolor{currentfill}{rgb}{0.121569,0.466667,0.705882}%
\pgfsetfillcolor{currentfill}%
\pgfsetlinewidth{1.003750pt}%
\definecolor{currentstroke}{rgb}{0.121569,0.466667,0.705882}%
\pgfsetstrokecolor{currentstroke}%
\pgfsetdash{}{0pt}%
\pgfpathmoveto{\pgfqpoint{2.281337in}{2.303481in}}%
\pgfpathcurveto{\pgfqpoint{2.292387in}{2.303481in}}{\pgfqpoint{2.302986in}{2.307871in}}{\pgfqpoint{2.310800in}{2.315685in}}%
\pgfpathcurveto{\pgfqpoint{2.318614in}{2.323499in}}{\pgfqpoint{2.323004in}{2.334098in}}{\pgfqpoint{2.323004in}{2.345148in}}%
\pgfpathcurveto{\pgfqpoint{2.323004in}{2.356198in}}{\pgfqpoint{2.318614in}{2.366797in}}{\pgfqpoint{2.310800in}{2.374611in}}%
\pgfpathcurveto{\pgfqpoint{2.302986in}{2.382424in}}{\pgfqpoint{2.292387in}{2.386814in}}{\pgfqpoint{2.281337in}{2.386814in}}%
\pgfpathcurveto{\pgfqpoint{2.270287in}{2.386814in}}{\pgfqpoint{2.259688in}{2.382424in}}{\pgfqpoint{2.251874in}{2.374611in}}%
\pgfpathcurveto{\pgfqpoint{2.244061in}{2.366797in}}{\pgfqpoint{2.239670in}{2.356198in}}{\pgfqpoint{2.239670in}{2.345148in}}%
\pgfpathcurveto{\pgfqpoint{2.239670in}{2.334098in}}{\pgfqpoint{2.244061in}{2.323499in}}{\pgfqpoint{2.251874in}{2.315685in}}%
\pgfpathcurveto{\pgfqpoint{2.259688in}{2.307871in}}{\pgfqpoint{2.270287in}{2.303481in}}{\pgfqpoint{2.281337in}{2.303481in}}%
\pgfpathclose%
\pgfusepath{stroke,fill}%
\end{pgfscope}%
\begin{pgfscope}%
\pgfpathrectangle{\pgfqpoint{0.600000in}{0.600000in}}{\pgfqpoint{3.900000in}{3.900000in}}%
\pgfusepath{clip}%
\pgfsetbuttcap%
\pgfsetroundjoin%
\definecolor{currentfill}{rgb}{0.121569,0.466667,0.705882}%
\pgfsetfillcolor{currentfill}%
\pgfsetlinewidth{1.003750pt}%
\definecolor{currentstroke}{rgb}{0.121569,0.466667,0.705882}%
\pgfsetstrokecolor{currentstroke}%
\pgfsetdash{}{0pt}%
\pgfpathmoveto{\pgfqpoint{2.214273in}{3.112578in}}%
\pgfpathcurveto{\pgfqpoint{2.225323in}{3.112578in}}{\pgfqpoint{2.235922in}{3.116968in}}{\pgfqpoint{2.243736in}{3.124782in}}%
\pgfpathcurveto{\pgfqpoint{2.251549in}{3.132595in}}{\pgfqpoint{2.255940in}{3.143194in}}{\pgfqpoint{2.255940in}{3.154244in}}%
\pgfpathcurveto{\pgfqpoint{2.255940in}{3.165294in}}{\pgfqpoint{2.251549in}{3.175894in}}{\pgfqpoint{2.243736in}{3.183707in}}%
\pgfpathcurveto{\pgfqpoint{2.235922in}{3.191521in}}{\pgfqpoint{2.225323in}{3.195911in}}{\pgfqpoint{2.214273in}{3.195911in}}%
\pgfpathcurveto{\pgfqpoint{2.203223in}{3.195911in}}{\pgfqpoint{2.192624in}{3.191521in}}{\pgfqpoint{2.184810in}{3.183707in}}%
\pgfpathcurveto{\pgfqpoint{2.176997in}{3.175894in}}{\pgfqpoint{2.172606in}{3.165294in}}{\pgfqpoint{2.172606in}{3.154244in}}%
\pgfpathcurveto{\pgfqpoint{2.172606in}{3.143194in}}{\pgfqpoint{2.176997in}{3.132595in}}{\pgfqpoint{2.184810in}{3.124782in}}%
\pgfpathcurveto{\pgfqpoint{2.192624in}{3.116968in}}{\pgfqpoint{2.203223in}{3.112578in}}{\pgfqpoint{2.214273in}{3.112578in}}%
\pgfpathclose%
\pgfusepath{stroke,fill}%
\end{pgfscope}%
\begin{pgfscope}%
\pgfpathrectangle{\pgfqpoint{0.600000in}{0.600000in}}{\pgfqpoint{3.900000in}{3.900000in}}%
\pgfusepath{clip}%
\pgfsetbuttcap%
\pgfsetroundjoin%
\definecolor{currentfill}{rgb}{0.121569,0.466667,0.705882}%
\pgfsetfillcolor{currentfill}%
\pgfsetlinewidth{1.003750pt}%
\definecolor{currentstroke}{rgb}{0.121569,0.466667,0.705882}%
\pgfsetstrokecolor{currentstroke}%
\pgfsetdash{}{0pt}%
\pgfpathmoveto{\pgfqpoint{3.399989in}{2.761020in}}%
\pgfpathcurveto{\pgfqpoint{3.411039in}{2.761020in}}{\pgfqpoint{3.421638in}{2.765410in}}{\pgfqpoint{3.429452in}{2.773224in}}%
\pgfpathcurveto{\pgfqpoint{3.437266in}{2.781037in}}{\pgfqpoint{3.441656in}{2.791636in}}{\pgfqpoint{3.441656in}{2.802687in}}%
\pgfpathcurveto{\pgfqpoint{3.441656in}{2.813737in}}{\pgfqpoint{3.437266in}{2.824336in}}{\pgfqpoint{3.429452in}{2.832149in}}%
\pgfpathcurveto{\pgfqpoint{3.421638in}{2.839963in}}{\pgfqpoint{3.411039in}{2.844353in}}{\pgfqpoint{3.399989in}{2.844353in}}%
\pgfpathcurveto{\pgfqpoint{3.388939in}{2.844353in}}{\pgfqpoint{3.378340in}{2.839963in}}{\pgfqpoint{3.370527in}{2.832149in}}%
\pgfpathcurveto{\pgfqpoint{3.362713in}{2.824336in}}{\pgfqpoint{3.358323in}{2.813737in}}{\pgfqpoint{3.358323in}{2.802687in}}%
\pgfpathcurveto{\pgfqpoint{3.358323in}{2.791636in}}{\pgfqpoint{3.362713in}{2.781037in}}{\pgfqpoint{3.370527in}{2.773224in}}%
\pgfpathcurveto{\pgfqpoint{3.378340in}{2.765410in}}{\pgfqpoint{3.388939in}{2.761020in}}{\pgfqpoint{3.399989in}{2.761020in}}%
\pgfpathclose%
\pgfusepath{stroke,fill}%
\end{pgfscope}%
\begin{pgfscope}%
\pgfpathrectangle{\pgfqpoint{0.600000in}{0.600000in}}{\pgfqpoint{3.900000in}{3.900000in}}%
\pgfusepath{clip}%
\pgfsetbuttcap%
\pgfsetroundjoin%
\definecolor{currentfill}{rgb}{0.121569,0.466667,0.705882}%
\pgfsetfillcolor{currentfill}%
\pgfsetlinewidth{1.003750pt}%
\definecolor{currentstroke}{rgb}{0.121569,0.466667,0.705882}%
\pgfsetstrokecolor{currentstroke}%
\pgfsetdash{}{0pt}%
\pgfpathmoveto{\pgfqpoint{2.894359in}{2.433247in}}%
\pgfpathcurveto{\pgfqpoint{2.905409in}{2.433247in}}{\pgfqpoint{2.916008in}{2.437637in}}{\pgfqpoint{2.923821in}{2.445450in}}%
\pgfpathcurveto{\pgfqpoint{2.931635in}{2.453264in}}{\pgfqpoint{2.936025in}{2.463863in}}{\pgfqpoint{2.936025in}{2.474913in}}%
\pgfpathcurveto{\pgfqpoint{2.936025in}{2.485963in}}{\pgfqpoint{2.931635in}{2.496562in}}{\pgfqpoint{2.923821in}{2.504376in}}%
\pgfpathcurveto{\pgfqpoint{2.916008in}{2.512190in}}{\pgfqpoint{2.905409in}{2.516580in}}{\pgfqpoint{2.894359in}{2.516580in}}%
\pgfpathcurveto{\pgfqpoint{2.883308in}{2.516580in}}{\pgfqpoint{2.872709in}{2.512190in}}{\pgfqpoint{2.864896in}{2.504376in}}%
\pgfpathcurveto{\pgfqpoint{2.857082in}{2.496562in}}{\pgfqpoint{2.852692in}{2.485963in}}{\pgfqpoint{2.852692in}{2.474913in}}%
\pgfpathcurveto{\pgfqpoint{2.852692in}{2.463863in}}{\pgfqpoint{2.857082in}{2.453264in}}{\pgfqpoint{2.864896in}{2.445450in}}%
\pgfpathcurveto{\pgfqpoint{2.872709in}{2.437637in}}{\pgfqpoint{2.883308in}{2.433247in}}{\pgfqpoint{2.894359in}{2.433247in}}%
\pgfpathclose%
\pgfusepath{stroke,fill}%
\end{pgfscope}%
\begin{pgfscope}%
\pgfpathrectangle{\pgfqpoint{0.600000in}{0.600000in}}{\pgfqpoint{3.900000in}{3.900000in}}%
\pgfusepath{clip}%
\pgfsetbuttcap%
\pgfsetroundjoin%
\definecolor{currentfill}{rgb}{0.121569,0.466667,0.705882}%
\pgfsetfillcolor{currentfill}%
\pgfsetlinewidth{1.003750pt}%
\definecolor{currentstroke}{rgb}{0.121569,0.466667,0.705882}%
\pgfsetstrokecolor{currentstroke}%
\pgfsetdash{}{0pt}%
\pgfpathmoveto{\pgfqpoint{2.508203in}{2.435857in}}%
\pgfpathcurveto{\pgfqpoint{2.519253in}{2.435857in}}{\pgfqpoint{2.529852in}{2.440247in}}{\pgfqpoint{2.537666in}{2.448061in}}%
\pgfpathcurveto{\pgfqpoint{2.545479in}{2.455874in}}{\pgfqpoint{2.549869in}{2.466473in}}{\pgfqpoint{2.549869in}{2.477524in}}%
\pgfpathcurveto{\pgfqpoint{2.549869in}{2.488574in}}{\pgfqpoint{2.545479in}{2.499173in}}{\pgfqpoint{2.537666in}{2.506986in}}%
\pgfpathcurveto{\pgfqpoint{2.529852in}{2.514800in}}{\pgfqpoint{2.519253in}{2.519190in}}{\pgfqpoint{2.508203in}{2.519190in}}%
\pgfpathcurveto{\pgfqpoint{2.497153in}{2.519190in}}{\pgfqpoint{2.486554in}{2.514800in}}{\pgfqpoint{2.478740in}{2.506986in}}%
\pgfpathcurveto{\pgfqpoint{2.470926in}{2.499173in}}{\pgfqpoint{2.466536in}{2.488574in}}{\pgfqpoint{2.466536in}{2.477524in}}%
\pgfpathcurveto{\pgfqpoint{2.466536in}{2.466473in}}{\pgfqpoint{2.470926in}{2.455874in}}{\pgfqpoint{2.478740in}{2.448061in}}%
\pgfpathcurveto{\pgfqpoint{2.486554in}{2.440247in}}{\pgfqpoint{2.497153in}{2.435857in}}{\pgfqpoint{2.508203in}{2.435857in}}%
\pgfpathclose%
\pgfusepath{stroke,fill}%
\end{pgfscope}%
\begin{pgfscope}%
\pgfpathrectangle{\pgfqpoint{0.600000in}{0.600000in}}{\pgfqpoint{3.900000in}{3.900000in}}%
\pgfusepath{clip}%
\pgfsetbuttcap%
\pgfsetroundjoin%
\definecolor{currentfill}{rgb}{0.121569,0.466667,0.705882}%
\pgfsetfillcolor{currentfill}%
\pgfsetlinewidth{1.003750pt}%
\definecolor{currentstroke}{rgb}{0.121569,0.466667,0.705882}%
\pgfsetstrokecolor{currentstroke}%
\pgfsetdash{}{0pt}%
\pgfpathmoveto{\pgfqpoint{2.568633in}{2.381298in}}%
\pgfpathcurveto{\pgfqpoint{2.579683in}{2.381298in}}{\pgfqpoint{2.590282in}{2.385688in}}{\pgfqpoint{2.598096in}{2.393502in}}%
\pgfpathcurveto{\pgfqpoint{2.605910in}{2.401316in}}{\pgfqpoint{2.610300in}{2.411915in}}{\pgfqpoint{2.610300in}{2.422965in}}%
\pgfpathcurveto{\pgfqpoint{2.610300in}{2.434015in}}{\pgfqpoint{2.605910in}{2.444614in}}{\pgfqpoint{2.598096in}{2.452427in}}%
\pgfpathcurveto{\pgfqpoint{2.590282in}{2.460241in}}{\pgfqpoint{2.579683in}{2.464631in}}{\pgfqpoint{2.568633in}{2.464631in}}%
\pgfpathcurveto{\pgfqpoint{2.557583in}{2.464631in}}{\pgfqpoint{2.546984in}{2.460241in}}{\pgfqpoint{2.539170in}{2.452427in}}%
\pgfpathcurveto{\pgfqpoint{2.531357in}{2.444614in}}{\pgfqpoint{2.526966in}{2.434015in}}{\pgfqpoint{2.526966in}{2.422965in}}%
\pgfpathcurveto{\pgfqpoint{2.526966in}{2.411915in}}{\pgfqpoint{2.531357in}{2.401316in}}{\pgfqpoint{2.539170in}{2.393502in}}%
\pgfpathcurveto{\pgfqpoint{2.546984in}{2.385688in}}{\pgfqpoint{2.557583in}{2.381298in}}{\pgfqpoint{2.568633in}{2.381298in}}%
\pgfpathclose%
\pgfusepath{stroke,fill}%
\end{pgfscope}%
\begin{pgfscope}%
\pgfpathrectangle{\pgfqpoint{0.600000in}{0.600000in}}{\pgfqpoint{3.900000in}{3.900000in}}%
\pgfusepath{clip}%
\pgfsetbuttcap%
\pgfsetroundjoin%
\definecolor{currentfill}{rgb}{0.121569,0.466667,0.705882}%
\pgfsetfillcolor{currentfill}%
\pgfsetlinewidth{1.003750pt}%
\definecolor{currentstroke}{rgb}{0.121569,0.466667,0.705882}%
\pgfsetstrokecolor{currentstroke}%
\pgfsetdash{}{0pt}%
\pgfpathmoveto{\pgfqpoint{1.768501in}{2.783570in}}%
\pgfpathcurveto{\pgfqpoint{1.779551in}{2.783570in}}{\pgfqpoint{1.790150in}{2.787960in}}{\pgfqpoint{1.797964in}{2.795774in}}%
\pgfpathcurveto{\pgfqpoint{1.805777in}{2.803587in}}{\pgfqpoint{1.810167in}{2.814186in}}{\pgfqpoint{1.810167in}{2.825236in}}%
\pgfpathcurveto{\pgfqpoint{1.810167in}{2.836286in}}{\pgfqpoint{1.805777in}{2.846886in}}{\pgfqpoint{1.797964in}{2.854699in}}%
\pgfpathcurveto{\pgfqpoint{1.790150in}{2.862513in}}{\pgfqpoint{1.779551in}{2.866903in}}{\pgfqpoint{1.768501in}{2.866903in}}%
\pgfpathcurveto{\pgfqpoint{1.757451in}{2.866903in}}{\pgfqpoint{1.746852in}{2.862513in}}{\pgfqpoint{1.739038in}{2.854699in}}%
\pgfpathcurveto{\pgfqpoint{1.731224in}{2.846886in}}{\pgfqpoint{1.726834in}{2.836286in}}{\pgfqpoint{1.726834in}{2.825236in}}%
\pgfpathcurveto{\pgfqpoint{1.726834in}{2.814186in}}{\pgfqpoint{1.731224in}{2.803587in}}{\pgfqpoint{1.739038in}{2.795774in}}%
\pgfpathcurveto{\pgfqpoint{1.746852in}{2.787960in}}{\pgfqpoint{1.757451in}{2.783570in}}{\pgfqpoint{1.768501in}{2.783570in}}%
\pgfpathclose%
\pgfusepath{stroke,fill}%
\end{pgfscope}%
\begin{pgfscope}%
\pgfpathrectangle{\pgfqpoint{0.600000in}{0.600000in}}{\pgfqpoint{3.900000in}{3.900000in}}%
\pgfusepath{clip}%
\pgfsetbuttcap%
\pgfsetroundjoin%
\definecolor{currentfill}{rgb}{0.121569,0.466667,0.705882}%
\pgfsetfillcolor{currentfill}%
\pgfsetlinewidth{1.003750pt}%
\definecolor{currentstroke}{rgb}{0.121569,0.466667,0.705882}%
\pgfsetstrokecolor{currentstroke}%
\pgfsetdash{}{0pt}%
\pgfpathmoveto{\pgfqpoint{2.174498in}{3.361719in}}%
\pgfpathcurveto{\pgfqpoint{2.185548in}{3.361719in}}{\pgfqpoint{2.196147in}{3.366109in}}{\pgfqpoint{2.203961in}{3.373922in}}%
\pgfpathcurveto{\pgfqpoint{2.211775in}{3.381736in}}{\pgfqpoint{2.216165in}{3.392335in}}{\pgfqpoint{2.216165in}{3.403385in}}%
\pgfpathcurveto{\pgfqpoint{2.216165in}{3.414435in}}{\pgfqpoint{2.211775in}{3.425034in}}{\pgfqpoint{2.203961in}{3.432848in}}%
\pgfpathcurveto{\pgfqpoint{2.196147in}{3.440662in}}{\pgfqpoint{2.185548in}{3.445052in}}{\pgfqpoint{2.174498in}{3.445052in}}%
\pgfpathcurveto{\pgfqpoint{2.163448in}{3.445052in}}{\pgfqpoint{2.152849in}{3.440662in}}{\pgfqpoint{2.145035in}{3.432848in}}%
\pgfpathcurveto{\pgfqpoint{2.137222in}{3.425034in}}{\pgfqpoint{2.132831in}{3.414435in}}{\pgfqpoint{2.132831in}{3.403385in}}%
\pgfpathcurveto{\pgfqpoint{2.132831in}{3.392335in}}{\pgfqpoint{2.137222in}{3.381736in}}{\pgfqpoint{2.145035in}{3.373922in}}%
\pgfpathcurveto{\pgfqpoint{2.152849in}{3.366109in}}{\pgfqpoint{2.163448in}{3.361719in}}{\pgfqpoint{2.174498in}{3.361719in}}%
\pgfpathclose%
\pgfusepath{stroke,fill}%
\end{pgfscope}%
\begin{pgfscope}%
\pgfpathrectangle{\pgfqpoint{0.600000in}{0.600000in}}{\pgfqpoint{3.900000in}{3.900000in}}%
\pgfusepath{clip}%
\pgfsetbuttcap%
\pgfsetroundjoin%
\definecolor{currentfill}{rgb}{0.121569,0.466667,0.705882}%
\pgfsetfillcolor{currentfill}%
\pgfsetlinewidth{1.003750pt}%
\definecolor{currentstroke}{rgb}{0.121569,0.466667,0.705882}%
\pgfsetstrokecolor{currentstroke}%
\pgfsetdash{}{0pt}%
\pgfpathmoveto{\pgfqpoint{1.569541in}{2.787877in}}%
\pgfpathcurveto{\pgfqpoint{1.580592in}{2.787877in}}{\pgfqpoint{1.591191in}{2.792268in}}{\pgfqpoint{1.599004in}{2.800081in}}%
\pgfpathcurveto{\pgfqpoint{1.606818in}{2.807895in}}{\pgfqpoint{1.611208in}{2.818494in}}{\pgfqpoint{1.611208in}{2.829544in}}%
\pgfpathcurveto{\pgfqpoint{1.611208in}{2.840594in}}{\pgfqpoint{1.606818in}{2.851193in}}{\pgfqpoint{1.599004in}{2.859007in}}%
\pgfpathcurveto{\pgfqpoint{1.591191in}{2.866820in}}{\pgfqpoint{1.580592in}{2.871211in}}{\pgfqpoint{1.569541in}{2.871211in}}%
\pgfpathcurveto{\pgfqpoint{1.558491in}{2.871211in}}{\pgfqpoint{1.547892in}{2.866820in}}{\pgfqpoint{1.540079in}{2.859007in}}%
\pgfpathcurveto{\pgfqpoint{1.532265in}{2.851193in}}{\pgfqpoint{1.527875in}{2.840594in}}{\pgfqpoint{1.527875in}{2.829544in}}%
\pgfpathcurveto{\pgfqpoint{1.527875in}{2.818494in}}{\pgfqpoint{1.532265in}{2.807895in}}{\pgfqpoint{1.540079in}{2.800081in}}%
\pgfpathcurveto{\pgfqpoint{1.547892in}{2.792268in}}{\pgfqpoint{1.558491in}{2.787877in}}{\pgfqpoint{1.569541in}{2.787877in}}%
\pgfpathclose%
\pgfusepath{stroke,fill}%
\end{pgfscope}%
\begin{pgfscope}%
\pgfpathrectangle{\pgfqpoint{0.600000in}{0.600000in}}{\pgfqpoint{3.900000in}{3.900000in}}%
\pgfusepath{clip}%
\pgfsetbuttcap%
\pgfsetroundjoin%
\definecolor{currentfill}{rgb}{0.121569,0.466667,0.705882}%
\pgfsetfillcolor{currentfill}%
\pgfsetlinewidth{1.003750pt}%
\definecolor{currentstroke}{rgb}{0.121569,0.466667,0.705882}%
\pgfsetstrokecolor{currentstroke}%
\pgfsetdash{}{0pt}%
\pgfpathmoveto{\pgfqpoint{3.099169in}{2.549265in}}%
\pgfpathcurveto{\pgfqpoint{3.110219in}{2.549265in}}{\pgfqpoint{3.120818in}{2.553655in}}{\pgfqpoint{3.128632in}{2.561468in}}%
\pgfpathcurveto{\pgfqpoint{3.136445in}{2.569282in}}{\pgfqpoint{3.140836in}{2.579881in}}{\pgfqpoint{3.140836in}{2.590931in}}%
\pgfpathcurveto{\pgfqpoint{3.140836in}{2.601981in}}{\pgfqpoint{3.136445in}{2.612580in}}{\pgfqpoint{3.128632in}{2.620394in}}%
\pgfpathcurveto{\pgfqpoint{3.120818in}{2.628208in}}{\pgfqpoint{3.110219in}{2.632598in}}{\pgfqpoint{3.099169in}{2.632598in}}%
\pgfpathcurveto{\pgfqpoint{3.088119in}{2.632598in}}{\pgfqpoint{3.077520in}{2.628208in}}{\pgfqpoint{3.069706in}{2.620394in}}%
\pgfpathcurveto{\pgfqpoint{3.061892in}{2.612580in}}{\pgfqpoint{3.057502in}{2.601981in}}{\pgfqpoint{3.057502in}{2.590931in}}%
\pgfpathcurveto{\pgfqpoint{3.057502in}{2.579881in}}{\pgfqpoint{3.061892in}{2.569282in}}{\pgfqpoint{3.069706in}{2.561468in}}%
\pgfpathcurveto{\pgfqpoint{3.077520in}{2.553655in}}{\pgfqpoint{3.088119in}{2.549265in}}{\pgfqpoint{3.099169in}{2.549265in}}%
\pgfpathclose%
\pgfusepath{stroke,fill}%
\end{pgfscope}%
\begin{pgfscope}%
\pgfpathrectangle{\pgfqpoint{0.600000in}{0.600000in}}{\pgfqpoint{3.900000in}{3.900000in}}%
\pgfusepath{clip}%
\pgfsetbuttcap%
\pgfsetroundjoin%
\definecolor{currentfill}{rgb}{0.121569,0.466667,0.705882}%
\pgfsetfillcolor{currentfill}%
\pgfsetlinewidth{1.003750pt}%
\definecolor{currentstroke}{rgb}{0.121569,0.466667,0.705882}%
\pgfsetstrokecolor{currentstroke}%
\pgfsetdash{}{0pt}%
\pgfpathmoveto{\pgfqpoint{2.186784in}{2.391430in}}%
\pgfpathcurveto{\pgfqpoint{2.197834in}{2.391430in}}{\pgfqpoint{2.208433in}{2.395821in}}{\pgfqpoint{2.216247in}{2.403634in}}%
\pgfpathcurveto{\pgfqpoint{2.224060in}{2.411448in}}{\pgfqpoint{2.228451in}{2.422047in}}{\pgfqpoint{2.228451in}{2.433097in}}%
\pgfpathcurveto{\pgfqpoint{2.228451in}{2.444147in}}{\pgfqpoint{2.224060in}{2.454746in}}{\pgfqpoint{2.216247in}{2.462560in}}%
\pgfpathcurveto{\pgfqpoint{2.208433in}{2.470373in}}{\pgfqpoint{2.197834in}{2.474764in}}{\pgfqpoint{2.186784in}{2.474764in}}%
\pgfpathcurveto{\pgfqpoint{2.175734in}{2.474764in}}{\pgfqpoint{2.165135in}{2.470373in}}{\pgfqpoint{2.157321in}{2.462560in}}%
\pgfpathcurveto{\pgfqpoint{2.149508in}{2.454746in}}{\pgfqpoint{2.145117in}{2.444147in}}{\pgfqpoint{2.145117in}{2.433097in}}%
\pgfpathcurveto{\pgfqpoint{2.145117in}{2.422047in}}{\pgfqpoint{2.149508in}{2.411448in}}{\pgfqpoint{2.157321in}{2.403634in}}%
\pgfpathcurveto{\pgfqpoint{2.165135in}{2.395821in}}{\pgfqpoint{2.175734in}{2.391430in}}{\pgfqpoint{2.186784in}{2.391430in}}%
\pgfpathclose%
\pgfusepath{stroke,fill}%
\end{pgfscope}%
\begin{pgfscope}%
\pgfpathrectangle{\pgfqpoint{0.600000in}{0.600000in}}{\pgfqpoint{3.900000in}{3.900000in}}%
\pgfusepath{clip}%
\pgfsetbuttcap%
\pgfsetroundjoin%
\definecolor{currentfill}{rgb}{0.121569,0.466667,0.705882}%
\pgfsetfillcolor{currentfill}%
\pgfsetlinewidth{1.003750pt}%
\definecolor{currentstroke}{rgb}{0.121569,0.466667,0.705882}%
\pgfsetstrokecolor{currentstroke}%
\pgfsetdash{}{0pt}%
\pgfpathmoveto{\pgfqpoint{2.991939in}{2.067210in}}%
\pgfpathcurveto{\pgfqpoint{3.002989in}{2.067210in}}{\pgfqpoint{3.013588in}{2.071600in}}{\pgfqpoint{3.021402in}{2.079413in}}%
\pgfpathcurveto{\pgfqpoint{3.029216in}{2.087227in}}{\pgfqpoint{3.033606in}{2.097826in}}{\pgfqpoint{3.033606in}{2.108876in}}%
\pgfpathcurveto{\pgfqpoint{3.033606in}{2.119926in}}{\pgfqpoint{3.029216in}{2.130525in}}{\pgfqpoint{3.021402in}{2.138339in}}%
\pgfpathcurveto{\pgfqpoint{3.013588in}{2.146153in}}{\pgfqpoint{3.002989in}{2.150543in}}{\pgfqpoint{2.991939in}{2.150543in}}%
\pgfpathcurveto{\pgfqpoint{2.980889in}{2.150543in}}{\pgfqpoint{2.970290in}{2.146153in}}{\pgfqpoint{2.962476in}{2.138339in}}%
\pgfpathcurveto{\pgfqpoint{2.954663in}{2.130525in}}{\pgfqpoint{2.950273in}{2.119926in}}{\pgfqpoint{2.950273in}{2.108876in}}%
\pgfpathcurveto{\pgfqpoint{2.950273in}{2.097826in}}{\pgfqpoint{2.954663in}{2.087227in}}{\pgfqpoint{2.962476in}{2.079413in}}%
\pgfpathcurveto{\pgfqpoint{2.970290in}{2.071600in}}{\pgfqpoint{2.980889in}{2.067210in}}{\pgfqpoint{2.991939in}{2.067210in}}%
\pgfpathclose%
\pgfusepath{stroke,fill}%
\end{pgfscope}%
\begin{pgfscope}%
\pgfpathrectangle{\pgfqpoint{0.600000in}{0.600000in}}{\pgfqpoint{3.900000in}{3.900000in}}%
\pgfusepath{clip}%
\pgfsetbuttcap%
\pgfsetroundjoin%
\definecolor{currentfill}{rgb}{0.121569,0.466667,0.705882}%
\pgfsetfillcolor{currentfill}%
\pgfsetlinewidth{1.003750pt}%
\definecolor{currentstroke}{rgb}{0.121569,0.466667,0.705882}%
\pgfsetstrokecolor{currentstroke}%
\pgfsetdash{}{0pt}%
\pgfpathmoveto{\pgfqpoint{2.381538in}{2.361547in}}%
\pgfpathcurveto{\pgfqpoint{2.392588in}{2.361547in}}{\pgfqpoint{2.403187in}{2.365938in}}{\pgfqpoint{2.411001in}{2.373751in}}%
\pgfpathcurveto{\pgfqpoint{2.418814in}{2.381565in}}{\pgfqpoint{2.423205in}{2.392164in}}{\pgfqpoint{2.423205in}{2.403214in}}%
\pgfpathcurveto{\pgfqpoint{2.423205in}{2.414264in}}{\pgfqpoint{2.418814in}{2.424863in}}{\pgfqpoint{2.411001in}{2.432677in}}%
\pgfpathcurveto{\pgfqpoint{2.403187in}{2.440490in}}{\pgfqpoint{2.392588in}{2.444881in}}{\pgfqpoint{2.381538in}{2.444881in}}%
\pgfpathcurveto{\pgfqpoint{2.370488in}{2.444881in}}{\pgfqpoint{2.359889in}{2.440490in}}{\pgfqpoint{2.352075in}{2.432677in}}%
\pgfpathcurveto{\pgfqpoint{2.344261in}{2.424863in}}{\pgfqpoint{2.339871in}{2.414264in}}{\pgfqpoint{2.339871in}{2.403214in}}%
\pgfpathcurveto{\pgfqpoint{2.339871in}{2.392164in}}{\pgfqpoint{2.344261in}{2.381565in}}{\pgfqpoint{2.352075in}{2.373751in}}%
\pgfpathcurveto{\pgfqpoint{2.359889in}{2.365938in}}{\pgfqpoint{2.370488in}{2.361547in}}{\pgfqpoint{2.381538in}{2.361547in}}%
\pgfpathclose%
\pgfusepath{stroke,fill}%
\end{pgfscope}%
\begin{pgfscope}%
\pgfpathrectangle{\pgfqpoint{0.600000in}{0.600000in}}{\pgfqpoint{3.900000in}{3.900000in}}%
\pgfusepath{clip}%
\pgfsetbuttcap%
\pgfsetroundjoin%
\definecolor{currentfill}{rgb}{0.121569,0.466667,0.705882}%
\pgfsetfillcolor{currentfill}%
\pgfsetlinewidth{1.003750pt}%
\definecolor{currentstroke}{rgb}{0.121569,0.466667,0.705882}%
\pgfsetstrokecolor{currentstroke}%
\pgfsetdash{}{0pt}%
\pgfpathmoveto{\pgfqpoint{2.729714in}{2.475078in}}%
\pgfpathcurveto{\pgfqpoint{2.740764in}{2.475078in}}{\pgfqpoint{2.751363in}{2.479468in}}{\pgfqpoint{2.759177in}{2.487282in}}%
\pgfpathcurveto{\pgfqpoint{2.766991in}{2.495095in}}{\pgfqpoint{2.771381in}{2.505695in}}{\pgfqpoint{2.771381in}{2.516745in}}%
\pgfpathcurveto{\pgfqpoint{2.771381in}{2.527795in}}{\pgfqpoint{2.766991in}{2.538394in}}{\pgfqpoint{2.759177in}{2.546207in}}%
\pgfpathcurveto{\pgfqpoint{2.751363in}{2.554021in}}{\pgfqpoint{2.740764in}{2.558411in}}{\pgfqpoint{2.729714in}{2.558411in}}%
\pgfpathcurveto{\pgfqpoint{2.718664in}{2.558411in}}{\pgfqpoint{2.708065in}{2.554021in}}{\pgfqpoint{2.700251in}{2.546207in}}%
\pgfpathcurveto{\pgfqpoint{2.692438in}{2.538394in}}{\pgfqpoint{2.688047in}{2.527795in}}{\pgfqpoint{2.688047in}{2.516745in}}%
\pgfpathcurveto{\pgfqpoint{2.688047in}{2.505695in}}{\pgfqpoint{2.692438in}{2.495095in}}{\pgfqpoint{2.700251in}{2.487282in}}%
\pgfpathcurveto{\pgfqpoint{2.708065in}{2.479468in}}{\pgfqpoint{2.718664in}{2.475078in}}{\pgfqpoint{2.729714in}{2.475078in}}%
\pgfpathclose%
\pgfusepath{stroke,fill}%
\end{pgfscope}%
\begin{pgfscope}%
\pgfpathrectangle{\pgfqpoint{0.600000in}{0.600000in}}{\pgfqpoint{3.900000in}{3.900000in}}%
\pgfusepath{clip}%
\pgfsetbuttcap%
\pgfsetroundjoin%
\definecolor{currentfill}{rgb}{0.121569,0.466667,0.705882}%
\pgfsetfillcolor{currentfill}%
\pgfsetlinewidth{1.003750pt}%
\definecolor{currentstroke}{rgb}{0.121569,0.466667,0.705882}%
\pgfsetstrokecolor{currentstroke}%
\pgfsetdash{}{0pt}%
\pgfpathmoveto{\pgfqpoint{2.143508in}{2.484008in}}%
\pgfpathcurveto{\pgfqpoint{2.154558in}{2.484008in}}{\pgfqpoint{2.165157in}{2.488399in}}{\pgfqpoint{2.172971in}{2.496212in}}%
\pgfpathcurveto{\pgfqpoint{2.180784in}{2.504026in}}{\pgfqpoint{2.185175in}{2.514625in}}{\pgfqpoint{2.185175in}{2.525675in}}%
\pgfpathcurveto{\pgfqpoint{2.185175in}{2.536725in}}{\pgfqpoint{2.180784in}{2.547324in}}{\pgfqpoint{2.172971in}{2.555138in}}%
\pgfpathcurveto{\pgfqpoint{2.165157in}{2.562951in}}{\pgfqpoint{2.154558in}{2.567342in}}{\pgfqpoint{2.143508in}{2.567342in}}%
\pgfpathcurveto{\pgfqpoint{2.132458in}{2.567342in}}{\pgfqpoint{2.121859in}{2.562951in}}{\pgfqpoint{2.114045in}{2.555138in}}%
\pgfpathcurveto{\pgfqpoint{2.106232in}{2.547324in}}{\pgfqpoint{2.101841in}{2.536725in}}{\pgfqpoint{2.101841in}{2.525675in}}%
\pgfpathcurveto{\pgfqpoint{2.101841in}{2.514625in}}{\pgfqpoint{2.106232in}{2.504026in}}{\pgfqpoint{2.114045in}{2.496212in}}%
\pgfpathcurveto{\pgfqpoint{2.121859in}{2.488399in}}{\pgfqpoint{2.132458in}{2.484008in}}{\pgfqpoint{2.143508in}{2.484008in}}%
\pgfpathclose%
\pgfusepath{stroke,fill}%
\end{pgfscope}%
\begin{pgfscope}%
\pgfpathrectangle{\pgfqpoint{0.600000in}{0.600000in}}{\pgfqpoint{3.900000in}{3.900000in}}%
\pgfusepath{clip}%
\pgfsetbuttcap%
\pgfsetroundjoin%
\definecolor{currentfill}{rgb}{0.121569,0.466667,0.705882}%
\pgfsetfillcolor{currentfill}%
\pgfsetlinewidth{1.003750pt}%
\definecolor{currentstroke}{rgb}{0.121569,0.466667,0.705882}%
\pgfsetstrokecolor{currentstroke}%
\pgfsetdash{}{0pt}%
\pgfpathmoveto{\pgfqpoint{3.310460in}{2.041024in}}%
\pgfpathcurveto{\pgfqpoint{3.321510in}{2.041024in}}{\pgfqpoint{3.332109in}{2.045414in}}{\pgfqpoint{3.339923in}{2.053228in}}%
\pgfpathcurveto{\pgfqpoint{3.347737in}{2.061041in}}{\pgfqpoint{3.352127in}{2.071640in}}{\pgfqpoint{3.352127in}{2.082690in}}%
\pgfpathcurveto{\pgfqpoint{3.352127in}{2.093741in}}{\pgfqpoint{3.347737in}{2.104340in}}{\pgfqpoint{3.339923in}{2.112153in}}%
\pgfpathcurveto{\pgfqpoint{3.332109in}{2.119967in}}{\pgfqpoint{3.321510in}{2.124357in}}{\pgfqpoint{3.310460in}{2.124357in}}%
\pgfpathcurveto{\pgfqpoint{3.299410in}{2.124357in}}{\pgfqpoint{3.288811in}{2.119967in}}{\pgfqpoint{3.280998in}{2.112153in}}%
\pgfpathcurveto{\pgfqpoint{3.273184in}{2.104340in}}{\pgfqpoint{3.268794in}{2.093741in}}{\pgfqpoint{3.268794in}{2.082690in}}%
\pgfpathcurveto{\pgfqpoint{3.268794in}{2.071640in}}{\pgfqpoint{3.273184in}{2.061041in}}{\pgfqpoint{3.280998in}{2.053228in}}%
\pgfpathcurveto{\pgfqpoint{3.288811in}{2.045414in}}{\pgfqpoint{3.299410in}{2.041024in}}{\pgfqpoint{3.310460in}{2.041024in}}%
\pgfpathclose%
\pgfusepath{stroke,fill}%
\end{pgfscope}%
\begin{pgfscope}%
\pgfpathrectangle{\pgfqpoint{0.600000in}{0.600000in}}{\pgfqpoint{3.900000in}{3.900000in}}%
\pgfusepath{clip}%
\pgfsetbuttcap%
\pgfsetroundjoin%
\definecolor{currentfill}{rgb}{0.121569,0.466667,0.705882}%
\pgfsetfillcolor{currentfill}%
\pgfsetlinewidth{1.003750pt}%
\definecolor{currentstroke}{rgb}{0.121569,0.466667,0.705882}%
\pgfsetstrokecolor{currentstroke}%
\pgfsetdash{}{0pt}%
\pgfpathmoveto{\pgfqpoint{1.628420in}{2.669441in}}%
\pgfpathcurveto{\pgfqpoint{1.639470in}{2.669441in}}{\pgfqpoint{1.650069in}{2.673831in}}{\pgfqpoint{1.657883in}{2.681644in}}%
\pgfpathcurveto{\pgfqpoint{1.665696in}{2.689458in}}{\pgfqpoint{1.670086in}{2.700057in}}{\pgfqpoint{1.670086in}{2.711107in}}%
\pgfpathcurveto{\pgfqpoint{1.670086in}{2.722157in}}{\pgfqpoint{1.665696in}{2.732756in}}{\pgfqpoint{1.657883in}{2.740570in}}%
\pgfpathcurveto{\pgfqpoint{1.650069in}{2.748384in}}{\pgfqpoint{1.639470in}{2.752774in}}{\pgfqpoint{1.628420in}{2.752774in}}%
\pgfpathcurveto{\pgfqpoint{1.617370in}{2.752774in}}{\pgfqpoint{1.606771in}{2.748384in}}{\pgfqpoint{1.598957in}{2.740570in}}%
\pgfpathcurveto{\pgfqpoint{1.591143in}{2.732756in}}{\pgfqpoint{1.586753in}{2.722157in}}{\pgfqpoint{1.586753in}{2.711107in}}%
\pgfpathcurveto{\pgfqpoint{1.586753in}{2.700057in}}{\pgfqpoint{1.591143in}{2.689458in}}{\pgfqpoint{1.598957in}{2.681644in}}%
\pgfpathcurveto{\pgfqpoint{1.606771in}{2.673831in}}{\pgfqpoint{1.617370in}{2.669441in}}{\pgfqpoint{1.628420in}{2.669441in}}%
\pgfpathclose%
\pgfusepath{stroke,fill}%
\end{pgfscope}%
\begin{pgfscope}%
\pgfpathrectangle{\pgfqpoint{0.600000in}{0.600000in}}{\pgfqpoint{3.900000in}{3.900000in}}%
\pgfusepath{clip}%
\pgfsetbuttcap%
\pgfsetroundjoin%
\definecolor{currentfill}{rgb}{0.121569,0.466667,0.705882}%
\pgfsetfillcolor{currentfill}%
\pgfsetlinewidth{1.003750pt}%
\definecolor{currentstroke}{rgb}{0.121569,0.466667,0.705882}%
\pgfsetstrokecolor{currentstroke}%
\pgfsetdash{}{0pt}%
\pgfpathmoveto{\pgfqpoint{2.639425in}{1.866864in}}%
\pgfpathcurveto{\pgfqpoint{2.650475in}{1.866864in}}{\pgfqpoint{2.661074in}{1.871255in}}{\pgfqpoint{2.668888in}{1.879068in}}%
\pgfpathcurveto{\pgfqpoint{2.676702in}{1.886882in}}{\pgfqpoint{2.681092in}{1.897481in}}{\pgfqpoint{2.681092in}{1.908531in}}%
\pgfpathcurveto{\pgfqpoint{2.681092in}{1.919581in}}{\pgfqpoint{2.676702in}{1.930180in}}{\pgfqpoint{2.668888in}{1.937994in}}%
\pgfpathcurveto{\pgfqpoint{2.661074in}{1.945807in}}{\pgfqpoint{2.650475in}{1.950198in}}{\pgfqpoint{2.639425in}{1.950198in}}%
\pgfpathcurveto{\pgfqpoint{2.628375in}{1.950198in}}{\pgfqpoint{2.617776in}{1.945807in}}{\pgfqpoint{2.609962in}{1.937994in}}%
\pgfpathcurveto{\pgfqpoint{2.602149in}{1.930180in}}{\pgfqpoint{2.597759in}{1.919581in}}{\pgfqpoint{2.597759in}{1.908531in}}%
\pgfpathcurveto{\pgfqpoint{2.597759in}{1.897481in}}{\pgfqpoint{2.602149in}{1.886882in}}{\pgfqpoint{2.609962in}{1.879068in}}%
\pgfpathcurveto{\pgfqpoint{2.617776in}{1.871255in}}{\pgfqpoint{2.628375in}{1.866864in}}{\pgfqpoint{2.639425in}{1.866864in}}%
\pgfpathclose%
\pgfusepath{stroke,fill}%
\end{pgfscope}%
\begin{pgfscope}%
\pgfpathrectangle{\pgfqpoint{0.600000in}{0.600000in}}{\pgfqpoint{3.900000in}{3.900000in}}%
\pgfusepath{clip}%
\pgfsetbuttcap%
\pgfsetroundjoin%
\definecolor{currentfill}{rgb}{0.121569,0.466667,0.705882}%
\pgfsetfillcolor{currentfill}%
\pgfsetlinewidth{1.003750pt}%
\definecolor{currentstroke}{rgb}{0.121569,0.466667,0.705882}%
\pgfsetstrokecolor{currentstroke}%
\pgfsetdash{}{0pt}%
\pgfpathmoveto{\pgfqpoint{1.865523in}{3.485079in}}%
\pgfpathcurveto{\pgfqpoint{1.876573in}{3.485079in}}{\pgfqpoint{1.887172in}{3.489469in}}{\pgfqpoint{1.894986in}{3.497283in}}%
\pgfpathcurveto{\pgfqpoint{1.902799in}{3.505096in}}{\pgfqpoint{1.907190in}{3.515695in}}{\pgfqpoint{1.907190in}{3.526745in}}%
\pgfpathcurveto{\pgfqpoint{1.907190in}{3.537795in}}{\pgfqpoint{1.902799in}{3.548394in}}{\pgfqpoint{1.894986in}{3.556208in}}%
\pgfpathcurveto{\pgfqpoint{1.887172in}{3.564022in}}{\pgfqpoint{1.876573in}{3.568412in}}{\pgfqpoint{1.865523in}{3.568412in}}%
\pgfpathcurveto{\pgfqpoint{1.854473in}{3.568412in}}{\pgfqpoint{1.843874in}{3.564022in}}{\pgfqpoint{1.836060in}{3.556208in}}%
\pgfpathcurveto{\pgfqpoint{1.828247in}{3.548394in}}{\pgfqpoint{1.823856in}{3.537795in}}{\pgfqpoint{1.823856in}{3.526745in}}%
\pgfpathcurveto{\pgfqpoint{1.823856in}{3.515695in}}{\pgfqpoint{1.828247in}{3.505096in}}{\pgfqpoint{1.836060in}{3.497283in}}%
\pgfpathcurveto{\pgfqpoint{1.843874in}{3.489469in}}{\pgfqpoint{1.854473in}{3.485079in}}{\pgfqpoint{1.865523in}{3.485079in}}%
\pgfpathclose%
\pgfusepath{stroke,fill}%
\end{pgfscope}%
\begin{pgfscope}%
\pgfpathrectangle{\pgfqpoint{0.600000in}{0.600000in}}{\pgfqpoint{3.900000in}{3.900000in}}%
\pgfusepath{clip}%
\pgfsetbuttcap%
\pgfsetroundjoin%
\definecolor{currentfill}{rgb}{0.121569,0.466667,0.705882}%
\pgfsetfillcolor{currentfill}%
\pgfsetlinewidth{1.003750pt}%
\definecolor{currentstroke}{rgb}{0.121569,0.466667,0.705882}%
\pgfsetstrokecolor{currentstroke}%
\pgfsetdash{}{0pt}%
\pgfpathmoveto{\pgfqpoint{2.879975in}{2.242860in}}%
\pgfpathcurveto{\pgfqpoint{2.891025in}{2.242860in}}{\pgfqpoint{2.901624in}{2.247251in}}{\pgfqpoint{2.909438in}{2.255064in}}%
\pgfpathcurveto{\pgfqpoint{2.917252in}{2.262878in}}{\pgfqpoint{2.921642in}{2.273477in}}{\pgfqpoint{2.921642in}{2.284527in}}%
\pgfpathcurveto{\pgfqpoint{2.921642in}{2.295577in}}{\pgfqpoint{2.917252in}{2.306176in}}{\pgfqpoint{2.909438in}{2.313990in}}%
\pgfpathcurveto{\pgfqpoint{2.901624in}{2.321804in}}{\pgfqpoint{2.891025in}{2.326194in}}{\pgfqpoint{2.879975in}{2.326194in}}%
\pgfpathcurveto{\pgfqpoint{2.868925in}{2.326194in}}{\pgfqpoint{2.858326in}{2.321804in}}{\pgfqpoint{2.850512in}{2.313990in}}%
\pgfpathcurveto{\pgfqpoint{2.842699in}{2.306176in}}{\pgfqpoint{2.838308in}{2.295577in}}{\pgfqpoint{2.838308in}{2.284527in}}%
\pgfpathcurveto{\pgfqpoint{2.838308in}{2.273477in}}{\pgfqpoint{2.842699in}{2.262878in}}{\pgfqpoint{2.850512in}{2.255064in}}%
\pgfpathcurveto{\pgfqpoint{2.858326in}{2.247251in}}{\pgfqpoint{2.868925in}{2.242860in}}{\pgfqpoint{2.879975in}{2.242860in}}%
\pgfpathclose%
\pgfusepath{stroke,fill}%
\end{pgfscope}%
\begin{pgfscope}%
\pgfpathrectangle{\pgfqpoint{0.600000in}{0.600000in}}{\pgfqpoint{3.900000in}{3.900000in}}%
\pgfusepath{clip}%
\pgfsetbuttcap%
\pgfsetroundjoin%
\definecolor{currentfill}{rgb}{0.121569,0.466667,0.705882}%
\pgfsetfillcolor{currentfill}%
\pgfsetlinewidth{1.003750pt}%
\definecolor{currentstroke}{rgb}{0.121569,0.466667,0.705882}%
\pgfsetstrokecolor{currentstroke}%
\pgfsetdash{}{0pt}%
\pgfpathmoveto{\pgfqpoint{2.630886in}{1.823358in}}%
\pgfpathcurveto{\pgfqpoint{2.641936in}{1.823358in}}{\pgfqpoint{2.652535in}{1.827748in}}{\pgfqpoint{2.660349in}{1.835562in}}%
\pgfpathcurveto{\pgfqpoint{2.668162in}{1.843375in}}{\pgfqpoint{2.672553in}{1.853974in}}{\pgfqpoint{2.672553in}{1.865024in}}%
\pgfpathcurveto{\pgfqpoint{2.672553in}{1.876074in}}{\pgfqpoint{2.668162in}{1.886673in}}{\pgfqpoint{2.660349in}{1.894487in}}%
\pgfpathcurveto{\pgfqpoint{2.652535in}{1.902301in}}{\pgfqpoint{2.641936in}{1.906691in}}{\pgfqpoint{2.630886in}{1.906691in}}%
\pgfpathcurveto{\pgfqpoint{2.619836in}{1.906691in}}{\pgfqpoint{2.609237in}{1.902301in}}{\pgfqpoint{2.601423in}{1.894487in}}%
\pgfpathcurveto{\pgfqpoint{2.593610in}{1.886673in}}{\pgfqpoint{2.589219in}{1.876074in}}{\pgfqpoint{2.589219in}{1.865024in}}%
\pgfpathcurveto{\pgfqpoint{2.589219in}{1.853974in}}{\pgfqpoint{2.593610in}{1.843375in}}{\pgfqpoint{2.601423in}{1.835562in}}%
\pgfpathcurveto{\pgfqpoint{2.609237in}{1.827748in}}{\pgfqpoint{2.619836in}{1.823358in}}{\pgfqpoint{2.630886in}{1.823358in}}%
\pgfpathclose%
\pgfusepath{stroke,fill}%
\end{pgfscope}%
\begin{pgfscope}%
\pgfpathrectangle{\pgfqpoint{0.600000in}{0.600000in}}{\pgfqpoint{3.900000in}{3.900000in}}%
\pgfusepath{clip}%
\pgfsetbuttcap%
\pgfsetroundjoin%
\definecolor{currentfill}{rgb}{0.121569,0.466667,0.705882}%
\pgfsetfillcolor{currentfill}%
\pgfsetlinewidth{1.003750pt}%
\definecolor{currentstroke}{rgb}{0.121569,0.466667,0.705882}%
\pgfsetstrokecolor{currentstroke}%
\pgfsetdash{}{0pt}%
\pgfpathmoveto{\pgfqpoint{2.513164in}{2.215097in}}%
\pgfpathcurveto{\pgfqpoint{2.524214in}{2.215097in}}{\pgfqpoint{2.534813in}{2.219487in}}{\pgfqpoint{2.542626in}{2.227301in}}%
\pgfpathcurveto{\pgfqpoint{2.550440in}{2.235114in}}{\pgfqpoint{2.554830in}{2.245713in}}{\pgfqpoint{2.554830in}{2.256764in}}%
\pgfpathcurveto{\pgfqpoint{2.554830in}{2.267814in}}{\pgfqpoint{2.550440in}{2.278413in}}{\pgfqpoint{2.542626in}{2.286226in}}%
\pgfpathcurveto{\pgfqpoint{2.534813in}{2.294040in}}{\pgfqpoint{2.524214in}{2.298430in}}{\pgfqpoint{2.513164in}{2.298430in}}%
\pgfpathcurveto{\pgfqpoint{2.502113in}{2.298430in}}{\pgfqpoint{2.491514in}{2.294040in}}{\pgfqpoint{2.483701in}{2.286226in}}%
\pgfpathcurveto{\pgfqpoint{2.475887in}{2.278413in}}{\pgfqpoint{2.471497in}{2.267814in}}{\pgfqpoint{2.471497in}{2.256764in}}%
\pgfpathcurveto{\pgfqpoint{2.471497in}{2.245713in}}{\pgfqpoint{2.475887in}{2.235114in}}{\pgfqpoint{2.483701in}{2.227301in}}%
\pgfpathcurveto{\pgfqpoint{2.491514in}{2.219487in}}{\pgfqpoint{2.502113in}{2.215097in}}{\pgfqpoint{2.513164in}{2.215097in}}%
\pgfpathclose%
\pgfusepath{stroke,fill}%
\end{pgfscope}%
\begin{pgfscope}%
\pgfpathrectangle{\pgfqpoint{0.600000in}{0.600000in}}{\pgfqpoint{3.900000in}{3.900000in}}%
\pgfusepath{clip}%
\pgfsetbuttcap%
\pgfsetroundjoin%
\definecolor{currentfill}{rgb}{0.121569,0.466667,0.705882}%
\pgfsetfillcolor{currentfill}%
\pgfsetlinewidth{1.003750pt}%
\definecolor{currentstroke}{rgb}{0.121569,0.466667,0.705882}%
\pgfsetstrokecolor{currentstroke}%
\pgfsetdash{}{0pt}%
\pgfpathmoveto{\pgfqpoint{2.132753in}{1.929431in}}%
\pgfpathcurveto{\pgfqpoint{2.143803in}{1.929431in}}{\pgfqpoint{2.154402in}{1.933821in}}{\pgfqpoint{2.162216in}{1.941635in}}%
\pgfpathcurveto{\pgfqpoint{2.170030in}{1.949448in}}{\pgfqpoint{2.174420in}{1.960047in}}{\pgfqpoint{2.174420in}{1.971097in}}%
\pgfpathcurveto{\pgfqpoint{2.174420in}{1.982148in}}{\pgfqpoint{2.170030in}{1.992747in}}{\pgfqpoint{2.162216in}{2.000560in}}%
\pgfpathcurveto{\pgfqpoint{2.154402in}{2.008374in}}{\pgfqpoint{2.143803in}{2.012764in}}{\pgfqpoint{2.132753in}{2.012764in}}%
\pgfpathcurveto{\pgfqpoint{2.121703in}{2.012764in}}{\pgfqpoint{2.111104in}{2.008374in}}{\pgfqpoint{2.103290in}{2.000560in}}%
\pgfpathcurveto{\pgfqpoint{2.095477in}{1.992747in}}{\pgfqpoint{2.091087in}{1.982148in}}{\pgfqpoint{2.091087in}{1.971097in}}%
\pgfpathcurveto{\pgfqpoint{2.091087in}{1.960047in}}{\pgfqpoint{2.095477in}{1.949448in}}{\pgfqpoint{2.103290in}{1.941635in}}%
\pgfpathcurveto{\pgfqpoint{2.111104in}{1.933821in}}{\pgfqpoint{2.121703in}{1.929431in}}{\pgfqpoint{2.132753in}{1.929431in}}%
\pgfpathclose%
\pgfusepath{stroke,fill}%
\end{pgfscope}%
\begin{pgfscope}%
\pgfpathrectangle{\pgfqpoint{0.600000in}{0.600000in}}{\pgfqpoint{3.900000in}{3.900000in}}%
\pgfusepath{clip}%
\pgfsetbuttcap%
\pgfsetroundjoin%
\definecolor{currentfill}{rgb}{0.121569,0.466667,0.705882}%
\pgfsetfillcolor{currentfill}%
\pgfsetlinewidth{1.003750pt}%
\definecolor{currentstroke}{rgb}{0.121569,0.466667,0.705882}%
\pgfsetstrokecolor{currentstroke}%
\pgfsetdash{}{0pt}%
\pgfpathmoveto{\pgfqpoint{1.643890in}{3.124197in}}%
\pgfpathcurveto{\pgfqpoint{1.654940in}{3.124197in}}{\pgfqpoint{1.665539in}{3.128587in}}{\pgfqpoint{1.673353in}{3.136401in}}%
\pgfpathcurveto{\pgfqpoint{1.681166in}{3.144214in}}{\pgfqpoint{1.685557in}{3.154813in}}{\pgfqpoint{1.685557in}{3.165864in}}%
\pgfpathcurveto{\pgfqpoint{1.685557in}{3.176914in}}{\pgfqpoint{1.681166in}{3.187513in}}{\pgfqpoint{1.673353in}{3.195326in}}%
\pgfpathcurveto{\pgfqpoint{1.665539in}{3.203140in}}{\pgfqpoint{1.654940in}{3.207530in}}{\pgfqpoint{1.643890in}{3.207530in}}%
\pgfpathcurveto{\pgfqpoint{1.632840in}{3.207530in}}{\pgfqpoint{1.622241in}{3.203140in}}{\pgfqpoint{1.614427in}{3.195326in}}%
\pgfpathcurveto{\pgfqpoint{1.606614in}{3.187513in}}{\pgfqpoint{1.602223in}{3.176914in}}{\pgfqpoint{1.602223in}{3.165864in}}%
\pgfpathcurveto{\pgfqpoint{1.602223in}{3.154813in}}{\pgfqpoint{1.606614in}{3.144214in}}{\pgfqpoint{1.614427in}{3.136401in}}%
\pgfpathcurveto{\pgfqpoint{1.622241in}{3.128587in}}{\pgfqpoint{1.632840in}{3.124197in}}{\pgfqpoint{1.643890in}{3.124197in}}%
\pgfpathclose%
\pgfusepath{stroke,fill}%
\end{pgfscope}%
\begin{pgfscope}%
\pgfpathrectangle{\pgfqpoint{0.600000in}{0.600000in}}{\pgfqpoint{3.900000in}{3.900000in}}%
\pgfusepath{clip}%
\pgfsetbuttcap%
\pgfsetroundjoin%
\definecolor{currentfill}{rgb}{0.121569,0.466667,0.705882}%
\pgfsetfillcolor{currentfill}%
\pgfsetlinewidth{1.003750pt}%
\definecolor{currentstroke}{rgb}{0.121569,0.466667,0.705882}%
\pgfsetstrokecolor{currentstroke}%
\pgfsetdash{}{0pt}%
\pgfpathmoveto{\pgfqpoint{2.232851in}{2.066033in}}%
\pgfpathcurveto{\pgfqpoint{2.243901in}{2.066033in}}{\pgfqpoint{2.254500in}{2.070423in}}{\pgfqpoint{2.262314in}{2.078237in}}%
\pgfpathcurveto{\pgfqpoint{2.270127in}{2.086050in}}{\pgfqpoint{2.274517in}{2.096649in}}{\pgfqpoint{2.274517in}{2.107700in}}%
\pgfpathcurveto{\pgfqpoint{2.274517in}{2.118750in}}{\pgfqpoint{2.270127in}{2.129349in}}{\pgfqpoint{2.262314in}{2.137162in}}%
\pgfpathcurveto{\pgfqpoint{2.254500in}{2.144976in}}{\pgfqpoint{2.243901in}{2.149366in}}{\pgfqpoint{2.232851in}{2.149366in}}%
\pgfpathcurveto{\pgfqpoint{2.221801in}{2.149366in}}{\pgfqpoint{2.211202in}{2.144976in}}{\pgfqpoint{2.203388in}{2.137162in}}%
\pgfpathcurveto{\pgfqpoint{2.195574in}{2.129349in}}{\pgfqpoint{2.191184in}{2.118750in}}{\pgfqpoint{2.191184in}{2.107700in}}%
\pgfpathcurveto{\pgfqpoint{2.191184in}{2.096649in}}{\pgfqpoint{2.195574in}{2.086050in}}{\pgfqpoint{2.203388in}{2.078237in}}%
\pgfpathcurveto{\pgfqpoint{2.211202in}{2.070423in}}{\pgfqpoint{2.221801in}{2.066033in}}{\pgfqpoint{2.232851in}{2.066033in}}%
\pgfpathclose%
\pgfusepath{stroke,fill}%
\end{pgfscope}%
\begin{pgfscope}%
\pgfpathrectangle{\pgfqpoint{0.600000in}{0.600000in}}{\pgfqpoint{3.900000in}{3.900000in}}%
\pgfusepath{clip}%
\pgfsetbuttcap%
\pgfsetroundjoin%
\definecolor{currentfill}{rgb}{0.121569,0.466667,0.705882}%
\pgfsetfillcolor{currentfill}%
\pgfsetlinewidth{1.003750pt}%
\definecolor{currentstroke}{rgb}{0.121569,0.466667,0.705882}%
\pgfsetstrokecolor{currentstroke}%
\pgfsetdash{}{0pt}%
\pgfpathmoveto{\pgfqpoint{1.803183in}{2.109250in}}%
\pgfpathcurveto{\pgfqpoint{1.814233in}{2.109250in}}{\pgfqpoint{1.824832in}{2.113640in}}{\pgfqpoint{1.832645in}{2.121454in}}%
\pgfpathcurveto{\pgfqpoint{1.840459in}{2.129267in}}{\pgfqpoint{1.844849in}{2.139866in}}{\pgfqpoint{1.844849in}{2.150917in}}%
\pgfpathcurveto{\pgfqpoint{1.844849in}{2.161967in}}{\pgfqpoint{1.840459in}{2.172566in}}{\pgfqpoint{1.832645in}{2.180379in}}%
\pgfpathcurveto{\pgfqpoint{1.824832in}{2.188193in}}{\pgfqpoint{1.814233in}{2.192583in}}{\pgfqpoint{1.803183in}{2.192583in}}%
\pgfpathcurveto{\pgfqpoint{1.792133in}{2.192583in}}{\pgfqpoint{1.781534in}{2.188193in}}{\pgfqpoint{1.773720in}{2.180379in}}%
\pgfpathcurveto{\pgfqpoint{1.765906in}{2.172566in}}{\pgfqpoint{1.761516in}{2.161967in}}{\pgfqpoint{1.761516in}{2.150917in}}%
\pgfpathcurveto{\pgfqpoint{1.761516in}{2.139866in}}{\pgfqpoint{1.765906in}{2.129267in}}{\pgfqpoint{1.773720in}{2.121454in}}%
\pgfpathcurveto{\pgfqpoint{1.781534in}{2.113640in}}{\pgfqpoint{1.792133in}{2.109250in}}{\pgfqpoint{1.803183in}{2.109250in}}%
\pgfpathclose%
\pgfusepath{stroke,fill}%
\end{pgfscope}%
\begin{pgfscope}%
\pgfpathrectangle{\pgfqpoint{0.600000in}{0.600000in}}{\pgfqpoint{3.900000in}{3.900000in}}%
\pgfusepath{clip}%
\pgfsetbuttcap%
\pgfsetroundjoin%
\definecolor{currentfill}{rgb}{0.121569,0.466667,0.705882}%
\pgfsetfillcolor{currentfill}%
\pgfsetlinewidth{1.003750pt}%
\definecolor{currentstroke}{rgb}{0.121569,0.466667,0.705882}%
\pgfsetstrokecolor{currentstroke}%
\pgfsetdash{}{0pt}%
\pgfpathmoveto{\pgfqpoint{3.345765in}{2.605714in}}%
\pgfpathcurveto{\pgfqpoint{3.356815in}{2.605714in}}{\pgfqpoint{3.367414in}{2.610104in}}{\pgfqpoint{3.375227in}{2.617918in}}%
\pgfpathcurveto{\pgfqpoint{3.383041in}{2.625731in}}{\pgfqpoint{3.387431in}{2.636330in}}{\pgfqpoint{3.387431in}{2.647381in}}%
\pgfpathcurveto{\pgfqpoint{3.387431in}{2.658431in}}{\pgfqpoint{3.383041in}{2.669030in}}{\pgfqpoint{3.375227in}{2.676843in}}%
\pgfpathcurveto{\pgfqpoint{3.367414in}{2.684657in}}{\pgfqpoint{3.356815in}{2.689047in}}{\pgfqpoint{3.345765in}{2.689047in}}%
\pgfpathcurveto{\pgfqpoint{3.334715in}{2.689047in}}{\pgfqpoint{3.324115in}{2.684657in}}{\pgfqpoint{3.316302in}{2.676843in}}%
\pgfpathcurveto{\pgfqpoint{3.308488in}{2.669030in}}{\pgfqpoint{3.304098in}{2.658431in}}{\pgfqpoint{3.304098in}{2.647381in}}%
\pgfpathcurveto{\pgfqpoint{3.304098in}{2.636330in}}{\pgfqpoint{3.308488in}{2.625731in}}{\pgfqpoint{3.316302in}{2.617918in}}%
\pgfpathcurveto{\pgfqpoint{3.324115in}{2.610104in}}{\pgfqpoint{3.334715in}{2.605714in}}{\pgfqpoint{3.345765in}{2.605714in}}%
\pgfpathclose%
\pgfusepath{stroke,fill}%
\end{pgfscope}%
\begin{pgfscope}%
\pgfpathrectangle{\pgfqpoint{0.600000in}{0.600000in}}{\pgfqpoint{3.900000in}{3.900000in}}%
\pgfusepath{clip}%
\pgfsetbuttcap%
\pgfsetroundjoin%
\definecolor{currentfill}{rgb}{0.121569,0.466667,0.705882}%
\pgfsetfillcolor{currentfill}%
\pgfsetlinewidth{1.003750pt}%
\definecolor{currentstroke}{rgb}{0.121569,0.466667,0.705882}%
\pgfsetstrokecolor{currentstroke}%
\pgfsetdash{}{0pt}%
\pgfpathmoveto{\pgfqpoint{2.025780in}{1.414998in}}%
\pgfpathcurveto{\pgfqpoint{2.036831in}{1.414998in}}{\pgfqpoint{2.047430in}{1.419388in}}{\pgfqpoint{2.055243in}{1.427202in}}%
\pgfpathcurveto{\pgfqpoint{2.063057in}{1.435016in}}{\pgfqpoint{2.067447in}{1.445615in}}{\pgfqpoint{2.067447in}{1.456665in}}%
\pgfpathcurveto{\pgfqpoint{2.067447in}{1.467715in}}{\pgfqpoint{2.063057in}{1.478314in}}{\pgfqpoint{2.055243in}{1.486128in}}%
\pgfpathcurveto{\pgfqpoint{2.047430in}{1.493941in}}{\pgfqpoint{2.036831in}{1.498332in}}{\pgfqpoint{2.025780in}{1.498332in}}%
\pgfpathcurveto{\pgfqpoint{2.014730in}{1.498332in}}{\pgfqpoint{2.004131in}{1.493941in}}{\pgfqpoint{1.996318in}{1.486128in}}%
\pgfpathcurveto{\pgfqpoint{1.988504in}{1.478314in}}{\pgfqpoint{1.984114in}{1.467715in}}{\pgfqpoint{1.984114in}{1.456665in}}%
\pgfpathcurveto{\pgfqpoint{1.984114in}{1.445615in}}{\pgfqpoint{1.988504in}{1.435016in}}{\pgfqpoint{1.996318in}{1.427202in}}%
\pgfpathcurveto{\pgfqpoint{2.004131in}{1.419388in}}{\pgfqpoint{2.014730in}{1.414998in}}{\pgfqpoint{2.025780in}{1.414998in}}%
\pgfpathclose%
\pgfusepath{stroke,fill}%
\end{pgfscope}%
\begin{pgfscope}%
\pgfpathrectangle{\pgfqpoint{0.600000in}{0.600000in}}{\pgfqpoint{3.900000in}{3.900000in}}%
\pgfusepath{clip}%
\pgfsetbuttcap%
\pgfsetroundjoin%
\definecolor{currentfill}{rgb}{0.121569,0.466667,0.705882}%
\pgfsetfillcolor{currentfill}%
\pgfsetlinewidth{1.003750pt}%
\definecolor{currentstroke}{rgb}{0.121569,0.466667,0.705882}%
\pgfsetstrokecolor{currentstroke}%
\pgfsetdash{}{0pt}%
\pgfpathmoveto{\pgfqpoint{3.454688in}{3.735145in}}%
\pgfpathcurveto{\pgfqpoint{3.465738in}{3.735145in}}{\pgfqpoint{3.476337in}{3.739535in}}{\pgfqpoint{3.484151in}{3.747349in}}%
\pgfpathcurveto{\pgfqpoint{3.491964in}{3.755162in}}{\pgfqpoint{3.496355in}{3.765761in}}{\pgfqpoint{3.496355in}{3.776811in}}%
\pgfpathcurveto{\pgfqpoint{3.496355in}{3.787862in}}{\pgfqpoint{3.491964in}{3.798461in}}{\pgfqpoint{3.484151in}{3.806274in}}%
\pgfpathcurveto{\pgfqpoint{3.476337in}{3.814088in}}{\pgfqpoint{3.465738in}{3.818478in}}{\pgfqpoint{3.454688in}{3.818478in}}%
\pgfpathcurveto{\pgfqpoint{3.443638in}{3.818478in}}{\pgfqpoint{3.433039in}{3.814088in}}{\pgfqpoint{3.425225in}{3.806274in}}%
\pgfpathcurveto{\pgfqpoint{3.417411in}{3.798461in}}{\pgfqpoint{3.413021in}{3.787862in}}{\pgfqpoint{3.413021in}{3.776811in}}%
\pgfpathcurveto{\pgfqpoint{3.413021in}{3.765761in}}{\pgfqpoint{3.417411in}{3.755162in}}{\pgfqpoint{3.425225in}{3.747349in}}%
\pgfpathcurveto{\pgfqpoint{3.433039in}{3.739535in}}{\pgfqpoint{3.443638in}{3.735145in}}{\pgfqpoint{3.454688in}{3.735145in}}%
\pgfpathclose%
\pgfusepath{stroke,fill}%
\end{pgfscope}%
\begin{pgfscope}%
\pgfpathrectangle{\pgfqpoint{0.600000in}{0.600000in}}{\pgfqpoint{3.900000in}{3.900000in}}%
\pgfusepath{clip}%
\pgfsetbuttcap%
\pgfsetroundjoin%
\definecolor{currentfill}{rgb}{0.121569,0.466667,0.705882}%
\pgfsetfillcolor{currentfill}%
\pgfsetlinewidth{1.003750pt}%
\definecolor{currentstroke}{rgb}{0.121569,0.466667,0.705882}%
\pgfsetstrokecolor{currentstroke}%
\pgfsetdash{}{0pt}%
\pgfpathmoveto{\pgfqpoint{1.486413in}{2.852287in}}%
\pgfpathcurveto{\pgfqpoint{1.497463in}{2.852287in}}{\pgfqpoint{1.508063in}{2.856677in}}{\pgfqpoint{1.515876in}{2.864491in}}%
\pgfpathcurveto{\pgfqpoint{1.523690in}{2.872305in}}{\pgfqpoint{1.528080in}{2.882904in}}{\pgfqpoint{1.528080in}{2.893954in}}%
\pgfpathcurveto{\pgfqpoint{1.528080in}{2.905004in}}{\pgfqpoint{1.523690in}{2.915603in}}{\pgfqpoint{1.515876in}{2.923417in}}%
\pgfpathcurveto{\pgfqpoint{1.508063in}{2.931230in}}{\pgfqpoint{1.497463in}{2.935620in}}{\pgfqpoint{1.486413in}{2.935620in}}%
\pgfpathcurveto{\pgfqpoint{1.475363in}{2.935620in}}{\pgfqpoint{1.464764in}{2.931230in}}{\pgfqpoint{1.456951in}{2.923417in}}%
\pgfpathcurveto{\pgfqpoint{1.449137in}{2.915603in}}{\pgfqpoint{1.444747in}{2.905004in}}{\pgfqpoint{1.444747in}{2.893954in}}%
\pgfpathcurveto{\pgfqpoint{1.444747in}{2.882904in}}{\pgfqpoint{1.449137in}{2.872305in}}{\pgfqpoint{1.456951in}{2.864491in}}%
\pgfpathcurveto{\pgfqpoint{1.464764in}{2.856677in}}{\pgfqpoint{1.475363in}{2.852287in}}{\pgfqpoint{1.486413in}{2.852287in}}%
\pgfpathclose%
\pgfusepath{stroke,fill}%
\end{pgfscope}%
\begin{pgfscope}%
\pgfpathrectangle{\pgfqpoint{0.600000in}{0.600000in}}{\pgfqpoint{3.900000in}{3.900000in}}%
\pgfusepath{clip}%
\pgfsetbuttcap%
\pgfsetroundjoin%
\definecolor{currentfill}{rgb}{0.121569,0.466667,0.705882}%
\pgfsetfillcolor{currentfill}%
\pgfsetlinewidth{1.003750pt}%
\definecolor{currentstroke}{rgb}{0.121569,0.466667,0.705882}%
\pgfsetstrokecolor{currentstroke}%
\pgfsetdash{}{0pt}%
\pgfpathmoveto{\pgfqpoint{2.195490in}{2.067449in}}%
\pgfpathcurveto{\pgfqpoint{2.206540in}{2.067449in}}{\pgfqpoint{2.217139in}{2.071839in}}{\pgfqpoint{2.224953in}{2.079653in}}%
\pgfpathcurveto{\pgfqpoint{2.232767in}{2.087466in}}{\pgfqpoint{2.237157in}{2.098065in}}{\pgfqpoint{2.237157in}{2.109116in}}%
\pgfpathcurveto{\pgfqpoint{2.237157in}{2.120166in}}{\pgfqpoint{2.232767in}{2.130765in}}{\pgfqpoint{2.224953in}{2.138578in}}%
\pgfpathcurveto{\pgfqpoint{2.217139in}{2.146392in}}{\pgfqpoint{2.206540in}{2.150782in}}{\pgfqpoint{2.195490in}{2.150782in}}%
\pgfpathcurveto{\pgfqpoint{2.184440in}{2.150782in}}{\pgfqpoint{2.173841in}{2.146392in}}{\pgfqpoint{2.166027in}{2.138578in}}%
\pgfpathcurveto{\pgfqpoint{2.158214in}{2.130765in}}{\pgfqpoint{2.153824in}{2.120166in}}{\pgfqpoint{2.153824in}{2.109116in}}%
\pgfpathcurveto{\pgfqpoint{2.153824in}{2.098065in}}{\pgfqpoint{2.158214in}{2.087466in}}{\pgfqpoint{2.166027in}{2.079653in}}%
\pgfpathcurveto{\pgfqpoint{2.173841in}{2.071839in}}{\pgfqpoint{2.184440in}{2.067449in}}{\pgfqpoint{2.195490in}{2.067449in}}%
\pgfpathclose%
\pgfusepath{stroke,fill}%
\end{pgfscope}%
\begin{pgfscope}%
\pgfpathrectangle{\pgfqpoint{0.600000in}{0.600000in}}{\pgfqpoint{3.900000in}{3.900000in}}%
\pgfusepath{clip}%
\pgfsetbuttcap%
\pgfsetroundjoin%
\definecolor{currentfill}{rgb}{0.121569,0.466667,0.705882}%
\pgfsetfillcolor{currentfill}%
\pgfsetlinewidth{1.003750pt}%
\definecolor{currentstroke}{rgb}{0.121569,0.466667,0.705882}%
\pgfsetstrokecolor{currentstroke}%
\pgfsetdash{}{0pt}%
\pgfpathmoveto{\pgfqpoint{3.011492in}{2.086642in}}%
\pgfpathcurveto{\pgfqpoint{3.022542in}{2.086642in}}{\pgfqpoint{3.033141in}{2.091032in}}{\pgfqpoint{3.040954in}{2.098846in}}%
\pgfpathcurveto{\pgfqpoint{3.048768in}{2.106659in}}{\pgfqpoint{3.053158in}{2.117258in}}{\pgfqpoint{3.053158in}{2.128308in}}%
\pgfpathcurveto{\pgfqpoint{3.053158in}{2.139359in}}{\pgfqpoint{3.048768in}{2.149958in}}{\pgfqpoint{3.040954in}{2.157771in}}%
\pgfpathcurveto{\pgfqpoint{3.033141in}{2.165585in}}{\pgfqpoint{3.022542in}{2.169975in}}{\pgfqpoint{3.011492in}{2.169975in}}%
\pgfpathcurveto{\pgfqpoint{3.000441in}{2.169975in}}{\pgfqpoint{2.989842in}{2.165585in}}{\pgfqpoint{2.982029in}{2.157771in}}%
\pgfpathcurveto{\pgfqpoint{2.974215in}{2.149958in}}{\pgfqpoint{2.969825in}{2.139359in}}{\pgfqpoint{2.969825in}{2.128308in}}%
\pgfpathcurveto{\pgfqpoint{2.969825in}{2.117258in}}{\pgfqpoint{2.974215in}{2.106659in}}{\pgfqpoint{2.982029in}{2.098846in}}%
\pgfpathcurveto{\pgfqpoint{2.989842in}{2.091032in}}{\pgfqpoint{3.000441in}{2.086642in}}{\pgfqpoint{3.011492in}{2.086642in}}%
\pgfpathclose%
\pgfusepath{stroke,fill}%
\end{pgfscope}%
\begin{pgfscope}%
\pgfpathrectangle{\pgfqpoint{0.600000in}{0.600000in}}{\pgfqpoint{3.900000in}{3.900000in}}%
\pgfusepath{clip}%
\pgfsetbuttcap%
\pgfsetroundjoin%
\definecolor{currentfill}{rgb}{0.121569,0.466667,0.705882}%
\pgfsetfillcolor{currentfill}%
\pgfsetlinewidth{1.003750pt}%
\definecolor{currentstroke}{rgb}{0.121569,0.466667,0.705882}%
\pgfsetstrokecolor{currentstroke}%
\pgfsetdash{}{0pt}%
\pgfpathmoveto{\pgfqpoint{2.697758in}{2.309336in}}%
\pgfpathcurveto{\pgfqpoint{2.708808in}{2.309336in}}{\pgfqpoint{2.719407in}{2.313726in}}{\pgfqpoint{2.727220in}{2.321540in}}%
\pgfpathcurveto{\pgfqpoint{2.735034in}{2.329354in}}{\pgfqpoint{2.739424in}{2.339953in}}{\pgfqpoint{2.739424in}{2.351003in}}%
\pgfpathcurveto{\pgfqpoint{2.739424in}{2.362053in}}{\pgfqpoint{2.735034in}{2.372652in}}{\pgfqpoint{2.727220in}{2.380465in}}%
\pgfpathcurveto{\pgfqpoint{2.719407in}{2.388279in}}{\pgfqpoint{2.708808in}{2.392669in}}{\pgfqpoint{2.697758in}{2.392669in}}%
\pgfpathcurveto{\pgfqpoint{2.686707in}{2.392669in}}{\pgfqpoint{2.676108in}{2.388279in}}{\pgfqpoint{2.668295in}{2.380465in}}%
\pgfpathcurveto{\pgfqpoint{2.660481in}{2.372652in}}{\pgfqpoint{2.656091in}{2.362053in}}{\pgfqpoint{2.656091in}{2.351003in}}%
\pgfpathcurveto{\pgfqpoint{2.656091in}{2.339953in}}{\pgfqpoint{2.660481in}{2.329354in}}{\pgfqpoint{2.668295in}{2.321540in}}%
\pgfpathcurveto{\pgfqpoint{2.676108in}{2.313726in}}{\pgfqpoint{2.686707in}{2.309336in}}{\pgfqpoint{2.697758in}{2.309336in}}%
\pgfpathclose%
\pgfusepath{stroke,fill}%
\end{pgfscope}%
\begin{pgfscope}%
\pgfpathrectangle{\pgfqpoint{0.600000in}{0.600000in}}{\pgfqpoint{3.900000in}{3.900000in}}%
\pgfusepath{clip}%
\pgfsetbuttcap%
\pgfsetroundjoin%
\definecolor{currentfill}{rgb}{0.121569,0.466667,0.705882}%
\pgfsetfillcolor{currentfill}%
\pgfsetlinewidth{1.003750pt}%
\definecolor{currentstroke}{rgb}{0.121569,0.466667,0.705882}%
\pgfsetstrokecolor{currentstroke}%
\pgfsetdash{}{0pt}%
\pgfpathmoveto{\pgfqpoint{2.275676in}{1.913504in}}%
\pgfpathcurveto{\pgfqpoint{2.286726in}{1.913504in}}{\pgfqpoint{2.297325in}{1.917894in}}{\pgfqpoint{2.305139in}{1.925708in}}%
\pgfpathcurveto{\pgfqpoint{2.312952in}{1.933522in}}{\pgfqpoint{2.317343in}{1.944121in}}{\pgfqpoint{2.317343in}{1.955171in}}%
\pgfpathcurveto{\pgfqpoint{2.317343in}{1.966221in}}{\pgfqpoint{2.312952in}{1.976820in}}{\pgfqpoint{2.305139in}{1.984633in}}%
\pgfpathcurveto{\pgfqpoint{2.297325in}{1.992447in}}{\pgfqpoint{2.286726in}{1.996837in}}{\pgfqpoint{2.275676in}{1.996837in}}%
\pgfpathcurveto{\pgfqpoint{2.264626in}{1.996837in}}{\pgfqpoint{2.254027in}{1.992447in}}{\pgfqpoint{2.246213in}{1.984633in}}%
\pgfpathcurveto{\pgfqpoint{2.238399in}{1.976820in}}{\pgfqpoint{2.234009in}{1.966221in}}{\pgfqpoint{2.234009in}{1.955171in}}%
\pgfpathcurveto{\pgfqpoint{2.234009in}{1.944121in}}{\pgfqpoint{2.238399in}{1.933522in}}{\pgfqpoint{2.246213in}{1.925708in}}%
\pgfpathcurveto{\pgfqpoint{2.254027in}{1.917894in}}{\pgfqpoint{2.264626in}{1.913504in}}{\pgfqpoint{2.275676in}{1.913504in}}%
\pgfpathclose%
\pgfusepath{stroke,fill}%
\end{pgfscope}%
\begin{pgfscope}%
\pgfpathrectangle{\pgfqpoint{0.600000in}{0.600000in}}{\pgfqpoint{3.900000in}{3.900000in}}%
\pgfusepath{clip}%
\pgfsetbuttcap%
\pgfsetroundjoin%
\definecolor{currentfill}{rgb}{0.121569,0.466667,0.705882}%
\pgfsetfillcolor{currentfill}%
\pgfsetlinewidth{1.003750pt}%
\definecolor{currentstroke}{rgb}{0.121569,0.466667,0.705882}%
\pgfsetstrokecolor{currentstroke}%
\pgfsetdash{}{0pt}%
\pgfpathmoveto{\pgfqpoint{2.815364in}{2.918073in}}%
\pgfpathcurveto{\pgfqpoint{2.826414in}{2.918073in}}{\pgfqpoint{2.837013in}{2.922463in}}{\pgfqpoint{2.844826in}{2.930277in}}%
\pgfpathcurveto{\pgfqpoint{2.852640in}{2.938091in}}{\pgfqpoint{2.857030in}{2.948690in}}{\pgfqpoint{2.857030in}{2.959740in}}%
\pgfpathcurveto{\pgfqpoint{2.857030in}{2.970790in}}{\pgfqpoint{2.852640in}{2.981389in}}{\pgfqpoint{2.844826in}{2.989203in}}%
\pgfpathcurveto{\pgfqpoint{2.837013in}{2.997016in}}{\pgfqpoint{2.826414in}{3.001406in}}{\pgfqpoint{2.815364in}{3.001406in}}%
\pgfpathcurveto{\pgfqpoint{2.804313in}{3.001406in}}{\pgfqpoint{2.793714in}{2.997016in}}{\pgfqpoint{2.785901in}{2.989203in}}%
\pgfpathcurveto{\pgfqpoint{2.778087in}{2.981389in}}{\pgfqpoint{2.773697in}{2.970790in}}{\pgfqpoint{2.773697in}{2.959740in}}%
\pgfpathcurveto{\pgfqpoint{2.773697in}{2.948690in}}{\pgfqpoint{2.778087in}{2.938091in}}{\pgfqpoint{2.785901in}{2.930277in}}%
\pgfpathcurveto{\pgfqpoint{2.793714in}{2.922463in}}{\pgfqpoint{2.804313in}{2.918073in}}{\pgfqpoint{2.815364in}{2.918073in}}%
\pgfpathclose%
\pgfusepath{stroke,fill}%
\end{pgfscope}%
\begin{pgfscope}%
\pgfpathrectangle{\pgfqpoint{0.600000in}{0.600000in}}{\pgfqpoint{3.900000in}{3.900000in}}%
\pgfusepath{clip}%
\pgfsetbuttcap%
\pgfsetroundjoin%
\definecolor{currentfill}{rgb}{0.121569,0.466667,0.705882}%
\pgfsetfillcolor{currentfill}%
\pgfsetlinewidth{1.003750pt}%
\definecolor{currentstroke}{rgb}{0.121569,0.466667,0.705882}%
\pgfsetstrokecolor{currentstroke}%
\pgfsetdash{}{0pt}%
\pgfpathmoveto{\pgfqpoint{3.424978in}{3.018478in}}%
\pgfpathcurveto{\pgfqpoint{3.436028in}{3.018478in}}{\pgfqpoint{3.446627in}{3.022868in}}{\pgfqpoint{3.454440in}{3.030682in}}%
\pgfpathcurveto{\pgfqpoint{3.462254in}{3.038495in}}{\pgfqpoint{3.466644in}{3.049094in}}{\pgfqpoint{3.466644in}{3.060144in}}%
\pgfpathcurveto{\pgfqpoint{3.466644in}{3.071194in}}{\pgfqpoint{3.462254in}{3.081793in}}{\pgfqpoint{3.454440in}{3.089607in}}%
\pgfpathcurveto{\pgfqpoint{3.446627in}{3.097421in}}{\pgfqpoint{3.436028in}{3.101811in}}{\pgfqpoint{3.424978in}{3.101811in}}%
\pgfpathcurveto{\pgfqpoint{3.413927in}{3.101811in}}{\pgfqpoint{3.403328in}{3.097421in}}{\pgfqpoint{3.395515in}{3.089607in}}%
\pgfpathcurveto{\pgfqpoint{3.387701in}{3.081793in}}{\pgfqpoint{3.383311in}{3.071194in}}{\pgfqpoint{3.383311in}{3.060144in}}%
\pgfpathcurveto{\pgfqpoint{3.383311in}{3.049094in}}{\pgfqpoint{3.387701in}{3.038495in}}{\pgfqpoint{3.395515in}{3.030682in}}%
\pgfpathcurveto{\pgfqpoint{3.403328in}{3.022868in}}{\pgfqpoint{3.413927in}{3.018478in}}{\pgfqpoint{3.424978in}{3.018478in}}%
\pgfpathclose%
\pgfusepath{stroke,fill}%
\end{pgfscope}%
\begin{pgfscope}%
\pgfpathrectangle{\pgfqpoint{0.600000in}{0.600000in}}{\pgfqpoint{3.900000in}{3.900000in}}%
\pgfusepath{clip}%
\pgfsetbuttcap%
\pgfsetroundjoin%
\definecolor{currentfill}{rgb}{0.121569,0.466667,0.705882}%
\pgfsetfillcolor{currentfill}%
\pgfsetlinewidth{1.003750pt}%
\definecolor{currentstroke}{rgb}{0.121569,0.466667,0.705882}%
\pgfsetstrokecolor{currentstroke}%
\pgfsetdash{}{0pt}%
\pgfpathmoveto{\pgfqpoint{3.232996in}{2.324365in}}%
\pgfpathcurveto{\pgfqpoint{3.244046in}{2.324365in}}{\pgfqpoint{3.254645in}{2.328755in}}{\pgfqpoint{3.262459in}{2.336569in}}%
\pgfpathcurveto{\pgfqpoint{3.270272in}{2.344383in}}{\pgfqpoint{3.274663in}{2.354982in}}{\pgfqpoint{3.274663in}{2.366032in}}%
\pgfpathcurveto{\pgfqpoint{3.274663in}{2.377082in}}{\pgfqpoint{3.270272in}{2.387681in}}{\pgfqpoint{3.262459in}{2.395494in}}%
\pgfpathcurveto{\pgfqpoint{3.254645in}{2.403308in}}{\pgfqpoint{3.244046in}{2.407698in}}{\pgfqpoint{3.232996in}{2.407698in}}%
\pgfpathcurveto{\pgfqpoint{3.221946in}{2.407698in}}{\pgfqpoint{3.211347in}{2.403308in}}{\pgfqpoint{3.203533in}{2.395494in}}%
\pgfpathcurveto{\pgfqpoint{3.195720in}{2.387681in}}{\pgfqpoint{3.191329in}{2.377082in}}{\pgfqpoint{3.191329in}{2.366032in}}%
\pgfpathcurveto{\pgfqpoint{3.191329in}{2.354982in}}{\pgfqpoint{3.195720in}{2.344383in}}{\pgfqpoint{3.203533in}{2.336569in}}%
\pgfpathcurveto{\pgfqpoint{3.211347in}{2.328755in}}{\pgfqpoint{3.221946in}{2.324365in}}{\pgfqpoint{3.232996in}{2.324365in}}%
\pgfpathclose%
\pgfusepath{stroke,fill}%
\end{pgfscope}%
\begin{pgfscope}%
\pgfpathrectangle{\pgfqpoint{0.600000in}{0.600000in}}{\pgfqpoint{3.900000in}{3.900000in}}%
\pgfusepath{clip}%
\pgfsetbuttcap%
\pgfsetroundjoin%
\definecolor{currentfill}{rgb}{0.121569,0.466667,0.705882}%
\pgfsetfillcolor{currentfill}%
\pgfsetlinewidth{1.003750pt}%
\definecolor{currentstroke}{rgb}{0.121569,0.466667,0.705882}%
\pgfsetstrokecolor{currentstroke}%
\pgfsetdash{}{0pt}%
\pgfpathmoveto{\pgfqpoint{2.553394in}{2.528922in}}%
\pgfpathcurveto{\pgfqpoint{2.564444in}{2.528922in}}{\pgfqpoint{2.575043in}{2.533313in}}{\pgfqpoint{2.582856in}{2.541126in}}%
\pgfpathcurveto{\pgfqpoint{2.590670in}{2.548940in}}{\pgfqpoint{2.595060in}{2.559539in}}{\pgfqpoint{2.595060in}{2.570589in}}%
\pgfpathcurveto{\pgfqpoint{2.595060in}{2.581639in}}{\pgfqpoint{2.590670in}{2.592238in}}{\pgfqpoint{2.582856in}{2.600052in}}%
\pgfpathcurveto{\pgfqpoint{2.575043in}{2.607866in}}{\pgfqpoint{2.564444in}{2.612256in}}{\pgfqpoint{2.553394in}{2.612256in}}%
\pgfpathcurveto{\pgfqpoint{2.542343in}{2.612256in}}{\pgfqpoint{2.531744in}{2.607866in}}{\pgfqpoint{2.523931in}{2.600052in}}%
\pgfpathcurveto{\pgfqpoint{2.516117in}{2.592238in}}{\pgfqpoint{2.511727in}{2.581639in}}{\pgfqpoint{2.511727in}{2.570589in}}%
\pgfpathcurveto{\pgfqpoint{2.511727in}{2.559539in}}{\pgfqpoint{2.516117in}{2.548940in}}{\pgfqpoint{2.523931in}{2.541126in}}%
\pgfpathcurveto{\pgfqpoint{2.531744in}{2.533313in}}{\pgfqpoint{2.542343in}{2.528922in}}{\pgfqpoint{2.553394in}{2.528922in}}%
\pgfpathclose%
\pgfusepath{stroke,fill}%
\end{pgfscope}%
\begin{pgfscope}%
\pgfpathrectangle{\pgfqpoint{0.600000in}{0.600000in}}{\pgfqpoint{3.900000in}{3.900000in}}%
\pgfusepath{clip}%
\pgfsetbuttcap%
\pgfsetroundjoin%
\definecolor{currentfill}{rgb}{0.121569,0.466667,0.705882}%
\pgfsetfillcolor{currentfill}%
\pgfsetlinewidth{1.003750pt}%
\definecolor{currentstroke}{rgb}{0.121569,0.466667,0.705882}%
\pgfsetstrokecolor{currentstroke}%
\pgfsetdash{}{0pt}%
\pgfpathmoveto{\pgfqpoint{2.244190in}{2.601557in}}%
\pgfpathcurveto{\pgfqpoint{2.255241in}{2.601557in}}{\pgfqpoint{2.265840in}{2.605947in}}{\pgfqpoint{2.273653in}{2.613761in}}%
\pgfpathcurveto{\pgfqpoint{2.281467in}{2.621574in}}{\pgfqpoint{2.285857in}{2.632173in}}{\pgfqpoint{2.285857in}{2.643223in}}%
\pgfpathcurveto{\pgfqpoint{2.285857in}{2.654273in}}{\pgfqpoint{2.281467in}{2.664872in}}{\pgfqpoint{2.273653in}{2.672686in}}%
\pgfpathcurveto{\pgfqpoint{2.265840in}{2.680500in}}{\pgfqpoint{2.255241in}{2.684890in}}{\pgfqpoint{2.244190in}{2.684890in}}%
\pgfpathcurveto{\pgfqpoint{2.233140in}{2.684890in}}{\pgfqpoint{2.222541in}{2.680500in}}{\pgfqpoint{2.214728in}{2.672686in}}%
\pgfpathcurveto{\pgfqpoint{2.206914in}{2.664872in}}{\pgfqpoint{2.202524in}{2.654273in}}{\pgfqpoint{2.202524in}{2.643223in}}%
\pgfpathcurveto{\pgfqpoint{2.202524in}{2.632173in}}{\pgfqpoint{2.206914in}{2.621574in}}{\pgfqpoint{2.214728in}{2.613761in}}%
\pgfpathcurveto{\pgfqpoint{2.222541in}{2.605947in}}{\pgfqpoint{2.233140in}{2.601557in}}{\pgfqpoint{2.244190in}{2.601557in}}%
\pgfpathclose%
\pgfusepath{stroke,fill}%
\end{pgfscope}%
\begin{pgfscope}%
\pgfpathrectangle{\pgfqpoint{0.600000in}{0.600000in}}{\pgfqpoint{3.900000in}{3.900000in}}%
\pgfusepath{clip}%
\pgfsetbuttcap%
\pgfsetroundjoin%
\definecolor{currentfill}{rgb}{0.121569,0.466667,0.705882}%
\pgfsetfillcolor{currentfill}%
\pgfsetlinewidth{1.003750pt}%
\definecolor{currentstroke}{rgb}{0.121569,0.466667,0.705882}%
\pgfsetstrokecolor{currentstroke}%
\pgfsetdash{}{0pt}%
\pgfpathmoveto{\pgfqpoint{2.281332in}{3.408343in}}%
\pgfpathcurveto{\pgfqpoint{2.292382in}{3.408343in}}{\pgfqpoint{2.302981in}{3.412733in}}{\pgfqpoint{2.310795in}{3.420547in}}%
\pgfpathcurveto{\pgfqpoint{2.318608in}{3.428361in}}{\pgfqpoint{2.322999in}{3.438960in}}{\pgfqpoint{2.322999in}{3.450010in}}%
\pgfpathcurveto{\pgfqpoint{2.322999in}{3.461060in}}{\pgfqpoint{2.318608in}{3.471659in}}{\pgfqpoint{2.310795in}{3.479473in}}%
\pgfpathcurveto{\pgfqpoint{2.302981in}{3.487286in}}{\pgfqpoint{2.292382in}{3.491677in}}{\pgfqpoint{2.281332in}{3.491677in}}%
\pgfpathcurveto{\pgfqpoint{2.270282in}{3.491677in}}{\pgfqpoint{2.259683in}{3.487286in}}{\pgfqpoint{2.251869in}{3.479473in}}%
\pgfpathcurveto{\pgfqpoint{2.244056in}{3.471659in}}{\pgfqpoint{2.239665in}{3.461060in}}{\pgfqpoint{2.239665in}{3.450010in}}%
\pgfpathcurveto{\pgfqpoint{2.239665in}{3.438960in}}{\pgfqpoint{2.244056in}{3.428361in}}{\pgfqpoint{2.251869in}{3.420547in}}%
\pgfpathcurveto{\pgfqpoint{2.259683in}{3.412733in}}{\pgfqpoint{2.270282in}{3.408343in}}{\pgfqpoint{2.281332in}{3.408343in}}%
\pgfpathclose%
\pgfusepath{stroke,fill}%
\end{pgfscope}%
\begin{pgfscope}%
\pgfpathrectangle{\pgfqpoint{0.600000in}{0.600000in}}{\pgfqpoint{3.900000in}{3.900000in}}%
\pgfusepath{clip}%
\pgfsetbuttcap%
\pgfsetroundjoin%
\definecolor{currentfill}{rgb}{0.121569,0.466667,0.705882}%
\pgfsetfillcolor{currentfill}%
\pgfsetlinewidth{1.003750pt}%
\definecolor{currentstroke}{rgb}{0.121569,0.466667,0.705882}%
\pgfsetstrokecolor{currentstroke}%
\pgfsetdash{}{0pt}%
\pgfpathmoveto{\pgfqpoint{2.949780in}{2.013283in}}%
\pgfpathcurveto{\pgfqpoint{2.960830in}{2.013283in}}{\pgfqpoint{2.971429in}{2.017673in}}{\pgfqpoint{2.979243in}{2.025486in}}%
\pgfpathcurveto{\pgfqpoint{2.987056in}{2.033300in}}{\pgfqpoint{2.991446in}{2.043899in}}{\pgfqpoint{2.991446in}{2.054949in}}%
\pgfpathcurveto{\pgfqpoint{2.991446in}{2.065999in}}{\pgfqpoint{2.987056in}{2.076598in}}{\pgfqpoint{2.979243in}{2.084412in}}%
\pgfpathcurveto{\pgfqpoint{2.971429in}{2.092226in}}{\pgfqpoint{2.960830in}{2.096616in}}{\pgfqpoint{2.949780in}{2.096616in}}%
\pgfpathcurveto{\pgfqpoint{2.938730in}{2.096616in}}{\pgfqpoint{2.928131in}{2.092226in}}{\pgfqpoint{2.920317in}{2.084412in}}%
\pgfpathcurveto{\pgfqpoint{2.912503in}{2.076598in}}{\pgfqpoint{2.908113in}{2.065999in}}{\pgfqpoint{2.908113in}{2.054949in}}%
\pgfpathcurveto{\pgfqpoint{2.908113in}{2.043899in}}{\pgfqpoint{2.912503in}{2.033300in}}{\pgfqpoint{2.920317in}{2.025486in}}%
\pgfpathcurveto{\pgfqpoint{2.928131in}{2.017673in}}{\pgfqpoint{2.938730in}{2.013283in}}{\pgfqpoint{2.949780in}{2.013283in}}%
\pgfpathclose%
\pgfusepath{stroke,fill}%
\end{pgfscope}%
\begin{pgfscope}%
\pgfpathrectangle{\pgfqpoint{0.600000in}{0.600000in}}{\pgfqpoint{3.900000in}{3.900000in}}%
\pgfusepath{clip}%
\pgfsetbuttcap%
\pgfsetroundjoin%
\definecolor{currentfill}{rgb}{0.121569,0.466667,0.705882}%
\pgfsetfillcolor{currentfill}%
\pgfsetlinewidth{1.003750pt}%
\definecolor{currentstroke}{rgb}{0.121569,0.466667,0.705882}%
\pgfsetstrokecolor{currentstroke}%
\pgfsetdash{}{0pt}%
\pgfpathmoveto{\pgfqpoint{3.335255in}{2.715902in}}%
\pgfpathcurveto{\pgfqpoint{3.346305in}{2.715902in}}{\pgfqpoint{3.356904in}{2.720292in}}{\pgfqpoint{3.364717in}{2.728106in}}%
\pgfpathcurveto{\pgfqpoint{3.372531in}{2.735919in}}{\pgfqpoint{3.376921in}{2.746518in}}{\pgfqpoint{3.376921in}{2.757568in}}%
\pgfpathcurveto{\pgfqpoint{3.376921in}{2.768618in}}{\pgfqpoint{3.372531in}{2.779218in}}{\pgfqpoint{3.364717in}{2.787031in}}%
\pgfpathcurveto{\pgfqpoint{3.356904in}{2.794845in}}{\pgfqpoint{3.346305in}{2.799235in}}{\pgfqpoint{3.335255in}{2.799235in}}%
\pgfpathcurveto{\pgfqpoint{3.324204in}{2.799235in}}{\pgfqpoint{3.313605in}{2.794845in}}{\pgfqpoint{3.305792in}{2.787031in}}%
\pgfpathcurveto{\pgfqpoint{3.297978in}{2.779218in}}{\pgfqpoint{3.293588in}{2.768618in}}{\pgfqpoint{3.293588in}{2.757568in}}%
\pgfpathcurveto{\pgfqpoint{3.293588in}{2.746518in}}{\pgfqpoint{3.297978in}{2.735919in}}{\pgfqpoint{3.305792in}{2.728106in}}%
\pgfpathcurveto{\pgfqpoint{3.313605in}{2.720292in}}{\pgfqpoint{3.324204in}{2.715902in}}{\pgfqpoint{3.335255in}{2.715902in}}%
\pgfpathclose%
\pgfusepath{stroke,fill}%
\end{pgfscope}%
\begin{pgfscope}%
\pgfpathrectangle{\pgfqpoint{0.600000in}{0.600000in}}{\pgfqpoint{3.900000in}{3.900000in}}%
\pgfusepath{clip}%
\pgfsetbuttcap%
\pgfsetroundjoin%
\definecolor{currentfill}{rgb}{0.121569,0.466667,0.705882}%
\pgfsetfillcolor{currentfill}%
\pgfsetlinewidth{1.003750pt}%
\definecolor{currentstroke}{rgb}{0.121569,0.466667,0.705882}%
\pgfsetstrokecolor{currentstroke}%
\pgfsetdash{}{0pt}%
\pgfpathmoveto{\pgfqpoint{2.187677in}{2.998585in}}%
\pgfpathcurveto{\pgfqpoint{2.198727in}{2.998585in}}{\pgfqpoint{2.209326in}{3.002975in}}{\pgfqpoint{2.217139in}{3.010789in}}%
\pgfpathcurveto{\pgfqpoint{2.224953in}{3.018602in}}{\pgfqpoint{2.229343in}{3.029201in}}{\pgfqpoint{2.229343in}{3.040251in}}%
\pgfpathcurveto{\pgfqpoint{2.229343in}{3.051302in}}{\pgfqpoint{2.224953in}{3.061901in}}{\pgfqpoint{2.217139in}{3.069714in}}%
\pgfpathcurveto{\pgfqpoint{2.209326in}{3.077528in}}{\pgfqpoint{2.198727in}{3.081918in}}{\pgfqpoint{2.187677in}{3.081918in}}%
\pgfpathcurveto{\pgfqpoint{2.176626in}{3.081918in}}{\pgfqpoint{2.166027in}{3.077528in}}{\pgfqpoint{2.158214in}{3.069714in}}%
\pgfpathcurveto{\pgfqpoint{2.150400in}{3.061901in}}{\pgfqpoint{2.146010in}{3.051302in}}{\pgfqpoint{2.146010in}{3.040251in}}%
\pgfpathcurveto{\pgfqpoint{2.146010in}{3.029201in}}{\pgfqpoint{2.150400in}{3.018602in}}{\pgfqpoint{2.158214in}{3.010789in}}%
\pgfpathcurveto{\pgfqpoint{2.166027in}{3.002975in}}{\pgfqpoint{2.176626in}{2.998585in}}{\pgfqpoint{2.187677in}{2.998585in}}%
\pgfpathclose%
\pgfusepath{stroke,fill}%
\end{pgfscope}%
\begin{pgfscope}%
\pgfpathrectangle{\pgfqpoint{0.600000in}{0.600000in}}{\pgfqpoint{3.900000in}{3.900000in}}%
\pgfusepath{clip}%
\pgfsetbuttcap%
\pgfsetroundjoin%
\definecolor{currentfill}{rgb}{0.121569,0.466667,0.705882}%
\pgfsetfillcolor{currentfill}%
\pgfsetlinewidth{1.003750pt}%
\definecolor{currentstroke}{rgb}{0.121569,0.466667,0.705882}%
\pgfsetstrokecolor{currentstroke}%
\pgfsetdash{}{0pt}%
\pgfpathmoveto{\pgfqpoint{1.943405in}{3.063562in}}%
\pgfpathcurveto{\pgfqpoint{1.954455in}{3.063562in}}{\pgfqpoint{1.965054in}{3.067952in}}{\pgfqpoint{1.972868in}{3.075766in}}%
\pgfpathcurveto{\pgfqpoint{1.980681in}{3.083579in}}{\pgfqpoint{1.985071in}{3.094178in}}{\pgfqpoint{1.985071in}{3.105229in}}%
\pgfpathcurveto{\pgfqpoint{1.985071in}{3.116279in}}{\pgfqpoint{1.980681in}{3.126878in}}{\pgfqpoint{1.972868in}{3.134691in}}%
\pgfpathcurveto{\pgfqpoint{1.965054in}{3.142505in}}{\pgfqpoint{1.954455in}{3.146895in}}{\pgfqpoint{1.943405in}{3.146895in}}%
\pgfpathcurveto{\pgfqpoint{1.932355in}{3.146895in}}{\pgfqpoint{1.921756in}{3.142505in}}{\pgfqpoint{1.913942in}{3.134691in}}%
\pgfpathcurveto{\pgfqpoint{1.906128in}{3.126878in}}{\pgfqpoint{1.901738in}{3.116279in}}{\pgfqpoint{1.901738in}{3.105229in}}%
\pgfpathcurveto{\pgfqpoint{1.901738in}{3.094178in}}{\pgfqpoint{1.906128in}{3.083579in}}{\pgfqpoint{1.913942in}{3.075766in}}%
\pgfpathcurveto{\pgfqpoint{1.921756in}{3.067952in}}{\pgfqpoint{1.932355in}{3.063562in}}{\pgfqpoint{1.943405in}{3.063562in}}%
\pgfpathclose%
\pgfusepath{stroke,fill}%
\end{pgfscope}%
\begin{pgfscope}%
\pgfpathrectangle{\pgfqpoint{0.600000in}{0.600000in}}{\pgfqpoint{3.900000in}{3.900000in}}%
\pgfusepath{clip}%
\pgfsetbuttcap%
\pgfsetroundjoin%
\definecolor{currentfill}{rgb}{0.121569,0.466667,0.705882}%
\pgfsetfillcolor{currentfill}%
\pgfsetlinewidth{1.003750pt}%
\definecolor{currentstroke}{rgb}{0.121569,0.466667,0.705882}%
\pgfsetstrokecolor{currentstroke}%
\pgfsetdash{}{0pt}%
\pgfpathmoveto{\pgfqpoint{2.997594in}{2.222598in}}%
\pgfpathcurveto{\pgfqpoint{3.008644in}{2.222598in}}{\pgfqpoint{3.019243in}{2.226988in}}{\pgfqpoint{3.027057in}{2.234802in}}%
\pgfpathcurveto{\pgfqpoint{3.034870in}{2.242616in}}{\pgfqpoint{3.039260in}{2.253215in}}{\pgfqpoint{3.039260in}{2.264265in}}%
\pgfpathcurveto{\pgfqpoint{3.039260in}{2.275315in}}{\pgfqpoint{3.034870in}{2.285914in}}{\pgfqpoint{3.027057in}{2.293728in}}%
\pgfpathcurveto{\pgfqpoint{3.019243in}{2.301541in}}{\pgfqpoint{3.008644in}{2.305931in}}{\pgfqpoint{2.997594in}{2.305931in}}%
\pgfpathcurveto{\pgfqpoint{2.986544in}{2.305931in}}{\pgfqpoint{2.975945in}{2.301541in}}{\pgfqpoint{2.968131in}{2.293728in}}%
\pgfpathcurveto{\pgfqpoint{2.960317in}{2.285914in}}{\pgfqpoint{2.955927in}{2.275315in}}{\pgfqpoint{2.955927in}{2.264265in}}%
\pgfpathcurveto{\pgfqpoint{2.955927in}{2.253215in}}{\pgfqpoint{2.960317in}{2.242616in}}{\pgfqpoint{2.968131in}{2.234802in}}%
\pgfpathcurveto{\pgfqpoint{2.975945in}{2.226988in}}{\pgfqpoint{2.986544in}{2.222598in}}{\pgfqpoint{2.997594in}{2.222598in}}%
\pgfpathclose%
\pgfusepath{stroke,fill}%
\end{pgfscope}%
\begin{pgfscope}%
\pgfpathrectangle{\pgfqpoint{0.600000in}{0.600000in}}{\pgfqpoint{3.900000in}{3.900000in}}%
\pgfusepath{clip}%
\pgfsetbuttcap%
\pgfsetroundjoin%
\definecolor{currentfill}{rgb}{0.121569,0.466667,0.705882}%
\pgfsetfillcolor{currentfill}%
\pgfsetlinewidth{1.003750pt}%
\definecolor{currentstroke}{rgb}{0.121569,0.466667,0.705882}%
\pgfsetstrokecolor{currentstroke}%
\pgfsetdash{}{0pt}%
\pgfpathmoveto{\pgfqpoint{2.648337in}{1.353673in}}%
\pgfpathcurveto{\pgfqpoint{2.659387in}{1.353673in}}{\pgfqpoint{2.669986in}{1.358063in}}{\pgfqpoint{2.677799in}{1.365876in}}%
\pgfpathcurveto{\pgfqpoint{2.685613in}{1.373690in}}{\pgfqpoint{2.690003in}{1.384289in}}{\pgfqpoint{2.690003in}{1.395339in}}%
\pgfpathcurveto{\pgfqpoint{2.690003in}{1.406389in}}{\pgfqpoint{2.685613in}{1.416988in}}{\pgfqpoint{2.677799in}{1.424802in}}%
\pgfpathcurveto{\pgfqpoint{2.669986in}{1.432616in}}{\pgfqpoint{2.659387in}{1.437006in}}{\pgfqpoint{2.648337in}{1.437006in}}%
\pgfpathcurveto{\pgfqpoint{2.637287in}{1.437006in}}{\pgfqpoint{2.626688in}{1.432616in}}{\pgfqpoint{2.618874in}{1.424802in}}%
\pgfpathcurveto{\pgfqpoint{2.611060in}{1.416988in}}{\pgfqpoint{2.606670in}{1.406389in}}{\pgfqpoint{2.606670in}{1.395339in}}%
\pgfpathcurveto{\pgfqpoint{2.606670in}{1.384289in}}{\pgfqpoint{2.611060in}{1.373690in}}{\pgfqpoint{2.618874in}{1.365876in}}%
\pgfpathcurveto{\pgfqpoint{2.626688in}{1.358063in}}{\pgfqpoint{2.637287in}{1.353673in}}{\pgfqpoint{2.648337in}{1.353673in}}%
\pgfpathclose%
\pgfusepath{stroke,fill}%
\end{pgfscope}%
\begin{pgfscope}%
\pgfpathrectangle{\pgfqpoint{0.600000in}{0.600000in}}{\pgfqpoint{3.900000in}{3.900000in}}%
\pgfusepath{clip}%
\pgfsetbuttcap%
\pgfsetroundjoin%
\definecolor{currentfill}{rgb}{0.121569,0.466667,0.705882}%
\pgfsetfillcolor{currentfill}%
\pgfsetlinewidth{1.003750pt}%
\definecolor{currentstroke}{rgb}{0.121569,0.466667,0.705882}%
\pgfsetstrokecolor{currentstroke}%
\pgfsetdash{}{0pt}%
\pgfpathmoveto{\pgfqpoint{2.782030in}{3.501972in}}%
\pgfpathcurveto{\pgfqpoint{2.793080in}{3.501972in}}{\pgfqpoint{2.803679in}{3.506362in}}{\pgfqpoint{2.811493in}{3.514176in}}%
\pgfpathcurveto{\pgfqpoint{2.819306in}{3.521990in}}{\pgfqpoint{2.823697in}{3.532589in}}{\pgfqpoint{2.823697in}{3.543639in}}%
\pgfpathcurveto{\pgfqpoint{2.823697in}{3.554689in}}{\pgfqpoint{2.819306in}{3.565288in}}{\pgfqpoint{2.811493in}{3.573102in}}%
\pgfpathcurveto{\pgfqpoint{2.803679in}{3.580915in}}{\pgfqpoint{2.793080in}{3.585305in}}{\pgfqpoint{2.782030in}{3.585305in}}%
\pgfpathcurveto{\pgfqpoint{2.770980in}{3.585305in}}{\pgfqpoint{2.760381in}{3.580915in}}{\pgfqpoint{2.752567in}{3.573102in}}%
\pgfpathcurveto{\pgfqpoint{2.744754in}{3.565288in}}{\pgfqpoint{2.740363in}{3.554689in}}{\pgfqpoint{2.740363in}{3.543639in}}%
\pgfpathcurveto{\pgfqpoint{2.740363in}{3.532589in}}{\pgfqpoint{2.744754in}{3.521990in}}{\pgfqpoint{2.752567in}{3.514176in}}%
\pgfpathcurveto{\pgfqpoint{2.760381in}{3.506362in}}{\pgfqpoint{2.770980in}{3.501972in}}{\pgfqpoint{2.782030in}{3.501972in}}%
\pgfpathclose%
\pgfusepath{stroke,fill}%
\end{pgfscope}%
\begin{pgfscope}%
\pgfpathrectangle{\pgfqpoint{0.600000in}{0.600000in}}{\pgfqpoint{3.900000in}{3.900000in}}%
\pgfusepath{clip}%
\pgfsetbuttcap%
\pgfsetroundjoin%
\definecolor{currentfill}{rgb}{0.121569,0.466667,0.705882}%
\pgfsetfillcolor{currentfill}%
\pgfsetlinewidth{1.003750pt}%
\definecolor{currentstroke}{rgb}{0.121569,0.466667,0.705882}%
\pgfsetstrokecolor{currentstroke}%
\pgfsetdash{}{0pt}%
\pgfpathmoveto{\pgfqpoint{3.017845in}{2.018855in}}%
\pgfpathcurveto{\pgfqpoint{3.028895in}{2.018855in}}{\pgfqpoint{3.039494in}{2.023245in}}{\pgfqpoint{3.047308in}{2.031059in}}%
\pgfpathcurveto{\pgfqpoint{3.055121in}{2.038872in}}{\pgfqpoint{3.059512in}{2.049471in}}{\pgfqpoint{3.059512in}{2.060522in}}%
\pgfpathcurveto{\pgfqpoint{3.059512in}{2.071572in}}{\pgfqpoint{3.055121in}{2.082171in}}{\pgfqpoint{3.047308in}{2.089984in}}%
\pgfpathcurveto{\pgfqpoint{3.039494in}{2.097798in}}{\pgfqpoint{3.028895in}{2.102188in}}{\pgfqpoint{3.017845in}{2.102188in}}%
\pgfpathcurveto{\pgfqpoint{3.006795in}{2.102188in}}{\pgfqpoint{2.996196in}{2.097798in}}{\pgfqpoint{2.988382in}{2.089984in}}%
\pgfpathcurveto{\pgfqpoint{2.980569in}{2.082171in}}{\pgfqpoint{2.976178in}{2.071572in}}{\pgfqpoint{2.976178in}{2.060522in}}%
\pgfpathcurveto{\pgfqpoint{2.976178in}{2.049471in}}{\pgfqpoint{2.980569in}{2.038872in}}{\pgfqpoint{2.988382in}{2.031059in}}%
\pgfpathcurveto{\pgfqpoint{2.996196in}{2.023245in}}{\pgfqpoint{3.006795in}{2.018855in}}{\pgfqpoint{3.017845in}{2.018855in}}%
\pgfpathclose%
\pgfusepath{stroke,fill}%
\end{pgfscope}%
\begin{pgfscope}%
\pgfpathrectangle{\pgfqpoint{0.600000in}{0.600000in}}{\pgfqpoint{3.900000in}{3.900000in}}%
\pgfusepath{clip}%
\pgfsetbuttcap%
\pgfsetroundjoin%
\definecolor{currentfill}{rgb}{0.121569,0.466667,0.705882}%
\pgfsetfillcolor{currentfill}%
\pgfsetlinewidth{1.003750pt}%
\definecolor{currentstroke}{rgb}{0.121569,0.466667,0.705882}%
\pgfsetstrokecolor{currentstroke}%
\pgfsetdash{}{0pt}%
\pgfpathmoveto{\pgfqpoint{1.952205in}{2.029485in}}%
\pgfpathcurveto{\pgfqpoint{1.963255in}{2.029485in}}{\pgfqpoint{1.973854in}{2.033875in}}{\pgfqpoint{1.981667in}{2.041689in}}%
\pgfpathcurveto{\pgfqpoint{1.989481in}{2.049503in}}{\pgfqpoint{1.993871in}{2.060102in}}{\pgfqpoint{1.993871in}{2.071152in}}%
\pgfpathcurveto{\pgfqpoint{1.993871in}{2.082202in}}{\pgfqpoint{1.989481in}{2.092801in}}{\pgfqpoint{1.981667in}{2.100615in}}%
\pgfpathcurveto{\pgfqpoint{1.973854in}{2.108428in}}{\pgfqpoint{1.963255in}{2.112819in}}{\pgfqpoint{1.952205in}{2.112819in}}%
\pgfpathcurveto{\pgfqpoint{1.941155in}{2.112819in}}{\pgfqpoint{1.930556in}{2.108428in}}{\pgfqpoint{1.922742in}{2.100615in}}%
\pgfpathcurveto{\pgfqpoint{1.914928in}{2.092801in}}{\pgfqpoint{1.910538in}{2.082202in}}{\pgfqpoint{1.910538in}{2.071152in}}%
\pgfpathcurveto{\pgfqpoint{1.910538in}{2.060102in}}{\pgfqpoint{1.914928in}{2.049503in}}{\pgfqpoint{1.922742in}{2.041689in}}%
\pgfpathcurveto{\pgfqpoint{1.930556in}{2.033875in}}{\pgfqpoint{1.941155in}{2.029485in}}{\pgfqpoint{1.952205in}{2.029485in}}%
\pgfpathclose%
\pgfusepath{stroke,fill}%
\end{pgfscope}%
\begin{pgfscope}%
\pgfpathrectangle{\pgfqpoint{0.600000in}{0.600000in}}{\pgfqpoint{3.900000in}{3.900000in}}%
\pgfusepath{clip}%
\pgfsetbuttcap%
\pgfsetroundjoin%
\definecolor{currentfill}{rgb}{0.121569,0.466667,0.705882}%
\pgfsetfillcolor{currentfill}%
\pgfsetlinewidth{1.003750pt}%
\definecolor{currentstroke}{rgb}{0.121569,0.466667,0.705882}%
\pgfsetstrokecolor{currentstroke}%
\pgfsetdash{}{0pt}%
\pgfpathmoveto{\pgfqpoint{2.734785in}{2.749131in}}%
\pgfpathcurveto{\pgfqpoint{2.745835in}{2.749131in}}{\pgfqpoint{2.756434in}{2.753521in}}{\pgfqpoint{2.764248in}{2.761334in}}%
\pgfpathcurveto{\pgfqpoint{2.772062in}{2.769148in}}{\pgfqpoint{2.776452in}{2.779747in}}{\pgfqpoint{2.776452in}{2.790797in}}%
\pgfpathcurveto{\pgfqpoint{2.776452in}{2.801847in}}{\pgfqpoint{2.772062in}{2.812446in}}{\pgfqpoint{2.764248in}{2.820260in}}%
\pgfpathcurveto{\pgfqpoint{2.756434in}{2.828074in}}{\pgfqpoint{2.745835in}{2.832464in}}{\pgfqpoint{2.734785in}{2.832464in}}%
\pgfpathcurveto{\pgfqpoint{2.723735in}{2.832464in}}{\pgfqpoint{2.713136in}{2.828074in}}{\pgfqpoint{2.705322in}{2.820260in}}%
\pgfpathcurveto{\pgfqpoint{2.697509in}{2.812446in}}{\pgfqpoint{2.693119in}{2.801847in}}{\pgfqpoint{2.693119in}{2.790797in}}%
\pgfpathcurveto{\pgfqpoint{2.693119in}{2.779747in}}{\pgfqpoint{2.697509in}{2.769148in}}{\pgfqpoint{2.705322in}{2.761334in}}%
\pgfpathcurveto{\pgfqpoint{2.713136in}{2.753521in}}{\pgfqpoint{2.723735in}{2.749131in}}{\pgfqpoint{2.734785in}{2.749131in}}%
\pgfpathclose%
\pgfusepath{stroke,fill}%
\end{pgfscope}%
\begin{pgfscope}%
\pgfpathrectangle{\pgfqpoint{0.600000in}{0.600000in}}{\pgfqpoint{3.900000in}{3.900000in}}%
\pgfusepath{clip}%
\pgfsetbuttcap%
\pgfsetroundjoin%
\definecolor{currentfill}{rgb}{0.121569,0.466667,0.705882}%
\pgfsetfillcolor{currentfill}%
\pgfsetlinewidth{1.003750pt}%
\definecolor{currentstroke}{rgb}{0.121569,0.466667,0.705882}%
\pgfsetstrokecolor{currentstroke}%
\pgfsetdash{}{0pt}%
\pgfpathmoveto{\pgfqpoint{2.687962in}{2.185715in}}%
\pgfpathcurveto{\pgfqpoint{2.699012in}{2.185715in}}{\pgfqpoint{2.709611in}{2.190105in}}{\pgfqpoint{2.717425in}{2.197918in}}%
\pgfpathcurveto{\pgfqpoint{2.725239in}{2.205732in}}{\pgfqpoint{2.729629in}{2.216331in}}{\pgfqpoint{2.729629in}{2.227381in}}%
\pgfpathcurveto{\pgfqpoint{2.729629in}{2.238431in}}{\pgfqpoint{2.725239in}{2.249030in}}{\pgfqpoint{2.717425in}{2.256844in}}%
\pgfpathcurveto{\pgfqpoint{2.709611in}{2.264658in}}{\pgfqpoint{2.699012in}{2.269048in}}{\pgfqpoint{2.687962in}{2.269048in}}%
\pgfpathcurveto{\pgfqpoint{2.676912in}{2.269048in}}{\pgfqpoint{2.666313in}{2.264658in}}{\pgfqpoint{2.658499in}{2.256844in}}%
\pgfpathcurveto{\pgfqpoint{2.650686in}{2.249030in}}{\pgfqpoint{2.646295in}{2.238431in}}{\pgfqpoint{2.646295in}{2.227381in}}%
\pgfpathcurveto{\pgfqpoint{2.646295in}{2.216331in}}{\pgfqpoint{2.650686in}{2.205732in}}{\pgfqpoint{2.658499in}{2.197918in}}%
\pgfpathcurveto{\pgfqpoint{2.666313in}{2.190105in}}{\pgfqpoint{2.676912in}{2.185715in}}{\pgfqpoint{2.687962in}{2.185715in}}%
\pgfpathclose%
\pgfusepath{stroke,fill}%
\end{pgfscope}%
\begin{pgfscope}%
\pgfpathrectangle{\pgfqpoint{0.600000in}{0.600000in}}{\pgfqpoint{3.900000in}{3.900000in}}%
\pgfusepath{clip}%
\pgfsetbuttcap%
\pgfsetroundjoin%
\definecolor{currentfill}{rgb}{0.121569,0.466667,0.705882}%
\pgfsetfillcolor{currentfill}%
\pgfsetlinewidth{1.003750pt}%
\definecolor{currentstroke}{rgb}{0.121569,0.466667,0.705882}%
\pgfsetstrokecolor{currentstroke}%
\pgfsetdash{}{0pt}%
\pgfpathmoveto{\pgfqpoint{2.481863in}{2.338166in}}%
\pgfpathcurveto{\pgfqpoint{2.492914in}{2.338166in}}{\pgfqpoint{2.503513in}{2.342556in}}{\pgfqpoint{2.511326in}{2.350370in}}%
\pgfpathcurveto{\pgfqpoint{2.519140in}{2.358183in}}{\pgfqpoint{2.523530in}{2.368782in}}{\pgfqpoint{2.523530in}{2.379832in}}%
\pgfpathcurveto{\pgfqpoint{2.523530in}{2.390882in}}{\pgfqpoint{2.519140in}{2.401482in}}{\pgfqpoint{2.511326in}{2.409295in}}%
\pgfpathcurveto{\pgfqpoint{2.503513in}{2.417109in}}{\pgfqpoint{2.492914in}{2.421499in}}{\pgfqpoint{2.481863in}{2.421499in}}%
\pgfpathcurveto{\pgfqpoint{2.470813in}{2.421499in}}{\pgfqpoint{2.460214in}{2.417109in}}{\pgfqpoint{2.452401in}{2.409295in}}%
\pgfpathcurveto{\pgfqpoint{2.444587in}{2.401482in}}{\pgfqpoint{2.440197in}{2.390882in}}{\pgfqpoint{2.440197in}{2.379832in}}%
\pgfpathcurveto{\pgfqpoint{2.440197in}{2.368782in}}{\pgfqpoint{2.444587in}{2.358183in}}{\pgfqpoint{2.452401in}{2.350370in}}%
\pgfpathcurveto{\pgfqpoint{2.460214in}{2.342556in}}{\pgfqpoint{2.470813in}{2.338166in}}{\pgfqpoint{2.481863in}{2.338166in}}%
\pgfpathclose%
\pgfusepath{stroke,fill}%
\end{pgfscope}%
\begin{pgfscope}%
\pgfpathrectangle{\pgfqpoint{0.600000in}{0.600000in}}{\pgfqpoint{3.900000in}{3.900000in}}%
\pgfusepath{clip}%
\pgfsetbuttcap%
\pgfsetroundjoin%
\definecolor{currentfill}{rgb}{0.121569,0.466667,0.705882}%
\pgfsetfillcolor{currentfill}%
\pgfsetlinewidth{1.003750pt}%
\definecolor{currentstroke}{rgb}{0.121569,0.466667,0.705882}%
\pgfsetstrokecolor{currentstroke}%
\pgfsetdash{}{0pt}%
\pgfpathmoveto{\pgfqpoint{2.356961in}{2.409660in}}%
\pgfpathcurveto{\pgfqpoint{2.368012in}{2.409660in}}{\pgfqpoint{2.378611in}{2.414050in}}{\pgfqpoint{2.386424in}{2.421864in}}%
\pgfpathcurveto{\pgfqpoint{2.394238in}{2.429677in}}{\pgfqpoint{2.398628in}{2.440276in}}{\pgfqpoint{2.398628in}{2.451327in}}%
\pgfpathcurveto{\pgfqpoint{2.398628in}{2.462377in}}{\pgfqpoint{2.394238in}{2.472976in}}{\pgfqpoint{2.386424in}{2.480789in}}%
\pgfpathcurveto{\pgfqpoint{2.378611in}{2.488603in}}{\pgfqpoint{2.368012in}{2.492993in}}{\pgfqpoint{2.356961in}{2.492993in}}%
\pgfpathcurveto{\pgfqpoint{2.345911in}{2.492993in}}{\pgfqpoint{2.335312in}{2.488603in}}{\pgfqpoint{2.327499in}{2.480789in}}%
\pgfpathcurveto{\pgfqpoint{2.319685in}{2.472976in}}{\pgfqpoint{2.315295in}{2.462377in}}{\pgfqpoint{2.315295in}{2.451327in}}%
\pgfpathcurveto{\pgfqpoint{2.315295in}{2.440276in}}{\pgfqpoint{2.319685in}{2.429677in}}{\pgfqpoint{2.327499in}{2.421864in}}%
\pgfpathcurveto{\pgfqpoint{2.335312in}{2.414050in}}{\pgfqpoint{2.345911in}{2.409660in}}{\pgfqpoint{2.356961in}{2.409660in}}%
\pgfpathclose%
\pgfusepath{stroke,fill}%
\end{pgfscope}%
\begin{pgfscope}%
\pgfpathrectangle{\pgfqpoint{0.600000in}{0.600000in}}{\pgfqpoint{3.900000in}{3.900000in}}%
\pgfusepath{clip}%
\pgfsetbuttcap%
\pgfsetroundjoin%
\definecolor{currentfill}{rgb}{0.121569,0.466667,0.705882}%
\pgfsetfillcolor{currentfill}%
\pgfsetlinewidth{1.003750pt}%
\definecolor{currentstroke}{rgb}{0.121569,0.466667,0.705882}%
\pgfsetstrokecolor{currentstroke}%
\pgfsetdash{}{0pt}%
\pgfpathmoveto{\pgfqpoint{2.716351in}{1.998279in}}%
\pgfpathcurveto{\pgfqpoint{2.727401in}{1.998279in}}{\pgfqpoint{2.738000in}{2.002670in}}{\pgfqpoint{2.745814in}{2.010483in}}%
\pgfpathcurveto{\pgfqpoint{2.753627in}{2.018297in}}{\pgfqpoint{2.758018in}{2.028896in}}{\pgfqpoint{2.758018in}{2.039946in}}%
\pgfpathcurveto{\pgfqpoint{2.758018in}{2.050996in}}{\pgfqpoint{2.753627in}{2.061595in}}{\pgfqpoint{2.745814in}{2.069409in}}%
\pgfpathcurveto{\pgfqpoint{2.738000in}{2.077222in}}{\pgfqpoint{2.727401in}{2.081613in}}{\pgfqpoint{2.716351in}{2.081613in}}%
\pgfpathcurveto{\pgfqpoint{2.705301in}{2.081613in}}{\pgfqpoint{2.694702in}{2.077222in}}{\pgfqpoint{2.686888in}{2.069409in}}%
\pgfpathcurveto{\pgfqpoint{2.679074in}{2.061595in}}{\pgfqpoint{2.674684in}{2.050996in}}{\pgfqpoint{2.674684in}{2.039946in}}%
\pgfpathcurveto{\pgfqpoint{2.674684in}{2.028896in}}{\pgfqpoint{2.679074in}{2.018297in}}{\pgfqpoint{2.686888in}{2.010483in}}%
\pgfpathcurveto{\pgfqpoint{2.694702in}{2.002670in}}{\pgfqpoint{2.705301in}{1.998279in}}{\pgfqpoint{2.716351in}{1.998279in}}%
\pgfpathclose%
\pgfusepath{stroke,fill}%
\end{pgfscope}%
\begin{pgfscope}%
\pgfpathrectangle{\pgfqpoint{0.600000in}{0.600000in}}{\pgfqpoint{3.900000in}{3.900000in}}%
\pgfusepath{clip}%
\pgfsetbuttcap%
\pgfsetroundjoin%
\definecolor{currentfill}{rgb}{0.121569,0.466667,0.705882}%
\pgfsetfillcolor{currentfill}%
\pgfsetlinewidth{1.003750pt}%
\definecolor{currentstroke}{rgb}{0.121569,0.466667,0.705882}%
\pgfsetstrokecolor{currentstroke}%
\pgfsetdash{}{0pt}%
\pgfpathmoveto{\pgfqpoint{2.076248in}{2.547620in}}%
\pgfpathcurveto{\pgfqpoint{2.087298in}{2.547620in}}{\pgfqpoint{2.097897in}{2.552010in}}{\pgfqpoint{2.105711in}{2.559824in}}%
\pgfpathcurveto{\pgfqpoint{2.113525in}{2.567637in}}{\pgfqpoint{2.117915in}{2.578236in}}{\pgfqpoint{2.117915in}{2.589286in}}%
\pgfpathcurveto{\pgfqpoint{2.117915in}{2.600336in}}{\pgfqpoint{2.113525in}{2.610935in}}{\pgfqpoint{2.105711in}{2.618749in}}%
\pgfpathcurveto{\pgfqpoint{2.097897in}{2.626563in}}{\pgfqpoint{2.087298in}{2.630953in}}{\pgfqpoint{2.076248in}{2.630953in}}%
\pgfpathcurveto{\pgfqpoint{2.065198in}{2.630953in}}{\pgfqpoint{2.054599in}{2.626563in}}{\pgfqpoint{2.046785in}{2.618749in}}%
\pgfpathcurveto{\pgfqpoint{2.038972in}{2.610935in}}{\pgfqpoint{2.034582in}{2.600336in}}{\pgfqpoint{2.034582in}{2.589286in}}%
\pgfpathcurveto{\pgfqpoint{2.034582in}{2.578236in}}{\pgfqpoint{2.038972in}{2.567637in}}{\pgfqpoint{2.046785in}{2.559824in}}%
\pgfpathcurveto{\pgfqpoint{2.054599in}{2.552010in}}{\pgfqpoint{2.065198in}{2.547620in}}{\pgfqpoint{2.076248in}{2.547620in}}%
\pgfpathclose%
\pgfusepath{stroke,fill}%
\end{pgfscope}%
\begin{pgfscope}%
\pgfpathrectangle{\pgfqpoint{0.600000in}{0.600000in}}{\pgfqpoint{3.900000in}{3.900000in}}%
\pgfusepath{clip}%
\pgfsetbuttcap%
\pgfsetroundjoin%
\definecolor{currentfill}{rgb}{0.121569,0.466667,0.705882}%
\pgfsetfillcolor{currentfill}%
\pgfsetlinewidth{1.003750pt}%
\definecolor{currentstroke}{rgb}{0.121569,0.466667,0.705882}%
\pgfsetstrokecolor{currentstroke}%
\pgfsetdash{}{0pt}%
\pgfpathmoveto{\pgfqpoint{2.716375in}{1.968214in}}%
\pgfpathcurveto{\pgfqpoint{2.727425in}{1.968214in}}{\pgfqpoint{2.738024in}{1.972605in}}{\pgfqpoint{2.745838in}{1.980418in}}%
\pgfpathcurveto{\pgfqpoint{2.753651in}{1.988232in}}{\pgfqpoint{2.758042in}{1.998831in}}{\pgfqpoint{2.758042in}{2.009881in}}%
\pgfpathcurveto{\pgfqpoint{2.758042in}{2.020931in}}{\pgfqpoint{2.753651in}{2.031530in}}{\pgfqpoint{2.745838in}{2.039344in}}%
\pgfpathcurveto{\pgfqpoint{2.738024in}{2.047158in}}{\pgfqpoint{2.727425in}{2.051548in}}{\pgfqpoint{2.716375in}{2.051548in}}%
\pgfpathcurveto{\pgfqpoint{2.705325in}{2.051548in}}{\pgfqpoint{2.694726in}{2.047158in}}{\pgfqpoint{2.686912in}{2.039344in}}%
\pgfpathcurveto{\pgfqpoint{2.679098in}{2.031530in}}{\pgfqpoint{2.674708in}{2.020931in}}{\pgfqpoint{2.674708in}{2.009881in}}%
\pgfpathcurveto{\pgfqpoint{2.674708in}{1.998831in}}{\pgfqpoint{2.679098in}{1.988232in}}{\pgfqpoint{2.686912in}{1.980418in}}%
\pgfpathcurveto{\pgfqpoint{2.694726in}{1.972605in}}{\pgfqpoint{2.705325in}{1.968214in}}{\pgfqpoint{2.716375in}{1.968214in}}%
\pgfpathclose%
\pgfusepath{stroke,fill}%
\end{pgfscope}%
\begin{pgfscope}%
\pgfpathrectangle{\pgfqpoint{0.600000in}{0.600000in}}{\pgfqpoint{3.900000in}{3.900000in}}%
\pgfusepath{clip}%
\pgfsetbuttcap%
\pgfsetroundjoin%
\definecolor{currentfill}{rgb}{0.121569,0.466667,0.705882}%
\pgfsetfillcolor{currentfill}%
\pgfsetlinewidth{1.003750pt}%
\definecolor{currentstroke}{rgb}{0.121569,0.466667,0.705882}%
\pgfsetstrokecolor{currentstroke}%
\pgfsetdash{}{0pt}%
\pgfpathmoveto{\pgfqpoint{2.513176in}{2.155812in}}%
\pgfpathcurveto{\pgfqpoint{2.524226in}{2.155812in}}{\pgfqpoint{2.534825in}{2.160203in}}{\pgfqpoint{2.542639in}{2.168016in}}%
\pgfpathcurveto{\pgfqpoint{2.550452in}{2.175830in}}{\pgfqpoint{2.554843in}{2.186429in}}{\pgfqpoint{2.554843in}{2.197479in}}%
\pgfpathcurveto{\pgfqpoint{2.554843in}{2.208529in}}{\pgfqpoint{2.550452in}{2.219128in}}{\pgfqpoint{2.542639in}{2.226942in}}%
\pgfpathcurveto{\pgfqpoint{2.534825in}{2.234755in}}{\pgfqpoint{2.524226in}{2.239146in}}{\pgfqpoint{2.513176in}{2.239146in}}%
\pgfpathcurveto{\pgfqpoint{2.502126in}{2.239146in}}{\pgfqpoint{2.491527in}{2.234755in}}{\pgfqpoint{2.483713in}{2.226942in}}%
\pgfpathcurveto{\pgfqpoint{2.475900in}{2.219128in}}{\pgfqpoint{2.471509in}{2.208529in}}{\pgfqpoint{2.471509in}{2.197479in}}%
\pgfpathcurveto{\pgfqpoint{2.471509in}{2.186429in}}{\pgfqpoint{2.475900in}{2.175830in}}{\pgfqpoint{2.483713in}{2.168016in}}%
\pgfpathcurveto{\pgfqpoint{2.491527in}{2.160203in}}{\pgfqpoint{2.502126in}{2.155812in}}{\pgfqpoint{2.513176in}{2.155812in}}%
\pgfpathclose%
\pgfusepath{stroke,fill}%
\end{pgfscope}%
\begin{pgfscope}%
\pgfpathrectangle{\pgfqpoint{0.600000in}{0.600000in}}{\pgfqpoint{3.900000in}{3.900000in}}%
\pgfusepath{clip}%
\pgfsetbuttcap%
\pgfsetroundjoin%
\definecolor{currentfill}{rgb}{0.121569,0.466667,0.705882}%
\pgfsetfillcolor{currentfill}%
\pgfsetlinewidth{1.003750pt}%
\definecolor{currentstroke}{rgb}{0.121569,0.466667,0.705882}%
\pgfsetstrokecolor{currentstroke}%
\pgfsetdash{}{0pt}%
\pgfpathmoveto{\pgfqpoint{2.635238in}{2.433332in}}%
\pgfpathcurveto{\pgfqpoint{2.646288in}{2.433332in}}{\pgfqpoint{2.656887in}{2.437722in}}{\pgfqpoint{2.664701in}{2.445536in}}%
\pgfpathcurveto{\pgfqpoint{2.672515in}{2.453349in}}{\pgfqpoint{2.676905in}{2.463948in}}{\pgfqpoint{2.676905in}{2.474998in}}%
\pgfpathcurveto{\pgfqpoint{2.676905in}{2.486048in}}{\pgfqpoint{2.672515in}{2.496648in}}{\pgfqpoint{2.664701in}{2.504461in}}%
\pgfpathcurveto{\pgfqpoint{2.656887in}{2.512275in}}{\pgfqpoint{2.646288in}{2.516665in}}{\pgfqpoint{2.635238in}{2.516665in}}%
\pgfpathcurveto{\pgfqpoint{2.624188in}{2.516665in}}{\pgfqpoint{2.613589in}{2.512275in}}{\pgfqpoint{2.605776in}{2.504461in}}%
\pgfpathcurveto{\pgfqpoint{2.597962in}{2.496648in}}{\pgfqpoint{2.593572in}{2.486048in}}{\pgfqpoint{2.593572in}{2.474998in}}%
\pgfpathcurveto{\pgfqpoint{2.593572in}{2.463948in}}{\pgfqpoint{2.597962in}{2.453349in}}{\pgfqpoint{2.605776in}{2.445536in}}%
\pgfpathcurveto{\pgfqpoint{2.613589in}{2.437722in}}{\pgfqpoint{2.624188in}{2.433332in}}{\pgfqpoint{2.635238in}{2.433332in}}%
\pgfpathclose%
\pgfusepath{stroke,fill}%
\end{pgfscope}%
\begin{pgfscope}%
\pgfpathrectangle{\pgfqpoint{0.600000in}{0.600000in}}{\pgfqpoint{3.900000in}{3.900000in}}%
\pgfusepath{clip}%
\pgfsetbuttcap%
\pgfsetroundjoin%
\definecolor{currentfill}{rgb}{0.121569,0.466667,0.705882}%
\pgfsetfillcolor{currentfill}%
\pgfsetlinewidth{1.003750pt}%
\definecolor{currentstroke}{rgb}{0.121569,0.466667,0.705882}%
\pgfsetstrokecolor{currentstroke}%
\pgfsetdash{}{0pt}%
\pgfpathmoveto{\pgfqpoint{2.472464in}{3.197055in}}%
\pgfpathcurveto{\pgfqpoint{2.483514in}{3.197055in}}{\pgfqpoint{2.494113in}{3.201445in}}{\pgfqpoint{2.501926in}{3.209259in}}%
\pgfpathcurveto{\pgfqpoint{2.509740in}{3.217073in}}{\pgfqpoint{2.514130in}{3.227672in}}{\pgfqpoint{2.514130in}{3.238722in}}%
\pgfpathcurveto{\pgfqpoint{2.514130in}{3.249772in}}{\pgfqpoint{2.509740in}{3.260371in}}{\pgfqpoint{2.501926in}{3.268185in}}%
\pgfpathcurveto{\pgfqpoint{2.494113in}{3.275998in}}{\pgfqpoint{2.483514in}{3.280388in}}{\pgfqpoint{2.472464in}{3.280388in}}%
\pgfpathcurveto{\pgfqpoint{2.461413in}{3.280388in}}{\pgfqpoint{2.450814in}{3.275998in}}{\pgfqpoint{2.443001in}{3.268185in}}%
\pgfpathcurveto{\pgfqpoint{2.435187in}{3.260371in}}{\pgfqpoint{2.430797in}{3.249772in}}{\pgfqpoint{2.430797in}{3.238722in}}%
\pgfpathcurveto{\pgfqpoint{2.430797in}{3.227672in}}{\pgfqpoint{2.435187in}{3.217073in}}{\pgfqpoint{2.443001in}{3.209259in}}%
\pgfpathcurveto{\pgfqpoint{2.450814in}{3.201445in}}{\pgfqpoint{2.461413in}{3.197055in}}{\pgfqpoint{2.472464in}{3.197055in}}%
\pgfpathclose%
\pgfusepath{stroke,fill}%
\end{pgfscope}%
\begin{pgfscope}%
\pgfpathrectangle{\pgfqpoint{0.600000in}{0.600000in}}{\pgfqpoint{3.900000in}{3.900000in}}%
\pgfusepath{clip}%
\pgfsetbuttcap%
\pgfsetroundjoin%
\definecolor{currentfill}{rgb}{0.121569,0.466667,0.705882}%
\pgfsetfillcolor{currentfill}%
\pgfsetlinewidth{1.003750pt}%
\definecolor{currentstroke}{rgb}{0.121569,0.466667,0.705882}%
\pgfsetstrokecolor{currentstroke}%
\pgfsetdash{}{0pt}%
\pgfpathmoveto{\pgfqpoint{3.821867in}{2.072872in}}%
\pgfpathcurveto{\pgfqpoint{3.832917in}{2.072872in}}{\pgfqpoint{3.843516in}{2.077263in}}{\pgfqpoint{3.851330in}{2.085076in}}%
\pgfpathcurveto{\pgfqpoint{3.859143in}{2.092890in}}{\pgfqpoint{3.863533in}{2.103489in}}{\pgfqpoint{3.863533in}{2.114539in}}%
\pgfpathcurveto{\pgfqpoint{3.863533in}{2.125589in}}{\pgfqpoint{3.859143in}{2.136188in}}{\pgfqpoint{3.851330in}{2.144002in}}%
\pgfpathcurveto{\pgfqpoint{3.843516in}{2.151815in}}{\pgfqpoint{3.832917in}{2.156206in}}{\pgfqpoint{3.821867in}{2.156206in}}%
\pgfpathcurveto{\pgfqpoint{3.810817in}{2.156206in}}{\pgfqpoint{3.800218in}{2.151815in}}{\pgfqpoint{3.792404in}{2.144002in}}%
\pgfpathcurveto{\pgfqpoint{3.784590in}{2.136188in}}{\pgfqpoint{3.780200in}{2.125589in}}{\pgfqpoint{3.780200in}{2.114539in}}%
\pgfpathcurveto{\pgfqpoint{3.780200in}{2.103489in}}{\pgfqpoint{3.784590in}{2.092890in}}{\pgfqpoint{3.792404in}{2.085076in}}%
\pgfpathcurveto{\pgfqpoint{3.800218in}{2.077263in}}{\pgfqpoint{3.810817in}{2.072872in}}{\pgfqpoint{3.821867in}{2.072872in}}%
\pgfpathclose%
\pgfusepath{stroke,fill}%
\end{pgfscope}%
\begin{pgfscope}%
\pgfpathrectangle{\pgfqpoint{0.600000in}{0.600000in}}{\pgfqpoint{3.900000in}{3.900000in}}%
\pgfusepath{clip}%
\pgfsetbuttcap%
\pgfsetroundjoin%
\definecolor{currentfill}{rgb}{0.121569,0.466667,0.705882}%
\pgfsetfillcolor{currentfill}%
\pgfsetlinewidth{1.003750pt}%
\definecolor{currentstroke}{rgb}{0.121569,0.466667,0.705882}%
\pgfsetstrokecolor{currentstroke}%
\pgfsetdash{}{0pt}%
\pgfpathmoveto{\pgfqpoint{1.282485in}{2.823043in}}%
\pgfpathcurveto{\pgfqpoint{1.293535in}{2.823043in}}{\pgfqpoint{1.304134in}{2.827433in}}{\pgfqpoint{1.311948in}{2.835247in}}%
\pgfpathcurveto{\pgfqpoint{1.319762in}{2.843061in}}{\pgfqpoint{1.324152in}{2.853660in}}{\pgfqpoint{1.324152in}{2.864710in}}%
\pgfpathcurveto{\pgfqpoint{1.324152in}{2.875760in}}{\pgfqpoint{1.319762in}{2.886359in}}{\pgfqpoint{1.311948in}{2.894173in}}%
\pgfpathcurveto{\pgfqpoint{1.304134in}{2.901986in}}{\pgfqpoint{1.293535in}{2.906377in}}{\pgfqpoint{1.282485in}{2.906377in}}%
\pgfpathcurveto{\pgfqpoint{1.271435in}{2.906377in}}{\pgfqpoint{1.260836in}{2.901986in}}{\pgfqpoint{1.253022in}{2.894173in}}%
\pgfpathcurveto{\pgfqpoint{1.245209in}{2.886359in}}{\pgfqpoint{1.240819in}{2.875760in}}{\pgfqpoint{1.240819in}{2.864710in}}%
\pgfpathcurveto{\pgfqpoint{1.240819in}{2.853660in}}{\pgfqpoint{1.245209in}{2.843061in}}{\pgfqpoint{1.253022in}{2.835247in}}%
\pgfpathcurveto{\pgfqpoint{1.260836in}{2.827433in}}{\pgfqpoint{1.271435in}{2.823043in}}{\pgfqpoint{1.282485in}{2.823043in}}%
\pgfpathclose%
\pgfusepath{stroke,fill}%
\end{pgfscope}%
\begin{pgfscope}%
\pgfpathrectangle{\pgfqpoint{0.600000in}{0.600000in}}{\pgfqpoint{3.900000in}{3.900000in}}%
\pgfusepath{clip}%
\pgfsetbuttcap%
\pgfsetroundjoin%
\definecolor{currentfill}{rgb}{0.121569,0.466667,0.705882}%
\pgfsetfillcolor{currentfill}%
\pgfsetlinewidth{1.003750pt}%
\definecolor{currentstroke}{rgb}{0.121569,0.466667,0.705882}%
\pgfsetstrokecolor{currentstroke}%
\pgfsetdash{}{0pt}%
\pgfpathmoveto{\pgfqpoint{2.215963in}{2.465840in}}%
\pgfpathcurveto{\pgfqpoint{2.227013in}{2.465840in}}{\pgfqpoint{2.237612in}{2.470230in}}{\pgfqpoint{2.245426in}{2.478044in}}%
\pgfpathcurveto{\pgfqpoint{2.253240in}{2.485857in}}{\pgfqpoint{2.257630in}{2.496456in}}{\pgfqpoint{2.257630in}{2.507506in}}%
\pgfpathcurveto{\pgfqpoint{2.257630in}{2.518557in}}{\pgfqpoint{2.253240in}{2.529156in}}{\pgfqpoint{2.245426in}{2.536969in}}%
\pgfpathcurveto{\pgfqpoint{2.237612in}{2.544783in}}{\pgfqpoint{2.227013in}{2.549173in}}{\pgfqpoint{2.215963in}{2.549173in}}%
\pgfpathcurveto{\pgfqpoint{2.204913in}{2.549173in}}{\pgfqpoint{2.194314in}{2.544783in}}{\pgfqpoint{2.186501in}{2.536969in}}%
\pgfpathcurveto{\pgfqpoint{2.178687in}{2.529156in}}{\pgfqpoint{2.174297in}{2.518557in}}{\pgfqpoint{2.174297in}{2.507506in}}%
\pgfpathcurveto{\pgfqpoint{2.174297in}{2.496456in}}{\pgfqpoint{2.178687in}{2.485857in}}{\pgfqpoint{2.186501in}{2.478044in}}%
\pgfpathcurveto{\pgfqpoint{2.194314in}{2.470230in}}{\pgfqpoint{2.204913in}{2.465840in}}{\pgfqpoint{2.215963in}{2.465840in}}%
\pgfpathclose%
\pgfusepath{stroke,fill}%
\end{pgfscope}%
\begin{pgfscope}%
\pgfpathrectangle{\pgfqpoint{0.600000in}{0.600000in}}{\pgfqpoint{3.900000in}{3.900000in}}%
\pgfusepath{clip}%
\pgfsetbuttcap%
\pgfsetroundjoin%
\definecolor{currentfill}{rgb}{0.121569,0.466667,0.705882}%
\pgfsetfillcolor{currentfill}%
\pgfsetlinewidth{1.003750pt}%
\definecolor{currentstroke}{rgb}{0.121569,0.466667,0.705882}%
\pgfsetstrokecolor{currentstroke}%
\pgfsetdash{}{0pt}%
\pgfpathmoveto{\pgfqpoint{2.477243in}{3.454151in}}%
\pgfpathcurveto{\pgfqpoint{2.488293in}{3.454151in}}{\pgfqpoint{2.498892in}{3.458541in}}{\pgfqpoint{2.506706in}{3.466355in}}%
\pgfpathcurveto{\pgfqpoint{2.514520in}{3.474169in}}{\pgfqpoint{2.518910in}{3.484768in}}{\pgfqpoint{2.518910in}{3.495818in}}%
\pgfpathcurveto{\pgfqpoint{2.518910in}{3.506868in}}{\pgfqpoint{2.514520in}{3.517467in}}{\pgfqpoint{2.506706in}{3.525280in}}%
\pgfpathcurveto{\pgfqpoint{2.498892in}{3.533094in}}{\pgfqpoint{2.488293in}{3.537484in}}{\pgfqpoint{2.477243in}{3.537484in}}%
\pgfpathcurveto{\pgfqpoint{2.466193in}{3.537484in}}{\pgfqpoint{2.455594in}{3.533094in}}{\pgfqpoint{2.447780in}{3.525280in}}%
\pgfpathcurveto{\pgfqpoint{2.439967in}{3.517467in}}{\pgfqpoint{2.435577in}{3.506868in}}{\pgfqpoint{2.435577in}{3.495818in}}%
\pgfpathcurveto{\pgfqpoint{2.435577in}{3.484768in}}{\pgfqpoint{2.439967in}{3.474169in}}{\pgfqpoint{2.447780in}{3.466355in}}%
\pgfpathcurveto{\pgfqpoint{2.455594in}{3.458541in}}{\pgfqpoint{2.466193in}{3.454151in}}{\pgfqpoint{2.477243in}{3.454151in}}%
\pgfpathclose%
\pgfusepath{stroke,fill}%
\end{pgfscope}%
\begin{pgfscope}%
\pgfpathrectangle{\pgfqpoint{0.600000in}{0.600000in}}{\pgfqpoint{3.900000in}{3.900000in}}%
\pgfusepath{clip}%
\pgfsetbuttcap%
\pgfsetroundjoin%
\definecolor{currentfill}{rgb}{0.121569,0.466667,0.705882}%
\pgfsetfillcolor{currentfill}%
\pgfsetlinewidth{1.003750pt}%
\definecolor{currentstroke}{rgb}{0.121569,0.466667,0.705882}%
\pgfsetstrokecolor{currentstroke}%
\pgfsetdash{}{0pt}%
\pgfpathmoveto{\pgfqpoint{2.281840in}{2.204875in}}%
\pgfpathcurveto{\pgfqpoint{2.292890in}{2.204875in}}{\pgfqpoint{2.303489in}{2.209266in}}{\pgfqpoint{2.311303in}{2.217079in}}%
\pgfpathcurveto{\pgfqpoint{2.319117in}{2.224893in}}{\pgfqpoint{2.323507in}{2.235492in}}{\pgfqpoint{2.323507in}{2.246542in}}%
\pgfpathcurveto{\pgfqpoint{2.323507in}{2.257592in}}{\pgfqpoint{2.319117in}{2.268191in}}{\pgfqpoint{2.311303in}{2.276005in}}%
\pgfpathcurveto{\pgfqpoint{2.303489in}{2.283818in}}{\pgfqpoint{2.292890in}{2.288209in}}{\pgfqpoint{2.281840in}{2.288209in}}%
\pgfpathcurveto{\pgfqpoint{2.270790in}{2.288209in}}{\pgfqpoint{2.260191in}{2.283818in}}{\pgfqpoint{2.252377in}{2.276005in}}%
\pgfpathcurveto{\pgfqpoint{2.244564in}{2.268191in}}{\pgfqpoint{2.240174in}{2.257592in}}{\pgfqpoint{2.240174in}{2.246542in}}%
\pgfpathcurveto{\pgfqpoint{2.240174in}{2.235492in}}{\pgfqpoint{2.244564in}{2.224893in}}{\pgfqpoint{2.252377in}{2.217079in}}%
\pgfpathcurveto{\pgfqpoint{2.260191in}{2.209266in}}{\pgfqpoint{2.270790in}{2.204875in}}{\pgfqpoint{2.281840in}{2.204875in}}%
\pgfpathclose%
\pgfusepath{stroke,fill}%
\end{pgfscope}%
\begin{pgfscope}%
\pgfpathrectangle{\pgfqpoint{0.600000in}{0.600000in}}{\pgfqpoint{3.900000in}{3.900000in}}%
\pgfusepath{clip}%
\pgfsetbuttcap%
\pgfsetroundjoin%
\definecolor{currentfill}{rgb}{0.121569,0.466667,0.705882}%
\pgfsetfillcolor{currentfill}%
\pgfsetlinewidth{1.003750pt}%
\definecolor{currentstroke}{rgb}{0.121569,0.466667,0.705882}%
\pgfsetstrokecolor{currentstroke}%
\pgfsetdash{}{0pt}%
\pgfpathmoveto{\pgfqpoint{1.932777in}{2.354724in}}%
\pgfpathcurveto{\pgfqpoint{1.943827in}{2.354724in}}{\pgfqpoint{1.954426in}{2.359114in}}{\pgfqpoint{1.962240in}{2.366928in}}%
\pgfpathcurveto{\pgfqpoint{1.970054in}{2.374741in}}{\pgfqpoint{1.974444in}{2.385340in}}{\pgfqpoint{1.974444in}{2.396391in}}%
\pgfpathcurveto{\pgfqpoint{1.974444in}{2.407441in}}{\pgfqpoint{1.970054in}{2.418040in}}{\pgfqpoint{1.962240in}{2.425853in}}%
\pgfpathcurveto{\pgfqpoint{1.954426in}{2.433667in}}{\pgfqpoint{1.943827in}{2.438057in}}{\pgfqpoint{1.932777in}{2.438057in}}%
\pgfpathcurveto{\pgfqpoint{1.921727in}{2.438057in}}{\pgfqpoint{1.911128in}{2.433667in}}{\pgfqpoint{1.903314in}{2.425853in}}%
\pgfpathcurveto{\pgfqpoint{1.895501in}{2.418040in}}{\pgfqpoint{1.891110in}{2.407441in}}{\pgfqpoint{1.891110in}{2.396391in}}%
\pgfpathcurveto{\pgfqpoint{1.891110in}{2.385340in}}{\pgfqpoint{1.895501in}{2.374741in}}{\pgfqpoint{1.903314in}{2.366928in}}%
\pgfpathcurveto{\pgfqpoint{1.911128in}{2.359114in}}{\pgfqpoint{1.921727in}{2.354724in}}{\pgfqpoint{1.932777in}{2.354724in}}%
\pgfpathclose%
\pgfusepath{stroke,fill}%
\end{pgfscope}%
\begin{pgfscope}%
\pgfpathrectangle{\pgfqpoint{0.600000in}{0.600000in}}{\pgfqpoint{3.900000in}{3.900000in}}%
\pgfusepath{clip}%
\pgfsetbuttcap%
\pgfsetroundjoin%
\definecolor{currentfill}{rgb}{0.121569,0.466667,0.705882}%
\pgfsetfillcolor{currentfill}%
\pgfsetlinewidth{1.003750pt}%
\definecolor{currentstroke}{rgb}{0.121569,0.466667,0.705882}%
\pgfsetstrokecolor{currentstroke}%
\pgfsetdash{}{0pt}%
\pgfpathmoveto{\pgfqpoint{2.411532in}{2.579801in}}%
\pgfpathcurveto{\pgfqpoint{2.422582in}{2.579801in}}{\pgfqpoint{2.433181in}{2.584192in}}{\pgfqpoint{2.440995in}{2.592005in}}%
\pgfpathcurveto{\pgfqpoint{2.448808in}{2.599819in}}{\pgfqpoint{2.453199in}{2.610418in}}{\pgfqpoint{2.453199in}{2.621468in}}%
\pgfpathcurveto{\pgfqpoint{2.453199in}{2.632518in}}{\pgfqpoint{2.448808in}{2.643117in}}{\pgfqpoint{2.440995in}{2.650931in}}%
\pgfpathcurveto{\pgfqpoint{2.433181in}{2.658744in}}{\pgfqpoint{2.422582in}{2.663135in}}{\pgfqpoint{2.411532in}{2.663135in}}%
\pgfpathcurveto{\pgfqpoint{2.400482in}{2.663135in}}{\pgfqpoint{2.389883in}{2.658744in}}{\pgfqpoint{2.382069in}{2.650931in}}%
\pgfpathcurveto{\pgfqpoint{2.374255in}{2.643117in}}{\pgfqpoint{2.369865in}{2.632518in}}{\pgfqpoint{2.369865in}{2.621468in}}%
\pgfpathcurveto{\pgfqpoint{2.369865in}{2.610418in}}{\pgfqpoint{2.374255in}{2.599819in}}{\pgfqpoint{2.382069in}{2.592005in}}%
\pgfpathcurveto{\pgfqpoint{2.389883in}{2.584192in}}{\pgfqpoint{2.400482in}{2.579801in}}{\pgfqpoint{2.411532in}{2.579801in}}%
\pgfpathclose%
\pgfusepath{stroke,fill}%
\end{pgfscope}%
\begin{pgfscope}%
\pgfpathrectangle{\pgfqpoint{0.600000in}{0.600000in}}{\pgfqpoint{3.900000in}{3.900000in}}%
\pgfusepath{clip}%
\pgfsetbuttcap%
\pgfsetroundjoin%
\definecolor{currentfill}{rgb}{0.121569,0.466667,0.705882}%
\pgfsetfillcolor{currentfill}%
\pgfsetlinewidth{1.003750pt}%
\definecolor{currentstroke}{rgb}{0.121569,0.466667,0.705882}%
\pgfsetstrokecolor{currentstroke}%
\pgfsetdash{}{0pt}%
\pgfpathmoveto{\pgfqpoint{2.699820in}{2.592387in}}%
\pgfpathcurveto{\pgfqpoint{2.710870in}{2.592387in}}{\pgfqpoint{2.721469in}{2.596777in}}{\pgfqpoint{2.729282in}{2.604591in}}%
\pgfpathcurveto{\pgfqpoint{2.737096in}{2.612404in}}{\pgfqpoint{2.741486in}{2.623003in}}{\pgfqpoint{2.741486in}{2.634054in}}%
\pgfpathcurveto{\pgfqpoint{2.741486in}{2.645104in}}{\pgfqpoint{2.737096in}{2.655703in}}{\pgfqpoint{2.729282in}{2.663516in}}%
\pgfpathcurveto{\pgfqpoint{2.721469in}{2.671330in}}{\pgfqpoint{2.710870in}{2.675720in}}{\pgfqpoint{2.699820in}{2.675720in}}%
\pgfpathcurveto{\pgfqpoint{2.688770in}{2.675720in}}{\pgfqpoint{2.678170in}{2.671330in}}{\pgfqpoint{2.670357in}{2.663516in}}%
\pgfpathcurveto{\pgfqpoint{2.662543in}{2.655703in}}{\pgfqpoint{2.658153in}{2.645104in}}{\pgfqpoint{2.658153in}{2.634054in}}%
\pgfpathcurveto{\pgfqpoint{2.658153in}{2.623003in}}{\pgfqpoint{2.662543in}{2.612404in}}{\pgfqpoint{2.670357in}{2.604591in}}%
\pgfpathcurveto{\pgfqpoint{2.678170in}{2.596777in}}{\pgfqpoint{2.688770in}{2.592387in}}{\pgfqpoint{2.699820in}{2.592387in}}%
\pgfpathclose%
\pgfusepath{stroke,fill}%
\end{pgfscope}%
\begin{pgfscope}%
\pgfpathrectangle{\pgfqpoint{0.600000in}{0.600000in}}{\pgfqpoint{3.900000in}{3.900000in}}%
\pgfusepath{clip}%
\pgfsetbuttcap%
\pgfsetroundjoin%
\definecolor{currentfill}{rgb}{0.121569,0.466667,0.705882}%
\pgfsetfillcolor{currentfill}%
\pgfsetlinewidth{1.003750pt}%
\definecolor{currentstroke}{rgb}{0.121569,0.466667,0.705882}%
\pgfsetstrokecolor{currentstroke}%
\pgfsetdash{}{0pt}%
\pgfpathmoveto{\pgfqpoint{2.054866in}{3.233335in}}%
\pgfpathcurveto{\pgfqpoint{2.065916in}{3.233335in}}{\pgfqpoint{2.076515in}{3.237725in}}{\pgfqpoint{2.084329in}{3.245539in}}%
\pgfpathcurveto{\pgfqpoint{2.092142in}{3.253353in}}{\pgfqpoint{2.096533in}{3.263952in}}{\pgfqpoint{2.096533in}{3.275002in}}%
\pgfpathcurveto{\pgfqpoint{2.096533in}{3.286052in}}{\pgfqpoint{2.092142in}{3.296651in}}{\pgfqpoint{2.084329in}{3.304465in}}%
\pgfpathcurveto{\pgfqpoint{2.076515in}{3.312278in}}{\pgfqpoint{2.065916in}{3.316668in}}{\pgfqpoint{2.054866in}{3.316668in}}%
\pgfpathcurveto{\pgfqpoint{2.043816in}{3.316668in}}{\pgfqpoint{2.033217in}{3.312278in}}{\pgfqpoint{2.025403in}{3.304465in}}%
\pgfpathcurveto{\pgfqpoint{2.017590in}{3.296651in}}{\pgfqpoint{2.013199in}{3.286052in}}{\pgfqpoint{2.013199in}{3.275002in}}%
\pgfpathcurveto{\pgfqpoint{2.013199in}{3.263952in}}{\pgfqpoint{2.017590in}{3.253353in}}{\pgfqpoint{2.025403in}{3.245539in}}%
\pgfpathcurveto{\pgfqpoint{2.033217in}{3.237725in}}{\pgfqpoint{2.043816in}{3.233335in}}{\pgfqpoint{2.054866in}{3.233335in}}%
\pgfpathclose%
\pgfusepath{stroke,fill}%
\end{pgfscope}%
\begin{pgfscope}%
\pgfpathrectangle{\pgfqpoint{0.600000in}{0.600000in}}{\pgfqpoint{3.900000in}{3.900000in}}%
\pgfusepath{clip}%
\pgfsetbuttcap%
\pgfsetroundjoin%
\definecolor{currentfill}{rgb}{0.121569,0.466667,0.705882}%
\pgfsetfillcolor{currentfill}%
\pgfsetlinewidth{1.003750pt}%
\definecolor{currentstroke}{rgb}{0.121569,0.466667,0.705882}%
\pgfsetstrokecolor{currentstroke}%
\pgfsetdash{}{0pt}%
\pgfpathmoveto{\pgfqpoint{2.790701in}{2.961071in}}%
\pgfpathcurveto{\pgfqpoint{2.801751in}{2.961071in}}{\pgfqpoint{2.812350in}{2.965461in}}{\pgfqpoint{2.820164in}{2.973274in}}%
\pgfpathcurveto{\pgfqpoint{2.827978in}{2.981088in}}{\pgfqpoint{2.832368in}{2.991687in}}{\pgfqpoint{2.832368in}{3.002737in}}%
\pgfpathcurveto{\pgfqpoint{2.832368in}{3.013787in}}{\pgfqpoint{2.827978in}{3.024386in}}{\pgfqpoint{2.820164in}{3.032200in}}%
\pgfpathcurveto{\pgfqpoint{2.812350in}{3.040014in}}{\pgfqpoint{2.801751in}{3.044404in}}{\pgfqpoint{2.790701in}{3.044404in}}%
\pgfpathcurveto{\pgfqpoint{2.779651in}{3.044404in}}{\pgfqpoint{2.769052in}{3.040014in}}{\pgfqpoint{2.761239in}{3.032200in}}%
\pgfpathcurveto{\pgfqpoint{2.753425in}{3.024386in}}{\pgfqpoint{2.749035in}{3.013787in}}{\pgfqpoint{2.749035in}{3.002737in}}%
\pgfpathcurveto{\pgfqpoint{2.749035in}{2.991687in}}{\pgfqpoint{2.753425in}{2.981088in}}{\pgfqpoint{2.761239in}{2.973274in}}%
\pgfpathcurveto{\pgfqpoint{2.769052in}{2.965461in}}{\pgfqpoint{2.779651in}{2.961071in}}{\pgfqpoint{2.790701in}{2.961071in}}%
\pgfpathclose%
\pgfusepath{stroke,fill}%
\end{pgfscope}%
\begin{pgfscope}%
\pgfpathrectangle{\pgfqpoint{0.600000in}{0.600000in}}{\pgfqpoint{3.900000in}{3.900000in}}%
\pgfusepath{clip}%
\pgfsetbuttcap%
\pgfsetroundjoin%
\definecolor{currentfill}{rgb}{0.121569,0.466667,0.705882}%
\pgfsetfillcolor{currentfill}%
\pgfsetlinewidth{1.003750pt}%
\definecolor{currentstroke}{rgb}{0.121569,0.466667,0.705882}%
\pgfsetstrokecolor{currentstroke}%
\pgfsetdash{}{0pt}%
\pgfpathmoveto{\pgfqpoint{2.600058in}{2.661263in}}%
\pgfpathcurveto{\pgfqpoint{2.611109in}{2.661263in}}{\pgfqpoint{2.621708in}{2.665653in}}{\pgfqpoint{2.629521in}{2.673467in}}%
\pgfpathcurveto{\pgfqpoint{2.637335in}{2.681280in}}{\pgfqpoint{2.641725in}{2.691879in}}{\pgfqpoint{2.641725in}{2.702929in}}%
\pgfpathcurveto{\pgfqpoint{2.641725in}{2.713980in}}{\pgfqpoint{2.637335in}{2.724579in}}{\pgfqpoint{2.629521in}{2.732392in}}%
\pgfpathcurveto{\pgfqpoint{2.621708in}{2.740206in}}{\pgfqpoint{2.611109in}{2.744596in}}{\pgfqpoint{2.600058in}{2.744596in}}%
\pgfpathcurveto{\pgfqpoint{2.589008in}{2.744596in}}{\pgfqpoint{2.578409in}{2.740206in}}{\pgfqpoint{2.570596in}{2.732392in}}%
\pgfpathcurveto{\pgfqpoint{2.562782in}{2.724579in}}{\pgfqpoint{2.558392in}{2.713980in}}{\pgfqpoint{2.558392in}{2.702929in}}%
\pgfpathcurveto{\pgfqpoint{2.558392in}{2.691879in}}{\pgfqpoint{2.562782in}{2.681280in}}{\pgfqpoint{2.570596in}{2.673467in}}%
\pgfpathcurveto{\pgfqpoint{2.578409in}{2.665653in}}{\pgfqpoint{2.589008in}{2.661263in}}{\pgfqpoint{2.600058in}{2.661263in}}%
\pgfpathclose%
\pgfusepath{stroke,fill}%
\end{pgfscope}%
\begin{pgfscope}%
\pgfpathrectangle{\pgfqpoint{0.600000in}{0.600000in}}{\pgfqpoint{3.900000in}{3.900000in}}%
\pgfusepath{clip}%
\pgfsetbuttcap%
\pgfsetroundjoin%
\definecolor{currentfill}{rgb}{0.121569,0.466667,0.705882}%
\pgfsetfillcolor{currentfill}%
\pgfsetlinewidth{1.003750pt}%
\definecolor{currentstroke}{rgb}{0.121569,0.466667,0.705882}%
\pgfsetstrokecolor{currentstroke}%
\pgfsetdash{}{0pt}%
\pgfpathmoveto{\pgfqpoint{1.930639in}{3.443591in}}%
\pgfpathcurveto{\pgfqpoint{1.941689in}{3.443591in}}{\pgfqpoint{1.952288in}{3.447981in}}{\pgfqpoint{1.960102in}{3.455795in}}%
\pgfpathcurveto{\pgfqpoint{1.967915in}{3.463608in}}{\pgfqpoint{1.972306in}{3.474207in}}{\pgfqpoint{1.972306in}{3.485257in}}%
\pgfpathcurveto{\pgfqpoint{1.972306in}{3.496308in}}{\pgfqpoint{1.967915in}{3.506907in}}{\pgfqpoint{1.960102in}{3.514720in}}%
\pgfpathcurveto{\pgfqpoint{1.952288in}{3.522534in}}{\pgfqpoint{1.941689in}{3.526924in}}{\pgfqpoint{1.930639in}{3.526924in}}%
\pgfpathcurveto{\pgfqpoint{1.919589in}{3.526924in}}{\pgfqpoint{1.908990in}{3.522534in}}{\pgfqpoint{1.901176in}{3.514720in}}%
\pgfpathcurveto{\pgfqpoint{1.893362in}{3.506907in}}{\pgfqpoint{1.888972in}{3.496308in}}{\pgfqpoint{1.888972in}{3.485257in}}%
\pgfpathcurveto{\pgfqpoint{1.888972in}{3.474207in}}{\pgfqpoint{1.893362in}{3.463608in}}{\pgfqpoint{1.901176in}{3.455795in}}%
\pgfpathcurveto{\pgfqpoint{1.908990in}{3.447981in}}{\pgfqpoint{1.919589in}{3.443591in}}{\pgfqpoint{1.930639in}{3.443591in}}%
\pgfpathclose%
\pgfusepath{stroke,fill}%
\end{pgfscope}%
\begin{pgfscope}%
\pgfpathrectangle{\pgfqpoint{0.600000in}{0.600000in}}{\pgfqpoint{3.900000in}{3.900000in}}%
\pgfusepath{clip}%
\pgfsetbuttcap%
\pgfsetroundjoin%
\definecolor{currentfill}{rgb}{0.121569,0.466667,0.705882}%
\pgfsetfillcolor{currentfill}%
\pgfsetlinewidth{1.003750pt}%
\definecolor{currentstroke}{rgb}{0.121569,0.466667,0.705882}%
\pgfsetstrokecolor{currentstroke}%
\pgfsetdash{}{0pt}%
\pgfpathmoveto{\pgfqpoint{2.804968in}{2.803316in}}%
\pgfpathcurveto{\pgfqpoint{2.816018in}{2.803316in}}{\pgfqpoint{2.826617in}{2.807707in}}{\pgfqpoint{2.834430in}{2.815520in}}%
\pgfpathcurveto{\pgfqpoint{2.842244in}{2.823334in}}{\pgfqpoint{2.846634in}{2.833933in}}{\pgfqpoint{2.846634in}{2.844983in}}%
\pgfpathcurveto{\pgfqpoint{2.846634in}{2.856033in}}{\pgfqpoint{2.842244in}{2.866632in}}{\pgfqpoint{2.834430in}{2.874446in}}%
\pgfpathcurveto{\pgfqpoint{2.826617in}{2.882259in}}{\pgfqpoint{2.816018in}{2.886650in}}{\pgfqpoint{2.804968in}{2.886650in}}%
\pgfpathcurveto{\pgfqpoint{2.793918in}{2.886650in}}{\pgfqpoint{2.783319in}{2.882259in}}{\pgfqpoint{2.775505in}{2.874446in}}%
\pgfpathcurveto{\pgfqpoint{2.767691in}{2.866632in}}{\pgfqpoint{2.763301in}{2.856033in}}{\pgfqpoint{2.763301in}{2.844983in}}%
\pgfpathcurveto{\pgfqpoint{2.763301in}{2.833933in}}{\pgfqpoint{2.767691in}{2.823334in}}{\pgfqpoint{2.775505in}{2.815520in}}%
\pgfpathcurveto{\pgfqpoint{2.783319in}{2.807707in}}{\pgfqpoint{2.793918in}{2.803316in}}{\pgfqpoint{2.804968in}{2.803316in}}%
\pgfpathclose%
\pgfusepath{stroke,fill}%
\end{pgfscope}%
\begin{pgfscope}%
\pgfpathrectangle{\pgfqpoint{0.600000in}{0.600000in}}{\pgfqpoint{3.900000in}{3.900000in}}%
\pgfusepath{clip}%
\pgfsetbuttcap%
\pgfsetroundjoin%
\definecolor{currentfill}{rgb}{0.121569,0.466667,0.705882}%
\pgfsetfillcolor{currentfill}%
\pgfsetlinewidth{1.003750pt}%
\definecolor{currentstroke}{rgb}{0.121569,0.466667,0.705882}%
\pgfsetstrokecolor{currentstroke}%
\pgfsetdash{}{0pt}%
\pgfpathmoveto{\pgfqpoint{2.136437in}{2.980437in}}%
\pgfpathcurveto{\pgfqpoint{2.147488in}{2.980437in}}{\pgfqpoint{2.158087in}{2.984827in}}{\pgfqpoint{2.165900in}{2.992640in}}%
\pgfpathcurveto{\pgfqpoint{2.173714in}{3.000454in}}{\pgfqpoint{2.178104in}{3.011053in}}{\pgfqpoint{2.178104in}{3.022103in}}%
\pgfpathcurveto{\pgfqpoint{2.178104in}{3.033153in}}{\pgfqpoint{2.173714in}{3.043752in}}{\pgfqpoint{2.165900in}{3.051566in}}%
\pgfpathcurveto{\pgfqpoint{2.158087in}{3.059380in}}{\pgfqpoint{2.147488in}{3.063770in}}{\pgfqpoint{2.136437in}{3.063770in}}%
\pgfpathcurveto{\pgfqpoint{2.125387in}{3.063770in}}{\pgfqpoint{2.114788in}{3.059380in}}{\pgfqpoint{2.106975in}{3.051566in}}%
\pgfpathcurveto{\pgfqpoint{2.099161in}{3.043752in}}{\pgfqpoint{2.094771in}{3.033153in}}{\pgfqpoint{2.094771in}{3.022103in}}%
\pgfpathcurveto{\pgfqpoint{2.094771in}{3.011053in}}{\pgfqpoint{2.099161in}{3.000454in}}{\pgfqpoint{2.106975in}{2.992640in}}%
\pgfpathcurveto{\pgfqpoint{2.114788in}{2.984827in}}{\pgfqpoint{2.125387in}{2.980437in}}{\pgfqpoint{2.136437in}{2.980437in}}%
\pgfpathclose%
\pgfusepath{stroke,fill}%
\end{pgfscope}%
\begin{pgfscope}%
\pgfpathrectangle{\pgfqpoint{0.600000in}{0.600000in}}{\pgfqpoint{3.900000in}{3.900000in}}%
\pgfusepath{clip}%
\pgfsetbuttcap%
\pgfsetroundjoin%
\definecolor{currentfill}{rgb}{0.121569,0.466667,0.705882}%
\pgfsetfillcolor{currentfill}%
\pgfsetlinewidth{1.003750pt}%
\definecolor{currentstroke}{rgb}{0.121569,0.466667,0.705882}%
\pgfsetstrokecolor{currentstroke}%
\pgfsetdash{}{0pt}%
\pgfpathmoveto{\pgfqpoint{2.212637in}{1.447243in}}%
\pgfpathcurveto{\pgfqpoint{2.223687in}{1.447243in}}{\pgfqpoint{2.234286in}{1.451633in}}{\pgfqpoint{2.242100in}{1.459447in}}%
\pgfpathcurveto{\pgfqpoint{2.249914in}{1.467260in}}{\pgfqpoint{2.254304in}{1.477859in}}{\pgfqpoint{2.254304in}{1.488910in}}%
\pgfpathcurveto{\pgfqpoint{2.254304in}{1.499960in}}{\pgfqpoint{2.249914in}{1.510559in}}{\pgfqpoint{2.242100in}{1.518372in}}%
\pgfpathcurveto{\pgfqpoint{2.234286in}{1.526186in}}{\pgfqpoint{2.223687in}{1.530576in}}{\pgfqpoint{2.212637in}{1.530576in}}%
\pgfpathcurveto{\pgfqpoint{2.201587in}{1.530576in}}{\pgfqpoint{2.190988in}{1.526186in}}{\pgfqpoint{2.183175in}{1.518372in}}%
\pgfpathcurveto{\pgfqpoint{2.175361in}{1.510559in}}{\pgfqpoint{2.170971in}{1.499960in}}{\pgfqpoint{2.170971in}{1.488910in}}%
\pgfpathcurveto{\pgfqpoint{2.170971in}{1.477859in}}{\pgfqpoint{2.175361in}{1.467260in}}{\pgfqpoint{2.183175in}{1.459447in}}%
\pgfpathcurveto{\pgfqpoint{2.190988in}{1.451633in}}{\pgfqpoint{2.201587in}{1.447243in}}{\pgfqpoint{2.212637in}{1.447243in}}%
\pgfpathclose%
\pgfusepath{stroke,fill}%
\end{pgfscope}%
\begin{pgfscope}%
\pgfpathrectangle{\pgfqpoint{0.600000in}{0.600000in}}{\pgfqpoint{3.900000in}{3.900000in}}%
\pgfusepath{clip}%
\pgfsetbuttcap%
\pgfsetroundjoin%
\definecolor{currentfill}{rgb}{0.121569,0.466667,0.705882}%
\pgfsetfillcolor{currentfill}%
\pgfsetlinewidth{1.003750pt}%
\definecolor{currentstroke}{rgb}{0.121569,0.466667,0.705882}%
\pgfsetstrokecolor{currentstroke}%
\pgfsetdash{}{0pt}%
\pgfpathmoveto{\pgfqpoint{3.323163in}{1.602290in}}%
\pgfpathcurveto{\pgfqpoint{3.334213in}{1.602290in}}{\pgfqpoint{3.344812in}{1.606680in}}{\pgfqpoint{3.352626in}{1.614494in}}%
\pgfpathcurveto{\pgfqpoint{3.360440in}{1.622307in}}{\pgfqpoint{3.364830in}{1.632906in}}{\pgfqpoint{3.364830in}{1.643956in}}%
\pgfpathcurveto{\pgfqpoint{3.364830in}{1.655007in}}{\pgfqpoint{3.360440in}{1.665606in}}{\pgfqpoint{3.352626in}{1.673419in}}%
\pgfpathcurveto{\pgfqpoint{3.344812in}{1.681233in}}{\pgfqpoint{3.334213in}{1.685623in}}{\pgfqpoint{3.323163in}{1.685623in}}%
\pgfpathcurveto{\pgfqpoint{3.312113in}{1.685623in}}{\pgfqpoint{3.301514in}{1.681233in}}{\pgfqpoint{3.293700in}{1.673419in}}%
\pgfpathcurveto{\pgfqpoint{3.285887in}{1.665606in}}{\pgfqpoint{3.281497in}{1.655007in}}{\pgfqpoint{3.281497in}{1.643956in}}%
\pgfpathcurveto{\pgfqpoint{3.281497in}{1.632906in}}{\pgfqpoint{3.285887in}{1.622307in}}{\pgfqpoint{3.293700in}{1.614494in}}%
\pgfpathcurveto{\pgfqpoint{3.301514in}{1.606680in}}{\pgfqpoint{3.312113in}{1.602290in}}{\pgfqpoint{3.323163in}{1.602290in}}%
\pgfpathclose%
\pgfusepath{stroke,fill}%
\end{pgfscope}%
\begin{pgfscope}%
\pgfpathrectangle{\pgfqpoint{0.600000in}{0.600000in}}{\pgfqpoint{3.900000in}{3.900000in}}%
\pgfusepath{clip}%
\pgfsetbuttcap%
\pgfsetroundjoin%
\definecolor{currentfill}{rgb}{0.121569,0.466667,0.705882}%
\pgfsetfillcolor{currentfill}%
\pgfsetlinewidth{1.003750pt}%
\definecolor{currentstroke}{rgb}{0.121569,0.466667,0.705882}%
\pgfsetstrokecolor{currentstroke}%
\pgfsetdash{}{0pt}%
\pgfpathmoveto{\pgfqpoint{2.562518in}{1.826170in}}%
\pgfpathcurveto{\pgfqpoint{2.573568in}{1.826170in}}{\pgfqpoint{2.584167in}{1.830560in}}{\pgfqpoint{2.591981in}{1.838374in}}%
\pgfpathcurveto{\pgfqpoint{2.599795in}{1.846187in}}{\pgfqpoint{2.604185in}{1.856787in}}{\pgfqpoint{2.604185in}{1.867837in}}%
\pgfpathcurveto{\pgfqpoint{2.604185in}{1.878887in}}{\pgfqpoint{2.599795in}{1.889486in}}{\pgfqpoint{2.591981in}{1.897299in}}%
\pgfpathcurveto{\pgfqpoint{2.584167in}{1.905113in}}{\pgfqpoint{2.573568in}{1.909503in}}{\pgfqpoint{2.562518in}{1.909503in}}%
\pgfpathcurveto{\pgfqpoint{2.551468in}{1.909503in}}{\pgfqpoint{2.540869in}{1.905113in}}{\pgfqpoint{2.533055in}{1.897299in}}%
\pgfpathcurveto{\pgfqpoint{2.525242in}{1.889486in}}{\pgfqpoint{2.520851in}{1.878887in}}{\pgfqpoint{2.520851in}{1.867837in}}%
\pgfpathcurveto{\pgfqpoint{2.520851in}{1.856787in}}{\pgfqpoint{2.525242in}{1.846187in}}{\pgfqpoint{2.533055in}{1.838374in}}%
\pgfpathcurveto{\pgfqpoint{2.540869in}{1.830560in}}{\pgfqpoint{2.551468in}{1.826170in}}{\pgfqpoint{2.562518in}{1.826170in}}%
\pgfpathclose%
\pgfusepath{stroke,fill}%
\end{pgfscope}%
\begin{pgfscope}%
\pgfpathrectangle{\pgfqpoint{0.600000in}{0.600000in}}{\pgfqpoint{3.900000in}{3.900000in}}%
\pgfusepath{clip}%
\pgfsetbuttcap%
\pgfsetroundjoin%
\definecolor{currentfill}{rgb}{0.121569,0.466667,0.705882}%
\pgfsetfillcolor{currentfill}%
\pgfsetlinewidth{1.003750pt}%
\definecolor{currentstroke}{rgb}{0.121569,0.466667,0.705882}%
\pgfsetstrokecolor{currentstroke}%
\pgfsetdash{}{0pt}%
\pgfpathmoveto{\pgfqpoint{2.184456in}{1.938770in}}%
\pgfpathcurveto{\pgfqpoint{2.195506in}{1.938770in}}{\pgfqpoint{2.206105in}{1.943160in}}{\pgfqpoint{2.213919in}{1.950973in}}%
\pgfpathcurveto{\pgfqpoint{2.221733in}{1.958787in}}{\pgfqpoint{2.226123in}{1.969386in}}{\pgfqpoint{2.226123in}{1.980436in}}%
\pgfpathcurveto{\pgfqpoint{2.226123in}{1.991486in}}{\pgfqpoint{2.221733in}{2.002085in}}{\pgfqpoint{2.213919in}{2.009899in}}%
\pgfpathcurveto{\pgfqpoint{2.206105in}{2.017713in}}{\pgfqpoint{2.195506in}{2.022103in}}{\pgfqpoint{2.184456in}{2.022103in}}%
\pgfpathcurveto{\pgfqpoint{2.173406in}{2.022103in}}{\pgfqpoint{2.162807in}{2.017713in}}{\pgfqpoint{2.154993in}{2.009899in}}%
\pgfpathcurveto{\pgfqpoint{2.147180in}{2.002085in}}{\pgfqpoint{2.142790in}{1.991486in}}{\pgfqpoint{2.142790in}{1.980436in}}%
\pgfpathcurveto{\pgfqpoint{2.142790in}{1.969386in}}{\pgfqpoint{2.147180in}{1.958787in}}{\pgfqpoint{2.154993in}{1.950973in}}%
\pgfpathcurveto{\pgfqpoint{2.162807in}{1.943160in}}{\pgfqpoint{2.173406in}{1.938770in}}{\pgfqpoint{2.184456in}{1.938770in}}%
\pgfpathclose%
\pgfusepath{stroke,fill}%
\end{pgfscope}%
\begin{pgfscope}%
\pgfpathrectangle{\pgfqpoint{0.600000in}{0.600000in}}{\pgfqpoint{3.900000in}{3.900000in}}%
\pgfusepath{clip}%
\pgfsetbuttcap%
\pgfsetroundjoin%
\definecolor{currentfill}{rgb}{0.121569,0.466667,0.705882}%
\pgfsetfillcolor{currentfill}%
\pgfsetlinewidth{1.003750pt}%
\definecolor{currentstroke}{rgb}{0.121569,0.466667,0.705882}%
\pgfsetstrokecolor{currentstroke}%
\pgfsetdash{}{0pt}%
\pgfpathmoveto{\pgfqpoint{1.903265in}{1.833376in}}%
\pgfpathcurveto{\pgfqpoint{1.914316in}{1.833376in}}{\pgfqpoint{1.924915in}{1.837766in}}{\pgfqpoint{1.932728in}{1.845580in}}%
\pgfpathcurveto{\pgfqpoint{1.940542in}{1.853394in}}{\pgfqpoint{1.944932in}{1.863993in}}{\pgfqpoint{1.944932in}{1.875043in}}%
\pgfpathcurveto{\pgfqpoint{1.944932in}{1.886093in}}{\pgfqpoint{1.940542in}{1.896692in}}{\pgfqpoint{1.932728in}{1.904505in}}%
\pgfpathcurveto{\pgfqpoint{1.924915in}{1.912319in}}{\pgfqpoint{1.914316in}{1.916709in}}{\pgfqpoint{1.903265in}{1.916709in}}%
\pgfpathcurveto{\pgfqpoint{1.892215in}{1.916709in}}{\pgfqpoint{1.881616in}{1.912319in}}{\pgfqpoint{1.873803in}{1.904505in}}%
\pgfpathcurveto{\pgfqpoint{1.865989in}{1.896692in}}{\pgfqpoint{1.861599in}{1.886093in}}{\pgfqpoint{1.861599in}{1.875043in}}%
\pgfpathcurveto{\pgfqpoint{1.861599in}{1.863993in}}{\pgfqpoint{1.865989in}{1.853394in}}{\pgfqpoint{1.873803in}{1.845580in}}%
\pgfpathcurveto{\pgfqpoint{1.881616in}{1.837766in}}{\pgfqpoint{1.892215in}{1.833376in}}{\pgfqpoint{1.903265in}{1.833376in}}%
\pgfpathclose%
\pgfusepath{stroke,fill}%
\end{pgfscope}%
\begin{pgfscope}%
\pgfpathrectangle{\pgfqpoint{0.600000in}{0.600000in}}{\pgfqpoint{3.900000in}{3.900000in}}%
\pgfusepath{clip}%
\pgfsetbuttcap%
\pgfsetroundjoin%
\definecolor{currentfill}{rgb}{0.121569,0.466667,0.705882}%
\pgfsetfillcolor{currentfill}%
\pgfsetlinewidth{1.003750pt}%
\definecolor{currentstroke}{rgb}{0.121569,0.466667,0.705882}%
\pgfsetstrokecolor{currentstroke}%
\pgfsetdash{}{0pt}%
\pgfpathmoveto{\pgfqpoint{2.473761in}{2.737990in}}%
\pgfpathcurveto{\pgfqpoint{2.484811in}{2.737990in}}{\pgfqpoint{2.495410in}{2.742381in}}{\pgfqpoint{2.503224in}{2.750194in}}%
\pgfpathcurveto{\pgfqpoint{2.511037in}{2.758008in}}{\pgfqpoint{2.515428in}{2.768607in}}{\pgfqpoint{2.515428in}{2.779657in}}%
\pgfpathcurveto{\pgfqpoint{2.515428in}{2.790707in}}{\pgfqpoint{2.511037in}{2.801306in}}{\pgfqpoint{2.503224in}{2.809120in}}%
\pgfpathcurveto{\pgfqpoint{2.495410in}{2.816933in}}{\pgfqpoint{2.484811in}{2.821324in}}{\pgfqpoint{2.473761in}{2.821324in}}%
\pgfpathcurveto{\pgfqpoint{2.462711in}{2.821324in}}{\pgfqpoint{2.452112in}{2.816933in}}{\pgfqpoint{2.444298in}{2.809120in}}%
\pgfpathcurveto{\pgfqpoint{2.436485in}{2.801306in}}{\pgfqpoint{2.432094in}{2.790707in}}{\pgfqpoint{2.432094in}{2.779657in}}%
\pgfpathcurveto{\pgfqpoint{2.432094in}{2.768607in}}{\pgfqpoint{2.436485in}{2.758008in}}{\pgfqpoint{2.444298in}{2.750194in}}%
\pgfpathcurveto{\pgfqpoint{2.452112in}{2.742381in}}{\pgfqpoint{2.462711in}{2.737990in}}{\pgfqpoint{2.473761in}{2.737990in}}%
\pgfpathclose%
\pgfusepath{stroke,fill}%
\end{pgfscope}%
\begin{pgfscope}%
\pgfpathrectangle{\pgfqpoint{0.600000in}{0.600000in}}{\pgfqpoint{3.900000in}{3.900000in}}%
\pgfusepath{clip}%
\pgfsetbuttcap%
\pgfsetroundjoin%
\definecolor{currentfill}{rgb}{0.121569,0.466667,0.705882}%
\pgfsetfillcolor{currentfill}%
\pgfsetlinewidth{1.003750pt}%
\definecolor{currentstroke}{rgb}{0.121569,0.466667,0.705882}%
\pgfsetstrokecolor{currentstroke}%
\pgfsetdash{}{0pt}%
\pgfpathmoveto{\pgfqpoint{2.059194in}{2.291653in}}%
\pgfpathcurveto{\pgfqpoint{2.070244in}{2.291653in}}{\pgfqpoint{2.080843in}{2.296043in}}{\pgfqpoint{2.088657in}{2.303857in}}%
\pgfpathcurveto{\pgfqpoint{2.096470in}{2.311671in}}{\pgfqpoint{2.100861in}{2.322270in}}{\pgfqpoint{2.100861in}{2.333320in}}%
\pgfpathcurveto{\pgfqpoint{2.100861in}{2.344370in}}{\pgfqpoint{2.096470in}{2.354969in}}{\pgfqpoint{2.088657in}{2.362783in}}%
\pgfpathcurveto{\pgfqpoint{2.080843in}{2.370596in}}{\pgfqpoint{2.070244in}{2.374986in}}{\pgfqpoint{2.059194in}{2.374986in}}%
\pgfpathcurveto{\pgfqpoint{2.048144in}{2.374986in}}{\pgfqpoint{2.037545in}{2.370596in}}{\pgfqpoint{2.029731in}{2.362783in}}%
\pgfpathcurveto{\pgfqpoint{2.021918in}{2.354969in}}{\pgfqpoint{2.017527in}{2.344370in}}{\pgfqpoint{2.017527in}{2.333320in}}%
\pgfpathcurveto{\pgfqpoint{2.017527in}{2.322270in}}{\pgfqpoint{2.021918in}{2.311671in}}{\pgfqpoint{2.029731in}{2.303857in}}%
\pgfpathcurveto{\pgfqpoint{2.037545in}{2.296043in}}{\pgfqpoint{2.048144in}{2.291653in}}{\pgfqpoint{2.059194in}{2.291653in}}%
\pgfpathclose%
\pgfusepath{stroke,fill}%
\end{pgfscope}%
\begin{pgfscope}%
\pgfpathrectangle{\pgfqpoint{0.600000in}{0.600000in}}{\pgfqpoint{3.900000in}{3.900000in}}%
\pgfusepath{clip}%
\pgfsetbuttcap%
\pgfsetroundjoin%
\definecolor{currentfill}{rgb}{0.121569,0.466667,0.705882}%
\pgfsetfillcolor{currentfill}%
\pgfsetlinewidth{1.003750pt}%
\definecolor{currentstroke}{rgb}{0.121569,0.466667,0.705882}%
\pgfsetstrokecolor{currentstroke}%
\pgfsetdash{}{0pt}%
\pgfpathmoveto{\pgfqpoint{1.672960in}{2.635053in}}%
\pgfpathcurveto{\pgfqpoint{1.684010in}{2.635053in}}{\pgfqpoint{1.694609in}{2.639444in}}{\pgfqpoint{1.702422in}{2.647257in}}%
\pgfpathcurveto{\pgfqpoint{1.710236in}{2.655071in}}{\pgfqpoint{1.714626in}{2.665670in}}{\pgfqpoint{1.714626in}{2.676720in}}%
\pgfpathcurveto{\pgfqpoint{1.714626in}{2.687770in}}{\pgfqpoint{1.710236in}{2.698369in}}{\pgfqpoint{1.702422in}{2.706183in}}%
\pgfpathcurveto{\pgfqpoint{1.694609in}{2.713996in}}{\pgfqpoint{1.684010in}{2.718387in}}{\pgfqpoint{1.672960in}{2.718387in}}%
\pgfpathcurveto{\pgfqpoint{1.661909in}{2.718387in}}{\pgfqpoint{1.651310in}{2.713996in}}{\pgfqpoint{1.643497in}{2.706183in}}%
\pgfpathcurveto{\pgfqpoint{1.635683in}{2.698369in}}{\pgfqpoint{1.631293in}{2.687770in}}{\pgfqpoint{1.631293in}{2.676720in}}%
\pgfpathcurveto{\pgfqpoint{1.631293in}{2.665670in}}{\pgfqpoint{1.635683in}{2.655071in}}{\pgfqpoint{1.643497in}{2.647257in}}%
\pgfpathcurveto{\pgfqpoint{1.651310in}{2.639444in}}{\pgfqpoint{1.661909in}{2.635053in}}{\pgfqpoint{1.672960in}{2.635053in}}%
\pgfpathclose%
\pgfusepath{stroke,fill}%
\end{pgfscope}%
\begin{pgfscope}%
\pgfpathrectangle{\pgfqpoint{0.600000in}{0.600000in}}{\pgfqpoint{3.900000in}{3.900000in}}%
\pgfusepath{clip}%
\pgfsetbuttcap%
\pgfsetroundjoin%
\definecolor{currentfill}{rgb}{0.121569,0.466667,0.705882}%
\pgfsetfillcolor{currentfill}%
\pgfsetlinewidth{1.003750pt}%
\definecolor{currentstroke}{rgb}{0.121569,0.466667,0.705882}%
\pgfsetstrokecolor{currentstroke}%
\pgfsetdash{}{0pt}%
\pgfpathmoveto{\pgfqpoint{2.032959in}{3.518085in}}%
\pgfpathcurveto{\pgfqpoint{2.044009in}{3.518085in}}{\pgfqpoint{2.054608in}{3.522476in}}{\pgfqpoint{2.062422in}{3.530289in}}%
\pgfpathcurveto{\pgfqpoint{2.070235in}{3.538103in}}{\pgfqpoint{2.074626in}{3.548702in}}{\pgfqpoint{2.074626in}{3.559752in}}%
\pgfpathcurveto{\pgfqpoint{2.074626in}{3.570802in}}{\pgfqpoint{2.070235in}{3.581401in}}{\pgfqpoint{2.062422in}{3.589215in}}%
\pgfpathcurveto{\pgfqpoint{2.054608in}{3.597028in}}{\pgfqpoint{2.044009in}{3.601419in}}{\pgfqpoint{2.032959in}{3.601419in}}%
\pgfpathcurveto{\pgfqpoint{2.021909in}{3.601419in}}{\pgfqpoint{2.011310in}{3.597028in}}{\pgfqpoint{2.003496in}{3.589215in}}%
\pgfpathcurveto{\pgfqpoint{1.995683in}{3.581401in}}{\pgfqpoint{1.991292in}{3.570802in}}{\pgfqpoint{1.991292in}{3.559752in}}%
\pgfpathcurveto{\pgfqpoint{1.991292in}{3.548702in}}{\pgfqpoint{1.995683in}{3.538103in}}{\pgfqpoint{2.003496in}{3.530289in}}%
\pgfpathcurveto{\pgfqpoint{2.011310in}{3.522476in}}{\pgfqpoint{2.021909in}{3.518085in}}{\pgfqpoint{2.032959in}{3.518085in}}%
\pgfpathclose%
\pgfusepath{stroke,fill}%
\end{pgfscope}%
\begin{pgfscope}%
\pgfpathrectangle{\pgfqpoint{0.600000in}{0.600000in}}{\pgfqpoint{3.900000in}{3.900000in}}%
\pgfusepath{clip}%
\pgfsetbuttcap%
\pgfsetroundjoin%
\definecolor{currentfill}{rgb}{0.121569,0.466667,0.705882}%
\pgfsetfillcolor{currentfill}%
\pgfsetlinewidth{1.003750pt}%
\definecolor{currentstroke}{rgb}{0.121569,0.466667,0.705882}%
\pgfsetstrokecolor{currentstroke}%
\pgfsetdash{}{0pt}%
\pgfpathmoveto{\pgfqpoint{2.894840in}{2.944614in}}%
\pgfpathcurveto{\pgfqpoint{2.905890in}{2.944614in}}{\pgfqpoint{2.916489in}{2.949005in}}{\pgfqpoint{2.924303in}{2.956818in}}%
\pgfpathcurveto{\pgfqpoint{2.932117in}{2.964632in}}{\pgfqpoint{2.936507in}{2.975231in}}{\pgfqpoint{2.936507in}{2.986281in}}%
\pgfpathcurveto{\pgfqpoint{2.936507in}{2.997331in}}{\pgfqpoint{2.932117in}{3.007930in}}{\pgfqpoint{2.924303in}{3.015744in}}%
\pgfpathcurveto{\pgfqpoint{2.916489in}{3.023557in}}{\pgfqpoint{2.905890in}{3.027948in}}{\pgfqpoint{2.894840in}{3.027948in}}%
\pgfpathcurveto{\pgfqpoint{2.883790in}{3.027948in}}{\pgfqpoint{2.873191in}{3.023557in}}{\pgfqpoint{2.865377in}{3.015744in}}%
\pgfpathcurveto{\pgfqpoint{2.857564in}{3.007930in}}{\pgfqpoint{2.853173in}{2.997331in}}{\pgfqpoint{2.853173in}{2.986281in}}%
\pgfpathcurveto{\pgfqpoint{2.853173in}{2.975231in}}{\pgfqpoint{2.857564in}{2.964632in}}{\pgfqpoint{2.865377in}{2.956818in}}%
\pgfpathcurveto{\pgfqpoint{2.873191in}{2.949005in}}{\pgfqpoint{2.883790in}{2.944614in}}{\pgfqpoint{2.894840in}{2.944614in}}%
\pgfpathclose%
\pgfusepath{stroke,fill}%
\end{pgfscope}%
\begin{pgfscope}%
\pgfpathrectangle{\pgfqpoint{0.600000in}{0.600000in}}{\pgfqpoint{3.900000in}{3.900000in}}%
\pgfusepath{clip}%
\pgfsetbuttcap%
\pgfsetroundjoin%
\definecolor{currentfill}{rgb}{0.121569,0.466667,0.705882}%
\pgfsetfillcolor{currentfill}%
\pgfsetlinewidth{1.003750pt}%
\definecolor{currentstroke}{rgb}{0.121569,0.466667,0.705882}%
\pgfsetstrokecolor{currentstroke}%
\pgfsetdash{}{0pt}%
\pgfpathmoveto{\pgfqpoint{3.090131in}{2.950147in}}%
\pgfpathcurveto{\pgfqpoint{3.101181in}{2.950147in}}{\pgfqpoint{3.111780in}{2.954537in}}{\pgfqpoint{3.119594in}{2.962351in}}%
\pgfpathcurveto{\pgfqpoint{3.127408in}{2.970165in}}{\pgfqpoint{3.131798in}{2.980764in}}{\pgfqpoint{3.131798in}{2.991814in}}%
\pgfpathcurveto{\pgfqpoint{3.131798in}{3.002864in}}{\pgfqpoint{3.127408in}{3.013463in}}{\pgfqpoint{3.119594in}{3.021276in}}%
\pgfpathcurveto{\pgfqpoint{3.111780in}{3.029090in}}{\pgfqpoint{3.101181in}{3.033480in}}{\pgfqpoint{3.090131in}{3.033480in}}%
\pgfpathcurveto{\pgfqpoint{3.079081in}{3.033480in}}{\pgfqpoint{3.068482in}{3.029090in}}{\pgfqpoint{3.060668in}{3.021276in}}%
\pgfpathcurveto{\pgfqpoint{3.052855in}{3.013463in}}{\pgfqpoint{3.048464in}{3.002864in}}{\pgfqpoint{3.048464in}{2.991814in}}%
\pgfpathcurveto{\pgfqpoint{3.048464in}{2.980764in}}{\pgfqpoint{3.052855in}{2.970165in}}{\pgfqpoint{3.060668in}{2.962351in}}%
\pgfpathcurveto{\pgfqpoint{3.068482in}{2.954537in}}{\pgfqpoint{3.079081in}{2.950147in}}{\pgfqpoint{3.090131in}{2.950147in}}%
\pgfpathclose%
\pgfusepath{stroke,fill}%
\end{pgfscope}%
\begin{pgfscope}%
\pgfpathrectangle{\pgfqpoint{0.600000in}{0.600000in}}{\pgfqpoint{3.900000in}{3.900000in}}%
\pgfusepath{clip}%
\pgfsetbuttcap%
\pgfsetroundjoin%
\definecolor{currentfill}{rgb}{0.121569,0.466667,0.705882}%
\pgfsetfillcolor{currentfill}%
\pgfsetlinewidth{1.003750pt}%
\definecolor{currentstroke}{rgb}{0.121569,0.466667,0.705882}%
\pgfsetstrokecolor{currentstroke}%
\pgfsetdash{}{0pt}%
\pgfpathmoveto{\pgfqpoint{3.073820in}{2.207024in}}%
\pgfpathcurveto{\pgfqpoint{3.084870in}{2.207024in}}{\pgfqpoint{3.095469in}{2.211415in}}{\pgfqpoint{3.103282in}{2.219228in}}%
\pgfpathcurveto{\pgfqpoint{3.111096in}{2.227042in}}{\pgfqpoint{3.115486in}{2.237641in}}{\pgfqpoint{3.115486in}{2.248691in}}%
\pgfpathcurveto{\pgfqpoint{3.115486in}{2.259741in}}{\pgfqpoint{3.111096in}{2.270340in}}{\pgfqpoint{3.103282in}{2.278154in}}%
\pgfpathcurveto{\pgfqpoint{3.095469in}{2.285967in}}{\pgfqpoint{3.084870in}{2.290358in}}{\pgfqpoint{3.073820in}{2.290358in}}%
\pgfpathcurveto{\pgfqpoint{3.062769in}{2.290358in}}{\pgfqpoint{3.052170in}{2.285967in}}{\pgfqpoint{3.044357in}{2.278154in}}%
\pgfpathcurveto{\pgfqpoint{3.036543in}{2.270340in}}{\pgfqpoint{3.032153in}{2.259741in}}{\pgfqpoint{3.032153in}{2.248691in}}%
\pgfpathcurveto{\pgfqpoint{3.032153in}{2.237641in}}{\pgfqpoint{3.036543in}{2.227042in}}{\pgfqpoint{3.044357in}{2.219228in}}%
\pgfpathcurveto{\pgfqpoint{3.052170in}{2.211415in}}{\pgfqpoint{3.062769in}{2.207024in}}{\pgfqpoint{3.073820in}{2.207024in}}%
\pgfpathclose%
\pgfusepath{stroke,fill}%
\end{pgfscope}%
\begin{pgfscope}%
\pgfpathrectangle{\pgfqpoint{0.600000in}{0.600000in}}{\pgfqpoint{3.900000in}{3.900000in}}%
\pgfusepath{clip}%
\pgfsetbuttcap%
\pgfsetroundjoin%
\definecolor{currentfill}{rgb}{0.121569,0.466667,0.705882}%
\pgfsetfillcolor{currentfill}%
\pgfsetlinewidth{1.003750pt}%
\definecolor{currentstroke}{rgb}{0.121569,0.466667,0.705882}%
\pgfsetstrokecolor{currentstroke}%
\pgfsetdash{}{0pt}%
\pgfpathmoveto{\pgfqpoint{2.368548in}{1.958973in}}%
\pgfpathcurveto{\pgfqpoint{2.379598in}{1.958973in}}{\pgfqpoint{2.390197in}{1.963363in}}{\pgfqpoint{2.398011in}{1.971177in}}%
\pgfpathcurveto{\pgfqpoint{2.405824in}{1.978990in}}{\pgfqpoint{2.410215in}{1.989589in}}{\pgfqpoint{2.410215in}{2.000639in}}%
\pgfpathcurveto{\pgfqpoint{2.410215in}{2.011689in}}{\pgfqpoint{2.405824in}{2.022288in}}{\pgfqpoint{2.398011in}{2.030102in}}%
\pgfpathcurveto{\pgfqpoint{2.390197in}{2.037916in}}{\pgfqpoint{2.379598in}{2.042306in}}{\pgfqpoint{2.368548in}{2.042306in}}%
\pgfpathcurveto{\pgfqpoint{2.357498in}{2.042306in}}{\pgfqpoint{2.346899in}{2.037916in}}{\pgfqpoint{2.339085in}{2.030102in}}%
\pgfpathcurveto{\pgfqpoint{2.331272in}{2.022288in}}{\pgfqpoint{2.326881in}{2.011689in}}{\pgfqpoint{2.326881in}{2.000639in}}%
\pgfpathcurveto{\pgfqpoint{2.326881in}{1.989589in}}{\pgfqpoint{2.331272in}{1.978990in}}{\pgfqpoint{2.339085in}{1.971177in}}%
\pgfpathcurveto{\pgfqpoint{2.346899in}{1.963363in}}{\pgfqpoint{2.357498in}{1.958973in}}{\pgfqpoint{2.368548in}{1.958973in}}%
\pgfpathclose%
\pgfusepath{stroke,fill}%
\end{pgfscope}%
\begin{pgfscope}%
\pgfpathrectangle{\pgfqpoint{0.600000in}{0.600000in}}{\pgfqpoint{3.900000in}{3.900000in}}%
\pgfusepath{clip}%
\pgfsetbuttcap%
\pgfsetroundjoin%
\definecolor{currentfill}{rgb}{0.121569,0.466667,0.705882}%
\pgfsetfillcolor{currentfill}%
\pgfsetlinewidth{1.003750pt}%
\definecolor{currentstroke}{rgb}{0.121569,0.466667,0.705882}%
\pgfsetstrokecolor{currentstroke}%
\pgfsetdash{}{0pt}%
\pgfpathmoveto{\pgfqpoint{2.829230in}{2.394861in}}%
\pgfpathcurveto{\pgfqpoint{2.840280in}{2.394861in}}{\pgfqpoint{2.850879in}{2.399251in}}{\pgfqpoint{2.858693in}{2.407065in}}%
\pgfpathcurveto{\pgfqpoint{2.866507in}{2.414879in}}{\pgfqpoint{2.870897in}{2.425478in}}{\pgfqpoint{2.870897in}{2.436528in}}%
\pgfpathcurveto{\pgfqpoint{2.870897in}{2.447578in}}{\pgfqpoint{2.866507in}{2.458177in}}{\pgfqpoint{2.858693in}{2.465991in}}%
\pgfpathcurveto{\pgfqpoint{2.850879in}{2.473804in}}{\pgfqpoint{2.840280in}{2.478194in}}{\pgfqpoint{2.829230in}{2.478194in}}%
\pgfpathcurveto{\pgfqpoint{2.818180in}{2.478194in}}{\pgfqpoint{2.807581in}{2.473804in}}{\pgfqpoint{2.799768in}{2.465991in}}%
\pgfpathcurveto{\pgfqpoint{2.791954in}{2.458177in}}{\pgfqpoint{2.787564in}{2.447578in}}{\pgfqpoint{2.787564in}{2.436528in}}%
\pgfpathcurveto{\pgfqpoint{2.787564in}{2.425478in}}{\pgfqpoint{2.791954in}{2.414879in}}{\pgfqpoint{2.799768in}{2.407065in}}%
\pgfpathcurveto{\pgfqpoint{2.807581in}{2.399251in}}{\pgfqpoint{2.818180in}{2.394861in}}{\pgfqpoint{2.829230in}{2.394861in}}%
\pgfpathclose%
\pgfusepath{stroke,fill}%
\end{pgfscope}%
\begin{pgfscope}%
\pgfpathrectangle{\pgfqpoint{0.600000in}{0.600000in}}{\pgfqpoint{3.900000in}{3.900000in}}%
\pgfusepath{clip}%
\pgfsetbuttcap%
\pgfsetroundjoin%
\definecolor{currentfill}{rgb}{0.121569,0.466667,0.705882}%
\pgfsetfillcolor{currentfill}%
\pgfsetlinewidth{1.003750pt}%
\definecolor{currentstroke}{rgb}{0.121569,0.466667,0.705882}%
\pgfsetstrokecolor{currentstroke}%
\pgfsetdash{}{0pt}%
\pgfpathmoveto{\pgfqpoint{1.935265in}{2.830843in}}%
\pgfpathcurveto{\pgfqpoint{1.946315in}{2.830843in}}{\pgfqpoint{1.956914in}{2.835233in}}{\pgfqpoint{1.964728in}{2.843047in}}%
\pgfpathcurveto{\pgfqpoint{1.972542in}{2.850861in}}{\pgfqpoint{1.976932in}{2.861460in}}{\pgfqpoint{1.976932in}{2.872510in}}%
\pgfpathcurveto{\pgfqpoint{1.976932in}{2.883560in}}{\pgfqpoint{1.972542in}{2.894159in}}{\pgfqpoint{1.964728in}{2.901973in}}%
\pgfpathcurveto{\pgfqpoint{1.956914in}{2.909786in}}{\pgfqpoint{1.946315in}{2.914176in}}{\pgfqpoint{1.935265in}{2.914176in}}%
\pgfpathcurveto{\pgfqpoint{1.924215in}{2.914176in}}{\pgfqpoint{1.913616in}{2.909786in}}{\pgfqpoint{1.905802in}{2.901973in}}%
\pgfpathcurveto{\pgfqpoint{1.897989in}{2.894159in}}{\pgfqpoint{1.893599in}{2.883560in}}{\pgfqpoint{1.893599in}{2.872510in}}%
\pgfpathcurveto{\pgfqpoint{1.893599in}{2.861460in}}{\pgfqpoint{1.897989in}{2.850861in}}{\pgfqpoint{1.905802in}{2.843047in}}%
\pgfpathcurveto{\pgfqpoint{1.913616in}{2.835233in}}{\pgfqpoint{1.924215in}{2.830843in}}{\pgfqpoint{1.935265in}{2.830843in}}%
\pgfpathclose%
\pgfusepath{stroke,fill}%
\end{pgfscope}%
\begin{pgfscope}%
\pgfpathrectangle{\pgfqpoint{0.600000in}{0.600000in}}{\pgfqpoint{3.900000in}{3.900000in}}%
\pgfusepath{clip}%
\pgfsetbuttcap%
\pgfsetroundjoin%
\definecolor{currentfill}{rgb}{0.121569,0.466667,0.705882}%
\pgfsetfillcolor{currentfill}%
\pgfsetlinewidth{1.003750pt}%
\definecolor{currentstroke}{rgb}{0.121569,0.466667,0.705882}%
\pgfsetstrokecolor{currentstroke}%
\pgfsetdash{}{0pt}%
\pgfpathmoveto{\pgfqpoint{2.573114in}{1.807303in}}%
\pgfpathcurveto{\pgfqpoint{2.584164in}{1.807303in}}{\pgfqpoint{2.594763in}{1.811693in}}{\pgfqpoint{2.602577in}{1.819507in}}%
\pgfpathcurveto{\pgfqpoint{2.610390in}{1.827320in}}{\pgfqpoint{2.614781in}{1.837919in}}{\pgfqpoint{2.614781in}{1.848970in}}%
\pgfpathcurveto{\pgfqpoint{2.614781in}{1.860020in}}{\pgfqpoint{2.610390in}{1.870619in}}{\pgfqpoint{2.602577in}{1.878432in}}%
\pgfpathcurveto{\pgfqpoint{2.594763in}{1.886246in}}{\pgfqpoint{2.584164in}{1.890636in}}{\pgfqpoint{2.573114in}{1.890636in}}%
\pgfpathcurveto{\pgfqpoint{2.562064in}{1.890636in}}{\pgfqpoint{2.551465in}{1.886246in}}{\pgfqpoint{2.543651in}{1.878432in}}%
\pgfpathcurveto{\pgfqpoint{2.535838in}{1.870619in}}{\pgfqpoint{2.531447in}{1.860020in}}{\pgfqpoint{2.531447in}{1.848970in}}%
\pgfpathcurveto{\pgfqpoint{2.531447in}{1.837919in}}{\pgfqpoint{2.535838in}{1.827320in}}{\pgfqpoint{2.543651in}{1.819507in}}%
\pgfpathcurveto{\pgfqpoint{2.551465in}{1.811693in}}{\pgfqpoint{2.562064in}{1.807303in}}{\pgfqpoint{2.573114in}{1.807303in}}%
\pgfpathclose%
\pgfusepath{stroke,fill}%
\end{pgfscope}%
\begin{pgfscope}%
\pgfpathrectangle{\pgfqpoint{0.600000in}{0.600000in}}{\pgfqpoint{3.900000in}{3.900000in}}%
\pgfusepath{clip}%
\pgfsetbuttcap%
\pgfsetroundjoin%
\definecolor{currentfill}{rgb}{0.121569,0.466667,0.705882}%
\pgfsetfillcolor{currentfill}%
\pgfsetlinewidth{1.003750pt}%
\definecolor{currentstroke}{rgb}{0.121569,0.466667,0.705882}%
\pgfsetstrokecolor{currentstroke}%
\pgfsetdash{}{0pt}%
\pgfpathmoveto{\pgfqpoint{2.555953in}{1.617258in}}%
\pgfpathcurveto{\pgfqpoint{2.567003in}{1.617258in}}{\pgfqpoint{2.577602in}{1.621649in}}{\pgfqpoint{2.585416in}{1.629462in}}%
\pgfpathcurveto{\pgfqpoint{2.593230in}{1.637276in}}{\pgfqpoint{2.597620in}{1.647875in}}{\pgfqpoint{2.597620in}{1.658925in}}%
\pgfpathcurveto{\pgfqpoint{2.597620in}{1.669975in}}{\pgfqpoint{2.593230in}{1.680574in}}{\pgfqpoint{2.585416in}{1.688388in}}%
\pgfpathcurveto{\pgfqpoint{2.577602in}{1.696201in}}{\pgfqpoint{2.567003in}{1.700592in}}{\pgfqpoint{2.555953in}{1.700592in}}%
\pgfpathcurveto{\pgfqpoint{2.544903in}{1.700592in}}{\pgfqpoint{2.534304in}{1.696201in}}{\pgfqpoint{2.526491in}{1.688388in}}%
\pgfpathcurveto{\pgfqpoint{2.518677in}{1.680574in}}{\pgfqpoint{2.514287in}{1.669975in}}{\pgfqpoint{2.514287in}{1.658925in}}%
\pgfpathcurveto{\pgfqpoint{2.514287in}{1.647875in}}{\pgfqpoint{2.518677in}{1.637276in}}{\pgfqpoint{2.526491in}{1.629462in}}%
\pgfpathcurveto{\pgfqpoint{2.534304in}{1.621649in}}{\pgfqpoint{2.544903in}{1.617258in}}{\pgfqpoint{2.555953in}{1.617258in}}%
\pgfpathclose%
\pgfusepath{stroke,fill}%
\end{pgfscope}%
\begin{pgfscope}%
\pgfpathrectangle{\pgfqpoint{0.600000in}{0.600000in}}{\pgfqpoint{3.900000in}{3.900000in}}%
\pgfusepath{clip}%
\pgfsetbuttcap%
\pgfsetroundjoin%
\definecolor{currentfill}{rgb}{0.121569,0.466667,0.705882}%
\pgfsetfillcolor{currentfill}%
\pgfsetlinewidth{1.003750pt}%
\definecolor{currentstroke}{rgb}{0.121569,0.466667,0.705882}%
\pgfsetstrokecolor{currentstroke}%
\pgfsetdash{}{0pt}%
\pgfpathmoveto{\pgfqpoint{2.310615in}{2.539972in}}%
\pgfpathcurveto{\pgfqpoint{2.321665in}{2.539972in}}{\pgfqpoint{2.332264in}{2.544362in}}{\pgfqpoint{2.340077in}{2.552176in}}%
\pgfpathcurveto{\pgfqpoint{2.347891in}{2.559989in}}{\pgfqpoint{2.352281in}{2.570588in}}{\pgfqpoint{2.352281in}{2.581638in}}%
\pgfpathcurveto{\pgfqpoint{2.352281in}{2.592689in}}{\pgfqpoint{2.347891in}{2.603288in}}{\pgfqpoint{2.340077in}{2.611101in}}%
\pgfpathcurveto{\pgfqpoint{2.332264in}{2.618915in}}{\pgfqpoint{2.321665in}{2.623305in}}{\pgfqpoint{2.310615in}{2.623305in}}%
\pgfpathcurveto{\pgfqpoint{2.299564in}{2.623305in}}{\pgfqpoint{2.288965in}{2.618915in}}{\pgfqpoint{2.281152in}{2.611101in}}%
\pgfpathcurveto{\pgfqpoint{2.273338in}{2.603288in}}{\pgfqpoint{2.268948in}{2.592689in}}{\pgfqpoint{2.268948in}{2.581638in}}%
\pgfpathcurveto{\pgfqpoint{2.268948in}{2.570588in}}{\pgfqpoint{2.273338in}{2.559989in}}{\pgfqpoint{2.281152in}{2.552176in}}%
\pgfpathcurveto{\pgfqpoint{2.288965in}{2.544362in}}{\pgfqpoint{2.299564in}{2.539972in}}{\pgfqpoint{2.310615in}{2.539972in}}%
\pgfpathclose%
\pgfusepath{stroke,fill}%
\end{pgfscope}%
\begin{pgfscope}%
\pgfpathrectangle{\pgfqpoint{0.600000in}{0.600000in}}{\pgfqpoint{3.900000in}{3.900000in}}%
\pgfusepath{clip}%
\pgfsetbuttcap%
\pgfsetroundjoin%
\definecolor{currentfill}{rgb}{0.121569,0.466667,0.705882}%
\pgfsetfillcolor{currentfill}%
\pgfsetlinewidth{1.003750pt}%
\definecolor{currentstroke}{rgb}{0.121569,0.466667,0.705882}%
\pgfsetstrokecolor{currentstroke}%
\pgfsetdash{}{0pt}%
\pgfpathmoveto{\pgfqpoint{2.025756in}{2.821347in}}%
\pgfpathcurveto{\pgfqpoint{2.036806in}{2.821347in}}{\pgfqpoint{2.047406in}{2.825737in}}{\pgfqpoint{2.055219in}{2.833551in}}%
\pgfpathcurveto{\pgfqpoint{2.063033in}{2.841365in}}{\pgfqpoint{2.067423in}{2.851964in}}{\pgfqpoint{2.067423in}{2.863014in}}%
\pgfpathcurveto{\pgfqpoint{2.067423in}{2.874064in}}{\pgfqpoint{2.063033in}{2.884663in}}{\pgfqpoint{2.055219in}{2.892476in}}%
\pgfpathcurveto{\pgfqpoint{2.047406in}{2.900290in}}{\pgfqpoint{2.036806in}{2.904680in}}{\pgfqpoint{2.025756in}{2.904680in}}%
\pgfpathcurveto{\pgfqpoint{2.014706in}{2.904680in}}{\pgfqpoint{2.004107in}{2.900290in}}{\pgfqpoint{1.996294in}{2.892476in}}%
\pgfpathcurveto{\pgfqpoint{1.988480in}{2.884663in}}{\pgfqpoint{1.984090in}{2.874064in}}{\pgfqpoint{1.984090in}{2.863014in}}%
\pgfpathcurveto{\pgfqpoint{1.984090in}{2.851964in}}{\pgfqpoint{1.988480in}{2.841365in}}{\pgfqpoint{1.996294in}{2.833551in}}%
\pgfpathcurveto{\pgfqpoint{2.004107in}{2.825737in}}{\pgfqpoint{2.014706in}{2.821347in}}{\pgfqpoint{2.025756in}{2.821347in}}%
\pgfpathclose%
\pgfusepath{stroke,fill}%
\end{pgfscope}%
\begin{pgfscope}%
\pgfpathrectangle{\pgfqpoint{0.600000in}{0.600000in}}{\pgfqpoint{3.900000in}{3.900000in}}%
\pgfusepath{clip}%
\pgfsetbuttcap%
\pgfsetroundjoin%
\definecolor{currentfill}{rgb}{0.121569,0.466667,0.705882}%
\pgfsetfillcolor{currentfill}%
\pgfsetlinewidth{1.003750pt}%
\definecolor{currentstroke}{rgb}{0.121569,0.466667,0.705882}%
\pgfsetstrokecolor{currentstroke}%
\pgfsetdash{}{0pt}%
\pgfpathmoveto{\pgfqpoint{2.376921in}{2.465198in}}%
\pgfpathcurveto{\pgfqpoint{2.387971in}{2.465198in}}{\pgfqpoint{2.398570in}{2.469589in}}{\pgfqpoint{2.406384in}{2.477402in}}%
\pgfpathcurveto{\pgfqpoint{2.414198in}{2.485216in}}{\pgfqpoint{2.418588in}{2.495815in}}{\pgfqpoint{2.418588in}{2.506865in}}%
\pgfpathcurveto{\pgfqpoint{2.418588in}{2.517915in}}{\pgfqpoint{2.414198in}{2.528514in}}{\pgfqpoint{2.406384in}{2.536328in}}%
\pgfpathcurveto{\pgfqpoint{2.398570in}{2.544142in}}{\pgfqpoint{2.387971in}{2.548532in}}{\pgfqpoint{2.376921in}{2.548532in}}%
\pgfpathcurveto{\pgfqpoint{2.365871in}{2.548532in}}{\pgfqpoint{2.355272in}{2.544142in}}{\pgfqpoint{2.347458in}{2.536328in}}%
\pgfpathcurveto{\pgfqpoint{2.339645in}{2.528514in}}{\pgfqpoint{2.335255in}{2.517915in}}{\pgfqpoint{2.335255in}{2.506865in}}%
\pgfpathcurveto{\pgfqpoint{2.335255in}{2.495815in}}{\pgfqpoint{2.339645in}{2.485216in}}{\pgfqpoint{2.347458in}{2.477402in}}%
\pgfpathcurveto{\pgfqpoint{2.355272in}{2.469589in}}{\pgfqpoint{2.365871in}{2.465198in}}{\pgfqpoint{2.376921in}{2.465198in}}%
\pgfpathclose%
\pgfusepath{stroke,fill}%
\end{pgfscope}%
\begin{pgfscope}%
\pgfpathrectangle{\pgfqpoint{0.600000in}{0.600000in}}{\pgfqpoint{3.900000in}{3.900000in}}%
\pgfusepath{clip}%
\pgfsetbuttcap%
\pgfsetroundjoin%
\definecolor{currentfill}{rgb}{0.121569,0.466667,0.705882}%
\pgfsetfillcolor{currentfill}%
\pgfsetlinewidth{1.003750pt}%
\definecolor{currentstroke}{rgb}{0.121569,0.466667,0.705882}%
\pgfsetstrokecolor{currentstroke}%
\pgfsetdash{}{0pt}%
\pgfpathmoveto{\pgfqpoint{2.586046in}{1.588959in}}%
\pgfpathcurveto{\pgfqpoint{2.597096in}{1.588959in}}{\pgfqpoint{2.607695in}{1.593349in}}{\pgfqpoint{2.615508in}{1.601162in}}%
\pgfpathcurveto{\pgfqpoint{2.623322in}{1.608976in}}{\pgfqpoint{2.627712in}{1.619575in}}{\pgfqpoint{2.627712in}{1.630625in}}%
\pgfpathcurveto{\pgfqpoint{2.627712in}{1.641675in}}{\pgfqpoint{2.623322in}{1.652274in}}{\pgfqpoint{2.615508in}{1.660088in}}%
\pgfpathcurveto{\pgfqpoint{2.607695in}{1.667902in}}{\pgfqpoint{2.597096in}{1.672292in}}{\pgfqpoint{2.586046in}{1.672292in}}%
\pgfpathcurveto{\pgfqpoint{2.574996in}{1.672292in}}{\pgfqpoint{2.564397in}{1.667902in}}{\pgfqpoint{2.556583in}{1.660088in}}%
\pgfpathcurveto{\pgfqpoint{2.548769in}{1.652274in}}{\pgfqpoint{2.544379in}{1.641675in}}{\pgfqpoint{2.544379in}{1.630625in}}%
\pgfpathcurveto{\pgfqpoint{2.544379in}{1.619575in}}{\pgfqpoint{2.548769in}{1.608976in}}{\pgfqpoint{2.556583in}{1.601162in}}%
\pgfpathcurveto{\pgfqpoint{2.564397in}{1.593349in}}{\pgfqpoint{2.574996in}{1.588959in}}{\pgfqpoint{2.586046in}{1.588959in}}%
\pgfpathclose%
\pgfusepath{stroke,fill}%
\end{pgfscope}%
\begin{pgfscope}%
\pgfpathrectangle{\pgfqpoint{0.600000in}{0.600000in}}{\pgfqpoint{3.900000in}{3.900000in}}%
\pgfusepath{clip}%
\pgfsetbuttcap%
\pgfsetroundjoin%
\definecolor{currentfill}{rgb}{0.121569,0.466667,0.705882}%
\pgfsetfillcolor{currentfill}%
\pgfsetlinewidth{1.003750pt}%
\definecolor{currentstroke}{rgb}{0.121569,0.466667,0.705882}%
\pgfsetstrokecolor{currentstroke}%
\pgfsetdash{}{0pt}%
\pgfpathmoveto{\pgfqpoint{2.798427in}{3.283876in}}%
\pgfpathcurveto{\pgfqpoint{2.809477in}{3.283876in}}{\pgfqpoint{2.820076in}{3.288266in}}{\pgfqpoint{2.827890in}{3.296080in}}%
\pgfpathcurveto{\pgfqpoint{2.835704in}{3.303893in}}{\pgfqpoint{2.840094in}{3.314492in}}{\pgfqpoint{2.840094in}{3.325542in}}%
\pgfpathcurveto{\pgfqpoint{2.840094in}{3.336593in}}{\pgfqpoint{2.835704in}{3.347192in}}{\pgfqpoint{2.827890in}{3.355005in}}%
\pgfpathcurveto{\pgfqpoint{2.820076in}{3.362819in}}{\pgfqpoint{2.809477in}{3.367209in}}{\pgfqpoint{2.798427in}{3.367209in}}%
\pgfpathcurveto{\pgfqpoint{2.787377in}{3.367209in}}{\pgfqpoint{2.776778in}{3.362819in}}{\pgfqpoint{2.768965in}{3.355005in}}%
\pgfpathcurveto{\pgfqpoint{2.761151in}{3.347192in}}{\pgfqpoint{2.756761in}{3.336593in}}{\pgfqpoint{2.756761in}{3.325542in}}%
\pgfpathcurveto{\pgfqpoint{2.756761in}{3.314492in}}{\pgfqpoint{2.761151in}{3.303893in}}{\pgfqpoint{2.768965in}{3.296080in}}%
\pgfpathcurveto{\pgfqpoint{2.776778in}{3.288266in}}{\pgfqpoint{2.787377in}{3.283876in}}{\pgfqpoint{2.798427in}{3.283876in}}%
\pgfpathclose%
\pgfusepath{stroke,fill}%
\end{pgfscope}%
\begin{pgfscope}%
\pgfpathrectangle{\pgfqpoint{0.600000in}{0.600000in}}{\pgfqpoint{3.900000in}{3.900000in}}%
\pgfusepath{clip}%
\pgfsetbuttcap%
\pgfsetroundjoin%
\definecolor{currentfill}{rgb}{0.121569,0.466667,0.705882}%
\pgfsetfillcolor{currentfill}%
\pgfsetlinewidth{1.003750pt}%
\definecolor{currentstroke}{rgb}{0.121569,0.466667,0.705882}%
\pgfsetstrokecolor{currentstroke}%
\pgfsetdash{}{0pt}%
\pgfpathmoveto{\pgfqpoint{2.087157in}{2.212335in}}%
\pgfpathcurveto{\pgfqpoint{2.098208in}{2.212335in}}{\pgfqpoint{2.108807in}{2.216725in}}{\pgfqpoint{2.116620in}{2.224539in}}%
\pgfpathcurveto{\pgfqpoint{2.124434in}{2.232352in}}{\pgfqpoint{2.128824in}{2.242951in}}{\pgfqpoint{2.128824in}{2.254002in}}%
\pgfpathcurveto{\pgfqpoint{2.128824in}{2.265052in}}{\pgfqpoint{2.124434in}{2.275651in}}{\pgfqpoint{2.116620in}{2.283464in}}%
\pgfpathcurveto{\pgfqpoint{2.108807in}{2.291278in}}{\pgfqpoint{2.098208in}{2.295668in}}{\pgfqpoint{2.087157in}{2.295668in}}%
\pgfpathcurveto{\pgfqpoint{2.076107in}{2.295668in}}{\pgfqpoint{2.065508in}{2.291278in}}{\pgfqpoint{2.057695in}{2.283464in}}%
\pgfpathcurveto{\pgfqpoint{2.049881in}{2.275651in}}{\pgfqpoint{2.045491in}{2.265052in}}{\pgfqpoint{2.045491in}{2.254002in}}%
\pgfpathcurveto{\pgfqpoint{2.045491in}{2.242951in}}{\pgfqpoint{2.049881in}{2.232352in}}{\pgfqpoint{2.057695in}{2.224539in}}%
\pgfpathcurveto{\pgfqpoint{2.065508in}{2.216725in}}{\pgfqpoint{2.076107in}{2.212335in}}{\pgfqpoint{2.087157in}{2.212335in}}%
\pgfpathclose%
\pgfusepath{stroke,fill}%
\end{pgfscope}%
\begin{pgfscope}%
\pgfpathrectangle{\pgfqpoint{0.600000in}{0.600000in}}{\pgfqpoint{3.900000in}{3.900000in}}%
\pgfusepath{clip}%
\pgfsetbuttcap%
\pgfsetroundjoin%
\definecolor{currentfill}{rgb}{0.121569,0.466667,0.705882}%
\pgfsetfillcolor{currentfill}%
\pgfsetlinewidth{1.003750pt}%
\definecolor{currentstroke}{rgb}{0.121569,0.466667,0.705882}%
\pgfsetstrokecolor{currentstroke}%
\pgfsetdash{}{0pt}%
\pgfpathmoveto{\pgfqpoint{2.397486in}{2.164796in}}%
\pgfpathcurveto{\pgfqpoint{2.408536in}{2.164796in}}{\pgfqpoint{2.419135in}{2.169187in}}{\pgfqpoint{2.426948in}{2.177000in}}%
\pgfpathcurveto{\pgfqpoint{2.434762in}{2.184814in}}{\pgfqpoint{2.439152in}{2.195413in}}{\pgfqpoint{2.439152in}{2.206463in}}%
\pgfpathcurveto{\pgfqpoint{2.439152in}{2.217513in}}{\pgfqpoint{2.434762in}{2.228112in}}{\pgfqpoint{2.426948in}{2.235926in}}%
\pgfpathcurveto{\pgfqpoint{2.419135in}{2.243740in}}{\pgfqpoint{2.408536in}{2.248130in}}{\pgfqpoint{2.397486in}{2.248130in}}%
\pgfpathcurveto{\pgfqpoint{2.386436in}{2.248130in}}{\pgfqpoint{2.375837in}{2.243740in}}{\pgfqpoint{2.368023in}{2.235926in}}%
\pgfpathcurveto{\pgfqpoint{2.360209in}{2.228112in}}{\pgfqpoint{2.355819in}{2.217513in}}{\pgfqpoint{2.355819in}{2.206463in}}%
\pgfpathcurveto{\pgfqpoint{2.355819in}{2.195413in}}{\pgfqpoint{2.360209in}{2.184814in}}{\pgfqpoint{2.368023in}{2.177000in}}%
\pgfpathcurveto{\pgfqpoint{2.375837in}{2.169187in}}{\pgfqpoint{2.386436in}{2.164796in}}{\pgfqpoint{2.397486in}{2.164796in}}%
\pgfpathclose%
\pgfusepath{stroke,fill}%
\end{pgfscope}%
\begin{pgfscope}%
\pgfpathrectangle{\pgfqpoint{0.600000in}{0.600000in}}{\pgfqpoint{3.900000in}{3.900000in}}%
\pgfusepath{clip}%
\pgfsetbuttcap%
\pgfsetroundjoin%
\definecolor{currentfill}{rgb}{0.121569,0.466667,0.705882}%
\pgfsetfillcolor{currentfill}%
\pgfsetlinewidth{1.003750pt}%
\definecolor{currentstroke}{rgb}{0.121569,0.466667,0.705882}%
\pgfsetstrokecolor{currentstroke}%
\pgfsetdash{}{0pt}%
\pgfpathmoveto{\pgfqpoint{2.054790in}{2.299681in}}%
\pgfpathcurveto{\pgfqpoint{2.065840in}{2.299681in}}{\pgfqpoint{2.076439in}{2.304071in}}{\pgfqpoint{2.084252in}{2.311885in}}%
\pgfpathcurveto{\pgfqpoint{2.092066in}{2.319699in}}{\pgfqpoint{2.096456in}{2.330298in}}{\pgfqpoint{2.096456in}{2.341348in}}%
\pgfpathcurveto{\pgfqpoint{2.096456in}{2.352398in}}{\pgfqpoint{2.092066in}{2.362997in}}{\pgfqpoint{2.084252in}{2.370811in}}%
\pgfpathcurveto{\pgfqpoint{2.076439in}{2.378624in}}{\pgfqpoint{2.065840in}{2.383014in}}{\pgfqpoint{2.054790in}{2.383014in}}%
\pgfpathcurveto{\pgfqpoint{2.043739in}{2.383014in}}{\pgfqpoint{2.033140in}{2.378624in}}{\pgfqpoint{2.025327in}{2.370811in}}%
\pgfpathcurveto{\pgfqpoint{2.017513in}{2.362997in}}{\pgfqpoint{2.013123in}{2.352398in}}{\pgfqpoint{2.013123in}{2.341348in}}%
\pgfpathcurveto{\pgfqpoint{2.013123in}{2.330298in}}{\pgfqpoint{2.017513in}{2.319699in}}{\pgfqpoint{2.025327in}{2.311885in}}%
\pgfpathcurveto{\pgfqpoint{2.033140in}{2.304071in}}{\pgfqpoint{2.043739in}{2.299681in}}{\pgfqpoint{2.054790in}{2.299681in}}%
\pgfpathclose%
\pgfusepath{stroke,fill}%
\end{pgfscope}%
\begin{pgfscope}%
\pgfpathrectangle{\pgfqpoint{0.600000in}{0.600000in}}{\pgfqpoint{3.900000in}{3.900000in}}%
\pgfusepath{clip}%
\pgfsetbuttcap%
\pgfsetroundjoin%
\definecolor{currentfill}{rgb}{0.121569,0.466667,0.705882}%
\pgfsetfillcolor{currentfill}%
\pgfsetlinewidth{1.003750pt}%
\definecolor{currentstroke}{rgb}{0.121569,0.466667,0.705882}%
\pgfsetstrokecolor{currentstroke}%
\pgfsetdash{}{0pt}%
\pgfpathmoveto{\pgfqpoint{2.071824in}{2.181460in}}%
\pgfpathcurveto{\pgfqpoint{2.082874in}{2.181460in}}{\pgfqpoint{2.093473in}{2.185850in}}{\pgfqpoint{2.101287in}{2.193664in}}%
\pgfpathcurveto{\pgfqpoint{2.109100in}{2.201477in}}{\pgfqpoint{2.113491in}{2.212076in}}{\pgfqpoint{2.113491in}{2.223126in}}%
\pgfpathcurveto{\pgfqpoint{2.113491in}{2.234177in}}{\pgfqpoint{2.109100in}{2.244776in}}{\pgfqpoint{2.101287in}{2.252589in}}%
\pgfpathcurveto{\pgfqpoint{2.093473in}{2.260403in}}{\pgfqpoint{2.082874in}{2.264793in}}{\pgfqpoint{2.071824in}{2.264793in}}%
\pgfpathcurveto{\pgfqpoint{2.060774in}{2.264793in}}{\pgfqpoint{2.050175in}{2.260403in}}{\pgfqpoint{2.042361in}{2.252589in}}%
\pgfpathcurveto{\pgfqpoint{2.034548in}{2.244776in}}{\pgfqpoint{2.030157in}{2.234177in}}{\pgfqpoint{2.030157in}{2.223126in}}%
\pgfpathcurveto{\pgfqpoint{2.030157in}{2.212076in}}{\pgfqpoint{2.034548in}{2.201477in}}{\pgfqpoint{2.042361in}{2.193664in}}%
\pgfpathcurveto{\pgfqpoint{2.050175in}{2.185850in}}{\pgfqpoint{2.060774in}{2.181460in}}{\pgfqpoint{2.071824in}{2.181460in}}%
\pgfpathclose%
\pgfusepath{stroke,fill}%
\end{pgfscope}%
\begin{pgfscope}%
\pgfpathrectangle{\pgfqpoint{0.600000in}{0.600000in}}{\pgfqpoint{3.900000in}{3.900000in}}%
\pgfusepath{clip}%
\pgfsetbuttcap%
\pgfsetroundjoin%
\definecolor{currentfill}{rgb}{0.121569,0.466667,0.705882}%
\pgfsetfillcolor{currentfill}%
\pgfsetlinewidth{1.003750pt}%
\definecolor{currentstroke}{rgb}{0.121569,0.466667,0.705882}%
\pgfsetstrokecolor{currentstroke}%
\pgfsetdash{}{0pt}%
\pgfpathmoveto{\pgfqpoint{2.801561in}{2.335775in}}%
\pgfpathcurveto{\pgfqpoint{2.812611in}{2.335775in}}{\pgfqpoint{2.823210in}{2.340165in}}{\pgfqpoint{2.831024in}{2.347979in}}%
\pgfpathcurveto{\pgfqpoint{2.838838in}{2.355793in}}{\pgfqpoint{2.843228in}{2.366392in}}{\pgfqpoint{2.843228in}{2.377442in}}%
\pgfpathcurveto{\pgfqpoint{2.843228in}{2.388492in}}{\pgfqpoint{2.838838in}{2.399091in}}{\pgfqpoint{2.831024in}{2.406905in}}%
\pgfpathcurveto{\pgfqpoint{2.823210in}{2.414718in}}{\pgfqpoint{2.812611in}{2.419108in}}{\pgfqpoint{2.801561in}{2.419108in}}%
\pgfpathcurveto{\pgfqpoint{2.790511in}{2.419108in}}{\pgfqpoint{2.779912in}{2.414718in}}{\pgfqpoint{2.772098in}{2.406905in}}%
\pgfpathcurveto{\pgfqpoint{2.764285in}{2.399091in}}{\pgfqpoint{2.759895in}{2.388492in}}{\pgfqpoint{2.759895in}{2.377442in}}%
\pgfpathcurveto{\pgfqpoint{2.759895in}{2.366392in}}{\pgfqpoint{2.764285in}{2.355793in}}{\pgfqpoint{2.772098in}{2.347979in}}%
\pgfpathcurveto{\pgfqpoint{2.779912in}{2.340165in}}{\pgfqpoint{2.790511in}{2.335775in}}{\pgfqpoint{2.801561in}{2.335775in}}%
\pgfpathclose%
\pgfusepath{stroke,fill}%
\end{pgfscope}%
\begin{pgfscope}%
\pgfpathrectangle{\pgfqpoint{0.600000in}{0.600000in}}{\pgfqpoint{3.900000in}{3.900000in}}%
\pgfusepath{clip}%
\pgfsetbuttcap%
\pgfsetroundjoin%
\definecolor{currentfill}{rgb}{0.121569,0.466667,0.705882}%
\pgfsetfillcolor{currentfill}%
\pgfsetlinewidth{1.003750pt}%
\definecolor{currentstroke}{rgb}{0.121569,0.466667,0.705882}%
\pgfsetstrokecolor{currentstroke}%
\pgfsetdash{}{0pt}%
\pgfpathmoveto{\pgfqpoint{2.019808in}{2.312686in}}%
\pgfpathcurveto{\pgfqpoint{2.030858in}{2.312686in}}{\pgfqpoint{2.041457in}{2.317076in}}{\pgfqpoint{2.049270in}{2.324890in}}%
\pgfpathcurveto{\pgfqpoint{2.057084in}{2.332704in}}{\pgfqpoint{2.061474in}{2.343303in}}{\pgfqpoint{2.061474in}{2.354353in}}%
\pgfpathcurveto{\pgfqpoint{2.061474in}{2.365403in}}{\pgfqpoint{2.057084in}{2.376002in}}{\pgfqpoint{2.049270in}{2.383816in}}%
\pgfpathcurveto{\pgfqpoint{2.041457in}{2.391629in}}{\pgfqpoint{2.030858in}{2.396019in}}{\pgfqpoint{2.019808in}{2.396019in}}%
\pgfpathcurveto{\pgfqpoint{2.008758in}{2.396019in}}{\pgfqpoint{1.998159in}{2.391629in}}{\pgfqpoint{1.990345in}{2.383816in}}%
\pgfpathcurveto{\pgfqpoint{1.982531in}{2.376002in}}{\pgfqpoint{1.978141in}{2.365403in}}{\pgfqpoint{1.978141in}{2.354353in}}%
\pgfpathcurveto{\pgfqpoint{1.978141in}{2.343303in}}{\pgfqpoint{1.982531in}{2.332704in}}{\pgfqpoint{1.990345in}{2.324890in}}%
\pgfpathcurveto{\pgfqpoint{1.998159in}{2.317076in}}{\pgfqpoint{2.008758in}{2.312686in}}{\pgfqpoint{2.019808in}{2.312686in}}%
\pgfpathclose%
\pgfusepath{stroke,fill}%
\end{pgfscope}%
\begin{pgfscope}%
\pgfpathrectangle{\pgfqpoint{0.600000in}{0.600000in}}{\pgfqpoint{3.900000in}{3.900000in}}%
\pgfusepath{clip}%
\pgfsetbuttcap%
\pgfsetroundjoin%
\definecolor{currentfill}{rgb}{0.121569,0.466667,0.705882}%
\pgfsetfillcolor{currentfill}%
\pgfsetlinewidth{1.003750pt}%
\definecolor{currentstroke}{rgb}{0.121569,0.466667,0.705882}%
\pgfsetstrokecolor{currentstroke}%
\pgfsetdash{}{0pt}%
\pgfpathmoveto{\pgfqpoint{2.650286in}{2.710772in}}%
\pgfpathcurveto{\pgfqpoint{2.661336in}{2.710772in}}{\pgfqpoint{2.671935in}{2.715162in}}{\pgfqpoint{2.679749in}{2.722976in}}%
\pgfpathcurveto{\pgfqpoint{2.687562in}{2.730789in}}{\pgfqpoint{2.691952in}{2.741388in}}{\pgfqpoint{2.691952in}{2.752439in}}%
\pgfpathcurveto{\pgfqpoint{2.691952in}{2.763489in}}{\pgfqpoint{2.687562in}{2.774088in}}{\pgfqpoint{2.679749in}{2.781901in}}%
\pgfpathcurveto{\pgfqpoint{2.671935in}{2.789715in}}{\pgfqpoint{2.661336in}{2.794105in}}{\pgfqpoint{2.650286in}{2.794105in}}%
\pgfpathcurveto{\pgfqpoint{2.639236in}{2.794105in}}{\pgfqpoint{2.628637in}{2.789715in}}{\pgfqpoint{2.620823in}{2.781901in}}%
\pgfpathcurveto{\pgfqpoint{2.613009in}{2.774088in}}{\pgfqpoint{2.608619in}{2.763489in}}{\pgfqpoint{2.608619in}{2.752439in}}%
\pgfpathcurveto{\pgfqpoint{2.608619in}{2.741388in}}{\pgfqpoint{2.613009in}{2.730789in}}{\pgfqpoint{2.620823in}{2.722976in}}%
\pgfpathcurveto{\pgfqpoint{2.628637in}{2.715162in}}{\pgfqpoint{2.639236in}{2.710772in}}{\pgfqpoint{2.650286in}{2.710772in}}%
\pgfpathclose%
\pgfusepath{stroke,fill}%
\end{pgfscope}%
\begin{pgfscope}%
\pgfpathrectangle{\pgfqpoint{0.600000in}{0.600000in}}{\pgfqpoint{3.900000in}{3.900000in}}%
\pgfusepath{clip}%
\pgfsetbuttcap%
\pgfsetroundjoin%
\definecolor{currentfill}{rgb}{0.121569,0.466667,0.705882}%
\pgfsetfillcolor{currentfill}%
\pgfsetlinewidth{1.003750pt}%
\definecolor{currentstroke}{rgb}{0.121569,0.466667,0.705882}%
\pgfsetstrokecolor{currentstroke}%
\pgfsetdash{}{0pt}%
\pgfpathmoveto{\pgfqpoint{2.625694in}{1.554034in}}%
\pgfpathcurveto{\pgfqpoint{2.636744in}{1.554034in}}{\pgfqpoint{2.647343in}{1.558424in}}{\pgfqpoint{2.655157in}{1.566238in}}%
\pgfpathcurveto{\pgfqpoint{2.662971in}{1.574052in}}{\pgfqpoint{2.667361in}{1.584651in}}{\pgfqpoint{2.667361in}{1.595701in}}%
\pgfpathcurveto{\pgfqpoint{2.667361in}{1.606751in}}{\pgfqpoint{2.662971in}{1.617350in}}{\pgfqpoint{2.655157in}{1.625164in}}%
\pgfpathcurveto{\pgfqpoint{2.647343in}{1.632977in}}{\pgfqpoint{2.636744in}{1.637368in}}{\pgfqpoint{2.625694in}{1.637368in}}%
\pgfpathcurveto{\pgfqpoint{2.614644in}{1.637368in}}{\pgfqpoint{2.604045in}{1.632977in}}{\pgfqpoint{2.596231in}{1.625164in}}%
\pgfpathcurveto{\pgfqpoint{2.588418in}{1.617350in}}{\pgfqpoint{2.584028in}{1.606751in}}{\pgfqpoint{2.584028in}{1.595701in}}%
\pgfpathcurveto{\pgfqpoint{2.584028in}{1.584651in}}{\pgfqpoint{2.588418in}{1.574052in}}{\pgfqpoint{2.596231in}{1.566238in}}%
\pgfpathcurveto{\pgfqpoint{2.604045in}{1.558424in}}{\pgfqpoint{2.614644in}{1.554034in}}{\pgfqpoint{2.625694in}{1.554034in}}%
\pgfpathclose%
\pgfusepath{stroke,fill}%
\end{pgfscope}%
\begin{pgfscope}%
\pgfpathrectangle{\pgfqpoint{0.600000in}{0.600000in}}{\pgfqpoint{3.900000in}{3.900000in}}%
\pgfusepath{clip}%
\pgfsetbuttcap%
\pgfsetroundjoin%
\definecolor{currentfill}{rgb}{0.121569,0.466667,0.705882}%
\pgfsetfillcolor{currentfill}%
\pgfsetlinewidth{1.003750pt}%
\definecolor{currentstroke}{rgb}{0.121569,0.466667,0.705882}%
\pgfsetstrokecolor{currentstroke}%
\pgfsetdash{}{0pt}%
\pgfpathmoveto{\pgfqpoint{2.459012in}{2.381408in}}%
\pgfpathcurveto{\pgfqpoint{2.470063in}{2.381408in}}{\pgfqpoint{2.480662in}{2.385798in}}{\pgfqpoint{2.488475in}{2.393612in}}%
\pgfpathcurveto{\pgfqpoint{2.496289in}{2.401426in}}{\pgfqpoint{2.500679in}{2.412025in}}{\pgfqpoint{2.500679in}{2.423075in}}%
\pgfpathcurveto{\pgfqpoint{2.500679in}{2.434125in}}{\pgfqpoint{2.496289in}{2.444724in}}{\pgfqpoint{2.488475in}{2.452538in}}%
\pgfpathcurveto{\pgfqpoint{2.480662in}{2.460351in}}{\pgfqpoint{2.470063in}{2.464742in}}{\pgfqpoint{2.459012in}{2.464742in}}%
\pgfpathcurveto{\pgfqpoint{2.447962in}{2.464742in}}{\pgfqpoint{2.437363in}{2.460351in}}{\pgfqpoint{2.429550in}{2.452538in}}%
\pgfpathcurveto{\pgfqpoint{2.421736in}{2.444724in}}{\pgfqpoint{2.417346in}{2.434125in}}{\pgfqpoint{2.417346in}{2.423075in}}%
\pgfpathcurveto{\pgfqpoint{2.417346in}{2.412025in}}{\pgfqpoint{2.421736in}{2.401426in}}{\pgfqpoint{2.429550in}{2.393612in}}%
\pgfpathcurveto{\pgfqpoint{2.437363in}{2.385798in}}{\pgfqpoint{2.447962in}{2.381408in}}{\pgfqpoint{2.459012in}{2.381408in}}%
\pgfpathclose%
\pgfusepath{stroke,fill}%
\end{pgfscope}%
\begin{pgfscope}%
\pgfpathrectangle{\pgfqpoint{0.600000in}{0.600000in}}{\pgfqpoint{3.900000in}{3.900000in}}%
\pgfusepath{clip}%
\pgfsetbuttcap%
\pgfsetroundjoin%
\definecolor{currentfill}{rgb}{0.121569,0.466667,0.705882}%
\pgfsetfillcolor{currentfill}%
\pgfsetlinewidth{1.003750pt}%
\definecolor{currentstroke}{rgb}{0.121569,0.466667,0.705882}%
\pgfsetstrokecolor{currentstroke}%
\pgfsetdash{}{0pt}%
\pgfpathmoveto{\pgfqpoint{2.709611in}{1.876851in}}%
\pgfpathcurveto{\pgfqpoint{2.720661in}{1.876851in}}{\pgfqpoint{2.731260in}{1.881241in}}{\pgfqpoint{2.739073in}{1.889055in}}%
\pgfpathcurveto{\pgfqpoint{2.746887in}{1.896868in}}{\pgfqpoint{2.751277in}{1.907467in}}{\pgfqpoint{2.751277in}{1.918517in}}%
\pgfpathcurveto{\pgfqpoint{2.751277in}{1.929568in}}{\pgfqpoint{2.746887in}{1.940167in}}{\pgfqpoint{2.739073in}{1.947980in}}%
\pgfpathcurveto{\pgfqpoint{2.731260in}{1.955794in}}{\pgfqpoint{2.720661in}{1.960184in}}{\pgfqpoint{2.709611in}{1.960184in}}%
\pgfpathcurveto{\pgfqpoint{2.698560in}{1.960184in}}{\pgfqpoint{2.687961in}{1.955794in}}{\pgfqpoint{2.680148in}{1.947980in}}%
\pgfpathcurveto{\pgfqpoint{2.672334in}{1.940167in}}{\pgfqpoint{2.667944in}{1.929568in}}{\pgfqpoint{2.667944in}{1.918517in}}%
\pgfpathcurveto{\pgfqpoint{2.667944in}{1.907467in}}{\pgfqpoint{2.672334in}{1.896868in}}{\pgfqpoint{2.680148in}{1.889055in}}%
\pgfpathcurveto{\pgfqpoint{2.687961in}{1.881241in}}{\pgfqpoint{2.698560in}{1.876851in}}{\pgfqpoint{2.709611in}{1.876851in}}%
\pgfpathclose%
\pgfusepath{stroke,fill}%
\end{pgfscope}%
\begin{pgfscope}%
\pgfpathrectangle{\pgfqpoint{0.600000in}{0.600000in}}{\pgfqpoint{3.900000in}{3.900000in}}%
\pgfusepath{clip}%
\pgfsetbuttcap%
\pgfsetroundjoin%
\definecolor{currentfill}{rgb}{0.121569,0.466667,0.705882}%
\pgfsetfillcolor{currentfill}%
\pgfsetlinewidth{1.003750pt}%
\definecolor{currentstroke}{rgb}{0.121569,0.466667,0.705882}%
\pgfsetstrokecolor{currentstroke}%
\pgfsetdash{}{0pt}%
\pgfpathmoveto{\pgfqpoint{2.485118in}{3.047217in}}%
\pgfpathcurveto{\pgfqpoint{2.496168in}{3.047217in}}{\pgfqpoint{2.506767in}{3.051607in}}{\pgfqpoint{2.514581in}{3.059421in}}%
\pgfpathcurveto{\pgfqpoint{2.522394in}{3.067234in}}{\pgfqpoint{2.526785in}{3.077833in}}{\pgfqpoint{2.526785in}{3.088883in}}%
\pgfpathcurveto{\pgfqpoint{2.526785in}{3.099933in}}{\pgfqpoint{2.522394in}{3.110533in}}{\pgfqpoint{2.514581in}{3.118346in}}%
\pgfpathcurveto{\pgfqpoint{2.506767in}{3.126160in}}{\pgfqpoint{2.496168in}{3.130550in}}{\pgfqpoint{2.485118in}{3.130550in}}%
\pgfpathcurveto{\pgfqpoint{2.474068in}{3.130550in}}{\pgfqpoint{2.463469in}{3.126160in}}{\pgfqpoint{2.455655in}{3.118346in}}%
\pgfpathcurveto{\pgfqpoint{2.447842in}{3.110533in}}{\pgfqpoint{2.443451in}{3.099933in}}{\pgfqpoint{2.443451in}{3.088883in}}%
\pgfpathcurveto{\pgfqpoint{2.443451in}{3.077833in}}{\pgfqpoint{2.447842in}{3.067234in}}{\pgfqpoint{2.455655in}{3.059421in}}%
\pgfpathcurveto{\pgfqpoint{2.463469in}{3.051607in}}{\pgfqpoint{2.474068in}{3.047217in}}{\pgfqpoint{2.485118in}{3.047217in}}%
\pgfpathclose%
\pgfusepath{stroke,fill}%
\end{pgfscope}%
\begin{pgfscope}%
\pgfpathrectangle{\pgfqpoint{0.600000in}{0.600000in}}{\pgfqpoint{3.900000in}{3.900000in}}%
\pgfusepath{clip}%
\pgfsetbuttcap%
\pgfsetroundjoin%
\definecolor{currentfill}{rgb}{0.121569,0.466667,0.705882}%
\pgfsetfillcolor{currentfill}%
\pgfsetlinewidth{1.003750pt}%
\definecolor{currentstroke}{rgb}{0.121569,0.466667,0.705882}%
\pgfsetstrokecolor{currentstroke}%
\pgfsetdash{}{0pt}%
\pgfpathmoveto{\pgfqpoint{2.414775in}{2.708243in}}%
\pgfpathcurveto{\pgfqpoint{2.425825in}{2.708243in}}{\pgfqpoint{2.436424in}{2.712634in}}{\pgfqpoint{2.444238in}{2.720447in}}%
\pgfpathcurveto{\pgfqpoint{2.452051in}{2.728261in}}{\pgfqpoint{2.456442in}{2.738860in}}{\pgfqpoint{2.456442in}{2.749910in}}%
\pgfpathcurveto{\pgfqpoint{2.456442in}{2.760960in}}{\pgfqpoint{2.452051in}{2.771559in}}{\pgfqpoint{2.444238in}{2.779373in}}%
\pgfpathcurveto{\pgfqpoint{2.436424in}{2.787187in}}{\pgfqpoint{2.425825in}{2.791577in}}{\pgfqpoint{2.414775in}{2.791577in}}%
\pgfpathcurveto{\pgfqpoint{2.403725in}{2.791577in}}{\pgfqpoint{2.393126in}{2.787187in}}{\pgfqpoint{2.385312in}{2.779373in}}%
\pgfpathcurveto{\pgfqpoint{2.377499in}{2.771559in}}{\pgfqpoint{2.373108in}{2.760960in}}{\pgfqpoint{2.373108in}{2.749910in}}%
\pgfpathcurveto{\pgfqpoint{2.373108in}{2.738860in}}{\pgfqpoint{2.377499in}{2.728261in}}{\pgfqpoint{2.385312in}{2.720447in}}%
\pgfpathcurveto{\pgfqpoint{2.393126in}{2.712634in}}{\pgfqpoint{2.403725in}{2.708243in}}{\pgfqpoint{2.414775in}{2.708243in}}%
\pgfpathclose%
\pgfusepath{stroke,fill}%
\end{pgfscope}%
\begin{pgfscope}%
\pgfpathrectangle{\pgfqpoint{0.600000in}{0.600000in}}{\pgfqpoint{3.900000in}{3.900000in}}%
\pgfusepath{clip}%
\pgfsetbuttcap%
\pgfsetroundjoin%
\definecolor{currentfill}{rgb}{0.121569,0.466667,0.705882}%
\pgfsetfillcolor{currentfill}%
\pgfsetlinewidth{1.003750pt}%
\definecolor{currentstroke}{rgb}{0.121569,0.466667,0.705882}%
\pgfsetstrokecolor{currentstroke}%
\pgfsetdash{}{0pt}%
\pgfpathmoveto{\pgfqpoint{2.380526in}{2.900869in}}%
\pgfpathcurveto{\pgfqpoint{2.391576in}{2.900869in}}{\pgfqpoint{2.402175in}{2.905259in}}{\pgfqpoint{2.409988in}{2.913072in}}%
\pgfpathcurveto{\pgfqpoint{2.417802in}{2.920886in}}{\pgfqpoint{2.422192in}{2.931485in}}{\pgfqpoint{2.422192in}{2.942535in}}%
\pgfpathcurveto{\pgfqpoint{2.422192in}{2.953585in}}{\pgfqpoint{2.417802in}{2.964184in}}{\pgfqpoint{2.409988in}{2.971998in}}%
\pgfpathcurveto{\pgfqpoint{2.402175in}{2.979812in}}{\pgfqpoint{2.391576in}{2.984202in}}{\pgfqpoint{2.380526in}{2.984202in}}%
\pgfpathcurveto{\pgfqpoint{2.369475in}{2.984202in}}{\pgfqpoint{2.358876in}{2.979812in}}{\pgfqpoint{2.351063in}{2.971998in}}%
\pgfpathcurveto{\pgfqpoint{2.343249in}{2.964184in}}{\pgfqpoint{2.338859in}{2.953585in}}{\pgfqpoint{2.338859in}{2.942535in}}%
\pgfpathcurveto{\pgfqpoint{2.338859in}{2.931485in}}{\pgfqpoint{2.343249in}{2.920886in}}{\pgfqpoint{2.351063in}{2.913072in}}%
\pgfpathcurveto{\pgfqpoint{2.358876in}{2.905259in}}{\pgfqpoint{2.369475in}{2.900869in}}{\pgfqpoint{2.380526in}{2.900869in}}%
\pgfpathclose%
\pgfusepath{stroke,fill}%
\end{pgfscope}%
\begin{pgfscope}%
\pgfpathrectangle{\pgfqpoint{0.600000in}{0.600000in}}{\pgfqpoint{3.900000in}{3.900000in}}%
\pgfusepath{clip}%
\pgfsetbuttcap%
\pgfsetroundjoin%
\definecolor{currentfill}{rgb}{0.121569,0.466667,0.705882}%
\pgfsetfillcolor{currentfill}%
\pgfsetlinewidth{1.003750pt}%
\definecolor{currentstroke}{rgb}{0.121569,0.466667,0.705882}%
\pgfsetstrokecolor{currentstroke}%
\pgfsetdash{}{0pt}%
\pgfpathmoveto{\pgfqpoint{3.608057in}{3.284305in}}%
\pgfpathcurveto{\pgfqpoint{3.619107in}{3.284305in}}{\pgfqpoint{3.629706in}{3.288696in}}{\pgfqpoint{3.637519in}{3.296509in}}%
\pgfpathcurveto{\pgfqpoint{3.645333in}{3.304323in}}{\pgfqpoint{3.649723in}{3.314922in}}{\pgfqpoint{3.649723in}{3.325972in}}%
\pgfpathcurveto{\pgfqpoint{3.649723in}{3.337022in}}{\pgfqpoint{3.645333in}{3.347621in}}{\pgfqpoint{3.637519in}{3.355435in}}%
\pgfpathcurveto{\pgfqpoint{3.629706in}{3.363248in}}{\pgfqpoint{3.619107in}{3.367639in}}{\pgfqpoint{3.608057in}{3.367639in}}%
\pgfpathcurveto{\pgfqpoint{3.597006in}{3.367639in}}{\pgfqpoint{3.586407in}{3.363248in}}{\pgfqpoint{3.578594in}{3.355435in}}%
\pgfpathcurveto{\pgfqpoint{3.570780in}{3.347621in}}{\pgfqpoint{3.566390in}{3.337022in}}{\pgfqpoint{3.566390in}{3.325972in}}%
\pgfpathcurveto{\pgfqpoint{3.566390in}{3.314922in}}{\pgfqpoint{3.570780in}{3.304323in}}{\pgfqpoint{3.578594in}{3.296509in}}%
\pgfpathcurveto{\pgfqpoint{3.586407in}{3.288696in}}{\pgfqpoint{3.597006in}{3.284305in}}{\pgfqpoint{3.608057in}{3.284305in}}%
\pgfpathclose%
\pgfusepath{stroke,fill}%
\end{pgfscope}%
\begin{pgfscope}%
\pgfpathrectangle{\pgfqpoint{0.600000in}{0.600000in}}{\pgfqpoint{3.900000in}{3.900000in}}%
\pgfusepath{clip}%
\pgfsetbuttcap%
\pgfsetroundjoin%
\definecolor{currentfill}{rgb}{0.121569,0.466667,0.705882}%
\pgfsetfillcolor{currentfill}%
\pgfsetlinewidth{1.003750pt}%
\definecolor{currentstroke}{rgb}{0.121569,0.466667,0.705882}%
\pgfsetstrokecolor{currentstroke}%
\pgfsetdash{}{0pt}%
\pgfpathmoveto{\pgfqpoint{1.959323in}{2.526361in}}%
\pgfpathcurveto{\pgfqpoint{1.970373in}{2.526361in}}{\pgfqpoint{1.980972in}{2.530751in}}{\pgfqpoint{1.988786in}{2.538564in}}%
\pgfpathcurveto{\pgfqpoint{1.996600in}{2.546378in}}{\pgfqpoint{2.000990in}{2.556977in}}{\pgfqpoint{2.000990in}{2.568027in}}%
\pgfpathcurveto{\pgfqpoint{2.000990in}{2.579077in}}{\pgfqpoint{1.996600in}{2.589676in}}{\pgfqpoint{1.988786in}{2.597490in}}%
\pgfpathcurveto{\pgfqpoint{1.980972in}{2.605304in}}{\pgfqpoint{1.970373in}{2.609694in}}{\pgfqpoint{1.959323in}{2.609694in}}%
\pgfpathcurveto{\pgfqpoint{1.948273in}{2.609694in}}{\pgfqpoint{1.937674in}{2.605304in}}{\pgfqpoint{1.929860in}{2.597490in}}%
\pgfpathcurveto{\pgfqpoint{1.922047in}{2.589676in}}{\pgfqpoint{1.917657in}{2.579077in}}{\pgfqpoint{1.917657in}{2.568027in}}%
\pgfpathcurveto{\pgfqpoint{1.917657in}{2.556977in}}{\pgfqpoint{1.922047in}{2.546378in}}{\pgfqpoint{1.929860in}{2.538564in}}%
\pgfpathcurveto{\pgfqpoint{1.937674in}{2.530751in}}{\pgfqpoint{1.948273in}{2.526361in}}{\pgfqpoint{1.959323in}{2.526361in}}%
\pgfpathclose%
\pgfusepath{stroke,fill}%
\end{pgfscope}%
\begin{pgfscope}%
\pgfpathrectangle{\pgfqpoint{0.600000in}{0.600000in}}{\pgfqpoint{3.900000in}{3.900000in}}%
\pgfusepath{clip}%
\pgfsetbuttcap%
\pgfsetroundjoin%
\definecolor{currentfill}{rgb}{0.121569,0.466667,0.705882}%
\pgfsetfillcolor{currentfill}%
\pgfsetlinewidth{1.003750pt}%
\definecolor{currentstroke}{rgb}{0.121569,0.466667,0.705882}%
\pgfsetstrokecolor{currentstroke}%
\pgfsetdash{}{0pt}%
\pgfpathmoveto{\pgfqpoint{2.342287in}{2.970252in}}%
\pgfpathcurveto{\pgfqpoint{2.353337in}{2.970252in}}{\pgfqpoint{2.363936in}{2.974642in}}{\pgfqpoint{2.371750in}{2.982456in}}%
\pgfpathcurveto{\pgfqpoint{2.379564in}{2.990270in}}{\pgfqpoint{2.383954in}{3.000869in}}{\pgfqpoint{2.383954in}{3.011919in}}%
\pgfpathcurveto{\pgfqpoint{2.383954in}{3.022969in}}{\pgfqpoint{2.379564in}{3.033568in}}{\pgfqpoint{2.371750in}{3.041381in}}%
\pgfpathcurveto{\pgfqpoint{2.363936in}{3.049195in}}{\pgfqpoint{2.353337in}{3.053585in}}{\pgfqpoint{2.342287in}{3.053585in}}%
\pgfpathcurveto{\pgfqpoint{2.331237in}{3.053585in}}{\pgfqpoint{2.320638in}{3.049195in}}{\pgfqpoint{2.312824in}{3.041381in}}%
\pgfpathcurveto{\pgfqpoint{2.305011in}{3.033568in}}{\pgfqpoint{2.300621in}{3.022969in}}{\pgfqpoint{2.300621in}{3.011919in}}%
\pgfpathcurveto{\pgfqpoint{2.300621in}{3.000869in}}{\pgfqpoint{2.305011in}{2.990270in}}{\pgfqpoint{2.312824in}{2.982456in}}%
\pgfpathcurveto{\pgfqpoint{2.320638in}{2.974642in}}{\pgfqpoint{2.331237in}{2.970252in}}{\pgfqpoint{2.342287in}{2.970252in}}%
\pgfpathclose%
\pgfusepath{stroke,fill}%
\end{pgfscope}%
\begin{pgfscope}%
\pgfpathrectangle{\pgfqpoint{0.600000in}{0.600000in}}{\pgfqpoint{3.900000in}{3.900000in}}%
\pgfusepath{clip}%
\pgfsetbuttcap%
\pgfsetroundjoin%
\definecolor{currentfill}{rgb}{0.121569,0.466667,0.705882}%
\pgfsetfillcolor{currentfill}%
\pgfsetlinewidth{1.003750pt}%
\definecolor{currentstroke}{rgb}{0.121569,0.466667,0.705882}%
\pgfsetstrokecolor{currentstroke}%
\pgfsetdash{}{0pt}%
\pgfpathmoveto{\pgfqpoint{3.206945in}{1.633777in}}%
\pgfpathcurveto{\pgfqpoint{3.217995in}{1.633777in}}{\pgfqpoint{3.228594in}{1.638167in}}{\pgfqpoint{3.236407in}{1.645981in}}%
\pgfpathcurveto{\pgfqpoint{3.244221in}{1.653794in}}{\pgfqpoint{3.248611in}{1.664393in}}{\pgfqpoint{3.248611in}{1.675444in}}%
\pgfpathcurveto{\pgfqpoint{3.248611in}{1.686494in}}{\pgfqpoint{3.244221in}{1.697093in}}{\pgfqpoint{3.236407in}{1.704906in}}%
\pgfpathcurveto{\pgfqpoint{3.228594in}{1.712720in}}{\pgfqpoint{3.217995in}{1.717110in}}{\pgfqpoint{3.206945in}{1.717110in}}%
\pgfpathcurveto{\pgfqpoint{3.195894in}{1.717110in}}{\pgfqpoint{3.185295in}{1.712720in}}{\pgfqpoint{3.177482in}{1.704906in}}%
\pgfpathcurveto{\pgfqpoint{3.169668in}{1.697093in}}{\pgfqpoint{3.165278in}{1.686494in}}{\pgfqpoint{3.165278in}{1.675444in}}%
\pgfpathcurveto{\pgfqpoint{3.165278in}{1.664393in}}{\pgfqpoint{3.169668in}{1.653794in}}{\pgfqpoint{3.177482in}{1.645981in}}%
\pgfpathcurveto{\pgfqpoint{3.185295in}{1.638167in}}{\pgfqpoint{3.195894in}{1.633777in}}{\pgfqpoint{3.206945in}{1.633777in}}%
\pgfpathclose%
\pgfusepath{stroke,fill}%
\end{pgfscope}%
\begin{pgfscope}%
\pgfpathrectangle{\pgfqpoint{0.600000in}{0.600000in}}{\pgfqpoint{3.900000in}{3.900000in}}%
\pgfusepath{clip}%
\pgfsetbuttcap%
\pgfsetroundjoin%
\definecolor{currentfill}{rgb}{0.121569,0.466667,0.705882}%
\pgfsetfillcolor{currentfill}%
\pgfsetlinewidth{1.003750pt}%
\definecolor{currentstroke}{rgb}{0.121569,0.466667,0.705882}%
\pgfsetstrokecolor{currentstroke}%
\pgfsetdash{}{0pt}%
\pgfpathmoveto{\pgfqpoint{2.754076in}{1.919521in}}%
\pgfpathcurveto{\pgfqpoint{2.765126in}{1.919521in}}{\pgfqpoint{2.775725in}{1.923911in}}{\pgfqpoint{2.783539in}{1.931725in}}%
\pgfpathcurveto{\pgfqpoint{2.791352in}{1.939538in}}{\pgfqpoint{2.795743in}{1.950138in}}{\pgfqpoint{2.795743in}{1.961188in}}%
\pgfpathcurveto{\pgfqpoint{2.795743in}{1.972238in}}{\pgfqpoint{2.791352in}{1.982837in}}{\pgfqpoint{2.783539in}{1.990650in}}%
\pgfpathcurveto{\pgfqpoint{2.775725in}{1.998464in}}{\pgfqpoint{2.765126in}{2.002854in}}{\pgfqpoint{2.754076in}{2.002854in}}%
\pgfpathcurveto{\pgfqpoint{2.743026in}{2.002854in}}{\pgfqpoint{2.732427in}{1.998464in}}{\pgfqpoint{2.724613in}{1.990650in}}%
\pgfpathcurveto{\pgfqpoint{2.716800in}{1.982837in}}{\pgfqpoint{2.712409in}{1.972238in}}{\pgfqpoint{2.712409in}{1.961188in}}%
\pgfpathcurveto{\pgfqpoint{2.712409in}{1.950138in}}{\pgfqpoint{2.716800in}{1.939538in}}{\pgfqpoint{2.724613in}{1.931725in}}%
\pgfpathcurveto{\pgfqpoint{2.732427in}{1.923911in}}{\pgfqpoint{2.743026in}{1.919521in}}{\pgfqpoint{2.754076in}{1.919521in}}%
\pgfpathclose%
\pgfusepath{stroke,fill}%
\end{pgfscope}%
\begin{pgfscope}%
\pgfpathrectangle{\pgfqpoint{0.600000in}{0.600000in}}{\pgfqpoint{3.900000in}{3.900000in}}%
\pgfusepath{clip}%
\pgfsetbuttcap%
\pgfsetroundjoin%
\definecolor{currentfill}{rgb}{0.121569,0.466667,0.705882}%
\pgfsetfillcolor{currentfill}%
\pgfsetlinewidth{1.003750pt}%
\definecolor{currentstroke}{rgb}{0.121569,0.466667,0.705882}%
\pgfsetstrokecolor{currentstroke}%
\pgfsetdash{}{0pt}%
\pgfpathmoveto{\pgfqpoint{2.294277in}{2.645795in}}%
\pgfpathcurveto{\pgfqpoint{2.305327in}{2.645795in}}{\pgfqpoint{2.315926in}{2.650185in}}{\pgfqpoint{2.323739in}{2.657998in}}%
\pgfpathcurveto{\pgfqpoint{2.331553in}{2.665812in}}{\pgfqpoint{2.335943in}{2.676411in}}{\pgfqpoint{2.335943in}{2.687461in}}%
\pgfpathcurveto{\pgfqpoint{2.335943in}{2.698511in}}{\pgfqpoint{2.331553in}{2.709110in}}{\pgfqpoint{2.323739in}{2.716924in}}%
\pgfpathcurveto{\pgfqpoint{2.315926in}{2.724738in}}{\pgfqpoint{2.305327in}{2.729128in}}{\pgfqpoint{2.294277in}{2.729128in}}%
\pgfpathcurveto{\pgfqpoint{2.283227in}{2.729128in}}{\pgfqpoint{2.272628in}{2.724738in}}{\pgfqpoint{2.264814in}{2.716924in}}%
\pgfpathcurveto{\pgfqpoint{2.257000in}{2.709110in}}{\pgfqpoint{2.252610in}{2.698511in}}{\pgfqpoint{2.252610in}{2.687461in}}%
\pgfpathcurveto{\pgfqpoint{2.252610in}{2.676411in}}{\pgfqpoint{2.257000in}{2.665812in}}{\pgfqpoint{2.264814in}{2.657998in}}%
\pgfpathcurveto{\pgfqpoint{2.272628in}{2.650185in}}{\pgfqpoint{2.283227in}{2.645795in}}{\pgfqpoint{2.294277in}{2.645795in}}%
\pgfpathclose%
\pgfusepath{stroke,fill}%
\end{pgfscope}%
\begin{pgfscope}%
\pgfpathrectangle{\pgfqpoint{0.600000in}{0.600000in}}{\pgfqpoint{3.900000in}{3.900000in}}%
\pgfusepath{clip}%
\pgfsetbuttcap%
\pgfsetroundjoin%
\definecolor{currentfill}{rgb}{0.121569,0.466667,0.705882}%
\pgfsetfillcolor{currentfill}%
\pgfsetlinewidth{1.003750pt}%
\definecolor{currentstroke}{rgb}{0.121569,0.466667,0.705882}%
\pgfsetstrokecolor{currentstroke}%
\pgfsetdash{}{0pt}%
\pgfpathmoveto{\pgfqpoint{2.469471in}{2.623731in}}%
\pgfpathcurveto{\pgfqpoint{2.480521in}{2.623731in}}{\pgfqpoint{2.491120in}{2.628122in}}{\pgfqpoint{2.498934in}{2.635935in}}%
\pgfpathcurveto{\pgfqpoint{2.506748in}{2.643749in}}{\pgfqpoint{2.511138in}{2.654348in}}{\pgfqpoint{2.511138in}{2.665398in}}%
\pgfpathcurveto{\pgfqpoint{2.511138in}{2.676448in}}{\pgfqpoint{2.506748in}{2.687047in}}{\pgfqpoint{2.498934in}{2.694861in}}%
\pgfpathcurveto{\pgfqpoint{2.491120in}{2.702674in}}{\pgfqpoint{2.480521in}{2.707065in}}{\pgfqpoint{2.469471in}{2.707065in}}%
\pgfpathcurveto{\pgfqpoint{2.458421in}{2.707065in}}{\pgfqpoint{2.447822in}{2.702674in}}{\pgfqpoint{2.440009in}{2.694861in}}%
\pgfpathcurveto{\pgfqpoint{2.432195in}{2.687047in}}{\pgfqpoint{2.427805in}{2.676448in}}{\pgfqpoint{2.427805in}{2.665398in}}%
\pgfpathcurveto{\pgfqpoint{2.427805in}{2.654348in}}{\pgfqpoint{2.432195in}{2.643749in}}{\pgfqpoint{2.440009in}{2.635935in}}%
\pgfpathcurveto{\pgfqpoint{2.447822in}{2.628122in}}{\pgfqpoint{2.458421in}{2.623731in}}{\pgfqpoint{2.469471in}{2.623731in}}%
\pgfpathclose%
\pgfusepath{stroke,fill}%
\end{pgfscope}%
\begin{pgfscope}%
\pgfpathrectangle{\pgfqpoint{0.600000in}{0.600000in}}{\pgfqpoint{3.900000in}{3.900000in}}%
\pgfusepath{clip}%
\pgfsetbuttcap%
\pgfsetroundjoin%
\definecolor{currentfill}{rgb}{0.121569,0.466667,0.705882}%
\pgfsetfillcolor{currentfill}%
\pgfsetlinewidth{1.003750pt}%
\definecolor{currentstroke}{rgb}{0.121569,0.466667,0.705882}%
\pgfsetstrokecolor{currentstroke}%
\pgfsetdash{}{0pt}%
\pgfpathmoveto{\pgfqpoint{2.700402in}{2.838454in}}%
\pgfpathcurveto{\pgfqpoint{2.711452in}{2.838454in}}{\pgfqpoint{2.722051in}{2.842844in}}{\pgfqpoint{2.729865in}{2.850658in}}%
\pgfpathcurveto{\pgfqpoint{2.737679in}{2.858471in}}{\pgfqpoint{2.742069in}{2.869070in}}{\pgfqpoint{2.742069in}{2.880121in}}%
\pgfpathcurveto{\pgfqpoint{2.742069in}{2.891171in}}{\pgfqpoint{2.737679in}{2.901770in}}{\pgfqpoint{2.729865in}{2.909583in}}%
\pgfpathcurveto{\pgfqpoint{2.722051in}{2.917397in}}{\pgfqpoint{2.711452in}{2.921787in}}{\pgfqpoint{2.700402in}{2.921787in}}%
\pgfpathcurveto{\pgfqpoint{2.689352in}{2.921787in}}{\pgfqpoint{2.678753in}{2.917397in}}{\pgfqpoint{2.670940in}{2.909583in}}%
\pgfpathcurveto{\pgfqpoint{2.663126in}{2.901770in}}{\pgfqpoint{2.658736in}{2.891171in}}{\pgfqpoint{2.658736in}{2.880121in}}%
\pgfpathcurveto{\pgfqpoint{2.658736in}{2.869070in}}{\pgfqpoint{2.663126in}{2.858471in}}{\pgfqpoint{2.670940in}{2.850658in}}%
\pgfpathcurveto{\pgfqpoint{2.678753in}{2.842844in}}{\pgfqpoint{2.689352in}{2.838454in}}{\pgfqpoint{2.700402in}{2.838454in}}%
\pgfpathclose%
\pgfusepath{stroke,fill}%
\end{pgfscope}%
\begin{pgfscope}%
\pgfpathrectangle{\pgfqpoint{0.600000in}{0.600000in}}{\pgfqpoint{3.900000in}{3.900000in}}%
\pgfusepath{clip}%
\pgfsetbuttcap%
\pgfsetroundjoin%
\definecolor{currentfill}{rgb}{0.121569,0.466667,0.705882}%
\pgfsetfillcolor{currentfill}%
\pgfsetlinewidth{1.003750pt}%
\definecolor{currentstroke}{rgb}{0.121569,0.466667,0.705882}%
\pgfsetstrokecolor{currentstroke}%
\pgfsetdash{}{0pt}%
\pgfpathmoveto{\pgfqpoint{2.513389in}{2.360463in}}%
\pgfpathcurveto{\pgfqpoint{2.524439in}{2.360463in}}{\pgfqpoint{2.535038in}{2.364853in}}{\pgfqpoint{2.542852in}{2.372667in}}%
\pgfpathcurveto{\pgfqpoint{2.550666in}{2.380480in}}{\pgfqpoint{2.555056in}{2.391079in}}{\pgfqpoint{2.555056in}{2.402129in}}%
\pgfpathcurveto{\pgfqpoint{2.555056in}{2.413179in}}{\pgfqpoint{2.550666in}{2.423778in}}{\pgfqpoint{2.542852in}{2.431592in}}%
\pgfpathcurveto{\pgfqpoint{2.535038in}{2.439406in}}{\pgfqpoint{2.524439in}{2.443796in}}{\pgfqpoint{2.513389in}{2.443796in}}%
\pgfpathcurveto{\pgfqpoint{2.502339in}{2.443796in}}{\pgfqpoint{2.491740in}{2.439406in}}{\pgfqpoint{2.483927in}{2.431592in}}%
\pgfpathcurveto{\pgfqpoint{2.476113in}{2.423778in}}{\pgfqpoint{2.471723in}{2.413179in}}{\pgfqpoint{2.471723in}{2.402129in}}%
\pgfpathcurveto{\pgfqpoint{2.471723in}{2.391079in}}{\pgfqpoint{2.476113in}{2.380480in}}{\pgfqpoint{2.483927in}{2.372667in}}%
\pgfpathcurveto{\pgfqpoint{2.491740in}{2.364853in}}{\pgfqpoint{2.502339in}{2.360463in}}{\pgfqpoint{2.513389in}{2.360463in}}%
\pgfpathclose%
\pgfusepath{stroke,fill}%
\end{pgfscope}%
\begin{pgfscope}%
\pgfpathrectangle{\pgfqpoint{0.600000in}{0.600000in}}{\pgfqpoint{3.900000in}{3.900000in}}%
\pgfusepath{clip}%
\pgfsetbuttcap%
\pgfsetroundjoin%
\definecolor{currentfill}{rgb}{0.121569,0.466667,0.705882}%
\pgfsetfillcolor{currentfill}%
\pgfsetlinewidth{1.003750pt}%
\definecolor{currentstroke}{rgb}{0.121569,0.466667,0.705882}%
\pgfsetstrokecolor{currentstroke}%
\pgfsetdash{}{0pt}%
\pgfpathmoveto{\pgfqpoint{3.512105in}{2.733876in}}%
\pgfpathcurveto{\pgfqpoint{3.523155in}{2.733876in}}{\pgfqpoint{3.533754in}{2.738266in}}{\pgfqpoint{3.541567in}{2.746080in}}%
\pgfpathcurveto{\pgfqpoint{3.549381in}{2.753894in}}{\pgfqpoint{3.553771in}{2.764493in}}{\pgfqpoint{3.553771in}{2.775543in}}%
\pgfpathcurveto{\pgfqpoint{3.553771in}{2.786593in}}{\pgfqpoint{3.549381in}{2.797192in}}{\pgfqpoint{3.541567in}{2.805006in}}%
\pgfpathcurveto{\pgfqpoint{3.533754in}{2.812819in}}{\pgfqpoint{3.523155in}{2.817210in}}{\pgfqpoint{3.512105in}{2.817210in}}%
\pgfpathcurveto{\pgfqpoint{3.501055in}{2.817210in}}{\pgfqpoint{3.490455in}{2.812819in}}{\pgfqpoint{3.482642in}{2.805006in}}%
\pgfpathcurveto{\pgfqpoint{3.474828in}{2.797192in}}{\pgfqpoint{3.470438in}{2.786593in}}{\pgfqpoint{3.470438in}{2.775543in}}%
\pgfpathcurveto{\pgfqpoint{3.470438in}{2.764493in}}{\pgfqpoint{3.474828in}{2.753894in}}{\pgfqpoint{3.482642in}{2.746080in}}%
\pgfpathcurveto{\pgfqpoint{3.490455in}{2.738266in}}{\pgfqpoint{3.501055in}{2.733876in}}{\pgfqpoint{3.512105in}{2.733876in}}%
\pgfpathclose%
\pgfusepath{stroke,fill}%
\end{pgfscope}%
\begin{pgfscope}%
\pgfpathrectangle{\pgfqpoint{0.600000in}{0.600000in}}{\pgfqpoint{3.900000in}{3.900000in}}%
\pgfusepath{clip}%
\pgfsetbuttcap%
\pgfsetroundjoin%
\definecolor{currentfill}{rgb}{0.121569,0.466667,0.705882}%
\pgfsetfillcolor{currentfill}%
\pgfsetlinewidth{1.003750pt}%
\definecolor{currentstroke}{rgb}{0.121569,0.466667,0.705882}%
\pgfsetstrokecolor{currentstroke}%
\pgfsetdash{}{0pt}%
\pgfpathmoveto{\pgfqpoint{2.556384in}{3.352235in}}%
\pgfpathcurveto{\pgfqpoint{2.567434in}{3.352235in}}{\pgfqpoint{2.578033in}{3.356625in}}{\pgfqpoint{2.585846in}{3.364439in}}%
\pgfpathcurveto{\pgfqpoint{2.593660in}{3.372253in}}{\pgfqpoint{2.598050in}{3.382852in}}{\pgfqpoint{2.598050in}{3.393902in}}%
\pgfpathcurveto{\pgfqpoint{2.598050in}{3.404952in}}{\pgfqpoint{2.593660in}{3.415551in}}{\pgfqpoint{2.585846in}{3.423365in}}%
\pgfpathcurveto{\pgfqpoint{2.578033in}{3.431178in}}{\pgfqpoint{2.567434in}{3.435568in}}{\pgfqpoint{2.556384in}{3.435568in}}%
\pgfpathcurveto{\pgfqpoint{2.545333in}{3.435568in}}{\pgfqpoint{2.534734in}{3.431178in}}{\pgfqpoint{2.526921in}{3.423365in}}%
\pgfpathcurveto{\pgfqpoint{2.519107in}{3.415551in}}{\pgfqpoint{2.514717in}{3.404952in}}{\pgfqpoint{2.514717in}{3.393902in}}%
\pgfpathcurveto{\pgfqpoint{2.514717in}{3.382852in}}{\pgfqpoint{2.519107in}{3.372253in}}{\pgfqpoint{2.526921in}{3.364439in}}%
\pgfpathcurveto{\pgfqpoint{2.534734in}{3.356625in}}{\pgfqpoint{2.545333in}{3.352235in}}{\pgfqpoint{2.556384in}{3.352235in}}%
\pgfpathclose%
\pgfusepath{stroke,fill}%
\end{pgfscope}%
\begin{pgfscope}%
\pgfpathrectangle{\pgfqpoint{0.600000in}{0.600000in}}{\pgfqpoint{3.900000in}{3.900000in}}%
\pgfusepath{clip}%
\pgfsetbuttcap%
\pgfsetroundjoin%
\definecolor{currentfill}{rgb}{0.121569,0.466667,0.705882}%
\pgfsetfillcolor{currentfill}%
\pgfsetlinewidth{1.003750pt}%
\definecolor{currentstroke}{rgb}{0.121569,0.466667,0.705882}%
\pgfsetstrokecolor{currentstroke}%
\pgfsetdash{}{0pt}%
\pgfpathmoveto{\pgfqpoint{3.067323in}{2.298885in}}%
\pgfpathcurveto{\pgfqpoint{3.078373in}{2.298885in}}{\pgfqpoint{3.088972in}{2.303275in}}{\pgfqpoint{3.096786in}{2.311089in}}%
\pgfpathcurveto{\pgfqpoint{3.104600in}{2.318903in}}{\pgfqpoint{3.108990in}{2.329502in}}{\pgfqpoint{3.108990in}{2.340552in}}%
\pgfpathcurveto{\pgfqpoint{3.108990in}{2.351602in}}{\pgfqpoint{3.104600in}{2.362201in}}{\pgfqpoint{3.096786in}{2.370015in}}%
\pgfpathcurveto{\pgfqpoint{3.088972in}{2.377828in}}{\pgfqpoint{3.078373in}{2.382219in}}{\pgfqpoint{3.067323in}{2.382219in}}%
\pgfpathcurveto{\pgfqpoint{3.056273in}{2.382219in}}{\pgfqpoint{3.045674in}{2.377828in}}{\pgfqpoint{3.037860in}{2.370015in}}%
\pgfpathcurveto{\pgfqpoint{3.030047in}{2.362201in}}{\pgfqpoint{3.025657in}{2.351602in}}{\pgfqpoint{3.025657in}{2.340552in}}%
\pgfpathcurveto{\pgfqpoint{3.025657in}{2.329502in}}{\pgfqpoint{3.030047in}{2.318903in}}{\pgfqpoint{3.037860in}{2.311089in}}%
\pgfpathcurveto{\pgfqpoint{3.045674in}{2.303275in}}{\pgfqpoint{3.056273in}{2.298885in}}{\pgfqpoint{3.067323in}{2.298885in}}%
\pgfpathclose%
\pgfusepath{stroke,fill}%
\end{pgfscope}%
\begin{pgfscope}%
\pgfpathrectangle{\pgfqpoint{0.600000in}{0.600000in}}{\pgfqpoint{3.900000in}{3.900000in}}%
\pgfusepath{clip}%
\pgfsetbuttcap%
\pgfsetroundjoin%
\definecolor{currentfill}{rgb}{0.121569,0.466667,0.705882}%
\pgfsetfillcolor{currentfill}%
\pgfsetlinewidth{1.003750pt}%
\definecolor{currentstroke}{rgb}{0.121569,0.466667,0.705882}%
\pgfsetstrokecolor{currentstroke}%
\pgfsetdash{}{0pt}%
\pgfpathmoveto{\pgfqpoint{2.645916in}{3.357942in}}%
\pgfpathcurveto{\pgfqpoint{2.656966in}{3.357942in}}{\pgfqpoint{2.667565in}{3.362333in}}{\pgfqpoint{2.675379in}{3.370146in}}%
\pgfpathcurveto{\pgfqpoint{2.683192in}{3.377960in}}{\pgfqpoint{2.687583in}{3.388559in}}{\pgfqpoint{2.687583in}{3.399609in}}%
\pgfpathcurveto{\pgfqpoint{2.687583in}{3.410659in}}{\pgfqpoint{2.683192in}{3.421258in}}{\pgfqpoint{2.675379in}{3.429072in}}%
\pgfpathcurveto{\pgfqpoint{2.667565in}{3.436885in}}{\pgfqpoint{2.656966in}{3.441276in}}{\pgfqpoint{2.645916in}{3.441276in}}%
\pgfpathcurveto{\pgfqpoint{2.634866in}{3.441276in}}{\pgfqpoint{2.624267in}{3.436885in}}{\pgfqpoint{2.616453in}{3.429072in}}%
\pgfpathcurveto{\pgfqpoint{2.608640in}{3.421258in}}{\pgfqpoint{2.604249in}{3.410659in}}{\pgfqpoint{2.604249in}{3.399609in}}%
\pgfpathcurveto{\pgfqpoint{2.604249in}{3.388559in}}{\pgfqpoint{2.608640in}{3.377960in}}{\pgfqpoint{2.616453in}{3.370146in}}%
\pgfpathcurveto{\pgfqpoint{2.624267in}{3.362333in}}{\pgfqpoint{2.634866in}{3.357942in}}{\pgfqpoint{2.645916in}{3.357942in}}%
\pgfpathclose%
\pgfusepath{stroke,fill}%
\end{pgfscope}%
\begin{pgfscope}%
\pgfpathrectangle{\pgfqpoint{0.600000in}{0.600000in}}{\pgfqpoint{3.900000in}{3.900000in}}%
\pgfusepath{clip}%
\pgfsetbuttcap%
\pgfsetroundjoin%
\definecolor{currentfill}{rgb}{0.121569,0.466667,0.705882}%
\pgfsetfillcolor{currentfill}%
\pgfsetlinewidth{1.003750pt}%
\definecolor{currentstroke}{rgb}{0.121569,0.466667,0.705882}%
\pgfsetstrokecolor{currentstroke}%
\pgfsetdash{}{0pt}%
\pgfpathmoveto{\pgfqpoint{3.522336in}{1.849630in}}%
\pgfpathcurveto{\pgfqpoint{3.533386in}{1.849630in}}{\pgfqpoint{3.543985in}{1.854020in}}{\pgfqpoint{3.551798in}{1.861834in}}%
\pgfpathcurveto{\pgfqpoint{3.559612in}{1.869648in}}{\pgfqpoint{3.564002in}{1.880247in}}{\pgfqpoint{3.564002in}{1.891297in}}%
\pgfpathcurveto{\pgfqpoint{3.564002in}{1.902347in}}{\pgfqpoint{3.559612in}{1.912946in}}{\pgfqpoint{3.551798in}{1.920759in}}%
\pgfpathcurveto{\pgfqpoint{3.543985in}{1.928573in}}{\pgfqpoint{3.533386in}{1.932963in}}{\pgfqpoint{3.522336in}{1.932963in}}%
\pgfpathcurveto{\pgfqpoint{3.511285in}{1.932963in}}{\pgfqpoint{3.500686in}{1.928573in}}{\pgfqpoint{3.492873in}{1.920759in}}%
\pgfpathcurveto{\pgfqpoint{3.485059in}{1.912946in}}{\pgfqpoint{3.480669in}{1.902347in}}{\pgfqpoint{3.480669in}{1.891297in}}%
\pgfpathcurveto{\pgfqpoint{3.480669in}{1.880247in}}{\pgfqpoint{3.485059in}{1.869648in}}{\pgfqpoint{3.492873in}{1.861834in}}%
\pgfpathcurveto{\pgfqpoint{3.500686in}{1.854020in}}{\pgfqpoint{3.511285in}{1.849630in}}{\pgfqpoint{3.522336in}{1.849630in}}%
\pgfpathclose%
\pgfusepath{stroke,fill}%
\end{pgfscope}%
\begin{pgfscope}%
\pgfpathrectangle{\pgfqpoint{0.600000in}{0.600000in}}{\pgfqpoint{3.900000in}{3.900000in}}%
\pgfusepath{clip}%
\pgfsetbuttcap%
\pgfsetroundjoin%
\definecolor{currentfill}{rgb}{0.121569,0.466667,0.705882}%
\pgfsetfillcolor{currentfill}%
\pgfsetlinewidth{1.003750pt}%
\definecolor{currentstroke}{rgb}{0.121569,0.466667,0.705882}%
\pgfsetstrokecolor{currentstroke}%
\pgfsetdash{}{0pt}%
\pgfpathmoveto{\pgfqpoint{3.052025in}{2.414455in}}%
\pgfpathcurveto{\pgfqpoint{3.063075in}{2.414455in}}{\pgfqpoint{3.073674in}{2.418846in}}{\pgfqpoint{3.081488in}{2.426659in}}%
\pgfpathcurveto{\pgfqpoint{3.089301in}{2.434473in}}{\pgfqpoint{3.093692in}{2.445072in}}{\pgfqpoint{3.093692in}{2.456122in}}%
\pgfpathcurveto{\pgfqpoint{3.093692in}{2.467172in}}{\pgfqpoint{3.089301in}{2.477771in}}{\pgfqpoint{3.081488in}{2.485585in}}%
\pgfpathcurveto{\pgfqpoint{3.073674in}{2.493398in}}{\pgfqpoint{3.063075in}{2.497789in}}{\pgfqpoint{3.052025in}{2.497789in}}%
\pgfpathcurveto{\pgfqpoint{3.040975in}{2.497789in}}{\pgfqpoint{3.030376in}{2.493398in}}{\pgfqpoint{3.022562in}{2.485585in}}%
\pgfpathcurveto{\pgfqpoint{3.014749in}{2.477771in}}{\pgfqpoint{3.010358in}{2.467172in}}{\pgfqpoint{3.010358in}{2.456122in}}%
\pgfpathcurveto{\pgfqpoint{3.010358in}{2.445072in}}{\pgfqpoint{3.014749in}{2.434473in}}{\pgfqpoint{3.022562in}{2.426659in}}%
\pgfpathcurveto{\pgfqpoint{3.030376in}{2.418846in}}{\pgfqpoint{3.040975in}{2.414455in}}{\pgfqpoint{3.052025in}{2.414455in}}%
\pgfpathclose%
\pgfusepath{stroke,fill}%
\end{pgfscope}%
\begin{pgfscope}%
\pgfpathrectangle{\pgfqpoint{0.600000in}{0.600000in}}{\pgfqpoint{3.900000in}{3.900000in}}%
\pgfusepath{clip}%
\pgfsetbuttcap%
\pgfsetroundjoin%
\definecolor{currentfill}{rgb}{0.121569,0.466667,0.705882}%
\pgfsetfillcolor{currentfill}%
\pgfsetlinewidth{1.003750pt}%
\definecolor{currentstroke}{rgb}{0.121569,0.466667,0.705882}%
\pgfsetstrokecolor{currentstroke}%
\pgfsetdash{}{0pt}%
\pgfpathmoveto{\pgfqpoint{2.043323in}{2.409958in}}%
\pgfpathcurveto{\pgfqpoint{2.054374in}{2.409958in}}{\pgfqpoint{2.064973in}{2.414348in}}{\pgfqpoint{2.072786in}{2.422161in}}%
\pgfpathcurveto{\pgfqpoint{2.080600in}{2.429975in}}{\pgfqpoint{2.084990in}{2.440574in}}{\pgfqpoint{2.084990in}{2.451624in}}%
\pgfpathcurveto{\pgfqpoint{2.084990in}{2.462674in}}{\pgfqpoint{2.080600in}{2.473273in}}{\pgfqpoint{2.072786in}{2.481087in}}%
\pgfpathcurveto{\pgfqpoint{2.064973in}{2.488901in}}{\pgfqpoint{2.054374in}{2.493291in}}{\pgfqpoint{2.043323in}{2.493291in}}%
\pgfpathcurveto{\pgfqpoint{2.032273in}{2.493291in}}{\pgfqpoint{2.021674in}{2.488901in}}{\pgfqpoint{2.013861in}{2.481087in}}%
\pgfpathcurveto{\pgfqpoint{2.006047in}{2.473273in}}{\pgfqpoint{2.001657in}{2.462674in}}{\pgfqpoint{2.001657in}{2.451624in}}%
\pgfpathcurveto{\pgfqpoint{2.001657in}{2.440574in}}{\pgfqpoint{2.006047in}{2.429975in}}{\pgfqpoint{2.013861in}{2.422161in}}%
\pgfpathcurveto{\pgfqpoint{2.021674in}{2.414348in}}{\pgfqpoint{2.032273in}{2.409958in}}{\pgfqpoint{2.043323in}{2.409958in}}%
\pgfpathclose%
\pgfusepath{stroke,fill}%
\end{pgfscope}%
\begin{pgfscope}%
\pgfpathrectangle{\pgfqpoint{0.600000in}{0.600000in}}{\pgfqpoint{3.900000in}{3.900000in}}%
\pgfusepath{clip}%
\pgfsetbuttcap%
\pgfsetroundjoin%
\definecolor{currentfill}{rgb}{0.121569,0.466667,0.705882}%
\pgfsetfillcolor{currentfill}%
\pgfsetlinewidth{1.003750pt}%
\definecolor{currentstroke}{rgb}{0.121569,0.466667,0.705882}%
\pgfsetstrokecolor{currentstroke}%
\pgfsetdash{}{0pt}%
\pgfpathmoveto{\pgfqpoint{2.841402in}{1.945175in}}%
\pgfpathcurveto{\pgfqpoint{2.852452in}{1.945175in}}{\pgfqpoint{2.863051in}{1.949565in}}{\pgfqpoint{2.870864in}{1.957379in}}%
\pgfpathcurveto{\pgfqpoint{2.878678in}{1.965192in}}{\pgfqpoint{2.883068in}{1.975791in}}{\pgfqpoint{2.883068in}{1.986842in}}%
\pgfpathcurveto{\pgfqpoint{2.883068in}{1.997892in}}{\pgfqpoint{2.878678in}{2.008491in}}{\pgfqpoint{2.870864in}{2.016304in}}%
\pgfpathcurveto{\pgfqpoint{2.863051in}{2.024118in}}{\pgfqpoint{2.852452in}{2.028508in}}{\pgfqpoint{2.841402in}{2.028508in}}%
\pgfpathcurveto{\pgfqpoint{2.830351in}{2.028508in}}{\pgfqpoint{2.819752in}{2.024118in}}{\pgfqpoint{2.811939in}{2.016304in}}%
\pgfpathcurveto{\pgfqpoint{2.804125in}{2.008491in}}{\pgfqpoint{2.799735in}{1.997892in}}{\pgfqpoint{2.799735in}{1.986842in}}%
\pgfpathcurveto{\pgfqpoint{2.799735in}{1.975791in}}{\pgfqpoint{2.804125in}{1.965192in}}{\pgfqpoint{2.811939in}{1.957379in}}%
\pgfpathcurveto{\pgfqpoint{2.819752in}{1.949565in}}{\pgfqpoint{2.830351in}{1.945175in}}{\pgfqpoint{2.841402in}{1.945175in}}%
\pgfpathclose%
\pgfusepath{stroke,fill}%
\end{pgfscope}%
\begin{pgfscope}%
\pgfpathrectangle{\pgfqpoint{0.600000in}{0.600000in}}{\pgfqpoint{3.900000in}{3.900000in}}%
\pgfusepath{clip}%
\pgfsetbuttcap%
\pgfsetroundjoin%
\definecolor{currentfill}{rgb}{0.121569,0.466667,0.705882}%
\pgfsetfillcolor{currentfill}%
\pgfsetlinewidth{1.003750pt}%
\definecolor{currentstroke}{rgb}{0.121569,0.466667,0.705882}%
\pgfsetstrokecolor{currentstroke}%
\pgfsetdash{}{0pt}%
\pgfpathmoveto{\pgfqpoint{2.155614in}{2.352816in}}%
\pgfpathcurveto{\pgfqpoint{2.166664in}{2.352816in}}{\pgfqpoint{2.177263in}{2.357207in}}{\pgfqpoint{2.185076in}{2.365020in}}%
\pgfpathcurveto{\pgfqpoint{2.192890in}{2.372834in}}{\pgfqpoint{2.197280in}{2.383433in}}{\pgfqpoint{2.197280in}{2.394483in}}%
\pgfpathcurveto{\pgfqpoint{2.197280in}{2.405533in}}{\pgfqpoint{2.192890in}{2.416132in}}{\pgfqpoint{2.185076in}{2.423946in}}%
\pgfpathcurveto{\pgfqpoint{2.177263in}{2.431759in}}{\pgfqpoint{2.166664in}{2.436150in}}{\pgfqpoint{2.155614in}{2.436150in}}%
\pgfpathcurveto{\pgfqpoint{2.144564in}{2.436150in}}{\pgfqpoint{2.133965in}{2.431759in}}{\pgfqpoint{2.126151in}{2.423946in}}%
\pgfpathcurveto{\pgfqpoint{2.118337in}{2.416132in}}{\pgfqpoint{2.113947in}{2.405533in}}{\pgfqpoint{2.113947in}{2.394483in}}%
\pgfpathcurveto{\pgfqpoint{2.113947in}{2.383433in}}{\pgfqpoint{2.118337in}{2.372834in}}{\pgfqpoint{2.126151in}{2.365020in}}%
\pgfpathcurveto{\pgfqpoint{2.133965in}{2.357207in}}{\pgfqpoint{2.144564in}{2.352816in}}{\pgfqpoint{2.155614in}{2.352816in}}%
\pgfpathclose%
\pgfusepath{stroke,fill}%
\end{pgfscope}%
\begin{pgfscope}%
\pgfpathrectangle{\pgfqpoint{0.600000in}{0.600000in}}{\pgfqpoint{3.900000in}{3.900000in}}%
\pgfusepath{clip}%
\pgfsetbuttcap%
\pgfsetroundjoin%
\definecolor{currentfill}{rgb}{0.121569,0.466667,0.705882}%
\pgfsetfillcolor{currentfill}%
\pgfsetlinewidth{1.003750pt}%
\definecolor{currentstroke}{rgb}{0.121569,0.466667,0.705882}%
\pgfsetstrokecolor{currentstroke}%
\pgfsetdash{}{0pt}%
\pgfpathmoveto{\pgfqpoint{2.300093in}{2.323727in}}%
\pgfpathcurveto{\pgfqpoint{2.311143in}{2.323727in}}{\pgfqpoint{2.321742in}{2.328117in}}{\pgfqpoint{2.329556in}{2.335931in}}%
\pgfpathcurveto{\pgfqpoint{2.337369in}{2.343744in}}{\pgfqpoint{2.341759in}{2.354343in}}{\pgfqpoint{2.341759in}{2.365393in}}%
\pgfpathcurveto{\pgfqpoint{2.341759in}{2.376444in}}{\pgfqpoint{2.337369in}{2.387043in}}{\pgfqpoint{2.329556in}{2.394856in}}%
\pgfpathcurveto{\pgfqpoint{2.321742in}{2.402670in}}{\pgfqpoint{2.311143in}{2.407060in}}{\pgfqpoint{2.300093in}{2.407060in}}%
\pgfpathcurveto{\pgfqpoint{2.289043in}{2.407060in}}{\pgfqpoint{2.278444in}{2.402670in}}{\pgfqpoint{2.270630in}{2.394856in}}%
\pgfpathcurveto{\pgfqpoint{2.262816in}{2.387043in}}{\pgfqpoint{2.258426in}{2.376444in}}{\pgfqpoint{2.258426in}{2.365393in}}%
\pgfpathcurveto{\pgfqpoint{2.258426in}{2.354343in}}{\pgfqpoint{2.262816in}{2.343744in}}{\pgfqpoint{2.270630in}{2.335931in}}%
\pgfpathcurveto{\pgfqpoint{2.278444in}{2.328117in}}{\pgfqpoint{2.289043in}{2.323727in}}{\pgfqpoint{2.300093in}{2.323727in}}%
\pgfpathclose%
\pgfusepath{stroke,fill}%
\end{pgfscope}%
\begin{pgfscope}%
\pgfpathrectangle{\pgfqpoint{0.600000in}{0.600000in}}{\pgfqpoint{3.900000in}{3.900000in}}%
\pgfusepath{clip}%
\pgfsetbuttcap%
\pgfsetroundjoin%
\definecolor{currentfill}{rgb}{0.121569,0.466667,0.705882}%
\pgfsetfillcolor{currentfill}%
\pgfsetlinewidth{1.003750pt}%
\definecolor{currentstroke}{rgb}{0.121569,0.466667,0.705882}%
\pgfsetstrokecolor{currentstroke}%
\pgfsetdash{}{0pt}%
\pgfpathmoveto{\pgfqpoint{2.613231in}{2.074260in}}%
\pgfpathcurveto{\pgfqpoint{2.624281in}{2.074260in}}{\pgfqpoint{2.634880in}{2.078650in}}{\pgfqpoint{2.642693in}{2.086464in}}%
\pgfpathcurveto{\pgfqpoint{2.650507in}{2.094278in}}{\pgfqpoint{2.654897in}{2.104877in}}{\pgfqpoint{2.654897in}{2.115927in}}%
\pgfpathcurveto{\pgfqpoint{2.654897in}{2.126977in}}{\pgfqpoint{2.650507in}{2.137576in}}{\pgfqpoint{2.642693in}{2.145390in}}%
\pgfpathcurveto{\pgfqpoint{2.634880in}{2.153203in}}{\pgfqpoint{2.624281in}{2.157593in}}{\pgfqpoint{2.613231in}{2.157593in}}%
\pgfpathcurveto{\pgfqpoint{2.602180in}{2.157593in}}{\pgfqpoint{2.591581in}{2.153203in}}{\pgfqpoint{2.583768in}{2.145390in}}%
\pgfpathcurveto{\pgfqpoint{2.575954in}{2.137576in}}{\pgfqpoint{2.571564in}{2.126977in}}{\pgfqpoint{2.571564in}{2.115927in}}%
\pgfpathcurveto{\pgfqpoint{2.571564in}{2.104877in}}{\pgfqpoint{2.575954in}{2.094278in}}{\pgfqpoint{2.583768in}{2.086464in}}%
\pgfpathcurveto{\pgfqpoint{2.591581in}{2.078650in}}{\pgfqpoint{2.602180in}{2.074260in}}{\pgfqpoint{2.613231in}{2.074260in}}%
\pgfpathclose%
\pgfusepath{stroke,fill}%
\end{pgfscope}%
\begin{pgfscope}%
\pgfpathrectangle{\pgfqpoint{0.600000in}{0.600000in}}{\pgfqpoint{3.900000in}{3.900000in}}%
\pgfusepath{clip}%
\pgfsetbuttcap%
\pgfsetroundjoin%
\definecolor{currentfill}{rgb}{0.121569,0.466667,0.705882}%
\pgfsetfillcolor{currentfill}%
\pgfsetlinewidth{1.003750pt}%
\definecolor{currentstroke}{rgb}{0.121569,0.466667,0.705882}%
\pgfsetstrokecolor{currentstroke}%
\pgfsetdash{}{0pt}%
\pgfpathmoveto{\pgfqpoint{3.458093in}{1.743908in}}%
\pgfpathcurveto{\pgfqpoint{3.469143in}{1.743908in}}{\pgfqpoint{3.479742in}{1.748298in}}{\pgfqpoint{3.487556in}{1.756112in}}%
\pgfpathcurveto{\pgfqpoint{3.495369in}{1.763925in}}{\pgfqpoint{3.499760in}{1.774524in}}{\pgfqpoint{3.499760in}{1.785574in}}%
\pgfpathcurveto{\pgfqpoint{3.499760in}{1.796624in}}{\pgfqpoint{3.495369in}{1.807223in}}{\pgfqpoint{3.487556in}{1.815037in}}%
\pgfpathcurveto{\pgfqpoint{3.479742in}{1.822851in}}{\pgfqpoint{3.469143in}{1.827241in}}{\pgfqpoint{3.458093in}{1.827241in}}%
\pgfpathcurveto{\pgfqpoint{3.447043in}{1.827241in}}{\pgfqpoint{3.436444in}{1.822851in}}{\pgfqpoint{3.428630in}{1.815037in}}%
\pgfpathcurveto{\pgfqpoint{3.420817in}{1.807223in}}{\pgfqpoint{3.416426in}{1.796624in}}{\pgfqpoint{3.416426in}{1.785574in}}%
\pgfpathcurveto{\pgfqpoint{3.416426in}{1.774524in}}{\pgfqpoint{3.420817in}{1.763925in}}{\pgfqpoint{3.428630in}{1.756112in}}%
\pgfpathcurveto{\pgfqpoint{3.436444in}{1.748298in}}{\pgfqpoint{3.447043in}{1.743908in}}{\pgfqpoint{3.458093in}{1.743908in}}%
\pgfpathclose%
\pgfusepath{stroke,fill}%
\end{pgfscope}%
\begin{pgfscope}%
\pgfpathrectangle{\pgfqpoint{0.600000in}{0.600000in}}{\pgfqpoint{3.900000in}{3.900000in}}%
\pgfusepath{clip}%
\pgfsetbuttcap%
\pgfsetroundjoin%
\definecolor{currentfill}{rgb}{0.121569,0.466667,0.705882}%
\pgfsetfillcolor{currentfill}%
\pgfsetlinewidth{1.003750pt}%
\definecolor{currentstroke}{rgb}{0.121569,0.466667,0.705882}%
\pgfsetstrokecolor{currentstroke}%
\pgfsetdash{}{0pt}%
\pgfpathmoveto{\pgfqpoint{2.229568in}{3.138940in}}%
\pgfpathcurveto{\pgfqpoint{2.240618in}{3.138940in}}{\pgfqpoint{2.251218in}{3.143330in}}{\pgfqpoint{2.259031in}{3.151144in}}%
\pgfpathcurveto{\pgfqpoint{2.266845in}{3.158957in}}{\pgfqpoint{2.271235in}{3.169556in}}{\pgfqpoint{2.271235in}{3.180607in}}%
\pgfpathcurveto{\pgfqpoint{2.271235in}{3.191657in}}{\pgfqpoint{2.266845in}{3.202256in}}{\pgfqpoint{2.259031in}{3.210069in}}%
\pgfpathcurveto{\pgfqpoint{2.251218in}{3.217883in}}{\pgfqpoint{2.240618in}{3.222273in}}{\pgfqpoint{2.229568in}{3.222273in}}%
\pgfpathcurveto{\pgfqpoint{2.218518in}{3.222273in}}{\pgfqpoint{2.207919in}{3.217883in}}{\pgfqpoint{2.200106in}{3.210069in}}%
\pgfpathcurveto{\pgfqpoint{2.192292in}{3.202256in}}{\pgfqpoint{2.187902in}{3.191657in}}{\pgfqpoint{2.187902in}{3.180607in}}%
\pgfpathcurveto{\pgfqpoint{2.187902in}{3.169556in}}{\pgfqpoint{2.192292in}{3.158957in}}{\pgfqpoint{2.200106in}{3.151144in}}%
\pgfpathcurveto{\pgfqpoint{2.207919in}{3.143330in}}{\pgfqpoint{2.218518in}{3.138940in}}{\pgfqpoint{2.229568in}{3.138940in}}%
\pgfpathclose%
\pgfusepath{stroke,fill}%
\end{pgfscope}%
\begin{pgfscope}%
\pgfpathrectangle{\pgfqpoint{0.600000in}{0.600000in}}{\pgfqpoint{3.900000in}{3.900000in}}%
\pgfusepath{clip}%
\pgfsetbuttcap%
\pgfsetroundjoin%
\definecolor{currentfill}{rgb}{0.121569,0.466667,0.705882}%
\pgfsetfillcolor{currentfill}%
\pgfsetlinewidth{1.003750pt}%
\definecolor{currentstroke}{rgb}{0.121569,0.466667,0.705882}%
\pgfsetstrokecolor{currentstroke}%
\pgfsetdash{}{0pt}%
\pgfpathmoveto{\pgfqpoint{2.622745in}{3.227543in}}%
\pgfpathcurveto{\pgfqpoint{2.633795in}{3.227543in}}{\pgfqpoint{2.644394in}{3.231933in}}{\pgfqpoint{2.652207in}{3.239747in}}%
\pgfpathcurveto{\pgfqpoint{2.660021in}{3.247561in}}{\pgfqpoint{2.664411in}{3.258160in}}{\pgfqpoint{2.664411in}{3.269210in}}%
\pgfpathcurveto{\pgfqpoint{2.664411in}{3.280260in}}{\pgfqpoint{2.660021in}{3.290859in}}{\pgfqpoint{2.652207in}{3.298673in}}%
\pgfpathcurveto{\pgfqpoint{2.644394in}{3.306486in}}{\pgfqpoint{2.633795in}{3.310877in}}{\pgfqpoint{2.622745in}{3.310877in}}%
\pgfpathcurveto{\pgfqpoint{2.611694in}{3.310877in}}{\pgfqpoint{2.601095in}{3.306486in}}{\pgfqpoint{2.593282in}{3.298673in}}%
\pgfpathcurveto{\pgfqpoint{2.585468in}{3.290859in}}{\pgfqpoint{2.581078in}{3.280260in}}{\pgfqpoint{2.581078in}{3.269210in}}%
\pgfpathcurveto{\pgfqpoint{2.581078in}{3.258160in}}{\pgfqpoint{2.585468in}{3.247561in}}{\pgfqpoint{2.593282in}{3.239747in}}%
\pgfpathcurveto{\pgfqpoint{2.601095in}{3.231933in}}{\pgfqpoint{2.611694in}{3.227543in}}{\pgfqpoint{2.622745in}{3.227543in}}%
\pgfpathclose%
\pgfusepath{stroke,fill}%
\end{pgfscope}%
\begin{pgfscope}%
\pgfpathrectangle{\pgfqpoint{0.600000in}{0.600000in}}{\pgfqpoint{3.900000in}{3.900000in}}%
\pgfusepath{clip}%
\pgfsetbuttcap%
\pgfsetroundjoin%
\definecolor{currentfill}{rgb}{0.121569,0.466667,0.705882}%
\pgfsetfillcolor{currentfill}%
\pgfsetlinewidth{1.003750pt}%
\definecolor{currentstroke}{rgb}{0.121569,0.466667,0.705882}%
\pgfsetstrokecolor{currentstroke}%
\pgfsetdash{}{0pt}%
\pgfpathmoveto{\pgfqpoint{1.897596in}{2.036347in}}%
\pgfpathcurveto{\pgfqpoint{1.908647in}{2.036347in}}{\pgfqpoint{1.919246in}{2.040737in}}{\pgfqpoint{1.927059in}{2.048551in}}%
\pgfpathcurveto{\pgfqpoint{1.934873in}{2.056365in}}{\pgfqpoint{1.939263in}{2.066964in}}{\pgfqpoint{1.939263in}{2.078014in}}%
\pgfpathcurveto{\pgfqpoint{1.939263in}{2.089064in}}{\pgfqpoint{1.934873in}{2.099663in}}{\pgfqpoint{1.927059in}{2.107477in}}%
\pgfpathcurveto{\pgfqpoint{1.919246in}{2.115290in}}{\pgfqpoint{1.908647in}{2.119681in}}{\pgfqpoint{1.897596in}{2.119681in}}%
\pgfpathcurveto{\pgfqpoint{1.886546in}{2.119681in}}{\pgfqpoint{1.875947in}{2.115290in}}{\pgfqpoint{1.868134in}{2.107477in}}%
\pgfpathcurveto{\pgfqpoint{1.860320in}{2.099663in}}{\pgfqpoint{1.855930in}{2.089064in}}{\pgfqpoint{1.855930in}{2.078014in}}%
\pgfpathcurveto{\pgfqpoint{1.855930in}{2.066964in}}{\pgfqpoint{1.860320in}{2.056365in}}{\pgfqpoint{1.868134in}{2.048551in}}%
\pgfpathcurveto{\pgfqpoint{1.875947in}{2.040737in}}{\pgfqpoint{1.886546in}{2.036347in}}{\pgfqpoint{1.897596in}{2.036347in}}%
\pgfpathclose%
\pgfusepath{stroke,fill}%
\end{pgfscope}%
\begin{pgfscope}%
\pgfpathrectangle{\pgfqpoint{0.600000in}{0.600000in}}{\pgfqpoint{3.900000in}{3.900000in}}%
\pgfusepath{clip}%
\pgfsetbuttcap%
\pgfsetroundjoin%
\definecolor{currentfill}{rgb}{0.121569,0.466667,0.705882}%
\pgfsetfillcolor{currentfill}%
\pgfsetlinewidth{1.003750pt}%
\definecolor{currentstroke}{rgb}{0.121569,0.466667,0.705882}%
\pgfsetstrokecolor{currentstroke}%
\pgfsetdash{}{0pt}%
\pgfpathmoveto{\pgfqpoint{2.276557in}{1.621732in}}%
\pgfpathcurveto{\pgfqpoint{2.287608in}{1.621732in}}{\pgfqpoint{2.298207in}{1.626122in}}{\pgfqpoint{2.306020in}{1.633936in}}%
\pgfpathcurveto{\pgfqpoint{2.313834in}{1.641749in}}{\pgfqpoint{2.318224in}{1.652348in}}{\pgfqpoint{2.318224in}{1.663398in}}%
\pgfpathcurveto{\pgfqpoint{2.318224in}{1.674448in}}{\pgfqpoint{2.313834in}{1.685047in}}{\pgfqpoint{2.306020in}{1.692861in}}%
\pgfpathcurveto{\pgfqpoint{2.298207in}{1.700675in}}{\pgfqpoint{2.287608in}{1.705065in}}{\pgfqpoint{2.276557in}{1.705065in}}%
\pgfpathcurveto{\pgfqpoint{2.265507in}{1.705065in}}{\pgfqpoint{2.254908in}{1.700675in}}{\pgfqpoint{2.247095in}{1.692861in}}%
\pgfpathcurveto{\pgfqpoint{2.239281in}{1.685047in}}{\pgfqpoint{2.234891in}{1.674448in}}{\pgfqpoint{2.234891in}{1.663398in}}%
\pgfpathcurveto{\pgfqpoint{2.234891in}{1.652348in}}{\pgfqpoint{2.239281in}{1.641749in}}{\pgfqpoint{2.247095in}{1.633936in}}%
\pgfpathcurveto{\pgfqpoint{2.254908in}{1.626122in}}{\pgfqpoint{2.265507in}{1.621732in}}{\pgfqpoint{2.276557in}{1.621732in}}%
\pgfpathclose%
\pgfusepath{stroke,fill}%
\end{pgfscope}%
\begin{pgfscope}%
\pgfpathrectangle{\pgfqpoint{0.600000in}{0.600000in}}{\pgfqpoint{3.900000in}{3.900000in}}%
\pgfusepath{clip}%
\pgfsetbuttcap%
\pgfsetroundjoin%
\definecolor{currentfill}{rgb}{0.121569,0.466667,0.705882}%
\pgfsetfillcolor{currentfill}%
\pgfsetlinewidth{1.003750pt}%
\definecolor{currentstroke}{rgb}{0.121569,0.466667,0.705882}%
\pgfsetstrokecolor{currentstroke}%
\pgfsetdash{}{0pt}%
\pgfpathmoveto{\pgfqpoint{2.871970in}{2.394571in}}%
\pgfpathcurveto{\pgfqpoint{2.883020in}{2.394571in}}{\pgfqpoint{2.893619in}{2.398962in}}{\pgfqpoint{2.901433in}{2.406775in}}%
\pgfpathcurveto{\pgfqpoint{2.909246in}{2.414589in}}{\pgfqpoint{2.913637in}{2.425188in}}{\pgfqpoint{2.913637in}{2.436238in}}%
\pgfpathcurveto{\pgfqpoint{2.913637in}{2.447288in}}{\pgfqpoint{2.909246in}{2.457887in}}{\pgfqpoint{2.901433in}{2.465701in}}%
\pgfpathcurveto{\pgfqpoint{2.893619in}{2.473515in}}{\pgfqpoint{2.883020in}{2.477905in}}{\pgfqpoint{2.871970in}{2.477905in}}%
\pgfpathcurveto{\pgfqpoint{2.860920in}{2.477905in}}{\pgfqpoint{2.850321in}{2.473515in}}{\pgfqpoint{2.842507in}{2.465701in}}%
\pgfpathcurveto{\pgfqpoint{2.834694in}{2.457887in}}{\pgfqpoint{2.830303in}{2.447288in}}{\pgfqpoint{2.830303in}{2.436238in}}%
\pgfpathcurveto{\pgfqpoint{2.830303in}{2.425188in}}{\pgfqpoint{2.834694in}{2.414589in}}{\pgfqpoint{2.842507in}{2.406775in}}%
\pgfpathcurveto{\pgfqpoint{2.850321in}{2.398962in}}{\pgfqpoint{2.860920in}{2.394571in}}{\pgfqpoint{2.871970in}{2.394571in}}%
\pgfpathclose%
\pgfusepath{stroke,fill}%
\end{pgfscope}%
\begin{pgfscope}%
\pgfpathrectangle{\pgfqpoint{0.600000in}{0.600000in}}{\pgfqpoint{3.900000in}{3.900000in}}%
\pgfusepath{clip}%
\pgfsetbuttcap%
\pgfsetroundjoin%
\definecolor{currentfill}{rgb}{0.121569,0.466667,0.705882}%
\pgfsetfillcolor{currentfill}%
\pgfsetlinewidth{1.003750pt}%
\definecolor{currentstroke}{rgb}{0.121569,0.466667,0.705882}%
\pgfsetstrokecolor{currentstroke}%
\pgfsetdash{}{0pt}%
\pgfpathmoveto{\pgfqpoint{2.947380in}{1.705905in}}%
\pgfpathcurveto{\pgfqpoint{2.958430in}{1.705905in}}{\pgfqpoint{2.969030in}{1.710295in}}{\pgfqpoint{2.976843in}{1.718109in}}%
\pgfpathcurveto{\pgfqpoint{2.984657in}{1.725922in}}{\pgfqpoint{2.989047in}{1.736521in}}{\pgfqpoint{2.989047in}{1.747572in}}%
\pgfpathcurveto{\pgfqpoint{2.989047in}{1.758622in}}{\pgfqpoint{2.984657in}{1.769221in}}{\pgfqpoint{2.976843in}{1.777034in}}%
\pgfpathcurveto{\pgfqpoint{2.969030in}{1.784848in}}{\pgfqpoint{2.958430in}{1.789238in}}{\pgfqpoint{2.947380in}{1.789238in}}%
\pgfpathcurveto{\pgfqpoint{2.936330in}{1.789238in}}{\pgfqpoint{2.925731in}{1.784848in}}{\pgfqpoint{2.917918in}{1.777034in}}%
\pgfpathcurveto{\pgfqpoint{2.910104in}{1.769221in}}{\pgfqpoint{2.905714in}{1.758622in}}{\pgfqpoint{2.905714in}{1.747572in}}%
\pgfpathcurveto{\pgfqpoint{2.905714in}{1.736521in}}{\pgfqpoint{2.910104in}{1.725922in}}{\pgfqpoint{2.917918in}{1.718109in}}%
\pgfpathcurveto{\pgfqpoint{2.925731in}{1.710295in}}{\pgfqpoint{2.936330in}{1.705905in}}{\pgfqpoint{2.947380in}{1.705905in}}%
\pgfpathclose%
\pgfusepath{stroke,fill}%
\end{pgfscope}%
\begin{pgfscope}%
\pgfpathrectangle{\pgfqpoint{0.600000in}{0.600000in}}{\pgfqpoint{3.900000in}{3.900000in}}%
\pgfusepath{clip}%
\pgfsetbuttcap%
\pgfsetroundjoin%
\definecolor{currentfill}{rgb}{0.121569,0.466667,0.705882}%
\pgfsetfillcolor{currentfill}%
\pgfsetlinewidth{1.003750pt}%
\definecolor{currentstroke}{rgb}{0.121569,0.466667,0.705882}%
\pgfsetstrokecolor{currentstroke}%
\pgfsetdash{}{0pt}%
\pgfpathmoveto{\pgfqpoint{2.754158in}{2.304653in}}%
\pgfpathcurveto{\pgfqpoint{2.765208in}{2.304653in}}{\pgfqpoint{2.775807in}{2.309043in}}{\pgfqpoint{2.783621in}{2.316857in}}%
\pgfpathcurveto{\pgfqpoint{2.791434in}{2.324670in}}{\pgfqpoint{2.795824in}{2.335269in}}{\pgfqpoint{2.795824in}{2.346319in}}%
\pgfpathcurveto{\pgfqpoint{2.795824in}{2.357370in}}{\pgfqpoint{2.791434in}{2.367969in}}{\pgfqpoint{2.783621in}{2.375782in}}%
\pgfpathcurveto{\pgfqpoint{2.775807in}{2.383596in}}{\pgfqpoint{2.765208in}{2.387986in}}{\pgfqpoint{2.754158in}{2.387986in}}%
\pgfpathcurveto{\pgfqpoint{2.743108in}{2.387986in}}{\pgfqpoint{2.732509in}{2.383596in}}{\pgfqpoint{2.724695in}{2.375782in}}%
\pgfpathcurveto{\pgfqpoint{2.716881in}{2.367969in}}{\pgfqpoint{2.712491in}{2.357370in}}{\pgfqpoint{2.712491in}{2.346319in}}%
\pgfpathcurveto{\pgfqpoint{2.712491in}{2.335269in}}{\pgfqpoint{2.716881in}{2.324670in}}{\pgfqpoint{2.724695in}{2.316857in}}%
\pgfpathcurveto{\pgfqpoint{2.732509in}{2.309043in}}{\pgfqpoint{2.743108in}{2.304653in}}{\pgfqpoint{2.754158in}{2.304653in}}%
\pgfpathclose%
\pgfusepath{stroke,fill}%
\end{pgfscope}%
\begin{pgfscope}%
\pgfpathrectangle{\pgfqpoint{0.600000in}{0.600000in}}{\pgfqpoint{3.900000in}{3.900000in}}%
\pgfusepath{clip}%
\pgfsetbuttcap%
\pgfsetroundjoin%
\definecolor{currentfill}{rgb}{0.121569,0.466667,0.705882}%
\pgfsetfillcolor{currentfill}%
\pgfsetlinewidth{1.003750pt}%
\definecolor{currentstroke}{rgb}{0.121569,0.466667,0.705882}%
\pgfsetstrokecolor{currentstroke}%
\pgfsetdash{}{0pt}%
\pgfpathmoveto{\pgfqpoint{3.158721in}{2.879274in}}%
\pgfpathcurveto{\pgfqpoint{3.169772in}{2.879274in}}{\pgfqpoint{3.180371in}{2.883664in}}{\pgfqpoint{3.188184in}{2.891478in}}%
\pgfpathcurveto{\pgfqpoint{3.195998in}{2.899292in}}{\pgfqpoint{3.200388in}{2.909891in}}{\pgfqpoint{3.200388in}{2.920941in}}%
\pgfpathcurveto{\pgfqpoint{3.200388in}{2.931991in}}{\pgfqpoint{3.195998in}{2.942590in}}{\pgfqpoint{3.188184in}{2.950404in}}%
\pgfpathcurveto{\pgfqpoint{3.180371in}{2.958217in}}{\pgfqpoint{3.169772in}{2.962608in}}{\pgfqpoint{3.158721in}{2.962608in}}%
\pgfpathcurveto{\pgfqpoint{3.147671in}{2.962608in}}{\pgfqpoint{3.137072in}{2.958217in}}{\pgfqpoint{3.129259in}{2.950404in}}%
\pgfpathcurveto{\pgfqpoint{3.121445in}{2.942590in}}{\pgfqpoint{3.117055in}{2.931991in}}{\pgfqpoint{3.117055in}{2.920941in}}%
\pgfpathcurveto{\pgfqpoint{3.117055in}{2.909891in}}{\pgfqpoint{3.121445in}{2.899292in}}{\pgfqpoint{3.129259in}{2.891478in}}%
\pgfpathcurveto{\pgfqpoint{3.137072in}{2.883664in}}{\pgfqpoint{3.147671in}{2.879274in}}{\pgfqpoint{3.158721in}{2.879274in}}%
\pgfpathclose%
\pgfusepath{stroke,fill}%
\end{pgfscope}%
\begin{pgfscope}%
\pgfpathrectangle{\pgfqpoint{0.600000in}{0.600000in}}{\pgfqpoint{3.900000in}{3.900000in}}%
\pgfusepath{clip}%
\pgfsetbuttcap%
\pgfsetroundjoin%
\definecolor{currentfill}{rgb}{0.121569,0.466667,0.705882}%
\pgfsetfillcolor{currentfill}%
\pgfsetlinewidth{1.003750pt}%
\definecolor{currentstroke}{rgb}{0.121569,0.466667,0.705882}%
\pgfsetstrokecolor{currentstroke}%
\pgfsetdash{}{0pt}%
\pgfpathmoveto{\pgfqpoint{1.942724in}{2.399088in}}%
\pgfpathcurveto{\pgfqpoint{1.953774in}{2.399088in}}{\pgfqpoint{1.964373in}{2.403479in}}{\pgfqpoint{1.972186in}{2.411292in}}%
\pgfpathcurveto{\pgfqpoint{1.980000in}{2.419106in}}{\pgfqpoint{1.984390in}{2.429705in}}{\pgfqpoint{1.984390in}{2.440755in}}%
\pgfpathcurveto{\pgfqpoint{1.984390in}{2.451805in}}{\pgfqpoint{1.980000in}{2.462404in}}{\pgfqpoint{1.972186in}{2.470218in}}%
\pgfpathcurveto{\pgfqpoint{1.964373in}{2.478031in}}{\pgfqpoint{1.953774in}{2.482422in}}{\pgfqpoint{1.942724in}{2.482422in}}%
\pgfpathcurveto{\pgfqpoint{1.931674in}{2.482422in}}{\pgfqpoint{1.921074in}{2.478031in}}{\pgfqpoint{1.913261in}{2.470218in}}%
\pgfpathcurveto{\pgfqpoint{1.905447in}{2.462404in}}{\pgfqpoint{1.901057in}{2.451805in}}{\pgfqpoint{1.901057in}{2.440755in}}%
\pgfpathcurveto{\pgfqpoint{1.901057in}{2.429705in}}{\pgfqpoint{1.905447in}{2.419106in}}{\pgfqpoint{1.913261in}{2.411292in}}%
\pgfpathcurveto{\pgfqpoint{1.921074in}{2.403479in}}{\pgfqpoint{1.931674in}{2.399088in}}{\pgfqpoint{1.942724in}{2.399088in}}%
\pgfpathclose%
\pgfusepath{stroke,fill}%
\end{pgfscope}%
\begin{pgfscope}%
\pgfpathrectangle{\pgfqpoint{0.600000in}{0.600000in}}{\pgfqpoint{3.900000in}{3.900000in}}%
\pgfusepath{clip}%
\pgfsetbuttcap%
\pgfsetroundjoin%
\definecolor{currentfill}{rgb}{0.121569,0.466667,0.705882}%
\pgfsetfillcolor{currentfill}%
\pgfsetlinewidth{1.003750pt}%
\definecolor{currentstroke}{rgb}{0.121569,0.466667,0.705882}%
\pgfsetstrokecolor{currentstroke}%
\pgfsetdash{}{0pt}%
\pgfpathmoveto{\pgfqpoint{2.373532in}{1.989316in}}%
\pgfpathcurveto{\pgfqpoint{2.384582in}{1.989316in}}{\pgfqpoint{2.395181in}{1.993706in}}{\pgfqpoint{2.402995in}{2.001520in}}%
\pgfpathcurveto{\pgfqpoint{2.410809in}{2.009334in}}{\pgfqpoint{2.415199in}{2.019933in}}{\pgfqpoint{2.415199in}{2.030983in}}%
\pgfpathcurveto{\pgfqpoint{2.415199in}{2.042033in}}{\pgfqpoint{2.410809in}{2.052632in}}{\pgfqpoint{2.402995in}{2.060446in}}%
\pgfpathcurveto{\pgfqpoint{2.395181in}{2.068259in}}{\pgfqpoint{2.384582in}{2.072649in}}{\pgfqpoint{2.373532in}{2.072649in}}%
\pgfpathcurveto{\pgfqpoint{2.362482in}{2.072649in}}{\pgfqpoint{2.351883in}{2.068259in}}{\pgfqpoint{2.344069in}{2.060446in}}%
\pgfpathcurveto{\pgfqpoint{2.336256in}{2.052632in}}{\pgfqpoint{2.331866in}{2.042033in}}{\pgfqpoint{2.331866in}{2.030983in}}%
\pgfpathcurveto{\pgfqpoint{2.331866in}{2.019933in}}{\pgfqpoint{2.336256in}{2.009334in}}{\pgfqpoint{2.344069in}{2.001520in}}%
\pgfpathcurveto{\pgfqpoint{2.351883in}{1.993706in}}{\pgfqpoint{2.362482in}{1.989316in}}{\pgfqpoint{2.373532in}{1.989316in}}%
\pgfpathclose%
\pgfusepath{stroke,fill}%
\end{pgfscope}%
\begin{pgfscope}%
\pgfpathrectangle{\pgfqpoint{0.600000in}{0.600000in}}{\pgfqpoint{3.900000in}{3.900000in}}%
\pgfusepath{clip}%
\pgfsetbuttcap%
\pgfsetroundjoin%
\definecolor{currentfill}{rgb}{0.121569,0.466667,0.705882}%
\pgfsetfillcolor{currentfill}%
\pgfsetlinewidth{1.003750pt}%
\definecolor{currentstroke}{rgb}{0.121569,0.466667,0.705882}%
\pgfsetstrokecolor{currentstroke}%
\pgfsetdash{}{0pt}%
\pgfpathmoveto{\pgfqpoint{2.027619in}{2.563675in}}%
\pgfpathcurveto{\pgfqpoint{2.038669in}{2.563675in}}{\pgfqpoint{2.049268in}{2.568065in}}{\pgfqpoint{2.057081in}{2.575879in}}%
\pgfpathcurveto{\pgfqpoint{2.064895in}{2.583692in}}{\pgfqpoint{2.069285in}{2.594291in}}{\pgfqpoint{2.069285in}{2.605341in}}%
\pgfpathcurveto{\pgfqpoint{2.069285in}{2.616391in}}{\pgfqpoint{2.064895in}{2.626991in}}{\pgfqpoint{2.057081in}{2.634804in}}%
\pgfpathcurveto{\pgfqpoint{2.049268in}{2.642618in}}{\pgfqpoint{2.038669in}{2.647008in}}{\pgfqpoint{2.027619in}{2.647008in}}%
\pgfpathcurveto{\pgfqpoint{2.016569in}{2.647008in}}{\pgfqpoint{2.005970in}{2.642618in}}{\pgfqpoint{1.998156in}{2.634804in}}%
\pgfpathcurveto{\pgfqpoint{1.990342in}{2.626991in}}{\pgfqpoint{1.985952in}{2.616391in}}{\pgfqpoint{1.985952in}{2.605341in}}%
\pgfpathcurveto{\pgfqpoint{1.985952in}{2.594291in}}{\pgfqpoint{1.990342in}{2.583692in}}{\pgfqpoint{1.998156in}{2.575879in}}%
\pgfpathcurveto{\pgfqpoint{2.005970in}{2.568065in}}{\pgfqpoint{2.016569in}{2.563675in}}{\pgfqpoint{2.027619in}{2.563675in}}%
\pgfpathclose%
\pgfusepath{stroke,fill}%
\end{pgfscope}%
\begin{pgfscope}%
\pgfpathrectangle{\pgfqpoint{0.600000in}{0.600000in}}{\pgfqpoint{3.900000in}{3.900000in}}%
\pgfusepath{clip}%
\pgfsetbuttcap%
\pgfsetroundjoin%
\definecolor{currentfill}{rgb}{0.121569,0.466667,0.705882}%
\pgfsetfillcolor{currentfill}%
\pgfsetlinewidth{1.003750pt}%
\definecolor{currentstroke}{rgb}{0.121569,0.466667,0.705882}%
\pgfsetstrokecolor{currentstroke}%
\pgfsetdash{}{0pt}%
\pgfpathmoveto{\pgfqpoint{1.453791in}{3.031390in}}%
\pgfpathcurveto{\pgfqpoint{1.464842in}{3.031390in}}{\pgfqpoint{1.475441in}{3.035780in}}{\pgfqpoint{1.483254in}{3.043593in}}%
\pgfpathcurveto{\pgfqpoint{1.491068in}{3.051407in}}{\pgfqpoint{1.495458in}{3.062006in}}{\pgfqpoint{1.495458in}{3.073056in}}%
\pgfpathcurveto{\pgfqpoint{1.495458in}{3.084106in}}{\pgfqpoint{1.491068in}{3.094705in}}{\pgfqpoint{1.483254in}{3.102519in}}%
\pgfpathcurveto{\pgfqpoint{1.475441in}{3.110333in}}{\pgfqpoint{1.464842in}{3.114723in}}{\pgfqpoint{1.453791in}{3.114723in}}%
\pgfpathcurveto{\pgfqpoint{1.442741in}{3.114723in}}{\pgfqpoint{1.432142in}{3.110333in}}{\pgfqpoint{1.424329in}{3.102519in}}%
\pgfpathcurveto{\pgfqpoint{1.416515in}{3.094705in}}{\pgfqpoint{1.412125in}{3.084106in}}{\pgfqpoint{1.412125in}{3.073056in}}%
\pgfpathcurveto{\pgfqpoint{1.412125in}{3.062006in}}{\pgfqpoint{1.416515in}{3.051407in}}{\pgfqpoint{1.424329in}{3.043593in}}%
\pgfpathcurveto{\pgfqpoint{1.432142in}{3.035780in}}{\pgfqpoint{1.442741in}{3.031390in}}{\pgfqpoint{1.453791in}{3.031390in}}%
\pgfpathclose%
\pgfusepath{stroke,fill}%
\end{pgfscope}%
\begin{pgfscope}%
\pgfpathrectangle{\pgfqpoint{0.600000in}{0.600000in}}{\pgfqpoint{3.900000in}{3.900000in}}%
\pgfusepath{clip}%
\pgfsetbuttcap%
\pgfsetroundjoin%
\definecolor{currentfill}{rgb}{0.121569,0.466667,0.705882}%
\pgfsetfillcolor{currentfill}%
\pgfsetlinewidth{1.003750pt}%
\definecolor{currentstroke}{rgb}{0.121569,0.466667,0.705882}%
\pgfsetstrokecolor{currentstroke}%
\pgfsetdash{}{0pt}%
\pgfpathmoveto{\pgfqpoint{1.998806in}{2.381927in}}%
\pgfpathcurveto{\pgfqpoint{2.009856in}{2.381927in}}{\pgfqpoint{2.020455in}{2.386317in}}{\pgfqpoint{2.028268in}{2.394131in}}%
\pgfpathcurveto{\pgfqpoint{2.036082in}{2.401944in}}{\pgfqpoint{2.040472in}{2.412543in}}{\pgfqpoint{2.040472in}{2.423593in}}%
\pgfpathcurveto{\pgfqpoint{2.040472in}{2.434643in}}{\pgfqpoint{2.036082in}{2.445243in}}{\pgfqpoint{2.028268in}{2.453056in}}%
\pgfpathcurveto{\pgfqpoint{2.020455in}{2.460870in}}{\pgfqpoint{2.009856in}{2.465260in}}{\pgfqpoint{1.998806in}{2.465260in}}%
\pgfpathcurveto{\pgfqpoint{1.987755in}{2.465260in}}{\pgfqpoint{1.977156in}{2.460870in}}{\pgfqpoint{1.969343in}{2.453056in}}%
\pgfpathcurveto{\pgfqpoint{1.961529in}{2.445243in}}{\pgfqpoint{1.957139in}{2.434643in}}{\pgfqpoint{1.957139in}{2.423593in}}%
\pgfpathcurveto{\pgfqpoint{1.957139in}{2.412543in}}{\pgfqpoint{1.961529in}{2.401944in}}{\pgfqpoint{1.969343in}{2.394131in}}%
\pgfpathcurveto{\pgfqpoint{1.977156in}{2.386317in}}{\pgfqpoint{1.987755in}{2.381927in}}{\pgfqpoint{1.998806in}{2.381927in}}%
\pgfpathclose%
\pgfusepath{stroke,fill}%
\end{pgfscope}%
\begin{pgfscope}%
\pgfpathrectangle{\pgfqpoint{0.600000in}{0.600000in}}{\pgfqpoint{3.900000in}{3.900000in}}%
\pgfusepath{clip}%
\pgfsetbuttcap%
\pgfsetroundjoin%
\definecolor{currentfill}{rgb}{0.121569,0.466667,0.705882}%
\pgfsetfillcolor{currentfill}%
\pgfsetlinewidth{1.003750pt}%
\definecolor{currentstroke}{rgb}{0.121569,0.466667,0.705882}%
\pgfsetstrokecolor{currentstroke}%
\pgfsetdash{}{0pt}%
\pgfpathmoveto{\pgfqpoint{2.321002in}{2.744102in}}%
\pgfpathcurveto{\pgfqpoint{2.332053in}{2.744102in}}{\pgfqpoint{2.342652in}{2.748493in}}{\pgfqpoint{2.350465in}{2.756306in}}%
\pgfpathcurveto{\pgfqpoint{2.358279in}{2.764120in}}{\pgfqpoint{2.362669in}{2.774719in}}{\pgfqpoint{2.362669in}{2.785769in}}%
\pgfpathcurveto{\pgfqpoint{2.362669in}{2.796819in}}{\pgfqpoint{2.358279in}{2.807418in}}{\pgfqpoint{2.350465in}{2.815232in}}%
\pgfpathcurveto{\pgfqpoint{2.342652in}{2.823045in}}{\pgfqpoint{2.332053in}{2.827436in}}{\pgfqpoint{2.321002in}{2.827436in}}%
\pgfpathcurveto{\pgfqpoint{2.309952in}{2.827436in}}{\pgfqpoint{2.299353in}{2.823045in}}{\pgfqpoint{2.291540in}{2.815232in}}%
\pgfpathcurveto{\pgfqpoint{2.283726in}{2.807418in}}{\pgfqpoint{2.279336in}{2.796819in}}{\pgfqpoint{2.279336in}{2.785769in}}%
\pgfpathcurveto{\pgfqpoint{2.279336in}{2.774719in}}{\pgfqpoint{2.283726in}{2.764120in}}{\pgfqpoint{2.291540in}{2.756306in}}%
\pgfpathcurveto{\pgfqpoint{2.299353in}{2.748493in}}{\pgfqpoint{2.309952in}{2.744102in}}{\pgfqpoint{2.321002in}{2.744102in}}%
\pgfpathclose%
\pgfusepath{stroke,fill}%
\end{pgfscope}%
\begin{pgfscope}%
\pgfpathrectangle{\pgfqpoint{0.600000in}{0.600000in}}{\pgfqpoint{3.900000in}{3.900000in}}%
\pgfusepath{clip}%
\pgfsetbuttcap%
\pgfsetroundjoin%
\definecolor{currentfill}{rgb}{0.121569,0.466667,0.705882}%
\pgfsetfillcolor{currentfill}%
\pgfsetlinewidth{1.003750pt}%
\definecolor{currentstroke}{rgb}{0.121569,0.466667,0.705882}%
\pgfsetstrokecolor{currentstroke}%
\pgfsetdash{}{0pt}%
\pgfpathmoveto{\pgfqpoint{3.081331in}{1.149773in}}%
\pgfpathcurveto{\pgfqpoint{3.092381in}{1.149773in}}{\pgfqpoint{3.102980in}{1.154163in}}{\pgfqpoint{3.110793in}{1.161977in}}%
\pgfpathcurveto{\pgfqpoint{3.118607in}{1.169790in}}{\pgfqpoint{3.122997in}{1.180389in}}{\pgfqpoint{3.122997in}{1.191439in}}%
\pgfpathcurveto{\pgfqpoint{3.122997in}{1.202490in}}{\pgfqpoint{3.118607in}{1.213089in}}{\pgfqpoint{3.110793in}{1.220902in}}%
\pgfpathcurveto{\pgfqpoint{3.102980in}{1.228716in}}{\pgfqpoint{3.092381in}{1.233106in}}{\pgfqpoint{3.081331in}{1.233106in}}%
\pgfpathcurveto{\pgfqpoint{3.070280in}{1.233106in}}{\pgfqpoint{3.059681in}{1.228716in}}{\pgfqpoint{3.051868in}{1.220902in}}%
\pgfpathcurveto{\pgfqpoint{3.044054in}{1.213089in}}{\pgfqpoint{3.039664in}{1.202490in}}{\pgfqpoint{3.039664in}{1.191439in}}%
\pgfpathcurveto{\pgfqpoint{3.039664in}{1.180389in}}{\pgfqpoint{3.044054in}{1.169790in}}{\pgfqpoint{3.051868in}{1.161977in}}%
\pgfpathcurveto{\pgfqpoint{3.059681in}{1.154163in}}{\pgfqpoint{3.070280in}{1.149773in}}{\pgfqpoint{3.081331in}{1.149773in}}%
\pgfpathclose%
\pgfusepath{stroke,fill}%
\end{pgfscope}%
\begin{pgfscope}%
\pgfpathrectangle{\pgfqpoint{0.600000in}{0.600000in}}{\pgfqpoint{3.900000in}{3.900000in}}%
\pgfusepath{clip}%
\pgfsetbuttcap%
\pgfsetroundjoin%
\definecolor{currentfill}{rgb}{0.121569,0.466667,0.705882}%
\pgfsetfillcolor{currentfill}%
\pgfsetlinewidth{1.003750pt}%
\definecolor{currentstroke}{rgb}{0.121569,0.466667,0.705882}%
\pgfsetstrokecolor{currentstroke}%
\pgfsetdash{}{0pt}%
\pgfpathmoveto{\pgfqpoint{3.358167in}{3.321232in}}%
\pgfpathcurveto{\pgfqpoint{3.369218in}{3.321232in}}{\pgfqpoint{3.379817in}{3.325622in}}{\pgfqpoint{3.387630in}{3.333436in}}%
\pgfpathcurveto{\pgfqpoint{3.395444in}{3.341250in}}{\pgfqpoint{3.399834in}{3.351849in}}{\pgfqpoint{3.399834in}{3.362899in}}%
\pgfpathcurveto{\pgfqpoint{3.399834in}{3.373949in}}{\pgfqpoint{3.395444in}{3.384548in}}{\pgfqpoint{3.387630in}{3.392362in}}%
\pgfpathcurveto{\pgfqpoint{3.379817in}{3.400175in}}{\pgfqpoint{3.369218in}{3.404566in}}{\pgfqpoint{3.358167in}{3.404566in}}%
\pgfpathcurveto{\pgfqpoint{3.347117in}{3.404566in}}{\pgfqpoint{3.336518in}{3.400175in}}{\pgfqpoint{3.328705in}{3.392362in}}%
\pgfpathcurveto{\pgfqpoint{3.320891in}{3.384548in}}{\pgfqpoint{3.316501in}{3.373949in}}{\pgfqpoint{3.316501in}{3.362899in}}%
\pgfpathcurveto{\pgfqpoint{3.316501in}{3.351849in}}{\pgfqpoint{3.320891in}{3.341250in}}{\pgfqpoint{3.328705in}{3.333436in}}%
\pgfpathcurveto{\pgfqpoint{3.336518in}{3.325622in}}{\pgfqpoint{3.347117in}{3.321232in}}{\pgfqpoint{3.358167in}{3.321232in}}%
\pgfpathclose%
\pgfusepath{stroke,fill}%
\end{pgfscope}%
\begin{pgfscope}%
\pgfpathrectangle{\pgfqpoint{0.600000in}{0.600000in}}{\pgfqpoint{3.900000in}{3.900000in}}%
\pgfusepath{clip}%
\pgfsetbuttcap%
\pgfsetroundjoin%
\definecolor{currentfill}{rgb}{0.121569,0.466667,0.705882}%
\pgfsetfillcolor{currentfill}%
\pgfsetlinewidth{1.003750pt}%
\definecolor{currentstroke}{rgb}{0.121569,0.466667,0.705882}%
\pgfsetstrokecolor{currentstroke}%
\pgfsetdash{}{0pt}%
\pgfpathmoveto{\pgfqpoint{2.504389in}{1.954082in}}%
\pgfpathcurveto{\pgfqpoint{2.515439in}{1.954082in}}{\pgfqpoint{2.526038in}{1.958472in}}{\pgfqpoint{2.533852in}{1.966286in}}%
\pgfpathcurveto{\pgfqpoint{2.541665in}{1.974100in}}{\pgfqpoint{2.546055in}{1.984699in}}{\pgfqpoint{2.546055in}{1.995749in}}%
\pgfpathcurveto{\pgfqpoint{2.546055in}{2.006799in}}{\pgfqpoint{2.541665in}{2.017398in}}{\pgfqpoint{2.533852in}{2.025212in}}%
\pgfpathcurveto{\pgfqpoint{2.526038in}{2.033025in}}{\pgfqpoint{2.515439in}{2.037416in}}{\pgfqpoint{2.504389in}{2.037416in}}%
\pgfpathcurveto{\pgfqpoint{2.493339in}{2.037416in}}{\pgfqpoint{2.482740in}{2.033025in}}{\pgfqpoint{2.474926in}{2.025212in}}%
\pgfpathcurveto{\pgfqpoint{2.467112in}{2.017398in}}{\pgfqpoint{2.462722in}{2.006799in}}{\pgfqpoint{2.462722in}{1.995749in}}%
\pgfpathcurveto{\pgfqpoint{2.462722in}{1.984699in}}{\pgfqpoint{2.467112in}{1.974100in}}{\pgfqpoint{2.474926in}{1.966286in}}%
\pgfpathcurveto{\pgfqpoint{2.482740in}{1.958472in}}{\pgfqpoint{2.493339in}{1.954082in}}{\pgfqpoint{2.504389in}{1.954082in}}%
\pgfpathclose%
\pgfusepath{stroke,fill}%
\end{pgfscope}%
\begin{pgfscope}%
\pgfpathrectangle{\pgfqpoint{0.600000in}{0.600000in}}{\pgfqpoint{3.900000in}{3.900000in}}%
\pgfusepath{clip}%
\pgfsetbuttcap%
\pgfsetroundjoin%
\definecolor{currentfill}{rgb}{0.121569,0.466667,0.705882}%
\pgfsetfillcolor{currentfill}%
\pgfsetlinewidth{1.003750pt}%
\definecolor{currentstroke}{rgb}{0.121569,0.466667,0.705882}%
\pgfsetstrokecolor{currentstroke}%
\pgfsetdash{}{0pt}%
\pgfpathmoveto{\pgfqpoint{3.085982in}{3.186041in}}%
\pgfpathcurveto{\pgfqpoint{3.097032in}{3.186041in}}{\pgfqpoint{3.107631in}{3.190432in}}{\pgfqpoint{3.115445in}{3.198245in}}%
\pgfpathcurveto{\pgfqpoint{3.123258in}{3.206059in}}{\pgfqpoint{3.127649in}{3.216658in}}{\pgfqpoint{3.127649in}{3.227708in}}%
\pgfpathcurveto{\pgfqpoint{3.127649in}{3.238758in}}{\pgfqpoint{3.123258in}{3.249357in}}{\pgfqpoint{3.115445in}{3.257171in}}%
\pgfpathcurveto{\pgfqpoint{3.107631in}{3.264984in}}{\pgfqpoint{3.097032in}{3.269375in}}{\pgfqpoint{3.085982in}{3.269375in}}%
\pgfpathcurveto{\pgfqpoint{3.074932in}{3.269375in}}{\pgfqpoint{3.064333in}{3.264984in}}{\pgfqpoint{3.056519in}{3.257171in}}%
\pgfpathcurveto{\pgfqpoint{3.048705in}{3.249357in}}{\pgfqpoint{3.044315in}{3.238758in}}{\pgfqpoint{3.044315in}{3.227708in}}%
\pgfpathcurveto{\pgfqpoint{3.044315in}{3.216658in}}{\pgfqpoint{3.048705in}{3.206059in}}{\pgfqpoint{3.056519in}{3.198245in}}%
\pgfpathcurveto{\pgfqpoint{3.064333in}{3.190432in}}{\pgfqpoint{3.074932in}{3.186041in}}{\pgfqpoint{3.085982in}{3.186041in}}%
\pgfpathclose%
\pgfusepath{stroke,fill}%
\end{pgfscope}%
\begin{pgfscope}%
\pgfpathrectangle{\pgfqpoint{0.600000in}{0.600000in}}{\pgfqpoint{3.900000in}{3.900000in}}%
\pgfusepath{clip}%
\pgfsetbuttcap%
\pgfsetroundjoin%
\definecolor{currentfill}{rgb}{0.121569,0.466667,0.705882}%
\pgfsetfillcolor{currentfill}%
\pgfsetlinewidth{1.003750pt}%
\definecolor{currentstroke}{rgb}{0.121569,0.466667,0.705882}%
\pgfsetstrokecolor{currentstroke}%
\pgfsetdash{}{0pt}%
\pgfpathmoveto{\pgfqpoint{1.917216in}{2.034602in}}%
\pgfpathcurveto{\pgfqpoint{1.928266in}{2.034602in}}{\pgfqpoint{1.938865in}{2.038992in}}{\pgfqpoint{1.946679in}{2.046806in}}%
\pgfpathcurveto{\pgfqpoint{1.954493in}{2.054620in}}{\pgfqpoint{1.958883in}{2.065219in}}{\pgfqpoint{1.958883in}{2.076269in}}%
\pgfpathcurveto{\pgfqpoint{1.958883in}{2.087319in}}{\pgfqpoint{1.954493in}{2.097918in}}{\pgfqpoint{1.946679in}{2.105732in}}%
\pgfpathcurveto{\pgfqpoint{1.938865in}{2.113545in}}{\pgfqpoint{1.928266in}{2.117935in}}{\pgfqpoint{1.917216in}{2.117935in}}%
\pgfpathcurveto{\pgfqpoint{1.906166in}{2.117935in}}{\pgfqpoint{1.895567in}{2.113545in}}{\pgfqpoint{1.887753in}{2.105732in}}%
\pgfpathcurveto{\pgfqpoint{1.879940in}{2.097918in}}{\pgfqpoint{1.875550in}{2.087319in}}{\pgfqpoint{1.875550in}{2.076269in}}%
\pgfpathcurveto{\pgfqpoint{1.875550in}{2.065219in}}{\pgfqpoint{1.879940in}{2.054620in}}{\pgfqpoint{1.887753in}{2.046806in}}%
\pgfpathcurveto{\pgfqpoint{1.895567in}{2.038992in}}{\pgfqpoint{1.906166in}{2.034602in}}{\pgfqpoint{1.917216in}{2.034602in}}%
\pgfpathclose%
\pgfusepath{stroke,fill}%
\end{pgfscope}%
\begin{pgfscope}%
\pgfpathrectangle{\pgfqpoint{0.600000in}{0.600000in}}{\pgfqpoint{3.900000in}{3.900000in}}%
\pgfusepath{clip}%
\pgfsetbuttcap%
\pgfsetroundjoin%
\definecolor{currentfill}{rgb}{0.121569,0.466667,0.705882}%
\pgfsetfillcolor{currentfill}%
\pgfsetlinewidth{1.003750pt}%
\definecolor{currentstroke}{rgb}{0.121569,0.466667,0.705882}%
\pgfsetstrokecolor{currentstroke}%
\pgfsetdash{}{0pt}%
\pgfpathmoveto{\pgfqpoint{3.248544in}{2.695462in}}%
\pgfpathcurveto{\pgfqpoint{3.259594in}{2.695462in}}{\pgfqpoint{3.270193in}{2.699852in}}{\pgfqpoint{3.278006in}{2.707666in}}%
\pgfpathcurveto{\pgfqpoint{3.285820in}{2.715479in}}{\pgfqpoint{3.290210in}{2.726078in}}{\pgfqpoint{3.290210in}{2.737128in}}%
\pgfpathcurveto{\pgfqpoint{3.290210in}{2.748179in}}{\pgfqpoint{3.285820in}{2.758778in}}{\pgfqpoint{3.278006in}{2.766591in}}%
\pgfpathcurveto{\pgfqpoint{3.270193in}{2.774405in}}{\pgfqpoint{3.259594in}{2.778795in}}{\pgfqpoint{3.248544in}{2.778795in}}%
\pgfpathcurveto{\pgfqpoint{3.237493in}{2.778795in}}{\pgfqpoint{3.226894in}{2.774405in}}{\pgfqpoint{3.219081in}{2.766591in}}%
\pgfpathcurveto{\pgfqpoint{3.211267in}{2.758778in}}{\pgfqpoint{3.206877in}{2.748179in}}{\pgfqpoint{3.206877in}{2.737128in}}%
\pgfpathcurveto{\pgfqpoint{3.206877in}{2.726078in}}{\pgfqpoint{3.211267in}{2.715479in}}{\pgfqpoint{3.219081in}{2.707666in}}%
\pgfpathcurveto{\pgfqpoint{3.226894in}{2.699852in}}{\pgfqpoint{3.237493in}{2.695462in}}{\pgfqpoint{3.248544in}{2.695462in}}%
\pgfpathclose%
\pgfusepath{stroke,fill}%
\end{pgfscope}%
\begin{pgfscope}%
\pgfpathrectangle{\pgfqpoint{0.600000in}{0.600000in}}{\pgfqpoint{3.900000in}{3.900000in}}%
\pgfusepath{clip}%
\pgfsetbuttcap%
\pgfsetroundjoin%
\definecolor{currentfill}{rgb}{0.121569,0.466667,0.705882}%
\pgfsetfillcolor{currentfill}%
\pgfsetlinewidth{1.003750pt}%
\definecolor{currentstroke}{rgb}{0.121569,0.466667,0.705882}%
\pgfsetstrokecolor{currentstroke}%
\pgfsetdash{}{0pt}%
\pgfpathmoveto{\pgfqpoint{2.437928in}{3.502200in}}%
\pgfpathcurveto{\pgfqpoint{2.448978in}{3.502200in}}{\pgfqpoint{2.459577in}{3.506591in}}{\pgfqpoint{2.467391in}{3.514404in}}%
\pgfpathcurveto{\pgfqpoint{2.475205in}{3.522218in}}{\pgfqpoint{2.479595in}{3.532817in}}{\pgfqpoint{2.479595in}{3.543867in}}%
\pgfpathcurveto{\pgfqpoint{2.479595in}{3.554917in}}{\pgfqpoint{2.475205in}{3.565516in}}{\pgfqpoint{2.467391in}{3.573330in}}%
\pgfpathcurveto{\pgfqpoint{2.459577in}{3.581143in}}{\pgfqpoint{2.448978in}{3.585534in}}{\pgfqpoint{2.437928in}{3.585534in}}%
\pgfpathcurveto{\pgfqpoint{2.426878in}{3.585534in}}{\pgfqpoint{2.416279in}{3.581143in}}{\pgfqpoint{2.408465in}{3.573330in}}%
\pgfpathcurveto{\pgfqpoint{2.400652in}{3.565516in}}{\pgfqpoint{2.396262in}{3.554917in}}{\pgfqpoint{2.396262in}{3.543867in}}%
\pgfpathcurveto{\pgfqpoint{2.396262in}{3.532817in}}{\pgfqpoint{2.400652in}{3.522218in}}{\pgfqpoint{2.408465in}{3.514404in}}%
\pgfpathcurveto{\pgfqpoint{2.416279in}{3.506591in}}{\pgfqpoint{2.426878in}{3.502200in}}{\pgfqpoint{2.437928in}{3.502200in}}%
\pgfpathclose%
\pgfusepath{stroke,fill}%
\end{pgfscope}%
\begin{pgfscope}%
\pgfpathrectangle{\pgfqpoint{0.600000in}{0.600000in}}{\pgfqpoint{3.900000in}{3.900000in}}%
\pgfusepath{clip}%
\pgfsetbuttcap%
\pgfsetroundjoin%
\definecolor{currentfill}{rgb}{0.121569,0.466667,0.705882}%
\pgfsetfillcolor{currentfill}%
\pgfsetlinewidth{1.003750pt}%
\definecolor{currentstroke}{rgb}{0.121569,0.466667,0.705882}%
\pgfsetstrokecolor{currentstroke}%
\pgfsetdash{}{0pt}%
\pgfpathmoveto{\pgfqpoint{2.297117in}{2.238131in}}%
\pgfpathcurveto{\pgfqpoint{2.308167in}{2.238131in}}{\pgfqpoint{2.318766in}{2.242521in}}{\pgfqpoint{2.326580in}{2.250335in}}%
\pgfpathcurveto{\pgfqpoint{2.334393in}{2.258148in}}{\pgfqpoint{2.338783in}{2.268747in}}{\pgfqpoint{2.338783in}{2.279797in}}%
\pgfpathcurveto{\pgfqpoint{2.338783in}{2.290848in}}{\pgfqpoint{2.334393in}{2.301447in}}{\pgfqpoint{2.326580in}{2.309260in}}%
\pgfpathcurveto{\pgfqpoint{2.318766in}{2.317074in}}{\pgfqpoint{2.308167in}{2.321464in}}{\pgfqpoint{2.297117in}{2.321464in}}%
\pgfpathcurveto{\pgfqpoint{2.286067in}{2.321464in}}{\pgfqpoint{2.275468in}{2.317074in}}{\pgfqpoint{2.267654in}{2.309260in}}%
\pgfpathcurveto{\pgfqpoint{2.259840in}{2.301447in}}{\pgfqpoint{2.255450in}{2.290848in}}{\pgfqpoint{2.255450in}{2.279797in}}%
\pgfpathcurveto{\pgfqpoint{2.255450in}{2.268747in}}{\pgfqpoint{2.259840in}{2.258148in}}{\pgfqpoint{2.267654in}{2.250335in}}%
\pgfpathcurveto{\pgfqpoint{2.275468in}{2.242521in}}{\pgfqpoint{2.286067in}{2.238131in}}{\pgfqpoint{2.297117in}{2.238131in}}%
\pgfpathclose%
\pgfusepath{stroke,fill}%
\end{pgfscope}%
\begin{pgfscope}%
\pgfpathrectangle{\pgfqpoint{0.600000in}{0.600000in}}{\pgfqpoint{3.900000in}{3.900000in}}%
\pgfusepath{clip}%
\pgfsetbuttcap%
\pgfsetroundjoin%
\definecolor{currentfill}{rgb}{0.121569,0.466667,0.705882}%
\pgfsetfillcolor{currentfill}%
\pgfsetlinewidth{1.003750pt}%
\definecolor{currentstroke}{rgb}{0.121569,0.466667,0.705882}%
\pgfsetstrokecolor{currentstroke}%
\pgfsetdash{}{0pt}%
\pgfpathmoveto{\pgfqpoint{2.504630in}{2.638008in}}%
\pgfpathcurveto{\pgfqpoint{2.515680in}{2.638008in}}{\pgfqpoint{2.526279in}{2.642398in}}{\pgfqpoint{2.534093in}{2.650212in}}%
\pgfpathcurveto{\pgfqpoint{2.541906in}{2.658026in}}{\pgfqpoint{2.546297in}{2.668625in}}{\pgfqpoint{2.546297in}{2.679675in}}%
\pgfpathcurveto{\pgfqpoint{2.546297in}{2.690725in}}{\pgfqpoint{2.541906in}{2.701324in}}{\pgfqpoint{2.534093in}{2.709138in}}%
\pgfpathcurveto{\pgfqpoint{2.526279in}{2.716951in}}{\pgfqpoint{2.515680in}{2.721342in}}{\pgfqpoint{2.504630in}{2.721342in}}%
\pgfpathcurveto{\pgfqpoint{2.493580in}{2.721342in}}{\pgfqpoint{2.482981in}{2.716951in}}{\pgfqpoint{2.475167in}{2.709138in}}%
\pgfpathcurveto{\pgfqpoint{2.467353in}{2.701324in}}{\pgfqpoint{2.462963in}{2.690725in}}{\pgfqpoint{2.462963in}{2.679675in}}%
\pgfpathcurveto{\pgfqpoint{2.462963in}{2.668625in}}{\pgfqpoint{2.467353in}{2.658026in}}{\pgfqpoint{2.475167in}{2.650212in}}%
\pgfpathcurveto{\pgfqpoint{2.482981in}{2.642398in}}{\pgfqpoint{2.493580in}{2.638008in}}{\pgfqpoint{2.504630in}{2.638008in}}%
\pgfpathclose%
\pgfusepath{stroke,fill}%
\end{pgfscope}%
\begin{pgfscope}%
\pgfpathrectangle{\pgfqpoint{0.600000in}{0.600000in}}{\pgfqpoint{3.900000in}{3.900000in}}%
\pgfusepath{clip}%
\pgfsetbuttcap%
\pgfsetroundjoin%
\definecolor{currentfill}{rgb}{0.121569,0.466667,0.705882}%
\pgfsetfillcolor{currentfill}%
\pgfsetlinewidth{1.003750pt}%
\definecolor{currentstroke}{rgb}{0.121569,0.466667,0.705882}%
\pgfsetstrokecolor{currentstroke}%
\pgfsetdash{}{0pt}%
\pgfpathmoveto{\pgfqpoint{1.529699in}{2.213903in}}%
\pgfpathcurveto{\pgfqpoint{1.540749in}{2.213903in}}{\pgfqpoint{1.551348in}{2.218293in}}{\pgfqpoint{1.559162in}{2.226107in}}%
\pgfpathcurveto{\pgfqpoint{1.566975in}{2.233920in}}{\pgfqpoint{1.571365in}{2.244519in}}{\pgfqpoint{1.571365in}{2.255570in}}%
\pgfpathcurveto{\pgfqpoint{1.571365in}{2.266620in}}{\pgfqpoint{1.566975in}{2.277219in}}{\pgfqpoint{1.559162in}{2.285032in}}%
\pgfpathcurveto{\pgfqpoint{1.551348in}{2.292846in}}{\pgfqpoint{1.540749in}{2.297236in}}{\pgfqpoint{1.529699in}{2.297236in}}%
\pgfpathcurveto{\pgfqpoint{1.518649in}{2.297236in}}{\pgfqpoint{1.508050in}{2.292846in}}{\pgfqpoint{1.500236in}{2.285032in}}%
\pgfpathcurveto{\pgfqpoint{1.492422in}{2.277219in}}{\pgfqpoint{1.488032in}{2.266620in}}{\pgfqpoint{1.488032in}{2.255570in}}%
\pgfpathcurveto{\pgfqpoint{1.488032in}{2.244519in}}{\pgfqpoint{1.492422in}{2.233920in}}{\pgfqpoint{1.500236in}{2.226107in}}%
\pgfpathcurveto{\pgfqpoint{1.508050in}{2.218293in}}{\pgfqpoint{1.518649in}{2.213903in}}{\pgfqpoint{1.529699in}{2.213903in}}%
\pgfpathclose%
\pgfusepath{stroke,fill}%
\end{pgfscope}%
\begin{pgfscope}%
\pgfpathrectangle{\pgfqpoint{0.600000in}{0.600000in}}{\pgfqpoint{3.900000in}{3.900000in}}%
\pgfusepath{clip}%
\pgfsetbuttcap%
\pgfsetroundjoin%
\definecolor{currentfill}{rgb}{0.121569,0.466667,0.705882}%
\pgfsetfillcolor{currentfill}%
\pgfsetlinewidth{1.003750pt}%
\definecolor{currentstroke}{rgb}{0.121569,0.466667,0.705882}%
\pgfsetstrokecolor{currentstroke}%
\pgfsetdash{}{0pt}%
\pgfpathmoveto{\pgfqpoint{1.957461in}{3.550706in}}%
\pgfpathcurveto{\pgfqpoint{1.968512in}{3.550706in}}{\pgfqpoint{1.979111in}{3.555096in}}{\pgfqpoint{1.986924in}{3.562909in}}%
\pgfpathcurveto{\pgfqpoint{1.994738in}{3.570723in}}{\pgfqpoint{1.999128in}{3.581322in}}{\pgfqpoint{1.999128in}{3.592372in}}%
\pgfpathcurveto{\pgfqpoint{1.999128in}{3.603422in}}{\pgfqpoint{1.994738in}{3.614021in}}{\pgfqpoint{1.986924in}{3.621835in}}%
\pgfpathcurveto{\pgfqpoint{1.979111in}{3.629649in}}{\pgfqpoint{1.968512in}{3.634039in}}{\pgfqpoint{1.957461in}{3.634039in}}%
\pgfpathcurveto{\pgfqpoint{1.946411in}{3.634039in}}{\pgfqpoint{1.935812in}{3.629649in}}{\pgfqpoint{1.927999in}{3.621835in}}%
\pgfpathcurveto{\pgfqpoint{1.920185in}{3.614021in}}{\pgfqpoint{1.915795in}{3.603422in}}{\pgfqpoint{1.915795in}{3.592372in}}%
\pgfpathcurveto{\pgfqpoint{1.915795in}{3.581322in}}{\pgfqpoint{1.920185in}{3.570723in}}{\pgfqpoint{1.927999in}{3.562909in}}%
\pgfpathcurveto{\pgfqpoint{1.935812in}{3.555096in}}{\pgfqpoint{1.946411in}{3.550706in}}{\pgfqpoint{1.957461in}{3.550706in}}%
\pgfpathclose%
\pgfusepath{stroke,fill}%
\end{pgfscope}%
\begin{pgfscope}%
\pgfpathrectangle{\pgfqpoint{0.600000in}{0.600000in}}{\pgfqpoint{3.900000in}{3.900000in}}%
\pgfusepath{clip}%
\pgfsetbuttcap%
\pgfsetroundjoin%
\definecolor{currentfill}{rgb}{0.121569,0.466667,0.705882}%
\pgfsetfillcolor{currentfill}%
\pgfsetlinewidth{1.003750pt}%
\definecolor{currentstroke}{rgb}{0.121569,0.466667,0.705882}%
\pgfsetstrokecolor{currentstroke}%
\pgfsetdash{}{0pt}%
\pgfpathmoveto{\pgfqpoint{2.051124in}{2.335282in}}%
\pgfpathcurveto{\pgfqpoint{2.062174in}{2.335282in}}{\pgfqpoint{2.072773in}{2.339672in}}{\pgfqpoint{2.080586in}{2.347486in}}%
\pgfpathcurveto{\pgfqpoint{2.088400in}{2.355299in}}{\pgfqpoint{2.092790in}{2.365898in}}{\pgfqpoint{2.092790in}{2.376948in}}%
\pgfpathcurveto{\pgfqpoint{2.092790in}{2.387998in}}{\pgfqpoint{2.088400in}{2.398598in}}{\pgfqpoint{2.080586in}{2.406411in}}%
\pgfpathcurveto{\pgfqpoint{2.072773in}{2.414225in}}{\pgfqpoint{2.062174in}{2.418615in}}{\pgfqpoint{2.051124in}{2.418615in}}%
\pgfpathcurveto{\pgfqpoint{2.040073in}{2.418615in}}{\pgfqpoint{2.029474in}{2.414225in}}{\pgfqpoint{2.021661in}{2.406411in}}%
\pgfpathcurveto{\pgfqpoint{2.013847in}{2.398598in}}{\pgfqpoint{2.009457in}{2.387998in}}{\pgfqpoint{2.009457in}{2.376948in}}%
\pgfpathcurveto{\pgfqpoint{2.009457in}{2.365898in}}{\pgfqpoint{2.013847in}{2.355299in}}{\pgfqpoint{2.021661in}{2.347486in}}%
\pgfpathcurveto{\pgfqpoint{2.029474in}{2.339672in}}{\pgfqpoint{2.040073in}{2.335282in}}{\pgfqpoint{2.051124in}{2.335282in}}%
\pgfpathclose%
\pgfusepath{stroke,fill}%
\end{pgfscope}%
\begin{pgfscope}%
\pgfpathrectangle{\pgfqpoint{0.600000in}{0.600000in}}{\pgfqpoint{3.900000in}{3.900000in}}%
\pgfusepath{clip}%
\pgfsetbuttcap%
\pgfsetroundjoin%
\definecolor{currentfill}{rgb}{0.121569,0.466667,0.705882}%
\pgfsetfillcolor{currentfill}%
\pgfsetlinewidth{1.003750pt}%
\definecolor{currentstroke}{rgb}{0.121569,0.466667,0.705882}%
\pgfsetstrokecolor{currentstroke}%
\pgfsetdash{}{0pt}%
\pgfpathmoveto{\pgfqpoint{3.776974in}{3.115779in}}%
\pgfpathcurveto{\pgfqpoint{3.788024in}{3.115779in}}{\pgfqpoint{3.798623in}{3.120170in}}{\pgfqpoint{3.806436in}{3.127983in}}%
\pgfpathcurveto{\pgfqpoint{3.814250in}{3.135797in}}{\pgfqpoint{3.818640in}{3.146396in}}{\pgfqpoint{3.818640in}{3.157446in}}%
\pgfpathcurveto{\pgfqpoint{3.818640in}{3.168496in}}{\pgfqpoint{3.814250in}{3.179095in}}{\pgfqpoint{3.806436in}{3.186909in}}%
\pgfpathcurveto{\pgfqpoint{3.798623in}{3.194723in}}{\pgfqpoint{3.788024in}{3.199113in}}{\pgfqpoint{3.776974in}{3.199113in}}%
\pgfpathcurveto{\pgfqpoint{3.765923in}{3.199113in}}{\pgfqpoint{3.755324in}{3.194723in}}{\pgfqpoint{3.747511in}{3.186909in}}%
\pgfpathcurveto{\pgfqpoint{3.739697in}{3.179095in}}{\pgfqpoint{3.735307in}{3.168496in}}{\pgfqpoint{3.735307in}{3.157446in}}%
\pgfpathcurveto{\pgfqpoint{3.735307in}{3.146396in}}{\pgfqpoint{3.739697in}{3.135797in}}{\pgfqpoint{3.747511in}{3.127983in}}%
\pgfpathcurveto{\pgfqpoint{3.755324in}{3.120170in}}{\pgfqpoint{3.765923in}{3.115779in}}{\pgfqpoint{3.776974in}{3.115779in}}%
\pgfpathclose%
\pgfusepath{stroke,fill}%
\end{pgfscope}%
\begin{pgfscope}%
\pgfpathrectangle{\pgfqpoint{0.600000in}{0.600000in}}{\pgfqpoint{3.900000in}{3.900000in}}%
\pgfusepath{clip}%
\pgfsetbuttcap%
\pgfsetroundjoin%
\definecolor{currentfill}{rgb}{0.121569,0.466667,0.705882}%
\pgfsetfillcolor{currentfill}%
\pgfsetlinewidth{1.003750pt}%
\definecolor{currentstroke}{rgb}{0.121569,0.466667,0.705882}%
\pgfsetstrokecolor{currentstroke}%
\pgfsetdash{}{0pt}%
\pgfpathmoveto{\pgfqpoint{3.101115in}{2.169920in}}%
\pgfpathcurveto{\pgfqpoint{3.112165in}{2.169920in}}{\pgfqpoint{3.122764in}{2.174310in}}{\pgfqpoint{3.130578in}{2.182123in}}%
\pgfpathcurveto{\pgfqpoint{3.138391in}{2.189937in}}{\pgfqpoint{3.142782in}{2.200536in}}{\pgfqpoint{3.142782in}{2.211586in}}%
\pgfpathcurveto{\pgfqpoint{3.142782in}{2.222636in}}{\pgfqpoint{3.138391in}{2.233235in}}{\pgfqpoint{3.130578in}{2.241049in}}%
\pgfpathcurveto{\pgfqpoint{3.122764in}{2.248863in}}{\pgfqpoint{3.112165in}{2.253253in}}{\pgfqpoint{3.101115in}{2.253253in}}%
\pgfpathcurveto{\pgfqpoint{3.090065in}{2.253253in}}{\pgfqpoint{3.079466in}{2.248863in}}{\pgfqpoint{3.071652in}{2.241049in}}%
\pgfpathcurveto{\pgfqpoint{3.063839in}{2.233235in}}{\pgfqpoint{3.059448in}{2.222636in}}{\pgfqpoint{3.059448in}{2.211586in}}%
\pgfpathcurveto{\pgfqpoint{3.059448in}{2.200536in}}{\pgfqpoint{3.063839in}{2.189937in}}{\pgfqpoint{3.071652in}{2.182123in}}%
\pgfpathcurveto{\pgfqpoint{3.079466in}{2.174310in}}{\pgfqpoint{3.090065in}{2.169920in}}{\pgfqpoint{3.101115in}{2.169920in}}%
\pgfpathclose%
\pgfusepath{stroke,fill}%
\end{pgfscope}%
\begin{pgfscope}%
\pgfpathrectangle{\pgfqpoint{0.600000in}{0.600000in}}{\pgfqpoint{3.900000in}{3.900000in}}%
\pgfusepath{clip}%
\pgfsetbuttcap%
\pgfsetroundjoin%
\definecolor{currentfill}{rgb}{0.121569,0.466667,0.705882}%
\pgfsetfillcolor{currentfill}%
\pgfsetlinewidth{1.003750pt}%
\definecolor{currentstroke}{rgb}{0.121569,0.466667,0.705882}%
\pgfsetstrokecolor{currentstroke}%
\pgfsetdash{}{0pt}%
\pgfpathmoveto{\pgfqpoint{1.676345in}{3.135956in}}%
\pgfpathcurveto{\pgfqpoint{1.687395in}{3.135956in}}{\pgfqpoint{1.697994in}{3.140347in}}{\pgfqpoint{1.705807in}{3.148160in}}%
\pgfpathcurveto{\pgfqpoint{1.713621in}{3.155974in}}{\pgfqpoint{1.718011in}{3.166573in}}{\pgfqpoint{1.718011in}{3.177623in}}%
\pgfpathcurveto{\pgfqpoint{1.718011in}{3.188673in}}{\pgfqpoint{1.713621in}{3.199272in}}{\pgfqpoint{1.705807in}{3.207086in}}%
\pgfpathcurveto{\pgfqpoint{1.697994in}{3.214900in}}{\pgfqpoint{1.687395in}{3.219290in}}{\pgfqpoint{1.676345in}{3.219290in}}%
\pgfpathcurveto{\pgfqpoint{1.665294in}{3.219290in}}{\pgfqpoint{1.654695in}{3.214900in}}{\pgfqpoint{1.646882in}{3.207086in}}%
\pgfpathcurveto{\pgfqpoint{1.639068in}{3.199272in}}{\pgfqpoint{1.634678in}{3.188673in}}{\pgfqpoint{1.634678in}{3.177623in}}%
\pgfpathcurveto{\pgfqpoint{1.634678in}{3.166573in}}{\pgfqpoint{1.639068in}{3.155974in}}{\pgfqpoint{1.646882in}{3.148160in}}%
\pgfpathcurveto{\pgfqpoint{1.654695in}{3.140347in}}{\pgfqpoint{1.665294in}{3.135956in}}{\pgfqpoint{1.676345in}{3.135956in}}%
\pgfpathclose%
\pgfusepath{stroke,fill}%
\end{pgfscope}%
\begin{pgfscope}%
\pgfpathrectangle{\pgfqpoint{0.600000in}{0.600000in}}{\pgfqpoint{3.900000in}{3.900000in}}%
\pgfusepath{clip}%
\pgfsetbuttcap%
\pgfsetroundjoin%
\definecolor{currentfill}{rgb}{0.121569,0.466667,0.705882}%
\pgfsetfillcolor{currentfill}%
\pgfsetlinewidth{1.003750pt}%
\definecolor{currentstroke}{rgb}{0.121569,0.466667,0.705882}%
\pgfsetstrokecolor{currentstroke}%
\pgfsetdash{}{0pt}%
\pgfpathmoveto{\pgfqpoint{2.879367in}{2.265170in}}%
\pgfpathcurveto{\pgfqpoint{2.890417in}{2.265170in}}{\pgfqpoint{2.901016in}{2.269561in}}{\pgfqpoint{2.908830in}{2.277374in}}%
\pgfpathcurveto{\pgfqpoint{2.916644in}{2.285188in}}{\pgfqpoint{2.921034in}{2.295787in}}{\pgfqpoint{2.921034in}{2.306837in}}%
\pgfpathcurveto{\pgfqpoint{2.921034in}{2.317887in}}{\pgfqpoint{2.916644in}{2.328486in}}{\pgfqpoint{2.908830in}{2.336300in}}%
\pgfpathcurveto{\pgfqpoint{2.901016in}{2.344113in}}{\pgfqpoint{2.890417in}{2.348504in}}{\pgfqpoint{2.879367in}{2.348504in}}%
\pgfpathcurveto{\pgfqpoint{2.868317in}{2.348504in}}{\pgfqpoint{2.857718in}{2.344113in}}{\pgfqpoint{2.849905in}{2.336300in}}%
\pgfpathcurveto{\pgfqpoint{2.842091in}{2.328486in}}{\pgfqpoint{2.837701in}{2.317887in}}{\pgfqpoint{2.837701in}{2.306837in}}%
\pgfpathcurveto{\pgfqpoint{2.837701in}{2.295787in}}{\pgfqpoint{2.842091in}{2.285188in}}{\pgfqpoint{2.849905in}{2.277374in}}%
\pgfpathcurveto{\pgfqpoint{2.857718in}{2.269561in}}{\pgfqpoint{2.868317in}{2.265170in}}{\pgfqpoint{2.879367in}{2.265170in}}%
\pgfpathclose%
\pgfusepath{stroke,fill}%
\end{pgfscope}%
\begin{pgfscope}%
\pgfsetbuttcap%
\pgfsetroundjoin%
\definecolor{currentfill}{rgb}{0.000000,0.000000,0.000000}%
\pgfsetfillcolor{currentfill}%
\pgfsetlinewidth{0.803000pt}%
\definecolor{currentstroke}{rgb}{0.000000,0.000000,0.000000}%
\pgfsetstrokecolor{currentstroke}%
\pgfsetdash{}{0pt}%
\pgfsys@defobject{currentmarker}{\pgfqpoint{0.000000in}{0.000000in}}{\pgfqpoint{0.000000in}{0.048611in}}{%
\pgfpathmoveto{\pgfqpoint{0.000000in}{0.000000in}}%
\pgfpathlineto{\pgfqpoint{0.000000in}{0.048611in}}%
\pgfusepath{stroke,fill}%
}%
\begin{pgfscope}%
\pgfsys@transformshift{1.010526in}{0.600000in}%
\pgfsys@useobject{currentmarker}{}%
\end{pgfscope}%
\end{pgfscope}%
\begin{pgfscope}%
\pgfsetbuttcap%
\pgfsetroundjoin%
\definecolor{currentfill}{rgb}{0.000000,0.000000,0.000000}%
\pgfsetfillcolor{currentfill}%
\pgfsetlinewidth{0.803000pt}%
\definecolor{currentstroke}{rgb}{0.000000,0.000000,0.000000}%
\pgfsetstrokecolor{currentstroke}%
\pgfsetdash{}{0pt}%
\pgfsys@defobject{currentmarker}{\pgfqpoint{0.000000in}{-0.048611in}}{\pgfqpoint{0.000000in}{0.000000in}}{%
\pgfpathmoveto{\pgfqpoint{0.000000in}{0.000000in}}%
\pgfpathlineto{\pgfqpoint{0.000000in}{-0.048611in}}%
\pgfusepath{stroke,fill}%
}%
\begin{pgfscope}%
\pgfsys@transformshift{1.010526in}{4.500000in}%
\pgfsys@useobject{currentmarker}{}%
\end{pgfscope}%
\end{pgfscope}%
\begin{pgfscope}%
\definecolor{textcolor}{rgb}{0.000000,0.000000,0.000000}%
\pgfsetstrokecolor{textcolor}%
\pgfsetfillcolor{textcolor}%
\pgftext[x=1.010526in,y=0.551389in,,top]{\color{textcolor}\sffamily\fontsize{10.000000}{12.000000}\selectfont −3}%
\end{pgfscope}%
\begin{pgfscope}%
\pgfsetbuttcap%
\pgfsetroundjoin%
\definecolor{currentfill}{rgb}{0.000000,0.000000,0.000000}%
\pgfsetfillcolor{currentfill}%
\pgfsetlinewidth{0.803000pt}%
\definecolor{currentstroke}{rgb}{0.000000,0.000000,0.000000}%
\pgfsetstrokecolor{currentstroke}%
\pgfsetdash{}{0pt}%
\pgfsys@defobject{currentmarker}{\pgfqpoint{0.000000in}{0.000000in}}{\pgfqpoint{0.000000in}{0.048611in}}{%
\pgfpathmoveto{\pgfqpoint{0.000000in}{0.000000in}}%
\pgfpathlineto{\pgfqpoint{0.000000in}{0.048611in}}%
\pgfusepath{stroke,fill}%
}%
\begin{pgfscope}%
\pgfsys@transformshift{1.523684in}{0.600000in}%
\pgfsys@useobject{currentmarker}{}%
\end{pgfscope}%
\end{pgfscope}%
\begin{pgfscope}%
\pgfsetbuttcap%
\pgfsetroundjoin%
\definecolor{currentfill}{rgb}{0.000000,0.000000,0.000000}%
\pgfsetfillcolor{currentfill}%
\pgfsetlinewidth{0.803000pt}%
\definecolor{currentstroke}{rgb}{0.000000,0.000000,0.000000}%
\pgfsetstrokecolor{currentstroke}%
\pgfsetdash{}{0pt}%
\pgfsys@defobject{currentmarker}{\pgfqpoint{0.000000in}{-0.048611in}}{\pgfqpoint{0.000000in}{0.000000in}}{%
\pgfpathmoveto{\pgfqpoint{0.000000in}{0.000000in}}%
\pgfpathlineto{\pgfqpoint{0.000000in}{-0.048611in}}%
\pgfusepath{stroke,fill}%
}%
\begin{pgfscope}%
\pgfsys@transformshift{1.523684in}{4.500000in}%
\pgfsys@useobject{currentmarker}{}%
\end{pgfscope}%
\end{pgfscope}%
\begin{pgfscope}%
\definecolor{textcolor}{rgb}{0.000000,0.000000,0.000000}%
\pgfsetstrokecolor{textcolor}%
\pgfsetfillcolor{textcolor}%
\pgftext[x=1.523684in,y=0.551389in,,top]{\color{textcolor}\sffamily\fontsize{10.000000}{12.000000}\selectfont −2}%
\end{pgfscope}%
\begin{pgfscope}%
\pgfsetbuttcap%
\pgfsetroundjoin%
\definecolor{currentfill}{rgb}{0.000000,0.000000,0.000000}%
\pgfsetfillcolor{currentfill}%
\pgfsetlinewidth{0.803000pt}%
\definecolor{currentstroke}{rgb}{0.000000,0.000000,0.000000}%
\pgfsetstrokecolor{currentstroke}%
\pgfsetdash{}{0pt}%
\pgfsys@defobject{currentmarker}{\pgfqpoint{0.000000in}{0.000000in}}{\pgfqpoint{0.000000in}{0.048611in}}{%
\pgfpathmoveto{\pgfqpoint{0.000000in}{0.000000in}}%
\pgfpathlineto{\pgfqpoint{0.000000in}{0.048611in}}%
\pgfusepath{stroke,fill}%
}%
\begin{pgfscope}%
\pgfsys@transformshift{2.036842in}{0.600000in}%
\pgfsys@useobject{currentmarker}{}%
\end{pgfscope}%
\end{pgfscope}%
\begin{pgfscope}%
\pgfsetbuttcap%
\pgfsetroundjoin%
\definecolor{currentfill}{rgb}{0.000000,0.000000,0.000000}%
\pgfsetfillcolor{currentfill}%
\pgfsetlinewidth{0.803000pt}%
\definecolor{currentstroke}{rgb}{0.000000,0.000000,0.000000}%
\pgfsetstrokecolor{currentstroke}%
\pgfsetdash{}{0pt}%
\pgfsys@defobject{currentmarker}{\pgfqpoint{0.000000in}{-0.048611in}}{\pgfqpoint{0.000000in}{0.000000in}}{%
\pgfpathmoveto{\pgfqpoint{0.000000in}{0.000000in}}%
\pgfpathlineto{\pgfqpoint{0.000000in}{-0.048611in}}%
\pgfusepath{stroke,fill}%
}%
\begin{pgfscope}%
\pgfsys@transformshift{2.036842in}{4.500000in}%
\pgfsys@useobject{currentmarker}{}%
\end{pgfscope}%
\end{pgfscope}%
\begin{pgfscope}%
\definecolor{textcolor}{rgb}{0.000000,0.000000,0.000000}%
\pgfsetstrokecolor{textcolor}%
\pgfsetfillcolor{textcolor}%
\pgftext[x=2.036842in,y=0.551389in,,top]{\color{textcolor}\sffamily\fontsize{10.000000}{12.000000}\selectfont −1}%
\end{pgfscope}%
\begin{pgfscope}%
\pgfsetbuttcap%
\pgfsetroundjoin%
\definecolor{currentfill}{rgb}{0.000000,0.000000,0.000000}%
\pgfsetfillcolor{currentfill}%
\pgfsetlinewidth{0.803000pt}%
\definecolor{currentstroke}{rgb}{0.000000,0.000000,0.000000}%
\pgfsetstrokecolor{currentstroke}%
\pgfsetdash{}{0pt}%
\pgfsys@defobject{currentmarker}{\pgfqpoint{0.000000in}{0.000000in}}{\pgfqpoint{0.000000in}{0.048611in}}{%
\pgfpathmoveto{\pgfqpoint{0.000000in}{0.000000in}}%
\pgfpathlineto{\pgfqpoint{0.000000in}{0.048611in}}%
\pgfusepath{stroke,fill}%
}%
\begin{pgfscope}%
\pgfsys@transformshift{2.550000in}{0.600000in}%
\pgfsys@useobject{currentmarker}{}%
\end{pgfscope}%
\end{pgfscope}%
\begin{pgfscope}%
\pgfsetbuttcap%
\pgfsetroundjoin%
\definecolor{currentfill}{rgb}{0.000000,0.000000,0.000000}%
\pgfsetfillcolor{currentfill}%
\pgfsetlinewidth{0.803000pt}%
\definecolor{currentstroke}{rgb}{0.000000,0.000000,0.000000}%
\pgfsetstrokecolor{currentstroke}%
\pgfsetdash{}{0pt}%
\pgfsys@defobject{currentmarker}{\pgfqpoint{0.000000in}{-0.048611in}}{\pgfqpoint{0.000000in}{0.000000in}}{%
\pgfpathmoveto{\pgfqpoint{0.000000in}{0.000000in}}%
\pgfpathlineto{\pgfqpoint{0.000000in}{-0.048611in}}%
\pgfusepath{stroke,fill}%
}%
\begin{pgfscope}%
\pgfsys@transformshift{2.550000in}{4.500000in}%
\pgfsys@useobject{currentmarker}{}%
\end{pgfscope}%
\end{pgfscope}%
\begin{pgfscope}%
\definecolor{textcolor}{rgb}{0.000000,0.000000,0.000000}%
\pgfsetstrokecolor{textcolor}%
\pgfsetfillcolor{textcolor}%
\pgftext[x=2.550000in,y=0.551389in,,top]{\color{textcolor}\sffamily\fontsize{10.000000}{12.000000}\selectfont 0}%
\end{pgfscope}%
\begin{pgfscope}%
\pgfsetbuttcap%
\pgfsetroundjoin%
\definecolor{currentfill}{rgb}{0.000000,0.000000,0.000000}%
\pgfsetfillcolor{currentfill}%
\pgfsetlinewidth{0.803000pt}%
\definecolor{currentstroke}{rgb}{0.000000,0.000000,0.000000}%
\pgfsetstrokecolor{currentstroke}%
\pgfsetdash{}{0pt}%
\pgfsys@defobject{currentmarker}{\pgfqpoint{0.000000in}{0.000000in}}{\pgfqpoint{0.000000in}{0.048611in}}{%
\pgfpathmoveto{\pgfqpoint{0.000000in}{0.000000in}}%
\pgfpathlineto{\pgfqpoint{0.000000in}{0.048611in}}%
\pgfusepath{stroke,fill}%
}%
\begin{pgfscope}%
\pgfsys@transformshift{3.063158in}{0.600000in}%
\pgfsys@useobject{currentmarker}{}%
\end{pgfscope}%
\end{pgfscope}%
\begin{pgfscope}%
\pgfsetbuttcap%
\pgfsetroundjoin%
\definecolor{currentfill}{rgb}{0.000000,0.000000,0.000000}%
\pgfsetfillcolor{currentfill}%
\pgfsetlinewidth{0.803000pt}%
\definecolor{currentstroke}{rgb}{0.000000,0.000000,0.000000}%
\pgfsetstrokecolor{currentstroke}%
\pgfsetdash{}{0pt}%
\pgfsys@defobject{currentmarker}{\pgfqpoint{0.000000in}{-0.048611in}}{\pgfqpoint{0.000000in}{0.000000in}}{%
\pgfpathmoveto{\pgfqpoint{0.000000in}{0.000000in}}%
\pgfpathlineto{\pgfqpoint{0.000000in}{-0.048611in}}%
\pgfusepath{stroke,fill}%
}%
\begin{pgfscope}%
\pgfsys@transformshift{3.063158in}{4.500000in}%
\pgfsys@useobject{currentmarker}{}%
\end{pgfscope}%
\end{pgfscope}%
\begin{pgfscope}%
\definecolor{textcolor}{rgb}{0.000000,0.000000,0.000000}%
\pgfsetstrokecolor{textcolor}%
\pgfsetfillcolor{textcolor}%
\pgftext[x=3.063158in,y=0.551389in,,top]{\color{textcolor}\sffamily\fontsize{10.000000}{12.000000}\selectfont 1}%
\end{pgfscope}%
\begin{pgfscope}%
\pgfsetbuttcap%
\pgfsetroundjoin%
\definecolor{currentfill}{rgb}{0.000000,0.000000,0.000000}%
\pgfsetfillcolor{currentfill}%
\pgfsetlinewidth{0.803000pt}%
\definecolor{currentstroke}{rgb}{0.000000,0.000000,0.000000}%
\pgfsetstrokecolor{currentstroke}%
\pgfsetdash{}{0pt}%
\pgfsys@defobject{currentmarker}{\pgfqpoint{0.000000in}{0.000000in}}{\pgfqpoint{0.000000in}{0.048611in}}{%
\pgfpathmoveto{\pgfqpoint{0.000000in}{0.000000in}}%
\pgfpathlineto{\pgfqpoint{0.000000in}{0.048611in}}%
\pgfusepath{stroke,fill}%
}%
\begin{pgfscope}%
\pgfsys@transformshift{3.576316in}{0.600000in}%
\pgfsys@useobject{currentmarker}{}%
\end{pgfscope}%
\end{pgfscope}%
\begin{pgfscope}%
\pgfsetbuttcap%
\pgfsetroundjoin%
\definecolor{currentfill}{rgb}{0.000000,0.000000,0.000000}%
\pgfsetfillcolor{currentfill}%
\pgfsetlinewidth{0.803000pt}%
\definecolor{currentstroke}{rgb}{0.000000,0.000000,0.000000}%
\pgfsetstrokecolor{currentstroke}%
\pgfsetdash{}{0pt}%
\pgfsys@defobject{currentmarker}{\pgfqpoint{0.000000in}{-0.048611in}}{\pgfqpoint{0.000000in}{0.000000in}}{%
\pgfpathmoveto{\pgfqpoint{0.000000in}{0.000000in}}%
\pgfpathlineto{\pgfqpoint{0.000000in}{-0.048611in}}%
\pgfusepath{stroke,fill}%
}%
\begin{pgfscope}%
\pgfsys@transformshift{3.576316in}{4.500000in}%
\pgfsys@useobject{currentmarker}{}%
\end{pgfscope}%
\end{pgfscope}%
\begin{pgfscope}%
\definecolor{textcolor}{rgb}{0.000000,0.000000,0.000000}%
\pgfsetstrokecolor{textcolor}%
\pgfsetfillcolor{textcolor}%
\pgftext[x=3.576316in,y=0.551389in,,top]{\color{textcolor}\sffamily\fontsize{10.000000}{12.000000}\selectfont 2}%
\end{pgfscope}%
\begin{pgfscope}%
\pgfsetbuttcap%
\pgfsetroundjoin%
\definecolor{currentfill}{rgb}{0.000000,0.000000,0.000000}%
\pgfsetfillcolor{currentfill}%
\pgfsetlinewidth{0.803000pt}%
\definecolor{currentstroke}{rgb}{0.000000,0.000000,0.000000}%
\pgfsetstrokecolor{currentstroke}%
\pgfsetdash{}{0pt}%
\pgfsys@defobject{currentmarker}{\pgfqpoint{0.000000in}{0.000000in}}{\pgfqpoint{0.000000in}{0.048611in}}{%
\pgfpathmoveto{\pgfqpoint{0.000000in}{0.000000in}}%
\pgfpathlineto{\pgfqpoint{0.000000in}{0.048611in}}%
\pgfusepath{stroke,fill}%
}%
\begin{pgfscope}%
\pgfsys@transformshift{4.089474in}{0.600000in}%
\pgfsys@useobject{currentmarker}{}%
\end{pgfscope}%
\end{pgfscope}%
\begin{pgfscope}%
\pgfsetbuttcap%
\pgfsetroundjoin%
\definecolor{currentfill}{rgb}{0.000000,0.000000,0.000000}%
\pgfsetfillcolor{currentfill}%
\pgfsetlinewidth{0.803000pt}%
\definecolor{currentstroke}{rgb}{0.000000,0.000000,0.000000}%
\pgfsetstrokecolor{currentstroke}%
\pgfsetdash{}{0pt}%
\pgfsys@defobject{currentmarker}{\pgfqpoint{0.000000in}{-0.048611in}}{\pgfqpoint{0.000000in}{0.000000in}}{%
\pgfpathmoveto{\pgfqpoint{0.000000in}{0.000000in}}%
\pgfpathlineto{\pgfqpoint{0.000000in}{-0.048611in}}%
\pgfusepath{stroke,fill}%
}%
\begin{pgfscope}%
\pgfsys@transformshift{4.089474in}{4.500000in}%
\pgfsys@useobject{currentmarker}{}%
\end{pgfscope}%
\end{pgfscope}%
\begin{pgfscope}%
\definecolor{textcolor}{rgb}{0.000000,0.000000,0.000000}%
\pgfsetstrokecolor{textcolor}%
\pgfsetfillcolor{textcolor}%
\pgftext[x=4.089474in,y=0.551389in,,top]{\color{textcolor}\sffamily\fontsize{10.000000}{12.000000}\selectfont 3}%
\end{pgfscope}%
\begin{pgfscope}%
\pgfsetbuttcap%
\pgfsetroundjoin%
\definecolor{currentfill}{rgb}{0.000000,0.000000,0.000000}%
\pgfsetfillcolor{currentfill}%
\pgfsetlinewidth{0.803000pt}%
\definecolor{currentstroke}{rgb}{0.000000,0.000000,0.000000}%
\pgfsetstrokecolor{currentstroke}%
\pgfsetdash{}{0pt}%
\pgfsys@defobject{currentmarker}{\pgfqpoint{0.000000in}{0.000000in}}{\pgfqpoint{0.048611in}{0.000000in}}{%
\pgfpathmoveto{\pgfqpoint{0.000000in}{0.000000in}}%
\pgfpathlineto{\pgfqpoint{0.048611in}{0.000000in}}%
\pgfusepath{stroke,fill}%
}%
\begin{pgfscope}%
\pgfsys@transformshift{0.600000in}{1.010526in}%
\pgfsys@useobject{currentmarker}{}%
\end{pgfscope}%
\end{pgfscope}%
\begin{pgfscope}%
\pgfsetbuttcap%
\pgfsetroundjoin%
\definecolor{currentfill}{rgb}{0.000000,0.000000,0.000000}%
\pgfsetfillcolor{currentfill}%
\pgfsetlinewidth{0.803000pt}%
\definecolor{currentstroke}{rgb}{0.000000,0.000000,0.000000}%
\pgfsetstrokecolor{currentstroke}%
\pgfsetdash{}{0pt}%
\pgfsys@defobject{currentmarker}{\pgfqpoint{-0.048611in}{0.000000in}}{\pgfqpoint{0.000000in}{0.000000in}}{%
\pgfpathmoveto{\pgfqpoint{0.000000in}{0.000000in}}%
\pgfpathlineto{\pgfqpoint{-0.048611in}{0.000000in}}%
\pgfusepath{stroke,fill}%
}%
\begin{pgfscope}%
\pgfsys@transformshift{4.500000in}{1.010526in}%
\pgfsys@useobject{currentmarker}{}%
\end{pgfscope}%
\end{pgfscope}%
\begin{pgfscope}%
\definecolor{textcolor}{rgb}{0.000000,0.000000,0.000000}%
\pgfsetstrokecolor{textcolor}%
\pgfsetfillcolor{textcolor}%
\pgftext[x=0.346650in,y=0.957765in,left,base]{\color{textcolor}\sffamily\fontsize{10.000000}{12.000000}\selectfont −3}%
\end{pgfscope}%
\begin{pgfscope}%
\pgfsetbuttcap%
\pgfsetroundjoin%
\definecolor{currentfill}{rgb}{0.000000,0.000000,0.000000}%
\pgfsetfillcolor{currentfill}%
\pgfsetlinewidth{0.803000pt}%
\definecolor{currentstroke}{rgb}{0.000000,0.000000,0.000000}%
\pgfsetstrokecolor{currentstroke}%
\pgfsetdash{}{0pt}%
\pgfsys@defobject{currentmarker}{\pgfqpoint{0.000000in}{0.000000in}}{\pgfqpoint{0.048611in}{0.000000in}}{%
\pgfpathmoveto{\pgfqpoint{0.000000in}{0.000000in}}%
\pgfpathlineto{\pgfqpoint{0.048611in}{0.000000in}}%
\pgfusepath{stroke,fill}%
}%
\begin{pgfscope}%
\pgfsys@transformshift{0.600000in}{1.523684in}%
\pgfsys@useobject{currentmarker}{}%
\end{pgfscope}%
\end{pgfscope}%
\begin{pgfscope}%
\pgfsetbuttcap%
\pgfsetroundjoin%
\definecolor{currentfill}{rgb}{0.000000,0.000000,0.000000}%
\pgfsetfillcolor{currentfill}%
\pgfsetlinewidth{0.803000pt}%
\definecolor{currentstroke}{rgb}{0.000000,0.000000,0.000000}%
\pgfsetstrokecolor{currentstroke}%
\pgfsetdash{}{0pt}%
\pgfsys@defobject{currentmarker}{\pgfqpoint{-0.048611in}{0.000000in}}{\pgfqpoint{0.000000in}{0.000000in}}{%
\pgfpathmoveto{\pgfqpoint{0.000000in}{0.000000in}}%
\pgfpathlineto{\pgfqpoint{-0.048611in}{0.000000in}}%
\pgfusepath{stroke,fill}%
}%
\begin{pgfscope}%
\pgfsys@transformshift{4.500000in}{1.523684in}%
\pgfsys@useobject{currentmarker}{}%
\end{pgfscope}%
\end{pgfscope}%
\begin{pgfscope}%
\definecolor{textcolor}{rgb}{0.000000,0.000000,0.000000}%
\pgfsetstrokecolor{textcolor}%
\pgfsetfillcolor{textcolor}%
\pgftext[x=0.346650in,y=1.470923in,left,base]{\color{textcolor}\sffamily\fontsize{10.000000}{12.000000}\selectfont −2}%
\end{pgfscope}%
\begin{pgfscope}%
\pgfsetbuttcap%
\pgfsetroundjoin%
\definecolor{currentfill}{rgb}{0.000000,0.000000,0.000000}%
\pgfsetfillcolor{currentfill}%
\pgfsetlinewidth{0.803000pt}%
\definecolor{currentstroke}{rgb}{0.000000,0.000000,0.000000}%
\pgfsetstrokecolor{currentstroke}%
\pgfsetdash{}{0pt}%
\pgfsys@defobject{currentmarker}{\pgfqpoint{0.000000in}{0.000000in}}{\pgfqpoint{0.048611in}{0.000000in}}{%
\pgfpathmoveto{\pgfqpoint{0.000000in}{0.000000in}}%
\pgfpathlineto{\pgfqpoint{0.048611in}{0.000000in}}%
\pgfusepath{stroke,fill}%
}%
\begin{pgfscope}%
\pgfsys@transformshift{0.600000in}{2.036842in}%
\pgfsys@useobject{currentmarker}{}%
\end{pgfscope}%
\end{pgfscope}%
\begin{pgfscope}%
\pgfsetbuttcap%
\pgfsetroundjoin%
\definecolor{currentfill}{rgb}{0.000000,0.000000,0.000000}%
\pgfsetfillcolor{currentfill}%
\pgfsetlinewidth{0.803000pt}%
\definecolor{currentstroke}{rgb}{0.000000,0.000000,0.000000}%
\pgfsetstrokecolor{currentstroke}%
\pgfsetdash{}{0pt}%
\pgfsys@defobject{currentmarker}{\pgfqpoint{-0.048611in}{0.000000in}}{\pgfqpoint{0.000000in}{0.000000in}}{%
\pgfpathmoveto{\pgfqpoint{0.000000in}{0.000000in}}%
\pgfpathlineto{\pgfqpoint{-0.048611in}{0.000000in}}%
\pgfusepath{stroke,fill}%
}%
\begin{pgfscope}%
\pgfsys@transformshift{4.500000in}{2.036842in}%
\pgfsys@useobject{currentmarker}{}%
\end{pgfscope}%
\end{pgfscope}%
\begin{pgfscope}%
\definecolor{textcolor}{rgb}{0.000000,0.000000,0.000000}%
\pgfsetstrokecolor{textcolor}%
\pgfsetfillcolor{textcolor}%
\pgftext[x=0.346650in,y=1.984081in,left,base]{\color{textcolor}\sffamily\fontsize{10.000000}{12.000000}\selectfont −1}%
\end{pgfscope}%
\begin{pgfscope}%
\pgfsetbuttcap%
\pgfsetroundjoin%
\definecolor{currentfill}{rgb}{0.000000,0.000000,0.000000}%
\pgfsetfillcolor{currentfill}%
\pgfsetlinewidth{0.803000pt}%
\definecolor{currentstroke}{rgb}{0.000000,0.000000,0.000000}%
\pgfsetstrokecolor{currentstroke}%
\pgfsetdash{}{0pt}%
\pgfsys@defobject{currentmarker}{\pgfqpoint{0.000000in}{0.000000in}}{\pgfqpoint{0.048611in}{0.000000in}}{%
\pgfpathmoveto{\pgfqpoint{0.000000in}{0.000000in}}%
\pgfpathlineto{\pgfqpoint{0.048611in}{0.000000in}}%
\pgfusepath{stroke,fill}%
}%
\begin{pgfscope}%
\pgfsys@transformshift{0.600000in}{2.550000in}%
\pgfsys@useobject{currentmarker}{}%
\end{pgfscope}%
\end{pgfscope}%
\begin{pgfscope}%
\pgfsetbuttcap%
\pgfsetroundjoin%
\definecolor{currentfill}{rgb}{0.000000,0.000000,0.000000}%
\pgfsetfillcolor{currentfill}%
\pgfsetlinewidth{0.803000pt}%
\definecolor{currentstroke}{rgb}{0.000000,0.000000,0.000000}%
\pgfsetstrokecolor{currentstroke}%
\pgfsetdash{}{0pt}%
\pgfsys@defobject{currentmarker}{\pgfqpoint{-0.048611in}{0.000000in}}{\pgfqpoint{0.000000in}{0.000000in}}{%
\pgfpathmoveto{\pgfqpoint{0.000000in}{0.000000in}}%
\pgfpathlineto{\pgfqpoint{-0.048611in}{0.000000in}}%
\pgfusepath{stroke,fill}%
}%
\begin{pgfscope}%
\pgfsys@transformshift{4.500000in}{2.550000in}%
\pgfsys@useobject{currentmarker}{}%
\end{pgfscope}%
\end{pgfscope}%
\begin{pgfscope}%
\definecolor{textcolor}{rgb}{0.000000,0.000000,0.000000}%
\pgfsetstrokecolor{textcolor}%
\pgfsetfillcolor{textcolor}%
\pgftext[x=0.463024in,y=2.497238in,left,base]{\color{textcolor}\sffamily\fontsize{10.000000}{12.000000}\selectfont 0}%
\end{pgfscope}%
\begin{pgfscope}%
\pgfsetbuttcap%
\pgfsetroundjoin%
\definecolor{currentfill}{rgb}{0.000000,0.000000,0.000000}%
\pgfsetfillcolor{currentfill}%
\pgfsetlinewidth{0.803000pt}%
\definecolor{currentstroke}{rgb}{0.000000,0.000000,0.000000}%
\pgfsetstrokecolor{currentstroke}%
\pgfsetdash{}{0pt}%
\pgfsys@defobject{currentmarker}{\pgfqpoint{0.000000in}{0.000000in}}{\pgfqpoint{0.048611in}{0.000000in}}{%
\pgfpathmoveto{\pgfqpoint{0.000000in}{0.000000in}}%
\pgfpathlineto{\pgfqpoint{0.048611in}{0.000000in}}%
\pgfusepath{stroke,fill}%
}%
\begin{pgfscope}%
\pgfsys@transformshift{0.600000in}{3.063158in}%
\pgfsys@useobject{currentmarker}{}%
\end{pgfscope}%
\end{pgfscope}%
\begin{pgfscope}%
\pgfsetbuttcap%
\pgfsetroundjoin%
\definecolor{currentfill}{rgb}{0.000000,0.000000,0.000000}%
\pgfsetfillcolor{currentfill}%
\pgfsetlinewidth{0.803000pt}%
\definecolor{currentstroke}{rgb}{0.000000,0.000000,0.000000}%
\pgfsetstrokecolor{currentstroke}%
\pgfsetdash{}{0pt}%
\pgfsys@defobject{currentmarker}{\pgfqpoint{-0.048611in}{0.000000in}}{\pgfqpoint{0.000000in}{0.000000in}}{%
\pgfpathmoveto{\pgfqpoint{0.000000in}{0.000000in}}%
\pgfpathlineto{\pgfqpoint{-0.048611in}{0.000000in}}%
\pgfusepath{stroke,fill}%
}%
\begin{pgfscope}%
\pgfsys@transformshift{4.500000in}{3.063158in}%
\pgfsys@useobject{currentmarker}{}%
\end{pgfscope}%
\end{pgfscope}%
\begin{pgfscope}%
\definecolor{textcolor}{rgb}{0.000000,0.000000,0.000000}%
\pgfsetstrokecolor{textcolor}%
\pgfsetfillcolor{textcolor}%
\pgftext[x=0.463024in,y=3.010396in,left,base]{\color{textcolor}\sffamily\fontsize{10.000000}{12.000000}\selectfont 1}%
\end{pgfscope}%
\begin{pgfscope}%
\pgfsetbuttcap%
\pgfsetroundjoin%
\definecolor{currentfill}{rgb}{0.000000,0.000000,0.000000}%
\pgfsetfillcolor{currentfill}%
\pgfsetlinewidth{0.803000pt}%
\definecolor{currentstroke}{rgb}{0.000000,0.000000,0.000000}%
\pgfsetstrokecolor{currentstroke}%
\pgfsetdash{}{0pt}%
\pgfsys@defobject{currentmarker}{\pgfqpoint{0.000000in}{0.000000in}}{\pgfqpoint{0.048611in}{0.000000in}}{%
\pgfpathmoveto{\pgfqpoint{0.000000in}{0.000000in}}%
\pgfpathlineto{\pgfqpoint{0.048611in}{0.000000in}}%
\pgfusepath{stroke,fill}%
}%
\begin{pgfscope}%
\pgfsys@transformshift{0.600000in}{3.576316in}%
\pgfsys@useobject{currentmarker}{}%
\end{pgfscope}%
\end{pgfscope}%
\begin{pgfscope}%
\pgfsetbuttcap%
\pgfsetroundjoin%
\definecolor{currentfill}{rgb}{0.000000,0.000000,0.000000}%
\pgfsetfillcolor{currentfill}%
\pgfsetlinewidth{0.803000pt}%
\definecolor{currentstroke}{rgb}{0.000000,0.000000,0.000000}%
\pgfsetstrokecolor{currentstroke}%
\pgfsetdash{}{0pt}%
\pgfsys@defobject{currentmarker}{\pgfqpoint{-0.048611in}{0.000000in}}{\pgfqpoint{0.000000in}{0.000000in}}{%
\pgfpathmoveto{\pgfqpoint{0.000000in}{0.000000in}}%
\pgfpathlineto{\pgfqpoint{-0.048611in}{0.000000in}}%
\pgfusepath{stroke,fill}%
}%
\begin{pgfscope}%
\pgfsys@transformshift{4.500000in}{3.576316in}%
\pgfsys@useobject{currentmarker}{}%
\end{pgfscope}%
\end{pgfscope}%
\begin{pgfscope}%
\definecolor{textcolor}{rgb}{0.000000,0.000000,0.000000}%
\pgfsetstrokecolor{textcolor}%
\pgfsetfillcolor{textcolor}%
\pgftext[x=0.463024in,y=3.523554in,left,base]{\color{textcolor}\sffamily\fontsize{10.000000}{12.000000}\selectfont 2}%
\end{pgfscope}%
\begin{pgfscope}%
\pgfsetbuttcap%
\pgfsetroundjoin%
\definecolor{currentfill}{rgb}{0.000000,0.000000,0.000000}%
\pgfsetfillcolor{currentfill}%
\pgfsetlinewidth{0.803000pt}%
\definecolor{currentstroke}{rgb}{0.000000,0.000000,0.000000}%
\pgfsetstrokecolor{currentstroke}%
\pgfsetdash{}{0pt}%
\pgfsys@defobject{currentmarker}{\pgfqpoint{0.000000in}{0.000000in}}{\pgfqpoint{0.048611in}{0.000000in}}{%
\pgfpathmoveto{\pgfqpoint{0.000000in}{0.000000in}}%
\pgfpathlineto{\pgfqpoint{0.048611in}{0.000000in}}%
\pgfusepath{stroke,fill}%
}%
\begin{pgfscope}%
\pgfsys@transformshift{0.600000in}{4.089474in}%
\pgfsys@useobject{currentmarker}{}%
\end{pgfscope}%
\end{pgfscope}%
\begin{pgfscope}%
\pgfsetbuttcap%
\pgfsetroundjoin%
\definecolor{currentfill}{rgb}{0.000000,0.000000,0.000000}%
\pgfsetfillcolor{currentfill}%
\pgfsetlinewidth{0.803000pt}%
\definecolor{currentstroke}{rgb}{0.000000,0.000000,0.000000}%
\pgfsetstrokecolor{currentstroke}%
\pgfsetdash{}{0pt}%
\pgfsys@defobject{currentmarker}{\pgfqpoint{-0.048611in}{0.000000in}}{\pgfqpoint{0.000000in}{0.000000in}}{%
\pgfpathmoveto{\pgfqpoint{0.000000in}{0.000000in}}%
\pgfpathlineto{\pgfqpoint{-0.048611in}{0.000000in}}%
\pgfusepath{stroke,fill}%
}%
\begin{pgfscope}%
\pgfsys@transformshift{4.500000in}{4.089474in}%
\pgfsys@useobject{currentmarker}{}%
\end{pgfscope}%
\end{pgfscope}%
\begin{pgfscope}%
\definecolor{textcolor}{rgb}{0.000000,0.000000,0.000000}%
\pgfsetstrokecolor{textcolor}%
\pgfsetfillcolor{textcolor}%
\pgftext[x=0.463024in,y=4.036712in,left,base]{\color{textcolor}\sffamily\fontsize{10.000000}{12.000000}\selectfont 3}%
\end{pgfscope}%
\begin{pgfscope}%
\pgfsetrectcap%
\pgfsetmiterjoin%
\pgfsetlinewidth{0.803000pt}%
\definecolor{currentstroke}{rgb}{0.000000,0.000000,0.000000}%
\pgfsetstrokecolor{currentstroke}%
\pgfsetdash{}{0pt}%
\pgfpathmoveto{\pgfqpoint{0.600000in}{0.600000in}}%
\pgfpathlineto{\pgfqpoint{0.600000in}{4.500000in}}%
\pgfusepath{stroke}%
\end{pgfscope}%
\begin{pgfscope}%
\pgfsetrectcap%
\pgfsetmiterjoin%
\pgfsetlinewidth{0.803000pt}%
\definecolor{currentstroke}{rgb}{0.000000,0.000000,0.000000}%
\pgfsetstrokecolor{currentstroke}%
\pgfsetdash{}{0pt}%
\pgfpathmoveto{\pgfqpoint{4.500000in}{0.600000in}}%
\pgfpathlineto{\pgfqpoint{4.500000in}{4.500000in}}%
\pgfusepath{stroke}%
\end{pgfscope}%
\begin{pgfscope}%
\pgfsetrectcap%
\pgfsetmiterjoin%
\pgfsetlinewidth{0.803000pt}%
\definecolor{currentstroke}{rgb}{0.000000,0.000000,0.000000}%
\pgfsetstrokecolor{currentstroke}%
\pgfsetdash{}{0pt}%
\pgfpathmoveto{\pgfqpoint{0.600000in}{0.600000in}}%
\pgfpathlineto{\pgfqpoint{4.500000in}{0.600000in}}%
\pgfusepath{stroke}%
\end{pgfscope}%
\begin{pgfscope}%
\pgfsetrectcap%
\pgfsetmiterjoin%
\pgfsetlinewidth{0.803000pt}%
\definecolor{currentstroke}{rgb}{0.000000,0.000000,0.000000}%
\pgfsetstrokecolor{currentstroke}%
\pgfsetdash{}{0pt}%
\pgfpathmoveto{\pgfqpoint{0.600000in}{4.500000in}}%
\pgfpathlineto{\pgfqpoint{4.500000in}{4.500000in}}%
\pgfusepath{stroke}%
\end{pgfscope}%
\begin{pgfscope}%
\pgfsetbuttcap%
\pgfsetmiterjoin%
\definecolor{currentfill}{rgb}{1.000000,1.000000,1.000000}%
\pgfsetfillcolor{currentfill}%
\pgfsetlinewidth{0.000000pt}%
\definecolor{currentstroke}{rgb}{0.000000,0.000000,0.000000}%
\pgfsetstrokecolor{currentstroke}%
\pgfsetstrokeopacity{0.000000}%
\pgfsetdash{}{0pt}%
\pgfpathmoveto{\pgfqpoint{0.600000in}{4.530000in}}%
\pgfpathlineto{\pgfqpoint{4.500000in}{4.530000in}}%
\pgfpathlineto{\pgfqpoint{4.500000in}{5.730000in}}%
\pgfpathlineto{\pgfqpoint{0.600000in}{5.730000in}}%
\pgfpathclose%
\pgfusepath{fill}%
\end{pgfscope}%
\begin{pgfscope}%
\pgfpathrectangle{\pgfqpoint{0.600000in}{4.530000in}}{\pgfqpoint{3.900000in}{1.200000in}}%
\pgfusepath{clip}%
\pgfsetbuttcap%
\pgfsetmiterjoin%
\definecolor{currentfill}{rgb}{0.121569,0.466667,0.705882}%
\pgfsetfillcolor{currentfill}%
\pgfsetlinewidth{0.000000pt}%
\definecolor{currentstroke}{rgb}{0.000000,0.000000,0.000000}%
\pgfsetstrokecolor{currentstroke}%
\pgfsetstrokeopacity{0.000000}%
\pgfsetdash{}{0pt}%
\pgfpathmoveto{\pgfqpoint{0.600000in}{4.530000in}}%
\pgfpathlineto{\pgfqpoint{0.728289in}{4.530000in}}%
\pgfpathlineto{\pgfqpoint{0.728289in}{4.539938in}}%
\pgfpathlineto{\pgfqpoint{0.600000in}{4.539938in}}%
\pgfpathclose%
\pgfusepath{fill}%
\end{pgfscope}%
\begin{pgfscope}%
\pgfpathrectangle{\pgfqpoint{0.600000in}{4.530000in}}{\pgfqpoint{3.900000in}{1.200000in}}%
\pgfusepath{clip}%
\pgfsetbuttcap%
\pgfsetmiterjoin%
\definecolor{currentfill}{rgb}{0.121569,0.466667,0.705882}%
\pgfsetfillcolor{currentfill}%
\pgfsetlinewidth{0.000000pt}%
\definecolor{currentstroke}{rgb}{0.000000,0.000000,0.000000}%
\pgfsetstrokecolor{currentstroke}%
\pgfsetstrokeopacity{0.000000}%
\pgfsetdash{}{0pt}%
\pgfpathmoveto{\pgfqpoint{0.728289in}{4.530000in}}%
\pgfpathlineto{\pgfqpoint{0.856579in}{4.530000in}}%
\pgfpathlineto{\pgfqpoint{0.856579in}{4.530000in}}%
\pgfpathlineto{\pgfqpoint{0.728289in}{4.530000in}}%
\pgfpathclose%
\pgfusepath{fill}%
\end{pgfscope}%
\begin{pgfscope}%
\pgfpathrectangle{\pgfqpoint{0.600000in}{4.530000in}}{\pgfqpoint{3.900000in}{1.200000in}}%
\pgfusepath{clip}%
\pgfsetbuttcap%
\pgfsetmiterjoin%
\definecolor{currentfill}{rgb}{0.121569,0.466667,0.705882}%
\pgfsetfillcolor{currentfill}%
\pgfsetlinewidth{0.000000pt}%
\definecolor{currentstroke}{rgb}{0.000000,0.000000,0.000000}%
\pgfsetstrokecolor{currentstroke}%
\pgfsetstrokeopacity{0.000000}%
\pgfsetdash{}{0pt}%
\pgfpathmoveto{\pgfqpoint{0.856579in}{4.530000in}}%
\pgfpathlineto{\pgfqpoint{0.984868in}{4.530000in}}%
\pgfpathlineto{\pgfqpoint{0.984868in}{4.539938in}}%
\pgfpathlineto{\pgfqpoint{0.856579in}{4.539938in}}%
\pgfpathclose%
\pgfusepath{fill}%
\end{pgfscope}%
\begin{pgfscope}%
\pgfpathrectangle{\pgfqpoint{0.600000in}{4.530000in}}{\pgfqpoint{3.900000in}{1.200000in}}%
\pgfusepath{clip}%
\pgfsetbuttcap%
\pgfsetmiterjoin%
\definecolor{currentfill}{rgb}{0.121569,0.466667,0.705882}%
\pgfsetfillcolor{currentfill}%
\pgfsetlinewidth{0.000000pt}%
\definecolor{currentstroke}{rgb}{0.000000,0.000000,0.000000}%
\pgfsetstrokecolor{currentstroke}%
\pgfsetstrokeopacity{0.000000}%
\pgfsetdash{}{0pt}%
\pgfpathmoveto{\pgfqpoint{0.984868in}{4.530000in}}%
\pgfpathlineto{\pgfqpoint{1.113158in}{4.530000in}}%
\pgfpathlineto{\pgfqpoint{1.113158in}{4.549876in}}%
\pgfpathlineto{\pgfqpoint{0.984868in}{4.549876in}}%
\pgfpathclose%
\pgfusepath{fill}%
\end{pgfscope}%
\begin{pgfscope}%
\pgfpathrectangle{\pgfqpoint{0.600000in}{4.530000in}}{\pgfqpoint{3.900000in}{1.200000in}}%
\pgfusepath{clip}%
\pgfsetbuttcap%
\pgfsetmiterjoin%
\definecolor{currentfill}{rgb}{0.121569,0.466667,0.705882}%
\pgfsetfillcolor{currentfill}%
\pgfsetlinewidth{0.000000pt}%
\definecolor{currentstroke}{rgb}{0.000000,0.000000,0.000000}%
\pgfsetstrokecolor{currentstroke}%
\pgfsetstrokeopacity{0.000000}%
\pgfsetdash{}{0pt}%
\pgfpathmoveto{\pgfqpoint{1.113158in}{4.530000in}}%
\pgfpathlineto{\pgfqpoint{1.241447in}{4.530000in}}%
\pgfpathlineto{\pgfqpoint{1.241447in}{4.539938in}}%
\pgfpathlineto{\pgfqpoint{1.113158in}{4.539938in}}%
\pgfpathclose%
\pgfusepath{fill}%
\end{pgfscope}%
\begin{pgfscope}%
\pgfpathrectangle{\pgfqpoint{0.600000in}{4.530000in}}{\pgfqpoint{3.900000in}{1.200000in}}%
\pgfusepath{clip}%
\pgfsetbuttcap%
\pgfsetmiterjoin%
\definecolor{currentfill}{rgb}{0.121569,0.466667,0.705882}%
\pgfsetfillcolor{currentfill}%
\pgfsetlinewidth{0.000000pt}%
\definecolor{currentstroke}{rgb}{0.000000,0.000000,0.000000}%
\pgfsetstrokecolor{currentstroke}%
\pgfsetstrokeopacity{0.000000}%
\pgfsetdash{}{0pt}%
\pgfpathmoveto{\pgfqpoint{1.241447in}{4.530000in}}%
\pgfpathlineto{\pgfqpoint{1.369737in}{4.530000in}}%
\pgfpathlineto{\pgfqpoint{1.369737in}{4.579689in}}%
\pgfpathlineto{\pgfqpoint{1.241447in}{4.579689in}}%
\pgfpathclose%
\pgfusepath{fill}%
\end{pgfscope}%
\begin{pgfscope}%
\pgfpathrectangle{\pgfqpoint{0.600000in}{4.530000in}}{\pgfqpoint{3.900000in}{1.200000in}}%
\pgfusepath{clip}%
\pgfsetbuttcap%
\pgfsetmiterjoin%
\definecolor{currentfill}{rgb}{0.121569,0.466667,0.705882}%
\pgfsetfillcolor{currentfill}%
\pgfsetlinewidth{0.000000pt}%
\definecolor{currentstroke}{rgb}{0.000000,0.000000,0.000000}%
\pgfsetstrokecolor{currentstroke}%
\pgfsetstrokeopacity{0.000000}%
\pgfsetdash{}{0pt}%
\pgfpathmoveto{\pgfqpoint{1.369737in}{4.530000in}}%
\pgfpathlineto{\pgfqpoint{1.498026in}{4.530000in}}%
\pgfpathlineto{\pgfqpoint{1.498026in}{4.629379in}}%
\pgfpathlineto{\pgfqpoint{1.369737in}{4.629379in}}%
\pgfpathclose%
\pgfusepath{fill}%
\end{pgfscope}%
\begin{pgfscope}%
\pgfpathrectangle{\pgfqpoint{0.600000in}{4.530000in}}{\pgfqpoint{3.900000in}{1.200000in}}%
\pgfusepath{clip}%
\pgfsetbuttcap%
\pgfsetmiterjoin%
\definecolor{currentfill}{rgb}{0.121569,0.466667,0.705882}%
\pgfsetfillcolor{currentfill}%
\pgfsetlinewidth{0.000000pt}%
\definecolor{currentstroke}{rgb}{0.000000,0.000000,0.000000}%
\pgfsetstrokecolor{currentstroke}%
\pgfsetstrokeopacity{0.000000}%
\pgfsetdash{}{0pt}%
\pgfpathmoveto{\pgfqpoint{1.498026in}{4.530000in}}%
\pgfpathlineto{\pgfqpoint{1.626316in}{4.530000in}}%
\pgfpathlineto{\pgfqpoint{1.626316in}{4.649255in}}%
\pgfpathlineto{\pgfqpoint{1.498026in}{4.649255in}}%
\pgfpathclose%
\pgfusepath{fill}%
\end{pgfscope}%
\begin{pgfscope}%
\pgfpathrectangle{\pgfqpoint{0.600000in}{4.530000in}}{\pgfqpoint{3.900000in}{1.200000in}}%
\pgfusepath{clip}%
\pgfsetbuttcap%
\pgfsetmiterjoin%
\definecolor{currentfill}{rgb}{0.121569,0.466667,0.705882}%
\pgfsetfillcolor{currentfill}%
\pgfsetlinewidth{0.000000pt}%
\definecolor{currentstroke}{rgb}{0.000000,0.000000,0.000000}%
\pgfsetstrokecolor{currentstroke}%
\pgfsetstrokeopacity{0.000000}%
\pgfsetdash{}{0pt}%
\pgfpathmoveto{\pgfqpoint{1.626316in}{4.530000in}}%
\pgfpathlineto{\pgfqpoint{1.754605in}{4.530000in}}%
\pgfpathlineto{\pgfqpoint{1.754605in}{4.748634in}}%
\pgfpathlineto{\pgfqpoint{1.626316in}{4.748634in}}%
\pgfpathclose%
\pgfusepath{fill}%
\end{pgfscope}%
\begin{pgfscope}%
\pgfpathrectangle{\pgfqpoint{0.600000in}{4.530000in}}{\pgfqpoint{3.900000in}{1.200000in}}%
\pgfusepath{clip}%
\pgfsetbuttcap%
\pgfsetmiterjoin%
\definecolor{currentfill}{rgb}{0.121569,0.466667,0.705882}%
\pgfsetfillcolor{currentfill}%
\pgfsetlinewidth{0.000000pt}%
\definecolor{currentstroke}{rgb}{0.000000,0.000000,0.000000}%
\pgfsetstrokecolor{currentstroke}%
\pgfsetstrokeopacity{0.000000}%
\pgfsetdash{}{0pt}%
\pgfpathmoveto{\pgfqpoint{1.754605in}{4.530000in}}%
\pgfpathlineto{\pgfqpoint{1.882895in}{4.530000in}}%
\pgfpathlineto{\pgfqpoint{1.882895in}{4.937453in}}%
\pgfpathlineto{\pgfqpoint{1.754605in}{4.937453in}}%
\pgfpathclose%
\pgfusepath{fill}%
\end{pgfscope}%
\begin{pgfscope}%
\pgfpathrectangle{\pgfqpoint{0.600000in}{4.530000in}}{\pgfqpoint{3.900000in}{1.200000in}}%
\pgfusepath{clip}%
\pgfsetbuttcap%
\pgfsetmiterjoin%
\definecolor{currentfill}{rgb}{0.121569,0.466667,0.705882}%
\pgfsetfillcolor{currentfill}%
\pgfsetlinewidth{0.000000pt}%
\definecolor{currentstroke}{rgb}{0.000000,0.000000,0.000000}%
\pgfsetstrokecolor{currentstroke}%
\pgfsetstrokeopacity{0.000000}%
\pgfsetdash{}{0pt}%
\pgfpathmoveto{\pgfqpoint{1.882895in}{4.530000in}}%
\pgfpathlineto{\pgfqpoint{2.011184in}{4.530000in}}%
\pgfpathlineto{\pgfqpoint{2.011184in}{5.007019in}}%
\pgfpathlineto{\pgfqpoint{1.882895in}{5.007019in}}%
\pgfpathclose%
\pgfusepath{fill}%
\end{pgfscope}%
\begin{pgfscope}%
\pgfpathrectangle{\pgfqpoint{0.600000in}{4.530000in}}{\pgfqpoint{3.900000in}{1.200000in}}%
\pgfusepath{clip}%
\pgfsetbuttcap%
\pgfsetmiterjoin%
\definecolor{currentfill}{rgb}{0.121569,0.466667,0.705882}%
\pgfsetfillcolor{currentfill}%
\pgfsetlinewidth{0.000000pt}%
\definecolor{currentstroke}{rgb}{0.000000,0.000000,0.000000}%
\pgfsetstrokecolor{currentstroke}%
\pgfsetstrokeopacity{0.000000}%
\pgfsetdash{}{0pt}%
\pgfpathmoveto{\pgfqpoint{2.011184in}{4.530000in}}%
\pgfpathlineto{\pgfqpoint{2.139474in}{4.530000in}}%
\pgfpathlineto{\pgfqpoint{2.139474in}{5.245528in}}%
\pgfpathlineto{\pgfqpoint{2.011184in}{5.245528in}}%
\pgfpathclose%
\pgfusepath{fill}%
\end{pgfscope}%
\begin{pgfscope}%
\pgfpathrectangle{\pgfqpoint{0.600000in}{4.530000in}}{\pgfqpoint{3.900000in}{1.200000in}}%
\pgfusepath{clip}%
\pgfsetbuttcap%
\pgfsetmiterjoin%
\definecolor{currentfill}{rgb}{0.121569,0.466667,0.705882}%
\pgfsetfillcolor{currentfill}%
\pgfsetlinewidth{0.000000pt}%
\definecolor{currentstroke}{rgb}{0.000000,0.000000,0.000000}%
\pgfsetstrokecolor{currentstroke}%
\pgfsetstrokeopacity{0.000000}%
\pgfsetdash{}{0pt}%
\pgfpathmoveto{\pgfqpoint{2.139474in}{4.530000in}}%
\pgfpathlineto{\pgfqpoint{2.267763in}{4.530000in}}%
\pgfpathlineto{\pgfqpoint{2.267763in}{5.265404in}}%
\pgfpathlineto{\pgfqpoint{2.139474in}{5.265404in}}%
\pgfpathclose%
\pgfusepath{fill}%
\end{pgfscope}%
\begin{pgfscope}%
\pgfpathrectangle{\pgfqpoint{0.600000in}{4.530000in}}{\pgfqpoint{3.900000in}{1.200000in}}%
\pgfusepath{clip}%
\pgfsetbuttcap%
\pgfsetmiterjoin%
\definecolor{currentfill}{rgb}{0.121569,0.466667,0.705882}%
\pgfsetfillcolor{currentfill}%
\pgfsetlinewidth{0.000000pt}%
\definecolor{currentstroke}{rgb}{0.000000,0.000000,0.000000}%
\pgfsetstrokecolor{currentstroke}%
\pgfsetstrokeopacity{0.000000}%
\pgfsetdash{}{0pt}%
\pgfpathmoveto{\pgfqpoint{2.267763in}{4.530000in}}%
\pgfpathlineto{\pgfqpoint{2.396053in}{4.530000in}}%
\pgfpathlineto{\pgfqpoint{2.396053in}{5.444286in}}%
\pgfpathlineto{\pgfqpoint{2.267763in}{5.444286in}}%
\pgfpathclose%
\pgfusepath{fill}%
\end{pgfscope}%
\begin{pgfscope}%
\pgfpathrectangle{\pgfqpoint{0.600000in}{4.530000in}}{\pgfqpoint{3.900000in}{1.200000in}}%
\pgfusepath{clip}%
\pgfsetbuttcap%
\pgfsetmiterjoin%
\definecolor{currentfill}{rgb}{0.121569,0.466667,0.705882}%
\pgfsetfillcolor{currentfill}%
\pgfsetlinewidth{0.000000pt}%
\definecolor{currentstroke}{rgb}{0.000000,0.000000,0.000000}%
\pgfsetstrokecolor{currentstroke}%
\pgfsetstrokeopacity{0.000000}%
\pgfsetdash{}{0pt}%
\pgfpathmoveto{\pgfqpoint{2.396053in}{4.530000in}}%
\pgfpathlineto{\pgfqpoint{2.524342in}{4.530000in}}%
\pgfpathlineto{\pgfqpoint{2.524342in}{5.533727in}}%
\pgfpathlineto{\pgfqpoint{2.396053in}{5.533727in}}%
\pgfpathclose%
\pgfusepath{fill}%
\end{pgfscope}%
\begin{pgfscope}%
\pgfpathrectangle{\pgfqpoint{0.600000in}{4.530000in}}{\pgfqpoint{3.900000in}{1.200000in}}%
\pgfusepath{clip}%
\pgfsetbuttcap%
\pgfsetmiterjoin%
\definecolor{currentfill}{rgb}{0.121569,0.466667,0.705882}%
\pgfsetfillcolor{currentfill}%
\pgfsetlinewidth{0.000000pt}%
\definecolor{currentstroke}{rgb}{0.000000,0.000000,0.000000}%
\pgfsetstrokecolor{currentstroke}%
\pgfsetstrokeopacity{0.000000}%
\pgfsetdash{}{0pt}%
\pgfpathmoveto{\pgfqpoint{2.524342in}{4.530000in}}%
\pgfpathlineto{\pgfqpoint{2.652632in}{4.530000in}}%
\pgfpathlineto{\pgfqpoint{2.652632in}{5.672857in}}%
\pgfpathlineto{\pgfqpoint{2.524342in}{5.672857in}}%
\pgfpathclose%
\pgfusepath{fill}%
\end{pgfscope}%
\begin{pgfscope}%
\pgfpathrectangle{\pgfqpoint{0.600000in}{4.530000in}}{\pgfqpoint{3.900000in}{1.200000in}}%
\pgfusepath{clip}%
\pgfsetbuttcap%
\pgfsetmiterjoin%
\definecolor{currentfill}{rgb}{0.121569,0.466667,0.705882}%
\pgfsetfillcolor{currentfill}%
\pgfsetlinewidth{0.000000pt}%
\definecolor{currentstroke}{rgb}{0.000000,0.000000,0.000000}%
\pgfsetstrokecolor{currentstroke}%
\pgfsetstrokeopacity{0.000000}%
\pgfsetdash{}{0pt}%
\pgfpathmoveto{\pgfqpoint{2.652632in}{4.530000in}}%
\pgfpathlineto{\pgfqpoint{2.780921in}{4.530000in}}%
\pgfpathlineto{\pgfqpoint{2.780921in}{5.503913in}}%
\pgfpathlineto{\pgfqpoint{2.652632in}{5.503913in}}%
\pgfpathclose%
\pgfusepath{fill}%
\end{pgfscope}%
\begin{pgfscope}%
\pgfpathrectangle{\pgfqpoint{0.600000in}{4.530000in}}{\pgfqpoint{3.900000in}{1.200000in}}%
\pgfusepath{clip}%
\pgfsetbuttcap%
\pgfsetmiterjoin%
\definecolor{currentfill}{rgb}{0.121569,0.466667,0.705882}%
\pgfsetfillcolor{currentfill}%
\pgfsetlinewidth{0.000000pt}%
\definecolor{currentstroke}{rgb}{0.000000,0.000000,0.000000}%
\pgfsetstrokecolor{currentstroke}%
\pgfsetstrokeopacity{0.000000}%
\pgfsetdash{}{0pt}%
\pgfpathmoveto{\pgfqpoint{2.780921in}{4.530000in}}%
\pgfpathlineto{\pgfqpoint{2.909211in}{4.530000in}}%
\pgfpathlineto{\pgfqpoint{2.909211in}{5.305155in}}%
\pgfpathlineto{\pgfqpoint{2.780921in}{5.305155in}}%
\pgfpathclose%
\pgfusepath{fill}%
\end{pgfscope}%
\begin{pgfscope}%
\pgfpathrectangle{\pgfqpoint{0.600000in}{4.530000in}}{\pgfqpoint{3.900000in}{1.200000in}}%
\pgfusepath{clip}%
\pgfsetbuttcap%
\pgfsetmiterjoin%
\definecolor{currentfill}{rgb}{0.121569,0.466667,0.705882}%
\pgfsetfillcolor{currentfill}%
\pgfsetlinewidth{0.000000pt}%
\definecolor{currentstroke}{rgb}{0.000000,0.000000,0.000000}%
\pgfsetstrokecolor{currentstroke}%
\pgfsetstrokeopacity{0.000000}%
\pgfsetdash{}{0pt}%
\pgfpathmoveto{\pgfqpoint{2.909211in}{4.530000in}}%
\pgfpathlineto{\pgfqpoint{3.037500in}{4.530000in}}%
\pgfpathlineto{\pgfqpoint{3.037500in}{5.255466in}}%
\pgfpathlineto{\pgfqpoint{2.909211in}{5.255466in}}%
\pgfpathclose%
\pgfusepath{fill}%
\end{pgfscope}%
\begin{pgfscope}%
\pgfpathrectangle{\pgfqpoint{0.600000in}{4.530000in}}{\pgfqpoint{3.900000in}{1.200000in}}%
\pgfusepath{clip}%
\pgfsetbuttcap%
\pgfsetmiterjoin%
\definecolor{currentfill}{rgb}{0.121569,0.466667,0.705882}%
\pgfsetfillcolor{currentfill}%
\pgfsetlinewidth{0.000000pt}%
\definecolor{currentstroke}{rgb}{0.000000,0.000000,0.000000}%
\pgfsetstrokecolor{currentstroke}%
\pgfsetstrokeopacity{0.000000}%
\pgfsetdash{}{0pt}%
\pgfpathmoveto{\pgfqpoint{3.037500in}{4.530000in}}%
\pgfpathlineto{\pgfqpoint{3.165789in}{4.530000in}}%
\pgfpathlineto{\pgfqpoint{3.165789in}{5.056708in}}%
\pgfpathlineto{\pgfqpoint{3.037500in}{5.056708in}}%
\pgfpathclose%
\pgfusepath{fill}%
\end{pgfscope}%
\begin{pgfscope}%
\pgfpathrectangle{\pgfqpoint{0.600000in}{4.530000in}}{\pgfqpoint{3.900000in}{1.200000in}}%
\pgfusepath{clip}%
\pgfsetbuttcap%
\pgfsetmiterjoin%
\definecolor{currentfill}{rgb}{0.121569,0.466667,0.705882}%
\pgfsetfillcolor{currentfill}%
\pgfsetlinewidth{0.000000pt}%
\definecolor{currentstroke}{rgb}{0.000000,0.000000,0.000000}%
\pgfsetstrokecolor{currentstroke}%
\pgfsetstrokeopacity{0.000000}%
\pgfsetdash{}{0pt}%
\pgfpathmoveto{\pgfqpoint{3.165789in}{4.530000in}}%
\pgfpathlineto{\pgfqpoint{3.294079in}{4.530000in}}%
\pgfpathlineto{\pgfqpoint{3.294079in}{4.877826in}}%
\pgfpathlineto{\pgfqpoint{3.165789in}{4.877826in}}%
\pgfpathclose%
\pgfusepath{fill}%
\end{pgfscope}%
\begin{pgfscope}%
\pgfpathrectangle{\pgfqpoint{0.600000in}{4.530000in}}{\pgfqpoint{3.900000in}{1.200000in}}%
\pgfusepath{clip}%
\pgfsetbuttcap%
\pgfsetmiterjoin%
\definecolor{currentfill}{rgb}{0.121569,0.466667,0.705882}%
\pgfsetfillcolor{currentfill}%
\pgfsetlinewidth{0.000000pt}%
\definecolor{currentstroke}{rgb}{0.000000,0.000000,0.000000}%
\pgfsetstrokecolor{currentstroke}%
\pgfsetstrokeopacity{0.000000}%
\pgfsetdash{}{0pt}%
\pgfpathmoveto{\pgfqpoint{3.294079in}{4.530000in}}%
\pgfpathlineto{\pgfqpoint{3.422368in}{4.530000in}}%
\pgfpathlineto{\pgfqpoint{3.422368in}{4.778447in}}%
\pgfpathlineto{\pgfqpoint{3.294079in}{4.778447in}}%
\pgfpathclose%
\pgfusepath{fill}%
\end{pgfscope}%
\begin{pgfscope}%
\pgfpathrectangle{\pgfqpoint{0.600000in}{4.530000in}}{\pgfqpoint{3.900000in}{1.200000in}}%
\pgfusepath{clip}%
\pgfsetbuttcap%
\pgfsetmiterjoin%
\definecolor{currentfill}{rgb}{0.121569,0.466667,0.705882}%
\pgfsetfillcolor{currentfill}%
\pgfsetlinewidth{0.000000pt}%
\definecolor{currentstroke}{rgb}{0.000000,0.000000,0.000000}%
\pgfsetstrokecolor{currentstroke}%
\pgfsetstrokeopacity{0.000000}%
\pgfsetdash{}{0pt}%
\pgfpathmoveto{\pgfqpoint{3.422368in}{4.530000in}}%
\pgfpathlineto{\pgfqpoint{3.550658in}{4.530000in}}%
\pgfpathlineto{\pgfqpoint{3.550658in}{4.718820in}}%
\pgfpathlineto{\pgfqpoint{3.422368in}{4.718820in}}%
\pgfpathclose%
\pgfusepath{fill}%
\end{pgfscope}%
\begin{pgfscope}%
\pgfpathrectangle{\pgfqpoint{0.600000in}{4.530000in}}{\pgfqpoint{3.900000in}{1.200000in}}%
\pgfusepath{clip}%
\pgfsetbuttcap%
\pgfsetmiterjoin%
\definecolor{currentfill}{rgb}{0.121569,0.466667,0.705882}%
\pgfsetfillcolor{currentfill}%
\pgfsetlinewidth{0.000000pt}%
\definecolor{currentstroke}{rgb}{0.000000,0.000000,0.000000}%
\pgfsetstrokecolor{currentstroke}%
\pgfsetstrokeopacity{0.000000}%
\pgfsetdash{}{0pt}%
\pgfpathmoveto{\pgfqpoint{3.550658in}{4.530000in}}%
\pgfpathlineto{\pgfqpoint{3.678947in}{4.530000in}}%
\pgfpathlineto{\pgfqpoint{3.678947in}{4.649255in}}%
\pgfpathlineto{\pgfqpoint{3.550658in}{4.649255in}}%
\pgfpathclose%
\pgfusepath{fill}%
\end{pgfscope}%
\begin{pgfscope}%
\pgfpathrectangle{\pgfqpoint{0.600000in}{4.530000in}}{\pgfqpoint{3.900000in}{1.200000in}}%
\pgfusepath{clip}%
\pgfsetbuttcap%
\pgfsetmiterjoin%
\definecolor{currentfill}{rgb}{0.121569,0.466667,0.705882}%
\pgfsetfillcolor{currentfill}%
\pgfsetlinewidth{0.000000pt}%
\definecolor{currentstroke}{rgb}{0.000000,0.000000,0.000000}%
\pgfsetstrokecolor{currentstroke}%
\pgfsetstrokeopacity{0.000000}%
\pgfsetdash{}{0pt}%
\pgfpathmoveto{\pgfqpoint{3.678947in}{4.530000in}}%
\pgfpathlineto{\pgfqpoint{3.807237in}{4.530000in}}%
\pgfpathlineto{\pgfqpoint{3.807237in}{4.549876in}}%
\pgfpathlineto{\pgfqpoint{3.678947in}{4.549876in}}%
\pgfpathclose%
\pgfusepath{fill}%
\end{pgfscope}%
\begin{pgfscope}%
\pgfpathrectangle{\pgfqpoint{0.600000in}{4.530000in}}{\pgfqpoint{3.900000in}{1.200000in}}%
\pgfusepath{clip}%
\pgfsetbuttcap%
\pgfsetmiterjoin%
\definecolor{currentfill}{rgb}{0.121569,0.466667,0.705882}%
\pgfsetfillcolor{currentfill}%
\pgfsetlinewidth{0.000000pt}%
\definecolor{currentstroke}{rgb}{0.000000,0.000000,0.000000}%
\pgfsetstrokecolor{currentstroke}%
\pgfsetstrokeopacity{0.000000}%
\pgfsetdash{}{0pt}%
\pgfpathmoveto{\pgfqpoint{3.807237in}{4.530000in}}%
\pgfpathlineto{\pgfqpoint{3.935526in}{4.530000in}}%
\pgfpathlineto{\pgfqpoint{3.935526in}{4.589627in}}%
\pgfpathlineto{\pgfqpoint{3.807237in}{4.589627in}}%
\pgfpathclose%
\pgfusepath{fill}%
\end{pgfscope}%
\begin{pgfscope}%
\pgfpathrectangle{\pgfqpoint{0.600000in}{4.530000in}}{\pgfqpoint{3.900000in}{1.200000in}}%
\pgfusepath{clip}%
\pgfsetbuttcap%
\pgfsetmiterjoin%
\definecolor{currentfill}{rgb}{0.121569,0.466667,0.705882}%
\pgfsetfillcolor{currentfill}%
\pgfsetlinewidth{0.000000pt}%
\definecolor{currentstroke}{rgb}{0.000000,0.000000,0.000000}%
\pgfsetstrokecolor{currentstroke}%
\pgfsetstrokeopacity{0.000000}%
\pgfsetdash{}{0pt}%
\pgfpathmoveto{\pgfqpoint{3.935526in}{4.530000in}}%
\pgfpathlineto{\pgfqpoint{4.063816in}{4.530000in}}%
\pgfpathlineto{\pgfqpoint{4.063816in}{4.549876in}}%
\pgfpathlineto{\pgfqpoint{3.935526in}{4.549876in}}%
\pgfpathclose%
\pgfusepath{fill}%
\end{pgfscope}%
\begin{pgfscope}%
\pgfpathrectangle{\pgfqpoint{0.600000in}{4.530000in}}{\pgfqpoint{3.900000in}{1.200000in}}%
\pgfusepath{clip}%
\pgfsetbuttcap%
\pgfsetmiterjoin%
\definecolor{currentfill}{rgb}{0.121569,0.466667,0.705882}%
\pgfsetfillcolor{currentfill}%
\pgfsetlinewidth{0.000000pt}%
\definecolor{currentstroke}{rgb}{0.000000,0.000000,0.000000}%
\pgfsetstrokecolor{currentstroke}%
\pgfsetstrokeopacity{0.000000}%
\pgfsetdash{}{0pt}%
\pgfpathmoveto{\pgfqpoint{4.063816in}{4.530000in}}%
\pgfpathlineto{\pgfqpoint{4.192105in}{4.530000in}}%
\pgfpathlineto{\pgfqpoint{4.192105in}{4.530000in}}%
\pgfpathlineto{\pgfqpoint{4.063816in}{4.530000in}}%
\pgfpathclose%
\pgfusepath{fill}%
\end{pgfscope}%
\begin{pgfscope}%
\pgfpathrectangle{\pgfqpoint{0.600000in}{4.530000in}}{\pgfqpoint{3.900000in}{1.200000in}}%
\pgfusepath{clip}%
\pgfsetbuttcap%
\pgfsetmiterjoin%
\definecolor{currentfill}{rgb}{0.121569,0.466667,0.705882}%
\pgfsetfillcolor{currentfill}%
\pgfsetlinewidth{0.000000pt}%
\definecolor{currentstroke}{rgb}{0.000000,0.000000,0.000000}%
\pgfsetstrokecolor{currentstroke}%
\pgfsetstrokeopacity{0.000000}%
\pgfsetdash{}{0pt}%
\pgfpathmoveto{\pgfqpoint{4.192105in}{4.530000in}}%
\pgfpathlineto{\pgfqpoint{4.320395in}{4.530000in}}%
\pgfpathlineto{\pgfqpoint{4.320395in}{4.530000in}}%
\pgfpathlineto{\pgfqpoint{4.192105in}{4.530000in}}%
\pgfpathclose%
\pgfusepath{fill}%
\end{pgfscope}%
\begin{pgfscope}%
\pgfpathrectangle{\pgfqpoint{0.600000in}{4.530000in}}{\pgfqpoint{3.900000in}{1.200000in}}%
\pgfusepath{clip}%
\pgfsetbuttcap%
\pgfsetmiterjoin%
\definecolor{currentfill}{rgb}{0.121569,0.466667,0.705882}%
\pgfsetfillcolor{currentfill}%
\pgfsetlinewidth{0.000000pt}%
\definecolor{currentstroke}{rgb}{0.000000,0.000000,0.000000}%
\pgfsetstrokecolor{currentstroke}%
\pgfsetstrokeopacity{0.000000}%
\pgfsetdash{}{0pt}%
\pgfpathmoveto{\pgfqpoint{4.320395in}{4.530000in}}%
\pgfpathlineto{\pgfqpoint{4.448684in}{4.530000in}}%
\pgfpathlineto{\pgfqpoint{4.448684in}{4.530000in}}%
\pgfpathlineto{\pgfqpoint{4.320395in}{4.530000in}}%
\pgfpathclose%
\pgfusepath{fill}%
\end{pgfscope}%
\begin{pgfscope}%
\pgfpathrectangle{\pgfqpoint{0.600000in}{4.530000in}}{\pgfqpoint{3.900000in}{1.200000in}}%
\pgfusepath{clip}%
\pgfsetbuttcap%
\pgfsetmiterjoin%
\definecolor{currentfill}{rgb}{0.121569,0.466667,0.705882}%
\pgfsetfillcolor{currentfill}%
\pgfsetlinewidth{0.000000pt}%
\definecolor{currentstroke}{rgb}{0.000000,0.000000,0.000000}%
\pgfsetstrokecolor{currentstroke}%
\pgfsetstrokeopacity{0.000000}%
\pgfsetdash{}{0pt}%
\pgfpathmoveto{\pgfqpoint{4.448684in}{4.530000in}}%
\pgfpathlineto{\pgfqpoint{4.576974in}{4.530000in}}%
\pgfpathlineto{\pgfqpoint{4.576974in}{4.530000in}}%
\pgfpathlineto{\pgfqpoint{4.448684in}{4.530000in}}%
\pgfpathclose%
\pgfusepath{fill}%
\end{pgfscope}%
\begin{pgfscope}%
\pgfsetbuttcap%
\pgfsetroundjoin%
\definecolor{currentfill}{rgb}{0.000000,0.000000,0.000000}%
\pgfsetfillcolor{currentfill}%
\pgfsetlinewidth{0.803000pt}%
\definecolor{currentstroke}{rgb}{0.000000,0.000000,0.000000}%
\pgfsetstrokecolor{currentstroke}%
\pgfsetdash{}{0pt}%
\pgfsys@defobject{currentmarker}{\pgfqpoint{0.000000in}{0.000000in}}{\pgfqpoint{0.000000in}{0.048611in}}{%
\pgfpathmoveto{\pgfqpoint{0.000000in}{0.000000in}}%
\pgfpathlineto{\pgfqpoint{0.000000in}{0.048611in}}%
\pgfusepath{stroke,fill}%
}%
\begin{pgfscope}%
\pgfsys@transformshift{1.010526in}{4.530000in}%
\pgfsys@useobject{currentmarker}{}%
\end{pgfscope}%
\end{pgfscope}%
\begin{pgfscope}%
\pgfsetbuttcap%
\pgfsetroundjoin%
\definecolor{currentfill}{rgb}{0.000000,0.000000,0.000000}%
\pgfsetfillcolor{currentfill}%
\pgfsetlinewidth{0.803000pt}%
\definecolor{currentstroke}{rgb}{0.000000,0.000000,0.000000}%
\pgfsetstrokecolor{currentstroke}%
\pgfsetdash{}{0pt}%
\pgfsys@defobject{currentmarker}{\pgfqpoint{0.000000in}{0.000000in}}{\pgfqpoint{0.000000in}{0.048611in}}{%
\pgfpathmoveto{\pgfqpoint{0.000000in}{0.000000in}}%
\pgfpathlineto{\pgfqpoint{0.000000in}{0.048611in}}%
\pgfusepath{stroke,fill}%
}%
\begin{pgfscope}%
\pgfsys@transformshift{1.523684in}{4.530000in}%
\pgfsys@useobject{currentmarker}{}%
\end{pgfscope}%
\end{pgfscope}%
\begin{pgfscope}%
\pgfsetbuttcap%
\pgfsetroundjoin%
\definecolor{currentfill}{rgb}{0.000000,0.000000,0.000000}%
\pgfsetfillcolor{currentfill}%
\pgfsetlinewidth{0.803000pt}%
\definecolor{currentstroke}{rgb}{0.000000,0.000000,0.000000}%
\pgfsetstrokecolor{currentstroke}%
\pgfsetdash{}{0pt}%
\pgfsys@defobject{currentmarker}{\pgfqpoint{0.000000in}{0.000000in}}{\pgfqpoint{0.000000in}{0.048611in}}{%
\pgfpathmoveto{\pgfqpoint{0.000000in}{0.000000in}}%
\pgfpathlineto{\pgfqpoint{0.000000in}{0.048611in}}%
\pgfusepath{stroke,fill}%
}%
\begin{pgfscope}%
\pgfsys@transformshift{2.036842in}{4.530000in}%
\pgfsys@useobject{currentmarker}{}%
\end{pgfscope}%
\end{pgfscope}%
\begin{pgfscope}%
\pgfsetbuttcap%
\pgfsetroundjoin%
\definecolor{currentfill}{rgb}{0.000000,0.000000,0.000000}%
\pgfsetfillcolor{currentfill}%
\pgfsetlinewidth{0.803000pt}%
\definecolor{currentstroke}{rgb}{0.000000,0.000000,0.000000}%
\pgfsetstrokecolor{currentstroke}%
\pgfsetdash{}{0pt}%
\pgfsys@defobject{currentmarker}{\pgfqpoint{0.000000in}{0.000000in}}{\pgfqpoint{0.000000in}{0.048611in}}{%
\pgfpathmoveto{\pgfqpoint{0.000000in}{0.000000in}}%
\pgfpathlineto{\pgfqpoint{0.000000in}{0.048611in}}%
\pgfusepath{stroke,fill}%
}%
\begin{pgfscope}%
\pgfsys@transformshift{2.550000in}{4.530000in}%
\pgfsys@useobject{currentmarker}{}%
\end{pgfscope}%
\end{pgfscope}%
\begin{pgfscope}%
\pgfsetbuttcap%
\pgfsetroundjoin%
\definecolor{currentfill}{rgb}{0.000000,0.000000,0.000000}%
\pgfsetfillcolor{currentfill}%
\pgfsetlinewidth{0.803000pt}%
\definecolor{currentstroke}{rgb}{0.000000,0.000000,0.000000}%
\pgfsetstrokecolor{currentstroke}%
\pgfsetdash{}{0pt}%
\pgfsys@defobject{currentmarker}{\pgfqpoint{0.000000in}{0.000000in}}{\pgfqpoint{0.000000in}{0.048611in}}{%
\pgfpathmoveto{\pgfqpoint{0.000000in}{0.000000in}}%
\pgfpathlineto{\pgfqpoint{0.000000in}{0.048611in}}%
\pgfusepath{stroke,fill}%
}%
\begin{pgfscope}%
\pgfsys@transformshift{3.063158in}{4.530000in}%
\pgfsys@useobject{currentmarker}{}%
\end{pgfscope}%
\end{pgfscope}%
\begin{pgfscope}%
\pgfsetbuttcap%
\pgfsetroundjoin%
\definecolor{currentfill}{rgb}{0.000000,0.000000,0.000000}%
\pgfsetfillcolor{currentfill}%
\pgfsetlinewidth{0.803000pt}%
\definecolor{currentstroke}{rgb}{0.000000,0.000000,0.000000}%
\pgfsetstrokecolor{currentstroke}%
\pgfsetdash{}{0pt}%
\pgfsys@defobject{currentmarker}{\pgfqpoint{0.000000in}{0.000000in}}{\pgfqpoint{0.000000in}{0.048611in}}{%
\pgfpathmoveto{\pgfqpoint{0.000000in}{0.000000in}}%
\pgfpathlineto{\pgfqpoint{0.000000in}{0.048611in}}%
\pgfusepath{stroke,fill}%
}%
\begin{pgfscope}%
\pgfsys@transformshift{3.576316in}{4.530000in}%
\pgfsys@useobject{currentmarker}{}%
\end{pgfscope}%
\end{pgfscope}%
\begin{pgfscope}%
\pgfsetbuttcap%
\pgfsetroundjoin%
\definecolor{currentfill}{rgb}{0.000000,0.000000,0.000000}%
\pgfsetfillcolor{currentfill}%
\pgfsetlinewidth{0.803000pt}%
\definecolor{currentstroke}{rgb}{0.000000,0.000000,0.000000}%
\pgfsetstrokecolor{currentstroke}%
\pgfsetdash{}{0pt}%
\pgfsys@defobject{currentmarker}{\pgfqpoint{0.000000in}{0.000000in}}{\pgfqpoint{0.000000in}{0.048611in}}{%
\pgfpathmoveto{\pgfqpoint{0.000000in}{0.000000in}}%
\pgfpathlineto{\pgfqpoint{0.000000in}{0.048611in}}%
\pgfusepath{stroke,fill}%
}%
\begin{pgfscope}%
\pgfsys@transformshift{4.089474in}{4.530000in}%
\pgfsys@useobject{currentmarker}{}%
\end{pgfscope}%
\end{pgfscope}%
\begin{pgfscope}%
\pgfsetbuttcap%
\pgfsetroundjoin%
\definecolor{currentfill}{rgb}{0.000000,0.000000,0.000000}%
\pgfsetfillcolor{currentfill}%
\pgfsetlinewidth{0.803000pt}%
\definecolor{currentstroke}{rgb}{0.000000,0.000000,0.000000}%
\pgfsetstrokecolor{currentstroke}%
\pgfsetdash{}{0pt}%
\pgfsys@defobject{currentmarker}{\pgfqpoint{0.000000in}{0.000000in}}{\pgfqpoint{0.048611in}{0.000000in}}{%
\pgfpathmoveto{\pgfqpoint{0.000000in}{0.000000in}}%
\pgfpathlineto{\pgfqpoint{0.048611in}{0.000000in}}%
\pgfusepath{stroke,fill}%
}%
\begin{pgfscope}%
\pgfsys@transformshift{0.600000in}{4.530000in}%
\pgfsys@useobject{currentmarker}{}%
\end{pgfscope}%
\end{pgfscope}%
\begin{pgfscope}%
\definecolor{textcolor}{rgb}{0.000000,0.000000,0.000000}%
\pgfsetstrokecolor{textcolor}%
\pgfsetfillcolor{textcolor}%
\pgftext[x=0.463024in,y=4.477238in,left,base]{\color{textcolor}\sffamily\fontsize{10.000000}{12.000000}\selectfont 0}%
\end{pgfscope}%
\begin{pgfscope}%
\pgfsetbuttcap%
\pgfsetroundjoin%
\definecolor{currentfill}{rgb}{0.000000,0.000000,0.000000}%
\pgfsetfillcolor{currentfill}%
\pgfsetlinewidth{0.803000pt}%
\definecolor{currentstroke}{rgb}{0.000000,0.000000,0.000000}%
\pgfsetstrokecolor{currentstroke}%
\pgfsetdash{}{0pt}%
\pgfsys@defobject{currentmarker}{\pgfqpoint{0.000000in}{0.000000in}}{\pgfqpoint{0.048611in}{0.000000in}}{%
\pgfpathmoveto{\pgfqpoint{0.000000in}{0.000000in}}%
\pgfpathlineto{\pgfqpoint{0.048611in}{0.000000in}}%
\pgfusepath{stroke,fill}%
}%
\begin{pgfscope}%
\pgfsys@transformshift{0.600000in}{5.026894in}%
\pgfsys@useobject{currentmarker}{}%
\end{pgfscope}%
\end{pgfscope}%
\begin{pgfscope}%
\definecolor{textcolor}{rgb}{0.000000,0.000000,0.000000}%
\pgfsetstrokecolor{textcolor}%
\pgfsetfillcolor{textcolor}%
\pgftext[x=0.374658in,y=4.974133in,left,base]{\color{textcolor}\sffamily\fontsize{10.000000}{12.000000}\selectfont 50}%
\end{pgfscope}%
\begin{pgfscope}%
\pgfsetbuttcap%
\pgfsetroundjoin%
\definecolor{currentfill}{rgb}{0.000000,0.000000,0.000000}%
\pgfsetfillcolor{currentfill}%
\pgfsetlinewidth{0.803000pt}%
\definecolor{currentstroke}{rgb}{0.000000,0.000000,0.000000}%
\pgfsetstrokecolor{currentstroke}%
\pgfsetdash{}{0pt}%
\pgfsys@defobject{currentmarker}{\pgfqpoint{0.000000in}{0.000000in}}{\pgfqpoint{0.048611in}{0.000000in}}{%
\pgfpathmoveto{\pgfqpoint{0.000000in}{0.000000in}}%
\pgfpathlineto{\pgfqpoint{0.048611in}{0.000000in}}%
\pgfusepath{stroke,fill}%
}%
\begin{pgfscope}%
\pgfsys@transformshift{0.600000in}{5.523789in}%
\pgfsys@useobject{currentmarker}{}%
\end{pgfscope}%
\end{pgfscope}%
\begin{pgfscope}%
\definecolor{textcolor}{rgb}{0.000000,0.000000,0.000000}%
\pgfsetstrokecolor{textcolor}%
\pgfsetfillcolor{textcolor}%
\pgftext[x=0.286293in,y=5.471027in,left,base]{\color{textcolor}\sffamily\fontsize{10.000000}{12.000000}\selectfont 100}%
\end{pgfscope}%
\begin{pgfscope}%
\pgfsetrectcap%
\pgfsetmiterjoin%
\pgfsetlinewidth{0.803000pt}%
\definecolor{currentstroke}{rgb}{0.000000,0.000000,0.000000}%
\pgfsetstrokecolor{currentstroke}%
\pgfsetdash{}{0pt}%
\pgfpathmoveto{\pgfqpoint{0.600000in}{4.530000in}}%
\pgfpathlineto{\pgfqpoint{0.600000in}{5.730000in}}%
\pgfusepath{stroke}%
\end{pgfscope}%
\begin{pgfscope}%
\pgfsetrectcap%
\pgfsetmiterjoin%
\pgfsetlinewidth{0.803000pt}%
\definecolor{currentstroke}{rgb}{0.000000,0.000000,0.000000}%
\pgfsetstrokecolor{currentstroke}%
\pgfsetdash{}{0pt}%
\pgfpathmoveto{\pgfqpoint{4.500000in}{4.530000in}}%
\pgfpathlineto{\pgfqpoint{4.500000in}{5.730000in}}%
\pgfusepath{stroke}%
\end{pgfscope}%
\begin{pgfscope}%
\pgfsetrectcap%
\pgfsetmiterjoin%
\pgfsetlinewidth{0.803000pt}%
\definecolor{currentstroke}{rgb}{0.000000,0.000000,0.000000}%
\pgfsetstrokecolor{currentstroke}%
\pgfsetdash{}{0pt}%
\pgfpathmoveto{\pgfqpoint{0.600000in}{4.530000in}}%
\pgfpathlineto{\pgfqpoint{4.500000in}{4.530000in}}%
\pgfusepath{stroke}%
\end{pgfscope}%
\begin{pgfscope}%
\pgfsetrectcap%
\pgfsetmiterjoin%
\pgfsetlinewidth{0.803000pt}%
\definecolor{currentstroke}{rgb}{0.000000,0.000000,0.000000}%
\pgfsetstrokecolor{currentstroke}%
\pgfsetdash{}{0pt}%
\pgfpathmoveto{\pgfqpoint{0.600000in}{5.730000in}}%
\pgfpathlineto{\pgfqpoint{4.500000in}{5.730000in}}%
\pgfusepath{stroke}%
\end{pgfscope}%
\begin{pgfscope}%
\pgfsetbuttcap%
\pgfsetmiterjoin%
\definecolor{currentfill}{rgb}{1.000000,1.000000,1.000000}%
\pgfsetfillcolor{currentfill}%
\pgfsetlinewidth{0.000000pt}%
\definecolor{currentstroke}{rgb}{0.000000,0.000000,0.000000}%
\pgfsetstrokecolor{currentstroke}%
\pgfsetstrokeopacity{0.000000}%
\pgfsetdash{}{0pt}%
\pgfpathmoveto{\pgfqpoint{4.530000in}{0.600000in}}%
\pgfpathlineto{\pgfqpoint{5.730000in}{0.600000in}}%
\pgfpathlineto{\pgfqpoint{5.730000in}{4.500000in}}%
\pgfpathlineto{\pgfqpoint{4.530000in}{4.500000in}}%
\pgfpathclose%
\pgfusepath{fill}%
\end{pgfscope}%
\begin{pgfscope}%
\pgfpathrectangle{\pgfqpoint{4.530000in}{0.600000in}}{\pgfqpoint{1.200000in}{3.900000in}}%
\pgfusepath{clip}%
\pgfsetbuttcap%
\pgfsetmiterjoin%
\definecolor{currentfill}{rgb}{0.121569,0.466667,0.705882}%
\pgfsetfillcolor{currentfill}%
\pgfsetlinewidth{0.000000pt}%
\definecolor{currentstroke}{rgb}{0.000000,0.000000,0.000000}%
\pgfsetstrokecolor{currentstroke}%
\pgfsetstrokeopacity{0.000000}%
\pgfsetdash{}{0pt}%
\pgfpathmoveto{\pgfqpoint{4.530000in}{0.600000in}}%
\pgfpathlineto{\pgfqpoint{4.530000in}{0.600000in}}%
\pgfpathlineto{\pgfqpoint{4.530000in}{0.728289in}}%
\pgfpathlineto{\pgfqpoint{4.530000in}{0.728289in}}%
\pgfpathclose%
\pgfusepath{fill}%
\end{pgfscope}%
\begin{pgfscope}%
\pgfpathrectangle{\pgfqpoint{4.530000in}{0.600000in}}{\pgfqpoint{1.200000in}{3.900000in}}%
\pgfusepath{clip}%
\pgfsetbuttcap%
\pgfsetmiterjoin%
\definecolor{currentfill}{rgb}{0.121569,0.466667,0.705882}%
\pgfsetfillcolor{currentfill}%
\pgfsetlinewidth{0.000000pt}%
\definecolor{currentstroke}{rgb}{0.000000,0.000000,0.000000}%
\pgfsetstrokecolor{currentstroke}%
\pgfsetstrokeopacity{0.000000}%
\pgfsetdash{}{0pt}%
\pgfpathmoveto{\pgfqpoint{4.530000in}{0.728289in}}%
\pgfpathlineto{\pgfqpoint{4.551164in}{0.728289in}}%
\pgfpathlineto{\pgfqpoint{4.551164in}{0.856579in}}%
\pgfpathlineto{\pgfqpoint{4.530000in}{0.856579in}}%
\pgfpathclose%
\pgfusepath{fill}%
\end{pgfscope}%
\begin{pgfscope}%
\pgfpathrectangle{\pgfqpoint{4.530000in}{0.600000in}}{\pgfqpoint{1.200000in}{3.900000in}}%
\pgfusepath{clip}%
\pgfsetbuttcap%
\pgfsetmiterjoin%
\definecolor{currentfill}{rgb}{0.121569,0.466667,0.705882}%
\pgfsetfillcolor{currentfill}%
\pgfsetlinewidth{0.000000pt}%
\definecolor{currentstroke}{rgb}{0.000000,0.000000,0.000000}%
\pgfsetstrokecolor{currentstroke}%
\pgfsetstrokeopacity{0.000000}%
\pgfsetdash{}{0pt}%
\pgfpathmoveto{\pgfqpoint{4.530000in}{0.856579in}}%
\pgfpathlineto{\pgfqpoint{4.551164in}{0.856579in}}%
\pgfpathlineto{\pgfqpoint{4.551164in}{0.984868in}}%
\pgfpathlineto{\pgfqpoint{4.530000in}{0.984868in}}%
\pgfpathclose%
\pgfusepath{fill}%
\end{pgfscope}%
\begin{pgfscope}%
\pgfpathrectangle{\pgfqpoint{4.530000in}{0.600000in}}{\pgfqpoint{1.200000in}{3.900000in}}%
\pgfusepath{clip}%
\pgfsetbuttcap%
\pgfsetmiterjoin%
\definecolor{currentfill}{rgb}{0.121569,0.466667,0.705882}%
\pgfsetfillcolor{currentfill}%
\pgfsetlinewidth{0.000000pt}%
\definecolor{currentstroke}{rgb}{0.000000,0.000000,0.000000}%
\pgfsetstrokecolor{currentstroke}%
\pgfsetstrokeopacity{0.000000}%
\pgfsetdash{}{0pt}%
\pgfpathmoveto{\pgfqpoint{4.530000in}{0.984868in}}%
\pgfpathlineto{\pgfqpoint{4.530000in}{0.984868in}}%
\pgfpathlineto{\pgfqpoint{4.530000in}{1.113158in}}%
\pgfpathlineto{\pgfqpoint{4.530000in}{1.113158in}}%
\pgfpathclose%
\pgfusepath{fill}%
\end{pgfscope}%
\begin{pgfscope}%
\pgfpathrectangle{\pgfqpoint{4.530000in}{0.600000in}}{\pgfqpoint{1.200000in}{3.900000in}}%
\pgfusepath{clip}%
\pgfsetbuttcap%
\pgfsetmiterjoin%
\definecolor{currentfill}{rgb}{0.121569,0.466667,0.705882}%
\pgfsetfillcolor{currentfill}%
\pgfsetlinewidth{0.000000pt}%
\definecolor{currentstroke}{rgb}{0.000000,0.000000,0.000000}%
\pgfsetstrokecolor{currentstroke}%
\pgfsetstrokeopacity{0.000000}%
\pgfsetdash{}{0pt}%
\pgfpathmoveto{\pgfqpoint{4.530000in}{1.113158in}}%
\pgfpathlineto{\pgfqpoint{4.572328in}{1.113158in}}%
\pgfpathlineto{\pgfqpoint{4.572328in}{1.241447in}}%
\pgfpathlineto{\pgfqpoint{4.530000in}{1.241447in}}%
\pgfpathclose%
\pgfusepath{fill}%
\end{pgfscope}%
\begin{pgfscope}%
\pgfpathrectangle{\pgfqpoint{4.530000in}{0.600000in}}{\pgfqpoint{1.200000in}{3.900000in}}%
\pgfusepath{clip}%
\pgfsetbuttcap%
\pgfsetmiterjoin%
\definecolor{currentfill}{rgb}{0.121569,0.466667,0.705882}%
\pgfsetfillcolor{currentfill}%
\pgfsetlinewidth{0.000000pt}%
\definecolor{currentstroke}{rgb}{0.000000,0.000000,0.000000}%
\pgfsetstrokecolor{currentstroke}%
\pgfsetstrokeopacity{0.000000}%
\pgfsetdash{}{0pt}%
\pgfpathmoveto{\pgfqpoint{4.530000in}{1.241447in}}%
\pgfpathlineto{\pgfqpoint{4.604074in}{1.241447in}}%
\pgfpathlineto{\pgfqpoint{4.604074in}{1.369737in}}%
\pgfpathlineto{\pgfqpoint{4.530000in}{1.369737in}}%
\pgfpathclose%
\pgfusepath{fill}%
\end{pgfscope}%
\begin{pgfscope}%
\pgfpathrectangle{\pgfqpoint{4.530000in}{0.600000in}}{\pgfqpoint{1.200000in}{3.900000in}}%
\pgfusepath{clip}%
\pgfsetbuttcap%
\pgfsetmiterjoin%
\definecolor{currentfill}{rgb}{0.121569,0.466667,0.705882}%
\pgfsetfillcolor{currentfill}%
\pgfsetlinewidth{0.000000pt}%
\definecolor{currentstroke}{rgb}{0.000000,0.000000,0.000000}%
\pgfsetstrokecolor{currentstroke}%
\pgfsetstrokeopacity{0.000000}%
\pgfsetdash{}{0pt}%
\pgfpathmoveto{\pgfqpoint{4.530000in}{1.369737in}}%
\pgfpathlineto{\pgfqpoint{4.678148in}{1.369737in}}%
\pgfpathlineto{\pgfqpoint{4.678148in}{1.498026in}}%
\pgfpathlineto{\pgfqpoint{4.530000in}{1.498026in}}%
\pgfpathclose%
\pgfusepath{fill}%
\end{pgfscope}%
\begin{pgfscope}%
\pgfpathrectangle{\pgfqpoint{4.530000in}{0.600000in}}{\pgfqpoint{1.200000in}{3.900000in}}%
\pgfusepath{clip}%
\pgfsetbuttcap%
\pgfsetmiterjoin%
\definecolor{currentfill}{rgb}{0.121569,0.466667,0.705882}%
\pgfsetfillcolor{currentfill}%
\pgfsetlinewidth{0.000000pt}%
\definecolor{currentstroke}{rgb}{0.000000,0.000000,0.000000}%
\pgfsetstrokecolor{currentstroke}%
\pgfsetstrokeopacity{0.000000}%
\pgfsetdash{}{0pt}%
\pgfpathmoveto{\pgfqpoint{4.530000in}{1.498026in}}%
\pgfpathlineto{\pgfqpoint{4.699312in}{1.498026in}}%
\pgfpathlineto{\pgfqpoint{4.699312in}{1.626316in}}%
\pgfpathlineto{\pgfqpoint{4.530000in}{1.626316in}}%
\pgfpathclose%
\pgfusepath{fill}%
\end{pgfscope}%
\begin{pgfscope}%
\pgfpathrectangle{\pgfqpoint{4.530000in}{0.600000in}}{\pgfqpoint{1.200000in}{3.900000in}}%
\pgfusepath{clip}%
\pgfsetbuttcap%
\pgfsetmiterjoin%
\definecolor{currentfill}{rgb}{0.121569,0.466667,0.705882}%
\pgfsetfillcolor{currentfill}%
\pgfsetlinewidth{0.000000pt}%
\definecolor{currentstroke}{rgb}{0.000000,0.000000,0.000000}%
\pgfsetstrokecolor{currentstroke}%
\pgfsetstrokeopacity{0.000000}%
\pgfsetdash{}{0pt}%
\pgfpathmoveto{\pgfqpoint{4.530000in}{1.626316in}}%
\pgfpathlineto{\pgfqpoint{4.900370in}{1.626316in}}%
\pgfpathlineto{\pgfqpoint{4.900370in}{1.754605in}}%
\pgfpathlineto{\pgfqpoint{4.530000in}{1.754605in}}%
\pgfpathclose%
\pgfusepath{fill}%
\end{pgfscope}%
\begin{pgfscope}%
\pgfpathrectangle{\pgfqpoint{4.530000in}{0.600000in}}{\pgfqpoint{1.200000in}{3.900000in}}%
\pgfusepath{clip}%
\pgfsetbuttcap%
\pgfsetmiterjoin%
\definecolor{currentfill}{rgb}{0.121569,0.466667,0.705882}%
\pgfsetfillcolor{currentfill}%
\pgfsetlinewidth{0.000000pt}%
\definecolor{currentstroke}{rgb}{0.000000,0.000000,0.000000}%
\pgfsetstrokecolor{currentstroke}%
\pgfsetstrokeopacity{0.000000}%
\pgfsetdash{}{0pt}%
\pgfpathmoveto{\pgfqpoint{4.530000in}{1.754605in}}%
\pgfpathlineto{\pgfqpoint{4.847460in}{1.754605in}}%
\pgfpathlineto{\pgfqpoint{4.847460in}{1.882895in}}%
\pgfpathlineto{\pgfqpoint{4.530000in}{1.882895in}}%
\pgfpathclose%
\pgfusepath{fill}%
\end{pgfscope}%
\begin{pgfscope}%
\pgfpathrectangle{\pgfqpoint{4.530000in}{0.600000in}}{\pgfqpoint{1.200000in}{3.900000in}}%
\pgfusepath{clip}%
\pgfsetbuttcap%
\pgfsetmiterjoin%
\definecolor{currentfill}{rgb}{0.121569,0.466667,0.705882}%
\pgfsetfillcolor{currentfill}%
\pgfsetlinewidth{0.000000pt}%
\definecolor{currentstroke}{rgb}{0.000000,0.000000,0.000000}%
\pgfsetstrokecolor{currentstroke}%
\pgfsetstrokeopacity{0.000000}%
\pgfsetdash{}{0pt}%
\pgfpathmoveto{\pgfqpoint{4.530000in}{1.882895in}}%
\pgfpathlineto{\pgfqpoint{5.059101in}{1.882895in}}%
\pgfpathlineto{\pgfqpoint{5.059101in}{2.011184in}}%
\pgfpathlineto{\pgfqpoint{4.530000in}{2.011184in}}%
\pgfpathclose%
\pgfusepath{fill}%
\end{pgfscope}%
\begin{pgfscope}%
\pgfpathrectangle{\pgfqpoint{4.530000in}{0.600000in}}{\pgfqpoint{1.200000in}{3.900000in}}%
\pgfusepath{clip}%
\pgfsetbuttcap%
\pgfsetmiterjoin%
\definecolor{currentfill}{rgb}{0.121569,0.466667,0.705882}%
\pgfsetfillcolor{currentfill}%
\pgfsetlinewidth{0.000000pt}%
\definecolor{currentstroke}{rgb}{0.000000,0.000000,0.000000}%
\pgfsetstrokecolor{currentstroke}%
\pgfsetstrokeopacity{0.000000}%
\pgfsetdash{}{0pt}%
\pgfpathmoveto{\pgfqpoint{4.530000in}{2.011184in}}%
\pgfpathlineto{\pgfqpoint{5.291905in}{2.011184in}}%
\pgfpathlineto{\pgfqpoint{5.291905in}{2.139474in}}%
\pgfpathlineto{\pgfqpoint{4.530000in}{2.139474in}}%
\pgfpathclose%
\pgfusepath{fill}%
\end{pgfscope}%
\begin{pgfscope}%
\pgfpathrectangle{\pgfqpoint{4.530000in}{0.600000in}}{\pgfqpoint{1.200000in}{3.900000in}}%
\pgfusepath{clip}%
\pgfsetbuttcap%
\pgfsetmiterjoin%
\definecolor{currentfill}{rgb}{0.121569,0.466667,0.705882}%
\pgfsetfillcolor{currentfill}%
\pgfsetlinewidth{0.000000pt}%
\definecolor{currentstroke}{rgb}{0.000000,0.000000,0.000000}%
\pgfsetstrokecolor{currentstroke}%
\pgfsetstrokeopacity{0.000000}%
\pgfsetdash{}{0pt}%
\pgfpathmoveto{\pgfqpoint{4.530000in}{2.139474in}}%
\pgfpathlineto{\pgfqpoint{5.270741in}{2.139474in}}%
\pgfpathlineto{\pgfqpoint{5.270741in}{2.267763in}}%
\pgfpathlineto{\pgfqpoint{4.530000in}{2.267763in}}%
\pgfpathclose%
\pgfusepath{fill}%
\end{pgfscope}%
\begin{pgfscope}%
\pgfpathrectangle{\pgfqpoint{4.530000in}{0.600000in}}{\pgfqpoint{1.200000in}{3.900000in}}%
\pgfusepath{clip}%
\pgfsetbuttcap%
\pgfsetmiterjoin%
\definecolor{currentfill}{rgb}{0.121569,0.466667,0.705882}%
\pgfsetfillcolor{currentfill}%
\pgfsetlinewidth{0.000000pt}%
\definecolor{currentstroke}{rgb}{0.000000,0.000000,0.000000}%
\pgfsetstrokecolor{currentstroke}%
\pgfsetstrokeopacity{0.000000}%
\pgfsetdash{}{0pt}%
\pgfpathmoveto{\pgfqpoint{4.530000in}{2.267763in}}%
\pgfpathlineto{\pgfqpoint{5.408307in}{2.267763in}}%
\pgfpathlineto{\pgfqpoint{5.408307in}{2.396053in}}%
\pgfpathlineto{\pgfqpoint{4.530000in}{2.396053in}}%
\pgfpathclose%
\pgfusepath{fill}%
\end{pgfscope}%
\begin{pgfscope}%
\pgfpathrectangle{\pgfqpoint{4.530000in}{0.600000in}}{\pgfqpoint{1.200000in}{3.900000in}}%
\pgfusepath{clip}%
\pgfsetbuttcap%
\pgfsetmiterjoin%
\definecolor{currentfill}{rgb}{0.121569,0.466667,0.705882}%
\pgfsetfillcolor{currentfill}%
\pgfsetlinewidth{0.000000pt}%
\definecolor{currentstroke}{rgb}{0.000000,0.000000,0.000000}%
\pgfsetstrokecolor{currentstroke}%
\pgfsetstrokeopacity{0.000000}%
\pgfsetdash{}{0pt}%
\pgfpathmoveto{\pgfqpoint{4.530000in}{2.396053in}}%
\pgfpathlineto{\pgfqpoint{5.577619in}{2.396053in}}%
\pgfpathlineto{\pgfqpoint{5.577619in}{2.524342in}}%
\pgfpathlineto{\pgfqpoint{4.530000in}{2.524342in}}%
\pgfpathclose%
\pgfusepath{fill}%
\end{pgfscope}%
\begin{pgfscope}%
\pgfpathrectangle{\pgfqpoint{4.530000in}{0.600000in}}{\pgfqpoint{1.200000in}{3.900000in}}%
\pgfusepath{clip}%
\pgfsetbuttcap%
\pgfsetmiterjoin%
\definecolor{currentfill}{rgb}{0.121569,0.466667,0.705882}%
\pgfsetfillcolor{currentfill}%
\pgfsetlinewidth{0.000000pt}%
\definecolor{currentstroke}{rgb}{0.000000,0.000000,0.000000}%
\pgfsetstrokecolor{currentstroke}%
\pgfsetstrokeopacity{0.000000}%
\pgfsetdash{}{0pt}%
\pgfpathmoveto{\pgfqpoint{4.530000in}{2.524342in}}%
\pgfpathlineto{\pgfqpoint{5.313069in}{2.524342in}}%
\pgfpathlineto{\pgfqpoint{5.313069in}{2.652632in}}%
\pgfpathlineto{\pgfqpoint{4.530000in}{2.652632in}}%
\pgfpathclose%
\pgfusepath{fill}%
\end{pgfscope}%
\begin{pgfscope}%
\pgfpathrectangle{\pgfqpoint{4.530000in}{0.600000in}}{\pgfqpoint{1.200000in}{3.900000in}}%
\pgfusepath{clip}%
\pgfsetbuttcap%
\pgfsetmiterjoin%
\definecolor{currentfill}{rgb}{0.121569,0.466667,0.705882}%
\pgfsetfillcolor{currentfill}%
\pgfsetlinewidth{0.000000pt}%
\definecolor{currentstroke}{rgb}{0.000000,0.000000,0.000000}%
\pgfsetstrokecolor{currentstroke}%
\pgfsetstrokeopacity{0.000000}%
\pgfsetdash{}{0pt}%
\pgfpathmoveto{\pgfqpoint{4.530000in}{2.652632in}}%
\pgfpathlineto{\pgfqpoint{5.672857in}{2.652632in}}%
\pgfpathlineto{\pgfqpoint{5.672857in}{2.780921in}}%
\pgfpathlineto{\pgfqpoint{4.530000in}{2.780921in}}%
\pgfpathclose%
\pgfusepath{fill}%
\end{pgfscope}%
\begin{pgfscope}%
\pgfpathrectangle{\pgfqpoint{4.530000in}{0.600000in}}{\pgfqpoint{1.200000in}{3.900000in}}%
\pgfusepath{clip}%
\pgfsetbuttcap%
\pgfsetmiterjoin%
\definecolor{currentfill}{rgb}{0.121569,0.466667,0.705882}%
\pgfsetfillcolor{currentfill}%
\pgfsetlinewidth{0.000000pt}%
\definecolor{currentstroke}{rgb}{0.000000,0.000000,0.000000}%
\pgfsetstrokecolor{currentstroke}%
\pgfsetstrokeopacity{0.000000}%
\pgfsetdash{}{0pt}%
\pgfpathmoveto{\pgfqpoint{4.530000in}{2.780921in}}%
\pgfpathlineto{\pgfqpoint{5.429471in}{2.780921in}}%
\pgfpathlineto{\pgfqpoint{5.429471in}{2.909211in}}%
\pgfpathlineto{\pgfqpoint{4.530000in}{2.909211in}}%
\pgfpathclose%
\pgfusepath{fill}%
\end{pgfscope}%
\begin{pgfscope}%
\pgfpathrectangle{\pgfqpoint{4.530000in}{0.600000in}}{\pgfqpoint{1.200000in}{3.900000in}}%
\pgfusepath{clip}%
\pgfsetbuttcap%
\pgfsetmiterjoin%
\definecolor{currentfill}{rgb}{0.121569,0.466667,0.705882}%
\pgfsetfillcolor{currentfill}%
\pgfsetlinewidth{0.000000pt}%
\definecolor{currentstroke}{rgb}{0.000000,0.000000,0.000000}%
\pgfsetstrokecolor{currentstroke}%
\pgfsetstrokeopacity{0.000000}%
\pgfsetdash{}{0pt}%
\pgfpathmoveto{\pgfqpoint{4.530000in}{2.909211in}}%
\pgfpathlineto{\pgfqpoint{5.186085in}{2.909211in}}%
\pgfpathlineto{\pgfqpoint{5.186085in}{3.037500in}}%
\pgfpathlineto{\pgfqpoint{4.530000in}{3.037500in}}%
\pgfpathclose%
\pgfusepath{fill}%
\end{pgfscope}%
\begin{pgfscope}%
\pgfpathrectangle{\pgfqpoint{4.530000in}{0.600000in}}{\pgfqpoint{1.200000in}{3.900000in}}%
\pgfusepath{clip}%
\pgfsetbuttcap%
\pgfsetmiterjoin%
\definecolor{currentfill}{rgb}{0.121569,0.466667,0.705882}%
\pgfsetfillcolor{currentfill}%
\pgfsetlinewidth{0.000000pt}%
\definecolor{currentstroke}{rgb}{0.000000,0.000000,0.000000}%
\pgfsetstrokecolor{currentstroke}%
\pgfsetstrokeopacity{0.000000}%
\pgfsetdash{}{0pt}%
\pgfpathmoveto{\pgfqpoint{4.530000in}{3.037500in}}%
\pgfpathlineto{\pgfqpoint{5.175503in}{3.037500in}}%
\pgfpathlineto{\pgfqpoint{5.175503in}{3.165789in}}%
\pgfpathlineto{\pgfqpoint{4.530000in}{3.165789in}}%
\pgfpathclose%
\pgfusepath{fill}%
\end{pgfscope}%
\begin{pgfscope}%
\pgfpathrectangle{\pgfqpoint{4.530000in}{0.600000in}}{\pgfqpoint{1.200000in}{3.900000in}}%
\pgfusepath{clip}%
\pgfsetbuttcap%
\pgfsetmiterjoin%
\definecolor{currentfill}{rgb}{0.121569,0.466667,0.705882}%
\pgfsetfillcolor{currentfill}%
\pgfsetlinewidth{0.000000pt}%
\definecolor{currentstroke}{rgb}{0.000000,0.000000,0.000000}%
\pgfsetstrokecolor{currentstroke}%
\pgfsetstrokeopacity{0.000000}%
\pgfsetdash{}{0pt}%
\pgfpathmoveto{\pgfqpoint{4.530000in}{3.165789in}}%
\pgfpathlineto{\pgfqpoint{4.974444in}{3.165789in}}%
\pgfpathlineto{\pgfqpoint{4.974444in}{3.294079in}}%
\pgfpathlineto{\pgfqpoint{4.530000in}{3.294079in}}%
\pgfpathclose%
\pgfusepath{fill}%
\end{pgfscope}%
\begin{pgfscope}%
\pgfpathrectangle{\pgfqpoint{4.530000in}{0.600000in}}{\pgfqpoint{1.200000in}{3.900000in}}%
\pgfusepath{clip}%
\pgfsetbuttcap%
\pgfsetmiterjoin%
\definecolor{currentfill}{rgb}{0.121569,0.466667,0.705882}%
\pgfsetfillcolor{currentfill}%
\pgfsetlinewidth{0.000000pt}%
\definecolor{currentstroke}{rgb}{0.000000,0.000000,0.000000}%
\pgfsetstrokecolor{currentstroke}%
\pgfsetstrokeopacity{0.000000}%
\pgfsetdash{}{0pt}%
\pgfpathmoveto{\pgfqpoint{4.530000in}{3.294079in}}%
\pgfpathlineto{\pgfqpoint{4.836878in}{3.294079in}}%
\pgfpathlineto{\pgfqpoint{4.836878in}{3.422368in}}%
\pgfpathlineto{\pgfqpoint{4.530000in}{3.422368in}}%
\pgfpathclose%
\pgfusepath{fill}%
\end{pgfscope}%
\begin{pgfscope}%
\pgfpathrectangle{\pgfqpoint{4.530000in}{0.600000in}}{\pgfqpoint{1.200000in}{3.900000in}}%
\pgfusepath{clip}%
\pgfsetbuttcap%
\pgfsetmiterjoin%
\definecolor{currentfill}{rgb}{0.121569,0.466667,0.705882}%
\pgfsetfillcolor{currentfill}%
\pgfsetlinewidth{0.000000pt}%
\definecolor{currentstroke}{rgb}{0.000000,0.000000,0.000000}%
\pgfsetstrokecolor{currentstroke}%
\pgfsetstrokeopacity{0.000000}%
\pgfsetdash{}{0pt}%
\pgfpathmoveto{\pgfqpoint{4.530000in}{3.422368in}}%
\pgfpathlineto{\pgfqpoint{4.836878in}{3.422368in}}%
\pgfpathlineto{\pgfqpoint{4.836878in}{3.550658in}}%
\pgfpathlineto{\pgfqpoint{4.530000in}{3.550658in}}%
\pgfpathclose%
\pgfusepath{fill}%
\end{pgfscope}%
\begin{pgfscope}%
\pgfpathrectangle{\pgfqpoint{4.530000in}{0.600000in}}{\pgfqpoint{1.200000in}{3.900000in}}%
\pgfusepath{clip}%
\pgfsetbuttcap%
\pgfsetmiterjoin%
\definecolor{currentfill}{rgb}{0.121569,0.466667,0.705882}%
\pgfsetfillcolor{currentfill}%
\pgfsetlinewidth{0.000000pt}%
\definecolor{currentstroke}{rgb}{0.000000,0.000000,0.000000}%
\pgfsetstrokecolor{currentstroke}%
\pgfsetstrokeopacity{0.000000}%
\pgfsetdash{}{0pt}%
\pgfpathmoveto{\pgfqpoint{4.530000in}{3.550658in}}%
\pgfpathlineto{\pgfqpoint{4.667566in}{3.550658in}}%
\pgfpathlineto{\pgfqpoint{4.667566in}{3.678947in}}%
\pgfpathlineto{\pgfqpoint{4.530000in}{3.678947in}}%
\pgfpathclose%
\pgfusepath{fill}%
\end{pgfscope}%
\begin{pgfscope}%
\pgfpathrectangle{\pgfqpoint{4.530000in}{0.600000in}}{\pgfqpoint{1.200000in}{3.900000in}}%
\pgfusepath{clip}%
\pgfsetbuttcap%
\pgfsetmiterjoin%
\definecolor{currentfill}{rgb}{0.121569,0.466667,0.705882}%
\pgfsetfillcolor{currentfill}%
\pgfsetlinewidth{0.000000pt}%
\definecolor{currentstroke}{rgb}{0.000000,0.000000,0.000000}%
\pgfsetstrokecolor{currentstroke}%
\pgfsetstrokeopacity{0.000000}%
\pgfsetdash{}{0pt}%
\pgfpathmoveto{\pgfqpoint{4.530000in}{3.678947in}}%
\pgfpathlineto{\pgfqpoint{4.561746in}{3.678947in}}%
\pgfpathlineto{\pgfqpoint{4.561746in}{3.807237in}}%
\pgfpathlineto{\pgfqpoint{4.530000in}{3.807237in}}%
\pgfpathclose%
\pgfusepath{fill}%
\end{pgfscope}%
\begin{pgfscope}%
\pgfpathrectangle{\pgfqpoint{4.530000in}{0.600000in}}{\pgfqpoint{1.200000in}{3.900000in}}%
\pgfusepath{clip}%
\pgfsetbuttcap%
\pgfsetmiterjoin%
\definecolor{currentfill}{rgb}{0.121569,0.466667,0.705882}%
\pgfsetfillcolor{currentfill}%
\pgfsetlinewidth{0.000000pt}%
\definecolor{currentstroke}{rgb}{0.000000,0.000000,0.000000}%
\pgfsetstrokecolor{currentstroke}%
\pgfsetstrokeopacity{0.000000}%
\pgfsetdash{}{0pt}%
\pgfpathmoveto{\pgfqpoint{4.530000in}{3.807237in}}%
\pgfpathlineto{\pgfqpoint{4.604074in}{3.807237in}}%
\pgfpathlineto{\pgfqpoint{4.604074in}{3.935526in}}%
\pgfpathlineto{\pgfqpoint{4.530000in}{3.935526in}}%
\pgfpathclose%
\pgfusepath{fill}%
\end{pgfscope}%
\begin{pgfscope}%
\pgfpathrectangle{\pgfqpoint{4.530000in}{0.600000in}}{\pgfqpoint{1.200000in}{3.900000in}}%
\pgfusepath{clip}%
\pgfsetbuttcap%
\pgfsetmiterjoin%
\definecolor{currentfill}{rgb}{0.121569,0.466667,0.705882}%
\pgfsetfillcolor{currentfill}%
\pgfsetlinewidth{0.000000pt}%
\definecolor{currentstroke}{rgb}{0.000000,0.000000,0.000000}%
\pgfsetstrokecolor{currentstroke}%
\pgfsetstrokeopacity{0.000000}%
\pgfsetdash{}{0pt}%
\pgfpathmoveto{\pgfqpoint{4.530000in}{3.935526in}}%
\pgfpathlineto{\pgfqpoint{4.540582in}{3.935526in}}%
\pgfpathlineto{\pgfqpoint{4.540582in}{4.063816in}}%
\pgfpathlineto{\pgfqpoint{4.530000in}{4.063816in}}%
\pgfpathclose%
\pgfusepath{fill}%
\end{pgfscope}%
\begin{pgfscope}%
\pgfpathrectangle{\pgfqpoint{4.530000in}{0.600000in}}{\pgfqpoint{1.200000in}{3.900000in}}%
\pgfusepath{clip}%
\pgfsetbuttcap%
\pgfsetmiterjoin%
\definecolor{currentfill}{rgb}{0.121569,0.466667,0.705882}%
\pgfsetfillcolor{currentfill}%
\pgfsetlinewidth{0.000000pt}%
\definecolor{currentstroke}{rgb}{0.000000,0.000000,0.000000}%
\pgfsetstrokecolor{currentstroke}%
\pgfsetstrokeopacity{0.000000}%
\pgfsetdash{}{0pt}%
\pgfpathmoveto{\pgfqpoint{4.530000in}{4.063816in}}%
\pgfpathlineto{\pgfqpoint{4.530000in}{4.063816in}}%
\pgfpathlineto{\pgfqpoint{4.530000in}{4.192105in}}%
\pgfpathlineto{\pgfqpoint{4.530000in}{4.192105in}}%
\pgfpathclose%
\pgfusepath{fill}%
\end{pgfscope}%
\begin{pgfscope}%
\pgfpathrectangle{\pgfqpoint{4.530000in}{0.600000in}}{\pgfqpoint{1.200000in}{3.900000in}}%
\pgfusepath{clip}%
\pgfsetbuttcap%
\pgfsetmiterjoin%
\definecolor{currentfill}{rgb}{0.121569,0.466667,0.705882}%
\pgfsetfillcolor{currentfill}%
\pgfsetlinewidth{0.000000pt}%
\definecolor{currentstroke}{rgb}{0.000000,0.000000,0.000000}%
\pgfsetstrokecolor{currentstroke}%
\pgfsetstrokeopacity{0.000000}%
\pgfsetdash{}{0pt}%
\pgfpathmoveto{\pgfqpoint{4.530000in}{4.192105in}}%
\pgfpathlineto{\pgfqpoint{4.540582in}{4.192105in}}%
\pgfpathlineto{\pgfqpoint{4.540582in}{4.320395in}}%
\pgfpathlineto{\pgfqpoint{4.530000in}{4.320395in}}%
\pgfpathclose%
\pgfusepath{fill}%
\end{pgfscope}%
\begin{pgfscope}%
\pgfpathrectangle{\pgfqpoint{4.530000in}{0.600000in}}{\pgfqpoint{1.200000in}{3.900000in}}%
\pgfusepath{clip}%
\pgfsetbuttcap%
\pgfsetmiterjoin%
\definecolor{currentfill}{rgb}{0.121569,0.466667,0.705882}%
\pgfsetfillcolor{currentfill}%
\pgfsetlinewidth{0.000000pt}%
\definecolor{currentstroke}{rgb}{0.000000,0.000000,0.000000}%
\pgfsetstrokecolor{currentstroke}%
\pgfsetstrokeopacity{0.000000}%
\pgfsetdash{}{0pt}%
\pgfpathmoveto{\pgfqpoint{4.530000in}{4.320395in}}%
\pgfpathlineto{\pgfqpoint{4.530000in}{4.320395in}}%
\pgfpathlineto{\pgfqpoint{4.530000in}{4.448684in}}%
\pgfpathlineto{\pgfqpoint{4.530000in}{4.448684in}}%
\pgfpathclose%
\pgfusepath{fill}%
\end{pgfscope}%
\begin{pgfscope}%
\pgfpathrectangle{\pgfqpoint{4.530000in}{0.600000in}}{\pgfqpoint{1.200000in}{3.900000in}}%
\pgfusepath{clip}%
\pgfsetbuttcap%
\pgfsetmiterjoin%
\definecolor{currentfill}{rgb}{0.121569,0.466667,0.705882}%
\pgfsetfillcolor{currentfill}%
\pgfsetlinewidth{0.000000pt}%
\definecolor{currentstroke}{rgb}{0.000000,0.000000,0.000000}%
\pgfsetstrokecolor{currentstroke}%
\pgfsetstrokeopacity{0.000000}%
\pgfsetdash{}{0pt}%
\pgfpathmoveto{\pgfqpoint{4.530000in}{4.448684in}}%
\pgfpathlineto{\pgfqpoint{4.530000in}{4.448684in}}%
\pgfpathlineto{\pgfqpoint{4.530000in}{4.576974in}}%
\pgfpathlineto{\pgfqpoint{4.530000in}{4.576974in}}%
\pgfpathclose%
\pgfusepath{fill}%
\end{pgfscope}%
\begin{pgfscope}%
\pgfsetbuttcap%
\pgfsetroundjoin%
\definecolor{currentfill}{rgb}{0.000000,0.000000,0.000000}%
\pgfsetfillcolor{currentfill}%
\pgfsetlinewidth{0.803000pt}%
\definecolor{currentstroke}{rgb}{0.000000,0.000000,0.000000}%
\pgfsetstrokecolor{currentstroke}%
\pgfsetdash{}{0pt}%
\pgfsys@defobject{currentmarker}{\pgfqpoint{0.000000in}{0.000000in}}{\pgfqpoint{0.000000in}{0.048611in}}{%
\pgfpathmoveto{\pgfqpoint{0.000000in}{0.000000in}}%
\pgfpathlineto{\pgfqpoint{0.000000in}{0.048611in}}%
\pgfusepath{stroke,fill}%
}%
\begin{pgfscope}%
\pgfsys@transformshift{4.530000in}{0.600000in}%
\pgfsys@useobject{currentmarker}{}%
\end{pgfscope}%
\end{pgfscope}%
\begin{pgfscope}%
\definecolor{textcolor}{rgb}{0.000000,0.000000,0.000000}%
\pgfsetstrokecolor{textcolor}%
\pgfsetfillcolor{textcolor}%
\pgftext[x=4.530000in,y=0.551389in,,top]{\color{textcolor}\sffamily\fontsize{10.000000}{12.000000}\selectfont 0}%
\end{pgfscope}%
\begin{pgfscope}%
\pgfsetbuttcap%
\pgfsetroundjoin%
\definecolor{currentfill}{rgb}{0.000000,0.000000,0.000000}%
\pgfsetfillcolor{currentfill}%
\pgfsetlinewidth{0.803000pt}%
\definecolor{currentstroke}{rgb}{0.000000,0.000000,0.000000}%
\pgfsetstrokecolor{currentstroke}%
\pgfsetdash{}{0pt}%
\pgfsys@defobject{currentmarker}{\pgfqpoint{0.000000in}{0.000000in}}{\pgfqpoint{0.000000in}{0.048611in}}{%
\pgfpathmoveto{\pgfqpoint{0.000000in}{0.000000in}}%
\pgfpathlineto{\pgfqpoint{0.000000in}{0.048611in}}%
\pgfusepath{stroke,fill}%
}%
\begin{pgfscope}%
\pgfsys@transformshift{5.588201in}{0.600000in}%
\pgfsys@useobject{currentmarker}{}%
\end{pgfscope}%
\end{pgfscope}%
\begin{pgfscope}%
\definecolor{textcolor}{rgb}{0.000000,0.000000,0.000000}%
\pgfsetstrokecolor{textcolor}%
\pgfsetfillcolor{textcolor}%
\pgftext[x=5.588201in,y=0.551389in,,top]{\color{textcolor}\sffamily\fontsize{10.000000}{12.000000}\selectfont 100}%
\end{pgfscope}%
\begin{pgfscope}%
\pgfsetbuttcap%
\pgfsetroundjoin%
\definecolor{currentfill}{rgb}{0.000000,0.000000,0.000000}%
\pgfsetfillcolor{currentfill}%
\pgfsetlinewidth{0.803000pt}%
\definecolor{currentstroke}{rgb}{0.000000,0.000000,0.000000}%
\pgfsetstrokecolor{currentstroke}%
\pgfsetdash{}{0pt}%
\pgfsys@defobject{currentmarker}{\pgfqpoint{0.000000in}{0.000000in}}{\pgfqpoint{0.048611in}{0.000000in}}{%
\pgfpathmoveto{\pgfqpoint{0.000000in}{0.000000in}}%
\pgfpathlineto{\pgfqpoint{0.048611in}{0.000000in}}%
\pgfusepath{stroke,fill}%
}%
\begin{pgfscope}%
\pgfsys@transformshift{4.530000in}{1.010526in}%
\pgfsys@useobject{currentmarker}{}%
\end{pgfscope}%
\end{pgfscope}%
\begin{pgfscope}%
\pgfsetbuttcap%
\pgfsetroundjoin%
\definecolor{currentfill}{rgb}{0.000000,0.000000,0.000000}%
\pgfsetfillcolor{currentfill}%
\pgfsetlinewidth{0.803000pt}%
\definecolor{currentstroke}{rgb}{0.000000,0.000000,0.000000}%
\pgfsetstrokecolor{currentstroke}%
\pgfsetdash{}{0pt}%
\pgfsys@defobject{currentmarker}{\pgfqpoint{0.000000in}{0.000000in}}{\pgfqpoint{0.048611in}{0.000000in}}{%
\pgfpathmoveto{\pgfqpoint{0.000000in}{0.000000in}}%
\pgfpathlineto{\pgfqpoint{0.048611in}{0.000000in}}%
\pgfusepath{stroke,fill}%
}%
\begin{pgfscope}%
\pgfsys@transformshift{4.530000in}{1.523684in}%
\pgfsys@useobject{currentmarker}{}%
\end{pgfscope}%
\end{pgfscope}%
\begin{pgfscope}%
\pgfsetbuttcap%
\pgfsetroundjoin%
\definecolor{currentfill}{rgb}{0.000000,0.000000,0.000000}%
\pgfsetfillcolor{currentfill}%
\pgfsetlinewidth{0.803000pt}%
\definecolor{currentstroke}{rgb}{0.000000,0.000000,0.000000}%
\pgfsetstrokecolor{currentstroke}%
\pgfsetdash{}{0pt}%
\pgfsys@defobject{currentmarker}{\pgfqpoint{0.000000in}{0.000000in}}{\pgfqpoint{0.048611in}{0.000000in}}{%
\pgfpathmoveto{\pgfqpoint{0.000000in}{0.000000in}}%
\pgfpathlineto{\pgfqpoint{0.048611in}{0.000000in}}%
\pgfusepath{stroke,fill}%
}%
\begin{pgfscope}%
\pgfsys@transformshift{4.530000in}{2.036842in}%
\pgfsys@useobject{currentmarker}{}%
\end{pgfscope}%
\end{pgfscope}%
\begin{pgfscope}%
\pgfsetbuttcap%
\pgfsetroundjoin%
\definecolor{currentfill}{rgb}{0.000000,0.000000,0.000000}%
\pgfsetfillcolor{currentfill}%
\pgfsetlinewidth{0.803000pt}%
\definecolor{currentstroke}{rgb}{0.000000,0.000000,0.000000}%
\pgfsetstrokecolor{currentstroke}%
\pgfsetdash{}{0pt}%
\pgfsys@defobject{currentmarker}{\pgfqpoint{0.000000in}{0.000000in}}{\pgfqpoint{0.048611in}{0.000000in}}{%
\pgfpathmoveto{\pgfqpoint{0.000000in}{0.000000in}}%
\pgfpathlineto{\pgfqpoint{0.048611in}{0.000000in}}%
\pgfusepath{stroke,fill}%
}%
\begin{pgfscope}%
\pgfsys@transformshift{4.530000in}{2.550000in}%
\pgfsys@useobject{currentmarker}{}%
\end{pgfscope}%
\end{pgfscope}%
\begin{pgfscope}%
\pgfsetbuttcap%
\pgfsetroundjoin%
\definecolor{currentfill}{rgb}{0.000000,0.000000,0.000000}%
\pgfsetfillcolor{currentfill}%
\pgfsetlinewidth{0.803000pt}%
\definecolor{currentstroke}{rgb}{0.000000,0.000000,0.000000}%
\pgfsetstrokecolor{currentstroke}%
\pgfsetdash{}{0pt}%
\pgfsys@defobject{currentmarker}{\pgfqpoint{0.000000in}{0.000000in}}{\pgfqpoint{0.048611in}{0.000000in}}{%
\pgfpathmoveto{\pgfqpoint{0.000000in}{0.000000in}}%
\pgfpathlineto{\pgfqpoint{0.048611in}{0.000000in}}%
\pgfusepath{stroke,fill}%
}%
\begin{pgfscope}%
\pgfsys@transformshift{4.530000in}{3.063158in}%
\pgfsys@useobject{currentmarker}{}%
\end{pgfscope}%
\end{pgfscope}%
\begin{pgfscope}%
\pgfsetbuttcap%
\pgfsetroundjoin%
\definecolor{currentfill}{rgb}{0.000000,0.000000,0.000000}%
\pgfsetfillcolor{currentfill}%
\pgfsetlinewidth{0.803000pt}%
\definecolor{currentstroke}{rgb}{0.000000,0.000000,0.000000}%
\pgfsetstrokecolor{currentstroke}%
\pgfsetdash{}{0pt}%
\pgfsys@defobject{currentmarker}{\pgfqpoint{0.000000in}{0.000000in}}{\pgfqpoint{0.048611in}{0.000000in}}{%
\pgfpathmoveto{\pgfqpoint{0.000000in}{0.000000in}}%
\pgfpathlineto{\pgfqpoint{0.048611in}{0.000000in}}%
\pgfusepath{stroke,fill}%
}%
\begin{pgfscope}%
\pgfsys@transformshift{4.530000in}{3.576316in}%
\pgfsys@useobject{currentmarker}{}%
\end{pgfscope}%
\end{pgfscope}%
\begin{pgfscope}%
\pgfsetbuttcap%
\pgfsetroundjoin%
\definecolor{currentfill}{rgb}{0.000000,0.000000,0.000000}%
\pgfsetfillcolor{currentfill}%
\pgfsetlinewidth{0.803000pt}%
\definecolor{currentstroke}{rgb}{0.000000,0.000000,0.000000}%
\pgfsetstrokecolor{currentstroke}%
\pgfsetdash{}{0pt}%
\pgfsys@defobject{currentmarker}{\pgfqpoint{0.000000in}{0.000000in}}{\pgfqpoint{0.048611in}{0.000000in}}{%
\pgfpathmoveto{\pgfqpoint{0.000000in}{0.000000in}}%
\pgfpathlineto{\pgfqpoint{0.048611in}{0.000000in}}%
\pgfusepath{stroke,fill}%
}%
\begin{pgfscope}%
\pgfsys@transformshift{4.530000in}{4.089474in}%
\pgfsys@useobject{currentmarker}{}%
\end{pgfscope}%
\end{pgfscope}%
\begin{pgfscope}%
\pgfsetrectcap%
\pgfsetmiterjoin%
\pgfsetlinewidth{0.803000pt}%
\definecolor{currentstroke}{rgb}{0.000000,0.000000,0.000000}%
\pgfsetstrokecolor{currentstroke}%
\pgfsetdash{}{0pt}%
\pgfpathmoveto{\pgfqpoint{4.530000in}{0.600000in}}%
\pgfpathlineto{\pgfqpoint{4.530000in}{4.500000in}}%
\pgfusepath{stroke}%
\end{pgfscope}%
\begin{pgfscope}%
\pgfsetrectcap%
\pgfsetmiterjoin%
\pgfsetlinewidth{0.803000pt}%
\definecolor{currentstroke}{rgb}{0.000000,0.000000,0.000000}%
\pgfsetstrokecolor{currentstroke}%
\pgfsetdash{}{0pt}%
\pgfpathmoveto{\pgfqpoint{5.730000in}{0.600000in}}%
\pgfpathlineto{\pgfqpoint{5.730000in}{4.500000in}}%
\pgfusepath{stroke}%
\end{pgfscope}%
\begin{pgfscope}%
\pgfsetrectcap%
\pgfsetmiterjoin%
\pgfsetlinewidth{0.803000pt}%
\definecolor{currentstroke}{rgb}{0.000000,0.000000,0.000000}%
\pgfsetstrokecolor{currentstroke}%
\pgfsetdash{}{0pt}%
\pgfpathmoveto{\pgfqpoint{4.530000in}{0.600000in}}%
\pgfpathlineto{\pgfqpoint{5.730000in}{0.600000in}}%
\pgfusepath{stroke}%
\end{pgfscope}%
\begin{pgfscope}%
\pgfsetrectcap%
\pgfsetmiterjoin%
\pgfsetlinewidth{0.803000pt}%
\definecolor{currentstroke}{rgb}{0.000000,0.000000,0.000000}%
\pgfsetstrokecolor{currentstroke}%
\pgfsetdash{}{0pt}%
\pgfpathmoveto{\pgfqpoint{4.530000in}{4.500000in}}%
\pgfpathlineto{\pgfqpoint{5.730000in}{4.500000in}}%
\pgfusepath{stroke}%
\end{pgfscope}%
\end{pgfpicture}%
\makeatother%
\endgroup%
}
\scalebox{0.75}{%% Creator: Matplotlib, PGF backend
%%
%% To include the figure in your LaTeX document, write
%%   \input{<filename>.pgf}
%%
%% Make sure the required packages are loaded in your preamble
%%   \usepackage{pgf}
%%
%% Figures using additional raster images can only be included by \input if
%% they are in the same directory as the main LaTeX file. For loading figures
%% from other directories you can use the `import` package
%%   \usepackage{import}
%% and then include the figures with
%%   \import{<path to file>}{<filename>.pgf}
%%
%% Matplotlib used the following preamble
%%   \usepackage{fontspec}
%%   \setmainfont{DejaVuSerif.ttf}[Path=/home/lzh/anaconda3/envs/numana/lib/python3.7/site-packages/matplotlib/mpl-data/fonts/ttf/]
%%   \setsansfont{DejaVuSans.ttf}[Path=/home/lzh/anaconda3/envs/numana/lib/python3.7/site-packages/matplotlib/mpl-data/fonts/ttf/]
%%   \setmonofont{DejaVuSansMono.ttf}[Path=/home/lzh/anaconda3/envs/numana/lib/python3.7/site-packages/matplotlib/mpl-data/fonts/ttf/]
%%
\begingroup%
\makeatletter%
\begin{pgfpicture}%
\pgfpathrectangle{\pgfpointorigin}{\pgfqpoint{12.000000in}{4.000000in}}%
\pgfusepath{use as bounding box, clip}%
\begin{pgfscope}%
\pgfsetbuttcap%
\pgfsetmiterjoin%
\definecolor{currentfill}{rgb}{1.000000,1.000000,1.000000}%
\pgfsetfillcolor{currentfill}%
\pgfsetlinewidth{0.000000pt}%
\definecolor{currentstroke}{rgb}{1.000000,1.000000,1.000000}%
\pgfsetstrokecolor{currentstroke}%
\pgfsetdash{}{0pt}%
\pgfpathmoveto{\pgfqpoint{0.000000in}{0.000000in}}%
\pgfpathlineto{\pgfqpoint{12.000000in}{0.000000in}}%
\pgfpathlineto{\pgfqpoint{12.000000in}{4.000000in}}%
\pgfpathlineto{\pgfqpoint{0.000000in}{4.000000in}}%
\pgfpathclose%
\pgfusepath{fill}%
\end{pgfscope}%
\begin{pgfscope}%
\pgfsys@transformshift{0.120000in}{0.130000in}%
\pgftext[left,bottom]{\pgfimage[interpolate=true,width=3.830000in,height=3.720000in]{Figure24-img0.png}}%
\end{pgfscope}%
\begin{pgfscope}%
\pgfsys@transformshift{4.100000in}{0.130000in}%
\pgftext[left,bottom]{\pgfimage[interpolate=true,width=3.930000in,height=3.720000in]{Figure24-img1.png}}%
\end{pgfscope}%
\begin{pgfscope}%
\pgfsys@transformshift{8.050000in}{0.150000in}%
\pgftext[left,bottom]{\pgfimage[interpolate=true,width=3.950000in,height=3.700000in]{Figure24-img2.png}}%
\end{pgfscope}%
\end{pgfpicture}%
\makeatother%
\endgroup%
}
\caption{Figures of the three-dimensional cubatures of $ D = 3 $}
\label{Fig:M3D3}
\end{figure}

In this case, the smallest $N$ is $6$.

\section{Discussion}

\subsection{Non-existence with $ M = 2 $, $ D = 2 $, $ N = 2 $}

We observe non-existence of solution when $ D = 2 $ and $ N = 2 $. The groups are $ \sbr{ 1, 2, 3 } $ with $ C = 3 $. We give a proof of the non-existence here. We consider on the simplex $S$ for simplicity.

If we have a numerical cubature with $ D = 2 $ and $ N = 2 $, we may find a line
\begin{equation}
\ell \rbr{ x, y } = A x + B y + C = 0,
\end{equation}
which we assume $ A^2 + B^2 \neq 0 $, passing through $\mathbf{x}_1$ and $\mathbf{x}_2$, even if the two points coincide. We consider the integration of $\ell^2$. Since $ \ell^2 \rbr{\mathbf{x}_1} = \ell^2 \rbr{\mathbf{y}_1} = 0 $, we immediately have
\begin{equation}
I \rbr{\ell^2} = w_1 \ell^2 \rbr{\mathbf{x}_1} + w_2 \ell^2 \rbr{\mathbf{x}_2} = 0.
\end{equation}
However, the integral itself
\begin{equation}
\int_S \ell^2 \rbr{ x, y } \sd x \sd y > 0
\end{equation}
since $\ell^2$ does not vanish identically over $S$. This contradicts the exactness of $I$ on $ P_2 \rbr{S} $.

In this case, the degree of freedom is $ 2 N + C = 6 $ and the number of constraints is $ L = 6 $. Although this coincides, the non-linear solution has no solution. The is caused by the non-linearity of the system. Compared to the one-dimensional case, orthogonal polynomials have $0$-level sets as curves, and we cannot easily continue the argument in the one-dimensional case. This exemplifies the difficulties of higher-dimensional Gaussian cubatures.

This results can be generalized to arbitrary dimension. For arbitrary $M$ and $ D = 2 $, $ N \le M $ is impossible.

\subsection{Symmetric cubatures}

It can be observed from the numerical results, symmetry gives rise to some cubatures.

For the case $ D = 1 $, we take nodes to be barycenter of the (unique) $M$-cell of $\hat{S}$. To be concrete, we consider $\mathbf{x}_1$ to be $ \mathbf{1} / M $, namely the barycenter of $\hat{S}$. We set $ w_1 = 1 / M ! $. In this case,
\begin{gather}
I \rbr{ \mathbf{x} \mapsto 1 } = \frac{1}{ M ! } = \int_{\hat{S}} 1 \sd \mathbf{x}, \\
I \rbr{ \mathbf{x} \mapsto x_t } = \frac{1}{ \rbr{ M + 1 } ! } = \int_{\hat{S}} x_t \sd \mathbf{x}.
\end{gather}

However for $ D = 2 $, the argument in the last subsection applies again, and we get $ N \ge M $. Hence, the degree of freedom is $ N M + C \ge M^2 + 1 $. However, $ \abs{\mathcal{I}_2} = \rbr{ M + 1 } \rbr{ M + 2 } / 2 $. This means a greater number of degree of freedom does not lead to solutions.

One intuitive choice is the barycenters of the $1$-cells of $\hat{S}$. More precisely, we set $ N = M + 1 $, $ \mathbf{x}_t = \mathbf{e}_t $, $ \mathbf{x}_{ M + 1 } = 0 $, with $ w_t = 1 / \rbr{ M + 1 } ! $. One may verify
\begin{gather}
I \rbr{ \mathbf{x} \mapsto 1 } = \frac{ M + 1 }{ \rbr{ M + 1 } ! } = \frac{1}{ M ! } \int_{\hat{S}} 1 \sd \mathbf{x}, \\
I \rbr{ \mathbf{x} \mapsto x_t } = \frac{1}{ \rbr{ M + 1 } ! } = \int_{\hat{S}} x_t \sd \mathbf{x}, \\
I \rbr{ \mathbf{x} \mapsto x_t x_s } = 0 \neq \frac{1}{ \rbr{ M + 2 } ! } = \int_{\hat{S}} x_t x_s \sd \mathbf{x}, \\
I \rbr{ \mathbf{x} \mapsto x_t^2 } = \frac{1}{ \rbr{ M + 1 } ! } \neq \frac{2}{ \rbr{ M + 2 } ! } = \int_{\hat{S}} x_t^2 \sd \mathbf{x}.
\end{gather}

Another choice is to choose the $ \rbr{ M - 1 } $-cells of $\hat{S}$. More precisely, we set $ N = M + 1 $, $ \mathbf{x}_t = \rbr{ \mathbf{1} - \mathbf{e}_t } / M M ! $, $ \mathbf{x}_{ M + 1 } = \mathbf{1} / M M ! $, and $ w_t = 1 / \rbr{ M + 1 } ! $. We have
\begin{gather}
I \rbr{ \mathbf{x} \mapsto 1 } = \frac{ M + 1 }{ \rbr{ M + 1 } ! } = \frac{1}{ M ! } = \int_{\hat{S}} 1 \sd \mathbf{x}, \\
I \rbr{ \mathbf{x} \mapsto x_t } = \frac{M}{ M \rbr{ M + 1 } ! } = \int_{\hat{S}} x_t \sd \mathbf{x}, \\
I \rbr{ \mathbf{x} \mapsto x_t x_s } = \frac{ M - 1 }{ M^2 \rbr{ M + 1 } ! } \neq \frac{1}{ \rbr{ M + 2 } ! } = \int_{\hat{S}} x_t x_s \sd \mathbf{x}, \\
I \rbr{ \mathbf{x} \mapsto x_t^2 } = \frac{M}{ M^2 \rbr{ M + 1 } ! } \neq \frac{2}{ \rbr{ M + 2 } ! } = \int_{\hat{S}} x_t^2 \sd \mathbf{x},
\end{gather}
unless $ M = 2 $. In the two-dimensional case, we have
\begin{equation}
\hat{I} (\hat{f}) = \frac{1}{6} \hat{f} \rbr{ 0, \frac{1}{2} } + \frac{1}{6} \hat{f} \rbr{ \frac{1}{2}, 0 } + \frac{1}{6} \hat{f} \rbr{ \frac{1}{2}, \frac{1}{2} }.
\end{equation}
This formula has second algebraic precision. This corresponds to the well-known Simpson's formula.

\end{document}
